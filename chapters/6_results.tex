\section{Results}

This section presents results from the different set of experiments mentioned in chapter \ref{sub:experiments} \nameref{sub:experiments}. For every experiment, a table comparing every algorithm's displacement and turn error is present. And an error percentage for that shape. And lastly, a 2D and 3D visualization of the best performing algorithm for that experiment is present for every test.

\subsection{Line}

The line shape consisted of moving the inertial system in a straight line for a determined distance. Three-line distances were tested: 4, 16, and 28 meter. The results are shown below:

\subsubsection{4 meter}

For the 4-meter line experiment, the OLEQ algorithm which had the lowest displacement error with an average of 0.13 meters (4.06\% of error margin), and ROLEQ with an average of 0.24 meters of turn error (6.06\% of error margin).

\begin{figure}[!h]
    \centering
    \begin{table}[H]
    \begin{center}
        \resizebox{1\linewidth}{!}{
            \begin{tabular}[t]{lcccc}
                \hline
                Algorithm   & Displacement Error[$m$] & Displacement Error[\%] & Turn Error[$m$] & Turn Error[\%] \\
                \hline
                AngularRate & 0.16                    & 3.90                   & 0.50            & 12.62          \\            AQUA            & 0.90  & 22.59 & 1.46 & 36.50              \\            Complementary            & 0.25  & 6.23 & 0.55 & 13.66              \\            Davenport            & 0.49  & 12.28 & 0.64 & 16.02              \\            EKF            & 1.63  & 40.64 & 1.41 & 35.35              \\            FAMC            & 0.17  & 4.25 & 0.50 & 12.57              \\            FLAE            & 0.49  & 12.29 & 0.64 & 16.04              \\            Fourati            & 0.32  & 8.01 & 0.54 & 13.39              \\            Madgwick            & 0.74  & 18.57 & 0.74 & 18.51              \\            Mahony            & 0.34  & 8.56 & 0.55 & 13.80              \\            OLEQ            & 0.13  & 3.24 & 0.41 & 10.28              \\            QUEST            & 0.78  & 19.62 & 0.96 & 24.00              \\            ROLEQ            & 0.16  & 4.05 & 0.24 & 6.06              \\            SAAM            & 0.34  & 8.53 & 0.55 & 13.82              \\            Tilt            & 0.34  & 8.53 & 0.55 & 13.82              \\
                \hline
                Average     & 0.48                    & 12.09                  & 0.68            & 17.10
            \end{tabular}
        }
        \caption{Accelerometer Specifications. }
        \label{tab:accelerometer_specification}
    \end{center}
\end{table}
\end{figure}

\begin{figure}[!h]
    \centering
    \begin{subfigure}{0.49\textwidth}
        \centering
        \resizebox{1\linewidth}{!}{%% Creator: Matplotlib, PGF backend
%%
%% To include the figure in your LaTeX document, write
%%   \input{<filename>.pgf}
%%
%% Make sure the required packages are loaded in your preamble
%%   \usepackage{pgf}
%%
%% and, on pdftex
%%   \usepackage[utf8]{inputenc}\DeclareUnicodeCharacter{2212}{-}
%%
%% or, on luatex and xetex
%%   \usepackage{unicode-math}
%%
%% Figures using additional raster images can only be included by \input if
%% they are in the same directory as the main LaTeX file. For loading figures
%% from other directories you can use the `import` package
%%   \usepackage{import}
%%
%% and then include the figures with
%%   \import{<path to file>}{<filename>.pgf}
%%
%% Matplotlib used the following preamble
%%   \usepackage{fontspec}
%%   \setmainfont{DejaVuSerif.ttf}[Path=C:/Users/Claudio/AppData/Local/Programs/Python/Python39/Lib/site-packages/matplotlib/mpl-data/fonts/ttf/]
%%   \setsansfont{DejaVuSans.ttf}[Path=C:/Users/Claudio/AppData/Local/Programs/Python/Python39/Lib/site-packages/matplotlib/mpl-data/fonts/ttf/]
%%   \setmonofont{DejaVuSansMono.ttf}[Path=C:/Users/Claudio/AppData/Local/Programs/Python/Python39/Lib/site-packages/matplotlib/mpl-data/fonts/ttf/]
%%
\begingroup%
\makeatletter%
\begin{pgfpicture}%
\pgfpathrectangle{\pgfpointorigin}{\pgfqpoint{4.342069in}{4.016728in}}%
\pgfusepath{use as bounding box, clip}%
\begin{pgfscope}%
\pgfsetbuttcap%
\pgfsetmiterjoin%
\definecolor{currentfill}{rgb}{1.000000,1.000000,1.000000}%
\pgfsetfillcolor{currentfill}%
\pgfsetlinewidth{0.000000pt}%
\definecolor{currentstroke}{rgb}{1.000000,1.000000,1.000000}%
\pgfsetstrokecolor{currentstroke}%
\pgfsetdash{}{0pt}%
\pgfpathmoveto{\pgfqpoint{0.000000in}{0.000000in}}%
\pgfpathlineto{\pgfqpoint{4.342069in}{0.000000in}}%
\pgfpathlineto{\pgfqpoint{4.342069in}{4.016728in}}%
\pgfpathlineto{\pgfqpoint{0.000000in}{4.016728in}}%
\pgfpathclose%
\pgfusepath{fill}%
\end{pgfscope}%
\begin{pgfscope}%
\pgfsetbuttcap%
\pgfsetmiterjoin%
\definecolor{currentfill}{rgb}{1.000000,1.000000,1.000000}%
\pgfsetfillcolor{currentfill}%
\pgfsetlinewidth{0.000000pt}%
\definecolor{currentstroke}{rgb}{0.000000,0.000000,0.000000}%
\pgfsetstrokecolor{currentstroke}%
\pgfsetstrokeopacity{0.000000}%
\pgfsetdash{}{0pt}%
\pgfpathmoveto{\pgfqpoint{0.100000in}{0.220728in}}%
\pgfpathlineto{\pgfqpoint{3.796000in}{0.220728in}}%
\pgfpathlineto{\pgfqpoint{3.796000in}{3.916728in}}%
\pgfpathlineto{\pgfqpoint{0.100000in}{3.916728in}}%
\pgfpathclose%
\pgfusepath{fill}%
\end{pgfscope}%
\begin{pgfscope}%
\pgfsetbuttcap%
\pgfsetmiterjoin%
\definecolor{currentfill}{rgb}{0.950000,0.950000,0.950000}%
\pgfsetfillcolor{currentfill}%
\pgfsetfillopacity{0.500000}%
\pgfsetlinewidth{1.003750pt}%
\definecolor{currentstroke}{rgb}{0.950000,0.950000,0.950000}%
\pgfsetstrokecolor{currentstroke}%
\pgfsetstrokeopacity{0.500000}%
\pgfsetdash{}{0pt}%
\pgfpathmoveto{\pgfqpoint{0.379073in}{1.132043in}}%
\pgfpathlineto{\pgfqpoint{1.599613in}{2.155124in}}%
\pgfpathlineto{\pgfqpoint{1.582647in}{3.630589in}}%
\pgfpathlineto{\pgfqpoint{0.303698in}{2.697271in}}%
\pgfusepath{stroke,fill}%
\end{pgfscope}%
\begin{pgfscope}%
\pgfsetbuttcap%
\pgfsetmiterjoin%
\definecolor{currentfill}{rgb}{0.900000,0.900000,0.900000}%
\pgfsetfillcolor{currentfill}%
\pgfsetfillopacity{0.500000}%
\pgfsetlinewidth{1.003750pt}%
\definecolor{currentstroke}{rgb}{0.900000,0.900000,0.900000}%
\pgfsetstrokecolor{currentstroke}%
\pgfsetstrokeopacity{0.500000}%
\pgfsetdash{}{0pt}%
\pgfpathmoveto{\pgfqpoint{1.599613in}{2.155124in}}%
\pgfpathlineto{\pgfqpoint{3.558144in}{1.585856in}}%
\pgfpathlineto{\pgfqpoint{3.628038in}{3.112142in}}%
\pgfpathlineto{\pgfqpoint{1.582647in}{3.630589in}}%
\pgfusepath{stroke,fill}%
\end{pgfscope}%
\begin{pgfscope}%
\pgfsetbuttcap%
\pgfsetmiterjoin%
\definecolor{currentfill}{rgb}{0.925000,0.925000,0.925000}%
\pgfsetfillcolor{currentfill}%
\pgfsetfillopacity{0.500000}%
\pgfsetlinewidth{1.003750pt}%
\definecolor{currentstroke}{rgb}{0.925000,0.925000,0.925000}%
\pgfsetstrokecolor{currentstroke}%
\pgfsetstrokeopacity{0.500000}%
\pgfsetdash{}{0pt}%
\pgfpathmoveto{\pgfqpoint{0.379073in}{1.132043in}}%
\pgfpathlineto{\pgfqpoint{2.455212in}{0.453976in}}%
\pgfpathlineto{\pgfqpoint{3.558144in}{1.585856in}}%
\pgfpathlineto{\pgfqpoint{1.599613in}{2.155124in}}%
\pgfusepath{stroke,fill}%
\end{pgfscope}%
\begin{pgfscope}%
\pgfsetrectcap%
\pgfsetroundjoin%
\pgfsetlinewidth{0.803000pt}%
\definecolor{currentstroke}{rgb}{0.000000,0.000000,0.000000}%
\pgfsetstrokecolor{currentstroke}%
\pgfsetdash{}{0pt}%
\pgfpathmoveto{\pgfqpoint{0.379073in}{1.132043in}}%
\pgfpathlineto{\pgfqpoint{2.455212in}{0.453976in}}%
\pgfusepath{stroke}%
\end{pgfscope}%
\begin{pgfscope}%
\definecolor{textcolor}{rgb}{0.000000,0.000000,0.000000}%
\pgfsetstrokecolor{textcolor}%
\pgfsetfillcolor{textcolor}%
\pgftext[x=0.697927in, y=0.423808in, left, base,rotate=341.912962]{\color{textcolor}\sffamily\fontsize{10.000000}{12.000000}\selectfont Position X [\(\displaystyle m\)]}%
\end{pgfscope}%
\begin{pgfscope}%
\pgfsetbuttcap%
\pgfsetroundjoin%
\pgfsetlinewidth{0.803000pt}%
\definecolor{currentstroke}{rgb}{0.690196,0.690196,0.690196}%
\pgfsetstrokecolor{currentstroke}%
\pgfsetdash{}{0pt}%
\pgfpathmoveto{\pgfqpoint{0.653031in}{1.042569in}}%
\pgfpathlineto{\pgfqpoint{1.859044in}{2.079718in}}%
\pgfpathlineto{\pgfqpoint{1.853087in}{3.562041in}}%
\pgfusepath{stroke}%
\end{pgfscope}%
\begin{pgfscope}%
\pgfsetbuttcap%
\pgfsetroundjoin%
\pgfsetlinewidth{0.803000pt}%
\definecolor{currentstroke}{rgb}{0.690196,0.690196,0.690196}%
\pgfsetstrokecolor{currentstroke}%
\pgfsetdash{}{0pt}%
\pgfpathmoveto{\pgfqpoint{1.055862in}{0.911004in}}%
\pgfpathlineto{\pgfqpoint{2.239965in}{1.968999in}}%
\pgfpathlineto{\pgfqpoint{2.250448in}{3.461321in}}%
\pgfusepath{stroke}%
\end{pgfscope}%
\begin{pgfscope}%
\pgfsetbuttcap%
\pgfsetroundjoin%
\pgfsetlinewidth{0.803000pt}%
\definecolor{currentstroke}{rgb}{0.690196,0.690196,0.690196}%
\pgfsetstrokecolor{currentstroke}%
\pgfsetdash{}{0pt}%
\pgfpathmoveto{\pgfqpoint{1.467187in}{0.776665in}}%
\pgfpathlineto{\pgfqpoint{2.628245in}{1.856141in}}%
\pgfpathlineto{\pgfqpoint{2.655821in}{3.358571in}}%
\pgfusepath{stroke}%
\end{pgfscope}%
\begin{pgfscope}%
\pgfsetbuttcap%
\pgfsetroundjoin%
\pgfsetlinewidth{0.803000pt}%
\definecolor{currentstroke}{rgb}{0.690196,0.690196,0.690196}%
\pgfsetstrokecolor{currentstroke}%
\pgfsetdash{}{0pt}%
\pgfpathmoveto{\pgfqpoint{1.887278in}{0.639464in}}%
\pgfpathlineto{\pgfqpoint{3.024099in}{1.741082in}}%
\pgfpathlineto{\pgfqpoint{3.069451in}{3.253728in}}%
\pgfusepath{stroke}%
\end{pgfscope}%
\begin{pgfscope}%
\pgfsetbuttcap%
\pgfsetroundjoin%
\pgfsetlinewidth{0.803000pt}%
\definecolor{currentstroke}{rgb}{0.690196,0.690196,0.690196}%
\pgfsetstrokecolor{currentstroke}%
\pgfsetdash{}{0pt}%
\pgfpathmoveto{\pgfqpoint{2.316418in}{0.499307in}}%
\pgfpathlineto{\pgfqpoint{3.427751in}{1.623757in}}%
\pgfpathlineto{\pgfqpoint{3.491592in}{3.146727in}}%
\pgfusepath{stroke}%
\end{pgfscope}%
\begin{pgfscope}%
\pgfsetrectcap%
\pgfsetroundjoin%
\pgfsetlinewidth{0.803000pt}%
\definecolor{currentstroke}{rgb}{0.000000,0.000000,0.000000}%
\pgfsetstrokecolor{currentstroke}%
\pgfsetdash{}{0pt}%
\pgfpathmoveto{\pgfqpoint{0.663536in}{1.051603in}}%
\pgfpathlineto{\pgfqpoint{0.631975in}{1.024461in}}%
\pgfusepath{stroke}%
\end{pgfscope}%
\begin{pgfscope}%
\definecolor{textcolor}{rgb}{0.000000,0.000000,0.000000}%
\pgfsetstrokecolor{textcolor}%
\pgfsetfillcolor{textcolor}%
\pgftext[x=0.548609in,y=0.823340in,,top]{\color{textcolor}\sffamily\fontsize{10.000000}{12.000000}\selectfont −4}%
\end{pgfscope}%
\begin{pgfscope}%
\pgfsetrectcap%
\pgfsetroundjoin%
\pgfsetlinewidth{0.803000pt}%
\definecolor{currentstroke}{rgb}{0.000000,0.000000,0.000000}%
\pgfsetstrokecolor{currentstroke}%
\pgfsetdash{}{0pt}%
\pgfpathmoveto{\pgfqpoint{1.066185in}{0.920228in}}%
\pgfpathlineto{\pgfqpoint{1.035171in}{0.892517in}}%
\pgfusepath{stroke}%
\end{pgfscope}%
\begin{pgfscope}%
\definecolor{textcolor}{rgb}{0.000000,0.000000,0.000000}%
\pgfsetstrokecolor{textcolor}%
\pgfsetfillcolor{textcolor}%
\pgftext[x=0.951866in,y=0.688983in,,top]{\color{textcolor}\sffamily\fontsize{10.000000}{12.000000}\selectfont −2}%
\end{pgfscope}%
\begin{pgfscope}%
\pgfsetrectcap%
\pgfsetroundjoin%
\pgfsetlinewidth{0.803000pt}%
\definecolor{currentstroke}{rgb}{0.000000,0.000000,0.000000}%
\pgfsetstrokecolor{currentstroke}%
\pgfsetdash{}{0pt}%
\pgfpathmoveto{\pgfqpoint{1.477318in}{0.786084in}}%
\pgfpathlineto{\pgfqpoint{1.446881in}{0.757786in}}%
\pgfusepath{stroke}%
\end{pgfscope}%
\begin{pgfscope}%
\definecolor{textcolor}{rgb}{0.000000,0.000000,0.000000}%
\pgfsetstrokecolor{textcolor}%
\pgfsetfillcolor{textcolor}%
\pgftext[x=1.363658in,y=0.551783in,,top]{\color{textcolor}\sffamily\fontsize{10.000000}{12.000000}\selectfont 0}%
\end{pgfscope}%
\begin{pgfscope}%
\pgfsetrectcap%
\pgfsetroundjoin%
\pgfsetlinewidth{0.803000pt}%
\definecolor{currentstroke}{rgb}{0.000000,0.000000,0.000000}%
\pgfsetstrokecolor{currentstroke}%
\pgfsetdash{}{0pt}%
\pgfpathmoveto{\pgfqpoint{1.897207in}{0.649084in}}%
\pgfpathlineto{\pgfqpoint{1.867378in}{0.620179in}}%
\pgfusepath{stroke}%
\end{pgfscope}%
\begin{pgfscope}%
\definecolor{textcolor}{rgb}{0.000000,0.000000,0.000000}%
\pgfsetstrokecolor{textcolor}%
\pgfsetfillcolor{textcolor}%
\pgftext[x=1.784260in,y=0.411647in,,top]{\color{textcolor}\sffamily\fontsize{10.000000}{12.000000}\selectfont 2}%
\end{pgfscope}%
\begin{pgfscope}%
\pgfsetrectcap%
\pgfsetroundjoin%
\pgfsetlinewidth{0.803000pt}%
\definecolor{currentstroke}{rgb}{0.000000,0.000000,0.000000}%
\pgfsetstrokecolor{currentstroke}%
\pgfsetdash{}{0pt}%
\pgfpathmoveto{\pgfqpoint{2.326132in}{0.509136in}}%
\pgfpathlineto{\pgfqpoint{2.296945in}{0.479604in}}%
\pgfusepath{stroke}%
\end{pgfscope}%
\begin{pgfscope}%
\definecolor{textcolor}{rgb}{0.000000,0.000000,0.000000}%
\pgfsetstrokecolor{textcolor}%
\pgfsetfillcolor{textcolor}%
\pgftext[x=2.213956in,y=0.268482in,,top]{\color{textcolor}\sffamily\fontsize{10.000000}{12.000000}\selectfont 4}%
\end{pgfscope}%
\begin{pgfscope}%
\pgfsetrectcap%
\pgfsetroundjoin%
\pgfsetlinewidth{0.803000pt}%
\definecolor{currentstroke}{rgb}{0.000000,0.000000,0.000000}%
\pgfsetstrokecolor{currentstroke}%
\pgfsetdash{}{0pt}%
\pgfpathmoveto{\pgfqpoint{3.558144in}{1.585856in}}%
\pgfpathlineto{\pgfqpoint{2.455212in}{0.453976in}}%
\pgfusepath{stroke}%
\end{pgfscope}%
\begin{pgfscope}%
\definecolor{textcolor}{rgb}{0.000000,0.000000,0.000000}%
\pgfsetstrokecolor{textcolor}%
\pgfsetfillcolor{textcolor}%
\pgftext[x=3.103916in, y=0.291339in, left, base,rotate=45.742112]{\color{textcolor}\sffamily\fontsize{10.000000}{12.000000}\selectfont Position Y [\(\displaystyle m\)]}%
\end{pgfscope}%
\begin{pgfscope}%
\pgfsetbuttcap%
\pgfsetroundjoin%
\pgfsetlinewidth{0.803000pt}%
\definecolor{currentstroke}{rgb}{0.690196,0.690196,0.690196}%
\pgfsetstrokecolor{currentstroke}%
\pgfsetdash{}{0pt}%
\pgfpathmoveto{\pgfqpoint{0.392140in}{2.761812in}}%
\pgfpathlineto{\pgfqpoint{0.463179in}{1.202542in}}%
\pgfpathlineto{\pgfqpoint{2.531526in}{0.532294in}}%
\pgfusepath{stroke}%
\end{pgfscope}%
\begin{pgfscope}%
\pgfsetbuttcap%
\pgfsetroundjoin%
\pgfsetlinewidth{0.803000pt}%
\definecolor{currentstroke}{rgb}{0.690196,0.690196,0.690196}%
\pgfsetstrokecolor{currentstroke}%
\pgfsetdash{}{0pt}%
\pgfpathmoveto{\pgfqpoint{0.687617in}{2.977438in}}%
\pgfpathlineto{\pgfqpoint{0.744487in}{1.438340in}}%
\pgfpathlineto{\pgfqpoint{2.786438in}{0.793895in}}%
\pgfusepath{stroke}%
\end{pgfscope}%
\begin{pgfscope}%
\pgfsetbuttcap%
\pgfsetroundjoin%
\pgfsetlinewidth{0.803000pt}%
\definecolor{currentstroke}{rgb}{0.690196,0.690196,0.690196}%
\pgfsetstrokecolor{currentstroke}%
\pgfsetdash{}{0pt}%
\pgfpathmoveto{\pgfqpoint{0.971622in}{3.184691in}}%
\pgfpathlineto{\pgfqpoint{1.015337in}{1.665372in}}%
\pgfpathlineto{\pgfqpoint{3.031383in}{1.045269in}}%
\pgfusepath{stroke}%
\end{pgfscope}%
\begin{pgfscope}%
\pgfsetbuttcap%
\pgfsetroundjoin%
\pgfsetlinewidth{0.803000pt}%
\definecolor{currentstroke}{rgb}{0.690196,0.690196,0.690196}%
\pgfsetstrokecolor{currentstroke}%
\pgfsetdash{}{0pt}%
\pgfpathmoveto{\pgfqpoint{1.244810in}{3.384051in}}%
\pgfpathlineto{\pgfqpoint{1.276303in}{1.884119in}}%
\pgfpathlineto{\pgfqpoint{3.266935in}{1.287004in}}%
\pgfusepath{stroke}%
\end{pgfscope}%
\begin{pgfscope}%
\pgfsetbuttcap%
\pgfsetroundjoin%
\pgfsetlinewidth{0.803000pt}%
\definecolor{currentstroke}{rgb}{0.690196,0.690196,0.690196}%
\pgfsetstrokecolor{currentstroke}%
\pgfsetdash{}{0pt}%
\pgfpathmoveto{\pgfqpoint{1.507787in}{3.575960in}}%
\pgfpathlineto{\pgfqpoint{1.527916in}{2.095026in}}%
\pgfpathlineto{\pgfqpoint{3.493625in}{1.519643in}}%
\pgfusepath{stroke}%
\end{pgfscope}%
\begin{pgfscope}%
\pgfsetrectcap%
\pgfsetroundjoin%
\pgfsetlinewidth{0.803000pt}%
\definecolor{currentstroke}{rgb}{0.000000,0.000000,0.000000}%
\pgfsetstrokecolor{currentstroke}%
\pgfsetdash{}{0pt}%
\pgfpathmoveto{\pgfqpoint{2.514096in}{0.537942in}}%
\pgfpathlineto{\pgfqpoint{2.566432in}{0.520983in}}%
\pgfusepath{stroke}%
\end{pgfscope}%
\begin{pgfscope}%
\definecolor{textcolor}{rgb}{0.000000,0.000000,0.000000}%
\pgfsetstrokecolor{textcolor}%
\pgfsetfillcolor{textcolor}%
\pgftext[x=2.710424in,y=0.345859in,,top]{\color{textcolor}\sffamily\fontsize{10.000000}{12.000000}\selectfont 0}%
\end{pgfscope}%
\begin{pgfscope}%
\pgfsetrectcap%
\pgfsetroundjoin%
\pgfsetlinewidth{0.803000pt}%
\definecolor{currentstroke}{rgb}{0.000000,0.000000,0.000000}%
\pgfsetstrokecolor{currentstroke}%
\pgfsetdash{}{0pt}%
\pgfpathmoveto{\pgfqpoint{2.769247in}{0.799320in}}%
\pgfpathlineto{\pgfqpoint{2.820862in}{0.783031in}}%
\pgfusepath{stroke}%
\end{pgfscope}%
\begin{pgfscope}%
\definecolor{textcolor}{rgb}{0.000000,0.000000,0.000000}%
\pgfsetstrokecolor{textcolor}%
\pgfsetfillcolor{textcolor}%
\pgftext[x=2.961916in,y=0.611333in,,top]{\color{textcolor}\sffamily\fontsize{10.000000}{12.000000}\selectfont 1}%
\end{pgfscope}%
\begin{pgfscope}%
\pgfsetrectcap%
\pgfsetroundjoin%
\pgfsetlinewidth{0.803000pt}%
\definecolor{currentstroke}{rgb}{0.000000,0.000000,0.000000}%
\pgfsetstrokecolor{currentstroke}%
\pgfsetdash{}{0pt}%
\pgfpathmoveto{\pgfqpoint{3.014427in}{1.050484in}}%
\pgfpathlineto{\pgfqpoint{3.065336in}{1.034826in}}%
\pgfusepath{stroke}%
\end{pgfscope}%
\begin{pgfscope}%
\definecolor{textcolor}{rgb}{0.000000,0.000000,0.000000}%
\pgfsetstrokecolor{textcolor}%
\pgfsetfillcolor{textcolor}%
\pgftext[x=3.203570in,y=0.866423in,,top]{\color{textcolor}\sffamily\fontsize{10.000000}{12.000000}\selectfont 2}%
\end{pgfscope}%
\begin{pgfscope}%
\pgfsetrectcap%
\pgfsetroundjoin%
\pgfsetlinewidth{0.803000pt}%
\definecolor{currentstroke}{rgb}{0.000000,0.000000,0.000000}%
\pgfsetstrokecolor{currentstroke}%
\pgfsetdash{}{0pt}%
\pgfpathmoveto{\pgfqpoint{3.250209in}{1.292021in}}%
\pgfpathlineto{\pgfqpoint{3.300428in}{1.276957in}}%
\pgfusepath{stroke}%
\end{pgfscope}%
\begin{pgfscope}%
\definecolor{textcolor}{rgb}{0.000000,0.000000,0.000000}%
\pgfsetstrokecolor{textcolor}%
\pgfsetfillcolor{textcolor}%
\pgftext[x=3.435952in,y=1.111725in,,top]{\color{textcolor}\sffamily\fontsize{10.000000}{12.000000}\selectfont 3}%
\end{pgfscope}%
\begin{pgfscope}%
\pgfsetrectcap%
\pgfsetroundjoin%
\pgfsetlinewidth{0.803000pt}%
\definecolor{currentstroke}{rgb}{0.000000,0.000000,0.000000}%
\pgfsetstrokecolor{currentstroke}%
\pgfsetdash{}{0pt}%
\pgfpathmoveto{\pgfqpoint{3.477123in}{1.524473in}}%
\pgfpathlineto{\pgfqpoint{3.526668in}{1.509971in}}%
\pgfusepath{stroke}%
\end{pgfscope}%
\begin{pgfscope}%
\definecolor{textcolor}{rgb}{0.000000,0.000000,0.000000}%
\pgfsetstrokecolor{textcolor}%
\pgfsetfillcolor{textcolor}%
\pgftext[x=3.659585in,y=1.347792in,,top]{\color{textcolor}\sffamily\fontsize{10.000000}{12.000000}\selectfont 4}%
\end{pgfscope}%
\begin{pgfscope}%
\pgfsetrectcap%
\pgfsetroundjoin%
\pgfsetlinewidth{0.803000pt}%
\definecolor{currentstroke}{rgb}{0.000000,0.000000,0.000000}%
\pgfsetstrokecolor{currentstroke}%
\pgfsetdash{}{0pt}%
\pgfpathmoveto{\pgfqpoint{3.558144in}{1.585856in}}%
\pgfpathlineto{\pgfqpoint{3.628038in}{3.112142in}}%
\pgfusepath{stroke}%
\end{pgfscope}%
\begin{pgfscope}%
\definecolor{textcolor}{rgb}{0.000000,0.000000,0.000000}%
\pgfsetstrokecolor{textcolor}%
\pgfsetfillcolor{textcolor}%
\pgftext[x=4.169544in, y=1.928890in, left, base,rotate=87.378092]{\color{textcolor}\sffamily\fontsize{10.000000}{12.000000}\selectfont Position Z [\(\displaystyle m\)]}%
\end{pgfscope}%
\begin{pgfscope}%
\pgfsetbuttcap%
\pgfsetroundjoin%
\pgfsetlinewidth{0.803000pt}%
\definecolor{currentstroke}{rgb}{0.690196,0.690196,0.690196}%
\pgfsetstrokecolor{currentstroke}%
\pgfsetdash{}{0pt}%
\pgfpathmoveto{\pgfqpoint{3.562413in}{1.679073in}}%
\pgfpathlineto{\pgfqpoint{1.598575in}{2.245416in}}%
\pgfpathlineto{\pgfqpoint{0.374477in}{1.227487in}}%
\pgfusepath{stroke}%
\end{pgfscope}%
\begin{pgfscope}%
\pgfsetbuttcap%
\pgfsetroundjoin%
\pgfsetlinewidth{0.803000pt}%
\definecolor{currentstroke}{rgb}{0.690196,0.690196,0.690196}%
\pgfsetstrokecolor{currentstroke}%
\pgfsetdash{}{0pt}%
\pgfpathmoveto{\pgfqpoint{3.574139in}{1.935135in}}%
\pgfpathlineto{\pgfqpoint{1.595724in}{2.493322in}}%
\pgfpathlineto{\pgfqpoint{0.361847in}{1.489767in}}%
\pgfusepath{stroke}%
\end{pgfscope}%
\begin{pgfscope}%
\pgfsetbuttcap%
\pgfsetroundjoin%
\pgfsetlinewidth{0.803000pt}%
\definecolor{currentstroke}{rgb}{0.690196,0.690196,0.690196}%
\pgfsetstrokecolor{currentstroke}%
\pgfsetdash{}{0pt}%
\pgfpathmoveto{\pgfqpoint{3.586042in}{2.195063in}}%
\pgfpathlineto{\pgfqpoint{1.592833in}{2.744793in}}%
\pgfpathlineto{\pgfqpoint{0.349018in}{1.756160in}}%
\pgfusepath{stroke}%
\end{pgfscope}%
\begin{pgfscope}%
\pgfsetbuttcap%
\pgfsetroundjoin%
\pgfsetlinewidth{0.803000pt}%
\definecolor{currentstroke}{rgb}{0.690196,0.690196,0.690196}%
\pgfsetstrokecolor{currentstroke}%
\pgfsetdash{}{0pt}%
\pgfpathmoveto{\pgfqpoint{3.598126in}{2.458947in}}%
\pgfpathlineto{\pgfqpoint{1.589899in}{2.999906in}}%
\pgfpathlineto{\pgfqpoint{0.335987in}{2.026764in}}%
\pgfusepath{stroke}%
\end{pgfscope}%
\begin{pgfscope}%
\pgfsetbuttcap%
\pgfsetroundjoin%
\pgfsetlinewidth{0.803000pt}%
\definecolor{currentstroke}{rgb}{0.690196,0.690196,0.690196}%
\pgfsetstrokecolor{currentstroke}%
\pgfsetdash{}{0pt}%
\pgfpathmoveto{\pgfqpoint{3.610395in}{2.726878in}}%
\pgfpathlineto{\pgfqpoint{1.586923in}{3.258741in}}%
\pgfpathlineto{\pgfqpoint{0.322748in}{2.301678in}}%
\pgfusepath{stroke}%
\end{pgfscope}%
\begin{pgfscope}%
\pgfsetbuttcap%
\pgfsetroundjoin%
\pgfsetlinewidth{0.803000pt}%
\definecolor{currentstroke}{rgb}{0.690196,0.690196,0.690196}%
\pgfsetstrokecolor{currentstroke}%
\pgfsetdash{}{0pt}%
\pgfpathmoveto{\pgfqpoint{3.622854in}{2.998949in}}%
\pgfpathlineto{\pgfqpoint{1.583903in}{3.521379in}}%
\pgfpathlineto{\pgfqpoint{0.309296in}{2.581008in}}%
\pgfusepath{stroke}%
\end{pgfscope}%
\begin{pgfscope}%
\pgfsetrectcap%
\pgfsetroundjoin%
\pgfsetlinewidth{0.803000pt}%
\definecolor{currentstroke}{rgb}{0.000000,0.000000,0.000000}%
\pgfsetstrokecolor{currentstroke}%
\pgfsetdash{}{0pt}%
\pgfpathmoveto{\pgfqpoint{3.545929in}{1.683827in}}%
\pgfpathlineto{\pgfqpoint{3.595421in}{1.669555in}}%
\pgfusepath{stroke}%
\end{pgfscope}%
\begin{pgfscope}%
\definecolor{textcolor}{rgb}{0.000000,0.000000,0.000000}%
\pgfsetstrokecolor{textcolor}%
\pgfsetfillcolor{textcolor}%
\pgftext[x=3.816545in,y=1.715072in,,top]{\color{textcolor}\sffamily\fontsize{10.000000}{12.000000}\selectfont 0.0}%
\end{pgfscope}%
\begin{pgfscope}%
\pgfsetrectcap%
\pgfsetroundjoin%
\pgfsetlinewidth{0.803000pt}%
\definecolor{currentstroke}{rgb}{0.000000,0.000000,0.000000}%
\pgfsetstrokecolor{currentstroke}%
\pgfsetdash{}{0pt}%
\pgfpathmoveto{\pgfqpoint{3.557527in}{1.939822in}}%
\pgfpathlineto{\pgfqpoint{3.607403in}{1.925749in}}%
\pgfusepath{stroke}%
\end{pgfscope}%
\begin{pgfscope}%
\definecolor{textcolor}{rgb}{0.000000,0.000000,0.000000}%
\pgfsetstrokecolor{textcolor}%
\pgfsetfillcolor{textcolor}%
\pgftext[x=3.830134in,y=1.970627in,,top]{\color{textcolor}\sffamily\fontsize{10.000000}{12.000000}\selectfont 0.1}%
\end{pgfscope}%
\begin{pgfscope}%
\pgfsetrectcap%
\pgfsetroundjoin%
\pgfsetlinewidth{0.803000pt}%
\definecolor{currentstroke}{rgb}{0.000000,0.000000,0.000000}%
\pgfsetstrokecolor{currentstroke}%
\pgfsetdash{}{0pt}%
\pgfpathmoveto{\pgfqpoint{3.569299in}{2.199681in}}%
\pgfpathlineto{\pgfqpoint{3.619567in}{2.185817in}}%
\pgfusepath{stroke}%
\end{pgfscope}%
\begin{pgfscope}%
\definecolor{textcolor}{rgb}{0.000000,0.000000,0.000000}%
\pgfsetstrokecolor{textcolor}%
\pgfsetfillcolor{textcolor}%
\pgftext[x=3.843928in,y=2.230029in,,top]{\color{textcolor}\sffamily\fontsize{10.000000}{12.000000}\selectfont 0.2}%
\end{pgfscope}%
\begin{pgfscope}%
\pgfsetrectcap%
\pgfsetroundjoin%
\pgfsetlinewidth{0.803000pt}%
\definecolor{currentstroke}{rgb}{0.000000,0.000000,0.000000}%
\pgfsetstrokecolor{currentstroke}%
\pgfsetdash{}{0pt}%
\pgfpathmoveto{\pgfqpoint{3.581251in}{2.463493in}}%
\pgfpathlineto{\pgfqpoint{3.631916in}{2.449845in}}%
\pgfusepath{stroke}%
\end{pgfscope}%
\begin{pgfscope}%
\definecolor{textcolor}{rgb}{0.000000,0.000000,0.000000}%
\pgfsetstrokecolor{textcolor}%
\pgfsetfillcolor{textcolor}%
\pgftext[x=3.857931in,y=2.493367in,,top]{\color{textcolor}\sffamily\fontsize{10.000000}{12.000000}\selectfont 0.3}%
\end{pgfscope}%
\begin{pgfscope}%
\pgfsetrectcap%
\pgfsetroundjoin%
\pgfsetlinewidth{0.803000pt}%
\definecolor{currentstroke}{rgb}{0.000000,0.000000,0.000000}%
\pgfsetstrokecolor{currentstroke}%
\pgfsetdash{}{0pt}%
\pgfpathmoveto{\pgfqpoint{3.593386in}{2.731349in}}%
\pgfpathlineto{\pgfqpoint{3.644455in}{2.717925in}}%
\pgfusepath{stroke}%
\end{pgfscope}%
\begin{pgfscope}%
\definecolor{textcolor}{rgb}{0.000000,0.000000,0.000000}%
\pgfsetstrokecolor{textcolor}%
\pgfsetfillcolor{textcolor}%
\pgftext[x=3.872148in,y=2.760731in,,top]{\color{textcolor}\sffamily\fontsize{10.000000}{12.000000}\selectfont 0.4}%
\end{pgfscope}%
\begin{pgfscope}%
\pgfsetrectcap%
\pgfsetroundjoin%
\pgfsetlinewidth{0.803000pt}%
\definecolor{currentstroke}{rgb}{0.000000,0.000000,0.000000}%
\pgfsetstrokecolor{currentstroke}%
\pgfsetdash{}{0pt}%
\pgfpathmoveto{\pgfqpoint{3.605708in}{3.003342in}}%
\pgfpathlineto{\pgfqpoint{3.657188in}{2.990152in}}%
\pgfusepath{stroke}%
\end{pgfscope}%
\begin{pgfscope}%
\definecolor{textcolor}{rgb}{0.000000,0.000000,0.000000}%
\pgfsetstrokecolor{textcolor}%
\pgfsetfillcolor{textcolor}%
\pgftext[x=3.886585in,y=3.032214in,,top]{\color{textcolor}\sffamily\fontsize{10.000000}{12.000000}\selectfont 0.5}%
\end{pgfscope}%
\begin{pgfscope}%
\pgfpathrectangle{\pgfqpoint{0.100000in}{0.220728in}}{\pgfqpoint{3.696000in}{3.696000in}}%
\pgfusepath{clip}%
\pgfsetrectcap%
\pgfsetroundjoin%
\pgfsetlinewidth{1.505625pt}%
\definecolor{currentstroke}{rgb}{0.121569,0.466667,0.705882}%
\pgfsetstrokecolor{currentstroke}%
\pgfsetdash{}{0pt}%
\pgfpathmoveto{\pgfqpoint{1.546059in}{0.947987in}}%
\pgfpathlineto{\pgfqpoint{2.561702in}{1.885039in}}%
\pgfpathlineto{\pgfqpoint{3.366464in}{1.651316in}}%
\pgfpathlineto{\pgfqpoint{2.394465in}{0.675130in}}%
\pgfpathlineto{\pgfqpoint{1.546059in}{0.947987in}}%
\pgfusepath{stroke}%
\end{pgfscope}%
\begin{pgfscope}%
\pgfpathrectangle{\pgfqpoint{0.100000in}{0.220728in}}{\pgfqpoint{3.696000in}{3.696000in}}%
\pgfusepath{clip}%
\pgfsetrectcap%
\pgfsetroundjoin%
\pgfsetlinewidth{1.505625pt}%
\definecolor{currentstroke}{rgb}{1.000000,0.000000,0.000000}%
\pgfsetstrokecolor{currentstroke}%
\pgfsetdash{}{0pt}%
\pgfpathmoveto{\pgfqpoint{1.545400in}{1.007576in}}%
\pgfpathlineto{\pgfqpoint{1.546059in}{0.947987in}}%
\pgfusepath{stroke}%
\end{pgfscope}%
\begin{pgfscope}%
\pgfpathrectangle{\pgfqpoint{0.100000in}{0.220728in}}{\pgfqpoint{3.696000in}{3.696000in}}%
\pgfusepath{clip}%
\pgfsetrectcap%
\pgfsetroundjoin%
\pgfsetlinewidth{1.505625pt}%
\definecolor{currentstroke}{rgb}{1.000000,0.000000,0.000000}%
\pgfsetstrokecolor{currentstroke}%
\pgfsetdash{}{0pt}%
\pgfpathmoveto{\pgfqpoint{1.542926in}{1.013246in}}%
\pgfpathlineto{\pgfqpoint{1.546059in}{0.947987in}}%
\pgfusepath{stroke}%
\end{pgfscope}%
\begin{pgfscope}%
\pgfpathrectangle{\pgfqpoint{0.100000in}{0.220728in}}{\pgfqpoint{3.696000in}{3.696000in}}%
\pgfusepath{clip}%
\pgfsetrectcap%
\pgfsetroundjoin%
\pgfsetlinewidth{1.505625pt}%
\definecolor{currentstroke}{rgb}{1.000000,0.000000,0.000000}%
\pgfsetstrokecolor{currentstroke}%
\pgfsetdash{}{0pt}%
\pgfpathmoveto{\pgfqpoint{1.537754in}{1.020646in}}%
\pgfpathlineto{\pgfqpoint{1.546059in}{0.947987in}}%
\pgfusepath{stroke}%
\end{pgfscope}%
\begin{pgfscope}%
\pgfpathrectangle{\pgfqpoint{0.100000in}{0.220728in}}{\pgfqpoint{3.696000in}{3.696000in}}%
\pgfusepath{clip}%
\pgfsetrectcap%
\pgfsetroundjoin%
\pgfsetlinewidth{1.505625pt}%
\definecolor{currentstroke}{rgb}{1.000000,0.000000,0.000000}%
\pgfsetstrokecolor{currentstroke}%
\pgfsetdash{}{0pt}%
\pgfpathmoveto{\pgfqpoint{1.530762in}{1.031364in}}%
\pgfpathlineto{\pgfqpoint{1.557126in}{0.958197in}}%
\pgfusepath{stroke}%
\end{pgfscope}%
\begin{pgfscope}%
\pgfpathrectangle{\pgfqpoint{0.100000in}{0.220728in}}{\pgfqpoint{3.696000in}{3.696000in}}%
\pgfusepath{clip}%
\pgfsetrectcap%
\pgfsetroundjoin%
\pgfsetlinewidth{1.505625pt}%
\definecolor{currentstroke}{rgb}{1.000000,0.000000,0.000000}%
\pgfsetstrokecolor{currentstroke}%
\pgfsetdash{}{0pt}%
\pgfpathmoveto{\pgfqpoint{1.517960in}{1.054742in}}%
\pgfpathlineto{\pgfqpoint{1.546059in}{0.947987in}}%
\pgfusepath{stroke}%
\end{pgfscope}%
\begin{pgfscope}%
\pgfpathrectangle{\pgfqpoint{0.100000in}{0.220728in}}{\pgfqpoint{3.696000in}{3.696000in}}%
\pgfusepath{clip}%
\pgfsetrectcap%
\pgfsetroundjoin%
\pgfsetlinewidth{1.505625pt}%
\definecolor{currentstroke}{rgb}{1.000000,0.000000,0.000000}%
\pgfsetstrokecolor{currentstroke}%
\pgfsetdash{}{0pt}%
\pgfpathmoveto{\pgfqpoint{1.512132in}{1.062280in}}%
\pgfpathlineto{\pgfqpoint{1.546059in}{0.947987in}}%
\pgfusepath{stroke}%
\end{pgfscope}%
\begin{pgfscope}%
\pgfpathrectangle{\pgfqpoint{0.100000in}{0.220728in}}{\pgfqpoint{3.696000in}{3.696000in}}%
\pgfusepath{clip}%
\pgfsetrectcap%
\pgfsetroundjoin%
\pgfsetlinewidth{1.505625pt}%
\definecolor{currentstroke}{rgb}{1.000000,0.000000,0.000000}%
\pgfsetstrokecolor{currentstroke}%
\pgfsetdash{}{0pt}%
\pgfpathmoveto{\pgfqpoint{1.509693in}{1.068735in}}%
\pgfpathlineto{\pgfqpoint{1.546059in}{0.947987in}}%
\pgfusepath{stroke}%
\end{pgfscope}%
\begin{pgfscope}%
\pgfpathrectangle{\pgfqpoint{0.100000in}{0.220728in}}{\pgfqpoint{3.696000in}{3.696000in}}%
\pgfusepath{clip}%
\pgfsetrectcap%
\pgfsetroundjoin%
\pgfsetlinewidth{1.505625pt}%
\definecolor{currentstroke}{rgb}{1.000000,0.000000,0.000000}%
\pgfsetstrokecolor{currentstroke}%
\pgfsetdash{}{0pt}%
\pgfpathmoveto{\pgfqpoint{1.503814in}{1.079344in}}%
\pgfpathlineto{\pgfqpoint{1.546059in}{0.947987in}}%
\pgfusepath{stroke}%
\end{pgfscope}%
\begin{pgfscope}%
\pgfpathrectangle{\pgfqpoint{0.100000in}{0.220728in}}{\pgfqpoint{3.696000in}{3.696000in}}%
\pgfusepath{clip}%
\pgfsetrectcap%
\pgfsetroundjoin%
\pgfsetlinewidth{1.505625pt}%
\definecolor{currentstroke}{rgb}{1.000000,0.000000,0.000000}%
\pgfsetstrokecolor{currentstroke}%
\pgfsetdash{}{0pt}%
\pgfpathmoveto{\pgfqpoint{1.497121in}{1.095821in}}%
\pgfpathlineto{\pgfqpoint{1.557126in}{0.958197in}}%
\pgfusepath{stroke}%
\end{pgfscope}%
\begin{pgfscope}%
\pgfpathrectangle{\pgfqpoint{0.100000in}{0.220728in}}{\pgfqpoint{3.696000in}{3.696000in}}%
\pgfusepath{clip}%
\pgfsetrectcap%
\pgfsetroundjoin%
\pgfsetlinewidth{1.505625pt}%
\definecolor{currentstroke}{rgb}{1.000000,0.000000,0.000000}%
\pgfsetstrokecolor{currentstroke}%
\pgfsetdash{}{0pt}%
\pgfpathmoveto{\pgfqpoint{1.484525in}{1.129840in}}%
\pgfpathlineto{\pgfqpoint{1.557126in}{0.958197in}}%
\pgfusepath{stroke}%
\end{pgfscope}%
\begin{pgfscope}%
\pgfpathrectangle{\pgfqpoint{0.100000in}{0.220728in}}{\pgfqpoint{3.696000in}{3.696000in}}%
\pgfusepath{clip}%
\pgfsetrectcap%
\pgfsetroundjoin%
\pgfsetlinewidth{1.505625pt}%
\definecolor{currentstroke}{rgb}{1.000000,0.000000,0.000000}%
\pgfsetstrokecolor{currentstroke}%
\pgfsetdash{}{0pt}%
\pgfpathmoveto{\pgfqpoint{1.472650in}{1.155319in}}%
\pgfpathlineto{\pgfqpoint{1.557126in}{0.958197in}}%
\pgfusepath{stroke}%
\end{pgfscope}%
\begin{pgfscope}%
\pgfpathrectangle{\pgfqpoint{0.100000in}{0.220728in}}{\pgfqpoint{3.696000in}{3.696000in}}%
\pgfusepath{clip}%
\pgfsetrectcap%
\pgfsetroundjoin%
\pgfsetlinewidth{1.505625pt}%
\definecolor{currentstroke}{rgb}{1.000000,0.000000,0.000000}%
\pgfsetstrokecolor{currentstroke}%
\pgfsetdash{}{0pt}%
\pgfpathmoveto{\pgfqpoint{1.466142in}{1.172498in}}%
\pgfpathlineto{\pgfqpoint{1.557126in}{0.958197in}}%
\pgfusepath{stroke}%
\end{pgfscope}%
\begin{pgfscope}%
\pgfpathrectangle{\pgfqpoint{0.100000in}{0.220728in}}{\pgfqpoint{3.696000in}{3.696000in}}%
\pgfusepath{clip}%
\pgfsetrectcap%
\pgfsetroundjoin%
\pgfsetlinewidth{1.505625pt}%
\definecolor{currentstroke}{rgb}{1.000000,0.000000,0.000000}%
\pgfsetstrokecolor{currentstroke}%
\pgfsetdash{}{0pt}%
\pgfpathmoveto{\pgfqpoint{1.462465in}{1.172083in}}%
\pgfpathlineto{\pgfqpoint{1.557126in}{0.958197in}}%
\pgfusepath{stroke}%
\end{pgfscope}%
\begin{pgfscope}%
\pgfpathrectangle{\pgfqpoint{0.100000in}{0.220728in}}{\pgfqpoint{3.696000in}{3.696000in}}%
\pgfusepath{clip}%
\pgfsetrectcap%
\pgfsetroundjoin%
\pgfsetlinewidth{1.505625pt}%
\definecolor{currentstroke}{rgb}{1.000000,0.000000,0.000000}%
\pgfsetstrokecolor{currentstroke}%
\pgfsetdash{}{0pt}%
\pgfpathmoveto{\pgfqpoint{1.457397in}{1.173192in}}%
\pgfpathlineto{\pgfqpoint{1.557126in}{0.958197in}}%
\pgfusepath{stroke}%
\end{pgfscope}%
\begin{pgfscope}%
\pgfpathrectangle{\pgfqpoint{0.100000in}{0.220728in}}{\pgfqpoint{3.696000in}{3.696000in}}%
\pgfusepath{clip}%
\pgfsetrectcap%
\pgfsetroundjoin%
\pgfsetlinewidth{1.505625pt}%
\definecolor{currentstroke}{rgb}{1.000000,0.000000,0.000000}%
\pgfsetstrokecolor{currentstroke}%
\pgfsetdash{}{0pt}%
\pgfpathmoveto{\pgfqpoint{1.445382in}{1.188913in}}%
\pgfpathlineto{\pgfqpoint{1.557126in}{0.958197in}}%
\pgfusepath{stroke}%
\end{pgfscope}%
\begin{pgfscope}%
\pgfpathrectangle{\pgfqpoint{0.100000in}{0.220728in}}{\pgfqpoint{3.696000in}{3.696000in}}%
\pgfusepath{clip}%
\pgfsetrectcap%
\pgfsetroundjoin%
\pgfsetlinewidth{1.505625pt}%
\definecolor{currentstroke}{rgb}{1.000000,0.000000,0.000000}%
\pgfsetstrokecolor{currentstroke}%
\pgfsetdash{}{0pt}%
\pgfpathmoveto{\pgfqpoint{1.433727in}{1.201492in}}%
\pgfpathlineto{\pgfqpoint{1.557126in}{0.958197in}}%
\pgfusepath{stroke}%
\end{pgfscope}%
\begin{pgfscope}%
\pgfpathrectangle{\pgfqpoint{0.100000in}{0.220728in}}{\pgfqpoint{3.696000in}{3.696000in}}%
\pgfusepath{clip}%
\pgfsetrectcap%
\pgfsetroundjoin%
\pgfsetlinewidth{1.505625pt}%
\definecolor{currentstroke}{rgb}{1.000000,0.000000,0.000000}%
\pgfsetstrokecolor{currentstroke}%
\pgfsetdash{}{0pt}%
\pgfpathmoveto{\pgfqpoint{1.424087in}{1.194318in}}%
\pgfpathlineto{\pgfqpoint{1.568175in}{0.968392in}}%
\pgfusepath{stroke}%
\end{pgfscope}%
\begin{pgfscope}%
\pgfpathrectangle{\pgfqpoint{0.100000in}{0.220728in}}{\pgfqpoint{3.696000in}{3.696000in}}%
\pgfusepath{clip}%
\pgfsetrectcap%
\pgfsetroundjoin%
\pgfsetlinewidth{1.505625pt}%
\definecolor{currentstroke}{rgb}{1.000000,0.000000,0.000000}%
\pgfsetstrokecolor{currentstroke}%
\pgfsetdash{}{0pt}%
\pgfpathmoveto{\pgfqpoint{1.406941in}{1.180379in}}%
\pgfpathlineto{\pgfqpoint{1.568175in}{0.968392in}}%
\pgfusepath{stroke}%
\end{pgfscope}%
\begin{pgfscope}%
\pgfpathrectangle{\pgfqpoint{0.100000in}{0.220728in}}{\pgfqpoint{3.696000in}{3.696000in}}%
\pgfusepath{clip}%
\pgfsetrectcap%
\pgfsetroundjoin%
\pgfsetlinewidth{1.505625pt}%
\definecolor{currentstroke}{rgb}{1.000000,0.000000,0.000000}%
\pgfsetstrokecolor{currentstroke}%
\pgfsetdash{}{0pt}%
\pgfpathmoveto{\pgfqpoint{1.390229in}{1.175495in}}%
\pgfpathlineto{\pgfqpoint{1.568175in}{0.968392in}}%
\pgfusepath{stroke}%
\end{pgfscope}%
\begin{pgfscope}%
\pgfpathrectangle{\pgfqpoint{0.100000in}{0.220728in}}{\pgfqpoint{3.696000in}{3.696000in}}%
\pgfusepath{clip}%
\pgfsetrectcap%
\pgfsetroundjoin%
\pgfsetlinewidth{1.505625pt}%
\definecolor{currentstroke}{rgb}{1.000000,0.000000,0.000000}%
\pgfsetstrokecolor{currentstroke}%
\pgfsetdash{}{0pt}%
\pgfpathmoveto{\pgfqpoint{1.380019in}{1.173193in}}%
\pgfpathlineto{\pgfqpoint{1.568175in}{0.968392in}}%
\pgfusepath{stroke}%
\end{pgfscope}%
\begin{pgfscope}%
\pgfpathrectangle{\pgfqpoint{0.100000in}{0.220728in}}{\pgfqpoint{3.696000in}{3.696000in}}%
\pgfusepath{clip}%
\pgfsetrectcap%
\pgfsetroundjoin%
\pgfsetlinewidth{1.505625pt}%
\definecolor{currentstroke}{rgb}{1.000000,0.000000,0.000000}%
\pgfsetstrokecolor{currentstroke}%
\pgfsetdash{}{0pt}%
\pgfpathmoveto{\pgfqpoint{1.367773in}{1.172411in}}%
\pgfpathlineto{\pgfqpoint{1.568175in}{0.968392in}}%
\pgfusepath{stroke}%
\end{pgfscope}%
\begin{pgfscope}%
\pgfpathrectangle{\pgfqpoint{0.100000in}{0.220728in}}{\pgfqpoint{3.696000in}{3.696000in}}%
\pgfusepath{clip}%
\pgfsetrectcap%
\pgfsetroundjoin%
\pgfsetlinewidth{1.505625pt}%
\definecolor{currentstroke}{rgb}{1.000000,0.000000,0.000000}%
\pgfsetstrokecolor{currentstroke}%
\pgfsetdash{}{0pt}%
\pgfpathmoveto{\pgfqpoint{1.351518in}{1.189113in}}%
\pgfpathlineto{\pgfqpoint{1.568175in}{0.968392in}}%
\pgfusepath{stroke}%
\end{pgfscope}%
\begin{pgfscope}%
\pgfpathrectangle{\pgfqpoint{0.100000in}{0.220728in}}{\pgfqpoint{3.696000in}{3.696000in}}%
\pgfusepath{clip}%
\pgfsetrectcap%
\pgfsetroundjoin%
\pgfsetlinewidth{1.505625pt}%
\definecolor{currentstroke}{rgb}{1.000000,0.000000,0.000000}%
\pgfsetstrokecolor{currentstroke}%
\pgfsetdash{}{0pt}%
\pgfpathmoveto{\pgfqpoint{1.336871in}{1.209573in}}%
\pgfpathlineto{\pgfqpoint{1.568175in}{0.968392in}}%
\pgfusepath{stroke}%
\end{pgfscope}%
\begin{pgfscope}%
\pgfpathrectangle{\pgfqpoint{0.100000in}{0.220728in}}{\pgfqpoint{3.696000in}{3.696000in}}%
\pgfusepath{clip}%
\pgfsetrectcap%
\pgfsetroundjoin%
\pgfsetlinewidth{1.505625pt}%
\definecolor{currentstroke}{rgb}{1.000000,0.000000,0.000000}%
\pgfsetstrokecolor{currentstroke}%
\pgfsetdash{}{0pt}%
\pgfpathmoveto{\pgfqpoint{1.312961in}{1.226899in}}%
\pgfpathlineto{\pgfqpoint{1.568175in}{0.968392in}}%
\pgfusepath{stroke}%
\end{pgfscope}%
\begin{pgfscope}%
\pgfpathrectangle{\pgfqpoint{0.100000in}{0.220728in}}{\pgfqpoint{3.696000in}{3.696000in}}%
\pgfusepath{clip}%
\pgfsetrectcap%
\pgfsetroundjoin%
\pgfsetlinewidth{1.505625pt}%
\definecolor{currentstroke}{rgb}{1.000000,0.000000,0.000000}%
\pgfsetstrokecolor{currentstroke}%
\pgfsetdash{}{0pt}%
\pgfpathmoveto{\pgfqpoint{1.292798in}{1.241489in}}%
\pgfpathlineto{\pgfqpoint{1.568175in}{0.968392in}}%
\pgfusepath{stroke}%
\end{pgfscope}%
\begin{pgfscope}%
\pgfpathrectangle{\pgfqpoint{0.100000in}{0.220728in}}{\pgfqpoint{3.696000in}{3.696000in}}%
\pgfusepath{clip}%
\pgfsetrectcap%
\pgfsetroundjoin%
\pgfsetlinewidth{1.505625pt}%
\definecolor{currentstroke}{rgb}{1.000000,0.000000,0.000000}%
\pgfsetstrokecolor{currentstroke}%
\pgfsetdash{}{0pt}%
\pgfpathmoveto{\pgfqpoint{1.269508in}{1.292194in}}%
\pgfpathlineto{\pgfqpoint{1.568175in}{0.968392in}}%
\pgfusepath{stroke}%
\end{pgfscope}%
\begin{pgfscope}%
\pgfpathrectangle{\pgfqpoint{0.100000in}{0.220728in}}{\pgfqpoint{3.696000in}{3.696000in}}%
\pgfusepath{clip}%
\pgfsetrectcap%
\pgfsetroundjoin%
\pgfsetlinewidth{1.505625pt}%
\definecolor{currentstroke}{rgb}{1.000000,0.000000,0.000000}%
\pgfsetstrokecolor{currentstroke}%
\pgfsetdash{}{0pt}%
\pgfpathmoveto{\pgfqpoint{1.246074in}{1.327585in}}%
\pgfpathlineto{\pgfqpoint{1.568175in}{0.968392in}}%
\pgfusepath{stroke}%
\end{pgfscope}%
\begin{pgfscope}%
\pgfpathrectangle{\pgfqpoint{0.100000in}{0.220728in}}{\pgfqpoint{3.696000in}{3.696000in}}%
\pgfusepath{clip}%
\pgfsetrectcap%
\pgfsetroundjoin%
\pgfsetlinewidth{1.505625pt}%
\definecolor{currentstroke}{rgb}{1.000000,0.000000,0.000000}%
\pgfsetstrokecolor{currentstroke}%
\pgfsetdash{}{0pt}%
\pgfpathmoveto{\pgfqpoint{1.233964in}{1.327535in}}%
\pgfpathlineto{\pgfqpoint{1.579206in}{0.978569in}}%
\pgfusepath{stroke}%
\end{pgfscope}%
\begin{pgfscope}%
\pgfpathrectangle{\pgfqpoint{0.100000in}{0.220728in}}{\pgfqpoint{3.696000in}{3.696000in}}%
\pgfusepath{clip}%
\pgfsetrectcap%
\pgfsetroundjoin%
\pgfsetlinewidth{1.505625pt}%
\definecolor{currentstroke}{rgb}{1.000000,0.000000,0.000000}%
\pgfsetstrokecolor{currentstroke}%
\pgfsetdash{}{0pt}%
\pgfpathmoveto{\pgfqpoint{1.219730in}{1.329329in}}%
\pgfpathlineto{\pgfqpoint{1.579206in}{0.978569in}}%
\pgfusepath{stroke}%
\end{pgfscope}%
\begin{pgfscope}%
\pgfpathrectangle{\pgfqpoint{0.100000in}{0.220728in}}{\pgfqpoint{3.696000in}{3.696000in}}%
\pgfusepath{clip}%
\pgfsetrectcap%
\pgfsetroundjoin%
\pgfsetlinewidth{1.505625pt}%
\definecolor{currentstroke}{rgb}{1.000000,0.000000,0.000000}%
\pgfsetstrokecolor{currentstroke}%
\pgfsetdash{}{0pt}%
\pgfpathmoveto{\pgfqpoint{1.211523in}{1.340235in}}%
\pgfpathlineto{\pgfqpoint{1.579206in}{0.978569in}}%
\pgfusepath{stroke}%
\end{pgfscope}%
\begin{pgfscope}%
\pgfpathrectangle{\pgfqpoint{0.100000in}{0.220728in}}{\pgfqpoint{3.696000in}{3.696000in}}%
\pgfusepath{clip}%
\pgfsetrectcap%
\pgfsetroundjoin%
\pgfsetlinewidth{1.505625pt}%
\definecolor{currentstroke}{rgb}{1.000000,0.000000,0.000000}%
\pgfsetstrokecolor{currentstroke}%
\pgfsetdash{}{0pt}%
\pgfpathmoveto{\pgfqpoint{1.195941in}{1.386841in}}%
\pgfpathlineto{\pgfqpoint{1.568175in}{0.968392in}}%
\pgfusepath{stroke}%
\end{pgfscope}%
\begin{pgfscope}%
\pgfpathrectangle{\pgfqpoint{0.100000in}{0.220728in}}{\pgfqpoint{3.696000in}{3.696000in}}%
\pgfusepath{clip}%
\pgfsetrectcap%
\pgfsetroundjoin%
\pgfsetlinewidth{1.505625pt}%
\definecolor{currentstroke}{rgb}{1.000000,0.000000,0.000000}%
\pgfsetstrokecolor{currentstroke}%
\pgfsetdash{}{0pt}%
\pgfpathmoveto{\pgfqpoint{1.187500in}{1.398004in}}%
\pgfpathlineto{\pgfqpoint{1.557126in}{0.958197in}}%
\pgfusepath{stroke}%
\end{pgfscope}%
\begin{pgfscope}%
\pgfpathrectangle{\pgfqpoint{0.100000in}{0.220728in}}{\pgfqpoint{3.696000in}{3.696000in}}%
\pgfusepath{clip}%
\pgfsetrectcap%
\pgfsetroundjoin%
\pgfsetlinewidth{1.505625pt}%
\definecolor{currentstroke}{rgb}{1.000000,0.000000,0.000000}%
\pgfsetstrokecolor{currentstroke}%
\pgfsetdash{}{0pt}%
\pgfpathmoveto{\pgfqpoint{1.180542in}{1.417909in}}%
\pgfpathlineto{\pgfqpoint{1.568175in}{0.968392in}}%
\pgfusepath{stroke}%
\end{pgfscope}%
\begin{pgfscope}%
\pgfpathrectangle{\pgfqpoint{0.100000in}{0.220728in}}{\pgfqpoint{3.696000in}{3.696000in}}%
\pgfusepath{clip}%
\pgfsetrectcap%
\pgfsetroundjoin%
\pgfsetlinewidth{1.505625pt}%
\definecolor{currentstroke}{rgb}{1.000000,0.000000,0.000000}%
\pgfsetstrokecolor{currentstroke}%
\pgfsetdash{}{0pt}%
\pgfpathmoveto{\pgfqpoint{1.170096in}{1.419948in}}%
\pgfpathlineto{\pgfqpoint{1.568175in}{0.968392in}}%
\pgfusepath{stroke}%
\end{pgfscope}%
\begin{pgfscope}%
\pgfpathrectangle{\pgfqpoint{0.100000in}{0.220728in}}{\pgfqpoint{3.696000in}{3.696000in}}%
\pgfusepath{clip}%
\pgfsetrectcap%
\pgfsetroundjoin%
\pgfsetlinewidth{1.505625pt}%
\definecolor{currentstroke}{rgb}{1.000000,0.000000,0.000000}%
\pgfsetstrokecolor{currentstroke}%
\pgfsetdash{}{0pt}%
\pgfpathmoveto{\pgfqpoint{1.155701in}{1.452953in}}%
\pgfpathlineto{\pgfqpoint{1.568175in}{0.968392in}}%
\pgfusepath{stroke}%
\end{pgfscope}%
\begin{pgfscope}%
\pgfpathrectangle{\pgfqpoint{0.100000in}{0.220728in}}{\pgfqpoint{3.696000in}{3.696000in}}%
\pgfusepath{clip}%
\pgfsetrectcap%
\pgfsetroundjoin%
\pgfsetlinewidth{1.505625pt}%
\definecolor{currentstroke}{rgb}{1.000000,0.000000,0.000000}%
\pgfsetstrokecolor{currentstroke}%
\pgfsetdash{}{0pt}%
\pgfpathmoveto{\pgfqpoint{1.139349in}{1.452326in}}%
\pgfpathlineto{\pgfqpoint{1.568175in}{0.968392in}}%
\pgfusepath{stroke}%
\end{pgfscope}%
\begin{pgfscope}%
\pgfpathrectangle{\pgfqpoint{0.100000in}{0.220728in}}{\pgfqpoint{3.696000in}{3.696000in}}%
\pgfusepath{clip}%
\pgfsetrectcap%
\pgfsetroundjoin%
\pgfsetlinewidth{1.505625pt}%
\definecolor{currentstroke}{rgb}{1.000000,0.000000,0.000000}%
\pgfsetstrokecolor{currentstroke}%
\pgfsetdash{}{0pt}%
\pgfpathmoveto{\pgfqpoint{1.117715in}{1.480104in}}%
\pgfpathlineto{\pgfqpoint{1.568175in}{0.968392in}}%
\pgfusepath{stroke}%
\end{pgfscope}%
\begin{pgfscope}%
\pgfpathrectangle{\pgfqpoint{0.100000in}{0.220728in}}{\pgfqpoint{3.696000in}{3.696000in}}%
\pgfusepath{clip}%
\pgfsetrectcap%
\pgfsetroundjoin%
\pgfsetlinewidth{1.505625pt}%
\definecolor{currentstroke}{rgb}{1.000000,0.000000,0.000000}%
\pgfsetstrokecolor{currentstroke}%
\pgfsetdash{}{0pt}%
\pgfpathmoveto{\pgfqpoint{1.098595in}{1.487029in}}%
\pgfpathlineto{\pgfqpoint{1.568175in}{0.968392in}}%
\pgfusepath{stroke}%
\end{pgfscope}%
\begin{pgfscope}%
\pgfpathrectangle{\pgfqpoint{0.100000in}{0.220728in}}{\pgfqpoint{3.696000in}{3.696000in}}%
\pgfusepath{clip}%
\pgfsetrectcap%
\pgfsetroundjoin%
\pgfsetlinewidth{1.505625pt}%
\definecolor{currentstroke}{rgb}{1.000000,0.000000,0.000000}%
\pgfsetstrokecolor{currentstroke}%
\pgfsetdash{}{0pt}%
\pgfpathmoveto{\pgfqpoint{1.074220in}{1.518721in}}%
\pgfpathlineto{\pgfqpoint{1.568175in}{0.968392in}}%
\pgfusepath{stroke}%
\end{pgfscope}%
\begin{pgfscope}%
\pgfpathrectangle{\pgfqpoint{0.100000in}{0.220728in}}{\pgfqpoint{3.696000in}{3.696000in}}%
\pgfusepath{clip}%
\pgfsetrectcap%
\pgfsetroundjoin%
\pgfsetlinewidth{1.505625pt}%
\definecolor{currentstroke}{rgb}{1.000000,0.000000,0.000000}%
\pgfsetstrokecolor{currentstroke}%
\pgfsetdash{}{0pt}%
\pgfpathmoveto{\pgfqpoint{1.062483in}{1.519983in}}%
\pgfpathlineto{\pgfqpoint{1.568175in}{0.968392in}}%
\pgfusepath{stroke}%
\end{pgfscope}%
\begin{pgfscope}%
\pgfpathrectangle{\pgfqpoint{0.100000in}{0.220728in}}{\pgfqpoint{3.696000in}{3.696000in}}%
\pgfusepath{clip}%
\pgfsetrectcap%
\pgfsetroundjoin%
\pgfsetlinewidth{1.505625pt}%
\definecolor{currentstroke}{rgb}{1.000000,0.000000,0.000000}%
\pgfsetstrokecolor{currentstroke}%
\pgfsetdash{}{0pt}%
\pgfpathmoveto{\pgfqpoint{1.053712in}{1.526448in}}%
\pgfpathlineto{\pgfqpoint{1.568175in}{0.968392in}}%
\pgfusepath{stroke}%
\end{pgfscope}%
\begin{pgfscope}%
\pgfpathrectangle{\pgfqpoint{0.100000in}{0.220728in}}{\pgfqpoint{3.696000in}{3.696000in}}%
\pgfusepath{clip}%
\pgfsetrectcap%
\pgfsetroundjoin%
\pgfsetlinewidth{1.505625pt}%
\definecolor{currentstroke}{rgb}{1.000000,0.000000,0.000000}%
\pgfsetstrokecolor{currentstroke}%
\pgfsetdash{}{0pt}%
\pgfpathmoveto{\pgfqpoint{1.049771in}{1.523354in}}%
\pgfpathlineto{\pgfqpoint{1.568175in}{0.968392in}}%
\pgfusepath{stroke}%
\end{pgfscope}%
\begin{pgfscope}%
\pgfpathrectangle{\pgfqpoint{0.100000in}{0.220728in}}{\pgfqpoint{3.696000in}{3.696000in}}%
\pgfusepath{clip}%
\pgfsetrectcap%
\pgfsetroundjoin%
\pgfsetlinewidth{1.505625pt}%
\definecolor{currentstroke}{rgb}{1.000000,0.000000,0.000000}%
\pgfsetstrokecolor{currentstroke}%
\pgfsetdash{}{0pt}%
\pgfpathmoveto{\pgfqpoint{1.047843in}{1.526003in}}%
\pgfpathlineto{\pgfqpoint{1.568175in}{0.968392in}}%
\pgfusepath{stroke}%
\end{pgfscope}%
\begin{pgfscope}%
\pgfpathrectangle{\pgfqpoint{0.100000in}{0.220728in}}{\pgfqpoint{3.696000in}{3.696000in}}%
\pgfusepath{clip}%
\pgfsetrectcap%
\pgfsetroundjoin%
\pgfsetlinewidth{1.505625pt}%
\definecolor{currentstroke}{rgb}{1.000000,0.000000,0.000000}%
\pgfsetstrokecolor{currentstroke}%
\pgfsetdash{}{0pt}%
\pgfpathmoveto{\pgfqpoint{1.043939in}{1.538484in}}%
\pgfpathlineto{\pgfqpoint{1.557126in}{0.958197in}}%
\pgfusepath{stroke}%
\end{pgfscope}%
\begin{pgfscope}%
\pgfpathrectangle{\pgfqpoint{0.100000in}{0.220728in}}{\pgfqpoint{3.696000in}{3.696000in}}%
\pgfusepath{clip}%
\pgfsetrectcap%
\pgfsetroundjoin%
\pgfsetlinewidth{1.505625pt}%
\definecolor{currentstroke}{rgb}{1.000000,0.000000,0.000000}%
\pgfsetstrokecolor{currentstroke}%
\pgfsetdash{}{0pt}%
\pgfpathmoveto{\pgfqpoint{1.038847in}{1.537514in}}%
\pgfpathlineto{\pgfqpoint{1.557126in}{0.958197in}}%
\pgfusepath{stroke}%
\end{pgfscope}%
\begin{pgfscope}%
\pgfpathrectangle{\pgfqpoint{0.100000in}{0.220728in}}{\pgfqpoint{3.696000in}{3.696000in}}%
\pgfusepath{clip}%
\pgfsetrectcap%
\pgfsetroundjoin%
\pgfsetlinewidth{1.505625pt}%
\definecolor{currentstroke}{rgb}{1.000000,0.000000,0.000000}%
\pgfsetstrokecolor{currentstroke}%
\pgfsetdash{}{0pt}%
\pgfpathmoveto{\pgfqpoint{1.029908in}{1.546605in}}%
\pgfpathlineto{\pgfqpoint{1.557126in}{0.958197in}}%
\pgfusepath{stroke}%
\end{pgfscope}%
\begin{pgfscope}%
\pgfpathrectangle{\pgfqpoint{0.100000in}{0.220728in}}{\pgfqpoint{3.696000in}{3.696000in}}%
\pgfusepath{clip}%
\pgfsetrectcap%
\pgfsetroundjoin%
\pgfsetlinewidth{1.505625pt}%
\definecolor{currentstroke}{rgb}{1.000000,0.000000,0.000000}%
\pgfsetstrokecolor{currentstroke}%
\pgfsetdash{}{0pt}%
\pgfpathmoveto{\pgfqpoint{1.019497in}{1.548322in}}%
\pgfpathlineto{\pgfqpoint{1.557126in}{0.958197in}}%
\pgfusepath{stroke}%
\end{pgfscope}%
\begin{pgfscope}%
\pgfpathrectangle{\pgfqpoint{0.100000in}{0.220728in}}{\pgfqpoint{3.696000in}{3.696000in}}%
\pgfusepath{clip}%
\pgfsetrectcap%
\pgfsetroundjoin%
\pgfsetlinewidth{1.505625pt}%
\definecolor{currentstroke}{rgb}{1.000000,0.000000,0.000000}%
\pgfsetstrokecolor{currentstroke}%
\pgfsetdash{}{0pt}%
\pgfpathmoveto{\pgfqpoint{1.000066in}{1.566684in}}%
\pgfpathlineto{\pgfqpoint{1.557126in}{0.958197in}}%
\pgfusepath{stroke}%
\end{pgfscope}%
\begin{pgfscope}%
\pgfpathrectangle{\pgfqpoint{0.100000in}{0.220728in}}{\pgfqpoint{3.696000in}{3.696000in}}%
\pgfusepath{clip}%
\pgfsetrectcap%
\pgfsetroundjoin%
\pgfsetlinewidth{1.505625pt}%
\definecolor{currentstroke}{rgb}{1.000000,0.000000,0.000000}%
\pgfsetstrokecolor{currentstroke}%
\pgfsetdash{}{0pt}%
\pgfpathmoveto{\pgfqpoint{0.984836in}{1.548205in}}%
\pgfpathlineto{\pgfqpoint{1.557126in}{0.958197in}}%
\pgfusepath{stroke}%
\end{pgfscope}%
\begin{pgfscope}%
\pgfpathrectangle{\pgfqpoint{0.100000in}{0.220728in}}{\pgfqpoint{3.696000in}{3.696000in}}%
\pgfusepath{clip}%
\pgfsetrectcap%
\pgfsetroundjoin%
\pgfsetlinewidth{1.505625pt}%
\definecolor{currentstroke}{rgb}{1.000000,0.000000,0.000000}%
\pgfsetstrokecolor{currentstroke}%
\pgfsetdash{}{0pt}%
\pgfpathmoveto{\pgfqpoint{0.965413in}{1.575841in}}%
\pgfpathlineto{\pgfqpoint{1.557126in}{0.958197in}}%
\pgfusepath{stroke}%
\end{pgfscope}%
\begin{pgfscope}%
\pgfpathrectangle{\pgfqpoint{0.100000in}{0.220728in}}{\pgfqpoint{3.696000in}{3.696000in}}%
\pgfusepath{clip}%
\pgfsetrectcap%
\pgfsetroundjoin%
\pgfsetlinewidth{1.505625pt}%
\definecolor{currentstroke}{rgb}{1.000000,0.000000,0.000000}%
\pgfsetstrokecolor{currentstroke}%
\pgfsetdash{}{0pt}%
\pgfpathmoveto{\pgfqpoint{0.955844in}{1.570650in}}%
\pgfpathlineto{\pgfqpoint{1.557126in}{0.958197in}}%
\pgfusepath{stroke}%
\end{pgfscope}%
\begin{pgfscope}%
\pgfpathrectangle{\pgfqpoint{0.100000in}{0.220728in}}{\pgfqpoint{3.696000in}{3.696000in}}%
\pgfusepath{clip}%
\pgfsetrectcap%
\pgfsetroundjoin%
\pgfsetlinewidth{1.505625pt}%
\definecolor{currentstroke}{rgb}{1.000000,0.000000,0.000000}%
\pgfsetstrokecolor{currentstroke}%
\pgfsetdash{}{0pt}%
\pgfpathmoveto{\pgfqpoint{0.950665in}{1.570346in}}%
\pgfpathlineto{\pgfqpoint{1.557126in}{0.958197in}}%
\pgfusepath{stroke}%
\end{pgfscope}%
\begin{pgfscope}%
\pgfpathrectangle{\pgfqpoint{0.100000in}{0.220728in}}{\pgfqpoint{3.696000in}{3.696000in}}%
\pgfusepath{clip}%
\pgfsetrectcap%
\pgfsetroundjoin%
\pgfsetlinewidth{1.505625pt}%
\definecolor{currentstroke}{rgb}{1.000000,0.000000,0.000000}%
\pgfsetstrokecolor{currentstroke}%
\pgfsetdash{}{0pt}%
\pgfpathmoveto{\pgfqpoint{0.943353in}{1.569048in}}%
\pgfpathlineto{\pgfqpoint{1.557126in}{0.958197in}}%
\pgfusepath{stroke}%
\end{pgfscope}%
\begin{pgfscope}%
\pgfpathrectangle{\pgfqpoint{0.100000in}{0.220728in}}{\pgfqpoint{3.696000in}{3.696000in}}%
\pgfusepath{clip}%
\pgfsetrectcap%
\pgfsetroundjoin%
\pgfsetlinewidth{1.505625pt}%
\definecolor{currentstroke}{rgb}{1.000000,0.000000,0.000000}%
\pgfsetstrokecolor{currentstroke}%
\pgfsetdash{}{0pt}%
\pgfpathmoveto{\pgfqpoint{0.939827in}{1.570655in}}%
\pgfpathlineto{\pgfqpoint{1.557126in}{0.958197in}}%
\pgfusepath{stroke}%
\end{pgfscope}%
\begin{pgfscope}%
\pgfpathrectangle{\pgfqpoint{0.100000in}{0.220728in}}{\pgfqpoint{3.696000in}{3.696000in}}%
\pgfusepath{clip}%
\pgfsetrectcap%
\pgfsetroundjoin%
\pgfsetlinewidth{1.505625pt}%
\definecolor{currentstroke}{rgb}{1.000000,0.000000,0.000000}%
\pgfsetstrokecolor{currentstroke}%
\pgfsetdash{}{0pt}%
\pgfpathmoveto{\pgfqpoint{0.935072in}{1.572308in}}%
\pgfpathlineto{\pgfqpoint{1.557126in}{0.958197in}}%
\pgfusepath{stroke}%
\end{pgfscope}%
\begin{pgfscope}%
\pgfpathrectangle{\pgfqpoint{0.100000in}{0.220728in}}{\pgfqpoint{3.696000in}{3.696000in}}%
\pgfusepath{clip}%
\pgfsetrectcap%
\pgfsetroundjoin%
\pgfsetlinewidth{1.505625pt}%
\definecolor{currentstroke}{rgb}{1.000000,0.000000,0.000000}%
\pgfsetstrokecolor{currentstroke}%
\pgfsetdash{}{0pt}%
\pgfpathmoveto{\pgfqpoint{0.929475in}{1.575333in}}%
\pgfpathlineto{\pgfqpoint{1.557126in}{0.958197in}}%
\pgfusepath{stroke}%
\end{pgfscope}%
\begin{pgfscope}%
\pgfpathrectangle{\pgfqpoint{0.100000in}{0.220728in}}{\pgfqpoint{3.696000in}{3.696000in}}%
\pgfusepath{clip}%
\pgfsetrectcap%
\pgfsetroundjoin%
\pgfsetlinewidth{1.505625pt}%
\definecolor{currentstroke}{rgb}{1.000000,0.000000,0.000000}%
\pgfsetstrokecolor{currentstroke}%
\pgfsetdash{}{0pt}%
\pgfpathmoveto{\pgfqpoint{0.927168in}{1.576220in}}%
\pgfpathlineto{\pgfqpoint{1.557126in}{0.958197in}}%
\pgfusepath{stroke}%
\end{pgfscope}%
\begin{pgfscope}%
\pgfpathrectangle{\pgfqpoint{0.100000in}{0.220728in}}{\pgfqpoint{3.696000in}{3.696000in}}%
\pgfusepath{clip}%
\pgfsetrectcap%
\pgfsetroundjoin%
\pgfsetlinewidth{1.505625pt}%
\definecolor{currentstroke}{rgb}{1.000000,0.000000,0.000000}%
\pgfsetstrokecolor{currentstroke}%
\pgfsetdash{}{0pt}%
\pgfpathmoveto{\pgfqpoint{0.923956in}{1.576385in}}%
\pgfpathlineto{\pgfqpoint{1.557126in}{0.958197in}}%
\pgfusepath{stroke}%
\end{pgfscope}%
\begin{pgfscope}%
\pgfpathrectangle{\pgfqpoint{0.100000in}{0.220728in}}{\pgfqpoint{3.696000in}{3.696000in}}%
\pgfusepath{clip}%
\pgfsetrectcap%
\pgfsetroundjoin%
\pgfsetlinewidth{1.505625pt}%
\definecolor{currentstroke}{rgb}{1.000000,0.000000,0.000000}%
\pgfsetstrokecolor{currentstroke}%
\pgfsetdash{}{0pt}%
\pgfpathmoveto{\pgfqpoint{0.917952in}{1.583795in}}%
\pgfpathlineto{\pgfqpoint{1.557126in}{0.958197in}}%
\pgfusepath{stroke}%
\end{pgfscope}%
\begin{pgfscope}%
\pgfpathrectangle{\pgfqpoint{0.100000in}{0.220728in}}{\pgfqpoint{3.696000in}{3.696000in}}%
\pgfusepath{clip}%
\pgfsetrectcap%
\pgfsetroundjoin%
\pgfsetlinewidth{1.505625pt}%
\definecolor{currentstroke}{rgb}{1.000000,0.000000,0.000000}%
\pgfsetstrokecolor{currentstroke}%
\pgfsetdash{}{0pt}%
\pgfpathmoveto{\pgfqpoint{0.910138in}{1.592689in}}%
\pgfpathlineto{\pgfqpoint{1.557126in}{0.958197in}}%
\pgfusepath{stroke}%
\end{pgfscope}%
\begin{pgfscope}%
\pgfpathrectangle{\pgfqpoint{0.100000in}{0.220728in}}{\pgfqpoint{3.696000in}{3.696000in}}%
\pgfusepath{clip}%
\pgfsetrectcap%
\pgfsetroundjoin%
\pgfsetlinewidth{1.505625pt}%
\definecolor{currentstroke}{rgb}{1.000000,0.000000,0.000000}%
\pgfsetstrokecolor{currentstroke}%
\pgfsetdash{}{0pt}%
\pgfpathmoveto{\pgfqpoint{0.906763in}{1.590345in}}%
\pgfpathlineto{\pgfqpoint{1.568175in}{0.968392in}}%
\pgfusepath{stroke}%
\end{pgfscope}%
\begin{pgfscope}%
\pgfpathrectangle{\pgfqpoint{0.100000in}{0.220728in}}{\pgfqpoint{3.696000in}{3.696000in}}%
\pgfusepath{clip}%
\pgfsetrectcap%
\pgfsetroundjoin%
\pgfsetlinewidth{1.505625pt}%
\definecolor{currentstroke}{rgb}{1.000000,0.000000,0.000000}%
\pgfsetstrokecolor{currentstroke}%
\pgfsetdash{}{0pt}%
\pgfpathmoveto{\pgfqpoint{0.897555in}{1.592339in}}%
\pgfpathlineto{\pgfqpoint{1.568175in}{0.968392in}}%
\pgfusepath{stroke}%
\end{pgfscope}%
\begin{pgfscope}%
\pgfpathrectangle{\pgfqpoint{0.100000in}{0.220728in}}{\pgfqpoint{3.696000in}{3.696000in}}%
\pgfusepath{clip}%
\pgfsetrectcap%
\pgfsetroundjoin%
\pgfsetlinewidth{1.505625pt}%
\definecolor{currentstroke}{rgb}{1.000000,0.000000,0.000000}%
\pgfsetstrokecolor{currentstroke}%
\pgfsetdash{}{0pt}%
\pgfpathmoveto{\pgfqpoint{0.887394in}{1.594502in}}%
\pgfpathlineto{\pgfqpoint{1.568175in}{0.968392in}}%
\pgfusepath{stroke}%
\end{pgfscope}%
\begin{pgfscope}%
\pgfpathrectangle{\pgfqpoint{0.100000in}{0.220728in}}{\pgfqpoint{3.696000in}{3.696000in}}%
\pgfusepath{clip}%
\pgfsetrectcap%
\pgfsetroundjoin%
\pgfsetlinewidth{1.505625pt}%
\definecolor{currentstroke}{rgb}{1.000000,0.000000,0.000000}%
\pgfsetstrokecolor{currentstroke}%
\pgfsetdash{}{0pt}%
\pgfpathmoveto{\pgfqpoint{0.878403in}{1.597834in}}%
\pgfpathlineto{\pgfqpoint{1.568175in}{0.968392in}}%
\pgfusepath{stroke}%
\end{pgfscope}%
\begin{pgfscope}%
\pgfpathrectangle{\pgfqpoint{0.100000in}{0.220728in}}{\pgfqpoint{3.696000in}{3.696000in}}%
\pgfusepath{clip}%
\pgfsetrectcap%
\pgfsetroundjoin%
\pgfsetlinewidth{1.505625pt}%
\definecolor{currentstroke}{rgb}{1.000000,0.000000,0.000000}%
\pgfsetstrokecolor{currentstroke}%
\pgfsetdash{}{0pt}%
\pgfpathmoveto{\pgfqpoint{0.864098in}{1.614470in}}%
\pgfpathlineto{\pgfqpoint{1.568175in}{0.968392in}}%
\pgfusepath{stroke}%
\end{pgfscope}%
\begin{pgfscope}%
\pgfpathrectangle{\pgfqpoint{0.100000in}{0.220728in}}{\pgfqpoint{3.696000in}{3.696000in}}%
\pgfusepath{clip}%
\pgfsetrectcap%
\pgfsetroundjoin%
\pgfsetlinewidth{1.505625pt}%
\definecolor{currentstroke}{rgb}{1.000000,0.000000,0.000000}%
\pgfsetstrokecolor{currentstroke}%
\pgfsetdash{}{0pt}%
\pgfpathmoveto{\pgfqpoint{0.850295in}{1.614697in}}%
\pgfpathlineto{\pgfqpoint{1.568175in}{0.968392in}}%
\pgfusepath{stroke}%
\end{pgfscope}%
\begin{pgfscope}%
\pgfpathrectangle{\pgfqpoint{0.100000in}{0.220728in}}{\pgfqpoint{3.696000in}{3.696000in}}%
\pgfusepath{clip}%
\pgfsetrectcap%
\pgfsetroundjoin%
\pgfsetlinewidth{1.505625pt}%
\definecolor{currentstroke}{rgb}{1.000000,0.000000,0.000000}%
\pgfsetstrokecolor{currentstroke}%
\pgfsetdash{}{0pt}%
\pgfpathmoveto{\pgfqpoint{0.838496in}{1.620549in}}%
\pgfpathlineto{\pgfqpoint{1.568175in}{0.968392in}}%
\pgfusepath{stroke}%
\end{pgfscope}%
\begin{pgfscope}%
\pgfpathrectangle{\pgfqpoint{0.100000in}{0.220728in}}{\pgfqpoint{3.696000in}{3.696000in}}%
\pgfusepath{clip}%
\pgfsetrectcap%
\pgfsetroundjoin%
\pgfsetlinewidth{1.505625pt}%
\definecolor{currentstroke}{rgb}{1.000000,0.000000,0.000000}%
\pgfsetstrokecolor{currentstroke}%
\pgfsetdash{}{0pt}%
\pgfpathmoveto{\pgfqpoint{0.823535in}{1.626061in}}%
\pgfpathlineto{\pgfqpoint{1.579206in}{0.978569in}}%
\pgfusepath{stroke}%
\end{pgfscope}%
\begin{pgfscope}%
\pgfpathrectangle{\pgfqpoint{0.100000in}{0.220728in}}{\pgfqpoint{3.696000in}{3.696000in}}%
\pgfusepath{clip}%
\pgfsetrectcap%
\pgfsetroundjoin%
\pgfsetlinewidth{1.505625pt}%
\definecolor{currentstroke}{rgb}{1.000000,0.000000,0.000000}%
\pgfsetstrokecolor{currentstroke}%
\pgfsetdash{}{0pt}%
\pgfpathmoveto{\pgfqpoint{0.808058in}{1.634183in}}%
\pgfpathlineto{\pgfqpoint{1.579206in}{0.978569in}}%
\pgfusepath{stroke}%
\end{pgfscope}%
\begin{pgfscope}%
\pgfpathrectangle{\pgfqpoint{0.100000in}{0.220728in}}{\pgfqpoint{3.696000in}{3.696000in}}%
\pgfusepath{clip}%
\pgfsetrectcap%
\pgfsetroundjoin%
\pgfsetlinewidth{1.505625pt}%
\definecolor{currentstroke}{rgb}{1.000000,0.000000,0.000000}%
\pgfsetstrokecolor{currentstroke}%
\pgfsetdash{}{0pt}%
\pgfpathmoveto{\pgfqpoint{0.788542in}{1.625623in}}%
\pgfpathlineto{\pgfqpoint{1.579206in}{0.978569in}}%
\pgfusepath{stroke}%
\end{pgfscope}%
\begin{pgfscope}%
\pgfpathrectangle{\pgfqpoint{0.100000in}{0.220728in}}{\pgfqpoint{3.696000in}{3.696000in}}%
\pgfusepath{clip}%
\pgfsetrectcap%
\pgfsetroundjoin%
\pgfsetlinewidth{1.505625pt}%
\definecolor{currentstroke}{rgb}{1.000000,0.000000,0.000000}%
\pgfsetstrokecolor{currentstroke}%
\pgfsetdash{}{0pt}%
\pgfpathmoveto{\pgfqpoint{0.768747in}{1.619331in}}%
\pgfpathlineto{\pgfqpoint{1.579206in}{0.978569in}}%
\pgfusepath{stroke}%
\end{pgfscope}%
\begin{pgfscope}%
\pgfpathrectangle{\pgfqpoint{0.100000in}{0.220728in}}{\pgfqpoint{3.696000in}{3.696000in}}%
\pgfusepath{clip}%
\pgfsetrectcap%
\pgfsetroundjoin%
\pgfsetlinewidth{1.505625pt}%
\definecolor{currentstroke}{rgb}{1.000000,0.000000,0.000000}%
\pgfsetstrokecolor{currentstroke}%
\pgfsetdash{}{0pt}%
\pgfpathmoveto{\pgfqpoint{0.746019in}{1.605191in}}%
\pgfpathlineto{\pgfqpoint{1.579206in}{0.978569in}}%
\pgfusepath{stroke}%
\end{pgfscope}%
\begin{pgfscope}%
\pgfpathrectangle{\pgfqpoint{0.100000in}{0.220728in}}{\pgfqpoint{3.696000in}{3.696000in}}%
\pgfusepath{clip}%
\pgfsetrectcap%
\pgfsetroundjoin%
\pgfsetlinewidth{1.505625pt}%
\definecolor{currentstroke}{rgb}{1.000000,0.000000,0.000000}%
\pgfsetstrokecolor{currentstroke}%
\pgfsetdash{}{0pt}%
\pgfpathmoveto{\pgfqpoint{0.734390in}{1.603401in}}%
\pgfpathlineto{\pgfqpoint{1.579206in}{0.978569in}}%
\pgfusepath{stroke}%
\end{pgfscope}%
\begin{pgfscope}%
\pgfpathrectangle{\pgfqpoint{0.100000in}{0.220728in}}{\pgfqpoint{3.696000in}{3.696000in}}%
\pgfusepath{clip}%
\pgfsetrectcap%
\pgfsetroundjoin%
\pgfsetlinewidth{1.505625pt}%
\definecolor{currentstroke}{rgb}{1.000000,0.000000,0.000000}%
\pgfsetstrokecolor{currentstroke}%
\pgfsetdash{}{0pt}%
\pgfpathmoveto{\pgfqpoint{0.727097in}{1.594463in}}%
\pgfpathlineto{\pgfqpoint{1.579206in}{0.978569in}}%
\pgfusepath{stroke}%
\end{pgfscope}%
\begin{pgfscope}%
\pgfpathrectangle{\pgfqpoint{0.100000in}{0.220728in}}{\pgfqpoint{3.696000in}{3.696000in}}%
\pgfusepath{clip}%
\pgfsetrectcap%
\pgfsetroundjoin%
\pgfsetlinewidth{1.505625pt}%
\definecolor{currentstroke}{rgb}{1.000000,0.000000,0.000000}%
\pgfsetstrokecolor{currentstroke}%
\pgfsetdash{}{0pt}%
\pgfpathmoveto{\pgfqpoint{0.723353in}{1.591919in}}%
\pgfpathlineto{\pgfqpoint{1.579206in}{0.978569in}}%
\pgfusepath{stroke}%
\end{pgfscope}%
\begin{pgfscope}%
\pgfpathrectangle{\pgfqpoint{0.100000in}{0.220728in}}{\pgfqpoint{3.696000in}{3.696000in}}%
\pgfusepath{clip}%
\pgfsetrectcap%
\pgfsetroundjoin%
\pgfsetlinewidth{1.505625pt}%
\definecolor{currentstroke}{rgb}{1.000000,0.000000,0.000000}%
\pgfsetstrokecolor{currentstroke}%
\pgfsetdash{}{0pt}%
\pgfpathmoveto{\pgfqpoint{0.721073in}{1.588656in}}%
\pgfpathlineto{\pgfqpoint{1.579206in}{0.978569in}}%
\pgfusepath{stroke}%
\end{pgfscope}%
\begin{pgfscope}%
\pgfpathrectangle{\pgfqpoint{0.100000in}{0.220728in}}{\pgfqpoint{3.696000in}{3.696000in}}%
\pgfusepath{clip}%
\pgfsetrectcap%
\pgfsetroundjoin%
\pgfsetlinewidth{1.505625pt}%
\definecolor{currentstroke}{rgb}{1.000000,0.000000,0.000000}%
\pgfsetstrokecolor{currentstroke}%
\pgfsetdash{}{0pt}%
\pgfpathmoveto{\pgfqpoint{0.719973in}{1.588144in}}%
\pgfpathlineto{\pgfqpoint{1.579206in}{0.978569in}}%
\pgfusepath{stroke}%
\end{pgfscope}%
\begin{pgfscope}%
\pgfpathrectangle{\pgfqpoint{0.100000in}{0.220728in}}{\pgfqpoint{3.696000in}{3.696000in}}%
\pgfusepath{clip}%
\pgfsetrectcap%
\pgfsetroundjoin%
\pgfsetlinewidth{1.505625pt}%
\definecolor{currentstroke}{rgb}{1.000000,0.000000,0.000000}%
\pgfsetstrokecolor{currentstroke}%
\pgfsetdash{}{0pt}%
\pgfpathmoveto{\pgfqpoint{0.719304in}{1.587350in}}%
\pgfpathlineto{\pgfqpoint{1.579206in}{0.978569in}}%
\pgfusepath{stroke}%
\end{pgfscope}%
\begin{pgfscope}%
\pgfpathrectangle{\pgfqpoint{0.100000in}{0.220728in}}{\pgfqpoint{3.696000in}{3.696000in}}%
\pgfusepath{clip}%
\pgfsetrectcap%
\pgfsetroundjoin%
\pgfsetlinewidth{1.505625pt}%
\definecolor{currentstroke}{rgb}{1.000000,0.000000,0.000000}%
\pgfsetstrokecolor{currentstroke}%
\pgfsetdash{}{0pt}%
\pgfpathmoveto{\pgfqpoint{0.718999in}{1.587287in}}%
\pgfpathlineto{\pgfqpoint{1.579206in}{0.978569in}}%
\pgfusepath{stroke}%
\end{pgfscope}%
\begin{pgfscope}%
\pgfpathrectangle{\pgfqpoint{0.100000in}{0.220728in}}{\pgfqpoint{3.696000in}{3.696000in}}%
\pgfusepath{clip}%
\pgfsetrectcap%
\pgfsetroundjoin%
\pgfsetlinewidth{1.505625pt}%
\definecolor{currentstroke}{rgb}{1.000000,0.000000,0.000000}%
\pgfsetstrokecolor{currentstroke}%
\pgfsetdash{}{0pt}%
\pgfpathmoveto{\pgfqpoint{0.718805in}{1.587118in}}%
\pgfpathlineto{\pgfqpoint{1.579206in}{0.978569in}}%
\pgfusepath{stroke}%
\end{pgfscope}%
\begin{pgfscope}%
\pgfpathrectangle{\pgfqpoint{0.100000in}{0.220728in}}{\pgfqpoint{3.696000in}{3.696000in}}%
\pgfusepath{clip}%
\pgfsetrectcap%
\pgfsetroundjoin%
\pgfsetlinewidth{1.505625pt}%
\definecolor{currentstroke}{rgb}{1.000000,0.000000,0.000000}%
\pgfsetstrokecolor{currentstroke}%
\pgfsetdash{}{0pt}%
\pgfpathmoveto{\pgfqpoint{0.718693in}{1.586967in}}%
\pgfpathlineto{\pgfqpoint{1.579206in}{0.978569in}}%
\pgfusepath{stroke}%
\end{pgfscope}%
\begin{pgfscope}%
\pgfpathrectangle{\pgfqpoint{0.100000in}{0.220728in}}{\pgfqpoint{3.696000in}{3.696000in}}%
\pgfusepath{clip}%
\pgfsetrectcap%
\pgfsetroundjoin%
\pgfsetlinewidth{1.505625pt}%
\definecolor{currentstroke}{rgb}{1.000000,0.000000,0.000000}%
\pgfsetstrokecolor{currentstroke}%
\pgfsetdash{}{0pt}%
\pgfpathmoveto{\pgfqpoint{0.718634in}{1.586909in}}%
\pgfpathlineto{\pgfqpoint{1.579206in}{0.978569in}}%
\pgfusepath{stroke}%
\end{pgfscope}%
\begin{pgfscope}%
\pgfpathrectangle{\pgfqpoint{0.100000in}{0.220728in}}{\pgfqpoint{3.696000in}{3.696000in}}%
\pgfusepath{clip}%
\pgfsetrectcap%
\pgfsetroundjoin%
\pgfsetlinewidth{1.505625pt}%
\definecolor{currentstroke}{rgb}{1.000000,0.000000,0.000000}%
\pgfsetstrokecolor{currentstroke}%
\pgfsetdash{}{0pt}%
\pgfpathmoveto{\pgfqpoint{0.718600in}{1.586858in}}%
\pgfpathlineto{\pgfqpoint{1.579206in}{0.978569in}}%
\pgfusepath{stroke}%
\end{pgfscope}%
\begin{pgfscope}%
\pgfpathrectangle{\pgfqpoint{0.100000in}{0.220728in}}{\pgfqpoint{3.696000in}{3.696000in}}%
\pgfusepath{clip}%
\pgfsetrectcap%
\pgfsetroundjoin%
\pgfsetlinewidth{1.505625pt}%
\definecolor{currentstroke}{rgb}{1.000000,0.000000,0.000000}%
\pgfsetstrokecolor{currentstroke}%
\pgfsetdash{}{0pt}%
\pgfpathmoveto{\pgfqpoint{0.718583in}{1.586846in}}%
\pgfpathlineto{\pgfqpoint{1.579206in}{0.978569in}}%
\pgfusepath{stroke}%
\end{pgfscope}%
\begin{pgfscope}%
\pgfpathrectangle{\pgfqpoint{0.100000in}{0.220728in}}{\pgfqpoint{3.696000in}{3.696000in}}%
\pgfusepath{clip}%
\pgfsetrectcap%
\pgfsetroundjoin%
\pgfsetlinewidth{1.505625pt}%
\definecolor{currentstroke}{rgb}{1.000000,0.000000,0.000000}%
\pgfsetstrokecolor{currentstroke}%
\pgfsetdash{}{0pt}%
\pgfpathmoveto{\pgfqpoint{0.718573in}{1.586839in}}%
\pgfpathlineto{\pgfqpoint{1.579206in}{0.978569in}}%
\pgfusepath{stroke}%
\end{pgfscope}%
\begin{pgfscope}%
\pgfpathrectangle{\pgfqpoint{0.100000in}{0.220728in}}{\pgfqpoint{3.696000in}{3.696000in}}%
\pgfusepath{clip}%
\pgfsetrectcap%
\pgfsetroundjoin%
\pgfsetlinewidth{1.505625pt}%
\definecolor{currentstroke}{rgb}{1.000000,0.000000,0.000000}%
\pgfsetstrokecolor{currentstroke}%
\pgfsetdash{}{0pt}%
\pgfpathmoveto{\pgfqpoint{0.718568in}{1.586835in}}%
\pgfpathlineto{\pgfqpoint{1.579206in}{0.978569in}}%
\pgfusepath{stroke}%
\end{pgfscope}%
\begin{pgfscope}%
\pgfpathrectangle{\pgfqpoint{0.100000in}{0.220728in}}{\pgfqpoint{3.696000in}{3.696000in}}%
\pgfusepath{clip}%
\pgfsetrectcap%
\pgfsetroundjoin%
\pgfsetlinewidth{1.505625pt}%
\definecolor{currentstroke}{rgb}{1.000000,0.000000,0.000000}%
\pgfsetstrokecolor{currentstroke}%
\pgfsetdash{}{0pt}%
\pgfpathmoveto{\pgfqpoint{0.718565in}{1.586830in}}%
\pgfpathlineto{\pgfqpoint{1.579206in}{0.978569in}}%
\pgfusepath{stroke}%
\end{pgfscope}%
\begin{pgfscope}%
\pgfpathrectangle{\pgfqpoint{0.100000in}{0.220728in}}{\pgfqpoint{3.696000in}{3.696000in}}%
\pgfusepath{clip}%
\pgfsetrectcap%
\pgfsetroundjoin%
\pgfsetlinewidth{1.505625pt}%
\definecolor{currentstroke}{rgb}{1.000000,0.000000,0.000000}%
\pgfsetstrokecolor{currentstroke}%
\pgfsetdash{}{0pt}%
\pgfpathmoveto{\pgfqpoint{0.718563in}{1.586829in}}%
\pgfpathlineto{\pgfqpoint{1.579206in}{0.978569in}}%
\pgfusepath{stroke}%
\end{pgfscope}%
\begin{pgfscope}%
\pgfpathrectangle{\pgfqpoint{0.100000in}{0.220728in}}{\pgfqpoint{3.696000in}{3.696000in}}%
\pgfusepath{clip}%
\pgfsetrectcap%
\pgfsetroundjoin%
\pgfsetlinewidth{1.505625pt}%
\definecolor{currentstroke}{rgb}{1.000000,0.000000,0.000000}%
\pgfsetstrokecolor{currentstroke}%
\pgfsetdash{}{0pt}%
\pgfpathmoveto{\pgfqpoint{0.718562in}{1.586828in}}%
\pgfpathlineto{\pgfqpoint{1.579206in}{0.978569in}}%
\pgfusepath{stroke}%
\end{pgfscope}%
\begin{pgfscope}%
\pgfpathrectangle{\pgfqpoint{0.100000in}{0.220728in}}{\pgfqpoint{3.696000in}{3.696000in}}%
\pgfusepath{clip}%
\pgfsetrectcap%
\pgfsetroundjoin%
\pgfsetlinewidth{1.505625pt}%
\definecolor{currentstroke}{rgb}{1.000000,0.000000,0.000000}%
\pgfsetstrokecolor{currentstroke}%
\pgfsetdash{}{0pt}%
\pgfpathmoveto{\pgfqpoint{0.718562in}{1.586827in}}%
\pgfpathlineto{\pgfqpoint{1.579206in}{0.978569in}}%
\pgfusepath{stroke}%
\end{pgfscope}%
\begin{pgfscope}%
\pgfpathrectangle{\pgfqpoint{0.100000in}{0.220728in}}{\pgfqpoint{3.696000in}{3.696000in}}%
\pgfusepath{clip}%
\pgfsetrectcap%
\pgfsetroundjoin%
\pgfsetlinewidth{1.505625pt}%
\definecolor{currentstroke}{rgb}{1.000000,0.000000,0.000000}%
\pgfsetstrokecolor{currentstroke}%
\pgfsetdash{}{0pt}%
\pgfpathmoveto{\pgfqpoint{0.718561in}{1.586827in}}%
\pgfpathlineto{\pgfqpoint{1.579206in}{0.978569in}}%
\pgfusepath{stroke}%
\end{pgfscope}%
\begin{pgfscope}%
\pgfpathrectangle{\pgfqpoint{0.100000in}{0.220728in}}{\pgfqpoint{3.696000in}{3.696000in}}%
\pgfusepath{clip}%
\pgfsetrectcap%
\pgfsetroundjoin%
\pgfsetlinewidth{1.505625pt}%
\definecolor{currentstroke}{rgb}{1.000000,0.000000,0.000000}%
\pgfsetstrokecolor{currentstroke}%
\pgfsetdash{}{0pt}%
\pgfpathmoveto{\pgfqpoint{0.718561in}{1.586827in}}%
\pgfpathlineto{\pgfqpoint{1.579206in}{0.978569in}}%
\pgfusepath{stroke}%
\end{pgfscope}%
\begin{pgfscope}%
\pgfpathrectangle{\pgfqpoint{0.100000in}{0.220728in}}{\pgfqpoint{3.696000in}{3.696000in}}%
\pgfusepath{clip}%
\pgfsetrectcap%
\pgfsetroundjoin%
\pgfsetlinewidth{1.505625pt}%
\definecolor{currentstroke}{rgb}{1.000000,0.000000,0.000000}%
\pgfsetstrokecolor{currentstroke}%
\pgfsetdash{}{0pt}%
\pgfpathmoveto{\pgfqpoint{0.718561in}{1.586827in}}%
\pgfpathlineto{\pgfqpoint{1.579206in}{0.978569in}}%
\pgfusepath{stroke}%
\end{pgfscope}%
\begin{pgfscope}%
\pgfpathrectangle{\pgfqpoint{0.100000in}{0.220728in}}{\pgfqpoint{3.696000in}{3.696000in}}%
\pgfusepath{clip}%
\pgfsetrectcap%
\pgfsetroundjoin%
\pgfsetlinewidth{1.505625pt}%
\definecolor{currentstroke}{rgb}{1.000000,0.000000,0.000000}%
\pgfsetstrokecolor{currentstroke}%
\pgfsetdash{}{0pt}%
\pgfpathmoveto{\pgfqpoint{0.718561in}{1.586827in}}%
\pgfpathlineto{\pgfqpoint{1.579206in}{0.978569in}}%
\pgfusepath{stroke}%
\end{pgfscope}%
\begin{pgfscope}%
\pgfpathrectangle{\pgfqpoint{0.100000in}{0.220728in}}{\pgfqpoint{3.696000in}{3.696000in}}%
\pgfusepath{clip}%
\pgfsetrectcap%
\pgfsetroundjoin%
\pgfsetlinewidth{1.505625pt}%
\definecolor{currentstroke}{rgb}{1.000000,0.000000,0.000000}%
\pgfsetstrokecolor{currentstroke}%
\pgfsetdash{}{0pt}%
\pgfpathmoveto{\pgfqpoint{0.718561in}{1.586827in}}%
\pgfpathlineto{\pgfqpoint{1.579206in}{0.978569in}}%
\pgfusepath{stroke}%
\end{pgfscope}%
\begin{pgfscope}%
\pgfpathrectangle{\pgfqpoint{0.100000in}{0.220728in}}{\pgfqpoint{3.696000in}{3.696000in}}%
\pgfusepath{clip}%
\pgfsetrectcap%
\pgfsetroundjoin%
\pgfsetlinewidth{1.505625pt}%
\definecolor{currentstroke}{rgb}{1.000000,0.000000,0.000000}%
\pgfsetstrokecolor{currentstroke}%
\pgfsetdash{}{0pt}%
\pgfpathmoveto{\pgfqpoint{0.718561in}{1.586827in}}%
\pgfpathlineto{\pgfqpoint{1.579206in}{0.978569in}}%
\pgfusepath{stroke}%
\end{pgfscope}%
\begin{pgfscope}%
\pgfpathrectangle{\pgfqpoint{0.100000in}{0.220728in}}{\pgfqpoint{3.696000in}{3.696000in}}%
\pgfusepath{clip}%
\pgfsetrectcap%
\pgfsetroundjoin%
\pgfsetlinewidth{1.505625pt}%
\definecolor{currentstroke}{rgb}{1.000000,0.000000,0.000000}%
\pgfsetstrokecolor{currentstroke}%
\pgfsetdash{}{0pt}%
\pgfpathmoveto{\pgfqpoint{0.718561in}{1.586827in}}%
\pgfpathlineto{\pgfqpoint{1.579206in}{0.978569in}}%
\pgfusepath{stroke}%
\end{pgfscope}%
\begin{pgfscope}%
\pgfpathrectangle{\pgfqpoint{0.100000in}{0.220728in}}{\pgfqpoint{3.696000in}{3.696000in}}%
\pgfusepath{clip}%
\pgfsetrectcap%
\pgfsetroundjoin%
\pgfsetlinewidth{1.505625pt}%
\definecolor{currentstroke}{rgb}{1.000000,0.000000,0.000000}%
\pgfsetstrokecolor{currentstroke}%
\pgfsetdash{}{0pt}%
\pgfpathmoveto{\pgfqpoint{0.718561in}{1.586827in}}%
\pgfpathlineto{\pgfqpoint{1.579206in}{0.978569in}}%
\pgfusepath{stroke}%
\end{pgfscope}%
\begin{pgfscope}%
\pgfpathrectangle{\pgfqpoint{0.100000in}{0.220728in}}{\pgfqpoint{3.696000in}{3.696000in}}%
\pgfusepath{clip}%
\pgfsetrectcap%
\pgfsetroundjoin%
\pgfsetlinewidth{1.505625pt}%
\definecolor{currentstroke}{rgb}{1.000000,0.000000,0.000000}%
\pgfsetstrokecolor{currentstroke}%
\pgfsetdash{}{0pt}%
\pgfpathmoveto{\pgfqpoint{0.718561in}{1.586827in}}%
\pgfpathlineto{\pgfqpoint{1.579206in}{0.978569in}}%
\pgfusepath{stroke}%
\end{pgfscope}%
\begin{pgfscope}%
\pgfpathrectangle{\pgfqpoint{0.100000in}{0.220728in}}{\pgfqpoint{3.696000in}{3.696000in}}%
\pgfusepath{clip}%
\pgfsetrectcap%
\pgfsetroundjoin%
\pgfsetlinewidth{1.505625pt}%
\definecolor{currentstroke}{rgb}{1.000000,0.000000,0.000000}%
\pgfsetstrokecolor{currentstroke}%
\pgfsetdash{}{0pt}%
\pgfpathmoveto{\pgfqpoint{0.718561in}{1.586827in}}%
\pgfpathlineto{\pgfqpoint{1.579206in}{0.978569in}}%
\pgfusepath{stroke}%
\end{pgfscope}%
\begin{pgfscope}%
\pgfpathrectangle{\pgfqpoint{0.100000in}{0.220728in}}{\pgfqpoint{3.696000in}{3.696000in}}%
\pgfusepath{clip}%
\pgfsetrectcap%
\pgfsetroundjoin%
\pgfsetlinewidth{1.505625pt}%
\definecolor{currentstroke}{rgb}{1.000000,0.000000,0.000000}%
\pgfsetstrokecolor{currentstroke}%
\pgfsetdash{}{0pt}%
\pgfpathmoveto{\pgfqpoint{0.718561in}{1.586827in}}%
\pgfpathlineto{\pgfqpoint{1.579206in}{0.978569in}}%
\pgfusepath{stroke}%
\end{pgfscope}%
\begin{pgfscope}%
\pgfpathrectangle{\pgfqpoint{0.100000in}{0.220728in}}{\pgfqpoint{3.696000in}{3.696000in}}%
\pgfusepath{clip}%
\pgfsetrectcap%
\pgfsetroundjoin%
\pgfsetlinewidth{1.505625pt}%
\definecolor{currentstroke}{rgb}{1.000000,0.000000,0.000000}%
\pgfsetstrokecolor{currentstroke}%
\pgfsetdash{}{0pt}%
\pgfpathmoveto{\pgfqpoint{0.718561in}{1.586827in}}%
\pgfpathlineto{\pgfqpoint{1.579206in}{0.978569in}}%
\pgfusepath{stroke}%
\end{pgfscope}%
\begin{pgfscope}%
\pgfpathrectangle{\pgfqpoint{0.100000in}{0.220728in}}{\pgfqpoint{3.696000in}{3.696000in}}%
\pgfusepath{clip}%
\pgfsetrectcap%
\pgfsetroundjoin%
\pgfsetlinewidth{1.505625pt}%
\definecolor{currentstroke}{rgb}{1.000000,0.000000,0.000000}%
\pgfsetstrokecolor{currentstroke}%
\pgfsetdash{}{0pt}%
\pgfpathmoveto{\pgfqpoint{0.718561in}{1.586827in}}%
\pgfpathlineto{\pgfqpoint{1.579206in}{0.978569in}}%
\pgfusepath{stroke}%
\end{pgfscope}%
\begin{pgfscope}%
\pgfpathrectangle{\pgfqpoint{0.100000in}{0.220728in}}{\pgfqpoint{3.696000in}{3.696000in}}%
\pgfusepath{clip}%
\pgfsetrectcap%
\pgfsetroundjoin%
\pgfsetlinewidth{1.505625pt}%
\definecolor{currentstroke}{rgb}{1.000000,0.000000,0.000000}%
\pgfsetstrokecolor{currentstroke}%
\pgfsetdash{}{0pt}%
\pgfpathmoveto{\pgfqpoint{0.717288in}{1.584480in}}%
\pgfpathlineto{\pgfqpoint{1.579206in}{0.978569in}}%
\pgfusepath{stroke}%
\end{pgfscope}%
\begin{pgfscope}%
\pgfpathrectangle{\pgfqpoint{0.100000in}{0.220728in}}{\pgfqpoint{3.696000in}{3.696000in}}%
\pgfusepath{clip}%
\pgfsetrectcap%
\pgfsetroundjoin%
\pgfsetlinewidth{1.505625pt}%
\definecolor{currentstroke}{rgb}{1.000000,0.000000,0.000000}%
\pgfsetstrokecolor{currentstroke}%
\pgfsetdash{}{0pt}%
\pgfpathmoveto{\pgfqpoint{0.716569in}{1.583184in}}%
\pgfpathlineto{\pgfqpoint{1.579206in}{0.978569in}}%
\pgfusepath{stroke}%
\end{pgfscope}%
\begin{pgfscope}%
\pgfpathrectangle{\pgfqpoint{0.100000in}{0.220728in}}{\pgfqpoint{3.696000in}{3.696000in}}%
\pgfusepath{clip}%
\pgfsetrectcap%
\pgfsetroundjoin%
\pgfsetlinewidth{1.505625pt}%
\definecolor{currentstroke}{rgb}{1.000000,0.000000,0.000000}%
\pgfsetstrokecolor{currentstroke}%
\pgfsetdash{}{0pt}%
\pgfpathmoveto{\pgfqpoint{0.716193in}{1.582713in}}%
\pgfpathlineto{\pgfqpoint{1.579206in}{0.978569in}}%
\pgfusepath{stroke}%
\end{pgfscope}%
\begin{pgfscope}%
\pgfpathrectangle{\pgfqpoint{0.100000in}{0.220728in}}{\pgfqpoint{3.696000in}{3.696000in}}%
\pgfusepath{clip}%
\pgfsetrectcap%
\pgfsetroundjoin%
\pgfsetlinewidth{1.505625pt}%
\definecolor{currentstroke}{rgb}{1.000000,0.000000,0.000000}%
\pgfsetstrokecolor{currentstroke}%
\pgfsetdash{}{0pt}%
\pgfpathmoveto{\pgfqpoint{0.715983in}{1.582306in}}%
\pgfpathlineto{\pgfqpoint{1.579206in}{0.978569in}}%
\pgfusepath{stroke}%
\end{pgfscope}%
\begin{pgfscope}%
\pgfpathrectangle{\pgfqpoint{0.100000in}{0.220728in}}{\pgfqpoint{3.696000in}{3.696000in}}%
\pgfusepath{clip}%
\pgfsetrectcap%
\pgfsetroundjoin%
\pgfsetlinewidth{1.505625pt}%
\definecolor{currentstroke}{rgb}{1.000000,0.000000,0.000000}%
\pgfsetstrokecolor{currentstroke}%
\pgfsetdash{}{0pt}%
\pgfpathmoveto{\pgfqpoint{0.715872in}{1.582189in}}%
\pgfpathlineto{\pgfqpoint{1.579206in}{0.978569in}}%
\pgfusepath{stroke}%
\end{pgfscope}%
\begin{pgfscope}%
\pgfpathrectangle{\pgfqpoint{0.100000in}{0.220728in}}{\pgfqpoint{3.696000in}{3.696000in}}%
\pgfusepath{clip}%
\pgfsetrectcap%
\pgfsetroundjoin%
\pgfsetlinewidth{1.505625pt}%
\definecolor{currentstroke}{rgb}{1.000000,0.000000,0.000000}%
\pgfsetstrokecolor{currentstroke}%
\pgfsetdash{}{0pt}%
\pgfpathmoveto{\pgfqpoint{0.715806in}{1.582098in}}%
\pgfpathlineto{\pgfqpoint{1.579206in}{0.978569in}}%
\pgfusepath{stroke}%
\end{pgfscope}%
\begin{pgfscope}%
\pgfpathrectangle{\pgfqpoint{0.100000in}{0.220728in}}{\pgfqpoint{3.696000in}{3.696000in}}%
\pgfusepath{clip}%
\pgfsetrectcap%
\pgfsetroundjoin%
\pgfsetlinewidth{1.505625pt}%
\definecolor{currentstroke}{rgb}{1.000000,0.000000,0.000000}%
\pgfsetstrokecolor{currentstroke}%
\pgfsetdash{}{0pt}%
\pgfpathmoveto{\pgfqpoint{0.715772in}{1.582042in}}%
\pgfpathlineto{\pgfqpoint{1.579206in}{0.978569in}}%
\pgfusepath{stroke}%
\end{pgfscope}%
\begin{pgfscope}%
\pgfpathrectangle{\pgfqpoint{0.100000in}{0.220728in}}{\pgfqpoint{3.696000in}{3.696000in}}%
\pgfusepath{clip}%
\pgfsetrectcap%
\pgfsetroundjoin%
\pgfsetlinewidth{1.505625pt}%
\definecolor{currentstroke}{rgb}{1.000000,0.000000,0.000000}%
\pgfsetstrokecolor{currentstroke}%
\pgfsetdash{}{0pt}%
\pgfpathmoveto{\pgfqpoint{0.715754in}{1.582032in}}%
\pgfpathlineto{\pgfqpoint{1.579206in}{0.978569in}}%
\pgfusepath{stroke}%
\end{pgfscope}%
\begin{pgfscope}%
\pgfpathrectangle{\pgfqpoint{0.100000in}{0.220728in}}{\pgfqpoint{3.696000in}{3.696000in}}%
\pgfusepath{clip}%
\pgfsetrectcap%
\pgfsetroundjoin%
\pgfsetlinewidth{1.505625pt}%
\definecolor{currentstroke}{rgb}{1.000000,0.000000,0.000000}%
\pgfsetstrokecolor{currentstroke}%
\pgfsetdash{}{0pt}%
\pgfpathmoveto{\pgfqpoint{0.715743in}{1.582040in}}%
\pgfpathlineto{\pgfqpoint{1.579206in}{0.978569in}}%
\pgfusepath{stroke}%
\end{pgfscope}%
\begin{pgfscope}%
\pgfpathrectangle{\pgfqpoint{0.100000in}{0.220728in}}{\pgfqpoint{3.696000in}{3.696000in}}%
\pgfusepath{clip}%
\pgfsetrectcap%
\pgfsetroundjoin%
\pgfsetlinewidth{1.505625pt}%
\definecolor{currentstroke}{rgb}{1.000000,0.000000,0.000000}%
\pgfsetstrokecolor{currentstroke}%
\pgfsetdash{}{0pt}%
\pgfpathmoveto{\pgfqpoint{0.715738in}{1.582051in}}%
\pgfpathlineto{\pgfqpoint{1.579206in}{0.978569in}}%
\pgfusepath{stroke}%
\end{pgfscope}%
\begin{pgfscope}%
\pgfpathrectangle{\pgfqpoint{0.100000in}{0.220728in}}{\pgfqpoint{3.696000in}{3.696000in}}%
\pgfusepath{clip}%
\pgfsetrectcap%
\pgfsetroundjoin%
\pgfsetlinewidth{1.505625pt}%
\definecolor{currentstroke}{rgb}{1.000000,0.000000,0.000000}%
\pgfsetstrokecolor{currentstroke}%
\pgfsetdash{}{0pt}%
\pgfpathmoveto{\pgfqpoint{0.707695in}{1.673614in}}%
\pgfpathlineto{\pgfqpoint{1.579206in}{0.978569in}}%
\pgfusepath{stroke}%
\end{pgfscope}%
\begin{pgfscope}%
\pgfpathrectangle{\pgfqpoint{0.100000in}{0.220728in}}{\pgfqpoint{3.696000in}{3.696000in}}%
\pgfusepath{clip}%
\pgfsetrectcap%
\pgfsetroundjoin%
\pgfsetlinewidth{1.505625pt}%
\definecolor{currentstroke}{rgb}{1.000000,0.000000,0.000000}%
\pgfsetstrokecolor{currentstroke}%
\pgfsetdash{}{0pt}%
\pgfpathmoveto{\pgfqpoint{0.676828in}{1.770615in}}%
\pgfpathlineto{\pgfqpoint{1.568175in}{0.968392in}}%
\pgfusepath{stroke}%
\end{pgfscope}%
\begin{pgfscope}%
\pgfpathrectangle{\pgfqpoint{0.100000in}{0.220728in}}{\pgfqpoint{3.696000in}{3.696000in}}%
\pgfusepath{clip}%
\pgfsetrectcap%
\pgfsetroundjoin%
\pgfsetlinewidth{1.505625pt}%
\definecolor{currentstroke}{rgb}{1.000000,0.000000,0.000000}%
\pgfsetstrokecolor{currentstroke}%
\pgfsetdash{}{0pt}%
\pgfpathmoveto{\pgfqpoint{0.666060in}{1.649961in}}%
\pgfpathlineto{\pgfqpoint{1.568175in}{0.968392in}}%
\pgfusepath{stroke}%
\end{pgfscope}%
\begin{pgfscope}%
\pgfpathrectangle{\pgfqpoint{0.100000in}{0.220728in}}{\pgfqpoint{3.696000in}{3.696000in}}%
\pgfusepath{clip}%
\pgfsetrectcap%
\pgfsetroundjoin%
\pgfsetlinewidth{1.505625pt}%
\definecolor{currentstroke}{rgb}{1.000000,0.000000,0.000000}%
\pgfsetstrokecolor{currentstroke}%
\pgfsetdash{}{0pt}%
\pgfpathmoveto{\pgfqpoint{0.659492in}{1.855427in}}%
\pgfpathlineto{\pgfqpoint{1.579206in}{0.978569in}}%
\pgfusepath{stroke}%
\end{pgfscope}%
\begin{pgfscope}%
\pgfpathrectangle{\pgfqpoint{0.100000in}{0.220728in}}{\pgfqpoint{3.696000in}{3.696000in}}%
\pgfusepath{clip}%
\pgfsetrectcap%
\pgfsetroundjoin%
\pgfsetlinewidth{1.505625pt}%
\definecolor{currentstroke}{rgb}{1.000000,0.000000,0.000000}%
\pgfsetstrokecolor{currentstroke}%
\pgfsetdash{}{0pt}%
\pgfpathmoveto{\pgfqpoint{0.657038in}{1.967122in}}%
\pgfpathlineto{\pgfqpoint{1.601216in}{0.998876in}}%
\pgfusepath{stroke}%
\end{pgfscope}%
\begin{pgfscope}%
\pgfpathrectangle{\pgfqpoint{0.100000in}{0.220728in}}{\pgfqpoint{3.696000in}{3.696000in}}%
\pgfusepath{clip}%
\pgfsetrectcap%
\pgfsetroundjoin%
\pgfsetlinewidth{1.505625pt}%
\definecolor{currentstroke}{rgb}{1.000000,0.000000,0.000000}%
\pgfsetstrokecolor{currentstroke}%
\pgfsetdash{}{0pt}%
\pgfpathmoveto{\pgfqpoint{0.649545in}{2.273724in}}%
\pgfpathlineto{\pgfqpoint{1.612195in}{1.009005in}}%
\pgfusepath{stroke}%
\end{pgfscope}%
\begin{pgfscope}%
\pgfpathrectangle{\pgfqpoint{0.100000in}{0.220728in}}{\pgfqpoint{3.696000in}{3.696000in}}%
\pgfusepath{clip}%
\pgfsetrectcap%
\pgfsetroundjoin%
\pgfsetlinewidth{1.505625pt}%
\definecolor{currentstroke}{rgb}{1.000000,0.000000,0.000000}%
\pgfsetstrokecolor{currentstroke}%
\pgfsetdash{}{0pt}%
\pgfpathmoveto{\pgfqpoint{0.641709in}{2.549091in}}%
\pgfpathlineto{\pgfqpoint{1.634100in}{1.029215in}}%
\pgfusepath{stroke}%
\end{pgfscope}%
\begin{pgfscope}%
\pgfpathrectangle{\pgfqpoint{0.100000in}{0.220728in}}{\pgfqpoint{3.696000in}{3.696000in}}%
\pgfusepath{clip}%
\pgfsetrectcap%
\pgfsetroundjoin%
\pgfsetlinewidth{1.505625pt}%
\definecolor{currentstroke}{rgb}{1.000000,0.000000,0.000000}%
\pgfsetstrokecolor{currentstroke}%
\pgfsetdash{}{0pt}%
\pgfpathmoveto{\pgfqpoint{0.637390in}{2.702955in}}%
\pgfpathlineto{\pgfqpoint{1.645026in}{1.039295in}}%
\pgfusepath{stroke}%
\end{pgfscope}%
\begin{pgfscope}%
\pgfpathrectangle{\pgfqpoint{0.100000in}{0.220728in}}{\pgfqpoint{3.696000in}{3.696000in}}%
\pgfusepath{clip}%
\pgfsetbuttcap%
\pgfsetroundjoin%
\definecolor{currentfill}{rgb}{0.121569,0.466667,0.705882}%
\pgfsetfillcolor{currentfill}%
\pgfsetfillopacity{0.300000}%
\pgfsetlinewidth{1.003750pt}%
\definecolor{currentstroke}{rgb}{0.121569,0.466667,0.705882}%
\pgfsetstrokecolor{currentstroke}%
\pgfsetstrokeopacity{0.300000}%
\pgfsetdash{}{0pt}%
\pgfpathmoveto{\pgfqpoint{0.715743in}{1.550983in}}%
\pgfpathcurveto{\pgfqpoint{0.723979in}{1.550983in}}{\pgfqpoint{0.731880in}{1.554255in}}{\pgfqpoint{0.737703in}{1.560079in}}%
\pgfpathcurveto{\pgfqpoint{0.743527in}{1.565903in}}{\pgfqpoint{0.746800in}{1.573803in}}{\pgfqpoint{0.746800in}{1.582040in}}%
\pgfpathcurveto{\pgfqpoint{0.746800in}{1.590276in}}{\pgfqpoint{0.743527in}{1.598176in}}{\pgfqpoint{0.737703in}{1.604000in}}%
\pgfpathcurveto{\pgfqpoint{0.731880in}{1.609824in}}{\pgfqpoint{0.723979in}{1.613096in}}{\pgfqpoint{0.715743in}{1.613096in}}%
\pgfpathcurveto{\pgfqpoint{0.707507in}{1.613096in}}{\pgfqpoint{0.699607in}{1.609824in}}{\pgfqpoint{0.693783in}{1.604000in}}%
\pgfpathcurveto{\pgfqpoint{0.687959in}{1.598176in}}{\pgfqpoint{0.684687in}{1.590276in}}{\pgfqpoint{0.684687in}{1.582040in}}%
\pgfpathcurveto{\pgfqpoint{0.684687in}{1.573803in}}{\pgfqpoint{0.687959in}{1.565903in}}{\pgfqpoint{0.693783in}{1.560079in}}%
\pgfpathcurveto{\pgfqpoint{0.699607in}{1.554255in}}{\pgfqpoint{0.707507in}{1.550983in}}{\pgfqpoint{0.715743in}{1.550983in}}%
\pgfpathclose%
\pgfusepath{stroke,fill}%
\end{pgfscope}%
\begin{pgfscope}%
\pgfpathrectangle{\pgfqpoint{0.100000in}{0.220728in}}{\pgfqpoint{3.696000in}{3.696000in}}%
\pgfusepath{clip}%
\pgfsetbuttcap%
\pgfsetroundjoin%
\definecolor{currentfill}{rgb}{0.121569,0.466667,0.705882}%
\pgfsetfillcolor{currentfill}%
\pgfsetfillopacity{0.300002}%
\pgfsetlinewidth{1.003750pt}%
\definecolor{currentstroke}{rgb}{0.121569,0.466667,0.705882}%
\pgfsetstrokecolor{currentstroke}%
\pgfsetstrokeopacity{0.300002}%
\pgfsetdash{}{0pt}%
\pgfpathmoveto{\pgfqpoint{0.715754in}{1.550975in}}%
\pgfpathcurveto{\pgfqpoint{0.723990in}{1.550975in}}{\pgfqpoint{0.731890in}{1.554248in}}{\pgfqpoint{0.737714in}{1.560071in}}%
\pgfpathcurveto{\pgfqpoint{0.743538in}{1.565895in}}{\pgfqpoint{0.746810in}{1.573795in}}{\pgfqpoint{0.746810in}{1.582032in}}%
\pgfpathcurveto{\pgfqpoint{0.746810in}{1.590268in}}{\pgfqpoint{0.743538in}{1.598168in}}{\pgfqpoint{0.737714in}{1.603992in}}%
\pgfpathcurveto{\pgfqpoint{0.731890in}{1.609816in}}{\pgfqpoint{0.723990in}{1.613088in}}{\pgfqpoint{0.715754in}{1.613088in}}%
\pgfpathcurveto{\pgfqpoint{0.707518in}{1.613088in}}{\pgfqpoint{0.699617in}{1.609816in}}{\pgfqpoint{0.693794in}{1.603992in}}%
\pgfpathcurveto{\pgfqpoint{0.687970in}{1.598168in}}{\pgfqpoint{0.684697in}{1.590268in}}{\pgfqpoint{0.684697in}{1.582032in}}%
\pgfpathcurveto{\pgfqpoint{0.684697in}{1.573795in}}{\pgfqpoint{0.687970in}{1.565895in}}{\pgfqpoint{0.693794in}{1.560071in}}%
\pgfpathcurveto{\pgfqpoint{0.699617in}{1.554248in}}{\pgfqpoint{0.707518in}{1.550975in}}{\pgfqpoint{0.715754in}{1.550975in}}%
\pgfpathclose%
\pgfusepath{stroke,fill}%
\end{pgfscope}%
\begin{pgfscope}%
\pgfpathrectangle{\pgfqpoint{0.100000in}{0.220728in}}{\pgfqpoint{3.696000in}{3.696000in}}%
\pgfusepath{clip}%
\pgfsetbuttcap%
\pgfsetroundjoin%
\definecolor{currentfill}{rgb}{0.121569,0.466667,0.705882}%
\pgfsetfillcolor{currentfill}%
\pgfsetfillopacity{0.300002}%
\pgfsetlinewidth{1.003750pt}%
\definecolor{currentstroke}{rgb}{0.121569,0.466667,0.705882}%
\pgfsetstrokecolor{currentstroke}%
\pgfsetstrokeopacity{0.300002}%
\pgfsetdash{}{0pt}%
\pgfpathmoveto{\pgfqpoint{0.715738in}{1.550994in}}%
\pgfpathcurveto{\pgfqpoint{0.723975in}{1.550994in}}{\pgfqpoint{0.731875in}{1.554267in}}{\pgfqpoint{0.737699in}{1.560091in}}%
\pgfpathcurveto{\pgfqpoint{0.743523in}{1.565915in}}{\pgfqpoint{0.746795in}{1.573815in}}{\pgfqpoint{0.746795in}{1.582051in}}%
\pgfpathcurveto{\pgfqpoint{0.746795in}{1.590287in}}{\pgfqpoint{0.743523in}{1.598187in}}{\pgfqpoint{0.737699in}{1.604011in}}%
\pgfpathcurveto{\pgfqpoint{0.731875in}{1.609835in}}{\pgfqpoint{0.723975in}{1.613107in}}{\pgfqpoint{0.715738in}{1.613107in}}%
\pgfpathcurveto{\pgfqpoint{0.707502in}{1.613107in}}{\pgfqpoint{0.699602in}{1.609835in}}{\pgfqpoint{0.693778in}{1.604011in}}%
\pgfpathcurveto{\pgfqpoint{0.687954in}{1.598187in}}{\pgfqpoint{0.684682in}{1.590287in}}{\pgfqpoint{0.684682in}{1.582051in}}%
\pgfpathcurveto{\pgfqpoint{0.684682in}{1.573815in}}{\pgfqpoint{0.687954in}{1.565915in}}{\pgfqpoint{0.693778in}{1.560091in}}%
\pgfpathcurveto{\pgfqpoint{0.699602in}{1.554267in}}{\pgfqpoint{0.707502in}{1.550994in}}{\pgfqpoint{0.715738in}{1.550994in}}%
\pgfpathclose%
\pgfusepath{stroke,fill}%
\end{pgfscope}%
\begin{pgfscope}%
\pgfpathrectangle{\pgfqpoint{0.100000in}{0.220728in}}{\pgfqpoint{3.696000in}{3.696000in}}%
\pgfusepath{clip}%
\pgfsetbuttcap%
\pgfsetroundjoin%
\definecolor{currentfill}{rgb}{0.121569,0.466667,0.705882}%
\pgfsetfillcolor{currentfill}%
\pgfsetfillopacity{0.300028}%
\pgfsetlinewidth{1.003750pt}%
\definecolor{currentstroke}{rgb}{0.121569,0.466667,0.705882}%
\pgfsetstrokecolor{currentstroke}%
\pgfsetstrokeopacity{0.300028}%
\pgfsetdash{}{0pt}%
\pgfpathmoveto{\pgfqpoint{0.715772in}{1.550985in}}%
\pgfpathcurveto{\pgfqpoint{0.724008in}{1.550985in}}{\pgfqpoint{0.731908in}{1.554258in}}{\pgfqpoint{0.737732in}{1.560081in}}%
\pgfpathcurveto{\pgfqpoint{0.743556in}{1.565905in}}{\pgfqpoint{0.746829in}{1.573805in}}{\pgfqpoint{0.746829in}{1.582042in}}%
\pgfpathcurveto{\pgfqpoint{0.746829in}{1.590278in}}{\pgfqpoint{0.743556in}{1.598178in}}{\pgfqpoint{0.737732in}{1.604002in}}%
\pgfpathcurveto{\pgfqpoint{0.731908in}{1.609826in}}{\pgfqpoint{0.724008in}{1.613098in}}{\pgfqpoint{0.715772in}{1.613098in}}%
\pgfpathcurveto{\pgfqpoint{0.707536in}{1.613098in}}{\pgfqpoint{0.699636in}{1.609826in}}{\pgfqpoint{0.693812in}{1.604002in}}%
\pgfpathcurveto{\pgfqpoint{0.687988in}{1.598178in}}{\pgfqpoint{0.684716in}{1.590278in}}{\pgfqpoint{0.684716in}{1.582042in}}%
\pgfpathcurveto{\pgfqpoint{0.684716in}{1.573805in}}{\pgfqpoint{0.687988in}{1.565905in}}{\pgfqpoint{0.693812in}{1.560081in}}%
\pgfpathcurveto{\pgfqpoint{0.699636in}{1.554258in}}{\pgfqpoint{0.707536in}{1.550985in}}{\pgfqpoint{0.715772in}{1.550985in}}%
\pgfpathclose%
\pgfusepath{stroke,fill}%
\end{pgfscope}%
\begin{pgfscope}%
\pgfpathrectangle{\pgfqpoint{0.100000in}{0.220728in}}{\pgfqpoint{3.696000in}{3.696000in}}%
\pgfusepath{clip}%
\pgfsetbuttcap%
\pgfsetroundjoin%
\definecolor{currentfill}{rgb}{0.121569,0.466667,0.705882}%
\pgfsetfillcolor{currentfill}%
\pgfsetfillopacity{0.300103}%
\pgfsetlinewidth{1.003750pt}%
\definecolor{currentstroke}{rgb}{0.121569,0.466667,0.705882}%
\pgfsetstrokecolor{currentstroke}%
\pgfsetstrokeopacity{0.300103}%
\pgfsetdash{}{0pt}%
\pgfpathmoveto{\pgfqpoint{0.715806in}{1.551041in}}%
\pgfpathcurveto{\pgfqpoint{0.724043in}{1.551041in}}{\pgfqpoint{0.731943in}{1.554313in}}{\pgfqpoint{0.737767in}{1.560137in}}%
\pgfpathcurveto{\pgfqpoint{0.743590in}{1.565961in}}{\pgfqpoint{0.746863in}{1.573861in}}{\pgfqpoint{0.746863in}{1.582098in}}%
\pgfpathcurveto{\pgfqpoint{0.746863in}{1.590334in}}{\pgfqpoint{0.743590in}{1.598234in}}{\pgfqpoint{0.737767in}{1.604058in}}%
\pgfpathcurveto{\pgfqpoint{0.731943in}{1.609882in}}{\pgfqpoint{0.724043in}{1.613154in}}{\pgfqpoint{0.715806in}{1.613154in}}%
\pgfpathcurveto{\pgfqpoint{0.707570in}{1.613154in}}{\pgfqpoint{0.699670in}{1.609882in}}{\pgfqpoint{0.693846in}{1.604058in}}%
\pgfpathcurveto{\pgfqpoint{0.688022in}{1.598234in}}{\pgfqpoint{0.684750in}{1.590334in}}{\pgfqpoint{0.684750in}{1.582098in}}%
\pgfpathcurveto{\pgfqpoint{0.684750in}{1.573861in}}{\pgfqpoint{0.688022in}{1.565961in}}{\pgfqpoint{0.693846in}{1.560137in}}%
\pgfpathcurveto{\pgfqpoint{0.699670in}{1.554313in}}{\pgfqpoint{0.707570in}{1.551041in}}{\pgfqpoint{0.715806in}{1.551041in}}%
\pgfpathclose%
\pgfusepath{stroke,fill}%
\end{pgfscope}%
\begin{pgfscope}%
\pgfpathrectangle{\pgfqpoint{0.100000in}{0.220728in}}{\pgfqpoint{3.696000in}{3.696000in}}%
\pgfusepath{clip}%
\pgfsetbuttcap%
\pgfsetroundjoin%
\definecolor{currentfill}{rgb}{0.121569,0.466667,0.705882}%
\pgfsetfillcolor{currentfill}%
\pgfsetfillopacity{0.300221}%
\pgfsetlinewidth{1.003750pt}%
\definecolor{currentstroke}{rgb}{0.121569,0.466667,0.705882}%
\pgfsetstrokecolor{currentstroke}%
\pgfsetstrokeopacity{0.300221}%
\pgfsetdash{}{0pt}%
\pgfpathmoveto{\pgfqpoint{0.715872in}{1.551133in}}%
\pgfpathcurveto{\pgfqpoint{0.724108in}{1.551133in}}{\pgfqpoint{0.732008in}{1.554405in}}{\pgfqpoint{0.737832in}{1.560229in}}%
\pgfpathcurveto{\pgfqpoint{0.743656in}{1.566053in}}{\pgfqpoint{0.746928in}{1.573953in}}{\pgfqpoint{0.746928in}{1.582189in}}%
\pgfpathcurveto{\pgfqpoint{0.746928in}{1.590425in}}{\pgfqpoint{0.743656in}{1.598325in}}{\pgfqpoint{0.737832in}{1.604149in}}%
\pgfpathcurveto{\pgfqpoint{0.732008in}{1.609973in}}{\pgfqpoint{0.724108in}{1.613246in}}{\pgfqpoint{0.715872in}{1.613246in}}%
\pgfpathcurveto{\pgfqpoint{0.707636in}{1.613246in}}{\pgfqpoint{0.699736in}{1.609973in}}{\pgfqpoint{0.693912in}{1.604149in}}%
\pgfpathcurveto{\pgfqpoint{0.688088in}{1.598325in}}{\pgfqpoint{0.684815in}{1.590425in}}{\pgfqpoint{0.684815in}{1.582189in}}%
\pgfpathcurveto{\pgfqpoint{0.684815in}{1.573953in}}{\pgfqpoint{0.688088in}{1.566053in}}{\pgfqpoint{0.693912in}{1.560229in}}%
\pgfpathcurveto{\pgfqpoint{0.699736in}{1.554405in}}{\pgfqpoint{0.707636in}{1.551133in}}{\pgfqpoint{0.715872in}{1.551133in}}%
\pgfpathclose%
\pgfusepath{stroke,fill}%
\end{pgfscope}%
\begin{pgfscope}%
\pgfpathrectangle{\pgfqpoint{0.100000in}{0.220728in}}{\pgfqpoint{3.696000in}{3.696000in}}%
\pgfusepath{clip}%
\pgfsetbuttcap%
\pgfsetroundjoin%
\definecolor{currentfill}{rgb}{0.121569,0.466667,0.705882}%
\pgfsetfillcolor{currentfill}%
\pgfsetfillopacity{0.300420}%
\pgfsetlinewidth{1.003750pt}%
\definecolor{currentstroke}{rgb}{0.121569,0.466667,0.705882}%
\pgfsetstrokecolor{currentstroke}%
\pgfsetstrokeopacity{0.300420}%
\pgfsetdash{}{0pt}%
\pgfpathmoveto{\pgfqpoint{0.715983in}{1.551249in}}%
\pgfpathcurveto{\pgfqpoint{0.724219in}{1.551249in}}{\pgfqpoint{0.732119in}{1.554521in}}{\pgfqpoint{0.737943in}{1.560345in}}%
\pgfpathcurveto{\pgfqpoint{0.743767in}{1.566169in}}{\pgfqpoint{0.747040in}{1.574069in}}{\pgfqpoint{0.747040in}{1.582306in}}%
\pgfpathcurveto{\pgfqpoint{0.747040in}{1.590542in}}{\pgfqpoint{0.743767in}{1.598442in}}{\pgfqpoint{0.737943in}{1.604266in}}%
\pgfpathcurveto{\pgfqpoint{0.732119in}{1.610090in}}{\pgfqpoint{0.724219in}{1.613362in}}{\pgfqpoint{0.715983in}{1.613362in}}%
\pgfpathcurveto{\pgfqpoint{0.707747in}{1.613362in}}{\pgfqpoint{0.699847in}{1.610090in}}{\pgfqpoint{0.694023in}{1.604266in}}%
\pgfpathcurveto{\pgfqpoint{0.688199in}{1.598442in}}{\pgfqpoint{0.684927in}{1.590542in}}{\pgfqpoint{0.684927in}{1.582306in}}%
\pgfpathcurveto{\pgfqpoint{0.684927in}{1.574069in}}{\pgfqpoint{0.688199in}{1.566169in}}{\pgfqpoint{0.694023in}{1.560345in}}%
\pgfpathcurveto{\pgfqpoint{0.699847in}{1.554521in}}{\pgfqpoint{0.707747in}{1.551249in}}{\pgfqpoint{0.715983in}{1.551249in}}%
\pgfpathclose%
\pgfusepath{stroke,fill}%
\end{pgfscope}%
\begin{pgfscope}%
\pgfpathrectangle{\pgfqpoint{0.100000in}{0.220728in}}{\pgfqpoint{3.696000in}{3.696000in}}%
\pgfusepath{clip}%
\pgfsetbuttcap%
\pgfsetroundjoin%
\definecolor{currentfill}{rgb}{0.121569,0.466667,0.705882}%
\pgfsetfillcolor{currentfill}%
\pgfsetfillopacity{0.300917}%
\pgfsetlinewidth{1.003750pt}%
\definecolor{currentstroke}{rgb}{0.121569,0.466667,0.705882}%
\pgfsetstrokecolor{currentstroke}%
\pgfsetstrokeopacity{0.300917}%
\pgfsetdash{}{0pt}%
\pgfpathmoveto{\pgfqpoint{0.716193in}{1.551656in}}%
\pgfpathcurveto{\pgfqpoint{0.724429in}{1.551656in}}{\pgfqpoint{0.732329in}{1.554928in}}{\pgfqpoint{0.738153in}{1.560752in}}%
\pgfpathcurveto{\pgfqpoint{0.743977in}{1.566576in}}{\pgfqpoint{0.747250in}{1.574476in}}{\pgfqpoint{0.747250in}{1.582713in}}%
\pgfpathcurveto{\pgfqpoint{0.747250in}{1.590949in}}{\pgfqpoint{0.743977in}{1.598849in}}{\pgfqpoint{0.738153in}{1.604673in}}%
\pgfpathcurveto{\pgfqpoint{0.732329in}{1.610497in}}{\pgfqpoint{0.724429in}{1.613769in}}{\pgfqpoint{0.716193in}{1.613769in}}%
\pgfpathcurveto{\pgfqpoint{0.707957in}{1.613769in}}{\pgfqpoint{0.700057in}{1.610497in}}{\pgfqpoint{0.694233in}{1.604673in}}%
\pgfpathcurveto{\pgfqpoint{0.688409in}{1.598849in}}{\pgfqpoint{0.685137in}{1.590949in}}{\pgfqpoint{0.685137in}{1.582713in}}%
\pgfpathcurveto{\pgfqpoint{0.685137in}{1.574476in}}{\pgfqpoint{0.688409in}{1.566576in}}{\pgfqpoint{0.694233in}{1.560752in}}%
\pgfpathcurveto{\pgfqpoint{0.700057in}{1.554928in}}{\pgfqpoint{0.707957in}{1.551656in}}{\pgfqpoint{0.716193in}{1.551656in}}%
\pgfpathclose%
\pgfusepath{stroke,fill}%
\end{pgfscope}%
\begin{pgfscope}%
\pgfpathrectangle{\pgfqpoint{0.100000in}{0.220728in}}{\pgfqpoint{3.696000in}{3.696000in}}%
\pgfusepath{clip}%
\pgfsetbuttcap%
\pgfsetroundjoin%
\definecolor{currentfill}{rgb}{0.121569,0.466667,0.705882}%
\pgfsetfillcolor{currentfill}%
\pgfsetfillopacity{0.301625}%
\pgfsetlinewidth{1.003750pt}%
\definecolor{currentstroke}{rgb}{0.121569,0.466667,0.705882}%
\pgfsetstrokecolor{currentstroke}%
\pgfsetstrokeopacity{0.301625}%
\pgfsetdash{}{0pt}%
\pgfpathmoveto{\pgfqpoint{0.716569in}{1.552127in}}%
\pgfpathcurveto{\pgfqpoint{0.724805in}{1.552127in}}{\pgfqpoint{0.732705in}{1.555399in}}{\pgfqpoint{0.738529in}{1.561223in}}%
\pgfpathcurveto{\pgfqpoint{0.744353in}{1.567047in}}{\pgfqpoint{0.747626in}{1.574947in}}{\pgfqpoint{0.747626in}{1.583184in}}%
\pgfpathcurveto{\pgfqpoint{0.747626in}{1.591420in}}{\pgfqpoint{0.744353in}{1.599320in}}{\pgfqpoint{0.738529in}{1.605144in}}%
\pgfpathcurveto{\pgfqpoint{0.732705in}{1.610968in}}{\pgfqpoint{0.724805in}{1.614240in}}{\pgfqpoint{0.716569in}{1.614240in}}%
\pgfpathcurveto{\pgfqpoint{0.708333in}{1.614240in}}{\pgfqpoint{0.700433in}{1.610968in}}{\pgfqpoint{0.694609in}{1.605144in}}%
\pgfpathcurveto{\pgfqpoint{0.688785in}{1.599320in}}{\pgfqpoint{0.685513in}{1.591420in}}{\pgfqpoint{0.685513in}{1.583184in}}%
\pgfpathcurveto{\pgfqpoint{0.685513in}{1.574947in}}{\pgfqpoint{0.688785in}{1.567047in}}{\pgfqpoint{0.694609in}{1.561223in}}%
\pgfpathcurveto{\pgfqpoint{0.700433in}{1.555399in}}{\pgfqpoint{0.708333in}{1.552127in}}{\pgfqpoint{0.716569in}{1.552127in}}%
\pgfpathclose%
\pgfusepath{stroke,fill}%
\end{pgfscope}%
\begin{pgfscope}%
\pgfpathrectangle{\pgfqpoint{0.100000in}{0.220728in}}{\pgfqpoint{3.696000in}{3.696000in}}%
\pgfusepath{clip}%
\pgfsetbuttcap%
\pgfsetroundjoin%
\definecolor{currentfill}{rgb}{0.121569,0.466667,0.705882}%
\pgfsetfillcolor{currentfill}%
\pgfsetfillopacity{0.303149}%
\pgfsetlinewidth{1.003750pt}%
\definecolor{currentstroke}{rgb}{0.121569,0.466667,0.705882}%
\pgfsetstrokecolor{currentstroke}%
\pgfsetstrokeopacity{0.303149}%
\pgfsetdash{}{0pt}%
\pgfpathmoveto{\pgfqpoint{0.717288in}{1.553423in}}%
\pgfpathcurveto{\pgfqpoint{0.725524in}{1.553423in}}{\pgfqpoint{0.733425in}{1.556695in}}{\pgfqpoint{0.739248in}{1.562519in}}%
\pgfpathcurveto{\pgfqpoint{0.745072in}{1.568343in}}{\pgfqpoint{0.748345in}{1.576243in}}{\pgfqpoint{0.748345in}{1.584480in}}%
\pgfpathcurveto{\pgfqpoint{0.748345in}{1.592716in}}{\pgfqpoint{0.745072in}{1.600616in}}{\pgfqpoint{0.739248in}{1.606440in}}%
\pgfpathcurveto{\pgfqpoint{0.733425in}{1.612264in}}{\pgfqpoint{0.725524in}{1.615536in}}{\pgfqpoint{0.717288in}{1.615536in}}%
\pgfpathcurveto{\pgfqpoint{0.709052in}{1.615536in}}{\pgfqpoint{0.701152in}{1.612264in}}{\pgfqpoint{0.695328in}{1.606440in}}%
\pgfpathcurveto{\pgfqpoint{0.689504in}{1.600616in}}{\pgfqpoint{0.686232in}{1.592716in}}{\pgfqpoint{0.686232in}{1.584480in}}%
\pgfpathcurveto{\pgfqpoint{0.686232in}{1.576243in}}{\pgfqpoint{0.689504in}{1.568343in}}{\pgfqpoint{0.695328in}{1.562519in}}%
\pgfpathcurveto{\pgfqpoint{0.701152in}{1.556695in}}{\pgfqpoint{0.709052in}{1.553423in}}{\pgfqpoint{0.717288in}{1.553423in}}%
\pgfpathclose%
\pgfusepath{stroke,fill}%
\end{pgfscope}%
\begin{pgfscope}%
\pgfpathrectangle{\pgfqpoint{0.100000in}{0.220728in}}{\pgfqpoint{3.696000in}{3.696000in}}%
\pgfusepath{clip}%
\pgfsetbuttcap%
\pgfsetroundjoin%
\definecolor{currentfill}{rgb}{0.121569,0.466667,0.705882}%
\pgfsetfillcolor{currentfill}%
\pgfsetfillopacity{0.306023}%
\pgfsetlinewidth{1.003750pt}%
\definecolor{currentstroke}{rgb}{0.121569,0.466667,0.705882}%
\pgfsetstrokecolor{currentstroke}%
\pgfsetstrokeopacity{0.306023}%
\pgfsetdash{}{0pt}%
\pgfpathmoveto{\pgfqpoint{0.718561in}{1.555770in}}%
\pgfpathcurveto{\pgfqpoint{0.726797in}{1.555770in}}{\pgfqpoint{0.734697in}{1.559043in}}{\pgfqpoint{0.740521in}{1.564867in}}%
\pgfpathcurveto{\pgfqpoint{0.746345in}{1.570691in}}{\pgfqpoint{0.749618in}{1.578591in}}{\pgfqpoint{0.749618in}{1.586827in}}%
\pgfpathcurveto{\pgfqpoint{0.749618in}{1.595063in}}{\pgfqpoint{0.746345in}{1.602963in}}{\pgfqpoint{0.740521in}{1.608787in}}%
\pgfpathcurveto{\pgfqpoint{0.734697in}{1.614611in}}{\pgfqpoint{0.726797in}{1.617883in}}{\pgfqpoint{0.718561in}{1.617883in}}%
\pgfpathcurveto{\pgfqpoint{0.710325in}{1.617883in}}{\pgfqpoint{0.702425in}{1.614611in}}{\pgfqpoint{0.696601in}{1.608787in}}%
\pgfpathcurveto{\pgfqpoint{0.690777in}{1.602963in}}{\pgfqpoint{0.687505in}{1.595063in}}{\pgfqpoint{0.687505in}{1.586827in}}%
\pgfpathcurveto{\pgfqpoint{0.687505in}{1.578591in}}{\pgfqpoint{0.690777in}{1.570691in}}{\pgfqpoint{0.696601in}{1.564867in}}%
\pgfpathcurveto{\pgfqpoint{0.702425in}{1.559043in}}{\pgfqpoint{0.710325in}{1.555770in}}{\pgfqpoint{0.718561in}{1.555770in}}%
\pgfpathclose%
\pgfusepath{stroke,fill}%
\end{pgfscope}%
\begin{pgfscope}%
\pgfpathrectangle{\pgfqpoint{0.100000in}{0.220728in}}{\pgfqpoint{3.696000in}{3.696000in}}%
\pgfusepath{clip}%
\pgfsetbuttcap%
\pgfsetroundjoin%
\definecolor{currentfill}{rgb}{0.121569,0.466667,0.705882}%
\pgfsetfillcolor{currentfill}%
\pgfsetfillopacity{0.306023}%
\pgfsetlinewidth{1.003750pt}%
\definecolor{currentstroke}{rgb}{0.121569,0.466667,0.705882}%
\pgfsetstrokecolor{currentstroke}%
\pgfsetstrokeopacity{0.306023}%
\pgfsetdash{}{0pt}%
\pgfpathmoveto{\pgfqpoint{0.718561in}{1.555770in}}%
\pgfpathcurveto{\pgfqpoint{0.726797in}{1.555770in}}{\pgfqpoint{0.734697in}{1.559043in}}{\pgfqpoint{0.740521in}{1.564867in}}%
\pgfpathcurveto{\pgfqpoint{0.746345in}{1.570691in}}{\pgfqpoint{0.749618in}{1.578591in}}{\pgfqpoint{0.749618in}{1.586827in}}%
\pgfpathcurveto{\pgfqpoint{0.749618in}{1.595063in}}{\pgfqpoint{0.746345in}{1.602963in}}{\pgfqpoint{0.740521in}{1.608787in}}%
\pgfpathcurveto{\pgfqpoint{0.734697in}{1.614611in}}{\pgfqpoint{0.726797in}{1.617883in}}{\pgfqpoint{0.718561in}{1.617883in}}%
\pgfpathcurveto{\pgfqpoint{0.710325in}{1.617883in}}{\pgfqpoint{0.702425in}{1.614611in}}{\pgfqpoint{0.696601in}{1.608787in}}%
\pgfpathcurveto{\pgfqpoint{0.690777in}{1.602963in}}{\pgfqpoint{0.687505in}{1.595063in}}{\pgfqpoint{0.687505in}{1.586827in}}%
\pgfpathcurveto{\pgfqpoint{0.687505in}{1.578591in}}{\pgfqpoint{0.690777in}{1.570691in}}{\pgfqpoint{0.696601in}{1.564867in}}%
\pgfpathcurveto{\pgfqpoint{0.702425in}{1.559043in}}{\pgfqpoint{0.710325in}{1.555770in}}{\pgfqpoint{0.718561in}{1.555770in}}%
\pgfpathclose%
\pgfusepath{stroke,fill}%
\end{pgfscope}%
\begin{pgfscope}%
\pgfpathrectangle{\pgfqpoint{0.100000in}{0.220728in}}{\pgfqpoint{3.696000in}{3.696000in}}%
\pgfusepath{clip}%
\pgfsetbuttcap%
\pgfsetroundjoin%
\definecolor{currentfill}{rgb}{0.121569,0.466667,0.705882}%
\pgfsetfillcolor{currentfill}%
\pgfsetfillopacity{0.306023}%
\pgfsetlinewidth{1.003750pt}%
\definecolor{currentstroke}{rgb}{0.121569,0.466667,0.705882}%
\pgfsetstrokecolor{currentstroke}%
\pgfsetstrokeopacity{0.306023}%
\pgfsetdash{}{0pt}%
\pgfpathmoveto{\pgfqpoint{0.718561in}{1.555770in}}%
\pgfpathcurveto{\pgfqpoint{0.726797in}{1.555770in}}{\pgfqpoint{0.734697in}{1.559043in}}{\pgfqpoint{0.740521in}{1.564867in}}%
\pgfpathcurveto{\pgfqpoint{0.746345in}{1.570691in}}{\pgfqpoint{0.749618in}{1.578591in}}{\pgfqpoint{0.749618in}{1.586827in}}%
\pgfpathcurveto{\pgfqpoint{0.749618in}{1.595063in}}{\pgfqpoint{0.746345in}{1.602963in}}{\pgfqpoint{0.740521in}{1.608787in}}%
\pgfpathcurveto{\pgfqpoint{0.734697in}{1.614611in}}{\pgfqpoint{0.726797in}{1.617883in}}{\pgfqpoint{0.718561in}{1.617883in}}%
\pgfpathcurveto{\pgfqpoint{0.710325in}{1.617883in}}{\pgfqpoint{0.702425in}{1.614611in}}{\pgfqpoint{0.696601in}{1.608787in}}%
\pgfpathcurveto{\pgfqpoint{0.690777in}{1.602963in}}{\pgfqpoint{0.687505in}{1.595063in}}{\pgfqpoint{0.687505in}{1.586827in}}%
\pgfpathcurveto{\pgfqpoint{0.687505in}{1.578591in}}{\pgfqpoint{0.690777in}{1.570691in}}{\pgfqpoint{0.696601in}{1.564867in}}%
\pgfpathcurveto{\pgfqpoint{0.702425in}{1.559043in}}{\pgfqpoint{0.710325in}{1.555770in}}{\pgfqpoint{0.718561in}{1.555770in}}%
\pgfpathclose%
\pgfusepath{stroke,fill}%
\end{pgfscope}%
\begin{pgfscope}%
\pgfpathrectangle{\pgfqpoint{0.100000in}{0.220728in}}{\pgfqpoint{3.696000in}{3.696000in}}%
\pgfusepath{clip}%
\pgfsetbuttcap%
\pgfsetroundjoin%
\definecolor{currentfill}{rgb}{0.121569,0.466667,0.705882}%
\pgfsetfillcolor{currentfill}%
\pgfsetfillopacity{0.306023}%
\pgfsetlinewidth{1.003750pt}%
\definecolor{currentstroke}{rgb}{0.121569,0.466667,0.705882}%
\pgfsetstrokecolor{currentstroke}%
\pgfsetstrokeopacity{0.306023}%
\pgfsetdash{}{0pt}%
\pgfpathmoveto{\pgfqpoint{0.718561in}{1.555770in}}%
\pgfpathcurveto{\pgfqpoint{0.726797in}{1.555770in}}{\pgfqpoint{0.734697in}{1.559043in}}{\pgfqpoint{0.740521in}{1.564867in}}%
\pgfpathcurveto{\pgfqpoint{0.746345in}{1.570691in}}{\pgfqpoint{0.749618in}{1.578591in}}{\pgfqpoint{0.749618in}{1.586827in}}%
\pgfpathcurveto{\pgfqpoint{0.749618in}{1.595063in}}{\pgfqpoint{0.746345in}{1.602963in}}{\pgfqpoint{0.740521in}{1.608787in}}%
\pgfpathcurveto{\pgfqpoint{0.734697in}{1.614611in}}{\pgfqpoint{0.726797in}{1.617883in}}{\pgfqpoint{0.718561in}{1.617883in}}%
\pgfpathcurveto{\pgfqpoint{0.710325in}{1.617883in}}{\pgfqpoint{0.702425in}{1.614611in}}{\pgfqpoint{0.696601in}{1.608787in}}%
\pgfpathcurveto{\pgfqpoint{0.690777in}{1.602963in}}{\pgfqpoint{0.687505in}{1.595063in}}{\pgfqpoint{0.687505in}{1.586827in}}%
\pgfpathcurveto{\pgfqpoint{0.687505in}{1.578591in}}{\pgfqpoint{0.690777in}{1.570691in}}{\pgfqpoint{0.696601in}{1.564867in}}%
\pgfpathcurveto{\pgfqpoint{0.702425in}{1.559043in}}{\pgfqpoint{0.710325in}{1.555770in}}{\pgfqpoint{0.718561in}{1.555770in}}%
\pgfpathclose%
\pgfusepath{stroke,fill}%
\end{pgfscope}%
\begin{pgfscope}%
\pgfpathrectangle{\pgfqpoint{0.100000in}{0.220728in}}{\pgfqpoint{3.696000in}{3.696000in}}%
\pgfusepath{clip}%
\pgfsetbuttcap%
\pgfsetroundjoin%
\definecolor{currentfill}{rgb}{0.121569,0.466667,0.705882}%
\pgfsetfillcolor{currentfill}%
\pgfsetfillopacity{0.306023}%
\pgfsetlinewidth{1.003750pt}%
\definecolor{currentstroke}{rgb}{0.121569,0.466667,0.705882}%
\pgfsetstrokecolor{currentstroke}%
\pgfsetstrokeopacity{0.306023}%
\pgfsetdash{}{0pt}%
\pgfpathmoveto{\pgfqpoint{0.718561in}{1.555770in}}%
\pgfpathcurveto{\pgfqpoint{0.726797in}{1.555770in}}{\pgfqpoint{0.734697in}{1.559043in}}{\pgfqpoint{0.740521in}{1.564867in}}%
\pgfpathcurveto{\pgfqpoint{0.746345in}{1.570691in}}{\pgfqpoint{0.749618in}{1.578591in}}{\pgfqpoint{0.749618in}{1.586827in}}%
\pgfpathcurveto{\pgfqpoint{0.749618in}{1.595063in}}{\pgfqpoint{0.746345in}{1.602963in}}{\pgfqpoint{0.740521in}{1.608787in}}%
\pgfpathcurveto{\pgfqpoint{0.734697in}{1.614611in}}{\pgfqpoint{0.726797in}{1.617883in}}{\pgfqpoint{0.718561in}{1.617883in}}%
\pgfpathcurveto{\pgfqpoint{0.710325in}{1.617883in}}{\pgfqpoint{0.702425in}{1.614611in}}{\pgfqpoint{0.696601in}{1.608787in}}%
\pgfpathcurveto{\pgfqpoint{0.690777in}{1.602963in}}{\pgfqpoint{0.687505in}{1.595063in}}{\pgfqpoint{0.687505in}{1.586827in}}%
\pgfpathcurveto{\pgfqpoint{0.687505in}{1.578591in}}{\pgfqpoint{0.690777in}{1.570691in}}{\pgfqpoint{0.696601in}{1.564867in}}%
\pgfpathcurveto{\pgfqpoint{0.702425in}{1.559043in}}{\pgfqpoint{0.710325in}{1.555770in}}{\pgfqpoint{0.718561in}{1.555770in}}%
\pgfpathclose%
\pgfusepath{stroke,fill}%
\end{pgfscope}%
\begin{pgfscope}%
\pgfpathrectangle{\pgfqpoint{0.100000in}{0.220728in}}{\pgfqpoint{3.696000in}{3.696000in}}%
\pgfusepath{clip}%
\pgfsetbuttcap%
\pgfsetroundjoin%
\definecolor{currentfill}{rgb}{0.121569,0.466667,0.705882}%
\pgfsetfillcolor{currentfill}%
\pgfsetfillopacity{0.306023}%
\pgfsetlinewidth{1.003750pt}%
\definecolor{currentstroke}{rgb}{0.121569,0.466667,0.705882}%
\pgfsetstrokecolor{currentstroke}%
\pgfsetstrokeopacity{0.306023}%
\pgfsetdash{}{0pt}%
\pgfpathmoveto{\pgfqpoint{0.718561in}{1.555770in}}%
\pgfpathcurveto{\pgfqpoint{0.726797in}{1.555770in}}{\pgfqpoint{0.734697in}{1.559043in}}{\pgfqpoint{0.740521in}{1.564867in}}%
\pgfpathcurveto{\pgfqpoint{0.746345in}{1.570691in}}{\pgfqpoint{0.749618in}{1.578591in}}{\pgfqpoint{0.749618in}{1.586827in}}%
\pgfpathcurveto{\pgfqpoint{0.749618in}{1.595063in}}{\pgfqpoint{0.746345in}{1.602963in}}{\pgfqpoint{0.740521in}{1.608787in}}%
\pgfpathcurveto{\pgfqpoint{0.734697in}{1.614611in}}{\pgfqpoint{0.726797in}{1.617883in}}{\pgfqpoint{0.718561in}{1.617883in}}%
\pgfpathcurveto{\pgfqpoint{0.710325in}{1.617883in}}{\pgfqpoint{0.702425in}{1.614611in}}{\pgfqpoint{0.696601in}{1.608787in}}%
\pgfpathcurveto{\pgfqpoint{0.690777in}{1.602963in}}{\pgfqpoint{0.687505in}{1.595063in}}{\pgfqpoint{0.687505in}{1.586827in}}%
\pgfpathcurveto{\pgfqpoint{0.687505in}{1.578591in}}{\pgfqpoint{0.690777in}{1.570691in}}{\pgfqpoint{0.696601in}{1.564867in}}%
\pgfpathcurveto{\pgfqpoint{0.702425in}{1.559043in}}{\pgfqpoint{0.710325in}{1.555770in}}{\pgfqpoint{0.718561in}{1.555770in}}%
\pgfpathclose%
\pgfusepath{stroke,fill}%
\end{pgfscope}%
\begin{pgfscope}%
\pgfpathrectangle{\pgfqpoint{0.100000in}{0.220728in}}{\pgfqpoint{3.696000in}{3.696000in}}%
\pgfusepath{clip}%
\pgfsetbuttcap%
\pgfsetroundjoin%
\definecolor{currentfill}{rgb}{0.121569,0.466667,0.705882}%
\pgfsetfillcolor{currentfill}%
\pgfsetfillopacity{0.306023}%
\pgfsetlinewidth{1.003750pt}%
\definecolor{currentstroke}{rgb}{0.121569,0.466667,0.705882}%
\pgfsetstrokecolor{currentstroke}%
\pgfsetstrokeopacity{0.306023}%
\pgfsetdash{}{0pt}%
\pgfpathmoveto{\pgfqpoint{0.718561in}{1.555770in}}%
\pgfpathcurveto{\pgfqpoint{0.726797in}{1.555770in}}{\pgfqpoint{0.734697in}{1.559043in}}{\pgfqpoint{0.740521in}{1.564867in}}%
\pgfpathcurveto{\pgfqpoint{0.746345in}{1.570691in}}{\pgfqpoint{0.749618in}{1.578591in}}{\pgfqpoint{0.749618in}{1.586827in}}%
\pgfpathcurveto{\pgfqpoint{0.749618in}{1.595063in}}{\pgfqpoint{0.746345in}{1.602963in}}{\pgfqpoint{0.740521in}{1.608787in}}%
\pgfpathcurveto{\pgfqpoint{0.734697in}{1.614611in}}{\pgfqpoint{0.726797in}{1.617883in}}{\pgfqpoint{0.718561in}{1.617883in}}%
\pgfpathcurveto{\pgfqpoint{0.710325in}{1.617883in}}{\pgfqpoint{0.702425in}{1.614611in}}{\pgfqpoint{0.696601in}{1.608787in}}%
\pgfpathcurveto{\pgfqpoint{0.690777in}{1.602963in}}{\pgfqpoint{0.687505in}{1.595063in}}{\pgfqpoint{0.687505in}{1.586827in}}%
\pgfpathcurveto{\pgfqpoint{0.687505in}{1.578591in}}{\pgfqpoint{0.690777in}{1.570691in}}{\pgfqpoint{0.696601in}{1.564867in}}%
\pgfpathcurveto{\pgfqpoint{0.702425in}{1.559043in}}{\pgfqpoint{0.710325in}{1.555770in}}{\pgfqpoint{0.718561in}{1.555770in}}%
\pgfpathclose%
\pgfusepath{stroke,fill}%
\end{pgfscope}%
\begin{pgfscope}%
\pgfpathrectangle{\pgfqpoint{0.100000in}{0.220728in}}{\pgfqpoint{3.696000in}{3.696000in}}%
\pgfusepath{clip}%
\pgfsetbuttcap%
\pgfsetroundjoin%
\definecolor{currentfill}{rgb}{0.121569,0.466667,0.705882}%
\pgfsetfillcolor{currentfill}%
\pgfsetfillopacity{0.306023}%
\pgfsetlinewidth{1.003750pt}%
\definecolor{currentstroke}{rgb}{0.121569,0.466667,0.705882}%
\pgfsetstrokecolor{currentstroke}%
\pgfsetstrokeopacity{0.306023}%
\pgfsetdash{}{0pt}%
\pgfpathmoveto{\pgfqpoint{0.718561in}{1.555770in}}%
\pgfpathcurveto{\pgfqpoint{0.726797in}{1.555770in}}{\pgfqpoint{0.734697in}{1.559043in}}{\pgfqpoint{0.740521in}{1.564867in}}%
\pgfpathcurveto{\pgfqpoint{0.746345in}{1.570691in}}{\pgfqpoint{0.749618in}{1.578591in}}{\pgfqpoint{0.749618in}{1.586827in}}%
\pgfpathcurveto{\pgfqpoint{0.749618in}{1.595063in}}{\pgfqpoint{0.746345in}{1.602963in}}{\pgfqpoint{0.740521in}{1.608787in}}%
\pgfpathcurveto{\pgfqpoint{0.734697in}{1.614611in}}{\pgfqpoint{0.726797in}{1.617883in}}{\pgfqpoint{0.718561in}{1.617883in}}%
\pgfpathcurveto{\pgfqpoint{0.710325in}{1.617883in}}{\pgfqpoint{0.702425in}{1.614611in}}{\pgfqpoint{0.696601in}{1.608787in}}%
\pgfpathcurveto{\pgfqpoint{0.690777in}{1.602963in}}{\pgfqpoint{0.687505in}{1.595063in}}{\pgfqpoint{0.687505in}{1.586827in}}%
\pgfpathcurveto{\pgfqpoint{0.687505in}{1.578591in}}{\pgfqpoint{0.690777in}{1.570691in}}{\pgfqpoint{0.696601in}{1.564867in}}%
\pgfpathcurveto{\pgfqpoint{0.702425in}{1.559043in}}{\pgfqpoint{0.710325in}{1.555770in}}{\pgfqpoint{0.718561in}{1.555770in}}%
\pgfpathclose%
\pgfusepath{stroke,fill}%
\end{pgfscope}%
\begin{pgfscope}%
\pgfpathrectangle{\pgfqpoint{0.100000in}{0.220728in}}{\pgfqpoint{3.696000in}{3.696000in}}%
\pgfusepath{clip}%
\pgfsetbuttcap%
\pgfsetroundjoin%
\definecolor{currentfill}{rgb}{0.121569,0.466667,0.705882}%
\pgfsetfillcolor{currentfill}%
\pgfsetfillopacity{0.306023}%
\pgfsetlinewidth{1.003750pt}%
\definecolor{currentstroke}{rgb}{0.121569,0.466667,0.705882}%
\pgfsetstrokecolor{currentstroke}%
\pgfsetstrokeopacity{0.306023}%
\pgfsetdash{}{0pt}%
\pgfpathmoveto{\pgfqpoint{0.718561in}{1.555770in}}%
\pgfpathcurveto{\pgfqpoint{0.726797in}{1.555770in}}{\pgfqpoint{0.734697in}{1.559043in}}{\pgfqpoint{0.740521in}{1.564867in}}%
\pgfpathcurveto{\pgfqpoint{0.746345in}{1.570691in}}{\pgfqpoint{0.749618in}{1.578591in}}{\pgfqpoint{0.749618in}{1.586827in}}%
\pgfpathcurveto{\pgfqpoint{0.749618in}{1.595063in}}{\pgfqpoint{0.746345in}{1.602963in}}{\pgfqpoint{0.740521in}{1.608787in}}%
\pgfpathcurveto{\pgfqpoint{0.734697in}{1.614611in}}{\pgfqpoint{0.726797in}{1.617883in}}{\pgfqpoint{0.718561in}{1.617883in}}%
\pgfpathcurveto{\pgfqpoint{0.710325in}{1.617883in}}{\pgfqpoint{0.702425in}{1.614611in}}{\pgfqpoint{0.696601in}{1.608787in}}%
\pgfpathcurveto{\pgfqpoint{0.690777in}{1.602963in}}{\pgfqpoint{0.687505in}{1.595063in}}{\pgfqpoint{0.687505in}{1.586827in}}%
\pgfpathcurveto{\pgfqpoint{0.687505in}{1.578591in}}{\pgfqpoint{0.690777in}{1.570691in}}{\pgfqpoint{0.696601in}{1.564867in}}%
\pgfpathcurveto{\pgfqpoint{0.702425in}{1.559043in}}{\pgfqpoint{0.710325in}{1.555770in}}{\pgfqpoint{0.718561in}{1.555770in}}%
\pgfpathclose%
\pgfusepath{stroke,fill}%
\end{pgfscope}%
\begin{pgfscope}%
\pgfpathrectangle{\pgfqpoint{0.100000in}{0.220728in}}{\pgfqpoint{3.696000in}{3.696000in}}%
\pgfusepath{clip}%
\pgfsetbuttcap%
\pgfsetroundjoin%
\definecolor{currentfill}{rgb}{0.121569,0.466667,0.705882}%
\pgfsetfillcolor{currentfill}%
\pgfsetfillopacity{0.306023}%
\pgfsetlinewidth{1.003750pt}%
\definecolor{currentstroke}{rgb}{0.121569,0.466667,0.705882}%
\pgfsetstrokecolor{currentstroke}%
\pgfsetstrokeopacity{0.306023}%
\pgfsetdash{}{0pt}%
\pgfpathmoveto{\pgfqpoint{0.718561in}{1.555770in}}%
\pgfpathcurveto{\pgfqpoint{0.726797in}{1.555770in}}{\pgfqpoint{0.734698in}{1.559043in}}{\pgfqpoint{0.740521in}{1.564867in}}%
\pgfpathcurveto{\pgfqpoint{0.746345in}{1.570691in}}{\pgfqpoint{0.749618in}{1.578591in}}{\pgfqpoint{0.749618in}{1.586827in}}%
\pgfpathcurveto{\pgfqpoint{0.749618in}{1.595063in}}{\pgfqpoint{0.746345in}{1.602963in}}{\pgfqpoint{0.740521in}{1.608787in}}%
\pgfpathcurveto{\pgfqpoint{0.734698in}{1.614611in}}{\pgfqpoint{0.726797in}{1.617883in}}{\pgfqpoint{0.718561in}{1.617883in}}%
\pgfpathcurveto{\pgfqpoint{0.710325in}{1.617883in}}{\pgfqpoint{0.702425in}{1.614611in}}{\pgfqpoint{0.696601in}{1.608787in}}%
\pgfpathcurveto{\pgfqpoint{0.690777in}{1.602963in}}{\pgfqpoint{0.687505in}{1.595063in}}{\pgfqpoint{0.687505in}{1.586827in}}%
\pgfpathcurveto{\pgfqpoint{0.687505in}{1.578591in}}{\pgfqpoint{0.690777in}{1.570691in}}{\pgfqpoint{0.696601in}{1.564867in}}%
\pgfpathcurveto{\pgfqpoint{0.702425in}{1.559043in}}{\pgfqpoint{0.710325in}{1.555770in}}{\pgfqpoint{0.718561in}{1.555770in}}%
\pgfpathclose%
\pgfusepath{stroke,fill}%
\end{pgfscope}%
\begin{pgfscope}%
\pgfpathrectangle{\pgfqpoint{0.100000in}{0.220728in}}{\pgfqpoint{3.696000in}{3.696000in}}%
\pgfusepath{clip}%
\pgfsetbuttcap%
\pgfsetroundjoin%
\definecolor{currentfill}{rgb}{0.121569,0.466667,0.705882}%
\pgfsetfillcolor{currentfill}%
\pgfsetfillopacity{0.306023}%
\pgfsetlinewidth{1.003750pt}%
\definecolor{currentstroke}{rgb}{0.121569,0.466667,0.705882}%
\pgfsetstrokecolor{currentstroke}%
\pgfsetstrokeopacity{0.306023}%
\pgfsetdash{}{0pt}%
\pgfpathmoveto{\pgfqpoint{0.718561in}{1.555770in}}%
\pgfpathcurveto{\pgfqpoint{0.726798in}{1.555770in}}{\pgfqpoint{0.734698in}{1.559043in}}{\pgfqpoint{0.740521in}{1.564867in}}%
\pgfpathcurveto{\pgfqpoint{0.746345in}{1.570691in}}{\pgfqpoint{0.749618in}{1.578591in}}{\pgfqpoint{0.749618in}{1.586827in}}%
\pgfpathcurveto{\pgfqpoint{0.749618in}{1.595063in}}{\pgfqpoint{0.746345in}{1.602963in}}{\pgfqpoint{0.740521in}{1.608787in}}%
\pgfpathcurveto{\pgfqpoint{0.734698in}{1.614611in}}{\pgfqpoint{0.726798in}{1.617883in}}{\pgfqpoint{0.718561in}{1.617883in}}%
\pgfpathcurveto{\pgfqpoint{0.710325in}{1.617883in}}{\pgfqpoint{0.702425in}{1.614611in}}{\pgfqpoint{0.696601in}{1.608787in}}%
\pgfpathcurveto{\pgfqpoint{0.690777in}{1.602963in}}{\pgfqpoint{0.687505in}{1.595063in}}{\pgfqpoint{0.687505in}{1.586827in}}%
\pgfpathcurveto{\pgfqpoint{0.687505in}{1.578591in}}{\pgfqpoint{0.690777in}{1.570691in}}{\pgfqpoint{0.696601in}{1.564867in}}%
\pgfpathcurveto{\pgfqpoint{0.702425in}{1.559043in}}{\pgfqpoint{0.710325in}{1.555770in}}{\pgfqpoint{0.718561in}{1.555770in}}%
\pgfpathclose%
\pgfusepath{stroke,fill}%
\end{pgfscope}%
\begin{pgfscope}%
\pgfpathrectangle{\pgfqpoint{0.100000in}{0.220728in}}{\pgfqpoint{3.696000in}{3.696000in}}%
\pgfusepath{clip}%
\pgfsetbuttcap%
\pgfsetroundjoin%
\definecolor{currentfill}{rgb}{0.121569,0.466667,0.705882}%
\pgfsetfillcolor{currentfill}%
\pgfsetfillopacity{0.306023}%
\pgfsetlinewidth{1.003750pt}%
\definecolor{currentstroke}{rgb}{0.121569,0.466667,0.705882}%
\pgfsetstrokecolor{currentstroke}%
\pgfsetstrokeopacity{0.306023}%
\pgfsetdash{}{0pt}%
\pgfpathmoveto{\pgfqpoint{0.718561in}{1.555770in}}%
\pgfpathcurveto{\pgfqpoint{0.726798in}{1.555770in}}{\pgfqpoint{0.734698in}{1.559043in}}{\pgfqpoint{0.740522in}{1.564867in}}%
\pgfpathcurveto{\pgfqpoint{0.746345in}{1.570691in}}{\pgfqpoint{0.749618in}{1.578591in}}{\pgfqpoint{0.749618in}{1.586827in}}%
\pgfpathcurveto{\pgfqpoint{0.749618in}{1.595063in}}{\pgfqpoint{0.746345in}{1.602963in}}{\pgfqpoint{0.740522in}{1.608787in}}%
\pgfpathcurveto{\pgfqpoint{0.734698in}{1.614611in}}{\pgfqpoint{0.726798in}{1.617883in}}{\pgfqpoint{0.718561in}{1.617883in}}%
\pgfpathcurveto{\pgfqpoint{0.710325in}{1.617883in}}{\pgfqpoint{0.702425in}{1.614611in}}{\pgfqpoint{0.696601in}{1.608787in}}%
\pgfpathcurveto{\pgfqpoint{0.690777in}{1.602963in}}{\pgfqpoint{0.687505in}{1.595063in}}{\pgfqpoint{0.687505in}{1.586827in}}%
\pgfpathcurveto{\pgfqpoint{0.687505in}{1.578591in}}{\pgfqpoint{0.690777in}{1.570691in}}{\pgfqpoint{0.696601in}{1.564867in}}%
\pgfpathcurveto{\pgfqpoint{0.702425in}{1.559043in}}{\pgfqpoint{0.710325in}{1.555770in}}{\pgfqpoint{0.718561in}{1.555770in}}%
\pgfpathclose%
\pgfusepath{stroke,fill}%
\end{pgfscope}%
\begin{pgfscope}%
\pgfpathrectangle{\pgfqpoint{0.100000in}{0.220728in}}{\pgfqpoint{3.696000in}{3.696000in}}%
\pgfusepath{clip}%
\pgfsetbuttcap%
\pgfsetroundjoin%
\definecolor{currentfill}{rgb}{0.121569,0.466667,0.705882}%
\pgfsetfillcolor{currentfill}%
\pgfsetfillopacity{0.306023}%
\pgfsetlinewidth{1.003750pt}%
\definecolor{currentstroke}{rgb}{0.121569,0.466667,0.705882}%
\pgfsetstrokecolor{currentstroke}%
\pgfsetstrokeopacity{0.306023}%
\pgfsetdash{}{0pt}%
\pgfpathmoveto{\pgfqpoint{0.718561in}{1.555771in}}%
\pgfpathcurveto{\pgfqpoint{0.726798in}{1.555771in}}{\pgfqpoint{0.734698in}{1.559043in}}{\pgfqpoint{0.740522in}{1.564867in}}%
\pgfpathcurveto{\pgfqpoint{0.746346in}{1.570691in}}{\pgfqpoint{0.749618in}{1.578591in}}{\pgfqpoint{0.749618in}{1.586827in}}%
\pgfpathcurveto{\pgfqpoint{0.749618in}{1.595063in}}{\pgfqpoint{0.746346in}{1.602963in}}{\pgfqpoint{0.740522in}{1.608787in}}%
\pgfpathcurveto{\pgfqpoint{0.734698in}{1.614611in}}{\pgfqpoint{0.726798in}{1.617884in}}{\pgfqpoint{0.718561in}{1.617884in}}%
\pgfpathcurveto{\pgfqpoint{0.710325in}{1.617884in}}{\pgfqpoint{0.702425in}{1.614611in}}{\pgfqpoint{0.696601in}{1.608787in}}%
\pgfpathcurveto{\pgfqpoint{0.690777in}{1.602963in}}{\pgfqpoint{0.687505in}{1.595063in}}{\pgfqpoint{0.687505in}{1.586827in}}%
\pgfpathcurveto{\pgfqpoint{0.687505in}{1.578591in}}{\pgfqpoint{0.690777in}{1.570691in}}{\pgfqpoint{0.696601in}{1.564867in}}%
\pgfpathcurveto{\pgfqpoint{0.702425in}{1.559043in}}{\pgfqpoint{0.710325in}{1.555771in}}{\pgfqpoint{0.718561in}{1.555771in}}%
\pgfpathclose%
\pgfusepath{stroke,fill}%
\end{pgfscope}%
\begin{pgfscope}%
\pgfpathrectangle{\pgfqpoint{0.100000in}{0.220728in}}{\pgfqpoint{3.696000in}{3.696000in}}%
\pgfusepath{clip}%
\pgfsetbuttcap%
\pgfsetroundjoin%
\definecolor{currentfill}{rgb}{0.121569,0.466667,0.705882}%
\pgfsetfillcolor{currentfill}%
\pgfsetfillopacity{0.306024}%
\pgfsetlinewidth{1.003750pt}%
\definecolor{currentstroke}{rgb}{0.121569,0.466667,0.705882}%
\pgfsetstrokecolor{currentstroke}%
\pgfsetstrokeopacity{0.306024}%
\pgfsetdash{}{0pt}%
\pgfpathmoveto{\pgfqpoint{0.718562in}{1.555771in}}%
\pgfpathcurveto{\pgfqpoint{0.726798in}{1.555771in}}{\pgfqpoint{0.734698in}{1.559043in}}{\pgfqpoint{0.740522in}{1.564867in}}%
\pgfpathcurveto{\pgfqpoint{0.746346in}{1.570691in}}{\pgfqpoint{0.749618in}{1.578591in}}{\pgfqpoint{0.749618in}{1.586827in}}%
\pgfpathcurveto{\pgfqpoint{0.749618in}{1.595064in}}{\pgfqpoint{0.746346in}{1.602964in}}{\pgfqpoint{0.740522in}{1.608788in}}%
\pgfpathcurveto{\pgfqpoint{0.734698in}{1.614612in}}{\pgfqpoint{0.726798in}{1.617884in}}{\pgfqpoint{0.718562in}{1.617884in}}%
\pgfpathcurveto{\pgfqpoint{0.710325in}{1.617884in}}{\pgfqpoint{0.702425in}{1.614612in}}{\pgfqpoint{0.696601in}{1.608788in}}%
\pgfpathcurveto{\pgfqpoint{0.690778in}{1.602964in}}{\pgfqpoint{0.687505in}{1.595064in}}{\pgfqpoint{0.687505in}{1.586827in}}%
\pgfpathcurveto{\pgfqpoint{0.687505in}{1.578591in}}{\pgfqpoint{0.690778in}{1.570691in}}{\pgfqpoint{0.696601in}{1.564867in}}%
\pgfpathcurveto{\pgfqpoint{0.702425in}{1.559043in}}{\pgfqpoint{0.710325in}{1.555771in}}{\pgfqpoint{0.718562in}{1.555771in}}%
\pgfpathclose%
\pgfusepath{stroke,fill}%
\end{pgfscope}%
\begin{pgfscope}%
\pgfpathrectangle{\pgfqpoint{0.100000in}{0.220728in}}{\pgfqpoint{3.696000in}{3.696000in}}%
\pgfusepath{clip}%
\pgfsetbuttcap%
\pgfsetroundjoin%
\definecolor{currentfill}{rgb}{0.121569,0.466667,0.705882}%
\pgfsetfillcolor{currentfill}%
\pgfsetfillopacity{0.306025}%
\pgfsetlinewidth{1.003750pt}%
\definecolor{currentstroke}{rgb}{0.121569,0.466667,0.705882}%
\pgfsetstrokecolor{currentstroke}%
\pgfsetstrokeopacity{0.306025}%
\pgfsetdash{}{0pt}%
\pgfpathmoveto{\pgfqpoint{0.718562in}{1.555772in}}%
\pgfpathcurveto{\pgfqpoint{0.726798in}{1.555772in}}{\pgfqpoint{0.734699in}{1.559044in}}{\pgfqpoint{0.740522in}{1.564868in}}%
\pgfpathcurveto{\pgfqpoint{0.746346in}{1.570692in}}{\pgfqpoint{0.749619in}{1.578592in}}{\pgfqpoint{0.749619in}{1.586828in}}%
\pgfpathcurveto{\pgfqpoint{0.749619in}{1.595064in}}{\pgfqpoint{0.746346in}{1.602964in}}{\pgfqpoint{0.740522in}{1.608788in}}%
\pgfpathcurveto{\pgfqpoint{0.734699in}{1.614612in}}{\pgfqpoint{0.726798in}{1.617885in}}{\pgfqpoint{0.718562in}{1.617885in}}%
\pgfpathcurveto{\pgfqpoint{0.710326in}{1.617885in}}{\pgfqpoint{0.702426in}{1.614612in}}{\pgfqpoint{0.696602in}{1.608788in}}%
\pgfpathcurveto{\pgfqpoint{0.690778in}{1.602964in}}{\pgfqpoint{0.687506in}{1.595064in}}{\pgfqpoint{0.687506in}{1.586828in}}%
\pgfpathcurveto{\pgfqpoint{0.687506in}{1.578592in}}{\pgfqpoint{0.690778in}{1.570692in}}{\pgfqpoint{0.696602in}{1.564868in}}%
\pgfpathcurveto{\pgfqpoint{0.702426in}{1.559044in}}{\pgfqpoint{0.710326in}{1.555772in}}{\pgfqpoint{0.718562in}{1.555772in}}%
\pgfpathclose%
\pgfusepath{stroke,fill}%
\end{pgfscope}%
\begin{pgfscope}%
\pgfpathrectangle{\pgfqpoint{0.100000in}{0.220728in}}{\pgfqpoint{3.696000in}{3.696000in}}%
\pgfusepath{clip}%
\pgfsetbuttcap%
\pgfsetroundjoin%
\definecolor{currentfill}{rgb}{0.121569,0.466667,0.705882}%
\pgfsetfillcolor{currentfill}%
\pgfsetfillopacity{0.306026}%
\pgfsetlinewidth{1.003750pt}%
\definecolor{currentstroke}{rgb}{0.121569,0.466667,0.705882}%
\pgfsetstrokecolor{currentstroke}%
\pgfsetstrokeopacity{0.306026}%
\pgfsetdash{}{0pt}%
\pgfpathmoveto{\pgfqpoint{0.718563in}{1.555772in}}%
\pgfpathcurveto{\pgfqpoint{0.726799in}{1.555772in}}{\pgfqpoint{0.734699in}{1.559045in}}{\pgfqpoint{0.740523in}{1.564869in}}%
\pgfpathcurveto{\pgfqpoint{0.746347in}{1.570693in}}{\pgfqpoint{0.749620in}{1.578593in}}{\pgfqpoint{0.749620in}{1.586829in}}%
\pgfpathcurveto{\pgfqpoint{0.749620in}{1.595065in}}{\pgfqpoint{0.746347in}{1.602965in}}{\pgfqpoint{0.740523in}{1.608789in}}%
\pgfpathcurveto{\pgfqpoint{0.734699in}{1.614613in}}{\pgfqpoint{0.726799in}{1.617885in}}{\pgfqpoint{0.718563in}{1.617885in}}%
\pgfpathcurveto{\pgfqpoint{0.710327in}{1.617885in}}{\pgfqpoint{0.702427in}{1.614613in}}{\pgfqpoint{0.696603in}{1.608789in}}%
\pgfpathcurveto{\pgfqpoint{0.690779in}{1.602965in}}{\pgfqpoint{0.687507in}{1.595065in}}{\pgfqpoint{0.687507in}{1.586829in}}%
\pgfpathcurveto{\pgfqpoint{0.687507in}{1.578593in}}{\pgfqpoint{0.690779in}{1.570693in}}{\pgfqpoint{0.696603in}{1.564869in}}%
\pgfpathcurveto{\pgfqpoint{0.702427in}{1.559045in}}{\pgfqpoint{0.710327in}{1.555772in}}{\pgfqpoint{0.718563in}{1.555772in}}%
\pgfpathclose%
\pgfusepath{stroke,fill}%
\end{pgfscope}%
\begin{pgfscope}%
\pgfpathrectangle{\pgfqpoint{0.100000in}{0.220728in}}{\pgfqpoint{3.696000in}{3.696000in}}%
\pgfusepath{clip}%
\pgfsetbuttcap%
\pgfsetroundjoin%
\definecolor{currentfill}{rgb}{0.121569,0.466667,0.705882}%
\pgfsetfillcolor{currentfill}%
\pgfsetfillopacity{0.306029}%
\pgfsetlinewidth{1.003750pt}%
\definecolor{currentstroke}{rgb}{0.121569,0.466667,0.705882}%
\pgfsetstrokecolor{currentstroke}%
\pgfsetstrokeopacity{0.306029}%
\pgfsetdash{}{0pt}%
\pgfpathmoveto{\pgfqpoint{0.718565in}{1.555774in}}%
\pgfpathcurveto{\pgfqpoint{0.726801in}{1.555774in}}{\pgfqpoint{0.734701in}{1.559046in}}{\pgfqpoint{0.740525in}{1.564870in}}%
\pgfpathcurveto{\pgfqpoint{0.746349in}{1.570694in}}{\pgfqpoint{0.749621in}{1.578594in}}{\pgfqpoint{0.749621in}{1.586830in}}%
\pgfpathcurveto{\pgfqpoint{0.749621in}{1.595067in}}{\pgfqpoint{0.746349in}{1.602967in}}{\pgfqpoint{0.740525in}{1.608791in}}%
\pgfpathcurveto{\pgfqpoint{0.734701in}{1.614614in}}{\pgfqpoint{0.726801in}{1.617887in}}{\pgfqpoint{0.718565in}{1.617887in}}%
\pgfpathcurveto{\pgfqpoint{0.710328in}{1.617887in}}{\pgfqpoint{0.702428in}{1.614614in}}{\pgfqpoint{0.696604in}{1.608791in}}%
\pgfpathcurveto{\pgfqpoint{0.690780in}{1.602967in}}{\pgfqpoint{0.687508in}{1.595067in}}{\pgfqpoint{0.687508in}{1.586830in}}%
\pgfpathcurveto{\pgfqpoint{0.687508in}{1.578594in}}{\pgfqpoint{0.690780in}{1.570694in}}{\pgfqpoint{0.696604in}{1.564870in}}%
\pgfpathcurveto{\pgfqpoint{0.702428in}{1.559046in}}{\pgfqpoint{0.710328in}{1.555774in}}{\pgfqpoint{0.718565in}{1.555774in}}%
\pgfpathclose%
\pgfusepath{stroke,fill}%
\end{pgfscope}%
\begin{pgfscope}%
\pgfpathrectangle{\pgfqpoint{0.100000in}{0.220728in}}{\pgfqpoint{3.696000in}{3.696000in}}%
\pgfusepath{clip}%
\pgfsetbuttcap%
\pgfsetroundjoin%
\definecolor{currentfill}{rgb}{0.121569,0.466667,0.705882}%
\pgfsetfillcolor{currentfill}%
\pgfsetfillopacity{0.306036}%
\pgfsetlinewidth{1.003750pt}%
\definecolor{currentstroke}{rgb}{0.121569,0.466667,0.705882}%
\pgfsetstrokecolor{currentstroke}%
\pgfsetstrokeopacity{0.306036}%
\pgfsetdash{}{0pt}%
\pgfpathmoveto{\pgfqpoint{0.718568in}{1.555778in}}%
\pgfpathcurveto{\pgfqpoint{0.726804in}{1.555778in}}{\pgfqpoint{0.734704in}{1.559051in}}{\pgfqpoint{0.740528in}{1.564875in}}%
\pgfpathcurveto{\pgfqpoint{0.746352in}{1.570699in}}{\pgfqpoint{0.749624in}{1.578599in}}{\pgfqpoint{0.749624in}{1.586835in}}%
\pgfpathcurveto{\pgfqpoint{0.749624in}{1.595071in}}{\pgfqpoint{0.746352in}{1.602971in}}{\pgfqpoint{0.740528in}{1.608795in}}%
\pgfpathcurveto{\pgfqpoint{0.734704in}{1.614619in}}{\pgfqpoint{0.726804in}{1.617891in}}{\pgfqpoint{0.718568in}{1.617891in}}%
\pgfpathcurveto{\pgfqpoint{0.710332in}{1.617891in}}{\pgfqpoint{0.702432in}{1.614619in}}{\pgfqpoint{0.696608in}{1.608795in}}%
\pgfpathcurveto{\pgfqpoint{0.690784in}{1.602971in}}{\pgfqpoint{0.687511in}{1.595071in}}{\pgfqpoint{0.687511in}{1.586835in}}%
\pgfpathcurveto{\pgfqpoint{0.687511in}{1.578599in}}{\pgfqpoint{0.690784in}{1.570699in}}{\pgfqpoint{0.696608in}{1.564875in}}%
\pgfpathcurveto{\pgfqpoint{0.702432in}{1.559051in}}{\pgfqpoint{0.710332in}{1.555778in}}{\pgfqpoint{0.718568in}{1.555778in}}%
\pgfpathclose%
\pgfusepath{stroke,fill}%
\end{pgfscope}%
\begin{pgfscope}%
\pgfpathrectangle{\pgfqpoint{0.100000in}{0.220728in}}{\pgfqpoint{3.696000in}{3.696000in}}%
\pgfusepath{clip}%
\pgfsetbuttcap%
\pgfsetroundjoin%
\definecolor{currentfill}{rgb}{0.121569,0.466667,0.705882}%
\pgfsetfillcolor{currentfill}%
\pgfsetfillopacity{0.306045}%
\pgfsetlinewidth{1.003750pt}%
\definecolor{currentstroke}{rgb}{0.121569,0.466667,0.705882}%
\pgfsetstrokecolor{currentstroke}%
\pgfsetstrokeopacity{0.306045}%
\pgfsetdash{}{0pt}%
\pgfpathmoveto{\pgfqpoint{0.718573in}{1.555783in}}%
\pgfpathcurveto{\pgfqpoint{0.726809in}{1.555783in}}{\pgfqpoint{0.734709in}{1.559055in}}{\pgfqpoint{0.740533in}{1.564879in}}%
\pgfpathcurveto{\pgfqpoint{0.746357in}{1.570703in}}{\pgfqpoint{0.749630in}{1.578603in}}{\pgfqpoint{0.749630in}{1.586839in}}%
\pgfpathcurveto{\pgfqpoint{0.749630in}{1.595075in}}{\pgfqpoint{0.746357in}{1.602975in}}{\pgfqpoint{0.740533in}{1.608799in}}%
\pgfpathcurveto{\pgfqpoint{0.734709in}{1.614623in}}{\pgfqpoint{0.726809in}{1.617896in}}{\pgfqpoint{0.718573in}{1.617896in}}%
\pgfpathcurveto{\pgfqpoint{0.710337in}{1.617896in}}{\pgfqpoint{0.702437in}{1.614623in}}{\pgfqpoint{0.696613in}{1.608799in}}%
\pgfpathcurveto{\pgfqpoint{0.690789in}{1.602975in}}{\pgfqpoint{0.687517in}{1.595075in}}{\pgfqpoint{0.687517in}{1.586839in}}%
\pgfpathcurveto{\pgfqpoint{0.687517in}{1.578603in}}{\pgfqpoint{0.690789in}{1.570703in}}{\pgfqpoint{0.696613in}{1.564879in}}%
\pgfpathcurveto{\pgfqpoint{0.702437in}{1.559055in}}{\pgfqpoint{0.710337in}{1.555783in}}{\pgfqpoint{0.718573in}{1.555783in}}%
\pgfpathclose%
\pgfusepath{stroke,fill}%
\end{pgfscope}%
\begin{pgfscope}%
\pgfpathrectangle{\pgfqpoint{0.100000in}{0.220728in}}{\pgfqpoint{3.696000in}{3.696000in}}%
\pgfusepath{clip}%
\pgfsetbuttcap%
\pgfsetroundjoin%
\definecolor{currentfill}{rgb}{0.121569,0.466667,0.705882}%
\pgfsetfillcolor{currentfill}%
\pgfsetfillopacity{0.306062}%
\pgfsetlinewidth{1.003750pt}%
\definecolor{currentstroke}{rgb}{0.121569,0.466667,0.705882}%
\pgfsetstrokecolor{currentstroke}%
\pgfsetstrokeopacity{0.306062}%
\pgfsetdash{}{0pt}%
\pgfpathmoveto{\pgfqpoint{0.718583in}{1.555789in}}%
\pgfpathcurveto{\pgfqpoint{0.726819in}{1.555789in}}{\pgfqpoint{0.734719in}{1.559062in}}{\pgfqpoint{0.740543in}{1.564885in}}%
\pgfpathcurveto{\pgfqpoint{0.746367in}{1.570709in}}{\pgfqpoint{0.749639in}{1.578609in}}{\pgfqpoint{0.749639in}{1.586846in}}%
\pgfpathcurveto{\pgfqpoint{0.749639in}{1.595082in}}{\pgfqpoint{0.746367in}{1.602982in}}{\pgfqpoint{0.740543in}{1.608806in}}%
\pgfpathcurveto{\pgfqpoint{0.734719in}{1.614630in}}{\pgfqpoint{0.726819in}{1.617902in}}{\pgfqpoint{0.718583in}{1.617902in}}%
\pgfpathcurveto{\pgfqpoint{0.710346in}{1.617902in}}{\pgfqpoint{0.702446in}{1.614630in}}{\pgfqpoint{0.696622in}{1.608806in}}%
\pgfpathcurveto{\pgfqpoint{0.690798in}{1.602982in}}{\pgfqpoint{0.687526in}{1.595082in}}{\pgfqpoint{0.687526in}{1.586846in}}%
\pgfpathcurveto{\pgfqpoint{0.687526in}{1.578609in}}{\pgfqpoint{0.690798in}{1.570709in}}{\pgfqpoint{0.696622in}{1.564885in}}%
\pgfpathcurveto{\pgfqpoint{0.702446in}{1.559062in}}{\pgfqpoint{0.710346in}{1.555789in}}{\pgfqpoint{0.718583in}{1.555789in}}%
\pgfpathclose%
\pgfusepath{stroke,fill}%
\end{pgfscope}%
\begin{pgfscope}%
\pgfpathrectangle{\pgfqpoint{0.100000in}{0.220728in}}{\pgfqpoint{3.696000in}{3.696000in}}%
\pgfusepath{clip}%
\pgfsetbuttcap%
\pgfsetroundjoin%
\definecolor{currentfill}{rgb}{0.121569,0.466667,0.705882}%
\pgfsetfillcolor{currentfill}%
\pgfsetfillopacity{0.306092}%
\pgfsetlinewidth{1.003750pt}%
\definecolor{currentstroke}{rgb}{0.121569,0.466667,0.705882}%
\pgfsetstrokecolor{currentstroke}%
\pgfsetstrokeopacity{0.306092}%
\pgfsetdash{}{0pt}%
\pgfpathmoveto{\pgfqpoint{0.718600in}{1.555802in}}%
\pgfpathcurveto{\pgfqpoint{0.726836in}{1.555802in}}{\pgfqpoint{0.734736in}{1.559074in}}{\pgfqpoint{0.740560in}{1.564898in}}%
\pgfpathcurveto{\pgfqpoint{0.746384in}{1.570722in}}{\pgfqpoint{0.749656in}{1.578622in}}{\pgfqpoint{0.749656in}{1.586858in}}%
\pgfpathcurveto{\pgfqpoint{0.749656in}{1.595094in}}{\pgfqpoint{0.746384in}{1.602994in}}{\pgfqpoint{0.740560in}{1.608818in}}%
\pgfpathcurveto{\pgfqpoint{0.734736in}{1.614642in}}{\pgfqpoint{0.726836in}{1.617915in}}{\pgfqpoint{0.718600in}{1.617915in}}%
\pgfpathcurveto{\pgfqpoint{0.710363in}{1.617915in}}{\pgfqpoint{0.702463in}{1.614642in}}{\pgfqpoint{0.696639in}{1.608818in}}%
\pgfpathcurveto{\pgfqpoint{0.690816in}{1.602994in}}{\pgfqpoint{0.687543in}{1.595094in}}{\pgfqpoint{0.687543in}{1.586858in}}%
\pgfpathcurveto{\pgfqpoint{0.687543in}{1.578622in}}{\pgfqpoint{0.690816in}{1.570722in}}{\pgfqpoint{0.696639in}{1.564898in}}%
\pgfpathcurveto{\pgfqpoint{0.702463in}{1.559074in}}{\pgfqpoint{0.710363in}{1.555802in}}{\pgfqpoint{0.718600in}{1.555802in}}%
\pgfpathclose%
\pgfusepath{stroke,fill}%
\end{pgfscope}%
\begin{pgfscope}%
\pgfpathrectangle{\pgfqpoint{0.100000in}{0.220728in}}{\pgfqpoint{3.696000in}{3.696000in}}%
\pgfusepath{clip}%
\pgfsetbuttcap%
\pgfsetroundjoin%
\definecolor{currentfill}{rgb}{0.121569,0.466667,0.705882}%
\pgfsetfillcolor{currentfill}%
\pgfsetfillopacity{0.306161}%
\pgfsetlinewidth{1.003750pt}%
\definecolor{currentstroke}{rgb}{0.121569,0.466667,0.705882}%
\pgfsetstrokecolor{currentstroke}%
\pgfsetstrokeopacity{0.306161}%
\pgfsetdash{}{0pt}%
\pgfpathmoveto{\pgfqpoint{0.718634in}{1.555853in}}%
\pgfpathcurveto{\pgfqpoint{0.726870in}{1.555853in}}{\pgfqpoint{0.734771in}{1.559125in}}{\pgfqpoint{0.740594in}{1.564949in}}%
\pgfpathcurveto{\pgfqpoint{0.746418in}{1.570773in}}{\pgfqpoint{0.749691in}{1.578673in}}{\pgfqpoint{0.749691in}{1.586909in}}%
\pgfpathcurveto{\pgfqpoint{0.749691in}{1.595145in}}{\pgfqpoint{0.746418in}{1.603045in}}{\pgfqpoint{0.740594in}{1.608869in}}%
\pgfpathcurveto{\pgfqpoint{0.734771in}{1.614693in}}{\pgfqpoint{0.726870in}{1.617966in}}{\pgfqpoint{0.718634in}{1.617966in}}%
\pgfpathcurveto{\pgfqpoint{0.710398in}{1.617966in}}{\pgfqpoint{0.702498in}{1.614693in}}{\pgfqpoint{0.696674in}{1.608869in}}%
\pgfpathcurveto{\pgfqpoint{0.690850in}{1.603045in}}{\pgfqpoint{0.687578in}{1.595145in}}{\pgfqpoint{0.687578in}{1.586909in}}%
\pgfpathcurveto{\pgfqpoint{0.687578in}{1.578673in}}{\pgfqpoint{0.690850in}{1.570773in}}{\pgfqpoint{0.696674in}{1.564949in}}%
\pgfpathcurveto{\pgfqpoint{0.702498in}{1.559125in}}{\pgfqpoint{0.710398in}{1.555853in}}{\pgfqpoint{0.718634in}{1.555853in}}%
\pgfpathclose%
\pgfusepath{stroke,fill}%
\end{pgfscope}%
\begin{pgfscope}%
\pgfpathrectangle{\pgfqpoint{0.100000in}{0.220728in}}{\pgfqpoint{3.696000in}{3.696000in}}%
\pgfusepath{clip}%
\pgfsetbuttcap%
\pgfsetroundjoin%
\definecolor{currentfill}{rgb}{0.121569,0.466667,0.705882}%
\pgfsetfillcolor{currentfill}%
\pgfsetfillopacity{0.306268}%
\pgfsetlinewidth{1.003750pt}%
\definecolor{currentstroke}{rgb}{0.121569,0.466667,0.705882}%
\pgfsetstrokecolor{currentstroke}%
\pgfsetstrokeopacity{0.306268}%
\pgfsetdash{}{0pt}%
\pgfpathmoveto{\pgfqpoint{0.718693in}{1.555911in}}%
\pgfpathcurveto{\pgfqpoint{0.726930in}{1.555911in}}{\pgfqpoint{0.734830in}{1.559183in}}{\pgfqpoint{0.740654in}{1.565007in}}%
\pgfpathcurveto{\pgfqpoint{0.746478in}{1.570831in}}{\pgfqpoint{0.749750in}{1.578731in}}{\pgfqpoint{0.749750in}{1.586967in}}%
\pgfpathcurveto{\pgfqpoint{0.749750in}{1.595204in}}{\pgfqpoint{0.746478in}{1.603104in}}{\pgfqpoint{0.740654in}{1.608928in}}%
\pgfpathcurveto{\pgfqpoint{0.734830in}{1.614752in}}{\pgfqpoint{0.726930in}{1.618024in}}{\pgfqpoint{0.718693in}{1.618024in}}%
\pgfpathcurveto{\pgfqpoint{0.710457in}{1.618024in}}{\pgfqpoint{0.702557in}{1.614752in}}{\pgfqpoint{0.696733in}{1.608928in}}%
\pgfpathcurveto{\pgfqpoint{0.690909in}{1.603104in}}{\pgfqpoint{0.687637in}{1.595204in}}{\pgfqpoint{0.687637in}{1.586967in}}%
\pgfpathcurveto{\pgfqpoint{0.687637in}{1.578731in}}{\pgfqpoint{0.690909in}{1.570831in}}{\pgfqpoint{0.696733in}{1.565007in}}%
\pgfpathcurveto{\pgfqpoint{0.702557in}{1.559183in}}{\pgfqpoint{0.710457in}{1.555911in}}{\pgfqpoint{0.718693in}{1.555911in}}%
\pgfpathclose%
\pgfusepath{stroke,fill}%
\end{pgfscope}%
\begin{pgfscope}%
\pgfpathrectangle{\pgfqpoint{0.100000in}{0.220728in}}{\pgfqpoint{3.696000in}{3.696000in}}%
\pgfusepath{clip}%
\pgfsetbuttcap%
\pgfsetroundjoin%
\definecolor{currentfill}{rgb}{0.121569,0.466667,0.705882}%
\pgfsetfillcolor{currentfill}%
\pgfsetfillopacity{0.306487}%
\pgfsetlinewidth{1.003750pt}%
\definecolor{currentstroke}{rgb}{0.121569,0.466667,0.705882}%
\pgfsetstrokecolor{currentstroke}%
\pgfsetstrokeopacity{0.306487}%
\pgfsetdash{}{0pt}%
\pgfpathmoveto{\pgfqpoint{0.718805in}{1.556062in}}%
\pgfpathcurveto{\pgfqpoint{0.727042in}{1.556062in}}{\pgfqpoint{0.734942in}{1.559334in}}{\pgfqpoint{0.740766in}{1.565158in}}%
\pgfpathcurveto{\pgfqpoint{0.746590in}{1.570982in}}{\pgfqpoint{0.749862in}{1.578882in}}{\pgfqpoint{0.749862in}{1.587118in}}%
\pgfpathcurveto{\pgfqpoint{0.749862in}{1.595354in}}{\pgfqpoint{0.746590in}{1.603254in}}{\pgfqpoint{0.740766in}{1.609078in}}%
\pgfpathcurveto{\pgfqpoint{0.734942in}{1.614902in}}{\pgfqpoint{0.727042in}{1.618175in}}{\pgfqpoint{0.718805in}{1.618175in}}%
\pgfpathcurveto{\pgfqpoint{0.710569in}{1.618175in}}{\pgfqpoint{0.702669in}{1.614902in}}{\pgfqpoint{0.696845in}{1.609078in}}%
\pgfpathcurveto{\pgfqpoint{0.691021in}{1.603254in}}{\pgfqpoint{0.687749in}{1.595354in}}{\pgfqpoint{0.687749in}{1.587118in}}%
\pgfpathcurveto{\pgfqpoint{0.687749in}{1.578882in}}{\pgfqpoint{0.691021in}{1.570982in}}{\pgfqpoint{0.696845in}{1.565158in}}%
\pgfpathcurveto{\pgfqpoint{0.702669in}{1.559334in}}{\pgfqpoint{0.710569in}{1.556062in}}{\pgfqpoint{0.718805in}{1.556062in}}%
\pgfpathclose%
\pgfusepath{stroke,fill}%
\end{pgfscope}%
\begin{pgfscope}%
\pgfpathrectangle{\pgfqpoint{0.100000in}{0.220728in}}{\pgfqpoint{3.696000in}{3.696000in}}%
\pgfusepath{clip}%
\pgfsetbuttcap%
\pgfsetroundjoin%
\definecolor{currentfill}{rgb}{0.121569,0.466667,0.705882}%
\pgfsetfillcolor{currentfill}%
\pgfsetfillopacity{0.306829}%
\pgfsetlinewidth{1.003750pt}%
\definecolor{currentstroke}{rgb}{0.121569,0.466667,0.705882}%
\pgfsetstrokecolor{currentstroke}%
\pgfsetstrokeopacity{0.306829}%
\pgfsetdash{}{0pt}%
\pgfpathmoveto{\pgfqpoint{0.718999in}{1.556230in}}%
\pgfpathcurveto{\pgfqpoint{0.727236in}{1.556230in}}{\pgfqpoint{0.735136in}{1.559502in}}{\pgfqpoint{0.740960in}{1.565326in}}%
\pgfpathcurveto{\pgfqpoint{0.746783in}{1.571150in}}{\pgfqpoint{0.750056in}{1.579050in}}{\pgfqpoint{0.750056in}{1.587287in}}%
\pgfpathcurveto{\pgfqpoint{0.750056in}{1.595523in}}{\pgfqpoint{0.746783in}{1.603423in}}{\pgfqpoint{0.740960in}{1.609247in}}%
\pgfpathcurveto{\pgfqpoint{0.735136in}{1.615071in}}{\pgfqpoint{0.727236in}{1.618343in}}{\pgfqpoint{0.718999in}{1.618343in}}%
\pgfpathcurveto{\pgfqpoint{0.710763in}{1.618343in}}{\pgfqpoint{0.702863in}{1.615071in}}{\pgfqpoint{0.697039in}{1.609247in}}%
\pgfpathcurveto{\pgfqpoint{0.691215in}{1.603423in}}{\pgfqpoint{0.687943in}{1.595523in}}{\pgfqpoint{0.687943in}{1.587287in}}%
\pgfpathcurveto{\pgfqpoint{0.687943in}{1.579050in}}{\pgfqpoint{0.691215in}{1.571150in}}{\pgfqpoint{0.697039in}{1.565326in}}%
\pgfpathcurveto{\pgfqpoint{0.702863in}{1.559502in}}{\pgfqpoint{0.710763in}{1.556230in}}{\pgfqpoint{0.718999in}{1.556230in}}%
\pgfpathclose%
\pgfusepath{stroke,fill}%
\end{pgfscope}%
\begin{pgfscope}%
\pgfpathrectangle{\pgfqpoint{0.100000in}{0.220728in}}{\pgfqpoint{3.696000in}{3.696000in}}%
\pgfusepath{clip}%
\pgfsetbuttcap%
\pgfsetroundjoin%
\definecolor{currentfill}{rgb}{0.121569,0.466667,0.705882}%
\pgfsetfillcolor{currentfill}%
\pgfsetfillopacity{0.307382}%
\pgfsetlinewidth{1.003750pt}%
\definecolor{currentstroke}{rgb}{0.121569,0.466667,0.705882}%
\pgfsetstrokecolor{currentstroke}%
\pgfsetstrokeopacity{0.307382}%
\pgfsetdash{}{0pt}%
\pgfpathmoveto{\pgfqpoint{0.719304in}{1.556294in}}%
\pgfpathcurveto{\pgfqpoint{0.727541in}{1.556294in}}{\pgfqpoint{0.735441in}{1.559566in}}{\pgfqpoint{0.741265in}{1.565390in}}%
\pgfpathcurveto{\pgfqpoint{0.747089in}{1.571214in}}{\pgfqpoint{0.750361in}{1.579114in}}{\pgfqpoint{0.750361in}{1.587350in}}%
\pgfpathcurveto{\pgfqpoint{0.750361in}{1.595587in}}{\pgfqpoint{0.747089in}{1.603487in}}{\pgfqpoint{0.741265in}{1.609311in}}%
\pgfpathcurveto{\pgfqpoint{0.735441in}{1.615135in}}{\pgfqpoint{0.727541in}{1.618407in}}{\pgfqpoint{0.719304in}{1.618407in}}%
\pgfpathcurveto{\pgfqpoint{0.711068in}{1.618407in}}{\pgfqpoint{0.703168in}{1.615135in}}{\pgfqpoint{0.697344in}{1.609311in}}%
\pgfpathcurveto{\pgfqpoint{0.691520in}{1.603487in}}{\pgfqpoint{0.688248in}{1.595587in}}{\pgfqpoint{0.688248in}{1.587350in}}%
\pgfpathcurveto{\pgfqpoint{0.688248in}{1.579114in}}{\pgfqpoint{0.691520in}{1.571214in}}{\pgfqpoint{0.697344in}{1.565390in}}%
\pgfpathcurveto{\pgfqpoint{0.703168in}{1.559566in}}{\pgfqpoint{0.711068in}{1.556294in}}{\pgfqpoint{0.719304in}{1.556294in}}%
\pgfpathclose%
\pgfusepath{stroke,fill}%
\end{pgfscope}%
\begin{pgfscope}%
\pgfpathrectangle{\pgfqpoint{0.100000in}{0.220728in}}{\pgfqpoint{3.696000in}{3.696000in}}%
\pgfusepath{clip}%
\pgfsetbuttcap%
\pgfsetroundjoin%
\definecolor{currentfill}{rgb}{0.121569,0.466667,0.705882}%
\pgfsetfillcolor{currentfill}%
\pgfsetfillopacity{0.308619}%
\pgfsetlinewidth{1.003750pt}%
\definecolor{currentstroke}{rgb}{0.121569,0.466667,0.705882}%
\pgfsetstrokecolor{currentstroke}%
\pgfsetstrokeopacity{0.308619}%
\pgfsetdash{}{0pt}%
\pgfpathmoveto{\pgfqpoint{0.719973in}{1.557088in}}%
\pgfpathcurveto{\pgfqpoint{0.728209in}{1.557088in}}{\pgfqpoint{0.736109in}{1.560360in}}{\pgfqpoint{0.741933in}{1.566184in}}%
\pgfpathcurveto{\pgfqpoint{0.747757in}{1.572008in}}{\pgfqpoint{0.751030in}{1.579908in}}{\pgfqpoint{0.751030in}{1.588144in}}%
\pgfpathcurveto{\pgfqpoint{0.751030in}{1.596381in}}{\pgfqpoint{0.747757in}{1.604281in}}{\pgfqpoint{0.741933in}{1.610105in}}%
\pgfpathcurveto{\pgfqpoint{0.736109in}{1.615929in}}{\pgfqpoint{0.728209in}{1.619201in}}{\pgfqpoint{0.719973in}{1.619201in}}%
\pgfpathcurveto{\pgfqpoint{0.711737in}{1.619201in}}{\pgfqpoint{0.703837in}{1.615929in}}{\pgfqpoint{0.698013in}{1.610105in}}%
\pgfpathcurveto{\pgfqpoint{0.692189in}{1.604281in}}{\pgfqpoint{0.688917in}{1.596381in}}{\pgfqpoint{0.688917in}{1.588144in}}%
\pgfpathcurveto{\pgfqpoint{0.688917in}{1.579908in}}{\pgfqpoint{0.692189in}{1.572008in}}{\pgfqpoint{0.698013in}{1.566184in}}%
\pgfpathcurveto{\pgfqpoint{0.703837in}{1.560360in}}{\pgfqpoint{0.711737in}{1.557088in}}{\pgfqpoint{0.719973in}{1.557088in}}%
\pgfpathclose%
\pgfusepath{stroke,fill}%
\end{pgfscope}%
\begin{pgfscope}%
\pgfpathrectangle{\pgfqpoint{0.100000in}{0.220728in}}{\pgfqpoint{3.696000in}{3.696000in}}%
\pgfusepath{clip}%
\pgfsetbuttcap%
\pgfsetroundjoin%
\definecolor{currentfill}{rgb}{0.121569,0.466667,0.705882}%
\pgfsetfillcolor{currentfill}%
\pgfsetfillopacity{0.310452}%
\pgfsetlinewidth{1.003750pt}%
\definecolor{currentstroke}{rgb}{0.121569,0.466667,0.705882}%
\pgfsetstrokecolor{currentstroke}%
\pgfsetstrokeopacity{0.310452}%
\pgfsetdash{}{0pt}%
\pgfpathmoveto{\pgfqpoint{0.721073in}{1.557599in}}%
\pgfpathcurveto{\pgfqpoint{0.729309in}{1.557599in}}{\pgfqpoint{0.737209in}{1.560872in}}{\pgfqpoint{0.743033in}{1.566696in}}%
\pgfpathcurveto{\pgfqpoint{0.748857in}{1.572519in}}{\pgfqpoint{0.752129in}{1.580419in}}{\pgfqpoint{0.752129in}{1.588656in}}%
\pgfpathcurveto{\pgfqpoint{0.752129in}{1.596892in}}{\pgfqpoint{0.748857in}{1.604792in}}{\pgfqpoint{0.743033in}{1.610616in}}%
\pgfpathcurveto{\pgfqpoint{0.737209in}{1.616440in}}{\pgfqpoint{0.729309in}{1.619712in}}{\pgfqpoint{0.721073in}{1.619712in}}%
\pgfpathcurveto{\pgfqpoint{0.712836in}{1.619712in}}{\pgfqpoint{0.704936in}{1.616440in}}{\pgfqpoint{0.699112in}{1.610616in}}%
\pgfpathcurveto{\pgfqpoint{0.693288in}{1.604792in}}{\pgfqpoint{0.690016in}{1.596892in}}{\pgfqpoint{0.690016in}{1.588656in}}%
\pgfpathcurveto{\pgfqpoint{0.690016in}{1.580419in}}{\pgfqpoint{0.693288in}{1.572519in}}{\pgfqpoint{0.699112in}{1.566696in}}%
\pgfpathcurveto{\pgfqpoint{0.704936in}{1.560872in}}{\pgfqpoint{0.712836in}{1.557599in}}{\pgfqpoint{0.721073in}{1.557599in}}%
\pgfpathclose%
\pgfusepath{stroke,fill}%
\end{pgfscope}%
\begin{pgfscope}%
\pgfpathrectangle{\pgfqpoint{0.100000in}{0.220728in}}{\pgfqpoint{3.696000in}{3.696000in}}%
\pgfusepath{clip}%
\pgfsetbuttcap%
\pgfsetroundjoin%
\definecolor{currentfill}{rgb}{0.121569,0.466667,0.705882}%
\pgfsetfillcolor{currentfill}%
\pgfsetfillopacity{0.314832}%
\pgfsetlinewidth{1.003750pt}%
\definecolor{currentstroke}{rgb}{0.121569,0.466667,0.705882}%
\pgfsetstrokecolor{currentstroke}%
\pgfsetstrokeopacity{0.314832}%
\pgfsetdash{}{0pt}%
\pgfpathmoveto{\pgfqpoint{0.723353in}{1.560862in}}%
\pgfpathcurveto{\pgfqpoint{0.731589in}{1.560862in}}{\pgfqpoint{0.739489in}{1.564134in}}{\pgfqpoint{0.745313in}{1.569958in}}%
\pgfpathcurveto{\pgfqpoint{0.751137in}{1.575782in}}{\pgfqpoint{0.754409in}{1.583682in}}{\pgfqpoint{0.754409in}{1.591919in}}%
\pgfpathcurveto{\pgfqpoint{0.754409in}{1.600155in}}{\pgfqpoint{0.751137in}{1.608055in}}{\pgfqpoint{0.745313in}{1.613879in}}%
\pgfpathcurveto{\pgfqpoint{0.739489in}{1.619703in}}{\pgfqpoint{0.731589in}{1.622975in}}{\pgfqpoint{0.723353in}{1.622975in}}%
\pgfpathcurveto{\pgfqpoint{0.715117in}{1.622975in}}{\pgfqpoint{0.707217in}{1.619703in}}{\pgfqpoint{0.701393in}{1.613879in}}%
\pgfpathcurveto{\pgfqpoint{0.695569in}{1.608055in}}{\pgfqpoint{0.692296in}{1.600155in}}{\pgfqpoint{0.692296in}{1.591919in}}%
\pgfpathcurveto{\pgfqpoint{0.692296in}{1.583682in}}{\pgfqpoint{0.695569in}{1.575782in}}{\pgfqpoint{0.701393in}{1.569958in}}%
\pgfpathcurveto{\pgfqpoint{0.707217in}{1.564134in}}{\pgfqpoint{0.715117in}{1.560862in}}{\pgfqpoint{0.723353in}{1.560862in}}%
\pgfpathclose%
\pgfusepath{stroke,fill}%
\end{pgfscope}%
\begin{pgfscope}%
\pgfpathrectangle{\pgfqpoint{0.100000in}{0.220728in}}{\pgfqpoint{3.696000in}{3.696000in}}%
\pgfusepath{clip}%
\pgfsetbuttcap%
\pgfsetroundjoin%
\definecolor{currentfill}{rgb}{0.121569,0.466667,0.705882}%
\pgfsetfillcolor{currentfill}%
\pgfsetfillopacity{0.321259}%
\pgfsetlinewidth{1.003750pt}%
\definecolor{currentstroke}{rgb}{0.121569,0.466667,0.705882}%
\pgfsetstrokecolor{currentstroke}%
\pgfsetstrokeopacity{0.321259}%
\pgfsetdash{}{0pt}%
\pgfpathmoveto{\pgfqpoint{0.727097in}{1.563407in}}%
\pgfpathcurveto{\pgfqpoint{0.735333in}{1.563407in}}{\pgfqpoint{0.743234in}{1.566679in}}{\pgfqpoint{0.749057in}{1.572503in}}%
\pgfpathcurveto{\pgfqpoint{0.754881in}{1.578327in}}{\pgfqpoint{0.758154in}{1.586227in}}{\pgfqpoint{0.758154in}{1.594463in}}%
\pgfpathcurveto{\pgfqpoint{0.758154in}{1.602699in}}{\pgfqpoint{0.754881in}{1.610599in}}{\pgfqpoint{0.749057in}{1.616423in}}%
\pgfpathcurveto{\pgfqpoint{0.743234in}{1.622247in}}{\pgfqpoint{0.735333in}{1.625520in}}{\pgfqpoint{0.727097in}{1.625520in}}%
\pgfpathcurveto{\pgfqpoint{0.718861in}{1.625520in}}{\pgfqpoint{0.710961in}{1.622247in}}{\pgfqpoint{0.705137in}{1.616423in}}%
\pgfpathcurveto{\pgfqpoint{0.699313in}{1.610599in}}{\pgfqpoint{0.696041in}{1.602699in}}{\pgfqpoint{0.696041in}{1.594463in}}%
\pgfpathcurveto{\pgfqpoint{0.696041in}{1.586227in}}{\pgfqpoint{0.699313in}{1.578327in}}{\pgfqpoint{0.705137in}{1.572503in}}%
\pgfpathcurveto{\pgfqpoint{0.710961in}{1.566679in}}{\pgfqpoint{0.718861in}{1.563407in}}{\pgfqpoint{0.727097in}{1.563407in}}%
\pgfpathclose%
\pgfusepath{stroke,fill}%
\end{pgfscope}%
\begin{pgfscope}%
\pgfpathrectangle{\pgfqpoint{0.100000in}{0.220728in}}{\pgfqpoint{3.696000in}{3.696000in}}%
\pgfusepath{clip}%
\pgfsetbuttcap%
\pgfsetroundjoin%
\definecolor{currentfill}{rgb}{0.121569,0.466667,0.705882}%
\pgfsetfillcolor{currentfill}%
\pgfsetfillopacity{0.335106}%
\pgfsetlinewidth{1.003750pt}%
\definecolor{currentstroke}{rgb}{0.121569,0.466667,0.705882}%
\pgfsetstrokecolor{currentstroke}%
\pgfsetstrokeopacity{0.335106}%
\pgfsetdash{}{0pt}%
\pgfpathmoveto{\pgfqpoint{0.734390in}{1.572345in}}%
\pgfpathcurveto{\pgfqpoint{0.742626in}{1.572345in}}{\pgfqpoint{0.750526in}{1.575617in}}{\pgfqpoint{0.756350in}{1.581441in}}%
\pgfpathcurveto{\pgfqpoint{0.762174in}{1.587265in}}{\pgfqpoint{0.765446in}{1.595165in}}{\pgfqpoint{0.765446in}{1.603401in}}%
\pgfpathcurveto{\pgfqpoint{0.765446in}{1.611637in}}{\pgfqpoint{0.762174in}{1.619537in}}{\pgfqpoint{0.756350in}{1.625361in}}%
\pgfpathcurveto{\pgfqpoint{0.750526in}{1.631185in}}{\pgfqpoint{0.742626in}{1.634458in}}{\pgfqpoint{0.734390in}{1.634458in}}%
\pgfpathcurveto{\pgfqpoint{0.726154in}{1.634458in}}{\pgfqpoint{0.718254in}{1.631185in}}{\pgfqpoint{0.712430in}{1.625361in}}%
\pgfpathcurveto{\pgfqpoint{0.706606in}{1.619537in}}{\pgfqpoint{0.703333in}{1.611637in}}{\pgfqpoint{0.703333in}{1.603401in}}%
\pgfpathcurveto{\pgfqpoint{0.703333in}{1.595165in}}{\pgfqpoint{0.706606in}{1.587265in}}{\pgfqpoint{0.712430in}{1.581441in}}%
\pgfpathcurveto{\pgfqpoint{0.718254in}{1.575617in}}{\pgfqpoint{0.726154in}{1.572345in}}{\pgfqpoint{0.734390in}{1.572345in}}%
\pgfpathclose%
\pgfusepath{stroke,fill}%
\end{pgfscope}%
\begin{pgfscope}%
\pgfpathrectangle{\pgfqpoint{0.100000in}{0.220728in}}{\pgfqpoint{3.696000in}{3.696000in}}%
\pgfusepath{clip}%
\pgfsetbuttcap%
\pgfsetroundjoin%
\definecolor{currentfill}{rgb}{0.121569,0.466667,0.705882}%
\pgfsetfillcolor{currentfill}%
\pgfsetfillopacity{0.353205}%
\pgfsetlinewidth{1.003750pt}%
\definecolor{currentstroke}{rgb}{0.121569,0.466667,0.705882}%
\pgfsetstrokecolor{currentstroke}%
\pgfsetstrokeopacity{0.353205}%
\pgfsetdash{}{0pt}%
\pgfpathmoveto{\pgfqpoint{0.746019in}{1.574135in}}%
\pgfpathcurveto{\pgfqpoint{0.754255in}{1.574135in}}{\pgfqpoint{0.762155in}{1.577407in}}{\pgfqpoint{0.767979in}{1.583231in}}%
\pgfpathcurveto{\pgfqpoint{0.773803in}{1.589055in}}{\pgfqpoint{0.777075in}{1.596955in}}{\pgfqpoint{0.777075in}{1.605191in}}%
\pgfpathcurveto{\pgfqpoint{0.777075in}{1.613428in}}{\pgfqpoint{0.773803in}{1.621328in}}{\pgfqpoint{0.767979in}{1.627152in}}%
\pgfpathcurveto{\pgfqpoint{0.762155in}{1.632975in}}{\pgfqpoint{0.754255in}{1.636248in}}{\pgfqpoint{0.746019in}{1.636248in}}%
\pgfpathcurveto{\pgfqpoint{0.737783in}{1.636248in}}{\pgfqpoint{0.729883in}{1.632975in}}{\pgfqpoint{0.724059in}{1.627152in}}%
\pgfpathcurveto{\pgfqpoint{0.718235in}{1.621328in}}{\pgfqpoint{0.714962in}{1.613428in}}{\pgfqpoint{0.714962in}{1.605191in}}%
\pgfpathcurveto{\pgfqpoint{0.714962in}{1.596955in}}{\pgfqpoint{0.718235in}{1.589055in}}{\pgfqpoint{0.724059in}{1.583231in}}%
\pgfpathcurveto{\pgfqpoint{0.729883in}{1.577407in}}{\pgfqpoint{0.737783in}{1.574135in}}{\pgfqpoint{0.746019in}{1.574135in}}%
\pgfpathclose%
\pgfusepath{stroke,fill}%
\end{pgfscope}%
\begin{pgfscope}%
\pgfpathrectangle{\pgfqpoint{0.100000in}{0.220728in}}{\pgfqpoint{3.696000in}{3.696000in}}%
\pgfusepath{clip}%
\pgfsetbuttcap%
\pgfsetroundjoin%
\definecolor{currentfill}{rgb}{0.121569,0.466667,0.705882}%
\pgfsetfillcolor{currentfill}%
\pgfsetfillopacity{0.359168}%
\pgfsetlinewidth{1.003750pt}%
\definecolor{currentstroke}{rgb}{0.121569,0.466667,0.705882}%
\pgfsetstrokecolor{currentstroke}%
\pgfsetstrokeopacity{0.359168}%
\pgfsetdash{}{0pt}%
\pgfpathmoveto{\pgfqpoint{0.707695in}{1.642558in}}%
\pgfpathcurveto{\pgfqpoint{0.715931in}{1.642558in}}{\pgfqpoint{0.723831in}{1.645830in}}{\pgfqpoint{0.729655in}{1.651654in}}%
\pgfpathcurveto{\pgfqpoint{0.735479in}{1.657478in}}{\pgfqpoint{0.738751in}{1.665378in}}{\pgfqpoint{0.738751in}{1.673614in}}%
\pgfpathcurveto{\pgfqpoint{0.738751in}{1.681850in}}{\pgfqpoint{0.735479in}{1.689750in}}{\pgfqpoint{0.729655in}{1.695574in}}%
\pgfpathcurveto{\pgfqpoint{0.723831in}{1.701398in}}{\pgfqpoint{0.715931in}{1.704671in}}{\pgfqpoint{0.707695in}{1.704671in}}%
\pgfpathcurveto{\pgfqpoint{0.699458in}{1.704671in}}{\pgfqpoint{0.691558in}{1.701398in}}{\pgfqpoint{0.685734in}{1.695574in}}%
\pgfpathcurveto{\pgfqpoint{0.679910in}{1.689750in}}{\pgfqpoint{0.676638in}{1.681850in}}{\pgfqpoint{0.676638in}{1.673614in}}%
\pgfpathcurveto{\pgfqpoint{0.676638in}{1.665378in}}{\pgfqpoint{0.679910in}{1.657478in}}{\pgfqpoint{0.685734in}{1.651654in}}%
\pgfpathcurveto{\pgfqpoint{0.691558in}{1.645830in}}{\pgfqpoint{0.699458in}{1.642558in}}{\pgfqpoint{0.707695in}{1.642558in}}%
\pgfpathclose%
\pgfusepath{stroke,fill}%
\end{pgfscope}%
\begin{pgfscope}%
\pgfpathrectangle{\pgfqpoint{0.100000in}{0.220728in}}{\pgfqpoint{3.696000in}{3.696000in}}%
\pgfusepath{clip}%
\pgfsetbuttcap%
\pgfsetroundjoin%
\definecolor{currentfill}{rgb}{0.121569,0.466667,0.705882}%
\pgfsetfillcolor{currentfill}%
\pgfsetfillopacity{0.363889}%
\pgfsetlinewidth{1.003750pt}%
\definecolor{currentstroke}{rgb}{0.121569,0.466667,0.705882}%
\pgfsetstrokecolor{currentstroke}%
\pgfsetstrokeopacity{0.363889}%
\pgfsetdash{}{0pt}%
\pgfpathmoveto{\pgfqpoint{0.666060in}{1.618904in}}%
\pgfpathcurveto{\pgfqpoint{0.674296in}{1.618904in}}{\pgfqpoint{0.682196in}{1.622177in}}{\pgfqpoint{0.688020in}{1.628001in}}%
\pgfpathcurveto{\pgfqpoint{0.693844in}{1.633825in}}{\pgfqpoint{0.697117in}{1.641725in}}{\pgfqpoint{0.697117in}{1.649961in}}%
\pgfpathcurveto{\pgfqpoint{0.697117in}{1.658197in}}{\pgfqpoint{0.693844in}{1.666097in}}{\pgfqpoint{0.688020in}{1.671921in}}%
\pgfpathcurveto{\pgfqpoint{0.682196in}{1.677745in}}{\pgfqpoint{0.674296in}{1.681017in}}{\pgfqpoint{0.666060in}{1.681017in}}%
\pgfpathcurveto{\pgfqpoint{0.657824in}{1.681017in}}{\pgfqpoint{0.649924in}{1.677745in}}{\pgfqpoint{0.644100in}{1.671921in}}%
\pgfpathcurveto{\pgfqpoint{0.638276in}{1.666097in}}{\pgfqpoint{0.635004in}{1.658197in}}{\pgfqpoint{0.635004in}{1.649961in}}%
\pgfpathcurveto{\pgfqpoint{0.635004in}{1.641725in}}{\pgfqpoint{0.638276in}{1.633825in}}{\pgfqpoint{0.644100in}{1.628001in}}%
\pgfpathcurveto{\pgfqpoint{0.649924in}{1.622177in}}{\pgfqpoint{0.657824in}{1.618904in}}{\pgfqpoint{0.666060in}{1.618904in}}%
\pgfpathclose%
\pgfusepath{stroke,fill}%
\end{pgfscope}%
\begin{pgfscope}%
\pgfpathrectangle{\pgfqpoint{0.100000in}{0.220728in}}{\pgfqpoint{3.696000in}{3.696000in}}%
\pgfusepath{clip}%
\pgfsetbuttcap%
\pgfsetroundjoin%
\definecolor{currentfill}{rgb}{0.121569,0.466667,0.705882}%
\pgfsetfillcolor{currentfill}%
\pgfsetfillopacity{0.390718}%
\pgfsetlinewidth{1.003750pt}%
\definecolor{currentstroke}{rgb}{0.121569,0.466667,0.705882}%
\pgfsetstrokecolor{currentstroke}%
\pgfsetstrokeopacity{0.390718}%
\pgfsetdash{}{0pt}%
\pgfpathmoveto{\pgfqpoint{0.768747in}{1.588275in}}%
\pgfpathcurveto{\pgfqpoint{0.776984in}{1.588275in}}{\pgfqpoint{0.784884in}{1.591547in}}{\pgfqpoint{0.790708in}{1.597371in}}%
\pgfpathcurveto{\pgfqpoint{0.796532in}{1.603195in}}{\pgfqpoint{0.799804in}{1.611095in}}{\pgfqpoint{0.799804in}{1.619331in}}%
\pgfpathcurveto{\pgfqpoint{0.799804in}{1.627567in}}{\pgfqpoint{0.796532in}{1.635467in}}{\pgfqpoint{0.790708in}{1.641291in}}%
\pgfpathcurveto{\pgfqpoint{0.784884in}{1.647115in}}{\pgfqpoint{0.776984in}{1.650388in}}{\pgfqpoint{0.768747in}{1.650388in}}%
\pgfpathcurveto{\pgfqpoint{0.760511in}{1.650388in}}{\pgfqpoint{0.752611in}{1.647115in}}{\pgfqpoint{0.746787in}{1.641291in}}%
\pgfpathcurveto{\pgfqpoint{0.740963in}{1.635467in}}{\pgfqpoint{0.737691in}{1.627567in}}{\pgfqpoint{0.737691in}{1.619331in}}%
\pgfpathcurveto{\pgfqpoint{0.737691in}{1.611095in}}{\pgfqpoint{0.740963in}{1.603195in}}{\pgfqpoint{0.746787in}{1.597371in}}%
\pgfpathcurveto{\pgfqpoint{0.752611in}{1.591547in}}{\pgfqpoint{0.760511in}{1.588275in}}{\pgfqpoint{0.768747in}{1.588275in}}%
\pgfpathclose%
\pgfusepath{stroke,fill}%
\end{pgfscope}%
\begin{pgfscope}%
\pgfpathrectangle{\pgfqpoint{0.100000in}{0.220728in}}{\pgfqpoint{3.696000in}{3.696000in}}%
\pgfusepath{clip}%
\pgfsetbuttcap%
\pgfsetroundjoin%
\definecolor{currentfill}{rgb}{0.121569,0.466667,0.705882}%
\pgfsetfillcolor{currentfill}%
\pgfsetfillopacity{0.423955}%
\pgfsetlinewidth{1.003750pt}%
\definecolor{currentstroke}{rgb}{0.121569,0.466667,0.705882}%
\pgfsetstrokecolor{currentstroke}%
\pgfsetstrokeopacity{0.423955}%
\pgfsetdash{}{0pt}%
\pgfpathmoveto{\pgfqpoint{0.788542in}{1.594566in}}%
\pgfpathcurveto{\pgfqpoint{0.796778in}{1.594566in}}{\pgfqpoint{0.804678in}{1.597838in}}{\pgfqpoint{0.810502in}{1.603662in}}%
\pgfpathcurveto{\pgfqpoint{0.816326in}{1.609486in}}{\pgfqpoint{0.819598in}{1.617386in}}{\pgfqpoint{0.819598in}{1.625623in}}%
\pgfpathcurveto{\pgfqpoint{0.819598in}{1.633859in}}{\pgfqpoint{0.816326in}{1.641759in}}{\pgfqpoint{0.810502in}{1.647583in}}%
\pgfpathcurveto{\pgfqpoint{0.804678in}{1.653407in}}{\pgfqpoint{0.796778in}{1.656679in}}{\pgfqpoint{0.788542in}{1.656679in}}%
\pgfpathcurveto{\pgfqpoint{0.780306in}{1.656679in}}{\pgfqpoint{0.772406in}{1.653407in}}{\pgfqpoint{0.766582in}{1.647583in}}%
\pgfpathcurveto{\pgfqpoint{0.760758in}{1.641759in}}{\pgfqpoint{0.757485in}{1.633859in}}{\pgfqpoint{0.757485in}{1.625623in}}%
\pgfpathcurveto{\pgfqpoint{0.757485in}{1.617386in}}{\pgfqpoint{0.760758in}{1.609486in}}{\pgfqpoint{0.766582in}{1.603662in}}%
\pgfpathcurveto{\pgfqpoint{0.772406in}{1.597838in}}{\pgfqpoint{0.780306in}{1.594566in}}{\pgfqpoint{0.788542in}{1.594566in}}%
\pgfpathclose%
\pgfusepath{stroke,fill}%
\end{pgfscope}%
\begin{pgfscope}%
\pgfpathrectangle{\pgfqpoint{0.100000in}{0.220728in}}{\pgfqpoint{3.696000in}{3.696000in}}%
\pgfusepath{clip}%
\pgfsetbuttcap%
\pgfsetroundjoin%
\definecolor{currentfill}{rgb}{0.121569,0.466667,0.705882}%
\pgfsetfillcolor{currentfill}%
\pgfsetfillopacity{0.454544}%
\pgfsetlinewidth{1.003750pt}%
\definecolor{currentstroke}{rgb}{0.121569,0.466667,0.705882}%
\pgfsetstrokecolor{currentstroke}%
\pgfsetstrokeopacity{0.454544}%
\pgfsetdash{}{0pt}%
\pgfpathmoveto{\pgfqpoint{0.808058in}{1.603127in}}%
\pgfpathcurveto{\pgfqpoint{0.816295in}{1.603127in}}{\pgfqpoint{0.824195in}{1.606399in}}{\pgfqpoint{0.830019in}{1.612223in}}%
\pgfpathcurveto{\pgfqpoint{0.835843in}{1.618047in}}{\pgfqpoint{0.839115in}{1.625947in}}{\pgfqpoint{0.839115in}{1.634183in}}%
\pgfpathcurveto{\pgfqpoint{0.839115in}{1.642420in}}{\pgfqpoint{0.835843in}{1.650320in}}{\pgfqpoint{0.830019in}{1.656144in}}%
\pgfpathcurveto{\pgfqpoint{0.824195in}{1.661968in}}{\pgfqpoint{0.816295in}{1.665240in}}{\pgfqpoint{0.808058in}{1.665240in}}%
\pgfpathcurveto{\pgfqpoint{0.799822in}{1.665240in}}{\pgfqpoint{0.791922in}{1.661968in}}{\pgfqpoint{0.786098in}{1.656144in}}%
\pgfpathcurveto{\pgfqpoint{0.780274in}{1.650320in}}{\pgfqpoint{0.777002in}{1.642420in}}{\pgfqpoint{0.777002in}{1.634183in}}%
\pgfpathcurveto{\pgfqpoint{0.777002in}{1.625947in}}{\pgfqpoint{0.780274in}{1.618047in}}{\pgfqpoint{0.786098in}{1.612223in}}%
\pgfpathcurveto{\pgfqpoint{0.791922in}{1.606399in}}{\pgfqpoint{0.799822in}{1.603127in}}{\pgfqpoint{0.808058in}{1.603127in}}%
\pgfpathclose%
\pgfusepath{stroke,fill}%
\end{pgfscope}%
\begin{pgfscope}%
\pgfpathrectangle{\pgfqpoint{0.100000in}{0.220728in}}{\pgfqpoint{3.696000in}{3.696000in}}%
\pgfusepath{clip}%
\pgfsetbuttcap%
\pgfsetroundjoin%
\definecolor{currentfill}{rgb}{0.121569,0.466667,0.705882}%
\pgfsetfillcolor{currentfill}%
\pgfsetfillopacity{0.464373}%
\pgfsetlinewidth{1.003750pt}%
\definecolor{currentstroke}{rgb}{0.121569,0.466667,0.705882}%
\pgfsetstrokecolor{currentstroke}%
\pgfsetstrokeopacity{0.464373}%
\pgfsetdash{}{0pt}%
\pgfpathmoveto{\pgfqpoint{0.676828in}{1.739559in}}%
\pgfpathcurveto{\pgfqpoint{0.685065in}{1.739559in}}{\pgfqpoint{0.692965in}{1.742831in}}{\pgfqpoint{0.698788in}{1.748655in}}%
\pgfpathcurveto{\pgfqpoint{0.704612in}{1.754479in}}{\pgfqpoint{0.707885in}{1.762379in}}{\pgfqpoint{0.707885in}{1.770615in}}%
\pgfpathcurveto{\pgfqpoint{0.707885in}{1.778851in}}{\pgfqpoint{0.704612in}{1.786751in}}{\pgfqpoint{0.698788in}{1.792575in}}%
\pgfpathcurveto{\pgfqpoint{0.692965in}{1.798399in}}{\pgfqpoint{0.685065in}{1.801672in}}{\pgfqpoint{0.676828in}{1.801672in}}%
\pgfpathcurveto{\pgfqpoint{0.668592in}{1.801672in}}{\pgfqpoint{0.660692in}{1.798399in}}{\pgfqpoint{0.654868in}{1.792575in}}%
\pgfpathcurveto{\pgfqpoint{0.649044in}{1.786751in}}{\pgfqpoint{0.645772in}{1.778851in}}{\pgfqpoint{0.645772in}{1.770615in}}%
\pgfpathcurveto{\pgfqpoint{0.645772in}{1.762379in}}{\pgfqpoint{0.649044in}{1.754479in}}{\pgfqpoint{0.654868in}{1.748655in}}%
\pgfpathcurveto{\pgfqpoint{0.660692in}{1.742831in}}{\pgfqpoint{0.668592in}{1.739559in}}{\pgfqpoint{0.676828in}{1.739559in}}%
\pgfpathclose%
\pgfusepath{stroke,fill}%
\end{pgfscope}%
\begin{pgfscope}%
\pgfpathrectangle{\pgfqpoint{0.100000in}{0.220728in}}{\pgfqpoint{3.696000in}{3.696000in}}%
\pgfusepath{clip}%
\pgfsetbuttcap%
\pgfsetroundjoin%
\definecolor{currentfill}{rgb}{0.121569,0.466667,0.705882}%
\pgfsetfillcolor{currentfill}%
\pgfsetfillopacity{0.477908}%
\pgfsetlinewidth{1.003750pt}%
\definecolor{currentstroke}{rgb}{0.121569,0.466667,0.705882}%
\pgfsetstrokecolor{currentstroke}%
\pgfsetstrokeopacity{0.477908}%
\pgfsetdash{}{0pt}%
\pgfpathmoveto{\pgfqpoint{0.823535in}{1.595005in}}%
\pgfpathcurveto{\pgfqpoint{0.831771in}{1.595005in}}{\pgfqpoint{0.839671in}{1.598277in}}{\pgfqpoint{0.845495in}{1.604101in}}%
\pgfpathcurveto{\pgfqpoint{0.851319in}{1.609925in}}{\pgfqpoint{0.854592in}{1.617825in}}{\pgfqpoint{0.854592in}{1.626061in}}%
\pgfpathcurveto{\pgfqpoint{0.854592in}{1.634298in}}{\pgfqpoint{0.851319in}{1.642198in}}{\pgfqpoint{0.845495in}{1.648021in}}%
\pgfpathcurveto{\pgfqpoint{0.839671in}{1.653845in}}{\pgfqpoint{0.831771in}{1.657118in}}{\pgfqpoint{0.823535in}{1.657118in}}%
\pgfpathcurveto{\pgfqpoint{0.815299in}{1.657118in}}{\pgfqpoint{0.807399in}{1.653845in}}{\pgfqpoint{0.801575in}{1.648021in}}%
\pgfpathcurveto{\pgfqpoint{0.795751in}{1.642198in}}{\pgfqpoint{0.792479in}{1.634298in}}{\pgfqpoint{0.792479in}{1.626061in}}%
\pgfpathcurveto{\pgfqpoint{0.792479in}{1.617825in}}{\pgfqpoint{0.795751in}{1.609925in}}{\pgfqpoint{0.801575in}{1.604101in}}%
\pgfpathcurveto{\pgfqpoint{0.807399in}{1.598277in}}{\pgfqpoint{0.815299in}{1.595005in}}{\pgfqpoint{0.823535in}{1.595005in}}%
\pgfpathclose%
\pgfusepath{stroke,fill}%
\end{pgfscope}%
\begin{pgfscope}%
\pgfpathrectangle{\pgfqpoint{0.100000in}{0.220728in}}{\pgfqpoint{3.696000in}{3.696000in}}%
\pgfusepath{clip}%
\pgfsetbuttcap%
\pgfsetroundjoin%
\definecolor{currentfill}{rgb}{0.121569,0.466667,0.705882}%
\pgfsetfillcolor{currentfill}%
\pgfsetfillopacity{0.494601}%
\pgfsetlinewidth{1.003750pt}%
\definecolor{currentstroke}{rgb}{0.121569,0.466667,0.705882}%
\pgfsetstrokecolor{currentstroke}%
\pgfsetstrokeopacity{0.494601}%
\pgfsetdash{}{0pt}%
\pgfpathmoveto{\pgfqpoint{0.659492in}{1.824370in}}%
\pgfpathcurveto{\pgfqpoint{0.667729in}{1.824370in}}{\pgfqpoint{0.675629in}{1.827643in}}{\pgfqpoint{0.681453in}{1.833467in}}%
\pgfpathcurveto{\pgfqpoint{0.687277in}{1.839291in}}{\pgfqpoint{0.690549in}{1.847191in}}{\pgfqpoint{0.690549in}{1.855427in}}%
\pgfpathcurveto{\pgfqpoint{0.690549in}{1.863663in}}{\pgfqpoint{0.687277in}{1.871563in}}{\pgfqpoint{0.681453in}{1.877387in}}%
\pgfpathcurveto{\pgfqpoint{0.675629in}{1.883211in}}{\pgfqpoint{0.667729in}{1.886483in}}{\pgfqpoint{0.659492in}{1.886483in}}%
\pgfpathcurveto{\pgfqpoint{0.651256in}{1.886483in}}{\pgfqpoint{0.643356in}{1.883211in}}{\pgfqpoint{0.637532in}{1.877387in}}%
\pgfpathcurveto{\pgfqpoint{0.631708in}{1.871563in}}{\pgfqpoint{0.628436in}{1.863663in}}{\pgfqpoint{0.628436in}{1.855427in}}%
\pgfpathcurveto{\pgfqpoint{0.628436in}{1.847191in}}{\pgfqpoint{0.631708in}{1.839291in}}{\pgfqpoint{0.637532in}{1.833467in}}%
\pgfpathcurveto{\pgfqpoint{0.643356in}{1.827643in}}{\pgfqpoint{0.651256in}{1.824370in}}{\pgfqpoint{0.659492in}{1.824370in}}%
\pgfpathclose%
\pgfusepath{stroke,fill}%
\end{pgfscope}%
\begin{pgfscope}%
\pgfpathrectangle{\pgfqpoint{0.100000in}{0.220728in}}{\pgfqpoint{3.696000in}{3.696000in}}%
\pgfusepath{clip}%
\pgfsetbuttcap%
\pgfsetroundjoin%
\definecolor{currentfill}{rgb}{0.121569,0.466667,0.705882}%
\pgfsetfillcolor{currentfill}%
\pgfsetfillopacity{0.500804}%
\pgfsetlinewidth{1.003750pt}%
\definecolor{currentstroke}{rgb}{0.121569,0.466667,0.705882}%
\pgfsetstrokecolor{currentstroke}%
\pgfsetstrokeopacity{0.500804}%
\pgfsetdash{}{0pt}%
\pgfpathmoveto{\pgfqpoint{0.838496in}{1.589492in}}%
\pgfpathcurveto{\pgfqpoint{0.846732in}{1.589492in}}{\pgfqpoint{0.854632in}{1.592765in}}{\pgfqpoint{0.860456in}{1.598589in}}%
\pgfpathcurveto{\pgfqpoint{0.866280in}{1.604412in}}{\pgfqpoint{0.869552in}{1.612312in}}{\pgfqpoint{0.869552in}{1.620549in}}%
\pgfpathcurveto{\pgfqpoint{0.869552in}{1.628785in}}{\pgfqpoint{0.866280in}{1.636685in}}{\pgfqpoint{0.860456in}{1.642509in}}%
\pgfpathcurveto{\pgfqpoint{0.854632in}{1.648333in}}{\pgfqpoint{0.846732in}{1.651605in}}{\pgfqpoint{0.838496in}{1.651605in}}%
\pgfpathcurveto{\pgfqpoint{0.830259in}{1.651605in}}{\pgfqpoint{0.822359in}{1.648333in}}{\pgfqpoint{0.816535in}{1.642509in}}%
\pgfpathcurveto{\pgfqpoint{0.810711in}{1.636685in}}{\pgfqpoint{0.807439in}{1.628785in}}{\pgfqpoint{0.807439in}{1.620549in}}%
\pgfpathcurveto{\pgfqpoint{0.807439in}{1.612312in}}{\pgfqpoint{0.810711in}{1.604412in}}{\pgfqpoint{0.816535in}{1.598589in}}%
\pgfpathcurveto{\pgfqpoint{0.822359in}{1.592765in}}{\pgfqpoint{0.830259in}{1.589492in}}{\pgfqpoint{0.838496in}{1.589492in}}%
\pgfpathclose%
\pgfusepath{stroke,fill}%
\end{pgfscope}%
\begin{pgfscope}%
\pgfpathrectangle{\pgfqpoint{0.100000in}{0.220728in}}{\pgfqpoint{3.696000in}{3.696000in}}%
\pgfusepath{clip}%
\pgfsetbuttcap%
\pgfsetroundjoin%
\definecolor{currentfill}{rgb}{0.121569,0.466667,0.705882}%
\pgfsetfillcolor{currentfill}%
\pgfsetfillopacity{0.526235}%
\pgfsetlinewidth{1.003750pt}%
\definecolor{currentstroke}{rgb}{0.121569,0.466667,0.705882}%
\pgfsetstrokecolor{currentstroke}%
\pgfsetstrokeopacity{0.526235}%
\pgfsetdash{}{0pt}%
\pgfpathmoveto{\pgfqpoint{0.850295in}{1.583640in}}%
\pgfpathcurveto{\pgfqpoint{0.858532in}{1.583640in}}{\pgfqpoint{0.866432in}{1.586912in}}{\pgfqpoint{0.872256in}{1.592736in}}%
\pgfpathcurveto{\pgfqpoint{0.878080in}{1.598560in}}{\pgfqpoint{0.881352in}{1.606460in}}{\pgfqpoint{0.881352in}{1.614697in}}%
\pgfpathcurveto{\pgfqpoint{0.881352in}{1.622933in}}{\pgfqpoint{0.878080in}{1.630833in}}{\pgfqpoint{0.872256in}{1.636657in}}%
\pgfpathcurveto{\pgfqpoint{0.866432in}{1.642481in}}{\pgfqpoint{0.858532in}{1.645753in}}{\pgfqpoint{0.850295in}{1.645753in}}%
\pgfpathcurveto{\pgfqpoint{0.842059in}{1.645753in}}{\pgfqpoint{0.834159in}{1.642481in}}{\pgfqpoint{0.828335in}{1.636657in}}%
\pgfpathcurveto{\pgfqpoint{0.822511in}{1.630833in}}{\pgfqpoint{0.819239in}{1.622933in}}{\pgfqpoint{0.819239in}{1.614697in}}%
\pgfpathcurveto{\pgfqpoint{0.819239in}{1.606460in}}{\pgfqpoint{0.822511in}{1.598560in}}{\pgfqpoint{0.828335in}{1.592736in}}%
\pgfpathcurveto{\pgfqpoint{0.834159in}{1.586912in}}{\pgfqpoint{0.842059in}{1.583640in}}{\pgfqpoint{0.850295in}{1.583640in}}%
\pgfpathclose%
\pgfusepath{stroke,fill}%
\end{pgfscope}%
\begin{pgfscope}%
\pgfpathrectangle{\pgfqpoint{0.100000in}{0.220728in}}{\pgfqpoint{3.696000in}{3.696000in}}%
\pgfusepath{clip}%
\pgfsetbuttcap%
\pgfsetroundjoin%
\definecolor{currentfill}{rgb}{0.121569,0.466667,0.705882}%
\pgfsetfillcolor{currentfill}%
\pgfsetfillopacity{0.532706}%
\pgfsetlinewidth{1.003750pt}%
\definecolor{currentstroke}{rgb}{0.121569,0.466667,0.705882}%
\pgfsetstrokecolor{currentstroke}%
\pgfsetstrokeopacity{0.532706}%
\pgfsetdash{}{0pt}%
\pgfpathmoveto{\pgfqpoint{0.657038in}{1.936066in}}%
\pgfpathcurveto{\pgfqpoint{0.665274in}{1.936066in}}{\pgfqpoint{0.673174in}{1.939338in}}{\pgfqpoint{0.678998in}{1.945162in}}%
\pgfpathcurveto{\pgfqpoint{0.684822in}{1.950986in}}{\pgfqpoint{0.688094in}{1.958886in}}{\pgfqpoint{0.688094in}{1.967122in}}%
\pgfpathcurveto{\pgfqpoint{0.688094in}{1.975358in}}{\pgfqpoint{0.684822in}{1.983258in}}{\pgfqpoint{0.678998in}{1.989082in}}%
\pgfpathcurveto{\pgfqpoint{0.673174in}{1.994906in}}{\pgfqpoint{0.665274in}{1.998179in}}{\pgfqpoint{0.657038in}{1.998179in}}%
\pgfpathcurveto{\pgfqpoint{0.648802in}{1.998179in}}{\pgfqpoint{0.640901in}{1.994906in}}{\pgfqpoint{0.635078in}{1.989082in}}%
\pgfpathcurveto{\pgfqpoint{0.629254in}{1.983258in}}{\pgfqpoint{0.625981in}{1.975358in}}{\pgfqpoint{0.625981in}{1.967122in}}%
\pgfpathcurveto{\pgfqpoint{0.625981in}{1.958886in}}{\pgfqpoint{0.629254in}{1.950986in}}{\pgfqpoint{0.635078in}{1.945162in}}%
\pgfpathcurveto{\pgfqpoint{0.640901in}{1.939338in}}{\pgfqpoint{0.648802in}{1.936066in}}{\pgfqpoint{0.657038in}{1.936066in}}%
\pgfpathclose%
\pgfusepath{stroke,fill}%
\end{pgfscope}%
\begin{pgfscope}%
\pgfpathrectangle{\pgfqpoint{0.100000in}{0.220728in}}{\pgfqpoint{3.696000in}{3.696000in}}%
\pgfusepath{clip}%
\pgfsetbuttcap%
\pgfsetroundjoin%
\definecolor{currentfill}{rgb}{0.121569,0.466667,0.705882}%
\pgfsetfillcolor{currentfill}%
\pgfsetfillopacity{0.546649}%
\pgfsetlinewidth{1.003750pt}%
\definecolor{currentstroke}{rgb}{0.121569,0.466667,0.705882}%
\pgfsetstrokecolor{currentstroke}%
\pgfsetstrokeopacity{0.546649}%
\pgfsetdash{}{0pt}%
\pgfpathmoveto{\pgfqpoint{0.864098in}{1.583413in}}%
\pgfpathcurveto{\pgfqpoint{0.872335in}{1.583413in}}{\pgfqpoint{0.880235in}{1.586686in}}{\pgfqpoint{0.886058in}{1.592510in}}%
\pgfpathcurveto{\pgfqpoint{0.891882in}{1.598334in}}{\pgfqpoint{0.895155in}{1.606234in}}{\pgfqpoint{0.895155in}{1.614470in}}%
\pgfpathcurveto{\pgfqpoint{0.895155in}{1.622706in}}{\pgfqpoint{0.891882in}{1.630606in}}{\pgfqpoint{0.886058in}{1.636430in}}%
\pgfpathcurveto{\pgfqpoint{0.880235in}{1.642254in}}{\pgfqpoint{0.872335in}{1.645526in}}{\pgfqpoint{0.864098in}{1.645526in}}%
\pgfpathcurveto{\pgfqpoint{0.855862in}{1.645526in}}{\pgfqpoint{0.847962in}{1.642254in}}{\pgfqpoint{0.842138in}{1.636430in}}%
\pgfpathcurveto{\pgfqpoint{0.836314in}{1.630606in}}{\pgfqpoint{0.833042in}{1.622706in}}{\pgfqpoint{0.833042in}{1.614470in}}%
\pgfpathcurveto{\pgfqpoint{0.833042in}{1.606234in}}{\pgfqpoint{0.836314in}{1.598334in}}{\pgfqpoint{0.842138in}{1.592510in}}%
\pgfpathcurveto{\pgfqpoint{0.847962in}{1.586686in}}{\pgfqpoint{0.855862in}{1.583413in}}{\pgfqpoint{0.864098in}{1.583413in}}%
\pgfpathclose%
\pgfusepath{stroke,fill}%
\end{pgfscope}%
\begin{pgfscope}%
\pgfpathrectangle{\pgfqpoint{0.100000in}{0.220728in}}{\pgfqpoint{3.696000in}{3.696000in}}%
\pgfusepath{clip}%
\pgfsetbuttcap%
\pgfsetroundjoin%
\definecolor{currentfill}{rgb}{0.121569,0.466667,0.705882}%
\pgfsetfillcolor{currentfill}%
\pgfsetfillopacity{0.546736}%
\pgfsetlinewidth{1.003750pt}%
\definecolor{currentstroke}{rgb}{0.121569,0.466667,0.705882}%
\pgfsetstrokecolor{currentstroke}%
\pgfsetstrokeopacity{0.546736}%
\pgfsetdash{}{0pt}%
\pgfpathmoveto{\pgfqpoint{0.878403in}{1.566778in}}%
\pgfpathcurveto{\pgfqpoint{0.886640in}{1.566778in}}{\pgfqpoint{0.894540in}{1.570050in}}{\pgfqpoint{0.900364in}{1.575874in}}%
\pgfpathcurveto{\pgfqpoint{0.906187in}{1.581698in}}{\pgfqpoint{0.909460in}{1.589598in}}{\pgfqpoint{0.909460in}{1.597834in}}%
\pgfpathcurveto{\pgfqpoint{0.909460in}{1.606071in}}{\pgfqpoint{0.906187in}{1.613971in}}{\pgfqpoint{0.900364in}{1.619794in}}%
\pgfpathcurveto{\pgfqpoint{0.894540in}{1.625618in}}{\pgfqpoint{0.886640in}{1.628891in}}{\pgfqpoint{0.878403in}{1.628891in}}%
\pgfpathcurveto{\pgfqpoint{0.870167in}{1.628891in}}{\pgfqpoint{0.862267in}{1.625618in}}{\pgfqpoint{0.856443in}{1.619794in}}%
\pgfpathcurveto{\pgfqpoint{0.850619in}{1.613971in}}{\pgfqpoint{0.847347in}{1.606071in}}{\pgfqpoint{0.847347in}{1.597834in}}%
\pgfpathcurveto{\pgfqpoint{0.847347in}{1.589598in}}{\pgfqpoint{0.850619in}{1.581698in}}{\pgfqpoint{0.856443in}{1.575874in}}%
\pgfpathcurveto{\pgfqpoint{0.862267in}{1.570050in}}{\pgfqpoint{0.870167in}{1.566778in}}{\pgfqpoint{0.878403in}{1.566778in}}%
\pgfpathclose%
\pgfusepath{stroke,fill}%
\end{pgfscope}%
\begin{pgfscope}%
\pgfpathrectangle{\pgfqpoint{0.100000in}{0.220728in}}{\pgfqpoint{3.696000in}{3.696000in}}%
\pgfusepath{clip}%
\pgfsetbuttcap%
\pgfsetroundjoin%
\definecolor{currentfill}{rgb}{0.121569,0.466667,0.705882}%
\pgfsetfillcolor{currentfill}%
\pgfsetfillopacity{0.566829}%
\pgfsetlinewidth{1.003750pt}%
\definecolor{currentstroke}{rgb}{0.121569,0.466667,0.705882}%
\pgfsetstrokecolor{currentstroke}%
\pgfsetstrokeopacity{0.566829}%
\pgfsetdash{}{0pt}%
\pgfpathmoveto{\pgfqpoint{0.887394in}{1.563445in}}%
\pgfpathcurveto{\pgfqpoint{0.895631in}{1.563445in}}{\pgfqpoint{0.903531in}{1.566718in}}{\pgfqpoint{0.909355in}{1.572542in}}%
\pgfpathcurveto{\pgfqpoint{0.915179in}{1.578366in}}{\pgfqpoint{0.918451in}{1.586266in}}{\pgfqpoint{0.918451in}{1.594502in}}%
\pgfpathcurveto{\pgfqpoint{0.918451in}{1.602738in}}{\pgfqpoint{0.915179in}{1.610638in}}{\pgfqpoint{0.909355in}{1.616462in}}%
\pgfpathcurveto{\pgfqpoint{0.903531in}{1.622286in}}{\pgfqpoint{0.895631in}{1.625558in}}{\pgfqpoint{0.887394in}{1.625558in}}%
\pgfpathcurveto{\pgfqpoint{0.879158in}{1.625558in}}{\pgfqpoint{0.871258in}{1.622286in}}{\pgfqpoint{0.865434in}{1.616462in}}%
\pgfpathcurveto{\pgfqpoint{0.859610in}{1.610638in}}{\pgfqpoint{0.856338in}{1.602738in}}{\pgfqpoint{0.856338in}{1.594502in}}%
\pgfpathcurveto{\pgfqpoint{0.856338in}{1.586266in}}{\pgfqpoint{0.859610in}{1.578366in}}{\pgfqpoint{0.865434in}{1.572542in}}%
\pgfpathcurveto{\pgfqpoint{0.871258in}{1.566718in}}{\pgfqpoint{0.879158in}{1.563445in}}{\pgfqpoint{0.887394in}{1.563445in}}%
\pgfpathclose%
\pgfusepath{stroke,fill}%
\end{pgfscope}%
\begin{pgfscope}%
\pgfpathrectangle{\pgfqpoint{0.100000in}{0.220728in}}{\pgfqpoint{3.696000in}{3.696000in}}%
\pgfusepath{clip}%
\pgfsetbuttcap%
\pgfsetroundjoin%
\definecolor{currentfill}{rgb}{0.121569,0.466667,0.705882}%
\pgfsetfillcolor{currentfill}%
\pgfsetfillopacity{0.580910}%
\pgfsetlinewidth{1.003750pt}%
\definecolor{currentstroke}{rgb}{0.121569,0.466667,0.705882}%
\pgfsetstrokecolor{currentstroke}%
\pgfsetstrokeopacity{0.580910}%
\pgfsetdash{}{0pt}%
\pgfpathmoveto{\pgfqpoint{0.897555in}{1.561282in}}%
\pgfpathcurveto{\pgfqpoint{0.905791in}{1.561282in}}{\pgfqpoint{0.913691in}{1.564554in}}{\pgfqpoint{0.919515in}{1.570378in}}%
\pgfpathcurveto{\pgfqpoint{0.925339in}{1.576202in}}{\pgfqpoint{0.928612in}{1.584102in}}{\pgfqpoint{0.928612in}{1.592339in}}%
\pgfpathcurveto{\pgfqpoint{0.928612in}{1.600575in}}{\pgfqpoint{0.925339in}{1.608475in}}{\pgfqpoint{0.919515in}{1.614299in}}%
\pgfpathcurveto{\pgfqpoint{0.913691in}{1.620123in}}{\pgfqpoint{0.905791in}{1.623395in}}{\pgfqpoint{0.897555in}{1.623395in}}%
\pgfpathcurveto{\pgfqpoint{0.889319in}{1.623395in}}{\pgfqpoint{0.881419in}{1.620123in}}{\pgfqpoint{0.875595in}{1.614299in}}%
\pgfpathcurveto{\pgfqpoint{0.869771in}{1.608475in}}{\pgfqpoint{0.866499in}{1.600575in}}{\pgfqpoint{0.866499in}{1.592339in}}%
\pgfpathcurveto{\pgfqpoint{0.866499in}{1.584102in}}{\pgfqpoint{0.869771in}{1.576202in}}{\pgfqpoint{0.875595in}{1.570378in}}%
\pgfpathcurveto{\pgfqpoint{0.881419in}{1.564554in}}{\pgfqpoint{0.889319in}{1.561282in}}{\pgfqpoint{0.897555in}{1.561282in}}%
\pgfpathclose%
\pgfusepath{stroke,fill}%
\end{pgfscope}%
\begin{pgfscope}%
\pgfpathrectangle{\pgfqpoint{0.100000in}{0.220728in}}{\pgfqpoint{3.696000in}{3.696000in}}%
\pgfusepath{clip}%
\pgfsetbuttcap%
\pgfsetroundjoin%
\definecolor{currentfill}{rgb}{0.121569,0.466667,0.705882}%
\pgfsetfillcolor{currentfill}%
\pgfsetfillopacity{0.592728}%
\pgfsetlinewidth{1.003750pt}%
\definecolor{currentstroke}{rgb}{0.121569,0.466667,0.705882}%
\pgfsetstrokecolor{currentstroke}%
\pgfsetstrokeopacity{0.592728}%
\pgfsetdash{}{0pt}%
\pgfpathmoveto{\pgfqpoint{0.906763in}{1.559289in}}%
\pgfpathcurveto{\pgfqpoint{0.914999in}{1.559289in}}{\pgfqpoint{0.922900in}{1.562561in}}{\pgfqpoint{0.928723in}{1.568385in}}%
\pgfpathcurveto{\pgfqpoint{0.934547in}{1.574209in}}{\pgfqpoint{0.937820in}{1.582109in}}{\pgfqpoint{0.937820in}{1.590345in}}%
\pgfpathcurveto{\pgfqpoint{0.937820in}{1.598582in}}{\pgfqpoint{0.934547in}{1.606482in}}{\pgfqpoint{0.928723in}{1.612306in}}%
\pgfpathcurveto{\pgfqpoint{0.922900in}{1.618130in}}{\pgfqpoint{0.914999in}{1.621402in}}{\pgfqpoint{0.906763in}{1.621402in}}%
\pgfpathcurveto{\pgfqpoint{0.898527in}{1.621402in}}{\pgfqpoint{0.890627in}{1.618130in}}{\pgfqpoint{0.884803in}{1.612306in}}%
\pgfpathcurveto{\pgfqpoint{0.878979in}{1.606482in}}{\pgfqpoint{0.875707in}{1.598582in}}{\pgfqpoint{0.875707in}{1.590345in}}%
\pgfpathcurveto{\pgfqpoint{0.875707in}{1.582109in}}{\pgfqpoint{0.878979in}{1.574209in}}{\pgfqpoint{0.884803in}{1.568385in}}%
\pgfpathcurveto{\pgfqpoint{0.890627in}{1.562561in}}{\pgfqpoint{0.898527in}{1.559289in}}{\pgfqpoint{0.906763in}{1.559289in}}%
\pgfpathclose%
\pgfusepath{stroke,fill}%
\end{pgfscope}%
\begin{pgfscope}%
\pgfpathrectangle{\pgfqpoint{0.100000in}{0.220728in}}{\pgfqpoint{3.696000in}{3.696000in}}%
\pgfusepath{clip}%
\pgfsetbuttcap%
\pgfsetroundjoin%
\definecolor{currentfill}{rgb}{0.121569,0.466667,0.705882}%
\pgfsetfillcolor{currentfill}%
\pgfsetfillopacity{0.618105}%
\pgfsetlinewidth{1.003750pt}%
\definecolor{currentstroke}{rgb}{0.121569,0.466667,0.705882}%
\pgfsetstrokecolor{currentstroke}%
\pgfsetstrokeopacity{0.618105}%
\pgfsetdash{}{0pt}%
\pgfpathmoveto{\pgfqpoint{0.910138in}{1.561633in}}%
\pgfpathcurveto{\pgfqpoint{0.918375in}{1.561633in}}{\pgfqpoint{0.926275in}{1.564905in}}{\pgfqpoint{0.932099in}{1.570729in}}%
\pgfpathcurveto{\pgfqpoint{0.937922in}{1.576553in}}{\pgfqpoint{0.941195in}{1.584453in}}{\pgfqpoint{0.941195in}{1.592689in}}%
\pgfpathcurveto{\pgfqpoint{0.941195in}{1.600926in}}{\pgfqpoint{0.937922in}{1.608826in}}{\pgfqpoint{0.932099in}{1.614650in}}%
\pgfpathcurveto{\pgfqpoint{0.926275in}{1.620473in}}{\pgfqpoint{0.918375in}{1.623746in}}{\pgfqpoint{0.910138in}{1.623746in}}%
\pgfpathcurveto{\pgfqpoint{0.901902in}{1.623746in}}{\pgfqpoint{0.894002in}{1.620473in}}{\pgfqpoint{0.888178in}{1.614650in}}%
\pgfpathcurveto{\pgfqpoint{0.882354in}{1.608826in}}{\pgfqpoint{0.879082in}{1.600926in}}{\pgfqpoint{0.879082in}{1.592689in}}%
\pgfpathcurveto{\pgfqpoint{0.879082in}{1.584453in}}{\pgfqpoint{0.882354in}{1.576553in}}{\pgfqpoint{0.888178in}{1.570729in}}%
\pgfpathcurveto{\pgfqpoint{0.894002in}{1.564905in}}{\pgfqpoint{0.901902in}{1.561633in}}{\pgfqpoint{0.910138in}{1.561633in}}%
\pgfpathclose%
\pgfusepath{stroke,fill}%
\end{pgfscope}%
\begin{pgfscope}%
\pgfpathrectangle{\pgfqpoint{0.100000in}{0.220728in}}{\pgfqpoint{3.696000in}{3.696000in}}%
\pgfusepath{clip}%
\pgfsetbuttcap%
\pgfsetroundjoin%
\definecolor{currentfill}{rgb}{0.121569,0.466667,0.705882}%
\pgfsetfillcolor{currentfill}%
\pgfsetfillopacity{0.618804}%
\pgfsetlinewidth{1.003750pt}%
\definecolor{currentstroke}{rgb}{0.121569,0.466667,0.705882}%
\pgfsetstrokecolor{currentstroke}%
\pgfsetstrokeopacity{0.618804}%
\pgfsetdash{}{0pt}%
\pgfpathmoveto{\pgfqpoint{0.917952in}{1.552738in}}%
\pgfpathcurveto{\pgfqpoint{0.926188in}{1.552738in}}{\pgfqpoint{0.934088in}{1.556011in}}{\pgfqpoint{0.939912in}{1.561834in}}%
\pgfpathcurveto{\pgfqpoint{0.945736in}{1.567658in}}{\pgfqpoint{0.949009in}{1.575558in}}{\pgfqpoint{0.949009in}{1.583795in}}%
\pgfpathcurveto{\pgfqpoint{0.949009in}{1.592031in}}{\pgfqpoint{0.945736in}{1.599931in}}{\pgfqpoint{0.939912in}{1.605755in}}%
\pgfpathcurveto{\pgfqpoint{0.934088in}{1.611579in}}{\pgfqpoint{0.926188in}{1.614851in}}{\pgfqpoint{0.917952in}{1.614851in}}%
\pgfpathcurveto{\pgfqpoint{0.909716in}{1.614851in}}{\pgfqpoint{0.901816in}{1.611579in}}{\pgfqpoint{0.895992in}{1.605755in}}%
\pgfpathcurveto{\pgfqpoint{0.890168in}{1.599931in}}{\pgfqpoint{0.886896in}{1.592031in}}{\pgfqpoint{0.886896in}{1.583795in}}%
\pgfpathcurveto{\pgfqpoint{0.886896in}{1.575558in}}{\pgfqpoint{0.890168in}{1.567658in}}{\pgfqpoint{0.895992in}{1.561834in}}%
\pgfpathcurveto{\pgfqpoint{0.901816in}{1.556011in}}{\pgfqpoint{0.909716in}{1.552738in}}{\pgfqpoint{0.917952in}{1.552738in}}%
\pgfpathclose%
\pgfusepath{stroke,fill}%
\end{pgfscope}%
\begin{pgfscope}%
\pgfpathrectangle{\pgfqpoint{0.100000in}{0.220728in}}{\pgfqpoint{3.696000in}{3.696000in}}%
\pgfusepath{clip}%
\pgfsetbuttcap%
\pgfsetroundjoin%
\definecolor{currentfill}{rgb}{0.121569,0.466667,0.705882}%
\pgfsetfillcolor{currentfill}%
\pgfsetfillopacity{0.619912}%
\pgfsetlinewidth{1.003750pt}%
\definecolor{currentstroke}{rgb}{0.121569,0.466667,0.705882}%
\pgfsetstrokecolor{currentstroke}%
\pgfsetstrokeopacity{0.619912}%
\pgfsetdash{}{0pt}%
\pgfpathmoveto{\pgfqpoint{0.923956in}{1.545328in}}%
\pgfpathcurveto{\pgfqpoint{0.932193in}{1.545328in}}{\pgfqpoint{0.940093in}{1.548600in}}{\pgfqpoint{0.945917in}{1.554424in}}%
\pgfpathcurveto{\pgfqpoint{0.951741in}{1.560248in}}{\pgfqpoint{0.955013in}{1.568148in}}{\pgfqpoint{0.955013in}{1.576385in}}%
\pgfpathcurveto{\pgfqpoint{0.955013in}{1.584621in}}{\pgfqpoint{0.951741in}{1.592521in}}{\pgfqpoint{0.945917in}{1.598345in}}%
\pgfpathcurveto{\pgfqpoint{0.940093in}{1.604169in}}{\pgfqpoint{0.932193in}{1.607441in}}{\pgfqpoint{0.923956in}{1.607441in}}%
\pgfpathcurveto{\pgfqpoint{0.915720in}{1.607441in}}{\pgfqpoint{0.907820in}{1.604169in}}{\pgfqpoint{0.901996in}{1.598345in}}%
\pgfpathcurveto{\pgfqpoint{0.896172in}{1.592521in}}{\pgfqpoint{0.892900in}{1.584621in}}{\pgfqpoint{0.892900in}{1.576385in}}%
\pgfpathcurveto{\pgfqpoint{0.892900in}{1.568148in}}{\pgfqpoint{0.896172in}{1.560248in}}{\pgfqpoint{0.901996in}{1.554424in}}%
\pgfpathcurveto{\pgfqpoint{0.907820in}{1.548600in}}{\pgfqpoint{0.915720in}{1.545328in}}{\pgfqpoint{0.923956in}{1.545328in}}%
\pgfpathclose%
\pgfusepath{stroke,fill}%
\end{pgfscope}%
\begin{pgfscope}%
\pgfpathrectangle{\pgfqpoint{0.100000in}{0.220728in}}{\pgfqpoint{3.696000in}{3.696000in}}%
\pgfusepath{clip}%
\pgfsetbuttcap%
\pgfsetroundjoin%
\definecolor{currentfill}{rgb}{0.121569,0.466667,0.705882}%
\pgfsetfillcolor{currentfill}%
\pgfsetfillopacity{0.625423}%
\pgfsetlinewidth{1.003750pt}%
\definecolor{currentstroke}{rgb}{0.121569,0.466667,0.705882}%
\pgfsetstrokecolor{currentstroke}%
\pgfsetstrokeopacity{0.625423}%
\pgfsetdash{}{0pt}%
\pgfpathmoveto{\pgfqpoint{1.211523in}{1.309179in}}%
\pgfpathcurveto{\pgfqpoint{1.219759in}{1.309179in}}{\pgfqpoint{1.227659in}{1.312451in}}{\pgfqpoint{1.233483in}{1.318275in}}%
\pgfpathcurveto{\pgfqpoint{1.239307in}{1.324099in}}{\pgfqpoint{1.242579in}{1.331999in}}{\pgfqpoint{1.242579in}{1.340235in}}%
\pgfpathcurveto{\pgfqpoint{1.242579in}{1.348472in}}{\pgfqpoint{1.239307in}{1.356372in}}{\pgfqpoint{1.233483in}{1.362195in}}%
\pgfpathcurveto{\pgfqpoint{1.227659in}{1.368019in}}{\pgfqpoint{1.219759in}{1.371292in}}{\pgfqpoint{1.211523in}{1.371292in}}%
\pgfpathcurveto{\pgfqpoint{1.203286in}{1.371292in}}{\pgfqpoint{1.195386in}{1.368019in}}{\pgfqpoint{1.189562in}{1.362195in}}%
\pgfpathcurveto{\pgfqpoint{1.183739in}{1.356372in}}{\pgfqpoint{1.180466in}{1.348472in}}{\pgfqpoint{1.180466in}{1.340235in}}%
\pgfpathcurveto{\pgfqpoint{1.180466in}{1.331999in}}{\pgfqpoint{1.183739in}{1.324099in}}{\pgfqpoint{1.189562in}{1.318275in}}%
\pgfpathcurveto{\pgfqpoint{1.195386in}{1.312451in}}{\pgfqpoint{1.203286in}{1.309179in}}{\pgfqpoint{1.211523in}{1.309179in}}%
\pgfpathclose%
\pgfusepath{stroke,fill}%
\end{pgfscope}%
\begin{pgfscope}%
\pgfpathrectangle{\pgfqpoint{0.100000in}{0.220728in}}{\pgfqpoint{3.696000in}{3.696000in}}%
\pgfusepath{clip}%
\pgfsetbuttcap%
\pgfsetroundjoin%
\definecolor{currentfill}{rgb}{0.121569,0.466667,0.705882}%
\pgfsetfillcolor{currentfill}%
\pgfsetfillopacity{0.627510}%
\pgfsetlinewidth{1.003750pt}%
\definecolor{currentstroke}{rgb}{0.121569,0.466667,0.705882}%
\pgfsetstrokecolor{currentstroke}%
\pgfsetstrokeopacity{0.627510}%
\pgfsetdash{}{0pt}%
\pgfpathmoveto{\pgfqpoint{1.219730in}{1.298273in}}%
\pgfpathcurveto{\pgfqpoint{1.227967in}{1.298273in}}{\pgfqpoint{1.235867in}{1.301545in}}{\pgfqpoint{1.241691in}{1.307369in}}%
\pgfpathcurveto{\pgfqpoint{1.247514in}{1.313193in}}{\pgfqpoint{1.250787in}{1.321093in}}{\pgfqpoint{1.250787in}{1.329329in}}%
\pgfpathcurveto{\pgfqpoint{1.250787in}{1.337566in}}{\pgfqpoint{1.247514in}{1.345466in}}{\pgfqpoint{1.241691in}{1.351290in}}%
\pgfpathcurveto{\pgfqpoint{1.235867in}{1.357114in}}{\pgfqpoint{1.227967in}{1.360386in}}{\pgfqpoint{1.219730in}{1.360386in}}%
\pgfpathcurveto{\pgfqpoint{1.211494in}{1.360386in}}{\pgfqpoint{1.203594in}{1.357114in}}{\pgfqpoint{1.197770in}{1.351290in}}%
\pgfpathcurveto{\pgfqpoint{1.191946in}{1.345466in}}{\pgfqpoint{1.188674in}{1.337566in}}{\pgfqpoint{1.188674in}{1.329329in}}%
\pgfpathcurveto{\pgfqpoint{1.188674in}{1.321093in}}{\pgfqpoint{1.191946in}{1.313193in}}{\pgfqpoint{1.197770in}{1.307369in}}%
\pgfpathcurveto{\pgfqpoint{1.203594in}{1.301545in}}{\pgfqpoint{1.211494in}{1.298273in}}{\pgfqpoint{1.219730in}{1.298273in}}%
\pgfpathclose%
\pgfusepath{stroke,fill}%
\end{pgfscope}%
\begin{pgfscope}%
\pgfpathrectangle{\pgfqpoint{0.100000in}{0.220728in}}{\pgfqpoint{3.696000in}{3.696000in}}%
\pgfusepath{clip}%
\pgfsetbuttcap%
\pgfsetroundjoin%
\definecolor{currentfill}{rgb}{0.121569,0.466667,0.705882}%
\pgfsetfillcolor{currentfill}%
\pgfsetfillopacity{0.629838}%
\pgfsetlinewidth{1.003750pt}%
\definecolor{currentstroke}{rgb}{0.121569,0.466667,0.705882}%
\pgfsetstrokecolor{currentstroke}%
\pgfsetstrokeopacity{0.629838}%
\pgfsetdash{}{0pt}%
\pgfpathmoveto{\pgfqpoint{0.927168in}{1.545164in}}%
\pgfpathcurveto{\pgfqpoint{0.935404in}{1.545164in}}{\pgfqpoint{0.943304in}{1.548436in}}{\pgfqpoint{0.949128in}{1.554260in}}%
\pgfpathcurveto{\pgfqpoint{0.954952in}{1.560084in}}{\pgfqpoint{0.958224in}{1.567984in}}{\pgfqpoint{0.958224in}{1.576220in}}%
\pgfpathcurveto{\pgfqpoint{0.958224in}{1.584457in}}{\pgfqpoint{0.954952in}{1.592357in}}{\pgfqpoint{0.949128in}{1.598181in}}%
\pgfpathcurveto{\pgfqpoint{0.943304in}{1.604005in}}{\pgfqpoint{0.935404in}{1.607277in}}{\pgfqpoint{0.927168in}{1.607277in}}%
\pgfpathcurveto{\pgfqpoint{0.918931in}{1.607277in}}{\pgfqpoint{0.911031in}{1.604005in}}{\pgfqpoint{0.905207in}{1.598181in}}%
\pgfpathcurveto{\pgfqpoint{0.899383in}{1.592357in}}{\pgfqpoint{0.896111in}{1.584457in}}{\pgfqpoint{0.896111in}{1.576220in}}%
\pgfpathcurveto{\pgfqpoint{0.896111in}{1.567984in}}{\pgfqpoint{0.899383in}{1.560084in}}{\pgfqpoint{0.905207in}{1.554260in}}%
\pgfpathcurveto{\pgfqpoint{0.911031in}{1.548436in}}{\pgfqpoint{0.918931in}{1.545164in}}{\pgfqpoint{0.927168in}{1.545164in}}%
\pgfpathclose%
\pgfusepath{stroke,fill}%
\end{pgfscope}%
\begin{pgfscope}%
\pgfpathrectangle{\pgfqpoint{0.100000in}{0.220728in}}{\pgfqpoint{3.696000in}{3.696000in}}%
\pgfusepath{clip}%
\pgfsetbuttcap%
\pgfsetroundjoin%
\definecolor{currentfill}{rgb}{0.121569,0.466667,0.705882}%
\pgfsetfillcolor{currentfill}%
\pgfsetfillopacity{0.637243}%
\pgfsetlinewidth{1.003750pt}%
\definecolor{currentstroke}{rgb}{0.121569,0.466667,0.705882}%
\pgfsetstrokecolor{currentstroke}%
\pgfsetstrokeopacity{0.637243}%
\pgfsetdash{}{0pt}%
\pgfpathmoveto{\pgfqpoint{0.929475in}{1.544277in}}%
\pgfpathcurveto{\pgfqpoint{0.937712in}{1.544277in}}{\pgfqpoint{0.945612in}{1.547549in}}{\pgfqpoint{0.951436in}{1.553373in}}%
\pgfpathcurveto{\pgfqpoint{0.957260in}{1.559197in}}{\pgfqpoint{0.960532in}{1.567097in}}{\pgfqpoint{0.960532in}{1.575333in}}%
\pgfpathcurveto{\pgfqpoint{0.960532in}{1.583570in}}{\pgfqpoint{0.957260in}{1.591470in}}{\pgfqpoint{0.951436in}{1.597294in}}%
\pgfpathcurveto{\pgfqpoint{0.945612in}{1.603117in}}{\pgfqpoint{0.937712in}{1.606390in}}{\pgfqpoint{0.929475in}{1.606390in}}%
\pgfpathcurveto{\pgfqpoint{0.921239in}{1.606390in}}{\pgfqpoint{0.913339in}{1.603117in}}{\pgfqpoint{0.907515in}{1.597294in}}%
\pgfpathcurveto{\pgfqpoint{0.901691in}{1.591470in}}{\pgfqpoint{0.898419in}{1.583570in}}{\pgfqpoint{0.898419in}{1.575333in}}%
\pgfpathcurveto{\pgfqpoint{0.898419in}{1.567097in}}{\pgfqpoint{0.901691in}{1.559197in}}{\pgfqpoint{0.907515in}{1.553373in}}%
\pgfpathcurveto{\pgfqpoint{0.913339in}{1.547549in}}{\pgfqpoint{0.921239in}{1.544277in}}{\pgfqpoint{0.929475in}{1.544277in}}%
\pgfpathclose%
\pgfusepath{stroke,fill}%
\end{pgfscope}%
\begin{pgfscope}%
\pgfpathrectangle{\pgfqpoint{0.100000in}{0.220728in}}{\pgfqpoint{3.696000in}{3.696000in}}%
\pgfusepath{clip}%
\pgfsetbuttcap%
\pgfsetroundjoin%
\definecolor{currentfill}{rgb}{0.121569,0.466667,0.705882}%
\pgfsetfillcolor{currentfill}%
\pgfsetfillopacity{0.646047}%
\pgfsetlinewidth{1.003750pt}%
\definecolor{currentstroke}{rgb}{0.121569,0.466667,0.705882}%
\pgfsetstrokecolor{currentstroke}%
\pgfsetstrokeopacity{0.646047}%
\pgfsetdash{}{0pt}%
\pgfpathmoveto{\pgfqpoint{0.935072in}{1.541252in}}%
\pgfpathcurveto{\pgfqpoint{0.943309in}{1.541252in}}{\pgfqpoint{0.951209in}{1.544524in}}{\pgfqpoint{0.957033in}{1.550348in}}%
\pgfpathcurveto{\pgfqpoint{0.962857in}{1.556172in}}{\pgfqpoint{0.966129in}{1.564072in}}{\pgfqpoint{0.966129in}{1.572308in}}%
\pgfpathcurveto{\pgfqpoint{0.966129in}{1.580544in}}{\pgfqpoint{0.962857in}{1.588444in}}{\pgfqpoint{0.957033in}{1.594268in}}%
\pgfpathcurveto{\pgfqpoint{0.951209in}{1.600092in}}{\pgfqpoint{0.943309in}{1.603365in}}{\pgfqpoint{0.935072in}{1.603365in}}%
\pgfpathcurveto{\pgfqpoint{0.926836in}{1.603365in}}{\pgfqpoint{0.918936in}{1.600092in}}{\pgfqpoint{0.913112in}{1.594268in}}%
\pgfpathcurveto{\pgfqpoint{0.907288in}{1.588444in}}{\pgfqpoint{0.904016in}{1.580544in}}{\pgfqpoint{0.904016in}{1.572308in}}%
\pgfpathcurveto{\pgfqpoint{0.904016in}{1.564072in}}{\pgfqpoint{0.907288in}{1.556172in}}{\pgfqpoint{0.913112in}{1.550348in}}%
\pgfpathcurveto{\pgfqpoint{0.918936in}{1.544524in}}{\pgfqpoint{0.926836in}{1.541252in}}{\pgfqpoint{0.935072in}{1.541252in}}%
\pgfpathclose%
\pgfusepath{stroke,fill}%
\end{pgfscope}%
\begin{pgfscope}%
\pgfpathrectangle{\pgfqpoint{0.100000in}{0.220728in}}{\pgfqpoint{3.696000in}{3.696000in}}%
\pgfusepath{clip}%
\pgfsetbuttcap%
\pgfsetroundjoin%
\definecolor{currentfill}{rgb}{0.121569,0.466667,0.705882}%
\pgfsetfillcolor{currentfill}%
\pgfsetfillopacity{0.649163}%
\pgfsetlinewidth{1.003750pt}%
\definecolor{currentstroke}{rgb}{0.121569,0.466667,0.705882}%
\pgfsetstrokecolor{currentstroke}%
\pgfsetstrokeopacity{0.649163}%
\pgfsetdash{}{0pt}%
\pgfpathmoveto{\pgfqpoint{1.233964in}{1.296479in}}%
\pgfpathcurveto{\pgfqpoint{1.242200in}{1.296479in}}{\pgfqpoint{1.250100in}{1.299751in}}{\pgfqpoint{1.255924in}{1.305575in}}%
\pgfpathcurveto{\pgfqpoint{1.261748in}{1.311399in}}{\pgfqpoint{1.265020in}{1.319299in}}{\pgfqpoint{1.265020in}{1.327535in}}%
\pgfpathcurveto{\pgfqpoint{1.265020in}{1.335772in}}{\pgfqpoint{1.261748in}{1.343672in}}{\pgfqpoint{1.255924in}{1.349496in}}%
\pgfpathcurveto{\pgfqpoint{1.250100in}{1.355320in}}{\pgfqpoint{1.242200in}{1.358592in}}{\pgfqpoint{1.233964in}{1.358592in}}%
\pgfpathcurveto{\pgfqpoint{1.225727in}{1.358592in}}{\pgfqpoint{1.217827in}{1.355320in}}{\pgfqpoint{1.212003in}{1.349496in}}%
\pgfpathcurveto{\pgfqpoint{1.206179in}{1.343672in}}{\pgfqpoint{1.202907in}{1.335772in}}{\pgfqpoint{1.202907in}{1.327535in}}%
\pgfpathcurveto{\pgfqpoint{1.202907in}{1.319299in}}{\pgfqpoint{1.206179in}{1.311399in}}{\pgfqpoint{1.212003in}{1.305575in}}%
\pgfpathcurveto{\pgfqpoint{1.217827in}{1.299751in}}{\pgfqpoint{1.225727in}{1.296479in}}{\pgfqpoint{1.233964in}{1.296479in}}%
\pgfpathclose%
\pgfusepath{stroke,fill}%
\end{pgfscope}%
\begin{pgfscope}%
\pgfpathrectangle{\pgfqpoint{0.100000in}{0.220728in}}{\pgfqpoint{3.696000in}{3.696000in}}%
\pgfusepath{clip}%
\pgfsetbuttcap%
\pgfsetroundjoin%
\definecolor{currentfill}{rgb}{0.121569,0.466667,0.705882}%
\pgfsetfillcolor{currentfill}%
\pgfsetfillopacity{0.652460}%
\pgfsetlinewidth{1.003750pt}%
\definecolor{currentstroke}{rgb}{0.121569,0.466667,0.705882}%
\pgfsetstrokecolor{currentstroke}%
\pgfsetstrokeopacity{0.652460}%
\pgfsetdash{}{0pt}%
\pgfpathmoveto{\pgfqpoint{0.939827in}{1.539598in}}%
\pgfpathcurveto{\pgfqpoint{0.948063in}{1.539598in}}{\pgfqpoint{0.955963in}{1.542870in}}{\pgfqpoint{0.961787in}{1.548694in}}%
\pgfpathcurveto{\pgfqpoint{0.967611in}{1.554518in}}{\pgfqpoint{0.970883in}{1.562418in}}{\pgfqpoint{0.970883in}{1.570655in}}%
\pgfpathcurveto{\pgfqpoint{0.970883in}{1.578891in}}{\pgfqpoint{0.967611in}{1.586791in}}{\pgfqpoint{0.961787in}{1.592615in}}%
\pgfpathcurveto{\pgfqpoint{0.955963in}{1.598439in}}{\pgfqpoint{0.948063in}{1.601711in}}{\pgfqpoint{0.939827in}{1.601711in}}%
\pgfpathcurveto{\pgfqpoint{0.931591in}{1.601711in}}{\pgfqpoint{0.923691in}{1.598439in}}{\pgfqpoint{0.917867in}{1.592615in}}%
\pgfpathcurveto{\pgfqpoint{0.912043in}{1.586791in}}{\pgfqpoint{0.908770in}{1.578891in}}{\pgfqpoint{0.908770in}{1.570655in}}%
\pgfpathcurveto{\pgfqpoint{0.908770in}{1.562418in}}{\pgfqpoint{0.912043in}{1.554518in}}{\pgfqpoint{0.917867in}{1.548694in}}%
\pgfpathcurveto{\pgfqpoint{0.923691in}{1.542870in}}{\pgfqpoint{0.931591in}{1.539598in}}{\pgfqpoint{0.939827in}{1.539598in}}%
\pgfpathclose%
\pgfusepath{stroke,fill}%
\end{pgfscope}%
\begin{pgfscope}%
\pgfpathrectangle{\pgfqpoint{0.100000in}{0.220728in}}{\pgfqpoint{3.696000in}{3.696000in}}%
\pgfusepath{clip}%
\pgfsetbuttcap%
\pgfsetroundjoin%
\definecolor{currentfill}{rgb}{0.121569,0.466667,0.705882}%
\pgfsetfillcolor{currentfill}%
\pgfsetfillopacity{0.657589}%
\pgfsetlinewidth{1.003750pt}%
\definecolor{currentstroke}{rgb}{0.121569,0.466667,0.705882}%
\pgfsetstrokecolor{currentstroke}%
\pgfsetstrokeopacity{0.657589}%
\pgfsetdash{}{0pt}%
\pgfpathmoveto{\pgfqpoint{0.943353in}{1.537991in}}%
\pgfpathcurveto{\pgfqpoint{0.951589in}{1.537991in}}{\pgfqpoint{0.959489in}{1.541264in}}{\pgfqpoint{0.965313in}{1.547088in}}%
\pgfpathcurveto{\pgfqpoint{0.971137in}{1.552911in}}{\pgfqpoint{0.974409in}{1.560812in}}{\pgfqpoint{0.974409in}{1.569048in}}%
\pgfpathcurveto{\pgfqpoint{0.974409in}{1.577284in}}{\pgfqpoint{0.971137in}{1.585184in}}{\pgfqpoint{0.965313in}{1.591008in}}%
\pgfpathcurveto{\pgfqpoint{0.959489in}{1.596832in}}{\pgfqpoint{0.951589in}{1.600104in}}{\pgfqpoint{0.943353in}{1.600104in}}%
\pgfpathcurveto{\pgfqpoint{0.935116in}{1.600104in}}{\pgfqpoint{0.927216in}{1.596832in}}{\pgfqpoint{0.921392in}{1.591008in}}%
\pgfpathcurveto{\pgfqpoint{0.915568in}{1.585184in}}{\pgfqpoint{0.912296in}{1.577284in}}{\pgfqpoint{0.912296in}{1.569048in}}%
\pgfpathcurveto{\pgfqpoint{0.912296in}{1.560812in}}{\pgfqpoint{0.915568in}{1.552911in}}{\pgfqpoint{0.921392in}{1.547088in}}%
\pgfpathcurveto{\pgfqpoint{0.927216in}{1.541264in}}{\pgfqpoint{0.935116in}{1.537991in}}{\pgfqpoint{0.943353in}{1.537991in}}%
\pgfpathclose%
\pgfusepath{stroke,fill}%
\end{pgfscope}%
\begin{pgfscope}%
\pgfpathrectangle{\pgfqpoint{0.100000in}{0.220728in}}{\pgfqpoint{3.696000in}{3.696000in}}%
\pgfusepath{clip}%
\pgfsetbuttcap%
\pgfsetroundjoin%
\definecolor{currentfill}{rgb}{0.121569,0.466667,0.705882}%
\pgfsetfillcolor{currentfill}%
\pgfsetfillopacity{0.668090}%
\pgfsetlinewidth{1.003750pt}%
\definecolor{currentstroke}{rgb}{0.121569,0.466667,0.705882}%
\pgfsetstrokecolor{currentstroke}%
\pgfsetstrokeopacity{0.668090}%
\pgfsetdash{}{0pt}%
\pgfpathmoveto{\pgfqpoint{0.950665in}{1.539289in}}%
\pgfpathcurveto{\pgfqpoint{0.958901in}{1.539289in}}{\pgfqpoint{0.966801in}{1.542561in}}{\pgfqpoint{0.972625in}{1.548385in}}%
\pgfpathcurveto{\pgfqpoint{0.978449in}{1.554209in}}{\pgfqpoint{0.981721in}{1.562109in}}{\pgfqpoint{0.981721in}{1.570346in}}%
\pgfpathcurveto{\pgfqpoint{0.981721in}{1.578582in}}{\pgfqpoint{0.978449in}{1.586482in}}{\pgfqpoint{0.972625in}{1.592306in}}%
\pgfpathcurveto{\pgfqpoint{0.966801in}{1.598130in}}{\pgfqpoint{0.958901in}{1.601402in}}{\pgfqpoint{0.950665in}{1.601402in}}%
\pgfpathcurveto{\pgfqpoint{0.942428in}{1.601402in}}{\pgfqpoint{0.934528in}{1.598130in}}{\pgfqpoint{0.928704in}{1.592306in}}%
\pgfpathcurveto{\pgfqpoint{0.922880in}{1.586482in}}{\pgfqpoint{0.919608in}{1.578582in}}{\pgfqpoint{0.919608in}{1.570346in}}%
\pgfpathcurveto{\pgfqpoint{0.919608in}{1.562109in}}{\pgfqpoint{0.922880in}{1.554209in}}{\pgfqpoint{0.928704in}{1.548385in}}%
\pgfpathcurveto{\pgfqpoint{0.934528in}{1.542561in}}{\pgfqpoint{0.942428in}{1.539289in}}{\pgfqpoint{0.950665in}{1.539289in}}%
\pgfpathclose%
\pgfusepath{stroke,fill}%
\end{pgfscope}%
\begin{pgfscope}%
\pgfpathrectangle{\pgfqpoint{0.100000in}{0.220728in}}{\pgfqpoint{3.696000in}{3.696000in}}%
\pgfusepath{clip}%
\pgfsetbuttcap%
\pgfsetroundjoin%
\definecolor{currentfill}{rgb}{0.121569,0.466667,0.705882}%
\pgfsetfillcolor{currentfill}%
\pgfsetfillopacity{0.668614}%
\pgfsetlinewidth{1.003750pt}%
\definecolor{currentstroke}{rgb}{0.121569,0.466667,0.705882}%
\pgfsetstrokecolor{currentstroke}%
\pgfsetstrokeopacity{0.668614}%
\pgfsetdash{}{0pt}%
\pgfpathmoveto{\pgfqpoint{1.292798in}{1.210433in}}%
\pgfpathcurveto{\pgfqpoint{1.301035in}{1.210433in}}{\pgfqpoint{1.308935in}{1.213705in}}{\pgfqpoint{1.314759in}{1.219529in}}%
\pgfpathcurveto{\pgfqpoint{1.320582in}{1.225353in}}{\pgfqpoint{1.323855in}{1.233253in}}{\pgfqpoint{1.323855in}{1.241489in}}%
\pgfpathcurveto{\pgfqpoint{1.323855in}{1.249725in}}{\pgfqpoint{1.320582in}{1.257625in}}{\pgfqpoint{1.314759in}{1.263449in}}%
\pgfpathcurveto{\pgfqpoint{1.308935in}{1.269273in}}{\pgfqpoint{1.301035in}{1.272546in}}{\pgfqpoint{1.292798in}{1.272546in}}%
\pgfpathcurveto{\pgfqpoint{1.284562in}{1.272546in}}{\pgfqpoint{1.276662in}{1.269273in}}{\pgfqpoint{1.270838in}{1.263449in}}%
\pgfpathcurveto{\pgfqpoint{1.265014in}{1.257625in}}{\pgfqpoint{1.261742in}{1.249725in}}{\pgfqpoint{1.261742in}{1.241489in}}%
\pgfpathcurveto{\pgfqpoint{1.261742in}{1.233253in}}{\pgfqpoint{1.265014in}{1.225353in}}{\pgfqpoint{1.270838in}{1.219529in}}%
\pgfpathcurveto{\pgfqpoint{1.276662in}{1.213705in}}{\pgfqpoint{1.284562in}{1.210433in}}{\pgfqpoint{1.292798in}{1.210433in}}%
\pgfpathclose%
\pgfusepath{stroke,fill}%
\end{pgfscope}%
\begin{pgfscope}%
\pgfpathrectangle{\pgfqpoint{0.100000in}{0.220728in}}{\pgfqpoint{3.696000in}{3.696000in}}%
\pgfusepath{clip}%
\pgfsetbuttcap%
\pgfsetroundjoin%
\definecolor{currentfill}{rgb}{0.121569,0.466667,0.705882}%
\pgfsetfillcolor{currentfill}%
\pgfsetfillopacity{0.672345}%
\pgfsetlinewidth{1.003750pt}%
\definecolor{currentstroke}{rgb}{0.121569,0.466667,0.705882}%
\pgfsetstrokecolor{currentstroke}%
\pgfsetstrokeopacity{0.672345}%
\pgfsetdash{}{0pt}%
\pgfpathmoveto{\pgfqpoint{1.246074in}{1.296528in}}%
\pgfpathcurveto{\pgfqpoint{1.254310in}{1.296528in}}{\pgfqpoint{1.262210in}{1.299801in}}{\pgfqpoint{1.268034in}{1.305625in}}%
\pgfpathcurveto{\pgfqpoint{1.273858in}{1.311449in}}{\pgfqpoint{1.277130in}{1.319349in}}{\pgfqpoint{1.277130in}{1.327585in}}%
\pgfpathcurveto{\pgfqpoint{1.277130in}{1.335821in}}{\pgfqpoint{1.273858in}{1.343721in}}{\pgfqpoint{1.268034in}{1.349545in}}%
\pgfpathcurveto{\pgfqpoint{1.262210in}{1.355369in}}{\pgfqpoint{1.254310in}{1.358641in}}{\pgfqpoint{1.246074in}{1.358641in}}%
\pgfpathcurveto{\pgfqpoint{1.237838in}{1.358641in}}{\pgfqpoint{1.229938in}{1.355369in}}{\pgfqpoint{1.224114in}{1.349545in}}%
\pgfpathcurveto{\pgfqpoint{1.218290in}{1.343721in}}{\pgfqpoint{1.215017in}{1.335821in}}{\pgfqpoint{1.215017in}{1.327585in}}%
\pgfpathcurveto{\pgfqpoint{1.215017in}{1.319349in}}{\pgfqpoint{1.218290in}{1.311449in}}{\pgfqpoint{1.224114in}{1.305625in}}%
\pgfpathcurveto{\pgfqpoint{1.229938in}{1.299801in}}{\pgfqpoint{1.237838in}{1.296528in}}{\pgfqpoint{1.246074in}{1.296528in}}%
\pgfpathclose%
\pgfusepath{stroke,fill}%
\end{pgfscope}%
\begin{pgfscope}%
\pgfpathrectangle{\pgfqpoint{0.100000in}{0.220728in}}{\pgfqpoint{3.696000in}{3.696000in}}%
\pgfusepath{clip}%
\pgfsetbuttcap%
\pgfsetroundjoin%
\definecolor{currentfill}{rgb}{0.121569,0.466667,0.705882}%
\pgfsetfillcolor{currentfill}%
\pgfsetfillopacity{0.674773}%
\pgfsetlinewidth{1.003750pt}%
\definecolor{currentstroke}{rgb}{0.121569,0.466667,0.705882}%
\pgfsetstrokecolor{currentstroke}%
\pgfsetstrokeopacity{0.674773}%
\pgfsetdash{}{0pt}%
\pgfpathmoveto{\pgfqpoint{0.955844in}{1.539594in}}%
\pgfpathcurveto{\pgfqpoint{0.964081in}{1.539594in}}{\pgfqpoint{0.971981in}{1.542866in}}{\pgfqpoint{0.977804in}{1.548690in}}%
\pgfpathcurveto{\pgfqpoint{0.983628in}{1.554514in}}{\pgfqpoint{0.986901in}{1.562414in}}{\pgfqpoint{0.986901in}{1.570650in}}%
\pgfpathcurveto{\pgfqpoint{0.986901in}{1.578887in}}{\pgfqpoint{0.983628in}{1.586787in}}{\pgfqpoint{0.977804in}{1.592611in}}%
\pgfpathcurveto{\pgfqpoint{0.971981in}{1.598435in}}{\pgfqpoint{0.964081in}{1.601707in}}{\pgfqpoint{0.955844in}{1.601707in}}%
\pgfpathcurveto{\pgfqpoint{0.947608in}{1.601707in}}{\pgfqpoint{0.939708in}{1.598435in}}{\pgfqpoint{0.933884in}{1.592611in}}%
\pgfpathcurveto{\pgfqpoint{0.928060in}{1.586787in}}{\pgfqpoint{0.924788in}{1.578887in}}{\pgfqpoint{0.924788in}{1.570650in}}%
\pgfpathcurveto{\pgfqpoint{0.924788in}{1.562414in}}{\pgfqpoint{0.928060in}{1.554514in}}{\pgfqpoint{0.933884in}{1.548690in}}%
\pgfpathcurveto{\pgfqpoint{0.939708in}{1.542866in}}{\pgfqpoint{0.947608in}{1.539594in}}{\pgfqpoint{0.955844in}{1.539594in}}%
\pgfpathclose%
\pgfusepath{stroke,fill}%
\end{pgfscope}%
\begin{pgfscope}%
\pgfpathrectangle{\pgfqpoint{0.100000in}{0.220728in}}{\pgfqpoint{3.696000in}{3.696000in}}%
\pgfusepath{clip}%
\pgfsetbuttcap%
\pgfsetroundjoin%
\definecolor{currentfill}{rgb}{0.121569,0.466667,0.705882}%
\pgfsetfillcolor{currentfill}%
\pgfsetfillopacity{0.676166}%
\pgfsetlinewidth{1.003750pt}%
\definecolor{currentstroke}{rgb}{0.121569,0.466667,0.705882}%
\pgfsetstrokecolor{currentstroke}%
\pgfsetstrokeopacity{0.676166}%
\pgfsetdash{}{0pt}%
\pgfpathmoveto{\pgfqpoint{1.047843in}{1.494947in}}%
\pgfpathcurveto{\pgfqpoint{1.056079in}{1.494947in}}{\pgfqpoint{1.063979in}{1.498219in}}{\pgfqpoint{1.069803in}{1.504043in}}%
\pgfpathcurveto{\pgfqpoint{1.075627in}{1.509867in}}{\pgfqpoint{1.078899in}{1.517767in}}{\pgfqpoint{1.078899in}{1.526003in}}%
\pgfpathcurveto{\pgfqpoint{1.078899in}{1.534240in}}{\pgfqpoint{1.075627in}{1.542140in}}{\pgfqpoint{1.069803in}{1.547964in}}%
\pgfpathcurveto{\pgfqpoint{1.063979in}{1.553788in}}{\pgfqpoint{1.056079in}{1.557060in}}{\pgfqpoint{1.047843in}{1.557060in}}%
\pgfpathcurveto{\pgfqpoint{1.039607in}{1.557060in}}{\pgfqpoint{1.031707in}{1.553788in}}{\pgfqpoint{1.025883in}{1.547964in}}%
\pgfpathcurveto{\pgfqpoint{1.020059in}{1.542140in}}{\pgfqpoint{1.016786in}{1.534240in}}{\pgfqpoint{1.016786in}{1.526003in}}%
\pgfpathcurveto{\pgfqpoint{1.016786in}{1.517767in}}{\pgfqpoint{1.020059in}{1.509867in}}{\pgfqpoint{1.025883in}{1.504043in}}%
\pgfpathcurveto{\pgfqpoint{1.031707in}{1.498219in}}{\pgfqpoint{1.039607in}{1.494947in}}{\pgfqpoint{1.047843in}{1.494947in}}%
\pgfpathclose%
\pgfusepath{stroke,fill}%
\end{pgfscope}%
\begin{pgfscope}%
\pgfpathrectangle{\pgfqpoint{0.100000in}{0.220728in}}{\pgfqpoint{3.696000in}{3.696000in}}%
\pgfusepath{clip}%
\pgfsetbuttcap%
\pgfsetroundjoin%
\definecolor{currentfill}{rgb}{0.121569,0.466667,0.705882}%
\pgfsetfillcolor{currentfill}%
\pgfsetfillopacity{0.676922}%
\pgfsetlinewidth{1.003750pt}%
\definecolor{currentstroke}{rgb}{0.121569,0.466667,0.705882}%
\pgfsetstrokecolor{currentstroke}%
\pgfsetstrokeopacity{0.676922}%
\pgfsetdash{}{0pt}%
\pgfpathmoveto{\pgfqpoint{1.049771in}{1.492297in}}%
\pgfpathcurveto{\pgfqpoint{1.058007in}{1.492297in}}{\pgfqpoint{1.065907in}{1.495570in}}{\pgfqpoint{1.071731in}{1.501394in}}%
\pgfpathcurveto{\pgfqpoint{1.077555in}{1.507218in}}{\pgfqpoint{1.080827in}{1.515118in}}{\pgfqpoint{1.080827in}{1.523354in}}%
\pgfpathcurveto{\pgfqpoint{1.080827in}{1.531590in}}{\pgfqpoint{1.077555in}{1.539490in}}{\pgfqpoint{1.071731in}{1.545314in}}%
\pgfpathcurveto{\pgfqpoint{1.065907in}{1.551138in}}{\pgfqpoint{1.058007in}{1.554410in}}{\pgfqpoint{1.049771in}{1.554410in}}%
\pgfpathcurveto{\pgfqpoint{1.041534in}{1.554410in}}{\pgfqpoint{1.033634in}{1.551138in}}{\pgfqpoint{1.027810in}{1.545314in}}%
\pgfpathcurveto{\pgfqpoint{1.021987in}{1.539490in}}{\pgfqpoint{1.018714in}{1.531590in}}{\pgfqpoint{1.018714in}{1.523354in}}%
\pgfpathcurveto{\pgfqpoint{1.018714in}{1.515118in}}{\pgfqpoint{1.021987in}{1.507218in}}{\pgfqpoint{1.027810in}{1.501394in}}%
\pgfpathcurveto{\pgfqpoint{1.033634in}{1.495570in}}{\pgfqpoint{1.041534in}{1.492297in}}{\pgfqpoint{1.049771in}{1.492297in}}%
\pgfpathclose%
\pgfusepath{stroke,fill}%
\end{pgfscope}%
\begin{pgfscope}%
\pgfpathrectangle{\pgfqpoint{0.100000in}{0.220728in}}{\pgfqpoint{3.696000in}{3.696000in}}%
\pgfusepath{clip}%
\pgfsetbuttcap%
\pgfsetroundjoin%
\definecolor{currentfill}{rgb}{0.121569,0.466667,0.705882}%
\pgfsetfillcolor{currentfill}%
\pgfsetfillopacity{0.679055}%
\pgfsetlinewidth{1.003750pt}%
\definecolor{currentstroke}{rgb}{0.121569,0.466667,0.705882}%
\pgfsetstrokecolor{currentstroke}%
\pgfsetstrokeopacity{0.679055}%
\pgfsetdash{}{0pt}%
\pgfpathmoveto{\pgfqpoint{1.269508in}{1.261137in}}%
\pgfpathcurveto{\pgfqpoint{1.277745in}{1.261137in}}{\pgfqpoint{1.285645in}{1.264410in}}{\pgfqpoint{1.291469in}{1.270234in}}%
\pgfpathcurveto{\pgfqpoint{1.297293in}{1.276058in}}{\pgfqpoint{1.300565in}{1.283958in}}{\pgfqpoint{1.300565in}{1.292194in}}%
\pgfpathcurveto{\pgfqpoint{1.300565in}{1.300430in}}{\pgfqpoint{1.297293in}{1.308330in}}{\pgfqpoint{1.291469in}{1.314154in}}%
\pgfpathcurveto{\pgfqpoint{1.285645in}{1.319978in}}{\pgfqpoint{1.277745in}{1.323250in}}{\pgfqpoint{1.269508in}{1.323250in}}%
\pgfpathcurveto{\pgfqpoint{1.261272in}{1.323250in}}{\pgfqpoint{1.253372in}{1.319978in}}{\pgfqpoint{1.247548in}{1.314154in}}%
\pgfpathcurveto{\pgfqpoint{1.241724in}{1.308330in}}{\pgfqpoint{1.238452in}{1.300430in}}{\pgfqpoint{1.238452in}{1.292194in}}%
\pgfpathcurveto{\pgfqpoint{1.238452in}{1.283958in}}{\pgfqpoint{1.241724in}{1.276058in}}{\pgfqpoint{1.247548in}{1.270234in}}%
\pgfpathcurveto{\pgfqpoint{1.253372in}{1.264410in}}{\pgfqpoint{1.261272in}{1.261137in}}{\pgfqpoint{1.269508in}{1.261137in}}%
\pgfpathclose%
\pgfusepath{stroke,fill}%
\end{pgfscope}%
\begin{pgfscope}%
\pgfpathrectangle{\pgfqpoint{0.100000in}{0.220728in}}{\pgfqpoint{3.696000in}{3.696000in}}%
\pgfusepath{clip}%
\pgfsetbuttcap%
\pgfsetroundjoin%
\definecolor{currentfill}{rgb}{0.121569,0.466667,0.705882}%
\pgfsetfillcolor{currentfill}%
\pgfsetfillopacity{0.680873}%
\pgfsetlinewidth{1.003750pt}%
\definecolor{currentstroke}{rgb}{0.121569,0.466667,0.705882}%
\pgfsetstrokecolor{currentstroke}%
\pgfsetstrokeopacity{0.680873}%
\pgfsetdash{}{0pt}%
\pgfpathmoveto{\pgfqpoint{1.062483in}{1.488926in}}%
\pgfpathcurveto{\pgfqpoint{1.070719in}{1.488926in}}{\pgfqpoint{1.078619in}{1.492199in}}{\pgfqpoint{1.084443in}{1.498023in}}%
\pgfpathcurveto{\pgfqpoint{1.090267in}{1.503847in}}{\pgfqpoint{1.093540in}{1.511747in}}{\pgfqpoint{1.093540in}{1.519983in}}%
\pgfpathcurveto{\pgfqpoint{1.093540in}{1.528219in}}{\pgfqpoint{1.090267in}{1.536119in}}{\pgfqpoint{1.084443in}{1.541943in}}%
\pgfpathcurveto{\pgfqpoint{1.078619in}{1.547767in}}{\pgfqpoint{1.070719in}{1.551039in}}{\pgfqpoint{1.062483in}{1.551039in}}%
\pgfpathcurveto{\pgfqpoint{1.054247in}{1.551039in}}{\pgfqpoint{1.046347in}{1.547767in}}{\pgfqpoint{1.040523in}{1.541943in}}%
\pgfpathcurveto{\pgfqpoint{1.034699in}{1.536119in}}{\pgfqpoint{1.031427in}{1.528219in}}{\pgfqpoint{1.031427in}{1.519983in}}%
\pgfpathcurveto{\pgfqpoint{1.031427in}{1.511747in}}{\pgfqpoint{1.034699in}{1.503847in}}{\pgfqpoint{1.040523in}{1.498023in}}%
\pgfpathcurveto{\pgfqpoint{1.046347in}{1.492199in}}{\pgfqpoint{1.054247in}{1.488926in}}{\pgfqpoint{1.062483in}{1.488926in}}%
\pgfpathclose%
\pgfusepath{stroke,fill}%
\end{pgfscope}%
\begin{pgfscope}%
\pgfpathrectangle{\pgfqpoint{0.100000in}{0.220728in}}{\pgfqpoint{3.696000in}{3.696000in}}%
\pgfusepath{clip}%
\pgfsetbuttcap%
\pgfsetroundjoin%
\definecolor{currentfill}{rgb}{0.121569,0.466667,0.705882}%
\pgfsetfillcolor{currentfill}%
\pgfsetfillopacity{0.681720}%
\pgfsetlinewidth{1.003750pt}%
\definecolor{currentstroke}{rgb}{0.121569,0.466667,0.705882}%
\pgfsetstrokecolor{currentstroke}%
\pgfsetstrokeopacity{0.681720}%
\pgfsetdash{}{0pt}%
\pgfpathmoveto{\pgfqpoint{0.984836in}{1.517148in}}%
\pgfpathcurveto{\pgfqpoint{0.993072in}{1.517148in}}{\pgfqpoint{1.000972in}{1.520421in}}{\pgfqpoint{1.006796in}{1.526245in}}%
\pgfpathcurveto{\pgfqpoint{1.012620in}{1.532068in}}{\pgfqpoint{1.015892in}{1.539969in}}{\pgfqpoint{1.015892in}{1.548205in}}%
\pgfpathcurveto{\pgfqpoint{1.015892in}{1.556441in}}{\pgfqpoint{1.012620in}{1.564341in}}{\pgfqpoint{1.006796in}{1.570165in}}%
\pgfpathcurveto{\pgfqpoint{1.000972in}{1.575989in}}{\pgfqpoint{0.993072in}{1.579261in}}{\pgfqpoint{0.984836in}{1.579261in}}%
\pgfpathcurveto{\pgfqpoint{0.976599in}{1.579261in}}{\pgfqpoint{0.968699in}{1.575989in}}{\pgfqpoint{0.962875in}{1.570165in}}%
\pgfpathcurveto{\pgfqpoint{0.957051in}{1.564341in}}{\pgfqpoint{0.953779in}{1.556441in}}{\pgfqpoint{0.953779in}{1.548205in}}%
\pgfpathcurveto{\pgfqpoint{0.953779in}{1.539969in}}{\pgfqpoint{0.957051in}{1.532068in}}{\pgfqpoint{0.962875in}{1.526245in}}%
\pgfpathcurveto{\pgfqpoint{0.968699in}{1.520421in}}{\pgfqpoint{0.976599in}{1.517148in}}{\pgfqpoint{0.984836in}{1.517148in}}%
\pgfpathclose%
\pgfusepath{stroke,fill}%
\end{pgfscope}%
\begin{pgfscope}%
\pgfpathrectangle{\pgfqpoint{0.100000in}{0.220728in}}{\pgfqpoint{3.696000in}{3.696000in}}%
\pgfusepath{clip}%
\pgfsetbuttcap%
\pgfsetroundjoin%
\definecolor{currentfill}{rgb}{0.121569,0.466667,0.705882}%
\pgfsetfillcolor{currentfill}%
\pgfsetfillopacity{0.683778}%
\pgfsetlinewidth{1.003750pt}%
\definecolor{currentstroke}{rgb}{0.121569,0.466667,0.705882}%
\pgfsetstrokecolor{currentstroke}%
\pgfsetstrokeopacity{0.683778}%
\pgfsetdash{}{0pt}%
\pgfpathmoveto{\pgfqpoint{1.053712in}{1.495391in}}%
\pgfpathcurveto{\pgfqpoint{1.061948in}{1.495391in}}{\pgfqpoint{1.069848in}{1.498664in}}{\pgfqpoint{1.075672in}{1.504488in}}%
\pgfpathcurveto{\pgfqpoint{1.081496in}{1.510312in}}{\pgfqpoint{1.084768in}{1.518212in}}{\pgfqpoint{1.084768in}{1.526448in}}%
\pgfpathcurveto{\pgfqpoint{1.084768in}{1.534684in}}{\pgfqpoint{1.081496in}{1.542584in}}{\pgfqpoint{1.075672in}{1.548408in}}%
\pgfpathcurveto{\pgfqpoint{1.069848in}{1.554232in}}{\pgfqpoint{1.061948in}{1.557504in}}{\pgfqpoint{1.053712in}{1.557504in}}%
\pgfpathcurveto{\pgfqpoint{1.045475in}{1.557504in}}{\pgfqpoint{1.037575in}{1.554232in}}{\pgfqpoint{1.031751in}{1.548408in}}%
\pgfpathcurveto{\pgfqpoint{1.025927in}{1.542584in}}{\pgfqpoint{1.022655in}{1.534684in}}{\pgfqpoint{1.022655in}{1.526448in}}%
\pgfpathcurveto{\pgfqpoint{1.022655in}{1.518212in}}{\pgfqpoint{1.025927in}{1.510312in}}{\pgfqpoint{1.031751in}{1.504488in}}%
\pgfpathcurveto{\pgfqpoint{1.037575in}{1.498664in}}{\pgfqpoint{1.045475in}{1.495391in}}{\pgfqpoint{1.053712in}{1.495391in}}%
\pgfpathclose%
\pgfusepath{stroke,fill}%
\end{pgfscope}%
\begin{pgfscope}%
\pgfpathrectangle{\pgfqpoint{0.100000in}{0.220728in}}{\pgfqpoint{3.696000in}{3.696000in}}%
\pgfusepath{clip}%
\pgfsetbuttcap%
\pgfsetroundjoin%
\definecolor{currentfill}{rgb}{0.121569,0.466667,0.705882}%
\pgfsetfillcolor{currentfill}%
\pgfsetfillopacity{0.688248}%
\pgfsetlinewidth{1.003750pt}%
\definecolor{currentstroke}{rgb}{0.121569,0.466667,0.705882}%
\pgfsetstrokecolor{currentstroke}%
\pgfsetstrokeopacity{0.688248}%
\pgfsetdash{}{0pt}%
\pgfpathmoveto{\pgfqpoint{1.336871in}{1.178516in}}%
\pgfpathcurveto{\pgfqpoint{1.345107in}{1.178516in}}{\pgfqpoint{1.353007in}{1.181789in}}{\pgfqpoint{1.358831in}{1.187613in}}%
\pgfpathcurveto{\pgfqpoint{1.364655in}{1.193437in}}{\pgfqpoint{1.367927in}{1.201337in}}{\pgfqpoint{1.367927in}{1.209573in}}%
\pgfpathcurveto{\pgfqpoint{1.367927in}{1.217809in}}{\pgfqpoint{1.364655in}{1.225709in}}{\pgfqpoint{1.358831in}{1.231533in}}%
\pgfpathcurveto{\pgfqpoint{1.353007in}{1.237357in}}{\pgfqpoint{1.345107in}{1.240629in}}{\pgfqpoint{1.336871in}{1.240629in}}%
\pgfpathcurveto{\pgfqpoint{1.328634in}{1.240629in}}{\pgfqpoint{1.320734in}{1.237357in}}{\pgfqpoint{1.314910in}{1.231533in}}%
\pgfpathcurveto{\pgfqpoint{1.309086in}{1.225709in}}{\pgfqpoint{1.305814in}{1.217809in}}{\pgfqpoint{1.305814in}{1.209573in}}%
\pgfpathcurveto{\pgfqpoint{1.305814in}{1.201337in}}{\pgfqpoint{1.309086in}{1.193437in}}{\pgfqpoint{1.314910in}{1.187613in}}%
\pgfpathcurveto{\pgfqpoint{1.320734in}{1.181789in}}{\pgfqpoint{1.328634in}{1.178516in}}{\pgfqpoint{1.336871in}{1.178516in}}%
\pgfpathclose%
\pgfusepath{stroke,fill}%
\end{pgfscope}%
\begin{pgfscope}%
\pgfpathrectangle{\pgfqpoint{0.100000in}{0.220728in}}{\pgfqpoint{3.696000in}{3.696000in}}%
\pgfusepath{clip}%
\pgfsetbuttcap%
\pgfsetroundjoin%
\definecolor{currentfill}{rgb}{0.121569,0.466667,0.705882}%
\pgfsetfillcolor{currentfill}%
\pgfsetfillopacity{0.689618}%
\pgfsetlinewidth{1.003750pt}%
\definecolor{currentstroke}{rgb}{0.121569,0.466667,0.705882}%
\pgfsetstrokecolor{currentstroke}%
\pgfsetstrokeopacity{0.689618}%
\pgfsetdash{}{0pt}%
\pgfpathmoveto{\pgfqpoint{1.367773in}{1.141355in}}%
\pgfpathcurveto{\pgfqpoint{1.376010in}{1.141355in}}{\pgfqpoint{1.383910in}{1.144627in}}{\pgfqpoint{1.389733in}{1.150451in}}%
\pgfpathcurveto{\pgfqpoint{1.395557in}{1.156275in}}{\pgfqpoint{1.398830in}{1.164175in}}{\pgfqpoint{1.398830in}{1.172411in}}%
\pgfpathcurveto{\pgfqpoint{1.398830in}{1.180647in}}{\pgfqpoint{1.395557in}{1.188547in}}{\pgfqpoint{1.389733in}{1.194371in}}%
\pgfpathcurveto{\pgfqpoint{1.383910in}{1.200195in}}{\pgfqpoint{1.376010in}{1.203468in}}{\pgfqpoint{1.367773in}{1.203468in}}%
\pgfpathcurveto{\pgfqpoint{1.359537in}{1.203468in}}{\pgfqpoint{1.351637in}{1.200195in}}{\pgfqpoint{1.345813in}{1.194371in}}%
\pgfpathcurveto{\pgfqpoint{1.339989in}{1.188547in}}{\pgfqpoint{1.336717in}{1.180647in}}{\pgfqpoint{1.336717in}{1.172411in}}%
\pgfpathcurveto{\pgfqpoint{1.336717in}{1.164175in}}{\pgfqpoint{1.339989in}{1.156275in}}{\pgfqpoint{1.345813in}{1.150451in}}%
\pgfpathcurveto{\pgfqpoint{1.351637in}{1.144627in}}{\pgfqpoint{1.359537in}{1.141355in}}{\pgfqpoint{1.367773in}{1.141355in}}%
\pgfpathclose%
\pgfusepath{stroke,fill}%
\end{pgfscope}%
\begin{pgfscope}%
\pgfpathrectangle{\pgfqpoint{0.100000in}{0.220728in}}{\pgfqpoint{3.696000in}{3.696000in}}%
\pgfusepath{clip}%
\pgfsetbuttcap%
\pgfsetroundjoin%
\definecolor{currentfill}{rgb}{0.121569,0.466667,0.705882}%
\pgfsetfillcolor{currentfill}%
\pgfsetfillopacity{0.690522}%
\pgfsetlinewidth{1.003750pt}%
\definecolor{currentstroke}{rgb}{0.121569,0.466667,0.705882}%
\pgfsetstrokecolor{currentstroke}%
\pgfsetstrokeopacity{0.690522}%
\pgfsetdash{}{0pt}%
\pgfpathmoveto{\pgfqpoint{0.965413in}{1.544784in}}%
\pgfpathcurveto{\pgfqpoint{0.973649in}{1.544784in}}{\pgfqpoint{0.981549in}{1.548057in}}{\pgfqpoint{0.987373in}{1.553880in}}%
\pgfpathcurveto{\pgfqpoint{0.993197in}{1.559704in}}{\pgfqpoint{0.996469in}{1.567604in}}{\pgfqpoint{0.996469in}{1.575841in}}%
\pgfpathcurveto{\pgfqpoint{0.996469in}{1.584077in}}{\pgfqpoint{0.993197in}{1.591977in}}{\pgfqpoint{0.987373in}{1.597801in}}%
\pgfpathcurveto{\pgfqpoint{0.981549in}{1.603625in}}{\pgfqpoint{0.973649in}{1.606897in}}{\pgfqpoint{0.965413in}{1.606897in}}%
\pgfpathcurveto{\pgfqpoint{0.957176in}{1.606897in}}{\pgfqpoint{0.949276in}{1.603625in}}{\pgfqpoint{0.943452in}{1.597801in}}%
\pgfpathcurveto{\pgfqpoint{0.937628in}{1.591977in}}{\pgfqpoint{0.934356in}{1.584077in}}{\pgfqpoint{0.934356in}{1.575841in}}%
\pgfpathcurveto{\pgfqpoint{0.934356in}{1.567604in}}{\pgfqpoint{0.937628in}{1.559704in}}{\pgfqpoint{0.943452in}{1.553880in}}%
\pgfpathcurveto{\pgfqpoint{0.949276in}{1.548057in}}{\pgfqpoint{0.957176in}{1.544784in}}{\pgfqpoint{0.965413in}{1.544784in}}%
\pgfpathclose%
\pgfusepath{stroke,fill}%
\end{pgfscope}%
\begin{pgfscope}%
\pgfpathrectangle{\pgfqpoint{0.100000in}{0.220728in}}{\pgfqpoint{3.696000in}{3.696000in}}%
\pgfusepath{clip}%
\pgfsetbuttcap%
\pgfsetroundjoin%
\definecolor{currentfill}{rgb}{0.121569,0.466667,0.705882}%
\pgfsetfillcolor{currentfill}%
\pgfsetfillopacity{0.691217}%
\pgfsetlinewidth{1.003750pt}%
\definecolor{currentstroke}{rgb}{0.121569,0.466667,0.705882}%
\pgfsetstrokecolor{currentstroke}%
\pgfsetstrokeopacity{0.691217}%
\pgfsetdash{}{0pt}%
\pgfpathmoveto{\pgfqpoint{1.019497in}{1.517266in}}%
\pgfpathcurveto{\pgfqpoint{1.027733in}{1.517266in}}{\pgfqpoint{1.035634in}{1.520538in}}{\pgfqpoint{1.041457in}{1.526362in}}%
\pgfpathcurveto{\pgfqpoint{1.047281in}{1.532186in}}{\pgfqpoint{1.050554in}{1.540086in}}{\pgfqpoint{1.050554in}{1.548322in}}%
\pgfpathcurveto{\pgfqpoint{1.050554in}{1.556558in}}{\pgfqpoint{1.047281in}{1.564458in}}{\pgfqpoint{1.041457in}{1.570282in}}%
\pgfpathcurveto{\pgfqpoint{1.035634in}{1.576106in}}{\pgfqpoint{1.027733in}{1.579379in}}{\pgfqpoint{1.019497in}{1.579379in}}%
\pgfpathcurveto{\pgfqpoint{1.011261in}{1.579379in}}{\pgfqpoint{1.003361in}{1.576106in}}{\pgfqpoint{0.997537in}{1.570282in}}%
\pgfpathcurveto{\pgfqpoint{0.991713in}{1.564458in}}{\pgfqpoint{0.988441in}{1.556558in}}{\pgfqpoint{0.988441in}{1.548322in}}%
\pgfpathcurveto{\pgfqpoint{0.988441in}{1.540086in}}{\pgfqpoint{0.991713in}{1.532186in}}{\pgfqpoint{0.997537in}{1.526362in}}%
\pgfpathcurveto{\pgfqpoint{1.003361in}{1.520538in}}{\pgfqpoint{1.011261in}{1.517266in}}{\pgfqpoint{1.019497in}{1.517266in}}%
\pgfpathclose%
\pgfusepath{stroke,fill}%
\end{pgfscope}%
\begin{pgfscope}%
\pgfpathrectangle{\pgfqpoint{0.100000in}{0.220728in}}{\pgfqpoint{3.696000in}{3.696000in}}%
\pgfusepath{clip}%
\pgfsetbuttcap%
\pgfsetroundjoin%
\definecolor{currentfill}{rgb}{0.121569,0.466667,0.705882}%
\pgfsetfillcolor{currentfill}%
\pgfsetfillopacity{0.692533}%
\pgfsetlinewidth{1.003750pt}%
\definecolor{currentstroke}{rgb}{0.121569,0.466667,0.705882}%
\pgfsetstrokecolor{currentstroke}%
\pgfsetstrokeopacity{0.692533}%
\pgfsetdash{}{0pt}%
\pgfpathmoveto{\pgfqpoint{1.312961in}{1.195842in}}%
\pgfpathcurveto{\pgfqpoint{1.321197in}{1.195842in}}{\pgfqpoint{1.329097in}{1.199115in}}{\pgfqpoint{1.334921in}{1.204938in}}%
\pgfpathcurveto{\pgfqpoint{1.340745in}{1.210762in}}{\pgfqpoint{1.344017in}{1.218662in}}{\pgfqpoint{1.344017in}{1.226899in}}%
\pgfpathcurveto{\pgfqpoint{1.344017in}{1.235135in}}{\pgfqpoint{1.340745in}{1.243035in}}{\pgfqpoint{1.334921in}{1.248859in}}%
\pgfpathcurveto{\pgfqpoint{1.329097in}{1.254683in}}{\pgfqpoint{1.321197in}{1.257955in}}{\pgfqpoint{1.312961in}{1.257955in}}%
\pgfpathcurveto{\pgfqpoint{1.304725in}{1.257955in}}{\pgfqpoint{1.296825in}{1.254683in}}{\pgfqpoint{1.291001in}{1.248859in}}%
\pgfpathcurveto{\pgfqpoint{1.285177in}{1.243035in}}{\pgfqpoint{1.281904in}{1.235135in}}{\pgfqpoint{1.281904in}{1.226899in}}%
\pgfpathcurveto{\pgfqpoint{1.281904in}{1.218662in}}{\pgfqpoint{1.285177in}{1.210762in}}{\pgfqpoint{1.291001in}{1.204938in}}%
\pgfpathcurveto{\pgfqpoint{1.296825in}{1.199115in}}{\pgfqpoint{1.304725in}{1.195842in}}{\pgfqpoint{1.312961in}{1.195842in}}%
\pgfpathclose%
\pgfusepath{stroke,fill}%
\end{pgfscope}%
\begin{pgfscope}%
\pgfpathrectangle{\pgfqpoint{0.100000in}{0.220728in}}{\pgfqpoint{3.696000in}{3.696000in}}%
\pgfusepath{clip}%
\pgfsetbuttcap%
\pgfsetroundjoin%
\definecolor{currentfill}{rgb}{0.121569,0.466667,0.705882}%
\pgfsetfillcolor{currentfill}%
\pgfsetfillopacity{0.694221}%
\pgfsetlinewidth{1.003750pt}%
\definecolor{currentstroke}{rgb}{0.121569,0.466667,0.705882}%
\pgfsetstrokecolor{currentstroke}%
\pgfsetstrokeopacity{0.694221}%
\pgfsetdash{}{0pt}%
\pgfpathmoveto{\pgfqpoint{1.351518in}{1.158057in}}%
\pgfpathcurveto{\pgfqpoint{1.359754in}{1.158057in}}{\pgfqpoint{1.367654in}{1.161329in}}{\pgfqpoint{1.373478in}{1.167153in}}%
\pgfpathcurveto{\pgfqpoint{1.379302in}{1.172977in}}{\pgfqpoint{1.382575in}{1.180877in}}{\pgfqpoint{1.382575in}{1.189113in}}%
\pgfpathcurveto{\pgfqpoint{1.382575in}{1.197349in}}{\pgfqpoint{1.379302in}{1.205250in}}{\pgfqpoint{1.373478in}{1.211073in}}%
\pgfpathcurveto{\pgfqpoint{1.367654in}{1.216897in}}{\pgfqpoint{1.359754in}{1.220170in}}{\pgfqpoint{1.351518in}{1.220170in}}%
\pgfpathcurveto{\pgfqpoint{1.343282in}{1.220170in}}{\pgfqpoint{1.335382in}{1.216897in}}{\pgfqpoint{1.329558in}{1.211073in}}%
\pgfpathcurveto{\pgfqpoint{1.323734in}{1.205250in}}{\pgfqpoint{1.320462in}{1.197349in}}{\pgfqpoint{1.320462in}{1.189113in}}%
\pgfpathcurveto{\pgfqpoint{1.320462in}{1.180877in}}{\pgfqpoint{1.323734in}{1.172977in}}{\pgfqpoint{1.329558in}{1.167153in}}%
\pgfpathcurveto{\pgfqpoint{1.335382in}{1.161329in}}{\pgfqpoint{1.343282in}{1.158057in}}{\pgfqpoint{1.351518in}{1.158057in}}%
\pgfpathclose%
\pgfusepath{stroke,fill}%
\end{pgfscope}%
\begin{pgfscope}%
\pgfpathrectangle{\pgfqpoint{0.100000in}{0.220728in}}{\pgfqpoint{3.696000in}{3.696000in}}%
\pgfusepath{clip}%
\pgfsetbuttcap%
\pgfsetroundjoin%
\definecolor{currentfill}{rgb}{0.121569,0.466667,0.705882}%
\pgfsetfillcolor{currentfill}%
\pgfsetfillopacity{0.694307}%
\pgfsetlinewidth{1.003750pt}%
\definecolor{currentstroke}{rgb}{0.121569,0.466667,0.705882}%
\pgfsetstrokecolor{currentstroke}%
\pgfsetstrokeopacity{0.694307}%
\pgfsetdash{}{0pt}%
\pgfpathmoveto{\pgfqpoint{1.098595in}{1.455972in}}%
\pgfpathcurveto{\pgfqpoint{1.106831in}{1.455972in}}{\pgfqpoint{1.114732in}{1.459245in}}{\pgfqpoint{1.120555in}{1.465069in}}%
\pgfpathcurveto{\pgfqpoint{1.126379in}{1.470893in}}{\pgfqpoint{1.129652in}{1.478793in}}{\pgfqpoint{1.129652in}{1.487029in}}%
\pgfpathcurveto{\pgfqpoint{1.129652in}{1.495265in}}{\pgfqpoint{1.126379in}{1.503165in}}{\pgfqpoint{1.120555in}{1.508989in}}%
\pgfpathcurveto{\pgfqpoint{1.114732in}{1.514813in}}{\pgfqpoint{1.106831in}{1.518085in}}{\pgfqpoint{1.098595in}{1.518085in}}%
\pgfpathcurveto{\pgfqpoint{1.090359in}{1.518085in}}{\pgfqpoint{1.082459in}{1.514813in}}{\pgfqpoint{1.076635in}{1.508989in}}%
\pgfpathcurveto{\pgfqpoint{1.070811in}{1.503165in}}{\pgfqpoint{1.067539in}{1.495265in}}{\pgfqpoint{1.067539in}{1.487029in}}%
\pgfpathcurveto{\pgfqpoint{1.067539in}{1.478793in}}{\pgfqpoint{1.070811in}{1.470893in}}{\pgfqpoint{1.076635in}{1.465069in}}%
\pgfpathcurveto{\pgfqpoint{1.082459in}{1.459245in}}{\pgfqpoint{1.090359in}{1.455972in}}{\pgfqpoint{1.098595in}{1.455972in}}%
\pgfpathclose%
\pgfusepath{stroke,fill}%
\end{pgfscope}%
\begin{pgfscope}%
\pgfpathrectangle{\pgfqpoint{0.100000in}{0.220728in}}{\pgfqpoint{3.696000in}{3.696000in}}%
\pgfusepath{clip}%
\pgfsetbuttcap%
\pgfsetroundjoin%
\definecolor{currentfill}{rgb}{0.121569,0.466667,0.705882}%
\pgfsetfillcolor{currentfill}%
\pgfsetfillopacity{0.697293}%
\pgfsetlinewidth{1.003750pt}%
\definecolor{currentstroke}{rgb}{0.121569,0.466667,0.705882}%
\pgfsetstrokecolor{currentstroke}%
\pgfsetstrokeopacity{0.697293}%
\pgfsetdash{}{0pt}%
\pgfpathmoveto{\pgfqpoint{1.038847in}{1.506457in}}%
\pgfpathcurveto{\pgfqpoint{1.047084in}{1.506457in}}{\pgfqpoint{1.054984in}{1.509730in}}{\pgfqpoint{1.060808in}{1.515554in}}%
\pgfpathcurveto{\pgfqpoint{1.066632in}{1.521378in}}{\pgfqpoint{1.069904in}{1.529278in}}{\pgfqpoint{1.069904in}{1.537514in}}%
\pgfpathcurveto{\pgfqpoint{1.069904in}{1.545750in}}{\pgfqpoint{1.066632in}{1.553650in}}{\pgfqpoint{1.060808in}{1.559474in}}%
\pgfpathcurveto{\pgfqpoint{1.054984in}{1.565298in}}{\pgfqpoint{1.047084in}{1.568570in}}{\pgfqpoint{1.038847in}{1.568570in}}%
\pgfpathcurveto{\pgfqpoint{1.030611in}{1.568570in}}{\pgfqpoint{1.022711in}{1.565298in}}{\pgfqpoint{1.016887in}{1.559474in}}%
\pgfpathcurveto{\pgfqpoint{1.011063in}{1.553650in}}{\pgfqpoint{1.007791in}{1.545750in}}{\pgfqpoint{1.007791in}{1.537514in}}%
\pgfpathcurveto{\pgfqpoint{1.007791in}{1.529278in}}{\pgfqpoint{1.011063in}{1.521378in}}{\pgfqpoint{1.016887in}{1.515554in}}%
\pgfpathcurveto{\pgfqpoint{1.022711in}{1.509730in}}{\pgfqpoint{1.030611in}{1.506457in}}{\pgfqpoint{1.038847in}{1.506457in}}%
\pgfpathclose%
\pgfusepath{stroke,fill}%
\end{pgfscope}%
\begin{pgfscope}%
\pgfpathrectangle{\pgfqpoint{0.100000in}{0.220728in}}{\pgfqpoint{3.696000in}{3.696000in}}%
\pgfusepath{clip}%
\pgfsetbuttcap%
\pgfsetroundjoin%
\definecolor{currentfill}{rgb}{0.121569,0.466667,0.705882}%
\pgfsetfillcolor{currentfill}%
\pgfsetfillopacity{0.697597}%
\pgfsetlinewidth{1.003750pt}%
\definecolor{currentstroke}{rgb}{0.121569,0.466667,0.705882}%
\pgfsetstrokecolor{currentstroke}%
\pgfsetstrokeopacity{0.697597}%
\pgfsetdash{}{0pt}%
\pgfpathmoveto{\pgfqpoint{1.074220in}{1.487664in}}%
\pgfpathcurveto{\pgfqpoint{1.082456in}{1.487664in}}{\pgfqpoint{1.090356in}{1.490936in}}{\pgfqpoint{1.096180in}{1.496760in}}%
\pgfpathcurveto{\pgfqpoint{1.102004in}{1.502584in}}{\pgfqpoint{1.105276in}{1.510484in}}{\pgfqpoint{1.105276in}{1.518721in}}%
\pgfpathcurveto{\pgfqpoint{1.105276in}{1.526957in}}{\pgfqpoint{1.102004in}{1.534857in}}{\pgfqpoint{1.096180in}{1.540681in}}%
\pgfpathcurveto{\pgfqpoint{1.090356in}{1.546505in}}{\pgfqpoint{1.082456in}{1.549777in}}{\pgfqpoint{1.074220in}{1.549777in}}%
\pgfpathcurveto{\pgfqpoint{1.065983in}{1.549777in}}{\pgfqpoint{1.058083in}{1.546505in}}{\pgfqpoint{1.052259in}{1.540681in}}%
\pgfpathcurveto{\pgfqpoint{1.046435in}{1.534857in}}{\pgfqpoint{1.043163in}{1.526957in}}{\pgfqpoint{1.043163in}{1.518721in}}%
\pgfpathcurveto{\pgfqpoint{1.043163in}{1.510484in}}{\pgfqpoint{1.046435in}{1.502584in}}{\pgfqpoint{1.052259in}{1.496760in}}%
\pgfpathcurveto{\pgfqpoint{1.058083in}{1.490936in}}{\pgfqpoint{1.065983in}{1.487664in}}{\pgfqpoint{1.074220in}{1.487664in}}%
\pgfpathclose%
\pgfusepath{stroke,fill}%
\end{pgfscope}%
\begin{pgfscope}%
\pgfpathrectangle{\pgfqpoint{0.100000in}{0.220728in}}{\pgfqpoint{3.696000in}{3.696000in}}%
\pgfusepath{clip}%
\pgfsetbuttcap%
\pgfsetroundjoin%
\definecolor{currentfill}{rgb}{0.121569,0.466667,0.705882}%
\pgfsetfillcolor{currentfill}%
\pgfsetfillopacity{0.703378}%
\pgfsetlinewidth{1.003750pt}%
\definecolor{currentstroke}{rgb}{0.121569,0.466667,0.705882}%
\pgfsetstrokecolor{currentstroke}%
\pgfsetstrokeopacity{0.703378}%
\pgfsetdash{}{0pt}%
\pgfpathmoveto{\pgfqpoint{1.029908in}{1.515548in}}%
\pgfpathcurveto{\pgfqpoint{1.038145in}{1.515548in}}{\pgfqpoint{1.046045in}{1.518821in}}{\pgfqpoint{1.051869in}{1.524644in}}%
\pgfpathcurveto{\pgfqpoint{1.057692in}{1.530468in}}{\pgfqpoint{1.060965in}{1.538368in}}{\pgfqpoint{1.060965in}{1.546605in}}%
\pgfpathcurveto{\pgfqpoint{1.060965in}{1.554841in}}{\pgfqpoint{1.057692in}{1.562741in}}{\pgfqpoint{1.051869in}{1.568565in}}%
\pgfpathcurveto{\pgfqpoint{1.046045in}{1.574389in}}{\pgfqpoint{1.038145in}{1.577661in}}{\pgfqpoint{1.029908in}{1.577661in}}%
\pgfpathcurveto{\pgfqpoint{1.021672in}{1.577661in}}{\pgfqpoint{1.013772in}{1.574389in}}{\pgfqpoint{1.007948in}{1.568565in}}%
\pgfpathcurveto{\pgfqpoint{1.002124in}{1.562741in}}{\pgfqpoint{0.998852in}{1.554841in}}{\pgfqpoint{0.998852in}{1.546605in}}%
\pgfpathcurveto{\pgfqpoint{0.998852in}{1.538368in}}{\pgfqpoint{1.002124in}{1.530468in}}{\pgfqpoint{1.007948in}{1.524644in}}%
\pgfpathcurveto{\pgfqpoint{1.013772in}{1.518821in}}{\pgfqpoint{1.021672in}{1.515548in}}{\pgfqpoint{1.029908in}{1.515548in}}%
\pgfpathclose%
\pgfusepath{stroke,fill}%
\end{pgfscope}%
\begin{pgfscope}%
\pgfpathrectangle{\pgfqpoint{0.100000in}{0.220728in}}{\pgfqpoint{3.696000in}{3.696000in}}%
\pgfusepath{clip}%
\pgfsetbuttcap%
\pgfsetroundjoin%
\definecolor{currentfill}{rgb}{0.121569,0.466667,0.705882}%
\pgfsetfillcolor{currentfill}%
\pgfsetfillopacity{0.704238}%
\pgfsetlinewidth{1.003750pt}%
\definecolor{currentstroke}{rgb}{0.121569,0.466667,0.705882}%
\pgfsetstrokecolor{currentstroke}%
\pgfsetstrokeopacity{0.704238}%
\pgfsetdash{}{0pt}%
\pgfpathmoveto{\pgfqpoint{1.043939in}{1.507427in}}%
\pgfpathcurveto{\pgfqpoint{1.052176in}{1.507427in}}{\pgfqpoint{1.060076in}{1.510700in}}{\pgfqpoint{1.065900in}{1.516524in}}%
\pgfpathcurveto{\pgfqpoint{1.071724in}{1.522348in}}{\pgfqpoint{1.074996in}{1.530248in}}{\pgfqpoint{1.074996in}{1.538484in}}%
\pgfpathcurveto{\pgfqpoint{1.074996in}{1.546720in}}{\pgfqpoint{1.071724in}{1.554620in}}{\pgfqpoint{1.065900in}{1.560444in}}%
\pgfpathcurveto{\pgfqpoint{1.060076in}{1.566268in}}{\pgfqpoint{1.052176in}{1.569540in}}{\pgfqpoint{1.043939in}{1.569540in}}%
\pgfpathcurveto{\pgfqpoint{1.035703in}{1.569540in}}{\pgfqpoint{1.027803in}{1.566268in}}{\pgfqpoint{1.021979in}{1.560444in}}%
\pgfpathcurveto{\pgfqpoint{1.016155in}{1.554620in}}{\pgfqpoint{1.012883in}{1.546720in}}{\pgfqpoint{1.012883in}{1.538484in}}%
\pgfpathcurveto{\pgfqpoint{1.012883in}{1.530248in}}{\pgfqpoint{1.016155in}{1.522348in}}{\pgfqpoint{1.021979in}{1.516524in}}%
\pgfpathcurveto{\pgfqpoint{1.027803in}{1.510700in}}{\pgfqpoint{1.035703in}{1.507427in}}{\pgfqpoint{1.043939in}{1.507427in}}%
\pgfpathclose%
\pgfusepath{stroke,fill}%
\end{pgfscope}%
\begin{pgfscope}%
\pgfpathrectangle{\pgfqpoint{0.100000in}{0.220728in}}{\pgfqpoint{3.696000in}{3.696000in}}%
\pgfusepath{clip}%
\pgfsetbuttcap%
\pgfsetroundjoin%
\definecolor{currentfill}{rgb}{0.121569,0.466667,0.705882}%
\pgfsetfillcolor{currentfill}%
\pgfsetfillopacity{0.704787}%
\pgfsetlinewidth{1.003750pt}%
\definecolor{currentstroke}{rgb}{0.121569,0.466667,0.705882}%
\pgfsetstrokecolor{currentstroke}%
\pgfsetstrokeopacity{0.704787}%
\pgfsetdash{}{0pt}%
\pgfpathmoveto{\pgfqpoint{1.380019in}{1.142137in}}%
\pgfpathcurveto{\pgfqpoint{1.388256in}{1.142137in}}{\pgfqpoint{1.396156in}{1.145409in}}{\pgfqpoint{1.401980in}{1.151233in}}%
\pgfpathcurveto{\pgfqpoint{1.407803in}{1.157057in}}{\pgfqpoint{1.411076in}{1.164957in}}{\pgfqpoint{1.411076in}{1.173193in}}%
\pgfpathcurveto{\pgfqpoint{1.411076in}{1.181429in}}{\pgfqpoint{1.407803in}{1.189329in}}{\pgfqpoint{1.401980in}{1.195153in}}%
\pgfpathcurveto{\pgfqpoint{1.396156in}{1.200977in}}{\pgfqpoint{1.388256in}{1.204250in}}{\pgfqpoint{1.380019in}{1.204250in}}%
\pgfpathcurveto{\pgfqpoint{1.371783in}{1.204250in}}{\pgfqpoint{1.363883in}{1.200977in}}{\pgfqpoint{1.358059in}{1.195153in}}%
\pgfpathcurveto{\pgfqpoint{1.352235in}{1.189329in}}{\pgfqpoint{1.348963in}{1.181429in}}{\pgfqpoint{1.348963in}{1.173193in}}%
\pgfpathcurveto{\pgfqpoint{1.348963in}{1.164957in}}{\pgfqpoint{1.352235in}{1.157057in}}{\pgfqpoint{1.358059in}{1.151233in}}%
\pgfpathcurveto{\pgfqpoint{1.363883in}{1.145409in}}{\pgfqpoint{1.371783in}{1.142137in}}{\pgfqpoint{1.380019in}{1.142137in}}%
\pgfpathclose%
\pgfusepath{stroke,fill}%
\end{pgfscope}%
\begin{pgfscope}%
\pgfpathrectangle{\pgfqpoint{0.100000in}{0.220728in}}{\pgfqpoint{3.696000in}{3.696000in}}%
\pgfusepath{clip}%
\pgfsetbuttcap%
\pgfsetroundjoin%
\definecolor{currentfill}{rgb}{0.121569,0.466667,0.705882}%
\pgfsetfillcolor{currentfill}%
\pgfsetfillopacity{0.707915}%
\pgfsetlinewidth{1.003750pt}%
\definecolor{currentstroke}{rgb}{0.121569,0.466667,0.705882}%
\pgfsetstrokecolor{currentstroke}%
\pgfsetstrokeopacity{0.707915}%
\pgfsetdash{}{0pt}%
\pgfpathmoveto{\pgfqpoint{1.139349in}{1.421270in}}%
\pgfpathcurveto{\pgfqpoint{1.147586in}{1.421270in}}{\pgfqpoint{1.155486in}{1.424542in}}{\pgfqpoint{1.161310in}{1.430366in}}%
\pgfpathcurveto{\pgfqpoint{1.167134in}{1.436190in}}{\pgfqpoint{1.170406in}{1.444090in}}{\pgfqpoint{1.170406in}{1.452326in}}%
\pgfpathcurveto{\pgfqpoint{1.170406in}{1.460562in}}{\pgfqpoint{1.167134in}{1.468462in}}{\pgfqpoint{1.161310in}{1.474286in}}%
\pgfpathcurveto{\pgfqpoint{1.155486in}{1.480110in}}{\pgfqpoint{1.147586in}{1.483383in}}{\pgfqpoint{1.139349in}{1.483383in}}%
\pgfpathcurveto{\pgfqpoint{1.131113in}{1.483383in}}{\pgfqpoint{1.123213in}{1.480110in}}{\pgfqpoint{1.117389in}{1.474286in}}%
\pgfpathcurveto{\pgfqpoint{1.111565in}{1.468462in}}{\pgfqpoint{1.108293in}{1.460562in}}{\pgfqpoint{1.108293in}{1.452326in}}%
\pgfpathcurveto{\pgfqpoint{1.108293in}{1.444090in}}{\pgfqpoint{1.111565in}{1.436190in}}{\pgfqpoint{1.117389in}{1.430366in}}%
\pgfpathcurveto{\pgfqpoint{1.123213in}{1.424542in}}{\pgfqpoint{1.131113in}{1.421270in}}{\pgfqpoint{1.139349in}{1.421270in}}%
\pgfpathclose%
\pgfusepath{stroke,fill}%
\end{pgfscope}%
\begin{pgfscope}%
\pgfpathrectangle{\pgfqpoint{0.100000in}{0.220728in}}{\pgfqpoint{3.696000in}{3.696000in}}%
\pgfusepath{clip}%
\pgfsetbuttcap%
\pgfsetroundjoin%
\definecolor{currentfill}{rgb}{0.121569,0.466667,0.705882}%
\pgfsetfillcolor{currentfill}%
\pgfsetfillopacity{0.713654}%
\pgfsetlinewidth{1.003750pt}%
\definecolor{currentstroke}{rgb}{0.121569,0.466667,0.705882}%
\pgfsetstrokecolor{currentstroke}%
\pgfsetstrokeopacity{0.713654}%
\pgfsetdash{}{0pt}%
\pgfpathmoveto{\pgfqpoint{1.000066in}{1.535627in}}%
\pgfpathcurveto{\pgfqpoint{1.008303in}{1.535627in}}{\pgfqpoint{1.016203in}{1.538900in}}{\pgfqpoint{1.022027in}{1.544724in}}%
\pgfpathcurveto{\pgfqpoint{1.027851in}{1.550547in}}{\pgfqpoint{1.031123in}{1.558448in}}{\pgfqpoint{1.031123in}{1.566684in}}%
\pgfpathcurveto{\pgfqpoint{1.031123in}{1.574920in}}{\pgfqpoint{1.027851in}{1.582820in}}{\pgfqpoint{1.022027in}{1.588644in}}%
\pgfpathcurveto{\pgfqpoint{1.016203in}{1.594468in}}{\pgfqpoint{1.008303in}{1.597740in}}{\pgfqpoint{1.000066in}{1.597740in}}%
\pgfpathcurveto{\pgfqpoint{0.991830in}{1.597740in}}{\pgfqpoint{0.983930in}{1.594468in}}{\pgfqpoint{0.978106in}{1.588644in}}%
\pgfpathcurveto{\pgfqpoint{0.972282in}{1.582820in}}{\pgfqpoint{0.969010in}{1.574920in}}{\pgfqpoint{0.969010in}{1.566684in}}%
\pgfpathcurveto{\pgfqpoint{0.969010in}{1.558448in}}{\pgfqpoint{0.972282in}{1.550547in}}{\pgfqpoint{0.978106in}{1.544724in}}%
\pgfpathcurveto{\pgfqpoint{0.983930in}{1.538900in}}{\pgfqpoint{0.991830in}{1.535627in}}{\pgfqpoint{1.000066in}{1.535627in}}%
\pgfpathclose%
\pgfusepath{stroke,fill}%
\end{pgfscope}%
\begin{pgfscope}%
\pgfpathrectangle{\pgfqpoint{0.100000in}{0.220728in}}{\pgfqpoint{3.696000in}{3.696000in}}%
\pgfusepath{clip}%
\pgfsetbuttcap%
\pgfsetroundjoin%
\definecolor{currentfill}{rgb}{0.121569,0.466667,0.705882}%
\pgfsetfillcolor{currentfill}%
\pgfsetfillopacity{0.714569}%
\pgfsetlinewidth{1.003750pt}%
\definecolor{currentstroke}{rgb}{0.121569,0.466667,0.705882}%
\pgfsetstrokecolor{currentstroke}%
\pgfsetstrokeopacity{0.714569}%
\pgfsetdash{}{0pt}%
\pgfpathmoveto{\pgfqpoint{1.195941in}{1.355785in}}%
\pgfpathcurveto{\pgfqpoint{1.204177in}{1.355785in}}{\pgfqpoint{1.212077in}{1.359057in}}{\pgfqpoint{1.217901in}{1.364881in}}%
\pgfpathcurveto{\pgfqpoint{1.223725in}{1.370705in}}{\pgfqpoint{1.226997in}{1.378605in}}{\pgfqpoint{1.226997in}{1.386841in}}%
\pgfpathcurveto{\pgfqpoint{1.226997in}{1.395077in}}{\pgfqpoint{1.223725in}{1.402977in}}{\pgfqpoint{1.217901in}{1.408801in}}%
\pgfpathcurveto{\pgfqpoint{1.212077in}{1.414625in}}{\pgfqpoint{1.204177in}{1.417898in}}{\pgfqpoint{1.195941in}{1.417898in}}%
\pgfpathcurveto{\pgfqpoint{1.187704in}{1.417898in}}{\pgfqpoint{1.179804in}{1.414625in}}{\pgfqpoint{1.173980in}{1.408801in}}%
\pgfpathcurveto{\pgfqpoint{1.168157in}{1.402977in}}{\pgfqpoint{1.164884in}{1.395077in}}{\pgfqpoint{1.164884in}{1.386841in}}%
\pgfpathcurveto{\pgfqpoint{1.164884in}{1.378605in}}{\pgfqpoint{1.168157in}{1.370705in}}{\pgfqpoint{1.173980in}{1.364881in}}%
\pgfpathcurveto{\pgfqpoint{1.179804in}{1.359057in}}{\pgfqpoint{1.187704in}{1.355785in}}{\pgfqpoint{1.195941in}{1.355785in}}%
\pgfpathclose%
\pgfusepath{stroke,fill}%
\end{pgfscope}%
\begin{pgfscope}%
\pgfpathrectangle{\pgfqpoint{0.100000in}{0.220728in}}{\pgfqpoint{3.696000in}{3.696000in}}%
\pgfusepath{clip}%
\pgfsetbuttcap%
\pgfsetroundjoin%
\definecolor{currentfill}{rgb}{0.121569,0.466667,0.705882}%
\pgfsetfillcolor{currentfill}%
\pgfsetfillopacity{0.717823}%
\pgfsetlinewidth{1.003750pt}%
\definecolor{currentstroke}{rgb}{0.121569,0.466667,0.705882}%
\pgfsetstrokecolor{currentstroke}%
\pgfsetstrokeopacity{0.717823}%
\pgfsetdash{}{0pt}%
\pgfpathmoveto{\pgfqpoint{1.117715in}{1.449047in}}%
\pgfpathcurveto{\pgfqpoint{1.125952in}{1.449047in}}{\pgfqpoint{1.133852in}{1.452320in}}{\pgfqpoint{1.139676in}{1.458144in}}%
\pgfpathcurveto{\pgfqpoint{1.145500in}{1.463968in}}{\pgfqpoint{1.148772in}{1.471868in}}{\pgfqpoint{1.148772in}{1.480104in}}%
\pgfpathcurveto{\pgfqpoint{1.148772in}{1.488340in}}{\pgfqpoint{1.145500in}{1.496240in}}{\pgfqpoint{1.139676in}{1.502064in}}%
\pgfpathcurveto{\pgfqpoint{1.133852in}{1.507888in}}{\pgfqpoint{1.125952in}{1.511160in}}{\pgfqpoint{1.117715in}{1.511160in}}%
\pgfpathcurveto{\pgfqpoint{1.109479in}{1.511160in}}{\pgfqpoint{1.101579in}{1.507888in}}{\pgfqpoint{1.095755in}{1.502064in}}%
\pgfpathcurveto{\pgfqpoint{1.089931in}{1.496240in}}{\pgfqpoint{1.086659in}{1.488340in}}{\pgfqpoint{1.086659in}{1.480104in}}%
\pgfpathcurveto{\pgfqpoint{1.086659in}{1.471868in}}{\pgfqpoint{1.089931in}{1.463968in}}{\pgfqpoint{1.095755in}{1.458144in}}%
\pgfpathcurveto{\pgfqpoint{1.101579in}{1.452320in}}{\pgfqpoint{1.109479in}{1.449047in}}{\pgfqpoint{1.117715in}{1.449047in}}%
\pgfpathclose%
\pgfusepath{stroke,fill}%
\end{pgfscope}%
\begin{pgfscope}%
\pgfpathrectangle{\pgfqpoint{0.100000in}{0.220728in}}{\pgfqpoint{3.696000in}{3.696000in}}%
\pgfusepath{clip}%
\pgfsetbuttcap%
\pgfsetroundjoin%
\definecolor{currentfill}{rgb}{0.121569,0.466667,0.705882}%
\pgfsetfillcolor{currentfill}%
\pgfsetfillopacity{0.719213}%
\pgfsetlinewidth{1.003750pt}%
\definecolor{currentstroke}{rgb}{0.121569,0.466667,0.705882}%
\pgfsetstrokecolor{currentstroke}%
\pgfsetstrokeopacity{0.719213}%
\pgfsetdash{}{0pt}%
\pgfpathmoveto{\pgfqpoint{1.170096in}{1.388892in}}%
\pgfpathcurveto{\pgfqpoint{1.178333in}{1.388892in}}{\pgfqpoint{1.186233in}{1.392164in}}{\pgfqpoint{1.192057in}{1.397988in}}%
\pgfpathcurveto{\pgfqpoint{1.197880in}{1.403812in}}{\pgfqpoint{1.201153in}{1.411712in}}{\pgfqpoint{1.201153in}{1.419948in}}%
\pgfpathcurveto{\pgfqpoint{1.201153in}{1.428184in}}{\pgfqpoint{1.197880in}{1.436084in}}{\pgfqpoint{1.192057in}{1.441908in}}%
\pgfpathcurveto{\pgfqpoint{1.186233in}{1.447732in}}{\pgfqpoint{1.178333in}{1.451005in}}{\pgfqpoint{1.170096in}{1.451005in}}%
\pgfpathcurveto{\pgfqpoint{1.161860in}{1.451005in}}{\pgfqpoint{1.153960in}{1.447732in}}{\pgfqpoint{1.148136in}{1.441908in}}%
\pgfpathcurveto{\pgfqpoint{1.142312in}{1.436084in}}{\pgfqpoint{1.139040in}{1.428184in}}{\pgfqpoint{1.139040in}{1.419948in}}%
\pgfpathcurveto{\pgfqpoint{1.139040in}{1.411712in}}{\pgfqpoint{1.142312in}{1.403812in}}{\pgfqpoint{1.148136in}{1.397988in}}%
\pgfpathcurveto{\pgfqpoint{1.153960in}{1.392164in}}{\pgfqpoint{1.161860in}{1.388892in}}{\pgfqpoint{1.170096in}{1.388892in}}%
\pgfpathclose%
\pgfusepath{stroke,fill}%
\end{pgfscope}%
\begin{pgfscope}%
\pgfpathrectangle{\pgfqpoint{0.100000in}{0.220728in}}{\pgfqpoint{3.696000in}{3.696000in}}%
\pgfusepath{clip}%
\pgfsetbuttcap%
\pgfsetroundjoin%
\definecolor{currentfill}{rgb}{0.121569,0.466667,0.705882}%
\pgfsetfillcolor{currentfill}%
\pgfsetfillopacity{0.720546}%
\pgfsetlinewidth{1.003750pt}%
\definecolor{currentstroke}{rgb}{0.121569,0.466667,0.705882}%
\pgfsetstrokecolor{currentstroke}%
\pgfsetstrokeopacity{0.720546}%
\pgfsetdash{}{0pt}%
\pgfpathmoveto{\pgfqpoint{1.390229in}{1.144438in}}%
\pgfpathcurveto{\pgfqpoint{1.398465in}{1.144438in}}{\pgfqpoint{1.406365in}{1.147711in}}{\pgfqpoint{1.412189in}{1.153535in}}%
\pgfpathcurveto{\pgfqpoint{1.418013in}{1.159359in}}{\pgfqpoint{1.421285in}{1.167259in}}{\pgfqpoint{1.421285in}{1.175495in}}%
\pgfpathcurveto{\pgfqpoint{1.421285in}{1.183731in}}{\pgfqpoint{1.418013in}{1.191631in}}{\pgfqpoint{1.412189in}{1.197455in}}%
\pgfpathcurveto{\pgfqpoint{1.406365in}{1.203279in}}{\pgfqpoint{1.398465in}{1.206551in}}{\pgfqpoint{1.390229in}{1.206551in}}%
\pgfpathcurveto{\pgfqpoint{1.381992in}{1.206551in}}{\pgfqpoint{1.374092in}{1.203279in}}{\pgfqpoint{1.368268in}{1.197455in}}%
\pgfpathcurveto{\pgfqpoint{1.362444in}{1.191631in}}{\pgfqpoint{1.359172in}{1.183731in}}{\pgfqpoint{1.359172in}{1.175495in}}%
\pgfpathcurveto{\pgfqpoint{1.359172in}{1.167259in}}{\pgfqpoint{1.362444in}{1.159359in}}{\pgfqpoint{1.368268in}{1.153535in}}%
\pgfpathcurveto{\pgfqpoint{1.374092in}{1.147711in}}{\pgfqpoint{1.381992in}{1.144438in}}{\pgfqpoint{1.390229in}{1.144438in}}%
\pgfpathclose%
\pgfusepath{stroke,fill}%
\end{pgfscope}%
\begin{pgfscope}%
\pgfpathrectangle{\pgfqpoint{0.100000in}{0.220728in}}{\pgfqpoint{3.696000in}{3.696000in}}%
\pgfusepath{clip}%
\pgfsetbuttcap%
\pgfsetroundjoin%
\definecolor{currentfill}{rgb}{0.121569,0.466667,0.705882}%
\pgfsetfillcolor{currentfill}%
\pgfsetfillopacity{0.726922}%
\pgfsetlinewidth{1.003750pt}%
\definecolor{currentstroke}{rgb}{0.121569,0.466667,0.705882}%
\pgfsetstrokecolor{currentstroke}%
\pgfsetstrokeopacity{0.726922}%
\pgfsetdash{}{0pt}%
\pgfpathmoveto{\pgfqpoint{1.187500in}{1.366947in}}%
\pgfpathcurveto{\pgfqpoint{1.195736in}{1.366947in}}{\pgfqpoint{1.203636in}{1.370219in}}{\pgfqpoint{1.209460in}{1.376043in}}%
\pgfpathcurveto{\pgfqpoint{1.215284in}{1.381867in}}{\pgfqpoint{1.218557in}{1.389767in}}{\pgfqpoint{1.218557in}{1.398004in}}%
\pgfpathcurveto{\pgfqpoint{1.218557in}{1.406240in}}{\pgfqpoint{1.215284in}{1.414140in}}{\pgfqpoint{1.209460in}{1.419964in}}%
\pgfpathcurveto{\pgfqpoint{1.203636in}{1.425788in}}{\pgfqpoint{1.195736in}{1.429060in}}{\pgfqpoint{1.187500in}{1.429060in}}%
\pgfpathcurveto{\pgfqpoint{1.179264in}{1.429060in}}{\pgfqpoint{1.171364in}{1.425788in}}{\pgfqpoint{1.165540in}{1.419964in}}%
\pgfpathcurveto{\pgfqpoint{1.159716in}{1.414140in}}{\pgfqpoint{1.156444in}{1.406240in}}{\pgfqpoint{1.156444in}{1.398004in}}%
\pgfpathcurveto{\pgfqpoint{1.156444in}{1.389767in}}{\pgfqpoint{1.159716in}{1.381867in}}{\pgfqpoint{1.165540in}{1.376043in}}%
\pgfpathcurveto{\pgfqpoint{1.171364in}{1.370219in}}{\pgfqpoint{1.179264in}{1.366947in}}{\pgfqpoint{1.187500in}{1.366947in}}%
\pgfpathclose%
\pgfusepath{stroke,fill}%
\end{pgfscope}%
\begin{pgfscope}%
\pgfpathrectangle{\pgfqpoint{0.100000in}{0.220728in}}{\pgfqpoint{3.696000in}{3.696000in}}%
\pgfusepath{clip}%
\pgfsetbuttcap%
\pgfsetroundjoin%
\definecolor{currentfill}{rgb}{0.121569,0.466667,0.705882}%
\pgfsetfillcolor{currentfill}%
\pgfsetfillopacity{0.730743}%
\pgfsetlinewidth{1.003750pt}%
\definecolor{currentstroke}{rgb}{0.121569,0.466667,0.705882}%
\pgfsetstrokecolor{currentstroke}%
\pgfsetstrokeopacity{0.730743}%
\pgfsetdash{}{0pt}%
\pgfpathmoveto{\pgfqpoint{1.155701in}{1.421896in}}%
\pgfpathcurveto{\pgfqpoint{1.163938in}{1.421896in}}{\pgfqpoint{1.171838in}{1.425169in}}{\pgfqpoint{1.177662in}{1.430993in}}%
\pgfpathcurveto{\pgfqpoint{1.183485in}{1.436817in}}{\pgfqpoint{1.186758in}{1.444717in}}{\pgfqpoint{1.186758in}{1.452953in}}%
\pgfpathcurveto{\pgfqpoint{1.186758in}{1.461189in}}{\pgfqpoint{1.183485in}{1.469089in}}{\pgfqpoint{1.177662in}{1.474913in}}%
\pgfpathcurveto{\pgfqpoint{1.171838in}{1.480737in}}{\pgfqpoint{1.163938in}{1.484009in}}{\pgfqpoint{1.155701in}{1.484009in}}%
\pgfpathcurveto{\pgfqpoint{1.147465in}{1.484009in}}{\pgfqpoint{1.139565in}{1.480737in}}{\pgfqpoint{1.133741in}{1.474913in}}%
\pgfpathcurveto{\pgfqpoint{1.127917in}{1.469089in}}{\pgfqpoint{1.124645in}{1.461189in}}{\pgfqpoint{1.124645in}{1.452953in}}%
\pgfpathcurveto{\pgfqpoint{1.124645in}{1.444717in}}{\pgfqpoint{1.127917in}{1.436817in}}{\pgfqpoint{1.133741in}{1.430993in}}%
\pgfpathcurveto{\pgfqpoint{1.139565in}{1.425169in}}{\pgfqpoint{1.147465in}{1.421896in}}{\pgfqpoint{1.155701in}{1.421896in}}%
\pgfpathclose%
\pgfusepath{stroke,fill}%
\end{pgfscope}%
\begin{pgfscope}%
\pgfpathrectangle{\pgfqpoint{0.100000in}{0.220728in}}{\pgfqpoint{3.696000in}{3.696000in}}%
\pgfusepath{clip}%
\pgfsetbuttcap%
\pgfsetroundjoin%
\definecolor{currentfill}{rgb}{0.121569,0.466667,0.705882}%
\pgfsetfillcolor{currentfill}%
\pgfsetfillopacity{0.731352}%
\pgfsetlinewidth{1.003750pt}%
\definecolor{currentstroke}{rgb}{0.121569,0.466667,0.705882}%
\pgfsetstrokecolor{currentstroke}%
\pgfsetstrokeopacity{0.731352}%
\pgfsetdash{}{0pt}%
\pgfpathmoveto{\pgfqpoint{1.180542in}{1.386853in}}%
\pgfpathcurveto{\pgfqpoint{1.188778in}{1.386853in}}{\pgfqpoint{1.196678in}{1.390125in}}{\pgfqpoint{1.202502in}{1.395949in}}%
\pgfpathcurveto{\pgfqpoint{1.208326in}{1.401773in}}{\pgfqpoint{1.211598in}{1.409673in}}{\pgfqpoint{1.211598in}{1.417909in}}%
\pgfpathcurveto{\pgfqpoint{1.211598in}{1.426146in}}{\pgfqpoint{1.208326in}{1.434046in}}{\pgfqpoint{1.202502in}{1.439870in}}%
\pgfpathcurveto{\pgfqpoint{1.196678in}{1.445694in}}{\pgfqpoint{1.188778in}{1.448966in}}{\pgfqpoint{1.180542in}{1.448966in}}%
\pgfpathcurveto{\pgfqpoint{1.172306in}{1.448966in}}{\pgfqpoint{1.164405in}{1.445694in}}{\pgfqpoint{1.158582in}{1.439870in}}%
\pgfpathcurveto{\pgfqpoint{1.152758in}{1.434046in}}{\pgfqpoint{1.149485in}{1.426146in}}{\pgfqpoint{1.149485in}{1.417909in}}%
\pgfpathcurveto{\pgfqpoint{1.149485in}{1.409673in}}{\pgfqpoint{1.152758in}{1.401773in}}{\pgfqpoint{1.158582in}{1.395949in}}%
\pgfpathcurveto{\pgfqpoint{1.164405in}{1.390125in}}{\pgfqpoint{1.172306in}{1.386853in}}{\pgfqpoint{1.180542in}{1.386853in}}%
\pgfpathclose%
\pgfusepath{stroke,fill}%
\end{pgfscope}%
\begin{pgfscope}%
\pgfpathrectangle{\pgfqpoint{0.100000in}{0.220728in}}{\pgfqpoint{3.696000in}{3.696000in}}%
\pgfusepath{clip}%
\pgfsetbuttcap%
\pgfsetroundjoin%
\definecolor{currentfill}{rgb}{0.121569,0.466667,0.705882}%
\pgfsetfillcolor{currentfill}%
\pgfsetfillopacity{0.735854}%
\pgfsetlinewidth{1.003750pt}%
\definecolor{currentstroke}{rgb}{0.121569,0.466667,0.705882}%
\pgfsetstrokecolor{currentstroke}%
\pgfsetstrokeopacity{0.735854}%
\pgfsetdash{}{0pt}%
\pgfpathmoveto{\pgfqpoint{0.649545in}{2.242668in}}%
\pgfpathcurveto{\pgfqpoint{0.657782in}{2.242668in}}{\pgfqpoint{0.665682in}{2.245940in}}{\pgfqpoint{0.671506in}{2.251764in}}%
\pgfpathcurveto{\pgfqpoint{0.677329in}{2.257588in}}{\pgfqpoint{0.680602in}{2.265488in}}{\pgfqpoint{0.680602in}{2.273724in}}%
\pgfpathcurveto{\pgfqpoint{0.680602in}{2.281960in}}{\pgfqpoint{0.677329in}{2.289860in}}{\pgfqpoint{0.671506in}{2.295684in}}%
\pgfpathcurveto{\pgfqpoint{0.665682in}{2.301508in}}{\pgfqpoint{0.657782in}{2.304781in}}{\pgfqpoint{0.649545in}{2.304781in}}%
\pgfpathcurveto{\pgfqpoint{0.641309in}{2.304781in}}{\pgfqpoint{0.633409in}{2.301508in}}{\pgfqpoint{0.627585in}{2.295684in}}%
\pgfpathcurveto{\pgfqpoint{0.621761in}{2.289860in}}{\pgfqpoint{0.618489in}{2.281960in}}{\pgfqpoint{0.618489in}{2.273724in}}%
\pgfpathcurveto{\pgfqpoint{0.618489in}{2.265488in}}{\pgfqpoint{0.621761in}{2.257588in}}{\pgfqpoint{0.627585in}{2.251764in}}%
\pgfpathcurveto{\pgfqpoint{0.633409in}{2.245940in}}{\pgfqpoint{0.641309in}{2.242668in}}{\pgfqpoint{0.649545in}{2.242668in}}%
\pgfpathclose%
\pgfusepath{stroke,fill}%
\end{pgfscope}%
\begin{pgfscope}%
\pgfpathrectangle{\pgfqpoint{0.100000in}{0.220728in}}{\pgfqpoint{3.696000in}{3.696000in}}%
\pgfusepath{clip}%
\pgfsetbuttcap%
\pgfsetroundjoin%
\definecolor{currentfill}{rgb}{0.121569,0.466667,0.705882}%
\pgfsetfillcolor{currentfill}%
\pgfsetfillopacity{0.755328}%
\pgfsetlinewidth{1.003750pt}%
\definecolor{currentstroke}{rgb}{0.121569,0.466667,0.705882}%
\pgfsetstrokecolor{currentstroke}%
\pgfsetstrokeopacity{0.755328}%
\pgfsetdash{}{0pt}%
\pgfpathmoveto{\pgfqpoint{1.406941in}{1.149323in}}%
\pgfpathcurveto{\pgfqpoint{1.415178in}{1.149323in}}{\pgfqpoint{1.423078in}{1.152595in}}{\pgfqpoint{1.428902in}{1.158419in}}%
\pgfpathcurveto{\pgfqpoint{1.434726in}{1.164243in}}{\pgfqpoint{1.437998in}{1.172143in}}{\pgfqpoint{1.437998in}{1.180379in}}%
\pgfpathcurveto{\pgfqpoint{1.437998in}{1.188616in}}{\pgfqpoint{1.434726in}{1.196516in}}{\pgfqpoint{1.428902in}{1.202340in}}%
\pgfpathcurveto{\pgfqpoint{1.423078in}{1.208163in}}{\pgfqpoint{1.415178in}{1.211436in}}{\pgfqpoint{1.406941in}{1.211436in}}%
\pgfpathcurveto{\pgfqpoint{1.398705in}{1.211436in}}{\pgfqpoint{1.390805in}{1.208163in}}{\pgfqpoint{1.384981in}{1.202340in}}%
\pgfpathcurveto{\pgfqpoint{1.379157in}{1.196516in}}{\pgfqpoint{1.375885in}{1.188616in}}{\pgfqpoint{1.375885in}{1.180379in}}%
\pgfpathcurveto{\pgfqpoint{1.375885in}{1.172143in}}{\pgfqpoint{1.379157in}{1.164243in}}{\pgfqpoint{1.384981in}{1.158419in}}%
\pgfpathcurveto{\pgfqpoint{1.390805in}{1.152595in}}{\pgfqpoint{1.398705in}{1.149323in}}{\pgfqpoint{1.406941in}{1.149323in}}%
\pgfpathclose%
\pgfusepath{stroke,fill}%
\end{pgfscope}%
\begin{pgfscope}%
\pgfpathrectangle{\pgfqpoint{0.100000in}{0.220728in}}{\pgfqpoint{3.696000in}{3.696000in}}%
\pgfusepath{clip}%
\pgfsetbuttcap%
\pgfsetroundjoin%
\definecolor{currentfill}{rgb}{0.121569,0.466667,0.705882}%
\pgfsetfillcolor{currentfill}%
\pgfsetfillopacity{0.785882}%
\pgfsetlinewidth{1.003750pt}%
\definecolor{currentstroke}{rgb}{0.121569,0.466667,0.705882}%
\pgfsetstrokecolor{currentstroke}%
\pgfsetstrokeopacity{0.785882}%
\pgfsetdash{}{0pt}%
\pgfpathmoveto{\pgfqpoint{1.424087in}{1.163261in}}%
\pgfpathcurveto{\pgfqpoint{1.432324in}{1.163261in}}{\pgfqpoint{1.440224in}{1.166533in}}{\pgfqpoint{1.446048in}{1.172357in}}%
\pgfpathcurveto{\pgfqpoint{1.451872in}{1.178181in}}{\pgfqpoint{1.455144in}{1.186081in}}{\pgfqpoint{1.455144in}{1.194318in}}%
\pgfpathcurveto{\pgfqpoint{1.455144in}{1.202554in}}{\pgfqpoint{1.451872in}{1.210454in}}{\pgfqpoint{1.446048in}{1.216278in}}%
\pgfpathcurveto{\pgfqpoint{1.440224in}{1.222102in}}{\pgfqpoint{1.432324in}{1.225374in}}{\pgfqpoint{1.424087in}{1.225374in}}%
\pgfpathcurveto{\pgfqpoint{1.415851in}{1.225374in}}{\pgfqpoint{1.407951in}{1.222102in}}{\pgfqpoint{1.402127in}{1.216278in}}%
\pgfpathcurveto{\pgfqpoint{1.396303in}{1.210454in}}{\pgfqpoint{1.393031in}{1.202554in}}{\pgfqpoint{1.393031in}{1.194318in}}%
\pgfpathcurveto{\pgfqpoint{1.393031in}{1.186081in}}{\pgfqpoint{1.396303in}{1.178181in}}{\pgfqpoint{1.402127in}{1.172357in}}%
\pgfpathcurveto{\pgfqpoint{1.407951in}{1.166533in}}{\pgfqpoint{1.415851in}{1.163261in}}{\pgfqpoint{1.424087in}{1.163261in}}%
\pgfpathclose%
\pgfusepath{stroke,fill}%
\end{pgfscope}%
\begin{pgfscope}%
\pgfpathrectangle{\pgfqpoint{0.100000in}{0.220728in}}{\pgfqpoint{3.696000in}{3.696000in}}%
\pgfusepath{clip}%
\pgfsetbuttcap%
\pgfsetroundjoin%
\definecolor{currentfill}{rgb}{0.121569,0.466667,0.705882}%
\pgfsetfillcolor{currentfill}%
\pgfsetfillopacity{0.794247}%
\pgfsetlinewidth{1.003750pt}%
\definecolor{currentstroke}{rgb}{0.121569,0.466667,0.705882}%
\pgfsetstrokecolor{currentstroke}%
\pgfsetstrokeopacity{0.794247}%
\pgfsetdash{}{0pt}%
\pgfpathmoveto{\pgfqpoint{1.530762in}{1.000308in}}%
\pgfpathcurveto{\pgfqpoint{1.538998in}{1.000308in}}{\pgfqpoint{1.546899in}{1.003580in}}{\pgfqpoint{1.552722in}{1.009404in}}%
\pgfpathcurveto{\pgfqpoint{1.558546in}{1.015228in}}{\pgfqpoint{1.561819in}{1.023128in}}{\pgfqpoint{1.561819in}{1.031364in}}%
\pgfpathcurveto{\pgfqpoint{1.561819in}{1.039600in}}{\pgfqpoint{1.558546in}{1.047500in}}{\pgfqpoint{1.552722in}{1.053324in}}%
\pgfpathcurveto{\pgfqpoint{1.546899in}{1.059148in}}{\pgfqpoint{1.538998in}{1.062421in}}{\pgfqpoint{1.530762in}{1.062421in}}%
\pgfpathcurveto{\pgfqpoint{1.522526in}{1.062421in}}{\pgfqpoint{1.514626in}{1.059148in}}{\pgfqpoint{1.508802in}{1.053324in}}%
\pgfpathcurveto{\pgfqpoint{1.502978in}{1.047500in}}{\pgfqpoint{1.499706in}{1.039600in}}{\pgfqpoint{1.499706in}{1.031364in}}%
\pgfpathcurveto{\pgfqpoint{1.499706in}{1.023128in}}{\pgfqpoint{1.502978in}{1.015228in}}{\pgfqpoint{1.508802in}{1.009404in}}%
\pgfpathcurveto{\pgfqpoint{1.514626in}{1.003580in}}{\pgfqpoint{1.522526in}{1.000308in}}{\pgfqpoint{1.530762in}{1.000308in}}%
\pgfpathclose%
\pgfusepath{stroke,fill}%
\end{pgfscope}%
\begin{pgfscope}%
\pgfpathrectangle{\pgfqpoint{0.100000in}{0.220728in}}{\pgfqpoint{3.696000in}{3.696000in}}%
\pgfusepath{clip}%
\pgfsetbuttcap%
\pgfsetroundjoin%
\definecolor{currentfill}{rgb}{0.121569,0.466667,0.705882}%
\pgfsetfillcolor{currentfill}%
\pgfsetfillopacity{0.796733}%
\pgfsetlinewidth{1.003750pt}%
\definecolor{currentstroke}{rgb}{0.121569,0.466667,0.705882}%
\pgfsetstrokecolor{currentstroke}%
\pgfsetstrokeopacity{0.796733}%
\pgfsetdash{}{0pt}%
\pgfpathmoveto{\pgfqpoint{1.545400in}{0.976519in}}%
\pgfpathcurveto{\pgfqpoint{1.553637in}{0.976519in}}{\pgfqpoint{1.561537in}{0.979792in}}{\pgfqpoint{1.567361in}{0.985616in}}%
\pgfpathcurveto{\pgfqpoint{1.573185in}{0.991439in}}{\pgfqpoint{1.576457in}{0.999340in}}{\pgfqpoint{1.576457in}{1.007576in}}%
\pgfpathcurveto{\pgfqpoint{1.576457in}{1.015812in}}{\pgfqpoint{1.573185in}{1.023712in}}{\pgfqpoint{1.567361in}{1.029536in}}%
\pgfpathcurveto{\pgfqpoint{1.561537in}{1.035360in}}{\pgfqpoint{1.553637in}{1.038632in}}{\pgfqpoint{1.545400in}{1.038632in}}%
\pgfpathcurveto{\pgfqpoint{1.537164in}{1.038632in}}{\pgfqpoint{1.529264in}{1.035360in}}{\pgfqpoint{1.523440in}{1.029536in}}%
\pgfpathcurveto{\pgfqpoint{1.517616in}{1.023712in}}{\pgfqpoint{1.514344in}{1.015812in}}{\pgfqpoint{1.514344in}{1.007576in}}%
\pgfpathcurveto{\pgfqpoint{1.514344in}{0.999340in}}{\pgfqpoint{1.517616in}{0.991439in}}{\pgfqpoint{1.523440in}{0.985616in}}%
\pgfpathcurveto{\pgfqpoint{1.529264in}{0.979792in}}{\pgfqpoint{1.537164in}{0.976519in}}{\pgfqpoint{1.545400in}{0.976519in}}%
\pgfpathclose%
\pgfusepath{stroke,fill}%
\end{pgfscope}%
\begin{pgfscope}%
\pgfpathrectangle{\pgfqpoint{0.100000in}{0.220728in}}{\pgfqpoint{3.696000in}{3.696000in}}%
\pgfusepath{clip}%
\pgfsetbuttcap%
\pgfsetroundjoin%
\definecolor{currentfill}{rgb}{0.121569,0.466667,0.705882}%
\pgfsetfillcolor{currentfill}%
\pgfsetfillopacity{0.796847}%
\pgfsetlinewidth{1.003750pt}%
\definecolor{currentstroke}{rgb}{0.121569,0.466667,0.705882}%
\pgfsetstrokecolor{currentstroke}%
\pgfsetstrokeopacity{0.796847}%
\pgfsetdash{}{0pt}%
\pgfpathmoveto{\pgfqpoint{1.542926in}{0.982189in}}%
\pgfpathcurveto{\pgfqpoint{1.551163in}{0.982189in}}{\pgfqpoint{1.559063in}{0.985462in}}{\pgfqpoint{1.564887in}{0.991286in}}%
\pgfpathcurveto{\pgfqpoint{1.570711in}{0.997110in}}{\pgfqpoint{1.573983in}{1.005010in}}{\pgfqpoint{1.573983in}{1.013246in}}%
\pgfpathcurveto{\pgfqpoint{1.573983in}{1.021482in}}{\pgfqpoint{1.570711in}{1.029382in}}{\pgfqpoint{1.564887in}{1.035206in}}%
\pgfpathcurveto{\pgfqpoint{1.559063in}{1.041030in}}{\pgfqpoint{1.551163in}{1.044302in}}{\pgfqpoint{1.542926in}{1.044302in}}%
\pgfpathcurveto{\pgfqpoint{1.534690in}{1.044302in}}{\pgfqpoint{1.526790in}{1.041030in}}{\pgfqpoint{1.520966in}{1.035206in}}%
\pgfpathcurveto{\pgfqpoint{1.515142in}{1.029382in}}{\pgfqpoint{1.511870in}{1.021482in}}{\pgfqpoint{1.511870in}{1.013246in}}%
\pgfpathcurveto{\pgfqpoint{1.511870in}{1.005010in}}{\pgfqpoint{1.515142in}{0.997110in}}{\pgfqpoint{1.520966in}{0.991286in}}%
\pgfpathcurveto{\pgfqpoint{1.526790in}{0.985462in}}{\pgfqpoint{1.534690in}{0.982189in}}{\pgfqpoint{1.542926in}{0.982189in}}%
\pgfpathclose%
\pgfusepath{stroke,fill}%
\end{pgfscope}%
\begin{pgfscope}%
\pgfpathrectangle{\pgfqpoint{0.100000in}{0.220728in}}{\pgfqpoint{3.696000in}{3.696000in}}%
\pgfusepath{clip}%
\pgfsetbuttcap%
\pgfsetroundjoin%
\definecolor{currentfill}{rgb}{0.121569,0.466667,0.705882}%
\pgfsetfillcolor{currentfill}%
\pgfsetfillopacity{0.797622}%
\pgfsetlinewidth{1.003750pt}%
\definecolor{currentstroke}{rgb}{0.121569,0.466667,0.705882}%
\pgfsetstrokecolor{currentstroke}%
\pgfsetstrokeopacity{0.797622}%
\pgfsetdash{}{0pt}%
\pgfpathmoveto{\pgfqpoint{1.537754in}{0.989589in}}%
\pgfpathcurveto{\pgfqpoint{1.545990in}{0.989589in}}{\pgfqpoint{1.553890in}{0.992862in}}{\pgfqpoint{1.559714in}{0.998686in}}%
\pgfpathcurveto{\pgfqpoint{1.565538in}{1.004510in}}{\pgfqpoint{1.568810in}{1.012410in}}{\pgfqpoint{1.568810in}{1.020646in}}%
\pgfpathcurveto{\pgfqpoint{1.568810in}{1.028882in}}{\pgfqpoint{1.565538in}{1.036782in}}{\pgfqpoint{1.559714in}{1.042606in}}%
\pgfpathcurveto{\pgfqpoint{1.553890in}{1.048430in}}{\pgfqpoint{1.545990in}{1.051702in}}{\pgfqpoint{1.537754in}{1.051702in}}%
\pgfpathcurveto{\pgfqpoint{1.529517in}{1.051702in}}{\pgfqpoint{1.521617in}{1.048430in}}{\pgfqpoint{1.515793in}{1.042606in}}%
\pgfpathcurveto{\pgfqpoint{1.509969in}{1.036782in}}{\pgfqpoint{1.506697in}{1.028882in}}{\pgfqpoint{1.506697in}{1.020646in}}%
\pgfpathcurveto{\pgfqpoint{1.506697in}{1.012410in}}{\pgfqpoint{1.509969in}{1.004510in}}{\pgfqpoint{1.515793in}{0.998686in}}%
\pgfpathcurveto{\pgfqpoint{1.521617in}{0.992862in}}{\pgfqpoint{1.529517in}{0.989589in}}{\pgfqpoint{1.537754in}{0.989589in}}%
\pgfpathclose%
\pgfusepath{stroke,fill}%
\end{pgfscope}%
\begin{pgfscope}%
\pgfpathrectangle{\pgfqpoint{0.100000in}{0.220728in}}{\pgfqpoint{3.696000in}{3.696000in}}%
\pgfusepath{clip}%
\pgfsetbuttcap%
\pgfsetroundjoin%
\definecolor{currentfill}{rgb}{0.121569,0.466667,0.705882}%
\pgfsetfillcolor{currentfill}%
\pgfsetfillopacity{0.811165}%
\pgfsetlinewidth{1.003750pt}%
\definecolor{currentstroke}{rgb}{0.121569,0.466667,0.705882}%
\pgfsetstrokecolor{currentstroke}%
\pgfsetstrokeopacity{0.811165}%
\pgfsetdash{}{0pt}%
\pgfpathmoveto{\pgfqpoint{1.517960in}{1.023685in}}%
\pgfpathcurveto{\pgfqpoint{1.526197in}{1.023685in}}{\pgfqpoint{1.534097in}{1.026958in}}{\pgfqpoint{1.539921in}{1.032782in}}%
\pgfpathcurveto{\pgfqpoint{1.545745in}{1.038606in}}{\pgfqpoint{1.549017in}{1.046506in}}{\pgfqpoint{1.549017in}{1.054742in}}%
\pgfpathcurveto{\pgfqpoint{1.549017in}{1.062978in}}{\pgfqpoint{1.545745in}{1.070878in}}{\pgfqpoint{1.539921in}{1.076702in}}%
\pgfpathcurveto{\pgfqpoint{1.534097in}{1.082526in}}{\pgfqpoint{1.526197in}{1.085798in}}{\pgfqpoint{1.517960in}{1.085798in}}%
\pgfpathcurveto{\pgfqpoint{1.509724in}{1.085798in}}{\pgfqpoint{1.501824in}{1.082526in}}{\pgfqpoint{1.496000in}{1.076702in}}%
\pgfpathcurveto{\pgfqpoint{1.490176in}{1.070878in}}{\pgfqpoint{1.486904in}{1.062978in}}{\pgfqpoint{1.486904in}{1.054742in}}%
\pgfpathcurveto{\pgfqpoint{1.486904in}{1.046506in}}{\pgfqpoint{1.490176in}{1.038606in}}{\pgfqpoint{1.496000in}{1.032782in}}%
\pgfpathcurveto{\pgfqpoint{1.501824in}{1.026958in}}{\pgfqpoint{1.509724in}{1.023685in}}{\pgfqpoint{1.517960in}{1.023685in}}%
\pgfpathclose%
\pgfusepath{stroke,fill}%
\end{pgfscope}%
\begin{pgfscope}%
\pgfpathrectangle{\pgfqpoint{0.100000in}{0.220728in}}{\pgfqpoint{3.696000in}{3.696000in}}%
\pgfusepath{clip}%
\pgfsetbuttcap%
\pgfsetroundjoin%
\definecolor{currentfill}{rgb}{0.121569,0.466667,0.705882}%
\pgfsetfillcolor{currentfill}%
\pgfsetfillopacity{0.811556}%
\pgfsetlinewidth{1.003750pt}%
\definecolor{currentstroke}{rgb}{0.121569,0.466667,0.705882}%
\pgfsetstrokecolor{currentstroke}%
\pgfsetstrokeopacity{0.811556}%
\pgfsetdash{}{0pt}%
\pgfpathmoveto{\pgfqpoint{1.512132in}{1.031224in}}%
\pgfpathcurveto{\pgfqpoint{1.520368in}{1.031224in}}{\pgfqpoint{1.528268in}{1.034496in}}{\pgfqpoint{1.534092in}{1.040320in}}%
\pgfpathcurveto{\pgfqpoint{1.539916in}{1.046144in}}{\pgfqpoint{1.543188in}{1.054044in}}{\pgfqpoint{1.543188in}{1.062280in}}%
\pgfpathcurveto{\pgfqpoint{1.543188in}{1.070517in}}{\pgfqpoint{1.539916in}{1.078417in}}{\pgfqpoint{1.534092in}{1.084241in}}%
\pgfpathcurveto{\pgfqpoint{1.528268in}{1.090065in}}{\pgfqpoint{1.520368in}{1.093337in}}{\pgfqpoint{1.512132in}{1.093337in}}%
\pgfpathcurveto{\pgfqpoint{1.503895in}{1.093337in}}{\pgfqpoint{1.495995in}{1.090065in}}{\pgfqpoint{1.490171in}{1.084241in}}%
\pgfpathcurveto{\pgfqpoint{1.484347in}{1.078417in}}{\pgfqpoint{1.481075in}{1.070517in}}{\pgfqpoint{1.481075in}{1.062280in}}%
\pgfpathcurveto{\pgfqpoint{1.481075in}{1.054044in}}{\pgfqpoint{1.484347in}{1.046144in}}{\pgfqpoint{1.490171in}{1.040320in}}%
\pgfpathcurveto{\pgfqpoint{1.495995in}{1.034496in}}{\pgfqpoint{1.503895in}{1.031224in}}{\pgfqpoint{1.512132in}{1.031224in}}%
\pgfpathclose%
\pgfusepath{stroke,fill}%
\end{pgfscope}%
\begin{pgfscope}%
\pgfpathrectangle{\pgfqpoint{0.100000in}{0.220728in}}{\pgfqpoint{3.696000in}{3.696000in}}%
\pgfusepath{clip}%
\pgfsetbuttcap%
\pgfsetroundjoin%
\definecolor{currentfill}{rgb}{0.121569,0.466667,0.705882}%
\pgfsetfillcolor{currentfill}%
\pgfsetfillopacity{0.811651}%
\pgfsetlinewidth{1.003750pt}%
\definecolor{currentstroke}{rgb}{0.121569,0.466667,0.705882}%
\pgfsetstrokecolor{currentstroke}%
\pgfsetstrokeopacity{0.811651}%
\pgfsetdash{}{0pt}%
\pgfpathmoveto{\pgfqpoint{1.509693in}{1.037678in}}%
\pgfpathcurveto{\pgfqpoint{1.517929in}{1.037678in}}{\pgfqpoint{1.525830in}{1.040950in}}{\pgfqpoint{1.531653in}{1.046774in}}%
\pgfpathcurveto{\pgfqpoint{1.537477in}{1.052598in}}{\pgfqpoint{1.540750in}{1.060498in}}{\pgfqpoint{1.540750in}{1.068735in}}%
\pgfpathcurveto{\pgfqpoint{1.540750in}{1.076971in}}{\pgfqpoint{1.537477in}{1.084871in}}{\pgfqpoint{1.531653in}{1.090695in}}%
\pgfpathcurveto{\pgfqpoint{1.525830in}{1.096519in}}{\pgfqpoint{1.517929in}{1.099791in}}{\pgfqpoint{1.509693in}{1.099791in}}%
\pgfpathcurveto{\pgfqpoint{1.501457in}{1.099791in}}{\pgfqpoint{1.493557in}{1.096519in}}{\pgfqpoint{1.487733in}{1.090695in}}%
\pgfpathcurveto{\pgfqpoint{1.481909in}{1.084871in}}{\pgfqpoint{1.478637in}{1.076971in}}{\pgfqpoint{1.478637in}{1.068735in}}%
\pgfpathcurveto{\pgfqpoint{1.478637in}{1.060498in}}{\pgfqpoint{1.481909in}{1.052598in}}{\pgfqpoint{1.487733in}{1.046774in}}%
\pgfpathcurveto{\pgfqpoint{1.493557in}{1.040950in}}{\pgfqpoint{1.501457in}{1.037678in}}{\pgfqpoint{1.509693in}{1.037678in}}%
\pgfpathclose%
\pgfusepath{stroke,fill}%
\end{pgfscope}%
\begin{pgfscope}%
\pgfpathrectangle{\pgfqpoint{0.100000in}{0.220728in}}{\pgfqpoint{3.696000in}{3.696000in}}%
\pgfusepath{clip}%
\pgfsetbuttcap%
\pgfsetroundjoin%
\definecolor{currentfill}{rgb}{0.121569,0.466667,0.705882}%
\pgfsetfillcolor{currentfill}%
\pgfsetfillopacity{0.819559}%
\pgfsetlinewidth{1.003750pt}%
\definecolor{currentstroke}{rgb}{0.121569,0.466667,0.705882}%
\pgfsetstrokecolor{currentstroke}%
\pgfsetstrokeopacity{0.819559}%
\pgfsetdash{}{0pt}%
\pgfpathmoveto{\pgfqpoint{1.497121in}{1.064765in}}%
\pgfpathcurveto{\pgfqpoint{1.505357in}{1.064765in}}{\pgfqpoint{1.513257in}{1.068037in}}{\pgfqpoint{1.519081in}{1.073861in}}%
\pgfpathcurveto{\pgfqpoint{1.524905in}{1.079685in}}{\pgfqpoint{1.528177in}{1.087585in}}{\pgfqpoint{1.528177in}{1.095821in}}%
\pgfpathcurveto{\pgfqpoint{1.528177in}{1.104058in}}{\pgfqpoint{1.524905in}{1.111958in}}{\pgfqpoint{1.519081in}{1.117782in}}%
\pgfpathcurveto{\pgfqpoint{1.513257in}{1.123606in}}{\pgfqpoint{1.505357in}{1.126878in}}{\pgfqpoint{1.497121in}{1.126878in}}%
\pgfpathcurveto{\pgfqpoint{1.488885in}{1.126878in}}{\pgfqpoint{1.480985in}{1.123606in}}{\pgfqpoint{1.475161in}{1.117782in}}%
\pgfpathcurveto{\pgfqpoint{1.469337in}{1.111958in}}{\pgfqpoint{1.466064in}{1.104058in}}{\pgfqpoint{1.466064in}{1.095821in}}%
\pgfpathcurveto{\pgfqpoint{1.466064in}{1.087585in}}{\pgfqpoint{1.469337in}{1.079685in}}{\pgfqpoint{1.475161in}{1.073861in}}%
\pgfpathcurveto{\pgfqpoint{1.480985in}{1.068037in}}{\pgfqpoint{1.488885in}{1.064765in}}{\pgfqpoint{1.497121in}{1.064765in}}%
\pgfpathclose%
\pgfusepath{stroke,fill}%
\end{pgfscope}%
\begin{pgfscope}%
\pgfpathrectangle{\pgfqpoint{0.100000in}{0.220728in}}{\pgfqpoint{3.696000in}{3.696000in}}%
\pgfusepath{clip}%
\pgfsetbuttcap%
\pgfsetroundjoin%
\definecolor{currentfill}{rgb}{0.121569,0.466667,0.705882}%
\pgfsetfillcolor{currentfill}%
\pgfsetfillopacity{0.819592}%
\pgfsetlinewidth{1.003750pt}%
\definecolor{currentstroke}{rgb}{0.121569,0.466667,0.705882}%
\pgfsetstrokecolor{currentstroke}%
\pgfsetstrokeopacity{0.819592}%
\pgfsetdash{}{0pt}%
\pgfpathmoveto{\pgfqpoint{1.457397in}{1.142135in}}%
\pgfpathcurveto{\pgfqpoint{1.465633in}{1.142135in}}{\pgfqpoint{1.473533in}{1.145408in}}{\pgfqpoint{1.479357in}{1.151232in}}%
\pgfpathcurveto{\pgfqpoint{1.485181in}{1.157055in}}{\pgfqpoint{1.488453in}{1.164956in}}{\pgfqpoint{1.488453in}{1.173192in}}%
\pgfpathcurveto{\pgfqpoint{1.488453in}{1.181428in}}{\pgfqpoint{1.485181in}{1.189328in}}{\pgfqpoint{1.479357in}{1.195152in}}%
\pgfpathcurveto{\pgfqpoint{1.473533in}{1.200976in}}{\pgfqpoint{1.465633in}{1.204248in}}{\pgfqpoint{1.457397in}{1.204248in}}%
\pgfpathcurveto{\pgfqpoint{1.449160in}{1.204248in}}{\pgfqpoint{1.441260in}{1.200976in}}{\pgfqpoint{1.435436in}{1.195152in}}%
\pgfpathcurveto{\pgfqpoint{1.429612in}{1.189328in}}{\pgfqpoint{1.426340in}{1.181428in}}{\pgfqpoint{1.426340in}{1.173192in}}%
\pgfpathcurveto{\pgfqpoint{1.426340in}{1.164956in}}{\pgfqpoint{1.429612in}{1.157055in}}{\pgfqpoint{1.435436in}{1.151232in}}%
\pgfpathcurveto{\pgfqpoint{1.441260in}{1.145408in}}{\pgfqpoint{1.449160in}{1.142135in}}{\pgfqpoint{1.457397in}{1.142135in}}%
\pgfpathclose%
\pgfusepath{stroke,fill}%
\end{pgfscope}%
\begin{pgfscope}%
\pgfpathrectangle{\pgfqpoint{0.100000in}{0.220728in}}{\pgfqpoint{3.696000in}{3.696000in}}%
\pgfusepath{clip}%
\pgfsetbuttcap%
\pgfsetroundjoin%
\definecolor{currentfill}{rgb}{0.121569,0.466667,0.705882}%
\pgfsetfillcolor{currentfill}%
\pgfsetfillopacity{0.821114}%
\pgfsetlinewidth{1.003750pt}%
\definecolor{currentstroke}{rgb}{0.121569,0.466667,0.705882}%
\pgfsetstrokecolor{currentstroke}%
\pgfsetstrokeopacity{0.821114}%
\pgfsetdash{}{0pt}%
\pgfpathmoveto{\pgfqpoint{1.503814in}{1.048288in}}%
\pgfpathcurveto{\pgfqpoint{1.512050in}{1.048288in}}{\pgfqpoint{1.519950in}{1.051560in}}{\pgfqpoint{1.525774in}{1.057384in}}%
\pgfpathcurveto{\pgfqpoint{1.531598in}{1.063208in}}{\pgfqpoint{1.534870in}{1.071108in}}{\pgfqpoint{1.534870in}{1.079344in}}%
\pgfpathcurveto{\pgfqpoint{1.534870in}{1.087580in}}{\pgfqpoint{1.531598in}{1.095480in}}{\pgfqpoint{1.525774in}{1.101304in}}%
\pgfpathcurveto{\pgfqpoint{1.519950in}{1.107128in}}{\pgfqpoint{1.512050in}{1.110401in}}{\pgfqpoint{1.503814in}{1.110401in}}%
\pgfpathcurveto{\pgfqpoint{1.495578in}{1.110401in}}{\pgfqpoint{1.487678in}{1.107128in}}{\pgfqpoint{1.481854in}{1.101304in}}%
\pgfpathcurveto{\pgfqpoint{1.476030in}{1.095480in}}{\pgfqpoint{1.472757in}{1.087580in}}{\pgfqpoint{1.472757in}{1.079344in}}%
\pgfpathcurveto{\pgfqpoint{1.472757in}{1.071108in}}{\pgfqpoint{1.476030in}{1.063208in}}{\pgfqpoint{1.481854in}{1.057384in}}%
\pgfpathcurveto{\pgfqpoint{1.487678in}{1.051560in}}{\pgfqpoint{1.495578in}{1.048288in}}{\pgfqpoint{1.503814in}{1.048288in}}%
\pgfpathclose%
\pgfusepath{stroke,fill}%
\end{pgfscope}%
\begin{pgfscope}%
\pgfpathrectangle{\pgfqpoint{0.100000in}{0.220728in}}{\pgfqpoint{3.696000in}{3.696000in}}%
\pgfusepath{clip}%
\pgfsetbuttcap%
\pgfsetroundjoin%
\definecolor{currentfill}{rgb}{0.121569,0.466667,0.705882}%
\pgfsetfillcolor{currentfill}%
\pgfsetfillopacity{0.826181}%
\pgfsetlinewidth{1.003750pt}%
\definecolor{currentstroke}{rgb}{0.121569,0.466667,0.705882}%
\pgfsetstrokecolor{currentstroke}%
\pgfsetstrokeopacity{0.826181}%
\pgfsetdash{}{0pt}%
\pgfpathmoveto{\pgfqpoint{1.433727in}{1.170435in}}%
\pgfpathcurveto{\pgfqpoint{1.441964in}{1.170435in}}{\pgfqpoint{1.449864in}{1.173708in}}{\pgfqpoint{1.455688in}{1.179531in}}%
\pgfpathcurveto{\pgfqpoint{1.461512in}{1.185355in}}{\pgfqpoint{1.464784in}{1.193255in}}{\pgfqpoint{1.464784in}{1.201492in}}%
\pgfpathcurveto{\pgfqpoint{1.464784in}{1.209728in}}{\pgfqpoint{1.461512in}{1.217628in}}{\pgfqpoint{1.455688in}{1.223452in}}%
\pgfpathcurveto{\pgfqpoint{1.449864in}{1.229276in}}{\pgfqpoint{1.441964in}{1.232548in}}{\pgfqpoint{1.433727in}{1.232548in}}%
\pgfpathcurveto{\pgfqpoint{1.425491in}{1.232548in}}{\pgfqpoint{1.417591in}{1.229276in}}{\pgfqpoint{1.411767in}{1.223452in}}%
\pgfpathcurveto{\pgfqpoint{1.405943in}{1.217628in}}{\pgfqpoint{1.402671in}{1.209728in}}{\pgfqpoint{1.402671in}{1.201492in}}%
\pgfpathcurveto{\pgfqpoint{1.402671in}{1.193255in}}{\pgfqpoint{1.405943in}{1.185355in}}{\pgfqpoint{1.411767in}{1.179531in}}%
\pgfpathcurveto{\pgfqpoint{1.417591in}{1.173708in}}{\pgfqpoint{1.425491in}{1.170435in}}{\pgfqpoint{1.433727in}{1.170435in}}%
\pgfpathclose%
\pgfusepath{stroke,fill}%
\end{pgfscope}%
\begin{pgfscope}%
\pgfpathrectangle{\pgfqpoint{0.100000in}{0.220728in}}{\pgfqpoint{3.696000in}{3.696000in}}%
\pgfusepath{clip}%
\pgfsetbuttcap%
\pgfsetroundjoin%
\definecolor{currentfill}{rgb}{0.121569,0.466667,0.705882}%
\pgfsetfillcolor{currentfill}%
\pgfsetfillopacity{0.827614}%
\pgfsetlinewidth{1.003750pt}%
\definecolor{currentstroke}{rgb}{0.121569,0.466667,0.705882}%
\pgfsetstrokecolor{currentstroke}%
\pgfsetstrokeopacity{0.827614}%
\pgfsetdash{}{0pt}%
\pgfpathmoveto{\pgfqpoint{1.462465in}{1.141026in}}%
\pgfpathcurveto{\pgfqpoint{1.470701in}{1.141026in}}{\pgfqpoint{1.478602in}{1.144299in}}{\pgfqpoint{1.484425in}{1.150123in}}%
\pgfpathcurveto{\pgfqpoint{1.490249in}{1.155947in}}{\pgfqpoint{1.493522in}{1.163847in}}{\pgfqpoint{1.493522in}{1.172083in}}%
\pgfpathcurveto{\pgfqpoint{1.493522in}{1.180319in}}{\pgfqpoint{1.490249in}{1.188219in}}{\pgfqpoint{1.484425in}{1.194043in}}%
\pgfpathcurveto{\pgfqpoint{1.478602in}{1.199867in}}{\pgfqpoint{1.470701in}{1.203139in}}{\pgfqpoint{1.462465in}{1.203139in}}%
\pgfpathcurveto{\pgfqpoint{1.454229in}{1.203139in}}{\pgfqpoint{1.446329in}{1.199867in}}{\pgfqpoint{1.440505in}{1.194043in}}%
\pgfpathcurveto{\pgfqpoint{1.434681in}{1.188219in}}{\pgfqpoint{1.431409in}{1.180319in}}{\pgfqpoint{1.431409in}{1.172083in}}%
\pgfpathcurveto{\pgfqpoint{1.431409in}{1.163847in}}{\pgfqpoint{1.434681in}{1.155947in}}{\pgfqpoint{1.440505in}{1.150123in}}%
\pgfpathcurveto{\pgfqpoint{1.446329in}{1.144299in}}{\pgfqpoint{1.454229in}{1.141026in}}{\pgfqpoint{1.462465in}{1.141026in}}%
\pgfpathclose%
\pgfusepath{stroke,fill}%
\end{pgfscope}%
\begin{pgfscope}%
\pgfpathrectangle{\pgfqpoint{0.100000in}{0.220728in}}{\pgfqpoint{3.696000in}{3.696000in}}%
\pgfusepath{clip}%
\pgfsetbuttcap%
\pgfsetroundjoin%
\definecolor{currentfill}{rgb}{0.121569,0.466667,0.705882}%
\pgfsetfillcolor{currentfill}%
\pgfsetfillopacity{0.833406}%
\pgfsetlinewidth{1.003750pt}%
\definecolor{currentstroke}{rgb}{0.121569,0.466667,0.705882}%
\pgfsetstrokecolor{currentstroke}%
\pgfsetstrokeopacity{0.833406}%
\pgfsetdash{}{0pt}%
\pgfpathmoveto{\pgfqpoint{1.472650in}{1.124263in}}%
\pgfpathcurveto{\pgfqpoint{1.480886in}{1.124263in}}{\pgfqpoint{1.488786in}{1.127535in}}{\pgfqpoint{1.494610in}{1.133359in}}%
\pgfpathcurveto{\pgfqpoint{1.500434in}{1.139183in}}{\pgfqpoint{1.503706in}{1.147083in}}{\pgfqpoint{1.503706in}{1.155319in}}%
\pgfpathcurveto{\pgfqpoint{1.503706in}{1.163556in}}{\pgfqpoint{1.500434in}{1.171456in}}{\pgfqpoint{1.494610in}{1.177280in}}%
\pgfpathcurveto{\pgfqpoint{1.488786in}{1.183104in}}{\pgfqpoint{1.480886in}{1.186376in}}{\pgfqpoint{1.472650in}{1.186376in}}%
\pgfpathcurveto{\pgfqpoint{1.464413in}{1.186376in}}{\pgfqpoint{1.456513in}{1.183104in}}{\pgfqpoint{1.450689in}{1.177280in}}%
\pgfpathcurveto{\pgfqpoint{1.444865in}{1.171456in}}{\pgfqpoint{1.441593in}{1.163556in}}{\pgfqpoint{1.441593in}{1.155319in}}%
\pgfpathcurveto{\pgfqpoint{1.441593in}{1.147083in}}{\pgfqpoint{1.444865in}{1.139183in}}{\pgfqpoint{1.450689in}{1.133359in}}%
\pgfpathcurveto{\pgfqpoint{1.456513in}{1.127535in}}{\pgfqpoint{1.464413in}{1.124263in}}{\pgfqpoint{1.472650in}{1.124263in}}%
\pgfpathclose%
\pgfusepath{stroke,fill}%
\end{pgfscope}%
\begin{pgfscope}%
\pgfpathrectangle{\pgfqpoint{0.100000in}{0.220728in}}{\pgfqpoint{3.696000in}{3.696000in}}%
\pgfusepath{clip}%
\pgfsetbuttcap%
\pgfsetroundjoin%
\definecolor{currentfill}{rgb}{0.121569,0.466667,0.705882}%
\pgfsetfillcolor{currentfill}%
\pgfsetfillopacity{0.835104}%
\pgfsetlinewidth{1.003750pt}%
\definecolor{currentstroke}{rgb}{0.121569,0.466667,0.705882}%
\pgfsetstrokecolor{currentstroke}%
\pgfsetstrokeopacity{0.835104}%
\pgfsetdash{}{0pt}%
\pgfpathmoveto{\pgfqpoint{1.484525in}{1.098784in}}%
\pgfpathcurveto{\pgfqpoint{1.492761in}{1.098784in}}{\pgfqpoint{1.500661in}{1.102056in}}{\pgfqpoint{1.506485in}{1.107880in}}%
\pgfpathcurveto{\pgfqpoint{1.512309in}{1.113704in}}{\pgfqpoint{1.515581in}{1.121604in}}{\pgfqpoint{1.515581in}{1.129840in}}%
\pgfpathcurveto{\pgfqpoint{1.515581in}{1.138076in}}{\pgfqpoint{1.512309in}{1.145976in}}{\pgfqpoint{1.506485in}{1.151800in}}%
\pgfpathcurveto{\pgfqpoint{1.500661in}{1.157624in}}{\pgfqpoint{1.492761in}{1.160897in}}{\pgfqpoint{1.484525in}{1.160897in}}%
\pgfpathcurveto{\pgfqpoint{1.476289in}{1.160897in}}{\pgfqpoint{1.468389in}{1.157624in}}{\pgfqpoint{1.462565in}{1.151800in}}%
\pgfpathcurveto{\pgfqpoint{1.456741in}{1.145976in}}{\pgfqpoint{1.453468in}{1.138076in}}{\pgfqpoint{1.453468in}{1.129840in}}%
\pgfpathcurveto{\pgfqpoint{1.453468in}{1.121604in}}{\pgfqpoint{1.456741in}{1.113704in}}{\pgfqpoint{1.462565in}{1.107880in}}%
\pgfpathcurveto{\pgfqpoint{1.468389in}{1.102056in}}{\pgfqpoint{1.476289in}{1.098784in}}{\pgfqpoint{1.484525in}{1.098784in}}%
\pgfpathclose%
\pgfusepath{stroke,fill}%
\end{pgfscope}%
\begin{pgfscope}%
\pgfpathrectangle{\pgfqpoint{0.100000in}{0.220728in}}{\pgfqpoint{3.696000in}{3.696000in}}%
\pgfusepath{clip}%
\pgfsetbuttcap%
\pgfsetroundjoin%
\definecolor{currentfill}{rgb}{0.121569,0.466667,0.705882}%
\pgfsetfillcolor{currentfill}%
\pgfsetfillopacity{0.835169}%
\pgfsetlinewidth{1.003750pt}%
\definecolor{currentstroke}{rgb}{0.121569,0.466667,0.705882}%
\pgfsetstrokecolor{currentstroke}%
\pgfsetstrokeopacity{0.835169}%
\pgfsetdash{}{0pt}%
\pgfpathmoveto{\pgfqpoint{1.445382in}{1.157856in}}%
\pgfpathcurveto{\pgfqpoint{1.453618in}{1.157856in}}{\pgfqpoint{1.461518in}{1.161129in}}{\pgfqpoint{1.467342in}{1.166953in}}%
\pgfpathcurveto{\pgfqpoint{1.473166in}{1.172777in}}{\pgfqpoint{1.476438in}{1.180677in}}{\pgfqpoint{1.476438in}{1.188913in}}%
\pgfpathcurveto{\pgfqpoint{1.476438in}{1.197149in}}{\pgfqpoint{1.473166in}{1.205049in}}{\pgfqpoint{1.467342in}{1.210873in}}%
\pgfpathcurveto{\pgfqpoint{1.461518in}{1.216697in}}{\pgfqpoint{1.453618in}{1.219969in}}{\pgfqpoint{1.445382in}{1.219969in}}%
\pgfpathcurveto{\pgfqpoint{1.437146in}{1.219969in}}{\pgfqpoint{1.429246in}{1.216697in}}{\pgfqpoint{1.423422in}{1.210873in}}%
\pgfpathcurveto{\pgfqpoint{1.417598in}{1.205049in}}{\pgfqpoint{1.414325in}{1.197149in}}{\pgfqpoint{1.414325in}{1.188913in}}%
\pgfpathcurveto{\pgfqpoint{1.414325in}{1.180677in}}{\pgfqpoint{1.417598in}{1.172777in}}{\pgfqpoint{1.423422in}{1.166953in}}%
\pgfpathcurveto{\pgfqpoint{1.429246in}{1.161129in}}{\pgfqpoint{1.437146in}{1.157856in}}{\pgfqpoint{1.445382in}{1.157856in}}%
\pgfpathclose%
\pgfusepath{stroke,fill}%
\end{pgfscope}%
\begin{pgfscope}%
\pgfpathrectangle{\pgfqpoint{0.100000in}{0.220728in}}{\pgfqpoint{3.696000in}{3.696000in}}%
\pgfusepath{clip}%
\pgfsetbuttcap%
\pgfsetroundjoin%
\definecolor{currentfill}{rgb}{0.121569,0.466667,0.705882}%
\pgfsetfillcolor{currentfill}%
\pgfsetfillopacity{0.835303}%
\pgfsetlinewidth{1.003750pt}%
\definecolor{currentstroke}{rgb}{0.121569,0.466667,0.705882}%
\pgfsetstrokecolor{currentstroke}%
\pgfsetstrokeopacity{0.835303}%
\pgfsetdash{}{0pt}%
\pgfpathmoveto{\pgfqpoint{1.466142in}{1.141441in}}%
\pgfpathcurveto{\pgfqpoint{1.474378in}{1.141441in}}{\pgfqpoint{1.482278in}{1.144714in}}{\pgfqpoint{1.488102in}{1.150538in}}%
\pgfpathcurveto{\pgfqpoint{1.493926in}{1.156362in}}{\pgfqpoint{1.497198in}{1.164262in}}{\pgfqpoint{1.497198in}{1.172498in}}%
\pgfpathcurveto{\pgfqpoint{1.497198in}{1.180734in}}{\pgfqpoint{1.493926in}{1.188634in}}{\pgfqpoint{1.488102in}{1.194458in}}%
\pgfpathcurveto{\pgfqpoint{1.482278in}{1.200282in}}{\pgfqpoint{1.474378in}{1.203554in}}{\pgfqpoint{1.466142in}{1.203554in}}%
\pgfpathcurveto{\pgfqpoint{1.457905in}{1.203554in}}{\pgfqpoint{1.450005in}{1.200282in}}{\pgfqpoint{1.444181in}{1.194458in}}%
\pgfpathcurveto{\pgfqpoint{1.438357in}{1.188634in}}{\pgfqpoint{1.435085in}{1.180734in}}{\pgfqpoint{1.435085in}{1.172498in}}%
\pgfpathcurveto{\pgfqpoint{1.435085in}{1.164262in}}{\pgfqpoint{1.438357in}{1.156362in}}{\pgfqpoint{1.444181in}{1.150538in}}%
\pgfpathcurveto{\pgfqpoint{1.450005in}{1.144714in}}{\pgfqpoint{1.457905in}{1.141441in}}{\pgfqpoint{1.466142in}{1.141441in}}%
\pgfpathclose%
\pgfusepath{stroke,fill}%
\end{pgfscope}%
\begin{pgfscope}%
\pgfpathrectangle{\pgfqpoint{0.100000in}{0.220728in}}{\pgfqpoint{3.696000in}{3.696000in}}%
\pgfusepath{clip}%
\pgfsetbuttcap%
\pgfsetroundjoin%
\definecolor{currentfill}{rgb}{0.121569,0.466667,0.705882}%
\pgfsetfillcolor{currentfill}%
\pgfsetfillopacity{0.905092}%
\pgfsetlinewidth{1.003750pt}%
\definecolor{currentstroke}{rgb}{0.121569,0.466667,0.705882}%
\pgfsetstrokecolor{currentstroke}%
\pgfsetstrokeopacity{0.905092}%
\pgfsetdash{}{0pt}%
\pgfpathmoveto{\pgfqpoint{0.641709in}{2.518035in}}%
\pgfpathcurveto{\pgfqpoint{0.649946in}{2.518035in}}{\pgfqpoint{0.657846in}{2.521307in}}{\pgfqpoint{0.663670in}{2.527131in}}%
\pgfpathcurveto{\pgfqpoint{0.669494in}{2.532955in}}{\pgfqpoint{0.672766in}{2.540855in}}{\pgfqpoint{0.672766in}{2.549091in}}%
\pgfpathcurveto{\pgfqpoint{0.672766in}{2.557328in}}{\pgfqpoint{0.669494in}{2.565228in}}{\pgfqpoint{0.663670in}{2.571052in}}%
\pgfpathcurveto{\pgfqpoint{0.657846in}{2.576875in}}{\pgfqpoint{0.649946in}{2.580148in}}{\pgfqpoint{0.641709in}{2.580148in}}%
\pgfpathcurveto{\pgfqpoint{0.633473in}{2.580148in}}{\pgfqpoint{0.625573in}{2.576875in}}{\pgfqpoint{0.619749in}{2.571052in}}%
\pgfpathcurveto{\pgfqpoint{0.613925in}{2.565228in}}{\pgfqpoint{0.610653in}{2.557328in}}{\pgfqpoint{0.610653in}{2.549091in}}%
\pgfpathcurveto{\pgfqpoint{0.610653in}{2.540855in}}{\pgfqpoint{0.613925in}{2.532955in}}{\pgfqpoint{0.619749in}{2.527131in}}%
\pgfpathcurveto{\pgfqpoint{0.625573in}{2.521307in}}{\pgfqpoint{0.633473in}{2.518035in}}{\pgfqpoint{0.641709in}{2.518035in}}%
\pgfpathclose%
\pgfusepath{stroke,fill}%
\end{pgfscope}%
\begin{pgfscope}%
\pgfpathrectangle{\pgfqpoint{0.100000in}{0.220728in}}{\pgfqpoint{3.696000in}{3.696000in}}%
\pgfusepath{clip}%
\pgfsetbuttcap%
\pgfsetroundjoin%
\definecolor{currentfill}{rgb}{0.121569,0.466667,0.705882}%
\pgfsetfillcolor{currentfill}%
\pgfsetlinewidth{1.003750pt}%
\definecolor{currentstroke}{rgb}{0.121569,0.466667,0.705882}%
\pgfsetstrokecolor{currentstroke}%
\pgfsetdash{}{0pt}%
\pgfpathmoveto{\pgfqpoint{0.637390in}{2.671899in}}%
\pgfpathcurveto{\pgfqpoint{0.645626in}{2.671899in}}{\pgfqpoint{0.653526in}{2.675171in}}{\pgfqpoint{0.659350in}{2.680995in}}%
\pgfpathcurveto{\pgfqpoint{0.665174in}{2.686819in}}{\pgfqpoint{0.668446in}{2.694719in}}{\pgfqpoint{0.668446in}{2.702955in}}%
\pgfpathcurveto{\pgfqpoint{0.668446in}{2.711191in}}{\pgfqpoint{0.665174in}{2.719092in}}{\pgfqpoint{0.659350in}{2.724915in}}%
\pgfpathcurveto{\pgfqpoint{0.653526in}{2.730739in}}{\pgfqpoint{0.645626in}{2.734012in}}{\pgfqpoint{0.637390in}{2.734012in}}%
\pgfpathcurveto{\pgfqpoint{0.629153in}{2.734012in}}{\pgfqpoint{0.621253in}{2.730739in}}{\pgfqpoint{0.615429in}{2.724915in}}%
\pgfpathcurveto{\pgfqpoint{0.609606in}{2.719092in}}{\pgfqpoint{0.606333in}{2.711191in}}{\pgfqpoint{0.606333in}{2.702955in}}%
\pgfpathcurveto{\pgfqpoint{0.606333in}{2.694719in}}{\pgfqpoint{0.609606in}{2.686819in}}{\pgfqpoint{0.615429in}{2.680995in}}%
\pgfpathcurveto{\pgfqpoint{0.621253in}{2.675171in}}{\pgfqpoint{0.629153in}{2.671899in}}{\pgfqpoint{0.637390in}{2.671899in}}%
\pgfpathclose%
\pgfusepath{stroke,fill}%
\end{pgfscope}%
\begin{pgfscope}%
\pgfsetbuttcap%
\pgfsetmiterjoin%
\definecolor{currentfill}{rgb}{1.000000,1.000000,1.000000}%
\pgfsetfillcolor{currentfill}%
\pgfsetfillopacity{0.800000}%
\pgfsetlinewidth{1.003750pt}%
\definecolor{currentstroke}{rgb}{0.800000,0.800000,0.800000}%
\pgfsetstrokecolor{currentstroke}%
\pgfsetstrokeopacity{0.800000}%
\pgfsetdash{}{0pt}%
\pgfpathmoveto{\pgfqpoint{1.958421in}{3.397902in}}%
\pgfpathlineto{\pgfqpoint{3.698778in}{3.397902in}}%
\pgfpathquadraticcurveto{\pgfqpoint{3.726556in}{3.397902in}}{\pgfqpoint{3.726556in}{3.425680in}}%
\pgfpathlineto{\pgfqpoint{3.726556in}{3.819506in}}%
\pgfpathquadraticcurveto{\pgfqpoint{3.726556in}{3.847284in}}{\pgfqpoint{3.698778in}{3.847284in}}%
\pgfpathlineto{\pgfqpoint{1.958421in}{3.847284in}}%
\pgfpathquadraticcurveto{\pgfqpoint{1.930644in}{3.847284in}}{\pgfqpoint{1.930644in}{3.819506in}}%
\pgfpathlineto{\pgfqpoint{1.930644in}{3.425680in}}%
\pgfpathquadraticcurveto{\pgfqpoint{1.930644in}{3.397902in}}{\pgfqpoint{1.958421in}{3.397902in}}%
\pgfpathclose%
\pgfusepath{stroke,fill}%
\end{pgfscope}%
\begin{pgfscope}%
\pgfsetrectcap%
\pgfsetroundjoin%
\pgfsetlinewidth{1.505625pt}%
\definecolor{currentstroke}{rgb}{0.121569,0.466667,0.705882}%
\pgfsetstrokecolor{currentstroke}%
\pgfsetdash{}{0pt}%
\pgfpathmoveto{\pgfqpoint{1.986199in}{3.734816in}}%
\pgfpathlineto{\pgfqpoint{2.263977in}{3.734816in}}%
\pgfusepath{stroke}%
\end{pgfscope}%
\begin{pgfscope}%
\definecolor{textcolor}{rgb}{0.000000,0.000000,0.000000}%
\pgfsetstrokecolor{textcolor}%
\pgfsetfillcolor{textcolor}%
\pgftext[x=2.375088in,y=3.686205in,left,base]{\color{textcolor}\sffamily\fontsize{10.000000}{12.000000}\selectfont Ground truth}%
\end{pgfscope}%
\begin{pgfscope}%
\pgfsetbuttcap%
\pgfsetroundjoin%
\definecolor{currentfill}{rgb}{0.121569,0.466667,0.705882}%
\pgfsetfillcolor{currentfill}%
\pgfsetlinewidth{1.003750pt}%
\definecolor{currentstroke}{rgb}{0.121569,0.466667,0.705882}%
\pgfsetstrokecolor{currentstroke}%
\pgfsetdash{}{0pt}%
\pgfsys@defobject{currentmarker}{\pgfqpoint{-0.031056in}{-0.031056in}}{\pgfqpoint{0.031056in}{0.031056in}}{%
\pgfpathmoveto{\pgfqpoint{0.000000in}{-0.031056in}}%
\pgfpathcurveto{\pgfqpoint{0.008236in}{-0.031056in}}{\pgfqpoint{0.016136in}{-0.027784in}}{\pgfqpoint{0.021960in}{-0.021960in}}%
\pgfpathcurveto{\pgfqpoint{0.027784in}{-0.016136in}}{\pgfqpoint{0.031056in}{-0.008236in}}{\pgfqpoint{0.031056in}{0.000000in}}%
\pgfpathcurveto{\pgfqpoint{0.031056in}{0.008236in}}{\pgfqpoint{0.027784in}{0.016136in}}{\pgfqpoint{0.021960in}{0.021960in}}%
\pgfpathcurveto{\pgfqpoint{0.016136in}{0.027784in}}{\pgfqpoint{0.008236in}{0.031056in}}{\pgfqpoint{0.000000in}{0.031056in}}%
\pgfpathcurveto{\pgfqpoint{-0.008236in}{0.031056in}}{\pgfqpoint{-0.016136in}{0.027784in}}{\pgfqpoint{-0.021960in}{0.021960in}}%
\pgfpathcurveto{\pgfqpoint{-0.027784in}{0.016136in}}{\pgfqpoint{-0.031056in}{0.008236in}}{\pgfqpoint{-0.031056in}{0.000000in}}%
\pgfpathcurveto{\pgfqpoint{-0.031056in}{-0.008236in}}{\pgfqpoint{-0.027784in}{-0.016136in}}{\pgfqpoint{-0.021960in}{-0.021960in}}%
\pgfpathcurveto{\pgfqpoint{-0.016136in}{-0.027784in}}{\pgfqpoint{-0.008236in}{-0.031056in}}{\pgfqpoint{0.000000in}{-0.031056in}}%
\pgfpathclose%
\pgfusepath{stroke,fill}%
}%
\begin{pgfscope}%
\pgfsys@transformshift{2.125088in}{3.518806in}%
\pgfsys@useobject{currentmarker}{}%
\end{pgfscope}%
\end{pgfscope}%
\begin{pgfscope}%
\definecolor{textcolor}{rgb}{0.000000,0.000000,0.000000}%
\pgfsetstrokecolor{textcolor}%
\pgfsetfillcolor{textcolor}%
\pgftext[x=2.375088in,y=3.482348in,left,base]{\color{textcolor}\sffamily\fontsize{10.000000}{12.000000}\selectfont Estimated position}%
\end{pgfscope}%
\end{pgfpicture}%
\makeatother%
\endgroup%
}
        \caption{ OLEQ's 3D position estimation had the lowest displacement error for the 4-meter line experiment. }
        \label{fig:line4_2D}
    \end{subfigure}
    \begin{subfigure}{0.49\textwidth}
        \centering
        \resizebox{1\linewidth}{!}{%% Creator: Matplotlib, PGF backend
%%
%% To include the figure in your LaTeX document, write
%%   \input{<filename>.pgf}
%%
%% Make sure the required packages are loaded in your preamble
%%   \usepackage{pgf}
%%
%% and, on pdftex
%%   \usepackage[utf8]{inputenc}\DeclareUnicodeCharacter{2212}{-}
%%
%% or, on luatex and xetex
%%   \usepackage{unicode-math}
%%
%% Figures using additional raster images can only be included by \input if
%% they are in the same directory as the main LaTeX file. For loading figures
%% from other directories you can use the `import` package
%%   \usepackage{import}
%%
%% and then include the figures with
%%   \import{<path to file>}{<filename>.pgf}
%%
%% Matplotlib used the following preamble
%%   \usepackage{fontspec}
%%
\begingroup%
\makeatletter%
\begin{pgfpicture}%
\pgfpathrectangle{\pgfpointorigin}{\pgfqpoint{4.342355in}{4.209289in}}%
\pgfusepath{use as bounding box, clip}%
\begin{pgfscope}%
\pgfsetbuttcap%
\pgfsetmiterjoin%
\definecolor{currentfill}{rgb}{1.000000,1.000000,1.000000}%
\pgfsetfillcolor{currentfill}%
\pgfsetlinewidth{0.000000pt}%
\definecolor{currentstroke}{rgb}{1.000000,1.000000,1.000000}%
\pgfsetstrokecolor{currentstroke}%
\pgfsetdash{}{0pt}%
\pgfpathmoveto{\pgfqpoint{0.000000in}{-0.000000in}}%
\pgfpathlineto{\pgfqpoint{4.342355in}{-0.000000in}}%
\pgfpathlineto{\pgfqpoint{4.342355in}{4.209289in}}%
\pgfpathlineto{\pgfqpoint{0.000000in}{4.209289in}}%
\pgfpathclose%
\pgfusepath{fill}%
\end{pgfscope}%
\begin{pgfscope}%
\pgfsetbuttcap%
\pgfsetmiterjoin%
\definecolor{currentfill}{rgb}{1.000000,1.000000,1.000000}%
\pgfsetfillcolor{currentfill}%
\pgfsetlinewidth{0.000000pt}%
\definecolor{currentstroke}{rgb}{0.000000,0.000000,0.000000}%
\pgfsetstrokecolor{currentstroke}%
\pgfsetstrokeopacity{0.000000}%
\pgfsetdash{}{0pt}%
\pgfpathmoveto{\pgfqpoint{0.100000in}{0.212622in}}%
\pgfpathlineto{\pgfqpoint{3.796000in}{0.212622in}}%
\pgfpathlineto{\pgfqpoint{3.796000in}{3.908622in}}%
\pgfpathlineto{\pgfqpoint{0.100000in}{3.908622in}}%
\pgfpathclose%
\pgfusepath{fill}%
\end{pgfscope}%
\begin{pgfscope}%
\pgfsetbuttcap%
\pgfsetmiterjoin%
\definecolor{currentfill}{rgb}{0.950000,0.950000,0.950000}%
\pgfsetfillcolor{currentfill}%
\pgfsetfillopacity{0.500000}%
\pgfsetlinewidth{1.003750pt}%
\definecolor{currentstroke}{rgb}{0.950000,0.950000,0.950000}%
\pgfsetstrokecolor{currentstroke}%
\pgfsetstrokeopacity{0.500000}%
\pgfsetdash{}{0pt}%
\pgfpathmoveto{\pgfqpoint{0.379073in}{1.123938in}}%
\pgfpathlineto{\pgfqpoint{1.599613in}{2.147018in}}%
\pgfpathlineto{\pgfqpoint{1.582647in}{3.622484in}}%
\pgfpathlineto{\pgfqpoint{0.303698in}{2.689165in}}%
\pgfusepath{stroke,fill}%
\end{pgfscope}%
\begin{pgfscope}%
\pgfsetbuttcap%
\pgfsetmiterjoin%
\definecolor{currentfill}{rgb}{0.900000,0.900000,0.900000}%
\pgfsetfillcolor{currentfill}%
\pgfsetfillopacity{0.500000}%
\pgfsetlinewidth{1.003750pt}%
\definecolor{currentstroke}{rgb}{0.900000,0.900000,0.900000}%
\pgfsetstrokecolor{currentstroke}%
\pgfsetstrokeopacity{0.500000}%
\pgfsetdash{}{0pt}%
\pgfpathmoveto{\pgfqpoint{1.599613in}{2.147018in}}%
\pgfpathlineto{\pgfqpoint{3.558144in}{1.577751in}}%
\pgfpathlineto{\pgfqpoint{3.628038in}{3.104037in}}%
\pgfpathlineto{\pgfqpoint{1.582647in}{3.622484in}}%
\pgfusepath{stroke,fill}%
\end{pgfscope}%
\begin{pgfscope}%
\pgfsetbuttcap%
\pgfsetmiterjoin%
\definecolor{currentfill}{rgb}{0.925000,0.925000,0.925000}%
\pgfsetfillcolor{currentfill}%
\pgfsetfillopacity{0.500000}%
\pgfsetlinewidth{1.003750pt}%
\definecolor{currentstroke}{rgb}{0.925000,0.925000,0.925000}%
\pgfsetstrokecolor{currentstroke}%
\pgfsetstrokeopacity{0.500000}%
\pgfsetdash{}{0pt}%
\pgfpathmoveto{\pgfqpoint{0.379073in}{1.123938in}}%
\pgfpathlineto{\pgfqpoint{2.455212in}{0.445871in}}%
\pgfpathlineto{\pgfqpoint{3.558144in}{1.577751in}}%
\pgfpathlineto{\pgfqpoint{1.599613in}{2.147018in}}%
\pgfusepath{stroke,fill}%
\end{pgfscope}%
\begin{pgfscope}%
\pgfsetrectcap%
\pgfsetroundjoin%
\pgfsetlinewidth{0.803000pt}%
\definecolor{currentstroke}{rgb}{0.000000,0.000000,0.000000}%
\pgfsetstrokecolor{currentstroke}%
\pgfsetdash{}{0pt}%
\pgfpathmoveto{\pgfqpoint{0.379073in}{1.123938in}}%
\pgfpathlineto{\pgfqpoint{2.455212in}{0.445871in}}%
\pgfusepath{stroke}%
\end{pgfscope}%
\begin{pgfscope}%
\definecolor{textcolor}{rgb}{0.000000,0.000000,0.000000}%
\pgfsetstrokecolor{textcolor}%
\pgfsetfillcolor{textcolor}%
\pgftext[x=0.730374in, y=0.408886in, left, base,rotate=341.912962]{\color{textcolor}\rmfamily\fontsize{10.000000}{12.000000}\selectfont Position X [\(\displaystyle m\)]}%
\end{pgfscope}%
\begin{pgfscope}%
\pgfsetbuttcap%
\pgfsetroundjoin%
\pgfsetlinewidth{0.803000pt}%
\definecolor{currentstroke}{rgb}{0.690196,0.690196,0.690196}%
\pgfsetstrokecolor{currentstroke}%
\pgfsetdash{}{0pt}%
\pgfpathmoveto{\pgfqpoint{0.526251in}{1.075869in}}%
\pgfpathlineto{\pgfqpoint{1.739024in}{2.106497in}}%
\pgfpathlineto{\pgfqpoint{1.727955in}{3.585652in}}%
\pgfusepath{stroke}%
\end{pgfscope}%
\begin{pgfscope}%
\pgfsetbuttcap%
\pgfsetroundjoin%
\pgfsetlinewidth{0.803000pt}%
\definecolor{currentstroke}{rgb}{0.690196,0.690196,0.690196}%
\pgfsetstrokecolor{currentstroke}%
\pgfsetdash{}{0pt}%
\pgfpathmoveto{\pgfqpoint{0.958652in}{0.934647in}}%
\pgfpathlineto{\pgfqpoint{2.148102in}{1.987594in}}%
\pgfpathlineto{\pgfqpoint{2.154591in}{3.477513in}}%
\pgfusepath{stroke}%
\end{pgfscope}%
\begin{pgfscope}%
\pgfsetbuttcap%
\pgfsetroundjoin%
\pgfsetlinewidth{0.803000pt}%
\definecolor{currentstroke}{rgb}{0.690196,0.690196,0.690196}%
\pgfsetstrokecolor{currentstroke}%
\pgfsetdash{}{0pt}%
\pgfpathmoveto{\pgfqpoint{1.400880in}{0.790216in}}%
\pgfpathlineto{\pgfqpoint{2.565699in}{1.866215in}}%
\pgfpathlineto{\pgfqpoint{2.590498in}{3.367023in}}%
\pgfusepath{stroke}%
\end{pgfscope}%
\begin{pgfscope}%
\pgfsetbuttcap%
\pgfsetroundjoin%
\pgfsetlinewidth{0.803000pt}%
\definecolor{currentstroke}{rgb}{0.690196,0.690196,0.690196}%
\pgfsetstrokecolor{currentstroke}%
\pgfsetdash{}{0pt}%
\pgfpathmoveto{\pgfqpoint{1.853274in}{0.642464in}}%
\pgfpathlineto{\pgfqpoint{2.992083in}{1.742282in}}%
\pgfpathlineto{\pgfqpoint{3.035984in}{3.254105in}}%
\pgfusepath{stroke}%
\end{pgfscope}%
\begin{pgfscope}%
\pgfsetbuttcap%
\pgfsetroundjoin%
\pgfsetlinewidth{0.803000pt}%
\definecolor{currentstroke}{rgb}{0.690196,0.690196,0.690196}%
\pgfsetstrokecolor{currentstroke}%
\pgfsetdash{}{0pt}%
\pgfpathmoveto{\pgfqpoint{2.316188in}{0.491276in}}%
\pgfpathlineto{\pgfqpoint{3.427535in}{1.615714in}}%
\pgfpathlineto{\pgfqpoint{3.491367in}{3.138679in}}%
\pgfusepath{stroke}%
\end{pgfscope}%
\begin{pgfscope}%
\pgfsetrectcap%
\pgfsetroundjoin%
\pgfsetlinewidth{0.803000pt}%
\definecolor{currentstroke}{rgb}{0.000000,0.000000,0.000000}%
\pgfsetstrokecolor{currentstroke}%
\pgfsetdash{}{0pt}%
\pgfpathmoveto{\pgfqpoint{0.536812in}{1.084844in}}%
\pgfpathlineto{\pgfqpoint{0.505083in}{1.057881in}}%
\pgfusepath{stroke}%
\end{pgfscope}%
\begin{pgfscope}%
\definecolor{textcolor}{rgb}{0.000000,0.000000,0.000000}%
\pgfsetstrokecolor{textcolor}%
\pgfsetfillcolor{textcolor}%
\pgftext[x=0.421702in,y=0.857517in,,top]{\color{textcolor}\rmfamily\fontsize{10.000000}{12.000000}\selectfont \(\displaystyle {0}\)}%
\end{pgfscope}%
\begin{pgfscope}%
\pgfsetrectcap%
\pgfsetroundjoin%
\pgfsetlinewidth{0.803000pt}%
\definecolor{currentstroke}{rgb}{0.000000,0.000000,0.000000}%
\pgfsetstrokecolor{currentstroke}%
\pgfsetdash{}{0pt}%
\pgfpathmoveto{\pgfqpoint{0.969020in}{0.943825in}}%
\pgfpathlineto{\pgfqpoint{0.937872in}{0.916252in}}%
\pgfusepath{stroke}%
\end{pgfscope}%
\begin{pgfscope}%
\definecolor{textcolor}{rgb}{0.000000,0.000000,0.000000}%
\pgfsetstrokecolor{textcolor}%
\pgfsetfillcolor{textcolor}%
\pgftext[x=0.854551in,y=0.713301in,,top]{\color{textcolor}\rmfamily\fontsize{10.000000}{12.000000}\selectfont \(\displaystyle {1}\)}%
\end{pgfscope}%
\begin{pgfscope}%
\pgfsetrectcap%
\pgfsetroundjoin%
\pgfsetlinewidth{0.803000pt}%
\definecolor{currentstroke}{rgb}{0.000000,0.000000,0.000000}%
\pgfsetstrokecolor{currentstroke}%
\pgfsetdash{}{0pt}%
\pgfpathmoveto{\pgfqpoint{1.411042in}{0.799603in}}%
\pgfpathlineto{\pgfqpoint{1.380511in}{0.771399in}}%
\pgfusepath{stroke}%
\end{pgfscope}%
\begin{pgfscope}%
\definecolor{textcolor}{rgb}{0.000000,0.000000,0.000000}%
\pgfsetstrokecolor{textcolor}%
\pgfsetfillcolor{textcolor}%
\pgftext[x=1.297273in,y=0.565795in,,top]{\color{textcolor}\rmfamily\fontsize{10.000000}{12.000000}\selectfont \(\displaystyle {2}\)}%
\end{pgfscope}%
\begin{pgfscope}%
\pgfsetrectcap%
\pgfsetroundjoin%
\pgfsetlinewidth{0.803000pt}%
\definecolor{currentstroke}{rgb}{0.000000,0.000000,0.000000}%
\pgfsetstrokecolor{currentstroke}%
\pgfsetdash{}{0pt}%
\pgfpathmoveto{\pgfqpoint{1.863219in}{0.652068in}}%
\pgfpathlineto{\pgfqpoint{1.833340in}{0.623212in}}%
\pgfusepath{stroke}%
\end{pgfscope}%
\begin{pgfscope}%
\definecolor{textcolor}{rgb}{0.000000,0.000000,0.000000}%
\pgfsetstrokecolor{textcolor}%
\pgfsetfillcolor{textcolor}%
\pgftext[x=1.750213in,y=0.414885in,,top]{\color{textcolor}\rmfamily\fontsize{10.000000}{12.000000}\selectfont \(\displaystyle {3}\)}%
\end{pgfscope}%
\begin{pgfscope}%
\pgfsetrectcap%
\pgfsetroundjoin%
\pgfsetlinewidth{0.803000pt}%
\definecolor{currentstroke}{rgb}{0.000000,0.000000,0.000000}%
\pgfsetstrokecolor{currentstroke}%
\pgfsetdash{}{0pt}%
\pgfpathmoveto{\pgfqpoint{2.325903in}{0.501105in}}%
\pgfpathlineto{\pgfqpoint{2.296715in}{0.471574in}}%
\pgfusepath{stroke}%
\end{pgfscope}%
\begin{pgfscope}%
\definecolor{textcolor}{rgb}{0.000000,0.000000,0.000000}%
\pgfsetstrokecolor{textcolor}%
\pgfsetfillcolor{textcolor}%
\pgftext[x=2.213726in,y=0.260453in,,top]{\color{textcolor}\rmfamily\fontsize{10.000000}{12.000000}\selectfont \(\displaystyle {4}\)}%
\end{pgfscope}%
\begin{pgfscope}%
\pgfsetrectcap%
\pgfsetroundjoin%
\pgfsetlinewidth{0.803000pt}%
\definecolor{currentstroke}{rgb}{0.000000,0.000000,0.000000}%
\pgfsetstrokecolor{currentstroke}%
\pgfsetdash{}{0pt}%
\pgfpathmoveto{\pgfqpoint{3.558144in}{1.577751in}}%
\pgfpathlineto{\pgfqpoint{2.455212in}{0.445871in}}%
\pgfusepath{stroke}%
\end{pgfscope}%
\begin{pgfscope}%
\definecolor{textcolor}{rgb}{0.000000,0.000000,0.000000}%
\pgfsetstrokecolor{textcolor}%
\pgfsetfillcolor{textcolor}%
\pgftext[x=3.120747in, y=0.305657in, left, base,rotate=45.742112]{\color{textcolor}\rmfamily\fontsize{10.000000}{12.000000}\selectfont Position Y [\(\displaystyle m\)]}%
\end{pgfscope}%
\begin{pgfscope}%
\pgfsetbuttcap%
\pgfsetroundjoin%
\pgfsetlinewidth{0.803000pt}%
\definecolor{currentstroke}{rgb}{0.690196,0.690196,0.690196}%
\pgfsetstrokecolor{currentstroke}%
\pgfsetdash{}{0pt}%
\pgfpathmoveto{\pgfqpoint{0.331607in}{2.709532in}}%
\pgfpathlineto{\pgfqpoint{0.405609in}{1.146180in}}%
\pgfpathlineto{\pgfqpoint{2.479295in}{0.470585in}}%
\pgfusepath{stroke}%
\end{pgfscope}%
\begin{pgfscope}%
\pgfsetbuttcap%
\pgfsetroundjoin%
\pgfsetlinewidth{0.803000pt}%
\definecolor{currentstroke}{rgb}{0.690196,0.690196,0.690196}%
\pgfsetstrokecolor{currentstroke}%
\pgfsetdash{}{0pt}%
\pgfpathmoveto{\pgfqpoint{0.496905in}{2.830159in}}%
\pgfpathlineto{\pgfqpoint{0.562864in}{1.277994in}}%
\pgfpathlineto{\pgfqpoint{2.621917in}{0.616951in}}%
\pgfusepath{stroke}%
\end{pgfscope}%
\begin{pgfscope}%
\pgfsetbuttcap%
\pgfsetroundjoin%
\pgfsetlinewidth{0.803000pt}%
\definecolor{currentstroke}{rgb}{0.690196,0.690196,0.690196}%
\pgfsetstrokecolor{currentstroke}%
\pgfsetdash{}{0pt}%
\pgfpathmoveto{\pgfqpoint{0.658597in}{2.948154in}}%
\pgfpathlineto{\pgfqpoint{0.716836in}{1.407057in}}%
\pgfpathlineto{\pgfqpoint{2.761405in}{0.760100in}}%
\pgfusepath{stroke}%
\end{pgfscope}%
\begin{pgfscope}%
\pgfsetbuttcap%
\pgfsetroundjoin%
\pgfsetlinewidth{0.803000pt}%
\definecolor{currentstroke}{rgb}{0.690196,0.690196,0.690196}%
\pgfsetstrokecolor{currentstroke}%
\pgfsetdash{}{0pt}%
\pgfpathmoveto{\pgfqpoint{0.816799in}{3.063603in}}%
\pgfpathlineto{\pgfqpoint{0.867628in}{1.533454in}}%
\pgfpathlineto{\pgfqpoint{2.897861in}{0.900137in}}%
\pgfusepath{stroke}%
\end{pgfscope}%
\begin{pgfscope}%
\pgfsetbuttcap%
\pgfsetroundjoin%
\pgfsetlinewidth{0.803000pt}%
\definecolor{currentstroke}{rgb}{0.690196,0.690196,0.690196}%
\pgfsetstrokecolor{currentstroke}%
\pgfsetdash{}{0pt}%
\pgfpathmoveto{\pgfqpoint{0.971622in}{3.176586in}}%
\pgfpathlineto{\pgfqpoint{1.015337in}{1.657267in}}%
\pgfpathlineto{\pgfqpoint{3.031383in}{1.037163in}}%
\pgfusepath{stroke}%
\end{pgfscope}%
\begin{pgfscope}%
\pgfsetbuttcap%
\pgfsetroundjoin%
\pgfsetlinewidth{0.803000pt}%
\definecolor{currentstroke}{rgb}{0.690196,0.690196,0.690196}%
\pgfsetstrokecolor{currentstroke}%
\pgfsetdash{}{0pt}%
\pgfpathmoveto{\pgfqpoint{1.123174in}{3.287181in}}%
\pgfpathlineto{\pgfqpoint{1.160057in}{1.778574in}}%
\pgfpathlineto{\pgfqpoint{3.162064in}{1.171275in}}%
\pgfusepath{stroke}%
\end{pgfscope}%
\begin{pgfscope}%
\pgfsetbuttcap%
\pgfsetroundjoin%
\pgfsetlinewidth{0.803000pt}%
\definecolor{currentstroke}{rgb}{0.690196,0.690196,0.690196}%
\pgfsetstrokecolor{currentstroke}%
\pgfsetdash{}{0pt}%
\pgfpathmoveto{\pgfqpoint{1.271557in}{3.395465in}}%
\pgfpathlineto{\pgfqpoint{1.301877in}{1.897450in}}%
\pgfpathlineto{\pgfqpoint{3.289994in}{1.302563in}}%
\pgfusepath{stroke}%
\end{pgfscope}%
\begin{pgfscope}%
\pgfsetbuttcap%
\pgfsetroundjoin%
\pgfsetlinewidth{0.803000pt}%
\definecolor{currentstroke}{rgb}{0.690196,0.690196,0.690196}%
\pgfsetstrokecolor{currentstroke}%
\pgfsetdash{}{0pt}%
\pgfpathmoveto{\pgfqpoint{1.416870in}{3.501507in}}%
\pgfpathlineto{\pgfqpoint{1.440884in}{2.013968in}}%
\pgfpathlineto{\pgfqpoint{3.415260in}{1.431116in}}%
\pgfusepath{stroke}%
\end{pgfscope}%
\begin{pgfscope}%
\pgfsetbuttcap%
\pgfsetroundjoin%
\pgfsetlinewidth{0.803000pt}%
\definecolor{currentstroke}{rgb}{0.690196,0.690196,0.690196}%
\pgfsetstrokecolor{currentstroke}%
\pgfsetdash{}{0pt}%
\pgfpathmoveto{\pgfqpoint{1.559207in}{3.605378in}}%
\pgfpathlineto{\pgfqpoint{1.577160in}{2.128198in}}%
\pgfpathlineto{\pgfqpoint{3.537943in}{1.557019in}}%
\pgfusepath{stroke}%
\end{pgfscope}%
\begin{pgfscope}%
\pgfsetrectcap%
\pgfsetroundjoin%
\pgfsetlinewidth{0.803000pt}%
\definecolor{currentstroke}{rgb}{0.000000,0.000000,0.000000}%
\pgfsetstrokecolor{currentstroke}%
\pgfsetdash{}{0pt}%
\pgfpathmoveto{\pgfqpoint{2.461816in}{0.476280in}}%
\pgfpathlineto{\pgfqpoint{2.514297in}{0.459182in}}%
\pgfusepath{stroke}%
\end{pgfscope}%
\begin{pgfscope}%
\definecolor{textcolor}{rgb}{0.000000,0.000000,0.000000}%
\pgfsetstrokecolor{textcolor}%
\pgfsetfillcolor{textcolor}%
\pgftext[x=2.658892in,y=0.283356in,,top]{\color{textcolor}\rmfamily\fontsize{10.000000}{12.000000}\selectfont \(\displaystyle {−2.0}\)}%
\end{pgfscope}%
\begin{pgfscope}%
\pgfsetrectcap%
\pgfsetroundjoin%
\pgfsetlinewidth{0.803000pt}%
\definecolor{currentstroke}{rgb}{0.000000,0.000000,0.000000}%
\pgfsetstrokecolor{currentstroke}%
\pgfsetdash{}{0pt}%
\pgfpathmoveto{\pgfqpoint{2.604571in}{0.622519in}}%
\pgfpathlineto{\pgfqpoint{2.656652in}{0.605799in}}%
\pgfusepath{stroke}%
\end{pgfscope}%
\begin{pgfscope}%
\definecolor{textcolor}{rgb}{0.000000,0.000000,0.000000}%
\pgfsetstrokecolor{textcolor}%
\pgfsetfillcolor{textcolor}%
\pgftext[x=2.799603in,y=0.431890in,,top]{\color{textcolor}\rmfamily\fontsize{10.000000}{12.000000}\selectfont \(\displaystyle {−1.5}\)}%
\end{pgfscope}%
\begin{pgfscope}%
\pgfsetrectcap%
\pgfsetroundjoin%
\pgfsetlinewidth{0.803000pt}%
\definecolor{currentstroke}{rgb}{0.000000,0.000000,0.000000}%
\pgfsetstrokecolor{currentstroke}%
\pgfsetdash{}{0pt}%
\pgfpathmoveto{\pgfqpoint{2.744191in}{0.765547in}}%
\pgfpathlineto{\pgfqpoint{2.795876in}{0.749192in}}%
\pgfusepath{stroke}%
\end{pgfscope}%
\begin{pgfscope}%
\definecolor{textcolor}{rgb}{0.000000,0.000000,0.000000}%
\pgfsetstrokecolor{textcolor}%
\pgfsetfillcolor{textcolor}%
\pgftext[x=2.937219in,y=0.577158in,,top]{\color{textcolor}\rmfamily\fontsize{10.000000}{12.000000}\selectfont \(\displaystyle {−1.0}\)}%
\end{pgfscope}%
\begin{pgfscope}%
\pgfsetrectcap%
\pgfsetroundjoin%
\pgfsetlinewidth{0.803000pt}%
\definecolor{currentstroke}{rgb}{0.000000,0.000000,0.000000}%
\pgfsetstrokecolor{currentstroke}%
\pgfsetdash{}{0pt}%
\pgfpathmoveto{\pgfqpoint{2.880777in}{0.905466in}}%
\pgfpathlineto{\pgfqpoint{2.932072in}{0.889465in}}%
\pgfusepath{stroke}%
\end{pgfscope}%
\begin{pgfscope}%
\definecolor{textcolor}{rgb}{0.000000,0.000000,0.000000}%
\pgfsetstrokecolor{textcolor}%
\pgfsetfillcolor{textcolor}%
\pgftext[x=3.071843in,y=0.719266in,,top]{\color{textcolor}\rmfamily\fontsize{10.000000}{12.000000}\selectfont \(\displaystyle {−0.5}\)}%
\end{pgfscope}%
\begin{pgfscope}%
\pgfsetrectcap%
\pgfsetroundjoin%
\pgfsetlinewidth{0.803000pt}%
\definecolor{currentstroke}{rgb}{0.000000,0.000000,0.000000}%
\pgfsetstrokecolor{currentstroke}%
\pgfsetdash{}{0pt}%
\pgfpathmoveto{\pgfqpoint{3.014427in}{1.042379in}}%
\pgfpathlineto{\pgfqpoint{3.065336in}{1.026720in}}%
\pgfusepath{stroke}%
\end{pgfscope}%
\begin{pgfscope}%
\definecolor{textcolor}{rgb}{0.000000,0.000000,0.000000}%
\pgfsetstrokecolor{textcolor}%
\pgfsetfillcolor{textcolor}%
\pgftext[x=3.203570in,y=0.858317in,,top]{\color{textcolor}\rmfamily\fontsize{10.000000}{12.000000}\selectfont \(\displaystyle {0.0}\)}%
\end{pgfscope}%
\begin{pgfscope}%
\pgfsetrectcap%
\pgfsetroundjoin%
\pgfsetlinewidth{0.803000pt}%
\definecolor{currentstroke}{rgb}{0.000000,0.000000,0.000000}%
\pgfsetstrokecolor{currentstroke}%
\pgfsetdash{}{0pt}%
\pgfpathmoveto{\pgfqpoint{3.145235in}{1.176380in}}%
\pgfpathlineto{\pgfqpoint{3.195763in}{1.161052in}}%
\pgfusepath{stroke}%
\end{pgfscope}%
\begin{pgfscope}%
\definecolor{textcolor}{rgb}{0.000000,0.000000,0.000000}%
\pgfsetstrokecolor{textcolor}%
\pgfsetfillcolor{textcolor}%
\pgftext[x=3.332493in,y=0.994408in,,top]{\color{textcolor}\rmfamily\fontsize{10.000000}{12.000000}\selectfont \(\displaystyle {0.5}\)}%
\end{pgfscope}%
\begin{pgfscope}%
\pgfsetrectcap%
\pgfsetroundjoin%
\pgfsetlinewidth{0.803000pt}%
\definecolor{currentstroke}{rgb}{0.000000,0.000000,0.000000}%
\pgfsetstrokecolor{currentstroke}%
\pgfsetdash{}{0pt}%
\pgfpathmoveto{\pgfqpoint{3.273291in}{1.307561in}}%
\pgfpathlineto{\pgfqpoint{3.323442in}{1.292554in}}%
\pgfusepath{stroke}%
\end{pgfscope}%
\begin{pgfscope}%
\definecolor{textcolor}{rgb}{0.000000,0.000000,0.000000}%
\pgfsetstrokecolor{textcolor}%
\pgfsetfillcolor{textcolor}%
\pgftext[x=3.458700in,y=1.127633in,,top]{\color{textcolor}\rmfamily\fontsize{10.000000}{12.000000}\selectfont \(\displaystyle {1.0}\)}%
\end{pgfscope}%
\begin{pgfscope}%
\pgfsetrectcap%
\pgfsetroundjoin%
\pgfsetlinewidth{0.803000pt}%
\definecolor{currentstroke}{rgb}{0.000000,0.000000,0.000000}%
\pgfsetstrokecolor{currentstroke}%
\pgfsetdash{}{0pt}%
\pgfpathmoveto{\pgfqpoint{3.398680in}{1.436010in}}%
\pgfpathlineto{\pgfqpoint{3.448459in}{1.421315in}}%
\pgfusepath{stroke}%
\end{pgfscope}%
\begin{pgfscope}%
\definecolor{textcolor}{rgb}{0.000000,0.000000,0.000000}%
\pgfsetstrokecolor{textcolor}%
\pgfsetfillcolor{textcolor}%
\pgftext[x=3.582278in,y=1.258080in,,top]{\color{textcolor}\rmfamily\fontsize{10.000000}{12.000000}\selectfont \(\displaystyle {1.5}\)}%
\end{pgfscope}%
\begin{pgfscope}%
\pgfsetrectcap%
\pgfsetroundjoin%
\pgfsetlinewidth{0.803000pt}%
\definecolor{currentstroke}{rgb}{0.000000,0.000000,0.000000}%
\pgfsetstrokecolor{currentstroke}%
\pgfsetdash{}{0pt}%
\pgfpathmoveto{\pgfqpoint{3.521485in}{1.561813in}}%
\pgfpathlineto{\pgfqpoint{3.570897in}{1.547419in}}%
\pgfusepath{stroke}%
\end{pgfscope}%
\begin{pgfscope}%
\definecolor{textcolor}{rgb}{0.000000,0.000000,0.000000}%
\pgfsetstrokecolor{textcolor}%
\pgfsetfillcolor{textcolor}%
\pgftext[x=3.703306in,y=1.385837in,,top]{\color{textcolor}\rmfamily\fontsize{10.000000}{12.000000}\selectfont \(\displaystyle {2.0}\)}%
\end{pgfscope}%
\begin{pgfscope}%
\pgfsetrectcap%
\pgfsetroundjoin%
\pgfsetlinewidth{0.803000pt}%
\definecolor{currentstroke}{rgb}{0.000000,0.000000,0.000000}%
\pgfsetstrokecolor{currentstroke}%
\pgfsetdash{}{0pt}%
\pgfpathmoveto{\pgfqpoint{3.558144in}{1.577751in}}%
\pgfpathlineto{\pgfqpoint{3.628038in}{3.104037in}}%
\pgfusepath{stroke}%
\end{pgfscope}%
\begin{pgfscope}%
\definecolor{textcolor}{rgb}{0.000000,0.000000,0.000000}%
\pgfsetstrokecolor{textcolor}%
\pgfsetfillcolor{textcolor}%
\pgftext[x=4.167903in, y=1.963517in, left, base,rotate=87.378092]{\color{textcolor}\rmfamily\fontsize{10.000000}{12.000000}\selectfont Position Z [\(\displaystyle m\)]}%
\end{pgfscope}%
\begin{pgfscope}%
\pgfsetbuttcap%
\pgfsetroundjoin%
\pgfsetlinewidth{0.803000pt}%
\definecolor{currentstroke}{rgb}{0.690196,0.690196,0.690196}%
\pgfsetstrokecolor{currentstroke}%
\pgfsetdash{}{0pt}%
\pgfpathmoveto{\pgfqpoint{3.559484in}{1.606993in}}%
\pgfpathlineto{\pgfqpoint{1.599288in}{2.175345in}}%
\pgfpathlineto{\pgfqpoint{0.377632in}{1.153876in}}%
\pgfusepath{stroke}%
\end{pgfscope}%
\begin{pgfscope}%
\pgfsetbuttcap%
\pgfsetroundjoin%
\pgfsetlinewidth{0.803000pt}%
\definecolor{currentstroke}{rgb}{0.690196,0.690196,0.690196}%
\pgfsetstrokecolor{currentstroke}%
\pgfsetdash{}{0pt}%
\pgfpathmoveto{\pgfqpoint{3.567566in}{1.783504in}}%
\pgfpathlineto{\pgfqpoint{1.597322in}{2.346284in}}%
\pgfpathlineto{\pgfqpoint{0.368927in}{1.334633in}}%
\pgfusepath{stroke}%
\end{pgfscope}%
\begin{pgfscope}%
\pgfsetbuttcap%
\pgfsetroundjoin%
\pgfsetlinewidth{0.803000pt}%
\definecolor{currentstroke}{rgb}{0.690196,0.690196,0.690196}%
\pgfsetstrokecolor{currentstroke}%
\pgfsetdash{}{0pt}%
\pgfpathmoveto{\pgfqpoint{3.575734in}{1.961852in}}%
\pgfpathlineto{\pgfqpoint{1.595337in}{2.518917in}}%
\pgfpathlineto{\pgfqpoint{0.360129in}{1.517342in}}%
\pgfusepath{stroke}%
\end{pgfscope}%
\begin{pgfscope}%
\pgfsetbuttcap%
\pgfsetroundjoin%
\pgfsetlinewidth{0.803000pt}%
\definecolor{currentstroke}{rgb}{0.690196,0.690196,0.690196}%
\pgfsetstrokecolor{currentstroke}%
\pgfsetdash{}{0pt}%
\pgfpathmoveto{\pgfqpoint{3.583986in}{2.142066in}}%
\pgfpathlineto{\pgfqpoint{1.593332in}{2.693270in}}%
\pgfpathlineto{\pgfqpoint{0.351234in}{1.702036in}}%
\pgfusepath{stroke}%
\end{pgfscope}%
\begin{pgfscope}%
\pgfsetbuttcap%
\pgfsetroundjoin%
\pgfsetlinewidth{0.803000pt}%
\definecolor{currentstroke}{rgb}{0.690196,0.690196,0.690196}%
\pgfsetstrokecolor{currentstroke}%
\pgfsetdash{}{0pt}%
\pgfpathmoveto{\pgfqpoint{3.592325in}{2.324175in}}%
\pgfpathlineto{\pgfqpoint{1.591307in}{2.869367in}}%
\pgfpathlineto{\pgfqpoint{0.342243in}{1.888746in}}%
\pgfusepath{stroke}%
\end{pgfscope}%
\begin{pgfscope}%
\pgfsetbuttcap%
\pgfsetroundjoin%
\pgfsetlinewidth{0.803000pt}%
\definecolor{currentstroke}{rgb}{0.690196,0.690196,0.690196}%
\pgfsetstrokecolor{currentstroke}%
\pgfsetdash{}{0pt}%
\pgfpathmoveto{\pgfqpoint{3.600753in}{2.508209in}}%
\pgfpathlineto{\pgfqpoint{1.589262in}{3.047236in}}%
\pgfpathlineto{\pgfqpoint{0.333153in}{2.077507in}}%
\pgfusepath{stroke}%
\end{pgfscope}%
\begin{pgfscope}%
\pgfsetbuttcap%
\pgfsetroundjoin%
\pgfsetlinewidth{0.803000pt}%
\definecolor{currentstroke}{rgb}{0.690196,0.690196,0.690196}%
\pgfsetstrokecolor{currentstroke}%
\pgfsetdash{}{0pt}%
\pgfpathmoveto{\pgfqpoint{3.609270in}{2.694199in}}%
\pgfpathlineto{\pgfqpoint{1.587196in}{3.226903in}}%
\pgfpathlineto{\pgfqpoint{0.323963in}{2.268352in}}%
\pgfusepath{stroke}%
\end{pgfscope}%
\begin{pgfscope}%
\pgfsetbuttcap%
\pgfsetroundjoin%
\pgfsetlinewidth{0.803000pt}%
\definecolor{currentstroke}{rgb}{0.690196,0.690196,0.690196}%
\pgfsetstrokecolor{currentstroke}%
\pgfsetdash{}{0pt}%
\pgfpathmoveto{\pgfqpoint{3.617878in}{2.882176in}}%
\pgfpathlineto{\pgfqpoint{1.585109in}{3.408397in}}%
\pgfpathlineto{\pgfqpoint{0.314670in}{2.461315in}}%
\pgfusepath{stroke}%
\end{pgfscope}%
\begin{pgfscope}%
\pgfsetbuttcap%
\pgfsetroundjoin%
\pgfsetlinewidth{0.803000pt}%
\definecolor{currentstroke}{rgb}{0.690196,0.690196,0.690196}%
\pgfsetstrokecolor{currentstroke}%
\pgfsetdash{}{0pt}%
\pgfpathmoveto{\pgfqpoint{3.626578in}{3.072172in}}%
\pgfpathlineto{\pgfqpoint{1.583000in}{3.591744in}}%
\pgfpathlineto{\pgfqpoint{0.305274in}{2.656433in}}%
\pgfusepath{stroke}%
\end{pgfscope}%
\begin{pgfscope}%
\pgfsetrectcap%
\pgfsetroundjoin%
\pgfsetlinewidth{0.803000pt}%
\definecolor{currentstroke}{rgb}{0.000000,0.000000,0.000000}%
\pgfsetstrokecolor{currentstroke}%
\pgfsetdash{}{0pt}%
\pgfpathmoveto{\pgfqpoint{3.543031in}{1.611763in}}%
\pgfpathlineto{\pgfqpoint{3.592427in}{1.597441in}}%
\pgfusepath{stroke}%
\end{pgfscope}%
\begin{pgfscope}%
\definecolor{textcolor}{rgb}{0.000000,0.000000,0.000000}%
\pgfsetstrokecolor{textcolor}%
\pgfsetfillcolor{textcolor}%
\pgftext[x=3.813150in,y=1.643116in,,top]{\color{textcolor}\rmfamily\fontsize{10.000000}{12.000000}\selectfont \(\displaystyle {0.0}\)}%
\end{pgfscope}%
\begin{pgfscope}%
\pgfsetrectcap%
\pgfsetroundjoin%
\pgfsetlinewidth{0.803000pt}%
\definecolor{currentstroke}{rgb}{0.000000,0.000000,0.000000}%
\pgfsetstrokecolor{currentstroke}%
\pgfsetdash{}{0pt}%
\pgfpathmoveto{\pgfqpoint{3.551026in}{1.788229in}}%
\pgfpathlineto{\pgfqpoint{3.600687in}{1.774043in}}%
\pgfusepath{stroke}%
\end{pgfscope}%
\begin{pgfscope}%
\definecolor{textcolor}{rgb}{0.000000,0.000000,0.000000}%
\pgfsetstrokecolor{textcolor}%
\pgfsetfillcolor{textcolor}%
\pgftext[x=3.822517in,y=1.819282in,,top]{\color{textcolor}\rmfamily\fontsize{10.000000}{12.000000}\selectfont \(\displaystyle {0.5}\)}%
\end{pgfscope}%
\begin{pgfscope}%
\pgfsetrectcap%
\pgfsetroundjoin%
\pgfsetlinewidth{0.803000pt}%
\definecolor{currentstroke}{rgb}{0.000000,0.000000,0.000000}%
\pgfsetstrokecolor{currentstroke}%
\pgfsetdash{}{0pt}%
\pgfpathmoveto{\pgfqpoint{3.559104in}{1.966530in}}%
\pgfpathlineto{\pgfqpoint{3.609033in}{1.952485in}}%
\pgfusepath{stroke}%
\end{pgfscope}%
\begin{pgfscope}%
\definecolor{textcolor}{rgb}{0.000000,0.000000,0.000000}%
\pgfsetstrokecolor{textcolor}%
\pgfsetfillcolor{textcolor}%
\pgftext[x=3.831982in,y=1.997275in,,top]{\color{textcolor}\rmfamily\fontsize{10.000000}{12.000000}\selectfont \(\displaystyle {1.0}\)}%
\end{pgfscope}%
\begin{pgfscope}%
\pgfsetrectcap%
\pgfsetroundjoin%
\pgfsetlinewidth{0.803000pt}%
\definecolor{currentstroke}{rgb}{0.000000,0.000000,0.000000}%
\pgfsetstrokecolor{currentstroke}%
\pgfsetdash{}{0pt}%
\pgfpathmoveto{\pgfqpoint{3.567266in}{2.146696in}}%
\pgfpathlineto{\pgfqpoint{3.617467in}{2.132795in}}%
\pgfusepath{stroke}%
\end{pgfscope}%
\begin{pgfscope}%
\definecolor{textcolor}{rgb}{0.000000,0.000000,0.000000}%
\pgfsetstrokecolor{textcolor}%
\pgfsetfillcolor{textcolor}%
\pgftext[x=3.841546in,y=2.177124in,,top]{\color{textcolor}\rmfamily\fontsize{10.000000}{12.000000}\selectfont \(\displaystyle {1.5}\)}%
\end{pgfscope}%
\begin{pgfscope}%
\pgfsetrectcap%
\pgfsetroundjoin%
\pgfsetlinewidth{0.803000pt}%
\definecolor{currentstroke}{rgb}{0.000000,0.000000,0.000000}%
\pgfsetstrokecolor{currentstroke}%
\pgfsetdash{}{0pt}%
\pgfpathmoveto{\pgfqpoint{3.575514in}{2.328755in}}%
\pgfpathlineto{\pgfqpoint{3.625989in}{2.315003in}}%
\pgfusepath{stroke}%
\end{pgfscope}%
\begin{pgfscope}%
\definecolor{textcolor}{rgb}{0.000000,0.000000,0.000000}%
\pgfsetstrokecolor{textcolor}%
\pgfsetfillcolor{textcolor}%
\pgftext[x=3.851210in,y=2.358859in,,top]{\color{textcolor}\rmfamily\fontsize{10.000000}{12.000000}\selectfont \(\displaystyle {2.0}\)}%
\end{pgfscope}%
\begin{pgfscope}%
\pgfsetrectcap%
\pgfsetroundjoin%
\pgfsetlinewidth{0.803000pt}%
\definecolor{currentstroke}{rgb}{0.000000,0.000000,0.000000}%
\pgfsetstrokecolor{currentstroke}%
\pgfsetdash{}{0pt}%
\pgfpathmoveto{\pgfqpoint{3.583849in}{2.512738in}}%
\pgfpathlineto{\pgfqpoint{3.634601in}{2.499138in}}%
\pgfusepath{stroke}%
\end{pgfscope}%
\begin{pgfscope}%
\definecolor{textcolor}{rgb}{0.000000,0.000000,0.000000}%
\pgfsetstrokecolor{textcolor}%
\pgfsetfillcolor{textcolor}%
\pgftext[x=3.860975in,y=2.542509in,,top]{\color{textcolor}\rmfamily\fontsize{10.000000}{12.000000}\selectfont \(\displaystyle {2.5}\)}%
\end{pgfscope}%
\begin{pgfscope}%
\pgfsetrectcap%
\pgfsetroundjoin%
\pgfsetlinewidth{0.803000pt}%
\definecolor{currentstroke}{rgb}{0.000000,0.000000,0.000000}%
\pgfsetstrokecolor{currentstroke}%
\pgfsetdash{}{0pt}%
\pgfpathmoveto{\pgfqpoint{3.592273in}{2.698676in}}%
\pgfpathlineto{\pgfqpoint{3.643305in}{2.685232in}}%
\pgfusepath{stroke}%
\end{pgfscope}%
\begin{pgfscope}%
\definecolor{textcolor}{rgb}{0.000000,0.000000,0.000000}%
\pgfsetstrokecolor{textcolor}%
\pgfsetfillcolor{textcolor}%
\pgftext[x=3.870844in,y=2.728104in,,top]{\color{textcolor}\rmfamily\fontsize{10.000000}{12.000000}\selectfont \(\displaystyle {3.0}\)}%
\end{pgfscope}%
\begin{pgfscope}%
\pgfsetrectcap%
\pgfsetroundjoin%
\pgfsetlinewidth{0.803000pt}%
\definecolor{currentstroke}{rgb}{0.000000,0.000000,0.000000}%
\pgfsetstrokecolor{currentstroke}%
\pgfsetdash{}{0pt}%
\pgfpathmoveto{\pgfqpoint{3.600787in}{2.886600in}}%
\pgfpathlineto{\pgfqpoint{3.652102in}{2.873316in}}%
\pgfusepath{stroke}%
\end{pgfscope}%
\begin{pgfscope}%
\definecolor{textcolor}{rgb}{0.000000,0.000000,0.000000}%
\pgfsetstrokecolor{textcolor}%
\pgfsetfillcolor{textcolor}%
\pgftext[x=3.880819in,y=2.915677in,,top]{\color{textcolor}\rmfamily\fontsize{10.000000}{12.000000}\selectfont \(\displaystyle {3.5}\)}%
\end{pgfscope}%
\begin{pgfscope}%
\pgfsetrectcap%
\pgfsetroundjoin%
\pgfsetlinewidth{0.803000pt}%
\definecolor{currentstroke}{rgb}{0.000000,0.000000,0.000000}%
\pgfsetstrokecolor{currentstroke}%
\pgfsetdash{}{0pt}%
\pgfpathmoveto{\pgfqpoint{3.609392in}{3.076542in}}%
\pgfpathlineto{\pgfqpoint{3.660994in}{3.063422in}}%
\pgfusepath{stroke}%
\end{pgfscope}%
\begin{pgfscope}%
\definecolor{textcolor}{rgb}{0.000000,0.000000,0.000000}%
\pgfsetstrokecolor{textcolor}%
\pgfsetfillcolor{textcolor}%
\pgftext[x=3.890900in,y=3.105258in,,top]{\color{textcolor}\rmfamily\fontsize{10.000000}{12.000000}\selectfont \(\displaystyle {4.0}\)}%
\end{pgfscope}%
\begin{pgfscope}%
\pgfpathrectangle{\pgfqpoint{0.100000in}{0.212622in}}{\pgfqpoint{3.696000in}{3.696000in}}%
\pgfusepath{clip}%
\pgfsetrectcap%
\pgfsetroundjoin%
\pgfsetlinewidth{1.505625pt}%
\definecolor{currentstroke}{rgb}{0.121569,0.466667,0.705882}%
\pgfsetstrokecolor{currentstroke}%
\pgfsetdash{}{0pt}%
\pgfpathmoveto{\pgfqpoint{1.157843in}{1.642400in}}%
\pgfpathlineto{\pgfqpoint{2.897476in}{1.108605in}}%
\pgfusepath{stroke}%
\end{pgfscope}%
\begin{pgfscope}%
\pgfpathrectangle{\pgfqpoint{0.100000in}{0.212622in}}{\pgfqpoint{3.696000in}{3.696000in}}%
\pgfusepath{clip}%
\pgfsetbuttcap%
\pgfsetroundjoin%
\definecolor{currentfill}{rgb}{0.121569,0.466667,0.705882}%
\pgfsetfillcolor{currentfill}%
\pgfsetfillopacity{0.300000}%
\pgfsetlinewidth{1.003750pt}%
\definecolor{currentstroke}{rgb}{0.121569,0.466667,0.705882}%
\pgfsetstrokecolor{currentstroke}%
\pgfsetstrokeopacity{0.300000}%
\pgfsetdash{}{0pt}%
\pgfpathmoveto{\pgfqpoint{1.171205in}{1.619877in}}%
\pgfpathcurveto{\pgfqpoint{1.179442in}{1.619877in}}{\pgfqpoint{1.187342in}{1.623150in}}{\pgfqpoint{1.193166in}{1.628974in}}%
\pgfpathcurveto{\pgfqpoint{1.198990in}{1.634797in}}{\pgfqpoint{1.202262in}{1.642698in}}{\pgfqpoint{1.202262in}{1.650934in}}%
\pgfpathcurveto{\pgfqpoint{1.202262in}{1.659170in}}{\pgfqpoint{1.198990in}{1.667070in}}{\pgfqpoint{1.193166in}{1.672894in}}%
\pgfpathcurveto{\pgfqpoint{1.187342in}{1.678718in}}{\pgfqpoint{1.179442in}{1.681990in}}{\pgfqpoint{1.171205in}{1.681990in}}%
\pgfpathcurveto{\pgfqpoint{1.162969in}{1.681990in}}{\pgfqpoint{1.155069in}{1.678718in}}{\pgfqpoint{1.149245in}{1.672894in}}%
\pgfpathcurveto{\pgfqpoint{1.143421in}{1.667070in}}{\pgfqpoint{1.140149in}{1.659170in}}{\pgfqpoint{1.140149in}{1.650934in}}%
\pgfpathcurveto{\pgfqpoint{1.140149in}{1.642698in}}{\pgfqpoint{1.143421in}{1.634797in}}{\pgfqpoint{1.149245in}{1.628974in}}%
\pgfpathcurveto{\pgfqpoint{1.155069in}{1.623150in}}{\pgfqpoint{1.162969in}{1.619877in}}{\pgfqpoint{1.171205in}{1.619877in}}%
\pgfpathclose%
\pgfusepath{stroke,fill}%
\end{pgfscope}%
\begin{pgfscope}%
\pgfpathrectangle{\pgfqpoint{0.100000in}{0.212622in}}{\pgfqpoint{3.696000in}{3.696000in}}%
\pgfusepath{clip}%
\pgfsetbuttcap%
\pgfsetroundjoin%
\definecolor{currentfill}{rgb}{0.121569,0.466667,0.705882}%
\pgfsetfillcolor{currentfill}%
\pgfsetfillopacity{0.306254}%
\pgfsetlinewidth{1.003750pt}%
\definecolor{currentstroke}{rgb}{0.121569,0.466667,0.705882}%
\pgfsetstrokecolor{currentstroke}%
\pgfsetstrokeopacity{0.306254}%
\pgfsetdash{}{0pt}%
\pgfpathmoveto{\pgfqpoint{1.172413in}{1.619979in}}%
\pgfpathcurveto{\pgfqpoint{1.180649in}{1.619979in}}{\pgfqpoint{1.188550in}{1.623251in}}{\pgfqpoint{1.194373in}{1.629075in}}%
\pgfpathcurveto{\pgfqpoint{1.200197in}{1.634899in}}{\pgfqpoint{1.203470in}{1.642799in}}{\pgfqpoint{1.203470in}{1.651035in}}%
\pgfpathcurveto{\pgfqpoint{1.203470in}{1.659271in}}{\pgfqpoint{1.200197in}{1.667172in}}{\pgfqpoint{1.194373in}{1.672995in}}%
\pgfpathcurveto{\pgfqpoint{1.188550in}{1.678819in}}{\pgfqpoint{1.180649in}{1.682092in}}{\pgfqpoint{1.172413in}{1.682092in}}%
\pgfpathcurveto{\pgfqpoint{1.164177in}{1.682092in}}{\pgfqpoint{1.156277in}{1.678819in}}{\pgfqpoint{1.150453in}{1.672995in}}%
\pgfpathcurveto{\pgfqpoint{1.144629in}{1.667172in}}{\pgfqpoint{1.141357in}{1.659271in}}{\pgfqpoint{1.141357in}{1.651035in}}%
\pgfpathcurveto{\pgfqpoint{1.141357in}{1.642799in}}{\pgfqpoint{1.144629in}{1.634899in}}{\pgfqpoint{1.150453in}{1.629075in}}%
\pgfpathcurveto{\pgfqpoint{1.156277in}{1.623251in}}{\pgfqpoint{1.164177in}{1.619979in}}{\pgfqpoint{1.172413in}{1.619979in}}%
\pgfpathclose%
\pgfusepath{stroke,fill}%
\end{pgfscope}%
\begin{pgfscope}%
\pgfpathrectangle{\pgfqpoint{0.100000in}{0.212622in}}{\pgfqpoint{3.696000in}{3.696000in}}%
\pgfusepath{clip}%
\pgfsetbuttcap%
\pgfsetroundjoin%
\definecolor{currentfill}{rgb}{0.121569,0.466667,0.705882}%
\pgfsetfillcolor{currentfill}%
\pgfsetfillopacity{0.308990}%
\pgfsetlinewidth{1.003750pt}%
\definecolor{currentstroke}{rgb}{0.121569,0.466667,0.705882}%
\pgfsetstrokecolor{currentstroke}%
\pgfsetstrokeopacity{0.308990}%
\pgfsetdash{}{0pt}%
\pgfpathmoveto{\pgfqpoint{1.159546in}{1.621810in}}%
\pgfpathcurveto{\pgfqpoint{1.167783in}{1.621810in}}{\pgfqpoint{1.175683in}{1.625082in}}{\pgfqpoint{1.181507in}{1.630906in}}%
\pgfpathcurveto{\pgfqpoint{1.187330in}{1.636730in}}{\pgfqpoint{1.190603in}{1.644630in}}{\pgfqpoint{1.190603in}{1.652867in}}%
\pgfpathcurveto{\pgfqpoint{1.190603in}{1.661103in}}{\pgfqpoint{1.187330in}{1.669003in}}{\pgfqpoint{1.181507in}{1.674827in}}%
\pgfpathcurveto{\pgfqpoint{1.175683in}{1.680651in}}{\pgfqpoint{1.167783in}{1.683923in}}{\pgfqpoint{1.159546in}{1.683923in}}%
\pgfpathcurveto{\pgfqpoint{1.151310in}{1.683923in}}{\pgfqpoint{1.143410in}{1.680651in}}{\pgfqpoint{1.137586in}{1.674827in}}%
\pgfpathcurveto{\pgfqpoint{1.131762in}{1.669003in}}{\pgfqpoint{1.128490in}{1.661103in}}{\pgfqpoint{1.128490in}{1.652867in}}%
\pgfpathcurveto{\pgfqpoint{1.128490in}{1.644630in}}{\pgfqpoint{1.131762in}{1.636730in}}{\pgfqpoint{1.137586in}{1.630906in}}%
\pgfpathcurveto{\pgfqpoint{1.143410in}{1.625082in}}{\pgfqpoint{1.151310in}{1.621810in}}{\pgfqpoint{1.159546in}{1.621810in}}%
\pgfpathclose%
\pgfusepath{stroke,fill}%
\end{pgfscope}%
\begin{pgfscope}%
\pgfpathrectangle{\pgfqpoint{0.100000in}{0.212622in}}{\pgfqpoint{3.696000in}{3.696000in}}%
\pgfusepath{clip}%
\pgfsetbuttcap%
\pgfsetroundjoin%
\definecolor{currentfill}{rgb}{0.121569,0.466667,0.705882}%
\pgfsetfillcolor{currentfill}%
\pgfsetfillopacity{0.317243}%
\pgfsetlinewidth{1.003750pt}%
\definecolor{currentstroke}{rgb}{0.121569,0.466667,0.705882}%
\pgfsetstrokecolor{currentstroke}%
\pgfsetstrokeopacity{0.317243}%
\pgfsetdash{}{0pt}%
\pgfpathmoveto{\pgfqpoint{1.139835in}{1.624157in}}%
\pgfpathcurveto{\pgfqpoint{1.148072in}{1.624157in}}{\pgfqpoint{1.155972in}{1.627429in}}{\pgfqpoint{1.161796in}{1.633253in}}%
\pgfpathcurveto{\pgfqpoint{1.167620in}{1.639077in}}{\pgfqpoint{1.170892in}{1.646977in}}{\pgfqpoint{1.170892in}{1.655214in}}%
\pgfpathcurveto{\pgfqpoint{1.170892in}{1.663450in}}{\pgfqpoint{1.167620in}{1.671350in}}{\pgfqpoint{1.161796in}{1.677174in}}%
\pgfpathcurveto{\pgfqpoint{1.155972in}{1.682998in}}{\pgfqpoint{1.148072in}{1.686270in}}{\pgfqpoint{1.139835in}{1.686270in}}%
\pgfpathcurveto{\pgfqpoint{1.131599in}{1.686270in}}{\pgfqpoint{1.123699in}{1.682998in}}{\pgfqpoint{1.117875in}{1.677174in}}%
\pgfpathcurveto{\pgfqpoint{1.112051in}{1.671350in}}{\pgfqpoint{1.108779in}{1.663450in}}{\pgfqpoint{1.108779in}{1.655214in}}%
\pgfpathcurveto{\pgfqpoint{1.108779in}{1.646977in}}{\pgfqpoint{1.112051in}{1.639077in}}{\pgfqpoint{1.117875in}{1.633253in}}%
\pgfpathcurveto{\pgfqpoint{1.123699in}{1.627429in}}{\pgfqpoint{1.131599in}{1.624157in}}{\pgfqpoint{1.139835in}{1.624157in}}%
\pgfpathclose%
\pgfusepath{stroke,fill}%
\end{pgfscope}%
\begin{pgfscope}%
\pgfpathrectangle{\pgfqpoint{0.100000in}{0.212622in}}{\pgfqpoint{3.696000in}{3.696000in}}%
\pgfusepath{clip}%
\pgfsetbuttcap%
\pgfsetroundjoin%
\definecolor{currentfill}{rgb}{0.121569,0.466667,0.705882}%
\pgfsetfillcolor{currentfill}%
\pgfsetfillopacity{0.328636}%
\pgfsetlinewidth{1.003750pt}%
\definecolor{currentstroke}{rgb}{0.121569,0.466667,0.705882}%
\pgfsetstrokecolor{currentstroke}%
\pgfsetstrokeopacity{0.328636}%
\pgfsetdash{}{0pt}%
\pgfpathmoveto{\pgfqpoint{1.115555in}{1.624496in}}%
\pgfpathcurveto{\pgfqpoint{1.123791in}{1.624496in}}{\pgfqpoint{1.131691in}{1.627768in}}{\pgfqpoint{1.137515in}{1.633592in}}%
\pgfpathcurveto{\pgfqpoint{1.143339in}{1.639416in}}{\pgfqpoint{1.146611in}{1.647316in}}{\pgfqpoint{1.146611in}{1.655553in}}%
\pgfpathcurveto{\pgfqpoint{1.146611in}{1.663789in}}{\pgfqpoint{1.143339in}{1.671689in}}{\pgfqpoint{1.137515in}{1.677513in}}%
\pgfpathcurveto{\pgfqpoint{1.131691in}{1.683337in}}{\pgfqpoint{1.123791in}{1.686609in}}{\pgfqpoint{1.115555in}{1.686609in}}%
\pgfpathcurveto{\pgfqpoint{1.107319in}{1.686609in}}{\pgfqpoint{1.099419in}{1.683337in}}{\pgfqpoint{1.093595in}{1.677513in}}%
\pgfpathcurveto{\pgfqpoint{1.087771in}{1.671689in}}{\pgfqpoint{1.084498in}{1.663789in}}{\pgfqpoint{1.084498in}{1.655553in}}%
\pgfpathcurveto{\pgfqpoint{1.084498in}{1.647316in}}{\pgfqpoint{1.087771in}{1.639416in}}{\pgfqpoint{1.093595in}{1.633592in}}%
\pgfpathcurveto{\pgfqpoint{1.099419in}{1.627768in}}{\pgfqpoint{1.107319in}{1.624496in}}{\pgfqpoint{1.115555in}{1.624496in}}%
\pgfpathclose%
\pgfusepath{stroke,fill}%
\end{pgfscope}%
\begin{pgfscope}%
\pgfpathrectangle{\pgfqpoint{0.100000in}{0.212622in}}{\pgfqpoint{3.696000in}{3.696000in}}%
\pgfusepath{clip}%
\pgfsetbuttcap%
\pgfsetroundjoin%
\definecolor{currentfill}{rgb}{0.121569,0.466667,0.705882}%
\pgfsetfillcolor{currentfill}%
\pgfsetfillopacity{0.337005}%
\pgfsetlinewidth{1.003750pt}%
\definecolor{currentstroke}{rgb}{0.121569,0.466667,0.705882}%
\pgfsetstrokecolor{currentstroke}%
\pgfsetstrokeopacity{0.337005}%
\pgfsetdash{}{0pt}%
\pgfpathmoveto{\pgfqpoint{1.103956in}{1.623614in}}%
\pgfpathcurveto{\pgfqpoint{1.112192in}{1.623614in}}{\pgfqpoint{1.120092in}{1.626886in}}{\pgfqpoint{1.125916in}{1.632710in}}%
\pgfpathcurveto{\pgfqpoint{1.131740in}{1.638534in}}{\pgfqpoint{1.135012in}{1.646434in}}{\pgfqpoint{1.135012in}{1.654671in}}%
\pgfpathcurveto{\pgfqpoint{1.135012in}{1.662907in}}{\pgfqpoint{1.131740in}{1.670807in}}{\pgfqpoint{1.125916in}{1.676631in}}%
\pgfpathcurveto{\pgfqpoint{1.120092in}{1.682455in}}{\pgfqpoint{1.112192in}{1.685727in}}{\pgfqpoint{1.103956in}{1.685727in}}%
\pgfpathcurveto{\pgfqpoint{1.095719in}{1.685727in}}{\pgfqpoint{1.087819in}{1.682455in}}{\pgfqpoint{1.081995in}{1.676631in}}%
\pgfpathcurveto{\pgfqpoint{1.076171in}{1.670807in}}{\pgfqpoint{1.072899in}{1.662907in}}{\pgfqpoint{1.072899in}{1.654671in}}%
\pgfpathcurveto{\pgfqpoint{1.072899in}{1.646434in}}{\pgfqpoint{1.076171in}{1.638534in}}{\pgfqpoint{1.081995in}{1.632710in}}%
\pgfpathcurveto{\pgfqpoint{1.087819in}{1.626886in}}{\pgfqpoint{1.095719in}{1.623614in}}{\pgfqpoint{1.103956in}{1.623614in}}%
\pgfpathclose%
\pgfusepath{stroke,fill}%
\end{pgfscope}%
\begin{pgfscope}%
\pgfpathrectangle{\pgfqpoint{0.100000in}{0.212622in}}{\pgfqpoint{3.696000in}{3.696000in}}%
\pgfusepath{clip}%
\pgfsetbuttcap%
\pgfsetroundjoin%
\definecolor{currentfill}{rgb}{0.121569,0.466667,0.705882}%
\pgfsetfillcolor{currentfill}%
\pgfsetfillopacity{0.342731}%
\pgfsetlinewidth{1.003750pt}%
\definecolor{currentstroke}{rgb}{0.121569,0.466667,0.705882}%
\pgfsetstrokecolor{currentstroke}%
\pgfsetstrokeopacity{0.342731}%
\pgfsetdash{}{0pt}%
\pgfpathmoveto{\pgfqpoint{1.099380in}{1.622599in}}%
\pgfpathcurveto{\pgfqpoint{1.107616in}{1.622599in}}{\pgfqpoint{1.115516in}{1.625872in}}{\pgfqpoint{1.121340in}{1.631695in}}%
\pgfpathcurveto{\pgfqpoint{1.127164in}{1.637519in}}{\pgfqpoint{1.130436in}{1.645419in}}{\pgfqpoint{1.130436in}{1.653656in}}%
\pgfpathcurveto{\pgfqpoint{1.130436in}{1.661892in}}{\pgfqpoint{1.127164in}{1.669792in}}{\pgfqpoint{1.121340in}{1.675616in}}%
\pgfpathcurveto{\pgfqpoint{1.115516in}{1.681440in}}{\pgfqpoint{1.107616in}{1.684712in}}{\pgfqpoint{1.099380in}{1.684712in}}%
\pgfpathcurveto{\pgfqpoint{1.091144in}{1.684712in}}{\pgfqpoint{1.083244in}{1.681440in}}{\pgfqpoint{1.077420in}{1.675616in}}%
\pgfpathcurveto{\pgfqpoint{1.071596in}{1.669792in}}{\pgfqpoint{1.068323in}{1.661892in}}{\pgfqpoint{1.068323in}{1.653656in}}%
\pgfpathcurveto{\pgfqpoint{1.068323in}{1.645419in}}{\pgfqpoint{1.071596in}{1.637519in}}{\pgfqpoint{1.077420in}{1.631695in}}%
\pgfpathcurveto{\pgfqpoint{1.083244in}{1.625872in}}{\pgfqpoint{1.091144in}{1.622599in}}{\pgfqpoint{1.099380in}{1.622599in}}%
\pgfpathclose%
\pgfusepath{stroke,fill}%
\end{pgfscope}%
\begin{pgfscope}%
\pgfpathrectangle{\pgfqpoint{0.100000in}{0.212622in}}{\pgfqpoint{3.696000in}{3.696000in}}%
\pgfusepath{clip}%
\pgfsetbuttcap%
\pgfsetroundjoin%
\definecolor{currentfill}{rgb}{0.121569,0.466667,0.705882}%
\pgfsetfillcolor{currentfill}%
\pgfsetfillopacity{0.351155}%
\pgfsetlinewidth{1.003750pt}%
\definecolor{currentstroke}{rgb}{0.121569,0.466667,0.705882}%
\pgfsetstrokecolor{currentstroke}%
\pgfsetstrokeopacity{0.351155}%
\pgfsetdash{}{0pt}%
\pgfpathmoveto{\pgfqpoint{1.095697in}{1.620039in}}%
\pgfpathcurveto{\pgfqpoint{1.103933in}{1.620039in}}{\pgfqpoint{1.111833in}{1.623311in}}{\pgfqpoint{1.117657in}{1.629135in}}%
\pgfpathcurveto{\pgfqpoint{1.123481in}{1.634959in}}{\pgfqpoint{1.126754in}{1.642859in}}{\pgfqpoint{1.126754in}{1.651095in}}%
\pgfpathcurveto{\pgfqpoint{1.126754in}{1.659332in}}{\pgfqpoint{1.123481in}{1.667232in}}{\pgfqpoint{1.117657in}{1.673056in}}%
\pgfpathcurveto{\pgfqpoint{1.111833in}{1.678879in}}{\pgfqpoint{1.103933in}{1.682152in}}{\pgfqpoint{1.095697in}{1.682152in}}%
\pgfpathcurveto{\pgfqpoint{1.087461in}{1.682152in}}{\pgfqpoint{1.079561in}{1.678879in}}{\pgfqpoint{1.073737in}{1.673056in}}%
\pgfpathcurveto{\pgfqpoint{1.067913in}{1.667232in}}{\pgfqpoint{1.064641in}{1.659332in}}{\pgfqpoint{1.064641in}{1.651095in}}%
\pgfpathcurveto{\pgfqpoint{1.064641in}{1.642859in}}{\pgfqpoint{1.067913in}{1.634959in}}{\pgfqpoint{1.073737in}{1.629135in}}%
\pgfpathcurveto{\pgfqpoint{1.079561in}{1.623311in}}{\pgfqpoint{1.087461in}{1.620039in}}{\pgfqpoint{1.095697in}{1.620039in}}%
\pgfpathclose%
\pgfusepath{stroke,fill}%
\end{pgfscope}%
\begin{pgfscope}%
\pgfpathrectangle{\pgfqpoint{0.100000in}{0.212622in}}{\pgfqpoint{3.696000in}{3.696000in}}%
\pgfusepath{clip}%
\pgfsetbuttcap%
\pgfsetroundjoin%
\definecolor{currentfill}{rgb}{0.121569,0.466667,0.705882}%
\pgfsetfillcolor{currentfill}%
\pgfsetfillopacity{0.368162}%
\pgfsetlinewidth{1.003750pt}%
\definecolor{currentstroke}{rgb}{0.121569,0.466667,0.705882}%
\pgfsetstrokecolor{currentstroke}%
\pgfsetstrokeopacity{0.368162}%
\pgfsetdash{}{0pt}%
\pgfpathmoveto{\pgfqpoint{1.096068in}{1.614909in}}%
\pgfpathcurveto{\pgfqpoint{1.104304in}{1.614909in}}{\pgfqpoint{1.112204in}{1.618181in}}{\pgfqpoint{1.118028in}{1.624005in}}%
\pgfpathcurveto{\pgfqpoint{1.123852in}{1.629829in}}{\pgfqpoint{1.127124in}{1.637729in}}{\pgfqpoint{1.127124in}{1.645965in}}%
\pgfpathcurveto{\pgfqpoint{1.127124in}{1.654202in}}{\pgfqpoint{1.123852in}{1.662102in}}{\pgfqpoint{1.118028in}{1.667926in}}%
\pgfpathcurveto{\pgfqpoint{1.112204in}{1.673750in}}{\pgfqpoint{1.104304in}{1.677022in}}{\pgfqpoint{1.096068in}{1.677022in}}%
\pgfpathcurveto{\pgfqpoint{1.087831in}{1.677022in}}{\pgfqpoint{1.079931in}{1.673750in}}{\pgfqpoint{1.074107in}{1.667926in}}%
\pgfpathcurveto{\pgfqpoint{1.068283in}{1.662102in}}{\pgfqpoint{1.065011in}{1.654202in}}{\pgfqpoint{1.065011in}{1.645965in}}%
\pgfpathcurveto{\pgfqpoint{1.065011in}{1.637729in}}{\pgfqpoint{1.068283in}{1.629829in}}{\pgfqpoint{1.074107in}{1.624005in}}%
\pgfpathcurveto{\pgfqpoint{1.079931in}{1.618181in}}{\pgfqpoint{1.087831in}{1.614909in}}{\pgfqpoint{1.096068in}{1.614909in}}%
\pgfpathclose%
\pgfusepath{stroke,fill}%
\end{pgfscope}%
\begin{pgfscope}%
\pgfpathrectangle{\pgfqpoint{0.100000in}{0.212622in}}{\pgfqpoint{3.696000in}{3.696000in}}%
\pgfusepath{clip}%
\pgfsetbuttcap%
\pgfsetroundjoin%
\definecolor{currentfill}{rgb}{0.121569,0.466667,0.705882}%
\pgfsetfillcolor{currentfill}%
\pgfsetfillopacity{0.390418}%
\pgfsetlinewidth{1.003750pt}%
\definecolor{currentstroke}{rgb}{0.121569,0.466667,0.705882}%
\pgfsetstrokecolor{currentstroke}%
\pgfsetstrokeopacity{0.390418}%
\pgfsetdash{}{0pt}%
\pgfpathmoveto{\pgfqpoint{1.106965in}{1.609279in}}%
\pgfpathcurveto{\pgfqpoint{1.115202in}{1.609279in}}{\pgfqpoint{1.123102in}{1.612552in}}{\pgfqpoint{1.128926in}{1.618375in}}%
\pgfpathcurveto{\pgfqpoint{1.134750in}{1.624199in}}{\pgfqpoint{1.138022in}{1.632099in}}{\pgfqpoint{1.138022in}{1.640336in}}%
\pgfpathcurveto{\pgfqpoint{1.138022in}{1.648572in}}{\pgfqpoint{1.134750in}{1.656472in}}{\pgfqpoint{1.128926in}{1.662296in}}%
\pgfpathcurveto{\pgfqpoint{1.123102in}{1.668120in}}{\pgfqpoint{1.115202in}{1.671392in}}{\pgfqpoint{1.106965in}{1.671392in}}%
\pgfpathcurveto{\pgfqpoint{1.098729in}{1.671392in}}{\pgfqpoint{1.090829in}{1.668120in}}{\pgfqpoint{1.085005in}{1.662296in}}%
\pgfpathcurveto{\pgfqpoint{1.079181in}{1.656472in}}{\pgfqpoint{1.075909in}{1.648572in}}{\pgfqpoint{1.075909in}{1.640336in}}%
\pgfpathcurveto{\pgfqpoint{1.075909in}{1.632099in}}{\pgfqpoint{1.079181in}{1.624199in}}{\pgfqpoint{1.085005in}{1.618375in}}%
\pgfpathcurveto{\pgfqpoint{1.090829in}{1.612552in}}{\pgfqpoint{1.098729in}{1.609279in}}{\pgfqpoint{1.106965in}{1.609279in}}%
\pgfpathclose%
\pgfusepath{stroke,fill}%
\end{pgfscope}%
\begin{pgfscope}%
\pgfpathrectangle{\pgfqpoint{0.100000in}{0.212622in}}{\pgfqpoint{3.696000in}{3.696000in}}%
\pgfusepath{clip}%
\pgfsetbuttcap%
\pgfsetroundjoin%
\definecolor{currentfill}{rgb}{0.121569,0.466667,0.705882}%
\pgfsetfillcolor{currentfill}%
\pgfsetfillopacity{0.412527}%
\pgfsetlinewidth{1.003750pt}%
\definecolor{currentstroke}{rgb}{0.121569,0.466667,0.705882}%
\pgfsetstrokecolor{currentstroke}%
\pgfsetstrokeopacity{0.412527}%
\pgfsetdash{}{0pt}%
\pgfpathmoveto{\pgfqpoint{1.124820in}{1.601175in}}%
\pgfpathcurveto{\pgfqpoint{1.133056in}{1.601175in}}{\pgfqpoint{1.140956in}{1.604447in}}{\pgfqpoint{1.146780in}{1.610271in}}%
\pgfpathcurveto{\pgfqpoint{1.152604in}{1.616095in}}{\pgfqpoint{1.155876in}{1.623995in}}{\pgfqpoint{1.155876in}{1.632231in}}%
\pgfpathcurveto{\pgfqpoint{1.155876in}{1.640468in}}{\pgfqpoint{1.152604in}{1.648368in}}{\pgfqpoint{1.146780in}{1.654192in}}%
\pgfpathcurveto{\pgfqpoint{1.140956in}{1.660016in}}{\pgfqpoint{1.133056in}{1.663288in}}{\pgfqpoint{1.124820in}{1.663288in}}%
\pgfpathcurveto{\pgfqpoint{1.116584in}{1.663288in}}{\pgfqpoint{1.108683in}{1.660016in}}{\pgfqpoint{1.102860in}{1.654192in}}%
\pgfpathcurveto{\pgfqpoint{1.097036in}{1.648368in}}{\pgfqpoint{1.093763in}{1.640468in}}{\pgfqpoint{1.093763in}{1.632231in}}%
\pgfpathcurveto{\pgfqpoint{1.093763in}{1.623995in}}{\pgfqpoint{1.097036in}{1.616095in}}{\pgfqpoint{1.102860in}{1.610271in}}%
\pgfpathcurveto{\pgfqpoint{1.108683in}{1.604447in}}{\pgfqpoint{1.116584in}{1.601175in}}{\pgfqpoint{1.124820in}{1.601175in}}%
\pgfpathclose%
\pgfusepath{stroke,fill}%
\end{pgfscope}%
\begin{pgfscope}%
\pgfpathrectangle{\pgfqpoint{0.100000in}{0.212622in}}{\pgfqpoint{3.696000in}{3.696000in}}%
\pgfusepath{clip}%
\pgfsetbuttcap%
\pgfsetroundjoin%
\definecolor{currentfill}{rgb}{0.121569,0.466667,0.705882}%
\pgfsetfillcolor{currentfill}%
\pgfsetfillopacity{0.422732}%
\pgfsetlinewidth{1.003750pt}%
\definecolor{currentstroke}{rgb}{0.121569,0.466667,0.705882}%
\pgfsetstrokecolor{currentstroke}%
\pgfsetstrokeopacity{0.422732}%
\pgfsetdash{}{0pt}%
\pgfpathmoveto{\pgfqpoint{1.137914in}{1.597681in}}%
\pgfpathcurveto{\pgfqpoint{1.146150in}{1.597681in}}{\pgfqpoint{1.154050in}{1.600954in}}{\pgfqpoint{1.159874in}{1.606778in}}%
\pgfpathcurveto{\pgfqpoint{1.165698in}{1.612602in}}{\pgfqpoint{1.168971in}{1.620502in}}{\pgfqpoint{1.168971in}{1.628738in}}%
\pgfpathcurveto{\pgfqpoint{1.168971in}{1.636974in}}{\pgfqpoint{1.165698in}{1.644874in}}{\pgfqpoint{1.159874in}{1.650698in}}%
\pgfpathcurveto{\pgfqpoint{1.154050in}{1.656522in}}{\pgfqpoint{1.146150in}{1.659794in}}{\pgfqpoint{1.137914in}{1.659794in}}%
\pgfpathcurveto{\pgfqpoint{1.129678in}{1.659794in}}{\pgfqpoint{1.121778in}{1.656522in}}{\pgfqpoint{1.115954in}{1.650698in}}%
\pgfpathcurveto{\pgfqpoint{1.110130in}{1.644874in}}{\pgfqpoint{1.106858in}{1.636974in}}{\pgfqpoint{1.106858in}{1.628738in}}%
\pgfpathcurveto{\pgfqpoint{1.106858in}{1.620502in}}{\pgfqpoint{1.110130in}{1.612602in}}{\pgfqpoint{1.115954in}{1.606778in}}%
\pgfpathcurveto{\pgfqpoint{1.121778in}{1.600954in}}{\pgfqpoint{1.129678in}{1.597681in}}{\pgfqpoint{1.137914in}{1.597681in}}%
\pgfpathclose%
\pgfusepath{stroke,fill}%
\end{pgfscope}%
\begin{pgfscope}%
\pgfpathrectangle{\pgfqpoint{0.100000in}{0.212622in}}{\pgfqpoint{3.696000in}{3.696000in}}%
\pgfusepath{clip}%
\pgfsetbuttcap%
\pgfsetroundjoin%
\definecolor{currentfill}{rgb}{0.121569,0.466667,0.705882}%
\pgfsetfillcolor{currentfill}%
\pgfsetfillopacity{0.427839}%
\pgfsetlinewidth{1.003750pt}%
\definecolor{currentstroke}{rgb}{0.121569,0.466667,0.705882}%
\pgfsetstrokecolor{currentstroke}%
\pgfsetstrokeopacity{0.427839}%
\pgfsetdash{}{0pt}%
\pgfpathmoveto{\pgfqpoint{1.145152in}{1.594470in}}%
\pgfpathcurveto{\pgfqpoint{1.153388in}{1.594470in}}{\pgfqpoint{1.161288in}{1.597743in}}{\pgfqpoint{1.167112in}{1.603567in}}%
\pgfpathcurveto{\pgfqpoint{1.172936in}{1.609390in}}{\pgfqpoint{1.176208in}{1.617291in}}{\pgfqpoint{1.176208in}{1.625527in}}%
\pgfpathcurveto{\pgfqpoint{1.176208in}{1.633763in}}{\pgfqpoint{1.172936in}{1.641663in}}{\pgfqpoint{1.167112in}{1.647487in}}%
\pgfpathcurveto{\pgfqpoint{1.161288in}{1.653311in}}{\pgfqpoint{1.153388in}{1.656583in}}{\pgfqpoint{1.145152in}{1.656583in}}%
\pgfpathcurveto{\pgfqpoint{1.136916in}{1.656583in}}{\pgfqpoint{1.129015in}{1.653311in}}{\pgfqpoint{1.123192in}{1.647487in}}%
\pgfpathcurveto{\pgfqpoint{1.117368in}{1.641663in}}{\pgfqpoint{1.114095in}{1.633763in}}{\pgfqpoint{1.114095in}{1.625527in}}%
\pgfpathcurveto{\pgfqpoint{1.114095in}{1.617291in}}{\pgfqpoint{1.117368in}{1.609390in}}{\pgfqpoint{1.123192in}{1.603567in}}%
\pgfpathcurveto{\pgfqpoint{1.129015in}{1.597743in}}{\pgfqpoint{1.136916in}{1.594470in}}{\pgfqpoint{1.145152in}{1.594470in}}%
\pgfpathclose%
\pgfusepath{stroke,fill}%
\end{pgfscope}%
\begin{pgfscope}%
\pgfpathrectangle{\pgfqpoint{0.100000in}{0.212622in}}{\pgfqpoint{3.696000in}{3.696000in}}%
\pgfusepath{clip}%
\pgfsetbuttcap%
\pgfsetroundjoin%
\definecolor{currentfill}{rgb}{0.121569,0.466667,0.705882}%
\pgfsetfillcolor{currentfill}%
\pgfsetfillopacity{0.433610}%
\pgfsetlinewidth{1.003750pt}%
\definecolor{currentstroke}{rgb}{0.121569,0.466667,0.705882}%
\pgfsetstrokecolor{currentstroke}%
\pgfsetstrokeopacity{0.433610}%
\pgfsetdash{}{0pt}%
\pgfpathmoveto{\pgfqpoint{1.155826in}{1.590441in}}%
\pgfpathcurveto{\pgfqpoint{1.164063in}{1.590441in}}{\pgfqpoint{1.171963in}{1.593714in}}{\pgfqpoint{1.177787in}{1.599538in}}%
\pgfpathcurveto{\pgfqpoint{1.183611in}{1.605362in}}{\pgfqpoint{1.186883in}{1.613262in}}{\pgfqpoint{1.186883in}{1.621498in}}%
\pgfpathcurveto{\pgfqpoint{1.186883in}{1.629734in}}{\pgfqpoint{1.183611in}{1.637634in}}{\pgfqpoint{1.177787in}{1.643458in}}%
\pgfpathcurveto{\pgfqpoint{1.171963in}{1.649282in}}{\pgfqpoint{1.164063in}{1.652554in}}{\pgfqpoint{1.155826in}{1.652554in}}%
\pgfpathcurveto{\pgfqpoint{1.147590in}{1.652554in}}{\pgfqpoint{1.139690in}{1.649282in}}{\pgfqpoint{1.133866in}{1.643458in}}%
\pgfpathcurveto{\pgfqpoint{1.128042in}{1.637634in}}{\pgfqpoint{1.124770in}{1.629734in}}{\pgfqpoint{1.124770in}{1.621498in}}%
\pgfpathcurveto{\pgfqpoint{1.124770in}{1.613262in}}{\pgfqpoint{1.128042in}{1.605362in}}{\pgfqpoint{1.133866in}{1.599538in}}%
\pgfpathcurveto{\pgfqpoint{1.139690in}{1.593714in}}{\pgfqpoint{1.147590in}{1.590441in}}{\pgfqpoint{1.155826in}{1.590441in}}%
\pgfpathclose%
\pgfusepath{stroke,fill}%
\end{pgfscope}%
\begin{pgfscope}%
\pgfpathrectangle{\pgfqpoint{0.100000in}{0.212622in}}{\pgfqpoint{3.696000in}{3.696000in}}%
\pgfusepath{clip}%
\pgfsetbuttcap%
\pgfsetroundjoin%
\definecolor{currentfill}{rgb}{0.121569,0.466667,0.705882}%
\pgfsetfillcolor{currentfill}%
\pgfsetfillopacity{0.441622}%
\pgfsetlinewidth{1.003750pt}%
\definecolor{currentstroke}{rgb}{0.121569,0.466667,0.705882}%
\pgfsetstrokecolor{currentstroke}%
\pgfsetstrokeopacity{0.441622}%
\pgfsetdash{}{0pt}%
\pgfpathmoveto{\pgfqpoint{1.172717in}{1.585047in}}%
\pgfpathcurveto{\pgfqpoint{1.180953in}{1.585047in}}{\pgfqpoint{1.188853in}{1.588319in}}{\pgfqpoint{1.194677in}{1.594143in}}%
\pgfpathcurveto{\pgfqpoint{1.200501in}{1.599967in}}{\pgfqpoint{1.203774in}{1.607867in}}{\pgfqpoint{1.203774in}{1.616103in}}%
\pgfpathcurveto{\pgfqpoint{1.203774in}{1.624339in}}{\pgfqpoint{1.200501in}{1.632239in}}{\pgfqpoint{1.194677in}{1.638063in}}%
\pgfpathcurveto{\pgfqpoint{1.188853in}{1.643887in}}{\pgfqpoint{1.180953in}{1.647160in}}{\pgfqpoint{1.172717in}{1.647160in}}%
\pgfpathcurveto{\pgfqpoint{1.164481in}{1.647160in}}{\pgfqpoint{1.156581in}{1.643887in}}{\pgfqpoint{1.150757in}{1.638063in}}%
\pgfpathcurveto{\pgfqpoint{1.144933in}{1.632239in}}{\pgfqpoint{1.141661in}{1.624339in}}{\pgfqpoint{1.141661in}{1.616103in}}%
\pgfpathcurveto{\pgfqpoint{1.141661in}{1.607867in}}{\pgfqpoint{1.144933in}{1.599967in}}{\pgfqpoint{1.150757in}{1.594143in}}%
\pgfpathcurveto{\pgfqpoint{1.156581in}{1.588319in}}{\pgfqpoint{1.164481in}{1.585047in}}{\pgfqpoint{1.172717in}{1.585047in}}%
\pgfpathclose%
\pgfusepath{stroke,fill}%
\end{pgfscope}%
\begin{pgfscope}%
\pgfpathrectangle{\pgfqpoint{0.100000in}{0.212622in}}{\pgfqpoint{3.696000in}{3.696000in}}%
\pgfusepath{clip}%
\pgfsetbuttcap%
\pgfsetroundjoin%
\definecolor{currentfill}{rgb}{0.121569,0.466667,0.705882}%
\pgfsetfillcolor{currentfill}%
\pgfsetfillopacity{0.452283}%
\pgfsetlinewidth{1.003750pt}%
\definecolor{currentstroke}{rgb}{0.121569,0.466667,0.705882}%
\pgfsetstrokecolor{currentstroke}%
\pgfsetstrokeopacity{0.452283}%
\pgfsetdash{}{0pt}%
\pgfpathmoveto{\pgfqpoint{1.199954in}{1.577658in}}%
\pgfpathcurveto{\pgfqpoint{1.208190in}{1.577658in}}{\pgfqpoint{1.216090in}{1.580930in}}{\pgfqpoint{1.221914in}{1.586754in}}%
\pgfpathcurveto{\pgfqpoint{1.227738in}{1.592578in}}{\pgfqpoint{1.231010in}{1.600478in}}{\pgfqpoint{1.231010in}{1.608714in}}%
\pgfpathcurveto{\pgfqpoint{1.231010in}{1.616951in}}{\pgfqpoint{1.227738in}{1.624851in}}{\pgfqpoint{1.221914in}{1.630675in}}%
\pgfpathcurveto{\pgfqpoint{1.216090in}{1.636498in}}{\pgfqpoint{1.208190in}{1.639771in}}{\pgfqpoint{1.199954in}{1.639771in}}%
\pgfpathcurveto{\pgfqpoint{1.191717in}{1.639771in}}{\pgfqpoint{1.183817in}{1.636498in}}{\pgfqpoint{1.177993in}{1.630675in}}%
\pgfpathcurveto{\pgfqpoint{1.172169in}{1.624851in}}{\pgfqpoint{1.168897in}{1.616951in}}{\pgfqpoint{1.168897in}{1.608714in}}%
\pgfpathcurveto{\pgfqpoint{1.168897in}{1.600478in}}{\pgfqpoint{1.172169in}{1.592578in}}{\pgfqpoint{1.177993in}{1.586754in}}%
\pgfpathcurveto{\pgfqpoint{1.183817in}{1.580930in}}{\pgfqpoint{1.191717in}{1.577658in}}{\pgfqpoint{1.199954in}{1.577658in}}%
\pgfpathclose%
\pgfusepath{stroke,fill}%
\end{pgfscope}%
\begin{pgfscope}%
\pgfpathrectangle{\pgfqpoint{0.100000in}{0.212622in}}{\pgfqpoint{3.696000in}{3.696000in}}%
\pgfusepath{clip}%
\pgfsetbuttcap%
\pgfsetroundjoin%
\definecolor{currentfill}{rgb}{0.121569,0.466667,0.705882}%
\pgfsetfillcolor{currentfill}%
\pgfsetfillopacity{0.464237}%
\pgfsetlinewidth{1.003750pt}%
\definecolor{currentstroke}{rgb}{0.121569,0.466667,0.705882}%
\pgfsetstrokecolor{currentstroke}%
\pgfsetstrokeopacity{0.464237}%
\pgfsetdash{}{0pt}%
\pgfpathmoveto{\pgfqpoint{1.231962in}{1.569058in}}%
\pgfpathcurveto{\pgfqpoint{1.240199in}{1.569058in}}{\pgfqpoint{1.248099in}{1.572330in}}{\pgfqpoint{1.253923in}{1.578154in}}%
\pgfpathcurveto{\pgfqpoint{1.259746in}{1.583978in}}{\pgfqpoint{1.263019in}{1.591878in}}{\pgfqpoint{1.263019in}{1.600115in}}%
\pgfpathcurveto{\pgfqpoint{1.263019in}{1.608351in}}{\pgfqpoint{1.259746in}{1.616251in}}{\pgfqpoint{1.253923in}{1.622075in}}%
\pgfpathcurveto{\pgfqpoint{1.248099in}{1.627899in}}{\pgfqpoint{1.240199in}{1.631171in}}{\pgfqpoint{1.231962in}{1.631171in}}%
\pgfpathcurveto{\pgfqpoint{1.223726in}{1.631171in}}{\pgfqpoint{1.215826in}{1.627899in}}{\pgfqpoint{1.210002in}{1.622075in}}%
\pgfpathcurveto{\pgfqpoint{1.204178in}{1.616251in}}{\pgfqpoint{1.200906in}{1.608351in}}{\pgfqpoint{1.200906in}{1.600115in}}%
\pgfpathcurveto{\pgfqpoint{1.200906in}{1.591878in}}{\pgfqpoint{1.204178in}{1.583978in}}{\pgfqpoint{1.210002in}{1.578154in}}%
\pgfpathcurveto{\pgfqpoint{1.215826in}{1.572330in}}{\pgfqpoint{1.223726in}{1.569058in}}{\pgfqpoint{1.231962in}{1.569058in}}%
\pgfpathclose%
\pgfusepath{stroke,fill}%
\end{pgfscope}%
\begin{pgfscope}%
\pgfpathrectangle{\pgfqpoint{0.100000in}{0.212622in}}{\pgfqpoint{3.696000in}{3.696000in}}%
\pgfusepath{clip}%
\pgfsetbuttcap%
\pgfsetroundjoin%
\definecolor{currentfill}{rgb}{0.121569,0.466667,0.705882}%
\pgfsetfillcolor{currentfill}%
\pgfsetfillopacity{0.477421}%
\pgfsetlinewidth{1.003750pt}%
\definecolor{currentstroke}{rgb}{0.121569,0.466667,0.705882}%
\pgfsetstrokecolor{currentstroke}%
\pgfsetstrokeopacity{0.477421}%
\pgfsetdash{}{0pt}%
\pgfpathmoveto{\pgfqpoint{1.265371in}{1.555672in}}%
\pgfpathcurveto{\pgfqpoint{1.273607in}{1.555672in}}{\pgfqpoint{1.281507in}{1.558944in}}{\pgfqpoint{1.287331in}{1.564768in}}%
\pgfpathcurveto{\pgfqpoint{1.293155in}{1.570592in}}{\pgfqpoint{1.296427in}{1.578492in}}{\pgfqpoint{1.296427in}{1.586728in}}%
\pgfpathcurveto{\pgfqpoint{1.296427in}{1.594964in}}{\pgfqpoint{1.293155in}{1.602865in}}{\pgfqpoint{1.287331in}{1.608688in}}%
\pgfpathcurveto{\pgfqpoint{1.281507in}{1.614512in}}{\pgfqpoint{1.273607in}{1.617785in}}{\pgfqpoint{1.265371in}{1.617785in}}%
\pgfpathcurveto{\pgfqpoint{1.257134in}{1.617785in}}{\pgfqpoint{1.249234in}{1.614512in}}{\pgfqpoint{1.243410in}{1.608688in}}%
\pgfpathcurveto{\pgfqpoint{1.237586in}{1.602865in}}{\pgfqpoint{1.234314in}{1.594964in}}{\pgfqpoint{1.234314in}{1.586728in}}%
\pgfpathcurveto{\pgfqpoint{1.234314in}{1.578492in}}{\pgfqpoint{1.237586in}{1.570592in}}{\pgfqpoint{1.243410in}{1.564768in}}%
\pgfpathcurveto{\pgfqpoint{1.249234in}{1.558944in}}{\pgfqpoint{1.257134in}{1.555672in}}{\pgfqpoint{1.265371in}{1.555672in}}%
\pgfpathclose%
\pgfusepath{stroke,fill}%
\end{pgfscope}%
\begin{pgfscope}%
\pgfpathrectangle{\pgfqpoint{0.100000in}{0.212622in}}{\pgfqpoint{3.696000in}{3.696000in}}%
\pgfusepath{clip}%
\pgfsetbuttcap%
\pgfsetroundjoin%
\definecolor{currentfill}{rgb}{0.121569,0.466667,0.705882}%
\pgfsetfillcolor{currentfill}%
\pgfsetfillopacity{0.492082}%
\pgfsetlinewidth{1.003750pt}%
\definecolor{currentstroke}{rgb}{0.121569,0.466667,0.705882}%
\pgfsetstrokecolor{currentstroke}%
\pgfsetstrokeopacity{0.492082}%
\pgfsetdash{}{0pt}%
\pgfpathmoveto{\pgfqpoint{1.305681in}{1.542949in}}%
\pgfpathcurveto{\pgfqpoint{1.313918in}{1.542949in}}{\pgfqpoint{1.321818in}{1.546221in}}{\pgfqpoint{1.327642in}{1.552045in}}%
\pgfpathcurveto{\pgfqpoint{1.333465in}{1.557869in}}{\pgfqpoint{1.336738in}{1.565769in}}{\pgfqpoint{1.336738in}{1.574005in}}%
\pgfpathcurveto{\pgfqpoint{1.336738in}{1.582242in}}{\pgfqpoint{1.333465in}{1.590142in}}{\pgfqpoint{1.327642in}{1.595966in}}%
\pgfpathcurveto{\pgfqpoint{1.321818in}{1.601790in}}{\pgfqpoint{1.313918in}{1.605062in}}{\pgfqpoint{1.305681in}{1.605062in}}%
\pgfpathcurveto{\pgfqpoint{1.297445in}{1.605062in}}{\pgfqpoint{1.289545in}{1.601790in}}{\pgfqpoint{1.283721in}{1.595966in}}%
\pgfpathcurveto{\pgfqpoint{1.277897in}{1.590142in}}{\pgfqpoint{1.274625in}{1.582242in}}{\pgfqpoint{1.274625in}{1.574005in}}%
\pgfpathcurveto{\pgfqpoint{1.274625in}{1.565769in}}{\pgfqpoint{1.277897in}{1.557869in}}{\pgfqpoint{1.283721in}{1.552045in}}%
\pgfpathcurveto{\pgfqpoint{1.289545in}{1.546221in}}{\pgfqpoint{1.297445in}{1.542949in}}{\pgfqpoint{1.305681in}{1.542949in}}%
\pgfpathclose%
\pgfusepath{stroke,fill}%
\end{pgfscope}%
\begin{pgfscope}%
\pgfpathrectangle{\pgfqpoint{0.100000in}{0.212622in}}{\pgfqpoint{3.696000in}{3.696000in}}%
\pgfusepath{clip}%
\pgfsetbuttcap%
\pgfsetroundjoin%
\definecolor{currentfill}{rgb}{0.121569,0.466667,0.705882}%
\pgfsetfillcolor{currentfill}%
\pgfsetfillopacity{0.500100}%
\pgfsetlinewidth{1.003750pt}%
\definecolor{currentstroke}{rgb}{0.121569,0.466667,0.705882}%
\pgfsetstrokecolor{currentstroke}%
\pgfsetstrokeopacity{0.500100}%
\pgfsetdash{}{0pt}%
\pgfpathmoveto{\pgfqpoint{1.327765in}{1.535406in}}%
\pgfpathcurveto{\pgfqpoint{1.336001in}{1.535406in}}{\pgfqpoint{1.343901in}{1.538679in}}{\pgfqpoint{1.349725in}{1.544502in}}%
\pgfpathcurveto{\pgfqpoint{1.355549in}{1.550326in}}{\pgfqpoint{1.358821in}{1.558226in}}{\pgfqpoint{1.358821in}{1.566463in}}%
\pgfpathcurveto{\pgfqpoint{1.358821in}{1.574699in}}{\pgfqpoint{1.355549in}{1.582599in}}{\pgfqpoint{1.349725in}{1.588423in}}%
\pgfpathcurveto{\pgfqpoint{1.343901in}{1.594247in}}{\pgfqpoint{1.336001in}{1.597519in}}{\pgfqpoint{1.327765in}{1.597519in}}%
\pgfpathcurveto{\pgfqpoint{1.319529in}{1.597519in}}{\pgfqpoint{1.311629in}{1.594247in}}{\pgfqpoint{1.305805in}{1.588423in}}%
\pgfpathcurveto{\pgfqpoint{1.299981in}{1.582599in}}{\pgfqpoint{1.296708in}{1.574699in}}{\pgfqpoint{1.296708in}{1.566463in}}%
\pgfpathcurveto{\pgfqpoint{1.296708in}{1.558226in}}{\pgfqpoint{1.299981in}{1.550326in}}{\pgfqpoint{1.305805in}{1.544502in}}%
\pgfpathcurveto{\pgfqpoint{1.311629in}{1.538679in}}{\pgfqpoint{1.319529in}{1.535406in}}{\pgfqpoint{1.327765in}{1.535406in}}%
\pgfpathclose%
\pgfusepath{stroke,fill}%
\end{pgfscope}%
\begin{pgfscope}%
\pgfpathrectangle{\pgfqpoint{0.100000in}{0.212622in}}{\pgfqpoint{3.696000in}{3.696000in}}%
\pgfusepath{clip}%
\pgfsetbuttcap%
\pgfsetroundjoin%
\definecolor{currentfill}{rgb}{0.121569,0.466667,0.705882}%
\pgfsetfillcolor{currentfill}%
\pgfsetfillopacity{0.509285}%
\pgfsetlinewidth{1.003750pt}%
\definecolor{currentstroke}{rgb}{0.121569,0.466667,0.705882}%
\pgfsetstrokecolor{currentstroke}%
\pgfsetstrokeopacity{0.509285}%
\pgfsetdash{}{0pt}%
\pgfpathmoveto{\pgfqpoint{1.353452in}{1.526822in}}%
\pgfpathcurveto{\pgfqpoint{1.361688in}{1.526822in}}{\pgfqpoint{1.369588in}{1.530094in}}{\pgfqpoint{1.375412in}{1.535918in}}%
\pgfpathcurveto{\pgfqpoint{1.381236in}{1.541742in}}{\pgfqpoint{1.384509in}{1.549642in}}{\pgfqpoint{1.384509in}{1.557879in}}%
\pgfpathcurveto{\pgfqpoint{1.384509in}{1.566115in}}{\pgfqpoint{1.381236in}{1.574015in}}{\pgfqpoint{1.375412in}{1.579839in}}%
\pgfpathcurveto{\pgfqpoint{1.369588in}{1.585663in}}{\pgfqpoint{1.361688in}{1.588935in}}{\pgfqpoint{1.353452in}{1.588935in}}%
\pgfpathcurveto{\pgfqpoint{1.345216in}{1.588935in}}{\pgfqpoint{1.337316in}{1.585663in}}{\pgfqpoint{1.331492in}{1.579839in}}%
\pgfpathcurveto{\pgfqpoint{1.325668in}{1.574015in}}{\pgfqpoint{1.322396in}{1.566115in}}{\pgfqpoint{1.322396in}{1.557879in}}%
\pgfpathcurveto{\pgfqpoint{1.322396in}{1.549642in}}{\pgfqpoint{1.325668in}{1.541742in}}{\pgfqpoint{1.331492in}{1.535918in}}%
\pgfpathcurveto{\pgfqpoint{1.337316in}{1.530094in}}{\pgfqpoint{1.345216in}{1.526822in}}{\pgfqpoint{1.353452in}{1.526822in}}%
\pgfpathclose%
\pgfusepath{stroke,fill}%
\end{pgfscope}%
\begin{pgfscope}%
\pgfpathrectangle{\pgfqpoint{0.100000in}{0.212622in}}{\pgfqpoint{3.696000in}{3.696000in}}%
\pgfusepath{clip}%
\pgfsetbuttcap%
\pgfsetroundjoin%
\definecolor{currentfill}{rgb}{0.121569,0.466667,0.705882}%
\pgfsetfillcolor{currentfill}%
\pgfsetfillopacity{0.519822}%
\pgfsetlinewidth{1.003750pt}%
\definecolor{currentstroke}{rgb}{0.121569,0.466667,0.705882}%
\pgfsetstrokecolor{currentstroke}%
\pgfsetstrokeopacity{0.519822}%
\pgfsetdash{}{0pt}%
\pgfpathmoveto{\pgfqpoint{1.384000in}{1.519049in}}%
\pgfpathcurveto{\pgfqpoint{1.392237in}{1.519049in}}{\pgfqpoint{1.400137in}{1.522321in}}{\pgfqpoint{1.405960in}{1.528145in}}%
\pgfpathcurveto{\pgfqpoint{1.411784in}{1.533969in}}{\pgfqpoint{1.415057in}{1.541869in}}{\pgfqpoint{1.415057in}{1.550105in}}%
\pgfpathcurveto{\pgfqpoint{1.415057in}{1.558342in}}{\pgfqpoint{1.411784in}{1.566242in}}{\pgfqpoint{1.405960in}{1.572066in}}%
\pgfpathcurveto{\pgfqpoint{1.400137in}{1.577889in}}{\pgfqpoint{1.392237in}{1.581162in}}{\pgfqpoint{1.384000in}{1.581162in}}%
\pgfpathcurveto{\pgfqpoint{1.375764in}{1.581162in}}{\pgfqpoint{1.367864in}{1.577889in}}{\pgfqpoint{1.362040in}{1.572066in}}%
\pgfpathcurveto{\pgfqpoint{1.356216in}{1.566242in}}{\pgfqpoint{1.352944in}{1.558342in}}{\pgfqpoint{1.352944in}{1.550105in}}%
\pgfpathcurveto{\pgfqpoint{1.352944in}{1.541869in}}{\pgfqpoint{1.356216in}{1.533969in}}{\pgfqpoint{1.362040in}{1.528145in}}%
\pgfpathcurveto{\pgfqpoint{1.367864in}{1.522321in}}{\pgfqpoint{1.375764in}{1.519049in}}{\pgfqpoint{1.384000in}{1.519049in}}%
\pgfpathclose%
\pgfusepath{stroke,fill}%
\end{pgfscope}%
\begin{pgfscope}%
\pgfpathrectangle{\pgfqpoint{0.100000in}{0.212622in}}{\pgfqpoint{3.696000in}{3.696000in}}%
\pgfusepath{clip}%
\pgfsetbuttcap%
\pgfsetroundjoin%
\definecolor{currentfill}{rgb}{0.121569,0.466667,0.705882}%
\pgfsetfillcolor{currentfill}%
\pgfsetfillopacity{0.531246}%
\pgfsetlinewidth{1.003750pt}%
\definecolor{currentstroke}{rgb}{0.121569,0.466667,0.705882}%
\pgfsetstrokecolor{currentstroke}%
\pgfsetstrokeopacity{0.531246}%
\pgfsetdash{}{0pt}%
\pgfpathmoveto{\pgfqpoint{1.420011in}{1.511384in}}%
\pgfpathcurveto{\pgfqpoint{1.428247in}{1.511384in}}{\pgfqpoint{1.436147in}{1.514656in}}{\pgfqpoint{1.441971in}{1.520480in}}%
\pgfpathcurveto{\pgfqpoint{1.447795in}{1.526304in}}{\pgfqpoint{1.451067in}{1.534204in}}{\pgfqpoint{1.451067in}{1.542440in}}%
\pgfpathcurveto{\pgfqpoint{1.451067in}{1.550677in}}{\pgfqpoint{1.447795in}{1.558577in}}{\pgfqpoint{1.441971in}{1.564401in}}%
\pgfpathcurveto{\pgfqpoint{1.436147in}{1.570224in}}{\pgfqpoint{1.428247in}{1.573497in}}{\pgfqpoint{1.420011in}{1.573497in}}%
\pgfpathcurveto{\pgfqpoint{1.411774in}{1.573497in}}{\pgfqpoint{1.403874in}{1.570224in}}{\pgfqpoint{1.398050in}{1.564401in}}%
\pgfpathcurveto{\pgfqpoint{1.392226in}{1.558577in}}{\pgfqpoint{1.388954in}{1.550677in}}{\pgfqpoint{1.388954in}{1.542440in}}%
\pgfpathcurveto{\pgfqpoint{1.388954in}{1.534204in}}{\pgfqpoint{1.392226in}{1.526304in}}{\pgfqpoint{1.398050in}{1.520480in}}%
\pgfpathcurveto{\pgfqpoint{1.403874in}{1.514656in}}{\pgfqpoint{1.411774in}{1.511384in}}{\pgfqpoint{1.420011in}{1.511384in}}%
\pgfpathclose%
\pgfusepath{stroke,fill}%
\end{pgfscope}%
\begin{pgfscope}%
\pgfpathrectangle{\pgfqpoint{0.100000in}{0.212622in}}{\pgfqpoint{3.696000in}{3.696000in}}%
\pgfusepath{clip}%
\pgfsetbuttcap%
\pgfsetroundjoin%
\definecolor{currentfill}{rgb}{0.121569,0.466667,0.705882}%
\pgfsetfillcolor{currentfill}%
\pgfsetfillopacity{0.546565}%
\pgfsetlinewidth{1.003750pt}%
\definecolor{currentstroke}{rgb}{0.121569,0.466667,0.705882}%
\pgfsetstrokecolor{currentstroke}%
\pgfsetstrokeopacity{0.546565}%
\pgfsetdash{}{0pt}%
\pgfpathmoveto{\pgfqpoint{1.462528in}{1.498712in}}%
\pgfpathcurveto{\pgfqpoint{1.470764in}{1.498712in}}{\pgfqpoint{1.478664in}{1.501984in}}{\pgfqpoint{1.484488in}{1.507808in}}%
\pgfpathcurveto{\pgfqpoint{1.490312in}{1.513632in}}{\pgfqpoint{1.493584in}{1.521532in}}{\pgfqpoint{1.493584in}{1.529768in}}%
\pgfpathcurveto{\pgfqpoint{1.493584in}{1.538005in}}{\pgfqpoint{1.490312in}{1.545905in}}{\pgfqpoint{1.484488in}{1.551729in}}%
\pgfpathcurveto{\pgfqpoint{1.478664in}{1.557552in}}{\pgfqpoint{1.470764in}{1.560825in}}{\pgfqpoint{1.462528in}{1.560825in}}%
\pgfpathcurveto{\pgfqpoint{1.454291in}{1.560825in}}{\pgfqpoint{1.446391in}{1.557552in}}{\pgfqpoint{1.440567in}{1.551729in}}%
\pgfpathcurveto{\pgfqpoint{1.434743in}{1.545905in}}{\pgfqpoint{1.431471in}{1.538005in}}{\pgfqpoint{1.431471in}{1.529768in}}%
\pgfpathcurveto{\pgfqpoint{1.431471in}{1.521532in}}{\pgfqpoint{1.434743in}{1.513632in}}{\pgfqpoint{1.440567in}{1.507808in}}%
\pgfpathcurveto{\pgfqpoint{1.446391in}{1.501984in}}{\pgfqpoint{1.454291in}{1.498712in}}{\pgfqpoint{1.462528in}{1.498712in}}%
\pgfpathclose%
\pgfusepath{stroke,fill}%
\end{pgfscope}%
\begin{pgfscope}%
\pgfpathrectangle{\pgfqpoint{0.100000in}{0.212622in}}{\pgfqpoint{3.696000in}{3.696000in}}%
\pgfusepath{clip}%
\pgfsetbuttcap%
\pgfsetroundjoin%
\definecolor{currentfill}{rgb}{0.121569,0.466667,0.705882}%
\pgfsetfillcolor{currentfill}%
\pgfsetfillopacity{0.563162}%
\pgfsetlinewidth{1.003750pt}%
\definecolor{currentstroke}{rgb}{0.121569,0.466667,0.705882}%
\pgfsetstrokecolor{currentstroke}%
\pgfsetstrokeopacity{0.563162}%
\pgfsetdash{}{0pt}%
\pgfpathmoveto{\pgfqpoint{1.513606in}{1.485062in}}%
\pgfpathcurveto{\pgfqpoint{1.521843in}{1.485062in}}{\pgfqpoint{1.529743in}{1.488334in}}{\pgfqpoint{1.535567in}{1.494158in}}%
\pgfpathcurveto{\pgfqpoint{1.541391in}{1.499982in}}{\pgfqpoint{1.544663in}{1.507882in}}{\pgfqpoint{1.544663in}{1.516119in}}%
\pgfpathcurveto{\pgfqpoint{1.544663in}{1.524355in}}{\pgfqpoint{1.541391in}{1.532255in}}{\pgfqpoint{1.535567in}{1.538079in}}%
\pgfpathcurveto{\pgfqpoint{1.529743in}{1.543903in}}{\pgfqpoint{1.521843in}{1.547175in}}{\pgfqpoint{1.513606in}{1.547175in}}%
\pgfpathcurveto{\pgfqpoint{1.505370in}{1.547175in}}{\pgfqpoint{1.497470in}{1.543903in}}{\pgfqpoint{1.491646in}{1.538079in}}%
\pgfpathcurveto{\pgfqpoint{1.485822in}{1.532255in}}{\pgfqpoint{1.482550in}{1.524355in}}{\pgfqpoint{1.482550in}{1.516119in}}%
\pgfpathcurveto{\pgfqpoint{1.482550in}{1.507882in}}{\pgfqpoint{1.485822in}{1.499982in}}{\pgfqpoint{1.491646in}{1.494158in}}%
\pgfpathcurveto{\pgfqpoint{1.497470in}{1.488334in}}{\pgfqpoint{1.505370in}{1.485062in}}{\pgfqpoint{1.513606in}{1.485062in}}%
\pgfpathclose%
\pgfusepath{stroke,fill}%
\end{pgfscope}%
\begin{pgfscope}%
\pgfpathrectangle{\pgfqpoint{0.100000in}{0.212622in}}{\pgfqpoint{3.696000in}{3.696000in}}%
\pgfusepath{clip}%
\pgfsetbuttcap%
\pgfsetroundjoin%
\definecolor{currentfill}{rgb}{0.121569,0.466667,0.705882}%
\pgfsetfillcolor{currentfill}%
\pgfsetfillopacity{0.579823}%
\pgfsetlinewidth{1.003750pt}%
\definecolor{currentstroke}{rgb}{0.121569,0.466667,0.705882}%
\pgfsetstrokecolor{currentstroke}%
\pgfsetstrokeopacity{0.579823}%
\pgfsetdash{}{0pt}%
\pgfpathmoveto{\pgfqpoint{1.570901in}{1.478429in}}%
\pgfpathcurveto{\pgfqpoint{1.579137in}{1.478429in}}{\pgfqpoint{1.587037in}{1.481702in}}{\pgfqpoint{1.592861in}{1.487525in}}%
\pgfpathcurveto{\pgfqpoint{1.598685in}{1.493349in}}{\pgfqpoint{1.601957in}{1.501249in}}{\pgfqpoint{1.601957in}{1.509486in}}%
\pgfpathcurveto{\pgfqpoint{1.601957in}{1.517722in}}{\pgfqpoint{1.598685in}{1.525622in}}{\pgfqpoint{1.592861in}{1.531446in}}%
\pgfpathcurveto{\pgfqpoint{1.587037in}{1.537270in}}{\pgfqpoint{1.579137in}{1.540542in}}{\pgfqpoint{1.570901in}{1.540542in}}%
\pgfpathcurveto{\pgfqpoint{1.562664in}{1.540542in}}{\pgfqpoint{1.554764in}{1.537270in}}{\pgfqpoint{1.548940in}{1.531446in}}%
\pgfpathcurveto{\pgfqpoint{1.543116in}{1.525622in}}{\pgfqpoint{1.539844in}{1.517722in}}{\pgfqpoint{1.539844in}{1.509486in}}%
\pgfpathcurveto{\pgfqpoint{1.539844in}{1.501249in}}{\pgfqpoint{1.543116in}{1.493349in}}{\pgfqpoint{1.548940in}{1.487525in}}%
\pgfpathcurveto{\pgfqpoint{1.554764in}{1.481702in}}{\pgfqpoint{1.562664in}{1.478429in}}{\pgfqpoint{1.570901in}{1.478429in}}%
\pgfpathclose%
\pgfusepath{stroke,fill}%
\end{pgfscope}%
\begin{pgfscope}%
\pgfpathrectangle{\pgfqpoint{0.100000in}{0.212622in}}{\pgfqpoint{3.696000in}{3.696000in}}%
\pgfusepath{clip}%
\pgfsetbuttcap%
\pgfsetroundjoin%
\definecolor{currentfill}{rgb}{0.121569,0.466667,0.705882}%
\pgfsetfillcolor{currentfill}%
\pgfsetfillopacity{0.597423}%
\pgfsetlinewidth{1.003750pt}%
\definecolor{currentstroke}{rgb}{0.121569,0.466667,0.705882}%
\pgfsetstrokecolor{currentstroke}%
\pgfsetstrokeopacity{0.597423}%
\pgfsetdash{}{0pt}%
\pgfpathmoveto{\pgfqpoint{1.629615in}{1.467473in}}%
\pgfpathcurveto{\pgfqpoint{1.637852in}{1.467473in}}{\pgfqpoint{1.645752in}{1.470745in}}{\pgfqpoint{1.651576in}{1.476569in}}%
\pgfpathcurveto{\pgfqpoint{1.657400in}{1.482393in}}{\pgfqpoint{1.660672in}{1.490293in}}{\pgfqpoint{1.660672in}{1.498530in}}%
\pgfpathcurveto{\pgfqpoint{1.660672in}{1.506766in}}{\pgfqpoint{1.657400in}{1.514666in}}{\pgfqpoint{1.651576in}{1.520490in}}%
\pgfpathcurveto{\pgfqpoint{1.645752in}{1.526314in}}{\pgfqpoint{1.637852in}{1.529586in}}{\pgfqpoint{1.629615in}{1.529586in}}%
\pgfpathcurveto{\pgfqpoint{1.621379in}{1.529586in}}{\pgfqpoint{1.613479in}{1.526314in}}{\pgfqpoint{1.607655in}{1.520490in}}%
\pgfpathcurveto{\pgfqpoint{1.601831in}{1.514666in}}{\pgfqpoint{1.598559in}{1.506766in}}{\pgfqpoint{1.598559in}{1.498530in}}%
\pgfpathcurveto{\pgfqpoint{1.598559in}{1.490293in}}{\pgfqpoint{1.601831in}{1.482393in}}{\pgfqpoint{1.607655in}{1.476569in}}%
\pgfpathcurveto{\pgfqpoint{1.613479in}{1.470745in}}{\pgfqpoint{1.621379in}{1.467473in}}{\pgfqpoint{1.629615in}{1.467473in}}%
\pgfpathclose%
\pgfusepath{stroke,fill}%
\end{pgfscope}%
\begin{pgfscope}%
\pgfpathrectangle{\pgfqpoint{0.100000in}{0.212622in}}{\pgfqpoint{3.696000in}{3.696000in}}%
\pgfusepath{clip}%
\pgfsetbuttcap%
\pgfsetroundjoin%
\definecolor{currentfill}{rgb}{0.121569,0.466667,0.705882}%
\pgfsetfillcolor{currentfill}%
\pgfsetfillopacity{0.607582}%
\pgfsetlinewidth{1.003750pt}%
\definecolor{currentstroke}{rgb}{0.121569,0.466667,0.705882}%
\pgfsetstrokecolor{currentstroke}%
\pgfsetstrokeopacity{0.607582}%
\pgfsetdash{}{0pt}%
\pgfpathmoveto{\pgfqpoint{1.661247in}{1.459061in}}%
\pgfpathcurveto{\pgfqpoint{1.669483in}{1.459061in}}{\pgfqpoint{1.677383in}{1.462333in}}{\pgfqpoint{1.683207in}{1.468157in}}%
\pgfpathcurveto{\pgfqpoint{1.689031in}{1.473981in}}{\pgfqpoint{1.692303in}{1.481881in}}{\pgfqpoint{1.692303in}{1.490117in}}%
\pgfpathcurveto{\pgfqpoint{1.692303in}{1.498354in}}{\pgfqpoint{1.689031in}{1.506254in}}{\pgfqpoint{1.683207in}{1.512078in}}%
\pgfpathcurveto{\pgfqpoint{1.677383in}{1.517902in}}{\pgfqpoint{1.669483in}{1.521174in}}{\pgfqpoint{1.661247in}{1.521174in}}%
\pgfpathcurveto{\pgfqpoint{1.653010in}{1.521174in}}{\pgfqpoint{1.645110in}{1.517902in}}{\pgfqpoint{1.639286in}{1.512078in}}%
\pgfpathcurveto{\pgfqpoint{1.633463in}{1.506254in}}{\pgfqpoint{1.630190in}{1.498354in}}{\pgfqpoint{1.630190in}{1.490117in}}%
\pgfpathcurveto{\pgfqpoint{1.630190in}{1.481881in}}{\pgfqpoint{1.633463in}{1.473981in}}{\pgfqpoint{1.639286in}{1.468157in}}%
\pgfpathcurveto{\pgfqpoint{1.645110in}{1.462333in}}{\pgfqpoint{1.653010in}{1.459061in}}{\pgfqpoint{1.661247in}{1.459061in}}%
\pgfpathclose%
\pgfusepath{stroke,fill}%
\end{pgfscope}%
\begin{pgfscope}%
\pgfpathrectangle{\pgfqpoint{0.100000in}{0.212622in}}{\pgfqpoint{3.696000in}{3.696000in}}%
\pgfusepath{clip}%
\pgfsetbuttcap%
\pgfsetroundjoin%
\definecolor{currentfill}{rgb}{0.121569,0.466667,0.705882}%
\pgfsetfillcolor{currentfill}%
\pgfsetfillopacity{0.619027}%
\pgfsetlinewidth{1.003750pt}%
\definecolor{currentstroke}{rgb}{0.121569,0.466667,0.705882}%
\pgfsetstrokecolor{currentstroke}%
\pgfsetstrokeopacity{0.619027}%
\pgfsetdash{}{0pt}%
\pgfpathmoveto{\pgfqpoint{1.696006in}{1.449049in}}%
\pgfpathcurveto{\pgfqpoint{1.704243in}{1.449049in}}{\pgfqpoint{1.712143in}{1.452321in}}{\pgfqpoint{1.717967in}{1.458145in}}%
\pgfpathcurveto{\pgfqpoint{1.723791in}{1.463969in}}{\pgfqpoint{1.727063in}{1.471869in}}{\pgfqpoint{1.727063in}{1.480106in}}%
\pgfpathcurveto{\pgfqpoint{1.727063in}{1.488342in}}{\pgfqpoint{1.723791in}{1.496242in}}{\pgfqpoint{1.717967in}{1.502066in}}%
\pgfpathcurveto{\pgfqpoint{1.712143in}{1.507890in}}{\pgfqpoint{1.704243in}{1.511162in}}{\pgfqpoint{1.696006in}{1.511162in}}%
\pgfpathcurveto{\pgfqpoint{1.687770in}{1.511162in}}{\pgfqpoint{1.679870in}{1.507890in}}{\pgfqpoint{1.674046in}{1.502066in}}%
\pgfpathcurveto{\pgfqpoint{1.668222in}{1.496242in}}{\pgfqpoint{1.664950in}{1.488342in}}{\pgfqpoint{1.664950in}{1.480106in}}%
\pgfpathcurveto{\pgfqpoint{1.664950in}{1.471869in}}{\pgfqpoint{1.668222in}{1.463969in}}{\pgfqpoint{1.674046in}{1.458145in}}%
\pgfpathcurveto{\pgfqpoint{1.679870in}{1.452321in}}{\pgfqpoint{1.687770in}{1.449049in}}{\pgfqpoint{1.696006in}{1.449049in}}%
\pgfpathclose%
\pgfusepath{stroke,fill}%
\end{pgfscope}%
\begin{pgfscope}%
\pgfpathrectangle{\pgfqpoint{0.100000in}{0.212622in}}{\pgfqpoint{3.696000in}{3.696000in}}%
\pgfusepath{clip}%
\pgfsetbuttcap%
\pgfsetroundjoin%
\definecolor{currentfill}{rgb}{0.121569,0.466667,0.705882}%
\pgfsetfillcolor{currentfill}%
\pgfsetfillopacity{0.625131}%
\pgfsetlinewidth{1.003750pt}%
\definecolor{currentstroke}{rgb}{0.121569,0.466667,0.705882}%
\pgfsetstrokecolor{currentstroke}%
\pgfsetstrokeopacity{0.625131}%
\pgfsetdash{}{0pt}%
\pgfpathmoveto{\pgfqpoint{1.715605in}{1.444933in}}%
\pgfpathcurveto{\pgfqpoint{1.723842in}{1.444933in}}{\pgfqpoint{1.731742in}{1.448206in}}{\pgfqpoint{1.737566in}{1.454030in}}%
\pgfpathcurveto{\pgfqpoint{1.743390in}{1.459854in}}{\pgfqpoint{1.746662in}{1.467754in}}{\pgfqpoint{1.746662in}{1.475990in}}%
\pgfpathcurveto{\pgfqpoint{1.746662in}{1.484226in}}{\pgfqpoint{1.743390in}{1.492126in}}{\pgfqpoint{1.737566in}{1.497950in}}%
\pgfpathcurveto{\pgfqpoint{1.731742in}{1.503774in}}{\pgfqpoint{1.723842in}{1.507046in}}{\pgfqpoint{1.715605in}{1.507046in}}%
\pgfpathcurveto{\pgfqpoint{1.707369in}{1.507046in}}{\pgfqpoint{1.699469in}{1.503774in}}{\pgfqpoint{1.693645in}{1.497950in}}%
\pgfpathcurveto{\pgfqpoint{1.687821in}{1.492126in}}{\pgfqpoint{1.684549in}{1.484226in}}{\pgfqpoint{1.684549in}{1.475990in}}%
\pgfpathcurveto{\pgfqpoint{1.684549in}{1.467754in}}{\pgfqpoint{1.687821in}{1.459854in}}{\pgfqpoint{1.693645in}{1.454030in}}%
\pgfpathcurveto{\pgfqpoint{1.699469in}{1.448206in}}{\pgfqpoint{1.707369in}{1.444933in}}{\pgfqpoint{1.715605in}{1.444933in}}%
\pgfpathclose%
\pgfusepath{stroke,fill}%
\end{pgfscope}%
\begin{pgfscope}%
\pgfpathrectangle{\pgfqpoint{0.100000in}{0.212622in}}{\pgfqpoint{3.696000in}{3.696000in}}%
\pgfusepath{clip}%
\pgfsetbuttcap%
\pgfsetroundjoin%
\definecolor{currentfill}{rgb}{0.121569,0.466667,0.705882}%
\pgfsetfillcolor{currentfill}%
\pgfsetfillopacity{0.632392}%
\pgfsetlinewidth{1.003750pt}%
\definecolor{currentstroke}{rgb}{0.121569,0.466667,0.705882}%
\pgfsetstrokecolor{currentstroke}%
\pgfsetstrokeopacity{0.632392}%
\pgfsetdash{}{0pt}%
\pgfpathmoveto{\pgfqpoint{1.739179in}{1.439862in}}%
\pgfpathcurveto{\pgfqpoint{1.747415in}{1.439862in}}{\pgfqpoint{1.755315in}{1.443134in}}{\pgfqpoint{1.761139in}{1.448958in}}%
\pgfpathcurveto{\pgfqpoint{1.766963in}{1.454782in}}{\pgfqpoint{1.770235in}{1.462682in}}{\pgfqpoint{1.770235in}{1.470919in}}%
\pgfpathcurveto{\pgfqpoint{1.770235in}{1.479155in}}{\pgfqpoint{1.766963in}{1.487055in}}{\pgfqpoint{1.761139in}{1.492879in}}%
\pgfpathcurveto{\pgfqpoint{1.755315in}{1.498703in}}{\pgfqpoint{1.747415in}{1.501975in}}{\pgfqpoint{1.739179in}{1.501975in}}%
\pgfpathcurveto{\pgfqpoint{1.730942in}{1.501975in}}{\pgfqpoint{1.723042in}{1.498703in}}{\pgfqpoint{1.717218in}{1.492879in}}%
\pgfpathcurveto{\pgfqpoint{1.711394in}{1.487055in}}{\pgfqpoint{1.708122in}{1.479155in}}{\pgfqpoint{1.708122in}{1.470919in}}%
\pgfpathcurveto{\pgfqpoint{1.708122in}{1.462682in}}{\pgfqpoint{1.711394in}{1.454782in}}{\pgfqpoint{1.717218in}{1.448958in}}%
\pgfpathcurveto{\pgfqpoint{1.723042in}{1.443134in}}{\pgfqpoint{1.730942in}{1.439862in}}{\pgfqpoint{1.739179in}{1.439862in}}%
\pgfpathclose%
\pgfusepath{stroke,fill}%
\end{pgfscope}%
\begin{pgfscope}%
\pgfpathrectangle{\pgfqpoint{0.100000in}{0.212622in}}{\pgfqpoint{3.696000in}{3.696000in}}%
\pgfusepath{clip}%
\pgfsetbuttcap%
\pgfsetroundjoin%
\definecolor{currentfill}{rgb}{0.121569,0.466667,0.705882}%
\pgfsetfillcolor{currentfill}%
\pgfsetfillopacity{0.636399}%
\pgfsetlinewidth{1.003750pt}%
\definecolor{currentstroke}{rgb}{0.121569,0.466667,0.705882}%
\pgfsetstrokecolor{currentstroke}%
\pgfsetstrokeopacity{0.636399}%
\pgfsetdash{}{0pt}%
\pgfpathmoveto{\pgfqpoint{1.752082in}{1.436804in}}%
\pgfpathcurveto{\pgfqpoint{1.760319in}{1.436804in}}{\pgfqpoint{1.768219in}{1.440076in}}{\pgfqpoint{1.774043in}{1.445900in}}%
\pgfpathcurveto{\pgfqpoint{1.779867in}{1.451724in}}{\pgfqpoint{1.783139in}{1.459624in}}{\pgfqpoint{1.783139in}{1.467860in}}%
\pgfpathcurveto{\pgfqpoint{1.783139in}{1.476096in}}{\pgfqpoint{1.779867in}{1.483996in}}{\pgfqpoint{1.774043in}{1.489820in}}%
\pgfpathcurveto{\pgfqpoint{1.768219in}{1.495644in}}{\pgfqpoint{1.760319in}{1.498917in}}{\pgfqpoint{1.752082in}{1.498917in}}%
\pgfpathcurveto{\pgfqpoint{1.743846in}{1.498917in}}{\pgfqpoint{1.735946in}{1.495644in}}{\pgfqpoint{1.730122in}{1.489820in}}%
\pgfpathcurveto{\pgfqpoint{1.724298in}{1.483996in}}{\pgfqpoint{1.721026in}{1.476096in}}{\pgfqpoint{1.721026in}{1.467860in}}%
\pgfpathcurveto{\pgfqpoint{1.721026in}{1.459624in}}{\pgfqpoint{1.724298in}{1.451724in}}{\pgfqpoint{1.730122in}{1.445900in}}%
\pgfpathcurveto{\pgfqpoint{1.735946in}{1.440076in}}{\pgfqpoint{1.743846in}{1.436804in}}{\pgfqpoint{1.752082in}{1.436804in}}%
\pgfpathclose%
\pgfusepath{stroke,fill}%
\end{pgfscope}%
\begin{pgfscope}%
\pgfpathrectangle{\pgfqpoint{0.100000in}{0.212622in}}{\pgfqpoint{3.696000in}{3.696000in}}%
\pgfusepath{clip}%
\pgfsetbuttcap%
\pgfsetroundjoin%
\definecolor{currentfill}{rgb}{0.121569,0.466667,0.705882}%
\pgfsetfillcolor{currentfill}%
\pgfsetfillopacity{0.641641}%
\pgfsetlinewidth{1.003750pt}%
\definecolor{currentstroke}{rgb}{0.121569,0.466667,0.705882}%
\pgfsetstrokecolor{currentstroke}%
\pgfsetstrokeopacity{0.641641}%
\pgfsetdash{}{0pt}%
\pgfpathmoveto{\pgfqpoint{1.771270in}{1.435210in}}%
\pgfpathcurveto{\pgfqpoint{1.779506in}{1.435210in}}{\pgfqpoint{1.787406in}{1.438482in}}{\pgfqpoint{1.793230in}{1.444306in}}%
\pgfpathcurveto{\pgfqpoint{1.799054in}{1.450130in}}{\pgfqpoint{1.802327in}{1.458030in}}{\pgfqpoint{1.802327in}{1.466266in}}%
\pgfpathcurveto{\pgfqpoint{1.802327in}{1.474502in}}{\pgfqpoint{1.799054in}{1.482403in}}{\pgfqpoint{1.793230in}{1.488226in}}%
\pgfpathcurveto{\pgfqpoint{1.787406in}{1.494050in}}{\pgfqpoint{1.779506in}{1.497323in}}{\pgfqpoint{1.771270in}{1.497323in}}%
\pgfpathcurveto{\pgfqpoint{1.763034in}{1.497323in}}{\pgfqpoint{1.755134in}{1.494050in}}{\pgfqpoint{1.749310in}{1.488226in}}%
\pgfpathcurveto{\pgfqpoint{1.743486in}{1.482403in}}{\pgfqpoint{1.740214in}{1.474502in}}{\pgfqpoint{1.740214in}{1.466266in}}%
\pgfpathcurveto{\pgfqpoint{1.740214in}{1.458030in}}{\pgfqpoint{1.743486in}{1.450130in}}{\pgfqpoint{1.749310in}{1.444306in}}%
\pgfpathcurveto{\pgfqpoint{1.755134in}{1.438482in}}{\pgfqpoint{1.763034in}{1.435210in}}{\pgfqpoint{1.771270in}{1.435210in}}%
\pgfpathclose%
\pgfusepath{stroke,fill}%
\end{pgfscope}%
\begin{pgfscope}%
\pgfpathrectangle{\pgfqpoint{0.100000in}{0.212622in}}{\pgfqpoint{3.696000in}{3.696000in}}%
\pgfusepath{clip}%
\pgfsetbuttcap%
\pgfsetroundjoin%
\definecolor{currentfill}{rgb}{0.121569,0.466667,0.705882}%
\pgfsetfillcolor{currentfill}%
\pgfsetfillopacity{0.649282}%
\pgfsetlinewidth{1.003750pt}%
\definecolor{currentstroke}{rgb}{0.121569,0.466667,0.705882}%
\pgfsetstrokecolor{currentstroke}%
\pgfsetstrokeopacity{0.649282}%
\pgfsetdash{}{0pt}%
\pgfpathmoveto{\pgfqpoint{1.795502in}{1.429129in}}%
\pgfpathcurveto{\pgfqpoint{1.803738in}{1.429129in}}{\pgfqpoint{1.811638in}{1.432402in}}{\pgfqpoint{1.817462in}{1.438226in}}%
\pgfpathcurveto{\pgfqpoint{1.823286in}{1.444049in}}{\pgfqpoint{1.826559in}{1.451949in}}{\pgfqpoint{1.826559in}{1.460186in}}%
\pgfpathcurveto{\pgfqpoint{1.826559in}{1.468422in}}{\pgfqpoint{1.823286in}{1.476322in}}{\pgfqpoint{1.817462in}{1.482146in}}%
\pgfpathcurveto{\pgfqpoint{1.811638in}{1.487970in}}{\pgfqpoint{1.803738in}{1.491242in}}{\pgfqpoint{1.795502in}{1.491242in}}%
\pgfpathcurveto{\pgfqpoint{1.787266in}{1.491242in}}{\pgfqpoint{1.779366in}{1.487970in}}{\pgfqpoint{1.773542in}{1.482146in}}%
\pgfpathcurveto{\pgfqpoint{1.767718in}{1.476322in}}{\pgfqpoint{1.764446in}{1.468422in}}{\pgfqpoint{1.764446in}{1.460186in}}%
\pgfpathcurveto{\pgfqpoint{1.764446in}{1.451949in}}{\pgfqpoint{1.767718in}{1.444049in}}{\pgfqpoint{1.773542in}{1.438226in}}%
\pgfpathcurveto{\pgfqpoint{1.779366in}{1.432402in}}{\pgfqpoint{1.787266in}{1.429129in}}{\pgfqpoint{1.795502in}{1.429129in}}%
\pgfpathclose%
\pgfusepath{stroke,fill}%
\end{pgfscope}%
\begin{pgfscope}%
\pgfpathrectangle{\pgfqpoint{0.100000in}{0.212622in}}{\pgfqpoint{3.696000in}{3.696000in}}%
\pgfusepath{clip}%
\pgfsetbuttcap%
\pgfsetroundjoin%
\definecolor{currentfill}{rgb}{0.121569,0.466667,0.705882}%
\pgfsetfillcolor{currentfill}%
\pgfsetfillopacity{0.658836}%
\pgfsetlinewidth{1.003750pt}%
\definecolor{currentstroke}{rgb}{0.121569,0.466667,0.705882}%
\pgfsetstrokecolor{currentstroke}%
\pgfsetstrokeopacity{0.658836}%
\pgfsetdash{}{0pt}%
\pgfpathmoveto{\pgfqpoint{1.829277in}{1.425563in}}%
\pgfpathcurveto{\pgfqpoint{1.837513in}{1.425563in}}{\pgfqpoint{1.845413in}{1.428835in}}{\pgfqpoint{1.851237in}{1.434659in}}%
\pgfpathcurveto{\pgfqpoint{1.857061in}{1.440483in}}{\pgfqpoint{1.860333in}{1.448383in}}{\pgfqpoint{1.860333in}{1.456620in}}%
\pgfpathcurveto{\pgfqpoint{1.860333in}{1.464856in}}{\pgfqpoint{1.857061in}{1.472756in}}{\pgfqpoint{1.851237in}{1.478580in}}%
\pgfpathcurveto{\pgfqpoint{1.845413in}{1.484404in}}{\pgfqpoint{1.837513in}{1.487676in}}{\pgfqpoint{1.829277in}{1.487676in}}%
\pgfpathcurveto{\pgfqpoint{1.821041in}{1.487676in}}{\pgfqpoint{1.813140in}{1.484404in}}{\pgfqpoint{1.807317in}{1.478580in}}%
\pgfpathcurveto{\pgfqpoint{1.801493in}{1.472756in}}{\pgfqpoint{1.798220in}{1.464856in}}{\pgfqpoint{1.798220in}{1.456620in}}%
\pgfpathcurveto{\pgfqpoint{1.798220in}{1.448383in}}{\pgfqpoint{1.801493in}{1.440483in}}{\pgfqpoint{1.807317in}{1.434659in}}%
\pgfpathcurveto{\pgfqpoint{1.813140in}{1.428835in}}{\pgfqpoint{1.821041in}{1.425563in}}{\pgfqpoint{1.829277in}{1.425563in}}%
\pgfpathclose%
\pgfusepath{stroke,fill}%
\end{pgfscope}%
\begin{pgfscope}%
\pgfpathrectangle{\pgfqpoint{0.100000in}{0.212622in}}{\pgfqpoint{3.696000in}{3.696000in}}%
\pgfusepath{clip}%
\pgfsetbuttcap%
\pgfsetroundjoin%
\definecolor{currentfill}{rgb}{0.121569,0.466667,0.705882}%
\pgfsetfillcolor{currentfill}%
\pgfsetfillopacity{0.671042}%
\pgfsetlinewidth{1.003750pt}%
\definecolor{currentstroke}{rgb}{0.121569,0.466667,0.705882}%
\pgfsetstrokecolor{currentstroke}%
\pgfsetstrokeopacity{0.671042}%
\pgfsetdash{}{0pt}%
\pgfpathmoveto{\pgfqpoint{1.867396in}{1.415248in}}%
\pgfpathcurveto{\pgfqpoint{1.875632in}{1.415248in}}{\pgfqpoint{1.883532in}{1.418520in}}{\pgfqpoint{1.889356in}{1.424344in}}%
\pgfpathcurveto{\pgfqpoint{1.895180in}{1.430168in}}{\pgfqpoint{1.898452in}{1.438068in}}{\pgfqpoint{1.898452in}{1.446304in}}%
\pgfpathcurveto{\pgfqpoint{1.898452in}{1.454540in}}{\pgfqpoint{1.895180in}{1.462440in}}{\pgfqpoint{1.889356in}{1.468264in}}%
\pgfpathcurveto{\pgfqpoint{1.883532in}{1.474088in}}{\pgfqpoint{1.875632in}{1.477361in}}{\pgfqpoint{1.867396in}{1.477361in}}%
\pgfpathcurveto{\pgfqpoint{1.859159in}{1.477361in}}{\pgfqpoint{1.851259in}{1.474088in}}{\pgfqpoint{1.845435in}{1.468264in}}%
\pgfpathcurveto{\pgfqpoint{1.839611in}{1.462440in}}{\pgfqpoint{1.836339in}{1.454540in}}{\pgfqpoint{1.836339in}{1.446304in}}%
\pgfpathcurveto{\pgfqpoint{1.836339in}{1.438068in}}{\pgfqpoint{1.839611in}{1.430168in}}{\pgfqpoint{1.845435in}{1.424344in}}%
\pgfpathcurveto{\pgfqpoint{1.851259in}{1.418520in}}{\pgfqpoint{1.859159in}{1.415248in}}{\pgfqpoint{1.867396in}{1.415248in}}%
\pgfpathclose%
\pgfusepath{stroke,fill}%
\end{pgfscope}%
\begin{pgfscope}%
\pgfpathrectangle{\pgfqpoint{0.100000in}{0.212622in}}{\pgfqpoint{3.696000in}{3.696000in}}%
\pgfusepath{clip}%
\pgfsetbuttcap%
\pgfsetroundjoin%
\definecolor{currentfill}{rgb}{0.121569,0.466667,0.705882}%
\pgfsetfillcolor{currentfill}%
\pgfsetfillopacity{0.684687}%
\pgfsetlinewidth{1.003750pt}%
\definecolor{currentstroke}{rgb}{0.121569,0.466667,0.705882}%
\pgfsetstrokecolor{currentstroke}%
\pgfsetstrokeopacity{0.684687}%
\pgfsetdash{}{0pt}%
\pgfpathmoveto{\pgfqpoint{1.910270in}{1.405998in}}%
\pgfpathcurveto{\pgfqpoint{1.918506in}{1.405998in}}{\pgfqpoint{1.926406in}{1.409270in}}{\pgfqpoint{1.932230in}{1.415094in}}%
\pgfpathcurveto{\pgfqpoint{1.938054in}{1.420918in}}{\pgfqpoint{1.941326in}{1.428818in}}{\pgfqpoint{1.941326in}{1.437054in}}%
\pgfpathcurveto{\pgfqpoint{1.941326in}{1.445291in}}{\pgfqpoint{1.938054in}{1.453191in}}{\pgfqpoint{1.932230in}{1.459015in}}%
\pgfpathcurveto{\pgfqpoint{1.926406in}{1.464838in}}{\pgfqpoint{1.918506in}{1.468111in}}{\pgfqpoint{1.910270in}{1.468111in}}%
\pgfpathcurveto{\pgfqpoint{1.902033in}{1.468111in}}{\pgfqpoint{1.894133in}{1.464838in}}{\pgfqpoint{1.888309in}{1.459015in}}%
\pgfpathcurveto{\pgfqpoint{1.882486in}{1.453191in}}{\pgfqpoint{1.879213in}{1.445291in}}{\pgfqpoint{1.879213in}{1.437054in}}%
\pgfpathcurveto{\pgfqpoint{1.879213in}{1.428818in}}{\pgfqpoint{1.882486in}{1.420918in}}{\pgfqpoint{1.888309in}{1.415094in}}%
\pgfpathcurveto{\pgfqpoint{1.894133in}{1.409270in}}{\pgfqpoint{1.902033in}{1.405998in}}{\pgfqpoint{1.910270in}{1.405998in}}%
\pgfpathclose%
\pgfusepath{stroke,fill}%
\end{pgfscope}%
\begin{pgfscope}%
\pgfpathrectangle{\pgfqpoint{0.100000in}{0.212622in}}{\pgfqpoint{3.696000in}{3.696000in}}%
\pgfusepath{clip}%
\pgfsetbuttcap%
\pgfsetroundjoin%
\definecolor{currentfill}{rgb}{0.121569,0.466667,0.705882}%
\pgfsetfillcolor{currentfill}%
\pgfsetfillopacity{0.699791}%
\pgfsetlinewidth{1.003750pt}%
\definecolor{currentstroke}{rgb}{0.121569,0.466667,0.705882}%
\pgfsetstrokecolor{currentstroke}%
\pgfsetstrokeopacity{0.699791}%
\pgfsetdash{}{0pt}%
\pgfpathmoveto{\pgfqpoint{1.956893in}{1.393131in}}%
\pgfpathcurveto{\pgfqpoint{1.965129in}{1.393131in}}{\pgfqpoint{1.973029in}{1.396404in}}{\pgfqpoint{1.978853in}{1.402228in}}%
\pgfpathcurveto{\pgfqpoint{1.984677in}{1.408052in}}{\pgfqpoint{1.987949in}{1.415952in}}{\pgfqpoint{1.987949in}{1.424188in}}%
\pgfpathcurveto{\pgfqpoint{1.987949in}{1.432424in}}{\pgfqpoint{1.984677in}{1.440324in}}{\pgfqpoint{1.978853in}{1.446148in}}%
\pgfpathcurveto{\pgfqpoint{1.973029in}{1.451972in}}{\pgfqpoint{1.965129in}{1.455244in}}{\pgfqpoint{1.956893in}{1.455244in}}%
\pgfpathcurveto{\pgfqpoint{1.948656in}{1.455244in}}{\pgfqpoint{1.940756in}{1.451972in}}{\pgfqpoint{1.934932in}{1.446148in}}%
\pgfpathcurveto{\pgfqpoint{1.929108in}{1.440324in}}{\pgfqpoint{1.925836in}{1.432424in}}{\pgfqpoint{1.925836in}{1.424188in}}%
\pgfpathcurveto{\pgfqpoint{1.925836in}{1.415952in}}{\pgfqpoint{1.929108in}{1.408052in}}{\pgfqpoint{1.934932in}{1.402228in}}%
\pgfpathcurveto{\pgfqpoint{1.940756in}{1.396404in}}{\pgfqpoint{1.948656in}{1.393131in}}{\pgfqpoint{1.956893in}{1.393131in}}%
\pgfpathclose%
\pgfusepath{stroke,fill}%
\end{pgfscope}%
\begin{pgfscope}%
\pgfpathrectangle{\pgfqpoint{0.100000in}{0.212622in}}{\pgfqpoint{3.696000in}{3.696000in}}%
\pgfusepath{clip}%
\pgfsetbuttcap%
\pgfsetroundjoin%
\definecolor{currentfill}{rgb}{0.121569,0.466667,0.705882}%
\pgfsetfillcolor{currentfill}%
\pgfsetfillopacity{0.716831}%
\pgfsetlinewidth{1.003750pt}%
\definecolor{currentstroke}{rgb}{0.121569,0.466667,0.705882}%
\pgfsetstrokecolor{currentstroke}%
\pgfsetstrokeopacity{0.716831}%
\pgfsetdash{}{0pt}%
\pgfpathmoveto{\pgfqpoint{2.009780in}{1.382123in}}%
\pgfpathcurveto{\pgfqpoint{2.018017in}{1.382123in}}{\pgfqpoint{2.025917in}{1.385395in}}{\pgfqpoint{2.031741in}{1.391219in}}%
\pgfpathcurveto{\pgfqpoint{2.037564in}{1.397043in}}{\pgfqpoint{2.040837in}{1.404943in}}{\pgfqpoint{2.040837in}{1.413180in}}%
\pgfpathcurveto{\pgfqpoint{2.040837in}{1.421416in}}{\pgfqpoint{2.037564in}{1.429316in}}{\pgfqpoint{2.031741in}{1.435140in}}%
\pgfpathcurveto{\pgfqpoint{2.025917in}{1.440964in}}{\pgfqpoint{2.018017in}{1.444236in}}{\pgfqpoint{2.009780in}{1.444236in}}%
\pgfpathcurveto{\pgfqpoint{2.001544in}{1.444236in}}{\pgfqpoint{1.993644in}{1.440964in}}{\pgfqpoint{1.987820in}{1.435140in}}%
\pgfpathcurveto{\pgfqpoint{1.981996in}{1.429316in}}{\pgfqpoint{1.978724in}{1.421416in}}{\pgfqpoint{1.978724in}{1.413180in}}%
\pgfpathcurveto{\pgfqpoint{1.978724in}{1.404943in}}{\pgfqpoint{1.981996in}{1.397043in}}{\pgfqpoint{1.987820in}{1.391219in}}%
\pgfpathcurveto{\pgfqpoint{1.993644in}{1.385395in}}{\pgfqpoint{2.001544in}{1.382123in}}{\pgfqpoint{2.009780in}{1.382123in}}%
\pgfpathclose%
\pgfusepath{stroke,fill}%
\end{pgfscope}%
\begin{pgfscope}%
\pgfpathrectangle{\pgfqpoint{0.100000in}{0.212622in}}{\pgfqpoint{3.696000in}{3.696000in}}%
\pgfusepath{clip}%
\pgfsetbuttcap%
\pgfsetroundjoin%
\definecolor{currentfill}{rgb}{0.121569,0.466667,0.705882}%
\pgfsetfillcolor{currentfill}%
\pgfsetfillopacity{0.726388}%
\pgfsetlinewidth{1.003750pt}%
\definecolor{currentstroke}{rgb}{0.121569,0.466667,0.705882}%
\pgfsetstrokecolor{currentstroke}%
\pgfsetstrokeopacity{0.726388}%
\pgfsetdash{}{0pt}%
\pgfpathmoveto{\pgfqpoint{2.038136in}{1.373868in}}%
\pgfpathcurveto{\pgfqpoint{2.046372in}{1.373868in}}{\pgfqpoint{2.054272in}{1.377141in}}{\pgfqpoint{2.060096in}{1.382965in}}%
\pgfpathcurveto{\pgfqpoint{2.065920in}{1.388789in}}{\pgfqpoint{2.069193in}{1.396689in}}{\pgfqpoint{2.069193in}{1.404925in}}%
\pgfpathcurveto{\pgfqpoint{2.069193in}{1.413161in}}{\pgfqpoint{2.065920in}{1.421061in}}{\pgfqpoint{2.060096in}{1.426885in}}%
\pgfpathcurveto{\pgfqpoint{2.054272in}{1.432709in}}{\pgfqpoint{2.046372in}{1.435981in}}{\pgfqpoint{2.038136in}{1.435981in}}%
\pgfpathcurveto{\pgfqpoint{2.029900in}{1.435981in}}{\pgfqpoint{2.022000in}{1.432709in}}{\pgfqpoint{2.016176in}{1.426885in}}%
\pgfpathcurveto{\pgfqpoint{2.010352in}{1.421061in}}{\pgfqpoint{2.007080in}{1.413161in}}{\pgfqpoint{2.007080in}{1.404925in}}%
\pgfpathcurveto{\pgfqpoint{2.007080in}{1.396689in}}{\pgfqpoint{2.010352in}{1.388789in}}{\pgfqpoint{2.016176in}{1.382965in}}%
\pgfpathcurveto{\pgfqpoint{2.022000in}{1.377141in}}{\pgfqpoint{2.029900in}{1.373868in}}{\pgfqpoint{2.038136in}{1.373868in}}%
\pgfpathclose%
\pgfusepath{stroke,fill}%
\end{pgfscope}%
\begin{pgfscope}%
\pgfpathrectangle{\pgfqpoint{0.100000in}{0.212622in}}{\pgfqpoint{3.696000in}{3.696000in}}%
\pgfusepath{clip}%
\pgfsetbuttcap%
\pgfsetroundjoin%
\definecolor{currentfill}{rgb}{0.121569,0.466667,0.705882}%
\pgfsetfillcolor{currentfill}%
\pgfsetfillopacity{0.731805}%
\pgfsetlinewidth{1.003750pt}%
\definecolor{currentstroke}{rgb}{0.121569,0.466667,0.705882}%
\pgfsetstrokecolor{currentstroke}%
\pgfsetstrokeopacity{0.731805}%
\pgfsetdash{}{0pt}%
\pgfpathmoveto{\pgfqpoint{2.053653in}{1.369299in}}%
\pgfpathcurveto{\pgfqpoint{2.061889in}{1.369299in}}{\pgfqpoint{2.069789in}{1.372571in}}{\pgfqpoint{2.075613in}{1.378395in}}%
\pgfpathcurveto{\pgfqpoint{2.081437in}{1.384219in}}{\pgfqpoint{2.084709in}{1.392119in}}{\pgfqpoint{2.084709in}{1.400355in}}%
\pgfpathcurveto{\pgfqpoint{2.084709in}{1.408591in}}{\pgfqpoint{2.081437in}{1.416491in}}{\pgfqpoint{2.075613in}{1.422315in}}%
\pgfpathcurveto{\pgfqpoint{2.069789in}{1.428139in}}{\pgfqpoint{2.061889in}{1.431412in}}{\pgfqpoint{2.053653in}{1.431412in}}%
\pgfpathcurveto{\pgfqpoint{2.045417in}{1.431412in}}{\pgfqpoint{2.037517in}{1.428139in}}{\pgfqpoint{2.031693in}{1.422315in}}%
\pgfpathcurveto{\pgfqpoint{2.025869in}{1.416491in}}{\pgfqpoint{2.022596in}{1.408591in}}{\pgfqpoint{2.022596in}{1.400355in}}%
\pgfpathcurveto{\pgfqpoint{2.022596in}{1.392119in}}{\pgfqpoint{2.025869in}{1.384219in}}{\pgfqpoint{2.031693in}{1.378395in}}%
\pgfpathcurveto{\pgfqpoint{2.037517in}{1.372571in}}{\pgfqpoint{2.045417in}{1.369299in}}{\pgfqpoint{2.053653in}{1.369299in}}%
\pgfpathclose%
\pgfusepath{stroke,fill}%
\end{pgfscope}%
\begin{pgfscope}%
\pgfpathrectangle{\pgfqpoint{0.100000in}{0.212622in}}{\pgfqpoint{3.696000in}{3.696000in}}%
\pgfusepath{clip}%
\pgfsetbuttcap%
\pgfsetroundjoin%
\definecolor{currentfill}{rgb}{0.121569,0.466667,0.705882}%
\pgfsetfillcolor{currentfill}%
\pgfsetfillopacity{0.734870}%
\pgfsetlinewidth{1.003750pt}%
\definecolor{currentstroke}{rgb}{0.121569,0.466667,0.705882}%
\pgfsetstrokecolor{currentstroke}%
\pgfsetstrokeopacity{0.734870}%
\pgfsetdash{}{0pt}%
\pgfpathmoveto{\pgfqpoint{2.061902in}{1.366088in}}%
\pgfpathcurveto{\pgfqpoint{2.070138in}{1.366088in}}{\pgfqpoint{2.078038in}{1.369360in}}{\pgfqpoint{2.083862in}{1.375184in}}%
\pgfpathcurveto{\pgfqpoint{2.089686in}{1.381008in}}{\pgfqpoint{2.092958in}{1.388908in}}{\pgfqpoint{2.092958in}{1.397144in}}%
\pgfpathcurveto{\pgfqpoint{2.092958in}{1.405380in}}{\pgfqpoint{2.089686in}{1.413280in}}{\pgfqpoint{2.083862in}{1.419104in}}%
\pgfpathcurveto{\pgfqpoint{2.078038in}{1.424928in}}{\pgfqpoint{2.070138in}{1.428201in}}{\pgfqpoint{2.061902in}{1.428201in}}%
\pgfpathcurveto{\pgfqpoint{2.053665in}{1.428201in}}{\pgfqpoint{2.045765in}{1.424928in}}{\pgfqpoint{2.039941in}{1.419104in}}%
\pgfpathcurveto{\pgfqpoint{2.034117in}{1.413280in}}{\pgfqpoint{2.030845in}{1.405380in}}{\pgfqpoint{2.030845in}{1.397144in}}%
\pgfpathcurveto{\pgfqpoint{2.030845in}{1.388908in}}{\pgfqpoint{2.034117in}{1.381008in}}{\pgfqpoint{2.039941in}{1.375184in}}%
\pgfpathcurveto{\pgfqpoint{2.045765in}{1.369360in}}{\pgfqpoint{2.053665in}{1.366088in}}{\pgfqpoint{2.061902in}{1.366088in}}%
\pgfpathclose%
\pgfusepath{stroke,fill}%
\end{pgfscope}%
\begin{pgfscope}%
\pgfpathrectangle{\pgfqpoint{0.100000in}{0.212622in}}{\pgfqpoint{3.696000in}{3.696000in}}%
\pgfusepath{clip}%
\pgfsetbuttcap%
\pgfsetroundjoin%
\definecolor{currentfill}{rgb}{0.121569,0.466667,0.705882}%
\pgfsetfillcolor{currentfill}%
\pgfsetfillopacity{0.736419}%
\pgfsetlinewidth{1.003750pt}%
\definecolor{currentstroke}{rgb}{0.121569,0.466667,0.705882}%
\pgfsetstrokecolor{currentstroke}%
\pgfsetstrokeopacity{0.736419}%
\pgfsetdash{}{0pt}%
\pgfpathmoveto{\pgfqpoint{2.066737in}{1.364986in}}%
\pgfpathcurveto{\pgfqpoint{2.074974in}{1.364986in}}{\pgfqpoint{2.082874in}{1.368258in}}{\pgfqpoint{2.088698in}{1.374082in}}%
\pgfpathcurveto{\pgfqpoint{2.094522in}{1.379906in}}{\pgfqpoint{2.097794in}{1.387806in}}{\pgfqpoint{2.097794in}{1.396042in}}%
\pgfpathcurveto{\pgfqpoint{2.097794in}{1.404278in}}{\pgfqpoint{2.094522in}{1.412178in}}{\pgfqpoint{2.088698in}{1.418002in}}%
\pgfpathcurveto{\pgfqpoint{2.082874in}{1.423826in}}{\pgfqpoint{2.074974in}{1.427099in}}{\pgfqpoint{2.066737in}{1.427099in}}%
\pgfpathcurveto{\pgfqpoint{2.058501in}{1.427099in}}{\pgfqpoint{2.050601in}{1.423826in}}{\pgfqpoint{2.044777in}{1.418002in}}%
\pgfpathcurveto{\pgfqpoint{2.038953in}{1.412178in}}{\pgfqpoint{2.035681in}{1.404278in}}{\pgfqpoint{2.035681in}{1.396042in}}%
\pgfpathcurveto{\pgfqpoint{2.035681in}{1.387806in}}{\pgfqpoint{2.038953in}{1.379906in}}{\pgfqpoint{2.044777in}{1.374082in}}%
\pgfpathcurveto{\pgfqpoint{2.050601in}{1.368258in}}{\pgfqpoint{2.058501in}{1.364986in}}{\pgfqpoint{2.066737in}{1.364986in}}%
\pgfpathclose%
\pgfusepath{stroke,fill}%
\end{pgfscope}%
\begin{pgfscope}%
\pgfpathrectangle{\pgfqpoint{0.100000in}{0.212622in}}{\pgfqpoint{3.696000in}{3.696000in}}%
\pgfusepath{clip}%
\pgfsetbuttcap%
\pgfsetroundjoin%
\definecolor{currentfill}{rgb}{0.121569,0.466667,0.705882}%
\pgfsetfillcolor{currentfill}%
\pgfsetfillopacity{0.739277}%
\pgfsetlinewidth{1.003750pt}%
\definecolor{currentstroke}{rgb}{0.121569,0.466667,0.705882}%
\pgfsetstrokecolor{currentstroke}%
\pgfsetstrokeopacity{0.739277}%
\pgfsetdash{}{0pt}%
\pgfpathmoveto{\pgfqpoint{2.074991in}{1.362549in}}%
\pgfpathcurveto{\pgfqpoint{2.083227in}{1.362549in}}{\pgfqpoint{2.091127in}{1.365822in}}{\pgfqpoint{2.096951in}{1.371646in}}%
\pgfpathcurveto{\pgfqpoint{2.102775in}{1.377469in}}{\pgfqpoint{2.106047in}{1.385369in}}{\pgfqpoint{2.106047in}{1.393606in}}%
\pgfpathcurveto{\pgfqpoint{2.106047in}{1.401842in}}{\pgfqpoint{2.102775in}{1.409742in}}{\pgfqpoint{2.096951in}{1.415566in}}%
\pgfpathcurveto{\pgfqpoint{2.091127in}{1.421390in}}{\pgfqpoint{2.083227in}{1.424662in}}{\pgfqpoint{2.074991in}{1.424662in}}%
\pgfpathcurveto{\pgfqpoint{2.066755in}{1.424662in}}{\pgfqpoint{2.058854in}{1.421390in}}{\pgfqpoint{2.053031in}{1.415566in}}%
\pgfpathcurveto{\pgfqpoint{2.047207in}{1.409742in}}{\pgfqpoint{2.043934in}{1.401842in}}{\pgfqpoint{2.043934in}{1.393606in}}%
\pgfpathcurveto{\pgfqpoint{2.043934in}{1.385369in}}{\pgfqpoint{2.047207in}{1.377469in}}{\pgfqpoint{2.053031in}{1.371646in}}%
\pgfpathcurveto{\pgfqpoint{2.058854in}{1.365822in}}{\pgfqpoint{2.066755in}{1.362549in}}{\pgfqpoint{2.074991in}{1.362549in}}%
\pgfpathclose%
\pgfusepath{stroke,fill}%
\end{pgfscope}%
\begin{pgfscope}%
\pgfpathrectangle{\pgfqpoint{0.100000in}{0.212622in}}{\pgfqpoint{3.696000in}{3.696000in}}%
\pgfusepath{clip}%
\pgfsetbuttcap%
\pgfsetroundjoin%
\definecolor{currentfill}{rgb}{0.121569,0.466667,0.705882}%
\pgfsetfillcolor{currentfill}%
\pgfsetfillopacity{0.743276}%
\pgfsetlinewidth{1.003750pt}%
\definecolor{currentstroke}{rgb}{0.121569,0.466667,0.705882}%
\pgfsetstrokecolor{currentstroke}%
\pgfsetstrokeopacity{0.743276}%
\pgfsetdash{}{0pt}%
\pgfpathmoveto{\pgfqpoint{2.086234in}{1.358814in}}%
\pgfpathcurveto{\pgfqpoint{2.094470in}{1.358814in}}{\pgfqpoint{2.102370in}{1.362086in}}{\pgfqpoint{2.108194in}{1.367910in}}%
\pgfpathcurveto{\pgfqpoint{2.114018in}{1.373734in}}{\pgfqpoint{2.117291in}{1.381634in}}{\pgfqpoint{2.117291in}{1.389871in}}%
\pgfpathcurveto{\pgfqpoint{2.117291in}{1.398107in}}{\pgfqpoint{2.114018in}{1.406007in}}{\pgfqpoint{2.108194in}{1.411831in}}%
\pgfpathcurveto{\pgfqpoint{2.102370in}{1.417655in}}{\pgfqpoint{2.094470in}{1.420927in}}{\pgfqpoint{2.086234in}{1.420927in}}%
\pgfpathcurveto{\pgfqpoint{2.077998in}{1.420927in}}{\pgfqpoint{2.070098in}{1.417655in}}{\pgfqpoint{2.064274in}{1.411831in}}%
\pgfpathcurveto{\pgfqpoint{2.058450in}{1.406007in}}{\pgfqpoint{2.055178in}{1.398107in}}{\pgfqpoint{2.055178in}{1.389871in}}%
\pgfpathcurveto{\pgfqpoint{2.055178in}{1.381634in}}{\pgfqpoint{2.058450in}{1.373734in}}{\pgfqpoint{2.064274in}{1.367910in}}%
\pgfpathcurveto{\pgfqpoint{2.070098in}{1.362086in}}{\pgfqpoint{2.077998in}{1.358814in}}{\pgfqpoint{2.086234in}{1.358814in}}%
\pgfpathclose%
\pgfusepath{stroke,fill}%
\end{pgfscope}%
\begin{pgfscope}%
\pgfpathrectangle{\pgfqpoint{0.100000in}{0.212622in}}{\pgfqpoint{3.696000in}{3.696000in}}%
\pgfusepath{clip}%
\pgfsetbuttcap%
\pgfsetroundjoin%
\definecolor{currentfill}{rgb}{0.121569,0.466667,0.705882}%
\pgfsetfillcolor{currentfill}%
\pgfsetfillopacity{0.748623}%
\pgfsetlinewidth{1.003750pt}%
\definecolor{currentstroke}{rgb}{0.121569,0.466667,0.705882}%
\pgfsetstrokecolor{currentstroke}%
\pgfsetstrokeopacity{0.748623}%
\pgfsetdash{}{0pt}%
\pgfpathmoveto{\pgfqpoint{2.101664in}{1.354514in}}%
\pgfpathcurveto{\pgfqpoint{2.109900in}{1.354514in}}{\pgfqpoint{2.117800in}{1.357786in}}{\pgfqpoint{2.123624in}{1.363610in}}%
\pgfpathcurveto{\pgfqpoint{2.129448in}{1.369434in}}{\pgfqpoint{2.132720in}{1.377334in}}{\pgfqpoint{2.132720in}{1.385570in}}%
\pgfpathcurveto{\pgfqpoint{2.132720in}{1.393807in}}{\pgfqpoint{2.129448in}{1.401707in}}{\pgfqpoint{2.123624in}{1.407531in}}%
\pgfpathcurveto{\pgfqpoint{2.117800in}{1.413354in}}{\pgfqpoint{2.109900in}{1.416627in}}{\pgfqpoint{2.101664in}{1.416627in}}%
\pgfpathcurveto{\pgfqpoint{2.093428in}{1.416627in}}{\pgfqpoint{2.085528in}{1.413354in}}{\pgfqpoint{2.079704in}{1.407531in}}%
\pgfpathcurveto{\pgfqpoint{2.073880in}{1.401707in}}{\pgfqpoint{2.070607in}{1.393807in}}{\pgfqpoint{2.070607in}{1.385570in}}%
\pgfpathcurveto{\pgfqpoint{2.070607in}{1.377334in}}{\pgfqpoint{2.073880in}{1.369434in}}{\pgfqpoint{2.079704in}{1.363610in}}%
\pgfpathcurveto{\pgfqpoint{2.085528in}{1.357786in}}{\pgfqpoint{2.093428in}{1.354514in}}{\pgfqpoint{2.101664in}{1.354514in}}%
\pgfpathclose%
\pgfusepath{stroke,fill}%
\end{pgfscope}%
\begin{pgfscope}%
\pgfpathrectangle{\pgfqpoint{0.100000in}{0.212622in}}{\pgfqpoint{3.696000in}{3.696000in}}%
\pgfusepath{clip}%
\pgfsetbuttcap%
\pgfsetroundjoin%
\definecolor{currentfill}{rgb}{0.121569,0.466667,0.705882}%
\pgfsetfillcolor{currentfill}%
\pgfsetfillopacity{0.756897}%
\pgfsetlinewidth{1.003750pt}%
\definecolor{currentstroke}{rgb}{0.121569,0.466667,0.705882}%
\pgfsetstrokecolor{currentstroke}%
\pgfsetstrokeopacity{0.756897}%
\pgfsetdash{}{0pt}%
\pgfpathmoveto{\pgfqpoint{2.125569in}{1.347238in}}%
\pgfpathcurveto{\pgfqpoint{2.133805in}{1.347238in}}{\pgfqpoint{2.141705in}{1.350510in}}{\pgfqpoint{2.147529in}{1.356334in}}%
\pgfpathcurveto{\pgfqpoint{2.153353in}{1.362158in}}{\pgfqpoint{2.156625in}{1.370058in}}{\pgfqpoint{2.156625in}{1.378294in}}%
\pgfpathcurveto{\pgfqpoint{2.156625in}{1.386531in}}{\pgfqpoint{2.153353in}{1.394431in}}{\pgfqpoint{2.147529in}{1.400255in}}%
\pgfpathcurveto{\pgfqpoint{2.141705in}{1.406079in}}{\pgfqpoint{2.133805in}{1.409351in}}{\pgfqpoint{2.125569in}{1.409351in}}%
\pgfpathcurveto{\pgfqpoint{2.117333in}{1.409351in}}{\pgfqpoint{2.109433in}{1.406079in}}{\pgfqpoint{2.103609in}{1.400255in}}%
\pgfpathcurveto{\pgfqpoint{2.097785in}{1.394431in}}{\pgfqpoint{2.094513in}{1.386531in}}{\pgfqpoint{2.094513in}{1.378294in}}%
\pgfpathcurveto{\pgfqpoint{2.094513in}{1.370058in}}{\pgfqpoint{2.097785in}{1.362158in}}{\pgfqpoint{2.103609in}{1.356334in}}%
\pgfpathcurveto{\pgfqpoint{2.109433in}{1.350510in}}{\pgfqpoint{2.117333in}{1.347238in}}{\pgfqpoint{2.125569in}{1.347238in}}%
\pgfpathclose%
\pgfusepath{stroke,fill}%
\end{pgfscope}%
\begin{pgfscope}%
\pgfpathrectangle{\pgfqpoint{0.100000in}{0.212622in}}{\pgfqpoint{3.696000in}{3.696000in}}%
\pgfusepath{clip}%
\pgfsetbuttcap%
\pgfsetroundjoin%
\definecolor{currentfill}{rgb}{0.121569,0.466667,0.705882}%
\pgfsetfillcolor{currentfill}%
\pgfsetfillopacity{0.767245}%
\pgfsetlinewidth{1.003750pt}%
\definecolor{currentstroke}{rgb}{0.121569,0.466667,0.705882}%
\pgfsetstrokecolor{currentstroke}%
\pgfsetstrokeopacity{0.767245}%
\pgfsetdash{}{0pt}%
\pgfpathmoveto{\pgfqpoint{2.154330in}{1.338216in}}%
\pgfpathcurveto{\pgfqpoint{2.162567in}{1.338216in}}{\pgfqpoint{2.170467in}{1.341488in}}{\pgfqpoint{2.176291in}{1.347312in}}%
\pgfpathcurveto{\pgfqpoint{2.182115in}{1.353136in}}{\pgfqpoint{2.185387in}{1.361036in}}{\pgfqpoint{2.185387in}{1.369272in}}%
\pgfpathcurveto{\pgfqpoint{2.185387in}{1.377509in}}{\pgfqpoint{2.182115in}{1.385409in}}{\pgfqpoint{2.176291in}{1.391233in}}%
\pgfpathcurveto{\pgfqpoint{2.170467in}{1.397057in}}{\pgfqpoint{2.162567in}{1.400329in}}{\pgfqpoint{2.154330in}{1.400329in}}%
\pgfpathcurveto{\pgfqpoint{2.146094in}{1.400329in}}{\pgfqpoint{2.138194in}{1.397057in}}{\pgfqpoint{2.132370in}{1.391233in}}%
\pgfpathcurveto{\pgfqpoint{2.126546in}{1.385409in}}{\pgfqpoint{2.123274in}{1.377509in}}{\pgfqpoint{2.123274in}{1.369272in}}%
\pgfpathcurveto{\pgfqpoint{2.123274in}{1.361036in}}{\pgfqpoint{2.126546in}{1.353136in}}{\pgfqpoint{2.132370in}{1.347312in}}%
\pgfpathcurveto{\pgfqpoint{2.138194in}{1.341488in}}{\pgfqpoint{2.146094in}{1.338216in}}{\pgfqpoint{2.154330in}{1.338216in}}%
\pgfpathclose%
\pgfusepath{stroke,fill}%
\end{pgfscope}%
\begin{pgfscope}%
\pgfpathrectangle{\pgfqpoint{0.100000in}{0.212622in}}{\pgfqpoint{3.696000in}{3.696000in}}%
\pgfusepath{clip}%
\pgfsetbuttcap%
\pgfsetroundjoin%
\definecolor{currentfill}{rgb}{0.121569,0.466667,0.705882}%
\pgfsetfillcolor{currentfill}%
\pgfsetfillopacity{0.779184}%
\pgfsetlinewidth{1.003750pt}%
\definecolor{currentstroke}{rgb}{0.121569,0.466667,0.705882}%
\pgfsetstrokecolor{currentstroke}%
\pgfsetstrokeopacity{0.779184}%
\pgfsetdash{}{0pt}%
\pgfpathmoveto{\pgfqpoint{2.186174in}{1.325044in}}%
\pgfpathcurveto{\pgfqpoint{2.194411in}{1.325044in}}{\pgfqpoint{2.202311in}{1.328316in}}{\pgfqpoint{2.208135in}{1.334140in}}%
\pgfpathcurveto{\pgfqpoint{2.213959in}{1.339964in}}{\pgfqpoint{2.217231in}{1.347864in}}{\pgfqpoint{2.217231in}{1.356100in}}%
\pgfpathcurveto{\pgfqpoint{2.217231in}{1.364337in}}{\pgfqpoint{2.213959in}{1.372237in}}{\pgfqpoint{2.208135in}{1.378061in}}%
\pgfpathcurveto{\pgfqpoint{2.202311in}{1.383885in}}{\pgfqpoint{2.194411in}{1.387157in}}{\pgfqpoint{2.186174in}{1.387157in}}%
\pgfpathcurveto{\pgfqpoint{2.177938in}{1.387157in}}{\pgfqpoint{2.170038in}{1.383885in}}{\pgfqpoint{2.164214in}{1.378061in}}%
\pgfpathcurveto{\pgfqpoint{2.158390in}{1.372237in}}{\pgfqpoint{2.155118in}{1.364337in}}{\pgfqpoint{2.155118in}{1.356100in}}%
\pgfpathcurveto{\pgfqpoint{2.155118in}{1.347864in}}{\pgfqpoint{2.158390in}{1.339964in}}{\pgfqpoint{2.164214in}{1.334140in}}%
\pgfpathcurveto{\pgfqpoint{2.170038in}{1.328316in}}{\pgfqpoint{2.177938in}{1.325044in}}{\pgfqpoint{2.186174in}{1.325044in}}%
\pgfpathclose%
\pgfusepath{stroke,fill}%
\end{pgfscope}%
\begin{pgfscope}%
\pgfpathrectangle{\pgfqpoint{0.100000in}{0.212622in}}{\pgfqpoint{3.696000in}{3.696000in}}%
\pgfusepath{clip}%
\pgfsetbuttcap%
\pgfsetroundjoin%
\definecolor{currentfill}{rgb}{0.121569,0.466667,0.705882}%
\pgfsetfillcolor{currentfill}%
\pgfsetfillopacity{0.792171}%
\pgfsetlinewidth{1.003750pt}%
\definecolor{currentstroke}{rgb}{0.121569,0.466667,0.705882}%
\pgfsetstrokecolor{currentstroke}%
\pgfsetstrokeopacity{0.792171}%
\pgfsetdash{}{0pt}%
\pgfpathmoveto{\pgfqpoint{2.226484in}{1.316851in}}%
\pgfpathcurveto{\pgfqpoint{2.234720in}{1.316851in}}{\pgfqpoint{2.242620in}{1.320123in}}{\pgfqpoint{2.248444in}{1.325947in}}%
\pgfpathcurveto{\pgfqpoint{2.254268in}{1.331771in}}{\pgfqpoint{2.257541in}{1.339671in}}{\pgfqpoint{2.257541in}{1.347908in}}%
\pgfpathcurveto{\pgfqpoint{2.257541in}{1.356144in}}{\pgfqpoint{2.254268in}{1.364044in}}{\pgfqpoint{2.248444in}{1.369868in}}%
\pgfpathcurveto{\pgfqpoint{2.242620in}{1.375692in}}{\pgfqpoint{2.234720in}{1.378964in}}{\pgfqpoint{2.226484in}{1.378964in}}%
\pgfpathcurveto{\pgfqpoint{2.218248in}{1.378964in}}{\pgfqpoint{2.210348in}{1.375692in}}{\pgfqpoint{2.204524in}{1.369868in}}%
\pgfpathcurveto{\pgfqpoint{2.198700in}{1.364044in}}{\pgfqpoint{2.195428in}{1.356144in}}{\pgfqpoint{2.195428in}{1.347908in}}%
\pgfpathcurveto{\pgfqpoint{2.195428in}{1.339671in}}{\pgfqpoint{2.198700in}{1.331771in}}{\pgfqpoint{2.204524in}{1.325947in}}%
\pgfpathcurveto{\pgfqpoint{2.210348in}{1.320123in}}{\pgfqpoint{2.218248in}{1.316851in}}{\pgfqpoint{2.226484in}{1.316851in}}%
\pgfpathclose%
\pgfusepath{stroke,fill}%
\end{pgfscope}%
\begin{pgfscope}%
\pgfpathrectangle{\pgfqpoint{0.100000in}{0.212622in}}{\pgfqpoint{3.696000in}{3.696000in}}%
\pgfusepath{clip}%
\pgfsetbuttcap%
\pgfsetroundjoin%
\definecolor{currentfill}{rgb}{0.121569,0.466667,0.705882}%
\pgfsetfillcolor{currentfill}%
\pgfsetfillopacity{0.799734}%
\pgfsetlinewidth{1.003750pt}%
\definecolor{currentstroke}{rgb}{0.121569,0.466667,0.705882}%
\pgfsetstrokecolor{currentstroke}%
\pgfsetstrokeopacity{0.799734}%
\pgfsetdash{}{0pt}%
\pgfpathmoveto{\pgfqpoint{2.247672in}{1.309462in}}%
\pgfpathcurveto{\pgfqpoint{2.255908in}{1.309462in}}{\pgfqpoint{2.263808in}{1.312734in}}{\pgfqpoint{2.269632in}{1.318558in}}%
\pgfpathcurveto{\pgfqpoint{2.275456in}{1.324382in}}{\pgfqpoint{2.278728in}{1.332282in}}{\pgfqpoint{2.278728in}{1.340518in}}%
\pgfpathcurveto{\pgfqpoint{2.278728in}{1.348754in}}{\pgfqpoint{2.275456in}{1.356654in}}{\pgfqpoint{2.269632in}{1.362478in}}%
\pgfpathcurveto{\pgfqpoint{2.263808in}{1.368302in}}{\pgfqpoint{2.255908in}{1.371575in}}{\pgfqpoint{2.247672in}{1.371575in}}%
\pgfpathcurveto{\pgfqpoint{2.239436in}{1.371575in}}{\pgfqpoint{2.231536in}{1.368302in}}{\pgfqpoint{2.225712in}{1.362478in}}%
\pgfpathcurveto{\pgfqpoint{2.219888in}{1.356654in}}{\pgfqpoint{2.216615in}{1.348754in}}{\pgfqpoint{2.216615in}{1.340518in}}%
\pgfpathcurveto{\pgfqpoint{2.216615in}{1.332282in}}{\pgfqpoint{2.219888in}{1.324382in}}{\pgfqpoint{2.225712in}{1.318558in}}%
\pgfpathcurveto{\pgfqpoint{2.231536in}{1.312734in}}{\pgfqpoint{2.239436in}{1.309462in}}{\pgfqpoint{2.247672in}{1.309462in}}%
\pgfpathclose%
\pgfusepath{stroke,fill}%
\end{pgfscope}%
\begin{pgfscope}%
\pgfpathrectangle{\pgfqpoint{0.100000in}{0.212622in}}{\pgfqpoint{3.696000in}{3.696000in}}%
\pgfusepath{clip}%
\pgfsetbuttcap%
\pgfsetroundjoin%
\definecolor{currentfill}{rgb}{0.121569,0.466667,0.705882}%
\pgfsetfillcolor{currentfill}%
\pgfsetfillopacity{0.803894}%
\pgfsetlinewidth{1.003750pt}%
\definecolor{currentstroke}{rgb}{0.121569,0.466667,0.705882}%
\pgfsetstrokecolor{currentstroke}%
\pgfsetstrokeopacity{0.803894}%
\pgfsetdash{}{0pt}%
\pgfpathmoveto{\pgfqpoint{2.259375in}{1.305641in}}%
\pgfpathcurveto{\pgfqpoint{2.267612in}{1.305641in}}{\pgfqpoint{2.275512in}{1.308913in}}{\pgfqpoint{2.281336in}{1.314737in}}%
\pgfpathcurveto{\pgfqpoint{2.287160in}{1.320561in}}{\pgfqpoint{2.290432in}{1.328461in}}{\pgfqpoint{2.290432in}{1.336697in}}%
\pgfpathcurveto{\pgfqpoint{2.290432in}{1.344934in}}{\pgfqpoint{2.287160in}{1.352834in}}{\pgfqpoint{2.281336in}{1.358658in}}%
\pgfpathcurveto{\pgfqpoint{2.275512in}{1.364482in}}{\pgfqpoint{2.267612in}{1.367754in}}{\pgfqpoint{2.259375in}{1.367754in}}%
\pgfpathcurveto{\pgfqpoint{2.251139in}{1.367754in}}{\pgfqpoint{2.243239in}{1.364482in}}{\pgfqpoint{2.237415in}{1.358658in}}%
\pgfpathcurveto{\pgfqpoint{2.231591in}{1.352834in}}{\pgfqpoint{2.228319in}{1.344934in}}{\pgfqpoint{2.228319in}{1.336697in}}%
\pgfpathcurveto{\pgfqpoint{2.228319in}{1.328461in}}{\pgfqpoint{2.231591in}{1.320561in}}{\pgfqpoint{2.237415in}{1.314737in}}%
\pgfpathcurveto{\pgfqpoint{2.243239in}{1.308913in}}{\pgfqpoint{2.251139in}{1.305641in}}{\pgfqpoint{2.259375in}{1.305641in}}%
\pgfpathclose%
\pgfusepath{stroke,fill}%
\end{pgfscope}%
\begin{pgfscope}%
\pgfpathrectangle{\pgfqpoint{0.100000in}{0.212622in}}{\pgfqpoint{3.696000in}{3.696000in}}%
\pgfusepath{clip}%
\pgfsetbuttcap%
\pgfsetroundjoin%
\definecolor{currentfill}{rgb}{0.121569,0.466667,0.705882}%
\pgfsetfillcolor{currentfill}%
\pgfsetfillopacity{0.809844}%
\pgfsetlinewidth{1.003750pt}%
\definecolor{currentstroke}{rgb}{0.121569,0.466667,0.705882}%
\pgfsetstrokecolor{currentstroke}%
\pgfsetstrokeopacity{0.809844}%
\pgfsetdash{}{0pt}%
\pgfpathmoveto{\pgfqpoint{2.276074in}{1.300099in}}%
\pgfpathcurveto{\pgfqpoint{2.284310in}{1.300099in}}{\pgfqpoint{2.292210in}{1.303371in}}{\pgfqpoint{2.298034in}{1.309195in}}%
\pgfpathcurveto{\pgfqpoint{2.303858in}{1.315019in}}{\pgfqpoint{2.307130in}{1.322919in}}{\pgfqpoint{2.307130in}{1.331156in}}%
\pgfpathcurveto{\pgfqpoint{2.307130in}{1.339392in}}{\pgfqpoint{2.303858in}{1.347292in}}{\pgfqpoint{2.298034in}{1.353116in}}%
\pgfpathcurveto{\pgfqpoint{2.292210in}{1.358940in}}{\pgfqpoint{2.284310in}{1.362212in}}{\pgfqpoint{2.276074in}{1.362212in}}%
\pgfpathcurveto{\pgfqpoint{2.267837in}{1.362212in}}{\pgfqpoint{2.259937in}{1.358940in}}{\pgfqpoint{2.254113in}{1.353116in}}%
\pgfpathcurveto{\pgfqpoint{2.248289in}{1.347292in}}{\pgfqpoint{2.245017in}{1.339392in}}{\pgfqpoint{2.245017in}{1.331156in}}%
\pgfpathcurveto{\pgfqpoint{2.245017in}{1.322919in}}{\pgfqpoint{2.248289in}{1.315019in}}{\pgfqpoint{2.254113in}{1.309195in}}%
\pgfpathcurveto{\pgfqpoint{2.259937in}{1.303371in}}{\pgfqpoint{2.267837in}{1.300099in}}{\pgfqpoint{2.276074in}{1.300099in}}%
\pgfpathclose%
\pgfusepath{stroke,fill}%
\end{pgfscope}%
\begin{pgfscope}%
\pgfpathrectangle{\pgfqpoint{0.100000in}{0.212622in}}{\pgfqpoint{3.696000in}{3.696000in}}%
\pgfusepath{clip}%
\pgfsetbuttcap%
\pgfsetroundjoin%
\definecolor{currentfill}{rgb}{0.121569,0.466667,0.705882}%
\pgfsetfillcolor{currentfill}%
\pgfsetfillopacity{0.812952}%
\pgfsetlinewidth{1.003750pt}%
\definecolor{currentstroke}{rgb}{0.121569,0.466667,0.705882}%
\pgfsetstrokecolor{currentstroke}%
\pgfsetstrokeopacity{0.812952}%
\pgfsetdash{}{0pt}%
\pgfpathmoveto{\pgfqpoint{2.285495in}{1.297586in}}%
\pgfpathcurveto{\pgfqpoint{2.293731in}{1.297586in}}{\pgfqpoint{2.301631in}{1.300858in}}{\pgfqpoint{2.307455in}{1.306682in}}%
\pgfpathcurveto{\pgfqpoint{2.313279in}{1.312506in}}{\pgfqpoint{2.316551in}{1.320406in}}{\pgfqpoint{2.316551in}{1.328642in}}%
\pgfpathcurveto{\pgfqpoint{2.316551in}{1.336879in}}{\pgfqpoint{2.313279in}{1.344779in}}{\pgfqpoint{2.307455in}{1.350603in}}%
\pgfpathcurveto{\pgfqpoint{2.301631in}{1.356426in}}{\pgfqpoint{2.293731in}{1.359699in}}{\pgfqpoint{2.285495in}{1.359699in}}%
\pgfpathcurveto{\pgfqpoint{2.277258in}{1.359699in}}{\pgfqpoint{2.269358in}{1.356426in}}{\pgfqpoint{2.263534in}{1.350603in}}%
\pgfpathcurveto{\pgfqpoint{2.257710in}{1.344779in}}{\pgfqpoint{2.254438in}{1.336879in}}{\pgfqpoint{2.254438in}{1.328642in}}%
\pgfpathcurveto{\pgfqpoint{2.254438in}{1.320406in}}{\pgfqpoint{2.257710in}{1.312506in}}{\pgfqpoint{2.263534in}{1.306682in}}%
\pgfpathcurveto{\pgfqpoint{2.269358in}{1.300858in}}{\pgfqpoint{2.277258in}{1.297586in}}{\pgfqpoint{2.285495in}{1.297586in}}%
\pgfpathclose%
\pgfusepath{stroke,fill}%
\end{pgfscope}%
\begin{pgfscope}%
\pgfpathrectangle{\pgfqpoint{0.100000in}{0.212622in}}{\pgfqpoint{3.696000in}{3.696000in}}%
\pgfusepath{clip}%
\pgfsetbuttcap%
\pgfsetroundjoin%
\definecolor{currentfill}{rgb}{0.121569,0.466667,0.705882}%
\pgfsetfillcolor{currentfill}%
\pgfsetfillopacity{0.816909}%
\pgfsetlinewidth{1.003750pt}%
\definecolor{currentstroke}{rgb}{0.121569,0.466667,0.705882}%
\pgfsetstrokecolor{currentstroke}%
\pgfsetstrokeopacity{0.816909}%
\pgfsetdash{}{0pt}%
\pgfpathmoveto{\pgfqpoint{2.297680in}{1.294328in}}%
\pgfpathcurveto{\pgfqpoint{2.305916in}{1.294328in}}{\pgfqpoint{2.313816in}{1.297600in}}{\pgfqpoint{2.319640in}{1.303424in}}%
\pgfpathcurveto{\pgfqpoint{2.325464in}{1.309248in}}{\pgfqpoint{2.328736in}{1.317148in}}{\pgfqpoint{2.328736in}{1.325384in}}%
\pgfpathcurveto{\pgfqpoint{2.328736in}{1.333621in}}{\pgfqpoint{2.325464in}{1.341521in}}{\pgfqpoint{2.319640in}{1.347345in}}%
\pgfpathcurveto{\pgfqpoint{2.313816in}{1.353169in}}{\pgfqpoint{2.305916in}{1.356441in}}{\pgfqpoint{2.297680in}{1.356441in}}%
\pgfpathcurveto{\pgfqpoint{2.289443in}{1.356441in}}{\pgfqpoint{2.281543in}{1.353169in}}{\pgfqpoint{2.275719in}{1.347345in}}%
\pgfpathcurveto{\pgfqpoint{2.269895in}{1.341521in}}{\pgfqpoint{2.266623in}{1.333621in}}{\pgfqpoint{2.266623in}{1.325384in}}%
\pgfpathcurveto{\pgfqpoint{2.266623in}{1.317148in}}{\pgfqpoint{2.269895in}{1.309248in}}{\pgfqpoint{2.275719in}{1.303424in}}%
\pgfpathcurveto{\pgfqpoint{2.281543in}{1.297600in}}{\pgfqpoint{2.289443in}{1.294328in}}{\pgfqpoint{2.297680in}{1.294328in}}%
\pgfpathclose%
\pgfusepath{stroke,fill}%
\end{pgfscope}%
\begin{pgfscope}%
\pgfpathrectangle{\pgfqpoint{0.100000in}{0.212622in}}{\pgfqpoint{3.696000in}{3.696000in}}%
\pgfusepath{clip}%
\pgfsetbuttcap%
\pgfsetroundjoin%
\definecolor{currentfill}{rgb}{0.121569,0.466667,0.705882}%
\pgfsetfillcolor{currentfill}%
\pgfsetfillopacity{0.821854}%
\pgfsetlinewidth{1.003750pt}%
\definecolor{currentstroke}{rgb}{0.121569,0.466667,0.705882}%
\pgfsetstrokecolor{currentstroke}%
\pgfsetstrokeopacity{0.821854}%
\pgfsetdash{}{0pt}%
\pgfpathmoveto{\pgfqpoint{2.313505in}{1.290462in}}%
\pgfpathcurveto{\pgfqpoint{2.321742in}{1.290462in}}{\pgfqpoint{2.329642in}{1.293734in}}{\pgfqpoint{2.335466in}{1.299558in}}%
\pgfpathcurveto{\pgfqpoint{2.341290in}{1.305382in}}{\pgfqpoint{2.344562in}{1.313282in}}{\pgfqpoint{2.344562in}{1.321518in}}%
\pgfpathcurveto{\pgfqpoint{2.344562in}{1.329754in}}{\pgfqpoint{2.341290in}{1.337654in}}{\pgfqpoint{2.335466in}{1.343478in}}%
\pgfpathcurveto{\pgfqpoint{2.329642in}{1.349302in}}{\pgfqpoint{2.321742in}{1.352575in}}{\pgfqpoint{2.313505in}{1.352575in}}%
\pgfpathcurveto{\pgfqpoint{2.305269in}{1.352575in}}{\pgfqpoint{2.297369in}{1.349302in}}{\pgfqpoint{2.291545in}{1.343478in}}%
\pgfpathcurveto{\pgfqpoint{2.285721in}{1.337654in}}{\pgfqpoint{2.282449in}{1.329754in}}{\pgfqpoint{2.282449in}{1.321518in}}%
\pgfpathcurveto{\pgfqpoint{2.282449in}{1.313282in}}{\pgfqpoint{2.285721in}{1.305382in}}{\pgfqpoint{2.291545in}{1.299558in}}%
\pgfpathcurveto{\pgfqpoint{2.297369in}{1.293734in}}{\pgfqpoint{2.305269in}{1.290462in}}{\pgfqpoint{2.313505in}{1.290462in}}%
\pgfpathclose%
\pgfusepath{stroke,fill}%
\end{pgfscope}%
\begin{pgfscope}%
\pgfpathrectangle{\pgfqpoint{0.100000in}{0.212622in}}{\pgfqpoint{3.696000in}{3.696000in}}%
\pgfusepath{clip}%
\pgfsetbuttcap%
\pgfsetroundjoin%
\definecolor{currentfill}{rgb}{0.121569,0.466667,0.705882}%
\pgfsetfillcolor{currentfill}%
\pgfsetfillopacity{0.824521}%
\pgfsetlinewidth{1.003750pt}%
\definecolor{currentstroke}{rgb}{0.121569,0.466667,0.705882}%
\pgfsetstrokecolor{currentstroke}%
\pgfsetstrokeopacity{0.824521}%
\pgfsetdash{}{0pt}%
\pgfpathmoveto{\pgfqpoint{2.322298in}{1.288471in}}%
\pgfpathcurveto{\pgfqpoint{2.330534in}{1.288471in}}{\pgfqpoint{2.338434in}{1.291743in}}{\pgfqpoint{2.344258in}{1.297567in}}%
\pgfpathcurveto{\pgfqpoint{2.350082in}{1.303391in}}{\pgfqpoint{2.353354in}{1.311291in}}{\pgfqpoint{2.353354in}{1.319527in}}%
\pgfpathcurveto{\pgfqpoint{2.353354in}{1.327764in}}{\pgfqpoint{2.350082in}{1.335664in}}{\pgfqpoint{2.344258in}{1.341488in}}%
\pgfpathcurveto{\pgfqpoint{2.338434in}{1.347312in}}{\pgfqpoint{2.330534in}{1.350584in}}{\pgfqpoint{2.322298in}{1.350584in}}%
\pgfpathcurveto{\pgfqpoint{2.314061in}{1.350584in}}{\pgfqpoint{2.306161in}{1.347312in}}{\pgfqpoint{2.300337in}{1.341488in}}%
\pgfpathcurveto{\pgfqpoint{2.294513in}{1.335664in}}{\pgfqpoint{2.291241in}{1.327764in}}{\pgfqpoint{2.291241in}{1.319527in}}%
\pgfpathcurveto{\pgfqpoint{2.291241in}{1.311291in}}{\pgfqpoint{2.294513in}{1.303391in}}{\pgfqpoint{2.300337in}{1.297567in}}%
\pgfpathcurveto{\pgfqpoint{2.306161in}{1.291743in}}{\pgfqpoint{2.314061in}{1.288471in}}{\pgfqpoint{2.322298in}{1.288471in}}%
\pgfpathclose%
\pgfusepath{stroke,fill}%
\end{pgfscope}%
\begin{pgfscope}%
\pgfpathrectangle{\pgfqpoint{0.100000in}{0.212622in}}{\pgfqpoint{3.696000in}{3.696000in}}%
\pgfusepath{clip}%
\pgfsetbuttcap%
\pgfsetroundjoin%
\definecolor{currentfill}{rgb}{0.121569,0.466667,0.705882}%
\pgfsetfillcolor{currentfill}%
\pgfsetfillopacity{0.828038}%
\pgfsetlinewidth{1.003750pt}%
\definecolor{currentstroke}{rgb}{0.121569,0.466667,0.705882}%
\pgfsetstrokecolor{currentstroke}%
\pgfsetstrokeopacity{0.828038}%
\pgfsetdash{}{0pt}%
\pgfpathmoveto{\pgfqpoint{2.333624in}{1.285713in}}%
\pgfpathcurveto{\pgfqpoint{2.341861in}{1.285713in}}{\pgfqpoint{2.349761in}{1.288985in}}{\pgfqpoint{2.355585in}{1.294809in}}%
\pgfpathcurveto{\pgfqpoint{2.361409in}{1.300633in}}{\pgfqpoint{2.364681in}{1.308533in}}{\pgfqpoint{2.364681in}{1.316769in}}%
\pgfpathcurveto{\pgfqpoint{2.364681in}{1.325006in}}{\pgfqpoint{2.361409in}{1.332906in}}{\pgfqpoint{2.355585in}{1.338730in}}%
\pgfpathcurveto{\pgfqpoint{2.349761in}{1.344553in}}{\pgfqpoint{2.341861in}{1.347826in}}{\pgfqpoint{2.333624in}{1.347826in}}%
\pgfpathcurveto{\pgfqpoint{2.325388in}{1.347826in}}{\pgfqpoint{2.317488in}{1.344553in}}{\pgfqpoint{2.311664in}{1.338730in}}%
\pgfpathcurveto{\pgfqpoint{2.305840in}{1.332906in}}{\pgfqpoint{2.302568in}{1.325006in}}{\pgfqpoint{2.302568in}{1.316769in}}%
\pgfpathcurveto{\pgfqpoint{2.302568in}{1.308533in}}{\pgfqpoint{2.305840in}{1.300633in}}{\pgfqpoint{2.311664in}{1.294809in}}%
\pgfpathcurveto{\pgfqpoint{2.317488in}{1.288985in}}{\pgfqpoint{2.325388in}{1.285713in}}{\pgfqpoint{2.333624in}{1.285713in}}%
\pgfpathclose%
\pgfusepath{stroke,fill}%
\end{pgfscope}%
\begin{pgfscope}%
\pgfpathrectangle{\pgfqpoint{0.100000in}{0.212622in}}{\pgfqpoint{3.696000in}{3.696000in}}%
\pgfusepath{clip}%
\pgfsetbuttcap%
\pgfsetroundjoin%
\definecolor{currentfill}{rgb}{0.121569,0.466667,0.705882}%
\pgfsetfillcolor{currentfill}%
\pgfsetfillopacity{0.832288}%
\pgfsetlinewidth{1.003750pt}%
\definecolor{currentstroke}{rgb}{0.121569,0.466667,0.705882}%
\pgfsetstrokecolor{currentstroke}%
\pgfsetstrokeopacity{0.832288}%
\pgfsetdash{}{0pt}%
\pgfpathmoveto{\pgfqpoint{2.347977in}{1.282763in}}%
\pgfpathcurveto{\pgfqpoint{2.356213in}{1.282763in}}{\pgfqpoint{2.364113in}{1.286035in}}{\pgfqpoint{2.369937in}{1.291859in}}%
\pgfpathcurveto{\pgfqpoint{2.375761in}{1.297683in}}{\pgfqpoint{2.379033in}{1.305583in}}{\pgfqpoint{2.379033in}{1.313819in}}%
\pgfpathcurveto{\pgfqpoint{2.379033in}{1.322056in}}{\pgfqpoint{2.375761in}{1.329956in}}{\pgfqpoint{2.369937in}{1.335780in}}%
\pgfpathcurveto{\pgfqpoint{2.364113in}{1.341603in}}{\pgfqpoint{2.356213in}{1.344876in}}{\pgfqpoint{2.347977in}{1.344876in}}%
\pgfpathcurveto{\pgfqpoint{2.339740in}{1.344876in}}{\pgfqpoint{2.331840in}{1.341603in}}{\pgfqpoint{2.326016in}{1.335780in}}%
\pgfpathcurveto{\pgfqpoint{2.320192in}{1.329956in}}{\pgfqpoint{2.316920in}{1.322056in}}{\pgfqpoint{2.316920in}{1.313819in}}%
\pgfpathcurveto{\pgfqpoint{2.316920in}{1.305583in}}{\pgfqpoint{2.320192in}{1.297683in}}{\pgfqpoint{2.326016in}{1.291859in}}%
\pgfpathcurveto{\pgfqpoint{2.331840in}{1.286035in}}{\pgfqpoint{2.339740in}{1.282763in}}{\pgfqpoint{2.347977in}{1.282763in}}%
\pgfpathclose%
\pgfusepath{stroke,fill}%
\end{pgfscope}%
\begin{pgfscope}%
\pgfpathrectangle{\pgfqpoint{0.100000in}{0.212622in}}{\pgfqpoint{3.696000in}{3.696000in}}%
\pgfusepath{clip}%
\pgfsetbuttcap%
\pgfsetroundjoin%
\definecolor{currentfill}{rgb}{0.121569,0.466667,0.705882}%
\pgfsetfillcolor{currentfill}%
\pgfsetfillopacity{0.837558}%
\pgfsetlinewidth{1.003750pt}%
\definecolor{currentstroke}{rgb}{0.121569,0.466667,0.705882}%
\pgfsetstrokecolor{currentstroke}%
\pgfsetstrokeopacity{0.837558}%
\pgfsetdash{}{0pt}%
\pgfpathmoveto{\pgfqpoint{2.365849in}{1.278992in}}%
\pgfpathcurveto{\pgfqpoint{2.374085in}{1.278992in}}{\pgfqpoint{2.381985in}{1.282264in}}{\pgfqpoint{2.387809in}{1.288088in}}%
\pgfpathcurveto{\pgfqpoint{2.393633in}{1.293912in}}{\pgfqpoint{2.396905in}{1.301812in}}{\pgfqpoint{2.396905in}{1.310048in}}%
\pgfpathcurveto{\pgfqpoint{2.396905in}{1.318285in}}{\pgfqpoint{2.393633in}{1.326185in}}{\pgfqpoint{2.387809in}{1.332008in}}%
\pgfpathcurveto{\pgfqpoint{2.381985in}{1.337832in}}{\pgfqpoint{2.374085in}{1.341105in}}{\pgfqpoint{2.365849in}{1.341105in}}%
\pgfpathcurveto{\pgfqpoint{2.357613in}{1.341105in}}{\pgfqpoint{2.349713in}{1.337832in}}{\pgfqpoint{2.343889in}{1.332008in}}%
\pgfpathcurveto{\pgfqpoint{2.338065in}{1.326185in}}{\pgfqpoint{2.334792in}{1.318285in}}{\pgfqpoint{2.334792in}{1.310048in}}%
\pgfpathcurveto{\pgfqpoint{2.334792in}{1.301812in}}{\pgfqpoint{2.338065in}{1.293912in}}{\pgfqpoint{2.343889in}{1.288088in}}%
\pgfpathcurveto{\pgfqpoint{2.349713in}{1.282264in}}{\pgfqpoint{2.357613in}{1.278992in}}{\pgfqpoint{2.365849in}{1.278992in}}%
\pgfpathclose%
\pgfusepath{stroke,fill}%
\end{pgfscope}%
\begin{pgfscope}%
\pgfpathrectangle{\pgfqpoint{0.100000in}{0.212622in}}{\pgfqpoint{3.696000in}{3.696000in}}%
\pgfusepath{clip}%
\pgfsetbuttcap%
\pgfsetroundjoin%
\definecolor{currentfill}{rgb}{0.121569,0.466667,0.705882}%
\pgfsetfillcolor{currentfill}%
\pgfsetfillopacity{0.843737}%
\pgfsetlinewidth{1.003750pt}%
\definecolor{currentstroke}{rgb}{0.121569,0.466667,0.705882}%
\pgfsetstrokecolor{currentstroke}%
\pgfsetstrokeopacity{0.843737}%
\pgfsetdash{}{0pt}%
\pgfpathmoveto{\pgfqpoint{2.386421in}{1.274238in}}%
\pgfpathcurveto{\pgfqpoint{2.394657in}{1.274238in}}{\pgfqpoint{2.402557in}{1.277511in}}{\pgfqpoint{2.408381in}{1.283335in}}%
\pgfpathcurveto{\pgfqpoint{2.414205in}{1.289159in}}{\pgfqpoint{2.417477in}{1.297059in}}{\pgfqpoint{2.417477in}{1.305295in}}%
\pgfpathcurveto{\pgfqpoint{2.417477in}{1.313531in}}{\pgfqpoint{2.414205in}{1.321431in}}{\pgfqpoint{2.408381in}{1.327255in}}%
\pgfpathcurveto{\pgfqpoint{2.402557in}{1.333079in}}{\pgfqpoint{2.394657in}{1.336351in}}{\pgfqpoint{2.386421in}{1.336351in}}%
\pgfpathcurveto{\pgfqpoint{2.378184in}{1.336351in}}{\pgfqpoint{2.370284in}{1.333079in}}{\pgfqpoint{2.364460in}{1.327255in}}%
\pgfpathcurveto{\pgfqpoint{2.358637in}{1.321431in}}{\pgfqpoint{2.355364in}{1.313531in}}{\pgfqpoint{2.355364in}{1.305295in}}%
\pgfpathcurveto{\pgfqpoint{2.355364in}{1.297059in}}{\pgfqpoint{2.358637in}{1.289159in}}{\pgfqpoint{2.364460in}{1.283335in}}%
\pgfpathcurveto{\pgfqpoint{2.370284in}{1.277511in}}{\pgfqpoint{2.378184in}{1.274238in}}{\pgfqpoint{2.386421in}{1.274238in}}%
\pgfpathclose%
\pgfusepath{stroke,fill}%
\end{pgfscope}%
\begin{pgfscope}%
\pgfpathrectangle{\pgfqpoint{0.100000in}{0.212622in}}{\pgfqpoint{3.696000in}{3.696000in}}%
\pgfusepath{clip}%
\pgfsetbuttcap%
\pgfsetroundjoin%
\definecolor{currentfill}{rgb}{0.121569,0.466667,0.705882}%
\pgfsetfillcolor{currentfill}%
\pgfsetfillopacity{0.850964}%
\pgfsetlinewidth{1.003750pt}%
\definecolor{currentstroke}{rgb}{0.121569,0.466667,0.705882}%
\pgfsetstrokecolor{currentstroke}%
\pgfsetstrokeopacity{0.850964}%
\pgfsetdash{}{0pt}%
\pgfpathmoveto{\pgfqpoint{2.409348in}{1.268099in}}%
\pgfpathcurveto{\pgfqpoint{2.417584in}{1.268099in}}{\pgfqpoint{2.425484in}{1.271371in}}{\pgfqpoint{2.431308in}{1.277195in}}%
\pgfpathcurveto{\pgfqpoint{2.437132in}{1.283019in}}{\pgfqpoint{2.440404in}{1.290919in}}{\pgfqpoint{2.440404in}{1.299155in}}%
\pgfpathcurveto{\pgfqpoint{2.440404in}{1.307391in}}{\pgfqpoint{2.437132in}{1.315291in}}{\pgfqpoint{2.431308in}{1.321115in}}%
\pgfpathcurveto{\pgfqpoint{2.425484in}{1.326939in}}{\pgfqpoint{2.417584in}{1.330212in}}{\pgfqpoint{2.409348in}{1.330212in}}%
\pgfpathcurveto{\pgfqpoint{2.401111in}{1.330212in}}{\pgfqpoint{2.393211in}{1.326939in}}{\pgfqpoint{2.387387in}{1.321115in}}%
\pgfpathcurveto{\pgfqpoint{2.381563in}{1.315291in}}{\pgfqpoint{2.378291in}{1.307391in}}{\pgfqpoint{2.378291in}{1.299155in}}%
\pgfpathcurveto{\pgfqpoint{2.378291in}{1.290919in}}{\pgfqpoint{2.381563in}{1.283019in}}{\pgfqpoint{2.387387in}{1.277195in}}%
\pgfpathcurveto{\pgfqpoint{2.393211in}{1.271371in}}{\pgfqpoint{2.401111in}{1.268099in}}{\pgfqpoint{2.409348in}{1.268099in}}%
\pgfpathclose%
\pgfusepath{stroke,fill}%
\end{pgfscope}%
\begin{pgfscope}%
\pgfpathrectangle{\pgfqpoint{0.100000in}{0.212622in}}{\pgfqpoint{3.696000in}{3.696000in}}%
\pgfusepath{clip}%
\pgfsetbuttcap%
\pgfsetroundjoin%
\definecolor{currentfill}{rgb}{0.121569,0.466667,0.705882}%
\pgfsetfillcolor{currentfill}%
\pgfsetfillopacity{0.859149}%
\pgfsetlinewidth{1.003750pt}%
\definecolor{currentstroke}{rgb}{0.121569,0.466667,0.705882}%
\pgfsetstrokecolor{currentstroke}%
\pgfsetstrokeopacity{0.859149}%
\pgfsetdash{}{0pt}%
\pgfpathmoveto{\pgfqpoint{2.435257in}{1.261022in}}%
\pgfpathcurveto{\pgfqpoint{2.443494in}{1.261022in}}{\pgfqpoint{2.451394in}{1.264294in}}{\pgfqpoint{2.457218in}{1.270118in}}%
\pgfpathcurveto{\pgfqpoint{2.463042in}{1.275942in}}{\pgfqpoint{2.466314in}{1.283842in}}{\pgfqpoint{2.466314in}{1.292078in}}%
\pgfpathcurveto{\pgfqpoint{2.466314in}{1.300315in}}{\pgfqpoint{2.463042in}{1.308215in}}{\pgfqpoint{2.457218in}{1.314039in}}%
\pgfpathcurveto{\pgfqpoint{2.451394in}{1.319863in}}{\pgfqpoint{2.443494in}{1.323135in}}{\pgfqpoint{2.435257in}{1.323135in}}%
\pgfpathcurveto{\pgfqpoint{2.427021in}{1.323135in}}{\pgfqpoint{2.419121in}{1.319863in}}{\pgfqpoint{2.413297in}{1.314039in}}%
\pgfpathcurveto{\pgfqpoint{2.407473in}{1.308215in}}{\pgfqpoint{2.404201in}{1.300315in}}{\pgfqpoint{2.404201in}{1.292078in}}%
\pgfpathcurveto{\pgfqpoint{2.404201in}{1.283842in}}{\pgfqpoint{2.407473in}{1.275942in}}{\pgfqpoint{2.413297in}{1.270118in}}%
\pgfpathcurveto{\pgfqpoint{2.419121in}{1.264294in}}{\pgfqpoint{2.427021in}{1.261022in}}{\pgfqpoint{2.435257in}{1.261022in}}%
\pgfpathclose%
\pgfusepath{stroke,fill}%
\end{pgfscope}%
\begin{pgfscope}%
\pgfpathrectangle{\pgfqpoint{0.100000in}{0.212622in}}{\pgfqpoint{3.696000in}{3.696000in}}%
\pgfusepath{clip}%
\pgfsetbuttcap%
\pgfsetroundjoin%
\definecolor{currentfill}{rgb}{0.121569,0.466667,0.705882}%
\pgfsetfillcolor{currentfill}%
\pgfsetfillopacity{0.867966}%
\pgfsetlinewidth{1.003750pt}%
\definecolor{currentstroke}{rgb}{0.121569,0.466667,0.705882}%
\pgfsetstrokecolor{currentstroke}%
\pgfsetstrokeopacity{0.867966}%
\pgfsetdash{}{0pt}%
\pgfpathmoveto{\pgfqpoint{2.463960in}{1.253982in}}%
\pgfpathcurveto{\pgfqpoint{2.472196in}{1.253982in}}{\pgfqpoint{2.480096in}{1.257254in}}{\pgfqpoint{2.485920in}{1.263078in}}%
\pgfpathcurveto{\pgfqpoint{2.491744in}{1.268902in}}{\pgfqpoint{2.495016in}{1.276802in}}{\pgfqpoint{2.495016in}{1.285039in}}%
\pgfpathcurveto{\pgfqpoint{2.495016in}{1.293275in}}{\pgfqpoint{2.491744in}{1.301175in}}{\pgfqpoint{2.485920in}{1.306999in}}%
\pgfpathcurveto{\pgfqpoint{2.480096in}{1.312823in}}{\pgfqpoint{2.472196in}{1.316095in}}{\pgfqpoint{2.463960in}{1.316095in}}%
\pgfpathcurveto{\pgfqpoint{2.455723in}{1.316095in}}{\pgfqpoint{2.447823in}{1.312823in}}{\pgfqpoint{2.441999in}{1.306999in}}%
\pgfpathcurveto{\pgfqpoint{2.436175in}{1.301175in}}{\pgfqpoint{2.432903in}{1.293275in}}{\pgfqpoint{2.432903in}{1.285039in}}%
\pgfpathcurveto{\pgfqpoint{2.432903in}{1.276802in}}{\pgfqpoint{2.436175in}{1.268902in}}{\pgfqpoint{2.441999in}{1.263078in}}%
\pgfpathcurveto{\pgfqpoint{2.447823in}{1.257254in}}{\pgfqpoint{2.455723in}{1.253982in}}{\pgfqpoint{2.463960in}{1.253982in}}%
\pgfpathclose%
\pgfusepath{stroke,fill}%
\end{pgfscope}%
\begin{pgfscope}%
\pgfpathrectangle{\pgfqpoint{0.100000in}{0.212622in}}{\pgfqpoint{3.696000in}{3.696000in}}%
\pgfusepath{clip}%
\pgfsetbuttcap%
\pgfsetroundjoin%
\definecolor{currentfill}{rgb}{0.121569,0.466667,0.705882}%
\pgfsetfillcolor{currentfill}%
\pgfsetfillopacity{0.877604}%
\pgfsetlinewidth{1.003750pt}%
\definecolor{currentstroke}{rgb}{0.121569,0.466667,0.705882}%
\pgfsetstrokecolor{currentstroke}%
\pgfsetstrokeopacity{0.877604}%
\pgfsetdash{}{0pt}%
\pgfpathmoveto{\pgfqpoint{2.496348in}{1.247090in}}%
\pgfpathcurveto{\pgfqpoint{2.504585in}{1.247090in}}{\pgfqpoint{2.512485in}{1.250362in}}{\pgfqpoint{2.518309in}{1.256186in}}%
\pgfpathcurveto{\pgfqpoint{2.524133in}{1.262010in}}{\pgfqpoint{2.527405in}{1.269910in}}{\pgfqpoint{2.527405in}{1.278146in}}%
\pgfpathcurveto{\pgfqpoint{2.527405in}{1.286383in}}{\pgfqpoint{2.524133in}{1.294283in}}{\pgfqpoint{2.518309in}{1.300107in}}%
\pgfpathcurveto{\pgfqpoint{2.512485in}{1.305931in}}{\pgfqpoint{2.504585in}{1.309203in}}{\pgfqpoint{2.496348in}{1.309203in}}%
\pgfpathcurveto{\pgfqpoint{2.488112in}{1.309203in}}{\pgfqpoint{2.480212in}{1.305931in}}{\pgfqpoint{2.474388in}{1.300107in}}%
\pgfpathcurveto{\pgfqpoint{2.468564in}{1.294283in}}{\pgfqpoint{2.465292in}{1.286383in}}{\pgfqpoint{2.465292in}{1.278146in}}%
\pgfpathcurveto{\pgfqpoint{2.465292in}{1.269910in}}{\pgfqpoint{2.468564in}{1.262010in}}{\pgfqpoint{2.474388in}{1.256186in}}%
\pgfpathcurveto{\pgfqpoint{2.480212in}{1.250362in}}{\pgfqpoint{2.488112in}{1.247090in}}{\pgfqpoint{2.496348in}{1.247090in}}%
\pgfpathclose%
\pgfusepath{stroke,fill}%
\end{pgfscope}%
\begin{pgfscope}%
\pgfpathrectangle{\pgfqpoint{0.100000in}{0.212622in}}{\pgfqpoint{3.696000in}{3.696000in}}%
\pgfusepath{clip}%
\pgfsetbuttcap%
\pgfsetroundjoin%
\definecolor{currentfill}{rgb}{0.121569,0.466667,0.705882}%
\pgfsetfillcolor{currentfill}%
\pgfsetfillopacity{0.888579}%
\pgfsetlinewidth{1.003750pt}%
\definecolor{currentstroke}{rgb}{0.121569,0.466667,0.705882}%
\pgfsetstrokecolor{currentstroke}%
\pgfsetstrokeopacity{0.888579}%
\pgfsetdash{}{0pt}%
\pgfpathmoveto{\pgfqpoint{2.531160in}{1.237428in}}%
\pgfpathcurveto{\pgfqpoint{2.539396in}{1.237428in}}{\pgfqpoint{2.547296in}{1.240701in}}{\pgfqpoint{2.553120in}{1.246525in}}%
\pgfpathcurveto{\pgfqpoint{2.558944in}{1.252349in}}{\pgfqpoint{2.562216in}{1.260249in}}{\pgfqpoint{2.562216in}{1.268485in}}%
\pgfpathcurveto{\pgfqpoint{2.562216in}{1.276721in}}{\pgfqpoint{2.558944in}{1.284621in}}{\pgfqpoint{2.553120in}{1.290445in}}%
\pgfpathcurveto{\pgfqpoint{2.547296in}{1.296269in}}{\pgfqpoint{2.539396in}{1.299541in}}{\pgfqpoint{2.531160in}{1.299541in}}%
\pgfpathcurveto{\pgfqpoint{2.522923in}{1.299541in}}{\pgfqpoint{2.515023in}{1.296269in}}{\pgfqpoint{2.509199in}{1.290445in}}%
\pgfpathcurveto{\pgfqpoint{2.503376in}{1.284621in}}{\pgfqpoint{2.500103in}{1.276721in}}{\pgfqpoint{2.500103in}{1.268485in}}%
\pgfpathcurveto{\pgfqpoint{2.500103in}{1.260249in}}{\pgfqpoint{2.503376in}{1.252349in}}{\pgfqpoint{2.509199in}{1.246525in}}%
\pgfpathcurveto{\pgfqpoint{2.515023in}{1.240701in}}{\pgfqpoint{2.522923in}{1.237428in}}{\pgfqpoint{2.531160in}{1.237428in}}%
\pgfpathclose%
\pgfusepath{stroke,fill}%
\end{pgfscope}%
\begin{pgfscope}%
\pgfpathrectangle{\pgfqpoint{0.100000in}{0.212622in}}{\pgfqpoint{3.696000in}{3.696000in}}%
\pgfusepath{clip}%
\pgfsetbuttcap%
\pgfsetroundjoin%
\definecolor{currentfill}{rgb}{0.121569,0.466667,0.705882}%
\pgfsetfillcolor{currentfill}%
\pgfsetfillopacity{0.900127}%
\pgfsetlinewidth{1.003750pt}%
\definecolor{currentstroke}{rgb}{0.121569,0.466667,0.705882}%
\pgfsetstrokecolor{currentstroke}%
\pgfsetstrokeopacity{0.900127}%
\pgfsetdash{}{0pt}%
\pgfpathmoveto{\pgfqpoint{2.569476in}{1.228303in}}%
\pgfpathcurveto{\pgfqpoint{2.577712in}{1.228303in}}{\pgfqpoint{2.585612in}{1.231576in}}{\pgfqpoint{2.591436in}{1.237399in}}%
\pgfpathcurveto{\pgfqpoint{2.597260in}{1.243223in}}{\pgfqpoint{2.600532in}{1.251123in}}{\pgfqpoint{2.600532in}{1.259360in}}%
\pgfpathcurveto{\pgfqpoint{2.600532in}{1.267596in}}{\pgfqpoint{2.597260in}{1.275496in}}{\pgfqpoint{2.591436in}{1.281320in}}%
\pgfpathcurveto{\pgfqpoint{2.585612in}{1.287144in}}{\pgfqpoint{2.577712in}{1.290416in}}{\pgfqpoint{2.569476in}{1.290416in}}%
\pgfpathcurveto{\pgfqpoint{2.561239in}{1.290416in}}{\pgfqpoint{2.553339in}{1.287144in}}{\pgfqpoint{2.547515in}{1.281320in}}%
\pgfpathcurveto{\pgfqpoint{2.541691in}{1.275496in}}{\pgfqpoint{2.538419in}{1.267596in}}{\pgfqpoint{2.538419in}{1.259360in}}%
\pgfpathcurveto{\pgfqpoint{2.538419in}{1.251123in}}{\pgfqpoint{2.541691in}{1.243223in}}{\pgfqpoint{2.547515in}{1.237399in}}%
\pgfpathcurveto{\pgfqpoint{2.553339in}{1.231576in}}{\pgfqpoint{2.561239in}{1.228303in}}{\pgfqpoint{2.569476in}{1.228303in}}%
\pgfpathclose%
\pgfusepath{stroke,fill}%
\end{pgfscope}%
\begin{pgfscope}%
\pgfpathrectangle{\pgfqpoint{0.100000in}{0.212622in}}{\pgfqpoint{3.696000in}{3.696000in}}%
\pgfusepath{clip}%
\pgfsetbuttcap%
\pgfsetroundjoin%
\definecolor{currentfill}{rgb}{0.121569,0.466667,0.705882}%
\pgfsetfillcolor{currentfill}%
\pgfsetfillopacity{0.912738}%
\pgfsetlinewidth{1.003750pt}%
\definecolor{currentstroke}{rgb}{0.121569,0.466667,0.705882}%
\pgfsetstrokecolor{currentstroke}%
\pgfsetstrokeopacity{0.912738}%
\pgfsetdash{}{0pt}%
\pgfpathmoveto{\pgfqpoint{2.611069in}{1.217959in}}%
\pgfpathcurveto{\pgfqpoint{2.619305in}{1.217959in}}{\pgfqpoint{2.627205in}{1.221231in}}{\pgfqpoint{2.633029in}{1.227055in}}%
\pgfpathcurveto{\pgfqpoint{2.638853in}{1.232879in}}{\pgfqpoint{2.642125in}{1.240779in}}{\pgfqpoint{2.642125in}{1.249015in}}%
\pgfpathcurveto{\pgfqpoint{2.642125in}{1.257252in}}{\pgfqpoint{2.638853in}{1.265152in}}{\pgfqpoint{2.633029in}{1.270976in}}%
\pgfpathcurveto{\pgfqpoint{2.627205in}{1.276800in}}{\pgfqpoint{2.619305in}{1.280072in}}{\pgfqpoint{2.611069in}{1.280072in}}%
\pgfpathcurveto{\pgfqpoint{2.602832in}{1.280072in}}{\pgfqpoint{2.594932in}{1.276800in}}{\pgfqpoint{2.589108in}{1.270976in}}%
\pgfpathcurveto{\pgfqpoint{2.583284in}{1.265152in}}{\pgfqpoint{2.580012in}{1.257252in}}{\pgfqpoint{2.580012in}{1.249015in}}%
\pgfpathcurveto{\pgfqpoint{2.580012in}{1.240779in}}{\pgfqpoint{2.583284in}{1.232879in}}{\pgfqpoint{2.589108in}{1.227055in}}%
\pgfpathcurveto{\pgfqpoint{2.594932in}{1.221231in}}{\pgfqpoint{2.602832in}{1.217959in}}{\pgfqpoint{2.611069in}{1.217959in}}%
\pgfpathclose%
\pgfusepath{stroke,fill}%
\end{pgfscope}%
\begin{pgfscope}%
\pgfpathrectangle{\pgfqpoint{0.100000in}{0.212622in}}{\pgfqpoint{3.696000in}{3.696000in}}%
\pgfusepath{clip}%
\pgfsetbuttcap%
\pgfsetroundjoin%
\definecolor{currentfill}{rgb}{0.121569,0.466667,0.705882}%
\pgfsetfillcolor{currentfill}%
\pgfsetfillopacity{0.926057}%
\pgfsetlinewidth{1.003750pt}%
\definecolor{currentstroke}{rgb}{0.121569,0.466667,0.705882}%
\pgfsetstrokecolor{currentstroke}%
\pgfsetstrokeopacity{0.926057}%
\pgfsetdash{}{0pt}%
\pgfpathmoveto{\pgfqpoint{2.655475in}{1.207401in}}%
\pgfpathcurveto{\pgfqpoint{2.663711in}{1.207401in}}{\pgfqpoint{2.671611in}{1.210674in}}{\pgfqpoint{2.677435in}{1.216498in}}%
\pgfpathcurveto{\pgfqpoint{2.683259in}{1.222321in}}{\pgfqpoint{2.686531in}{1.230222in}}{\pgfqpoint{2.686531in}{1.238458in}}%
\pgfpathcurveto{\pgfqpoint{2.686531in}{1.246694in}}{\pgfqpoint{2.683259in}{1.254594in}}{\pgfqpoint{2.677435in}{1.260418in}}%
\pgfpathcurveto{\pgfqpoint{2.671611in}{1.266242in}}{\pgfqpoint{2.663711in}{1.269514in}}{\pgfqpoint{2.655475in}{1.269514in}}%
\pgfpathcurveto{\pgfqpoint{2.647238in}{1.269514in}}{\pgfqpoint{2.639338in}{1.266242in}}{\pgfqpoint{2.633514in}{1.260418in}}%
\pgfpathcurveto{\pgfqpoint{2.627690in}{1.254594in}}{\pgfqpoint{2.624418in}{1.246694in}}{\pgfqpoint{2.624418in}{1.238458in}}%
\pgfpathcurveto{\pgfqpoint{2.624418in}{1.230222in}}{\pgfqpoint{2.627690in}{1.222321in}}{\pgfqpoint{2.633514in}{1.216498in}}%
\pgfpathcurveto{\pgfqpoint{2.639338in}{1.210674in}}{\pgfqpoint{2.647238in}{1.207401in}}{\pgfqpoint{2.655475in}{1.207401in}}%
\pgfpathclose%
\pgfusepath{stroke,fill}%
\end{pgfscope}%
\begin{pgfscope}%
\pgfpathrectangle{\pgfqpoint{0.100000in}{0.212622in}}{\pgfqpoint{3.696000in}{3.696000in}}%
\pgfusepath{clip}%
\pgfsetbuttcap%
\pgfsetroundjoin%
\definecolor{currentfill}{rgb}{0.121569,0.466667,0.705882}%
\pgfsetfillcolor{currentfill}%
\pgfsetfillopacity{0.941125}%
\pgfsetlinewidth{1.003750pt}%
\definecolor{currentstroke}{rgb}{0.121569,0.466667,0.705882}%
\pgfsetstrokecolor{currentstroke}%
\pgfsetstrokeopacity{0.941125}%
\pgfsetdash{}{0pt}%
\pgfpathmoveto{\pgfqpoint{2.701152in}{1.192342in}}%
\pgfpathcurveto{\pgfqpoint{2.709389in}{1.192342in}}{\pgfqpoint{2.717289in}{1.195614in}}{\pgfqpoint{2.723113in}{1.201438in}}%
\pgfpathcurveto{\pgfqpoint{2.728937in}{1.207262in}}{\pgfqpoint{2.732209in}{1.215162in}}{\pgfqpoint{2.732209in}{1.223398in}}%
\pgfpathcurveto{\pgfqpoint{2.732209in}{1.231635in}}{\pgfqpoint{2.728937in}{1.239535in}}{\pgfqpoint{2.723113in}{1.245359in}}%
\pgfpathcurveto{\pgfqpoint{2.717289in}{1.251183in}}{\pgfqpoint{2.709389in}{1.254455in}}{\pgfqpoint{2.701152in}{1.254455in}}%
\pgfpathcurveto{\pgfqpoint{2.692916in}{1.254455in}}{\pgfqpoint{2.685016in}{1.251183in}}{\pgfqpoint{2.679192in}{1.245359in}}%
\pgfpathcurveto{\pgfqpoint{2.673368in}{1.239535in}}{\pgfqpoint{2.670096in}{1.231635in}}{\pgfqpoint{2.670096in}{1.223398in}}%
\pgfpathcurveto{\pgfqpoint{2.670096in}{1.215162in}}{\pgfqpoint{2.673368in}{1.207262in}}{\pgfqpoint{2.679192in}{1.201438in}}%
\pgfpathcurveto{\pgfqpoint{2.685016in}{1.195614in}}{\pgfqpoint{2.692916in}{1.192342in}}{\pgfqpoint{2.701152in}{1.192342in}}%
\pgfpathclose%
\pgfusepath{stroke,fill}%
\end{pgfscope}%
\begin{pgfscope}%
\pgfpathrectangle{\pgfqpoint{0.100000in}{0.212622in}}{\pgfqpoint{3.696000in}{3.696000in}}%
\pgfusepath{clip}%
\pgfsetbuttcap%
\pgfsetroundjoin%
\definecolor{currentfill}{rgb}{0.121569,0.466667,0.705882}%
\pgfsetfillcolor{currentfill}%
\pgfsetfillopacity{0.957163}%
\pgfsetlinewidth{1.003750pt}%
\definecolor{currentstroke}{rgb}{0.121569,0.466667,0.705882}%
\pgfsetstrokecolor{currentstroke}%
\pgfsetstrokeopacity{0.957163}%
\pgfsetdash{}{0pt}%
\pgfpathmoveto{\pgfqpoint{2.750282in}{1.176494in}}%
\pgfpathcurveto{\pgfqpoint{2.758518in}{1.176494in}}{\pgfqpoint{2.766418in}{1.179766in}}{\pgfqpoint{2.772242in}{1.185590in}}%
\pgfpathcurveto{\pgfqpoint{2.778066in}{1.191414in}}{\pgfqpoint{2.781339in}{1.199314in}}{\pgfqpoint{2.781339in}{1.207551in}}%
\pgfpathcurveto{\pgfqpoint{2.781339in}{1.215787in}}{\pgfqpoint{2.778066in}{1.223687in}}{\pgfqpoint{2.772242in}{1.229511in}}%
\pgfpathcurveto{\pgfqpoint{2.766418in}{1.235335in}}{\pgfqpoint{2.758518in}{1.238607in}}{\pgfqpoint{2.750282in}{1.238607in}}%
\pgfpathcurveto{\pgfqpoint{2.742046in}{1.238607in}}{\pgfqpoint{2.734146in}{1.235335in}}{\pgfqpoint{2.728322in}{1.229511in}}%
\pgfpathcurveto{\pgfqpoint{2.722498in}{1.223687in}}{\pgfqpoint{2.719226in}{1.215787in}}{\pgfqpoint{2.719226in}{1.207551in}}%
\pgfpathcurveto{\pgfqpoint{2.719226in}{1.199314in}}{\pgfqpoint{2.722498in}{1.191414in}}{\pgfqpoint{2.728322in}{1.185590in}}%
\pgfpathcurveto{\pgfqpoint{2.734146in}{1.179766in}}{\pgfqpoint{2.742046in}{1.176494in}}{\pgfqpoint{2.750282in}{1.176494in}}%
\pgfpathclose%
\pgfusepath{stroke,fill}%
\end{pgfscope}%
\begin{pgfscope}%
\pgfpathrectangle{\pgfqpoint{0.100000in}{0.212622in}}{\pgfqpoint{3.696000in}{3.696000in}}%
\pgfusepath{clip}%
\pgfsetbuttcap%
\pgfsetroundjoin%
\definecolor{currentfill}{rgb}{0.121569,0.466667,0.705882}%
\pgfsetfillcolor{currentfill}%
\pgfsetfillopacity{0.975059}%
\pgfsetlinewidth{1.003750pt}%
\definecolor{currentstroke}{rgb}{0.121569,0.466667,0.705882}%
\pgfsetstrokecolor{currentstroke}%
\pgfsetstrokeopacity{0.975059}%
\pgfsetdash{}{0pt}%
\pgfpathmoveto{\pgfqpoint{2.802347in}{1.157676in}}%
\pgfpathcurveto{\pgfqpoint{2.810583in}{1.157676in}}{\pgfqpoint{2.818483in}{1.160948in}}{\pgfqpoint{2.824307in}{1.166772in}}%
\pgfpathcurveto{\pgfqpoint{2.830131in}{1.172596in}}{\pgfqpoint{2.833403in}{1.180496in}}{\pgfqpoint{2.833403in}{1.188732in}}%
\pgfpathcurveto{\pgfqpoint{2.833403in}{1.196969in}}{\pgfqpoint{2.830131in}{1.204869in}}{\pgfqpoint{2.824307in}{1.210693in}}%
\pgfpathcurveto{\pgfqpoint{2.818483in}{1.216517in}}{\pgfqpoint{2.810583in}{1.219789in}}{\pgfqpoint{2.802347in}{1.219789in}}%
\pgfpathcurveto{\pgfqpoint{2.794110in}{1.219789in}}{\pgfqpoint{2.786210in}{1.216517in}}{\pgfqpoint{2.780386in}{1.210693in}}%
\pgfpathcurveto{\pgfqpoint{2.774563in}{1.204869in}}{\pgfqpoint{2.771290in}{1.196969in}}{\pgfqpoint{2.771290in}{1.188732in}}%
\pgfpathcurveto{\pgfqpoint{2.771290in}{1.180496in}}{\pgfqpoint{2.774563in}{1.172596in}}{\pgfqpoint{2.780386in}{1.166772in}}%
\pgfpathcurveto{\pgfqpoint{2.786210in}{1.160948in}}{\pgfqpoint{2.794110in}{1.157676in}}{\pgfqpoint{2.802347in}{1.157676in}}%
\pgfpathclose%
\pgfusepath{stroke,fill}%
\end{pgfscope}%
\begin{pgfscope}%
\pgfpathrectangle{\pgfqpoint{0.100000in}{0.212622in}}{\pgfqpoint{3.696000in}{3.696000in}}%
\pgfusepath{clip}%
\pgfsetbuttcap%
\pgfsetroundjoin%
\definecolor{currentfill}{rgb}{0.121569,0.466667,0.705882}%
\pgfsetfillcolor{currentfill}%
\pgfsetfillopacity{0.984813}%
\pgfsetlinewidth{1.003750pt}%
\definecolor{currentstroke}{rgb}{0.121569,0.466667,0.705882}%
\pgfsetstrokecolor{currentstroke}%
\pgfsetstrokeopacity{0.984813}%
\pgfsetdash{}{0pt}%
\pgfpathmoveto{\pgfqpoint{2.831443in}{1.148412in}}%
\pgfpathcurveto{\pgfqpoint{2.839679in}{1.148412in}}{\pgfqpoint{2.847579in}{1.151685in}}{\pgfqpoint{2.853403in}{1.157509in}}%
\pgfpathcurveto{\pgfqpoint{2.859227in}{1.163333in}}{\pgfqpoint{2.862499in}{1.171233in}}{\pgfqpoint{2.862499in}{1.179469in}}%
\pgfpathcurveto{\pgfqpoint{2.862499in}{1.187705in}}{\pgfqpoint{2.859227in}{1.195605in}}{\pgfqpoint{2.853403in}{1.201429in}}%
\pgfpathcurveto{\pgfqpoint{2.847579in}{1.207253in}}{\pgfqpoint{2.839679in}{1.210525in}}{\pgfqpoint{2.831443in}{1.210525in}}%
\pgfpathcurveto{\pgfqpoint{2.823207in}{1.210525in}}{\pgfqpoint{2.815306in}{1.207253in}}{\pgfqpoint{2.809483in}{1.201429in}}%
\pgfpathcurveto{\pgfqpoint{2.803659in}{1.195605in}}{\pgfqpoint{2.800386in}{1.187705in}}{\pgfqpoint{2.800386in}{1.179469in}}%
\pgfpathcurveto{\pgfqpoint{2.800386in}{1.171233in}}{\pgfqpoint{2.803659in}{1.163333in}}{\pgfqpoint{2.809483in}{1.157509in}}%
\pgfpathcurveto{\pgfqpoint{2.815306in}{1.151685in}}{\pgfqpoint{2.823207in}{1.148412in}}{\pgfqpoint{2.831443in}{1.148412in}}%
\pgfpathclose%
\pgfusepath{stroke,fill}%
\end{pgfscope}%
\begin{pgfscope}%
\pgfpathrectangle{\pgfqpoint{0.100000in}{0.212622in}}{\pgfqpoint{3.696000in}{3.696000in}}%
\pgfusepath{clip}%
\pgfsetbuttcap%
\pgfsetroundjoin%
\definecolor{currentfill}{rgb}{0.121569,0.466667,0.705882}%
\pgfsetfillcolor{currentfill}%
\pgfsetfillopacity{0.990571}%
\pgfsetlinewidth{1.003750pt}%
\definecolor{currentstroke}{rgb}{0.121569,0.466667,0.705882}%
\pgfsetstrokecolor{currentstroke}%
\pgfsetstrokeopacity{0.990571}%
\pgfsetdash{}{0pt}%
\pgfpathmoveto{\pgfqpoint{2.846719in}{1.141667in}}%
\pgfpathcurveto{\pgfqpoint{2.854956in}{1.141667in}}{\pgfqpoint{2.862856in}{1.144939in}}{\pgfqpoint{2.868680in}{1.150763in}}%
\pgfpathcurveto{\pgfqpoint{2.874503in}{1.156587in}}{\pgfqpoint{2.877776in}{1.164487in}}{\pgfqpoint{2.877776in}{1.172724in}}%
\pgfpathcurveto{\pgfqpoint{2.877776in}{1.180960in}}{\pgfqpoint{2.874503in}{1.188860in}}{\pgfqpoint{2.868680in}{1.194684in}}%
\pgfpathcurveto{\pgfqpoint{2.862856in}{1.200508in}}{\pgfqpoint{2.854956in}{1.203780in}}{\pgfqpoint{2.846719in}{1.203780in}}%
\pgfpathcurveto{\pgfqpoint{2.838483in}{1.203780in}}{\pgfqpoint{2.830583in}{1.200508in}}{\pgfqpoint{2.824759in}{1.194684in}}%
\pgfpathcurveto{\pgfqpoint{2.818935in}{1.188860in}}{\pgfqpoint{2.815663in}{1.180960in}}{\pgfqpoint{2.815663in}{1.172724in}}%
\pgfpathcurveto{\pgfqpoint{2.815663in}{1.164487in}}{\pgfqpoint{2.818935in}{1.156587in}}{\pgfqpoint{2.824759in}{1.150763in}}%
\pgfpathcurveto{\pgfqpoint{2.830583in}{1.144939in}}{\pgfqpoint{2.838483in}{1.141667in}}{\pgfqpoint{2.846719in}{1.141667in}}%
\pgfpathclose%
\pgfusepath{stroke,fill}%
\end{pgfscope}%
\begin{pgfscope}%
\pgfpathrectangle{\pgfqpoint{0.100000in}{0.212622in}}{\pgfqpoint{3.696000in}{3.696000in}}%
\pgfusepath{clip}%
\pgfsetbuttcap%
\pgfsetroundjoin%
\definecolor{currentfill}{rgb}{0.121569,0.466667,0.705882}%
\pgfsetfillcolor{currentfill}%
\pgfsetfillopacity{0.993691}%
\pgfsetlinewidth{1.003750pt}%
\definecolor{currentstroke}{rgb}{0.121569,0.466667,0.705882}%
\pgfsetstrokecolor{currentstroke}%
\pgfsetstrokeopacity{0.993691}%
\pgfsetdash{}{0pt}%
\pgfpathmoveto{\pgfqpoint{2.855264in}{1.138371in}}%
\pgfpathcurveto{\pgfqpoint{2.863501in}{1.138371in}}{\pgfqpoint{2.871401in}{1.141644in}}{\pgfqpoint{2.877225in}{1.147467in}}%
\pgfpathcurveto{\pgfqpoint{2.883048in}{1.153291in}}{\pgfqpoint{2.886321in}{1.161191in}}{\pgfqpoint{2.886321in}{1.169428in}}%
\pgfpathcurveto{\pgfqpoint{2.886321in}{1.177664in}}{\pgfqpoint{2.883048in}{1.185564in}}{\pgfqpoint{2.877225in}{1.191388in}}%
\pgfpathcurveto{\pgfqpoint{2.871401in}{1.197212in}}{\pgfqpoint{2.863501in}{1.200484in}}{\pgfqpoint{2.855264in}{1.200484in}}%
\pgfpathcurveto{\pgfqpoint{2.847028in}{1.200484in}}{\pgfqpoint{2.839128in}{1.197212in}}{\pgfqpoint{2.833304in}{1.191388in}}%
\pgfpathcurveto{\pgfqpoint{2.827480in}{1.185564in}}{\pgfqpoint{2.824208in}{1.177664in}}{\pgfqpoint{2.824208in}{1.169428in}}%
\pgfpathcurveto{\pgfqpoint{2.824208in}{1.161191in}}{\pgfqpoint{2.827480in}{1.153291in}}{\pgfqpoint{2.833304in}{1.147467in}}%
\pgfpathcurveto{\pgfqpoint{2.839128in}{1.141644in}}{\pgfqpoint{2.847028in}{1.138371in}}{\pgfqpoint{2.855264in}{1.138371in}}%
\pgfpathclose%
\pgfusepath{stroke,fill}%
\end{pgfscope}%
\begin{pgfscope}%
\pgfpathrectangle{\pgfqpoint{0.100000in}{0.212622in}}{\pgfqpoint{3.696000in}{3.696000in}}%
\pgfusepath{clip}%
\pgfsetbuttcap%
\pgfsetroundjoin%
\definecolor{currentfill}{rgb}{0.121569,0.466667,0.705882}%
\pgfsetfillcolor{currentfill}%
\pgfsetfillopacity{0.995507}%
\pgfsetlinewidth{1.003750pt}%
\definecolor{currentstroke}{rgb}{0.121569,0.466667,0.705882}%
\pgfsetstrokecolor{currentstroke}%
\pgfsetstrokeopacity{0.995507}%
\pgfsetdash{}{0pt}%
\pgfpathmoveto{\pgfqpoint{2.859767in}{1.136165in}}%
\pgfpathcurveto{\pgfqpoint{2.868004in}{1.136165in}}{\pgfqpoint{2.875904in}{1.139437in}}{\pgfqpoint{2.881728in}{1.145261in}}%
\pgfpathcurveto{\pgfqpoint{2.887551in}{1.151085in}}{\pgfqpoint{2.890824in}{1.158985in}}{\pgfqpoint{2.890824in}{1.167221in}}%
\pgfpathcurveto{\pgfqpoint{2.890824in}{1.175457in}}{\pgfqpoint{2.887551in}{1.183357in}}{\pgfqpoint{2.881728in}{1.189181in}}%
\pgfpathcurveto{\pgfqpoint{2.875904in}{1.195005in}}{\pgfqpoint{2.868004in}{1.198278in}}{\pgfqpoint{2.859767in}{1.198278in}}%
\pgfpathcurveto{\pgfqpoint{2.851531in}{1.198278in}}{\pgfqpoint{2.843631in}{1.195005in}}{\pgfqpoint{2.837807in}{1.189181in}}%
\pgfpathcurveto{\pgfqpoint{2.831983in}{1.183357in}}{\pgfqpoint{2.828711in}{1.175457in}}{\pgfqpoint{2.828711in}{1.167221in}}%
\pgfpathcurveto{\pgfqpoint{2.828711in}{1.158985in}}{\pgfqpoint{2.831983in}{1.151085in}}{\pgfqpoint{2.837807in}{1.145261in}}%
\pgfpathcurveto{\pgfqpoint{2.843631in}{1.139437in}}{\pgfqpoint{2.851531in}{1.136165in}}{\pgfqpoint{2.859767in}{1.136165in}}%
\pgfpathclose%
\pgfusepath{stroke,fill}%
\end{pgfscope}%
\begin{pgfscope}%
\pgfpathrectangle{\pgfqpoint{0.100000in}{0.212622in}}{\pgfqpoint{3.696000in}{3.696000in}}%
\pgfusepath{clip}%
\pgfsetbuttcap%
\pgfsetroundjoin%
\definecolor{currentfill}{rgb}{0.121569,0.466667,0.705882}%
\pgfsetfillcolor{currentfill}%
\pgfsetfillopacity{0.996460}%
\pgfsetlinewidth{1.003750pt}%
\definecolor{currentstroke}{rgb}{0.121569,0.466667,0.705882}%
\pgfsetstrokecolor{currentstroke}%
\pgfsetstrokeopacity{0.996460}%
\pgfsetdash{}{0pt}%
\pgfpathmoveto{\pgfqpoint{2.862368in}{1.135206in}}%
\pgfpathcurveto{\pgfqpoint{2.870604in}{1.135206in}}{\pgfqpoint{2.878504in}{1.138478in}}{\pgfqpoint{2.884328in}{1.144302in}}%
\pgfpathcurveto{\pgfqpoint{2.890152in}{1.150126in}}{\pgfqpoint{2.893424in}{1.158026in}}{\pgfqpoint{2.893424in}{1.166262in}}%
\pgfpathcurveto{\pgfqpoint{2.893424in}{1.174499in}}{\pgfqpoint{2.890152in}{1.182399in}}{\pgfqpoint{2.884328in}{1.188223in}}%
\pgfpathcurveto{\pgfqpoint{2.878504in}{1.194046in}}{\pgfqpoint{2.870604in}{1.197319in}}{\pgfqpoint{2.862368in}{1.197319in}}%
\pgfpathcurveto{\pgfqpoint{2.854131in}{1.197319in}}{\pgfqpoint{2.846231in}{1.194046in}}{\pgfqpoint{2.840407in}{1.188223in}}%
\pgfpathcurveto{\pgfqpoint{2.834583in}{1.182399in}}{\pgfqpoint{2.831311in}{1.174499in}}{\pgfqpoint{2.831311in}{1.166262in}}%
\pgfpathcurveto{\pgfqpoint{2.831311in}{1.158026in}}{\pgfqpoint{2.834583in}{1.150126in}}{\pgfqpoint{2.840407in}{1.144302in}}%
\pgfpathcurveto{\pgfqpoint{2.846231in}{1.138478in}}{\pgfqpoint{2.854131in}{1.135206in}}{\pgfqpoint{2.862368in}{1.135206in}}%
\pgfpathclose%
\pgfusepath{stroke,fill}%
\end{pgfscope}%
\begin{pgfscope}%
\pgfpathrectangle{\pgfqpoint{0.100000in}{0.212622in}}{\pgfqpoint{3.696000in}{3.696000in}}%
\pgfusepath{clip}%
\pgfsetbuttcap%
\pgfsetroundjoin%
\definecolor{currentfill}{rgb}{0.121569,0.466667,0.705882}%
\pgfsetfillcolor{currentfill}%
\pgfsetfillopacity{0.997006}%
\pgfsetlinewidth{1.003750pt}%
\definecolor{currentstroke}{rgb}{0.121569,0.466667,0.705882}%
\pgfsetstrokecolor{currentstroke}%
\pgfsetstrokeopacity{0.997006}%
\pgfsetdash{}{0pt}%
\pgfpathmoveto{\pgfqpoint{2.863749in}{1.134580in}}%
\pgfpathcurveto{\pgfqpoint{2.871985in}{1.134580in}}{\pgfqpoint{2.879885in}{1.137852in}}{\pgfqpoint{2.885709in}{1.143676in}}%
\pgfpathcurveto{\pgfqpoint{2.891533in}{1.149500in}}{\pgfqpoint{2.894805in}{1.157400in}}{\pgfqpoint{2.894805in}{1.165637in}}%
\pgfpathcurveto{\pgfqpoint{2.894805in}{1.173873in}}{\pgfqpoint{2.891533in}{1.181773in}}{\pgfqpoint{2.885709in}{1.187597in}}%
\pgfpathcurveto{\pgfqpoint{2.879885in}{1.193421in}}{\pgfqpoint{2.871985in}{1.196693in}}{\pgfqpoint{2.863749in}{1.196693in}}%
\pgfpathcurveto{\pgfqpoint{2.855513in}{1.196693in}}{\pgfqpoint{2.847612in}{1.193421in}}{\pgfqpoint{2.841789in}{1.187597in}}%
\pgfpathcurveto{\pgfqpoint{2.835965in}{1.181773in}}{\pgfqpoint{2.832692in}{1.173873in}}{\pgfqpoint{2.832692in}{1.165637in}}%
\pgfpathcurveto{\pgfqpoint{2.832692in}{1.157400in}}{\pgfqpoint{2.835965in}{1.149500in}}{\pgfqpoint{2.841789in}{1.143676in}}%
\pgfpathcurveto{\pgfqpoint{2.847612in}{1.137852in}}{\pgfqpoint{2.855513in}{1.134580in}}{\pgfqpoint{2.863749in}{1.134580in}}%
\pgfpathclose%
\pgfusepath{stroke,fill}%
\end{pgfscope}%
\begin{pgfscope}%
\pgfpathrectangle{\pgfqpoint{0.100000in}{0.212622in}}{\pgfqpoint{3.696000in}{3.696000in}}%
\pgfusepath{clip}%
\pgfsetbuttcap%
\pgfsetroundjoin%
\definecolor{currentfill}{rgb}{0.121569,0.466667,0.705882}%
\pgfsetfillcolor{currentfill}%
\pgfsetfillopacity{0.997288}%
\pgfsetlinewidth{1.003750pt}%
\definecolor{currentstroke}{rgb}{0.121569,0.466667,0.705882}%
\pgfsetstrokecolor{currentstroke}%
\pgfsetstrokeopacity{0.997288}%
\pgfsetdash{}{0pt}%
\pgfpathmoveto{\pgfqpoint{2.864547in}{1.134321in}}%
\pgfpathcurveto{\pgfqpoint{2.872784in}{1.134321in}}{\pgfqpoint{2.880684in}{1.137593in}}{\pgfqpoint{2.886508in}{1.143417in}}%
\pgfpathcurveto{\pgfqpoint{2.892332in}{1.149241in}}{\pgfqpoint{2.895604in}{1.157141in}}{\pgfqpoint{2.895604in}{1.165377in}}%
\pgfpathcurveto{\pgfqpoint{2.895604in}{1.173614in}}{\pgfqpoint{2.892332in}{1.181514in}}{\pgfqpoint{2.886508in}{1.187338in}}%
\pgfpathcurveto{\pgfqpoint{2.880684in}{1.193162in}}{\pgfqpoint{2.872784in}{1.196434in}}{\pgfqpoint{2.864547in}{1.196434in}}%
\pgfpathcurveto{\pgfqpoint{2.856311in}{1.196434in}}{\pgfqpoint{2.848411in}{1.193162in}}{\pgfqpoint{2.842587in}{1.187338in}}%
\pgfpathcurveto{\pgfqpoint{2.836763in}{1.181514in}}{\pgfqpoint{2.833491in}{1.173614in}}{\pgfqpoint{2.833491in}{1.165377in}}%
\pgfpathcurveto{\pgfqpoint{2.833491in}{1.157141in}}{\pgfqpoint{2.836763in}{1.149241in}}{\pgfqpoint{2.842587in}{1.143417in}}%
\pgfpathcurveto{\pgfqpoint{2.848411in}{1.137593in}}{\pgfqpoint{2.856311in}{1.134321in}}{\pgfqpoint{2.864547in}{1.134321in}}%
\pgfpathclose%
\pgfusepath{stroke,fill}%
\end{pgfscope}%
\begin{pgfscope}%
\pgfpathrectangle{\pgfqpoint{0.100000in}{0.212622in}}{\pgfqpoint{3.696000in}{3.696000in}}%
\pgfusepath{clip}%
\pgfsetbuttcap%
\pgfsetroundjoin%
\definecolor{currentfill}{rgb}{0.121569,0.466667,0.705882}%
\pgfsetfillcolor{currentfill}%
\pgfsetfillopacity{0.997450}%
\pgfsetlinewidth{1.003750pt}%
\definecolor{currentstroke}{rgb}{0.121569,0.466667,0.705882}%
\pgfsetstrokecolor{currentstroke}%
\pgfsetstrokeopacity{0.997450}%
\pgfsetdash{}{0pt}%
\pgfpathmoveto{\pgfqpoint{2.864974in}{1.134149in}}%
\pgfpathcurveto{\pgfqpoint{2.873211in}{1.134149in}}{\pgfqpoint{2.881111in}{1.137421in}}{\pgfqpoint{2.886935in}{1.143245in}}%
\pgfpathcurveto{\pgfqpoint{2.892759in}{1.149069in}}{\pgfqpoint{2.896031in}{1.156969in}}{\pgfqpoint{2.896031in}{1.165205in}}%
\pgfpathcurveto{\pgfqpoint{2.896031in}{1.173441in}}{\pgfqpoint{2.892759in}{1.181341in}}{\pgfqpoint{2.886935in}{1.187165in}}%
\pgfpathcurveto{\pgfqpoint{2.881111in}{1.192989in}}{\pgfqpoint{2.873211in}{1.196262in}}{\pgfqpoint{2.864974in}{1.196262in}}%
\pgfpathcurveto{\pgfqpoint{2.856738in}{1.196262in}}{\pgfqpoint{2.848838in}{1.192989in}}{\pgfqpoint{2.843014in}{1.187165in}}%
\pgfpathcurveto{\pgfqpoint{2.837190in}{1.181341in}}{\pgfqpoint{2.833918in}{1.173441in}}{\pgfqpoint{2.833918in}{1.165205in}}%
\pgfpathcurveto{\pgfqpoint{2.833918in}{1.156969in}}{\pgfqpoint{2.837190in}{1.149069in}}{\pgfqpoint{2.843014in}{1.143245in}}%
\pgfpathcurveto{\pgfqpoint{2.848838in}{1.137421in}}{\pgfqpoint{2.856738in}{1.134149in}}{\pgfqpoint{2.864974in}{1.134149in}}%
\pgfpathclose%
\pgfusepath{stroke,fill}%
\end{pgfscope}%
\begin{pgfscope}%
\pgfpathrectangle{\pgfqpoint{0.100000in}{0.212622in}}{\pgfqpoint{3.696000in}{3.696000in}}%
\pgfusepath{clip}%
\pgfsetbuttcap%
\pgfsetroundjoin%
\definecolor{currentfill}{rgb}{0.121569,0.466667,0.705882}%
\pgfsetfillcolor{currentfill}%
\pgfsetfillopacity{0.997541}%
\pgfsetlinewidth{1.003750pt}%
\definecolor{currentstroke}{rgb}{0.121569,0.466667,0.705882}%
\pgfsetstrokecolor{currentstroke}%
\pgfsetstrokeopacity{0.997541}%
\pgfsetdash{}{0pt}%
\pgfpathmoveto{\pgfqpoint{2.865203in}{1.134040in}}%
\pgfpathcurveto{\pgfqpoint{2.873439in}{1.134040in}}{\pgfqpoint{2.881339in}{1.137313in}}{\pgfqpoint{2.887163in}{1.143137in}}%
\pgfpathcurveto{\pgfqpoint{2.892987in}{1.148961in}}{\pgfqpoint{2.896259in}{1.156861in}}{\pgfqpoint{2.896259in}{1.165097in}}%
\pgfpathcurveto{\pgfqpoint{2.896259in}{1.173333in}}{\pgfqpoint{2.892987in}{1.181233in}}{\pgfqpoint{2.887163in}{1.187057in}}%
\pgfpathcurveto{\pgfqpoint{2.881339in}{1.192881in}}{\pgfqpoint{2.873439in}{1.196153in}}{\pgfqpoint{2.865203in}{1.196153in}}%
\pgfpathcurveto{\pgfqpoint{2.856967in}{1.196153in}}{\pgfqpoint{2.849067in}{1.192881in}}{\pgfqpoint{2.843243in}{1.187057in}}%
\pgfpathcurveto{\pgfqpoint{2.837419in}{1.181233in}}{\pgfqpoint{2.834146in}{1.173333in}}{\pgfqpoint{2.834146in}{1.165097in}}%
\pgfpathcurveto{\pgfqpoint{2.834146in}{1.156861in}}{\pgfqpoint{2.837419in}{1.148961in}}{\pgfqpoint{2.843243in}{1.143137in}}%
\pgfpathcurveto{\pgfqpoint{2.849067in}{1.137313in}}{\pgfqpoint{2.856967in}{1.134040in}}{\pgfqpoint{2.865203in}{1.134040in}}%
\pgfpathclose%
\pgfusepath{stroke,fill}%
\end{pgfscope}%
\begin{pgfscope}%
\pgfpathrectangle{\pgfqpoint{0.100000in}{0.212622in}}{\pgfqpoint{3.696000in}{3.696000in}}%
\pgfusepath{clip}%
\pgfsetbuttcap%
\pgfsetroundjoin%
\definecolor{currentfill}{rgb}{0.121569,0.466667,0.705882}%
\pgfsetfillcolor{currentfill}%
\pgfsetfillopacity{0.997591}%
\pgfsetlinewidth{1.003750pt}%
\definecolor{currentstroke}{rgb}{0.121569,0.466667,0.705882}%
\pgfsetstrokecolor{currentstroke}%
\pgfsetstrokeopacity{0.997591}%
\pgfsetdash{}{0pt}%
\pgfpathmoveto{\pgfqpoint{2.865330in}{1.133985in}}%
\pgfpathcurveto{\pgfqpoint{2.873566in}{1.133985in}}{\pgfqpoint{2.881467in}{1.137258in}}{\pgfqpoint{2.887290in}{1.143082in}}%
\pgfpathcurveto{\pgfqpoint{2.893114in}{1.148906in}}{\pgfqpoint{2.896387in}{1.156806in}}{\pgfqpoint{2.896387in}{1.165042in}}%
\pgfpathcurveto{\pgfqpoint{2.896387in}{1.173278in}}{\pgfqpoint{2.893114in}{1.181178in}}{\pgfqpoint{2.887290in}{1.187002in}}%
\pgfpathcurveto{\pgfqpoint{2.881467in}{1.192826in}}{\pgfqpoint{2.873566in}{1.196098in}}{\pgfqpoint{2.865330in}{1.196098in}}%
\pgfpathcurveto{\pgfqpoint{2.857094in}{1.196098in}}{\pgfqpoint{2.849194in}{1.192826in}}{\pgfqpoint{2.843370in}{1.187002in}}%
\pgfpathcurveto{\pgfqpoint{2.837546in}{1.181178in}}{\pgfqpoint{2.834274in}{1.173278in}}{\pgfqpoint{2.834274in}{1.165042in}}%
\pgfpathcurveto{\pgfqpoint{2.834274in}{1.156806in}}{\pgfqpoint{2.837546in}{1.148906in}}{\pgfqpoint{2.843370in}{1.143082in}}%
\pgfpathcurveto{\pgfqpoint{2.849194in}{1.137258in}}{\pgfqpoint{2.857094in}{1.133985in}}{\pgfqpoint{2.865330in}{1.133985in}}%
\pgfpathclose%
\pgfusepath{stroke,fill}%
\end{pgfscope}%
\begin{pgfscope}%
\pgfpathrectangle{\pgfqpoint{0.100000in}{0.212622in}}{\pgfqpoint{3.696000in}{3.696000in}}%
\pgfusepath{clip}%
\pgfsetbuttcap%
\pgfsetroundjoin%
\definecolor{currentfill}{rgb}{0.121569,0.466667,0.705882}%
\pgfsetfillcolor{currentfill}%
\pgfsetfillopacity{0.997619}%
\pgfsetlinewidth{1.003750pt}%
\definecolor{currentstroke}{rgb}{0.121569,0.466667,0.705882}%
\pgfsetstrokecolor{currentstroke}%
\pgfsetstrokeopacity{0.997619}%
\pgfsetdash{}{0pt}%
\pgfpathmoveto{\pgfqpoint{2.865398in}{1.133951in}}%
\pgfpathcurveto{\pgfqpoint{2.873634in}{1.133951in}}{\pgfqpoint{2.881534in}{1.137223in}}{\pgfqpoint{2.887358in}{1.143047in}}%
\pgfpathcurveto{\pgfqpoint{2.893182in}{1.148871in}}{\pgfqpoint{2.896455in}{1.156771in}}{\pgfqpoint{2.896455in}{1.165008in}}%
\pgfpathcurveto{\pgfqpoint{2.896455in}{1.173244in}}{\pgfqpoint{2.893182in}{1.181144in}}{\pgfqpoint{2.887358in}{1.186968in}}%
\pgfpathcurveto{\pgfqpoint{2.881534in}{1.192792in}}{\pgfqpoint{2.873634in}{1.196064in}}{\pgfqpoint{2.865398in}{1.196064in}}%
\pgfpathcurveto{\pgfqpoint{2.857162in}{1.196064in}}{\pgfqpoint{2.849262in}{1.192792in}}{\pgfqpoint{2.843438in}{1.186968in}}%
\pgfpathcurveto{\pgfqpoint{2.837614in}{1.181144in}}{\pgfqpoint{2.834342in}{1.173244in}}{\pgfqpoint{2.834342in}{1.165008in}}%
\pgfpathcurveto{\pgfqpoint{2.834342in}{1.156771in}}{\pgfqpoint{2.837614in}{1.148871in}}{\pgfqpoint{2.843438in}{1.143047in}}%
\pgfpathcurveto{\pgfqpoint{2.849262in}{1.137223in}}{\pgfqpoint{2.857162in}{1.133951in}}{\pgfqpoint{2.865398in}{1.133951in}}%
\pgfpathclose%
\pgfusepath{stroke,fill}%
\end{pgfscope}%
\begin{pgfscope}%
\pgfpathrectangle{\pgfqpoint{0.100000in}{0.212622in}}{\pgfqpoint{3.696000in}{3.696000in}}%
\pgfusepath{clip}%
\pgfsetbuttcap%
\pgfsetroundjoin%
\definecolor{currentfill}{rgb}{0.121569,0.466667,0.705882}%
\pgfsetfillcolor{currentfill}%
\pgfsetfillopacity{0.997634}%
\pgfsetlinewidth{1.003750pt}%
\definecolor{currentstroke}{rgb}{0.121569,0.466667,0.705882}%
\pgfsetstrokecolor{currentstroke}%
\pgfsetstrokeopacity{0.997634}%
\pgfsetdash{}{0pt}%
\pgfpathmoveto{\pgfqpoint{2.865437in}{1.133936in}}%
\pgfpathcurveto{\pgfqpoint{2.873673in}{1.133936in}}{\pgfqpoint{2.881573in}{1.137208in}}{\pgfqpoint{2.887397in}{1.143032in}}%
\pgfpathcurveto{\pgfqpoint{2.893221in}{1.148856in}}{\pgfqpoint{2.896494in}{1.156756in}}{\pgfqpoint{2.896494in}{1.164992in}}%
\pgfpathcurveto{\pgfqpoint{2.896494in}{1.173228in}}{\pgfqpoint{2.893221in}{1.181128in}}{\pgfqpoint{2.887397in}{1.186952in}}%
\pgfpathcurveto{\pgfqpoint{2.881573in}{1.192776in}}{\pgfqpoint{2.873673in}{1.196049in}}{\pgfqpoint{2.865437in}{1.196049in}}%
\pgfpathcurveto{\pgfqpoint{2.857201in}{1.196049in}}{\pgfqpoint{2.849301in}{1.192776in}}{\pgfqpoint{2.843477in}{1.186952in}}%
\pgfpathcurveto{\pgfqpoint{2.837653in}{1.181128in}}{\pgfqpoint{2.834381in}{1.173228in}}{\pgfqpoint{2.834381in}{1.164992in}}%
\pgfpathcurveto{\pgfqpoint{2.834381in}{1.156756in}}{\pgfqpoint{2.837653in}{1.148856in}}{\pgfqpoint{2.843477in}{1.143032in}}%
\pgfpathcurveto{\pgfqpoint{2.849301in}{1.137208in}}{\pgfqpoint{2.857201in}{1.133936in}}{\pgfqpoint{2.865437in}{1.133936in}}%
\pgfpathclose%
\pgfusepath{stroke,fill}%
\end{pgfscope}%
\begin{pgfscope}%
\pgfpathrectangle{\pgfqpoint{0.100000in}{0.212622in}}{\pgfqpoint{3.696000in}{3.696000in}}%
\pgfusepath{clip}%
\pgfsetbuttcap%
\pgfsetroundjoin%
\definecolor{currentfill}{rgb}{0.121569,0.466667,0.705882}%
\pgfsetfillcolor{currentfill}%
\pgfsetfillopacity{0.997643}%
\pgfsetlinewidth{1.003750pt}%
\definecolor{currentstroke}{rgb}{0.121569,0.466667,0.705882}%
\pgfsetstrokecolor{currentstroke}%
\pgfsetstrokeopacity{0.997643}%
\pgfsetdash{}{0pt}%
\pgfpathmoveto{\pgfqpoint{2.865459in}{1.133927in}}%
\pgfpathcurveto{\pgfqpoint{2.873695in}{1.133927in}}{\pgfqpoint{2.881595in}{1.137200in}}{\pgfqpoint{2.887419in}{1.143023in}}%
\pgfpathcurveto{\pgfqpoint{2.893243in}{1.148847in}}{\pgfqpoint{2.896515in}{1.156747in}}{\pgfqpoint{2.896515in}{1.164984in}}%
\pgfpathcurveto{\pgfqpoint{2.896515in}{1.173220in}}{\pgfqpoint{2.893243in}{1.181120in}}{\pgfqpoint{2.887419in}{1.186944in}}%
\pgfpathcurveto{\pgfqpoint{2.881595in}{1.192768in}}{\pgfqpoint{2.873695in}{1.196040in}}{\pgfqpoint{2.865459in}{1.196040in}}%
\pgfpathcurveto{\pgfqpoint{2.857222in}{1.196040in}}{\pgfqpoint{2.849322in}{1.192768in}}{\pgfqpoint{2.843498in}{1.186944in}}%
\pgfpathcurveto{\pgfqpoint{2.837674in}{1.181120in}}{\pgfqpoint{2.834402in}{1.173220in}}{\pgfqpoint{2.834402in}{1.164984in}}%
\pgfpathcurveto{\pgfqpoint{2.834402in}{1.156747in}}{\pgfqpoint{2.837674in}{1.148847in}}{\pgfqpoint{2.843498in}{1.143023in}}%
\pgfpathcurveto{\pgfqpoint{2.849322in}{1.137200in}}{\pgfqpoint{2.857222in}{1.133927in}}{\pgfqpoint{2.865459in}{1.133927in}}%
\pgfpathclose%
\pgfusepath{stroke,fill}%
\end{pgfscope}%
\begin{pgfscope}%
\pgfpathrectangle{\pgfqpoint{0.100000in}{0.212622in}}{\pgfqpoint{3.696000in}{3.696000in}}%
\pgfusepath{clip}%
\pgfsetbuttcap%
\pgfsetroundjoin%
\definecolor{currentfill}{rgb}{0.121569,0.466667,0.705882}%
\pgfsetfillcolor{currentfill}%
\pgfsetfillopacity{0.997647}%
\pgfsetlinewidth{1.003750pt}%
\definecolor{currentstroke}{rgb}{0.121569,0.466667,0.705882}%
\pgfsetstrokecolor{currentstroke}%
\pgfsetstrokeopacity{0.997647}%
\pgfsetdash{}{0pt}%
\pgfpathmoveto{\pgfqpoint{2.865470in}{1.133922in}}%
\pgfpathcurveto{\pgfqpoint{2.873707in}{1.133922in}}{\pgfqpoint{2.881607in}{1.137195in}}{\pgfqpoint{2.887431in}{1.143019in}}%
\pgfpathcurveto{\pgfqpoint{2.893255in}{1.148843in}}{\pgfqpoint{2.896527in}{1.156743in}}{\pgfqpoint{2.896527in}{1.164979in}}%
\pgfpathcurveto{\pgfqpoint{2.896527in}{1.173215in}}{\pgfqpoint{2.893255in}{1.181115in}}{\pgfqpoint{2.887431in}{1.186939in}}%
\pgfpathcurveto{\pgfqpoint{2.881607in}{1.192763in}}{\pgfqpoint{2.873707in}{1.196035in}}{\pgfqpoint{2.865470in}{1.196035in}}%
\pgfpathcurveto{\pgfqpoint{2.857234in}{1.196035in}}{\pgfqpoint{2.849334in}{1.192763in}}{\pgfqpoint{2.843510in}{1.186939in}}%
\pgfpathcurveto{\pgfqpoint{2.837686in}{1.181115in}}{\pgfqpoint{2.834414in}{1.173215in}}{\pgfqpoint{2.834414in}{1.164979in}}%
\pgfpathcurveto{\pgfqpoint{2.834414in}{1.156743in}}{\pgfqpoint{2.837686in}{1.148843in}}{\pgfqpoint{2.843510in}{1.143019in}}%
\pgfpathcurveto{\pgfqpoint{2.849334in}{1.137195in}}{\pgfqpoint{2.857234in}{1.133922in}}{\pgfqpoint{2.865470in}{1.133922in}}%
\pgfpathclose%
\pgfusepath{stroke,fill}%
\end{pgfscope}%
\begin{pgfscope}%
\pgfpathrectangle{\pgfqpoint{0.100000in}{0.212622in}}{\pgfqpoint{3.696000in}{3.696000in}}%
\pgfusepath{clip}%
\pgfsetbuttcap%
\pgfsetroundjoin%
\definecolor{currentfill}{rgb}{0.121569,0.466667,0.705882}%
\pgfsetfillcolor{currentfill}%
\pgfsetfillopacity{0.997650}%
\pgfsetlinewidth{1.003750pt}%
\definecolor{currentstroke}{rgb}{0.121569,0.466667,0.705882}%
\pgfsetstrokecolor{currentstroke}%
\pgfsetstrokeopacity{0.997650}%
\pgfsetdash{}{0pt}%
\pgfpathmoveto{\pgfqpoint{2.865477in}{1.133919in}}%
\pgfpathcurveto{\pgfqpoint{2.873713in}{1.133919in}}{\pgfqpoint{2.881613in}{1.137192in}}{\pgfqpoint{2.887437in}{1.143016in}}%
\pgfpathcurveto{\pgfqpoint{2.893261in}{1.148840in}}{\pgfqpoint{2.896533in}{1.156740in}}{\pgfqpoint{2.896533in}{1.164976in}}%
\pgfpathcurveto{\pgfqpoint{2.896533in}{1.173212in}}{\pgfqpoint{2.893261in}{1.181112in}}{\pgfqpoint{2.887437in}{1.186936in}}%
\pgfpathcurveto{\pgfqpoint{2.881613in}{1.192760in}}{\pgfqpoint{2.873713in}{1.196032in}}{\pgfqpoint{2.865477in}{1.196032in}}%
\pgfpathcurveto{\pgfqpoint{2.857240in}{1.196032in}}{\pgfqpoint{2.849340in}{1.192760in}}{\pgfqpoint{2.843516in}{1.186936in}}%
\pgfpathcurveto{\pgfqpoint{2.837693in}{1.181112in}}{\pgfqpoint{2.834420in}{1.173212in}}{\pgfqpoint{2.834420in}{1.164976in}}%
\pgfpathcurveto{\pgfqpoint{2.834420in}{1.156740in}}{\pgfqpoint{2.837693in}{1.148840in}}{\pgfqpoint{2.843516in}{1.143016in}}%
\pgfpathcurveto{\pgfqpoint{2.849340in}{1.137192in}}{\pgfqpoint{2.857240in}{1.133919in}}{\pgfqpoint{2.865477in}{1.133919in}}%
\pgfpathclose%
\pgfusepath{stroke,fill}%
\end{pgfscope}%
\begin{pgfscope}%
\pgfpathrectangle{\pgfqpoint{0.100000in}{0.212622in}}{\pgfqpoint{3.696000in}{3.696000in}}%
\pgfusepath{clip}%
\pgfsetbuttcap%
\pgfsetroundjoin%
\definecolor{currentfill}{rgb}{0.121569,0.466667,0.705882}%
\pgfsetfillcolor{currentfill}%
\pgfsetfillopacity{0.997651}%
\pgfsetlinewidth{1.003750pt}%
\definecolor{currentstroke}{rgb}{0.121569,0.466667,0.705882}%
\pgfsetstrokecolor{currentstroke}%
\pgfsetstrokeopacity{0.997651}%
\pgfsetdash{}{0pt}%
\pgfpathmoveto{\pgfqpoint{2.865480in}{1.133918in}}%
\pgfpathcurveto{\pgfqpoint{2.873717in}{1.133918in}}{\pgfqpoint{2.881617in}{1.137190in}}{\pgfqpoint{2.887441in}{1.143014in}}%
\pgfpathcurveto{\pgfqpoint{2.893264in}{1.148838in}}{\pgfqpoint{2.896537in}{1.156738in}}{\pgfqpoint{2.896537in}{1.164974in}}%
\pgfpathcurveto{\pgfqpoint{2.896537in}{1.173211in}}{\pgfqpoint{2.893264in}{1.181111in}}{\pgfqpoint{2.887441in}{1.186935in}}%
\pgfpathcurveto{\pgfqpoint{2.881617in}{1.192759in}}{\pgfqpoint{2.873717in}{1.196031in}}{\pgfqpoint{2.865480in}{1.196031in}}%
\pgfpathcurveto{\pgfqpoint{2.857244in}{1.196031in}}{\pgfqpoint{2.849344in}{1.192759in}}{\pgfqpoint{2.843520in}{1.186935in}}%
\pgfpathcurveto{\pgfqpoint{2.837696in}{1.181111in}}{\pgfqpoint{2.834424in}{1.173211in}}{\pgfqpoint{2.834424in}{1.164974in}}%
\pgfpathcurveto{\pgfqpoint{2.834424in}{1.156738in}}{\pgfqpoint{2.837696in}{1.148838in}}{\pgfqpoint{2.843520in}{1.143014in}}%
\pgfpathcurveto{\pgfqpoint{2.849344in}{1.137190in}}{\pgfqpoint{2.857244in}{1.133918in}}{\pgfqpoint{2.865480in}{1.133918in}}%
\pgfpathclose%
\pgfusepath{stroke,fill}%
\end{pgfscope}%
\begin{pgfscope}%
\pgfpathrectangle{\pgfqpoint{0.100000in}{0.212622in}}{\pgfqpoint{3.696000in}{3.696000in}}%
\pgfusepath{clip}%
\pgfsetbuttcap%
\pgfsetroundjoin%
\definecolor{currentfill}{rgb}{0.121569,0.466667,0.705882}%
\pgfsetfillcolor{currentfill}%
\pgfsetfillopacity{0.997652}%
\pgfsetlinewidth{1.003750pt}%
\definecolor{currentstroke}{rgb}{0.121569,0.466667,0.705882}%
\pgfsetstrokecolor{currentstroke}%
\pgfsetstrokeopacity{0.997652}%
\pgfsetdash{}{0pt}%
\pgfpathmoveto{\pgfqpoint{2.865482in}{1.133917in}}%
\pgfpathcurveto{\pgfqpoint{2.873719in}{1.133917in}}{\pgfqpoint{2.881619in}{1.137189in}}{\pgfqpoint{2.887443in}{1.143013in}}%
\pgfpathcurveto{\pgfqpoint{2.893266in}{1.148837in}}{\pgfqpoint{2.896539in}{1.156737in}}{\pgfqpoint{2.896539in}{1.164974in}}%
\pgfpathcurveto{\pgfqpoint{2.896539in}{1.173210in}}{\pgfqpoint{2.893266in}{1.181110in}}{\pgfqpoint{2.887443in}{1.186934in}}%
\pgfpathcurveto{\pgfqpoint{2.881619in}{1.192758in}}{\pgfqpoint{2.873719in}{1.196030in}}{\pgfqpoint{2.865482in}{1.196030in}}%
\pgfpathcurveto{\pgfqpoint{2.857246in}{1.196030in}}{\pgfqpoint{2.849346in}{1.192758in}}{\pgfqpoint{2.843522in}{1.186934in}}%
\pgfpathcurveto{\pgfqpoint{2.837698in}{1.181110in}}{\pgfqpoint{2.834426in}{1.173210in}}{\pgfqpoint{2.834426in}{1.164974in}}%
\pgfpathcurveto{\pgfqpoint{2.834426in}{1.156737in}}{\pgfqpoint{2.837698in}{1.148837in}}{\pgfqpoint{2.843522in}{1.143013in}}%
\pgfpathcurveto{\pgfqpoint{2.849346in}{1.137189in}}{\pgfqpoint{2.857246in}{1.133917in}}{\pgfqpoint{2.865482in}{1.133917in}}%
\pgfpathclose%
\pgfusepath{stroke,fill}%
\end{pgfscope}%
\begin{pgfscope}%
\pgfpathrectangle{\pgfqpoint{0.100000in}{0.212622in}}{\pgfqpoint{3.696000in}{3.696000in}}%
\pgfusepath{clip}%
\pgfsetbuttcap%
\pgfsetroundjoin%
\definecolor{currentfill}{rgb}{0.121569,0.466667,0.705882}%
\pgfsetfillcolor{currentfill}%
\pgfsetfillopacity{0.997652}%
\pgfsetlinewidth{1.003750pt}%
\definecolor{currentstroke}{rgb}{0.121569,0.466667,0.705882}%
\pgfsetstrokecolor{currentstroke}%
\pgfsetstrokeopacity{0.997652}%
\pgfsetdash{}{0pt}%
\pgfpathmoveto{\pgfqpoint{2.865483in}{1.133917in}}%
\pgfpathcurveto{\pgfqpoint{2.873720in}{1.133917in}}{\pgfqpoint{2.881620in}{1.137189in}}{\pgfqpoint{2.887444in}{1.143013in}}%
\pgfpathcurveto{\pgfqpoint{2.893267in}{1.148837in}}{\pgfqpoint{2.896540in}{1.156737in}}{\pgfqpoint{2.896540in}{1.164973in}}%
\pgfpathcurveto{\pgfqpoint{2.896540in}{1.173209in}}{\pgfqpoint{2.893267in}{1.181109in}}{\pgfqpoint{2.887444in}{1.186933in}}%
\pgfpathcurveto{\pgfqpoint{2.881620in}{1.192757in}}{\pgfqpoint{2.873720in}{1.196030in}}{\pgfqpoint{2.865483in}{1.196030in}}%
\pgfpathcurveto{\pgfqpoint{2.857247in}{1.196030in}}{\pgfqpoint{2.849347in}{1.192757in}}{\pgfqpoint{2.843523in}{1.186933in}}%
\pgfpathcurveto{\pgfqpoint{2.837699in}{1.181109in}}{\pgfqpoint{2.834427in}{1.173209in}}{\pgfqpoint{2.834427in}{1.164973in}}%
\pgfpathcurveto{\pgfqpoint{2.834427in}{1.156737in}}{\pgfqpoint{2.837699in}{1.148837in}}{\pgfqpoint{2.843523in}{1.143013in}}%
\pgfpathcurveto{\pgfqpoint{2.849347in}{1.137189in}}{\pgfqpoint{2.857247in}{1.133917in}}{\pgfqpoint{2.865483in}{1.133917in}}%
\pgfpathclose%
\pgfusepath{stroke,fill}%
\end{pgfscope}%
\begin{pgfscope}%
\pgfpathrectangle{\pgfqpoint{0.100000in}{0.212622in}}{\pgfqpoint{3.696000in}{3.696000in}}%
\pgfusepath{clip}%
\pgfsetbuttcap%
\pgfsetroundjoin%
\definecolor{currentfill}{rgb}{0.121569,0.466667,0.705882}%
\pgfsetfillcolor{currentfill}%
\pgfsetfillopacity{0.997653}%
\pgfsetlinewidth{1.003750pt}%
\definecolor{currentstroke}{rgb}{0.121569,0.466667,0.705882}%
\pgfsetstrokecolor{currentstroke}%
\pgfsetstrokeopacity{0.997653}%
\pgfsetdash{}{0pt}%
\pgfpathmoveto{\pgfqpoint{2.865484in}{1.133916in}}%
\pgfpathcurveto{\pgfqpoint{2.873720in}{1.133916in}}{\pgfqpoint{2.881620in}{1.137189in}}{\pgfqpoint{2.887444in}{1.143013in}}%
\pgfpathcurveto{\pgfqpoint{2.893268in}{1.148836in}}{\pgfqpoint{2.896540in}{1.156737in}}{\pgfqpoint{2.896540in}{1.164973in}}%
\pgfpathcurveto{\pgfqpoint{2.896540in}{1.173209in}}{\pgfqpoint{2.893268in}{1.181109in}}{\pgfqpoint{2.887444in}{1.186933in}}%
\pgfpathcurveto{\pgfqpoint{2.881620in}{1.192757in}}{\pgfqpoint{2.873720in}{1.196029in}}{\pgfqpoint{2.865484in}{1.196029in}}%
\pgfpathcurveto{\pgfqpoint{2.857248in}{1.196029in}}{\pgfqpoint{2.849348in}{1.192757in}}{\pgfqpoint{2.843524in}{1.186933in}}%
\pgfpathcurveto{\pgfqpoint{2.837700in}{1.181109in}}{\pgfqpoint{2.834427in}{1.173209in}}{\pgfqpoint{2.834427in}{1.164973in}}%
\pgfpathcurveto{\pgfqpoint{2.834427in}{1.156737in}}{\pgfqpoint{2.837700in}{1.148836in}}{\pgfqpoint{2.843524in}{1.143013in}}%
\pgfpathcurveto{\pgfqpoint{2.849348in}{1.137189in}}{\pgfqpoint{2.857248in}{1.133916in}}{\pgfqpoint{2.865484in}{1.133916in}}%
\pgfpathclose%
\pgfusepath{stroke,fill}%
\end{pgfscope}%
\begin{pgfscope}%
\pgfpathrectangle{\pgfqpoint{0.100000in}{0.212622in}}{\pgfqpoint{3.696000in}{3.696000in}}%
\pgfusepath{clip}%
\pgfsetbuttcap%
\pgfsetroundjoin%
\definecolor{currentfill}{rgb}{0.121569,0.466667,0.705882}%
\pgfsetfillcolor{currentfill}%
\pgfsetfillopacity{0.997653}%
\pgfsetlinewidth{1.003750pt}%
\definecolor{currentstroke}{rgb}{0.121569,0.466667,0.705882}%
\pgfsetstrokecolor{currentstroke}%
\pgfsetstrokeopacity{0.997653}%
\pgfsetdash{}{0pt}%
\pgfpathmoveto{\pgfqpoint{2.865484in}{1.133916in}}%
\pgfpathcurveto{\pgfqpoint{2.873720in}{1.133916in}}{\pgfqpoint{2.881621in}{1.137188in}}{\pgfqpoint{2.887444in}{1.143012in}}%
\pgfpathcurveto{\pgfqpoint{2.893268in}{1.148836in}}{\pgfqpoint{2.896541in}{1.156736in}}{\pgfqpoint{2.896541in}{1.164973in}}%
\pgfpathcurveto{\pgfqpoint{2.896541in}{1.173209in}}{\pgfqpoint{2.893268in}{1.181109in}}{\pgfqpoint{2.887444in}{1.186933in}}%
\pgfpathcurveto{\pgfqpoint{2.881621in}{1.192757in}}{\pgfqpoint{2.873720in}{1.196029in}}{\pgfqpoint{2.865484in}{1.196029in}}%
\pgfpathcurveto{\pgfqpoint{2.857248in}{1.196029in}}{\pgfqpoint{2.849348in}{1.192757in}}{\pgfqpoint{2.843524in}{1.186933in}}%
\pgfpathcurveto{\pgfqpoint{2.837700in}{1.181109in}}{\pgfqpoint{2.834428in}{1.173209in}}{\pgfqpoint{2.834428in}{1.164973in}}%
\pgfpathcurveto{\pgfqpoint{2.834428in}{1.156736in}}{\pgfqpoint{2.837700in}{1.148836in}}{\pgfqpoint{2.843524in}{1.143012in}}%
\pgfpathcurveto{\pgfqpoint{2.849348in}{1.137188in}}{\pgfqpoint{2.857248in}{1.133916in}}{\pgfqpoint{2.865484in}{1.133916in}}%
\pgfpathclose%
\pgfusepath{stroke,fill}%
\end{pgfscope}%
\begin{pgfscope}%
\pgfpathrectangle{\pgfqpoint{0.100000in}{0.212622in}}{\pgfqpoint{3.696000in}{3.696000in}}%
\pgfusepath{clip}%
\pgfsetbuttcap%
\pgfsetroundjoin%
\definecolor{currentfill}{rgb}{0.121569,0.466667,0.705882}%
\pgfsetfillcolor{currentfill}%
\pgfsetfillopacity{0.997653}%
\pgfsetlinewidth{1.003750pt}%
\definecolor{currentstroke}{rgb}{0.121569,0.466667,0.705882}%
\pgfsetstrokecolor{currentstroke}%
\pgfsetstrokeopacity{0.997653}%
\pgfsetdash{}{0pt}%
\pgfpathmoveto{\pgfqpoint{2.865484in}{1.133916in}}%
\pgfpathcurveto{\pgfqpoint{2.873721in}{1.133916in}}{\pgfqpoint{2.881621in}{1.137188in}}{\pgfqpoint{2.887445in}{1.143012in}}%
\pgfpathcurveto{\pgfqpoint{2.893269in}{1.148836in}}{\pgfqpoint{2.896541in}{1.156736in}}{\pgfqpoint{2.896541in}{1.164973in}}%
\pgfpathcurveto{\pgfqpoint{2.896541in}{1.173209in}}{\pgfqpoint{2.893269in}{1.181109in}}{\pgfqpoint{2.887445in}{1.186933in}}%
\pgfpathcurveto{\pgfqpoint{2.881621in}{1.192757in}}{\pgfqpoint{2.873721in}{1.196029in}}{\pgfqpoint{2.865484in}{1.196029in}}%
\pgfpathcurveto{\pgfqpoint{2.857248in}{1.196029in}}{\pgfqpoint{2.849348in}{1.192757in}}{\pgfqpoint{2.843524in}{1.186933in}}%
\pgfpathcurveto{\pgfqpoint{2.837700in}{1.181109in}}{\pgfqpoint{2.834428in}{1.173209in}}{\pgfqpoint{2.834428in}{1.164973in}}%
\pgfpathcurveto{\pgfqpoint{2.834428in}{1.156736in}}{\pgfqpoint{2.837700in}{1.148836in}}{\pgfqpoint{2.843524in}{1.143012in}}%
\pgfpathcurveto{\pgfqpoint{2.849348in}{1.137188in}}{\pgfqpoint{2.857248in}{1.133916in}}{\pgfqpoint{2.865484in}{1.133916in}}%
\pgfpathclose%
\pgfusepath{stroke,fill}%
\end{pgfscope}%
\begin{pgfscope}%
\pgfpathrectangle{\pgfqpoint{0.100000in}{0.212622in}}{\pgfqpoint{3.696000in}{3.696000in}}%
\pgfusepath{clip}%
\pgfsetbuttcap%
\pgfsetroundjoin%
\definecolor{currentfill}{rgb}{0.121569,0.466667,0.705882}%
\pgfsetfillcolor{currentfill}%
\pgfsetfillopacity{0.997653}%
\pgfsetlinewidth{1.003750pt}%
\definecolor{currentstroke}{rgb}{0.121569,0.466667,0.705882}%
\pgfsetstrokecolor{currentstroke}%
\pgfsetstrokeopacity{0.997653}%
\pgfsetdash{}{0pt}%
\pgfpathmoveto{\pgfqpoint{2.865484in}{1.133916in}}%
\pgfpathcurveto{\pgfqpoint{2.873721in}{1.133916in}}{\pgfqpoint{2.881621in}{1.137188in}}{\pgfqpoint{2.887445in}{1.143012in}}%
\pgfpathcurveto{\pgfqpoint{2.893269in}{1.148836in}}{\pgfqpoint{2.896541in}{1.156736in}}{\pgfqpoint{2.896541in}{1.164973in}}%
\pgfpathcurveto{\pgfqpoint{2.896541in}{1.173209in}}{\pgfqpoint{2.893269in}{1.181109in}}{\pgfqpoint{2.887445in}{1.186933in}}%
\pgfpathcurveto{\pgfqpoint{2.881621in}{1.192757in}}{\pgfqpoint{2.873721in}{1.196029in}}{\pgfqpoint{2.865484in}{1.196029in}}%
\pgfpathcurveto{\pgfqpoint{2.857248in}{1.196029in}}{\pgfqpoint{2.849348in}{1.192757in}}{\pgfqpoint{2.843524in}{1.186933in}}%
\pgfpathcurveto{\pgfqpoint{2.837700in}{1.181109in}}{\pgfqpoint{2.834428in}{1.173209in}}{\pgfqpoint{2.834428in}{1.164973in}}%
\pgfpathcurveto{\pgfqpoint{2.834428in}{1.156736in}}{\pgfqpoint{2.837700in}{1.148836in}}{\pgfqpoint{2.843524in}{1.143012in}}%
\pgfpathcurveto{\pgfqpoint{2.849348in}{1.137188in}}{\pgfqpoint{2.857248in}{1.133916in}}{\pgfqpoint{2.865484in}{1.133916in}}%
\pgfpathclose%
\pgfusepath{stroke,fill}%
\end{pgfscope}%
\begin{pgfscope}%
\pgfpathrectangle{\pgfqpoint{0.100000in}{0.212622in}}{\pgfqpoint{3.696000in}{3.696000in}}%
\pgfusepath{clip}%
\pgfsetbuttcap%
\pgfsetroundjoin%
\definecolor{currentfill}{rgb}{0.121569,0.466667,0.705882}%
\pgfsetfillcolor{currentfill}%
\pgfsetfillopacity{0.997653}%
\pgfsetlinewidth{1.003750pt}%
\definecolor{currentstroke}{rgb}{0.121569,0.466667,0.705882}%
\pgfsetstrokecolor{currentstroke}%
\pgfsetstrokeopacity{0.997653}%
\pgfsetdash{}{0pt}%
\pgfpathmoveto{\pgfqpoint{2.865485in}{1.133916in}}%
\pgfpathcurveto{\pgfqpoint{2.873721in}{1.133916in}}{\pgfqpoint{2.881621in}{1.137188in}}{\pgfqpoint{2.887445in}{1.143012in}}%
\pgfpathcurveto{\pgfqpoint{2.893269in}{1.148836in}}{\pgfqpoint{2.896541in}{1.156736in}}{\pgfqpoint{2.896541in}{1.164973in}}%
\pgfpathcurveto{\pgfqpoint{2.896541in}{1.173209in}}{\pgfqpoint{2.893269in}{1.181109in}}{\pgfqpoint{2.887445in}{1.186933in}}%
\pgfpathcurveto{\pgfqpoint{2.881621in}{1.192757in}}{\pgfqpoint{2.873721in}{1.196029in}}{\pgfqpoint{2.865485in}{1.196029in}}%
\pgfpathcurveto{\pgfqpoint{2.857248in}{1.196029in}}{\pgfqpoint{2.849348in}{1.192757in}}{\pgfqpoint{2.843524in}{1.186933in}}%
\pgfpathcurveto{\pgfqpoint{2.837700in}{1.181109in}}{\pgfqpoint{2.834428in}{1.173209in}}{\pgfqpoint{2.834428in}{1.164973in}}%
\pgfpathcurveto{\pgfqpoint{2.834428in}{1.156736in}}{\pgfqpoint{2.837700in}{1.148836in}}{\pgfqpoint{2.843524in}{1.143012in}}%
\pgfpathcurveto{\pgfqpoint{2.849348in}{1.137188in}}{\pgfqpoint{2.857248in}{1.133916in}}{\pgfqpoint{2.865485in}{1.133916in}}%
\pgfpathclose%
\pgfusepath{stroke,fill}%
\end{pgfscope}%
\begin{pgfscope}%
\pgfpathrectangle{\pgfqpoint{0.100000in}{0.212622in}}{\pgfqpoint{3.696000in}{3.696000in}}%
\pgfusepath{clip}%
\pgfsetbuttcap%
\pgfsetroundjoin%
\definecolor{currentfill}{rgb}{0.121569,0.466667,0.705882}%
\pgfsetfillcolor{currentfill}%
\pgfsetfillopacity{0.997653}%
\pgfsetlinewidth{1.003750pt}%
\definecolor{currentstroke}{rgb}{0.121569,0.466667,0.705882}%
\pgfsetstrokecolor{currentstroke}%
\pgfsetstrokeopacity{0.997653}%
\pgfsetdash{}{0pt}%
\pgfpathmoveto{\pgfqpoint{2.865485in}{1.133916in}}%
\pgfpathcurveto{\pgfqpoint{2.873721in}{1.133916in}}{\pgfqpoint{2.881621in}{1.137188in}}{\pgfqpoint{2.887445in}{1.143012in}}%
\pgfpathcurveto{\pgfqpoint{2.893269in}{1.148836in}}{\pgfqpoint{2.896541in}{1.156736in}}{\pgfqpoint{2.896541in}{1.164973in}}%
\pgfpathcurveto{\pgfqpoint{2.896541in}{1.173209in}}{\pgfqpoint{2.893269in}{1.181109in}}{\pgfqpoint{2.887445in}{1.186933in}}%
\pgfpathcurveto{\pgfqpoint{2.881621in}{1.192757in}}{\pgfqpoint{2.873721in}{1.196029in}}{\pgfqpoint{2.865485in}{1.196029in}}%
\pgfpathcurveto{\pgfqpoint{2.857248in}{1.196029in}}{\pgfqpoint{2.849348in}{1.192757in}}{\pgfqpoint{2.843524in}{1.186933in}}%
\pgfpathcurveto{\pgfqpoint{2.837700in}{1.181109in}}{\pgfqpoint{2.834428in}{1.173209in}}{\pgfqpoint{2.834428in}{1.164973in}}%
\pgfpathcurveto{\pgfqpoint{2.834428in}{1.156736in}}{\pgfqpoint{2.837700in}{1.148836in}}{\pgfqpoint{2.843524in}{1.143012in}}%
\pgfpathcurveto{\pgfqpoint{2.849348in}{1.137188in}}{\pgfqpoint{2.857248in}{1.133916in}}{\pgfqpoint{2.865485in}{1.133916in}}%
\pgfpathclose%
\pgfusepath{stroke,fill}%
\end{pgfscope}%
\begin{pgfscope}%
\pgfpathrectangle{\pgfqpoint{0.100000in}{0.212622in}}{\pgfqpoint{3.696000in}{3.696000in}}%
\pgfusepath{clip}%
\pgfsetbuttcap%
\pgfsetroundjoin%
\definecolor{currentfill}{rgb}{0.121569,0.466667,0.705882}%
\pgfsetfillcolor{currentfill}%
\pgfsetfillopacity{0.997653}%
\pgfsetlinewidth{1.003750pt}%
\definecolor{currentstroke}{rgb}{0.121569,0.466667,0.705882}%
\pgfsetstrokecolor{currentstroke}%
\pgfsetstrokeopacity{0.997653}%
\pgfsetdash{}{0pt}%
\pgfpathmoveto{\pgfqpoint{2.865485in}{1.133916in}}%
\pgfpathcurveto{\pgfqpoint{2.873721in}{1.133916in}}{\pgfqpoint{2.881621in}{1.137188in}}{\pgfqpoint{2.887445in}{1.143012in}}%
\pgfpathcurveto{\pgfqpoint{2.893269in}{1.148836in}}{\pgfqpoint{2.896541in}{1.156736in}}{\pgfqpoint{2.896541in}{1.164973in}}%
\pgfpathcurveto{\pgfqpoint{2.896541in}{1.173209in}}{\pgfqpoint{2.893269in}{1.181109in}}{\pgfqpoint{2.887445in}{1.186933in}}%
\pgfpathcurveto{\pgfqpoint{2.881621in}{1.192757in}}{\pgfqpoint{2.873721in}{1.196029in}}{\pgfqpoint{2.865485in}{1.196029in}}%
\pgfpathcurveto{\pgfqpoint{2.857248in}{1.196029in}}{\pgfqpoint{2.849348in}{1.192757in}}{\pgfqpoint{2.843524in}{1.186933in}}%
\pgfpathcurveto{\pgfqpoint{2.837700in}{1.181109in}}{\pgfqpoint{2.834428in}{1.173209in}}{\pgfqpoint{2.834428in}{1.164973in}}%
\pgfpathcurveto{\pgfqpoint{2.834428in}{1.156736in}}{\pgfqpoint{2.837700in}{1.148836in}}{\pgfqpoint{2.843524in}{1.143012in}}%
\pgfpathcurveto{\pgfqpoint{2.849348in}{1.137188in}}{\pgfqpoint{2.857248in}{1.133916in}}{\pgfqpoint{2.865485in}{1.133916in}}%
\pgfpathclose%
\pgfusepath{stroke,fill}%
\end{pgfscope}%
\begin{pgfscope}%
\pgfpathrectangle{\pgfqpoint{0.100000in}{0.212622in}}{\pgfqpoint{3.696000in}{3.696000in}}%
\pgfusepath{clip}%
\pgfsetbuttcap%
\pgfsetroundjoin%
\definecolor{currentfill}{rgb}{0.121569,0.466667,0.705882}%
\pgfsetfillcolor{currentfill}%
\pgfsetfillopacity{0.997653}%
\pgfsetlinewidth{1.003750pt}%
\definecolor{currentstroke}{rgb}{0.121569,0.466667,0.705882}%
\pgfsetstrokecolor{currentstroke}%
\pgfsetstrokeopacity{0.997653}%
\pgfsetdash{}{0pt}%
\pgfpathmoveto{\pgfqpoint{2.865485in}{1.133916in}}%
\pgfpathcurveto{\pgfqpoint{2.873721in}{1.133916in}}{\pgfqpoint{2.881621in}{1.137188in}}{\pgfqpoint{2.887445in}{1.143012in}}%
\pgfpathcurveto{\pgfqpoint{2.893269in}{1.148836in}}{\pgfqpoint{2.896541in}{1.156736in}}{\pgfqpoint{2.896541in}{1.164973in}}%
\pgfpathcurveto{\pgfqpoint{2.896541in}{1.173209in}}{\pgfqpoint{2.893269in}{1.181109in}}{\pgfqpoint{2.887445in}{1.186933in}}%
\pgfpathcurveto{\pgfqpoint{2.881621in}{1.192757in}}{\pgfqpoint{2.873721in}{1.196029in}}{\pgfqpoint{2.865485in}{1.196029in}}%
\pgfpathcurveto{\pgfqpoint{2.857248in}{1.196029in}}{\pgfqpoint{2.849348in}{1.192757in}}{\pgfqpoint{2.843524in}{1.186933in}}%
\pgfpathcurveto{\pgfqpoint{2.837700in}{1.181109in}}{\pgfqpoint{2.834428in}{1.173209in}}{\pgfqpoint{2.834428in}{1.164973in}}%
\pgfpathcurveto{\pgfqpoint{2.834428in}{1.156736in}}{\pgfqpoint{2.837700in}{1.148836in}}{\pgfqpoint{2.843524in}{1.143012in}}%
\pgfpathcurveto{\pgfqpoint{2.849348in}{1.137188in}}{\pgfqpoint{2.857248in}{1.133916in}}{\pgfqpoint{2.865485in}{1.133916in}}%
\pgfpathclose%
\pgfusepath{stroke,fill}%
\end{pgfscope}%
\begin{pgfscope}%
\pgfpathrectangle{\pgfqpoint{0.100000in}{0.212622in}}{\pgfqpoint{3.696000in}{3.696000in}}%
\pgfusepath{clip}%
\pgfsetbuttcap%
\pgfsetroundjoin%
\definecolor{currentfill}{rgb}{0.121569,0.466667,0.705882}%
\pgfsetfillcolor{currentfill}%
\pgfsetfillopacity{0.997653}%
\pgfsetlinewidth{1.003750pt}%
\definecolor{currentstroke}{rgb}{0.121569,0.466667,0.705882}%
\pgfsetstrokecolor{currentstroke}%
\pgfsetstrokeopacity{0.997653}%
\pgfsetdash{}{0pt}%
\pgfpathmoveto{\pgfqpoint{2.865485in}{1.133916in}}%
\pgfpathcurveto{\pgfqpoint{2.873721in}{1.133916in}}{\pgfqpoint{2.881621in}{1.137188in}}{\pgfqpoint{2.887445in}{1.143012in}}%
\pgfpathcurveto{\pgfqpoint{2.893269in}{1.148836in}}{\pgfqpoint{2.896541in}{1.156736in}}{\pgfqpoint{2.896541in}{1.164973in}}%
\pgfpathcurveto{\pgfqpoint{2.896541in}{1.173209in}}{\pgfqpoint{2.893269in}{1.181109in}}{\pgfqpoint{2.887445in}{1.186933in}}%
\pgfpathcurveto{\pgfqpoint{2.881621in}{1.192757in}}{\pgfqpoint{2.873721in}{1.196029in}}{\pgfqpoint{2.865485in}{1.196029in}}%
\pgfpathcurveto{\pgfqpoint{2.857248in}{1.196029in}}{\pgfqpoint{2.849348in}{1.192757in}}{\pgfqpoint{2.843524in}{1.186933in}}%
\pgfpathcurveto{\pgfqpoint{2.837700in}{1.181109in}}{\pgfqpoint{2.834428in}{1.173209in}}{\pgfqpoint{2.834428in}{1.164973in}}%
\pgfpathcurveto{\pgfqpoint{2.834428in}{1.156736in}}{\pgfqpoint{2.837700in}{1.148836in}}{\pgfqpoint{2.843524in}{1.143012in}}%
\pgfpathcurveto{\pgfqpoint{2.849348in}{1.137188in}}{\pgfqpoint{2.857248in}{1.133916in}}{\pgfqpoint{2.865485in}{1.133916in}}%
\pgfpathclose%
\pgfusepath{stroke,fill}%
\end{pgfscope}%
\begin{pgfscope}%
\pgfpathrectangle{\pgfqpoint{0.100000in}{0.212622in}}{\pgfqpoint{3.696000in}{3.696000in}}%
\pgfusepath{clip}%
\pgfsetbuttcap%
\pgfsetroundjoin%
\definecolor{currentfill}{rgb}{0.121569,0.466667,0.705882}%
\pgfsetfillcolor{currentfill}%
\pgfsetfillopacity{0.997653}%
\pgfsetlinewidth{1.003750pt}%
\definecolor{currentstroke}{rgb}{0.121569,0.466667,0.705882}%
\pgfsetstrokecolor{currentstroke}%
\pgfsetstrokeopacity{0.997653}%
\pgfsetdash{}{0pt}%
\pgfpathmoveto{\pgfqpoint{2.865485in}{1.133916in}}%
\pgfpathcurveto{\pgfqpoint{2.873721in}{1.133916in}}{\pgfqpoint{2.881621in}{1.137188in}}{\pgfqpoint{2.887445in}{1.143012in}}%
\pgfpathcurveto{\pgfqpoint{2.893269in}{1.148836in}}{\pgfqpoint{2.896541in}{1.156736in}}{\pgfqpoint{2.896541in}{1.164973in}}%
\pgfpathcurveto{\pgfqpoint{2.896541in}{1.173209in}}{\pgfqpoint{2.893269in}{1.181109in}}{\pgfqpoint{2.887445in}{1.186933in}}%
\pgfpathcurveto{\pgfqpoint{2.881621in}{1.192757in}}{\pgfqpoint{2.873721in}{1.196029in}}{\pgfqpoint{2.865485in}{1.196029in}}%
\pgfpathcurveto{\pgfqpoint{2.857248in}{1.196029in}}{\pgfqpoint{2.849348in}{1.192757in}}{\pgfqpoint{2.843524in}{1.186933in}}%
\pgfpathcurveto{\pgfqpoint{2.837700in}{1.181109in}}{\pgfqpoint{2.834428in}{1.173209in}}{\pgfqpoint{2.834428in}{1.164973in}}%
\pgfpathcurveto{\pgfqpoint{2.834428in}{1.156736in}}{\pgfqpoint{2.837700in}{1.148836in}}{\pgfqpoint{2.843524in}{1.143012in}}%
\pgfpathcurveto{\pgfqpoint{2.849348in}{1.137188in}}{\pgfqpoint{2.857248in}{1.133916in}}{\pgfqpoint{2.865485in}{1.133916in}}%
\pgfpathclose%
\pgfusepath{stroke,fill}%
\end{pgfscope}%
\begin{pgfscope}%
\pgfpathrectangle{\pgfqpoint{0.100000in}{0.212622in}}{\pgfqpoint{3.696000in}{3.696000in}}%
\pgfusepath{clip}%
\pgfsetbuttcap%
\pgfsetroundjoin%
\definecolor{currentfill}{rgb}{0.121569,0.466667,0.705882}%
\pgfsetfillcolor{currentfill}%
\pgfsetfillopacity{0.997653}%
\pgfsetlinewidth{1.003750pt}%
\definecolor{currentstroke}{rgb}{0.121569,0.466667,0.705882}%
\pgfsetstrokecolor{currentstroke}%
\pgfsetstrokeopacity{0.997653}%
\pgfsetdash{}{0pt}%
\pgfpathmoveto{\pgfqpoint{2.865485in}{1.133916in}}%
\pgfpathcurveto{\pgfqpoint{2.873721in}{1.133916in}}{\pgfqpoint{2.881621in}{1.137188in}}{\pgfqpoint{2.887445in}{1.143012in}}%
\pgfpathcurveto{\pgfqpoint{2.893269in}{1.148836in}}{\pgfqpoint{2.896541in}{1.156736in}}{\pgfqpoint{2.896541in}{1.164973in}}%
\pgfpathcurveto{\pgfqpoint{2.896541in}{1.173209in}}{\pgfqpoint{2.893269in}{1.181109in}}{\pgfqpoint{2.887445in}{1.186933in}}%
\pgfpathcurveto{\pgfqpoint{2.881621in}{1.192757in}}{\pgfqpoint{2.873721in}{1.196029in}}{\pgfqpoint{2.865485in}{1.196029in}}%
\pgfpathcurveto{\pgfqpoint{2.857248in}{1.196029in}}{\pgfqpoint{2.849348in}{1.192757in}}{\pgfqpoint{2.843524in}{1.186933in}}%
\pgfpathcurveto{\pgfqpoint{2.837700in}{1.181109in}}{\pgfqpoint{2.834428in}{1.173209in}}{\pgfqpoint{2.834428in}{1.164973in}}%
\pgfpathcurveto{\pgfqpoint{2.834428in}{1.156736in}}{\pgfqpoint{2.837700in}{1.148836in}}{\pgfqpoint{2.843524in}{1.143012in}}%
\pgfpathcurveto{\pgfqpoint{2.849348in}{1.137188in}}{\pgfqpoint{2.857248in}{1.133916in}}{\pgfqpoint{2.865485in}{1.133916in}}%
\pgfpathclose%
\pgfusepath{stroke,fill}%
\end{pgfscope}%
\begin{pgfscope}%
\pgfpathrectangle{\pgfqpoint{0.100000in}{0.212622in}}{\pgfqpoint{3.696000in}{3.696000in}}%
\pgfusepath{clip}%
\pgfsetbuttcap%
\pgfsetroundjoin%
\definecolor{currentfill}{rgb}{0.121569,0.466667,0.705882}%
\pgfsetfillcolor{currentfill}%
\pgfsetfillopacity{0.997653}%
\pgfsetlinewidth{1.003750pt}%
\definecolor{currentstroke}{rgb}{0.121569,0.466667,0.705882}%
\pgfsetstrokecolor{currentstroke}%
\pgfsetstrokeopacity{0.997653}%
\pgfsetdash{}{0pt}%
\pgfpathmoveto{\pgfqpoint{2.865485in}{1.133916in}}%
\pgfpathcurveto{\pgfqpoint{2.873721in}{1.133916in}}{\pgfqpoint{2.881621in}{1.137188in}}{\pgfqpoint{2.887445in}{1.143012in}}%
\pgfpathcurveto{\pgfqpoint{2.893269in}{1.148836in}}{\pgfqpoint{2.896541in}{1.156736in}}{\pgfqpoint{2.896541in}{1.164973in}}%
\pgfpathcurveto{\pgfqpoint{2.896541in}{1.173209in}}{\pgfqpoint{2.893269in}{1.181109in}}{\pgfqpoint{2.887445in}{1.186933in}}%
\pgfpathcurveto{\pgfqpoint{2.881621in}{1.192757in}}{\pgfqpoint{2.873721in}{1.196029in}}{\pgfqpoint{2.865485in}{1.196029in}}%
\pgfpathcurveto{\pgfqpoint{2.857248in}{1.196029in}}{\pgfqpoint{2.849348in}{1.192757in}}{\pgfqpoint{2.843524in}{1.186933in}}%
\pgfpathcurveto{\pgfqpoint{2.837700in}{1.181109in}}{\pgfqpoint{2.834428in}{1.173209in}}{\pgfqpoint{2.834428in}{1.164973in}}%
\pgfpathcurveto{\pgfqpoint{2.834428in}{1.156736in}}{\pgfqpoint{2.837700in}{1.148836in}}{\pgfqpoint{2.843524in}{1.143012in}}%
\pgfpathcurveto{\pgfqpoint{2.849348in}{1.137188in}}{\pgfqpoint{2.857248in}{1.133916in}}{\pgfqpoint{2.865485in}{1.133916in}}%
\pgfpathclose%
\pgfusepath{stroke,fill}%
\end{pgfscope}%
\begin{pgfscope}%
\pgfpathrectangle{\pgfqpoint{0.100000in}{0.212622in}}{\pgfqpoint{3.696000in}{3.696000in}}%
\pgfusepath{clip}%
\pgfsetbuttcap%
\pgfsetroundjoin%
\definecolor{currentfill}{rgb}{0.121569,0.466667,0.705882}%
\pgfsetfillcolor{currentfill}%
\pgfsetfillopacity{0.997653}%
\pgfsetlinewidth{1.003750pt}%
\definecolor{currentstroke}{rgb}{0.121569,0.466667,0.705882}%
\pgfsetstrokecolor{currentstroke}%
\pgfsetstrokeopacity{0.997653}%
\pgfsetdash{}{0pt}%
\pgfpathmoveto{\pgfqpoint{2.865485in}{1.133916in}}%
\pgfpathcurveto{\pgfqpoint{2.873721in}{1.133916in}}{\pgfqpoint{2.881621in}{1.137188in}}{\pgfqpoint{2.887445in}{1.143012in}}%
\pgfpathcurveto{\pgfqpoint{2.893269in}{1.148836in}}{\pgfqpoint{2.896541in}{1.156736in}}{\pgfqpoint{2.896541in}{1.164973in}}%
\pgfpathcurveto{\pgfqpoint{2.896541in}{1.173209in}}{\pgfqpoint{2.893269in}{1.181109in}}{\pgfqpoint{2.887445in}{1.186933in}}%
\pgfpathcurveto{\pgfqpoint{2.881621in}{1.192757in}}{\pgfqpoint{2.873721in}{1.196029in}}{\pgfqpoint{2.865485in}{1.196029in}}%
\pgfpathcurveto{\pgfqpoint{2.857248in}{1.196029in}}{\pgfqpoint{2.849348in}{1.192757in}}{\pgfqpoint{2.843524in}{1.186933in}}%
\pgfpathcurveto{\pgfqpoint{2.837700in}{1.181109in}}{\pgfqpoint{2.834428in}{1.173209in}}{\pgfqpoint{2.834428in}{1.164973in}}%
\pgfpathcurveto{\pgfqpoint{2.834428in}{1.156736in}}{\pgfqpoint{2.837700in}{1.148836in}}{\pgfqpoint{2.843524in}{1.143012in}}%
\pgfpathcurveto{\pgfqpoint{2.849348in}{1.137188in}}{\pgfqpoint{2.857248in}{1.133916in}}{\pgfqpoint{2.865485in}{1.133916in}}%
\pgfpathclose%
\pgfusepath{stroke,fill}%
\end{pgfscope}%
\begin{pgfscope}%
\pgfpathrectangle{\pgfqpoint{0.100000in}{0.212622in}}{\pgfqpoint{3.696000in}{3.696000in}}%
\pgfusepath{clip}%
\pgfsetbuttcap%
\pgfsetroundjoin%
\definecolor{currentfill}{rgb}{0.121569,0.466667,0.705882}%
\pgfsetfillcolor{currentfill}%
\pgfsetfillopacity{0.998712}%
\pgfsetlinewidth{1.003750pt}%
\definecolor{currentstroke}{rgb}{0.121569,0.466667,0.705882}%
\pgfsetstrokecolor{currentstroke}%
\pgfsetstrokeopacity{0.998712}%
\pgfsetdash{}{0pt}%
\pgfpathmoveto{\pgfqpoint{2.867931in}{1.132571in}}%
\pgfpathcurveto{\pgfqpoint{2.876168in}{1.132571in}}{\pgfqpoint{2.884068in}{1.135843in}}{\pgfqpoint{2.889892in}{1.141667in}}%
\pgfpathcurveto{\pgfqpoint{2.895716in}{1.147491in}}{\pgfqpoint{2.898988in}{1.155391in}}{\pgfqpoint{2.898988in}{1.163628in}}%
\pgfpathcurveto{\pgfqpoint{2.898988in}{1.171864in}}{\pgfqpoint{2.895716in}{1.179764in}}{\pgfqpoint{2.889892in}{1.185588in}}%
\pgfpathcurveto{\pgfqpoint{2.884068in}{1.191412in}}{\pgfqpoint{2.876168in}{1.194684in}}{\pgfqpoint{2.867931in}{1.194684in}}%
\pgfpathcurveto{\pgfqpoint{2.859695in}{1.194684in}}{\pgfqpoint{2.851795in}{1.191412in}}{\pgfqpoint{2.845971in}{1.185588in}}%
\pgfpathcurveto{\pgfqpoint{2.840147in}{1.179764in}}{\pgfqpoint{2.836875in}{1.171864in}}{\pgfqpoint{2.836875in}{1.163628in}}%
\pgfpathcurveto{\pgfqpoint{2.836875in}{1.155391in}}{\pgfqpoint{2.840147in}{1.147491in}}{\pgfqpoint{2.845971in}{1.141667in}}%
\pgfpathcurveto{\pgfqpoint{2.851795in}{1.135843in}}{\pgfqpoint{2.859695in}{1.132571in}}{\pgfqpoint{2.867931in}{1.132571in}}%
\pgfpathclose%
\pgfusepath{stroke,fill}%
\end{pgfscope}%
\begin{pgfscope}%
\pgfpathrectangle{\pgfqpoint{0.100000in}{0.212622in}}{\pgfqpoint{3.696000in}{3.696000in}}%
\pgfusepath{clip}%
\pgfsetbuttcap%
\pgfsetroundjoin%
\definecolor{currentfill}{rgb}{0.121569,0.466667,0.705882}%
\pgfsetfillcolor{currentfill}%
\pgfsetfillopacity{0.999304}%
\pgfsetlinewidth{1.003750pt}%
\definecolor{currentstroke}{rgb}{0.121569,0.466667,0.705882}%
\pgfsetstrokecolor{currentstroke}%
\pgfsetstrokeopacity{0.999304}%
\pgfsetdash{}{0pt}%
\pgfpathmoveto{\pgfqpoint{2.869265in}{1.131816in}}%
\pgfpathcurveto{\pgfqpoint{2.877501in}{1.131816in}}{\pgfqpoint{2.885401in}{1.135088in}}{\pgfqpoint{2.891225in}{1.140912in}}%
\pgfpathcurveto{\pgfqpoint{2.897049in}{1.146736in}}{\pgfqpoint{2.900321in}{1.154636in}}{\pgfqpoint{2.900321in}{1.162873in}}%
\pgfpathcurveto{\pgfqpoint{2.900321in}{1.171109in}}{\pgfqpoint{2.897049in}{1.179009in}}{\pgfqpoint{2.891225in}{1.184833in}}%
\pgfpathcurveto{\pgfqpoint{2.885401in}{1.190657in}}{\pgfqpoint{2.877501in}{1.193929in}}{\pgfqpoint{2.869265in}{1.193929in}}%
\pgfpathcurveto{\pgfqpoint{2.861029in}{1.193929in}}{\pgfqpoint{2.853129in}{1.190657in}}{\pgfqpoint{2.847305in}{1.184833in}}%
\pgfpathcurveto{\pgfqpoint{2.841481in}{1.179009in}}{\pgfqpoint{2.838208in}{1.171109in}}{\pgfqpoint{2.838208in}{1.162873in}}%
\pgfpathcurveto{\pgfqpoint{2.838208in}{1.154636in}}{\pgfqpoint{2.841481in}{1.146736in}}{\pgfqpoint{2.847305in}{1.140912in}}%
\pgfpathcurveto{\pgfqpoint{2.853129in}{1.135088in}}{\pgfqpoint{2.861029in}{1.131816in}}{\pgfqpoint{2.869265in}{1.131816in}}%
\pgfpathclose%
\pgfusepath{stroke,fill}%
\end{pgfscope}%
\begin{pgfscope}%
\pgfpathrectangle{\pgfqpoint{0.100000in}{0.212622in}}{\pgfqpoint{3.696000in}{3.696000in}}%
\pgfusepath{clip}%
\pgfsetbuttcap%
\pgfsetroundjoin%
\definecolor{currentfill}{rgb}{0.121569,0.466667,0.705882}%
\pgfsetfillcolor{currentfill}%
\pgfsetfillopacity{0.999624}%
\pgfsetlinewidth{1.003750pt}%
\definecolor{currentstroke}{rgb}{0.121569,0.466667,0.705882}%
\pgfsetstrokecolor{currentstroke}%
\pgfsetstrokeopacity{0.999624}%
\pgfsetdash{}{0pt}%
\pgfpathmoveto{\pgfqpoint{2.870022in}{1.131447in}}%
\pgfpathcurveto{\pgfqpoint{2.878258in}{1.131447in}}{\pgfqpoint{2.886158in}{1.134719in}}{\pgfqpoint{2.891982in}{1.140543in}}%
\pgfpathcurveto{\pgfqpoint{2.897806in}{1.146367in}}{\pgfqpoint{2.901078in}{1.154267in}}{\pgfqpoint{2.901078in}{1.162503in}}%
\pgfpathcurveto{\pgfqpoint{2.901078in}{1.170740in}}{\pgfqpoint{2.897806in}{1.178640in}}{\pgfqpoint{2.891982in}{1.184464in}}%
\pgfpathcurveto{\pgfqpoint{2.886158in}{1.190288in}}{\pgfqpoint{2.878258in}{1.193560in}}{\pgfqpoint{2.870022in}{1.193560in}}%
\pgfpathcurveto{\pgfqpoint{2.861785in}{1.193560in}}{\pgfqpoint{2.853885in}{1.190288in}}{\pgfqpoint{2.848061in}{1.184464in}}%
\pgfpathcurveto{\pgfqpoint{2.842237in}{1.178640in}}{\pgfqpoint{2.838965in}{1.170740in}}{\pgfqpoint{2.838965in}{1.162503in}}%
\pgfpathcurveto{\pgfqpoint{2.838965in}{1.154267in}}{\pgfqpoint{2.842237in}{1.146367in}}{\pgfqpoint{2.848061in}{1.140543in}}%
\pgfpathcurveto{\pgfqpoint{2.853885in}{1.134719in}}{\pgfqpoint{2.861785in}{1.131447in}}{\pgfqpoint{2.870022in}{1.131447in}}%
\pgfpathclose%
\pgfusepath{stroke,fill}%
\end{pgfscope}%
\begin{pgfscope}%
\pgfpathrectangle{\pgfqpoint{0.100000in}{0.212622in}}{\pgfqpoint{3.696000in}{3.696000in}}%
\pgfusepath{clip}%
\pgfsetbuttcap%
\pgfsetroundjoin%
\definecolor{currentfill}{rgb}{0.121569,0.466667,0.705882}%
\pgfsetfillcolor{currentfill}%
\pgfsetfillopacity{0.999802}%
\pgfsetlinewidth{1.003750pt}%
\definecolor{currentstroke}{rgb}{0.121569,0.466667,0.705882}%
\pgfsetstrokecolor{currentstroke}%
\pgfsetstrokeopacity{0.999802}%
\pgfsetdash{}{0pt}%
\pgfpathmoveto{\pgfqpoint{2.870426in}{1.131221in}}%
\pgfpathcurveto{\pgfqpoint{2.878662in}{1.131221in}}{\pgfqpoint{2.886562in}{1.134493in}}{\pgfqpoint{2.892386in}{1.140317in}}%
\pgfpathcurveto{\pgfqpoint{2.898210in}{1.146141in}}{\pgfqpoint{2.901482in}{1.154041in}}{\pgfqpoint{2.901482in}{1.162277in}}%
\pgfpathcurveto{\pgfqpoint{2.901482in}{1.170513in}}{\pgfqpoint{2.898210in}{1.178413in}}{\pgfqpoint{2.892386in}{1.184237in}}%
\pgfpathcurveto{\pgfqpoint{2.886562in}{1.190061in}}{\pgfqpoint{2.878662in}{1.193334in}}{\pgfqpoint{2.870426in}{1.193334in}}%
\pgfpathcurveto{\pgfqpoint{2.862190in}{1.193334in}}{\pgfqpoint{2.854290in}{1.190061in}}{\pgfqpoint{2.848466in}{1.184237in}}%
\pgfpathcurveto{\pgfqpoint{2.842642in}{1.178413in}}{\pgfqpoint{2.839369in}{1.170513in}}{\pgfqpoint{2.839369in}{1.162277in}}%
\pgfpathcurveto{\pgfqpoint{2.839369in}{1.154041in}}{\pgfqpoint{2.842642in}{1.146141in}}{\pgfqpoint{2.848466in}{1.140317in}}%
\pgfpathcurveto{\pgfqpoint{2.854290in}{1.134493in}}{\pgfqpoint{2.862190in}{1.131221in}}{\pgfqpoint{2.870426in}{1.131221in}}%
\pgfpathclose%
\pgfusepath{stroke,fill}%
\end{pgfscope}%
\begin{pgfscope}%
\pgfpathrectangle{\pgfqpoint{0.100000in}{0.212622in}}{\pgfqpoint{3.696000in}{3.696000in}}%
\pgfusepath{clip}%
\pgfsetbuttcap%
\pgfsetroundjoin%
\definecolor{currentfill}{rgb}{0.121569,0.466667,0.705882}%
\pgfsetfillcolor{currentfill}%
\pgfsetfillopacity{0.999899}%
\pgfsetlinewidth{1.003750pt}%
\definecolor{currentstroke}{rgb}{0.121569,0.466667,0.705882}%
\pgfsetstrokecolor{currentstroke}%
\pgfsetstrokeopacity{0.999899}%
\pgfsetdash{}{0pt}%
\pgfpathmoveto{\pgfqpoint{2.870656in}{1.131115in}}%
\pgfpathcurveto{\pgfqpoint{2.878892in}{1.131115in}}{\pgfqpoint{2.886792in}{1.134387in}}{\pgfqpoint{2.892616in}{1.140211in}}%
\pgfpathcurveto{\pgfqpoint{2.898440in}{1.146035in}}{\pgfqpoint{2.901712in}{1.153935in}}{\pgfqpoint{2.901712in}{1.162171in}}%
\pgfpathcurveto{\pgfqpoint{2.901712in}{1.170407in}}{\pgfqpoint{2.898440in}{1.178307in}}{\pgfqpoint{2.892616in}{1.184131in}}%
\pgfpathcurveto{\pgfqpoint{2.886792in}{1.189955in}}{\pgfqpoint{2.878892in}{1.193228in}}{\pgfqpoint{2.870656in}{1.193228in}}%
\pgfpathcurveto{\pgfqpoint{2.862420in}{1.193228in}}{\pgfqpoint{2.854520in}{1.189955in}}{\pgfqpoint{2.848696in}{1.184131in}}%
\pgfpathcurveto{\pgfqpoint{2.842872in}{1.178307in}}{\pgfqpoint{2.839599in}{1.170407in}}{\pgfqpoint{2.839599in}{1.162171in}}%
\pgfpathcurveto{\pgfqpoint{2.839599in}{1.153935in}}{\pgfqpoint{2.842872in}{1.146035in}}{\pgfqpoint{2.848696in}{1.140211in}}%
\pgfpathcurveto{\pgfqpoint{2.854520in}{1.134387in}}{\pgfqpoint{2.862420in}{1.131115in}}{\pgfqpoint{2.870656in}{1.131115in}}%
\pgfpathclose%
\pgfusepath{stroke,fill}%
\end{pgfscope}%
\begin{pgfscope}%
\pgfpathrectangle{\pgfqpoint{0.100000in}{0.212622in}}{\pgfqpoint{3.696000in}{3.696000in}}%
\pgfusepath{clip}%
\pgfsetbuttcap%
\pgfsetroundjoin%
\definecolor{currentfill}{rgb}{0.121569,0.466667,0.705882}%
\pgfsetfillcolor{currentfill}%
\pgfsetfillopacity{0.999954}%
\pgfsetlinewidth{1.003750pt}%
\definecolor{currentstroke}{rgb}{0.121569,0.466667,0.705882}%
\pgfsetstrokecolor{currentstroke}%
\pgfsetstrokeopacity{0.999954}%
\pgfsetdash{}{0pt}%
\pgfpathmoveto{\pgfqpoint{2.870779in}{1.131050in}}%
\pgfpathcurveto{\pgfqpoint{2.879015in}{1.131050in}}{\pgfqpoint{2.886915in}{1.134322in}}{\pgfqpoint{2.892739in}{1.140146in}}%
\pgfpathcurveto{\pgfqpoint{2.898563in}{1.145970in}}{\pgfqpoint{2.901835in}{1.153870in}}{\pgfqpoint{2.901835in}{1.162106in}}%
\pgfpathcurveto{\pgfqpoint{2.901835in}{1.170343in}}{\pgfqpoint{2.898563in}{1.178243in}}{\pgfqpoint{2.892739in}{1.184067in}}%
\pgfpathcurveto{\pgfqpoint{2.886915in}{1.189890in}}{\pgfqpoint{2.879015in}{1.193163in}}{\pgfqpoint{2.870779in}{1.193163in}}%
\pgfpathcurveto{\pgfqpoint{2.862542in}{1.193163in}}{\pgfqpoint{2.854642in}{1.189890in}}{\pgfqpoint{2.848818in}{1.184067in}}%
\pgfpathcurveto{\pgfqpoint{2.842994in}{1.178243in}}{\pgfqpoint{2.839722in}{1.170343in}}{\pgfqpoint{2.839722in}{1.162106in}}%
\pgfpathcurveto{\pgfqpoint{2.839722in}{1.153870in}}{\pgfqpoint{2.842994in}{1.145970in}}{\pgfqpoint{2.848818in}{1.140146in}}%
\pgfpathcurveto{\pgfqpoint{2.854642in}{1.134322in}}{\pgfqpoint{2.862542in}{1.131050in}}{\pgfqpoint{2.870779in}{1.131050in}}%
\pgfpathclose%
\pgfusepath{stroke,fill}%
\end{pgfscope}%
\begin{pgfscope}%
\pgfpathrectangle{\pgfqpoint{0.100000in}{0.212622in}}{\pgfqpoint{3.696000in}{3.696000in}}%
\pgfusepath{clip}%
\pgfsetbuttcap%
\pgfsetroundjoin%
\definecolor{currentfill}{rgb}{0.121569,0.466667,0.705882}%
\pgfsetfillcolor{currentfill}%
\pgfsetfillopacity{0.999984}%
\pgfsetlinewidth{1.003750pt}%
\definecolor{currentstroke}{rgb}{0.121569,0.466667,0.705882}%
\pgfsetstrokecolor{currentstroke}%
\pgfsetstrokeopacity{0.999984}%
\pgfsetdash{}{0pt}%
\pgfpathmoveto{\pgfqpoint{2.870846in}{1.131014in}}%
\pgfpathcurveto{\pgfqpoint{2.879082in}{1.131014in}}{\pgfqpoint{2.886982in}{1.134286in}}{\pgfqpoint{2.892806in}{1.140110in}}%
\pgfpathcurveto{\pgfqpoint{2.898630in}{1.145934in}}{\pgfqpoint{2.901903in}{1.153834in}}{\pgfqpoint{2.901903in}{1.162071in}}%
\pgfpathcurveto{\pgfqpoint{2.901903in}{1.170307in}}{\pgfqpoint{2.898630in}{1.178207in}}{\pgfqpoint{2.892806in}{1.184031in}}%
\pgfpathcurveto{\pgfqpoint{2.886982in}{1.189855in}}{\pgfqpoint{2.879082in}{1.193127in}}{\pgfqpoint{2.870846in}{1.193127in}}%
\pgfpathcurveto{\pgfqpoint{2.862610in}{1.193127in}}{\pgfqpoint{2.854710in}{1.189855in}}{\pgfqpoint{2.848886in}{1.184031in}}%
\pgfpathcurveto{\pgfqpoint{2.843062in}{1.178207in}}{\pgfqpoint{2.839790in}{1.170307in}}{\pgfqpoint{2.839790in}{1.162071in}}%
\pgfpathcurveto{\pgfqpoint{2.839790in}{1.153834in}}{\pgfqpoint{2.843062in}{1.145934in}}{\pgfqpoint{2.848886in}{1.140110in}}%
\pgfpathcurveto{\pgfqpoint{2.854710in}{1.134286in}}{\pgfqpoint{2.862610in}{1.131014in}}{\pgfqpoint{2.870846in}{1.131014in}}%
\pgfpathclose%
\pgfusepath{stroke,fill}%
\end{pgfscope}%
\begin{pgfscope}%
\pgfpathrectangle{\pgfqpoint{0.100000in}{0.212622in}}{\pgfqpoint{3.696000in}{3.696000in}}%
\pgfusepath{clip}%
\pgfsetbuttcap%
\pgfsetroundjoin%
\definecolor{currentfill}{rgb}{0.121569,0.466667,0.705882}%
\pgfsetfillcolor{currentfill}%
\pgfsetlinewidth{1.003750pt}%
\definecolor{currentstroke}{rgb}{0.121569,0.466667,0.705882}%
\pgfsetstrokecolor{currentstroke}%
\pgfsetdash{}{0pt}%
\pgfpathmoveto{\pgfqpoint{2.870885in}{1.130998in}}%
\pgfpathcurveto{\pgfqpoint{2.879121in}{1.130998in}}{\pgfqpoint{2.887021in}{1.134270in}}{\pgfqpoint{2.892845in}{1.140094in}}%
\pgfpathcurveto{\pgfqpoint{2.898669in}{1.145918in}}{\pgfqpoint{2.901941in}{1.153818in}}{\pgfqpoint{2.901941in}{1.162054in}}%
\pgfpathcurveto{\pgfqpoint{2.901941in}{1.170291in}}{\pgfqpoint{2.898669in}{1.178191in}}{\pgfqpoint{2.892845in}{1.184015in}}%
\pgfpathcurveto{\pgfqpoint{2.887021in}{1.189839in}}{\pgfqpoint{2.879121in}{1.193111in}}{\pgfqpoint{2.870885in}{1.193111in}}%
\pgfpathcurveto{\pgfqpoint{2.862649in}{1.193111in}}{\pgfqpoint{2.854749in}{1.189839in}}{\pgfqpoint{2.848925in}{1.184015in}}%
\pgfpathcurveto{\pgfqpoint{2.843101in}{1.178191in}}{\pgfqpoint{2.839828in}{1.170291in}}{\pgfqpoint{2.839828in}{1.162054in}}%
\pgfpathcurveto{\pgfqpoint{2.839828in}{1.153818in}}{\pgfqpoint{2.843101in}{1.145918in}}{\pgfqpoint{2.848925in}{1.140094in}}%
\pgfpathcurveto{\pgfqpoint{2.854749in}{1.134270in}}{\pgfqpoint{2.862649in}{1.130998in}}{\pgfqpoint{2.870885in}{1.130998in}}%
\pgfpathclose%
\pgfusepath{stroke,fill}%
\end{pgfscope}%
\begin{pgfscope}%
\definecolor{textcolor}{rgb}{0.000000,0.000000,0.000000}%
\pgfsetstrokecolor{textcolor}%
\pgfsetfillcolor{textcolor}%
\pgftext[x=1.948000in,y=3.991956in,,base]{\color{textcolor}\rmfamily\fontsize{12.000000}{14.400000}\selectfont ROLEQ}%
\end{pgfscope}%
\begin{pgfscope}%
\pgfpathrectangle{\pgfqpoint{0.100000in}{0.212622in}}{\pgfqpoint{3.696000in}{3.696000in}}%
\pgfusepath{clip}%
\pgfsetbuttcap%
\pgfsetroundjoin%
\definecolor{currentfill}{rgb}{1.000000,0.498039,0.054902}%
\pgfsetfillcolor{currentfill}%
\pgfsetlinewidth{1.003750pt}%
\definecolor{currentstroke}{rgb}{1.000000,0.498039,0.054902}%
\pgfsetstrokecolor{currentstroke}%
\pgfsetdash{}{0pt}%
\pgfsys@defobject{currentmarker}{\pgfqpoint{-0.031056in}{-0.031056in}}{\pgfqpoint{0.031056in}{0.031056in}}{%
\pgfpathmoveto{\pgfqpoint{0.000000in}{-0.031056in}}%
\pgfpathcurveto{\pgfqpoint{0.008236in}{-0.031056in}}{\pgfqpoint{0.016136in}{-0.027784in}}{\pgfqpoint{0.021960in}{-0.021960in}}%
\pgfpathcurveto{\pgfqpoint{0.027784in}{-0.016136in}}{\pgfqpoint{0.031056in}{-0.008236in}}{\pgfqpoint{0.031056in}{0.000000in}}%
\pgfpathcurveto{\pgfqpoint{0.031056in}{0.008236in}}{\pgfqpoint{0.027784in}{0.016136in}}{\pgfqpoint{0.021960in}{0.021960in}}%
\pgfpathcurveto{\pgfqpoint{0.016136in}{0.027784in}}{\pgfqpoint{0.008236in}{0.031056in}}{\pgfqpoint{0.000000in}{0.031056in}}%
\pgfpathcurveto{\pgfqpoint{-0.008236in}{0.031056in}}{\pgfqpoint{-0.016136in}{0.027784in}}{\pgfqpoint{-0.021960in}{0.021960in}}%
\pgfpathcurveto{\pgfqpoint{-0.027784in}{0.016136in}}{\pgfqpoint{-0.031056in}{0.008236in}}{\pgfqpoint{-0.031056in}{0.000000in}}%
\pgfpathcurveto{\pgfqpoint{-0.031056in}{-0.008236in}}{\pgfqpoint{-0.027784in}{-0.016136in}}{\pgfqpoint{-0.021960in}{-0.021960in}}%
\pgfpathcurveto{\pgfqpoint{-0.016136in}{-0.027784in}}{\pgfqpoint{-0.008236in}{-0.031056in}}{\pgfqpoint{0.000000in}{-0.031056in}}%
\pgfpathclose%
\pgfusepath{stroke,fill}%
}%
\begin{pgfscope}%
\pgfsys@transformshift{2.870907in}{1.162048in}%
\pgfsys@useobject{currentmarker}{}%
\end{pgfscope}%
\end{pgfscope}%
\begin{pgfscope}%
\pgfsetbuttcap%
\pgfsetmiterjoin%
\definecolor{currentfill}{rgb}{1.000000,1.000000,1.000000}%
\pgfsetfillcolor{currentfill}%
\pgfsetfillopacity{0.800000}%
\pgfsetlinewidth{1.003750pt}%
\definecolor{currentstroke}{rgb}{0.800000,0.800000,0.800000}%
\pgfsetstrokecolor{currentstroke}%
\pgfsetstrokeopacity{0.800000}%
\pgfsetdash{}{0pt}%
\pgfpathmoveto{\pgfqpoint{2.104889in}{3.216678in}}%
\pgfpathlineto{\pgfqpoint{3.698778in}{3.216678in}}%
\pgfpathquadraticcurveto{\pgfqpoint{3.726556in}{3.216678in}}{\pgfqpoint{3.726556in}{3.244456in}}%
\pgfpathlineto{\pgfqpoint{3.726556in}{3.811400in}}%
\pgfpathquadraticcurveto{\pgfqpoint{3.726556in}{3.839178in}}{\pgfqpoint{3.698778in}{3.839178in}}%
\pgfpathlineto{\pgfqpoint{2.104889in}{3.839178in}}%
\pgfpathquadraticcurveto{\pgfqpoint{2.077111in}{3.839178in}}{\pgfqpoint{2.077111in}{3.811400in}}%
\pgfpathlineto{\pgfqpoint{2.077111in}{3.244456in}}%
\pgfpathquadraticcurveto{\pgfqpoint{2.077111in}{3.216678in}}{\pgfqpoint{2.104889in}{3.216678in}}%
\pgfpathclose%
\pgfusepath{stroke,fill}%
\end{pgfscope}%
\begin{pgfscope}%
\pgfsetrectcap%
\pgfsetroundjoin%
\pgfsetlinewidth{1.505625pt}%
\definecolor{currentstroke}{rgb}{0.121569,0.466667,0.705882}%
\pgfsetstrokecolor{currentstroke}%
\pgfsetdash{}{0pt}%
\pgfpathmoveto{\pgfqpoint{2.132667in}{3.735011in}}%
\pgfpathlineto{\pgfqpoint{2.410444in}{3.735011in}}%
\pgfusepath{stroke}%
\end{pgfscope}%
\begin{pgfscope}%
\definecolor{textcolor}{rgb}{0.000000,0.000000,0.000000}%
\pgfsetstrokecolor{textcolor}%
\pgfsetfillcolor{textcolor}%
\pgftext[x=2.521555in,y=3.686400in,left,base]{\color{textcolor}\rmfamily\fontsize{10.000000}{12.000000}\selectfont Ground truth}%
\end{pgfscope}%
\begin{pgfscope}%
\pgfsetbuttcap%
\pgfsetroundjoin%
\definecolor{currentfill}{rgb}{0.121569,0.466667,0.705882}%
\pgfsetfillcolor{currentfill}%
\pgfsetlinewidth{1.003750pt}%
\definecolor{currentstroke}{rgb}{0.121569,0.466667,0.705882}%
\pgfsetstrokecolor{currentstroke}%
\pgfsetdash{}{0pt}%
\pgfsys@defobject{currentmarker}{\pgfqpoint{-0.031056in}{-0.031056in}}{\pgfqpoint{0.031056in}{0.031056in}}{%
\pgfpathmoveto{\pgfqpoint{0.000000in}{-0.031056in}}%
\pgfpathcurveto{\pgfqpoint{0.008236in}{-0.031056in}}{\pgfqpoint{0.016136in}{-0.027784in}}{\pgfqpoint{0.021960in}{-0.021960in}}%
\pgfpathcurveto{\pgfqpoint{0.027784in}{-0.016136in}}{\pgfqpoint{0.031056in}{-0.008236in}}{\pgfqpoint{0.031056in}{0.000000in}}%
\pgfpathcurveto{\pgfqpoint{0.031056in}{0.008236in}}{\pgfqpoint{0.027784in}{0.016136in}}{\pgfqpoint{0.021960in}{0.021960in}}%
\pgfpathcurveto{\pgfqpoint{0.016136in}{0.027784in}}{\pgfqpoint{0.008236in}{0.031056in}}{\pgfqpoint{0.000000in}{0.031056in}}%
\pgfpathcurveto{\pgfqpoint{-0.008236in}{0.031056in}}{\pgfqpoint{-0.016136in}{0.027784in}}{\pgfqpoint{-0.021960in}{0.021960in}}%
\pgfpathcurveto{\pgfqpoint{-0.027784in}{0.016136in}}{\pgfqpoint{-0.031056in}{0.008236in}}{\pgfqpoint{-0.031056in}{0.000000in}}%
\pgfpathcurveto{\pgfqpoint{-0.031056in}{-0.008236in}}{\pgfqpoint{-0.027784in}{-0.016136in}}{\pgfqpoint{-0.021960in}{-0.021960in}}%
\pgfpathcurveto{\pgfqpoint{-0.016136in}{-0.027784in}}{\pgfqpoint{-0.008236in}{-0.031056in}}{\pgfqpoint{0.000000in}{-0.031056in}}%
\pgfpathclose%
\pgfusepath{stroke,fill}%
}%
\begin{pgfscope}%
\pgfsys@transformshift{2.271555in}{3.529248in}%
\pgfsys@useobject{currentmarker}{}%
\end{pgfscope}%
\end{pgfscope}%
\begin{pgfscope}%
\definecolor{textcolor}{rgb}{0.000000,0.000000,0.000000}%
\pgfsetstrokecolor{textcolor}%
\pgfsetfillcolor{textcolor}%
\pgftext[x=2.521555in,y=3.492789in,left,base]{\color{textcolor}\rmfamily\fontsize{10.000000}{12.000000}\selectfont Estimated position}%
\end{pgfscope}%
\begin{pgfscope}%
\pgfsetbuttcap%
\pgfsetroundjoin%
\definecolor{currentfill}{rgb}{1.000000,0.498039,0.054902}%
\pgfsetfillcolor{currentfill}%
\pgfsetlinewidth{1.003750pt}%
\definecolor{currentstroke}{rgb}{1.000000,0.498039,0.054902}%
\pgfsetstrokecolor{currentstroke}%
\pgfsetdash{}{0pt}%
\pgfsys@defobject{currentmarker}{\pgfqpoint{-0.031056in}{-0.031056in}}{\pgfqpoint{0.031056in}{0.031056in}}{%
\pgfpathmoveto{\pgfqpoint{0.000000in}{-0.031056in}}%
\pgfpathcurveto{\pgfqpoint{0.008236in}{-0.031056in}}{\pgfqpoint{0.016136in}{-0.027784in}}{\pgfqpoint{0.021960in}{-0.021960in}}%
\pgfpathcurveto{\pgfqpoint{0.027784in}{-0.016136in}}{\pgfqpoint{0.031056in}{-0.008236in}}{\pgfqpoint{0.031056in}{0.000000in}}%
\pgfpathcurveto{\pgfqpoint{0.031056in}{0.008236in}}{\pgfqpoint{0.027784in}{0.016136in}}{\pgfqpoint{0.021960in}{0.021960in}}%
\pgfpathcurveto{\pgfqpoint{0.016136in}{0.027784in}}{\pgfqpoint{0.008236in}{0.031056in}}{\pgfqpoint{0.000000in}{0.031056in}}%
\pgfpathcurveto{\pgfqpoint{-0.008236in}{0.031056in}}{\pgfqpoint{-0.016136in}{0.027784in}}{\pgfqpoint{-0.021960in}{0.021960in}}%
\pgfpathcurveto{\pgfqpoint{-0.027784in}{0.016136in}}{\pgfqpoint{-0.031056in}{0.008236in}}{\pgfqpoint{-0.031056in}{0.000000in}}%
\pgfpathcurveto{\pgfqpoint{-0.031056in}{-0.008236in}}{\pgfqpoint{-0.027784in}{-0.016136in}}{\pgfqpoint{-0.021960in}{-0.021960in}}%
\pgfpathcurveto{\pgfqpoint{-0.016136in}{-0.027784in}}{\pgfqpoint{-0.008236in}{-0.031056in}}{\pgfqpoint{0.000000in}{-0.031056in}}%
\pgfpathclose%
\pgfusepath{stroke,fill}%
}%
\begin{pgfscope}%
\pgfsys@transformshift{2.271555in}{3.335637in}%
\pgfsys@useobject{currentmarker}{}%
\end{pgfscope}%
\end{pgfscope}%
\begin{pgfscope}%
\definecolor{textcolor}{rgb}{0.000000,0.000000,0.000000}%
\pgfsetstrokecolor{textcolor}%
\pgfsetfillcolor{textcolor}%
\pgftext[x=2.521555in,y=3.299178in,left,base]{\color{textcolor}\rmfamily\fontsize{10.000000}{12.000000}\selectfont Estimated turn}%
\end{pgfscope}%
\end{pgfpicture}%
\makeatother%
\endgroup%
}
        \caption{ ROLEQ's 3D position estimation had the lowest turn error for the 4-meter line experiment. }
        \label{fig:line4_3D}
    \end{subfigure}
    \caption{Position estimation by the best performing algorithms in the 4-meter line experiment.}
    \label{fig:line4}
\end{figure}

% \subsubsection{16 meter}

% For the 16-meter line experiment, the FAMC algorithm which had the lowest displacement error with an average of 0.43 meters (2.69\% of error margin), and Mahony with an average of 2.11 meters of turn error (13.21\% of error margin).

% \begin{figure}[!h]
%     \centering
%     \begin{table}[H]
    \begin{center}
        \resizebox{1\linewidth}{!}{
            \begin{tabular}[t]{lcccc}
                \hline
                Algorithm   & Displacement Error[$m$] & Displacement Error[\%] & Turn Error[$m$] & Turn Error[\%] \\
                \hline
                AngularRate & 0.89                    & 5.59                   & 3.97            & 24.78          \\            AQUA            & 1.47  & 9.21 & 4.23 & 26.46              \\            Complementary            & 0.50  & 3.09 & 2.16 & 13.47              \\            Davenport            & 0.51  & 3.17 & 2.16 & 13.52              \\            EKF            & 0.72  & 4.53 & 2.22 & 13.86              \\            FAMC            & 0.43  & 2.69 & 2.16 & 13.49              \\            FLAE            & 0.51  & 3.17 & 2.16 & 13.52              \\            Fourati            & 1.09  & 6.83 & 4.80 & 29.98              \\            Madgwick            & 0.49  & 3.05 & 2.15 & 13.47              \\            Mahony            & 0.48  & 3.00 & 2.11 & 13.21              \\            OLEQ            & 0.68  & 4.28 & 3.54 & 22.14              \\            QUEST            & 1.67  & 10.44 & 3.79 & 23.72              \\            ROLEQ            & 0.72  & 4.51 & 3.56 & 22.26              \\            SAAM            & 0.48  & 3.02 & 2.14 & 13.35              \\            Tilt            & 0.48  & 3.02 & 2.14 & 13.35              \\
                \hline
                Average     & 0.74                    & 4.64                   & 2.89            & 18.04
            \end{tabular}
        }
        \caption{Accelerometer Specifications. }
        \label{tab:accelerometer_specification}
    \end{center}
\end{table}
% \end{figure}

% \begin{figure}[!h]
%     \centering
%     \begin{subfigure}{0.49\textwidth}
%         \centering
%         \resizebox{1\linewidth}{!}{%% Creator: Matplotlib, PGF backend
%%
%% To include the figure in your LaTeX document, write
%%   \input{<filename>.pgf}
%%
%% Make sure the required packages are loaded in your preamble
%%   \usepackage{pgf}
%%
%% and, on pdftex
%%   \usepackage[utf8]{inputenc}\DeclareUnicodeCharacter{2212}{-}
%%
%% or, on luatex and xetex
%%   \usepackage{unicode-math}
%%
%% Figures using additional raster images can only be included by \input if
%% they are in the same directory as the main LaTeX file. For loading figures
%% from other directories you can use the `import` package
%%   \usepackage{import}
%%
%% and then include the figures with
%%   \import{<path to file>}{<filename>.pgf}
%%
%% Matplotlib used the following preamble
%%   \usepackage{fontspec}
%%
\begingroup%
\makeatletter%
\begin{pgfpicture}%
\pgfpathrectangle{\pgfpointorigin}{\pgfqpoint{5.629167in}{4.325970in}}%
\pgfusepath{use as bounding box, clip}%
\begin{pgfscope}%
\pgfsetbuttcap%
\pgfsetmiterjoin%
\definecolor{currentfill}{rgb}{1.000000,1.000000,1.000000}%
\pgfsetfillcolor{currentfill}%
\pgfsetlinewidth{0.000000pt}%
\definecolor{currentstroke}{rgb}{1.000000,1.000000,1.000000}%
\pgfsetstrokecolor{currentstroke}%
\pgfsetdash{}{0pt}%
\pgfpathmoveto{\pgfqpoint{0.000000in}{0.000000in}}%
\pgfpathlineto{\pgfqpoint{5.629167in}{0.000000in}}%
\pgfpathlineto{\pgfqpoint{5.629167in}{4.325970in}}%
\pgfpathlineto{\pgfqpoint{0.000000in}{4.325970in}}%
\pgfpathclose%
\pgfusepath{fill}%
\end{pgfscope}%
\begin{pgfscope}%
\pgfsetbuttcap%
\pgfsetmiterjoin%
\definecolor{currentfill}{rgb}{1.000000,1.000000,1.000000}%
\pgfsetfillcolor{currentfill}%
\pgfsetlinewidth{0.000000pt}%
\definecolor{currentstroke}{rgb}{0.000000,0.000000,0.000000}%
\pgfsetstrokecolor{currentstroke}%
\pgfsetstrokeopacity{0.000000}%
\pgfsetdash{}{0pt}%
\pgfpathmoveto{\pgfqpoint{0.569167in}{0.515000in}}%
\pgfpathlineto{\pgfqpoint{5.529167in}{0.515000in}}%
\pgfpathlineto{\pgfqpoint{5.529167in}{4.211000in}}%
\pgfpathlineto{\pgfqpoint{0.569167in}{4.211000in}}%
\pgfpathclose%
\pgfusepath{fill}%
\end{pgfscope}%
\begin{pgfscope}%
\pgfpathrectangle{\pgfqpoint{0.569167in}{0.515000in}}{\pgfqpoint{4.960000in}{3.696000in}}%
\pgfusepath{clip}%
\pgfsetbuttcap%
\pgfsetroundjoin%
\definecolor{currentfill}{rgb}{0.121569,0.466667,0.705882}%
\pgfsetfillcolor{currentfill}%
\pgfsetlinewidth{1.003750pt}%
\definecolor{currentstroke}{rgb}{0.121569,0.466667,0.705882}%
\pgfsetstrokecolor{currentstroke}%
\pgfsetdash{}{0pt}%
\pgfsys@defobject{currentmarker}{\pgfqpoint{-0.041667in}{-0.041667in}}{\pgfqpoint{0.041667in}{0.041667in}}{%
\pgfpathmoveto{\pgfqpoint{0.000000in}{-0.041667in}}%
\pgfpathcurveto{\pgfqpoint{0.011050in}{-0.041667in}}{\pgfqpoint{0.021649in}{-0.037276in}}{\pgfqpoint{0.029463in}{-0.029463in}}%
\pgfpathcurveto{\pgfqpoint{0.037276in}{-0.021649in}}{\pgfqpoint{0.041667in}{-0.011050in}}{\pgfqpoint{0.041667in}{0.000000in}}%
\pgfpathcurveto{\pgfqpoint{0.041667in}{0.011050in}}{\pgfqpoint{0.037276in}{0.021649in}}{\pgfqpoint{0.029463in}{0.029463in}}%
\pgfpathcurveto{\pgfqpoint{0.021649in}{0.037276in}}{\pgfqpoint{0.011050in}{0.041667in}}{\pgfqpoint{0.000000in}{0.041667in}}%
\pgfpathcurveto{\pgfqpoint{-0.011050in}{0.041667in}}{\pgfqpoint{-0.021649in}{0.037276in}}{\pgfqpoint{-0.029463in}{0.029463in}}%
\pgfpathcurveto{\pgfqpoint{-0.037276in}{0.021649in}}{\pgfqpoint{-0.041667in}{0.011050in}}{\pgfqpoint{-0.041667in}{0.000000in}}%
\pgfpathcurveto{\pgfqpoint{-0.041667in}{-0.011050in}}{\pgfqpoint{-0.037276in}{-0.021649in}}{\pgfqpoint{-0.029463in}{-0.029463in}}%
\pgfpathcurveto{\pgfqpoint{-0.021649in}{-0.037276in}}{\pgfqpoint{-0.011050in}{-0.041667in}}{\pgfqpoint{0.000000in}{-0.041667in}}%
\pgfpathclose%
\pgfusepath{stroke,fill}%
}%
\begin{pgfscope}%
\pgfsys@transformshift{0.794632in}{2.394379in}%
\pgfsys@useobject{currentmarker}{}%
\end{pgfscope}%
\begin{pgfscope}%
\pgfsys@transformshift{0.794639in}{2.394375in}%
\pgfsys@useobject{currentmarker}{}%
\end{pgfscope}%
\begin{pgfscope}%
\pgfsys@transformshift{0.794643in}{2.394375in}%
\pgfsys@useobject{currentmarker}{}%
\end{pgfscope}%
\begin{pgfscope}%
\pgfsys@transformshift{0.794645in}{2.394374in}%
\pgfsys@useobject{currentmarker}{}%
\end{pgfscope}%
\begin{pgfscope}%
\pgfsys@transformshift{0.794647in}{2.394374in}%
\pgfsys@useobject{currentmarker}{}%
\end{pgfscope}%
\begin{pgfscope}%
\pgfsys@transformshift{0.794647in}{2.394374in}%
\pgfsys@useobject{currentmarker}{}%
\end{pgfscope}%
\begin{pgfscope}%
\pgfsys@transformshift{0.794648in}{2.394374in}%
\pgfsys@useobject{currentmarker}{}%
\end{pgfscope}%
\begin{pgfscope}%
\pgfsys@transformshift{0.794648in}{2.394374in}%
\pgfsys@useobject{currentmarker}{}%
\end{pgfscope}%
\begin{pgfscope}%
\pgfsys@transformshift{0.794648in}{2.394374in}%
\pgfsys@useobject{currentmarker}{}%
\end{pgfscope}%
\begin{pgfscope}%
\pgfsys@transformshift{0.794648in}{2.394374in}%
\pgfsys@useobject{currentmarker}{}%
\end{pgfscope}%
\begin{pgfscope}%
\pgfsys@transformshift{0.794648in}{2.394374in}%
\pgfsys@useobject{currentmarker}{}%
\end{pgfscope}%
\begin{pgfscope}%
\pgfsys@transformshift{0.794648in}{2.394374in}%
\pgfsys@useobject{currentmarker}{}%
\end{pgfscope}%
\begin{pgfscope}%
\pgfsys@transformshift{0.794648in}{2.394374in}%
\pgfsys@useobject{currentmarker}{}%
\end{pgfscope}%
\begin{pgfscope}%
\pgfsys@transformshift{0.794648in}{2.394374in}%
\pgfsys@useobject{currentmarker}{}%
\end{pgfscope}%
\begin{pgfscope}%
\pgfsys@transformshift{0.794648in}{2.394374in}%
\pgfsys@useobject{currentmarker}{}%
\end{pgfscope}%
\begin{pgfscope}%
\pgfsys@transformshift{0.794648in}{2.394374in}%
\pgfsys@useobject{currentmarker}{}%
\end{pgfscope}%
\begin{pgfscope}%
\pgfsys@transformshift{0.794648in}{2.394374in}%
\pgfsys@useobject{currentmarker}{}%
\end{pgfscope}%
\begin{pgfscope}%
\pgfsys@transformshift{0.794648in}{2.394374in}%
\pgfsys@useobject{currentmarker}{}%
\end{pgfscope}%
\begin{pgfscope}%
\pgfsys@transformshift{0.794648in}{2.394374in}%
\pgfsys@useobject{currentmarker}{}%
\end{pgfscope}%
\begin{pgfscope}%
\pgfsys@transformshift{0.794648in}{2.394374in}%
\pgfsys@useobject{currentmarker}{}%
\end{pgfscope}%
\begin{pgfscope}%
\pgfsys@transformshift{0.794648in}{2.394374in}%
\pgfsys@useobject{currentmarker}{}%
\end{pgfscope}%
\begin{pgfscope}%
\pgfsys@transformshift{0.794648in}{2.394374in}%
\pgfsys@useobject{currentmarker}{}%
\end{pgfscope}%
\begin{pgfscope}%
\pgfsys@transformshift{0.794648in}{2.394374in}%
\pgfsys@useobject{currentmarker}{}%
\end{pgfscope}%
\begin{pgfscope}%
\pgfsys@transformshift{0.794648in}{2.394374in}%
\pgfsys@useobject{currentmarker}{}%
\end{pgfscope}%
\begin{pgfscope}%
\pgfsys@transformshift{0.794648in}{2.394374in}%
\pgfsys@useobject{currentmarker}{}%
\end{pgfscope}%
\begin{pgfscope}%
\pgfsys@transformshift{0.794648in}{2.394374in}%
\pgfsys@useobject{currentmarker}{}%
\end{pgfscope}%
\begin{pgfscope}%
\pgfsys@transformshift{0.794648in}{2.394374in}%
\pgfsys@useobject{currentmarker}{}%
\end{pgfscope}%
\begin{pgfscope}%
\pgfsys@transformshift{0.794648in}{2.394374in}%
\pgfsys@useobject{currentmarker}{}%
\end{pgfscope}%
\begin{pgfscope}%
\pgfsys@transformshift{0.794648in}{2.394374in}%
\pgfsys@useobject{currentmarker}{}%
\end{pgfscope}%
\begin{pgfscope}%
\pgfsys@transformshift{0.794648in}{2.394374in}%
\pgfsys@useobject{currentmarker}{}%
\end{pgfscope}%
\begin{pgfscope}%
\pgfsys@transformshift{0.794648in}{2.394374in}%
\pgfsys@useobject{currentmarker}{}%
\end{pgfscope}%
\begin{pgfscope}%
\pgfsys@transformshift{0.794648in}{2.394374in}%
\pgfsys@useobject{currentmarker}{}%
\end{pgfscope}%
\begin{pgfscope}%
\pgfsys@transformshift{0.794648in}{2.394374in}%
\pgfsys@useobject{currentmarker}{}%
\end{pgfscope}%
\begin{pgfscope}%
\pgfsys@transformshift{0.794648in}{2.394374in}%
\pgfsys@useobject{currentmarker}{}%
\end{pgfscope}%
\begin{pgfscope}%
\pgfsys@transformshift{0.794648in}{2.394374in}%
\pgfsys@useobject{currentmarker}{}%
\end{pgfscope}%
\begin{pgfscope}%
\pgfsys@transformshift{0.794648in}{2.394374in}%
\pgfsys@useobject{currentmarker}{}%
\end{pgfscope}%
\begin{pgfscope}%
\pgfsys@transformshift{0.794648in}{2.394374in}%
\pgfsys@useobject{currentmarker}{}%
\end{pgfscope}%
\begin{pgfscope}%
\pgfsys@transformshift{0.794648in}{2.394374in}%
\pgfsys@useobject{currentmarker}{}%
\end{pgfscope}%
\begin{pgfscope}%
\pgfsys@transformshift{0.794648in}{2.394374in}%
\pgfsys@useobject{currentmarker}{}%
\end{pgfscope}%
\begin{pgfscope}%
\pgfsys@transformshift{0.795787in}{2.394285in}%
\pgfsys@useobject{currentmarker}{}%
\end{pgfscope}%
\begin{pgfscope}%
\pgfsys@transformshift{0.796410in}{2.394205in}%
\pgfsys@useobject{currentmarker}{}%
\end{pgfscope}%
\begin{pgfscope}%
\pgfsys@transformshift{0.796755in}{2.394198in}%
\pgfsys@useobject{currentmarker}{}%
\end{pgfscope}%
\begin{pgfscope}%
\pgfsys@transformshift{0.796945in}{2.394188in}%
\pgfsys@useobject{currentmarker}{}%
\end{pgfscope}%
\begin{pgfscope}%
\pgfsys@transformshift{0.797049in}{2.394187in}%
\pgfsys@useobject{currentmarker}{}%
\end{pgfscope}%
\begin{pgfscope}%
\pgfsys@transformshift{0.797107in}{2.394189in}%
\pgfsys@useobject{currentmarker}{}%
\end{pgfscope}%
\begin{pgfscope}%
\pgfsys@transformshift{0.797138in}{2.394191in}%
\pgfsys@useobject{currentmarker}{}%
\end{pgfscope}%
\begin{pgfscope}%
\pgfsys@transformshift{0.797156in}{2.394191in}%
\pgfsys@useobject{currentmarker}{}%
\end{pgfscope}%
\begin{pgfscope}%
\pgfsys@transformshift{0.797165in}{2.394191in}%
\pgfsys@useobject{currentmarker}{}%
\end{pgfscope}%
\begin{pgfscope}%
\pgfsys@transformshift{0.797171in}{2.394191in}%
\pgfsys@useobject{currentmarker}{}%
\end{pgfscope}%
\begin{pgfscope}%
\pgfsys@transformshift{0.797174in}{2.394191in}%
\pgfsys@useobject{currentmarker}{}%
\end{pgfscope}%
\begin{pgfscope}%
\pgfsys@transformshift{0.797175in}{2.394192in}%
\pgfsys@useobject{currentmarker}{}%
\end{pgfscope}%
\begin{pgfscope}%
\pgfsys@transformshift{0.797176in}{2.394192in}%
\pgfsys@useobject{currentmarker}{}%
\end{pgfscope}%
\begin{pgfscope}%
\pgfsys@transformshift{0.797177in}{2.394192in}%
\pgfsys@useobject{currentmarker}{}%
\end{pgfscope}%
\begin{pgfscope}%
\pgfsys@transformshift{0.797177in}{2.394192in}%
\pgfsys@useobject{currentmarker}{}%
\end{pgfscope}%
\begin{pgfscope}%
\pgfsys@transformshift{0.797177in}{2.394192in}%
\pgfsys@useobject{currentmarker}{}%
\end{pgfscope}%
\begin{pgfscope}%
\pgfsys@transformshift{0.797177in}{2.394192in}%
\pgfsys@useobject{currentmarker}{}%
\end{pgfscope}%
\begin{pgfscope}%
\pgfsys@transformshift{0.797177in}{2.394192in}%
\pgfsys@useobject{currentmarker}{}%
\end{pgfscope}%
\begin{pgfscope}%
\pgfsys@transformshift{0.797177in}{2.394192in}%
\pgfsys@useobject{currentmarker}{}%
\end{pgfscope}%
\begin{pgfscope}%
\pgfsys@transformshift{0.797177in}{2.394192in}%
\pgfsys@useobject{currentmarker}{}%
\end{pgfscope}%
\begin{pgfscope}%
\pgfsys@transformshift{0.797177in}{2.394192in}%
\pgfsys@useobject{currentmarker}{}%
\end{pgfscope}%
\begin{pgfscope}%
\pgfsys@transformshift{0.797177in}{2.394192in}%
\pgfsys@useobject{currentmarker}{}%
\end{pgfscope}%
\begin{pgfscope}%
\pgfsys@transformshift{0.797177in}{2.394192in}%
\pgfsys@useobject{currentmarker}{}%
\end{pgfscope}%
\begin{pgfscope}%
\pgfsys@transformshift{0.797177in}{2.394192in}%
\pgfsys@useobject{currentmarker}{}%
\end{pgfscope}%
\begin{pgfscope}%
\pgfsys@transformshift{0.797177in}{2.394192in}%
\pgfsys@useobject{currentmarker}{}%
\end{pgfscope}%
\begin{pgfscope}%
\pgfsys@transformshift{0.797177in}{2.394192in}%
\pgfsys@useobject{currentmarker}{}%
\end{pgfscope}%
\begin{pgfscope}%
\pgfsys@transformshift{0.797177in}{2.394192in}%
\pgfsys@useobject{currentmarker}{}%
\end{pgfscope}%
\begin{pgfscope}%
\pgfsys@transformshift{0.797177in}{2.394192in}%
\pgfsys@useobject{currentmarker}{}%
\end{pgfscope}%
\begin{pgfscope}%
\pgfsys@transformshift{0.797177in}{2.394192in}%
\pgfsys@useobject{currentmarker}{}%
\end{pgfscope}%
\begin{pgfscope}%
\pgfsys@transformshift{0.797177in}{2.394192in}%
\pgfsys@useobject{currentmarker}{}%
\end{pgfscope}%
\begin{pgfscope}%
\pgfsys@transformshift{0.797177in}{2.394192in}%
\pgfsys@useobject{currentmarker}{}%
\end{pgfscope}%
\begin{pgfscope}%
\pgfsys@transformshift{0.797177in}{2.394192in}%
\pgfsys@useobject{currentmarker}{}%
\end{pgfscope}%
\begin{pgfscope}%
\pgfsys@transformshift{0.797177in}{2.394192in}%
\pgfsys@useobject{currentmarker}{}%
\end{pgfscope}%
\begin{pgfscope}%
\pgfsys@transformshift{0.797177in}{2.394192in}%
\pgfsys@useobject{currentmarker}{}%
\end{pgfscope}%
\begin{pgfscope}%
\pgfsys@transformshift{0.797177in}{2.394192in}%
\pgfsys@useobject{currentmarker}{}%
\end{pgfscope}%
\begin{pgfscope}%
\pgfsys@transformshift{0.797177in}{2.394192in}%
\pgfsys@useobject{currentmarker}{}%
\end{pgfscope}%
\begin{pgfscope}%
\pgfsys@transformshift{0.797177in}{2.394192in}%
\pgfsys@useobject{currentmarker}{}%
\end{pgfscope}%
\begin{pgfscope}%
\pgfsys@transformshift{0.797177in}{2.394192in}%
\pgfsys@useobject{currentmarker}{}%
\end{pgfscope}%
\begin{pgfscope}%
\pgfsys@transformshift{0.797177in}{2.394192in}%
\pgfsys@useobject{currentmarker}{}%
\end{pgfscope}%
\begin{pgfscope}%
\pgfsys@transformshift{0.797177in}{2.394192in}%
\pgfsys@useobject{currentmarker}{}%
\end{pgfscope}%
\begin{pgfscope}%
\pgfsys@transformshift{0.797177in}{2.394192in}%
\pgfsys@useobject{currentmarker}{}%
\end{pgfscope}%
\begin{pgfscope}%
\pgfsys@transformshift{0.797177in}{2.394192in}%
\pgfsys@useobject{currentmarker}{}%
\end{pgfscope}%
\begin{pgfscope}%
\pgfsys@transformshift{0.797177in}{2.394192in}%
\pgfsys@useobject{currentmarker}{}%
\end{pgfscope}%
\begin{pgfscope}%
\pgfsys@transformshift{0.797177in}{2.394192in}%
\pgfsys@useobject{currentmarker}{}%
\end{pgfscope}%
\begin{pgfscope}%
\pgfsys@transformshift{0.797177in}{2.394192in}%
\pgfsys@useobject{currentmarker}{}%
\end{pgfscope}%
\begin{pgfscope}%
\pgfsys@transformshift{0.797177in}{2.394192in}%
\pgfsys@useobject{currentmarker}{}%
\end{pgfscope}%
\begin{pgfscope}%
\pgfsys@transformshift{0.797177in}{2.394192in}%
\pgfsys@useobject{currentmarker}{}%
\end{pgfscope}%
\begin{pgfscope}%
\pgfsys@transformshift{0.797177in}{2.394192in}%
\pgfsys@useobject{currentmarker}{}%
\end{pgfscope}%
\begin{pgfscope}%
\pgfsys@transformshift{0.797177in}{2.394192in}%
\pgfsys@useobject{currentmarker}{}%
\end{pgfscope}%
\begin{pgfscope}%
\pgfsys@transformshift{0.797177in}{2.394192in}%
\pgfsys@useobject{currentmarker}{}%
\end{pgfscope}%
\begin{pgfscope}%
\pgfsys@transformshift{0.797177in}{2.394192in}%
\pgfsys@useobject{currentmarker}{}%
\end{pgfscope}%
\begin{pgfscope}%
\pgfsys@transformshift{0.797177in}{2.394192in}%
\pgfsys@useobject{currentmarker}{}%
\end{pgfscope}%
\begin{pgfscope}%
\pgfsys@transformshift{0.797177in}{2.394192in}%
\pgfsys@useobject{currentmarker}{}%
\end{pgfscope}%
\begin{pgfscope}%
\pgfsys@transformshift{0.797177in}{2.394192in}%
\pgfsys@useobject{currentmarker}{}%
\end{pgfscope}%
\begin{pgfscope}%
\pgfsys@transformshift{0.797177in}{2.394192in}%
\pgfsys@useobject{currentmarker}{}%
\end{pgfscope}%
\begin{pgfscope}%
\pgfsys@transformshift{0.797177in}{2.394192in}%
\pgfsys@useobject{currentmarker}{}%
\end{pgfscope}%
\begin{pgfscope}%
\pgfsys@transformshift{0.797177in}{2.394192in}%
\pgfsys@useobject{currentmarker}{}%
\end{pgfscope}%
\begin{pgfscope}%
\pgfsys@transformshift{0.797177in}{2.394192in}%
\pgfsys@useobject{currentmarker}{}%
\end{pgfscope}%
\begin{pgfscope}%
\pgfsys@transformshift{0.797177in}{2.394192in}%
\pgfsys@useobject{currentmarker}{}%
\end{pgfscope}%
\begin{pgfscope}%
\pgfsys@transformshift{0.797177in}{2.394192in}%
\pgfsys@useobject{currentmarker}{}%
\end{pgfscope}%
\begin{pgfscope}%
\pgfsys@transformshift{0.797177in}{2.394192in}%
\pgfsys@useobject{currentmarker}{}%
\end{pgfscope}%
\begin{pgfscope}%
\pgfsys@transformshift{0.797177in}{2.394192in}%
\pgfsys@useobject{currentmarker}{}%
\end{pgfscope}%
\begin{pgfscope}%
\pgfsys@transformshift{0.797177in}{2.394192in}%
\pgfsys@useobject{currentmarker}{}%
\end{pgfscope}%
\begin{pgfscope}%
\pgfsys@transformshift{0.797177in}{2.394192in}%
\pgfsys@useobject{currentmarker}{}%
\end{pgfscope}%
\begin{pgfscope}%
\pgfsys@transformshift{0.797177in}{2.394192in}%
\pgfsys@useobject{currentmarker}{}%
\end{pgfscope}%
\begin{pgfscope}%
\pgfsys@transformshift{0.797177in}{2.394192in}%
\pgfsys@useobject{currentmarker}{}%
\end{pgfscope}%
\begin{pgfscope}%
\pgfsys@transformshift{0.797177in}{2.394192in}%
\pgfsys@useobject{currentmarker}{}%
\end{pgfscope}%
\begin{pgfscope}%
\pgfsys@transformshift{0.797177in}{2.394192in}%
\pgfsys@useobject{currentmarker}{}%
\end{pgfscope}%
\begin{pgfscope}%
\pgfsys@transformshift{0.797177in}{2.394192in}%
\pgfsys@useobject{currentmarker}{}%
\end{pgfscope}%
\begin{pgfscope}%
\pgfsys@transformshift{0.797177in}{2.394192in}%
\pgfsys@useobject{currentmarker}{}%
\end{pgfscope}%
\begin{pgfscope}%
\pgfsys@transformshift{0.797177in}{2.394192in}%
\pgfsys@useobject{currentmarker}{}%
\end{pgfscope}%
\begin{pgfscope}%
\pgfsys@transformshift{0.797177in}{2.394192in}%
\pgfsys@useobject{currentmarker}{}%
\end{pgfscope}%
\begin{pgfscope}%
\pgfsys@transformshift{0.797177in}{2.394192in}%
\pgfsys@useobject{currentmarker}{}%
\end{pgfscope}%
\begin{pgfscope}%
\pgfsys@transformshift{0.797177in}{2.394192in}%
\pgfsys@useobject{currentmarker}{}%
\end{pgfscope}%
\begin{pgfscope}%
\pgfsys@transformshift{0.797177in}{2.394192in}%
\pgfsys@useobject{currentmarker}{}%
\end{pgfscope}%
\begin{pgfscope}%
\pgfsys@transformshift{0.797177in}{2.394192in}%
\pgfsys@useobject{currentmarker}{}%
\end{pgfscope}%
\begin{pgfscope}%
\pgfsys@transformshift{0.797177in}{2.394192in}%
\pgfsys@useobject{currentmarker}{}%
\end{pgfscope}%
\begin{pgfscope}%
\pgfsys@transformshift{0.797177in}{2.394192in}%
\pgfsys@useobject{currentmarker}{}%
\end{pgfscope}%
\begin{pgfscope}%
\pgfsys@transformshift{0.797177in}{2.394192in}%
\pgfsys@useobject{currentmarker}{}%
\end{pgfscope}%
\begin{pgfscope}%
\pgfsys@transformshift{0.797177in}{2.394192in}%
\pgfsys@useobject{currentmarker}{}%
\end{pgfscope}%
\begin{pgfscope}%
\pgfsys@transformshift{0.797177in}{2.394192in}%
\pgfsys@useobject{currentmarker}{}%
\end{pgfscope}%
\begin{pgfscope}%
\pgfsys@transformshift{0.797177in}{2.394192in}%
\pgfsys@useobject{currentmarker}{}%
\end{pgfscope}%
\begin{pgfscope}%
\pgfsys@transformshift{0.798416in}{2.394261in}%
\pgfsys@useobject{currentmarker}{}%
\end{pgfscope}%
\begin{pgfscope}%
\pgfsys@transformshift{0.799096in}{2.394309in}%
\pgfsys@useobject{currentmarker}{}%
\end{pgfscope}%
\begin{pgfscope}%
\pgfsys@transformshift{0.799471in}{2.394332in}%
\pgfsys@useobject{currentmarker}{}%
\end{pgfscope}%
\begin{pgfscope}%
\pgfsys@transformshift{0.799676in}{2.394354in}%
\pgfsys@useobject{currentmarker}{}%
\end{pgfscope}%
\begin{pgfscope}%
\pgfsys@transformshift{0.799790in}{2.394356in}%
\pgfsys@useobject{currentmarker}{}%
\end{pgfscope}%
\begin{pgfscope}%
\pgfsys@transformshift{0.799852in}{2.394359in}%
\pgfsys@useobject{currentmarker}{}%
\end{pgfscope}%
\begin{pgfscope}%
\pgfsys@transformshift{0.799886in}{2.394358in}%
\pgfsys@useobject{currentmarker}{}%
\end{pgfscope}%
\begin{pgfscope}%
\pgfsys@transformshift{0.799905in}{2.394361in}%
\pgfsys@useobject{currentmarker}{}%
\end{pgfscope}%
\begin{pgfscope}%
\pgfsys@transformshift{0.799915in}{2.394362in}%
\pgfsys@useobject{currentmarker}{}%
\end{pgfscope}%
\begin{pgfscope}%
\pgfsys@transformshift{0.799921in}{2.394363in}%
\pgfsys@useobject{currentmarker}{}%
\end{pgfscope}%
\begin{pgfscope}%
\pgfsys@transformshift{0.799924in}{2.394363in}%
\pgfsys@useobject{currentmarker}{}%
\end{pgfscope}%
\begin{pgfscope}%
\pgfsys@transformshift{0.799926in}{2.394363in}%
\pgfsys@useobject{currentmarker}{}%
\end{pgfscope}%
\begin{pgfscope}%
\pgfsys@transformshift{0.799927in}{2.394363in}%
\pgfsys@useobject{currentmarker}{}%
\end{pgfscope}%
\begin{pgfscope}%
\pgfsys@transformshift{0.799927in}{2.394363in}%
\pgfsys@useobject{currentmarker}{}%
\end{pgfscope}%
\begin{pgfscope}%
\pgfsys@transformshift{0.799928in}{2.394363in}%
\pgfsys@useobject{currentmarker}{}%
\end{pgfscope}%
\begin{pgfscope}%
\pgfsys@transformshift{0.799928in}{2.394363in}%
\pgfsys@useobject{currentmarker}{}%
\end{pgfscope}%
\begin{pgfscope}%
\pgfsys@transformshift{0.799928in}{2.394363in}%
\pgfsys@useobject{currentmarker}{}%
\end{pgfscope}%
\begin{pgfscope}%
\pgfsys@transformshift{0.799928in}{2.394363in}%
\pgfsys@useobject{currentmarker}{}%
\end{pgfscope}%
\begin{pgfscope}%
\pgfsys@transformshift{0.799928in}{2.394363in}%
\pgfsys@useobject{currentmarker}{}%
\end{pgfscope}%
\begin{pgfscope}%
\pgfsys@transformshift{0.799928in}{2.394363in}%
\pgfsys@useobject{currentmarker}{}%
\end{pgfscope}%
\begin{pgfscope}%
\pgfsys@transformshift{0.799928in}{2.394363in}%
\pgfsys@useobject{currentmarker}{}%
\end{pgfscope}%
\begin{pgfscope}%
\pgfsys@transformshift{0.799928in}{2.394363in}%
\pgfsys@useobject{currentmarker}{}%
\end{pgfscope}%
\begin{pgfscope}%
\pgfsys@transformshift{0.799928in}{2.394363in}%
\pgfsys@useobject{currentmarker}{}%
\end{pgfscope}%
\begin{pgfscope}%
\pgfsys@transformshift{0.799928in}{2.394363in}%
\pgfsys@useobject{currentmarker}{}%
\end{pgfscope}%
\begin{pgfscope}%
\pgfsys@transformshift{0.799928in}{2.394363in}%
\pgfsys@useobject{currentmarker}{}%
\end{pgfscope}%
\begin{pgfscope}%
\pgfsys@transformshift{0.799928in}{2.394363in}%
\pgfsys@useobject{currentmarker}{}%
\end{pgfscope}%
\begin{pgfscope}%
\pgfsys@transformshift{0.799928in}{2.394363in}%
\pgfsys@useobject{currentmarker}{}%
\end{pgfscope}%
\begin{pgfscope}%
\pgfsys@transformshift{0.799928in}{2.394363in}%
\pgfsys@useobject{currentmarker}{}%
\end{pgfscope}%
\begin{pgfscope}%
\pgfsys@transformshift{0.799928in}{2.394363in}%
\pgfsys@useobject{currentmarker}{}%
\end{pgfscope}%
\begin{pgfscope}%
\pgfsys@transformshift{0.799928in}{2.394363in}%
\pgfsys@useobject{currentmarker}{}%
\end{pgfscope}%
\begin{pgfscope}%
\pgfsys@transformshift{0.799928in}{2.394363in}%
\pgfsys@useobject{currentmarker}{}%
\end{pgfscope}%
\begin{pgfscope}%
\pgfsys@transformshift{0.799928in}{2.394363in}%
\pgfsys@useobject{currentmarker}{}%
\end{pgfscope}%
\begin{pgfscope}%
\pgfsys@transformshift{0.799928in}{2.394363in}%
\pgfsys@useobject{currentmarker}{}%
\end{pgfscope}%
\begin{pgfscope}%
\pgfsys@transformshift{0.799928in}{2.394363in}%
\pgfsys@useobject{currentmarker}{}%
\end{pgfscope}%
\begin{pgfscope}%
\pgfsys@transformshift{0.799928in}{2.394363in}%
\pgfsys@useobject{currentmarker}{}%
\end{pgfscope}%
\begin{pgfscope}%
\pgfsys@transformshift{0.799928in}{2.394363in}%
\pgfsys@useobject{currentmarker}{}%
\end{pgfscope}%
\begin{pgfscope}%
\pgfsys@transformshift{0.799928in}{2.394363in}%
\pgfsys@useobject{currentmarker}{}%
\end{pgfscope}%
\begin{pgfscope}%
\pgfsys@transformshift{0.799928in}{2.394363in}%
\pgfsys@useobject{currentmarker}{}%
\end{pgfscope}%
\begin{pgfscope}%
\pgfsys@transformshift{0.799928in}{2.394363in}%
\pgfsys@useobject{currentmarker}{}%
\end{pgfscope}%
\begin{pgfscope}%
\pgfsys@transformshift{0.799928in}{2.394363in}%
\pgfsys@useobject{currentmarker}{}%
\end{pgfscope}%
\begin{pgfscope}%
\pgfsys@transformshift{0.799928in}{2.394363in}%
\pgfsys@useobject{currentmarker}{}%
\end{pgfscope}%
\begin{pgfscope}%
\pgfsys@transformshift{0.799928in}{2.394363in}%
\pgfsys@useobject{currentmarker}{}%
\end{pgfscope}%
\begin{pgfscope}%
\pgfsys@transformshift{0.799928in}{2.394363in}%
\pgfsys@useobject{currentmarker}{}%
\end{pgfscope}%
\begin{pgfscope}%
\pgfsys@transformshift{0.799928in}{2.394363in}%
\pgfsys@useobject{currentmarker}{}%
\end{pgfscope}%
\begin{pgfscope}%
\pgfsys@transformshift{0.799928in}{2.394363in}%
\pgfsys@useobject{currentmarker}{}%
\end{pgfscope}%
\begin{pgfscope}%
\pgfsys@transformshift{0.799928in}{2.394363in}%
\pgfsys@useobject{currentmarker}{}%
\end{pgfscope}%
\begin{pgfscope}%
\pgfsys@transformshift{0.799928in}{2.394363in}%
\pgfsys@useobject{currentmarker}{}%
\end{pgfscope}%
\begin{pgfscope}%
\pgfsys@transformshift{0.799928in}{2.394363in}%
\pgfsys@useobject{currentmarker}{}%
\end{pgfscope}%
\begin{pgfscope}%
\pgfsys@transformshift{0.799928in}{2.394363in}%
\pgfsys@useobject{currentmarker}{}%
\end{pgfscope}%
\begin{pgfscope}%
\pgfsys@transformshift{0.799928in}{2.394363in}%
\pgfsys@useobject{currentmarker}{}%
\end{pgfscope}%
\begin{pgfscope}%
\pgfsys@transformshift{0.799928in}{2.394363in}%
\pgfsys@useobject{currentmarker}{}%
\end{pgfscope}%
\begin{pgfscope}%
\pgfsys@transformshift{0.799928in}{2.394363in}%
\pgfsys@useobject{currentmarker}{}%
\end{pgfscope}%
\begin{pgfscope}%
\pgfsys@transformshift{0.799928in}{2.394363in}%
\pgfsys@useobject{currentmarker}{}%
\end{pgfscope}%
\begin{pgfscope}%
\pgfsys@transformshift{0.799928in}{2.394363in}%
\pgfsys@useobject{currentmarker}{}%
\end{pgfscope}%
\begin{pgfscope}%
\pgfsys@transformshift{0.799928in}{2.394363in}%
\pgfsys@useobject{currentmarker}{}%
\end{pgfscope}%
\begin{pgfscope}%
\pgfsys@transformshift{0.799928in}{2.394363in}%
\pgfsys@useobject{currentmarker}{}%
\end{pgfscope}%
\begin{pgfscope}%
\pgfsys@transformshift{0.799928in}{2.394363in}%
\pgfsys@useobject{currentmarker}{}%
\end{pgfscope}%
\begin{pgfscope}%
\pgfsys@transformshift{0.799928in}{2.394363in}%
\pgfsys@useobject{currentmarker}{}%
\end{pgfscope}%
\begin{pgfscope}%
\pgfsys@transformshift{0.799928in}{2.394363in}%
\pgfsys@useobject{currentmarker}{}%
\end{pgfscope}%
\begin{pgfscope}%
\pgfsys@transformshift{0.799928in}{2.394363in}%
\pgfsys@useobject{currentmarker}{}%
\end{pgfscope}%
\begin{pgfscope}%
\pgfsys@transformshift{0.799928in}{2.394363in}%
\pgfsys@useobject{currentmarker}{}%
\end{pgfscope}%
\begin{pgfscope}%
\pgfsys@transformshift{0.799928in}{2.394363in}%
\pgfsys@useobject{currentmarker}{}%
\end{pgfscope}%
\begin{pgfscope}%
\pgfsys@transformshift{0.799928in}{2.394363in}%
\pgfsys@useobject{currentmarker}{}%
\end{pgfscope}%
\begin{pgfscope}%
\pgfsys@transformshift{0.802077in}{2.394375in}%
\pgfsys@useobject{currentmarker}{}%
\end{pgfscope}%
\begin{pgfscope}%
\pgfsys@transformshift{0.803259in}{2.394403in}%
\pgfsys@useobject{currentmarker}{}%
\end{pgfscope}%
\begin{pgfscope}%
\pgfsys@transformshift{0.803909in}{2.394396in}%
\pgfsys@useobject{currentmarker}{}%
\end{pgfscope}%
\begin{pgfscope}%
\pgfsys@transformshift{0.806207in}{2.394287in}%
\pgfsys@useobject{currentmarker}{}%
\end{pgfscope}%
\begin{pgfscope}%
\pgfsys@transformshift{0.807431in}{2.393963in}%
\pgfsys@useobject{currentmarker}{}%
\end{pgfscope}%
\begin{pgfscope}%
\pgfsys@transformshift{0.810003in}{2.393959in}%
\pgfsys@useobject{currentmarker}{}%
\end{pgfscope}%
\begin{pgfscope}%
\pgfsys@transformshift{0.813860in}{2.393285in}%
\pgfsys@useobject{currentmarker}{}%
\end{pgfscope}%
\begin{pgfscope}%
\pgfsys@transformshift{0.816012in}{2.393180in}%
\pgfsys@useobject{currentmarker}{}%
\end{pgfscope}%
\begin{pgfscope}%
\pgfsys@transformshift{0.820017in}{2.392723in}%
\pgfsys@useobject{currentmarker}{}%
\end{pgfscope}%
\begin{pgfscope}%
\pgfsys@transformshift{0.825286in}{2.392048in}%
\pgfsys@useobject{currentmarker}{}%
\end{pgfscope}%
\begin{pgfscope}%
\pgfsys@transformshift{0.828207in}{2.391992in}%
\pgfsys@useobject{currentmarker}{}%
\end{pgfscope}%
\begin{pgfscope}%
\pgfsys@transformshift{0.832991in}{2.391626in}%
\pgfsys@useobject{currentmarker}{}%
\end{pgfscope}%
\begin{pgfscope}%
\pgfsys@transformshift{0.839849in}{2.390838in}%
\pgfsys@useobject{currentmarker}{}%
\end{pgfscope}%
\begin{pgfscope}%
\pgfsys@transformshift{0.843638in}{2.390591in}%
\pgfsys@useobject{currentmarker}{}%
\end{pgfscope}%
\begin{pgfscope}%
\pgfsys@transformshift{0.848519in}{2.389815in}%
\pgfsys@useobject{currentmarker}{}%
\end{pgfscope}%
\begin{pgfscope}%
\pgfsys@transformshift{0.855081in}{2.388426in}%
\pgfsys@useobject{currentmarker}{}%
\end{pgfscope}%
\begin{pgfscope}%
\pgfsys@transformshift{0.858724in}{2.387847in}%
\pgfsys@useobject{currentmarker}{}%
\end{pgfscope}%
\begin{pgfscope}%
\pgfsys@transformshift{0.868515in}{2.386637in}%
\pgfsys@useobject{currentmarker}{}%
\end{pgfscope}%
\begin{pgfscope}%
\pgfsys@transformshift{0.880044in}{2.384343in}%
\pgfsys@useobject{currentmarker}{}%
\end{pgfscope}%
\begin{pgfscope}%
\pgfsys@transformshift{0.886508in}{2.384470in}%
\pgfsys@useobject{currentmarker}{}%
\end{pgfscope}%
\begin{pgfscope}%
\pgfsys@transformshift{0.890028in}{2.383969in}%
\pgfsys@useobject{currentmarker}{}%
\end{pgfscope}%
\begin{pgfscope}%
\pgfsys@transformshift{0.894950in}{2.383441in}%
\pgfsys@useobject{currentmarker}{}%
\end{pgfscope}%
\begin{pgfscope}%
\pgfsys@transformshift{0.897577in}{2.382726in}%
\pgfsys@useobject{currentmarker}{}%
\end{pgfscope}%
\begin{pgfscope}%
\pgfsys@transformshift{0.899026in}{2.382349in}%
\pgfsys@useobject{currentmarker}{}%
\end{pgfscope}%
\begin{pgfscope}%
\pgfsys@transformshift{0.899849in}{2.382350in}%
\pgfsys@useobject{currentmarker}{}%
\end{pgfscope}%
\begin{pgfscope}%
\pgfsys@transformshift{0.902247in}{2.381779in}%
\pgfsys@useobject{currentmarker}{}%
\end{pgfscope}%
\begin{pgfscope}%
\pgfsys@transformshift{0.903598in}{2.381667in}%
\pgfsys@useobject{currentmarker}{}%
\end{pgfscope}%
\begin{pgfscope}%
\pgfsys@transformshift{0.904339in}{2.381589in}%
\pgfsys@useobject{currentmarker}{}%
\end{pgfscope}%
\begin{pgfscope}%
\pgfsys@transformshift{0.906672in}{2.381429in}%
\pgfsys@useobject{currentmarker}{}%
\end{pgfscope}%
\begin{pgfscope}%
\pgfsys@transformshift{0.907958in}{2.381442in}%
\pgfsys@useobject{currentmarker}{}%
\end{pgfscope}%
\begin{pgfscope}%
\pgfsys@transformshift{0.908659in}{2.381353in}%
\pgfsys@useobject{currentmarker}{}%
\end{pgfscope}%
\begin{pgfscope}%
\pgfsys@transformshift{0.909048in}{2.381354in}%
\pgfsys@useobject{currentmarker}{}%
\end{pgfscope}%
\begin{pgfscope}%
\pgfsys@transformshift{0.914704in}{2.380271in}%
\pgfsys@useobject{currentmarker}{}%
\end{pgfscope}%
\begin{pgfscope}%
\pgfsys@transformshift{0.921850in}{2.380701in}%
\pgfsys@useobject{currentmarker}{}%
\end{pgfscope}%
\begin{pgfscope}%
\pgfsys@transformshift{0.930361in}{2.379960in}%
\pgfsys@useobject{currentmarker}{}%
\end{pgfscope}%
\begin{pgfscope}%
\pgfsys@transformshift{0.935058in}{2.380091in}%
\pgfsys@useobject{currentmarker}{}%
\end{pgfscope}%
\begin{pgfscope}%
\pgfsys@transformshift{0.941422in}{2.379495in}%
\pgfsys@useobject{currentmarker}{}%
\end{pgfscope}%
\begin{pgfscope}%
\pgfsys@transformshift{0.944931in}{2.379288in}%
\pgfsys@useobject{currentmarker}{}%
\end{pgfscope}%
\begin{pgfscope}%
\pgfsys@transformshift{0.950169in}{2.378892in}%
\pgfsys@useobject{currentmarker}{}%
\end{pgfscope}%
\begin{pgfscope}%
\pgfsys@transformshift{0.956813in}{2.378857in}%
\pgfsys@useobject{currentmarker}{}%
\end{pgfscope}%
\begin{pgfscope}%
\pgfsys@transformshift{0.960460in}{2.378635in}%
\pgfsys@useobject{currentmarker}{}%
\end{pgfscope}%
\begin{pgfscope}%
\pgfsys@transformshift{0.962458in}{2.378410in}%
\pgfsys@useobject{currentmarker}{}%
\end{pgfscope}%
\begin{pgfscope}%
\pgfsys@transformshift{0.965985in}{2.378500in}%
\pgfsys@useobject{currentmarker}{}%
\end{pgfscope}%
\begin{pgfscope}%
\pgfsys@transformshift{0.967845in}{2.377944in}%
\pgfsys@useobject{currentmarker}{}%
\end{pgfscope}%
\begin{pgfscope}%
\pgfsys@transformshift{0.968909in}{2.377863in}%
\pgfsys@useobject{currentmarker}{}%
\end{pgfscope}%
\begin{pgfscope}%
\pgfsys@transformshift{0.972872in}{2.377476in}%
\pgfsys@useobject{currentmarker}{}%
\end{pgfscope}%
\begin{pgfscope}%
\pgfsys@transformshift{0.975058in}{2.377352in}%
\pgfsys@useobject{currentmarker}{}%
\end{pgfscope}%
\begin{pgfscope}%
\pgfsys@transformshift{0.976258in}{2.377248in}%
\pgfsys@useobject{currentmarker}{}%
\end{pgfscope}%
\begin{pgfscope}%
\pgfsys@transformshift{0.979000in}{2.377119in}%
\pgfsys@useobject{currentmarker}{}%
\end{pgfscope}%
\begin{pgfscope}%
\pgfsys@transformshift{0.980509in}{2.377082in}%
\pgfsys@useobject{currentmarker}{}%
\end{pgfscope}%
\begin{pgfscope}%
\pgfsys@transformshift{0.983911in}{2.376740in}%
\pgfsys@useobject{currentmarker}{}%
\end{pgfscope}%
\begin{pgfscope}%
\pgfsys@transformshift{0.989688in}{2.377210in}%
\pgfsys@useobject{currentmarker}{}%
\end{pgfscope}%
\begin{pgfscope}%
\pgfsys@transformshift{0.992796in}{2.376505in}%
\pgfsys@useobject{currentmarker}{}%
\end{pgfscope}%
\begin{pgfscope}%
\pgfsys@transformshift{1.000465in}{2.376722in}%
\pgfsys@useobject{currentmarker}{}%
\end{pgfscope}%
\begin{pgfscope}%
\pgfsys@transformshift{1.011185in}{2.375537in}%
\pgfsys@useobject{currentmarker}{}%
\end{pgfscope}%
\begin{pgfscope}%
\pgfsys@transformshift{1.017061in}{2.376350in}%
\pgfsys@useobject{currentmarker}{}%
\end{pgfscope}%
\begin{pgfscope}%
\pgfsys@transformshift{1.020305in}{2.375994in}%
\pgfsys@useobject{currentmarker}{}%
\end{pgfscope}%
\begin{pgfscope}%
\pgfsys@transformshift{1.032133in}{2.376689in}%
\pgfsys@useobject{currentmarker}{}%
\end{pgfscope}%
\begin{pgfscope}%
\pgfsys@transformshift{1.047290in}{2.375208in}%
\pgfsys@useobject{currentmarker}{}%
\end{pgfscope}%
\begin{pgfscope}%
\pgfsys@transformshift{1.055659in}{2.375542in}%
\pgfsys@useobject{currentmarker}{}%
\end{pgfscope}%
\begin{pgfscope}%
\pgfsys@transformshift{1.060103in}{2.374329in}%
\pgfsys@useobject{currentmarker}{}%
\end{pgfscope}%
\begin{pgfscope}%
\pgfsys@transformshift{1.067382in}{2.374375in}%
\pgfsys@useobject{currentmarker}{}%
\end{pgfscope}%
\begin{pgfscope}%
\pgfsys@transformshift{1.076391in}{2.373104in}%
\pgfsys@useobject{currentmarker}{}%
\end{pgfscope}%
\begin{pgfscope}%
\pgfsys@transformshift{1.090583in}{2.372652in}%
\pgfsys@useobject{currentmarker}{}%
\end{pgfscope}%
\begin{pgfscope}%
\pgfsys@transformshift{1.106391in}{2.367111in}%
\pgfsys@useobject{currentmarker}{}%
\end{pgfscope}%
\begin{pgfscope}%
\pgfsys@transformshift{1.115588in}{2.366567in}%
\pgfsys@useobject{currentmarker}{}%
\end{pgfscope}%
\begin{pgfscope}%
\pgfsys@transformshift{1.127720in}{2.364631in}%
\pgfsys@useobject{currentmarker}{}%
\end{pgfscope}%
\begin{pgfscope}%
\pgfsys@transformshift{1.134424in}{2.363788in}%
\pgfsys@useobject{currentmarker}{}%
\end{pgfscope}%
\begin{pgfscope}%
\pgfsys@transformshift{1.138103in}{2.363262in}%
\pgfsys@useobject{currentmarker}{}%
\end{pgfscope}%
\begin{pgfscope}%
\pgfsys@transformshift{1.145799in}{2.362045in}%
\pgfsys@useobject{currentmarker}{}%
\end{pgfscope}%
\begin{pgfscope}%
\pgfsys@transformshift{1.150076in}{2.361764in}%
\pgfsys@useobject{currentmarker}{}%
\end{pgfscope}%
\begin{pgfscope}%
\pgfsys@transformshift{1.155899in}{2.361518in}%
\pgfsys@useobject{currentmarker}{}%
\end{pgfscope}%
\begin{pgfscope}%
\pgfsys@transformshift{1.164926in}{2.361449in}%
\pgfsys@useobject{currentmarker}{}%
\end{pgfscope}%
\begin{pgfscope}%
\pgfsys@transformshift{1.169890in}{2.361591in}%
\pgfsys@useobject{currentmarker}{}%
\end{pgfscope}%
\begin{pgfscope}%
\pgfsys@transformshift{1.177684in}{2.361630in}%
\pgfsys@useobject{currentmarker}{}%
\end{pgfscope}%
\begin{pgfscope}%
\pgfsys@transformshift{1.188219in}{2.360656in}%
\pgfsys@useobject{currentmarker}{}%
\end{pgfscope}%
\begin{pgfscope}%
\pgfsys@transformshift{1.194037in}{2.360537in}%
\pgfsys@useobject{currentmarker}{}%
\end{pgfscope}%
\begin{pgfscope}%
\pgfsys@transformshift{1.203185in}{2.361134in}%
\pgfsys@useobject{currentmarker}{}%
\end{pgfscope}%
\begin{pgfscope}%
\pgfsys@transformshift{1.214941in}{2.359994in}%
\pgfsys@useobject{currentmarker}{}%
\end{pgfscope}%
\begin{pgfscope}%
\pgfsys@transformshift{1.228478in}{2.362428in}%
\pgfsys@useobject{currentmarker}{}%
\end{pgfscope}%
\begin{pgfscope}%
\pgfsys@transformshift{1.245700in}{2.361062in}%
\pgfsys@useobject{currentmarker}{}%
\end{pgfscope}%
\begin{pgfscope}%
\pgfsys@transformshift{1.264391in}{2.360913in}%
\pgfsys@useobject{currentmarker}{}%
\end{pgfscope}%
\begin{pgfscope}%
\pgfsys@transformshift{1.284714in}{2.360561in}%
\pgfsys@useobject{currentmarker}{}%
\end{pgfscope}%
\begin{pgfscope}%
\pgfsys@transformshift{1.310563in}{2.362608in}%
\pgfsys@useobject{currentmarker}{}%
\end{pgfscope}%
\begin{pgfscope}%
\pgfsys@transformshift{1.338811in}{2.360543in}%
\pgfsys@useobject{currentmarker}{}%
\end{pgfscope}%
\begin{pgfscope}%
\pgfsys@transformshift{1.354375in}{2.361211in}%
\pgfsys@useobject{currentmarker}{}%
\end{pgfscope}%
\begin{pgfscope}%
\pgfsys@transformshift{1.372121in}{2.361359in}%
\pgfsys@useobject{currentmarker}{}%
\end{pgfscope}%
\begin{pgfscope}%
\pgfsys@transformshift{1.392634in}{2.362340in}%
\pgfsys@useobject{currentmarker}{}%
\end{pgfscope}%
\begin{pgfscope}%
\pgfsys@transformshift{1.414607in}{2.362044in}%
\pgfsys@useobject{currentmarker}{}%
\end{pgfscope}%
\begin{pgfscope}%
\pgfsys@transformshift{1.439993in}{2.360583in}%
\pgfsys@useobject{currentmarker}{}%
\end{pgfscope}%
\begin{pgfscope}%
\pgfsys@transformshift{1.467128in}{2.362977in}%
\pgfsys@useobject{currentmarker}{}%
\end{pgfscope}%
\begin{pgfscope}%
\pgfsys@transformshift{1.482095in}{2.363666in}%
\pgfsys@useobject{currentmarker}{}%
\end{pgfscope}%
\begin{pgfscope}%
\pgfsys@transformshift{1.490330in}{2.363962in}%
\pgfsys@useobject{currentmarker}{}%
\end{pgfscope}%
\begin{pgfscope}%
\pgfsys@transformshift{1.501338in}{2.363540in}%
\pgfsys@useobject{currentmarker}{}%
\end{pgfscope}%
\begin{pgfscope}%
\pgfsys@transformshift{1.507396in}{2.363622in}%
\pgfsys@useobject{currentmarker}{}%
\end{pgfscope}%
\begin{pgfscope}%
\pgfsys@transformshift{1.514606in}{2.363862in}%
\pgfsys@useobject{currentmarker}{}%
\end{pgfscope}%
\begin{pgfscope}%
\pgfsys@transformshift{1.523876in}{2.363453in}%
\pgfsys@useobject{currentmarker}{}%
\end{pgfscope}%
\begin{pgfscope}%
\pgfsys@transformshift{1.528921in}{2.364222in}%
\pgfsys@useobject{currentmarker}{}%
\end{pgfscope}%
\begin{pgfscope}%
\pgfsys@transformshift{1.531728in}{2.364231in}%
\pgfsys@useobject{currentmarker}{}%
\end{pgfscope}%
\begin{pgfscope}%
\pgfsys@transformshift{1.538902in}{2.363910in}%
\pgfsys@useobject{currentmarker}{}%
\end{pgfscope}%
\begin{pgfscope}%
\pgfsys@transformshift{1.549121in}{2.364903in}%
\pgfsys@useobject{currentmarker}{}%
\end{pgfscope}%
\begin{pgfscope}%
\pgfsys@transformshift{1.560465in}{2.363914in}%
\pgfsys@useobject{currentmarker}{}%
\end{pgfscope}%
\begin{pgfscope}%
\pgfsys@transformshift{1.573695in}{2.364090in}%
\pgfsys@useobject{currentmarker}{}%
\end{pgfscope}%
\begin{pgfscope}%
\pgfsys@transformshift{1.592057in}{2.363149in}%
\pgfsys@useobject{currentmarker}{}%
\end{pgfscope}%
\begin{pgfscope}%
\pgfsys@transformshift{1.612036in}{2.362439in}%
\pgfsys@useobject{currentmarker}{}%
\end{pgfscope}%
\begin{pgfscope}%
\pgfsys@transformshift{1.623028in}{2.362172in}%
\pgfsys@useobject{currentmarker}{}%
\end{pgfscope}%
\begin{pgfscope}%
\pgfsys@transformshift{1.629076in}{2.362170in}%
\pgfsys@useobject{currentmarker}{}%
\end{pgfscope}%
\begin{pgfscope}%
\pgfsys@transformshift{1.638477in}{2.362123in}%
\pgfsys@useobject{currentmarker}{}%
\end{pgfscope}%
\begin{pgfscope}%
\pgfsys@transformshift{1.649724in}{2.361684in}%
\pgfsys@useobject{currentmarker}{}%
\end{pgfscope}%
\begin{pgfscope}%
\pgfsys@transformshift{1.655911in}{2.361466in}%
\pgfsys@useobject{currentmarker}{}%
\end{pgfscope}%
\begin{pgfscope}%
\pgfsys@transformshift{1.667379in}{2.360501in}%
\pgfsys@useobject{currentmarker}{}%
\end{pgfscope}%
\begin{pgfscope}%
\pgfsys@transformshift{1.681345in}{2.362716in}%
\pgfsys@useobject{currentmarker}{}%
\end{pgfscope}%
\begin{pgfscope}%
\pgfsys@transformshift{1.696317in}{2.359475in}%
\pgfsys@useobject{currentmarker}{}%
\end{pgfscope}%
\begin{pgfscope}%
\pgfsys@transformshift{1.704742in}{2.359557in}%
\pgfsys@useobject{currentmarker}{}%
\end{pgfscope}%
\begin{pgfscope}%
\pgfsys@transformshift{1.717316in}{2.357441in}%
\pgfsys@useobject{currentmarker}{}%
\end{pgfscope}%
\begin{pgfscope}%
\pgfsys@transformshift{1.734371in}{2.356842in}%
\pgfsys@useobject{currentmarker}{}%
\end{pgfscope}%
\begin{pgfscope}%
\pgfsys@transformshift{1.754263in}{2.356237in}%
\pgfsys@useobject{currentmarker}{}%
\end{pgfscope}%
\begin{pgfscope}%
\pgfsys@transformshift{1.765208in}{2.356145in}%
\pgfsys@useobject{currentmarker}{}%
\end{pgfscope}%
\begin{pgfscope}%
\pgfsys@transformshift{1.781298in}{2.354725in}%
\pgfsys@useobject{currentmarker}{}%
\end{pgfscope}%
\begin{pgfscope}%
\pgfsys@transformshift{1.800440in}{2.354610in}%
\pgfsys@useobject{currentmarker}{}%
\end{pgfscope}%
\begin{pgfscope}%
\pgfsys@transformshift{1.810968in}{2.354517in}%
\pgfsys@useobject{currentmarker}{}%
\end{pgfscope}%
\begin{pgfscope}%
\pgfsys@transformshift{1.823742in}{2.353345in}%
\pgfsys@useobject{currentmarker}{}%
\end{pgfscope}%
\begin{pgfscope}%
\pgfsys@transformshift{1.840013in}{2.352254in}%
\pgfsys@useobject{currentmarker}{}%
\end{pgfscope}%
\begin{pgfscope}%
\pgfsys@transformshift{1.858474in}{2.350557in}%
\pgfsys@useobject{currentmarker}{}%
\end{pgfscope}%
\begin{pgfscope}%
\pgfsys@transformshift{1.868669in}{2.350363in}%
\pgfsys@useobject{currentmarker}{}%
\end{pgfscope}%
\begin{pgfscope}%
\pgfsys@transformshift{1.880379in}{2.350165in}%
\pgfsys@useobject{currentmarker}{}%
\end{pgfscope}%
\begin{pgfscope}%
\pgfsys@transformshift{1.896276in}{2.348806in}%
\pgfsys@useobject{currentmarker}{}%
\end{pgfscope}%
\begin{pgfscope}%
\pgfsys@transformshift{1.915788in}{2.347793in}%
\pgfsys@useobject{currentmarker}{}%
\end{pgfscope}%
\begin{pgfscope}%
\pgfsys@transformshift{1.926508in}{2.347045in}%
\pgfsys@useobject{currentmarker}{}%
\end{pgfscope}%
\begin{pgfscope}%
\pgfsys@transformshift{1.938914in}{2.346525in}%
\pgfsys@useobject{currentmarker}{}%
\end{pgfscope}%
\begin{pgfscope}%
\pgfsys@transformshift{1.954016in}{2.346163in}%
\pgfsys@useobject{currentmarker}{}%
\end{pgfscope}%
\begin{pgfscope}%
\pgfsys@transformshift{1.971518in}{2.343210in}%
\pgfsys@useobject{currentmarker}{}%
\end{pgfscope}%
\begin{pgfscope}%
\pgfsys@transformshift{1.981261in}{2.343834in}%
\pgfsys@useobject{currentmarker}{}%
\end{pgfscope}%
\begin{pgfscope}%
\pgfsys@transformshift{1.986629in}{2.343720in}%
\pgfsys@useobject{currentmarker}{}%
\end{pgfscope}%
\begin{pgfscope}%
\pgfsys@transformshift{1.994975in}{2.344060in}%
\pgfsys@useobject{currentmarker}{}%
\end{pgfscope}%
\begin{pgfscope}%
\pgfsys@transformshift{2.005437in}{2.345426in}%
\pgfsys@useobject{currentmarker}{}%
\end{pgfscope}%
\begin{pgfscope}%
\pgfsys@transformshift{2.011215in}{2.344883in}%
\pgfsys@useobject{currentmarker}{}%
\end{pgfscope}%
\begin{pgfscope}%
\pgfsys@transformshift{2.019195in}{2.345243in}%
\pgfsys@useobject{currentmarker}{}%
\end{pgfscope}%
\begin{pgfscope}%
\pgfsys@transformshift{2.030025in}{2.345172in}%
\pgfsys@useobject{currentmarker}{}%
\end{pgfscope}%
\begin{pgfscope}%
\pgfsys@transformshift{2.042350in}{2.344811in}%
\pgfsys@useobject{currentmarker}{}%
\end{pgfscope}%
\begin{pgfscope}%
\pgfsys@transformshift{2.049128in}{2.344592in}%
\pgfsys@useobject{currentmarker}{}%
\end{pgfscope}%
\begin{pgfscope}%
\pgfsys@transformshift{2.052858in}{2.344565in}%
\pgfsys@useobject{currentmarker}{}%
\end{pgfscope}%
\begin{pgfscope}%
\pgfsys@transformshift{2.062386in}{2.344169in}%
\pgfsys@useobject{currentmarker}{}%
\end{pgfscope}%
\begin{pgfscope}%
\pgfsys@transformshift{2.073423in}{2.343758in}%
\pgfsys@useobject{currentmarker}{}%
\end{pgfscope}%
\begin{pgfscope}%
\pgfsys@transformshift{2.079481in}{2.343321in}%
\pgfsys@useobject{currentmarker}{}%
\end{pgfscope}%
\begin{pgfscope}%
\pgfsys@transformshift{2.086903in}{2.342718in}%
\pgfsys@useobject{currentmarker}{}%
\end{pgfscope}%
\begin{pgfscope}%
\pgfsys@transformshift{2.097789in}{2.342588in}%
\pgfsys@useobject{currentmarker}{}%
\end{pgfscope}%
\begin{pgfscope}%
\pgfsys@transformshift{2.111370in}{2.340936in}%
\pgfsys@useobject{currentmarker}{}%
\end{pgfscope}%
\begin{pgfscope}%
\pgfsys@transformshift{2.118894in}{2.340958in}%
\pgfsys@useobject{currentmarker}{}%
\end{pgfscope}%
\begin{pgfscope}%
\pgfsys@transformshift{2.123032in}{2.340927in}%
\pgfsys@useobject{currentmarker}{}%
\end{pgfscope}%
\begin{pgfscope}%
\pgfsys@transformshift{2.130121in}{2.341174in}%
\pgfsys@useobject{currentmarker}{}%
\end{pgfscope}%
\begin{pgfscope}%
\pgfsys@transformshift{2.140807in}{2.340404in}%
\pgfsys@useobject{currentmarker}{}%
\end{pgfscope}%
\begin{pgfscope}%
\pgfsys@transformshift{2.153454in}{2.340178in}%
\pgfsys@useobject{currentmarker}{}%
\end{pgfscope}%
\begin{pgfscope}%
\pgfsys@transformshift{2.160332in}{2.339127in}%
\pgfsys@useobject{currentmarker}{}%
\end{pgfscope}%
\begin{pgfscope}%
\pgfsys@transformshift{2.164158in}{2.339178in}%
\pgfsys@useobject{currentmarker}{}%
\end{pgfscope}%
\begin{pgfscope}%
\pgfsys@transformshift{2.171459in}{2.338498in}%
\pgfsys@useobject{currentmarker}{}%
\end{pgfscope}%
\begin{pgfscope}%
\pgfsys@transformshift{2.180246in}{2.338962in}%
\pgfsys@useobject{currentmarker}{}%
\end{pgfscope}%
\begin{pgfscope}%
\pgfsys@transformshift{2.185079in}{2.338707in}%
\pgfsys@useobject{currentmarker}{}%
\end{pgfscope}%
\begin{pgfscope}%
\pgfsys@transformshift{2.192310in}{2.337769in}%
\pgfsys@useobject{currentmarker}{}%
\end{pgfscope}%
\begin{pgfscope}%
\pgfsys@transformshift{2.203278in}{2.337629in}%
\pgfsys@useobject{currentmarker}{}%
\end{pgfscope}%
\begin{pgfscope}%
\pgfsys@transformshift{2.215762in}{2.337241in}%
\pgfsys@useobject{currentmarker}{}%
\end{pgfscope}%
\begin{pgfscope}%
\pgfsys@transformshift{2.222609in}{2.337796in}%
\pgfsys@useobject{currentmarker}{}%
\end{pgfscope}%
\begin{pgfscope}%
\pgfsys@transformshift{2.226385in}{2.337928in}%
\pgfsys@useobject{currentmarker}{}%
\end{pgfscope}%
\begin{pgfscope}%
\pgfsys@transformshift{2.235176in}{2.337616in}%
\pgfsys@useobject{currentmarker}{}%
\end{pgfscope}%
\begin{pgfscope}%
\pgfsys@transformshift{2.248449in}{2.339359in}%
\pgfsys@useobject{currentmarker}{}%
\end{pgfscope}%
\begin{pgfscope}%
\pgfsys@transformshift{2.263383in}{2.338976in}%
\pgfsys@useobject{currentmarker}{}%
\end{pgfscope}%
\begin{pgfscope}%
\pgfsys@transformshift{2.271576in}{2.338354in}%
\pgfsys@useobject{currentmarker}{}%
\end{pgfscope}%
\begin{pgfscope}%
\pgfsys@transformshift{2.276081in}{2.338710in}%
\pgfsys@useobject{currentmarker}{}%
\end{pgfscope}%
\begin{pgfscope}%
\pgfsys@transformshift{2.283770in}{2.338184in}%
\pgfsys@useobject{currentmarker}{}%
\end{pgfscope}%
\begin{pgfscope}%
\pgfsys@transformshift{2.294939in}{2.339969in}%
\pgfsys@useobject{currentmarker}{}%
\end{pgfscope}%
\begin{pgfscope}%
\pgfsys@transformshift{2.307339in}{2.336829in}%
\pgfsys@useobject{currentmarker}{}%
\end{pgfscope}%
\begin{pgfscope}%
\pgfsys@transformshift{2.314363in}{2.337224in}%
\pgfsys@useobject{currentmarker}{}%
\end{pgfscope}%
\begin{pgfscope}%
\pgfsys@transformshift{2.318230in}{2.337096in}%
\pgfsys@useobject{currentmarker}{}%
\end{pgfscope}%
\begin{pgfscope}%
\pgfsys@transformshift{2.326070in}{2.337494in}%
\pgfsys@useobject{currentmarker}{}%
\end{pgfscope}%
\begin{pgfscope}%
\pgfsys@transformshift{2.338231in}{2.337157in}%
\pgfsys@useobject{currentmarker}{}%
\end{pgfscope}%
\begin{pgfscope}%
\pgfsys@transformshift{2.352811in}{2.335880in}%
\pgfsys@useobject{currentmarker}{}%
\end{pgfscope}%
\begin{pgfscope}%
\pgfsys@transformshift{2.369231in}{2.337616in}%
\pgfsys@useobject{currentmarker}{}%
\end{pgfscope}%
\begin{pgfscope}%
\pgfsys@transformshift{2.378309in}{2.337378in}%
\pgfsys@useobject{currentmarker}{}%
\end{pgfscope}%
\begin{pgfscope}%
\pgfsys@transformshift{2.389455in}{2.336624in}%
\pgfsys@useobject{currentmarker}{}%
\end{pgfscope}%
\begin{pgfscope}%
\pgfsys@transformshift{2.404886in}{2.336351in}%
\pgfsys@useobject{currentmarker}{}%
\end{pgfscope}%
\begin{pgfscope}%
\pgfsys@transformshift{2.423754in}{2.334875in}%
\pgfsys@useobject{currentmarker}{}%
\end{pgfscope}%
\begin{pgfscope}%
\pgfsys@transformshift{2.445571in}{2.334017in}%
\pgfsys@useobject{currentmarker}{}%
\end{pgfscope}%
\begin{pgfscope}%
\pgfsys@transformshift{2.468682in}{2.333286in}%
\pgfsys@useobject{currentmarker}{}%
\end{pgfscope}%
\begin{pgfscope}%
\pgfsys@transformshift{2.481398in}{2.333486in}%
\pgfsys@useobject{currentmarker}{}%
\end{pgfscope}%
\begin{pgfscope}%
\pgfsys@transformshift{2.488384in}{2.333133in}%
\pgfsys@useobject{currentmarker}{}%
\end{pgfscope}%
\begin{pgfscope}%
\pgfsys@transformshift{2.498835in}{2.333819in}%
\pgfsys@useobject{currentmarker}{}%
\end{pgfscope}%
\begin{pgfscope}%
\pgfsys@transformshift{2.511378in}{2.334303in}%
\pgfsys@useobject{currentmarker}{}%
\end{pgfscope}%
\begin{pgfscope}%
\pgfsys@transformshift{2.518280in}{2.334189in}%
\pgfsys@useobject{currentmarker}{}%
\end{pgfscope}%
\begin{pgfscope}%
\pgfsys@transformshift{2.526376in}{2.333511in}%
\pgfsys@useobject{currentmarker}{}%
\end{pgfscope}%
\begin{pgfscope}%
\pgfsys@transformshift{2.537410in}{2.334205in}%
\pgfsys@useobject{currentmarker}{}%
\end{pgfscope}%
\begin{pgfscope}%
\pgfsys@transformshift{2.552317in}{2.332584in}%
\pgfsys@useobject{currentmarker}{}%
\end{pgfscope}%
\begin{pgfscope}%
\pgfsys@transformshift{2.570182in}{2.333675in}%
\pgfsys@useobject{currentmarker}{}%
\end{pgfscope}%
\begin{pgfscope}%
\pgfsys@transformshift{2.589497in}{2.331889in}%
\pgfsys@useobject{currentmarker}{}%
\end{pgfscope}%
\begin{pgfscope}%
\pgfsys@transformshift{2.610107in}{2.331755in}%
\pgfsys@useobject{currentmarker}{}%
\end{pgfscope}%
\begin{pgfscope}%
\pgfsys@transformshift{2.621443in}{2.331792in}%
\pgfsys@useobject{currentmarker}{}%
\end{pgfscope}%
\begin{pgfscope}%
\pgfsys@transformshift{2.636340in}{2.331798in}%
\pgfsys@useobject{currentmarker}{}%
\end{pgfscope}%
\begin{pgfscope}%
\pgfsys@transformshift{2.654291in}{2.334136in}%
\pgfsys@useobject{currentmarker}{}%
\end{pgfscope}%
\begin{pgfscope}%
\pgfsys@transformshift{2.675195in}{2.331596in}%
\pgfsys@useobject{currentmarker}{}%
\end{pgfscope}%
\begin{pgfscope}%
\pgfsys@transformshift{2.696953in}{2.336600in}%
\pgfsys@useobject{currentmarker}{}%
\end{pgfscope}%
\begin{pgfscope}%
\pgfsys@transformshift{2.709205in}{2.335785in}%
\pgfsys@useobject{currentmarker}{}%
\end{pgfscope}%
\begin{pgfscope}%
\pgfsys@transformshift{2.725496in}{2.338419in}%
\pgfsys@useobject{currentmarker}{}%
\end{pgfscope}%
\begin{pgfscope}%
\pgfsys@transformshift{2.744569in}{2.340801in}%
\pgfsys@useobject{currentmarker}{}%
\end{pgfscope}%
\begin{pgfscope}%
\pgfsys@transformshift{2.765786in}{2.338685in}%
\pgfsys@useobject{currentmarker}{}%
\end{pgfscope}%
\begin{pgfscope}%
\pgfsys@transformshift{2.777287in}{2.340977in}%
\pgfsys@useobject{currentmarker}{}%
\end{pgfscope}%
\begin{pgfscope}%
\pgfsys@transformshift{2.783730in}{2.340663in}%
\pgfsys@useobject{currentmarker}{}%
\end{pgfscope}%
\begin{pgfscope}%
\pgfsys@transformshift{2.793478in}{2.341808in}%
\pgfsys@useobject{currentmarker}{}%
\end{pgfscope}%
\begin{pgfscope}%
\pgfsys@transformshift{2.806910in}{2.342768in}%
\pgfsys@useobject{currentmarker}{}%
\end{pgfscope}%
\begin{pgfscope}%
\pgfsys@transformshift{2.823512in}{2.340333in}%
\pgfsys@useobject{currentmarker}{}%
\end{pgfscope}%
\begin{pgfscope}%
\pgfsys@transformshift{2.841433in}{2.342831in}%
\pgfsys@useobject{currentmarker}{}%
\end{pgfscope}%
\begin{pgfscope}%
\pgfsys@transformshift{2.851378in}{2.342438in}%
\pgfsys@useobject{currentmarker}{}%
\end{pgfscope}%
\begin{pgfscope}%
\pgfsys@transformshift{2.864499in}{2.343893in}%
\pgfsys@useobject{currentmarker}{}%
\end{pgfscope}%
\begin{pgfscope}%
\pgfsys@transformshift{2.880965in}{2.343284in}%
\pgfsys@useobject{currentmarker}{}%
\end{pgfscope}%
\begin{pgfscope}%
\pgfsys@transformshift{2.889999in}{2.342563in}%
\pgfsys@useobject{currentmarker}{}%
\end{pgfscope}%
\begin{pgfscope}%
\pgfsys@transformshift{2.900584in}{2.343757in}%
\pgfsys@useobject{currentmarker}{}%
\end{pgfscope}%
\begin{pgfscope}%
\pgfsys@transformshift{2.906442in}{2.343699in}%
\pgfsys@useobject{currentmarker}{}%
\end{pgfscope}%
\begin{pgfscope}%
\pgfsys@transformshift{2.914737in}{2.343362in}%
\pgfsys@useobject{currentmarker}{}%
\end{pgfscope}%
\begin{pgfscope}%
\pgfsys@transformshift{2.926335in}{2.343621in}%
\pgfsys@useobject{currentmarker}{}%
\end{pgfscope}%
\begin{pgfscope}%
\pgfsys@transformshift{2.941056in}{2.342132in}%
\pgfsys@useobject{currentmarker}{}%
\end{pgfscope}%
\begin{pgfscope}%
\pgfsys@transformshift{2.957600in}{2.344304in}%
\pgfsys@useobject{currentmarker}{}%
\end{pgfscope}%
\begin{pgfscope}%
\pgfsys@transformshift{2.975073in}{2.340603in}%
\pgfsys@useobject{currentmarker}{}%
\end{pgfscope}%
\begin{pgfscope}%
\pgfsys@transformshift{2.995290in}{2.342729in}%
\pgfsys@useobject{currentmarker}{}%
\end{pgfscope}%
\begin{pgfscope}%
\pgfsys@transformshift{3.018792in}{2.340908in}%
\pgfsys@useobject{currentmarker}{}%
\end{pgfscope}%
\begin{pgfscope}%
\pgfsys@transformshift{3.044057in}{2.344919in}%
\pgfsys@useobject{currentmarker}{}%
\end{pgfscope}%
\begin{pgfscope}%
\pgfsys@transformshift{3.058126in}{2.345106in}%
\pgfsys@useobject{currentmarker}{}%
\end{pgfscope}%
\begin{pgfscope}%
\pgfsys@transformshift{3.065862in}{2.344913in}%
\pgfsys@useobject{currentmarker}{}%
\end{pgfscope}%
\begin{pgfscope}%
\pgfsys@transformshift{3.075162in}{2.344221in}%
\pgfsys@useobject{currentmarker}{}%
\end{pgfscope}%
\begin{pgfscope}%
\pgfsys@transformshift{3.086354in}{2.344440in}%
\pgfsys@useobject{currentmarker}{}%
\end{pgfscope}%
\begin{pgfscope}%
\pgfsys@transformshift{3.100978in}{2.345706in}%
\pgfsys@useobject{currentmarker}{}%
\end{pgfscope}%
\begin{pgfscope}%
\pgfsys@transformshift{3.119777in}{2.347768in}%
\pgfsys@useobject{currentmarker}{}%
\end{pgfscope}%
\begin{pgfscope}%
\pgfsys@transformshift{3.139912in}{2.348377in}%
\pgfsys@useobject{currentmarker}{}%
\end{pgfscope}%
\begin{pgfscope}%
\pgfsys@transformshift{3.150977in}{2.348960in}%
\pgfsys@useobject{currentmarker}{}%
\end{pgfscope}%
\begin{pgfscope}%
\pgfsys@transformshift{3.157070in}{2.348862in}%
\pgfsys@useobject{currentmarker}{}%
\end{pgfscope}%
\begin{pgfscope}%
\pgfsys@transformshift{3.165967in}{2.348783in}%
\pgfsys@useobject{currentmarker}{}%
\end{pgfscope}%
\begin{pgfscope}%
\pgfsys@transformshift{3.180681in}{2.348845in}%
\pgfsys@useobject{currentmarker}{}%
\end{pgfscope}%
\begin{pgfscope}%
\pgfsys@transformshift{3.198074in}{2.346532in}%
\pgfsys@useobject{currentmarker}{}%
\end{pgfscope}%
\begin{pgfscope}%
\pgfsys@transformshift{3.217209in}{2.347926in}%
\pgfsys@useobject{currentmarker}{}%
\end{pgfscope}%
\begin{pgfscope}%
\pgfsys@transformshift{3.227714in}{2.346924in}%
\pgfsys@useobject{currentmarker}{}%
\end{pgfscope}%
\begin{pgfscope}%
\pgfsys@transformshift{3.240633in}{2.347286in}%
\pgfsys@useobject{currentmarker}{}%
\end{pgfscope}%
\begin{pgfscope}%
\pgfsys@transformshift{3.255294in}{2.347017in}%
\pgfsys@useobject{currentmarker}{}%
\end{pgfscope}%
\begin{pgfscope}%
\pgfsys@transformshift{3.272440in}{2.347005in}%
\pgfsys@useobject{currentmarker}{}%
\end{pgfscope}%
\begin{pgfscope}%
\pgfsys@transformshift{3.291943in}{2.346456in}%
\pgfsys@useobject{currentmarker}{}%
\end{pgfscope}%
\begin{pgfscope}%
\pgfsys@transformshift{3.312652in}{2.346016in}%
\pgfsys@useobject{currentmarker}{}%
\end{pgfscope}%
\begin{pgfscope}%
\pgfsys@transformshift{3.324026in}{2.345381in}%
\pgfsys@useobject{currentmarker}{}%
\end{pgfscope}%
\begin{pgfscope}%
\pgfsys@transformshift{3.337728in}{2.346037in}%
\pgfsys@useobject{currentmarker}{}%
\end{pgfscope}%
\begin{pgfscope}%
\pgfsys@transformshift{3.353608in}{2.345972in}%
\pgfsys@useobject{currentmarker}{}%
\end{pgfscope}%
\begin{pgfscope}%
\pgfsys@transformshift{3.372423in}{2.344919in}%
\pgfsys@useobject{currentmarker}{}%
\end{pgfscope}%
\begin{pgfscope}%
\pgfsys@transformshift{3.394072in}{2.343333in}%
\pgfsys@useobject{currentmarker}{}%
\end{pgfscope}%
\begin{pgfscope}%
\pgfsys@transformshift{3.417412in}{2.343603in}%
\pgfsys@useobject{currentmarker}{}%
\end{pgfscope}%
\begin{pgfscope}%
\pgfsys@transformshift{3.430232in}{2.342936in}%
\pgfsys@useobject{currentmarker}{}%
\end{pgfscope}%
\begin{pgfscope}%
\pgfsys@transformshift{3.445496in}{2.343541in}%
\pgfsys@useobject{currentmarker}{}%
\end{pgfscope}%
\begin{pgfscope}%
\pgfsys@transformshift{3.464460in}{2.343260in}%
\pgfsys@useobject{currentmarker}{}%
\end{pgfscope}%
\begin{pgfscope}%
\pgfsys@transformshift{3.486151in}{2.342067in}%
\pgfsys@useobject{currentmarker}{}%
\end{pgfscope}%
\begin{pgfscope}%
\pgfsys@transformshift{3.509761in}{2.341368in}%
\pgfsys@useobject{currentmarker}{}%
\end{pgfscope}%
\begin{pgfscope}%
\pgfsys@transformshift{3.534567in}{2.343273in}%
\pgfsys@useobject{currentmarker}{}%
\end{pgfscope}%
\begin{pgfscope}%
\pgfsys@transformshift{3.548240in}{2.342755in}%
\pgfsys@useobject{currentmarker}{}%
\end{pgfscope}%
\begin{pgfscope}%
\pgfsys@transformshift{3.555742in}{2.343356in}%
\pgfsys@useobject{currentmarker}{}%
\end{pgfscope}%
\begin{pgfscope}%
\pgfsys@transformshift{3.565668in}{2.341662in}%
\pgfsys@useobject{currentmarker}{}%
\end{pgfscope}%
\begin{pgfscope}%
\pgfsys@transformshift{3.580371in}{2.341998in}%
\pgfsys@useobject{currentmarker}{}%
\end{pgfscope}%
\begin{pgfscope}%
\pgfsys@transformshift{3.598980in}{2.340674in}%
\pgfsys@useobject{currentmarker}{}%
\end{pgfscope}%
\begin{pgfscope}%
\pgfsys@transformshift{3.620548in}{2.341075in}%
\pgfsys@useobject{currentmarker}{}%
\end{pgfscope}%
\begin{pgfscope}%
\pgfsys@transformshift{3.643455in}{2.342935in}%
\pgfsys@useobject{currentmarker}{}%
\end{pgfscope}%
\begin{pgfscope}%
\pgfsys@transformshift{3.656029in}{2.344225in}%
\pgfsys@useobject{currentmarker}{}%
\end{pgfscope}%
\begin{pgfscope}%
\pgfsys@transformshift{3.662979in}{2.344039in}%
\pgfsys@useobject{currentmarker}{}%
\end{pgfscope}%
\begin{pgfscope}%
\pgfsys@transformshift{3.672229in}{2.345609in}%
\pgfsys@useobject{currentmarker}{}%
\end{pgfscope}%
\begin{pgfscope}%
\pgfsys@transformshift{3.685148in}{2.345654in}%
\pgfsys@useobject{currentmarker}{}%
\end{pgfscope}%
\begin{pgfscope}%
\pgfsys@transformshift{3.701125in}{2.346431in}%
\pgfsys@useobject{currentmarker}{}%
\end{pgfscope}%
\begin{pgfscope}%
\pgfsys@transformshift{3.719647in}{2.347319in}%
\pgfsys@useobject{currentmarker}{}%
\end{pgfscope}%
\begin{pgfscope}%
\pgfsys@transformshift{3.740787in}{2.347118in}%
\pgfsys@useobject{currentmarker}{}%
\end{pgfscope}%
\begin{pgfscope}%
\pgfsys@transformshift{3.763395in}{2.346466in}%
\pgfsys@useobject{currentmarker}{}%
\end{pgfscope}%
\begin{pgfscope}%
\pgfsys@transformshift{3.775813in}{2.347197in}%
\pgfsys@useobject{currentmarker}{}%
\end{pgfscope}%
\begin{pgfscope}%
\pgfsys@transformshift{3.790488in}{2.346252in}%
\pgfsys@useobject{currentmarker}{}%
\end{pgfscope}%
\begin{pgfscope}%
\pgfsys@transformshift{3.808247in}{2.346202in}%
\pgfsys@useobject{currentmarker}{}%
\end{pgfscope}%
\begin{pgfscope}%
\pgfsys@transformshift{3.828533in}{2.347026in}%
\pgfsys@useobject{currentmarker}{}%
\end{pgfscope}%
\begin{pgfscope}%
\pgfsys@transformshift{3.852154in}{2.349079in}%
\pgfsys@useobject{currentmarker}{}%
\end{pgfscope}%
\begin{pgfscope}%
\pgfsys@transformshift{3.877100in}{2.347601in}%
\pgfsys@useobject{currentmarker}{}%
\end{pgfscope}%
\begin{pgfscope}%
\pgfsys@transformshift{3.890685in}{2.349685in}%
\pgfsys@useobject{currentmarker}{}%
\end{pgfscope}%
\begin{pgfscope}%
\pgfsys@transformshift{3.898232in}{2.349263in}%
\pgfsys@useobject{currentmarker}{}%
\end{pgfscope}%
\begin{pgfscope}%
\pgfsys@transformshift{3.907440in}{2.350132in}%
\pgfsys@useobject{currentmarker}{}%
\end{pgfscope}%
\begin{pgfscope}%
\pgfsys@transformshift{3.917845in}{2.350193in}%
\pgfsys@useobject{currentmarker}{}%
\end{pgfscope}%
\begin{pgfscope}%
\pgfsys@transformshift{3.932211in}{2.351266in}%
\pgfsys@useobject{currentmarker}{}%
\end{pgfscope}%
\begin{pgfscope}%
\pgfsys@transformshift{3.949583in}{2.352411in}%
\pgfsys@useobject{currentmarker}{}%
\end{pgfscope}%
\begin{pgfscope}%
\pgfsys@transformshift{3.969090in}{2.353468in}%
\pgfsys@useobject{currentmarker}{}%
\end{pgfscope}%
\begin{pgfscope}%
\pgfsys@transformshift{3.979826in}{2.353891in}%
\pgfsys@useobject{currentmarker}{}%
\end{pgfscope}%
\begin{pgfscope}%
\pgfsys@transformshift{3.985732in}{2.354108in}%
\pgfsys@useobject{currentmarker}{}%
\end{pgfscope}%
\begin{pgfscope}%
\pgfsys@transformshift{3.988979in}{2.354254in}%
\pgfsys@useobject{currentmarker}{}%
\end{pgfscope}%
\begin{pgfscope}%
\pgfsys@transformshift{3.990767in}{2.354276in}%
\pgfsys@useobject{currentmarker}{}%
\end{pgfscope}%
\begin{pgfscope}%
\pgfsys@transformshift{3.995300in}{2.354619in}%
\pgfsys@useobject{currentmarker}{}%
\end{pgfscope}%
\begin{pgfscope}%
\pgfsys@transformshift{4.003800in}{2.354772in}%
\pgfsys@useobject{currentmarker}{}%
\end{pgfscope}%
\begin{pgfscope}%
\pgfsys@transformshift{4.014800in}{2.355507in}%
\pgfsys@useobject{currentmarker}{}%
\end{pgfscope}%
\begin{pgfscope}%
\pgfsys@transformshift{4.020836in}{2.356078in}%
\pgfsys@useobject{currentmarker}{}%
\end{pgfscope}%
\begin{pgfscope}%
\pgfsys@transformshift{4.030271in}{2.357046in}%
\pgfsys@useobject{currentmarker}{}%
\end{pgfscope}%
\begin{pgfscope}%
\pgfsys@transformshift{4.041190in}{2.358314in}%
\pgfsys@useobject{currentmarker}{}%
\end{pgfscope}%
\begin{pgfscope}%
\pgfsys@transformshift{4.053330in}{2.358669in}%
\pgfsys@useobject{currentmarker}{}%
\end{pgfscope}%
\begin{pgfscope}%
\pgfsys@transformshift{4.059953in}{2.359539in}%
\pgfsys@useobject{currentmarker}{}%
\end{pgfscope}%
\begin{pgfscope}%
\pgfsys@transformshift{4.063610in}{2.359189in}%
\pgfsys@useobject{currentmarker}{}%
\end{pgfscope}%
\begin{pgfscope}%
\pgfsys@transformshift{4.069738in}{2.360054in}%
\pgfsys@useobject{currentmarker}{}%
\end{pgfscope}%
\begin{pgfscope}%
\pgfsys@transformshift{4.078101in}{2.360023in}%
\pgfsys@useobject{currentmarker}{}%
\end{pgfscope}%
\begin{pgfscope}%
\pgfsys@transformshift{4.089026in}{2.360937in}%
\pgfsys@useobject{currentmarker}{}%
\end{pgfscope}%
\begin{pgfscope}%
\pgfsys@transformshift{4.103658in}{2.361836in}%
\pgfsys@useobject{currentmarker}{}%
\end{pgfscope}%
\begin{pgfscope}%
\pgfsys@transformshift{4.119722in}{2.362202in}%
\pgfsys@useobject{currentmarker}{}%
\end{pgfscope}%
\begin{pgfscope}%
\pgfsys@transformshift{4.128536in}{2.362833in}%
\pgfsys@useobject{currentmarker}{}%
\end{pgfscope}%
\begin{pgfscope}%
\pgfsys@transformshift{4.138761in}{2.362135in}%
\pgfsys@useobject{currentmarker}{}%
\end{pgfscope}%
\begin{pgfscope}%
\pgfsys@transformshift{4.151269in}{2.362001in}%
\pgfsys@useobject{currentmarker}{}%
\end{pgfscope}%
\begin{pgfscope}%
\pgfsys@transformshift{4.158088in}{2.362911in}%
\pgfsys@useobject{currentmarker}{}%
\end{pgfscope}%
\begin{pgfscope}%
\pgfsys@transformshift{4.168101in}{2.361659in}%
\pgfsys@useobject{currentmarker}{}%
\end{pgfscope}%
\begin{pgfscope}%
\pgfsys@transformshift{4.183590in}{2.362561in}%
\pgfsys@useobject{currentmarker}{}%
\end{pgfscope}%
\begin{pgfscope}%
\pgfsys@transformshift{4.201703in}{2.361808in}%
\pgfsys@useobject{currentmarker}{}%
\end{pgfscope}%
\begin{pgfscope}%
\pgfsys@transformshift{4.211608in}{2.362955in}%
\pgfsys@useobject{currentmarker}{}%
\end{pgfscope}%
\begin{pgfscope}%
\pgfsys@transformshift{4.217085in}{2.362665in}%
\pgfsys@useobject{currentmarker}{}%
\end{pgfscope}%
\begin{pgfscope}%
\pgfsys@transformshift{4.220099in}{2.362781in}%
\pgfsys@useobject{currentmarker}{}%
\end{pgfscope}%
\begin{pgfscope}%
\pgfsys@transformshift{4.225179in}{2.362813in}%
\pgfsys@useobject{currentmarker}{}%
\end{pgfscope}%
\begin{pgfscope}%
\pgfsys@transformshift{4.233415in}{2.362778in}%
\pgfsys@useobject{currentmarker}{}%
\end{pgfscope}%
\begin{pgfscope}%
\pgfsys@transformshift{4.246253in}{2.363948in}%
\pgfsys@useobject{currentmarker}{}%
\end{pgfscope}%
\begin{pgfscope}%
\pgfsys@transformshift{4.262091in}{2.361256in}%
\pgfsys@useobject{currentmarker}{}%
\end{pgfscope}%
\begin{pgfscope}%
\pgfsys@transformshift{4.280249in}{2.363477in}%
\pgfsys@useobject{currentmarker}{}%
\end{pgfscope}%
\begin{pgfscope}%
\pgfsys@transformshift{4.290226in}{2.362178in}%
\pgfsys@useobject{currentmarker}{}%
\end{pgfscope}%
\begin{pgfscope}%
\pgfsys@transformshift{4.295760in}{2.362122in}%
\pgfsys@useobject{currentmarker}{}%
\end{pgfscope}%
\begin{pgfscope}%
\pgfsys@transformshift{4.298803in}{2.362165in}%
\pgfsys@useobject{currentmarker}{}%
\end{pgfscope}%
\begin{pgfscope}%
\pgfsys@transformshift{4.305433in}{2.362420in}%
\pgfsys@useobject{currentmarker}{}%
\end{pgfscope}%
\begin{pgfscope}%
\pgfsys@transformshift{4.318511in}{2.362971in}%
\pgfsys@useobject{currentmarker}{}%
\end{pgfscope}%
\begin{pgfscope}%
\pgfsys@transformshift{4.336332in}{2.361667in}%
\pgfsys@useobject{currentmarker}{}%
\end{pgfscope}%
\begin{pgfscope}%
\pgfsys@transformshift{4.356063in}{2.363077in}%
\pgfsys@useobject{currentmarker}{}%
\end{pgfscope}%
\begin{pgfscope}%
\pgfsys@transformshift{4.366872in}{2.361843in}%
\pgfsys@useobject{currentmarker}{}%
\end{pgfscope}%
\begin{pgfscope}%
\pgfsys@transformshift{4.372826in}{2.362442in}%
\pgfsys@useobject{currentmarker}{}%
\end{pgfscope}%
\begin{pgfscope}%
\pgfsys@transformshift{4.379911in}{2.361902in}%
\pgfsys@useobject{currentmarker}{}%
\end{pgfscope}%
\begin{pgfscope}%
\pgfsys@transformshift{4.389744in}{2.362425in}%
\pgfsys@useobject{currentmarker}{}%
\end{pgfscope}%
\begin{pgfscope}%
\pgfsys@transformshift{4.401835in}{2.362160in}%
\pgfsys@useobject{currentmarker}{}%
\end{pgfscope}%
\begin{pgfscope}%
\pgfsys@transformshift{4.416797in}{2.362668in}%
\pgfsys@useobject{currentmarker}{}%
\end{pgfscope}%
\begin{pgfscope}%
\pgfsys@transformshift{4.435392in}{2.363778in}%
\pgfsys@useobject{currentmarker}{}%
\end{pgfscope}%
\begin{pgfscope}%
\pgfsys@transformshift{4.457249in}{2.362388in}%
\pgfsys@useobject{currentmarker}{}%
\end{pgfscope}%
\begin{pgfscope}%
\pgfsys@transformshift{4.482146in}{2.364950in}%
\pgfsys@useobject{currentmarker}{}%
\end{pgfscope}%
\begin{pgfscope}%
\pgfsys@transformshift{4.495735in}{2.362754in}%
\pgfsys@useobject{currentmarker}{}%
\end{pgfscope}%
\begin{pgfscope}%
\pgfsys@transformshift{4.510690in}{2.364294in}%
\pgfsys@useobject{currentmarker}{}%
\end{pgfscope}%
\begin{pgfscope}%
\pgfsys@transformshift{4.527877in}{2.362760in}%
\pgfsys@useobject{currentmarker}{}%
\end{pgfscope}%
\begin{pgfscope}%
\pgfsys@transformshift{4.547825in}{2.363011in}%
\pgfsys@useobject{currentmarker}{}%
\end{pgfscope}%
\begin{pgfscope}%
\pgfsys@transformshift{4.572775in}{2.363454in}%
\pgfsys@useobject{currentmarker}{}%
\end{pgfscope}%
\begin{pgfscope}%
\pgfsys@transformshift{4.601655in}{2.366958in}%
\pgfsys@useobject{currentmarker}{}%
\end{pgfscope}%
\begin{pgfscope}%
\pgfsys@transformshift{4.633150in}{2.369115in}%
\pgfsys@useobject{currentmarker}{}%
\end{pgfscope}%
\begin{pgfscope}%
\pgfsys@transformshift{4.666100in}{2.371717in}%
\pgfsys@useobject{currentmarker}{}%
\end{pgfscope}%
\begin{pgfscope}%
\pgfsys@transformshift{4.700900in}{2.376105in}%
\pgfsys@useobject{currentmarker}{}%
\end{pgfscope}%
\begin{pgfscope}%
\pgfsys@transformshift{4.737031in}{2.371361in}%
\pgfsys@useobject{currentmarker}{}%
\end{pgfscope}%
\begin{pgfscope}%
\pgfsys@transformshift{4.756554in}{2.375892in}%
\pgfsys@useobject{currentmarker}{}%
\end{pgfscope}%
\begin{pgfscope}%
\pgfsys@transformshift{4.767455in}{2.374249in}%
\pgfsys@useobject{currentmarker}{}%
\end{pgfscope}%
\begin{pgfscope}%
\pgfsys@transformshift{4.780282in}{2.375609in}%
\pgfsys@useobject{currentmarker}{}%
\end{pgfscope}%
\begin{pgfscope}%
\pgfsys@transformshift{4.795786in}{2.375018in}%
\pgfsys@useobject{currentmarker}{}%
\end{pgfscope}%
\begin{pgfscope}%
\pgfsys@transformshift{4.814242in}{2.376490in}%
\pgfsys@useobject{currentmarker}{}%
\end{pgfscope}%
\begin{pgfscope}%
\pgfsys@transformshift{4.837015in}{2.375298in}%
\pgfsys@useobject{currentmarker}{}%
\end{pgfscope}%
\begin{pgfscope}%
\pgfsys@transformshift{4.862152in}{2.378344in}%
\pgfsys@useobject{currentmarker}{}%
\end{pgfscope}%
\begin{pgfscope}%
\pgfsys@transformshift{4.889239in}{2.380510in}%
\pgfsys@useobject{currentmarker}{}%
\end{pgfscope}%
\begin{pgfscope}%
\pgfsys@transformshift{4.918008in}{2.374690in}%
\pgfsys@useobject{currentmarker}{}%
\end{pgfscope}%
\begin{pgfscope}%
\pgfsys@transformshift{4.934067in}{2.376336in}%
\pgfsys@useobject{currentmarker}{}%
\end{pgfscope}%
\begin{pgfscope}%
\pgfsys@transformshift{4.942820in}{2.374849in}%
\pgfsys@useobject{currentmarker}{}%
\end{pgfscope}%
\begin{pgfscope}%
\pgfsys@transformshift{4.953032in}{2.375535in}%
\pgfsys@useobject{currentmarker}{}%
\end{pgfscope}%
\begin{pgfscope}%
\pgfsys@transformshift{4.958628in}{2.374924in}%
\pgfsys@useobject{currentmarker}{}%
\end{pgfscope}%
\begin{pgfscope}%
\pgfsys@transformshift{4.967787in}{2.375761in}%
\pgfsys@useobject{currentmarker}{}%
\end{pgfscope}%
\begin{pgfscope}%
\pgfsys@transformshift{4.979787in}{2.375317in}%
\pgfsys@useobject{currentmarker}{}%
\end{pgfscope}%
\begin{pgfscope}%
\pgfsys@transformshift{4.994665in}{2.376916in}%
\pgfsys@useobject{currentmarker}{}%
\end{pgfscope}%
\begin{pgfscope}%
\pgfsys@transformshift{5.011816in}{2.377226in}%
\pgfsys@useobject{currentmarker}{}%
\end{pgfscope}%
\begin{pgfscope}%
\pgfsys@transformshift{5.021245in}{2.377559in}%
\pgfsys@useobject{currentmarker}{}%
\end{pgfscope}%
\begin{pgfscope}%
\pgfsys@transformshift{5.026425in}{2.377864in}%
\pgfsys@useobject{currentmarker}{}%
\end{pgfscope}%
\begin{pgfscope}%
\pgfsys@transformshift{5.029271in}{2.377658in}%
\pgfsys@useobject{currentmarker}{}%
\end{pgfscope}%
\begin{pgfscope}%
\pgfsys@transformshift{5.035258in}{2.377352in}%
\pgfsys@useobject{currentmarker}{}%
\end{pgfscope}%
\begin{pgfscope}%
\pgfsys@transformshift{5.044049in}{2.377506in}%
\pgfsys@useobject{currentmarker}{}%
\end{pgfscope}%
\begin{pgfscope}%
\pgfsys@transformshift{5.054851in}{2.377270in}%
\pgfsys@useobject{currentmarker}{}%
\end{pgfscope}%
\begin{pgfscope}%
\pgfsys@transformshift{5.068477in}{2.378761in}%
\pgfsys@useobject{currentmarker}{}%
\end{pgfscope}%
\begin{pgfscope}%
\pgfsys@transformshift{5.084115in}{2.376116in}%
\pgfsys@useobject{currentmarker}{}%
\end{pgfscope}%
\begin{pgfscope}%
\pgfsys@transformshift{5.092716in}{2.377571in}%
\pgfsys@useobject{currentmarker}{}%
\end{pgfscope}%
\begin{pgfscope}%
\pgfsys@transformshift{5.097391in}{2.376491in}%
\pgfsys@useobject{currentmarker}{}%
\end{pgfscope}%
\begin{pgfscope}%
\pgfsys@transformshift{5.103876in}{2.376993in}%
\pgfsys@useobject{currentmarker}{}%
\end{pgfscope}%
\begin{pgfscope}%
\pgfsys@transformshift{5.111551in}{2.376508in}%
\pgfsys@useobject{currentmarker}{}%
\end{pgfscope}%
\begin{pgfscope}%
\pgfsys@transformshift{5.122519in}{2.377470in}%
\pgfsys@useobject{currentmarker}{}%
\end{pgfscope}%
\begin{pgfscope}%
\pgfsys@transformshift{5.135583in}{2.377376in}%
\pgfsys@useobject{currentmarker}{}%
\end{pgfscope}%
\begin{pgfscope}%
\pgfsys@transformshift{5.150805in}{2.376911in}%
\pgfsys@useobject{currentmarker}{}%
\end{pgfscope}%
\begin{pgfscope}%
\pgfsys@transformshift{5.168012in}{2.377013in}%
\pgfsys@useobject{currentmarker}{}%
\end{pgfscope}%
\begin{pgfscope}%
\pgfsys@transformshift{5.177430in}{2.376087in}%
\pgfsys@useobject{currentmarker}{}%
\end{pgfscope}%
\begin{pgfscope}%
\pgfsys@transformshift{5.182632in}{2.375929in}%
\pgfsys@useobject{currentmarker}{}%
\end{pgfscope}%
\begin{pgfscope}%
\pgfsys@transformshift{5.189218in}{2.375319in}%
\pgfsys@useobject{currentmarker}{}%
\end{pgfscope}%
\begin{pgfscope}%
\pgfsys@transformshift{5.197651in}{2.375125in}%
\pgfsys@useobject{currentmarker}{}%
\end{pgfscope}%
\begin{pgfscope}%
\pgfsys@transformshift{5.202287in}{2.374981in}%
\pgfsys@useobject{currentmarker}{}%
\end{pgfscope}%
\begin{pgfscope}%
\pgfsys@transformshift{5.204832in}{2.374789in}%
\pgfsys@useobject{currentmarker}{}%
\end{pgfscope}%
\begin{pgfscope}%
\pgfsys@transformshift{5.206204in}{2.374494in}%
\pgfsys@useobject{currentmarker}{}%
\end{pgfscope}%
\begin{pgfscope}%
\pgfsys@transformshift{5.206974in}{2.374441in}%
\pgfsys@useobject{currentmarker}{}%
\end{pgfscope}%
\begin{pgfscope}%
\pgfsys@transformshift{5.210086in}{2.374212in}%
\pgfsys@useobject{currentmarker}{}%
\end{pgfscope}%
\begin{pgfscope}%
\pgfsys@transformshift{5.216393in}{2.374002in}%
\pgfsys@useobject{currentmarker}{}%
\end{pgfscope}%
\begin{pgfscope}%
\pgfsys@transformshift{5.224315in}{2.372856in}%
\pgfsys@useobject{currentmarker}{}%
\end{pgfscope}%
\begin{pgfscope}%
\pgfsys@transformshift{5.234105in}{2.372669in}%
\pgfsys@useobject{currentmarker}{}%
\end{pgfscope}%
\begin{pgfscope}%
\pgfsys@transformshift{5.246425in}{2.369796in}%
\pgfsys@useobject{currentmarker}{}%
\end{pgfscope}%
\begin{pgfscope}%
\pgfsys@transformshift{5.265343in}{2.366735in}%
\pgfsys@useobject{currentmarker}{}%
\end{pgfscope}%
\begin{pgfscope}%
\pgfsys@transformshift{5.275817in}{2.365564in}%
\pgfsys@useobject{currentmarker}{}%
\end{pgfscope}%
\begin{pgfscope}%
\pgfsys@transformshift{5.281583in}{2.364960in}%
\pgfsys@useobject{currentmarker}{}%
\end{pgfscope}%
\begin{pgfscope}%
\pgfsys@transformshift{5.284741in}{2.364518in}%
\pgfsys@useobject{currentmarker}{}%
\end{pgfscope}%
\begin{pgfscope}%
\pgfsys@transformshift{5.286490in}{2.364397in}%
\pgfsys@useobject{currentmarker}{}%
\end{pgfscope}%
\begin{pgfscope}%
\pgfsys@transformshift{5.287448in}{2.364283in}%
\pgfsys@useobject{currentmarker}{}%
\end{pgfscope}%
\begin{pgfscope}%
\pgfsys@transformshift{5.289529in}{2.364395in}%
\pgfsys@useobject{currentmarker}{}%
\end{pgfscope}%
\begin{pgfscope}%
\pgfsys@transformshift{5.292818in}{2.364444in}%
\pgfsys@useobject{currentmarker}{}%
\end{pgfscope}%
\begin{pgfscope}%
\pgfsys@transformshift{5.294549in}{2.364971in}%
\pgfsys@useobject{currentmarker}{}%
\end{pgfscope}%
\begin{pgfscope}%
\pgfsys@transformshift{5.297651in}{2.366048in}%
\pgfsys@useobject{currentmarker}{}%
\end{pgfscope}%
\begin{pgfscope}%
\pgfsys@transformshift{5.299390in}{2.366533in}%
\pgfsys@useobject{currentmarker}{}%
\end{pgfscope}%
\begin{pgfscope}%
\pgfsys@transformshift{5.300186in}{2.367127in}%
\pgfsys@useobject{currentmarker}{}%
\end{pgfscope}%
\begin{pgfscope}%
\pgfsys@transformshift{5.302110in}{2.368020in}%
\pgfsys@useobject{currentmarker}{}%
\end{pgfscope}%
\begin{pgfscope}%
\pgfsys@transformshift{5.302943in}{2.368837in}%
\pgfsys@useobject{currentmarker}{}%
\end{pgfscope}%
\begin{pgfscope}%
\pgfsys@transformshift{5.303439in}{2.369244in}%
\pgfsys@useobject{currentmarker}{}%
\end{pgfscope}%
\begin{pgfscope}%
\pgfsys@transformshift{5.303712in}{2.369467in}%
\pgfsys@useobject{currentmarker}{}%
\end{pgfscope}%
\end{pgfscope}%
\begin{pgfscope}%
\pgfsetbuttcap%
\pgfsetroundjoin%
\definecolor{currentfill}{rgb}{0.000000,0.000000,0.000000}%
\pgfsetfillcolor{currentfill}%
\pgfsetlinewidth{0.803000pt}%
\definecolor{currentstroke}{rgb}{0.000000,0.000000,0.000000}%
\pgfsetstrokecolor{currentstroke}%
\pgfsetdash{}{0pt}%
\pgfsys@defobject{currentmarker}{\pgfqpoint{0.000000in}{-0.048611in}}{\pgfqpoint{0.000000in}{0.000000in}}{%
\pgfpathmoveto{\pgfqpoint{0.000000in}{0.000000in}}%
\pgfpathlineto{\pgfqpoint{0.000000in}{-0.048611in}}%
\pgfusepath{stroke,fill}%
}%
\begin{pgfscope}%
\pgfsys@transformshift{0.794621in}{0.515000in}%
\pgfsys@useobject{currentmarker}{}%
\end{pgfscope}%
\end{pgfscope}%
\begin{pgfscope}%
\definecolor{textcolor}{rgb}{0.000000,0.000000,0.000000}%
\pgfsetstrokecolor{textcolor}%
\pgfsetfillcolor{textcolor}%
\pgftext[x=0.794621in,y=0.417777in,,top]{\color{textcolor}\rmfamily\fontsize{10.000000}{12.000000}\selectfont \(\displaystyle {0.0}\)}%
\end{pgfscope}%
\begin{pgfscope}%
\pgfsetbuttcap%
\pgfsetroundjoin%
\definecolor{currentfill}{rgb}{0.000000,0.000000,0.000000}%
\pgfsetfillcolor{currentfill}%
\pgfsetlinewidth{0.803000pt}%
\definecolor{currentstroke}{rgb}{0.000000,0.000000,0.000000}%
\pgfsetstrokecolor{currentstroke}%
\pgfsetdash{}{0pt}%
\pgfsys@defobject{currentmarker}{\pgfqpoint{0.000000in}{-0.048611in}}{\pgfqpoint{0.000000in}{0.000000in}}{%
\pgfpathmoveto{\pgfqpoint{0.000000in}{0.000000in}}%
\pgfpathlineto{\pgfqpoint{0.000000in}{-0.048611in}}%
\pgfusepath{stroke,fill}%
}%
\begin{pgfscope}%
\pgfsys@transformshift{1.351935in}{0.515000in}%
\pgfsys@useobject{currentmarker}{}%
\end{pgfscope}%
\end{pgfscope}%
\begin{pgfscope}%
\definecolor{textcolor}{rgb}{0.000000,0.000000,0.000000}%
\pgfsetstrokecolor{textcolor}%
\pgfsetfillcolor{textcolor}%
\pgftext[x=1.351935in,y=0.417777in,,top]{\color{textcolor}\rmfamily\fontsize{10.000000}{12.000000}\selectfont \(\displaystyle {2.5}\)}%
\end{pgfscope}%
\begin{pgfscope}%
\pgfsetbuttcap%
\pgfsetroundjoin%
\definecolor{currentfill}{rgb}{0.000000,0.000000,0.000000}%
\pgfsetfillcolor{currentfill}%
\pgfsetlinewidth{0.803000pt}%
\definecolor{currentstroke}{rgb}{0.000000,0.000000,0.000000}%
\pgfsetstrokecolor{currentstroke}%
\pgfsetdash{}{0pt}%
\pgfsys@defobject{currentmarker}{\pgfqpoint{0.000000in}{-0.048611in}}{\pgfqpoint{0.000000in}{0.000000in}}{%
\pgfpathmoveto{\pgfqpoint{0.000000in}{0.000000in}}%
\pgfpathlineto{\pgfqpoint{0.000000in}{-0.048611in}}%
\pgfusepath{stroke,fill}%
}%
\begin{pgfscope}%
\pgfsys@transformshift{1.909250in}{0.515000in}%
\pgfsys@useobject{currentmarker}{}%
\end{pgfscope}%
\end{pgfscope}%
\begin{pgfscope}%
\definecolor{textcolor}{rgb}{0.000000,0.000000,0.000000}%
\pgfsetstrokecolor{textcolor}%
\pgfsetfillcolor{textcolor}%
\pgftext[x=1.909250in,y=0.417777in,,top]{\color{textcolor}\rmfamily\fontsize{10.000000}{12.000000}\selectfont \(\displaystyle {5.0}\)}%
\end{pgfscope}%
\begin{pgfscope}%
\pgfsetbuttcap%
\pgfsetroundjoin%
\definecolor{currentfill}{rgb}{0.000000,0.000000,0.000000}%
\pgfsetfillcolor{currentfill}%
\pgfsetlinewidth{0.803000pt}%
\definecolor{currentstroke}{rgb}{0.000000,0.000000,0.000000}%
\pgfsetstrokecolor{currentstroke}%
\pgfsetdash{}{0pt}%
\pgfsys@defobject{currentmarker}{\pgfqpoint{0.000000in}{-0.048611in}}{\pgfqpoint{0.000000in}{0.000000in}}{%
\pgfpathmoveto{\pgfqpoint{0.000000in}{0.000000in}}%
\pgfpathlineto{\pgfqpoint{0.000000in}{-0.048611in}}%
\pgfusepath{stroke,fill}%
}%
\begin{pgfscope}%
\pgfsys@transformshift{2.466564in}{0.515000in}%
\pgfsys@useobject{currentmarker}{}%
\end{pgfscope}%
\end{pgfscope}%
\begin{pgfscope}%
\definecolor{textcolor}{rgb}{0.000000,0.000000,0.000000}%
\pgfsetstrokecolor{textcolor}%
\pgfsetfillcolor{textcolor}%
\pgftext[x=2.466564in,y=0.417777in,,top]{\color{textcolor}\rmfamily\fontsize{10.000000}{12.000000}\selectfont \(\displaystyle {7.5}\)}%
\end{pgfscope}%
\begin{pgfscope}%
\pgfsetbuttcap%
\pgfsetroundjoin%
\definecolor{currentfill}{rgb}{0.000000,0.000000,0.000000}%
\pgfsetfillcolor{currentfill}%
\pgfsetlinewidth{0.803000pt}%
\definecolor{currentstroke}{rgb}{0.000000,0.000000,0.000000}%
\pgfsetstrokecolor{currentstroke}%
\pgfsetdash{}{0pt}%
\pgfsys@defobject{currentmarker}{\pgfqpoint{0.000000in}{-0.048611in}}{\pgfqpoint{0.000000in}{0.000000in}}{%
\pgfpathmoveto{\pgfqpoint{0.000000in}{0.000000in}}%
\pgfpathlineto{\pgfqpoint{0.000000in}{-0.048611in}}%
\pgfusepath{stroke,fill}%
}%
\begin{pgfscope}%
\pgfsys@transformshift{3.023878in}{0.515000in}%
\pgfsys@useobject{currentmarker}{}%
\end{pgfscope}%
\end{pgfscope}%
\begin{pgfscope}%
\definecolor{textcolor}{rgb}{0.000000,0.000000,0.000000}%
\pgfsetstrokecolor{textcolor}%
\pgfsetfillcolor{textcolor}%
\pgftext[x=3.023878in,y=0.417777in,,top]{\color{textcolor}\rmfamily\fontsize{10.000000}{12.000000}\selectfont \(\displaystyle {10.0}\)}%
\end{pgfscope}%
\begin{pgfscope}%
\pgfsetbuttcap%
\pgfsetroundjoin%
\definecolor{currentfill}{rgb}{0.000000,0.000000,0.000000}%
\pgfsetfillcolor{currentfill}%
\pgfsetlinewidth{0.803000pt}%
\definecolor{currentstroke}{rgb}{0.000000,0.000000,0.000000}%
\pgfsetstrokecolor{currentstroke}%
\pgfsetdash{}{0pt}%
\pgfsys@defobject{currentmarker}{\pgfqpoint{0.000000in}{-0.048611in}}{\pgfqpoint{0.000000in}{0.000000in}}{%
\pgfpathmoveto{\pgfqpoint{0.000000in}{0.000000in}}%
\pgfpathlineto{\pgfqpoint{0.000000in}{-0.048611in}}%
\pgfusepath{stroke,fill}%
}%
\begin{pgfscope}%
\pgfsys@transformshift{3.581192in}{0.515000in}%
\pgfsys@useobject{currentmarker}{}%
\end{pgfscope}%
\end{pgfscope}%
\begin{pgfscope}%
\definecolor{textcolor}{rgb}{0.000000,0.000000,0.000000}%
\pgfsetstrokecolor{textcolor}%
\pgfsetfillcolor{textcolor}%
\pgftext[x=3.581192in,y=0.417777in,,top]{\color{textcolor}\rmfamily\fontsize{10.000000}{12.000000}\selectfont \(\displaystyle {12.5}\)}%
\end{pgfscope}%
\begin{pgfscope}%
\pgfsetbuttcap%
\pgfsetroundjoin%
\definecolor{currentfill}{rgb}{0.000000,0.000000,0.000000}%
\pgfsetfillcolor{currentfill}%
\pgfsetlinewidth{0.803000pt}%
\definecolor{currentstroke}{rgb}{0.000000,0.000000,0.000000}%
\pgfsetstrokecolor{currentstroke}%
\pgfsetdash{}{0pt}%
\pgfsys@defobject{currentmarker}{\pgfqpoint{0.000000in}{-0.048611in}}{\pgfqpoint{0.000000in}{0.000000in}}{%
\pgfpathmoveto{\pgfqpoint{0.000000in}{0.000000in}}%
\pgfpathlineto{\pgfqpoint{0.000000in}{-0.048611in}}%
\pgfusepath{stroke,fill}%
}%
\begin{pgfscope}%
\pgfsys@transformshift{4.138507in}{0.515000in}%
\pgfsys@useobject{currentmarker}{}%
\end{pgfscope}%
\end{pgfscope}%
\begin{pgfscope}%
\definecolor{textcolor}{rgb}{0.000000,0.000000,0.000000}%
\pgfsetstrokecolor{textcolor}%
\pgfsetfillcolor{textcolor}%
\pgftext[x=4.138507in,y=0.417777in,,top]{\color{textcolor}\rmfamily\fontsize{10.000000}{12.000000}\selectfont \(\displaystyle {15.0}\)}%
\end{pgfscope}%
\begin{pgfscope}%
\pgfsetbuttcap%
\pgfsetroundjoin%
\definecolor{currentfill}{rgb}{0.000000,0.000000,0.000000}%
\pgfsetfillcolor{currentfill}%
\pgfsetlinewidth{0.803000pt}%
\definecolor{currentstroke}{rgb}{0.000000,0.000000,0.000000}%
\pgfsetstrokecolor{currentstroke}%
\pgfsetdash{}{0pt}%
\pgfsys@defobject{currentmarker}{\pgfqpoint{0.000000in}{-0.048611in}}{\pgfqpoint{0.000000in}{0.000000in}}{%
\pgfpathmoveto{\pgfqpoint{0.000000in}{0.000000in}}%
\pgfpathlineto{\pgfqpoint{0.000000in}{-0.048611in}}%
\pgfusepath{stroke,fill}%
}%
\begin{pgfscope}%
\pgfsys@transformshift{4.695821in}{0.515000in}%
\pgfsys@useobject{currentmarker}{}%
\end{pgfscope}%
\end{pgfscope}%
\begin{pgfscope}%
\definecolor{textcolor}{rgb}{0.000000,0.000000,0.000000}%
\pgfsetstrokecolor{textcolor}%
\pgfsetfillcolor{textcolor}%
\pgftext[x=4.695821in,y=0.417777in,,top]{\color{textcolor}\rmfamily\fontsize{10.000000}{12.000000}\selectfont \(\displaystyle {17.5}\)}%
\end{pgfscope}%
\begin{pgfscope}%
\pgfsetbuttcap%
\pgfsetroundjoin%
\definecolor{currentfill}{rgb}{0.000000,0.000000,0.000000}%
\pgfsetfillcolor{currentfill}%
\pgfsetlinewidth{0.803000pt}%
\definecolor{currentstroke}{rgb}{0.000000,0.000000,0.000000}%
\pgfsetstrokecolor{currentstroke}%
\pgfsetdash{}{0pt}%
\pgfsys@defobject{currentmarker}{\pgfqpoint{0.000000in}{-0.048611in}}{\pgfqpoint{0.000000in}{0.000000in}}{%
\pgfpathmoveto{\pgfqpoint{0.000000in}{0.000000in}}%
\pgfpathlineto{\pgfqpoint{0.000000in}{-0.048611in}}%
\pgfusepath{stroke,fill}%
}%
\begin{pgfscope}%
\pgfsys@transformshift{5.253135in}{0.515000in}%
\pgfsys@useobject{currentmarker}{}%
\end{pgfscope}%
\end{pgfscope}%
\begin{pgfscope}%
\definecolor{textcolor}{rgb}{0.000000,0.000000,0.000000}%
\pgfsetstrokecolor{textcolor}%
\pgfsetfillcolor{textcolor}%
\pgftext[x=5.253135in,y=0.417777in,,top]{\color{textcolor}\rmfamily\fontsize{10.000000}{12.000000}\selectfont \(\displaystyle {20.0}\)}%
\end{pgfscope}%
\begin{pgfscope}%
\definecolor{textcolor}{rgb}{0.000000,0.000000,0.000000}%
\pgfsetstrokecolor{textcolor}%
\pgfsetfillcolor{textcolor}%
\pgftext[x=3.049167in,y=0.238889in,,top]{\color{textcolor}\rmfamily\fontsize{10.000000}{12.000000}\selectfont Position X [\(\displaystyle m\)]}%
\end{pgfscope}%
\begin{pgfscope}%
\pgfsetbuttcap%
\pgfsetroundjoin%
\definecolor{currentfill}{rgb}{0.000000,0.000000,0.000000}%
\pgfsetfillcolor{currentfill}%
\pgfsetlinewidth{0.803000pt}%
\definecolor{currentstroke}{rgb}{0.000000,0.000000,0.000000}%
\pgfsetstrokecolor{currentstroke}%
\pgfsetdash{}{0pt}%
\pgfsys@defobject{currentmarker}{\pgfqpoint{-0.048611in}{0.000000in}}{\pgfqpoint{-0.000000in}{0.000000in}}{%
\pgfpathmoveto{\pgfqpoint{-0.000000in}{0.000000in}}%
\pgfpathlineto{\pgfqpoint{-0.048611in}{0.000000in}}%
\pgfusepath{stroke,fill}%
}%
\begin{pgfscope}%
\pgfsys@transformshift{0.569167in}{0.610964in}%
\pgfsys@useobject{currentmarker}{}%
\end{pgfscope}%
\end{pgfscope}%
\begin{pgfscope}%
\definecolor{textcolor}{rgb}{0.000000,0.000000,0.000000}%
\pgfsetstrokecolor{textcolor}%
\pgfsetfillcolor{textcolor}%
\pgftext[x=0.294444in, y=0.562770in, left, base]{\color{textcolor}\rmfamily\fontsize{10.000000}{12.000000}\selectfont \(\displaystyle {-8}\)}%
\end{pgfscope}%
\begin{pgfscope}%
\pgfsetbuttcap%
\pgfsetroundjoin%
\definecolor{currentfill}{rgb}{0.000000,0.000000,0.000000}%
\pgfsetfillcolor{currentfill}%
\pgfsetlinewidth{0.803000pt}%
\definecolor{currentstroke}{rgb}{0.000000,0.000000,0.000000}%
\pgfsetstrokecolor{currentstroke}%
\pgfsetdash{}{0pt}%
\pgfsys@defobject{currentmarker}{\pgfqpoint{-0.048611in}{0.000000in}}{\pgfqpoint{-0.000000in}{0.000000in}}{%
\pgfpathmoveto{\pgfqpoint{-0.000000in}{0.000000in}}%
\pgfpathlineto{\pgfqpoint{-0.048611in}{0.000000in}}%
\pgfusepath{stroke,fill}%
}%
\begin{pgfscope}%
\pgfsys@transformshift{0.569167in}{1.056815in}%
\pgfsys@useobject{currentmarker}{}%
\end{pgfscope}%
\end{pgfscope}%
\begin{pgfscope}%
\definecolor{textcolor}{rgb}{0.000000,0.000000,0.000000}%
\pgfsetstrokecolor{textcolor}%
\pgfsetfillcolor{textcolor}%
\pgftext[x=0.294444in, y=1.008621in, left, base]{\color{textcolor}\rmfamily\fontsize{10.000000}{12.000000}\selectfont \(\displaystyle {-6}\)}%
\end{pgfscope}%
\begin{pgfscope}%
\pgfsetbuttcap%
\pgfsetroundjoin%
\definecolor{currentfill}{rgb}{0.000000,0.000000,0.000000}%
\pgfsetfillcolor{currentfill}%
\pgfsetlinewidth{0.803000pt}%
\definecolor{currentstroke}{rgb}{0.000000,0.000000,0.000000}%
\pgfsetstrokecolor{currentstroke}%
\pgfsetdash{}{0pt}%
\pgfsys@defobject{currentmarker}{\pgfqpoint{-0.048611in}{0.000000in}}{\pgfqpoint{-0.000000in}{0.000000in}}{%
\pgfpathmoveto{\pgfqpoint{-0.000000in}{0.000000in}}%
\pgfpathlineto{\pgfqpoint{-0.048611in}{0.000000in}}%
\pgfusepath{stroke,fill}%
}%
\begin{pgfscope}%
\pgfsys@transformshift{0.569167in}{1.502667in}%
\pgfsys@useobject{currentmarker}{}%
\end{pgfscope}%
\end{pgfscope}%
\begin{pgfscope}%
\definecolor{textcolor}{rgb}{0.000000,0.000000,0.000000}%
\pgfsetstrokecolor{textcolor}%
\pgfsetfillcolor{textcolor}%
\pgftext[x=0.294444in, y=1.454472in, left, base]{\color{textcolor}\rmfamily\fontsize{10.000000}{12.000000}\selectfont \(\displaystyle {-4}\)}%
\end{pgfscope}%
\begin{pgfscope}%
\pgfsetbuttcap%
\pgfsetroundjoin%
\definecolor{currentfill}{rgb}{0.000000,0.000000,0.000000}%
\pgfsetfillcolor{currentfill}%
\pgfsetlinewidth{0.803000pt}%
\definecolor{currentstroke}{rgb}{0.000000,0.000000,0.000000}%
\pgfsetstrokecolor{currentstroke}%
\pgfsetdash{}{0pt}%
\pgfsys@defobject{currentmarker}{\pgfqpoint{-0.048611in}{0.000000in}}{\pgfqpoint{-0.000000in}{0.000000in}}{%
\pgfpathmoveto{\pgfqpoint{-0.000000in}{0.000000in}}%
\pgfpathlineto{\pgfqpoint{-0.048611in}{0.000000in}}%
\pgfusepath{stroke,fill}%
}%
\begin{pgfscope}%
\pgfsys@transformshift{0.569167in}{1.948518in}%
\pgfsys@useobject{currentmarker}{}%
\end{pgfscope}%
\end{pgfscope}%
\begin{pgfscope}%
\definecolor{textcolor}{rgb}{0.000000,0.000000,0.000000}%
\pgfsetstrokecolor{textcolor}%
\pgfsetfillcolor{textcolor}%
\pgftext[x=0.294444in, y=1.900324in, left, base]{\color{textcolor}\rmfamily\fontsize{10.000000}{12.000000}\selectfont \(\displaystyle {-2}\)}%
\end{pgfscope}%
\begin{pgfscope}%
\pgfsetbuttcap%
\pgfsetroundjoin%
\definecolor{currentfill}{rgb}{0.000000,0.000000,0.000000}%
\pgfsetfillcolor{currentfill}%
\pgfsetlinewidth{0.803000pt}%
\definecolor{currentstroke}{rgb}{0.000000,0.000000,0.000000}%
\pgfsetstrokecolor{currentstroke}%
\pgfsetdash{}{0pt}%
\pgfsys@defobject{currentmarker}{\pgfqpoint{-0.048611in}{0.000000in}}{\pgfqpoint{-0.000000in}{0.000000in}}{%
\pgfpathmoveto{\pgfqpoint{-0.000000in}{0.000000in}}%
\pgfpathlineto{\pgfqpoint{-0.048611in}{0.000000in}}%
\pgfusepath{stroke,fill}%
}%
\begin{pgfscope}%
\pgfsys@transformshift{0.569167in}{2.394370in}%
\pgfsys@useobject{currentmarker}{}%
\end{pgfscope}%
\end{pgfscope}%
\begin{pgfscope}%
\definecolor{textcolor}{rgb}{0.000000,0.000000,0.000000}%
\pgfsetstrokecolor{textcolor}%
\pgfsetfillcolor{textcolor}%
\pgftext[x=0.402500in, y=2.346175in, left, base]{\color{textcolor}\rmfamily\fontsize{10.000000}{12.000000}\selectfont \(\displaystyle {0}\)}%
\end{pgfscope}%
\begin{pgfscope}%
\pgfsetbuttcap%
\pgfsetroundjoin%
\definecolor{currentfill}{rgb}{0.000000,0.000000,0.000000}%
\pgfsetfillcolor{currentfill}%
\pgfsetlinewidth{0.803000pt}%
\definecolor{currentstroke}{rgb}{0.000000,0.000000,0.000000}%
\pgfsetstrokecolor{currentstroke}%
\pgfsetdash{}{0pt}%
\pgfsys@defobject{currentmarker}{\pgfqpoint{-0.048611in}{0.000000in}}{\pgfqpoint{-0.000000in}{0.000000in}}{%
\pgfpathmoveto{\pgfqpoint{-0.000000in}{0.000000in}}%
\pgfpathlineto{\pgfqpoint{-0.048611in}{0.000000in}}%
\pgfusepath{stroke,fill}%
}%
\begin{pgfscope}%
\pgfsys@transformshift{0.569167in}{2.840221in}%
\pgfsys@useobject{currentmarker}{}%
\end{pgfscope}%
\end{pgfscope}%
\begin{pgfscope}%
\definecolor{textcolor}{rgb}{0.000000,0.000000,0.000000}%
\pgfsetstrokecolor{textcolor}%
\pgfsetfillcolor{textcolor}%
\pgftext[x=0.402500in, y=2.792027in, left, base]{\color{textcolor}\rmfamily\fontsize{10.000000}{12.000000}\selectfont \(\displaystyle {2}\)}%
\end{pgfscope}%
\begin{pgfscope}%
\pgfsetbuttcap%
\pgfsetroundjoin%
\definecolor{currentfill}{rgb}{0.000000,0.000000,0.000000}%
\pgfsetfillcolor{currentfill}%
\pgfsetlinewidth{0.803000pt}%
\definecolor{currentstroke}{rgb}{0.000000,0.000000,0.000000}%
\pgfsetstrokecolor{currentstroke}%
\pgfsetdash{}{0pt}%
\pgfsys@defobject{currentmarker}{\pgfqpoint{-0.048611in}{0.000000in}}{\pgfqpoint{-0.000000in}{0.000000in}}{%
\pgfpathmoveto{\pgfqpoint{-0.000000in}{0.000000in}}%
\pgfpathlineto{\pgfqpoint{-0.048611in}{0.000000in}}%
\pgfusepath{stroke,fill}%
}%
\begin{pgfscope}%
\pgfsys@transformshift{0.569167in}{3.286072in}%
\pgfsys@useobject{currentmarker}{}%
\end{pgfscope}%
\end{pgfscope}%
\begin{pgfscope}%
\definecolor{textcolor}{rgb}{0.000000,0.000000,0.000000}%
\pgfsetstrokecolor{textcolor}%
\pgfsetfillcolor{textcolor}%
\pgftext[x=0.402500in, y=3.237878in, left, base]{\color{textcolor}\rmfamily\fontsize{10.000000}{12.000000}\selectfont \(\displaystyle {4}\)}%
\end{pgfscope}%
\begin{pgfscope}%
\pgfsetbuttcap%
\pgfsetroundjoin%
\definecolor{currentfill}{rgb}{0.000000,0.000000,0.000000}%
\pgfsetfillcolor{currentfill}%
\pgfsetlinewidth{0.803000pt}%
\definecolor{currentstroke}{rgb}{0.000000,0.000000,0.000000}%
\pgfsetstrokecolor{currentstroke}%
\pgfsetdash{}{0pt}%
\pgfsys@defobject{currentmarker}{\pgfqpoint{-0.048611in}{0.000000in}}{\pgfqpoint{-0.000000in}{0.000000in}}{%
\pgfpathmoveto{\pgfqpoint{-0.000000in}{0.000000in}}%
\pgfpathlineto{\pgfqpoint{-0.048611in}{0.000000in}}%
\pgfusepath{stroke,fill}%
}%
\begin{pgfscope}%
\pgfsys@transformshift{0.569167in}{3.731924in}%
\pgfsys@useobject{currentmarker}{}%
\end{pgfscope}%
\end{pgfscope}%
\begin{pgfscope}%
\definecolor{textcolor}{rgb}{0.000000,0.000000,0.000000}%
\pgfsetstrokecolor{textcolor}%
\pgfsetfillcolor{textcolor}%
\pgftext[x=0.402500in, y=3.683729in, left, base]{\color{textcolor}\rmfamily\fontsize{10.000000}{12.000000}\selectfont \(\displaystyle {6}\)}%
\end{pgfscope}%
\begin{pgfscope}%
\pgfsetbuttcap%
\pgfsetroundjoin%
\definecolor{currentfill}{rgb}{0.000000,0.000000,0.000000}%
\pgfsetfillcolor{currentfill}%
\pgfsetlinewidth{0.803000pt}%
\definecolor{currentstroke}{rgb}{0.000000,0.000000,0.000000}%
\pgfsetstrokecolor{currentstroke}%
\pgfsetdash{}{0pt}%
\pgfsys@defobject{currentmarker}{\pgfqpoint{-0.048611in}{0.000000in}}{\pgfqpoint{-0.000000in}{0.000000in}}{%
\pgfpathmoveto{\pgfqpoint{-0.000000in}{0.000000in}}%
\pgfpathlineto{\pgfqpoint{-0.048611in}{0.000000in}}%
\pgfusepath{stroke,fill}%
}%
\begin{pgfscope}%
\pgfsys@transformshift{0.569167in}{4.177775in}%
\pgfsys@useobject{currentmarker}{}%
\end{pgfscope}%
\end{pgfscope}%
\begin{pgfscope}%
\definecolor{textcolor}{rgb}{0.000000,0.000000,0.000000}%
\pgfsetstrokecolor{textcolor}%
\pgfsetfillcolor{textcolor}%
\pgftext[x=0.402500in, y=4.129581in, left, base]{\color{textcolor}\rmfamily\fontsize{10.000000}{12.000000}\selectfont \(\displaystyle {8}\)}%
\end{pgfscope}%
\begin{pgfscope}%
\definecolor{textcolor}{rgb}{0.000000,0.000000,0.000000}%
\pgfsetstrokecolor{textcolor}%
\pgfsetfillcolor{textcolor}%
\pgftext[x=0.238889in,y=2.363000in,,bottom,rotate=90.000000]{\color{textcolor}\rmfamily\fontsize{10.000000}{12.000000}\selectfont Position Y [\(\displaystyle m\)]}%
\end{pgfscope}%
\begin{pgfscope}%
\pgfpathrectangle{\pgfqpoint{0.569167in}{0.515000in}}{\pgfqpoint{4.960000in}{3.696000in}}%
\pgfusepath{clip}%
\pgfsetrectcap%
\pgfsetroundjoin%
\pgfsetlinewidth{1.505625pt}%
\definecolor{currentstroke}{rgb}{0.121569,0.466667,0.705882}%
\pgfsetstrokecolor{currentstroke}%
\pgfsetdash{}{0pt}%
\pgfpathmoveto{\pgfqpoint{0.794621in}{2.394370in}}%
\pgfpathlineto{\pgfqpoint{4.361432in}{2.394370in}}%
\pgfusepath{stroke}%
\end{pgfscope}%
\begin{pgfscope}%
\pgfsetrectcap%
\pgfsetmiterjoin%
\pgfsetlinewidth{0.803000pt}%
\definecolor{currentstroke}{rgb}{0.000000,0.000000,0.000000}%
\pgfsetstrokecolor{currentstroke}%
\pgfsetdash{}{0pt}%
\pgfpathmoveto{\pgfqpoint{0.569167in}{0.515000in}}%
\pgfpathlineto{\pgfqpoint{0.569167in}{4.211000in}}%
\pgfusepath{stroke}%
\end{pgfscope}%
\begin{pgfscope}%
\pgfsetrectcap%
\pgfsetmiterjoin%
\pgfsetlinewidth{0.803000pt}%
\definecolor{currentstroke}{rgb}{0.000000,0.000000,0.000000}%
\pgfsetstrokecolor{currentstroke}%
\pgfsetdash{}{0pt}%
\pgfpathmoveto{\pgfqpoint{5.529167in}{0.515000in}}%
\pgfpathlineto{\pgfqpoint{5.529167in}{4.211000in}}%
\pgfusepath{stroke}%
\end{pgfscope}%
\begin{pgfscope}%
\pgfsetrectcap%
\pgfsetmiterjoin%
\pgfsetlinewidth{0.803000pt}%
\definecolor{currentstroke}{rgb}{0.000000,0.000000,0.000000}%
\pgfsetstrokecolor{currentstroke}%
\pgfsetdash{}{0pt}%
\pgfpathmoveto{\pgfqpoint{0.569167in}{0.515000in}}%
\pgfpathlineto{\pgfqpoint{5.529167in}{0.515000in}}%
\pgfusepath{stroke}%
\end{pgfscope}%
\begin{pgfscope}%
\pgfsetrectcap%
\pgfsetmiterjoin%
\pgfsetlinewidth{0.803000pt}%
\definecolor{currentstroke}{rgb}{0.000000,0.000000,0.000000}%
\pgfsetstrokecolor{currentstroke}%
\pgfsetdash{}{0pt}%
\pgfpathmoveto{\pgfqpoint{0.569167in}{4.211000in}}%
\pgfpathlineto{\pgfqpoint{5.529167in}{4.211000in}}%
\pgfusepath{stroke}%
\end{pgfscope}%
\begin{pgfscope}%
\pgfsetbuttcap%
\pgfsetmiterjoin%
\definecolor{currentfill}{rgb}{1.000000,1.000000,1.000000}%
\pgfsetfillcolor{currentfill}%
\pgfsetfillopacity{0.800000}%
\pgfsetlinewidth{1.003750pt}%
\definecolor{currentstroke}{rgb}{0.800000,0.800000,0.800000}%
\pgfsetstrokecolor{currentstroke}%
\pgfsetstrokeopacity{0.800000}%
\pgfsetdash{}{0pt}%
\pgfpathmoveto{\pgfqpoint{3.838055in}{3.712667in}}%
\pgfpathlineto{\pgfqpoint{5.431944in}{3.712667in}}%
\pgfpathquadraticcurveto{\pgfqpoint{5.459722in}{3.712667in}}{\pgfqpoint{5.459722in}{3.740444in}}%
\pgfpathlineto{\pgfqpoint{5.459722in}{4.113777in}}%
\pgfpathquadraticcurveto{\pgfqpoint{5.459722in}{4.141555in}}{\pgfqpoint{5.431944in}{4.141555in}}%
\pgfpathlineto{\pgfqpoint{3.838055in}{4.141555in}}%
\pgfpathquadraticcurveto{\pgfqpoint{3.810278in}{4.141555in}}{\pgfqpoint{3.810278in}{4.113777in}}%
\pgfpathlineto{\pgfqpoint{3.810278in}{3.740444in}}%
\pgfpathquadraticcurveto{\pgfqpoint{3.810278in}{3.712667in}}{\pgfqpoint{3.838055in}{3.712667in}}%
\pgfpathclose%
\pgfusepath{stroke,fill}%
\end{pgfscope}%
\begin{pgfscope}%
\pgfsetrectcap%
\pgfsetroundjoin%
\pgfsetlinewidth{1.505625pt}%
\definecolor{currentstroke}{rgb}{0.121569,0.466667,0.705882}%
\pgfsetstrokecolor{currentstroke}%
\pgfsetdash{}{0pt}%
\pgfpathmoveto{\pgfqpoint{3.865833in}{4.037388in}}%
\pgfpathlineto{\pgfqpoint{4.143611in}{4.037388in}}%
\pgfusepath{stroke}%
\end{pgfscope}%
\begin{pgfscope}%
\definecolor{textcolor}{rgb}{0.000000,0.000000,0.000000}%
\pgfsetstrokecolor{textcolor}%
\pgfsetfillcolor{textcolor}%
\pgftext[x=4.254722in,y=3.988777in,left,base]{\color{textcolor}\rmfamily\fontsize{10.000000}{12.000000}\selectfont Ground truth}%
\end{pgfscope}%
\begin{pgfscope}%
\pgfsetbuttcap%
\pgfsetroundjoin%
\definecolor{currentfill}{rgb}{0.121569,0.466667,0.705882}%
\pgfsetfillcolor{currentfill}%
\pgfsetlinewidth{1.003750pt}%
\definecolor{currentstroke}{rgb}{0.121569,0.466667,0.705882}%
\pgfsetstrokecolor{currentstroke}%
\pgfsetdash{}{0pt}%
\pgfsys@defobject{currentmarker}{\pgfqpoint{-0.041667in}{-0.041667in}}{\pgfqpoint{0.041667in}{0.041667in}}{%
\pgfpathmoveto{\pgfqpoint{0.000000in}{-0.041667in}}%
\pgfpathcurveto{\pgfqpoint{0.011050in}{-0.041667in}}{\pgfqpoint{0.021649in}{-0.037276in}}{\pgfqpoint{0.029463in}{-0.029463in}}%
\pgfpathcurveto{\pgfqpoint{0.037276in}{-0.021649in}}{\pgfqpoint{0.041667in}{-0.011050in}}{\pgfqpoint{0.041667in}{0.000000in}}%
\pgfpathcurveto{\pgfqpoint{0.041667in}{0.011050in}}{\pgfqpoint{0.037276in}{0.021649in}}{\pgfqpoint{0.029463in}{0.029463in}}%
\pgfpathcurveto{\pgfqpoint{0.021649in}{0.037276in}}{\pgfqpoint{0.011050in}{0.041667in}}{\pgfqpoint{0.000000in}{0.041667in}}%
\pgfpathcurveto{\pgfqpoint{-0.011050in}{0.041667in}}{\pgfqpoint{-0.021649in}{0.037276in}}{\pgfqpoint{-0.029463in}{0.029463in}}%
\pgfpathcurveto{\pgfqpoint{-0.037276in}{0.021649in}}{\pgfqpoint{-0.041667in}{0.011050in}}{\pgfqpoint{-0.041667in}{0.000000in}}%
\pgfpathcurveto{\pgfqpoint{-0.041667in}{-0.011050in}}{\pgfqpoint{-0.037276in}{-0.021649in}}{\pgfqpoint{-0.029463in}{-0.029463in}}%
\pgfpathcurveto{\pgfqpoint{-0.021649in}{-0.037276in}}{\pgfqpoint{-0.011050in}{-0.041667in}}{\pgfqpoint{0.000000in}{-0.041667in}}%
\pgfpathclose%
\pgfusepath{stroke,fill}%
}%
\begin{pgfscope}%
\pgfsys@transformshift{4.004722in}{3.831625in}%
\pgfsys@useobject{currentmarker}{}%
\end{pgfscope}%
\end{pgfscope}%
\begin{pgfscope}%
\definecolor{textcolor}{rgb}{0.000000,0.000000,0.000000}%
\pgfsetstrokecolor{textcolor}%
\pgfsetfillcolor{textcolor}%
\pgftext[x=4.254722in,y=3.795166in,left,base]{\color{textcolor}\rmfamily\fontsize{10.000000}{12.000000}\selectfont Estimated position}%
\end{pgfscope}%
\end{pgfpicture}%
\makeatother%
\endgroup%
}
%         \caption{ FAMC's 3D position estimation had the lowest displacement error for the 16-meter line experiment. }
%         \label{fig:line16_2D}
%     \end{subfigure}
%     \begin{subfigure}{0.49\textwidth}
%         \centering
%         \resizebox{1\linewidth}{!}{%% Creator: Matplotlib, PGF backend
%%
%% To include the figure in your LaTeX document, write
%%   \input{<filename>.pgf}
%%
%% Make sure the required packages are loaded in your preamble
%%   \usepackage{pgf}
%%
%% and, on pdftex
%%   \usepackage[utf8]{inputenc}\DeclareUnicodeCharacter{2212}{-}
%%
%% or, on luatex and xetex
%%   \usepackage{unicode-math}
%%
%% Figures using additional raster images can only be included by \input if
%% they are in the same directory as the main LaTeX file. For loading figures
%% from other directories you can use the `import` package
%%   \usepackage{import}
%%
%% and then include the figures with
%%   \import{<path to file>}{<filename>.pgf}
%%
%% Matplotlib used the following preamble
%%   \usepackage{fontspec}
%%
\begingroup%
\makeatletter%
\begin{pgfpicture}%
\pgfpathrectangle{\pgfpointorigin}{\pgfqpoint{4.342355in}{4.008622in}}%
\pgfusepath{use as bounding box, clip}%
\begin{pgfscope}%
\pgfsetbuttcap%
\pgfsetmiterjoin%
\definecolor{currentfill}{rgb}{1.000000,1.000000,1.000000}%
\pgfsetfillcolor{currentfill}%
\pgfsetlinewidth{0.000000pt}%
\definecolor{currentstroke}{rgb}{1.000000,1.000000,1.000000}%
\pgfsetstrokecolor{currentstroke}%
\pgfsetdash{}{0pt}%
\pgfpathmoveto{\pgfqpoint{0.000000in}{0.000000in}}%
\pgfpathlineto{\pgfqpoint{4.342355in}{0.000000in}}%
\pgfpathlineto{\pgfqpoint{4.342355in}{4.008622in}}%
\pgfpathlineto{\pgfqpoint{0.000000in}{4.008622in}}%
\pgfpathclose%
\pgfusepath{fill}%
\end{pgfscope}%
\begin{pgfscope}%
\pgfsetbuttcap%
\pgfsetmiterjoin%
\definecolor{currentfill}{rgb}{1.000000,1.000000,1.000000}%
\pgfsetfillcolor{currentfill}%
\pgfsetlinewidth{0.000000pt}%
\definecolor{currentstroke}{rgb}{0.000000,0.000000,0.000000}%
\pgfsetstrokecolor{currentstroke}%
\pgfsetstrokeopacity{0.000000}%
\pgfsetdash{}{0pt}%
\pgfpathmoveto{\pgfqpoint{0.100000in}{0.212622in}}%
\pgfpathlineto{\pgfqpoint{3.796000in}{0.212622in}}%
\pgfpathlineto{\pgfqpoint{3.796000in}{3.908622in}}%
\pgfpathlineto{\pgfqpoint{0.100000in}{3.908622in}}%
\pgfpathclose%
\pgfusepath{fill}%
\end{pgfscope}%
\begin{pgfscope}%
\pgfsetbuttcap%
\pgfsetmiterjoin%
\definecolor{currentfill}{rgb}{0.950000,0.950000,0.950000}%
\pgfsetfillcolor{currentfill}%
\pgfsetfillopacity{0.500000}%
\pgfsetlinewidth{1.003750pt}%
\definecolor{currentstroke}{rgb}{0.950000,0.950000,0.950000}%
\pgfsetstrokecolor{currentstroke}%
\pgfsetstrokeopacity{0.500000}%
\pgfsetdash{}{0pt}%
\pgfpathmoveto{\pgfqpoint{0.379073in}{1.123938in}}%
\pgfpathlineto{\pgfqpoint{1.599613in}{2.147018in}}%
\pgfpathlineto{\pgfqpoint{1.582647in}{3.622484in}}%
\pgfpathlineto{\pgfqpoint{0.303698in}{2.689165in}}%
\pgfusepath{stroke,fill}%
\end{pgfscope}%
\begin{pgfscope}%
\pgfsetbuttcap%
\pgfsetmiterjoin%
\definecolor{currentfill}{rgb}{0.900000,0.900000,0.900000}%
\pgfsetfillcolor{currentfill}%
\pgfsetfillopacity{0.500000}%
\pgfsetlinewidth{1.003750pt}%
\definecolor{currentstroke}{rgb}{0.900000,0.900000,0.900000}%
\pgfsetstrokecolor{currentstroke}%
\pgfsetstrokeopacity{0.500000}%
\pgfsetdash{}{0pt}%
\pgfpathmoveto{\pgfqpoint{1.599613in}{2.147018in}}%
\pgfpathlineto{\pgfqpoint{3.558144in}{1.577751in}}%
\pgfpathlineto{\pgfqpoint{3.628038in}{3.104037in}}%
\pgfpathlineto{\pgfqpoint{1.582647in}{3.622484in}}%
\pgfusepath{stroke,fill}%
\end{pgfscope}%
\begin{pgfscope}%
\pgfsetbuttcap%
\pgfsetmiterjoin%
\definecolor{currentfill}{rgb}{0.925000,0.925000,0.925000}%
\pgfsetfillcolor{currentfill}%
\pgfsetfillopacity{0.500000}%
\pgfsetlinewidth{1.003750pt}%
\definecolor{currentstroke}{rgb}{0.925000,0.925000,0.925000}%
\pgfsetstrokecolor{currentstroke}%
\pgfsetstrokeopacity{0.500000}%
\pgfsetdash{}{0pt}%
\pgfpathmoveto{\pgfqpoint{0.379073in}{1.123938in}}%
\pgfpathlineto{\pgfqpoint{2.455212in}{0.445871in}}%
\pgfpathlineto{\pgfqpoint{3.558144in}{1.577751in}}%
\pgfpathlineto{\pgfqpoint{1.599613in}{2.147018in}}%
\pgfusepath{stroke,fill}%
\end{pgfscope}%
\begin{pgfscope}%
\pgfsetrectcap%
\pgfsetroundjoin%
\pgfsetlinewidth{0.803000pt}%
\definecolor{currentstroke}{rgb}{0.000000,0.000000,0.000000}%
\pgfsetstrokecolor{currentstroke}%
\pgfsetdash{}{0pt}%
\pgfpathmoveto{\pgfqpoint{0.379073in}{1.123938in}}%
\pgfpathlineto{\pgfqpoint{2.455212in}{0.445871in}}%
\pgfusepath{stroke}%
\end{pgfscope}%
\begin{pgfscope}%
\definecolor{textcolor}{rgb}{0.000000,0.000000,0.000000}%
\pgfsetstrokecolor{textcolor}%
\pgfsetfillcolor{textcolor}%
\pgftext[x=0.730374in, y=0.408886in, left, base,rotate=341.912962]{\color{textcolor}\rmfamily\fontsize{10.000000}{12.000000}\selectfont Position X [\(\displaystyle m\)]}%
\end{pgfscope}%
\begin{pgfscope}%
\pgfsetbuttcap%
\pgfsetroundjoin%
\pgfsetlinewidth{0.803000pt}%
\definecolor{currentstroke}{rgb}{0.690196,0.690196,0.690196}%
\pgfsetstrokecolor{currentstroke}%
\pgfsetdash{}{0pt}%
\pgfpathmoveto{\pgfqpoint{0.504815in}{1.082870in}}%
\pgfpathlineto{\pgfqpoint{1.718725in}{2.112397in}}%
\pgfpathlineto{\pgfqpoint{1.706795in}{3.591016in}}%
\pgfusepath{stroke}%
\end{pgfscope}%
\begin{pgfscope}%
\pgfsetbuttcap%
\pgfsetroundjoin%
\pgfsetlinewidth{0.803000pt}%
\definecolor{currentstroke}{rgb}{0.690196,0.690196,0.690196}%
\pgfsetstrokecolor{currentstroke}%
\pgfsetdash{}{0pt}%
\pgfpathmoveto{\pgfqpoint{0.937302in}{0.941620in}}%
\pgfpathlineto{\pgfqpoint{2.127921in}{1.993460in}}%
\pgfpathlineto{\pgfqpoint{2.133534in}{3.482850in}}%
\pgfusepath{stroke}%
\end{pgfscope}%
\begin{pgfscope}%
\pgfsetbuttcap%
\pgfsetroundjoin%
\pgfsetlinewidth{0.803000pt}%
\definecolor{currentstroke}{rgb}{0.690196,0.690196,0.690196}%
\pgfsetstrokecolor{currentstroke}%
\pgfsetdash{}{0pt}%
\pgfpathmoveto{\pgfqpoint{1.379624in}{0.797158in}}%
\pgfpathlineto{\pgfqpoint{2.545645in}{1.872044in}}%
\pgfpathlineto{\pgfqpoint{2.569556in}{3.372331in}}%
\pgfusepath{stroke}%
\end{pgfscope}%
\begin{pgfscope}%
\pgfsetbuttcap%
\pgfsetroundjoin%
\pgfsetlinewidth{0.803000pt}%
\definecolor{currentstroke}{rgb}{0.690196,0.690196,0.690196}%
\pgfsetstrokecolor{currentstroke}%
\pgfsetdash{}{0pt}%
\pgfpathmoveto{\pgfqpoint{1.832122in}{0.649372in}}%
\pgfpathlineto{\pgfqpoint{2.972166in}{1.748072in}}%
\pgfpathlineto{\pgfqpoint{3.015165in}{3.259382in}}%
\pgfusepath{stroke}%
\end{pgfscope}%
\begin{pgfscope}%
\pgfsetbuttcap%
\pgfsetroundjoin%
\pgfsetlinewidth{0.803000pt}%
\definecolor{currentstroke}{rgb}{0.690196,0.690196,0.690196}%
\pgfsetstrokecolor{currentstroke}%
\pgfsetdash{}{0pt}%
\pgfpathmoveto{\pgfqpoint{2.295152in}{0.498146in}}%
\pgfpathlineto{\pgfqpoint{3.407765in}{1.621460in}}%
\pgfpathlineto{\pgfqpoint{3.470683in}{3.143922in}}%
\pgfusepath{stroke}%
\end{pgfscope}%
\begin{pgfscope}%
\pgfsetrectcap%
\pgfsetroundjoin%
\pgfsetlinewidth{0.803000pt}%
\definecolor{currentstroke}{rgb}{0.000000,0.000000,0.000000}%
\pgfsetstrokecolor{currentstroke}%
\pgfsetdash{}{0pt}%
\pgfpathmoveto{\pgfqpoint{0.515386in}{1.091835in}}%
\pgfpathlineto{\pgfqpoint{0.483629in}{1.064902in}}%
\pgfusepath{stroke}%
\end{pgfscope}%
\begin{pgfscope}%
\definecolor{textcolor}{rgb}{0.000000,0.000000,0.000000}%
\pgfsetstrokecolor{textcolor}%
\pgfsetfillcolor{textcolor}%
\pgftext[x=0.400245in,y=0.864666in,,top]{\color{textcolor}\rmfamily\fontsize{10.000000}{12.000000}\selectfont \(\displaystyle {0}\)}%
\end{pgfscope}%
\begin{pgfscope}%
\pgfsetrectcap%
\pgfsetroundjoin%
\pgfsetlinewidth{0.803000pt}%
\definecolor{currentstroke}{rgb}{0.000000,0.000000,0.000000}%
\pgfsetstrokecolor{currentstroke}%
\pgfsetdash{}{0pt}%
\pgfpathmoveto{\pgfqpoint{0.947679in}{0.950788in}}%
\pgfpathlineto{\pgfqpoint{0.916502in}{0.923245in}}%
\pgfusepath{stroke}%
\end{pgfscope}%
\begin{pgfscope}%
\definecolor{textcolor}{rgb}{0.000000,0.000000,0.000000}%
\pgfsetstrokecolor{textcolor}%
\pgfsetfillcolor{textcolor}%
\pgftext[x=0.833177in,y=0.720422in,,top]{\color{textcolor}\rmfamily\fontsize{10.000000}{12.000000}\selectfont \(\displaystyle {5}\)}%
\end{pgfscope}%
\begin{pgfscope}%
\pgfsetrectcap%
\pgfsetroundjoin%
\pgfsetlinewidth{0.803000pt}%
\definecolor{currentstroke}{rgb}{0.000000,0.000000,0.000000}%
\pgfsetstrokecolor{currentstroke}%
\pgfsetdash{}{0pt}%
\pgfpathmoveto{\pgfqpoint{1.389796in}{0.806535in}}%
\pgfpathlineto{\pgfqpoint{1.359235in}{0.778362in}}%
\pgfusepath{stroke}%
\end{pgfscope}%
\begin{pgfscope}%
\definecolor{textcolor}{rgb}{0.000000,0.000000,0.000000}%
\pgfsetstrokecolor{textcolor}%
\pgfsetfillcolor{textcolor}%
\pgftext[x=1.275993in,y=0.572885in,,top]{\color{textcolor}\rmfamily\fontsize{10.000000}{12.000000}\selectfont \(\displaystyle {10}\)}%
\end{pgfscope}%
\begin{pgfscope}%
\pgfsetrectcap%
\pgfsetroundjoin%
\pgfsetlinewidth{0.803000pt}%
\definecolor{currentstroke}{rgb}{0.000000,0.000000,0.000000}%
\pgfsetstrokecolor{currentstroke}%
\pgfsetdash{}{0pt}%
\pgfpathmoveto{\pgfqpoint{1.842078in}{0.658966in}}%
\pgfpathlineto{\pgfqpoint{1.812168in}{0.630141in}}%
\pgfusepath{stroke}%
\end{pgfscope}%
\begin{pgfscope}%
\definecolor{textcolor}{rgb}{0.000000,0.000000,0.000000}%
\pgfsetstrokecolor{textcolor}%
\pgfsetfillcolor{textcolor}%
\pgftext[x=1.729035in,y=0.421941in,,top]{\color{textcolor}\rmfamily\fontsize{10.000000}{12.000000}\selectfont \(\displaystyle {15}\)}%
\end{pgfscope}%
\begin{pgfscope}%
\pgfsetrectcap%
\pgfsetroundjoin%
\pgfsetlinewidth{0.803000pt}%
\definecolor{currentstroke}{rgb}{0.000000,0.000000,0.000000}%
\pgfsetstrokecolor{currentstroke}%
\pgfsetdash{}{0pt}%
\pgfpathmoveto{\pgfqpoint{2.304877in}{0.507965in}}%
\pgfpathlineto{\pgfqpoint{2.275658in}{0.478465in}}%
\pgfusepath{stroke}%
\end{pgfscope}%
\begin{pgfscope}%
\definecolor{textcolor}{rgb}{0.000000,0.000000,0.000000}%
\pgfsetstrokecolor{textcolor}%
\pgfsetfillcolor{textcolor}%
\pgftext[x=2.192662in,y=0.267471in,,top]{\color{textcolor}\rmfamily\fontsize{10.000000}{12.000000}\selectfont \(\displaystyle {20}\)}%
\end{pgfscope}%
\begin{pgfscope}%
\pgfsetrectcap%
\pgfsetroundjoin%
\pgfsetlinewidth{0.803000pt}%
\definecolor{currentstroke}{rgb}{0.000000,0.000000,0.000000}%
\pgfsetstrokecolor{currentstroke}%
\pgfsetdash{}{0pt}%
\pgfpathmoveto{\pgfqpoint{3.558144in}{1.577751in}}%
\pgfpathlineto{\pgfqpoint{2.455212in}{0.445871in}}%
\pgfusepath{stroke}%
\end{pgfscope}%
\begin{pgfscope}%
\definecolor{textcolor}{rgb}{0.000000,0.000000,0.000000}%
\pgfsetstrokecolor{textcolor}%
\pgfsetfillcolor{textcolor}%
\pgftext[x=3.120747in, y=0.305657in, left, base,rotate=45.742112]{\color{textcolor}\rmfamily\fontsize{10.000000}{12.000000}\selectfont Position Y [\(\displaystyle m\)]}%
\end{pgfscope}%
\begin{pgfscope}%
\pgfsetbuttcap%
\pgfsetroundjoin%
\pgfsetlinewidth{0.803000pt}%
\definecolor{currentstroke}{rgb}{0.690196,0.690196,0.690196}%
\pgfsetstrokecolor{currentstroke}%
\pgfsetdash{}{0pt}%
\pgfpathmoveto{\pgfqpoint{0.526119in}{2.851478in}}%
\pgfpathlineto{\pgfqpoint{0.590672in}{1.301303in}}%
\pgfpathlineto{\pgfqpoint{2.647120in}{0.642816in}}%
\pgfusepath{stroke}%
\end{pgfscope}%
\begin{pgfscope}%
\pgfsetbuttcap%
\pgfsetroundjoin%
\pgfsetlinewidth{0.803000pt}%
\definecolor{currentstroke}{rgb}{0.690196,0.690196,0.690196}%
\pgfsetstrokecolor{currentstroke}%
\pgfsetdash{}{0pt}%
\pgfpathmoveto{\pgfqpoint{0.733340in}{3.002698in}}%
\pgfpathlineto{\pgfqpoint{0.788061in}{1.466759in}}%
\pgfpathlineto{\pgfqpoint{2.825876in}{0.826263in}}%
\pgfusepath{stroke}%
\end{pgfscope}%
\begin{pgfscope}%
\pgfsetbuttcap%
\pgfsetroundjoin%
\pgfsetlinewidth{0.803000pt}%
\definecolor{currentstroke}{rgb}{0.690196,0.690196,0.690196}%
\pgfsetstrokecolor{currentstroke}%
\pgfsetdash{}{0pt}%
\pgfpathmoveto{\pgfqpoint{0.934834in}{3.149740in}}%
\pgfpathlineto{\pgfqpoint{0.980228in}{1.627837in}}%
\pgfpathlineto{\pgfqpoint{2.999658in}{1.004606in}}%
\pgfusepath{stroke}%
\end{pgfscope}%
\begin{pgfscope}%
\pgfsetbuttcap%
\pgfsetroundjoin%
\pgfsetlinewidth{0.803000pt}%
\definecolor{currentstroke}{rgb}{0.690196,0.690196,0.690196}%
\pgfsetstrokecolor{currentstroke}%
\pgfsetdash{}{0pt}%
\pgfpathmoveto{\pgfqpoint{1.130837in}{3.292774in}}%
\pgfpathlineto{\pgfqpoint{1.167378in}{1.784710in}}%
\pgfpathlineto{\pgfqpoint{3.168671in}{1.178055in}}%
\pgfusepath{stroke}%
\end{pgfscope}%
\begin{pgfscope}%
\pgfsetbuttcap%
\pgfsetroundjoin%
\pgfsetlinewidth{0.803000pt}%
\definecolor{currentstroke}{rgb}{0.690196,0.690196,0.690196}%
\pgfsetstrokecolor{currentstroke}%
\pgfsetdash{}{0pt}%
\pgfpathmoveto{\pgfqpoint{1.321569in}{3.431961in}}%
\pgfpathlineto{\pgfqpoint{1.349705in}{1.937540in}}%
\pgfpathlineto{\pgfqpoint{3.333109in}{1.346809in}}%
\pgfusepath{stroke}%
\end{pgfscope}%
\begin{pgfscope}%
\pgfsetbuttcap%
\pgfsetroundjoin%
\pgfsetlinewidth{0.803000pt}%
\definecolor{currentstroke}{rgb}{0.690196,0.690196,0.690196}%
\pgfsetstrokecolor{currentstroke}%
\pgfsetdash{}{0pt}%
\pgfpathmoveto{\pgfqpoint{1.507240in}{3.567456in}}%
\pgfpathlineto{\pgfqpoint{1.527393in}{2.086482in}}%
\pgfpathlineto{\pgfqpoint{3.493154in}{1.511054in}}%
\pgfusepath{stroke}%
\end{pgfscope}%
\begin{pgfscope}%
\pgfsetrectcap%
\pgfsetroundjoin%
\pgfsetlinewidth{0.803000pt}%
\definecolor{currentstroke}{rgb}{0.000000,0.000000,0.000000}%
\pgfsetstrokecolor{currentstroke}%
\pgfsetdash{}{0pt}%
\pgfpathmoveto{\pgfqpoint{2.629798in}{0.648362in}}%
\pgfpathlineto{\pgfqpoint{2.681808in}{0.631708in}}%
\pgfusepath{stroke}%
\end{pgfscope}%
\begin{pgfscope}%
\definecolor{textcolor}{rgb}{0.000000,0.000000,0.000000}%
\pgfsetstrokecolor{textcolor}%
\pgfsetfillcolor{textcolor}%
\pgftext[x=2.824468in,y=0.458138in,,top]{\color{textcolor}\rmfamily\fontsize{10.000000}{12.000000}\selectfont \(\displaystyle {-0.25}\)}%
\end{pgfscope}%
\begin{pgfscope}%
\pgfsetrectcap%
\pgfsetroundjoin%
\pgfsetlinewidth{0.803000pt}%
\definecolor{currentstroke}{rgb}{0.000000,0.000000,0.000000}%
\pgfsetstrokecolor{currentstroke}%
\pgfsetdash{}{0pt}%
\pgfpathmoveto{\pgfqpoint{2.808724in}{0.831655in}}%
\pgfpathlineto{\pgfqpoint{2.860225in}{0.815467in}}%
\pgfusepath{stroke}%
\end{pgfscope}%
\begin{pgfscope}%
\definecolor{textcolor}{rgb}{0.000000,0.000000,0.000000}%
\pgfsetstrokecolor{textcolor}%
\pgfsetfillcolor{textcolor}%
\pgftext[x=3.000825in,y=0.644300in,,top]{\color{textcolor}\rmfamily\fontsize{10.000000}{12.000000}\selectfont \(\displaystyle {-0.20}\)}%
\end{pgfscope}%
\begin{pgfscope}%
\pgfsetrectcap%
\pgfsetroundjoin%
\pgfsetlinewidth{0.803000pt}%
\definecolor{currentstroke}{rgb}{0.000000,0.000000,0.000000}%
\pgfsetstrokecolor{currentstroke}%
\pgfsetdash{}{0pt}%
\pgfpathmoveto{\pgfqpoint{2.982672in}{1.009849in}}%
\pgfpathlineto{\pgfqpoint{3.033673in}{0.994109in}}%
\pgfusepath{stroke}%
\end{pgfscope}%
\begin{pgfscope}%
\definecolor{textcolor}{rgb}{0.000000,0.000000,0.000000}%
\pgfsetstrokecolor{textcolor}%
\pgfsetfillcolor{textcolor}%
\pgftext[x=3.172272in,y=0.825279in,,top]{\color{textcolor}\rmfamily\fontsize{10.000000}{12.000000}\selectfont \(\displaystyle {-0.15}\)}%
\end{pgfscope}%
\begin{pgfscope}%
\pgfsetrectcap%
\pgfsetroundjoin%
\pgfsetlinewidth{0.803000pt}%
\definecolor{currentstroke}{rgb}{0.000000,0.000000,0.000000}%
\pgfsetstrokecolor{currentstroke}%
\pgfsetdash{}{0pt}%
\pgfpathmoveto{\pgfqpoint{3.151849in}{1.183155in}}%
\pgfpathlineto{\pgfqpoint{3.202357in}{1.167844in}}%
\pgfusepath{stroke}%
\end{pgfscope}%
\begin{pgfscope}%
\definecolor{textcolor}{rgb}{0.000000,0.000000,0.000000}%
\pgfsetstrokecolor{textcolor}%
\pgfsetfillcolor{textcolor}%
\pgftext[x=3.339011in,y=1.001289in,,top]{\color{textcolor}\rmfamily\fontsize{10.000000}{12.000000}\selectfont \(\displaystyle {-0.10}\)}%
\end{pgfscope}%
\begin{pgfscope}%
\pgfsetrectcap%
\pgfsetroundjoin%
\pgfsetlinewidth{0.803000pt}%
\definecolor{currentstroke}{rgb}{0.000000,0.000000,0.000000}%
\pgfsetstrokecolor{currentstroke}%
\pgfsetdash{}{0pt}%
\pgfpathmoveto{\pgfqpoint{3.316447in}{1.351771in}}%
\pgfpathlineto{\pgfqpoint{3.366471in}{1.336872in}}%
\pgfusepath{stroke}%
\end{pgfscope}%
\begin{pgfscope}%
\definecolor{textcolor}{rgb}{0.000000,0.000000,0.000000}%
\pgfsetstrokecolor{textcolor}%
\pgfsetfillcolor{textcolor}%
\pgftext[x=3.501234in,y=1.172531in,,top]{\color{textcolor}\rmfamily\fontsize{10.000000}{12.000000}\selectfont \(\displaystyle {-0.05}\)}%
\end{pgfscope}%
\begin{pgfscope}%
\pgfsetrectcap%
\pgfsetroundjoin%
\pgfsetlinewidth{0.803000pt}%
\definecolor{currentstroke}{rgb}{0.000000,0.000000,0.000000}%
\pgfsetstrokecolor{currentstroke}%
\pgfsetdash{}{0pt}%
\pgfpathmoveto{\pgfqpoint{3.476651in}{1.515885in}}%
\pgfpathlineto{\pgfqpoint{3.526198in}{1.501381in}}%
\pgfusepath{stroke}%
\end{pgfscope}%
\begin{pgfscope}%
\definecolor{textcolor}{rgb}{0.000000,0.000000,0.000000}%
\pgfsetstrokecolor{textcolor}%
\pgfsetfillcolor{textcolor}%
\pgftext[x=3.659121in,y=1.339196in,,top]{\color{textcolor}\rmfamily\fontsize{10.000000}{12.000000}\selectfont \(\displaystyle {0.00}\)}%
\end{pgfscope}%
\begin{pgfscope}%
\pgfsetrectcap%
\pgfsetroundjoin%
\pgfsetlinewidth{0.803000pt}%
\definecolor{currentstroke}{rgb}{0.000000,0.000000,0.000000}%
\pgfsetstrokecolor{currentstroke}%
\pgfsetdash{}{0pt}%
\pgfpathmoveto{\pgfqpoint{3.558144in}{1.577751in}}%
\pgfpathlineto{\pgfqpoint{3.628038in}{3.104037in}}%
\pgfusepath{stroke}%
\end{pgfscope}%
\begin{pgfscope}%
\definecolor{textcolor}{rgb}{0.000000,0.000000,0.000000}%
\pgfsetstrokecolor{textcolor}%
\pgfsetfillcolor{textcolor}%
\pgftext[x=4.167903in, y=1.963517in, left, base,rotate=87.378092]{\color{textcolor}\rmfamily\fontsize{10.000000}{12.000000}\selectfont Position Z [\(\displaystyle m\)]}%
\end{pgfscope}%
\begin{pgfscope}%
\pgfsetbuttcap%
\pgfsetroundjoin%
\pgfsetlinewidth{0.803000pt}%
\definecolor{currentstroke}{rgb}{0.690196,0.690196,0.690196}%
\pgfsetstrokecolor{currentstroke}%
\pgfsetdash{}{0pt}%
\pgfpathmoveto{\pgfqpoint{3.562413in}{1.670971in}}%
\pgfpathlineto{\pgfqpoint{1.598575in}{2.237314in}}%
\pgfpathlineto{\pgfqpoint{0.374477in}{1.219385in}}%
\pgfusepath{stroke}%
\end{pgfscope}%
\begin{pgfscope}%
\pgfsetbuttcap%
\pgfsetroundjoin%
\pgfsetlinewidth{0.803000pt}%
\definecolor{currentstroke}{rgb}{0.690196,0.690196,0.690196}%
\pgfsetstrokecolor{currentstroke}%
\pgfsetdash{}{0pt}%
\pgfpathmoveto{\pgfqpoint{3.576199in}{1.972019in}}%
\pgfpathlineto{\pgfqpoint{1.595224in}{2.528756in}}%
\pgfpathlineto{\pgfqpoint{0.359627in}{1.527760in}}%
\pgfusepath{stroke}%
\end{pgfscope}%
\begin{pgfscope}%
\pgfsetbuttcap%
\pgfsetroundjoin%
\pgfsetlinewidth{0.803000pt}%
\definecolor{currentstroke}{rgb}{0.690196,0.690196,0.690196}%
\pgfsetstrokecolor{currentstroke}%
\pgfsetdash{}{0pt}%
\pgfpathmoveto{\pgfqpoint{3.590230in}{2.278419in}}%
\pgfpathlineto{\pgfqpoint{1.591816in}{2.825130in}}%
\pgfpathlineto{\pgfqpoint{0.344502in}{1.841827in}}%
\pgfusepath{stroke}%
\end{pgfscope}%
\begin{pgfscope}%
\pgfsetbuttcap%
\pgfsetroundjoin%
\pgfsetlinewidth{0.803000pt}%
\definecolor{currentstroke}{rgb}{0.690196,0.690196,0.690196}%
\pgfsetstrokecolor{currentstroke}%
\pgfsetdash{}{0pt}%
\pgfpathmoveto{\pgfqpoint{3.604513in}{2.590316in}}%
\pgfpathlineto{\pgfqpoint{1.588349in}{3.126563in}}%
\pgfpathlineto{\pgfqpoint{0.329096in}{2.161748in}}%
\pgfusepath{stroke}%
\end{pgfscope}%
\begin{pgfscope}%
\pgfsetbuttcap%
\pgfsetroundjoin%
\pgfsetlinewidth{0.803000pt}%
\definecolor{currentstroke}{rgb}{0.690196,0.690196,0.690196}%
\pgfsetstrokecolor{currentstroke}%
\pgfsetdash{}{0pt}%
\pgfpathmoveto{\pgfqpoint{3.619054in}{2.907859in}}%
\pgfpathlineto{\pgfqpoint{1.584823in}{3.433186in}}%
\pgfpathlineto{\pgfqpoint{0.313400in}{2.487686in}}%
\pgfusepath{stroke}%
\end{pgfscope}%
\begin{pgfscope}%
\pgfsetrectcap%
\pgfsetroundjoin%
\pgfsetlinewidth{0.803000pt}%
\definecolor{currentstroke}{rgb}{0.000000,0.000000,0.000000}%
\pgfsetstrokecolor{currentstroke}%
\pgfsetdash{}{0pt}%
\pgfpathmoveto{\pgfqpoint{3.545929in}{1.675725in}}%
\pgfpathlineto{\pgfqpoint{3.595421in}{1.661453in}}%
\pgfusepath{stroke}%
\end{pgfscope}%
\begin{pgfscope}%
\definecolor{textcolor}{rgb}{0.000000,0.000000,0.000000}%
\pgfsetstrokecolor{textcolor}%
\pgfsetfillcolor{textcolor}%
\pgftext[x=3.816545in,y=1.706970in,,top]{\color{textcolor}\rmfamily\fontsize{10.000000}{12.000000}\selectfont \(\displaystyle {0.00}\)}%
\end{pgfscope}%
\begin{pgfscope}%
\pgfsetrectcap%
\pgfsetroundjoin%
\pgfsetlinewidth{0.803000pt}%
\definecolor{currentstroke}{rgb}{0.000000,0.000000,0.000000}%
\pgfsetstrokecolor{currentstroke}%
\pgfsetdash{}{0pt}%
\pgfpathmoveto{\pgfqpoint{3.559564in}{1.976694in}}%
\pgfpathlineto{\pgfqpoint{3.609509in}{1.962657in}}%
\pgfusepath{stroke}%
\end{pgfscope}%
\begin{pgfscope}%
\definecolor{textcolor}{rgb}{0.000000,0.000000,0.000000}%
\pgfsetstrokecolor{textcolor}%
\pgfsetfillcolor{textcolor}%
\pgftext[x=3.832522in,y=2.007421in,,top]{\color{textcolor}\rmfamily\fontsize{10.000000}{12.000000}\selectfont \(\displaystyle {0.01}\)}%
\end{pgfscope}%
\begin{pgfscope}%
\pgfsetrectcap%
\pgfsetroundjoin%
\pgfsetlinewidth{0.803000pt}%
\definecolor{currentstroke}{rgb}{0.000000,0.000000,0.000000}%
\pgfsetstrokecolor{currentstroke}%
\pgfsetdash{}{0pt}%
\pgfpathmoveto{\pgfqpoint{3.573442in}{2.283012in}}%
\pgfpathlineto{\pgfqpoint{3.623848in}{2.269222in}}%
\pgfusepath{stroke}%
\end{pgfscope}%
\begin{pgfscope}%
\definecolor{textcolor}{rgb}{0.000000,0.000000,0.000000}%
\pgfsetstrokecolor{textcolor}%
\pgfsetfillcolor{textcolor}%
\pgftext[x=3.848782in,y=2.313197in,,top]{\color{textcolor}\rmfamily\fontsize{10.000000}{12.000000}\selectfont \(\displaystyle {0.02}\)}%
\end{pgfscope}%
\begin{pgfscope}%
\pgfsetrectcap%
\pgfsetroundjoin%
\pgfsetlinewidth{0.803000pt}%
\definecolor{currentstroke}{rgb}{0.000000,0.000000,0.000000}%
\pgfsetstrokecolor{currentstroke}%
\pgfsetdash{}{0pt}%
\pgfpathmoveto{\pgfqpoint{3.587568in}{2.594823in}}%
\pgfpathlineto{\pgfqpoint{3.638444in}{2.581291in}}%
\pgfusepath{stroke}%
\end{pgfscope}%
\begin{pgfscope}%
\definecolor{textcolor}{rgb}{0.000000,0.000000,0.000000}%
\pgfsetstrokecolor{textcolor}%
\pgfsetfillcolor{textcolor}%
\pgftext[x=3.865332in,y=2.624442in,,top]{\color{textcolor}\rmfamily\fontsize{10.000000}{12.000000}\selectfont \(\displaystyle {0.03}\)}%
\end{pgfscope}%
\begin{pgfscope}%
\pgfsetrectcap%
\pgfsetroundjoin%
\pgfsetlinewidth{0.803000pt}%
\definecolor{currentstroke}{rgb}{0.000000,0.000000,0.000000}%
\pgfsetstrokecolor{currentstroke}%
\pgfsetdash{}{0pt}%
\pgfpathmoveto{\pgfqpoint{3.601950in}{2.912276in}}%
\pgfpathlineto{\pgfqpoint{3.653304in}{2.899014in}}%
\pgfusepath{stroke}%
\end{pgfscope}%
\begin{pgfscope}%
\definecolor{textcolor}{rgb}{0.000000,0.000000,0.000000}%
\pgfsetstrokecolor{textcolor}%
\pgfsetfillcolor{textcolor}%
\pgftext[x=3.882181in,y=2.941304in,,top]{\color{textcolor}\rmfamily\fontsize{10.000000}{12.000000}\selectfont \(\displaystyle {0.04}\)}%
\end{pgfscope}%
\begin{pgfscope}%
\pgfpathrectangle{\pgfqpoint{0.100000in}{0.212622in}}{\pgfqpoint{3.696000in}{3.696000in}}%
\pgfusepath{clip}%
\pgfsetrectcap%
\pgfsetroundjoin%
\pgfsetlinewidth{1.505625pt}%
\definecolor{currentstroke}{rgb}{0.121569,0.466667,0.705882}%
\pgfsetstrokecolor{currentstroke}%
\pgfsetdash{}{0pt}%
\pgfpathmoveto{\pgfqpoint{1.645993in}{2.142291in}}%
\pgfpathlineto{\pgfqpoint{2.994319in}{1.750684in}}%
\pgfusepath{stroke}%
\end{pgfscope}%
\begin{pgfscope}%
\pgfpathrectangle{\pgfqpoint{0.100000in}{0.212622in}}{\pgfqpoint{3.696000in}{3.696000in}}%
\pgfusepath{clip}%
\pgfsetbuttcap%
\pgfsetroundjoin%
\definecolor{currentfill}{rgb}{0.121569,0.466667,0.705882}%
\pgfsetfillcolor{currentfill}%
\pgfsetfillopacity{0.300000}%
\pgfsetlinewidth{1.003750pt}%
\definecolor{currentstroke}{rgb}{0.121569,0.466667,0.705882}%
\pgfsetstrokecolor{currentstroke}%
\pgfsetstrokeopacity{0.300000}%
\pgfsetdash{}{0pt}%
\pgfpathmoveto{\pgfqpoint{1.646136in}{2.111348in}}%
\pgfpathcurveto{\pgfqpoint{1.654373in}{2.111348in}}{\pgfqpoint{1.662273in}{2.114620in}}{\pgfqpoint{1.668097in}{2.120444in}}%
\pgfpathcurveto{\pgfqpoint{1.673921in}{2.126268in}}{\pgfqpoint{1.677193in}{2.134168in}}{\pgfqpoint{1.677193in}{2.142404in}}%
\pgfpathcurveto{\pgfqpoint{1.677193in}{2.150640in}}{\pgfqpoint{1.673921in}{2.158540in}}{\pgfqpoint{1.668097in}{2.164364in}}%
\pgfpathcurveto{\pgfqpoint{1.662273in}{2.170188in}}{\pgfqpoint{1.654373in}{2.173461in}}{\pgfqpoint{1.646136in}{2.173461in}}%
\pgfpathcurveto{\pgfqpoint{1.637900in}{2.173461in}}{\pgfqpoint{1.630000in}{2.170188in}}{\pgfqpoint{1.624176in}{2.164364in}}%
\pgfpathcurveto{\pgfqpoint{1.618352in}{2.158540in}}{\pgfqpoint{1.615080in}{2.150640in}}{\pgfqpoint{1.615080in}{2.142404in}}%
\pgfpathcurveto{\pgfqpoint{1.615080in}{2.134168in}}{\pgfqpoint{1.618352in}{2.126268in}}{\pgfqpoint{1.624176in}{2.120444in}}%
\pgfpathcurveto{\pgfqpoint{1.630000in}{2.114620in}}{\pgfqpoint{1.637900in}{2.111348in}}{\pgfqpoint{1.646136in}{2.111348in}}%
\pgfpathclose%
\pgfusepath{stroke,fill}%
\end{pgfscope}%
\begin{pgfscope}%
\pgfpathrectangle{\pgfqpoint{0.100000in}{0.212622in}}{\pgfqpoint{3.696000in}{3.696000in}}%
\pgfusepath{clip}%
\pgfsetbuttcap%
\pgfsetroundjoin%
\definecolor{currentfill}{rgb}{0.121569,0.466667,0.705882}%
\pgfsetfillcolor{currentfill}%
\pgfsetfillopacity{0.300023}%
\pgfsetlinewidth{1.003750pt}%
\definecolor{currentstroke}{rgb}{0.121569,0.466667,0.705882}%
\pgfsetstrokecolor{currentstroke}%
\pgfsetstrokeopacity{0.300023}%
\pgfsetdash{}{0pt}%
\pgfpathmoveto{\pgfqpoint{1.646083in}{2.111303in}}%
\pgfpathcurveto{\pgfqpoint{1.654319in}{2.111303in}}{\pgfqpoint{1.662219in}{2.114575in}}{\pgfqpoint{1.668043in}{2.120399in}}%
\pgfpathcurveto{\pgfqpoint{1.673867in}{2.126223in}}{\pgfqpoint{1.677140in}{2.134123in}}{\pgfqpoint{1.677140in}{2.142359in}}%
\pgfpathcurveto{\pgfqpoint{1.677140in}{2.150596in}}{\pgfqpoint{1.673867in}{2.158496in}}{\pgfqpoint{1.668043in}{2.164320in}}%
\pgfpathcurveto{\pgfqpoint{1.662219in}{2.170144in}}{\pgfqpoint{1.654319in}{2.173416in}}{\pgfqpoint{1.646083in}{2.173416in}}%
\pgfpathcurveto{\pgfqpoint{1.637847in}{2.173416in}}{\pgfqpoint{1.629947in}{2.170144in}}{\pgfqpoint{1.624123in}{2.164320in}}%
\pgfpathcurveto{\pgfqpoint{1.618299in}{2.158496in}}{\pgfqpoint{1.615027in}{2.150596in}}{\pgfqpoint{1.615027in}{2.142359in}}%
\pgfpathcurveto{\pgfqpoint{1.615027in}{2.134123in}}{\pgfqpoint{1.618299in}{2.126223in}}{\pgfqpoint{1.624123in}{2.120399in}}%
\pgfpathcurveto{\pgfqpoint{1.629947in}{2.114575in}}{\pgfqpoint{1.637847in}{2.111303in}}{\pgfqpoint{1.646083in}{2.111303in}}%
\pgfpathclose%
\pgfusepath{stroke,fill}%
\end{pgfscope}%
\begin{pgfscope}%
\pgfpathrectangle{\pgfqpoint{0.100000in}{0.212622in}}{\pgfqpoint{3.696000in}{3.696000in}}%
\pgfusepath{clip}%
\pgfsetbuttcap%
\pgfsetroundjoin%
\definecolor{currentfill}{rgb}{0.121569,0.466667,0.705882}%
\pgfsetfillcolor{currentfill}%
\pgfsetfillopacity{0.300026}%
\pgfsetlinewidth{1.003750pt}%
\definecolor{currentstroke}{rgb}{0.121569,0.466667,0.705882}%
\pgfsetstrokecolor{currentstroke}%
\pgfsetstrokeopacity{0.300026}%
\pgfsetdash{}{0pt}%
\pgfpathmoveto{\pgfqpoint{1.646079in}{2.111303in}}%
\pgfpathcurveto{\pgfqpoint{1.654315in}{2.111303in}}{\pgfqpoint{1.662215in}{2.114575in}}{\pgfqpoint{1.668039in}{2.120399in}}%
\pgfpathcurveto{\pgfqpoint{1.673863in}{2.126223in}}{\pgfqpoint{1.677135in}{2.134123in}}{\pgfqpoint{1.677135in}{2.142359in}}%
\pgfpathcurveto{\pgfqpoint{1.677135in}{2.150595in}}{\pgfqpoint{1.673863in}{2.158495in}}{\pgfqpoint{1.668039in}{2.164319in}}%
\pgfpathcurveto{\pgfqpoint{1.662215in}{2.170143in}}{\pgfqpoint{1.654315in}{2.173416in}}{\pgfqpoint{1.646079in}{2.173416in}}%
\pgfpathcurveto{\pgfqpoint{1.637843in}{2.173416in}}{\pgfqpoint{1.629943in}{2.170143in}}{\pgfqpoint{1.624119in}{2.164319in}}%
\pgfpathcurveto{\pgfqpoint{1.618295in}{2.158495in}}{\pgfqpoint{1.615022in}{2.150595in}}{\pgfqpoint{1.615022in}{2.142359in}}%
\pgfpathcurveto{\pgfqpoint{1.615022in}{2.134123in}}{\pgfqpoint{1.618295in}{2.126223in}}{\pgfqpoint{1.624119in}{2.120399in}}%
\pgfpathcurveto{\pgfqpoint{1.629943in}{2.114575in}}{\pgfqpoint{1.637843in}{2.111303in}}{\pgfqpoint{1.646079in}{2.111303in}}%
\pgfpathclose%
\pgfusepath{stroke,fill}%
\end{pgfscope}%
\begin{pgfscope}%
\pgfpathrectangle{\pgfqpoint{0.100000in}{0.212622in}}{\pgfqpoint{3.696000in}{3.696000in}}%
\pgfusepath{clip}%
\pgfsetbuttcap%
\pgfsetroundjoin%
\definecolor{currentfill}{rgb}{0.121569,0.466667,0.705882}%
\pgfsetfillcolor{currentfill}%
\pgfsetfillopacity{0.300030}%
\pgfsetlinewidth{1.003750pt}%
\definecolor{currentstroke}{rgb}{0.121569,0.466667,0.705882}%
\pgfsetstrokecolor{currentstroke}%
\pgfsetstrokeopacity{0.300030}%
\pgfsetdash{}{0pt}%
\pgfpathmoveto{\pgfqpoint{1.646070in}{2.111293in}}%
\pgfpathcurveto{\pgfqpoint{1.654306in}{2.111293in}}{\pgfqpoint{1.662206in}{2.114566in}}{\pgfqpoint{1.668030in}{2.120390in}}%
\pgfpathcurveto{\pgfqpoint{1.673854in}{2.126213in}}{\pgfqpoint{1.677126in}{2.134114in}}{\pgfqpoint{1.677126in}{2.142350in}}%
\pgfpathcurveto{\pgfqpoint{1.677126in}{2.150586in}}{\pgfqpoint{1.673854in}{2.158486in}}{\pgfqpoint{1.668030in}{2.164310in}}%
\pgfpathcurveto{\pgfqpoint{1.662206in}{2.170134in}}{\pgfqpoint{1.654306in}{2.173406in}}{\pgfqpoint{1.646070in}{2.173406in}}%
\pgfpathcurveto{\pgfqpoint{1.637833in}{2.173406in}}{\pgfqpoint{1.629933in}{2.170134in}}{\pgfqpoint{1.624109in}{2.164310in}}%
\pgfpathcurveto{\pgfqpoint{1.618286in}{2.158486in}}{\pgfqpoint{1.615013in}{2.150586in}}{\pgfqpoint{1.615013in}{2.142350in}}%
\pgfpathcurveto{\pgfqpoint{1.615013in}{2.134114in}}{\pgfqpoint{1.618286in}{2.126213in}}{\pgfqpoint{1.624109in}{2.120390in}}%
\pgfpathcurveto{\pgfqpoint{1.629933in}{2.114566in}}{\pgfqpoint{1.637833in}{2.111293in}}{\pgfqpoint{1.646070in}{2.111293in}}%
\pgfpathclose%
\pgfusepath{stroke,fill}%
\end{pgfscope}%
\begin{pgfscope}%
\pgfpathrectangle{\pgfqpoint{0.100000in}{0.212622in}}{\pgfqpoint{3.696000in}{3.696000in}}%
\pgfusepath{clip}%
\pgfsetbuttcap%
\pgfsetroundjoin%
\definecolor{currentfill}{rgb}{0.121569,0.466667,0.705882}%
\pgfsetfillcolor{currentfill}%
\pgfsetfillopacity{0.300031}%
\pgfsetlinewidth{1.003750pt}%
\definecolor{currentstroke}{rgb}{0.121569,0.466667,0.705882}%
\pgfsetstrokecolor{currentstroke}%
\pgfsetstrokeopacity{0.300031}%
\pgfsetdash{}{0pt}%
\pgfpathmoveto{\pgfqpoint{1.646068in}{2.111293in}}%
\pgfpathcurveto{\pgfqpoint{1.654305in}{2.111293in}}{\pgfqpoint{1.662205in}{2.114565in}}{\pgfqpoint{1.668029in}{2.120389in}}%
\pgfpathcurveto{\pgfqpoint{1.673853in}{2.126213in}}{\pgfqpoint{1.677125in}{2.134113in}}{\pgfqpoint{1.677125in}{2.142349in}}%
\pgfpathcurveto{\pgfqpoint{1.677125in}{2.150586in}}{\pgfqpoint{1.673853in}{2.158486in}}{\pgfqpoint{1.668029in}{2.164310in}}%
\pgfpathcurveto{\pgfqpoint{1.662205in}{2.170133in}}{\pgfqpoint{1.654305in}{2.173406in}}{\pgfqpoint{1.646068in}{2.173406in}}%
\pgfpathcurveto{\pgfqpoint{1.637832in}{2.173406in}}{\pgfqpoint{1.629932in}{2.170133in}}{\pgfqpoint{1.624108in}{2.164310in}}%
\pgfpathcurveto{\pgfqpoint{1.618284in}{2.158486in}}{\pgfqpoint{1.615012in}{2.150586in}}{\pgfqpoint{1.615012in}{2.142349in}}%
\pgfpathcurveto{\pgfqpoint{1.615012in}{2.134113in}}{\pgfqpoint{1.618284in}{2.126213in}}{\pgfqpoint{1.624108in}{2.120389in}}%
\pgfpathcurveto{\pgfqpoint{1.629932in}{2.114565in}}{\pgfqpoint{1.637832in}{2.111293in}}{\pgfqpoint{1.646068in}{2.111293in}}%
\pgfpathclose%
\pgfusepath{stroke,fill}%
\end{pgfscope}%
\begin{pgfscope}%
\pgfpathrectangle{\pgfqpoint{0.100000in}{0.212622in}}{\pgfqpoint{3.696000in}{3.696000in}}%
\pgfusepath{clip}%
\pgfsetbuttcap%
\pgfsetroundjoin%
\definecolor{currentfill}{rgb}{0.121569,0.466667,0.705882}%
\pgfsetfillcolor{currentfill}%
\pgfsetfillopacity{0.300032}%
\pgfsetlinewidth{1.003750pt}%
\definecolor{currentstroke}{rgb}{0.121569,0.466667,0.705882}%
\pgfsetstrokecolor{currentstroke}%
\pgfsetstrokeopacity{0.300032}%
\pgfsetdash{}{0pt}%
\pgfpathmoveto{\pgfqpoint{1.646067in}{2.111291in}}%
\pgfpathcurveto{\pgfqpoint{1.654303in}{2.111291in}}{\pgfqpoint{1.662203in}{2.114563in}}{\pgfqpoint{1.668027in}{2.120387in}}%
\pgfpathcurveto{\pgfqpoint{1.673851in}{2.126211in}}{\pgfqpoint{1.677123in}{2.134111in}}{\pgfqpoint{1.677123in}{2.142348in}}%
\pgfpathcurveto{\pgfqpoint{1.677123in}{2.150584in}}{\pgfqpoint{1.673851in}{2.158484in}}{\pgfqpoint{1.668027in}{2.164308in}}%
\pgfpathcurveto{\pgfqpoint{1.662203in}{2.170132in}}{\pgfqpoint{1.654303in}{2.173404in}}{\pgfqpoint{1.646067in}{2.173404in}}%
\pgfpathcurveto{\pgfqpoint{1.637831in}{2.173404in}}{\pgfqpoint{1.629931in}{2.170132in}}{\pgfqpoint{1.624107in}{2.164308in}}%
\pgfpathcurveto{\pgfqpoint{1.618283in}{2.158484in}}{\pgfqpoint{1.615010in}{2.150584in}}{\pgfqpoint{1.615010in}{2.142348in}}%
\pgfpathcurveto{\pgfqpoint{1.615010in}{2.134111in}}{\pgfqpoint{1.618283in}{2.126211in}}{\pgfqpoint{1.624107in}{2.120387in}}%
\pgfpathcurveto{\pgfqpoint{1.629931in}{2.114563in}}{\pgfqpoint{1.637831in}{2.111291in}}{\pgfqpoint{1.646067in}{2.111291in}}%
\pgfpathclose%
\pgfusepath{stroke,fill}%
\end{pgfscope}%
\begin{pgfscope}%
\pgfpathrectangle{\pgfqpoint{0.100000in}{0.212622in}}{\pgfqpoint{3.696000in}{3.696000in}}%
\pgfusepath{clip}%
\pgfsetbuttcap%
\pgfsetroundjoin%
\definecolor{currentfill}{rgb}{0.121569,0.466667,0.705882}%
\pgfsetfillcolor{currentfill}%
\pgfsetfillopacity{0.300032}%
\pgfsetlinewidth{1.003750pt}%
\definecolor{currentstroke}{rgb}{0.121569,0.466667,0.705882}%
\pgfsetstrokecolor{currentstroke}%
\pgfsetstrokeopacity{0.300032}%
\pgfsetdash{}{0pt}%
\pgfpathmoveto{\pgfqpoint{1.646066in}{2.111291in}}%
\pgfpathcurveto{\pgfqpoint{1.654303in}{2.111291in}}{\pgfqpoint{1.662203in}{2.114563in}}{\pgfqpoint{1.668027in}{2.120387in}}%
\pgfpathcurveto{\pgfqpoint{1.673851in}{2.126211in}}{\pgfqpoint{1.677123in}{2.134111in}}{\pgfqpoint{1.677123in}{2.142347in}}%
\pgfpathcurveto{\pgfqpoint{1.677123in}{2.150584in}}{\pgfqpoint{1.673851in}{2.158484in}}{\pgfqpoint{1.668027in}{2.164308in}}%
\pgfpathcurveto{\pgfqpoint{1.662203in}{2.170132in}}{\pgfqpoint{1.654303in}{2.173404in}}{\pgfqpoint{1.646066in}{2.173404in}}%
\pgfpathcurveto{\pgfqpoint{1.637830in}{2.173404in}}{\pgfqpoint{1.629930in}{2.170132in}}{\pgfqpoint{1.624106in}{2.164308in}}%
\pgfpathcurveto{\pgfqpoint{1.618282in}{2.158484in}}{\pgfqpoint{1.615010in}{2.150584in}}{\pgfqpoint{1.615010in}{2.142347in}}%
\pgfpathcurveto{\pgfqpoint{1.615010in}{2.134111in}}{\pgfqpoint{1.618282in}{2.126211in}}{\pgfqpoint{1.624106in}{2.120387in}}%
\pgfpathcurveto{\pgfqpoint{1.629930in}{2.114563in}}{\pgfqpoint{1.637830in}{2.111291in}}{\pgfqpoint{1.646066in}{2.111291in}}%
\pgfpathclose%
\pgfusepath{stroke,fill}%
\end{pgfscope}%
\begin{pgfscope}%
\pgfpathrectangle{\pgfqpoint{0.100000in}{0.212622in}}{\pgfqpoint{3.696000in}{3.696000in}}%
\pgfusepath{clip}%
\pgfsetbuttcap%
\pgfsetroundjoin%
\definecolor{currentfill}{rgb}{0.121569,0.466667,0.705882}%
\pgfsetfillcolor{currentfill}%
\pgfsetfillopacity{0.300032}%
\pgfsetlinewidth{1.003750pt}%
\definecolor{currentstroke}{rgb}{0.121569,0.466667,0.705882}%
\pgfsetstrokecolor{currentstroke}%
\pgfsetstrokeopacity{0.300032}%
\pgfsetdash{}{0pt}%
\pgfpathmoveto{\pgfqpoint{1.646066in}{2.111291in}}%
\pgfpathcurveto{\pgfqpoint{1.654302in}{2.111291in}}{\pgfqpoint{1.662202in}{2.114563in}}{\pgfqpoint{1.668026in}{2.120387in}}%
\pgfpathcurveto{\pgfqpoint{1.673850in}{2.126211in}}{\pgfqpoint{1.677123in}{2.134111in}}{\pgfqpoint{1.677123in}{2.142347in}}%
\pgfpathcurveto{\pgfqpoint{1.677123in}{2.150583in}}{\pgfqpoint{1.673850in}{2.158483in}}{\pgfqpoint{1.668026in}{2.164307in}}%
\pgfpathcurveto{\pgfqpoint{1.662202in}{2.170131in}}{\pgfqpoint{1.654302in}{2.173404in}}{\pgfqpoint{1.646066in}{2.173404in}}%
\pgfpathcurveto{\pgfqpoint{1.637830in}{2.173404in}}{\pgfqpoint{1.629930in}{2.170131in}}{\pgfqpoint{1.624106in}{2.164307in}}%
\pgfpathcurveto{\pgfqpoint{1.618282in}{2.158483in}}{\pgfqpoint{1.615010in}{2.150583in}}{\pgfqpoint{1.615010in}{2.142347in}}%
\pgfpathcurveto{\pgfqpoint{1.615010in}{2.134111in}}{\pgfqpoint{1.618282in}{2.126211in}}{\pgfqpoint{1.624106in}{2.120387in}}%
\pgfpathcurveto{\pgfqpoint{1.629930in}{2.114563in}}{\pgfqpoint{1.637830in}{2.111291in}}{\pgfqpoint{1.646066in}{2.111291in}}%
\pgfpathclose%
\pgfusepath{stroke,fill}%
\end{pgfscope}%
\begin{pgfscope}%
\pgfpathrectangle{\pgfqpoint{0.100000in}{0.212622in}}{\pgfqpoint{3.696000in}{3.696000in}}%
\pgfusepath{clip}%
\pgfsetbuttcap%
\pgfsetroundjoin%
\definecolor{currentfill}{rgb}{0.121569,0.466667,0.705882}%
\pgfsetfillcolor{currentfill}%
\pgfsetfillopacity{0.300032}%
\pgfsetlinewidth{1.003750pt}%
\definecolor{currentstroke}{rgb}{0.121569,0.466667,0.705882}%
\pgfsetstrokecolor{currentstroke}%
\pgfsetstrokeopacity{0.300032}%
\pgfsetdash{}{0pt}%
\pgfpathmoveto{\pgfqpoint{1.646066in}{2.111290in}}%
\pgfpathcurveto{\pgfqpoint{1.654302in}{2.111290in}}{\pgfqpoint{1.662202in}{2.114563in}}{\pgfqpoint{1.668026in}{2.120387in}}%
\pgfpathcurveto{\pgfqpoint{1.673850in}{2.126211in}}{\pgfqpoint{1.677123in}{2.134111in}}{\pgfqpoint{1.677123in}{2.142347in}}%
\pgfpathcurveto{\pgfqpoint{1.677123in}{2.150583in}}{\pgfqpoint{1.673850in}{2.158483in}}{\pgfqpoint{1.668026in}{2.164307in}}%
\pgfpathcurveto{\pgfqpoint{1.662202in}{2.170131in}}{\pgfqpoint{1.654302in}{2.173403in}}{\pgfqpoint{1.646066in}{2.173403in}}%
\pgfpathcurveto{\pgfqpoint{1.637830in}{2.173403in}}{\pgfqpoint{1.629930in}{2.170131in}}{\pgfqpoint{1.624106in}{2.164307in}}%
\pgfpathcurveto{\pgfqpoint{1.618282in}{2.158483in}}{\pgfqpoint{1.615010in}{2.150583in}}{\pgfqpoint{1.615010in}{2.142347in}}%
\pgfpathcurveto{\pgfqpoint{1.615010in}{2.134111in}}{\pgfqpoint{1.618282in}{2.126211in}}{\pgfqpoint{1.624106in}{2.120387in}}%
\pgfpathcurveto{\pgfqpoint{1.629930in}{2.114563in}}{\pgfqpoint{1.637830in}{2.111290in}}{\pgfqpoint{1.646066in}{2.111290in}}%
\pgfpathclose%
\pgfusepath{stroke,fill}%
\end{pgfscope}%
\begin{pgfscope}%
\pgfpathrectangle{\pgfqpoint{0.100000in}{0.212622in}}{\pgfqpoint{3.696000in}{3.696000in}}%
\pgfusepath{clip}%
\pgfsetbuttcap%
\pgfsetroundjoin%
\definecolor{currentfill}{rgb}{0.121569,0.466667,0.705882}%
\pgfsetfillcolor{currentfill}%
\pgfsetfillopacity{0.300032}%
\pgfsetlinewidth{1.003750pt}%
\definecolor{currentstroke}{rgb}{0.121569,0.466667,0.705882}%
\pgfsetstrokecolor{currentstroke}%
\pgfsetstrokeopacity{0.300032}%
\pgfsetdash{}{0pt}%
\pgfpathmoveto{\pgfqpoint{1.646066in}{2.111290in}}%
\pgfpathcurveto{\pgfqpoint{1.654302in}{2.111290in}}{\pgfqpoint{1.662202in}{2.114563in}}{\pgfqpoint{1.668026in}{2.120387in}}%
\pgfpathcurveto{\pgfqpoint{1.673850in}{2.126211in}}{\pgfqpoint{1.677122in}{2.134111in}}{\pgfqpoint{1.677122in}{2.142347in}}%
\pgfpathcurveto{\pgfqpoint{1.677122in}{2.150583in}}{\pgfqpoint{1.673850in}{2.158483in}}{\pgfqpoint{1.668026in}{2.164307in}}%
\pgfpathcurveto{\pgfqpoint{1.662202in}{2.170131in}}{\pgfqpoint{1.654302in}{2.173403in}}{\pgfqpoint{1.646066in}{2.173403in}}%
\pgfpathcurveto{\pgfqpoint{1.637830in}{2.173403in}}{\pgfqpoint{1.629930in}{2.170131in}}{\pgfqpoint{1.624106in}{2.164307in}}%
\pgfpathcurveto{\pgfqpoint{1.618282in}{2.158483in}}{\pgfqpoint{1.615009in}{2.150583in}}{\pgfqpoint{1.615009in}{2.142347in}}%
\pgfpathcurveto{\pgfqpoint{1.615009in}{2.134111in}}{\pgfqpoint{1.618282in}{2.126211in}}{\pgfqpoint{1.624106in}{2.120387in}}%
\pgfpathcurveto{\pgfqpoint{1.629930in}{2.114563in}}{\pgfqpoint{1.637830in}{2.111290in}}{\pgfqpoint{1.646066in}{2.111290in}}%
\pgfpathclose%
\pgfusepath{stroke,fill}%
\end{pgfscope}%
\begin{pgfscope}%
\pgfpathrectangle{\pgfqpoint{0.100000in}{0.212622in}}{\pgfqpoint{3.696000in}{3.696000in}}%
\pgfusepath{clip}%
\pgfsetbuttcap%
\pgfsetroundjoin%
\definecolor{currentfill}{rgb}{0.121569,0.466667,0.705882}%
\pgfsetfillcolor{currentfill}%
\pgfsetfillopacity{0.300032}%
\pgfsetlinewidth{1.003750pt}%
\definecolor{currentstroke}{rgb}{0.121569,0.466667,0.705882}%
\pgfsetstrokecolor{currentstroke}%
\pgfsetstrokeopacity{0.300032}%
\pgfsetdash{}{0pt}%
\pgfpathmoveto{\pgfqpoint{1.646066in}{2.111290in}}%
\pgfpathcurveto{\pgfqpoint{1.654302in}{2.111290in}}{\pgfqpoint{1.662202in}{2.114563in}}{\pgfqpoint{1.668026in}{2.120387in}}%
\pgfpathcurveto{\pgfqpoint{1.673850in}{2.126210in}}{\pgfqpoint{1.677122in}{2.134111in}}{\pgfqpoint{1.677122in}{2.142347in}}%
\pgfpathcurveto{\pgfqpoint{1.677122in}{2.150583in}}{\pgfqpoint{1.673850in}{2.158483in}}{\pgfqpoint{1.668026in}{2.164307in}}%
\pgfpathcurveto{\pgfqpoint{1.662202in}{2.170131in}}{\pgfqpoint{1.654302in}{2.173403in}}{\pgfqpoint{1.646066in}{2.173403in}}%
\pgfpathcurveto{\pgfqpoint{1.637830in}{2.173403in}}{\pgfqpoint{1.629929in}{2.170131in}}{\pgfqpoint{1.624106in}{2.164307in}}%
\pgfpathcurveto{\pgfqpoint{1.618282in}{2.158483in}}{\pgfqpoint{1.615009in}{2.150583in}}{\pgfqpoint{1.615009in}{2.142347in}}%
\pgfpathcurveto{\pgfqpoint{1.615009in}{2.134111in}}{\pgfqpoint{1.618282in}{2.126210in}}{\pgfqpoint{1.624106in}{2.120387in}}%
\pgfpathcurveto{\pgfqpoint{1.629929in}{2.114563in}}{\pgfqpoint{1.637830in}{2.111290in}}{\pgfqpoint{1.646066in}{2.111290in}}%
\pgfpathclose%
\pgfusepath{stroke,fill}%
\end{pgfscope}%
\begin{pgfscope}%
\pgfpathrectangle{\pgfqpoint{0.100000in}{0.212622in}}{\pgfqpoint{3.696000in}{3.696000in}}%
\pgfusepath{clip}%
\pgfsetbuttcap%
\pgfsetroundjoin%
\definecolor{currentfill}{rgb}{0.121569,0.466667,0.705882}%
\pgfsetfillcolor{currentfill}%
\pgfsetfillopacity{0.300032}%
\pgfsetlinewidth{1.003750pt}%
\definecolor{currentstroke}{rgb}{0.121569,0.466667,0.705882}%
\pgfsetstrokecolor{currentstroke}%
\pgfsetstrokeopacity{0.300032}%
\pgfsetdash{}{0pt}%
\pgfpathmoveto{\pgfqpoint{1.646066in}{2.111290in}}%
\pgfpathcurveto{\pgfqpoint{1.654302in}{2.111290in}}{\pgfqpoint{1.662202in}{2.114563in}}{\pgfqpoint{1.668026in}{2.120387in}}%
\pgfpathcurveto{\pgfqpoint{1.673850in}{2.126210in}}{\pgfqpoint{1.677122in}{2.134111in}}{\pgfqpoint{1.677122in}{2.142347in}}%
\pgfpathcurveto{\pgfqpoint{1.677122in}{2.150583in}}{\pgfqpoint{1.673850in}{2.158483in}}{\pgfqpoint{1.668026in}{2.164307in}}%
\pgfpathcurveto{\pgfqpoint{1.662202in}{2.170131in}}{\pgfqpoint{1.654302in}{2.173403in}}{\pgfqpoint{1.646066in}{2.173403in}}%
\pgfpathcurveto{\pgfqpoint{1.637830in}{2.173403in}}{\pgfqpoint{1.629929in}{2.170131in}}{\pgfqpoint{1.624106in}{2.164307in}}%
\pgfpathcurveto{\pgfqpoint{1.618282in}{2.158483in}}{\pgfqpoint{1.615009in}{2.150583in}}{\pgfqpoint{1.615009in}{2.142347in}}%
\pgfpathcurveto{\pgfqpoint{1.615009in}{2.134111in}}{\pgfqpoint{1.618282in}{2.126210in}}{\pgfqpoint{1.624106in}{2.120387in}}%
\pgfpathcurveto{\pgfqpoint{1.629929in}{2.114563in}}{\pgfqpoint{1.637830in}{2.111290in}}{\pgfqpoint{1.646066in}{2.111290in}}%
\pgfpathclose%
\pgfusepath{stroke,fill}%
\end{pgfscope}%
\begin{pgfscope}%
\pgfpathrectangle{\pgfqpoint{0.100000in}{0.212622in}}{\pgfqpoint{3.696000in}{3.696000in}}%
\pgfusepath{clip}%
\pgfsetbuttcap%
\pgfsetroundjoin%
\definecolor{currentfill}{rgb}{0.121569,0.466667,0.705882}%
\pgfsetfillcolor{currentfill}%
\pgfsetfillopacity{0.300032}%
\pgfsetlinewidth{1.003750pt}%
\definecolor{currentstroke}{rgb}{0.121569,0.466667,0.705882}%
\pgfsetstrokecolor{currentstroke}%
\pgfsetstrokeopacity{0.300032}%
\pgfsetdash{}{0pt}%
\pgfpathmoveto{\pgfqpoint{1.646066in}{2.111290in}}%
\pgfpathcurveto{\pgfqpoint{1.654302in}{2.111290in}}{\pgfqpoint{1.662202in}{2.114563in}}{\pgfqpoint{1.668026in}{2.120387in}}%
\pgfpathcurveto{\pgfqpoint{1.673850in}{2.126210in}}{\pgfqpoint{1.677122in}{2.134110in}}{\pgfqpoint{1.677122in}{2.142347in}}%
\pgfpathcurveto{\pgfqpoint{1.677122in}{2.150583in}}{\pgfqpoint{1.673850in}{2.158483in}}{\pgfqpoint{1.668026in}{2.164307in}}%
\pgfpathcurveto{\pgfqpoint{1.662202in}{2.170131in}}{\pgfqpoint{1.654302in}{2.173403in}}{\pgfqpoint{1.646066in}{2.173403in}}%
\pgfpathcurveto{\pgfqpoint{1.637829in}{2.173403in}}{\pgfqpoint{1.629929in}{2.170131in}}{\pgfqpoint{1.624106in}{2.164307in}}%
\pgfpathcurveto{\pgfqpoint{1.618282in}{2.158483in}}{\pgfqpoint{1.615009in}{2.150583in}}{\pgfqpoint{1.615009in}{2.142347in}}%
\pgfpathcurveto{\pgfqpoint{1.615009in}{2.134110in}}{\pgfqpoint{1.618282in}{2.126210in}}{\pgfqpoint{1.624106in}{2.120387in}}%
\pgfpathcurveto{\pgfqpoint{1.629929in}{2.114563in}}{\pgfqpoint{1.637829in}{2.111290in}}{\pgfqpoint{1.646066in}{2.111290in}}%
\pgfpathclose%
\pgfusepath{stroke,fill}%
\end{pgfscope}%
\begin{pgfscope}%
\pgfpathrectangle{\pgfqpoint{0.100000in}{0.212622in}}{\pgfqpoint{3.696000in}{3.696000in}}%
\pgfusepath{clip}%
\pgfsetbuttcap%
\pgfsetroundjoin%
\definecolor{currentfill}{rgb}{0.121569,0.466667,0.705882}%
\pgfsetfillcolor{currentfill}%
\pgfsetfillopacity{0.300032}%
\pgfsetlinewidth{1.003750pt}%
\definecolor{currentstroke}{rgb}{0.121569,0.466667,0.705882}%
\pgfsetstrokecolor{currentstroke}%
\pgfsetstrokeopacity{0.300032}%
\pgfsetdash{}{0pt}%
\pgfpathmoveto{\pgfqpoint{1.646066in}{2.111290in}}%
\pgfpathcurveto{\pgfqpoint{1.654302in}{2.111290in}}{\pgfqpoint{1.662202in}{2.114563in}}{\pgfqpoint{1.668026in}{2.120386in}}%
\pgfpathcurveto{\pgfqpoint{1.673850in}{2.126210in}}{\pgfqpoint{1.677122in}{2.134110in}}{\pgfqpoint{1.677122in}{2.142347in}}%
\pgfpathcurveto{\pgfqpoint{1.677122in}{2.150583in}}{\pgfqpoint{1.673850in}{2.158483in}}{\pgfqpoint{1.668026in}{2.164307in}}%
\pgfpathcurveto{\pgfqpoint{1.662202in}{2.170131in}}{\pgfqpoint{1.654302in}{2.173403in}}{\pgfqpoint{1.646066in}{2.173403in}}%
\pgfpathcurveto{\pgfqpoint{1.637829in}{2.173403in}}{\pgfqpoint{1.629929in}{2.170131in}}{\pgfqpoint{1.624105in}{2.164307in}}%
\pgfpathcurveto{\pgfqpoint{1.618282in}{2.158483in}}{\pgfqpoint{1.615009in}{2.150583in}}{\pgfqpoint{1.615009in}{2.142347in}}%
\pgfpathcurveto{\pgfqpoint{1.615009in}{2.134110in}}{\pgfqpoint{1.618282in}{2.126210in}}{\pgfqpoint{1.624105in}{2.120386in}}%
\pgfpathcurveto{\pgfqpoint{1.629929in}{2.114563in}}{\pgfqpoint{1.637829in}{2.111290in}}{\pgfqpoint{1.646066in}{2.111290in}}%
\pgfpathclose%
\pgfusepath{stroke,fill}%
\end{pgfscope}%
\begin{pgfscope}%
\pgfpathrectangle{\pgfqpoint{0.100000in}{0.212622in}}{\pgfqpoint{3.696000in}{3.696000in}}%
\pgfusepath{clip}%
\pgfsetbuttcap%
\pgfsetroundjoin%
\definecolor{currentfill}{rgb}{0.121569,0.466667,0.705882}%
\pgfsetfillcolor{currentfill}%
\pgfsetfillopacity{0.300032}%
\pgfsetlinewidth{1.003750pt}%
\definecolor{currentstroke}{rgb}{0.121569,0.466667,0.705882}%
\pgfsetstrokecolor{currentstroke}%
\pgfsetstrokeopacity{0.300032}%
\pgfsetdash{}{0pt}%
\pgfpathmoveto{\pgfqpoint{1.646066in}{2.111290in}}%
\pgfpathcurveto{\pgfqpoint{1.654302in}{2.111290in}}{\pgfqpoint{1.662202in}{2.114563in}}{\pgfqpoint{1.668026in}{2.120386in}}%
\pgfpathcurveto{\pgfqpoint{1.673850in}{2.126210in}}{\pgfqpoint{1.677122in}{2.134110in}}{\pgfqpoint{1.677122in}{2.142347in}}%
\pgfpathcurveto{\pgfqpoint{1.677122in}{2.150583in}}{\pgfqpoint{1.673850in}{2.158483in}}{\pgfqpoint{1.668026in}{2.164307in}}%
\pgfpathcurveto{\pgfqpoint{1.662202in}{2.170131in}}{\pgfqpoint{1.654302in}{2.173403in}}{\pgfqpoint{1.646066in}{2.173403in}}%
\pgfpathcurveto{\pgfqpoint{1.637829in}{2.173403in}}{\pgfqpoint{1.629929in}{2.170131in}}{\pgfqpoint{1.624105in}{2.164307in}}%
\pgfpathcurveto{\pgfqpoint{1.618282in}{2.158483in}}{\pgfqpoint{1.615009in}{2.150583in}}{\pgfqpoint{1.615009in}{2.142347in}}%
\pgfpathcurveto{\pgfqpoint{1.615009in}{2.134110in}}{\pgfqpoint{1.618282in}{2.126210in}}{\pgfqpoint{1.624105in}{2.120386in}}%
\pgfpathcurveto{\pgfqpoint{1.629929in}{2.114563in}}{\pgfqpoint{1.637829in}{2.111290in}}{\pgfqpoint{1.646066in}{2.111290in}}%
\pgfpathclose%
\pgfusepath{stroke,fill}%
\end{pgfscope}%
\begin{pgfscope}%
\pgfpathrectangle{\pgfqpoint{0.100000in}{0.212622in}}{\pgfqpoint{3.696000in}{3.696000in}}%
\pgfusepath{clip}%
\pgfsetbuttcap%
\pgfsetroundjoin%
\definecolor{currentfill}{rgb}{0.121569,0.466667,0.705882}%
\pgfsetfillcolor{currentfill}%
\pgfsetfillopacity{0.300032}%
\pgfsetlinewidth{1.003750pt}%
\definecolor{currentstroke}{rgb}{0.121569,0.466667,0.705882}%
\pgfsetstrokecolor{currentstroke}%
\pgfsetstrokeopacity{0.300032}%
\pgfsetdash{}{0pt}%
\pgfpathmoveto{\pgfqpoint{1.646066in}{2.111290in}}%
\pgfpathcurveto{\pgfqpoint{1.654302in}{2.111290in}}{\pgfqpoint{1.662202in}{2.114563in}}{\pgfqpoint{1.668026in}{2.120386in}}%
\pgfpathcurveto{\pgfqpoint{1.673850in}{2.126210in}}{\pgfqpoint{1.677122in}{2.134110in}}{\pgfqpoint{1.677122in}{2.142347in}}%
\pgfpathcurveto{\pgfqpoint{1.677122in}{2.150583in}}{\pgfqpoint{1.673850in}{2.158483in}}{\pgfqpoint{1.668026in}{2.164307in}}%
\pgfpathcurveto{\pgfqpoint{1.662202in}{2.170131in}}{\pgfqpoint{1.654302in}{2.173403in}}{\pgfqpoint{1.646066in}{2.173403in}}%
\pgfpathcurveto{\pgfqpoint{1.637829in}{2.173403in}}{\pgfqpoint{1.629929in}{2.170131in}}{\pgfqpoint{1.624105in}{2.164307in}}%
\pgfpathcurveto{\pgfqpoint{1.618282in}{2.158483in}}{\pgfqpoint{1.615009in}{2.150583in}}{\pgfqpoint{1.615009in}{2.142347in}}%
\pgfpathcurveto{\pgfqpoint{1.615009in}{2.134110in}}{\pgfqpoint{1.618282in}{2.126210in}}{\pgfqpoint{1.624105in}{2.120386in}}%
\pgfpathcurveto{\pgfqpoint{1.629929in}{2.114563in}}{\pgfqpoint{1.637829in}{2.111290in}}{\pgfqpoint{1.646066in}{2.111290in}}%
\pgfpathclose%
\pgfusepath{stroke,fill}%
\end{pgfscope}%
\begin{pgfscope}%
\pgfpathrectangle{\pgfqpoint{0.100000in}{0.212622in}}{\pgfqpoint{3.696000in}{3.696000in}}%
\pgfusepath{clip}%
\pgfsetbuttcap%
\pgfsetroundjoin%
\definecolor{currentfill}{rgb}{0.121569,0.466667,0.705882}%
\pgfsetfillcolor{currentfill}%
\pgfsetfillopacity{0.300032}%
\pgfsetlinewidth{1.003750pt}%
\definecolor{currentstroke}{rgb}{0.121569,0.466667,0.705882}%
\pgfsetstrokecolor{currentstroke}%
\pgfsetstrokeopacity{0.300032}%
\pgfsetdash{}{0pt}%
\pgfpathmoveto{\pgfqpoint{1.646066in}{2.111290in}}%
\pgfpathcurveto{\pgfqpoint{1.654302in}{2.111290in}}{\pgfqpoint{1.662202in}{2.114563in}}{\pgfqpoint{1.668026in}{2.120386in}}%
\pgfpathcurveto{\pgfqpoint{1.673850in}{2.126210in}}{\pgfqpoint{1.677122in}{2.134110in}}{\pgfqpoint{1.677122in}{2.142347in}}%
\pgfpathcurveto{\pgfqpoint{1.677122in}{2.150583in}}{\pgfqpoint{1.673850in}{2.158483in}}{\pgfqpoint{1.668026in}{2.164307in}}%
\pgfpathcurveto{\pgfqpoint{1.662202in}{2.170131in}}{\pgfqpoint{1.654302in}{2.173403in}}{\pgfqpoint{1.646066in}{2.173403in}}%
\pgfpathcurveto{\pgfqpoint{1.637829in}{2.173403in}}{\pgfqpoint{1.629929in}{2.170131in}}{\pgfqpoint{1.624105in}{2.164307in}}%
\pgfpathcurveto{\pgfqpoint{1.618282in}{2.158483in}}{\pgfqpoint{1.615009in}{2.150583in}}{\pgfqpoint{1.615009in}{2.142347in}}%
\pgfpathcurveto{\pgfqpoint{1.615009in}{2.134110in}}{\pgfqpoint{1.618282in}{2.126210in}}{\pgfqpoint{1.624105in}{2.120386in}}%
\pgfpathcurveto{\pgfqpoint{1.629929in}{2.114563in}}{\pgfqpoint{1.637829in}{2.111290in}}{\pgfqpoint{1.646066in}{2.111290in}}%
\pgfpathclose%
\pgfusepath{stroke,fill}%
\end{pgfscope}%
\begin{pgfscope}%
\pgfpathrectangle{\pgfqpoint{0.100000in}{0.212622in}}{\pgfqpoint{3.696000in}{3.696000in}}%
\pgfusepath{clip}%
\pgfsetbuttcap%
\pgfsetroundjoin%
\definecolor{currentfill}{rgb}{0.121569,0.466667,0.705882}%
\pgfsetfillcolor{currentfill}%
\pgfsetfillopacity{0.300032}%
\pgfsetlinewidth{1.003750pt}%
\definecolor{currentstroke}{rgb}{0.121569,0.466667,0.705882}%
\pgfsetstrokecolor{currentstroke}%
\pgfsetstrokeopacity{0.300032}%
\pgfsetdash{}{0pt}%
\pgfpathmoveto{\pgfqpoint{1.646066in}{2.111290in}}%
\pgfpathcurveto{\pgfqpoint{1.654302in}{2.111290in}}{\pgfqpoint{1.662202in}{2.114563in}}{\pgfqpoint{1.668026in}{2.120386in}}%
\pgfpathcurveto{\pgfqpoint{1.673850in}{2.126210in}}{\pgfqpoint{1.677122in}{2.134110in}}{\pgfqpoint{1.677122in}{2.142347in}}%
\pgfpathcurveto{\pgfqpoint{1.677122in}{2.150583in}}{\pgfqpoint{1.673850in}{2.158483in}}{\pgfqpoint{1.668026in}{2.164307in}}%
\pgfpathcurveto{\pgfqpoint{1.662202in}{2.170131in}}{\pgfqpoint{1.654302in}{2.173403in}}{\pgfqpoint{1.646066in}{2.173403in}}%
\pgfpathcurveto{\pgfqpoint{1.637829in}{2.173403in}}{\pgfqpoint{1.629929in}{2.170131in}}{\pgfqpoint{1.624105in}{2.164307in}}%
\pgfpathcurveto{\pgfqpoint{1.618282in}{2.158483in}}{\pgfqpoint{1.615009in}{2.150583in}}{\pgfqpoint{1.615009in}{2.142347in}}%
\pgfpathcurveto{\pgfqpoint{1.615009in}{2.134110in}}{\pgfqpoint{1.618282in}{2.126210in}}{\pgfqpoint{1.624105in}{2.120386in}}%
\pgfpathcurveto{\pgfqpoint{1.629929in}{2.114563in}}{\pgfqpoint{1.637829in}{2.111290in}}{\pgfqpoint{1.646066in}{2.111290in}}%
\pgfpathclose%
\pgfusepath{stroke,fill}%
\end{pgfscope}%
\begin{pgfscope}%
\pgfpathrectangle{\pgfqpoint{0.100000in}{0.212622in}}{\pgfqpoint{3.696000in}{3.696000in}}%
\pgfusepath{clip}%
\pgfsetbuttcap%
\pgfsetroundjoin%
\definecolor{currentfill}{rgb}{0.121569,0.466667,0.705882}%
\pgfsetfillcolor{currentfill}%
\pgfsetfillopacity{0.300032}%
\pgfsetlinewidth{1.003750pt}%
\definecolor{currentstroke}{rgb}{0.121569,0.466667,0.705882}%
\pgfsetstrokecolor{currentstroke}%
\pgfsetstrokeopacity{0.300032}%
\pgfsetdash{}{0pt}%
\pgfpathmoveto{\pgfqpoint{1.646066in}{2.111290in}}%
\pgfpathcurveto{\pgfqpoint{1.654302in}{2.111290in}}{\pgfqpoint{1.662202in}{2.114563in}}{\pgfqpoint{1.668026in}{2.120386in}}%
\pgfpathcurveto{\pgfqpoint{1.673850in}{2.126210in}}{\pgfqpoint{1.677122in}{2.134110in}}{\pgfqpoint{1.677122in}{2.142347in}}%
\pgfpathcurveto{\pgfqpoint{1.677122in}{2.150583in}}{\pgfqpoint{1.673850in}{2.158483in}}{\pgfqpoint{1.668026in}{2.164307in}}%
\pgfpathcurveto{\pgfqpoint{1.662202in}{2.170131in}}{\pgfqpoint{1.654302in}{2.173403in}}{\pgfqpoint{1.646066in}{2.173403in}}%
\pgfpathcurveto{\pgfqpoint{1.637829in}{2.173403in}}{\pgfqpoint{1.629929in}{2.170131in}}{\pgfqpoint{1.624105in}{2.164307in}}%
\pgfpathcurveto{\pgfqpoint{1.618282in}{2.158483in}}{\pgfqpoint{1.615009in}{2.150583in}}{\pgfqpoint{1.615009in}{2.142347in}}%
\pgfpathcurveto{\pgfqpoint{1.615009in}{2.134110in}}{\pgfqpoint{1.618282in}{2.126210in}}{\pgfqpoint{1.624105in}{2.120386in}}%
\pgfpathcurveto{\pgfqpoint{1.629929in}{2.114563in}}{\pgfqpoint{1.637829in}{2.111290in}}{\pgfqpoint{1.646066in}{2.111290in}}%
\pgfpathclose%
\pgfusepath{stroke,fill}%
\end{pgfscope}%
\begin{pgfscope}%
\pgfpathrectangle{\pgfqpoint{0.100000in}{0.212622in}}{\pgfqpoint{3.696000in}{3.696000in}}%
\pgfusepath{clip}%
\pgfsetbuttcap%
\pgfsetroundjoin%
\definecolor{currentfill}{rgb}{0.121569,0.466667,0.705882}%
\pgfsetfillcolor{currentfill}%
\pgfsetfillopacity{0.300032}%
\pgfsetlinewidth{1.003750pt}%
\definecolor{currentstroke}{rgb}{0.121569,0.466667,0.705882}%
\pgfsetstrokecolor{currentstroke}%
\pgfsetstrokeopacity{0.300032}%
\pgfsetdash{}{0pt}%
\pgfpathmoveto{\pgfqpoint{1.646066in}{2.111290in}}%
\pgfpathcurveto{\pgfqpoint{1.654302in}{2.111290in}}{\pgfqpoint{1.662202in}{2.114563in}}{\pgfqpoint{1.668026in}{2.120386in}}%
\pgfpathcurveto{\pgfqpoint{1.673850in}{2.126210in}}{\pgfqpoint{1.677122in}{2.134110in}}{\pgfqpoint{1.677122in}{2.142347in}}%
\pgfpathcurveto{\pgfqpoint{1.677122in}{2.150583in}}{\pgfqpoint{1.673850in}{2.158483in}}{\pgfqpoint{1.668026in}{2.164307in}}%
\pgfpathcurveto{\pgfqpoint{1.662202in}{2.170131in}}{\pgfqpoint{1.654302in}{2.173403in}}{\pgfqpoint{1.646066in}{2.173403in}}%
\pgfpathcurveto{\pgfqpoint{1.637829in}{2.173403in}}{\pgfqpoint{1.629929in}{2.170131in}}{\pgfqpoint{1.624105in}{2.164307in}}%
\pgfpathcurveto{\pgfqpoint{1.618282in}{2.158483in}}{\pgfqpoint{1.615009in}{2.150583in}}{\pgfqpoint{1.615009in}{2.142347in}}%
\pgfpathcurveto{\pgfqpoint{1.615009in}{2.134110in}}{\pgfqpoint{1.618282in}{2.126210in}}{\pgfqpoint{1.624105in}{2.120386in}}%
\pgfpathcurveto{\pgfqpoint{1.629929in}{2.114563in}}{\pgfqpoint{1.637829in}{2.111290in}}{\pgfqpoint{1.646066in}{2.111290in}}%
\pgfpathclose%
\pgfusepath{stroke,fill}%
\end{pgfscope}%
\begin{pgfscope}%
\pgfpathrectangle{\pgfqpoint{0.100000in}{0.212622in}}{\pgfqpoint{3.696000in}{3.696000in}}%
\pgfusepath{clip}%
\pgfsetbuttcap%
\pgfsetroundjoin%
\definecolor{currentfill}{rgb}{0.121569,0.466667,0.705882}%
\pgfsetfillcolor{currentfill}%
\pgfsetfillopacity{0.300032}%
\pgfsetlinewidth{1.003750pt}%
\definecolor{currentstroke}{rgb}{0.121569,0.466667,0.705882}%
\pgfsetstrokecolor{currentstroke}%
\pgfsetstrokeopacity{0.300032}%
\pgfsetdash{}{0pt}%
\pgfpathmoveto{\pgfqpoint{1.646066in}{2.111290in}}%
\pgfpathcurveto{\pgfqpoint{1.654302in}{2.111290in}}{\pgfqpoint{1.662202in}{2.114563in}}{\pgfqpoint{1.668026in}{2.120386in}}%
\pgfpathcurveto{\pgfqpoint{1.673850in}{2.126210in}}{\pgfqpoint{1.677122in}{2.134110in}}{\pgfqpoint{1.677122in}{2.142347in}}%
\pgfpathcurveto{\pgfqpoint{1.677122in}{2.150583in}}{\pgfqpoint{1.673850in}{2.158483in}}{\pgfqpoint{1.668026in}{2.164307in}}%
\pgfpathcurveto{\pgfqpoint{1.662202in}{2.170131in}}{\pgfqpoint{1.654302in}{2.173403in}}{\pgfqpoint{1.646066in}{2.173403in}}%
\pgfpathcurveto{\pgfqpoint{1.637829in}{2.173403in}}{\pgfqpoint{1.629929in}{2.170131in}}{\pgfqpoint{1.624105in}{2.164307in}}%
\pgfpathcurveto{\pgfqpoint{1.618282in}{2.158483in}}{\pgfqpoint{1.615009in}{2.150583in}}{\pgfqpoint{1.615009in}{2.142347in}}%
\pgfpathcurveto{\pgfqpoint{1.615009in}{2.134110in}}{\pgfqpoint{1.618282in}{2.126210in}}{\pgfqpoint{1.624105in}{2.120386in}}%
\pgfpathcurveto{\pgfqpoint{1.629929in}{2.114563in}}{\pgfqpoint{1.637829in}{2.111290in}}{\pgfqpoint{1.646066in}{2.111290in}}%
\pgfpathclose%
\pgfusepath{stroke,fill}%
\end{pgfscope}%
\begin{pgfscope}%
\pgfpathrectangle{\pgfqpoint{0.100000in}{0.212622in}}{\pgfqpoint{3.696000in}{3.696000in}}%
\pgfusepath{clip}%
\pgfsetbuttcap%
\pgfsetroundjoin%
\definecolor{currentfill}{rgb}{0.121569,0.466667,0.705882}%
\pgfsetfillcolor{currentfill}%
\pgfsetfillopacity{0.300032}%
\pgfsetlinewidth{1.003750pt}%
\definecolor{currentstroke}{rgb}{0.121569,0.466667,0.705882}%
\pgfsetstrokecolor{currentstroke}%
\pgfsetstrokeopacity{0.300032}%
\pgfsetdash{}{0pt}%
\pgfpathmoveto{\pgfqpoint{1.646066in}{2.111290in}}%
\pgfpathcurveto{\pgfqpoint{1.654302in}{2.111290in}}{\pgfqpoint{1.662202in}{2.114563in}}{\pgfqpoint{1.668026in}{2.120386in}}%
\pgfpathcurveto{\pgfqpoint{1.673850in}{2.126210in}}{\pgfqpoint{1.677122in}{2.134110in}}{\pgfqpoint{1.677122in}{2.142347in}}%
\pgfpathcurveto{\pgfqpoint{1.677122in}{2.150583in}}{\pgfqpoint{1.673850in}{2.158483in}}{\pgfqpoint{1.668026in}{2.164307in}}%
\pgfpathcurveto{\pgfqpoint{1.662202in}{2.170131in}}{\pgfqpoint{1.654302in}{2.173403in}}{\pgfqpoint{1.646066in}{2.173403in}}%
\pgfpathcurveto{\pgfqpoint{1.637829in}{2.173403in}}{\pgfqpoint{1.629929in}{2.170131in}}{\pgfqpoint{1.624105in}{2.164307in}}%
\pgfpathcurveto{\pgfqpoint{1.618282in}{2.158483in}}{\pgfqpoint{1.615009in}{2.150583in}}{\pgfqpoint{1.615009in}{2.142347in}}%
\pgfpathcurveto{\pgfqpoint{1.615009in}{2.134110in}}{\pgfqpoint{1.618282in}{2.126210in}}{\pgfqpoint{1.624105in}{2.120386in}}%
\pgfpathcurveto{\pgfqpoint{1.629929in}{2.114563in}}{\pgfqpoint{1.637829in}{2.111290in}}{\pgfqpoint{1.646066in}{2.111290in}}%
\pgfpathclose%
\pgfusepath{stroke,fill}%
\end{pgfscope}%
\begin{pgfscope}%
\pgfpathrectangle{\pgfqpoint{0.100000in}{0.212622in}}{\pgfqpoint{3.696000in}{3.696000in}}%
\pgfusepath{clip}%
\pgfsetbuttcap%
\pgfsetroundjoin%
\definecolor{currentfill}{rgb}{0.121569,0.466667,0.705882}%
\pgfsetfillcolor{currentfill}%
\pgfsetfillopacity{0.300032}%
\pgfsetlinewidth{1.003750pt}%
\definecolor{currentstroke}{rgb}{0.121569,0.466667,0.705882}%
\pgfsetstrokecolor{currentstroke}%
\pgfsetstrokeopacity{0.300032}%
\pgfsetdash{}{0pt}%
\pgfpathmoveto{\pgfqpoint{1.646066in}{2.111290in}}%
\pgfpathcurveto{\pgfqpoint{1.654302in}{2.111290in}}{\pgfqpoint{1.662202in}{2.114563in}}{\pgfqpoint{1.668026in}{2.120386in}}%
\pgfpathcurveto{\pgfqpoint{1.673850in}{2.126210in}}{\pgfqpoint{1.677122in}{2.134110in}}{\pgfqpoint{1.677122in}{2.142347in}}%
\pgfpathcurveto{\pgfqpoint{1.677122in}{2.150583in}}{\pgfqpoint{1.673850in}{2.158483in}}{\pgfqpoint{1.668026in}{2.164307in}}%
\pgfpathcurveto{\pgfqpoint{1.662202in}{2.170131in}}{\pgfqpoint{1.654302in}{2.173403in}}{\pgfqpoint{1.646066in}{2.173403in}}%
\pgfpathcurveto{\pgfqpoint{1.637829in}{2.173403in}}{\pgfqpoint{1.629929in}{2.170131in}}{\pgfqpoint{1.624105in}{2.164307in}}%
\pgfpathcurveto{\pgfqpoint{1.618282in}{2.158483in}}{\pgfqpoint{1.615009in}{2.150583in}}{\pgfqpoint{1.615009in}{2.142347in}}%
\pgfpathcurveto{\pgfqpoint{1.615009in}{2.134110in}}{\pgfqpoint{1.618282in}{2.126210in}}{\pgfqpoint{1.624105in}{2.120386in}}%
\pgfpathcurveto{\pgfqpoint{1.629929in}{2.114563in}}{\pgfqpoint{1.637829in}{2.111290in}}{\pgfqpoint{1.646066in}{2.111290in}}%
\pgfpathclose%
\pgfusepath{stroke,fill}%
\end{pgfscope}%
\begin{pgfscope}%
\pgfpathrectangle{\pgfqpoint{0.100000in}{0.212622in}}{\pgfqpoint{3.696000in}{3.696000in}}%
\pgfusepath{clip}%
\pgfsetbuttcap%
\pgfsetroundjoin%
\definecolor{currentfill}{rgb}{0.121569,0.466667,0.705882}%
\pgfsetfillcolor{currentfill}%
\pgfsetfillopacity{0.300032}%
\pgfsetlinewidth{1.003750pt}%
\definecolor{currentstroke}{rgb}{0.121569,0.466667,0.705882}%
\pgfsetstrokecolor{currentstroke}%
\pgfsetstrokeopacity{0.300032}%
\pgfsetdash{}{0pt}%
\pgfpathmoveto{\pgfqpoint{1.646066in}{2.111290in}}%
\pgfpathcurveto{\pgfqpoint{1.654302in}{2.111290in}}{\pgfqpoint{1.662202in}{2.114563in}}{\pgfqpoint{1.668026in}{2.120386in}}%
\pgfpathcurveto{\pgfqpoint{1.673850in}{2.126210in}}{\pgfqpoint{1.677122in}{2.134110in}}{\pgfqpoint{1.677122in}{2.142347in}}%
\pgfpathcurveto{\pgfqpoint{1.677122in}{2.150583in}}{\pgfqpoint{1.673850in}{2.158483in}}{\pgfqpoint{1.668026in}{2.164307in}}%
\pgfpathcurveto{\pgfqpoint{1.662202in}{2.170131in}}{\pgfqpoint{1.654302in}{2.173403in}}{\pgfqpoint{1.646066in}{2.173403in}}%
\pgfpathcurveto{\pgfqpoint{1.637829in}{2.173403in}}{\pgfqpoint{1.629929in}{2.170131in}}{\pgfqpoint{1.624105in}{2.164307in}}%
\pgfpathcurveto{\pgfqpoint{1.618282in}{2.158483in}}{\pgfqpoint{1.615009in}{2.150583in}}{\pgfqpoint{1.615009in}{2.142347in}}%
\pgfpathcurveto{\pgfqpoint{1.615009in}{2.134110in}}{\pgfqpoint{1.618282in}{2.126210in}}{\pgfqpoint{1.624105in}{2.120386in}}%
\pgfpathcurveto{\pgfqpoint{1.629929in}{2.114563in}}{\pgfqpoint{1.637829in}{2.111290in}}{\pgfqpoint{1.646066in}{2.111290in}}%
\pgfpathclose%
\pgfusepath{stroke,fill}%
\end{pgfscope}%
\begin{pgfscope}%
\pgfpathrectangle{\pgfqpoint{0.100000in}{0.212622in}}{\pgfqpoint{3.696000in}{3.696000in}}%
\pgfusepath{clip}%
\pgfsetbuttcap%
\pgfsetroundjoin%
\definecolor{currentfill}{rgb}{0.121569,0.466667,0.705882}%
\pgfsetfillcolor{currentfill}%
\pgfsetfillopacity{0.300032}%
\pgfsetlinewidth{1.003750pt}%
\definecolor{currentstroke}{rgb}{0.121569,0.466667,0.705882}%
\pgfsetstrokecolor{currentstroke}%
\pgfsetstrokeopacity{0.300032}%
\pgfsetdash{}{0pt}%
\pgfpathmoveto{\pgfqpoint{1.646066in}{2.111290in}}%
\pgfpathcurveto{\pgfqpoint{1.654302in}{2.111290in}}{\pgfqpoint{1.662202in}{2.114563in}}{\pgfqpoint{1.668026in}{2.120386in}}%
\pgfpathcurveto{\pgfqpoint{1.673850in}{2.126210in}}{\pgfqpoint{1.677122in}{2.134110in}}{\pgfqpoint{1.677122in}{2.142347in}}%
\pgfpathcurveto{\pgfqpoint{1.677122in}{2.150583in}}{\pgfqpoint{1.673850in}{2.158483in}}{\pgfqpoint{1.668026in}{2.164307in}}%
\pgfpathcurveto{\pgfqpoint{1.662202in}{2.170131in}}{\pgfqpoint{1.654302in}{2.173403in}}{\pgfqpoint{1.646066in}{2.173403in}}%
\pgfpathcurveto{\pgfqpoint{1.637829in}{2.173403in}}{\pgfqpoint{1.629929in}{2.170131in}}{\pgfqpoint{1.624105in}{2.164307in}}%
\pgfpathcurveto{\pgfqpoint{1.618282in}{2.158483in}}{\pgfqpoint{1.615009in}{2.150583in}}{\pgfqpoint{1.615009in}{2.142347in}}%
\pgfpathcurveto{\pgfqpoint{1.615009in}{2.134110in}}{\pgfqpoint{1.618282in}{2.126210in}}{\pgfqpoint{1.624105in}{2.120386in}}%
\pgfpathcurveto{\pgfqpoint{1.629929in}{2.114563in}}{\pgfqpoint{1.637829in}{2.111290in}}{\pgfqpoint{1.646066in}{2.111290in}}%
\pgfpathclose%
\pgfusepath{stroke,fill}%
\end{pgfscope}%
\begin{pgfscope}%
\pgfpathrectangle{\pgfqpoint{0.100000in}{0.212622in}}{\pgfqpoint{3.696000in}{3.696000in}}%
\pgfusepath{clip}%
\pgfsetbuttcap%
\pgfsetroundjoin%
\definecolor{currentfill}{rgb}{0.121569,0.466667,0.705882}%
\pgfsetfillcolor{currentfill}%
\pgfsetfillopacity{0.300032}%
\pgfsetlinewidth{1.003750pt}%
\definecolor{currentstroke}{rgb}{0.121569,0.466667,0.705882}%
\pgfsetstrokecolor{currentstroke}%
\pgfsetstrokeopacity{0.300032}%
\pgfsetdash{}{0pt}%
\pgfpathmoveto{\pgfqpoint{1.646066in}{2.111290in}}%
\pgfpathcurveto{\pgfqpoint{1.654302in}{2.111290in}}{\pgfqpoint{1.662202in}{2.114563in}}{\pgfqpoint{1.668026in}{2.120386in}}%
\pgfpathcurveto{\pgfqpoint{1.673850in}{2.126210in}}{\pgfqpoint{1.677122in}{2.134110in}}{\pgfqpoint{1.677122in}{2.142347in}}%
\pgfpathcurveto{\pgfqpoint{1.677122in}{2.150583in}}{\pgfqpoint{1.673850in}{2.158483in}}{\pgfqpoint{1.668026in}{2.164307in}}%
\pgfpathcurveto{\pgfqpoint{1.662202in}{2.170131in}}{\pgfqpoint{1.654302in}{2.173403in}}{\pgfqpoint{1.646066in}{2.173403in}}%
\pgfpathcurveto{\pgfqpoint{1.637829in}{2.173403in}}{\pgfqpoint{1.629929in}{2.170131in}}{\pgfqpoint{1.624105in}{2.164307in}}%
\pgfpathcurveto{\pgfqpoint{1.618282in}{2.158483in}}{\pgfqpoint{1.615009in}{2.150583in}}{\pgfqpoint{1.615009in}{2.142347in}}%
\pgfpathcurveto{\pgfqpoint{1.615009in}{2.134110in}}{\pgfqpoint{1.618282in}{2.126210in}}{\pgfqpoint{1.624105in}{2.120386in}}%
\pgfpathcurveto{\pgfqpoint{1.629929in}{2.114563in}}{\pgfqpoint{1.637829in}{2.111290in}}{\pgfqpoint{1.646066in}{2.111290in}}%
\pgfpathclose%
\pgfusepath{stroke,fill}%
\end{pgfscope}%
\begin{pgfscope}%
\pgfpathrectangle{\pgfqpoint{0.100000in}{0.212622in}}{\pgfqpoint{3.696000in}{3.696000in}}%
\pgfusepath{clip}%
\pgfsetbuttcap%
\pgfsetroundjoin%
\definecolor{currentfill}{rgb}{0.121569,0.466667,0.705882}%
\pgfsetfillcolor{currentfill}%
\pgfsetfillopacity{0.300032}%
\pgfsetlinewidth{1.003750pt}%
\definecolor{currentstroke}{rgb}{0.121569,0.466667,0.705882}%
\pgfsetstrokecolor{currentstroke}%
\pgfsetstrokeopacity{0.300032}%
\pgfsetdash{}{0pt}%
\pgfpathmoveto{\pgfqpoint{1.646066in}{2.111290in}}%
\pgfpathcurveto{\pgfqpoint{1.654302in}{2.111290in}}{\pgfqpoint{1.662202in}{2.114563in}}{\pgfqpoint{1.668026in}{2.120386in}}%
\pgfpathcurveto{\pgfqpoint{1.673850in}{2.126210in}}{\pgfqpoint{1.677122in}{2.134110in}}{\pgfqpoint{1.677122in}{2.142347in}}%
\pgfpathcurveto{\pgfqpoint{1.677122in}{2.150583in}}{\pgfqpoint{1.673850in}{2.158483in}}{\pgfqpoint{1.668026in}{2.164307in}}%
\pgfpathcurveto{\pgfqpoint{1.662202in}{2.170131in}}{\pgfqpoint{1.654302in}{2.173403in}}{\pgfqpoint{1.646066in}{2.173403in}}%
\pgfpathcurveto{\pgfqpoint{1.637829in}{2.173403in}}{\pgfqpoint{1.629929in}{2.170131in}}{\pgfqpoint{1.624105in}{2.164307in}}%
\pgfpathcurveto{\pgfqpoint{1.618282in}{2.158483in}}{\pgfqpoint{1.615009in}{2.150583in}}{\pgfqpoint{1.615009in}{2.142347in}}%
\pgfpathcurveto{\pgfqpoint{1.615009in}{2.134110in}}{\pgfqpoint{1.618282in}{2.126210in}}{\pgfqpoint{1.624105in}{2.120386in}}%
\pgfpathcurveto{\pgfqpoint{1.629929in}{2.114563in}}{\pgfqpoint{1.637829in}{2.111290in}}{\pgfqpoint{1.646066in}{2.111290in}}%
\pgfpathclose%
\pgfusepath{stroke,fill}%
\end{pgfscope}%
\begin{pgfscope}%
\pgfpathrectangle{\pgfqpoint{0.100000in}{0.212622in}}{\pgfqpoint{3.696000in}{3.696000in}}%
\pgfusepath{clip}%
\pgfsetbuttcap%
\pgfsetroundjoin%
\definecolor{currentfill}{rgb}{0.121569,0.466667,0.705882}%
\pgfsetfillcolor{currentfill}%
\pgfsetfillopacity{0.300032}%
\pgfsetlinewidth{1.003750pt}%
\definecolor{currentstroke}{rgb}{0.121569,0.466667,0.705882}%
\pgfsetstrokecolor{currentstroke}%
\pgfsetstrokeopacity{0.300032}%
\pgfsetdash{}{0pt}%
\pgfpathmoveto{\pgfqpoint{1.646066in}{2.111290in}}%
\pgfpathcurveto{\pgfqpoint{1.654302in}{2.111290in}}{\pgfqpoint{1.662202in}{2.114563in}}{\pgfqpoint{1.668026in}{2.120386in}}%
\pgfpathcurveto{\pgfqpoint{1.673850in}{2.126210in}}{\pgfqpoint{1.677122in}{2.134110in}}{\pgfqpoint{1.677122in}{2.142347in}}%
\pgfpathcurveto{\pgfqpoint{1.677122in}{2.150583in}}{\pgfqpoint{1.673850in}{2.158483in}}{\pgfqpoint{1.668026in}{2.164307in}}%
\pgfpathcurveto{\pgfqpoint{1.662202in}{2.170131in}}{\pgfqpoint{1.654302in}{2.173403in}}{\pgfqpoint{1.646066in}{2.173403in}}%
\pgfpathcurveto{\pgfqpoint{1.637829in}{2.173403in}}{\pgfqpoint{1.629929in}{2.170131in}}{\pgfqpoint{1.624105in}{2.164307in}}%
\pgfpathcurveto{\pgfqpoint{1.618282in}{2.158483in}}{\pgfqpoint{1.615009in}{2.150583in}}{\pgfqpoint{1.615009in}{2.142347in}}%
\pgfpathcurveto{\pgfqpoint{1.615009in}{2.134110in}}{\pgfqpoint{1.618282in}{2.126210in}}{\pgfqpoint{1.624105in}{2.120386in}}%
\pgfpathcurveto{\pgfqpoint{1.629929in}{2.114563in}}{\pgfqpoint{1.637829in}{2.111290in}}{\pgfqpoint{1.646066in}{2.111290in}}%
\pgfpathclose%
\pgfusepath{stroke,fill}%
\end{pgfscope}%
\begin{pgfscope}%
\pgfpathrectangle{\pgfqpoint{0.100000in}{0.212622in}}{\pgfqpoint{3.696000in}{3.696000in}}%
\pgfusepath{clip}%
\pgfsetbuttcap%
\pgfsetroundjoin%
\definecolor{currentfill}{rgb}{0.121569,0.466667,0.705882}%
\pgfsetfillcolor{currentfill}%
\pgfsetfillopacity{0.300032}%
\pgfsetlinewidth{1.003750pt}%
\definecolor{currentstroke}{rgb}{0.121569,0.466667,0.705882}%
\pgfsetstrokecolor{currentstroke}%
\pgfsetstrokeopacity{0.300032}%
\pgfsetdash{}{0pt}%
\pgfpathmoveto{\pgfqpoint{1.646066in}{2.111290in}}%
\pgfpathcurveto{\pgfqpoint{1.654302in}{2.111290in}}{\pgfqpoint{1.662202in}{2.114563in}}{\pgfqpoint{1.668026in}{2.120386in}}%
\pgfpathcurveto{\pgfqpoint{1.673850in}{2.126210in}}{\pgfqpoint{1.677122in}{2.134110in}}{\pgfqpoint{1.677122in}{2.142347in}}%
\pgfpathcurveto{\pgfqpoint{1.677122in}{2.150583in}}{\pgfqpoint{1.673850in}{2.158483in}}{\pgfqpoint{1.668026in}{2.164307in}}%
\pgfpathcurveto{\pgfqpoint{1.662202in}{2.170131in}}{\pgfqpoint{1.654302in}{2.173403in}}{\pgfqpoint{1.646066in}{2.173403in}}%
\pgfpathcurveto{\pgfqpoint{1.637829in}{2.173403in}}{\pgfqpoint{1.629929in}{2.170131in}}{\pgfqpoint{1.624105in}{2.164307in}}%
\pgfpathcurveto{\pgfqpoint{1.618282in}{2.158483in}}{\pgfqpoint{1.615009in}{2.150583in}}{\pgfqpoint{1.615009in}{2.142347in}}%
\pgfpathcurveto{\pgfqpoint{1.615009in}{2.134110in}}{\pgfqpoint{1.618282in}{2.126210in}}{\pgfqpoint{1.624105in}{2.120386in}}%
\pgfpathcurveto{\pgfqpoint{1.629929in}{2.114563in}}{\pgfqpoint{1.637829in}{2.111290in}}{\pgfqpoint{1.646066in}{2.111290in}}%
\pgfpathclose%
\pgfusepath{stroke,fill}%
\end{pgfscope}%
\begin{pgfscope}%
\pgfpathrectangle{\pgfqpoint{0.100000in}{0.212622in}}{\pgfqpoint{3.696000in}{3.696000in}}%
\pgfusepath{clip}%
\pgfsetbuttcap%
\pgfsetroundjoin%
\definecolor{currentfill}{rgb}{0.121569,0.466667,0.705882}%
\pgfsetfillcolor{currentfill}%
\pgfsetfillopacity{0.300032}%
\pgfsetlinewidth{1.003750pt}%
\definecolor{currentstroke}{rgb}{0.121569,0.466667,0.705882}%
\pgfsetstrokecolor{currentstroke}%
\pgfsetstrokeopacity{0.300032}%
\pgfsetdash{}{0pt}%
\pgfpathmoveto{\pgfqpoint{1.646066in}{2.111290in}}%
\pgfpathcurveto{\pgfqpoint{1.654302in}{2.111290in}}{\pgfqpoint{1.662202in}{2.114563in}}{\pgfqpoint{1.668026in}{2.120386in}}%
\pgfpathcurveto{\pgfqpoint{1.673850in}{2.126210in}}{\pgfqpoint{1.677122in}{2.134110in}}{\pgfqpoint{1.677122in}{2.142347in}}%
\pgfpathcurveto{\pgfqpoint{1.677122in}{2.150583in}}{\pgfqpoint{1.673850in}{2.158483in}}{\pgfqpoint{1.668026in}{2.164307in}}%
\pgfpathcurveto{\pgfqpoint{1.662202in}{2.170131in}}{\pgfqpoint{1.654302in}{2.173403in}}{\pgfqpoint{1.646066in}{2.173403in}}%
\pgfpathcurveto{\pgfqpoint{1.637829in}{2.173403in}}{\pgfqpoint{1.629929in}{2.170131in}}{\pgfqpoint{1.624105in}{2.164307in}}%
\pgfpathcurveto{\pgfqpoint{1.618282in}{2.158483in}}{\pgfqpoint{1.615009in}{2.150583in}}{\pgfqpoint{1.615009in}{2.142347in}}%
\pgfpathcurveto{\pgfqpoint{1.615009in}{2.134110in}}{\pgfqpoint{1.618282in}{2.126210in}}{\pgfqpoint{1.624105in}{2.120386in}}%
\pgfpathcurveto{\pgfqpoint{1.629929in}{2.114563in}}{\pgfqpoint{1.637829in}{2.111290in}}{\pgfqpoint{1.646066in}{2.111290in}}%
\pgfpathclose%
\pgfusepath{stroke,fill}%
\end{pgfscope}%
\begin{pgfscope}%
\pgfpathrectangle{\pgfqpoint{0.100000in}{0.212622in}}{\pgfqpoint{3.696000in}{3.696000in}}%
\pgfusepath{clip}%
\pgfsetbuttcap%
\pgfsetroundjoin%
\definecolor{currentfill}{rgb}{0.121569,0.466667,0.705882}%
\pgfsetfillcolor{currentfill}%
\pgfsetfillopacity{0.300032}%
\pgfsetlinewidth{1.003750pt}%
\definecolor{currentstroke}{rgb}{0.121569,0.466667,0.705882}%
\pgfsetstrokecolor{currentstroke}%
\pgfsetstrokeopacity{0.300032}%
\pgfsetdash{}{0pt}%
\pgfpathmoveto{\pgfqpoint{1.646066in}{2.111290in}}%
\pgfpathcurveto{\pgfqpoint{1.654302in}{2.111290in}}{\pgfqpoint{1.662202in}{2.114563in}}{\pgfqpoint{1.668026in}{2.120386in}}%
\pgfpathcurveto{\pgfqpoint{1.673850in}{2.126210in}}{\pgfqpoint{1.677122in}{2.134110in}}{\pgfqpoint{1.677122in}{2.142347in}}%
\pgfpathcurveto{\pgfqpoint{1.677122in}{2.150583in}}{\pgfqpoint{1.673850in}{2.158483in}}{\pgfqpoint{1.668026in}{2.164307in}}%
\pgfpathcurveto{\pgfqpoint{1.662202in}{2.170131in}}{\pgfqpoint{1.654302in}{2.173403in}}{\pgfqpoint{1.646066in}{2.173403in}}%
\pgfpathcurveto{\pgfqpoint{1.637829in}{2.173403in}}{\pgfqpoint{1.629929in}{2.170131in}}{\pgfqpoint{1.624105in}{2.164307in}}%
\pgfpathcurveto{\pgfqpoint{1.618282in}{2.158483in}}{\pgfqpoint{1.615009in}{2.150583in}}{\pgfqpoint{1.615009in}{2.142347in}}%
\pgfpathcurveto{\pgfqpoint{1.615009in}{2.134110in}}{\pgfqpoint{1.618282in}{2.126210in}}{\pgfqpoint{1.624105in}{2.120386in}}%
\pgfpathcurveto{\pgfqpoint{1.629929in}{2.114563in}}{\pgfqpoint{1.637829in}{2.111290in}}{\pgfqpoint{1.646066in}{2.111290in}}%
\pgfpathclose%
\pgfusepath{stroke,fill}%
\end{pgfscope}%
\begin{pgfscope}%
\pgfpathrectangle{\pgfqpoint{0.100000in}{0.212622in}}{\pgfqpoint{3.696000in}{3.696000in}}%
\pgfusepath{clip}%
\pgfsetbuttcap%
\pgfsetroundjoin%
\definecolor{currentfill}{rgb}{0.121569,0.466667,0.705882}%
\pgfsetfillcolor{currentfill}%
\pgfsetfillopacity{0.300032}%
\pgfsetlinewidth{1.003750pt}%
\definecolor{currentstroke}{rgb}{0.121569,0.466667,0.705882}%
\pgfsetstrokecolor{currentstroke}%
\pgfsetstrokeopacity{0.300032}%
\pgfsetdash{}{0pt}%
\pgfpathmoveto{\pgfqpoint{1.646066in}{2.111290in}}%
\pgfpathcurveto{\pgfqpoint{1.654302in}{2.111290in}}{\pgfqpoint{1.662202in}{2.114563in}}{\pgfqpoint{1.668026in}{2.120386in}}%
\pgfpathcurveto{\pgfqpoint{1.673850in}{2.126210in}}{\pgfqpoint{1.677122in}{2.134110in}}{\pgfqpoint{1.677122in}{2.142347in}}%
\pgfpathcurveto{\pgfqpoint{1.677122in}{2.150583in}}{\pgfqpoint{1.673850in}{2.158483in}}{\pgfqpoint{1.668026in}{2.164307in}}%
\pgfpathcurveto{\pgfqpoint{1.662202in}{2.170131in}}{\pgfqpoint{1.654302in}{2.173403in}}{\pgfqpoint{1.646066in}{2.173403in}}%
\pgfpathcurveto{\pgfqpoint{1.637829in}{2.173403in}}{\pgfqpoint{1.629929in}{2.170131in}}{\pgfqpoint{1.624105in}{2.164307in}}%
\pgfpathcurveto{\pgfqpoint{1.618282in}{2.158483in}}{\pgfqpoint{1.615009in}{2.150583in}}{\pgfqpoint{1.615009in}{2.142347in}}%
\pgfpathcurveto{\pgfqpoint{1.615009in}{2.134110in}}{\pgfqpoint{1.618282in}{2.126210in}}{\pgfqpoint{1.624105in}{2.120386in}}%
\pgfpathcurveto{\pgfqpoint{1.629929in}{2.114563in}}{\pgfqpoint{1.637829in}{2.111290in}}{\pgfqpoint{1.646066in}{2.111290in}}%
\pgfpathclose%
\pgfusepath{stroke,fill}%
\end{pgfscope}%
\begin{pgfscope}%
\pgfpathrectangle{\pgfqpoint{0.100000in}{0.212622in}}{\pgfqpoint{3.696000in}{3.696000in}}%
\pgfusepath{clip}%
\pgfsetbuttcap%
\pgfsetroundjoin%
\definecolor{currentfill}{rgb}{0.121569,0.466667,0.705882}%
\pgfsetfillcolor{currentfill}%
\pgfsetfillopacity{0.300032}%
\pgfsetlinewidth{1.003750pt}%
\definecolor{currentstroke}{rgb}{0.121569,0.466667,0.705882}%
\pgfsetstrokecolor{currentstroke}%
\pgfsetstrokeopacity{0.300032}%
\pgfsetdash{}{0pt}%
\pgfpathmoveto{\pgfqpoint{1.646066in}{2.111290in}}%
\pgfpathcurveto{\pgfqpoint{1.654302in}{2.111290in}}{\pgfqpoint{1.662202in}{2.114563in}}{\pgfqpoint{1.668026in}{2.120386in}}%
\pgfpathcurveto{\pgfqpoint{1.673850in}{2.126210in}}{\pgfqpoint{1.677122in}{2.134110in}}{\pgfqpoint{1.677122in}{2.142347in}}%
\pgfpathcurveto{\pgfqpoint{1.677122in}{2.150583in}}{\pgfqpoint{1.673850in}{2.158483in}}{\pgfqpoint{1.668026in}{2.164307in}}%
\pgfpathcurveto{\pgfqpoint{1.662202in}{2.170131in}}{\pgfqpoint{1.654302in}{2.173403in}}{\pgfqpoint{1.646066in}{2.173403in}}%
\pgfpathcurveto{\pgfqpoint{1.637829in}{2.173403in}}{\pgfqpoint{1.629929in}{2.170131in}}{\pgfqpoint{1.624105in}{2.164307in}}%
\pgfpathcurveto{\pgfqpoint{1.618282in}{2.158483in}}{\pgfqpoint{1.615009in}{2.150583in}}{\pgfqpoint{1.615009in}{2.142347in}}%
\pgfpathcurveto{\pgfqpoint{1.615009in}{2.134110in}}{\pgfqpoint{1.618282in}{2.126210in}}{\pgfqpoint{1.624105in}{2.120386in}}%
\pgfpathcurveto{\pgfqpoint{1.629929in}{2.114563in}}{\pgfqpoint{1.637829in}{2.111290in}}{\pgfqpoint{1.646066in}{2.111290in}}%
\pgfpathclose%
\pgfusepath{stroke,fill}%
\end{pgfscope}%
\begin{pgfscope}%
\pgfpathrectangle{\pgfqpoint{0.100000in}{0.212622in}}{\pgfqpoint{3.696000in}{3.696000in}}%
\pgfusepath{clip}%
\pgfsetbuttcap%
\pgfsetroundjoin%
\definecolor{currentfill}{rgb}{0.121569,0.466667,0.705882}%
\pgfsetfillcolor{currentfill}%
\pgfsetfillopacity{0.300032}%
\pgfsetlinewidth{1.003750pt}%
\definecolor{currentstroke}{rgb}{0.121569,0.466667,0.705882}%
\pgfsetstrokecolor{currentstroke}%
\pgfsetstrokeopacity{0.300032}%
\pgfsetdash{}{0pt}%
\pgfpathmoveto{\pgfqpoint{1.646066in}{2.111290in}}%
\pgfpathcurveto{\pgfqpoint{1.654302in}{2.111290in}}{\pgfqpoint{1.662202in}{2.114563in}}{\pgfqpoint{1.668026in}{2.120386in}}%
\pgfpathcurveto{\pgfqpoint{1.673850in}{2.126210in}}{\pgfqpoint{1.677122in}{2.134110in}}{\pgfqpoint{1.677122in}{2.142347in}}%
\pgfpathcurveto{\pgfqpoint{1.677122in}{2.150583in}}{\pgfqpoint{1.673850in}{2.158483in}}{\pgfqpoint{1.668026in}{2.164307in}}%
\pgfpathcurveto{\pgfqpoint{1.662202in}{2.170131in}}{\pgfqpoint{1.654302in}{2.173403in}}{\pgfqpoint{1.646066in}{2.173403in}}%
\pgfpathcurveto{\pgfqpoint{1.637829in}{2.173403in}}{\pgfqpoint{1.629929in}{2.170131in}}{\pgfqpoint{1.624105in}{2.164307in}}%
\pgfpathcurveto{\pgfqpoint{1.618282in}{2.158483in}}{\pgfqpoint{1.615009in}{2.150583in}}{\pgfqpoint{1.615009in}{2.142347in}}%
\pgfpathcurveto{\pgfqpoint{1.615009in}{2.134110in}}{\pgfqpoint{1.618282in}{2.126210in}}{\pgfqpoint{1.624105in}{2.120386in}}%
\pgfpathcurveto{\pgfqpoint{1.629929in}{2.114563in}}{\pgfqpoint{1.637829in}{2.111290in}}{\pgfqpoint{1.646066in}{2.111290in}}%
\pgfpathclose%
\pgfusepath{stroke,fill}%
\end{pgfscope}%
\begin{pgfscope}%
\pgfpathrectangle{\pgfqpoint{0.100000in}{0.212622in}}{\pgfqpoint{3.696000in}{3.696000in}}%
\pgfusepath{clip}%
\pgfsetbuttcap%
\pgfsetroundjoin%
\definecolor{currentfill}{rgb}{0.121569,0.466667,0.705882}%
\pgfsetfillcolor{currentfill}%
\pgfsetfillopacity{0.300032}%
\pgfsetlinewidth{1.003750pt}%
\definecolor{currentstroke}{rgb}{0.121569,0.466667,0.705882}%
\pgfsetstrokecolor{currentstroke}%
\pgfsetstrokeopacity{0.300032}%
\pgfsetdash{}{0pt}%
\pgfpathmoveto{\pgfqpoint{1.646066in}{2.111290in}}%
\pgfpathcurveto{\pgfqpoint{1.654302in}{2.111290in}}{\pgfqpoint{1.662202in}{2.114563in}}{\pgfqpoint{1.668026in}{2.120386in}}%
\pgfpathcurveto{\pgfqpoint{1.673850in}{2.126210in}}{\pgfqpoint{1.677122in}{2.134110in}}{\pgfqpoint{1.677122in}{2.142347in}}%
\pgfpathcurveto{\pgfqpoint{1.677122in}{2.150583in}}{\pgfqpoint{1.673850in}{2.158483in}}{\pgfqpoint{1.668026in}{2.164307in}}%
\pgfpathcurveto{\pgfqpoint{1.662202in}{2.170131in}}{\pgfqpoint{1.654302in}{2.173403in}}{\pgfqpoint{1.646066in}{2.173403in}}%
\pgfpathcurveto{\pgfqpoint{1.637829in}{2.173403in}}{\pgfqpoint{1.629929in}{2.170131in}}{\pgfqpoint{1.624105in}{2.164307in}}%
\pgfpathcurveto{\pgfqpoint{1.618282in}{2.158483in}}{\pgfqpoint{1.615009in}{2.150583in}}{\pgfqpoint{1.615009in}{2.142347in}}%
\pgfpathcurveto{\pgfqpoint{1.615009in}{2.134110in}}{\pgfqpoint{1.618282in}{2.126210in}}{\pgfqpoint{1.624105in}{2.120386in}}%
\pgfpathcurveto{\pgfqpoint{1.629929in}{2.114563in}}{\pgfqpoint{1.637829in}{2.111290in}}{\pgfqpoint{1.646066in}{2.111290in}}%
\pgfpathclose%
\pgfusepath{stroke,fill}%
\end{pgfscope}%
\begin{pgfscope}%
\pgfpathrectangle{\pgfqpoint{0.100000in}{0.212622in}}{\pgfqpoint{3.696000in}{3.696000in}}%
\pgfusepath{clip}%
\pgfsetbuttcap%
\pgfsetroundjoin%
\definecolor{currentfill}{rgb}{0.121569,0.466667,0.705882}%
\pgfsetfillcolor{currentfill}%
\pgfsetfillopacity{0.300032}%
\pgfsetlinewidth{1.003750pt}%
\definecolor{currentstroke}{rgb}{0.121569,0.466667,0.705882}%
\pgfsetstrokecolor{currentstroke}%
\pgfsetstrokeopacity{0.300032}%
\pgfsetdash{}{0pt}%
\pgfpathmoveto{\pgfqpoint{1.646066in}{2.111290in}}%
\pgfpathcurveto{\pgfqpoint{1.654302in}{2.111290in}}{\pgfqpoint{1.662202in}{2.114563in}}{\pgfqpoint{1.668026in}{2.120386in}}%
\pgfpathcurveto{\pgfqpoint{1.673850in}{2.126210in}}{\pgfqpoint{1.677122in}{2.134110in}}{\pgfqpoint{1.677122in}{2.142347in}}%
\pgfpathcurveto{\pgfqpoint{1.677122in}{2.150583in}}{\pgfqpoint{1.673850in}{2.158483in}}{\pgfqpoint{1.668026in}{2.164307in}}%
\pgfpathcurveto{\pgfqpoint{1.662202in}{2.170131in}}{\pgfqpoint{1.654302in}{2.173403in}}{\pgfqpoint{1.646066in}{2.173403in}}%
\pgfpathcurveto{\pgfqpoint{1.637829in}{2.173403in}}{\pgfqpoint{1.629929in}{2.170131in}}{\pgfqpoint{1.624105in}{2.164307in}}%
\pgfpathcurveto{\pgfqpoint{1.618282in}{2.158483in}}{\pgfqpoint{1.615009in}{2.150583in}}{\pgfqpoint{1.615009in}{2.142347in}}%
\pgfpathcurveto{\pgfqpoint{1.615009in}{2.134110in}}{\pgfqpoint{1.618282in}{2.126210in}}{\pgfqpoint{1.624105in}{2.120386in}}%
\pgfpathcurveto{\pgfqpoint{1.629929in}{2.114563in}}{\pgfqpoint{1.637829in}{2.111290in}}{\pgfqpoint{1.646066in}{2.111290in}}%
\pgfpathclose%
\pgfusepath{stroke,fill}%
\end{pgfscope}%
\begin{pgfscope}%
\pgfpathrectangle{\pgfqpoint{0.100000in}{0.212622in}}{\pgfqpoint{3.696000in}{3.696000in}}%
\pgfusepath{clip}%
\pgfsetbuttcap%
\pgfsetroundjoin%
\definecolor{currentfill}{rgb}{0.121569,0.466667,0.705882}%
\pgfsetfillcolor{currentfill}%
\pgfsetfillopacity{0.300032}%
\pgfsetlinewidth{1.003750pt}%
\definecolor{currentstroke}{rgb}{0.121569,0.466667,0.705882}%
\pgfsetstrokecolor{currentstroke}%
\pgfsetstrokeopacity{0.300032}%
\pgfsetdash{}{0pt}%
\pgfpathmoveto{\pgfqpoint{1.646066in}{2.111290in}}%
\pgfpathcurveto{\pgfqpoint{1.654302in}{2.111290in}}{\pgfqpoint{1.662202in}{2.114563in}}{\pgfqpoint{1.668026in}{2.120386in}}%
\pgfpathcurveto{\pgfqpoint{1.673850in}{2.126210in}}{\pgfqpoint{1.677122in}{2.134110in}}{\pgfqpoint{1.677122in}{2.142347in}}%
\pgfpathcurveto{\pgfqpoint{1.677122in}{2.150583in}}{\pgfqpoint{1.673850in}{2.158483in}}{\pgfqpoint{1.668026in}{2.164307in}}%
\pgfpathcurveto{\pgfqpoint{1.662202in}{2.170131in}}{\pgfqpoint{1.654302in}{2.173403in}}{\pgfqpoint{1.646066in}{2.173403in}}%
\pgfpathcurveto{\pgfqpoint{1.637829in}{2.173403in}}{\pgfqpoint{1.629929in}{2.170131in}}{\pgfqpoint{1.624105in}{2.164307in}}%
\pgfpathcurveto{\pgfqpoint{1.618282in}{2.158483in}}{\pgfqpoint{1.615009in}{2.150583in}}{\pgfqpoint{1.615009in}{2.142347in}}%
\pgfpathcurveto{\pgfqpoint{1.615009in}{2.134110in}}{\pgfqpoint{1.618282in}{2.126210in}}{\pgfqpoint{1.624105in}{2.120386in}}%
\pgfpathcurveto{\pgfqpoint{1.629929in}{2.114563in}}{\pgfqpoint{1.637829in}{2.111290in}}{\pgfqpoint{1.646066in}{2.111290in}}%
\pgfpathclose%
\pgfusepath{stroke,fill}%
\end{pgfscope}%
\begin{pgfscope}%
\pgfpathrectangle{\pgfqpoint{0.100000in}{0.212622in}}{\pgfqpoint{3.696000in}{3.696000in}}%
\pgfusepath{clip}%
\pgfsetbuttcap%
\pgfsetroundjoin%
\definecolor{currentfill}{rgb}{0.121569,0.466667,0.705882}%
\pgfsetfillcolor{currentfill}%
\pgfsetfillopacity{0.300032}%
\pgfsetlinewidth{1.003750pt}%
\definecolor{currentstroke}{rgb}{0.121569,0.466667,0.705882}%
\pgfsetstrokecolor{currentstroke}%
\pgfsetstrokeopacity{0.300032}%
\pgfsetdash{}{0pt}%
\pgfpathmoveto{\pgfqpoint{1.646066in}{2.111290in}}%
\pgfpathcurveto{\pgfqpoint{1.654302in}{2.111290in}}{\pgfqpoint{1.662202in}{2.114563in}}{\pgfqpoint{1.668026in}{2.120386in}}%
\pgfpathcurveto{\pgfqpoint{1.673850in}{2.126210in}}{\pgfqpoint{1.677122in}{2.134110in}}{\pgfqpoint{1.677122in}{2.142347in}}%
\pgfpathcurveto{\pgfqpoint{1.677122in}{2.150583in}}{\pgfqpoint{1.673850in}{2.158483in}}{\pgfqpoint{1.668026in}{2.164307in}}%
\pgfpathcurveto{\pgfqpoint{1.662202in}{2.170131in}}{\pgfqpoint{1.654302in}{2.173403in}}{\pgfqpoint{1.646066in}{2.173403in}}%
\pgfpathcurveto{\pgfqpoint{1.637829in}{2.173403in}}{\pgfqpoint{1.629929in}{2.170131in}}{\pgfqpoint{1.624105in}{2.164307in}}%
\pgfpathcurveto{\pgfqpoint{1.618282in}{2.158483in}}{\pgfqpoint{1.615009in}{2.150583in}}{\pgfqpoint{1.615009in}{2.142347in}}%
\pgfpathcurveto{\pgfqpoint{1.615009in}{2.134110in}}{\pgfqpoint{1.618282in}{2.126210in}}{\pgfqpoint{1.624105in}{2.120386in}}%
\pgfpathcurveto{\pgfqpoint{1.629929in}{2.114563in}}{\pgfqpoint{1.637829in}{2.111290in}}{\pgfqpoint{1.646066in}{2.111290in}}%
\pgfpathclose%
\pgfusepath{stroke,fill}%
\end{pgfscope}%
\begin{pgfscope}%
\pgfpathrectangle{\pgfqpoint{0.100000in}{0.212622in}}{\pgfqpoint{3.696000in}{3.696000in}}%
\pgfusepath{clip}%
\pgfsetbuttcap%
\pgfsetroundjoin%
\definecolor{currentfill}{rgb}{0.121569,0.466667,0.705882}%
\pgfsetfillcolor{currentfill}%
\pgfsetfillopacity{0.300032}%
\pgfsetlinewidth{1.003750pt}%
\definecolor{currentstroke}{rgb}{0.121569,0.466667,0.705882}%
\pgfsetstrokecolor{currentstroke}%
\pgfsetstrokeopacity{0.300032}%
\pgfsetdash{}{0pt}%
\pgfpathmoveto{\pgfqpoint{1.646066in}{2.111290in}}%
\pgfpathcurveto{\pgfqpoint{1.654302in}{2.111290in}}{\pgfqpoint{1.662202in}{2.114563in}}{\pgfqpoint{1.668026in}{2.120386in}}%
\pgfpathcurveto{\pgfqpoint{1.673850in}{2.126210in}}{\pgfqpoint{1.677122in}{2.134110in}}{\pgfqpoint{1.677122in}{2.142347in}}%
\pgfpathcurveto{\pgfqpoint{1.677122in}{2.150583in}}{\pgfqpoint{1.673850in}{2.158483in}}{\pgfqpoint{1.668026in}{2.164307in}}%
\pgfpathcurveto{\pgfqpoint{1.662202in}{2.170131in}}{\pgfqpoint{1.654302in}{2.173403in}}{\pgfqpoint{1.646066in}{2.173403in}}%
\pgfpathcurveto{\pgfqpoint{1.637829in}{2.173403in}}{\pgfqpoint{1.629929in}{2.170131in}}{\pgfqpoint{1.624105in}{2.164307in}}%
\pgfpathcurveto{\pgfqpoint{1.618282in}{2.158483in}}{\pgfqpoint{1.615009in}{2.150583in}}{\pgfqpoint{1.615009in}{2.142347in}}%
\pgfpathcurveto{\pgfqpoint{1.615009in}{2.134110in}}{\pgfqpoint{1.618282in}{2.126210in}}{\pgfqpoint{1.624105in}{2.120386in}}%
\pgfpathcurveto{\pgfqpoint{1.629929in}{2.114563in}}{\pgfqpoint{1.637829in}{2.111290in}}{\pgfqpoint{1.646066in}{2.111290in}}%
\pgfpathclose%
\pgfusepath{stroke,fill}%
\end{pgfscope}%
\begin{pgfscope}%
\pgfpathrectangle{\pgfqpoint{0.100000in}{0.212622in}}{\pgfqpoint{3.696000in}{3.696000in}}%
\pgfusepath{clip}%
\pgfsetbuttcap%
\pgfsetroundjoin%
\definecolor{currentfill}{rgb}{0.121569,0.466667,0.705882}%
\pgfsetfillcolor{currentfill}%
\pgfsetfillopacity{0.300651}%
\pgfsetlinewidth{1.003750pt}%
\definecolor{currentstroke}{rgb}{0.121569,0.466667,0.705882}%
\pgfsetstrokecolor{currentstroke}%
\pgfsetstrokeopacity{0.300651}%
\pgfsetdash{}{0pt}%
\pgfpathmoveto{\pgfqpoint{1.647814in}{2.112464in}}%
\pgfpathcurveto{\pgfqpoint{1.656050in}{2.112464in}}{\pgfqpoint{1.663950in}{2.115736in}}{\pgfqpoint{1.669774in}{2.121560in}}%
\pgfpathcurveto{\pgfqpoint{1.675598in}{2.127384in}}{\pgfqpoint{1.678871in}{2.135284in}}{\pgfqpoint{1.678871in}{2.143520in}}%
\pgfpathcurveto{\pgfqpoint{1.678871in}{2.151757in}}{\pgfqpoint{1.675598in}{2.159657in}}{\pgfqpoint{1.669774in}{2.165481in}}%
\pgfpathcurveto{\pgfqpoint{1.663950in}{2.171305in}}{\pgfqpoint{1.656050in}{2.174577in}}{\pgfqpoint{1.647814in}{2.174577in}}%
\pgfpathcurveto{\pgfqpoint{1.639578in}{2.174577in}}{\pgfqpoint{1.631678in}{2.171305in}}{\pgfqpoint{1.625854in}{2.165481in}}%
\pgfpathcurveto{\pgfqpoint{1.620030in}{2.159657in}}{\pgfqpoint{1.616758in}{2.151757in}}{\pgfqpoint{1.616758in}{2.143520in}}%
\pgfpathcurveto{\pgfqpoint{1.616758in}{2.135284in}}{\pgfqpoint{1.620030in}{2.127384in}}{\pgfqpoint{1.625854in}{2.121560in}}%
\pgfpathcurveto{\pgfqpoint{1.631678in}{2.115736in}}{\pgfqpoint{1.639578in}{2.112464in}}{\pgfqpoint{1.647814in}{2.112464in}}%
\pgfpathclose%
\pgfusepath{stroke,fill}%
\end{pgfscope}%
\begin{pgfscope}%
\pgfpathrectangle{\pgfqpoint{0.100000in}{0.212622in}}{\pgfqpoint{3.696000in}{3.696000in}}%
\pgfusepath{clip}%
\pgfsetbuttcap%
\pgfsetroundjoin%
\definecolor{currentfill}{rgb}{0.121569,0.466667,0.705882}%
\pgfsetfillcolor{currentfill}%
\pgfsetfillopacity{0.300651}%
\pgfsetlinewidth{1.003750pt}%
\definecolor{currentstroke}{rgb}{0.121569,0.466667,0.705882}%
\pgfsetstrokecolor{currentstroke}%
\pgfsetstrokeopacity{0.300651}%
\pgfsetdash{}{0pt}%
\pgfpathmoveto{\pgfqpoint{1.647814in}{2.112464in}}%
\pgfpathcurveto{\pgfqpoint{1.656050in}{2.112464in}}{\pgfqpoint{1.663950in}{2.115736in}}{\pgfqpoint{1.669774in}{2.121560in}}%
\pgfpathcurveto{\pgfqpoint{1.675598in}{2.127384in}}{\pgfqpoint{1.678871in}{2.135284in}}{\pgfqpoint{1.678871in}{2.143520in}}%
\pgfpathcurveto{\pgfqpoint{1.678871in}{2.151757in}}{\pgfqpoint{1.675598in}{2.159657in}}{\pgfqpoint{1.669774in}{2.165481in}}%
\pgfpathcurveto{\pgfqpoint{1.663950in}{2.171305in}}{\pgfqpoint{1.656050in}{2.174577in}}{\pgfqpoint{1.647814in}{2.174577in}}%
\pgfpathcurveto{\pgfqpoint{1.639578in}{2.174577in}}{\pgfqpoint{1.631678in}{2.171305in}}{\pgfqpoint{1.625854in}{2.165481in}}%
\pgfpathcurveto{\pgfqpoint{1.620030in}{2.159657in}}{\pgfqpoint{1.616758in}{2.151757in}}{\pgfqpoint{1.616758in}{2.143520in}}%
\pgfpathcurveto{\pgfqpoint{1.616758in}{2.135284in}}{\pgfqpoint{1.620030in}{2.127384in}}{\pgfqpoint{1.625854in}{2.121560in}}%
\pgfpathcurveto{\pgfqpoint{1.631678in}{2.115736in}}{\pgfqpoint{1.639578in}{2.112464in}}{\pgfqpoint{1.647814in}{2.112464in}}%
\pgfpathclose%
\pgfusepath{stroke,fill}%
\end{pgfscope}%
\begin{pgfscope}%
\pgfpathrectangle{\pgfqpoint{0.100000in}{0.212622in}}{\pgfqpoint{3.696000in}{3.696000in}}%
\pgfusepath{clip}%
\pgfsetbuttcap%
\pgfsetroundjoin%
\definecolor{currentfill}{rgb}{0.121569,0.466667,0.705882}%
\pgfsetfillcolor{currentfill}%
\pgfsetfillopacity{0.300651}%
\pgfsetlinewidth{1.003750pt}%
\definecolor{currentstroke}{rgb}{0.121569,0.466667,0.705882}%
\pgfsetstrokecolor{currentstroke}%
\pgfsetstrokeopacity{0.300651}%
\pgfsetdash{}{0pt}%
\pgfpathmoveto{\pgfqpoint{1.647814in}{2.112464in}}%
\pgfpathcurveto{\pgfqpoint{1.656050in}{2.112464in}}{\pgfqpoint{1.663950in}{2.115736in}}{\pgfqpoint{1.669774in}{2.121560in}}%
\pgfpathcurveto{\pgfqpoint{1.675598in}{2.127384in}}{\pgfqpoint{1.678871in}{2.135284in}}{\pgfqpoint{1.678871in}{2.143520in}}%
\pgfpathcurveto{\pgfqpoint{1.678871in}{2.151757in}}{\pgfqpoint{1.675598in}{2.159657in}}{\pgfqpoint{1.669774in}{2.165481in}}%
\pgfpathcurveto{\pgfqpoint{1.663950in}{2.171305in}}{\pgfqpoint{1.656050in}{2.174577in}}{\pgfqpoint{1.647814in}{2.174577in}}%
\pgfpathcurveto{\pgfqpoint{1.639578in}{2.174577in}}{\pgfqpoint{1.631678in}{2.171305in}}{\pgfqpoint{1.625854in}{2.165481in}}%
\pgfpathcurveto{\pgfqpoint{1.620030in}{2.159657in}}{\pgfqpoint{1.616758in}{2.151757in}}{\pgfqpoint{1.616758in}{2.143520in}}%
\pgfpathcurveto{\pgfqpoint{1.616758in}{2.135284in}}{\pgfqpoint{1.620030in}{2.127384in}}{\pgfqpoint{1.625854in}{2.121560in}}%
\pgfpathcurveto{\pgfqpoint{1.631678in}{2.115736in}}{\pgfqpoint{1.639578in}{2.112464in}}{\pgfqpoint{1.647814in}{2.112464in}}%
\pgfpathclose%
\pgfusepath{stroke,fill}%
\end{pgfscope}%
\begin{pgfscope}%
\pgfpathrectangle{\pgfqpoint{0.100000in}{0.212622in}}{\pgfqpoint{3.696000in}{3.696000in}}%
\pgfusepath{clip}%
\pgfsetbuttcap%
\pgfsetroundjoin%
\definecolor{currentfill}{rgb}{0.121569,0.466667,0.705882}%
\pgfsetfillcolor{currentfill}%
\pgfsetfillopacity{0.300651}%
\pgfsetlinewidth{1.003750pt}%
\definecolor{currentstroke}{rgb}{0.121569,0.466667,0.705882}%
\pgfsetstrokecolor{currentstroke}%
\pgfsetstrokeopacity{0.300651}%
\pgfsetdash{}{0pt}%
\pgfpathmoveto{\pgfqpoint{1.647814in}{2.112464in}}%
\pgfpathcurveto{\pgfqpoint{1.656050in}{2.112464in}}{\pgfqpoint{1.663950in}{2.115736in}}{\pgfqpoint{1.669774in}{2.121560in}}%
\pgfpathcurveto{\pgfqpoint{1.675598in}{2.127384in}}{\pgfqpoint{1.678871in}{2.135284in}}{\pgfqpoint{1.678871in}{2.143520in}}%
\pgfpathcurveto{\pgfqpoint{1.678871in}{2.151757in}}{\pgfqpoint{1.675598in}{2.159657in}}{\pgfqpoint{1.669774in}{2.165481in}}%
\pgfpathcurveto{\pgfqpoint{1.663950in}{2.171305in}}{\pgfqpoint{1.656050in}{2.174577in}}{\pgfqpoint{1.647814in}{2.174577in}}%
\pgfpathcurveto{\pgfqpoint{1.639578in}{2.174577in}}{\pgfqpoint{1.631678in}{2.171305in}}{\pgfqpoint{1.625854in}{2.165481in}}%
\pgfpathcurveto{\pgfqpoint{1.620030in}{2.159657in}}{\pgfqpoint{1.616758in}{2.151757in}}{\pgfqpoint{1.616758in}{2.143520in}}%
\pgfpathcurveto{\pgfqpoint{1.616758in}{2.135284in}}{\pgfqpoint{1.620030in}{2.127384in}}{\pgfqpoint{1.625854in}{2.121560in}}%
\pgfpathcurveto{\pgfqpoint{1.631678in}{2.115736in}}{\pgfqpoint{1.639578in}{2.112464in}}{\pgfqpoint{1.647814in}{2.112464in}}%
\pgfpathclose%
\pgfusepath{stroke,fill}%
\end{pgfscope}%
\begin{pgfscope}%
\pgfpathrectangle{\pgfqpoint{0.100000in}{0.212622in}}{\pgfqpoint{3.696000in}{3.696000in}}%
\pgfusepath{clip}%
\pgfsetbuttcap%
\pgfsetroundjoin%
\definecolor{currentfill}{rgb}{0.121569,0.466667,0.705882}%
\pgfsetfillcolor{currentfill}%
\pgfsetfillopacity{0.300651}%
\pgfsetlinewidth{1.003750pt}%
\definecolor{currentstroke}{rgb}{0.121569,0.466667,0.705882}%
\pgfsetstrokecolor{currentstroke}%
\pgfsetstrokeopacity{0.300651}%
\pgfsetdash{}{0pt}%
\pgfpathmoveto{\pgfqpoint{1.647814in}{2.112464in}}%
\pgfpathcurveto{\pgfqpoint{1.656050in}{2.112464in}}{\pgfqpoint{1.663950in}{2.115736in}}{\pgfqpoint{1.669774in}{2.121560in}}%
\pgfpathcurveto{\pgfqpoint{1.675598in}{2.127384in}}{\pgfqpoint{1.678871in}{2.135284in}}{\pgfqpoint{1.678871in}{2.143520in}}%
\pgfpathcurveto{\pgfqpoint{1.678871in}{2.151757in}}{\pgfqpoint{1.675598in}{2.159657in}}{\pgfqpoint{1.669774in}{2.165481in}}%
\pgfpathcurveto{\pgfqpoint{1.663950in}{2.171305in}}{\pgfqpoint{1.656050in}{2.174577in}}{\pgfqpoint{1.647814in}{2.174577in}}%
\pgfpathcurveto{\pgfqpoint{1.639578in}{2.174577in}}{\pgfqpoint{1.631678in}{2.171305in}}{\pgfqpoint{1.625854in}{2.165481in}}%
\pgfpathcurveto{\pgfqpoint{1.620030in}{2.159657in}}{\pgfqpoint{1.616758in}{2.151757in}}{\pgfqpoint{1.616758in}{2.143520in}}%
\pgfpathcurveto{\pgfqpoint{1.616758in}{2.135284in}}{\pgfqpoint{1.620030in}{2.127384in}}{\pgfqpoint{1.625854in}{2.121560in}}%
\pgfpathcurveto{\pgfqpoint{1.631678in}{2.115736in}}{\pgfqpoint{1.639578in}{2.112464in}}{\pgfqpoint{1.647814in}{2.112464in}}%
\pgfpathclose%
\pgfusepath{stroke,fill}%
\end{pgfscope}%
\begin{pgfscope}%
\pgfpathrectangle{\pgfqpoint{0.100000in}{0.212622in}}{\pgfqpoint{3.696000in}{3.696000in}}%
\pgfusepath{clip}%
\pgfsetbuttcap%
\pgfsetroundjoin%
\definecolor{currentfill}{rgb}{0.121569,0.466667,0.705882}%
\pgfsetfillcolor{currentfill}%
\pgfsetfillopacity{0.300651}%
\pgfsetlinewidth{1.003750pt}%
\definecolor{currentstroke}{rgb}{0.121569,0.466667,0.705882}%
\pgfsetstrokecolor{currentstroke}%
\pgfsetstrokeopacity{0.300651}%
\pgfsetdash{}{0pt}%
\pgfpathmoveto{\pgfqpoint{1.647814in}{2.112464in}}%
\pgfpathcurveto{\pgfqpoint{1.656050in}{2.112464in}}{\pgfqpoint{1.663950in}{2.115736in}}{\pgfqpoint{1.669774in}{2.121560in}}%
\pgfpathcurveto{\pgfqpoint{1.675598in}{2.127384in}}{\pgfqpoint{1.678871in}{2.135284in}}{\pgfqpoint{1.678871in}{2.143520in}}%
\pgfpathcurveto{\pgfqpoint{1.678871in}{2.151757in}}{\pgfqpoint{1.675598in}{2.159657in}}{\pgfqpoint{1.669774in}{2.165481in}}%
\pgfpathcurveto{\pgfqpoint{1.663950in}{2.171305in}}{\pgfqpoint{1.656050in}{2.174577in}}{\pgfqpoint{1.647814in}{2.174577in}}%
\pgfpathcurveto{\pgfqpoint{1.639578in}{2.174577in}}{\pgfqpoint{1.631678in}{2.171305in}}{\pgfqpoint{1.625854in}{2.165481in}}%
\pgfpathcurveto{\pgfqpoint{1.620030in}{2.159657in}}{\pgfqpoint{1.616758in}{2.151757in}}{\pgfqpoint{1.616758in}{2.143520in}}%
\pgfpathcurveto{\pgfqpoint{1.616758in}{2.135284in}}{\pgfqpoint{1.620030in}{2.127384in}}{\pgfqpoint{1.625854in}{2.121560in}}%
\pgfpathcurveto{\pgfqpoint{1.631678in}{2.115736in}}{\pgfqpoint{1.639578in}{2.112464in}}{\pgfqpoint{1.647814in}{2.112464in}}%
\pgfpathclose%
\pgfusepath{stroke,fill}%
\end{pgfscope}%
\begin{pgfscope}%
\pgfpathrectangle{\pgfqpoint{0.100000in}{0.212622in}}{\pgfqpoint{3.696000in}{3.696000in}}%
\pgfusepath{clip}%
\pgfsetbuttcap%
\pgfsetroundjoin%
\definecolor{currentfill}{rgb}{0.121569,0.466667,0.705882}%
\pgfsetfillcolor{currentfill}%
\pgfsetfillopacity{0.300651}%
\pgfsetlinewidth{1.003750pt}%
\definecolor{currentstroke}{rgb}{0.121569,0.466667,0.705882}%
\pgfsetstrokecolor{currentstroke}%
\pgfsetstrokeopacity{0.300651}%
\pgfsetdash{}{0pt}%
\pgfpathmoveto{\pgfqpoint{1.647814in}{2.112464in}}%
\pgfpathcurveto{\pgfqpoint{1.656050in}{2.112464in}}{\pgfqpoint{1.663950in}{2.115736in}}{\pgfqpoint{1.669774in}{2.121560in}}%
\pgfpathcurveto{\pgfqpoint{1.675598in}{2.127384in}}{\pgfqpoint{1.678871in}{2.135284in}}{\pgfqpoint{1.678871in}{2.143520in}}%
\pgfpathcurveto{\pgfqpoint{1.678871in}{2.151757in}}{\pgfqpoint{1.675598in}{2.159657in}}{\pgfqpoint{1.669774in}{2.165481in}}%
\pgfpathcurveto{\pgfqpoint{1.663950in}{2.171305in}}{\pgfqpoint{1.656050in}{2.174577in}}{\pgfqpoint{1.647814in}{2.174577in}}%
\pgfpathcurveto{\pgfqpoint{1.639578in}{2.174577in}}{\pgfqpoint{1.631678in}{2.171305in}}{\pgfqpoint{1.625854in}{2.165481in}}%
\pgfpathcurveto{\pgfqpoint{1.620030in}{2.159657in}}{\pgfqpoint{1.616758in}{2.151757in}}{\pgfqpoint{1.616758in}{2.143520in}}%
\pgfpathcurveto{\pgfqpoint{1.616758in}{2.135284in}}{\pgfqpoint{1.620030in}{2.127384in}}{\pgfqpoint{1.625854in}{2.121560in}}%
\pgfpathcurveto{\pgfqpoint{1.631678in}{2.115736in}}{\pgfqpoint{1.639578in}{2.112464in}}{\pgfqpoint{1.647814in}{2.112464in}}%
\pgfpathclose%
\pgfusepath{stroke,fill}%
\end{pgfscope}%
\begin{pgfscope}%
\pgfpathrectangle{\pgfqpoint{0.100000in}{0.212622in}}{\pgfqpoint{3.696000in}{3.696000in}}%
\pgfusepath{clip}%
\pgfsetbuttcap%
\pgfsetroundjoin%
\definecolor{currentfill}{rgb}{0.121569,0.466667,0.705882}%
\pgfsetfillcolor{currentfill}%
\pgfsetfillopacity{0.300651}%
\pgfsetlinewidth{1.003750pt}%
\definecolor{currentstroke}{rgb}{0.121569,0.466667,0.705882}%
\pgfsetstrokecolor{currentstroke}%
\pgfsetstrokeopacity{0.300651}%
\pgfsetdash{}{0pt}%
\pgfpathmoveto{\pgfqpoint{1.647814in}{2.112464in}}%
\pgfpathcurveto{\pgfqpoint{1.656050in}{2.112464in}}{\pgfqpoint{1.663950in}{2.115736in}}{\pgfqpoint{1.669774in}{2.121560in}}%
\pgfpathcurveto{\pgfqpoint{1.675598in}{2.127384in}}{\pgfqpoint{1.678871in}{2.135284in}}{\pgfqpoint{1.678871in}{2.143520in}}%
\pgfpathcurveto{\pgfqpoint{1.678871in}{2.151757in}}{\pgfqpoint{1.675598in}{2.159657in}}{\pgfqpoint{1.669774in}{2.165481in}}%
\pgfpathcurveto{\pgfqpoint{1.663950in}{2.171305in}}{\pgfqpoint{1.656050in}{2.174577in}}{\pgfqpoint{1.647814in}{2.174577in}}%
\pgfpathcurveto{\pgfqpoint{1.639578in}{2.174577in}}{\pgfqpoint{1.631678in}{2.171305in}}{\pgfqpoint{1.625854in}{2.165481in}}%
\pgfpathcurveto{\pgfqpoint{1.620030in}{2.159657in}}{\pgfqpoint{1.616758in}{2.151757in}}{\pgfqpoint{1.616758in}{2.143520in}}%
\pgfpathcurveto{\pgfqpoint{1.616758in}{2.135284in}}{\pgfqpoint{1.620030in}{2.127384in}}{\pgfqpoint{1.625854in}{2.121560in}}%
\pgfpathcurveto{\pgfqpoint{1.631678in}{2.115736in}}{\pgfqpoint{1.639578in}{2.112464in}}{\pgfqpoint{1.647814in}{2.112464in}}%
\pgfpathclose%
\pgfusepath{stroke,fill}%
\end{pgfscope}%
\begin{pgfscope}%
\pgfpathrectangle{\pgfqpoint{0.100000in}{0.212622in}}{\pgfqpoint{3.696000in}{3.696000in}}%
\pgfusepath{clip}%
\pgfsetbuttcap%
\pgfsetroundjoin%
\definecolor{currentfill}{rgb}{0.121569,0.466667,0.705882}%
\pgfsetfillcolor{currentfill}%
\pgfsetfillopacity{0.300651}%
\pgfsetlinewidth{1.003750pt}%
\definecolor{currentstroke}{rgb}{0.121569,0.466667,0.705882}%
\pgfsetstrokecolor{currentstroke}%
\pgfsetstrokeopacity{0.300651}%
\pgfsetdash{}{0pt}%
\pgfpathmoveto{\pgfqpoint{1.647814in}{2.112464in}}%
\pgfpathcurveto{\pgfqpoint{1.656050in}{2.112464in}}{\pgfqpoint{1.663950in}{2.115736in}}{\pgfqpoint{1.669774in}{2.121560in}}%
\pgfpathcurveto{\pgfqpoint{1.675598in}{2.127384in}}{\pgfqpoint{1.678871in}{2.135284in}}{\pgfqpoint{1.678871in}{2.143520in}}%
\pgfpathcurveto{\pgfqpoint{1.678871in}{2.151757in}}{\pgfqpoint{1.675598in}{2.159657in}}{\pgfqpoint{1.669774in}{2.165481in}}%
\pgfpathcurveto{\pgfqpoint{1.663950in}{2.171305in}}{\pgfqpoint{1.656050in}{2.174577in}}{\pgfqpoint{1.647814in}{2.174577in}}%
\pgfpathcurveto{\pgfqpoint{1.639578in}{2.174577in}}{\pgfqpoint{1.631678in}{2.171305in}}{\pgfqpoint{1.625854in}{2.165481in}}%
\pgfpathcurveto{\pgfqpoint{1.620030in}{2.159657in}}{\pgfqpoint{1.616758in}{2.151757in}}{\pgfqpoint{1.616758in}{2.143520in}}%
\pgfpathcurveto{\pgfqpoint{1.616758in}{2.135284in}}{\pgfqpoint{1.620030in}{2.127384in}}{\pgfqpoint{1.625854in}{2.121560in}}%
\pgfpathcurveto{\pgfqpoint{1.631678in}{2.115736in}}{\pgfqpoint{1.639578in}{2.112464in}}{\pgfqpoint{1.647814in}{2.112464in}}%
\pgfpathclose%
\pgfusepath{stroke,fill}%
\end{pgfscope}%
\begin{pgfscope}%
\pgfpathrectangle{\pgfqpoint{0.100000in}{0.212622in}}{\pgfqpoint{3.696000in}{3.696000in}}%
\pgfusepath{clip}%
\pgfsetbuttcap%
\pgfsetroundjoin%
\definecolor{currentfill}{rgb}{0.121569,0.466667,0.705882}%
\pgfsetfillcolor{currentfill}%
\pgfsetfillopacity{0.300651}%
\pgfsetlinewidth{1.003750pt}%
\definecolor{currentstroke}{rgb}{0.121569,0.466667,0.705882}%
\pgfsetstrokecolor{currentstroke}%
\pgfsetstrokeopacity{0.300651}%
\pgfsetdash{}{0pt}%
\pgfpathmoveto{\pgfqpoint{1.647814in}{2.112464in}}%
\pgfpathcurveto{\pgfqpoint{1.656050in}{2.112464in}}{\pgfqpoint{1.663950in}{2.115736in}}{\pgfqpoint{1.669774in}{2.121560in}}%
\pgfpathcurveto{\pgfqpoint{1.675598in}{2.127384in}}{\pgfqpoint{1.678871in}{2.135284in}}{\pgfqpoint{1.678871in}{2.143520in}}%
\pgfpathcurveto{\pgfqpoint{1.678871in}{2.151757in}}{\pgfqpoint{1.675598in}{2.159657in}}{\pgfqpoint{1.669774in}{2.165481in}}%
\pgfpathcurveto{\pgfqpoint{1.663950in}{2.171305in}}{\pgfqpoint{1.656050in}{2.174577in}}{\pgfqpoint{1.647814in}{2.174577in}}%
\pgfpathcurveto{\pgfqpoint{1.639578in}{2.174577in}}{\pgfqpoint{1.631678in}{2.171305in}}{\pgfqpoint{1.625854in}{2.165481in}}%
\pgfpathcurveto{\pgfqpoint{1.620030in}{2.159657in}}{\pgfqpoint{1.616758in}{2.151757in}}{\pgfqpoint{1.616758in}{2.143520in}}%
\pgfpathcurveto{\pgfqpoint{1.616758in}{2.135284in}}{\pgfqpoint{1.620030in}{2.127384in}}{\pgfqpoint{1.625854in}{2.121560in}}%
\pgfpathcurveto{\pgfqpoint{1.631678in}{2.115736in}}{\pgfqpoint{1.639578in}{2.112464in}}{\pgfqpoint{1.647814in}{2.112464in}}%
\pgfpathclose%
\pgfusepath{stroke,fill}%
\end{pgfscope}%
\begin{pgfscope}%
\pgfpathrectangle{\pgfqpoint{0.100000in}{0.212622in}}{\pgfqpoint{3.696000in}{3.696000in}}%
\pgfusepath{clip}%
\pgfsetbuttcap%
\pgfsetroundjoin%
\definecolor{currentfill}{rgb}{0.121569,0.466667,0.705882}%
\pgfsetfillcolor{currentfill}%
\pgfsetfillopacity{0.300651}%
\pgfsetlinewidth{1.003750pt}%
\definecolor{currentstroke}{rgb}{0.121569,0.466667,0.705882}%
\pgfsetstrokecolor{currentstroke}%
\pgfsetstrokeopacity{0.300651}%
\pgfsetdash{}{0pt}%
\pgfpathmoveto{\pgfqpoint{1.647814in}{2.112464in}}%
\pgfpathcurveto{\pgfqpoint{1.656050in}{2.112464in}}{\pgfqpoint{1.663950in}{2.115736in}}{\pgfqpoint{1.669774in}{2.121560in}}%
\pgfpathcurveto{\pgfqpoint{1.675598in}{2.127384in}}{\pgfqpoint{1.678871in}{2.135284in}}{\pgfqpoint{1.678871in}{2.143520in}}%
\pgfpathcurveto{\pgfqpoint{1.678871in}{2.151757in}}{\pgfqpoint{1.675598in}{2.159657in}}{\pgfqpoint{1.669774in}{2.165481in}}%
\pgfpathcurveto{\pgfqpoint{1.663950in}{2.171305in}}{\pgfqpoint{1.656050in}{2.174577in}}{\pgfqpoint{1.647814in}{2.174577in}}%
\pgfpathcurveto{\pgfqpoint{1.639578in}{2.174577in}}{\pgfqpoint{1.631678in}{2.171305in}}{\pgfqpoint{1.625854in}{2.165481in}}%
\pgfpathcurveto{\pgfqpoint{1.620030in}{2.159657in}}{\pgfqpoint{1.616758in}{2.151757in}}{\pgfqpoint{1.616758in}{2.143520in}}%
\pgfpathcurveto{\pgfqpoint{1.616758in}{2.135284in}}{\pgfqpoint{1.620030in}{2.127384in}}{\pgfqpoint{1.625854in}{2.121560in}}%
\pgfpathcurveto{\pgfqpoint{1.631678in}{2.115736in}}{\pgfqpoint{1.639578in}{2.112464in}}{\pgfqpoint{1.647814in}{2.112464in}}%
\pgfpathclose%
\pgfusepath{stroke,fill}%
\end{pgfscope}%
\begin{pgfscope}%
\pgfpathrectangle{\pgfqpoint{0.100000in}{0.212622in}}{\pgfqpoint{3.696000in}{3.696000in}}%
\pgfusepath{clip}%
\pgfsetbuttcap%
\pgfsetroundjoin%
\definecolor{currentfill}{rgb}{0.121569,0.466667,0.705882}%
\pgfsetfillcolor{currentfill}%
\pgfsetfillopacity{0.300651}%
\pgfsetlinewidth{1.003750pt}%
\definecolor{currentstroke}{rgb}{0.121569,0.466667,0.705882}%
\pgfsetstrokecolor{currentstroke}%
\pgfsetstrokeopacity{0.300651}%
\pgfsetdash{}{0pt}%
\pgfpathmoveto{\pgfqpoint{1.647814in}{2.112464in}}%
\pgfpathcurveto{\pgfqpoint{1.656050in}{2.112464in}}{\pgfqpoint{1.663950in}{2.115736in}}{\pgfqpoint{1.669774in}{2.121560in}}%
\pgfpathcurveto{\pgfqpoint{1.675598in}{2.127384in}}{\pgfqpoint{1.678871in}{2.135284in}}{\pgfqpoint{1.678871in}{2.143520in}}%
\pgfpathcurveto{\pgfqpoint{1.678871in}{2.151757in}}{\pgfqpoint{1.675598in}{2.159657in}}{\pgfqpoint{1.669774in}{2.165481in}}%
\pgfpathcurveto{\pgfqpoint{1.663950in}{2.171305in}}{\pgfqpoint{1.656050in}{2.174577in}}{\pgfqpoint{1.647814in}{2.174577in}}%
\pgfpathcurveto{\pgfqpoint{1.639578in}{2.174577in}}{\pgfqpoint{1.631678in}{2.171305in}}{\pgfqpoint{1.625854in}{2.165481in}}%
\pgfpathcurveto{\pgfqpoint{1.620030in}{2.159657in}}{\pgfqpoint{1.616758in}{2.151757in}}{\pgfqpoint{1.616758in}{2.143520in}}%
\pgfpathcurveto{\pgfqpoint{1.616758in}{2.135284in}}{\pgfqpoint{1.620030in}{2.127384in}}{\pgfqpoint{1.625854in}{2.121560in}}%
\pgfpathcurveto{\pgfqpoint{1.631678in}{2.115736in}}{\pgfqpoint{1.639578in}{2.112464in}}{\pgfqpoint{1.647814in}{2.112464in}}%
\pgfpathclose%
\pgfusepath{stroke,fill}%
\end{pgfscope}%
\begin{pgfscope}%
\pgfpathrectangle{\pgfqpoint{0.100000in}{0.212622in}}{\pgfqpoint{3.696000in}{3.696000in}}%
\pgfusepath{clip}%
\pgfsetbuttcap%
\pgfsetroundjoin%
\definecolor{currentfill}{rgb}{0.121569,0.466667,0.705882}%
\pgfsetfillcolor{currentfill}%
\pgfsetfillopacity{0.300651}%
\pgfsetlinewidth{1.003750pt}%
\definecolor{currentstroke}{rgb}{0.121569,0.466667,0.705882}%
\pgfsetstrokecolor{currentstroke}%
\pgfsetstrokeopacity{0.300651}%
\pgfsetdash{}{0pt}%
\pgfpathmoveto{\pgfqpoint{1.647814in}{2.112464in}}%
\pgfpathcurveto{\pgfqpoint{1.656050in}{2.112464in}}{\pgfqpoint{1.663950in}{2.115736in}}{\pgfqpoint{1.669774in}{2.121560in}}%
\pgfpathcurveto{\pgfqpoint{1.675598in}{2.127384in}}{\pgfqpoint{1.678871in}{2.135284in}}{\pgfqpoint{1.678871in}{2.143520in}}%
\pgfpathcurveto{\pgfqpoint{1.678871in}{2.151757in}}{\pgfqpoint{1.675598in}{2.159657in}}{\pgfqpoint{1.669774in}{2.165481in}}%
\pgfpathcurveto{\pgfqpoint{1.663950in}{2.171305in}}{\pgfqpoint{1.656050in}{2.174577in}}{\pgfqpoint{1.647814in}{2.174577in}}%
\pgfpathcurveto{\pgfqpoint{1.639578in}{2.174577in}}{\pgfqpoint{1.631678in}{2.171305in}}{\pgfqpoint{1.625854in}{2.165481in}}%
\pgfpathcurveto{\pgfqpoint{1.620030in}{2.159657in}}{\pgfqpoint{1.616758in}{2.151757in}}{\pgfqpoint{1.616758in}{2.143520in}}%
\pgfpathcurveto{\pgfqpoint{1.616758in}{2.135284in}}{\pgfqpoint{1.620030in}{2.127384in}}{\pgfqpoint{1.625854in}{2.121560in}}%
\pgfpathcurveto{\pgfqpoint{1.631678in}{2.115736in}}{\pgfqpoint{1.639578in}{2.112464in}}{\pgfqpoint{1.647814in}{2.112464in}}%
\pgfpathclose%
\pgfusepath{stroke,fill}%
\end{pgfscope}%
\begin{pgfscope}%
\pgfpathrectangle{\pgfqpoint{0.100000in}{0.212622in}}{\pgfqpoint{3.696000in}{3.696000in}}%
\pgfusepath{clip}%
\pgfsetbuttcap%
\pgfsetroundjoin%
\definecolor{currentfill}{rgb}{0.121569,0.466667,0.705882}%
\pgfsetfillcolor{currentfill}%
\pgfsetfillopacity{0.300651}%
\pgfsetlinewidth{1.003750pt}%
\definecolor{currentstroke}{rgb}{0.121569,0.466667,0.705882}%
\pgfsetstrokecolor{currentstroke}%
\pgfsetstrokeopacity{0.300651}%
\pgfsetdash{}{0pt}%
\pgfpathmoveto{\pgfqpoint{1.647814in}{2.112464in}}%
\pgfpathcurveto{\pgfqpoint{1.656050in}{2.112464in}}{\pgfqpoint{1.663950in}{2.115736in}}{\pgfqpoint{1.669774in}{2.121560in}}%
\pgfpathcurveto{\pgfqpoint{1.675598in}{2.127384in}}{\pgfqpoint{1.678871in}{2.135284in}}{\pgfqpoint{1.678871in}{2.143520in}}%
\pgfpathcurveto{\pgfqpoint{1.678871in}{2.151757in}}{\pgfqpoint{1.675598in}{2.159657in}}{\pgfqpoint{1.669774in}{2.165481in}}%
\pgfpathcurveto{\pgfqpoint{1.663950in}{2.171305in}}{\pgfqpoint{1.656050in}{2.174577in}}{\pgfqpoint{1.647814in}{2.174577in}}%
\pgfpathcurveto{\pgfqpoint{1.639578in}{2.174577in}}{\pgfqpoint{1.631678in}{2.171305in}}{\pgfqpoint{1.625854in}{2.165481in}}%
\pgfpathcurveto{\pgfqpoint{1.620030in}{2.159657in}}{\pgfqpoint{1.616758in}{2.151757in}}{\pgfqpoint{1.616758in}{2.143520in}}%
\pgfpathcurveto{\pgfqpoint{1.616758in}{2.135284in}}{\pgfqpoint{1.620030in}{2.127384in}}{\pgfqpoint{1.625854in}{2.121560in}}%
\pgfpathcurveto{\pgfqpoint{1.631678in}{2.115736in}}{\pgfqpoint{1.639578in}{2.112464in}}{\pgfqpoint{1.647814in}{2.112464in}}%
\pgfpathclose%
\pgfusepath{stroke,fill}%
\end{pgfscope}%
\begin{pgfscope}%
\pgfpathrectangle{\pgfqpoint{0.100000in}{0.212622in}}{\pgfqpoint{3.696000in}{3.696000in}}%
\pgfusepath{clip}%
\pgfsetbuttcap%
\pgfsetroundjoin%
\definecolor{currentfill}{rgb}{0.121569,0.466667,0.705882}%
\pgfsetfillcolor{currentfill}%
\pgfsetfillopacity{0.300651}%
\pgfsetlinewidth{1.003750pt}%
\definecolor{currentstroke}{rgb}{0.121569,0.466667,0.705882}%
\pgfsetstrokecolor{currentstroke}%
\pgfsetstrokeopacity{0.300651}%
\pgfsetdash{}{0pt}%
\pgfpathmoveto{\pgfqpoint{1.647814in}{2.112464in}}%
\pgfpathcurveto{\pgfqpoint{1.656050in}{2.112464in}}{\pgfqpoint{1.663950in}{2.115736in}}{\pgfqpoint{1.669774in}{2.121560in}}%
\pgfpathcurveto{\pgfqpoint{1.675598in}{2.127384in}}{\pgfqpoint{1.678871in}{2.135284in}}{\pgfqpoint{1.678871in}{2.143520in}}%
\pgfpathcurveto{\pgfqpoint{1.678871in}{2.151757in}}{\pgfqpoint{1.675598in}{2.159657in}}{\pgfqpoint{1.669774in}{2.165481in}}%
\pgfpathcurveto{\pgfqpoint{1.663950in}{2.171305in}}{\pgfqpoint{1.656050in}{2.174577in}}{\pgfqpoint{1.647814in}{2.174577in}}%
\pgfpathcurveto{\pgfqpoint{1.639578in}{2.174577in}}{\pgfqpoint{1.631678in}{2.171305in}}{\pgfqpoint{1.625854in}{2.165481in}}%
\pgfpathcurveto{\pgfqpoint{1.620030in}{2.159657in}}{\pgfqpoint{1.616758in}{2.151757in}}{\pgfqpoint{1.616758in}{2.143520in}}%
\pgfpathcurveto{\pgfqpoint{1.616758in}{2.135284in}}{\pgfqpoint{1.620030in}{2.127384in}}{\pgfqpoint{1.625854in}{2.121560in}}%
\pgfpathcurveto{\pgfqpoint{1.631678in}{2.115736in}}{\pgfqpoint{1.639578in}{2.112464in}}{\pgfqpoint{1.647814in}{2.112464in}}%
\pgfpathclose%
\pgfusepath{stroke,fill}%
\end{pgfscope}%
\begin{pgfscope}%
\pgfpathrectangle{\pgfqpoint{0.100000in}{0.212622in}}{\pgfqpoint{3.696000in}{3.696000in}}%
\pgfusepath{clip}%
\pgfsetbuttcap%
\pgfsetroundjoin%
\definecolor{currentfill}{rgb}{0.121569,0.466667,0.705882}%
\pgfsetfillcolor{currentfill}%
\pgfsetfillopacity{0.300651}%
\pgfsetlinewidth{1.003750pt}%
\definecolor{currentstroke}{rgb}{0.121569,0.466667,0.705882}%
\pgfsetstrokecolor{currentstroke}%
\pgfsetstrokeopacity{0.300651}%
\pgfsetdash{}{0pt}%
\pgfpathmoveto{\pgfqpoint{1.647814in}{2.112464in}}%
\pgfpathcurveto{\pgfqpoint{1.656050in}{2.112464in}}{\pgfqpoint{1.663950in}{2.115736in}}{\pgfqpoint{1.669774in}{2.121560in}}%
\pgfpathcurveto{\pgfqpoint{1.675598in}{2.127384in}}{\pgfqpoint{1.678871in}{2.135284in}}{\pgfqpoint{1.678871in}{2.143520in}}%
\pgfpathcurveto{\pgfqpoint{1.678871in}{2.151757in}}{\pgfqpoint{1.675598in}{2.159657in}}{\pgfqpoint{1.669774in}{2.165481in}}%
\pgfpathcurveto{\pgfqpoint{1.663950in}{2.171305in}}{\pgfqpoint{1.656050in}{2.174577in}}{\pgfqpoint{1.647814in}{2.174577in}}%
\pgfpathcurveto{\pgfqpoint{1.639578in}{2.174577in}}{\pgfqpoint{1.631678in}{2.171305in}}{\pgfqpoint{1.625854in}{2.165481in}}%
\pgfpathcurveto{\pgfqpoint{1.620030in}{2.159657in}}{\pgfqpoint{1.616758in}{2.151757in}}{\pgfqpoint{1.616758in}{2.143520in}}%
\pgfpathcurveto{\pgfqpoint{1.616758in}{2.135284in}}{\pgfqpoint{1.620030in}{2.127384in}}{\pgfqpoint{1.625854in}{2.121560in}}%
\pgfpathcurveto{\pgfqpoint{1.631678in}{2.115736in}}{\pgfqpoint{1.639578in}{2.112464in}}{\pgfqpoint{1.647814in}{2.112464in}}%
\pgfpathclose%
\pgfusepath{stroke,fill}%
\end{pgfscope}%
\begin{pgfscope}%
\pgfpathrectangle{\pgfqpoint{0.100000in}{0.212622in}}{\pgfqpoint{3.696000in}{3.696000in}}%
\pgfusepath{clip}%
\pgfsetbuttcap%
\pgfsetroundjoin%
\definecolor{currentfill}{rgb}{0.121569,0.466667,0.705882}%
\pgfsetfillcolor{currentfill}%
\pgfsetfillopacity{0.300651}%
\pgfsetlinewidth{1.003750pt}%
\definecolor{currentstroke}{rgb}{0.121569,0.466667,0.705882}%
\pgfsetstrokecolor{currentstroke}%
\pgfsetstrokeopacity{0.300651}%
\pgfsetdash{}{0pt}%
\pgfpathmoveto{\pgfqpoint{1.647814in}{2.112464in}}%
\pgfpathcurveto{\pgfqpoint{1.656050in}{2.112464in}}{\pgfqpoint{1.663950in}{2.115736in}}{\pgfqpoint{1.669774in}{2.121560in}}%
\pgfpathcurveto{\pgfqpoint{1.675598in}{2.127384in}}{\pgfqpoint{1.678871in}{2.135284in}}{\pgfqpoint{1.678871in}{2.143520in}}%
\pgfpathcurveto{\pgfqpoint{1.678871in}{2.151757in}}{\pgfqpoint{1.675598in}{2.159657in}}{\pgfqpoint{1.669774in}{2.165481in}}%
\pgfpathcurveto{\pgfqpoint{1.663950in}{2.171305in}}{\pgfqpoint{1.656050in}{2.174577in}}{\pgfqpoint{1.647814in}{2.174577in}}%
\pgfpathcurveto{\pgfqpoint{1.639578in}{2.174577in}}{\pgfqpoint{1.631678in}{2.171305in}}{\pgfqpoint{1.625854in}{2.165481in}}%
\pgfpathcurveto{\pgfqpoint{1.620030in}{2.159657in}}{\pgfqpoint{1.616758in}{2.151757in}}{\pgfqpoint{1.616758in}{2.143520in}}%
\pgfpathcurveto{\pgfqpoint{1.616758in}{2.135284in}}{\pgfqpoint{1.620030in}{2.127384in}}{\pgfqpoint{1.625854in}{2.121560in}}%
\pgfpathcurveto{\pgfqpoint{1.631678in}{2.115736in}}{\pgfqpoint{1.639578in}{2.112464in}}{\pgfqpoint{1.647814in}{2.112464in}}%
\pgfpathclose%
\pgfusepath{stroke,fill}%
\end{pgfscope}%
\begin{pgfscope}%
\pgfpathrectangle{\pgfqpoint{0.100000in}{0.212622in}}{\pgfqpoint{3.696000in}{3.696000in}}%
\pgfusepath{clip}%
\pgfsetbuttcap%
\pgfsetroundjoin%
\definecolor{currentfill}{rgb}{0.121569,0.466667,0.705882}%
\pgfsetfillcolor{currentfill}%
\pgfsetfillopacity{0.300651}%
\pgfsetlinewidth{1.003750pt}%
\definecolor{currentstroke}{rgb}{0.121569,0.466667,0.705882}%
\pgfsetstrokecolor{currentstroke}%
\pgfsetstrokeopacity{0.300651}%
\pgfsetdash{}{0pt}%
\pgfpathmoveto{\pgfqpoint{1.647814in}{2.112464in}}%
\pgfpathcurveto{\pgfqpoint{1.656050in}{2.112464in}}{\pgfqpoint{1.663950in}{2.115736in}}{\pgfqpoint{1.669774in}{2.121560in}}%
\pgfpathcurveto{\pgfqpoint{1.675598in}{2.127384in}}{\pgfqpoint{1.678871in}{2.135284in}}{\pgfqpoint{1.678871in}{2.143520in}}%
\pgfpathcurveto{\pgfqpoint{1.678871in}{2.151757in}}{\pgfqpoint{1.675598in}{2.159657in}}{\pgfqpoint{1.669774in}{2.165481in}}%
\pgfpathcurveto{\pgfqpoint{1.663950in}{2.171305in}}{\pgfqpoint{1.656050in}{2.174577in}}{\pgfqpoint{1.647814in}{2.174577in}}%
\pgfpathcurveto{\pgfqpoint{1.639578in}{2.174577in}}{\pgfqpoint{1.631678in}{2.171305in}}{\pgfqpoint{1.625854in}{2.165481in}}%
\pgfpathcurveto{\pgfqpoint{1.620030in}{2.159657in}}{\pgfqpoint{1.616758in}{2.151757in}}{\pgfqpoint{1.616758in}{2.143520in}}%
\pgfpathcurveto{\pgfqpoint{1.616758in}{2.135284in}}{\pgfqpoint{1.620030in}{2.127384in}}{\pgfqpoint{1.625854in}{2.121560in}}%
\pgfpathcurveto{\pgfqpoint{1.631678in}{2.115736in}}{\pgfqpoint{1.639578in}{2.112464in}}{\pgfqpoint{1.647814in}{2.112464in}}%
\pgfpathclose%
\pgfusepath{stroke,fill}%
\end{pgfscope}%
\begin{pgfscope}%
\pgfpathrectangle{\pgfqpoint{0.100000in}{0.212622in}}{\pgfqpoint{3.696000in}{3.696000in}}%
\pgfusepath{clip}%
\pgfsetbuttcap%
\pgfsetroundjoin%
\definecolor{currentfill}{rgb}{0.121569,0.466667,0.705882}%
\pgfsetfillcolor{currentfill}%
\pgfsetfillopacity{0.300651}%
\pgfsetlinewidth{1.003750pt}%
\definecolor{currentstroke}{rgb}{0.121569,0.466667,0.705882}%
\pgfsetstrokecolor{currentstroke}%
\pgfsetstrokeopacity{0.300651}%
\pgfsetdash{}{0pt}%
\pgfpathmoveto{\pgfqpoint{1.647814in}{2.112464in}}%
\pgfpathcurveto{\pgfqpoint{1.656050in}{2.112464in}}{\pgfqpoint{1.663950in}{2.115736in}}{\pgfqpoint{1.669774in}{2.121560in}}%
\pgfpathcurveto{\pgfqpoint{1.675598in}{2.127384in}}{\pgfqpoint{1.678871in}{2.135284in}}{\pgfqpoint{1.678871in}{2.143520in}}%
\pgfpathcurveto{\pgfqpoint{1.678871in}{2.151757in}}{\pgfqpoint{1.675598in}{2.159657in}}{\pgfqpoint{1.669774in}{2.165481in}}%
\pgfpathcurveto{\pgfqpoint{1.663950in}{2.171305in}}{\pgfqpoint{1.656050in}{2.174577in}}{\pgfqpoint{1.647814in}{2.174577in}}%
\pgfpathcurveto{\pgfqpoint{1.639578in}{2.174577in}}{\pgfqpoint{1.631678in}{2.171305in}}{\pgfqpoint{1.625854in}{2.165481in}}%
\pgfpathcurveto{\pgfqpoint{1.620030in}{2.159657in}}{\pgfqpoint{1.616758in}{2.151757in}}{\pgfqpoint{1.616758in}{2.143520in}}%
\pgfpathcurveto{\pgfqpoint{1.616758in}{2.135284in}}{\pgfqpoint{1.620030in}{2.127384in}}{\pgfqpoint{1.625854in}{2.121560in}}%
\pgfpathcurveto{\pgfqpoint{1.631678in}{2.115736in}}{\pgfqpoint{1.639578in}{2.112464in}}{\pgfqpoint{1.647814in}{2.112464in}}%
\pgfpathclose%
\pgfusepath{stroke,fill}%
\end{pgfscope}%
\begin{pgfscope}%
\pgfpathrectangle{\pgfqpoint{0.100000in}{0.212622in}}{\pgfqpoint{3.696000in}{3.696000in}}%
\pgfusepath{clip}%
\pgfsetbuttcap%
\pgfsetroundjoin%
\definecolor{currentfill}{rgb}{0.121569,0.466667,0.705882}%
\pgfsetfillcolor{currentfill}%
\pgfsetfillopacity{0.300651}%
\pgfsetlinewidth{1.003750pt}%
\definecolor{currentstroke}{rgb}{0.121569,0.466667,0.705882}%
\pgfsetstrokecolor{currentstroke}%
\pgfsetstrokeopacity{0.300651}%
\pgfsetdash{}{0pt}%
\pgfpathmoveto{\pgfqpoint{1.647814in}{2.112464in}}%
\pgfpathcurveto{\pgfqpoint{1.656050in}{2.112464in}}{\pgfqpoint{1.663950in}{2.115736in}}{\pgfqpoint{1.669774in}{2.121560in}}%
\pgfpathcurveto{\pgfqpoint{1.675598in}{2.127384in}}{\pgfqpoint{1.678871in}{2.135284in}}{\pgfqpoint{1.678871in}{2.143520in}}%
\pgfpathcurveto{\pgfqpoint{1.678871in}{2.151757in}}{\pgfqpoint{1.675598in}{2.159657in}}{\pgfqpoint{1.669774in}{2.165481in}}%
\pgfpathcurveto{\pgfqpoint{1.663950in}{2.171305in}}{\pgfqpoint{1.656050in}{2.174577in}}{\pgfqpoint{1.647814in}{2.174577in}}%
\pgfpathcurveto{\pgfqpoint{1.639578in}{2.174577in}}{\pgfqpoint{1.631678in}{2.171305in}}{\pgfqpoint{1.625854in}{2.165481in}}%
\pgfpathcurveto{\pgfqpoint{1.620030in}{2.159657in}}{\pgfqpoint{1.616758in}{2.151757in}}{\pgfqpoint{1.616758in}{2.143520in}}%
\pgfpathcurveto{\pgfqpoint{1.616758in}{2.135284in}}{\pgfqpoint{1.620030in}{2.127384in}}{\pgfqpoint{1.625854in}{2.121560in}}%
\pgfpathcurveto{\pgfqpoint{1.631678in}{2.115736in}}{\pgfqpoint{1.639578in}{2.112464in}}{\pgfqpoint{1.647814in}{2.112464in}}%
\pgfpathclose%
\pgfusepath{stroke,fill}%
\end{pgfscope}%
\begin{pgfscope}%
\pgfpathrectangle{\pgfqpoint{0.100000in}{0.212622in}}{\pgfqpoint{3.696000in}{3.696000in}}%
\pgfusepath{clip}%
\pgfsetbuttcap%
\pgfsetroundjoin%
\definecolor{currentfill}{rgb}{0.121569,0.466667,0.705882}%
\pgfsetfillcolor{currentfill}%
\pgfsetfillopacity{0.300651}%
\pgfsetlinewidth{1.003750pt}%
\definecolor{currentstroke}{rgb}{0.121569,0.466667,0.705882}%
\pgfsetstrokecolor{currentstroke}%
\pgfsetstrokeopacity{0.300651}%
\pgfsetdash{}{0pt}%
\pgfpathmoveto{\pgfqpoint{1.647814in}{2.112464in}}%
\pgfpathcurveto{\pgfqpoint{1.656050in}{2.112464in}}{\pgfqpoint{1.663950in}{2.115736in}}{\pgfqpoint{1.669774in}{2.121560in}}%
\pgfpathcurveto{\pgfqpoint{1.675598in}{2.127384in}}{\pgfqpoint{1.678871in}{2.135284in}}{\pgfqpoint{1.678871in}{2.143520in}}%
\pgfpathcurveto{\pgfqpoint{1.678871in}{2.151757in}}{\pgfqpoint{1.675598in}{2.159657in}}{\pgfqpoint{1.669774in}{2.165481in}}%
\pgfpathcurveto{\pgfqpoint{1.663950in}{2.171305in}}{\pgfqpoint{1.656050in}{2.174577in}}{\pgfqpoint{1.647814in}{2.174577in}}%
\pgfpathcurveto{\pgfqpoint{1.639578in}{2.174577in}}{\pgfqpoint{1.631678in}{2.171305in}}{\pgfqpoint{1.625854in}{2.165481in}}%
\pgfpathcurveto{\pgfqpoint{1.620030in}{2.159657in}}{\pgfqpoint{1.616758in}{2.151757in}}{\pgfqpoint{1.616758in}{2.143520in}}%
\pgfpathcurveto{\pgfqpoint{1.616758in}{2.135284in}}{\pgfqpoint{1.620030in}{2.127384in}}{\pgfqpoint{1.625854in}{2.121560in}}%
\pgfpathcurveto{\pgfqpoint{1.631678in}{2.115736in}}{\pgfqpoint{1.639578in}{2.112464in}}{\pgfqpoint{1.647814in}{2.112464in}}%
\pgfpathclose%
\pgfusepath{stroke,fill}%
\end{pgfscope}%
\begin{pgfscope}%
\pgfpathrectangle{\pgfqpoint{0.100000in}{0.212622in}}{\pgfqpoint{3.696000in}{3.696000in}}%
\pgfusepath{clip}%
\pgfsetbuttcap%
\pgfsetroundjoin%
\definecolor{currentfill}{rgb}{0.121569,0.466667,0.705882}%
\pgfsetfillcolor{currentfill}%
\pgfsetfillopacity{0.300651}%
\pgfsetlinewidth{1.003750pt}%
\definecolor{currentstroke}{rgb}{0.121569,0.466667,0.705882}%
\pgfsetstrokecolor{currentstroke}%
\pgfsetstrokeopacity{0.300651}%
\pgfsetdash{}{0pt}%
\pgfpathmoveto{\pgfqpoint{1.647814in}{2.112464in}}%
\pgfpathcurveto{\pgfqpoint{1.656050in}{2.112464in}}{\pgfqpoint{1.663950in}{2.115736in}}{\pgfqpoint{1.669774in}{2.121560in}}%
\pgfpathcurveto{\pgfqpoint{1.675598in}{2.127384in}}{\pgfqpoint{1.678871in}{2.135284in}}{\pgfqpoint{1.678871in}{2.143520in}}%
\pgfpathcurveto{\pgfqpoint{1.678871in}{2.151757in}}{\pgfqpoint{1.675598in}{2.159657in}}{\pgfqpoint{1.669774in}{2.165481in}}%
\pgfpathcurveto{\pgfqpoint{1.663950in}{2.171305in}}{\pgfqpoint{1.656050in}{2.174577in}}{\pgfqpoint{1.647814in}{2.174577in}}%
\pgfpathcurveto{\pgfqpoint{1.639578in}{2.174577in}}{\pgfqpoint{1.631678in}{2.171305in}}{\pgfqpoint{1.625854in}{2.165481in}}%
\pgfpathcurveto{\pgfqpoint{1.620030in}{2.159657in}}{\pgfqpoint{1.616758in}{2.151757in}}{\pgfqpoint{1.616758in}{2.143520in}}%
\pgfpathcurveto{\pgfqpoint{1.616758in}{2.135284in}}{\pgfqpoint{1.620030in}{2.127384in}}{\pgfqpoint{1.625854in}{2.121560in}}%
\pgfpathcurveto{\pgfqpoint{1.631678in}{2.115736in}}{\pgfqpoint{1.639578in}{2.112464in}}{\pgfqpoint{1.647814in}{2.112464in}}%
\pgfpathclose%
\pgfusepath{stroke,fill}%
\end{pgfscope}%
\begin{pgfscope}%
\pgfpathrectangle{\pgfqpoint{0.100000in}{0.212622in}}{\pgfqpoint{3.696000in}{3.696000in}}%
\pgfusepath{clip}%
\pgfsetbuttcap%
\pgfsetroundjoin%
\definecolor{currentfill}{rgb}{0.121569,0.466667,0.705882}%
\pgfsetfillcolor{currentfill}%
\pgfsetfillopacity{0.300651}%
\pgfsetlinewidth{1.003750pt}%
\definecolor{currentstroke}{rgb}{0.121569,0.466667,0.705882}%
\pgfsetstrokecolor{currentstroke}%
\pgfsetstrokeopacity{0.300651}%
\pgfsetdash{}{0pt}%
\pgfpathmoveto{\pgfqpoint{1.647814in}{2.112464in}}%
\pgfpathcurveto{\pgfqpoint{1.656050in}{2.112464in}}{\pgfqpoint{1.663950in}{2.115736in}}{\pgfqpoint{1.669774in}{2.121560in}}%
\pgfpathcurveto{\pgfqpoint{1.675598in}{2.127384in}}{\pgfqpoint{1.678871in}{2.135284in}}{\pgfqpoint{1.678871in}{2.143520in}}%
\pgfpathcurveto{\pgfqpoint{1.678871in}{2.151757in}}{\pgfqpoint{1.675598in}{2.159657in}}{\pgfqpoint{1.669774in}{2.165481in}}%
\pgfpathcurveto{\pgfqpoint{1.663950in}{2.171305in}}{\pgfqpoint{1.656050in}{2.174577in}}{\pgfqpoint{1.647814in}{2.174577in}}%
\pgfpathcurveto{\pgfqpoint{1.639578in}{2.174577in}}{\pgfqpoint{1.631678in}{2.171305in}}{\pgfqpoint{1.625854in}{2.165481in}}%
\pgfpathcurveto{\pgfqpoint{1.620030in}{2.159657in}}{\pgfqpoint{1.616758in}{2.151757in}}{\pgfqpoint{1.616758in}{2.143520in}}%
\pgfpathcurveto{\pgfqpoint{1.616758in}{2.135284in}}{\pgfqpoint{1.620030in}{2.127384in}}{\pgfqpoint{1.625854in}{2.121560in}}%
\pgfpathcurveto{\pgfqpoint{1.631678in}{2.115736in}}{\pgfqpoint{1.639578in}{2.112464in}}{\pgfqpoint{1.647814in}{2.112464in}}%
\pgfpathclose%
\pgfusepath{stroke,fill}%
\end{pgfscope}%
\begin{pgfscope}%
\pgfpathrectangle{\pgfqpoint{0.100000in}{0.212622in}}{\pgfqpoint{3.696000in}{3.696000in}}%
\pgfusepath{clip}%
\pgfsetbuttcap%
\pgfsetroundjoin%
\definecolor{currentfill}{rgb}{0.121569,0.466667,0.705882}%
\pgfsetfillcolor{currentfill}%
\pgfsetfillopacity{0.300651}%
\pgfsetlinewidth{1.003750pt}%
\definecolor{currentstroke}{rgb}{0.121569,0.466667,0.705882}%
\pgfsetstrokecolor{currentstroke}%
\pgfsetstrokeopacity{0.300651}%
\pgfsetdash{}{0pt}%
\pgfpathmoveto{\pgfqpoint{1.647814in}{2.112464in}}%
\pgfpathcurveto{\pgfqpoint{1.656050in}{2.112464in}}{\pgfqpoint{1.663950in}{2.115736in}}{\pgfqpoint{1.669774in}{2.121560in}}%
\pgfpathcurveto{\pgfqpoint{1.675598in}{2.127384in}}{\pgfqpoint{1.678871in}{2.135284in}}{\pgfqpoint{1.678871in}{2.143520in}}%
\pgfpathcurveto{\pgfqpoint{1.678871in}{2.151757in}}{\pgfqpoint{1.675598in}{2.159657in}}{\pgfqpoint{1.669774in}{2.165481in}}%
\pgfpathcurveto{\pgfqpoint{1.663950in}{2.171305in}}{\pgfqpoint{1.656050in}{2.174577in}}{\pgfqpoint{1.647814in}{2.174577in}}%
\pgfpathcurveto{\pgfqpoint{1.639578in}{2.174577in}}{\pgfqpoint{1.631678in}{2.171305in}}{\pgfqpoint{1.625854in}{2.165481in}}%
\pgfpathcurveto{\pgfqpoint{1.620030in}{2.159657in}}{\pgfqpoint{1.616758in}{2.151757in}}{\pgfqpoint{1.616758in}{2.143520in}}%
\pgfpathcurveto{\pgfqpoint{1.616758in}{2.135284in}}{\pgfqpoint{1.620030in}{2.127384in}}{\pgfqpoint{1.625854in}{2.121560in}}%
\pgfpathcurveto{\pgfqpoint{1.631678in}{2.115736in}}{\pgfqpoint{1.639578in}{2.112464in}}{\pgfqpoint{1.647814in}{2.112464in}}%
\pgfpathclose%
\pgfusepath{stroke,fill}%
\end{pgfscope}%
\begin{pgfscope}%
\pgfpathrectangle{\pgfqpoint{0.100000in}{0.212622in}}{\pgfqpoint{3.696000in}{3.696000in}}%
\pgfusepath{clip}%
\pgfsetbuttcap%
\pgfsetroundjoin%
\definecolor{currentfill}{rgb}{0.121569,0.466667,0.705882}%
\pgfsetfillcolor{currentfill}%
\pgfsetfillopacity{0.300651}%
\pgfsetlinewidth{1.003750pt}%
\definecolor{currentstroke}{rgb}{0.121569,0.466667,0.705882}%
\pgfsetstrokecolor{currentstroke}%
\pgfsetstrokeopacity{0.300651}%
\pgfsetdash{}{0pt}%
\pgfpathmoveto{\pgfqpoint{1.647814in}{2.112464in}}%
\pgfpathcurveto{\pgfqpoint{1.656050in}{2.112464in}}{\pgfqpoint{1.663950in}{2.115736in}}{\pgfqpoint{1.669774in}{2.121560in}}%
\pgfpathcurveto{\pgfqpoint{1.675598in}{2.127384in}}{\pgfqpoint{1.678871in}{2.135284in}}{\pgfqpoint{1.678871in}{2.143520in}}%
\pgfpathcurveto{\pgfqpoint{1.678871in}{2.151757in}}{\pgfqpoint{1.675598in}{2.159657in}}{\pgfqpoint{1.669774in}{2.165481in}}%
\pgfpathcurveto{\pgfqpoint{1.663950in}{2.171305in}}{\pgfqpoint{1.656050in}{2.174577in}}{\pgfqpoint{1.647814in}{2.174577in}}%
\pgfpathcurveto{\pgfqpoint{1.639578in}{2.174577in}}{\pgfqpoint{1.631678in}{2.171305in}}{\pgfqpoint{1.625854in}{2.165481in}}%
\pgfpathcurveto{\pgfqpoint{1.620030in}{2.159657in}}{\pgfqpoint{1.616758in}{2.151757in}}{\pgfqpoint{1.616758in}{2.143520in}}%
\pgfpathcurveto{\pgfqpoint{1.616758in}{2.135284in}}{\pgfqpoint{1.620030in}{2.127384in}}{\pgfqpoint{1.625854in}{2.121560in}}%
\pgfpathcurveto{\pgfqpoint{1.631678in}{2.115736in}}{\pgfqpoint{1.639578in}{2.112464in}}{\pgfqpoint{1.647814in}{2.112464in}}%
\pgfpathclose%
\pgfusepath{stroke,fill}%
\end{pgfscope}%
\begin{pgfscope}%
\pgfpathrectangle{\pgfqpoint{0.100000in}{0.212622in}}{\pgfqpoint{3.696000in}{3.696000in}}%
\pgfusepath{clip}%
\pgfsetbuttcap%
\pgfsetroundjoin%
\definecolor{currentfill}{rgb}{0.121569,0.466667,0.705882}%
\pgfsetfillcolor{currentfill}%
\pgfsetfillopacity{0.300651}%
\pgfsetlinewidth{1.003750pt}%
\definecolor{currentstroke}{rgb}{0.121569,0.466667,0.705882}%
\pgfsetstrokecolor{currentstroke}%
\pgfsetstrokeopacity{0.300651}%
\pgfsetdash{}{0pt}%
\pgfpathmoveto{\pgfqpoint{1.647814in}{2.112464in}}%
\pgfpathcurveto{\pgfqpoint{1.656050in}{2.112464in}}{\pgfqpoint{1.663950in}{2.115736in}}{\pgfqpoint{1.669774in}{2.121560in}}%
\pgfpathcurveto{\pgfqpoint{1.675598in}{2.127384in}}{\pgfqpoint{1.678871in}{2.135284in}}{\pgfqpoint{1.678871in}{2.143520in}}%
\pgfpathcurveto{\pgfqpoint{1.678871in}{2.151757in}}{\pgfqpoint{1.675598in}{2.159657in}}{\pgfqpoint{1.669774in}{2.165481in}}%
\pgfpathcurveto{\pgfqpoint{1.663950in}{2.171305in}}{\pgfqpoint{1.656050in}{2.174577in}}{\pgfqpoint{1.647814in}{2.174577in}}%
\pgfpathcurveto{\pgfqpoint{1.639578in}{2.174577in}}{\pgfqpoint{1.631678in}{2.171305in}}{\pgfqpoint{1.625854in}{2.165481in}}%
\pgfpathcurveto{\pgfqpoint{1.620030in}{2.159657in}}{\pgfqpoint{1.616758in}{2.151757in}}{\pgfqpoint{1.616758in}{2.143520in}}%
\pgfpathcurveto{\pgfqpoint{1.616758in}{2.135284in}}{\pgfqpoint{1.620030in}{2.127384in}}{\pgfqpoint{1.625854in}{2.121560in}}%
\pgfpathcurveto{\pgfqpoint{1.631678in}{2.115736in}}{\pgfqpoint{1.639578in}{2.112464in}}{\pgfqpoint{1.647814in}{2.112464in}}%
\pgfpathclose%
\pgfusepath{stroke,fill}%
\end{pgfscope}%
\begin{pgfscope}%
\pgfpathrectangle{\pgfqpoint{0.100000in}{0.212622in}}{\pgfqpoint{3.696000in}{3.696000in}}%
\pgfusepath{clip}%
\pgfsetbuttcap%
\pgfsetroundjoin%
\definecolor{currentfill}{rgb}{0.121569,0.466667,0.705882}%
\pgfsetfillcolor{currentfill}%
\pgfsetfillopacity{0.300651}%
\pgfsetlinewidth{1.003750pt}%
\definecolor{currentstroke}{rgb}{0.121569,0.466667,0.705882}%
\pgfsetstrokecolor{currentstroke}%
\pgfsetstrokeopacity{0.300651}%
\pgfsetdash{}{0pt}%
\pgfpathmoveto{\pgfqpoint{1.647814in}{2.112464in}}%
\pgfpathcurveto{\pgfqpoint{1.656050in}{2.112464in}}{\pgfqpoint{1.663950in}{2.115736in}}{\pgfqpoint{1.669774in}{2.121560in}}%
\pgfpathcurveto{\pgfqpoint{1.675598in}{2.127384in}}{\pgfqpoint{1.678871in}{2.135284in}}{\pgfqpoint{1.678871in}{2.143520in}}%
\pgfpathcurveto{\pgfqpoint{1.678871in}{2.151757in}}{\pgfqpoint{1.675598in}{2.159657in}}{\pgfqpoint{1.669774in}{2.165481in}}%
\pgfpathcurveto{\pgfqpoint{1.663950in}{2.171305in}}{\pgfqpoint{1.656050in}{2.174577in}}{\pgfqpoint{1.647814in}{2.174577in}}%
\pgfpathcurveto{\pgfqpoint{1.639578in}{2.174577in}}{\pgfqpoint{1.631678in}{2.171305in}}{\pgfqpoint{1.625854in}{2.165481in}}%
\pgfpathcurveto{\pgfqpoint{1.620030in}{2.159657in}}{\pgfqpoint{1.616758in}{2.151757in}}{\pgfqpoint{1.616758in}{2.143520in}}%
\pgfpathcurveto{\pgfqpoint{1.616758in}{2.135284in}}{\pgfqpoint{1.620030in}{2.127384in}}{\pgfqpoint{1.625854in}{2.121560in}}%
\pgfpathcurveto{\pgfqpoint{1.631678in}{2.115736in}}{\pgfqpoint{1.639578in}{2.112464in}}{\pgfqpoint{1.647814in}{2.112464in}}%
\pgfpathclose%
\pgfusepath{stroke,fill}%
\end{pgfscope}%
\begin{pgfscope}%
\pgfpathrectangle{\pgfqpoint{0.100000in}{0.212622in}}{\pgfqpoint{3.696000in}{3.696000in}}%
\pgfusepath{clip}%
\pgfsetbuttcap%
\pgfsetroundjoin%
\definecolor{currentfill}{rgb}{0.121569,0.466667,0.705882}%
\pgfsetfillcolor{currentfill}%
\pgfsetfillopacity{0.300651}%
\pgfsetlinewidth{1.003750pt}%
\definecolor{currentstroke}{rgb}{0.121569,0.466667,0.705882}%
\pgfsetstrokecolor{currentstroke}%
\pgfsetstrokeopacity{0.300651}%
\pgfsetdash{}{0pt}%
\pgfpathmoveto{\pgfqpoint{1.647814in}{2.112464in}}%
\pgfpathcurveto{\pgfqpoint{1.656050in}{2.112464in}}{\pgfqpoint{1.663950in}{2.115736in}}{\pgfqpoint{1.669774in}{2.121560in}}%
\pgfpathcurveto{\pgfqpoint{1.675598in}{2.127384in}}{\pgfqpoint{1.678871in}{2.135284in}}{\pgfqpoint{1.678871in}{2.143520in}}%
\pgfpathcurveto{\pgfqpoint{1.678871in}{2.151757in}}{\pgfqpoint{1.675598in}{2.159657in}}{\pgfqpoint{1.669774in}{2.165481in}}%
\pgfpathcurveto{\pgfqpoint{1.663950in}{2.171305in}}{\pgfqpoint{1.656050in}{2.174577in}}{\pgfqpoint{1.647814in}{2.174577in}}%
\pgfpathcurveto{\pgfqpoint{1.639578in}{2.174577in}}{\pgfqpoint{1.631678in}{2.171305in}}{\pgfqpoint{1.625854in}{2.165481in}}%
\pgfpathcurveto{\pgfqpoint{1.620030in}{2.159657in}}{\pgfqpoint{1.616758in}{2.151757in}}{\pgfqpoint{1.616758in}{2.143520in}}%
\pgfpathcurveto{\pgfqpoint{1.616758in}{2.135284in}}{\pgfqpoint{1.620030in}{2.127384in}}{\pgfqpoint{1.625854in}{2.121560in}}%
\pgfpathcurveto{\pgfqpoint{1.631678in}{2.115736in}}{\pgfqpoint{1.639578in}{2.112464in}}{\pgfqpoint{1.647814in}{2.112464in}}%
\pgfpathclose%
\pgfusepath{stroke,fill}%
\end{pgfscope}%
\begin{pgfscope}%
\pgfpathrectangle{\pgfqpoint{0.100000in}{0.212622in}}{\pgfqpoint{3.696000in}{3.696000in}}%
\pgfusepath{clip}%
\pgfsetbuttcap%
\pgfsetroundjoin%
\definecolor{currentfill}{rgb}{0.121569,0.466667,0.705882}%
\pgfsetfillcolor{currentfill}%
\pgfsetfillopacity{0.300651}%
\pgfsetlinewidth{1.003750pt}%
\definecolor{currentstroke}{rgb}{0.121569,0.466667,0.705882}%
\pgfsetstrokecolor{currentstroke}%
\pgfsetstrokeopacity{0.300651}%
\pgfsetdash{}{0pt}%
\pgfpathmoveto{\pgfqpoint{1.647814in}{2.112464in}}%
\pgfpathcurveto{\pgfqpoint{1.656050in}{2.112464in}}{\pgfqpoint{1.663950in}{2.115736in}}{\pgfqpoint{1.669774in}{2.121560in}}%
\pgfpathcurveto{\pgfqpoint{1.675598in}{2.127384in}}{\pgfqpoint{1.678871in}{2.135284in}}{\pgfqpoint{1.678871in}{2.143520in}}%
\pgfpathcurveto{\pgfqpoint{1.678871in}{2.151757in}}{\pgfqpoint{1.675598in}{2.159657in}}{\pgfqpoint{1.669774in}{2.165481in}}%
\pgfpathcurveto{\pgfqpoint{1.663950in}{2.171305in}}{\pgfqpoint{1.656050in}{2.174577in}}{\pgfqpoint{1.647814in}{2.174577in}}%
\pgfpathcurveto{\pgfqpoint{1.639578in}{2.174577in}}{\pgfqpoint{1.631678in}{2.171305in}}{\pgfqpoint{1.625854in}{2.165481in}}%
\pgfpathcurveto{\pgfqpoint{1.620030in}{2.159657in}}{\pgfqpoint{1.616758in}{2.151757in}}{\pgfqpoint{1.616758in}{2.143520in}}%
\pgfpathcurveto{\pgfqpoint{1.616758in}{2.135284in}}{\pgfqpoint{1.620030in}{2.127384in}}{\pgfqpoint{1.625854in}{2.121560in}}%
\pgfpathcurveto{\pgfqpoint{1.631678in}{2.115736in}}{\pgfqpoint{1.639578in}{2.112464in}}{\pgfqpoint{1.647814in}{2.112464in}}%
\pgfpathclose%
\pgfusepath{stroke,fill}%
\end{pgfscope}%
\begin{pgfscope}%
\pgfpathrectangle{\pgfqpoint{0.100000in}{0.212622in}}{\pgfqpoint{3.696000in}{3.696000in}}%
\pgfusepath{clip}%
\pgfsetbuttcap%
\pgfsetroundjoin%
\definecolor{currentfill}{rgb}{0.121569,0.466667,0.705882}%
\pgfsetfillcolor{currentfill}%
\pgfsetfillopacity{0.300651}%
\pgfsetlinewidth{1.003750pt}%
\definecolor{currentstroke}{rgb}{0.121569,0.466667,0.705882}%
\pgfsetstrokecolor{currentstroke}%
\pgfsetstrokeopacity{0.300651}%
\pgfsetdash{}{0pt}%
\pgfpathmoveto{\pgfqpoint{1.647814in}{2.112464in}}%
\pgfpathcurveto{\pgfqpoint{1.656050in}{2.112464in}}{\pgfqpoint{1.663950in}{2.115736in}}{\pgfqpoint{1.669774in}{2.121560in}}%
\pgfpathcurveto{\pgfqpoint{1.675598in}{2.127384in}}{\pgfqpoint{1.678871in}{2.135284in}}{\pgfqpoint{1.678871in}{2.143520in}}%
\pgfpathcurveto{\pgfqpoint{1.678871in}{2.151757in}}{\pgfqpoint{1.675598in}{2.159657in}}{\pgfqpoint{1.669774in}{2.165481in}}%
\pgfpathcurveto{\pgfqpoint{1.663950in}{2.171305in}}{\pgfqpoint{1.656050in}{2.174577in}}{\pgfqpoint{1.647814in}{2.174577in}}%
\pgfpathcurveto{\pgfqpoint{1.639578in}{2.174577in}}{\pgfqpoint{1.631678in}{2.171305in}}{\pgfqpoint{1.625854in}{2.165481in}}%
\pgfpathcurveto{\pgfqpoint{1.620030in}{2.159657in}}{\pgfqpoint{1.616758in}{2.151757in}}{\pgfqpoint{1.616758in}{2.143520in}}%
\pgfpathcurveto{\pgfqpoint{1.616758in}{2.135284in}}{\pgfqpoint{1.620030in}{2.127384in}}{\pgfqpoint{1.625854in}{2.121560in}}%
\pgfpathcurveto{\pgfqpoint{1.631678in}{2.115736in}}{\pgfqpoint{1.639578in}{2.112464in}}{\pgfqpoint{1.647814in}{2.112464in}}%
\pgfpathclose%
\pgfusepath{stroke,fill}%
\end{pgfscope}%
\begin{pgfscope}%
\pgfpathrectangle{\pgfqpoint{0.100000in}{0.212622in}}{\pgfqpoint{3.696000in}{3.696000in}}%
\pgfusepath{clip}%
\pgfsetbuttcap%
\pgfsetroundjoin%
\definecolor{currentfill}{rgb}{0.121569,0.466667,0.705882}%
\pgfsetfillcolor{currentfill}%
\pgfsetfillopacity{0.300651}%
\pgfsetlinewidth{1.003750pt}%
\definecolor{currentstroke}{rgb}{0.121569,0.466667,0.705882}%
\pgfsetstrokecolor{currentstroke}%
\pgfsetstrokeopacity{0.300651}%
\pgfsetdash{}{0pt}%
\pgfpathmoveto{\pgfqpoint{1.647814in}{2.112464in}}%
\pgfpathcurveto{\pgfqpoint{1.656050in}{2.112464in}}{\pgfqpoint{1.663950in}{2.115736in}}{\pgfqpoint{1.669774in}{2.121560in}}%
\pgfpathcurveto{\pgfqpoint{1.675598in}{2.127384in}}{\pgfqpoint{1.678871in}{2.135284in}}{\pgfqpoint{1.678871in}{2.143520in}}%
\pgfpathcurveto{\pgfqpoint{1.678871in}{2.151757in}}{\pgfqpoint{1.675598in}{2.159657in}}{\pgfqpoint{1.669774in}{2.165481in}}%
\pgfpathcurveto{\pgfqpoint{1.663950in}{2.171305in}}{\pgfqpoint{1.656050in}{2.174577in}}{\pgfqpoint{1.647814in}{2.174577in}}%
\pgfpathcurveto{\pgfqpoint{1.639578in}{2.174577in}}{\pgfqpoint{1.631678in}{2.171305in}}{\pgfqpoint{1.625854in}{2.165481in}}%
\pgfpathcurveto{\pgfqpoint{1.620030in}{2.159657in}}{\pgfqpoint{1.616758in}{2.151757in}}{\pgfqpoint{1.616758in}{2.143520in}}%
\pgfpathcurveto{\pgfqpoint{1.616758in}{2.135284in}}{\pgfqpoint{1.620030in}{2.127384in}}{\pgfqpoint{1.625854in}{2.121560in}}%
\pgfpathcurveto{\pgfqpoint{1.631678in}{2.115736in}}{\pgfqpoint{1.639578in}{2.112464in}}{\pgfqpoint{1.647814in}{2.112464in}}%
\pgfpathclose%
\pgfusepath{stroke,fill}%
\end{pgfscope}%
\begin{pgfscope}%
\pgfpathrectangle{\pgfqpoint{0.100000in}{0.212622in}}{\pgfqpoint{3.696000in}{3.696000in}}%
\pgfusepath{clip}%
\pgfsetbuttcap%
\pgfsetroundjoin%
\definecolor{currentfill}{rgb}{0.121569,0.466667,0.705882}%
\pgfsetfillcolor{currentfill}%
\pgfsetfillopacity{0.300651}%
\pgfsetlinewidth{1.003750pt}%
\definecolor{currentstroke}{rgb}{0.121569,0.466667,0.705882}%
\pgfsetstrokecolor{currentstroke}%
\pgfsetstrokeopacity{0.300651}%
\pgfsetdash{}{0pt}%
\pgfpathmoveto{\pgfqpoint{1.647814in}{2.112464in}}%
\pgfpathcurveto{\pgfqpoint{1.656050in}{2.112464in}}{\pgfqpoint{1.663950in}{2.115736in}}{\pgfqpoint{1.669774in}{2.121560in}}%
\pgfpathcurveto{\pgfqpoint{1.675598in}{2.127384in}}{\pgfqpoint{1.678871in}{2.135284in}}{\pgfqpoint{1.678871in}{2.143520in}}%
\pgfpathcurveto{\pgfqpoint{1.678871in}{2.151757in}}{\pgfqpoint{1.675598in}{2.159657in}}{\pgfqpoint{1.669774in}{2.165481in}}%
\pgfpathcurveto{\pgfqpoint{1.663950in}{2.171305in}}{\pgfqpoint{1.656050in}{2.174577in}}{\pgfqpoint{1.647814in}{2.174577in}}%
\pgfpathcurveto{\pgfqpoint{1.639578in}{2.174577in}}{\pgfqpoint{1.631678in}{2.171305in}}{\pgfqpoint{1.625854in}{2.165481in}}%
\pgfpathcurveto{\pgfqpoint{1.620030in}{2.159657in}}{\pgfqpoint{1.616758in}{2.151757in}}{\pgfqpoint{1.616758in}{2.143520in}}%
\pgfpathcurveto{\pgfqpoint{1.616758in}{2.135284in}}{\pgfqpoint{1.620030in}{2.127384in}}{\pgfqpoint{1.625854in}{2.121560in}}%
\pgfpathcurveto{\pgfqpoint{1.631678in}{2.115736in}}{\pgfqpoint{1.639578in}{2.112464in}}{\pgfqpoint{1.647814in}{2.112464in}}%
\pgfpathclose%
\pgfusepath{stroke,fill}%
\end{pgfscope}%
\begin{pgfscope}%
\pgfpathrectangle{\pgfqpoint{0.100000in}{0.212622in}}{\pgfqpoint{3.696000in}{3.696000in}}%
\pgfusepath{clip}%
\pgfsetbuttcap%
\pgfsetroundjoin%
\definecolor{currentfill}{rgb}{0.121569,0.466667,0.705882}%
\pgfsetfillcolor{currentfill}%
\pgfsetfillopacity{0.300651}%
\pgfsetlinewidth{1.003750pt}%
\definecolor{currentstroke}{rgb}{0.121569,0.466667,0.705882}%
\pgfsetstrokecolor{currentstroke}%
\pgfsetstrokeopacity{0.300651}%
\pgfsetdash{}{0pt}%
\pgfpathmoveto{\pgfqpoint{1.647814in}{2.112464in}}%
\pgfpathcurveto{\pgfqpoint{1.656050in}{2.112464in}}{\pgfqpoint{1.663950in}{2.115736in}}{\pgfqpoint{1.669774in}{2.121560in}}%
\pgfpathcurveto{\pgfqpoint{1.675598in}{2.127384in}}{\pgfqpoint{1.678871in}{2.135284in}}{\pgfqpoint{1.678871in}{2.143520in}}%
\pgfpathcurveto{\pgfqpoint{1.678871in}{2.151757in}}{\pgfqpoint{1.675598in}{2.159657in}}{\pgfqpoint{1.669774in}{2.165481in}}%
\pgfpathcurveto{\pgfqpoint{1.663950in}{2.171305in}}{\pgfqpoint{1.656050in}{2.174577in}}{\pgfqpoint{1.647814in}{2.174577in}}%
\pgfpathcurveto{\pgfqpoint{1.639578in}{2.174577in}}{\pgfqpoint{1.631678in}{2.171305in}}{\pgfqpoint{1.625854in}{2.165481in}}%
\pgfpathcurveto{\pgfqpoint{1.620030in}{2.159657in}}{\pgfqpoint{1.616758in}{2.151757in}}{\pgfqpoint{1.616758in}{2.143520in}}%
\pgfpathcurveto{\pgfqpoint{1.616758in}{2.135284in}}{\pgfqpoint{1.620030in}{2.127384in}}{\pgfqpoint{1.625854in}{2.121560in}}%
\pgfpathcurveto{\pgfqpoint{1.631678in}{2.115736in}}{\pgfqpoint{1.639578in}{2.112464in}}{\pgfqpoint{1.647814in}{2.112464in}}%
\pgfpathclose%
\pgfusepath{stroke,fill}%
\end{pgfscope}%
\begin{pgfscope}%
\pgfpathrectangle{\pgfqpoint{0.100000in}{0.212622in}}{\pgfqpoint{3.696000in}{3.696000in}}%
\pgfusepath{clip}%
\pgfsetbuttcap%
\pgfsetroundjoin%
\definecolor{currentfill}{rgb}{0.121569,0.466667,0.705882}%
\pgfsetfillcolor{currentfill}%
\pgfsetfillopacity{0.300651}%
\pgfsetlinewidth{1.003750pt}%
\definecolor{currentstroke}{rgb}{0.121569,0.466667,0.705882}%
\pgfsetstrokecolor{currentstroke}%
\pgfsetstrokeopacity{0.300651}%
\pgfsetdash{}{0pt}%
\pgfpathmoveto{\pgfqpoint{1.647814in}{2.112464in}}%
\pgfpathcurveto{\pgfqpoint{1.656050in}{2.112464in}}{\pgfqpoint{1.663950in}{2.115736in}}{\pgfqpoint{1.669774in}{2.121560in}}%
\pgfpathcurveto{\pgfqpoint{1.675598in}{2.127384in}}{\pgfqpoint{1.678871in}{2.135284in}}{\pgfqpoint{1.678871in}{2.143520in}}%
\pgfpathcurveto{\pgfqpoint{1.678871in}{2.151757in}}{\pgfqpoint{1.675598in}{2.159657in}}{\pgfqpoint{1.669774in}{2.165481in}}%
\pgfpathcurveto{\pgfqpoint{1.663950in}{2.171305in}}{\pgfqpoint{1.656050in}{2.174577in}}{\pgfqpoint{1.647814in}{2.174577in}}%
\pgfpathcurveto{\pgfqpoint{1.639578in}{2.174577in}}{\pgfqpoint{1.631678in}{2.171305in}}{\pgfqpoint{1.625854in}{2.165481in}}%
\pgfpathcurveto{\pgfqpoint{1.620030in}{2.159657in}}{\pgfqpoint{1.616758in}{2.151757in}}{\pgfqpoint{1.616758in}{2.143520in}}%
\pgfpathcurveto{\pgfqpoint{1.616758in}{2.135284in}}{\pgfqpoint{1.620030in}{2.127384in}}{\pgfqpoint{1.625854in}{2.121560in}}%
\pgfpathcurveto{\pgfqpoint{1.631678in}{2.115736in}}{\pgfqpoint{1.639578in}{2.112464in}}{\pgfqpoint{1.647814in}{2.112464in}}%
\pgfpathclose%
\pgfusepath{stroke,fill}%
\end{pgfscope}%
\begin{pgfscope}%
\pgfpathrectangle{\pgfqpoint{0.100000in}{0.212622in}}{\pgfqpoint{3.696000in}{3.696000in}}%
\pgfusepath{clip}%
\pgfsetbuttcap%
\pgfsetroundjoin%
\definecolor{currentfill}{rgb}{0.121569,0.466667,0.705882}%
\pgfsetfillcolor{currentfill}%
\pgfsetfillopacity{0.300651}%
\pgfsetlinewidth{1.003750pt}%
\definecolor{currentstroke}{rgb}{0.121569,0.466667,0.705882}%
\pgfsetstrokecolor{currentstroke}%
\pgfsetstrokeopacity{0.300651}%
\pgfsetdash{}{0pt}%
\pgfpathmoveto{\pgfqpoint{1.647814in}{2.112464in}}%
\pgfpathcurveto{\pgfqpoint{1.656050in}{2.112464in}}{\pgfqpoint{1.663950in}{2.115736in}}{\pgfqpoint{1.669774in}{2.121560in}}%
\pgfpathcurveto{\pgfqpoint{1.675598in}{2.127384in}}{\pgfqpoint{1.678871in}{2.135284in}}{\pgfqpoint{1.678871in}{2.143520in}}%
\pgfpathcurveto{\pgfqpoint{1.678871in}{2.151757in}}{\pgfqpoint{1.675598in}{2.159657in}}{\pgfqpoint{1.669774in}{2.165481in}}%
\pgfpathcurveto{\pgfqpoint{1.663950in}{2.171305in}}{\pgfqpoint{1.656050in}{2.174577in}}{\pgfqpoint{1.647814in}{2.174577in}}%
\pgfpathcurveto{\pgfqpoint{1.639578in}{2.174577in}}{\pgfqpoint{1.631678in}{2.171305in}}{\pgfqpoint{1.625854in}{2.165481in}}%
\pgfpathcurveto{\pgfqpoint{1.620030in}{2.159657in}}{\pgfqpoint{1.616758in}{2.151757in}}{\pgfqpoint{1.616758in}{2.143520in}}%
\pgfpathcurveto{\pgfqpoint{1.616758in}{2.135284in}}{\pgfqpoint{1.620030in}{2.127384in}}{\pgfqpoint{1.625854in}{2.121560in}}%
\pgfpathcurveto{\pgfqpoint{1.631678in}{2.115736in}}{\pgfqpoint{1.639578in}{2.112464in}}{\pgfqpoint{1.647814in}{2.112464in}}%
\pgfpathclose%
\pgfusepath{stroke,fill}%
\end{pgfscope}%
\begin{pgfscope}%
\pgfpathrectangle{\pgfqpoint{0.100000in}{0.212622in}}{\pgfqpoint{3.696000in}{3.696000in}}%
\pgfusepath{clip}%
\pgfsetbuttcap%
\pgfsetroundjoin%
\definecolor{currentfill}{rgb}{0.121569,0.466667,0.705882}%
\pgfsetfillcolor{currentfill}%
\pgfsetfillopacity{0.300651}%
\pgfsetlinewidth{1.003750pt}%
\definecolor{currentstroke}{rgb}{0.121569,0.466667,0.705882}%
\pgfsetstrokecolor{currentstroke}%
\pgfsetstrokeopacity{0.300651}%
\pgfsetdash{}{0pt}%
\pgfpathmoveto{\pgfqpoint{1.647814in}{2.112464in}}%
\pgfpathcurveto{\pgfqpoint{1.656050in}{2.112464in}}{\pgfqpoint{1.663950in}{2.115736in}}{\pgfqpoint{1.669774in}{2.121560in}}%
\pgfpathcurveto{\pgfqpoint{1.675598in}{2.127384in}}{\pgfqpoint{1.678871in}{2.135284in}}{\pgfqpoint{1.678871in}{2.143520in}}%
\pgfpathcurveto{\pgfqpoint{1.678871in}{2.151757in}}{\pgfqpoint{1.675598in}{2.159657in}}{\pgfqpoint{1.669774in}{2.165481in}}%
\pgfpathcurveto{\pgfqpoint{1.663950in}{2.171305in}}{\pgfqpoint{1.656050in}{2.174577in}}{\pgfqpoint{1.647814in}{2.174577in}}%
\pgfpathcurveto{\pgfqpoint{1.639578in}{2.174577in}}{\pgfqpoint{1.631678in}{2.171305in}}{\pgfqpoint{1.625854in}{2.165481in}}%
\pgfpathcurveto{\pgfqpoint{1.620030in}{2.159657in}}{\pgfqpoint{1.616758in}{2.151757in}}{\pgfqpoint{1.616758in}{2.143520in}}%
\pgfpathcurveto{\pgfqpoint{1.616758in}{2.135284in}}{\pgfqpoint{1.620030in}{2.127384in}}{\pgfqpoint{1.625854in}{2.121560in}}%
\pgfpathcurveto{\pgfqpoint{1.631678in}{2.115736in}}{\pgfqpoint{1.639578in}{2.112464in}}{\pgfqpoint{1.647814in}{2.112464in}}%
\pgfpathclose%
\pgfusepath{stroke,fill}%
\end{pgfscope}%
\begin{pgfscope}%
\pgfpathrectangle{\pgfqpoint{0.100000in}{0.212622in}}{\pgfqpoint{3.696000in}{3.696000in}}%
\pgfusepath{clip}%
\pgfsetbuttcap%
\pgfsetroundjoin%
\definecolor{currentfill}{rgb}{0.121569,0.466667,0.705882}%
\pgfsetfillcolor{currentfill}%
\pgfsetfillopacity{0.300651}%
\pgfsetlinewidth{1.003750pt}%
\definecolor{currentstroke}{rgb}{0.121569,0.466667,0.705882}%
\pgfsetstrokecolor{currentstroke}%
\pgfsetstrokeopacity{0.300651}%
\pgfsetdash{}{0pt}%
\pgfpathmoveto{\pgfqpoint{1.647814in}{2.112464in}}%
\pgfpathcurveto{\pgfqpoint{1.656050in}{2.112464in}}{\pgfqpoint{1.663950in}{2.115736in}}{\pgfqpoint{1.669774in}{2.121560in}}%
\pgfpathcurveto{\pgfqpoint{1.675598in}{2.127384in}}{\pgfqpoint{1.678871in}{2.135284in}}{\pgfqpoint{1.678871in}{2.143520in}}%
\pgfpathcurveto{\pgfqpoint{1.678871in}{2.151757in}}{\pgfqpoint{1.675598in}{2.159657in}}{\pgfqpoint{1.669774in}{2.165481in}}%
\pgfpathcurveto{\pgfqpoint{1.663950in}{2.171305in}}{\pgfqpoint{1.656050in}{2.174577in}}{\pgfqpoint{1.647814in}{2.174577in}}%
\pgfpathcurveto{\pgfqpoint{1.639578in}{2.174577in}}{\pgfqpoint{1.631678in}{2.171305in}}{\pgfqpoint{1.625854in}{2.165481in}}%
\pgfpathcurveto{\pgfqpoint{1.620030in}{2.159657in}}{\pgfqpoint{1.616758in}{2.151757in}}{\pgfqpoint{1.616758in}{2.143520in}}%
\pgfpathcurveto{\pgfqpoint{1.616758in}{2.135284in}}{\pgfqpoint{1.620030in}{2.127384in}}{\pgfqpoint{1.625854in}{2.121560in}}%
\pgfpathcurveto{\pgfqpoint{1.631678in}{2.115736in}}{\pgfqpoint{1.639578in}{2.112464in}}{\pgfqpoint{1.647814in}{2.112464in}}%
\pgfpathclose%
\pgfusepath{stroke,fill}%
\end{pgfscope}%
\begin{pgfscope}%
\pgfpathrectangle{\pgfqpoint{0.100000in}{0.212622in}}{\pgfqpoint{3.696000in}{3.696000in}}%
\pgfusepath{clip}%
\pgfsetbuttcap%
\pgfsetroundjoin%
\definecolor{currentfill}{rgb}{0.121569,0.466667,0.705882}%
\pgfsetfillcolor{currentfill}%
\pgfsetfillopacity{0.300651}%
\pgfsetlinewidth{1.003750pt}%
\definecolor{currentstroke}{rgb}{0.121569,0.466667,0.705882}%
\pgfsetstrokecolor{currentstroke}%
\pgfsetstrokeopacity{0.300651}%
\pgfsetdash{}{0pt}%
\pgfpathmoveto{\pgfqpoint{1.647814in}{2.112464in}}%
\pgfpathcurveto{\pgfqpoint{1.656050in}{2.112464in}}{\pgfqpoint{1.663950in}{2.115736in}}{\pgfqpoint{1.669774in}{2.121560in}}%
\pgfpathcurveto{\pgfqpoint{1.675598in}{2.127384in}}{\pgfqpoint{1.678871in}{2.135284in}}{\pgfqpoint{1.678871in}{2.143520in}}%
\pgfpathcurveto{\pgfqpoint{1.678871in}{2.151757in}}{\pgfqpoint{1.675598in}{2.159657in}}{\pgfqpoint{1.669774in}{2.165481in}}%
\pgfpathcurveto{\pgfqpoint{1.663950in}{2.171305in}}{\pgfqpoint{1.656050in}{2.174577in}}{\pgfqpoint{1.647814in}{2.174577in}}%
\pgfpathcurveto{\pgfqpoint{1.639578in}{2.174577in}}{\pgfqpoint{1.631678in}{2.171305in}}{\pgfqpoint{1.625854in}{2.165481in}}%
\pgfpathcurveto{\pgfqpoint{1.620030in}{2.159657in}}{\pgfqpoint{1.616758in}{2.151757in}}{\pgfqpoint{1.616758in}{2.143520in}}%
\pgfpathcurveto{\pgfqpoint{1.616758in}{2.135284in}}{\pgfqpoint{1.620030in}{2.127384in}}{\pgfqpoint{1.625854in}{2.121560in}}%
\pgfpathcurveto{\pgfqpoint{1.631678in}{2.115736in}}{\pgfqpoint{1.639578in}{2.112464in}}{\pgfqpoint{1.647814in}{2.112464in}}%
\pgfpathclose%
\pgfusepath{stroke,fill}%
\end{pgfscope}%
\begin{pgfscope}%
\pgfpathrectangle{\pgfqpoint{0.100000in}{0.212622in}}{\pgfqpoint{3.696000in}{3.696000in}}%
\pgfusepath{clip}%
\pgfsetbuttcap%
\pgfsetroundjoin%
\definecolor{currentfill}{rgb}{0.121569,0.466667,0.705882}%
\pgfsetfillcolor{currentfill}%
\pgfsetfillopacity{0.300651}%
\pgfsetlinewidth{1.003750pt}%
\definecolor{currentstroke}{rgb}{0.121569,0.466667,0.705882}%
\pgfsetstrokecolor{currentstroke}%
\pgfsetstrokeopacity{0.300651}%
\pgfsetdash{}{0pt}%
\pgfpathmoveto{\pgfqpoint{1.647814in}{2.112464in}}%
\pgfpathcurveto{\pgfqpoint{1.656050in}{2.112464in}}{\pgfqpoint{1.663950in}{2.115736in}}{\pgfqpoint{1.669774in}{2.121560in}}%
\pgfpathcurveto{\pgfqpoint{1.675598in}{2.127384in}}{\pgfqpoint{1.678871in}{2.135284in}}{\pgfqpoint{1.678871in}{2.143520in}}%
\pgfpathcurveto{\pgfqpoint{1.678871in}{2.151757in}}{\pgfqpoint{1.675598in}{2.159657in}}{\pgfqpoint{1.669774in}{2.165481in}}%
\pgfpathcurveto{\pgfqpoint{1.663950in}{2.171305in}}{\pgfqpoint{1.656050in}{2.174577in}}{\pgfqpoint{1.647814in}{2.174577in}}%
\pgfpathcurveto{\pgfqpoint{1.639578in}{2.174577in}}{\pgfqpoint{1.631678in}{2.171305in}}{\pgfqpoint{1.625854in}{2.165481in}}%
\pgfpathcurveto{\pgfqpoint{1.620030in}{2.159657in}}{\pgfqpoint{1.616758in}{2.151757in}}{\pgfqpoint{1.616758in}{2.143520in}}%
\pgfpathcurveto{\pgfqpoint{1.616758in}{2.135284in}}{\pgfqpoint{1.620030in}{2.127384in}}{\pgfqpoint{1.625854in}{2.121560in}}%
\pgfpathcurveto{\pgfqpoint{1.631678in}{2.115736in}}{\pgfqpoint{1.639578in}{2.112464in}}{\pgfqpoint{1.647814in}{2.112464in}}%
\pgfpathclose%
\pgfusepath{stroke,fill}%
\end{pgfscope}%
\begin{pgfscope}%
\pgfpathrectangle{\pgfqpoint{0.100000in}{0.212622in}}{\pgfqpoint{3.696000in}{3.696000in}}%
\pgfusepath{clip}%
\pgfsetbuttcap%
\pgfsetroundjoin%
\definecolor{currentfill}{rgb}{0.121569,0.466667,0.705882}%
\pgfsetfillcolor{currentfill}%
\pgfsetfillopacity{0.300651}%
\pgfsetlinewidth{1.003750pt}%
\definecolor{currentstroke}{rgb}{0.121569,0.466667,0.705882}%
\pgfsetstrokecolor{currentstroke}%
\pgfsetstrokeopacity{0.300651}%
\pgfsetdash{}{0pt}%
\pgfpathmoveto{\pgfqpoint{1.647814in}{2.112464in}}%
\pgfpathcurveto{\pgfqpoint{1.656050in}{2.112464in}}{\pgfqpoint{1.663950in}{2.115736in}}{\pgfqpoint{1.669774in}{2.121560in}}%
\pgfpathcurveto{\pgfqpoint{1.675598in}{2.127384in}}{\pgfqpoint{1.678871in}{2.135284in}}{\pgfqpoint{1.678871in}{2.143520in}}%
\pgfpathcurveto{\pgfqpoint{1.678871in}{2.151757in}}{\pgfqpoint{1.675598in}{2.159657in}}{\pgfqpoint{1.669774in}{2.165481in}}%
\pgfpathcurveto{\pgfqpoint{1.663950in}{2.171305in}}{\pgfqpoint{1.656050in}{2.174577in}}{\pgfqpoint{1.647814in}{2.174577in}}%
\pgfpathcurveto{\pgfqpoint{1.639578in}{2.174577in}}{\pgfqpoint{1.631678in}{2.171305in}}{\pgfqpoint{1.625854in}{2.165481in}}%
\pgfpathcurveto{\pgfqpoint{1.620030in}{2.159657in}}{\pgfqpoint{1.616758in}{2.151757in}}{\pgfqpoint{1.616758in}{2.143520in}}%
\pgfpathcurveto{\pgfqpoint{1.616758in}{2.135284in}}{\pgfqpoint{1.620030in}{2.127384in}}{\pgfqpoint{1.625854in}{2.121560in}}%
\pgfpathcurveto{\pgfqpoint{1.631678in}{2.115736in}}{\pgfqpoint{1.639578in}{2.112464in}}{\pgfqpoint{1.647814in}{2.112464in}}%
\pgfpathclose%
\pgfusepath{stroke,fill}%
\end{pgfscope}%
\begin{pgfscope}%
\pgfpathrectangle{\pgfqpoint{0.100000in}{0.212622in}}{\pgfqpoint{3.696000in}{3.696000in}}%
\pgfusepath{clip}%
\pgfsetbuttcap%
\pgfsetroundjoin%
\definecolor{currentfill}{rgb}{0.121569,0.466667,0.705882}%
\pgfsetfillcolor{currentfill}%
\pgfsetfillopacity{0.300651}%
\pgfsetlinewidth{1.003750pt}%
\definecolor{currentstroke}{rgb}{0.121569,0.466667,0.705882}%
\pgfsetstrokecolor{currentstroke}%
\pgfsetstrokeopacity{0.300651}%
\pgfsetdash{}{0pt}%
\pgfpathmoveto{\pgfqpoint{1.647814in}{2.112464in}}%
\pgfpathcurveto{\pgfqpoint{1.656050in}{2.112464in}}{\pgfqpoint{1.663950in}{2.115736in}}{\pgfqpoint{1.669774in}{2.121560in}}%
\pgfpathcurveto{\pgfqpoint{1.675598in}{2.127384in}}{\pgfqpoint{1.678871in}{2.135284in}}{\pgfqpoint{1.678871in}{2.143520in}}%
\pgfpathcurveto{\pgfqpoint{1.678871in}{2.151757in}}{\pgfqpoint{1.675598in}{2.159657in}}{\pgfqpoint{1.669774in}{2.165481in}}%
\pgfpathcurveto{\pgfqpoint{1.663950in}{2.171305in}}{\pgfqpoint{1.656050in}{2.174577in}}{\pgfqpoint{1.647814in}{2.174577in}}%
\pgfpathcurveto{\pgfqpoint{1.639578in}{2.174577in}}{\pgfqpoint{1.631678in}{2.171305in}}{\pgfqpoint{1.625854in}{2.165481in}}%
\pgfpathcurveto{\pgfqpoint{1.620030in}{2.159657in}}{\pgfqpoint{1.616758in}{2.151757in}}{\pgfqpoint{1.616758in}{2.143520in}}%
\pgfpathcurveto{\pgfqpoint{1.616758in}{2.135284in}}{\pgfqpoint{1.620030in}{2.127384in}}{\pgfqpoint{1.625854in}{2.121560in}}%
\pgfpathcurveto{\pgfqpoint{1.631678in}{2.115736in}}{\pgfqpoint{1.639578in}{2.112464in}}{\pgfqpoint{1.647814in}{2.112464in}}%
\pgfpathclose%
\pgfusepath{stroke,fill}%
\end{pgfscope}%
\begin{pgfscope}%
\pgfpathrectangle{\pgfqpoint{0.100000in}{0.212622in}}{\pgfqpoint{3.696000in}{3.696000in}}%
\pgfusepath{clip}%
\pgfsetbuttcap%
\pgfsetroundjoin%
\definecolor{currentfill}{rgb}{0.121569,0.466667,0.705882}%
\pgfsetfillcolor{currentfill}%
\pgfsetfillopacity{0.300651}%
\pgfsetlinewidth{1.003750pt}%
\definecolor{currentstroke}{rgb}{0.121569,0.466667,0.705882}%
\pgfsetstrokecolor{currentstroke}%
\pgfsetstrokeopacity{0.300651}%
\pgfsetdash{}{0pt}%
\pgfpathmoveto{\pgfqpoint{1.647814in}{2.112464in}}%
\pgfpathcurveto{\pgfqpoint{1.656050in}{2.112464in}}{\pgfqpoint{1.663950in}{2.115736in}}{\pgfqpoint{1.669774in}{2.121560in}}%
\pgfpathcurveto{\pgfqpoint{1.675598in}{2.127384in}}{\pgfqpoint{1.678871in}{2.135284in}}{\pgfqpoint{1.678871in}{2.143520in}}%
\pgfpathcurveto{\pgfqpoint{1.678871in}{2.151757in}}{\pgfqpoint{1.675598in}{2.159657in}}{\pgfqpoint{1.669774in}{2.165481in}}%
\pgfpathcurveto{\pgfqpoint{1.663950in}{2.171305in}}{\pgfqpoint{1.656050in}{2.174577in}}{\pgfqpoint{1.647814in}{2.174577in}}%
\pgfpathcurveto{\pgfqpoint{1.639578in}{2.174577in}}{\pgfqpoint{1.631678in}{2.171305in}}{\pgfqpoint{1.625854in}{2.165481in}}%
\pgfpathcurveto{\pgfqpoint{1.620030in}{2.159657in}}{\pgfqpoint{1.616758in}{2.151757in}}{\pgfqpoint{1.616758in}{2.143520in}}%
\pgfpathcurveto{\pgfqpoint{1.616758in}{2.135284in}}{\pgfqpoint{1.620030in}{2.127384in}}{\pgfqpoint{1.625854in}{2.121560in}}%
\pgfpathcurveto{\pgfqpoint{1.631678in}{2.115736in}}{\pgfqpoint{1.639578in}{2.112464in}}{\pgfqpoint{1.647814in}{2.112464in}}%
\pgfpathclose%
\pgfusepath{stroke,fill}%
\end{pgfscope}%
\begin{pgfscope}%
\pgfpathrectangle{\pgfqpoint{0.100000in}{0.212622in}}{\pgfqpoint{3.696000in}{3.696000in}}%
\pgfusepath{clip}%
\pgfsetbuttcap%
\pgfsetroundjoin%
\definecolor{currentfill}{rgb}{0.121569,0.466667,0.705882}%
\pgfsetfillcolor{currentfill}%
\pgfsetfillopacity{0.300651}%
\pgfsetlinewidth{1.003750pt}%
\definecolor{currentstroke}{rgb}{0.121569,0.466667,0.705882}%
\pgfsetstrokecolor{currentstroke}%
\pgfsetstrokeopacity{0.300651}%
\pgfsetdash{}{0pt}%
\pgfpathmoveto{\pgfqpoint{1.647814in}{2.112464in}}%
\pgfpathcurveto{\pgfqpoint{1.656050in}{2.112464in}}{\pgfqpoint{1.663950in}{2.115736in}}{\pgfqpoint{1.669774in}{2.121560in}}%
\pgfpathcurveto{\pgfqpoint{1.675598in}{2.127384in}}{\pgfqpoint{1.678871in}{2.135284in}}{\pgfqpoint{1.678871in}{2.143520in}}%
\pgfpathcurveto{\pgfqpoint{1.678871in}{2.151757in}}{\pgfqpoint{1.675598in}{2.159657in}}{\pgfqpoint{1.669774in}{2.165481in}}%
\pgfpathcurveto{\pgfqpoint{1.663950in}{2.171305in}}{\pgfqpoint{1.656050in}{2.174577in}}{\pgfqpoint{1.647814in}{2.174577in}}%
\pgfpathcurveto{\pgfqpoint{1.639578in}{2.174577in}}{\pgfqpoint{1.631678in}{2.171305in}}{\pgfqpoint{1.625854in}{2.165481in}}%
\pgfpathcurveto{\pgfqpoint{1.620030in}{2.159657in}}{\pgfqpoint{1.616758in}{2.151757in}}{\pgfqpoint{1.616758in}{2.143520in}}%
\pgfpathcurveto{\pgfqpoint{1.616758in}{2.135284in}}{\pgfqpoint{1.620030in}{2.127384in}}{\pgfqpoint{1.625854in}{2.121560in}}%
\pgfpathcurveto{\pgfqpoint{1.631678in}{2.115736in}}{\pgfqpoint{1.639578in}{2.112464in}}{\pgfqpoint{1.647814in}{2.112464in}}%
\pgfpathclose%
\pgfusepath{stroke,fill}%
\end{pgfscope}%
\begin{pgfscope}%
\pgfpathrectangle{\pgfqpoint{0.100000in}{0.212622in}}{\pgfqpoint{3.696000in}{3.696000in}}%
\pgfusepath{clip}%
\pgfsetbuttcap%
\pgfsetroundjoin%
\definecolor{currentfill}{rgb}{0.121569,0.466667,0.705882}%
\pgfsetfillcolor{currentfill}%
\pgfsetfillopacity{0.300651}%
\pgfsetlinewidth{1.003750pt}%
\definecolor{currentstroke}{rgb}{0.121569,0.466667,0.705882}%
\pgfsetstrokecolor{currentstroke}%
\pgfsetstrokeopacity{0.300651}%
\pgfsetdash{}{0pt}%
\pgfpathmoveto{\pgfqpoint{1.647814in}{2.112464in}}%
\pgfpathcurveto{\pgfqpoint{1.656050in}{2.112464in}}{\pgfqpoint{1.663950in}{2.115736in}}{\pgfqpoint{1.669774in}{2.121560in}}%
\pgfpathcurveto{\pgfqpoint{1.675598in}{2.127384in}}{\pgfqpoint{1.678871in}{2.135284in}}{\pgfqpoint{1.678871in}{2.143520in}}%
\pgfpathcurveto{\pgfqpoint{1.678871in}{2.151757in}}{\pgfqpoint{1.675598in}{2.159657in}}{\pgfqpoint{1.669774in}{2.165481in}}%
\pgfpathcurveto{\pgfqpoint{1.663950in}{2.171305in}}{\pgfqpoint{1.656050in}{2.174577in}}{\pgfqpoint{1.647814in}{2.174577in}}%
\pgfpathcurveto{\pgfqpoint{1.639578in}{2.174577in}}{\pgfqpoint{1.631678in}{2.171305in}}{\pgfqpoint{1.625854in}{2.165481in}}%
\pgfpathcurveto{\pgfqpoint{1.620030in}{2.159657in}}{\pgfqpoint{1.616758in}{2.151757in}}{\pgfqpoint{1.616758in}{2.143520in}}%
\pgfpathcurveto{\pgfqpoint{1.616758in}{2.135284in}}{\pgfqpoint{1.620030in}{2.127384in}}{\pgfqpoint{1.625854in}{2.121560in}}%
\pgfpathcurveto{\pgfqpoint{1.631678in}{2.115736in}}{\pgfqpoint{1.639578in}{2.112464in}}{\pgfqpoint{1.647814in}{2.112464in}}%
\pgfpathclose%
\pgfusepath{stroke,fill}%
\end{pgfscope}%
\begin{pgfscope}%
\pgfpathrectangle{\pgfqpoint{0.100000in}{0.212622in}}{\pgfqpoint{3.696000in}{3.696000in}}%
\pgfusepath{clip}%
\pgfsetbuttcap%
\pgfsetroundjoin%
\definecolor{currentfill}{rgb}{0.121569,0.466667,0.705882}%
\pgfsetfillcolor{currentfill}%
\pgfsetfillopacity{0.300651}%
\pgfsetlinewidth{1.003750pt}%
\definecolor{currentstroke}{rgb}{0.121569,0.466667,0.705882}%
\pgfsetstrokecolor{currentstroke}%
\pgfsetstrokeopacity{0.300651}%
\pgfsetdash{}{0pt}%
\pgfpathmoveto{\pgfqpoint{1.647814in}{2.112464in}}%
\pgfpathcurveto{\pgfqpoint{1.656050in}{2.112464in}}{\pgfqpoint{1.663950in}{2.115736in}}{\pgfqpoint{1.669774in}{2.121560in}}%
\pgfpathcurveto{\pgfqpoint{1.675598in}{2.127384in}}{\pgfqpoint{1.678871in}{2.135284in}}{\pgfqpoint{1.678871in}{2.143520in}}%
\pgfpathcurveto{\pgfqpoint{1.678871in}{2.151757in}}{\pgfqpoint{1.675598in}{2.159657in}}{\pgfqpoint{1.669774in}{2.165481in}}%
\pgfpathcurveto{\pgfqpoint{1.663950in}{2.171305in}}{\pgfqpoint{1.656050in}{2.174577in}}{\pgfqpoint{1.647814in}{2.174577in}}%
\pgfpathcurveto{\pgfqpoint{1.639578in}{2.174577in}}{\pgfqpoint{1.631678in}{2.171305in}}{\pgfqpoint{1.625854in}{2.165481in}}%
\pgfpathcurveto{\pgfqpoint{1.620030in}{2.159657in}}{\pgfqpoint{1.616758in}{2.151757in}}{\pgfqpoint{1.616758in}{2.143520in}}%
\pgfpathcurveto{\pgfqpoint{1.616758in}{2.135284in}}{\pgfqpoint{1.620030in}{2.127384in}}{\pgfqpoint{1.625854in}{2.121560in}}%
\pgfpathcurveto{\pgfqpoint{1.631678in}{2.115736in}}{\pgfqpoint{1.639578in}{2.112464in}}{\pgfqpoint{1.647814in}{2.112464in}}%
\pgfpathclose%
\pgfusepath{stroke,fill}%
\end{pgfscope}%
\begin{pgfscope}%
\pgfpathrectangle{\pgfqpoint{0.100000in}{0.212622in}}{\pgfqpoint{3.696000in}{3.696000in}}%
\pgfusepath{clip}%
\pgfsetbuttcap%
\pgfsetroundjoin%
\definecolor{currentfill}{rgb}{0.121569,0.466667,0.705882}%
\pgfsetfillcolor{currentfill}%
\pgfsetfillopacity{0.300651}%
\pgfsetlinewidth{1.003750pt}%
\definecolor{currentstroke}{rgb}{0.121569,0.466667,0.705882}%
\pgfsetstrokecolor{currentstroke}%
\pgfsetstrokeopacity{0.300651}%
\pgfsetdash{}{0pt}%
\pgfpathmoveto{\pgfqpoint{1.647814in}{2.112464in}}%
\pgfpathcurveto{\pgfqpoint{1.656050in}{2.112464in}}{\pgfqpoint{1.663950in}{2.115736in}}{\pgfqpoint{1.669774in}{2.121560in}}%
\pgfpathcurveto{\pgfqpoint{1.675598in}{2.127384in}}{\pgfqpoint{1.678871in}{2.135284in}}{\pgfqpoint{1.678871in}{2.143520in}}%
\pgfpathcurveto{\pgfqpoint{1.678871in}{2.151757in}}{\pgfqpoint{1.675598in}{2.159657in}}{\pgfqpoint{1.669774in}{2.165481in}}%
\pgfpathcurveto{\pgfqpoint{1.663950in}{2.171305in}}{\pgfqpoint{1.656050in}{2.174577in}}{\pgfqpoint{1.647814in}{2.174577in}}%
\pgfpathcurveto{\pgfqpoint{1.639578in}{2.174577in}}{\pgfqpoint{1.631678in}{2.171305in}}{\pgfqpoint{1.625854in}{2.165481in}}%
\pgfpathcurveto{\pgfqpoint{1.620030in}{2.159657in}}{\pgfqpoint{1.616758in}{2.151757in}}{\pgfqpoint{1.616758in}{2.143520in}}%
\pgfpathcurveto{\pgfqpoint{1.616758in}{2.135284in}}{\pgfqpoint{1.620030in}{2.127384in}}{\pgfqpoint{1.625854in}{2.121560in}}%
\pgfpathcurveto{\pgfqpoint{1.631678in}{2.115736in}}{\pgfqpoint{1.639578in}{2.112464in}}{\pgfqpoint{1.647814in}{2.112464in}}%
\pgfpathclose%
\pgfusepath{stroke,fill}%
\end{pgfscope}%
\begin{pgfscope}%
\pgfpathrectangle{\pgfqpoint{0.100000in}{0.212622in}}{\pgfqpoint{3.696000in}{3.696000in}}%
\pgfusepath{clip}%
\pgfsetbuttcap%
\pgfsetroundjoin%
\definecolor{currentfill}{rgb}{0.121569,0.466667,0.705882}%
\pgfsetfillcolor{currentfill}%
\pgfsetfillopacity{0.300651}%
\pgfsetlinewidth{1.003750pt}%
\definecolor{currentstroke}{rgb}{0.121569,0.466667,0.705882}%
\pgfsetstrokecolor{currentstroke}%
\pgfsetstrokeopacity{0.300651}%
\pgfsetdash{}{0pt}%
\pgfpathmoveto{\pgfqpoint{1.647814in}{2.112464in}}%
\pgfpathcurveto{\pgfqpoint{1.656050in}{2.112464in}}{\pgfqpoint{1.663950in}{2.115736in}}{\pgfqpoint{1.669774in}{2.121560in}}%
\pgfpathcurveto{\pgfqpoint{1.675598in}{2.127384in}}{\pgfqpoint{1.678871in}{2.135284in}}{\pgfqpoint{1.678871in}{2.143520in}}%
\pgfpathcurveto{\pgfqpoint{1.678871in}{2.151757in}}{\pgfqpoint{1.675598in}{2.159657in}}{\pgfqpoint{1.669774in}{2.165481in}}%
\pgfpathcurveto{\pgfqpoint{1.663950in}{2.171305in}}{\pgfqpoint{1.656050in}{2.174577in}}{\pgfqpoint{1.647814in}{2.174577in}}%
\pgfpathcurveto{\pgfqpoint{1.639578in}{2.174577in}}{\pgfqpoint{1.631678in}{2.171305in}}{\pgfqpoint{1.625854in}{2.165481in}}%
\pgfpathcurveto{\pgfqpoint{1.620030in}{2.159657in}}{\pgfqpoint{1.616758in}{2.151757in}}{\pgfqpoint{1.616758in}{2.143520in}}%
\pgfpathcurveto{\pgfqpoint{1.616758in}{2.135284in}}{\pgfqpoint{1.620030in}{2.127384in}}{\pgfqpoint{1.625854in}{2.121560in}}%
\pgfpathcurveto{\pgfqpoint{1.631678in}{2.115736in}}{\pgfqpoint{1.639578in}{2.112464in}}{\pgfqpoint{1.647814in}{2.112464in}}%
\pgfpathclose%
\pgfusepath{stroke,fill}%
\end{pgfscope}%
\begin{pgfscope}%
\pgfpathrectangle{\pgfqpoint{0.100000in}{0.212622in}}{\pgfqpoint{3.696000in}{3.696000in}}%
\pgfusepath{clip}%
\pgfsetbuttcap%
\pgfsetroundjoin%
\definecolor{currentfill}{rgb}{0.121569,0.466667,0.705882}%
\pgfsetfillcolor{currentfill}%
\pgfsetfillopacity{0.300651}%
\pgfsetlinewidth{1.003750pt}%
\definecolor{currentstroke}{rgb}{0.121569,0.466667,0.705882}%
\pgfsetstrokecolor{currentstroke}%
\pgfsetstrokeopacity{0.300651}%
\pgfsetdash{}{0pt}%
\pgfpathmoveto{\pgfqpoint{1.647814in}{2.112464in}}%
\pgfpathcurveto{\pgfqpoint{1.656050in}{2.112464in}}{\pgfqpoint{1.663950in}{2.115736in}}{\pgfqpoint{1.669774in}{2.121560in}}%
\pgfpathcurveto{\pgfqpoint{1.675598in}{2.127384in}}{\pgfqpoint{1.678870in}{2.135284in}}{\pgfqpoint{1.678870in}{2.143520in}}%
\pgfpathcurveto{\pgfqpoint{1.678870in}{2.151757in}}{\pgfqpoint{1.675598in}{2.159657in}}{\pgfqpoint{1.669774in}{2.165480in}}%
\pgfpathcurveto{\pgfqpoint{1.663950in}{2.171304in}}{\pgfqpoint{1.656050in}{2.174577in}}{\pgfqpoint{1.647814in}{2.174577in}}%
\pgfpathcurveto{\pgfqpoint{1.639578in}{2.174577in}}{\pgfqpoint{1.631678in}{2.171304in}}{\pgfqpoint{1.625854in}{2.165480in}}%
\pgfpathcurveto{\pgfqpoint{1.620030in}{2.159657in}}{\pgfqpoint{1.616757in}{2.151757in}}{\pgfqpoint{1.616757in}{2.143520in}}%
\pgfpathcurveto{\pgfqpoint{1.616757in}{2.135284in}}{\pgfqpoint{1.620030in}{2.127384in}}{\pgfqpoint{1.625854in}{2.121560in}}%
\pgfpathcurveto{\pgfqpoint{1.631678in}{2.115736in}}{\pgfqpoint{1.639578in}{2.112464in}}{\pgfqpoint{1.647814in}{2.112464in}}%
\pgfpathclose%
\pgfusepath{stroke,fill}%
\end{pgfscope}%
\begin{pgfscope}%
\pgfpathrectangle{\pgfqpoint{0.100000in}{0.212622in}}{\pgfqpoint{3.696000in}{3.696000in}}%
\pgfusepath{clip}%
\pgfsetbuttcap%
\pgfsetroundjoin%
\definecolor{currentfill}{rgb}{0.121569,0.466667,0.705882}%
\pgfsetfillcolor{currentfill}%
\pgfsetfillopacity{0.300651}%
\pgfsetlinewidth{1.003750pt}%
\definecolor{currentstroke}{rgb}{0.121569,0.466667,0.705882}%
\pgfsetstrokecolor{currentstroke}%
\pgfsetstrokeopacity{0.300651}%
\pgfsetdash{}{0pt}%
\pgfpathmoveto{\pgfqpoint{1.647814in}{2.112464in}}%
\pgfpathcurveto{\pgfqpoint{1.656050in}{2.112464in}}{\pgfqpoint{1.663950in}{2.115736in}}{\pgfqpoint{1.669774in}{2.121560in}}%
\pgfpathcurveto{\pgfqpoint{1.675598in}{2.127384in}}{\pgfqpoint{1.678870in}{2.135284in}}{\pgfqpoint{1.678870in}{2.143520in}}%
\pgfpathcurveto{\pgfqpoint{1.678870in}{2.151756in}}{\pgfqpoint{1.675598in}{2.159657in}}{\pgfqpoint{1.669774in}{2.165480in}}%
\pgfpathcurveto{\pgfqpoint{1.663950in}{2.171304in}}{\pgfqpoint{1.656050in}{2.174577in}}{\pgfqpoint{1.647814in}{2.174577in}}%
\pgfpathcurveto{\pgfqpoint{1.639578in}{2.174577in}}{\pgfqpoint{1.631678in}{2.171304in}}{\pgfqpoint{1.625854in}{2.165480in}}%
\pgfpathcurveto{\pgfqpoint{1.620030in}{2.159657in}}{\pgfqpoint{1.616757in}{2.151756in}}{\pgfqpoint{1.616757in}{2.143520in}}%
\pgfpathcurveto{\pgfqpoint{1.616757in}{2.135284in}}{\pgfqpoint{1.620030in}{2.127384in}}{\pgfqpoint{1.625854in}{2.121560in}}%
\pgfpathcurveto{\pgfqpoint{1.631678in}{2.115736in}}{\pgfqpoint{1.639578in}{2.112464in}}{\pgfqpoint{1.647814in}{2.112464in}}%
\pgfpathclose%
\pgfusepath{stroke,fill}%
\end{pgfscope}%
\begin{pgfscope}%
\pgfpathrectangle{\pgfqpoint{0.100000in}{0.212622in}}{\pgfqpoint{3.696000in}{3.696000in}}%
\pgfusepath{clip}%
\pgfsetbuttcap%
\pgfsetroundjoin%
\definecolor{currentfill}{rgb}{0.121569,0.466667,0.705882}%
\pgfsetfillcolor{currentfill}%
\pgfsetfillopacity{0.300651}%
\pgfsetlinewidth{1.003750pt}%
\definecolor{currentstroke}{rgb}{0.121569,0.466667,0.705882}%
\pgfsetstrokecolor{currentstroke}%
\pgfsetstrokeopacity{0.300651}%
\pgfsetdash{}{0pt}%
\pgfpathmoveto{\pgfqpoint{1.647813in}{2.112463in}}%
\pgfpathcurveto{\pgfqpoint{1.656050in}{2.112463in}}{\pgfqpoint{1.663950in}{2.115736in}}{\pgfqpoint{1.669774in}{2.121560in}}%
\pgfpathcurveto{\pgfqpoint{1.675598in}{2.127384in}}{\pgfqpoint{1.678870in}{2.135284in}}{\pgfqpoint{1.678870in}{2.143520in}}%
\pgfpathcurveto{\pgfqpoint{1.678870in}{2.151756in}}{\pgfqpoint{1.675598in}{2.159656in}}{\pgfqpoint{1.669774in}{2.165480in}}%
\pgfpathcurveto{\pgfqpoint{1.663950in}{2.171304in}}{\pgfqpoint{1.656050in}{2.174576in}}{\pgfqpoint{1.647813in}{2.174576in}}%
\pgfpathcurveto{\pgfqpoint{1.639577in}{2.174576in}}{\pgfqpoint{1.631677in}{2.171304in}}{\pgfqpoint{1.625853in}{2.165480in}}%
\pgfpathcurveto{\pgfqpoint{1.620029in}{2.159656in}}{\pgfqpoint{1.616757in}{2.151756in}}{\pgfqpoint{1.616757in}{2.143520in}}%
\pgfpathcurveto{\pgfqpoint{1.616757in}{2.135284in}}{\pgfqpoint{1.620029in}{2.127384in}}{\pgfqpoint{1.625853in}{2.121560in}}%
\pgfpathcurveto{\pgfqpoint{1.631677in}{2.115736in}}{\pgfqpoint{1.639577in}{2.112463in}}{\pgfqpoint{1.647813in}{2.112463in}}%
\pgfpathclose%
\pgfusepath{stroke,fill}%
\end{pgfscope}%
\begin{pgfscope}%
\pgfpathrectangle{\pgfqpoint{0.100000in}{0.212622in}}{\pgfqpoint{3.696000in}{3.696000in}}%
\pgfusepath{clip}%
\pgfsetbuttcap%
\pgfsetroundjoin%
\definecolor{currentfill}{rgb}{0.121569,0.466667,0.705882}%
\pgfsetfillcolor{currentfill}%
\pgfsetfillopacity{0.300651}%
\pgfsetlinewidth{1.003750pt}%
\definecolor{currentstroke}{rgb}{0.121569,0.466667,0.705882}%
\pgfsetstrokecolor{currentstroke}%
\pgfsetstrokeopacity{0.300651}%
\pgfsetdash{}{0pt}%
\pgfpathmoveto{\pgfqpoint{1.647812in}{2.112462in}}%
\pgfpathcurveto{\pgfqpoint{1.656048in}{2.112462in}}{\pgfqpoint{1.663948in}{2.115735in}}{\pgfqpoint{1.669772in}{2.121559in}}%
\pgfpathcurveto{\pgfqpoint{1.675596in}{2.127383in}}{\pgfqpoint{1.678869in}{2.135283in}}{\pgfqpoint{1.678869in}{2.143519in}}%
\pgfpathcurveto{\pgfqpoint{1.678869in}{2.151755in}}{\pgfqpoint{1.675596in}{2.159655in}}{\pgfqpoint{1.669772in}{2.165479in}}%
\pgfpathcurveto{\pgfqpoint{1.663948in}{2.171303in}}{\pgfqpoint{1.656048in}{2.174575in}}{\pgfqpoint{1.647812in}{2.174575in}}%
\pgfpathcurveto{\pgfqpoint{1.639576in}{2.174575in}}{\pgfqpoint{1.631676in}{2.171303in}}{\pgfqpoint{1.625852in}{2.165479in}}%
\pgfpathcurveto{\pgfqpoint{1.620028in}{2.159655in}}{\pgfqpoint{1.616756in}{2.151755in}}{\pgfqpoint{1.616756in}{2.143519in}}%
\pgfpathcurveto{\pgfqpoint{1.616756in}{2.135283in}}{\pgfqpoint{1.620028in}{2.127383in}}{\pgfqpoint{1.625852in}{2.121559in}}%
\pgfpathcurveto{\pgfqpoint{1.631676in}{2.115735in}}{\pgfqpoint{1.639576in}{2.112462in}}{\pgfqpoint{1.647812in}{2.112462in}}%
\pgfpathclose%
\pgfusepath{stroke,fill}%
\end{pgfscope}%
\begin{pgfscope}%
\pgfpathrectangle{\pgfqpoint{0.100000in}{0.212622in}}{\pgfqpoint{3.696000in}{3.696000in}}%
\pgfusepath{clip}%
\pgfsetbuttcap%
\pgfsetroundjoin%
\definecolor{currentfill}{rgb}{0.121569,0.466667,0.705882}%
\pgfsetfillcolor{currentfill}%
\pgfsetfillopacity{0.300652}%
\pgfsetlinewidth{1.003750pt}%
\definecolor{currentstroke}{rgb}{0.121569,0.466667,0.705882}%
\pgfsetstrokecolor{currentstroke}%
\pgfsetstrokeopacity{0.300652}%
\pgfsetdash{}{0pt}%
\pgfpathmoveto{\pgfqpoint{1.647810in}{2.112461in}}%
\pgfpathcurveto{\pgfqpoint{1.656047in}{2.112461in}}{\pgfqpoint{1.663947in}{2.115733in}}{\pgfqpoint{1.669771in}{2.121557in}}%
\pgfpathcurveto{\pgfqpoint{1.675595in}{2.127381in}}{\pgfqpoint{1.678867in}{2.135281in}}{\pgfqpoint{1.678867in}{2.143518in}}%
\pgfpathcurveto{\pgfqpoint{1.678867in}{2.151754in}}{\pgfqpoint{1.675595in}{2.159654in}}{\pgfqpoint{1.669771in}{2.165478in}}%
\pgfpathcurveto{\pgfqpoint{1.663947in}{2.171302in}}{\pgfqpoint{1.656047in}{2.174574in}}{\pgfqpoint{1.647810in}{2.174574in}}%
\pgfpathcurveto{\pgfqpoint{1.639574in}{2.174574in}}{\pgfqpoint{1.631674in}{2.171302in}}{\pgfqpoint{1.625850in}{2.165478in}}%
\pgfpathcurveto{\pgfqpoint{1.620026in}{2.159654in}}{\pgfqpoint{1.616754in}{2.151754in}}{\pgfqpoint{1.616754in}{2.143518in}}%
\pgfpathcurveto{\pgfqpoint{1.616754in}{2.135281in}}{\pgfqpoint{1.620026in}{2.127381in}}{\pgfqpoint{1.625850in}{2.121557in}}%
\pgfpathcurveto{\pgfqpoint{1.631674in}{2.115733in}}{\pgfqpoint{1.639574in}{2.112461in}}{\pgfqpoint{1.647810in}{2.112461in}}%
\pgfpathclose%
\pgfusepath{stroke,fill}%
\end{pgfscope}%
\begin{pgfscope}%
\pgfpathrectangle{\pgfqpoint{0.100000in}{0.212622in}}{\pgfqpoint{3.696000in}{3.696000in}}%
\pgfusepath{clip}%
\pgfsetbuttcap%
\pgfsetroundjoin%
\definecolor{currentfill}{rgb}{0.121569,0.466667,0.705882}%
\pgfsetfillcolor{currentfill}%
\pgfsetfillopacity{0.300653}%
\pgfsetlinewidth{1.003750pt}%
\definecolor{currentstroke}{rgb}{0.121569,0.466667,0.705882}%
\pgfsetstrokecolor{currentstroke}%
\pgfsetstrokeopacity{0.300653}%
\pgfsetdash{}{0pt}%
\pgfpathmoveto{\pgfqpoint{1.647808in}{2.112459in}}%
\pgfpathcurveto{\pgfqpoint{1.656044in}{2.112459in}}{\pgfqpoint{1.663944in}{2.115731in}}{\pgfqpoint{1.669768in}{2.121555in}}%
\pgfpathcurveto{\pgfqpoint{1.675592in}{2.127379in}}{\pgfqpoint{1.678864in}{2.135279in}}{\pgfqpoint{1.678864in}{2.143515in}}%
\pgfpathcurveto{\pgfqpoint{1.678864in}{2.151751in}}{\pgfqpoint{1.675592in}{2.159652in}}{\pgfqpoint{1.669768in}{2.165475in}}%
\pgfpathcurveto{\pgfqpoint{1.663944in}{2.171299in}}{\pgfqpoint{1.656044in}{2.174572in}}{\pgfqpoint{1.647808in}{2.174572in}}%
\pgfpathcurveto{\pgfqpoint{1.639571in}{2.174572in}}{\pgfqpoint{1.631671in}{2.171299in}}{\pgfqpoint{1.625847in}{2.165475in}}%
\pgfpathcurveto{\pgfqpoint{1.620023in}{2.159652in}}{\pgfqpoint{1.616751in}{2.151751in}}{\pgfqpoint{1.616751in}{2.143515in}}%
\pgfpathcurveto{\pgfqpoint{1.616751in}{2.135279in}}{\pgfqpoint{1.620023in}{2.127379in}}{\pgfqpoint{1.625847in}{2.121555in}}%
\pgfpathcurveto{\pgfqpoint{1.631671in}{2.115731in}}{\pgfqpoint{1.639571in}{2.112459in}}{\pgfqpoint{1.647808in}{2.112459in}}%
\pgfpathclose%
\pgfusepath{stroke,fill}%
\end{pgfscope}%
\begin{pgfscope}%
\pgfpathrectangle{\pgfqpoint{0.100000in}{0.212622in}}{\pgfqpoint{3.696000in}{3.696000in}}%
\pgfusepath{clip}%
\pgfsetbuttcap%
\pgfsetroundjoin%
\definecolor{currentfill}{rgb}{0.121569,0.466667,0.705882}%
\pgfsetfillcolor{currentfill}%
\pgfsetfillopacity{0.300654}%
\pgfsetlinewidth{1.003750pt}%
\definecolor{currentstroke}{rgb}{0.121569,0.466667,0.705882}%
\pgfsetstrokecolor{currentstroke}%
\pgfsetstrokeopacity{0.300654}%
\pgfsetdash{}{0pt}%
\pgfpathmoveto{\pgfqpoint{1.647803in}{2.112456in}}%
\pgfpathcurveto{\pgfqpoint{1.656040in}{2.112456in}}{\pgfqpoint{1.663940in}{2.115729in}}{\pgfqpoint{1.669764in}{2.121553in}}%
\pgfpathcurveto{\pgfqpoint{1.675588in}{2.127377in}}{\pgfqpoint{1.678860in}{2.135277in}}{\pgfqpoint{1.678860in}{2.143513in}}%
\pgfpathcurveto{\pgfqpoint{1.678860in}{2.151749in}}{\pgfqpoint{1.675588in}{2.159649in}}{\pgfqpoint{1.669764in}{2.165473in}}%
\pgfpathcurveto{\pgfqpoint{1.663940in}{2.171297in}}{\pgfqpoint{1.656040in}{2.174569in}}{\pgfqpoint{1.647803in}{2.174569in}}%
\pgfpathcurveto{\pgfqpoint{1.639567in}{2.174569in}}{\pgfqpoint{1.631667in}{2.171297in}}{\pgfqpoint{1.625843in}{2.165473in}}%
\pgfpathcurveto{\pgfqpoint{1.620019in}{2.159649in}}{\pgfqpoint{1.616747in}{2.151749in}}{\pgfqpoint{1.616747in}{2.143513in}}%
\pgfpathcurveto{\pgfqpoint{1.616747in}{2.135277in}}{\pgfqpoint{1.620019in}{2.127377in}}{\pgfqpoint{1.625843in}{2.121553in}}%
\pgfpathcurveto{\pgfqpoint{1.631667in}{2.115729in}}{\pgfqpoint{1.639567in}{2.112456in}}{\pgfqpoint{1.647803in}{2.112456in}}%
\pgfpathclose%
\pgfusepath{stroke,fill}%
\end{pgfscope}%
\begin{pgfscope}%
\pgfpathrectangle{\pgfqpoint{0.100000in}{0.212622in}}{\pgfqpoint{3.696000in}{3.696000in}}%
\pgfusepath{clip}%
\pgfsetbuttcap%
\pgfsetroundjoin%
\definecolor{currentfill}{rgb}{0.121569,0.466667,0.705882}%
\pgfsetfillcolor{currentfill}%
\pgfsetfillopacity{0.300659}%
\pgfsetlinewidth{1.003750pt}%
\definecolor{currentstroke}{rgb}{0.121569,0.466667,0.705882}%
\pgfsetstrokecolor{currentstroke}%
\pgfsetstrokeopacity{0.300659}%
\pgfsetdash{}{0pt}%
\pgfpathmoveto{\pgfqpoint{1.647787in}{2.112442in}}%
\pgfpathcurveto{\pgfqpoint{1.656023in}{2.112442in}}{\pgfqpoint{1.663923in}{2.115714in}}{\pgfqpoint{1.669747in}{2.121538in}}%
\pgfpathcurveto{\pgfqpoint{1.675571in}{2.127362in}}{\pgfqpoint{1.678843in}{2.135262in}}{\pgfqpoint{1.678843in}{2.143498in}}%
\pgfpathcurveto{\pgfqpoint{1.678843in}{2.151734in}}{\pgfqpoint{1.675571in}{2.159634in}}{\pgfqpoint{1.669747in}{2.165458in}}%
\pgfpathcurveto{\pgfqpoint{1.663923in}{2.171282in}}{\pgfqpoint{1.656023in}{2.174555in}}{\pgfqpoint{1.647787in}{2.174555in}}%
\pgfpathcurveto{\pgfqpoint{1.639551in}{2.174555in}}{\pgfqpoint{1.631651in}{2.171282in}}{\pgfqpoint{1.625827in}{2.165458in}}%
\pgfpathcurveto{\pgfqpoint{1.620003in}{2.159634in}}{\pgfqpoint{1.616730in}{2.151734in}}{\pgfqpoint{1.616730in}{2.143498in}}%
\pgfpathcurveto{\pgfqpoint{1.616730in}{2.135262in}}{\pgfqpoint{1.620003in}{2.127362in}}{\pgfqpoint{1.625827in}{2.121538in}}%
\pgfpathcurveto{\pgfqpoint{1.631651in}{2.115714in}}{\pgfqpoint{1.639551in}{2.112442in}}{\pgfqpoint{1.647787in}{2.112442in}}%
\pgfpathclose%
\pgfusepath{stroke,fill}%
\end{pgfscope}%
\begin{pgfscope}%
\pgfpathrectangle{\pgfqpoint{0.100000in}{0.212622in}}{\pgfqpoint{3.696000in}{3.696000in}}%
\pgfusepath{clip}%
\pgfsetbuttcap%
\pgfsetroundjoin%
\definecolor{currentfill}{rgb}{0.121569,0.466667,0.705882}%
\pgfsetfillcolor{currentfill}%
\pgfsetfillopacity{0.300661}%
\pgfsetlinewidth{1.003750pt}%
\definecolor{currentstroke}{rgb}{0.121569,0.466667,0.705882}%
\pgfsetstrokecolor{currentstroke}%
\pgfsetstrokeopacity{0.300661}%
\pgfsetdash{}{0pt}%
\pgfpathmoveto{\pgfqpoint{1.647775in}{2.112436in}}%
\pgfpathcurveto{\pgfqpoint{1.656011in}{2.112436in}}{\pgfqpoint{1.663911in}{2.115708in}}{\pgfqpoint{1.669735in}{2.121532in}}%
\pgfpathcurveto{\pgfqpoint{1.675559in}{2.127356in}}{\pgfqpoint{1.678831in}{2.135256in}}{\pgfqpoint{1.678831in}{2.143492in}}%
\pgfpathcurveto{\pgfqpoint{1.678831in}{2.151729in}}{\pgfqpoint{1.675559in}{2.159629in}}{\pgfqpoint{1.669735in}{2.165453in}}%
\pgfpathcurveto{\pgfqpoint{1.663911in}{2.171277in}}{\pgfqpoint{1.656011in}{2.174549in}}{\pgfqpoint{1.647775in}{2.174549in}}%
\pgfpathcurveto{\pgfqpoint{1.639538in}{2.174549in}}{\pgfqpoint{1.631638in}{2.171277in}}{\pgfqpoint{1.625814in}{2.165453in}}%
\pgfpathcurveto{\pgfqpoint{1.619990in}{2.159629in}}{\pgfqpoint{1.616718in}{2.151729in}}{\pgfqpoint{1.616718in}{2.143492in}}%
\pgfpathcurveto{\pgfqpoint{1.616718in}{2.135256in}}{\pgfqpoint{1.619990in}{2.127356in}}{\pgfqpoint{1.625814in}{2.121532in}}%
\pgfpathcurveto{\pgfqpoint{1.631638in}{2.115708in}}{\pgfqpoint{1.639538in}{2.112436in}}{\pgfqpoint{1.647775in}{2.112436in}}%
\pgfpathclose%
\pgfusepath{stroke,fill}%
\end{pgfscope}%
\begin{pgfscope}%
\pgfpathrectangle{\pgfqpoint{0.100000in}{0.212622in}}{\pgfqpoint{3.696000in}{3.696000in}}%
\pgfusepath{clip}%
\pgfsetbuttcap%
\pgfsetroundjoin%
\definecolor{currentfill}{rgb}{0.121569,0.466667,0.705882}%
\pgfsetfillcolor{currentfill}%
\pgfsetfillopacity{0.300672}%
\pgfsetlinewidth{1.003750pt}%
\definecolor{currentstroke}{rgb}{0.121569,0.466667,0.705882}%
\pgfsetstrokecolor{currentstroke}%
\pgfsetstrokeopacity{0.300672}%
\pgfsetdash{}{0pt}%
\pgfpathmoveto{\pgfqpoint{1.647716in}{2.112391in}}%
\pgfpathcurveto{\pgfqpoint{1.655952in}{2.112391in}}{\pgfqpoint{1.663852in}{2.115664in}}{\pgfqpoint{1.669676in}{2.121487in}}%
\pgfpathcurveto{\pgfqpoint{1.675500in}{2.127311in}}{\pgfqpoint{1.678772in}{2.135211in}}{\pgfqpoint{1.678772in}{2.143448in}}%
\pgfpathcurveto{\pgfqpoint{1.678772in}{2.151684in}}{\pgfqpoint{1.675500in}{2.159584in}}{\pgfqpoint{1.669676in}{2.165408in}}%
\pgfpathcurveto{\pgfqpoint{1.663852in}{2.171232in}}{\pgfqpoint{1.655952in}{2.174504in}}{\pgfqpoint{1.647716in}{2.174504in}}%
\pgfpathcurveto{\pgfqpoint{1.639479in}{2.174504in}}{\pgfqpoint{1.631579in}{2.171232in}}{\pgfqpoint{1.625755in}{2.165408in}}%
\pgfpathcurveto{\pgfqpoint{1.619932in}{2.159584in}}{\pgfqpoint{1.616659in}{2.151684in}}{\pgfqpoint{1.616659in}{2.143448in}}%
\pgfpathcurveto{\pgfqpoint{1.616659in}{2.135211in}}{\pgfqpoint{1.619932in}{2.127311in}}{\pgfqpoint{1.625755in}{2.121487in}}%
\pgfpathcurveto{\pgfqpoint{1.631579in}{2.115664in}}{\pgfqpoint{1.639479in}{2.112391in}}{\pgfqpoint{1.647716in}{2.112391in}}%
\pgfpathclose%
\pgfusepath{stroke,fill}%
\end{pgfscope}%
\begin{pgfscope}%
\pgfpathrectangle{\pgfqpoint{0.100000in}{0.212622in}}{\pgfqpoint{3.696000in}{3.696000in}}%
\pgfusepath{clip}%
\pgfsetbuttcap%
\pgfsetroundjoin%
\definecolor{currentfill}{rgb}{0.121569,0.466667,0.705882}%
\pgfsetfillcolor{currentfill}%
\pgfsetfillopacity{0.300677}%
\pgfsetlinewidth{1.003750pt}%
\definecolor{currentstroke}{rgb}{0.121569,0.466667,0.705882}%
\pgfsetstrokecolor{currentstroke}%
\pgfsetstrokeopacity{0.300677}%
\pgfsetdash{}{0pt}%
\pgfpathmoveto{\pgfqpoint{1.647720in}{2.112384in}}%
\pgfpathcurveto{\pgfqpoint{1.655956in}{2.112384in}}{\pgfqpoint{1.663856in}{2.115657in}}{\pgfqpoint{1.669680in}{2.121481in}}%
\pgfpathcurveto{\pgfqpoint{1.675504in}{2.127304in}}{\pgfqpoint{1.678776in}{2.135204in}}{\pgfqpoint{1.678776in}{2.143441in}}%
\pgfpathcurveto{\pgfqpoint{1.678776in}{2.151677in}}{\pgfqpoint{1.675504in}{2.159577in}}{\pgfqpoint{1.669680in}{2.165401in}}%
\pgfpathcurveto{\pgfqpoint{1.663856in}{2.171225in}}{\pgfqpoint{1.655956in}{2.174497in}}{\pgfqpoint{1.647720in}{2.174497in}}%
\pgfpathcurveto{\pgfqpoint{1.639484in}{2.174497in}}{\pgfqpoint{1.631584in}{2.171225in}}{\pgfqpoint{1.625760in}{2.165401in}}%
\pgfpathcurveto{\pgfqpoint{1.619936in}{2.159577in}}{\pgfqpoint{1.616663in}{2.151677in}}{\pgfqpoint{1.616663in}{2.143441in}}%
\pgfpathcurveto{\pgfqpoint{1.616663in}{2.135204in}}{\pgfqpoint{1.619936in}{2.127304in}}{\pgfqpoint{1.625760in}{2.121481in}}%
\pgfpathcurveto{\pgfqpoint{1.631584in}{2.115657in}}{\pgfqpoint{1.639484in}{2.112384in}}{\pgfqpoint{1.647720in}{2.112384in}}%
\pgfpathclose%
\pgfusepath{stroke,fill}%
\end{pgfscope}%
\begin{pgfscope}%
\pgfpathrectangle{\pgfqpoint{0.100000in}{0.212622in}}{\pgfqpoint{3.696000in}{3.696000in}}%
\pgfusepath{clip}%
\pgfsetbuttcap%
\pgfsetroundjoin%
\definecolor{currentfill}{rgb}{0.121569,0.466667,0.705882}%
\pgfsetfillcolor{currentfill}%
\pgfsetfillopacity{0.300684}%
\pgfsetlinewidth{1.003750pt}%
\definecolor{currentstroke}{rgb}{0.121569,0.466667,0.705882}%
\pgfsetstrokecolor{currentstroke}%
\pgfsetstrokeopacity{0.300684}%
\pgfsetdash{}{0pt}%
\pgfpathmoveto{\pgfqpoint{1.647644in}{2.112331in}}%
\pgfpathcurveto{\pgfqpoint{1.655881in}{2.112331in}}{\pgfqpoint{1.663781in}{2.115603in}}{\pgfqpoint{1.669605in}{2.121427in}}%
\pgfpathcurveto{\pgfqpoint{1.675429in}{2.127251in}}{\pgfqpoint{1.678701in}{2.135151in}}{\pgfqpoint{1.678701in}{2.143388in}}%
\pgfpathcurveto{\pgfqpoint{1.678701in}{2.151624in}}{\pgfqpoint{1.675429in}{2.159524in}}{\pgfqpoint{1.669605in}{2.165348in}}%
\pgfpathcurveto{\pgfqpoint{1.663781in}{2.171172in}}{\pgfqpoint{1.655881in}{2.174444in}}{\pgfqpoint{1.647644in}{2.174444in}}%
\pgfpathcurveto{\pgfqpoint{1.639408in}{2.174444in}}{\pgfqpoint{1.631508in}{2.171172in}}{\pgfqpoint{1.625684in}{2.165348in}}%
\pgfpathcurveto{\pgfqpoint{1.619860in}{2.159524in}}{\pgfqpoint{1.616588in}{2.151624in}}{\pgfqpoint{1.616588in}{2.143388in}}%
\pgfpathcurveto{\pgfqpoint{1.616588in}{2.135151in}}{\pgfqpoint{1.619860in}{2.127251in}}{\pgfqpoint{1.625684in}{2.121427in}}%
\pgfpathcurveto{\pgfqpoint{1.631508in}{2.115603in}}{\pgfqpoint{1.639408in}{2.112331in}}{\pgfqpoint{1.647644in}{2.112331in}}%
\pgfpathclose%
\pgfusepath{stroke,fill}%
\end{pgfscope}%
\begin{pgfscope}%
\pgfpathrectangle{\pgfqpoint{0.100000in}{0.212622in}}{\pgfqpoint{3.696000in}{3.696000in}}%
\pgfusepath{clip}%
\pgfsetbuttcap%
\pgfsetroundjoin%
\definecolor{currentfill}{rgb}{0.121569,0.466667,0.705882}%
\pgfsetfillcolor{currentfill}%
\pgfsetfillopacity{0.300689}%
\pgfsetlinewidth{1.003750pt}%
\definecolor{currentstroke}{rgb}{0.121569,0.466667,0.705882}%
\pgfsetstrokecolor{currentstroke}%
\pgfsetstrokeopacity{0.300689}%
\pgfsetdash{}{0pt}%
\pgfpathmoveto{\pgfqpoint{1.647571in}{2.112305in}}%
\pgfpathcurveto{\pgfqpoint{1.655807in}{2.112305in}}{\pgfqpoint{1.663707in}{2.115577in}}{\pgfqpoint{1.669531in}{2.121401in}}%
\pgfpathcurveto{\pgfqpoint{1.675355in}{2.127225in}}{\pgfqpoint{1.678627in}{2.135125in}}{\pgfqpoint{1.678627in}{2.143361in}}%
\pgfpathcurveto{\pgfqpoint{1.678627in}{2.151598in}}{\pgfqpoint{1.675355in}{2.159498in}}{\pgfqpoint{1.669531in}{2.165322in}}%
\pgfpathcurveto{\pgfqpoint{1.663707in}{2.171146in}}{\pgfqpoint{1.655807in}{2.174418in}}{\pgfqpoint{1.647571in}{2.174418in}}%
\pgfpathcurveto{\pgfqpoint{1.639334in}{2.174418in}}{\pgfqpoint{1.631434in}{2.171146in}}{\pgfqpoint{1.625611in}{2.165322in}}%
\pgfpathcurveto{\pgfqpoint{1.619787in}{2.159498in}}{\pgfqpoint{1.616514in}{2.151598in}}{\pgfqpoint{1.616514in}{2.143361in}}%
\pgfpathcurveto{\pgfqpoint{1.616514in}{2.135125in}}{\pgfqpoint{1.619787in}{2.127225in}}{\pgfqpoint{1.625611in}{2.121401in}}%
\pgfpathcurveto{\pgfqpoint{1.631434in}{2.115577in}}{\pgfqpoint{1.639334in}{2.112305in}}{\pgfqpoint{1.647571in}{2.112305in}}%
\pgfpathclose%
\pgfusepath{stroke,fill}%
\end{pgfscope}%
\begin{pgfscope}%
\pgfpathrectangle{\pgfqpoint{0.100000in}{0.212622in}}{\pgfqpoint{3.696000in}{3.696000in}}%
\pgfusepath{clip}%
\pgfsetbuttcap%
\pgfsetroundjoin%
\definecolor{currentfill}{rgb}{0.121569,0.466667,0.705882}%
\pgfsetfillcolor{currentfill}%
\pgfsetfillopacity{0.300712}%
\pgfsetlinewidth{1.003750pt}%
\definecolor{currentstroke}{rgb}{0.121569,0.466667,0.705882}%
\pgfsetstrokecolor{currentstroke}%
\pgfsetstrokeopacity{0.300712}%
\pgfsetdash{}{0pt}%
\pgfpathmoveto{\pgfqpoint{1.649644in}{2.113535in}}%
\pgfpathcurveto{\pgfqpoint{1.657881in}{2.113535in}}{\pgfqpoint{1.665781in}{2.116807in}}{\pgfqpoint{1.671605in}{2.122631in}}%
\pgfpathcurveto{\pgfqpoint{1.677428in}{2.128455in}}{\pgfqpoint{1.680701in}{2.136355in}}{\pgfqpoint{1.680701in}{2.144591in}}%
\pgfpathcurveto{\pgfqpoint{1.680701in}{2.152828in}}{\pgfqpoint{1.677428in}{2.160728in}}{\pgfqpoint{1.671605in}{2.166551in}}%
\pgfpathcurveto{\pgfqpoint{1.665781in}{2.172375in}}{\pgfqpoint{1.657881in}{2.175648in}}{\pgfqpoint{1.649644in}{2.175648in}}%
\pgfpathcurveto{\pgfqpoint{1.641408in}{2.175648in}}{\pgfqpoint{1.633508in}{2.172375in}}{\pgfqpoint{1.627684in}{2.166551in}}%
\pgfpathcurveto{\pgfqpoint{1.621860in}{2.160728in}}{\pgfqpoint{1.618588in}{2.152828in}}{\pgfqpoint{1.618588in}{2.144591in}}%
\pgfpathcurveto{\pgfqpoint{1.618588in}{2.136355in}}{\pgfqpoint{1.621860in}{2.128455in}}{\pgfqpoint{1.627684in}{2.122631in}}%
\pgfpathcurveto{\pgfqpoint{1.633508in}{2.116807in}}{\pgfqpoint{1.641408in}{2.113535in}}{\pgfqpoint{1.649644in}{2.113535in}}%
\pgfpathclose%
\pgfusepath{stroke,fill}%
\end{pgfscope}%
\begin{pgfscope}%
\pgfpathrectangle{\pgfqpoint{0.100000in}{0.212622in}}{\pgfqpoint{3.696000in}{3.696000in}}%
\pgfusepath{clip}%
\pgfsetbuttcap%
\pgfsetroundjoin%
\definecolor{currentfill}{rgb}{0.121569,0.466667,0.705882}%
\pgfsetfillcolor{currentfill}%
\pgfsetfillopacity{0.300727}%
\pgfsetlinewidth{1.003750pt}%
\definecolor{currentstroke}{rgb}{0.121569,0.466667,0.705882}%
\pgfsetstrokecolor{currentstroke}%
\pgfsetstrokeopacity{0.300727}%
\pgfsetdash{}{0pt}%
\pgfpathmoveto{\pgfqpoint{1.645086in}{2.110574in}}%
\pgfpathcurveto{\pgfqpoint{1.653322in}{2.110574in}}{\pgfqpoint{1.661222in}{2.113846in}}{\pgfqpoint{1.667046in}{2.119670in}}%
\pgfpathcurveto{\pgfqpoint{1.672870in}{2.125494in}}{\pgfqpoint{1.676142in}{2.133394in}}{\pgfqpoint{1.676142in}{2.141630in}}%
\pgfpathcurveto{\pgfqpoint{1.676142in}{2.149867in}}{\pgfqpoint{1.672870in}{2.157767in}}{\pgfqpoint{1.667046in}{2.163591in}}%
\pgfpathcurveto{\pgfqpoint{1.661222in}{2.169415in}}{\pgfqpoint{1.653322in}{2.172687in}}{\pgfqpoint{1.645086in}{2.172687in}}%
\pgfpathcurveto{\pgfqpoint{1.636850in}{2.172687in}}{\pgfqpoint{1.628949in}{2.169415in}}{\pgfqpoint{1.623126in}{2.163591in}}%
\pgfpathcurveto{\pgfqpoint{1.617302in}{2.157767in}}{\pgfqpoint{1.614029in}{2.149867in}}{\pgfqpoint{1.614029in}{2.141630in}}%
\pgfpathcurveto{\pgfqpoint{1.614029in}{2.133394in}}{\pgfqpoint{1.617302in}{2.125494in}}{\pgfqpoint{1.623126in}{2.119670in}}%
\pgfpathcurveto{\pgfqpoint{1.628949in}{2.113846in}}{\pgfqpoint{1.636850in}{2.110574in}}{\pgfqpoint{1.645086in}{2.110574in}}%
\pgfpathclose%
\pgfusepath{stroke,fill}%
\end{pgfscope}%
\begin{pgfscope}%
\pgfpathrectangle{\pgfqpoint{0.100000in}{0.212622in}}{\pgfqpoint{3.696000in}{3.696000in}}%
\pgfusepath{clip}%
\pgfsetbuttcap%
\pgfsetroundjoin%
\definecolor{currentfill}{rgb}{0.121569,0.466667,0.705882}%
\pgfsetfillcolor{currentfill}%
\pgfsetfillopacity{0.300766}%
\pgfsetlinewidth{1.003750pt}%
\definecolor{currentstroke}{rgb}{0.121569,0.466667,0.705882}%
\pgfsetstrokecolor{currentstroke}%
\pgfsetstrokeopacity{0.300766}%
\pgfsetdash{}{0pt}%
\pgfpathmoveto{\pgfqpoint{1.648784in}{2.112925in}}%
\pgfpathcurveto{\pgfqpoint{1.657021in}{2.112925in}}{\pgfqpoint{1.664921in}{2.116197in}}{\pgfqpoint{1.670745in}{2.122021in}}%
\pgfpathcurveto{\pgfqpoint{1.676568in}{2.127845in}}{\pgfqpoint{1.679841in}{2.135745in}}{\pgfqpoint{1.679841in}{2.143981in}}%
\pgfpathcurveto{\pgfqpoint{1.679841in}{2.152217in}}{\pgfqpoint{1.676568in}{2.160117in}}{\pgfqpoint{1.670745in}{2.165941in}}%
\pgfpathcurveto{\pgfqpoint{1.664921in}{2.171765in}}{\pgfqpoint{1.657021in}{2.175038in}}{\pgfqpoint{1.648784in}{2.175038in}}%
\pgfpathcurveto{\pgfqpoint{1.640548in}{2.175038in}}{\pgfqpoint{1.632648in}{2.171765in}}{\pgfqpoint{1.626824in}{2.165941in}}%
\pgfpathcurveto{\pgfqpoint{1.621000in}{2.160117in}}{\pgfqpoint{1.617728in}{2.152217in}}{\pgfqpoint{1.617728in}{2.143981in}}%
\pgfpathcurveto{\pgfqpoint{1.617728in}{2.135745in}}{\pgfqpoint{1.621000in}{2.127845in}}{\pgfqpoint{1.626824in}{2.122021in}}%
\pgfpathcurveto{\pgfqpoint{1.632648in}{2.116197in}}{\pgfqpoint{1.640548in}{2.112925in}}{\pgfqpoint{1.648784in}{2.112925in}}%
\pgfpathclose%
\pgfusepath{stroke,fill}%
\end{pgfscope}%
\begin{pgfscope}%
\pgfpathrectangle{\pgfqpoint{0.100000in}{0.212622in}}{\pgfqpoint{3.696000in}{3.696000in}}%
\pgfusepath{clip}%
\pgfsetbuttcap%
\pgfsetroundjoin%
\definecolor{currentfill}{rgb}{0.121569,0.466667,0.705882}%
\pgfsetfillcolor{currentfill}%
\pgfsetfillopacity{0.300799}%
\pgfsetlinewidth{1.003750pt}%
\definecolor{currentstroke}{rgb}{0.121569,0.466667,0.705882}%
\pgfsetstrokecolor{currentstroke}%
\pgfsetstrokeopacity{0.300799}%
\pgfsetdash{}{0pt}%
\pgfpathmoveto{\pgfqpoint{1.647160in}{2.111960in}}%
\pgfpathcurveto{\pgfqpoint{1.655396in}{2.111960in}}{\pgfqpoint{1.663296in}{2.115232in}}{\pgfqpoint{1.669120in}{2.121056in}}%
\pgfpathcurveto{\pgfqpoint{1.674944in}{2.126880in}}{\pgfqpoint{1.678216in}{2.134780in}}{\pgfqpoint{1.678216in}{2.143016in}}%
\pgfpathcurveto{\pgfqpoint{1.678216in}{2.151253in}}{\pgfqpoint{1.674944in}{2.159153in}}{\pgfqpoint{1.669120in}{2.164977in}}%
\pgfpathcurveto{\pgfqpoint{1.663296in}{2.170801in}}{\pgfqpoint{1.655396in}{2.174073in}}{\pgfqpoint{1.647160in}{2.174073in}}%
\pgfpathcurveto{\pgfqpoint{1.638923in}{2.174073in}}{\pgfqpoint{1.631023in}{2.170801in}}{\pgfqpoint{1.625199in}{2.164977in}}%
\pgfpathcurveto{\pgfqpoint{1.619375in}{2.159153in}}{\pgfqpoint{1.616103in}{2.151253in}}{\pgfqpoint{1.616103in}{2.143016in}}%
\pgfpathcurveto{\pgfqpoint{1.616103in}{2.134780in}}{\pgfqpoint{1.619375in}{2.126880in}}{\pgfqpoint{1.625199in}{2.121056in}}%
\pgfpathcurveto{\pgfqpoint{1.631023in}{2.115232in}}{\pgfqpoint{1.638923in}{2.111960in}}{\pgfqpoint{1.647160in}{2.111960in}}%
\pgfpathclose%
\pgfusepath{stroke,fill}%
\end{pgfscope}%
\begin{pgfscope}%
\pgfpathrectangle{\pgfqpoint{0.100000in}{0.212622in}}{\pgfqpoint{3.696000in}{3.696000in}}%
\pgfusepath{clip}%
\pgfsetbuttcap%
\pgfsetroundjoin%
\definecolor{currentfill}{rgb}{0.121569,0.466667,0.705882}%
\pgfsetfillcolor{currentfill}%
\pgfsetfillopacity{0.300829}%
\pgfsetlinewidth{1.003750pt}%
\definecolor{currentstroke}{rgb}{0.121569,0.466667,0.705882}%
\pgfsetstrokecolor{currentstroke}%
\pgfsetstrokeopacity{0.300829}%
\pgfsetdash{}{0pt}%
\pgfpathmoveto{\pgfqpoint{1.649775in}{2.113659in}}%
\pgfpathcurveto{\pgfqpoint{1.658011in}{2.113659in}}{\pgfqpoint{1.665911in}{2.116931in}}{\pgfqpoint{1.671735in}{2.122755in}}%
\pgfpathcurveto{\pgfqpoint{1.677559in}{2.128579in}}{\pgfqpoint{1.680832in}{2.136479in}}{\pgfqpoint{1.680832in}{2.144715in}}%
\pgfpathcurveto{\pgfqpoint{1.680832in}{2.152951in}}{\pgfqpoint{1.677559in}{2.160852in}}{\pgfqpoint{1.671735in}{2.166675in}}%
\pgfpathcurveto{\pgfqpoint{1.665911in}{2.172499in}}{\pgfqpoint{1.658011in}{2.175772in}}{\pgfqpoint{1.649775in}{2.175772in}}%
\pgfpathcurveto{\pgfqpoint{1.641539in}{2.175772in}}{\pgfqpoint{1.633639in}{2.172499in}}{\pgfqpoint{1.627815in}{2.166675in}}%
\pgfpathcurveto{\pgfqpoint{1.621991in}{2.160852in}}{\pgfqpoint{1.618719in}{2.152951in}}{\pgfqpoint{1.618719in}{2.144715in}}%
\pgfpathcurveto{\pgfqpoint{1.618719in}{2.136479in}}{\pgfqpoint{1.621991in}{2.128579in}}{\pgfqpoint{1.627815in}{2.122755in}}%
\pgfpathcurveto{\pgfqpoint{1.633639in}{2.116931in}}{\pgfqpoint{1.641539in}{2.113659in}}{\pgfqpoint{1.649775in}{2.113659in}}%
\pgfpathclose%
\pgfusepath{stroke,fill}%
\end{pgfscope}%
\begin{pgfscope}%
\pgfpathrectangle{\pgfqpoint{0.100000in}{0.212622in}}{\pgfqpoint{3.696000in}{3.696000in}}%
\pgfusepath{clip}%
\pgfsetbuttcap%
\pgfsetroundjoin%
\definecolor{currentfill}{rgb}{0.121569,0.466667,0.705882}%
\pgfsetfillcolor{currentfill}%
\pgfsetfillopacity{0.300909}%
\pgfsetlinewidth{1.003750pt}%
\definecolor{currentstroke}{rgb}{0.121569,0.466667,0.705882}%
\pgfsetstrokecolor{currentstroke}%
\pgfsetstrokeopacity{0.300909}%
\pgfsetdash{}{0pt}%
\pgfpathmoveto{\pgfqpoint{1.646660in}{2.111594in}}%
\pgfpathcurveto{\pgfqpoint{1.654896in}{2.111594in}}{\pgfqpoint{1.662796in}{2.114866in}}{\pgfqpoint{1.668620in}{2.120690in}}%
\pgfpathcurveto{\pgfqpoint{1.674444in}{2.126514in}}{\pgfqpoint{1.677716in}{2.134414in}}{\pgfqpoint{1.677716in}{2.142651in}}%
\pgfpathcurveto{\pgfqpoint{1.677716in}{2.150887in}}{\pgfqpoint{1.674444in}{2.158787in}}{\pgfqpoint{1.668620in}{2.164611in}}%
\pgfpathcurveto{\pgfqpoint{1.662796in}{2.170435in}}{\pgfqpoint{1.654896in}{2.173707in}}{\pgfqpoint{1.646660in}{2.173707in}}%
\pgfpathcurveto{\pgfqpoint{1.638424in}{2.173707in}}{\pgfqpoint{1.630524in}{2.170435in}}{\pgfqpoint{1.624700in}{2.164611in}}%
\pgfpathcurveto{\pgfqpoint{1.618876in}{2.158787in}}{\pgfqpoint{1.615603in}{2.150887in}}{\pgfqpoint{1.615603in}{2.142651in}}%
\pgfpathcurveto{\pgfqpoint{1.615603in}{2.134414in}}{\pgfqpoint{1.618876in}{2.126514in}}{\pgfqpoint{1.624700in}{2.120690in}}%
\pgfpathcurveto{\pgfqpoint{1.630524in}{2.114866in}}{\pgfqpoint{1.638424in}{2.111594in}}{\pgfqpoint{1.646660in}{2.111594in}}%
\pgfpathclose%
\pgfusepath{stroke,fill}%
\end{pgfscope}%
\begin{pgfscope}%
\pgfpathrectangle{\pgfqpoint{0.100000in}{0.212622in}}{\pgfqpoint{3.696000in}{3.696000in}}%
\pgfusepath{clip}%
\pgfsetbuttcap%
\pgfsetroundjoin%
\definecolor{currentfill}{rgb}{0.121569,0.466667,0.705882}%
\pgfsetfillcolor{currentfill}%
\pgfsetfillopacity{0.301136}%
\pgfsetlinewidth{1.003750pt}%
\definecolor{currentstroke}{rgb}{0.121569,0.466667,0.705882}%
\pgfsetstrokecolor{currentstroke}%
\pgfsetstrokeopacity{0.301136}%
\pgfsetdash{}{0pt}%
\pgfpathmoveto{\pgfqpoint{1.645661in}{2.110805in}}%
\pgfpathcurveto{\pgfqpoint{1.653898in}{2.110805in}}{\pgfqpoint{1.661798in}{2.114078in}}{\pgfqpoint{1.667622in}{2.119902in}}%
\pgfpathcurveto{\pgfqpoint{1.673446in}{2.125725in}}{\pgfqpoint{1.676718in}{2.133626in}}{\pgfqpoint{1.676718in}{2.141862in}}%
\pgfpathcurveto{\pgfqpoint{1.676718in}{2.150098in}}{\pgfqpoint{1.673446in}{2.157998in}}{\pgfqpoint{1.667622in}{2.163822in}}%
\pgfpathcurveto{\pgfqpoint{1.661798in}{2.169646in}}{\pgfqpoint{1.653898in}{2.172918in}}{\pgfqpoint{1.645661in}{2.172918in}}%
\pgfpathcurveto{\pgfqpoint{1.637425in}{2.172918in}}{\pgfqpoint{1.629525in}{2.169646in}}{\pgfqpoint{1.623701in}{2.163822in}}%
\pgfpathcurveto{\pgfqpoint{1.617877in}{2.157998in}}{\pgfqpoint{1.614605in}{2.150098in}}{\pgfqpoint{1.614605in}{2.141862in}}%
\pgfpathcurveto{\pgfqpoint{1.614605in}{2.133626in}}{\pgfqpoint{1.617877in}{2.125725in}}{\pgfqpoint{1.623701in}{2.119902in}}%
\pgfpathcurveto{\pgfqpoint{1.629525in}{2.114078in}}{\pgfqpoint{1.637425in}{2.110805in}}{\pgfqpoint{1.645661in}{2.110805in}}%
\pgfpathclose%
\pgfusepath{stroke,fill}%
\end{pgfscope}%
\begin{pgfscope}%
\pgfpathrectangle{\pgfqpoint{0.100000in}{0.212622in}}{\pgfqpoint{3.696000in}{3.696000in}}%
\pgfusepath{clip}%
\pgfsetbuttcap%
\pgfsetroundjoin%
\definecolor{currentfill}{rgb}{0.121569,0.466667,0.705882}%
\pgfsetfillcolor{currentfill}%
\pgfsetfillopacity{0.301276}%
\pgfsetlinewidth{1.003750pt}%
\definecolor{currentstroke}{rgb}{0.121569,0.466667,0.705882}%
\pgfsetstrokecolor{currentstroke}%
\pgfsetstrokeopacity{0.301276}%
\pgfsetdash{}{0pt}%
\pgfpathmoveto{\pgfqpoint{1.644055in}{2.109588in}}%
\pgfpathcurveto{\pgfqpoint{1.652291in}{2.109588in}}{\pgfqpoint{1.660191in}{2.112861in}}{\pgfqpoint{1.666015in}{2.118684in}}%
\pgfpathcurveto{\pgfqpoint{1.671839in}{2.124508in}}{\pgfqpoint{1.675112in}{2.132408in}}{\pgfqpoint{1.675112in}{2.140645in}}%
\pgfpathcurveto{\pgfqpoint{1.675112in}{2.148881in}}{\pgfqpoint{1.671839in}{2.156781in}}{\pgfqpoint{1.666015in}{2.162605in}}%
\pgfpathcurveto{\pgfqpoint{1.660191in}{2.168429in}}{\pgfqpoint{1.652291in}{2.171701in}}{\pgfqpoint{1.644055in}{2.171701in}}%
\pgfpathcurveto{\pgfqpoint{1.635819in}{2.171701in}}{\pgfqpoint{1.627919in}{2.168429in}}{\pgfqpoint{1.622095in}{2.162605in}}%
\pgfpathcurveto{\pgfqpoint{1.616271in}{2.156781in}}{\pgfqpoint{1.612999in}{2.148881in}}{\pgfqpoint{1.612999in}{2.140645in}}%
\pgfpathcurveto{\pgfqpoint{1.612999in}{2.132408in}}{\pgfqpoint{1.616271in}{2.124508in}}{\pgfqpoint{1.622095in}{2.118684in}}%
\pgfpathcurveto{\pgfqpoint{1.627919in}{2.112861in}}{\pgfqpoint{1.635819in}{2.109588in}}{\pgfqpoint{1.644055in}{2.109588in}}%
\pgfpathclose%
\pgfusepath{stroke,fill}%
\end{pgfscope}%
\begin{pgfscope}%
\pgfpathrectangle{\pgfqpoint{0.100000in}{0.212622in}}{\pgfqpoint{3.696000in}{3.696000in}}%
\pgfusepath{clip}%
\pgfsetbuttcap%
\pgfsetroundjoin%
\definecolor{currentfill}{rgb}{0.121569,0.466667,0.705882}%
\pgfsetfillcolor{currentfill}%
\pgfsetfillopacity{0.301367}%
\pgfsetlinewidth{1.003750pt}%
\definecolor{currentstroke}{rgb}{0.121569,0.466667,0.705882}%
\pgfsetstrokecolor{currentstroke}%
\pgfsetstrokeopacity{0.301367}%
\pgfsetdash{}{0pt}%
\pgfpathmoveto{\pgfqpoint{1.644075in}{2.109677in}}%
\pgfpathcurveto{\pgfqpoint{1.652311in}{2.109677in}}{\pgfqpoint{1.660211in}{2.112949in}}{\pgfqpoint{1.666035in}{2.118773in}}%
\pgfpathcurveto{\pgfqpoint{1.671859in}{2.124597in}}{\pgfqpoint{1.675131in}{2.132497in}}{\pgfqpoint{1.675131in}{2.140733in}}%
\pgfpathcurveto{\pgfqpoint{1.675131in}{2.148970in}}{\pgfqpoint{1.671859in}{2.156870in}}{\pgfqpoint{1.666035in}{2.162694in}}%
\pgfpathcurveto{\pgfqpoint{1.660211in}{2.168518in}}{\pgfqpoint{1.652311in}{2.171790in}}{\pgfqpoint{1.644075in}{2.171790in}}%
\pgfpathcurveto{\pgfqpoint{1.635838in}{2.171790in}}{\pgfqpoint{1.627938in}{2.168518in}}{\pgfqpoint{1.622114in}{2.162694in}}%
\pgfpathcurveto{\pgfqpoint{1.616290in}{2.156870in}}{\pgfqpoint{1.613018in}{2.148970in}}{\pgfqpoint{1.613018in}{2.140733in}}%
\pgfpathcurveto{\pgfqpoint{1.613018in}{2.132497in}}{\pgfqpoint{1.616290in}{2.124597in}}{\pgfqpoint{1.622114in}{2.118773in}}%
\pgfpathcurveto{\pgfqpoint{1.627938in}{2.112949in}}{\pgfqpoint{1.635838in}{2.109677in}}{\pgfqpoint{1.644075in}{2.109677in}}%
\pgfpathclose%
\pgfusepath{stroke,fill}%
\end{pgfscope}%
\begin{pgfscope}%
\pgfpathrectangle{\pgfqpoint{0.100000in}{0.212622in}}{\pgfqpoint{3.696000in}{3.696000in}}%
\pgfusepath{clip}%
\pgfsetbuttcap%
\pgfsetroundjoin%
\definecolor{currentfill}{rgb}{0.121569,0.466667,0.705882}%
\pgfsetfillcolor{currentfill}%
\pgfsetfillopacity{0.301445}%
\pgfsetlinewidth{1.003750pt}%
\definecolor{currentstroke}{rgb}{0.121569,0.466667,0.705882}%
\pgfsetstrokecolor{currentstroke}%
\pgfsetstrokeopacity{0.301445}%
\pgfsetdash{}{0pt}%
\pgfpathmoveto{\pgfqpoint{1.643984in}{2.109553in}}%
\pgfpathcurveto{\pgfqpoint{1.652220in}{2.109553in}}{\pgfqpoint{1.660120in}{2.112826in}}{\pgfqpoint{1.665944in}{2.118649in}}%
\pgfpathcurveto{\pgfqpoint{1.671768in}{2.124473in}}{\pgfqpoint{1.675041in}{2.132373in}}{\pgfqpoint{1.675041in}{2.140610in}}%
\pgfpathcurveto{\pgfqpoint{1.675041in}{2.148846in}}{\pgfqpoint{1.671768in}{2.156746in}}{\pgfqpoint{1.665944in}{2.162570in}}%
\pgfpathcurveto{\pgfqpoint{1.660120in}{2.168394in}}{\pgfqpoint{1.652220in}{2.171666in}}{\pgfqpoint{1.643984in}{2.171666in}}%
\pgfpathcurveto{\pgfqpoint{1.635748in}{2.171666in}}{\pgfqpoint{1.627848in}{2.168394in}}{\pgfqpoint{1.622024in}{2.162570in}}%
\pgfpathcurveto{\pgfqpoint{1.616200in}{2.156746in}}{\pgfqpoint{1.612928in}{2.148846in}}{\pgfqpoint{1.612928in}{2.140610in}}%
\pgfpathcurveto{\pgfqpoint{1.612928in}{2.132373in}}{\pgfqpoint{1.616200in}{2.124473in}}{\pgfqpoint{1.622024in}{2.118649in}}%
\pgfpathcurveto{\pgfqpoint{1.627848in}{2.112826in}}{\pgfqpoint{1.635748in}{2.109553in}}{\pgfqpoint{1.643984in}{2.109553in}}%
\pgfpathclose%
\pgfusepath{stroke,fill}%
\end{pgfscope}%
\begin{pgfscope}%
\pgfpathrectangle{\pgfqpoint{0.100000in}{0.212622in}}{\pgfqpoint{3.696000in}{3.696000in}}%
\pgfusepath{clip}%
\pgfsetbuttcap%
\pgfsetroundjoin%
\definecolor{currentfill}{rgb}{0.121569,0.466667,0.705882}%
\pgfsetfillcolor{currentfill}%
\pgfsetfillopacity{0.301449}%
\pgfsetlinewidth{1.003750pt}%
\definecolor{currentstroke}{rgb}{0.121569,0.466667,0.705882}%
\pgfsetstrokecolor{currentstroke}%
\pgfsetstrokeopacity{0.301449}%
\pgfsetdash{}{0pt}%
\pgfpathmoveto{\pgfqpoint{1.644116in}{2.109663in}}%
\pgfpathcurveto{\pgfqpoint{1.652352in}{2.109663in}}{\pgfqpoint{1.660253in}{2.112935in}}{\pgfqpoint{1.666076in}{2.118759in}}%
\pgfpathcurveto{\pgfqpoint{1.671900in}{2.124583in}}{\pgfqpoint{1.675173in}{2.132483in}}{\pgfqpoint{1.675173in}{2.140719in}}%
\pgfpathcurveto{\pgfqpoint{1.675173in}{2.148956in}}{\pgfqpoint{1.671900in}{2.156856in}}{\pgfqpoint{1.666076in}{2.162680in}}%
\pgfpathcurveto{\pgfqpoint{1.660253in}{2.168503in}}{\pgfqpoint{1.652352in}{2.171776in}}{\pgfqpoint{1.644116in}{2.171776in}}%
\pgfpathcurveto{\pgfqpoint{1.635880in}{2.171776in}}{\pgfqpoint{1.627980in}{2.168503in}}{\pgfqpoint{1.622156in}{2.162680in}}%
\pgfpathcurveto{\pgfqpoint{1.616332in}{2.156856in}}{\pgfqpoint{1.613060in}{2.148956in}}{\pgfqpoint{1.613060in}{2.140719in}}%
\pgfpathcurveto{\pgfqpoint{1.613060in}{2.132483in}}{\pgfqpoint{1.616332in}{2.124583in}}{\pgfqpoint{1.622156in}{2.118759in}}%
\pgfpathcurveto{\pgfqpoint{1.627980in}{2.112935in}}{\pgfqpoint{1.635880in}{2.109663in}}{\pgfqpoint{1.644116in}{2.109663in}}%
\pgfpathclose%
\pgfusepath{stroke,fill}%
\end{pgfscope}%
\begin{pgfscope}%
\pgfpathrectangle{\pgfqpoint{0.100000in}{0.212622in}}{\pgfqpoint{3.696000in}{3.696000in}}%
\pgfusepath{clip}%
\pgfsetbuttcap%
\pgfsetroundjoin%
\definecolor{currentfill}{rgb}{0.121569,0.466667,0.705882}%
\pgfsetfillcolor{currentfill}%
\pgfsetfillopacity{0.301449}%
\pgfsetlinewidth{1.003750pt}%
\definecolor{currentstroke}{rgb}{0.121569,0.466667,0.705882}%
\pgfsetstrokecolor{currentstroke}%
\pgfsetstrokeopacity{0.301449}%
\pgfsetdash{}{0pt}%
\pgfpathmoveto{\pgfqpoint{1.644123in}{2.109667in}}%
\pgfpathcurveto{\pgfqpoint{1.652359in}{2.109667in}}{\pgfqpoint{1.660259in}{2.112939in}}{\pgfqpoint{1.666083in}{2.118763in}}%
\pgfpathcurveto{\pgfqpoint{1.671907in}{2.124587in}}{\pgfqpoint{1.675180in}{2.132487in}}{\pgfqpoint{1.675180in}{2.140724in}}%
\pgfpathcurveto{\pgfqpoint{1.675180in}{2.148960in}}{\pgfqpoint{1.671907in}{2.156860in}}{\pgfqpoint{1.666083in}{2.162684in}}%
\pgfpathcurveto{\pgfqpoint{1.660259in}{2.168508in}}{\pgfqpoint{1.652359in}{2.171780in}}{\pgfqpoint{1.644123in}{2.171780in}}%
\pgfpathcurveto{\pgfqpoint{1.635887in}{2.171780in}}{\pgfqpoint{1.627987in}{2.168508in}}{\pgfqpoint{1.622163in}{2.162684in}}%
\pgfpathcurveto{\pgfqpoint{1.616339in}{2.156860in}}{\pgfqpoint{1.613067in}{2.148960in}}{\pgfqpoint{1.613067in}{2.140724in}}%
\pgfpathcurveto{\pgfqpoint{1.613067in}{2.132487in}}{\pgfqpoint{1.616339in}{2.124587in}}{\pgfqpoint{1.622163in}{2.118763in}}%
\pgfpathcurveto{\pgfqpoint{1.627987in}{2.112939in}}{\pgfqpoint{1.635887in}{2.109667in}}{\pgfqpoint{1.644123in}{2.109667in}}%
\pgfpathclose%
\pgfusepath{stroke,fill}%
\end{pgfscope}%
\begin{pgfscope}%
\pgfpathrectangle{\pgfqpoint{0.100000in}{0.212622in}}{\pgfqpoint{3.696000in}{3.696000in}}%
\pgfusepath{clip}%
\pgfsetbuttcap%
\pgfsetroundjoin%
\definecolor{currentfill}{rgb}{0.121569,0.466667,0.705882}%
\pgfsetfillcolor{currentfill}%
\pgfsetfillopacity{0.301449}%
\pgfsetlinewidth{1.003750pt}%
\definecolor{currentstroke}{rgb}{0.121569,0.466667,0.705882}%
\pgfsetstrokecolor{currentstroke}%
\pgfsetstrokeopacity{0.301449}%
\pgfsetdash{}{0pt}%
\pgfpathmoveto{\pgfqpoint{1.644123in}{2.109667in}}%
\pgfpathcurveto{\pgfqpoint{1.652359in}{2.109667in}}{\pgfqpoint{1.660259in}{2.112939in}}{\pgfqpoint{1.666083in}{2.118763in}}%
\pgfpathcurveto{\pgfqpoint{1.671907in}{2.124587in}}{\pgfqpoint{1.675180in}{2.132487in}}{\pgfqpoint{1.675180in}{2.140724in}}%
\pgfpathcurveto{\pgfqpoint{1.675180in}{2.148960in}}{\pgfqpoint{1.671907in}{2.156860in}}{\pgfqpoint{1.666083in}{2.162684in}}%
\pgfpathcurveto{\pgfqpoint{1.660259in}{2.168508in}}{\pgfqpoint{1.652359in}{2.171780in}}{\pgfqpoint{1.644123in}{2.171780in}}%
\pgfpathcurveto{\pgfqpoint{1.635887in}{2.171780in}}{\pgfqpoint{1.627987in}{2.168508in}}{\pgfqpoint{1.622163in}{2.162684in}}%
\pgfpathcurveto{\pgfqpoint{1.616339in}{2.156860in}}{\pgfqpoint{1.613067in}{2.148960in}}{\pgfqpoint{1.613067in}{2.140724in}}%
\pgfpathcurveto{\pgfqpoint{1.613067in}{2.132487in}}{\pgfqpoint{1.616339in}{2.124587in}}{\pgfqpoint{1.622163in}{2.118763in}}%
\pgfpathcurveto{\pgfqpoint{1.627987in}{2.112939in}}{\pgfqpoint{1.635887in}{2.109667in}}{\pgfqpoint{1.644123in}{2.109667in}}%
\pgfpathclose%
\pgfusepath{stroke,fill}%
\end{pgfscope}%
\begin{pgfscope}%
\pgfpathrectangle{\pgfqpoint{0.100000in}{0.212622in}}{\pgfqpoint{3.696000in}{3.696000in}}%
\pgfusepath{clip}%
\pgfsetbuttcap%
\pgfsetroundjoin%
\definecolor{currentfill}{rgb}{0.121569,0.466667,0.705882}%
\pgfsetfillcolor{currentfill}%
\pgfsetfillopacity{0.301449}%
\pgfsetlinewidth{1.003750pt}%
\definecolor{currentstroke}{rgb}{0.121569,0.466667,0.705882}%
\pgfsetstrokecolor{currentstroke}%
\pgfsetstrokeopacity{0.301449}%
\pgfsetdash{}{0pt}%
\pgfpathmoveto{\pgfqpoint{1.644123in}{2.109667in}}%
\pgfpathcurveto{\pgfqpoint{1.652359in}{2.109667in}}{\pgfqpoint{1.660259in}{2.112939in}}{\pgfqpoint{1.666083in}{2.118763in}}%
\pgfpathcurveto{\pgfqpoint{1.671907in}{2.124587in}}{\pgfqpoint{1.675180in}{2.132487in}}{\pgfqpoint{1.675180in}{2.140724in}}%
\pgfpathcurveto{\pgfqpoint{1.675180in}{2.148960in}}{\pgfqpoint{1.671907in}{2.156860in}}{\pgfqpoint{1.666083in}{2.162684in}}%
\pgfpathcurveto{\pgfqpoint{1.660259in}{2.168508in}}{\pgfqpoint{1.652359in}{2.171780in}}{\pgfqpoint{1.644123in}{2.171780in}}%
\pgfpathcurveto{\pgfqpoint{1.635887in}{2.171780in}}{\pgfqpoint{1.627987in}{2.168508in}}{\pgfqpoint{1.622163in}{2.162684in}}%
\pgfpathcurveto{\pgfqpoint{1.616339in}{2.156860in}}{\pgfqpoint{1.613067in}{2.148960in}}{\pgfqpoint{1.613067in}{2.140724in}}%
\pgfpathcurveto{\pgfqpoint{1.613067in}{2.132487in}}{\pgfqpoint{1.616339in}{2.124587in}}{\pgfqpoint{1.622163in}{2.118763in}}%
\pgfpathcurveto{\pgfqpoint{1.627987in}{2.112939in}}{\pgfqpoint{1.635887in}{2.109667in}}{\pgfqpoint{1.644123in}{2.109667in}}%
\pgfpathclose%
\pgfusepath{stroke,fill}%
\end{pgfscope}%
\begin{pgfscope}%
\pgfpathrectangle{\pgfqpoint{0.100000in}{0.212622in}}{\pgfqpoint{3.696000in}{3.696000in}}%
\pgfusepath{clip}%
\pgfsetbuttcap%
\pgfsetroundjoin%
\definecolor{currentfill}{rgb}{0.121569,0.466667,0.705882}%
\pgfsetfillcolor{currentfill}%
\pgfsetfillopacity{0.301449}%
\pgfsetlinewidth{1.003750pt}%
\definecolor{currentstroke}{rgb}{0.121569,0.466667,0.705882}%
\pgfsetstrokecolor{currentstroke}%
\pgfsetstrokeopacity{0.301449}%
\pgfsetdash{}{0pt}%
\pgfpathmoveto{\pgfqpoint{1.644123in}{2.109667in}}%
\pgfpathcurveto{\pgfqpoint{1.652359in}{2.109667in}}{\pgfqpoint{1.660259in}{2.112939in}}{\pgfqpoint{1.666083in}{2.118763in}}%
\pgfpathcurveto{\pgfqpoint{1.671907in}{2.124587in}}{\pgfqpoint{1.675180in}{2.132487in}}{\pgfqpoint{1.675180in}{2.140724in}}%
\pgfpathcurveto{\pgfqpoint{1.675180in}{2.148960in}}{\pgfqpoint{1.671907in}{2.156860in}}{\pgfqpoint{1.666083in}{2.162684in}}%
\pgfpathcurveto{\pgfqpoint{1.660259in}{2.168508in}}{\pgfqpoint{1.652359in}{2.171780in}}{\pgfqpoint{1.644123in}{2.171780in}}%
\pgfpathcurveto{\pgfqpoint{1.635887in}{2.171780in}}{\pgfqpoint{1.627987in}{2.168508in}}{\pgfqpoint{1.622163in}{2.162684in}}%
\pgfpathcurveto{\pgfqpoint{1.616339in}{2.156860in}}{\pgfqpoint{1.613067in}{2.148960in}}{\pgfqpoint{1.613067in}{2.140724in}}%
\pgfpathcurveto{\pgfqpoint{1.613067in}{2.132487in}}{\pgfqpoint{1.616339in}{2.124587in}}{\pgfqpoint{1.622163in}{2.118763in}}%
\pgfpathcurveto{\pgfqpoint{1.627987in}{2.112939in}}{\pgfqpoint{1.635887in}{2.109667in}}{\pgfqpoint{1.644123in}{2.109667in}}%
\pgfpathclose%
\pgfusepath{stroke,fill}%
\end{pgfscope}%
\begin{pgfscope}%
\pgfpathrectangle{\pgfqpoint{0.100000in}{0.212622in}}{\pgfqpoint{3.696000in}{3.696000in}}%
\pgfusepath{clip}%
\pgfsetbuttcap%
\pgfsetroundjoin%
\definecolor{currentfill}{rgb}{0.121569,0.466667,0.705882}%
\pgfsetfillcolor{currentfill}%
\pgfsetfillopacity{0.301449}%
\pgfsetlinewidth{1.003750pt}%
\definecolor{currentstroke}{rgb}{0.121569,0.466667,0.705882}%
\pgfsetstrokecolor{currentstroke}%
\pgfsetstrokeopacity{0.301449}%
\pgfsetdash{}{0pt}%
\pgfpathmoveto{\pgfqpoint{1.644123in}{2.109667in}}%
\pgfpathcurveto{\pgfqpoint{1.652359in}{2.109667in}}{\pgfqpoint{1.660259in}{2.112939in}}{\pgfqpoint{1.666083in}{2.118763in}}%
\pgfpathcurveto{\pgfqpoint{1.671907in}{2.124587in}}{\pgfqpoint{1.675180in}{2.132487in}}{\pgfqpoint{1.675180in}{2.140724in}}%
\pgfpathcurveto{\pgfqpoint{1.675180in}{2.148960in}}{\pgfqpoint{1.671907in}{2.156860in}}{\pgfqpoint{1.666083in}{2.162684in}}%
\pgfpathcurveto{\pgfqpoint{1.660259in}{2.168508in}}{\pgfqpoint{1.652359in}{2.171780in}}{\pgfqpoint{1.644123in}{2.171780in}}%
\pgfpathcurveto{\pgfqpoint{1.635887in}{2.171780in}}{\pgfqpoint{1.627987in}{2.168508in}}{\pgfqpoint{1.622163in}{2.162684in}}%
\pgfpathcurveto{\pgfqpoint{1.616339in}{2.156860in}}{\pgfqpoint{1.613067in}{2.148960in}}{\pgfqpoint{1.613067in}{2.140724in}}%
\pgfpathcurveto{\pgfqpoint{1.613067in}{2.132487in}}{\pgfqpoint{1.616339in}{2.124587in}}{\pgfqpoint{1.622163in}{2.118763in}}%
\pgfpathcurveto{\pgfqpoint{1.627987in}{2.112939in}}{\pgfqpoint{1.635887in}{2.109667in}}{\pgfqpoint{1.644123in}{2.109667in}}%
\pgfpathclose%
\pgfusepath{stroke,fill}%
\end{pgfscope}%
\begin{pgfscope}%
\pgfpathrectangle{\pgfqpoint{0.100000in}{0.212622in}}{\pgfqpoint{3.696000in}{3.696000in}}%
\pgfusepath{clip}%
\pgfsetbuttcap%
\pgfsetroundjoin%
\definecolor{currentfill}{rgb}{0.121569,0.466667,0.705882}%
\pgfsetfillcolor{currentfill}%
\pgfsetfillopacity{0.301449}%
\pgfsetlinewidth{1.003750pt}%
\definecolor{currentstroke}{rgb}{0.121569,0.466667,0.705882}%
\pgfsetstrokecolor{currentstroke}%
\pgfsetstrokeopacity{0.301449}%
\pgfsetdash{}{0pt}%
\pgfpathmoveto{\pgfqpoint{1.644123in}{2.109667in}}%
\pgfpathcurveto{\pgfqpoint{1.652359in}{2.109667in}}{\pgfqpoint{1.660259in}{2.112939in}}{\pgfqpoint{1.666083in}{2.118763in}}%
\pgfpathcurveto{\pgfqpoint{1.671907in}{2.124587in}}{\pgfqpoint{1.675180in}{2.132487in}}{\pgfqpoint{1.675180in}{2.140724in}}%
\pgfpathcurveto{\pgfqpoint{1.675180in}{2.148960in}}{\pgfqpoint{1.671907in}{2.156860in}}{\pgfqpoint{1.666083in}{2.162684in}}%
\pgfpathcurveto{\pgfqpoint{1.660259in}{2.168508in}}{\pgfqpoint{1.652359in}{2.171780in}}{\pgfqpoint{1.644123in}{2.171780in}}%
\pgfpathcurveto{\pgfqpoint{1.635887in}{2.171780in}}{\pgfqpoint{1.627987in}{2.168508in}}{\pgfqpoint{1.622163in}{2.162684in}}%
\pgfpathcurveto{\pgfqpoint{1.616339in}{2.156860in}}{\pgfqpoint{1.613067in}{2.148960in}}{\pgfqpoint{1.613067in}{2.140724in}}%
\pgfpathcurveto{\pgfqpoint{1.613067in}{2.132487in}}{\pgfqpoint{1.616339in}{2.124587in}}{\pgfqpoint{1.622163in}{2.118763in}}%
\pgfpathcurveto{\pgfqpoint{1.627987in}{2.112939in}}{\pgfqpoint{1.635887in}{2.109667in}}{\pgfqpoint{1.644123in}{2.109667in}}%
\pgfpathclose%
\pgfusepath{stroke,fill}%
\end{pgfscope}%
\begin{pgfscope}%
\pgfpathrectangle{\pgfqpoint{0.100000in}{0.212622in}}{\pgfqpoint{3.696000in}{3.696000in}}%
\pgfusepath{clip}%
\pgfsetbuttcap%
\pgfsetroundjoin%
\definecolor{currentfill}{rgb}{0.121569,0.466667,0.705882}%
\pgfsetfillcolor{currentfill}%
\pgfsetfillopacity{0.301449}%
\pgfsetlinewidth{1.003750pt}%
\definecolor{currentstroke}{rgb}{0.121569,0.466667,0.705882}%
\pgfsetstrokecolor{currentstroke}%
\pgfsetstrokeopacity{0.301449}%
\pgfsetdash{}{0pt}%
\pgfpathmoveto{\pgfqpoint{1.644123in}{2.109667in}}%
\pgfpathcurveto{\pgfqpoint{1.652359in}{2.109667in}}{\pgfqpoint{1.660259in}{2.112939in}}{\pgfqpoint{1.666083in}{2.118763in}}%
\pgfpathcurveto{\pgfqpoint{1.671907in}{2.124587in}}{\pgfqpoint{1.675180in}{2.132487in}}{\pgfqpoint{1.675180in}{2.140724in}}%
\pgfpathcurveto{\pgfqpoint{1.675180in}{2.148960in}}{\pgfqpoint{1.671907in}{2.156860in}}{\pgfqpoint{1.666083in}{2.162684in}}%
\pgfpathcurveto{\pgfqpoint{1.660259in}{2.168508in}}{\pgfqpoint{1.652359in}{2.171780in}}{\pgfqpoint{1.644123in}{2.171780in}}%
\pgfpathcurveto{\pgfqpoint{1.635887in}{2.171780in}}{\pgfqpoint{1.627987in}{2.168508in}}{\pgfqpoint{1.622163in}{2.162684in}}%
\pgfpathcurveto{\pgfqpoint{1.616339in}{2.156860in}}{\pgfqpoint{1.613067in}{2.148960in}}{\pgfqpoint{1.613067in}{2.140724in}}%
\pgfpathcurveto{\pgfqpoint{1.613067in}{2.132487in}}{\pgfqpoint{1.616339in}{2.124587in}}{\pgfqpoint{1.622163in}{2.118763in}}%
\pgfpathcurveto{\pgfqpoint{1.627987in}{2.112939in}}{\pgfqpoint{1.635887in}{2.109667in}}{\pgfqpoint{1.644123in}{2.109667in}}%
\pgfpathclose%
\pgfusepath{stroke,fill}%
\end{pgfscope}%
\begin{pgfscope}%
\pgfpathrectangle{\pgfqpoint{0.100000in}{0.212622in}}{\pgfqpoint{3.696000in}{3.696000in}}%
\pgfusepath{clip}%
\pgfsetbuttcap%
\pgfsetroundjoin%
\definecolor{currentfill}{rgb}{0.121569,0.466667,0.705882}%
\pgfsetfillcolor{currentfill}%
\pgfsetfillopacity{0.301449}%
\pgfsetlinewidth{1.003750pt}%
\definecolor{currentstroke}{rgb}{0.121569,0.466667,0.705882}%
\pgfsetstrokecolor{currentstroke}%
\pgfsetstrokeopacity{0.301449}%
\pgfsetdash{}{0pt}%
\pgfpathmoveto{\pgfqpoint{1.644123in}{2.109667in}}%
\pgfpathcurveto{\pgfqpoint{1.652359in}{2.109667in}}{\pgfqpoint{1.660259in}{2.112939in}}{\pgfqpoint{1.666083in}{2.118763in}}%
\pgfpathcurveto{\pgfqpoint{1.671907in}{2.124587in}}{\pgfqpoint{1.675180in}{2.132487in}}{\pgfqpoint{1.675180in}{2.140724in}}%
\pgfpathcurveto{\pgfqpoint{1.675180in}{2.148960in}}{\pgfqpoint{1.671907in}{2.156860in}}{\pgfqpoint{1.666083in}{2.162684in}}%
\pgfpathcurveto{\pgfqpoint{1.660259in}{2.168508in}}{\pgfqpoint{1.652359in}{2.171780in}}{\pgfqpoint{1.644123in}{2.171780in}}%
\pgfpathcurveto{\pgfqpoint{1.635887in}{2.171780in}}{\pgfqpoint{1.627987in}{2.168508in}}{\pgfqpoint{1.622163in}{2.162684in}}%
\pgfpathcurveto{\pgfqpoint{1.616339in}{2.156860in}}{\pgfqpoint{1.613067in}{2.148960in}}{\pgfqpoint{1.613067in}{2.140724in}}%
\pgfpathcurveto{\pgfqpoint{1.613067in}{2.132487in}}{\pgfqpoint{1.616339in}{2.124587in}}{\pgfqpoint{1.622163in}{2.118763in}}%
\pgfpathcurveto{\pgfqpoint{1.627987in}{2.112939in}}{\pgfqpoint{1.635887in}{2.109667in}}{\pgfqpoint{1.644123in}{2.109667in}}%
\pgfpathclose%
\pgfusepath{stroke,fill}%
\end{pgfscope}%
\begin{pgfscope}%
\pgfpathrectangle{\pgfqpoint{0.100000in}{0.212622in}}{\pgfqpoint{3.696000in}{3.696000in}}%
\pgfusepath{clip}%
\pgfsetbuttcap%
\pgfsetroundjoin%
\definecolor{currentfill}{rgb}{0.121569,0.466667,0.705882}%
\pgfsetfillcolor{currentfill}%
\pgfsetfillopacity{0.301449}%
\pgfsetlinewidth{1.003750pt}%
\definecolor{currentstroke}{rgb}{0.121569,0.466667,0.705882}%
\pgfsetstrokecolor{currentstroke}%
\pgfsetstrokeopacity{0.301449}%
\pgfsetdash{}{0pt}%
\pgfpathmoveto{\pgfqpoint{1.644123in}{2.109667in}}%
\pgfpathcurveto{\pgfqpoint{1.652359in}{2.109667in}}{\pgfqpoint{1.660259in}{2.112939in}}{\pgfqpoint{1.666083in}{2.118763in}}%
\pgfpathcurveto{\pgfqpoint{1.671907in}{2.124587in}}{\pgfqpoint{1.675180in}{2.132487in}}{\pgfqpoint{1.675180in}{2.140724in}}%
\pgfpathcurveto{\pgfqpoint{1.675180in}{2.148960in}}{\pgfqpoint{1.671907in}{2.156860in}}{\pgfqpoint{1.666083in}{2.162684in}}%
\pgfpathcurveto{\pgfqpoint{1.660259in}{2.168508in}}{\pgfqpoint{1.652359in}{2.171780in}}{\pgfqpoint{1.644123in}{2.171780in}}%
\pgfpathcurveto{\pgfqpoint{1.635887in}{2.171780in}}{\pgfqpoint{1.627987in}{2.168508in}}{\pgfqpoint{1.622163in}{2.162684in}}%
\pgfpathcurveto{\pgfqpoint{1.616339in}{2.156860in}}{\pgfqpoint{1.613067in}{2.148960in}}{\pgfqpoint{1.613067in}{2.140724in}}%
\pgfpathcurveto{\pgfqpoint{1.613067in}{2.132487in}}{\pgfqpoint{1.616339in}{2.124587in}}{\pgfqpoint{1.622163in}{2.118763in}}%
\pgfpathcurveto{\pgfqpoint{1.627987in}{2.112939in}}{\pgfqpoint{1.635887in}{2.109667in}}{\pgfqpoint{1.644123in}{2.109667in}}%
\pgfpathclose%
\pgfusepath{stroke,fill}%
\end{pgfscope}%
\begin{pgfscope}%
\pgfpathrectangle{\pgfqpoint{0.100000in}{0.212622in}}{\pgfqpoint{3.696000in}{3.696000in}}%
\pgfusepath{clip}%
\pgfsetbuttcap%
\pgfsetroundjoin%
\definecolor{currentfill}{rgb}{0.121569,0.466667,0.705882}%
\pgfsetfillcolor{currentfill}%
\pgfsetfillopacity{0.301449}%
\pgfsetlinewidth{1.003750pt}%
\definecolor{currentstroke}{rgb}{0.121569,0.466667,0.705882}%
\pgfsetstrokecolor{currentstroke}%
\pgfsetstrokeopacity{0.301449}%
\pgfsetdash{}{0pt}%
\pgfpathmoveto{\pgfqpoint{1.644123in}{2.109667in}}%
\pgfpathcurveto{\pgfqpoint{1.652359in}{2.109667in}}{\pgfqpoint{1.660259in}{2.112939in}}{\pgfqpoint{1.666083in}{2.118763in}}%
\pgfpathcurveto{\pgfqpoint{1.671907in}{2.124587in}}{\pgfqpoint{1.675180in}{2.132487in}}{\pgfqpoint{1.675180in}{2.140724in}}%
\pgfpathcurveto{\pgfqpoint{1.675180in}{2.148960in}}{\pgfqpoint{1.671907in}{2.156860in}}{\pgfqpoint{1.666083in}{2.162684in}}%
\pgfpathcurveto{\pgfqpoint{1.660259in}{2.168508in}}{\pgfqpoint{1.652359in}{2.171780in}}{\pgfqpoint{1.644123in}{2.171780in}}%
\pgfpathcurveto{\pgfqpoint{1.635887in}{2.171780in}}{\pgfqpoint{1.627987in}{2.168508in}}{\pgfqpoint{1.622163in}{2.162684in}}%
\pgfpathcurveto{\pgfqpoint{1.616339in}{2.156860in}}{\pgfqpoint{1.613067in}{2.148960in}}{\pgfqpoint{1.613067in}{2.140724in}}%
\pgfpathcurveto{\pgfqpoint{1.613067in}{2.132487in}}{\pgfqpoint{1.616339in}{2.124587in}}{\pgfqpoint{1.622163in}{2.118763in}}%
\pgfpathcurveto{\pgfqpoint{1.627987in}{2.112939in}}{\pgfqpoint{1.635887in}{2.109667in}}{\pgfqpoint{1.644123in}{2.109667in}}%
\pgfpathclose%
\pgfusepath{stroke,fill}%
\end{pgfscope}%
\begin{pgfscope}%
\pgfpathrectangle{\pgfqpoint{0.100000in}{0.212622in}}{\pgfqpoint{3.696000in}{3.696000in}}%
\pgfusepath{clip}%
\pgfsetbuttcap%
\pgfsetroundjoin%
\definecolor{currentfill}{rgb}{0.121569,0.466667,0.705882}%
\pgfsetfillcolor{currentfill}%
\pgfsetfillopacity{0.301449}%
\pgfsetlinewidth{1.003750pt}%
\definecolor{currentstroke}{rgb}{0.121569,0.466667,0.705882}%
\pgfsetstrokecolor{currentstroke}%
\pgfsetstrokeopacity{0.301449}%
\pgfsetdash{}{0pt}%
\pgfpathmoveto{\pgfqpoint{1.644123in}{2.109667in}}%
\pgfpathcurveto{\pgfqpoint{1.652359in}{2.109667in}}{\pgfqpoint{1.660259in}{2.112939in}}{\pgfqpoint{1.666083in}{2.118763in}}%
\pgfpathcurveto{\pgfqpoint{1.671907in}{2.124587in}}{\pgfqpoint{1.675180in}{2.132487in}}{\pgfqpoint{1.675180in}{2.140724in}}%
\pgfpathcurveto{\pgfqpoint{1.675180in}{2.148960in}}{\pgfqpoint{1.671907in}{2.156860in}}{\pgfqpoint{1.666083in}{2.162684in}}%
\pgfpathcurveto{\pgfqpoint{1.660259in}{2.168508in}}{\pgfqpoint{1.652359in}{2.171780in}}{\pgfqpoint{1.644123in}{2.171780in}}%
\pgfpathcurveto{\pgfqpoint{1.635887in}{2.171780in}}{\pgfqpoint{1.627987in}{2.168508in}}{\pgfqpoint{1.622163in}{2.162684in}}%
\pgfpathcurveto{\pgfqpoint{1.616339in}{2.156860in}}{\pgfqpoint{1.613067in}{2.148960in}}{\pgfqpoint{1.613067in}{2.140724in}}%
\pgfpathcurveto{\pgfqpoint{1.613067in}{2.132487in}}{\pgfqpoint{1.616339in}{2.124587in}}{\pgfqpoint{1.622163in}{2.118763in}}%
\pgfpathcurveto{\pgfqpoint{1.627987in}{2.112939in}}{\pgfqpoint{1.635887in}{2.109667in}}{\pgfqpoint{1.644123in}{2.109667in}}%
\pgfpathclose%
\pgfusepath{stroke,fill}%
\end{pgfscope}%
\begin{pgfscope}%
\pgfpathrectangle{\pgfqpoint{0.100000in}{0.212622in}}{\pgfqpoint{3.696000in}{3.696000in}}%
\pgfusepath{clip}%
\pgfsetbuttcap%
\pgfsetroundjoin%
\definecolor{currentfill}{rgb}{0.121569,0.466667,0.705882}%
\pgfsetfillcolor{currentfill}%
\pgfsetfillopacity{0.301449}%
\pgfsetlinewidth{1.003750pt}%
\definecolor{currentstroke}{rgb}{0.121569,0.466667,0.705882}%
\pgfsetstrokecolor{currentstroke}%
\pgfsetstrokeopacity{0.301449}%
\pgfsetdash{}{0pt}%
\pgfpathmoveto{\pgfqpoint{1.644123in}{2.109667in}}%
\pgfpathcurveto{\pgfqpoint{1.652359in}{2.109667in}}{\pgfqpoint{1.660259in}{2.112939in}}{\pgfqpoint{1.666083in}{2.118763in}}%
\pgfpathcurveto{\pgfqpoint{1.671907in}{2.124587in}}{\pgfqpoint{1.675180in}{2.132487in}}{\pgfqpoint{1.675180in}{2.140724in}}%
\pgfpathcurveto{\pgfqpoint{1.675180in}{2.148960in}}{\pgfqpoint{1.671907in}{2.156860in}}{\pgfqpoint{1.666083in}{2.162684in}}%
\pgfpathcurveto{\pgfqpoint{1.660259in}{2.168508in}}{\pgfqpoint{1.652359in}{2.171780in}}{\pgfqpoint{1.644123in}{2.171780in}}%
\pgfpathcurveto{\pgfqpoint{1.635887in}{2.171780in}}{\pgfqpoint{1.627987in}{2.168508in}}{\pgfqpoint{1.622163in}{2.162684in}}%
\pgfpathcurveto{\pgfqpoint{1.616339in}{2.156860in}}{\pgfqpoint{1.613067in}{2.148960in}}{\pgfqpoint{1.613067in}{2.140724in}}%
\pgfpathcurveto{\pgfqpoint{1.613067in}{2.132487in}}{\pgfqpoint{1.616339in}{2.124587in}}{\pgfqpoint{1.622163in}{2.118763in}}%
\pgfpathcurveto{\pgfqpoint{1.627987in}{2.112939in}}{\pgfqpoint{1.635887in}{2.109667in}}{\pgfqpoint{1.644123in}{2.109667in}}%
\pgfpathclose%
\pgfusepath{stroke,fill}%
\end{pgfscope}%
\begin{pgfscope}%
\pgfpathrectangle{\pgfqpoint{0.100000in}{0.212622in}}{\pgfqpoint{3.696000in}{3.696000in}}%
\pgfusepath{clip}%
\pgfsetbuttcap%
\pgfsetroundjoin%
\definecolor{currentfill}{rgb}{0.121569,0.466667,0.705882}%
\pgfsetfillcolor{currentfill}%
\pgfsetfillopacity{0.301449}%
\pgfsetlinewidth{1.003750pt}%
\definecolor{currentstroke}{rgb}{0.121569,0.466667,0.705882}%
\pgfsetstrokecolor{currentstroke}%
\pgfsetstrokeopacity{0.301449}%
\pgfsetdash{}{0pt}%
\pgfpathmoveto{\pgfqpoint{1.644123in}{2.109667in}}%
\pgfpathcurveto{\pgfqpoint{1.652359in}{2.109667in}}{\pgfqpoint{1.660259in}{2.112939in}}{\pgfqpoint{1.666083in}{2.118763in}}%
\pgfpathcurveto{\pgfqpoint{1.671907in}{2.124587in}}{\pgfqpoint{1.675180in}{2.132487in}}{\pgfqpoint{1.675180in}{2.140724in}}%
\pgfpathcurveto{\pgfqpoint{1.675180in}{2.148960in}}{\pgfqpoint{1.671907in}{2.156860in}}{\pgfqpoint{1.666083in}{2.162684in}}%
\pgfpathcurveto{\pgfqpoint{1.660259in}{2.168508in}}{\pgfqpoint{1.652359in}{2.171780in}}{\pgfqpoint{1.644123in}{2.171780in}}%
\pgfpathcurveto{\pgfqpoint{1.635887in}{2.171780in}}{\pgfqpoint{1.627987in}{2.168508in}}{\pgfqpoint{1.622163in}{2.162684in}}%
\pgfpathcurveto{\pgfqpoint{1.616339in}{2.156860in}}{\pgfqpoint{1.613067in}{2.148960in}}{\pgfqpoint{1.613067in}{2.140724in}}%
\pgfpathcurveto{\pgfqpoint{1.613067in}{2.132487in}}{\pgfqpoint{1.616339in}{2.124587in}}{\pgfqpoint{1.622163in}{2.118763in}}%
\pgfpathcurveto{\pgfqpoint{1.627987in}{2.112939in}}{\pgfqpoint{1.635887in}{2.109667in}}{\pgfqpoint{1.644123in}{2.109667in}}%
\pgfpathclose%
\pgfusepath{stroke,fill}%
\end{pgfscope}%
\begin{pgfscope}%
\pgfpathrectangle{\pgfqpoint{0.100000in}{0.212622in}}{\pgfqpoint{3.696000in}{3.696000in}}%
\pgfusepath{clip}%
\pgfsetbuttcap%
\pgfsetroundjoin%
\definecolor{currentfill}{rgb}{0.121569,0.466667,0.705882}%
\pgfsetfillcolor{currentfill}%
\pgfsetfillopacity{0.301449}%
\pgfsetlinewidth{1.003750pt}%
\definecolor{currentstroke}{rgb}{0.121569,0.466667,0.705882}%
\pgfsetstrokecolor{currentstroke}%
\pgfsetstrokeopacity{0.301449}%
\pgfsetdash{}{0pt}%
\pgfpathmoveto{\pgfqpoint{1.644123in}{2.109667in}}%
\pgfpathcurveto{\pgfqpoint{1.652359in}{2.109667in}}{\pgfqpoint{1.660259in}{2.112939in}}{\pgfqpoint{1.666083in}{2.118763in}}%
\pgfpathcurveto{\pgfqpoint{1.671907in}{2.124587in}}{\pgfqpoint{1.675180in}{2.132487in}}{\pgfqpoint{1.675180in}{2.140724in}}%
\pgfpathcurveto{\pgfqpoint{1.675180in}{2.148960in}}{\pgfqpoint{1.671907in}{2.156860in}}{\pgfqpoint{1.666083in}{2.162684in}}%
\pgfpathcurveto{\pgfqpoint{1.660259in}{2.168508in}}{\pgfqpoint{1.652359in}{2.171780in}}{\pgfqpoint{1.644123in}{2.171780in}}%
\pgfpathcurveto{\pgfqpoint{1.635887in}{2.171780in}}{\pgfqpoint{1.627987in}{2.168508in}}{\pgfqpoint{1.622163in}{2.162684in}}%
\pgfpathcurveto{\pgfqpoint{1.616339in}{2.156860in}}{\pgfqpoint{1.613067in}{2.148960in}}{\pgfqpoint{1.613067in}{2.140724in}}%
\pgfpathcurveto{\pgfqpoint{1.613067in}{2.132487in}}{\pgfqpoint{1.616339in}{2.124587in}}{\pgfqpoint{1.622163in}{2.118763in}}%
\pgfpathcurveto{\pgfqpoint{1.627987in}{2.112939in}}{\pgfqpoint{1.635887in}{2.109667in}}{\pgfqpoint{1.644123in}{2.109667in}}%
\pgfpathclose%
\pgfusepath{stroke,fill}%
\end{pgfscope}%
\begin{pgfscope}%
\pgfpathrectangle{\pgfqpoint{0.100000in}{0.212622in}}{\pgfqpoint{3.696000in}{3.696000in}}%
\pgfusepath{clip}%
\pgfsetbuttcap%
\pgfsetroundjoin%
\definecolor{currentfill}{rgb}{0.121569,0.466667,0.705882}%
\pgfsetfillcolor{currentfill}%
\pgfsetfillopacity{0.301449}%
\pgfsetlinewidth{1.003750pt}%
\definecolor{currentstroke}{rgb}{0.121569,0.466667,0.705882}%
\pgfsetstrokecolor{currentstroke}%
\pgfsetstrokeopacity{0.301449}%
\pgfsetdash{}{0pt}%
\pgfpathmoveto{\pgfqpoint{1.644123in}{2.109667in}}%
\pgfpathcurveto{\pgfqpoint{1.652359in}{2.109667in}}{\pgfqpoint{1.660259in}{2.112939in}}{\pgfqpoint{1.666083in}{2.118763in}}%
\pgfpathcurveto{\pgfqpoint{1.671907in}{2.124587in}}{\pgfqpoint{1.675180in}{2.132487in}}{\pgfqpoint{1.675180in}{2.140724in}}%
\pgfpathcurveto{\pgfqpoint{1.675180in}{2.148960in}}{\pgfqpoint{1.671907in}{2.156860in}}{\pgfqpoint{1.666083in}{2.162684in}}%
\pgfpathcurveto{\pgfqpoint{1.660259in}{2.168508in}}{\pgfqpoint{1.652359in}{2.171780in}}{\pgfqpoint{1.644123in}{2.171780in}}%
\pgfpathcurveto{\pgfqpoint{1.635887in}{2.171780in}}{\pgfqpoint{1.627987in}{2.168508in}}{\pgfqpoint{1.622163in}{2.162684in}}%
\pgfpathcurveto{\pgfqpoint{1.616339in}{2.156860in}}{\pgfqpoint{1.613067in}{2.148960in}}{\pgfqpoint{1.613067in}{2.140724in}}%
\pgfpathcurveto{\pgfqpoint{1.613067in}{2.132487in}}{\pgfqpoint{1.616339in}{2.124587in}}{\pgfqpoint{1.622163in}{2.118763in}}%
\pgfpathcurveto{\pgfqpoint{1.627987in}{2.112939in}}{\pgfqpoint{1.635887in}{2.109667in}}{\pgfqpoint{1.644123in}{2.109667in}}%
\pgfpathclose%
\pgfusepath{stroke,fill}%
\end{pgfscope}%
\begin{pgfscope}%
\pgfpathrectangle{\pgfqpoint{0.100000in}{0.212622in}}{\pgfqpoint{3.696000in}{3.696000in}}%
\pgfusepath{clip}%
\pgfsetbuttcap%
\pgfsetroundjoin%
\definecolor{currentfill}{rgb}{0.121569,0.466667,0.705882}%
\pgfsetfillcolor{currentfill}%
\pgfsetfillopacity{0.301449}%
\pgfsetlinewidth{1.003750pt}%
\definecolor{currentstroke}{rgb}{0.121569,0.466667,0.705882}%
\pgfsetstrokecolor{currentstroke}%
\pgfsetstrokeopacity{0.301449}%
\pgfsetdash{}{0pt}%
\pgfpathmoveto{\pgfqpoint{1.644123in}{2.109667in}}%
\pgfpathcurveto{\pgfqpoint{1.652359in}{2.109667in}}{\pgfqpoint{1.660259in}{2.112939in}}{\pgfqpoint{1.666083in}{2.118763in}}%
\pgfpathcurveto{\pgfqpoint{1.671907in}{2.124587in}}{\pgfqpoint{1.675180in}{2.132487in}}{\pgfqpoint{1.675180in}{2.140724in}}%
\pgfpathcurveto{\pgfqpoint{1.675180in}{2.148960in}}{\pgfqpoint{1.671907in}{2.156860in}}{\pgfqpoint{1.666083in}{2.162684in}}%
\pgfpathcurveto{\pgfqpoint{1.660259in}{2.168508in}}{\pgfqpoint{1.652359in}{2.171780in}}{\pgfqpoint{1.644123in}{2.171780in}}%
\pgfpathcurveto{\pgfqpoint{1.635887in}{2.171780in}}{\pgfqpoint{1.627987in}{2.168508in}}{\pgfqpoint{1.622163in}{2.162684in}}%
\pgfpathcurveto{\pgfqpoint{1.616339in}{2.156860in}}{\pgfqpoint{1.613067in}{2.148960in}}{\pgfqpoint{1.613067in}{2.140724in}}%
\pgfpathcurveto{\pgfqpoint{1.613067in}{2.132487in}}{\pgfqpoint{1.616339in}{2.124587in}}{\pgfqpoint{1.622163in}{2.118763in}}%
\pgfpathcurveto{\pgfqpoint{1.627987in}{2.112939in}}{\pgfqpoint{1.635887in}{2.109667in}}{\pgfqpoint{1.644123in}{2.109667in}}%
\pgfpathclose%
\pgfusepath{stroke,fill}%
\end{pgfscope}%
\begin{pgfscope}%
\pgfpathrectangle{\pgfqpoint{0.100000in}{0.212622in}}{\pgfqpoint{3.696000in}{3.696000in}}%
\pgfusepath{clip}%
\pgfsetbuttcap%
\pgfsetroundjoin%
\definecolor{currentfill}{rgb}{0.121569,0.466667,0.705882}%
\pgfsetfillcolor{currentfill}%
\pgfsetfillopacity{0.301449}%
\pgfsetlinewidth{1.003750pt}%
\definecolor{currentstroke}{rgb}{0.121569,0.466667,0.705882}%
\pgfsetstrokecolor{currentstroke}%
\pgfsetstrokeopacity{0.301449}%
\pgfsetdash{}{0pt}%
\pgfpathmoveto{\pgfqpoint{1.644123in}{2.109667in}}%
\pgfpathcurveto{\pgfqpoint{1.652359in}{2.109667in}}{\pgfqpoint{1.660259in}{2.112939in}}{\pgfqpoint{1.666083in}{2.118763in}}%
\pgfpathcurveto{\pgfqpoint{1.671907in}{2.124587in}}{\pgfqpoint{1.675180in}{2.132487in}}{\pgfqpoint{1.675180in}{2.140724in}}%
\pgfpathcurveto{\pgfqpoint{1.675180in}{2.148960in}}{\pgfqpoint{1.671907in}{2.156860in}}{\pgfqpoint{1.666083in}{2.162684in}}%
\pgfpathcurveto{\pgfqpoint{1.660259in}{2.168508in}}{\pgfqpoint{1.652359in}{2.171780in}}{\pgfqpoint{1.644123in}{2.171780in}}%
\pgfpathcurveto{\pgfqpoint{1.635887in}{2.171780in}}{\pgfqpoint{1.627987in}{2.168508in}}{\pgfqpoint{1.622163in}{2.162684in}}%
\pgfpathcurveto{\pgfqpoint{1.616339in}{2.156860in}}{\pgfqpoint{1.613067in}{2.148960in}}{\pgfqpoint{1.613067in}{2.140724in}}%
\pgfpathcurveto{\pgfqpoint{1.613067in}{2.132487in}}{\pgfqpoint{1.616339in}{2.124587in}}{\pgfqpoint{1.622163in}{2.118763in}}%
\pgfpathcurveto{\pgfqpoint{1.627987in}{2.112939in}}{\pgfqpoint{1.635887in}{2.109667in}}{\pgfqpoint{1.644123in}{2.109667in}}%
\pgfpathclose%
\pgfusepath{stroke,fill}%
\end{pgfscope}%
\begin{pgfscope}%
\pgfpathrectangle{\pgfqpoint{0.100000in}{0.212622in}}{\pgfqpoint{3.696000in}{3.696000in}}%
\pgfusepath{clip}%
\pgfsetbuttcap%
\pgfsetroundjoin%
\definecolor{currentfill}{rgb}{0.121569,0.466667,0.705882}%
\pgfsetfillcolor{currentfill}%
\pgfsetfillopacity{0.301449}%
\pgfsetlinewidth{1.003750pt}%
\definecolor{currentstroke}{rgb}{0.121569,0.466667,0.705882}%
\pgfsetstrokecolor{currentstroke}%
\pgfsetstrokeopacity{0.301449}%
\pgfsetdash{}{0pt}%
\pgfpathmoveto{\pgfqpoint{1.644123in}{2.109667in}}%
\pgfpathcurveto{\pgfqpoint{1.652359in}{2.109667in}}{\pgfqpoint{1.660259in}{2.112939in}}{\pgfqpoint{1.666083in}{2.118763in}}%
\pgfpathcurveto{\pgfqpoint{1.671907in}{2.124587in}}{\pgfqpoint{1.675180in}{2.132487in}}{\pgfqpoint{1.675180in}{2.140724in}}%
\pgfpathcurveto{\pgfqpoint{1.675180in}{2.148960in}}{\pgfqpoint{1.671907in}{2.156860in}}{\pgfqpoint{1.666083in}{2.162684in}}%
\pgfpathcurveto{\pgfqpoint{1.660259in}{2.168508in}}{\pgfqpoint{1.652359in}{2.171780in}}{\pgfqpoint{1.644123in}{2.171780in}}%
\pgfpathcurveto{\pgfqpoint{1.635887in}{2.171780in}}{\pgfqpoint{1.627987in}{2.168508in}}{\pgfqpoint{1.622163in}{2.162684in}}%
\pgfpathcurveto{\pgfqpoint{1.616339in}{2.156860in}}{\pgfqpoint{1.613067in}{2.148960in}}{\pgfqpoint{1.613067in}{2.140724in}}%
\pgfpathcurveto{\pgfqpoint{1.613067in}{2.132487in}}{\pgfqpoint{1.616339in}{2.124587in}}{\pgfqpoint{1.622163in}{2.118763in}}%
\pgfpathcurveto{\pgfqpoint{1.627987in}{2.112939in}}{\pgfqpoint{1.635887in}{2.109667in}}{\pgfqpoint{1.644123in}{2.109667in}}%
\pgfpathclose%
\pgfusepath{stroke,fill}%
\end{pgfscope}%
\begin{pgfscope}%
\pgfpathrectangle{\pgfqpoint{0.100000in}{0.212622in}}{\pgfqpoint{3.696000in}{3.696000in}}%
\pgfusepath{clip}%
\pgfsetbuttcap%
\pgfsetroundjoin%
\definecolor{currentfill}{rgb}{0.121569,0.466667,0.705882}%
\pgfsetfillcolor{currentfill}%
\pgfsetfillopacity{0.301449}%
\pgfsetlinewidth{1.003750pt}%
\definecolor{currentstroke}{rgb}{0.121569,0.466667,0.705882}%
\pgfsetstrokecolor{currentstroke}%
\pgfsetstrokeopacity{0.301449}%
\pgfsetdash{}{0pt}%
\pgfpathmoveto{\pgfqpoint{1.644123in}{2.109667in}}%
\pgfpathcurveto{\pgfqpoint{1.652359in}{2.109667in}}{\pgfqpoint{1.660259in}{2.112939in}}{\pgfqpoint{1.666083in}{2.118763in}}%
\pgfpathcurveto{\pgfqpoint{1.671907in}{2.124587in}}{\pgfqpoint{1.675180in}{2.132487in}}{\pgfqpoint{1.675180in}{2.140724in}}%
\pgfpathcurveto{\pgfqpoint{1.675180in}{2.148960in}}{\pgfqpoint{1.671907in}{2.156860in}}{\pgfqpoint{1.666083in}{2.162684in}}%
\pgfpathcurveto{\pgfqpoint{1.660259in}{2.168508in}}{\pgfqpoint{1.652359in}{2.171780in}}{\pgfqpoint{1.644123in}{2.171780in}}%
\pgfpathcurveto{\pgfqpoint{1.635887in}{2.171780in}}{\pgfqpoint{1.627987in}{2.168508in}}{\pgfqpoint{1.622163in}{2.162684in}}%
\pgfpathcurveto{\pgfqpoint{1.616339in}{2.156860in}}{\pgfqpoint{1.613067in}{2.148960in}}{\pgfqpoint{1.613067in}{2.140724in}}%
\pgfpathcurveto{\pgfqpoint{1.613067in}{2.132487in}}{\pgfqpoint{1.616339in}{2.124587in}}{\pgfqpoint{1.622163in}{2.118763in}}%
\pgfpathcurveto{\pgfqpoint{1.627987in}{2.112939in}}{\pgfqpoint{1.635887in}{2.109667in}}{\pgfqpoint{1.644123in}{2.109667in}}%
\pgfpathclose%
\pgfusepath{stroke,fill}%
\end{pgfscope}%
\begin{pgfscope}%
\pgfpathrectangle{\pgfqpoint{0.100000in}{0.212622in}}{\pgfqpoint{3.696000in}{3.696000in}}%
\pgfusepath{clip}%
\pgfsetbuttcap%
\pgfsetroundjoin%
\definecolor{currentfill}{rgb}{0.121569,0.466667,0.705882}%
\pgfsetfillcolor{currentfill}%
\pgfsetfillopacity{0.301449}%
\pgfsetlinewidth{1.003750pt}%
\definecolor{currentstroke}{rgb}{0.121569,0.466667,0.705882}%
\pgfsetstrokecolor{currentstroke}%
\pgfsetstrokeopacity{0.301449}%
\pgfsetdash{}{0pt}%
\pgfpathmoveto{\pgfqpoint{1.644123in}{2.109667in}}%
\pgfpathcurveto{\pgfqpoint{1.652359in}{2.109667in}}{\pgfqpoint{1.660259in}{2.112939in}}{\pgfqpoint{1.666083in}{2.118763in}}%
\pgfpathcurveto{\pgfqpoint{1.671907in}{2.124587in}}{\pgfqpoint{1.675180in}{2.132487in}}{\pgfqpoint{1.675180in}{2.140724in}}%
\pgfpathcurveto{\pgfqpoint{1.675180in}{2.148960in}}{\pgfqpoint{1.671907in}{2.156860in}}{\pgfqpoint{1.666083in}{2.162684in}}%
\pgfpathcurveto{\pgfqpoint{1.660259in}{2.168508in}}{\pgfqpoint{1.652359in}{2.171780in}}{\pgfqpoint{1.644123in}{2.171780in}}%
\pgfpathcurveto{\pgfqpoint{1.635887in}{2.171780in}}{\pgfqpoint{1.627987in}{2.168508in}}{\pgfqpoint{1.622163in}{2.162684in}}%
\pgfpathcurveto{\pgfqpoint{1.616339in}{2.156860in}}{\pgfqpoint{1.613067in}{2.148960in}}{\pgfqpoint{1.613067in}{2.140724in}}%
\pgfpathcurveto{\pgfqpoint{1.613067in}{2.132487in}}{\pgfqpoint{1.616339in}{2.124587in}}{\pgfqpoint{1.622163in}{2.118763in}}%
\pgfpathcurveto{\pgfqpoint{1.627987in}{2.112939in}}{\pgfqpoint{1.635887in}{2.109667in}}{\pgfqpoint{1.644123in}{2.109667in}}%
\pgfpathclose%
\pgfusepath{stroke,fill}%
\end{pgfscope}%
\begin{pgfscope}%
\pgfpathrectangle{\pgfqpoint{0.100000in}{0.212622in}}{\pgfqpoint{3.696000in}{3.696000in}}%
\pgfusepath{clip}%
\pgfsetbuttcap%
\pgfsetroundjoin%
\definecolor{currentfill}{rgb}{0.121569,0.466667,0.705882}%
\pgfsetfillcolor{currentfill}%
\pgfsetfillopacity{0.301449}%
\pgfsetlinewidth{1.003750pt}%
\definecolor{currentstroke}{rgb}{0.121569,0.466667,0.705882}%
\pgfsetstrokecolor{currentstroke}%
\pgfsetstrokeopacity{0.301449}%
\pgfsetdash{}{0pt}%
\pgfpathmoveto{\pgfqpoint{1.644123in}{2.109667in}}%
\pgfpathcurveto{\pgfqpoint{1.652359in}{2.109667in}}{\pgfqpoint{1.660259in}{2.112939in}}{\pgfqpoint{1.666083in}{2.118763in}}%
\pgfpathcurveto{\pgfqpoint{1.671907in}{2.124587in}}{\pgfqpoint{1.675180in}{2.132487in}}{\pgfqpoint{1.675180in}{2.140724in}}%
\pgfpathcurveto{\pgfqpoint{1.675180in}{2.148960in}}{\pgfqpoint{1.671907in}{2.156860in}}{\pgfqpoint{1.666083in}{2.162684in}}%
\pgfpathcurveto{\pgfqpoint{1.660259in}{2.168508in}}{\pgfqpoint{1.652359in}{2.171780in}}{\pgfqpoint{1.644123in}{2.171780in}}%
\pgfpathcurveto{\pgfqpoint{1.635887in}{2.171780in}}{\pgfqpoint{1.627987in}{2.168508in}}{\pgfqpoint{1.622163in}{2.162684in}}%
\pgfpathcurveto{\pgfqpoint{1.616339in}{2.156860in}}{\pgfqpoint{1.613067in}{2.148960in}}{\pgfqpoint{1.613067in}{2.140724in}}%
\pgfpathcurveto{\pgfqpoint{1.613067in}{2.132487in}}{\pgfqpoint{1.616339in}{2.124587in}}{\pgfqpoint{1.622163in}{2.118763in}}%
\pgfpathcurveto{\pgfqpoint{1.627987in}{2.112939in}}{\pgfqpoint{1.635887in}{2.109667in}}{\pgfqpoint{1.644123in}{2.109667in}}%
\pgfpathclose%
\pgfusepath{stroke,fill}%
\end{pgfscope}%
\begin{pgfscope}%
\pgfpathrectangle{\pgfqpoint{0.100000in}{0.212622in}}{\pgfqpoint{3.696000in}{3.696000in}}%
\pgfusepath{clip}%
\pgfsetbuttcap%
\pgfsetroundjoin%
\definecolor{currentfill}{rgb}{0.121569,0.466667,0.705882}%
\pgfsetfillcolor{currentfill}%
\pgfsetfillopacity{0.301449}%
\pgfsetlinewidth{1.003750pt}%
\definecolor{currentstroke}{rgb}{0.121569,0.466667,0.705882}%
\pgfsetstrokecolor{currentstroke}%
\pgfsetstrokeopacity{0.301449}%
\pgfsetdash{}{0pt}%
\pgfpathmoveto{\pgfqpoint{1.644123in}{2.109667in}}%
\pgfpathcurveto{\pgfqpoint{1.652359in}{2.109667in}}{\pgfqpoint{1.660259in}{2.112939in}}{\pgfqpoint{1.666083in}{2.118763in}}%
\pgfpathcurveto{\pgfqpoint{1.671907in}{2.124587in}}{\pgfqpoint{1.675180in}{2.132487in}}{\pgfqpoint{1.675180in}{2.140724in}}%
\pgfpathcurveto{\pgfqpoint{1.675180in}{2.148960in}}{\pgfqpoint{1.671907in}{2.156860in}}{\pgfqpoint{1.666083in}{2.162684in}}%
\pgfpathcurveto{\pgfqpoint{1.660259in}{2.168508in}}{\pgfqpoint{1.652359in}{2.171780in}}{\pgfqpoint{1.644123in}{2.171780in}}%
\pgfpathcurveto{\pgfqpoint{1.635887in}{2.171780in}}{\pgfqpoint{1.627987in}{2.168508in}}{\pgfqpoint{1.622163in}{2.162684in}}%
\pgfpathcurveto{\pgfqpoint{1.616339in}{2.156860in}}{\pgfqpoint{1.613067in}{2.148960in}}{\pgfqpoint{1.613067in}{2.140724in}}%
\pgfpathcurveto{\pgfqpoint{1.613067in}{2.132487in}}{\pgfqpoint{1.616339in}{2.124587in}}{\pgfqpoint{1.622163in}{2.118763in}}%
\pgfpathcurveto{\pgfqpoint{1.627987in}{2.112939in}}{\pgfqpoint{1.635887in}{2.109667in}}{\pgfqpoint{1.644123in}{2.109667in}}%
\pgfpathclose%
\pgfusepath{stroke,fill}%
\end{pgfscope}%
\begin{pgfscope}%
\pgfpathrectangle{\pgfqpoint{0.100000in}{0.212622in}}{\pgfqpoint{3.696000in}{3.696000in}}%
\pgfusepath{clip}%
\pgfsetbuttcap%
\pgfsetroundjoin%
\definecolor{currentfill}{rgb}{0.121569,0.466667,0.705882}%
\pgfsetfillcolor{currentfill}%
\pgfsetfillopacity{0.301449}%
\pgfsetlinewidth{1.003750pt}%
\definecolor{currentstroke}{rgb}{0.121569,0.466667,0.705882}%
\pgfsetstrokecolor{currentstroke}%
\pgfsetstrokeopacity{0.301449}%
\pgfsetdash{}{0pt}%
\pgfpathmoveto{\pgfqpoint{1.644123in}{2.109667in}}%
\pgfpathcurveto{\pgfqpoint{1.652359in}{2.109667in}}{\pgfqpoint{1.660259in}{2.112939in}}{\pgfqpoint{1.666083in}{2.118763in}}%
\pgfpathcurveto{\pgfqpoint{1.671907in}{2.124587in}}{\pgfqpoint{1.675180in}{2.132487in}}{\pgfqpoint{1.675180in}{2.140724in}}%
\pgfpathcurveto{\pgfqpoint{1.675180in}{2.148960in}}{\pgfqpoint{1.671907in}{2.156860in}}{\pgfqpoint{1.666083in}{2.162684in}}%
\pgfpathcurveto{\pgfqpoint{1.660259in}{2.168508in}}{\pgfqpoint{1.652359in}{2.171780in}}{\pgfqpoint{1.644123in}{2.171780in}}%
\pgfpathcurveto{\pgfqpoint{1.635887in}{2.171780in}}{\pgfqpoint{1.627987in}{2.168508in}}{\pgfqpoint{1.622163in}{2.162684in}}%
\pgfpathcurveto{\pgfqpoint{1.616339in}{2.156860in}}{\pgfqpoint{1.613067in}{2.148960in}}{\pgfqpoint{1.613067in}{2.140724in}}%
\pgfpathcurveto{\pgfqpoint{1.613067in}{2.132487in}}{\pgfqpoint{1.616339in}{2.124587in}}{\pgfqpoint{1.622163in}{2.118763in}}%
\pgfpathcurveto{\pgfqpoint{1.627987in}{2.112939in}}{\pgfqpoint{1.635887in}{2.109667in}}{\pgfqpoint{1.644123in}{2.109667in}}%
\pgfpathclose%
\pgfusepath{stroke,fill}%
\end{pgfscope}%
\begin{pgfscope}%
\pgfpathrectangle{\pgfqpoint{0.100000in}{0.212622in}}{\pgfqpoint{3.696000in}{3.696000in}}%
\pgfusepath{clip}%
\pgfsetbuttcap%
\pgfsetroundjoin%
\definecolor{currentfill}{rgb}{0.121569,0.466667,0.705882}%
\pgfsetfillcolor{currentfill}%
\pgfsetfillopacity{0.301449}%
\pgfsetlinewidth{1.003750pt}%
\definecolor{currentstroke}{rgb}{0.121569,0.466667,0.705882}%
\pgfsetstrokecolor{currentstroke}%
\pgfsetstrokeopacity{0.301449}%
\pgfsetdash{}{0pt}%
\pgfpathmoveto{\pgfqpoint{1.644123in}{2.109667in}}%
\pgfpathcurveto{\pgfqpoint{1.652359in}{2.109667in}}{\pgfqpoint{1.660259in}{2.112939in}}{\pgfqpoint{1.666083in}{2.118763in}}%
\pgfpathcurveto{\pgfqpoint{1.671907in}{2.124587in}}{\pgfqpoint{1.675180in}{2.132487in}}{\pgfqpoint{1.675180in}{2.140724in}}%
\pgfpathcurveto{\pgfqpoint{1.675180in}{2.148960in}}{\pgfqpoint{1.671907in}{2.156860in}}{\pgfqpoint{1.666083in}{2.162684in}}%
\pgfpathcurveto{\pgfqpoint{1.660259in}{2.168508in}}{\pgfqpoint{1.652359in}{2.171780in}}{\pgfqpoint{1.644123in}{2.171780in}}%
\pgfpathcurveto{\pgfqpoint{1.635887in}{2.171780in}}{\pgfqpoint{1.627987in}{2.168508in}}{\pgfqpoint{1.622163in}{2.162684in}}%
\pgfpathcurveto{\pgfqpoint{1.616339in}{2.156860in}}{\pgfqpoint{1.613067in}{2.148960in}}{\pgfqpoint{1.613067in}{2.140724in}}%
\pgfpathcurveto{\pgfqpoint{1.613067in}{2.132487in}}{\pgfqpoint{1.616339in}{2.124587in}}{\pgfqpoint{1.622163in}{2.118763in}}%
\pgfpathcurveto{\pgfqpoint{1.627987in}{2.112939in}}{\pgfqpoint{1.635887in}{2.109667in}}{\pgfqpoint{1.644123in}{2.109667in}}%
\pgfpathclose%
\pgfusepath{stroke,fill}%
\end{pgfscope}%
\begin{pgfscope}%
\pgfpathrectangle{\pgfqpoint{0.100000in}{0.212622in}}{\pgfqpoint{3.696000in}{3.696000in}}%
\pgfusepath{clip}%
\pgfsetbuttcap%
\pgfsetroundjoin%
\definecolor{currentfill}{rgb}{0.121569,0.466667,0.705882}%
\pgfsetfillcolor{currentfill}%
\pgfsetfillopacity{0.301449}%
\pgfsetlinewidth{1.003750pt}%
\definecolor{currentstroke}{rgb}{0.121569,0.466667,0.705882}%
\pgfsetstrokecolor{currentstroke}%
\pgfsetstrokeopacity{0.301449}%
\pgfsetdash{}{0pt}%
\pgfpathmoveto{\pgfqpoint{1.644123in}{2.109667in}}%
\pgfpathcurveto{\pgfqpoint{1.652359in}{2.109667in}}{\pgfqpoint{1.660259in}{2.112939in}}{\pgfqpoint{1.666083in}{2.118763in}}%
\pgfpathcurveto{\pgfqpoint{1.671907in}{2.124587in}}{\pgfqpoint{1.675180in}{2.132487in}}{\pgfqpoint{1.675180in}{2.140724in}}%
\pgfpathcurveto{\pgfqpoint{1.675180in}{2.148960in}}{\pgfqpoint{1.671907in}{2.156860in}}{\pgfqpoint{1.666083in}{2.162684in}}%
\pgfpathcurveto{\pgfqpoint{1.660259in}{2.168508in}}{\pgfqpoint{1.652359in}{2.171780in}}{\pgfqpoint{1.644123in}{2.171780in}}%
\pgfpathcurveto{\pgfqpoint{1.635887in}{2.171780in}}{\pgfqpoint{1.627987in}{2.168508in}}{\pgfqpoint{1.622163in}{2.162684in}}%
\pgfpathcurveto{\pgfqpoint{1.616339in}{2.156860in}}{\pgfqpoint{1.613067in}{2.148960in}}{\pgfqpoint{1.613067in}{2.140724in}}%
\pgfpathcurveto{\pgfqpoint{1.613067in}{2.132487in}}{\pgfqpoint{1.616339in}{2.124587in}}{\pgfqpoint{1.622163in}{2.118763in}}%
\pgfpathcurveto{\pgfqpoint{1.627987in}{2.112939in}}{\pgfqpoint{1.635887in}{2.109667in}}{\pgfqpoint{1.644123in}{2.109667in}}%
\pgfpathclose%
\pgfusepath{stroke,fill}%
\end{pgfscope}%
\begin{pgfscope}%
\pgfpathrectangle{\pgfqpoint{0.100000in}{0.212622in}}{\pgfqpoint{3.696000in}{3.696000in}}%
\pgfusepath{clip}%
\pgfsetbuttcap%
\pgfsetroundjoin%
\definecolor{currentfill}{rgb}{0.121569,0.466667,0.705882}%
\pgfsetfillcolor{currentfill}%
\pgfsetfillopacity{0.301449}%
\pgfsetlinewidth{1.003750pt}%
\definecolor{currentstroke}{rgb}{0.121569,0.466667,0.705882}%
\pgfsetstrokecolor{currentstroke}%
\pgfsetstrokeopacity{0.301449}%
\pgfsetdash{}{0pt}%
\pgfpathmoveto{\pgfqpoint{1.644123in}{2.109667in}}%
\pgfpathcurveto{\pgfqpoint{1.652359in}{2.109667in}}{\pgfqpoint{1.660259in}{2.112939in}}{\pgfqpoint{1.666083in}{2.118763in}}%
\pgfpathcurveto{\pgfqpoint{1.671907in}{2.124587in}}{\pgfqpoint{1.675180in}{2.132487in}}{\pgfqpoint{1.675180in}{2.140724in}}%
\pgfpathcurveto{\pgfqpoint{1.675180in}{2.148960in}}{\pgfqpoint{1.671907in}{2.156860in}}{\pgfqpoint{1.666083in}{2.162684in}}%
\pgfpathcurveto{\pgfqpoint{1.660259in}{2.168508in}}{\pgfqpoint{1.652359in}{2.171780in}}{\pgfqpoint{1.644123in}{2.171780in}}%
\pgfpathcurveto{\pgfqpoint{1.635887in}{2.171780in}}{\pgfqpoint{1.627987in}{2.168508in}}{\pgfqpoint{1.622163in}{2.162684in}}%
\pgfpathcurveto{\pgfqpoint{1.616339in}{2.156860in}}{\pgfqpoint{1.613067in}{2.148960in}}{\pgfqpoint{1.613067in}{2.140724in}}%
\pgfpathcurveto{\pgfqpoint{1.613067in}{2.132487in}}{\pgfqpoint{1.616339in}{2.124587in}}{\pgfqpoint{1.622163in}{2.118763in}}%
\pgfpathcurveto{\pgfqpoint{1.627987in}{2.112939in}}{\pgfqpoint{1.635887in}{2.109667in}}{\pgfqpoint{1.644123in}{2.109667in}}%
\pgfpathclose%
\pgfusepath{stroke,fill}%
\end{pgfscope}%
\begin{pgfscope}%
\pgfpathrectangle{\pgfqpoint{0.100000in}{0.212622in}}{\pgfqpoint{3.696000in}{3.696000in}}%
\pgfusepath{clip}%
\pgfsetbuttcap%
\pgfsetroundjoin%
\definecolor{currentfill}{rgb}{0.121569,0.466667,0.705882}%
\pgfsetfillcolor{currentfill}%
\pgfsetfillopacity{0.301449}%
\pgfsetlinewidth{1.003750pt}%
\definecolor{currentstroke}{rgb}{0.121569,0.466667,0.705882}%
\pgfsetstrokecolor{currentstroke}%
\pgfsetstrokeopacity{0.301449}%
\pgfsetdash{}{0pt}%
\pgfpathmoveto{\pgfqpoint{1.644123in}{2.109667in}}%
\pgfpathcurveto{\pgfqpoint{1.652359in}{2.109667in}}{\pgfqpoint{1.660259in}{2.112939in}}{\pgfqpoint{1.666083in}{2.118763in}}%
\pgfpathcurveto{\pgfqpoint{1.671907in}{2.124587in}}{\pgfqpoint{1.675180in}{2.132487in}}{\pgfqpoint{1.675180in}{2.140724in}}%
\pgfpathcurveto{\pgfqpoint{1.675180in}{2.148960in}}{\pgfqpoint{1.671907in}{2.156860in}}{\pgfqpoint{1.666083in}{2.162684in}}%
\pgfpathcurveto{\pgfqpoint{1.660259in}{2.168508in}}{\pgfqpoint{1.652359in}{2.171780in}}{\pgfqpoint{1.644123in}{2.171780in}}%
\pgfpathcurveto{\pgfqpoint{1.635887in}{2.171780in}}{\pgfqpoint{1.627987in}{2.168508in}}{\pgfqpoint{1.622163in}{2.162684in}}%
\pgfpathcurveto{\pgfqpoint{1.616339in}{2.156860in}}{\pgfqpoint{1.613067in}{2.148960in}}{\pgfqpoint{1.613067in}{2.140724in}}%
\pgfpathcurveto{\pgfqpoint{1.613067in}{2.132487in}}{\pgfqpoint{1.616339in}{2.124587in}}{\pgfqpoint{1.622163in}{2.118763in}}%
\pgfpathcurveto{\pgfqpoint{1.627987in}{2.112939in}}{\pgfqpoint{1.635887in}{2.109667in}}{\pgfqpoint{1.644123in}{2.109667in}}%
\pgfpathclose%
\pgfusepath{stroke,fill}%
\end{pgfscope}%
\begin{pgfscope}%
\pgfpathrectangle{\pgfqpoint{0.100000in}{0.212622in}}{\pgfqpoint{3.696000in}{3.696000in}}%
\pgfusepath{clip}%
\pgfsetbuttcap%
\pgfsetroundjoin%
\definecolor{currentfill}{rgb}{0.121569,0.466667,0.705882}%
\pgfsetfillcolor{currentfill}%
\pgfsetfillopacity{0.301449}%
\pgfsetlinewidth{1.003750pt}%
\definecolor{currentstroke}{rgb}{0.121569,0.466667,0.705882}%
\pgfsetstrokecolor{currentstroke}%
\pgfsetstrokeopacity{0.301449}%
\pgfsetdash{}{0pt}%
\pgfpathmoveto{\pgfqpoint{1.644123in}{2.109667in}}%
\pgfpathcurveto{\pgfqpoint{1.652359in}{2.109667in}}{\pgfqpoint{1.660259in}{2.112939in}}{\pgfqpoint{1.666083in}{2.118763in}}%
\pgfpathcurveto{\pgfqpoint{1.671907in}{2.124587in}}{\pgfqpoint{1.675180in}{2.132487in}}{\pgfqpoint{1.675180in}{2.140724in}}%
\pgfpathcurveto{\pgfqpoint{1.675180in}{2.148960in}}{\pgfqpoint{1.671907in}{2.156860in}}{\pgfqpoint{1.666083in}{2.162684in}}%
\pgfpathcurveto{\pgfqpoint{1.660259in}{2.168508in}}{\pgfqpoint{1.652359in}{2.171780in}}{\pgfqpoint{1.644123in}{2.171780in}}%
\pgfpathcurveto{\pgfqpoint{1.635887in}{2.171780in}}{\pgfqpoint{1.627987in}{2.168508in}}{\pgfqpoint{1.622163in}{2.162684in}}%
\pgfpathcurveto{\pgfqpoint{1.616339in}{2.156860in}}{\pgfqpoint{1.613067in}{2.148960in}}{\pgfqpoint{1.613067in}{2.140724in}}%
\pgfpathcurveto{\pgfqpoint{1.613067in}{2.132487in}}{\pgfqpoint{1.616339in}{2.124587in}}{\pgfqpoint{1.622163in}{2.118763in}}%
\pgfpathcurveto{\pgfqpoint{1.627987in}{2.112939in}}{\pgfqpoint{1.635887in}{2.109667in}}{\pgfqpoint{1.644123in}{2.109667in}}%
\pgfpathclose%
\pgfusepath{stroke,fill}%
\end{pgfscope}%
\begin{pgfscope}%
\pgfpathrectangle{\pgfqpoint{0.100000in}{0.212622in}}{\pgfqpoint{3.696000in}{3.696000in}}%
\pgfusepath{clip}%
\pgfsetbuttcap%
\pgfsetroundjoin%
\definecolor{currentfill}{rgb}{0.121569,0.466667,0.705882}%
\pgfsetfillcolor{currentfill}%
\pgfsetfillopacity{0.301449}%
\pgfsetlinewidth{1.003750pt}%
\definecolor{currentstroke}{rgb}{0.121569,0.466667,0.705882}%
\pgfsetstrokecolor{currentstroke}%
\pgfsetstrokeopacity{0.301449}%
\pgfsetdash{}{0pt}%
\pgfpathmoveto{\pgfqpoint{1.644123in}{2.109667in}}%
\pgfpathcurveto{\pgfqpoint{1.652359in}{2.109667in}}{\pgfqpoint{1.660259in}{2.112939in}}{\pgfqpoint{1.666083in}{2.118763in}}%
\pgfpathcurveto{\pgfqpoint{1.671907in}{2.124587in}}{\pgfqpoint{1.675180in}{2.132487in}}{\pgfqpoint{1.675180in}{2.140724in}}%
\pgfpathcurveto{\pgfqpoint{1.675180in}{2.148960in}}{\pgfqpoint{1.671907in}{2.156860in}}{\pgfqpoint{1.666083in}{2.162684in}}%
\pgfpathcurveto{\pgfqpoint{1.660259in}{2.168508in}}{\pgfqpoint{1.652359in}{2.171780in}}{\pgfqpoint{1.644123in}{2.171780in}}%
\pgfpathcurveto{\pgfqpoint{1.635887in}{2.171780in}}{\pgfqpoint{1.627987in}{2.168508in}}{\pgfqpoint{1.622163in}{2.162684in}}%
\pgfpathcurveto{\pgfqpoint{1.616339in}{2.156860in}}{\pgfqpoint{1.613067in}{2.148960in}}{\pgfqpoint{1.613067in}{2.140724in}}%
\pgfpathcurveto{\pgfqpoint{1.613067in}{2.132487in}}{\pgfqpoint{1.616339in}{2.124587in}}{\pgfqpoint{1.622163in}{2.118763in}}%
\pgfpathcurveto{\pgfqpoint{1.627987in}{2.112939in}}{\pgfqpoint{1.635887in}{2.109667in}}{\pgfqpoint{1.644123in}{2.109667in}}%
\pgfpathclose%
\pgfusepath{stroke,fill}%
\end{pgfscope}%
\begin{pgfscope}%
\pgfpathrectangle{\pgfqpoint{0.100000in}{0.212622in}}{\pgfqpoint{3.696000in}{3.696000in}}%
\pgfusepath{clip}%
\pgfsetbuttcap%
\pgfsetroundjoin%
\definecolor{currentfill}{rgb}{0.121569,0.466667,0.705882}%
\pgfsetfillcolor{currentfill}%
\pgfsetfillopacity{0.301449}%
\pgfsetlinewidth{1.003750pt}%
\definecolor{currentstroke}{rgb}{0.121569,0.466667,0.705882}%
\pgfsetstrokecolor{currentstroke}%
\pgfsetstrokeopacity{0.301449}%
\pgfsetdash{}{0pt}%
\pgfpathmoveto{\pgfqpoint{1.644123in}{2.109667in}}%
\pgfpathcurveto{\pgfqpoint{1.652359in}{2.109667in}}{\pgfqpoint{1.660259in}{2.112939in}}{\pgfqpoint{1.666083in}{2.118763in}}%
\pgfpathcurveto{\pgfqpoint{1.671907in}{2.124587in}}{\pgfqpoint{1.675180in}{2.132487in}}{\pgfqpoint{1.675180in}{2.140724in}}%
\pgfpathcurveto{\pgfqpoint{1.675180in}{2.148960in}}{\pgfqpoint{1.671907in}{2.156860in}}{\pgfqpoint{1.666083in}{2.162684in}}%
\pgfpathcurveto{\pgfqpoint{1.660259in}{2.168508in}}{\pgfqpoint{1.652359in}{2.171780in}}{\pgfqpoint{1.644123in}{2.171780in}}%
\pgfpathcurveto{\pgfqpoint{1.635887in}{2.171780in}}{\pgfqpoint{1.627987in}{2.168508in}}{\pgfqpoint{1.622163in}{2.162684in}}%
\pgfpathcurveto{\pgfqpoint{1.616339in}{2.156860in}}{\pgfqpoint{1.613067in}{2.148960in}}{\pgfqpoint{1.613067in}{2.140724in}}%
\pgfpathcurveto{\pgfqpoint{1.613067in}{2.132487in}}{\pgfqpoint{1.616339in}{2.124587in}}{\pgfqpoint{1.622163in}{2.118763in}}%
\pgfpathcurveto{\pgfqpoint{1.627987in}{2.112939in}}{\pgfqpoint{1.635887in}{2.109667in}}{\pgfqpoint{1.644123in}{2.109667in}}%
\pgfpathclose%
\pgfusepath{stroke,fill}%
\end{pgfscope}%
\begin{pgfscope}%
\pgfpathrectangle{\pgfqpoint{0.100000in}{0.212622in}}{\pgfqpoint{3.696000in}{3.696000in}}%
\pgfusepath{clip}%
\pgfsetbuttcap%
\pgfsetroundjoin%
\definecolor{currentfill}{rgb}{0.121569,0.466667,0.705882}%
\pgfsetfillcolor{currentfill}%
\pgfsetfillopacity{0.301449}%
\pgfsetlinewidth{1.003750pt}%
\definecolor{currentstroke}{rgb}{0.121569,0.466667,0.705882}%
\pgfsetstrokecolor{currentstroke}%
\pgfsetstrokeopacity{0.301449}%
\pgfsetdash{}{0pt}%
\pgfpathmoveto{\pgfqpoint{1.644123in}{2.109667in}}%
\pgfpathcurveto{\pgfqpoint{1.652359in}{2.109667in}}{\pgfqpoint{1.660259in}{2.112939in}}{\pgfqpoint{1.666083in}{2.118763in}}%
\pgfpathcurveto{\pgfqpoint{1.671907in}{2.124587in}}{\pgfqpoint{1.675180in}{2.132487in}}{\pgfqpoint{1.675180in}{2.140724in}}%
\pgfpathcurveto{\pgfqpoint{1.675180in}{2.148960in}}{\pgfqpoint{1.671907in}{2.156860in}}{\pgfqpoint{1.666083in}{2.162684in}}%
\pgfpathcurveto{\pgfqpoint{1.660259in}{2.168508in}}{\pgfqpoint{1.652359in}{2.171780in}}{\pgfqpoint{1.644123in}{2.171780in}}%
\pgfpathcurveto{\pgfqpoint{1.635887in}{2.171780in}}{\pgfqpoint{1.627987in}{2.168508in}}{\pgfqpoint{1.622163in}{2.162684in}}%
\pgfpathcurveto{\pgfqpoint{1.616339in}{2.156860in}}{\pgfqpoint{1.613067in}{2.148960in}}{\pgfqpoint{1.613067in}{2.140724in}}%
\pgfpathcurveto{\pgfqpoint{1.613067in}{2.132487in}}{\pgfqpoint{1.616339in}{2.124587in}}{\pgfqpoint{1.622163in}{2.118763in}}%
\pgfpathcurveto{\pgfqpoint{1.627987in}{2.112939in}}{\pgfqpoint{1.635887in}{2.109667in}}{\pgfqpoint{1.644123in}{2.109667in}}%
\pgfpathclose%
\pgfusepath{stroke,fill}%
\end{pgfscope}%
\begin{pgfscope}%
\pgfpathrectangle{\pgfqpoint{0.100000in}{0.212622in}}{\pgfqpoint{3.696000in}{3.696000in}}%
\pgfusepath{clip}%
\pgfsetbuttcap%
\pgfsetroundjoin%
\definecolor{currentfill}{rgb}{0.121569,0.466667,0.705882}%
\pgfsetfillcolor{currentfill}%
\pgfsetfillopacity{0.301449}%
\pgfsetlinewidth{1.003750pt}%
\definecolor{currentstroke}{rgb}{0.121569,0.466667,0.705882}%
\pgfsetstrokecolor{currentstroke}%
\pgfsetstrokeopacity{0.301449}%
\pgfsetdash{}{0pt}%
\pgfpathmoveto{\pgfqpoint{1.644123in}{2.109667in}}%
\pgfpathcurveto{\pgfqpoint{1.652359in}{2.109667in}}{\pgfqpoint{1.660259in}{2.112939in}}{\pgfqpoint{1.666083in}{2.118763in}}%
\pgfpathcurveto{\pgfqpoint{1.671907in}{2.124587in}}{\pgfqpoint{1.675180in}{2.132487in}}{\pgfqpoint{1.675180in}{2.140724in}}%
\pgfpathcurveto{\pgfqpoint{1.675180in}{2.148960in}}{\pgfqpoint{1.671907in}{2.156860in}}{\pgfqpoint{1.666083in}{2.162684in}}%
\pgfpathcurveto{\pgfqpoint{1.660259in}{2.168508in}}{\pgfqpoint{1.652359in}{2.171780in}}{\pgfqpoint{1.644123in}{2.171780in}}%
\pgfpathcurveto{\pgfqpoint{1.635887in}{2.171780in}}{\pgfqpoint{1.627987in}{2.168508in}}{\pgfqpoint{1.622163in}{2.162684in}}%
\pgfpathcurveto{\pgfqpoint{1.616339in}{2.156860in}}{\pgfqpoint{1.613067in}{2.148960in}}{\pgfqpoint{1.613067in}{2.140724in}}%
\pgfpathcurveto{\pgfqpoint{1.613067in}{2.132487in}}{\pgfqpoint{1.616339in}{2.124587in}}{\pgfqpoint{1.622163in}{2.118763in}}%
\pgfpathcurveto{\pgfqpoint{1.627987in}{2.112939in}}{\pgfqpoint{1.635887in}{2.109667in}}{\pgfqpoint{1.644123in}{2.109667in}}%
\pgfpathclose%
\pgfusepath{stroke,fill}%
\end{pgfscope}%
\begin{pgfscope}%
\pgfpathrectangle{\pgfqpoint{0.100000in}{0.212622in}}{\pgfqpoint{3.696000in}{3.696000in}}%
\pgfusepath{clip}%
\pgfsetbuttcap%
\pgfsetroundjoin%
\definecolor{currentfill}{rgb}{0.121569,0.466667,0.705882}%
\pgfsetfillcolor{currentfill}%
\pgfsetfillopacity{0.301449}%
\pgfsetlinewidth{1.003750pt}%
\definecolor{currentstroke}{rgb}{0.121569,0.466667,0.705882}%
\pgfsetstrokecolor{currentstroke}%
\pgfsetstrokeopacity{0.301449}%
\pgfsetdash{}{0pt}%
\pgfpathmoveto{\pgfqpoint{1.644123in}{2.109667in}}%
\pgfpathcurveto{\pgfqpoint{1.652359in}{2.109667in}}{\pgfqpoint{1.660259in}{2.112939in}}{\pgfqpoint{1.666083in}{2.118763in}}%
\pgfpathcurveto{\pgfqpoint{1.671907in}{2.124587in}}{\pgfqpoint{1.675180in}{2.132487in}}{\pgfqpoint{1.675180in}{2.140724in}}%
\pgfpathcurveto{\pgfqpoint{1.675180in}{2.148960in}}{\pgfqpoint{1.671907in}{2.156860in}}{\pgfqpoint{1.666083in}{2.162684in}}%
\pgfpathcurveto{\pgfqpoint{1.660259in}{2.168508in}}{\pgfqpoint{1.652359in}{2.171780in}}{\pgfqpoint{1.644123in}{2.171780in}}%
\pgfpathcurveto{\pgfqpoint{1.635887in}{2.171780in}}{\pgfqpoint{1.627987in}{2.168508in}}{\pgfqpoint{1.622163in}{2.162684in}}%
\pgfpathcurveto{\pgfqpoint{1.616339in}{2.156860in}}{\pgfqpoint{1.613067in}{2.148960in}}{\pgfqpoint{1.613067in}{2.140724in}}%
\pgfpathcurveto{\pgfqpoint{1.613067in}{2.132487in}}{\pgfqpoint{1.616339in}{2.124587in}}{\pgfqpoint{1.622163in}{2.118763in}}%
\pgfpathcurveto{\pgfqpoint{1.627987in}{2.112939in}}{\pgfqpoint{1.635887in}{2.109667in}}{\pgfqpoint{1.644123in}{2.109667in}}%
\pgfpathclose%
\pgfusepath{stroke,fill}%
\end{pgfscope}%
\begin{pgfscope}%
\pgfpathrectangle{\pgfqpoint{0.100000in}{0.212622in}}{\pgfqpoint{3.696000in}{3.696000in}}%
\pgfusepath{clip}%
\pgfsetbuttcap%
\pgfsetroundjoin%
\definecolor{currentfill}{rgb}{0.121569,0.466667,0.705882}%
\pgfsetfillcolor{currentfill}%
\pgfsetfillopacity{0.301449}%
\pgfsetlinewidth{1.003750pt}%
\definecolor{currentstroke}{rgb}{0.121569,0.466667,0.705882}%
\pgfsetstrokecolor{currentstroke}%
\pgfsetstrokeopacity{0.301449}%
\pgfsetdash{}{0pt}%
\pgfpathmoveto{\pgfqpoint{1.644123in}{2.109667in}}%
\pgfpathcurveto{\pgfqpoint{1.652359in}{2.109667in}}{\pgfqpoint{1.660259in}{2.112939in}}{\pgfqpoint{1.666083in}{2.118763in}}%
\pgfpathcurveto{\pgfqpoint{1.671907in}{2.124587in}}{\pgfqpoint{1.675180in}{2.132487in}}{\pgfqpoint{1.675180in}{2.140724in}}%
\pgfpathcurveto{\pgfqpoint{1.675180in}{2.148960in}}{\pgfqpoint{1.671907in}{2.156860in}}{\pgfqpoint{1.666083in}{2.162684in}}%
\pgfpathcurveto{\pgfqpoint{1.660259in}{2.168508in}}{\pgfqpoint{1.652359in}{2.171780in}}{\pgfqpoint{1.644123in}{2.171780in}}%
\pgfpathcurveto{\pgfqpoint{1.635887in}{2.171780in}}{\pgfqpoint{1.627987in}{2.168508in}}{\pgfqpoint{1.622163in}{2.162684in}}%
\pgfpathcurveto{\pgfqpoint{1.616339in}{2.156860in}}{\pgfqpoint{1.613067in}{2.148960in}}{\pgfqpoint{1.613067in}{2.140724in}}%
\pgfpathcurveto{\pgfqpoint{1.613067in}{2.132487in}}{\pgfqpoint{1.616339in}{2.124587in}}{\pgfqpoint{1.622163in}{2.118763in}}%
\pgfpathcurveto{\pgfqpoint{1.627987in}{2.112939in}}{\pgfqpoint{1.635887in}{2.109667in}}{\pgfqpoint{1.644123in}{2.109667in}}%
\pgfpathclose%
\pgfusepath{stroke,fill}%
\end{pgfscope}%
\begin{pgfscope}%
\pgfpathrectangle{\pgfqpoint{0.100000in}{0.212622in}}{\pgfqpoint{3.696000in}{3.696000in}}%
\pgfusepath{clip}%
\pgfsetbuttcap%
\pgfsetroundjoin%
\definecolor{currentfill}{rgb}{0.121569,0.466667,0.705882}%
\pgfsetfillcolor{currentfill}%
\pgfsetfillopacity{0.301449}%
\pgfsetlinewidth{1.003750pt}%
\definecolor{currentstroke}{rgb}{0.121569,0.466667,0.705882}%
\pgfsetstrokecolor{currentstroke}%
\pgfsetstrokeopacity{0.301449}%
\pgfsetdash{}{0pt}%
\pgfpathmoveto{\pgfqpoint{1.644123in}{2.109667in}}%
\pgfpathcurveto{\pgfqpoint{1.652359in}{2.109667in}}{\pgfqpoint{1.660259in}{2.112939in}}{\pgfqpoint{1.666083in}{2.118763in}}%
\pgfpathcurveto{\pgfqpoint{1.671907in}{2.124587in}}{\pgfqpoint{1.675180in}{2.132487in}}{\pgfqpoint{1.675180in}{2.140724in}}%
\pgfpathcurveto{\pgfqpoint{1.675180in}{2.148960in}}{\pgfqpoint{1.671907in}{2.156860in}}{\pgfqpoint{1.666083in}{2.162684in}}%
\pgfpathcurveto{\pgfqpoint{1.660259in}{2.168508in}}{\pgfqpoint{1.652359in}{2.171780in}}{\pgfqpoint{1.644123in}{2.171780in}}%
\pgfpathcurveto{\pgfqpoint{1.635887in}{2.171780in}}{\pgfqpoint{1.627987in}{2.168508in}}{\pgfqpoint{1.622163in}{2.162684in}}%
\pgfpathcurveto{\pgfqpoint{1.616339in}{2.156860in}}{\pgfqpoint{1.613067in}{2.148960in}}{\pgfqpoint{1.613067in}{2.140724in}}%
\pgfpathcurveto{\pgfqpoint{1.613067in}{2.132487in}}{\pgfqpoint{1.616339in}{2.124587in}}{\pgfqpoint{1.622163in}{2.118763in}}%
\pgfpathcurveto{\pgfqpoint{1.627987in}{2.112939in}}{\pgfqpoint{1.635887in}{2.109667in}}{\pgfqpoint{1.644123in}{2.109667in}}%
\pgfpathclose%
\pgfusepath{stroke,fill}%
\end{pgfscope}%
\begin{pgfscope}%
\pgfpathrectangle{\pgfqpoint{0.100000in}{0.212622in}}{\pgfqpoint{3.696000in}{3.696000in}}%
\pgfusepath{clip}%
\pgfsetbuttcap%
\pgfsetroundjoin%
\definecolor{currentfill}{rgb}{0.121569,0.466667,0.705882}%
\pgfsetfillcolor{currentfill}%
\pgfsetfillopacity{0.301449}%
\pgfsetlinewidth{1.003750pt}%
\definecolor{currentstroke}{rgb}{0.121569,0.466667,0.705882}%
\pgfsetstrokecolor{currentstroke}%
\pgfsetstrokeopacity{0.301449}%
\pgfsetdash{}{0pt}%
\pgfpathmoveto{\pgfqpoint{1.644123in}{2.109667in}}%
\pgfpathcurveto{\pgfqpoint{1.652359in}{2.109667in}}{\pgfqpoint{1.660259in}{2.112939in}}{\pgfqpoint{1.666083in}{2.118763in}}%
\pgfpathcurveto{\pgfqpoint{1.671907in}{2.124587in}}{\pgfqpoint{1.675180in}{2.132487in}}{\pgfqpoint{1.675180in}{2.140724in}}%
\pgfpathcurveto{\pgfqpoint{1.675180in}{2.148960in}}{\pgfqpoint{1.671907in}{2.156860in}}{\pgfqpoint{1.666083in}{2.162684in}}%
\pgfpathcurveto{\pgfqpoint{1.660259in}{2.168508in}}{\pgfqpoint{1.652359in}{2.171780in}}{\pgfqpoint{1.644123in}{2.171780in}}%
\pgfpathcurveto{\pgfqpoint{1.635887in}{2.171780in}}{\pgfqpoint{1.627987in}{2.168508in}}{\pgfqpoint{1.622163in}{2.162684in}}%
\pgfpathcurveto{\pgfqpoint{1.616339in}{2.156860in}}{\pgfqpoint{1.613067in}{2.148960in}}{\pgfqpoint{1.613067in}{2.140724in}}%
\pgfpathcurveto{\pgfqpoint{1.613067in}{2.132487in}}{\pgfqpoint{1.616339in}{2.124587in}}{\pgfqpoint{1.622163in}{2.118763in}}%
\pgfpathcurveto{\pgfqpoint{1.627987in}{2.112939in}}{\pgfqpoint{1.635887in}{2.109667in}}{\pgfqpoint{1.644123in}{2.109667in}}%
\pgfpathclose%
\pgfusepath{stroke,fill}%
\end{pgfscope}%
\begin{pgfscope}%
\pgfpathrectangle{\pgfqpoint{0.100000in}{0.212622in}}{\pgfqpoint{3.696000in}{3.696000in}}%
\pgfusepath{clip}%
\pgfsetbuttcap%
\pgfsetroundjoin%
\definecolor{currentfill}{rgb}{0.121569,0.466667,0.705882}%
\pgfsetfillcolor{currentfill}%
\pgfsetfillopacity{0.301449}%
\pgfsetlinewidth{1.003750pt}%
\definecolor{currentstroke}{rgb}{0.121569,0.466667,0.705882}%
\pgfsetstrokecolor{currentstroke}%
\pgfsetstrokeopacity{0.301449}%
\pgfsetdash{}{0pt}%
\pgfpathmoveto{\pgfqpoint{1.644123in}{2.109667in}}%
\pgfpathcurveto{\pgfqpoint{1.652359in}{2.109667in}}{\pgfqpoint{1.660259in}{2.112939in}}{\pgfqpoint{1.666083in}{2.118763in}}%
\pgfpathcurveto{\pgfqpoint{1.671907in}{2.124587in}}{\pgfqpoint{1.675180in}{2.132487in}}{\pgfqpoint{1.675180in}{2.140724in}}%
\pgfpathcurveto{\pgfqpoint{1.675180in}{2.148960in}}{\pgfqpoint{1.671907in}{2.156860in}}{\pgfqpoint{1.666083in}{2.162684in}}%
\pgfpathcurveto{\pgfqpoint{1.660259in}{2.168508in}}{\pgfqpoint{1.652359in}{2.171780in}}{\pgfqpoint{1.644123in}{2.171780in}}%
\pgfpathcurveto{\pgfqpoint{1.635887in}{2.171780in}}{\pgfqpoint{1.627987in}{2.168508in}}{\pgfqpoint{1.622163in}{2.162684in}}%
\pgfpathcurveto{\pgfqpoint{1.616339in}{2.156860in}}{\pgfqpoint{1.613067in}{2.148960in}}{\pgfqpoint{1.613067in}{2.140724in}}%
\pgfpathcurveto{\pgfqpoint{1.613067in}{2.132487in}}{\pgfqpoint{1.616339in}{2.124587in}}{\pgfqpoint{1.622163in}{2.118763in}}%
\pgfpathcurveto{\pgfqpoint{1.627987in}{2.112939in}}{\pgfqpoint{1.635887in}{2.109667in}}{\pgfqpoint{1.644123in}{2.109667in}}%
\pgfpathclose%
\pgfusepath{stroke,fill}%
\end{pgfscope}%
\begin{pgfscope}%
\pgfpathrectangle{\pgfqpoint{0.100000in}{0.212622in}}{\pgfqpoint{3.696000in}{3.696000in}}%
\pgfusepath{clip}%
\pgfsetbuttcap%
\pgfsetroundjoin%
\definecolor{currentfill}{rgb}{0.121569,0.466667,0.705882}%
\pgfsetfillcolor{currentfill}%
\pgfsetfillopacity{0.301449}%
\pgfsetlinewidth{1.003750pt}%
\definecolor{currentstroke}{rgb}{0.121569,0.466667,0.705882}%
\pgfsetstrokecolor{currentstroke}%
\pgfsetstrokeopacity{0.301449}%
\pgfsetdash{}{0pt}%
\pgfpathmoveto{\pgfqpoint{1.644123in}{2.109667in}}%
\pgfpathcurveto{\pgfqpoint{1.652359in}{2.109667in}}{\pgfqpoint{1.660259in}{2.112939in}}{\pgfqpoint{1.666083in}{2.118763in}}%
\pgfpathcurveto{\pgfqpoint{1.671907in}{2.124587in}}{\pgfqpoint{1.675180in}{2.132487in}}{\pgfqpoint{1.675180in}{2.140724in}}%
\pgfpathcurveto{\pgfqpoint{1.675180in}{2.148960in}}{\pgfqpoint{1.671907in}{2.156860in}}{\pgfqpoint{1.666083in}{2.162684in}}%
\pgfpathcurveto{\pgfqpoint{1.660259in}{2.168508in}}{\pgfqpoint{1.652359in}{2.171780in}}{\pgfqpoint{1.644123in}{2.171780in}}%
\pgfpathcurveto{\pgfqpoint{1.635887in}{2.171780in}}{\pgfqpoint{1.627987in}{2.168508in}}{\pgfqpoint{1.622163in}{2.162684in}}%
\pgfpathcurveto{\pgfqpoint{1.616339in}{2.156860in}}{\pgfqpoint{1.613067in}{2.148960in}}{\pgfqpoint{1.613067in}{2.140724in}}%
\pgfpathcurveto{\pgfqpoint{1.613067in}{2.132487in}}{\pgfqpoint{1.616339in}{2.124587in}}{\pgfqpoint{1.622163in}{2.118763in}}%
\pgfpathcurveto{\pgfqpoint{1.627987in}{2.112939in}}{\pgfqpoint{1.635887in}{2.109667in}}{\pgfqpoint{1.644123in}{2.109667in}}%
\pgfpathclose%
\pgfusepath{stroke,fill}%
\end{pgfscope}%
\begin{pgfscope}%
\pgfpathrectangle{\pgfqpoint{0.100000in}{0.212622in}}{\pgfqpoint{3.696000in}{3.696000in}}%
\pgfusepath{clip}%
\pgfsetbuttcap%
\pgfsetroundjoin%
\definecolor{currentfill}{rgb}{0.121569,0.466667,0.705882}%
\pgfsetfillcolor{currentfill}%
\pgfsetfillopacity{0.301449}%
\pgfsetlinewidth{1.003750pt}%
\definecolor{currentstroke}{rgb}{0.121569,0.466667,0.705882}%
\pgfsetstrokecolor{currentstroke}%
\pgfsetstrokeopacity{0.301449}%
\pgfsetdash{}{0pt}%
\pgfpathmoveto{\pgfqpoint{1.644123in}{2.109667in}}%
\pgfpathcurveto{\pgfqpoint{1.652359in}{2.109667in}}{\pgfqpoint{1.660259in}{2.112939in}}{\pgfqpoint{1.666083in}{2.118763in}}%
\pgfpathcurveto{\pgfqpoint{1.671907in}{2.124587in}}{\pgfqpoint{1.675180in}{2.132487in}}{\pgfqpoint{1.675180in}{2.140724in}}%
\pgfpathcurveto{\pgfqpoint{1.675180in}{2.148960in}}{\pgfqpoint{1.671907in}{2.156860in}}{\pgfqpoint{1.666083in}{2.162684in}}%
\pgfpathcurveto{\pgfqpoint{1.660259in}{2.168508in}}{\pgfqpoint{1.652359in}{2.171780in}}{\pgfqpoint{1.644123in}{2.171780in}}%
\pgfpathcurveto{\pgfqpoint{1.635887in}{2.171780in}}{\pgfqpoint{1.627987in}{2.168508in}}{\pgfqpoint{1.622163in}{2.162684in}}%
\pgfpathcurveto{\pgfqpoint{1.616339in}{2.156860in}}{\pgfqpoint{1.613067in}{2.148960in}}{\pgfqpoint{1.613067in}{2.140724in}}%
\pgfpathcurveto{\pgfqpoint{1.613067in}{2.132487in}}{\pgfqpoint{1.616339in}{2.124587in}}{\pgfqpoint{1.622163in}{2.118763in}}%
\pgfpathcurveto{\pgfqpoint{1.627987in}{2.112939in}}{\pgfqpoint{1.635887in}{2.109667in}}{\pgfqpoint{1.644123in}{2.109667in}}%
\pgfpathclose%
\pgfusepath{stroke,fill}%
\end{pgfscope}%
\begin{pgfscope}%
\pgfpathrectangle{\pgfqpoint{0.100000in}{0.212622in}}{\pgfqpoint{3.696000in}{3.696000in}}%
\pgfusepath{clip}%
\pgfsetbuttcap%
\pgfsetroundjoin%
\definecolor{currentfill}{rgb}{0.121569,0.466667,0.705882}%
\pgfsetfillcolor{currentfill}%
\pgfsetfillopacity{0.301449}%
\pgfsetlinewidth{1.003750pt}%
\definecolor{currentstroke}{rgb}{0.121569,0.466667,0.705882}%
\pgfsetstrokecolor{currentstroke}%
\pgfsetstrokeopacity{0.301449}%
\pgfsetdash{}{0pt}%
\pgfpathmoveto{\pgfqpoint{1.644123in}{2.109667in}}%
\pgfpathcurveto{\pgfqpoint{1.652359in}{2.109667in}}{\pgfqpoint{1.660259in}{2.112939in}}{\pgfqpoint{1.666083in}{2.118763in}}%
\pgfpathcurveto{\pgfqpoint{1.671907in}{2.124587in}}{\pgfqpoint{1.675180in}{2.132487in}}{\pgfqpoint{1.675180in}{2.140724in}}%
\pgfpathcurveto{\pgfqpoint{1.675180in}{2.148960in}}{\pgfqpoint{1.671907in}{2.156860in}}{\pgfqpoint{1.666083in}{2.162684in}}%
\pgfpathcurveto{\pgfqpoint{1.660259in}{2.168508in}}{\pgfqpoint{1.652359in}{2.171780in}}{\pgfqpoint{1.644123in}{2.171780in}}%
\pgfpathcurveto{\pgfqpoint{1.635887in}{2.171780in}}{\pgfqpoint{1.627987in}{2.168508in}}{\pgfqpoint{1.622163in}{2.162684in}}%
\pgfpathcurveto{\pgfqpoint{1.616339in}{2.156860in}}{\pgfqpoint{1.613067in}{2.148960in}}{\pgfqpoint{1.613067in}{2.140724in}}%
\pgfpathcurveto{\pgfqpoint{1.613067in}{2.132487in}}{\pgfqpoint{1.616339in}{2.124587in}}{\pgfqpoint{1.622163in}{2.118763in}}%
\pgfpathcurveto{\pgfqpoint{1.627987in}{2.112939in}}{\pgfqpoint{1.635887in}{2.109667in}}{\pgfqpoint{1.644123in}{2.109667in}}%
\pgfpathclose%
\pgfusepath{stroke,fill}%
\end{pgfscope}%
\begin{pgfscope}%
\pgfpathrectangle{\pgfqpoint{0.100000in}{0.212622in}}{\pgfqpoint{3.696000in}{3.696000in}}%
\pgfusepath{clip}%
\pgfsetbuttcap%
\pgfsetroundjoin%
\definecolor{currentfill}{rgb}{0.121569,0.466667,0.705882}%
\pgfsetfillcolor{currentfill}%
\pgfsetfillopacity{0.301449}%
\pgfsetlinewidth{1.003750pt}%
\definecolor{currentstroke}{rgb}{0.121569,0.466667,0.705882}%
\pgfsetstrokecolor{currentstroke}%
\pgfsetstrokeopacity{0.301449}%
\pgfsetdash{}{0pt}%
\pgfpathmoveto{\pgfqpoint{1.644123in}{2.109667in}}%
\pgfpathcurveto{\pgfqpoint{1.652359in}{2.109667in}}{\pgfqpoint{1.660259in}{2.112939in}}{\pgfqpoint{1.666083in}{2.118763in}}%
\pgfpathcurveto{\pgfqpoint{1.671907in}{2.124587in}}{\pgfqpoint{1.675180in}{2.132487in}}{\pgfqpoint{1.675180in}{2.140724in}}%
\pgfpathcurveto{\pgfqpoint{1.675180in}{2.148960in}}{\pgfqpoint{1.671907in}{2.156860in}}{\pgfqpoint{1.666083in}{2.162684in}}%
\pgfpathcurveto{\pgfqpoint{1.660259in}{2.168508in}}{\pgfqpoint{1.652359in}{2.171780in}}{\pgfqpoint{1.644123in}{2.171780in}}%
\pgfpathcurveto{\pgfqpoint{1.635887in}{2.171780in}}{\pgfqpoint{1.627987in}{2.168508in}}{\pgfqpoint{1.622163in}{2.162684in}}%
\pgfpathcurveto{\pgfqpoint{1.616339in}{2.156860in}}{\pgfqpoint{1.613067in}{2.148960in}}{\pgfqpoint{1.613067in}{2.140724in}}%
\pgfpathcurveto{\pgfqpoint{1.613067in}{2.132487in}}{\pgfqpoint{1.616339in}{2.124587in}}{\pgfqpoint{1.622163in}{2.118763in}}%
\pgfpathcurveto{\pgfqpoint{1.627987in}{2.112939in}}{\pgfqpoint{1.635887in}{2.109667in}}{\pgfqpoint{1.644123in}{2.109667in}}%
\pgfpathclose%
\pgfusepath{stroke,fill}%
\end{pgfscope}%
\begin{pgfscope}%
\pgfpathrectangle{\pgfqpoint{0.100000in}{0.212622in}}{\pgfqpoint{3.696000in}{3.696000in}}%
\pgfusepath{clip}%
\pgfsetbuttcap%
\pgfsetroundjoin%
\definecolor{currentfill}{rgb}{0.121569,0.466667,0.705882}%
\pgfsetfillcolor{currentfill}%
\pgfsetfillopacity{0.301449}%
\pgfsetlinewidth{1.003750pt}%
\definecolor{currentstroke}{rgb}{0.121569,0.466667,0.705882}%
\pgfsetstrokecolor{currentstroke}%
\pgfsetstrokeopacity{0.301449}%
\pgfsetdash{}{0pt}%
\pgfpathmoveto{\pgfqpoint{1.644123in}{2.109667in}}%
\pgfpathcurveto{\pgfqpoint{1.652359in}{2.109667in}}{\pgfqpoint{1.660259in}{2.112939in}}{\pgfqpoint{1.666083in}{2.118763in}}%
\pgfpathcurveto{\pgfqpoint{1.671907in}{2.124587in}}{\pgfqpoint{1.675180in}{2.132487in}}{\pgfqpoint{1.675180in}{2.140724in}}%
\pgfpathcurveto{\pgfqpoint{1.675180in}{2.148960in}}{\pgfqpoint{1.671907in}{2.156860in}}{\pgfqpoint{1.666083in}{2.162684in}}%
\pgfpathcurveto{\pgfqpoint{1.660259in}{2.168508in}}{\pgfqpoint{1.652359in}{2.171780in}}{\pgfqpoint{1.644123in}{2.171780in}}%
\pgfpathcurveto{\pgfqpoint{1.635887in}{2.171780in}}{\pgfqpoint{1.627987in}{2.168508in}}{\pgfqpoint{1.622163in}{2.162684in}}%
\pgfpathcurveto{\pgfqpoint{1.616339in}{2.156860in}}{\pgfqpoint{1.613067in}{2.148960in}}{\pgfqpoint{1.613067in}{2.140724in}}%
\pgfpathcurveto{\pgfqpoint{1.613067in}{2.132487in}}{\pgfqpoint{1.616339in}{2.124587in}}{\pgfqpoint{1.622163in}{2.118763in}}%
\pgfpathcurveto{\pgfqpoint{1.627987in}{2.112939in}}{\pgfqpoint{1.635887in}{2.109667in}}{\pgfqpoint{1.644123in}{2.109667in}}%
\pgfpathclose%
\pgfusepath{stroke,fill}%
\end{pgfscope}%
\begin{pgfscope}%
\pgfpathrectangle{\pgfqpoint{0.100000in}{0.212622in}}{\pgfqpoint{3.696000in}{3.696000in}}%
\pgfusepath{clip}%
\pgfsetbuttcap%
\pgfsetroundjoin%
\definecolor{currentfill}{rgb}{0.121569,0.466667,0.705882}%
\pgfsetfillcolor{currentfill}%
\pgfsetfillopacity{0.301449}%
\pgfsetlinewidth{1.003750pt}%
\definecolor{currentstroke}{rgb}{0.121569,0.466667,0.705882}%
\pgfsetstrokecolor{currentstroke}%
\pgfsetstrokeopacity{0.301449}%
\pgfsetdash{}{0pt}%
\pgfpathmoveto{\pgfqpoint{1.644123in}{2.109667in}}%
\pgfpathcurveto{\pgfqpoint{1.652359in}{2.109667in}}{\pgfqpoint{1.660259in}{2.112939in}}{\pgfqpoint{1.666083in}{2.118763in}}%
\pgfpathcurveto{\pgfqpoint{1.671907in}{2.124587in}}{\pgfqpoint{1.675180in}{2.132487in}}{\pgfqpoint{1.675180in}{2.140724in}}%
\pgfpathcurveto{\pgfqpoint{1.675180in}{2.148960in}}{\pgfqpoint{1.671907in}{2.156860in}}{\pgfqpoint{1.666083in}{2.162684in}}%
\pgfpathcurveto{\pgfqpoint{1.660259in}{2.168508in}}{\pgfqpoint{1.652359in}{2.171780in}}{\pgfqpoint{1.644123in}{2.171780in}}%
\pgfpathcurveto{\pgfqpoint{1.635887in}{2.171780in}}{\pgfqpoint{1.627987in}{2.168508in}}{\pgfqpoint{1.622163in}{2.162684in}}%
\pgfpathcurveto{\pgfqpoint{1.616339in}{2.156860in}}{\pgfqpoint{1.613067in}{2.148960in}}{\pgfqpoint{1.613067in}{2.140724in}}%
\pgfpathcurveto{\pgfqpoint{1.613067in}{2.132487in}}{\pgfqpoint{1.616339in}{2.124587in}}{\pgfqpoint{1.622163in}{2.118763in}}%
\pgfpathcurveto{\pgfqpoint{1.627987in}{2.112939in}}{\pgfqpoint{1.635887in}{2.109667in}}{\pgfqpoint{1.644123in}{2.109667in}}%
\pgfpathclose%
\pgfusepath{stroke,fill}%
\end{pgfscope}%
\begin{pgfscope}%
\pgfpathrectangle{\pgfqpoint{0.100000in}{0.212622in}}{\pgfqpoint{3.696000in}{3.696000in}}%
\pgfusepath{clip}%
\pgfsetbuttcap%
\pgfsetroundjoin%
\definecolor{currentfill}{rgb}{0.121569,0.466667,0.705882}%
\pgfsetfillcolor{currentfill}%
\pgfsetfillopacity{0.301449}%
\pgfsetlinewidth{1.003750pt}%
\definecolor{currentstroke}{rgb}{0.121569,0.466667,0.705882}%
\pgfsetstrokecolor{currentstroke}%
\pgfsetstrokeopacity{0.301449}%
\pgfsetdash{}{0pt}%
\pgfpathmoveto{\pgfqpoint{1.644123in}{2.109667in}}%
\pgfpathcurveto{\pgfqpoint{1.652359in}{2.109667in}}{\pgfqpoint{1.660259in}{2.112939in}}{\pgfqpoint{1.666083in}{2.118763in}}%
\pgfpathcurveto{\pgfqpoint{1.671907in}{2.124587in}}{\pgfqpoint{1.675180in}{2.132487in}}{\pgfqpoint{1.675180in}{2.140724in}}%
\pgfpathcurveto{\pgfqpoint{1.675180in}{2.148960in}}{\pgfqpoint{1.671907in}{2.156860in}}{\pgfqpoint{1.666083in}{2.162684in}}%
\pgfpathcurveto{\pgfqpoint{1.660259in}{2.168508in}}{\pgfqpoint{1.652359in}{2.171780in}}{\pgfqpoint{1.644123in}{2.171780in}}%
\pgfpathcurveto{\pgfqpoint{1.635887in}{2.171780in}}{\pgfqpoint{1.627987in}{2.168508in}}{\pgfqpoint{1.622163in}{2.162684in}}%
\pgfpathcurveto{\pgfqpoint{1.616339in}{2.156860in}}{\pgfqpoint{1.613067in}{2.148960in}}{\pgfqpoint{1.613067in}{2.140724in}}%
\pgfpathcurveto{\pgfqpoint{1.613067in}{2.132487in}}{\pgfqpoint{1.616339in}{2.124587in}}{\pgfqpoint{1.622163in}{2.118763in}}%
\pgfpathcurveto{\pgfqpoint{1.627987in}{2.112939in}}{\pgfqpoint{1.635887in}{2.109667in}}{\pgfqpoint{1.644123in}{2.109667in}}%
\pgfpathclose%
\pgfusepath{stroke,fill}%
\end{pgfscope}%
\begin{pgfscope}%
\pgfpathrectangle{\pgfqpoint{0.100000in}{0.212622in}}{\pgfqpoint{3.696000in}{3.696000in}}%
\pgfusepath{clip}%
\pgfsetbuttcap%
\pgfsetroundjoin%
\definecolor{currentfill}{rgb}{0.121569,0.466667,0.705882}%
\pgfsetfillcolor{currentfill}%
\pgfsetfillopacity{0.301449}%
\pgfsetlinewidth{1.003750pt}%
\definecolor{currentstroke}{rgb}{0.121569,0.466667,0.705882}%
\pgfsetstrokecolor{currentstroke}%
\pgfsetstrokeopacity{0.301449}%
\pgfsetdash{}{0pt}%
\pgfpathmoveto{\pgfqpoint{1.644123in}{2.109667in}}%
\pgfpathcurveto{\pgfqpoint{1.652359in}{2.109667in}}{\pgfqpoint{1.660259in}{2.112939in}}{\pgfqpoint{1.666083in}{2.118763in}}%
\pgfpathcurveto{\pgfqpoint{1.671907in}{2.124587in}}{\pgfqpoint{1.675180in}{2.132487in}}{\pgfqpoint{1.675180in}{2.140724in}}%
\pgfpathcurveto{\pgfqpoint{1.675180in}{2.148960in}}{\pgfqpoint{1.671907in}{2.156860in}}{\pgfqpoint{1.666083in}{2.162684in}}%
\pgfpathcurveto{\pgfqpoint{1.660259in}{2.168508in}}{\pgfqpoint{1.652359in}{2.171780in}}{\pgfqpoint{1.644123in}{2.171780in}}%
\pgfpathcurveto{\pgfqpoint{1.635887in}{2.171780in}}{\pgfqpoint{1.627987in}{2.168508in}}{\pgfqpoint{1.622163in}{2.162684in}}%
\pgfpathcurveto{\pgfqpoint{1.616339in}{2.156860in}}{\pgfqpoint{1.613067in}{2.148960in}}{\pgfqpoint{1.613067in}{2.140724in}}%
\pgfpathcurveto{\pgfqpoint{1.613067in}{2.132487in}}{\pgfqpoint{1.616339in}{2.124587in}}{\pgfqpoint{1.622163in}{2.118763in}}%
\pgfpathcurveto{\pgfqpoint{1.627987in}{2.112939in}}{\pgfqpoint{1.635887in}{2.109667in}}{\pgfqpoint{1.644123in}{2.109667in}}%
\pgfpathclose%
\pgfusepath{stroke,fill}%
\end{pgfscope}%
\begin{pgfscope}%
\pgfpathrectangle{\pgfqpoint{0.100000in}{0.212622in}}{\pgfqpoint{3.696000in}{3.696000in}}%
\pgfusepath{clip}%
\pgfsetbuttcap%
\pgfsetroundjoin%
\definecolor{currentfill}{rgb}{0.121569,0.466667,0.705882}%
\pgfsetfillcolor{currentfill}%
\pgfsetfillopacity{0.301449}%
\pgfsetlinewidth{1.003750pt}%
\definecolor{currentstroke}{rgb}{0.121569,0.466667,0.705882}%
\pgfsetstrokecolor{currentstroke}%
\pgfsetstrokeopacity{0.301449}%
\pgfsetdash{}{0pt}%
\pgfpathmoveto{\pgfqpoint{1.644123in}{2.109667in}}%
\pgfpathcurveto{\pgfqpoint{1.652359in}{2.109667in}}{\pgfqpoint{1.660259in}{2.112939in}}{\pgfqpoint{1.666083in}{2.118763in}}%
\pgfpathcurveto{\pgfqpoint{1.671907in}{2.124587in}}{\pgfqpoint{1.675180in}{2.132487in}}{\pgfqpoint{1.675180in}{2.140724in}}%
\pgfpathcurveto{\pgfqpoint{1.675180in}{2.148960in}}{\pgfqpoint{1.671907in}{2.156860in}}{\pgfqpoint{1.666083in}{2.162684in}}%
\pgfpathcurveto{\pgfqpoint{1.660259in}{2.168508in}}{\pgfqpoint{1.652359in}{2.171780in}}{\pgfqpoint{1.644123in}{2.171780in}}%
\pgfpathcurveto{\pgfqpoint{1.635887in}{2.171780in}}{\pgfqpoint{1.627987in}{2.168508in}}{\pgfqpoint{1.622163in}{2.162684in}}%
\pgfpathcurveto{\pgfqpoint{1.616339in}{2.156860in}}{\pgfqpoint{1.613067in}{2.148960in}}{\pgfqpoint{1.613067in}{2.140724in}}%
\pgfpathcurveto{\pgfqpoint{1.613067in}{2.132487in}}{\pgfqpoint{1.616339in}{2.124587in}}{\pgfqpoint{1.622163in}{2.118763in}}%
\pgfpathcurveto{\pgfqpoint{1.627987in}{2.112939in}}{\pgfqpoint{1.635887in}{2.109667in}}{\pgfqpoint{1.644123in}{2.109667in}}%
\pgfpathclose%
\pgfusepath{stroke,fill}%
\end{pgfscope}%
\begin{pgfscope}%
\pgfpathrectangle{\pgfqpoint{0.100000in}{0.212622in}}{\pgfqpoint{3.696000in}{3.696000in}}%
\pgfusepath{clip}%
\pgfsetbuttcap%
\pgfsetroundjoin%
\definecolor{currentfill}{rgb}{0.121569,0.466667,0.705882}%
\pgfsetfillcolor{currentfill}%
\pgfsetfillopacity{0.301449}%
\pgfsetlinewidth{1.003750pt}%
\definecolor{currentstroke}{rgb}{0.121569,0.466667,0.705882}%
\pgfsetstrokecolor{currentstroke}%
\pgfsetstrokeopacity{0.301449}%
\pgfsetdash{}{0pt}%
\pgfpathmoveto{\pgfqpoint{1.644123in}{2.109667in}}%
\pgfpathcurveto{\pgfqpoint{1.652359in}{2.109667in}}{\pgfqpoint{1.660259in}{2.112939in}}{\pgfqpoint{1.666083in}{2.118763in}}%
\pgfpathcurveto{\pgfqpoint{1.671907in}{2.124587in}}{\pgfqpoint{1.675180in}{2.132487in}}{\pgfqpoint{1.675180in}{2.140724in}}%
\pgfpathcurveto{\pgfqpoint{1.675180in}{2.148960in}}{\pgfqpoint{1.671907in}{2.156860in}}{\pgfqpoint{1.666083in}{2.162684in}}%
\pgfpathcurveto{\pgfqpoint{1.660259in}{2.168508in}}{\pgfqpoint{1.652359in}{2.171780in}}{\pgfqpoint{1.644123in}{2.171780in}}%
\pgfpathcurveto{\pgfqpoint{1.635887in}{2.171780in}}{\pgfqpoint{1.627987in}{2.168508in}}{\pgfqpoint{1.622163in}{2.162684in}}%
\pgfpathcurveto{\pgfqpoint{1.616339in}{2.156860in}}{\pgfqpoint{1.613067in}{2.148960in}}{\pgfqpoint{1.613067in}{2.140724in}}%
\pgfpathcurveto{\pgfqpoint{1.613067in}{2.132487in}}{\pgfqpoint{1.616339in}{2.124587in}}{\pgfqpoint{1.622163in}{2.118763in}}%
\pgfpathcurveto{\pgfqpoint{1.627987in}{2.112939in}}{\pgfqpoint{1.635887in}{2.109667in}}{\pgfqpoint{1.644123in}{2.109667in}}%
\pgfpathclose%
\pgfusepath{stroke,fill}%
\end{pgfscope}%
\begin{pgfscope}%
\pgfpathrectangle{\pgfqpoint{0.100000in}{0.212622in}}{\pgfqpoint{3.696000in}{3.696000in}}%
\pgfusepath{clip}%
\pgfsetbuttcap%
\pgfsetroundjoin%
\definecolor{currentfill}{rgb}{0.121569,0.466667,0.705882}%
\pgfsetfillcolor{currentfill}%
\pgfsetfillopacity{0.301449}%
\pgfsetlinewidth{1.003750pt}%
\definecolor{currentstroke}{rgb}{0.121569,0.466667,0.705882}%
\pgfsetstrokecolor{currentstroke}%
\pgfsetstrokeopacity{0.301449}%
\pgfsetdash{}{0pt}%
\pgfpathmoveto{\pgfqpoint{1.644123in}{2.109667in}}%
\pgfpathcurveto{\pgfqpoint{1.652359in}{2.109667in}}{\pgfqpoint{1.660259in}{2.112939in}}{\pgfqpoint{1.666083in}{2.118763in}}%
\pgfpathcurveto{\pgfqpoint{1.671907in}{2.124587in}}{\pgfqpoint{1.675180in}{2.132487in}}{\pgfqpoint{1.675180in}{2.140724in}}%
\pgfpathcurveto{\pgfqpoint{1.675180in}{2.148960in}}{\pgfqpoint{1.671907in}{2.156860in}}{\pgfqpoint{1.666083in}{2.162684in}}%
\pgfpathcurveto{\pgfqpoint{1.660259in}{2.168508in}}{\pgfqpoint{1.652359in}{2.171780in}}{\pgfqpoint{1.644123in}{2.171780in}}%
\pgfpathcurveto{\pgfqpoint{1.635887in}{2.171780in}}{\pgfqpoint{1.627987in}{2.168508in}}{\pgfqpoint{1.622163in}{2.162684in}}%
\pgfpathcurveto{\pgfqpoint{1.616339in}{2.156860in}}{\pgfqpoint{1.613067in}{2.148960in}}{\pgfqpoint{1.613067in}{2.140724in}}%
\pgfpathcurveto{\pgfqpoint{1.613067in}{2.132487in}}{\pgfqpoint{1.616339in}{2.124587in}}{\pgfqpoint{1.622163in}{2.118763in}}%
\pgfpathcurveto{\pgfqpoint{1.627987in}{2.112939in}}{\pgfqpoint{1.635887in}{2.109667in}}{\pgfqpoint{1.644123in}{2.109667in}}%
\pgfpathclose%
\pgfusepath{stroke,fill}%
\end{pgfscope}%
\begin{pgfscope}%
\pgfpathrectangle{\pgfqpoint{0.100000in}{0.212622in}}{\pgfqpoint{3.696000in}{3.696000in}}%
\pgfusepath{clip}%
\pgfsetbuttcap%
\pgfsetroundjoin%
\definecolor{currentfill}{rgb}{0.121569,0.466667,0.705882}%
\pgfsetfillcolor{currentfill}%
\pgfsetfillopacity{0.301449}%
\pgfsetlinewidth{1.003750pt}%
\definecolor{currentstroke}{rgb}{0.121569,0.466667,0.705882}%
\pgfsetstrokecolor{currentstroke}%
\pgfsetstrokeopacity{0.301449}%
\pgfsetdash{}{0pt}%
\pgfpathmoveto{\pgfqpoint{1.644123in}{2.109667in}}%
\pgfpathcurveto{\pgfqpoint{1.652359in}{2.109667in}}{\pgfqpoint{1.660259in}{2.112939in}}{\pgfqpoint{1.666083in}{2.118763in}}%
\pgfpathcurveto{\pgfqpoint{1.671907in}{2.124587in}}{\pgfqpoint{1.675180in}{2.132487in}}{\pgfqpoint{1.675180in}{2.140724in}}%
\pgfpathcurveto{\pgfqpoint{1.675180in}{2.148960in}}{\pgfqpoint{1.671907in}{2.156860in}}{\pgfqpoint{1.666083in}{2.162684in}}%
\pgfpathcurveto{\pgfqpoint{1.660259in}{2.168508in}}{\pgfqpoint{1.652359in}{2.171780in}}{\pgfqpoint{1.644123in}{2.171780in}}%
\pgfpathcurveto{\pgfqpoint{1.635887in}{2.171780in}}{\pgfqpoint{1.627987in}{2.168508in}}{\pgfqpoint{1.622163in}{2.162684in}}%
\pgfpathcurveto{\pgfqpoint{1.616339in}{2.156860in}}{\pgfqpoint{1.613067in}{2.148960in}}{\pgfqpoint{1.613067in}{2.140724in}}%
\pgfpathcurveto{\pgfqpoint{1.613067in}{2.132487in}}{\pgfqpoint{1.616339in}{2.124587in}}{\pgfqpoint{1.622163in}{2.118763in}}%
\pgfpathcurveto{\pgfqpoint{1.627987in}{2.112939in}}{\pgfqpoint{1.635887in}{2.109667in}}{\pgfqpoint{1.644123in}{2.109667in}}%
\pgfpathclose%
\pgfusepath{stroke,fill}%
\end{pgfscope}%
\begin{pgfscope}%
\pgfpathrectangle{\pgfqpoint{0.100000in}{0.212622in}}{\pgfqpoint{3.696000in}{3.696000in}}%
\pgfusepath{clip}%
\pgfsetbuttcap%
\pgfsetroundjoin%
\definecolor{currentfill}{rgb}{0.121569,0.466667,0.705882}%
\pgfsetfillcolor{currentfill}%
\pgfsetfillopacity{0.301449}%
\pgfsetlinewidth{1.003750pt}%
\definecolor{currentstroke}{rgb}{0.121569,0.466667,0.705882}%
\pgfsetstrokecolor{currentstroke}%
\pgfsetstrokeopacity{0.301449}%
\pgfsetdash{}{0pt}%
\pgfpathmoveto{\pgfqpoint{1.644123in}{2.109667in}}%
\pgfpathcurveto{\pgfqpoint{1.652359in}{2.109667in}}{\pgfqpoint{1.660259in}{2.112939in}}{\pgfqpoint{1.666083in}{2.118763in}}%
\pgfpathcurveto{\pgfqpoint{1.671907in}{2.124587in}}{\pgfqpoint{1.675180in}{2.132487in}}{\pgfqpoint{1.675180in}{2.140724in}}%
\pgfpathcurveto{\pgfqpoint{1.675180in}{2.148960in}}{\pgfqpoint{1.671907in}{2.156860in}}{\pgfqpoint{1.666083in}{2.162684in}}%
\pgfpathcurveto{\pgfqpoint{1.660259in}{2.168508in}}{\pgfqpoint{1.652359in}{2.171780in}}{\pgfqpoint{1.644123in}{2.171780in}}%
\pgfpathcurveto{\pgfqpoint{1.635887in}{2.171780in}}{\pgfqpoint{1.627987in}{2.168508in}}{\pgfqpoint{1.622163in}{2.162684in}}%
\pgfpathcurveto{\pgfqpoint{1.616339in}{2.156860in}}{\pgfqpoint{1.613067in}{2.148960in}}{\pgfqpoint{1.613067in}{2.140724in}}%
\pgfpathcurveto{\pgfqpoint{1.613067in}{2.132487in}}{\pgfqpoint{1.616339in}{2.124587in}}{\pgfqpoint{1.622163in}{2.118763in}}%
\pgfpathcurveto{\pgfqpoint{1.627987in}{2.112939in}}{\pgfqpoint{1.635887in}{2.109667in}}{\pgfqpoint{1.644123in}{2.109667in}}%
\pgfpathclose%
\pgfusepath{stroke,fill}%
\end{pgfscope}%
\begin{pgfscope}%
\pgfpathrectangle{\pgfqpoint{0.100000in}{0.212622in}}{\pgfqpoint{3.696000in}{3.696000in}}%
\pgfusepath{clip}%
\pgfsetbuttcap%
\pgfsetroundjoin%
\definecolor{currentfill}{rgb}{0.121569,0.466667,0.705882}%
\pgfsetfillcolor{currentfill}%
\pgfsetfillopacity{0.301449}%
\pgfsetlinewidth{1.003750pt}%
\definecolor{currentstroke}{rgb}{0.121569,0.466667,0.705882}%
\pgfsetstrokecolor{currentstroke}%
\pgfsetstrokeopacity{0.301449}%
\pgfsetdash{}{0pt}%
\pgfpathmoveto{\pgfqpoint{1.644123in}{2.109667in}}%
\pgfpathcurveto{\pgfqpoint{1.652359in}{2.109667in}}{\pgfqpoint{1.660259in}{2.112939in}}{\pgfqpoint{1.666083in}{2.118763in}}%
\pgfpathcurveto{\pgfqpoint{1.671907in}{2.124587in}}{\pgfqpoint{1.675180in}{2.132487in}}{\pgfqpoint{1.675180in}{2.140724in}}%
\pgfpathcurveto{\pgfqpoint{1.675180in}{2.148960in}}{\pgfqpoint{1.671907in}{2.156860in}}{\pgfqpoint{1.666083in}{2.162684in}}%
\pgfpathcurveto{\pgfqpoint{1.660259in}{2.168508in}}{\pgfqpoint{1.652359in}{2.171780in}}{\pgfqpoint{1.644123in}{2.171780in}}%
\pgfpathcurveto{\pgfqpoint{1.635887in}{2.171780in}}{\pgfqpoint{1.627987in}{2.168508in}}{\pgfqpoint{1.622163in}{2.162684in}}%
\pgfpathcurveto{\pgfqpoint{1.616339in}{2.156860in}}{\pgfqpoint{1.613067in}{2.148960in}}{\pgfqpoint{1.613067in}{2.140724in}}%
\pgfpathcurveto{\pgfqpoint{1.613067in}{2.132487in}}{\pgfqpoint{1.616339in}{2.124587in}}{\pgfqpoint{1.622163in}{2.118763in}}%
\pgfpathcurveto{\pgfqpoint{1.627987in}{2.112939in}}{\pgfqpoint{1.635887in}{2.109667in}}{\pgfqpoint{1.644123in}{2.109667in}}%
\pgfpathclose%
\pgfusepath{stroke,fill}%
\end{pgfscope}%
\begin{pgfscope}%
\pgfpathrectangle{\pgfqpoint{0.100000in}{0.212622in}}{\pgfqpoint{3.696000in}{3.696000in}}%
\pgfusepath{clip}%
\pgfsetbuttcap%
\pgfsetroundjoin%
\definecolor{currentfill}{rgb}{0.121569,0.466667,0.705882}%
\pgfsetfillcolor{currentfill}%
\pgfsetfillopacity{0.301449}%
\pgfsetlinewidth{1.003750pt}%
\definecolor{currentstroke}{rgb}{0.121569,0.466667,0.705882}%
\pgfsetstrokecolor{currentstroke}%
\pgfsetstrokeopacity{0.301449}%
\pgfsetdash{}{0pt}%
\pgfpathmoveto{\pgfqpoint{1.644123in}{2.109667in}}%
\pgfpathcurveto{\pgfqpoint{1.652359in}{2.109667in}}{\pgfqpoint{1.660259in}{2.112939in}}{\pgfqpoint{1.666083in}{2.118763in}}%
\pgfpathcurveto{\pgfqpoint{1.671907in}{2.124587in}}{\pgfqpoint{1.675180in}{2.132487in}}{\pgfqpoint{1.675180in}{2.140724in}}%
\pgfpathcurveto{\pgfqpoint{1.675180in}{2.148960in}}{\pgfqpoint{1.671907in}{2.156860in}}{\pgfqpoint{1.666083in}{2.162684in}}%
\pgfpathcurveto{\pgfqpoint{1.660259in}{2.168508in}}{\pgfqpoint{1.652359in}{2.171780in}}{\pgfqpoint{1.644123in}{2.171780in}}%
\pgfpathcurveto{\pgfqpoint{1.635887in}{2.171780in}}{\pgfqpoint{1.627987in}{2.168508in}}{\pgfqpoint{1.622163in}{2.162684in}}%
\pgfpathcurveto{\pgfqpoint{1.616339in}{2.156860in}}{\pgfqpoint{1.613067in}{2.148960in}}{\pgfqpoint{1.613067in}{2.140724in}}%
\pgfpathcurveto{\pgfqpoint{1.613067in}{2.132487in}}{\pgfqpoint{1.616339in}{2.124587in}}{\pgfqpoint{1.622163in}{2.118763in}}%
\pgfpathcurveto{\pgfqpoint{1.627987in}{2.112939in}}{\pgfqpoint{1.635887in}{2.109667in}}{\pgfqpoint{1.644123in}{2.109667in}}%
\pgfpathclose%
\pgfusepath{stroke,fill}%
\end{pgfscope}%
\begin{pgfscope}%
\pgfpathrectangle{\pgfqpoint{0.100000in}{0.212622in}}{\pgfqpoint{3.696000in}{3.696000in}}%
\pgfusepath{clip}%
\pgfsetbuttcap%
\pgfsetroundjoin%
\definecolor{currentfill}{rgb}{0.121569,0.466667,0.705882}%
\pgfsetfillcolor{currentfill}%
\pgfsetfillopacity{0.301449}%
\pgfsetlinewidth{1.003750pt}%
\definecolor{currentstroke}{rgb}{0.121569,0.466667,0.705882}%
\pgfsetstrokecolor{currentstroke}%
\pgfsetstrokeopacity{0.301449}%
\pgfsetdash{}{0pt}%
\pgfpathmoveto{\pgfqpoint{1.644123in}{2.109667in}}%
\pgfpathcurveto{\pgfqpoint{1.652359in}{2.109667in}}{\pgfqpoint{1.660259in}{2.112939in}}{\pgfqpoint{1.666083in}{2.118763in}}%
\pgfpathcurveto{\pgfqpoint{1.671907in}{2.124587in}}{\pgfqpoint{1.675180in}{2.132487in}}{\pgfqpoint{1.675180in}{2.140724in}}%
\pgfpathcurveto{\pgfqpoint{1.675180in}{2.148960in}}{\pgfqpoint{1.671907in}{2.156860in}}{\pgfqpoint{1.666083in}{2.162684in}}%
\pgfpathcurveto{\pgfqpoint{1.660259in}{2.168508in}}{\pgfqpoint{1.652359in}{2.171780in}}{\pgfqpoint{1.644123in}{2.171780in}}%
\pgfpathcurveto{\pgfqpoint{1.635887in}{2.171780in}}{\pgfqpoint{1.627987in}{2.168508in}}{\pgfqpoint{1.622163in}{2.162684in}}%
\pgfpathcurveto{\pgfqpoint{1.616339in}{2.156860in}}{\pgfqpoint{1.613067in}{2.148960in}}{\pgfqpoint{1.613067in}{2.140724in}}%
\pgfpathcurveto{\pgfqpoint{1.613067in}{2.132487in}}{\pgfqpoint{1.616339in}{2.124587in}}{\pgfqpoint{1.622163in}{2.118763in}}%
\pgfpathcurveto{\pgfqpoint{1.627987in}{2.112939in}}{\pgfqpoint{1.635887in}{2.109667in}}{\pgfqpoint{1.644123in}{2.109667in}}%
\pgfpathclose%
\pgfusepath{stroke,fill}%
\end{pgfscope}%
\begin{pgfscope}%
\pgfpathrectangle{\pgfqpoint{0.100000in}{0.212622in}}{\pgfqpoint{3.696000in}{3.696000in}}%
\pgfusepath{clip}%
\pgfsetbuttcap%
\pgfsetroundjoin%
\definecolor{currentfill}{rgb}{0.121569,0.466667,0.705882}%
\pgfsetfillcolor{currentfill}%
\pgfsetfillopacity{0.301449}%
\pgfsetlinewidth{1.003750pt}%
\definecolor{currentstroke}{rgb}{0.121569,0.466667,0.705882}%
\pgfsetstrokecolor{currentstroke}%
\pgfsetstrokeopacity{0.301449}%
\pgfsetdash{}{0pt}%
\pgfpathmoveto{\pgfqpoint{1.644123in}{2.109667in}}%
\pgfpathcurveto{\pgfqpoint{1.652359in}{2.109667in}}{\pgfqpoint{1.660259in}{2.112939in}}{\pgfqpoint{1.666083in}{2.118763in}}%
\pgfpathcurveto{\pgfqpoint{1.671907in}{2.124587in}}{\pgfqpoint{1.675180in}{2.132487in}}{\pgfqpoint{1.675180in}{2.140724in}}%
\pgfpathcurveto{\pgfqpoint{1.675180in}{2.148960in}}{\pgfqpoint{1.671907in}{2.156860in}}{\pgfqpoint{1.666083in}{2.162684in}}%
\pgfpathcurveto{\pgfqpoint{1.660259in}{2.168508in}}{\pgfqpoint{1.652359in}{2.171780in}}{\pgfqpoint{1.644123in}{2.171780in}}%
\pgfpathcurveto{\pgfqpoint{1.635887in}{2.171780in}}{\pgfqpoint{1.627987in}{2.168508in}}{\pgfqpoint{1.622163in}{2.162684in}}%
\pgfpathcurveto{\pgfqpoint{1.616339in}{2.156860in}}{\pgfqpoint{1.613067in}{2.148960in}}{\pgfqpoint{1.613067in}{2.140724in}}%
\pgfpathcurveto{\pgfqpoint{1.613067in}{2.132487in}}{\pgfqpoint{1.616339in}{2.124587in}}{\pgfqpoint{1.622163in}{2.118763in}}%
\pgfpathcurveto{\pgfqpoint{1.627987in}{2.112939in}}{\pgfqpoint{1.635887in}{2.109667in}}{\pgfqpoint{1.644123in}{2.109667in}}%
\pgfpathclose%
\pgfusepath{stroke,fill}%
\end{pgfscope}%
\begin{pgfscope}%
\pgfpathrectangle{\pgfqpoint{0.100000in}{0.212622in}}{\pgfqpoint{3.696000in}{3.696000in}}%
\pgfusepath{clip}%
\pgfsetbuttcap%
\pgfsetroundjoin%
\definecolor{currentfill}{rgb}{0.121569,0.466667,0.705882}%
\pgfsetfillcolor{currentfill}%
\pgfsetfillopacity{0.301449}%
\pgfsetlinewidth{1.003750pt}%
\definecolor{currentstroke}{rgb}{0.121569,0.466667,0.705882}%
\pgfsetstrokecolor{currentstroke}%
\pgfsetstrokeopacity{0.301449}%
\pgfsetdash{}{0pt}%
\pgfpathmoveto{\pgfqpoint{1.644123in}{2.109667in}}%
\pgfpathcurveto{\pgfqpoint{1.652359in}{2.109667in}}{\pgfqpoint{1.660259in}{2.112939in}}{\pgfqpoint{1.666083in}{2.118763in}}%
\pgfpathcurveto{\pgfqpoint{1.671907in}{2.124587in}}{\pgfqpoint{1.675180in}{2.132487in}}{\pgfqpoint{1.675180in}{2.140724in}}%
\pgfpathcurveto{\pgfqpoint{1.675180in}{2.148960in}}{\pgfqpoint{1.671907in}{2.156860in}}{\pgfqpoint{1.666083in}{2.162684in}}%
\pgfpathcurveto{\pgfqpoint{1.660259in}{2.168508in}}{\pgfqpoint{1.652359in}{2.171780in}}{\pgfqpoint{1.644123in}{2.171780in}}%
\pgfpathcurveto{\pgfqpoint{1.635887in}{2.171780in}}{\pgfqpoint{1.627987in}{2.168508in}}{\pgfqpoint{1.622163in}{2.162684in}}%
\pgfpathcurveto{\pgfqpoint{1.616339in}{2.156860in}}{\pgfqpoint{1.613067in}{2.148960in}}{\pgfqpoint{1.613067in}{2.140724in}}%
\pgfpathcurveto{\pgfqpoint{1.613067in}{2.132487in}}{\pgfqpoint{1.616339in}{2.124587in}}{\pgfqpoint{1.622163in}{2.118763in}}%
\pgfpathcurveto{\pgfqpoint{1.627987in}{2.112939in}}{\pgfqpoint{1.635887in}{2.109667in}}{\pgfqpoint{1.644123in}{2.109667in}}%
\pgfpathclose%
\pgfusepath{stroke,fill}%
\end{pgfscope}%
\begin{pgfscope}%
\pgfpathrectangle{\pgfqpoint{0.100000in}{0.212622in}}{\pgfqpoint{3.696000in}{3.696000in}}%
\pgfusepath{clip}%
\pgfsetbuttcap%
\pgfsetroundjoin%
\definecolor{currentfill}{rgb}{0.121569,0.466667,0.705882}%
\pgfsetfillcolor{currentfill}%
\pgfsetfillopacity{0.301449}%
\pgfsetlinewidth{1.003750pt}%
\definecolor{currentstroke}{rgb}{0.121569,0.466667,0.705882}%
\pgfsetstrokecolor{currentstroke}%
\pgfsetstrokeopacity{0.301449}%
\pgfsetdash{}{0pt}%
\pgfpathmoveto{\pgfqpoint{1.644123in}{2.109667in}}%
\pgfpathcurveto{\pgfqpoint{1.652359in}{2.109667in}}{\pgfqpoint{1.660259in}{2.112939in}}{\pgfqpoint{1.666083in}{2.118763in}}%
\pgfpathcurveto{\pgfqpoint{1.671907in}{2.124587in}}{\pgfqpoint{1.675180in}{2.132487in}}{\pgfqpoint{1.675180in}{2.140724in}}%
\pgfpathcurveto{\pgfqpoint{1.675180in}{2.148960in}}{\pgfqpoint{1.671907in}{2.156860in}}{\pgfqpoint{1.666083in}{2.162684in}}%
\pgfpathcurveto{\pgfqpoint{1.660259in}{2.168508in}}{\pgfqpoint{1.652359in}{2.171780in}}{\pgfqpoint{1.644123in}{2.171780in}}%
\pgfpathcurveto{\pgfqpoint{1.635887in}{2.171780in}}{\pgfqpoint{1.627987in}{2.168508in}}{\pgfqpoint{1.622163in}{2.162684in}}%
\pgfpathcurveto{\pgfqpoint{1.616339in}{2.156860in}}{\pgfqpoint{1.613067in}{2.148960in}}{\pgfqpoint{1.613067in}{2.140724in}}%
\pgfpathcurveto{\pgfqpoint{1.613067in}{2.132487in}}{\pgfqpoint{1.616339in}{2.124587in}}{\pgfqpoint{1.622163in}{2.118763in}}%
\pgfpathcurveto{\pgfqpoint{1.627987in}{2.112939in}}{\pgfqpoint{1.635887in}{2.109667in}}{\pgfqpoint{1.644123in}{2.109667in}}%
\pgfpathclose%
\pgfusepath{stroke,fill}%
\end{pgfscope}%
\begin{pgfscope}%
\pgfpathrectangle{\pgfqpoint{0.100000in}{0.212622in}}{\pgfqpoint{3.696000in}{3.696000in}}%
\pgfusepath{clip}%
\pgfsetbuttcap%
\pgfsetroundjoin%
\definecolor{currentfill}{rgb}{0.121569,0.466667,0.705882}%
\pgfsetfillcolor{currentfill}%
\pgfsetfillopacity{0.301449}%
\pgfsetlinewidth{1.003750pt}%
\definecolor{currentstroke}{rgb}{0.121569,0.466667,0.705882}%
\pgfsetstrokecolor{currentstroke}%
\pgfsetstrokeopacity{0.301449}%
\pgfsetdash{}{0pt}%
\pgfpathmoveto{\pgfqpoint{1.644123in}{2.109667in}}%
\pgfpathcurveto{\pgfqpoint{1.652359in}{2.109667in}}{\pgfqpoint{1.660259in}{2.112939in}}{\pgfqpoint{1.666083in}{2.118763in}}%
\pgfpathcurveto{\pgfqpoint{1.671907in}{2.124587in}}{\pgfqpoint{1.675180in}{2.132487in}}{\pgfqpoint{1.675180in}{2.140724in}}%
\pgfpathcurveto{\pgfqpoint{1.675180in}{2.148960in}}{\pgfqpoint{1.671907in}{2.156860in}}{\pgfqpoint{1.666083in}{2.162684in}}%
\pgfpathcurveto{\pgfqpoint{1.660259in}{2.168508in}}{\pgfqpoint{1.652359in}{2.171780in}}{\pgfqpoint{1.644123in}{2.171780in}}%
\pgfpathcurveto{\pgfqpoint{1.635887in}{2.171780in}}{\pgfqpoint{1.627987in}{2.168508in}}{\pgfqpoint{1.622163in}{2.162684in}}%
\pgfpathcurveto{\pgfqpoint{1.616339in}{2.156860in}}{\pgfqpoint{1.613067in}{2.148960in}}{\pgfqpoint{1.613067in}{2.140724in}}%
\pgfpathcurveto{\pgfqpoint{1.613067in}{2.132487in}}{\pgfqpoint{1.616339in}{2.124587in}}{\pgfqpoint{1.622163in}{2.118763in}}%
\pgfpathcurveto{\pgfqpoint{1.627987in}{2.112939in}}{\pgfqpoint{1.635887in}{2.109667in}}{\pgfqpoint{1.644123in}{2.109667in}}%
\pgfpathclose%
\pgfusepath{stroke,fill}%
\end{pgfscope}%
\begin{pgfscope}%
\pgfpathrectangle{\pgfqpoint{0.100000in}{0.212622in}}{\pgfqpoint{3.696000in}{3.696000in}}%
\pgfusepath{clip}%
\pgfsetbuttcap%
\pgfsetroundjoin%
\definecolor{currentfill}{rgb}{0.121569,0.466667,0.705882}%
\pgfsetfillcolor{currentfill}%
\pgfsetfillopacity{0.301449}%
\pgfsetlinewidth{1.003750pt}%
\definecolor{currentstroke}{rgb}{0.121569,0.466667,0.705882}%
\pgfsetstrokecolor{currentstroke}%
\pgfsetstrokeopacity{0.301449}%
\pgfsetdash{}{0pt}%
\pgfpathmoveto{\pgfqpoint{1.644123in}{2.109667in}}%
\pgfpathcurveto{\pgfqpoint{1.652359in}{2.109667in}}{\pgfqpoint{1.660259in}{2.112939in}}{\pgfqpoint{1.666083in}{2.118763in}}%
\pgfpathcurveto{\pgfqpoint{1.671907in}{2.124587in}}{\pgfqpoint{1.675180in}{2.132487in}}{\pgfqpoint{1.675180in}{2.140724in}}%
\pgfpathcurveto{\pgfqpoint{1.675180in}{2.148960in}}{\pgfqpoint{1.671907in}{2.156860in}}{\pgfqpoint{1.666083in}{2.162684in}}%
\pgfpathcurveto{\pgfqpoint{1.660259in}{2.168508in}}{\pgfqpoint{1.652359in}{2.171780in}}{\pgfqpoint{1.644123in}{2.171780in}}%
\pgfpathcurveto{\pgfqpoint{1.635887in}{2.171780in}}{\pgfqpoint{1.627987in}{2.168508in}}{\pgfqpoint{1.622163in}{2.162684in}}%
\pgfpathcurveto{\pgfqpoint{1.616339in}{2.156860in}}{\pgfqpoint{1.613067in}{2.148960in}}{\pgfqpoint{1.613067in}{2.140724in}}%
\pgfpathcurveto{\pgfqpoint{1.613067in}{2.132487in}}{\pgfqpoint{1.616339in}{2.124587in}}{\pgfqpoint{1.622163in}{2.118763in}}%
\pgfpathcurveto{\pgfqpoint{1.627987in}{2.112939in}}{\pgfqpoint{1.635887in}{2.109667in}}{\pgfqpoint{1.644123in}{2.109667in}}%
\pgfpathclose%
\pgfusepath{stroke,fill}%
\end{pgfscope}%
\begin{pgfscope}%
\pgfpathrectangle{\pgfqpoint{0.100000in}{0.212622in}}{\pgfqpoint{3.696000in}{3.696000in}}%
\pgfusepath{clip}%
\pgfsetbuttcap%
\pgfsetroundjoin%
\definecolor{currentfill}{rgb}{0.121569,0.466667,0.705882}%
\pgfsetfillcolor{currentfill}%
\pgfsetfillopacity{0.301449}%
\pgfsetlinewidth{1.003750pt}%
\definecolor{currentstroke}{rgb}{0.121569,0.466667,0.705882}%
\pgfsetstrokecolor{currentstroke}%
\pgfsetstrokeopacity{0.301449}%
\pgfsetdash{}{0pt}%
\pgfpathmoveto{\pgfqpoint{1.644123in}{2.109667in}}%
\pgfpathcurveto{\pgfqpoint{1.652359in}{2.109667in}}{\pgfqpoint{1.660259in}{2.112939in}}{\pgfqpoint{1.666083in}{2.118763in}}%
\pgfpathcurveto{\pgfqpoint{1.671907in}{2.124587in}}{\pgfqpoint{1.675180in}{2.132487in}}{\pgfqpoint{1.675180in}{2.140724in}}%
\pgfpathcurveto{\pgfqpoint{1.675180in}{2.148960in}}{\pgfqpoint{1.671907in}{2.156860in}}{\pgfqpoint{1.666083in}{2.162684in}}%
\pgfpathcurveto{\pgfqpoint{1.660259in}{2.168508in}}{\pgfqpoint{1.652359in}{2.171780in}}{\pgfqpoint{1.644123in}{2.171780in}}%
\pgfpathcurveto{\pgfqpoint{1.635887in}{2.171780in}}{\pgfqpoint{1.627987in}{2.168508in}}{\pgfqpoint{1.622163in}{2.162684in}}%
\pgfpathcurveto{\pgfqpoint{1.616339in}{2.156860in}}{\pgfqpoint{1.613067in}{2.148960in}}{\pgfqpoint{1.613067in}{2.140724in}}%
\pgfpathcurveto{\pgfqpoint{1.613067in}{2.132487in}}{\pgfqpoint{1.616339in}{2.124587in}}{\pgfqpoint{1.622163in}{2.118763in}}%
\pgfpathcurveto{\pgfqpoint{1.627987in}{2.112939in}}{\pgfqpoint{1.635887in}{2.109667in}}{\pgfqpoint{1.644123in}{2.109667in}}%
\pgfpathclose%
\pgfusepath{stroke,fill}%
\end{pgfscope}%
\begin{pgfscope}%
\pgfpathrectangle{\pgfqpoint{0.100000in}{0.212622in}}{\pgfqpoint{3.696000in}{3.696000in}}%
\pgfusepath{clip}%
\pgfsetbuttcap%
\pgfsetroundjoin%
\definecolor{currentfill}{rgb}{0.121569,0.466667,0.705882}%
\pgfsetfillcolor{currentfill}%
\pgfsetfillopacity{0.301449}%
\pgfsetlinewidth{1.003750pt}%
\definecolor{currentstroke}{rgb}{0.121569,0.466667,0.705882}%
\pgfsetstrokecolor{currentstroke}%
\pgfsetstrokeopacity{0.301449}%
\pgfsetdash{}{0pt}%
\pgfpathmoveto{\pgfqpoint{1.644123in}{2.109667in}}%
\pgfpathcurveto{\pgfqpoint{1.652359in}{2.109667in}}{\pgfqpoint{1.660259in}{2.112939in}}{\pgfqpoint{1.666083in}{2.118763in}}%
\pgfpathcurveto{\pgfqpoint{1.671907in}{2.124587in}}{\pgfqpoint{1.675180in}{2.132487in}}{\pgfqpoint{1.675180in}{2.140724in}}%
\pgfpathcurveto{\pgfqpoint{1.675180in}{2.148960in}}{\pgfqpoint{1.671907in}{2.156860in}}{\pgfqpoint{1.666083in}{2.162684in}}%
\pgfpathcurveto{\pgfqpoint{1.660259in}{2.168508in}}{\pgfqpoint{1.652359in}{2.171780in}}{\pgfqpoint{1.644123in}{2.171780in}}%
\pgfpathcurveto{\pgfqpoint{1.635887in}{2.171780in}}{\pgfqpoint{1.627987in}{2.168508in}}{\pgfqpoint{1.622163in}{2.162684in}}%
\pgfpathcurveto{\pgfqpoint{1.616339in}{2.156860in}}{\pgfqpoint{1.613067in}{2.148960in}}{\pgfqpoint{1.613067in}{2.140724in}}%
\pgfpathcurveto{\pgfqpoint{1.613067in}{2.132487in}}{\pgfqpoint{1.616339in}{2.124587in}}{\pgfqpoint{1.622163in}{2.118763in}}%
\pgfpathcurveto{\pgfqpoint{1.627987in}{2.112939in}}{\pgfqpoint{1.635887in}{2.109667in}}{\pgfqpoint{1.644123in}{2.109667in}}%
\pgfpathclose%
\pgfusepath{stroke,fill}%
\end{pgfscope}%
\begin{pgfscope}%
\pgfpathrectangle{\pgfqpoint{0.100000in}{0.212622in}}{\pgfqpoint{3.696000in}{3.696000in}}%
\pgfusepath{clip}%
\pgfsetbuttcap%
\pgfsetroundjoin%
\definecolor{currentfill}{rgb}{0.121569,0.466667,0.705882}%
\pgfsetfillcolor{currentfill}%
\pgfsetfillopacity{0.301449}%
\pgfsetlinewidth{1.003750pt}%
\definecolor{currentstroke}{rgb}{0.121569,0.466667,0.705882}%
\pgfsetstrokecolor{currentstroke}%
\pgfsetstrokeopacity{0.301449}%
\pgfsetdash{}{0pt}%
\pgfpathmoveto{\pgfqpoint{1.644123in}{2.109667in}}%
\pgfpathcurveto{\pgfqpoint{1.652359in}{2.109667in}}{\pgfqpoint{1.660259in}{2.112939in}}{\pgfqpoint{1.666083in}{2.118763in}}%
\pgfpathcurveto{\pgfqpoint{1.671907in}{2.124587in}}{\pgfqpoint{1.675179in}{2.132487in}}{\pgfqpoint{1.675179in}{2.140724in}}%
\pgfpathcurveto{\pgfqpoint{1.675179in}{2.148960in}}{\pgfqpoint{1.671907in}{2.156860in}}{\pgfqpoint{1.666083in}{2.162684in}}%
\pgfpathcurveto{\pgfqpoint{1.660259in}{2.168508in}}{\pgfqpoint{1.652359in}{2.171780in}}{\pgfqpoint{1.644123in}{2.171780in}}%
\pgfpathcurveto{\pgfqpoint{1.635887in}{2.171780in}}{\pgfqpoint{1.627987in}{2.168508in}}{\pgfqpoint{1.622163in}{2.162684in}}%
\pgfpathcurveto{\pgfqpoint{1.616339in}{2.156860in}}{\pgfqpoint{1.613066in}{2.148960in}}{\pgfqpoint{1.613066in}{2.140724in}}%
\pgfpathcurveto{\pgfqpoint{1.613066in}{2.132487in}}{\pgfqpoint{1.616339in}{2.124587in}}{\pgfqpoint{1.622163in}{2.118763in}}%
\pgfpathcurveto{\pgfqpoint{1.627987in}{2.112939in}}{\pgfqpoint{1.635887in}{2.109667in}}{\pgfqpoint{1.644123in}{2.109667in}}%
\pgfpathclose%
\pgfusepath{stroke,fill}%
\end{pgfscope}%
\begin{pgfscope}%
\pgfpathrectangle{\pgfqpoint{0.100000in}{0.212622in}}{\pgfqpoint{3.696000in}{3.696000in}}%
\pgfusepath{clip}%
\pgfsetbuttcap%
\pgfsetroundjoin%
\definecolor{currentfill}{rgb}{0.121569,0.466667,0.705882}%
\pgfsetfillcolor{currentfill}%
\pgfsetfillopacity{0.301449}%
\pgfsetlinewidth{1.003750pt}%
\definecolor{currentstroke}{rgb}{0.121569,0.466667,0.705882}%
\pgfsetstrokecolor{currentstroke}%
\pgfsetstrokeopacity{0.301449}%
\pgfsetdash{}{0pt}%
\pgfpathmoveto{\pgfqpoint{1.644123in}{2.109667in}}%
\pgfpathcurveto{\pgfqpoint{1.652359in}{2.109667in}}{\pgfqpoint{1.660259in}{2.112939in}}{\pgfqpoint{1.666083in}{2.118763in}}%
\pgfpathcurveto{\pgfqpoint{1.671907in}{2.124587in}}{\pgfqpoint{1.675180in}{2.132487in}}{\pgfqpoint{1.675180in}{2.140724in}}%
\pgfpathcurveto{\pgfqpoint{1.675180in}{2.148960in}}{\pgfqpoint{1.671907in}{2.156860in}}{\pgfqpoint{1.666083in}{2.162684in}}%
\pgfpathcurveto{\pgfqpoint{1.660259in}{2.168508in}}{\pgfqpoint{1.652359in}{2.171780in}}{\pgfqpoint{1.644123in}{2.171780in}}%
\pgfpathcurveto{\pgfqpoint{1.635887in}{2.171780in}}{\pgfqpoint{1.627987in}{2.168508in}}{\pgfqpoint{1.622163in}{2.162684in}}%
\pgfpathcurveto{\pgfqpoint{1.616339in}{2.156860in}}{\pgfqpoint{1.613067in}{2.148960in}}{\pgfqpoint{1.613067in}{2.140724in}}%
\pgfpathcurveto{\pgfqpoint{1.613067in}{2.132487in}}{\pgfqpoint{1.616339in}{2.124587in}}{\pgfqpoint{1.622163in}{2.118763in}}%
\pgfpathcurveto{\pgfqpoint{1.627987in}{2.112939in}}{\pgfqpoint{1.635887in}{2.109667in}}{\pgfqpoint{1.644123in}{2.109667in}}%
\pgfpathclose%
\pgfusepath{stroke,fill}%
\end{pgfscope}%
\begin{pgfscope}%
\pgfpathrectangle{\pgfqpoint{0.100000in}{0.212622in}}{\pgfqpoint{3.696000in}{3.696000in}}%
\pgfusepath{clip}%
\pgfsetbuttcap%
\pgfsetroundjoin%
\definecolor{currentfill}{rgb}{0.121569,0.466667,0.705882}%
\pgfsetfillcolor{currentfill}%
\pgfsetfillopacity{0.301449}%
\pgfsetlinewidth{1.003750pt}%
\definecolor{currentstroke}{rgb}{0.121569,0.466667,0.705882}%
\pgfsetstrokecolor{currentstroke}%
\pgfsetstrokeopacity{0.301449}%
\pgfsetdash{}{0pt}%
\pgfpathmoveto{\pgfqpoint{1.644123in}{2.109667in}}%
\pgfpathcurveto{\pgfqpoint{1.652359in}{2.109667in}}{\pgfqpoint{1.660259in}{2.112939in}}{\pgfqpoint{1.666083in}{2.118763in}}%
\pgfpathcurveto{\pgfqpoint{1.671907in}{2.124587in}}{\pgfqpoint{1.675180in}{2.132487in}}{\pgfqpoint{1.675180in}{2.140724in}}%
\pgfpathcurveto{\pgfqpoint{1.675180in}{2.148960in}}{\pgfqpoint{1.671907in}{2.156860in}}{\pgfqpoint{1.666083in}{2.162684in}}%
\pgfpathcurveto{\pgfqpoint{1.660259in}{2.168508in}}{\pgfqpoint{1.652359in}{2.171780in}}{\pgfqpoint{1.644123in}{2.171780in}}%
\pgfpathcurveto{\pgfqpoint{1.635887in}{2.171780in}}{\pgfqpoint{1.627987in}{2.168508in}}{\pgfqpoint{1.622163in}{2.162684in}}%
\pgfpathcurveto{\pgfqpoint{1.616339in}{2.156860in}}{\pgfqpoint{1.613067in}{2.148960in}}{\pgfqpoint{1.613067in}{2.140724in}}%
\pgfpathcurveto{\pgfqpoint{1.613067in}{2.132487in}}{\pgfqpoint{1.616339in}{2.124587in}}{\pgfqpoint{1.622163in}{2.118763in}}%
\pgfpathcurveto{\pgfqpoint{1.627987in}{2.112939in}}{\pgfqpoint{1.635887in}{2.109667in}}{\pgfqpoint{1.644123in}{2.109667in}}%
\pgfpathclose%
\pgfusepath{stroke,fill}%
\end{pgfscope}%
\begin{pgfscope}%
\pgfpathrectangle{\pgfqpoint{0.100000in}{0.212622in}}{\pgfqpoint{3.696000in}{3.696000in}}%
\pgfusepath{clip}%
\pgfsetbuttcap%
\pgfsetroundjoin%
\definecolor{currentfill}{rgb}{0.121569,0.466667,0.705882}%
\pgfsetfillcolor{currentfill}%
\pgfsetfillopacity{0.301449}%
\pgfsetlinewidth{1.003750pt}%
\definecolor{currentstroke}{rgb}{0.121569,0.466667,0.705882}%
\pgfsetstrokecolor{currentstroke}%
\pgfsetstrokeopacity{0.301449}%
\pgfsetdash{}{0pt}%
\pgfpathmoveto{\pgfqpoint{1.644123in}{2.109667in}}%
\pgfpathcurveto{\pgfqpoint{1.652359in}{2.109667in}}{\pgfqpoint{1.660259in}{2.112939in}}{\pgfqpoint{1.666083in}{2.118763in}}%
\pgfpathcurveto{\pgfqpoint{1.671907in}{2.124587in}}{\pgfqpoint{1.675180in}{2.132487in}}{\pgfqpoint{1.675180in}{2.140724in}}%
\pgfpathcurveto{\pgfqpoint{1.675180in}{2.148960in}}{\pgfqpoint{1.671907in}{2.156860in}}{\pgfqpoint{1.666083in}{2.162684in}}%
\pgfpathcurveto{\pgfqpoint{1.660259in}{2.168508in}}{\pgfqpoint{1.652359in}{2.171780in}}{\pgfqpoint{1.644123in}{2.171780in}}%
\pgfpathcurveto{\pgfqpoint{1.635887in}{2.171780in}}{\pgfqpoint{1.627987in}{2.168508in}}{\pgfqpoint{1.622163in}{2.162684in}}%
\pgfpathcurveto{\pgfqpoint{1.616339in}{2.156860in}}{\pgfqpoint{1.613067in}{2.148960in}}{\pgfqpoint{1.613067in}{2.140724in}}%
\pgfpathcurveto{\pgfqpoint{1.613067in}{2.132487in}}{\pgfqpoint{1.616339in}{2.124587in}}{\pgfqpoint{1.622163in}{2.118763in}}%
\pgfpathcurveto{\pgfqpoint{1.627987in}{2.112939in}}{\pgfqpoint{1.635887in}{2.109667in}}{\pgfqpoint{1.644123in}{2.109667in}}%
\pgfpathclose%
\pgfusepath{stroke,fill}%
\end{pgfscope}%
\begin{pgfscope}%
\pgfpathrectangle{\pgfqpoint{0.100000in}{0.212622in}}{\pgfqpoint{3.696000in}{3.696000in}}%
\pgfusepath{clip}%
\pgfsetbuttcap%
\pgfsetroundjoin%
\definecolor{currentfill}{rgb}{0.121569,0.466667,0.705882}%
\pgfsetfillcolor{currentfill}%
\pgfsetfillopacity{0.301449}%
\pgfsetlinewidth{1.003750pt}%
\definecolor{currentstroke}{rgb}{0.121569,0.466667,0.705882}%
\pgfsetstrokecolor{currentstroke}%
\pgfsetstrokeopacity{0.301449}%
\pgfsetdash{}{0pt}%
\pgfpathmoveto{\pgfqpoint{1.644123in}{2.109667in}}%
\pgfpathcurveto{\pgfqpoint{1.652359in}{2.109667in}}{\pgfqpoint{1.660259in}{2.112939in}}{\pgfqpoint{1.666083in}{2.118763in}}%
\pgfpathcurveto{\pgfqpoint{1.671907in}{2.124587in}}{\pgfqpoint{1.675180in}{2.132487in}}{\pgfqpoint{1.675180in}{2.140724in}}%
\pgfpathcurveto{\pgfqpoint{1.675180in}{2.148960in}}{\pgfqpoint{1.671907in}{2.156860in}}{\pgfqpoint{1.666083in}{2.162684in}}%
\pgfpathcurveto{\pgfqpoint{1.660259in}{2.168508in}}{\pgfqpoint{1.652359in}{2.171780in}}{\pgfqpoint{1.644123in}{2.171780in}}%
\pgfpathcurveto{\pgfqpoint{1.635887in}{2.171780in}}{\pgfqpoint{1.627987in}{2.168508in}}{\pgfqpoint{1.622163in}{2.162684in}}%
\pgfpathcurveto{\pgfqpoint{1.616339in}{2.156860in}}{\pgfqpoint{1.613067in}{2.148960in}}{\pgfqpoint{1.613067in}{2.140724in}}%
\pgfpathcurveto{\pgfqpoint{1.613067in}{2.132487in}}{\pgfqpoint{1.616339in}{2.124587in}}{\pgfqpoint{1.622163in}{2.118763in}}%
\pgfpathcurveto{\pgfqpoint{1.627987in}{2.112939in}}{\pgfqpoint{1.635887in}{2.109667in}}{\pgfqpoint{1.644123in}{2.109667in}}%
\pgfpathclose%
\pgfusepath{stroke,fill}%
\end{pgfscope}%
\begin{pgfscope}%
\pgfpathrectangle{\pgfqpoint{0.100000in}{0.212622in}}{\pgfqpoint{3.696000in}{3.696000in}}%
\pgfusepath{clip}%
\pgfsetbuttcap%
\pgfsetroundjoin%
\definecolor{currentfill}{rgb}{0.121569,0.466667,0.705882}%
\pgfsetfillcolor{currentfill}%
\pgfsetfillopacity{0.301449}%
\pgfsetlinewidth{1.003750pt}%
\definecolor{currentstroke}{rgb}{0.121569,0.466667,0.705882}%
\pgfsetstrokecolor{currentstroke}%
\pgfsetstrokeopacity{0.301449}%
\pgfsetdash{}{0pt}%
\pgfpathmoveto{\pgfqpoint{1.644123in}{2.109667in}}%
\pgfpathcurveto{\pgfqpoint{1.652359in}{2.109667in}}{\pgfqpoint{1.660259in}{2.112939in}}{\pgfqpoint{1.666083in}{2.118763in}}%
\pgfpathcurveto{\pgfqpoint{1.671907in}{2.124587in}}{\pgfqpoint{1.675179in}{2.132487in}}{\pgfqpoint{1.675179in}{2.140724in}}%
\pgfpathcurveto{\pgfqpoint{1.675179in}{2.148960in}}{\pgfqpoint{1.671907in}{2.156860in}}{\pgfqpoint{1.666083in}{2.162684in}}%
\pgfpathcurveto{\pgfqpoint{1.660259in}{2.168508in}}{\pgfqpoint{1.652359in}{2.171780in}}{\pgfqpoint{1.644123in}{2.171780in}}%
\pgfpathcurveto{\pgfqpoint{1.635887in}{2.171780in}}{\pgfqpoint{1.627987in}{2.168508in}}{\pgfqpoint{1.622163in}{2.162684in}}%
\pgfpathcurveto{\pgfqpoint{1.616339in}{2.156860in}}{\pgfqpoint{1.613066in}{2.148960in}}{\pgfqpoint{1.613066in}{2.140724in}}%
\pgfpathcurveto{\pgfqpoint{1.613066in}{2.132487in}}{\pgfqpoint{1.616339in}{2.124587in}}{\pgfqpoint{1.622163in}{2.118763in}}%
\pgfpathcurveto{\pgfqpoint{1.627987in}{2.112939in}}{\pgfqpoint{1.635887in}{2.109667in}}{\pgfqpoint{1.644123in}{2.109667in}}%
\pgfpathclose%
\pgfusepath{stroke,fill}%
\end{pgfscope}%
\begin{pgfscope}%
\pgfpathrectangle{\pgfqpoint{0.100000in}{0.212622in}}{\pgfqpoint{3.696000in}{3.696000in}}%
\pgfusepath{clip}%
\pgfsetbuttcap%
\pgfsetroundjoin%
\definecolor{currentfill}{rgb}{0.121569,0.466667,0.705882}%
\pgfsetfillcolor{currentfill}%
\pgfsetfillopacity{0.301449}%
\pgfsetlinewidth{1.003750pt}%
\definecolor{currentstroke}{rgb}{0.121569,0.466667,0.705882}%
\pgfsetstrokecolor{currentstroke}%
\pgfsetstrokeopacity{0.301449}%
\pgfsetdash{}{0pt}%
\pgfpathmoveto{\pgfqpoint{1.644123in}{2.109667in}}%
\pgfpathcurveto{\pgfqpoint{1.652359in}{2.109667in}}{\pgfqpoint{1.660259in}{2.112939in}}{\pgfqpoint{1.666083in}{2.118763in}}%
\pgfpathcurveto{\pgfqpoint{1.671907in}{2.124587in}}{\pgfqpoint{1.675179in}{2.132487in}}{\pgfqpoint{1.675179in}{2.140723in}}%
\pgfpathcurveto{\pgfqpoint{1.675179in}{2.148960in}}{\pgfqpoint{1.671907in}{2.156860in}}{\pgfqpoint{1.666083in}{2.162684in}}%
\pgfpathcurveto{\pgfqpoint{1.660259in}{2.168508in}}{\pgfqpoint{1.652359in}{2.171780in}}{\pgfqpoint{1.644123in}{2.171780in}}%
\pgfpathcurveto{\pgfqpoint{1.635887in}{2.171780in}}{\pgfqpoint{1.627986in}{2.168508in}}{\pgfqpoint{1.622163in}{2.162684in}}%
\pgfpathcurveto{\pgfqpoint{1.616339in}{2.156860in}}{\pgfqpoint{1.613066in}{2.148960in}}{\pgfqpoint{1.613066in}{2.140723in}}%
\pgfpathcurveto{\pgfqpoint{1.613066in}{2.132487in}}{\pgfqpoint{1.616339in}{2.124587in}}{\pgfqpoint{1.622163in}{2.118763in}}%
\pgfpathcurveto{\pgfqpoint{1.627986in}{2.112939in}}{\pgfqpoint{1.635887in}{2.109667in}}{\pgfqpoint{1.644123in}{2.109667in}}%
\pgfpathclose%
\pgfusepath{stroke,fill}%
\end{pgfscope}%
\begin{pgfscope}%
\pgfpathrectangle{\pgfqpoint{0.100000in}{0.212622in}}{\pgfqpoint{3.696000in}{3.696000in}}%
\pgfusepath{clip}%
\pgfsetbuttcap%
\pgfsetroundjoin%
\definecolor{currentfill}{rgb}{0.121569,0.466667,0.705882}%
\pgfsetfillcolor{currentfill}%
\pgfsetfillopacity{0.301449}%
\pgfsetlinewidth{1.003750pt}%
\definecolor{currentstroke}{rgb}{0.121569,0.466667,0.705882}%
\pgfsetstrokecolor{currentstroke}%
\pgfsetstrokeopacity{0.301449}%
\pgfsetdash{}{0pt}%
\pgfpathmoveto{\pgfqpoint{1.644123in}{2.109667in}}%
\pgfpathcurveto{\pgfqpoint{1.652359in}{2.109667in}}{\pgfqpoint{1.660259in}{2.112939in}}{\pgfqpoint{1.666083in}{2.118763in}}%
\pgfpathcurveto{\pgfqpoint{1.671907in}{2.124587in}}{\pgfqpoint{1.675179in}{2.132487in}}{\pgfqpoint{1.675179in}{2.140723in}}%
\pgfpathcurveto{\pgfqpoint{1.675179in}{2.148960in}}{\pgfqpoint{1.671907in}{2.156860in}}{\pgfqpoint{1.666083in}{2.162684in}}%
\pgfpathcurveto{\pgfqpoint{1.660259in}{2.168508in}}{\pgfqpoint{1.652359in}{2.171780in}}{\pgfqpoint{1.644123in}{2.171780in}}%
\pgfpathcurveto{\pgfqpoint{1.635886in}{2.171780in}}{\pgfqpoint{1.627986in}{2.168508in}}{\pgfqpoint{1.622162in}{2.162684in}}%
\pgfpathcurveto{\pgfqpoint{1.616339in}{2.156860in}}{\pgfqpoint{1.613066in}{2.148960in}}{\pgfqpoint{1.613066in}{2.140723in}}%
\pgfpathcurveto{\pgfqpoint{1.613066in}{2.132487in}}{\pgfqpoint{1.616339in}{2.124587in}}{\pgfqpoint{1.622162in}{2.118763in}}%
\pgfpathcurveto{\pgfqpoint{1.627986in}{2.112939in}}{\pgfqpoint{1.635886in}{2.109667in}}{\pgfqpoint{1.644123in}{2.109667in}}%
\pgfpathclose%
\pgfusepath{stroke,fill}%
\end{pgfscope}%
\begin{pgfscope}%
\pgfpathrectangle{\pgfqpoint{0.100000in}{0.212622in}}{\pgfqpoint{3.696000in}{3.696000in}}%
\pgfusepath{clip}%
\pgfsetbuttcap%
\pgfsetroundjoin%
\definecolor{currentfill}{rgb}{0.121569,0.466667,0.705882}%
\pgfsetfillcolor{currentfill}%
\pgfsetfillopacity{0.301449}%
\pgfsetlinewidth{1.003750pt}%
\definecolor{currentstroke}{rgb}{0.121569,0.466667,0.705882}%
\pgfsetstrokecolor{currentstroke}%
\pgfsetstrokeopacity{0.301449}%
\pgfsetdash{}{0pt}%
\pgfpathmoveto{\pgfqpoint{1.644122in}{2.109666in}}%
\pgfpathcurveto{\pgfqpoint{1.652358in}{2.109666in}}{\pgfqpoint{1.660259in}{2.112939in}}{\pgfqpoint{1.666082in}{2.118763in}}%
\pgfpathcurveto{\pgfqpoint{1.671906in}{2.124587in}}{\pgfqpoint{1.675179in}{2.132487in}}{\pgfqpoint{1.675179in}{2.140723in}}%
\pgfpathcurveto{\pgfqpoint{1.675179in}{2.148959in}}{\pgfqpoint{1.671906in}{2.156859in}}{\pgfqpoint{1.666082in}{2.162683in}}%
\pgfpathcurveto{\pgfqpoint{1.660259in}{2.168507in}}{\pgfqpoint{1.652358in}{2.171779in}}{\pgfqpoint{1.644122in}{2.171779in}}%
\pgfpathcurveto{\pgfqpoint{1.635886in}{2.171779in}}{\pgfqpoint{1.627986in}{2.168507in}}{\pgfqpoint{1.622162in}{2.162683in}}%
\pgfpathcurveto{\pgfqpoint{1.616338in}{2.156859in}}{\pgfqpoint{1.613066in}{2.148959in}}{\pgfqpoint{1.613066in}{2.140723in}}%
\pgfpathcurveto{\pgfqpoint{1.613066in}{2.132487in}}{\pgfqpoint{1.616338in}{2.124587in}}{\pgfqpoint{1.622162in}{2.118763in}}%
\pgfpathcurveto{\pgfqpoint{1.627986in}{2.112939in}}{\pgfqpoint{1.635886in}{2.109666in}}{\pgfqpoint{1.644122in}{2.109666in}}%
\pgfpathclose%
\pgfusepath{stroke,fill}%
\end{pgfscope}%
\begin{pgfscope}%
\pgfpathrectangle{\pgfqpoint{0.100000in}{0.212622in}}{\pgfqpoint{3.696000in}{3.696000in}}%
\pgfusepath{clip}%
\pgfsetbuttcap%
\pgfsetroundjoin%
\definecolor{currentfill}{rgb}{0.121569,0.466667,0.705882}%
\pgfsetfillcolor{currentfill}%
\pgfsetfillopacity{0.301449}%
\pgfsetlinewidth{1.003750pt}%
\definecolor{currentstroke}{rgb}{0.121569,0.466667,0.705882}%
\pgfsetstrokecolor{currentstroke}%
\pgfsetstrokeopacity{0.301449}%
\pgfsetdash{}{0pt}%
\pgfpathmoveto{\pgfqpoint{1.644122in}{2.109666in}}%
\pgfpathcurveto{\pgfqpoint{1.652358in}{2.109666in}}{\pgfqpoint{1.660258in}{2.112938in}}{\pgfqpoint{1.666082in}{2.118762in}}%
\pgfpathcurveto{\pgfqpoint{1.671906in}{2.124586in}}{\pgfqpoint{1.675178in}{2.132486in}}{\pgfqpoint{1.675178in}{2.140723in}}%
\pgfpathcurveto{\pgfqpoint{1.675178in}{2.148959in}}{\pgfqpoint{1.671906in}{2.156859in}}{\pgfqpoint{1.666082in}{2.162683in}}%
\pgfpathcurveto{\pgfqpoint{1.660258in}{2.168507in}}{\pgfqpoint{1.652358in}{2.171779in}}{\pgfqpoint{1.644122in}{2.171779in}}%
\pgfpathcurveto{\pgfqpoint{1.635885in}{2.171779in}}{\pgfqpoint{1.627985in}{2.168507in}}{\pgfqpoint{1.622161in}{2.162683in}}%
\pgfpathcurveto{\pgfqpoint{1.616338in}{2.156859in}}{\pgfqpoint{1.613065in}{2.148959in}}{\pgfqpoint{1.613065in}{2.140723in}}%
\pgfpathcurveto{\pgfqpoint{1.613065in}{2.132486in}}{\pgfqpoint{1.616338in}{2.124586in}}{\pgfqpoint{1.622161in}{2.118762in}}%
\pgfpathcurveto{\pgfqpoint{1.627985in}{2.112938in}}{\pgfqpoint{1.635885in}{2.109666in}}{\pgfqpoint{1.644122in}{2.109666in}}%
\pgfpathclose%
\pgfusepath{stroke,fill}%
\end{pgfscope}%
\begin{pgfscope}%
\pgfpathrectangle{\pgfqpoint{0.100000in}{0.212622in}}{\pgfqpoint{3.696000in}{3.696000in}}%
\pgfusepath{clip}%
\pgfsetbuttcap%
\pgfsetroundjoin%
\definecolor{currentfill}{rgb}{0.121569,0.466667,0.705882}%
\pgfsetfillcolor{currentfill}%
\pgfsetfillopacity{0.301449}%
\pgfsetlinewidth{1.003750pt}%
\definecolor{currentstroke}{rgb}{0.121569,0.466667,0.705882}%
\pgfsetstrokecolor{currentstroke}%
\pgfsetstrokeopacity{0.301449}%
\pgfsetdash{}{0pt}%
\pgfpathmoveto{\pgfqpoint{1.644120in}{2.109665in}}%
\pgfpathcurveto{\pgfqpoint{1.652357in}{2.109665in}}{\pgfqpoint{1.660257in}{2.112937in}}{\pgfqpoint{1.666081in}{2.118761in}}%
\pgfpathcurveto{\pgfqpoint{1.671905in}{2.124585in}}{\pgfqpoint{1.675177in}{2.132485in}}{\pgfqpoint{1.675177in}{2.140722in}}%
\pgfpathcurveto{\pgfqpoint{1.675177in}{2.148958in}}{\pgfqpoint{1.671905in}{2.156858in}}{\pgfqpoint{1.666081in}{2.162682in}}%
\pgfpathcurveto{\pgfqpoint{1.660257in}{2.168506in}}{\pgfqpoint{1.652357in}{2.171778in}}{\pgfqpoint{1.644120in}{2.171778in}}%
\pgfpathcurveto{\pgfqpoint{1.635884in}{2.171778in}}{\pgfqpoint{1.627984in}{2.168506in}}{\pgfqpoint{1.622160in}{2.162682in}}%
\pgfpathcurveto{\pgfqpoint{1.616336in}{2.156858in}}{\pgfqpoint{1.613064in}{2.148958in}}{\pgfqpoint{1.613064in}{2.140722in}}%
\pgfpathcurveto{\pgfqpoint{1.613064in}{2.132485in}}{\pgfqpoint{1.616336in}{2.124585in}}{\pgfqpoint{1.622160in}{2.118761in}}%
\pgfpathcurveto{\pgfqpoint{1.627984in}{2.112937in}}{\pgfqpoint{1.635884in}{2.109665in}}{\pgfqpoint{1.644120in}{2.109665in}}%
\pgfpathclose%
\pgfusepath{stroke,fill}%
\end{pgfscope}%
\begin{pgfscope}%
\pgfpathrectangle{\pgfqpoint{0.100000in}{0.212622in}}{\pgfqpoint{3.696000in}{3.696000in}}%
\pgfusepath{clip}%
\pgfsetbuttcap%
\pgfsetroundjoin%
\definecolor{currentfill}{rgb}{0.121569,0.466667,0.705882}%
\pgfsetfillcolor{currentfill}%
\pgfsetfillopacity{0.301449}%
\pgfsetlinewidth{1.003750pt}%
\definecolor{currentstroke}{rgb}{0.121569,0.466667,0.705882}%
\pgfsetstrokecolor{currentstroke}%
\pgfsetstrokeopacity{0.301449}%
\pgfsetdash{}{0pt}%
\pgfpathmoveto{\pgfqpoint{1.644119in}{2.109665in}}%
\pgfpathcurveto{\pgfqpoint{1.652356in}{2.109665in}}{\pgfqpoint{1.660256in}{2.112937in}}{\pgfqpoint{1.666080in}{2.118761in}}%
\pgfpathcurveto{\pgfqpoint{1.671904in}{2.124585in}}{\pgfqpoint{1.675176in}{2.132485in}}{\pgfqpoint{1.675176in}{2.140721in}}%
\pgfpathcurveto{\pgfqpoint{1.675176in}{2.148958in}}{\pgfqpoint{1.671904in}{2.156858in}}{\pgfqpoint{1.666080in}{2.162681in}}%
\pgfpathcurveto{\pgfqpoint{1.660256in}{2.168505in}}{\pgfqpoint{1.652356in}{2.171778in}}{\pgfqpoint{1.644119in}{2.171778in}}%
\pgfpathcurveto{\pgfqpoint{1.635883in}{2.171778in}}{\pgfqpoint{1.627983in}{2.168505in}}{\pgfqpoint{1.622159in}{2.162681in}}%
\pgfpathcurveto{\pgfqpoint{1.616335in}{2.156858in}}{\pgfqpoint{1.613063in}{2.148958in}}{\pgfqpoint{1.613063in}{2.140721in}}%
\pgfpathcurveto{\pgfqpoint{1.613063in}{2.132485in}}{\pgfqpoint{1.616335in}{2.124585in}}{\pgfqpoint{1.622159in}{2.118761in}}%
\pgfpathcurveto{\pgfqpoint{1.627983in}{2.112937in}}{\pgfqpoint{1.635883in}{2.109665in}}{\pgfqpoint{1.644119in}{2.109665in}}%
\pgfpathclose%
\pgfusepath{stroke,fill}%
\end{pgfscope}%
\begin{pgfscope}%
\pgfpathrectangle{\pgfqpoint{0.100000in}{0.212622in}}{\pgfqpoint{3.696000in}{3.696000in}}%
\pgfusepath{clip}%
\pgfsetbuttcap%
\pgfsetroundjoin%
\definecolor{currentfill}{rgb}{0.121569,0.466667,0.705882}%
\pgfsetfillcolor{currentfill}%
\pgfsetfillopacity{0.301450}%
\pgfsetlinewidth{1.003750pt}%
\definecolor{currentstroke}{rgb}{0.121569,0.466667,0.705882}%
\pgfsetstrokecolor{currentstroke}%
\pgfsetstrokeopacity{0.301450}%
\pgfsetdash{}{0pt}%
\pgfpathmoveto{\pgfqpoint{1.644115in}{2.109660in}}%
\pgfpathcurveto{\pgfqpoint{1.652351in}{2.109660in}}{\pgfqpoint{1.660251in}{2.112932in}}{\pgfqpoint{1.666075in}{2.118756in}}%
\pgfpathcurveto{\pgfqpoint{1.671899in}{2.124580in}}{\pgfqpoint{1.675171in}{2.132480in}}{\pgfqpoint{1.675171in}{2.140716in}}%
\pgfpathcurveto{\pgfqpoint{1.675171in}{2.148953in}}{\pgfqpoint{1.671899in}{2.156853in}}{\pgfqpoint{1.666075in}{2.162677in}}%
\pgfpathcurveto{\pgfqpoint{1.660251in}{2.168501in}}{\pgfqpoint{1.652351in}{2.171773in}}{\pgfqpoint{1.644115in}{2.171773in}}%
\pgfpathcurveto{\pgfqpoint{1.635879in}{2.171773in}}{\pgfqpoint{1.627979in}{2.168501in}}{\pgfqpoint{1.622155in}{2.162677in}}%
\pgfpathcurveto{\pgfqpoint{1.616331in}{2.156853in}}{\pgfqpoint{1.613058in}{2.148953in}}{\pgfqpoint{1.613058in}{2.140716in}}%
\pgfpathcurveto{\pgfqpoint{1.613058in}{2.132480in}}{\pgfqpoint{1.616331in}{2.124580in}}{\pgfqpoint{1.622155in}{2.118756in}}%
\pgfpathcurveto{\pgfqpoint{1.627979in}{2.112932in}}{\pgfqpoint{1.635879in}{2.109660in}}{\pgfqpoint{1.644115in}{2.109660in}}%
\pgfpathclose%
\pgfusepath{stroke,fill}%
\end{pgfscope}%
\begin{pgfscope}%
\pgfpathrectangle{\pgfqpoint{0.100000in}{0.212622in}}{\pgfqpoint{3.696000in}{3.696000in}}%
\pgfusepath{clip}%
\pgfsetbuttcap%
\pgfsetroundjoin%
\definecolor{currentfill}{rgb}{0.121569,0.466667,0.705882}%
\pgfsetfillcolor{currentfill}%
\pgfsetfillopacity{0.301451}%
\pgfsetlinewidth{1.003750pt}%
\definecolor{currentstroke}{rgb}{0.121569,0.466667,0.705882}%
\pgfsetstrokecolor{currentstroke}%
\pgfsetstrokeopacity{0.301451}%
\pgfsetdash{}{0pt}%
\pgfpathmoveto{\pgfqpoint{1.644105in}{2.109653in}}%
\pgfpathcurveto{\pgfqpoint{1.652341in}{2.109653in}}{\pgfqpoint{1.660241in}{2.112926in}}{\pgfqpoint{1.666065in}{2.118750in}}%
\pgfpathcurveto{\pgfqpoint{1.671889in}{2.124573in}}{\pgfqpoint{1.675162in}{2.132474in}}{\pgfqpoint{1.675162in}{2.140710in}}%
\pgfpathcurveto{\pgfqpoint{1.675162in}{2.148946in}}{\pgfqpoint{1.671889in}{2.156846in}}{\pgfqpoint{1.666065in}{2.162670in}}%
\pgfpathcurveto{\pgfqpoint{1.660241in}{2.168494in}}{\pgfqpoint{1.652341in}{2.171766in}}{\pgfqpoint{1.644105in}{2.171766in}}%
\pgfpathcurveto{\pgfqpoint{1.635869in}{2.171766in}}{\pgfqpoint{1.627969in}{2.168494in}}{\pgfqpoint{1.622145in}{2.162670in}}%
\pgfpathcurveto{\pgfqpoint{1.616321in}{2.156846in}}{\pgfqpoint{1.613049in}{2.148946in}}{\pgfqpoint{1.613049in}{2.140710in}}%
\pgfpathcurveto{\pgfqpoint{1.613049in}{2.132474in}}{\pgfqpoint{1.616321in}{2.124573in}}{\pgfqpoint{1.622145in}{2.118750in}}%
\pgfpathcurveto{\pgfqpoint{1.627969in}{2.112926in}}{\pgfqpoint{1.635869in}{2.109653in}}{\pgfqpoint{1.644105in}{2.109653in}}%
\pgfpathclose%
\pgfusepath{stroke,fill}%
\end{pgfscope}%
\begin{pgfscope}%
\pgfpathrectangle{\pgfqpoint{0.100000in}{0.212622in}}{\pgfqpoint{3.696000in}{3.696000in}}%
\pgfusepath{clip}%
\pgfsetbuttcap%
\pgfsetroundjoin%
\definecolor{currentfill}{rgb}{0.121569,0.466667,0.705882}%
\pgfsetfillcolor{currentfill}%
\pgfsetfillopacity{0.301451}%
\pgfsetlinewidth{1.003750pt}%
\definecolor{currentstroke}{rgb}{0.121569,0.466667,0.705882}%
\pgfsetstrokecolor{currentstroke}%
\pgfsetstrokeopacity{0.301451}%
\pgfsetdash{}{0pt}%
\pgfpathmoveto{\pgfqpoint{1.644095in}{2.109648in}}%
\pgfpathcurveto{\pgfqpoint{1.652331in}{2.109648in}}{\pgfqpoint{1.660231in}{2.112921in}}{\pgfqpoint{1.666055in}{2.118745in}}%
\pgfpathcurveto{\pgfqpoint{1.671879in}{2.124569in}}{\pgfqpoint{1.675151in}{2.132469in}}{\pgfqpoint{1.675151in}{2.140705in}}%
\pgfpathcurveto{\pgfqpoint{1.675151in}{2.148941in}}{\pgfqpoint{1.671879in}{2.156841in}}{\pgfqpoint{1.666055in}{2.162665in}}%
\pgfpathcurveto{\pgfqpoint{1.660231in}{2.168489in}}{\pgfqpoint{1.652331in}{2.171761in}}{\pgfqpoint{1.644095in}{2.171761in}}%
\pgfpathcurveto{\pgfqpoint{1.635859in}{2.171761in}}{\pgfqpoint{1.627959in}{2.168489in}}{\pgfqpoint{1.622135in}{2.162665in}}%
\pgfpathcurveto{\pgfqpoint{1.616311in}{2.156841in}}{\pgfqpoint{1.613038in}{2.148941in}}{\pgfqpoint{1.613038in}{2.140705in}}%
\pgfpathcurveto{\pgfqpoint{1.613038in}{2.132469in}}{\pgfqpoint{1.616311in}{2.124569in}}{\pgfqpoint{1.622135in}{2.118745in}}%
\pgfpathcurveto{\pgfqpoint{1.627959in}{2.112921in}}{\pgfqpoint{1.635859in}{2.109648in}}{\pgfqpoint{1.644095in}{2.109648in}}%
\pgfpathclose%
\pgfusepath{stroke,fill}%
\end{pgfscope}%
\begin{pgfscope}%
\pgfpathrectangle{\pgfqpoint{0.100000in}{0.212622in}}{\pgfqpoint{3.696000in}{3.696000in}}%
\pgfusepath{clip}%
\pgfsetbuttcap%
\pgfsetroundjoin%
\definecolor{currentfill}{rgb}{0.121569,0.466667,0.705882}%
\pgfsetfillcolor{currentfill}%
\pgfsetfillopacity{0.301456}%
\pgfsetlinewidth{1.003750pt}%
\definecolor{currentstroke}{rgb}{0.121569,0.466667,0.705882}%
\pgfsetstrokecolor{currentstroke}%
\pgfsetstrokeopacity{0.301456}%
\pgfsetdash{}{0pt}%
\pgfpathmoveto{\pgfqpoint{1.644060in}{2.109619in}}%
\pgfpathcurveto{\pgfqpoint{1.652296in}{2.109619in}}{\pgfqpoint{1.660196in}{2.112891in}}{\pgfqpoint{1.666020in}{2.118715in}}%
\pgfpathcurveto{\pgfqpoint{1.671844in}{2.124539in}}{\pgfqpoint{1.675116in}{2.132439in}}{\pgfqpoint{1.675116in}{2.140675in}}%
\pgfpathcurveto{\pgfqpoint{1.675116in}{2.148912in}}{\pgfqpoint{1.671844in}{2.156812in}}{\pgfqpoint{1.666020in}{2.162636in}}%
\pgfpathcurveto{\pgfqpoint{1.660196in}{2.168460in}}{\pgfqpoint{1.652296in}{2.171732in}}{\pgfqpoint{1.644060in}{2.171732in}}%
\pgfpathcurveto{\pgfqpoint{1.635824in}{2.171732in}}{\pgfqpoint{1.627924in}{2.168460in}}{\pgfqpoint{1.622100in}{2.162636in}}%
\pgfpathcurveto{\pgfqpoint{1.616276in}{2.156812in}}{\pgfqpoint{1.613003in}{2.148912in}}{\pgfqpoint{1.613003in}{2.140675in}}%
\pgfpathcurveto{\pgfqpoint{1.613003in}{2.132439in}}{\pgfqpoint{1.616276in}{2.124539in}}{\pgfqpoint{1.622100in}{2.118715in}}%
\pgfpathcurveto{\pgfqpoint{1.627924in}{2.112891in}}{\pgfqpoint{1.635824in}{2.109619in}}{\pgfqpoint{1.644060in}{2.109619in}}%
\pgfpathclose%
\pgfusepath{stroke,fill}%
\end{pgfscope}%
\begin{pgfscope}%
\pgfpathrectangle{\pgfqpoint{0.100000in}{0.212622in}}{\pgfqpoint{3.696000in}{3.696000in}}%
\pgfusepath{clip}%
\pgfsetbuttcap%
\pgfsetroundjoin%
\definecolor{currentfill}{rgb}{0.121569,0.466667,0.705882}%
\pgfsetfillcolor{currentfill}%
\pgfsetfillopacity{0.301464}%
\pgfsetlinewidth{1.003750pt}%
\definecolor{currentstroke}{rgb}{0.121569,0.466667,0.705882}%
\pgfsetstrokecolor{currentstroke}%
\pgfsetstrokeopacity{0.301464}%
\pgfsetdash{}{0pt}%
\pgfpathmoveto{\pgfqpoint{1.644005in}{2.109576in}}%
\pgfpathcurveto{\pgfqpoint{1.652241in}{2.109576in}}{\pgfqpoint{1.660142in}{2.112848in}}{\pgfqpoint{1.665965in}{2.118672in}}%
\pgfpathcurveto{\pgfqpoint{1.671789in}{2.124496in}}{\pgfqpoint{1.675062in}{2.132396in}}{\pgfqpoint{1.675062in}{2.140632in}}%
\pgfpathcurveto{\pgfqpoint{1.675062in}{2.148868in}}{\pgfqpoint{1.671789in}{2.156768in}}{\pgfqpoint{1.665965in}{2.162592in}}%
\pgfpathcurveto{\pgfqpoint{1.660142in}{2.168416in}}{\pgfqpoint{1.652241in}{2.171689in}}{\pgfqpoint{1.644005in}{2.171689in}}%
\pgfpathcurveto{\pgfqpoint{1.635769in}{2.171689in}}{\pgfqpoint{1.627869in}{2.168416in}}{\pgfqpoint{1.622045in}{2.162592in}}%
\pgfpathcurveto{\pgfqpoint{1.616221in}{2.156768in}}{\pgfqpoint{1.612949in}{2.148868in}}{\pgfqpoint{1.612949in}{2.140632in}}%
\pgfpathcurveto{\pgfqpoint{1.612949in}{2.132396in}}{\pgfqpoint{1.616221in}{2.124496in}}{\pgfqpoint{1.622045in}{2.118672in}}%
\pgfpathcurveto{\pgfqpoint{1.627869in}{2.112848in}}{\pgfqpoint{1.635769in}{2.109576in}}{\pgfqpoint{1.644005in}{2.109576in}}%
\pgfpathclose%
\pgfusepath{stroke,fill}%
\end{pgfscope}%
\begin{pgfscope}%
\pgfpathrectangle{\pgfqpoint{0.100000in}{0.212622in}}{\pgfqpoint{3.696000in}{3.696000in}}%
\pgfusepath{clip}%
\pgfsetbuttcap%
\pgfsetroundjoin%
\definecolor{currentfill}{rgb}{0.121569,0.466667,0.705882}%
\pgfsetfillcolor{currentfill}%
\pgfsetfillopacity{0.302100}%
\pgfsetlinewidth{1.003750pt}%
\definecolor{currentstroke}{rgb}{0.121569,0.466667,0.705882}%
\pgfsetstrokecolor{currentstroke}%
\pgfsetstrokeopacity{0.302100}%
\pgfsetdash{}{0pt}%
\pgfpathmoveto{\pgfqpoint{1.648865in}{2.115035in}}%
\pgfpathcurveto{\pgfqpoint{1.657101in}{2.115035in}}{\pgfqpoint{1.665001in}{2.118308in}}{\pgfqpoint{1.670825in}{2.124132in}}%
\pgfpathcurveto{\pgfqpoint{1.676649in}{2.129956in}}{\pgfqpoint{1.679922in}{2.137856in}}{\pgfqpoint{1.679922in}{2.146092in}}%
\pgfpathcurveto{\pgfqpoint{1.679922in}{2.154328in}}{\pgfqpoint{1.676649in}{2.162228in}}{\pgfqpoint{1.670825in}{2.168052in}}%
\pgfpathcurveto{\pgfqpoint{1.665001in}{2.173876in}}{\pgfqpoint{1.657101in}{2.177148in}}{\pgfqpoint{1.648865in}{2.177148in}}%
\pgfpathcurveto{\pgfqpoint{1.640629in}{2.177148in}}{\pgfqpoint{1.632729in}{2.173876in}}{\pgfqpoint{1.626905in}{2.168052in}}%
\pgfpathcurveto{\pgfqpoint{1.621081in}{2.162228in}}{\pgfqpoint{1.617809in}{2.154328in}}{\pgfqpoint{1.617809in}{2.146092in}}%
\pgfpathcurveto{\pgfqpoint{1.617809in}{2.137856in}}{\pgfqpoint{1.621081in}{2.129956in}}{\pgfqpoint{1.626905in}{2.124132in}}%
\pgfpathcurveto{\pgfqpoint{1.632729in}{2.118308in}}{\pgfqpoint{1.640629in}{2.115035in}}{\pgfqpoint{1.648865in}{2.115035in}}%
\pgfpathclose%
\pgfusepath{stroke,fill}%
\end{pgfscope}%
\begin{pgfscope}%
\pgfpathrectangle{\pgfqpoint{0.100000in}{0.212622in}}{\pgfqpoint{3.696000in}{3.696000in}}%
\pgfusepath{clip}%
\pgfsetbuttcap%
\pgfsetroundjoin%
\definecolor{currentfill}{rgb}{0.121569,0.466667,0.705882}%
\pgfsetfillcolor{currentfill}%
\pgfsetfillopacity{0.304056}%
\pgfsetlinewidth{1.003750pt}%
\definecolor{currentstroke}{rgb}{0.121569,0.466667,0.705882}%
\pgfsetstrokecolor{currentstroke}%
\pgfsetstrokeopacity{0.304056}%
\pgfsetdash{}{0pt}%
\pgfpathmoveto{\pgfqpoint{1.644245in}{2.109937in}}%
\pgfpathcurveto{\pgfqpoint{1.652481in}{2.109937in}}{\pgfqpoint{1.660381in}{2.113210in}}{\pgfqpoint{1.666205in}{2.119034in}}%
\pgfpathcurveto{\pgfqpoint{1.672029in}{2.124857in}}{\pgfqpoint{1.675301in}{2.132757in}}{\pgfqpoint{1.675301in}{2.140994in}}%
\pgfpathcurveto{\pgfqpoint{1.675301in}{2.149230in}}{\pgfqpoint{1.672029in}{2.157130in}}{\pgfqpoint{1.666205in}{2.162954in}}%
\pgfpathcurveto{\pgfqpoint{1.660381in}{2.168778in}}{\pgfqpoint{1.652481in}{2.172050in}}{\pgfqpoint{1.644245in}{2.172050in}}%
\pgfpathcurveto{\pgfqpoint{1.636008in}{2.172050in}}{\pgfqpoint{1.628108in}{2.168778in}}{\pgfqpoint{1.622284in}{2.162954in}}%
\pgfpathcurveto{\pgfqpoint{1.616460in}{2.157130in}}{\pgfqpoint{1.613188in}{2.149230in}}{\pgfqpoint{1.613188in}{2.140994in}}%
\pgfpathcurveto{\pgfqpoint{1.613188in}{2.132757in}}{\pgfqpoint{1.616460in}{2.124857in}}{\pgfqpoint{1.622284in}{2.119034in}}%
\pgfpathcurveto{\pgfqpoint{1.628108in}{2.113210in}}{\pgfqpoint{1.636008in}{2.109937in}}{\pgfqpoint{1.644245in}{2.109937in}}%
\pgfpathclose%
\pgfusepath{stroke,fill}%
\end{pgfscope}%
\begin{pgfscope}%
\pgfpathrectangle{\pgfqpoint{0.100000in}{0.212622in}}{\pgfqpoint{3.696000in}{3.696000in}}%
\pgfusepath{clip}%
\pgfsetbuttcap%
\pgfsetroundjoin%
\definecolor{currentfill}{rgb}{0.121569,0.466667,0.705882}%
\pgfsetfillcolor{currentfill}%
\pgfsetfillopacity{0.304764}%
\pgfsetlinewidth{1.003750pt}%
\definecolor{currentstroke}{rgb}{0.121569,0.466667,0.705882}%
\pgfsetstrokecolor{currentstroke}%
\pgfsetstrokeopacity{0.304764}%
\pgfsetdash{}{0pt}%
\pgfpathmoveto{\pgfqpoint{1.645084in}{2.113164in}}%
\pgfpathcurveto{\pgfqpoint{1.653320in}{2.113164in}}{\pgfqpoint{1.661220in}{2.116436in}}{\pgfqpoint{1.667044in}{2.122260in}}%
\pgfpathcurveto{\pgfqpoint{1.672868in}{2.128084in}}{\pgfqpoint{1.676140in}{2.135984in}}{\pgfqpoint{1.676140in}{2.144220in}}%
\pgfpathcurveto{\pgfqpoint{1.676140in}{2.152457in}}{\pgfqpoint{1.672868in}{2.160357in}}{\pgfqpoint{1.667044in}{2.166181in}}%
\pgfpathcurveto{\pgfqpoint{1.661220in}{2.172005in}}{\pgfqpoint{1.653320in}{2.175277in}}{\pgfqpoint{1.645084in}{2.175277in}}%
\pgfpathcurveto{\pgfqpoint{1.636848in}{2.175277in}}{\pgfqpoint{1.628948in}{2.172005in}}{\pgfqpoint{1.623124in}{2.166181in}}%
\pgfpathcurveto{\pgfqpoint{1.617300in}{2.160357in}}{\pgfqpoint{1.614027in}{2.152457in}}{\pgfqpoint{1.614027in}{2.144220in}}%
\pgfpathcurveto{\pgfqpoint{1.614027in}{2.135984in}}{\pgfqpoint{1.617300in}{2.128084in}}{\pgfqpoint{1.623124in}{2.122260in}}%
\pgfpathcurveto{\pgfqpoint{1.628948in}{2.116436in}}{\pgfqpoint{1.636848in}{2.113164in}}{\pgfqpoint{1.645084in}{2.113164in}}%
\pgfpathclose%
\pgfusepath{stroke,fill}%
\end{pgfscope}%
\begin{pgfscope}%
\pgfpathrectangle{\pgfqpoint{0.100000in}{0.212622in}}{\pgfqpoint{3.696000in}{3.696000in}}%
\pgfusepath{clip}%
\pgfsetbuttcap%
\pgfsetroundjoin%
\definecolor{currentfill}{rgb}{0.121569,0.466667,0.705882}%
\pgfsetfillcolor{currentfill}%
\pgfsetfillopacity{0.309200}%
\pgfsetlinewidth{1.003750pt}%
\definecolor{currentstroke}{rgb}{0.121569,0.466667,0.705882}%
\pgfsetstrokecolor{currentstroke}%
\pgfsetstrokeopacity{0.309200}%
\pgfsetdash{}{0pt}%
\pgfpathmoveto{\pgfqpoint{1.635893in}{2.104212in}}%
\pgfpathcurveto{\pgfqpoint{1.644129in}{2.104212in}}{\pgfqpoint{1.652029in}{2.107484in}}{\pgfqpoint{1.657853in}{2.113308in}}%
\pgfpathcurveto{\pgfqpoint{1.663677in}{2.119132in}}{\pgfqpoint{1.666950in}{2.127032in}}{\pgfqpoint{1.666950in}{2.135268in}}%
\pgfpathcurveto{\pgfqpoint{1.666950in}{2.143504in}}{\pgfqpoint{1.663677in}{2.151405in}}{\pgfqpoint{1.657853in}{2.157228in}}%
\pgfpathcurveto{\pgfqpoint{1.652029in}{2.163052in}}{\pgfqpoint{1.644129in}{2.166325in}}{\pgfqpoint{1.635893in}{2.166325in}}%
\pgfpathcurveto{\pgfqpoint{1.627657in}{2.166325in}}{\pgfqpoint{1.619757in}{2.163052in}}{\pgfqpoint{1.613933in}{2.157228in}}%
\pgfpathcurveto{\pgfqpoint{1.608109in}{2.151405in}}{\pgfqpoint{1.604837in}{2.143504in}}{\pgfqpoint{1.604837in}{2.135268in}}%
\pgfpathcurveto{\pgfqpoint{1.604837in}{2.127032in}}{\pgfqpoint{1.608109in}{2.119132in}}{\pgfqpoint{1.613933in}{2.113308in}}%
\pgfpathcurveto{\pgfqpoint{1.619757in}{2.107484in}}{\pgfqpoint{1.627657in}{2.104212in}}{\pgfqpoint{1.635893in}{2.104212in}}%
\pgfpathclose%
\pgfusepath{stroke,fill}%
\end{pgfscope}%
\begin{pgfscope}%
\pgfpathrectangle{\pgfqpoint{0.100000in}{0.212622in}}{\pgfqpoint{3.696000in}{3.696000in}}%
\pgfusepath{clip}%
\pgfsetbuttcap%
\pgfsetroundjoin%
\definecolor{currentfill}{rgb}{0.121569,0.466667,0.705882}%
\pgfsetfillcolor{currentfill}%
\pgfsetfillopacity{0.310041}%
\pgfsetlinewidth{1.003750pt}%
\definecolor{currentstroke}{rgb}{0.121569,0.466667,0.705882}%
\pgfsetstrokecolor{currentstroke}%
\pgfsetstrokeopacity{0.310041}%
\pgfsetdash{}{0pt}%
\pgfpathmoveto{\pgfqpoint{1.635020in}{2.103100in}}%
\pgfpathcurveto{\pgfqpoint{1.643257in}{2.103100in}}{\pgfqpoint{1.651157in}{2.106372in}}{\pgfqpoint{1.656981in}{2.112196in}}%
\pgfpathcurveto{\pgfqpoint{1.662804in}{2.118020in}}{\pgfqpoint{1.666077in}{2.125920in}}{\pgfqpoint{1.666077in}{2.134157in}}%
\pgfpathcurveto{\pgfqpoint{1.666077in}{2.142393in}}{\pgfqpoint{1.662804in}{2.150293in}}{\pgfqpoint{1.656981in}{2.156117in}}%
\pgfpathcurveto{\pgfqpoint{1.651157in}{2.161941in}}{\pgfqpoint{1.643257in}{2.165213in}}{\pgfqpoint{1.635020in}{2.165213in}}%
\pgfpathcurveto{\pgfqpoint{1.626784in}{2.165213in}}{\pgfqpoint{1.618884in}{2.161941in}}{\pgfqpoint{1.613060in}{2.156117in}}%
\pgfpathcurveto{\pgfqpoint{1.607236in}{2.150293in}}{\pgfqpoint{1.603964in}{2.142393in}}{\pgfqpoint{1.603964in}{2.134157in}}%
\pgfpathcurveto{\pgfqpoint{1.603964in}{2.125920in}}{\pgfqpoint{1.607236in}{2.118020in}}{\pgfqpoint{1.613060in}{2.112196in}}%
\pgfpathcurveto{\pgfqpoint{1.618884in}{2.106372in}}{\pgfqpoint{1.626784in}{2.103100in}}{\pgfqpoint{1.635020in}{2.103100in}}%
\pgfpathclose%
\pgfusepath{stroke,fill}%
\end{pgfscope}%
\begin{pgfscope}%
\pgfpathrectangle{\pgfqpoint{0.100000in}{0.212622in}}{\pgfqpoint{3.696000in}{3.696000in}}%
\pgfusepath{clip}%
\pgfsetbuttcap%
\pgfsetroundjoin%
\definecolor{currentfill}{rgb}{0.121569,0.466667,0.705882}%
\pgfsetfillcolor{currentfill}%
\pgfsetfillopacity{0.313502}%
\pgfsetlinewidth{1.003750pt}%
\definecolor{currentstroke}{rgb}{0.121569,0.466667,0.705882}%
\pgfsetstrokecolor{currentstroke}%
\pgfsetstrokeopacity{0.313502}%
\pgfsetdash{}{0pt}%
\pgfpathmoveto{\pgfqpoint{1.629264in}{2.099304in}}%
\pgfpathcurveto{\pgfqpoint{1.637500in}{2.099304in}}{\pgfqpoint{1.645400in}{2.102576in}}{\pgfqpoint{1.651224in}{2.108400in}}%
\pgfpathcurveto{\pgfqpoint{1.657048in}{2.114224in}}{\pgfqpoint{1.660320in}{2.122124in}}{\pgfqpoint{1.660320in}{2.130360in}}%
\pgfpathcurveto{\pgfqpoint{1.660320in}{2.138596in}}{\pgfqpoint{1.657048in}{2.146496in}}{\pgfqpoint{1.651224in}{2.152320in}}%
\pgfpathcurveto{\pgfqpoint{1.645400in}{2.158144in}}{\pgfqpoint{1.637500in}{2.161417in}}{\pgfqpoint{1.629264in}{2.161417in}}%
\pgfpathcurveto{\pgfqpoint{1.621028in}{2.161417in}}{\pgfqpoint{1.613127in}{2.158144in}}{\pgfqpoint{1.607304in}{2.152320in}}%
\pgfpathcurveto{\pgfqpoint{1.601480in}{2.146496in}}{\pgfqpoint{1.598207in}{2.138596in}}{\pgfqpoint{1.598207in}{2.130360in}}%
\pgfpathcurveto{\pgfqpoint{1.598207in}{2.122124in}}{\pgfqpoint{1.601480in}{2.114224in}}{\pgfqpoint{1.607304in}{2.108400in}}%
\pgfpathcurveto{\pgfqpoint{1.613127in}{2.102576in}}{\pgfqpoint{1.621028in}{2.099304in}}{\pgfqpoint{1.629264in}{2.099304in}}%
\pgfpathclose%
\pgfusepath{stroke,fill}%
\end{pgfscope}%
\begin{pgfscope}%
\pgfpathrectangle{\pgfqpoint{0.100000in}{0.212622in}}{\pgfqpoint{3.696000in}{3.696000in}}%
\pgfusepath{clip}%
\pgfsetbuttcap%
\pgfsetroundjoin%
\definecolor{currentfill}{rgb}{0.121569,0.466667,0.705882}%
\pgfsetfillcolor{currentfill}%
\pgfsetfillopacity{0.318043}%
\pgfsetlinewidth{1.003750pt}%
\definecolor{currentstroke}{rgb}{0.121569,0.466667,0.705882}%
\pgfsetstrokecolor{currentstroke}%
\pgfsetstrokeopacity{0.318043}%
\pgfsetdash{}{0pt}%
\pgfpathmoveto{\pgfqpoint{1.620560in}{2.090294in}}%
\pgfpathcurveto{\pgfqpoint{1.628796in}{2.090294in}}{\pgfqpoint{1.636696in}{2.093567in}}{\pgfqpoint{1.642520in}{2.099391in}}%
\pgfpathcurveto{\pgfqpoint{1.648344in}{2.105215in}}{\pgfqpoint{1.651616in}{2.113115in}}{\pgfqpoint{1.651616in}{2.121351in}}%
\pgfpathcurveto{\pgfqpoint{1.651616in}{2.129587in}}{\pgfqpoint{1.648344in}{2.137487in}}{\pgfqpoint{1.642520in}{2.143311in}}%
\pgfpathcurveto{\pgfqpoint{1.636696in}{2.149135in}}{\pgfqpoint{1.628796in}{2.152407in}}{\pgfqpoint{1.620560in}{2.152407in}}%
\pgfpathcurveto{\pgfqpoint{1.612324in}{2.152407in}}{\pgfqpoint{1.604424in}{2.149135in}}{\pgfqpoint{1.598600in}{2.143311in}}%
\pgfpathcurveto{\pgfqpoint{1.592776in}{2.137487in}}{\pgfqpoint{1.589503in}{2.129587in}}{\pgfqpoint{1.589503in}{2.121351in}}%
\pgfpathcurveto{\pgfqpoint{1.589503in}{2.113115in}}{\pgfqpoint{1.592776in}{2.105215in}}{\pgfqpoint{1.598600in}{2.099391in}}%
\pgfpathcurveto{\pgfqpoint{1.604424in}{2.093567in}}{\pgfqpoint{1.612324in}{2.090294in}}{\pgfqpoint{1.620560in}{2.090294in}}%
\pgfpathclose%
\pgfusepath{stroke,fill}%
\end{pgfscope}%
\begin{pgfscope}%
\pgfpathrectangle{\pgfqpoint{0.100000in}{0.212622in}}{\pgfqpoint{3.696000in}{3.696000in}}%
\pgfusepath{clip}%
\pgfsetbuttcap%
\pgfsetroundjoin%
\definecolor{currentfill}{rgb}{0.121569,0.466667,0.705882}%
\pgfsetfillcolor{currentfill}%
\pgfsetfillopacity{0.318753}%
\pgfsetlinewidth{1.003750pt}%
\definecolor{currentstroke}{rgb}{0.121569,0.466667,0.705882}%
\pgfsetstrokecolor{currentstroke}%
\pgfsetstrokeopacity{0.318753}%
\pgfsetdash{}{0pt}%
\pgfpathmoveto{\pgfqpoint{1.620726in}{2.090593in}}%
\pgfpathcurveto{\pgfqpoint{1.628962in}{2.090593in}}{\pgfqpoint{1.636863in}{2.093865in}}{\pgfqpoint{1.642686in}{2.099689in}}%
\pgfpathcurveto{\pgfqpoint{1.648510in}{2.105513in}}{\pgfqpoint{1.651783in}{2.113413in}}{\pgfqpoint{1.651783in}{2.121649in}}%
\pgfpathcurveto{\pgfqpoint{1.651783in}{2.129886in}}{\pgfqpoint{1.648510in}{2.137786in}}{\pgfqpoint{1.642686in}{2.143610in}}%
\pgfpathcurveto{\pgfqpoint{1.636863in}{2.149434in}}{\pgfqpoint{1.628962in}{2.152706in}}{\pgfqpoint{1.620726in}{2.152706in}}%
\pgfpathcurveto{\pgfqpoint{1.612490in}{2.152706in}}{\pgfqpoint{1.604590in}{2.149434in}}{\pgfqpoint{1.598766in}{2.143610in}}%
\pgfpathcurveto{\pgfqpoint{1.592942in}{2.137786in}}{\pgfqpoint{1.589670in}{2.129886in}}{\pgfqpoint{1.589670in}{2.121649in}}%
\pgfpathcurveto{\pgfqpoint{1.589670in}{2.113413in}}{\pgfqpoint{1.592942in}{2.105513in}}{\pgfqpoint{1.598766in}{2.099689in}}%
\pgfpathcurveto{\pgfqpoint{1.604590in}{2.093865in}}{\pgfqpoint{1.612490in}{2.090593in}}{\pgfqpoint{1.620726in}{2.090593in}}%
\pgfpathclose%
\pgfusepath{stroke,fill}%
\end{pgfscope}%
\begin{pgfscope}%
\pgfpathrectangle{\pgfqpoint{0.100000in}{0.212622in}}{\pgfqpoint{3.696000in}{3.696000in}}%
\pgfusepath{clip}%
\pgfsetbuttcap%
\pgfsetroundjoin%
\definecolor{currentfill}{rgb}{0.121569,0.466667,0.705882}%
\pgfsetfillcolor{currentfill}%
\pgfsetfillopacity{0.321804}%
\pgfsetlinewidth{1.003750pt}%
\definecolor{currentstroke}{rgb}{0.121569,0.466667,0.705882}%
\pgfsetstrokecolor{currentstroke}%
\pgfsetstrokeopacity{0.321804}%
\pgfsetdash{}{0pt}%
\pgfpathmoveto{\pgfqpoint{1.616670in}{2.088607in}}%
\pgfpathcurveto{\pgfqpoint{1.624906in}{2.088607in}}{\pgfqpoint{1.632806in}{2.091880in}}{\pgfqpoint{1.638630in}{2.097704in}}%
\pgfpathcurveto{\pgfqpoint{1.644454in}{2.103527in}}{\pgfqpoint{1.647726in}{2.111428in}}{\pgfqpoint{1.647726in}{2.119664in}}%
\pgfpathcurveto{\pgfqpoint{1.647726in}{2.127900in}}{\pgfqpoint{1.644454in}{2.135800in}}{\pgfqpoint{1.638630in}{2.141624in}}%
\pgfpathcurveto{\pgfqpoint{1.632806in}{2.147448in}}{\pgfqpoint{1.624906in}{2.150720in}}{\pgfqpoint{1.616670in}{2.150720in}}%
\pgfpathcurveto{\pgfqpoint{1.608433in}{2.150720in}}{\pgfqpoint{1.600533in}{2.147448in}}{\pgfqpoint{1.594710in}{2.141624in}}%
\pgfpathcurveto{\pgfqpoint{1.588886in}{2.135800in}}{\pgfqpoint{1.585613in}{2.127900in}}{\pgfqpoint{1.585613in}{2.119664in}}%
\pgfpathcurveto{\pgfqpoint{1.585613in}{2.111428in}}{\pgfqpoint{1.588886in}{2.103527in}}{\pgfqpoint{1.594710in}{2.097704in}}%
\pgfpathcurveto{\pgfqpoint{1.600533in}{2.091880in}}{\pgfqpoint{1.608433in}{2.088607in}}{\pgfqpoint{1.616670in}{2.088607in}}%
\pgfpathclose%
\pgfusepath{stroke,fill}%
\end{pgfscope}%
\begin{pgfscope}%
\pgfpathrectangle{\pgfqpoint{0.100000in}{0.212622in}}{\pgfqpoint{3.696000in}{3.696000in}}%
\pgfusepath{clip}%
\pgfsetbuttcap%
\pgfsetroundjoin%
\definecolor{currentfill}{rgb}{0.121569,0.466667,0.705882}%
\pgfsetfillcolor{currentfill}%
\pgfsetfillopacity{0.327055}%
\pgfsetlinewidth{1.003750pt}%
\definecolor{currentstroke}{rgb}{0.121569,0.466667,0.705882}%
\pgfsetstrokecolor{currentstroke}%
\pgfsetstrokeopacity{0.327055}%
\pgfsetdash{}{0pt}%
\pgfpathmoveto{\pgfqpoint{1.606738in}{2.077301in}}%
\pgfpathcurveto{\pgfqpoint{1.614974in}{2.077301in}}{\pgfqpoint{1.622874in}{2.080573in}}{\pgfqpoint{1.628698in}{2.086397in}}%
\pgfpathcurveto{\pgfqpoint{1.634522in}{2.092221in}}{\pgfqpoint{1.637794in}{2.100121in}}{\pgfqpoint{1.637794in}{2.108358in}}%
\pgfpathcurveto{\pgfqpoint{1.637794in}{2.116594in}}{\pgfqpoint{1.634522in}{2.124494in}}{\pgfqpoint{1.628698in}{2.130318in}}%
\pgfpathcurveto{\pgfqpoint{1.622874in}{2.136142in}}{\pgfqpoint{1.614974in}{2.139414in}}{\pgfqpoint{1.606738in}{2.139414in}}%
\pgfpathcurveto{\pgfqpoint{1.598501in}{2.139414in}}{\pgfqpoint{1.590601in}{2.136142in}}{\pgfqpoint{1.584777in}{2.130318in}}%
\pgfpathcurveto{\pgfqpoint{1.578953in}{2.124494in}}{\pgfqpoint{1.575681in}{2.116594in}}{\pgfqpoint{1.575681in}{2.108358in}}%
\pgfpathcurveto{\pgfqpoint{1.575681in}{2.100121in}}{\pgfqpoint{1.578953in}{2.092221in}}{\pgfqpoint{1.584777in}{2.086397in}}%
\pgfpathcurveto{\pgfqpoint{1.590601in}{2.080573in}}{\pgfqpoint{1.598501in}{2.077301in}}{\pgfqpoint{1.606738in}{2.077301in}}%
\pgfpathclose%
\pgfusepath{stroke,fill}%
\end{pgfscope}%
\begin{pgfscope}%
\pgfpathrectangle{\pgfqpoint{0.100000in}{0.212622in}}{\pgfqpoint{3.696000in}{3.696000in}}%
\pgfusepath{clip}%
\pgfsetbuttcap%
\pgfsetroundjoin%
\definecolor{currentfill}{rgb}{0.121569,0.466667,0.705882}%
\pgfsetfillcolor{currentfill}%
\pgfsetfillopacity{0.329111}%
\pgfsetlinewidth{1.003750pt}%
\definecolor{currentstroke}{rgb}{0.121569,0.466667,0.705882}%
\pgfsetstrokecolor{currentstroke}%
\pgfsetstrokeopacity{0.329111}%
\pgfsetdash{}{0pt}%
\pgfpathmoveto{\pgfqpoint{1.604192in}{2.075634in}}%
\pgfpathcurveto{\pgfqpoint{1.612428in}{2.075634in}}{\pgfqpoint{1.620328in}{2.078906in}}{\pgfqpoint{1.626152in}{2.084730in}}%
\pgfpathcurveto{\pgfqpoint{1.631976in}{2.090554in}}{\pgfqpoint{1.635248in}{2.098454in}}{\pgfqpoint{1.635248in}{2.106690in}}%
\pgfpathcurveto{\pgfqpoint{1.635248in}{2.114926in}}{\pgfqpoint{1.631976in}{2.122826in}}{\pgfqpoint{1.626152in}{2.128650in}}%
\pgfpathcurveto{\pgfqpoint{1.620328in}{2.134474in}}{\pgfqpoint{1.612428in}{2.137746in}}{\pgfqpoint{1.604192in}{2.137746in}}%
\pgfpathcurveto{\pgfqpoint{1.595955in}{2.137746in}}{\pgfqpoint{1.588055in}{2.134474in}}{\pgfqpoint{1.582231in}{2.128650in}}%
\pgfpathcurveto{\pgfqpoint{1.576408in}{2.122826in}}{\pgfqpoint{1.573135in}{2.114926in}}{\pgfqpoint{1.573135in}{2.106690in}}%
\pgfpathcurveto{\pgfqpoint{1.573135in}{2.098454in}}{\pgfqpoint{1.576408in}{2.090554in}}{\pgfqpoint{1.582231in}{2.084730in}}%
\pgfpathcurveto{\pgfqpoint{1.588055in}{2.078906in}}{\pgfqpoint{1.595955in}{2.075634in}}{\pgfqpoint{1.604192in}{2.075634in}}%
\pgfpathclose%
\pgfusepath{stroke,fill}%
\end{pgfscope}%
\begin{pgfscope}%
\pgfpathrectangle{\pgfqpoint{0.100000in}{0.212622in}}{\pgfqpoint{3.696000in}{3.696000in}}%
\pgfusepath{clip}%
\pgfsetbuttcap%
\pgfsetroundjoin%
\definecolor{currentfill}{rgb}{0.121569,0.466667,0.705882}%
\pgfsetfillcolor{currentfill}%
\pgfsetfillopacity{0.334386}%
\pgfsetlinewidth{1.003750pt}%
\definecolor{currentstroke}{rgb}{0.121569,0.466667,0.705882}%
\pgfsetstrokecolor{currentstroke}%
\pgfsetstrokeopacity{0.334386}%
\pgfsetdash{}{0pt}%
\pgfpathmoveto{\pgfqpoint{1.593683in}{2.065847in}}%
\pgfpathcurveto{\pgfqpoint{1.601919in}{2.065847in}}{\pgfqpoint{1.609819in}{2.069119in}}{\pgfqpoint{1.615643in}{2.074943in}}%
\pgfpathcurveto{\pgfqpoint{1.621467in}{2.080767in}}{\pgfqpoint{1.624739in}{2.088667in}}{\pgfqpoint{1.624739in}{2.096903in}}%
\pgfpathcurveto{\pgfqpoint{1.624739in}{2.105140in}}{\pgfqpoint{1.621467in}{2.113040in}}{\pgfqpoint{1.615643in}{2.118864in}}%
\pgfpathcurveto{\pgfqpoint{1.609819in}{2.124688in}}{\pgfqpoint{1.601919in}{2.127960in}}{\pgfqpoint{1.593683in}{2.127960in}}%
\pgfpathcurveto{\pgfqpoint{1.585446in}{2.127960in}}{\pgfqpoint{1.577546in}{2.124688in}}{\pgfqpoint{1.571722in}{2.118864in}}%
\pgfpathcurveto{\pgfqpoint{1.565898in}{2.113040in}}{\pgfqpoint{1.562626in}{2.105140in}}{\pgfqpoint{1.562626in}{2.096903in}}%
\pgfpathcurveto{\pgfqpoint{1.562626in}{2.088667in}}{\pgfqpoint{1.565898in}{2.080767in}}{\pgfqpoint{1.571722in}{2.074943in}}%
\pgfpathcurveto{\pgfqpoint{1.577546in}{2.069119in}}{\pgfqpoint{1.585446in}{2.065847in}}{\pgfqpoint{1.593683in}{2.065847in}}%
\pgfpathclose%
\pgfusepath{stroke,fill}%
\end{pgfscope}%
\begin{pgfscope}%
\pgfpathrectangle{\pgfqpoint{0.100000in}{0.212622in}}{\pgfqpoint{3.696000in}{3.696000in}}%
\pgfusepath{clip}%
\pgfsetbuttcap%
\pgfsetroundjoin%
\definecolor{currentfill}{rgb}{0.121569,0.466667,0.705882}%
\pgfsetfillcolor{currentfill}%
\pgfsetfillopacity{0.343882}%
\pgfsetlinewidth{1.003750pt}%
\definecolor{currentstroke}{rgb}{0.121569,0.466667,0.705882}%
\pgfsetstrokecolor{currentstroke}%
\pgfsetstrokeopacity{0.343882}%
\pgfsetdash{}{0pt}%
\pgfpathmoveto{\pgfqpoint{1.574016in}{2.049415in}}%
\pgfpathcurveto{\pgfqpoint{1.582252in}{2.049415in}}{\pgfqpoint{1.590152in}{2.052687in}}{\pgfqpoint{1.595976in}{2.058511in}}%
\pgfpathcurveto{\pgfqpoint{1.601800in}{2.064335in}}{\pgfqpoint{1.605072in}{2.072235in}}{\pgfqpoint{1.605072in}{2.080472in}}%
\pgfpathcurveto{\pgfqpoint{1.605072in}{2.088708in}}{\pgfqpoint{1.601800in}{2.096608in}}{\pgfqpoint{1.595976in}{2.102432in}}%
\pgfpathcurveto{\pgfqpoint{1.590152in}{2.108256in}}{\pgfqpoint{1.582252in}{2.111528in}}{\pgfqpoint{1.574016in}{2.111528in}}%
\pgfpathcurveto{\pgfqpoint{1.565779in}{2.111528in}}{\pgfqpoint{1.557879in}{2.108256in}}{\pgfqpoint{1.552055in}{2.102432in}}%
\pgfpathcurveto{\pgfqpoint{1.546231in}{2.096608in}}{\pgfqpoint{1.542959in}{2.088708in}}{\pgfqpoint{1.542959in}{2.080472in}}%
\pgfpathcurveto{\pgfqpoint{1.542959in}{2.072235in}}{\pgfqpoint{1.546231in}{2.064335in}}{\pgfqpoint{1.552055in}{2.058511in}}%
\pgfpathcurveto{\pgfqpoint{1.557879in}{2.052687in}}{\pgfqpoint{1.565779in}{2.049415in}}{\pgfqpoint{1.574016in}{2.049415in}}%
\pgfpathclose%
\pgfusepath{stroke,fill}%
\end{pgfscope}%
\begin{pgfscope}%
\pgfpathrectangle{\pgfqpoint{0.100000in}{0.212622in}}{\pgfqpoint{3.696000in}{3.696000in}}%
\pgfusepath{clip}%
\pgfsetbuttcap%
\pgfsetroundjoin%
\definecolor{currentfill}{rgb}{0.121569,0.466667,0.705882}%
\pgfsetfillcolor{currentfill}%
\pgfsetfillopacity{0.347858}%
\pgfsetlinewidth{1.003750pt}%
\definecolor{currentstroke}{rgb}{0.121569,0.466667,0.705882}%
\pgfsetstrokecolor{currentstroke}%
\pgfsetstrokeopacity{0.347858}%
\pgfsetdash{}{0pt}%
\pgfpathmoveto{\pgfqpoint{1.566120in}{2.042151in}}%
\pgfpathcurveto{\pgfqpoint{1.574356in}{2.042151in}}{\pgfqpoint{1.582256in}{2.045423in}}{\pgfqpoint{1.588080in}{2.051247in}}%
\pgfpathcurveto{\pgfqpoint{1.593904in}{2.057071in}}{\pgfqpoint{1.597176in}{2.064971in}}{\pgfqpoint{1.597176in}{2.073207in}}%
\pgfpathcurveto{\pgfqpoint{1.597176in}{2.081444in}}{\pgfqpoint{1.593904in}{2.089344in}}{\pgfqpoint{1.588080in}{2.095168in}}%
\pgfpathcurveto{\pgfqpoint{1.582256in}{2.100992in}}{\pgfqpoint{1.574356in}{2.104264in}}{\pgfqpoint{1.566120in}{2.104264in}}%
\pgfpathcurveto{\pgfqpoint{1.557883in}{2.104264in}}{\pgfqpoint{1.549983in}{2.100992in}}{\pgfqpoint{1.544159in}{2.095168in}}%
\pgfpathcurveto{\pgfqpoint{1.538336in}{2.089344in}}{\pgfqpoint{1.535063in}{2.081444in}}{\pgfqpoint{1.535063in}{2.073207in}}%
\pgfpathcurveto{\pgfqpoint{1.535063in}{2.064971in}}{\pgfqpoint{1.538336in}{2.057071in}}{\pgfqpoint{1.544159in}{2.051247in}}%
\pgfpathcurveto{\pgfqpoint{1.549983in}{2.045423in}}{\pgfqpoint{1.557883in}{2.042151in}}{\pgfqpoint{1.566120in}{2.042151in}}%
\pgfpathclose%
\pgfusepath{stroke,fill}%
\end{pgfscope}%
\begin{pgfscope}%
\pgfpathrectangle{\pgfqpoint{0.100000in}{0.212622in}}{\pgfqpoint{3.696000in}{3.696000in}}%
\pgfusepath{clip}%
\pgfsetbuttcap%
\pgfsetroundjoin%
\definecolor{currentfill}{rgb}{0.121569,0.466667,0.705882}%
\pgfsetfillcolor{currentfill}%
\pgfsetfillopacity{0.356925}%
\pgfsetlinewidth{1.003750pt}%
\definecolor{currentstroke}{rgb}{0.121569,0.466667,0.705882}%
\pgfsetstrokecolor{currentstroke}%
\pgfsetstrokeopacity{0.356925}%
\pgfsetdash{}{0pt}%
\pgfpathmoveto{\pgfqpoint{1.550367in}{2.030823in}}%
\pgfpathcurveto{\pgfqpoint{1.558603in}{2.030823in}}{\pgfqpoint{1.566503in}{2.034095in}}{\pgfqpoint{1.572327in}{2.039919in}}%
\pgfpathcurveto{\pgfqpoint{1.578151in}{2.045743in}}{\pgfqpoint{1.581423in}{2.053643in}}{\pgfqpoint{1.581423in}{2.061879in}}%
\pgfpathcurveto{\pgfqpoint{1.581423in}{2.070116in}}{\pgfqpoint{1.578151in}{2.078016in}}{\pgfqpoint{1.572327in}{2.083840in}}%
\pgfpathcurveto{\pgfqpoint{1.566503in}{2.089664in}}{\pgfqpoint{1.558603in}{2.092936in}}{\pgfqpoint{1.550367in}{2.092936in}}%
\pgfpathcurveto{\pgfqpoint{1.542131in}{2.092936in}}{\pgfqpoint{1.534231in}{2.089664in}}{\pgfqpoint{1.528407in}{2.083840in}}%
\pgfpathcurveto{\pgfqpoint{1.522583in}{2.078016in}}{\pgfqpoint{1.519310in}{2.070116in}}{\pgfqpoint{1.519310in}{2.061879in}}%
\pgfpathcurveto{\pgfqpoint{1.519310in}{2.053643in}}{\pgfqpoint{1.522583in}{2.045743in}}{\pgfqpoint{1.528407in}{2.039919in}}%
\pgfpathcurveto{\pgfqpoint{1.534231in}{2.034095in}}{\pgfqpoint{1.542131in}{2.030823in}}{\pgfqpoint{1.550367in}{2.030823in}}%
\pgfpathclose%
\pgfusepath{stroke,fill}%
\end{pgfscope}%
\begin{pgfscope}%
\pgfpathrectangle{\pgfqpoint{0.100000in}{0.212622in}}{\pgfqpoint{3.696000in}{3.696000in}}%
\pgfusepath{clip}%
\pgfsetbuttcap%
\pgfsetroundjoin%
\definecolor{currentfill}{rgb}{0.121569,0.466667,0.705882}%
\pgfsetfillcolor{currentfill}%
\pgfsetfillopacity{0.371656}%
\pgfsetlinewidth{1.003750pt}%
\definecolor{currentstroke}{rgb}{0.121569,0.466667,0.705882}%
\pgfsetstrokecolor{currentstroke}%
\pgfsetstrokeopacity{0.371656}%
\pgfsetdash{}{0pt}%
\pgfpathmoveto{\pgfqpoint{1.517928in}{1.996131in}}%
\pgfpathcurveto{\pgfqpoint{1.526164in}{1.996131in}}{\pgfqpoint{1.534064in}{1.999404in}}{\pgfqpoint{1.539888in}{2.005228in}}%
\pgfpathcurveto{\pgfqpoint{1.545712in}{2.011051in}}{\pgfqpoint{1.548984in}{2.018952in}}{\pgfqpoint{1.548984in}{2.027188in}}%
\pgfpathcurveto{\pgfqpoint{1.548984in}{2.035424in}}{\pgfqpoint{1.545712in}{2.043324in}}{\pgfqpoint{1.539888in}{2.049148in}}%
\pgfpathcurveto{\pgfqpoint{1.534064in}{2.054972in}}{\pgfqpoint{1.526164in}{2.058244in}}{\pgfqpoint{1.517928in}{2.058244in}}%
\pgfpathcurveto{\pgfqpoint{1.509691in}{2.058244in}}{\pgfqpoint{1.501791in}{2.054972in}}{\pgfqpoint{1.495967in}{2.049148in}}%
\pgfpathcurveto{\pgfqpoint{1.490143in}{2.043324in}}{\pgfqpoint{1.486871in}{2.035424in}}{\pgfqpoint{1.486871in}{2.027188in}}%
\pgfpathcurveto{\pgfqpoint{1.486871in}{2.018952in}}{\pgfqpoint{1.490143in}{2.011051in}}{\pgfqpoint{1.495967in}{2.005228in}}%
\pgfpathcurveto{\pgfqpoint{1.501791in}{1.999404in}}{\pgfqpoint{1.509691in}{1.996131in}}{\pgfqpoint{1.517928in}{1.996131in}}%
\pgfpathclose%
\pgfusepath{stroke,fill}%
\end{pgfscope}%
\begin{pgfscope}%
\pgfpathrectangle{\pgfqpoint{0.100000in}{0.212622in}}{\pgfqpoint{3.696000in}{3.696000in}}%
\pgfusepath{clip}%
\pgfsetbuttcap%
\pgfsetroundjoin%
\definecolor{currentfill}{rgb}{0.121569,0.466667,0.705882}%
\pgfsetfillcolor{currentfill}%
\pgfsetfillopacity{0.371903}%
\pgfsetlinewidth{1.003750pt}%
\definecolor{currentstroke}{rgb}{0.121569,0.466667,0.705882}%
\pgfsetstrokecolor{currentstroke}%
\pgfsetstrokeopacity{0.371903}%
\pgfsetdash{}{0pt}%
\pgfpathmoveto{\pgfqpoint{1.522284in}{2.001698in}}%
\pgfpathcurveto{\pgfqpoint{1.530520in}{2.001698in}}{\pgfqpoint{1.538420in}{2.004970in}}{\pgfqpoint{1.544244in}{2.010794in}}%
\pgfpathcurveto{\pgfqpoint{1.550068in}{2.016618in}}{\pgfqpoint{1.553340in}{2.024518in}}{\pgfqpoint{1.553340in}{2.032755in}}%
\pgfpathcurveto{\pgfqpoint{1.553340in}{2.040991in}}{\pgfqpoint{1.550068in}{2.048891in}}{\pgfqpoint{1.544244in}{2.054715in}}%
\pgfpathcurveto{\pgfqpoint{1.538420in}{2.060539in}}{\pgfqpoint{1.530520in}{2.063811in}}{\pgfqpoint{1.522284in}{2.063811in}}%
\pgfpathcurveto{\pgfqpoint{1.514047in}{2.063811in}}{\pgfqpoint{1.506147in}{2.060539in}}{\pgfqpoint{1.500323in}{2.054715in}}%
\pgfpathcurveto{\pgfqpoint{1.494499in}{2.048891in}}{\pgfqpoint{1.491227in}{2.040991in}}{\pgfqpoint{1.491227in}{2.032755in}}%
\pgfpathcurveto{\pgfqpoint{1.491227in}{2.024518in}}{\pgfqpoint{1.494499in}{2.016618in}}{\pgfqpoint{1.500323in}{2.010794in}}%
\pgfpathcurveto{\pgfqpoint{1.506147in}{2.004970in}}{\pgfqpoint{1.514047in}{2.001698in}}{\pgfqpoint{1.522284in}{2.001698in}}%
\pgfpathclose%
\pgfusepath{stroke,fill}%
\end{pgfscope}%
\begin{pgfscope}%
\pgfpathrectangle{\pgfqpoint{0.100000in}{0.212622in}}{\pgfqpoint{3.696000in}{3.696000in}}%
\pgfusepath{clip}%
\pgfsetbuttcap%
\pgfsetroundjoin%
\definecolor{currentfill}{rgb}{0.121569,0.466667,0.705882}%
\pgfsetfillcolor{currentfill}%
\pgfsetfillopacity{0.375179}%
\pgfsetlinewidth{1.003750pt}%
\definecolor{currentstroke}{rgb}{0.121569,0.466667,0.705882}%
\pgfsetstrokecolor{currentstroke}%
\pgfsetstrokeopacity{0.375179}%
\pgfsetdash{}{0pt}%
\pgfpathmoveto{\pgfqpoint{1.515546in}{1.993958in}}%
\pgfpathcurveto{\pgfqpoint{1.523782in}{1.993958in}}{\pgfqpoint{1.531682in}{1.997231in}}{\pgfqpoint{1.537506in}{2.003055in}}%
\pgfpathcurveto{\pgfqpoint{1.543330in}{2.008879in}}{\pgfqpoint{1.546603in}{2.016779in}}{\pgfqpoint{1.546603in}{2.025015in}}%
\pgfpathcurveto{\pgfqpoint{1.546603in}{2.033251in}}{\pgfqpoint{1.543330in}{2.041151in}}{\pgfqpoint{1.537506in}{2.046975in}}%
\pgfpathcurveto{\pgfqpoint{1.531682in}{2.052799in}}{\pgfqpoint{1.523782in}{2.056071in}}{\pgfqpoint{1.515546in}{2.056071in}}%
\pgfpathcurveto{\pgfqpoint{1.507310in}{2.056071in}}{\pgfqpoint{1.499410in}{2.052799in}}{\pgfqpoint{1.493586in}{2.046975in}}%
\pgfpathcurveto{\pgfqpoint{1.487762in}{2.041151in}}{\pgfqpoint{1.484490in}{2.033251in}}{\pgfqpoint{1.484490in}{2.025015in}}%
\pgfpathcurveto{\pgfqpoint{1.484490in}{2.016779in}}{\pgfqpoint{1.487762in}{2.008879in}}{\pgfqpoint{1.493586in}{2.003055in}}%
\pgfpathcurveto{\pgfqpoint{1.499410in}{1.997231in}}{\pgfqpoint{1.507310in}{1.993958in}}{\pgfqpoint{1.515546in}{1.993958in}}%
\pgfpathclose%
\pgfusepath{stroke,fill}%
\end{pgfscope}%
\begin{pgfscope}%
\pgfpathrectangle{\pgfqpoint{0.100000in}{0.212622in}}{\pgfqpoint{3.696000in}{3.696000in}}%
\pgfusepath{clip}%
\pgfsetbuttcap%
\pgfsetroundjoin%
\definecolor{currentfill}{rgb}{0.121569,0.466667,0.705882}%
\pgfsetfillcolor{currentfill}%
\pgfsetfillopacity{0.378836}%
\pgfsetlinewidth{1.003750pt}%
\definecolor{currentstroke}{rgb}{0.121569,0.466667,0.705882}%
\pgfsetstrokecolor{currentstroke}%
\pgfsetstrokeopacity{0.378836}%
\pgfsetdash{}{0pt}%
\pgfpathmoveto{\pgfqpoint{1.508887in}{1.986586in}}%
\pgfpathcurveto{\pgfqpoint{1.517124in}{1.986586in}}{\pgfqpoint{1.525024in}{1.989858in}}{\pgfqpoint{1.530848in}{1.995682in}}%
\pgfpathcurveto{\pgfqpoint{1.536672in}{2.001506in}}{\pgfqpoint{1.539944in}{2.009406in}}{\pgfqpoint{1.539944in}{2.017643in}}%
\pgfpathcurveto{\pgfqpoint{1.539944in}{2.025879in}}{\pgfqpoint{1.536672in}{2.033779in}}{\pgfqpoint{1.530848in}{2.039603in}}%
\pgfpathcurveto{\pgfqpoint{1.525024in}{2.045427in}}{\pgfqpoint{1.517124in}{2.048699in}}{\pgfqpoint{1.508887in}{2.048699in}}%
\pgfpathcurveto{\pgfqpoint{1.500651in}{2.048699in}}{\pgfqpoint{1.492751in}{2.045427in}}{\pgfqpoint{1.486927in}{2.039603in}}%
\pgfpathcurveto{\pgfqpoint{1.481103in}{2.033779in}}{\pgfqpoint{1.477831in}{2.025879in}}{\pgfqpoint{1.477831in}{2.017643in}}%
\pgfpathcurveto{\pgfqpoint{1.477831in}{2.009406in}}{\pgfqpoint{1.481103in}{2.001506in}}{\pgfqpoint{1.486927in}{1.995682in}}%
\pgfpathcurveto{\pgfqpoint{1.492751in}{1.989858in}}{\pgfqpoint{1.500651in}{1.986586in}}{\pgfqpoint{1.508887in}{1.986586in}}%
\pgfpathclose%
\pgfusepath{stroke,fill}%
\end{pgfscope}%
\begin{pgfscope}%
\pgfpathrectangle{\pgfqpoint{0.100000in}{0.212622in}}{\pgfqpoint{3.696000in}{3.696000in}}%
\pgfusepath{clip}%
\pgfsetbuttcap%
\pgfsetroundjoin%
\definecolor{currentfill}{rgb}{0.121569,0.466667,0.705882}%
\pgfsetfillcolor{currentfill}%
\pgfsetfillopacity{0.383772}%
\pgfsetlinewidth{1.003750pt}%
\definecolor{currentstroke}{rgb}{0.121569,0.466667,0.705882}%
\pgfsetstrokecolor{currentstroke}%
\pgfsetstrokeopacity{0.383772}%
\pgfsetdash{}{0pt}%
\pgfpathmoveto{\pgfqpoint{1.498329in}{1.978121in}}%
\pgfpathcurveto{\pgfqpoint{1.506566in}{1.978121in}}{\pgfqpoint{1.514466in}{1.981393in}}{\pgfqpoint{1.520290in}{1.987217in}}%
\pgfpathcurveto{\pgfqpoint{1.526114in}{1.993041in}}{\pgfqpoint{1.529386in}{2.000941in}}{\pgfqpoint{1.529386in}{2.009178in}}%
\pgfpathcurveto{\pgfqpoint{1.529386in}{2.017414in}}{\pgfqpoint{1.526114in}{2.025314in}}{\pgfqpoint{1.520290in}{2.031138in}}%
\pgfpathcurveto{\pgfqpoint{1.514466in}{2.036962in}}{\pgfqpoint{1.506566in}{2.040234in}}{\pgfqpoint{1.498329in}{2.040234in}}%
\pgfpathcurveto{\pgfqpoint{1.490093in}{2.040234in}}{\pgfqpoint{1.482193in}{2.036962in}}{\pgfqpoint{1.476369in}{2.031138in}}%
\pgfpathcurveto{\pgfqpoint{1.470545in}{2.025314in}}{\pgfqpoint{1.467273in}{2.017414in}}{\pgfqpoint{1.467273in}{2.009178in}}%
\pgfpathcurveto{\pgfqpoint{1.467273in}{2.000941in}}{\pgfqpoint{1.470545in}{1.993041in}}{\pgfqpoint{1.476369in}{1.987217in}}%
\pgfpathcurveto{\pgfqpoint{1.482193in}{1.981393in}}{\pgfqpoint{1.490093in}{1.978121in}}{\pgfqpoint{1.498329in}{1.978121in}}%
\pgfpathclose%
\pgfusepath{stroke,fill}%
\end{pgfscope}%
\begin{pgfscope}%
\pgfpathrectangle{\pgfqpoint{0.100000in}{0.212622in}}{\pgfqpoint{3.696000in}{3.696000in}}%
\pgfusepath{clip}%
\pgfsetbuttcap%
\pgfsetroundjoin%
\definecolor{currentfill}{rgb}{0.121569,0.466667,0.705882}%
\pgfsetfillcolor{currentfill}%
\pgfsetfillopacity{0.386239}%
\pgfsetlinewidth{1.003750pt}%
\definecolor{currentstroke}{rgb}{0.121569,0.466667,0.705882}%
\pgfsetstrokecolor{currentstroke}%
\pgfsetstrokeopacity{0.386239}%
\pgfsetdash{}{0pt}%
\pgfpathmoveto{\pgfqpoint{1.492793in}{1.972757in}}%
\pgfpathcurveto{\pgfqpoint{1.501029in}{1.972757in}}{\pgfqpoint{1.508929in}{1.976029in}}{\pgfqpoint{1.514753in}{1.981853in}}%
\pgfpathcurveto{\pgfqpoint{1.520577in}{1.987677in}}{\pgfqpoint{1.523849in}{1.995577in}}{\pgfqpoint{1.523849in}{2.003813in}}%
\pgfpathcurveto{\pgfqpoint{1.523849in}{2.012049in}}{\pgfqpoint{1.520577in}{2.019949in}}{\pgfqpoint{1.514753in}{2.025773in}}%
\pgfpathcurveto{\pgfqpoint{1.508929in}{2.031597in}}{\pgfqpoint{1.501029in}{2.034870in}}{\pgfqpoint{1.492793in}{2.034870in}}%
\pgfpathcurveto{\pgfqpoint{1.484557in}{2.034870in}}{\pgfqpoint{1.476657in}{2.031597in}}{\pgfqpoint{1.470833in}{2.025773in}}%
\pgfpathcurveto{\pgfqpoint{1.465009in}{2.019949in}}{\pgfqpoint{1.461736in}{2.012049in}}{\pgfqpoint{1.461736in}{2.003813in}}%
\pgfpathcurveto{\pgfqpoint{1.461736in}{1.995577in}}{\pgfqpoint{1.465009in}{1.987677in}}{\pgfqpoint{1.470833in}{1.981853in}}%
\pgfpathcurveto{\pgfqpoint{1.476657in}{1.976029in}}{\pgfqpoint{1.484557in}{1.972757in}}{\pgfqpoint{1.492793in}{1.972757in}}%
\pgfpathclose%
\pgfusepath{stroke,fill}%
\end{pgfscope}%
\begin{pgfscope}%
\pgfpathrectangle{\pgfqpoint{0.100000in}{0.212622in}}{\pgfqpoint{3.696000in}{3.696000in}}%
\pgfusepath{clip}%
\pgfsetbuttcap%
\pgfsetroundjoin%
\definecolor{currentfill}{rgb}{0.121569,0.466667,0.705882}%
\pgfsetfillcolor{currentfill}%
\pgfsetfillopacity{0.386394}%
\pgfsetlinewidth{1.003750pt}%
\definecolor{currentstroke}{rgb}{0.121569,0.466667,0.705882}%
\pgfsetstrokecolor{currentstroke}%
\pgfsetstrokeopacity{0.386394}%
\pgfsetdash{}{0pt}%
\pgfpathmoveto{\pgfqpoint{1.493099in}{1.973414in}}%
\pgfpathcurveto{\pgfqpoint{1.501335in}{1.973414in}}{\pgfqpoint{1.509235in}{1.976686in}}{\pgfqpoint{1.515059in}{1.982510in}}%
\pgfpathcurveto{\pgfqpoint{1.520883in}{1.988334in}}{\pgfqpoint{1.524155in}{1.996234in}}{\pgfqpoint{1.524155in}{2.004470in}}%
\pgfpathcurveto{\pgfqpoint{1.524155in}{2.012707in}}{\pgfqpoint{1.520883in}{2.020607in}}{\pgfqpoint{1.515059in}{2.026431in}}%
\pgfpathcurveto{\pgfqpoint{1.509235in}{2.032255in}}{\pgfqpoint{1.501335in}{2.035527in}}{\pgfqpoint{1.493099in}{2.035527in}}%
\pgfpathcurveto{\pgfqpoint{1.484863in}{2.035527in}}{\pgfqpoint{1.476963in}{2.032255in}}{\pgfqpoint{1.471139in}{2.026431in}}%
\pgfpathcurveto{\pgfqpoint{1.465315in}{2.020607in}}{\pgfqpoint{1.462042in}{2.012707in}}{\pgfqpoint{1.462042in}{2.004470in}}%
\pgfpathcurveto{\pgfqpoint{1.462042in}{1.996234in}}{\pgfqpoint{1.465315in}{1.988334in}}{\pgfqpoint{1.471139in}{1.982510in}}%
\pgfpathcurveto{\pgfqpoint{1.476963in}{1.976686in}}{\pgfqpoint{1.484863in}{1.973414in}}{\pgfqpoint{1.493099in}{1.973414in}}%
\pgfpathclose%
\pgfusepath{stroke,fill}%
\end{pgfscope}%
\begin{pgfscope}%
\pgfpathrectangle{\pgfqpoint{0.100000in}{0.212622in}}{\pgfqpoint{3.696000in}{3.696000in}}%
\pgfusepath{clip}%
\pgfsetbuttcap%
\pgfsetroundjoin%
\definecolor{currentfill}{rgb}{0.121569,0.466667,0.705882}%
\pgfsetfillcolor{currentfill}%
\pgfsetfillopacity{0.390162}%
\pgfsetlinewidth{1.003750pt}%
\definecolor{currentstroke}{rgb}{0.121569,0.466667,0.705882}%
\pgfsetstrokecolor{currentstroke}%
\pgfsetstrokeopacity{0.390162}%
\pgfsetdash{}{0pt}%
\pgfpathmoveto{\pgfqpoint{1.484773in}{1.965342in}}%
\pgfpathcurveto{\pgfqpoint{1.493009in}{1.965342in}}{\pgfqpoint{1.500909in}{1.968614in}}{\pgfqpoint{1.506733in}{1.974438in}}%
\pgfpathcurveto{\pgfqpoint{1.512557in}{1.980262in}}{\pgfqpoint{1.515829in}{1.988162in}}{\pgfqpoint{1.515829in}{1.996398in}}%
\pgfpathcurveto{\pgfqpoint{1.515829in}{2.004634in}}{\pgfqpoint{1.512557in}{2.012534in}}{\pgfqpoint{1.506733in}{2.018358in}}%
\pgfpathcurveto{\pgfqpoint{1.500909in}{2.024182in}}{\pgfqpoint{1.493009in}{2.027455in}}{\pgfqpoint{1.484773in}{2.027455in}}%
\pgfpathcurveto{\pgfqpoint{1.476536in}{2.027455in}}{\pgfqpoint{1.468636in}{2.024182in}}{\pgfqpoint{1.462812in}{2.018358in}}%
\pgfpathcurveto{\pgfqpoint{1.456988in}{2.012534in}}{\pgfqpoint{1.453716in}{2.004634in}}{\pgfqpoint{1.453716in}{1.996398in}}%
\pgfpathcurveto{\pgfqpoint{1.453716in}{1.988162in}}{\pgfqpoint{1.456988in}{1.980262in}}{\pgfqpoint{1.462812in}{1.974438in}}%
\pgfpathcurveto{\pgfqpoint{1.468636in}{1.968614in}}{\pgfqpoint{1.476536in}{1.965342in}}{\pgfqpoint{1.484773in}{1.965342in}}%
\pgfpathclose%
\pgfusepath{stroke,fill}%
\end{pgfscope}%
\begin{pgfscope}%
\pgfpathrectangle{\pgfqpoint{0.100000in}{0.212622in}}{\pgfqpoint{3.696000in}{3.696000in}}%
\pgfusepath{clip}%
\pgfsetbuttcap%
\pgfsetroundjoin%
\definecolor{currentfill}{rgb}{0.121569,0.466667,0.705882}%
\pgfsetfillcolor{currentfill}%
\pgfsetfillopacity{0.391158}%
\pgfsetlinewidth{1.003750pt}%
\definecolor{currentstroke}{rgb}{0.121569,0.466667,0.705882}%
\pgfsetstrokecolor{currentstroke}%
\pgfsetstrokeopacity{0.391158}%
\pgfsetdash{}{0pt}%
\pgfpathmoveto{\pgfqpoint{1.483444in}{1.964985in}}%
\pgfpathcurveto{\pgfqpoint{1.491680in}{1.964985in}}{\pgfqpoint{1.499580in}{1.968257in}}{\pgfqpoint{1.505404in}{1.974081in}}%
\pgfpathcurveto{\pgfqpoint{1.511228in}{1.979905in}}{\pgfqpoint{1.514501in}{1.987805in}}{\pgfqpoint{1.514501in}{1.996041in}}%
\pgfpathcurveto{\pgfqpoint{1.514501in}{2.004278in}}{\pgfqpoint{1.511228in}{2.012178in}}{\pgfqpoint{1.505404in}{2.018002in}}%
\pgfpathcurveto{\pgfqpoint{1.499580in}{2.023826in}}{\pgfqpoint{1.491680in}{2.027098in}}{\pgfqpoint{1.483444in}{2.027098in}}%
\pgfpathcurveto{\pgfqpoint{1.475208in}{2.027098in}}{\pgfqpoint{1.467308in}{2.023826in}}{\pgfqpoint{1.461484in}{2.018002in}}%
\pgfpathcurveto{\pgfqpoint{1.455660in}{2.012178in}}{\pgfqpoint{1.452388in}{2.004278in}}{\pgfqpoint{1.452388in}{1.996041in}}%
\pgfpathcurveto{\pgfqpoint{1.452388in}{1.987805in}}{\pgfqpoint{1.455660in}{1.979905in}}{\pgfqpoint{1.461484in}{1.974081in}}%
\pgfpathcurveto{\pgfqpoint{1.467308in}{1.968257in}}{\pgfqpoint{1.475208in}{1.964985in}}{\pgfqpoint{1.483444in}{1.964985in}}%
\pgfpathclose%
\pgfusepath{stroke,fill}%
\end{pgfscope}%
\begin{pgfscope}%
\pgfpathrectangle{\pgfqpoint{0.100000in}{0.212622in}}{\pgfqpoint{3.696000in}{3.696000in}}%
\pgfusepath{clip}%
\pgfsetbuttcap%
\pgfsetroundjoin%
\definecolor{currentfill}{rgb}{0.121569,0.466667,0.705882}%
\pgfsetfillcolor{currentfill}%
\pgfsetfillopacity{0.391747}%
\pgfsetlinewidth{1.003750pt}%
\definecolor{currentstroke}{rgb}{0.121569,0.466667,0.705882}%
\pgfsetstrokecolor{currentstroke}%
\pgfsetstrokeopacity{0.391747}%
\pgfsetdash{}{0pt}%
\pgfpathmoveto{\pgfqpoint{1.482455in}{1.964169in}}%
\pgfpathcurveto{\pgfqpoint{1.490691in}{1.964169in}}{\pgfqpoint{1.498591in}{1.967441in}}{\pgfqpoint{1.504415in}{1.973265in}}%
\pgfpathcurveto{\pgfqpoint{1.510239in}{1.979089in}}{\pgfqpoint{1.513512in}{1.986989in}}{\pgfqpoint{1.513512in}{1.995226in}}%
\pgfpathcurveto{\pgfqpoint{1.513512in}{2.003462in}}{\pgfqpoint{1.510239in}{2.011362in}}{\pgfqpoint{1.504415in}{2.017186in}}%
\pgfpathcurveto{\pgfqpoint{1.498591in}{2.023010in}}{\pgfqpoint{1.490691in}{2.026282in}}{\pgfqpoint{1.482455in}{2.026282in}}%
\pgfpathcurveto{\pgfqpoint{1.474219in}{2.026282in}}{\pgfqpoint{1.466319in}{2.023010in}}{\pgfqpoint{1.460495in}{2.017186in}}%
\pgfpathcurveto{\pgfqpoint{1.454671in}{2.011362in}}{\pgfqpoint{1.451399in}{2.003462in}}{\pgfqpoint{1.451399in}{1.995226in}}%
\pgfpathcurveto{\pgfqpoint{1.451399in}{1.986989in}}{\pgfqpoint{1.454671in}{1.979089in}}{\pgfqpoint{1.460495in}{1.973265in}}%
\pgfpathcurveto{\pgfqpoint{1.466319in}{1.967441in}}{\pgfqpoint{1.474219in}{1.964169in}}{\pgfqpoint{1.482455in}{1.964169in}}%
\pgfpathclose%
\pgfusepath{stroke,fill}%
\end{pgfscope}%
\begin{pgfscope}%
\pgfpathrectangle{\pgfqpoint{0.100000in}{0.212622in}}{\pgfqpoint{3.696000in}{3.696000in}}%
\pgfusepath{clip}%
\pgfsetbuttcap%
\pgfsetroundjoin%
\definecolor{currentfill}{rgb}{0.121569,0.466667,0.705882}%
\pgfsetfillcolor{currentfill}%
\pgfsetfillopacity{0.392937}%
\pgfsetlinewidth{1.003750pt}%
\definecolor{currentstroke}{rgb}{0.121569,0.466667,0.705882}%
\pgfsetstrokecolor{currentstroke}%
\pgfsetstrokeopacity{0.392937}%
\pgfsetdash{}{0pt}%
\pgfpathmoveto{\pgfqpoint{1.480729in}{1.962016in}}%
\pgfpathcurveto{\pgfqpoint{1.488965in}{1.962016in}}{\pgfqpoint{1.496865in}{1.965288in}}{\pgfqpoint{1.502689in}{1.971112in}}%
\pgfpathcurveto{\pgfqpoint{1.508513in}{1.976936in}}{\pgfqpoint{1.511785in}{1.984836in}}{\pgfqpoint{1.511785in}{1.993072in}}%
\pgfpathcurveto{\pgfqpoint{1.511785in}{2.001308in}}{\pgfqpoint{1.508513in}{2.009209in}}{\pgfqpoint{1.502689in}{2.015032in}}%
\pgfpathcurveto{\pgfqpoint{1.496865in}{2.020856in}}{\pgfqpoint{1.488965in}{2.024129in}}{\pgfqpoint{1.480729in}{2.024129in}}%
\pgfpathcurveto{\pgfqpoint{1.472492in}{2.024129in}}{\pgfqpoint{1.464592in}{2.020856in}}{\pgfqpoint{1.458768in}{2.015032in}}%
\pgfpathcurveto{\pgfqpoint{1.452944in}{2.009209in}}{\pgfqpoint{1.449672in}{2.001308in}}{\pgfqpoint{1.449672in}{1.993072in}}%
\pgfpathcurveto{\pgfqpoint{1.449672in}{1.984836in}}{\pgfqpoint{1.452944in}{1.976936in}}{\pgfqpoint{1.458768in}{1.971112in}}%
\pgfpathcurveto{\pgfqpoint{1.464592in}{1.965288in}}{\pgfqpoint{1.472492in}{1.962016in}}{\pgfqpoint{1.480729in}{1.962016in}}%
\pgfpathclose%
\pgfusepath{stroke,fill}%
\end{pgfscope}%
\begin{pgfscope}%
\pgfpathrectangle{\pgfqpoint{0.100000in}{0.212622in}}{\pgfqpoint{3.696000in}{3.696000in}}%
\pgfusepath{clip}%
\pgfsetbuttcap%
\pgfsetroundjoin%
\definecolor{currentfill}{rgb}{0.121569,0.466667,0.705882}%
\pgfsetfillcolor{currentfill}%
\pgfsetfillopacity{0.393120}%
\pgfsetlinewidth{1.003750pt}%
\definecolor{currentstroke}{rgb}{0.121569,0.466667,0.705882}%
\pgfsetstrokecolor{currentstroke}%
\pgfsetstrokeopacity{0.393120}%
\pgfsetdash{}{0pt}%
\pgfpathmoveto{\pgfqpoint{1.481402in}{1.963329in}}%
\pgfpathcurveto{\pgfqpoint{1.489638in}{1.963329in}}{\pgfqpoint{1.497538in}{1.966601in}}{\pgfqpoint{1.503362in}{1.972425in}}%
\pgfpathcurveto{\pgfqpoint{1.509186in}{1.978249in}}{\pgfqpoint{1.512458in}{1.986149in}}{\pgfqpoint{1.512458in}{1.994386in}}%
\pgfpathcurveto{\pgfqpoint{1.512458in}{2.002622in}}{\pgfqpoint{1.509186in}{2.010522in}}{\pgfqpoint{1.503362in}{2.016346in}}%
\pgfpathcurveto{\pgfqpoint{1.497538in}{2.022170in}}{\pgfqpoint{1.489638in}{2.025442in}}{\pgfqpoint{1.481402in}{2.025442in}}%
\pgfpathcurveto{\pgfqpoint{1.473165in}{2.025442in}}{\pgfqpoint{1.465265in}{2.022170in}}{\pgfqpoint{1.459441in}{2.016346in}}%
\pgfpathcurveto{\pgfqpoint{1.453617in}{2.010522in}}{\pgfqpoint{1.450345in}{2.002622in}}{\pgfqpoint{1.450345in}{1.994386in}}%
\pgfpathcurveto{\pgfqpoint{1.450345in}{1.986149in}}{\pgfqpoint{1.453617in}{1.978249in}}{\pgfqpoint{1.459441in}{1.972425in}}%
\pgfpathcurveto{\pgfqpoint{1.465265in}{1.966601in}}{\pgfqpoint{1.473165in}{1.963329in}}{\pgfqpoint{1.481402in}{1.963329in}}%
\pgfpathclose%
\pgfusepath{stroke,fill}%
\end{pgfscope}%
\begin{pgfscope}%
\pgfpathrectangle{\pgfqpoint{0.100000in}{0.212622in}}{\pgfqpoint{3.696000in}{3.696000in}}%
\pgfusepath{clip}%
\pgfsetbuttcap%
\pgfsetroundjoin%
\definecolor{currentfill}{rgb}{0.121569,0.466667,0.705882}%
\pgfsetfillcolor{currentfill}%
\pgfsetfillopacity{0.393705}%
\pgfsetlinewidth{1.003750pt}%
\definecolor{currentstroke}{rgb}{0.121569,0.466667,0.705882}%
\pgfsetstrokecolor{currentstroke}%
\pgfsetstrokeopacity{0.393705}%
\pgfsetdash{}{0pt}%
\pgfpathmoveto{\pgfqpoint{1.480218in}{1.961888in}}%
\pgfpathcurveto{\pgfqpoint{1.488455in}{1.961888in}}{\pgfqpoint{1.496355in}{1.965160in}}{\pgfqpoint{1.502179in}{1.970984in}}%
\pgfpathcurveto{\pgfqpoint{1.508003in}{1.976808in}}{\pgfqpoint{1.511275in}{1.984708in}}{\pgfqpoint{1.511275in}{1.992944in}}%
\pgfpathcurveto{\pgfqpoint{1.511275in}{2.001180in}}{\pgfqpoint{1.508003in}{2.009080in}}{\pgfqpoint{1.502179in}{2.014904in}}%
\pgfpathcurveto{\pgfqpoint{1.496355in}{2.020728in}}{\pgfqpoint{1.488455in}{2.024001in}}{\pgfqpoint{1.480218in}{2.024001in}}%
\pgfpathcurveto{\pgfqpoint{1.471982in}{2.024001in}}{\pgfqpoint{1.464082in}{2.020728in}}{\pgfqpoint{1.458258in}{2.014904in}}%
\pgfpathcurveto{\pgfqpoint{1.452434in}{2.009080in}}{\pgfqpoint{1.449162in}{2.001180in}}{\pgfqpoint{1.449162in}{1.992944in}}%
\pgfpathcurveto{\pgfqpoint{1.449162in}{1.984708in}}{\pgfqpoint{1.452434in}{1.976808in}}{\pgfqpoint{1.458258in}{1.970984in}}%
\pgfpathcurveto{\pgfqpoint{1.464082in}{1.965160in}}{\pgfqpoint{1.471982in}{1.961888in}}{\pgfqpoint{1.480218in}{1.961888in}}%
\pgfpathclose%
\pgfusepath{stroke,fill}%
\end{pgfscope}%
\begin{pgfscope}%
\pgfpathrectangle{\pgfqpoint{0.100000in}{0.212622in}}{\pgfqpoint{3.696000in}{3.696000in}}%
\pgfusepath{clip}%
\pgfsetbuttcap%
\pgfsetroundjoin%
\definecolor{currentfill}{rgb}{0.121569,0.466667,0.705882}%
\pgfsetfillcolor{currentfill}%
\pgfsetfillopacity{0.393773}%
\pgfsetlinewidth{1.003750pt}%
\definecolor{currentstroke}{rgb}{0.121569,0.466667,0.705882}%
\pgfsetstrokecolor{currentstroke}%
\pgfsetstrokeopacity{0.393773}%
\pgfsetdash{}{0pt}%
\pgfpathmoveto{\pgfqpoint{1.480372in}{1.962193in}}%
\pgfpathcurveto{\pgfqpoint{1.488608in}{1.962193in}}{\pgfqpoint{1.496509in}{1.965466in}}{\pgfqpoint{1.502332in}{1.971289in}}%
\pgfpathcurveto{\pgfqpoint{1.508156in}{1.977113in}}{\pgfqpoint{1.511429in}{1.985013in}}{\pgfqpoint{1.511429in}{1.993250in}}%
\pgfpathcurveto{\pgfqpoint{1.511429in}{2.001486in}}{\pgfqpoint{1.508156in}{2.009386in}}{\pgfqpoint{1.502332in}{2.015210in}}%
\pgfpathcurveto{\pgfqpoint{1.496509in}{2.021034in}}{\pgfqpoint{1.488608in}{2.024306in}}{\pgfqpoint{1.480372in}{2.024306in}}%
\pgfpathcurveto{\pgfqpoint{1.472136in}{2.024306in}}{\pgfqpoint{1.464236in}{2.021034in}}{\pgfqpoint{1.458412in}{2.015210in}}%
\pgfpathcurveto{\pgfqpoint{1.452588in}{2.009386in}}{\pgfqpoint{1.449316in}{2.001486in}}{\pgfqpoint{1.449316in}{1.993250in}}%
\pgfpathcurveto{\pgfqpoint{1.449316in}{1.985013in}}{\pgfqpoint{1.452588in}{1.977113in}}{\pgfqpoint{1.458412in}{1.971289in}}%
\pgfpathcurveto{\pgfqpoint{1.464236in}{1.965466in}}{\pgfqpoint{1.472136in}{1.962193in}}{\pgfqpoint{1.480372in}{1.962193in}}%
\pgfpathclose%
\pgfusepath{stroke,fill}%
\end{pgfscope}%
\begin{pgfscope}%
\pgfpathrectangle{\pgfqpoint{0.100000in}{0.212622in}}{\pgfqpoint{3.696000in}{3.696000in}}%
\pgfusepath{clip}%
\pgfsetbuttcap%
\pgfsetroundjoin%
\definecolor{currentfill}{rgb}{0.121569,0.466667,0.705882}%
\pgfsetfillcolor{currentfill}%
\pgfsetfillopacity{0.400056}%
\pgfsetlinewidth{1.003750pt}%
\definecolor{currentstroke}{rgb}{0.121569,0.466667,0.705882}%
\pgfsetstrokecolor{currentstroke}%
\pgfsetstrokeopacity{0.400056}%
\pgfsetdash{}{0pt}%
\pgfpathmoveto{\pgfqpoint{1.474433in}{1.960947in}}%
\pgfpathcurveto{\pgfqpoint{1.482669in}{1.960947in}}{\pgfqpoint{1.490569in}{1.964220in}}{\pgfqpoint{1.496393in}{1.970043in}}%
\pgfpathcurveto{\pgfqpoint{1.502217in}{1.975867in}}{\pgfqpoint{1.505489in}{1.983767in}}{\pgfqpoint{1.505489in}{1.992004in}}%
\pgfpathcurveto{\pgfqpoint{1.505489in}{2.000240in}}{\pgfqpoint{1.502217in}{2.008140in}}{\pgfqpoint{1.496393in}{2.013964in}}%
\pgfpathcurveto{\pgfqpoint{1.490569in}{2.019788in}}{\pgfqpoint{1.482669in}{2.023060in}}{\pgfqpoint{1.474433in}{2.023060in}}%
\pgfpathcurveto{\pgfqpoint{1.466196in}{2.023060in}}{\pgfqpoint{1.458296in}{2.019788in}}{\pgfqpoint{1.452472in}{2.013964in}}%
\pgfpathcurveto{\pgfqpoint{1.446649in}{2.008140in}}{\pgfqpoint{1.443376in}{2.000240in}}{\pgfqpoint{1.443376in}{1.992004in}}%
\pgfpathcurveto{\pgfqpoint{1.443376in}{1.983767in}}{\pgfqpoint{1.446649in}{1.975867in}}{\pgfqpoint{1.452472in}{1.970043in}}%
\pgfpathcurveto{\pgfqpoint{1.458296in}{1.964220in}}{\pgfqpoint{1.466196in}{1.960947in}}{\pgfqpoint{1.474433in}{1.960947in}}%
\pgfpathclose%
\pgfusepath{stroke,fill}%
\end{pgfscope}%
\begin{pgfscope}%
\pgfpathrectangle{\pgfqpoint{0.100000in}{0.212622in}}{\pgfqpoint{3.696000in}{3.696000in}}%
\pgfusepath{clip}%
\pgfsetbuttcap%
\pgfsetroundjoin%
\definecolor{currentfill}{rgb}{0.121569,0.466667,0.705882}%
\pgfsetfillcolor{currentfill}%
\pgfsetfillopacity{0.401131}%
\pgfsetlinewidth{1.003750pt}%
\definecolor{currentstroke}{rgb}{0.121569,0.466667,0.705882}%
\pgfsetstrokecolor{currentstroke}%
\pgfsetstrokeopacity{0.401131}%
\pgfsetdash{}{0pt}%
\pgfpathmoveto{\pgfqpoint{1.464949in}{1.947601in}}%
\pgfpathcurveto{\pgfqpoint{1.473186in}{1.947601in}}{\pgfqpoint{1.481086in}{1.950873in}}{\pgfqpoint{1.486910in}{1.956697in}}%
\pgfpathcurveto{\pgfqpoint{1.492733in}{1.962521in}}{\pgfqpoint{1.496006in}{1.970421in}}{\pgfqpoint{1.496006in}{1.978657in}}%
\pgfpathcurveto{\pgfqpoint{1.496006in}{1.986893in}}{\pgfqpoint{1.492733in}{1.994793in}}{\pgfqpoint{1.486910in}{2.000617in}}%
\pgfpathcurveto{\pgfqpoint{1.481086in}{2.006441in}}{\pgfqpoint{1.473186in}{2.009714in}}{\pgfqpoint{1.464949in}{2.009714in}}%
\pgfpathcurveto{\pgfqpoint{1.456713in}{2.009714in}}{\pgfqpoint{1.448813in}{2.006441in}}{\pgfqpoint{1.442989in}{2.000617in}}%
\pgfpathcurveto{\pgfqpoint{1.437165in}{1.994793in}}{\pgfqpoint{1.433893in}{1.986893in}}{\pgfqpoint{1.433893in}{1.978657in}}%
\pgfpathcurveto{\pgfqpoint{1.433893in}{1.970421in}}{\pgfqpoint{1.437165in}{1.962521in}}{\pgfqpoint{1.442989in}{1.956697in}}%
\pgfpathcurveto{\pgfqpoint{1.448813in}{1.950873in}}{\pgfqpoint{1.456713in}{1.947601in}}{\pgfqpoint{1.464949in}{1.947601in}}%
\pgfpathclose%
\pgfusepath{stroke,fill}%
\end{pgfscope}%
\begin{pgfscope}%
\pgfpathrectangle{\pgfqpoint{0.100000in}{0.212622in}}{\pgfqpoint{3.696000in}{3.696000in}}%
\pgfusepath{clip}%
\pgfsetbuttcap%
\pgfsetroundjoin%
\definecolor{currentfill}{rgb}{0.121569,0.466667,0.705882}%
\pgfsetfillcolor{currentfill}%
\pgfsetfillopacity{0.404943}%
\pgfsetlinewidth{1.003750pt}%
\definecolor{currentstroke}{rgb}{0.121569,0.466667,0.705882}%
\pgfsetstrokecolor{currentstroke}%
\pgfsetstrokeopacity{0.404943}%
\pgfsetdash{}{0pt}%
\pgfpathmoveto{\pgfqpoint{1.469423in}{1.952945in}}%
\pgfpathcurveto{\pgfqpoint{1.477659in}{1.952945in}}{\pgfqpoint{1.485559in}{1.956217in}}{\pgfqpoint{1.491383in}{1.962041in}}%
\pgfpathcurveto{\pgfqpoint{1.497207in}{1.967865in}}{\pgfqpoint{1.500479in}{1.975765in}}{\pgfqpoint{1.500479in}{1.984001in}}%
\pgfpathcurveto{\pgfqpoint{1.500479in}{1.992238in}}{\pgfqpoint{1.497207in}{2.000138in}}{\pgfqpoint{1.491383in}{2.005962in}}%
\pgfpathcurveto{\pgfqpoint{1.485559in}{2.011785in}}{\pgfqpoint{1.477659in}{2.015058in}}{\pgfqpoint{1.469423in}{2.015058in}}%
\pgfpathcurveto{\pgfqpoint{1.461186in}{2.015058in}}{\pgfqpoint{1.453286in}{2.011785in}}{\pgfqpoint{1.447462in}{2.005962in}}%
\pgfpathcurveto{\pgfqpoint{1.441639in}{2.000138in}}{\pgfqpoint{1.438366in}{1.992238in}}{\pgfqpoint{1.438366in}{1.984001in}}%
\pgfpathcurveto{\pgfqpoint{1.438366in}{1.975765in}}{\pgfqpoint{1.441639in}{1.967865in}}{\pgfqpoint{1.447462in}{1.962041in}}%
\pgfpathcurveto{\pgfqpoint{1.453286in}{1.956217in}}{\pgfqpoint{1.461186in}{1.952945in}}{\pgfqpoint{1.469423in}{1.952945in}}%
\pgfpathclose%
\pgfusepath{stroke,fill}%
\end{pgfscope}%
\begin{pgfscope}%
\pgfpathrectangle{\pgfqpoint{0.100000in}{0.212622in}}{\pgfqpoint{3.696000in}{3.696000in}}%
\pgfusepath{clip}%
\pgfsetbuttcap%
\pgfsetroundjoin%
\definecolor{currentfill}{rgb}{0.121569,0.466667,0.705882}%
\pgfsetfillcolor{currentfill}%
\pgfsetfillopacity{0.405345}%
\pgfsetlinewidth{1.003750pt}%
\definecolor{currentstroke}{rgb}{0.121569,0.466667,0.705882}%
\pgfsetstrokecolor{currentstroke}%
\pgfsetstrokeopacity{0.405345}%
\pgfsetdash{}{0pt}%
\pgfpathmoveto{\pgfqpoint{1.465579in}{1.950438in}}%
\pgfpathcurveto{\pgfqpoint{1.473816in}{1.950438in}}{\pgfqpoint{1.481716in}{1.953710in}}{\pgfqpoint{1.487540in}{1.959534in}}%
\pgfpathcurveto{\pgfqpoint{1.493364in}{1.965358in}}{\pgfqpoint{1.496636in}{1.973258in}}{\pgfqpoint{1.496636in}{1.981494in}}%
\pgfpathcurveto{\pgfqpoint{1.496636in}{1.989731in}}{\pgfqpoint{1.493364in}{1.997631in}}{\pgfqpoint{1.487540in}{2.003455in}}%
\pgfpathcurveto{\pgfqpoint{1.481716in}{2.009278in}}{\pgfqpoint{1.473816in}{2.012551in}}{\pgfqpoint{1.465579in}{2.012551in}}%
\pgfpathcurveto{\pgfqpoint{1.457343in}{2.012551in}}{\pgfqpoint{1.449443in}{2.009278in}}{\pgfqpoint{1.443619in}{2.003455in}}%
\pgfpathcurveto{\pgfqpoint{1.437795in}{1.997631in}}{\pgfqpoint{1.434523in}{1.989731in}}{\pgfqpoint{1.434523in}{1.981494in}}%
\pgfpathcurveto{\pgfqpoint{1.434523in}{1.973258in}}{\pgfqpoint{1.437795in}{1.965358in}}{\pgfqpoint{1.443619in}{1.959534in}}%
\pgfpathcurveto{\pgfqpoint{1.449443in}{1.953710in}}{\pgfqpoint{1.457343in}{1.950438in}}{\pgfqpoint{1.465579in}{1.950438in}}%
\pgfpathclose%
\pgfusepath{stroke,fill}%
\end{pgfscope}%
\begin{pgfscope}%
\pgfpathrectangle{\pgfqpoint{0.100000in}{0.212622in}}{\pgfqpoint{3.696000in}{3.696000in}}%
\pgfusepath{clip}%
\pgfsetbuttcap%
\pgfsetroundjoin%
\definecolor{currentfill}{rgb}{0.121569,0.466667,0.705882}%
\pgfsetfillcolor{currentfill}%
\pgfsetfillopacity{0.409224}%
\pgfsetlinewidth{1.003750pt}%
\definecolor{currentstroke}{rgb}{0.121569,0.466667,0.705882}%
\pgfsetstrokecolor{currentstroke}%
\pgfsetstrokeopacity{0.409224}%
\pgfsetdash{}{0pt}%
\pgfpathmoveto{\pgfqpoint{1.462113in}{1.944829in}}%
\pgfpathcurveto{\pgfqpoint{1.470349in}{1.944829in}}{\pgfqpoint{1.478249in}{1.948101in}}{\pgfqpoint{1.484073in}{1.953925in}}%
\pgfpathcurveto{\pgfqpoint{1.489897in}{1.959749in}}{\pgfqpoint{1.493169in}{1.967649in}}{\pgfqpoint{1.493169in}{1.975886in}}%
\pgfpathcurveto{\pgfqpoint{1.493169in}{1.984122in}}{\pgfqpoint{1.489897in}{1.992022in}}{\pgfqpoint{1.484073in}{1.997846in}}%
\pgfpathcurveto{\pgfqpoint{1.478249in}{2.003670in}}{\pgfqpoint{1.470349in}{2.006942in}}{\pgfqpoint{1.462113in}{2.006942in}}%
\pgfpathcurveto{\pgfqpoint{1.453877in}{2.006942in}}{\pgfqpoint{1.445977in}{2.003670in}}{\pgfqpoint{1.440153in}{1.997846in}}%
\pgfpathcurveto{\pgfqpoint{1.434329in}{1.992022in}}{\pgfqpoint{1.431056in}{1.984122in}}{\pgfqpoint{1.431056in}{1.975886in}}%
\pgfpathcurveto{\pgfqpoint{1.431056in}{1.967649in}}{\pgfqpoint{1.434329in}{1.959749in}}{\pgfqpoint{1.440153in}{1.953925in}}%
\pgfpathcurveto{\pgfqpoint{1.445977in}{1.948101in}}{\pgfqpoint{1.453877in}{1.944829in}}{\pgfqpoint{1.462113in}{1.944829in}}%
\pgfpathclose%
\pgfusepath{stroke,fill}%
\end{pgfscope}%
\begin{pgfscope}%
\pgfpathrectangle{\pgfqpoint{0.100000in}{0.212622in}}{\pgfqpoint{3.696000in}{3.696000in}}%
\pgfusepath{clip}%
\pgfsetbuttcap%
\pgfsetroundjoin%
\definecolor{currentfill}{rgb}{0.121569,0.466667,0.705882}%
\pgfsetfillcolor{currentfill}%
\pgfsetfillopacity{0.411004}%
\pgfsetlinewidth{1.003750pt}%
\definecolor{currentstroke}{rgb}{0.121569,0.466667,0.705882}%
\pgfsetstrokecolor{currentstroke}%
\pgfsetstrokeopacity{0.411004}%
\pgfsetdash{}{0pt}%
\pgfpathmoveto{\pgfqpoint{1.460021in}{1.943232in}}%
\pgfpathcurveto{\pgfqpoint{1.468258in}{1.943232in}}{\pgfqpoint{1.476158in}{1.946504in}}{\pgfqpoint{1.481982in}{1.952328in}}%
\pgfpathcurveto{\pgfqpoint{1.487806in}{1.958152in}}{\pgfqpoint{1.491078in}{1.966052in}}{\pgfqpoint{1.491078in}{1.974288in}}%
\pgfpathcurveto{\pgfqpoint{1.491078in}{1.982524in}}{\pgfqpoint{1.487806in}{1.990424in}}{\pgfqpoint{1.481982in}{1.996248in}}%
\pgfpathcurveto{\pgfqpoint{1.476158in}{2.002072in}}{\pgfqpoint{1.468258in}{2.005345in}}{\pgfqpoint{1.460021in}{2.005345in}}%
\pgfpathcurveto{\pgfqpoint{1.451785in}{2.005345in}}{\pgfqpoint{1.443885in}{2.002072in}}{\pgfqpoint{1.438061in}{1.996248in}}%
\pgfpathcurveto{\pgfqpoint{1.432237in}{1.990424in}}{\pgfqpoint{1.428965in}{1.982524in}}{\pgfqpoint{1.428965in}{1.974288in}}%
\pgfpathcurveto{\pgfqpoint{1.428965in}{1.966052in}}{\pgfqpoint{1.432237in}{1.958152in}}{\pgfqpoint{1.438061in}{1.952328in}}%
\pgfpathcurveto{\pgfqpoint{1.443885in}{1.946504in}}{\pgfqpoint{1.451785in}{1.943232in}}{\pgfqpoint{1.460021in}{1.943232in}}%
\pgfpathclose%
\pgfusepath{stroke,fill}%
\end{pgfscope}%
\begin{pgfscope}%
\pgfpathrectangle{\pgfqpoint{0.100000in}{0.212622in}}{\pgfqpoint{3.696000in}{3.696000in}}%
\pgfusepath{clip}%
\pgfsetbuttcap%
\pgfsetroundjoin%
\definecolor{currentfill}{rgb}{0.121569,0.466667,0.705882}%
\pgfsetfillcolor{currentfill}%
\pgfsetfillopacity{0.414156}%
\pgfsetlinewidth{1.003750pt}%
\definecolor{currentstroke}{rgb}{0.121569,0.466667,0.705882}%
\pgfsetstrokecolor{currentstroke}%
\pgfsetstrokeopacity{0.414156}%
\pgfsetdash{}{0pt}%
\pgfpathmoveto{\pgfqpoint{1.455510in}{1.939296in}}%
\pgfpathcurveto{\pgfqpoint{1.463746in}{1.939296in}}{\pgfqpoint{1.471646in}{1.942569in}}{\pgfqpoint{1.477470in}{1.948393in}}%
\pgfpathcurveto{\pgfqpoint{1.483294in}{1.954217in}}{\pgfqpoint{1.486567in}{1.962117in}}{\pgfqpoint{1.486567in}{1.970353in}}%
\pgfpathcurveto{\pgfqpoint{1.486567in}{1.978589in}}{\pgfqpoint{1.483294in}{1.986489in}}{\pgfqpoint{1.477470in}{1.992313in}}%
\pgfpathcurveto{\pgfqpoint{1.471646in}{1.998137in}}{\pgfqpoint{1.463746in}{2.001409in}}{\pgfqpoint{1.455510in}{2.001409in}}%
\pgfpathcurveto{\pgfqpoint{1.447274in}{2.001409in}}{\pgfqpoint{1.439374in}{1.998137in}}{\pgfqpoint{1.433550in}{1.992313in}}%
\pgfpathcurveto{\pgfqpoint{1.427726in}{1.986489in}}{\pgfqpoint{1.424454in}{1.978589in}}{\pgfqpoint{1.424454in}{1.970353in}}%
\pgfpathcurveto{\pgfqpoint{1.424454in}{1.962117in}}{\pgfqpoint{1.427726in}{1.954217in}}{\pgfqpoint{1.433550in}{1.948393in}}%
\pgfpathcurveto{\pgfqpoint{1.439374in}{1.942569in}}{\pgfqpoint{1.447274in}{1.939296in}}{\pgfqpoint{1.455510in}{1.939296in}}%
\pgfpathclose%
\pgfusepath{stroke,fill}%
\end{pgfscope}%
\begin{pgfscope}%
\pgfpathrectangle{\pgfqpoint{0.100000in}{0.212622in}}{\pgfqpoint{3.696000in}{3.696000in}}%
\pgfusepath{clip}%
\pgfsetbuttcap%
\pgfsetroundjoin%
\definecolor{currentfill}{rgb}{0.121569,0.466667,0.705882}%
\pgfsetfillcolor{currentfill}%
\pgfsetfillopacity{0.415765}%
\pgfsetlinewidth{1.003750pt}%
\definecolor{currentstroke}{rgb}{0.121569,0.466667,0.705882}%
\pgfsetstrokecolor{currentstroke}%
\pgfsetstrokeopacity{0.415765}%
\pgfsetdash{}{0pt}%
\pgfpathmoveto{\pgfqpoint{1.457308in}{1.944584in}}%
\pgfpathcurveto{\pgfqpoint{1.465545in}{1.944584in}}{\pgfqpoint{1.473445in}{1.947856in}}{\pgfqpoint{1.479269in}{1.953680in}}%
\pgfpathcurveto{\pgfqpoint{1.485093in}{1.959504in}}{\pgfqpoint{1.488365in}{1.967404in}}{\pgfqpoint{1.488365in}{1.975641in}}%
\pgfpathcurveto{\pgfqpoint{1.488365in}{1.983877in}}{\pgfqpoint{1.485093in}{1.991777in}}{\pgfqpoint{1.479269in}{1.997601in}}%
\pgfpathcurveto{\pgfqpoint{1.473445in}{2.003425in}}{\pgfqpoint{1.465545in}{2.006697in}}{\pgfqpoint{1.457308in}{2.006697in}}%
\pgfpathcurveto{\pgfqpoint{1.449072in}{2.006697in}}{\pgfqpoint{1.441172in}{2.003425in}}{\pgfqpoint{1.435348in}{1.997601in}}%
\pgfpathcurveto{\pgfqpoint{1.429524in}{1.991777in}}{\pgfqpoint{1.426252in}{1.983877in}}{\pgfqpoint{1.426252in}{1.975641in}}%
\pgfpathcurveto{\pgfqpoint{1.426252in}{1.967404in}}{\pgfqpoint{1.429524in}{1.959504in}}{\pgfqpoint{1.435348in}{1.953680in}}%
\pgfpathcurveto{\pgfqpoint{1.441172in}{1.947856in}}{\pgfqpoint{1.449072in}{1.944584in}}{\pgfqpoint{1.457308in}{1.944584in}}%
\pgfpathclose%
\pgfusepath{stroke,fill}%
\end{pgfscope}%
\begin{pgfscope}%
\pgfpathrectangle{\pgfqpoint{0.100000in}{0.212622in}}{\pgfqpoint{3.696000in}{3.696000in}}%
\pgfusepath{clip}%
\pgfsetbuttcap%
\pgfsetroundjoin%
\definecolor{currentfill}{rgb}{0.121569,0.466667,0.705882}%
\pgfsetfillcolor{currentfill}%
\pgfsetfillopacity{0.417676}%
\pgfsetlinewidth{1.003750pt}%
\definecolor{currentstroke}{rgb}{0.121569,0.466667,0.705882}%
\pgfsetstrokecolor{currentstroke}%
\pgfsetstrokeopacity{0.417676}%
\pgfsetdash{}{0pt}%
\pgfpathmoveto{\pgfqpoint{1.455020in}{1.942913in}}%
\pgfpathcurveto{\pgfqpoint{1.463257in}{1.942913in}}{\pgfqpoint{1.471157in}{1.946185in}}{\pgfqpoint{1.476981in}{1.952009in}}%
\pgfpathcurveto{\pgfqpoint{1.482804in}{1.957833in}}{\pgfqpoint{1.486077in}{1.965733in}}{\pgfqpoint{1.486077in}{1.973969in}}%
\pgfpathcurveto{\pgfqpoint{1.486077in}{1.982205in}}{\pgfqpoint{1.482804in}{1.990105in}}{\pgfqpoint{1.476981in}{1.995929in}}%
\pgfpathcurveto{\pgfqpoint{1.471157in}{2.001753in}}{\pgfqpoint{1.463257in}{2.005026in}}{\pgfqpoint{1.455020in}{2.005026in}}%
\pgfpathcurveto{\pgfqpoint{1.446784in}{2.005026in}}{\pgfqpoint{1.438884in}{2.001753in}}{\pgfqpoint{1.433060in}{1.995929in}}%
\pgfpathcurveto{\pgfqpoint{1.427236in}{1.990105in}}{\pgfqpoint{1.423964in}{1.982205in}}{\pgfqpoint{1.423964in}{1.973969in}}%
\pgfpathcurveto{\pgfqpoint{1.423964in}{1.965733in}}{\pgfqpoint{1.427236in}{1.957833in}}{\pgfqpoint{1.433060in}{1.952009in}}%
\pgfpathcurveto{\pgfqpoint{1.438884in}{1.946185in}}{\pgfqpoint{1.446784in}{1.942913in}}{\pgfqpoint{1.455020in}{1.942913in}}%
\pgfpathclose%
\pgfusepath{stroke,fill}%
\end{pgfscope}%
\begin{pgfscope}%
\pgfpathrectangle{\pgfqpoint{0.100000in}{0.212622in}}{\pgfqpoint{3.696000in}{3.696000in}}%
\pgfusepath{clip}%
\pgfsetbuttcap%
\pgfsetroundjoin%
\definecolor{currentfill}{rgb}{0.121569,0.466667,0.705882}%
\pgfsetfillcolor{currentfill}%
\pgfsetfillopacity{0.419382}%
\pgfsetlinewidth{1.003750pt}%
\definecolor{currentstroke}{rgb}{0.121569,0.466667,0.705882}%
\pgfsetstrokecolor{currentstroke}%
\pgfsetstrokeopacity{0.419382}%
\pgfsetdash{}{0pt}%
\pgfpathmoveto{\pgfqpoint{1.452100in}{1.940575in}}%
\pgfpathcurveto{\pgfqpoint{1.460337in}{1.940575in}}{\pgfqpoint{1.468237in}{1.943847in}}{\pgfqpoint{1.474061in}{1.949671in}}%
\pgfpathcurveto{\pgfqpoint{1.479884in}{1.955495in}}{\pgfqpoint{1.483157in}{1.963395in}}{\pgfqpoint{1.483157in}{1.971631in}}%
\pgfpathcurveto{\pgfqpoint{1.483157in}{1.979867in}}{\pgfqpoint{1.479884in}{1.987767in}}{\pgfqpoint{1.474061in}{1.993591in}}%
\pgfpathcurveto{\pgfqpoint{1.468237in}{1.999415in}}{\pgfqpoint{1.460337in}{2.002688in}}{\pgfqpoint{1.452100in}{2.002688in}}%
\pgfpathcurveto{\pgfqpoint{1.443864in}{2.002688in}}{\pgfqpoint{1.435964in}{1.999415in}}{\pgfqpoint{1.430140in}{1.993591in}}%
\pgfpathcurveto{\pgfqpoint{1.424316in}{1.987767in}}{\pgfqpoint{1.421044in}{1.979867in}}{\pgfqpoint{1.421044in}{1.971631in}}%
\pgfpathcurveto{\pgfqpoint{1.421044in}{1.963395in}}{\pgfqpoint{1.424316in}{1.955495in}}{\pgfqpoint{1.430140in}{1.949671in}}%
\pgfpathcurveto{\pgfqpoint{1.435964in}{1.943847in}}{\pgfqpoint{1.443864in}{1.940575in}}{\pgfqpoint{1.452100in}{1.940575in}}%
\pgfpathclose%
\pgfusepath{stroke,fill}%
\end{pgfscope}%
\begin{pgfscope}%
\pgfpathrectangle{\pgfqpoint{0.100000in}{0.212622in}}{\pgfqpoint{3.696000in}{3.696000in}}%
\pgfusepath{clip}%
\pgfsetbuttcap%
\pgfsetroundjoin%
\definecolor{currentfill}{rgb}{0.121569,0.466667,0.705882}%
\pgfsetfillcolor{currentfill}%
\pgfsetfillopacity{0.420421}%
\pgfsetlinewidth{1.003750pt}%
\definecolor{currentstroke}{rgb}{0.121569,0.466667,0.705882}%
\pgfsetstrokecolor{currentstroke}%
\pgfsetstrokeopacity{0.420421}%
\pgfsetdash{}{0pt}%
\pgfpathmoveto{\pgfqpoint{1.454734in}{1.950404in}}%
\pgfpathcurveto{\pgfqpoint{1.462970in}{1.950404in}}{\pgfqpoint{1.470870in}{1.953676in}}{\pgfqpoint{1.476694in}{1.959500in}}%
\pgfpathcurveto{\pgfqpoint{1.482518in}{1.965324in}}{\pgfqpoint{1.485790in}{1.973224in}}{\pgfqpoint{1.485790in}{1.981460in}}%
\pgfpathcurveto{\pgfqpoint{1.485790in}{1.989697in}}{\pgfqpoint{1.482518in}{1.997597in}}{\pgfqpoint{1.476694in}{2.003421in}}%
\pgfpathcurveto{\pgfqpoint{1.470870in}{2.009244in}}{\pgfqpoint{1.462970in}{2.012517in}}{\pgfqpoint{1.454734in}{2.012517in}}%
\pgfpathcurveto{\pgfqpoint{1.446498in}{2.012517in}}{\pgfqpoint{1.438598in}{2.009244in}}{\pgfqpoint{1.432774in}{2.003421in}}%
\pgfpathcurveto{\pgfqpoint{1.426950in}{1.997597in}}{\pgfqpoint{1.423677in}{1.989697in}}{\pgfqpoint{1.423677in}{1.981460in}}%
\pgfpathcurveto{\pgfqpoint{1.423677in}{1.973224in}}{\pgfqpoint{1.426950in}{1.965324in}}{\pgfqpoint{1.432774in}{1.959500in}}%
\pgfpathcurveto{\pgfqpoint{1.438598in}{1.953676in}}{\pgfqpoint{1.446498in}{1.950404in}}{\pgfqpoint{1.454734in}{1.950404in}}%
\pgfpathclose%
\pgfusepath{stroke,fill}%
\end{pgfscope}%
\begin{pgfscope}%
\pgfpathrectangle{\pgfqpoint{0.100000in}{0.212622in}}{\pgfqpoint{3.696000in}{3.696000in}}%
\pgfusepath{clip}%
\pgfsetbuttcap%
\pgfsetroundjoin%
\definecolor{currentfill}{rgb}{0.121569,0.466667,0.705882}%
\pgfsetfillcolor{currentfill}%
\pgfsetfillopacity{0.424248}%
\pgfsetlinewidth{1.003750pt}%
\definecolor{currentstroke}{rgb}{0.121569,0.466667,0.705882}%
\pgfsetstrokecolor{currentstroke}%
\pgfsetstrokeopacity{0.424248}%
\pgfsetdash{}{0pt}%
\pgfpathmoveto{\pgfqpoint{1.446351in}{1.943357in}}%
\pgfpathcurveto{\pgfqpoint{1.454587in}{1.943357in}}{\pgfqpoint{1.462487in}{1.946629in}}{\pgfqpoint{1.468311in}{1.952453in}}%
\pgfpathcurveto{\pgfqpoint{1.474135in}{1.958277in}}{\pgfqpoint{1.477407in}{1.966177in}}{\pgfqpoint{1.477407in}{1.974413in}}%
\pgfpathcurveto{\pgfqpoint{1.477407in}{1.982650in}}{\pgfqpoint{1.474135in}{1.990550in}}{\pgfqpoint{1.468311in}{1.996374in}}%
\pgfpathcurveto{\pgfqpoint{1.462487in}{2.002198in}}{\pgfqpoint{1.454587in}{2.005470in}}{\pgfqpoint{1.446351in}{2.005470in}}%
\pgfpathcurveto{\pgfqpoint{1.438114in}{2.005470in}}{\pgfqpoint{1.430214in}{2.002198in}}{\pgfqpoint{1.424390in}{1.996374in}}%
\pgfpathcurveto{\pgfqpoint{1.418566in}{1.990550in}}{\pgfqpoint{1.415294in}{1.982650in}}{\pgfqpoint{1.415294in}{1.974413in}}%
\pgfpathcurveto{\pgfqpoint{1.415294in}{1.966177in}}{\pgfqpoint{1.418566in}{1.958277in}}{\pgfqpoint{1.424390in}{1.952453in}}%
\pgfpathcurveto{\pgfqpoint{1.430214in}{1.946629in}}{\pgfqpoint{1.438114in}{1.943357in}}{\pgfqpoint{1.446351in}{1.943357in}}%
\pgfpathclose%
\pgfusepath{stroke,fill}%
\end{pgfscope}%
\begin{pgfscope}%
\pgfpathrectangle{\pgfqpoint{0.100000in}{0.212622in}}{\pgfqpoint{3.696000in}{3.696000in}}%
\pgfusepath{clip}%
\pgfsetbuttcap%
\pgfsetroundjoin%
\definecolor{currentfill}{rgb}{0.121569,0.466667,0.705882}%
\pgfsetfillcolor{currentfill}%
\pgfsetfillopacity{0.424930}%
\pgfsetlinewidth{1.003750pt}%
\definecolor{currentstroke}{rgb}{0.121569,0.466667,0.705882}%
\pgfsetstrokecolor{currentstroke}%
\pgfsetstrokeopacity{0.424930}%
\pgfsetdash{}{0pt}%
\pgfpathmoveto{\pgfqpoint{1.445419in}{1.942768in}}%
\pgfpathcurveto{\pgfqpoint{1.453655in}{1.942768in}}{\pgfqpoint{1.461555in}{1.946041in}}{\pgfqpoint{1.467379in}{1.951864in}}%
\pgfpathcurveto{\pgfqpoint{1.473203in}{1.957688in}}{\pgfqpoint{1.476476in}{1.965588in}}{\pgfqpoint{1.476476in}{1.973825in}}%
\pgfpathcurveto{\pgfqpoint{1.476476in}{1.982061in}}{\pgfqpoint{1.473203in}{1.989961in}}{\pgfqpoint{1.467379in}{1.995785in}}%
\pgfpathcurveto{\pgfqpoint{1.461555in}{2.001609in}}{\pgfqpoint{1.453655in}{2.004881in}}{\pgfqpoint{1.445419in}{2.004881in}}%
\pgfpathcurveto{\pgfqpoint{1.437183in}{2.004881in}}{\pgfqpoint{1.429283in}{2.001609in}}{\pgfqpoint{1.423459in}{1.995785in}}%
\pgfpathcurveto{\pgfqpoint{1.417635in}{1.989961in}}{\pgfqpoint{1.414363in}{1.982061in}}{\pgfqpoint{1.414363in}{1.973825in}}%
\pgfpathcurveto{\pgfqpoint{1.414363in}{1.965588in}}{\pgfqpoint{1.417635in}{1.957688in}}{\pgfqpoint{1.423459in}{1.951864in}}%
\pgfpathcurveto{\pgfqpoint{1.429283in}{1.946041in}}{\pgfqpoint{1.437183in}{1.942768in}}{\pgfqpoint{1.445419in}{1.942768in}}%
\pgfpathclose%
\pgfusepath{stroke,fill}%
\end{pgfscope}%
\begin{pgfscope}%
\pgfpathrectangle{\pgfqpoint{0.100000in}{0.212622in}}{\pgfqpoint{3.696000in}{3.696000in}}%
\pgfusepath{clip}%
\pgfsetbuttcap%
\pgfsetroundjoin%
\definecolor{currentfill}{rgb}{0.121569,0.466667,0.705882}%
\pgfsetfillcolor{currentfill}%
\pgfsetfillopacity{0.427642}%
\pgfsetlinewidth{1.003750pt}%
\definecolor{currentstroke}{rgb}{0.121569,0.466667,0.705882}%
\pgfsetstrokecolor{currentstroke}%
\pgfsetstrokeopacity{0.427642}%
\pgfsetdash{}{0pt}%
\pgfpathmoveto{\pgfqpoint{1.440591in}{1.937011in}}%
\pgfpathcurveto{\pgfqpoint{1.448827in}{1.937011in}}{\pgfqpoint{1.456728in}{1.940283in}}{\pgfqpoint{1.462551in}{1.946107in}}%
\pgfpathcurveto{\pgfqpoint{1.468375in}{1.951931in}}{\pgfqpoint{1.471648in}{1.959831in}}{\pgfqpoint{1.471648in}{1.968067in}}%
\pgfpathcurveto{\pgfqpoint{1.471648in}{1.976304in}}{\pgfqpoint{1.468375in}{1.984204in}}{\pgfqpoint{1.462551in}{1.990027in}}%
\pgfpathcurveto{\pgfqpoint{1.456728in}{1.995851in}}{\pgfqpoint{1.448827in}{1.999124in}}{\pgfqpoint{1.440591in}{1.999124in}}%
\pgfpathcurveto{\pgfqpoint{1.432355in}{1.999124in}}{\pgfqpoint{1.424455in}{1.995851in}}{\pgfqpoint{1.418631in}{1.990027in}}%
\pgfpathcurveto{\pgfqpoint{1.412807in}{1.984204in}}{\pgfqpoint{1.409535in}{1.976304in}}{\pgfqpoint{1.409535in}{1.968067in}}%
\pgfpathcurveto{\pgfqpoint{1.409535in}{1.959831in}}{\pgfqpoint{1.412807in}{1.951931in}}{\pgfqpoint{1.418631in}{1.946107in}}%
\pgfpathcurveto{\pgfqpoint{1.424455in}{1.940283in}}{\pgfqpoint{1.432355in}{1.937011in}}{\pgfqpoint{1.440591in}{1.937011in}}%
\pgfpathclose%
\pgfusepath{stroke,fill}%
\end{pgfscope}%
\begin{pgfscope}%
\pgfpathrectangle{\pgfqpoint{0.100000in}{0.212622in}}{\pgfqpoint{3.696000in}{3.696000in}}%
\pgfusepath{clip}%
\pgfsetbuttcap%
\pgfsetroundjoin%
\definecolor{currentfill}{rgb}{0.121569,0.466667,0.705882}%
\pgfsetfillcolor{currentfill}%
\pgfsetfillopacity{0.428712}%
\pgfsetlinewidth{1.003750pt}%
\definecolor{currentstroke}{rgb}{0.121569,0.466667,0.705882}%
\pgfsetstrokecolor{currentstroke}%
\pgfsetstrokeopacity{0.428712}%
\pgfsetdash{}{0pt}%
\pgfpathmoveto{\pgfqpoint{1.439371in}{1.936024in}}%
\pgfpathcurveto{\pgfqpoint{1.447607in}{1.936024in}}{\pgfqpoint{1.455507in}{1.939296in}}{\pgfqpoint{1.461331in}{1.945120in}}%
\pgfpathcurveto{\pgfqpoint{1.467155in}{1.950944in}}{\pgfqpoint{1.470427in}{1.958844in}}{\pgfqpoint{1.470427in}{1.967081in}}%
\pgfpathcurveto{\pgfqpoint{1.470427in}{1.975317in}}{\pgfqpoint{1.467155in}{1.983217in}}{\pgfqpoint{1.461331in}{1.989041in}}%
\pgfpathcurveto{\pgfqpoint{1.455507in}{1.994865in}}{\pgfqpoint{1.447607in}{1.998137in}}{\pgfqpoint{1.439371in}{1.998137in}}%
\pgfpathcurveto{\pgfqpoint{1.431134in}{1.998137in}}{\pgfqpoint{1.423234in}{1.994865in}}{\pgfqpoint{1.417410in}{1.989041in}}%
\pgfpathcurveto{\pgfqpoint{1.411587in}{1.983217in}}{\pgfqpoint{1.408314in}{1.975317in}}{\pgfqpoint{1.408314in}{1.967081in}}%
\pgfpathcurveto{\pgfqpoint{1.408314in}{1.958844in}}{\pgfqpoint{1.411587in}{1.950944in}}{\pgfqpoint{1.417410in}{1.945120in}}%
\pgfpathcurveto{\pgfqpoint{1.423234in}{1.939296in}}{\pgfqpoint{1.431134in}{1.936024in}}{\pgfqpoint{1.439371in}{1.936024in}}%
\pgfpathclose%
\pgfusepath{stroke,fill}%
\end{pgfscope}%
\begin{pgfscope}%
\pgfpathrectangle{\pgfqpoint{0.100000in}{0.212622in}}{\pgfqpoint{3.696000in}{3.696000in}}%
\pgfusepath{clip}%
\pgfsetbuttcap%
\pgfsetroundjoin%
\definecolor{currentfill}{rgb}{0.121569,0.466667,0.705882}%
\pgfsetfillcolor{currentfill}%
\pgfsetfillopacity{0.429431}%
\pgfsetlinewidth{1.003750pt}%
\definecolor{currentstroke}{rgb}{0.121569,0.466667,0.705882}%
\pgfsetstrokecolor{currentstroke}%
\pgfsetstrokeopacity{0.429431}%
\pgfsetdash{}{0pt}%
\pgfpathmoveto{\pgfqpoint{1.438125in}{1.934357in}}%
\pgfpathcurveto{\pgfqpoint{1.446361in}{1.934357in}}{\pgfqpoint{1.454261in}{1.937629in}}{\pgfqpoint{1.460085in}{1.943453in}}%
\pgfpathcurveto{\pgfqpoint{1.465909in}{1.949277in}}{\pgfqpoint{1.469182in}{1.957177in}}{\pgfqpoint{1.469182in}{1.965413in}}%
\pgfpathcurveto{\pgfqpoint{1.469182in}{1.973649in}}{\pgfqpoint{1.465909in}{1.981549in}}{\pgfqpoint{1.460085in}{1.987373in}}%
\pgfpathcurveto{\pgfqpoint{1.454261in}{1.993197in}}{\pgfqpoint{1.446361in}{1.996470in}}{\pgfqpoint{1.438125in}{1.996470in}}%
\pgfpathcurveto{\pgfqpoint{1.429889in}{1.996470in}}{\pgfqpoint{1.421989in}{1.993197in}}{\pgfqpoint{1.416165in}{1.987373in}}%
\pgfpathcurveto{\pgfqpoint{1.410341in}{1.981549in}}{\pgfqpoint{1.407069in}{1.973649in}}{\pgfqpoint{1.407069in}{1.965413in}}%
\pgfpathcurveto{\pgfqpoint{1.407069in}{1.957177in}}{\pgfqpoint{1.410341in}{1.949277in}}{\pgfqpoint{1.416165in}{1.943453in}}%
\pgfpathcurveto{\pgfqpoint{1.421989in}{1.937629in}}{\pgfqpoint{1.429889in}{1.934357in}}{\pgfqpoint{1.438125in}{1.934357in}}%
\pgfpathclose%
\pgfusepath{stroke,fill}%
\end{pgfscope}%
\begin{pgfscope}%
\pgfpathrectangle{\pgfqpoint{0.100000in}{0.212622in}}{\pgfqpoint{3.696000in}{3.696000in}}%
\pgfusepath{clip}%
\pgfsetbuttcap%
\pgfsetroundjoin%
\definecolor{currentfill}{rgb}{0.121569,0.466667,0.705882}%
\pgfsetfillcolor{currentfill}%
\pgfsetfillopacity{0.430547}%
\pgfsetlinewidth{1.003750pt}%
\definecolor{currentstroke}{rgb}{0.121569,0.466667,0.705882}%
\pgfsetstrokecolor{currentstroke}%
\pgfsetstrokeopacity{0.430547}%
\pgfsetdash{}{0pt}%
\pgfpathmoveto{\pgfqpoint{1.437018in}{1.933097in}}%
\pgfpathcurveto{\pgfqpoint{1.445255in}{1.933097in}}{\pgfqpoint{1.453155in}{1.936369in}}{\pgfqpoint{1.458979in}{1.942193in}}%
\pgfpathcurveto{\pgfqpoint{1.464803in}{1.948017in}}{\pgfqpoint{1.468075in}{1.955917in}}{\pgfqpoint{1.468075in}{1.964154in}}%
\pgfpathcurveto{\pgfqpoint{1.468075in}{1.972390in}}{\pgfqpoint{1.464803in}{1.980290in}}{\pgfqpoint{1.458979in}{1.986114in}}%
\pgfpathcurveto{\pgfqpoint{1.453155in}{1.991938in}}{\pgfqpoint{1.445255in}{1.995210in}}{\pgfqpoint{1.437018in}{1.995210in}}%
\pgfpathcurveto{\pgfqpoint{1.428782in}{1.995210in}}{\pgfqpoint{1.420882in}{1.991938in}}{\pgfqpoint{1.415058in}{1.986114in}}%
\pgfpathcurveto{\pgfqpoint{1.409234in}{1.980290in}}{\pgfqpoint{1.405962in}{1.972390in}}{\pgfqpoint{1.405962in}{1.964154in}}%
\pgfpathcurveto{\pgfqpoint{1.405962in}{1.955917in}}{\pgfqpoint{1.409234in}{1.948017in}}{\pgfqpoint{1.415058in}{1.942193in}}%
\pgfpathcurveto{\pgfqpoint{1.420882in}{1.936369in}}{\pgfqpoint{1.428782in}{1.933097in}}{\pgfqpoint{1.437018in}{1.933097in}}%
\pgfpathclose%
\pgfusepath{stroke,fill}%
\end{pgfscope}%
\begin{pgfscope}%
\pgfpathrectangle{\pgfqpoint{0.100000in}{0.212622in}}{\pgfqpoint{3.696000in}{3.696000in}}%
\pgfusepath{clip}%
\pgfsetbuttcap%
\pgfsetroundjoin%
\definecolor{currentfill}{rgb}{0.121569,0.466667,0.705882}%
\pgfsetfillcolor{currentfill}%
\pgfsetfillopacity{0.431200}%
\pgfsetlinewidth{1.003750pt}%
\definecolor{currentstroke}{rgb}{0.121569,0.466667,0.705882}%
\pgfsetstrokecolor{currentstroke}%
\pgfsetstrokeopacity{0.431200}%
\pgfsetdash{}{0pt}%
\pgfpathmoveto{\pgfqpoint{1.436945in}{1.934510in}}%
\pgfpathcurveto{\pgfqpoint{1.445182in}{1.934510in}}{\pgfqpoint{1.453082in}{1.937783in}}{\pgfqpoint{1.458906in}{1.943607in}}%
\pgfpathcurveto{\pgfqpoint{1.464730in}{1.949431in}}{\pgfqpoint{1.468002in}{1.957331in}}{\pgfqpoint{1.468002in}{1.965567in}}%
\pgfpathcurveto{\pgfqpoint{1.468002in}{1.973803in}}{\pgfqpoint{1.464730in}{1.981703in}}{\pgfqpoint{1.458906in}{1.987527in}}%
\pgfpathcurveto{\pgfqpoint{1.453082in}{1.993351in}}{\pgfqpoint{1.445182in}{1.996623in}}{\pgfqpoint{1.436945in}{1.996623in}}%
\pgfpathcurveto{\pgfqpoint{1.428709in}{1.996623in}}{\pgfqpoint{1.420809in}{1.993351in}}{\pgfqpoint{1.414985in}{1.987527in}}%
\pgfpathcurveto{\pgfqpoint{1.409161in}{1.981703in}}{\pgfqpoint{1.405889in}{1.973803in}}{\pgfqpoint{1.405889in}{1.965567in}}%
\pgfpathcurveto{\pgfqpoint{1.405889in}{1.957331in}}{\pgfqpoint{1.409161in}{1.949431in}}{\pgfqpoint{1.414985in}{1.943607in}}%
\pgfpathcurveto{\pgfqpoint{1.420809in}{1.937783in}}{\pgfqpoint{1.428709in}{1.934510in}}{\pgfqpoint{1.436945in}{1.934510in}}%
\pgfpathclose%
\pgfusepath{stroke,fill}%
\end{pgfscope}%
\begin{pgfscope}%
\pgfpathrectangle{\pgfqpoint{0.100000in}{0.212622in}}{\pgfqpoint{3.696000in}{3.696000in}}%
\pgfusepath{clip}%
\pgfsetbuttcap%
\pgfsetroundjoin%
\definecolor{currentfill}{rgb}{0.121569,0.466667,0.705882}%
\pgfsetfillcolor{currentfill}%
\pgfsetfillopacity{0.432218}%
\pgfsetlinewidth{1.003750pt}%
\definecolor{currentstroke}{rgb}{0.121569,0.466667,0.705882}%
\pgfsetstrokecolor{currentstroke}%
\pgfsetstrokeopacity{0.432218}%
\pgfsetdash{}{0pt}%
\pgfpathmoveto{\pgfqpoint{1.442314in}{1.943801in}}%
\pgfpathcurveto{\pgfqpoint{1.450551in}{1.943801in}}{\pgfqpoint{1.458451in}{1.947073in}}{\pgfqpoint{1.464275in}{1.952897in}}%
\pgfpathcurveto{\pgfqpoint{1.470099in}{1.958721in}}{\pgfqpoint{1.473371in}{1.966621in}}{\pgfqpoint{1.473371in}{1.974858in}}%
\pgfpathcurveto{\pgfqpoint{1.473371in}{1.983094in}}{\pgfqpoint{1.470099in}{1.990994in}}{\pgfqpoint{1.464275in}{1.996818in}}%
\pgfpathcurveto{\pgfqpoint{1.458451in}{2.002642in}}{\pgfqpoint{1.450551in}{2.005914in}}{\pgfqpoint{1.442314in}{2.005914in}}%
\pgfpathcurveto{\pgfqpoint{1.434078in}{2.005914in}}{\pgfqpoint{1.426178in}{2.002642in}}{\pgfqpoint{1.420354in}{1.996818in}}%
\pgfpathcurveto{\pgfqpoint{1.414530in}{1.990994in}}{\pgfqpoint{1.411258in}{1.983094in}}{\pgfqpoint{1.411258in}{1.974858in}}%
\pgfpathcurveto{\pgfqpoint{1.411258in}{1.966621in}}{\pgfqpoint{1.414530in}{1.958721in}}{\pgfqpoint{1.420354in}{1.952897in}}%
\pgfpathcurveto{\pgfqpoint{1.426178in}{1.947073in}}{\pgfqpoint{1.434078in}{1.943801in}}{\pgfqpoint{1.442314in}{1.943801in}}%
\pgfpathclose%
\pgfusepath{stroke,fill}%
\end{pgfscope}%
\begin{pgfscope}%
\pgfpathrectangle{\pgfqpoint{0.100000in}{0.212622in}}{\pgfqpoint{3.696000in}{3.696000in}}%
\pgfusepath{clip}%
\pgfsetbuttcap%
\pgfsetroundjoin%
\definecolor{currentfill}{rgb}{0.121569,0.466667,0.705882}%
\pgfsetfillcolor{currentfill}%
\pgfsetfillopacity{0.433567}%
\pgfsetlinewidth{1.003750pt}%
\definecolor{currentstroke}{rgb}{0.121569,0.466667,0.705882}%
\pgfsetstrokecolor{currentstroke}%
\pgfsetstrokeopacity{0.433567}%
\pgfsetdash{}{0pt}%
\pgfpathmoveto{\pgfqpoint{1.432633in}{1.929225in}}%
\pgfpathcurveto{\pgfqpoint{1.440869in}{1.929225in}}{\pgfqpoint{1.448769in}{1.932497in}}{\pgfqpoint{1.454593in}{1.938321in}}%
\pgfpathcurveto{\pgfqpoint{1.460417in}{1.944145in}}{\pgfqpoint{1.463689in}{1.952045in}}{\pgfqpoint{1.463689in}{1.960281in}}%
\pgfpathcurveto{\pgfqpoint{1.463689in}{1.968518in}}{\pgfqpoint{1.460417in}{1.976418in}}{\pgfqpoint{1.454593in}{1.982242in}}%
\pgfpathcurveto{\pgfqpoint{1.448769in}{1.988066in}}{\pgfqpoint{1.440869in}{1.991338in}}{\pgfqpoint{1.432633in}{1.991338in}}%
\pgfpathcurveto{\pgfqpoint{1.424397in}{1.991338in}}{\pgfqpoint{1.416497in}{1.988066in}}{\pgfqpoint{1.410673in}{1.982242in}}%
\pgfpathcurveto{\pgfqpoint{1.404849in}{1.976418in}}{\pgfqpoint{1.401576in}{1.968518in}}{\pgfqpoint{1.401576in}{1.960281in}}%
\pgfpathcurveto{\pgfqpoint{1.401576in}{1.952045in}}{\pgfqpoint{1.404849in}{1.944145in}}{\pgfqpoint{1.410673in}{1.938321in}}%
\pgfpathcurveto{\pgfqpoint{1.416497in}{1.932497in}}{\pgfqpoint{1.424397in}{1.929225in}}{\pgfqpoint{1.432633in}{1.929225in}}%
\pgfpathclose%
\pgfusepath{stroke,fill}%
\end{pgfscope}%
\begin{pgfscope}%
\pgfpathrectangle{\pgfqpoint{0.100000in}{0.212622in}}{\pgfqpoint{3.696000in}{3.696000in}}%
\pgfusepath{clip}%
\pgfsetbuttcap%
\pgfsetroundjoin%
\definecolor{currentfill}{rgb}{0.121569,0.466667,0.705882}%
\pgfsetfillcolor{currentfill}%
\pgfsetfillopacity{0.436790}%
\pgfsetlinewidth{1.003750pt}%
\definecolor{currentstroke}{rgb}{0.121569,0.466667,0.705882}%
\pgfsetstrokecolor{currentstroke}%
\pgfsetstrokeopacity{0.436790}%
\pgfsetdash{}{0pt}%
\pgfpathmoveto{\pgfqpoint{1.431979in}{1.932679in}}%
\pgfpathcurveto{\pgfqpoint{1.440216in}{1.932679in}}{\pgfqpoint{1.448116in}{1.935952in}}{\pgfqpoint{1.453940in}{1.941776in}}%
\pgfpathcurveto{\pgfqpoint{1.459764in}{1.947599in}}{\pgfqpoint{1.463036in}{1.955499in}}{\pgfqpoint{1.463036in}{1.963736in}}%
\pgfpathcurveto{\pgfqpoint{1.463036in}{1.971972in}}{\pgfqpoint{1.459764in}{1.979872in}}{\pgfqpoint{1.453940in}{1.985696in}}%
\pgfpathcurveto{\pgfqpoint{1.448116in}{1.991520in}}{\pgfqpoint{1.440216in}{1.994792in}}{\pgfqpoint{1.431979in}{1.994792in}}%
\pgfpathcurveto{\pgfqpoint{1.423743in}{1.994792in}}{\pgfqpoint{1.415843in}{1.991520in}}{\pgfqpoint{1.410019in}{1.985696in}}%
\pgfpathcurveto{\pgfqpoint{1.404195in}{1.979872in}}{\pgfqpoint{1.400923in}{1.971972in}}{\pgfqpoint{1.400923in}{1.963736in}}%
\pgfpathcurveto{\pgfqpoint{1.400923in}{1.955499in}}{\pgfqpoint{1.404195in}{1.947599in}}{\pgfqpoint{1.410019in}{1.941776in}}%
\pgfpathcurveto{\pgfqpoint{1.415843in}{1.935952in}}{\pgfqpoint{1.423743in}{1.932679in}}{\pgfqpoint{1.431979in}{1.932679in}}%
\pgfpathclose%
\pgfusepath{stroke,fill}%
\end{pgfscope}%
\begin{pgfscope}%
\pgfpathrectangle{\pgfqpoint{0.100000in}{0.212622in}}{\pgfqpoint{3.696000in}{3.696000in}}%
\pgfusepath{clip}%
\pgfsetbuttcap%
\pgfsetroundjoin%
\definecolor{currentfill}{rgb}{0.121569,0.466667,0.705882}%
\pgfsetfillcolor{currentfill}%
\pgfsetfillopacity{0.437227}%
\pgfsetlinewidth{1.003750pt}%
\definecolor{currentstroke}{rgb}{0.121569,0.466667,0.705882}%
\pgfsetstrokecolor{currentstroke}%
\pgfsetstrokeopacity{0.437227}%
\pgfsetdash{}{0pt}%
\pgfpathmoveto{\pgfqpoint{1.438246in}{1.943868in}}%
\pgfpathcurveto{\pgfqpoint{1.446483in}{1.943868in}}{\pgfqpoint{1.454383in}{1.947140in}}{\pgfqpoint{1.460207in}{1.952964in}}%
\pgfpathcurveto{\pgfqpoint{1.466030in}{1.958788in}}{\pgfqpoint{1.469303in}{1.966688in}}{\pgfqpoint{1.469303in}{1.974924in}}%
\pgfpathcurveto{\pgfqpoint{1.469303in}{1.983160in}}{\pgfqpoint{1.466030in}{1.991060in}}{\pgfqpoint{1.460207in}{1.996884in}}%
\pgfpathcurveto{\pgfqpoint{1.454383in}{2.002708in}}{\pgfqpoint{1.446483in}{2.005981in}}{\pgfqpoint{1.438246in}{2.005981in}}%
\pgfpathcurveto{\pgfqpoint{1.430010in}{2.005981in}}{\pgfqpoint{1.422110in}{2.002708in}}{\pgfqpoint{1.416286in}{1.996884in}}%
\pgfpathcurveto{\pgfqpoint{1.410462in}{1.991060in}}{\pgfqpoint{1.407190in}{1.983160in}}{\pgfqpoint{1.407190in}{1.974924in}}%
\pgfpathcurveto{\pgfqpoint{1.407190in}{1.966688in}}{\pgfqpoint{1.410462in}{1.958788in}}{\pgfqpoint{1.416286in}{1.952964in}}%
\pgfpathcurveto{\pgfqpoint{1.422110in}{1.947140in}}{\pgfqpoint{1.430010in}{1.943868in}}{\pgfqpoint{1.438246in}{1.943868in}}%
\pgfpathclose%
\pgfusepath{stroke,fill}%
\end{pgfscope}%
\begin{pgfscope}%
\pgfpathrectangle{\pgfqpoint{0.100000in}{0.212622in}}{\pgfqpoint{3.696000in}{3.696000in}}%
\pgfusepath{clip}%
\pgfsetbuttcap%
\pgfsetroundjoin%
\definecolor{currentfill}{rgb}{0.121569,0.466667,0.705882}%
\pgfsetfillcolor{currentfill}%
\pgfsetfillopacity{0.440196}%
\pgfsetlinewidth{1.003750pt}%
\definecolor{currentstroke}{rgb}{0.121569,0.466667,0.705882}%
\pgfsetstrokecolor{currentstroke}%
\pgfsetstrokeopacity{0.440196}%
\pgfsetdash{}{0pt}%
\pgfpathmoveto{\pgfqpoint{1.449403in}{1.946712in}}%
\pgfpathcurveto{\pgfqpoint{1.457639in}{1.946712in}}{\pgfqpoint{1.465539in}{1.949984in}}{\pgfqpoint{1.471363in}{1.955808in}}%
\pgfpathcurveto{\pgfqpoint{1.477187in}{1.961632in}}{\pgfqpoint{1.480459in}{1.969532in}}{\pgfqpoint{1.480459in}{1.977768in}}%
\pgfpathcurveto{\pgfqpoint{1.480459in}{1.986005in}}{\pgfqpoint{1.477187in}{1.993905in}}{\pgfqpoint{1.471363in}{1.999729in}}%
\pgfpathcurveto{\pgfqpoint{1.465539in}{2.005553in}}{\pgfqpoint{1.457639in}{2.008825in}}{\pgfqpoint{1.449403in}{2.008825in}}%
\pgfpathcurveto{\pgfqpoint{1.441166in}{2.008825in}}{\pgfqpoint{1.433266in}{2.005553in}}{\pgfqpoint{1.427442in}{1.999729in}}%
\pgfpathcurveto{\pgfqpoint{1.421618in}{1.993905in}}{\pgfqpoint{1.418346in}{1.986005in}}{\pgfqpoint{1.418346in}{1.977768in}}%
\pgfpathcurveto{\pgfqpoint{1.418346in}{1.969532in}}{\pgfqpoint{1.421618in}{1.961632in}}{\pgfqpoint{1.427442in}{1.955808in}}%
\pgfpathcurveto{\pgfqpoint{1.433266in}{1.949984in}}{\pgfqpoint{1.441166in}{1.946712in}}{\pgfqpoint{1.449403in}{1.946712in}}%
\pgfpathclose%
\pgfusepath{stroke,fill}%
\end{pgfscope}%
\begin{pgfscope}%
\pgfpathrectangle{\pgfqpoint{0.100000in}{0.212622in}}{\pgfqpoint{3.696000in}{3.696000in}}%
\pgfusepath{clip}%
\pgfsetbuttcap%
\pgfsetroundjoin%
\definecolor{currentfill}{rgb}{0.121569,0.466667,0.705882}%
\pgfsetfillcolor{currentfill}%
\pgfsetfillopacity{0.440334}%
\pgfsetlinewidth{1.003750pt}%
\definecolor{currentstroke}{rgb}{0.121569,0.466667,0.705882}%
\pgfsetstrokecolor{currentstroke}%
\pgfsetstrokeopacity{0.440334}%
\pgfsetdash{}{0pt}%
\pgfpathmoveto{\pgfqpoint{1.438367in}{1.935731in}}%
\pgfpathcurveto{\pgfqpoint{1.446603in}{1.935731in}}{\pgfqpoint{1.454503in}{1.939003in}}{\pgfqpoint{1.460327in}{1.944827in}}%
\pgfpathcurveto{\pgfqpoint{1.466151in}{1.950651in}}{\pgfqpoint{1.469423in}{1.958551in}}{\pgfqpoint{1.469423in}{1.966787in}}%
\pgfpathcurveto{\pgfqpoint{1.469423in}{1.975023in}}{\pgfqpoint{1.466151in}{1.982923in}}{\pgfqpoint{1.460327in}{1.988747in}}%
\pgfpathcurveto{\pgfqpoint{1.454503in}{1.994571in}}{\pgfqpoint{1.446603in}{1.997844in}}{\pgfqpoint{1.438367in}{1.997844in}}%
\pgfpathcurveto{\pgfqpoint{1.430131in}{1.997844in}}{\pgfqpoint{1.422231in}{1.994571in}}{\pgfqpoint{1.416407in}{1.988747in}}%
\pgfpathcurveto{\pgfqpoint{1.410583in}{1.982923in}}{\pgfqpoint{1.407310in}{1.975023in}}{\pgfqpoint{1.407310in}{1.966787in}}%
\pgfpathcurveto{\pgfqpoint{1.407310in}{1.958551in}}{\pgfqpoint{1.410583in}{1.950651in}}{\pgfqpoint{1.416407in}{1.944827in}}%
\pgfpathcurveto{\pgfqpoint{1.422231in}{1.939003in}}{\pgfqpoint{1.430131in}{1.935731in}}{\pgfqpoint{1.438367in}{1.935731in}}%
\pgfpathclose%
\pgfusepath{stroke,fill}%
\end{pgfscope}%
\begin{pgfscope}%
\pgfpathrectangle{\pgfqpoint{0.100000in}{0.212622in}}{\pgfqpoint{3.696000in}{3.696000in}}%
\pgfusepath{clip}%
\pgfsetbuttcap%
\pgfsetroundjoin%
\definecolor{currentfill}{rgb}{0.121569,0.466667,0.705882}%
\pgfsetfillcolor{currentfill}%
\pgfsetfillopacity{0.442497}%
\pgfsetlinewidth{1.003750pt}%
\definecolor{currentstroke}{rgb}{0.121569,0.466667,0.705882}%
\pgfsetstrokecolor{currentstroke}%
\pgfsetstrokeopacity{0.442497}%
\pgfsetdash{}{0pt}%
\pgfpathmoveto{\pgfqpoint{1.433790in}{1.928433in}}%
\pgfpathcurveto{\pgfqpoint{1.442026in}{1.928433in}}{\pgfqpoint{1.449926in}{1.931706in}}{\pgfqpoint{1.455750in}{1.937530in}}%
\pgfpathcurveto{\pgfqpoint{1.461574in}{1.943354in}}{\pgfqpoint{1.464846in}{1.951254in}}{\pgfqpoint{1.464846in}{1.959490in}}%
\pgfpathcurveto{\pgfqpoint{1.464846in}{1.967726in}}{\pgfqpoint{1.461574in}{1.975626in}}{\pgfqpoint{1.455750in}{1.981450in}}%
\pgfpathcurveto{\pgfqpoint{1.449926in}{1.987274in}}{\pgfqpoint{1.442026in}{1.990546in}}{\pgfqpoint{1.433790in}{1.990546in}}%
\pgfpathcurveto{\pgfqpoint{1.425554in}{1.990546in}}{\pgfqpoint{1.417653in}{1.987274in}}{\pgfqpoint{1.411830in}{1.981450in}}%
\pgfpathcurveto{\pgfqpoint{1.406006in}{1.975626in}}{\pgfqpoint{1.402733in}{1.967726in}}{\pgfqpoint{1.402733in}{1.959490in}}%
\pgfpathcurveto{\pgfqpoint{1.402733in}{1.951254in}}{\pgfqpoint{1.406006in}{1.943354in}}{\pgfqpoint{1.411830in}{1.937530in}}%
\pgfpathcurveto{\pgfqpoint{1.417653in}{1.931706in}}{\pgfqpoint{1.425554in}{1.928433in}}{\pgfqpoint{1.433790in}{1.928433in}}%
\pgfpathclose%
\pgfusepath{stroke,fill}%
\end{pgfscope}%
\begin{pgfscope}%
\pgfpathrectangle{\pgfqpoint{0.100000in}{0.212622in}}{\pgfqpoint{3.696000in}{3.696000in}}%
\pgfusepath{clip}%
\pgfsetbuttcap%
\pgfsetroundjoin%
\definecolor{currentfill}{rgb}{0.121569,0.466667,0.705882}%
\pgfsetfillcolor{currentfill}%
\pgfsetfillopacity{0.444235}%
\pgfsetlinewidth{1.003750pt}%
\definecolor{currentstroke}{rgb}{0.121569,0.466667,0.705882}%
\pgfsetstrokecolor{currentstroke}%
\pgfsetstrokeopacity{0.444235}%
\pgfsetdash{}{0pt}%
\pgfpathmoveto{\pgfqpoint{1.422990in}{1.918400in}}%
\pgfpathcurveto{\pgfqpoint{1.431226in}{1.918400in}}{\pgfqpoint{1.439126in}{1.921672in}}{\pgfqpoint{1.444950in}{1.927496in}}%
\pgfpathcurveto{\pgfqpoint{1.450774in}{1.933320in}}{\pgfqpoint{1.454046in}{1.941220in}}{\pgfqpoint{1.454046in}{1.949456in}}%
\pgfpathcurveto{\pgfqpoint{1.454046in}{1.957693in}}{\pgfqpoint{1.450774in}{1.965593in}}{\pgfqpoint{1.444950in}{1.971417in}}%
\pgfpathcurveto{\pgfqpoint{1.439126in}{1.977241in}}{\pgfqpoint{1.431226in}{1.980513in}}{\pgfqpoint{1.422990in}{1.980513in}}%
\pgfpathcurveto{\pgfqpoint{1.414754in}{1.980513in}}{\pgfqpoint{1.406854in}{1.977241in}}{\pgfqpoint{1.401030in}{1.971417in}}%
\pgfpathcurveto{\pgfqpoint{1.395206in}{1.965593in}}{\pgfqpoint{1.391933in}{1.957693in}}{\pgfqpoint{1.391933in}{1.949456in}}%
\pgfpathcurveto{\pgfqpoint{1.391933in}{1.941220in}}{\pgfqpoint{1.395206in}{1.933320in}}{\pgfqpoint{1.401030in}{1.927496in}}%
\pgfpathcurveto{\pgfqpoint{1.406854in}{1.921672in}}{\pgfqpoint{1.414754in}{1.918400in}}{\pgfqpoint{1.422990in}{1.918400in}}%
\pgfpathclose%
\pgfusepath{stroke,fill}%
\end{pgfscope}%
\begin{pgfscope}%
\pgfpathrectangle{\pgfqpoint{0.100000in}{0.212622in}}{\pgfqpoint{3.696000in}{3.696000in}}%
\pgfusepath{clip}%
\pgfsetbuttcap%
\pgfsetroundjoin%
\definecolor{currentfill}{rgb}{0.121569,0.466667,0.705882}%
\pgfsetfillcolor{currentfill}%
\pgfsetfillopacity{0.449105}%
\pgfsetlinewidth{1.003750pt}%
\definecolor{currentstroke}{rgb}{0.121569,0.466667,0.705882}%
\pgfsetstrokecolor{currentstroke}%
\pgfsetstrokeopacity{0.449105}%
\pgfsetdash{}{0pt}%
\pgfpathmoveto{\pgfqpoint{1.439418in}{1.929490in}}%
\pgfpathcurveto{\pgfqpoint{1.447654in}{1.929490in}}{\pgfqpoint{1.455554in}{1.932762in}}{\pgfqpoint{1.461378in}{1.938586in}}%
\pgfpathcurveto{\pgfqpoint{1.467202in}{1.944410in}}{\pgfqpoint{1.470474in}{1.952310in}}{\pgfqpoint{1.470474in}{1.960546in}}%
\pgfpathcurveto{\pgfqpoint{1.470474in}{1.968783in}}{\pgfqpoint{1.467202in}{1.976683in}}{\pgfqpoint{1.461378in}{1.982507in}}%
\pgfpathcurveto{\pgfqpoint{1.455554in}{1.988331in}}{\pgfqpoint{1.447654in}{1.991603in}}{\pgfqpoint{1.439418in}{1.991603in}}%
\pgfpathcurveto{\pgfqpoint{1.431181in}{1.991603in}}{\pgfqpoint{1.423281in}{1.988331in}}{\pgfqpoint{1.417457in}{1.982507in}}%
\pgfpathcurveto{\pgfqpoint{1.411634in}{1.976683in}}{\pgfqpoint{1.408361in}{1.968783in}}{\pgfqpoint{1.408361in}{1.960546in}}%
\pgfpathcurveto{\pgfqpoint{1.408361in}{1.952310in}}{\pgfqpoint{1.411634in}{1.944410in}}{\pgfqpoint{1.417457in}{1.938586in}}%
\pgfpathcurveto{\pgfqpoint{1.423281in}{1.932762in}}{\pgfqpoint{1.431181in}{1.929490in}}{\pgfqpoint{1.439418in}{1.929490in}}%
\pgfpathclose%
\pgfusepath{stroke,fill}%
\end{pgfscope}%
\begin{pgfscope}%
\pgfpathrectangle{\pgfqpoint{0.100000in}{0.212622in}}{\pgfqpoint{3.696000in}{3.696000in}}%
\pgfusepath{clip}%
\pgfsetbuttcap%
\pgfsetroundjoin%
\definecolor{currentfill}{rgb}{0.121569,0.466667,0.705882}%
\pgfsetfillcolor{currentfill}%
\pgfsetfillopacity{0.450414}%
\pgfsetlinewidth{1.003750pt}%
\definecolor{currentstroke}{rgb}{0.121569,0.466667,0.705882}%
\pgfsetstrokecolor{currentstroke}%
\pgfsetstrokeopacity{0.450414}%
\pgfsetdash{}{0pt}%
\pgfpathmoveto{\pgfqpoint{1.430871in}{1.923133in}}%
\pgfpathcurveto{\pgfqpoint{1.439107in}{1.923133in}}{\pgfqpoint{1.447007in}{1.926405in}}{\pgfqpoint{1.452831in}{1.932229in}}%
\pgfpathcurveto{\pgfqpoint{1.458655in}{1.938053in}}{\pgfqpoint{1.461927in}{1.945953in}}{\pgfqpoint{1.461927in}{1.954190in}}%
\pgfpathcurveto{\pgfqpoint{1.461927in}{1.962426in}}{\pgfqpoint{1.458655in}{1.970326in}}{\pgfqpoint{1.452831in}{1.976150in}}%
\pgfpathcurveto{\pgfqpoint{1.447007in}{1.981974in}}{\pgfqpoint{1.439107in}{1.985246in}}{\pgfqpoint{1.430871in}{1.985246in}}%
\pgfpathcurveto{\pgfqpoint{1.422634in}{1.985246in}}{\pgfqpoint{1.414734in}{1.981974in}}{\pgfqpoint{1.408910in}{1.976150in}}%
\pgfpathcurveto{\pgfqpoint{1.403086in}{1.970326in}}{\pgfqpoint{1.399814in}{1.962426in}}{\pgfqpoint{1.399814in}{1.954190in}}%
\pgfpathcurveto{\pgfqpoint{1.399814in}{1.945953in}}{\pgfqpoint{1.403086in}{1.938053in}}{\pgfqpoint{1.408910in}{1.932229in}}%
\pgfpathcurveto{\pgfqpoint{1.414734in}{1.926405in}}{\pgfqpoint{1.422634in}{1.923133in}}{\pgfqpoint{1.430871in}{1.923133in}}%
\pgfpathclose%
\pgfusepath{stroke,fill}%
\end{pgfscope}%
\begin{pgfscope}%
\pgfpathrectangle{\pgfqpoint{0.100000in}{0.212622in}}{\pgfqpoint{3.696000in}{3.696000in}}%
\pgfusepath{clip}%
\pgfsetbuttcap%
\pgfsetroundjoin%
\definecolor{currentfill}{rgb}{0.121569,0.466667,0.705882}%
\pgfsetfillcolor{currentfill}%
\pgfsetfillopacity{0.456671}%
\pgfsetlinewidth{1.003750pt}%
\definecolor{currentstroke}{rgb}{0.121569,0.466667,0.705882}%
\pgfsetstrokecolor{currentstroke}%
\pgfsetstrokeopacity{0.456671}%
\pgfsetdash{}{0pt}%
\pgfpathmoveto{\pgfqpoint{1.421266in}{1.908491in}}%
\pgfpathcurveto{\pgfqpoint{1.429502in}{1.908491in}}{\pgfqpoint{1.437402in}{1.911763in}}{\pgfqpoint{1.443226in}{1.917587in}}%
\pgfpathcurveto{\pgfqpoint{1.449050in}{1.923411in}}{\pgfqpoint{1.452322in}{1.931311in}}{\pgfqpoint{1.452322in}{1.939548in}}%
\pgfpathcurveto{\pgfqpoint{1.452322in}{1.947784in}}{\pgfqpoint{1.449050in}{1.955684in}}{\pgfqpoint{1.443226in}{1.961508in}}%
\pgfpathcurveto{\pgfqpoint{1.437402in}{1.967332in}}{\pgfqpoint{1.429502in}{1.970604in}}{\pgfqpoint{1.421266in}{1.970604in}}%
\pgfpathcurveto{\pgfqpoint{1.413029in}{1.970604in}}{\pgfqpoint{1.405129in}{1.967332in}}{\pgfqpoint{1.399305in}{1.961508in}}%
\pgfpathcurveto{\pgfqpoint{1.393481in}{1.955684in}}{\pgfqpoint{1.390209in}{1.947784in}}{\pgfqpoint{1.390209in}{1.939548in}}%
\pgfpathcurveto{\pgfqpoint{1.390209in}{1.931311in}}{\pgfqpoint{1.393481in}{1.923411in}}{\pgfqpoint{1.399305in}{1.917587in}}%
\pgfpathcurveto{\pgfqpoint{1.405129in}{1.911763in}}{\pgfqpoint{1.413029in}{1.908491in}}{\pgfqpoint{1.421266in}{1.908491in}}%
\pgfpathclose%
\pgfusepath{stroke,fill}%
\end{pgfscope}%
\begin{pgfscope}%
\pgfpathrectangle{\pgfqpoint{0.100000in}{0.212622in}}{\pgfqpoint{3.696000in}{3.696000in}}%
\pgfusepath{clip}%
\pgfsetbuttcap%
\pgfsetroundjoin%
\definecolor{currentfill}{rgb}{0.121569,0.466667,0.705882}%
\pgfsetfillcolor{currentfill}%
\pgfsetfillopacity{0.457792}%
\pgfsetlinewidth{1.003750pt}%
\definecolor{currentstroke}{rgb}{0.121569,0.466667,0.705882}%
\pgfsetstrokecolor{currentstroke}%
\pgfsetstrokeopacity{0.457792}%
\pgfsetdash{}{0pt}%
\pgfpathmoveto{\pgfqpoint{1.424625in}{1.914728in}}%
\pgfpathcurveto{\pgfqpoint{1.432861in}{1.914728in}}{\pgfqpoint{1.440761in}{1.918000in}}{\pgfqpoint{1.446585in}{1.923824in}}%
\pgfpathcurveto{\pgfqpoint{1.452409in}{1.929648in}}{\pgfqpoint{1.455682in}{1.937548in}}{\pgfqpoint{1.455682in}{1.945785in}}%
\pgfpathcurveto{\pgfqpoint{1.455682in}{1.954021in}}{\pgfqpoint{1.452409in}{1.961921in}}{\pgfqpoint{1.446585in}{1.967745in}}%
\pgfpathcurveto{\pgfqpoint{1.440761in}{1.973569in}}{\pgfqpoint{1.432861in}{1.976841in}}{\pgfqpoint{1.424625in}{1.976841in}}%
\pgfpathcurveto{\pgfqpoint{1.416389in}{1.976841in}}{\pgfqpoint{1.408489in}{1.973569in}}{\pgfqpoint{1.402665in}{1.967745in}}%
\pgfpathcurveto{\pgfqpoint{1.396841in}{1.961921in}}{\pgfqpoint{1.393569in}{1.954021in}}{\pgfqpoint{1.393569in}{1.945785in}}%
\pgfpathcurveto{\pgfqpoint{1.393569in}{1.937548in}}{\pgfqpoint{1.396841in}{1.929648in}}{\pgfqpoint{1.402665in}{1.923824in}}%
\pgfpathcurveto{\pgfqpoint{1.408489in}{1.918000in}}{\pgfqpoint{1.416389in}{1.914728in}}{\pgfqpoint{1.424625in}{1.914728in}}%
\pgfpathclose%
\pgfusepath{stroke,fill}%
\end{pgfscope}%
\begin{pgfscope}%
\pgfpathrectangle{\pgfqpoint{0.100000in}{0.212622in}}{\pgfqpoint{3.696000in}{3.696000in}}%
\pgfusepath{clip}%
\pgfsetbuttcap%
\pgfsetroundjoin%
\definecolor{currentfill}{rgb}{0.121569,0.466667,0.705882}%
\pgfsetfillcolor{currentfill}%
\pgfsetfillopacity{0.467406}%
\pgfsetlinewidth{1.003750pt}%
\definecolor{currentstroke}{rgb}{0.121569,0.466667,0.705882}%
\pgfsetstrokecolor{currentstroke}%
\pgfsetstrokeopacity{0.467406}%
\pgfsetdash{}{0pt}%
\pgfpathmoveto{\pgfqpoint{1.407029in}{1.901027in}}%
\pgfpathcurveto{\pgfqpoint{1.415265in}{1.901027in}}{\pgfqpoint{1.423165in}{1.904299in}}{\pgfqpoint{1.428989in}{1.910123in}}%
\pgfpathcurveto{\pgfqpoint{1.434813in}{1.915947in}}{\pgfqpoint{1.438085in}{1.923847in}}{\pgfqpoint{1.438085in}{1.932083in}}%
\pgfpathcurveto{\pgfqpoint{1.438085in}{1.940320in}}{\pgfqpoint{1.434813in}{1.948220in}}{\pgfqpoint{1.428989in}{1.954044in}}%
\pgfpathcurveto{\pgfqpoint{1.423165in}{1.959868in}}{\pgfqpoint{1.415265in}{1.963140in}}{\pgfqpoint{1.407029in}{1.963140in}}%
\pgfpathcurveto{\pgfqpoint{1.398792in}{1.963140in}}{\pgfqpoint{1.390892in}{1.959868in}}{\pgfqpoint{1.385068in}{1.954044in}}%
\pgfpathcurveto{\pgfqpoint{1.379244in}{1.948220in}}{\pgfqpoint{1.375972in}{1.940320in}}{\pgfqpoint{1.375972in}{1.932083in}}%
\pgfpathcurveto{\pgfqpoint{1.375972in}{1.923847in}}{\pgfqpoint{1.379244in}{1.915947in}}{\pgfqpoint{1.385068in}{1.910123in}}%
\pgfpathcurveto{\pgfqpoint{1.390892in}{1.904299in}}{\pgfqpoint{1.398792in}{1.901027in}}{\pgfqpoint{1.407029in}{1.901027in}}%
\pgfpathclose%
\pgfusepath{stroke,fill}%
\end{pgfscope}%
\begin{pgfscope}%
\pgfpathrectangle{\pgfqpoint{0.100000in}{0.212622in}}{\pgfqpoint{3.696000in}{3.696000in}}%
\pgfusepath{clip}%
\pgfsetbuttcap%
\pgfsetroundjoin%
\definecolor{currentfill}{rgb}{0.121569,0.466667,0.705882}%
\pgfsetfillcolor{currentfill}%
\pgfsetfillopacity{0.474934}%
\pgfsetlinewidth{1.003750pt}%
\definecolor{currentstroke}{rgb}{0.121569,0.466667,0.705882}%
\pgfsetstrokecolor{currentstroke}%
\pgfsetstrokeopacity{0.474934}%
\pgfsetdash{}{0pt}%
\pgfpathmoveto{\pgfqpoint{1.404508in}{1.916709in}}%
\pgfpathcurveto{\pgfqpoint{1.412744in}{1.916709in}}{\pgfqpoint{1.420645in}{1.919981in}}{\pgfqpoint{1.426468in}{1.925805in}}%
\pgfpathcurveto{\pgfqpoint{1.432292in}{1.931629in}}{\pgfqpoint{1.435565in}{1.939529in}}{\pgfqpoint{1.435565in}{1.947765in}}%
\pgfpathcurveto{\pgfqpoint{1.435565in}{1.956001in}}{\pgfqpoint{1.432292in}{1.963901in}}{\pgfqpoint{1.426468in}{1.969725in}}%
\pgfpathcurveto{\pgfqpoint{1.420645in}{1.975549in}}{\pgfqpoint{1.412744in}{1.978822in}}{\pgfqpoint{1.404508in}{1.978822in}}%
\pgfpathcurveto{\pgfqpoint{1.396272in}{1.978822in}}{\pgfqpoint{1.388372in}{1.975549in}}{\pgfqpoint{1.382548in}{1.969725in}}%
\pgfpathcurveto{\pgfqpoint{1.376724in}{1.963901in}}{\pgfqpoint{1.373452in}{1.956001in}}{\pgfqpoint{1.373452in}{1.947765in}}%
\pgfpathcurveto{\pgfqpoint{1.373452in}{1.939529in}}{\pgfqpoint{1.376724in}{1.931629in}}{\pgfqpoint{1.382548in}{1.925805in}}%
\pgfpathcurveto{\pgfqpoint{1.388372in}{1.919981in}}{\pgfqpoint{1.396272in}{1.916709in}}{\pgfqpoint{1.404508in}{1.916709in}}%
\pgfpathclose%
\pgfusepath{stroke,fill}%
\end{pgfscope}%
\begin{pgfscope}%
\pgfpathrectangle{\pgfqpoint{0.100000in}{0.212622in}}{\pgfqpoint{3.696000in}{3.696000in}}%
\pgfusepath{clip}%
\pgfsetbuttcap%
\pgfsetroundjoin%
\definecolor{currentfill}{rgb}{0.121569,0.466667,0.705882}%
\pgfsetfillcolor{currentfill}%
\pgfsetfillopacity{0.512435}%
\pgfsetlinewidth{1.003750pt}%
\definecolor{currentstroke}{rgb}{0.121569,0.466667,0.705882}%
\pgfsetstrokecolor{currentstroke}%
\pgfsetstrokeopacity{0.512435}%
\pgfsetdash{}{0pt}%
\pgfpathmoveto{\pgfqpoint{1.318523in}{1.836646in}}%
\pgfpathcurveto{\pgfqpoint{1.326759in}{1.836646in}}{\pgfqpoint{1.334659in}{1.839918in}}{\pgfqpoint{1.340483in}{1.845742in}}%
\pgfpathcurveto{\pgfqpoint{1.346307in}{1.851566in}}{\pgfqpoint{1.349579in}{1.859466in}}{\pgfqpoint{1.349579in}{1.867702in}}%
\pgfpathcurveto{\pgfqpoint{1.349579in}{1.875939in}}{\pgfqpoint{1.346307in}{1.883839in}}{\pgfqpoint{1.340483in}{1.889663in}}%
\pgfpathcurveto{\pgfqpoint{1.334659in}{1.895486in}}{\pgfqpoint{1.326759in}{1.898759in}}{\pgfqpoint{1.318523in}{1.898759in}}%
\pgfpathcurveto{\pgfqpoint{1.310287in}{1.898759in}}{\pgfqpoint{1.302387in}{1.895486in}}{\pgfqpoint{1.296563in}{1.889663in}}%
\pgfpathcurveto{\pgfqpoint{1.290739in}{1.883839in}}{\pgfqpoint{1.287466in}{1.875939in}}{\pgfqpoint{1.287466in}{1.867702in}}%
\pgfpathcurveto{\pgfqpoint{1.287466in}{1.859466in}}{\pgfqpoint{1.290739in}{1.851566in}}{\pgfqpoint{1.296563in}{1.845742in}}%
\pgfpathcurveto{\pgfqpoint{1.302387in}{1.839918in}}{\pgfqpoint{1.310287in}{1.836646in}}{\pgfqpoint{1.318523in}{1.836646in}}%
\pgfpathclose%
\pgfusepath{stroke,fill}%
\end{pgfscope}%
\begin{pgfscope}%
\pgfpathrectangle{\pgfqpoint{0.100000in}{0.212622in}}{\pgfqpoint{3.696000in}{3.696000in}}%
\pgfusepath{clip}%
\pgfsetbuttcap%
\pgfsetroundjoin%
\definecolor{currentfill}{rgb}{0.121569,0.466667,0.705882}%
\pgfsetfillcolor{currentfill}%
\pgfsetfillopacity{0.517644}%
\pgfsetlinewidth{1.003750pt}%
\definecolor{currentstroke}{rgb}{0.121569,0.466667,0.705882}%
\pgfsetstrokecolor{currentstroke}%
\pgfsetstrokeopacity{0.517644}%
\pgfsetdash{}{0pt}%
\pgfpathmoveto{\pgfqpoint{1.312766in}{1.834521in}}%
\pgfpathcurveto{\pgfqpoint{1.321003in}{1.834521in}}{\pgfqpoint{1.328903in}{1.837794in}}{\pgfqpoint{1.334727in}{1.843618in}}%
\pgfpathcurveto{\pgfqpoint{1.340551in}{1.849442in}}{\pgfqpoint{1.343823in}{1.857342in}}{\pgfqpoint{1.343823in}{1.865578in}}%
\pgfpathcurveto{\pgfqpoint{1.343823in}{1.873814in}}{\pgfqpoint{1.340551in}{1.881714in}}{\pgfqpoint{1.334727in}{1.887538in}}%
\pgfpathcurveto{\pgfqpoint{1.328903in}{1.893362in}}{\pgfqpoint{1.321003in}{1.896634in}}{\pgfqpoint{1.312766in}{1.896634in}}%
\pgfpathcurveto{\pgfqpoint{1.304530in}{1.896634in}}{\pgfqpoint{1.296630in}{1.893362in}}{\pgfqpoint{1.290806in}{1.887538in}}%
\pgfpathcurveto{\pgfqpoint{1.284982in}{1.881714in}}{\pgfqpoint{1.281710in}{1.873814in}}{\pgfqpoint{1.281710in}{1.865578in}}%
\pgfpathcurveto{\pgfqpoint{1.281710in}{1.857342in}}{\pgfqpoint{1.284982in}{1.849442in}}{\pgfqpoint{1.290806in}{1.843618in}}%
\pgfpathcurveto{\pgfqpoint{1.296630in}{1.837794in}}{\pgfqpoint{1.304530in}{1.834521in}}{\pgfqpoint{1.312766in}{1.834521in}}%
\pgfpathclose%
\pgfusepath{stroke,fill}%
\end{pgfscope}%
\begin{pgfscope}%
\pgfpathrectangle{\pgfqpoint{0.100000in}{0.212622in}}{\pgfqpoint{3.696000in}{3.696000in}}%
\pgfusepath{clip}%
\pgfsetbuttcap%
\pgfsetroundjoin%
\definecolor{currentfill}{rgb}{0.121569,0.466667,0.705882}%
\pgfsetfillcolor{currentfill}%
\pgfsetfillopacity{0.531870}%
\pgfsetlinewidth{1.003750pt}%
\definecolor{currentstroke}{rgb}{0.121569,0.466667,0.705882}%
\pgfsetstrokecolor{currentstroke}%
\pgfsetstrokeopacity{0.531870}%
\pgfsetdash{}{0pt}%
\pgfpathmoveto{\pgfqpoint{1.284819in}{1.809282in}}%
\pgfpathcurveto{\pgfqpoint{1.293056in}{1.809282in}}{\pgfqpoint{1.300956in}{1.812554in}}{\pgfqpoint{1.306780in}{1.818378in}}%
\pgfpathcurveto{\pgfqpoint{1.312604in}{1.824202in}}{\pgfqpoint{1.315876in}{1.832102in}}{\pgfqpoint{1.315876in}{1.840338in}}%
\pgfpathcurveto{\pgfqpoint{1.315876in}{1.848575in}}{\pgfqpoint{1.312604in}{1.856475in}}{\pgfqpoint{1.306780in}{1.862299in}}%
\pgfpathcurveto{\pgfqpoint{1.300956in}{1.868123in}}{\pgfqpoint{1.293056in}{1.871395in}}{\pgfqpoint{1.284819in}{1.871395in}}%
\pgfpathcurveto{\pgfqpoint{1.276583in}{1.871395in}}{\pgfqpoint{1.268683in}{1.868123in}}{\pgfqpoint{1.262859in}{1.862299in}}%
\pgfpathcurveto{\pgfqpoint{1.257035in}{1.856475in}}{\pgfqpoint{1.253763in}{1.848575in}}{\pgfqpoint{1.253763in}{1.840338in}}%
\pgfpathcurveto{\pgfqpoint{1.253763in}{1.832102in}}{\pgfqpoint{1.257035in}{1.824202in}}{\pgfqpoint{1.262859in}{1.818378in}}%
\pgfpathcurveto{\pgfqpoint{1.268683in}{1.812554in}}{\pgfqpoint{1.276583in}{1.809282in}}{\pgfqpoint{1.284819in}{1.809282in}}%
\pgfpathclose%
\pgfusepath{stroke,fill}%
\end{pgfscope}%
\begin{pgfscope}%
\pgfpathrectangle{\pgfqpoint{0.100000in}{0.212622in}}{\pgfqpoint{3.696000in}{3.696000in}}%
\pgfusepath{clip}%
\pgfsetbuttcap%
\pgfsetroundjoin%
\definecolor{currentfill}{rgb}{0.121569,0.466667,0.705882}%
\pgfsetfillcolor{currentfill}%
\pgfsetfillopacity{0.539235}%
\pgfsetlinewidth{1.003750pt}%
\definecolor{currentstroke}{rgb}{0.121569,0.466667,0.705882}%
\pgfsetstrokecolor{currentstroke}%
\pgfsetstrokeopacity{0.539235}%
\pgfsetdash{}{0pt}%
\pgfpathmoveto{\pgfqpoint{1.273022in}{1.804666in}}%
\pgfpathcurveto{\pgfqpoint{1.281258in}{1.804666in}}{\pgfqpoint{1.289158in}{1.807939in}}{\pgfqpoint{1.294982in}{1.813762in}}%
\pgfpathcurveto{\pgfqpoint{1.300806in}{1.819586in}}{\pgfqpoint{1.304078in}{1.827486in}}{\pgfqpoint{1.304078in}{1.835723in}}%
\pgfpathcurveto{\pgfqpoint{1.304078in}{1.843959in}}{\pgfqpoint{1.300806in}{1.851859in}}{\pgfqpoint{1.294982in}{1.857683in}}%
\pgfpathcurveto{\pgfqpoint{1.289158in}{1.863507in}}{\pgfqpoint{1.281258in}{1.866779in}}{\pgfqpoint{1.273022in}{1.866779in}}%
\pgfpathcurveto{\pgfqpoint{1.264785in}{1.866779in}}{\pgfqpoint{1.256885in}{1.863507in}}{\pgfqpoint{1.251061in}{1.857683in}}%
\pgfpathcurveto{\pgfqpoint{1.245237in}{1.851859in}}{\pgfqpoint{1.241965in}{1.843959in}}{\pgfqpoint{1.241965in}{1.835723in}}%
\pgfpathcurveto{\pgfqpoint{1.241965in}{1.827486in}}{\pgfqpoint{1.245237in}{1.819586in}}{\pgfqpoint{1.251061in}{1.813762in}}%
\pgfpathcurveto{\pgfqpoint{1.256885in}{1.807939in}}{\pgfqpoint{1.264785in}{1.804666in}}{\pgfqpoint{1.273022in}{1.804666in}}%
\pgfpathclose%
\pgfusepath{stroke,fill}%
\end{pgfscope}%
\begin{pgfscope}%
\pgfpathrectangle{\pgfqpoint{0.100000in}{0.212622in}}{\pgfqpoint{3.696000in}{3.696000in}}%
\pgfusepath{clip}%
\pgfsetbuttcap%
\pgfsetroundjoin%
\definecolor{currentfill}{rgb}{0.121569,0.466667,0.705882}%
\pgfsetfillcolor{currentfill}%
\pgfsetfillopacity{0.543724}%
\pgfsetlinewidth{1.003750pt}%
\definecolor{currentstroke}{rgb}{0.121569,0.466667,0.705882}%
\pgfsetstrokecolor{currentstroke}%
\pgfsetstrokeopacity{0.543724}%
\pgfsetdash{}{0pt}%
\pgfpathmoveto{\pgfqpoint{1.265447in}{1.801236in}}%
\pgfpathcurveto{\pgfqpoint{1.273683in}{1.801236in}}{\pgfqpoint{1.281583in}{1.804508in}}{\pgfqpoint{1.287407in}{1.810332in}}%
\pgfpathcurveto{\pgfqpoint{1.293231in}{1.816156in}}{\pgfqpoint{1.296503in}{1.824056in}}{\pgfqpoint{1.296503in}{1.832293in}}%
\pgfpathcurveto{\pgfqpoint{1.296503in}{1.840529in}}{\pgfqpoint{1.293231in}{1.848429in}}{\pgfqpoint{1.287407in}{1.854253in}}%
\pgfpathcurveto{\pgfqpoint{1.281583in}{1.860077in}}{\pgfqpoint{1.273683in}{1.863349in}}{\pgfqpoint{1.265447in}{1.863349in}}%
\pgfpathcurveto{\pgfqpoint{1.257210in}{1.863349in}}{\pgfqpoint{1.249310in}{1.860077in}}{\pgfqpoint{1.243486in}{1.854253in}}%
\pgfpathcurveto{\pgfqpoint{1.237662in}{1.848429in}}{\pgfqpoint{1.234390in}{1.840529in}}{\pgfqpoint{1.234390in}{1.832293in}}%
\pgfpathcurveto{\pgfqpoint{1.234390in}{1.824056in}}{\pgfqpoint{1.237662in}{1.816156in}}{\pgfqpoint{1.243486in}{1.810332in}}%
\pgfpathcurveto{\pgfqpoint{1.249310in}{1.804508in}}{\pgfqpoint{1.257210in}{1.801236in}}{\pgfqpoint{1.265447in}{1.801236in}}%
\pgfpathclose%
\pgfusepath{stroke,fill}%
\end{pgfscope}%
\begin{pgfscope}%
\pgfpathrectangle{\pgfqpoint{0.100000in}{0.212622in}}{\pgfqpoint{3.696000in}{3.696000in}}%
\pgfusepath{clip}%
\pgfsetbuttcap%
\pgfsetroundjoin%
\definecolor{currentfill}{rgb}{0.121569,0.466667,0.705882}%
\pgfsetfillcolor{currentfill}%
\pgfsetfillopacity{0.553548}%
\pgfsetlinewidth{1.003750pt}%
\definecolor{currentstroke}{rgb}{0.121569,0.466667,0.705882}%
\pgfsetstrokecolor{currentstroke}%
\pgfsetstrokeopacity{0.553548}%
\pgfsetdash{}{0pt}%
\pgfpathmoveto{\pgfqpoint{1.247664in}{1.790147in}}%
\pgfpathcurveto{\pgfqpoint{1.255900in}{1.790147in}}{\pgfqpoint{1.263800in}{1.793420in}}{\pgfqpoint{1.269624in}{1.799243in}}%
\pgfpathcurveto{\pgfqpoint{1.275448in}{1.805067in}}{\pgfqpoint{1.278720in}{1.812967in}}{\pgfqpoint{1.278720in}{1.821204in}}%
\pgfpathcurveto{\pgfqpoint{1.278720in}{1.829440in}}{\pgfqpoint{1.275448in}{1.837340in}}{\pgfqpoint{1.269624in}{1.843164in}}%
\pgfpathcurveto{\pgfqpoint{1.263800in}{1.848988in}}{\pgfqpoint{1.255900in}{1.852260in}}{\pgfqpoint{1.247664in}{1.852260in}}%
\pgfpathcurveto{\pgfqpoint{1.239427in}{1.852260in}}{\pgfqpoint{1.231527in}{1.848988in}}{\pgfqpoint{1.225704in}{1.843164in}}%
\pgfpathcurveto{\pgfqpoint{1.219880in}{1.837340in}}{\pgfqpoint{1.216607in}{1.829440in}}{\pgfqpoint{1.216607in}{1.821204in}}%
\pgfpathcurveto{\pgfqpoint{1.216607in}{1.812967in}}{\pgfqpoint{1.219880in}{1.805067in}}{\pgfqpoint{1.225704in}{1.799243in}}%
\pgfpathcurveto{\pgfqpoint{1.231527in}{1.793420in}}{\pgfqpoint{1.239427in}{1.790147in}}{\pgfqpoint{1.247664in}{1.790147in}}%
\pgfpathclose%
\pgfusepath{stroke,fill}%
\end{pgfscope}%
\begin{pgfscope}%
\pgfpathrectangle{\pgfqpoint{0.100000in}{0.212622in}}{\pgfqpoint{3.696000in}{3.696000in}}%
\pgfusepath{clip}%
\pgfsetbuttcap%
\pgfsetroundjoin%
\definecolor{currentfill}{rgb}{0.121569,0.466667,0.705882}%
\pgfsetfillcolor{currentfill}%
\pgfsetfillopacity{0.556023}%
\pgfsetlinewidth{1.003750pt}%
\definecolor{currentstroke}{rgb}{0.121569,0.466667,0.705882}%
\pgfsetstrokecolor{currentstroke}%
\pgfsetstrokeopacity{0.556023}%
\pgfsetdash{}{0pt}%
\pgfpathmoveto{\pgfqpoint{1.244498in}{1.787729in}}%
\pgfpathcurveto{\pgfqpoint{1.252734in}{1.787729in}}{\pgfqpoint{1.260634in}{1.791002in}}{\pgfqpoint{1.266458in}{1.796825in}}%
\pgfpathcurveto{\pgfqpoint{1.272282in}{1.802649in}}{\pgfqpoint{1.275554in}{1.810549in}}{\pgfqpoint{1.275554in}{1.818786in}}%
\pgfpathcurveto{\pgfqpoint{1.275554in}{1.827022in}}{\pgfqpoint{1.272282in}{1.834922in}}{\pgfqpoint{1.266458in}{1.840746in}}%
\pgfpathcurveto{\pgfqpoint{1.260634in}{1.846570in}}{\pgfqpoint{1.252734in}{1.849842in}}{\pgfqpoint{1.244498in}{1.849842in}}%
\pgfpathcurveto{\pgfqpoint{1.236262in}{1.849842in}}{\pgfqpoint{1.228362in}{1.846570in}}{\pgfqpoint{1.222538in}{1.840746in}}%
\pgfpathcurveto{\pgfqpoint{1.216714in}{1.834922in}}{\pgfqpoint{1.213441in}{1.827022in}}{\pgfqpoint{1.213441in}{1.818786in}}%
\pgfpathcurveto{\pgfqpoint{1.213441in}{1.810549in}}{\pgfqpoint{1.216714in}{1.802649in}}{\pgfqpoint{1.222538in}{1.796825in}}%
\pgfpathcurveto{\pgfqpoint{1.228362in}{1.791002in}}{\pgfqpoint{1.236262in}{1.787729in}}{\pgfqpoint{1.244498in}{1.787729in}}%
\pgfpathclose%
\pgfusepath{stroke,fill}%
\end{pgfscope}%
\begin{pgfscope}%
\pgfpathrectangle{\pgfqpoint{0.100000in}{0.212622in}}{\pgfqpoint{3.696000in}{3.696000in}}%
\pgfusepath{clip}%
\pgfsetbuttcap%
\pgfsetroundjoin%
\definecolor{currentfill}{rgb}{0.121569,0.466667,0.705882}%
\pgfsetfillcolor{currentfill}%
\pgfsetfillopacity{0.558868}%
\pgfsetlinewidth{1.003750pt}%
\definecolor{currentstroke}{rgb}{0.121569,0.466667,0.705882}%
\pgfsetstrokecolor{currentstroke}%
\pgfsetstrokeopacity{0.558868}%
\pgfsetdash{}{0pt}%
\pgfpathmoveto{\pgfqpoint{1.242420in}{1.788727in}}%
\pgfpathcurveto{\pgfqpoint{1.250656in}{1.788727in}}{\pgfqpoint{1.258556in}{1.792000in}}{\pgfqpoint{1.264380in}{1.797824in}}%
\pgfpathcurveto{\pgfqpoint{1.270204in}{1.803648in}}{\pgfqpoint{1.273476in}{1.811548in}}{\pgfqpoint{1.273476in}{1.819784in}}%
\pgfpathcurveto{\pgfqpoint{1.273476in}{1.828020in}}{\pgfqpoint{1.270204in}{1.835920in}}{\pgfqpoint{1.264380in}{1.841744in}}%
\pgfpathcurveto{\pgfqpoint{1.258556in}{1.847568in}}{\pgfqpoint{1.250656in}{1.850840in}}{\pgfqpoint{1.242420in}{1.850840in}}%
\pgfpathcurveto{\pgfqpoint{1.234183in}{1.850840in}}{\pgfqpoint{1.226283in}{1.847568in}}{\pgfqpoint{1.220459in}{1.841744in}}%
\pgfpathcurveto{\pgfqpoint{1.214636in}{1.835920in}}{\pgfqpoint{1.211363in}{1.828020in}}{\pgfqpoint{1.211363in}{1.819784in}}%
\pgfpathcurveto{\pgfqpoint{1.211363in}{1.811548in}}{\pgfqpoint{1.214636in}{1.803648in}}{\pgfqpoint{1.220459in}{1.797824in}}%
\pgfpathcurveto{\pgfqpoint{1.226283in}{1.792000in}}{\pgfqpoint{1.234183in}{1.788727in}}{\pgfqpoint{1.242420in}{1.788727in}}%
\pgfpathclose%
\pgfusepath{stroke,fill}%
\end{pgfscope}%
\begin{pgfscope}%
\pgfpathrectangle{\pgfqpoint{0.100000in}{0.212622in}}{\pgfqpoint{3.696000in}{3.696000in}}%
\pgfusepath{clip}%
\pgfsetbuttcap%
\pgfsetroundjoin%
\definecolor{currentfill}{rgb}{0.121569,0.466667,0.705882}%
\pgfsetfillcolor{currentfill}%
\pgfsetfillopacity{0.561025}%
\pgfsetlinewidth{1.003750pt}%
\definecolor{currentstroke}{rgb}{0.121569,0.466667,0.705882}%
\pgfsetstrokecolor{currentstroke}%
\pgfsetstrokeopacity{0.561025}%
\pgfsetdash{}{0pt}%
\pgfpathmoveto{\pgfqpoint{1.244516in}{1.793960in}}%
\pgfpathcurveto{\pgfqpoint{1.252753in}{1.793960in}}{\pgfqpoint{1.260653in}{1.797232in}}{\pgfqpoint{1.266477in}{1.803056in}}%
\pgfpathcurveto{\pgfqpoint{1.272301in}{1.808880in}}{\pgfqpoint{1.275573in}{1.816780in}}{\pgfqpoint{1.275573in}{1.825017in}}%
\pgfpathcurveto{\pgfqpoint{1.275573in}{1.833253in}}{\pgfqpoint{1.272301in}{1.841153in}}{\pgfqpoint{1.266477in}{1.846977in}}%
\pgfpathcurveto{\pgfqpoint{1.260653in}{1.852801in}}{\pgfqpoint{1.252753in}{1.856073in}}{\pgfqpoint{1.244516in}{1.856073in}}%
\pgfpathcurveto{\pgfqpoint{1.236280in}{1.856073in}}{\pgfqpoint{1.228380in}{1.852801in}}{\pgfqpoint{1.222556in}{1.846977in}}%
\pgfpathcurveto{\pgfqpoint{1.216732in}{1.841153in}}{\pgfqpoint{1.213460in}{1.833253in}}{\pgfqpoint{1.213460in}{1.825017in}}%
\pgfpathcurveto{\pgfqpoint{1.213460in}{1.816780in}}{\pgfqpoint{1.216732in}{1.808880in}}{\pgfqpoint{1.222556in}{1.803056in}}%
\pgfpathcurveto{\pgfqpoint{1.228380in}{1.797232in}}{\pgfqpoint{1.236280in}{1.793960in}}{\pgfqpoint{1.244516in}{1.793960in}}%
\pgfpathclose%
\pgfusepath{stroke,fill}%
\end{pgfscope}%
\begin{pgfscope}%
\pgfpathrectangle{\pgfqpoint{0.100000in}{0.212622in}}{\pgfqpoint{3.696000in}{3.696000in}}%
\pgfusepath{clip}%
\pgfsetbuttcap%
\pgfsetroundjoin%
\definecolor{currentfill}{rgb}{0.121569,0.466667,0.705882}%
\pgfsetfillcolor{currentfill}%
\pgfsetfillopacity{0.561042}%
\pgfsetlinewidth{1.003750pt}%
\definecolor{currentstroke}{rgb}{0.121569,0.466667,0.705882}%
\pgfsetstrokecolor{currentstroke}%
\pgfsetstrokeopacity{0.561042}%
\pgfsetdash{}{0pt}%
\pgfpathmoveto{\pgfqpoint{1.248704in}{1.799711in}}%
\pgfpathcurveto{\pgfqpoint{1.256940in}{1.799711in}}{\pgfqpoint{1.264840in}{1.802983in}}{\pgfqpoint{1.270664in}{1.808807in}}%
\pgfpathcurveto{\pgfqpoint{1.276488in}{1.814631in}}{\pgfqpoint{1.279760in}{1.822531in}}{\pgfqpoint{1.279760in}{1.830767in}}%
\pgfpathcurveto{\pgfqpoint{1.279760in}{1.839003in}}{\pgfqpoint{1.276488in}{1.846903in}}{\pgfqpoint{1.270664in}{1.852727in}}%
\pgfpathcurveto{\pgfqpoint{1.264840in}{1.858551in}}{\pgfqpoint{1.256940in}{1.861824in}}{\pgfqpoint{1.248704in}{1.861824in}}%
\pgfpathcurveto{\pgfqpoint{1.240468in}{1.861824in}}{\pgfqpoint{1.232567in}{1.858551in}}{\pgfqpoint{1.226744in}{1.852727in}}%
\pgfpathcurveto{\pgfqpoint{1.220920in}{1.846903in}}{\pgfqpoint{1.217647in}{1.839003in}}{\pgfqpoint{1.217647in}{1.830767in}}%
\pgfpathcurveto{\pgfqpoint{1.217647in}{1.822531in}}{\pgfqpoint{1.220920in}{1.814631in}}{\pgfqpoint{1.226744in}{1.808807in}}%
\pgfpathcurveto{\pgfqpoint{1.232567in}{1.802983in}}{\pgfqpoint{1.240468in}{1.799711in}}{\pgfqpoint{1.248704in}{1.799711in}}%
\pgfpathclose%
\pgfusepath{stroke,fill}%
\end{pgfscope}%
\begin{pgfscope}%
\pgfpathrectangle{\pgfqpoint{0.100000in}{0.212622in}}{\pgfqpoint{3.696000in}{3.696000in}}%
\pgfusepath{clip}%
\pgfsetbuttcap%
\pgfsetroundjoin%
\definecolor{currentfill}{rgb}{0.121569,0.466667,0.705882}%
\pgfsetfillcolor{currentfill}%
\pgfsetfillopacity{0.562523}%
\pgfsetlinewidth{1.003750pt}%
\definecolor{currentstroke}{rgb}{0.121569,0.466667,0.705882}%
\pgfsetstrokecolor{currentstroke}%
\pgfsetstrokeopacity{0.562523}%
\pgfsetdash{}{0pt}%
\pgfpathmoveto{\pgfqpoint{1.252152in}{1.807401in}}%
\pgfpathcurveto{\pgfqpoint{1.260388in}{1.807401in}}{\pgfqpoint{1.268288in}{1.810673in}}{\pgfqpoint{1.274112in}{1.816497in}}%
\pgfpathcurveto{\pgfqpoint{1.279936in}{1.822321in}}{\pgfqpoint{1.283208in}{1.830221in}}{\pgfqpoint{1.283208in}{1.838457in}}%
\pgfpathcurveto{\pgfqpoint{1.283208in}{1.846693in}}{\pgfqpoint{1.279936in}{1.854594in}}{\pgfqpoint{1.274112in}{1.860417in}}%
\pgfpathcurveto{\pgfqpoint{1.268288in}{1.866241in}}{\pgfqpoint{1.260388in}{1.869514in}}{\pgfqpoint{1.252152in}{1.869514in}}%
\pgfpathcurveto{\pgfqpoint{1.243915in}{1.869514in}}{\pgfqpoint{1.236015in}{1.866241in}}{\pgfqpoint{1.230191in}{1.860417in}}%
\pgfpathcurveto{\pgfqpoint{1.224367in}{1.854594in}}{\pgfqpoint{1.221095in}{1.846693in}}{\pgfqpoint{1.221095in}{1.838457in}}%
\pgfpathcurveto{\pgfqpoint{1.221095in}{1.830221in}}{\pgfqpoint{1.224367in}{1.822321in}}{\pgfqpoint{1.230191in}{1.816497in}}%
\pgfpathcurveto{\pgfqpoint{1.236015in}{1.810673in}}{\pgfqpoint{1.243915in}{1.807401in}}{\pgfqpoint{1.252152in}{1.807401in}}%
\pgfpathclose%
\pgfusepath{stroke,fill}%
\end{pgfscope}%
\begin{pgfscope}%
\pgfpathrectangle{\pgfqpoint{0.100000in}{0.212622in}}{\pgfqpoint{3.696000in}{3.696000in}}%
\pgfusepath{clip}%
\pgfsetbuttcap%
\pgfsetroundjoin%
\definecolor{currentfill}{rgb}{0.121569,0.466667,0.705882}%
\pgfsetfillcolor{currentfill}%
\pgfsetfillopacity{0.566048}%
\pgfsetlinewidth{1.003750pt}%
\definecolor{currentstroke}{rgb}{0.121569,0.466667,0.705882}%
\pgfsetstrokecolor{currentstroke}%
\pgfsetstrokeopacity{0.566048}%
\pgfsetdash{}{0pt}%
\pgfpathmoveto{\pgfqpoint{1.284153in}{1.850471in}}%
\pgfpathcurveto{\pgfqpoint{1.292390in}{1.850471in}}{\pgfqpoint{1.300290in}{1.853743in}}{\pgfqpoint{1.306113in}{1.859567in}}%
\pgfpathcurveto{\pgfqpoint{1.311937in}{1.865391in}}{\pgfqpoint{1.315210in}{1.873291in}}{\pgfqpoint{1.315210in}{1.881528in}}%
\pgfpathcurveto{\pgfqpoint{1.315210in}{1.889764in}}{\pgfqpoint{1.311937in}{1.897664in}}{\pgfqpoint{1.306113in}{1.903488in}}%
\pgfpathcurveto{\pgfqpoint{1.300290in}{1.909312in}}{\pgfqpoint{1.292390in}{1.912584in}}{\pgfqpoint{1.284153in}{1.912584in}}%
\pgfpathcurveto{\pgfqpoint{1.275917in}{1.912584in}}{\pgfqpoint{1.268017in}{1.909312in}}{\pgfqpoint{1.262193in}{1.903488in}}%
\pgfpathcurveto{\pgfqpoint{1.256369in}{1.897664in}}{\pgfqpoint{1.253097in}{1.889764in}}{\pgfqpoint{1.253097in}{1.881528in}}%
\pgfpathcurveto{\pgfqpoint{1.253097in}{1.873291in}}{\pgfqpoint{1.256369in}{1.865391in}}{\pgfqpoint{1.262193in}{1.859567in}}%
\pgfpathcurveto{\pgfqpoint{1.268017in}{1.853743in}}{\pgfqpoint{1.275917in}{1.850471in}}{\pgfqpoint{1.284153in}{1.850471in}}%
\pgfpathclose%
\pgfusepath{stroke,fill}%
\end{pgfscope}%
\begin{pgfscope}%
\pgfpathrectangle{\pgfqpoint{0.100000in}{0.212622in}}{\pgfqpoint{3.696000in}{3.696000in}}%
\pgfusepath{clip}%
\pgfsetbuttcap%
\pgfsetroundjoin%
\definecolor{currentfill}{rgb}{0.121569,0.466667,0.705882}%
\pgfsetfillcolor{currentfill}%
\pgfsetfillopacity{0.569795}%
\pgfsetlinewidth{1.003750pt}%
\definecolor{currentstroke}{rgb}{0.121569,0.466667,0.705882}%
\pgfsetstrokecolor{currentstroke}%
\pgfsetstrokeopacity{0.569795}%
\pgfsetdash{}{0pt}%
\pgfpathmoveto{\pgfqpoint{1.253102in}{1.812541in}}%
\pgfpathcurveto{\pgfqpoint{1.261338in}{1.812541in}}{\pgfqpoint{1.269238in}{1.815813in}}{\pgfqpoint{1.275062in}{1.821637in}}%
\pgfpathcurveto{\pgfqpoint{1.280886in}{1.827461in}}{\pgfqpoint{1.284158in}{1.835361in}}{\pgfqpoint{1.284158in}{1.843597in}}%
\pgfpathcurveto{\pgfqpoint{1.284158in}{1.851834in}}{\pgfqpoint{1.280886in}{1.859734in}}{\pgfqpoint{1.275062in}{1.865558in}}%
\pgfpathcurveto{\pgfqpoint{1.269238in}{1.871382in}}{\pgfqpoint{1.261338in}{1.874654in}}{\pgfqpoint{1.253102in}{1.874654in}}%
\pgfpathcurveto{\pgfqpoint{1.244865in}{1.874654in}}{\pgfqpoint{1.236965in}{1.871382in}}{\pgfqpoint{1.231141in}{1.865558in}}%
\pgfpathcurveto{\pgfqpoint{1.225317in}{1.859734in}}{\pgfqpoint{1.222045in}{1.851834in}}{\pgfqpoint{1.222045in}{1.843597in}}%
\pgfpathcurveto{\pgfqpoint{1.222045in}{1.835361in}}{\pgfqpoint{1.225317in}{1.827461in}}{\pgfqpoint{1.231141in}{1.821637in}}%
\pgfpathcurveto{\pgfqpoint{1.236965in}{1.815813in}}{\pgfqpoint{1.244865in}{1.812541in}}{\pgfqpoint{1.253102in}{1.812541in}}%
\pgfpathclose%
\pgfusepath{stroke,fill}%
\end{pgfscope}%
\begin{pgfscope}%
\pgfpathrectangle{\pgfqpoint{0.100000in}{0.212622in}}{\pgfqpoint{3.696000in}{3.696000in}}%
\pgfusepath{clip}%
\pgfsetbuttcap%
\pgfsetroundjoin%
\definecolor{currentfill}{rgb}{0.121569,0.466667,0.705882}%
\pgfsetfillcolor{currentfill}%
\pgfsetfillopacity{0.570476}%
\pgfsetlinewidth{1.003750pt}%
\definecolor{currentstroke}{rgb}{0.121569,0.466667,0.705882}%
\pgfsetstrokecolor{currentstroke}%
\pgfsetstrokeopacity{0.570476}%
\pgfsetdash{}{0pt}%
\pgfpathmoveto{\pgfqpoint{1.239544in}{1.796789in}}%
\pgfpathcurveto{\pgfqpoint{1.247781in}{1.796789in}}{\pgfqpoint{1.255681in}{1.800062in}}{\pgfqpoint{1.261505in}{1.805886in}}%
\pgfpathcurveto{\pgfqpoint{1.267329in}{1.811710in}}{\pgfqpoint{1.270601in}{1.819610in}}{\pgfqpoint{1.270601in}{1.827846in}}%
\pgfpathcurveto{\pgfqpoint{1.270601in}{1.836082in}}{\pgfqpoint{1.267329in}{1.843982in}}{\pgfqpoint{1.261505in}{1.849806in}}%
\pgfpathcurveto{\pgfqpoint{1.255681in}{1.855630in}}{\pgfqpoint{1.247781in}{1.858902in}}{\pgfqpoint{1.239544in}{1.858902in}}%
\pgfpathcurveto{\pgfqpoint{1.231308in}{1.858902in}}{\pgfqpoint{1.223408in}{1.855630in}}{\pgfqpoint{1.217584in}{1.849806in}}%
\pgfpathcurveto{\pgfqpoint{1.211760in}{1.843982in}}{\pgfqpoint{1.208488in}{1.836082in}}{\pgfqpoint{1.208488in}{1.827846in}}%
\pgfpathcurveto{\pgfqpoint{1.208488in}{1.819610in}}{\pgfqpoint{1.211760in}{1.811710in}}{\pgfqpoint{1.217584in}{1.805886in}}%
\pgfpathcurveto{\pgfqpoint{1.223408in}{1.800062in}}{\pgfqpoint{1.231308in}{1.796789in}}{\pgfqpoint{1.239544in}{1.796789in}}%
\pgfpathclose%
\pgfusepath{stroke,fill}%
\end{pgfscope}%
\begin{pgfscope}%
\pgfpathrectangle{\pgfqpoint{0.100000in}{0.212622in}}{\pgfqpoint{3.696000in}{3.696000in}}%
\pgfusepath{clip}%
\pgfsetbuttcap%
\pgfsetroundjoin%
\definecolor{currentfill}{rgb}{0.121569,0.466667,0.705882}%
\pgfsetfillcolor{currentfill}%
\pgfsetfillopacity{0.571924}%
\pgfsetlinewidth{1.003750pt}%
\definecolor{currentstroke}{rgb}{0.121569,0.466667,0.705882}%
\pgfsetstrokecolor{currentstroke}%
\pgfsetstrokeopacity{0.571924}%
\pgfsetdash{}{0pt}%
\pgfpathmoveto{\pgfqpoint{1.239700in}{1.796362in}}%
\pgfpathcurveto{\pgfqpoint{1.247936in}{1.796362in}}{\pgfqpoint{1.255836in}{1.799634in}}{\pgfqpoint{1.261660in}{1.805458in}}%
\pgfpathcurveto{\pgfqpoint{1.267484in}{1.811282in}}{\pgfqpoint{1.270757in}{1.819182in}}{\pgfqpoint{1.270757in}{1.827418in}}%
\pgfpathcurveto{\pgfqpoint{1.270757in}{1.835654in}}{\pgfqpoint{1.267484in}{1.843554in}}{\pgfqpoint{1.261660in}{1.849378in}}%
\pgfpathcurveto{\pgfqpoint{1.255836in}{1.855202in}}{\pgfqpoint{1.247936in}{1.858475in}}{\pgfqpoint{1.239700in}{1.858475in}}%
\pgfpathcurveto{\pgfqpoint{1.231464in}{1.858475in}}{\pgfqpoint{1.223564in}{1.855202in}}{\pgfqpoint{1.217740in}{1.849378in}}%
\pgfpathcurveto{\pgfqpoint{1.211916in}{1.843554in}}{\pgfqpoint{1.208644in}{1.835654in}}{\pgfqpoint{1.208644in}{1.827418in}}%
\pgfpathcurveto{\pgfqpoint{1.208644in}{1.819182in}}{\pgfqpoint{1.211916in}{1.811282in}}{\pgfqpoint{1.217740in}{1.805458in}}%
\pgfpathcurveto{\pgfqpoint{1.223564in}{1.799634in}}{\pgfqpoint{1.231464in}{1.796362in}}{\pgfqpoint{1.239700in}{1.796362in}}%
\pgfpathclose%
\pgfusepath{stroke,fill}%
\end{pgfscope}%
\begin{pgfscope}%
\pgfpathrectangle{\pgfqpoint{0.100000in}{0.212622in}}{\pgfqpoint{3.696000in}{3.696000in}}%
\pgfusepath{clip}%
\pgfsetbuttcap%
\pgfsetroundjoin%
\definecolor{currentfill}{rgb}{0.121569,0.466667,0.705882}%
\pgfsetfillcolor{currentfill}%
\pgfsetfillopacity{0.576861}%
\pgfsetlinewidth{1.003750pt}%
\definecolor{currentstroke}{rgb}{0.121569,0.466667,0.705882}%
\pgfsetstrokecolor{currentstroke}%
\pgfsetstrokeopacity{0.576861}%
\pgfsetdash{}{0pt}%
\pgfpathmoveto{\pgfqpoint{1.267504in}{1.832352in}}%
\pgfpathcurveto{\pgfqpoint{1.275741in}{1.832352in}}{\pgfqpoint{1.283641in}{1.835624in}}{\pgfqpoint{1.289465in}{1.841448in}}%
\pgfpathcurveto{\pgfqpoint{1.295288in}{1.847272in}}{\pgfqpoint{1.298561in}{1.855172in}}{\pgfqpoint{1.298561in}{1.863409in}}%
\pgfpathcurveto{\pgfqpoint{1.298561in}{1.871645in}}{\pgfqpoint{1.295288in}{1.879545in}}{\pgfqpoint{1.289465in}{1.885369in}}%
\pgfpathcurveto{\pgfqpoint{1.283641in}{1.891193in}}{\pgfqpoint{1.275741in}{1.894465in}}{\pgfqpoint{1.267504in}{1.894465in}}%
\pgfpathcurveto{\pgfqpoint{1.259268in}{1.894465in}}{\pgfqpoint{1.251368in}{1.891193in}}{\pgfqpoint{1.245544in}{1.885369in}}%
\pgfpathcurveto{\pgfqpoint{1.239720in}{1.879545in}}{\pgfqpoint{1.236448in}{1.871645in}}{\pgfqpoint{1.236448in}{1.863409in}}%
\pgfpathcurveto{\pgfqpoint{1.236448in}{1.855172in}}{\pgfqpoint{1.239720in}{1.847272in}}{\pgfqpoint{1.245544in}{1.841448in}}%
\pgfpathcurveto{\pgfqpoint{1.251368in}{1.835624in}}{\pgfqpoint{1.259268in}{1.832352in}}{\pgfqpoint{1.267504in}{1.832352in}}%
\pgfpathclose%
\pgfusepath{stroke,fill}%
\end{pgfscope}%
\begin{pgfscope}%
\pgfpathrectangle{\pgfqpoint{0.100000in}{0.212622in}}{\pgfqpoint{3.696000in}{3.696000in}}%
\pgfusepath{clip}%
\pgfsetbuttcap%
\pgfsetroundjoin%
\definecolor{currentfill}{rgb}{0.121569,0.466667,0.705882}%
\pgfsetfillcolor{currentfill}%
\pgfsetfillopacity{0.577882}%
\pgfsetlinewidth{1.003750pt}%
\definecolor{currentstroke}{rgb}{0.121569,0.466667,0.705882}%
\pgfsetstrokecolor{currentstroke}%
\pgfsetstrokeopacity{0.577882}%
\pgfsetdash{}{0pt}%
\pgfpathmoveto{\pgfqpoint{1.317381in}{1.895583in}}%
\pgfpathcurveto{\pgfqpoint{1.325617in}{1.895583in}}{\pgfqpoint{1.333517in}{1.898856in}}{\pgfqpoint{1.339341in}{1.904680in}}%
\pgfpathcurveto{\pgfqpoint{1.345165in}{1.910504in}}{\pgfqpoint{1.348437in}{1.918404in}}{\pgfqpoint{1.348437in}{1.926640in}}%
\pgfpathcurveto{\pgfqpoint{1.348437in}{1.934876in}}{\pgfqpoint{1.345165in}{1.942776in}}{\pgfqpoint{1.339341in}{1.948600in}}%
\pgfpathcurveto{\pgfqpoint{1.333517in}{1.954424in}}{\pgfqpoint{1.325617in}{1.957696in}}{\pgfqpoint{1.317381in}{1.957696in}}%
\pgfpathcurveto{\pgfqpoint{1.309145in}{1.957696in}}{\pgfqpoint{1.301245in}{1.954424in}}{\pgfqpoint{1.295421in}{1.948600in}}%
\pgfpathcurveto{\pgfqpoint{1.289597in}{1.942776in}}{\pgfqpoint{1.286324in}{1.934876in}}{\pgfqpoint{1.286324in}{1.926640in}}%
\pgfpathcurveto{\pgfqpoint{1.286324in}{1.918404in}}{\pgfqpoint{1.289597in}{1.910504in}}{\pgfqpoint{1.295421in}{1.904680in}}%
\pgfpathcurveto{\pgfqpoint{1.301245in}{1.898856in}}{\pgfqpoint{1.309145in}{1.895583in}}{\pgfqpoint{1.317381in}{1.895583in}}%
\pgfpathclose%
\pgfusepath{stroke,fill}%
\end{pgfscope}%
\begin{pgfscope}%
\pgfpathrectangle{\pgfqpoint{0.100000in}{0.212622in}}{\pgfqpoint{3.696000in}{3.696000in}}%
\pgfusepath{clip}%
\pgfsetbuttcap%
\pgfsetroundjoin%
\definecolor{currentfill}{rgb}{0.121569,0.466667,0.705882}%
\pgfsetfillcolor{currentfill}%
\pgfsetfillopacity{0.578362}%
\pgfsetlinewidth{1.003750pt}%
\definecolor{currentstroke}{rgb}{0.121569,0.466667,0.705882}%
\pgfsetstrokecolor{currentstroke}%
\pgfsetstrokeopacity{0.578362}%
\pgfsetdash{}{0pt}%
\pgfpathmoveto{\pgfqpoint{1.238220in}{1.795772in}}%
\pgfpathcurveto{\pgfqpoint{1.246456in}{1.795772in}}{\pgfqpoint{1.254356in}{1.799044in}}{\pgfqpoint{1.260180in}{1.804868in}}%
\pgfpathcurveto{\pgfqpoint{1.266004in}{1.810692in}}{\pgfqpoint{1.269276in}{1.818592in}}{\pgfqpoint{1.269276in}{1.826828in}}%
\pgfpathcurveto{\pgfqpoint{1.269276in}{1.835064in}}{\pgfqpoint{1.266004in}{1.842965in}}{\pgfqpoint{1.260180in}{1.848788in}}%
\pgfpathcurveto{\pgfqpoint{1.254356in}{1.854612in}}{\pgfqpoint{1.246456in}{1.857885in}}{\pgfqpoint{1.238220in}{1.857885in}}%
\pgfpathcurveto{\pgfqpoint{1.229984in}{1.857885in}}{\pgfqpoint{1.222084in}{1.854612in}}{\pgfqpoint{1.216260in}{1.848788in}}%
\pgfpathcurveto{\pgfqpoint{1.210436in}{1.842965in}}{\pgfqpoint{1.207163in}{1.835064in}}{\pgfqpoint{1.207163in}{1.826828in}}%
\pgfpathcurveto{\pgfqpoint{1.207163in}{1.818592in}}{\pgfqpoint{1.210436in}{1.810692in}}{\pgfqpoint{1.216260in}{1.804868in}}%
\pgfpathcurveto{\pgfqpoint{1.222084in}{1.799044in}}{\pgfqpoint{1.229984in}{1.795772in}}{\pgfqpoint{1.238220in}{1.795772in}}%
\pgfpathclose%
\pgfusepath{stroke,fill}%
\end{pgfscope}%
\begin{pgfscope}%
\pgfpathrectangle{\pgfqpoint{0.100000in}{0.212622in}}{\pgfqpoint{3.696000in}{3.696000in}}%
\pgfusepath{clip}%
\pgfsetbuttcap%
\pgfsetroundjoin%
\definecolor{currentfill}{rgb}{0.121569,0.466667,0.705882}%
\pgfsetfillcolor{currentfill}%
\pgfsetfillopacity{0.581067}%
\pgfsetlinewidth{1.003750pt}%
\definecolor{currentstroke}{rgb}{0.121569,0.466667,0.705882}%
\pgfsetstrokecolor{currentstroke}%
\pgfsetstrokeopacity{0.581067}%
\pgfsetdash{}{0pt}%
\pgfpathmoveto{\pgfqpoint{1.271821in}{1.841163in}}%
\pgfpathcurveto{\pgfqpoint{1.280057in}{1.841163in}}{\pgfqpoint{1.287957in}{1.844436in}}{\pgfqpoint{1.293781in}{1.850260in}}%
\pgfpathcurveto{\pgfqpoint{1.299605in}{1.856084in}}{\pgfqpoint{1.302877in}{1.863984in}}{\pgfqpoint{1.302877in}{1.872220in}}%
\pgfpathcurveto{\pgfqpoint{1.302877in}{1.880456in}}{\pgfqpoint{1.299605in}{1.888356in}}{\pgfqpoint{1.293781in}{1.894180in}}%
\pgfpathcurveto{\pgfqpoint{1.287957in}{1.900004in}}{\pgfqpoint{1.280057in}{1.903276in}}{\pgfqpoint{1.271821in}{1.903276in}}%
\pgfpathcurveto{\pgfqpoint{1.263584in}{1.903276in}}{\pgfqpoint{1.255684in}{1.900004in}}{\pgfqpoint{1.249860in}{1.894180in}}%
\pgfpathcurveto{\pgfqpoint{1.244036in}{1.888356in}}{\pgfqpoint{1.240764in}{1.880456in}}{\pgfqpoint{1.240764in}{1.872220in}}%
\pgfpathcurveto{\pgfqpoint{1.240764in}{1.863984in}}{\pgfqpoint{1.244036in}{1.856084in}}{\pgfqpoint{1.249860in}{1.850260in}}%
\pgfpathcurveto{\pgfqpoint{1.255684in}{1.844436in}}{\pgfqpoint{1.263584in}{1.841163in}}{\pgfqpoint{1.271821in}{1.841163in}}%
\pgfpathclose%
\pgfusepath{stroke,fill}%
\end{pgfscope}%
\begin{pgfscope}%
\pgfpathrectangle{\pgfqpoint{0.100000in}{0.212622in}}{\pgfqpoint{3.696000in}{3.696000in}}%
\pgfusepath{clip}%
\pgfsetbuttcap%
\pgfsetroundjoin%
\definecolor{currentfill}{rgb}{0.121569,0.466667,0.705882}%
\pgfsetfillcolor{currentfill}%
\pgfsetfillopacity{0.585267}%
\pgfsetlinewidth{1.003750pt}%
\definecolor{currentstroke}{rgb}{0.121569,0.466667,0.705882}%
\pgfsetstrokecolor{currentstroke}%
\pgfsetstrokeopacity{0.585267}%
\pgfsetdash{}{0pt}%
\pgfpathmoveto{\pgfqpoint{1.273516in}{1.837887in}}%
\pgfpathcurveto{\pgfqpoint{1.281753in}{1.837887in}}{\pgfqpoint{1.289653in}{1.841159in}}{\pgfqpoint{1.295477in}{1.846983in}}%
\pgfpathcurveto{\pgfqpoint{1.301301in}{1.852807in}}{\pgfqpoint{1.304573in}{1.860707in}}{\pgfqpoint{1.304573in}{1.868944in}}%
\pgfpathcurveto{\pgfqpoint{1.304573in}{1.877180in}}{\pgfqpoint{1.301301in}{1.885080in}}{\pgfqpoint{1.295477in}{1.890904in}}%
\pgfpathcurveto{\pgfqpoint{1.289653in}{1.896728in}}{\pgfqpoint{1.281753in}{1.900000in}}{\pgfqpoint{1.273516in}{1.900000in}}%
\pgfpathcurveto{\pgfqpoint{1.265280in}{1.900000in}}{\pgfqpoint{1.257380in}{1.896728in}}{\pgfqpoint{1.251556in}{1.890904in}}%
\pgfpathcurveto{\pgfqpoint{1.245732in}{1.885080in}}{\pgfqpoint{1.242460in}{1.877180in}}{\pgfqpoint{1.242460in}{1.868944in}}%
\pgfpathcurveto{\pgfqpoint{1.242460in}{1.860707in}}{\pgfqpoint{1.245732in}{1.852807in}}{\pgfqpoint{1.251556in}{1.846983in}}%
\pgfpathcurveto{\pgfqpoint{1.257380in}{1.841159in}}{\pgfqpoint{1.265280in}{1.837887in}}{\pgfqpoint{1.273516in}{1.837887in}}%
\pgfpathclose%
\pgfusepath{stroke,fill}%
\end{pgfscope}%
\begin{pgfscope}%
\pgfpathrectangle{\pgfqpoint{0.100000in}{0.212622in}}{\pgfqpoint{3.696000in}{3.696000in}}%
\pgfusepath{clip}%
\pgfsetbuttcap%
\pgfsetroundjoin%
\definecolor{currentfill}{rgb}{0.121569,0.466667,0.705882}%
\pgfsetfillcolor{currentfill}%
\pgfsetfillopacity{0.589746}%
\pgfsetlinewidth{1.003750pt}%
\definecolor{currentstroke}{rgb}{0.121569,0.466667,0.705882}%
\pgfsetstrokecolor{currentstroke}%
\pgfsetstrokeopacity{0.589746}%
\pgfsetdash{}{0pt}%
\pgfpathmoveto{\pgfqpoint{1.343534in}{1.915797in}}%
\pgfpathcurveto{\pgfqpoint{1.351770in}{1.915797in}}{\pgfqpoint{1.359670in}{1.919069in}}{\pgfqpoint{1.365494in}{1.924893in}}%
\pgfpathcurveto{\pgfqpoint{1.371318in}{1.930717in}}{\pgfqpoint{1.374590in}{1.938617in}}{\pgfqpoint{1.374590in}{1.946854in}}%
\pgfpathcurveto{\pgfqpoint{1.374590in}{1.955090in}}{\pgfqpoint{1.371318in}{1.962990in}}{\pgfqpoint{1.365494in}{1.968814in}}%
\pgfpathcurveto{\pgfqpoint{1.359670in}{1.974638in}}{\pgfqpoint{1.351770in}{1.977910in}}{\pgfqpoint{1.343534in}{1.977910in}}%
\pgfpathcurveto{\pgfqpoint{1.335298in}{1.977910in}}{\pgfqpoint{1.327398in}{1.974638in}}{\pgfqpoint{1.321574in}{1.968814in}}%
\pgfpathcurveto{\pgfqpoint{1.315750in}{1.962990in}}{\pgfqpoint{1.312477in}{1.955090in}}{\pgfqpoint{1.312477in}{1.946854in}}%
\pgfpathcurveto{\pgfqpoint{1.312477in}{1.938617in}}{\pgfqpoint{1.315750in}{1.930717in}}{\pgfqpoint{1.321574in}{1.924893in}}%
\pgfpathcurveto{\pgfqpoint{1.327398in}{1.919069in}}{\pgfqpoint{1.335298in}{1.915797in}}{\pgfqpoint{1.343534in}{1.915797in}}%
\pgfpathclose%
\pgfusepath{stroke,fill}%
\end{pgfscope}%
\begin{pgfscope}%
\pgfpathrectangle{\pgfqpoint{0.100000in}{0.212622in}}{\pgfqpoint{3.696000in}{3.696000in}}%
\pgfusepath{clip}%
\pgfsetbuttcap%
\pgfsetroundjoin%
\definecolor{currentfill}{rgb}{0.121569,0.466667,0.705882}%
\pgfsetfillcolor{currentfill}%
\pgfsetfillopacity{0.592582}%
\pgfsetlinewidth{1.003750pt}%
\definecolor{currentstroke}{rgb}{0.121569,0.466667,0.705882}%
\pgfsetstrokecolor{currentstroke}%
\pgfsetstrokeopacity{0.592582}%
\pgfsetdash{}{0pt}%
\pgfpathmoveto{\pgfqpoint{1.310141in}{1.887148in}}%
\pgfpathcurveto{\pgfqpoint{1.318377in}{1.887148in}}{\pgfqpoint{1.326277in}{1.890420in}}{\pgfqpoint{1.332101in}{1.896244in}}%
\pgfpathcurveto{\pgfqpoint{1.337925in}{1.902068in}}{\pgfqpoint{1.341197in}{1.909968in}}{\pgfqpoint{1.341197in}{1.918205in}}%
\pgfpathcurveto{\pgfqpoint{1.341197in}{1.926441in}}{\pgfqpoint{1.337925in}{1.934341in}}{\pgfqpoint{1.332101in}{1.940165in}}%
\pgfpathcurveto{\pgfqpoint{1.326277in}{1.945989in}}{\pgfqpoint{1.318377in}{1.949261in}}{\pgfqpoint{1.310141in}{1.949261in}}%
\pgfpathcurveto{\pgfqpoint{1.301904in}{1.949261in}}{\pgfqpoint{1.294004in}{1.945989in}}{\pgfqpoint{1.288180in}{1.940165in}}%
\pgfpathcurveto{\pgfqpoint{1.282356in}{1.934341in}}{\pgfqpoint{1.279084in}{1.926441in}}{\pgfqpoint{1.279084in}{1.918205in}}%
\pgfpathcurveto{\pgfqpoint{1.279084in}{1.909968in}}{\pgfqpoint{1.282356in}{1.902068in}}{\pgfqpoint{1.288180in}{1.896244in}}%
\pgfpathcurveto{\pgfqpoint{1.294004in}{1.890420in}}{\pgfqpoint{1.301904in}{1.887148in}}{\pgfqpoint{1.310141in}{1.887148in}}%
\pgfpathclose%
\pgfusepath{stroke,fill}%
\end{pgfscope}%
\begin{pgfscope}%
\pgfpathrectangle{\pgfqpoint{0.100000in}{0.212622in}}{\pgfqpoint{3.696000in}{3.696000in}}%
\pgfusepath{clip}%
\pgfsetbuttcap%
\pgfsetroundjoin%
\definecolor{currentfill}{rgb}{0.121569,0.466667,0.705882}%
\pgfsetfillcolor{currentfill}%
\pgfsetfillopacity{0.593374}%
\pgfsetlinewidth{1.003750pt}%
\definecolor{currentstroke}{rgb}{0.121569,0.466667,0.705882}%
\pgfsetstrokecolor{currentstroke}%
\pgfsetstrokeopacity{0.593374}%
\pgfsetdash{}{0pt}%
\pgfpathmoveto{\pgfqpoint{1.445296in}{2.010664in}}%
\pgfpathcurveto{\pgfqpoint{1.453532in}{2.010664in}}{\pgfqpoint{1.461432in}{2.013937in}}{\pgfqpoint{1.467256in}{2.019761in}}%
\pgfpathcurveto{\pgfqpoint{1.473080in}{2.025584in}}{\pgfqpoint{1.476352in}{2.033484in}}{\pgfqpoint{1.476352in}{2.041721in}}%
\pgfpathcurveto{\pgfqpoint{1.476352in}{2.049957in}}{\pgfqpoint{1.473080in}{2.057857in}}{\pgfqpoint{1.467256in}{2.063681in}}%
\pgfpathcurveto{\pgfqpoint{1.461432in}{2.069505in}}{\pgfqpoint{1.453532in}{2.072777in}}{\pgfqpoint{1.445296in}{2.072777in}}%
\pgfpathcurveto{\pgfqpoint{1.437060in}{2.072777in}}{\pgfqpoint{1.429159in}{2.069505in}}{\pgfqpoint{1.423336in}{2.063681in}}%
\pgfpathcurveto{\pgfqpoint{1.417512in}{2.057857in}}{\pgfqpoint{1.414239in}{2.049957in}}{\pgfqpoint{1.414239in}{2.041721in}}%
\pgfpathcurveto{\pgfqpoint{1.414239in}{2.033484in}}{\pgfqpoint{1.417512in}{2.025584in}}{\pgfqpoint{1.423336in}{2.019761in}}%
\pgfpathcurveto{\pgfqpoint{1.429159in}{2.013937in}}{\pgfqpoint{1.437060in}{2.010664in}}{\pgfqpoint{1.445296in}{2.010664in}}%
\pgfpathclose%
\pgfusepath{stroke,fill}%
\end{pgfscope}%
\begin{pgfscope}%
\pgfpathrectangle{\pgfqpoint{0.100000in}{0.212622in}}{\pgfqpoint{3.696000in}{3.696000in}}%
\pgfusepath{clip}%
\pgfsetbuttcap%
\pgfsetroundjoin%
\definecolor{currentfill}{rgb}{0.121569,0.466667,0.705882}%
\pgfsetfillcolor{currentfill}%
\pgfsetfillopacity{0.593562}%
\pgfsetlinewidth{1.003750pt}%
\definecolor{currentstroke}{rgb}{0.121569,0.466667,0.705882}%
\pgfsetstrokecolor{currentstroke}%
\pgfsetstrokeopacity{0.593562}%
\pgfsetdash{}{0pt}%
\pgfpathmoveto{\pgfqpoint{1.319301in}{1.893341in}}%
\pgfpathcurveto{\pgfqpoint{1.327537in}{1.893341in}}{\pgfqpoint{1.335437in}{1.896613in}}{\pgfqpoint{1.341261in}{1.902437in}}%
\pgfpathcurveto{\pgfqpoint{1.347085in}{1.908261in}}{\pgfqpoint{1.350357in}{1.916161in}}{\pgfqpoint{1.350357in}{1.924397in}}%
\pgfpathcurveto{\pgfqpoint{1.350357in}{1.932633in}}{\pgfqpoint{1.347085in}{1.940534in}}{\pgfqpoint{1.341261in}{1.946357in}}%
\pgfpathcurveto{\pgfqpoint{1.335437in}{1.952181in}}{\pgfqpoint{1.327537in}{1.955454in}}{\pgfqpoint{1.319301in}{1.955454in}}%
\pgfpathcurveto{\pgfqpoint{1.311064in}{1.955454in}}{\pgfqpoint{1.303164in}{1.952181in}}{\pgfqpoint{1.297340in}{1.946357in}}%
\pgfpathcurveto{\pgfqpoint{1.291516in}{1.940534in}}{\pgfqpoint{1.288244in}{1.932633in}}{\pgfqpoint{1.288244in}{1.924397in}}%
\pgfpathcurveto{\pgfqpoint{1.288244in}{1.916161in}}{\pgfqpoint{1.291516in}{1.908261in}}{\pgfqpoint{1.297340in}{1.902437in}}%
\pgfpathcurveto{\pgfqpoint{1.303164in}{1.896613in}}{\pgfqpoint{1.311064in}{1.893341in}}{\pgfqpoint{1.319301in}{1.893341in}}%
\pgfpathclose%
\pgfusepath{stroke,fill}%
\end{pgfscope}%
\begin{pgfscope}%
\pgfpathrectangle{\pgfqpoint{0.100000in}{0.212622in}}{\pgfqpoint{3.696000in}{3.696000in}}%
\pgfusepath{clip}%
\pgfsetbuttcap%
\pgfsetroundjoin%
\definecolor{currentfill}{rgb}{0.121569,0.466667,0.705882}%
\pgfsetfillcolor{currentfill}%
\pgfsetfillopacity{0.594378}%
\pgfsetlinewidth{1.003750pt}%
\definecolor{currentstroke}{rgb}{0.121569,0.466667,0.705882}%
\pgfsetstrokecolor{currentstroke}%
\pgfsetstrokeopacity{0.594378}%
\pgfsetdash{}{0pt}%
\pgfpathmoveto{\pgfqpoint{1.293100in}{1.867698in}}%
\pgfpathcurveto{\pgfqpoint{1.301336in}{1.867698in}}{\pgfqpoint{1.309236in}{1.870971in}}{\pgfqpoint{1.315060in}{1.876795in}}%
\pgfpathcurveto{\pgfqpoint{1.320884in}{1.882619in}}{\pgfqpoint{1.324157in}{1.890519in}}{\pgfqpoint{1.324157in}{1.898755in}}%
\pgfpathcurveto{\pgfqpoint{1.324157in}{1.906991in}}{\pgfqpoint{1.320884in}{1.914891in}}{\pgfqpoint{1.315060in}{1.920715in}}%
\pgfpathcurveto{\pgfqpoint{1.309236in}{1.926539in}}{\pgfqpoint{1.301336in}{1.929811in}}{\pgfqpoint{1.293100in}{1.929811in}}%
\pgfpathcurveto{\pgfqpoint{1.284864in}{1.929811in}}{\pgfqpoint{1.276964in}{1.926539in}}{\pgfqpoint{1.271140in}{1.920715in}}%
\pgfpathcurveto{\pgfqpoint{1.265316in}{1.914891in}}{\pgfqpoint{1.262044in}{1.906991in}}{\pgfqpoint{1.262044in}{1.898755in}}%
\pgfpathcurveto{\pgfqpoint{1.262044in}{1.890519in}}{\pgfqpoint{1.265316in}{1.882619in}}{\pgfqpoint{1.271140in}{1.876795in}}%
\pgfpathcurveto{\pgfqpoint{1.276964in}{1.870971in}}{\pgfqpoint{1.284864in}{1.867698in}}{\pgfqpoint{1.293100in}{1.867698in}}%
\pgfpathclose%
\pgfusepath{stroke,fill}%
\end{pgfscope}%
\begin{pgfscope}%
\pgfpathrectangle{\pgfqpoint{0.100000in}{0.212622in}}{\pgfqpoint{3.696000in}{3.696000in}}%
\pgfusepath{clip}%
\pgfsetbuttcap%
\pgfsetroundjoin%
\definecolor{currentfill}{rgb}{0.121569,0.466667,0.705882}%
\pgfsetfillcolor{currentfill}%
\pgfsetfillopacity{0.594442}%
\pgfsetlinewidth{1.003750pt}%
\definecolor{currentstroke}{rgb}{0.121569,0.466667,0.705882}%
\pgfsetstrokecolor{currentstroke}%
\pgfsetstrokeopacity{0.594442}%
\pgfsetdash{}{0pt}%
\pgfpathmoveto{\pgfqpoint{1.407142in}{1.990957in}}%
\pgfpathcurveto{\pgfqpoint{1.415378in}{1.990957in}}{\pgfqpoint{1.423278in}{1.994229in}}{\pgfqpoint{1.429102in}{2.000053in}}%
\pgfpathcurveto{\pgfqpoint{1.434926in}{2.005877in}}{\pgfqpoint{1.438198in}{2.013777in}}{\pgfqpoint{1.438198in}{2.022013in}}%
\pgfpathcurveto{\pgfqpoint{1.438198in}{2.030250in}}{\pgfqpoint{1.434926in}{2.038150in}}{\pgfqpoint{1.429102in}{2.043974in}}%
\pgfpathcurveto{\pgfqpoint{1.423278in}{2.049798in}}{\pgfqpoint{1.415378in}{2.053070in}}{\pgfqpoint{1.407142in}{2.053070in}}%
\pgfpathcurveto{\pgfqpoint{1.398905in}{2.053070in}}{\pgfqpoint{1.391005in}{2.049798in}}{\pgfqpoint{1.385181in}{2.043974in}}%
\pgfpathcurveto{\pgfqpoint{1.379357in}{2.038150in}}{\pgfqpoint{1.376085in}{2.030250in}}{\pgfqpoint{1.376085in}{2.022013in}}%
\pgfpathcurveto{\pgfqpoint{1.376085in}{2.013777in}}{\pgfqpoint{1.379357in}{2.005877in}}{\pgfqpoint{1.385181in}{2.000053in}}%
\pgfpathcurveto{\pgfqpoint{1.391005in}{1.994229in}}{\pgfqpoint{1.398905in}{1.990957in}}{\pgfqpoint{1.407142in}{1.990957in}}%
\pgfpathclose%
\pgfusepath{stroke,fill}%
\end{pgfscope}%
\begin{pgfscope}%
\pgfpathrectangle{\pgfqpoint{0.100000in}{0.212622in}}{\pgfqpoint{3.696000in}{3.696000in}}%
\pgfusepath{clip}%
\pgfsetbuttcap%
\pgfsetroundjoin%
\definecolor{currentfill}{rgb}{0.121569,0.466667,0.705882}%
\pgfsetfillcolor{currentfill}%
\pgfsetfillopacity{0.594890}%
\pgfsetlinewidth{1.003750pt}%
\definecolor{currentstroke}{rgb}{0.121569,0.466667,0.705882}%
\pgfsetstrokecolor{currentstroke}%
\pgfsetstrokeopacity{0.594890}%
\pgfsetdash{}{0pt}%
\pgfpathmoveto{\pgfqpoint{1.426296in}{1.990858in}}%
\pgfpathcurveto{\pgfqpoint{1.434533in}{1.990858in}}{\pgfqpoint{1.442433in}{1.994130in}}{\pgfqpoint{1.448257in}{1.999954in}}%
\pgfpathcurveto{\pgfqpoint{1.454081in}{2.005778in}}{\pgfqpoint{1.457353in}{2.013678in}}{\pgfqpoint{1.457353in}{2.021914in}}%
\pgfpathcurveto{\pgfqpoint{1.457353in}{2.030150in}}{\pgfqpoint{1.454081in}{2.038050in}}{\pgfqpoint{1.448257in}{2.043874in}}%
\pgfpathcurveto{\pgfqpoint{1.442433in}{2.049698in}}{\pgfqpoint{1.434533in}{2.052971in}}{\pgfqpoint{1.426296in}{2.052971in}}%
\pgfpathcurveto{\pgfqpoint{1.418060in}{2.052971in}}{\pgfqpoint{1.410160in}{2.049698in}}{\pgfqpoint{1.404336in}{2.043874in}}%
\pgfpathcurveto{\pgfqpoint{1.398512in}{2.038050in}}{\pgfqpoint{1.395240in}{2.030150in}}{\pgfqpoint{1.395240in}{2.021914in}}%
\pgfpathcurveto{\pgfqpoint{1.395240in}{2.013678in}}{\pgfqpoint{1.398512in}{2.005778in}}{\pgfqpoint{1.404336in}{1.999954in}}%
\pgfpathcurveto{\pgfqpoint{1.410160in}{1.994130in}}{\pgfqpoint{1.418060in}{1.990858in}}{\pgfqpoint{1.426296in}{1.990858in}}%
\pgfpathclose%
\pgfusepath{stroke,fill}%
\end{pgfscope}%
\begin{pgfscope}%
\pgfpathrectangle{\pgfqpoint{0.100000in}{0.212622in}}{\pgfqpoint{3.696000in}{3.696000in}}%
\pgfusepath{clip}%
\pgfsetbuttcap%
\pgfsetroundjoin%
\definecolor{currentfill}{rgb}{0.121569,0.466667,0.705882}%
\pgfsetfillcolor{currentfill}%
\pgfsetfillopacity{0.595090}%
\pgfsetlinewidth{1.003750pt}%
\definecolor{currentstroke}{rgb}{0.121569,0.466667,0.705882}%
\pgfsetstrokecolor{currentstroke}%
\pgfsetstrokeopacity{0.595090}%
\pgfsetdash{}{0pt}%
\pgfpathmoveto{\pgfqpoint{1.427521in}{1.991339in}}%
\pgfpathcurveto{\pgfqpoint{1.435757in}{1.991339in}}{\pgfqpoint{1.443657in}{1.994612in}}{\pgfqpoint{1.449481in}{2.000435in}}%
\pgfpathcurveto{\pgfqpoint{1.455305in}{2.006259in}}{\pgfqpoint{1.458577in}{2.014159in}}{\pgfqpoint{1.458577in}{2.022396in}}%
\pgfpathcurveto{\pgfqpoint{1.458577in}{2.030632in}}{\pgfqpoint{1.455305in}{2.038532in}}{\pgfqpoint{1.449481in}{2.044356in}}%
\pgfpathcurveto{\pgfqpoint{1.443657in}{2.050180in}}{\pgfqpoint{1.435757in}{2.053452in}}{\pgfqpoint{1.427521in}{2.053452in}}%
\pgfpathcurveto{\pgfqpoint{1.419284in}{2.053452in}}{\pgfqpoint{1.411384in}{2.050180in}}{\pgfqpoint{1.405560in}{2.044356in}}%
\pgfpathcurveto{\pgfqpoint{1.399736in}{2.038532in}}{\pgfqpoint{1.396464in}{2.030632in}}{\pgfqpoint{1.396464in}{2.022396in}}%
\pgfpathcurveto{\pgfqpoint{1.396464in}{2.014159in}}{\pgfqpoint{1.399736in}{2.006259in}}{\pgfqpoint{1.405560in}{2.000435in}}%
\pgfpathcurveto{\pgfqpoint{1.411384in}{1.994612in}}{\pgfqpoint{1.419284in}{1.991339in}}{\pgfqpoint{1.427521in}{1.991339in}}%
\pgfpathclose%
\pgfusepath{stroke,fill}%
\end{pgfscope}%
\begin{pgfscope}%
\pgfpathrectangle{\pgfqpoint{0.100000in}{0.212622in}}{\pgfqpoint{3.696000in}{3.696000in}}%
\pgfusepath{clip}%
\pgfsetbuttcap%
\pgfsetroundjoin%
\definecolor{currentfill}{rgb}{0.121569,0.466667,0.705882}%
\pgfsetfillcolor{currentfill}%
\pgfsetfillopacity{0.595194}%
\pgfsetlinewidth{1.003750pt}%
\definecolor{currentstroke}{rgb}{0.121569,0.466667,0.705882}%
\pgfsetstrokecolor{currentstroke}%
\pgfsetstrokeopacity{0.595194}%
\pgfsetdash{}{0pt}%
\pgfpathmoveto{\pgfqpoint{1.399091in}{1.982701in}}%
\pgfpathcurveto{\pgfqpoint{1.407327in}{1.982701in}}{\pgfqpoint{1.415227in}{1.985973in}}{\pgfqpoint{1.421051in}{1.991797in}}%
\pgfpathcurveto{\pgfqpoint{1.426875in}{1.997621in}}{\pgfqpoint{1.430147in}{2.005521in}}{\pgfqpoint{1.430147in}{2.013757in}}%
\pgfpathcurveto{\pgfqpoint{1.430147in}{2.021993in}}{\pgfqpoint{1.426875in}{2.029893in}}{\pgfqpoint{1.421051in}{2.035717in}}%
\pgfpathcurveto{\pgfqpoint{1.415227in}{2.041541in}}{\pgfqpoint{1.407327in}{2.044814in}}{\pgfqpoint{1.399091in}{2.044814in}}%
\pgfpathcurveto{\pgfqpoint{1.390854in}{2.044814in}}{\pgfqpoint{1.382954in}{2.041541in}}{\pgfqpoint{1.377130in}{2.035717in}}%
\pgfpathcurveto{\pgfqpoint{1.371306in}{2.029893in}}{\pgfqpoint{1.368034in}{2.021993in}}{\pgfqpoint{1.368034in}{2.013757in}}%
\pgfpathcurveto{\pgfqpoint{1.368034in}{2.005521in}}{\pgfqpoint{1.371306in}{1.997621in}}{\pgfqpoint{1.377130in}{1.991797in}}%
\pgfpathcurveto{\pgfqpoint{1.382954in}{1.985973in}}{\pgfqpoint{1.390854in}{1.982701in}}{\pgfqpoint{1.399091in}{1.982701in}}%
\pgfpathclose%
\pgfusepath{stroke,fill}%
\end{pgfscope}%
\begin{pgfscope}%
\pgfpathrectangle{\pgfqpoint{0.100000in}{0.212622in}}{\pgfqpoint{3.696000in}{3.696000in}}%
\pgfusepath{clip}%
\pgfsetbuttcap%
\pgfsetroundjoin%
\definecolor{currentfill}{rgb}{0.121569,0.466667,0.705882}%
\pgfsetfillcolor{currentfill}%
\pgfsetfillopacity{0.597004}%
\pgfsetlinewidth{1.003750pt}%
\definecolor{currentstroke}{rgb}{0.121569,0.466667,0.705882}%
\pgfsetstrokecolor{currentstroke}%
\pgfsetstrokeopacity{0.597004}%
\pgfsetdash{}{0pt}%
\pgfpathmoveto{\pgfqpoint{1.346482in}{1.933395in}}%
\pgfpathcurveto{\pgfqpoint{1.354718in}{1.933395in}}{\pgfqpoint{1.362618in}{1.936667in}}{\pgfqpoint{1.368442in}{1.942491in}}%
\pgfpathcurveto{\pgfqpoint{1.374266in}{1.948315in}}{\pgfqpoint{1.377538in}{1.956215in}}{\pgfqpoint{1.377538in}{1.964451in}}%
\pgfpathcurveto{\pgfqpoint{1.377538in}{1.972688in}}{\pgfqpoint{1.374266in}{1.980588in}}{\pgfqpoint{1.368442in}{1.986412in}}%
\pgfpathcurveto{\pgfqpoint{1.362618in}{1.992236in}}{\pgfqpoint{1.354718in}{1.995508in}}{\pgfqpoint{1.346482in}{1.995508in}}%
\pgfpathcurveto{\pgfqpoint{1.338245in}{1.995508in}}{\pgfqpoint{1.330345in}{1.992236in}}{\pgfqpoint{1.324521in}{1.986412in}}%
\pgfpathcurveto{\pgfqpoint{1.318697in}{1.980588in}}{\pgfqpoint{1.315425in}{1.972688in}}{\pgfqpoint{1.315425in}{1.964451in}}%
\pgfpathcurveto{\pgfqpoint{1.315425in}{1.956215in}}{\pgfqpoint{1.318697in}{1.948315in}}{\pgfqpoint{1.324521in}{1.942491in}}%
\pgfpathcurveto{\pgfqpoint{1.330345in}{1.936667in}}{\pgfqpoint{1.338245in}{1.933395in}}{\pgfqpoint{1.346482in}{1.933395in}}%
\pgfpathclose%
\pgfusepath{stroke,fill}%
\end{pgfscope}%
\begin{pgfscope}%
\pgfpathrectangle{\pgfqpoint{0.100000in}{0.212622in}}{\pgfqpoint{3.696000in}{3.696000in}}%
\pgfusepath{clip}%
\pgfsetbuttcap%
\pgfsetroundjoin%
\definecolor{currentfill}{rgb}{0.121569,0.466667,0.705882}%
\pgfsetfillcolor{currentfill}%
\pgfsetfillopacity{0.597075}%
\pgfsetlinewidth{1.003750pt}%
\definecolor{currentstroke}{rgb}{0.121569,0.466667,0.705882}%
\pgfsetstrokecolor{currentstroke}%
\pgfsetstrokeopacity{0.597075}%
\pgfsetdash{}{0pt}%
\pgfpathmoveto{\pgfqpoint{1.414715in}{1.990483in}}%
\pgfpathcurveto{\pgfqpoint{1.422951in}{1.990483in}}{\pgfqpoint{1.430851in}{1.993755in}}{\pgfqpoint{1.436675in}{1.999579in}}%
\pgfpathcurveto{\pgfqpoint{1.442499in}{2.005403in}}{\pgfqpoint{1.445771in}{2.013303in}}{\pgfqpoint{1.445771in}{2.021539in}}%
\pgfpathcurveto{\pgfqpoint{1.445771in}{2.029775in}}{\pgfqpoint{1.442499in}{2.037676in}}{\pgfqpoint{1.436675in}{2.043499in}}%
\pgfpathcurveto{\pgfqpoint{1.430851in}{2.049323in}}{\pgfqpoint{1.422951in}{2.052596in}}{\pgfqpoint{1.414715in}{2.052596in}}%
\pgfpathcurveto{\pgfqpoint{1.406478in}{2.052596in}}{\pgfqpoint{1.398578in}{2.049323in}}{\pgfqpoint{1.392754in}{2.043499in}}%
\pgfpathcurveto{\pgfqpoint{1.386930in}{2.037676in}}{\pgfqpoint{1.383658in}{2.029775in}}{\pgfqpoint{1.383658in}{2.021539in}}%
\pgfpathcurveto{\pgfqpoint{1.383658in}{2.013303in}}{\pgfqpoint{1.386930in}{2.005403in}}{\pgfqpoint{1.392754in}{1.999579in}}%
\pgfpathcurveto{\pgfqpoint{1.398578in}{1.993755in}}{\pgfqpoint{1.406478in}{1.990483in}}{\pgfqpoint{1.414715in}{1.990483in}}%
\pgfpathclose%
\pgfusepath{stroke,fill}%
\end{pgfscope}%
\begin{pgfscope}%
\pgfpathrectangle{\pgfqpoint{0.100000in}{0.212622in}}{\pgfqpoint{3.696000in}{3.696000in}}%
\pgfusepath{clip}%
\pgfsetbuttcap%
\pgfsetroundjoin%
\definecolor{currentfill}{rgb}{0.121569,0.466667,0.705882}%
\pgfsetfillcolor{currentfill}%
\pgfsetfillopacity{0.597095}%
\pgfsetlinewidth{1.003750pt}%
\definecolor{currentstroke}{rgb}{0.121569,0.466667,0.705882}%
\pgfsetstrokecolor{currentstroke}%
\pgfsetstrokeopacity{0.597095}%
\pgfsetdash{}{0pt}%
\pgfpathmoveto{\pgfqpoint{1.381957in}{1.962467in}}%
\pgfpathcurveto{\pgfqpoint{1.390193in}{1.962467in}}{\pgfqpoint{1.398093in}{1.965740in}}{\pgfqpoint{1.403917in}{1.971563in}}%
\pgfpathcurveto{\pgfqpoint{1.409741in}{1.977387in}}{\pgfqpoint{1.413013in}{1.985287in}}{\pgfqpoint{1.413013in}{1.993524in}}%
\pgfpathcurveto{\pgfqpoint{1.413013in}{2.001760in}}{\pgfqpoint{1.409741in}{2.009660in}}{\pgfqpoint{1.403917in}{2.015484in}}%
\pgfpathcurveto{\pgfqpoint{1.398093in}{2.021308in}}{\pgfqpoint{1.390193in}{2.024580in}}{\pgfqpoint{1.381957in}{2.024580in}}%
\pgfpathcurveto{\pgfqpoint{1.373721in}{2.024580in}}{\pgfqpoint{1.365821in}{2.021308in}}{\pgfqpoint{1.359997in}{2.015484in}}%
\pgfpathcurveto{\pgfqpoint{1.354173in}{2.009660in}}{\pgfqpoint{1.350900in}{2.001760in}}{\pgfqpoint{1.350900in}{1.993524in}}%
\pgfpathcurveto{\pgfqpoint{1.350900in}{1.985287in}}{\pgfqpoint{1.354173in}{1.977387in}}{\pgfqpoint{1.359997in}{1.971563in}}%
\pgfpathcurveto{\pgfqpoint{1.365821in}{1.965740in}}{\pgfqpoint{1.373721in}{1.962467in}}{\pgfqpoint{1.381957in}{1.962467in}}%
\pgfpathclose%
\pgfusepath{stroke,fill}%
\end{pgfscope}%
\begin{pgfscope}%
\pgfpathrectangle{\pgfqpoint{0.100000in}{0.212622in}}{\pgfqpoint{3.696000in}{3.696000in}}%
\pgfusepath{clip}%
\pgfsetbuttcap%
\pgfsetroundjoin%
\definecolor{currentfill}{rgb}{0.121569,0.466667,0.705882}%
\pgfsetfillcolor{currentfill}%
\pgfsetfillopacity{0.597360}%
\pgfsetlinewidth{1.003750pt}%
\definecolor{currentstroke}{rgb}{0.121569,0.466667,0.705882}%
\pgfsetstrokecolor{currentstroke}%
\pgfsetstrokeopacity{0.597360}%
\pgfsetdash{}{0pt}%
\pgfpathmoveto{\pgfqpoint{1.407994in}{1.981991in}}%
\pgfpathcurveto{\pgfqpoint{1.416230in}{1.981991in}}{\pgfqpoint{1.424130in}{1.985263in}}{\pgfqpoint{1.429954in}{1.991087in}}%
\pgfpathcurveto{\pgfqpoint{1.435778in}{1.996911in}}{\pgfqpoint{1.439050in}{2.004811in}}{\pgfqpoint{1.439050in}{2.013047in}}%
\pgfpathcurveto{\pgfqpoint{1.439050in}{2.021283in}}{\pgfqpoint{1.435778in}{2.029183in}}{\pgfqpoint{1.429954in}{2.035007in}}%
\pgfpathcurveto{\pgfqpoint{1.424130in}{2.040831in}}{\pgfqpoint{1.416230in}{2.044104in}}{\pgfqpoint{1.407994in}{2.044104in}}%
\pgfpathcurveto{\pgfqpoint{1.399757in}{2.044104in}}{\pgfqpoint{1.391857in}{2.040831in}}{\pgfqpoint{1.386033in}{2.035007in}}%
\pgfpathcurveto{\pgfqpoint{1.380209in}{2.029183in}}{\pgfqpoint{1.376937in}{2.021283in}}{\pgfqpoint{1.376937in}{2.013047in}}%
\pgfpathcurveto{\pgfqpoint{1.376937in}{2.004811in}}{\pgfqpoint{1.380209in}{1.996911in}}{\pgfqpoint{1.386033in}{1.991087in}}%
\pgfpathcurveto{\pgfqpoint{1.391857in}{1.985263in}}{\pgfqpoint{1.399757in}{1.981991in}}{\pgfqpoint{1.407994in}{1.981991in}}%
\pgfpathclose%
\pgfusepath{stroke,fill}%
\end{pgfscope}%
\begin{pgfscope}%
\pgfpathrectangle{\pgfqpoint{0.100000in}{0.212622in}}{\pgfqpoint{3.696000in}{3.696000in}}%
\pgfusepath{clip}%
\pgfsetbuttcap%
\pgfsetroundjoin%
\definecolor{currentfill}{rgb}{0.121569,0.466667,0.705882}%
\pgfsetfillcolor{currentfill}%
\pgfsetfillopacity{0.597665}%
\pgfsetlinewidth{1.003750pt}%
\definecolor{currentstroke}{rgb}{0.121569,0.466667,0.705882}%
\pgfsetstrokecolor{currentstroke}%
\pgfsetstrokeopacity{0.597665}%
\pgfsetdash{}{0pt}%
\pgfpathmoveto{\pgfqpoint{1.424874in}{1.985792in}}%
\pgfpathcurveto{\pgfqpoint{1.433110in}{1.985792in}}{\pgfqpoint{1.441010in}{1.989065in}}{\pgfqpoint{1.446834in}{1.994888in}}%
\pgfpathcurveto{\pgfqpoint{1.452658in}{2.000712in}}{\pgfqpoint{1.455930in}{2.008612in}}{\pgfqpoint{1.455930in}{2.016849in}}%
\pgfpathcurveto{\pgfqpoint{1.455930in}{2.025085in}}{\pgfqpoint{1.452658in}{2.032985in}}{\pgfqpoint{1.446834in}{2.038809in}}%
\pgfpathcurveto{\pgfqpoint{1.441010in}{2.044633in}}{\pgfqpoint{1.433110in}{2.047905in}}{\pgfqpoint{1.424874in}{2.047905in}}%
\pgfpathcurveto{\pgfqpoint{1.416638in}{2.047905in}}{\pgfqpoint{1.408738in}{2.044633in}}{\pgfqpoint{1.402914in}{2.038809in}}%
\pgfpathcurveto{\pgfqpoint{1.397090in}{2.032985in}}{\pgfqpoint{1.393817in}{2.025085in}}{\pgfqpoint{1.393817in}{2.016849in}}%
\pgfpathcurveto{\pgfqpoint{1.393817in}{2.008612in}}{\pgfqpoint{1.397090in}{2.000712in}}{\pgfqpoint{1.402914in}{1.994888in}}%
\pgfpathcurveto{\pgfqpoint{1.408738in}{1.989065in}}{\pgfqpoint{1.416638in}{1.985792in}}{\pgfqpoint{1.424874in}{1.985792in}}%
\pgfpathclose%
\pgfusepath{stroke,fill}%
\end{pgfscope}%
\begin{pgfscope}%
\pgfpathrectangle{\pgfqpoint{0.100000in}{0.212622in}}{\pgfqpoint{3.696000in}{3.696000in}}%
\pgfusepath{clip}%
\pgfsetbuttcap%
\pgfsetroundjoin%
\definecolor{currentfill}{rgb}{0.121569,0.466667,0.705882}%
\pgfsetfillcolor{currentfill}%
\pgfsetfillopacity{0.597755}%
\pgfsetlinewidth{1.003750pt}%
\definecolor{currentstroke}{rgb}{0.121569,0.466667,0.705882}%
\pgfsetstrokecolor{currentstroke}%
\pgfsetstrokeopacity{0.597755}%
\pgfsetdash{}{0pt}%
\pgfpathmoveto{\pgfqpoint{1.404281in}{1.982560in}}%
\pgfpathcurveto{\pgfqpoint{1.412517in}{1.982560in}}{\pgfqpoint{1.420417in}{1.985832in}}{\pgfqpoint{1.426241in}{1.991656in}}%
\pgfpathcurveto{\pgfqpoint{1.432065in}{1.997480in}}{\pgfqpoint{1.435337in}{2.005380in}}{\pgfqpoint{1.435337in}{2.013616in}}%
\pgfpathcurveto{\pgfqpoint{1.435337in}{2.021853in}}{\pgfqpoint{1.432065in}{2.029753in}}{\pgfqpoint{1.426241in}{2.035577in}}%
\pgfpathcurveto{\pgfqpoint{1.420417in}{2.041401in}}{\pgfqpoint{1.412517in}{2.044673in}}{\pgfqpoint{1.404281in}{2.044673in}}%
\pgfpathcurveto{\pgfqpoint{1.396044in}{2.044673in}}{\pgfqpoint{1.388144in}{2.041401in}}{\pgfqpoint{1.382320in}{2.035577in}}%
\pgfpathcurveto{\pgfqpoint{1.376497in}{2.029753in}}{\pgfqpoint{1.373224in}{2.021853in}}{\pgfqpoint{1.373224in}{2.013616in}}%
\pgfpathcurveto{\pgfqpoint{1.373224in}{2.005380in}}{\pgfqpoint{1.376497in}{1.997480in}}{\pgfqpoint{1.382320in}{1.991656in}}%
\pgfpathcurveto{\pgfqpoint{1.388144in}{1.985832in}}{\pgfqpoint{1.396044in}{1.982560in}}{\pgfqpoint{1.404281in}{1.982560in}}%
\pgfpathclose%
\pgfusepath{stroke,fill}%
\end{pgfscope}%
\begin{pgfscope}%
\pgfpathrectangle{\pgfqpoint{0.100000in}{0.212622in}}{\pgfqpoint{3.696000in}{3.696000in}}%
\pgfusepath{clip}%
\pgfsetbuttcap%
\pgfsetroundjoin%
\definecolor{currentfill}{rgb}{0.121569,0.466667,0.705882}%
\pgfsetfillcolor{currentfill}%
\pgfsetfillopacity{0.599445}%
\pgfsetlinewidth{1.003750pt}%
\definecolor{currentstroke}{rgb}{0.121569,0.466667,0.705882}%
\pgfsetstrokecolor{currentstroke}%
\pgfsetstrokeopacity{0.599445}%
\pgfsetdash{}{0pt}%
\pgfpathmoveto{\pgfqpoint{1.411468in}{1.977789in}}%
\pgfpathcurveto{\pgfqpoint{1.419704in}{1.977789in}}{\pgfqpoint{1.427604in}{1.981062in}}{\pgfqpoint{1.433428in}{1.986886in}}%
\pgfpathcurveto{\pgfqpoint{1.439252in}{1.992710in}}{\pgfqpoint{1.442525in}{2.000610in}}{\pgfqpoint{1.442525in}{2.008846in}}%
\pgfpathcurveto{\pgfqpoint{1.442525in}{2.017082in}}{\pgfqpoint{1.439252in}{2.024982in}}{\pgfqpoint{1.433428in}{2.030806in}}%
\pgfpathcurveto{\pgfqpoint{1.427604in}{2.036630in}}{\pgfqpoint{1.419704in}{2.039902in}}{\pgfqpoint{1.411468in}{2.039902in}}%
\pgfpathcurveto{\pgfqpoint{1.403232in}{2.039902in}}{\pgfqpoint{1.395332in}{2.036630in}}{\pgfqpoint{1.389508in}{2.030806in}}%
\pgfpathcurveto{\pgfqpoint{1.383684in}{2.024982in}}{\pgfqpoint{1.380412in}{2.017082in}}{\pgfqpoint{1.380412in}{2.008846in}}%
\pgfpathcurveto{\pgfqpoint{1.380412in}{2.000610in}}{\pgfqpoint{1.383684in}{1.992710in}}{\pgfqpoint{1.389508in}{1.986886in}}%
\pgfpathcurveto{\pgfqpoint{1.395332in}{1.981062in}}{\pgfqpoint{1.403232in}{1.977789in}}{\pgfqpoint{1.411468in}{1.977789in}}%
\pgfpathclose%
\pgfusepath{stroke,fill}%
\end{pgfscope}%
\begin{pgfscope}%
\pgfpathrectangle{\pgfqpoint{0.100000in}{0.212622in}}{\pgfqpoint{3.696000in}{3.696000in}}%
\pgfusepath{clip}%
\pgfsetbuttcap%
\pgfsetroundjoin%
\definecolor{currentfill}{rgb}{0.121569,0.466667,0.705882}%
\pgfsetfillcolor{currentfill}%
\pgfsetfillopacity{0.601495}%
\pgfsetlinewidth{1.003750pt}%
\definecolor{currentstroke}{rgb}{0.121569,0.466667,0.705882}%
\pgfsetstrokecolor{currentstroke}%
\pgfsetstrokeopacity{0.601495}%
\pgfsetdash{}{0pt}%
\pgfpathmoveto{\pgfqpoint{1.432946in}{1.999532in}}%
\pgfpathcurveto{\pgfqpoint{1.441182in}{1.999532in}}{\pgfqpoint{1.449082in}{2.002805in}}{\pgfqpoint{1.454906in}{2.008629in}}%
\pgfpathcurveto{\pgfqpoint{1.460730in}{2.014453in}}{\pgfqpoint{1.464002in}{2.022353in}}{\pgfqpoint{1.464002in}{2.030589in}}%
\pgfpathcurveto{\pgfqpoint{1.464002in}{2.038825in}}{\pgfqpoint{1.460730in}{2.046725in}}{\pgfqpoint{1.454906in}{2.052549in}}%
\pgfpathcurveto{\pgfqpoint{1.449082in}{2.058373in}}{\pgfqpoint{1.441182in}{2.061645in}}{\pgfqpoint{1.432946in}{2.061645in}}%
\pgfpathcurveto{\pgfqpoint{1.424709in}{2.061645in}}{\pgfqpoint{1.416809in}{2.058373in}}{\pgfqpoint{1.410985in}{2.052549in}}%
\pgfpathcurveto{\pgfqpoint{1.405161in}{2.046725in}}{\pgfqpoint{1.401889in}{2.038825in}}{\pgfqpoint{1.401889in}{2.030589in}}%
\pgfpathcurveto{\pgfqpoint{1.401889in}{2.022353in}}{\pgfqpoint{1.405161in}{2.014453in}}{\pgfqpoint{1.410985in}{2.008629in}}%
\pgfpathcurveto{\pgfqpoint{1.416809in}{2.002805in}}{\pgfqpoint{1.424709in}{1.999532in}}{\pgfqpoint{1.432946in}{1.999532in}}%
\pgfpathclose%
\pgfusepath{stroke,fill}%
\end{pgfscope}%
\begin{pgfscope}%
\pgfpathrectangle{\pgfqpoint{0.100000in}{0.212622in}}{\pgfqpoint{3.696000in}{3.696000in}}%
\pgfusepath{clip}%
\pgfsetbuttcap%
\pgfsetroundjoin%
\definecolor{currentfill}{rgb}{0.121569,0.466667,0.705882}%
\pgfsetfillcolor{currentfill}%
\pgfsetfillopacity{0.603243}%
\pgfsetlinewidth{1.003750pt}%
\definecolor{currentstroke}{rgb}{0.121569,0.466667,0.705882}%
\pgfsetstrokecolor{currentstroke}%
\pgfsetstrokeopacity{0.603243}%
\pgfsetdash{}{0pt}%
\pgfpathmoveto{\pgfqpoint{1.440778in}{2.014052in}}%
\pgfpathcurveto{\pgfqpoint{1.449014in}{2.014052in}}{\pgfqpoint{1.456914in}{2.017325in}}{\pgfqpoint{1.462738in}{2.023149in}}%
\pgfpathcurveto{\pgfqpoint{1.468562in}{2.028973in}}{\pgfqpoint{1.471834in}{2.036873in}}{\pgfqpoint{1.471834in}{2.045109in}}%
\pgfpathcurveto{\pgfqpoint{1.471834in}{2.053345in}}{\pgfqpoint{1.468562in}{2.061245in}}{\pgfqpoint{1.462738in}{2.067069in}}%
\pgfpathcurveto{\pgfqpoint{1.456914in}{2.072893in}}{\pgfqpoint{1.449014in}{2.076165in}}{\pgfqpoint{1.440778in}{2.076165in}}%
\pgfpathcurveto{\pgfqpoint{1.432541in}{2.076165in}}{\pgfqpoint{1.424641in}{2.072893in}}{\pgfqpoint{1.418817in}{2.067069in}}%
\pgfpathcurveto{\pgfqpoint{1.412993in}{2.061245in}}{\pgfqpoint{1.409721in}{2.053345in}}{\pgfqpoint{1.409721in}{2.045109in}}%
\pgfpathcurveto{\pgfqpoint{1.409721in}{2.036873in}}{\pgfqpoint{1.412993in}{2.028973in}}{\pgfqpoint{1.418817in}{2.023149in}}%
\pgfpathcurveto{\pgfqpoint{1.424641in}{2.017325in}}{\pgfqpoint{1.432541in}{2.014052in}}{\pgfqpoint{1.440778in}{2.014052in}}%
\pgfpathclose%
\pgfusepath{stroke,fill}%
\end{pgfscope}%
\begin{pgfscope}%
\pgfpathrectangle{\pgfqpoint{0.100000in}{0.212622in}}{\pgfqpoint{3.696000in}{3.696000in}}%
\pgfusepath{clip}%
\pgfsetbuttcap%
\pgfsetroundjoin%
\definecolor{currentfill}{rgb}{0.121569,0.466667,0.705882}%
\pgfsetfillcolor{currentfill}%
\pgfsetfillopacity{0.610833}%
\pgfsetlinewidth{1.003750pt}%
\definecolor{currentstroke}{rgb}{0.121569,0.466667,0.705882}%
\pgfsetstrokecolor{currentstroke}%
\pgfsetstrokeopacity{0.610833}%
\pgfsetdash{}{0pt}%
\pgfpathmoveto{\pgfqpoint{1.331183in}{1.923882in}}%
\pgfpathcurveto{\pgfqpoint{1.339419in}{1.923882in}}{\pgfqpoint{1.347319in}{1.927154in}}{\pgfqpoint{1.353143in}{1.932978in}}%
\pgfpathcurveto{\pgfqpoint{1.358967in}{1.938802in}}{\pgfqpoint{1.362239in}{1.946702in}}{\pgfqpoint{1.362239in}{1.954938in}}%
\pgfpathcurveto{\pgfqpoint{1.362239in}{1.963175in}}{\pgfqpoint{1.358967in}{1.971075in}}{\pgfqpoint{1.353143in}{1.976899in}}%
\pgfpathcurveto{\pgfqpoint{1.347319in}{1.982723in}}{\pgfqpoint{1.339419in}{1.985995in}}{\pgfqpoint{1.331183in}{1.985995in}}%
\pgfpathcurveto{\pgfqpoint{1.322947in}{1.985995in}}{\pgfqpoint{1.315047in}{1.982723in}}{\pgfqpoint{1.309223in}{1.976899in}}%
\pgfpathcurveto{\pgfqpoint{1.303399in}{1.971075in}}{\pgfqpoint{1.300126in}{1.963175in}}{\pgfqpoint{1.300126in}{1.954938in}}%
\pgfpathcurveto{\pgfqpoint{1.300126in}{1.946702in}}{\pgfqpoint{1.303399in}{1.938802in}}{\pgfqpoint{1.309223in}{1.932978in}}%
\pgfpathcurveto{\pgfqpoint{1.315047in}{1.927154in}}{\pgfqpoint{1.322947in}{1.923882in}}{\pgfqpoint{1.331183in}{1.923882in}}%
\pgfpathclose%
\pgfusepath{stroke,fill}%
\end{pgfscope}%
\begin{pgfscope}%
\pgfpathrectangle{\pgfqpoint{0.100000in}{0.212622in}}{\pgfqpoint{3.696000in}{3.696000in}}%
\pgfusepath{clip}%
\pgfsetbuttcap%
\pgfsetroundjoin%
\definecolor{currentfill}{rgb}{0.121569,0.466667,0.705882}%
\pgfsetfillcolor{currentfill}%
\pgfsetfillopacity{0.611840}%
\pgfsetlinewidth{1.003750pt}%
\definecolor{currentstroke}{rgb}{0.121569,0.466667,0.705882}%
\pgfsetstrokecolor{currentstroke}%
\pgfsetstrokeopacity{0.611840}%
\pgfsetdash{}{0pt}%
\pgfpathmoveto{\pgfqpoint{1.431876in}{2.005161in}}%
\pgfpathcurveto{\pgfqpoint{1.440112in}{2.005161in}}{\pgfqpoint{1.448012in}{2.008433in}}{\pgfqpoint{1.453836in}{2.014257in}}%
\pgfpathcurveto{\pgfqpoint{1.459660in}{2.020081in}}{\pgfqpoint{1.462932in}{2.027981in}}{\pgfqpoint{1.462932in}{2.036217in}}%
\pgfpathcurveto{\pgfqpoint{1.462932in}{2.044453in}}{\pgfqpoint{1.459660in}{2.052353in}}{\pgfqpoint{1.453836in}{2.058177in}}%
\pgfpathcurveto{\pgfqpoint{1.448012in}{2.064001in}}{\pgfqpoint{1.440112in}{2.067274in}}{\pgfqpoint{1.431876in}{2.067274in}}%
\pgfpathcurveto{\pgfqpoint{1.423639in}{2.067274in}}{\pgfqpoint{1.415739in}{2.064001in}}{\pgfqpoint{1.409915in}{2.058177in}}%
\pgfpathcurveto{\pgfqpoint{1.404091in}{2.052353in}}{\pgfqpoint{1.400819in}{2.044453in}}{\pgfqpoint{1.400819in}{2.036217in}}%
\pgfpathcurveto{\pgfqpoint{1.400819in}{2.027981in}}{\pgfqpoint{1.404091in}{2.020081in}}{\pgfqpoint{1.409915in}{2.014257in}}%
\pgfpathcurveto{\pgfqpoint{1.415739in}{2.008433in}}{\pgfqpoint{1.423639in}{2.005161in}}{\pgfqpoint{1.431876in}{2.005161in}}%
\pgfpathclose%
\pgfusepath{stroke,fill}%
\end{pgfscope}%
\begin{pgfscope}%
\pgfpathrectangle{\pgfqpoint{0.100000in}{0.212622in}}{\pgfqpoint{3.696000in}{3.696000in}}%
\pgfusepath{clip}%
\pgfsetbuttcap%
\pgfsetroundjoin%
\definecolor{currentfill}{rgb}{0.121569,0.466667,0.705882}%
\pgfsetfillcolor{currentfill}%
\pgfsetfillopacity{0.619560}%
\pgfsetlinewidth{1.003750pt}%
\definecolor{currentstroke}{rgb}{0.121569,0.466667,0.705882}%
\pgfsetstrokecolor{currentstroke}%
\pgfsetstrokeopacity{0.619560}%
\pgfsetdash{}{0pt}%
\pgfpathmoveto{\pgfqpoint{1.427416in}{2.003035in}}%
\pgfpathcurveto{\pgfqpoint{1.435652in}{2.003035in}}{\pgfqpoint{1.443552in}{2.006308in}}{\pgfqpoint{1.449376in}{2.012132in}}%
\pgfpathcurveto{\pgfqpoint{1.455200in}{2.017956in}}{\pgfqpoint{1.458472in}{2.025856in}}{\pgfqpoint{1.458472in}{2.034092in}}%
\pgfpathcurveto{\pgfqpoint{1.458472in}{2.042328in}}{\pgfqpoint{1.455200in}{2.050228in}}{\pgfqpoint{1.449376in}{2.056052in}}%
\pgfpathcurveto{\pgfqpoint{1.443552in}{2.061876in}}{\pgfqpoint{1.435652in}{2.065148in}}{\pgfqpoint{1.427416in}{2.065148in}}%
\pgfpathcurveto{\pgfqpoint{1.419179in}{2.065148in}}{\pgfqpoint{1.411279in}{2.061876in}}{\pgfqpoint{1.405455in}{2.056052in}}%
\pgfpathcurveto{\pgfqpoint{1.399632in}{2.050228in}}{\pgfqpoint{1.396359in}{2.042328in}}{\pgfqpoint{1.396359in}{2.034092in}}%
\pgfpathcurveto{\pgfqpoint{1.396359in}{2.025856in}}{\pgfqpoint{1.399632in}{2.017956in}}{\pgfqpoint{1.405455in}{2.012132in}}%
\pgfpathcurveto{\pgfqpoint{1.411279in}{2.006308in}}{\pgfqpoint{1.419179in}{2.003035in}}{\pgfqpoint{1.427416in}{2.003035in}}%
\pgfpathclose%
\pgfusepath{stroke,fill}%
\end{pgfscope}%
\begin{pgfscope}%
\pgfpathrectangle{\pgfqpoint{0.100000in}{0.212622in}}{\pgfqpoint{3.696000in}{3.696000in}}%
\pgfusepath{clip}%
\pgfsetbuttcap%
\pgfsetroundjoin%
\definecolor{currentfill}{rgb}{0.121569,0.466667,0.705882}%
\pgfsetfillcolor{currentfill}%
\pgfsetfillopacity{0.622357}%
\pgfsetlinewidth{1.003750pt}%
\definecolor{currentstroke}{rgb}{0.121569,0.466667,0.705882}%
\pgfsetstrokecolor{currentstroke}%
\pgfsetstrokeopacity{0.622357}%
\pgfsetdash{}{0pt}%
\pgfpathmoveto{\pgfqpoint{1.427122in}{2.000043in}}%
\pgfpathcurveto{\pgfqpoint{1.435358in}{2.000043in}}{\pgfqpoint{1.443258in}{2.003315in}}{\pgfqpoint{1.449082in}{2.009139in}}%
\pgfpathcurveto{\pgfqpoint{1.454906in}{2.014963in}}{\pgfqpoint{1.458178in}{2.022863in}}{\pgfqpoint{1.458178in}{2.031099in}}%
\pgfpathcurveto{\pgfqpoint{1.458178in}{2.039335in}}{\pgfqpoint{1.454906in}{2.047235in}}{\pgfqpoint{1.449082in}{2.053059in}}%
\pgfpathcurveto{\pgfqpoint{1.443258in}{2.058883in}}{\pgfqpoint{1.435358in}{2.062156in}}{\pgfqpoint{1.427122in}{2.062156in}}%
\pgfpathcurveto{\pgfqpoint{1.418885in}{2.062156in}}{\pgfqpoint{1.410985in}{2.058883in}}{\pgfqpoint{1.405161in}{2.053059in}}%
\pgfpathcurveto{\pgfqpoint{1.399338in}{2.047235in}}{\pgfqpoint{1.396065in}{2.039335in}}{\pgfqpoint{1.396065in}{2.031099in}}%
\pgfpathcurveto{\pgfqpoint{1.396065in}{2.022863in}}{\pgfqpoint{1.399338in}{2.014963in}}{\pgfqpoint{1.405161in}{2.009139in}}%
\pgfpathcurveto{\pgfqpoint{1.410985in}{2.003315in}}{\pgfqpoint{1.418885in}{2.000043in}}{\pgfqpoint{1.427122in}{2.000043in}}%
\pgfpathclose%
\pgfusepath{stroke,fill}%
\end{pgfscope}%
\begin{pgfscope}%
\pgfpathrectangle{\pgfqpoint{0.100000in}{0.212622in}}{\pgfqpoint{3.696000in}{3.696000in}}%
\pgfusepath{clip}%
\pgfsetbuttcap%
\pgfsetroundjoin%
\definecolor{currentfill}{rgb}{0.121569,0.466667,0.705882}%
\pgfsetfillcolor{currentfill}%
\pgfsetfillopacity{0.623205}%
\pgfsetlinewidth{1.003750pt}%
\definecolor{currentstroke}{rgb}{0.121569,0.466667,0.705882}%
\pgfsetstrokecolor{currentstroke}%
\pgfsetstrokeopacity{0.623205}%
\pgfsetdash{}{0pt}%
\pgfpathmoveto{\pgfqpoint{1.429354in}{2.002272in}}%
\pgfpathcurveto{\pgfqpoint{1.437590in}{2.002272in}}{\pgfqpoint{1.445490in}{2.005544in}}{\pgfqpoint{1.451314in}{2.011368in}}%
\pgfpathcurveto{\pgfqpoint{1.457138in}{2.017192in}}{\pgfqpoint{1.460410in}{2.025092in}}{\pgfqpoint{1.460410in}{2.033328in}}%
\pgfpathcurveto{\pgfqpoint{1.460410in}{2.041564in}}{\pgfqpoint{1.457138in}{2.049465in}}{\pgfqpoint{1.451314in}{2.055288in}}%
\pgfpathcurveto{\pgfqpoint{1.445490in}{2.061112in}}{\pgfqpoint{1.437590in}{2.064385in}}{\pgfqpoint{1.429354in}{2.064385in}}%
\pgfpathcurveto{\pgfqpoint{1.421117in}{2.064385in}}{\pgfqpoint{1.413217in}{2.061112in}}{\pgfqpoint{1.407393in}{2.055288in}}%
\pgfpathcurveto{\pgfqpoint{1.401570in}{2.049465in}}{\pgfqpoint{1.398297in}{2.041564in}}{\pgfqpoint{1.398297in}{2.033328in}}%
\pgfpathcurveto{\pgfqpoint{1.398297in}{2.025092in}}{\pgfqpoint{1.401570in}{2.017192in}}{\pgfqpoint{1.407393in}{2.011368in}}%
\pgfpathcurveto{\pgfqpoint{1.413217in}{2.005544in}}{\pgfqpoint{1.421117in}{2.002272in}}{\pgfqpoint{1.429354in}{2.002272in}}%
\pgfpathclose%
\pgfusepath{stroke,fill}%
\end{pgfscope}%
\begin{pgfscope}%
\pgfpathrectangle{\pgfqpoint{0.100000in}{0.212622in}}{\pgfqpoint{3.696000in}{3.696000in}}%
\pgfusepath{clip}%
\pgfsetbuttcap%
\pgfsetroundjoin%
\definecolor{currentfill}{rgb}{0.121569,0.466667,0.705882}%
\pgfsetfillcolor{currentfill}%
\pgfsetfillopacity{0.624779}%
\pgfsetlinewidth{1.003750pt}%
\definecolor{currentstroke}{rgb}{0.121569,0.466667,0.705882}%
\pgfsetstrokecolor{currentstroke}%
\pgfsetstrokeopacity{0.624779}%
\pgfsetdash{}{0pt}%
\pgfpathmoveto{\pgfqpoint{1.432116in}{2.004912in}}%
\pgfpathcurveto{\pgfqpoint{1.440352in}{2.004912in}}{\pgfqpoint{1.448252in}{2.008185in}}{\pgfqpoint{1.454076in}{2.014009in}}%
\pgfpathcurveto{\pgfqpoint{1.459900in}{2.019833in}}{\pgfqpoint{1.463172in}{2.027733in}}{\pgfqpoint{1.463172in}{2.035969in}}%
\pgfpathcurveto{\pgfqpoint{1.463172in}{2.044205in}}{\pgfqpoint{1.459900in}{2.052105in}}{\pgfqpoint{1.454076in}{2.057929in}}%
\pgfpathcurveto{\pgfqpoint{1.448252in}{2.063753in}}{\pgfqpoint{1.440352in}{2.067025in}}{\pgfqpoint{1.432116in}{2.067025in}}%
\pgfpathcurveto{\pgfqpoint{1.423880in}{2.067025in}}{\pgfqpoint{1.415980in}{2.063753in}}{\pgfqpoint{1.410156in}{2.057929in}}%
\pgfpathcurveto{\pgfqpoint{1.404332in}{2.052105in}}{\pgfqpoint{1.401059in}{2.044205in}}{\pgfqpoint{1.401059in}{2.035969in}}%
\pgfpathcurveto{\pgfqpoint{1.401059in}{2.027733in}}{\pgfqpoint{1.404332in}{2.019833in}}{\pgfqpoint{1.410156in}{2.014009in}}%
\pgfpathcurveto{\pgfqpoint{1.415980in}{2.008185in}}{\pgfqpoint{1.423880in}{2.004912in}}{\pgfqpoint{1.432116in}{2.004912in}}%
\pgfpathclose%
\pgfusepath{stroke,fill}%
\end{pgfscope}%
\begin{pgfscope}%
\pgfpathrectangle{\pgfqpoint{0.100000in}{0.212622in}}{\pgfqpoint{3.696000in}{3.696000in}}%
\pgfusepath{clip}%
\pgfsetbuttcap%
\pgfsetroundjoin%
\definecolor{currentfill}{rgb}{0.121569,0.466667,0.705882}%
\pgfsetfillcolor{currentfill}%
\pgfsetfillopacity{0.628568}%
\pgfsetlinewidth{1.003750pt}%
\definecolor{currentstroke}{rgb}{0.121569,0.466667,0.705882}%
\pgfsetstrokecolor{currentstroke}%
\pgfsetstrokeopacity{0.628568}%
\pgfsetdash{}{0pt}%
\pgfpathmoveto{\pgfqpoint{1.458129in}{2.041724in}}%
\pgfpathcurveto{\pgfqpoint{1.466365in}{2.041724in}}{\pgfqpoint{1.474265in}{2.044996in}}{\pgfqpoint{1.480089in}{2.050820in}}%
\pgfpathcurveto{\pgfqpoint{1.485913in}{2.056644in}}{\pgfqpoint{1.489185in}{2.064544in}}{\pgfqpoint{1.489185in}{2.072780in}}%
\pgfpathcurveto{\pgfqpoint{1.489185in}{2.081017in}}{\pgfqpoint{1.485913in}{2.088917in}}{\pgfqpoint{1.480089in}{2.094741in}}%
\pgfpathcurveto{\pgfqpoint{1.474265in}{2.100565in}}{\pgfqpoint{1.466365in}{2.103837in}}{\pgfqpoint{1.458129in}{2.103837in}}%
\pgfpathcurveto{\pgfqpoint{1.449892in}{2.103837in}}{\pgfqpoint{1.441992in}{2.100565in}}{\pgfqpoint{1.436168in}{2.094741in}}%
\pgfpathcurveto{\pgfqpoint{1.430345in}{2.088917in}}{\pgfqpoint{1.427072in}{2.081017in}}{\pgfqpoint{1.427072in}{2.072780in}}%
\pgfpathcurveto{\pgfqpoint{1.427072in}{2.064544in}}{\pgfqpoint{1.430345in}{2.056644in}}{\pgfqpoint{1.436168in}{2.050820in}}%
\pgfpathcurveto{\pgfqpoint{1.441992in}{2.044996in}}{\pgfqpoint{1.449892in}{2.041724in}}{\pgfqpoint{1.458129in}{2.041724in}}%
\pgfpathclose%
\pgfusepath{stroke,fill}%
\end{pgfscope}%
\begin{pgfscope}%
\pgfpathrectangle{\pgfqpoint{0.100000in}{0.212622in}}{\pgfqpoint{3.696000in}{3.696000in}}%
\pgfusepath{clip}%
\pgfsetbuttcap%
\pgfsetroundjoin%
\definecolor{currentfill}{rgb}{0.121569,0.466667,0.705882}%
\pgfsetfillcolor{currentfill}%
\pgfsetfillopacity{0.628963}%
\pgfsetlinewidth{1.003750pt}%
\definecolor{currentstroke}{rgb}{0.121569,0.466667,0.705882}%
\pgfsetstrokecolor{currentstroke}%
\pgfsetstrokeopacity{0.628963}%
\pgfsetdash{}{0pt}%
\pgfpathmoveto{\pgfqpoint{1.428980in}{2.000418in}}%
\pgfpathcurveto{\pgfqpoint{1.437217in}{2.000418in}}{\pgfqpoint{1.445117in}{2.003690in}}{\pgfqpoint{1.450941in}{2.009514in}}%
\pgfpathcurveto{\pgfqpoint{1.456764in}{2.015338in}}{\pgfqpoint{1.460037in}{2.023238in}}{\pgfqpoint{1.460037in}{2.031474in}}%
\pgfpathcurveto{\pgfqpoint{1.460037in}{2.039710in}}{\pgfqpoint{1.456764in}{2.047610in}}{\pgfqpoint{1.450941in}{2.053434in}}%
\pgfpathcurveto{\pgfqpoint{1.445117in}{2.059258in}}{\pgfqpoint{1.437217in}{2.062531in}}{\pgfqpoint{1.428980in}{2.062531in}}%
\pgfpathcurveto{\pgfqpoint{1.420744in}{2.062531in}}{\pgfqpoint{1.412844in}{2.059258in}}{\pgfqpoint{1.407020in}{2.053434in}}%
\pgfpathcurveto{\pgfqpoint{1.401196in}{2.047610in}}{\pgfqpoint{1.397924in}{2.039710in}}{\pgfqpoint{1.397924in}{2.031474in}}%
\pgfpathcurveto{\pgfqpoint{1.397924in}{2.023238in}}{\pgfqpoint{1.401196in}{2.015338in}}{\pgfqpoint{1.407020in}{2.009514in}}%
\pgfpathcurveto{\pgfqpoint{1.412844in}{2.003690in}}{\pgfqpoint{1.420744in}{2.000418in}}{\pgfqpoint{1.428980in}{2.000418in}}%
\pgfpathclose%
\pgfusepath{stroke,fill}%
\end{pgfscope}%
\begin{pgfscope}%
\pgfpathrectangle{\pgfqpoint{0.100000in}{0.212622in}}{\pgfqpoint{3.696000in}{3.696000in}}%
\pgfusepath{clip}%
\pgfsetbuttcap%
\pgfsetroundjoin%
\definecolor{currentfill}{rgb}{0.121569,0.466667,0.705882}%
\pgfsetfillcolor{currentfill}%
\pgfsetfillopacity{0.631491}%
\pgfsetlinewidth{1.003750pt}%
\definecolor{currentstroke}{rgb}{0.121569,0.466667,0.705882}%
\pgfsetstrokecolor{currentstroke}%
\pgfsetstrokeopacity{0.631491}%
\pgfsetdash{}{0pt}%
\pgfpathmoveto{\pgfqpoint{1.427615in}{2.000623in}}%
\pgfpathcurveto{\pgfqpoint{1.435851in}{2.000623in}}{\pgfqpoint{1.443751in}{2.003895in}}{\pgfqpoint{1.449575in}{2.009719in}}%
\pgfpathcurveto{\pgfqpoint{1.455399in}{2.015543in}}{\pgfqpoint{1.458672in}{2.023443in}}{\pgfqpoint{1.458672in}{2.031679in}}%
\pgfpathcurveto{\pgfqpoint{1.458672in}{2.039915in}}{\pgfqpoint{1.455399in}{2.047815in}}{\pgfqpoint{1.449575in}{2.053639in}}%
\pgfpathcurveto{\pgfqpoint{1.443751in}{2.059463in}}{\pgfqpoint{1.435851in}{2.062736in}}{\pgfqpoint{1.427615in}{2.062736in}}%
\pgfpathcurveto{\pgfqpoint{1.419379in}{2.062736in}}{\pgfqpoint{1.411479in}{2.059463in}}{\pgfqpoint{1.405655in}{2.053639in}}%
\pgfpathcurveto{\pgfqpoint{1.399831in}{2.047815in}}{\pgfqpoint{1.396559in}{2.039915in}}{\pgfqpoint{1.396559in}{2.031679in}}%
\pgfpathcurveto{\pgfqpoint{1.396559in}{2.023443in}}{\pgfqpoint{1.399831in}{2.015543in}}{\pgfqpoint{1.405655in}{2.009719in}}%
\pgfpathcurveto{\pgfqpoint{1.411479in}{2.003895in}}{\pgfqpoint{1.419379in}{2.000623in}}{\pgfqpoint{1.427615in}{2.000623in}}%
\pgfpathclose%
\pgfusepath{stroke,fill}%
\end{pgfscope}%
\begin{pgfscope}%
\pgfpathrectangle{\pgfqpoint{0.100000in}{0.212622in}}{\pgfqpoint{3.696000in}{3.696000in}}%
\pgfusepath{clip}%
\pgfsetbuttcap%
\pgfsetroundjoin%
\definecolor{currentfill}{rgb}{0.121569,0.466667,0.705882}%
\pgfsetfillcolor{currentfill}%
\pgfsetfillopacity{0.639007}%
\pgfsetlinewidth{1.003750pt}%
\definecolor{currentstroke}{rgb}{0.121569,0.466667,0.705882}%
\pgfsetstrokecolor{currentstroke}%
\pgfsetstrokeopacity{0.639007}%
\pgfsetdash{}{0pt}%
\pgfpathmoveto{\pgfqpoint{1.415681in}{1.986437in}}%
\pgfpathcurveto{\pgfqpoint{1.423918in}{1.986437in}}{\pgfqpoint{1.431818in}{1.989709in}}{\pgfqpoint{1.437642in}{1.995533in}}%
\pgfpathcurveto{\pgfqpoint{1.443466in}{2.001357in}}{\pgfqpoint{1.446738in}{2.009257in}}{\pgfqpoint{1.446738in}{2.017494in}}%
\pgfpathcurveto{\pgfqpoint{1.446738in}{2.025730in}}{\pgfqpoint{1.443466in}{2.033630in}}{\pgfqpoint{1.437642in}{2.039454in}}%
\pgfpathcurveto{\pgfqpoint{1.431818in}{2.045278in}}{\pgfqpoint{1.423918in}{2.048550in}}{\pgfqpoint{1.415681in}{2.048550in}}%
\pgfpathcurveto{\pgfqpoint{1.407445in}{2.048550in}}{\pgfqpoint{1.399545in}{2.045278in}}{\pgfqpoint{1.393721in}{2.039454in}}%
\pgfpathcurveto{\pgfqpoint{1.387897in}{2.033630in}}{\pgfqpoint{1.384625in}{2.025730in}}{\pgfqpoint{1.384625in}{2.017494in}}%
\pgfpathcurveto{\pgfqpoint{1.384625in}{2.009257in}}{\pgfqpoint{1.387897in}{2.001357in}}{\pgfqpoint{1.393721in}{1.995533in}}%
\pgfpathcurveto{\pgfqpoint{1.399545in}{1.989709in}}{\pgfqpoint{1.407445in}{1.986437in}}{\pgfqpoint{1.415681in}{1.986437in}}%
\pgfpathclose%
\pgfusepath{stroke,fill}%
\end{pgfscope}%
\begin{pgfscope}%
\pgfpathrectangle{\pgfqpoint{0.100000in}{0.212622in}}{\pgfqpoint{3.696000in}{3.696000in}}%
\pgfusepath{clip}%
\pgfsetbuttcap%
\pgfsetroundjoin%
\definecolor{currentfill}{rgb}{0.121569,0.466667,0.705882}%
\pgfsetfillcolor{currentfill}%
\pgfsetfillopacity{0.651879}%
\pgfsetlinewidth{1.003750pt}%
\definecolor{currentstroke}{rgb}{0.121569,0.466667,0.705882}%
\pgfsetstrokecolor{currentstroke}%
\pgfsetstrokeopacity{0.651879}%
\pgfsetdash{}{0pt}%
\pgfpathmoveto{\pgfqpoint{1.409037in}{1.992931in}}%
\pgfpathcurveto{\pgfqpoint{1.417274in}{1.992931in}}{\pgfqpoint{1.425174in}{1.996203in}}{\pgfqpoint{1.430998in}{2.002027in}}%
\pgfpathcurveto{\pgfqpoint{1.436822in}{2.007851in}}{\pgfqpoint{1.440094in}{2.015751in}}{\pgfqpoint{1.440094in}{2.023987in}}%
\pgfpathcurveto{\pgfqpoint{1.440094in}{2.032224in}}{\pgfqpoint{1.436822in}{2.040124in}}{\pgfqpoint{1.430998in}{2.045948in}}%
\pgfpathcurveto{\pgfqpoint{1.425174in}{2.051771in}}{\pgfqpoint{1.417274in}{2.055044in}}{\pgfqpoint{1.409037in}{2.055044in}}%
\pgfpathcurveto{\pgfqpoint{1.400801in}{2.055044in}}{\pgfqpoint{1.392901in}{2.051771in}}{\pgfqpoint{1.387077in}{2.045948in}}%
\pgfpathcurveto{\pgfqpoint{1.381253in}{2.040124in}}{\pgfqpoint{1.377981in}{2.032224in}}{\pgfqpoint{1.377981in}{2.023987in}}%
\pgfpathcurveto{\pgfqpoint{1.377981in}{2.015751in}}{\pgfqpoint{1.381253in}{2.007851in}}{\pgfqpoint{1.387077in}{2.002027in}}%
\pgfpathcurveto{\pgfqpoint{1.392901in}{1.996203in}}{\pgfqpoint{1.400801in}{1.992931in}}{\pgfqpoint{1.409037in}{1.992931in}}%
\pgfpathclose%
\pgfusepath{stroke,fill}%
\end{pgfscope}%
\begin{pgfscope}%
\pgfpathrectangle{\pgfqpoint{0.100000in}{0.212622in}}{\pgfqpoint{3.696000in}{3.696000in}}%
\pgfusepath{clip}%
\pgfsetbuttcap%
\pgfsetroundjoin%
\definecolor{currentfill}{rgb}{0.121569,0.466667,0.705882}%
\pgfsetfillcolor{currentfill}%
\pgfsetfillopacity{0.652877}%
\pgfsetlinewidth{1.003750pt}%
\definecolor{currentstroke}{rgb}{0.121569,0.466667,0.705882}%
\pgfsetstrokecolor{currentstroke}%
\pgfsetstrokeopacity{0.652877}%
\pgfsetdash{}{0pt}%
\pgfpathmoveto{\pgfqpoint{1.413575in}{1.999708in}}%
\pgfpathcurveto{\pgfqpoint{1.421812in}{1.999708in}}{\pgfqpoint{1.429712in}{2.002980in}}{\pgfqpoint{1.435536in}{2.008804in}}%
\pgfpathcurveto{\pgfqpoint{1.441360in}{2.014628in}}{\pgfqpoint{1.444632in}{2.022528in}}{\pgfqpoint{1.444632in}{2.030764in}}%
\pgfpathcurveto{\pgfqpoint{1.444632in}{2.039001in}}{\pgfqpoint{1.441360in}{2.046901in}}{\pgfqpoint{1.435536in}{2.052725in}}%
\pgfpathcurveto{\pgfqpoint{1.429712in}{2.058549in}}{\pgfqpoint{1.421812in}{2.061821in}}{\pgfqpoint{1.413575in}{2.061821in}}%
\pgfpathcurveto{\pgfqpoint{1.405339in}{2.061821in}}{\pgfqpoint{1.397439in}{2.058549in}}{\pgfqpoint{1.391615in}{2.052725in}}%
\pgfpathcurveto{\pgfqpoint{1.385791in}{2.046901in}}{\pgfqpoint{1.382519in}{2.039001in}}{\pgfqpoint{1.382519in}{2.030764in}}%
\pgfpathcurveto{\pgfqpoint{1.382519in}{2.022528in}}{\pgfqpoint{1.385791in}{2.014628in}}{\pgfqpoint{1.391615in}{2.008804in}}%
\pgfpathcurveto{\pgfqpoint{1.397439in}{2.002980in}}{\pgfqpoint{1.405339in}{1.999708in}}{\pgfqpoint{1.413575in}{1.999708in}}%
\pgfpathclose%
\pgfusepath{stroke,fill}%
\end{pgfscope}%
\begin{pgfscope}%
\pgfpathrectangle{\pgfqpoint{0.100000in}{0.212622in}}{\pgfqpoint{3.696000in}{3.696000in}}%
\pgfusepath{clip}%
\pgfsetbuttcap%
\pgfsetroundjoin%
\definecolor{currentfill}{rgb}{0.121569,0.466667,0.705882}%
\pgfsetfillcolor{currentfill}%
\pgfsetfillopacity{0.669139}%
\pgfsetlinewidth{1.003750pt}%
\definecolor{currentstroke}{rgb}{0.121569,0.466667,0.705882}%
\pgfsetstrokecolor{currentstroke}%
\pgfsetstrokeopacity{0.669139}%
\pgfsetdash{}{0pt}%
\pgfpathmoveto{\pgfqpoint{1.382326in}{1.972002in}}%
\pgfpathcurveto{\pgfqpoint{1.390563in}{1.972002in}}{\pgfqpoint{1.398463in}{1.975275in}}{\pgfqpoint{1.404287in}{1.981099in}}%
\pgfpathcurveto{\pgfqpoint{1.410110in}{1.986923in}}{\pgfqpoint{1.413383in}{1.994823in}}{\pgfqpoint{1.413383in}{2.003059in}}%
\pgfpathcurveto{\pgfqpoint{1.413383in}{2.011295in}}{\pgfqpoint{1.410110in}{2.019195in}}{\pgfqpoint{1.404287in}{2.025019in}}%
\pgfpathcurveto{\pgfqpoint{1.398463in}{2.030843in}}{\pgfqpoint{1.390563in}{2.034115in}}{\pgfqpoint{1.382326in}{2.034115in}}%
\pgfpathcurveto{\pgfqpoint{1.374090in}{2.034115in}}{\pgfqpoint{1.366190in}{2.030843in}}{\pgfqpoint{1.360366in}{2.025019in}}%
\pgfpathcurveto{\pgfqpoint{1.354542in}{2.019195in}}{\pgfqpoint{1.351270in}{2.011295in}}{\pgfqpoint{1.351270in}{2.003059in}}%
\pgfpathcurveto{\pgfqpoint{1.351270in}{1.994823in}}{\pgfqpoint{1.354542in}{1.986923in}}{\pgfqpoint{1.360366in}{1.981099in}}%
\pgfpathcurveto{\pgfqpoint{1.366190in}{1.975275in}}{\pgfqpoint{1.374090in}{1.972002in}}{\pgfqpoint{1.382326in}{1.972002in}}%
\pgfpathclose%
\pgfusepath{stroke,fill}%
\end{pgfscope}%
\begin{pgfscope}%
\pgfpathrectangle{\pgfqpoint{0.100000in}{0.212622in}}{\pgfqpoint{3.696000in}{3.696000in}}%
\pgfusepath{clip}%
\pgfsetbuttcap%
\pgfsetroundjoin%
\definecolor{currentfill}{rgb}{0.121569,0.466667,0.705882}%
\pgfsetfillcolor{currentfill}%
\pgfsetfillopacity{0.675593}%
\pgfsetlinewidth{1.003750pt}%
\definecolor{currentstroke}{rgb}{0.121569,0.466667,0.705882}%
\pgfsetstrokecolor{currentstroke}%
\pgfsetstrokeopacity{0.675593}%
\pgfsetdash{}{0pt}%
\pgfpathmoveto{\pgfqpoint{1.378564in}{1.968914in}}%
\pgfpathcurveto{\pgfqpoint{1.386800in}{1.968914in}}{\pgfqpoint{1.394700in}{1.972186in}}{\pgfqpoint{1.400524in}{1.978010in}}%
\pgfpathcurveto{\pgfqpoint{1.406348in}{1.983834in}}{\pgfqpoint{1.409620in}{1.991734in}}{\pgfqpoint{1.409620in}{1.999970in}}%
\pgfpathcurveto{\pgfqpoint{1.409620in}{2.008206in}}{\pgfqpoint{1.406348in}{2.016106in}}{\pgfqpoint{1.400524in}{2.021930in}}%
\pgfpathcurveto{\pgfqpoint{1.394700in}{2.027754in}}{\pgfqpoint{1.386800in}{2.031027in}}{\pgfqpoint{1.378564in}{2.031027in}}%
\pgfpathcurveto{\pgfqpoint{1.370327in}{2.031027in}}{\pgfqpoint{1.362427in}{2.027754in}}{\pgfqpoint{1.356603in}{2.021930in}}%
\pgfpathcurveto{\pgfqpoint{1.350779in}{2.016106in}}{\pgfqpoint{1.347507in}{2.008206in}}{\pgfqpoint{1.347507in}{1.999970in}}%
\pgfpathcurveto{\pgfqpoint{1.347507in}{1.991734in}}{\pgfqpoint{1.350779in}{1.983834in}}{\pgfqpoint{1.356603in}{1.978010in}}%
\pgfpathcurveto{\pgfqpoint{1.362427in}{1.972186in}}{\pgfqpoint{1.370327in}{1.968914in}}{\pgfqpoint{1.378564in}{1.968914in}}%
\pgfpathclose%
\pgfusepath{stroke,fill}%
\end{pgfscope}%
\begin{pgfscope}%
\pgfpathrectangle{\pgfqpoint{0.100000in}{0.212622in}}{\pgfqpoint{3.696000in}{3.696000in}}%
\pgfusepath{clip}%
\pgfsetbuttcap%
\pgfsetroundjoin%
\definecolor{currentfill}{rgb}{0.121569,0.466667,0.705882}%
\pgfsetfillcolor{currentfill}%
\pgfsetfillopacity{0.682191}%
\pgfsetlinewidth{1.003750pt}%
\definecolor{currentstroke}{rgb}{0.121569,0.466667,0.705882}%
\pgfsetstrokecolor{currentstroke}%
\pgfsetstrokeopacity{0.682191}%
\pgfsetdash{}{0pt}%
\pgfpathmoveto{\pgfqpoint{1.375809in}{1.964993in}}%
\pgfpathcurveto{\pgfqpoint{1.384046in}{1.964993in}}{\pgfqpoint{1.391946in}{1.968265in}}{\pgfqpoint{1.397770in}{1.974089in}}%
\pgfpathcurveto{\pgfqpoint{1.403593in}{1.979913in}}{\pgfqpoint{1.406866in}{1.987813in}}{\pgfqpoint{1.406866in}{1.996049in}}%
\pgfpathcurveto{\pgfqpoint{1.406866in}{2.004285in}}{\pgfqpoint{1.403593in}{2.012185in}}{\pgfqpoint{1.397770in}{2.018009in}}%
\pgfpathcurveto{\pgfqpoint{1.391946in}{2.023833in}}{\pgfqpoint{1.384046in}{2.027106in}}{\pgfqpoint{1.375809in}{2.027106in}}%
\pgfpathcurveto{\pgfqpoint{1.367573in}{2.027106in}}{\pgfqpoint{1.359673in}{2.023833in}}{\pgfqpoint{1.353849in}{2.018009in}}%
\pgfpathcurveto{\pgfqpoint{1.348025in}{2.012185in}}{\pgfqpoint{1.344753in}{2.004285in}}{\pgfqpoint{1.344753in}{1.996049in}}%
\pgfpathcurveto{\pgfqpoint{1.344753in}{1.987813in}}{\pgfqpoint{1.348025in}{1.979913in}}{\pgfqpoint{1.353849in}{1.974089in}}%
\pgfpathcurveto{\pgfqpoint{1.359673in}{1.968265in}}{\pgfqpoint{1.367573in}{1.964993in}}{\pgfqpoint{1.375809in}{1.964993in}}%
\pgfpathclose%
\pgfusepath{stroke,fill}%
\end{pgfscope}%
\begin{pgfscope}%
\pgfpathrectangle{\pgfqpoint{0.100000in}{0.212622in}}{\pgfqpoint{3.696000in}{3.696000in}}%
\pgfusepath{clip}%
\pgfsetbuttcap%
\pgfsetroundjoin%
\definecolor{currentfill}{rgb}{0.121569,0.466667,0.705882}%
\pgfsetfillcolor{currentfill}%
\pgfsetfillopacity{0.685161}%
\pgfsetlinewidth{1.003750pt}%
\definecolor{currentstroke}{rgb}{0.121569,0.466667,0.705882}%
\pgfsetstrokecolor{currentstroke}%
\pgfsetstrokeopacity{0.685161}%
\pgfsetdash{}{0pt}%
\pgfpathmoveto{\pgfqpoint{1.378301in}{1.972485in}}%
\pgfpathcurveto{\pgfqpoint{1.386537in}{1.972485in}}{\pgfqpoint{1.394437in}{1.975757in}}{\pgfqpoint{1.400261in}{1.981581in}}%
\pgfpathcurveto{\pgfqpoint{1.406085in}{1.987405in}}{\pgfqpoint{1.409357in}{1.995305in}}{\pgfqpoint{1.409357in}{2.003541in}}%
\pgfpathcurveto{\pgfqpoint{1.409357in}{2.011777in}}{\pgfqpoint{1.406085in}{2.019677in}}{\pgfqpoint{1.400261in}{2.025501in}}%
\pgfpathcurveto{\pgfqpoint{1.394437in}{2.031325in}}{\pgfqpoint{1.386537in}{2.034598in}}{\pgfqpoint{1.378301in}{2.034598in}}%
\pgfpathcurveto{\pgfqpoint{1.370064in}{2.034598in}}{\pgfqpoint{1.362164in}{2.031325in}}{\pgfqpoint{1.356340in}{2.025501in}}%
\pgfpathcurveto{\pgfqpoint{1.350517in}{2.019677in}}{\pgfqpoint{1.347244in}{2.011777in}}{\pgfqpoint{1.347244in}{2.003541in}}%
\pgfpathcurveto{\pgfqpoint{1.347244in}{1.995305in}}{\pgfqpoint{1.350517in}{1.987405in}}{\pgfqpoint{1.356340in}{1.981581in}}%
\pgfpathcurveto{\pgfqpoint{1.362164in}{1.975757in}}{\pgfqpoint{1.370064in}{1.972485in}}{\pgfqpoint{1.378301in}{1.972485in}}%
\pgfpathclose%
\pgfusepath{stroke,fill}%
\end{pgfscope}%
\begin{pgfscope}%
\pgfpathrectangle{\pgfqpoint{0.100000in}{0.212622in}}{\pgfqpoint{3.696000in}{3.696000in}}%
\pgfusepath{clip}%
\pgfsetbuttcap%
\pgfsetroundjoin%
\definecolor{currentfill}{rgb}{0.121569,0.466667,0.705882}%
\pgfsetfillcolor{currentfill}%
\pgfsetfillopacity{0.696567}%
\pgfsetlinewidth{1.003750pt}%
\definecolor{currentstroke}{rgb}{0.121569,0.466667,0.705882}%
\pgfsetstrokecolor{currentstroke}%
\pgfsetstrokeopacity{0.696567}%
\pgfsetdash{}{0pt}%
\pgfpathmoveto{\pgfqpoint{1.360156in}{1.952534in}}%
\pgfpathcurveto{\pgfqpoint{1.368392in}{1.952534in}}{\pgfqpoint{1.376292in}{1.955806in}}{\pgfqpoint{1.382116in}{1.961630in}}%
\pgfpathcurveto{\pgfqpoint{1.387940in}{1.967454in}}{\pgfqpoint{1.391212in}{1.975354in}}{\pgfqpoint{1.391212in}{1.983590in}}%
\pgfpathcurveto{\pgfqpoint{1.391212in}{1.991827in}}{\pgfqpoint{1.387940in}{1.999727in}}{\pgfqpoint{1.382116in}{2.005551in}}%
\pgfpathcurveto{\pgfqpoint{1.376292in}{2.011375in}}{\pgfqpoint{1.368392in}{2.014647in}}{\pgfqpoint{1.360156in}{2.014647in}}%
\pgfpathcurveto{\pgfqpoint{1.351920in}{2.014647in}}{\pgfqpoint{1.344019in}{2.011375in}}{\pgfqpoint{1.338196in}{2.005551in}}%
\pgfpathcurveto{\pgfqpoint{1.332372in}{1.999727in}}{\pgfqpoint{1.329099in}{1.991827in}}{\pgfqpoint{1.329099in}{1.983590in}}%
\pgfpathcurveto{\pgfqpoint{1.329099in}{1.975354in}}{\pgfqpoint{1.332372in}{1.967454in}}{\pgfqpoint{1.338196in}{1.961630in}}%
\pgfpathcurveto{\pgfqpoint{1.344019in}{1.955806in}}{\pgfqpoint{1.351920in}{1.952534in}}{\pgfqpoint{1.360156in}{1.952534in}}%
\pgfpathclose%
\pgfusepath{stroke,fill}%
\end{pgfscope}%
\begin{pgfscope}%
\pgfpathrectangle{\pgfqpoint{0.100000in}{0.212622in}}{\pgfqpoint{3.696000in}{3.696000in}}%
\pgfusepath{clip}%
\pgfsetbuttcap%
\pgfsetroundjoin%
\definecolor{currentfill}{rgb}{0.121569,0.466667,0.705882}%
\pgfsetfillcolor{currentfill}%
\pgfsetfillopacity{0.700918}%
\pgfsetlinewidth{1.003750pt}%
\definecolor{currentstroke}{rgb}{0.121569,0.466667,0.705882}%
\pgfsetstrokecolor{currentstroke}%
\pgfsetstrokeopacity{0.700918}%
\pgfsetdash{}{0pt}%
\pgfpathmoveto{\pgfqpoint{1.365367in}{1.963117in}}%
\pgfpathcurveto{\pgfqpoint{1.373603in}{1.963117in}}{\pgfqpoint{1.381503in}{1.966390in}}{\pgfqpoint{1.387327in}{1.972213in}}%
\pgfpathcurveto{\pgfqpoint{1.393151in}{1.978037in}}{\pgfqpoint{1.396423in}{1.985937in}}{\pgfqpoint{1.396423in}{1.994174in}}%
\pgfpathcurveto{\pgfqpoint{1.396423in}{2.002410in}}{\pgfqpoint{1.393151in}{2.010310in}}{\pgfqpoint{1.387327in}{2.016134in}}%
\pgfpathcurveto{\pgfqpoint{1.381503in}{2.021958in}}{\pgfqpoint{1.373603in}{2.025230in}}{\pgfqpoint{1.365367in}{2.025230in}}%
\pgfpathcurveto{\pgfqpoint{1.357131in}{2.025230in}}{\pgfqpoint{1.349231in}{2.021958in}}{\pgfqpoint{1.343407in}{2.016134in}}%
\pgfpathcurveto{\pgfqpoint{1.337583in}{2.010310in}}{\pgfqpoint{1.334310in}{2.002410in}}{\pgfqpoint{1.334310in}{1.994174in}}%
\pgfpathcurveto{\pgfqpoint{1.334310in}{1.985937in}}{\pgfqpoint{1.337583in}{1.978037in}}{\pgfqpoint{1.343407in}{1.972213in}}%
\pgfpathcurveto{\pgfqpoint{1.349231in}{1.966390in}}{\pgfqpoint{1.357131in}{1.963117in}}{\pgfqpoint{1.365367in}{1.963117in}}%
\pgfpathclose%
\pgfusepath{stroke,fill}%
\end{pgfscope}%
\begin{pgfscope}%
\pgfpathrectangle{\pgfqpoint{0.100000in}{0.212622in}}{\pgfqpoint{3.696000in}{3.696000in}}%
\pgfusepath{clip}%
\pgfsetbuttcap%
\pgfsetroundjoin%
\definecolor{currentfill}{rgb}{0.121569,0.466667,0.705882}%
\pgfsetfillcolor{currentfill}%
\pgfsetfillopacity{0.703475}%
\pgfsetlinewidth{1.003750pt}%
\definecolor{currentstroke}{rgb}{0.121569,0.466667,0.705882}%
\pgfsetstrokecolor{currentstroke}%
\pgfsetstrokeopacity{0.703475}%
\pgfsetdash{}{0pt}%
\pgfpathmoveto{\pgfqpoint{1.367733in}{1.968229in}}%
\pgfpathcurveto{\pgfqpoint{1.375970in}{1.968229in}}{\pgfqpoint{1.383870in}{1.971501in}}{\pgfqpoint{1.389694in}{1.977325in}}%
\pgfpathcurveto{\pgfqpoint{1.395518in}{1.983149in}}{\pgfqpoint{1.398790in}{1.991049in}}{\pgfqpoint{1.398790in}{1.999285in}}%
\pgfpathcurveto{\pgfqpoint{1.398790in}{2.007521in}}{\pgfqpoint{1.395518in}{2.015421in}}{\pgfqpoint{1.389694in}{2.021245in}}%
\pgfpathcurveto{\pgfqpoint{1.383870in}{2.027069in}}{\pgfqpoint{1.375970in}{2.030342in}}{\pgfqpoint{1.367733in}{2.030342in}}%
\pgfpathcurveto{\pgfqpoint{1.359497in}{2.030342in}}{\pgfqpoint{1.351597in}{2.027069in}}{\pgfqpoint{1.345773in}{2.021245in}}%
\pgfpathcurveto{\pgfqpoint{1.339949in}{2.015421in}}{\pgfqpoint{1.336677in}{2.007521in}}{\pgfqpoint{1.336677in}{1.999285in}}%
\pgfpathcurveto{\pgfqpoint{1.336677in}{1.991049in}}{\pgfqpoint{1.339949in}{1.983149in}}{\pgfqpoint{1.345773in}{1.977325in}}%
\pgfpathcurveto{\pgfqpoint{1.351597in}{1.971501in}}{\pgfqpoint{1.359497in}{1.968229in}}{\pgfqpoint{1.367733in}{1.968229in}}%
\pgfpathclose%
\pgfusepath{stroke,fill}%
\end{pgfscope}%
\begin{pgfscope}%
\pgfpathrectangle{\pgfqpoint{0.100000in}{0.212622in}}{\pgfqpoint{3.696000in}{3.696000in}}%
\pgfusepath{clip}%
\pgfsetbuttcap%
\pgfsetroundjoin%
\definecolor{currentfill}{rgb}{0.121569,0.466667,0.705882}%
\pgfsetfillcolor{currentfill}%
\pgfsetfillopacity{0.713335}%
\pgfsetlinewidth{1.003750pt}%
\definecolor{currentstroke}{rgb}{0.121569,0.466667,0.705882}%
\pgfsetstrokecolor{currentstroke}%
\pgfsetstrokeopacity{0.713335}%
\pgfsetdash{}{0pt}%
\pgfpathmoveto{\pgfqpoint{1.352447in}{1.954227in}}%
\pgfpathcurveto{\pgfqpoint{1.360684in}{1.954227in}}{\pgfqpoint{1.368584in}{1.957499in}}{\pgfqpoint{1.374408in}{1.963323in}}%
\pgfpathcurveto{\pgfqpoint{1.380232in}{1.969147in}}{\pgfqpoint{1.383504in}{1.977047in}}{\pgfqpoint{1.383504in}{1.985284in}}%
\pgfpathcurveto{\pgfqpoint{1.383504in}{1.993520in}}{\pgfqpoint{1.380232in}{2.001420in}}{\pgfqpoint{1.374408in}{2.007244in}}%
\pgfpathcurveto{\pgfqpoint{1.368584in}{2.013068in}}{\pgfqpoint{1.360684in}{2.016340in}}{\pgfqpoint{1.352447in}{2.016340in}}%
\pgfpathcurveto{\pgfqpoint{1.344211in}{2.016340in}}{\pgfqpoint{1.336311in}{2.013068in}}{\pgfqpoint{1.330487in}{2.007244in}}%
\pgfpathcurveto{\pgfqpoint{1.324663in}{2.001420in}}{\pgfqpoint{1.321391in}{1.993520in}}{\pgfqpoint{1.321391in}{1.985284in}}%
\pgfpathcurveto{\pgfqpoint{1.321391in}{1.977047in}}{\pgfqpoint{1.324663in}{1.969147in}}{\pgfqpoint{1.330487in}{1.963323in}}%
\pgfpathcurveto{\pgfqpoint{1.336311in}{1.957499in}}{\pgfqpoint{1.344211in}{1.954227in}}{\pgfqpoint{1.352447in}{1.954227in}}%
\pgfpathclose%
\pgfusepath{stroke,fill}%
\end{pgfscope}%
\begin{pgfscope}%
\pgfpathrectangle{\pgfqpoint{0.100000in}{0.212622in}}{\pgfqpoint{3.696000in}{3.696000in}}%
\pgfusepath{clip}%
\pgfsetbuttcap%
\pgfsetroundjoin%
\definecolor{currentfill}{rgb}{0.121569,0.466667,0.705882}%
\pgfsetfillcolor{currentfill}%
\pgfsetfillopacity{0.725143}%
\pgfsetlinewidth{1.003750pt}%
\definecolor{currentstroke}{rgb}{0.121569,0.466667,0.705882}%
\pgfsetstrokecolor{currentstroke}%
\pgfsetstrokeopacity{0.725143}%
\pgfsetdash{}{0pt}%
\pgfpathmoveto{\pgfqpoint{1.339614in}{1.954525in}}%
\pgfpathcurveto{\pgfqpoint{1.347850in}{1.954525in}}{\pgfqpoint{1.355751in}{1.957797in}}{\pgfqpoint{1.361574in}{1.963621in}}%
\pgfpathcurveto{\pgfqpoint{1.367398in}{1.969445in}}{\pgfqpoint{1.370671in}{1.977345in}}{\pgfqpoint{1.370671in}{1.985581in}}%
\pgfpathcurveto{\pgfqpoint{1.370671in}{1.993818in}}{\pgfqpoint{1.367398in}{2.001718in}}{\pgfqpoint{1.361574in}{2.007542in}}%
\pgfpathcurveto{\pgfqpoint{1.355751in}{2.013366in}}{\pgfqpoint{1.347850in}{2.016638in}}{\pgfqpoint{1.339614in}{2.016638in}}%
\pgfpathcurveto{\pgfqpoint{1.331378in}{2.016638in}}{\pgfqpoint{1.323478in}{2.013366in}}{\pgfqpoint{1.317654in}{2.007542in}}%
\pgfpathcurveto{\pgfqpoint{1.311830in}{2.001718in}}{\pgfqpoint{1.308558in}{1.993818in}}{\pgfqpoint{1.308558in}{1.985581in}}%
\pgfpathcurveto{\pgfqpoint{1.308558in}{1.977345in}}{\pgfqpoint{1.311830in}{1.969445in}}{\pgfqpoint{1.317654in}{1.963621in}}%
\pgfpathcurveto{\pgfqpoint{1.323478in}{1.957797in}}{\pgfqpoint{1.331378in}{1.954525in}}{\pgfqpoint{1.339614in}{1.954525in}}%
\pgfpathclose%
\pgfusepath{stroke,fill}%
\end{pgfscope}%
\begin{pgfscope}%
\pgfpathrectangle{\pgfqpoint{0.100000in}{0.212622in}}{\pgfqpoint{3.696000in}{3.696000in}}%
\pgfusepath{clip}%
\pgfsetbuttcap%
\pgfsetroundjoin%
\definecolor{currentfill}{rgb}{0.121569,0.466667,0.705882}%
\pgfsetfillcolor{currentfill}%
\pgfsetfillopacity{0.738336}%
\pgfsetlinewidth{1.003750pt}%
\definecolor{currentstroke}{rgb}{0.121569,0.466667,0.705882}%
\pgfsetstrokecolor{currentstroke}%
\pgfsetstrokeopacity{0.738336}%
\pgfsetdash{}{0pt}%
\pgfpathmoveto{\pgfqpoint{1.317435in}{1.927039in}}%
\pgfpathcurveto{\pgfqpoint{1.325671in}{1.927039in}}{\pgfqpoint{1.333571in}{1.930311in}}{\pgfqpoint{1.339395in}{1.936135in}}%
\pgfpathcurveto{\pgfqpoint{1.345219in}{1.941959in}}{\pgfqpoint{1.348492in}{1.949859in}}{\pgfqpoint{1.348492in}{1.958095in}}%
\pgfpathcurveto{\pgfqpoint{1.348492in}{1.966332in}}{\pgfqpoint{1.345219in}{1.974232in}}{\pgfqpoint{1.339395in}{1.980056in}}%
\pgfpathcurveto{\pgfqpoint{1.333571in}{1.985880in}}{\pgfqpoint{1.325671in}{1.989152in}}{\pgfqpoint{1.317435in}{1.989152in}}%
\pgfpathcurveto{\pgfqpoint{1.309199in}{1.989152in}}{\pgfqpoint{1.301299in}{1.985880in}}{\pgfqpoint{1.295475in}{1.980056in}}%
\pgfpathcurveto{\pgfqpoint{1.289651in}{1.974232in}}{\pgfqpoint{1.286379in}{1.966332in}}{\pgfqpoint{1.286379in}{1.958095in}}%
\pgfpathcurveto{\pgfqpoint{1.286379in}{1.949859in}}{\pgfqpoint{1.289651in}{1.941959in}}{\pgfqpoint{1.295475in}{1.936135in}}%
\pgfpathcurveto{\pgfqpoint{1.301299in}{1.930311in}}{\pgfqpoint{1.309199in}{1.927039in}}{\pgfqpoint{1.317435in}{1.927039in}}%
\pgfpathclose%
\pgfusepath{stroke,fill}%
\end{pgfscope}%
\begin{pgfscope}%
\pgfpathrectangle{\pgfqpoint{0.100000in}{0.212622in}}{\pgfqpoint{3.696000in}{3.696000in}}%
\pgfusepath{clip}%
\pgfsetbuttcap%
\pgfsetroundjoin%
\definecolor{currentfill}{rgb}{0.121569,0.466667,0.705882}%
\pgfsetfillcolor{currentfill}%
\pgfsetfillopacity{0.740948}%
\pgfsetlinewidth{1.003750pt}%
\definecolor{currentstroke}{rgb}{0.121569,0.466667,0.705882}%
\pgfsetstrokecolor{currentstroke}%
\pgfsetstrokeopacity{0.740948}%
\pgfsetdash{}{0pt}%
\pgfpathmoveto{\pgfqpoint{1.317993in}{1.926666in}}%
\pgfpathcurveto{\pgfqpoint{1.326229in}{1.926666in}}{\pgfqpoint{1.334129in}{1.929938in}}{\pgfqpoint{1.339953in}{1.935762in}}%
\pgfpathcurveto{\pgfqpoint{1.345777in}{1.941586in}}{\pgfqpoint{1.349049in}{1.949486in}}{\pgfqpoint{1.349049in}{1.957722in}}%
\pgfpathcurveto{\pgfqpoint{1.349049in}{1.965958in}}{\pgfqpoint{1.345777in}{1.973858in}}{\pgfqpoint{1.339953in}{1.979682in}}%
\pgfpathcurveto{\pgfqpoint{1.334129in}{1.985506in}}{\pgfqpoint{1.326229in}{1.988779in}}{\pgfqpoint{1.317993in}{1.988779in}}%
\pgfpathcurveto{\pgfqpoint{1.309757in}{1.988779in}}{\pgfqpoint{1.301856in}{1.985506in}}{\pgfqpoint{1.296033in}{1.979682in}}%
\pgfpathcurveto{\pgfqpoint{1.290209in}{1.973858in}}{\pgfqpoint{1.286936in}{1.965958in}}{\pgfqpoint{1.286936in}{1.957722in}}%
\pgfpathcurveto{\pgfqpoint{1.286936in}{1.949486in}}{\pgfqpoint{1.290209in}{1.941586in}}{\pgfqpoint{1.296033in}{1.935762in}}%
\pgfpathcurveto{\pgfqpoint{1.301856in}{1.929938in}}{\pgfqpoint{1.309757in}{1.926666in}}{\pgfqpoint{1.317993in}{1.926666in}}%
\pgfpathclose%
\pgfusepath{stroke,fill}%
\end{pgfscope}%
\begin{pgfscope}%
\pgfpathrectangle{\pgfqpoint{0.100000in}{0.212622in}}{\pgfqpoint{3.696000in}{3.696000in}}%
\pgfusepath{clip}%
\pgfsetbuttcap%
\pgfsetroundjoin%
\definecolor{currentfill}{rgb}{0.121569,0.466667,0.705882}%
\pgfsetfillcolor{currentfill}%
\pgfsetfillopacity{0.744171}%
\pgfsetlinewidth{1.003750pt}%
\definecolor{currentstroke}{rgb}{0.121569,0.466667,0.705882}%
\pgfsetstrokecolor{currentstroke}%
\pgfsetstrokeopacity{0.744171}%
\pgfsetdash{}{0pt}%
\pgfpathmoveto{\pgfqpoint{1.319000in}{1.928898in}}%
\pgfpathcurveto{\pgfqpoint{1.327236in}{1.928898in}}{\pgfqpoint{1.335136in}{1.932171in}}{\pgfqpoint{1.340960in}{1.937995in}}%
\pgfpathcurveto{\pgfqpoint{1.346784in}{1.943818in}}{\pgfqpoint{1.350056in}{1.951719in}}{\pgfqpoint{1.350056in}{1.959955in}}%
\pgfpathcurveto{\pgfqpoint{1.350056in}{1.968191in}}{\pgfqpoint{1.346784in}{1.976091in}}{\pgfqpoint{1.340960in}{1.981915in}}%
\pgfpathcurveto{\pgfqpoint{1.335136in}{1.987739in}}{\pgfqpoint{1.327236in}{1.991011in}}{\pgfqpoint{1.319000in}{1.991011in}}%
\pgfpathcurveto{\pgfqpoint{1.310764in}{1.991011in}}{\pgfqpoint{1.302864in}{1.987739in}}{\pgfqpoint{1.297040in}{1.981915in}}%
\pgfpathcurveto{\pgfqpoint{1.291216in}{1.976091in}}{\pgfqpoint{1.287943in}{1.968191in}}{\pgfqpoint{1.287943in}{1.959955in}}%
\pgfpathcurveto{\pgfqpoint{1.287943in}{1.951719in}}{\pgfqpoint{1.291216in}{1.943818in}}{\pgfqpoint{1.297040in}{1.937995in}}%
\pgfpathcurveto{\pgfqpoint{1.302864in}{1.932171in}}{\pgfqpoint{1.310764in}{1.928898in}}{\pgfqpoint{1.319000in}{1.928898in}}%
\pgfpathclose%
\pgfusepath{stroke,fill}%
\end{pgfscope}%
\begin{pgfscope}%
\pgfpathrectangle{\pgfqpoint{0.100000in}{0.212622in}}{\pgfqpoint{3.696000in}{3.696000in}}%
\pgfusepath{clip}%
\pgfsetbuttcap%
\pgfsetroundjoin%
\definecolor{currentfill}{rgb}{0.121569,0.466667,0.705882}%
\pgfsetfillcolor{currentfill}%
\pgfsetfillopacity{0.754813}%
\pgfsetlinewidth{1.003750pt}%
\definecolor{currentstroke}{rgb}{0.121569,0.466667,0.705882}%
\pgfsetstrokecolor{currentstroke}%
\pgfsetstrokeopacity{0.754813}%
\pgfsetdash{}{0pt}%
\pgfpathmoveto{\pgfqpoint{1.301604in}{1.906364in}}%
\pgfpathcurveto{\pgfqpoint{1.309840in}{1.906364in}}{\pgfqpoint{1.317740in}{1.909636in}}{\pgfqpoint{1.323564in}{1.915460in}}%
\pgfpathcurveto{\pgfqpoint{1.329388in}{1.921284in}}{\pgfqpoint{1.332661in}{1.929184in}}{\pgfqpoint{1.332661in}{1.937421in}}%
\pgfpathcurveto{\pgfqpoint{1.332661in}{1.945657in}}{\pgfqpoint{1.329388in}{1.953557in}}{\pgfqpoint{1.323564in}{1.959381in}}%
\pgfpathcurveto{\pgfqpoint{1.317740in}{1.965205in}}{\pgfqpoint{1.309840in}{1.968477in}}{\pgfqpoint{1.301604in}{1.968477in}}%
\pgfpathcurveto{\pgfqpoint{1.293368in}{1.968477in}}{\pgfqpoint{1.285468in}{1.965205in}}{\pgfqpoint{1.279644in}{1.959381in}}%
\pgfpathcurveto{\pgfqpoint{1.273820in}{1.953557in}}{\pgfqpoint{1.270548in}{1.945657in}}{\pgfqpoint{1.270548in}{1.937421in}}%
\pgfpathcurveto{\pgfqpoint{1.270548in}{1.929184in}}{\pgfqpoint{1.273820in}{1.921284in}}{\pgfqpoint{1.279644in}{1.915460in}}%
\pgfpathcurveto{\pgfqpoint{1.285468in}{1.909636in}}{\pgfqpoint{1.293368in}{1.906364in}}{\pgfqpoint{1.301604in}{1.906364in}}%
\pgfpathclose%
\pgfusepath{stroke,fill}%
\end{pgfscope}%
\begin{pgfscope}%
\pgfpathrectangle{\pgfqpoint{0.100000in}{0.212622in}}{\pgfqpoint{3.696000in}{3.696000in}}%
\pgfusepath{clip}%
\pgfsetbuttcap%
\pgfsetroundjoin%
\definecolor{currentfill}{rgb}{0.121569,0.466667,0.705882}%
\pgfsetfillcolor{currentfill}%
\pgfsetfillopacity{0.765027}%
\pgfsetlinewidth{1.003750pt}%
\definecolor{currentstroke}{rgb}{0.121569,0.466667,0.705882}%
\pgfsetstrokecolor{currentstroke}%
\pgfsetstrokeopacity{0.765027}%
\pgfsetdash{}{0pt}%
\pgfpathmoveto{\pgfqpoint{1.291375in}{1.899175in}}%
\pgfpathcurveto{\pgfqpoint{1.299611in}{1.899175in}}{\pgfqpoint{1.307511in}{1.902447in}}{\pgfqpoint{1.313335in}{1.908271in}}%
\pgfpathcurveto{\pgfqpoint{1.319159in}{1.914095in}}{\pgfqpoint{1.322431in}{1.921995in}}{\pgfqpoint{1.322431in}{1.930231in}}%
\pgfpathcurveto{\pgfqpoint{1.322431in}{1.938468in}}{\pgfqpoint{1.319159in}{1.946368in}}{\pgfqpoint{1.313335in}{1.952192in}}%
\pgfpathcurveto{\pgfqpoint{1.307511in}{1.958016in}}{\pgfqpoint{1.299611in}{1.961288in}}{\pgfqpoint{1.291375in}{1.961288in}}%
\pgfpathcurveto{\pgfqpoint{1.283139in}{1.961288in}}{\pgfqpoint{1.275239in}{1.958016in}}{\pgfqpoint{1.269415in}{1.952192in}}%
\pgfpathcurveto{\pgfqpoint{1.263591in}{1.946368in}}{\pgfqpoint{1.260318in}{1.938468in}}{\pgfqpoint{1.260318in}{1.930231in}}%
\pgfpathcurveto{\pgfqpoint{1.260318in}{1.921995in}}{\pgfqpoint{1.263591in}{1.914095in}}{\pgfqpoint{1.269415in}{1.908271in}}%
\pgfpathcurveto{\pgfqpoint{1.275239in}{1.902447in}}{\pgfqpoint{1.283139in}{1.899175in}}{\pgfqpoint{1.291375in}{1.899175in}}%
\pgfpathclose%
\pgfusepath{stroke,fill}%
\end{pgfscope}%
\begin{pgfscope}%
\pgfpathrectangle{\pgfqpoint{0.100000in}{0.212622in}}{\pgfqpoint{3.696000in}{3.696000in}}%
\pgfusepath{clip}%
\pgfsetbuttcap%
\pgfsetroundjoin%
\definecolor{currentfill}{rgb}{0.121569,0.466667,0.705882}%
\pgfsetfillcolor{currentfill}%
\pgfsetfillopacity{0.772122}%
\pgfsetlinewidth{1.003750pt}%
\definecolor{currentstroke}{rgb}{0.121569,0.466667,0.705882}%
\pgfsetstrokecolor{currentstroke}%
\pgfsetstrokeopacity{0.772122}%
\pgfsetdash{}{0pt}%
\pgfpathmoveto{\pgfqpoint{1.282389in}{1.892836in}}%
\pgfpathcurveto{\pgfqpoint{1.290625in}{1.892836in}}{\pgfqpoint{1.298525in}{1.896109in}}{\pgfqpoint{1.304349in}{1.901933in}}%
\pgfpathcurveto{\pgfqpoint{1.310173in}{1.907757in}}{\pgfqpoint{1.313445in}{1.915657in}}{\pgfqpoint{1.313445in}{1.923893in}}%
\pgfpathcurveto{\pgfqpoint{1.313445in}{1.932129in}}{\pgfqpoint{1.310173in}{1.940029in}}{\pgfqpoint{1.304349in}{1.945853in}}%
\pgfpathcurveto{\pgfqpoint{1.298525in}{1.951677in}}{\pgfqpoint{1.290625in}{1.954949in}}{\pgfqpoint{1.282389in}{1.954949in}}%
\pgfpathcurveto{\pgfqpoint{1.274152in}{1.954949in}}{\pgfqpoint{1.266252in}{1.951677in}}{\pgfqpoint{1.260428in}{1.945853in}}%
\pgfpathcurveto{\pgfqpoint{1.254604in}{1.940029in}}{\pgfqpoint{1.251332in}{1.932129in}}{\pgfqpoint{1.251332in}{1.923893in}}%
\pgfpathcurveto{\pgfqpoint{1.251332in}{1.915657in}}{\pgfqpoint{1.254604in}{1.907757in}}{\pgfqpoint{1.260428in}{1.901933in}}%
\pgfpathcurveto{\pgfqpoint{1.266252in}{1.896109in}}{\pgfqpoint{1.274152in}{1.892836in}}{\pgfqpoint{1.282389in}{1.892836in}}%
\pgfpathclose%
\pgfusepath{stroke,fill}%
\end{pgfscope}%
\begin{pgfscope}%
\pgfpathrectangle{\pgfqpoint{0.100000in}{0.212622in}}{\pgfqpoint{3.696000in}{3.696000in}}%
\pgfusepath{clip}%
\pgfsetbuttcap%
\pgfsetroundjoin%
\definecolor{currentfill}{rgb}{0.121569,0.466667,0.705882}%
\pgfsetfillcolor{currentfill}%
\pgfsetfillopacity{0.777020}%
\pgfsetlinewidth{1.003750pt}%
\definecolor{currentstroke}{rgb}{0.121569,0.466667,0.705882}%
\pgfsetstrokecolor{currentstroke}%
\pgfsetstrokeopacity{0.777020}%
\pgfsetdash{}{0pt}%
\pgfpathmoveto{\pgfqpoint{1.278125in}{1.885872in}}%
\pgfpathcurveto{\pgfqpoint{1.286361in}{1.885872in}}{\pgfqpoint{1.294261in}{1.889144in}}{\pgfqpoint{1.300085in}{1.894968in}}%
\pgfpathcurveto{\pgfqpoint{1.305909in}{1.900792in}}{\pgfqpoint{1.309181in}{1.908692in}}{\pgfqpoint{1.309181in}{1.916928in}}%
\pgfpathcurveto{\pgfqpoint{1.309181in}{1.925164in}}{\pgfqpoint{1.305909in}{1.933064in}}{\pgfqpoint{1.300085in}{1.938888in}}%
\pgfpathcurveto{\pgfqpoint{1.294261in}{1.944712in}}{\pgfqpoint{1.286361in}{1.947985in}}{\pgfqpoint{1.278125in}{1.947985in}}%
\pgfpathcurveto{\pgfqpoint{1.269889in}{1.947985in}}{\pgfqpoint{1.261989in}{1.944712in}}{\pgfqpoint{1.256165in}{1.938888in}}%
\pgfpathcurveto{\pgfqpoint{1.250341in}{1.933064in}}{\pgfqpoint{1.247068in}{1.925164in}}{\pgfqpoint{1.247068in}{1.916928in}}%
\pgfpathcurveto{\pgfqpoint{1.247068in}{1.908692in}}{\pgfqpoint{1.250341in}{1.900792in}}{\pgfqpoint{1.256165in}{1.894968in}}%
\pgfpathcurveto{\pgfqpoint{1.261989in}{1.889144in}}{\pgfqpoint{1.269889in}{1.885872in}}{\pgfqpoint{1.278125in}{1.885872in}}%
\pgfpathclose%
\pgfusepath{stroke,fill}%
\end{pgfscope}%
\begin{pgfscope}%
\pgfpathrectangle{\pgfqpoint{0.100000in}{0.212622in}}{\pgfqpoint{3.696000in}{3.696000in}}%
\pgfusepath{clip}%
\pgfsetbuttcap%
\pgfsetroundjoin%
\definecolor{currentfill}{rgb}{0.121569,0.466667,0.705882}%
\pgfsetfillcolor{currentfill}%
\pgfsetfillopacity{0.782565}%
\pgfsetlinewidth{1.003750pt}%
\definecolor{currentstroke}{rgb}{0.121569,0.466667,0.705882}%
\pgfsetstrokecolor{currentstroke}%
\pgfsetstrokeopacity{0.782565}%
\pgfsetdash{}{0pt}%
\pgfpathmoveto{\pgfqpoint{1.277471in}{1.890431in}}%
\pgfpathcurveto{\pgfqpoint{1.285707in}{1.890431in}}{\pgfqpoint{1.293607in}{1.893704in}}{\pgfqpoint{1.299431in}{1.899528in}}%
\pgfpathcurveto{\pgfqpoint{1.305255in}{1.905351in}}{\pgfqpoint{1.308527in}{1.913252in}}{\pgfqpoint{1.308527in}{1.921488in}}%
\pgfpathcurveto{\pgfqpoint{1.308527in}{1.929724in}}{\pgfqpoint{1.305255in}{1.937624in}}{\pgfqpoint{1.299431in}{1.943448in}}%
\pgfpathcurveto{\pgfqpoint{1.293607in}{1.949272in}}{\pgfqpoint{1.285707in}{1.952544in}}{\pgfqpoint{1.277471in}{1.952544in}}%
\pgfpathcurveto{\pgfqpoint{1.269234in}{1.952544in}}{\pgfqpoint{1.261334in}{1.949272in}}{\pgfqpoint{1.255510in}{1.943448in}}%
\pgfpathcurveto{\pgfqpoint{1.249686in}{1.937624in}}{\pgfqpoint{1.246414in}{1.929724in}}{\pgfqpoint{1.246414in}{1.921488in}}%
\pgfpathcurveto{\pgfqpoint{1.246414in}{1.913252in}}{\pgfqpoint{1.249686in}{1.905351in}}{\pgfqpoint{1.255510in}{1.899528in}}%
\pgfpathcurveto{\pgfqpoint{1.261334in}{1.893704in}}{\pgfqpoint{1.269234in}{1.890431in}}{\pgfqpoint{1.277471in}{1.890431in}}%
\pgfpathclose%
\pgfusepath{stroke,fill}%
\end{pgfscope}%
\begin{pgfscope}%
\pgfpathrectangle{\pgfqpoint{0.100000in}{0.212622in}}{\pgfqpoint{3.696000in}{3.696000in}}%
\pgfusepath{clip}%
\pgfsetbuttcap%
\pgfsetroundjoin%
\definecolor{currentfill}{rgb}{0.121569,0.466667,0.705882}%
\pgfsetfillcolor{currentfill}%
\pgfsetfillopacity{0.792238}%
\pgfsetlinewidth{1.003750pt}%
\definecolor{currentstroke}{rgb}{0.121569,0.466667,0.705882}%
\pgfsetstrokecolor{currentstroke}%
\pgfsetstrokeopacity{0.792238}%
\pgfsetdash{}{0pt}%
\pgfpathmoveto{\pgfqpoint{1.284692in}{1.880617in}}%
\pgfpathcurveto{\pgfqpoint{1.292929in}{1.880617in}}{\pgfqpoint{1.300829in}{1.883889in}}{\pgfqpoint{1.306653in}{1.889713in}}%
\pgfpathcurveto{\pgfqpoint{1.312476in}{1.895537in}}{\pgfqpoint{1.315749in}{1.903437in}}{\pgfqpoint{1.315749in}{1.911674in}}%
\pgfpathcurveto{\pgfqpoint{1.315749in}{1.919910in}}{\pgfqpoint{1.312476in}{1.927810in}}{\pgfqpoint{1.306653in}{1.933634in}}%
\pgfpathcurveto{\pgfqpoint{1.300829in}{1.939458in}}{\pgfqpoint{1.292929in}{1.942730in}}{\pgfqpoint{1.284692in}{1.942730in}}%
\pgfpathcurveto{\pgfqpoint{1.276456in}{1.942730in}}{\pgfqpoint{1.268556in}{1.939458in}}{\pgfqpoint{1.262732in}{1.933634in}}%
\pgfpathcurveto{\pgfqpoint{1.256908in}{1.927810in}}{\pgfqpoint{1.253636in}{1.919910in}}{\pgfqpoint{1.253636in}{1.911674in}}%
\pgfpathcurveto{\pgfqpoint{1.253636in}{1.903437in}}{\pgfqpoint{1.256908in}{1.895537in}}{\pgfqpoint{1.262732in}{1.889713in}}%
\pgfpathcurveto{\pgfqpoint{1.268556in}{1.883889in}}{\pgfqpoint{1.276456in}{1.880617in}}{\pgfqpoint{1.284692in}{1.880617in}}%
\pgfpathclose%
\pgfusepath{stroke,fill}%
\end{pgfscope}%
\begin{pgfscope}%
\pgfpathrectangle{\pgfqpoint{0.100000in}{0.212622in}}{\pgfqpoint{3.696000in}{3.696000in}}%
\pgfusepath{clip}%
\pgfsetbuttcap%
\pgfsetroundjoin%
\definecolor{currentfill}{rgb}{0.121569,0.466667,0.705882}%
\pgfsetfillcolor{currentfill}%
\pgfsetfillopacity{0.795367}%
\pgfsetlinewidth{1.003750pt}%
\definecolor{currentstroke}{rgb}{0.121569,0.466667,0.705882}%
\pgfsetstrokecolor{currentstroke}%
\pgfsetstrokeopacity{0.795367}%
\pgfsetdash{}{0pt}%
\pgfpathmoveto{\pgfqpoint{1.286801in}{1.881943in}}%
\pgfpathcurveto{\pgfqpoint{1.295038in}{1.881943in}}{\pgfqpoint{1.302938in}{1.885215in}}{\pgfqpoint{1.308762in}{1.891039in}}%
\pgfpathcurveto{\pgfqpoint{1.314585in}{1.896863in}}{\pgfqpoint{1.317858in}{1.904763in}}{\pgfqpoint{1.317858in}{1.912999in}}%
\pgfpathcurveto{\pgfqpoint{1.317858in}{1.921236in}}{\pgfqpoint{1.314585in}{1.929136in}}{\pgfqpoint{1.308762in}{1.934960in}}%
\pgfpathcurveto{\pgfqpoint{1.302938in}{1.940784in}}{\pgfqpoint{1.295038in}{1.944056in}}{\pgfqpoint{1.286801in}{1.944056in}}%
\pgfpathcurveto{\pgfqpoint{1.278565in}{1.944056in}}{\pgfqpoint{1.270665in}{1.940784in}}{\pgfqpoint{1.264841in}{1.934960in}}%
\pgfpathcurveto{\pgfqpoint{1.259017in}{1.929136in}}{\pgfqpoint{1.255745in}{1.921236in}}{\pgfqpoint{1.255745in}{1.912999in}}%
\pgfpathcurveto{\pgfqpoint{1.255745in}{1.904763in}}{\pgfqpoint{1.259017in}{1.896863in}}{\pgfqpoint{1.264841in}{1.891039in}}%
\pgfpathcurveto{\pgfqpoint{1.270665in}{1.885215in}}{\pgfqpoint{1.278565in}{1.881943in}}{\pgfqpoint{1.286801in}{1.881943in}}%
\pgfpathclose%
\pgfusepath{stroke,fill}%
\end{pgfscope}%
\begin{pgfscope}%
\pgfpathrectangle{\pgfqpoint{0.100000in}{0.212622in}}{\pgfqpoint{3.696000in}{3.696000in}}%
\pgfusepath{clip}%
\pgfsetbuttcap%
\pgfsetroundjoin%
\definecolor{currentfill}{rgb}{0.121569,0.466667,0.705882}%
\pgfsetfillcolor{currentfill}%
\pgfsetfillopacity{0.796444}%
\pgfsetlinewidth{1.003750pt}%
\definecolor{currentstroke}{rgb}{0.121569,0.466667,0.705882}%
\pgfsetstrokecolor{currentstroke}%
\pgfsetstrokeopacity{0.796444}%
\pgfsetdash{}{0pt}%
\pgfpathmoveto{\pgfqpoint{1.277448in}{1.871130in}}%
\pgfpathcurveto{\pgfqpoint{1.285685in}{1.871130in}}{\pgfqpoint{1.293585in}{1.874402in}}{\pgfqpoint{1.299409in}{1.880226in}}%
\pgfpathcurveto{\pgfqpoint{1.305232in}{1.886050in}}{\pgfqpoint{1.308505in}{1.893950in}}{\pgfqpoint{1.308505in}{1.902186in}}%
\pgfpathcurveto{\pgfqpoint{1.308505in}{1.910422in}}{\pgfqpoint{1.305232in}{1.918322in}}{\pgfqpoint{1.299409in}{1.924146in}}%
\pgfpathcurveto{\pgfqpoint{1.293585in}{1.929970in}}{\pgfqpoint{1.285685in}{1.933243in}}{\pgfqpoint{1.277448in}{1.933243in}}%
\pgfpathcurveto{\pgfqpoint{1.269212in}{1.933243in}}{\pgfqpoint{1.261312in}{1.929970in}}{\pgfqpoint{1.255488in}{1.924146in}}%
\pgfpathcurveto{\pgfqpoint{1.249664in}{1.918322in}}{\pgfqpoint{1.246392in}{1.910422in}}{\pgfqpoint{1.246392in}{1.902186in}}%
\pgfpathcurveto{\pgfqpoint{1.246392in}{1.893950in}}{\pgfqpoint{1.249664in}{1.886050in}}{\pgfqpoint{1.255488in}{1.880226in}}%
\pgfpathcurveto{\pgfqpoint{1.261312in}{1.874402in}}{\pgfqpoint{1.269212in}{1.871130in}}{\pgfqpoint{1.277448in}{1.871130in}}%
\pgfpathclose%
\pgfusepath{stroke,fill}%
\end{pgfscope}%
\begin{pgfscope}%
\pgfpathrectangle{\pgfqpoint{0.100000in}{0.212622in}}{\pgfqpoint{3.696000in}{3.696000in}}%
\pgfusepath{clip}%
\pgfsetbuttcap%
\pgfsetroundjoin%
\definecolor{currentfill}{rgb}{0.121569,0.466667,0.705882}%
\pgfsetfillcolor{currentfill}%
\pgfsetfillopacity{0.797810}%
\pgfsetlinewidth{1.003750pt}%
\definecolor{currentstroke}{rgb}{0.121569,0.466667,0.705882}%
\pgfsetstrokecolor{currentstroke}%
\pgfsetstrokeopacity{0.797810}%
\pgfsetdash{}{0pt}%
\pgfpathmoveto{\pgfqpoint{1.289674in}{1.887032in}}%
\pgfpathcurveto{\pgfqpoint{1.297910in}{1.887032in}}{\pgfqpoint{1.305810in}{1.890305in}}{\pgfqpoint{1.311634in}{1.896129in}}%
\pgfpathcurveto{\pgfqpoint{1.317458in}{1.901953in}}{\pgfqpoint{1.320730in}{1.909853in}}{\pgfqpoint{1.320730in}{1.918089in}}%
\pgfpathcurveto{\pgfqpoint{1.320730in}{1.926325in}}{\pgfqpoint{1.317458in}{1.934225in}}{\pgfqpoint{1.311634in}{1.940049in}}%
\pgfpathcurveto{\pgfqpoint{1.305810in}{1.945873in}}{\pgfqpoint{1.297910in}{1.949145in}}{\pgfqpoint{1.289674in}{1.949145in}}%
\pgfpathcurveto{\pgfqpoint{1.281437in}{1.949145in}}{\pgfqpoint{1.273537in}{1.945873in}}{\pgfqpoint{1.267713in}{1.940049in}}%
\pgfpathcurveto{\pgfqpoint{1.261889in}{1.934225in}}{\pgfqpoint{1.258617in}{1.926325in}}{\pgfqpoint{1.258617in}{1.918089in}}%
\pgfpathcurveto{\pgfqpoint{1.258617in}{1.909853in}}{\pgfqpoint{1.261889in}{1.901953in}}{\pgfqpoint{1.267713in}{1.896129in}}%
\pgfpathcurveto{\pgfqpoint{1.273537in}{1.890305in}}{\pgfqpoint{1.281437in}{1.887032in}}{\pgfqpoint{1.289674in}{1.887032in}}%
\pgfpathclose%
\pgfusepath{stroke,fill}%
\end{pgfscope}%
\begin{pgfscope}%
\pgfpathrectangle{\pgfqpoint{0.100000in}{0.212622in}}{\pgfqpoint{3.696000in}{3.696000in}}%
\pgfusepath{clip}%
\pgfsetbuttcap%
\pgfsetroundjoin%
\definecolor{currentfill}{rgb}{0.121569,0.466667,0.705882}%
\pgfsetfillcolor{currentfill}%
\pgfsetfillopacity{0.800095}%
\pgfsetlinewidth{1.003750pt}%
\definecolor{currentstroke}{rgb}{0.121569,0.466667,0.705882}%
\pgfsetstrokecolor{currentstroke}%
\pgfsetstrokeopacity{0.800095}%
\pgfsetdash{}{0pt}%
\pgfpathmoveto{\pgfqpoint{1.256710in}{1.852384in}}%
\pgfpathcurveto{\pgfqpoint{1.264946in}{1.852384in}}{\pgfqpoint{1.272846in}{1.855657in}}{\pgfqpoint{1.278670in}{1.861481in}}%
\pgfpathcurveto{\pgfqpoint{1.284494in}{1.867305in}}{\pgfqpoint{1.287766in}{1.875205in}}{\pgfqpoint{1.287766in}{1.883441in}}%
\pgfpathcurveto{\pgfqpoint{1.287766in}{1.891677in}}{\pgfqpoint{1.284494in}{1.899577in}}{\pgfqpoint{1.278670in}{1.905401in}}%
\pgfpathcurveto{\pgfqpoint{1.272846in}{1.911225in}}{\pgfqpoint{1.264946in}{1.914497in}}{\pgfqpoint{1.256710in}{1.914497in}}%
\pgfpathcurveto{\pgfqpoint{1.248473in}{1.914497in}}{\pgfqpoint{1.240573in}{1.911225in}}{\pgfqpoint{1.234750in}{1.905401in}}%
\pgfpathcurveto{\pgfqpoint{1.228926in}{1.899577in}}{\pgfqpoint{1.225653in}{1.891677in}}{\pgfqpoint{1.225653in}{1.883441in}}%
\pgfpathcurveto{\pgfqpoint{1.225653in}{1.875205in}}{\pgfqpoint{1.228926in}{1.867305in}}{\pgfqpoint{1.234750in}{1.861481in}}%
\pgfpathcurveto{\pgfqpoint{1.240573in}{1.855657in}}{\pgfqpoint{1.248473in}{1.852384in}}{\pgfqpoint{1.256710in}{1.852384in}}%
\pgfpathclose%
\pgfusepath{stroke,fill}%
\end{pgfscope}%
\begin{pgfscope}%
\pgfpathrectangle{\pgfqpoint{0.100000in}{0.212622in}}{\pgfqpoint{3.696000in}{3.696000in}}%
\pgfusepath{clip}%
\pgfsetbuttcap%
\pgfsetroundjoin%
\definecolor{currentfill}{rgb}{0.121569,0.466667,0.705882}%
\pgfsetfillcolor{currentfill}%
\pgfsetfillopacity{0.801566}%
\pgfsetlinewidth{1.003750pt}%
\definecolor{currentstroke}{rgb}{0.121569,0.466667,0.705882}%
\pgfsetstrokecolor{currentstroke}%
\pgfsetstrokeopacity{0.801566}%
\pgfsetdash{}{0pt}%
\pgfpathmoveto{\pgfqpoint{1.247208in}{1.854226in}}%
\pgfpathcurveto{\pgfqpoint{1.255445in}{1.854226in}}{\pgfqpoint{1.263345in}{1.857499in}}{\pgfqpoint{1.269169in}{1.863323in}}%
\pgfpathcurveto{\pgfqpoint{1.274993in}{1.869147in}}{\pgfqpoint{1.278265in}{1.877047in}}{\pgfqpoint{1.278265in}{1.885283in}}%
\pgfpathcurveto{\pgfqpoint{1.278265in}{1.893519in}}{\pgfqpoint{1.274993in}{1.901419in}}{\pgfqpoint{1.269169in}{1.907243in}}%
\pgfpathcurveto{\pgfqpoint{1.263345in}{1.913067in}}{\pgfqpoint{1.255445in}{1.916339in}}{\pgfqpoint{1.247208in}{1.916339in}}%
\pgfpathcurveto{\pgfqpoint{1.238972in}{1.916339in}}{\pgfqpoint{1.231072in}{1.913067in}}{\pgfqpoint{1.225248in}{1.907243in}}%
\pgfpathcurveto{\pgfqpoint{1.219424in}{1.901419in}}{\pgfqpoint{1.216152in}{1.893519in}}{\pgfqpoint{1.216152in}{1.885283in}}%
\pgfpathcurveto{\pgfqpoint{1.216152in}{1.877047in}}{\pgfqpoint{1.219424in}{1.869147in}}{\pgfqpoint{1.225248in}{1.863323in}}%
\pgfpathcurveto{\pgfqpoint{1.231072in}{1.857499in}}{\pgfqpoint{1.238972in}{1.854226in}}{\pgfqpoint{1.247208in}{1.854226in}}%
\pgfpathclose%
\pgfusepath{stroke,fill}%
\end{pgfscope}%
\begin{pgfscope}%
\pgfpathrectangle{\pgfqpoint{0.100000in}{0.212622in}}{\pgfqpoint{3.696000in}{3.696000in}}%
\pgfusepath{clip}%
\pgfsetbuttcap%
\pgfsetroundjoin%
\definecolor{currentfill}{rgb}{0.121569,0.466667,0.705882}%
\pgfsetfillcolor{currentfill}%
\pgfsetfillopacity{0.801651}%
\pgfsetlinewidth{1.003750pt}%
\definecolor{currentstroke}{rgb}{0.121569,0.466667,0.705882}%
\pgfsetstrokecolor{currentstroke}%
\pgfsetstrokeopacity{0.801651}%
\pgfsetdash{}{0pt}%
\pgfpathmoveto{\pgfqpoint{1.288158in}{1.882870in}}%
\pgfpathcurveto{\pgfqpoint{1.296394in}{1.882870in}}{\pgfqpoint{1.304294in}{1.886142in}}{\pgfqpoint{1.310118in}{1.891966in}}%
\pgfpathcurveto{\pgfqpoint{1.315942in}{1.897790in}}{\pgfqpoint{1.319215in}{1.905690in}}{\pgfqpoint{1.319215in}{1.913927in}}%
\pgfpathcurveto{\pgfqpoint{1.319215in}{1.922163in}}{\pgfqpoint{1.315942in}{1.930063in}}{\pgfqpoint{1.310118in}{1.935887in}}%
\pgfpathcurveto{\pgfqpoint{1.304294in}{1.941711in}}{\pgfqpoint{1.296394in}{1.944983in}}{\pgfqpoint{1.288158in}{1.944983in}}%
\pgfpathcurveto{\pgfqpoint{1.279922in}{1.944983in}}{\pgfqpoint{1.272022in}{1.941711in}}{\pgfqpoint{1.266198in}{1.935887in}}%
\pgfpathcurveto{\pgfqpoint{1.260374in}{1.930063in}}{\pgfqpoint{1.257102in}{1.922163in}}{\pgfqpoint{1.257102in}{1.913927in}}%
\pgfpathcurveto{\pgfqpoint{1.257102in}{1.905690in}}{\pgfqpoint{1.260374in}{1.897790in}}{\pgfqpoint{1.266198in}{1.891966in}}%
\pgfpathcurveto{\pgfqpoint{1.272022in}{1.886142in}}{\pgfqpoint{1.279922in}{1.882870in}}{\pgfqpoint{1.288158in}{1.882870in}}%
\pgfpathclose%
\pgfusepath{stroke,fill}%
\end{pgfscope}%
\begin{pgfscope}%
\pgfpathrectangle{\pgfqpoint{0.100000in}{0.212622in}}{\pgfqpoint{3.696000in}{3.696000in}}%
\pgfusepath{clip}%
\pgfsetbuttcap%
\pgfsetroundjoin%
\definecolor{currentfill}{rgb}{0.121569,0.466667,0.705882}%
\pgfsetfillcolor{currentfill}%
\pgfsetfillopacity{0.802177}%
\pgfsetlinewidth{1.003750pt}%
\definecolor{currentstroke}{rgb}{0.121569,0.466667,0.705882}%
\pgfsetstrokecolor{currentstroke}%
\pgfsetstrokeopacity{0.802177}%
\pgfsetdash{}{0pt}%
\pgfpathmoveto{\pgfqpoint{1.247402in}{1.848736in}}%
\pgfpathcurveto{\pgfqpoint{1.255639in}{1.848736in}}{\pgfqpoint{1.263539in}{1.852008in}}{\pgfqpoint{1.269362in}{1.857832in}}%
\pgfpathcurveto{\pgfqpoint{1.275186in}{1.863656in}}{\pgfqpoint{1.278459in}{1.871556in}}{\pgfqpoint{1.278459in}{1.879792in}}%
\pgfpathcurveto{\pgfqpoint{1.278459in}{1.888028in}}{\pgfqpoint{1.275186in}{1.895929in}}{\pgfqpoint{1.269362in}{1.901752in}}%
\pgfpathcurveto{\pgfqpoint{1.263539in}{1.907576in}}{\pgfqpoint{1.255639in}{1.910849in}}{\pgfqpoint{1.247402in}{1.910849in}}%
\pgfpathcurveto{\pgfqpoint{1.239166in}{1.910849in}}{\pgfqpoint{1.231266in}{1.907576in}}{\pgfqpoint{1.225442in}{1.901752in}}%
\pgfpathcurveto{\pgfqpoint{1.219618in}{1.895929in}}{\pgfqpoint{1.216346in}{1.888028in}}{\pgfqpoint{1.216346in}{1.879792in}}%
\pgfpathcurveto{\pgfqpoint{1.216346in}{1.871556in}}{\pgfqpoint{1.219618in}{1.863656in}}{\pgfqpoint{1.225442in}{1.857832in}}%
\pgfpathcurveto{\pgfqpoint{1.231266in}{1.852008in}}{\pgfqpoint{1.239166in}{1.848736in}}{\pgfqpoint{1.247402in}{1.848736in}}%
\pgfpathclose%
\pgfusepath{stroke,fill}%
\end{pgfscope}%
\begin{pgfscope}%
\pgfpathrectangle{\pgfqpoint{0.100000in}{0.212622in}}{\pgfqpoint{3.696000in}{3.696000in}}%
\pgfusepath{clip}%
\pgfsetbuttcap%
\pgfsetroundjoin%
\definecolor{currentfill}{rgb}{0.121569,0.466667,0.705882}%
\pgfsetfillcolor{currentfill}%
\pgfsetfillopacity{0.803635}%
\pgfsetlinewidth{1.003750pt}%
\definecolor{currentstroke}{rgb}{0.121569,0.466667,0.705882}%
\pgfsetstrokecolor{currentstroke}%
\pgfsetstrokeopacity{0.803635}%
\pgfsetdash{}{0pt}%
\pgfpathmoveto{\pgfqpoint{1.286987in}{1.878593in}}%
\pgfpathcurveto{\pgfqpoint{1.295223in}{1.878593in}}{\pgfqpoint{1.303123in}{1.881865in}}{\pgfqpoint{1.308947in}{1.887689in}}%
\pgfpathcurveto{\pgfqpoint{1.314771in}{1.893513in}}{\pgfqpoint{1.318043in}{1.901413in}}{\pgfqpoint{1.318043in}{1.909649in}}%
\pgfpathcurveto{\pgfqpoint{1.318043in}{1.917885in}}{\pgfqpoint{1.314771in}{1.925785in}}{\pgfqpoint{1.308947in}{1.931609in}}%
\pgfpathcurveto{\pgfqpoint{1.303123in}{1.937433in}}{\pgfqpoint{1.295223in}{1.940706in}}{\pgfqpoint{1.286987in}{1.940706in}}%
\pgfpathcurveto{\pgfqpoint{1.278750in}{1.940706in}}{\pgfqpoint{1.270850in}{1.937433in}}{\pgfqpoint{1.265026in}{1.931609in}}%
\pgfpathcurveto{\pgfqpoint{1.259202in}{1.925785in}}{\pgfqpoint{1.255930in}{1.917885in}}{\pgfqpoint{1.255930in}{1.909649in}}%
\pgfpathcurveto{\pgfqpoint{1.255930in}{1.901413in}}{\pgfqpoint{1.259202in}{1.893513in}}{\pgfqpoint{1.265026in}{1.887689in}}%
\pgfpathcurveto{\pgfqpoint{1.270850in}{1.881865in}}{\pgfqpoint{1.278750in}{1.878593in}}{\pgfqpoint{1.286987in}{1.878593in}}%
\pgfpathclose%
\pgfusepath{stroke,fill}%
\end{pgfscope}%
\begin{pgfscope}%
\pgfpathrectangle{\pgfqpoint{0.100000in}{0.212622in}}{\pgfqpoint{3.696000in}{3.696000in}}%
\pgfusepath{clip}%
\pgfsetbuttcap%
\pgfsetroundjoin%
\definecolor{currentfill}{rgb}{0.121569,0.466667,0.705882}%
\pgfsetfillcolor{currentfill}%
\pgfsetfillopacity{0.804393}%
\pgfsetlinewidth{1.003750pt}%
\definecolor{currentstroke}{rgb}{0.121569,0.466667,0.705882}%
\pgfsetstrokecolor{currentstroke}%
\pgfsetstrokeopacity{0.804393}%
\pgfsetdash{}{0pt}%
\pgfpathmoveto{\pgfqpoint{1.287948in}{1.879713in}}%
\pgfpathcurveto{\pgfqpoint{1.296184in}{1.879713in}}{\pgfqpoint{1.304084in}{1.882986in}}{\pgfqpoint{1.309908in}{1.888810in}}%
\pgfpathcurveto{\pgfqpoint{1.315732in}{1.894633in}}{\pgfqpoint{1.319004in}{1.902534in}}{\pgfqpoint{1.319004in}{1.910770in}}%
\pgfpathcurveto{\pgfqpoint{1.319004in}{1.919006in}}{\pgfqpoint{1.315732in}{1.926906in}}{\pgfqpoint{1.309908in}{1.932730in}}%
\pgfpathcurveto{\pgfqpoint{1.304084in}{1.938554in}}{\pgfqpoint{1.296184in}{1.941826in}}{\pgfqpoint{1.287948in}{1.941826in}}%
\pgfpathcurveto{\pgfqpoint{1.279712in}{1.941826in}}{\pgfqpoint{1.271811in}{1.938554in}}{\pgfqpoint{1.265988in}{1.932730in}}%
\pgfpathcurveto{\pgfqpoint{1.260164in}{1.926906in}}{\pgfqpoint{1.256891in}{1.919006in}}{\pgfqpoint{1.256891in}{1.910770in}}%
\pgfpathcurveto{\pgfqpoint{1.256891in}{1.902534in}}{\pgfqpoint{1.260164in}{1.894633in}}{\pgfqpoint{1.265988in}{1.888810in}}%
\pgfpathcurveto{\pgfqpoint{1.271811in}{1.882986in}}{\pgfqpoint{1.279712in}{1.879713in}}{\pgfqpoint{1.287948in}{1.879713in}}%
\pgfpathclose%
\pgfusepath{stroke,fill}%
\end{pgfscope}%
\begin{pgfscope}%
\pgfpathrectangle{\pgfqpoint{0.100000in}{0.212622in}}{\pgfqpoint{3.696000in}{3.696000in}}%
\pgfusepath{clip}%
\pgfsetbuttcap%
\pgfsetroundjoin%
\definecolor{currentfill}{rgb}{0.121569,0.466667,0.705882}%
\pgfsetfillcolor{currentfill}%
\pgfsetfillopacity{0.804839}%
\pgfsetlinewidth{1.003750pt}%
\definecolor{currentstroke}{rgb}{0.121569,0.466667,0.705882}%
\pgfsetstrokecolor{currentstroke}%
\pgfsetstrokeopacity{0.804839}%
\pgfsetdash{}{0pt}%
\pgfpathmoveto{\pgfqpoint{1.232439in}{1.841745in}}%
\pgfpathcurveto{\pgfqpoint{1.240675in}{1.841745in}}{\pgfqpoint{1.248575in}{1.845017in}}{\pgfqpoint{1.254399in}{1.850841in}}%
\pgfpathcurveto{\pgfqpoint{1.260223in}{1.856665in}}{\pgfqpoint{1.263495in}{1.864565in}}{\pgfqpoint{1.263495in}{1.872801in}}%
\pgfpathcurveto{\pgfqpoint{1.263495in}{1.881038in}}{\pgfqpoint{1.260223in}{1.888938in}}{\pgfqpoint{1.254399in}{1.894762in}}%
\pgfpathcurveto{\pgfqpoint{1.248575in}{1.900586in}}{\pgfqpoint{1.240675in}{1.903858in}}{\pgfqpoint{1.232439in}{1.903858in}}%
\pgfpathcurveto{\pgfqpoint{1.224202in}{1.903858in}}{\pgfqpoint{1.216302in}{1.900586in}}{\pgfqpoint{1.210478in}{1.894762in}}%
\pgfpathcurveto{\pgfqpoint{1.204655in}{1.888938in}}{\pgfqpoint{1.201382in}{1.881038in}}{\pgfqpoint{1.201382in}{1.872801in}}%
\pgfpathcurveto{\pgfqpoint{1.201382in}{1.864565in}}{\pgfqpoint{1.204655in}{1.856665in}}{\pgfqpoint{1.210478in}{1.850841in}}%
\pgfpathcurveto{\pgfqpoint{1.216302in}{1.845017in}}{\pgfqpoint{1.224202in}{1.841745in}}{\pgfqpoint{1.232439in}{1.841745in}}%
\pgfpathclose%
\pgfusepath{stroke,fill}%
\end{pgfscope}%
\begin{pgfscope}%
\pgfpathrectangle{\pgfqpoint{0.100000in}{0.212622in}}{\pgfqpoint{3.696000in}{3.696000in}}%
\pgfusepath{clip}%
\pgfsetbuttcap%
\pgfsetroundjoin%
\definecolor{currentfill}{rgb}{0.121569,0.466667,0.705882}%
\pgfsetfillcolor{currentfill}%
\pgfsetfillopacity{0.808319}%
\pgfsetlinewidth{1.003750pt}%
\definecolor{currentstroke}{rgb}{0.121569,0.466667,0.705882}%
\pgfsetstrokecolor{currentstroke}%
\pgfsetstrokeopacity{0.808319}%
\pgfsetdash{}{0pt}%
\pgfpathmoveto{\pgfqpoint{1.284700in}{1.875240in}}%
\pgfpathcurveto{\pgfqpoint{1.292937in}{1.875240in}}{\pgfqpoint{1.300837in}{1.878512in}}{\pgfqpoint{1.306661in}{1.884336in}}%
\pgfpathcurveto{\pgfqpoint{1.312485in}{1.890160in}}{\pgfqpoint{1.315757in}{1.898060in}}{\pgfqpoint{1.315757in}{1.906297in}}%
\pgfpathcurveto{\pgfqpoint{1.315757in}{1.914533in}}{\pgfqpoint{1.312485in}{1.922433in}}{\pgfqpoint{1.306661in}{1.928257in}}%
\pgfpathcurveto{\pgfqpoint{1.300837in}{1.934081in}}{\pgfqpoint{1.292937in}{1.937353in}}{\pgfqpoint{1.284700in}{1.937353in}}%
\pgfpathcurveto{\pgfqpoint{1.276464in}{1.937353in}}{\pgfqpoint{1.268564in}{1.934081in}}{\pgfqpoint{1.262740in}{1.928257in}}%
\pgfpathcurveto{\pgfqpoint{1.256916in}{1.922433in}}{\pgfqpoint{1.253644in}{1.914533in}}{\pgfqpoint{1.253644in}{1.906297in}}%
\pgfpathcurveto{\pgfqpoint{1.253644in}{1.898060in}}{\pgfqpoint{1.256916in}{1.890160in}}{\pgfqpoint{1.262740in}{1.884336in}}%
\pgfpathcurveto{\pgfqpoint{1.268564in}{1.878512in}}{\pgfqpoint{1.276464in}{1.875240in}}{\pgfqpoint{1.284700in}{1.875240in}}%
\pgfpathclose%
\pgfusepath{stroke,fill}%
\end{pgfscope}%
\begin{pgfscope}%
\pgfpathrectangle{\pgfqpoint{0.100000in}{0.212622in}}{\pgfqpoint{3.696000in}{3.696000in}}%
\pgfusepath{clip}%
\pgfsetbuttcap%
\pgfsetroundjoin%
\definecolor{currentfill}{rgb}{0.121569,0.466667,0.705882}%
\pgfsetfillcolor{currentfill}%
\pgfsetfillopacity{0.809922}%
\pgfsetlinewidth{1.003750pt}%
\definecolor{currentstroke}{rgb}{0.121569,0.466667,0.705882}%
\pgfsetstrokecolor{currentstroke}%
\pgfsetstrokeopacity{0.809922}%
\pgfsetdash{}{0pt}%
\pgfpathmoveto{\pgfqpoint{3.033755in}{2.673330in}}%
\pgfpathcurveto{\pgfqpoint{3.041992in}{2.673330in}}{\pgfqpoint{3.049892in}{2.676602in}}{\pgfqpoint{3.055716in}{2.682426in}}%
\pgfpathcurveto{\pgfqpoint{3.061539in}{2.688250in}}{\pgfqpoint{3.064812in}{2.696150in}}{\pgfqpoint{3.064812in}{2.704386in}}%
\pgfpathcurveto{\pgfqpoint{3.064812in}{2.712623in}}{\pgfqpoint{3.061539in}{2.720523in}}{\pgfqpoint{3.055716in}{2.726347in}}%
\pgfpathcurveto{\pgfqpoint{3.049892in}{2.732170in}}{\pgfqpoint{3.041992in}{2.735443in}}{\pgfqpoint{3.033755in}{2.735443in}}%
\pgfpathcurveto{\pgfqpoint{3.025519in}{2.735443in}}{\pgfqpoint{3.017619in}{2.732170in}}{\pgfqpoint{3.011795in}{2.726347in}}%
\pgfpathcurveto{\pgfqpoint{3.005971in}{2.720523in}}{\pgfqpoint{3.002699in}{2.712623in}}{\pgfqpoint{3.002699in}{2.704386in}}%
\pgfpathcurveto{\pgfqpoint{3.002699in}{2.696150in}}{\pgfqpoint{3.005971in}{2.688250in}}{\pgfqpoint{3.011795in}{2.682426in}}%
\pgfpathcurveto{\pgfqpoint{3.017619in}{2.676602in}}{\pgfqpoint{3.025519in}{2.673330in}}{\pgfqpoint{3.033755in}{2.673330in}}%
\pgfpathclose%
\pgfusepath{stroke,fill}%
\end{pgfscope}%
\begin{pgfscope}%
\pgfpathrectangle{\pgfqpoint{0.100000in}{0.212622in}}{\pgfqpoint{3.696000in}{3.696000in}}%
\pgfusepath{clip}%
\pgfsetbuttcap%
\pgfsetroundjoin%
\definecolor{currentfill}{rgb}{0.121569,0.466667,0.705882}%
\pgfsetfillcolor{currentfill}%
\pgfsetfillopacity{0.812995}%
\pgfsetlinewidth{1.003750pt}%
\definecolor{currentstroke}{rgb}{0.121569,0.466667,0.705882}%
\pgfsetstrokecolor{currentstroke}%
\pgfsetstrokeopacity{0.812995}%
\pgfsetdash{}{0pt}%
\pgfpathmoveto{\pgfqpoint{1.281712in}{1.873513in}}%
\pgfpathcurveto{\pgfqpoint{1.289948in}{1.873513in}}{\pgfqpoint{1.297848in}{1.876785in}}{\pgfqpoint{1.303672in}{1.882609in}}%
\pgfpathcurveto{\pgfqpoint{1.309496in}{1.888433in}}{\pgfqpoint{1.312768in}{1.896333in}}{\pgfqpoint{1.312768in}{1.904569in}}%
\pgfpathcurveto{\pgfqpoint{1.312768in}{1.912806in}}{\pgfqpoint{1.309496in}{1.920706in}}{\pgfqpoint{1.303672in}{1.926530in}}%
\pgfpathcurveto{\pgfqpoint{1.297848in}{1.932354in}}{\pgfqpoint{1.289948in}{1.935626in}}{\pgfqpoint{1.281712in}{1.935626in}}%
\pgfpathcurveto{\pgfqpoint{1.273476in}{1.935626in}}{\pgfqpoint{1.265576in}{1.932354in}}{\pgfqpoint{1.259752in}{1.926530in}}%
\pgfpathcurveto{\pgfqpoint{1.253928in}{1.920706in}}{\pgfqpoint{1.250655in}{1.912806in}}{\pgfqpoint{1.250655in}{1.904569in}}%
\pgfpathcurveto{\pgfqpoint{1.250655in}{1.896333in}}{\pgfqpoint{1.253928in}{1.888433in}}{\pgfqpoint{1.259752in}{1.882609in}}%
\pgfpathcurveto{\pgfqpoint{1.265576in}{1.876785in}}{\pgfqpoint{1.273476in}{1.873513in}}{\pgfqpoint{1.281712in}{1.873513in}}%
\pgfpathclose%
\pgfusepath{stroke,fill}%
\end{pgfscope}%
\begin{pgfscope}%
\pgfpathrectangle{\pgfqpoint{0.100000in}{0.212622in}}{\pgfqpoint{3.696000in}{3.696000in}}%
\pgfusepath{clip}%
\pgfsetbuttcap%
\pgfsetroundjoin%
\definecolor{currentfill}{rgb}{0.121569,0.466667,0.705882}%
\pgfsetfillcolor{currentfill}%
\pgfsetfillopacity{0.816774}%
\pgfsetlinewidth{1.003750pt}%
\definecolor{currentstroke}{rgb}{0.121569,0.466667,0.705882}%
\pgfsetstrokecolor{currentstroke}%
\pgfsetstrokeopacity{0.816774}%
\pgfsetdash{}{0pt}%
\pgfpathmoveto{\pgfqpoint{1.276387in}{1.867354in}}%
\pgfpathcurveto{\pgfqpoint{1.284623in}{1.867354in}}{\pgfqpoint{1.292523in}{1.870626in}}{\pgfqpoint{1.298347in}{1.876450in}}%
\pgfpathcurveto{\pgfqpoint{1.304171in}{1.882274in}}{\pgfqpoint{1.307443in}{1.890174in}}{\pgfqpoint{1.307443in}{1.898410in}}%
\pgfpathcurveto{\pgfqpoint{1.307443in}{1.906647in}}{\pgfqpoint{1.304171in}{1.914547in}}{\pgfqpoint{1.298347in}{1.920371in}}%
\pgfpathcurveto{\pgfqpoint{1.292523in}{1.926195in}}{\pgfqpoint{1.284623in}{1.929467in}}{\pgfqpoint{1.276387in}{1.929467in}}%
\pgfpathcurveto{\pgfqpoint{1.268150in}{1.929467in}}{\pgfqpoint{1.260250in}{1.926195in}}{\pgfqpoint{1.254426in}{1.920371in}}%
\pgfpathcurveto{\pgfqpoint{1.248603in}{1.914547in}}{\pgfqpoint{1.245330in}{1.906647in}}{\pgfqpoint{1.245330in}{1.898410in}}%
\pgfpathcurveto{\pgfqpoint{1.245330in}{1.890174in}}{\pgfqpoint{1.248603in}{1.882274in}}{\pgfqpoint{1.254426in}{1.876450in}}%
\pgfpathcurveto{\pgfqpoint{1.260250in}{1.870626in}}{\pgfqpoint{1.268150in}{1.867354in}}{\pgfqpoint{1.276387in}{1.867354in}}%
\pgfpathclose%
\pgfusepath{stroke,fill}%
\end{pgfscope}%
\begin{pgfscope}%
\pgfpathrectangle{\pgfqpoint{0.100000in}{0.212622in}}{\pgfqpoint{3.696000in}{3.696000in}}%
\pgfusepath{clip}%
\pgfsetbuttcap%
\pgfsetroundjoin%
\definecolor{currentfill}{rgb}{0.121569,0.466667,0.705882}%
\pgfsetfillcolor{currentfill}%
\pgfsetfillopacity{0.821908}%
\pgfsetlinewidth{1.003750pt}%
\definecolor{currentstroke}{rgb}{0.121569,0.466667,0.705882}%
\pgfsetstrokecolor{currentstroke}%
\pgfsetstrokeopacity{0.821908}%
\pgfsetdash{}{0pt}%
\pgfpathmoveto{\pgfqpoint{1.268642in}{1.858704in}}%
\pgfpathcurveto{\pgfqpoint{1.276878in}{1.858704in}}{\pgfqpoint{1.284778in}{1.861976in}}{\pgfqpoint{1.290602in}{1.867800in}}%
\pgfpathcurveto{\pgfqpoint{1.296426in}{1.873624in}}{\pgfqpoint{1.299698in}{1.881524in}}{\pgfqpoint{1.299698in}{1.889760in}}%
\pgfpathcurveto{\pgfqpoint{1.299698in}{1.897996in}}{\pgfqpoint{1.296426in}{1.905896in}}{\pgfqpoint{1.290602in}{1.911720in}}%
\pgfpathcurveto{\pgfqpoint{1.284778in}{1.917544in}}{\pgfqpoint{1.276878in}{1.920817in}}{\pgfqpoint{1.268642in}{1.920817in}}%
\pgfpathcurveto{\pgfqpoint{1.260406in}{1.920817in}}{\pgfqpoint{1.252506in}{1.917544in}}{\pgfqpoint{1.246682in}{1.911720in}}%
\pgfpathcurveto{\pgfqpoint{1.240858in}{1.905896in}}{\pgfqpoint{1.237585in}{1.897996in}}{\pgfqpoint{1.237585in}{1.889760in}}%
\pgfpathcurveto{\pgfqpoint{1.237585in}{1.881524in}}{\pgfqpoint{1.240858in}{1.873624in}}{\pgfqpoint{1.246682in}{1.867800in}}%
\pgfpathcurveto{\pgfqpoint{1.252506in}{1.861976in}}{\pgfqpoint{1.260406in}{1.858704in}}{\pgfqpoint{1.268642in}{1.858704in}}%
\pgfpathclose%
\pgfusepath{stroke,fill}%
\end{pgfscope}%
\begin{pgfscope}%
\pgfpathrectangle{\pgfqpoint{0.100000in}{0.212622in}}{\pgfqpoint{3.696000in}{3.696000in}}%
\pgfusepath{clip}%
\pgfsetbuttcap%
\pgfsetroundjoin%
\definecolor{currentfill}{rgb}{0.121569,0.466667,0.705882}%
\pgfsetfillcolor{currentfill}%
\pgfsetfillopacity{0.822750}%
\pgfsetlinewidth{1.003750pt}%
\definecolor{currentstroke}{rgb}{0.121569,0.466667,0.705882}%
\pgfsetstrokecolor{currentstroke}%
\pgfsetstrokeopacity{0.822750}%
\pgfsetdash{}{0pt}%
\pgfpathmoveto{\pgfqpoint{2.988738in}{2.636693in}}%
\pgfpathcurveto{\pgfqpoint{2.996975in}{2.636693in}}{\pgfqpoint{3.004875in}{2.639965in}}{\pgfqpoint{3.010699in}{2.645789in}}%
\pgfpathcurveto{\pgfqpoint{3.016523in}{2.651613in}}{\pgfqpoint{3.019795in}{2.659513in}}{\pgfqpoint{3.019795in}{2.667749in}}%
\pgfpathcurveto{\pgfqpoint{3.019795in}{2.675985in}}{\pgfqpoint{3.016523in}{2.683886in}}{\pgfqpoint{3.010699in}{2.689709in}}%
\pgfpathcurveto{\pgfqpoint{3.004875in}{2.695533in}}{\pgfqpoint{2.996975in}{2.698806in}}{\pgfqpoint{2.988738in}{2.698806in}}%
\pgfpathcurveto{\pgfqpoint{2.980502in}{2.698806in}}{\pgfqpoint{2.972602in}{2.695533in}}{\pgfqpoint{2.966778in}{2.689709in}}%
\pgfpathcurveto{\pgfqpoint{2.960954in}{2.683886in}}{\pgfqpoint{2.957682in}{2.675985in}}{\pgfqpoint{2.957682in}{2.667749in}}%
\pgfpathcurveto{\pgfqpoint{2.957682in}{2.659513in}}{\pgfqpoint{2.960954in}{2.651613in}}{\pgfqpoint{2.966778in}{2.645789in}}%
\pgfpathcurveto{\pgfqpoint{2.972602in}{2.639965in}}{\pgfqpoint{2.980502in}{2.636693in}}{\pgfqpoint{2.988738in}{2.636693in}}%
\pgfpathclose%
\pgfusepath{stroke,fill}%
\end{pgfscope}%
\begin{pgfscope}%
\pgfpathrectangle{\pgfqpoint{0.100000in}{0.212622in}}{\pgfqpoint{3.696000in}{3.696000in}}%
\pgfusepath{clip}%
\pgfsetbuttcap%
\pgfsetroundjoin%
\definecolor{currentfill}{rgb}{0.121569,0.466667,0.705882}%
\pgfsetfillcolor{currentfill}%
\pgfsetfillopacity{0.826018}%
\pgfsetlinewidth{1.003750pt}%
\definecolor{currentstroke}{rgb}{0.121569,0.466667,0.705882}%
\pgfsetstrokecolor{currentstroke}%
\pgfsetstrokeopacity{0.826018}%
\pgfsetdash{}{0pt}%
\pgfpathmoveto{\pgfqpoint{1.270357in}{1.869936in}}%
\pgfpathcurveto{\pgfqpoint{1.278593in}{1.869936in}}{\pgfqpoint{1.286493in}{1.873209in}}{\pgfqpoint{1.292317in}{1.879032in}}%
\pgfpathcurveto{\pgfqpoint{1.298141in}{1.884856in}}{\pgfqpoint{1.301414in}{1.892756in}}{\pgfqpoint{1.301414in}{1.900993in}}%
\pgfpathcurveto{\pgfqpoint{1.301414in}{1.909229in}}{\pgfqpoint{1.298141in}{1.917129in}}{\pgfqpoint{1.292317in}{1.922953in}}%
\pgfpathcurveto{\pgfqpoint{1.286493in}{1.928777in}}{\pgfqpoint{1.278593in}{1.932049in}}{\pgfqpoint{1.270357in}{1.932049in}}%
\pgfpathcurveto{\pgfqpoint{1.262121in}{1.932049in}}{\pgfqpoint{1.254221in}{1.928777in}}{\pgfqpoint{1.248397in}{1.922953in}}%
\pgfpathcurveto{\pgfqpoint{1.242573in}{1.917129in}}{\pgfqpoint{1.239301in}{1.909229in}}{\pgfqpoint{1.239301in}{1.900993in}}%
\pgfpathcurveto{\pgfqpoint{1.239301in}{1.892756in}}{\pgfqpoint{1.242573in}{1.884856in}}{\pgfqpoint{1.248397in}{1.879032in}}%
\pgfpathcurveto{\pgfqpoint{1.254221in}{1.873209in}}{\pgfqpoint{1.262121in}{1.869936in}}{\pgfqpoint{1.270357in}{1.869936in}}%
\pgfpathclose%
\pgfusepath{stroke,fill}%
\end{pgfscope}%
\begin{pgfscope}%
\pgfpathrectangle{\pgfqpoint{0.100000in}{0.212622in}}{\pgfqpoint{3.696000in}{3.696000in}}%
\pgfusepath{clip}%
\pgfsetbuttcap%
\pgfsetroundjoin%
\definecolor{currentfill}{rgb}{0.121569,0.466667,0.705882}%
\pgfsetfillcolor{currentfill}%
\pgfsetfillopacity{0.827985}%
\pgfsetlinewidth{1.003750pt}%
\definecolor{currentstroke}{rgb}{0.121569,0.466667,0.705882}%
\pgfsetstrokecolor{currentstroke}%
\pgfsetstrokeopacity{0.827985}%
\pgfsetdash{}{0pt}%
\pgfpathmoveto{\pgfqpoint{2.886731in}{2.612984in}}%
\pgfpathcurveto{\pgfqpoint{2.894968in}{2.612984in}}{\pgfqpoint{2.902868in}{2.616257in}}{\pgfqpoint{2.908692in}{2.622080in}}%
\pgfpathcurveto{\pgfqpoint{2.914516in}{2.627904in}}{\pgfqpoint{2.917788in}{2.635804in}}{\pgfqpoint{2.917788in}{2.644041in}}%
\pgfpathcurveto{\pgfqpoint{2.917788in}{2.652277in}}{\pgfqpoint{2.914516in}{2.660177in}}{\pgfqpoint{2.908692in}{2.666001in}}%
\pgfpathcurveto{\pgfqpoint{2.902868in}{2.671825in}}{\pgfqpoint{2.894968in}{2.675097in}}{\pgfqpoint{2.886731in}{2.675097in}}%
\pgfpathcurveto{\pgfqpoint{2.878495in}{2.675097in}}{\pgfqpoint{2.870595in}{2.671825in}}{\pgfqpoint{2.864771in}{2.666001in}}%
\pgfpathcurveto{\pgfqpoint{2.858947in}{2.660177in}}{\pgfqpoint{2.855675in}{2.652277in}}{\pgfqpoint{2.855675in}{2.644041in}}%
\pgfpathcurveto{\pgfqpoint{2.855675in}{2.635804in}}{\pgfqpoint{2.858947in}{2.627904in}}{\pgfqpoint{2.864771in}{2.622080in}}%
\pgfpathcurveto{\pgfqpoint{2.870595in}{2.616257in}}{\pgfqpoint{2.878495in}{2.612984in}}{\pgfqpoint{2.886731in}{2.612984in}}%
\pgfpathclose%
\pgfusepath{stroke,fill}%
\end{pgfscope}%
\begin{pgfscope}%
\pgfpathrectangle{\pgfqpoint{0.100000in}{0.212622in}}{\pgfqpoint{3.696000in}{3.696000in}}%
\pgfusepath{clip}%
\pgfsetbuttcap%
\pgfsetroundjoin%
\definecolor{currentfill}{rgb}{0.121569,0.466667,0.705882}%
\pgfsetfillcolor{currentfill}%
\pgfsetfillopacity{0.831487}%
\pgfsetlinewidth{1.003750pt}%
\definecolor{currentstroke}{rgb}{0.121569,0.466667,0.705882}%
\pgfsetstrokecolor{currentstroke}%
\pgfsetstrokeopacity{0.831487}%
\pgfsetdash{}{0pt}%
\pgfpathmoveto{\pgfqpoint{2.906230in}{2.595345in}}%
\pgfpathcurveto{\pgfqpoint{2.914467in}{2.595345in}}{\pgfqpoint{2.922367in}{2.598617in}}{\pgfqpoint{2.928191in}{2.604441in}}%
\pgfpathcurveto{\pgfqpoint{2.934015in}{2.610265in}}{\pgfqpoint{2.937287in}{2.618165in}}{\pgfqpoint{2.937287in}{2.626401in}}%
\pgfpathcurveto{\pgfqpoint{2.937287in}{2.634637in}}{\pgfqpoint{2.934015in}{2.642538in}}{\pgfqpoint{2.928191in}{2.648361in}}%
\pgfpathcurveto{\pgfqpoint{2.922367in}{2.654185in}}{\pgfqpoint{2.914467in}{2.657458in}}{\pgfqpoint{2.906230in}{2.657458in}}%
\pgfpathcurveto{\pgfqpoint{2.897994in}{2.657458in}}{\pgfqpoint{2.890094in}{2.654185in}}{\pgfqpoint{2.884270in}{2.648361in}}%
\pgfpathcurveto{\pgfqpoint{2.878446in}{2.642538in}}{\pgfqpoint{2.875174in}{2.634637in}}{\pgfqpoint{2.875174in}{2.626401in}}%
\pgfpathcurveto{\pgfqpoint{2.875174in}{2.618165in}}{\pgfqpoint{2.878446in}{2.610265in}}{\pgfqpoint{2.884270in}{2.604441in}}%
\pgfpathcurveto{\pgfqpoint{2.890094in}{2.598617in}}{\pgfqpoint{2.897994in}{2.595345in}}{\pgfqpoint{2.906230in}{2.595345in}}%
\pgfpathclose%
\pgfusepath{stroke,fill}%
\end{pgfscope}%
\begin{pgfscope}%
\pgfpathrectangle{\pgfqpoint{0.100000in}{0.212622in}}{\pgfqpoint{3.696000in}{3.696000in}}%
\pgfusepath{clip}%
\pgfsetbuttcap%
\pgfsetroundjoin%
\definecolor{currentfill}{rgb}{0.121569,0.466667,0.705882}%
\pgfsetfillcolor{currentfill}%
\pgfsetfillopacity{0.833970}%
\pgfsetlinewidth{1.003750pt}%
\definecolor{currentstroke}{rgb}{0.121569,0.466667,0.705882}%
\pgfsetstrokecolor{currentstroke}%
\pgfsetstrokeopacity{0.833970}%
\pgfsetdash{}{0pt}%
\pgfpathmoveto{\pgfqpoint{2.940048in}{2.617886in}}%
\pgfpathcurveto{\pgfqpoint{2.948284in}{2.617886in}}{\pgfqpoint{2.956184in}{2.621158in}}{\pgfqpoint{2.962008in}{2.626982in}}%
\pgfpathcurveto{\pgfqpoint{2.967832in}{2.632806in}}{\pgfqpoint{2.971105in}{2.640706in}}{\pgfqpoint{2.971105in}{2.648942in}}%
\pgfpathcurveto{\pgfqpoint{2.971105in}{2.657179in}}{\pgfqpoint{2.967832in}{2.665079in}}{\pgfqpoint{2.962008in}{2.670903in}}%
\pgfpathcurveto{\pgfqpoint{2.956184in}{2.676727in}}{\pgfqpoint{2.948284in}{2.679999in}}{\pgfqpoint{2.940048in}{2.679999in}}%
\pgfpathcurveto{\pgfqpoint{2.931812in}{2.679999in}}{\pgfqpoint{2.923912in}{2.676727in}}{\pgfqpoint{2.918088in}{2.670903in}}%
\pgfpathcurveto{\pgfqpoint{2.912264in}{2.665079in}}{\pgfqpoint{2.908992in}{2.657179in}}{\pgfqpoint{2.908992in}{2.648942in}}%
\pgfpathcurveto{\pgfqpoint{2.908992in}{2.640706in}}{\pgfqpoint{2.912264in}{2.632806in}}{\pgfqpoint{2.918088in}{2.626982in}}%
\pgfpathcurveto{\pgfqpoint{2.923912in}{2.621158in}}{\pgfqpoint{2.931812in}{2.617886in}}{\pgfqpoint{2.940048in}{2.617886in}}%
\pgfpathclose%
\pgfusepath{stroke,fill}%
\end{pgfscope}%
\begin{pgfscope}%
\pgfpathrectangle{\pgfqpoint{0.100000in}{0.212622in}}{\pgfqpoint{3.696000in}{3.696000in}}%
\pgfusepath{clip}%
\pgfsetbuttcap%
\pgfsetroundjoin%
\definecolor{currentfill}{rgb}{0.121569,0.466667,0.705882}%
\pgfsetfillcolor{currentfill}%
\pgfsetfillopacity{0.837503}%
\pgfsetlinewidth{1.003750pt}%
\definecolor{currentstroke}{rgb}{0.121569,0.466667,0.705882}%
\pgfsetstrokecolor{currentstroke}%
\pgfsetstrokeopacity{0.837503}%
\pgfsetdash{}{0pt}%
\pgfpathmoveto{\pgfqpoint{2.912012in}{2.603385in}}%
\pgfpathcurveto{\pgfqpoint{2.920248in}{2.603385in}}{\pgfqpoint{2.928148in}{2.606658in}}{\pgfqpoint{2.933972in}{2.612482in}}%
\pgfpathcurveto{\pgfqpoint{2.939796in}{2.618306in}}{\pgfqpoint{2.943068in}{2.626206in}}{\pgfqpoint{2.943068in}{2.634442in}}%
\pgfpathcurveto{\pgfqpoint{2.943068in}{2.642678in}}{\pgfqpoint{2.939796in}{2.650578in}}{\pgfqpoint{2.933972in}{2.656402in}}%
\pgfpathcurveto{\pgfqpoint{2.928148in}{2.662226in}}{\pgfqpoint{2.920248in}{2.665498in}}{\pgfqpoint{2.912012in}{2.665498in}}%
\pgfpathcurveto{\pgfqpoint{2.903775in}{2.665498in}}{\pgfqpoint{2.895875in}{2.662226in}}{\pgfqpoint{2.890051in}{2.656402in}}%
\pgfpathcurveto{\pgfqpoint{2.884228in}{2.650578in}}{\pgfqpoint{2.880955in}{2.642678in}}{\pgfqpoint{2.880955in}{2.634442in}}%
\pgfpathcurveto{\pgfqpoint{2.880955in}{2.626206in}}{\pgfqpoint{2.884228in}{2.618306in}}{\pgfqpoint{2.890051in}{2.612482in}}%
\pgfpathcurveto{\pgfqpoint{2.895875in}{2.606658in}}{\pgfqpoint{2.903775in}{2.603385in}}{\pgfqpoint{2.912012in}{2.603385in}}%
\pgfpathclose%
\pgfusepath{stroke,fill}%
\end{pgfscope}%
\begin{pgfscope}%
\pgfpathrectangle{\pgfqpoint{0.100000in}{0.212622in}}{\pgfqpoint{3.696000in}{3.696000in}}%
\pgfusepath{clip}%
\pgfsetbuttcap%
\pgfsetroundjoin%
\definecolor{currentfill}{rgb}{0.121569,0.466667,0.705882}%
\pgfsetfillcolor{currentfill}%
\pgfsetfillopacity{0.839833}%
\pgfsetlinewidth{1.003750pt}%
\definecolor{currentstroke}{rgb}{0.121569,0.466667,0.705882}%
\pgfsetstrokecolor{currentstroke}%
\pgfsetstrokeopacity{0.839833}%
\pgfsetdash{}{0pt}%
\pgfpathmoveto{\pgfqpoint{1.246400in}{1.847644in}}%
\pgfpathcurveto{\pgfqpoint{1.254636in}{1.847644in}}{\pgfqpoint{1.262536in}{1.850916in}}{\pgfqpoint{1.268360in}{1.856740in}}%
\pgfpathcurveto{\pgfqpoint{1.274184in}{1.862564in}}{\pgfqpoint{1.277457in}{1.870464in}}{\pgfqpoint{1.277457in}{1.878701in}}%
\pgfpathcurveto{\pgfqpoint{1.277457in}{1.886937in}}{\pgfqpoint{1.274184in}{1.894837in}}{\pgfqpoint{1.268360in}{1.900661in}}%
\pgfpathcurveto{\pgfqpoint{1.262536in}{1.906485in}}{\pgfqpoint{1.254636in}{1.909757in}}{\pgfqpoint{1.246400in}{1.909757in}}%
\pgfpathcurveto{\pgfqpoint{1.238164in}{1.909757in}}{\pgfqpoint{1.230264in}{1.906485in}}{\pgfqpoint{1.224440in}{1.900661in}}%
\pgfpathcurveto{\pgfqpoint{1.218616in}{1.894837in}}{\pgfqpoint{1.215344in}{1.886937in}}{\pgfqpoint{1.215344in}{1.878701in}}%
\pgfpathcurveto{\pgfqpoint{1.215344in}{1.870464in}}{\pgfqpoint{1.218616in}{1.862564in}}{\pgfqpoint{1.224440in}{1.856740in}}%
\pgfpathcurveto{\pgfqpoint{1.230264in}{1.850916in}}{\pgfqpoint{1.238164in}{1.847644in}}{\pgfqpoint{1.246400in}{1.847644in}}%
\pgfpathclose%
\pgfusepath{stroke,fill}%
\end{pgfscope}%
\begin{pgfscope}%
\pgfpathrectangle{\pgfqpoint{0.100000in}{0.212622in}}{\pgfqpoint{3.696000in}{3.696000in}}%
\pgfusepath{clip}%
\pgfsetbuttcap%
\pgfsetroundjoin%
\definecolor{currentfill}{rgb}{0.121569,0.466667,0.705882}%
\pgfsetfillcolor{currentfill}%
\pgfsetfillopacity{0.840036}%
\pgfsetlinewidth{1.003750pt}%
\definecolor{currentstroke}{rgb}{0.121569,0.466667,0.705882}%
\pgfsetstrokecolor{currentstroke}%
\pgfsetstrokeopacity{0.840036}%
\pgfsetdash{}{0pt}%
\pgfpathmoveto{\pgfqpoint{1.257977in}{1.853875in}}%
\pgfpathcurveto{\pgfqpoint{1.266213in}{1.853875in}}{\pgfqpoint{1.274113in}{1.857147in}}{\pgfqpoint{1.279937in}{1.862971in}}%
\pgfpathcurveto{\pgfqpoint{1.285761in}{1.868795in}}{\pgfqpoint{1.289033in}{1.876695in}}{\pgfqpoint{1.289033in}{1.884932in}}%
\pgfpathcurveto{\pgfqpoint{1.289033in}{1.893168in}}{\pgfqpoint{1.285761in}{1.901068in}}{\pgfqpoint{1.279937in}{1.906892in}}%
\pgfpathcurveto{\pgfqpoint{1.274113in}{1.912716in}}{\pgfqpoint{1.266213in}{1.915988in}}{\pgfqpoint{1.257977in}{1.915988in}}%
\pgfpathcurveto{\pgfqpoint{1.249741in}{1.915988in}}{\pgfqpoint{1.241841in}{1.912716in}}{\pgfqpoint{1.236017in}{1.906892in}}%
\pgfpathcurveto{\pgfqpoint{1.230193in}{1.901068in}}{\pgfqpoint{1.226920in}{1.893168in}}{\pgfqpoint{1.226920in}{1.884932in}}%
\pgfpathcurveto{\pgfqpoint{1.226920in}{1.876695in}}{\pgfqpoint{1.230193in}{1.868795in}}{\pgfqpoint{1.236017in}{1.862971in}}%
\pgfpathcurveto{\pgfqpoint{1.241841in}{1.857147in}}{\pgfqpoint{1.249741in}{1.853875in}}{\pgfqpoint{1.257977in}{1.853875in}}%
\pgfpathclose%
\pgfusepath{stroke,fill}%
\end{pgfscope}%
\begin{pgfscope}%
\pgfpathrectangle{\pgfqpoint{0.100000in}{0.212622in}}{\pgfqpoint{3.696000in}{3.696000in}}%
\pgfusepath{clip}%
\pgfsetbuttcap%
\pgfsetroundjoin%
\definecolor{currentfill}{rgb}{0.121569,0.466667,0.705882}%
\pgfsetfillcolor{currentfill}%
\pgfsetfillopacity{0.840203}%
\pgfsetlinewidth{1.003750pt}%
\definecolor{currentstroke}{rgb}{0.121569,0.466667,0.705882}%
\pgfsetstrokecolor{currentstroke}%
\pgfsetstrokeopacity{0.840203}%
\pgfsetdash{}{0pt}%
\pgfpathmoveto{\pgfqpoint{3.082382in}{2.668001in}}%
\pgfpathcurveto{\pgfqpoint{3.090618in}{2.668001in}}{\pgfqpoint{3.098519in}{2.671274in}}{\pgfqpoint{3.104342in}{2.677098in}}%
\pgfpathcurveto{\pgfqpoint{3.110166in}{2.682922in}}{\pgfqpoint{3.113439in}{2.690822in}}{\pgfqpoint{3.113439in}{2.699058in}}%
\pgfpathcurveto{\pgfqpoint{3.113439in}{2.707294in}}{\pgfqpoint{3.110166in}{2.715194in}}{\pgfqpoint{3.104342in}{2.721018in}}%
\pgfpathcurveto{\pgfqpoint{3.098519in}{2.726842in}}{\pgfqpoint{3.090618in}{2.730114in}}{\pgfqpoint{3.082382in}{2.730114in}}%
\pgfpathcurveto{\pgfqpoint{3.074146in}{2.730114in}}{\pgfqpoint{3.066246in}{2.726842in}}{\pgfqpoint{3.060422in}{2.721018in}}%
\pgfpathcurveto{\pgfqpoint{3.054598in}{2.715194in}}{\pgfqpoint{3.051326in}{2.707294in}}{\pgfqpoint{3.051326in}{2.699058in}}%
\pgfpathcurveto{\pgfqpoint{3.051326in}{2.690822in}}{\pgfqpoint{3.054598in}{2.682922in}}{\pgfqpoint{3.060422in}{2.677098in}}%
\pgfpathcurveto{\pgfqpoint{3.066246in}{2.671274in}}{\pgfqpoint{3.074146in}{2.668001in}}{\pgfqpoint{3.082382in}{2.668001in}}%
\pgfpathclose%
\pgfusepath{stroke,fill}%
\end{pgfscope}%
\begin{pgfscope}%
\pgfpathrectangle{\pgfqpoint{0.100000in}{0.212622in}}{\pgfqpoint{3.696000in}{3.696000in}}%
\pgfusepath{clip}%
\pgfsetbuttcap%
\pgfsetroundjoin%
\definecolor{currentfill}{rgb}{0.121569,0.466667,0.705882}%
\pgfsetfillcolor{currentfill}%
\pgfsetfillopacity{0.840807}%
\pgfsetlinewidth{1.003750pt}%
\definecolor{currentstroke}{rgb}{0.121569,0.466667,0.705882}%
\pgfsetstrokecolor{currentstroke}%
\pgfsetstrokeopacity{0.840807}%
\pgfsetdash{}{0pt}%
\pgfpathmoveto{\pgfqpoint{1.249693in}{1.851017in}}%
\pgfpathcurveto{\pgfqpoint{1.257930in}{1.851017in}}{\pgfqpoint{1.265830in}{1.854289in}}{\pgfqpoint{1.271654in}{1.860113in}}%
\pgfpathcurveto{\pgfqpoint{1.277477in}{1.865937in}}{\pgfqpoint{1.280750in}{1.873837in}}{\pgfqpoint{1.280750in}{1.882073in}}%
\pgfpathcurveto{\pgfqpoint{1.280750in}{1.890310in}}{\pgfqpoint{1.277477in}{1.898210in}}{\pgfqpoint{1.271654in}{1.904034in}}%
\pgfpathcurveto{\pgfqpoint{1.265830in}{1.909858in}}{\pgfqpoint{1.257930in}{1.913130in}}{\pgfqpoint{1.249693in}{1.913130in}}%
\pgfpathcurveto{\pgfqpoint{1.241457in}{1.913130in}}{\pgfqpoint{1.233557in}{1.909858in}}{\pgfqpoint{1.227733in}{1.904034in}}%
\pgfpathcurveto{\pgfqpoint{1.221909in}{1.898210in}}{\pgfqpoint{1.218637in}{1.890310in}}{\pgfqpoint{1.218637in}{1.882073in}}%
\pgfpathcurveto{\pgfqpoint{1.218637in}{1.873837in}}{\pgfqpoint{1.221909in}{1.865937in}}{\pgfqpoint{1.227733in}{1.860113in}}%
\pgfpathcurveto{\pgfqpoint{1.233557in}{1.854289in}}{\pgfqpoint{1.241457in}{1.851017in}}{\pgfqpoint{1.249693in}{1.851017in}}%
\pgfpathclose%
\pgfusepath{stroke,fill}%
\end{pgfscope}%
\begin{pgfscope}%
\pgfpathrectangle{\pgfqpoint{0.100000in}{0.212622in}}{\pgfqpoint{3.696000in}{3.696000in}}%
\pgfusepath{clip}%
\pgfsetbuttcap%
\pgfsetroundjoin%
\definecolor{currentfill}{rgb}{0.121569,0.466667,0.705882}%
\pgfsetfillcolor{currentfill}%
\pgfsetfillopacity{0.841169}%
\pgfsetlinewidth{1.003750pt}%
\definecolor{currentstroke}{rgb}{0.121569,0.466667,0.705882}%
\pgfsetstrokecolor{currentstroke}%
\pgfsetstrokeopacity{0.841169}%
\pgfsetdash{}{0pt}%
\pgfpathmoveto{\pgfqpoint{1.250807in}{1.849379in}}%
\pgfpathcurveto{\pgfqpoint{1.259043in}{1.849379in}}{\pgfqpoint{1.266943in}{1.852651in}}{\pgfqpoint{1.272767in}{1.858475in}}%
\pgfpathcurveto{\pgfqpoint{1.278591in}{1.864299in}}{\pgfqpoint{1.281863in}{1.872199in}}{\pgfqpoint{1.281863in}{1.880435in}}%
\pgfpathcurveto{\pgfqpoint{1.281863in}{1.888672in}}{\pgfqpoint{1.278591in}{1.896572in}}{\pgfqpoint{1.272767in}{1.902396in}}%
\pgfpathcurveto{\pgfqpoint{1.266943in}{1.908220in}}{\pgfqpoint{1.259043in}{1.911492in}}{\pgfqpoint{1.250807in}{1.911492in}}%
\pgfpathcurveto{\pgfqpoint{1.242571in}{1.911492in}}{\pgfqpoint{1.234671in}{1.908220in}}{\pgfqpoint{1.228847in}{1.902396in}}%
\pgfpathcurveto{\pgfqpoint{1.223023in}{1.896572in}}{\pgfqpoint{1.219750in}{1.888672in}}{\pgfqpoint{1.219750in}{1.880435in}}%
\pgfpathcurveto{\pgfqpoint{1.219750in}{1.872199in}}{\pgfqpoint{1.223023in}{1.864299in}}{\pgfqpoint{1.228847in}{1.858475in}}%
\pgfpathcurveto{\pgfqpoint{1.234671in}{1.852651in}}{\pgfqpoint{1.242571in}{1.849379in}}{\pgfqpoint{1.250807in}{1.849379in}}%
\pgfpathclose%
\pgfusepath{stroke,fill}%
\end{pgfscope}%
\begin{pgfscope}%
\pgfpathrectangle{\pgfqpoint{0.100000in}{0.212622in}}{\pgfqpoint{3.696000in}{3.696000in}}%
\pgfusepath{clip}%
\pgfsetbuttcap%
\pgfsetroundjoin%
\definecolor{currentfill}{rgb}{0.121569,0.466667,0.705882}%
\pgfsetfillcolor{currentfill}%
\pgfsetfillopacity{0.841300}%
\pgfsetlinewidth{1.003750pt}%
\definecolor{currentstroke}{rgb}{0.121569,0.466667,0.705882}%
\pgfsetstrokecolor{currentstroke}%
\pgfsetstrokeopacity{0.841300}%
\pgfsetdash{}{0pt}%
\pgfpathmoveto{\pgfqpoint{3.050437in}{2.641975in}}%
\pgfpathcurveto{\pgfqpoint{3.058673in}{2.641975in}}{\pgfqpoint{3.066573in}{2.645248in}}{\pgfqpoint{3.072397in}{2.651072in}}%
\pgfpathcurveto{\pgfqpoint{3.078221in}{2.656895in}}{\pgfqpoint{3.081493in}{2.664796in}}{\pgfqpoint{3.081493in}{2.673032in}}%
\pgfpathcurveto{\pgfqpoint{3.081493in}{2.681268in}}{\pgfqpoint{3.078221in}{2.689168in}}{\pgfqpoint{3.072397in}{2.694992in}}%
\pgfpathcurveto{\pgfqpoint{3.066573in}{2.700816in}}{\pgfqpoint{3.058673in}{2.704088in}}{\pgfqpoint{3.050437in}{2.704088in}}%
\pgfpathcurveto{\pgfqpoint{3.042200in}{2.704088in}}{\pgfqpoint{3.034300in}{2.700816in}}{\pgfqpoint{3.028476in}{2.694992in}}%
\pgfpathcurveto{\pgfqpoint{3.022652in}{2.689168in}}{\pgfqpoint{3.019380in}{2.681268in}}{\pgfqpoint{3.019380in}{2.673032in}}%
\pgfpathcurveto{\pgfqpoint{3.019380in}{2.664796in}}{\pgfqpoint{3.022652in}{2.656895in}}{\pgfqpoint{3.028476in}{2.651072in}}%
\pgfpathcurveto{\pgfqpoint{3.034300in}{2.645248in}}{\pgfqpoint{3.042200in}{2.641975in}}{\pgfqpoint{3.050437in}{2.641975in}}%
\pgfpathclose%
\pgfusepath{stroke,fill}%
\end{pgfscope}%
\begin{pgfscope}%
\pgfpathrectangle{\pgfqpoint{0.100000in}{0.212622in}}{\pgfqpoint{3.696000in}{3.696000in}}%
\pgfusepath{clip}%
\pgfsetbuttcap%
\pgfsetroundjoin%
\definecolor{currentfill}{rgb}{0.121569,0.466667,0.705882}%
\pgfsetfillcolor{currentfill}%
\pgfsetfillopacity{0.841515}%
\pgfsetlinewidth{1.003750pt}%
\definecolor{currentstroke}{rgb}{0.121569,0.466667,0.705882}%
\pgfsetstrokecolor{currentstroke}%
\pgfsetstrokeopacity{0.841515}%
\pgfsetdash{}{0pt}%
\pgfpathmoveto{\pgfqpoint{2.908994in}{2.585486in}}%
\pgfpathcurveto{\pgfqpoint{2.917230in}{2.585486in}}{\pgfqpoint{2.925130in}{2.588759in}}{\pgfqpoint{2.930954in}{2.594583in}}%
\pgfpathcurveto{\pgfqpoint{2.936778in}{2.600407in}}{\pgfqpoint{2.940050in}{2.608307in}}{\pgfqpoint{2.940050in}{2.616543in}}%
\pgfpathcurveto{\pgfqpoint{2.940050in}{2.624779in}}{\pgfqpoint{2.936778in}{2.632679in}}{\pgfqpoint{2.930954in}{2.638503in}}%
\pgfpathcurveto{\pgfqpoint{2.925130in}{2.644327in}}{\pgfqpoint{2.917230in}{2.647599in}}{\pgfqpoint{2.908994in}{2.647599in}}%
\pgfpathcurveto{\pgfqpoint{2.900757in}{2.647599in}}{\pgfqpoint{2.892857in}{2.644327in}}{\pgfqpoint{2.887033in}{2.638503in}}%
\pgfpathcurveto{\pgfqpoint{2.881209in}{2.632679in}}{\pgfqpoint{2.877937in}{2.624779in}}{\pgfqpoint{2.877937in}{2.616543in}}%
\pgfpathcurveto{\pgfqpoint{2.877937in}{2.608307in}}{\pgfqpoint{2.881209in}{2.600407in}}{\pgfqpoint{2.887033in}{2.594583in}}%
\pgfpathcurveto{\pgfqpoint{2.892857in}{2.588759in}}{\pgfqpoint{2.900757in}{2.585486in}}{\pgfqpoint{2.908994in}{2.585486in}}%
\pgfpathclose%
\pgfusepath{stroke,fill}%
\end{pgfscope}%
\begin{pgfscope}%
\pgfpathrectangle{\pgfqpoint{0.100000in}{0.212622in}}{\pgfqpoint{3.696000in}{3.696000in}}%
\pgfusepath{clip}%
\pgfsetbuttcap%
\pgfsetroundjoin%
\definecolor{currentfill}{rgb}{0.121569,0.466667,0.705882}%
\pgfsetfillcolor{currentfill}%
\pgfsetfillopacity{0.842755}%
\pgfsetlinewidth{1.003750pt}%
\definecolor{currentstroke}{rgb}{0.121569,0.466667,0.705882}%
\pgfsetstrokecolor{currentstroke}%
\pgfsetstrokeopacity{0.842755}%
\pgfsetdash{}{0pt}%
\pgfpathmoveto{\pgfqpoint{3.043462in}{2.635256in}}%
\pgfpathcurveto{\pgfqpoint{3.051698in}{2.635256in}}{\pgfqpoint{3.059598in}{2.638529in}}{\pgfqpoint{3.065422in}{2.644353in}}%
\pgfpathcurveto{\pgfqpoint{3.071246in}{2.650177in}}{\pgfqpoint{3.074518in}{2.658077in}}{\pgfqpoint{3.074518in}{2.666313in}}%
\pgfpathcurveto{\pgfqpoint{3.074518in}{2.674549in}}{\pgfqpoint{3.071246in}{2.682449in}}{\pgfqpoint{3.065422in}{2.688273in}}%
\pgfpathcurveto{\pgfqpoint{3.059598in}{2.694097in}}{\pgfqpoint{3.051698in}{2.697369in}}{\pgfqpoint{3.043462in}{2.697369in}}%
\pgfpathcurveto{\pgfqpoint{3.035225in}{2.697369in}}{\pgfqpoint{3.027325in}{2.694097in}}{\pgfqpoint{3.021501in}{2.688273in}}%
\pgfpathcurveto{\pgfqpoint{3.015678in}{2.682449in}}{\pgfqpoint{3.012405in}{2.674549in}}{\pgfqpoint{3.012405in}{2.666313in}}%
\pgfpathcurveto{\pgfqpoint{3.012405in}{2.658077in}}{\pgfqpoint{3.015678in}{2.650177in}}{\pgfqpoint{3.021501in}{2.644353in}}%
\pgfpathcurveto{\pgfqpoint{3.027325in}{2.638529in}}{\pgfqpoint{3.035225in}{2.635256in}}{\pgfqpoint{3.043462in}{2.635256in}}%
\pgfpathclose%
\pgfusepath{stroke,fill}%
\end{pgfscope}%
\begin{pgfscope}%
\pgfpathrectangle{\pgfqpoint{0.100000in}{0.212622in}}{\pgfqpoint{3.696000in}{3.696000in}}%
\pgfusepath{clip}%
\pgfsetbuttcap%
\pgfsetroundjoin%
\definecolor{currentfill}{rgb}{0.121569,0.466667,0.705882}%
\pgfsetfillcolor{currentfill}%
\pgfsetfillopacity{0.842872}%
\pgfsetlinewidth{1.003750pt}%
\definecolor{currentstroke}{rgb}{0.121569,0.466667,0.705882}%
\pgfsetstrokecolor{currentstroke}%
\pgfsetstrokeopacity{0.842872}%
\pgfsetdash{}{0pt}%
\pgfpathmoveto{\pgfqpoint{2.987513in}{2.605517in}}%
\pgfpathcurveto{\pgfqpoint{2.995749in}{2.605517in}}{\pgfqpoint{3.003649in}{2.608789in}}{\pgfqpoint{3.009473in}{2.614613in}}%
\pgfpathcurveto{\pgfqpoint{3.015297in}{2.620437in}}{\pgfqpoint{3.018569in}{2.628337in}}{\pgfqpoint{3.018569in}{2.636573in}}%
\pgfpathcurveto{\pgfqpoint{3.018569in}{2.644809in}}{\pgfqpoint{3.015297in}{2.652709in}}{\pgfqpoint{3.009473in}{2.658533in}}%
\pgfpathcurveto{\pgfqpoint{3.003649in}{2.664357in}}{\pgfqpoint{2.995749in}{2.667630in}}{\pgfqpoint{2.987513in}{2.667630in}}%
\pgfpathcurveto{\pgfqpoint{2.979276in}{2.667630in}}{\pgfqpoint{2.971376in}{2.664357in}}{\pgfqpoint{2.965552in}{2.658533in}}%
\pgfpathcurveto{\pgfqpoint{2.959728in}{2.652709in}}{\pgfqpoint{2.956456in}{2.644809in}}{\pgfqpoint{2.956456in}{2.636573in}}%
\pgfpathcurveto{\pgfqpoint{2.956456in}{2.628337in}}{\pgfqpoint{2.959728in}{2.620437in}}{\pgfqpoint{2.965552in}{2.614613in}}%
\pgfpathcurveto{\pgfqpoint{2.971376in}{2.608789in}}{\pgfqpoint{2.979276in}{2.605517in}}{\pgfqpoint{2.987513in}{2.605517in}}%
\pgfpathclose%
\pgfusepath{stroke,fill}%
\end{pgfscope}%
\begin{pgfscope}%
\pgfpathrectangle{\pgfqpoint{0.100000in}{0.212622in}}{\pgfqpoint{3.696000in}{3.696000in}}%
\pgfusepath{clip}%
\pgfsetbuttcap%
\pgfsetroundjoin%
\definecolor{currentfill}{rgb}{0.121569,0.466667,0.705882}%
\pgfsetfillcolor{currentfill}%
\pgfsetfillopacity{0.843093}%
\pgfsetlinewidth{1.003750pt}%
\definecolor{currentstroke}{rgb}{0.121569,0.466667,0.705882}%
\pgfsetstrokecolor{currentstroke}%
\pgfsetstrokeopacity{0.843093}%
\pgfsetdash{}{0pt}%
\pgfpathmoveto{\pgfqpoint{3.048439in}{2.639242in}}%
\pgfpathcurveto{\pgfqpoint{3.056675in}{2.639242in}}{\pgfqpoint{3.064575in}{2.642515in}}{\pgfqpoint{3.070399in}{2.648338in}}%
\pgfpathcurveto{\pgfqpoint{3.076223in}{2.654162in}}{\pgfqpoint{3.079496in}{2.662062in}}{\pgfqpoint{3.079496in}{2.670299in}}%
\pgfpathcurveto{\pgfqpoint{3.079496in}{2.678535in}}{\pgfqpoint{3.076223in}{2.686435in}}{\pgfqpoint{3.070399in}{2.692259in}}%
\pgfpathcurveto{\pgfqpoint{3.064575in}{2.698083in}}{\pgfqpoint{3.056675in}{2.701355in}}{\pgfqpoint{3.048439in}{2.701355in}}%
\pgfpathcurveto{\pgfqpoint{3.040203in}{2.701355in}}{\pgfqpoint{3.032303in}{2.698083in}}{\pgfqpoint{3.026479in}{2.692259in}}%
\pgfpathcurveto{\pgfqpoint{3.020655in}{2.686435in}}{\pgfqpoint{3.017383in}{2.678535in}}{\pgfqpoint{3.017383in}{2.670299in}}%
\pgfpathcurveto{\pgfqpoint{3.017383in}{2.662062in}}{\pgfqpoint{3.020655in}{2.654162in}}{\pgfqpoint{3.026479in}{2.648338in}}%
\pgfpathcurveto{\pgfqpoint{3.032303in}{2.642515in}}{\pgfqpoint{3.040203in}{2.639242in}}{\pgfqpoint{3.048439in}{2.639242in}}%
\pgfpathclose%
\pgfusepath{stroke,fill}%
\end{pgfscope}%
\begin{pgfscope}%
\pgfpathrectangle{\pgfqpoint{0.100000in}{0.212622in}}{\pgfqpoint{3.696000in}{3.696000in}}%
\pgfusepath{clip}%
\pgfsetbuttcap%
\pgfsetroundjoin%
\definecolor{currentfill}{rgb}{0.121569,0.466667,0.705882}%
\pgfsetfillcolor{currentfill}%
\pgfsetfillopacity{0.843193}%
\pgfsetlinewidth{1.003750pt}%
\definecolor{currentstroke}{rgb}{0.121569,0.466667,0.705882}%
\pgfsetstrokecolor{currentstroke}%
\pgfsetstrokeopacity{0.843193}%
\pgfsetdash{}{0pt}%
\pgfpathmoveto{\pgfqpoint{2.930693in}{2.592384in}}%
\pgfpathcurveto{\pgfqpoint{2.938929in}{2.592384in}}{\pgfqpoint{2.946830in}{2.595657in}}{\pgfqpoint{2.952653in}{2.601481in}}%
\pgfpathcurveto{\pgfqpoint{2.958477in}{2.607304in}}{\pgfqpoint{2.961750in}{2.615205in}}{\pgfqpoint{2.961750in}{2.623441in}}%
\pgfpathcurveto{\pgfqpoint{2.961750in}{2.631677in}}{\pgfqpoint{2.958477in}{2.639577in}}{\pgfqpoint{2.952653in}{2.645401in}}%
\pgfpathcurveto{\pgfqpoint{2.946830in}{2.651225in}}{\pgfqpoint{2.938929in}{2.654497in}}{\pgfqpoint{2.930693in}{2.654497in}}%
\pgfpathcurveto{\pgfqpoint{2.922457in}{2.654497in}}{\pgfqpoint{2.914557in}{2.651225in}}{\pgfqpoint{2.908733in}{2.645401in}}%
\pgfpathcurveto{\pgfqpoint{2.902909in}{2.639577in}}{\pgfqpoint{2.899637in}{2.631677in}}{\pgfqpoint{2.899637in}{2.623441in}}%
\pgfpathcurveto{\pgfqpoint{2.899637in}{2.615205in}}{\pgfqpoint{2.902909in}{2.607304in}}{\pgfqpoint{2.908733in}{2.601481in}}%
\pgfpathcurveto{\pgfqpoint{2.914557in}{2.595657in}}{\pgfqpoint{2.922457in}{2.592384in}}{\pgfqpoint{2.930693in}{2.592384in}}%
\pgfpathclose%
\pgfusepath{stroke,fill}%
\end{pgfscope}%
\begin{pgfscope}%
\pgfpathrectangle{\pgfqpoint{0.100000in}{0.212622in}}{\pgfqpoint{3.696000in}{3.696000in}}%
\pgfusepath{clip}%
\pgfsetbuttcap%
\pgfsetroundjoin%
\definecolor{currentfill}{rgb}{0.121569,0.466667,0.705882}%
\pgfsetfillcolor{currentfill}%
\pgfsetfillopacity{0.844271}%
\pgfsetlinewidth{1.003750pt}%
\definecolor{currentstroke}{rgb}{0.121569,0.466667,0.705882}%
\pgfsetstrokecolor{currentstroke}%
\pgfsetstrokeopacity{0.844271}%
\pgfsetdash{}{0pt}%
\pgfpathmoveto{\pgfqpoint{2.884962in}{2.570209in}}%
\pgfpathcurveto{\pgfqpoint{2.893198in}{2.570209in}}{\pgfqpoint{2.901098in}{2.573482in}}{\pgfqpoint{2.906922in}{2.579306in}}%
\pgfpathcurveto{\pgfqpoint{2.912746in}{2.585130in}}{\pgfqpoint{2.916018in}{2.593030in}}{\pgfqpoint{2.916018in}{2.601266in}}%
\pgfpathcurveto{\pgfqpoint{2.916018in}{2.609502in}}{\pgfqpoint{2.912746in}{2.617402in}}{\pgfqpoint{2.906922in}{2.623226in}}%
\pgfpathcurveto{\pgfqpoint{2.901098in}{2.629050in}}{\pgfqpoint{2.893198in}{2.632322in}}{\pgfqpoint{2.884962in}{2.632322in}}%
\pgfpathcurveto{\pgfqpoint{2.876726in}{2.632322in}}{\pgfqpoint{2.868826in}{2.629050in}}{\pgfqpoint{2.863002in}{2.623226in}}%
\pgfpathcurveto{\pgfqpoint{2.857178in}{2.617402in}}{\pgfqpoint{2.853905in}{2.609502in}}{\pgfqpoint{2.853905in}{2.601266in}}%
\pgfpathcurveto{\pgfqpoint{2.853905in}{2.593030in}}{\pgfqpoint{2.857178in}{2.585130in}}{\pgfqpoint{2.863002in}{2.579306in}}%
\pgfpathcurveto{\pgfqpoint{2.868826in}{2.573482in}}{\pgfqpoint{2.876726in}{2.570209in}}{\pgfqpoint{2.884962in}{2.570209in}}%
\pgfpathclose%
\pgfusepath{stroke,fill}%
\end{pgfscope}%
\begin{pgfscope}%
\pgfpathrectangle{\pgfqpoint{0.100000in}{0.212622in}}{\pgfqpoint{3.696000in}{3.696000in}}%
\pgfusepath{clip}%
\pgfsetbuttcap%
\pgfsetroundjoin%
\definecolor{currentfill}{rgb}{0.121569,0.466667,0.705882}%
\pgfsetfillcolor{currentfill}%
\pgfsetfillopacity{0.844435}%
\pgfsetlinewidth{1.003750pt}%
\definecolor{currentstroke}{rgb}{0.121569,0.466667,0.705882}%
\pgfsetstrokecolor{currentstroke}%
\pgfsetstrokeopacity{0.844435}%
\pgfsetdash{}{0pt}%
\pgfpathmoveto{\pgfqpoint{3.034310in}{2.630162in}}%
\pgfpathcurveto{\pgfqpoint{3.042547in}{2.630162in}}{\pgfqpoint{3.050447in}{2.633434in}}{\pgfqpoint{3.056271in}{2.639258in}}%
\pgfpathcurveto{\pgfqpoint{3.062095in}{2.645082in}}{\pgfqpoint{3.065367in}{2.652982in}}{\pgfqpoint{3.065367in}{2.661218in}}%
\pgfpathcurveto{\pgfqpoint{3.065367in}{2.669455in}}{\pgfqpoint{3.062095in}{2.677355in}}{\pgfqpoint{3.056271in}{2.683179in}}%
\pgfpathcurveto{\pgfqpoint{3.050447in}{2.689003in}}{\pgfqpoint{3.042547in}{2.692275in}}{\pgfqpoint{3.034310in}{2.692275in}}%
\pgfpathcurveto{\pgfqpoint{3.026074in}{2.692275in}}{\pgfqpoint{3.018174in}{2.689003in}}{\pgfqpoint{3.012350in}{2.683179in}}%
\pgfpathcurveto{\pgfqpoint{3.006526in}{2.677355in}}{\pgfqpoint{3.003254in}{2.669455in}}{\pgfqpoint{3.003254in}{2.661218in}}%
\pgfpathcurveto{\pgfqpoint{3.003254in}{2.652982in}}{\pgfqpoint{3.006526in}{2.645082in}}{\pgfqpoint{3.012350in}{2.639258in}}%
\pgfpathcurveto{\pgfqpoint{3.018174in}{2.633434in}}{\pgfqpoint{3.026074in}{2.630162in}}{\pgfqpoint{3.034310in}{2.630162in}}%
\pgfpathclose%
\pgfusepath{stroke,fill}%
\end{pgfscope}%
\begin{pgfscope}%
\pgfpathrectangle{\pgfqpoint{0.100000in}{0.212622in}}{\pgfqpoint{3.696000in}{3.696000in}}%
\pgfusepath{clip}%
\pgfsetbuttcap%
\pgfsetroundjoin%
\definecolor{currentfill}{rgb}{0.121569,0.466667,0.705882}%
\pgfsetfillcolor{currentfill}%
\pgfsetfillopacity{0.845902}%
\pgfsetlinewidth{1.003750pt}%
\definecolor{currentstroke}{rgb}{0.121569,0.466667,0.705882}%
\pgfsetstrokecolor{currentstroke}%
\pgfsetstrokeopacity{0.845902}%
\pgfsetdash{}{0pt}%
\pgfpathmoveto{\pgfqpoint{3.022386in}{2.628824in}}%
\pgfpathcurveto{\pgfqpoint{3.030622in}{2.628824in}}{\pgfqpoint{3.038522in}{2.632097in}}{\pgfqpoint{3.044346in}{2.637921in}}%
\pgfpathcurveto{\pgfqpoint{3.050170in}{2.643744in}}{\pgfqpoint{3.053443in}{2.651644in}}{\pgfqpoint{3.053443in}{2.659881in}}%
\pgfpathcurveto{\pgfqpoint{3.053443in}{2.668117in}}{\pgfqpoint{3.050170in}{2.676017in}}{\pgfqpoint{3.044346in}{2.681841in}}%
\pgfpathcurveto{\pgfqpoint{3.038522in}{2.687665in}}{\pgfqpoint{3.030622in}{2.690937in}}{\pgfqpoint{3.022386in}{2.690937in}}%
\pgfpathcurveto{\pgfqpoint{3.014150in}{2.690937in}}{\pgfqpoint{3.006250in}{2.687665in}}{\pgfqpoint{3.000426in}{2.681841in}}%
\pgfpathcurveto{\pgfqpoint{2.994602in}{2.676017in}}{\pgfqpoint{2.991330in}{2.668117in}}{\pgfqpoint{2.991330in}{2.659881in}}%
\pgfpathcurveto{\pgfqpoint{2.991330in}{2.651644in}}{\pgfqpoint{2.994602in}{2.643744in}}{\pgfqpoint{3.000426in}{2.637921in}}%
\pgfpathcurveto{\pgfqpoint{3.006250in}{2.632097in}}{\pgfqpoint{3.014150in}{2.628824in}}{\pgfqpoint{3.022386in}{2.628824in}}%
\pgfpathclose%
\pgfusepath{stroke,fill}%
\end{pgfscope}%
\begin{pgfscope}%
\pgfpathrectangle{\pgfqpoint{0.100000in}{0.212622in}}{\pgfqpoint{3.696000in}{3.696000in}}%
\pgfusepath{clip}%
\pgfsetbuttcap%
\pgfsetroundjoin%
\definecolor{currentfill}{rgb}{0.121569,0.466667,0.705882}%
\pgfsetfillcolor{currentfill}%
\pgfsetfillopacity{0.845964}%
\pgfsetlinewidth{1.003750pt}%
\definecolor{currentstroke}{rgb}{0.121569,0.466667,0.705882}%
\pgfsetstrokecolor{currentstroke}%
\pgfsetstrokeopacity{0.845964}%
\pgfsetdash{}{0pt}%
\pgfpathmoveto{\pgfqpoint{3.052395in}{2.640468in}}%
\pgfpathcurveto{\pgfqpoint{3.060631in}{2.640468in}}{\pgfqpoint{3.068531in}{2.643741in}}{\pgfqpoint{3.074355in}{2.649565in}}%
\pgfpathcurveto{\pgfqpoint{3.080179in}{2.655388in}}{\pgfqpoint{3.083451in}{2.663288in}}{\pgfqpoint{3.083451in}{2.671525in}}%
\pgfpathcurveto{\pgfqpoint{3.083451in}{2.679761in}}{\pgfqpoint{3.080179in}{2.687661in}}{\pgfqpoint{3.074355in}{2.693485in}}%
\pgfpathcurveto{\pgfqpoint{3.068531in}{2.699309in}}{\pgfqpoint{3.060631in}{2.702581in}}{\pgfqpoint{3.052395in}{2.702581in}}%
\pgfpathcurveto{\pgfqpoint{3.044158in}{2.702581in}}{\pgfqpoint{3.036258in}{2.699309in}}{\pgfqpoint{3.030434in}{2.693485in}}%
\pgfpathcurveto{\pgfqpoint{3.024610in}{2.687661in}}{\pgfqpoint{3.021338in}{2.679761in}}{\pgfqpoint{3.021338in}{2.671525in}}%
\pgfpathcurveto{\pgfqpoint{3.021338in}{2.663288in}}{\pgfqpoint{3.024610in}{2.655388in}}{\pgfqpoint{3.030434in}{2.649565in}}%
\pgfpathcurveto{\pgfqpoint{3.036258in}{2.643741in}}{\pgfqpoint{3.044158in}{2.640468in}}{\pgfqpoint{3.052395in}{2.640468in}}%
\pgfpathclose%
\pgfusepath{stroke,fill}%
\end{pgfscope}%
\begin{pgfscope}%
\pgfpathrectangle{\pgfqpoint{0.100000in}{0.212622in}}{\pgfqpoint{3.696000in}{3.696000in}}%
\pgfusepath{clip}%
\pgfsetbuttcap%
\pgfsetroundjoin%
\definecolor{currentfill}{rgb}{0.121569,0.466667,0.705882}%
\pgfsetfillcolor{currentfill}%
\pgfsetfillopacity{0.846058}%
\pgfsetlinewidth{1.003750pt}%
\definecolor{currentstroke}{rgb}{0.121569,0.466667,0.705882}%
\pgfsetstrokecolor{currentstroke}%
\pgfsetstrokeopacity{0.846058}%
\pgfsetdash{}{0pt}%
\pgfpathmoveto{\pgfqpoint{1.248600in}{1.838836in}}%
\pgfpathcurveto{\pgfqpoint{1.256837in}{1.838836in}}{\pgfqpoint{1.264737in}{1.842109in}}{\pgfqpoint{1.270561in}{1.847933in}}%
\pgfpathcurveto{\pgfqpoint{1.276385in}{1.853757in}}{\pgfqpoint{1.279657in}{1.861657in}}{\pgfqpoint{1.279657in}{1.869893in}}%
\pgfpathcurveto{\pgfqpoint{1.279657in}{1.878129in}}{\pgfqpoint{1.276385in}{1.886029in}}{\pgfqpoint{1.270561in}{1.891853in}}%
\pgfpathcurveto{\pgfqpoint{1.264737in}{1.897677in}}{\pgfqpoint{1.256837in}{1.900949in}}{\pgfqpoint{1.248600in}{1.900949in}}%
\pgfpathcurveto{\pgfqpoint{1.240364in}{1.900949in}}{\pgfqpoint{1.232464in}{1.897677in}}{\pgfqpoint{1.226640in}{1.891853in}}%
\pgfpathcurveto{\pgfqpoint{1.220816in}{1.886029in}}{\pgfqpoint{1.217544in}{1.878129in}}{\pgfqpoint{1.217544in}{1.869893in}}%
\pgfpathcurveto{\pgfqpoint{1.217544in}{1.861657in}}{\pgfqpoint{1.220816in}{1.853757in}}{\pgfqpoint{1.226640in}{1.847933in}}%
\pgfpathcurveto{\pgfqpoint{1.232464in}{1.842109in}}{\pgfqpoint{1.240364in}{1.838836in}}{\pgfqpoint{1.248600in}{1.838836in}}%
\pgfpathclose%
\pgfusepath{stroke,fill}%
\end{pgfscope}%
\begin{pgfscope}%
\pgfpathrectangle{\pgfqpoint{0.100000in}{0.212622in}}{\pgfqpoint{3.696000in}{3.696000in}}%
\pgfusepath{clip}%
\pgfsetbuttcap%
\pgfsetroundjoin%
\definecolor{currentfill}{rgb}{0.121569,0.466667,0.705882}%
\pgfsetfillcolor{currentfill}%
\pgfsetfillopacity{0.846229}%
\pgfsetlinewidth{1.003750pt}%
\definecolor{currentstroke}{rgb}{0.121569,0.466667,0.705882}%
\pgfsetstrokecolor{currentstroke}%
\pgfsetstrokeopacity{0.846229}%
\pgfsetdash{}{0pt}%
\pgfpathmoveto{\pgfqpoint{3.046276in}{2.637170in}}%
\pgfpathcurveto{\pgfqpoint{3.054512in}{2.637170in}}{\pgfqpoint{3.062412in}{2.640442in}}{\pgfqpoint{3.068236in}{2.646266in}}%
\pgfpathcurveto{\pgfqpoint{3.074060in}{2.652090in}}{\pgfqpoint{3.077332in}{2.659990in}}{\pgfqpoint{3.077332in}{2.668226in}}%
\pgfpathcurveto{\pgfqpoint{3.077332in}{2.676462in}}{\pgfqpoint{3.074060in}{2.684362in}}{\pgfqpoint{3.068236in}{2.690186in}}%
\pgfpathcurveto{\pgfqpoint{3.062412in}{2.696010in}}{\pgfqpoint{3.054512in}{2.699283in}}{\pgfqpoint{3.046276in}{2.699283in}}%
\pgfpathcurveto{\pgfqpoint{3.038040in}{2.699283in}}{\pgfqpoint{3.030140in}{2.696010in}}{\pgfqpoint{3.024316in}{2.690186in}}%
\pgfpathcurveto{\pgfqpoint{3.018492in}{2.684362in}}{\pgfqpoint{3.015219in}{2.676462in}}{\pgfqpoint{3.015219in}{2.668226in}}%
\pgfpathcurveto{\pgfqpoint{3.015219in}{2.659990in}}{\pgfqpoint{3.018492in}{2.652090in}}{\pgfqpoint{3.024316in}{2.646266in}}%
\pgfpathcurveto{\pgfqpoint{3.030140in}{2.640442in}}{\pgfqpoint{3.038040in}{2.637170in}}{\pgfqpoint{3.046276in}{2.637170in}}%
\pgfpathclose%
\pgfusepath{stroke,fill}%
\end{pgfscope}%
\begin{pgfscope}%
\pgfpathrectangle{\pgfqpoint{0.100000in}{0.212622in}}{\pgfqpoint{3.696000in}{3.696000in}}%
\pgfusepath{clip}%
\pgfsetbuttcap%
\pgfsetroundjoin%
\definecolor{currentfill}{rgb}{0.121569,0.466667,0.705882}%
\pgfsetfillcolor{currentfill}%
\pgfsetfillopacity{0.847917}%
\pgfsetlinewidth{1.003750pt}%
\definecolor{currentstroke}{rgb}{0.121569,0.466667,0.705882}%
\pgfsetstrokecolor{currentstroke}%
\pgfsetstrokeopacity{0.847917}%
\pgfsetdash{}{0pt}%
\pgfpathmoveto{\pgfqpoint{3.053147in}{2.633166in}}%
\pgfpathcurveto{\pgfqpoint{3.061383in}{2.633166in}}{\pgfqpoint{3.069283in}{2.636438in}}{\pgfqpoint{3.075107in}{2.642262in}}%
\pgfpathcurveto{\pgfqpoint{3.080931in}{2.648086in}}{\pgfqpoint{3.084204in}{2.655986in}}{\pgfqpoint{3.084204in}{2.664222in}}%
\pgfpathcurveto{\pgfqpoint{3.084204in}{2.672458in}}{\pgfqpoint{3.080931in}{2.680358in}}{\pgfqpoint{3.075107in}{2.686182in}}%
\pgfpathcurveto{\pgfqpoint{3.069283in}{2.692006in}}{\pgfqpoint{3.061383in}{2.695279in}}{\pgfqpoint{3.053147in}{2.695279in}}%
\pgfpathcurveto{\pgfqpoint{3.044911in}{2.695279in}}{\pgfqpoint{3.037011in}{2.692006in}}{\pgfqpoint{3.031187in}{2.686182in}}%
\pgfpathcurveto{\pgfqpoint{3.025363in}{2.680358in}}{\pgfqpoint{3.022091in}{2.672458in}}{\pgfqpoint{3.022091in}{2.664222in}}%
\pgfpathcurveto{\pgfqpoint{3.022091in}{2.655986in}}{\pgfqpoint{3.025363in}{2.648086in}}{\pgfqpoint{3.031187in}{2.642262in}}%
\pgfpathcurveto{\pgfqpoint{3.037011in}{2.636438in}}{\pgfqpoint{3.044911in}{2.633166in}}{\pgfqpoint{3.053147in}{2.633166in}}%
\pgfpathclose%
\pgfusepath{stroke,fill}%
\end{pgfscope}%
\begin{pgfscope}%
\pgfpathrectangle{\pgfqpoint{0.100000in}{0.212622in}}{\pgfqpoint{3.696000in}{3.696000in}}%
\pgfusepath{clip}%
\pgfsetbuttcap%
\pgfsetroundjoin%
\definecolor{currentfill}{rgb}{0.121569,0.466667,0.705882}%
\pgfsetfillcolor{currentfill}%
\pgfsetfillopacity{0.849842}%
\pgfsetlinewidth{1.003750pt}%
\definecolor{currentstroke}{rgb}{0.121569,0.466667,0.705882}%
\pgfsetstrokecolor{currentstroke}%
\pgfsetstrokeopacity{0.849842}%
\pgfsetdash{}{0pt}%
\pgfpathmoveto{\pgfqpoint{1.249468in}{1.841646in}}%
\pgfpathcurveto{\pgfqpoint{1.257704in}{1.841646in}}{\pgfqpoint{1.265604in}{1.844918in}}{\pgfqpoint{1.271428in}{1.850742in}}%
\pgfpathcurveto{\pgfqpoint{1.277252in}{1.856566in}}{\pgfqpoint{1.280524in}{1.864466in}}{\pgfqpoint{1.280524in}{1.872702in}}%
\pgfpathcurveto{\pgfqpoint{1.280524in}{1.880939in}}{\pgfqpoint{1.277252in}{1.888839in}}{\pgfqpoint{1.271428in}{1.894663in}}%
\pgfpathcurveto{\pgfqpoint{1.265604in}{1.900487in}}{\pgfqpoint{1.257704in}{1.903759in}}{\pgfqpoint{1.249468in}{1.903759in}}%
\pgfpathcurveto{\pgfqpoint{1.241231in}{1.903759in}}{\pgfqpoint{1.233331in}{1.900487in}}{\pgfqpoint{1.227507in}{1.894663in}}%
\pgfpathcurveto{\pgfqpoint{1.221683in}{1.888839in}}{\pgfqpoint{1.218411in}{1.880939in}}{\pgfqpoint{1.218411in}{1.872702in}}%
\pgfpathcurveto{\pgfqpoint{1.218411in}{1.864466in}}{\pgfqpoint{1.221683in}{1.856566in}}{\pgfqpoint{1.227507in}{1.850742in}}%
\pgfpathcurveto{\pgfqpoint{1.233331in}{1.844918in}}{\pgfqpoint{1.241231in}{1.841646in}}{\pgfqpoint{1.249468in}{1.841646in}}%
\pgfpathclose%
\pgfusepath{stroke,fill}%
\end{pgfscope}%
\begin{pgfscope}%
\pgfpathrectangle{\pgfqpoint{0.100000in}{0.212622in}}{\pgfqpoint{3.696000in}{3.696000in}}%
\pgfusepath{clip}%
\pgfsetbuttcap%
\pgfsetroundjoin%
\definecolor{currentfill}{rgb}{0.121569,0.466667,0.705882}%
\pgfsetfillcolor{currentfill}%
\pgfsetfillopacity{0.850824}%
\pgfsetlinewidth{1.003750pt}%
\definecolor{currentstroke}{rgb}{0.121569,0.466667,0.705882}%
\pgfsetstrokecolor{currentstroke}%
\pgfsetstrokeopacity{0.850824}%
\pgfsetdash{}{0pt}%
\pgfpathmoveto{\pgfqpoint{3.074137in}{2.650199in}}%
\pgfpathcurveto{\pgfqpoint{3.082373in}{2.650199in}}{\pgfqpoint{3.090273in}{2.653472in}}{\pgfqpoint{3.096097in}{2.659296in}}%
\pgfpathcurveto{\pgfqpoint{3.101921in}{2.665120in}}{\pgfqpoint{3.105193in}{2.673020in}}{\pgfqpoint{3.105193in}{2.681256in}}%
\pgfpathcurveto{\pgfqpoint{3.105193in}{2.689492in}}{\pgfqpoint{3.101921in}{2.697392in}}{\pgfqpoint{3.096097in}{2.703216in}}%
\pgfpathcurveto{\pgfqpoint{3.090273in}{2.709040in}}{\pgfqpoint{3.082373in}{2.712312in}}{\pgfqpoint{3.074137in}{2.712312in}}%
\pgfpathcurveto{\pgfqpoint{3.065901in}{2.712312in}}{\pgfqpoint{3.058001in}{2.709040in}}{\pgfqpoint{3.052177in}{2.703216in}}%
\pgfpathcurveto{\pgfqpoint{3.046353in}{2.697392in}}{\pgfqpoint{3.043080in}{2.689492in}}{\pgfqpoint{3.043080in}{2.681256in}}%
\pgfpathcurveto{\pgfqpoint{3.043080in}{2.673020in}}{\pgfqpoint{3.046353in}{2.665120in}}{\pgfqpoint{3.052177in}{2.659296in}}%
\pgfpathcurveto{\pgfqpoint{3.058001in}{2.653472in}}{\pgfqpoint{3.065901in}{2.650199in}}{\pgfqpoint{3.074137in}{2.650199in}}%
\pgfpathclose%
\pgfusepath{stroke,fill}%
\end{pgfscope}%
\begin{pgfscope}%
\pgfpathrectangle{\pgfqpoint{0.100000in}{0.212622in}}{\pgfqpoint{3.696000in}{3.696000in}}%
\pgfusepath{clip}%
\pgfsetbuttcap%
\pgfsetroundjoin%
\definecolor{currentfill}{rgb}{0.121569,0.466667,0.705882}%
\pgfsetfillcolor{currentfill}%
\pgfsetfillopacity{0.850921}%
\pgfsetlinewidth{1.003750pt}%
\definecolor{currentstroke}{rgb}{0.121569,0.466667,0.705882}%
\pgfsetstrokecolor{currentstroke}%
\pgfsetstrokeopacity{0.850921}%
\pgfsetdash{}{0pt}%
\pgfpathmoveto{\pgfqpoint{2.954582in}{2.564574in}}%
\pgfpathcurveto{\pgfqpoint{2.962818in}{2.564574in}}{\pgfqpoint{2.970718in}{2.567846in}}{\pgfqpoint{2.976542in}{2.573670in}}%
\pgfpathcurveto{\pgfqpoint{2.982366in}{2.579494in}}{\pgfqpoint{2.985638in}{2.587394in}}{\pgfqpoint{2.985638in}{2.595631in}}%
\pgfpathcurveto{\pgfqpoint{2.985638in}{2.603867in}}{\pgfqpoint{2.982366in}{2.611767in}}{\pgfqpoint{2.976542in}{2.617591in}}%
\pgfpathcurveto{\pgfqpoint{2.970718in}{2.623415in}}{\pgfqpoint{2.962818in}{2.626687in}}{\pgfqpoint{2.954582in}{2.626687in}}%
\pgfpathcurveto{\pgfqpoint{2.946345in}{2.626687in}}{\pgfqpoint{2.938445in}{2.623415in}}{\pgfqpoint{2.932621in}{2.617591in}}%
\pgfpathcurveto{\pgfqpoint{2.926797in}{2.611767in}}{\pgfqpoint{2.923525in}{2.603867in}}{\pgfqpoint{2.923525in}{2.595631in}}%
\pgfpathcurveto{\pgfqpoint{2.923525in}{2.587394in}}{\pgfqpoint{2.926797in}{2.579494in}}{\pgfqpoint{2.932621in}{2.573670in}}%
\pgfpathcurveto{\pgfqpoint{2.938445in}{2.567846in}}{\pgfqpoint{2.946345in}{2.564574in}}{\pgfqpoint{2.954582in}{2.564574in}}%
\pgfpathclose%
\pgfusepath{stroke,fill}%
\end{pgfscope}%
\begin{pgfscope}%
\pgfpathrectangle{\pgfqpoint{0.100000in}{0.212622in}}{\pgfqpoint{3.696000in}{3.696000in}}%
\pgfusepath{clip}%
\pgfsetbuttcap%
\pgfsetroundjoin%
\definecolor{currentfill}{rgb}{0.121569,0.466667,0.705882}%
\pgfsetfillcolor{currentfill}%
\pgfsetfillopacity{0.851979}%
\pgfsetlinewidth{1.003750pt}%
\definecolor{currentstroke}{rgb}{0.121569,0.466667,0.705882}%
\pgfsetstrokecolor{currentstroke}%
\pgfsetstrokeopacity{0.851979}%
\pgfsetdash{}{0pt}%
\pgfpathmoveto{\pgfqpoint{2.983232in}{2.603711in}}%
\pgfpathcurveto{\pgfqpoint{2.991468in}{2.603711in}}{\pgfqpoint{2.999368in}{2.606983in}}{\pgfqpoint{3.005192in}{2.612807in}}%
\pgfpathcurveto{\pgfqpoint{3.011016in}{2.618631in}}{\pgfqpoint{3.014288in}{2.626531in}}{\pgfqpoint{3.014288in}{2.634767in}}%
\pgfpathcurveto{\pgfqpoint{3.014288in}{2.643004in}}{\pgfqpoint{3.011016in}{2.650904in}}{\pgfqpoint{3.005192in}{2.656728in}}%
\pgfpathcurveto{\pgfqpoint{2.999368in}{2.662552in}}{\pgfqpoint{2.991468in}{2.665824in}}{\pgfqpoint{2.983232in}{2.665824in}}%
\pgfpathcurveto{\pgfqpoint{2.974995in}{2.665824in}}{\pgfqpoint{2.967095in}{2.662552in}}{\pgfqpoint{2.961271in}{2.656728in}}%
\pgfpathcurveto{\pgfqpoint{2.955448in}{2.650904in}}{\pgfqpoint{2.952175in}{2.643004in}}{\pgfqpoint{2.952175in}{2.634767in}}%
\pgfpathcurveto{\pgfqpoint{2.952175in}{2.626531in}}{\pgfqpoint{2.955448in}{2.618631in}}{\pgfqpoint{2.961271in}{2.612807in}}%
\pgfpathcurveto{\pgfqpoint{2.967095in}{2.606983in}}{\pgfqpoint{2.974995in}{2.603711in}}{\pgfqpoint{2.983232in}{2.603711in}}%
\pgfpathclose%
\pgfusepath{stroke,fill}%
\end{pgfscope}%
\begin{pgfscope}%
\pgfpathrectangle{\pgfqpoint{0.100000in}{0.212622in}}{\pgfqpoint{3.696000in}{3.696000in}}%
\pgfusepath{clip}%
\pgfsetbuttcap%
\pgfsetroundjoin%
\definecolor{currentfill}{rgb}{0.121569,0.466667,0.705882}%
\pgfsetfillcolor{currentfill}%
\pgfsetfillopacity{0.852669}%
\pgfsetlinewidth{1.003750pt}%
\definecolor{currentstroke}{rgb}{0.121569,0.466667,0.705882}%
\pgfsetstrokecolor{currentstroke}%
\pgfsetstrokeopacity{0.852669}%
\pgfsetdash{}{0pt}%
\pgfpathmoveto{\pgfqpoint{2.993156in}{2.614054in}}%
\pgfpathcurveto{\pgfqpoint{3.001392in}{2.614054in}}{\pgfqpoint{3.009292in}{2.617326in}}{\pgfqpoint{3.015116in}{2.623150in}}%
\pgfpathcurveto{\pgfqpoint{3.020940in}{2.628974in}}{\pgfqpoint{3.024212in}{2.636874in}}{\pgfqpoint{3.024212in}{2.645110in}}%
\pgfpathcurveto{\pgfqpoint{3.024212in}{2.653347in}}{\pgfqpoint{3.020940in}{2.661247in}}{\pgfqpoint{3.015116in}{2.667071in}}%
\pgfpathcurveto{\pgfqpoint{3.009292in}{2.672894in}}{\pgfqpoint{3.001392in}{2.676167in}}{\pgfqpoint{2.993156in}{2.676167in}}%
\pgfpathcurveto{\pgfqpoint{2.984919in}{2.676167in}}{\pgfqpoint{2.977019in}{2.672894in}}{\pgfqpoint{2.971195in}{2.667071in}}%
\pgfpathcurveto{\pgfqpoint{2.965372in}{2.661247in}}{\pgfqpoint{2.962099in}{2.653347in}}{\pgfqpoint{2.962099in}{2.645110in}}%
\pgfpathcurveto{\pgfqpoint{2.962099in}{2.636874in}}{\pgfqpoint{2.965372in}{2.628974in}}{\pgfqpoint{2.971195in}{2.623150in}}%
\pgfpathcurveto{\pgfqpoint{2.977019in}{2.617326in}}{\pgfqpoint{2.984919in}{2.614054in}}{\pgfqpoint{2.993156in}{2.614054in}}%
\pgfpathclose%
\pgfusepath{stroke,fill}%
\end{pgfscope}%
\begin{pgfscope}%
\pgfpathrectangle{\pgfqpoint{0.100000in}{0.212622in}}{\pgfqpoint{3.696000in}{3.696000in}}%
\pgfusepath{clip}%
\pgfsetbuttcap%
\pgfsetroundjoin%
\definecolor{currentfill}{rgb}{0.121569,0.466667,0.705882}%
\pgfsetfillcolor{currentfill}%
\pgfsetfillopacity{0.854484}%
\pgfsetlinewidth{1.003750pt}%
\definecolor{currentstroke}{rgb}{0.121569,0.466667,0.705882}%
\pgfsetstrokecolor{currentstroke}%
\pgfsetstrokeopacity{0.854484}%
\pgfsetdash{}{0pt}%
\pgfpathmoveto{\pgfqpoint{3.085212in}{2.651936in}}%
\pgfpathcurveto{\pgfqpoint{3.093448in}{2.651936in}}{\pgfqpoint{3.101348in}{2.655208in}}{\pgfqpoint{3.107172in}{2.661032in}}%
\pgfpathcurveto{\pgfqpoint{3.112996in}{2.666856in}}{\pgfqpoint{3.116268in}{2.674756in}}{\pgfqpoint{3.116268in}{2.682992in}}%
\pgfpathcurveto{\pgfqpoint{3.116268in}{2.691228in}}{\pgfqpoint{3.112996in}{2.699128in}}{\pgfqpoint{3.107172in}{2.704952in}}%
\pgfpathcurveto{\pgfqpoint{3.101348in}{2.710776in}}{\pgfqpoint{3.093448in}{2.714049in}}{\pgfqpoint{3.085212in}{2.714049in}}%
\pgfpathcurveto{\pgfqpoint{3.076975in}{2.714049in}}{\pgfqpoint{3.069075in}{2.710776in}}{\pgfqpoint{3.063251in}{2.704952in}}%
\pgfpathcurveto{\pgfqpoint{3.057427in}{2.699128in}}{\pgfqpoint{3.054155in}{2.691228in}}{\pgfqpoint{3.054155in}{2.682992in}}%
\pgfpathcurveto{\pgfqpoint{3.054155in}{2.674756in}}{\pgfqpoint{3.057427in}{2.666856in}}{\pgfqpoint{3.063251in}{2.661032in}}%
\pgfpathcurveto{\pgfqpoint{3.069075in}{2.655208in}}{\pgfqpoint{3.076975in}{2.651936in}}{\pgfqpoint{3.085212in}{2.651936in}}%
\pgfpathclose%
\pgfusepath{stroke,fill}%
\end{pgfscope}%
\begin{pgfscope}%
\pgfpathrectangle{\pgfqpoint{0.100000in}{0.212622in}}{\pgfqpoint{3.696000in}{3.696000in}}%
\pgfusepath{clip}%
\pgfsetbuttcap%
\pgfsetroundjoin%
\definecolor{currentfill}{rgb}{0.121569,0.466667,0.705882}%
\pgfsetfillcolor{currentfill}%
\pgfsetfillopacity{0.854587}%
\pgfsetlinewidth{1.003750pt}%
\definecolor{currentstroke}{rgb}{0.121569,0.466667,0.705882}%
\pgfsetstrokecolor{currentstroke}%
\pgfsetstrokeopacity{0.854587}%
\pgfsetdash{}{0pt}%
\pgfpathmoveto{\pgfqpoint{2.967948in}{2.583863in}}%
\pgfpathcurveto{\pgfqpoint{2.976184in}{2.583863in}}{\pgfqpoint{2.984084in}{2.587135in}}{\pgfqpoint{2.989908in}{2.592959in}}%
\pgfpathcurveto{\pgfqpoint{2.995732in}{2.598783in}}{\pgfqpoint{2.999005in}{2.606683in}}{\pgfqpoint{2.999005in}{2.614919in}}%
\pgfpathcurveto{\pgfqpoint{2.999005in}{2.623156in}}{\pgfqpoint{2.995732in}{2.631056in}}{\pgfqpoint{2.989908in}{2.636880in}}%
\pgfpathcurveto{\pgfqpoint{2.984084in}{2.642703in}}{\pgfqpoint{2.976184in}{2.645976in}}{\pgfqpoint{2.967948in}{2.645976in}}%
\pgfpathcurveto{\pgfqpoint{2.959712in}{2.645976in}}{\pgfqpoint{2.951812in}{2.642703in}}{\pgfqpoint{2.945988in}{2.636880in}}%
\pgfpathcurveto{\pgfqpoint{2.940164in}{2.631056in}}{\pgfqpoint{2.936892in}{2.623156in}}{\pgfqpoint{2.936892in}{2.614919in}}%
\pgfpathcurveto{\pgfqpoint{2.936892in}{2.606683in}}{\pgfqpoint{2.940164in}{2.598783in}}{\pgfqpoint{2.945988in}{2.592959in}}%
\pgfpathcurveto{\pgfqpoint{2.951812in}{2.587135in}}{\pgfqpoint{2.959712in}{2.583863in}}{\pgfqpoint{2.967948in}{2.583863in}}%
\pgfpathclose%
\pgfusepath{stroke,fill}%
\end{pgfscope}%
\begin{pgfscope}%
\pgfpathrectangle{\pgfqpoint{0.100000in}{0.212622in}}{\pgfqpoint{3.696000in}{3.696000in}}%
\pgfusepath{clip}%
\pgfsetbuttcap%
\pgfsetroundjoin%
\definecolor{currentfill}{rgb}{0.121569,0.466667,0.705882}%
\pgfsetfillcolor{currentfill}%
\pgfsetfillopacity{0.854878}%
\pgfsetlinewidth{1.003750pt}%
\definecolor{currentstroke}{rgb}{0.121569,0.466667,0.705882}%
\pgfsetstrokecolor{currentstroke}%
\pgfsetstrokeopacity{0.854878}%
\pgfsetdash{}{0pt}%
\pgfpathmoveto{\pgfqpoint{2.802634in}{2.532641in}}%
\pgfpathcurveto{\pgfqpoint{2.810870in}{2.532641in}}{\pgfqpoint{2.818770in}{2.535913in}}{\pgfqpoint{2.824594in}{2.541737in}}%
\pgfpathcurveto{\pgfqpoint{2.830418in}{2.547561in}}{\pgfqpoint{2.833691in}{2.555461in}}{\pgfqpoint{2.833691in}{2.563697in}}%
\pgfpathcurveto{\pgfqpoint{2.833691in}{2.571933in}}{\pgfqpoint{2.830418in}{2.579834in}}{\pgfqpoint{2.824594in}{2.585657in}}%
\pgfpathcurveto{\pgfqpoint{2.818770in}{2.591481in}}{\pgfqpoint{2.810870in}{2.594754in}}{\pgfqpoint{2.802634in}{2.594754in}}%
\pgfpathcurveto{\pgfqpoint{2.794398in}{2.594754in}}{\pgfqpoint{2.786498in}{2.591481in}}{\pgfqpoint{2.780674in}{2.585657in}}%
\pgfpathcurveto{\pgfqpoint{2.774850in}{2.579834in}}{\pgfqpoint{2.771578in}{2.571933in}}{\pgfqpoint{2.771578in}{2.563697in}}%
\pgfpathcurveto{\pgfqpoint{2.771578in}{2.555461in}}{\pgfqpoint{2.774850in}{2.547561in}}{\pgfqpoint{2.780674in}{2.541737in}}%
\pgfpathcurveto{\pgfqpoint{2.786498in}{2.535913in}}{\pgfqpoint{2.794398in}{2.532641in}}{\pgfqpoint{2.802634in}{2.532641in}}%
\pgfpathclose%
\pgfusepath{stroke,fill}%
\end{pgfscope}%
\begin{pgfscope}%
\pgfpathrectangle{\pgfqpoint{0.100000in}{0.212622in}}{\pgfqpoint{3.696000in}{3.696000in}}%
\pgfusepath{clip}%
\pgfsetbuttcap%
\pgfsetroundjoin%
\definecolor{currentfill}{rgb}{0.121569,0.466667,0.705882}%
\pgfsetfillcolor{currentfill}%
\pgfsetfillopacity{0.855868}%
\pgfsetlinewidth{1.003750pt}%
\definecolor{currentstroke}{rgb}{0.121569,0.466667,0.705882}%
\pgfsetstrokecolor{currentstroke}%
\pgfsetstrokeopacity{0.855868}%
\pgfsetdash{}{0pt}%
\pgfpathmoveto{\pgfqpoint{2.991095in}{2.601419in}}%
\pgfpathcurveto{\pgfqpoint{2.999332in}{2.601419in}}{\pgfqpoint{3.007232in}{2.604692in}}{\pgfqpoint{3.013055in}{2.610516in}}%
\pgfpathcurveto{\pgfqpoint{3.018879in}{2.616340in}}{\pgfqpoint{3.022152in}{2.624240in}}{\pgfqpoint{3.022152in}{2.632476in}}%
\pgfpathcurveto{\pgfqpoint{3.022152in}{2.640712in}}{\pgfqpoint{3.018879in}{2.648612in}}{\pgfqpoint{3.013055in}{2.654436in}}%
\pgfpathcurveto{\pgfqpoint{3.007232in}{2.660260in}}{\pgfqpoint{2.999332in}{2.663532in}}{\pgfqpoint{2.991095in}{2.663532in}}%
\pgfpathcurveto{\pgfqpoint{2.982859in}{2.663532in}}{\pgfqpoint{2.974959in}{2.660260in}}{\pgfqpoint{2.969135in}{2.654436in}}%
\pgfpathcurveto{\pgfqpoint{2.963311in}{2.648612in}}{\pgfqpoint{2.960039in}{2.640712in}}{\pgfqpoint{2.960039in}{2.632476in}}%
\pgfpathcurveto{\pgfqpoint{2.960039in}{2.624240in}}{\pgfqpoint{2.963311in}{2.616340in}}{\pgfqpoint{2.969135in}{2.610516in}}%
\pgfpathcurveto{\pgfqpoint{2.974959in}{2.604692in}}{\pgfqpoint{2.982859in}{2.601419in}}{\pgfqpoint{2.991095in}{2.601419in}}%
\pgfpathclose%
\pgfusepath{stroke,fill}%
\end{pgfscope}%
\begin{pgfscope}%
\pgfpathrectangle{\pgfqpoint{0.100000in}{0.212622in}}{\pgfqpoint{3.696000in}{3.696000in}}%
\pgfusepath{clip}%
\pgfsetbuttcap%
\pgfsetroundjoin%
\definecolor{currentfill}{rgb}{0.121569,0.466667,0.705882}%
\pgfsetfillcolor{currentfill}%
\pgfsetfillopacity{0.856450}%
\pgfsetlinewidth{1.003750pt}%
\definecolor{currentstroke}{rgb}{0.121569,0.466667,0.705882}%
\pgfsetstrokecolor{currentstroke}%
\pgfsetstrokeopacity{0.856450}%
\pgfsetdash{}{0pt}%
\pgfpathmoveto{\pgfqpoint{3.089299in}{2.651924in}}%
\pgfpathcurveto{\pgfqpoint{3.097535in}{2.651924in}}{\pgfqpoint{3.105435in}{2.655196in}}{\pgfqpoint{3.111259in}{2.661020in}}%
\pgfpathcurveto{\pgfqpoint{3.117083in}{2.666844in}}{\pgfqpoint{3.120355in}{2.674744in}}{\pgfqpoint{3.120355in}{2.682980in}}%
\pgfpathcurveto{\pgfqpoint{3.120355in}{2.691217in}}{\pgfqpoint{3.117083in}{2.699117in}}{\pgfqpoint{3.111259in}{2.704941in}}%
\pgfpathcurveto{\pgfqpoint{3.105435in}{2.710765in}}{\pgfqpoint{3.097535in}{2.714037in}}{\pgfqpoint{3.089299in}{2.714037in}}%
\pgfpathcurveto{\pgfqpoint{3.081063in}{2.714037in}}{\pgfqpoint{3.073163in}{2.710765in}}{\pgfqpoint{3.067339in}{2.704941in}}%
\pgfpathcurveto{\pgfqpoint{3.061515in}{2.699117in}}{\pgfqpoint{3.058242in}{2.691217in}}{\pgfqpoint{3.058242in}{2.682980in}}%
\pgfpathcurveto{\pgfqpoint{3.058242in}{2.674744in}}{\pgfqpoint{3.061515in}{2.666844in}}{\pgfqpoint{3.067339in}{2.661020in}}%
\pgfpathcurveto{\pgfqpoint{3.073163in}{2.655196in}}{\pgfqpoint{3.081063in}{2.651924in}}{\pgfqpoint{3.089299in}{2.651924in}}%
\pgfpathclose%
\pgfusepath{stroke,fill}%
\end{pgfscope}%
\begin{pgfscope}%
\pgfpathrectangle{\pgfqpoint{0.100000in}{0.212622in}}{\pgfqpoint{3.696000in}{3.696000in}}%
\pgfusepath{clip}%
\pgfsetbuttcap%
\pgfsetroundjoin%
\definecolor{currentfill}{rgb}{0.121569,0.466667,0.705882}%
\pgfsetfillcolor{currentfill}%
\pgfsetfillopacity{0.856746}%
\pgfsetlinewidth{1.003750pt}%
\definecolor{currentstroke}{rgb}{0.121569,0.466667,0.705882}%
\pgfsetstrokecolor{currentstroke}%
\pgfsetstrokeopacity{0.856746}%
\pgfsetdash{}{0pt}%
\pgfpathmoveto{\pgfqpoint{3.069991in}{2.646194in}}%
\pgfpathcurveto{\pgfqpoint{3.078227in}{2.646194in}}{\pgfqpoint{3.086127in}{2.649467in}}{\pgfqpoint{3.091951in}{2.655291in}}%
\pgfpathcurveto{\pgfqpoint{3.097775in}{2.661115in}}{\pgfqpoint{3.101047in}{2.669015in}}{\pgfqpoint{3.101047in}{2.677251in}}%
\pgfpathcurveto{\pgfqpoint{3.101047in}{2.685487in}}{\pgfqpoint{3.097775in}{2.693387in}}{\pgfqpoint{3.091951in}{2.699211in}}%
\pgfpathcurveto{\pgfqpoint{3.086127in}{2.705035in}}{\pgfqpoint{3.078227in}{2.708307in}}{\pgfqpoint{3.069991in}{2.708307in}}%
\pgfpathcurveto{\pgfqpoint{3.061755in}{2.708307in}}{\pgfqpoint{3.053854in}{2.705035in}}{\pgfqpoint{3.048031in}{2.699211in}}%
\pgfpathcurveto{\pgfqpoint{3.042207in}{2.693387in}}{\pgfqpoint{3.038934in}{2.685487in}}{\pgfqpoint{3.038934in}{2.677251in}}%
\pgfpathcurveto{\pgfqpoint{3.038934in}{2.669015in}}{\pgfqpoint{3.042207in}{2.661115in}}{\pgfqpoint{3.048031in}{2.655291in}}%
\pgfpathcurveto{\pgfqpoint{3.053854in}{2.649467in}}{\pgfqpoint{3.061755in}{2.646194in}}{\pgfqpoint{3.069991in}{2.646194in}}%
\pgfpathclose%
\pgfusepath{stroke,fill}%
\end{pgfscope}%
\begin{pgfscope}%
\pgfpathrectangle{\pgfqpoint{0.100000in}{0.212622in}}{\pgfqpoint{3.696000in}{3.696000in}}%
\pgfusepath{clip}%
\pgfsetbuttcap%
\pgfsetroundjoin%
\definecolor{currentfill}{rgb}{0.121569,0.466667,0.705882}%
\pgfsetfillcolor{currentfill}%
\pgfsetfillopacity{0.856748}%
\pgfsetlinewidth{1.003750pt}%
\definecolor{currentstroke}{rgb}{0.121569,0.466667,0.705882}%
\pgfsetstrokecolor{currentstroke}%
\pgfsetstrokeopacity{0.856748}%
\pgfsetdash{}{0pt}%
\pgfpathmoveto{\pgfqpoint{2.975996in}{2.593567in}}%
\pgfpathcurveto{\pgfqpoint{2.984232in}{2.593567in}}{\pgfqpoint{2.992132in}{2.596839in}}{\pgfqpoint{2.997956in}{2.602663in}}%
\pgfpathcurveto{\pgfqpoint{3.003780in}{2.608487in}}{\pgfqpoint{3.007053in}{2.616387in}}{\pgfqpoint{3.007053in}{2.624623in}}%
\pgfpathcurveto{\pgfqpoint{3.007053in}{2.632860in}}{\pgfqpoint{3.003780in}{2.640760in}}{\pgfqpoint{2.997956in}{2.646584in}}%
\pgfpathcurveto{\pgfqpoint{2.992132in}{2.652408in}}{\pgfqpoint{2.984232in}{2.655680in}}{\pgfqpoint{2.975996in}{2.655680in}}%
\pgfpathcurveto{\pgfqpoint{2.967760in}{2.655680in}}{\pgfqpoint{2.959860in}{2.652408in}}{\pgfqpoint{2.954036in}{2.646584in}}%
\pgfpathcurveto{\pgfqpoint{2.948212in}{2.640760in}}{\pgfqpoint{2.944940in}{2.632860in}}{\pgfqpoint{2.944940in}{2.624623in}}%
\pgfpathcurveto{\pgfqpoint{2.944940in}{2.616387in}}{\pgfqpoint{2.948212in}{2.608487in}}{\pgfqpoint{2.954036in}{2.602663in}}%
\pgfpathcurveto{\pgfqpoint{2.959860in}{2.596839in}}{\pgfqpoint{2.967760in}{2.593567in}}{\pgfqpoint{2.975996in}{2.593567in}}%
\pgfpathclose%
\pgfusepath{stroke,fill}%
\end{pgfscope}%
\begin{pgfscope}%
\pgfpathrectangle{\pgfqpoint{0.100000in}{0.212622in}}{\pgfqpoint{3.696000in}{3.696000in}}%
\pgfusepath{clip}%
\pgfsetbuttcap%
\pgfsetroundjoin%
\definecolor{currentfill}{rgb}{0.121569,0.466667,0.705882}%
\pgfsetfillcolor{currentfill}%
\pgfsetfillopacity{0.857378}%
\pgfsetlinewidth{1.003750pt}%
\definecolor{currentstroke}{rgb}{0.121569,0.466667,0.705882}%
\pgfsetstrokecolor{currentstroke}%
\pgfsetstrokeopacity{0.857378}%
\pgfsetdash{}{0pt}%
\pgfpathmoveto{\pgfqpoint{1.233584in}{1.820556in}}%
\pgfpathcurveto{\pgfqpoint{1.241820in}{1.820556in}}{\pgfqpoint{1.249720in}{1.823829in}}{\pgfqpoint{1.255544in}{1.829652in}}%
\pgfpathcurveto{\pgfqpoint{1.261368in}{1.835476in}}{\pgfqpoint{1.264640in}{1.843376in}}{\pgfqpoint{1.264640in}{1.851613in}}%
\pgfpathcurveto{\pgfqpoint{1.264640in}{1.859849in}}{\pgfqpoint{1.261368in}{1.867749in}}{\pgfqpoint{1.255544in}{1.873573in}}%
\pgfpathcurveto{\pgfqpoint{1.249720in}{1.879397in}}{\pgfqpoint{1.241820in}{1.882669in}}{\pgfqpoint{1.233584in}{1.882669in}}%
\pgfpathcurveto{\pgfqpoint{1.225347in}{1.882669in}}{\pgfqpoint{1.217447in}{1.879397in}}{\pgfqpoint{1.211623in}{1.873573in}}%
\pgfpathcurveto{\pgfqpoint{1.205800in}{1.867749in}}{\pgfqpoint{1.202527in}{1.859849in}}{\pgfqpoint{1.202527in}{1.851613in}}%
\pgfpathcurveto{\pgfqpoint{1.202527in}{1.843376in}}{\pgfqpoint{1.205800in}{1.835476in}}{\pgfqpoint{1.211623in}{1.829652in}}%
\pgfpathcurveto{\pgfqpoint{1.217447in}{1.823829in}}{\pgfqpoint{1.225347in}{1.820556in}}{\pgfqpoint{1.233584in}{1.820556in}}%
\pgfpathclose%
\pgfusepath{stroke,fill}%
\end{pgfscope}%
\begin{pgfscope}%
\pgfpathrectangle{\pgfqpoint{0.100000in}{0.212622in}}{\pgfqpoint{3.696000in}{3.696000in}}%
\pgfusepath{clip}%
\pgfsetbuttcap%
\pgfsetroundjoin%
\definecolor{currentfill}{rgb}{0.121569,0.466667,0.705882}%
\pgfsetfillcolor{currentfill}%
\pgfsetfillopacity{0.857525}%
\pgfsetlinewidth{1.003750pt}%
\definecolor{currentstroke}{rgb}{0.121569,0.466667,0.705882}%
\pgfsetstrokecolor{currentstroke}%
\pgfsetstrokeopacity{0.857525}%
\pgfsetdash{}{0pt}%
\pgfpathmoveto{\pgfqpoint{1.235977in}{1.822571in}}%
\pgfpathcurveto{\pgfqpoint{1.244213in}{1.822571in}}{\pgfqpoint{1.252113in}{1.825843in}}{\pgfqpoint{1.257937in}{1.831667in}}%
\pgfpathcurveto{\pgfqpoint{1.263761in}{1.837491in}}{\pgfqpoint{1.267033in}{1.845391in}}{\pgfqpoint{1.267033in}{1.853628in}}%
\pgfpathcurveto{\pgfqpoint{1.267033in}{1.861864in}}{\pgfqpoint{1.263761in}{1.869764in}}{\pgfqpoint{1.257937in}{1.875588in}}%
\pgfpathcurveto{\pgfqpoint{1.252113in}{1.881412in}}{\pgfqpoint{1.244213in}{1.884684in}}{\pgfqpoint{1.235977in}{1.884684in}}%
\pgfpathcurveto{\pgfqpoint{1.227740in}{1.884684in}}{\pgfqpoint{1.219840in}{1.881412in}}{\pgfqpoint{1.214016in}{1.875588in}}%
\pgfpathcurveto{\pgfqpoint{1.208192in}{1.869764in}}{\pgfqpoint{1.204920in}{1.861864in}}{\pgfqpoint{1.204920in}{1.853628in}}%
\pgfpathcurveto{\pgfqpoint{1.204920in}{1.845391in}}{\pgfqpoint{1.208192in}{1.837491in}}{\pgfqpoint{1.214016in}{1.831667in}}%
\pgfpathcurveto{\pgfqpoint{1.219840in}{1.825843in}}{\pgfqpoint{1.227740in}{1.822571in}}{\pgfqpoint{1.235977in}{1.822571in}}%
\pgfpathclose%
\pgfusepath{stroke,fill}%
\end{pgfscope}%
\begin{pgfscope}%
\pgfpathrectangle{\pgfqpoint{0.100000in}{0.212622in}}{\pgfqpoint{3.696000in}{3.696000in}}%
\pgfusepath{clip}%
\pgfsetbuttcap%
\pgfsetroundjoin%
\definecolor{currentfill}{rgb}{0.121569,0.466667,0.705882}%
\pgfsetfillcolor{currentfill}%
\pgfsetfillopacity{0.858360}%
\pgfsetlinewidth{1.003750pt}%
\definecolor{currentstroke}{rgb}{0.121569,0.466667,0.705882}%
\pgfsetstrokecolor{currentstroke}%
\pgfsetstrokeopacity{0.858360}%
\pgfsetdash{}{0pt}%
\pgfpathmoveto{\pgfqpoint{2.825966in}{2.499010in}}%
\pgfpathcurveto{\pgfqpoint{2.834202in}{2.499010in}}{\pgfqpoint{2.842102in}{2.502283in}}{\pgfqpoint{2.847926in}{2.508107in}}%
\pgfpathcurveto{\pgfqpoint{2.853750in}{2.513931in}}{\pgfqpoint{2.857022in}{2.521831in}}{\pgfqpoint{2.857022in}{2.530067in}}%
\pgfpathcurveto{\pgfqpoint{2.857022in}{2.538303in}}{\pgfqpoint{2.853750in}{2.546203in}}{\pgfqpoint{2.847926in}{2.552027in}}%
\pgfpathcurveto{\pgfqpoint{2.842102in}{2.557851in}}{\pgfqpoint{2.834202in}{2.561123in}}{\pgfqpoint{2.825966in}{2.561123in}}%
\pgfpathcurveto{\pgfqpoint{2.817730in}{2.561123in}}{\pgfqpoint{2.809830in}{2.557851in}}{\pgfqpoint{2.804006in}{2.552027in}}%
\pgfpathcurveto{\pgfqpoint{2.798182in}{2.546203in}}{\pgfqpoint{2.794909in}{2.538303in}}{\pgfqpoint{2.794909in}{2.530067in}}%
\pgfpathcurveto{\pgfqpoint{2.794909in}{2.521831in}}{\pgfqpoint{2.798182in}{2.513931in}}{\pgfqpoint{2.804006in}{2.508107in}}%
\pgfpathcurveto{\pgfqpoint{2.809830in}{2.502283in}}{\pgfqpoint{2.817730in}{2.499010in}}{\pgfqpoint{2.825966in}{2.499010in}}%
\pgfpathclose%
\pgfusepath{stroke,fill}%
\end{pgfscope}%
\begin{pgfscope}%
\pgfpathrectangle{\pgfqpoint{0.100000in}{0.212622in}}{\pgfqpoint{3.696000in}{3.696000in}}%
\pgfusepath{clip}%
\pgfsetbuttcap%
\pgfsetroundjoin%
\definecolor{currentfill}{rgb}{0.121569,0.466667,0.705882}%
\pgfsetfillcolor{currentfill}%
\pgfsetfillopacity{0.858664}%
\pgfsetlinewidth{1.003750pt}%
\definecolor{currentstroke}{rgb}{0.121569,0.466667,0.705882}%
\pgfsetstrokecolor{currentstroke}%
\pgfsetstrokeopacity{0.858664}%
\pgfsetdash{}{0pt}%
\pgfpathmoveto{\pgfqpoint{3.047508in}{2.615598in}}%
\pgfpathcurveto{\pgfqpoint{3.055744in}{2.615598in}}{\pgfqpoint{3.063644in}{2.618870in}}{\pgfqpoint{3.069468in}{2.624694in}}%
\pgfpathcurveto{\pgfqpoint{3.075292in}{2.630518in}}{\pgfqpoint{3.078564in}{2.638418in}}{\pgfqpoint{3.078564in}{2.646654in}}%
\pgfpathcurveto{\pgfqpoint{3.078564in}{2.654891in}}{\pgfqpoint{3.075292in}{2.662791in}}{\pgfqpoint{3.069468in}{2.668615in}}%
\pgfpathcurveto{\pgfqpoint{3.063644in}{2.674439in}}{\pgfqpoint{3.055744in}{2.677711in}}{\pgfqpoint{3.047508in}{2.677711in}}%
\pgfpathcurveto{\pgfqpoint{3.039272in}{2.677711in}}{\pgfqpoint{3.031372in}{2.674439in}}{\pgfqpoint{3.025548in}{2.668615in}}%
\pgfpathcurveto{\pgfqpoint{3.019724in}{2.662791in}}{\pgfqpoint{3.016451in}{2.654891in}}{\pgfqpoint{3.016451in}{2.646654in}}%
\pgfpathcurveto{\pgfqpoint{3.016451in}{2.638418in}}{\pgfqpoint{3.019724in}{2.630518in}}{\pgfqpoint{3.025548in}{2.624694in}}%
\pgfpathcurveto{\pgfqpoint{3.031372in}{2.618870in}}{\pgfqpoint{3.039272in}{2.615598in}}{\pgfqpoint{3.047508in}{2.615598in}}%
\pgfpathclose%
\pgfusepath{stroke,fill}%
\end{pgfscope}%
\begin{pgfscope}%
\pgfpathrectangle{\pgfqpoint{0.100000in}{0.212622in}}{\pgfqpoint{3.696000in}{3.696000in}}%
\pgfusepath{clip}%
\pgfsetbuttcap%
\pgfsetroundjoin%
\definecolor{currentfill}{rgb}{0.121569,0.466667,0.705882}%
\pgfsetfillcolor{currentfill}%
\pgfsetfillopacity{0.858961}%
\pgfsetlinewidth{1.003750pt}%
\definecolor{currentstroke}{rgb}{0.121569,0.466667,0.705882}%
\pgfsetstrokecolor{currentstroke}%
\pgfsetstrokeopacity{0.858961}%
\pgfsetdash{}{0pt}%
\pgfpathmoveto{\pgfqpoint{3.059278in}{2.632984in}}%
\pgfpathcurveto{\pgfqpoint{3.067514in}{2.632984in}}{\pgfqpoint{3.075414in}{2.636257in}}{\pgfqpoint{3.081238in}{2.642081in}}%
\pgfpathcurveto{\pgfqpoint{3.087062in}{2.647905in}}{\pgfqpoint{3.090334in}{2.655805in}}{\pgfqpoint{3.090334in}{2.664041in}}%
\pgfpathcurveto{\pgfqpoint{3.090334in}{2.672277in}}{\pgfqpoint{3.087062in}{2.680177in}}{\pgfqpoint{3.081238in}{2.686001in}}%
\pgfpathcurveto{\pgfqpoint{3.075414in}{2.691825in}}{\pgfqpoint{3.067514in}{2.695097in}}{\pgfqpoint{3.059278in}{2.695097in}}%
\pgfpathcurveto{\pgfqpoint{3.051041in}{2.695097in}}{\pgfqpoint{3.043141in}{2.691825in}}{\pgfqpoint{3.037317in}{2.686001in}}%
\pgfpathcurveto{\pgfqpoint{3.031493in}{2.680177in}}{\pgfqpoint{3.028221in}{2.672277in}}{\pgfqpoint{3.028221in}{2.664041in}}%
\pgfpathcurveto{\pgfqpoint{3.028221in}{2.655805in}}{\pgfqpoint{3.031493in}{2.647905in}}{\pgfqpoint{3.037317in}{2.642081in}}%
\pgfpathcurveto{\pgfqpoint{3.043141in}{2.636257in}}{\pgfqpoint{3.051041in}{2.632984in}}{\pgfqpoint{3.059278in}{2.632984in}}%
\pgfpathclose%
\pgfusepath{stroke,fill}%
\end{pgfscope}%
\begin{pgfscope}%
\pgfpathrectangle{\pgfqpoint{0.100000in}{0.212622in}}{\pgfqpoint{3.696000in}{3.696000in}}%
\pgfusepath{clip}%
\pgfsetbuttcap%
\pgfsetroundjoin%
\definecolor{currentfill}{rgb}{0.121569,0.466667,0.705882}%
\pgfsetfillcolor{currentfill}%
\pgfsetfillopacity{0.860032}%
\pgfsetlinewidth{1.003750pt}%
\definecolor{currentstroke}{rgb}{0.121569,0.466667,0.705882}%
\pgfsetstrokecolor{currentstroke}%
\pgfsetstrokeopacity{0.860032}%
\pgfsetdash{}{0pt}%
\pgfpathmoveto{\pgfqpoint{3.065523in}{2.633653in}}%
\pgfpathcurveto{\pgfqpoint{3.073759in}{2.633653in}}{\pgfqpoint{3.081659in}{2.636925in}}{\pgfqpoint{3.087483in}{2.642749in}}%
\pgfpathcurveto{\pgfqpoint{3.093307in}{2.648573in}}{\pgfqpoint{3.096579in}{2.656473in}}{\pgfqpoint{3.096579in}{2.664709in}}%
\pgfpathcurveto{\pgfqpoint{3.096579in}{2.672945in}}{\pgfqpoint{3.093307in}{2.680845in}}{\pgfqpoint{3.087483in}{2.686669in}}%
\pgfpathcurveto{\pgfqpoint{3.081659in}{2.692493in}}{\pgfqpoint{3.073759in}{2.695766in}}{\pgfqpoint{3.065523in}{2.695766in}}%
\pgfpathcurveto{\pgfqpoint{3.057287in}{2.695766in}}{\pgfqpoint{3.049387in}{2.692493in}}{\pgfqpoint{3.043563in}{2.686669in}}%
\pgfpathcurveto{\pgfqpoint{3.037739in}{2.680845in}}{\pgfqpoint{3.034466in}{2.672945in}}{\pgfqpoint{3.034466in}{2.664709in}}%
\pgfpathcurveto{\pgfqpoint{3.034466in}{2.656473in}}{\pgfqpoint{3.037739in}{2.648573in}}{\pgfqpoint{3.043563in}{2.642749in}}%
\pgfpathcurveto{\pgfqpoint{3.049387in}{2.636925in}}{\pgfqpoint{3.057287in}{2.633653in}}{\pgfqpoint{3.065523in}{2.633653in}}%
\pgfpathclose%
\pgfusepath{stroke,fill}%
\end{pgfscope}%
\begin{pgfscope}%
\pgfpathrectangle{\pgfqpoint{0.100000in}{0.212622in}}{\pgfqpoint{3.696000in}{3.696000in}}%
\pgfusepath{clip}%
\pgfsetbuttcap%
\pgfsetroundjoin%
\definecolor{currentfill}{rgb}{0.121569,0.466667,0.705882}%
\pgfsetfillcolor{currentfill}%
\pgfsetfillopacity{0.860679}%
\pgfsetlinewidth{1.003750pt}%
\definecolor{currentstroke}{rgb}{0.121569,0.466667,0.705882}%
\pgfsetstrokecolor{currentstroke}%
\pgfsetstrokeopacity{0.860679}%
\pgfsetdash{}{0pt}%
\pgfpathmoveto{\pgfqpoint{3.088419in}{2.642562in}}%
\pgfpathcurveto{\pgfqpoint{3.096655in}{2.642562in}}{\pgfqpoint{3.104555in}{2.645835in}}{\pgfqpoint{3.110379in}{2.651659in}}%
\pgfpathcurveto{\pgfqpoint{3.116203in}{2.657483in}}{\pgfqpoint{3.119475in}{2.665383in}}{\pgfqpoint{3.119475in}{2.673619in}}%
\pgfpathcurveto{\pgfqpoint{3.119475in}{2.681855in}}{\pgfqpoint{3.116203in}{2.689755in}}{\pgfqpoint{3.110379in}{2.695579in}}%
\pgfpathcurveto{\pgfqpoint{3.104555in}{2.701403in}}{\pgfqpoint{3.096655in}{2.704675in}}{\pgfqpoint{3.088419in}{2.704675in}}%
\pgfpathcurveto{\pgfqpoint{3.080182in}{2.704675in}}{\pgfqpoint{3.072282in}{2.701403in}}{\pgfqpoint{3.066458in}{2.695579in}}%
\pgfpathcurveto{\pgfqpoint{3.060634in}{2.689755in}}{\pgfqpoint{3.057362in}{2.681855in}}{\pgfqpoint{3.057362in}{2.673619in}}%
\pgfpathcurveto{\pgfqpoint{3.057362in}{2.665383in}}{\pgfqpoint{3.060634in}{2.657483in}}{\pgfqpoint{3.066458in}{2.651659in}}%
\pgfpathcurveto{\pgfqpoint{3.072282in}{2.645835in}}{\pgfqpoint{3.080182in}{2.642562in}}{\pgfqpoint{3.088419in}{2.642562in}}%
\pgfpathclose%
\pgfusepath{stroke,fill}%
\end{pgfscope}%
\begin{pgfscope}%
\pgfpathrectangle{\pgfqpoint{0.100000in}{0.212622in}}{\pgfqpoint{3.696000in}{3.696000in}}%
\pgfusepath{clip}%
\pgfsetbuttcap%
\pgfsetroundjoin%
\definecolor{currentfill}{rgb}{0.121569,0.466667,0.705882}%
\pgfsetfillcolor{currentfill}%
\pgfsetfillopacity{0.861531}%
\pgfsetlinewidth{1.003750pt}%
\definecolor{currentstroke}{rgb}{0.121569,0.466667,0.705882}%
\pgfsetstrokecolor{currentstroke}%
\pgfsetstrokeopacity{0.861531}%
\pgfsetdash{}{0pt}%
\pgfpathmoveto{\pgfqpoint{3.097306in}{2.645479in}}%
\pgfpathcurveto{\pgfqpoint{3.105542in}{2.645479in}}{\pgfqpoint{3.113442in}{2.648751in}}{\pgfqpoint{3.119266in}{2.654575in}}%
\pgfpathcurveto{\pgfqpoint{3.125090in}{2.660399in}}{\pgfqpoint{3.128362in}{2.668299in}}{\pgfqpoint{3.128362in}{2.676536in}}%
\pgfpathcurveto{\pgfqpoint{3.128362in}{2.684772in}}{\pgfqpoint{3.125090in}{2.692672in}}{\pgfqpoint{3.119266in}{2.698496in}}%
\pgfpathcurveto{\pgfqpoint{3.113442in}{2.704320in}}{\pgfqpoint{3.105542in}{2.707592in}}{\pgfqpoint{3.097306in}{2.707592in}}%
\pgfpathcurveto{\pgfqpoint{3.089070in}{2.707592in}}{\pgfqpoint{3.081170in}{2.704320in}}{\pgfqpoint{3.075346in}{2.698496in}}%
\pgfpathcurveto{\pgfqpoint{3.069522in}{2.692672in}}{\pgfqpoint{3.066249in}{2.684772in}}{\pgfqpoint{3.066249in}{2.676536in}}%
\pgfpathcurveto{\pgfqpoint{3.066249in}{2.668299in}}{\pgfqpoint{3.069522in}{2.660399in}}{\pgfqpoint{3.075346in}{2.654575in}}%
\pgfpathcurveto{\pgfqpoint{3.081170in}{2.648751in}}{\pgfqpoint{3.089070in}{2.645479in}}{\pgfqpoint{3.097306in}{2.645479in}}%
\pgfpathclose%
\pgfusepath{stroke,fill}%
\end{pgfscope}%
\begin{pgfscope}%
\pgfpathrectangle{\pgfqpoint{0.100000in}{0.212622in}}{\pgfqpoint{3.696000in}{3.696000in}}%
\pgfusepath{clip}%
\pgfsetbuttcap%
\pgfsetroundjoin%
\definecolor{currentfill}{rgb}{0.121569,0.466667,0.705882}%
\pgfsetfillcolor{currentfill}%
\pgfsetfillopacity{0.861551}%
\pgfsetlinewidth{1.003750pt}%
\definecolor{currentstroke}{rgb}{0.121569,0.466667,0.705882}%
\pgfsetstrokecolor{currentstroke}%
\pgfsetstrokeopacity{0.861551}%
\pgfsetdash{}{0pt}%
\pgfpathmoveto{\pgfqpoint{1.238367in}{1.825613in}}%
\pgfpathcurveto{\pgfqpoint{1.246604in}{1.825613in}}{\pgfqpoint{1.254504in}{1.828886in}}{\pgfqpoint{1.260328in}{1.834710in}}%
\pgfpathcurveto{\pgfqpoint{1.266152in}{1.840534in}}{\pgfqpoint{1.269424in}{1.848434in}}{\pgfqpoint{1.269424in}{1.856670in}}%
\pgfpathcurveto{\pgfqpoint{1.269424in}{1.864906in}}{\pgfqpoint{1.266152in}{1.872806in}}{\pgfqpoint{1.260328in}{1.878630in}}%
\pgfpathcurveto{\pgfqpoint{1.254504in}{1.884454in}}{\pgfqpoint{1.246604in}{1.887726in}}{\pgfqpoint{1.238367in}{1.887726in}}%
\pgfpathcurveto{\pgfqpoint{1.230131in}{1.887726in}}{\pgfqpoint{1.222231in}{1.884454in}}{\pgfqpoint{1.216407in}{1.878630in}}%
\pgfpathcurveto{\pgfqpoint{1.210583in}{1.872806in}}{\pgfqpoint{1.207311in}{1.864906in}}{\pgfqpoint{1.207311in}{1.856670in}}%
\pgfpathcurveto{\pgfqpoint{1.207311in}{1.848434in}}{\pgfqpoint{1.210583in}{1.840534in}}{\pgfqpoint{1.216407in}{1.834710in}}%
\pgfpathcurveto{\pgfqpoint{1.222231in}{1.828886in}}{\pgfqpoint{1.230131in}{1.825613in}}{\pgfqpoint{1.238367in}{1.825613in}}%
\pgfpathclose%
\pgfusepath{stroke,fill}%
\end{pgfscope}%
\begin{pgfscope}%
\pgfpathrectangle{\pgfqpoint{0.100000in}{0.212622in}}{\pgfqpoint{3.696000in}{3.696000in}}%
\pgfusepath{clip}%
\pgfsetbuttcap%
\pgfsetroundjoin%
\definecolor{currentfill}{rgb}{0.121569,0.466667,0.705882}%
\pgfsetfillcolor{currentfill}%
\pgfsetfillopacity{0.862930}%
\pgfsetlinewidth{1.003750pt}%
\definecolor{currentstroke}{rgb}{0.121569,0.466667,0.705882}%
\pgfsetstrokecolor{currentstroke}%
\pgfsetstrokeopacity{0.862930}%
\pgfsetdash{}{0pt}%
\pgfpathmoveto{\pgfqpoint{1.226774in}{1.810545in}}%
\pgfpathcurveto{\pgfqpoint{1.235010in}{1.810545in}}{\pgfqpoint{1.242910in}{1.813818in}}{\pgfqpoint{1.248734in}{1.819642in}}%
\pgfpathcurveto{\pgfqpoint{1.254558in}{1.825466in}}{\pgfqpoint{1.257830in}{1.833366in}}{\pgfqpoint{1.257830in}{1.841602in}}%
\pgfpathcurveto{\pgfqpoint{1.257830in}{1.849838in}}{\pgfqpoint{1.254558in}{1.857738in}}{\pgfqpoint{1.248734in}{1.863562in}}%
\pgfpathcurveto{\pgfqpoint{1.242910in}{1.869386in}}{\pgfqpoint{1.235010in}{1.872658in}}{\pgfqpoint{1.226774in}{1.872658in}}%
\pgfpathcurveto{\pgfqpoint{1.218537in}{1.872658in}}{\pgfqpoint{1.210637in}{1.869386in}}{\pgfqpoint{1.204813in}{1.863562in}}%
\pgfpathcurveto{\pgfqpoint{1.198990in}{1.857738in}}{\pgfqpoint{1.195717in}{1.849838in}}{\pgfqpoint{1.195717in}{1.841602in}}%
\pgfpathcurveto{\pgfqpoint{1.195717in}{1.833366in}}{\pgfqpoint{1.198990in}{1.825466in}}{\pgfqpoint{1.204813in}{1.819642in}}%
\pgfpathcurveto{\pgfqpoint{1.210637in}{1.813818in}}{\pgfqpoint{1.218537in}{1.810545in}}{\pgfqpoint{1.226774in}{1.810545in}}%
\pgfpathclose%
\pgfusepath{stroke,fill}%
\end{pgfscope}%
\begin{pgfscope}%
\pgfpathrectangle{\pgfqpoint{0.100000in}{0.212622in}}{\pgfqpoint{3.696000in}{3.696000in}}%
\pgfusepath{clip}%
\pgfsetbuttcap%
\pgfsetroundjoin%
\definecolor{currentfill}{rgb}{0.121569,0.466667,0.705882}%
\pgfsetfillcolor{currentfill}%
\pgfsetfillopacity{0.863845}%
\pgfsetlinewidth{1.003750pt}%
\definecolor{currentstroke}{rgb}{0.121569,0.466667,0.705882}%
\pgfsetstrokecolor{currentstroke}%
\pgfsetstrokeopacity{0.863845}%
\pgfsetdash{}{0pt}%
\pgfpathmoveto{\pgfqpoint{1.272561in}{1.829978in}}%
\pgfpathcurveto{\pgfqpoint{1.280797in}{1.829978in}}{\pgfqpoint{1.288698in}{1.833250in}}{\pgfqpoint{1.294521in}{1.839074in}}%
\pgfpathcurveto{\pgfqpoint{1.300345in}{1.844898in}}{\pgfqpoint{1.303618in}{1.852798in}}{\pgfqpoint{1.303618in}{1.861035in}}%
\pgfpathcurveto{\pgfqpoint{1.303618in}{1.869271in}}{\pgfqpoint{1.300345in}{1.877171in}}{\pgfqpoint{1.294521in}{1.882995in}}%
\pgfpathcurveto{\pgfqpoint{1.288698in}{1.888819in}}{\pgfqpoint{1.280797in}{1.892091in}}{\pgfqpoint{1.272561in}{1.892091in}}%
\pgfpathcurveto{\pgfqpoint{1.264325in}{1.892091in}}{\pgfqpoint{1.256425in}{1.888819in}}{\pgfqpoint{1.250601in}{1.882995in}}%
\pgfpathcurveto{\pgfqpoint{1.244777in}{1.877171in}}{\pgfqpoint{1.241505in}{1.869271in}}{\pgfqpoint{1.241505in}{1.861035in}}%
\pgfpathcurveto{\pgfqpoint{1.241505in}{1.852798in}}{\pgfqpoint{1.244777in}{1.844898in}}{\pgfqpoint{1.250601in}{1.839074in}}%
\pgfpathcurveto{\pgfqpoint{1.256425in}{1.833250in}}{\pgfqpoint{1.264325in}{1.829978in}}{\pgfqpoint{1.272561in}{1.829978in}}%
\pgfpathclose%
\pgfusepath{stroke,fill}%
\end{pgfscope}%
\begin{pgfscope}%
\pgfpathrectangle{\pgfqpoint{0.100000in}{0.212622in}}{\pgfqpoint{3.696000in}{3.696000in}}%
\pgfusepath{clip}%
\pgfsetbuttcap%
\pgfsetroundjoin%
\definecolor{currentfill}{rgb}{0.121569,0.466667,0.705882}%
\pgfsetfillcolor{currentfill}%
\pgfsetfillopacity{0.864181}%
\pgfsetlinewidth{1.003750pt}%
\definecolor{currentstroke}{rgb}{0.121569,0.466667,0.705882}%
\pgfsetstrokecolor{currentstroke}%
\pgfsetstrokeopacity{0.864181}%
\pgfsetdash{}{0pt}%
\pgfpathmoveto{\pgfqpoint{1.235695in}{1.823593in}}%
\pgfpathcurveto{\pgfqpoint{1.243931in}{1.823593in}}{\pgfqpoint{1.251831in}{1.826865in}}{\pgfqpoint{1.257655in}{1.832689in}}%
\pgfpathcurveto{\pgfqpoint{1.263479in}{1.838513in}}{\pgfqpoint{1.266751in}{1.846413in}}{\pgfqpoint{1.266751in}{1.854649in}}%
\pgfpathcurveto{\pgfqpoint{1.266751in}{1.862885in}}{\pgfqpoint{1.263479in}{1.870785in}}{\pgfqpoint{1.257655in}{1.876609in}}%
\pgfpathcurveto{\pgfqpoint{1.251831in}{1.882433in}}{\pgfqpoint{1.243931in}{1.885706in}}{\pgfqpoint{1.235695in}{1.885706in}}%
\pgfpathcurveto{\pgfqpoint{1.227458in}{1.885706in}}{\pgfqpoint{1.219558in}{1.882433in}}{\pgfqpoint{1.213734in}{1.876609in}}%
\pgfpathcurveto{\pgfqpoint{1.207910in}{1.870785in}}{\pgfqpoint{1.204638in}{1.862885in}}{\pgfqpoint{1.204638in}{1.854649in}}%
\pgfpathcurveto{\pgfqpoint{1.204638in}{1.846413in}}{\pgfqpoint{1.207910in}{1.838513in}}{\pgfqpoint{1.213734in}{1.832689in}}%
\pgfpathcurveto{\pgfqpoint{1.219558in}{1.826865in}}{\pgfqpoint{1.227458in}{1.823593in}}{\pgfqpoint{1.235695in}{1.823593in}}%
\pgfpathclose%
\pgfusepath{stroke,fill}%
\end{pgfscope}%
\begin{pgfscope}%
\pgfpathrectangle{\pgfqpoint{0.100000in}{0.212622in}}{\pgfqpoint{3.696000in}{3.696000in}}%
\pgfusepath{clip}%
\pgfsetbuttcap%
\pgfsetroundjoin%
\definecolor{currentfill}{rgb}{0.121569,0.466667,0.705882}%
\pgfsetfillcolor{currentfill}%
\pgfsetfillopacity{0.865175}%
\pgfsetlinewidth{1.003750pt}%
\definecolor{currentstroke}{rgb}{0.121569,0.466667,0.705882}%
\pgfsetstrokecolor{currentstroke}%
\pgfsetstrokeopacity{0.865175}%
\pgfsetdash{}{0pt}%
\pgfpathmoveto{\pgfqpoint{1.301648in}{1.851478in}}%
\pgfpathcurveto{\pgfqpoint{1.309884in}{1.851478in}}{\pgfqpoint{1.317784in}{1.854750in}}{\pgfqpoint{1.323608in}{1.860574in}}%
\pgfpathcurveto{\pgfqpoint{1.329432in}{1.866398in}}{\pgfqpoint{1.332704in}{1.874298in}}{\pgfqpoint{1.332704in}{1.882534in}}%
\pgfpathcurveto{\pgfqpoint{1.332704in}{1.890771in}}{\pgfqpoint{1.329432in}{1.898671in}}{\pgfqpoint{1.323608in}{1.904495in}}%
\pgfpathcurveto{\pgfqpoint{1.317784in}{1.910318in}}{\pgfqpoint{1.309884in}{1.913591in}}{\pgfqpoint{1.301648in}{1.913591in}}%
\pgfpathcurveto{\pgfqpoint{1.293412in}{1.913591in}}{\pgfqpoint{1.285512in}{1.910318in}}{\pgfqpoint{1.279688in}{1.904495in}}%
\pgfpathcurveto{\pgfqpoint{1.273864in}{1.898671in}}{\pgfqpoint{1.270591in}{1.890771in}}{\pgfqpoint{1.270591in}{1.882534in}}%
\pgfpathcurveto{\pgfqpoint{1.270591in}{1.874298in}}{\pgfqpoint{1.273864in}{1.866398in}}{\pgfqpoint{1.279688in}{1.860574in}}%
\pgfpathcurveto{\pgfqpoint{1.285512in}{1.854750in}}{\pgfqpoint{1.293412in}{1.851478in}}{\pgfqpoint{1.301648in}{1.851478in}}%
\pgfpathclose%
\pgfusepath{stroke,fill}%
\end{pgfscope}%
\begin{pgfscope}%
\pgfpathrectangle{\pgfqpoint{0.100000in}{0.212622in}}{\pgfqpoint{3.696000in}{3.696000in}}%
\pgfusepath{clip}%
\pgfsetbuttcap%
\pgfsetroundjoin%
\definecolor{currentfill}{rgb}{0.121569,0.466667,0.705882}%
\pgfsetfillcolor{currentfill}%
\pgfsetfillopacity{0.868810}%
\pgfsetlinewidth{1.003750pt}%
\definecolor{currentstroke}{rgb}{0.121569,0.466667,0.705882}%
\pgfsetstrokecolor{currentstroke}%
\pgfsetstrokeopacity{0.868810}%
\pgfsetdash{}{0pt}%
\pgfpathmoveto{\pgfqpoint{1.271571in}{1.829156in}}%
\pgfpathcurveto{\pgfqpoint{1.279807in}{1.829156in}}{\pgfqpoint{1.287707in}{1.832428in}}{\pgfqpoint{1.293531in}{1.838252in}}%
\pgfpathcurveto{\pgfqpoint{1.299355in}{1.844076in}}{\pgfqpoint{1.302628in}{1.851976in}}{\pgfqpoint{1.302628in}{1.860212in}}%
\pgfpathcurveto{\pgfqpoint{1.302628in}{1.868448in}}{\pgfqpoint{1.299355in}{1.876348in}}{\pgfqpoint{1.293531in}{1.882172in}}%
\pgfpathcurveto{\pgfqpoint{1.287707in}{1.887996in}}{\pgfqpoint{1.279807in}{1.891269in}}{\pgfqpoint{1.271571in}{1.891269in}}%
\pgfpathcurveto{\pgfqpoint{1.263335in}{1.891269in}}{\pgfqpoint{1.255435in}{1.887996in}}{\pgfqpoint{1.249611in}{1.882172in}}%
\pgfpathcurveto{\pgfqpoint{1.243787in}{1.876348in}}{\pgfqpoint{1.240515in}{1.868448in}}{\pgfqpoint{1.240515in}{1.860212in}}%
\pgfpathcurveto{\pgfqpoint{1.240515in}{1.851976in}}{\pgfqpoint{1.243787in}{1.844076in}}{\pgfqpoint{1.249611in}{1.838252in}}%
\pgfpathcurveto{\pgfqpoint{1.255435in}{1.832428in}}{\pgfqpoint{1.263335in}{1.829156in}}{\pgfqpoint{1.271571in}{1.829156in}}%
\pgfpathclose%
\pgfusepath{stroke,fill}%
\end{pgfscope}%
\begin{pgfscope}%
\pgfpathrectangle{\pgfqpoint{0.100000in}{0.212622in}}{\pgfqpoint{3.696000in}{3.696000in}}%
\pgfusepath{clip}%
\pgfsetbuttcap%
\pgfsetroundjoin%
\definecolor{currentfill}{rgb}{0.121569,0.466667,0.705882}%
\pgfsetfillcolor{currentfill}%
\pgfsetfillopacity{0.869660}%
\pgfsetlinewidth{1.003750pt}%
\definecolor{currentstroke}{rgb}{0.121569,0.466667,0.705882}%
\pgfsetstrokecolor{currentstroke}%
\pgfsetstrokeopacity{0.869660}%
\pgfsetdash{}{0pt}%
\pgfpathmoveto{\pgfqpoint{3.087001in}{2.634851in}}%
\pgfpathcurveto{\pgfqpoint{3.095237in}{2.634851in}}{\pgfqpoint{3.103137in}{2.638123in}}{\pgfqpoint{3.108961in}{2.643947in}}%
\pgfpathcurveto{\pgfqpoint{3.114785in}{2.649771in}}{\pgfqpoint{3.118057in}{2.657671in}}{\pgfqpoint{3.118057in}{2.665908in}}%
\pgfpathcurveto{\pgfqpoint{3.118057in}{2.674144in}}{\pgfqpoint{3.114785in}{2.682044in}}{\pgfqpoint{3.108961in}{2.687868in}}%
\pgfpathcurveto{\pgfqpoint{3.103137in}{2.693692in}}{\pgfqpoint{3.095237in}{2.696964in}}{\pgfqpoint{3.087001in}{2.696964in}}%
\pgfpathcurveto{\pgfqpoint{3.078765in}{2.696964in}}{\pgfqpoint{3.070864in}{2.693692in}}{\pgfqpoint{3.065041in}{2.687868in}}%
\pgfpathcurveto{\pgfqpoint{3.059217in}{2.682044in}}{\pgfqpoint{3.055944in}{2.674144in}}{\pgfqpoint{3.055944in}{2.665908in}}%
\pgfpathcurveto{\pgfqpoint{3.055944in}{2.657671in}}{\pgfqpoint{3.059217in}{2.649771in}}{\pgfqpoint{3.065041in}{2.643947in}}%
\pgfpathcurveto{\pgfqpoint{3.070864in}{2.638123in}}{\pgfqpoint{3.078765in}{2.634851in}}{\pgfqpoint{3.087001in}{2.634851in}}%
\pgfpathclose%
\pgfusepath{stroke,fill}%
\end{pgfscope}%
\begin{pgfscope}%
\pgfpathrectangle{\pgfqpoint{0.100000in}{0.212622in}}{\pgfqpoint{3.696000in}{3.696000in}}%
\pgfusepath{clip}%
\pgfsetbuttcap%
\pgfsetroundjoin%
\definecolor{currentfill}{rgb}{0.121569,0.466667,0.705882}%
\pgfsetfillcolor{currentfill}%
\pgfsetfillopacity{0.871094}%
\pgfsetlinewidth{1.003750pt}%
\definecolor{currentstroke}{rgb}{0.121569,0.466667,0.705882}%
\pgfsetstrokecolor{currentstroke}%
\pgfsetstrokeopacity{0.871094}%
\pgfsetdash{}{0pt}%
\pgfpathmoveto{\pgfqpoint{2.435366in}{2.361062in}}%
\pgfpathcurveto{\pgfqpoint{2.443603in}{2.361062in}}{\pgfqpoint{2.451503in}{2.364335in}}{\pgfqpoint{2.457327in}{2.370158in}}%
\pgfpathcurveto{\pgfqpoint{2.463151in}{2.375982in}}{\pgfqpoint{2.466423in}{2.383882in}}{\pgfqpoint{2.466423in}{2.392119in}}%
\pgfpathcurveto{\pgfqpoint{2.466423in}{2.400355in}}{\pgfqpoint{2.463151in}{2.408255in}}{\pgfqpoint{2.457327in}{2.414079in}}%
\pgfpathcurveto{\pgfqpoint{2.451503in}{2.419903in}}{\pgfqpoint{2.443603in}{2.423175in}}{\pgfqpoint{2.435366in}{2.423175in}}%
\pgfpathcurveto{\pgfqpoint{2.427130in}{2.423175in}}{\pgfqpoint{2.419230in}{2.419903in}}{\pgfqpoint{2.413406in}{2.414079in}}%
\pgfpathcurveto{\pgfqpoint{2.407582in}{2.408255in}}{\pgfqpoint{2.404310in}{2.400355in}}{\pgfqpoint{2.404310in}{2.392119in}}%
\pgfpathcurveto{\pgfqpoint{2.404310in}{2.383882in}}{\pgfqpoint{2.407582in}{2.375982in}}{\pgfqpoint{2.413406in}{2.370158in}}%
\pgfpathcurveto{\pgfqpoint{2.419230in}{2.364335in}}{\pgfqpoint{2.427130in}{2.361062in}}{\pgfqpoint{2.435366in}{2.361062in}}%
\pgfpathclose%
\pgfusepath{stroke,fill}%
\end{pgfscope}%
\begin{pgfscope}%
\pgfpathrectangle{\pgfqpoint{0.100000in}{0.212622in}}{\pgfqpoint{3.696000in}{3.696000in}}%
\pgfusepath{clip}%
\pgfsetbuttcap%
\pgfsetroundjoin%
\definecolor{currentfill}{rgb}{0.121569,0.466667,0.705882}%
\pgfsetfillcolor{currentfill}%
\pgfsetfillopacity{0.871300}%
\pgfsetlinewidth{1.003750pt}%
\definecolor{currentstroke}{rgb}{0.121569,0.466667,0.705882}%
\pgfsetstrokecolor{currentstroke}%
\pgfsetstrokeopacity{0.871300}%
\pgfsetdash{}{0pt}%
\pgfpathmoveto{\pgfqpoint{1.271961in}{1.821976in}}%
\pgfpathcurveto{\pgfqpoint{1.280197in}{1.821976in}}{\pgfqpoint{1.288097in}{1.825248in}}{\pgfqpoint{1.293921in}{1.831072in}}%
\pgfpathcurveto{\pgfqpoint{1.299745in}{1.836896in}}{\pgfqpoint{1.303018in}{1.844796in}}{\pgfqpoint{1.303018in}{1.853033in}}%
\pgfpathcurveto{\pgfqpoint{1.303018in}{1.861269in}}{\pgfqpoint{1.299745in}{1.869169in}}{\pgfqpoint{1.293921in}{1.874993in}}%
\pgfpathcurveto{\pgfqpoint{1.288097in}{1.880817in}}{\pgfqpoint{1.280197in}{1.884089in}}{\pgfqpoint{1.271961in}{1.884089in}}%
\pgfpathcurveto{\pgfqpoint{1.263725in}{1.884089in}}{\pgfqpoint{1.255825in}{1.880817in}}{\pgfqpoint{1.250001in}{1.874993in}}%
\pgfpathcurveto{\pgfqpoint{1.244177in}{1.869169in}}{\pgfqpoint{1.240905in}{1.861269in}}{\pgfqpoint{1.240905in}{1.853033in}}%
\pgfpathcurveto{\pgfqpoint{1.240905in}{1.844796in}}{\pgfqpoint{1.244177in}{1.836896in}}{\pgfqpoint{1.250001in}{1.831072in}}%
\pgfpathcurveto{\pgfqpoint{1.255825in}{1.825248in}}{\pgfqpoint{1.263725in}{1.821976in}}{\pgfqpoint{1.271961in}{1.821976in}}%
\pgfpathclose%
\pgfusepath{stroke,fill}%
\end{pgfscope}%
\begin{pgfscope}%
\pgfpathrectangle{\pgfqpoint{0.100000in}{0.212622in}}{\pgfqpoint{3.696000in}{3.696000in}}%
\pgfusepath{clip}%
\pgfsetbuttcap%
\pgfsetroundjoin%
\definecolor{currentfill}{rgb}{0.121569,0.466667,0.705882}%
\pgfsetfillcolor{currentfill}%
\pgfsetfillopacity{0.871606}%
\pgfsetlinewidth{1.003750pt}%
\definecolor{currentstroke}{rgb}{0.121569,0.466667,0.705882}%
\pgfsetstrokecolor{currentstroke}%
\pgfsetstrokeopacity{0.871606}%
\pgfsetdash{}{0pt}%
\pgfpathmoveto{\pgfqpoint{1.221834in}{1.808053in}}%
\pgfpathcurveto{\pgfqpoint{1.230070in}{1.808053in}}{\pgfqpoint{1.237970in}{1.811325in}}{\pgfqpoint{1.243794in}{1.817149in}}%
\pgfpathcurveto{\pgfqpoint{1.249618in}{1.822973in}}{\pgfqpoint{1.252890in}{1.830873in}}{\pgfqpoint{1.252890in}{1.839109in}}%
\pgfpathcurveto{\pgfqpoint{1.252890in}{1.847345in}}{\pgfqpoint{1.249618in}{1.855246in}}{\pgfqpoint{1.243794in}{1.861069in}}%
\pgfpathcurveto{\pgfqpoint{1.237970in}{1.866893in}}{\pgfqpoint{1.230070in}{1.870166in}}{\pgfqpoint{1.221834in}{1.870166in}}%
\pgfpathcurveto{\pgfqpoint{1.213597in}{1.870166in}}{\pgfqpoint{1.205697in}{1.866893in}}{\pgfqpoint{1.199874in}{1.861069in}}%
\pgfpathcurveto{\pgfqpoint{1.194050in}{1.855246in}}{\pgfqpoint{1.190777in}{1.847345in}}{\pgfqpoint{1.190777in}{1.839109in}}%
\pgfpathcurveto{\pgfqpoint{1.190777in}{1.830873in}}{\pgfqpoint{1.194050in}{1.822973in}}{\pgfqpoint{1.199874in}{1.817149in}}%
\pgfpathcurveto{\pgfqpoint{1.205697in}{1.811325in}}{\pgfqpoint{1.213597in}{1.808053in}}{\pgfqpoint{1.221834in}{1.808053in}}%
\pgfpathclose%
\pgfusepath{stroke,fill}%
\end{pgfscope}%
\begin{pgfscope}%
\pgfpathrectangle{\pgfqpoint{0.100000in}{0.212622in}}{\pgfqpoint{3.696000in}{3.696000in}}%
\pgfusepath{clip}%
\pgfsetbuttcap%
\pgfsetroundjoin%
\definecolor{currentfill}{rgb}{0.121569,0.466667,0.705882}%
\pgfsetfillcolor{currentfill}%
\pgfsetfillopacity{0.872104}%
\pgfsetlinewidth{1.003750pt}%
\definecolor{currentstroke}{rgb}{0.121569,0.466667,0.705882}%
\pgfsetstrokecolor{currentstroke}%
\pgfsetstrokeopacity{0.872104}%
\pgfsetdash{}{0pt}%
\pgfpathmoveto{\pgfqpoint{3.086891in}{2.637534in}}%
\pgfpathcurveto{\pgfqpoint{3.095127in}{2.637534in}}{\pgfqpoint{3.103027in}{2.640806in}}{\pgfqpoint{3.108851in}{2.646630in}}%
\pgfpathcurveto{\pgfqpoint{3.114675in}{2.652454in}}{\pgfqpoint{3.117947in}{2.660354in}}{\pgfqpoint{3.117947in}{2.668591in}}%
\pgfpathcurveto{\pgfqpoint{3.117947in}{2.676827in}}{\pgfqpoint{3.114675in}{2.684727in}}{\pgfqpoint{3.108851in}{2.690551in}}%
\pgfpathcurveto{\pgfqpoint{3.103027in}{2.696375in}}{\pgfqpoint{3.095127in}{2.699647in}}{\pgfqpoint{3.086891in}{2.699647in}}%
\pgfpathcurveto{\pgfqpoint{3.078655in}{2.699647in}}{\pgfqpoint{3.070754in}{2.696375in}}{\pgfqpoint{3.064931in}{2.690551in}}%
\pgfpathcurveto{\pgfqpoint{3.059107in}{2.684727in}}{\pgfqpoint{3.055834in}{2.676827in}}{\pgfqpoint{3.055834in}{2.668591in}}%
\pgfpathcurveto{\pgfqpoint{3.055834in}{2.660354in}}{\pgfqpoint{3.059107in}{2.652454in}}{\pgfqpoint{3.064931in}{2.646630in}}%
\pgfpathcurveto{\pgfqpoint{3.070754in}{2.640806in}}{\pgfqpoint{3.078655in}{2.637534in}}{\pgfqpoint{3.086891in}{2.637534in}}%
\pgfpathclose%
\pgfusepath{stroke,fill}%
\end{pgfscope}%
\begin{pgfscope}%
\pgfpathrectangle{\pgfqpoint{0.100000in}{0.212622in}}{\pgfqpoint{3.696000in}{3.696000in}}%
\pgfusepath{clip}%
\pgfsetbuttcap%
\pgfsetroundjoin%
\definecolor{currentfill}{rgb}{0.121569,0.466667,0.705882}%
\pgfsetfillcolor{currentfill}%
\pgfsetfillopacity{0.873190}%
\pgfsetlinewidth{1.003750pt}%
\definecolor{currentstroke}{rgb}{0.121569,0.466667,0.705882}%
\pgfsetstrokecolor{currentstroke}%
\pgfsetstrokeopacity{0.873190}%
\pgfsetdash{}{0pt}%
\pgfpathmoveto{\pgfqpoint{1.263868in}{1.814116in}}%
\pgfpathcurveto{\pgfqpoint{1.272104in}{1.814116in}}{\pgfqpoint{1.280004in}{1.817388in}}{\pgfqpoint{1.285828in}{1.823212in}}%
\pgfpathcurveto{\pgfqpoint{1.291652in}{1.829036in}}{\pgfqpoint{1.294924in}{1.836936in}}{\pgfqpoint{1.294924in}{1.845172in}}%
\pgfpathcurveto{\pgfqpoint{1.294924in}{1.853409in}}{\pgfqpoint{1.291652in}{1.861309in}}{\pgfqpoint{1.285828in}{1.867133in}}%
\pgfpathcurveto{\pgfqpoint{1.280004in}{1.872957in}}{\pgfqpoint{1.272104in}{1.876229in}}{\pgfqpoint{1.263868in}{1.876229in}}%
\pgfpathcurveto{\pgfqpoint{1.255632in}{1.876229in}}{\pgfqpoint{1.247732in}{1.872957in}}{\pgfqpoint{1.241908in}{1.867133in}}%
\pgfpathcurveto{\pgfqpoint{1.236084in}{1.861309in}}{\pgfqpoint{1.232811in}{1.853409in}}{\pgfqpoint{1.232811in}{1.845172in}}%
\pgfpathcurveto{\pgfqpoint{1.232811in}{1.836936in}}{\pgfqpoint{1.236084in}{1.829036in}}{\pgfqpoint{1.241908in}{1.823212in}}%
\pgfpathcurveto{\pgfqpoint{1.247732in}{1.817388in}}{\pgfqpoint{1.255632in}{1.814116in}}{\pgfqpoint{1.263868in}{1.814116in}}%
\pgfpathclose%
\pgfusepath{stroke,fill}%
\end{pgfscope}%
\begin{pgfscope}%
\pgfpathrectangle{\pgfqpoint{0.100000in}{0.212622in}}{\pgfqpoint{3.696000in}{3.696000in}}%
\pgfusepath{clip}%
\pgfsetbuttcap%
\pgfsetroundjoin%
\definecolor{currentfill}{rgb}{0.121569,0.466667,0.705882}%
\pgfsetfillcolor{currentfill}%
\pgfsetfillopacity{0.873469}%
\pgfsetlinewidth{1.003750pt}%
\definecolor{currentstroke}{rgb}{0.121569,0.466667,0.705882}%
\pgfsetstrokecolor{currentstroke}%
\pgfsetstrokeopacity{0.873469}%
\pgfsetdash{}{0pt}%
\pgfpathmoveto{\pgfqpoint{2.448931in}{2.365640in}}%
\pgfpathcurveto{\pgfqpoint{2.457167in}{2.365640in}}{\pgfqpoint{2.465067in}{2.368913in}}{\pgfqpoint{2.470891in}{2.374737in}}%
\pgfpathcurveto{\pgfqpoint{2.476715in}{2.380560in}}{\pgfqpoint{2.479987in}{2.388460in}}{\pgfqpoint{2.479987in}{2.396697in}}%
\pgfpathcurveto{\pgfqpoint{2.479987in}{2.404933in}}{\pgfqpoint{2.476715in}{2.412833in}}{\pgfqpoint{2.470891in}{2.418657in}}%
\pgfpathcurveto{\pgfqpoint{2.465067in}{2.424481in}}{\pgfqpoint{2.457167in}{2.427753in}}{\pgfqpoint{2.448931in}{2.427753in}}%
\pgfpathcurveto{\pgfqpoint{2.440695in}{2.427753in}}{\pgfqpoint{2.432795in}{2.424481in}}{\pgfqpoint{2.426971in}{2.418657in}}%
\pgfpathcurveto{\pgfqpoint{2.421147in}{2.412833in}}{\pgfqpoint{2.417874in}{2.404933in}}{\pgfqpoint{2.417874in}{2.396697in}}%
\pgfpathcurveto{\pgfqpoint{2.417874in}{2.388460in}}{\pgfqpoint{2.421147in}{2.380560in}}{\pgfqpoint{2.426971in}{2.374737in}}%
\pgfpathcurveto{\pgfqpoint{2.432795in}{2.368913in}}{\pgfqpoint{2.440695in}{2.365640in}}{\pgfqpoint{2.448931in}{2.365640in}}%
\pgfpathclose%
\pgfusepath{stroke,fill}%
\end{pgfscope}%
\begin{pgfscope}%
\pgfpathrectangle{\pgfqpoint{0.100000in}{0.212622in}}{\pgfqpoint{3.696000in}{3.696000in}}%
\pgfusepath{clip}%
\pgfsetbuttcap%
\pgfsetroundjoin%
\definecolor{currentfill}{rgb}{0.121569,0.466667,0.705882}%
\pgfsetfillcolor{currentfill}%
\pgfsetfillopacity{0.873893}%
\pgfsetlinewidth{1.003750pt}%
\definecolor{currentstroke}{rgb}{0.121569,0.466667,0.705882}%
\pgfsetstrokecolor{currentstroke}%
\pgfsetstrokeopacity{0.873893}%
\pgfsetdash{}{0pt}%
\pgfpathmoveto{\pgfqpoint{1.238128in}{1.814465in}}%
\pgfpathcurveto{\pgfqpoint{1.246364in}{1.814465in}}{\pgfqpoint{1.254264in}{1.817738in}}{\pgfqpoint{1.260088in}{1.823562in}}%
\pgfpathcurveto{\pgfqpoint{1.265912in}{1.829385in}}{\pgfqpoint{1.269184in}{1.837286in}}{\pgfqpoint{1.269184in}{1.845522in}}%
\pgfpathcurveto{\pgfqpoint{1.269184in}{1.853758in}}{\pgfqpoint{1.265912in}{1.861658in}}{\pgfqpoint{1.260088in}{1.867482in}}%
\pgfpathcurveto{\pgfqpoint{1.254264in}{1.873306in}}{\pgfqpoint{1.246364in}{1.876578in}}{\pgfqpoint{1.238128in}{1.876578in}}%
\pgfpathcurveto{\pgfqpoint{1.229892in}{1.876578in}}{\pgfqpoint{1.221991in}{1.873306in}}{\pgfqpoint{1.216168in}{1.867482in}}%
\pgfpathcurveto{\pgfqpoint{1.210344in}{1.861658in}}{\pgfqpoint{1.207071in}{1.853758in}}{\pgfqpoint{1.207071in}{1.845522in}}%
\pgfpathcurveto{\pgfqpoint{1.207071in}{1.837286in}}{\pgfqpoint{1.210344in}{1.829385in}}{\pgfqpoint{1.216168in}{1.823562in}}%
\pgfpathcurveto{\pgfqpoint{1.221991in}{1.817738in}}{\pgfqpoint{1.229892in}{1.814465in}}{\pgfqpoint{1.238128in}{1.814465in}}%
\pgfpathclose%
\pgfusepath{stroke,fill}%
\end{pgfscope}%
\begin{pgfscope}%
\pgfpathrectangle{\pgfqpoint{0.100000in}{0.212622in}}{\pgfqpoint{3.696000in}{3.696000in}}%
\pgfusepath{clip}%
\pgfsetbuttcap%
\pgfsetroundjoin%
\definecolor{currentfill}{rgb}{0.121569,0.466667,0.705882}%
\pgfsetfillcolor{currentfill}%
\pgfsetfillopacity{0.874201}%
\pgfsetlinewidth{1.003750pt}%
\definecolor{currentstroke}{rgb}{0.121569,0.466667,0.705882}%
\pgfsetstrokecolor{currentstroke}%
\pgfsetstrokeopacity{0.874201}%
\pgfsetdash{}{0pt}%
\pgfpathmoveto{\pgfqpoint{2.421323in}{2.346409in}}%
\pgfpathcurveto{\pgfqpoint{2.429559in}{2.346409in}}{\pgfqpoint{2.437459in}{2.349681in}}{\pgfqpoint{2.443283in}{2.355505in}}%
\pgfpathcurveto{\pgfqpoint{2.449107in}{2.361329in}}{\pgfqpoint{2.452379in}{2.369229in}}{\pgfqpoint{2.452379in}{2.377465in}}%
\pgfpathcurveto{\pgfqpoint{2.452379in}{2.385702in}}{\pgfqpoint{2.449107in}{2.393602in}}{\pgfqpoint{2.443283in}{2.399426in}}%
\pgfpathcurveto{\pgfqpoint{2.437459in}{2.405250in}}{\pgfqpoint{2.429559in}{2.408522in}}{\pgfqpoint{2.421323in}{2.408522in}}%
\pgfpathcurveto{\pgfqpoint{2.413086in}{2.408522in}}{\pgfqpoint{2.405186in}{2.405250in}}{\pgfqpoint{2.399362in}{2.399426in}}%
\pgfpathcurveto{\pgfqpoint{2.393539in}{2.393602in}}{\pgfqpoint{2.390266in}{2.385702in}}{\pgfqpoint{2.390266in}{2.377465in}}%
\pgfpathcurveto{\pgfqpoint{2.390266in}{2.369229in}}{\pgfqpoint{2.393539in}{2.361329in}}{\pgfqpoint{2.399362in}{2.355505in}}%
\pgfpathcurveto{\pgfqpoint{2.405186in}{2.349681in}}{\pgfqpoint{2.413086in}{2.346409in}}{\pgfqpoint{2.421323in}{2.346409in}}%
\pgfpathclose%
\pgfusepath{stroke,fill}%
\end{pgfscope}%
\begin{pgfscope}%
\pgfpathrectangle{\pgfqpoint{0.100000in}{0.212622in}}{\pgfqpoint{3.696000in}{3.696000in}}%
\pgfusepath{clip}%
\pgfsetbuttcap%
\pgfsetroundjoin%
\definecolor{currentfill}{rgb}{0.121569,0.466667,0.705882}%
\pgfsetfillcolor{currentfill}%
\pgfsetfillopacity{0.874396}%
\pgfsetlinewidth{1.003750pt}%
\definecolor{currentstroke}{rgb}{0.121569,0.466667,0.705882}%
\pgfsetstrokecolor{currentstroke}%
\pgfsetstrokeopacity{0.874396}%
\pgfsetdash{}{0pt}%
\pgfpathmoveto{\pgfqpoint{1.234299in}{1.811438in}}%
\pgfpathcurveto{\pgfqpoint{1.242535in}{1.811438in}}{\pgfqpoint{1.250435in}{1.814711in}}{\pgfqpoint{1.256259in}{1.820534in}}%
\pgfpathcurveto{\pgfqpoint{1.262083in}{1.826358in}}{\pgfqpoint{1.265356in}{1.834258in}}{\pgfqpoint{1.265356in}{1.842495in}}%
\pgfpathcurveto{\pgfqpoint{1.265356in}{1.850731in}}{\pgfqpoint{1.262083in}{1.858631in}}{\pgfqpoint{1.256259in}{1.864455in}}%
\pgfpathcurveto{\pgfqpoint{1.250435in}{1.870279in}}{\pgfqpoint{1.242535in}{1.873551in}}{\pgfqpoint{1.234299in}{1.873551in}}%
\pgfpathcurveto{\pgfqpoint{1.226063in}{1.873551in}}{\pgfqpoint{1.218163in}{1.870279in}}{\pgfqpoint{1.212339in}{1.864455in}}%
\pgfpathcurveto{\pgfqpoint{1.206515in}{1.858631in}}{\pgfqpoint{1.203243in}{1.850731in}}{\pgfqpoint{1.203243in}{1.842495in}}%
\pgfpathcurveto{\pgfqpoint{1.203243in}{1.834258in}}{\pgfqpoint{1.206515in}{1.826358in}}{\pgfqpoint{1.212339in}{1.820534in}}%
\pgfpathcurveto{\pgfqpoint{1.218163in}{1.814711in}}{\pgfqpoint{1.226063in}{1.811438in}}{\pgfqpoint{1.234299in}{1.811438in}}%
\pgfpathclose%
\pgfusepath{stroke,fill}%
\end{pgfscope}%
\begin{pgfscope}%
\pgfpathrectangle{\pgfqpoint{0.100000in}{0.212622in}}{\pgfqpoint{3.696000in}{3.696000in}}%
\pgfusepath{clip}%
\pgfsetbuttcap%
\pgfsetroundjoin%
\definecolor{currentfill}{rgb}{0.121569,0.466667,0.705882}%
\pgfsetfillcolor{currentfill}%
\pgfsetfillopacity{0.874412}%
\pgfsetlinewidth{1.003750pt}%
\definecolor{currentstroke}{rgb}{0.121569,0.466667,0.705882}%
\pgfsetstrokecolor{currentstroke}%
\pgfsetstrokeopacity{0.874412}%
\pgfsetdash{}{0pt}%
\pgfpathmoveto{\pgfqpoint{1.223589in}{1.810872in}}%
\pgfpathcurveto{\pgfqpoint{1.231825in}{1.810872in}}{\pgfqpoint{1.239725in}{1.814145in}}{\pgfqpoint{1.245549in}{1.819969in}}%
\pgfpathcurveto{\pgfqpoint{1.251373in}{1.825793in}}{\pgfqpoint{1.254645in}{1.833693in}}{\pgfqpoint{1.254645in}{1.841929in}}%
\pgfpathcurveto{\pgfqpoint{1.254645in}{1.850165in}}{\pgfqpoint{1.251373in}{1.858065in}}{\pgfqpoint{1.245549in}{1.863889in}}%
\pgfpathcurveto{\pgfqpoint{1.239725in}{1.869713in}}{\pgfqpoint{1.231825in}{1.872985in}}{\pgfqpoint{1.223589in}{1.872985in}}%
\pgfpathcurveto{\pgfqpoint{1.215352in}{1.872985in}}{\pgfqpoint{1.207452in}{1.869713in}}{\pgfqpoint{1.201628in}{1.863889in}}%
\pgfpathcurveto{\pgfqpoint{1.195805in}{1.858065in}}{\pgfqpoint{1.192532in}{1.850165in}}{\pgfqpoint{1.192532in}{1.841929in}}%
\pgfpathcurveto{\pgfqpoint{1.192532in}{1.833693in}}{\pgfqpoint{1.195805in}{1.825793in}}{\pgfqpoint{1.201628in}{1.819969in}}%
\pgfpathcurveto{\pgfqpoint{1.207452in}{1.814145in}}{\pgfqpoint{1.215352in}{1.810872in}}{\pgfqpoint{1.223589in}{1.810872in}}%
\pgfpathclose%
\pgfusepath{stroke,fill}%
\end{pgfscope}%
\begin{pgfscope}%
\pgfpathrectangle{\pgfqpoint{0.100000in}{0.212622in}}{\pgfqpoint{3.696000in}{3.696000in}}%
\pgfusepath{clip}%
\pgfsetbuttcap%
\pgfsetroundjoin%
\definecolor{currentfill}{rgb}{0.121569,0.466667,0.705882}%
\pgfsetfillcolor{currentfill}%
\pgfsetfillopacity{0.874553}%
\pgfsetlinewidth{1.003750pt}%
\definecolor{currentstroke}{rgb}{0.121569,0.466667,0.705882}%
\pgfsetstrokecolor{currentstroke}%
\pgfsetstrokeopacity{0.874553}%
\pgfsetdash{}{0pt}%
\pgfpathmoveto{\pgfqpoint{1.265805in}{1.806261in}}%
\pgfpathcurveto{\pgfqpoint{1.274042in}{1.806261in}}{\pgfqpoint{1.281942in}{1.809534in}}{\pgfqpoint{1.287766in}{1.815357in}}%
\pgfpathcurveto{\pgfqpoint{1.293590in}{1.821181in}}{\pgfqpoint{1.296862in}{1.829081in}}{\pgfqpoint{1.296862in}{1.837318in}}%
\pgfpathcurveto{\pgfqpoint{1.296862in}{1.845554in}}{\pgfqpoint{1.293590in}{1.853454in}}{\pgfqpoint{1.287766in}{1.859278in}}%
\pgfpathcurveto{\pgfqpoint{1.281942in}{1.865102in}}{\pgfqpoint{1.274042in}{1.868374in}}{\pgfqpoint{1.265805in}{1.868374in}}%
\pgfpathcurveto{\pgfqpoint{1.257569in}{1.868374in}}{\pgfqpoint{1.249669in}{1.865102in}}{\pgfqpoint{1.243845in}{1.859278in}}%
\pgfpathcurveto{\pgfqpoint{1.238021in}{1.853454in}}{\pgfqpoint{1.234749in}{1.845554in}}{\pgfqpoint{1.234749in}{1.837318in}}%
\pgfpathcurveto{\pgfqpoint{1.234749in}{1.829081in}}{\pgfqpoint{1.238021in}{1.821181in}}{\pgfqpoint{1.243845in}{1.815357in}}%
\pgfpathcurveto{\pgfqpoint{1.249669in}{1.809534in}}{\pgfqpoint{1.257569in}{1.806261in}}{\pgfqpoint{1.265805in}{1.806261in}}%
\pgfpathclose%
\pgfusepath{stroke,fill}%
\end{pgfscope}%
\begin{pgfscope}%
\pgfpathrectangle{\pgfqpoint{0.100000in}{0.212622in}}{\pgfqpoint{3.696000in}{3.696000in}}%
\pgfusepath{clip}%
\pgfsetbuttcap%
\pgfsetroundjoin%
\definecolor{currentfill}{rgb}{0.121569,0.466667,0.705882}%
\pgfsetfillcolor{currentfill}%
\pgfsetfillopacity{0.874731}%
\pgfsetlinewidth{1.003750pt}%
\definecolor{currentstroke}{rgb}{0.121569,0.466667,0.705882}%
\pgfsetstrokecolor{currentstroke}%
\pgfsetstrokeopacity{0.874731}%
\pgfsetdash{}{0pt}%
\pgfpathmoveto{\pgfqpoint{2.747857in}{2.512696in}}%
\pgfpathcurveto{\pgfqpoint{2.756093in}{2.512696in}}{\pgfqpoint{2.763994in}{2.515968in}}{\pgfqpoint{2.769817in}{2.521792in}}%
\pgfpathcurveto{\pgfqpoint{2.775641in}{2.527616in}}{\pgfqpoint{2.778914in}{2.535516in}}{\pgfqpoint{2.778914in}{2.543752in}}%
\pgfpathcurveto{\pgfqpoint{2.778914in}{2.551989in}}{\pgfqpoint{2.775641in}{2.559889in}}{\pgfqpoint{2.769817in}{2.565713in}}%
\pgfpathcurveto{\pgfqpoint{2.763994in}{2.571537in}}{\pgfqpoint{2.756093in}{2.574809in}}{\pgfqpoint{2.747857in}{2.574809in}}%
\pgfpathcurveto{\pgfqpoint{2.739621in}{2.574809in}}{\pgfqpoint{2.731721in}{2.571537in}}{\pgfqpoint{2.725897in}{2.565713in}}%
\pgfpathcurveto{\pgfqpoint{2.720073in}{2.559889in}}{\pgfqpoint{2.716801in}{2.551989in}}{\pgfqpoint{2.716801in}{2.543752in}}%
\pgfpathcurveto{\pgfqpoint{2.716801in}{2.535516in}}{\pgfqpoint{2.720073in}{2.527616in}}{\pgfqpoint{2.725897in}{2.521792in}}%
\pgfpathcurveto{\pgfqpoint{2.731721in}{2.515968in}}{\pgfqpoint{2.739621in}{2.512696in}}{\pgfqpoint{2.747857in}{2.512696in}}%
\pgfpathclose%
\pgfusepath{stroke,fill}%
\end{pgfscope}%
\begin{pgfscope}%
\pgfpathrectangle{\pgfqpoint{0.100000in}{0.212622in}}{\pgfqpoint{3.696000in}{3.696000in}}%
\pgfusepath{clip}%
\pgfsetbuttcap%
\pgfsetroundjoin%
\definecolor{currentfill}{rgb}{0.121569,0.466667,0.705882}%
\pgfsetfillcolor{currentfill}%
\pgfsetfillopacity{0.874744}%
\pgfsetlinewidth{1.003750pt}%
\definecolor{currentstroke}{rgb}{0.121569,0.466667,0.705882}%
\pgfsetstrokecolor{currentstroke}%
\pgfsetstrokeopacity{0.874744}%
\pgfsetdash{}{0pt}%
\pgfpathmoveto{\pgfqpoint{2.502540in}{2.392274in}}%
\pgfpathcurveto{\pgfqpoint{2.510776in}{2.392274in}}{\pgfqpoint{2.518676in}{2.395547in}}{\pgfqpoint{2.524500in}{2.401371in}}%
\pgfpathcurveto{\pgfqpoint{2.530324in}{2.407195in}}{\pgfqpoint{2.533596in}{2.415095in}}{\pgfqpoint{2.533596in}{2.423331in}}%
\pgfpathcurveto{\pgfqpoint{2.533596in}{2.431567in}}{\pgfqpoint{2.530324in}{2.439467in}}{\pgfqpoint{2.524500in}{2.445291in}}%
\pgfpathcurveto{\pgfqpoint{2.518676in}{2.451115in}}{\pgfqpoint{2.510776in}{2.454387in}}{\pgfqpoint{2.502540in}{2.454387in}}%
\pgfpathcurveto{\pgfqpoint{2.494303in}{2.454387in}}{\pgfqpoint{2.486403in}{2.451115in}}{\pgfqpoint{2.480579in}{2.445291in}}%
\pgfpathcurveto{\pgfqpoint{2.474755in}{2.439467in}}{\pgfqpoint{2.471483in}{2.431567in}}{\pgfqpoint{2.471483in}{2.423331in}}%
\pgfpathcurveto{\pgfqpoint{2.471483in}{2.415095in}}{\pgfqpoint{2.474755in}{2.407195in}}{\pgfqpoint{2.480579in}{2.401371in}}%
\pgfpathcurveto{\pgfqpoint{2.486403in}{2.395547in}}{\pgfqpoint{2.494303in}{2.392274in}}{\pgfqpoint{2.502540in}{2.392274in}}%
\pgfpathclose%
\pgfusepath{stroke,fill}%
\end{pgfscope}%
\begin{pgfscope}%
\pgfpathrectangle{\pgfqpoint{0.100000in}{0.212622in}}{\pgfqpoint{3.696000in}{3.696000in}}%
\pgfusepath{clip}%
\pgfsetbuttcap%
\pgfsetroundjoin%
\definecolor{currentfill}{rgb}{0.121569,0.466667,0.705882}%
\pgfsetfillcolor{currentfill}%
\pgfsetfillopacity{0.875461}%
\pgfsetlinewidth{1.003750pt}%
\definecolor{currentstroke}{rgb}{0.121569,0.466667,0.705882}%
\pgfsetstrokecolor{currentstroke}%
\pgfsetstrokeopacity{0.875461}%
\pgfsetdash{}{0pt}%
\pgfpathmoveto{\pgfqpoint{1.236241in}{1.801072in}}%
\pgfpathcurveto{\pgfqpoint{1.244478in}{1.801072in}}{\pgfqpoint{1.252378in}{1.804344in}}{\pgfqpoint{1.258202in}{1.810168in}}%
\pgfpathcurveto{\pgfqpoint{1.264026in}{1.815992in}}{\pgfqpoint{1.267298in}{1.823892in}}{\pgfqpoint{1.267298in}{1.832128in}}%
\pgfpathcurveto{\pgfqpoint{1.267298in}{1.840364in}}{\pgfqpoint{1.264026in}{1.848264in}}{\pgfqpoint{1.258202in}{1.854088in}}%
\pgfpathcurveto{\pgfqpoint{1.252378in}{1.859912in}}{\pgfqpoint{1.244478in}{1.863185in}}{\pgfqpoint{1.236241in}{1.863185in}}%
\pgfpathcurveto{\pgfqpoint{1.228005in}{1.863185in}}{\pgfqpoint{1.220105in}{1.859912in}}{\pgfqpoint{1.214281in}{1.854088in}}%
\pgfpathcurveto{\pgfqpoint{1.208457in}{1.848264in}}{\pgfqpoint{1.205185in}{1.840364in}}{\pgfqpoint{1.205185in}{1.832128in}}%
\pgfpathcurveto{\pgfqpoint{1.205185in}{1.823892in}}{\pgfqpoint{1.208457in}{1.815992in}}{\pgfqpoint{1.214281in}{1.810168in}}%
\pgfpathcurveto{\pgfqpoint{1.220105in}{1.804344in}}{\pgfqpoint{1.228005in}{1.801072in}}{\pgfqpoint{1.236241in}{1.801072in}}%
\pgfpathclose%
\pgfusepath{stroke,fill}%
\end{pgfscope}%
\begin{pgfscope}%
\pgfpathrectangle{\pgfqpoint{0.100000in}{0.212622in}}{\pgfqpoint{3.696000in}{3.696000in}}%
\pgfusepath{clip}%
\pgfsetbuttcap%
\pgfsetroundjoin%
\definecolor{currentfill}{rgb}{0.121569,0.466667,0.705882}%
\pgfsetfillcolor{currentfill}%
\pgfsetfillopacity{0.875980}%
\pgfsetlinewidth{1.003750pt}%
\definecolor{currentstroke}{rgb}{0.121569,0.466667,0.705882}%
\pgfsetstrokecolor{currentstroke}%
\pgfsetstrokeopacity{0.875980}%
\pgfsetdash{}{0pt}%
\pgfpathmoveto{\pgfqpoint{2.408709in}{2.342960in}}%
\pgfpathcurveto{\pgfqpoint{2.416945in}{2.342960in}}{\pgfqpoint{2.424845in}{2.346232in}}{\pgfqpoint{2.430669in}{2.352056in}}%
\pgfpathcurveto{\pgfqpoint{2.436493in}{2.357880in}}{\pgfqpoint{2.439765in}{2.365780in}}{\pgfqpoint{2.439765in}{2.374016in}}%
\pgfpathcurveto{\pgfqpoint{2.439765in}{2.382252in}}{\pgfqpoint{2.436493in}{2.390152in}}{\pgfqpoint{2.430669in}{2.395976in}}%
\pgfpathcurveto{\pgfqpoint{2.424845in}{2.401800in}}{\pgfqpoint{2.416945in}{2.405073in}}{\pgfqpoint{2.408709in}{2.405073in}}%
\pgfpathcurveto{\pgfqpoint{2.400473in}{2.405073in}}{\pgfqpoint{2.392573in}{2.401800in}}{\pgfqpoint{2.386749in}{2.395976in}}%
\pgfpathcurveto{\pgfqpoint{2.380925in}{2.390152in}}{\pgfqpoint{2.377652in}{2.382252in}}{\pgfqpoint{2.377652in}{2.374016in}}%
\pgfpathcurveto{\pgfqpoint{2.377652in}{2.365780in}}{\pgfqpoint{2.380925in}{2.357880in}}{\pgfqpoint{2.386749in}{2.352056in}}%
\pgfpathcurveto{\pgfqpoint{2.392573in}{2.346232in}}{\pgfqpoint{2.400473in}{2.342960in}}{\pgfqpoint{2.408709in}{2.342960in}}%
\pgfpathclose%
\pgfusepath{stroke,fill}%
\end{pgfscope}%
\begin{pgfscope}%
\pgfpathrectangle{\pgfqpoint{0.100000in}{0.212622in}}{\pgfqpoint{3.696000in}{3.696000in}}%
\pgfusepath{clip}%
\pgfsetbuttcap%
\pgfsetroundjoin%
\definecolor{currentfill}{rgb}{0.121569,0.466667,0.705882}%
\pgfsetfillcolor{currentfill}%
\pgfsetfillopacity{0.877216}%
\pgfsetlinewidth{1.003750pt}%
\definecolor{currentstroke}{rgb}{0.121569,0.466667,0.705882}%
\pgfsetstrokecolor{currentstroke}%
\pgfsetstrokeopacity{0.877216}%
\pgfsetdash{}{0pt}%
\pgfpathmoveto{\pgfqpoint{2.471797in}{2.364852in}}%
\pgfpathcurveto{\pgfqpoint{2.480033in}{2.364852in}}{\pgfqpoint{2.487933in}{2.368124in}}{\pgfqpoint{2.493757in}{2.373948in}}%
\pgfpathcurveto{\pgfqpoint{2.499581in}{2.379772in}}{\pgfqpoint{2.502854in}{2.387672in}}{\pgfqpoint{2.502854in}{2.395908in}}%
\pgfpathcurveto{\pgfqpoint{2.502854in}{2.404145in}}{\pgfqpoint{2.499581in}{2.412045in}}{\pgfqpoint{2.493757in}{2.417869in}}%
\pgfpathcurveto{\pgfqpoint{2.487933in}{2.423692in}}{\pgfqpoint{2.480033in}{2.426965in}}{\pgfqpoint{2.471797in}{2.426965in}}%
\pgfpathcurveto{\pgfqpoint{2.463561in}{2.426965in}}{\pgfqpoint{2.455661in}{2.423692in}}{\pgfqpoint{2.449837in}{2.417869in}}%
\pgfpathcurveto{\pgfqpoint{2.444013in}{2.412045in}}{\pgfqpoint{2.440741in}{2.404145in}}{\pgfqpoint{2.440741in}{2.395908in}}%
\pgfpathcurveto{\pgfqpoint{2.440741in}{2.387672in}}{\pgfqpoint{2.444013in}{2.379772in}}{\pgfqpoint{2.449837in}{2.373948in}}%
\pgfpathcurveto{\pgfqpoint{2.455661in}{2.368124in}}{\pgfqpoint{2.463561in}{2.364852in}}{\pgfqpoint{2.471797in}{2.364852in}}%
\pgfpathclose%
\pgfusepath{stroke,fill}%
\end{pgfscope}%
\begin{pgfscope}%
\pgfpathrectangle{\pgfqpoint{0.100000in}{0.212622in}}{\pgfqpoint{3.696000in}{3.696000in}}%
\pgfusepath{clip}%
\pgfsetbuttcap%
\pgfsetroundjoin%
\definecolor{currentfill}{rgb}{0.121569,0.466667,0.705882}%
\pgfsetfillcolor{currentfill}%
\pgfsetfillopacity{0.877380}%
\pgfsetlinewidth{1.003750pt}%
\definecolor{currentstroke}{rgb}{0.121569,0.466667,0.705882}%
\pgfsetstrokecolor{currentstroke}%
\pgfsetstrokeopacity{0.877380}%
\pgfsetdash{}{0pt}%
\pgfpathmoveto{\pgfqpoint{3.080219in}{2.629762in}}%
\pgfpathcurveto{\pgfqpoint{3.088455in}{2.629762in}}{\pgfqpoint{3.096355in}{2.633034in}}{\pgfqpoint{3.102179in}{2.638858in}}%
\pgfpathcurveto{\pgfqpoint{3.108003in}{2.644682in}}{\pgfqpoint{3.111275in}{2.652582in}}{\pgfqpoint{3.111275in}{2.660818in}}%
\pgfpathcurveto{\pgfqpoint{3.111275in}{2.669054in}}{\pgfqpoint{3.108003in}{2.676954in}}{\pgfqpoint{3.102179in}{2.682778in}}%
\pgfpathcurveto{\pgfqpoint{3.096355in}{2.688602in}}{\pgfqpoint{3.088455in}{2.691875in}}{\pgfqpoint{3.080219in}{2.691875in}}%
\pgfpathcurveto{\pgfqpoint{3.071983in}{2.691875in}}{\pgfqpoint{3.064083in}{2.688602in}}{\pgfqpoint{3.058259in}{2.682778in}}%
\pgfpathcurveto{\pgfqpoint{3.052435in}{2.676954in}}{\pgfqpoint{3.049162in}{2.669054in}}{\pgfqpoint{3.049162in}{2.660818in}}%
\pgfpathcurveto{\pgfqpoint{3.049162in}{2.652582in}}{\pgfqpoint{3.052435in}{2.644682in}}{\pgfqpoint{3.058259in}{2.638858in}}%
\pgfpathcurveto{\pgfqpoint{3.064083in}{2.633034in}}{\pgfqpoint{3.071983in}{2.629762in}}{\pgfqpoint{3.080219in}{2.629762in}}%
\pgfpathclose%
\pgfusepath{stroke,fill}%
\end{pgfscope}%
\begin{pgfscope}%
\pgfpathrectangle{\pgfqpoint{0.100000in}{0.212622in}}{\pgfqpoint{3.696000in}{3.696000in}}%
\pgfusepath{clip}%
\pgfsetbuttcap%
\pgfsetroundjoin%
\definecolor{currentfill}{rgb}{0.121569,0.466667,0.705882}%
\pgfsetfillcolor{currentfill}%
\pgfsetfillopacity{0.877441}%
\pgfsetlinewidth{1.003750pt}%
\definecolor{currentstroke}{rgb}{0.121569,0.466667,0.705882}%
\pgfsetstrokecolor{currentstroke}%
\pgfsetstrokeopacity{0.877441}%
\pgfsetdash{}{0pt}%
\pgfpathmoveto{\pgfqpoint{2.428201in}{2.352463in}}%
\pgfpathcurveto{\pgfqpoint{2.436437in}{2.352463in}}{\pgfqpoint{2.444337in}{2.355736in}}{\pgfqpoint{2.450161in}{2.361560in}}%
\pgfpathcurveto{\pgfqpoint{2.455985in}{2.367384in}}{\pgfqpoint{2.459258in}{2.375284in}}{\pgfqpoint{2.459258in}{2.383520in}}%
\pgfpathcurveto{\pgfqpoint{2.459258in}{2.391756in}}{\pgfqpoint{2.455985in}{2.399656in}}{\pgfqpoint{2.450161in}{2.405480in}}%
\pgfpathcurveto{\pgfqpoint{2.444337in}{2.411304in}}{\pgfqpoint{2.436437in}{2.414576in}}{\pgfqpoint{2.428201in}{2.414576in}}%
\pgfpathcurveto{\pgfqpoint{2.419965in}{2.414576in}}{\pgfqpoint{2.412065in}{2.411304in}}{\pgfqpoint{2.406241in}{2.405480in}}%
\pgfpathcurveto{\pgfqpoint{2.400417in}{2.399656in}}{\pgfqpoint{2.397145in}{2.391756in}}{\pgfqpoint{2.397145in}{2.383520in}}%
\pgfpathcurveto{\pgfqpoint{2.397145in}{2.375284in}}{\pgfqpoint{2.400417in}{2.367384in}}{\pgfqpoint{2.406241in}{2.361560in}}%
\pgfpathcurveto{\pgfqpoint{2.412065in}{2.355736in}}{\pgfqpoint{2.419965in}{2.352463in}}{\pgfqpoint{2.428201in}{2.352463in}}%
\pgfpathclose%
\pgfusepath{stroke,fill}%
\end{pgfscope}%
\begin{pgfscope}%
\pgfpathrectangle{\pgfqpoint{0.100000in}{0.212622in}}{\pgfqpoint{3.696000in}{3.696000in}}%
\pgfusepath{clip}%
\pgfsetbuttcap%
\pgfsetroundjoin%
\definecolor{currentfill}{rgb}{0.121569,0.466667,0.705882}%
\pgfsetfillcolor{currentfill}%
\pgfsetfillopacity{0.877764}%
\pgfsetlinewidth{1.003750pt}%
\definecolor{currentstroke}{rgb}{0.121569,0.466667,0.705882}%
\pgfsetstrokecolor{currentstroke}%
\pgfsetstrokeopacity{0.877764}%
\pgfsetdash{}{0pt}%
\pgfpathmoveto{\pgfqpoint{1.221672in}{1.801476in}}%
\pgfpathcurveto{\pgfqpoint{1.229908in}{1.801476in}}{\pgfqpoint{1.237808in}{1.804749in}}{\pgfqpoint{1.243632in}{1.810573in}}%
\pgfpathcurveto{\pgfqpoint{1.249456in}{1.816396in}}{\pgfqpoint{1.252728in}{1.824296in}}{\pgfqpoint{1.252728in}{1.832533in}}%
\pgfpathcurveto{\pgfqpoint{1.252728in}{1.840769in}}{\pgfqpoint{1.249456in}{1.848669in}}{\pgfqpoint{1.243632in}{1.854493in}}%
\pgfpathcurveto{\pgfqpoint{1.237808in}{1.860317in}}{\pgfqpoint{1.229908in}{1.863589in}}{\pgfqpoint{1.221672in}{1.863589in}}%
\pgfpathcurveto{\pgfqpoint{1.213435in}{1.863589in}}{\pgfqpoint{1.205535in}{1.860317in}}{\pgfqpoint{1.199711in}{1.854493in}}%
\pgfpathcurveto{\pgfqpoint{1.193888in}{1.848669in}}{\pgfqpoint{1.190615in}{1.840769in}}{\pgfqpoint{1.190615in}{1.832533in}}%
\pgfpathcurveto{\pgfqpoint{1.190615in}{1.824296in}}{\pgfqpoint{1.193888in}{1.816396in}}{\pgfqpoint{1.199711in}{1.810573in}}%
\pgfpathcurveto{\pgfqpoint{1.205535in}{1.804749in}}{\pgfqpoint{1.213435in}{1.801476in}}{\pgfqpoint{1.221672in}{1.801476in}}%
\pgfpathclose%
\pgfusepath{stroke,fill}%
\end{pgfscope}%
\begin{pgfscope}%
\pgfpathrectangle{\pgfqpoint{0.100000in}{0.212622in}}{\pgfqpoint{3.696000in}{3.696000in}}%
\pgfusepath{clip}%
\pgfsetbuttcap%
\pgfsetroundjoin%
\definecolor{currentfill}{rgb}{0.121569,0.466667,0.705882}%
\pgfsetfillcolor{currentfill}%
\pgfsetfillopacity{0.878368}%
\pgfsetlinewidth{1.003750pt}%
\definecolor{currentstroke}{rgb}{0.121569,0.466667,0.705882}%
\pgfsetstrokecolor{currentstroke}%
\pgfsetstrokeopacity{0.878368}%
\pgfsetdash{}{0pt}%
\pgfpathmoveto{\pgfqpoint{2.453779in}{2.361696in}}%
\pgfpathcurveto{\pgfqpoint{2.462015in}{2.361696in}}{\pgfqpoint{2.469915in}{2.364969in}}{\pgfqpoint{2.475739in}{2.370793in}}%
\pgfpathcurveto{\pgfqpoint{2.481563in}{2.376617in}}{\pgfqpoint{2.484836in}{2.384517in}}{\pgfqpoint{2.484836in}{2.392753in}}%
\pgfpathcurveto{\pgfqpoint{2.484836in}{2.400989in}}{\pgfqpoint{2.481563in}{2.408889in}}{\pgfqpoint{2.475739in}{2.414713in}}%
\pgfpathcurveto{\pgfqpoint{2.469915in}{2.420537in}}{\pgfqpoint{2.462015in}{2.423809in}}{\pgfqpoint{2.453779in}{2.423809in}}%
\pgfpathcurveto{\pgfqpoint{2.445543in}{2.423809in}}{\pgfqpoint{2.437643in}{2.420537in}}{\pgfqpoint{2.431819in}{2.414713in}}%
\pgfpathcurveto{\pgfqpoint{2.425995in}{2.408889in}}{\pgfqpoint{2.422723in}{2.400989in}}{\pgfqpoint{2.422723in}{2.392753in}}%
\pgfpathcurveto{\pgfqpoint{2.422723in}{2.384517in}}{\pgfqpoint{2.425995in}{2.376617in}}{\pgfqpoint{2.431819in}{2.370793in}}%
\pgfpathcurveto{\pgfqpoint{2.437643in}{2.364969in}}{\pgfqpoint{2.445543in}{2.361696in}}{\pgfqpoint{2.453779in}{2.361696in}}%
\pgfpathclose%
\pgfusepath{stroke,fill}%
\end{pgfscope}%
\begin{pgfscope}%
\pgfpathrectangle{\pgfqpoint{0.100000in}{0.212622in}}{\pgfqpoint{3.696000in}{3.696000in}}%
\pgfusepath{clip}%
\pgfsetbuttcap%
\pgfsetroundjoin%
\definecolor{currentfill}{rgb}{0.121569,0.466667,0.705882}%
\pgfsetfillcolor{currentfill}%
\pgfsetfillopacity{0.878846}%
\pgfsetlinewidth{1.003750pt}%
\definecolor{currentstroke}{rgb}{0.121569,0.466667,0.705882}%
\pgfsetstrokecolor{currentstroke}%
\pgfsetstrokeopacity{0.878846}%
\pgfsetdash{}{0pt}%
\pgfpathmoveto{\pgfqpoint{2.431144in}{2.346923in}}%
\pgfpathcurveto{\pgfqpoint{2.439381in}{2.346923in}}{\pgfqpoint{2.447281in}{2.350196in}}{\pgfqpoint{2.453105in}{2.356020in}}%
\pgfpathcurveto{\pgfqpoint{2.458929in}{2.361844in}}{\pgfqpoint{2.462201in}{2.369744in}}{\pgfqpoint{2.462201in}{2.377980in}}%
\pgfpathcurveto{\pgfqpoint{2.462201in}{2.386216in}}{\pgfqpoint{2.458929in}{2.394116in}}{\pgfqpoint{2.453105in}{2.399940in}}%
\pgfpathcurveto{\pgfqpoint{2.447281in}{2.405764in}}{\pgfqpoint{2.439381in}{2.409036in}}{\pgfqpoint{2.431144in}{2.409036in}}%
\pgfpathcurveto{\pgfqpoint{2.422908in}{2.409036in}}{\pgfqpoint{2.415008in}{2.405764in}}{\pgfqpoint{2.409184in}{2.399940in}}%
\pgfpathcurveto{\pgfqpoint{2.403360in}{2.394116in}}{\pgfqpoint{2.400088in}{2.386216in}}{\pgfqpoint{2.400088in}{2.377980in}}%
\pgfpathcurveto{\pgfqpoint{2.400088in}{2.369744in}}{\pgfqpoint{2.403360in}{2.361844in}}{\pgfqpoint{2.409184in}{2.356020in}}%
\pgfpathcurveto{\pgfqpoint{2.415008in}{2.350196in}}{\pgfqpoint{2.422908in}{2.346923in}}{\pgfqpoint{2.431144in}{2.346923in}}%
\pgfpathclose%
\pgfusepath{stroke,fill}%
\end{pgfscope}%
\begin{pgfscope}%
\pgfpathrectangle{\pgfqpoint{0.100000in}{0.212622in}}{\pgfqpoint{3.696000in}{3.696000in}}%
\pgfusepath{clip}%
\pgfsetbuttcap%
\pgfsetroundjoin%
\definecolor{currentfill}{rgb}{0.121569,0.466667,0.705882}%
\pgfsetfillcolor{currentfill}%
\pgfsetfillopacity{0.879341}%
\pgfsetlinewidth{1.003750pt}%
\definecolor{currentstroke}{rgb}{0.121569,0.466667,0.705882}%
\pgfsetstrokecolor{currentstroke}%
\pgfsetstrokeopacity{0.879341}%
\pgfsetdash{}{0pt}%
\pgfpathmoveto{\pgfqpoint{2.472492in}{2.363074in}}%
\pgfpathcurveto{\pgfqpoint{2.480728in}{2.363074in}}{\pgfqpoint{2.488628in}{2.366346in}}{\pgfqpoint{2.494452in}{2.372170in}}%
\pgfpathcurveto{\pgfqpoint{2.500276in}{2.377994in}}{\pgfqpoint{2.503548in}{2.385894in}}{\pgfqpoint{2.503548in}{2.394130in}}%
\pgfpathcurveto{\pgfqpoint{2.503548in}{2.402366in}}{\pgfqpoint{2.500276in}{2.410266in}}{\pgfqpoint{2.494452in}{2.416090in}}%
\pgfpathcurveto{\pgfqpoint{2.488628in}{2.421914in}}{\pgfqpoint{2.480728in}{2.425187in}}{\pgfqpoint{2.472492in}{2.425187in}}%
\pgfpathcurveto{\pgfqpoint{2.464256in}{2.425187in}}{\pgfqpoint{2.456356in}{2.421914in}}{\pgfqpoint{2.450532in}{2.416090in}}%
\pgfpathcurveto{\pgfqpoint{2.444708in}{2.410266in}}{\pgfqpoint{2.441435in}{2.402366in}}{\pgfqpoint{2.441435in}{2.394130in}}%
\pgfpathcurveto{\pgfqpoint{2.441435in}{2.385894in}}{\pgfqpoint{2.444708in}{2.377994in}}{\pgfqpoint{2.450532in}{2.372170in}}%
\pgfpathcurveto{\pgfqpoint{2.456356in}{2.366346in}}{\pgfqpoint{2.464256in}{2.363074in}}{\pgfqpoint{2.472492in}{2.363074in}}%
\pgfpathclose%
\pgfusepath{stroke,fill}%
\end{pgfscope}%
\begin{pgfscope}%
\pgfpathrectangle{\pgfqpoint{0.100000in}{0.212622in}}{\pgfqpoint{3.696000in}{3.696000in}}%
\pgfusepath{clip}%
\pgfsetbuttcap%
\pgfsetroundjoin%
\definecolor{currentfill}{rgb}{0.121569,0.466667,0.705882}%
\pgfsetfillcolor{currentfill}%
\pgfsetfillopacity{0.879514}%
\pgfsetlinewidth{1.003750pt}%
\definecolor{currentstroke}{rgb}{0.121569,0.466667,0.705882}%
\pgfsetstrokecolor{currentstroke}%
\pgfsetstrokeopacity{0.879514}%
\pgfsetdash{}{0pt}%
\pgfpathmoveto{\pgfqpoint{2.475134in}{2.363606in}}%
\pgfpathcurveto{\pgfqpoint{2.483371in}{2.363606in}}{\pgfqpoint{2.491271in}{2.366878in}}{\pgfqpoint{2.497094in}{2.372702in}}%
\pgfpathcurveto{\pgfqpoint{2.502918in}{2.378526in}}{\pgfqpoint{2.506191in}{2.386426in}}{\pgfqpoint{2.506191in}{2.394662in}}%
\pgfpathcurveto{\pgfqpoint{2.506191in}{2.402898in}}{\pgfqpoint{2.502918in}{2.410798in}}{\pgfqpoint{2.497094in}{2.416622in}}%
\pgfpathcurveto{\pgfqpoint{2.491271in}{2.422446in}}{\pgfqpoint{2.483371in}{2.425719in}}{\pgfqpoint{2.475134in}{2.425719in}}%
\pgfpathcurveto{\pgfqpoint{2.466898in}{2.425719in}}{\pgfqpoint{2.458998in}{2.422446in}}{\pgfqpoint{2.453174in}{2.416622in}}%
\pgfpathcurveto{\pgfqpoint{2.447350in}{2.410798in}}{\pgfqpoint{2.444078in}{2.402898in}}{\pgfqpoint{2.444078in}{2.394662in}}%
\pgfpathcurveto{\pgfqpoint{2.444078in}{2.386426in}}{\pgfqpoint{2.447350in}{2.378526in}}{\pgfqpoint{2.453174in}{2.372702in}}%
\pgfpathcurveto{\pgfqpoint{2.458998in}{2.366878in}}{\pgfqpoint{2.466898in}{2.363606in}}{\pgfqpoint{2.475134in}{2.363606in}}%
\pgfpathclose%
\pgfusepath{stroke,fill}%
\end{pgfscope}%
\begin{pgfscope}%
\pgfpathrectangle{\pgfqpoint{0.100000in}{0.212622in}}{\pgfqpoint{3.696000in}{3.696000in}}%
\pgfusepath{clip}%
\pgfsetbuttcap%
\pgfsetroundjoin%
\definecolor{currentfill}{rgb}{0.121569,0.466667,0.705882}%
\pgfsetfillcolor{currentfill}%
\pgfsetfillopacity{0.879985}%
\pgfsetlinewidth{1.003750pt}%
\definecolor{currentstroke}{rgb}{0.121569,0.466667,0.705882}%
\pgfsetstrokecolor{currentstroke}%
\pgfsetstrokeopacity{0.879985}%
\pgfsetdash{}{0pt}%
\pgfpathmoveto{\pgfqpoint{3.080852in}{2.629748in}}%
\pgfpathcurveto{\pgfqpoint{3.089088in}{2.629748in}}{\pgfqpoint{3.096988in}{2.633020in}}{\pgfqpoint{3.102812in}{2.638844in}}%
\pgfpathcurveto{\pgfqpoint{3.108636in}{2.644668in}}{\pgfqpoint{3.111908in}{2.652568in}}{\pgfqpoint{3.111908in}{2.660804in}}%
\pgfpathcurveto{\pgfqpoint{3.111908in}{2.669040in}}{\pgfqpoint{3.108636in}{2.676941in}}{\pgfqpoint{3.102812in}{2.682764in}}%
\pgfpathcurveto{\pgfqpoint{3.096988in}{2.688588in}}{\pgfqpoint{3.089088in}{2.691861in}}{\pgfqpoint{3.080852in}{2.691861in}}%
\pgfpathcurveto{\pgfqpoint{3.072615in}{2.691861in}}{\pgfqpoint{3.064715in}{2.688588in}}{\pgfqpoint{3.058891in}{2.682764in}}%
\pgfpathcurveto{\pgfqpoint{3.053067in}{2.676941in}}{\pgfqpoint{3.049795in}{2.669040in}}{\pgfqpoint{3.049795in}{2.660804in}}%
\pgfpathcurveto{\pgfqpoint{3.049795in}{2.652568in}}{\pgfqpoint{3.053067in}{2.644668in}}{\pgfqpoint{3.058891in}{2.638844in}}%
\pgfpathcurveto{\pgfqpoint{3.064715in}{2.633020in}}{\pgfqpoint{3.072615in}{2.629748in}}{\pgfqpoint{3.080852in}{2.629748in}}%
\pgfpathclose%
\pgfusepath{stroke,fill}%
\end{pgfscope}%
\begin{pgfscope}%
\pgfpathrectangle{\pgfqpoint{0.100000in}{0.212622in}}{\pgfqpoint{3.696000in}{3.696000in}}%
\pgfusepath{clip}%
\pgfsetbuttcap%
\pgfsetroundjoin%
\definecolor{currentfill}{rgb}{0.121569,0.466667,0.705882}%
\pgfsetfillcolor{currentfill}%
\pgfsetfillopacity{0.880034}%
\pgfsetlinewidth{1.003750pt}%
\definecolor{currentstroke}{rgb}{0.121569,0.466667,0.705882}%
\pgfsetstrokecolor{currentstroke}%
\pgfsetstrokeopacity{0.880034}%
\pgfsetdash{}{0pt}%
\pgfpathmoveto{\pgfqpoint{2.469353in}{2.361745in}}%
\pgfpathcurveto{\pgfqpoint{2.477589in}{2.361745in}}{\pgfqpoint{2.485489in}{2.365017in}}{\pgfqpoint{2.491313in}{2.370841in}}%
\pgfpathcurveto{\pgfqpoint{2.497137in}{2.376665in}}{\pgfqpoint{2.500409in}{2.384565in}}{\pgfqpoint{2.500409in}{2.392801in}}%
\pgfpathcurveto{\pgfqpoint{2.500409in}{2.401037in}}{\pgfqpoint{2.497137in}{2.408937in}}{\pgfqpoint{2.491313in}{2.414761in}}%
\pgfpathcurveto{\pgfqpoint{2.485489in}{2.420585in}}{\pgfqpoint{2.477589in}{2.423858in}}{\pgfqpoint{2.469353in}{2.423858in}}%
\pgfpathcurveto{\pgfqpoint{2.461116in}{2.423858in}}{\pgfqpoint{2.453216in}{2.420585in}}{\pgfqpoint{2.447392in}{2.414761in}}%
\pgfpathcurveto{\pgfqpoint{2.441568in}{2.408937in}}{\pgfqpoint{2.438296in}{2.401037in}}{\pgfqpoint{2.438296in}{2.392801in}}%
\pgfpathcurveto{\pgfqpoint{2.438296in}{2.384565in}}{\pgfqpoint{2.441568in}{2.376665in}}{\pgfqpoint{2.447392in}{2.370841in}}%
\pgfpathcurveto{\pgfqpoint{2.453216in}{2.365017in}}{\pgfqpoint{2.461116in}{2.361745in}}{\pgfqpoint{2.469353in}{2.361745in}}%
\pgfpathclose%
\pgfusepath{stroke,fill}%
\end{pgfscope}%
\begin{pgfscope}%
\pgfpathrectangle{\pgfqpoint{0.100000in}{0.212622in}}{\pgfqpoint{3.696000in}{3.696000in}}%
\pgfusepath{clip}%
\pgfsetbuttcap%
\pgfsetroundjoin%
\definecolor{currentfill}{rgb}{0.121569,0.466667,0.705882}%
\pgfsetfillcolor{currentfill}%
\pgfsetfillopacity{0.880265}%
\pgfsetlinewidth{1.003750pt}%
\definecolor{currentstroke}{rgb}{0.121569,0.466667,0.705882}%
\pgfsetstrokecolor{currentstroke}%
\pgfsetstrokeopacity{0.880265}%
\pgfsetdash{}{0pt}%
\pgfpathmoveto{\pgfqpoint{2.477955in}{2.362014in}}%
\pgfpathcurveto{\pgfqpoint{2.486191in}{2.362014in}}{\pgfqpoint{2.494091in}{2.365286in}}{\pgfqpoint{2.499915in}{2.371110in}}%
\pgfpathcurveto{\pgfqpoint{2.505739in}{2.376934in}}{\pgfqpoint{2.509011in}{2.384834in}}{\pgfqpoint{2.509011in}{2.393071in}}%
\pgfpathcurveto{\pgfqpoint{2.509011in}{2.401307in}}{\pgfqpoint{2.505739in}{2.409207in}}{\pgfqpoint{2.499915in}{2.415031in}}%
\pgfpathcurveto{\pgfqpoint{2.494091in}{2.420855in}}{\pgfqpoint{2.486191in}{2.424127in}}{\pgfqpoint{2.477955in}{2.424127in}}%
\pgfpathcurveto{\pgfqpoint{2.469718in}{2.424127in}}{\pgfqpoint{2.461818in}{2.420855in}}{\pgfqpoint{2.455994in}{2.415031in}}%
\pgfpathcurveto{\pgfqpoint{2.450170in}{2.409207in}}{\pgfqpoint{2.446898in}{2.401307in}}{\pgfqpoint{2.446898in}{2.393071in}}%
\pgfpathcurveto{\pgfqpoint{2.446898in}{2.384834in}}{\pgfqpoint{2.450170in}{2.376934in}}{\pgfqpoint{2.455994in}{2.371110in}}%
\pgfpathcurveto{\pgfqpoint{2.461818in}{2.365286in}}{\pgfqpoint{2.469718in}{2.362014in}}{\pgfqpoint{2.477955in}{2.362014in}}%
\pgfpathclose%
\pgfusepath{stroke,fill}%
\end{pgfscope}%
\begin{pgfscope}%
\pgfpathrectangle{\pgfqpoint{0.100000in}{0.212622in}}{\pgfqpoint{3.696000in}{3.696000in}}%
\pgfusepath{clip}%
\pgfsetbuttcap%
\pgfsetroundjoin%
\definecolor{currentfill}{rgb}{0.121569,0.466667,0.705882}%
\pgfsetfillcolor{currentfill}%
\pgfsetfillopacity{0.881352}%
\pgfsetlinewidth{1.003750pt}%
\definecolor{currentstroke}{rgb}{0.121569,0.466667,0.705882}%
\pgfsetstrokecolor{currentstroke}%
\pgfsetstrokeopacity{0.881352}%
\pgfsetdash{}{0pt}%
\pgfpathmoveto{\pgfqpoint{3.080557in}{2.627162in}}%
\pgfpathcurveto{\pgfqpoint{3.088794in}{2.627162in}}{\pgfqpoint{3.096694in}{2.630434in}}{\pgfqpoint{3.102518in}{2.636258in}}%
\pgfpathcurveto{\pgfqpoint{3.108341in}{2.642082in}}{\pgfqpoint{3.111614in}{2.649982in}}{\pgfqpoint{3.111614in}{2.658218in}}%
\pgfpathcurveto{\pgfqpoint{3.111614in}{2.666454in}}{\pgfqpoint{3.108341in}{2.674354in}}{\pgfqpoint{3.102518in}{2.680178in}}%
\pgfpathcurveto{\pgfqpoint{3.096694in}{2.686002in}}{\pgfqpoint{3.088794in}{2.689275in}}{\pgfqpoint{3.080557in}{2.689275in}}%
\pgfpathcurveto{\pgfqpoint{3.072321in}{2.689275in}}{\pgfqpoint{3.064421in}{2.686002in}}{\pgfqpoint{3.058597in}{2.680178in}}%
\pgfpathcurveto{\pgfqpoint{3.052773in}{2.674354in}}{\pgfqpoint{3.049501in}{2.666454in}}{\pgfqpoint{3.049501in}{2.658218in}}%
\pgfpathcurveto{\pgfqpoint{3.049501in}{2.649982in}}{\pgfqpoint{3.052773in}{2.642082in}}{\pgfqpoint{3.058597in}{2.636258in}}%
\pgfpathcurveto{\pgfqpoint{3.064421in}{2.630434in}}{\pgfqpoint{3.072321in}{2.627162in}}{\pgfqpoint{3.080557in}{2.627162in}}%
\pgfpathclose%
\pgfusepath{stroke,fill}%
\end{pgfscope}%
\begin{pgfscope}%
\pgfpathrectangle{\pgfqpoint{0.100000in}{0.212622in}}{\pgfqpoint{3.696000in}{3.696000in}}%
\pgfusepath{clip}%
\pgfsetbuttcap%
\pgfsetroundjoin%
\definecolor{currentfill}{rgb}{0.121569,0.466667,0.705882}%
\pgfsetfillcolor{currentfill}%
\pgfsetfillopacity{0.882240}%
\pgfsetlinewidth{1.003750pt}%
\definecolor{currentstroke}{rgb}{0.121569,0.466667,0.705882}%
\pgfsetstrokecolor{currentstroke}%
\pgfsetstrokeopacity{0.882240}%
\pgfsetdash{}{0pt}%
\pgfpathmoveto{\pgfqpoint{2.387930in}{2.333667in}}%
\pgfpathcurveto{\pgfqpoint{2.396166in}{2.333667in}}{\pgfqpoint{2.404066in}{2.336939in}}{\pgfqpoint{2.409890in}{2.342763in}}%
\pgfpathcurveto{\pgfqpoint{2.415714in}{2.348587in}}{\pgfqpoint{2.418987in}{2.356487in}}{\pgfqpoint{2.418987in}{2.364723in}}%
\pgfpathcurveto{\pgfqpoint{2.418987in}{2.372960in}}{\pgfqpoint{2.415714in}{2.380860in}}{\pgfqpoint{2.409890in}{2.386684in}}%
\pgfpathcurveto{\pgfqpoint{2.404066in}{2.392507in}}{\pgfqpoint{2.396166in}{2.395780in}}{\pgfqpoint{2.387930in}{2.395780in}}%
\pgfpathcurveto{\pgfqpoint{2.379694in}{2.395780in}}{\pgfqpoint{2.371794in}{2.392507in}}{\pgfqpoint{2.365970in}{2.386684in}}%
\pgfpathcurveto{\pgfqpoint{2.360146in}{2.380860in}}{\pgfqpoint{2.356874in}{2.372960in}}{\pgfqpoint{2.356874in}{2.364723in}}%
\pgfpathcurveto{\pgfqpoint{2.356874in}{2.356487in}}{\pgfqpoint{2.360146in}{2.348587in}}{\pgfqpoint{2.365970in}{2.342763in}}%
\pgfpathcurveto{\pgfqpoint{2.371794in}{2.336939in}}{\pgfqpoint{2.379694in}{2.333667in}}{\pgfqpoint{2.387930in}{2.333667in}}%
\pgfpathclose%
\pgfusepath{stroke,fill}%
\end{pgfscope}%
\begin{pgfscope}%
\pgfpathrectangle{\pgfqpoint{0.100000in}{0.212622in}}{\pgfqpoint{3.696000in}{3.696000in}}%
\pgfusepath{clip}%
\pgfsetbuttcap%
\pgfsetroundjoin%
\definecolor{currentfill}{rgb}{0.121569,0.466667,0.705882}%
\pgfsetfillcolor{currentfill}%
\pgfsetfillopacity{0.882976}%
\pgfsetlinewidth{1.003750pt}%
\definecolor{currentstroke}{rgb}{0.121569,0.466667,0.705882}%
\pgfsetstrokecolor{currentstroke}%
\pgfsetstrokeopacity{0.882976}%
\pgfsetdash{}{0pt}%
\pgfpathmoveto{\pgfqpoint{2.432523in}{2.343003in}}%
\pgfpathcurveto{\pgfqpoint{2.440760in}{2.343003in}}{\pgfqpoint{2.448660in}{2.346275in}}{\pgfqpoint{2.454484in}{2.352099in}}%
\pgfpathcurveto{\pgfqpoint{2.460308in}{2.357923in}}{\pgfqpoint{2.463580in}{2.365823in}}{\pgfqpoint{2.463580in}{2.374060in}}%
\pgfpathcurveto{\pgfqpoint{2.463580in}{2.382296in}}{\pgfqpoint{2.460308in}{2.390196in}}{\pgfqpoint{2.454484in}{2.396020in}}%
\pgfpathcurveto{\pgfqpoint{2.448660in}{2.401844in}}{\pgfqpoint{2.440760in}{2.405116in}}{\pgfqpoint{2.432523in}{2.405116in}}%
\pgfpathcurveto{\pgfqpoint{2.424287in}{2.405116in}}{\pgfqpoint{2.416387in}{2.401844in}}{\pgfqpoint{2.410563in}{2.396020in}}%
\pgfpathcurveto{\pgfqpoint{2.404739in}{2.390196in}}{\pgfqpoint{2.401467in}{2.382296in}}{\pgfqpoint{2.401467in}{2.374060in}}%
\pgfpathcurveto{\pgfqpoint{2.401467in}{2.365823in}}{\pgfqpoint{2.404739in}{2.357923in}}{\pgfqpoint{2.410563in}{2.352099in}}%
\pgfpathcurveto{\pgfqpoint{2.416387in}{2.346275in}}{\pgfqpoint{2.424287in}{2.343003in}}{\pgfqpoint{2.432523in}{2.343003in}}%
\pgfpathclose%
\pgfusepath{stroke,fill}%
\end{pgfscope}%
\begin{pgfscope}%
\pgfpathrectangle{\pgfqpoint{0.100000in}{0.212622in}}{\pgfqpoint{3.696000in}{3.696000in}}%
\pgfusepath{clip}%
\pgfsetbuttcap%
\pgfsetroundjoin%
\definecolor{currentfill}{rgb}{0.121569,0.466667,0.705882}%
\pgfsetfillcolor{currentfill}%
\pgfsetfillopacity{0.883363}%
\pgfsetlinewidth{1.003750pt}%
\definecolor{currentstroke}{rgb}{0.121569,0.466667,0.705882}%
\pgfsetstrokecolor{currentstroke}%
\pgfsetstrokeopacity{0.883363}%
\pgfsetdash{}{0pt}%
\pgfpathmoveto{\pgfqpoint{3.078709in}{2.626526in}}%
\pgfpathcurveto{\pgfqpoint{3.086946in}{2.626526in}}{\pgfqpoint{3.094846in}{2.629798in}}{\pgfqpoint{3.100670in}{2.635622in}}%
\pgfpathcurveto{\pgfqpoint{3.106494in}{2.641446in}}{\pgfqpoint{3.109766in}{2.649346in}}{\pgfqpoint{3.109766in}{2.657582in}}%
\pgfpathcurveto{\pgfqpoint{3.109766in}{2.665819in}}{\pgfqpoint{3.106494in}{2.673719in}}{\pgfqpoint{3.100670in}{2.679543in}}%
\pgfpathcurveto{\pgfqpoint{3.094846in}{2.685367in}}{\pgfqpoint{3.086946in}{2.688639in}}{\pgfqpoint{3.078709in}{2.688639in}}%
\pgfpathcurveto{\pgfqpoint{3.070473in}{2.688639in}}{\pgfqpoint{3.062573in}{2.685367in}}{\pgfqpoint{3.056749in}{2.679543in}}%
\pgfpathcurveto{\pgfqpoint{3.050925in}{2.673719in}}{\pgfqpoint{3.047653in}{2.665819in}}{\pgfqpoint{3.047653in}{2.657582in}}%
\pgfpathcurveto{\pgfqpoint{3.047653in}{2.649346in}}{\pgfqpoint{3.050925in}{2.641446in}}{\pgfqpoint{3.056749in}{2.635622in}}%
\pgfpathcurveto{\pgfqpoint{3.062573in}{2.629798in}}{\pgfqpoint{3.070473in}{2.626526in}}{\pgfqpoint{3.078709in}{2.626526in}}%
\pgfpathclose%
\pgfusepath{stroke,fill}%
\end{pgfscope}%
\begin{pgfscope}%
\pgfpathrectangle{\pgfqpoint{0.100000in}{0.212622in}}{\pgfqpoint{3.696000in}{3.696000in}}%
\pgfusepath{clip}%
\pgfsetbuttcap%
\pgfsetroundjoin%
\definecolor{currentfill}{rgb}{0.121569,0.466667,0.705882}%
\pgfsetfillcolor{currentfill}%
\pgfsetfillopacity{0.883505}%
\pgfsetlinewidth{1.003750pt}%
\definecolor{currentstroke}{rgb}{0.121569,0.466667,0.705882}%
\pgfsetstrokecolor{currentstroke}%
\pgfsetstrokeopacity{0.883505}%
\pgfsetdash{}{0pt}%
\pgfpathmoveto{\pgfqpoint{2.509194in}{2.399841in}}%
\pgfpathcurveto{\pgfqpoint{2.517431in}{2.399841in}}{\pgfqpoint{2.525331in}{2.403113in}}{\pgfqpoint{2.531155in}{2.408937in}}%
\pgfpathcurveto{\pgfqpoint{2.536979in}{2.414761in}}{\pgfqpoint{2.540251in}{2.422661in}}{\pgfqpoint{2.540251in}{2.430897in}}%
\pgfpathcurveto{\pgfqpoint{2.540251in}{2.439134in}}{\pgfqpoint{2.536979in}{2.447034in}}{\pgfqpoint{2.531155in}{2.452858in}}%
\pgfpathcurveto{\pgfqpoint{2.525331in}{2.458682in}}{\pgfqpoint{2.517431in}{2.461954in}}{\pgfqpoint{2.509194in}{2.461954in}}%
\pgfpathcurveto{\pgfqpoint{2.500958in}{2.461954in}}{\pgfqpoint{2.493058in}{2.458682in}}{\pgfqpoint{2.487234in}{2.452858in}}%
\pgfpathcurveto{\pgfqpoint{2.481410in}{2.447034in}}{\pgfqpoint{2.478138in}{2.439134in}}{\pgfqpoint{2.478138in}{2.430897in}}%
\pgfpathcurveto{\pgfqpoint{2.478138in}{2.422661in}}{\pgfqpoint{2.481410in}{2.414761in}}{\pgfqpoint{2.487234in}{2.408937in}}%
\pgfpathcurveto{\pgfqpoint{2.493058in}{2.403113in}}{\pgfqpoint{2.500958in}{2.399841in}}{\pgfqpoint{2.509194in}{2.399841in}}%
\pgfpathclose%
\pgfusepath{stroke,fill}%
\end{pgfscope}%
\begin{pgfscope}%
\pgfpathrectangle{\pgfqpoint{0.100000in}{0.212622in}}{\pgfqpoint{3.696000in}{3.696000in}}%
\pgfusepath{clip}%
\pgfsetbuttcap%
\pgfsetroundjoin%
\definecolor{currentfill}{rgb}{0.121569,0.466667,0.705882}%
\pgfsetfillcolor{currentfill}%
\pgfsetfillopacity{0.883976}%
\pgfsetlinewidth{1.003750pt}%
\definecolor{currentstroke}{rgb}{0.121569,0.466667,0.705882}%
\pgfsetstrokecolor{currentstroke}%
\pgfsetstrokeopacity{0.883976}%
\pgfsetdash{}{0pt}%
\pgfpathmoveto{\pgfqpoint{2.448858in}{2.341733in}}%
\pgfpathcurveto{\pgfqpoint{2.457094in}{2.341733in}}{\pgfqpoint{2.464995in}{2.345006in}}{\pgfqpoint{2.470818in}{2.350829in}}%
\pgfpathcurveto{\pgfqpoint{2.476642in}{2.356653in}}{\pgfqpoint{2.479915in}{2.364553in}}{\pgfqpoint{2.479915in}{2.372790in}}%
\pgfpathcurveto{\pgfqpoint{2.479915in}{2.381026in}}{\pgfqpoint{2.476642in}{2.388926in}}{\pgfqpoint{2.470818in}{2.394750in}}%
\pgfpathcurveto{\pgfqpoint{2.464995in}{2.400574in}}{\pgfqpoint{2.457094in}{2.403846in}}{\pgfqpoint{2.448858in}{2.403846in}}%
\pgfpathcurveto{\pgfqpoint{2.440622in}{2.403846in}}{\pgfqpoint{2.432722in}{2.400574in}}{\pgfqpoint{2.426898in}{2.394750in}}%
\pgfpathcurveto{\pgfqpoint{2.421074in}{2.388926in}}{\pgfqpoint{2.417802in}{2.381026in}}{\pgfqpoint{2.417802in}{2.372790in}}%
\pgfpathcurveto{\pgfqpoint{2.417802in}{2.364553in}}{\pgfqpoint{2.421074in}{2.356653in}}{\pgfqpoint{2.426898in}{2.350829in}}%
\pgfpathcurveto{\pgfqpoint{2.432722in}{2.345006in}}{\pgfqpoint{2.440622in}{2.341733in}}{\pgfqpoint{2.448858in}{2.341733in}}%
\pgfpathclose%
\pgfusepath{stroke,fill}%
\end{pgfscope}%
\begin{pgfscope}%
\pgfpathrectangle{\pgfqpoint{0.100000in}{0.212622in}}{\pgfqpoint{3.696000in}{3.696000in}}%
\pgfusepath{clip}%
\pgfsetbuttcap%
\pgfsetroundjoin%
\definecolor{currentfill}{rgb}{0.121569,0.466667,0.705882}%
\pgfsetfillcolor{currentfill}%
\pgfsetfillopacity{0.885661}%
\pgfsetlinewidth{1.003750pt}%
\definecolor{currentstroke}{rgb}{0.121569,0.466667,0.705882}%
\pgfsetstrokecolor{currentstroke}%
\pgfsetstrokeopacity{0.885661}%
\pgfsetdash{}{0pt}%
\pgfpathmoveto{\pgfqpoint{3.074690in}{2.622089in}}%
\pgfpathcurveto{\pgfqpoint{3.082926in}{2.622089in}}{\pgfqpoint{3.090826in}{2.625361in}}{\pgfqpoint{3.096650in}{2.631185in}}%
\pgfpathcurveto{\pgfqpoint{3.102474in}{2.637009in}}{\pgfqpoint{3.105746in}{2.644909in}}{\pgfqpoint{3.105746in}{2.653145in}}%
\pgfpathcurveto{\pgfqpoint{3.105746in}{2.661381in}}{\pgfqpoint{3.102474in}{2.669281in}}{\pgfqpoint{3.096650in}{2.675105in}}%
\pgfpathcurveto{\pgfqpoint{3.090826in}{2.680929in}}{\pgfqpoint{3.082926in}{2.684202in}}{\pgfqpoint{3.074690in}{2.684202in}}%
\pgfpathcurveto{\pgfqpoint{3.066453in}{2.684202in}}{\pgfqpoint{3.058553in}{2.680929in}}{\pgfqpoint{3.052729in}{2.675105in}}%
\pgfpathcurveto{\pgfqpoint{3.046906in}{2.669281in}}{\pgfqpoint{3.043633in}{2.661381in}}{\pgfqpoint{3.043633in}{2.653145in}}%
\pgfpathcurveto{\pgfqpoint{3.043633in}{2.644909in}}{\pgfqpoint{3.046906in}{2.637009in}}{\pgfqpoint{3.052729in}{2.631185in}}%
\pgfpathcurveto{\pgfqpoint{3.058553in}{2.625361in}}{\pgfqpoint{3.066453in}{2.622089in}}{\pgfqpoint{3.074690in}{2.622089in}}%
\pgfpathclose%
\pgfusepath{stroke,fill}%
\end{pgfscope}%
\begin{pgfscope}%
\pgfpathrectangle{\pgfqpoint{0.100000in}{0.212622in}}{\pgfqpoint{3.696000in}{3.696000in}}%
\pgfusepath{clip}%
\pgfsetbuttcap%
\pgfsetroundjoin%
\definecolor{currentfill}{rgb}{0.121569,0.466667,0.705882}%
\pgfsetfillcolor{currentfill}%
\pgfsetfillopacity{0.886138}%
\pgfsetlinewidth{1.003750pt}%
\definecolor{currentstroke}{rgb}{0.121569,0.466667,0.705882}%
\pgfsetstrokecolor{currentstroke}%
\pgfsetstrokeopacity{0.886138}%
\pgfsetdash{}{0pt}%
\pgfpathmoveto{\pgfqpoint{3.074196in}{2.621425in}}%
\pgfpathcurveto{\pgfqpoint{3.082432in}{2.621425in}}{\pgfqpoint{3.090332in}{2.624697in}}{\pgfqpoint{3.096156in}{2.630521in}}%
\pgfpathcurveto{\pgfqpoint{3.101980in}{2.636345in}}{\pgfqpoint{3.105252in}{2.644245in}}{\pgfqpoint{3.105252in}{2.652481in}}%
\pgfpathcurveto{\pgfqpoint{3.105252in}{2.660717in}}{\pgfqpoint{3.101980in}{2.668617in}}{\pgfqpoint{3.096156in}{2.674441in}}%
\pgfpathcurveto{\pgfqpoint{3.090332in}{2.680265in}}{\pgfqpoint{3.082432in}{2.683537in}}{\pgfqpoint{3.074196in}{2.683537in}}%
\pgfpathcurveto{\pgfqpoint{3.065959in}{2.683537in}}{\pgfqpoint{3.058059in}{2.680265in}}{\pgfqpoint{3.052235in}{2.674441in}}%
\pgfpathcurveto{\pgfqpoint{3.046412in}{2.668617in}}{\pgfqpoint{3.043139in}{2.660717in}}{\pgfqpoint{3.043139in}{2.652481in}}%
\pgfpathcurveto{\pgfqpoint{3.043139in}{2.644245in}}{\pgfqpoint{3.046412in}{2.636345in}}{\pgfqpoint{3.052235in}{2.630521in}}%
\pgfpathcurveto{\pgfqpoint{3.058059in}{2.624697in}}{\pgfqpoint{3.065959in}{2.621425in}}{\pgfqpoint{3.074196in}{2.621425in}}%
\pgfpathclose%
\pgfusepath{stroke,fill}%
\end{pgfscope}%
\begin{pgfscope}%
\pgfpathrectangle{\pgfqpoint{0.100000in}{0.212622in}}{\pgfqpoint{3.696000in}{3.696000in}}%
\pgfusepath{clip}%
\pgfsetbuttcap%
\pgfsetroundjoin%
\definecolor{currentfill}{rgb}{0.121569,0.466667,0.705882}%
\pgfsetfillcolor{currentfill}%
\pgfsetfillopacity{0.886840}%
\pgfsetlinewidth{1.003750pt}%
\definecolor{currentstroke}{rgb}{0.121569,0.466667,0.705882}%
\pgfsetstrokecolor{currentstroke}%
\pgfsetstrokeopacity{0.886840}%
\pgfsetdash{}{0pt}%
\pgfpathmoveto{\pgfqpoint{1.269897in}{1.816441in}}%
\pgfpathcurveto{\pgfqpoint{1.278134in}{1.816441in}}{\pgfqpoint{1.286034in}{1.819713in}}{\pgfqpoint{1.291858in}{1.825537in}}%
\pgfpathcurveto{\pgfqpoint{1.297682in}{1.831361in}}{\pgfqpoint{1.300954in}{1.839261in}}{\pgfqpoint{1.300954in}{1.847497in}}%
\pgfpathcurveto{\pgfqpoint{1.300954in}{1.855734in}}{\pgfqpoint{1.297682in}{1.863634in}}{\pgfqpoint{1.291858in}{1.869458in}}%
\pgfpathcurveto{\pgfqpoint{1.286034in}{1.875282in}}{\pgfqpoint{1.278134in}{1.878554in}}{\pgfqpoint{1.269897in}{1.878554in}}%
\pgfpathcurveto{\pgfqpoint{1.261661in}{1.878554in}}{\pgfqpoint{1.253761in}{1.875282in}}{\pgfqpoint{1.247937in}{1.869458in}}%
\pgfpathcurveto{\pgfqpoint{1.242113in}{1.863634in}}{\pgfqpoint{1.238841in}{1.855734in}}{\pgfqpoint{1.238841in}{1.847497in}}%
\pgfpathcurveto{\pgfqpoint{1.238841in}{1.839261in}}{\pgfqpoint{1.242113in}{1.831361in}}{\pgfqpoint{1.247937in}{1.825537in}}%
\pgfpathcurveto{\pgfqpoint{1.253761in}{1.819713in}}{\pgfqpoint{1.261661in}{1.816441in}}{\pgfqpoint{1.269897in}{1.816441in}}%
\pgfpathclose%
\pgfusepath{stroke,fill}%
\end{pgfscope}%
\begin{pgfscope}%
\pgfpathrectangle{\pgfqpoint{0.100000in}{0.212622in}}{\pgfqpoint{3.696000in}{3.696000in}}%
\pgfusepath{clip}%
\pgfsetbuttcap%
\pgfsetroundjoin%
\definecolor{currentfill}{rgb}{0.121569,0.466667,0.705882}%
\pgfsetfillcolor{currentfill}%
\pgfsetfillopacity{0.887586}%
\pgfsetlinewidth{1.003750pt}%
\definecolor{currentstroke}{rgb}{0.121569,0.466667,0.705882}%
\pgfsetstrokecolor{currentstroke}%
\pgfsetstrokeopacity{0.887586}%
\pgfsetdash{}{0pt}%
\pgfpathmoveto{\pgfqpoint{1.260441in}{1.810394in}}%
\pgfpathcurveto{\pgfqpoint{1.268677in}{1.810394in}}{\pgfqpoint{1.276577in}{1.813666in}}{\pgfqpoint{1.282401in}{1.819490in}}%
\pgfpathcurveto{\pgfqpoint{1.288225in}{1.825314in}}{\pgfqpoint{1.291498in}{1.833214in}}{\pgfqpoint{1.291498in}{1.841450in}}%
\pgfpathcurveto{\pgfqpoint{1.291498in}{1.849687in}}{\pgfqpoint{1.288225in}{1.857587in}}{\pgfqpoint{1.282401in}{1.863411in}}%
\pgfpathcurveto{\pgfqpoint{1.276577in}{1.869234in}}{\pgfqpoint{1.268677in}{1.872507in}}{\pgfqpoint{1.260441in}{1.872507in}}%
\pgfpathcurveto{\pgfqpoint{1.252205in}{1.872507in}}{\pgfqpoint{1.244305in}{1.869234in}}{\pgfqpoint{1.238481in}{1.863411in}}%
\pgfpathcurveto{\pgfqpoint{1.232657in}{1.857587in}}{\pgfqpoint{1.229385in}{1.849687in}}{\pgfqpoint{1.229385in}{1.841450in}}%
\pgfpathcurveto{\pgfqpoint{1.229385in}{1.833214in}}{\pgfqpoint{1.232657in}{1.825314in}}{\pgfqpoint{1.238481in}{1.819490in}}%
\pgfpathcurveto{\pgfqpoint{1.244305in}{1.813666in}}{\pgfqpoint{1.252205in}{1.810394in}}{\pgfqpoint{1.260441in}{1.810394in}}%
\pgfpathclose%
\pgfusepath{stroke,fill}%
\end{pgfscope}%
\begin{pgfscope}%
\pgfpathrectangle{\pgfqpoint{0.100000in}{0.212622in}}{\pgfqpoint{3.696000in}{3.696000in}}%
\pgfusepath{clip}%
\pgfsetbuttcap%
\pgfsetroundjoin%
\definecolor{currentfill}{rgb}{0.121569,0.466667,0.705882}%
\pgfsetfillcolor{currentfill}%
\pgfsetfillopacity{0.888095}%
\pgfsetlinewidth{1.003750pt}%
\definecolor{currentstroke}{rgb}{0.121569,0.466667,0.705882}%
\pgfsetstrokecolor{currentstroke}%
\pgfsetstrokeopacity{0.888095}%
\pgfsetdash{}{0pt}%
\pgfpathmoveto{\pgfqpoint{2.368343in}{2.319283in}}%
\pgfpathcurveto{\pgfqpoint{2.376579in}{2.319283in}}{\pgfqpoint{2.384479in}{2.322555in}}{\pgfqpoint{2.390303in}{2.328379in}}%
\pgfpathcurveto{\pgfqpoint{2.396127in}{2.334203in}}{\pgfqpoint{2.399399in}{2.342103in}}{\pgfqpoint{2.399399in}{2.350339in}}%
\pgfpathcurveto{\pgfqpoint{2.399399in}{2.358575in}}{\pgfqpoint{2.396127in}{2.366475in}}{\pgfqpoint{2.390303in}{2.372299in}}%
\pgfpathcurveto{\pgfqpoint{2.384479in}{2.378123in}}{\pgfqpoint{2.376579in}{2.381396in}}{\pgfqpoint{2.368343in}{2.381396in}}%
\pgfpathcurveto{\pgfqpoint{2.360106in}{2.381396in}}{\pgfqpoint{2.352206in}{2.378123in}}{\pgfqpoint{2.346382in}{2.372299in}}%
\pgfpathcurveto{\pgfqpoint{2.340559in}{2.366475in}}{\pgfqpoint{2.337286in}{2.358575in}}{\pgfqpoint{2.337286in}{2.350339in}}%
\pgfpathcurveto{\pgfqpoint{2.337286in}{2.342103in}}{\pgfqpoint{2.340559in}{2.334203in}}{\pgfqpoint{2.346382in}{2.328379in}}%
\pgfpathcurveto{\pgfqpoint{2.352206in}{2.322555in}}{\pgfqpoint{2.360106in}{2.319283in}}{\pgfqpoint{2.368343in}{2.319283in}}%
\pgfpathclose%
\pgfusepath{stroke,fill}%
\end{pgfscope}%
\begin{pgfscope}%
\pgfpathrectangle{\pgfqpoint{0.100000in}{0.212622in}}{\pgfqpoint{3.696000in}{3.696000in}}%
\pgfusepath{clip}%
\pgfsetbuttcap%
\pgfsetroundjoin%
\definecolor{currentfill}{rgb}{0.121569,0.466667,0.705882}%
\pgfsetfillcolor{currentfill}%
\pgfsetfillopacity{0.888142}%
\pgfsetlinewidth{1.003750pt}%
\definecolor{currentstroke}{rgb}{0.121569,0.466667,0.705882}%
\pgfsetstrokecolor{currentstroke}%
\pgfsetstrokeopacity{0.888142}%
\pgfsetdash{}{0pt}%
\pgfpathmoveto{\pgfqpoint{2.365453in}{2.325336in}}%
\pgfpathcurveto{\pgfqpoint{2.373689in}{2.325336in}}{\pgfqpoint{2.381589in}{2.328609in}}{\pgfqpoint{2.387413in}{2.334433in}}%
\pgfpathcurveto{\pgfqpoint{2.393237in}{2.340257in}}{\pgfqpoint{2.396509in}{2.348157in}}{\pgfqpoint{2.396509in}{2.356393in}}%
\pgfpathcurveto{\pgfqpoint{2.396509in}{2.364629in}}{\pgfqpoint{2.393237in}{2.372529in}}{\pgfqpoint{2.387413in}{2.378353in}}%
\pgfpathcurveto{\pgfqpoint{2.381589in}{2.384177in}}{\pgfqpoint{2.373689in}{2.387449in}}{\pgfqpoint{2.365453in}{2.387449in}}%
\pgfpathcurveto{\pgfqpoint{2.357217in}{2.387449in}}{\pgfqpoint{2.349316in}{2.384177in}}{\pgfqpoint{2.343493in}{2.378353in}}%
\pgfpathcurveto{\pgfqpoint{2.337669in}{2.372529in}}{\pgfqpoint{2.334396in}{2.364629in}}{\pgfqpoint{2.334396in}{2.356393in}}%
\pgfpathcurveto{\pgfqpoint{2.334396in}{2.348157in}}{\pgfqpoint{2.337669in}{2.340257in}}{\pgfqpoint{2.343493in}{2.334433in}}%
\pgfpathcurveto{\pgfqpoint{2.349316in}{2.328609in}}{\pgfqpoint{2.357217in}{2.325336in}}{\pgfqpoint{2.365453in}{2.325336in}}%
\pgfpathclose%
\pgfusepath{stroke,fill}%
\end{pgfscope}%
\begin{pgfscope}%
\pgfpathrectangle{\pgfqpoint{0.100000in}{0.212622in}}{\pgfqpoint{3.696000in}{3.696000in}}%
\pgfusepath{clip}%
\pgfsetbuttcap%
\pgfsetroundjoin%
\definecolor{currentfill}{rgb}{0.121569,0.466667,0.705882}%
\pgfsetfillcolor{currentfill}%
\pgfsetfillopacity{0.888239}%
\pgfsetlinewidth{1.003750pt}%
\definecolor{currentstroke}{rgb}{0.121569,0.466667,0.705882}%
\pgfsetstrokecolor{currentstroke}%
\pgfsetstrokeopacity{0.888239}%
\pgfsetdash{}{0pt}%
\pgfpathmoveto{\pgfqpoint{3.071960in}{2.618842in}}%
\pgfpathcurveto{\pgfqpoint{3.080196in}{2.618842in}}{\pgfqpoint{3.088096in}{2.622114in}}{\pgfqpoint{3.093920in}{2.627938in}}%
\pgfpathcurveto{\pgfqpoint{3.099744in}{2.633762in}}{\pgfqpoint{3.103017in}{2.641662in}}{\pgfqpoint{3.103017in}{2.649898in}}%
\pgfpathcurveto{\pgfqpoint{3.103017in}{2.658135in}}{\pgfqpoint{3.099744in}{2.666035in}}{\pgfqpoint{3.093920in}{2.671859in}}%
\pgfpathcurveto{\pgfqpoint{3.088096in}{2.677683in}}{\pgfqpoint{3.080196in}{2.680955in}}{\pgfqpoint{3.071960in}{2.680955in}}%
\pgfpathcurveto{\pgfqpoint{3.063724in}{2.680955in}}{\pgfqpoint{3.055824in}{2.677683in}}{\pgfqpoint{3.050000in}{2.671859in}}%
\pgfpathcurveto{\pgfqpoint{3.044176in}{2.666035in}}{\pgfqpoint{3.040904in}{2.658135in}}{\pgfqpoint{3.040904in}{2.649898in}}%
\pgfpathcurveto{\pgfqpoint{3.040904in}{2.641662in}}{\pgfqpoint{3.044176in}{2.633762in}}{\pgfqpoint{3.050000in}{2.627938in}}%
\pgfpathcurveto{\pgfqpoint{3.055824in}{2.622114in}}{\pgfqpoint{3.063724in}{2.618842in}}{\pgfqpoint{3.071960in}{2.618842in}}%
\pgfpathclose%
\pgfusepath{stroke,fill}%
\end{pgfscope}%
\begin{pgfscope}%
\pgfpathrectangle{\pgfqpoint{0.100000in}{0.212622in}}{\pgfqpoint{3.696000in}{3.696000in}}%
\pgfusepath{clip}%
\pgfsetbuttcap%
\pgfsetroundjoin%
\definecolor{currentfill}{rgb}{0.121569,0.466667,0.705882}%
\pgfsetfillcolor{currentfill}%
\pgfsetfillopacity{0.888762}%
\pgfsetlinewidth{1.003750pt}%
\definecolor{currentstroke}{rgb}{0.121569,0.466667,0.705882}%
\pgfsetstrokecolor{currentstroke}%
\pgfsetstrokeopacity{0.888762}%
\pgfsetdash{}{0pt}%
\pgfpathmoveto{\pgfqpoint{1.259707in}{1.807797in}}%
\pgfpathcurveto{\pgfqpoint{1.267943in}{1.807797in}}{\pgfqpoint{1.275843in}{1.811070in}}{\pgfqpoint{1.281667in}{1.816894in}}%
\pgfpathcurveto{\pgfqpoint{1.287491in}{1.822718in}}{\pgfqpoint{1.290763in}{1.830618in}}{\pgfqpoint{1.290763in}{1.838854in}}%
\pgfpathcurveto{\pgfqpoint{1.290763in}{1.847090in}}{\pgfqpoint{1.287491in}{1.854990in}}{\pgfqpoint{1.281667in}{1.860814in}}%
\pgfpathcurveto{\pgfqpoint{1.275843in}{1.866638in}}{\pgfqpoint{1.267943in}{1.869910in}}{\pgfqpoint{1.259707in}{1.869910in}}%
\pgfpathcurveto{\pgfqpoint{1.251471in}{1.869910in}}{\pgfqpoint{1.243571in}{1.866638in}}{\pgfqpoint{1.237747in}{1.860814in}}%
\pgfpathcurveto{\pgfqpoint{1.231923in}{1.854990in}}{\pgfqpoint{1.228650in}{1.847090in}}{\pgfqpoint{1.228650in}{1.838854in}}%
\pgfpathcurveto{\pgfqpoint{1.228650in}{1.830618in}}{\pgfqpoint{1.231923in}{1.822718in}}{\pgfqpoint{1.237747in}{1.816894in}}%
\pgfpathcurveto{\pgfqpoint{1.243571in}{1.811070in}}{\pgfqpoint{1.251471in}{1.807797in}}{\pgfqpoint{1.259707in}{1.807797in}}%
\pgfpathclose%
\pgfusepath{stroke,fill}%
\end{pgfscope}%
\begin{pgfscope}%
\pgfpathrectangle{\pgfqpoint{0.100000in}{0.212622in}}{\pgfqpoint{3.696000in}{3.696000in}}%
\pgfusepath{clip}%
\pgfsetbuttcap%
\pgfsetroundjoin%
\definecolor{currentfill}{rgb}{0.121569,0.466667,0.705882}%
\pgfsetfillcolor{currentfill}%
\pgfsetfillopacity{0.888997}%
\pgfsetlinewidth{1.003750pt}%
\definecolor{currentstroke}{rgb}{0.121569,0.466667,0.705882}%
\pgfsetstrokecolor{currentstroke}%
\pgfsetstrokeopacity{0.888997}%
\pgfsetdash{}{0pt}%
\pgfpathmoveto{\pgfqpoint{1.250700in}{1.798628in}}%
\pgfpathcurveto{\pgfqpoint{1.258936in}{1.798628in}}{\pgfqpoint{1.266836in}{1.801901in}}{\pgfqpoint{1.272660in}{1.807725in}}%
\pgfpathcurveto{\pgfqpoint{1.278484in}{1.813548in}}{\pgfqpoint{1.281757in}{1.821449in}}{\pgfqpoint{1.281757in}{1.829685in}}%
\pgfpathcurveto{\pgfqpoint{1.281757in}{1.837921in}}{\pgfqpoint{1.278484in}{1.845821in}}{\pgfqpoint{1.272660in}{1.851645in}}%
\pgfpathcurveto{\pgfqpoint{1.266836in}{1.857469in}}{\pgfqpoint{1.258936in}{1.860741in}}{\pgfqpoint{1.250700in}{1.860741in}}%
\pgfpathcurveto{\pgfqpoint{1.242464in}{1.860741in}}{\pgfqpoint{1.234564in}{1.857469in}}{\pgfqpoint{1.228740in}{1.851645in}}%
\pgfpathcurveto{\pgfqpoint{1.222916in}{1.845821in}}{\pgfqpoint{1.219644in}{1.837921in}}{\pgfqpoint{1.219644in}{1.829685in}}%
\pgfpathcurveto{\pgfqpoint{1.219644in}{1.821449in}}{\pgfqpoint{1.222916in}{1.813548in}}{\pgfqpoint{1.228740in}{1.807725in}}%
\pgfpathcurveto{\pgfqpoint{1.234564in}{1.801901in}}{\pgfqpoint{1.242464in}{1.798628in}}{\pgfqpoint{1.250700in}{1.798628in}}%
\pgfpathclose%
\pgfusepath{stroke,fill}%
\end{pgfscope}%
\begin{pgfscope}%
\pgfpathrectangle{\pgfqpoint{0.100000in}{0.212622in}}{\pgfqpoint{3.696000in}{3.696000in}}%
\pgfusepath{clip}%
\pgfsetbuttcap%
\pgfsetroundjoin%
\definecolor{currentfill}{rgb}{0.121569,0.466667,0.705882}%
\pgfsetfillcolor{currentfill}%
\pgfsetfillopacity{0.889351}%
\pgfsetlinewidth{1.003750pt}%
\definecolor{currentstroke}{rgb}{0.121569,0.466667,0.705882}%
\pgfsetstrokecolor{currentstroke}%
\pgfsetstrokeopacity{0.889351}%
\pgfsetdash{}{0pt}%
\pgfpathmoveto{\pgfqpoint{1.289313in}{1.817613in}}%
\pgfpathcurveto{\pgfqpoint{1.297549in}{1.817613in}}{\pgfqpoint{1.305449in}{1.820885in}}{\pgfqpoint{1.311273in}{1.826709in}}%
\pgfpathcurveto{\pgfqpoint{1.317097in}{1.832533in}}{\pgfqpoint{1.320369in}{1.840433in}}{\pgfqpoint{1.320369in}{1.848669in}}%
\pgfpathcurveto{\pgfqpoint{1.320369in}{1.856905in}}{\pgfqpoint{1.317097in}{1.864805in}}{\pgfqpoint{1.311273in}{1.870629in}}%
\pgfpathcurveto{\pgfqpoint{1.305449in}{1.876453in}}{\pgfqpoint{1.297549in}{1.879726in}}{\pgfqpoint{1.289313in}{1.879726in}}%
\pgfpathcurveto{\pgfqpoint{1.281076in}{1.879726in}}{\pgfqpoint{1.273176in}{1.876453in}}{\pgfqpoint{1.267352in}{1.870629in}}%
\pgfpathcurveto{\pgfqpoint{1.261528in}{1.864805in}}{\pgfqpoint{1.258256in}{1.856905in}}{\pgfqpoint{1.258256in}{1.848669in}}%
\pgfpathcurveto{\pgfqpoint{1.258256in}{1.840433in}}{\pgfqpoint{1.261528in}{1.832533in}}{\pgfqpoint{1.267352in}{1.826709in}}%
\pgfpathcurveto{\pgfqpoint{1.273176in}{1.820885in}}{\pgfqpoint{1.281076in}{1.817613in}}{\pgfqpoint{1.289313in}{1.817613in}}%
\pgfpathclose%
\pgfusepath{stroke,fill}%
\end{pgfscope}%
\begin{pgfscope}%
\pgfpathrectangle{\pgfqpoint{0.100000in}{0.212622in}}{\pgfqpoint{3.696000in}{3.696000in}}%
\pgfusepath{clip}%
\pgfsetbuttcap%
\pgfsetroundjoin%
\definecolor{currentfill}{rgb}{0.121569,0.466667,0.705882}%
\pgfsetfillcolor{currentfill}%
\pgfsetfillopacity{0.889591}%
\pgfsetlinewidth{1.003750pt}%
\definecolor{currentstroke}{rgb}{0.121569,0.466667,0.705882}%
\pgfsetstrokecolor{currentstroke}%
\pgfsetstrokeopacity{0.889591}%
\pgfsetdash{}{0pt}%
\pgfpathmoveto{\pgfqpoint{1.801509in}{2.116109in}}%
\pgfpathcurveto{\pgfqpoint{1.809745in}{2.116109in}}{\pgfqpoint{1.817645in}{2.119381in}}{\pgfqpoint{1.823469in}{2.125205in}}%
\pgfpathcurveto{\pgfqpoint{1.829293in}{2.131029in}}{\pgfqpoint{1.832565in}{2.138929in}}{\pgfqpoint{1.832565in}{2.147166in}}%
\pgfpathcurveto{\pgfqpoint{1.832565in}{2.155402in}}{\pgfqpoint{1.829293in}{2.163302in}}{\pgfqpoint{1.823469in}{2.169126in}}%
\pgfpathcurveto{\pgfqpoint{1.817645in}{2.174950in}}{\pgfqpoint{1.809745in}{2.178222in}}{\pgfqpoint{1.801509in}{2.178222in}}%
\pgfpathcurveto{\pgfqpoint{1.793273in}{2.178222in}}{\pgfqpoint{1.785372in}{2.174950in}}{\pgfqpoint{1.779549in}{2.169126in}}%
\pgfpathcurveto{\pgfqpoint{1.773725in}{2.163302in}}{\pgfqpoint{1.770452in}{2.155402in}}{\pgfqpoint{1.770452in}{2.147166in}}%
\pgfpathcurveto{\pgfqpoint{1.770452in}{2.138929in}}{\pgfqpoint{1.773725in}{2.131029in}}{\pgfqpoint{1.779549in}{2.125205in}}%
\pgfpathcurveto{\pgfqpoint{1.785372in}{2.119381in}}{\pgfqpoint{1.793273in}{2.116109in}}{\pgfqpoint{1.801509in}{2.116109in}}%
\pgfpathclose%
\pgfusepath{stroke,fill}%
\end{pgfscope}%
\begin{pgfscope}%
\pgfpathrectangle{\pgfqpoint{0.100000in}{0.212622in}}{\pgfqpoint{3.696000in}{3.696000in}}%
\pgfusepath{clip}%
\pgfsetbuttcap%
\pgfsetroundjoin%
\definecolor{currentfill}{rgb}{0.121569,0.466667,0.705882}%
\pgfsetfillcolor{currentfill}%
\pgfsetfillopacity{0.890392}%
\pgfsetlinewidth{1.003750pt}%
\definecolor{currentstroke}{rgb}{0.121569,0.466667,0.705882}%
\pgfsetstrokecolor{currentstroke}%
\pgfsetstrokeopacity{0.890392}%
\pgfsetdash{}{0pt}%
\pgfpathmoveto{\pgfqpoint{2.352857in}{2.312825in}}%
\pgfpathcurveto{\pgfqpoint{2.361093in}{2.312825in}}{\pgfqpoint{2.368993in}{2.316097in}}{\pgfqpoint{2.374817in}{2.321921in}}%
\pgfpathcurveto{\pgfqpoint{2.380641in}{2.327745in}}{\pgfqpoint{2.383913in}{2.335645in}}{\pgfqpoint{2.383913in}{2.343881in}}%
\pgfpathcurveto{\pgfqpoint{2.383913in}{2.352118in}}{\pgfqpoint{2.380641in}{2.360018in}}{\pgfqpoint{2.374817in}{2.365842in}}%
\pgfpathcurveto{\pgfqpoint{2.368993in}{2.371665in}}{\pgfqpoint{2.361093in}{2.374938in}}{\pgfqpoint{2.352857in}{2.374938in}}%
\pgfpathcurveto{\pgfqpoint{2.344621in}{2.374938in}}{\pgfqpoint{2.336721in}{2.371665in}}{\pgfqpoint{2.330897in}{2.365842in}}%
\pgfpathcurveto{\pgfqpoint{2.325073in}{2.360018in}}{\pgfqpoint{2.321800in}{2.352118in}}{\pgfqpoint{2.321800in}{2.343881in}}%
\pgfpathcurveto{\pgfqpoint{2.321800in}{2.335645in}}{\pgfqpoint{2.325073in}{2.327745in}}{\pgfqpoint{2.330897in}{2.321921in}}%
\pgfpathcurveto{\pgfqpoint{2.336721in}{2.316097in}}{\pgfqpoint{2.344621in}{2.312825in}}{\pgfqpoint{2.352857in}{2.312825in}}%
\pgfpathclose%
\pgfusepath{stroke,fill}%
\end{pgfscope}%
\begin{pgfscope}%
\pgfpathrectangle{\pgfqpoint{0.100000in}{0.212622in}}{\pgfqpoint{3.696000in}{3.696000in}}%
\pgfusepath{clip}%
\pgfsetbuttcap%
\pgfsetroundjoin%
\definecolor{currentfill}{rgb}{0.121569,0.466667,0.705882}%
\pgfsetfillcolor{currentfill}%
\pgfsetfillopacity{0.890780}%
\pgfsetlinewidth{1.003750pt}%
\definecolor{currentstroke}{rgb}{0.121569,0.466667,0.705882}%
\pgfsetstrokecolor{currentstroke}%
\pgfsetstrokeopacity{0.890780}%
\pgfsetdash{}{0pt}%
\pgfpathmoveto{\pgfqpoint{3.071435in}{2.618449in}}%
\pgfpathcurveto{\pgfqpoint{3.079672in}{2.618449in}}{\pgfqpoint{3.087572in}{2.621721in}}{\pgfqpoint{3.093396in}{2.627545in}}%
\pgfpathcurveto{\pgfqpoint{3.099220in}{2.633369in}}{\pgfqpoint{3.102492in}{2.641269in}}{\pgfqpoint{3.102492in}{2.649505in}}%
\pgfpathcurveto{\pgfqpoint{3.102492in}{2.657742in}}{\pgfqpoint{3.099220in}{2.665642in}}{\pgfqpoint{3.093396in}{2.671466in}}%
\pgfpathcurveto{\pgfqpoint{3.087572in}{2.677290in}}{\pgfqpoint{3.079672in}{2.680562in}}{\pgfqpoint{3.071435in}{2.680562in}}%
\pgfpathcurveto{\pgfqpoint{3.063199in}{2.680562in}}{\pgfqpoint{3.055299in}{2.677290in}}{\pgfqpoint{3.049475in}{2.671466in}}%
\pgfpathcurveto{\pgfqpoint{3.043651in}{2.665642in}}{\pgfqpoint{3.040379in}{2.657742in}}{\pgfqpoint{3.040379in}{2.649505in}}%
\pgfpathcurveto{\pgfqpoint{3.040379in}{2.641269in}}{\pgfqpoint{3.043651in}{2.633369in}}{\pgfqpoint{3.049475in}{2.627545in}}%
\pgfpathcurveto{\pgfqpoint{3.055299in}{2.621721in}}{\pgfqpoint{3.063199in}{2.618449in}}{\pgfqpoint{3.071435in}{2.618449in}}%
\pgfpathclose%
\pgfusepath{stroke,fill}%
\end{pgfscope}%
\begin{pgfscope}%
\pgfpathrectangle{\pgfqpoint{0.100000in}{0.212622in}}{\pgfqpoint{3.696000in}{3.696000in}}%
\pgfusepath{clip}%
\pgfsetbuttcap%
\pgfsetroundjoin%
\definecolor{currentfill}{rgb}{0.121569,0.466667,0.705882}%
\pgfsetfillcolor{currentfill}%
\pgfsetfillopacity{0.890972}%
\pgfsetlinewidth{1.003750pt}%
\definecolor{currentstroke}{rgb}{0.121569,0.466667,0.705882}%
\pgfsetstrokecolor{currentstroke}%
\pgfsetstrokeopacity{0.890972}%
\pgfsetdash{}{0pt}%
\pgfpathmoveto{\pgfqpoint{2.700416in}{2.496158in}}%
\pgfpathcurveto{\pgfqpoint{2.708652in}{2.496158in}}{\pgfqpoint{2.716552in}{2.499431in}}{\pgfqpoint{2.722376in}{2.505254in}}%
\pgfpathcurveto{\pgfqpoint{2.728200in}{2.511078in}}{\pgfqpoint{2.731472in}{2.518978in}}{\pgfqpoint{2.731472in}{2.527215in}}%
\pgfpathcurveto{\pgfqpoint{2.731472in}{2.535451in}}{\pgfqpoint{2.728200in}{2.543351in}}{\pgfqpoint{2.722376in}{2.549175in}}%
\pgfpathcurveto{\pgfqpoint{2.716552in}{2.554999in}}{\pgfqpoint{2.708652in}{2.558271in}}{\pgfqpoint{2.700416in}{2.558271in}}%
\pgfpathcurveto{\pgfqpoint{2.692179in}{2.558271in}}{\pgfqpoint{2.684279in}{2.554999in}}{\pgfqpoint{2.678455in}{2.549175in}}%
\pgfpathcurveto{\pgfqpoint{2.672631in}{2.543351in}}{\pgfqpoint{2.669359in}{2.535451in}}{\pgfqpoint{2.669359in}{2.527215in}}%
\pgfpathcurveto{\pgfqpoint{2.669359in}{2.518978in}}{\pgfqpoint{2.672631in}{2.511078in}}{\pgfqpoint{2.678455in}{2.505254in}}%
\pgfpathcurveto{\pgfqpoint{2.684279in}{2.499431in}}{\pgfqpoint{2.692179in}{2.496158in}}{\pgfqpoint{2.700416in}{2.496158in}}%
\pgfpathclose%
\pgfusepath{stroke,fill}%
\end{pgfscope}%
\begin{pgfscope}%
\pgfpathrectangle{\pgfqpoint{0.100000in}{0.212622in}}{\pgfqpoint{3.696000in}{3.696000in}}%
\pgfusepath{clip}%
\pgfsetbuttcap%
\pgfsetroundjoin%
\definecolor{currentfill}{rgb}{0.121569,0.466667,0.705882}%
\pgfsetfillcolor{currentfill}%
\pgfsetfillopacity{0.891044}%
\pgfsetlinewidth{1.003750pt}%
\definecolor{currentstroke}{rgb}{0.121569,0.466667,0.705882}%
\pgfsetstrokecolor{currentstroke}%
\pgfsetstrokeopacity{0.891044}%
\pgfsetdash{}{0pt}%
\pgfpathmoveto{\pgfqpoint{1.802295in}{2.115894in}}%
\pgfpathcurveto{\pgfqpoint{1.810531in}{2.115894in}}{\pgfqpoint{1.818431in}{2.119166in}}{\pgfqpoint{1.824255in}{2.124990in}}%
\pgfpathcurveto{\pgfqpoint{1.830079in}{2.130814in}}{\pgfqpoint{1.833352in}{2.138714in}}{\pgfqpoint{1.833352in}{2.146950in}}%
\pgfpathcurveto{\pgfqpoint{1.833352in}{2.155187in}}{\pgfqpoint{1.830079in}{2.163087in}}{\pgfqpoint{1.824255in}{2.168911in}}%
\pgfpathcurveto{\pgfqpoint{1.818431in}{2.174735in}}{\pgfqpoint{1.810531in}{2.178007in}}{\pgfqpoint{1.802295in}{2.178007in}}%
\pgfpathcurveto{\pgfqpoint{1.794059in}{2.178007in}}{\pgfqpoint{1.786159in}{2.174735in}}{\pgfqpoint{1.780335in}{2.168911in}}%
\pgfpathcurveto{\pgfqpoint{1.774511in}{2.163087in}}{\pgfqpoint{1.771239in}{2.155187in}}{\pgfqpoint{1.771239in}{2.146950in}}%
\pgfpathcurveto{\pgfqpoint{1.771239in}{2.138714in}}{\pgfqpoint{1.774511in}{2.130814in}}{\pgfqpoint{1.780335in}{2.124990in}}%
\pgfpathcurveto{\pgfqpoint{1.786159in}{2.119166in}}{\pgfqpoint{1.794059in}{2.115894in}}{\pgfqpoint{1.802295in}{2.115894in}}%
\pgfpathclose%
\pgfusepath{stroke,fill}%
\end{pgfscope}%
\begin{pgfscope}%
\pgfpathrectangle{\pgfqpoint{0.100000in}{0.212622in}}{\pgfqpoint{3.696000in}{3.696000in}}%
\pgfusepath{clip}%
\pgfsetbuttcap%
\pgfsetroundjoin%
\definecolor{currentfill}{rgb}{0.121569,0.466667,0.705882}%
\pgfsetfillcolor{currentfill}%
\pgfsetfillopacity{0.891133}%
\pgfsetlinewidth{1.003750pt}%
\definecolor{currentstroke}{rgb}{0.121569,0.466667,0.705882}%
\pgfsetstrokecolor{currentstroke}%
\pgfsetstrokeopacity{0.891133}%
\pgfsetdash{}{0pt}%
\pgfpathmoveto{\pgfqpoint{1.268634in}{1.815539in}}%
\pgfpathcurveto{\pgfqpoint{1.276870in}{1.815539in}}{\pgfqpoint{1.284770in}{1.818811in}}{\pgfqpoint{1.290594in}{1.824635in}}%
\pgfpathcurveto{\pgfqpoint{1.296418in}{1.830459in}}{\pgfqpoint{1.299691in}{1.838359in}}{\pgfqpoint{1.299691in}{1.846595in}}%
\pgfpathcurveto{\pgfqpoint{1.299691in}{1.854832in}}{\pgfqpoint{1.296418in}{1.862732in}}{\pgfqpoint{1.290594in}{1.868556in}}%
\pgfpathcurveto{\pgfqpoint{1.284770in}{1.874379in}}{\pgfqpoint{1.276870in}{1.877652in}}{\pgfqpoint{1.268634in}{1.877652in}}%
\pgfpathcurveto{\pgfqpoint{1.260398in}{1.877652in}}{\pgfqpoint{1.252498in}{1.874379in}}{\pgfqpoint{1.246674in}{1.868556in}}%
\pgfpathcurveto{\pgfqpoint{1.240850in}{1.862732in}}{\pgfqpoint{1.237578in}{1.854832in}}{\pgfqpoint{1.237578in}{1.846595in}}%
\pgfpathcurveto{\pgfqpoint{1.237578in}{1.838359in}}{\pgfqpoint{1.240850in}{1.830459in}}{\pgfqpoint{1.246674in}{1.824635in}}%
\pgfpathcurveto{\pgfqpoint{1.252498in}{1.818811in}}{\pgfqpoint{1.260398in}{1.815539in}}{\pgfqpoint{1.268634in}{1.815539in}}%
\pgfpathclose%
\pgfusepath{stroke,fill}%
\end{pgfscope}%
\begin{pgfscope}%
\pgfpathrectangle{\pgfqpoint{0.100000in}{0.212622in}}{\pgfqpoint{3.696000in}{3.696000in}}%
\pgfusepath{clip}%
\pgfsetbuttcap%
\pgfsetroundjoin%
\definecolor{currentfill}{rgb}{0.121569,0.466667,0.705882}%
\pgfsetfillcolor{currentfill}%
\pgfsetfillopacity{0.891159}%
\pgfsetlinewidth{1.003750pt}%
\definecolor{currentstroke}{rgb}{0.121569,0.466667,0.705882}%
\pgfsetstrokecolor{currentstroke}%
\pgfsetstrokeopacity{0.891159}%
\pgfsetdash{}{0pt}%
\pgfpathmoveto{\pgfqpoint{2.516936in}{2.396790in}}%
\pgfpathcurveto{\pgfqpoint{2.525172in}{2.396790in}}{\pgfqpoint{2.533072in}{2.400062in}}{\pgfqpoint{2.538896in}{2.405886in}}%
\pgfpathcurveto{\pgfqpoint{2.544720in}{2.411710in}}{\pgfqpoint{2.547992in}{2.419610in}}{\pgfqpoint{2.547992in}{2.427846in}}%
\pgfpathcurveto{\pgfqpoint{2.547992in}{2.436083in}}{\pgfqpoint{2.544720in}{2.443983in}}{\pgfqpoint{2.538896in}{2.449807in}}%
\pgfpathcurveto{\pgfqpoint{2.533072in}{2.455630in}}{\pgfqpoint{2.525172in}{2.458903in}}{\pgfqpoint{2.516936in}{2.458903in}}%
\pgfpathcurveto{\pgfqpoint{2.508699in}{2.458903in}}{\pgfqpoint{2.500799in}{2.455630in}}{\pgfqpoint{2.494975in}{2.449807in}}%
\pgfpathcurveto{\pgfqpoint{2.489151in}{2.443983in}}{\pgfqpoint{2.485879in}{2.436083in}}{\pgfqpoint{2.485879in}{2.427846in}}%
\pgfpathcurveto{\pgfqpoint{2.485879in}{2.419610in}}{\pgfqpoint{2.489151in}{2.411710in}}{\pgfqpoint{2.494975in}{2.405886in}}%
\pgfpathcurveto{\pgfqpoint{2.500799in}{2.400062in}}{\pgfqpoint{2.508699in}{2.396790in}}{\pgfqpoint{2.516936in}{2.396790in}}%
\pgfpathclose%
\pgfusepath{stroke,fill}%
\end{pgfscope}%
\begin{pgfscope}%
\pgfpathrectangle{\pgfqpoint{0.100000in}{0.212622in}}{\pgfqpoint{3.696000in}{3.696000in}}%
\pgfusepath{clip}%
\pgfsetbuttcap%
\pgfsetroundjoin%
\definecolor{currentfill}{rgb}{0.121569,0.466667,0.705882}%
\pgfsetfillcolor{currentfill}%
\pgfsetfillopacity{0.892399}%
\pgfsetlinewidth{1.003750pt}%
\definecolor{currentstroke}{rgb}{0.121569,0.466667,0.705882}%
\pgfsetstrokecolor{currentstroke}%
\pgfsetstrokeopacity{0.892399}%
\pgfsetdash{}{0pt}%
\pgfpathmoveto{\pgfqpoint{1.288626in}{1.816818in}}%
\pgfpathcurveto{\pgfqpoint{1.296863in}{1.816818in}}{\pgfqpoint{1.304763in}{1.820090in}}{\pgfqpoint{1.310587in}{1.825914in}}%
\pgfpathcurveto{\pgfqpoint{1.316410in}{1.831738in}}{\pgfqpoint{1.319683in}{1.839638in}}{\pgfqpoint{1.319683in}{1.847874in}}%
\pgfpathcurveto{\pgfqpoint{1.319683in}{1.856111in}}{\pgfqpoint{1.316410in}{1.864011in}}{\pgfqpoint{1.310587in}{1.869835in}}%
\pgfpathcurveto{\pgfqpoint{1.304763in}{1.875659in}}{\pgfqpoint{1.296863in}{1.878931in}}{\pgfqpoint{1.288626in}{1.878931in}}%
\pgfpathcurveto{\pgfqpoint{1.280390in}{1.878931in}}{\pgfqpoint{1.272490in}{1.875659in}}{\pgfqpoint{1.266666in}{1.869835in}}%
\pgfpathcurveto{\pgfqpoint{1.260842in}{1.864011in}}{\pgfqpoint{1.257570in}{1.856111in}}{\pgfqpoint{1.257570in}{1.847874in}}%
\pgfpathcurveto{\pgfqpoint{1.257570in}{1.839638in}}{\pgfqpoint{1.260842in}{1.831738in}}{\pgfqpoint{1.266666in}{1.825914in}}%
\pgfpathcurveto{\pgfqpoint{1.272490in}{1.820090in}}{\pgfqpoint{1.280390in}{1.816818in}}{\pgfqpoint{1.288626in}{1.816818in}}%
\pgfpathclose%
\pgfusepath{stroke,fill}%
\end{pgfscope}%
\begin{pgfscope}%
\pgfpathrectangle{\pgfqpoint{0.100000in}{0.212622in}}{\pgfqpoint{3.696000in}{3.696000in}}%
\pgfusepath{clip}%
\pgfsetbuttcap%
\pgfsetroundjoin%
\definecolor{currentfill}{rgb}{0.121569,0.466667,0.705882}%
\pgfsetfillcolor{currentfill}%
\pgfsetfillopacity{0.892478}%
\pgfsetlinewidth{1.003750pt}%
\definecolor{currentstroke}{rgb}{0.121569,0.466667,0.705882}%
\pgfsetstrokecolor{currentstroke}%
\pgfsetstrokeopacity{0.892478}%
\pgfsetdash{}{0pt}%
\pgfpathmoveto{\pgfqpoint{1.804538in}{2.115305in}}%
\pgfpathcurveto{\pgfqpoint{1.812774in}{2.115305in}}{\pgfqpoint{1.820675in}{2.118578in}}{\pgfqpoint{1.826498in}{2.124402in}}%
\pgfpathcurveto{\pgfqpoint{1.832322in}{2.130226in}}{\pgfqpoint{1.835595in}{2.138126in}}{\pgfqpoint{1.835595in}{2.146362in}}%
\pgfpathcurveto{\pgfqpoint{1.835595in}{2.154598in}}{\pgfqpoint{1.832322in}{2.162498in}}{\pgfqpoint{1.826498in}{2.168322in}}%
\pgfpathcurveto{\pgfqpoint{1.820675in}{2.174146in}}{\pgfqpoint{1.812774in}{2.177418in}}{\pgfqpoint{1.804538in}{2.177418in}}%
\pgfpathcurveto{\pgfqpoint{1.796302in}{2.177418in}}{\pgfqpoint{1.788402in}{2.174146in}}{\pgfqpoint{1.782578in}{2.168322in}}%
\pgfpathcurveto{\pgfqpoint{1.776754in}{2.162498in}}{\pgfqpoint{1.773482in}{2.154598in}}{\pgfqpoint{1.773482in}{2.146362in}}%
\pgfpathcurveto{\pgfqpoint{1.773482in}{2.138126in}}{\pgfqpoint{1.776754in}{2.130226in}}{\pgfqpoint{1.782578in}{2.124402in}}%
\pgfpathcurveto{\pgfqpoint{1.788402in}{2.118578in}}{\pgfqpoint{1.796302in}{2.115305in}}{\pgfqpoint{1.804538in}{2.115305in}}%
\pgfpathclose%
\pgfusepath{stroke,fill}%
\end{pgfscope}%
\begin{pgfscope}%
\pgfpathrectangle{\pgfqpoint{0.100000in}{0.212622in}}{\pgfqpoint{3.696000in}{3.696000in}}%
\pgfusepath{clip}%
\pgfsetbuttcap%
\pgfsetroundjoin%
\definecolor{currentfill}{rgb}{0.121569,0.466667,0.705882}%
\pgfsetfillcolor{currentfill}%
\pgfsetfillopacity{0.892991}%
\pgfsetlinewidth{1.003750pt}%
\definecolor{currentstroke}{rgb}{0.121569,0.466667,0.705882}%
\pgfsetstrokecolor{currentstroke}%
\pgfsetstrokeopacity{0.892991}%
\pgfsetdash{}{0pt}%
\pgfpathmoveto{\pgfqpoint{1.786973in}{2.106748in}}%
\pgfpathcurveto{\pgfqpoint{1.795209in}{2.106748in}}{\pgfqpoint{1.803109in}{2.110020in}}{\pgfqpoint{1.808933in}{2.115844in}}%
\pgfpathcurveto{\pgfqpoint{1.814757in}{2.121668in}}{\pgfqpoint{1.818029in}{2.129568in}}{\pgfqpoint{1.818029in}{2.137804in}}%
\pgfpathcurveto{\pgfqpoint{1.818029in}{2.146041in}}{\pgfqpoint{1.814757in}{2.153941in}}{\pgfqpoint{1.808933in}{2.159765in}}%
\pgfpathcurveto{\pgfqpoint{1.803109in}{2.165589in}}{\pgfqpoint{1.795209in}{2.168861in}}{\pgfqpoint{1.786973in}{2.168861in}}%
\pgfpathcurveto{\pgfqpoint{1.778737in}{2.168861in}}{\pgfqpoint{1.770837in}{2.165589in}}{\pgfqpoint{1.765013in}{2.159765in}}%
\pgfpathcurveto{\pgfqpoint{1.759189in}{2.153941in}}{\pgfqpoint{1.755916in}{2.146041in}}{\pgfqpoint{1.755916in}{2.137804in}}%
\pgfpathcurveto{\pgfqpoint{1.755916in}{2.129568in}}{\pgfqpoint{1.759189in}{2.121668in}}{\pgfqpoint{1.765013in}{2.115844in}}%
\pgfpathcurveto{\pgfqpoint{1.770837in}{2.110020in}}{\pgfqpoint{1.778737in}{2.106748in}}{\pgfqpoint{1.786973in}{2.106748in}}%
\pgfpathclose%
\pgfusepath{stroke,fill}%
\end{pgfscope}%
\begin{pgfscope}%
\pgfpathrectangle{\pgfqpoint{0.100000in}{0.212622in}}{\pgfqpoint{3.696000in}{3.696000in}}%
\pgfusepath{clip}%
\pgfsetbuttcap%
\pgfsetroundjoin%
\definecolor{currentfill}{rgb}{0.121569,0.466667,0.705882}%
\pgfsetfillcolor{currentfill}%
\pgfsetfillopacity{0.893132}%
\pgfsetlinewidth{1.003750pt}%
\definecolor{currentstroke}{rgb}{0.121569,0.466667,0.705882}%
\pgfsetstrokecolor{currentstroke}%
\pgfsetstrokeopacity{0.893132}%
\pgfsetdash{}{0pt}%
\pgfpathmoveto{\pgfqpoint{2.348585in}{2.306500in}}%
\pgfpathcurveto{\pgfqpoint{2.356821in}{2.306500in}}{\pgfqpoint{2.364721in}{2.309773in}}{\pgfqpoint{2.370545in}{2.315597in}}%
\pgfpathcurveto{\pgfqpoint{2.376369in}{2.321421in}}{\pgfqpoint{2.379641in}{2.329321in}}{\pgfqpoint{2.379641in}{2.337557in}}%
\pgfpathcurveto{\pgfqpoint{2.379641in}{2.345793in}}{\pgfqpoint{2.376369in}{2.353693in}}{\pgfqpoint{2.370545in}{2.359517in}}%
\pgfpathcurveto{\pgfqpoint{2.364721in}{2.365341in}}{\pgfqpoint{2.356821in}{2.368613in}}{\pgfqpoint{2.348585in}{2.368613in}}%
\pgfpathcurveto{\pgfqpoint{2.340348in}{2.368613in}}{\pgfqpoint{2.332448in}{2.365341in}}{\pgfqpoint{2.326624in}{2.359517in}}%
\pgfpathcurveto{\pgfqpoint{2.320800in}{2.353693in}}{\pgfqpoint{2.317528in}{2.345793in}}{\pgfqpoint{2.317528in}{2.337557in}}%
\pgfpathcurveto{\pgfqpoint{2.317528in}{2.329321in}}{\pgfqpoint{2.320800in}{2.321421in}}{\pgfqpoint{2.326624in}{2.315597in}}%
\pgfpathcurveto{\pgfqpoint{2.332448in}{2.309773in}}{\pgfqpoint{2.340348in}{2.306500in}}{\pgfqpoint{2.348585in}{2.306500in}}%
\pgfpathclose%
\pgfusepath{stroke,fill}%
\end{pgfscope}%
\begin{pgfscope}%
\pgfpathrectangle{\pgfqpoint{0.100000in}{0.212622in}}{\pgfqpoint{3.696000in}{3.696000in}}%
\pgfusepath{clip}%
\pgfsetbuttcap%
\pgfsetroundjoin%
\definecolor{currentfill}{rgb}{0.121569,0.466667,0.705882}%
\pgfsetfillcolor{currentfill}%
\pgfsetfillopacity{0.893184}%
\pgfsetlinewidth{1.003750pt}%
\definecolor{currentstroke}{rgb}{0.121569,0.466667,0.705882}%
\pgfsetstrokecolor{currentstroke}%
\pgfsetstrokeopacity{0.893184}%
\pgfsetdash{}{0pt}%
\pgfpathmoveto{\pgfqpoint{2.502398in}{2.382366in}}%
\pgfpathcurveto{\pgfqpoint{2.510635in}{2.382366in}}{\pgfqpoint{2.518535in}{2.385638in}}{\pgfqpoint{2.524359in}{2.391462in}}%
\pgfpathcurveto{\pgfqpoint{2.530183in}{2.397286in}}{\pgfqpoint{2.533455in}{2.405186in}}{\pgfqpoint{2.533455in}{2.413422in}}%
\pgfpathcurveto{\pgfqpoint{2.533455in}{2.421658in}}{\pgfqpoint{2.530183in}{2.429558in}}{\pgfqpoint{2.524359in}{2.435382in}}%
\pgfpathcurveto{\pgfqpoint{2.518535in}{2.441206in}}{\pgfqpoint{2.510635in}{2.444479in}}{\pgfqpoint{2.502398in}{2.444479in}}%
\pgfpathcurveto{\pgfqpoint{2.494162in}{2.444479in}}{\pgfqpoint{2.486262in}{2.441206in}}{\pgfqpoint{2.480438in}{2.435382in}}%
\pgfpathcurveto{\pgfqpoint{2.474614in}{2.429558in}}{\pgfqpoint{2.471342in}{2.421658in}}{\pgfqpoint{2.471342in}{2.413422in}}%
\pgfpathcurveto{\pgfqpoint{2.471342in}{2.405186in}}{\pgfqpoint{2.474614in}{2.397286in}}{\pgfqpoint{2.480438in}{2.391462in}}%
\pgfpathcurveto{\pgfqpoint{2.486262in}{2.385638in}}{\pgfqpoint{2.494162in}{2.382366in}}{\pgfqpoint{2.502398in}{2.382366in}}%
\pgfpathclose%
\pgfusepath{stroke,fill}%
\end{pgfscope}%
\begin{pgfscope}%
\pgfpathrectangle{\pgfqpoint{0.100000in}{0.212622in}}{\pgfqpoint{3.696000in}{3.696000in}}%
\pgfusepath{clip}%
\pgfsetbuttcap%
\pgfsetroundjoin%
\definecolor{currentfill}{rgb}{0.121569,0.466667,0.705882}%
\pgfsetfillcolor{currentfill}%
\pgfsetfillopacity{0.893294}%
\pgfsetlinewidth{1.003750pt}%
\definecolor{currentstroke}{rgb}{0.121569,0.466667,0.705882}%
\pgfsetstrokecolor{currentstroke}%
\pgfsetstrokeopacity{0.893294}%
\pgfsetdash{}{0pt}%
\pgfpathmoveto{\pgfqpoint{2.492082in}{2.375990in}}%
\pgfpathcurveto{\pgfqpoint{2.500319in}{2.375990in}}{\pgfqpoint{2.508219in}{2.379263in}}{\pgfqpoint{2.514043in}{2.385087in}}%
\pgfpathcurveto{\pgfqpoint{2.519867in}{2.390911in}}{\pgfqpoint{2.523139in}{2.398811in}}{\pgfqpoint{2.523139in}{2.407047in}}%
\pgfpathcurveto{\pgfqpoint{2.523139in}{2.415283in}}{\pgfqpoint{2.519867in}{2.423183in}}{\pgfqpoint{2.514043in}{2.429007in}}%
\pgfpathcurveto{\pgfqpoint{2.508219in}{2.434831in}}{\pgfqpoint{2.500319in}{2.438103in}}{\pgfqpoint{2.492082in}{2.438103in}}%
\pgfpathcurveto{\pgfqpoint{2.483846in}{2.438103in}}{\pgfqpoint{2.475946in}{2.434831in}}{\pgfqpoint{2.470122in}{2.429007in}}%
\pgfpathcurveto{\pgfqpoint{2.464298in}{2.423183in}}{\pgfqpoint{2.461026in}{2.415283in}}{\pgfqpoint{2.461026in}{2.407047in}}%
\pgfpathcurveto{\pgfqpoint{2.461026in}{2.398811in}}{\pgfqpoint{2.464298in}{2.390911in}}{\pgfqpoint{2.470122in}{2.385087in}}%
\pgfpathcurveto{\pgfqpoint{2.475946in}{2.379263in}}{\pgfqpoint{2.483846in}{2.375990in}}{\pgfqpoint{2.492082in}{2.375990in}}%
\pgfpathclose%
\pgfusepath{stroke,fill}%
\end{pgfscope}%
\begin{pgfscope}%
\pgfpathrectangle{\pgfqpoint{0.100000in}{0.212622in}}{\pgfqpoint{3.696000in}{3.696000in}}%
\pgfusepath{clip}%
\pgfsetbuttcap%
\pgfsetroundjoin%
\definecolor{currentfill}{rgb}{0.121569,0.466667,0.705882}%
\pgfsetfillcolor{currentfill}%
\pgfsetfillopacity{0.893591}%
\pgfsetlinewidth{1.003750pt}%
\definecolor{currentstroke}{rgb}{0.121569,0.466667,0.705882}%
\pgfsetstrokecolor{currentstroke}%
\pgfsetstrokeopacity{0.893591}%
\pgfsetdash{}{0pt}%
\pgfpathmoveto{\pgfqpoint{1.811593in}{2.118488in}}%
\pgfpathcurveto{\pgfqpoint{1.819829in}{2.118488in}}{\pgfqpoint{1.827729in}{2.121760in}}{\pgfqpoint{1.833553in}{2.127584in}}%
\pgfpathcurveto{\pgfqpoint{1.839377in}{2.133408in}}{\pgfqpoint{1.842649in}{2.141308in}}{\pgfqpoint{1.842649in}{2.149544in}}%
\pgfpathcurveto{\pgfqpoint{1.842649in}{2.157780in}}{\pgfqpoint{1.839377in}{2.165680in}}{\pgfqpoint{1.833553in}{2.171504in}}%
\pgfpathcurveto{\pgfqpoint{1.827729in}{2.177328in}}{\pgfqpoint{1.819829in}{2.180601in}}{\pgfqpoint{1.811593in}{2.180601in}}%
\pgfpathcurveto{\pgfqpoint{1.803357in}{2.180601in}}{\pgfqpoint{1.795457in}{2.177328in}}{\pgfqpoint{1.789633in}{2.171504in}}%
\pgfpathcurveto{\pgfqpoint{1.783809in}{2.165680in}}{\pgfqpoint{1.780536in}{2.157780in}}{\pgfqpoint{1.780536in}{2.149544in}}%
\pgfpathcurveto{\pgfqpoint{1.780536in}{2.141308in}}{\pgfqpoint{1.783809in}{2.133408in}}{\pgfqpoint{1.789633in}{2.127584in}}%
\pgfpathcurveto{\pgfqpoint{1.795457in}{2.121760in}}{\pgfqpoint{1.803357in}{2.118488in}}{\pgfqpoint{1.811593in}{2.118488in}}%
\pgfpathclose%
\pgfusepath{stroke,fill}%
\end{pgfscope}%
\begin{pgfscope}%
\pgfpathrectangle{\pgfqpoint{0.100000in}{0.212622in}}{\pgfqpoint{3.696000in}{3.696000in}}%
\pgfusepath{clip}%
\pgfsetbuttcap%
\pgfsetroundjoin%
\definecolor{currentfill}{rgb}{0.121569,0.466667,0.705882}%
\pgfsetfillcolor{currentfill}%
\pgfsetfillopacity{0.893941}%
\pgfsetlinewidth{1.003750pt}%
\definecolor{currentstroke}{rgb}{0.121569,0.466667,0.705882}%
\pgfsetstrokecolor{currentstroke}%
\pgfsetstrokeopacity{0.893941}%
\pgfsetdash{}{0pt}%
\pgfpathmoveto{\pgfqpoint{2.534340in}{2.401460in}}%
\pgfpathcurveto{\pgfqpoint{2.542576in}{2.401460in}}{\pgfqpoint{2.550476in}{2.404732in}}{\pgfqpoint{2.556300in}{2.410556in}}%
\pgfpathcurveto{\pgfqpoint{2.562124in}{2.416380in}}{\pgfqpoint{2.565397in}{2.424280in}}{\pgfqpoint{2.565397in}{2.432516in}}%
\pgfpathcurveto{\pgfqpoint{2.565397in}{2.440753in}}{\pgfqpoint{2.562124in}{2.448653in}}{\pgfqpoint{2.556300in}{2.454477in}}%
\pgfpathcurveto{\pgfqpoint{2.550476in}{2.460301in}}{\pgfqpoint{2.542576in}{2.463573in}}{\pgfqpoint{2.534340in}{2.463573in}}%
\pgfpathcurveto{\pgfqpoint{2.526104in}{2.463573in}}{\pgfqpoint{2.518204in}{2.460301in}}{\pgfqpoint{2.512380in}{2.454477in}}%
\pgfpathcurveto{\pgfqpoint{2.506556in}{2.448653in}}{\pgfqpoint{2.503284in}{2.440753in}}{\pgfqpoint{2.503284in}{2.432516in}}%
\pgfpathcurveto{\pgfqpoint{2.503284in}{2.424280in}}{\pgfqpoint{2.506556in}{2.416380in}}{\pgfqpoint{2.512380in}{2.410556in}}%
\pgfpathcurveto{\pgfqpoint{2.518204in}{2.404732in}}{\pgfqpoint{2.526104in}{2.401460in}}{\pgfqpoint{2.534340in}{2.401460in}}%
\pgfpathclose%
\pgfusepath{stroke,fill}%
\end{pgfscope}%
\begin{pgfscope}%
\pgfpathrectangle{\pgfqpoint{0.100000in}{0.212622in}}{\pgfqpoint{3.696000in}{3.696000in}}%
\pgfusepath{clip}%
\pgfsetbuttcap%
\pgfsetroundjoin%
\definecolor{currentfill}{rgb}{0.121569,0.466667,0.705882}%
\pgfsetfillcolor{currentfill}%
\pgfsetfillopacity{0.894382}%
\pgfsetlinewidth{1.003750pt}%
\definecolor{currentstroke}{rgb}{0.121569,0.466667,0.705882}%
\pgfsetstrokecolor{currentstroke}%
\pgfsetstrokeopacity{0.894382}%
\pgfsetdash{}{0pt}%
\pgfpathmoveto{\pgfqpoint{2.495466in}{2.377525in}}%
\pgfpathcurveto{\pgfqpoint{2.503703in}{2.377525in}}{\pgfqpoint{2.511603in}{2.380797in}}{\pgfqpoint{2.517427in}{2.386621in}}%
\pgfpathcurveto{\pgfqpoint{2.523251in}{2.392445in}}{\pgfqpoint{2.526523in}{2.400345in}}{\pgfqpoint{2.526523in}{2.408581in}}%
\pgfpathcurveto{\pgfqpoint{2.526523in}{2.416818in}}{\pgfqpoint{2.523251in}{2.424718in}}{\pgfqpoint{2.517427in}{2.430542in}}%
\pgfpathcurveto{\pgfqpoint{2.511603in}{2.436365in}}{\pgfqpoint{2.503703in}{2.439638in}}{\pgfqpoint{2.495466in}{2.439638in}}%
\pgfpathcurveto{\pgfqpoint{2.487230in}{2.439638in}}{\pgfqpoint{2.479330in}{2.436365in}}{\pgfqpoint{2.473506in}{2.430542in}}%
\pgfpathcurveto{\pgfqpoint{2.467682in}{2.424718in}}{\pgfqpoint{2.464410in}{2.416818in}}{\pgfqpoint{2.464410in}{2.408581in}}%
\pgfpathcurveto{\pgfqpoint{2.464410in}{2.400345in}}{\pgfqpoint{2.467682in}{2.392445in}}{\pgfqpoint{2.473506in}{2.386621in}}%
\pgfpathcurveto{\pgfqpoint{2.479330in}{2.380797in}}{\pgfqpoint{2.487230in}{2.377525in}}{\pgfqpoint{2.495466in}{2.377525in}}%
\pgfpathclose%
\pgfusepath{stroke,fill}%
\end{pgfscope}%
\begin{pgfscope}%
\pgfpathrectangle{\pgfqpoint{0.100000in}{0.212622in}}{\pgfqpoint{3.696000in}{3.696000in}}%
\pgfusepath{clip}%
\pgfsetbuttcap%
\pgfsetroundjoin%
\definecolor{currentfill}{rgb}{0.121569,0.466667,0.705882}%
\pgfsetfillcolor{currentfill}%
\pgfsetfillopacity{0.894488}%
\pgfsetlinewidth{1.003750pt}%
\definecolor{currentstroke}{rgb}{0.121569,0.466667,0.705882}%
\pgfsetstrokecolor{currentstroke}%
\pgfsetstrokeopacity{0.894488}%
\pgfsetdash{}{0pt}%
\pgfpathmoveto{\pgfqpoint{2.493500in}{2.377057in}}%
\pgfpathcurveto{\pgfqpoint{2.501736in}{2.377057in}}{\pgfqpoint{2.509636in}{2.380329in}}{\pgfqpoint{2.515460in}{2.386153in}}%
\pgfpathcurveto{\pgfqpoint{2.521284in}{2.391977in}}{\pgfqpoint{2.524556in}{2.399877in}}{\pgfqpoint{2.524556in}{2.408113in}}%
\pgfpathcurveto{\pgfqpoint{2.524556in}{2.416350in}}{\pgfqpoint{2.521284in}{2.424250in}}{\pgfqpoint{2.515460in}{2.430074in}}%
\pgfpathcurveto{\pgfqpoint{2.509636in}{2.435897in}}{\pgfqpoint{2.501736in}{2.439170in}}{\pgfqpoint{2.493500in}{2.439170in}}%
\pgfpathcurveto{\pgfqpoint{2.485263in}{2.439170in}}{\pgfqpoint{2.477363in}{2.435897in}}{\pgfqpoint{2.471539in}{2.430074in}}%
\pgfpathcurveto{\pgfqpoint{2.465715in}{2.424250in}}{\pgfqpoint{2.462443in}{2.416350in}}{\pgfqpoint{2.462443in}{2.408113in}}%
\pgfpathcurveto{\pgfqpoint{2.462443in}{2.399877in}}{\pgfqpoint{2.465715in}{2.391977in}}{\pgfqpoint{2.471539in}{2.386153in}}%
\pgfpathcurveto{\pgfqpoint{2.477363in}{2.380329in}}{\pgfqpoint{2.485263in}{2.377057in}}{\pgfqpoint{2.493500in}{2.377057in}}%
\pgfpathclose%
\pgfusepath{stroke,fill}%
\end{pgfscope}%
\begin{pgfscope}%
\pgfpathrectangle{\pgfqpoint{0.100000in}{0.212622in}}{\pgfqpoint{3.696000in}{3.696000in}}%
\pgfusepath{clip}%
\pgfsetbuttcap%
\pgfsetroundjoin%
\definecolor{currentfill}{rgb}{0.121569,0.466667,0.705882}%
\pgfsetfillcolor{currentfill}%
\pgfsetfillopacity{0.894761}%
\pgfsetlinewidth{1.003750pt}%
\definecolor{currentstroke}{rgb}{0.121569,0.466667,0.705882}%
\pgfsetstrokecolor{currentstroke}%
\pgfsetstrokeopacity{0.894761}%
\pgfsetdash{}{0pt}%
\pgfpathmoveto{\pgfqpoint{2.465036in}{2.343649in}}%
\pgfpathcurveto{\pgfqpoint{2.473272in}{2.343649in}}{\pgfqpoint{2.481172in}{2.346922in}}{\pgfqpoint{2.486996in}{2.352746in}}%
\pgfpathcurveto{\pgfqpoint{2.492820in}{2.358570in}}{\pgfqpoint{2.496093in}{2.366470in}}{\pgfqpoint{2.496093in}{2.374706in}}%
\pgfpathcurveto{\pgfqpoint{2.496093in}{2.382942in}}{\pgfqpoint{2.492820in}{2.390842in}}{\pgfqpoint{2.486996in}{2.396666in}}%
\pgfpathcurveto{\pgfqpoint{2.481172in}{2.402490in}}{\pgfqpoint{2.473272in}{2.405762in}}{\pgfqpoint{2.465036in}{2.405762in}}%
\pgfpathcurveto{\pgfqpoint{2.456800in}{2.405762in}}{\pgfqpoint{2.448900in}{2.402490in}}{\pgfqpoint{2.443076in}{2.396666in}}%
\pgfpathcurveto{\pgfqpoint{2.437252in}{2.390842in}}{\pgfqpoint{2.433980in}{2.382942in}}{\pgfqpoint{2.433980in}{2.374706in}}%
\pgfpathcurveto{\pgfqpoint{2.433980in}{2.366470in}}{\pgfqpoint{2.437252in}{2.358570in}}{\pgfqpoint{2.443076in}{2.352746in}}%
\pgfpathcurveto{\pgfqpoint{2.448900in}{2.346922in}}{\pgfqpoint{2.456800in}{2.343649in}}{\pgfqpoint{2.465036in}{2.343649in}}%
\pgfpathclose%
\pgfusepath{stroke,fill}%
\end{pgfscope}%
\begin{pgfscope}%
\pgfpathrectangle{\pgfqpoint{0.100000in}{0.212622in}}{\pgfqpoint{3.696000in}{3.696000in}}%
\pgfusepath{clip}%
\pgfsetbuttcap%
\pgfsetroundjoin%
\definecolor{currentfill}{rgb}{0.121569,0.466667,0.705882}%
\pgfsetfillcolor{currentfill}%
\pgfsetfillopacity{0.895181}%
\pgfsetlinewidth{1.003750pt}%
\definecolor{currentstroke}{rgb}{0.121569,0.466667,0.705882}%
\pgfsetstrokecolor{currentstroke}%
\pgfsetstrokeopacity{0.895181}%
\pgfsetdash{}{0pt}%
\pgfpathmoveto{\pgfqpoint{1.768318in}{2.093149in}}%
\pgfpathcurveto{\pgfqpoint{1.776555in}{2.093149in}}{\pgfqpoint{1.784455in}{2.096422in}}{\pgfqpoint{1.790279in}{2.102246in}}%
\pgfpathcurveto{\pgfqpoint{1.796103in}{2.108070in}}{\pgfqpoint{1.799375in}{2.115970in}}{\pgfqpoint{1.799375in}{2.124206in}}%
\pgfpathcurveto{\pgfqpoint{1.799375in}{2.132442in}}{\pgfqpoint{1.796103in}{2.140342in}}{\pgfqpoint{1.790279in}{2.146166in}}%
\pgfpathcurveto{\pgfqpoint{1.784455in}{2.151990in}}{\pgfqpoint{1.776555in}{2.155262in}}{\pgfqpoint{1.768318in}{2.155262in}}%
\pgfpathcurveto{\pgfqpoint{1.760082in}{2.155262in}}{\pgfqpoint{1.752182in}{2.151990in}}{\pgfqpoint{1.746358in}{2.146166in}}%
\pgfpathcurveto{\pgfqpoint{1.740534in}{2.140342in}}{\pgfqpoint{1.737262in}{2.132442in}}{\pgfqpoint{1.737262in}{2.124206in}}%
\pgfpathcurveto{\pgfqpoint{1.737262in}{2.115970in}}{\pgfqpoint{1.740534in}{2.108070in}}{\pgfqpoint{1.746358in}{2.102246in}}%
\pgfpathcurveto{\pgfqpoint{1.752182in}{2.096422in}}{\pgfqpoint{1.760082in}{2.093149in}}{\pgfqpoint{1.768318in}{2.093149in}}%
\pgfpathclose%
\pgfusepath{stroke,fill}%
\end{pgfscope}%
\begin{pgfscope}%
\pgfpathrectangle{\pgfqpoint{0.100000in}{0.212622in}}{\pgfqpoint{3.696000in}{3.696000in}}%
\pgfusepath{clip}%
\pgfsetbuttcap%
\pgfsetroundjoin%
\definecolor{currentfill}{rgb}{0.121569,0.466667,0.705882}%
\pgfsetfillcolor{currentfill}%
\pgfsetfillopacity{0.895536}%
\pgfsetlinewidth{1.003750pt}%
\definecolor{currentstroke}{rgb}{0.121569,0.466667,0.705882}%
\pgfsetstrokecolor{currentstroke}%
\pgfsetstrokeopacity{0.895536}%
\pgfsetdash{}{0pt}%
\pgfpathmoveto{\pgfqpoint{2.617929in}{2.461330in}}%
\pgfpathcurveto{\pgfqpoint{2.626165in}{2.461330in}}{\pgfqpoint{2.634065in}{2.464603in}}{\pgfqpoint{2.639889in}{2.470427in}}%
\pgfpathcurveto{\pgfqpoint{2.645713in}{2.476250in}}{\pgfqpoint{2.648985in}{2.484150in}}{\pgfqpoint{2.648985in}{2.492387in}}%
\pgfpathcurveto{\pgfqpoint{2.648985in}{2.500623in}}{\pgfqpoint{2.645713in}{2.508523in}}{\pgfqpoint{2.639889in}{2.514347in}}%
\pgfpathcurveto{\pgfqpoint{2.634065in}{2.520171in}}{\pgfqpoint{2.626165in}{2.523443in}}{\pgfqpoint{2.617929in}{2.523443in}}%
\pgfpathcurveto{\pgfqpoint{2.609693in}{2.523443in}}{\pgfqpoint{2.601792in}{2.520171in}}{\pgfqpoint{2.595969in}{2.514347in}}%
\pgfpathcurveto{\pgfqpoint{2.590145in}{2.508523in}}{\pgfqpoint{2.586872in}{2.500623in}}{\pgfqpoint{2.586872in}{2.492387in}}%
\pgfpathcurveto{\pgfqpoint{2.586872in}{2.484150in}}{\pgfqpoint{2.590145in}{2.476250in}}{\pgfqpoint{2.595969in}{2.470427in}}%
\pgfpathcurveto{\pgfqpoint{2.601792in}{2.464603in}}{\pgfqpoint{2.609693in}{2.461330in}}{\pgfqpoint{2.617929in}{2.461330in}}%
\pgfpathclose%
\pgfusepath{stroke,fill}%
\end{pgfscope}%
\begin{pgfscope}%
\pgfpathrectangle{\pgfqpoint{0.100000in}{0.212622in}}{\pgfqpoint{3.696000in}{3.696000in}}%
\pgfusepath{clip}%
\pgfsetbuttcap%
\pgfsetroundjoin%
\definecolor{currentfill}{rgb}{0.121569,0.466667,0.705882}%
\pgfsetfillcolor{currentfill}%
\pgfsetfillopacity{0.895981}%
\pgfsetlinewidth{1.003750pt}%
\definecolor{currentstroke}{rgb}{0.121569,0.466667,0.705882}%
\pgfsetstrokecolor{currentstroke}%
\pgfsetstrokeopacity{0.895981}%
\pgfsetdash{}{0pt}%
\pgfpathmoveto{\pgfqpoint{2.335702in}{2.297314in}}%
\pgfpathcurveto{\pgfqpoint{2.343938in}{2.297314in}}{\pgfqpoint{2.351838in}{2.300586in}}{\pgfqpoint{2.357662in}{2.306410in}}%
\pgfpathcurveto{\pgfqpoint{2.363486in}{2.312234in}}{\pgfqpoint{2.366758in}{2.320134in}}{\pgfqpoint{2.366758in}{2.328371in}}%
\pgfpathcurveto{\pgfqpoint{2.366758in}{2.336607in}}{\pgfqpoint{2.363486in}{2.344507in}}{\pgfqpoint{2.357662in}{2.350331in}}%
\pgfpathcurveto{\pgfqpoint{2.351838in}{2.356155in}}{\pgfqpoint{2.343938in}{2.359427in}}{\pgfqpoint{2.335702in}{2.359427in}}%
\pgfpathcurveto{\pgfqpoint{2.327465in}{2.359427in}}{\pgfqpoint{2.319565in}{2.356155in}}{\pgfqpoint{2.313741in}{2.350331in}}%
\pgfpathcurveto{\pgfqpoint{2.307917in}{2.344507in}}{\pgfqpoint{2.304645in}{2.336607in}}{\pgfqpoint{2.304645in}{2.328371in}}%
\pgfpathcurveto{\pgfqpoint{2.304645in}{2.320134in}}{\pgfqpoint{2.307917in}{2.312234in}}{\pgfqpoint{2.313741in}{2.306410in}}%
\pgfpathcurveto{\pgfqpoint{2.319565in}{2.300586in}}{\pgfqpoint{2.327465in}{2.297314in}}{\pgfqpoint{2.335702in}{2.297314in}}%
\pgfpathclose%
\pgfusepath{stroke,fill}%
\end{pgfscope}%
\begin{pgfscope}%
\pgfpathrectangle{\pgfqpoint{0.100000in}{0.212622in}}{\pgfqpoint{3.696000in}{3.696000in}}%
\pgfusepath{clip}%
\pgfsetbuttcap%
\pgfsetroundjoin%
\definecolor{currentfill}{rgb}{0.121569,0.466667,0.705882}%
\pgfsetfillcolor{currentfill}%
\pgfsetfillopacity{0.897153}%
\pgfsetlinewidth{1.003750pt}%
\definecolor{currentstroke}{rgb}{0.121569,0.466667,0.705882}%
\pgfsetstrokecolor{currentstroke}%
\pgfsetstrokeopacity{0.897153}%
\pgfsetdash{}{0pt}%
\pgfpathmoveto{\pgfqpoint{2.579042in}{2.423953in}}%
\pgfpathcurveto{\pgfqpoint{2.587278in}{2.423953in}}{\pgfqpoint{2.595178in}{2.427226in}}{\pgfqpoint{2.601002in}{2.433050in}}%
\pgfpathcurveto{\pgfqpoint{2.606826in}{2.438874in}}{\pgfqpoint{2.610098in}{2.446774in}}{\pgfqpoint{2.610098in}{2.455010in}}%
\pgfpathcurveto{\pgfqpoint{2.610098in}{2.463246in}}{\pgfqpoint{2.606826in}{2.471146in}}{\pgfqpoint{2.601002in}{2.476970in}}%
\pgfpathcurveto{\pgfqpoint{2.595178in}{2.482794in}}{\pgfqpoint{2.587278in}{2.486066in}}{\pgfqpoint{2.579042in}{2.486066in}}%
\pgfpathcurveto{\pgfqpoint{2.570805in}{2.486066in}}{\pgfqpoint{2.562905in}{2.482794in}}{\pgfqpoint{2.557081in}{2.476970in}}%
\pgfpathcurveto{\pgfqpoint{2.551257in}{2.471146in}}{\pgfqpoint{2.547985in}{2.463246in}}{\pgfqpoint{2.547985in}{2.455010in}}%
\pgfpathcurveto{\pgfqpoint{2.547985in}{2.446774in}}{\pgfqpoint{2.551257in}{2.438874in}}{\pgfqpoint{2.557081in}{2.433050in}}%
\pgfpathcurveto{\pgfqpoint{2.562905in}{2.427226in}}{\pgfqpoint{2.570805in}{2.423953in}}{\pgfqpoint{2.579042in}{2.423953in}}%
\pgfpathclose%
\pgfusepath{stroke,fill}%
\end{pgfscope}%
\begin{pgfscope}%
\pgfpathrectangle{\pgfqpoint{0.100000in}{0.212622in}}{\pgfqpoint{3.696000in}{3.696000in}}%
\pgfusepath{clip}%
\pgfsetbuttcap%
\pgfsetroundjoin%
\definecolor{currentfill}{rgb}{0.121569,0.466667,0.705882}%
\pgfsetfillcolor{currentfill}%
\pgfsetfillopacity{0.897571}%
\pgfsetlinewidth{1.003750pt}%
\definecolor{currentstroke}{rgb}{0.121569,0.466667,0.705882}%
\pgfsetstrokecolor{currentstroke}%
\pgfsetstrokeopacity{0.897571}%
\pgfsetdash{}{0pt}%
\pgfpathmoveto{\pgfqpoint{1.597838in}{1.999768in}}%
\pgfpathcurveto{\pgfqpoint{1.606074in}{1.999768in}}{\pgfqpoint{1.613974in}{2.003041in}}{\pgfqpoint{1.619798in}{2.008864in}}%
\pgfpathcurveto{\pgfqpoint{1.625622in}{2.014688in}}{\pgfqpoint{1.628894in}{2.022588in}}{\pgfqpoint{1.628894in}{2.030825in}}%
\pgfpathcurveto{\pgfqpoint{1.628894in}{2.039061in}}{\pgfqpoint{1.625622in}{2.046961in}}{\pgfqpoint{1.619798in}{2.052785in}}%
\pgfpathcurveto{\pgfqpoint{1.613974in}{2.058609in}}{\pgfqpoint{1.606074in}{2.061881in}}{\pgfqpoint{1.597838in}{2.061881in}}%
\pgfpathcurveto{\pgfqpoint{1.589601in}{2.061881in}}{\pgfqpoint{1.581701in}{2.058609in}}{\pgfqpoint{1.575877in}{2.052785in}}%
\pgfpathcurveto{\pgfqpoint{1.570053in}{2.046961in}}{\pgfqpoint{1.566781in}{2.039061in}}{\pgfqpoint{1.566781in}{2.030825in}}%
\pgfpathcurveto{\pgfqpoint{1.566781in}{2.022588in}}{\pgfqpoint{1.570053in}{2.014688in}}{\pgfqpoint{1.575877in}{2.008864in}}%
\pgfpathcurveto{\pgfqpoint{1.581701in}{2.003041in}}{\pgfqpoint{1.589601in}{1.999768in}}{\pgfqpoint{1.597838in}{1.999768in}}%
\pgfpathclose%
\pgfusepath{stroke,fill}%
\end{pgfscope}%
\begin{pgfscope}%
\pgfpathrectangle{\pgfqpoint{0.100000in}{0.212622in}}{\pgfqpoint{3.696000in}{3.696000in}}%
\pgfusepath{clip}%
\pgfsetbuttcap%
\pgfsetroundjoin%
\definecolor{currentfill}{rgb}{0.121569,0.466667,0.705882}%
\pgfsetfillcolor{currentfill}%
\pgfsetfillopacity{0.898607}%
\pgfsetlinewidth{1.003750pt}%
\definecolor{currentstroke}{rgb}{0.121569,0.466667,0.705882}%
\pgfsetstrokecolor{currentstroke}%
\pgfsetstrokeopacity{0.898607}%
\pgfsetdash{}{0pt}%
\pgfpathmoveto{\pgfqpoint{2.324853in}{2.297462in}}%
\pgfpathcurveto{\pgfqpoint{2.333089in}{2.297462in}}{\pgfqpoint{2.340989in}{2.300734in}}{\pgfqpoint{2.346813in}{2.306558in}}%
\pgfpathcurveto{\pgfqpoint{2.352637in}{2.312382in}}{\pgfqpoint{2.355909in}{2.320282in}}{\pgfqpoint{2.355909in}{2.328518in}}%
\pgfpathcurveto{\pgfqpoint{2.355909in}{2.336754in}}{\pgfqpoint{2.352637in}{2.344654in}}{\pgfqpoint{2.346813in}{2.350478in}}%
\pgfpathcurveto{\pgfqpoint{2.340989in}{2.356302in}}{\pgfqpoint{2.333089in}{2.359575in}}{\pgfqpoint{2.324853in}{2.359575in}}%
\pgfpathcurveto{\pgfqpoint{2.316617in}{2.359575in}}{\pgfqpoint{2.308717in}{2.356302in}}{\pgfqpoint{2.302893in}{2.350478in}}%
\pgfpathcurveto{\pgfqpoint{2.297069in}{2.344654in}}{\pgfqpoint{2.293796in}{2.336754in}}{\pgfqpoint{2.293796in}{2.328518in}}%
\pgfpathcurveto{\pgfqpoint{2.293796in}{2.320282in}}{\pgfqpoint{2.297069in}{2.312382in}}{\pgfqpoint{2.302893in}{2.306558in}}%
\pgfpathcurveto{\pgfqpoint{2.308717in}{2.300734in}}{\pgfqpoint{2.316617in}{2.297462in}}{\pgfqpoint{2.324853in}{2.297462in}}%
\pgfpathclose%
\pgfusepath{stroke,fill}%
\end{pgfscope}%
\begin{pgfscope}%
\pgfpathrectangle{\pgfqpoint{0.100000in}{0.212622in}}{\pgfqpoint{3.696000in}{3.696000in}}%
\pgfusepath{clip}%
\pgfsetbuttcap%
\pgfsetroundjoin%
\definecolor{currentfill}{rgb}{0.121569,0.466667,0.705882}%
\pgfsetfillcolor{currentfill}%
\pgfsetfillopacity{0.899357}%
\pgfsetlinewidth{1.003750pt}%
\definecolor{currentstroke}{rgb}{0.121569,0.466667,0.705882}%
\pgfsetstrokecolor{currentstroke}%
\pgfsetstrokeopacity{0.899357}%
\pgfsetdash{}{0pt}%
\pgfpathmoveto{\pgfqpoint{1.251920in}{1.780702in}}%
\pgfpathcurveto{\pgfqpoint{1.260156in}{1.780702in}}{\pgfqpoint{1.268056in}{1.783974in}}{\pgfqpoint{1.273880in}{1.789798in}}%
\pgfpathcurveto{\pgfqpoint{1.279704in}{1.795622in}}{\pgfqpoint{1.282977in}{1.803522in}}{\pgfqpoint{1.282977in}{1.811758in}}%
\pgfpathcurveto{\pgfqpoint{1.282977in}{1.819995in}}{\pgfqpoint{1.279704in}{1.827895in}}{\pgfqpoint{1.273880in}{1.833719in}}%
\pgfpathcurveto{\pgfqpoint{1.268056in}{1.839542in}}{\pgfqpoint{1.260156in}{1.842815in}}{\pgfqpoint{1.251920in}{1.842815in}}%
\pgfpathcurveto{\pgfqpoint{1.243684in}{1.842815in}}{\pgfqpoint{1.235784in}{1.839542in}}{\pgfqpoint{1.229960in}{1.833719in}}%
\pgfpathcurveto{\pgfqpoint{1.224136in}{1.827895in}}{\pgfqpoint{1.220864in}{1.819995in}}{\pgfqpoint{1.220864in}{1.811758in}}%
\pgfpathcurveto{\pgfqpoint{1.220864in}{1.803522in}}{\pgfqpoint{1.224136in}{1.795622in}}{\pgfqpoint{1.229960in}{1.789798in}}%
\pgfpathcurveto{\pgfqpoint{1.235784in}{1.783974in}}{\pgfqpoint{1.243684in}{1.780702in}}{\pgfqpoint{1.251920in}{1.780702in}}%
\pgfpathclose%
\pgfusepath{stroke,fill}%
\end{pgfscope}%
\begin{pgfscope}%
\pgfpathrectangle{\pgfqpoint{0.100000in}{0.212622in}}{\pgfqpoint{3.696000in}{3.696000in}}%
\pgfusepath{clip}%
\pgfsetbuttcap%
\pgfsetroundjoin%
\definecolor{currentfill}{rgb}{0.121569,0.466667,0.705882}%
\pgfsetfillcolor{currentfill}%
\pgfsetfillopacity{0.899478}%
\pgfsetlinewidth{1.003750pt}%
\definecolor{currentstroke}{rgb}{0.121569,0.466667,0.705882}%
\pgfsetstrokecolor{currentstroke}%
\pgfsetstrokeopacity{0.899478}%
\pgfsetdash{}{0pt}%
\pgfpathmoveto{\pgfqpoint{3.056773in}{2.598334in}}%
\pgfpathcurveto{\pgfqpoint{3.065009in}{2.598334in}}{\pgfqpoint{3.072909in}{2.601606in}}{\pgfqpoint{3.078733in}{2.607430in}}%
\pgfpathcurveto{\pgfqpoint{3.084557in}{2.613254in}}{\pgfqpoint{3.087829in}{2.621154in}}{\pgfqpoint{3.087829in}{2.629391in}}%
\pgfpathcurveto{\pgfqpoint{3.087829in}{2.637627in}}{\pgfqpoint{3.084557in}{2.645527in}}{\pgfqpoint{3.078733in}{2.651351in}}%
\pgfpathcurveto{\pgfqpoint{3.072909in}{2.657175in}}{\pgfqpoint{3.065009in}{2.660447in}}{\pgfqpoint{3.056773in}{2.660447in}}%
\pgfpathcurveto{\pgfqpoint{3.048536in}{2.660447in}}{\pgfqpoint{3.040636in}{2.657175in}}{\pgfqpoint{3.034812in}{2.651351in}}%
\pgfpathcurveto{\pgfqpoint{3.028988in}{2.645527in}}{\pgfqpoint{3.025716in}{2.637627in}}{\pgfqpoint{3.025716in}{2.629391in}}%
\pgfpathcurveto{\pgfqpoint{3.025716in}{2.621154in}}{\pgfqpoint{3.028988in}{2.613254in}}{\pgfqpoint{3.034812in}{2.607430in}}%
\pgfpathcurveto{\pgfqpoint{3.040636in}{2.601606in}}{\pgfqpoint{3.048536in}{2.598334in}}{\pgfqpoint{3.056773in}{2.598334in}}%
\pgfpathclose%
\pgfusepath{stroke,fill}%
\end{pgfscope}%
\begin{pgfscope}%
\pgfpathrectangle{\pgfqpoint{0.100000in}{0.212622in}}{\pgfqpoint{3.696000in}{3.696000in}}%
\pgfusepath{clip}%
\pgfsetbuttcap%
\pgfsetroundjoin%
\definecolor{currentfill}{rgb}{0.121569,0.466667,0.705882}%
\pgfsetfillcolor{currentfill}%
\pgfsetfillopacity{0.899700}%
\pgfsetlinewidth{1.003750pt}%
\definecolor{currentstroke}{rgb}{0.121569,0.466667,0.705882}%
\pgfsetstrokecolor{currentstroke}%
\pgfsetstrokeopacity{0.899700}%
\pgfsetdash{}{0pt}%
\pgfpathmoveto{\pgfqpoint{2.530952in}{2.389802in}}%
\pgfpathcurveto{\pgfqpoint{2.539189in}{2.389802in}}{\pgfqpoint{2.547089in}{2.393074in}}{\pgfqpoint{2.552913in}{2.398898in}}%
\pgfpathcurveto{\pgfqpoint{2.558737in}{2.404722in}}{\pgfqpoint{2.562009in}{2.412622in}}{\pgfqpoint{2.562009in}{2.420858in}}%
\pgfpathcurveto{\pgfqpoint{2.562009in}{2.429094in}}{\pgfqpoint{2.558737in}{2.436994in}}{\pgfqpoint{2.552913in}{2.442818in}}%
\pgfpathcurveto{\pgfqpoint{2.547089in}{2.448642in}}{\pgfqpoint{2.539189in}{2.451915in}}{\pgfqpoint{2.530952in}{2.451915in}}%
\pgfpathcurveto{\pgfqpoint{2.522716in}{2.451915in}}{\pgfqpoint{2.514816in}{2.448642in}}{\pgfqpoint{2.508992in}{2.442818in}}%
\pgfpathcurveto{\pgfqpoint{2.503168in}{2.436994in}}{\pgfqpoint{2.499896in}{2.429094in}}{\pgfqpoint{2.499896in}{2.420858in}}%
\pgfpathcurveto{\pgfqpoint{2.499896in}{2.412622in}}{\pgfqpoint{2.503168in}{2.404722in}}{\pgfqpoint{2.508992in}{2.398898in}}%
\pgfpathcurveto{\pgfqpoint{2.514816in}{2.393074in}}{\pgfqpoint{2.522716in}{2.389802in}}{\pgfqpoint{2.530952in}{2.389802in}}%
\pgfpathclose%
\pgfusepath{stroke,fill}%
\end{pgfscope}%
\begin{pgfscope}%
\pgfpathrectangle{\pgfqpoint{0.100000in}{0.212622in}}{\pgfqpoint{3.696000in}{3.696000in}}%
\pgfusepath{clip}%
\pgfsetbuttcap%
\pgfsetroundjoin%
\definecolor{currentfill}{rgb}{0.121569,0.466667,0.705882}%
\pgfsetfillcolor{currentfill}%
\pgfsetfillopacity{0.899828}%
\pgfsetlinewidth{1.003750pt}%
\definecolor{currentstroke}{rgb}{0.121569,0.466667,0.705882}%
\pgfsetstrokecolor{currentstroke}%
\pgfsetstrokeopacity{0.899828}%
\pgfsetdash{}{0pt}%
\pgfpathmoveto{\pgfqpoint{1.279505in}{1.809793in}}%
\pgfpathcurveto{\pgfqpoint{1.287742in}{1.809793in}}{\pgfqpoint{1.295642in}{1.813065in}}{\pgfqpoint{1.301466in}{1.818889in}}%
\pgfpathcurveto{\pgfqpoint{1.307289in}{1.824713in}}{\pgfqpoint{1.310562in}{1.832613in}}{\pgfqpoint{1.310562in}{1.840849in}}%
\pgfpathcurveto{\pgfqpoint{1.310562in}{1.849085in}}{\pgfqpoint{1.307289in}{1.856985in}}{\pgfqpoint{1.301466in}{1.862809in}}%
\pgfpathcurveto{\pgfqpoint{1.295642in}{1.868633in}}{\pgfqpoint{1.287742in}{1.871906in}}{\pgfqpoint{1.279505in}{1.871906in}}%
\pgfpathcurveto{\pgfqpoint{1.271269in}{1.871906in}}{\pgfqpoint{1.263369in}{1.868633in}}{\pgfqpoint{1.257545in}{1.862809in}}%
\pgfpathcurveto{\pgfqpoint{1.251721in}{1.856985in}}{\pgfqpoint{1.248449in}{1.849085in}}{\pgfqpoint{1.248449in}{1.840849in}}%
\pgfpathcurveto{\pgfqpoint{1.248449in}{1.832613in}}{\pgfqpoint{1.251721in}{1.824713in}}{\pgfqpoint{1.257545in}{1.818889in}}%
\pgfpathcurveto{\pgfqpoint{1.263369in}{1.813065in}}{\pgfqpoint{1.271269in}{1.809793in}}{\pgfqpoint{1.279505in}{1.809793in}}%
\pgfpathclose%
\pgfusepath{stroke,fill}%
\end{pgfscope}%
\begin{pgfscope}%
\pgfpathrectangle{\pgfqpoint{0.100000in}{0.212622in}}{\pgfqpoint{3.696000in}{3.696000in}}%
\pgfusepath{clip}%
\pgfsetbuttcap%
\pgfsetroundjoin%
\definecolor{currentfill}{rgb}{0.121569,0.466667,0.705882}%
\pgfsetfillcolor{currentfill}%
\pgfsetfillopacity{0.900947}%
\pgfsetlinewidth{1.003750pt}%
\definecolor{currentstroke}{rgb}{0.121569,0.466667,0.705882}%
\pgfsetstrokecolor{currentstroke}%
\pgfsetstrokeopacity{0.900947}%
\pgfsetdash{}{0pt}%
\pgfpathmoveto{\pgfqpoint{2.502929in}{2.368702in}}%
\pgfpathcurveto{\pgfqpoint{2.511165in}{2.368702in}}{\pgfqpoint{2.519065in}{2.371975in}}{\pgfqpoint{2.524889in}{2.377798in}}%
\pgfpathcurveto{\pgfqpoint{2.530713in}{2.383622in}}{\pgfqpoint{2.533986in}{2.391522in}}{\pgfqpoint{2.533986in}{2.399759in}}%
\pgfpathcurveto{\pgfqpoint{2.533986in}{2.407995in}}{\pgfqpoint{2.530713in}{2.415895in}}{\pgfqpoint{2.524889in}{2.421719in}}%
\pgfpathcurveto{\pgfqpoint{2.519065in}{2.427543in}}{\pgfqpoint{2.511165in}{2.430815in}}{\pgfqpoint{2.502929in}{2.430815in}}%
\pgfpathcurveto{\pgfqpoint{2.494693in}{2.430815in}}{\pgfqpoint{2.486793in}{2.427543in}}{\pgfqpoint{2.480969in}{2.421719in}}%
\pgfpathcurveto{\pgfqpoint{2.475145in}{2.415895in}}{\pgfqpoint{2.471873in}{2.407995in}}{\pgfqpoint{2.471873in}{2.399759in}}%
\pgfpathcurveto{\pgfqpoint{2.471873in}{2.391522in}}{\pgfqpoint{2.475145in}{2.383622in}}{\pgfqpoint{2.480969in}{2.377798in}}%
\pgfpathcurveto{\pgfqpoint{2.486793in}{2.371975in}}{\pgfqpoint{2.494693in}{2.368702in}}{\pgfqpoint{2.502929in}{2.368702in}}%
\pgfpathclose%
\pgfusepath{stroke,fill}%
\end{pgfscope}%
\begin{pgfscope}%
\pgfpathrectangle{\pgfqpoint{0.100000in}{0.212622in}}{\pgfqpoint{3.696000in}{3.696000in}}%
\pgfusepath{clip}%
\pgfsetbuttcap%
\pgfsetroundjoin%
\definecolor{currentfill}{rgb}{0.121569,0.466667,0.705882}%
\pgfsetfillcolor{currentfill}%
\pgfsetfillopacity{0.901752}%
\pgfsetlinewidth{1.003750pt}%
\definecolor{currentstroke}{rgb}{0.121569,0.466667,0.705882}%
\pgfsetstrokecolor{currentstroke}%
\pgfsetstrokeopacity{0.901752}%
\pgfsetdash{}{0pt}%
\pgfpathmoveto{\pgfqpoint{2.553004in}{2.394247in}}%
\pgfpathcurveto{\pgfqpoint{2.561240in}{2.394247in}}{\pgfqpoint{2.569140in}{2.397519in}}{\pgfqpoint{2.574964in}{2.403343in}}%
\pgfpathcurveto{\pgfqpoint{2.580788in}{2.409167in}}{\pgfqpoint{2.584060in}{2.417067in}}{\pgfqpoint{2.584060in}{2.425303in}}%
\pgfpathcurveto{\pgfqpoint{2.584060in}{2.433539in}}{\pgfqpoint{2.580788in}{2.441439in}}{\pgfqpoint{2.574964in}{2.447263in}}%
\pgfpathcurveto{\pgfqpoint{2.569140in}{2.453087in}}{\pgfqpoint{2.561240in}{2.456360in}}{\pgfqpoint{2.553004in}{2.456360in}}%
\pgfpathcurveto{\pgfqpoint{2.544768in}{2.456360in}}{\pgfqpoint{2.536868in}{2.453087in}}{\pgfqpoint{2.531044in}{2.447263in}}%
\pgfpathcurveto{\pgfqpoint{2.525220in}{2.441439in}}{\pgfqpoint{2.521947in}{2.433539in}}{\pgfqpoint{2.521947in}{2.425303in}}%
\pgfpathcurveto{\pgfqpoint{2.521947in}{2.417067in}}{\pgfqpoint{2.525220in}{2.409167in}}{\pgfqpoint{2.531044in}{2.403343in}}%
\pgfpathcurveto{\pgfqpoint{2.536868in}{2.397519in}}{\pgfqpoint{2.544768in}{2.394247in}}{\pgfqpoint{2.553004in}{2.394247in}}%
\pgfpathclose%
\pgfusepath{stroke,fill}%
\end{pgfscope}%
\begin{pgfscope}%
\pgfpathrectangle{\pgfqpoint{0.100000in}{0.212622in}}{\pgfqpoint{3.696000in}{3.696000in}}%
\pgfusepath{clip}%
\pgfsetbuttcap%
\pgfsetroundjoin%
\definecolor{currentfill}{rgb}{0.121569,0.466667,0.705882}%
\pgfsetfillcolor{currentfill}%
\pgfsetfillopacity{0.902209}%
\pgfsetlinewidth{1.003750pt}%
\definecolor{currentstroke}{rgb}{0.121569,0.466667,0.705882}%
\pgfsetstrokecolor{currentstroke}%
\pgfsetstrokeopacity{0.902209}%
\pgfsetdash{}{0pt}%
\pgfpathmoveto{\pgfqpoint{1.554795in}{1.987518in}}%
\pgfpathcurveto{\pgfqpoint{1.563031in}{1.987518in}}{\pgfqpoint{1.570931in}{1.990791in}}{\pgfqpoint{1.576755in}{1.996615in}}%
\pgfpathcurveto{\pgfqpoint{1.582579in}{2.002438in}}{\pgfqpoint{1.585851in}{2.010338in}}{\pgfqpoint{1.585851in}{2.018575in}}%
\pgfpathcurveto{\pgfqpoint{1.585851in}{2.026811in}}{\pgfqpoint{1.582579in}{2.034711in}}{\pgfqpoint{1.576755in}{2.040535in}}%
\pgfpathcurveto{\pgfqpoint{1.570931in}{2.046359in}}{\pgfqpoint{1.563031in}{2.049631in}}{\pgfqpoint{1.554795in}{2.049631in}}%
\pgfpathcurveto{\pgfqpoint{1.546559in}{2.049631in}}{\pgfqpoint{1.538659in}{2.046359in}}{\pgfqpoint{1.532835in}{2.040535in}}%
\pgfpathcurveto{\pgfqpoint{1.527011in}{2.034711in}}{\pgfqpoint{1.523738in}{2.026811in}}{\pgfqpoint{1.523738in}{2.018575in}}%
\pgfpathcurveto{\pgfqpoint{1.523738in}{2.010338in}}{\pgfqpoint{1.527011in}{2.002438in}}{\pgfqpoint{1.532835in}{1.996615in}}%
\pgfpathcurveto{\pgfqpoint{1.538659in}{1.990791in}}{\pgfqpoint{1.546559in}{1.987518in}}{\pgfqpoint{1.554795in}{1.987518in}}%
\pgfpathclose%
\pgfusepath{stroke,fill}%
\end{pgfscope}%
\begin{pgfscope}%
\pgfpathrectangle{\pgfqpoint{0.100000in}{0.212622in}}{\pgfqpoint{3.696000in}{3.696000in}}%
\pgfusepath{clip}%
\pgfsetbuttcap%
\pgfsetroundjoin%
\definecolor{currentfill}{rgb}{0.121569,0.466667,0.705882}%
\pgfsetfillcolor{currentfill}%
\pgfsetfillopacity{0.902430}%
\pgfsetlinewidth{1.003750pt}%
\definecolor{currentstroke}{rgb}{0.121569,0.466667,0.705882}%
\pgfsetstrokecolor{currentstroke}%
\pgfsetstrokeopacity{0.902430}%
\pgfsetdash{}{0pt}%
\pgfpathmoveto{\pgfqpoint{1.610000in}{2.002685in}}%
\pgfpathcurveto{\pgfqpoint{1.618236in}{2.002685in}}{\pgfqpoint{1.626136in}{2.005958in}}{\pgfqpoint{1.631960in}{2.011782in}}%
\pgfpathcurveto{\pgfqpoint{1.637784in}{2.017606in}}{\pgfqpoint{1.641056in}{2.025506in}}{\pgfqpoint{1.641056in}{2.033742in}}%
\pgfpathcurveto{\pgfqpoint{1.641056in}{2.041978in}}{\pgfqpoint{1.637784in}{2.049878in}}{\pgfqpoint{1.631960in}{2.055702in}}%
\pgfpathcurveto{\pgfqpoint{1.626136in}{2.061526in}}{\pgfqpoint{1.618236in}{2.064798in}}{\pgfqpoint{1.610000in}{2.064798in}}%
\pgfpathcurveto{\pgfqpoint{1.601763in}{2.064798in}}{\pgfqpoint{1.593863in}{2.061526in}}{\pgfqpoint{1.588040in}{2.055702in}}%
\pgfpathcurveto{\pgfqpoint{1.582216in}{2.049878in}}{\pgfqpoint{1.578943in}{2.041978in}}{\pgfqpoint{1.578943in}{2.033742in}}%
\pgfpathcurveto{\pgfqpoint{1.578943in}{2.025506in}}{\pgfqpoint{1.582216in}{2.017606in}}{\pgfqpoint{1.588040in}{2.011782in}}%
\pgfpathcurveto{\pgfqpoint{1.593863in}{2.005958in}}{\pgfqpoint{1.601763in}{2.002685in}}{\pgfqpoint{1.610000in}{2.002685in}}%
\pgfpathclose%
\pgfusepath{stroke,fill}%
\end{pgfscope}%
\begin{pgfscope}%
\pgfpathrectangle{\pgfqpoint{0.100000in}{0.212622in}}{\pgfqpoint{3.696000in}{3.696000in}}%
\pgfusepath{clip}%
\pgfsetbuttcap%
\pgfsetroundjoin%
\definecolor{currentfill}{rgb}{0.121569,0.466667,0.705882}%
\pgfsetfillcolor{currentfill}%
\pgfsetfillopacity{0.902452}%
\pgfsetlinewidth{1.003750pt}%
\definecolor{currentstroke}{rgb}{0.121569,0.466667,0.705882}%
\pgfsetstrokecolor{currentstroke}%
\pgfsetstrokeopacity{0.902452}%
\pgfsetdash{}{0pt}%
\pgfpathmoveto{\pgfqpoint{2.537846in}{2.396827in}}%
\pgfpathcurveto{\pgfqpoint{2.546082in}{2.396827in}}{\pgfqpoint{2.553982in}{2.400099in}}{\pgfqpoint{2.559806in}{2.405923in}}%
\pgfpathcurveto{\pgfqpoint{2.565630in}{2.411747in}}{\pgfqpoint{2.568903in}{2.419647in}}{\pgfqpoint{2.568903in}{2.427883in}}%
\pgfpathcurveto{\pgfqpoint{2.568903in}{2.436119in}}{\pgfqpoint{2.565630in}{2.444020in}}{\pgfqpoint{2.559806in}{2.449843in}}%
\pgfpathcurveto{\pgfqpoint{2.553982in}{2.455667in}}{\pgfqpoint{2.546082in}{2.458940in}}{\pgfqpoint{2.537846in}{2.458940in}}%
\pgfpathcurveto{\pgfqpoint{2.529610in}{2.458940in}}{\pgfqpoint{2.521710in}{2.455667in}}{\pgfqpoint{2.515886in}{2.449843in}}%
\pgfpathcurveto{\pgfqpoint{2.510062in}{2.444020in}}{\pgfqpoint{2.506790in}{2.436119in}}{\pgfqpoint{2.506790in}{2.427883in}}%
\pgfpathcurveto{\pgfqpoint{2.506790in}{2.419647in}}{\pgfqpoint{2.510062in}{2.411747in}}{\pgfqpoint{2.515886in}{2.405923in}}%
\pgfpathcurveto{\pgfqpoint{2.521710in}{2.400099in}}{\pgfqpoint{2.529610in}{2.396827in}}{\pgfqpoint{2.537846in}{2.396827in}}%
\pgfpathclose%
\pgfusepath{stroke,fill}%
\end{pgfscope}%
\begin{pgfscope}%
\pgfpathrectangle{\pgfqpoint{0.100000in}{0.212622in}}{\pgfqpoint{3.696000in}{3.696000in}}%
\pgfusepath{clip}%
\pgfsetbuttcap%
\pgfsetroundjoin%
\definecolor{currentfill}{rgb}{0.121569,0.466667,0.705882}%
\pgfsetfillcolor{currentfill}%
\pgfsetfillopacity{0.902526}%
\pgfsetlinewidth{1.003750pt}%
\definecolor{currentstroke}{rgb}{0.121569,0.466667,0.705882}%
\pgfsetstrokecolor{currentstroke}%
\pgfsetstrokeopacity{0.902526}%
\pgfsetdash{}{0pt}%
\pgfpathmoveto{\pgfqpoint{1.569995in}{1.976697in}}%
\pgfpathcurveto{\pgfqpoint{1.578231in}{1.976697in}}{\pgfqpoint{1.586131in}{1.979969in}}{\pgfqpoint{1.591955in}{1.985793in}}%
\pgfpathcurveto{\pgfqpoint{1.597779in}{1.991617in}}{\pgfqpoint{1.601052in}{1.999517in}}{\pgfqpoint{1.601052in}{2.007754in}}%
\pgfpathcurveto{\pgfqpoint{1.601052in}{2.015990in}}{\pgfqpoint{1.597779in}{2.023890in}}{\pgfqpoint{1.591955in}{2.029714in}}%
\pgfpathcurveto{\pgfqpoint{1.586131in}{2.035538in}}{\pgfqpoint{1.578231in}{2.038810in}}{\pgfqpoint{1.569995in}{2.038810in}}%
\pgfpathcurveto{\pgfqpoint{1.561759in}{2.038810in}}{\pgfqpoint{1.553859in}{2.035538in}}{\pgfqpoint{1.548035in}{2.029714in}}%
\pgfpathcurveto{\pgfqpoint{1.542211in}{2.023890in}}{\pgfqpoint{1.538939in}{2.015990in}}{\pgfqpoint{1.538939in}{2.007754in}}%
\pgfpathcurveto{\pgfqpoint{1.538939in}{1.999517in}}{\pgfqpoint{1.542211in}{1.991617in}}{\pgfqpoint{1.548035in}{1.985793in}}%
\pgfpathcurveto{\pgfqpoint{1.553859in}{1.979969in}}{\pgfqpoint{1.561759in}{1.976697in}}{\pgfqpoint{1.569995in}{1.976697in}}%
\pgfpathclose%
\pgfusepath{stroke,fill}%
\end{pgfscope}%
\begin{pgfscope}%
\pgfpathrectangle{\pgfqpoint{0.100000in}{0.212622in}}{\pgfqpoint{3.696000in}{3.696000in}}%
\pgfusepath{clip}%
\pgfsetbuttcap%
\pgfsetroundjoin%
\definecolor{currentfill}{rgb}{0.121569,0.466667,0.705882}%
\pgfsetfillcolor{currentfill}%
\pgfsetfillopacity{0.902870}%
\pgfsetlinewidth{1.003750pt}%
\definecolor{currentstroke}{rgb}{0.121569,0.466667,0.705882}%
\pgfsetstrokecolor{currentstroke}%
\pgfsetstrokeopacity{0.902870}%
\pgfsetdash{}{0pt}%
\pgfpathmoveto{\pgfqpoint{2.518615in}{2.375825in}}%
\pgfpathcurveto{\pgfqpoint{2.526851in}{2.375825in}}{\pgfqpoint{2.534751in}{2.379097in}}{\pgfqpoint{2.540575in}{2.384921in}}%
\pgfpathcurveto{\pgfqpoint{2.546399in}{2.390745in}}{\pgfqpoint{2.549671in}{2.398645in}}{\pgfqpoint{2.549671in}{2.406881in}}%
\pgfpathcurveto{\pgfqpoint{2.549671in}{2.415118in}}{\pgfqpoint{2.546399in}{2.423018in}}{\pgfqpoint{2.540575in}{2.428842in}}%
\pgfpathcurveto{\pgfqpoint{2.534751in}{2.434666in}}{\pgfqpoint{2.526851in}{2.437938in}}{\pgfqpoint{2.518615in}{2.437938in}}%
\pgfpathcurveto{\pgfqpoint{2.510378in}{2.437938in}}{\pgfqpoint{2.502478in}{2.434666in}}{\pgfqpoint{2.496654in}{2.428842in}}%
\pgfpathcurveto{\pgfqpoint{2.490830in}{2.423018in}}{\pgfqpoint{2.487558in}{2.415118in}}{\pgfqpoint{2.487558in}{2.406881in}}%
\pgfpathcurveto{\pgfqpoint{2.487558in}{2.398645in}}{\pgfqpoint{2.490830in}{2.390745in}}{\pgfqpoint{2.496654in}{2.384921in}}%
\pgfpathcurveto{\pgfqpoint{2.502478in}{2.379097in}}{\pgfqpoint{2.510378in}{2.375825in}}{\pgfqpoint{2.518615in}{2.375825in}}%
\pgfpathclose%
\pgfusepath{stroke,fill}%
\end{pgfscope}%
\begin{pgfscope}%
\pgfpathrectangle{\pgfqpoint{0.100000in}{0.212622in}}{\pgfqpoint{3.696000in}{3.696000in}}%
\pgfusepath{clip}%
\pgfsetbuttcap%
\pgfsetroundjoin%
\definecolor{currentfill}{rgb}{0.121569,0.466667,0.705882}%
\pgfsetfillcolor{currentfill}%
\pgfsetfillopacity{0.903067}%
\pgfsetlinewidth{1.003750pt}%
\definecolor{currentstroke}{rgb}{0.121569,0.466667,0.705882}%
\pgfsetstrokecolor{currentstroke}%
\pgfsetstrokeopacity{0.903067}%
\pgfsetdash{}{0pt}%
\pgfpathmoveto{\pgfqpoint{3.058254in}{2.603761in}}%
\pgfpathcurveto{\pgfqpoint{3.066490in}{2.603761in}}{\pgfqpoint{3.074391in}{2.607034in}}{\pgfqpoint{3.080214in}{2.612857in}}%
\pgfpathcurveto{\pgfqpoint{3.086038in}{2.618681in}}{\pgfqpoint{3.089311in}{2.626581in}}{\pgfqpoint{3.089311in}{2.634818in}}%
\pgfpathcurveto{\pgfqpoint{3.089311in}{2.643054in}}{\pgfqpoint{3.086038in}{2.650954in}}{\pgfqpoint{3.080214in}{2.656778in}}%
\pgfpathcurveto{\pgfqpoint{3.074391in}{2.662602in}}{\pgfqpoint{3.066490in}{2.665874in}}{\pgfqpoint{3.058254in}{2.665874in}}%
\pgfpathcurveto{\pgfqpoint{3.050018in}{2.665874in}}{\pgfqpoint{3.042118in}{2.662602in}}{\pgfqpoint{3.036294in}{2.656778in}}%
\pgfpathcurveto{\pgfqpoint{3.030470in}{2.650954in}}{\pgfqpoint{3.027198in}{2.643054in}}{\pgfqpoint{3.027198in}{2.634818in}}%
\pgfpathcurveto{\pgfqpoint{3.027198in}{2.626581in}}{\pgfqpoint{3.030470in}{2.618681in}}{\pgfqpoint{3.036294in}{2.612857in}}%
\pgfpathcurveto{\pgfqpoint{3.042118in}{2.607034in}}{\pgfqpoint{3.050018in}{2.603761in}}{\pgfqpoint{3.058254in}{2.603761in}}%
\pgfpathclose%
\pgfusepath{stroke,fill}%
\end{pgfscope}%
\begin{pgfscope}%
\pgfpathrectangle{\pgfqpoint{0.100000in}{0.212622in}}{\pgfqpoint{3.696000in}{3.696000in}}%
\pgfusepath{clip}%
\pgfsetbuttcap%
\pgfsetroundjoin%
\definecolor{currentfill}{rgb}{0.121569,0.466667,0.705882}%
\pgfsetfillcolor{currentfill}%
\pgfsetfillopacity{0.903515}%
\pgfsetlinewidth{1.003750pt}%
\definecolor{currentstroke}{rgb}{0.121569,0.466667,0.705882}%
\pgfsetstrokecolor{currentstroke}%
\pgfsetstrokeopacity{0.903515}%
\pgfsetdash{}{0pt}%
\pgfpathmoveto{\pgfqpoint{1.611337in}{2.002712in}}%
\pgfpathcurveto{\pgfqpoint{1.619573in}{2.002712in}}{\pgfqpoint{1.627473in}{2.005984in}}{\pgfqpoint{1.633297in}{2.011808in}}%
\pgfpathcurveto{\pgfqpoint{1.639121in}{2.017632in}}{\pgfqpoint{1.642394in}{2.025532in}}{\pgfqpoint{1.642394in}{2.033768in}}%
\pgfpathcurveto{\pgfqpoint{1.642394in}{2.042005in}}{\pgfqpoint{1.639121in}{2.049905in}}{\pgfqpoint{1.633297in}{2.055729in}}%
\pgfpathcurveto{\pgfqpoint{1.627473in}{2.061553in}}{\pgfqpoint{1.619573in}{2.064825in}}{\pgfqpoint{1.611337in}{2.064825in}}%
\pgfpathcurveto{\pgfqpoint{1.603101in}{2.064825in}}{\pgfqpoint{1.595201in}{2.061553in}}{\pgfqpoint{1.589377in}{2.055729in}}%
\pgfpathcurveto{\pgfqpoint{1.583553in}{2.049905in}}{\pgfqpoint{1.580281in}{2.042005in}}{\pgfqpoint{1.580281in}{2.033768in}}%
\pgfpathcurveto{\pgfqpoint{1.580281in}{2.025532in}}{\pgfqpoint{1.583553in}{2.017632in}}{\pgfqpoint{1.589377in}{2.011808in}}%
\pgfpathcurveto{\pgfqpoint{1.595201in}{2.005984in}}{\pgfqpoint{1.603101in}{2.002712in}}{\pgfqpoint{1.611337in}{2.002712in}}%
\pgfpathclose%
\pgfusepath{stroke,fill}%
\end{pgfscope}%
\begin{pgfscope}%
\pgfpathrectangle{\pgfqpoint{0.100000in}{0.212622in}}{\pgfqpoint{3.696000in}{3.696000in}}%
\pgfusepath{clip}%
\pgfsetbuttcap%
\pgfsetroundjoin%
\definecolor{currentfill}{rgb}{0.121569,0.466667,0.705882}%
\pgfsetfillcolor{currentfill}%
\pgfsetfillopacity{0.904023}%
\pgfsetlinewidth{1.003750pt}%
\definecolor{currentstroke}{rgb}{0.121569,0.466667,0.705882}%
\pgfsetstrokecolor{currentstroke}%
\pgfsetstrokeopacity{0.904023}%
\pgfsetdash{}{0pt}%
\pgfpathmoveto{\pgfqpoint{1.280698in}{1.810469in}}%
\pgfpathcurveto{\pgfqpoint{1.288935in}{1.810469in}}{\pgfqpoint{1.296835in}{1.813742in}}{\pgfqpoint{1.302659in}{1.819566in}}%
\pgfpathcurveto{\pgfqpoint{1.308482in}{1.825390in}}{\pgfqpoint{1.311755in}{1.833290in}}{\pgfqpoint{1.311755in}{1.841526in}}%
\pgfpathcurveto{\pgfqpoint{1.311755in}{1.849762in}}{\pgfqpoint{1.308482in}{1.857662in}}{\pgfqpoint{1.302659in}{1.863486in}}%
\pgfpathcurveto{\pgfqpoint{1.296835in}{1.869310in}}{\pgfqpoint{1.288935in}{1.872582in}}{\pgfqpoint{1.280698in}{1.872582in}}%
\pgfpathcurveto{\pgfqpoint{1.272462in}{1.872582in}}{\pgfqpoint{1.264562in}{1.869310in}}{\pgfqpoint{1.258738in}{1.863486in}}%
\pgfpathcurveto{\pgfqpoint{1.252914in}{1.857662in}}{\pgfqpoint{1.249642in}{1.849762in}}{\pgfqpoint{1.249642in}{1.841526in}}%
\pgfpathcurveto{\pgfqpoint{1.249642in}{1.833290in}}{\pgfqpoint{1.252914in}{1.825390in}}{\pgfqpoint{1.258738in}{1.819566in}}%
\pgfpathcurveto{\pgfqpoint{1.264562in}{1.813742in}}{\pgfqpoint{1.272462in}{1.810469in}}{\pgfqpoint{1.280698in}{1.810469in}}%
\pgfpathclose%
\pgfusepath{stroke,fill}%
\end{pgfscope}%
\begin{pgfscope}%
\pgfpathrectangle{\pgfqpoint{0.100000in}{0.212622in}}{\pgfqpoint{3.696000in}{3.696000in}}%
\pgfusepath{clip}%
\pgfsetbuttcap%
\pgfsetroundjoin%
\definecolor{currentfill}{rgb}{0.121569,0.466667,0.705882}%
\pgfsetfillcolor{currentfill}%
\pgfsetfillopacity{0.904159}%
\pgfsetlinewidth{1.003750pt}%
\definecolor{currentstroke}{rgb}{0.121569,0.466667,0.705882}%
\pgfsetstrokecolor{currentstroke}%
\pgfsetstrokeopacity{0.904159}%
\pgfsetdash{}{0pt}%
\pgfpathmoveto{\pgfqpoint{1.593796in}{1.994062in}}%
\pgfpathcurveto{\pgfqpoint{1.602033in}{1.994062in}}{\pgfqpoint{1.609933in}{1.997335in}}{\pgfqpoint{1.615757in}{2.003159in}}%
\pgfpathcurveto{\pgfqpoint{1.621581in}{2.008983in}}{\pgfqpoint{1.624853in}{2.016883in}}{\pgfqpoint{1.624853in}{2.025119in}}%
\pgfpathcurveto{\pgfqpoint{1.624853in}{2.033355in}}{\pgfqpoint{1.621581in}{2.041255in}}{\pgfqpoint{1.615757in}{2.047079in}}%
\pgfpathcurveto{\pgfqpoint{1.609933in}{2.052903in}}{\pgfqpoint{1.602033in}{2.056175in}}{\pgfqpoint{1.593796in}{2.056175in}}%
\pgfpathcurveto{\pgfqpoint{1.585560in}{2.056175in}}{\pgfqpoint{1.577660in}{2.052903in}}{\pgfqpoint{1.571836in}{2.047079in}}%
\pgfpathcurveto{\pgfqpoint{1.566012in}{2.041255in}}{\pgfqpoint{1.562740in}{2.033355in}}{\pgfqpoint{1.562740in}{2.025119in}}%
\pgfpathcurveto{\pgfqpoint{1.562740in}{2.016883in}}{\pgfqpoint{1.566012in}{2.008983in}}{\pgfqpoint{1.571836in}{2.003159in}}%
\pgfpathcurveto{\pgfqpoint{1.577660in}{1.997335in}}{\pgfqpoint{1.585560in}{1.994062in}}{\pgfqpoint{1.593796in}{1.994062in}}%
\pgfpathclose%
\pgfusepath{stroke,fill}%
\end{pgfscope}%
\begin{pgfscope}%
\pgfpathrectangle{\pgfqpoint{0.100000in}{0.212622in}}{\pgfqpoint{3.696000in}{3.696000in}}%
\pgfusepath{clip}%
\pgfsetbuttcap%
\pgfsetroundjoin%
\definecolor{currentfill}{rgb}{0.121569,0.466667,0.705882}%
\pgfsetfillcolor{currentfill}%
\pgfsetfillopacity{0.904440}%
\pgfsetlinewidth{1.003750pt}%
\definecolor{currentstroke}{rgb}{0.121569,0.466667,0.705882}%
\pgfsetstrokecolor{currentstroke}%
\pgfsetstrokeopacity{0.904440}%
\pgfsetdash{}{0pt}%
\pgfpathmoveto{\pgfqpoint{2.619400in}{2.461172in}}%
\pgfpathcurveto{\pgfqpoint{2.627636in}{2.461172in}}{\pgfqpoint{2.635536in}{2.464444in}}{\pgfqpoint{2.641360in}{2.470268in}}%
\pgfpathcurveto{\pgfqpoint{2.647184in}{2.476092in}}{\pgfqpoint{2.650456in}{2.483992in}}{\pgfqpoint{2.650456in}{2.492228in}}%
\pgfpathcurveto{\pgfqpoint{2.650456in}{2.500464in}}{\pgfqpoint{2.647184in}{2.508364in}}{\pgfqpoint{2.641360in}{2.514188in}}%
\pgfpathcurveto{\pgfqpoint{2.635536in}{2.520012in}}{\pgfqpoint{2.627636in}{2.523285in}}{\pgfqpoint{2.619400in}{2.523285in}}%
\pgfpathcurveto{\pgfqpoint{2.611163in}{2.523285in}}{\pgfqpoint{2.603263in}{2.520012in}}{\pgfqpoint{2.597439in}{2.514188in}}%
\pgfpathcurveto{\pgfqpoint{2.591616in}{2.508364in}}{\pgfqpoint{2.588343in}{2.500464in}}{\pgfqpoint{2.588343in}{2.492228in}}%
\pgfpathcurveto{\pgfqpoint{2.588343in}{2.483992in}}{\pgfqpoint{2.591616in}{2.476092in}}{\pgfqpoint{2.597439in}{2.470268in}}%
\pgfpathcurveto{\pgfqpoint{2.603263in}{2.464444in}}{\pgfqpoint{2.611163in}{2.461172in}}{\pgfqpoint{2.619400in}{2.461172in}}%
\pgfpathclose%
\pgfusepath{stroke,fill}%
\end{pgfscope}%
\begin{pgfscope}%
\pgfpathrectangle{\pgfqpoint{0.100000in}{0.212622in}}{\pgfqpoint{3.696000in}{3.696000in}}%
\pgfusepath{clip}%
\pgfsetbuttcap%
\pgfsetroundjoin%
\definecolor{currentfill}{rgb}{0.121569,0.466667,0.705882}%
\pgfsetfillcolor{currentfill}%
\pgfsetfillopacity{0.904440}%
\pgfsetlinewidth{1.003750pt}%
\definecolor{currentstroke}{rgb}{0.121569,0.466667,0.705882}%
\pgfsetstrokecolor{currentstroke}%
\pgfsetstrokeopacity{0.904440}%
\pgfsetdash{}{0pt}%
\pgfpathmoveto{\pgfqpoint{2.525094in}{2.382238in}}%
\pgfpathcurveto{\pgfqpoint{2.533331in}{2.382238in}}{\pgfqpoint{2.541231in}{2.385511in}}{\pgfqpoint{2.547055in}{2.391335in}}%
\pgfpathcurveto{\pgfqpoint{2.552879in}{2.397159in}}{\pgfqpoint{2.556151in}{2.405059in}}{\pgfqpoint{2.556151in}{2.413295in}}%
\pgfpathcurveto{\pgfqpoint{2.556151in}{2.421531in}}{\pgfqpoint{2.552879in}{2.429431in}}{\pgfqpoint{2.547055in}{2.435255in}}%
\pgfpathcurveto{\pgfqpoint{2.541231in}{2.441079in}}{\pgfqpoint{2.533331in}{2.444351in}}{\pgfqpoint{2.525094in}{2.444351in}}%
\pgfpathcurveto{\pgfqpoint{2.516858in}{2.444351in}}{\pgfqpoint{2.508958in}{2.441079in}}{\pgfqpoint{2.503134in}{2.435255in}}%
\pgfpathcurveto{\pgfqpoint{2.497310in}{2.429431in}}{\pgfqpoint{2.494038in}{2.421531in}}{\pgfqpoint{2.494038in}{2.413295in}}%
\pgfpathcurveto{\pgfqpoint{2.494038in}{2.405059in}}{\pgfqpoint{2.497310in}{2.397159in}}{\pgfqpoint{2.503134in}{2.391335in}}%
\pgfpathcurveto{\pgfqpoint{2.508958in}{2.385511in}}{\pgfqpoint{2.516858in}{2.382238in}}{\pgfqpoint{2.525094in}{2.382238in}}%
\pgfpathclose%
\pgfusepath{stroke,fill}%
\end{pgfscope}%
\begin{pgfscope}%
\pgfpathrectangle{\pgfqpoint{0.100000in}{0.212622in}}{\pgfqpoint{3.696000in}{3.696000in}}%
\pgfusepath{clip}%
\pgfsetbuttcap%
\pgfsetroundjoin%
\definecolor{currentfill}{rgb}{0.121569,0.466667,0.705882}%
\pgfsetfillcolor{currentfill}%
\pgfsetfillopacity{0.904521}%
\pgfsetlinewidth{1.003750pt}%
\definecolor{currentstroke}{rgb}{0.121569,0.466667,0.705882}%
\pgfsetstrokecolor{currentstroke}%
\pgfsetstrokeopacity{0.904521}%
\pgfsetdash{}{0pt}%
\pgfpathmoveto{\pgfqpoint{2.538482in}{2.387524in}}%
\pgfpathcurveto{\pgfqpoint{2.546718in}{2.387524in}}{\pgfqpoint{2.554618in}{2.390797in}}{\pgfqpoint{2.560442in}{2.396621in}}%
\pgfpathcurveto{\pgfqpoint{2.566266in}{2.402445in}}{\pgfqpoint{2.569538in}{2.410345in}}{\pgfqpoint{2.569538in}{2.418581in}}%
\pgfpathcurveto{\pgfqpoint{2.569538in}{2.426817in}}{\pgfqpoint{2.566266in}{2.434717in}}{\pgfqpoint{2.560442in}{2.440541in}}%
\pgfpathcurveto{\pgfqpoint{2.554618in}{2.446365in}}{\pgfqpoint{2.546718in}{2.449637in}}{\pgfqpoint{2.538482in}{2.449637in}}%
\pgfpathcurveto{\pgfqpoint{2.530246in}{2.449637in}}{\pgfqpoint{2.522345in}{2.446365in}}{\pgfqpoint{2.516522in}{2.440541in}}%
\pgfpathcurveto{\pgfqpoint{2.510698in}{2.434717in}}{\pgfqpoint{2.507425in}{2.426817in}}{\pgfqpoint{2.507425in}{2.418581in}}%
\pgfpathcurveto{\pgfqpoint{2.507425in}{2.410345in}}{\pgfqpoint{2.510698in}{2.402445in}}{\pgfqpoint{2.516522in}{2.396621in}}%
\pgfpathcurveto{\pgfqpoint{2.522345in}{2.390797in}}{\pgfqpoint{2.530246in}{2.387524in}}{\pgfqpoint{2.538482in}{2.387524in}}%
\pgfpathclose%
\pgfusepath{stroke,fill}%
\end{pgfscope}%
\begin{pgfscope}%
\pgfpathrectangle{\pgfqpoint{0.100000in}{0.212622in}}{\pgfqpoint{3.696000in}{3.696000in}}%
\pgfusepath{clip}%
\pgfsetbuttcap%
\pgfsetroundjoin%
\definecolor{currentfill}{rgb}{0.121569,0.466667,0.705882}%
\pgfsetfillcolor{currentfill}%
\pgfsetfillopacity{0.904689}%
\pgfsetlinewidth{1.003750pt}%
\definecolor{currentstroke}{rgb}{0.121569,0.466667,0.705882}%
\pgfsetstrokecolor{currentstroke}%
\pgfsetstrokeopacity{0.904689}%
\pgfsetdash{}{0pt}%
\pgfpathmoveto{\pgfqpoint{1.810601in}{2.111260in}}%
\pgfpathcurveto{\pgfqpoint{1.818837in}{2.111260in}}{\pgfqpoint{1.826737in}{2.114532in}}{\pgfqpoint{1.832561in}{2.120356in}}%
\pgfpathcurveto{\pgfqpoint{1.838385in}{2.126180in}}{\pgfqpoint{1.841657in}{2.134080in}}{\pgfqpoint{1.841657in}{2.142317in}}%
\pgfpathcurveto{\pgfqpoint{1.841657in}{2.150553in}}{\pgfqpoint{1.838385in}{2.158453in}}{\pgfqpoint{1.832561in}{2.164277in}}%
\pgfpathcurveto{\pgfqpoint{1.826737in}{2.170101in}}{\pgfqpoint{1.818837in}{2.173373in}}{\pgfqpoint{1.810601in}{2.173373in}}%
\pgfpathcurveto{\pgfqpoint{1.802365in}{2.173373in}}{\pgfqpoint{1.794465in}{2.170101in}}{\pgfqpoint{1.788641in}{2.164277in}}%
\pgfpathcurveto{\pgfqpoint{1.782817in}{2.158453in}}{\pgfqpoint{1.779544in}{2.150553in}}{\pgfqpoint{1.779544in}{2.142317in}}%
\pgfpathcurveto{\pgfqpoint{1.779544in}{2.134080in}}{\pgfqpoint{1.782817in}{2.126180in}}{\pgfqpoint{1.788641in}{2.120356in}}%
\pgfpathcurveto{\pgfqpoint{1.794465in}{2.114532in}}{\pgfqpoint{1.802365in}{2.111260in}}{\pgfqpoint{1.810601in}{2.111260in}}%
\pgfpathclose%
\pgfusepath{stroke,fill}%
\end{pgfscope}%
\begin{pgfscope}%
\pgfpathrectangle{\pgfqpoint{0.100000in}{0.212622in}}{\pgfqpoint{3.696000in}{3.696000in}}%
\pgfusepath{clip}%
\pgfsetbuttcap%
\pgfsetroundjoin%
\definecolor{currentfill}{rgb}{0.121569,0.466667,0.705882}%
\pgfsetfillcolor{currentfill}%
\pgfsetfillopacity{0.906334}%
\pgfsetlinewidth{1.003750pt}%
\definecolor{currentstroke}{rgb}{0.121569,0.466667,0.705882}%
\pgfsetstrokecolor{currentstroke}%
\pgfsetstrokeopacity{0.906334}%
\pgfsetdash{}{0pt}%
\pgfpathmoveto{\pgfqpoint{1.642444in}{2.030385in}}%
\pgfpathcurveto{\pgfqpoint{1.650680in}{2.030385in}}{\pgfqpoint{1.658580in}{2.033657in}}{\pgfqpoint{1.664404in}{2.039481in}}%
\pgfpathcurveto{\pgfqpoint{1.670228in}{2.045305in}}{\pgfqpoint{1.673500in}{2.053205in}}{\pgfqpoint{1.673500in}{2.061441in}}%
\pgfpathcurveto{\pgfqpoint{1.673500in}{2.069677in}}{\pgfqpoint{1.670228in}{2.077577in}}{\pgfqpoint{1.664404in}{2.083401in}}%
\pgfpathcurveto{\pgfqpoint{1.658580in}{2.089225in}}{\pgfqpoint{1.650680in}{2.092498in}}{\pgfqpoint{1.642444in}{2.092498in}}%
\pgfpathcurveto{\pgfqpoint{1.634207in}{2.092498in}}{\pgfqpoint{1.626307in}{2.089225in}}{\pgfqpoint{1.620483in}{2.083401in}}%
\pgfpathcurveto{\pgfqpoint{1.614659in}{2.077577in}}{\pgfqpoint{1.611387in}{2.069677in}}{\pgfqpoint{1.611387in}{2.061441in}}%
\pgfpathcurveto{\pgfqpoint{1.611387in}{2.053205in}}{\pgfqpoint{1.614659in}{2.045305in}}{\pgfqpoint{1.620483in}{2.039481in}}%
\pgfpathcurveto{\pgfqpoint{1.626307in}{2.033657in}}{\pgfqpoint{1.634207in}{2.030385in}}{\pgfqpoint{1.642444in}{2.030385in}}%
\pgfpathclose%
\pgfusepath{stroke,fill}%
\end{pgfscope}%
\begin{pgfscope}%
\pgfpathrectangle{\pgfqpoint{0.100000in}{0.212622in}}{\pgfqpoint{3.696000in}{3.696000in}}%
\pgfusepath{clip}%
\pgfsetbuttcap%
\pgfsetroundjoin%
\definecolor{currentfill}{rgb}{0.121569,0.466667,0.705882}%
\pgfsetfillcolor{currentfill}%
\pgfsetfillopacity{0.906445}%
\pgfsetlinewidth{1.003750pt}%
\definecolor{currentstroke}{rgb}{0.121569,0.466667,0.705882}%
\pgfsetstrokecolor{currentstroke}%
\pgfsetstrokeopacity{0.906445}%
\pgfsetdash{}{0pt}%
\pgfpathmoveto{\pgfqpoint{1.567110in}{1.971376in}}%
\pgfpathcurveto{\pgfqpoint{1.575347in}{1.971376in}}{\pgfqpoint{1.583247in}{1.974649in}}{\pgfqpoint{1.589071in}{1.980473in}}%
\pgfpathcurveto{\pgfqpoint{1.594894in}{1.986297in}}{\pgfqpoint{1.598167in}{1.994197in}}{\pgfqpoint{1.598167in}{2.002433in}}%
\pgfpathcurveto{\pgfqpoint{1.598167in}{2.010669in}}{\pgfqpoint{1.594894in}{2.018569in}}{\pgfqpoint{1.589071in}{2.024393in}}%
\pgfpathcurveto{\pgfqpoint{1.583247in}{2.030217in}}{\pgfqpoint{1.575347in}{2.033489in}}{\pgfqpoint{1.567110in}{2.033489in}}%
\pgfpathcurveto{\pgfqpoint{1.558874in}{2.033489in}}{\pgfqpoint{1.550974in}{2.030217in}}{\pgfqpoint{1.545150in}{2.024393in}}%
\pgfpathcurveto{\pgfqpoint{1.539326in}{2.018569in}}{\pgfqpoint{1.536054in}{2.010669in}}{\pgfqpoint{1.536054in}{2.002433in}}%
\pgfpathcurveto{\pgfqpoint{1.536054in}{1.994197in}}{\pgfqpoint{1.539326in}{1.986297in}}{\pgfqpoint{1.545150in}{1.980473in}}%
\pgfpathcurveto{\pgfqpoint{1.550974in}{1.974649in}}{\pgfqpoint{1.558874in}{1.971376in}}{\pgfqpoint{1.567110in}{1.971376in}}%
\pgfpathclose%
\pgfusepath{stroke,fill}%
\end{pgfscope}%
\begin{pgfscope}%
\pgfpathrectangle{\pgfqpoint{0.100000in}{0.212622in}}{\pgfqpoint{3.696000in}{3.696000in}}%
\pgfusepath{clip}%
\pgfsetbuttcap%
\pgfsetroundjoin%
\definecolor{currentfill}{rgb}{0.121569,0.466667,0.705882}%
\pgfsetfillcolor{currentfill}%
\pgfsetfillopacity{0.906560}%
\pgfsetlinewidth{1.003750pt}%
\definecolor{currentstroke}{rgb}{0.121569,0.466667,0.705882}%
\pgfsetstrokecolor{currentstroke}%
\pgfsetstrokeopacity{0.906560}%
\pgfsetdash{}{0pt}%
\pgfpathmoveto{\pgfqpoint{1.617987in}{2.006263in}}%
\pgfpathcurveto{\pgfqpoint{1.626224in}{2.006263in}}{\pgfqpoint{1.634124in}{2.009535in}}{\pgfqpoint{1.639948in}{2.015359in}}%
\pgfpathcurveto{\pgfqpoint{1.645772in}{2.021183in}}{\pgfqpoint{1.649044in}{2.029083in}}{\pgfqpoint{1.649044in}{2.037320in}}%
\pgfpathcurveto{\pgfqpoint{1.649044in}{2.045556in}}{\pgfqpoint{1.645772in}{2.053456in}}{\pgfqpoint{1.639948in}{2.059280in}}%
\pgfpathcurveto{\pgfqpoint{1.634124in}{2.065104in}}{\pgfqpoint{1.626224in}{2.068376in}}{\pgfqpoint{1.617987in}{2.068376in}}%
\pgfpathcurveto{\pgfqpoint{1.609751in}{2.068376in}}{\pgfqpoint{1.601851in}{2.065104in}}{\pgfqpoint{1.596027in}{2.059280in}}%
\pgfpathcurveto{\pgfqpoint{1.590203in}{2.053456in}}{\pgfqpoint{1.586931in}{2.045556in}}{\pgfqpoint{1.586931in}{2.037320in}}%
\pgfpathcurveto{\pgfqpoint{1.586931in}{2.029083in}}{\pgfqpoint{1.590203in}{2.021183in}}{\pgfqpoint{1.596027in}{2.015359in}}%
\pgfpathcurveto{\pgfqpoint{1.601851in}{2.009535in}}{\pgfqpoint{1.609751in}{2.006263in}}{\pgfqpoint{1.617987in}{2.006263in}}%
\pgfpathclose%
\pgfusepath{stroke,fill}%
\end{pgfscope}%
\begin{pgfscope}%
\pgfpathrectangle{\pgfqpoint{0.100000in}{0.212622in}}{\pgfqpoint{3.696000in}{3.696000in}}%
\pgfusepath{clip}%
\pgfsetbuttcap%
\pgfsetroundjoin%
\definecolor{currentfill}{rgb}{0.121569,0.466667,0.705882}%
\pgfsetfillcolor{currentfill}%
\pgfsetfillopacity{0.906880}%
\pgfsetlinewidth{1.003750pt}%
\definecolor{currentstroke}{rgb}{0.121569,0.466667,0.705882}%
\pgfsetstrokecolor{currentstroke}%
\pgfsetstrokeopacity{0.906880}%
\pgfsetdash{}{0pt}%
\pgfpathmoveto{\pgfqpoint{1.608800in}{1.998328in}}%
\pgfpathcurveto{\pgfqpoint{1.617036in}{1.998328in}}{\pgfqpoint{1.624936in}{2.001600in}}{\pgfqpoint{1.630760in}{2.007424in}}%
\pgfpathcurveto{\pgfqpoint{1.636584in}{2.013248in}}{\pgfqpoint{1.639857in}{2.021148in}}{\pgfqpoint{1.639857in}{2.029384in}}%
\pgfpathcurveto{\pgfqpoint{1.639857in}{2.037620in}}{\pgfqpoint{1.636584in}{2.045520in}}{\pgfqpoint{1.630760in}{2.051344in}}%
\pgfpathcurveto{\pgfqpoint{1.624936in}{2.057168in}}{\pgfqpoint{1.617036in}{2.060441in}}{\pgfqpoint{1.608800in}{2.060441in}}%
\pgfpathcurveto{\pgfqpoint{1.600564in}{2.060441in}}{\pgfqpoint{1.592664in}{2.057168in}}{\pgfqpoint{1.586840in}{2.051344in}}%
\pgfpathcurveto{\pgfqpoint{1.581016in}{2.045520in}}{\pgfqpoint{1.577744in}{2.037620in}}{\pgfqpoint{1.577744in}{2.029384in}}%
\pgfpathcurveto{\pgfqpoint{1.577744in}{2.021148in}}{\pgfqpoint{1.581016in}{2.013248in}}{\pgfqpoint{1.586840in}{2.007424in}}%
\pgfpathcurveto{\pgfqpoint{1.592664in}{2.001600in}}{\pgfqpoint{1.600564in}{1.998328in}}{\pgfqpoint{1.608800in}{1.998328in}}%
\pgfpathclose%
\pgfusepath{stroke,fill}%
\end{pgfscope}%
\begin{pgfscope}%
\pgfpathrectangle{\pgfqpoint{0.100000in}{0.212622in}}{\pgfqpoint{3.696000in}{3.696000in}}%
\pgfusepath{clip}%
\pgfsetbuttcap%
\pgfsetroundjoin%
\definecolor{currentfill}{rgb}{0.121569,0.466667,0.705882}%
\pgfsetfillcolor{currentfill}%
\pgfsetfillopacity{0.907241}%
\pgfsetlinewidth{1.003750pt}%
\definecolor{currentstroke}{rgb}{0.121569,0.466667,0.705882}%
\pgfsetstrokecolor{currentstroke}%
\pgfsetstrokeopacity{0.907241}%
\pgfsetdash{}{0pt}%
\pgfpathmoveto{\pgfqpoint{2.299268in}{2.278074in}}%
\pgfpathcurveto{\pgfqpoint{2.307504in}{2.278074in}}{\pgfqpoint{2.315404in}{2.281347in}}{\pgfqpoint{2.321228in}{2.287170in}}%
\pgfpathcurveto{\pgfqpoint{2.327052in}{2.292994in}}{\pgfqpoint{2.330324in}{2.300894in}}{\pgfqpoint{2.330324in}{2.309131in}}%
\pgfpathcurveto{\pgfqpoint{2.330324in}{2.317367in}}{\pgfqpoint{2.327052in}{2.325267in}}{\pgfqpoint{2.321228in}{2.331091in}}%
\pgfpathcurveto{\pgfqpoint{2.315404in}{2.336915in}}{\pgfqpoint{2.307504in}{2.340187in}}{\pgfqpoint{2.299268in}{2.340187in}}%
\pgfpathcurveto{\pgfqpoint{2.291031in}{2.340187in}}{\pgfqpoint{2.283131in}{2.336915in}}{\pgfqpoint{2.277307in}{2.331091in}}%
\pgfpathcurveto{\pgfqpoint{2.271484in}{2.325267in}}{\pgfqpoint{2.268211in}{2.317367in}}{\pgfqpoint{2.268211in}{2.309131in}}%
\pgfpathcurveto{\pgfqpoint{2.268211in}{2.300894in}}{\pgfqpoint{2.271484in}{2.292994in}}{\pgfqpoint{2.277307in}{2.287170in}}%
\pgfpathcurveto{\pgfqpoint{2.283131in}{2.281347in}}{\pgfqpoint{2.291031in}{2.278074in}}{\pgfqpoint{2.299268in}{2.278074in}}%
\pgfpathclose%
\pgfusepath{stroke,fill}%
\end{pgfscope}%
\begin{pgfscope}%
\pgfpathrectangle{\pgfqpoint{0.100000in}{0.212622in}}{\pgfqpoint{3.696000in}{3.696000in}}%
\pgfusepath{clip}%
\pgfsetbuttcap%
\pgfsetroundjoin%
\definecolor{currentfill}{rgb}{0.121569,0.466667,0.705882}%
\pgfsetfillcolor{currentfill}%
\pgfsetfillopacity{0.907471}%
\pgfsetlinewidth{1.003750pt}%
\definecolor{currentstroke}{rgb}{0.121569,0.466667,0.705882}%
\pgfsetstrokecolor{currentstroke}%
\pgfsetstrokeopacity{0.907471}%
\pgfsetdash{}{0pt}%
\pgfpathmoveto{\pgfqpoint{1.532559in}{1.967583in}}%
\pgfpathcurveto{\pgfqpoint{1.540795in}{1.967583in}}{\pgfqpoint{1.548695in}{1.970855in}}{\pgfqpoint{1.554519in}{1.976679in}}%
\pgfpathcurveto{\pgfqpoint{1.560343in}{1.982503in}}{\pgfqpoint{1.563616in}{1.990403in}}{\pgfqpoint{1.563616in}{1.998639in}}%
\pgfpathcurveto{\pgfqpoint{1.563616in}{2.006875in}}{\pgfqpoint{1.560343in}{2.014775in}}{\pgfqpoint{1.554519in}{2.020599in}}%
\pgfpathcurveto{\pgfqpoint{1.548695in}{2.026423in}}{\pgfqpoint{1.540795in}{2.029696in}}{\pgfqpoint{1.532559in}{2.029696in}}%
\pgfpathcurveto{\pgfqpoint{1.524323in}{2.029696in}}{\pgfqpoint{1.516423in}{2.026423in}}{\pgfqpoint{1.510599in}{2.020599in}}%
\pgfpathcurveto{\pgfqpoint{1.504775in}{2.014775in}}{\pgfqpoint{1.501503in}{2.006875in}}{\pgfqpoint{1.501503in}{1.998639in}}%
\pgfpathcurveto{\pgfqpoint{1.501503in}{1.990403in}}{\pgfqpoint{1.504775in}{1.982503in}}{\pgfqpoint{1.510599in}{1.976679in}}%
\pgfpathcurveto{\pgfqpoint{1.516423in}{1.970855in}}{\pgfqpoint{1.524323in}{1.967583in}}{\pgfqpoint{1.532559in}{1.967583in}}%
\pgfpathclose%
\pgfusepath{stroke,fill}%
\end{pgfscope}%
\begin{pgfscope}%
\pgfpathrectangle{\pgfqpoint{0.100000in}{0.212622in}}{\pgfqpoint{3.696000in}{3.696000in}}%
\pgfusepath{clip}%
\pgfsetbuttcap%
\pgfsetroundjoin%
\definecolor{currentfill}{rgb}{0.121569,0.466667,0.705882}%
\pgfsetfillcolor{currentfill}%
\pgfsetfillopacity{0.907575}%
\pgfsetlinewidth{1.003750pt}%
\definecolor{currentstroke}{rgb}{0.121569,0.466667,0.705882}%
\pgfsetstrokecolor{currentstroke}%
\pgfsetstrokeopacity{0.907575}%
\pgfsetdash{}{0pt}%
\pgfpathmoveto{\pgfqpoint{2.565372in}{2.392469in}}%
\pgfpathcurveto{\pgfqpoint{2.573608in}{2.392469in}}{\pgfqpoint{2.581508in}{2.395741in}}{\pgfqpoint{2.587332in}{2.401565in}}%
\pgfpathcurveto{\pgfqpoint{2.593156in}{2.407389in}}{\pgfqpoint{2.596428in}{2.415289in}}{\pgfqpoint{2.596428in}{2.423526in}}%
\pgfpathcurveto{\pgfqpoint{2.596428in}{2.431762in}}{\pgfqpoint{2.593156in}{2.439662in}}{\pgfqpoint{2.587332in}{2.445486in}}%
\pgfpathcurveto{\pgfqpoint{2.581508in}{2.451310in}}{\pgfqpoint{2.573608in}{2.454582in}}{\pgfqpoint{2.565372in}{2.454582in}}%
\pgfpathcurveto{\pgfqpoint{2.557135in}{2.454582in}}{\pgfqpoint{2.549235in}{2.451310in}}{\pgfqpoint{2.543411in}{2.445486in}}%
\pgfpathcurveto{\pgfqpoint{2.537587in}{2.439662in}}{\pgfqpoint{2.534315in}{2.431762in}}{\pgfqpoint{2.534315in}{2.423526in}}%
\pgfpathcurveto{\pgfqpoint{2.534315in}{2.415289in}}{\pgfqpoint{2.537587in}{2.407389in}}{\pgfqpoint{2.543411in}{2.401565in}}%
\pgfpathcurveto{\pgfqpoint{2.549235in}{2.395741in}}{\pgfqpoint{2.557135in}{2.392469in}}{\pgfqpoint{2.565372in}{2.392469in}}%
\pgfpathclose%
\pgfusepath{stroke,fill}%
\end{pgfscope}%
\begin{pgfscope}%
\pgfpathrectangle{\pgfqpoint{0.100000in}{0.212622in}}{\pgfqpoint{3.696000in}{3.696000in}}%
\pgfusepath{clip}%
\pgfsetbuttcap%
\pgfsetroundjoin%
\definecolor{currentfill}{rgb}{0.121569,0.466667,0.705882}%
\pgfsetfillcolor{currentfill}%
\pgfsetfillopacity{0.909473}%
\pgfsetlinewidth{1.003750pt}%
\definecolor{currentstroke}{rgb}{0.121569,0.466667,0.705882}%
\pgfsetstrokecolor{currentstroke}%
\pgfsetstrokeopacity{0.909473}%
\pgfsetdash{}{0pt}%
\pgfpathmoveto{\pgfqpoint{1.584887in}{1.977893in}}%
\pgfpathcurveto{\pgfqpoint{1.593123in}{1.977893in}}{\pgfqpoint{1.601023in}{1.981165in}}{\pgfqpoint{1.606847in}{1.986989in}}%
\pgfpathcurveto{\pgfqpoint{1.612671in}{1.992813in}}{\pgfqpoint{1.615943in}{2.000713in}}{\pgfqpoint{1.615943in}{2.008949in}}%
\pgfpathcurveto{\pgfqpoint{1.615943in}{2.017185in}}{\pgfqpoint{1.612671in}{2.025085in}}{\pgfqpoint{1.606847in}{2.030909in}}%
\pgfpathcurveto{\pgfqpoint{1.601023in}{2.036733in}}{\pgfqpoint{1.593123in}{2.040006in}}{\pgfqpoint{1.584887in}{2.040006in}}%
\pgfpathcurveto{\pgfqpoint{1.576651in}{2.040006in}}{\pgfqpoint{1.568751in}{2.036733in}}{\pgfqpoint{1.562927in}{2.030909in}}%
\pgfpathcurveto{\pgfqpoint{1.557103in}{2.025085in}}{\pgfqpoint{1.553830in}{2.017185in}}{\pgfqpoint{1.553830in}{2.008949in}}%
\pgfpathcurveto{\pgfqpoint{1.553830in}{2.000713in}}{\pgfqpoint{1.557103in}{1.992813in}}{\pgfqpoint{1.562927in}{1.986989in}}%
\pgfpathcurveto{\pgfqpoint{1.568751in}{1.981165in}}{\pgfqpoint{1.576651in}{1.977893in}}{\pgfqpoint{1.584887in}{1.977893in}}%
\pgfpathclose%
\pgfusepath{stroke,fill}%
\end{pgfscope}%
\begin{pgfscope}%
\pgfpathrectangle{\pgfqpoint{0.100000in}{0.212622in}}{\pgfqpoint{3.696000in}{3.696000in}}%
\pgfusepath{clip}%
\pgfsetbuttcap%
\pgfsetroundjoin%
\definecolor{currentfill}{rgb}{0.121569,0.466667,0.705882}%
\pgfsetfillcolor{currentfill}%
\pgfsetfillopacity{0.910557}%
\pgfsetlinewidth{1.003750pt}%
\definecolor{currentstroke}{rgb}{0.121569,0.466667,0.705882}%
\pgfsetstrokecolor{currentstroke}%
\pgfsetstrokeopacity{0.910557}%
\pgfsetdash{}{0pt}%
\pgfpathmoveto{\pgfqpoint{1.778829in}{2.072004in}}%
\pgfpathcurveto{\pgfqpoint{1.787065in}{2.072004in}}{\pgfqpoint{1.794966in}{2.075276in}}{\pgfqpoint{1.800789in}{2.081100in}}%
\pgfpathcurveto{\pgfqpoint{1.806613in}{2.086924in}}{\pgfqpoint{1.809886in}{2.094824in}}{\pgfqpoint{1.809886in}{2.103060in}}%
\pgfpathcurveto{\pgfqpoint{1.809886in}{2.111296in}}{\pgfqpoint{1.806613in}{2.119196in}}{\pgfqpoint{1.800789in}{2.125020in}}%
\pgfpathcurveto{\pgfqpoint{1.794966in}{2.130844in}}{\pgfqpoint{1.787065in}{2.134117in}}{\pgfqpoint{1.778829in}{2.134117in}}%
\pgfpathcurveto{\pgfqpoint{1.770593in}{2.134117in}}{\pgfqpoint{1.762693in}{2.130844in}}{\pgfqpoint{1.756869in}{2.125020in}}%
\pgfpathcurveto{\pgfqpoint{1.751045in}{2.119196in}}{\pgfqpoint{1.747773in}{2.111296in}}{\pgfqpoint{1.747773in}{2.103060in}}%
\pgfpathcurveto{\pgfqpoint{1.747773in}{2.094824in}}{\pgfqpoint{1.751045in}{2.086924in}}{\pgfqpoint{1.756869in}{2.081100in}}%
\pgfpathcurveto{\pgfqpoint{1.762693in}{2.075276in}}{\pgfqpoint{1.770593in}{2.072004in}}{\pgfqpoint{1.778829in}{2.072004in}}%
\pgfpathclose%
\pgfusepath{stroke,fill}%
\end{pgfscope}%
\begin{pgfscope}%
\pgfpathrectangle{\pgfqpoint{0.100000in}{0.212622in}}{\pgfqpoint{3.696000in}{3.696000in}}%
\pgfusepath{clip}%
\pgfsetbuttcap%
\pgfsetroundjoin%
\definecolor{currentfill}{rgb}{0.121569,0.466667,0.705882}%
\pgfsetfillcolor{currentfill}%
\pgfsetfillopacity{0.910649}%
\pgfsetlinewidth{1.003750pt}%
\definecolor{currentstroke}{rgb}{0.121569,0.466667,0.705882}%
\pgfsetstrokecolor{currentstroke}%
\pgfsetstrokeopacity{0.910649}%
\pgfsetdash{}{0pt}%
\pgfpathmoveto{\pgfqpoint{1.724908in}{2.069887in}}%
\pgfpathcurveto{\pgfqpoint{1.733144in}{2.069887in}}{\pgfqpoint{1.741044in}{2.073160in}}{\pgfqpoint{1.746868in}{2.078983in}}%
\pgfpathcurveto{\pgfqpoint{1.752692in}{2.084807in}}{\pgfqpoint{1.755964in}{2.092707in}}{\pgfqpoint{1.755964in}{2.100944in}}%
\pgfpathcurveto{\pgfqpoint{1.755964in}{2.109180in}}{\pgfqpoint{1.752692in}{2.117080in}}{\pgfqpoint{1.746868in}{2.122904in}}%
\pgfpathcurveto{\pgfqpoint{1.741044in}{2.128728in}}{\pgfqpoint{1.733144in}{2.132000in}}{\pgfqpoint{1.724908in}{2.132000in}}%
\pgfpathcurveto{\pgfqpoint{1.716672in}{2.132000in}}{\pgfqpoint{1.708772in}{2.128728in}}{\pgfqpoint{1.702948in}{2.122904in}}%
\pgfpathcurveto{\pgfqpoint{1.697124in}{2.117080in}}{\pgfqpoint{1.693851in}{2.109180in}}{\pgfqpoint{1.693851in}{2.100944in}}%
\pgfpathcurveto{\pgfqpoint{1.693851in}{2.092707in}}{\pgfqpoint{1.697124in}{2.084807in}}{\pgfqpoint{1.702948in}{2.078983in}}%
\pgfpathcurveto{\pgfqpoint{1.708772in}{2.073160in}}{\pgfqpoint{1.716672in}{2.069887in}}{\pgfqpoint{1.724908in}{2.069887in}}%
\pgfpathclose%
\pgfusepath{stroke,fill}%
\end{pgfscope}%
\begin{pgfscope}%
\pgfpathrectangle{\pgfqpoint{0.100000in}{0.212622in}}{\pgfqpoint{3.696000in}{3.696000in}}%
\pgfusepath{clip}%
\pgfsetbuttcap%
\pgfsetroundjoin%
\definecolor{currentfill}{rgb}{0.121569,0.466667,0.705882}%
\pgfsetfillcolor{currentfill}%
\pgfsetfillopacity{0.911356}%
\pgfsetlinewidth{1.003750pt}%
\definecolor{currentstroke}{rgb}{0.121569,0.466667,0.705882}%
\pgfsetstrokecolor{currentstroke}%
\pgfsetstrokeopacity{0.911356}%
\pgfsetdash{}{0pt}%
\pgfpathmoveto{\pgfqpoint{1.495046in}{1.953769in}}%
\pgfpathcurveto{\pgfqpoint{1.503283in}{1.953769in}}{\pgfqpoint{1.511183in}{1.957042in}}{\pgfqpoint{1.517007in}{1.962866in}}%
\pgfpathcurveto{\pgfqpoint{1.522831in}{1.968690in}}{\pgfqpoint{1.526103in}{1.976590in}}{\pgfqpoint{1.526103in}{1.984826in}}%
\pgfpathcurveto{\pgfqpoint{1.526103in}{1.993062in}}{\pgfqpoint{1.522831in}{2.000962in}}{\pgfqpoint{1.517007in}{2.006786in}}%
\pgfpathcurveto{\pgfqpoint{1.511183in}{2.012610in}}{\pgfqpoint{1.503283in}{2.015882in}}{\pgfqpoint{1.495046in}{2.015882in}}%
\pgfpathcurveto{\pgfqpoint{1.486810in}{2.015882in}}{\pgfqpoint{1.478910in}{2.012610in}}{\pgfqpoint{1.473086in}{2.006786in}}%
\pgfpathcurveto{\pgfqpoint{1.467262in}{2.000962in}}{\pgfqpoint{1.463990in}{1.993062in}}{\pgfqpoint{1.463990in}{1.984826in}}%
\pgfpathcurveto{\pgfqpoint{1.463990in}{1.976590in}}{\pgfqpoint{1.467262in}{1.968690in}}{\pgfqpoint{1.473086in}{1.962866in}}%
\pgfpathcurveto{\pgfqpoint{1.478910in}{1.957042in}}{\pgfqpoint{1.486810in}{1.953769in}}{\pgfqpoint{1.495046in}{1.953769in}}%
\pgfpathclose%
\pgfusepath{stroke,fill}%
\end{pgfscope}%
\begin{pgfscope}%
\pgfpathrectangle{\pgfqpoint{0.100000in}{0.212622in}}{\pgfqpoint{3.696000in}{3.696000in}}%
\pgfusepath{clip}%
\pgfsetbuttcap%
\pgfsetroundjoin%
\definecolor{currentfill}{rgb}{0.121569,0.466667,0.705882}%
\pgfsetfillcolor{currentfill}%
\pgfsetfillopacity{0.911516}%
\pgfsetlinewidth{1.003750pt}%
\definecolor{currentstroke}{rgb}{0.121569,0.466667,0.705882}%
\pgfsetstrokecolor{currentstroke}%
\pgfsetstrokeopacity{0.911516}%
\pgfsetdash{}{0pt}%
\pgfpathmoveto{\pgfqpoint{1.697065in}{2.059842in}}%
\pgfpathcurveto{\pgfqpoint{1.705301in}{2.059842in}}{\pgfqpoint{1.713201in}{2.063114in}}{\pgfqpoint{1.719025in}{2.068938in}}%
\pgfpathcurveto{\pgfqpoint{1.724849in}{2.074762in}}{\pgfqpoint{1.728122in}{2.082662in}}{\pgfqpoint{1.728122in}{2.090898in}}%
\pgfpathcurveto{\pgfqpoint{1.728122in}{2.099134in}}{\pgfqpoint{1.724849in}{2.107034in}}{\pgfqpoint{1.719025in}{2.112858in}}%
\pgfpathcurveto{\pgfqpoint{1.713201in}{2.118682in}}{\pgfqpoint{1.705301in}{2.121955in}}{\pgfqpoint{1.697065in}{2.121955in}}%
\pgfpathcurveto{\pgfqpoint{1.688829in}{2.121955in}}{\pgfqpoint{1.680929in}{2.118682in}}{\pgfqpoint{1.675105in}{2.112858in}}%
\pgfpathcurveto{\pgfqpoint{1.669281in}{2.107034in}}{\pgfqpoint{1.666009in}{2.099134in}}{\pgfqpoint{1.666009in}{2.090898in}}%
\pgfpathcurveto{\pgfqpoint{1.666009in}{2.082662in}}{\pgfqpoint{1.669281in}{2.074762in}}{\pgfqpoint{1.675105in}{2.068938in}}%
\pgfpathcurveto{\pgfqpoint{1.680929in}{2.063114in}}{\pgfqpoint{1.688829in}{2.059842in}}{\pgfqpoint{1.697065in}{2.059842in}}%
\pgfpathclose%
\pgfusepath{stroke,fill}%
\end{pgfscope}%
\begin{pgfscope}%
\pgfpathrectangle{\pgfqpoint{0.100000in}{0.212622in}}{\pgfqpoint{3.696000in}{3.696000in}}%
\pgfusepath{clip}%
\pgfsetbuttcap%
\pgfsetroundjoin%
\definecolor{currentfill}{rgb}{0.121569,0.466667,0.705882}%
\pgfsetfillcolor{currentfill}%
\pgfsetfillopacity{0.911607}%
\pgfsetlinewidth{1.003750pt}%
\definecolor{currentstroke}{rgb}{0.121569,0.466667,0.705882}%
\pgfsetstrokecolor{currentstroke}%
\pgfsetstrokeopacity{0.911607}%
\pgfsetdash{}{0pt}%
\pgfpathmoveto{\pgfqpoint{1.688092in}{2.055984in}}%
\pgfpathcurveto{\pgfqpoint{1.696328in}{2.055984in}}{\pgfqpoint{1.704228in}{2.059256in}}{\pgfqpoint{1.710052in}{2.065080in}}%
\pgfpathcurveto{\pgfqpoint{1.715876in}{2.070904in}}{\pgfqpoint{1.719148in}{2.078804in}}{\pgfqpoint{1.719148in}{2.087040in}}%
\pgfpathcurveto{\pgfqpoint{1.719148in}{2.095277in}}{\pgfqpoint{1.715876in}{2.103177in}}{\pgfqpoint{1.710052in}{2.109001in}}%
\pgfpathcurveto{\pgfqpoint{1.704228in}{2.114825in}}{\pgfqpoint{1.696328in}{2.118097in}}{\pgfqpoint{1.688092in}{2.118097in}}%
\pgfpathcurveto{\pgfqpoint{1.679855in}{2.118097in}}{\pgfqpoint{1.671955in}{2.114825in}}{\pgfqpoint{1.666131in}{2.109001in}}%
\pgfpathcurveto{\pgfqpoint{1.660307in}{2.103177in}}{\pgfqpoint{1.657035in}{2.095277in}}{\pgfqpoint{1.657035in}{2.087040in}}%
\pgfpathcurveto{\pgfqpoint{1.657035in}{2.078804in}}{\pgfqpoint{1.660307in}{2.070904in}}{\pgfqpoint{1.666131in}{2.065080in}}%
\pgfpathcurveto{\pgfqpoint{1.671955in}{2.059256in}}{\pgfqpoint{1.679855in}{2.055984in}}{\pgfqpoint{1.688092in}{2.055984in}}%
\pgfpathclose%
\pgfusepath{stroke,fill}%
\end{pgfscope}%
\begin{pgfscope}%
\pgfpathrectangle{\pgfqpoint{0.100000in}{0.212622in}}{\pgfqpoint{3.696000in}{3.696000in}}%
\pgfusepath{clip}%
\pgfsetbuttcap%
\pgfsetroundjoin%
\definecolor{currentfill}{rgb}{0.121569,0.466667,0.705882}%
\pgfsetfillcolor{currentfill}%
\pgfsetfillopacity{0.912124}%
\pgfsetlinewidth{1.003750pt}%
\definecolor{currentstroke}{rgb}{0.121569,0.466667,0.705882}%
\pgfsetstrokecolor{currentstroke}%
\pgfsetstrokeopacity{0.912124}%
\pgfsetdash{}{0pt}%
\pgfpathmoveto{\pgfqpoint{1.809084in}{2.105312in}}%
\pgfpathcurveto{\pgfqpoint{1.817320in}{2.105312in}}{\pgfqpoint{1.825220in}{2.108584in}}{\pgfqpoint{1.831044in}{2.114408in}}%
\pgfpathcurveto{\pgfqpoint{1.836868in}{2.120232in}}{\pgfqpoint{1.840141in}{2.128132in}}{\pgfqpoint{1.840141in}{2.136369in}}%
\pgfpathcurveto{\pgfqpoint{1.840141in}{2.144605in}}{\pgfqpoint{1.836868in}{2.152505in}}{\pgfqpoint{1.831044in}{2.158329in}}%
\pgfpathcurveto{\pgfqpoint{1.825220in}{2.164153in}}{\pgfqpoint{1.817320in}{2.167425in}}{\pgfqpoint{1.809084in}{2.167425in}}%
\pgfpathcurveto{\pgfqpoint{1.800848in}{2.167425in}}{\pgfqpoint{1.792948in}{2.164153in}}{\pgfqpoint{1.787124in}{2.158329in}}%
\pgfpathcurveto{\pgfqpoint{1.781300in}{2.152505in}}{\pgfqpoint{1.778028in}{2.144605in}}{\pgfqpoint{1.778028in}{2.136369in}}%
\pgfpathcurveto{\pgfqpoint{1.778028in}{2.128132in}}{\pgfqpoint{1.781300in}{2.120232in}}{\pgfqpoint{1.787124in}{2.114408in}}%
\pgfpathcurveto{\pgfqpoint{1.792948in}{2.108584in}}{\pgfqpoint{1.800848in}{2.105312in}}{\pgfqpoint{1.809084in}{2.105312in}}%
\pgfpathclose%
\pgfusepath{stroke,fill}%
\end{pgfscope}%
\begin{pgfscope}%
\pgfpathrectangle{\pgfqpoint{0.100000in}{0.212622in}}{\pgfqpoint{3.696000in}{3.696000in}}%
\pgfusepath{clip}%
\pgfsetbuttcap%
\pgfsetroundjoin%
\definecolor{currentfill}{rgb}{0.121569,0.466667,0.705882}%
\pgfsetfillcolor{currentfill}%
\pgfsetfillopacity{0.912825}%
\pgfsetlinewidth{1.003750pt}%
\definecolor{currentstroke}{rgb}{0.121569,0.466667,0.705882}%
\pgfsetstrokecolor{currentstroke}%
\pgfsetstrokeopacity{0.912825}%
\pgfsetdash{}{0pt}%
\pgfpathmoveto{\pgfqpoint{1.511361in}{1.957083in}}%
\pgfpathcurveto{\pgfqpoint{1.519598in}{1.957083in}}{\pgfqpoint{1.527498in}{1.960355in}}{\pgfqpoint{1.533322in}{1.966179in}}%
\pgfpathcurveto{\pgfqpoint{1.539146in}{1.972003in}}{\pgfqpoint{1.542418in}{1.979903in}}{\pgfqpoint{1.542418in}{1.988139in}}%
\pgfpathcurveto{\pgfqpoint{1.542418in}{1.996375in}}{\pgfqpoint{1.539146in}{2.004275in}}{\pgfqpoint{1.533322in}{2.010099in}}%
\pgfpathcurveto{\pgfqpoint{1.527498in}{2.015923in}}{\pgfqpoint{1.519598in}{2.019196in}}{\pgfqpoint{1.511361in}{2.019196in}}%
\pgfpathcurveto{\pgfqpoint{1.503125in}{2.019196in}}{\pgfqpoint{1.495225in}{2.015923in}}{\pgfqpoint{1.489401in}{2.010099in}}%
\pgfpathcurveto{\pgfqpoint{1.483577in}{2.004275in}}{\pgfqpoint{1.480305in}{1.996375in}}{\pgfqpoint{1.480305in}{1.988139in}}%
\pgfpathcurveto{\pgfqpoint{1.480305in}{1.979903in}}{\pgfqpoint{1.483577in}{1.972003in}}{\pgfqpoint{1.489401in}{1.966179in}}%
\pgfpathcurveto{\pgfqpoint{1.495225in}{1.960355in}}{\pgfqpoint{1.503125in}{1.957083in}}{\pgfqpoint{1.511361in}{1.957083in}}%
\pgfpathclose%
\pgfusepath{stroke,fill}%
\end{pgfscope}%
\begin{pgfscope}%
\pgfpathrectangle{\pgfqpoint{0.100000in}{0.212622in}}{\pgfqpoint{3.696000in}{3.696000in}}%
\pgfusepath{clip}%
\pgfsetbuttcap%
\pgfsetroundjoin%
\definecolor{currentfill}{rgb}{0.121569,0.466667,0.705882}%
\pgfsetfillcolor{currentfill}%
\pgfsetfillopacity{0.912858}%
\pgfsetlinewidth{1.003750pt}%
\definecolor{currentstroke}{rgb}{0.121569,0.466667,0.705882}%
\pgfsetstrokecolor{currentstroke}%
\pgfsetstrokeopacity{0.912858}%
\pgfsetdash{}{0pt}%
\pgfpathmoveto{\pgfqpoint{2.587912in}{2.426543in}}%
\pgfpathcurveto{\pgfqpoint{2.596148in}{2.426543in}}{\pgfqpoint{2.604048in}{2.429816in}}{\pgfqpoint{2.609872in}{2.435640in}}%
\pgfpathcurveto{\pgfqpoint{2.615696in}{2.441464in}}{\pgfqpoint{2.618968in}{2.449364in}}{\pgfqpoint{2.618968in}{2.457600in}}%
\pgfpathcurveto{\pgfqpoint{2.618968in}{2.465836in}}{\pgfqpoint{2.615696in}{2.473736in}}{\pgfqpoint{2.609872in}{2.479560in}}%
\pgfpathcurveto{\pgfqpoint{2.604048in}{2.485384in}}{\pgfqpoint{2.596148in}{2.488656in}}{\pgfqpoint{2.587912in}{2.488656in}}%
\pgfpathcurveto{\pgfqpoint{2.579675in}{2.488656in}}{\pgfqpoint{2.571775in}{2.485384in}}{\pgfqpoint{2.565951in}{2.479560in}}%
\pgfpathcurveto{\pgfqpoint{2.560127in}{2.473736in}}{\pgfqpoint{2.556855in}{2.465836in}}{\pgfqpoint{2.556855in}{2.457600in}}%
\pgfpathcurveto{\pgfqpoint{2.556855in}{2.449364in}}{\pgfqpoint{2.560127in}{2.441464in}}{\pgfqpoint{2.565951in}{2.435640in}}%
\pgfpathcurveto{\pgfqpoint{2.571775in}{2.429816in}}{\pgfqpoint{2.579675in}{2.426543in}}{\pgfqpoint{2.587912in}{2.426543in}}%
\pgfpathclose%
\pgfusepath{stroke,fill}%
\end{pgfscope}%
\begin{pgfscope}%
\pgfpathrectangle{\pgfqpoint{0.100000in}{0.212622in}}{\pgfqpoint{3.696000in}{3.696000in}}%
\pgfusepath{clip}%
\pgfsetbuttcap%
\pgfsetroundjoin%
\definecolor{currentfill}{rgb}{0.121569,0.466667,0.705882}%
\pgfsetfillcolor{currentfill}%
\pgfsetfillopacity{0.912900}%
\pgfsetlinewidth{1.003750pt}%
\definecolor{currentstroke}{rgb}{0.121569,0.466667,0.705882}%
\pgfsetstrokecolor{currentstroke}%
\pgfsetstrokeopacity{0.912900}%
\pgfsetdash{}{0pt}%
\pgfpathmoveto{\pgfqpoint{1.797597in}{2.094585in}}%
\pgfpathcurveto{\pgfqpoint{1.805833in}{2.094585in}}{\pgfqpoint{1.813733in}{2.097857in}}{\pgfqpoint{1.819557in}{2.103681in}}%
\pgfpathcurveto{\pgfqpoint{1.825381in}{2.109505in}}{\pgfqpoint{1.828654in}{2.117405in}}{\pgfqpoint{1.828654in}{2.125641in}}%
\pgfpathcurveto{\pgfqpoint{1.828654in}{2.133878in}}{\pgfqpoint{1.825381in}{2.141778in}}{\pgfqpoint{1.819557in}{2.147602in}}%
\pgfpathcurveto{\pgfqpoint{1.813733in}{2.153425in}}{\pgfqpoint{1.805833in}{2.156698in}}{\pgfqpoint{1.797597in}{2.156698in}}%
\pgfpathcurveto{\pgfqpoint{1.789361in}{2.156698in}}{\pgfqpoint{1.781461in}{2.153425in}}{\pgfqpoint{1.775637in}{2.147602in}}%
\pgfpathcurveto{\pgfqpoint{1.769813in}{2.141778in}}{\pgfqpoint{1.766541in}{2.133878in}}{\pgfqpoint{1.766541in}{2.125641in}}%
\pgfpathcurveto{\pgfqpoint{1.766541in}{2.117405in}}{\pgfqpoint{1.769813in}{2.109505in}}{\pgfqpoint{1.775637in}{2.103681in}}%
\pgfpathcurveto{\pgfqpoint{1.781461in}{2.097857in}}{\pgfqpoint{1.789361in}{2.094585in}}{\pgfqpoint{1.797597in}{2.094585in}}%
\pgfpathclose%
\pgfusepath{stroke,fill}%
\end{pgfscope}%
\begin{pgfscope}%
\pgfpathrectangle{\pgfqpoint{0.100000in}{0.212622in}}{\pgfqpoint{3.696000in}{3.696000in}}%
\pgfusepath{clip}%
\pgfsetbuttcap%
\pgfsetroundjoin%
\definecolor{currentfill}{rgb}{0.121569,0.466667,0.705882}%
\pgfsetfillcolor{currentfill}%
\pgfsetfillopacity{0.914238}%
\pgfsetlinewidth{1.003750pt}%
\definecolor{currentstroke}{rgb}{0.121569,0.466667,0.705882}%
\pgfsetstrokecolor{currentstroke}%
\pgfsetstrokeopacity{0.914238}%
\pgfsetdash{}{0pt}%
\pgfpathmoveto{\pgfqpoint{2.279257in}{2.266015in}}%
\pgfpathcurveto{\pgfqpoint{2.287493in}{2.266015in}}{\pgfqpoint{2.295393in}{2.269288in}}{\pgfqpoint{2.301217in}{2.275112in}}%
\pgfpathcurveto{\pgfqpoint{2.307041in}{2.280936in}}{\pgfqpoint{2.310313in}{2.288836in}}{\pgfqpoint{2.310313in}{2.297072in}}%
\pgfpathcurveto{\pgfqpoint{2.310313in}{2.305308in}}{\pgfqpoint{2.307041in}{2.313208in}}{\pgfqpoint{2.301217in}{2.319032in}}%
\pgfpathcurveto{\pgfqpoint{2.295393in}{2.324856in}}{\pgfqpoint{2.287493in}{2.328128in}}{\pgfqpoint{2.279257in}{2.328128in}}%
\pgfpathcurveto{\pgfqpoint{2.271021in}{2.328128in}}{\pgfqpoint{2.263120in}{2.324856in}}{\pgfqpoint{2.257297in}{2.319032in}}%
\pgfpathcurveto{\pgfqpoint{2.251473in}{2.313208in}}{\pgfqpoint{2.248200in}{2.305308in}}{\pgfqpoint{2.248200in}{2.297072in}}%
\pgfpathcurveto{\pgfqpoint{2.248200in}{2.288836in}}{\pgfqpoint{2.251473in}{2.280936in}}{\pgfqpoint{2.257297in}{2.275112in}}%
\pgfpathcurveto{\pgfqpoint{2.263120in}{2.269288in}}{\pgfqpoint{2.271021in}{2.266015in}}{\pgfqpoint{2.279257in}{2.266015in}}%
\pgfpathclose%
\pgfusepath{stroke,fill}%
\end{pgfscope}%
\begin{pgfscope}%
\pgfpathrectangle{\pgfqpoint{0.100000in}{0.212622in}}{\pgfqpoint{3.696000in}{3.696000in}}%
\pgfusepath{clip}%
\pgfsetbuttcap%
\pgfsetroundjoin%
\definecolor{currentfill}{rgb}{0.121569,0.466667,0.705882}%
\pgfsetfillcolor{currentfill}%
\pgfsetfillopacity{0.914510}%
\pgfsetlinewidth{1.003750pt}%
\definecolor{currentstroke}{rgb}{0.121569,0.466667,0.705882}%
\pgfsetstrokecolor{currentstroke}%
\pgfsetstrokeopacity{0.914510}%
\pgfsetdash{}{0pt}%
\pgfpathmoveto{\pgfqpoint{2.631639in}{2.440884in}}%
\pgfpathcurveto{\pgfqpoint{2.639876in}{2.440884in}}{\pgfqpoint{2.647776in}{2.444157in}}{\pgfqpoint{2.653600in}{2.449981in}}%
\pgfpathcurveto{\pgfqpoint{2.659424in}{2.455804in}}{\pgfqpoint{2.662696in}{2.463705in}}{\pgfqpoint{2.662696in}{2.471941in}}%
\pgfpathcurveto{\pgfqpoint{2.662696in}{2.480177in}}{\pgfqpoint{2.659424in}{2.488077in}}{\pgfqpoint{2.653600in}{2.493901in}}%
\pgfpathcurveto{\pgfqpoint{2.647776in}{2.499725in}}{\pgfqpoint{2.639876in}{2.502997in}}{\pgfqpoint{2.631639in}{2.502997in}}%
\pgfpathcurveto{\pgfqpoint{2.623403in}{2.502997in}}{\pgfqpoint{2.615503in}{2.499725in}}{\pgfqpoint{2.609679in}{2.493901in}}%
\pgfpathcurveto{\pgfqpoint{2.603855in}{2.488077in}}{\pgfqpoint{2.600583in}{2.480177in}}{\pgfqpoint{2.600583in}{2.471941in}}%
\pgfpathcurveto{\pgfqpoint{2.600583in}{2.463705in}}{\pgfqpoint{2.603855in}{2.455804in}}{\pgfqpoint{2.609679in}{2.449981in}}%
\pgfpathcurveto{\pgfqpoint{2.615503in}{2.444157in}}{\pgfqpoint{2.623403in}{2.440884in}}{\pgfqpoint{2.631639in}{2.440884in}}%
\pgfpathclose%
\pgfusepath{stroke,fill}%
\end{pgfscope}%
\begin{pgfscope}%
\pgfpathrectangle{\pgfqpoint{0.100000in}{0.212622in}}{\pgfqpoint{3.696000in}{3.696000in}}%
\pgfusepath{clip}%
\pgfsetbuttcap%
\pgfsetroundjoin%
\definecolor{currentfill}{rgb}{0.121569,0.466667,0.705882}%
\pgfsetfillcolor{currentfill}%
\pgfsetfillopacity{0.914745}%
\pgfsetlinewidth{1.003750pt}%
\definecolor{currentstroke}{rgb}{0.121569,0.466667,0.705882}%
\pgfsetstrokecolor{currentstroke}%
\pgfsetstrokeopacity{0.914745}%
\pgfsetdash{}{0pt}%
\pgfpathmoveto{\pgfqpoint{1.696796in}{2.063390in}}%
\pgfpathcurveto{\pgfqpoint{1.705033in}{2.063390in}}{\pgfqpoint{1.712933in}{2.066662in}}{\pgfqpoint{1.718756in}{2.072486in}}%
\pgfpathcurveto{\pgfqpoint{1.724580in}{2.078310in}}{\pgfqpoint{1.727853in}{2.086210in}}{\pgfqpoint{1.727853in}{2.094446in}}%
\pgfpathcurveto{\pgfqpoint{1.727853in}{2.102683in}}{\pgfqpoint{1.724580in}{2.110583in}}{\pgfqpoint{1.718756in}{2.116406in}}%
\pgfpathcurveto{\pgfqpoint{1.712933in}{2.122230in}}{\pgfqpoint{1.705033in}{2.125503in}}{\pgfqpoint{1.696796in}{2.125503in}}%
\pgfpathcurveto{\pgfqpoint{1.688560in}{2.125503in}}{\pgfqpoint{1.680660in}{2.122230in}}{\pgfqpoint{1.674836in}{2.116406in}}%
\pgfpathcurveto{\pgfqpoint{1.669012in}{2.110583in}}{\pgfqpoint{1.665740in}{2.102683in}}{\pgfqpoint{1.665740in}{2.094446in}}%
\pgfpathcurveto{\pgfqpoint{1.665740in}{2.086210in}}{\pgfqpoint{1.669012in}{2.078310in}}{\pgfqpoint{1.674836in}{2.072486in}}%
\pgfpathcurveto{\pgfqpoint{1.680660in}{2.066662in}}{\pgfqpoint{1.688560in}{2.063390in}}{\pgfqpoint{1.696796in}{2.063390in}}%
\pgfpathclose%
\pgfusepath{stroke,fill}%
\end{pgfscope}%
\begin{pgfscope}%
\pgfpathrectangle{\pgfqpoint{0.100000in}{0.212622in}}{\pgfqpoint{3.696000in}{3.696000in}}%
\pgfusepath{clip}%
\pgfsetbuttcap%
\pgfsetroundjoin%
\definecolor{currentfill}{rgb}{0.121569,0.466667,0.705882}%
\pgfsetfillcolor{currentfill}%
\pgfsetfillopacity{0.915086}%
\pgfsetlinewidth{1.003750pt}%
\definecolor{currentstroke}{rgb}{0.121569,0.466667,0.705882}%
\pgfsetstrokecolor{currentstroke}%
\pgfsetstrokeopacity{0.915086}%
\pgfsetdash{}{0pt}%
\pgfpathmoveto{\pgfqpoint{1.810440in}{2.099587in}}%
\pgfpathcurveto{\pgfqpoint{1.818677in}{2.099587in}}{\pgfqpoint{1.826577in}{2.102859in}}{\pgfqpoint{1.832401in}{2.108683in}}%
\pgfpathcurveto{\pgfqpoint{1.838224in}{2.114507in}}{\pgfqpoint{1.841497in}{2.122407in}}{\pgfqpoint{1.841497in}{2.130643in}}%
\pgfpathcurveto{\pgfqpoint{1.841497in}{2.138879in}}{\pgfqpoint{1.838224in}{2.146779in}}{\pgfqpoint{1.832401in}{2.152603in}}%
\pgfpathcurveto{\pgfqpoint{1.826577in}{2.158427in}}{\pgfqpoint{1.818677in}{2.161700in}}{\pgfqpoint{1.810440in}{2.161700in}}%
\pgfpathcurveto{\pgfqpoint{1.802204in}{2.161700in}}{\pgfqpoint{1.794304in}{2.158427in}}{\pgfqpoint{1.788480in}{2.152603in}}%
\pgfpathcurveto{\pgfqpoint{1.782656in}{2.146779in}}{\pgfqpoint{1.779384in}{2.138879in}}{\pgfqpoint{1.779384in}{2.130643in}}%
\pgfpathcurveto{\pgfqpoint{1.779384in}{2.122407in}}{\pgfqpoint{1.782656in}{2.114507in}}{\pgfqpoint{1.788480in}{2.108683in}}%
\pgfpathcurveto{\pgfqpoint{1.794304in}{2.102859in}}{\pgfqpoint{1.802204in}{2.099587in}}{\pgfqpoint{1.810440in}{2.099587in}}%
\pgfpathclose%
\pgfusepath{stroke,fill}%
\end{pgfscope}%
\begin{pgfscope}%
\pgfpathrectangle{\pgfqpoint{0.100000in}{0.212622in}}{\pgfqpoint{3.696000in}{3.696000in}}%
\pgfusepath{clip}%
\pgfsetbuttcap%
\pgfsetroundjoin%
\definecolor{currentfill}{rgb}{0.121569,0.466667,0.705882}%
\pgfsetfillcolor{currentfill}%
\pgfsetfillopacity{0.915414}%
\pgfsetlinewidth{1.003750pt}%
\definecolor{currentstroke}{rgb}{0.121569,0.466667,0.705882}%
\pgfsetstrokecolor{currentstroke}%
\pgfsetstrokeopacity{0.915414}%
\pgfsetdash{}{0pt}%
\pgfpathmoveto{\pgfqpoint{1.508447in}{1.950350in}}%
\pgfpathcurveto{\pgfqpoint{1.516683in}{1.950350in}}{\pgfqpoint{1.524583in}{1.953623in}}{\pgfqpoint{1.530407in}{1.959446in}}%
\pgfpathcurveto{\pgfqpoint{1.536231in}{1.965270in}}{\pgfqpoint{1.539504in}{1.973170in}}{\pgfqpoint{1.539504in}{1.981407in}}%
\pgfpathcurveto{\pgfqpoint{1.539504in}{1.989643in}}{\pgfqpoint{1.536231in}{1.997543in}}{\pgfqpoint{1.530407in}{2.003367in}}%
\pgfpathcurveto{\pgfqpoint{1.524583in}{2.009191in}}{\pgfqpoint{1.516683in}{2.012463in}}{\pgfqpoint{1.508447in}{2.012463in}}%
\pgfpathcurveto{\pgfqpoint{1.500211in}{2.012463in}}{\pgfqpoint{1.492311in}{2.009191in}}{\pgfqpoint{1.486487in}{2.003367in}}%
\pgfpathcurveto{\pgfqpoint{1.480663in}{1.997543in}}{\pgfqpoint{1.477391in}{1.989643in}}{\pgfqpoint{1.477391in}{1.981407in}}%
\pgfpathcurveto{\pgfqpoint{1.477391in}{1.973170in}}{\pgfqpoint{1.480663in}{1.965270in}}{\pgfqpoint{1.486487in}{1.959446in}}%
\pgfpathcurveto{\pgfqpoint{1.492311in}{1.953623in}}{\pgfqpoint{1.500211in}{1.950350in}}{\pgfqpoint{1.508447in}{1.950350in}}%
\pgfpathclose%
\pgfusepath{stroke,fill}%
\end{pgfscope}%
\begin{pgfscope}%
\pgfpathrectangle{\pgfqpoint{0.100000in}{0.212622in}}{\pgfqpoint{3.696000in}{3.696000in}}%
\pgfusepath{clip}%
\pgfsetbuttcap%
\pgfsetroundjoin%
\definecolor{currentfill}{rgb}{0.121569,0.466667,0.705882}%
\pgfsetfillcolor{currentfill}%
\pgfsetfillopacity{0.915768}%
\pgfsetlinewidth{1.003750pt}%
\definecolor{currentstroke}{rgb}{0.121569,0.466667,0.705882}%
\pgfsetstrokecolor{currentstroke}%
\pgfsetstrokeopacity{0.915768}%
\pgfsetdash{}{0pt}%
\pgfpathmoveto{\pgfqpoint{1.817256in}{2.094660in}}%
\pgfpathcurveto{\pgfqpoint{1.825493in}{2.094660in}}{\pgfqpoint{1.833393in}{2.097932in}}{\pgfqpoint{1.839217in}{2.103756in}}%
\pgfpathcurveto{\pgfqpoint{1.845040in}{2.109580in}}{\pgfqpoint{1.848313in}{2.117480in}}{\pgfqpoint{1.848313in}{2.125716in}}%
\pgfpathcurveto{\pgfqpoint{1.848313in}{2.133953in}}{\pgfqpoint{1.845040in}{2.141853in}}{\pgfqpoint{1.839217in}{2.147677in}}%
\pgfpathcurveto{\pgfqpoint{1.833393in}{2.153501in}}{\pgfqpoint{1.825493in}{2.156773in}}{\pgfqpoint{1.817256in}{2.156773in}}%
\pgfpathcurveto{\pgfqpoint{1.809020in}{2.156773in}}{\pgfqpoint{1.801120in}{2.153501in}}{\pgfqpoint{1.795296in}{2.147677in}}%
\pgfpathcurveto{\pgfqpoint{1.789472in}{2.141853in}}{\pgfqpoint{1.786200in}{2.133953in}}{\pgfqpoint{1.786200in}{2.125716in}}%
\pgfpathcurveto{\pgfqpoint{1.786200in}{2.117480in}}{\pgfqpoint{1.789472in}{2.109580in}}{\pgfqpoint{1.795296in}{2.103756in}}%
\pgfpathcurveto{\pgfqpoint{1.801120in}{2.097932in}}{\pgfqpoint{1.809020in}{2.094660in}}{\pgfqpoint{1.817256in}{2.094660in}}%
\pgfpathclose%
\pgfusepath{stroke,fill}%
\end{pgfscope}%
\begin{pgfscope}%
\pgfpathrectangle{\pgfqpoint{0.100000in}{0.212622in}}{\pgfqpoint{3.696000in}{3.696000in}}%
\pgfusepath{clip}%
\pgfsetbuttcap%
\pgfsetroundjoin%
\definecolor{currentfill}{rgb}{0.121569,0.466667,0.705882}%
\pgfsetfillcolor{currentfill}%
\pgfsetfillopacity{0.916277}%
\pgfsetlinewidth{1.003750pt}%
\definecolor{currentstroke}{rgb}{0.121569,0.466667,0.705882}%
\pgfsetstrokecolor{currentstroke}%
\pgfsetstrokeopacity{0.916277}%
\pgfsetdash{}{0pt}%
\pgfpathmoveto{\pgfqpoint{1.261837in}{1.784544in}}%
\pgfpathcurveto{\pgfqpoint{1.270073in}{1.784544in}}{\pgfqpoint{1.277973in}{1.787816in}}{\pgfqpoint{1.283797in}{1.793640in}}%
\pgfpathcurveto{\pgfqpoint{1.289621in}{1.799464in}}{\pgfqpoint{1.292893in}{1.807364in}}{\pgfqpoint{1.292893in}{1.815600in}}%
\pgfpathcurveto{\pgfqpoint{1.292893in}{1.823837in}}{\pgfqpoint{1.289621in}{1.831737in}}{\pgfqpoint{1.283797in}{1.837561in}}%
\pgfpathcurveto{\pgfqpoint{1.277973in}{1.843385in}}{\pgfqpoint{1.270073in}{1.846657in}}{\pgfqpoint{1.261837in}{1.846657in}}%
\pgfpathcurveto{\pgfqpoint{1.253601in}{1.846657in}}{\pgfqpoint{1.245700in}{1.843385in}}{\pgfqpoint{1.239877in}{1.837561in}}%
\pgfpathcurveto{\pgfqpoint{1.234053in}{1.831737in}}{\pgfqpoint{1.230780in}{1.823837in}}{\pgfqpoint{1.230780in}{1.815600in}}%
\pgfpathcurveto{\pgfqpoint{1.230780in}{1.807364in}}{\pgfqpoint{1.234053in}{1.799464in}}{\pgfqpoint{1.239877in}{1.793640in}}%
\pgfpathcurveto{\pgfqpoint{1.245700in}{1.787816in}}{\pgfqpoint{1.253601in}{1.784544in}}{\pgfqpoint{1.261837in}{1.784544in}}%
\pgfpathclose%
\pgfusepath{stroke,fill}%
\end{pgfscope}%
\begin{pgfscope}%
\pgfpathrectangle{\pgfqpoint{0.100000in}{0.212622in}}{\pgfqpoint{3.696000in}{3.696000in}}%
\pgfusepath{clip}%
\pgfsetbuttcap%
\pgfsetroundjoin%
\definecolor{currentfill}{rgb}{0.121569,0.466667,0.705882}%
\pgfsetfillcolor{currentfill}%
\pgfsetfillopacity{0.917285}%
\pgfsetlinewidth{1.003750pt}%
\definecolor{currentstroke}{rgb}{0.121569,0.466667,0.705882}%
\pgfsetstrokecolor{currentstroke}%
\pgfsetstrokeopacity{0.917285}%
\pgfsetdash{}{0pt}%
\pgfpathmoveto{\pgfqpoint{2.601652in}{2.437378in}}%
\pgfpathcurveto{\pgfqpoint{2.609888in}{2.437378in}}{\pgfqpoint{2.617788in}{2.440651in}}{\pgfqpoint{2.623612in}{2.446475in}}%
\pgfpathcurveto{\pgfqpoint{2.629436in}{2.452299in}}{\pgfqpoint{2.632708in}{2.460199in}}{\pgfqpoint{2.632708in}{2.468435in}}%
\pgfpathcurveto{\pgfqpoint{2.632708in}{2.476671in}}{\pgfqpoint{2.629436in}{2.484571in}}{\pgfqpoint{2.623612in}{2.490395in}}%
\pgfpathcurveto{\pgfqpoint{2.617788in}{2.496219in}}{\pgfqpoint{2.609888in}{2.499491in}}{\pgfqpoint{2.601652in}{2.499491in}}%
\pgfpathcurveto{\pgfqpoint{2.593415in}{2.499491in}}{\pgfqpoint{2.585515in}{2.496219in}}{\pgfqpoint{2.579692in}{2.490395in}}%
\pgfpathcurveto{\pgfqpoint{2.573868in}{2.484571in}}{\pgfqpoint{2.570595in}{2.476671in}}{\pgfqpoint{2.570595in}{2.468435in}}%
\pgfpathcurveto{\pgfqpoint{2.570595in}{2.460199in}}{\pgfqpoint{2.573868in}{2.452299in}}{\pgfqpoint{2.579692in}{2.446475in}}%
\pgfpathcurveto{\pgfqpoint{2.585515in}{2.440651in}}{\pgfqpoint{2.593415in}{2.437378in}}{\pgfqpoint{2.601652in}{2.437378in}}%
\pgfpathclose%
\pgfusepath{stroke,fill}%
\end{pgfscope}%
\begin{pgfscope}%
\pgfpathrectangle{\pgfqpoint{0.100000in}{0.212622in}}{\pgfqpoint{3.696000in}{3.696000in}}%
\pgfusepath{clip}%
\pgfsetbuttcap%
\pgfsetroundjoin%
\definecolor{currentfill}{rgb}{0.121569,0.466667,0.705882}%
\pgfsetfillcolor{currentfill}%
\pgfsetfillopacity{0.917479}%
\pgfsetlinewidth{1.003750pt}%
\definecolor{currentstroke}{rgb}{0.121569,0.466667,0.705882}%
\pgfsetstrokecolor{currentstroke}%
\pgfsetstrokeopacity{0.917479}%
\pgfsetdash{}{0pt}%
\pgfpathmoveto{\pgfqpoint{1.597966in}{1.974189in}}%
\pgfpathcurveto{\pgfqpoint{1.606202in}{1.974189in}}{\pgfqpoint{1.614102in}{1.977462in}}{\pgfqpoint{1.619926in}{1.983286in}}%
\pgfpathcurveto{\pgfqpoint{1.625750in}{1.989110in}}{\pgfqpoint{1.629022in}{1.997010in}}{\pgfqpoint{1.629022in}{2.005246in}}%
\pgfpathcurveto{\pgfqpoint{1.629022in}{2.013482in}}{\pgfqpoint{1.625750in}{2.021382in}}{\pgfqpoint{1.619926in}{2.027206in}}%
\pgfpathcurveto{\pgfqpoint{1.614102in}{2.033030in}}{\pgfqpoint{1.606202in}{2.036302in}}{\pgfqpoint{1.597966in}{2.036302in}}%
\pgfpathcurveto{\pgfqpoint{1.589729in}{2.036302in}}{\pgfqpoint{1.581829in}{2.033030in}}{\pgfqpoint{1.576005in}{2.027206in}}%
\pgfpathcurveto{\pgfqpoint{1.570181in}{2.021382in}}{\pgfqpoint{1.566909in}{2.013482in}}{\pgfqpoint{1.566909in}{2.005246in}}%
\pgfpathcurveto{\pgfqpoint{1.566909in}{1.997010in}}{\pgfqpoint{1.570181in}{1.989110in}}{\pgfqpoint{1.576005in}{1.983286in}}%
\pgfpathcurveto{\pgfqpoint{1.581829in}{1.977462in}}{\pgfqpoint{1.589729in}{1.974189in}}{\pgfqpoint{1.597966in}{1.974189in}}%
\pgfpathclose%
\pgfusepath{stroke,fill}%
\end{pgfscope}%
\begin{pgfscope}%
\pgfpathrectangle{\pgfqpoint{0.100000in}{0.212622in}}{\pgfqpoint{3.696000in}{3.696000in}}%
\pgfusepath{clip}%
\pgfsetbuttcap%
\pgfsetroundjoin%
\definecolor{currentfill}{rgb}{0.121569,0.466667,0.705882}%
\pgfsetfillcolor{currentfill}%
\pgfsetfillopacity{0.918229}%
\pgfsetlinewidth{1.003750pt}%
\definecolor{currentstroke}{rgb}{0.121569,0.466667,0.705882}%
\pgfsetstrokecolor{currentstroke}%
\pgfsetstrokeopacity{0.918229}%
\pgfsetdash{}{0pt}%
\pgfpathmoveto{\pgfqpoint{2.267234in}{2.258293in}}%
\pgfpathcurveto{\pgfqpoint{2.275470in}{2.258293in}}{\pgfqpoint{2.283370in}{2.261565in}}{\pgfqpoint{2.289194in}{2.267389in}}%
\pgfpathcurveto{\pgfqpoint{2.295018in}{2.273213in}}{\pgfqpoint{2.298290in}{2.281113in}}{\pgfqpoint{2.298290in}{2.289349in}}%
\pgfpathcurveto{\pgfqpoint{2.298290in}{2.297586in}}{\pgfqpoint{2.295018in}{2.305486in}}{\pgfqpoint{2.289194in}{2.311310in}}%
\pgfpathcurveto{\pgfqpoint{2.283370in}{2.317134in}}{\pgfqpoint{2.275470in}{2.320406in}}{\pgfqpoint{2.267234in}{2.320406in}}%
\pgfpathcurveto{\pgfqpoint{2.258997in}{2.320406in}}{\pgfqpoint{2.251097in}{2.317134in}}{\pgfqpoint{2.245273in}{2.311310in}}%
\pgfpathcurveto{\pgfqpoint{2.239449in}{2.305486in}}{\pgfqpoint{2.236177in}{2.297586in}}{\pgfqpoint{2.236177in}{2.289349in}}%
\pgfpathcurveto{\pgfqpoint{2.236177in}{2.281113in}}{\pgfqpoint{2.239449in}{2.273213in}}{\pgfqpoint{2.245273in}{2.267389in}}%
\pgfpathcurveto{\pgfqpoint{2.251097in}{2.261565in}}{\pgfqpoint{2.258997in}{2.258293in}}{\pgfqpoint{2.267234in}{2.258293in}}%
\pgfpathclose%
\pgfusepath{stroke,fill}%
\end{pgfscope}%
\begin{pgfscope}%
\pgfpathrectangle{\pgfqpoint{0.100000in}{0.212622in}}{\pgfqpoint{3.696000in}{3.696000in}}%
\pgfusepath{clip}%
\pgfsetbuttcap%
\pgfsetroundjoin%
\definecolor{currentfill}{rgb}{0.121569,0.466667,0.705882}%
\pgfsetfillcolor{currentfill}%
\pgfsetfillopacity{0.918308}%
\pgfsetlinewidth{1.003750pt}%
\definecolor{currentstroke}{rgb}{0.121569,0.466667,0.705882}%
\pgfsetstrokecolor{currentstroke}%
\pgfsetstrokeopacity{0.918308}%
\pgfsetdash{}{0pt}%
\pgfpathmoveto{\pgfqpoint{1.518943in}{1.928591in}}%
\pgfpathcurveto{\pgfqpoint{1.527179in}{1.928591in}}{\pgfqpoint{1.535079in}{1.931864in}}{\pgfqpoint{1.540903in}{1.937687in}}%
\pgfpathcurveto{\pgfqpoint{1.546727in}{1.943511in}}{\pgfqpoint{1.550000in}{1.951411in}}{\pgfqpoint{1.550000in}{1.959648in}}%
\pgfpathcurveto{\pgfqpoint{1.550000in}{1.967884in}}{\pgfqpoint{1.546727in}{1.975784in}}{\pgfqpoint{1.540903in}{1.981608in}}%
\pgfpathcurveto{\pgfqpoint{1.535079in}{1.987432in}}{\pgfqpoint{1.527179in}{1.990704in}}{\pgfqpoint{1.518943in}{1.990704in}}%
\pgfpathcurveto{\pgfqpoint{1.510707in}{1.990704in}}{\pgfqpoint{1.502807in}{1.987432in}}{\pgfqpoint{1.496983in}{1.981608in}}%
\pgfpathcurveto{\pgfqpoint{1.491159in}{1.975784in}}{\pgfqpoint{1.487887in}{1.967884in}}{\pgfqpoint{1.487887in}{1.959648in}}%
\pgfpathcurveto{\pgfqpoint{1.487887in}{1.951411in}}{\pgfqpoint{1.491159in}{1.943511in}}{\pgfqpoint{1.496983in}{1.937687in}}%
\pgfpathcurveto{\pgfqpoint{1.502807in}{1.931864in}}{\pgfqpoint{1.510707in}{1.928591in}}{\pgfqpoint{1.518943in}{1.928591in}}%
\pgfpathclose%
\pgfusepath{stroke,fill}%
\end{pgfscope}%
\begin{pgfscope}%
\pgfpathrectangle{\pgfqpoint{0.100000in}{0.212622in}}{\pgfqpoint{3.696000in}{3.696000in}}%
\pgfusepath{clip}%
\pgfsetbuttcap%
\pgfsetroundjoin%
\definecolor{currentfill}{rgb}{0.121569,0.466667,0.705882}%
\pgfsetfillcolor{currentfill}%
\pgfsetfillopacity{0.918878}%
\pgfsetlinewidth{1.003750pt}%
\definecolor{currentstroke}{rgb}{0.121569,0.466667,0.705882}%
\pgfsetstrokecolor{currentstroke}%
\pgfsetstrokeopacity{0.918878}%
\pgfsetdash{}{0pt}%
\pgfpathmoveto{\pgfqpoint{2.614300in}{2.452021in}}%
\pgfpathcurveto{\pgfqpoint{2.622536in}{2.452021in}}{\pgfqpoint{2.630436in}{2.455293in}}{\pgfqpoint{2.636260in}{2.461117in}}%
\pgfpathcurveto{\pgfqpoint{2.642084in}{2.466941in}}{\pgfqpoint{2.645356in}{2.474841in}}{\pgfqpoint{2.645356in}{2.483078in}}%
\pgfpathcurveto{\pgfqpoint{2.645356in}{2.491314in}}{\pgfqpoint{2.642084in}{2.499214in}}{\pgfqpoint{2.636260in}{2.505038in}}%
\pgfpathcurveto{\pgfqpoint{2.630436in}{2.510862in}}{\pgfqpoint{2.622536in}{2.514134in}}{\pgfqpoint{2.614300in}{2.514134in}}%
\pgfpathcurveto{\pgfqpoint{2.606063in}{2.514134in}}{\pgfqpoint{2.598163in}{2.510862in}}{\pgfqpoint{2.592340in}{2.505038in}}%
\pgfpathcurveto{\pgfqpoint{2.586516in}{2.499214in}}{\pgfqpoint{2.583243in}{2.491314in}}{\pgfqpoint{2.583243in}{2.483078in}}%
\pgfpathcurveto{\pgfqpoint{2.583243in}{2.474841in}}{\pgfqpoint{2.586516in}{2.466941in}}{\pgfqpoint{2.592340in}{2.461117in}}%
\pgfpathcurveto{\pgfqpoint{2.598163in}{2.455293in}}{\pgfqpoint{2.606063in}{2.452021in}}{\pgfqpoint{2.614300in}{2.452021in}}%
\pgfpathclose%
\pgfusepath{stroke,fill}%
\end{pgfscope}%
\begin{pgfscope}%
\pgfpathrectangle{\pgfqpoint{0.100000in}{0.212622in}}{\pgfqpoint{3.696000in}{3.696000in}}%
\pgfusepath{clip}%
\pgfsetbuttcap%
\pgfsetroundjoin%
\definecolor{currentfill}{rgb}{0.121569,0.466667,0.705882}%
\pgfsetfillcolor{currentfill}%
\pgfsetfillopacity{0.920282}%
\pgfsetlinewidth{1.003750pt}%
\definecolor{currentstroke}{rgb}{0.121569,0.466667,0.705882}%
\pgfsetstrokecolor{currentstroke}%
\pgfsetstrokeopacity{0.920282}%
\pgfsetdash{}{0pt}%
\pgfpathmoveto{\pgfqpoint{1.696874in}{2.057571in}}%
\pgfpathcurveto{\pgfqpoint{1.705110in}{2.057571in}}{\pgfqpoint{1.713010in}{2.060843in}}{\pgfqpoint{1.718834in}{2.066667in}}%
\pgfpathcurveto{\pgfqpoint{1.724658in}{2.072491in}}{\pgfqpoint{1.727930in}{2.080391in}}{\pgfqpoint{1.727930in}{2.088628in}}%
\pgfpathcurveto{\pgfqpoint{1.727930in}{2.096864in}}{\pgfqpoint{1.724658in}{2.104764in}}{\pgfqpoint{1.718834in}{2.110588in}}%
\pgfpathcurveto{\pgfqpoint{1.713010in}{2.116412in}}{\pgfqpoint{1.705110in}{2.119684in}}{\pgfqpoint{1.696874in}{2.119684in}}%
\pgfpathcurveto{\pgfqpoint{1.688638in}{2.119684in}}{\pgfqpoint{1.680738in}{2.116412in}}{\pgfqpoint{1.674914in}{2.110588in}}%
\pgfpathcurveto{\pgfqpoint{1.669090in}{2.104764in}}{\pgfqpoint{1.665817in}{2.096864in}}{\pgfqpoint{1.665817in}{2.088628in}}%
\pgfpathcurveto{\pgfqpoint{1.665817in}{2.080391in}}{\pgfqpoint{1.669090in}{2.072491in}}{\pgfqpoint{1.674914in}{2.066667in}}%
\pgfpathcurveto{\pgfqpoint{1.680738in}{2.060843in}}{\pgfqpoint{1.688638in}{2.057571in}}{\pgfqpoint{1.696874in}{2.057571in}}%
\pgfpathclose%
\pgfusepath{stroke,fill}%
\end{pgfscope}%
\begin{pgfscope}%
\pgfpathrectangle{\pgfqpoint{0.100000in}{0.212622in}}{\pgfqpoint{3.696000in}{3.696000in}}%
\pgfusepath{clip}%
\pgfsetbuttcap%
\pgfsetroundjoin%
\definecolor{currentfill}{rgb}{0.121569,0.466667,0.705882}%
\pgfsetfillcolor{currentfill}%
\pgfsetfillopacity{0.920659}%
\pgfsetlinewidth{1.003750pt}%
\definecolor{currentstroke}{rgb}{0.121569,0.466667,0.705882}%
\pgfsetstrokecolor{currentstroke}%
\pgfsetstrokeopacity{0.920659}%
\pgfsetdash{}{0pt}%
\pgfpathmoveto{\pgfqpoint{1.688522in}{2.052130in}}%
\pgfpathcurveto{\pgfqpoint{1.696758in}{2.052130in}}{\pgfqpoint{1.704658in}{2.055403in}}{\pgfqpoint{1.710482in}{2.061227in}}%
\pgfpathcurveto{\pgfqpoint{1.716306in}{2.067051in}}{\pgfqpoint{1.719578in}{2.074951in}}{\pgfqpoint{1.719578in}{2.083187in}}%
\pgfpathcurveto{\pgfqpoint{1.719578in}{2.091423in}}{\pgfqpoint{1.716306in}{2.099323in}}{\pgfqpoint{1.710482in}{2.105147in}}%
\pgfpathcurveto{\pgfqpoint{1.704658in}{2.110971in}}{\pgfqpoint{1.696758in}{2.114243in}}{\pgfqpoint{1.688522in}{2.114243in}}%
\pgfpathcurveto{\pgfqpoint{1.680285in}{2.114243in}}{\pgfqpoint{1.672385in}{2.110971in}}{\pgfqpoint{1.666561in}{2.105147in}}%
\pgfpathcurveto{\pgfqpoint{1.660737in}{2.099323in}}{\pgfqpoint{1.657465in}{2.091423in}}{\pgfqpoint{1.657465in}{2.083187in}}%
\pgfpathcurveto{\pgfqpoint{1.657465in}{2.074951in}}{\pgfqpoint{1.660737in}{2.067051in}}{\pgfqpoint{1.666561in}{2.061227in}}%
\pgfpathcurveto{\pgfqpoint{1.672385in}{2.055403in}}{\pgfqpoint{1.680285in}{2.052130in}}{\pgfqpoint{1.688522in}{2.052130in}}%
\pgfpathclose%
\pgfusepath{stroke,fill}%
\end{pgfscope}%
\begin{pgfscope}%
\pgfpathrectangle{\pgfqpoint{0.100000in}{0.212622in}}{\pgfqpoint{3.696000in}{3.696000in}}%
\pgfusepath{clip}%
\pgfsetbuttcap%
\pgfsetroundjoin%
\definecolor{currentfill}{rgb}{0.121569,0.466667,0.705882}%
\pgfsetfillcolor{currentfill}%
\pgfsetfillopacity{0.920775}%
\pgfsetlinewidth{1.003750pt}%
\definecolor{currentstroke}{rgb}{0.121569,0.466667,0.705882}%
\pgfsetstrokecolor{currentstroke}%
\pgfsetstrokeopacity{0.920775}%
\pgfsetdash{}{0pt}%
\pgfpathmoveto{\pgfqpoint{1.815768in}{2.081824in}}%
\pgfpathcurveto{\pgfqpoint{1.824004in}{2.081824in}}{\pgfqpoint{1.831904in}{2.085096in}}{\pgfqpoint{1.837728in}{2.090920in}}%
\pgfpathcurveto{\pgfqpoint{1.843552in}{2.096744in}}{\pgfqpoint{1.846824in}{2.104644in}}{\pgfqpoint{1.846824in}{2.112880in}}%
\pgfpathcurveto{\pgfqpoint{1.846824in}{2.121116in}}{\pgfqpoint{1.843552in}{2.129016in}}{\pgfqpoint{1.837728in}{2.134840in}}%
\pgfpathcurveto{\pgfqpoint{1.831904in}{2.140664in}}{\pgfqpoint{1.824004in}{2.143937in}}{\pgfqpoint{1.815768in}{2.143937in}}%
\pgfpathcurveto{\pgfqpoint{1.807532in}{2.143937in}}{\pgfqpoint{1.799632in}{2.140664in}}{\pgfqpoint{1.793808in}{2.134840in}}%
\pgfpathcurveto{\pgfqpoint{1.787984in}{2.129016in}}{\pgfqpoint{1.784711in}{2.121116in}}{\pgfqpoint{1.784711in}{2.112880in}}%
\pgfpathcurveto{\pgfqpoint{1.784711in}{2.104644in}}{\pgfqpoint{1.787984in}{2.096744in}}{\pgfqpoint{1.793808in}{2.090920in}}%
\pgfpathcurveto{\pgfqpoint{1.799632in}{2.085096in}}{\pgfqpoint{1.807532in}{2.081824in}}{\pgfqpoint{1.815768in}{2.081824in}}%
\pgfpathclose%
\pgfusepath{stroke,fill}%
\end{pgfscope}%
\begin{pgfscope}%
\pgfpathrectangle{\pgfqpoint{0.100000in}{0.212622in}}{\pgfqpoint{3.696000in}{3.696000in}}%
\pgfusepath{clip}%
\pgfsetbuttcap%
\pgfsetroundjoin%
\definecolor{currentfill}{rgb}{0.121569,0.466667,0.705882}%
\pgfsetfillcolor{currentfill}%
\pgfsetfillopacity{0.922937}%
\pgfsetlinewidth{1.003750pt}%
\definecolor{currentstroke}{rgb}{0.121569,0.466667,0.705882}%
\pgfsetstrokecolor{currentstroke}%
\pgfsetstrokeopacity{0.922937}%
\pgfsetdash{}{0pt}%
\pgfpathmoveto{\pgfqpoint{2.250395in}{2.247374in}}%
\pgfpathcurveto{\pgfqpoint{2.258632in}{2.247374in}}{\pgfqpoint{2.266532in}{2.250647in}}{\pgfqpoint{2.272356in}{2.256471in}}%
\pgfpathcurveto{\pgfqpoint{2.278180in}{2.262294in}}{\pgfqpoint{2.281452in}{2.270194in}}{\pgfqpoint{2.281452in}{2.278431in}}%
\pgfpathcurveto{\pgfqpoint{2.281452in}{2.286667in}}{\pgfqpoint{2.278180in}{2.294567in}}{\pgfqpoint{2.272356in}{2.300391in}}%
\pgfpathcurveto{\pgfqpoint{2.266532in}{2.306215in}}{\pgfqpoint{2.258632in}{2.309487in}}{\pgfqpoint{2.250395in}{2.309487in}}%
\pgfpathcurveto{\pgfqpoint{2.242159in}{2.309487in}}{\pgfqpoint{2.234259in}{2.306215in}}{\pgfqpoint{2.228435in}{2.300391in}}%
\pgfpathcurveto{\pgfqpoint{2.222611in}{2.294567in}}{\pgfqpoint{2.219339in}{2.286667in}}{\pgfqpoint{2.219339in}{2.278431in}}%
\pgfpathcurveto{\pgfqpoint{2.219339in}{2.270194in}}{\pgfqpoint{2.222611in}{2.262294in}}{\pgfqpoint{2.228435in}{2.256471in}}%
\pgfpathcurveto{\pgfqpoint{2.234259in}{2.250647in}}{\pgfqpoint{2.242159in}{2.247374in}}{\pgfqpoint{2.250395in}{2.247374in}}%
\pgfpathclose%
\pgfusepath{stroke,fill}%
\end{pgfscope}%
\begin{pgfscope}%
\pgfpathrectangle{\pgfqpoint{0.100000in}{0.212622in}}{\pgfqpoint{3.696000in}{3.696000in}}%
\pgfusepath{clip}%
\pgfsetbuttcap%
\pgfsetroundjoin%
\definecolor{currentfill}{rgb}{0.121569,0.466667,0.705882}%
\pgfsetfillcolor{currentfill}%
\pgfsetfillopacity{0.923374}%
\pgfsetlinewidth{1.003750pt}%
\definecolor{currentstroke}{rgb}{0.121569,0.466667,0.705882}%
\pgfsetstrokecolor{currentstroke}%
\pgfsetstrokeopacity{0.923374}%
\pgfsetdash{}{0pt}%
\pgfpathmoveto{\pgfqpoint{2.244308in}{2.244902in}}%
\pgfpathcurveto{\pgfqpoint{2.252544in}{2.244902in}}{\pgfqpoint{2.260444in}{2.248174in}}{\pgfqpoint{2.266268in}{2.253998in}}%
\pgfpathcurveto{\pgfqpoint{2.272092in}{2.259822in}}{\pgfqpoint{2.275364in}{2.267722in}}{\pgfqpoint{2.275364in}{2.275958in}}%
\pgfpathcurveto{\pgfqpoint{2.275364in}{2.284194in}}{\pgfqpoint{2.272092in}{2.292094in}}{\pgfqpoint{2.266268in}{2.297918in}}%
\pgfpathcurveto{\pgfqpoint{2.260444in}{2.303742in}}{\pgfqpoint{2.252544in}{2.307015in}}{\pgfqpoint{2.244308in}{2.307015in}}%
\pgfpathcurveto{\pgfqpoint{2.236072in}{2.307015in}}{\pgfqpoint{2.228171in}{2.303742in}}{\pgfqpoint{2.222348in}{2.297918in}}%
\pgfpathcurveto{\pgfqpoint{2.216524in}{2.292094in}}{\pgfqpoint{2.213251in}{2.284194in}}{\pgfqpoint{2.213251in}{2.275958in}}%
\pgfpathcurveto{\pgfqpoint{2.213251in}{2.267722in}}{\pgfqpoint{2.216524in}{2.259822in}}{\pgfqpoint{2.222348in}{2.253998in}}%
\pgfpathcurveto{\pgfqpoint{2.228171in}{2.248174in}}{\pgfqpoint{2.236072in}{2.244902in}}{\pgfqpoint{2.244308in}{2.244902in}}%
\pgfpathclose%
\pgfusepath{stroke,fill}%
\end{pgfscope}%
\begin{pgfscope}%
\pgfpathrectangle{\pgfqpoint{0.100000in}{0.212622in}}{\pgfqpoint{3.696000in}{3.696000in}}%
\pgfusepath{clip}%
\pgfsetbuttcap%
\pgfsetroundjoin%
\definecolor{currentfill}{rgb}{0.121569,0.466667,0.705882}%
\pgfsetfillcolor{currentfill}%
\pgfsetfillopacity{0.924488}%
\pgfsetlinewidth{1.003750pt}%
\definecolor{currentstroke}{rgb}{0.121569,0.466667,0.705882}%
\pgfsetstrokecolor{currentstroke}%
\pgfsetstrokeopacity{0.924488}%
\pgfsetdash{}{0pt}%
\pgfpathmoveto{\pgfqpoint{1.445461in}{1.897464in}}%
\pgfpathcurveto{\pgfqpoint{1.453697in}{1.897464in}}{\pgfqpoint{1.461597in}{1.900737in}}{\pgfqpoint{1.467421in}{1.906560in}}%
\pgfpathcurveto{\pgfqpoint{1.473245in}{1.912384in}}{\pgfqpoint{1.476517in}{1.920284in}}{\pgfqpoint{1.476517in}{1.928521in}}%
\pgfpathcurveto{\pgfqpoint{1.476517in}{1.936757in}}{\pgfqpoint{1.473245in}{1.944657in}}{\pgfqpoint{1.467421in}{1.950481in}}%
\pgfpathcurveto{\pgfqpoint{1.461597in}{1.956305in}}{\pgfqpoint{1.453697in}{1.959577in}}{\pgfqpoint{1.445461in}{1.959577in}}%
\pgfpathcurveto{\pgfqpoint{1.437224in}{1.959577in}}{\pgfqpoint{1.429324in}{1.956305in}}{\pgfqpoint{1.423501in}{1.950481in}}%
\pgfpathcurveto{\pgfqpoint{1.417677in}{1.944657in}}{\pgfqpoint{1.414404in}{1.936757in}}{\pgfqpoint{1.414404in}{1.928521in}}%
\pgfpathcurveto{\pgfqpoint{1.414404in}{1.920284in}}{\pgfqpoint{1.417677in}{1.912384in}}{\pgfqpoint{1.423501in}{1.906560in}}%
\pgfpathcurveto{\pgfqpoint{1.429324in}{1.900737in}}{\pgfqpoint{1.437224in}{1.897464in}}{\pgfqpoint{1.445461in}{1.897464in}}%
\pgfpathclose%
\pgfusepath{stroke,fill}%
\end{pgfscope}%
\begin{pgfscope}%
\pgfpathrectangle{\pgfqpoint{0.100000in}{0.212622in}}{\pgfqpoint{3.696000in}{3.696000in}}%
\pgfusepath{clip}%
\pgfsetbuttcap%
\pgfsetroundjoin%
\definecolor{currentfill}{rgb}{0.121569,0.466667,0.705882}%
\pgfsetfillcolor{currentfill}%
\pgfsetfillopacity{0.925172}%
\pgfsetlinewidth{1.003750pt}%
\definecolor{currentstroke}{rgb}{0.121569,0.466667,0.705882}%
\pgfsetstrokecolor{currentstroke}%
\pgfsetstrokeopacity{0.925172}%
\pgfsetdash{}{0pt}%
\pgfpathmoveto{\pgfqpoint{1.629966in}{2.027874in}}%
\pgfpathcurveto{\pgfqpoint{1.638203in}{2.027874in}}{\pgfqpoint{1.646103in}{2.031146in}}{\pgfqpoint{1.651926in}{2.036970in}}%
\pgfpathcurveto{\pgfqpoint{1.657750in}{2.042794in}}{\pgfqpoint{1.661023in}{2.050694in}}{\pgfqpoint{1.661023in}{2.058930in}}%
\pgfpathcurveto{\pgfqpoint{1.661023in}{2.067167in}}{\pgfqpoint{1.657750in}{2.075067in}}{\pgfqpoint{1.651926in}{2.080891in}}%
\pgfpathcurveto{\pgfqpoint{1.646103in}{2.086714in}}{\pgfqpoint{1.638203in}{2.089987in}}{\pgfqpoint{1.629966in}{2.089987in}}%
\pgfpathcurveto{\pgfqpoint{1.621730in}{2.089987in}}{\pgfqpoint{1.613830in}{2.086714in}}{\pgfqpoint{1.608006in}{2.080891in}}%
\pgfpathcurveto{\pgfqpoint{1.602182in}{2.075067in}}{\pgfqpoint{1.598910in}{2.067167in}}{\pgfqpoint{1.598910in}{2.058930in}}%
\pgfpathcurveto{\pgfqpoint{1.598910in}{2.050694in}}{\pgfqpoint{1.602182in}{2.042794in}}{\pgfqpoint{1.608006in}{2.036970in}}%
\pgfpathcurveto{\pgfqpoint{1.613830in}{2.031146in}}{\pgfqpoint{1.621730in}{2.027874in}}{\pgfqpoint{1.629966in}{2.027874in}}%
\pgfpathclose%
\pgfusepath{stroke,fill}%
\end{pgfscope}%
\begin{pgfscope}%
\pgfpathrectangle{\pgfqpoint{0.100000in}{0.212622in}}{\pgfqpoint{3.696000in}{3.696000in}}%
\pgfusepath{clip}%
\pgfsetbuttcap%
\pgfsetroundjoin%
\definecolor{currentfill}{rgb}{0.121569,0.466667,0.705882}%
\pgfsetfillcolor{currentfill}%
\pgfsetfillopacity{0.925283}%
\pgfsetlinewidth{1.003750pt}%
\definecolor{currentstroke}{rgb}{0.121569,0.466667,0.705882}%
\pgfsetstrokecolor{currentstroke}%
\pgfsetstrokeopacity{0.925283}%
\pgfsetdash{}{0pt}%
\pgfpathmoveto{\pgfqpoint{2.235553in}{2.237348in}}%
\pgfpathcurveto{\pgfqpoint{2.243789in}{2.237348in}}{\pgfqpoint{2.251689in}{2.240621in}}{\pgfqpoint{2.257513in}{2.246445in}}%
\pgfpathcurveto{\pgfqpoint{2.263337in}{2.252269in}}{\pgfqpoint{2.266609in}{2.260169in}}{\pgfqpoint{2.266609in}{2.268405in}}%
\pgfpathcurveto{\pgfqpoint{2.266609in}{2.276641in}}{\pgfqpoint{2.263337in}{2.284541in}}{\pgfqpoint{2.257513in}{2.290365in}}%
\pgfpathcurveto{\pgfqpoint{2.251689in}{2.296189in}}{\pgfqpoint{2.243789in}{2.299461in}}{\pgfqpoint{2.235553in}{2.299461in}}%
\pgfpathcurveto{\pgfqpoint{2.227316in}{2.299461in}}{\pgfqpoint{2.219416in}{2.296189in}}{\pgfqpoint{2.213592in}{2.290365in}}%
\pgfpathcurveto{\pgfqpoint{2.207769in}{2.284541in}}{\pgfqpoint{2.204496in}{2.276641in}}{\pgfqpoint{2.204496in}{2.268405in}}%
\pgfpathcurveto{\pgfqpoint{2.204496in}{2.260169in}}{\pgfqpoint{2.207769in}{2.252269in}}{\pgfqpoint{2.213592in}{2.246445in}}%
\pgfpathcurveto{\pgfqpoint{2.219416in}{2.240621in}}{\pgfqpoint{2.227316in}{2.237348in}}{\pgfqpoint{2.235553in}{2.237348in}}%
\pgfpathclose%
\pgfusepath{stroke,fill}%
\end{pgfscope}%
\begin{pgfscope}%
\pgfpathrectangle{\pgfqpoint{0.100000in}{0.212622in}}{\pgfqpoint{3.696000in}{3.696000in}}%
\pgfusepath{clip}%
\pgfsetbuttcap%
\pgfsetroundjoin%
\definecolor{currentfill}{rgb}{0.121569,0.466667,0.705882}%
\pgfsetfillcolor{currentfill}%
\pgfsetfillopacity{0.925302}%
\pgfsetlinewidth{1.003750pt}%
\definecolor{currentstroke}{rgb}{0.121569,0.466667,0.705882}%
\pgfsetstrokecolor{currentstroke}%
\pgfsetstrokeopacity{0.925302}%
\pgfsetdash{}{0pt}%
\pgfpathmoveto{\pgfqpoint{2.236664in}{2.237907in}}%
\pgfpathcurveto{\pgfqpoint{2.244900in}{2.237907in}}{\pgfqpoint{2.252800in}{2.241179in}}{\pgfqpoint{2.258624in}{2.247003in}}%
\pgfpathcurveto{\pgfqpoint{2.264448in}{2.252827in}}{\pgfqpoint{2.267720in}{2.260727in}}{\pgfqpoint{2.267720in}{2.268963in}}%
\pgfpathcurveto{\pgfqpoint{2.267720in}{2.277200in}}{\pgfqpoint{2.264448in}{2.285100in}}{\pgfqpoint{2.258624in}{2.290924in}}%
\pgfpathcurveto{\pgfqpoint{2.252800in}{2.296748in}}{\pgfqpoint{2.244900in}{2.300020in}}{\pgfqpoint{2.236664in}{2.300020in}}%
\pgfpathcurveto{\pgfqpoint{2.228427in}{2.300020in}}{\pgfqpoint{2.220527in}{2.296748in}}{\pgfqpoint{2.214703in}{2.290924in}}%
\pgfpathcurveto{\pgfqpoint{2.208879in}{2.285100in}}{\pgfqpoint{2.205607in}{2.277200in}}{\pgfqpoint{2.205607in}{2.268963in}}%
\pgfpathcurveto{\pgfqpoint{2.205607in}{2.260727in}}{\pgfqpoint{2.208879in}{2.252827in}}{\pgfqpoint{2.214703in}{2.247003in}}%
\pgfpathcurveto{\pgfqpoint{2.220527in}{2.241179in}}{\pgfqpoint{2.228427in}{2.237907in}}{\pgfqpoint{2.236664in}{2.237907in}}%
\pgfpathclose%
\pgfusepath{stroke,fill}%
\end{pgfscope}%
\begin{pgfscope}%
\pgfpathrectangle{\pgfqpoint{0.100000in}{0.212622in}}{\pgfqpoint{3.696000in}{3.696000in}}%
\pgfusepath{clip}%
\pgfsetbuttcap%
\pgfsetroundjoin%
\definecolor{currentfill}{rgb}{0.121569,0.466667,0.705882}%
\pgfsetfillcolor{currentfill}%
\pgfsetfillopacity{0.925935}%
\pgfsetlinewidth{1.003750pt}%
\definecolor{currentstroke}{rgb}{0.121569,0.466667,0.705882}%
\pgfsetstrokecolor{currentstroke}%
\pgfsetstrokeopacity{0.925935}%
\pgfsetdash{}{0pt}%
\pgfpathmoveto{\pgfqpoint{3.018341in}{2.562030in}}%
\pgfpathcurveto{\pgfqpoint{3.026577in}{2.562030in}}{\pgfqpoint{3.034477in}{2.565303in}}{\pgfqpoint{3.040301in}{2.571126in}}%
\pgfpathcurveto{\pgfqpoint{3.046125in}{2.576950in}}{\pgfqpoint{3.049397in}{2.584850in}}{\pgfqpoint{3.049397in}{2.593087in}}%
\pgfpathcurveto{\pgfqpoint{3.049397in}{2.601323in}}{\pgfqpoint{3.046125in}{2.609223in}}{\pgfqpoint{3.040301in}{2.615047in}}%
\pgfpathcurveto{\pgfqpoint{3.034477in}{2.620871in}}{\pgfqpoint{3.026577in}{2.624143in}}{\pgfqpoint{3.018341in}{2.624143in}}%
\pgfpathcurveto{\pgfqpoint{3.010105in}{2.624143in}}{\pgfqpoint{3.002205in}{2.620871in}}{\pgfqpoint{2.996381in}{2.615047in}}%
\pgfpathcurveto{\pgfqpoint{2.990557in}{2.609223in}}{\pgfqpoint{2.987284in}{2.601323in}}{\pgfqpoint{2.987284in}{2.593087in}}%
\pgfpathcurveto{\pgfqpoint{2.987284in}{2.584850in}}{\pgfqpoint{2.990557in}{2.576950in}}{\pgfqpoint{2.996381in}{2.571126in}}%
\pgfpathcurveto{\pgfqpoint{3.002205in}{2.565303in}}{\pgfqpoint{3.010105in}{2.562030in}}{\pgfqpoint{3.018341in}{2.562030in}}%
\pgfpathclose%
\pgfusepath{stroke,fill}%
\end{pgfscope}%
\begin{pgfscope}%
\pgfpathrectangle{\pgfqpoint{0.100000in}{0.212622in}}{\pgfqpoint{3.696000in}{3.696000in}}%
\pgfusepath{clip}%
\pgfsetbuttcap%
\pgfsetroundjoin%
\definecolor{currentfill}{rgb}{0.121569,0.466667,0.705882}%
\pgfsetfillcolor{currentfill}%
\pgfsetfillopacity{0.926158}%
\pgfsetlinewidth{1.003750pt}%
\definecolor{currentstroke}{rgb}{0.121569,0.466667,0.705882}%
\pgfsetstrokecolor{currentstroke}%
\pgfsetstrokeopacity{0.926158}%
\pgfsetdash{}{0pt}%
\pgfpathmoveto{\pgfqpoint{2.231764in}{2.235333in}}%
\pgfpathcurveto{\pgfqpoint{2.240000in}{2.235333in}}{\pgfqpoint{2.247900in}{2.238605in}}{\pgfqpoint{2.253724in}{2.244429in}}%
\pgfpathcurveto{\pgfqpoint{2.259548in}{2.250253in}}{\pgfqpoint{2.262820in}{2.258153in}}{\pgfqpoint{2.262820in}{2.266389in}}%
\pgfpathcurveto{\pgfqpoint{2.262820in}{2.274625in}}{\pgfqpoint{2.259548in}{2.282525in}}{\pgfqpoint{2.253724in}{2.288349in}}%
\pgfpathcurveto{\pgfqpoint{2.247900in}{2.294173in}}{\pgfqpoint{2.240000in}{2.297446in}}{\pgfqpoint{2.231764in}{2.297446in}}%
\pgfpathcurveto{\pgfqpoint{2.223527in}{2.297446in}}{\pgfqpoint{2.215627in}{2.294173in}}{\pgfqpoint{2.209803in}{2.288349in}}%
\pgfpathcurveto{\pgfqpoint{2.203979in}{2.282525in}}{\pgfqpoint{2.200707in}{2.274625in}}{\pgfqpoint{2.200707in}{2.266389in}}%
\pgfpathcurveto{\pgfqpoint{2.200707in}{2.258153in}}{\pgfqpoint{2.203979in}{2.250253in}}{\pgfqpoint{2.209803in}{2.244429in}}%
\pgfpathcurveto{\pgfqpoint{2.215627in}{2.238605in}}{\pgfqpoint{2.223527in}{2.235333in}}{\pgfqpoint{2.231764in}{2.235333in}}%
\pgfpathclose%
\pgfusepath{stroke,fill}%
\end{pgfscope}%
\begin{pgfscope}%
\pgfpathrectangle{\pgfqpoint{0.100000in}{0.212622in}}{\pgfqpoint{3.696000in}{3.696000in}}%
\pgfusepath{clip}%
\pgfsetbuttcap%
\pgfsetroundjoin%
\definecolor{currentfill}{rgb}{0.121569,0.466667,0.705882}%
\pgfsetfillcolor{currentfill}%
\pgfsetfillopacity{0.926241}%
\pgfsetlinewidth{1.003750pt}%
\definecolor{currentstroke}{rgb}{0.121569,0.466667,0.705882}%
\pgfsetstrokecolor{currentstroke}%
\pgfsetstrokeopacity{0.926241}%
\pgfsetdash{}{0pt}%
\pgfpathmoveto{\pgfqpoint{1.254956in}{1.780337in}}%
\pgfpathcurveto{\pgfqpoint{1.263192in}{1.780337in}}{\pgfqpoint{1.271092in}{1.783609in}}{\pgfqpoint{1.276916in}{1.789433in}}%
\pgfpathcurveto{\pgfqpoint{1.282740in}{1.795257in}}{\pgfqpoint{1.286012in}{1.803157in}}{\pgfqpoint{1.286012in}{1.811393in}}%
\pgfpathcurveto{\pgfqpoint{1.286012in}{1.819629in}}{\pgfqpoint{1.282740in}{1.827529in}}{\pgfqpoint{1.276916in}{1.833353in}}%
\pgfpathcurveto{\pgfqpoint{1.271092in}{1.839177in}}{\pgfqpoint{1.263192in}{1.842450in}}{\pgfqpoint{1.254956in}{1.842450in}}%
\pgfpathcurveto{\pgfqpoint{1.246719in}{1.842450in}}{\pgfqpoint{1.238819in}{1.839177in}}{\pgfqpoint{1.232995in}{1.833353in}}%
\pgfpathcurveto{\pgfqpoint{1.227171in}{1.827529in}}{\pgfqpoint{1.223899in}{1.819629in}}{\pgfqpoint{1.223899in}{1.811393in}}%
\pgfpathcurveto{\pgfqpoint{1.223899in}{1.803157in}}{\pgfqpoint{1.227171in}{1.795257in}}{\pgfqpoint{1.232995in}{1.789433in}}%
\pgfpathcurveto{\pgfqpoint{1.238819in}{1.783609in}}{\pgfqpoint{1.246719in}{1.780337in}}{\pgfqpoint{1.254956in}{1.780337in}}%
\pgfpathclose%
\pgfusepath{stroke,fill}%
\end{pgfscope}%
\begin{pgfscope}%
\pgfpathrectangle{\pgfqpoint{0.100000in}{0.212622in}}{\pgfqpoint{3.696000in}{3.696000in}}%
\pgfusepath{clip}%
\pgfsetbuttcap%
\pgfsetroundjoin%
\definecolor{currentfill}{rgb}{0.121569,0.466667,0.705882}%
\pgfsetfillcolor{currentfill}%
\pgfsetfillopacity{0.927182}%
\pgfsetlinewidth{1.003750pt}%
\definecolor{currentstroke}{rgb}{0.121569,0.466667,0.705882}%
\pgfsetstrokecolor{currentstroke}%
\pgfsetstrokeopacity{0.927182}%
\pgfsetdash{}{0pt}%
\pgfpathmoveto{\pgfqpoint{2.225665in}{2.231211in}}%
\pgfpathcurveto{\pgfqpoint{2.233901in}{2.231211in}}{\pgfqpoint{2.241801in}{2.234483in}}{\pgfqpoint{2.247625in}{2.240307in}}%
\pgfpathcurveto{\pgfqpoint{2.253449in}{2.246131in}}{\pgfqpoint{2.256721in}{2.254031in}}{\pgfqpoint{2.256721in}{2.262267in}}%
\pgfpathcurveto{\pgfqpoint{2.256721in}{2.270503in}}{\pgfqpoint{2.253449in}{2.278403in}}{\pgfqpoint{2.247625in}{2.284227in}}%
\pgfpathcurveto{\pgfqpoint{2.241801in}{2.290051in}}{\pgfqpoint{2.233901in}{2.293324in}}{\pgfqpoint{2.225665in}{2.293324in}}%
\pgfpathcurveto{\pgfqpoint{2.217428in}{2.293324in}}{\pgfqpoint{2.209528in}{2.290051in}}{\pgfqpoint{2.203704in}{2.284227in}}%
\pgfpathcurveto{\pgfqpoint{2.197881in}{2.278403in}}{\pgfqpoint{2.194608in}{2.270503in}}{\pgfqpoint{2.194608in}{2.262267in}}%
\pgfpathcurveto{\pgfqpoint{2.194608in}{2.254031in}}{\pgfqpoint{2.197881in}{2.246131in}}{\pgfqpoint{2.203704in}{2.240307in}}%
\pgfpathcurveto{\pgfqpoint{2.209528in}{2.234483in}}{\pgfqpoint{2.217428in}{2.231211in}}{\pgfqpoint{2.225665in}{2.231211in}}%
\pgfpathclose%
\pgfusepath{stroke,fill}%
\end{pgfscope}%
\begin{pgfscope}%
\pgfpathrectangle{\pgfqpoint{0.100000in}{0.212622in}}{\pgfqpoint{3.696000in}{3.696000in}}%
\pgfusepath{clip}%
\pgfsetbuttcap%
\pgfsetroundjoin%
\definecolor{currentfill}{rgb}{0.121569,0.466667,0.705882}%
\pgfsetfillcolor{currentfill}%
\pgfsetfillopacity{0.927368}%
\pgfsetlinewidth{1.003750pt}%
\definecolor{currentstroke}{rgb}{0.121569,0.466667,0.705882}%
\pgfsetstrokecolor{currentstroke}%
\pgfsetstrokeopacity{0.927368}%
\pgfsetdash{}{0pt}%
\pgfpathmoveto{\pgfqpoint{1.816583in}{2.083892in}}%
\pgfpathcurveto{\pgfqpoint{1.824820in}{2.083892in}}{\pgfqpoint{1.832720in}{2.087165in}}{\pgfqpoint{1.838544in}{2.092989in}}%
\pgfpathcurveto{\pgfqpoint{1.844367in}{2.098813in}}{\pgfqpoint{1.847640in}{2.106713in}}{\pgfqpoint{1.847640in}{2.114949in}}%
\pgfpathcurveto{\pgfqpoint{1.847640in}{2.123185in}}{\pgfqpoint{1.844367in}{2.131085in}}{\pgfqpoint{1.838544in}{2.136909in}}%
\pgfpathcurveto{\pgfqpoint{1.832720in}{2.142733in}}{\pgfqpoint{1.824820in}{2.146005in}}{\pgfqpoint{1.816583in}{2.146005in}}%
\pgfpathcurveto{\pgfqpoint{1.808347in}{2.146005in}}{\pgfqpoint{1.800447in}{2.142733in}}{\pgfqpoint{1.794623in}{2.136909in}}%
\pgfpathcurveto{\pgfqpoint{1.788799in}{2.131085in}}{\pgfqpoint{1.785527in}{2.123185in}}{\pgfqpoint{1.785527in}{2.114949in}}%
\pgfpathcurveto{\pgfqpoint{1.785527in}{2.106713in}}{\pgfqpoint{1.788799in}{2.098813in}}{\pgfqpoint{1.794623in}{2.092989in}}%
\pgfpathcurveto{\pgfqpoint{1.800447in}{2.087165in}}{\pgfqpoint{1.808347in}{2.083892in}}{\pgfqpoint{1.816583in}{2.083892in}}%
\pgfpathclose%
\pgfusepath{stroke,fill}%
\end{pgfscope}%
\begin{pgfscope}%
\pgfpathrectangle{\pgfqpoint{0.100000in}{0.212622in}}{\pgfqpoint{3.696000in}{3.696000in}}%
\pgfusepath{clip}%
\pgfsetbuttcap%
\pgfsetroundjoin%
\definecolor{currentfill}{rgb}{0.121569,0.466667,0.705882}%
\pgfsetfillcolor{currentfill}%
\pgfsetfillopacity{0.927389}%
\pgfsetlinewidth{1.003750pt}%
\definecolor{currentstroke}{rgb}{0.121569,0.466667,0.705882}%
\pgfsetstrokecolor{currentstroke}%
\pgfsetstrokeopacity{0.927389}%
\pgfsetdash{}{0pt}%
\pgfpathmoveto{\pgfqpoint{1.466332in}{1.910658in}}%
\pgfpathcurveto{\pgfqpoint{1.474568in}{1.910658in}}{\pgfqpoint{1.482468in}{1.913930in}}{\pgfqpoint{1.488292in}{1.919754in}}%
\pgfpathcurveto{\pgfqpoint{1.494116in}{1.925578in}}{\pgfqpoint{1.497388in}{1.933478in}}{\pgfqpoint{1.497388in}{1.941715in}}%
\pgfpathcurveto{\pgfqpoint{1.497388in}{1.949951in}}{\pgfqpoint{1.494116in}{1.957851in}}{\pgfqpoint{1.488292in}{1.963675in}}%
\pgfpathcurveto{\pgfqpoint{1.482468in}{1.969499in}}{\pgfqpoint{1.474568in}{1.972771in}}{\pgfqpoint{1.466332in}{1.972771in}}%
\pgfpathcurveto{\pgfqpoint{1.458096in}{1.972771in}}{\pgfqpoint{1.450196in}{1.969499in}}{\pgfqpoint{1.444372in}{1.963675in}}%
\pgfpathcurveto{\pgfqpoint{1.438548in}{1.957851in}}{\pgfqpoint{1.435275in}{1.949951in}}{\pgfqpoint{1.435275in}{1.941715in}}%
\pgfpathcurveto{\pgfqpoint{1.435275in}{1.933478in}}{\pgfqpoint{1.438548in}{1.925578in}}{\pgfqpoint{1.444372in}{1.919754in}}%
\pgfpathcurveto{\pgfqpoint{1.450196in}{1.913930in}}{\pgfqpoint{1.458096in}{1.910658in}}{\pgfqpoint{1.466332in}{1.910658in}}%
\pgfpathclose%
\pgfusepath{stroke,fill}%
\end{pgfscope}%
\begin{pgfscope}%
\pgfpathrectangle{\pgfqpoint{0.100000in}{0.212622in}}{\pgfqpoint{3.696000in}{3.696000in}}%
\pgfusepath{clip}%
\pgfsetbuttcap%
\pgfsetroundjoin%
\definecolor{currentfill}{rgb}{0.121569,0.466667,0.705882}%
\pgfsetfillcolor{currentfill}%
\pgfsetfillopacity{0.928970}%
\pgfsetlinewidth{1.003750pt}%
\definecolor{currentstroke}{rgb}{0.121569,0.466667,0.705882}%
\pgfsetstrokecolor{currentstroke}%
\pgfsetstrokeopacity{0.928970}%
\pgfsetdash{}{0pt}%
\pgfpathmoveto{\pgfqpoint{1.827235in}{2.082467in}}%
\pgfpathcurveto{\pgfqpoint{1.835471in}{2.082467in}}{\pgfqpoint{1.843371in}{2.085739in}}{\pgfqpoint{1.849195in}{2.091563in}}%
\pgfpathcurveto{\pgfqpoint{1.855019in}{2.097387in}}{\pgfqpoint{1.858291in}{2.105287in}}{\pgfqpoint{1.858291in}{2.113524in}}%
\pgfpathcurveto{\pgfqpoint{1.858291in}{2.121760in}}{\pgfqpoint{1.855019in}{2.129660in}}{\pgfqpoint{1.849195in}{2.135484in}}%
\pgfpathcurveto{\pgfqpoint{1.843371in}{2.141308in}}{\pgfqpoint{1.835471in}{2.144580in}}{\pgfqpoint{1.827235in}{2.144580in}}%
\pgfpathcurveto{\pgfqpoint{1.818999in}{2.144580in}}{\pgfqpoint{1.811099in}{2.141308in}}{\pgfqpoint{1.805275in}{2.135484in}}%
\pgfpathcurveto{\pgfqpoint{1.799451in}{2.129660in}}{\pgfqpoint{1.796178in}{2.121760in}}{\pgfqpoint{1.796178in}{2.113524in}}%
\pgfpathcurveto{\pgfqpoint{1.796178in}{2.105287in}}{\pgfqpoint{1.799451in}{2.097387in}}{\pgfqpoint{1.805275in}{2.091563in}}%
\pgfpathcurveto{\pgfqpoint{1.811099in}{2.085739in}}{\pgfqpoint{1.818999in}{2.082467in}}{\pgfqpoint{1.827235in}{2.082467in}}%
\pgfpathclose%
\pgfusepath{stroke,fill}%
\end{pgfscope}%
\begin{pgfscope}%
\pgfpathrectangle{\pgfqpoint{0.100000in}{0.212622in}}{\pgfqpoint{3.696000in}{3.696000in}}%
\pgfusepath{clip}%
\pgfsetbuttcap%
\pgfsetroundjoin%
\definecolor{currentfill}{rgb}{0.121569,0.466667,0.705882}%
\pgfsetfillcolor{currentfill}%
\pgfsetfillopacity{0.929801}%
\pgfsetlinewidth{1.003750pt}%
\definecolor{currentstroke}{rgb}{0.121569,0.466667,0.705882}%
\pgfsetstrokecolor{currentstroke}%
\pgfsetstrokeopacity{0.929801}%
\pgfsetdash{}{0pt}%
\pgfpathmoveto{\pgfqpoint{2.214114in}{2.226399in}}%
\pgfpathcurveto{\pgfqpoint{2.222350in}{2.226399in}}{\pgfqpoint{2.230250in}{2.229671in}}{\pgfqpoint{2.236074in}{2.235495in}}%
\pgfpathcurveto{\pgfqpoint{2.241898in}{2.241319in}}{\pgfqpoint{2.245171in}{2.249219in}}{\pgfqpoint{2.245171in}{2.257455in}}%
\pgfpathcurveto{\pgfqpoint{2.245171in}{2.265691in}}{\pgfqpoint{2.241898in}{2.273591in}}{\pgfqpoint{2.236074in}{2.279415in}}%
\pgfpathcurveto{\pgfqpoint{2.230250in}{2.285239in}}{\pgfqpoint{2.222350in}{2.288512in}}{\pgfqpoint{2.214114in}{2.288512in}}%
\pgfpathcurveto{\pgfqpoint{2.205878in}{2.288512in}}{\pgfqpoint{2.197978in}{2.285239in}}{\pgfqpoint{2.192154in}{2.279415in}}%
\pgfpathcurveto{\pgfqpoint{2.186330in}{2.273591in}}{\pgfqpoint{2.183058in}{2.265691in}}{\pgfqpoint{2.183058in}{2.257455in}}%
\pgfpathcurveto{\pgfqpoint{2.183058in}{2.249219in}}{\pgfqpoint{2.186330in}{2.241319in}}{\pgfqpoint{2.192154in}{2.235495in}}%
\pgfpathcurveto{\pgfqpoint{2.197978in}{2.229671in}}{\pgfqpoint{2.205878in}{2.226399in}}{\pgfqpoint{2.214114in}{2.226399in}}%
\pgfpathclose%
\pgfusepath{stroke,fill}%
\end{pgfscope}%
\begin{pgfscope}%
\pgfpathrectangle{\pgfqpoint{0.100000in}{0.212622in}}{\pgfqpoint{3.696000in}{3.696000in}}%
\pgfusepath{clip}%
\pgfsetbuttcap%
\pgfsetroundjoin%
\definecolor{currentfill}{rgb}{0.121569,0.466667,0.705882}%
\pgfsetfillcolor{currentfill}%
\pgfsetfillopacity{0.930035}%
\pgfsetlinewidth{1.003750pt}%
\definecolor{currentstroke}{rgb}{0.121569,0.466667,0.705882}%
\pgfsetstrokecolor{currentstroke}%
\pgfsetstrokeopacity{0.930035}%
\pgfsetdash{}{0pt}%
\pgfpathmoveto{\pgfqpoint{1.286489in}{1.787618in}}%
\pgfpathcurveto{\pgfqpoint{1.294725in}{1.787618in}}{\pgfqpoint{1.302625in}{1.790890in}}{\pgfqpoint{1.308449in}{1.796714in}}%
\pgfpathcurveto{\pgfqpoint{1.314273in}{1.802538in}}{\pgfqpoint{1.317545in}{1.810438in}}{\pgfqpoint{1.317545in}{1.818675in}}%
\pgfpathcurveto{\pgfqpoint{1.317545in}{1.826911in}}{\pgfqpoint{1.314273in}{1.834811in}}{\pgfqpoint{1.308449in}{1.840635in}}%
\pgfpathcurveto{\pgfqpoint{1.302625in}{1.846459in}}{\pgfqpoint{1.294725in}{1.849731in}}{\pgfqpoint{1.286489in}{1.849731in}}%
\pgfpathcurveto{\pgfqpoint{1.278252in}{1.849731in}}{\pgfqpoint{1.270352in}{1.846459in}}{\pgfqpoint{1.264528in}{1.840635in}}%
\pgfpathcurveto{\pgfqpoint{1.258704in}{1.834811in}}{\pgfqpoint{1.255432in}{1.826911in}}{\pgfqpoint{1.255432in}{1.818675in}}%
\pgfpathcurveto{\pgfqpoint{1.255432in}{1.810438in}}{\pgfqpoint{1.258704in}{1.802538in}}{\pgfqpoint{1.264528in}{1.796714in}}%
\pgfpathcurveto{\pgfqpoint{1.270352in}{1.790890in}}{\pgfqpoint{1.278252in}{1.787618in}}{\pgfqpoint{1.286489in}{1.787618in}}%
\pgfpathclose%
\pgfusepath{stroke,fill}%
\end{pgfscope}%
\begin{pgfscope}%
\pgfpathrectangle{\pgfqpoint{0.100000in}{0.212622in}}{\pgfqpoint{3.696000in}{3.696000in}}%
\pgfusepath{clip}%
\pgfsetbuttcap%
\pgfsetroundjoin%
\definecolor{currentfill}{rgb}{0.121569,0.466667,0.705882}%
\pgfsetfillcolor{currentfill}%
\pgfsetfillopacity{0.932024}%
\pgfsetlinewidth{1.003750pt}%
\definecolor{currentstroke}{rgb}{0.121569,0.466667,0.705882}%
\pgfsetstrokecolor{currentstroke}%
\pgfsetstrokeopacity{0.932024}%
\pgfsetdash{}{0pt}%
\pgfpathmoveto{\pgfqpoint{1.832571in}{2.088577in}}%
\pgfpathcurveto{\pgfqpoint{1.840807in}{2.088577in}}{\pgfqpoint{1.848707in}{2.091850in}}{\pgfqpoint{1.854531in}{2.097673in}}%
\pgfpathcurveto{\pgfqpoint{1.860355in}{2.103497in}}{\pgfqpoint{1.863627in}{2.111397in}}{\pgfqpoint{1.863627in}{2.119634in}}%
\pgfpathcurveto{\pgfqpoint{1.863627in}{2.127870in}}{\pgfqpoint{1.860355in}{2.135770in}}{\pgfqpoint{1.854531in}{2.141594in}}%
\pgfpathcurveto{\pgfqpoint{1.848707in}{2.147418in}}{\pgfqpoint{1.840807in}{2.150690in}}{\pgfqpoint{1.832571in}{2.150690in}}%
\pgfpathcurveto{\pgfqpoint{1.824335in}{2.150690in}}{\pgfqpoint{1.816435in}{2.147418in}}{\pgfqpoint{1.810611in}{2.141594in}}%
\pgfpathcurveto{\pgfqpoint{1.804787in}{2.135770in}}{\pgfqpoint{1.801515in}{2.127870in}}{\pgfqpoint{1.801515in}{2.119634in}}%
\pgfpathcurveto{\pgfqpoint{1.801515in}{2.111397in}}{\pgfqpoint{1.804787in}{2.103497in}}{\pgfqpoint{1.810611in}{2.097673in}}%
\pgfpathcurveto{\pgfqpoint{1.816435in}{2.091850in}}{\pgfqpoint{1.824335in}{2.088577in}}{\pgfqpoint{1.832571in}{2.088577in}}%
\pgfpathclose%
\pgfusepath{stroke,fill}%
\end{pgfscope}%
\begin{pgfscope}%
\pgfpathrectangle{\pgfqpoint{0.100000in}{0.212622in}}{\pgfqpoint{3.696000in}{3.696000in}}%
\pgfusepath{clip}%
\pgfsetbuttcap%
\pgfsetroundjoin%
\definecolor{currentfill}{rgb}{0.121569,0.466667,0.705882}%
\pgfsetfillcolor{currentfill}%
\pgfsetfillopacity{0.932278}%
\pgfsetlinewidth{1.003750pt}%
\definecolor{currentstroke}{rgb}{0.121569,0.466667,0.705882}%
\pgfsetstrokecolor{currentstroke}%
\pgfsetstrokeopacity{0.932278}%
\pgfsetdash{}{0pt}%
\pgfpathmoveto{\pgfqpoint{1.287110in}{1.790332in}}%
\pgfpathcurveto{\pgfqpoint{1.295347in}{1.790332in}}{\pgfqpoint{1.303247in}{1.793604in}}{\pgfqpoint{1.309071in}{1.799428in}}%
\pgfpathcurveto{\pgfqpoint{1.314894in}{1.805252in}}{\pgfqpoint{1.318167in}{1.813152in}}{\pgfqpoint{1.318167in}{1.821388in}}%
\pgfpathcurveto{\pgfqpoint{1.318167in}{1.829624in}}{\pgfqpoint{1.314894in}{1.837524in}}{\pgfqpoint{1.309071in}{1.843348in}}%
\pgfpathcurveto{\pgfqpoint{1.303247in}{1.849172in}}{\pgfqpoint{1.295347in}{1.852445in}}{\pgfqpoint{1.287110in}{1.852445in}}%
\pgfpathcurveto{\pgfqpoint{1.278874in}{1.852445in}}{\pgfqpoint{1.270974in}{1.849172in}}{\pgfqpoint{1.265150in}{1.843348in}}%
\pgfpathcurveto{\pgfqpoint{1.259326in}{1.837524in}}{\pgfqpoint{1.256054in}{1.829624in}}{\pgfqpoint{1.256054in}{1.821388in}}%
\pgfpathcurveto{\pgfqpoint{1.256054in}{1.813152in}}{\pgfqpoint{1.259326in}{1.805252in}}{\pgfqpoint{1.265150in}{1.799428in}}%
\pgfpathcurveto{\pgfqpoint{1.270974in}{1.793604in}}{\pgfqpoint{1.278874in}{1.790332in}}{\pgfqpoint{1.287110in}{1.790332in}}%
\pgfpathclose%
\pgfusepath{stroke,fill}%
\end{pgfscope}%
\begin{pgfscope}%
\pgfpathrectangle{\pgfqpoint{0.100000in}{0.212622in}}{\pgfqpoint{3.696000in}{3.696000in}}%
\pgfusepath{clip}%
\pgfsetbuttcap%
\pgfsetroundjoin%
\definecolor{currentfill}{rgb}{0.121569,0.466667,0.705882}%
\pgfsetfillcolor{currentfill}%
\pgfsetfillopacity{0.932471}%
\pgfsetlinewidth{1.003750pt}%
\definecolor{currentstroke}{rgb}{0.121569,0.466667,0.705882}%
\pgfsetstrokecolor{currentstroke}%
\pgfsetstrokeopacity{0.932471}%
\pgfsetdash{}{0pt}%
\pgfpathmoveto{\pgfqpoint{1.272767in}{1.778825in}}%
\pgfpathcurveto{\pgfqpoint{1.281003in}{1.778825in}}{\pgfqpoint{1.288903in}{1.782097in}}{\pgfqpoint{1.294727in}{1.787921in}}%
\pgfpathcurveto{\pgfqpoint{1.300551in}{1.793745in}}{\pgfqpoint{1.303823in}{1.801645in}}{\pgfqpoint{1.303823in}{1.809882in}}%
\pgfpathcurveto{\pgfqpoint{1.303823in}{1.818118in}}{\pgfqpoint{1.300551in}{1.826018in}}{\pgfqpoint{1.294727in}{1.831842in}}%
\pgfpathcurveto{\pgfqpoint{1.288903in}{1.837666in}}{\pgfqpoint{1.281003in}{1.840938in}}{\pgfqpoint{1.272767in}{1.840938in}}%
\pgfpathcurveto{\pgfqpoint{1.264530in}{1.840938in}}{\pgfqpoint{1.256630in}{1.837666in}}{\pgfqpoint{1.250806in}{1.831842in}}%
\pgfpathcurveto{\pgfqpoint{1.244983in}{1.826018in}}{\pgfqpoint{1.241710in}{1.818118in}}{\pgfqpoint{1.241710in}{1.809882in}}%
\pgfpathcurveto{\pgfqpoint{1.241710in}{1.801645in}}{\pgfqpoint{1.244983in}{1.793745in}}{\pgfqpoint{1.250806in}{1.787921in}}%
\pgfpathcurveto{\pgfqpoint{1.256630in}{1.782097in}}{\pgfqpoint{1.264530in}{1.778825in}}{\pgfqpoint{1.272767in}{1.778825in}}%
\pgfpathclose%
\pgfusepath{stroke,fill}%
\end{pgfscope}%
\begin{pgfscope}%
\pgfpathrectangle{\pgfqpoint{0.100000in}{0.212622in}}{\pgfqpoint{3.696000in}{3.696000in}}%
\pgfusepath{clip}%
\pgfsetbuttcap%
\pgfsetroundjoin%
\definecolor{currentfill}{rgb}{0.121569,0.466667,0.705882}%
\pgfsetfillcolor{currentfill}%
\pgfsetfillopacity{0.932968}%
\pgfsetlinewidth{1.003750pt}%
\definecolor{currentstroke}{rgb}{0.121569,0.466667,0.705882}%
\pgfsetstrokecolor{currentstroke}%
\pgfsetstrokeopacity{0.932968}%
\pgfsetdash{}{0pt}%
\pgfpathmoveto{\pgfqpoint{1.810265in}{2.073129in}}%
\pgfpathcurveto{\pgfqpoint{1.818501in}{2.073129in}}{\pgfqpoint{1.826401in}{2.076401in}}{\pgfqpoint{1.832225in}{2.082225in}}%
\pgfpathcurveto{\pgfqpoint{1.838049in}{2.088049in}}{\pgfqpoint{1.841321in}{2.095949in}}{\pgfqpoint{1.841321in}{2.104185in}}%
\pgfpathcurveto{\pgfqpoint{1.841321in}{2.112421in}}{\pgfqpoint{1.838049in}{2.120321in}}{\pgfqpoint{1.832225in}{2.126145in}}%
\pgfpathcurveto{\pgfqpoint{1.826401in}{2.131969in}}{\pgfqpoint{1.818501in}{2.135242in}}{\pgfqpoint{1.810265in}{2.135242in}}%
\pgfpathcurveto{\pgfqpoint{1.802028in}{2.135242in}}{\pgfqpoint{1.794128in}{2.131969in}}{\pgfqpoint{1.788304in}{2.126145in}}%
\pgfpathcurveto{\pgfqpoint{1.782481in}{2.120321in}}{\pgfqpoint{1.779208in}{2.112421in}}{\pgfqpoint{1.779208in}{2.104185in}}%
\pgfpathcurveto{\pgfqpoint{1.779208in}{2.095949in}}{\pgfqpoint{1.782481in}{2.088049in}}{\pgfqpoint{1.788304in}{2.082225in}}%
\pgfpathcurveto{\pgfqpoint{1.794128in}{2.076401in}}{\pgfqpoint{1.802028in}{2.073129in}}{\pgfqpoint{1.810265in}{2.073129in}}%
\pgfpathclose%
\pgfusepath{stroke,fill}%
\end{pgfscope}%
\begin{pgfscope}%
\pgfpathrectangle{\pgfqpoint{0.100000in}{0.212622in}}{\pgfqpoint{3.696000in}{3.696000in}}%
\pgfusepath{clip}%
\pgfsetbuttcap%
\pgfsetroundjoin%
\definecolor{currentfill}{rgb}{0.121569,0.466667,0.705882}%
\pgfsetfillcolor{currentfill}%
\pgfsetfillopacity{0.933380}%
\pgfsetlinewidth{1.003750pt}%
\definecolor{currentstroke}{rgb}{0.121569,0.466667,0.705882}%
\pgfsetstrokecolor{currentstroke}%
\pgfsetstrokeopacity{0.933380}%
\pgfsetdash{}{0pt}%
\pgfpathmoveto{\pgfqpoint{1.259781in}{1.772590in}}%
\pgfpathcurveto{\pgfqpoint{1.268018in}{1.772590in}}{\pgfqpoint{1.275918in}{1.775863in}}{\pgfqpoint{1.281742in}{1.781687in}}%
\pgfpathcurveto{\pgfqpoint{1.287566in}{1.787510in}}{\pgfqpoint{1.290838in}{1.795410in}}{\pgfqpoint{1.290838in}{1.803647in}}%
\pgfpathcurveto{\pgfqpoint{1.290838in}{1.811883in}}{\pgfqpoint{1.287566in}{1.819783in}}{\pgfqpoint{1.281742in}{1.825607in}}%
\pgfpathcurveto{\pgfqpoint{1.275918in}{1.831431in}}{\pgfqpoint{1.268018in}{1.834703in}}{\pgfqpoint{1.259781in}{1.834703in}}%
\pgfpathcurveto{\pgfqpoint{1.251545in}{1.834703in}}{\pgfqpoint{1.243645in}{1.831431in}}{\pgfqpoint{1.237821in}{1.825607in}}%
\pgfpathcurveto{\pgfqpoint{1.231997in}{1.819783in}}{\pgfqpoint{1.228725in}{1.811883in}}{\pgfqpoint{1.228725in}{1.803647in}}%
\pgfpathcurveto{\pgfqpoint{1.228725in}{1.795410in}}{\pgfqpoint{1.231997in}{1.787510in}}{\pgfqpoint{1.237821in}{1.781687in}}%
\pgfpathcurveto{\pgfqpoint{1.243645in}{1.775863in}}{\pgfqpoint{1.251545in}{1.772590in}}{\pgfqpoint{1.259781in}{1.772590in}}%
\pgfpathclose%
\pgfusepath{stroke,fill}%
\end{pgfscope}%
\begin{pgfscope}%
\pgfpathrectangle{\pgfqpoint{0.100000in}{0.212622in}}{\pgfqpoint{3.696000in}{3.696000in}}%
\pgfusepath{clip}%
\pgfsetbuttcap%
\pgfsetroundjoin%
\definecolor{currentfill}{rgb}{0.121569,0.466667,0.705882}%
\pgfsetfillcolor{currentfill}%
\pgfsetfillopacity{0.934014}%
\pgfsetlinewidth{1.003750pt}%
\definecolor{currentstroke}{rgb}{0.121569,0.466667,0.705882}%
\pgfsetstrokecolor{currentstroke}%
\pgfsetstrokeopacity{0.934014}%
\pgfsetdash{}{0pt}%
\pgfpathmoveto{\pgfqpoint{1.251033in}{1.770372in}}%
\pgfpathcurveto{\pgfqpoint{1.259269in}{1.770372in}}{\pgfqpoint{1.267169in}{1.773645in}}{\pgfqpoint{1.272993in}{1.779469in}}%
\pgfpathcurveto{\pgfqpoint{1.278817in}{1.785293in}}{\pgfqpoint{1.282090in}{1.793193in}}{\pgfqpoint{1.282090in}{1.801429in}}%
\pgfpathcurveto{\pgfqpoint{1.282090in}{1.809665in}}{\pgfqpoint{1.278817in}{1.817565in}}{\pgfqpoint{1.272993in}{1.823389in}}%
\pgfpathcurveto{\pgfqpoint{1.267169in}{1.829213in}}{\pgfqpoint{1.259269in}{1.832485in}}{\pgfqpoint{1.251033in}{1.832485in}}%
\pgfpathcurveto{\pgfqpoint{1.242797in}{1.832485in}}{\pgfqpoint{1.234897in}{1.829213in}}{\pgfqpoint{1.229073in}{1.823389in}}%
\pgfpathcurveto{\pgfqpoint{1.223249in}{1.817565in}}{\pgfqpoint{1.219977in}{1.809665in}}{\pgfqpoint{1.219977in}{1.801429in}}%
\pgfpathcurveto{\pgfqpoint{1.219977in}{1.793193in}}{\pgfqpoint{1.223249in}{1.785293in}}{\pgfqpoint{1.229073in}{1.779469in}}%
\pgfpathcurveto{\pgfqpoint{1.234897in}{1.773645in}}{\pgfqpoint{1.242797in}{1.770372in}}{\pgfqpoint{1.251033in}{1.770372in}}%
\pgfpathclose%
\pgfusepath{stroke,fill}%
\end{pgfscope}%
\begin{pgfscope}%
\pgfpathrectangle{\pgfqpoint{0.100000in}{0.212622in}}{\pgfqpoint{3.696000in}{3.696000in}}%
\pgfusepath{clip}%
\pgfsetbuttcap%
\pgfsetroundjoin%
\definecolor{currentfill}{rgb}{0.121569,0.466667,0.705882}%
\pgfsetfillcolor{currentfill}%
\pgfsetfillopacity{0.934637}%
\pgfsetlinewidth{1.003750pt}%
\definecolor{currentstroke}{rgb}{0.121569,0.466667,0.705882}%
\pgfsetstrokecolor{currentstroke}%
\pgfsetstrokeopacity{0.934637}%
\pgfsetdash{}{0pt}%
\pgfpathmoveto{\pgfqpoint{1.294995in}{1.795617in}}%
\pgfpathcurveto{\pgfqpoint{1.303231in}{1.795617in}}{\pgfqpoint{1.311131in}{1.798889in}}{\pgfqpoint{1.316955in}{1.804713in}}%
\pgfpathcurveto{\pgfqpoint{1.322779in}{1.810537in}}{\pgfqpoint{1.326051in}{1.818437in}}{\pgfqpoint{1.326051in}{1.826674in}}%
\pgfpathcurveto{\pgfqpoint{1.326051in}{1.834910in}}{\pgfqpoint{1.322779in}{1.842810in}}{\pgfqpoint{1.316955in}{1.848634in}}%
\pgfpathcurveto{\pgfqpoint{1.311131in}{1.854458in}}{\pgfqpoint{1.303231in}{1.857730in}}{\pgfqpoint{1.294995in}{1.857730in}}%
\pgfpathcurveto{\pgfqpoint{1.286758in}{1.857730in}}{\pgfqpoint{1.278858in}{1.854458in}}{\pgfqpoint{1.273034in}{1.848634in}}%
\pgfpathcurveto{\pgfqpoint{1.267211in}{1.842810in}}{\pgfqpoint{1.263938in}{1.834910in}}{\pgfqpoint{1.263938in}{1.826674in}}%
\pgfpathcurveto{\pgfqpoint{1.263938in}{1.818437in}}{\pgfqpoint{1.267211in}{1.810537in}}{\pgfqpoint{1.273034in}{1.804713in}}%
\pgfpathcurveto{\pgfqpoint{1.278858in}{1.798889in}}{\pgfqpoint{1.286758in}{1.795617in}}{\pgfqpoint{1.294995in}{1.795617in}}%
\pgfpathclose%
\pgfusepath{stroke,fill}%
\end{pgfscope}%
\begin{pgfscope}%
\pgfpathrectangle{\pgfqpoint{0.100000in}{0.212622in}}{\pgfqpoint{3.696000in}{3.696000in}}%
\pgfusepath{clip}%
\pgfsetbuttcap%
\pgfsetroundjoin%
\definecolor{currentfill}{rgb}{0.121569,0.466667,0.705882}%
\pgfsetfillcolor{currentfill}%
\pgfsetfillopacity{0.935541}%
\pgfsetlinewidth{1.003750pt}%
\definecolor{currentstroke}{rgb}{0.121569,0.466667,0.705882}%
\pgfsetstrokecolor{currentstroke}%
\pgfsetstrokeopacity{0.935541}%
\pgfsetdash{}{0pt}%
\pgfpathmoveto{\pgfqpoint{2.188227in}{2.206271in}}%
\pgfpathcurveto{\pgfqpoint{2.196463in}{2.206271in}}{\pgfqpoint{2.204363in}{2.209543in}}{\pgfqpoint{2.210187in}{2.215367in}}%
\pgfpathcurveto{\pgfqpoint{2.216011in}{2.221191in}}{\pgfqpoint{2.219284in}{2.229091in}}{\pgfqpoint{2.219284in}{2.237328in}}%
\pgfpathcurveto{\pgfqpoint{2.219284in}{2.245564in}}{\pgfqpoint{2.216011in}{2.253464in}}{\pgfqpoint{2.210187in}{2.259288in}}%
\pgfpathcurveto{\pgfqpoint{2.204363in}{2.265112in}}{\pgfqpoint{2.196463in}{2.268384in}}{\pgfqpoint{2.188227in}{2.268384in}}%
\pgfpathcurveto{\pgfqpoint{2.179991in}{2.268384in}}{\pgfqpoint{2.172091in}{2.265112in}}{\pgfqpoint{2.166267in}{2.259288in}}%
\pgfpathcurveto{\pgfqpoint{2.160443in}{2.253464in}}{\pgfqpoint{2.157171in}{2.245564in}}{\pgfqpoint{2.157171in}{2.237328in}}%
\pgfpathcurveto{\pgfqpoint{2.157171in}{2.229091in}}{\pgfqpoint{2.160443in}{2.221191in}}{\pgfqpoint{2.166267in}{2.215367in}}%
\pgfpathcurveto{\pgfqpoint{2.172091in}{2.209543in}}{\pgfqpoint{2.179991in}{2.206271in}}{\pgfqpoint{2.188227in}{2.206271in}}%
\pgfpathclose%
\pgfusepath{stroke,fill}%
\end{pgfscope}%
\begin{pgfscope}%
\pgfpathrectangle{\pgfqpoint{0.100000in}{0.212622in}}{\pgfqpoint{3.696000in}{3.696000in}}%
\pgfusepath{clip}%
\pgfsetbuttcap%
\pgfsetroundjoin%
\definecolor{currentfill}{rgb}{0.121569,0.466667,0.705882}%
\pgfsetfillcolor{currentfill}%
\pgfsetfillopacity{0.935764}%
\pgfsetlinewidth{1.003750pt}%
\definecolor{currentstroke}{rgb}{0.121569,0.466667,0.705882}%
\pgfsetstrokecolor{currentstroke}%
\pgfsetstrokeopacity{0.935764}%
\pgfsetdash{}{0pt}%
\pgfpathmoveto{\pgfqpoint{1.401582in}{1.867898in}}%
\pgfpathcurveto{\pgfqpoint{1.409818in}{1.867898in}}{\pgfqpoint{1.417718in}{1.871170in}}{\pgfqpoint{1.423542in}{1.876994in}}%
\pgfpathcurveto{\pgfqpoint{1.429366in}{1.882818in}}{\pgfqpoint{1.432638in}{1.890718in}}{\pgfqpoint{1.432638in}{1.898954in}}%
\pgfpathcurveto{\pgfqpoint{1.432638in}{1.907190in}}{\pgfqpoint{1.429366in}{1.915090in}}{\pgfqpoint{1.423542in}{1.920914in}}%
\pgfpathcurveto{\pgfqpoint{1.417718in}{1.926738in}}{\pgfqpoint{1.409818in}{1.930011in}}{\pgfqpoint{1.401582in}{1.930011in}}%
\pgfpathcurveto{\pgfqpoint{1.393345in}{1.930011in}}{\pgfqpoint{1.385445in}{1.926738in}}{\pgfqpoint{1.379621in}{1.920914in}}%
\pgfpathcurveto{\pgfqpoint{1.373797in}{1.915090in}}{\pgfqpoint{1.370525in}{1.907190in}}{\pgfqpoint{1.370525in}{1.898954in}}%
\pgfpathcurveto{\pgfqpoint{1.370525in}{1.890718in}}{\pgfqpoint{1.373797in}{1.882818in}}{\pgfqpoint{1.379621in}{1.876994in}}%
\pgfpathcurveto{\pgfqpoint{1.385445in}{1.871170in}}{\pgfqpoint{1.393345in}{1.867898in}}{\pgfqpoint{1.401582in}{1.867898in}}%
\pgfpathclose%
\pgfusepath{stroke,fill}%
\end{pgfscope}%
\begin{pgfscope}%
\pgfpathrectangle{\pgfqpoint{0.100000in}{0.212622in}}{\pgfqpoint{3.696000in}{3.696000in}}%
\pgfusepath{clip}%
\pgfsetbuttcap%
\pgfsetroundjoin%
\definecolor{currentfill}{rgb}{0.121569,0.466667,0.705882}%
\pgfsetfillcolor{currentfill}%
\pgfsetfillopacity{0.935982}%
\pgfsetlinewidth{1.003750pt}%
\definecolor{currentstroke}{rgb}{0.121569,0.466667,0.705882}%
\pgfsetstrokecolor{currentstroke}%
\pgfsetstrokeopacity{0.935982}%
\pgfsetdash{}{0pt}%
\pgfpathmoveto{\pgfqpoint{1.584641in}{1.973705in}}%
\pgfpathcurveto{\pgfqpoint{1.592877in}{1.973705in}}{\pgfqpoint{1.600777in}{1.976977in}}{\pgfqpoint{1.606601in}{1.982801in}}%
\pgfpathcurveto{\pgfqpoint{1.612425in}{1.988625in}}{\pgfqpoint{1.615697in}{1.996525in}}{\pgfqpoint{1.615697in}{2.004761in}}%
\pgfpathcurveto{\pgfqpoint{1.615697in}{2.012997in}}{\pgfqpoint{1.612425in}{2.020897in}}{\pgfqpoint{1.606601in}{2.026721in}}%
\pgfpathcurveto{\pgfqpoint{1.600777in}{2.032545in}}{\pgfqpoint{1.592877in}{2.035818in}}{\pgfqpoint{1.584641in}{2.035818in}}%
\pgfpathcurveto{\pgfqpoint{1.576405in}{2.035818in}}{\pgfqpoint{1.568504in}{2.032545in}}{\pgfqpoint{1.562681in}{2.026721in}}%
\pgfpathcurveto{\pgfqpoint{1.556857in}{2.020897in}}{\pgfqpoint{1.553584in}{2.012997in}}{\pgfqpoint{1.553584in}{2.004761in}}%
\pgfpathcurveto{\pgfqpoint{1.553584in}{1.996525in}}{\pgfqpoint{1.556857in}{1.988625in}}{\pgfqpoint{1.562681in}{1.982801in}}%
\pgfpathcurveto{\pgfqpoint{1.568504in}{1.976977in}}{\pgfqpoint{1.576405in}{1.973705in}}{\pgfqpoint{1.584641in}{1.973705in}}%
\pgfpathclose%
\pgfusepath{stroke,fill}%
\end{pgfscope}%
\begin{pgfscope}%
\pgfpathrectangle{\pgfqpoint{0.100000in}{0.212622in}}{\pgfqpoint{3.696000in}{3.696000in}}%
\pgfusepath{clip}%
\pgfsetbuttcap%
\pgfsetroundjoin%
\definecolor{currentfill}{rgb}{0.121569,0.466667,0.705882}%
\pgfsetfillcolor{currentfill}%
\pgfsetfillopacity{0.936729}%
\pgfsetlinewidth{1.003750pt}%
\definecolor{currentstroke}{rgb}{0.121569,0.466667,0.705882}%
\pgfsetstrokecolor{currentstroke}%
\pgfsetstrokeopacity{0.936729}%
\pgfsetdash{}{0pt}%
\pgfpathmoveto{\pgfqpoint{1.256214in}{1.767406in}}%
\pgfpathcurveto{\pgfqpoint{1.264450in}{1.767406in}}{\pgfqpoint{1.272350in}{1.770678in}}{\pgfqpoint{1.278174in}{1.776502in}}%
\pgfpathcurveto{\pgfqpoint{1.283998in}{1.782326in}}{\pgfqpoint{1.287270in}{1.790226in}}{\pgfqpoint{1.287270in}{1.798462in}}%
\pgfpathcurveto{\pgfqpoint{1.287270in}{1.806698in}}{\pgfqpoint{1.283998in}{1.814598in}}{\pgfqpoint{1.278174in}{1.820422in}}%
\pgfpathcurveto{\pgfqpoint{1.272350in}{1.826246in}}{\pgfqpoint{1.264450in}{1.829519in}}{\pgfqpoint{1.256214in}{1.829519in}}%
\pgfpathcurveto{\pgfqpoint{1.247977in}{1.829519in}}{\pgfqpoint{1.240077in}{1.826246in}}{\pgfqpoint{1.234253in}{1.820422in}}%
\pgfpathcurveto{\pgfqpoint{1.228429in}{1.814598in}}{\pgfqpoint{1.225157in}{1.806698in}}{\pgfqpoint{1.225157in}{1.798462in}}%
\pgfpathcurveto{\pgfqpoint{1.225157in}{1.790226in}}{\pgfqpoint{1.228429in}{1.782326in}}{\pgfqpoint{1.234253in}{1.776502in}}%
\pgfpathcurveto{\pgfqpoint{1.240077in}{1.770678in}}{\pgfqpoint{1.247977in}{1.767406in}}{\pgfqpoint{1.256214in}{1.767406in}}%
\pgfpathclose%
\pgfusepath{stroke,fill}%
\end{pgfscope}%
\begin{pgfscope}%
\pgfpathrectangle{\pgfqpoint{0.100000in}{0.212622in}}{\pgfqpoint{3.696000in}{3.696000in}}%
\pgfusepath{clip}%
\pgfsetbuttcap%
\pgfsetroundjoin%
\definecolor{currentfill}{rgb}{0.121569,0.466667,0.705882}%
\pgfsetfillcolor{currentfill}%
\pgfsetfillopacity{0.937473}%
\pgfsetlinewidth{1.003750pt}%
\definecolor{currentstroke}{rgb}{0.121569,0.466667,0.705882}%
\pgfsetstrokecolor{currentstroke}%
\pgfsetstrokeopacity{0.937473}%
\pgfsetdash{}{0pt}%
\pgfpathmoveto{\pgfqpoint{1.607779in}{1.979465in}}%
\pgfpathcurveto{\pgfqpoint{1.616015in}{1.979465in}}{\pgfqpoint{1.623915in}{1.982738in}}{\pgfqpoint{1.629739in}{1.988562in}}%
\pgfpathcurveto{\pgfqpoint{1.635563in}{1.994386in}}{\pgfqpoint{1.638835in}{2.002286in}}{\pgfqpoint{1.638835in}{2.010522in}}%
\pgfpathcurveto{\pgfqpoint{1.638835in}{2.018758in}}{\pgfqpoint{1.635563in}{2.026658in}}{\pgfqpoint{1.629739in}{2.032482in}}%
\pgfpathcurveto{\pgfqpoint{1.623915in}{2.038306in}}{\pgfqpoint{1.616015in}{2.041578in}}{\pgfqpoint{1.607779in}{2.041578in}}%
\pgfpathcurveto{\pgfqpoint{1.599543in}{2.041578in}}{\pgfqpoint{1.591643in}{2.038306in}}{\pgfqpoint{1.585819in}{2.032482in}}%
\pgfpathcurveto{\pgfqpoint{1.579995in}{2.026658in}}{\pgfqpoint{1.576722in}{2.018758in}}{\pgfqpoint{1.576722in}{2.010522in}}%
\pgfpathcurveto{\pgfqpoint{1.576722in}{2.002286in}}{\pgfqpoint{1.579995in}{1.994386in}}{\pgfqpoint{1.585819in}{1.988562in}}%
\pgfpathcurveto{\pgfqpoint{1.591643in}{1.982738in}}{\pgfqpoint{1.599543in}{1.979465in}}{\pgfqpoint{1.607779in}{1.979465in}}%
\pgfpathclose%
\pgfusepath{stroke,fill}%
\end{pgfscope}%
\begin{pgfscope}%
\pgfpathrectangle{\pgfqpoint{0.100000in}{0.212622in}}{\pgfqpoint{3.696000in}{3.696000in}}%
\pgfusepath{clip}%
\pgfsetbuttcap%
\pgfsetroundjoin%
\definecolor{currentfill}{rgb}{0.121569,0.466667,0.705882}%
\pgfsetfillcolor{currentfill}%
\pgfsetfillopacity{0.938144}%
\pgfsetlinewidth{1.003750pt}%
\definecolor{currentstroke}{rgb}{0.121569,0.466667,0.705882}%
\pgfsetstrokecolor{currentstroke}%
\pgfsetstrokeopacity{0.938144}%
\pgfsetdash{}{0pt}%
\pgfpathmoveto{\pgfqpoint{1.278178in}{1.779763in}}%
\pgfpathcurveto{\pgfqpoint{1.286414in}{1.779763in}}{\pgfqpoint{1.294314in}{1.783035in}}{\pgfqpoint{1.300138in}{1.788859in}}%
\pgfpathcurveto{\pgfqpoint{1.305962in}{1.794683in}}{\pgfqpoint{1.309234in}{1.802583in}}{\pgfqpoint{1.309234in}{1.810819in}}%
\pgfpathcurveto{\pgfqpoint{1.309234in}{1.819056in}}{\pgfqpoint{1.305962in}{1.826956in}}{\pgfqpoint{1.300138in}{1.832780in}}%
\pgfpathcurveto{\pgfqpoint{1.294314in}{1.838604in}}{\pgfqpoint{1.286414in}{1.841876in}}{\pgfqpoint{1.278178in}{1.841876in}}%
\pgfpathcurveto{\pgfqpoint{1.269941in}{1.841876in}}{\pgfqpoint{1.262041in}{1.838604in}}{\pgfqpoint{1.256217in}{1.832780in}}%
\pgfpathcurveto{\pgfqpoint{1.250394in}{1.826956in}}{\pgfqpoint{1.247121in}{1.819056in}}{\pgfqpoint{1.247121in}{1.810819in}}%
\pgfpathcurveto{\pgfqpoint{1.247121in}{1.802583in}}{\pgfqpoint{1.250394in}{1.794683in}}{\pgfqpoint{1.256217in}{1.788859in}}%
\pgfpathcurveto{\pgfqpoint{1.262041in}{1.783035in}}{\pgfqpoint{1.269941in}{1.779763in}}{\pgfqpoint{1.278178in}{1.779763in}}%
\pgfpathclose%
\pgfusepath{stroke,fill}%
\end{pgfscope}%
\begin{pgfscope}%
\pgfpathrectangle{\pgfqpoint{0.100000in}{0.212622in}}{\pgfqpoint{3.696000in}{3.696000in}}%
\pgfusepath{clip}%
\pgfsetbuttcap%
\pgfsetroundjoin%
\definecolor{currentfill}{rgb}{0.121569,0.466667,0.705882}%
\pgfsetfillcolor{currentfill}%
\pgfsetfillopacity{0.941201}%
\pgfsetlinewidth{1.003750pt}%
\definecolor{currentstroke}{rgb}{0.121569,0.466667,0.705882}%
\pgfsetstrokecolor{currentstroke}%
\pgfsetstrokeopacity{0.941201}%
\pgfsetdash{}{0pt}%
\pgfpathmoveto{\pgfqpoint{3.038951in}{2.604249in}}%
\pgfpathcurveto{\pgfqpoint{3.047187in}{2.604249in}}{\pgfqpoint{3.055087in}{2.607522in}}{\pgfqpoint{3.060911in}{2.613346in}}%
\pgfpathcurveto{\pgfqpoint{3.066735in}{2.619169in}}{\pgfqpoint{3.070008in}{2.627069in}}{\pgfqpoint{3.070008in}{2.635306in}}%
\pgfpathcurveto{\pgfqpoint{3.070008in}{2.643542in}}{\pgfqpoint{3.066735in}{2.651442in}}{\pgfqpoint{3.060911in}{2.657266in}}%
\pgfpathcurveto{\pgfqpoint{3.055087in}{2.663090in}}{\pgfqpoint{3.047187in}{2.666362in}}{\pgfqpoint{3.038951in}{2.666362in}}%
\pgfpathcurveto{\pgfqpoint{3.030715in}{2.666362in}}{\pgfqpoint{3.022815in}{2.663090in}}{\pgfqpoint{3.016991in}{2.657266in}}%
\pgfpathcurveto{\pgfqpoint{3.011167in}{2.651442in}}{\pgfqpoint{3.007895in}{2.643542in}}{\pgfqpoint{3.007895in}{2.635306in}}%
\pgfpathcurveto{\pgfqpoint{3.007895in}{2.627069in}}{\pgfqpoint{3.011167in}{2.619169in}}{\pgfqpoint{3.016991in}{2.613346in}}%
\pgfpathcurveto{\pgfqpoint{3.022815in}{2.607522in}}{\pgfqpoint{3.030715in}{2.604249in}}{\pgfqpoint{3.038951in}{2.604249in}}%
\pgfpathclose%
\pgfusepath{stroke,fill}%
\end{pgfscope}%
\begin{pgfscope}%
\pgfpathrectangle{\pgfqpoint{0.100000in}{0.212622in}}{\pgfqpoint{3.696000in}{3.696000in}}%
\pgfusepath{clip}%
\pgfsetbuttcap%
\pgfsetroundjoin%
\definecolor{currentfill}{rgb}{0.121569,0.466667,0.705882}%
\pgfsetfillcolor{currentfill}%
\pgfsetfillopacity{0.942051}%
\pgfsetlinewidth{1.003750pt}%
\definecolor{currentstroke}{rgb}{0.121569,0.466667,0.705882}%
\pgfsetstrokecolor{currentstroke}%
\pgfsetstrokeopacity{0.942051}%
\pgfsetdash{}{0pt}%
\pgfpathmoveto{\pgfqpoint{1.392071in}{1.849612in}}%
\pgfpathcurveto{\pgfqpoint{1.400307in}{1.849612in}}{\pgfqpoint{1.408207in}{1.852885in}}{\pgfqpoint{1.414031in}{1.858709in}}%
\pgfpathcurveto{\pgfqpoint{1.419855in}{1.864533in}}{\pgfqpoint{1.423127in}{1.872433in}}{\pgfqpoint{1.423127in}{1.880669in}}%
\pgfpathcurveto{\pgfqpoint{1.423127in}{1.888905in}}{\pgfqpoint{1.419855in}{1.896805in}}{\pgfqpoint{1.414031in}{1.902629in}}%
\pgfpathcurveto{\pgfqpoint{1.408207in}{1.908453in}}{\pgfqpoint{1.400307in}{1.911725in}}{\pgfqpoint{1.392071in}{1.911725in}}%
\pgfpathcurveto{\pgfqpoint{1.383835in}{1.911725in}}{\pgfqpoint{1.375935in}{1.908453in}}{\pgfqpoint{1.370111in}{1.902629in}}%
\pgfpathcurveto{\pgfqpoint{1.364287in}{1.896805in}}{\pgfqpoint{1.361014in}{1.888905in}}{\pgfqpoint{1.361014in}{1.880669in}}%
\pgfpathcurveto{\pgfqpoint{1.361014in}{1.872433in}}{\pgfqpoint{1.364287in}{1.864533in}}{\pgfqpoint{1.370111in}{1.858709in}}%
\pgfpathcurveto{\pgfqpoint{1.375935in}{1.852885in}}{\pgfqpoint{1.383835in}{1.849612in}}{\pgfqpoint{1.392071in}{1.849612in}}%
\pgfpathclose%
\pgfusepath{stroke,fill}%
\end{pgfscope}%
\begin{pgfscope}%
\pgfpathrectangle{\pgfqpoint{0.100000in}{0.212622in}}{\pgfqpoint{3.696000in}{3.696000in}}%
\pgfusepath{clip}%
\pgfsetbuttcap%
\pgfsetroundjoin%
\definecolor{currentfill}{rgb}{0.121569,0.466667,0.705882}%
\pgfsetfillcolor{currentfill}%
\pgfsetfillopacity{0.942684}%
\pgfsetlinewidth{1.003750pt}%
\definecolor{currentstroke}{rgb}{0.121569,0.466667,0.705882}%
\pgfsetstrokecolor{currentstroke}%
\pgfsetstrokeopacity{0.942684}%
\pgfsetdash{}{0pt}%
\pgfpathmoveto{\pgfqpoint{1.822039in}{2.078471in}}%
\pgfpathcurveto{\pgfqpoint{1.830275in}{2.078471in}}{\pgfqpoint{1.838175in}{2.081743in}}{\pgfqpoint{1.843999in}{2.087567in}}%
\pgfpathcurveto{\pgfqpoint{1.849823in}{2.093391in}}{\pgfqpoint{1.853095in}{2.101291in}}{\pgfqpoint{1.853095in}{2.109527in}}%
\pgfpathcurveto{\pgfqpoint{1.853095in}{2.117763in}}{\pgfqpoint{1.849823in}{2.125664in}}{\pgfqpoint{1.843999in}{2.131487in}}%
\pgfpathcurveto{\pgfqpoint{1.838175in}{2.137311in}}{\pgfqpoint{1.830275in}{2.140584in}}{\pgfqpoint{1.822039in}{2.140584in}}%
\pgfpathcurveto{\pgfqpoint{1.813802in}{2.140584in}}{\pgfqpoint{1.805902in}{2.137311in}}{\pgfqpoint{1.800078in}{2.131487in}}%
\pgfpathcurveto{\pgfqpoint{1.794255in}{2.125664in}}{\pgfqpoint{1.790982in}{2.117763in}}{\pgfqpoint{1.790982in}{2.109527in}}%
\pgfpathcurveto{\pgfqpoint{1.790982in}{2.101291in}}{\pgfqpoint{1.794255in}{2.093391in}}{\pgfqpoint{1.800078in}{2.087567in}}%
\pgfpathcurveto{\pgfqpoint{1.805902in}{2.081743in}}{\pgfqpoint{1.813802in}{2.078471in}}{\pgfqpoint{1.822039in}{2.078471in}}%
\pgfpathclose%
\pgfusepath{stroke,fill}%
\end{pgfscope}%
\begin{pgfscope}%
\pgfpathrectangle{\pgfqpoint{0.100000in}{0.212622in}}{\pgfqpoint{3.696000in}{3.696000in}}%
\pgfusepath{clip}%
\pgfsetbuttcap%
\pgfsetroundjoin%
\definecolor{currentfill}{rgb}{0.121569,0.466667,0.705882}%
\pgfsetfillcolor{currentfill}%
\pgfsetfillopacity{0.942935}%
\pgfsetlinewidth{1.003750pt}%
\definecolor{currentstroke}{rgb}{0.121569,0.466667,0.705882}%
\pgfsetstrokecolor{currentstroke}%
\pgfsetstrokeopacity{0.942935}%
\pgfsetdash{}{0pt}%
\pgfpathmoveto{\pgfqpoint{3.035310in}{2.601216in}}%
\pgfpathcurveto{\pgfqpoint{3.043546in}{2.601216in}}{\pgfqpoint{3.051446in}{2.604489in}}{\pgfqpoint{3.057270in}{2.610313in}}%
\pgfpathcurveto{\pgfqpoint{3.063094in}{2.616136in}}{\pgfqpoint{3.066367in}{2.624037in}}{\pgfqpoint{3.066367in}{2.632273in}}%
\pgfpathcurveto{\pgfqpoint{3.066367in}{2.640509in}}{\pgfqpoint{3.063094in}{2.648409in}}{\pgfqpoint{3.057270in}{2.654233in}}%
\pgfpathcurveto{\pgfqpoint{3.051446in}{2.660057in}}{\pgfqpoint{3.043546in}{2.663329in}}{\pgfqpoint{3.035310in}{2.663329in}}%
\pgfpathcurveto{\pgfqpoint{3.027074in}{2.663329in}}{\pgfqpoint{3.019174in}{2.660057in}}{\pgfqpoint{3.013350in}{2.654233in}}%
\pgfpathcurveto{\pgfqpoint{3.007526in}{2.648409in}}{\pgfqpoint{3.004254in}{2.640509in}}{\pgfqpoint{3.004254in}{2.632273in}}%
\pgfpathcurveto{\pgfqpoint{3.004254in}{2.624037in}}{\pgfqpoint{3.007526in}{2.616136in}}{\pgfqpoint{3.013350in}{2.610313in}}%
\pgfpathcurveto{\pgfqpoint{3.019174in}{2.604489in}}{\pgfqpoint{3.027074in}{2.601216in}}{\pgfqpoint{3.035310in}{2.601216in}}%
\pgfpathclose%
\pgfusepath{stroke,fill}%
\end{pgfscope}%
\begin{pgfscope}%
\pgfpathrectangle{\pgfqpoint{0.100000in}{0.212622in}}{\pgfqpoint{3.696000in}{3.696000in}}%
\pgfusepath{clip}%
\pgfsetbuttcap%
\pgfsetroundjoin%
\definecolor{currentfill}{rgb}{0.121569,0.466667,0.705882}%
\pgfsetfillcolor{currentfill}%
\pgfsetfillopacity{0.943375}%
\pgfsetlinewidth{1.003750pt}%
\definecolor{currentstroke}{rgb}{0.121569,0.466667,0.705882}%
\pgfsetstrokecolor{currentstroke}%
\pgfsetstrokeopacity{0.943375}%
\pgfsetdash{}{0pt}%
\pgfpathmoveto{\pgfqpoint{1.298430in}{1.798303in}}%
\pgfpathcurveto{\pgfqpoint{1.306666in}{1.798303in}}{\pgfqpoint{1.314566in}{1.801576in}}{\pgfqpoint{1.320390in}{1.807400in}}%
\pgfpathcurveto{\pgfqpoint{1.326214in}{1.813224in}}{\pgfqpoint{1.329486in}{1.821124in}}{\pgfqpoint{1.329486in}{1.829360in}}%
\pgfpathcurveto{\pgfqpoint{1.329486in}{1.837596in}}{\pgfqpoint{1.326214in}{1.845496in}}{\pgfqpoint{1.320390in}{1.851320in}}%
\pgfpathcurveto{\pgfqpoint{1.314566in}{1.857144in}}{\pgfqpoint{1.306666in}{1.860416in}}{\pgfqpoint{1.298430in}{1.860416in}}%
\pgfpathcurveto{\pgfqpoint{1.290193in}{1.860416in}}{\pgfqpoint{1.282293in}{1.857144in}}{\pgfqpoint{1.276469in}{1.851320in}}%
\pgfpathcurveto{\pgfqpoint{1.270646in}{1.845496in}}{\pgfqpoint{1.267373in}{1.837596in}}{\pgfqpoint{1.267373in}{1.829360in}}%
\pgfpathcurveto{\pgfqpoint{1.267373in}{1.821124in}}{\pgfqpoint{1.270646in}{1.813224in}}{\pgfqpoint{1.276469in}{1.807400in}}%
\pgfpathcurveto{\pgfqpoint{1.282293in}{1.801576in}}{\pgfqpoint{1.290193in}{1.798303in}}{\pgfqpoint{1.298430in}{1.798303in}}%
\pgfpathclose%
\pgfusepath{stroke,fill}%
\end{pgfscope}%
\begin{pgfscope}%
\pgfpathrectangle{\pgfqpoint{0.100000in}{0.212622in}}{\pgfqpoint{3.696000in}{3.696000in}}%
\pgfusepath{clip}%
\pgfsetbuttcap%
\pgfsetroundjoin%
\definecolor{currentfill}{rgb}{0.121569,0.466667,0.705882}%
\pgfsetfillcolor{currentfill}%
\pgfsetfillopacity{0.943676}%
\pgfsetlinewidth{1.003750pt}%
\definecolor{currentstroke}{rgb}{0.121569,0.466667,0.705882}%
\pgfsetstrokecolor{currentstroke}%
\pgfsetstrokeopacity{0.943676}%
\pgfsetdash{}{0pt}%
\pgfpathmoveto{\pgfqpoint{2.161705in}{2.193492in}}%
\pgfpathcurveto{\pgfqpoint{2.169941in}{2.193492in}}{\pgfqpoint{2.177841in}{2.196764in}}{\pgfqpoint{2.183665in}{2.202588in}}%
\pgfpathcurveto{\pgfqpoint{2.189489in}{2.208412in}}{\pgfqpoint{2.192762in}{2.216312in}}{\pgfqpoint{2.192762in}{2.224549in}}%
\pgfpathcurveto{\pgfqpoint{2.192762in}{2.232785in}}{\pgfqpoint{2.189489in}{2.240685in}}{\pgfqpoint{2.183665in}{2.246509in}}%
\pgfpathcurveto{\pgfqpoint{2.177841in}{2.252333in}}{\pgfqpoint{2.169941in}{2.255605in}}{\pgfqpoint{2.161705in}{2.255605in}}%
\pgfpathcurveto{\pgfqpoint{2.153469in}{2.255605in}}{\pgfqpoint{2.145569in}{2.252333in}}{\pgfqpoint{2.139745in}{2.246509in}}%
\pgfpathcurveto{\pgfqpoint{2.133921in}{2.240685in}}{\pgfqpoint{2.130649in}{2.232785in}}{\pgfqpoint{2.130649in}{2.224549in}}%
\pgfpathcurveto{\pgfqpoint{2.130649in}{2.216312in}}{\pgfqpoint{2.133921in}{2.208412in}}{\pgfqpoint{2.139745in}{2.202588in}}%
\pgfpathcurveto{\pgfqpoint{2.145569in}{2.196764in}}{\pgfqpoint{2.153469in}{2.193492in}}{\pgfqpoint{2.161705in}{2.193492in}}%
\pgfpathclose%
\pgfusepath{stroke,fill}%
\end{pgfscope}%
\begin{pgfscope}%
\pgfpathrectangle{\pgfqpoint{0.100000in}{0.212622in}}{\pgfqpoint{3.696000in}{3.696000in}}%
\pgfusepath{clip}%
\pgfsetbuttcap%
\pgfsetroundjoin%
\definecolor{currentfill}{rgb}{0.121569,0.466667,0.705882}%
\pgfsetfillcolor{currentfill}%
\pgfsetfillopacity{0.946091}%
\pgfsetlinewidth{1.003750pt}%
\definecolor{currentstroke}{rgb}{0.121569,0.466667,0.705882}%
\pgfsetstrokecolor{currentstroke}%
\pgfsetstrokeopacity{0.946091}%
\pgfsetdash{}{0pt}%
\pgfpathmoveto{\pgfqpoint{3.028685in}{2.595697in}}%
\pgfpathcurveto{\pgfqpoint{3.036921in}{2.595697in}}{\pgfqpoint{3.044821in}{2.598970in}}{\pgfqpoint{3.050645in}{2.604794in}}%
\pgfpathcurveto{\pgfqpoint{3.056469in}{2.610617in}}{\pgfqpoint{3.059741in}{2.618517in}}{\pgfqpoint{3.059741in}{2.626754in}}%
\pgfpathcurveto{\pgfqpoint{3.059741in}{2.634990in}}{\pgfqpoint{3.056469in}{2.642890in}}{\pgfqpoint{3.050645in}{2.648714in}}%
\pgfpathcurveto{\pgfqpoint{3.044821in}{2.654538in}}{\pgfqpoint{3.036921in}{2.657810in}}{\pgfqpoint{3.028685in}{2.657810in}}%
\pgfpathcurveto{\pgfqpoint{3.020449in}{2.657810in}}{\pgfqpoint{3.012549in}{2.654538in}}{\pgfqpoint{3.006725in}{2.648714in}}%
\pgfpathcurveto{\pgfqpoint{3.000901in}{2.642890in}}{\pgfqpoint{2.997628in}{2.634990in}}{\pgfqpoint{2.997628in}{2.626754in}}%
\pgfpathcurveto{\pgfqpoint{2.997628in}{2.618517in}}{\pgfqpoint{3.000901in}{2.610617in}}{\pgfqpoint{3.006725in}{2.604794in}}%
\pgfpathcurveto{\pgfqpoint{3.012549in}{2.598970in}}{\pgfqpoint{3.020449in}{2.595697in}}{\pgfqpoint{3.028685in}{2.595697in}}%
\pgfpathclose%
\pgfusepath{stroke,fill}%
\end{pgfscope}%
\begin{pgfscope}%
\pgfpathrectangle{\pgfqpoint{0.100000in}{0.212622in}}{\pgfqpoint{3.696000in}{3.696000in}}%
\pgfusepath{clip}%
\pgfsetbuttcap%
\pgfsetroundjoin%
\definecolor{currentfill}{rgb}{0.121569,0.466667,0.705882}%
\pgfsetfillcolor{currentfill}%
\pgfsetfillopacity{0.947066}%
\pgfsetlinewidth{1.003750pt}%
\definecolor{currentstroke}{rgb}{0.121569,0.466667,0.705882}%
\pgfsetstrokecolor{currentstroke}%
\pgfsetstrokeopacity{0.947066}%
\pgfsetdash{}{0pt}%
\pgfpathmoveto{\pgfqpoint{1.271955in}{1.763142in}}%
\pgfpathcurveto{\pgfqpoint{1.280191in}{1.763142in}}{\pgfqpoint{1.288091in}{1.766414in}}{\pgfqpoint{1.293915in}{1.772238in}}%
\pgfpathcurveto{\pgfqpoint{1.299739in}{1.778062in}}{\pgfqpoint{1.303011in}{1.785962in}}{\pgfqpoint{1.303011in}{1.794198in}}%
\pgfpathcurveto{\pgfqpoint{1.303011in}{1.802434in}}{\pgfqpoint{1.299739in}{1.810335in}}{\pgfqpoint{1.293915in}{1.816158in}}%
\pgfpathcurveto{\pgfqpoint{1.288091in}{1.821982in}}{\pgfqpoint{1.280191in}{1.825255in}}{\pgfqpoint{1.271955in}{1.825255in}}%
\pgfpathcurveto{\pgfqpoint{1.263718in}{1.825255in}}{\pgfqpoint{1.255818in}{1.821982in}}{\pgfqpoint{1.249994in}{1.816158in}}%
\pgfpathcurveto{\pgfqpoint{1.244170in}{1.810335in}}{\pgfqpoint{1.240898in}{1.802434in}}{\pgfqpoint{1.240898in}{1.794198in}}%
\pgfpathcurveto{\pgfqpoint{1.240898in}{1.785962in}}{\pgfqpoint{1.244170in}{1.778062in}}{\pgfqpoint{1.249994in}{1.772238in}}%
\pgfpathcurveto{\pgfqpoint{1.255818in}{1.766414in}}{\pgfqpoint{1.263718in}{1.763142in}}{\pgfqpoint{1.271955in}{1.763142in}}%
\pgfpathclose%
\pgfusepath{stroke,fill}%
\end{pgfscope}%
\begin{pgfscope}%
\pgfpathrectangle{\pgfqpoint{0.100000in}{0.212622in}}{\pgfqpoint{3.696000in}{3.696000in}}%
\pgfusepath{clip}%
\pgfsetbuttcap%
\pgfsetroundjoin%
\definecolor{currentfill}{rgb}{0.121569,0.466667,0.705882}%
\pgfsetfillcolor{currentfill}%
\pgfsetfillopacity{0.950057}%
\pgfsetlinewidth{1.003750pt}%
\definecolor{currentstroke}{rgb}{0.121569,0.466667,0.705882}%
\pgfsetstrokecolor{currentstroke}%
\pgfsetstrokeopacity{0.950057}%
\pgfsetdash{}{0pt}%
\pgfpathmoveto{\pgfqpoint{2.132230in}{2.174769in}}%
\pgfpathcurveto{\pgfqpoint{2.140467in}{2.174769in}}{\pgfqpoint{2.148367in}{2.178041in}}{\pgfqpoint{2.154191in}{2.183865in}}%
\pgfpathcurveto{\pgfqpoint{2.160015in}{2.189689in}}{\pgfqpoint{2.163287in}{2.197589in}}{\pgfqpoint{2.163287in}{2.205825in}}%
\pgfpathcurveto{\pgfqpoint{2.163287in}{2.214061in}}{\pgfqpoint{2.160015in}{2.221961in}}{\pgfqpoint{2.154191in}{2.227785in}}%
\pgfpathcurveto{\pgfqpoint{2.148367in}{2.233609in}}{\pgfqpoint{2.140467in}{2.236882in}}{\pgfqpoint{2.132230in}{2.236882in}}%
\pgfpathcurveto{\pgfqpoint{2.123994in}{2.236882in}}{\pgfqpoint{2.116094in}{2.233609in}}{\pgfqpoint{2.110270in}{2.227785in}}%
\pgfpathcurveto{\pgfqpoint{2.104446in}{2.221961in}}{\pgfqpoint{2.101174in}{2.214061in}}{\pgfqpoint{2.101174in}{2.205825in}}%
\pgfpathcurveto{\pgfqpoint{2.101174in}{2.197589in}}{\pgfqpoint{2.104446in}{2.189689in}}{\pgfqpoint{2.110270in}{2.183865in}}%
\pgfpathcurveto{\pgfqpoint{2.116094in}{2.178041in}}{\pgfqpoint{2.123994in}{2.174769in}}{\pgfqpoint{2.132230in}{2.174769in}}%
\pgfpathclose%
\pgfusepath{stroke,fill}%
\end{pgfscope}%
\begin{pgfscope}%
\pgfpathrectangle{\pgfqpoint{0.100000in}{0.212622in}}{\pgfqpoint{3.696000in}{3.696000in}}%
\pgfusepath{clip}%
\pgfsetbuttcap%
\pgfsetroundjoin%
\definecolor{currentfill}{rgb}{0.121569,0.466667,0.705882}%
\pgfsetfillcolor{currentfill}%
\pgfsetfillopacity{0.950214}%
\pgfsetlinewidth{1.003750pt}%
\definecolor{currentstroke}{rgb}{0.121569,0.466667,0.705882}%
\pgfsetstrokecolor{currentstroke}%
\pgfsetstrokeopacity{0.950214}%
\pgfsetdash{}{0pt}%
\pgfpathmoveto{\pgfqpoint{2.137580in}{2.174328in}}%
\pgfpathcurveto{\pgfqpoint{2.145817in}{2.174328in}}{\pgfqpoint{2.153717in}{2.177600in}}{\pgfqpoint{2.159541in}{2.183424in}}%
\pgfpathcurveto{\pgfqpoint{2.165365in}{2.189248in}}{\pgfqpoint{2.168637in}{2.197148in}}{\pgfqpoint{2.168637in}{2.205384in}}%
\pgfpathcurveto{\pgfqpoint{2.168637in}{2.213621in}}{\pgfqpoint{2.165365in}{2.221521in}}{\pgfqpoint{2.159541in}{2.227345in}}%
\pgfpathcurveto{\pgfqpoint{2.153717in}{2.233169in}}{\pgfqpoint{2.145817in}{2.236441in}}{\pgfqpoint{2.137580in}{2.236441in}}%
\pgfpathcurveto{\pgfqpoint{2.129344in}{2.236441in}}{\pgfqpoint{2.121444in}{2.233169in}}{\pgfqpoint{2.115620in}{2.227345in}}%
\pgfpathcurveto{\pgfqpoint{2.109796in}{2.221521in}}{\pgfqpoint{2.106524in}{2.213621in}}{\pgfqpoint{2.106524in}{2.205384in}}%
\pgfpathcurveto{\pgfqpoint{2.106524in}{2.197148in}}{\pgfqpoint{2.109796in}{2.189248in}}{\pgfqpoint{2.115620in}{2.183424in}}%
\pgfpathcurveto{\pgfqpoint{2.121444in}{2.177600in}}{\pgfqpoint{2.129344in}{2.174328in}}{\pgfqpoint{2.137580in}{2.174328in}}%
\pgfpathclose%
\pgfusepath{stroke,fill}%
\end{pgfscope}%
\begin{pgfscope}%
\pgfpathrectangle{\pgfqpoint{0.100000in}{0.212622in}}{\pgfqpoint{3.696000in}{3.696000in}}%
\pgfusepath{clip}%
\pgfsetbuttcap%
\pgfsetroundjoin%
\definecolor{currentfill}{rgb}{0.121569,0.466667,0.705882}%
\pgfsetfillcolor{currentfill}%
\pgfsetfillopacity{0.951297}%
\pgfsetlinewidth{1.003750pt}%
\definecolor{currentstroke}{rgb}{0.121569,0.466667,0.705882}%
\pgfsetstrokecolor{currentstroke}%
\pgfsetstrokeopacity{0.951297}%
\pgfsetdash{}{0pt}%
\pgfpathmoveto{\pgfqpoint{2.117663in}{2.163315in}}%
\pgfpathcurveto{\pgfqpoint{2.125899in}{2.163315in}}{\pgfqpoint{2.133799in}{2.166587in}}{\pgfqpoint{2.139623in}{2.172411in}}%
\pgfpathcurveto{\pgfqpoint{2.145447in}{2.178235in}}{\pgfqpoint{2.148719in}{2.186135in}}{\pgfqpoint{2.148719in}{2.194371in}}%
\pgfpathcurveto{\pgfqpoint{2.148719in}{2.202608in}}{\pgfqpoint{2.145447in}{2.210508in}}{\pgfqpoint{2.139623in}{2.216332in}}%
\pgfpathcurveto{\pgfqpoint{2.133799in}{2.222156in}}{\pgfqpoint{2.125899in}{2.225428in}}{\pgfqpoint{2.117663in}{2.225428in}}%
\pgfpathcurveto{\pgfqpoint{2.109426in}{2.225428in}}{\pgfqpoint{2.101526in}{2.222156in}}{\pgfqpoint{2.095702in}{2.216332in}}%
\pgfpathcurveto{\pgfqpoint{2.089878in}{2.210508in}}{\pgfqpoint{2.086606in}{2.202608in}}{\pgfqpoint{2.086606in}{2.194371in}}%
\pgfpathcurveto{\pgfqpoint{2.086606in}{2.186135in}}{\pgfqpoint{2.089878in}{2.178235in}}{\pgfqpoint{2.095702in}{2.172411in}}%
\pgfpathcurveto{\pgfqpoint{2.101526in}{2.166587in}}{\pgfqpoint{2.109426in}{2.163315in}}{\pgfqpoint{2.117663in}{2.163315in}}%
\pgfpathclose%
\pgfusepath{stroke,fill}%
\end{pgfscope}%
\begin{pgfscope}%
\pgfpathrectangle{\pgfqpoint{0.100000in}{0.212622in}}{\pgfqpoint{3.696000in}{3.696000in}}%
\pgfusepath{clip}%
\pgfsetbuttcap%
\pgfsetroundjoin%
\definecolor{currentfill}{rgb}{0.121569,0.466667,0.705882}%
\pgfsetfillcolor{currentfill}%
\pgfsetfillopacity{0.951351}%
\pgfsetlinewidth{1.003750pt}%
\definecolor{currentstroke}{rgb}{0.121569,0.466667,0.705882}%
\pgfsetstrokecolor{currentstroke}%
\pgfsetstrokeopacity{0.951351}%
\pgfsetdash{}{0pt}%
\pgfpathmoveto{\pgfqpoint{1.340174in}{1.830876in}}%
\pgfpathcurveto{\pgfqpoint{1.348410in}{1.830876in}}{\pgfqpoint{1.356310in}{1.834148in}}{\pgfqpoint{1.362134in}{1.839972in}}%
\pgfpathcurveto{\pgfqpoint{1.367958in}{1.845796in}}{\pgfqpoint{1.371230in}{1.853696in}}{\pgfqpoint{1.371230in}{1.861932in}}%
\pgfpathcurveto{\pgfqpoint{1.371230in}{1.870168in}}{\pgfqpoint{1.367958in}{1.878068in}}{\pgfqpoint{1.362134in}{1.883892in}}%
\pgfpathcurveto{\pgfqpoint{1.356310in}{1.889716in}}{\pgfqpoint{1.348410in}{1.892989in}}{\pgfqpoint{1.340174in}{1.892989in}}%
\pgfpathcurveto{\pgfqpoint{1.331938in}{1.892989in}}{\pgfqpoint{1.324037in}{1.889716in}}{\pgfqpoint{1.318214in}{1.883892in}}%
\pgfpathcurveto{\pgfqpoint{1.312390in}{1.878068in}}{\pgfqpoint{1.309117in}{1.870168in}}{\pgfqpoint{1.309117in}{1.861932in}}%
\pgfpathcurveto{\pgfqpoint{1.309117in}{1.853696in}}{\pgfqpoint{1.312390in}{1.845796in}}{\pgfqpoint{1.318214in}{1.839972in}}%
\pgfpathcurveto{\pgfqpoint{1.324037in}{1.834148in}}{\pgfqpoint{1.331938in}{1.830876in}}{\pgfqpoint{1.340174in}{1.830876in}}%
\pgfpathclose%
\pgfusepath{stroke,fill}%
\end{pgfscope}%
\begin{pgfscope}%
\pgfpathrectangle{\pgfqpoint{0.100000in}{0.212622in}}{\pgfqpoint{3.696000in}{3.696000in}}%
\pgfusepath{clip}%
\pgfsetbuttcap%
\pgfsetroundjoin%
\definecolor{currentfill}{rgb}{0.121569,0.466667,0.705882}%
\pgfsetfillcolor{currentfill}%
\pgfsetfillopacity{0.952139}%
\pgfsetlinewidth{1.003750pt}%
\definecolor{currentstroke}{rgb}{0.121569,0.466667,0.705882}%
\pgfsetstrokecolor{currentstroke}%
\pgfsetstrokeopacity{0.952139}%
\pgfsetdash{}{0pt}%
\pgfpathmoveto{\pgfqpoint{3.015367in}{2.582743in}}%
\pgfpathcurveto{\pgfqpoint{3.023603in}{2.582743in}}{\pgfqpoint{3.031503in}{2.586015in}}{\pgfqpoint{3.037327in}{2.591839in}}%
\pgfpathcurveto{\pgfqpoint{3.043151in}{2.597663in}}{\pgfqpoint{3.046423in}{2.605563in}}{\pgfqpoint{3.046423in}{2.613800in}}%
\pgfpathcurveto{\pgfqpoint{3.046423in}{2.622036in}}{\pgfqpoint{3.043151in}{2.629936in}}{\pgfqpoint{3.037327in}{2.635760in}}%
\pgfpathcurveto{\pgfqpoint{3.031503in}{2.641584in}}{\pgfqpoint{3.023603in}{2.644856in}}{\pgfqpoint{3.015367in}{2.644856in}}%
\pgfpathcurveto{\pgfqpoint{3.007130in}{2.644856in}}{\pgfqpoint{2.999230in}{2.641584in}}{\pgfqpoint{2.993406in}{2.635760in}}%
\pgfpathcurveto{\pgfqpoint{2.987582in}{2.629936in}}{\pgfqpoint{2.984310in}{2.622036in}}{\pgfqpoint{2.984310in}{2.613800in}}%
\pgfpathcurveto{\pgfqpoint{2.984310in}{2.605563in}}{\pgfqpoint{2.987582in}{2.597663in}}{\pgfqpoint{2.993406in}{2.591839in}}%
\pgfpathcurveto{\pgfqpoint{2.999230in}{2.586015in}}{\pgfqpoint{3.007130in}{2.582743in}}{\pgfqpoint{3.015367in}{2.582743in}}%
\pgfpathclose%
\pgfusepath{stroke,fill}%
\end{pgfscope}%
\begin{pgfscope}%
\pgfpathrectangle{\pgfqpoint{0.100000in}{0.212622in}}{\pgfqpoint{3.696000in}{3.696000in}}%
\pgfusepath{clip}%
\pgfsetbuttcap%
\pgfsetroundjoin%
\definecolor{currentfill}{rgb}{0.121569,0.466667,0.705882}%
\pgfsetfillcolor{currentfill}%
\pgfsetfillopacity{0.953525}%
\pgfsetlinewidth{1.003750pt}%
\definecolor{currentstroke}{rgb}{0.121569,0.466667,0.705882}%
\pgfsetstrokecolor{currentstroke}%
\pgfsetstrokeopacity{0.953525}%
\pgfsetdash{}{0pt}%
\pgfpathmoveto{\pgfqpoint{2.091390in}{2.158528in}}%
\pgfpathcurveto{\pgfqpoint{2.099626in}{2.158528in}}{\pgfqpoint{2.107526in}{2.161800in}}{\pgfqpoint{2.113350in}{2.167624in}}%
\pgfpathcurveto{\pgfqpoint{2.119174in}{2.173448in}}{\pgfqpoint{2.122447in}{2.181348in}}{\pgfqpoint{2.122447in}{2.189585in}}%
\pgfpathcurveto{\pgfqpoint{2.122447in}{2.197821in}}{\pgfqpoint{2.119174in}{2.205721in}}{\pgfqpoint{2.113350in}{2.211545in}}%
\pgfpathcurveto{\pgfqpoint{2.107526in}{2.217369in}}{\pgfqpoint{2.099626in}{2.220641in}}{\pgfqpoint{2.091390in}{2.220641in}}%
\pgfpathcurveto{\pgfqpoint{2.083154in}{2.220641in}}{\pgfqpoint{2.075254in}{2.217369in}}{\pgfqpoint{2.069430in}{2.211545in}}%
\pgfpathcurveto{\pgfqpoint{2.063606in}{2.205721in}}{\pgfqpoint{2.060334in}{2.197821in}}{\pgfqpoint{2.060334in}{2.189585in}}%
\pgfpathcurveto{\pgfqpoint{2.060334in}{2.181348in}}{\pgfqpoint{2.063606in}{2.173448in}}{\pgfqpoint{2.069430in}{2.167624in}}%
\pgfpathcurveto{\pgfqpoint{2.075254in}{2.161800in}}{\pgfqpoint{2.083154in}{2.158528in}}{\pgfqpoint{2.091390in}{2.158528in}}%
\pgfpathclose%
\pgfusepath{stroke,fill}%
\end{pgfscope}%
\begin{pgfscope}%
\pgfpathrectangle{\pgfqpoint{0.100000in}{0.212622in}}{\pgfqpoint{3.696000in}{3.696000in}}%
\pgfusepath{clip}%
\pgfsetbuttcap%
\pgfsetroundjoin%
\definecolor{currentfill}{rgb}{0.121569,0.466667,0.705882}%
\pgfsetfillcolor{currentfill}%
\pgfsetfillopacity{0.954707}%
\pgfsetlinewidth{1.003750pt}%
\definecolor{currentstroke}{rgb}{0.121569,0.466667,0.705882}%
\pgfsetstrokecolor{currentstroke}%
\pgfsetstrokeopacity{0.954707}%
\pgfsetdash{}{0pt}%
\pgfpathmoveto{\pgfqpoint{2.978573in}{2.538937in}}%
\pgfpathcurveto{\pgfqpoint{2.986809in}{2.538937in}}{\pgfqpoint{2.994709in}{2.542210in}}{\pgfqpoint{3.000533in}{2.548033in}}%
\pgfpathcurveto{\pgfqpoint{3.006357in}{2.553857in}}{\pgfqpoint{3.009629in}{2.561757in}}{\pgfqpoint{3.009629in}{2.569994in}}%
\pgfpathcurveto{\pgfqpoint{3.009629in}{2.578230in}}{\pgfqpoint{3.006357in}{2.586130in}}{\pgfqpoint{3.000533in}{2.591954in}}%
\pgfpathcurveto{\pgfqpoint{2.994709in}{2.597778in}}{\pgfqpoint{2.986809in}{2.601050in}}{\pgfqpoint{2.978573in}{2.601050in}}%
\pgfpathcurveto{\pgfqpoint{2.970336in}{2.601050in}}{\pgfqpoint{2.962436in}{2.597778in}}{\pgfqpoint{2.956612in}{2.591954in}}%
\pgfpathcurveto{\pgfqpoint{2.950788in}{2.586130in}}{\pgfqpoint{2.947516in}{2.578230in}}{\pgfqpoint{2.947516in}{2.569994in}}%
\pgfpathcurveto{\pgfqpoint{2.947516in}{2.561757in}}{\pgfqpoint{2.950788in}{2.553857in}}{\pgfqpoint{2.956612in}{2.548033in}}%
\pgfpathcurveto{\pgfqpoint{2.962436in}{2.542210in}}{\pgfqpoint{2.970336in}{2.538937in}}{\pgfqpoint{2.978573in}{2.538937in}}%
\pgfpathclose%
\pgfusepath{stroke,fill}%
\end{pgfscope}%
\begin{pgfscope}%
\pgfpathrectangle{\pgfqpoint{0.100000in}{0.212622in}}{\pgfqpoint{3.696000in}{3.696000in}}%
\pgfusepath{clip}%
\pgfsetbuttcap%
\pgfsetroundjoin%
\definecolor{currentfill}{rgb}{0.121569,0.466667,0.705882}%
\pgfsetfillcolor{currentfill}%
\pgfsetfillopacity{0.954770}%
\pgfsetlinewidth{1.003750pt}%
\definecolor{currentstroke}{rgb}{0.121569,0.466667,0.705882}%
\pgfsetstrokecolor{currentstroke}%
\pgfsetstrokeopacity{0.954770}%
\pgfsetdash{}{0pt}%
\pgfpathmoveto{\pgfqpoint{2.006429in}{2.124414in}}%
\pgfpathcurveto{\pgfqpoint{2.014665in}{2.124414in}}{\pgfqpoint{2.022565in}{2.127686in}}{\pgfqpoint{2.028389in}{2.133510in}}%
\pgfpathcurveto{\pgfqpoint{2.034213in}{2.139334in}}{\pgfqpoint{2.037485in}{2.147234in}}{\pgfqpoint{2.037485in}{2.155470in}}%
\pgfpathcurveto{\pgfqpoint{2.037485in}{2.163706in}}{\pgfqpoint{2.034213in}{2.171606in}}{\pgfqpoint{2.028389in}{2.177430in}}%
\pgfpathcurveto{\pgfqpoint{2.022565in}{2.183254in}}{\pgfqpoint{2.014665in}{2.186527in}}{\pgfqpoint{2.006429in}{2.186527in}}%
\pgfpathcurveto{\pgfqpoint{1.998192in}{2.186527in}}{\pgfqpoint{1.990292in}{2.183254in}}{\pgfqpoint{1.984468in}{2.177430in}}%
\pgfpathcurveto{\pgfqpoint{1.978644in}{2.171606in}}{\pgfqpoint{1.975372in}{2.163706in}}{\pgfqpoint{1.975372in}{2.155470in}}%
\pgfpathcurveto{\pgfqpoint{1.975372in}{2.147234in}}{\pgfqpoint{1.978644in}{2.139334in}}{\pgfqpoint{1.984468in}{2.133510in}}%
\pgfpathcurveto{\pgfqpoint{1.990292in}{2.127686in}}{\pgfqpoint{1.998192in}{2.124414in}}{\pgfqpoint{2.006429in}{2.124414in}}%
\pgfpathclose%
\pgfusepath{stroke,fill}%
\end{pgfscope}%
\begin{pgfscope}%
\pgfpathrectangle{\pgfqpoint{0.100000in}{0.212622in}}{\pgfqpoint{3.696000in}{3.696000in}}%
\pgfusepath{clip}%
\pgfsetbuttcap%
\pgfsetroundjoin%
\definecolor{currentfill}{rgb}{0.121569,0.466667,0.705882}%
\pgfsetfillcolor{currentfill}%
\pgfsetfillopacity{0.955381}%
\pgfsetlinewidth{1.003750pt}%
\definecolor{currentstroke}{rgb}{0.121569,0.466667,0.705882}%
\pgfsetstrokecolor{currentstroke}%
\pgfsetstrokeopacity{0.955381}%
\pgfsetdash{}{0pt}%
\pgfpathmoveto{\pgfqpoint{2.113632in}{2.158088in}}%
\pgfpathcurveto{\pgfqpoint{2.121868in}{2.158088in}}{\pgfqpoint{2.129768in}{2.161360in}}{\pgfqpoint{2.135592in}{2.167184in}}%
\pgfpathcurveto{\pgfqpoint{2.141416in}{2.173008in}}{\pgfqpoint{2.144689in}{2.180908in}}{\pgfqpoint{2.144689in}{2.189144in}}%
\pgfpathcurveto{\pgfqpoint{2.144689in}{2.197381in}}{\pgfqpoint{2.141416in}{2.205281in}}{\pgfqpoint{2.135592in}{2.211105in}}%
\pgfpathcurveto{\pgfqpoint{2.129768in}{2.216929in}}{\pgfqpoint{2.121868in}{2.220201in}}{\pgfqpoint{2.113632in}{2.220201in}}%
\pgfpathcurveto{\pgfqpoint{2.105396in}{2.220201in}}{\pgfqpoint{2.097496in}{2.216929in}}{\pgfqpoint{2.091672in}{2.211105in}}%
\pgfpathcurveto{\pgfqpoint{2.085848in}{2.205281in}}{\pgfqpoint{2.082576in}{2.197381in}}{\pgfqpoint{2.082576in}{2.189144in}}%
\pgfpathcurveto{\pgfqpoint{2.082576in}{2.180908in}}{\pgfqpoint{2.085848in}{2.173008in}}{\pgfqpoint{2.091672in}{2.167184in}}%
\pgfpathcurveto{\pgfqpoint{2.097496in}{2.161360in}}{\pgfqpoint{2.105396in}{2.158088in}}{\pgfqpoint{2.113632in}{2.158088in}}%
\pgfpathclose%
\pgfusepath{stroke,fill}%
\end{pgfscope}%
\begin{pgfscope}%
\pgfpathrectangle{\pgfqpoint{0.100000in}{0.212622in}}{\pgfqpoint{3.696000in}{3.696000in}}%
\pgfusepath{clip}%
\pgfsetbuttcap%
\pgfsetroundjoin%
\definecolor{currentfill}{rgb}{0.121569,0.466667,0.705882}%
\pgfsetfillcolor{currentfill}%
\pgfsetfillopacity{0.956690}%
\pgfsetlinewidth{1.003750pt}%
\definecolor{currentstroke}{rgb}{0.121569,0.466667,0.705882}%
\pgfsetstrokecolor{currentstroke}%
\pgfsetstrokeopacity{0.956690}%
\pgfsetdash{}{0pt}%
\pgfpathmoveto{\pgfqpoint{1.803378in}{2.054162in}}%
\pgfpathcurveto{\pgfqpoint{1.811615in}{2.054162in}}{\pgfqpoint{1.819515in}{2.057434in}}{\pgfqpoint{1.825339in}{2.063258in}}%
\pgfpathcurveto{\pgfqpoint{1.831162in}{2.069082in}}{\pgfqpoint{1.834435in}{2.076982in}}{\pgfqpoint{1.834435in}{2.085218in}}%
\pgfpathcurveto{\pgfqpoint{1.834435in}{2.093455in}}{\pgfqpoint{1.831162in}{2.101355in}}{\pgfqpoint{1.825339in}{2.107178in}}%
\pgfpathcurveto{\pgfqpoint{1.819515in}{2.113002in}}{\pgfqpoint{1.811615in}{2.116275in}}{\pgfqpoint{1.803378in}{2.116275in}}%
\pgfpathcurveto{\pgfqpoint{1.795142in}{2.116275in}}{\pgfqpoint{1.787242in}{2.113002in}}{\pgfqpoint{1.781418in}{2.107178in}}%
\pgfpathcurveto{\pgfqpoint{1.775594in}{2.101355in}}{\pgfqpoint{1.772322in}{2.093455in}}{\pgfqpoint{1.772322in}{2.085218in}}%
\pgfpathcurveto{\pgfqpoint{1.772322in}{2.076982in}}{\pgfqpoint{1.775594in}{2.069082in}}{\pgfqpoint{1.781418in}{2.063258in}}%
\pgfpathcurveto{\pgfqpoint{1.787242in}{2.057434in}}{\pgfqpoint{1.795142in}{2.054162in}}{\pgfqpoint{1.803378in}{2.054162in}}%
\pgfpathclose%
\pgfusepath{stroke,fill}%
\end{pgfscope}%
\begin{pgfscope}%
\pgfpathrectangle{\pgfqpoint{0.100000in}{0.212622in}}{\pgfqpoint{3.696000in}{3.696000in}}%
\pgfusepath{clip}%
\pgfsetbuttcap%
\pgfsetroundjoin%
\definecolor{currentfill}{rgb}{0.121569,0.466667,0.705882}%
\pgfsetfillcolor{currentfill}%
\pgfsetfillopacity{0.958427}%
\pgfsetlinewidth{1.003750pt}%
\definecolor{currentstroke}{rgb}{0.121569,0.466667,0.705882}%
\pgfsetstrokecolor{currentstroke}%
\pgfsetstrokeopacity{0.958427}%
\pgfsetdash{}{0pt}%
\pgfpathmoveto{\pgfqpoint{3.000322in}{2.567157in}}%
\pgfpathcurveto{\pgfqpoint{3.008559in}{2.567157in}}{\pgfqpoint{3.016459in}{2.570429in}}{\pgfqpoint{3.022283in}{2.576253in}}%
\pgfpathcurveto{\pgfqpoint{3.028106in}{2.582077in}}{\pgfqpoint{3.031379in}{2.589977in}}{\pgfqpoint{3.031379in}{2.598214in}}%
\pgfpathcurveto{\pgfqpoint{3.031379in}{2.606450in}}{\pgfqpoint{3.028106in}{2.614350in}}{\pgfqpoint{3.022283in}{2.620174in}}%
\pgfpathcurveto{\pgfqpoint{3.016459in}{2.625998in}}{\pgfqpoint{3.008559in}{2.629270in}}{\pgfqpoint{3.000322in}{2.629270in}}%
\pgfpathcurveto{\pgfqpoint{2.992086in}{2.629270in}}{\pgfqpoint{2.984186in}{2.625998in}}{\pgfqpoint{2.978362in}{2.620174in}}%
\pgfpathcurveto{\pgfqpoint{2.972538in}{2.614350in}}{\pgfqpoint{2.969266in}{2.606450in}}{\pgfqpoint{2.969266in}{2.598214in}}%
\pgfpathcurveto{\pgfqpoint{2.969266in}{2.589977in}}{\pgfqpoint{2.972538in}{2.582077in}}{\pgfqpoint{2.978362in}{2.576253in}}%
\pgfpathcurveto{\pgfqpoint{2.984186in}{2.570429in}}{\pgfqpoint{2.992086in}{2.567157in}}{\pgfqpoint{3.000322in}{2.567157in}}%
\pgfpathclose%
\pgfusepath{stroke,fill}%
\end{pgfscope}%
\begin{pgfscope}%
\pgfpathrectangle{\pgfqpoint{0.100000in}{0.212622in}}{\pgfqpoint{3.696000in}{3.696000in}}%
\pgfusepath{clip}%
\pgfsetbuttcap%
\pgfsetroundjoin%
\definecolor{currentfill}{rgb}{0.121569,0.466667,0.705882}%
\pgfsetfillcolor{currentfill}%
\pgfsetfillopacity{0.958953}%
\pgfsetlinewidth{1.003750pt}%
\definecolor{currentstroke}{rgb}{0.121569,0.466667,0.705882}%
\pgfsetstrokecolor{currentstroke}%
\pgfsetstrokeopacity{0.958953}%
\pgfsetdash{}{0pt}%
\pgfpathmoveto{\pgfqpoint{2.011735in}{2.126330in}}%
\pgfpathcurveto{\pgfqpoint{2.019971in}{2.126330in}}{\pgfqpoint{2.027871in}{2.129603in}}{\pgfqpoint{2.033695in}{2.135427in}}%
\pgfpathcurveto{\pgfqpoint{2.039519in}{2.141251in}}{\pgfqpoint{2.042791in}{2.149151in}}{\pgfqpoint{2.042791in}{2.157387in}}%
\pgfpathcurveto{\pgfqpoint{2.042791in}{2.165623in}}{\pgfqpoint{2.039519in}{2.173523in}}{\pgfqpoint{2.033695in}{2.179347in}}%
\pgfpathcurveto{\pgfqpoint{2.027871in}{2.185171in}}{\pgfqpoint{2.019971in}{2.188443in}}{\pgfqpoint{2.011735in}{2.188443in}}%
\pgfpathcurveto{\pgfqpoint{2.003499in}{2.188443in}}{\pgfqpoint{1.995598in}{2.185171in}}{\pgfqpoint{1.989775in}{2.179347in}}%
\pgfpathcurveto{\pgfqpoint{1.983951in}{2.173523in}}{\pgfqpoint{1.980678in}{2.165623in}}{\pgfqpoint{1.980678in}{2.157387in}}%
\pgfpathcurveto{\pgfqpoint{1.980678in}{2.149151in}}{\pgfqpoint{1.983951in}{2.141251in}}{\pgfqpoint{1.989775in}{2.135427in}}%
\pgfpathcurveto{\pgfqpoint{1.995598in}{2.129603in}}{\pgfqpoint{2.003499in}{2.126330in}}{\pgfqpoint{2.011735in}{2.126330in}}%
\pgfpathclose%
\pgfusepath{stroke,fill}%
\end{pgfscope}%
\begin{pgfscope}%
\pgfpathrectangle{\pgfqpoint{0.100000in}{0.212622in}}{\pgfqpoint{3.696000in}{3.696000in}}%
\pgfusepath{clip}%
\pgfsetbuttcap%
\pgfsetroundjoin%
\definecolor{currentfill}{rgb}{0.121569,0.466667,0.705882}%
\pgfsetfillcolor{currentfill}%
\pgfsetfillopacity{0.959271}%
\pgfsetlinewidth{1.003750pt}%
\definecolor{currentstroke}{rgb}{0.121569,0.466667,0.705882}%
\pgfsetstrokecolor{currentstroke}%
\pgfsetstrokeopacity{0.959271}%
\pgfsetdash{}{0pt}%
\pgfpathmoveto{\pgfqpoint{1.817590in}{2.073812in}}%
\pgfpathcurveto{\pgfqpoint{1.825826in}{2.073812in}}{\pgfqpoint{1.833726in}{2.077085in}}{\pgfqpoint{1.839550in}{2.082909in}}%
\pgfpathcurveto{\pgfqpoint{1.845374in}{2.088733in}}{\pgfqpoint{1.848646in}{2.096633in}}{\pgfqpoint{1.848646in}{2.104869in}}%
\pgfpathcurveto{\pgfqpoint{1.848646in}{2.113105in}}{\pgfqpoint{1.845374in}{2.121005in}}{\pgfqpoint{1.839550in}{2.126829in}}%
\pgfpathcurveto{\pgfqpoint{1.833726in}{2.132653in}}{\pgfqpoint{1.825826in}{2.135925in}}{\pgfqpoint{1.817590in}{2.135925in}}%
\pgfpathcurveto{\pgfqpoint{1.809354in}{2.135925in}}{\pgfqpoint{1.801454in}{2.132653in}}{\pgfqpoint{1.795630in}{2.126829in}}%
\pgfpathcurveto{\pgfqpoint{1.789806in}{2.121005in}}{\pgfqpoint{1.786533in}{2.113105in}}{\pgfqpoint{1.786533in}{2.104869in}}%
\pgfpathcurveto{\pgfqpoint{1.786533in}{2.096633in}}{\pgfqpoint{1.789806in}{2.088733in}}{\pgfqpoint{1.795630in}{2.082909in}}%
\pgfpathcurveto{\pgfqpoint{1.801454in}{2.077085in}}{\pgfqpoint{1.809354in}{2.073812in}}{\pgfqpoint{1.817590in}{2.073812in}}%
\pgfpathclose%
\pgfusepath{stroke,fill}%
\end{pgfscope}%
\begin{pgfscope}%
\pgfpathrectangle{\pgfqpoint{0.100000in}{0.212622in}}{\pgfqpoint{3.696000in}{3.696000in}}%
\pgfusepath{clip}%
\pgfsetbuttcap%
\pgfsetroundjoin%
\definecolor{currentfill}{rgb}{0.121569,0.466667,0.705882}%
\pgfsetfillcolor{currentfill}%
\pgfsetfillopacity{0.959447}%
\pgfsetlinewidth{1.003750pt}%
\definecolor{currentstroke}{rgb}{0.121569,0.466667,0.705882}%
\pgfsetstrokecolor{currentstroke}%
\pgfsetstrokeopacity{0.959447}%
\pgfsetdash{}{0pt}%
\pgfpathmoveto{\pgfqpoint{1.983534in}{2.107286in}}%
\pgfpathcurveto{\pgfqpoint{1.991770in}{2.107286in}}{\pgfqpoint{1.999670in}{2.110558in}}{\pgfqpoint{2.005494in}{2.116382in}}%
\pgfpathcurveto{\pgfqpoint{2.011318in}{2.122206in}}{\pgfqpoint{2.014590in}{2.130106in}}{\pgfqpoint{2.014590in}{2.138342in}}%
\pgfpathcurveto{\pgfqpoint{2.014590in}{2.146578in}}{\pgfqpoint{2.011318in}{2.154478in}}{\pgfqpoint{2.005494in}{2.160302in}}%
\pgfpathcurveto{\pgfqpoint{1.999670in}{2.166126in}}{\pgfqpoint{1.991770in}{2.169399in}}{\pgfqpoint{1.983534in}{2.169399in}}%
\pgfpathcurveto{\pgfqpoint{1.975298in}{2.169399in}}{\pgfqpoint{1.967398in}{2.166126in}}{\pgfqpoint{1.961574in}{2.160302in}}%
\pgfpathcurveto{\pgfqpoint{1.955750in}{2.154478in}}{\pgfqpoint{1.952477in}{2.146578in}}{\pgfqpoint{1.952477in}{2.138342in}}%
\pgfpathcurveto{\pgfqpoint{1.952477in}{2.130106in}}{\pgfqpoint{1.955750in}{2.122206in}}{\pgfqpoint{1.961574in}{2.116382in}}%
\pgfpathcurveto{\pgfqpoint{1.967398in}{2.110558in}}{\pgfqpoint{1.975298in}{2.107286in}}{\pgfqpoint{1.983534in}{2.107286in}}%
\pgfpathclose%
\pgfusepath{stroke,fill}%
\end{pgfscope}%
\begin{pgfscope}%
\pgfpathrectangle{\pgfqpoint{0.100000in}{0.212622in}}{\pgfqpoint{3.696000in}{3.696000in}}%
\pgfusepath{clip}%
\pgfsetbuttcap%
\pgfsetroundjoin%
\definecolor{currentfill}{rgb}{0.121569,0.466667,0.705882}%
\pgfsetfillcolor{currentfill}%
\pgfsetfillopacity{0.959587}%
\pgfsetlinewidth{1.003750pt}%
\definecolor{currentstroke}{rgb}{0.121569,0.466667,0.705882}%
\pgfsetstrokecolor{currentstroke}%
\pgfsetstrokeopacity{0.959587}%
\pgfsetdash{}{0pt}%
\pgfpathmoveto{\pgfqpoint{1.273918in}{1.775146in}}%
\pgfpathcurveto{\pgfqpoint{1.282154in}{1.775146in}}{\pgfqpoint{1.290054in}{1.778419in}}{\pgfqpoint{1.295878in}{1.784243in}}%
\pgfpathcurveto{\pgfqpoint{1.301702in}{1.790066in}}{\pgfqpoint{1.304974in}{1.797967in}}{\pgfqpoint{1.304974in}{1.806203in}}%
\pgfpathcurveto{\pgfqpoint{1.304974in}{1.814439in}}{\pgfqpoint{1.301702in}{1.822339in}}{\pgfqpoint{1.295878in}{1.828163in}}%
\pgfpathcurveto{\pgfqpoint{1.290054in}{1.833987in}}{\pgfqpoint{1.282154in}{1.837259in}}{\pgfqpoint{1.273918in}{1.837259in}}%
\pgfpathcurveto{\pgfqpoint{1.265681in}{1.837259in}}{\pgfqpoint{1.257781in}{1.833987in}}{\pgfqpoint{1.251957in}{1.828163in}}%
\pgfpathcurveto{\pgfqpoint{1.246134in}{1.822339in}}{\pgfqpoint{1.242861in}{1.814439in}}{\pgfqpoint{1.242861in}{1.806203in}}%
\pgfpathcurveto{\pgfqpoint{1.242861in}{1.797967in}}{\pgfqpoint{1.246134in}{1.790066in}}{\pgfqpoint{1.251957in}{1.784243in}}%
\pgfpathcurveto{\pgfqpoint{1.257781in}{1.778419in}}{\pgfqpoint{1.265681in}{1.775146in}}{\pgfqpoint{1.273918in}{1.775146in}}%
\pgfpathclose%
\pgfusepath{stroke,fill}%
\end{pgfscope}%
\begin{pgfscope}%
\pgfpathrectangle{\pgfqpoint{0.100000in}{0.212622in}}{\pgfqpoint{3.696000in}{3.696000in}}%
\pgfusepath{clip}%
\pgfsetbuttcap%
\pgfsetroundjoin%
\definecolor{currentfill}{rgb}{0.121569,0.466667,0.705882}%
\pgfsetfillcolor{currentfill}%
\pgfsetfillopacity{0.961950}%
\pgfsetlinewidth{1.003750pt}%
\definecolor{currentstroke}{rgb}{0.121569,0.466667,0.705882}%
\pgfsetstrokecolor{currentstroke}%
\pgfsetstrokeopacity{0.961950}%
\pgfsetdash{}{0pt}%
\pgfpathmoveto{\pgfqpoint{1.828078in}{2.072021in}}%
\pgfpathcurveto{\pgfqpoint{1.836314in}{2.072021in}}{\pgfqpoint{1.844214in}{2.075293in}}{\pgfqpoint{1.850038in}{2.081117in}}%
\pgfpathcurveto{\pgfqpoint{1.855862in}{2.086941in}}{\pgfqpoint{1.859134in}{2.094841in}}{\pgfqpoint{1.859134in}{2.103077in}}%
\pgfpathcurveto{\pgfqpoint{1.859134in}{2.111314in}}{\pgfqpoint{1.855862in}{2.119214in}}{\pgfqpoint{1.850038in}{2.125038in}}%
\pgfpathcurveto{\pgfqpoint{1.844214in}{2.130861in}}{\pgfqpoint{1.836314in}{2.134134in}}{\pgfqpoint{1.828078in}{2.134134in}}%
\pgfpathcurveto{\pgfqpoint{1.819841in}{2.134134in}}{\pgfqpoint{1.811941in}{2.130861in}}{\pgfqpoint{1.806117in}{2.125038in}}%
\pgfpathcurveto{\pgfqpoint{1.800293in}{2.119214in}}{\pgfqpoint{1.797021in}{2.111314in}}{\pgfqpoint{1.797021in}{2.103077in}}%
\pgfpathcurveto{\pgfqpoint{1.797021in}{2.094841in}}{\pgfqpoint{1.800293in}{2.086941in}}{\pgfqpoint{1.806117in}{2.081117in}}%
\pgfpathcurveto{\pgfqpoint{1.811941in}{2.075293in}}{\pgfqpoint{1.819841in}{2.072021in}}{\pgfqpoint{1.828078in}{2.072021in}}%
\pgfpathclose%
\pgfusepath{stroke,fill}%
\end{pgfscope}%
\begin{pgfscope}%
\pgfpathrectangle{\pgfqpoint{0.100000in}{0.212622in}}{\pgfqpoint{3.696000in}{3.696000in}}%
\pgfusepath{clip}%
\pgfsetbuttcap%
\pgfsetroundjoin%
\definecolor{currentfill}{rgb}{0.121569,0.466667,0.705882}%
\pgfsetfillcolor{currentfill}%
\pgfsetfillopacity{0.962683}%
\pgfsetlinewidth{1.003750pt}%
\definecolor{currentstroke}{rgb}{0.121569,0.466667,0.705882}%
\pgfsetstrokecolor{currentstroke}%
\pgfsetstrokeopacity{0.962683}%
\pgfsetdash{}{0pt}%
\pgfpathmoveto{\pgfqpoint{2.027606in}{2.135630in}}%
\pgfpathcurveto{\pgfqpoint{2.035842in}{2.135630in}}{\pgfqpoint{2.043742in}{2.138902in}}{\pgfqpoint{2.049566in}{2.144726in}}%
\pgfpathcurveto{\pgfqpoint{2.055390in}{2.150550in}}{\pgfqpoint{2.058662in}{2.158450in}}{\pgfqpoint{2.058662in}{2.166687in}}%
\pgfpathcurveto{\pgfqpoint{2.058662in}{2.174923in}}{\pgfqpoint{2.055390in}{2.182823in}}{\pgfqpoint{2.049566in}{2.188647in}}%
\pgfpathcurveto{\pgfqpoint{2.043742in}{2.194471in}}{\pgfqpoint{2.035842in}{2.197743in}}{\pgfqpoint{2.027606in}{2.197743in}}%
\pgfpathcurveto{\pgfqpoint{2.019369in}{2.197743in}}{\pgfqpoint{2.011469in}{2.194471in}}{\pgfqpoint{2.005645in}{2.188647in}}%
\pgfpathcurveto{\pgfqpoint{1.999821in}{2.182823in}}{\pgfqpoint{1.996549in}{2.174923in}}{\pgfqpoint{1.996549in}{2.166687in}}%
\pgfpathcurveto{\pgfqpoint{1.996549in}{2.158450in}}{\pgfqpoint{1.999821in}{2.150550in}}{\pgfqpoint{2.005645in}{2.144726in}}%
\pgfpathcurveto{\pgfqpoint{2.011469in}{2.138902in}}{\pgfqpoint{2.019369in}{2.135630in}}{\pgfqpoint{2.027606in}{2.135630in}}%
\pgfpathclose%
\pgfusepath{stroke,fill}%
\end{pgfscope}%
\begin{pgfscope}%
\pgfpathrectangle{\pgfqpoint{0.100000in}{0.212622in}}{\pgfqpoint{3.696000in}{3.696000in}}%
\pgfusepath{clip}%
\pgfsetbuttcap%
\pgfsetroundjoin%
\definecolor{currentfill}{rgb}{0.121569,0.466667,0.705882}%
\pgfsetfillcolor{currentfill}%
\pgfsetfillopacity{0.962786}%
\pgfsetlinewidth{1.003750pt}%
\definecolor{currentstroke}{rgb}{0.121569,0.466667,0.705882}%
\pgfsetstrokecolor{currentstroke}%
\pgfsetstrokeopacity{0.962786}%
\pgfsetdash{}{0pt}%
\pgfpathmoveto{\pgfqpoint{2.990520in}{2.557495in}}%
\pgfpathcurveto{\pgfqpoint{2.998756in}{2.557495in}}{\pgfqpoint{3.006656in}{2.560767in}}{\pgfqpoint{3.012480in}{2.566591in}}%
\pgfpathcurveto{\pgfqpoint{3.018304in}{2.572415in}}{\pgfqpoint{3.021576in}{2.580315in}}{\pgfqpoint{3.021576in}{2.588552in}}%
\pgfpathcurveto{\pgfqpoint{3.021576in}{2.596788in}}{\pgfqpoint{3.018304in}{2.604688in}}{\pgfqpoint{3.012480in}{2.610512in}}%
\pgfpathcurveto{\pgfqpoint{3.006656in}{2.616336in}}{\pgfqpoint{2.998756in}{2.619608in}}{\pgfqpoint{2.990520in}{2.619608in}}%
\pgfpathcurveto{\pgfqpoint{2.982283in}{2.619608in}}{\pgfqpoint{2.974383in}{2.616336in}}{\pgfqpoint{2.968559in}{2.610512in}}%
\pgfpathcurveto{\pgfqpoint{2.962735in}{2.604688in}}{\pgfqpoint{2.959463in}{2.596788in}}{\pgfqpoint{2.959463in}{2.588552in}}%
\pgfpathcurveto{\pgfqpoint{2.959463in}{2.580315in}}{\pgfqpoint{2.962735in}{2.572415in}}{\pgfqpoint{2.968559in}{2.566591in}}%
\pgfpathcurveto{\pgfqpoint{2.974383in}{2.560767in}}{\pgfqpoint{2.982283in}{2.557495in}}{\pgfqpoint{2.990520in}{2.557495in}}%
\pgfpathclose%
\pgfusepath{stroke,fill}%
\end{pgfscope}%
\begin{pgfscope}%
\pgfpathrectangle{\pgfqpoint{0.100000in}{0.212622in}}{\pgfqpoint{3.696000in}{3.696000in}}%
\pgfusepath{clip}%
\pgfsetbuttcap%
\pgfsetroundjoin%
\definecolor{currentfill}{rgb}{0.121569,0.466667,0.705882}%
\pgfsetfillcolor{currentfill}%
\pgfsetfillopacity{0.963494}%
\pgfsetlinewidth{1.003750pt}%
\definecolor{currentstroke}{rgb}{0.121569,0.466667,0.705882}%
\pgfsetstrokecolor{currentstroke}%
\pgfsetstrokeopacity{0.963494}%
\pgfsetdash{}{0pt}%
\pgfpathmoveto{\pgfqpoint{2.076506in}{2.115887in}}%
\pgfpathcurveto{\pgfqpoint{2.084742in}{2.115887in}}{\pgfqpoint{2.092642in}{2.119160in}}{\pgfqpoint{2.098466in}{2.124984in}}%
\pgfpathcurveto{\pgfqpoint{2.104290in}{2.130808in}}{\pgfqpoint{2.107562in}{2.138708in}}{\pgfqpoint{2.107562in}{2.146944in}}%
\pgfpathcurveto{\pgfqpoint{2.107562in}{2.155180in}}{\pgfqpoint{2.104290in}{2.163080in}}{\pgfqpoint{2.098466in}{2.168904in}}%
\pgfpathcurveto{\pgfqpoint{2.092642in}{2.174728in}}{\pgfqpoint{2.084742in}{2.178000in}}{\pgfqpoint{2.076506in}{2.178000in}}%
\pgfpathcurveto{\pgfqpoint{2.068269in}{2.178000in}}{\pgfqpoint{2.060369in}{2.174728in}}{\pgfqpoint{2.054545in}{2.168904in}}%
\pgfpathcurveto{\pgfqpoint{2.048721in}{2.163080in}}{\pgfqpoint{2.045449in}{2.155180in}}{\pgfqpoint{2.045449in}{2.146944in}}%
\pgfpathcurveto{\pgfqpoint{2.045449in}{2.138708in}}{\pgfqpoint{2.048721in}{2.130808in}}{\pgfqpoint{2.054545in}{2.124984in}}%
\pgfpathcurveto{\pgfqpoint{2.060369in}{2.119160in}}{\pgfqpoint{2.068269in}{2.115887in}}{\pgfqpoint{2.076506in}{2.115887in}}%
\pgfpathclose%
\pgfusepath{stroke,fill}%
\end{pgfscope}%
\begin{pgfscope}%
\pgfpathrectangle{\pgfqpoint{0.100000in}{0.212622in}}{\pgfqpoint{3.696000in}{3.696000in}}%
\pgfusepath{clip}%
\pgfsetbuttcap%
\pgfsetroundjoin%
\definecolor{currentfill}{rgb}{0.121569,0.466667,0.705882}%
\pgfsetfillcolor{currentfill}%
\pgfsetfillopacity{0.964724}%
\pgfsetlinewidth{1.003750pt}%
\definecolor{currentstroke}{rgb}{0.121569,0.466667,0.705882}%
\pgfsetstrokecolor{currentstroke}%
\pgfsetstrokeopacity{0.964724}%
\pgfsetdash{}{0pt}%
\pgfpathmoveto{\pgfqpoint{1.291142in}{1.776419in}}%
\pgfpathcurveto{\pgfqpoint{1.299379in}{1.776419in}}{\pgfqpoint{1.307279in}{1.779691in}}{\pgfqpoint{1.313103in}{1.785515in}}%
\pgfpathcurveto{\pgfqpoint{1.318927in}{1.791339in}}{\pgfqpoint{1.322199in}{1.799239in}}{\pgfqpoint{1.322199in}{1.807475in}}%
\pgfpathcurveto{\pgfqpoint{1.322199in}{1.815712in}}{\pgfqpoint{1.318927in}{1.823612in}}{\pgfqpoint{1.313103in}{1.829436in}}%
\pgfpathcurveto{\pgfqpoint{1.307279in}{1.835260in}}{\pgfqpoint{1.299379in}{1.838532in}}{\pgfqpoint{1.291142in}{1.838532in}}%
\pgfpathcurveto{\pgfqpoint{1.282906in}{1.838532in}}{\pgfqpoint{1.275006in}{1.835260in}}{\pgfqpoint{1.269182in}{1.829436in}}%
\pgfpathcurveto{\pgfqpoint{1.263358in}{1.823612in}}{\pgfqpoint{1.260086in}{1.815712in}}{\pgfqpoint{1.260086in}{1.807475in}}%
\pgfpathcurveto{\pgfqpoint{1.260086in}{1.799239in}}{\pgfqpoint{1.263358in}{1.791339in}}{\pgfqpoint{1.269182in}{1.785515in}}%
\pgfpathcurveto{\pgfqpoint{1.275006in}{1.779691in}}{\pgfqpoint{1.282906in}{1.776419in}}{\pgfqpoint{1.291142in}{1.776419in}}%
\pgfpathclose%
\pgfusepath{stroke,fill}%
\end{pgfscope}%
\begin{pgfscope}%
\pgfpathrectangle{\pgfqpoint{0.100000in}{0.212622in}}{\pgfqpoint{3.696000in}{3.696000in}}%
\pgfusepath{clip}%
\pgfsetbuttcap%
\pgfsetroundjoin%
\definecolor{currentfill}{rgb}{0.121569,0.466667,0.705882}%
\pgfsetfillcolor{currentfill}%
\pgfsetfillopacity{0.964817}%
\pgfsetlinewidth{1.003750pt}%
\definecolor{currentstroke}{rgb}{0.121569,0.466667,0.705882}%
\pgfsetstrokecolor{currentstroke}%
\pgfsetstrokeopacity{0.964817}%
\pgfsetdash{}{0pt}%
\pgfpathmoveto{\pgfqpoint{1.279530in}{1.786955in}}%
\pgfpathcurveto{\pgfqpoint{1.287766in}{1.786955in}}{\pgfqpoint{1.295666in}{1.790227in}}{\pgfqpoint{1.301490in}{1.796051in}}%
\pgfpathcurveto{\pgfqpoint{1.307314in}{1.801875in}}{\pgfqpoint{1.310586in}{1.809775in}}{\pgfqpoint{1.310586in}{1.818012in}}%
\pgfpathcurveto{\pgfqpoint{1.310586in}{1.826248in}}{\pgfqpoint{1.307314in}{1.834148in}}{\pgfqpoint{1.301490in}{1.839972in}}%
\pgfpathcurveto{\pgfqpoint{1.295666in}{1.845796in}}{\pgfqpoint{1.287766in}{1.849068in}}{\pgfqpoint{1.279530in}{1.849068in}}%
\pgfpathcurveto{\pgfqpoint{1.271294in}{1.849068in}}{\pgfqpoint{1.263394in}{1.845796in}}{\pgfqpoint{1.257570in}{1.839972in}}%
\pgfpathcurveto{\pgfqpoint{1.251746in}{1.834148in}}{\pgfqpoint{1.248473in}{1.826248in}}{\pgfqpoint{1.248473in}{1.818012in}}%
\pgfpathcurveto{\pgfqpoint{1.248473in}{1.809775in}}{\pgfqpoint{1.251746in}{1.801875in}}{\pgfqpoint{1.257570in}{1.796051in}}%
\pgfpathcurveto{\pgfqpoint{1.263394in}{1.790227in}}{\pgfqpoint{1.271294in}{1.786955in}}{\pgfqpoint{1.279530in}{1.786955in}}%
\pgfpathclose%
\pgfusepath{stroke,fill}%
\end{pgfscope}%
\begin{pgfscope}%
\pgfpathrectangle{\pgfqpoint{0.100000in}{0.212622in}}{\pgfqpoint{3.696000in}{3.696000in}}%
\pgfusepath{clip}%
\pgfsetbuttcap%
\pgfsetroundjoin%
\definecolor{currentfill}{rgb}{0.121569,0.466667,0.705882}%
\pgfsetfillcolor{currentfill}%
\pgfsetfillopacity{0.965091}%
\pgfsetlinewidth{1.003750pt}%
\definecolor{currentstroke}{rgb}{0.121569,0.466667,0.705882}%
\pgfsetstrokecolor{currentstroke}%
\pgfsetstrokeopacity{0.965091}%
\pgfsetdash{}{0pt}%
\pgfpathmoveto{\pgfqpoint{2.964475in}{2.524424in}}%
\pgfpathcurveto{\pgfqpoint{2.972711in}{2.524424in}}{\pgfqpoint{2.980611in}{2.527697in}}{\pgfqpoint{2.986435in}{2.533521in}}%
\pgfpathcurveto{\pgfqpoint{2.992259in}{2.539345in}}{\pgfqpoint{2.995531in}{2.547245in}}{\pgfqpoint{2.995531in}{2.555481in}}%
\pgfpathcurveto{\pgfqpoint{2.995531in}{2.563717in}}{\pgfqpoint{2.992259in}{2.571617in}}{\pgfqpoint{2.986435in}{2.577441in}}%
\pgfpathcurveto{\pgfqpoint{2.980611in}{2.583265in}}{\pgfqpoint{2.972711in}{2.586537in}}{\pgfqpoint{2.964475in}{2.586537in}}%
\pgfpathcurveto{\pgfqpoint{2.956239in}{2.586537in}}{\pgfqpoint{2.948339in}{2.583265in}}{\pgfqpoint{2.942515in}{2.577441in}}%
\pgfpathcurveto{\pgfqpoint{2.936691in}{2.571617in}}{\pgfqpoint{2.933418in}{2.563717in}}{\pgfqpoint{2.933418in}{2.555481in}}%
\pgfpathcurveto{\pgfqpoint{2.933418in}{2.547245in}}{\pgfqpoint{2.936691in}{2.539345in}}{\pgfqpoint{2.942515in}{2.533521in}}%
\pgfpathcurveto{\pgfqpoint{2.948339in}{2.527697in}}{\pgfqpoint{2.956239in}{2.524424in}}{\pgfqpoint{2.964475in}{2.524424in}}%
\pgfpathclose%
\pgfusepath{stroke,fill}%
\end{pgfscope}%
\begin{pgfscope}%
\pgfpathrectangle{\pgfqpoint{0.100000in}{0.212622in}}{\pgfqpoint{3.696000in}{3.696000in}}%
\pgfusepath{clip}%
\pgfsetbuttcap%
\pgfsetroundjoin%
\definecolor{currentfill}{rgb}{0.121569,0.466667,0.705882}%
\pgfsetfillcolor{currentfill}%
\pgfsetfillopacity{0.965333}%
\pgfsetlinewidth{1.003750pt}%
\definecolor{currentstroke}{rgb}{0.121569,0.466667,0.705882}%
\pgfsetstrokecolor{currentstroke}%
\pgfsetstrokeopacity{0.965333}%
\pgfsetdash{}{0pt}%
\pgfpathmoveto{\pgfqpoint{1.811268in}{2.062658in}}%
\pgfpathcurveto{\pgfqpoint{1.819505in}{2.062658in}}{\pgfqpoint{1.827405in}{2.065931in}}{\pgfqpoint{1.833229in}{2.071755in}}%
\pgfpathcurveto{\pgfqpoint{1.839053in}{2.077579in}}{\pgfqpoint{1.842325in}{2.085479in}}{\pgfqpoint{1.842325in}{2.093715in}}%
\pgfpathcurveto{\pgfqpoint{1.842325in}{2.101951in}}{\pgfqpoint{1.839053in}{2.109851in}}{\pgfqpoint{1.833229in}{2.115675in}}%
\pgfpathcurveto{\pgfqpoint{1.827405in}{2.121499in}}{\pgfqpoint{1.819505in}{2.124771in}}{\pgfqpoint{1.811268in}{2.124771in}}%
\pgfpathcurveto{\pgfqpoint{1.803032in}{2.124771in}}{\pgfqpoint{1.795132in}{2.121499in}}{\pgfqpoint{1.789308in}{2.115675in}}%
\pgfpathcurveto{\pgfqpoint{1.783484in}{2.109851in}}{\pgfqpoint{1.780212in}{2.101951in}}{\pgfqpoint{1.780212in}{2.093715in}}%
\pgfpathcurveto{\pgfqpoint{1.780212in}{2.085479in}}{\pgfqpoint{1.783484in}{2.077579in}}{\pgfqpoint{1.789308in}{2.071755in}}%
\pgfpathcurveto{\pgfqpoint{1.795132in}{2.065931in}}{\pgfqpoint{1.803032in}{2.062658in}}{\pgfqpoint{1.811268in}{2.062658in}}%
\pgfpathclose%
\pgfusepath{stroke,fill}%
\end{pgfscope}%
\begin{pgfscope}%
\pgfpathrectangle{\pgfqpoint{0.100000in}{0.212622in}}{\pgfqpoint{3.696000in}{3.696000in}}%
\pgfusepath{clip}%
\pgfsetbuttcap%
\pgfsetroundjoin%
\definecolor{currentfill}{rgb}{0.121569,0.466667,0.705882}%
\pgfsetfillcolor{currentfill}%
\pgfsetfillopacity{0.965946}%
\pgfsetlinewidth{1.003750pt}%
\definecolor{currentstroke}{rgb}{0.121569,0.466667,0.705882}%
\pgfsetstrokecolor{currentstroke}%
\pgfsetstrokeopacity{0.965946}%
\pgfsetdash{}{0pt}%
\pgfpathmoveto{\pgfqpoint{1.284717in}{1.790083in}}%
\pgfpathcurveto{\pgfqpoint{1.292954in}{1.790083in}}{\pgfqpoint{1.300854in}{1.793355in}}{\pgfqpoint{1.306678in}{1.799179in}}%
\pgfpathcurveto{\pgfqpoint{1.312502in}{1.805003in}}{\pgfqpoint{1.315774in}{1.812903in}}{\pgfqpoint{1.315774in}{1.821140in}}%
\pgfpathcurveto{\pgfqpoint{1.315774in}{1.829376in}}{\pgfqpoint{1.312502in}{1.837276in}}{\pgfqpoint{1.306678in}{1.843100in}}%
\pgfpathcurveto{\pgfqpoint{1.300854in}{1.848924in}}{\pgfqpoint{1.292954in}{1.852196in}}{\pgfqpoint{1.284717in}{1.852196in}}%
\pgfpathcurveto{\pgfqpoint{1.276481in}{1.852196in}}{\pgfqpoint{1.268581in}{1.848924in}}{\pgfqpoint{1.262757in}{1.843100in}}%
\pgfpathcurveto{\pgfqpoint{1.256933in}{1.837276in}}{\pgfqpoint{1.253661in}{1.829376in}}{\pgfqpoint{1.253661in}{1.821140in}}%
\pgfpathcurveto{\pgfqpoint{1.253661in}{1.812903in}}{\pgfqpoint{1.256933in}{1.805003in}}{\pgfqpoint{1.262757in}{1.799179in}}%
\pgfpathcurveto{\pgfqpoint{1.268581in}{1.793355in}}{\pgfqpoint{1.276481in}{1.790083in}}{\pgfqpoint{1.284717in}{1.790083in}}%
\pgfpathclose%
\pgfusepath{stroke,fill}%
\end{pgfscope}%
\begin{pgfscope}%
\pgfpathrectangle{\pgfqpoint{0.100000in}{0.212622in}}{\pgfqpoint{3.696000in}{3.696000in}}%
\pgfusepath{clip}%
\pgfsetbuttcap%
\pgfsetroundjoin%
\definecolor{currentfill}{rgb}{0.121569,0.466667,0.705882}%
\pgfsetfillcolor{currentfill}%
\pgfsetfillopacity{0.966041}%
\pgfsetlinewidth{1.003750pt}%
\definecolor{currentstroke}{rgb}{0.121569,0.466667,0.705882}%
\pgfsetstrokecolor{currentstroke}%
\pgfsetstrokeopacity{0.966041}%
\pgfsetdash{}{0pt}%
\pgfpathmoveto{\pgfqpoint{2.009930in}{2.117326in}}%
\pgfpathcurveto{\pgfqpoint{2.018166in}{2.117326in}}{\pgfqpoint{2.026066in}{2.120598in}}{\pgfqpoint{2.031890in}{2.126422in}}%
\pgfpathcurveto{\pgfqpoint{2.037714in}{2.132246in}}{\pgfqpoint{2.040986in}{2.140146in}}{\pgfqpoint{2.040986in}{2.148382in}}%
\pgfpathcurveto{\pgfqpoint{2.040986in}{2.156619in}}{\pgfqpoint{2.037714in}{2.164519in}}{\pgfqpoint{2.031890in}{2.170343in}}%
\pgfpathcurveto{\pgfqpoint{2.026066in}{2.176166in}}{\pgfqpoint{2.018166in}{2.179439in}}{\pgfqpoint{2.009930in}{2.179439in}}%
\pgfpathcurveto{\pgfqpoint{2.001694in}{2.179439in}}{\pgfqpoint{1.993793in}{2.176166in}}{\pgfqpoint{1.987970in}{2.170343in}}%
\pgfpathcurveto{\pgfqpoint{1.982146in}{2.164519in}}{\pgfqpoint{1.978873in}{2.156619in}}{\pgfqpoint{1.978873in}{2.148382in}}%
\pgfpathcurveto{\pgfqpoint{1.978873in}{2.140146in}}{\pgfqpoint{1.982146in}{2.132246in}}{\pgfqpoint{1.987970in}{2.126422in}}%
\pgfpathcurveto{\pgfqpoint{1.993793in}{2.120598in}}{\pgfqpoint{2.001694in}{2.117326in}}{\pgfqpoint{2.009930in}{2.117326in}}%
\pgfpathclose%
\pgfusepath{stroke,fill}%
\end{pgfscope}%
\begin{pgfscope}%
\pgfpathrectangle{\pgfqpoint{0.100000in}{0.212622in}}{\pgfqpoint{3.696000in}{3.696000in}}%
\pgfusepath{clip}%
\pgfsetbuttcap%
\pgfsetroundjoin%
\definecolor{currentfill}{rgb}{0.121569,0.466667,0.705882}%
\pgfsetfillcolor{currentfill}%
\pgfsetfillopacity{0.966149}%
\pgfsetlinewidth{1.003750pt}%
\definecolor{currentstroke}{rgb}{0.121569,0.466667,0.705882}%
\pgfsetstrokecolor{currentstroke}%
\pgfsetstrokeopacity{0.966149}%
\pgfsetdash{}{0pt}%
\pgfpathmoveto{\pgfqpoint{1.830990in}{2.070211in}}%
\pgfpathcurveto{\pgfqpoint{1.839227in}{2.070211in}}{\pgfqpoint{1.847127in}{2.073483in}}{\pgfqpoint{1.852951in}{2.079307in}}%
\pgfpathcurveto{\pgfqpoint{1.858775in}{2.085131in}}{\pgfqpoint{1.862047in}{2.093031in}}{\pgfqpoint{1.862047in}{2.101268in}}%
\pgfpathcurveto{\pgfqpoint{1.862047in}{2.109504in}}{\pgfqpoint{1.858775in}{2.117404in}}{\pgfqpoint{1.852951in}{2.123228in}}%
\pgfpathcurveto{\pgfqpoint{1.847127in}{2.129052in}}{\pgfqpoint{1.839227in}{2.132324in}}{\pgfqpoint{1.830990in}{2.132324in}}%
\pgfpathcurveto{\pgfqpoint{1.822754in}{2.132324in}}{\pgfqpoint{1.814854in}{2.129052in}}{\pgfqpoint{1.809030in}{2.123228in}}%
\pgfpathcurveto{\pgfqpoint{1.803206in}{2.117404in}}{\pgfqpoint{1.799934in}{2.109504in}}{\pgfqpoint{1.799934in}{2.101268in}}%
\pgfpathcurveto{\pgfqpoint{1.799934in}{2.093031in}}{\pgfqpoint{1.803206in}{2.085131in}}{\pgfqpoint{1.809030in}{2.079307in}}%
\pgfpathcurveto{\pgfqpoint{1.814854in}{2.073483in}}{\pgfqpoint{1.822754in}{2.070211in}}{\pgfqpoint{1.830990in}{2.070211in}}%
\pgfpathclose%
\pgfusepath{stroke,fill}%
\end{pgfscope}%
\begin{pgfscope}%
\pgfpathrectangle{\pgfqpoint{0.100000in}{0.212622in}}{\pgfqpoint{3.696000in}{3.696000in}}%
\pgfusepath{clip}%
\pgfsetbuttcap%
\pgfsetroundjoin%
\definecolor{currentfill}{rgb}{0.121569,0.466667,0.705882}%
\pgfsetfillcolor{currentfill}%
\pgfsetfillopacity{0.966285}%
\pgfsetlinewidth{1.003750pt}%
\definecolor{currentstroke}{rgb}{0.121569,0.466667,0.705882}%
\pgfsetstrokecolor{currentstroke}%
\pgfsetstrokeopacity{0.966285}%
\pgfsetdash{}{0pt}%
\pgfpathmoveto{\pgfqpoint{1.963551in}{2.107822in}}%
\pgfpathcurveto{\pgfqpoint{1.971787in}{2.107822in}}{\pgfqpoint{1.979687in}{2.111094in}}{\pgfqpoint{1.985511in}{2.116918in}}%
\pgfpathcurveto{\pgfqpoint{1.991335in}{2.122742in}}{\pgfqpoint{1.994607in}{2.130642in}}{\pgfqpoint{1.994607in}{2.138878in}}%
\pgfpathcurveto{\pgfqpoint{1.994607in}{2.147115in}}{\pgfqpoint{1.991335in}{2.155015in}}{\pgfqpoint{1.985511in}{2.160839in}}%
\pgfpathcurveto{\pgfqpoint{1.979687in}{2.166663in}}{\pgfqpoint{1.971787in}{2.169935in}}{\pgfqpoint{1.963551in}{2.169935in}}%
\pgfpathcurveto{\pgfqpoint{1.955314in}{2.169935in}}{\pgfqpoint{1.947414in}{2.166663in}}{\pgfqpoint{1.941590in}{2.160839in}}%
\pgfpathcurveto{\pgfqpoint{1.935767in}{2.155015in}}{\pgfqpoint{1.932494in}{2.147115in}}{\pgfqpoint{1.932494in}{2.138878in}}%
\pgfpathcurveto{\pgfqpoint{1.932494in}{2.130642in}}{\pgfqpoint{1.935767in}{2.122742in}}{\pgfqpoint{1.941590in}{2.116918in}}%
\pgfpathcurveto{\pgfqpoint{1.947414in}{2.111094in}}{\pgfqpoint{1.955314in}{2.107822in}}{\pgfqpoint{1.963551in}{2.107822in}}%
\pgfpathclose%
\pgfusepath{stroke,fill}%
\end{pgfscope}%
\begin{pgfscope}%
\pgfpathrectangle{\pgfqpoint{0.100000in}{0.212622in}}{\pgfqpoint{3.696000in}{3.696000in}}%
\pgfusepath{clip}%
\pgfsetbuttcap%
\pgfsetroundjoin%
\definecolor{currentfill}{rgb}{0.121569,0.466667,0.705882}%
\pgfsetfillcolor{currentfill}%
\pgfsetfillopacity{0.966349}%
\pgfsetlinewidth{1.003750pt}%
\definecolor{currentstroke}{rgb}{0.121569,0.466667,0.705882}%
\pgfsetstrokecolor{currentstroke}%
\pgfsetstrokeopacity{0.966349}%
\pgfsetdash{}{0pt}%
\pgfpathmoveto{\pgfqpoint{2.982037in}{2.550057in}}%
\pgfpathcurveto{\pgfqpoint{2.990273in}{2.550057in}}{\pgfqpoint{2.998173in}{2.553329in}}{\pgfqpoint{3.003997in}{2.559153in}}%
\pgfpathcurveto{\pgfqpoint{3.009821in}{2.564977in}}{\pgfqpoint{3.013093in}{2.572877in}}{\pgfqpoint{3.013093in}{2.581113in}}%
\pgfpathcurveto{\pgfqpoint{3.013093in}{2.589350in}}{\pgfqpoint{3.009821in}{2.597250in}}{\pgfqpoint{3.003997in}{2.603074in}}%
\pgfpathcurveto{\pgfqpoint{2.998173in}{2.608898in}}{\pgfqpoint{2.990273in}{2.612170in}}{\pgfqpoint{2.982037in}{2.612170in}}%
\pgfpathcurveto{\pgfqpoint{2.973800in}{2.612170in}}{\pgfqpoint{2.965900in}{2.608898in}}{\pgfqpoint{2.960076in}{2.603074in}}%
\pgfpathcurveto{\pgfqpoint{2.954253in}{2.597250in}}{\pgfqpoint{2.950980in}{2.589350in}}{\pgfqpoint{2.950980in}{2.581113in}}%
\pgfpathcurveto{\pgfqpoint{2.950980in}{2.572877in}}{\pgfqpoint{2.954253in}{2.564977in}}{\pgfqpoint{2.960076in}{2.559153in}}%
\pgfpathcurveto{\pgfqpoint{2.965900in}{2.553329in}}{\pgfqpoint{2.973800in}{2.550057in}}{\pgfqpoint{2.982037in}{2.550057in}}%
\pgfpathclose%
\pgfusepath{stroke,fill}%
\end{pgfscope}%
\begin{pgfscope}%
\pgfpathrectangle{\pgfqpoint{0.100000in}{0.212622in}}{\pgfqpoint{3.696000in}{3.696000in}}%
\pgfusepath{clip}%
\pgfsetbuttcap%
\pgfsetroundjoin%
\definecolor{currentfill}{rgb}{0.121569,0.466667,0.705882}%
\pgfsetfillcolor{currentfill}%
\pgfsetfillopacity{0.967271}%
\pgfsetlinewidth{1.003750pt}%
\definecolor{currentstroke}{rgb}{0.121569,0.466667,0.705882}%
\pgfsetstrokecolor{currentstroke}%
\pgfsetstrokeopacity{0.967271}%
\pgfsetdash{}{0pt}%
\pgfpathmoveto{\pgfqpoint{1.957422in}{2.116972in}}%
\pgfpathcurveto{\pgfqpoint{1.965658in}{2.116972in}}{\pgfqpoint{1.973558in}{2.120244in}}{\pgfqpoint{1.979382in}{2.126068in}}%
\pgfpathcurveto{\pgfqpoint{1.985206in}{2.131892in}}{\pgfqpoint{1.988478in}{2.139792in}}{\pgfqpoint{1.988478in}{2.148028in}}%
\pgfpathcurveto{\pgfqpoint{1.988478in}{2.156265in}}{\pgfqpoint{1.985206in}{2.164165in}}{\pgfqpoint{1.979382in}{2.169989in}}%
\pgfpathcurveto{\pgfqpoint{1.973558in}{2.175812in}}{\pgfqpoint{1.965658in}{2.179085in}}{\pgfqpoint{1.957422in}{2.179085in}}%
\pgfpathcurveto{\pgfqpoint{1.949185in}{2.179085in}}{\pgfqpoint{1.941285in}{2.175812in}}{\pgfqpoint{1.935461in}{2.169989in}}%
\pgfpathcurveto{\pgfqpoint{1.929637in}{2.164165in}}{\pgfqpoint{1.926365in}{2.156265in}}{\pgfqpoint{1.926365in}{2.148028in}}%
\pgfpathcurveto{\pgfqpoint{1.926365in}{2.139792in}}{\pgfqpoint{1.929637in}{2.131892in}}{\pgfqpoint{1.935461in}{2.126068in}}%
\pgfpathcurveto{\pgfqpoint{1.941285in}{2.120244in}}{\pgfqpoint{1.949185in}{2.116972in}}{\pgfqpoint{1.957422in}{2.116972in}}%
\pgfpathclose%
\pgfusepath{stroke,fill}%
\end{pgfscope}%
\begin{pgfscope}%
\pgfpathrectangle{\pgfqpoint{0.100000in}{0.212622in}}{\pgfqpoint{3.696000in}{3.696000in}}%
\pgfusepath{clip}%
\pgfsetbuttcap%
\pgfsetroundjoin%
\definecolor{currentfill}{rgb}{0.121569,0.466667,0.705882}%
\pgfsetfillcolor{currentfill}%
\pgfsetfillopacity{0.967394}%
\pgfsetlinewidth{1.003750pt}%
\definecolor{currentstroke}{rgb}{0.121569,0.466667,0.705882}%
\pgfsetstrokecolor{currentstroke}%
\pgfsetstrokeopacity{0.967394}%
\pgfsetdash{}{0pt}%
\pgfpathmoveto{\pgfqpoint{2.046544in}{2.127197in}}%
\pgfpathcurveto{\pgfqpoint{2.054780in}{2.127197in}}{\pgfqpoint{2.062680in}{2.130470in}}{\pgfqpoint{2.068504in}{2.136293in}}%
\pgfpathcurveto{\pgfqpoint{2.074328in}{2.142117in}}{\pgfqpoint{2.077600in}{2.150017in}}{\pgfqpoint{2.077600in}{2.158254in}}%
\pgfpathcurveto{\pgfqpoint{2.077600in}{2.166490in}}{\pgfqpoint{2.074328in}{2.174390in}}{\pgfqpoint{2.068504in}{2.180214in}}%
\pgfpathcurveto{\pgfqpoint{2.062680in}{2.186038in}}{\pgfqpoint{2.054780in}{2.189310in}}{\pgfqpoint{2.046544in}{2.189310in}}%
\pgfpathcurveto{\pgfqpoint{2.038307in}{2.189310in}}{\pgfqpoint{2.030407in}{2.186038in}}{\pgfqpoint{2.024584in}{2.180214in}}%
\pgfpathcurveto{\pgfqpoint{2.018760in}{2.174390in}}{\pgfqpoint{2.015487in}{2.166490in}}{\pgfqpoint{2.015487in}{2.158254in}}%
\pgfpathcurveto{\pgfqpoint{2.015487in}{2.150017in}}{\pgfqpoint{2.018760in}{2.142117in}}{\pgfqpoint{2.024584in}{2.136293in}}%
\pgfpathcurveto{\pgfqpoint{2.030407in}{2.130470in}}{\pgfqpoint{2.038307in}{2.127197in}}{\pgfqpoint{2.046544in}{2.127197in}}%
\pgfpathclose%
\pgfusepath{stroke,fill}%
\end{pgfscope}%
\begin{pgfscope}%
\pgfpathrectangle{\pgfqpoint{0.100000in}{0.212622in}}{\pgfqpoint{3.696000in}{3.696000in}}%
\pgfusepath{clip}%
\pgfsetbuttcap%
\pgfsetroundjoin%
\definecolor{currentfill}{rgb}{0.121569,0.466667,0.705882}%
\pgfsetfillcolor{currentfill}%
\pgfsetfillopacity{0.968886}%
\pgfsetlinewidth{1.003750pt}%
\definecolor{currentstroke}{rgb}{0.121569,0.466667,0.705882}%
\pgfsetstrokecolor{currentstroke}%
\pgfsetstrokeopacity{0.968886}%
\pgfsetdash{}{0pt}%
\pgfpathmoveto{\pgfqpoint{1.303297in}{1.769372in}}%
\pgfpathcurveto{\pgfqpoint{1.311534in}{1.769372in}}{\pgfqpoint{1.319434in}{1.772644in}}{\pgfqpoint{1.325258in}{1.778468in}}%
\pgfpathcurveto{\pgfqpoint{1.331082in}{1.784292in}}{\pgfqpoint{1.334354in}{1.792192in}}{\pgfqpoint{1.334354in}{1.800428in}}%
\pgfpathcurveto{\pgfqpoint{1.334354in}{1.808665in}}{\pgfqpoint{1.331082in}{1.816565in}}{\pgfqpoint{1.325258in}{1.822389in}}%
\pgfpathcurveto{\pgfqpoint{1.319434in}{1.828212in}}{\pgfqpoint{1.311534in}{1.831485in}}{\pgfqpoint{1.303297in}{1.831485in}}%
\pgfpathcurveto{\pgfqpoint{1.295061in}{1.831485in}}{\pgfqpoint{1.287161in}{1.828212in}}{\pgfqpoint{1.281337in}{1.822389in}}%
\pgfpathcurveto{\pgfqpoint{1.275513in}{1.816565in}}{\pgfqpoint{1.272241in}{1.808665in}}{\pgfqpoint{1.272241in}{1.800428in}}%
\pgfpathcurveto{\pgfqpoint{1.272241in}{1.792192in}}{\pgfqpoint{1.275513in}{1.784292in}}{\pgfqpoint{1.281337in}{1.778468in}}%
\pgfpathcurveto{\pgfqpoint{1.287161in}{1.772644in}}{\pgfqpoint{1.295061in}{1.769372in}}{\pgfqpoint{1.303297in}{1.769372in}}%
\pgfpathclose%
\pgfusepath{stroke,fill}%
\end{pgfscope}%
\begin{pgfscope}%
\pgfpathrectangle{\pgfqpoint{0.100000in}{0.212622in}}{\pgfqpoint{3.696000in}{3.696000in}}%
\pgfusepath{clip}%
\pgfsetbuttcap%
\pgfsetroundjoin%
\definecolor{currentfill}{rgb}{0.121569,0.466667,0.705882}%
\pgfsetfillcolor{currentfill}%
\pgfsetfillopacity{0.970375}%
\pgfsetlinewidth{1.003750pt}%
\definecolor{currentstroke}{rgb}{0.121569,0.466667,0.705882}%
\pgfsetstrokecolor{currentstroke}%
\pgfsetstrokeopacity{0.970375}%
\pgfsetdash{}{0pt}%
\pgfpathmoveto{\pgfqpoint{2.957334in}{2.516294in}}%
\pgfpathcurveto{\pgfqpoint{2.965570in}{2.516294in}}{\pgfqpoint{2.973470in}{2.519567in}}{\pgfqpoint{2.979294in}{2.525391in}}%
\pgfpathcurveto{\pgfqpoint{2.985118in}{2.531214in}}{\pgfqpoint{2.988390in}{2.539115in}}{\pgfqpoint{2.988390in}{2.547351in}}%
\pgfpathcurveto{\pgfqpoint{2.988390in}{2.555587in}}{\pgfqpoint{2.985118in}{2.563487in}}{\pgfqpoint{2.979294in}{2.569311in}}%
\pgfpathcurveto{\pgfqpoint{2.973470in}{2.575135in}}{\pgfqpoint{2.965570in}{2.578407in}}{\pgfqpoint{2.957334in}{2.578407in}}%
\pgfpathcurveto{\pgfqpoint{2.949097in}{2.578407in}}{\pgfqpoint{2.941197in}{2.575135in}}{\pgfqpoint{2.935373in}{2.569311in}}%
\pgfpathcurveto{\pgfqpoint{2.929550in}{2.563487in}}{\pgfqpoint{2.926277in}{2.555587in}}{\pgfqpoint{2.926277in}{2.547351in}}%
\pgfpathcurveto{\pgfqpoint{2.926277in}{2.539115in}}{\pgfqpoint{2.929550in}{2.531214in}}{\pgfqpoint{2.935373in}{2.525391in}}%
\pgfpathcurveto{\pgfqpoint{2.941197in}{2.519567in}}{\pgfqpoint{2.949097in}{2.516294in}}{\pgfqpoint{2.957334in}{2.516294in}}%
\pgfpathclose%
\pgfusepath{stroke,fill}%
\end{pgfscope}%
\begin{pgfscope}%
\pgfpathrectangle{\pgfqpoint{0.100000in}{0.212622in}}{\pgfqpoint{3.696000in}{3.696000in}}%
\pgfusepath{clip}%
\pgfsetbuttcap%
\pgfsetroundjoin%
\definecolor{currentfill}{rgb}{0.121569,0.466667,0.705882}%
\pgfsetfillcolor{currentfill}%
\pgfsetfillopacity{0.971014}%
\pgfsetlinewidth{1.003750pt}%
\definecolor{currentstroke}{rgb}{0.121569,0.466667,0.705882}%
\pgfsetstrokecolor{currentstroke}%
\pgfsetstrokeopacity{0.971014}%
\pgfsetdash{}{0pt}%
\pgfpathmoveto{\pgfqpoint{2.017502in}{2.120059in}}%
\pgfpathcurveto{\pgfqpoint{2.025738in}{2.120059in}}{\pgfqpoint{2.033638in}{2.123331in}}{\pgfqpoint{2.039462in}{2.129155in}}%
\pgfpathcurveto{\pgfqpoint{2.045286in}{2.134979in}}{\pgfqpoint{2.048558in}{2.142879in}}{\pgfqpoint{2.048558in}{2.151115in}}%
\pgfpathcurveto{\pgfqpoint{2.048558in}{2.159352in}}{\pgfqpoint{2.045286in}{2.167252in}}{\pgfqpoint{2.039462in}{2.173076in}}%
\pgfpathcurveto{\pgfqpoint{2.033638in}{2.178900in}}{\pgfqpoint{2.025738in}{2.182172in}}{\pgfqpoint{2.017502in}{2.182172in}}%
\pgfpathcurveto{\pgfqpoint{2.009265in}{2.182172in}}{\pgfqpoint{2.001365in}{2.178900in}}{\pgfqpoint{1.995541in}{2.173076in}}%
\pgfpathcurveto{\pgfqpoint{1.989718in}{2.167252in}}{\pgfqpoint{1.986445in}{2.159352in}}{\pgfqpoint{1.986445in}{2.151115in}}%
\pgfpathcurveto{\pgfqpoint{1.986445in}{2.142879in}}{\pgfqpoint{1.989718in}{2.134979in}}{\pgfqpoint{1.995541in}{2.129155in}}%
\pgfpathcurveto{\pgfqpoint{2.001365in}{2.123331in}}{\pgfqpoint{2.009265in}{2.120059in}}{\pgfqpoint{2.017502in}{2.120059in}}%
\pgfpathclose%
\pgfusepath{stroke,fill}%
\end{pgfscope}%
\begin{pgfscope}%
\pgfpathrectangle{\pgfqpoint{0.100000in}{0.212622in}}{\pgfqpoint{3.696000in}{3.696000in}}%
\pgfusepath{clip}%
\pgfsetbuttcap%
\pgfsetroundjoin%
\definecolor{currentfill}{rgb}{0.121569,0.466667,0.705882}%
\pgfsetfillcolor{currentfill}%
\pgfsetfillopacity{0.972939}%
\pgfsetlinewidth{1.003750pt}%
\definecolor{currentstroke}{rgb}{0.121569,0.466667,0.705882}%
\pgfsetstrokecolor{currentstroke}%
\pgfsetstrokeopacity{0.972939}%
\pgfsetdash{}{0pt}%
\pgfpathmoveto{\pgfqpoint{2.963310in}{2.525559in}}%
\pgfpathcurveto{\pgfqpoint{2.971546in}{2.525559in}}{\pgfqpoint{2.979446in}{2.528831in}}{\pgfqpoint{2.985270in}{2.534655in}}%
\pgfpathcurveto{\pgfqpoint{2.991094in}{2.540479in}}{\pgfqpoint{2.994366in}{2.548379in}}{\pgfqpoint{2.994366in}{2.556616in}}%
\pgfpathcurveto{\pgfqpoint{2.994366in}{2.564852in}}{\pgfqpoint{2.991094in}{2.572752in}}{\pgfqpoint{2.985270in}{2.578576in}}%
\pgfpathcurveto{\pgfqpoint{2.979446in}{2.584400in}}{\pgfqpoint{2.971546in}{2.587672in}}{\pgfqpoint{2.963310in}{2.587672in}}%
\pgfpathcurveto{\pgfqpoint{2.955073in}{2.587672in}}{\pgfqpoint{2.947173in}{2.584400in}}{\pgfqpoint{2.941349in}{2.578576in}}%
\pgfpathcurveto{\pgfqpoint{2.935525in}{2.572752in}}{\pgfqpoint{2.932253in}{2.564852in}}{\pgfqpoint{2.932253in}{2.556616in}}%
\pgfpathcurveto{\pgfqpoint{2.932253in}{2.548379in}}{\pgfqpoint{2.935525in}{2.540479in}}{\pgfqpoint{2.941349in}{2.534655in}}%
\pgfpathcurveto{\pgfqpoint{2.947173in}{2.528831in}}{\pgfqpoint{2.955073in}{2.525559in}}{\pgfqpoint{2.963310in}{2.525559in}}%
\pgfpathclose%
\pgfusepath{stroke,fill}%
\end{pgfscope}%
\begin{pgfscope}%
\pgfpathrectangle{\pgfqpoint{0.100000in}{0.212622in}}{\pgfqpoint{3.696000in}{3.696000in}}%
\pgfusepath{clip}%
\pgfsetbuttcap%
\pgfsetroundjoin%
\definecolor{currentfill}{rgb}{0.121569,0.466667,0.705882}%
\pgfsetfillcolor{currentfill}%
\pgfsetfillopacity{0.973194}%
\pgfsetlinewidth{1.003750pt}%
\definecolor{currentstroke}{rgb}{0.121569,0.466667,0.705882}%
\pgfsetstrokecolor{currentstroke}%
\pgfsetstrokeopacity{0.973194}%
\pgfsetdash{}{0pt}%
\pgfpathmoveto{\pgfqpoint{2.024009in}{2.121005in}}%
\pgfpathcurveto{\pgfqpoint{2.032246in}{2.121005in}}{\pgfqpoint{2.040146in}{2.124277in}}{\pgfqpoint{2.045970in}{2.130101in}}%
\pgfpathcurveto{\pgfqpoint{2.051794in}{2.135925in}}{\pgfqpoint{2.055066in}{2.143825in}}{\pgfqpoint{2.055066in}{2.152061in}}%
\pgfpathcurveto{\pgfqpoint{2.055066in}{2.160298in}}{\pgfqpoint{2.051794in}{2.168198in}}{\pgfqpoint{2.045970in}{2.174022in}}%
\pgfpathcurveto{\pgfqpoint{2.040146in}{2.179846in}}{\pgfqpoint{2.032246in}{2.183118in}}{\pgfqpoint{2.024009in}{2.183118in}}%
\pgfpathcurveto{\pgfqpoint{2.015773in}{2.183118in}}{\pgfqpoint{2.007873in}{2.179846in}}{\pgfqpoint{2.002049in}{2.174022in}}%
\pgfpathcurveto{\pgfqpoint{1.996225in}{2.168198in}}{\pgfqpoint{1.992953in}{2.160298in}}{\pgfqpoint{1.992953in}{2.152061in}}%
\pgfpathcurveto{\pgfqpoint{1.992953in}{2.143825in}}{\pgfqpoint{1.996225in}{2.135925in}}{\pgfqpoint{2.002049in}{2.130101in}}%
\pgfpathcurveto{\pgfqpoint{2.007873in}{2.124277in}}{\pgfqpoint{2.015773in}{2.121005in}}{\pgfqpoint{2.024009in}{2.121005in}}%
\pgfpathclose%
\pgfusepath{stroke,fill}%
\end{pgfscope}%
\begin{pgfscope}%
\pgfpathrectangle{\pgfqpoint{0.100000in}{0.212622in}}{\pgfqpoint{3.696000in}{3.696000in}}%
\pgfusepath{clip}%
\pgfsetbuttcap%
\pgfsetroundjoin%
\definecolor{currentfill}{rgb}{0.121569,0.466667,0.705882}%
\pgfsetfillcolor{currentfill}%
\pgfsetfillopacity{0.974098}%
\pgfsetlinewidth{1.003750pt}%
\definecolor{currentstroke}{rgb}{0.121569,0.466667,0.705882}%
\pgfsetstrokecolor{currentstroke}%
\pgfsetstrokeopacity{0.974098}%
\pgfsetdash{}{0pt}%
\pgfpathmoveto{\pgfqpoint{2.951636in}{2.509974in}}%
\pgfpathcurveto{\pgfqpoint{2.959872in}{2.509974in}}{\pgfqpoint{2.967772in}{2.513246in}}{\pgfqpoint{2.973596in}{2.519070in}}%
\pgfpathcurveto{\pgfqpoint{2.979420in}{2.524894in}}{\pgfqpoint{2.982692in}{2.532794in}}{\pgfqpoint{2.982692in}{2.541031in}}%
\pgfpathcurveto{\pgfqpoint{2.982692in}{2.549267in}}{\pgfqpoint{2.979420in}{2.557167in}}{\pgfqpoint{2.973596in}{2.562991in}}%
\pgfpathcurveto{\pgfqpoint{2.967772in}{2.568815in}}{\pgfqpoint{2.959872in}{2.572087in}}{\pgfqpoint{2.951636in}{2.572087in}}%
\pgfpathcurveto{\pgfqpoint{2.943399in}{2.572087in}}{\pgfqpoint{2.935499in}{2.568815in}}{\pgfqpoint{2.929675in}{2.562991in}}%
\pgfpathcurveto{\pgfqpoint{2.923851in}{2.557167in}}{\pgfqpoint{2.920579in}{2.549267in}}{\pgfqpoint{2.920579in}{2.541031in}}%
\pgfpathcurveto{\pgfqpoint{2.920579in}{2.532794in}}{\pgfqpoint{2.923851in}{2.524894in}}{\pgfqpoint{2.929675in}{2.519070in}}%
\pgfpathcurveto{\pgfqpoint{2.935499in}{2.513246in}}{\pgfqpoint{2.943399in}{2.509974in}}{\pgfqpoint{2.951636in}{2.509974in}}%
\pgfpathclose%
\pgfusepath{stroke,fill}%
\end{pgfscope}%
\begin{pgfscope}%
\pgfpathrectangle{\pgfqpoint{0.100000in}{0.212622in}}{\pgfqpoint{3.696000in}{3.696000in}}%
\pgfusepath{clip}%
\pgfsetbuttcap%
\pgfsetroundjoin%
\definecolor{currentfill}{rgb}{0.121569,0.466667,0.705882}%
\pgfsetfillcolor{currentfill}%
\pgfsetfillopacity{0.975143}%
\pgfsetlinewidth{1.003750pt}%
\definecolor{currentstroke}{rgb}{0.121569,0.466667,0.705882}%
\pgfsetstrokecolor{currentstroke}%
\pgfsetstrokeopacity{0.975143}%
\pgfsetdash{}{0pt}%
\pgfpathmoveto{\pgfqpoint{1.860055in}{2.088721in}}%
\pgfpathcurveto{\pgfqpoint{1.868291in}{2.088721in}}{\pgfqpoint{1.876191in}{2.091993in}}{\pgfqpoint{1.882015in}{2.097817in}}%
\pgfpathcurveto{\pgfqpoint{1.887839in}{2.103641in}}{\pgfqpoint{1.891111in}{2.111541in}}{\pgfqpoint{1.891111in}{2.119777in}}%
\pgfpathcurveto{\pgfqpoint{1.891111in}{2.128013in}}{\pgfqpoint{1.887839in}{2.135913in}}{\pgfqpoint{1.882015in}{2.141737in}}%
\pgfpathcurveto{\pgfqpoint{1.876191in}{2.147561in}}{\pgfqpoint{1.868291in}{2.150834in}}{\pgfqpoint{1.860055in}{2.150834in}}%
\pgfpathcurveto{\pgfqpoint{1.851818in}{2.150834in}}{\pgfqpoint{1.843918in}{2.147561in}}{\pgfqpoint{1.838094in}{2.141737in}}%
\pgfpathcurveto{\pgfqpoint{1.832270in}{2.135913in}}{\pgfqpoint{1.828998in}{2.128013in}}{\pgfqpoint{1.828998in}{2.119777in}}%
\pgfpathcurveto{\pgfqpoint{1.828998in}{2.111541in}}{\pgfqpoint{1.832270in}{2.103641in}}{\pgfqpoint{1.838094in}{2.097817in}}%
\pgfpathcurveto{\pgfqpoint{1.843918in}{2.091993in}}{\pgfqpoint{1.851818in}{2.088721in}}{\pgfqpoint{1.860055in}{2.088721in}}%
\pgfpathclose%
\pgfusepath{stroke,fill}%
\end{pgfscope}%
\begin{pgfscope}%
\pgfpathrectangle{\pgfqpoint{0.100000in}{0.212622in}}{\pgfqpoint{3.696000in}{3.696000in}}%
\pgfusepath{clip}%
\pgfsetbuttcap%
\pgfsetroundjoin%
\definecolor{currentfill}{rgb}{0.121569,0.466667,0.705882}%
\pgfsetfillcolor{currentfill}%
\pgfsetfillopacity{0.975268}%
\pgfsetlinewidth{1.003750pt}%
\definecolor{currentstroke}{rgb}{0.121569,0.466667,0.705882}%
\pgfsetstrokecolor{currentstroke}%
\pgfsetstrokeopacity{0.975268}%
\pgfsetdash{}{0pt}%
\pgfpathmoveto{\pgfqpoint{2.950465in}{2.508687in}}%
\pgfpathcurveto{\pgfqpoint{2.958701in}{2.508687in}}{\pgfqpoint{2.966601in}{2.511959in}}{\pgfqpoint{2.972425in}{2.517783in}}%
\pgfpathcurveto{\pgfqpoint{2.978249in}{2.523607in}}{\pgfqpoint{2.981522in}{2.531507in}}{\pgfqpoint{2.981522in}{2.539744in}}%
\pgfpathcurveto{\pgfqpoint{2.981522in}{2.547980in}}{\pgfqpoint{2.978249in}{2.555880in}}{\pgfqpoint{2.972425in}{2.561704in}}%
\pgfpathcurveto{\pgfqpoint{2.966601in}{2.567528in}}{\pgfqpoint{2.958701in}{2.570800in}}{\pgfqpoint{2.950465in}{2.570800in}}%
\pgfpathcurveto{\pgfqpoint{2.942229in}{2.570800in}}{\pgfqpoint{2.934329in}{2.567528in}}{\pgfqpoint{2.928505in}{2.561704in}}%
\pgfpathcurveto{\pgfqpoint{2.922681in}{2.555880in}}{\pgfqpoint{2.919409in}{2.547980in}}{\pgfqpoint{2.919409in}{2.539744in}}%
\pgfpathcurveto{\pgfqpoint{2.919409in}{2.531507in}}{\pgfqpoint{2.922681in}{2.523607in}}{\pgfqpoint{2.928505in}{2.517783in}}%
\pgfpathcurveto{\pgfqpoint{2.934329in}{2.511959in}}{\pgfqpoint{2.942229in}{2.508687in}}{\pgfqpoint{2.950465in}{2.508687in}}%
\pgfpathclose%
\pgfusepath{stroke,fill}%
\end{pgfscope}%
\begin{pgfscope}%
\pgfpathrectangle{\pgfqpoint{0.100000in}{0.212622in}}{\pgfqpoint{3.696000in}{3.696000in}}%
\pgfusepath{clip}%
\pgfsetbuttcap%
\pgfsetroundjoin%
\definecolor{currentfill}{rgb}{0.121569,0.466667,0.705882}%
\pgfsetfillcolor{currentfill}%
\pgfsetfillopacity{0.975428}%
\pgfsetlinewidth{1.003750pt}%
\definecolor{currentstroke}{rgb}{0.121569,0.466667,0.705882}%
\pgfsetstrokecolor{currentstroke}%
\pgfsetstrokeopacity{0.975428}%
\pgfsetdash{}{0pt}%
\pgfpathmoveto{\pgfqpoint{1.926836in}{2.088885in}}%
\pgfpathcurveto{\pgfqpoint{1.935073in}{2.088885in}}{\pgfqpoint{1.942973in}{2.092157in}}{\pgfqpoint{1.948797in}{2.097981in}}%
\pgfpathcurveto{\pgfqpoint{1.954621in}{2.103805in}}{\pgfqpoint{1.957893in}{2.111705in}}{\pgfqpoint{1.957893in}{2.119941in}}%
\pgfpathcurveto{\pgfqpoint{1.957893in}{2.128178in}}{\pgfqpoint{1.954621in}{2.136078in}}{\pgfqpoint{1.948797in}{2.141902in}}%
\pgfpathcurveto{\pgfqpoint{1.942973in}{2.147726in}}{\pgfqpoint{1.935073in}{2.150998in}}{\pgfqpoint{1.926836in}{2.150998in}}%
\pgfpathcurveto{\pgfqpoint{1.918600in}{2.150998in}}{\pgfqpoint{1.910700in}{2.147726in}}{\pgfqpoint{1.904876in}{2.141902in}}%
\pgfpathcurveto{\pgfqpoint{1.899052in}{2.136078in}}{\pgfqpoint{1.895780in}{2.128178in}}{\pgfqpoint{1.895780in}{2.119941in}}%
\pgfpathcurveto{\pgfqpoint{1.895780in}{2.111705in}}{\pgfqpoint{1.899052in}{2.103805in}}{\pgfqpoint{1.904876in}{2.097981in}}%
\pgfpathcurveto{\pgfqpoint{1.910700in}{2.092157in}}{\pgfqpoint{1.918600in}{2.088885in}}{\pgfqpoint{1.926836in}{2.088885in}}%
\pgfpathclose%
\pgfusepath{stroke,fill}%
\end{pgfscope}%
\begin{pgfscope}%
\pgfpathrectangle{\pgfqpoint{0.100000in}{0.212622in}}{\pgfqpoint{3.696000in}{3.696000in}}%
\pgfusepath{clip}%
\pgfsetbuttcap%
\pgfsetroundjoin%
\definecolor{currentfill}{rgb}{0.121569,0.466667,0.705882}%
\pgfsetfillcolor{currentfill}%
\pgfsetfillopacity{0.975659}%
\pgfsetlinewidth{1.003750pt}%
\definecolor{currentstroke}{rgb}{0.121569,0.466667,0.705882}%
\pgfsetstrokecolor{currentstroke}%
\pgfsetstrokeopacity{0.975659}%
\pgfsetdash{}{0pt}%
\pgfpathmoveto{\pgfqpoint{2.951747in}{2.509347in}}%
\pgfpathcurveto{\pgfqpoint{2.959983in}{2.509347in}}{\pgfqpoint{2.967883in}{2.512619in}}{\pgfqpoint{2.973707in}{2.518443in}}%
\pgfpathcurveto{\pgfqpoint{2.979531in}{2.524267in}}{\pgfqpoint{2.982804in}{2.532167in}}{\pgfqpoint{2.982804in}{2.540404in}}%
\pgfpathcurveto{\pgfqpoint{2.982804in}{2.548640in}}{\pgfqpoint{2.979531in}{2.556540in}}{\pgfqpoint{2.973707in}{2.562364in}}%
\pgfpathcurveto{\pgfqpoint{2.967883in}{2.568188in}}{\pgfqpoint{2.959983in}{2.571460in}}{\pgfqpoint{2.951747in}{2.571460in}}%
\pgfpathcurveto{\pgfqpoint{2.943511in}{2.571460in}}{\pgfqpoint{2.935611in}{2.568188in}}{\pgfqpoint{2.929787in}{2.562364in}}%
\pgfpathcurveto{\pgfqpoint{2.923963in}{2.556540in}}{\pgfqpoint{2.920691in}{2.548640in}}{\pgfqpoint{2.920691in}{2.540404in}}%
\pgfpathcurveto{\pgfqpoint{2.920691in}{2.532167in}}{\pgfqpoint{2.923963in}{2.524267in}}{\pgfqpoint{2.929787in}{2.518443in}}%
\pgfpathcurveto{\pgfqpoint{2.935611in}{2.512619in}}{\pgfqpoint{2.943511in}{2.509347in}}{\pgfqpoint{2.951747in}{2.509347in}}%
\pgfpathclose%
\pgfusepath{stroke,fill}%
\end{pgfscope}%
\begin{pgfscope}%
\pgfpathrectangle{\pgfqpoint{0.100000in}{0.212622in}}{\pgfqpoint{3.696000in}{3.696000in}}%
\pgfusepath{clip}%
\pgfsetbuttcap%
\pgfsetroundjoin%
\definecolor{currentfill}{rgb}{0.121569,0.466667,0.705882}%
\pgfsetfillcolor{currentfill}%
\pgfsetfillopacity{0.975953}%
\pgfsetlinewidth{1.003750pt}%
\definecolor{currentstroke}{rgb}{0.121569,0.466667,0.705882}%
\pgfsetstrokecolor{currentstroke}%
\pgfsetstrokeopacity{0.975953}%
\pgfsetdash{}{0pt}%
\pgfpathmoveto{\pgfqpoint{1.870263in}{2.088177in}}%
\pgfpathcurveto{\pgfqpoint{1.878500in}{2.088177in}}{\pgfqpoint{1.886400in}{2.091449in}}{\pgfqpoint{1.892224in}{2.097273in}}%
\pgfpathcurveto{\pgfqpoint{1.898048in}{2.103097in}}{\pgfqpoint{1.901320in}{2.110997in}}{\pgfqpoint{1.901320in}{2.119233in}}%
\pgfpathcurveto{\pgfqpoint{1.901320in}{2.127470in}}{\pgfqpoint{1.898048in}{2.135370in}}{\pgfqpoint{1.892224in}{2.141194in}}%
\pgfpathcurveto{\pgfqpoint{1.886400in}{2.147017in}}{\pgfqpoint{1.878500in}{2.150290in}}{\pgfqpoint{1.870263in}{2.150290in}}%
\pgfpathcurveto{\pgfqpoint{1.862027in}{2.150290in}}{\pgfqpoint{1.854127in}{2.147017in}}{\pgfqpoint{1.848303in}{2.141194in}}%
\pgfpathcurveto{\pgfqpoint{1.842479in}{2.135370in}}{\pgfqpoint{1.839207in}{2.127470in}}{\pgfqpoint{1.839207in}{2.119233in}}%
\pgfpathcurveto{\pgfqpoint{1.839207in}{2.110997in}}{\pgfqpoint{1.842479in}{2.103097in}}{\pgfqpoint{1.848303in}{2.097273in}}%
\pgfpathcurveto{\pgfqpoint{1.854127in}{2.091449in}}{\pgfqpoint{1.862027in}{2.088177in}}{\pgfqpoint{1.870263in}{2.088177in}}%
\pgfpathclose%
\pgfusepath{stroke,fill}%
\end{pgfscope}%
\begin{pgfscope}%
\pgfpathrectangle{\pgfqpoint{0.100000in}{0.212622in}}{\pgfqpoint{3.696000in}{3.696000in}}%
\pgfusepath{clip}%
\pgfsetbuttcap%
\pgfsetroundjoin%
\definecolor{currentfill}{rgb}{0.121569,0.466667,0.705882}%
\pgfsetfillcolor{currentfill}%
\pgfsetfillopacity{0.976187}%
\pgfsetlinewidth{1.003750pt}%
\definecolor{currentstroke}{rgb}{0.121569,0.466667,0.705882}%
\pgfsetstrokecolor{currentstroke}%
\pgfsetstrokeopacity{0.976187}%
\pgfsetdash{}{0pt}%
\pgfpathmoveto{\pgfqpoint{2.949051in}{2.506755in}}%
\pgfpathcurveto{\pgfqpoint{2.957287in}{2.506755in}}{\pgfqpoint{2.965187in}{2.510027in}}{\pgfqpoint{2.971011in}{2.515851in}}%
\pgfpathcurveto{\pgfqpoint{2.976835in}{2.521675in}}{\pgfqpoint{2.980107in}{2.529575in}}{\pgfqpoint{2.980107in}{2.537811in}}%
\pgfpathcurveto{\pgfqpoint{2.980107in}{2.546047in}}{\pgfqpoint{2.976835in}{2.553947in}}{\pgfqpoint{2.971011in}{2.559771in}}%
\pgfpathcurveto{\pgfqpoint{2.965187in}{2.565595in}}{\pgfqpoint{2.957287in}{2.568868in}}{\pgfqpoint{2.949051in}{2.568868in}}%
\pgfpathcurveto{\pgfqpoint{2.940814in}{2.568868in}}{\pgfqpoint{2.932914in}{2.565595in}}{\pgfqpoint{2.927091in}{2.559771in}}%
\pgfpathcurveto{\pgfqpoint{2.921267in}{2.553947in}}{\pgfqpoint{2.917994in}{2.546047in}}{\pgfqpoint{2.917994in}{2.537811in}}%
\pgfpathcurveto{\pgfqpoint{2.917994in}{2.529575in}}{\pgfqpoint{2.921267in}{2.521675in}}{\pgfqpoint{2.927091in}{2.515851in}}%
\pgfpathcurveto{\pgfqpoint{2.932914in}{2.510027in}}{\pgfqpoint{2.940814in}{2.506755in}}{\pgfqpoint{2.949051in}{2.506755in}}%
\pgfpathclose%
\pgfusepath{stroke,fill}%
\end{pgfscope}%
\begin{pgfscope}%
\pgfpathrectangle{\pgfqpoint{0.100000in}{0.212622in}}{\pgfqpoint{3.696000in}{3.696000in}}%
\pgfusepath{clip}%
\pgfsetbuttcap%
\pgfsetroundjoin%
\definecolor{currentfill}{rgb}{0.121569,0.466667,0.705882}%
\pgfsetfillcolor{currentfill}%
\pgfsetfillopacity{0.976531}%
\pgfsetlinewidth{1.003750pt}%
\definecolor{currentstroke}{rgb}{0.121569,0.466667,0.705882}%
\pgfsetstrokecolor{currentstroke}%
\pgfsetstrokeopacity{0.976531}%
\pgfsetdash{}{0pt}%
\pgfpathmoveto{\pgfqpoint{2.954099in}{2.515744in}}%
\pgfpathcurveto{\pgfqpoint{2.962335in}{2.515744in}}{\pgfqpoint{2.970235in}{2.519017in}}{\pgfqpoint{2.976059in}{2.524841in}}%
\pgfpathcurveto{\pgfqpoint{2.981883in}{2.530665in}}{\pgfqpoint{2.985155in}{2.538565in}}{\pgfqpoint{2.985155in}{2.546801in}}%
\pgfpathcurveto{\pgfqpoint{2.985155in}{2.555037in}}{\pgfqpoint{2.981883in}{2.562937in}}{\pgfqpoint{2.976059in}{2.568761in}}%
\pgfpathcurveto{\pgfqpoint{2.970235in}{2.574585in}}{\pgfqpoint{2.962335in}{2.577857in}}{\pgfqpoint{2.954099in}{2.577857in}}%
\pgfpathcurveto{\pgfqpoint{2.945863in}{2.577857in}}{\pgfqpoint{2.937962in}{2.574585in}}{\pgfqpoint{2.932139in}{2.568761in}}%
\pgfpathcurveto{\pgfqpoint{2.926315in}{2.562937in}}{\pgfqpoint{2.923042in}{2.555037in}}{\pgfqpoint{2.923042in}{2.546801in}}%
\pgfpathcurveto{\pgfqpoint{2.923042in}{2.538565in}}{\pgfqpoint{2.926315in}{2.530665in}}{\pgfqpoint{2.932139in}{2.524841in}}%
\pgfpathcurveto{\pgfqpoint{2.937962in}{2.519017in}}{\pgfqpoint{2.945863in}{2.515744in}}{\pgfqpoint{2.954099in}{2.515744in}}%
\pgfpathclose%
\pgfusepath{stroke,fill}%
\end{pgfscope}%
\begin{pgfscope}%
\pgfpathrectangle{\pgfqpoint{0.100000in}{0.212622in}}{\pgfqpoint{3.696000in}{3.696000in}}%
\pgfusepath{clip}%
\pgfsetbuttcap%
\pgfsetroundjoin%
\definecolor{currentfill}{rgb}{0.121569,0.466667,0.705882}%
\pgfsetfillcolor{currentfill}%
\pgfsetfillopacity{0.977362}%
\pgfsetlinewidth{1.003750pt}%
\definecolor{currentstroke}{rgb}{0.121569,0.466667,0.705882}%
\pgfsetstrokecolor{currentstroke}%
\pgfsetstrokeopacity{0.977362}%
\pgfsetdash{}{0pt}%
\pgfpathmoveto{\pgfqpoint{1.819162in}{2.052733in}}%
\pgfpathcurveto{\pgfqpoint{1.827398in}{2.052733in}}{\pgfqpoint{1.835298in}{2.056005in}}{\pgfqpoint{1.841122in}{2.061829in}}%
\pgfpathcurveto{\pgfqpoint{1.846946in}{2.067653in}}{\pgfqpoint{1.850219in}{2.075553in}}{\pgfqpoint{1.850219in}{2.083789in}}%
\pgfpathcurveto{\pgfqpoint{1.850219in}{2.092025in}}{\pgfqpoint{1.846946in}{2.099925in}}{\pgfqpoint{1.841122in}{2.105749in}}%
\pgfpathcurveto{\pgfqpoint{1.835298in}{2.111573in}}{\pgfqpoint{1.827398in}{2.114846in}}{\pgfqpoint{1.819162in}{2.114846in}}%
\pgfpathcurveto{\pgfqpoint{1.810926in}{2.114846in}}{\pgfqpoint{1.803026in}{2.111573in}}{\pgfqpoint{1.797202in}{2.105749in}}%
\pgfpathcurveto{\pgfqpoint{1.791378in}{2.099925in}}{\pgfqpoint{1.788106in}{2.092025in}}{\pgfqpoint{1.788106in}{2.083789in}}%
\pgfpathcurveto{\pgfqpoint{1.788106in}{2.075553in}}{\pgfqpoint{1.791378in}{2.067653in}}{\pgfqpoint{1.797202in}{2.061829in}}%
\pgfpathcurveto{\pgfqpoint{1.803026in}{2.056005in}}{\pgfqpoint{1.810926in}{2.052733in}}{\pgfqpoint{1.819162in}{2.052733in}}%
\pgfpathclose%
\pgfusepath{stroke,fill}%
\end{pgfscope}%
\begin{pgfscope}%
\pgfpathrectangle{\pgfqpoint{0.100000in}{0.212622in}}{\pgfqpoint{3.696000in}{3.696000in}}%
\pgfusepath{clip}%
\pgfsetbuttcap%
\pgfsetroundjoin%
\definecolor{currentfill}{rgb}{0.121569,0.466667,0.705882}%
\pgfsetfillcolor{currentfill}%
\pgfsetfillopacity{0.977367}%
\pgfsetlinewidth{1.003750pt}%
\definecolor{currentstroke}{rgb}{0.121569,0.466667,0.705882}%
\pgfsetstrokecolor{currentstroke}%
\pgfsetstrokeopacity{0.977367}%
\pgfsetdash{}{0pt}%
\pgfpathmoveto{\pgfqpoint{1.926500in}{2.085532in}}%
\pgfpathcurveto{\pgfqpoint{1.934737in}{2.085532in}}{\pgfqpoint{1.942637in}{2.088804in}}{\pgfqpoint{1.948460in}{2.094628in}}%
\pgfpathcurveto{\pgfqpoint{1.954284in}{2.100452in}}{\pgfqpoint{1.957557in}{2.108352in}}{\pgfqpoint{1.957557in}{2.116588in}}%
\pgfpathcurveto{\pgfqpoint{1.957557in}{2.124825in}}{\pgfqpoint{1.954284in}{2.132725in}}{\pgfqpoint{1.948460in}{2.138548in}}%
\pgfpathcurveto{\pgfqpoint{1.942637in}{2.144372in}}{\pgfqpoint{1.934737in}{2.147645in}}{\pgfqpoint{1.926500in}{2.147645in}}%
\pgfpathcurveto{\pgfqpoint{1.918264in}{2.147645in}}{\pgfqpoint{1.910364in}{2.144372in}}{\pgfqpoint{1.904540in}{2.138548in}}%
\pgfpathcurveto{\pgfqpoint{1.898716in}{2.132725in}}{\pgfqpoint{1.895444in}{2.124825in}}{\pgfqpoint{1.895444in}{2.116588in}}%
\pgfpathcurveto{\pgfqpoint{1.895444in}{2.108352in}}{\pgfqpoint{1.898716in}{2.100452in}}{\pgfqpoint{1.904540in}{2.094628in}}%
\pgfpathcurveto{\pgfqpoint{1.910364in}{2.088804in}}{\pgfqpoint{1.918264in}{2.085532in}}{\pgfqpoint{1.926500in}{2.085532in}}%
\pgfpathclose%
\pgfusepath{stroke,fill}%
\end{pgfscope}%
\begin{pgfscope}%
\pgfpathrectangle{\pgfqpoint{0.100000in}{0.212622in}}{\pgfqpoint{3.696000in}{3.696000in}}%
\pgfusepath{clip}%
\pgfsetbuttcap%
\pgfsetroundjoin%
\definecolor{currentfill}{rgb}{0.121569,0.466667,0.705882}%
\pgfsetfillcolor{currentfill}%
\pgfsetfillopacity{0.978797}%
\pgfsetlinewidth{1.003750pt}%
\definecolor{currentstroke}{rgb}{0.121569,0.466667,0.705882}%
\pgfsetstrokecolor{currentstroke}%
\pgfsetstrokeopacity{0.978797}%
\pgfsetdash{}{0pt}%
\pgfpathmoveto{\pgfqpoint{1.856735in}{2.072128in}}%
\pgfpathcurveto{\pgfqpoint{1.864971in}{2.072128in}}{\pgfqpoint{1.872871in}{2.075400in}}{\pgfqpoint{1.878695in}{2.081224in}}%
\pgfpathcurveto{\pgfqpoint{1.884519in}{2.087048in}}{\pgfqpoint{1.887791in}{2.094948in}}{\pgfqpoint{1.887791in}{2.103185in}}%
\pgfpathcurveto{\pgfqpoint{1.887791in}{2.111421in}}{\pgfqpoint{1.884519in}{2.119321in}}{\pgfqpoint{1.878695in}{2.125145in}}%
\pgfpathcurveto{\pgfqpoint{1.872871in}{2.130969in}}{\pgfqpoint{1.864971in}{2.134241in}}{\pgfqpoint{1.856735in}{2.134241in}}%
\pgfpathcurveto{\pgfqpoint{1.848499in}{2.134241in}}{\pgfqpoint{1.840599in}{2.130969in}}{\pgfqpoint{1.834775in}{2.125145in}}%
\pgfpathcurveto{\pgfqpoint{1.828951in}{2.119321in}}{\pgfqpoint{1.825678in}{2.111421in}}{\pgfqpoint{1.825678in}{2.103185in}}%
\pgfpathcurveto{\pgfqpoint{1.825678in}{2.094948in}}{\pgfqpoint{1.828951in}{2.087048in}}{\pgfqpoint{1.834775in}{2.081224in}}%
\pgfpathcurveto{\pgfqpoint{1.840599in}{2.075400in}}{\pgfqpoint{1.848499in}{2.072128in}}{\pgfqpoint{1.856735in}{2.072128in}}%
\pgfpathclose%
\pgfusepath{stroke,fill}%
\end{pgfscope}%
\begin{pgfscope}%
\pgfpathrectangle{\pgfqpoint{0.100000in}{0.212622in}}{\pgfqpoint{3.696000in}{3.696000in}}%
\pgfusepath{clip}%
\pgfsetbuttcap%
\pgfsetroundjoin%
\definecolor{currentfill}{rgb}{0.121569,0.466667,0.705882}%
\pgfsetfillcolor{currentfill}%
\pgfsetfillopacity{0.983602}%
\pgfsetlinewidth{1.003750pt}%
\definecolor{currentstroke}{rgb}{0.121569,0.466667,0.705882}%
\pgfsetstrokecolor{currentstroke}%
\pgfsetstrokeopacity{0.983602}%
\pgfsetdash{}{0pt}%
\pgfpathmoveto{\pgfqpoint{1.899301in}{2.064587in}}%
\pgfpathcurveto{\pgfqpoint{1.907537in}{2.064587in}}{\pgfqpoint{1.915437in}{2.067860in}}{\pgfqpoint{1.921261in}{2.073684in}}%
\pgfpathcurveto{\pgfqpoint{1.927085in}{2.079507in}}{\pgfqpoint{1.930358in}{2.087408in}}{\pgfqpoint{1.930358in}{2.095644in}}%
\pgfpathcurveto{\pgfqpoint{1.930358in}{2.103880in}}{\pgfqpoint{1.927085in}{2.111780in}}{\pgfqpoint{1.921261in}{2.117604in}}%
\pgfpathcurveto{\pgfqpoint{1.915437in}{2.123428in}}{\pgfqpoint{1.907537in}{2.126700in}}{\pgfqpoint{1.899301in}{2.126700in}}%
\pgfpathcurveto{\pgfqpoint{1.891065in}{2.126700in}}{\pgfqpoint{1.883165in}{2.123428in}}{\pgfqpoint{1.877341in}{2.117604in}}%
\pgfpathcurveto{\pgfqpoint{1.871517in}{2.111780in}}{\pgfqpoint{1.868245in}{2.103880in}}{\pgfqpoint{1.868245in}{2.095644in}}%
\pgfpathcurveto{\pgfqpoint{1.868245in}{2.087408in}}{\pgfqpoint{1.871517in}{2.079507in}}{\pgfqpoint{1.877341in}{2.073684in}}%
\pgfpathcurveto{\pgfqpoint{1.883165in}{2.067860in}}{\pgfqpoint{1.891065in}{2.064587in}}{\pgfqpoint{1.899301in}{2.064587in}}%
\pgfpathclose%
\pgfusepath{stroke,fill}%
\end{pgfscope}%
\begin{pgfscope}%
\pgfpathrectangle{\pgfqpoint{0.100000in}{0.212622in}}{\pgfqpoint{3.696000in}{3.696000in}}%
\pgfusepath{clip}%
\pgfsetbuttcap%
\pgfsetroundjoin%
\definecolor{currentfill}{rgb}{0.121569,0.466667,0.705882}%
\pgfsetfillcolor{currentfill}%
\pgfsetfillopacity{0.984795}%
\pgfsetlinewidth{1.003750pt}%
\definecolor{currentstroke}{rgb}{0.121569,0.466667,0.705882}%
\pgfsetstrokecolor{currentstroke}%
\pgfsetstrokeopacity{0.984795}%
\pgfsetdash{}{0pt}%
\pgfpathmoveto{\pgfqpoint{1.816717in}{2.042029in}}%
\pgfpathcurveto{\pgfqpoint{1.824953in}{2.042029in}}{\pgfqpoint{1.832853in}{2.045301in}}{\pgfqpoint{1.838677in}{2.051125in}}%
\pgfpathcurveto{\pgfqpoint{1.844501in}{2.056949in}}{\pgfqpoint{1.847773in}{2.064849in}}{\pgfqpoint{1.847773in}{2.073085in}}%
\pgfpathcurveto{\pgfqpoint{1.847773in}{2.081322in}}{\pgfqpoint{1.844501in}{2.089222in}}{\pgfqpoint{1.838677in}{2.095045in}}%
\pgfpathcurveto{\pgfqpoint{1.832853in}{2.100869in}}{\pgfqpoint{1.824953in}{2.104142in}}{\pgfqpoint{1.816717in}{2.104142in}}%
\pgfpathcurveto{\pgfqpoint{1.808480in}{2.104142in}}{\pgfqpoint{1.800580in}{2.100869in}}{\pgfqpoint{1.794756in}{2.095045in}}%
\pgfpathcurveto{\pgfqpoint{1.788932in}{2.089222in}}{\pgfqpoint{1.785660in}{2.081322in}}{\pgfqpoint{1.785660in}{2.073085in}}%
\pgfpathcurveto{\pgfqpoint{1.785660in}{2.064849in}}{\pgfqpoint{1.788932in}{2.056949in}}{\pgfqpoint{1.794756in}{2.051125in}}%
\pgfpathcurveto{\pgfqpoint{1.800580in}{2.045301in}}{\pgfqpoint{1.808480in}{2.042029in}}{\pgfqpoint{1.816717in}{2.042029in}}%
\pgfpathclose%
\pgfusepath{stroke,fill}%
\end{pgfscope}%
\begin{pgfscope}%
\pgfpathrectangle{\pgfqpoint{0.100000in}{0.212622in}}{\pgfqpoint{3.696000in}{3.696000in}}%
\pgfusepath{clip}%
\pgfsetbuttcap%
\pgfsetroundjoin%
\definecolor{currentfill}{rgb}{0.121569,0.466667,0.705882}%
\pgfsetfillcolor{currentfill}%
\pgfsetfillopacity{0.988793}%
\pgfsetlinewidth{1.003750pt}%
\definecolor{currentstroke}{rgb}{0.121569,0.466667,0.705882}%
\pgfsetstrokecolor{currentstroke}%
\pgfsetstrokeopacity{0.988793}%
\pgfsetdash{}{0pt}%
\pgfpathmoveto{\pgfqpoint{1.856826in}{2.068472in}}%
\pgfpathcurveto{\pgfqpoint{1.865062in}{2.068472in}}{\pgfqpoint{1.872962in}{2.071744in}}{\pgfqpoint{1.878786in}{2.077568in}}%
\pgfpathcurveto{\pgfqpoint{1.884610in}{2.083392in}}{\pgfqpoint{1.887883in}{2.091292in}}{\pgfqpoint{1.887883in}{2.099528in}}%
\pgfpathcurveto{\pgfqpoint{1.887883in}{2.107764in}}{\pgfqpoint{1.884610in}{2.115665in}}{\pgfqpoint{1.878786in}{2.121488in}}%
\pgfpathcurveto{\pgfqpoint{1.872962in}{2.127312in}}{\pgfqpoint{1.865062in}{2.130585in}}{\pgfqpoint{1.856826in}{2.130585in}}%
\pgfpathcurveto{\pgfqpoint{1.848590in}{2.130585in}}{\pgfqpoint{1.840690in}{2.127312in}}{\pgfqpoint{1.834866in}{2.121488in}}%
\pgfpathcurveto{\pgfqpoint{1.829042in}{2.115665in}}{\pgfqpoint{1.825770in}{2.107764in}}{\pgfqpoint{1.825770in}{2.099528in}}%
\pgfpathcurveto{\pgfqpoint{1.825770in}{2.091292in}}{\pgfqpoint{1.829042in}{2.083392in}}{\pgfqpoint{1.834866in}{2.077568in}}%
\pgfpathcurveto{\pgfqpoint{1.840690in}{2.071744in}}{\pgfqpoint{1.848590in}{2.068472in}}{\pgfqpoint{1.856826in}{2.068472in}}%
\pgfpathclose%
\pgfusepath{stroke,fill}%
\end{pgfscope}%
\begin{pgfscope}%
\pgfpathrectangle{\pgfqpoint{0.100000in}{0.212622in}}{\pgfqpoint{3.696000in}{3.696000in}}%
\pgfusepath{clip}%
\pgfsetbuttcap%
\pgfsetroundjoin%
\definecolor{currentfill}{rgb}{0.121569,0.466667,0.705882}%
\pgfsetfillcolor{currentfill}%
\pgfsetfillopacity{0.988996}%
\pgfsetlinewidth{1.003750pt}%
\definecolor{currentstroke}{rgb}{0.121569,0.466667,0.705882}%
\pgfsetstrokecolor{currentstroke}%
\pgfsetstrokeopacity{0.988996}%
\pgfsetdash{}{0pt}%
\pgfpathmoveto{\pgfqpoint{1.844910in}{2.056610in}}%
\pgfpathcurveto{\pgfqpoint{1.853147in}{2.056610in}}{\pgfqpoint{1.861047in}{2.059882in}}{\pgfqpoint{1.866871in}{2.065706in}}%
\pgfpathcurveto{\pgfqpoint{1.872694in}{2.071530in}}{\pgfqpoint{1.875967in}{2.079430in}}{\pgfqpoint{1.875967in}{2.087667in}}%
\pgfpathcurveto{\pgfqpoint{1.875967in}{2.095903in}}{\pgfqpoint{1.872694in}{2.103803in}}{\pgfqpoint{1.866871in}{2.109627in}}%
\pgfpathcurveto{\pgfqpoint{1.861047in}{2.115451in}}{\pgfqpoint{1.853147in}{2.118723in}}{\pgfqpoint{1.844910in}{2.118723in}}%
\pgfpathcurveto{\pgfqpoint{1.836674in}{2.118723in}}{\pgfqpoint{1.828774in}{2.115451in}}{\pgfqpoint{1.822950in}{2.109627in}}%
\pgfpathcurveto{\pgfqpoint{1.817126in}{2.103803in}}{\pgfqpoint{1.813854in}{2.095903in}}{\pgfqpoint{1.813854in}{2.087667in}}%
\pgfpathcurveto{\pgfqpoint{1.813854in}{2.079430in}}{\pgfqpoint{1.817126in}{2.071530in}}{\pgfqpoint{1.822950in}{2.065706in}}%
\pgfpathcurveto{\pgfqpoint{1.828774in}{2.059882in}}{\pgfqpoint{1.836674in}{2.056610in}}{\pgfqpoint{1.844910in}{2.056610in}}%
\pgfpathclose%
\pgfusepath{stroke,fill}%
\end{pgfscope}%
\begin{pgfscope}%
\pgfpathrectangle{\pgfqpoint{0.100000in}{0.212622in}}{\pgfqpoint{3.696000in}{3.696000in}}%
\pgfusepath{clip}%
\pgfsetbuttcap%
\pgfsetroundjoin%
\definecolor{currentfill}{rgb}{0.121569,0.466667,0.705882}%
\pgfsetfillcolor{currentfill}%
\pgfsetfillopacity{0.996073}%
\pgfsetlinewidth{1.003750pt}%
\definecolor{currentstroke}{rgb}{0.121569,0.466667,0.705882}%
\pgfsetstrokecolor{currentstroke}%
\pgfsetstrokeopacity{0.996073}%
\pgfsetdash{}{0pt}%
\pgfpathmoveto{\pgfqpoint{1.857446in}{2.035338in}}%
\pgfpathcurveto{\pgfqpoint{1.865683in}{2.035338in}}{\pgfqpoint{1.873583in}{2.038610in}}{\pgfqpoint{1.879407in}{2.044434in}}%
\pgfpathcurveto{\pgfqpoint{1.885231in}{2.050258in}}{\pgfqpoint{1.888503in}{2.058158in}}{\pgfqpoint{1.888503in}{2.066394in}}%
\pgfpathcurveto{\pgfqpoint{1.888503in}{2.074631in}}{\pgfqpoint{1.885231in}{2.082531in}}{\pgfqpoint{1.879407in}{2.088355in}}%
\pgfpathcurveto{\pgfqpoint{1.873583in}{2.094179in}}{\pgfqpoint{1.865683in}{2.097451in}}{\pgfqpoint{1.857446in}{2.097451in}}%
\pgfpathcurveto{\pgfqpoint{1.849210in}{2.097451in}}{\pgfqpoint{1.841310in}{2.094179in}}{\pgfqpoint{1.835486in}{2.088355in}}%
\pgfpathcurveto{\pgfqpoint{1.829662in}{2.082531in}}{\pgfqpoint{1.826390in}{2.074631in}}{\pgfqpoint{1.826390in}{2.066394in}}%
\pgfpathcurveto{\pgfqpoint{1.826390in}{2.058158in}}{\pgfqpoint{1.829662in}{2.050258in}}{\pgfqpoint{1.835486in}{2.044434in}}%
\pgfpathcurveto{\pgfqpoint{1.841310in}{2.038610in}}{\pgfqpoint{1.849210in}{2.035338in}}{\pgfqpoint{1.857446in}{2.035338in}}%
\pgfpathclose%
\pgfusepath{stroke,fill}%
\end{pgfscope}%
\begin{pgfscope}%
\pgfpathrectangle{\pgfqpoint{0.100000in}{0.212622in}}{\pgfqpoint{3.696000in}{3.696000in}}%
\pgfusepath{clip}%
\pgfsetbuttcap%
\pgfsetroundjoin%
\definecolor{currentfill}{rgb}{0.121569,0.466667,0.705882}%
\pgfsetfillcolor{currentfill}%
\pgfsetlinewidth{1.003750pt}%
\definecolor{currentstroke}{rgb}{0.121569,0.466667,0.705882}%
\pgfsetstrokecolor{currentstroke}%
\pgfsetdash{}{0pt}%
\pgfpathmoveto{\pgfqpoint{1.841454in}{2.044049in}}%
\pgfpathcurveto{\pgfqpoint{1.849690in}{2.044049in}}{\pgfqpoint{1.857591in}{2.047321in}}{\pgfqpoint{1.863414in}{2.053145in}}%
\pgfpathcurveto{\pgfqpoint{1.869238in}{2.058969in}}{\pgfqpoint{1.872511in}{2.066869in}}{\pgfqpoint{1.872511in}{2.075105in}}%
\pgfpathcurveto{\pgfqpoint{1.872511in}{2.083342in}}{\pgfqpoint{1.869238in}{2.091242in}}{\pgfqpoint{1.863414in}{2.097066in}}%
\pgfpathcurveto{\pgfqpoint{1.857591in}{2.102889in}}{\pgfqpoint{1.849690in}{2.106162in}}{\pgfqpoint{1.841454in}{2.106162in}}%
\pgfpathcurveto{\pgfqpoint{1.833218in}{2.106162in}}{\pgfqpoint{1.825318in}{2.102889in}}{\pgfqpoint{1.819494in}{2.097066in}}%
\pgfpathcurveto{\pgfqpoint{1.813670in}{2.091242in}}{\pgfqpoint{1.810398in}{2.083342in}}{\pgfqpoint{1.810398in}{2.075105in}}%
\pgfpathcurveto{\pgfqpoint{1.810398in}{2.066869in}}{\pgfqpoint{1.813670in}{2.058969in}}{\pgfqpoint{1.819494in}{2.053145in}}%
\pgfpathcurveto{\pgfqpoint{1.825318in}{2.047321in}}{\pgfqpoint{1.833218in}{2.044049in}}{\pgfqpoint{1.841454in}{2.044049in}}%
\pgfpathclose%
\pgfusepath{stroke,fill}%
\end{pgfscope}%
\begin{pgfscope}%
\pgfsetbuttcap%
\pgfsetmiterjoin%
\definecolor{currentfill}{rgb}{1.000000,1.000000,1.000000}%
\pgfsetfillcolor{currentfill}%
\pgfsetfillopacity{0.800000}%
\pgfsetlinewidth{1.003750pt}%
\definecolor{currentstroke}{rgb}{0.800000,0.800000,0.800000}%
\pgfsetstrokecolor{currentstroke}%
\pgfsetstrokeopacity{0.800000}%
\pgfsetdash{}{0pt}%
\pgfpathmoveto{\pgfqpoint{2.104889in}{3.410289in}}%
\pgfpathlineto{\pgfqpoint{3.698778in}{3.410289in}}%
\pgfpathquadraticcurveto{\pgfqpoint{3.726556in}{3.410289in}}{\pgfqpoint{3.726556in}{3.438067in}}%
\pgfpathlineto{\pgfqpoint{3.726556in}{3.811400in}}%
\pgfpathquadraticcurveto{\pgfqpoint{3.726556in}{3.839178in}}{\pgfqpoint{3.698778in}{3.839178in}}%
\pgfpathlineto{\pgfqpoint{2.104889in}{3.839178in}}%
\pgfpathquadraticcurveto{\pgfqpoint{2.077111in}{3.839178in}}{\pgfqpoint{2.077111in}{3.811400in}}%
\pgfpathlineto{\pgfqpoint{2.077111in}{3.438067in}}%
\pgfpathquadraticcurveto{\pgfqpoint{2.077111in}{3.410289in}}{\pgfqpoint{2.104889in}{3.410289in}}%
\pgfpathclose%
\pgfusepath{stroke,fill}%
\end{pgfscope}%
\begin{pgfscope}%
\pgfsetrectcap%
\pgfsetroundjoin%
\pgfsetlinewidth{1.505625pt}%
\definecolor{currentstroke}{rgb}{0.121569,0.466667,0.705882}%
\pgfsetstrokecolor{currentstroke}%
\pgfsetdash{}{0pt}%
\pgfpathmoveto{\pgfqpoint{2.132667in}{3.735011in}}%
\pgfpathlineto{\pgfqpoint{2.410444in}{3.735011in}}%
\pgfusepath{stroke}%
\end{pgfscope}%
\begin{pgfscope}%
\definecolor{textcolor}{rgb}{0.000000,0.000000,0.000000}%
\pgfsetstrokecolor{textcolor}%
\pgfsetfillcolor{textcolor}%
\pgftext[x=2.521555in,y=3.686400in,left,base]{\color{textcolor}\rmfamily\fontsize{10.000000}{12.000000}\selectfont Ground truth}%
\end{pgfscope}%
\begin{pgfscope}%
\pgfsetbuttcap%
\pgfsetroundjoin%
\definecolor{currentfill}{rgb}{0.121569,0.466667,0.705882}%
\pgfsetfillcolor{currentfill}%
\pgfsetlinewidth{1.003750pt}%
\definecolor{currentstroke}{rgb}{0.121569,0.466667,0.705882}%
\pgfsetstrokecolor{currentstroke}%
\pgfsetdash{}{0pt}%
\pgfsys@defobject{currentmarker}{\pgfqpoint{-0.031056in}{-0.031056in}}{\pgfqpoint{0.031056in}{0.031056in}}{%
\pgfpathmoveto{\pgfqpoint{0.000000in}{-0.031056in}}%
\pgfpathcurveto{\pgfqpoint{0.008236in}{-0.031056in}}{\pgfqpoint{0.016136in}{-0.027784in}}{\pgfqpoint{0.021960in}{-0.021960in}}%
\pgfpathcurveto{\pgfqpoint{0.027784in}{-0.016136in}}{\pgfqpoint{0.031056in}{-0.008236in}}{\pgfqpoint{0.031056in}{0.000000in}}%
\pgfpathcurveto{\pgfqpoint{0.031056in}{0.008236in}}{\pgfqpoint{0.027784in}{0.016136in}}{\pgfqpoint{0.021960in}{0.021960in}}%
\pgfpathcurveto{\pgfqpoint{0.016136in}{0.027784in}}{\pgfqpoint{0.008236in}{0.031056in}}{\pgfqpoint{0.000000in}{0.031056in}}%
\pgfpathcurveto{\pgfqpoint{-0.008236in}{0.031056in}}{\pgfqpoint{-0.016136in}{0.027784in}}{\pgfqpoint{-0.021960in}{0.021960in}}%
\pgfpathcurveto{\pgfqpoint{-0.027784in}{0.016136in}}{\pgfqpoint{-0.031056in}{0.008236in}}{\pgfqpoint{-0.031056in}{0.000000in}}%
\pgfpathcurveto{\pgfqpoint{-0.031056in}{-0.008236in}}{\pgfqpoint{-0.027784in}{-0.016136in}}{\pgfqpoint{-0.021960in}{-0.021960in}}%
\pgfpathcurveto{\pgfqpoint{-0.016136in}{-0.027784in}}{\pgfqpoint{-0.008236in}{-0.031056in}}{\pgfqpoint{0.000000in}{-0.031056in}}%
\pgfpathclose%
\pgfusepath{stroke,fill}%
}%
\begin{pgfscope}%
\pgfsys@transformshift{2.271555in}{3.529248in}%
\pgfsys@useobject{currentmarker}{}%
\end{pgfscope}%
\end{pgfscope}%
\begin{pgfscope}%
\definecolor{textcolor}{rgb}{0.000000,0.000000,0.000000}%
\pgfsetstrokecolor{textcolor}%
\pgfsetfillcolor{textcolor}%
\pgftext[x=2.521555in,y=3.492789in,left,base]{\color{textcolor}\rmfamily\fontsize{10.000000}{12.000000}\selectfont Estimated position}%
\end{pgfscope}%
\end{pgfpicture}%
\makeatother%
\endgroup%
}
%         \caption{Mahony's 3D position estimation had the lowest turn error for the 16-meter line experiment.}
%         \label{fig:line16_3D}
%     \end{subfigure}
%     \caption{Position estimation by the best performing algorithms in the 16-meter line experiment.}
%     \label{fig:line16}
% \end{figure}

% \subsubsection{28 meter}

% For the 28-meter line experiment, the Complementary algorithm which had the lowest displacement error with an average of 0.52 meters (1.85\% of error margin), and SAAM with an average of 4.23 meters of turn error (15.09\% of error margin).

% \begin{figure}[!h]
%     \centering
%     \begin{table}[H]
    \begin{center}
    \resizebox{1\linewidth}{!}{

        \begin{tabular}[t]{lcccc}
            \hline
            Algorithm                   & Displacement Error[$m$] & Displacement Error[\%]      & Turn Error[$m$]  & Turn Error[\%]             \\
            \hline 
            AngularRate            & 1.29  & 4.61 & 6.08 & 21.73              \\            AQUA            & 8.66  & 30.93 & 14.77 & 52.75              \\            Complementary            & 0.52  & 1.85 & 4.31 & 15.41              \\            Davenport            & 0.73  & 2.60 & 5.47 & 19.53              \\            EKF            & 0.76  & 2.73 & 4.35 & 15.55              \\            FAMC            & 0.50  & 1.80 & 4.33 & 15.45              \\            FLAE            & 0.73  & 2.59 & 5.47 & 19.53              \\            Fourati            & 1.07  & 3.81 & 6.21 & 22.18              \\            Madgwick            & 0.55  & 1.96 & 4.29 & 15.32              \\            Mahony            & 0.53  & 1.88 & 4.19 & 14.98              \\            OLEQ            & 0.69  & 2.47 & 5.35 & 19.11              \\            QUEST            & 3.08  & 11.01 & 11.72 & 41.87              \\            ROLEQ            & 0.83  & 2.95 & 5.39 & 19.24              \\            SAAM            & 0.51  & 1.81 & 4.23 & 15.09              \\            Tilt            & 0.51  & 1.81 & 4.23 & 15.09              \\
            \hline
            Average & 1.40 & 4.99 & 6.03 & 21.52
        \end{tabular}
        }
        \caption{Accelerometer Specifications. }
        \label{tab:accelerometer_specification}
    \end{center}
\end{table}
% \end{figure}

% \begin{figure}[!h]
%     \centering
%     \begin{subfigure}{0.49\textwidth}
%         \centering
%         \resizebox{1\linewidth}{!}{%% Creator: Matplotlib, PGF backend
%%
%% To include the figure in your LaTeX document, write
%%   \input{<filename>.pgf}
%%
%% Make sure the required packages are loaded in your preamble
%%   \usepackage{pgf}
%%
%% and, on pdftex
%%   \usepackage[utf8]{inputenc}\DeclareUnicodeCharacter{2212}{-}
%%
%% or, on luatex and xetex
%%   \usepackage{unicode-math}
%%
%% Figures using additional raster images can only be included by \input if
%% they are in the same directory as the main LaTeX file. For loading figures
%% from other directories you can use the `import` package
%%   \usepackage{import}
%%
%% and then include the figures with
%%   \import{<path to file>}{<filename>.pgf}
%%
%% Matplotlib used the following preamble
%%   \usepackage{fontspec}
%%
\begingroup%
\makeatletter%
\begin{pgfpicture}%
\pgfpathrectangle{\pgfpointorigin}{\pgfqpoint{5.698611in}{4.311000in}}%
\pgfusepath{use as bounding box, clip}%
\begin{pgfscope}%
\pgfsetbuttcap%
\pgfsetmiterjoin%
\definecolor{currentfill}{rgb}{1.000000,1.000000,1.000000}%
\pgfsetfillcolor{currentfill}%
\pgfsetlinewidth{0.000000pt}%
\definecolor{currentstroke}{rgb}{1.000000,1.000000,1.000000}%
\pgfsetstrokecolor{currentstroke}%
\pgfsetdash{}{0pt}%
\pgfpathmoveto{\pgfqpoint{0.000000in}{0.000000in}}%
\pgfpathlineto{\pgfqpoint{5.698611in}{0.000000in}}%
\pgfpathlineto{\pgfqpoint{5.698611in}{4.311000in}}%
\pgfpathlineto{\pgfqpoint{0.000000in}{4.311000in}}%
\pgfpathclose%
\pgfusepath{fill}%
\end{pgfscope}%
\begin{pgfscope}%
\pgfsetbuttcap%
\pgfsetmiterjoin%
\definecolor{currentfill}{rgb}{1.000000,1.000000,1.000000}%
\pgfsetfillcolor{currentfill}%
\pgfsetlinewidth{0.000000pt}%
\definecolor{currentstroke}{rgb}{0.000000,0.000000,0.000000}%
\pgfsetstrokecolor{currentstroke}%
\pgfsetstrokeopacity{0.000000}%
\pgfsetdash{}{0pt}%
\pgfpathmoveto{\pgfqpoint{0.638611in}{0.515000in}}%
\pgfpathlineto{\pgfqpoint{5.598611in}{0.515000in}}%
\pgfpathlineto{\pgfqpoint{5.598611in}{4.211000in}}%
\pgfpathlineto{\pgfqpoint{0.638611in}{4.211000in}}%
\pgfpathclose%
\pgfusepath{fill}%
\end{pgfscope}%
\begin{pgfscope}%
\pgfpathrectangle{\pgfqpoint{0.638611in}{0.515000in}}{\pgfqpoint{4.960000in}{3.696000in}}%
\pgfusepath{clip}%
\pgfsetbuttcap%
\pgfsetroundjoin%
\definecolor{currentfill}{rgb}{0.121569,0.466667,0.705882}%
\pgfsetfillcolor{currentfill}%
\pgfsetlinewidth{1.003750pt}%
\definecolor{currentstroke}{rgb}{0.121569,0.466667,0.705882}%
\pgfsetstrokecolor{currentstroke}%
\pgfsetdash{}{0pt}%
\pgfsys@defobject{currentmarker}{\pgfqpoint{-0.041667in}{-0.041667in}}{\pgfqpoint{0.041667in}{0.041667in}}{%
\pgfpathmoveto{\pgfqpoint{0.000000in}{-0.041667in}}%
\pgfpathcurveto{\pgfqpoint{0.011050in}{-0.041667in}}{\pgfqpoint{0.021649in}{-0.037276in}}{\pgfqpoint{0.029463in}{-0.029463in}}%
\pgfpathcurveto{\pgfqpoint{0.037276in}{-0.021649in}}{\pgfqpoint{0.041667in}{-0.011050in}}{\pgfqpoint{0.041667in}{0.000000in}}%
\pgfpathcurveto{\pgfqpoint{0.041667in}{0.011050in}}{\pgfqpoint{0.037276in}{0.021649in}}{\pgfqpoint{0.029463in}{0.029463in}}%
\pgfpathcurveto{\pgfqpoint{0.021649in}{0.037276in}}{\pgfqpoint{0.011050in}{0.041667in}}{\pgfqpoint{0.000000in}{0.041667in}}%
\pgfpathcurveto{\pgfqpoint{-0.011050in}{0.041667in}}{\pgfqpoint{-0.021649in}{0.037276in}}{\pgfqpoint{-0.029463in}{0.029463in}}%
\pgfpathcurveto{\pgfqpoint{-0.037276in}{0.021649in}}{\pgfqpoint{-0.041667in}{0.011050in}}{\pgfqpoint{-0.041667in}{0.000000in}}%
\pgfpathcurveto{\pgfqpoint{-0.041667in}{-0.011050in}}{\pgfqpoint{-0.037276in}{-0.021649in}}{\pgfqpoint{-0.029463in}{-0.029463in}}%
\pgfpathcurveto{\pgfqpoint{-0.021649in}{-0.037276in}}{\pgfqpoint{-0.011050in}{-0.041667in}}{\pgfqpoint{0.000000in}{-0.041667in}}%
\pgfpathclose%
\pgfusepath{stroke,fill}%
}%
\begin{pgfscope}%
\pgfsys@transformshift{0.864072in}{2.390003in}%
\pgfsys@useobject{currentmarker}{}%
\end{pgfscope}%
\begin{pgfscope}%
\pgfsys@transformshift{0.864076in}{2.390000in}%
\pgfsys@useobject{currentmarker}{}%
\end{pgfscope}%
\begin{pgfscope}%
\pgfsys@transformshift{0.864079in}{2.390000in}%
\pgfsys@useobject{currentmarker}{}%
\end{pgfscope}%
\begin{pgfscope}%
\pgfsys@transformshift{0.864080in}{2.390000in}%
\pgfsys@useobject{currentmarker}{}%
\end{pgfscope}%
\begin{pgfscope}%
\pgfsys@transformshift{0.864081in}{2.390000in}%
\pgfsys@useobject{currentmarker}{}%
\end{pgfscope}%
\begin{pgfscope}%
\pgfsys@transformshift{0.864081in}{2.390000in}%
\pgfsys@useobject{currentmarker}{}%
\end{pgfscope}%
\begin{pgfscope}%
\pgfsys@transformshift{0.864082in}{2.390000in}%
\pgfsys@useobject{currentmarker}{}%
\end{pgfscope}%
\begin{pgfscope}%
\pgfsys@transformshift{0.864815in}{2.390024in}%
\pgfsys@useobject{currentmarker}{}%
\end{pgfscope}%
\begin{pgfscope}%
\pgfsys@transformshift{0.866170in}{2.390066in}%
\pgfsys@useobject{currentmarker}{}%
\end{pgfscope}%
\begin{pgfscope}%
\pgfsys@transformshift{0.866915in}{2.390090in}%
\pgfsys@useobject{currentmarker}{}%
\end{pgfscope}%
\begin{pgfscope}%
\pgfsys@transformshift{0.867325in}{2.390110in}%
\pgfsys@useobject{currentmarker}{}%
\end{pgfscope}%
\begin{pgfscope}%
\pgfsys@transformshift{0.867551in}{2.390116in}%
\pgfsys@useobject{currentmarker}{}%
\end{pgfscope}%
\begin{pgfscope}%
\pgfsys@transformshift{0.867674in}{2.390125in}%
\pgfsys@useobject{currentmarker}{}%
\end{pgfscope}%
\begin{pgfscope}%
\pgfsys@transformshift{0.867742in}{2.390128in}%
\pgfsys@useobject{currentmarker}{}%
\end{pgfscope}%
\begin{pgfscope}%
\pgfsys@transformshift{0.867780in}{2.390131in}%
\pgfsys@useobject{currentmarker}{}%
\end{pgfscope}%
\begin{pgfscope}%
\pgfsys@transformshift{0.867800in}{2.390133in}%
\pgfsys@useobject{currentmarker}{}%
\end{pgfscope}%
\begin{pgfscope}%
\pgfsys@transformshift{0.867812in}{2.390133in}%
\pgfsys@useobject{currentmarker}{}%
\end{pgfscope}%
\begin{pgfscope}%
\pgfsys@transformshift{0.867818in}{2.390134in}%
\pgfsys@useobject{currentmarker}{}%
\end{pgfscope}%
\begin{pgfscope}%
\pgfsys@transformshift{0.867821in}{2.390134in}%
\pgfsys@useobject{currentmarker}{}%
\end{pgfscope}%
\begin{pgfscope}%
\pgfsys@transformshift{0.867823in}{2.390134in}%
\pgfsys@useobject{currentmarker}{}%
\end{pgfscope}%
\begin{pgfscope}%
\pgfsys@transformshift{0.867824in}{2.390134in}%
\pgfsys@useobject{currentmarker}{}%
\end{pgfscope}%
\begin{pgfscope}%
\pgfsys@transformshift{0.867825in}{2.390134in}%
\pgfsys@useobject{currentmarker}{}%
\end{pgfscope}%
\begin{pgfscope}%
\pgfsys@transformshift{0.867825in}{2.390134in}%
\pgfsys@useobject{currentmarker}{}%
\end{pgfscope}%
\begin{pgfscope}%
\pgfsys@transformshift{0.867825in}{2.390134in}%
\pgfsys@useobject{currentmarker}{}%
\end{pgfscope}%
\begin{pgfscope}%
\pgfsys@transformshift{0.867825in}{2.390134in}%
\pgfsys@useobject{currentmarker}{}%
\end{pgfscope}%
\begin{pgfscope}%
\pgfsys@transformshift{0.867825in}{2.390134in}%
\pgfsys@useobject{currentmarker}{}%
\end{pgfscope}%
\begin{pgfscope}%
\pgfsys@transformshift{0.867825in}{2.390134in}%
\pgfsys@useobject{currentmarker}{}%
\end{pgfscope}%
\begin{pgfscope}%
\pgfsys@transformshift{0.867825in}{2.390134in}%
\pgfsys@useobject{currentmarker}{}%
\end{pgfscope}%
\begin{pgfscope}%
\pgfsys@transformshift{0.867825in}{2.390134in}%
\pgfsys@useobject{currentmarker}{}%
\end{pgfscope}%
\begin{pgfscope}%
\pgfsys@transformshift{0.867825in}{2.390134in}%
\pgfsys@useobject{currentmarker}{}%
\end{pgfscope}%
\begin{pgfscope}%
\pgfsys@transformshift{0.867826in}{2.390134in}%
\pgfsys@useobject{currentmarker}{}%
\end{pgfscope}%
\begin{pgfscope}%
\pgfsys@transformshift{0.867826in}{2.390134in}%
\pgfsys@useobject{currentmarker}{}%
\end{pgfscope}%
\begin{pgfscope}%
\pgfsys@transformshift{0.867826in}{2.390134in}%
\pgfsys@useobject{currentmarker}{}%
\end{pgfscope}%
\begin{pgfscope}%
\pgfsys@transformshift{0.867826in}{2.390134in}%
\pgfsys@useobject{currentmarker}{}%
\end{pgfscope}%
\begin{pgfscope}%
\pgfsys@transformshift{0.867826in}{2.390134in}%
\pgfsys@useobject{currentmarker}{}%
\end{pgfscope}%
\begin{pgfscope}%
\pgfsys@transformshift{0.867826in}{2.390134in}%
\pgfsys@useobject{currentmarker}{}%
\end{pgfscope}%
\begin{pgfscope}%
\pgfsys@transformshift{0.867826in}{2.390134in}%
\pgfsys@useobject{currentmarker}{}%
\end{pgfscope}%
\begin{pgfscope}%
\pgfsys@transformshift{0.867826in}{2.390134in}%
\pgfsys@useobject{currentmarker}{}%
\end{pgfscope}%
\begin{pgfscope}%
\pgfsys@transformshift{0.867826in}{2.390134in}%
\pgfsys@useobject{currentmarker}{}%
\end{pgfscope}%
\begin{pgfscope}%
\pgfsys@transformshift{0.867826in}{2.390134in}%
\pgfsys@useobject{currentmarker}{}%
\end{pgfscope}%
\begin{pgfscope}%
\pgfsys@transformshift{0.867826in}{2.390134in}%
\pgfsys@useobject{currentmarker}{}%
\end{pgfscope}%
\begin{pgfscope}%
\pgfsys@transformshift{0.867826in}{2.390134in}%
\pgfsys@useobject{currentmarker}{}%
\end{pgfscope}%
\begin{pgfscope}%
\pgfsys@transformshift{0.867826in}{2.390134in}%
\pgfsys@useobject{currentmarker}{}%
\end{pgfscope}%
\begin{pgfscope}%
\pgfsys@transformshift{0.867826in}{2.390134in}%
\pgfsys@useobject{currentmarker}{}%
\end{pgfscope}%
\begin{pgfscope}%
\pgfsys@transformshift{0.867826in}{2.390134in}%
\pgfsys@useobject{currentmarker}{}%
\end{pgfscope}%
\begin{pgfscope}%
\pgfsys@transformshift{0.867826in}{2.390134in}%
\pgfsys@useobject{currentmarker}{}%
\end{pgfscope}%
\begin{pgfscope}%
\pgfsys@transformshift{0.867826in}{2.390134in}%
\pgfsys@useobject{currentmarker}{}%
\end{pgfscope}%
\begin{pgfscope}%
\pgfsys@transformshift{0.867826in}{2.390134in}%
\pgfsys@useobject{currentmarker}{}%
\end{pgfscope}%
\begin{pgfscope}%
\pgfsys@transformshift{0.867826in}{2.390134in}%
\pgfsys@useobject{currentmarker}{}%
\end{pgfscope}%
\begin{pgfscope}%
\pgfsys@transformshift{0.867826in}{2.390134in}%
\pgfsys@useobject{currentmarker}{}%
\end{pgfscope}%
\begin{pgfscope}%
\pgfsys@transformshift{0.867826in}{2.390134in}%
\pgfsys@useobject{currentmarker}{}%
\end{pgfscope}%
\begin{pgfscope}%
\pgfsys@transformshift{0.867826in}{2.390134in}%
\pgfsys@useobject{currentmarker}{}%
\end{pgfscope}%
\begin{pgfscope}%
\pgfsys@transformshift{0.867826in}{2.390134in}%
\pgfsys@useobject{currentmarker}{}%
\end{pgfscope}%
\begin{pgfscope}%
\pgfsys@transformshift{0.867826in}{2.390134in}%
\pgfsys@useobject{currentmarker}{}%
\end{pgfscope}%
\begin{pgfscope}%
\pgfsys@transformshift{0.867826in}{2.390134in}%
\pgfsys@useobject{currentmarker}{}%
\end{pgfscope}%
\begin{pgfscope}%
\pgfsys@transformshift{0.867826in}{2.390134in}%
\pgfsys@useobject{currentmarker}{}%
\end{pgfscope}%
\begin{pgfscope}%
\pgfsys@transformshift{0.867826in}{2.390134in}%
\pgfsys@useobject{currentmarker}{}%
\end{pgfscope}%
\begin{pgfscope}%
\pgfsys@transformshift{0.867826in}{2.390134in}%
\pgfsys@useobject{currentmarker}{}%
\end{pgfscope}%
\begin{pgfscope}%
\pgfsys@transformshift{0.867826in}{2.390134in}%
\pgfsys@useobject{currentmarker}{}%
\end{pgfscope}%
\begin{pgfscope}%
\pgfsys@transformshift{0.867826in}{2.390134in}%
\pgfsys@useobject{currentmarker}{}%
\end{pgfscope}%
\begin{pgfscope}%
\pgfsys@transformshift{0.867826in}{2.390134in}%
\pgfsys@useobject{currentmarker}{}%
\end{pgfscope}%
\begin{pgfscope}%
\pgfsys@transformshift{0.867826in}{2.390134in}%
\pgfsys@useobject{currentmarker}{}%
\end{pgfscope}%
\begin{pgfscope}%
\pgfsys@transformshift{0.867826in}{2.390134in}%
\pgfsys@useobject{currentmarker}{}%
\end{pgfscope}%
\begin{pgfscope}%
\pgfsys@transformshift{0.867826in}{2.390134in}%
\pgfsys@useobject{currentmarker}{}%
\end{pgfscope}%
\begin{pgfscope}%
\pgfsys@transformshift{0.867826in}{2.390134in}%
\pgfsys@useobject{currentmarker}{}%
\end{pgfscope}%
\begin{pgfscope}%
\pgfsys@transformshift{0.867826in}{2.390134in}%
\pgfsys@useobject{currentmarker}{}%
\end{pgfscope}%
\begin{pgfscope}%
\pgfsys@transformshift{0.867826in}{2.390134in}%
\pgfsys@useobject{currentmarker}{}%
\end{pgfscope}%
\begin{pgfscope}%
\pgfsys@transformshift{0.867826in}{2.390134in}%
\pgfsys@useobject{currentmarker}{}%
\end{pgfscope}%
\begin{pgfscope}%
\pgfsys@transformshift{0.867826in}{2.390134in}%
\pgfsys@useobject{currentmarker}{}%
\end{pgfscope}%
\begin{pgfscope}%
\pgfsys@transformshift{0.867826in}{2.390134in}%
\pgfsys@useobject{currentmarker}{}%
\end{pgfscope}%
\begin{pgfscope}%
\pgfsys@transformshift{0.867826in}{2.390134in}%
\pgfsys@useobject{currentmarker}{}%
\end{pgfscope}%
\begin{pgfscope}%
\pgfsys@transformshift{0.867826in}{2.390134in}%
\pgfsys@useobject{currentmarker}{}%
\end{pgfscope}%
\begin{pgfscope}%
\pgfsys@transformshift{0.867826in}{2.390134in}%
\pgfsys@useobject{currentmarker}{}%
\end{pgfscope}%
\begin{pgfscope}%
\pgfsys@transformshift{0.867826in}{2.390134in}%
\pgfsys@useobject{currentmarker}{}%
\end{pgfscope}%
\begin{pgfscope}%
\pgfsys@transformshift{0.867826in}{2.390134in}%
\pgfsys@useobject{currentmarker}{}%
\end{pgfscope}%
\begin{pgfscope}%
\pgfsys@transformshift{0.867826in}{2.390134in}%
\pgfsys@useobject{currentmarker}{}%
\end{pgfscope}%
\begin{pgfscope}%
\pgfsys@transformshift{0.867826in}{2.390134in}%
\pgfsys@useobject{currentmarker}{}%
\end{pgfscope}%
\begin{pgfscope}%
\pgfsys@transformshift{0.867826in}{2.390134in}%
\pgfsys@useobject{currentmarker}{}%
\end{pgfscope}%
\begin{pgfscope}%
\pgfsys@transformshift{0.867826in}{2.390134in}%
\pgfsys@useobject{currentmarker}{}%
\end{pgfscope}%
\begin{pgfscope}%
\pgfsys@transformshift{0.867826in}{2.390134in}%
\pgfsys@useobject{currentmarker}{}%
\end{pgfscope}%
\begin{pgfscope}%
\pgfsys@transformshift{0.867826in}{2.390134in}%
\pgfsys@useobject{currentmarker}{}%
\end{pgfscope}%
\begin{pgfscope}%
\pgfsys@transformshift{0.867826in}{2.390134in}%
\pgfsys@useobject{currentmarker}{}%
\end{pgfscope}%
\begin{pgfscope}%
\pgfsys@transformshift{0.867826in}{2.390134in}%
\pgfsys@useobject{currentmarker}{}%
\end{pgfscope}%
\begin{pgfscope}%
\pgfsys@transformshift{0.867826in}{2.390134in}%
\pgfsys@useobject{currentmarker}{}%
\end{pgfscope}%
\begin{pgfscope}%
\pgfsys@transformshift{0.867826in}{2.390134in}%
\pgfsys@useobject{currentmarker}{}%
\end{pgfscope}%
\begin{pgfscope}%
\pgfsys@transformshift{0.867826in}{2.390134in}%
\pgfsys@useobject{currentmarker}{}%
\end{pgfscope}%
\begin{pgfscope}%
\pgfsys@transformshift{0.867826in}{2.390134in}%
\pgfsys@useobject{currentmarker}{}%
\end{pgfscope}%
\begin{pgfscope}%
\pgfsys@transformshift{0.867826in}{2.390134in}%
\pgfsys@useobject{currentmarker}{}%
\end{pgfscope}%
\begin{pgfscope}%
\pgfsys@transformshift{0.867826in}{2.390134in}%
\pgfsys@useobject{currentmarker}{}%
\end{pgfscope}%
\begin{pgfscope}%
\pgfsys@transformshift{0.867826in}{2.390134in}%
\pgfsys@useobject{currentmarker}{}%
\end{pgfscope}%
\begin{pgfscope}%
\pgfsys@transformshift{0.867826in}{2.390134in}%
\pgfsys@useobject{currentmarker}{}%
\end{pgfscope}%
\begin{pgfscope}%
\pgfsys@transformshift{0.867826in}{2.390134in}%
\pgfsys@useobject{currentmarker}{}%
\end{pgfscope}%
\begin{pgfscope}%
\pgfsys@transformshift{0.867826in}{2.390134in}%
\pgfsys@useobject{currentmarker}{}%
\end{pgfscope}%
\begin{pgfscope}%
\pgfsys@transformshift{0.867826in}{2.390134in}%
\pgfsys@useobject{currentmarker}{}%
\end{pgfscope}%
\begin{pgfscope}%
\pgfsys@transformshift{0.867826in}{2.390134in}%
\pgfsys@useobject{currentmarker}{}%
\end{pgfscope}%
\begin{pgfscope}%
\pgfsys@transformshift{0.867826in}{2.390134in}%
\pgfsys@useobject{currentmarker}{}%
\end{pgfscope}%
\begin{pgfscope}%
\pgfsys@transformshift{0.867826in}{2.390134in}%
\pgfsys@useobject{currentmarker}{}%
\end{pgfscope}%
\begin{pgfscope}%
\pgfsys@transformshift{0.867826in}{2.390134in}%
\pgfsys@useobject{currentmarker}{}%
\end{pgfscope}%
\begin{pgfscope}%
\pgfsys@transformshift{0.867826in}{2.390134in}%
\pgfsys@useobject{currentmarker}{}%
\end{pgfscope}%
\begin{pgfscope}%
\pgfsys@transformshift{0.867826in}{2.390134in}%
\pgfsys@useobject{currentmarker}{}%
\end{pgfscope}%
\begin{pgfscope}%
\pgfsys@transformshift{0.867826in}{2.390134in}%
\pgfsys@useobject{currentmarker}{}%
\end{pgfscope}%
\begin{pgfscope}%
\pgfsys@transformshift{0.867826in}{2.390134in}%
\pgfsys@useobject{currentmarker}{}%
\end{pgfscope}%
\begin{pgfscope}%
\pgfsys@transformshift{0.867826in}{2.390134in}%
\pgfsys@useobject{currentmarker}{}%
\end{pgfscope}%
\begin{pgfscope}%
\pgfsys@transformshift{0.867826in}{2.390134in}%
\pgfsys@useobject{currentmarker}{}%
\end{pgfscope}%
\begin{pgfscope}%
\pgfsys@transformshift{0.867826in}{2.390134in}%
\pgfsys@useobject{currentmarker}{}%
\end{pgfscope}%
\begin{pgfscope}%
\pgfsys@transformshift{0.867826in}{2.390134in}%
\pgfsys@useobject{currentmarker}{}%
\end{pgfscope}%
\begin{pgfscope}%
\pgfsys@transformshift{0.867826in}{2.390134in}%
\pgfsys@useobject{currentmarker}{}%
\end{pgfscope}%
\begin{pgfscope}%
\pgfsys@transformshift{0.867826in}{2.390134in}%
\pgfsys@useobject{currentmarker}{}%
\end{pgfscope}%
\begin{pgfscope}%
\pgfsys@transformshift{0.867826in}{2.390134in}%
\pgfsys@useobject{currentmarker}{}%
\end{pgfscope}%
\begin{pgfscope}%
\pgfsys@transformshift{0.867826in}{2.390134in}%
\pgfsys@useobject{currentmarker}{}%
\end{pgfscope}%
\begin{pgfscope}%
\pgfsys@transformshift{0.867826in}{2.390134in}%
\pgfsys@useobject{currentmarker}{}%
\end{pgfscope}%
\begin{pgfscope}%
\pgfsys@transformshift{0.867826in}{2.390134in}%
\pgfsys@useobject{currentmarker}{}%
\end{pgfscope}%
\begin{pgfscope}%
\pgfsys@transformshift{0.867826in}{2.390134in}%
\pgfsys@useobject{currentmarker}{}%
\end{pgfscope}%
\begin{pgfscope}%
\pgfsys@transformshift{0.867826in}{2.390134in}%
\pgfsys@useobject{currentmarker}{}%
\end{pgfscope}%
\begin{pgfscope}%
\pgfsys@transformshift{0.867826in}{2.390134in}%
\pgfsys@useobject{currentmarker}{}%
\end{pgfscope}%
\begin{pgfscope}%
\pgfsys@transformshift{0.867826in}{2.390134in}%
\pgfsys@useobject{currentmarker}{}%
\end{pgfscope}%
\begin{pgfscope}%
\pgfsys@transformshift{0.867826in}{2.390134in}%
\pgfsys@useobject{currentmarker}{}%
\end{pgfscope}%
\begin{pgfscope}%
\pgfsys@transformshift{0.867826in}{2.390134in}%
\pgfsys@useobject{currentmarker}{}%
\end{pgfscope}%
\begin{pgfscope}%
\pgfsys@transformshift{0.868561in}{2.390275in}%
\pgfsys@useobject{currentmarker}{}%
\end{pgfscope}%
\begin{pgfscope}%
\pgfsys@transformshift{0.868973in}{2.390251in}%
\pgfsys@useobject{currentmarker}{}%
\end{pgfscope}%
\begin{pgfscope}%
\pgfsys@transformshift{0.869196in}{2.390289in}%
\pgfsys@useobject{currentmarker}{}%
\end{pgfscope}%
\begin{pgfscope}%
\pgfsys@transformshift{0.869320in}{2.390280in}%
\pgfsys@useobject{currentmarker}{}%
\end{pgfscope}%
\begin{pgfscope}%
\pgfsys@transformshift{0.869388in}{2.390292in}%
\pgfsys@useobject{currentmarker}{}%
\end{pgfscope}%
\begin{pgfscope}%
\pgfsys@transformshift{0.869426in}{2.390291in}%
\pgfsys@useobject{currentmarker}{}%
\end{pgfscope}%
\begin{pgfscope}%
\pgfsys@transformshift{0.869446in}{2.390293in}%
\pgfsys@useobject{currentmarker}{}%
\end{pgfscope}%
\begin{pgfscope}%
\pgfsys@transformshift{0.869458in}{2.390293in}%
\pgfsys@useobject{currentmarker}{}%
\end{pgfscope}%
\begin{pgfscope}%
\pgfsys@transformshift{0.869464in}{2.390294in}%
\pgfsys@useobject{currentmarker}{}%
\end{pgfscope}%
\begin{pgfscope}%
\pgfsys@transformshift{0.869467in}{2.390294in}%
\pgfsys@useobject{currentmarker}{}%
\end{pgfscope}%
\begin{pgfscope}%
\pgfsys@transformshift{0.869469in}{2.390294in}%
\pgfsys@useobject{currentmarker}{}%
\end{pgfscope}%
\begin{pgfscope}%
\pgfsys@transformshift{0.869470in}{2.390294in}%
\pgfsys@useobject{currentmarker}{}%
\end{pgfscope}%
\begin{pgfscope}%
\pgfsys@transformshift{0.869471in}{2.390294in}%
\pgfsys@useobject{currentmarker}{}%
\end{pgfscope}%
\begin{pgfscope}%
\pgfsys@transformshift{0.869471in}{2.390294in}%
\pgfsys@useobject{currentmarker}{}%
\end{pgfscope}%
\begin{pgfscope}%
\pgfsys@transformshift{0.869471in}{2.390294in}%
\pgfsys@useobject{currentmarker}{}%
\end{pgfscope}%
\begin{pgfscope}%
\pgfsys@transformshift{0.869471in}{2.390294in}%
\pgfsys@useobject{currentmarker}{}%
\end{pgfscope}%
\begin{pgfscope}%
\pgfsys@transformshift{0.869471in}{2.390294in}%
\pgfsys@useobject{currentmarker}{}%
\end{pgfscope}%
\begin{pgfscope}%
\pgfsys@transformshift{0.869471in}{2.390294in}%
\pgfsys@useobject{currentmarker}{}%
\end{pgfscope}%
\begin{pgfscope}%
\pgfsys@transformshift{0.869471in}{2.390294in}%
\pgfsys@useobject{currentmarker}{}%
\end{pgfscope}%
\begin{pgfscope}%
\pgfsys@transformshift{0.869471in}{2.390294in}%
\pgfsys@useobject{currentmarker}{}%
\end{pgfscope}%
\begin{pgfscope}%
\pgfsys@transformshift{0.869471in}{2.390294in}%
\pgfsys@useobject{currentmarker}{}%
\end{pgfscope}%
\begin{pgfscope}%
\pgfsys@transformshift{0.869471in}{2.390294in}%
\pgfsys@useobject{currentmarker}{}%
\end{pgfscope}%
\begin{pgfscope}%
\pgfsys@transformshift{0.869471in}{2.390294in}%
\pgfsys@useobject{currentmarker}{}%
\end{pgfscope}%
\begin{pgfscope}%
\pgfsys@transformshift{0.869471in}{2.390294in}%
\pgfsys@useobject{currentmarker}{}%
\end{pgfscope}%
\begin{pgfscope}%
\pgfsys@transformshift{0.869471in}{2.390294in}%
\pgfsys@useobject{currentmarker}{}%
\end{pgfscope}%
\begin{pgfscope}%
\pgfsys@transformshift{0.869471in}{2.390294in}%
\pgfsys@useobject{currentmarker}{}%
\end{pgfscope}%
\begin{pgfscope}%
\pgfsys@transformshift{0.869471in}{2.390294in}%
\pgfsys@useobject{currentmarker}{}%
\end{pgfscope}%
\begin{pgfscope}%
\pgfsys@transformshift{0.869471in}{2.390294in}%
\pgfsys@useobject{currentmarker}{}%
\end{pgfscope}%
\begin{pgfscope}%
\pgfsys@transformshift{0.869471in}{2.390294in}%
\pgfsys@useobject{currentmarker}{}%
\end{pgfscope}%
\begin{pgfscope}%
\pgfsys@transformshift{0.869471in}{2.390294in}%
\pgfsys@useobject{currentmarker}{}%
\end{pgfscope}%
\begin{pgfscope}%
\pgfsys@transformshift{0.869471in}{2.390294in}%
\pgfsys@useobject{currentmarker}{}%
\end{pgfscope}%
\begin{pgfscope}%
\pgfsys@transformshift{0.869471in}{2.390294in}%
\pgfsys@useobject{currentmarker}{}%
\end{pgfscope}%
\begin{pgfscope}%
\pgfsys@transformshift{0.869471in}{2.390294in}%
\pgfsys@useobject{currentmarker}{}%
\end{pgfscope}%
\begin{pgfscope}%
\pgfsys@transformshift{0.869471in}{2.390294in}%
\pgfsys@useobject{currentmarker}{}%
\end{pgfscope}%
\begin{pgfscope}%
\pgfsys@transformshift{0.869471in}{2.390294in}%
\pgfsys@useobject{currentmarker}{}%
\end{pgfscope}%
\begin{pgfscope}%
\pgfsys@transformshift{0.869471in}{2.390294in}%
\pgfsys@useobject{currentmarker}{}%
\end{pgfscope}%
\begin{pgfscope}%
\pgfsys@transformshift{0.869471in}{2.390294in}%
\pgfsys@useobject{currentmarker}{}%
\end{pgfscope}%
\begin{pgfscope}%
\pgfsys@transformshift{0.869471in}{2.390294in}%
\pgfsys@useobject{currentmarker}{}%
\end{pgfscope}%
\begin{pgfscope}%
\pgfsys@transformshift{0.869471in}{2.390294in}%
\pgfsys@useobject{currentmarker}{}%
\end{pgfscope}%
\begin{pgfscope}%
\pgfsys@transformshift{0.869471in}{2.390294in}%
\pgfsys@useobject{currentmarker}{}%
\end{pgfscope}%
\begin{pgfscope}%
\pgfsys@transformshift{0.869471in}{2.390294in}%
\pgfsys@useobject{currentmarker}{}%
\end{pgfscope}%
\begin{pgfscope}%
\pgfsys@transformshift{0.869471in}{2.390294in}%
\pgfsys@useobject{currentmarker}{}%
\end{pgfscope}%
\begin{pgfscope}%
\pgfsys@transformshift{0.869471in}{2.390294in}%
\pgfsys@useobject{currentmarker}{}%
\end{pgfscope}%
\begin{pgfscope}%
\pgfsys@transformshift{0.869471in}{2.390294in}%
\pgfsys@useobject{currentmarker}{}%
\end{pgfscope}%
\begin{pgfscope}%
\pgfsys@transformshift{0.869471in}{2.390294in}%
\pgfsys@useobject{currentmarker}{}%
\end{pgfscope}%
\begin{pgfscope}%
\pgfsys@transformshift{0.869471in}{2.390294in}%
\pgfsys@useobject{currentmarker}{}%
\end{pgfscope}%
\begin{pgfscope}%
\pgfsys@transformshift{0.869471in}{2.390294in}%
\pgfsys@useobject{currentmarker}{}%
\end{pgfscope}%
\begin{pgfscope}%
\pgfsys@transformshift{0.869471in}{2.390294in}%
\pgfsys@useobject{currentmarker}{}%
\end{pgfscope}%
\begin{pgfscope}%
\pgfsys@transformshift{0.869471in}{2.390294in}%
\pgfsys@useobject{currentmarker}{}%
\end{pgfscope}%
\begin{pgfscope}%
\pgfsys@transformshift{0.869471in}{2.390294in}%
\pgfsys@useobject{currentmarker}{}%
\end{pgfscope}%
\begin{pgfscope}%
\pgfsys@transformshift{0.869471in}{2.390294in}%
\pgfsys@useobject{currentmarker}{}%
\end{pgfscope}%
\begin{pgfscope}%
\pgfsys@transformshift{0.869471in}{2.390294in}%
\pgfsys@useobject{currentmarker}{}%
\end{pgfscope}%
\begin{pgfscope}%
\pgfsys@transformshift{0.869471in}{2.390294in}%
\pgfsys@useobject{currentmarker}{}%
\end{pgfscope}%
\begin{pgfscope}%
\pgfsys@transformshift{0.869471in}{2.390294in}%
\pgfsys@useobject{currentmarker}{}%
\end{pgfscope}%
\begin{pgfscope}%
\pgfsys@transformshift{0.869471in}{2.390294in}%
\pgfsys@useobject{currentmarker}{}%
\end{pgfscope}%
\begin{pgfscope}%
\pgfsys@transformshift{0.869471in}{2.390294in}%
\pgfsys@useobject{currentmarker}{}%
\end{pgfscope}%
\begin{pgfscope}%
\pgfsys@transformshift{0.869471in}{2.390294in}%
\pgfsys@useobject{currentmarker}{}%
\end{pgfscope}%
\begin{pgfscope}%
\pgfsys@transformshift{0.869471in}{2.390294in}%
\pgfsys@useobject{currentmarker}{}%
\end{pgfscope}%
\begin{pgfscope}%
\pgfsys@transformshift{0.869471in}{2.390294in}%
\pgfsys@useobject{currentmarker}{}%
\end{pgfscope}%
\begin{pgfscope}%
\pgfsys@transformshift{0.869471in}{2.390294in}%
\pgfsys@useobject{currentmarker}{}%
\end{pgfscope}%
\begin{pgfscope}%
\pgfsys@transformshift{0.869471in}{2.390294in}%
\pgfsys@useobject{currentmarker}{}%
\end{pgfscope}%
\begin{pgfscope}%
\pgfsys@transformshift{0.869471in}{2.390294in}%
\pgfsys@useobject{currentmarker}{}%
\end{pgfscope}%
\begin{pgfscope}%
\pgfsys@transformshift{0.869471in}{2.390294in}%
\pgfsys@useobject{currentmarker}{}%
\end{pgfscope}%
\begin{pgfscope}%
\pgfsys@transformshift{0.869471in}{2.390294in}%
\pgfsys@useobject{currentmarker}{}%
\end{pgfscope}%
\begin{pgfscope}%
\pgfsys@transformshift{0.869471in}{2.390294in}%
\pgfsys@useobject{currentmarker}{}%
\end{pgfscope}%
\begin{pgfscope}%
\pgfsys@transformshift{0.869471in}{2.390294in}%
\pgfsys@useobject{currentmarker}{}%
\end{pgfscope}%
\begin{pgfscope}%
\pgfsys@transformshift{0.869471in}{2.390294in}%
\pgfsys@useobject{currentmarker}{}%
\end{pgfscope}%
\begin{pgfscope}%
\pgfsys@transformshift{0.869471in}{2.390294in}%
\pgfsys@useobject{currentmarker}{}%
\end{pgfscope}%
\begin{pgfscope}%
\pgfsys@transformshift{0.869471in}{2.390294in}%
\pgfsys@useobject{currentmarker}{}%
\end{pgfscope}%
\begin{pgfscope}%
\pgfsys@transformshift{0.869471in}{2.390294in}%
\pgfsys@useobject{currentmarker}{}%
\end{pgfscope}%
\begin{pgfscope}%
\pgfsys@transformshift{0.869471in}{2.390294in}%
\pgfsys@useobject{currentmarker}{}%
\end{pgfscope}%
\begin{pgfscope}%
\pgfsys@transformshift{0.869471in}{2.390294in}%
\pgfsys@useobject{currentmarker}{}%
\end{pgfscope}%
\begin{pgfscope}%
\pgfsys@transformshift{0.869471in}{2.390294in}%
\pgfsys@useobject{currentmarker}{}%
\end{pgfscope}%
\begin{pgfscope}%
\pgfsys@transformshift{0.869471in}{2.390294in}%
\pgfsys@useobject{currentmarker}{}%
\end{pgfscope}%
\begin{pgfscope}%
\pgfsys@transformshift{0.869471in}{2.390294in}%
\pgfsys@useobject{currentmarker}{}%
\end{pgfscope}%
\begin{pgfscope}%
\pgfsys@transformshift{0.869471in}{2.390294in}%
\pgfsys@useobject{currentmarker}{}%
\end{pgfscope}%
\begin{pgfscope}%
\pgfsys@transformshift{0.869471in}{2.390294in}%
\pgfsys@useobject{currentmarker}{}%
\end{pgfscope}%
\begin{pgfscope}%
\pgfsys@transformshift{0.869471in}{2.390294in}%
\pgfsys@useobject{currentmarker}{}%
\end{pgfscope}%
\begin{pgfscope}%
\pgfsys@transformshift{0.869471in}{2.390294in}%
\pgfsys@useobject{currentmarker}{}%
\end{pgfscope}%
\begin{pgfscope}%
\pgfsys@transformshift{0.869471in}{2.390294in}%
\pgfsys@useobject{currentmarker}{}%
\end{pgfscope}%
\begin{pgfscope}%
\pgfsys@transformshift{0.869471in}{2.390294in}%
\pgfsys@useobject{currentmarker}{}%
\end{pgfscope}%
\begin{pgfscope}%
\pgfsys@transformshift{0.869471in}{2.390294in}%
\pgfsys@useobject{currentmarker}{}%
\end{pgfscope}%
\begin{pgfscope}%
\pgfsys@transformshift{0.869471in}{2.390294in}%
\pgfsys@useobject{currentmarker}{}%
\end{pgfscope}%
\begin{pgfscope}%
\pgfsys@transformshift{0.869471in}{2.390294in}%
\pgfsys@useobject{currentmarker}{}%
\end{pgfscope}%
\begin{pgfscope}%
\pgfsys@transformshift{0.869471in}{2.390294in}%
\pgfsys@useobject{currentmarker}{}%
\end{pgfscope}%
\begin{pgfscope}%
\pgfsys@transformshift{0.869471in}{2.390294in}%
\pgfsys@useobject{currentmarker}{}%
\end{pgfscope}%
\begin{pgfscope}%
\pgfsys@transformshift{0.869471in}{2.390294in}%
\pgfsys@useobject{currentmarker}{}%
\end{pgfscope}%
\begin{pgfscope}%
\pgfsys@transformshift{0.869471in}{2.390294in}%
\pgfsys@useobject{currentmarker}{}%
\end{pgfscope}%
\begin{pgfscope}%
\pgfsys@transformshift{0.869471in}{2.390294in}%
\pgfsys@useobject{currentmarker}{}%
\end{pgfscope}%
\begin{pgfscope}%
\pgfsys@transformshift{0.869471in}{2.390294in}%
\pgfsys@useobject{currentmarker}{}%
\end{pgfscope}%
\begin{pgfscope}%
\pgfsys@transformshift{0.869471in}{2.390294in}%
\pgfsys@useobject{currentmarker}{}%
\end{pgfscope}%
\begin{pgfscope}%
\pgfsys@transformshift{0.869471in}{2.390294in}%
\pgfsys@useobject{currentmarker}{}%
\end{pgfscope}%
\begin{pgfscope}%
\pgfsys@transformshift{0.869471in}{2.390294in}%
\pgfsys@useobject{currentmarker}{}%
\end{pgfscope}%
\begin{pgfscope}%
\pgfsys@transformshift{0.869471in}{2.390294in}%
\pgfsys@useobject{currentmarker}{}%
\end{pgfscope}%
\begin{pgfscope}%
\pgfsys@transformshift{0.869471in}{2.390294in}%
\pgfsys@useobject{currentmarker}{}%
\end{pgfscope}%
\begin{pgfscope}%
\pgfsys@transformshift{0.869471in}{2.390294in}%
\pgfsys@useobject{currentmarker}{}%
\end{pgfscope}%
\begin{pgfscope}%
\pgfsys@transformshift{0.869471in}{2.390294in}%
\pgfsys@useobject{currentmarker}{}%
\end{pgfscope}%
\begin{pgfscope}%
\pgfsys@transformshift{0.869471in}{2.390294in}%
\pgfsys@useobject{currentmarker}{}%
\end{pgfscope}%
\begin{pgfscope}%
\pgfsys@transformshift{0.869471in}{2.390294in}%
\pgfsys@useobject{currentmarker}{}%
\end{pgfscope}%
\begin{pgfscope}%
\pgfsys@transformshift{0.869471in}{2.390294in}%
\pgfsys@useobject{currentmarker}{}%
\end{pgfscope}%
\begin{pgfscope}%
\pgfsys@transformshift{0.869471in}{2.390294in}%
\pgfsys@useobject{currentmarker}{}%
\end{pgfscope}%
\begin{pgfscope}%
\pgfsys@transformshift{0.869471in}{2.390294in}%
\pgfsys@useobject{currentmarker}{}%
\end{pgfscope}%
\begin{pgfscope}%
\pgfsys@transformshift{0.869471in}{2.390294in}%
\pgfsys@useobject{currentmarker}{}%
\end{pgfscope}%
\begin{pgfscope}%
\pgfsys@transformshift{0.869471in}{2.390294in}%
\pgfsys@useobject{currentmarker}{}%
\end{pgfscope}%
\begin{pgfscope}%
\pgfsys@transformshift{0.869471in}{2.390294in}%
\pgfsys@useobject{currentmarker}{}%
\end{pgfscope}%
\begin{pgfscope}%
\pgfsys@transformshift{0.869471in}{2.390294in}%
\pgfsys@useobject{currentmarker}{}%
\end{pgfscope}%
\begin{pgfscope}%
\pgfsys@transformshift{0.869471in}{2.390294in}%
\pgfsys@useobject{currentmarker}{}%
\end{pgfscope}%
\begin{pgfscope}%
\pgfsys@transformshift{0.869471in}{2.390294in}%
\pgfsys@useobject{currentmarker}{}%
\end{pgfscope}%
\begin{pgfscope}%
\pgfsys@transformshift{0.869471in}{2.390294in}%
\pgfsys@useobject{currentmarker}{}%
\end{pgfscope}%
\begin{pgfscope}%
\pgfsys@transformshift{0.869471in}{2.390294in}%
\pgfsys@useobject{currentmarker}{}%
\end{pgfscope}%
\begin{pgfscope}%
\pgfsys@transformshift{0.869471in}{2.390294in}%
\pgfsys@useobject{currentmarker}{}%
\end{pgfscope}%
\begin{pgfscope}%
\pgfsys@transformshift{0.869471in}{2.390294in}%
\pgfsys@useobject{currentmarker}{}%
\end{pgfscope}%
\begin{pgfscope}%
\pgfsys@transformshift{0.869471in}{2.390294in}%
\pgfsys@useobject{currentmarker}{}%
\end{pgfscope}%
\begin{pgfscope}%
\pgfsys@transformshift{0.869471in}{2.390294in}%
\pgfsys@useobject{currentmarker}{}%
\end{pgfscope}%
\begin{pgfscope}%
\pgfsys@transformshift{0.869471in}{2.390294in}%
\pgfsys@useobject{currentmarker}{}%
\end{pgfscope}%
\begin{pgfscope}%
\pgfsys@transformshift{0.869471in}{2.390294in}%
\pgfsys@useobject{currentmarker}{}%
\end{pgfscope}%
\begin{pgfscope}%
\pgfsys@transformshift{0.869471in}{2.390294in}%
\pgfsys@useobject{currentmarker}{}%
\end{pgfscope}%
\begin{pgfscope}%
\pgfsys@transformshift{0.869471in}{2.390294in}%
\pgfsys@useobject{currentmarker}{}%
\end{pgfscope}%
\begin{pgfscope}%
\pgfsys@transformshift{0.869471in}{2.390294in}%
\pgfsys@useobject{currentmarker}{}%
\end{pgfscope}%
\begin{pgfscope}%
\pgfsys@transformshift{0.869471in}{2.390294in}%
\pgfsys@useobject{currentmarker}{}%
\end{pgfscope}%
\begin{pgfscope}%
\pgfsys@transformshift{0.869471in}{2.390294in}%
\pgfsys@useobject{currentmarker}{}%
\end{pgfscope}%
\begin{pgfscope}%
\pgfsys@transformshift{0.869471in}{2.390294in}%
\pgfsys@useobject{currentmarker}{}%
\end{pgfscope}%
\begin{pgfscope}%
\pgfsys@transformshift{0.869471in}{2.390294in}%
\pgfsys@useobject{currentmarker}{}%
\end{pgfscope}%
\begin{pgfscope}%
\pgfsys@transformshift{0.876765in}{2.388591in}%
\pgfsys@useobject{currentmarker}{}%
\end{pgfscope}%
\begin{pgfscope}%
\pgfsys@transformshift{0.884756in}{2.385879in}%
\pgfsys@useobject{currentmarker}{}%
\end{pgfscope}%
\begin{pgfscope}%
\pgfsys@transformshift{0.889302in}{2.384946in}%
\pgfsys@useobject{currentmarker}{}%
\end{pgfscope}%
\begin{pgfscope}%
\pgfsys@transformshift{0.894592in}{2.384904in}%
\pgfsys@useobject{currentmarker}{}%
\end{pgfscope}%
\begin{pgfscope}%
\pgfsys@transformshift{0.903626in}{2.382738in}%
\pgfsys@useobject{currentmarker}{}%
\end{pgfscope}%
\begin{pgfscope}%
\pgfsys@transformshift{0.908729in}{2.382493in}%
\pgfsys@useobject{currentmarker}{}%
\end{pgfscope}%
\begin{pgfscope}%
\pgfsys@transformshift{0.915577in}{2.381887in}%
\pgfsys@useobject{currentmarker}{}%
\end{pgfscope}%
\begin{pgfscope}%
\pgfsys@transformshift{0.923450in}{2.379980in}%
\pgfsys@useobject{currentmarker}{}%
\end{pgfscope}%
\begin{pgfscope}%
\pgfsys@transformshift{0.927900in}{2.379760in}%
\pgfsys@useobject{currentmarker}{}%
\end{pgfscope}%
\begin{pgfscope}%
\pgfsys@transformshift{0.933447in}{2.378642in}%
\pgfsys@useobject{currentmarker}{}%
\end{pgfscope}%
\begin{pgfscope}%
\pgfsys@transformshift{0.939869in}{2.378071in}%
\pgfsys@useobject{currentmarker}{}%
\end{pgfscope}%
\begin{pgfscope}%
\pgfsys@transformshift{0.943409in}{2.378289in}%
\pgfsys@useobject{currentmarker}{}%
\end{pgfscope}%
\begin{pgfscope}%
\pgfsys@transformshift{0.948553in}{2.378709in}%
\pgfsys@useobject{currentmarker}{}%
\end{pgfscope}%
\begin{pgfscope}%
\pgfsys@transformshift{0.954553in}{2.378865in}%
\pgfsys@useobject{currentmarker}{}%
\end{pgfscope}%
\begin{pgfscope}%
\pgfsys@transformshift{0.957827in}{2.379284in}%
\pgfsys@useobject{currentmarker}{}%
\end{pgfscope}%
\begin{pgfscope}%
\pgfsys@transformshift{0.962456in}{2.378980in}%
\pgfsys@useobject{currentmarker}{}%
\end{pgfscope}%
\begin{pgfscope}%
\pgfsys@transformshift{0.969335in}{2.379043in}%
\pgfsys@useobject{currentmarker}{}%
\end{pgfscope}%
\begin{pgfscope}%
\pgfsys@transformshift{0.973119in}{2.378985in}%
\pgfsys@useobject{currentmarker}{}%
\end{pgfscope}%
\begin{pgfscope}%
\pgfsys@transformshift{0.978142in}{2.378290in}%
\pgfsys@useobject{currentmarker}{}%
\end{pgfscope}%
\begin{pgfscope}%
\pgfsys@transformshift{0.984956in}{2.378337in}%
\pgfsys@useobject{currentmarker}{}%
\end{pgfscope}%
\begin{pgfscope}%
\pgfsys@transformshift{0.993550in}{2.377378in}%
\pgfsys@useobject{currentmarker}{}%
\end{pgfscope}%
\begin{pgfscope}%
\pgfsys@transformshift{0.998305in}{2.377447in}%
\pgfsys@useobject{currentmarker}{}%
\end{pgfscope}%
\begin{pgfscope}%
\pgfsys@transformshift{1.000921in}{2.377490in}%
\pgfsys@useobject{currentmarker}{}%
\end{pgfscope}%
\begin{pgfscope}%
\pgfsys@transformshift{1.005189in}{2.377399in}%
\pgfsys@useobject{currentmarker}{}%
\end{pgfscope}%
\begin{pgfscope}%
\pgfsys@transformshift{1.011066in}{2.377579in}%
\pgfsys@useobject{currentmarker}{}%
\end{pgfscope}%
\begin{pgfscope}%
\pgfsys@transformshift{1.017715in}{2.377276in}%
\pgfsys@useobject{currentmarker}{}%
\end{pgfscope}%
\begin{pgfscope}%
\pgfsys@transformshift{1.025396in}{2.378151in}%
\pgfsys@useobject{currentmarker}{}%
\end{pgfscope}%
\begin{pgfscope}%
\pgfsys@transformshift{1.035132in}{2.379404in}%
\pgfsys@useobject{currentmarker}{}%
\end{pgfscope}%
\begin{pgfscope}%
\pgfsys@transformshift{1.045826in}{2.378320in}%
\pgfsys@useobject{currentmarker}{}%
\end{pgfscope}%
\begin{pgfscope}%
\pgfsys@transformshift{1.051737in}{2.378446in}%
\pgfsys@useobject{currentmarker}{}%
\end{pgfscope}%
\begin{pgfscope}%
\pgfsys@transformshift{1.054984in}{2.378610in}%
\pgfsys@useobject{currentmarker}{}%
\end{pgfscope}%
\begin{pgfscope}%
\pgfsys@transformshift{1.060036in}{2.378518in}%
\pgfsys@useobject{currentmarker}{}%
\end{pgfscope}%
\begin{pgfscope}%
\pgfsys@transformshift{1.066468in}{2.379107in}%
\pgfsys@useobject{currentmarker}{}%
\end{pgfscope}%
\begin{pgfscope}%
\pgfsys@transformshift{1.070009in}{2.378826in}%
\pgfsys@useobject{currentmarker}{}%
\end{pgfscope}%
\begin{pgfscope}%
\pgfsys@transformshift{1.074468in}{2.379056in}%
\pgfsys@useobject{currentmarker}{}%
\end{pgfscope}%
\begin{pgfscope}%
\pgfsys@transformshift{1.081170in}{2.379095in}%
\pgfsys@useobject{currentmarker}{}%
\end{pgfscope}%
\begin{pgfscope}%
\pgfsys@transformshift{1.088870in}{2.378596in}%
\pgfsys@useobject{currentmarker}{}%
\end{pgfscope}%
\begin{pgfscope}%
\pgfsys@transformshift{1.093110in}{2.378423in}%
\pgfsys@useobject{currentmarker}{}%
\end{pgfscope}%
\begin{pgfscope}%
\pgfsys@transformshift{1.095443in}{2.378485in}%
\pgfsys@useobject{currentmarker}{}%
\end{pgfscope}%
\begin{pgfscope}%
\pgfsys@transformshift{1.100298in}{2.377927in}%
\pgfsys@useobject{currentmarker}{}%
\end{pgfscope}%
\begin{pgfscope}%
\pgfsys@transformshift{1.106899in}{2.378475in}%
\pgfsys@useobject{currentmarker}{}%
\end{pgfscope}%
\begin{pgfscope}%
\pgfsys@transformshift{1.114358in}{2.377823in}%
\pgfsys@useobject{currentmarker}{}%
\end{pgfscope}%
\begin{pgfscope}%
\pgfsys@transformshift{1.118466in}{2.377542in}%
\pgfsys@useobject{currentmarker}{}%
\end{pgfscope}%
\begin{pgfscope}%
\pgfsys@transformshift{1.123760in}{2.377466in}%
\pgfsys@useobject{currentmarker}{}%
\end{pgfscope}%
\begin{pgfscope}%
\pgfsys@transformshift{1.131959in}{2.376928in}%
\pgfsys@useobject{currentmarker}{}%
\end{pgfscope}%
\begin{pgfscope}%
\pgfsys@transformshift{1.142477in}{2.377373in}%
\pgfsys@useobject{currentmarker}{}%
\end{pgfscope}%
\begin{pgfscope}%
\pgfsys@transformshift{1.153715in}{2.376141in}%
\pgfsys@useobject{currentmarker}{}%
\end{pgfscope}%
\begin{pgfscope}%
\pgfsys@transformshift{1.165687in}{2.375477in}%
\pgfsys@useobject{currentmarker}{}%
\end{pgfscope}%
\begin{pgfscope}%
\pgfsys@transformshift{1.178960in}{2.375138in}%
\pgfsys@useobject{currentmarker}{}%
\end{pgfscope}%
\begin{pgfscope}%
\pgfsys@transformshift{1.194579in}{2.373985in}%
\pgfsys@useobject{currentmarker}{}%
\end{pgfscope}%
\begin{pgfscope}%
\pgfsys@transformshift{1.211351in}{2.373981in}%
\pgfsys@useobject{currentmarker}{}%
\end{pgfscope}%
\begin{pgfscope}%
\pgfsys@transformshift{1.220562in}{2.373482in}%
\pgfsys@useobject{currentmarker}{}%
\end{pgfscope}%
\begin{pgfscope}%
\pgfsys@transformshift{1.230498in}{2.373713in}%
\pgfsys@useobject{currentmarker}{}%
\end{pgfscope}%
\begin{pgfscope}%
\pgfsys@transformshift{1.235949in}{2.373309in}%
\pgfsys@useobject{currentmarker}{}%
\end{pgfscope}%
\begin{pgfscope}%
\pgfsys@transformshift{1.243074in}{2.373668in}%
\pgfsys@useobject{currentmarker}{}%
\end{pgfscope}%
\begin{pgfscope}%
\pgfsys@transformshift{1.251713in}{2.373833in}%
\pgfsys@useobject{currentmarker}{}%
\end{pgfscope}%
\begin{pgfscope}%
\pgfsys@transformshift{1.261941in}{2.373424in}%
\pgfsys@useobject{currentmarker}{}%
\end{pgfscope}%
\begin{pgfscope}%
\pgfsys@transformshift{1.272720in}{2.374876in}%
\pgfsys@useobject{currentmarker}{}%
\end{pgfscope}%
\begin{pgfscope}%
\pgfsys@transformshift{1.284232in}{2.374431in}%
\pgfsys@useobject{currentmarker}{}%
\end{pgfscope}%
\begin{pgfscope}%
\pgfsys@transformshift{1.296712in}{2.375734in}%
\pgfsys@useobject{currentmarker}{}%
\end{pgfscope}%
\begin{pgfscope}%
\pgfsys@transformshift{1.311305in}{2.376830in}%
\pgfsys@useobject{currentmarker}{}%
\end{pgfscope}%
\begin{pgfscope}%
\pgfsys@transformshift{1.327413in}{2.376246in}%
\pgfsys@useobject{currentmarker}{}%
\end{pgfscope}%
\begin{pgfscope}%
\pgfsys@transformshift{1.343980in}{2.379302in}%
\pgfsys@useobject{currentmarker}{}%
\end{pgfscope}%
\begin{pgfscope}%
\pgfsys@transformshift{1.361447in}{2.376493in}%
\pgfsys@useobject{currentmarker}{}%
\end{pgfscope}%
\begin{pgfscope}%
\pgfsys@transformshift{1.380062in}{2.379493in}%
\pgfsys@useobject{currentmarker}{}%
\end{pgfscope}%
\begin{pgfscope}%
\pgfsys@transformshift{1.400499in}{2.379333in}%
\pgfsys@useobject{currentmarker}{}%
\end{pgfscope}%
\begin{pgfscope}%
\pgfsys@transformshift{1.422365in}{2.380881in}%
\pgfsys@useobject{currentmarker}{}%
\end{pgfscope}%
\begin{pgfscope}%
\pgfsys@transformshift{1.445697in}{2.385098in}%
\pgfsys@useobject{currentmarker}{}%
\end{pgfscope}%
\begin{pgfscope}%
\pgfsys@transformshift{1.470372in}{2.383694in}%
\pgfsys@useobject{currentmarker}{}%
\end{pgfscope}%
\begin{pgfscope}%
\pgfsys@transformshift{1.483809in}{2.385751in}%
\pgfsys@useobject{currentmarker}{}%
\end{pgfscope}%
\begin{pgfscope}%
\pgfsys@transformshift{1.491191in}{2.384566in}%
\pgfsys@useobject{currentmarker}{}%
\end{pgfscope}%
\begin{pgfscope}%
\pgfsys@transformshift{1.500082in}{2.385807in}%
\pgfsys@useobject{currentmarker}{}%
\end{pgfscope}%
\begin{pgfscope}%
\pgfsys@transformshift{1.510394in}{2.385774in}%
\pgfsys@useobject{currentmarker}{}%
\end{pgfscope}%
\begin{pgfscope}%
\pgfsys@transformshift{1.522692in}{2.387116in}%
\pgfsys@useobject{currentmarker}{}%
\end{pgfscope}%
\begin{pgfscope}%
\pgfsys@transformshift{1.536630in}{2.386919in}%
\pgfsys@useobject{currentmarker}{}%
\end{pgfscope}%
\begin{pgfscope}%
\pgfsys@transformshift{1.551068in}{2.388851in}%
\pgfsys@useobject{currentmarker}{}%
\end{pgfscope}%
\begin{pgfscope}%
\pgfsys@transformshift{1.566236in}{2.386925in}%
\pgfsys@useobject{currentmarker}{}%
\end{pgfscope}%
\begin{pgfscope}%
\pgfsys@transformshift{1.582518in}{2.386982in}%
\pgfsys@useobject{currentmarker}{}%
\end{pgfscope}%
\begin{pgfscope}%
\pgfsys@transformshift{1.600908in}{2.387444in}%
\pgfsys@useobject{currentmarker}{}%
\end{pgfscope}%
\begin{pgfscope}%
\pgfsys@transformshift{1.620268in}{2.387579in}%
\pgfsys@useobject{currentmarker}{}%
\end{pgfscope}%
\begin{pgfscope}%
\pgfsys@transformshift{1.640413in}{2.386407in}%
\pgfsys@useobject{currentmarker}{}%
\end{pgfscope}%
\begin{pgfscope}%
\pgfsys@transformshift{1.651479in}{2.387263in}%
\pgfsys@useobject{currentmarker}{}%
\end{pgfscope}%
\begin{pgfscope}%
\pgfsys@transformshift{1.657583in}{2.387243in}%
\pgfsys@useobject{currentmarker}{}%
\end{pgfscope}%
\begin{pgfscope}%
\pgfsys@transformshift{1.666350in}{2.387753in}%
\pgfsys@useobject{currentmarker}{}%
\end{pgfscope}%
\begin{pgfscope}%
\pgfsys@transformshift{1.676781in}{2.387961in}%
\pgfsys@useobject{currentmarker}{}%
\end{pgfscope}%
\begin{pgfscope}%
\pgfsys@transformshift{1.688375in}{2.388820in}%
\pgfsys@useobject{currentmarker}{}%
\end{pgfscope}%
\begin{pgfscope}%
\pgfsys@transformshift{1.694731in}{2.389520in}%
\pgfsys@useobject{currentmarker}{}%
\end{pgfscope}%
\begin{pgfscope}%
\pgfsys@transformshift{1.698248in}{2.389497in}%
\pgfsys@useobject{currentmarker}{}%
\end{pgfscope}%
\begin{pgfscope}%
\pgfsys@transformshift{1.703623in}{2.389736in}%
\pgfsys@useobject{currentmarker}{}%
\end{pgfscope}%
\begin{pgfscope}%
\pgfsys@transformshift{1.709909in}{2.390081in}%
\pgfsys@useobject{currentmarker}{}%
\end{pgfscope}%
\begin{pgfscope}%
\pgfsys@transformshift{1.717544in}{2.389980in}%
\pgfsys@useobject{currentmarker}{}%
\end{pgfscope}%
\begin{pgfscope}%
\pgfsys@transformshift{1.721738in}{2.390189in}%
\pgfsys@useobject{currentmarker}{}%
\end{pgfscope}%
\begin{pgfscope}%
\pgfsys@transformshift{1.726714in}{2.389748in}%
\pgfsys@useobject{currentmarker}{}%
\end{pgfscope}%
\begin{pgfscope}%
\pgfsys@transformshift{1.729460in}{2.389847in}%
\pgfsys@useobject{currentmarker}{}%
\end{pgfscope}%
\begin{pgfscope}%
\pgfsys@transformshift{1.733082in}{2.389789in}%
\pgfsys@useobject{currentmarker}{}%
\end{pgfscope}%
\begin{pgfscope}%
\pgfsys@transformshift{1.737903in}{2.390425in}%
\pgfsys@useobject{currentmarker}{}%
\end{pgfscope}%
\begin{pgfscope}%
\pgfsys@transformshift{1.744399in}{2.391114in}%
\pgfsys@useobject{currentmarker}{}%
\end{pgfscope}%
\begin{pgfscope}%
\pgfsys@transformshift{1.752023in}{2.391231in}%
\pgfsys@useobject{currentmarker}{}%
\end{pgfscope}%
\begin{pgfscope}%
\pgfsys@transformshift{1.756217in}{2.391295in}%
\pgfsys@useobject{currentmarker}{}%
\end{pgfscope}%
\begin{pgfscope}%
\pgfsys@transformshift{1.758521in}{2.391402in}%
\pgfsys@useobject{currentmarker}{}%
\end{pgfscope}%
\begin{pgfscope}%
\pgfsys@transformshift{1.762138in}{2.391189in}%
\pgfsys@useobject{currentmarker}{}%
\end{pgfscope}%
\begin{pgfscope}%
\pgfsys@transformshift{1.767592in}{2.391709in}%
\pgfsys@useobject{currentmarker}{}%
\end{pgfscope}%
\begin{pgfscope}%
\pgfsys@transformshift{1.775770in}{2.389907in}%
\pgfsys@useobject{currentmarker}{}%
\end{pgfscope}%
\begin{pgfscope}%
\pgfsys@transformshift{1.784813in}{2.391317in}%
\pgfsys@useobject{currentmarker}{}%
\end{pgfscope}%
\begin{pgfscope}%
\pgfsys@transformshift{1.789675in}{2.390017in}%
\pgfsys@useobject{currentmarker}{}%
\end{pgfscope}%
\begin{pgfscope}%
\pgfsys@transformshift{1.792391in}{2.390556in}%
\pgfsys@useobject{currentmarker}{}%
\end{pgfscope}%
\begin{pgfscope}%
\pgfsys@transformshift{1.796372in}{2.390274in}%
\pgfsys@useobject{currentmarker}{}%
\end{pgfscope}%
\begin{pgfscope}%
\pgfsys@transformshift{1.802135in}{2.391460in}%
\pgfsys@useobject{currentmarker}{}%
\end{pgfscope}%
\begin{pgfscope}%
\pgfsys@transformshift{1.809830in}{2.391487in}%
\pgfsys@useobject{currentmarker}{}%
\end{pgfscope}%
\begin{pgfscope}%
\pgfsys@transformshift{1.818664in}{2.392707in}%
\pgfsys@useobject{currentmarker}{}%
\end{pgfscope}%
\begin{pgfscope}%
\pgfsys@transformshift{1.823569in}{2.392739in}%
\pgfsys@useobject{currentmarker}{}%
\end{pgfscope}%
\begin{pgfscope}%
\pgfsys@transformshift{1.829404in}{2.391396in}%
\pgfsys@useobject{currentmarker}{}%
\end{pgfscope}%
\begin{pgfscope}%
\pgfsys@transformshift{1.832694in}{2.391245in}%
\pgfsys@useobject{currentmarker}{}%
\end{pgfscope}%
\begin{pgfscope}%
\pgfsys@transformshift{1.836791in}{2.390789in}%
\pgfsys@useobject{currentmarker}{}%
\end{pgfscope}%
\begin{pgfscope}%
\pgfsys@transformshift{1.839056in}{2.390693in}%
\pgfsys@useobject{currentmarker}{}%
\end{pgfscope}%
\begin{pgfscope}%
\pgfsys@transformshift{1.840277in}{2.390441in}%
\pgfsys@useobject{currentmarker}{}%
\end{pgfscope}%
\begin{pgfscope}%
\pgfsys@transformshift{1.842329in}{2.390458in}%
\pgfsys@useobject{currentmarker}{}%
\end{pgfscope}%
\begin{pgfscope}%
\pgfsys@transformshift{1.845301in}{2.390185in}%
\pgfsys@useobject{currentmarker}{}%
\end{pgfscope}%
\begin{pgfscope}%
\pgfsys@transformshift{1.849646in}{2.390482in}%
\pgfsys@useobject{currentmarker}{}%
\end{pgfscope}%
\begin{pgfscope}%
\pgfsys@transformshift{1.855231in}{2.390369in}%
\pgfsys@useobject{currentmarker}{}%
\end{pgfscope}%
\begin{pgfscope}%
\pgfsys@transformshift{1.862575in}{2.390340in}%
\pgfsys@useobject{currentmarker}{}%
\end{pgfscope}%
\begin{pgfscope}%
\pgfsys@transformshift{1.872169in}{2.390585in}%
\pgfsys@useobject{currentmarker}{}%
\end{pgfscope}%
\begin{pgfscope}%
\pgfsys@transformshift{1.882636in}{2.390679in}%
\pgfsys@useobject{currentmarker}{}%
\end{pgfscope}%
\begin{pgfscope}%
\pgfsys@transformshift{1.894111in}{2.390850in}%
\pgfsys@useobject{currentmarker}{}%
\end{pgfscope}%
\begin{pgfscope}%
\pgfsys@transformshift{1.906344in}{2.389594in}%
\pgfsys@useobject{currentmarker}{}%
\end{pgfscope}%
\begin{pgfscope}%
\pgfsys@transformshift{1.913107in}{2.389516in}%
\pgfsys@useobject{currentmarker}{}%
\end{pgfscope}%
\begin{pgfscope}%
\pgfsys@transformshift{1.920636in}{2.389452in}%
\pgfsys@useobject{currentmarker}{}%
\end{pgfscope}%
\begin{pgfscope}%
\pgfsys@transformshift{1.929176in}{2.389083in}%
\pgfsys@useobject{currentmarker}{}%
\end{pgfscope}%
\begin{pgfscope}%
\pgfsys@transformshift{1.938993in}{2.388369in}%
\pgfsys@useobject{currentmarker}{}%
\end{pgfscope}%
\begin{pgfscope}%
\pgfsys@transformshift{1.951289in}{2.388531in}%
\pgfsys@useobject{currentmarker}{}%
\end{pgfscope}%
\begin{pgfscope}%
\pgfsys@transformshift{1.967272in}{2.389636in}%
\pgfsys@useobject{currentmarker}{}%
\end{pgfscope}%
\begin{pgfscope}%
\pgfsys@transformshift{1.986184in}{2.386936in}%
\pgfsys@useobject{currentmarker}{}%
\end{pgfscope}%
\begin{pgfscope}%
\pgfsys@transformshift{2.007162in}{2.387643in}%
\pgfsys@useobject{currentmarker}{}%
\end{pgfscope}%
\begin{pgfscope}%
\pgfsys@transformshift{2.028881in}{2.387008in}%
\pgfsys@useobject{currentmarker}{}%
\end{pgfscope}%
\begin{pgfscope}%
\pgfsys@transformshift{2.040828in}{2.386768in}%
\pgfsys@useobject{currentmarker}{}%
\end{pgfscope}%
\begin{pgfscope}%
\pgfsys@transformshift{2.053681in}{2.385226in}%
\pgfsys@useobject{currentmarker}{}%
\end{pgfscope}%
\begin{pgfscope}%
\pgfsys@transformshift{2.060767in}{2.384534in}%
\pgfsys@useobject{currentmarker}{}%
\end{pgfscope}%
\begin{pgfscope}%
\pgfsys@transformshift{2.064659in}{2.384102in}%
\pgfsys@useobject{currentmarker}{}%
\end{pgfscope}%
\begin{pgfscope}%
\pgfsys@transformshift{2.069597in}{2.383887in}%
\pgfsys@useobject{currentmarker}{}%
\end{pgfscope}%
\begin{pgfscope}%
\pgfsys@transformshift{2.075697in}{2.384020in}%
\pgfsys@useobject{currentmarker}{}%
\end{pgfscope}%
\begin{pgfscope}%
\pgfsys@transformshift{2.082731in}{2.383451in}%
\pgfsys@useobject{currentmarker}{}%
\end{pgfscope}%
\begin{pgfscope}%
\pgfsys@transformshift{2.091090in}{2.383625in}%
\pgfsys@useobject{currentmarker}{}%
\end{pgfscope}%
\begin{pgfscope}%
\pgfsys@transformshift{2.100687in}{2.383685in}%
\pgfsys@useobject{currentmarker}{}%
\end{pgfscope}%
\begin{pgfscope}%
\pgfsys@transformshift{2.111972in}{2.384911in}%
\pgfsys@useobject{currentmarker}{}%
\end{pgfscope}%
\begin{pgfscope}%
\pgfsys@transformshift{2.124734in}{2.383445in}%
\pgfsys@useobject{currentmarker}{}%
\end{pgfscope}%
\begin{pgfscope}%
\pgfsys@transformshift{2.138936in}{2.382486in}%
\pgfsys@useobject{currentmarker}{}%
\end{pgfscope}%
\begin{pgfscope}%
\pgfsys@transformshift{2.154046in}{2.383197in}%
\pgfsys@useobject{currentmarker}{}%
\end{pgfscope}%
\begin{pgfscope}%
\pgfsys@transformshift{2.162348in}{2.383748in}%
\pgfsys@useobject{currentmarker}{}%
\end{pgfscope}%
\begin{pgfscope}%
\pgfsys@transformshift{2.171315in}{2.383039in}%
\pgfsys@useobject{currentmarker}{}%
\end{pgfscope}%
\begin{pgfscope}%
\pgfsys@transformshift{2.176202in}{2.383805in}%
\pgfsys@useobject{currentmarker}{}%
\end{pgfscope}%
\begin{pgfscope}%
\pgfsys@transformshift{2.181812in}{2.383628in}%
\pgfsys@useobject{currentmarker}{}%
\end{pgfscope}%
\begin{pgfscope}%
\pgfsys@transformshift{2.184840in}{2.384233in}%
\pgfsys@useobject{currentmarker}{}%
\end{pgfscope}%
\begin{pgfscope}%
\pgfsys@transformshift{2.188927in}{2.384723in}%
\pgfsys@useobject{currentmarker}{}%
\end{pgfscope}%
\begin{pgfscope}%
\pgfsys@transformshift{2.194538in}{2.385779in}%
\pgfsys@useobject{currentmarker}{}%
\end{pgfscope}%
\begin{pgfscope}%
\pgfsys@transformshift{2.202132in}{2.386329in}%
\pgfsys@useobject{currentmarker}{}%
\end{pgfscope}%
\begin{pgfscope}%
\pgfsys@transformshift{2.210958in}{2.388414in}%
\pgfsys@useobject{currentmarker}{}%
\end{pgfscope}%
\begin{pgfscope}%
\pgfsys@transformshift{2.222320in}{2.391539in}%
\pgfsys@useobject{currentmarker}{}%
\end{pgfscope}%
\begin{pgfscope}%
\pgfsys@transformshift{2.235110in}{2.391228in}%
\pgfsys@useobject{currentmarker}{}%
\end{pgfscope}%
\begin{pgfscope}%
\pgfsys@transformshift{2.248546in}{2.394762in}%
\pgfsys@useobject{currentmarker}{}%
\end{pgfscope}%
\begin{pgfscope}%
\pgfsys@transformshift{2.256076in}{2.393469in}%
\pgfsys@useobject{currentmarker}{}%
\end{pgfscope}%
\begin{pgfscope}%
\pgfsys@transformshift{2.260213in}{2.394211in}%
\pgfsys@useobject{currentmarker}{}%
\end{pgfscope}%
\begin{pgfscope}%
\pgfsys@transformshift{2.265011in}{2.393305in}%
\pgfsys@useobject{currentmarker}{}%
\end{pgfscope}%
\begin{pgfscope}%
\pgfsys@transformshift{2.271160in}{2.394315in}%
\pgfsys@useobject{currentmarker}{}%
\end{pgfscope}%
\begin{pgfscope}%
\pgfsys@transformshift{2.278296in}{2.394358in}%
\pgfsys@useobject{currentmarker}{}%
\end{pgfscope}%
\begin{pgfscope}%
\pgfsys@transformshift{2.287293in}{2.395276in}%
\pgfsys@useobject{currentmarker}{}%
\end{pgfscope}%
\begin{pgfscope}%
\pgfsys@transformshift{2.299439in}{2.395282in}%
\pgfsys@useobject{currentmarker}{}%
\end{pgfscope}%
\begin{pgfscope}%
\pgfsys@transformshift{2.312613in}{2.396213in}%
\pgfsys@useobject{currentmarker}{}%
\end{pgfscope}%
\begin{pgfscope}%
\pgfsys@transformshift{2.319876in}{2.396312in}%
\pgfsys@useobject{currentmarker}{}%
\end{pgfscope}%
\begin{pgfscope}%
\pgfsys@transformshift{2.328128in}{2.394952in}%
\pgfsys@useobject{currentmarker}{}%
\end{pgfscope}%
\begin{pgfscope}%
\pgfsys@transformshift{2.332722in}{2.395184in}%
\pgfsys@useobject{currentmarker}{}%
\end{pgfscope}%
\begin{pgfscope}%
\pgfsys@transformshift{2.335252in}{2.395117in}%
\pgfsys@useobject{currentmarker}{}%
\end{pgfscope}%
\begin{pgfscope}%
\pgfsys@transformshift{2.338659in}{2.395009in}%
\pgfsys@useobject{currentmarker}{}%
\end{pgfscope}%
\begin{pgfscope}%
\pgfsys@transformshift{2.340534in}{2.395015in}%
\pgfsys@useobject{currentmarker}{}%
\end{pgfscope}%
\begin{pgfscope}%
\pgfsys@transformshift{2.343641in}{2.394828in}%
\pgfsys@useobject{currentmarker}{}%
\end{pgfscope}%
\begin{pgfscope}%
\pgfsys@transformshift{2.348891in}{2.394711in}%
\pgfsys@useobject{currentmarker}{}%
\end{pgfscope}%
\begin{pgfscope}%
\pgfsys@transformshift{2.356180in}{2.393979in}%
\pgfsys@useobject{currentmarker}{}%
\end{pgfscope}%
\begin{pgfscope}%
\pgfsys@transformshift{2.366015in}{2.394187in}%
\pgfsys@useobject{currentmarker}{}%
\end{pgfscope}%
\begin{pgfscope}%
\pgfsys@transformshift{2.378549in}{2.393096in}%
\pgfsys@useobject{currentmarker}{}%
\end{pgfscope}%
\begin{pgfscope}%
\pgfsys@transformshift{2.393228in}{2.394123in}%
\pgfsys@useobject{currentmarker}{}%
\end{pgfscope}%
\begin{pgfscope}%
\pgfsys@transformshift{2.408453in}{2.391359in}%
\pgfsys@useobject{currentmarker}{}%
\end{pgfscope}%
\begin{pgfscope}%
\pgfsys@transformshift{2.425226in}{2.393387in}%
\pgfsys@useobject{currentmarker}{}%
\end{pgfscope}%
\begin{pgfscope}%
\pgfsys@transformshift{2.442313in}{2.389020in}%
\pgfsys@useobject{currentmarker}{}%
\end{pgfscope}%
\begin{pgfscope}%
\pgfsys@transformshift{2.451966in}{2.389965in}%
\pgfsys@useobject{currentmarker}{}%
\end{pgfscope}%
\begin{pgfscope}%
\pgfsys@transformshift{2.462937in}{2.388059in}%
\pgfsys@useobject{currentmarker}{}%
\end{pgfscope}%
\begin{pgfscope}%
\pgfsys@transformshift{2.469061in}{2.388044in}%
\pgfsys@useobject{currentmarker}{}%
\end{pgfscope}%
\begin{pgfscope}%
\pgfsys@transformshift{2.472425in}{2.387858in}%
\pgfsys@useobject{currentmarker}{}%
\end{pgfscope}%
\begin{pgfscope}%
\pgfsys@transformshift{2.476851in}{2.387436in}%
\pgfsys@useobject{currentmarker}{}%
\end{pgfscope}%
\begin{pgfscope}%
\pgfsys@transformshift{2.479295in}{2.387507in}%
\pgfsys@useobject{currentmarker}{}%
\end{pgfscope}%
\begin{pgfscope}%
\pgfsys@transformshift{2.483303in}{2.387443in}%
\pgfsys@useobject{currentmarker}{}%
\end{pgfscope}%
\begin{pgfscope}%
\pgfsys@transformshift{2.489011in}{2.387051in}%
\pgfsys@useobject{currentmarker}{}%
\end{pgfscope}%
\begin{pgfscope}%
\pgfsys@transformshift{2.496771in}{2.386878in}%
\pgfsys@useobject{currentmarker}{}%
\end{pgfscope}%
\begin{pgfscope}%
\pgfsys@transformshift{2.506683in}{2.386892in}%
\pgfsys@useobject{currentmarker}{}%
\end{pgfscope}%
\begin{pgfscope}%
\pgfsys@transformshift{2.518744in}{2.384800in}%
\pgfsys@useobject{currentmarker}{}%
\end{pgfscope}%
\begin{pgfscope}%
\pgfsys@transformshift{2.532840in}{2.383839in}%
\pgfsys@useobject{currentmarker}{}%
\end{pgfscope}%
\begin{pgfscope}%
\pgfsys@transformshift{2.548201in}{2.381594in}%
\pgfsys@useobject{currentmarker}{}%
\end{pgfscope}%
\begin{pgfscope}%
\pgfsys@transformshift{2.566419in}{2.379648in}%
\pgfsys@useobject{currentmarker}{}%
\end{pgfscope}%
\begin{pgfscope}%
\pgfsys@transformshift{2.586336in}{2.377188in}%
\pgfsys@useobject{currentmarker}{}%
\end{pgfscope}%
\begin{pgfscope}%
\pgfsys@transformshift{2.608855in}{2.374127in}%
\pgfsys@useobject{currentmarker}{}%
\end{pgfscope}%
\begin{pgfscope}%
\pgfsys@transformshift{2.635156in}{2.375854in}%
\pgfsys@useobject{currentmarker}{}%
\end{pgfscope}%
\begin{pgfscope}%
\pgfsys@transformshift{2.664994in}{2.375866in}%
\pgfsys@useobject{currentmarker}{}%
\end{pgfscope}%
\begin{pgfscope}%
\pgfsys@transformshift{2.698038in}{2.374286in}%
\pgfsys@useobject{currentmarker}{}%
\end{pgfscope}%
\begin{pgfscope}%
\pgfsys@transformshift{2.733860in}{2.373884in}%
\pgfsys@useobject{currentmarker}{}%
\end{pgfscope}%
\begin{pgfscope}%
\pgfsys@transformshift{2.770827in}{2.370817in}%
\pgfsys@useobject{currentmarker}{}%
\end{pgfscope}%
\begin{pgfscope}%
\pgfsys@transformshift{2.810117in}{2.371584in}%
\pgfsys@useobject{currentmarker}{}%
\end{pgfscope}%
\begin{pgfscope}%
\pgfsys@transformshift{2.850722in}{2.370269in}%
\pgfsys@useobject{currentmarker}{}%
\end{pgfscope}%
\begin{pgfscope}%
\pgfsys@transformshift{2.892565in}{2.367797in}%
\pgfsys@useobject{currentmarker}{}%
\end{pgfscope}%
\begin{pgfscope}%
\pgfsys@transformshift{2.935198in}{2.366686in}%
\pgfsys@useobject{currentmarker}{}%
\end{pgfscope}%
\begin{pgfscope}%
\pgfsys@transformshift{2.978467in}{2.365169in}%
\pgfsys@useobject{currentmarker}{}%
\end{pgfscope}%
\begin{pgfscope}%
\pgfsys@transformshift{3.022730in}{2.359694in}%
\pgfsys@useobject{currentmarker}{}%
\end{pgfscope}%
\begin{pgfscope}%
\pgfsys@transformshift{3.047259in}{2.359923in}%
\pgfsys@useobject{currentmarker}{}%
\end{pgfscope}%
\begin{pgfscope}%
\pgfsys@transformshift{3.071926in}{2.354169in}%
\pgfsys@useobject{currentmarker}{}%
\end{pgfscope}%
\begin{pgfscope}%
\pgfsys@transformshift{3.085841in}{2.354828in}%
\pgfsys@useobject{currentmarker}{}%
\end{pgfscope}%
\begin{pgfscope}%
\pgfsys@transformshift{3.093467in}{2.354078in}%
\pgfsys@useobject{currentmarker}{}%
\end{pgfscope}%
\begin{pgfscope}%
\pgfsys@transformshift{3.101880in}{2.352993in}%
\pgfsys@useobject{currentmarker}{}%
\end{pgfscope}%
\begin{pgfscope}%
\pgfsys@transformshift{3.111369in}{2.352791in}%
\pgfsys@useobject{currentmarker}{}%
\end{pgfscope}%
\begin{pgfscope}%
\pgfsys@transformshift{3.122161in}{2.352867in}%
\pgfsys@useobject{currentmarker}{}%
\end{pgfscope}%
\begin{pgfscope}%
\pgfsys@transformshift{3.134396in}{2.351963in}%
\pgfsys@useobject{currentmarker}{}%
\end{pgfscope}%
\begin{pgfscope}%
\pgfsys@transformshift{3.147646in}{2.351042in}%
\pgfsys@useobject{currentmarker}{}%
\end{pgfscope}%
\begin{pgfscope}%
\pgfsys@transformshift{3.162499in}{2.350703in}%
\pgfsys@useobject{currentmarker}{}%
\end{pgfscope}%
\begin{pgfscope}%
\pgfsys@transformshift{3.180016in}{2.351681in}%
\pgfsys@useobject{currentmarker}{}%
\end{pgfscope}%
\begin{pgfscope}%
\pgfsys@transformshift{3.199114in}{2.352471in}%
\pgfsys@useobject{currentmarker}{}%
\end{pgfscope}%
\begin{pgfscope}%
\pgfsys@transformshift{3.219194in}{2.351913in}%
\pgfsys@useobject{currentmarker}{}%
\end{pgfscope}%
\begin{pgfscope}%
\pgfsys@transformshift{3.240220in}{2.353175in}%
\pgfsys@useobject{currentmarker}{}%
\end{pgfscope}%
\begin{pgfscope}%
\pgfsys@transformshift{3.262149in}{2.351948in}%
\pgfsys@useobject{currentmarker}{}%
\end{pgfscope}%
\begin{pgfscope}%
\pgfsys@transformshift{3.284975in}{2.352991in}%
\pgfsys@useobject{currentmarker}{}%
\end{pgfscope}%
\begin{pgfscope}%
\pgfsys@transformshift{3.308782in}{2.350312in}%
\pgfsys@useobject{currentmarker}{}%
\end{pgfscope}%
\begin{pgfscope}%
\pgfsys@transformshift{3.321785in}{2.352443in}%
\pgfsys@useobject{currentmarker}{}%
\end{pgfscope}%
\begin{pgfscope}%
\pgfsys@transformshift{3.335540in}{2.350458in}%
\pgfsys@useobject{currentmarker}{}%
\end{pgfscope}%
\begin{pgfscope}%
\pgfsys@transformshift{3.343157in}{2.351093in}%
\pgfsys@useobject{currentmarker}{}%
\end{pgfscope}%
\begin{pgfscope}%
\pgfsys@transformshift{3.347344in}{2.350715in}%
\pgfsys@useobject{currentmarker}{}%
\end{pgfscope}%
\begin{pgfscope}%
\pgfsys@transformshift{3.352601in}{2.351138in}%
\pgfsys@useobject{currentmarker}{}%
\end{pgfscope}%
\begin{pgfscope}%
\pgfsys@transformshift{3.359187in}{2.351063in}%
\pgfsys@useobject{currentmarker}{}%
\end{pgfscope}%
\begin{pgfscope}%
\pgfsys@transformshift{3.362797in}{2.351375in}%
\pgfsys@useobject{currentmarker}{}%
\end{pgfscope}%
\begin{pgfscope}%
\pgfsys@transformshift{3.369087in}{2.350329in}%
\pgfsys@useobject{currentmarker}{}%
\end{pgfscope}%
\begin{pgfscope}%
\pgfsys@transformshift{3.376976in}{2.350796in}%
\pgfsys@useobject{currentmarker}{}%
\end{pgfscope}%
\begin{pgfscope}%
\pgfsys@transformshift{3.385700in}{2.349905in}%
\pgfsys@useobject{currentmarker}{}%
\end{pgfscope}%
\begin{pgfscope}%
\pgfsys@transformshift{3.395661in}{2.349996in}%
\pgfsys@useobject{currentmarker}{}%
\end{pgfscope}%
\begin{pgfscope}%
\pgfsys@transformshift{3.406591in}{2.349382in}%
\pgfsys@useobject{currentmarker}{}%
\end{pgfscope}%
\begin{pgfscope}%
\pgfsys@transformshift{3.412597in}{2.349802in}%
\pgfsys@useobject{currentmarker}{}%
\end{pgfscope}%
\begin{pgfscope}%
\pgfsys@transformshift{3.420315in}{2.348792in}%
\pgfsys@useobject{currentmarker}{}%
\end{pgfscope}%
\begin{pgfscope}%
\pgfsys@transformshift{3.429159in}{2.349208in}%
\pgfsys@useobject{currentmarker}{}%
\end{pgfscope}%
\begin{pgfscope}%
\pgfsys@transformshift{3.439335in}{2.348321in}%
\pgfsys@useobject{currentmarker}{}%
\end{pgfscope}%
\begin{pgfscope}%
\pgfsys@transformshift{3.451228in}{2.348368in}%
\pgfsys@useobject{currentmarker}{}%
\end{pgfscope}%
\begin{pgfscope}%
\pgfsys@transformshift{3.464589in}{2.348168in}%
\pgfsys@useobject{currentmarker}{}%
\end{pgfscope}%
\begin{pgfscope}%
\pgfsys@transformshift{3.478784in}{2.348950in}%
\pgfsys@useobject{currentmarker}{}%
\end{pgfscope}%
\begin{pgfscope}%
\pgfsys@transformshift{3.494816in}{2.346899in}%
\pgfsys@useobject{currentmarker}{}%
\end{pgfscope}%
\begin{pgfscope}%
\pgfsys@transformshift{3.511950in}{2.348568in}%
\pgfsys@useobject{currentmarker}{}%
\end{pgfscope}%
\begin{pgfscope}%
\pgfsys@transformshift{3.530164in}{2.347051in}%
\pgfsys@useobject{currentmarker}{}%
\end{pgfscope}%
\begin{pgfscope}%
\pgfsys@transformshift{3.549410in}{2.347310in}%
\pgfsys@useobject{currentmarker}{}%
\end{pgfscope}%
\begin{pgfscope}%
\pgfsys@transformshift{3.569523in}{2.349173in}%
\pgfsys@useobject{currentmarker}{}%
\end{pgfscope}%
\begin{pgfscope}%
\pgfsys@transformshift{3.590400in}{2.350564in}%
\pgfsys@useobject{currentmarker}{}%
\end{pgfscope}%
\begin{pgfscope}%
\pgfsys@transformshift{3.612889in}{2.351512in}%
\pgfsys@useobject{currentmarker}{}%
\end{pgfscope}%
\begin{pgfscope}%
\pgfsys@transformshift{3.625260in}{2.352001in}%
\pgfsys@useobject{currentmarker}{}%
\end{pgfscope}%
\begin{pgfscope}%
\pgfsys@transformshift{3.638149in}{2.349679in}%
\pgfsys@useobject{currentmarker}{}%
\end{pgfscope}%
\begin{pgfscope}%
\pgfsys@transformshift{3.645330in}{2.350234in}%
\pgfsys@useobject{currentmarker}{}%
\end{pgfscope}%
\begin{pgfscope}%
\pgfsys@transformshift{3.653100in}{2.348232in}%
\pgfsys@useobject{currentmarker}{}%
\end{pgfscope}%
\begin{pgfscope}%
\pgfsys@transformshift{3.657511in}{2.348347in}%
\pgfsys@useobject{currentmarker}{}%
\end{pgfscope}%
\begin{pgfscope}%
\pgfsys@transformshift{3.659890in}{2.347866in}%
\pgfsys@useobject{currentmarker}{}%
\end{pgfscope}%
\begin{pgfscope}%
\pgfsys@transformshift{3.663019in}{2.348124in}%
\pgfsys@useobject{currentmarker}{}%
\end{pgfscope}%
\begin{pgfscope}%
\pgfsys@transformshift{3.664731in}{2.347901in}%
\pgfsys@useobject{currentmarker}{}%
\end{pgfscope}%
\begin{pgfscope}%
\pgfsys@transformshift{3.667520in}{2.347928in}%
\pgfsys@useobject{currentmarker}{}%
\end{pgfscope}%
\begin{pgfscope}%
\pgfsys@transformshift{3.671177in}{2.347822in}%
\pgfsys@useobject{currentmarker}{}%
\end{pgfscope}%
\begin{pgfscope}%
\pgfsys@transformshift{3.675516in}{2.347434in}%
\pgfsys@useobject{currentmarker}{}%
\end{pgfscope}%
\begin{pgfscope}%
\pgfsys@transformshift{3.681044in}{2.348056in}%
\pgfsys@useobject{currentmarker}{}%
\end{pgfscope}%
\begin{pgfscope}%
\pgfsys@transformshift{3.687901in}{2.347263in}%
\pgfsys@useobject{currentmarker}{}%
\end{pgfscope}%
\begin{pgfscope}%
\pgfsys@transformshift{3.696959in}{2.347755in}%
\pgfsys@useobject{currentmarker}{}%
\end{pgfscope}%
\begin{pgfscope}%
\pgfsys@transformshift{3.707670in}{2.346850in}%
\pgfsys@useobject{currentmarker}{}%
\end{pgfscope}%
\begin{pgfscope}%
\pgfsys@transformshift{3.720057in}{2.346766in}%
\pgfsys@useobject{currentmarker}{}%
\end{pgfscope}%
\begin{pgfscope}%
\pgfsys@transformshift{3.734252in}{2.345531in}%
\pgfsys@useobject{currentmarker}{}%
\end{pgfscope}%
\begin{pgfscope}%
\pgfsys@transformshift{3.751020in}{2.345465in}%
\pgfsys@useobject{currentmarker}{}%
\end{pgfscope}%
\begin{pgfscope}%
\pgfsys@transformshift{3.769182in}{2.343294in}%
\pgfsys@useobject{currentmarker}{}%
\end{pgfscope}%
\begin{pgfscope}%
\pgfsys@transformshift{3.789618in}{2.342846in}%
\pgfsys@useobject{currentmarker}{}%
\end{pgfscope}%
\begin{pgfscope}%
\pgfsys@transformshift{3.812291in}{2.343018in}%
\pgfsys@useobject{currentmarker}{}%
\end{pgfscope}%
\begin{pgfscope}%
\pgfsys@transformshift{3.837809in}{2.342063in}%
\pgfsys@useobject{currentmarker}{}%
\end{pgfscope}%
\begin{pgfscope}%
\pgfsys@transformshift{3.864977in}{2.343300in}%
\pgfsys@useobject{currentmarker}{}%
\end{pgfscope}%
\begin{pgfscope}%
\pgfsys@transformshift{3.893633in}{2.343666in}%
\pgfsys@useobject{currentmarker}{}%
\end{pgfscope}%
\begin{pgfscope}%
\pgfsys@transformshift{3.923671in}{2.345492in}%
\pgfsys@useobject{currentmarker}{}%
\end{pgfscope}%
\begin{pgfscope}%
\pgfsys@transformshift{3.955551in}{2.344930in}%
\pgfsys@useobject{currentmarker}{}%
\end{pgfscope}%
\begin{pgfscope}%
\pgfsys@transformshift{3.988758in}{2.346013in}%
\pgfsys@useobject{currentmarker}{}%
\end{pgfscope}%
\begin{pgfscope}%
\pgfsys@transformshift{4.023462in}{2.343250in}%
\pgfsys@useobject{currentmarker}{}%
\end{pgfscope}%
\begin{pgfscope}%
\pgfsys@transformshift{4.060031in}{2.344794in}%
\pgfsys@useobject{currentmarker}{}%
\end{pgfscope}%
\begin{pgfscope}%
\pgfsys@transformshift{4.098185in}{2.343716in}%
\pgfsys@useobject{currentmarker}{}%
\end{pgfscope}%
\begin{pgfscope}%
\pgfsys@transformshift{4.137630in}{2.345799in}%
\pgfsys@useobject{currentmarker}{}%
\end{pgfscope}%
\begin{pgfscope}%
\pgfsys@transformshift{4.178453in}{2.347534in}%
\pgfsys@useobject{currentmarker}{}%
\end{pgfscope}%
\begin{pgfscope}%
\pgfsys@transformshift{4.220980in}{2.346837in}%
\pgfsys@useobject{currentmarker}{}%
\end{pgfscope}%
\begin{pgfscope}%
\pgfsys@transformshift{4.264466in}{2.349966in}%
\pgfsys@useobject{currentmarker}{}%
\end{pgfscope}%
\begin{pgfscope}%
\pgfsys@transformshift{4.308745in}{2.345755in}%
\pgfsys@useobject{currentmarker}{}%
\end{pgfscope}%
\begin{pgfscope}%
\pgfsys@transformshift{4.332960in}{2.349231in}%
\pgfsys@useobject{currentmarker}{}%
\end{pgfscope}%
\begin{pgfscope}%
\pgfsys@transformshift{4.358025in}{2.346947in}%
\pgfsys@useobject{currentmarker}{}%
\end{pgfscope}%
\begin{pgfscope}%
\pgfsys@transformshift{4.371580in}{2.349757in}%
\pgfsys@useobject{currentmarker}{}%
\end{pgfscope}%
\begin{pgfscope}%
\pgfsys@transformshift{4.379189in}{2.349493in}%
\pgfsys@useobject{currentmarker}{}%
\end{pgfscope}%
\begin{pgfscope}%
\pgfsys@transformshift{4.387319in}{2.350854in}%
\pgfsys@useobject{currentmarker}{}%
\end{pgfscope}%
\begin{pgfscope}%
\pgfsys@transformshift{4.391852in}{2.350909in}%
\pgfsys@useobject{currentmarker}{}%
\end{pgfscope}%
\begin{pgfscope}%
\pgfsys@transformshift{4.397551in}{2.351011in}%
\pgfsys@useobject{currentmarker}{}%
\end{pgfscope}%
\begin{pgfscope}%
\pgfsys@transformshift{4.400683in}{2.351144in}%
\pgfsys@useobject{currentmarker}{}%
\end{pgfscope}%
\begin{pgfscope}%
\pgfsys@transformshift{4.404832in}{2.351243in}%
\pgfsys@useobject{currentmarker}{}%
\end{pgfscope}%
\begin{pgfscope}%
\pgfsys@transformshift{4.407098in}{2.351514in}%
\pgfsys@useobject{currentmarker}{}%
\end{pgfscope}%
\begin{pgfscope}%
\pgfsys@transformshift{4.411359in}{2.350636in}%
\pgfsys@useobject{currentmarker}{}%
\end{pgfscope}%
\begin{pgfscope}%
\pgfsys@transformshift{4.417370in}{2.351056in}%
\pgfsys@useobject{currentmarker}{}%
\end{pgfscope}%
\begin{pgfscope}%
\pgfsys@transformshift{4.424537in}{2.350103in}%
\pgfsys@useobject{currentmarker}{}%
\end{pgfscope}%
\begin{pgfscope}%
\pgfsys@transformshift{4.432390in}{2.350504in}%
\pgfsys@useobject{currentmarker}{}%
\end{pgfscope}%
\begin{pgfscope}%
\pgfsys@transformshift{4.442857in}{2.348947in}%
\pgfsys@useobject{currentmarker}{}%
\end{pgfscope}%
\begin{pgfscope}%
\pgfsys@transformshift{4.455470in}{2.349062in}%
\pgfsys@useobject{currentmarker}{}%
\end{pgfscope}%
\begin{pgfscope}%
\pgfsys@transformshift{4.469717in}{2.346698in}%
\pgfsys@useobject{currentmarker}{}%
\end{pgfscope}%
\begin{pgfscope}%
\pgfsys@transformshift{4.486482in}{2.347546in}%
\pgfsys@useobject{currentmarker}{}%
\end{pgfscope}%
\begin{pgfscope}%
\pgfsys@transformshift{4.504672in}{2.346597in}%
\pgfsys@useobject{currentmarker}{}%
\end{pgfscope}%
\begin{pgfscope}%
\pgfsys@transformshift{4.523994in}{2.346299in}%
\pgfsys@useobject{currentmarker}{}%
\end{pgfscope}%
\begin{pgfscope}%
\pgfsys@transformshift{4.547183in}{2.345321in}%
\pgfsys@useobject{currentmarker}{}%
\end{pgfscope}%
\begin{pgfscope}%
\pgfsys@transformshift{4.572588in}{2.344944in}%
\pgfsys@useobject{currentmarker}{}%
\end{pgfscope}%
\begin{pgfscope}%
\pgfsys@transformshift{4.600595in}{2.345682in}%
\pgfsys@useobject{currentmarker}{}%
\end{pgfscope}%
\begin{pgfscope}%
\pgfsys@transformshift{4.630965in}{2.346591in}%
\pgfsys@useobject{currentmarker}{}%
\end{pgfscope}%
\begin{pgfscope}%
\pgfsys@transformshift{4.663093in}{2.345144in}%
\pgfsys@useobject{currentmarker}{}%
\end{pgfscope}%
\begin{pgfscope}%
\pgfsys@transformshift{4.698210in}{2.346472in}%
\pgfsys@useobject{currentmarker}{}%
\end{pgfscope}%
\begin{pgfscope}%
\pgfsys@transformshift{4.735232in}{2.344265in}%
\pgfsys@useobject{currentmarker}{}%
\end{pgfscope}%
\begin{pgfscope}%
\pgfsys@transformshift{4.774584in}{2.346413in}%
\pgfsys@useobject{currentmarker}{}%
\end{pgfscope}%
\begin{pgfscope}%
\pgfsys@transformshift{4.815519in}{2.341666in}%
\pgfsys@useobject{currentmarker}{}%
\end{pgfscope}%
\begin{pgfscope}%
\pgfsys@transformshift{4.858979in}{2.344520in}%
\pgfsys@useobject{currentmarker}{}%
\end{pgfscope}%
\begin{pgfscope}%
\pgfsys@transformshift{4.903392in}{2.340251in}%
\pgfsys@useobject{currentmarker}{}%
\end{pgfscope}%
\begin{pgfscope}%
\pgfsys@transformshift{4.949234in}{2.341940in}%
\pgfsys@useobject{currentmarker}{}%
\end{pgfscope}%
\begin{pgfscope}%
\pgfsys@transformshift{4.995802in}{2.337136in}%
\pgfsys@useobject{currentmarker}{}%
\end{pgfscope}%
\begin{pgfscope}%
\pgfsys@transformshift{5.044023in}{2.339920in}%
\pgfsys@useobject{currentmarker}{}%
\end{pgfscope}%
\begin{pgfscope}%
\pgfsys@transformshift{5.093680in}{2.336423in}%
\pgfsys@useobject{currentmarker}{}%
\end{pgfscope}%
\begin{pgfscope}%
\pgfsys@transformshift{5.144027in}{2.332948in}%
\pgfsys@useobject{currentmarker}{}%
\end{pgfscope}%
\begin{pgfscope}%
\pgfsys@transformshift{5.195812in}{2.331479in}%
\pgfsys@useobject{currentmarker}{}%
\end{pgfscope}%
\begin{pgfscope}%
\pgfsys@transformshift{5.224304in}{2.331784in}%
\pgfsys@useobject{currentmarker}{}%
\end{pgfscope}%
\begin{pgfscope}%
\pgfsys@transformshift{5.254026in}{2.331547in}%
\pgfsys@useobject{currentmarker}{}%
\end{pgfscope}%
\begin{pgfscope}%
\pgfsys@transformshift{5.270267in}{2.329687in}%
\pgfsys@useobject{currentmarker}{}%
\end{pgfscope}%
\begin{pgfscope}%
\pgfsys@transformshift{5.279249in}{2.330086in}%
\pgfsys@useobject{currentmarker}{}%
\end{pgfscope}%
\begin{pgfscope}%
\pgfsys@transformshift{5.289503in}{2.330058in}%
\pgfsys@useobject{currentmarker}{}%
\end{pgfscope}%
\begin{pgfscope}%
\pgfsys@transformshift{5.300751in}{2.331587in}%
\pgfsys@useobject{currentmarker}{}%
\end{pgfscope}%
\begin{pgfscope}%
\pgfsys@transformshift{5.306983in}{2.331951in}%
\pgfsys@useobject{currentmarker}{}%
\end{pgfscope}%
\begin{pgfscope}%
\pgfsys@transformshift{5.310384in}{2.332424in}%
\pgfsys@useobject{currentmarker}{}%
\end{pgfscope}%
\begin{pgfscope}%
\pgfsys@transformshift{5.312141in}{2.333117in}%
\pgfsys@useobject{currentmarker}{}%
\end{pgfscope}%
\begin{pgfscope}%
\pgfsys@transformshift{5.316423in}{2.338131in}%
\pgfsys@useobject{currentmarker}{}%
\end{pgfscope}%
\begin{pgfscope}%
\pgfsys@transformshift{5.322956in}{2.343077in}%
\pgfsys@useobject{currentmarker}{}%
\end{pgfscope}%
\begin{pgfscope}%
\pgfsys@transformshift{5.328873in}{2.352365in}%
\pgfsys@useobject{currentmarker}{}%
\end{pgfscope}%
\begin{pgfscope}%
\pgfsys@transformshift{5.343086in}{2.358210in}%
\pgfsys@useobject{currentmarker}{}%
\end{pgfscope}%
\begin{pgfscope}%
\pgfsys@transformshift{5.349861in}{2.363263in}%
\pgfsys@useobject{currentmarker}{}%
\end{pgfscope}%
\begin{pgfscope}%
\pgfsys@transformshift{5.353934in}{2.365505in}%
\pgfsys@useobject{currentmarker}{}%
\end{pgfscope}%
\begin{pgfscope}%
\pgfsys@transformshift{5.358738in}{2.368935in}%
\pgfsys@useobject{currentmarker}{}%
\end{pgfscope}%
\begin{pgfscope}%
\pgfsys@transformshift{5.364649in}{2.372362in}%
\pgfsys@useobject{currentmarker}{}%
\end{pgfscope}%
\begin{pgfscope}%
\pgfsys@transformshift{5.367860in}{2.374315in}%
\pgfsys@useobject{currentmarker}{}%
\end{pgfscope}%
\begin{pgfscope}%
\pgfsys@transformshift{5.369704in}{2.375249in}%
\pgfsys@useobject{currentmarker}{}%
\end{pgfscope}%
\begin{pgfscope}%
\pgfsys@transformshift{5.371931in}{2.376943in}%
\pgfsys@useobject{currentmarker}{}%
\end{pgfscope}%
\begin{pgfscope}%
\pgfsys@transformshift{5.373157in}{2.377875in}%
\pgfsys@useobject{currentmarker}{}%
\end{pgfscope}%
\end{pgfscope}%
\begin{pgfscope}%
\pgfsetbuttcap%
\pgfsetroundjoin%
\definecolor{currentfill}{rgb}{0.000000,0.000000,0.000000}%
\pgfsetfillcolor{currentfill}%
\pgfsetlinewidth{0.803000pt}%
\definecolor{currentstroke}{rgb}{0.000000,0.000000,0.000000}%
\pgfsetstrokecolor{currentstroke}%
\pgfsetdash{}{0pt}%
\pgfsys@defobject{currentmarker}{\pgfqpoint{0.000000in}{-0.048611in}}{\pgfqpoint{0.000000in}{0.000000in}}{%
\pgfpathmoveto{\pgfqpoint{0.000000in}{0.000000in}}%
\pgfpathlineto{\pgfqpoint{0.000000in}{-0.048611in}}%
\pgfusepath{stroke,fill}%
}%
\begin{pgfscope}%
\pgfsys@transformshift{0.864066in}{0.515000in}%
\pgfsys@useobject{currentmarker}{}%
\end{pgfscope}%
\end{pgfscope}%
\begin{pgfscope}%
\definecolor{textcolor}{rgb}{0.000000,0.000000,0.000000}%
\pgfsetstrokecolor{textcolor}%
\pgfsetfillcolor{textcolor}%
\pgftext[x=0.864066in,y=0.417777in,,top]{\color{textcolor}\rmfamily\fontsize{10.000000}{12.000000}\selectfont \(\displaystyle {0}\)}%
\end{pgfscope}%
\begin{pgfscope}%
\pgfsetbuttcap%
\pgfsetroundjoin%
\definecolor{currentfill}{rgb}{0.000000,0.000000,0.000000}%
\pgfsetfillcolor{currentfill}%
\pgfsetlinewidth{0.803000pt}%
\definecolor{currentstroke}{rgb}{0.000000,0.000000,0.000000}%
\pgfsetstrokecolor{currentstroke}%
\pgfsetdash{}{0pt}%
\pgfsys@defobject{currentmarker}{\pgfqpoint{0.000000in}{-0.048611in}}{\pgfqpoint{0.000000in}{0.000000in}}{%
\pgfpathmoveto{\pgfqpoint{0.000000in}{0.000000in}}%
\pgfpathlineto{\pgfqpoint{0.000000in}{-0.048611in}}%
\pgfusepath{stroke,fill}%
}%
\begin{pgfscope}%
\pgfsys@transformshift{1.483661in}{0.515000in}%
\pgfsys@useobject{currentmarker}{}%
\end{pgfscope}%
\end{pgfscope}%
\begin{pgfscope}%
\definecolor{textcolor}{rgb}{0.000000,0.000000,0.000000}%
\pgfsetstrokecolor{textcolor}%
\pgfsetfillcolor{textcolor}%
\pgftext[x=1.483661in,y=0.417777in,,top]{\color{textcolor}\rmfamily\fontsize{10.000000}{12.000000}\selectfont \(\displaystyle {5}\)}%
\end{pgfscope}%
\begin{pgfscope}%
\pgfsetbuttcap%
\pgfsetroundjoin%
\definecolor{currentfill}{rgb}{0.000000,0.000000,0.000000}%
\pgfsetfillcolor{currentfill}%
\pgfsetlinewidth{0.803000pt}%
\definecolor{currentstroke}{rgb}{0.000000,0.000000,0.000000}%
\pgfsetstrokecolor{currentstroke}%
\pgfsetdash{}{0pt}%
\pgfsys@defobject{currentmarker}{\pgfqpoint{0.000000in}{-0.048611in}}{\pgfqpoint{0.000000in}{0.000000in}}{%
\pgfpathmoveto{\pgfqpoint{0.000000in}{0.000000in}}%
\pgfpathlineto{\pgfqpoint{0.000000in}{-0.048611in}}%
\pgfusepath{stroke,fill}%
}%
\begin{pgfscope}%
\pgfsys@transformshift{2.103257in}{0.515000in}%
\pgfsys@useobject{currentmarker}{}%
\end{pgfscope}%
\end{pgfscope}%
\begin{pgfscope}%
\definecolor{textcolor}{rgb}{0.000000,0.000000,0.000000}%
\pgfsetstrokecolor{textcolor}%
\pgfsetfillcolor{textcolor}%
\pgftext[x=2.103257in,y=0.417777in,,top]{\color{textcolor}\rmfamily\fontsize{10.000000}{12.000000}\selectfont \(\displaystyle {10}\)}%
\end{pgfscope}%
\begin{pgfscope}%
\pgfsetbuttcap%
\pgfsetroundjoin%
\definecolor{currentfill}{rgb}{0.000000,0.000000,0.000000}%
\pgfsetfillcolor{currentfill}%
\pgfsetlinewidth{0.803000pt}%
\definecolor{currentstroke}{rgb}{0.000000,0.000000,0.000000}%
\pgfsetstrokecolor{currentstroke}%
\pgfsetdash{}{0pt}%
\pgfsys@defobject{currentmarker}{\pgfqpoint{0.000000in}{-0.048611in}}{\pgfqpoint{0.000000in}{0.000000in}}{%
\pgfpathmoveto{\pgfqpoint{0.000000in}{0.000000in}}%
\pgfpathlineto{\pgfqpoint{0.000000in}{-0.048611in}}%
\pgfusepath{stroke,fill}%
}%
\begin{pgfscope}%
\pgfsys@transformshift{2.722853in}{0.515000in}%
\pgfsys@useobject{currentmarker}{}%
\end{pgfscope}%
\end{pgfscope}%
\begin{pgfscope}%
\definecolor{textcolor}{rgb}{0.000000,0.000000,0.000000}%
\pgfsetstrokecolor{textcolor}%
\pgfsetfillcolor{textcolor}%
\pgftext[x=2.722853in,y=0.417777in,,top]{\color{textcolor}\rmfamily\fontsize{10.000000}{12.000000}\selectfont \(\displaystyle {15}\)}%
\end{pgfscope}%
\begin{pgfscope}%
\pgfsetbuttcap%
\pgfsetroundjoin%
\definecolor{currentfill}{rgb}{0.000000,0.000000,0.000000}%
\pgfsetfillcolor{currentfill}%
\pgfsetlinewidth{0.803000pt}%
\definecolor{currentstroke}{rgb}{0.000000,0.000000,0.000000}%
\pgfsetstrokecolor{currentstroke}%
\pgfsetdash{}{0pt}%
\pgfsys@defobject{currentmarker}{\pgfqpoint{0.000000in}{-0.048611in}}{\pgfqpoint{0.000000in}{0.000000in}}{%
\pgfpathmoveto{\pgfqpoint{0.000000in}{0.000000in}}%
\pgfpathlineto{\pgfqpoint{0.000000in}{-0.048611in}}%
\pgfusepath{stroke,fill}%
}%
\begin{pgfscope}%
\pgfsys@transformshift{3.342448in}{0.515000in}%
\pgfsys@useobject{currentmarker}{}%
\end{pgfscope}%
\end{pgfscope}%
\begin{pgfscope}%
\definecolor{textcolor}{rgb}{0.000000,0.000000,0.000000}%
\pgfsetstrokecolor{textcolor}%
\pgfsetfillcolor{textcolor}%
\pgftext[x=3.342448in,y=0.417777in,,top]{\color{textcolor}\rmfamily\fontsize{10.000000}{12.000000}\selectfont \(\displaystyle {20}\)}%
\end{pgfscope}%
\begin{pgfscope}%
\pgfsetbuttcap%
\pgfsetroundjoin%
\definecolor{currentfill}{rgb}{0.000000,0.000000,0.000000}%
\pgfsetfillcolor{currentfill}%
\pgfsetlinewidth{0.803000pt}%
\definecolor{currentstroke}{rgb}{0.000000,0.000000,0.000000}%
\pgfsetstrokecolor{currentstroke}%
\pgfsetdash{}{0pt}%
\pgfsys@defobject{currentmarker}{\pgfqpoint{0.000000in}{-0.048611in}}{\pgfqpoint{0.000000in}{0.000000in}}{%
\pgfpathmoveto{\pgfqpoint{0.000000in}{0.000000in}}%
\pgfpathlineto{\pgfqpoint{0.000000in}{-0.048611in}}%
\pgfusepath{stroke,fill}%
}%
\begin{pgfscope}%
\pgfsys@transformshift{3.962044in}{0.515000in}%
\pgfsys@useobject{currentmarker}{}%
\end{pgfscope}%
\end{pgfscope}%
\begin{pgfscope}%
\definecolor{textcolor}{rgb}{0.000000,0.000000,0.000000}%
\pgfsetstrokecolor{textcolor}%
\pgfsetfillcolor{textcolor}%
\pgftext[x=3.962044in,y=0.417777in,,top]{\color{textcolor}\rmfamily\fontsize{10.000000}{12.000000}\selectfont \(\displaystyle {25}\)}%
\end{pgfscope}%
\begin{pgfscope}%
\pgfsetbuttcap%
\pgfsetroundjoin%
\definecolor{currentfill}{rgb}{0.000000,0.000000,0.000000}%
\pgfsetfillcolor{currentfill}%
\pgfsetlinewidth{0.803000pt}%
\definecolor{currentstroke}{rgb}{0.000000,0.000000,0.000000}%
\pgfsetstrokecolor{currentstroke}%
\pgfsetdash{}{0pt}%
\pgfsys@defobject{currentmarker}{\pgfqpoint{0.000000in}{-0.048611in}}{\pgfqpoint{0.000000in}{0.000000in}}{%
\pgfpathmoveto{\pgfqpoint{0.000000in}{0.000000in}}%
\pgfpathlineto{\pgfqpoint{0.000000in}{-0.048611in}}%
\pgfusepath{stroke,fill}%
}%
\begin{pgfscope}%
\pgfsys@transformshift{4.581640in}{0.515000in}%
\pgfsys@useobject{currentmarker}{}%
\end{pgfscope}%
\end{pgfscope}%
\begin{pgfscope}%
\definecolor{textcolor}{rgb}{0.000000,0.000000,0.000000}%
\pgfsetstrokecolor{textcolor}%
\pgfsetfillcolor{textcolor}%
\pgftext[x=4.581640in,y=0.417777in,,top]{\color{textcolor}\rmfamily\fontsize{10.000000}{12.000000}\selectfont \(\displaystyle {30}\)}%
\end{pgfscope}%
\begin{pgfscope}%
\pgfsetbuttcap%
\pgfsetroundjoin%
\definecolor{currentfill}{rgb}{0.000000,0.000000,0.000000}%
\pgfsetfillcolor{currentfill}%
\pgfsetlinewidth{0.803000pt}%
\definecolor{currentstroke}{rgb}{0.000000,0.000000,0.000000}%
\pgfsetstrokecolor{currentstroke}%
\pgfsetdash{}{0pt}%
\pgfsys@defobject{currentmarker}{\pgfqpoint{0.000000in}{-0.048611in}}{\pgfqpoint{0.000000in}{0.000000in}}{%
\pgfpathmoveto{\pgfqpoint{0.000000in}{0.000000in}}%
\pgfpathlineto{\pgfqpoint{0.000000in}{-0.048611in}}%
\pgfusepath{stroke,fill}%
}%
\begin{pgfscope}%
\pgfsys@transformshift{5.201235in}{0.515000in}%
\pgfsys@useobject{currentmarker}{}%
\end{pgfscope}%
\end{pgfscope}%
\begin{pgfscope}%
\definecolor{textcolor}{rgb}{0.000000,0.000000,0.000000}%
\pgfsetstrokecolor{textcolor}%
\pgfsetfillcolor{textcolor}%
\pgftext[x=5.201235in,y=0.417777in,,top]{\color{textcolor}\rmfamily\fontsize{10.000000}{12.000000}\selectfont \(\displaystyle {35}\)}%
\end{pgfscope}%
\begin{pgfscope}%
\definecolor{textcolor}{rgb}{0.000000,0.000000,0.000000}%
\pgfsetstrokecolor{textcolor}%
\pgfsetfillcolor{textcolor}%
\pgftext[x=3.118611in,y=0.238889in,,top]{\color{textcolor}\rmfamily\fontsize{10.000000}{12.000000}\selectfont Position X [\(\displaystyle m\)]}%
\end{pgfscope}%
\begin{pgfscope}%
\pgfsetbuttcap%
\pgfsetroundjoin%
\definecolor{currentfill}{rgb}{0.000000,0.000000,0.000000}%
\pgfsetfillcolor{currentfill}%
\pgfsetlinewidth{0.803000pt}%
\definecolor{currentstroke}{rgb}{0.000000,0.000000,0.000000}%
\pgfsetstrokecolor{currentstroke}%
\pgfsetdash{}{0pt}%
\pgfsys@defobject{currentmarker}{\pgfqpoint{-0.048611in}{0.000000in}}{\pgfqpoint{-0.000000in}{0.000000in}}{%
\pgfpathmoveto{\pgfqpoint{-0.000000in}{0.000000in}}%
\pgfpathlineto{\pgfqpoint{-0.048611in}{0.000000in}}%
\pgfusepath{stroke,fill}%
}%
\begin{pgfscope}%
\pgfsys@transformshift{0.638611in}{0.531211in}%
\pgfsys@useobject{currentmarker}{}%
\end{pgfscope}%
\end{pgfscope}%
\begin{pgfscope}%
\definecolor{textcolor}{rgb}{0.000000,0.000000,0.000000}%
\pgfsetstrokecolor{textcolor}%
\pgfsetfillcolor{textcolor}%
\pgftext[x=0.294444in, y=0.483016in, left, base]{\color{textcolor}\rmfamily\fontsize{10.000000}{12.000000}\selectfont \(\displaystyle {−15}\)}%
\end{pgfscope}%
\begin{pgfscope}%
\pgfsetbuttcap%
\pgfsetroundjoin%
\definecolor{currentfill}{rgb}{0.000000,0.000000,0.000000}%
\pgfsetfillcolor{currentfill}%
\pgfsetlinewidth{0.803000pt}%
\definecolor{currentstroke}{rgb}{0.000000,0.000000,0.000000}%
\pgfsetstrokecolor{currentstroke}%
\pgfsetdash{}{0pt}%
\pgfsys@defobject{currentmarker}{\pgfqpoint{-0.048611in}{0.000000in}}{\pgfqpoint{-0.000000in}{0.000000in}}{%
\pgfpathmoveto{\pgfqpoint{-0.000000in}{0.000000in}}%
\pgfpathlineto{\pgfqpoint{-0.048611in}{0.000000in}}%
\pgfusepath{stroke,fill}%
}%
\begin{pgfscope}%
\pgfsys@transformshift{0.638611in}{1.150806in}%
\pgfsys@useobject{currentmarker}{}%
\end{pgfscope}%
\end{pgfscope}%
\begin{pgfscope}%
\definecolor{textcolor}{rgb}{0.000000,0.000000,0.000000}%
\pgfsetstrokecolor{textcolor}%
\pgfsetfillcolor{textcolor}%
\pgftext[x=0.294444in, y=1.102612in, left, base]{\color{textcolor}\rmfamily\fontsize{10.000000}{12.000000}\selectfont \(\displaystyle {−10}\)}%
\end{pgfscope}%
\begin{pgfscope}%
\pgfsetbuttcap%
\pgfsetroundjoin%
\definecolor{currentfill}{rgb}{0.000000,0.000000,0.000000}%
\pgfsetfillcolor{currentfill}%
\pgfsetlinewidth{0.803000pt}%
\definecolor{currentstroke}{rgb}{0.000000,0.000000,0.000000}%
\pgfsetstrokecolor{currentstroke}%
\pgfsetdash{}{0pt}%
\pgfsys@defobject{currentmarker}{\pgfqpoint{-0.048611in}{0.000000in}}{\pgfqpoint{-0.000000in}{0.000000in}}{%
\pgfpathmoveto{\pgfqpoint{-0.000000in}{0.000000in}}%
\pgfpathlineto{\pgfqpoint{-0.048611in}{0.000000in}}%
\pgfusepath{stroke,fill}%
}%
\begin{pgfscope}%
\pgfsys@transformshift{0.638611in}{1.770402in}%
\pgfsys@useobject{currentmarker}{}%
\end{pgfscope}%
\end{pgfscope}%
\begin{pgfscope}%
\definecolor{textcolor}{rgb}{0.000000,0.000000,0.000000}%
\pgfsetstrokecolor{textcolor}%
\pgfsetfillcolor{textcolor}%
\pgftext[x=0.363889in, y=1.722208in, left, base]{\color{textcolor}\rmfamily\fontsize{10.000000}{12.000000}\selectfont \(\displaystyle {−5}\)}%
\end{pgfscope}%
\begin{pgfscope}%
\pgfsetbuttcap%
\pgfsetroundjoin%
\definecolor{currentfill}{rgb}{0.000000,0.000000,0.000000}%
\pgfsetfillcolor{currentfill}%
\pgfsetlinewidth{0.803000pt}%
\definecolor{currentstroke}{rgb}{0.000000,0.000000,0.000000}%
\pgfsetstrokecolor{currentstroke}%
\pgfsetdash{}{0pt}%
\pgfsys@defobject{currentmarker}{\pgfqpoint{-0.048611in}{0.000000in}}{\pgfqpoint{-0.000000in}{0.000000in}}{%
\pgfpathmoveto{\pgfqpoint{-0.000000in}{0.000000in}}%
\pgfpathlineto{\pgfqpoint{-0.048611in}{0.000000in}}%
\pgfusepath{stroke,fill}%
}%
\begin{pgfscope}%
\pgfsys@transformshift{0.638611in}{2.389998in}%
\pgfsys@useobject{currentmarker}{}%
\end{pgfscope}%
\end{pgfscope}%
\begin{pgfscope}%
\definecolor{textcolor}{rgb}{0.000000,0.000000,0.000000}%
\pgfsetstrokecolor{textcolor}%
\pgfsetfillcolor{textcolor}%
\pgftext[x=0.471944in, y=2.341803in, left, base]{\color{textcolor}\rmfamily\fontsize{10.000000}{12.000000}\selectfont \(\displaystyle {0}\)}%
\end{pgfscope}%
\begin{pgfscope}%
\pgfsetbuttcap%
\pgfsetroundjoin%
\definecolor{currentfill}{rgb}{0.000000,0.000000,0.000000}%
\pgfsetfillcolor{currentfill}%
\pgfsetlinewidth{0.803000pt}%
\definecolor{currentstroke}{rgb}{0.000000,0.000000,0.000000}%
\pgfsetstrokecolor{currentstroke}%
\pgfsetdash{}{0pt}%
\pgfsys@defobject{currentmarker}{\pgfqpoint{-0.048611in}{0.000000in}}{\pgfqpoint{-0.000000in}{0.000000in}}{%
\pgfpathmoveto{\pgfqpoint{-0.000000in}{0.000000in}}%
\pgfpathlineto{\pgfqpoint{-0.048611in}{0.000000in}}%
\pgfusepath{stroke,fill}%
}%
\begin{pgfscope}%
\pgfsys@transformshift{0.638611in}{3.009593in}%
\pgfsys@useobject{currentmarker}{}%
\end{pgfscope}%
\end{pgfscope}%
\begin{pgfscope}%
\definecolor{textcolor}{rgb}{0.000000,0.000000,0.000000}%
\pgfsetstrokecolor{textcolor}%
\pgfsetfillcolor{textcolor}%
\pgftext[x=0.471944in, y=2.961399in, left, base]{\color{textcolor}\rmfamily\fontsize{10.000000}{12.000000}\selectfont \(\displaystyle {5}\)}%
\end{pgfscope}%
\begin{pgfscope}%
\pgfsetbuttcap%
\pgfsetroundjoin%
\definecolor{currentfill}{rgb}{0.000000,0.000000,0.000000}%
\pgfsetfillcolor{currentfill}%
\pgfsetlinewidth{0.803000pt}%
\definecolor{currentstroke}{rgb}{0.000000,0.000000,0.000000}%
\pgfsetstrokecolor{currentstroke}%
\pgfsetdash{}{0pt}%
\pgfsys@defobject{currentmarker}{\pgfqpoint{-0.048611in}{0.000000in}}{\pgfqpoint{-0.000000in}{0.000000in}}{%
\pgfpathmoveto{\pgfqpoint{-0.000000in}{0.000000in}}%
\pgfpathlineto{\pgfqpoint{-0.048611in}{0.000000in}}%
\pgfusepath{stroke,fill}%
}%
\begin{pgfscope}%
\pgfsys@transformshift{0.638611in}{3.629189in}%
\pgfsys@useobject{currentmarker}{}%
\end{pgfscope}%
\end{pgfscope}%
\begin{pgfscope}%
\definecolor{textcolor}{rgb}{0.000000,0.000000,0.000000}%
\pgfsetstrokecolor{textcolor}%
\pgfsetfillcolor{textcolor}%
\pgftext[x=0.402500in, y=3.580995in, left, base]{\color{textcolor}\rmfamily\fontsize{10.000000}{12.000000}\selectfont \(\displaystyle {10}\)}%
\end{pgfscope}%
\begin{pgfscope}%
\definecolor{textcolor}{rgb}{0.000000,0.000000,0.000000}%
\pgfsetstrokecolor{textcolor}%
\pgfsetfillcolor{textcolor}%
\pgftext[x=0.238889in,y=2.363000in,,bottom,rotate=90.000000]{\color{textcolor}\rmfamily\fontsize{10.000000}{12.000000}\selectfont Position Y [\(\displaystyle m\)]}%
\end{pgfscope}%
\begin{pgfscope}%
\pgfpathrectangle{\pgfqpoint{0.638611in}{0.515000in}}{\pgfqpoint{4.960000in}{3.696000in}}%
\pgfusepath{clip}%
\pgfsetrectcap%
\pgfsetroundjoin%
\pgfsetlinewidth{1.505625pt}%
\definecolor{currentstroke}{rgb}{0.121569,0.466667,0.705882}%
\pgfsetstrokecolor{currentstroke}%
\pgfsetdash{}{0pt}%
\pgfpathmoveto{\pgfqpoint{0.864066in}{2.389998in}}%
\pgfpathlineto{\pgfqpoint{4.333801in}{2.389998in}}%
\pgfusepath{stroke}%
\end{pgfscope}%
\begin{pgfscope}%
\pgfsetrectcap%
\pgfsetmiterjoin%
\pgfsetlinewidth{0.803000pt}%
\definecolor{currentstroke}{rgb}{0.000000,0.000000,0.000000}%
\pgfsetstrokecolor{currentstroke}%
\pgfsetdash{}{0pt}%
\pgfpathmoveto{\pgfqpoint{0.638611in}{0.515000in}}%
\pgfpathlineto{\pgfqpoint{0.638611in}{4.211000in}}%
\pgfusepath{stroke}%
\end{pgfscope}%
\begin{pgfscope}%
\pgfsetrectcap%
\pgfsetmiterjoin%
\pgfsetlinewidth{0.803000pt}%
\definecolor{currentstroke}{rgb}{0.000000,0.000000,0.000000}%
\pgfsetstrokecolor{currentstroke}%
\pgfsetdash{}{0pt}%
\pgfpathmoveto{\pgfqpoint{5.598611in}{0.515000in}}%
\pgfpathlineto{\pgfqpoint{5.598611in}{4.211000in}}%
\pgfusepath{stroke}%
\end{pgfscope}%
\begin{pgfscope}%
\pgfsetrectcap%
\pgfsetmiterjoin%
\pgfsetlinewidth{0.803000pt}%
\definecolor{currentstroke}{rgb}{0.000000,0.000000,0.000000}%
\pgfsetstrokecolor{currentstroke}%
\pgfsetdash{}{0pt}%
\pgfpathmoveto{\pgfqpoint{0.638611in}{0.515000in}}%
\pgfpathlineto{\pgfqpoint{5.598611in}{0.515000in}}%
\pgfusepath{stroke}%
\end{pgfscope}%
\begin{pgfscope}%
\pgfsetrectcap%
\pgfsetmiterjoin%
\pgfsetlinewidth{0.803000pt}%
\definecolor{currentstroke}{rgb}{0.000000,0.000000,0.000000}%
\pgfsetstrokecolor{currentstroke}%
\pgfsetdash{}{0pt}%
\pgfpathmoveto{\pgfqpoint{0.638611in}{4.211000in}}%
\pgfpathlineto{\pgfqpoint{5.598611in}{4.211000in}}%
\pgfusepath{stroke}%
\end{pgfscope}%
\begin{pgfscope}%
\pgfsetbuttcap%
\pgfsetmiterjoin%
\definecolor{currentfill}{rgb}{1.000000,1.000000,1.000000}%
\pgfsetfillcolor{currentfill}%
\pgfsetfillopacity{0.800000}%
\pgfsetlinewidth{1.003750pt}%
\definecolor{currentstroke}{rgb}{0.800000,0.800000,0.800000}%
\pgfsetstrokecolor{currentstroke}%
\pgfsetstrokeopacity{0.800000}%
\pgfsetdash{}{0pt}%
\pgfpathmoveto{\pgfqpoint{3.907500in}{3.712667in}}%
\pgfpathlineto{\pgfqpoint{5.501389in}{3.712667in}}%
\pgfpathquadraticcurveto{\pgfqpoint{5.529167in}{3.712667in}}{\pgfqpoint{5.529167in}{3.740444in}}%
\pgfpathlineto{\pgfqpoint{5.529167in}{4.113777in}}%
\pgfpathquadraticcurveto{\pgfqpoint{5.529167in}{4.141555in}}{\pgfqpoint{5.501389in}{4.141555in}}%
\pgfpathlineto{\pgfqpoint{3.907500in}{4.141555in}}%
\pgfpathquadraticcurveto{\pgfqpoint{3.879722in}{4.141555in}}{\pgfqpoint{3.879722in}{4.113777in}}%
\pgfpathlineto{\pgfqpoint{3.879722in}{3.740444in}}%
\pgfpathquadraticcurveto{\pgfqpoint{3.879722in}{3.712667in}}{\pgfqpoint{3.907500in}{3.712667in}}%
\pgfpathclose%
\pgfusepath{stroke,fill}%
\end{pgfscope}%
\begin{pgfscope}%
\pgfsetrectcap%
\pgfsetroundjoin%
\pgfsetlinewidth{1.505625pt}%
\definecolor{currentstroke}{rgb}{0.121569,0.466667,0.705882}%
\pgfsetstrokecolor{currentstroke}%
\pgfsetdash{}{0pt}%
\pgfpathmoveto{\pgfqpoint{3.935278in}{4.037388in}}%
\pgfpathlineto{\pgfqpoint{4.213056in}{4.037388in}}%
\pgfusepath{stroke}%
\end{pgfscope}%
\begin{pgfscope}%
\definecolor{textcolor}{rgb}{0.000000,0.000000,0.000000}%
\pgfsetstrokecolor{textcolor}%
\pgfsetfillcolor{textcolor}%
\pgftext[x=4.324167in,y=3.988777in,left,base]{\color{textcolor}\rmfamily\fontsize{10.000000}{12.000000}\selectfont Ground truth}%
\end{pgfscope}%
\begin{pgfscope}%
\pgfsetbuttcap%
\pgfsetroundjoin%
\definecolor{currentfill}{rgb}{0.121569,0.466667,0.705882}%
\pgfsetfillcolor{currentfill}%
\pgfsetlinewidth{1.003750pt}%
\definecolor{currentstroke}{rgb}{0.121569,0.466667,0.705882}%
\pgfsetstrokecolor{currentstroke}%
\pgfsetdash{}{0pt}%
\pgfsys@defobject{currentmarker}{\pgfqpoint{-0.041667in}{-0.041667in}}{\pgfqpoint{0.041667in}{0.041667in}}{%
\pgfpathmoveto{\pgfqpoint{0.000000in}{-0.041667in}}%
\pgfpathcurveto{\pgfqpoint{0.011050in}{-0.041667in}}{\pgfqpoint{0.021649in}{-0.037276in}}{\pgfqpoint{0.029463in}{-0.029463in}}%
\pgfpathcurveto{\pgfqpoint{0.037276in}{-0.021649in}}{\pgfqpoint{0.041667in}{-0.011050in}}{\pgfqpoint{0.041667in}{0.000000in}}%
\pgfpathcurveto{\pgfqpoint{0.041667in}{0.011050in}}{\pgfqpoint{0.037276in}{0.021649in}}{\pgfqpoint{0.029463in}{0.029463in}}%
\pgfpathcurveto{\pgfqpoint{0.021649in}{0.037276in}}{\pgfqpoint{0.011050in}{0.041667in}}{\pgfqpoint{0.000000in}{0.041667in}}%
\pgfpathcurveto{\pgfqpoint{-0.011050in}{0.041667in}}{\pgfqpoint{-0.021649in}{0.037276in}}{\pgfqpoint{-0.029463in}{0.029463in}}%
\pgfpathcurveto{\pgfqpoint{-0.037276in}{0.021649in}}{\pgfqpoint{-0.041667in}{0.011050in}}{\pgfqpoint{-0.041667in}{0.000000in}}%
\pgfpathcurveto{\pgfqpoint{-0.041667in}{-0.011050in}}{\pgfqpoint{-0.037276in}{-0.021649in}}{\pgfqpoint{-0.029463in}{-0.029463in}}%
\pgfpathcurveto{\pgfqpoint{-0.021649in}{-0.037276in}}{\pgfqpoint{-0.011050in}{-0.041667in}}{\pgfqpoint{0.000000in}{-0.041667in}}%
\pgfpathclose%
\pgfusepath{stroke,fill}%
}%
\begin{pgfscope}%
\pgfsys@transformshift{4.074167in}{3.831625in}%
\pgfsys@useobject{currentmarker}{}%
\end{pgfscope}%
\end{pgfscope}%
\begin{pgfscope}%
\definecolor{textcolor}{rgb}{0.000000,0.000000,0.000000}%
\pgfsetstrokecolor{textcolor}%
\pgfsetfillcolor{textcolor}%
\pgftext[x=4.324167in,y=3.795166in,left,base]{\color{textcolor}\rmfamily\fontsize{10.000000}{12.000000}\selectfont Estimated position}%
\end{pgfscope}%
\end{pgfpicture}%
\makeatother%
\endgroup%
}
%         \caption{Complementary's 3D position estimation had the lowest displacement error for the 28-meter line experiment.}
%         \label{fig:line28_2D}
%     \end{subfigure}
%     \begin{subfigure}{0.49\textwidth}
%         \centering
%         \resizebox{1\linewidth}{!}{%% Creator: Matplotlib, PGF backend
%%
%% To include the figure in your LaTeX document, write
%%   \input{<filename>.pgf}
%%
%% Make sure the required packages are loaded in your preamble
%%   \usepackage{pgf}
%%
%% and, on pdftex
%%   \usepackage[utf8]{inputenc}\DeclareUnicodeCharacter{2212}{-}
%%
%% or, on luatex and xetex
%%   \usepackage{unicode-math}
%%
%% Figures using additional raster images can only be included by \input if
%% they are in the same directory as the main LaTeX file. For loading figures
%% from other directories you can use the `import` package
%%   \usepackage{import}
%%
%% and then include the figures with
%%   \import{<path to file>}{<filename>.pgf}
%%
%% Matplotlib used the following preamble
%%   \usepackage{fontspec}
%%
\begingroup%
\makeatletter%
\begin{pgfpicture}%
\pgfpathrectangle{\pgfpointorigin}{\pgfqpoint{4.342355in}{4.008622in}}%
\pgfusepath{use as bounding box, clip}%
\begin{pgfscope}%
\pgfsetbuttcap%
\pgfsetmiterjoin%
\definecolor{currentfill}{rgb}{1.000000,1.000000,1.000000}%
\pgfsetfillcolor{currentfill}%
\pgfsetlinewidth{0.000000pt}%
\definecolor{currentstroke}{rgb}{1.000000,1.000000,1.000000}%
\pgfsetstrokecolor{currentstroke}%
\pgfsetdash{}{0pt}%
\pgfpathmoveto{\pgfqpoint{0.000000in}{-0.000000in}}%
\pgfpathlineto{\pgfqpoint{4.342355in}{-0.000000in}}%
\pgfpathlineto{\pgfqpoint{4.342355in}{4.008622in}}%
\pgfpathlineto{\pgfqpoint{0.000000in}{4.008622in}}%
\pgfpathclose%
\pgfusepath{fill}%
\end{pgfscope}%
\begin{pgfscope}%
\pgfsetbuttcap%
\pgfsetmiterjoin%
\definecolor{currentfill}{rgb}{1.000000,1.000000,1.000000}%
\pgfsetfillcolor{currentfill}%
\pgfsetlinewidth{0.000000pt}%
\definecolor{currentstroke}{rgb}{0.000000,0.000000,0.000000}%
\pgfsetstrokecolor{currentstroke}%
\pgfsetstrokeopacity{0.000000}%
\pgfsetdash{}{0pt}%
\pgfpathmoveto{\pgfqpoint{0.100000in}{0.212622in}}%
\pgfpathlineto{\pgfqpoint{3.796000in}{0.212622in}}%
\pgfpathlineto{\pgfqpoint{3.796000in}{3.908622in}}%
\pgfpathlineto{\pgfqpoint{0.100000in}{3.908622in}}%
\pgfpathclose%
\pgfusepath{fill}%
\end{pgfscope}%
\begin{pgfscope}%
\pgfsetbuttcap%
\pgfsetmiterjoin%
\definecolor{currentfill}{rgb}{0.950000,0.950000,0.950000}%
\pgfsetfillcolor{currentfill}%
\pgfsetfillopacity{0.500000}%
\pgfsetlinewidth{1.003750pt}%
\definecolor{currentstroke}{rgb}{0.950000,0.950000,0.950000}%
\pgfsetstrokecolor{currentstroke}%
\pgfsetstrokeopacity{0.500000}%
\pgfsetdash{}{0pt}%
\pgfpathmoveto{\pgfqpoint{0.379073in}{1.123938in}}%
\pgfpathlineto{\pgfqpoint{1.599613in}{2.147018in}}%
\pgfpathlineto{\pgfqpoint{1.582647in}{3.622484in}}%
\pgfpathlineto{\pgfqpoint{0.303698in}{2.689165in}}%
\pgfusepath{stroke,fill}%
\end{pgfscope}%
\begin{pgfscope}%
\pgfsetbuttcap%
\pgfsetmiterjoin%
\definecolor{currentfill}{rgb}{0.900000,0.900000,0.900000}%
\pgfsetfillcolor{currentfill}%
\pgfsetfillopacity{0.500000}%
\pgfsetlinewidth{1.003750pt}%
\definecolor{currentstroke}{rgb}{0.900000,0.900000,0.900000}%
\pgfsetstrokecolor{currentstroke}%
\pgfsetstrokeopacity{0.500000}%
\pgfsetdash{}{0pt}%
\pgfpathmoveto{\pgfqpoint{1.599613in}{2.147018in}}%
\pgfpathlineto{\pgfqpoint{3.558144in}{1.577751in}}%
\pgfpathlineto{\pgfqpoint{3.628038in}{3.104037in}}%
\pgfpathlineto{\pgfqpoint{1.582647in}{3.622484in}}%
\pgfusepath{stroke,fill}%
\end{pgfscope}%
\begin{pgfscope}%
\pgfsetbuttcap%
\pgfsetmiterjoin%
\definecolor{currentfill}{rgb}{0.925000,0.925000,0.925000}%
\pgfsetfillcolor{currentfill}%
\pgfsetfillopacity{0.500000}%
\pgfsetlinewidth{1.003750pt}%
\definecolor{currentstroke}{rgb}{0.925000,0.925000,0.925000}%
\pgfsetstrokecolor{currentstroke}%
\pgfsetstrokeopacity{0.500000}%
\pgfsetdash{}{0pt}%
\pgfpathmoveto{\pgfqpoint{0.379073in}{1.123938in}}%
\pgfpathlineto{\pgfqpoint{2.455212in}{0.445871in}}%
\pgfpathlineto{\pgfqpoint{3.558144in}{1.577751in}}%
\pgfpathlineto{\pgfqpoint{1.599613in}{2.147018in}}%
\pgfusepath{stroke,fill}%
\end{pgfscope}%
\begin{pgfscope}%
\pgfsetrectcap%
\pgfsetroundjoin%
\pgfsetlinewidth{0.803000pt}%
\definecolor{currentstroke}{rgb}{0.000000,0.000000,0.000000}%
\pgfsetstrokecolor{currentstroke}%
\pgfsetdash{}{0pt}%
\pgfpathmoveto{\pgfqpoint{0.379073in}{1.123938in}}%
\pgfpathlineto{\pgfqpoint{2.455212in}{0.445871in}}%
\pgfusepath{stroke}%
\end{pgfscope}%
\begin{pgfscope}%
\definecolor{textcolor}{rgb}{0.000000,0.000000,0.000000}%
\pgfsetstrokecolor{textcolor}%
\pgfsetfillcolor{textcolor}%
\pgftext[x=0.730374in, y=0.408886in, left, base,rotate=341.912962]{\color{textcolor}\rmfamily\fontsize{10.000000}{12.000000}\selectfont Position X [\(\displaystyle m\)]}%
\end{pgfscope}%
\begin{pgfscope}%
\pgfsetbuttcap%
\pgfsetroundjoin%
\pgfsetlinewidth{0.803000pt}%
\definecolor{currentstroke}{rgb}{0.690196,0.690196,0.690196}%
\pgfsetstrokecolor{currentstroke}%
\pgfsetdash{}{0pt}%
\pgfpathmoveto{\pgfqpoint{0.504815in}{1.082870in}}%
\pgfpathlineto{\pgfqpoint{1.718725in}{2.112397in}}%
\pgfpathlineto{\pgfqpoint{1.706795in}{3.591016in}}%
\pgfusepath{stroke}%
\end{pgfscope}%
\begin{pgfscope}%
\pgfsetbuttcap%
\pgfsetroundjoin%
\pgfsetlinewidth{0.803000pt}%
\definecolor{currentstroke}{rgb}{0.690196,0.690196,0.690196}%
\pgfsetstrokecolor{currentstroke}%
\pgfsetdash{}{0pt}%
\pgfpathmoveto{\pgfqpoint{0.986238in}{0.925638in}}%
\pgfpathlineto{\pgfqpoint{2.174175in}{1.980016in}}%
\pgfpathlineto{\pgfqpoint{2.181795in}{3.470617in}}%
\pgfusepath{stroke}%
\end{pgfscope}%
\begin{pgfscope}%
\pgfsetbuttcap%
\pgfsetroundjoin%
\pgfsetlinewidth{0.803000pt}%
\definecolor{currentstroke}{rgb}{0.690196,0.690196,0.690196}%
\pgfsetstrokecolor{currentstroke}%
\pgfsetdash{}{0pt}%
\pgfpathmoveto{\pgfqpoint{1.479865in}{0.764419in}}%
\pgfpathlineto{\pgfqpoint{2.640201in}{1.844561in}}%
\pgfpathlineto{\pgfqpoint{2.668309in}{3.347300in}}%
\pgfusepath{stroke}%
\end{pgfscope}%
\begin{pgfscope}%
\pgfsetbuttcap%
\pgfsetroundjoin%
\pgfsetlinewidth{0.803000pt}%
\definecolor{currentstroke}{rgb}{0.690196,0.690196,0.690196}%
\pgfsetstrokecolor{currentstroke}%
\pgfsetdash{}{0pt}%
\pgfpathmoveto{\pgfqpoint{1.986165in}{0.599061in}}%
\pgfpathlineto{\pgfqpoint{3.117178in}{1.705922in}}%
\pgfpathlineto{\pgfqpoint{3.166761in}{3.220957in}}%
\pgfusepath{stroke}%
\end{pgfscope}%
\begin{pgfscope}%
\pgfsetrectcap%
\pgfsetroundjoin%
\pgfsetlinewidth{0.803000pt}%
\definecolor{currentstroke}{rgb}{0.000000,0.000000,0.000000}%
\pgfsetstrokecolor{currentstroke}%
\pgfsetdash{}{0pt}%
\pgfpathmoveto{\pgfqpoint{0.515386in}{1.091835in}}%
\pgfpathlineto{\pgfqpoint{0.483629in}{1.064902in}}%
\pgfusepath{stroke}%
\end{pgfscope}%
\begin{pgfscope}%
\definecolor{textcolor}{rgb}{0.000000,0.000000,0.000000}%
\pgfsetstrokecolor{textcolor}%
\pgfsetfillcolor{textcolor}%
\pgftext[x=0.400245in,y=0.864666in,,top]{\color{textcolor}\rmfamily\fontsize{10.000000}{12.000000}\selectfont \(\displaystyle {0}\)}%
\end{pgfscope}%
\begin{pgfscope}%
\pgfsetrectcap%
\pgfsetroundjoin%
\pgfsetlinewidth{0.803000pt}%
\definecolor{currentstroke}{rgb}{0.000000,0.000000,0.000000}%
\pgfsetstrokecolor{currentstroke}%
\pgfsetdash{}{0pt}%
\pgfpathmoveto{\pgfqpoint{0.996593in}{0.934828in}}%
\pgfpathlineto{\pgfqpoint{0.965483in}{0.907216in}}%
\pgfusepath{stroke}%
\end{pgfscope}%
\begin{pgfscope}%
\definecolor{textcolor}{rgb}{0.000000,0.000000,0.000000}%
\pgfsetstrokecolor{textcolor}%
\pgfsetfillcolor{textcolor}%
\pgftext[x=0.882166in,y=0.704100in,,top]{\color{textcolor}\rmfamily\fontsize{10.000000}{12.000000}\selectfont \(\displaystyle {10}\)}%
\end{pgfscope}%
\begin{pgfscope}%
\pgfsetrectcap%
\pgfsetroundjoin%
\pgfsetlinewidth{0.803000pt}%
\definecolor{currentstroke}{rgb}{0.000000,0.000000,0.000000}%
\pgfsetstrokecolor{currentstroke}%
\pgfsetdash{}{0pt}%
\pgfpathmoveto{\pgfqpoint{1.489990in}{0.773844in}}%
\pgfpathlineto{\pgfqpoint{1.459570in}{0.745527in}}%
\pgfusepath{stroke}%
\end{pgfscope}%
\begin{pgfscope}%
\definecolor{textcolor}{rgb}{0.000000,0.000000,0.000000}%
\pgfsetstrokecolor{textcolor}%
\pgfsetfillcolor{textcolor}%
\pgftext[x=1.376350in,y=0.539448in,,top]{\color{textcolor}\rmfamily\fontsize{10.000000}{12.000000}\selectfont \(\displaystyle {20}\)}%
\end{pgfscope}%
\begin{pgfscope}%
\pgfsetrectcap%
\pgfsetroundjoin%
\pgfsetlinewidth{0.803000pt}%
\definecolor{currentstroke}{rgb}{0.000000,0.000000,0.000000}%
\pgfsetstrokecolor{currentstroke}%
\pgfsetdash{}{0pt}%
\pgfpathmoveto{\pgfqpoint{1.996045in}{0.608730in}}%
\pgfpathlineto{\pgfqpoint{1.966362in}{0.579681in}}%
\pgfusepath{stroke}%
\end{pgfscope}%
\begin{pgfscope}%
\definecolor{textcolor}{rgb}{0.000000,0.000000,0.000000}%
\pgfsetstrokecolor{textcolor}%
\pgfsetfillcolor{textcolor}%
\pgftext[x=1.883272in,y=0.370553in,,top]{\color{textcolor}\rmfamily\fontsize{10.000000}{12.000000}\selectfont \(\displaystyle {30}\)}%
\end{pgfscope}%
\begin{pgfscope}%
\pgfsetrectcap%
\pgfsetroundjoin%
\pgfsetlinewidth{0.803000pt}%
\definecolor{currentstroke}{rgb}{0.000000,0.000000,0.000000}%
\pgfsetstrokecolor{currentstroke}%
\pgfsetdash{}{0pt}%
\pgfpathmoveto{\pgfqpoint{3.558144in}{1.577751in}}%
\pgfpathlineto{\pgfqpoint{2.455212in}{0.445871in}}%
\pgfusepath{stroke}%
\end{pgfscope}%
\begin{pgfscope}%
\definecolor{textcolor}{rgb}{0.000000,0.000000,0.000000}%
\pgfsetstrokecolor{textcolor}%
\pgfsetfillcolor{textcolor}%
\pgftext[x=3.120747in, y=0.305657in, left, base,rotate=45.742112]{\color{textcolor}\rmfamily\fontsize{10.000000}{12.000000}\selectfont Position Y [\(\displaystyle m\)]}%
\end{pgfscope}%
\begin{pgfscope}%
\pgfsetbuttcap%
\pgfsetroundjoin%
\pgfsetlinewidth{0.803000pt}%
\definecolor{currentstroke}{rgb}{0.690196,0.690196,0.690196}%
\pgfsetstrokecolor{currentstroke}%
\pgfsetdash{}{0pt}%
\pgfpathmoveto{\pgfqpoint{0.362236in}{2.731884in}}%
\pgfpathlineto{\pgfqpoint{0.434737in}{1.170596in}}%
\pgfpathlineto{\pgfqpoint{2.505724in}{0.497709in}}%
\pgfusepath{stroke}%
\end{pgfscope}%
\begin{pgfscope}%
\pgfsetbuttcap%
\pgfsetroundjoin%
\pgfsetlinewidth{0.803000pt}%
\definecolor{currentstroke}{rgb}{0.690196,0.690196,0.690196}%
\pgfsetstrokecolor{currentstroke}%
\pgfsetdash{}{0pt}%
\pgfpathmoveto{\pgfqpoint{0.584065in}{2.893765in}}%
\pgfpathlineto{\pgfqpoint{0.645845in}{1.347550in}}%
\pgfpathlineto{\pgfqpoint{2.697111in}{0.694118in}}%
\pgfusepath{stroke}%
\end{pgfscope}%
\begin{pgfscope}%
\pgfsetbuttcap%
\pgfsetroundjoin%
\pgfsetlinewidth{0.803000pt}%
\definecolor{currentstroke}{rgb}{0.690196,0.690196,0.690196}%
\pgfsetstrokecolor{currentstroke}%
\pgfsetdash{}{0pt}%
\pgfpathmoveto{\pgfqpoint{0.799410in}{3.050913in}}%
\pgfpathlineto{\pgfqpoint{0.851047in}{1.519555in}}%
\pgfpathlineto{\pgfqpoint{2.882863in}{0.884746in}}%
\pgfusepath{stroke}%
\end{pgfscope}%
\begin{pgfscope}%
\pgfsetbuttcap%
\pgfsetroundjoin%
\pgfsetlinewidth{0.803000pt}%
\definecolor{currentstroke}{rgb}{0.690196,0.690196,0.690196}%
\pgfsetstrokecolor{currentstroke}%
\pgfsetdash{}{0pt}%
\pgfpathmoveto{\pgfqpoint{1.008549in}{3.203534in}}%
\pgfpathlineto{\pgfqpoint{1.050588in}{1.686814in}}%
\pgfpathlineto{\pgfqpoint{3.063226in}{1.069843in}}%
\pgfusepath{stroke}%
\end{pgfscope}%
\begin{pgfscope}%
\pgfsetbuttcap%
\pgfsetroundjoin%
\pgfsetlinewidth{0.803000pt}%
\definecolor{currentstroke}{rgb}{0.690196,0.690196,0.690196}%
\pgfsetstrokecolor{currentstroke}%
\pgfsetdash{}{0pt}%
\pgfpathmoveto{\pgfqpoint{1.211749in}{3.351819in}}%
\pgfpathlineto{\pgfqpoint{1.244699in}{1.849522in}}%
\pgfpathlineto{\pgfqpoint{3.238432in}{1.249647in}}%
\pgfusepath{stroke}%
\end{pgfscope}%
\begin{pgfscope}%
\pgfsetbuttcap%
\pgfsetroundjoin%
\pgfsetlinewidth{0.803000pt}%
\definecolor{currentstroke}{rgb}{0.690196,0.690196,0.690196}%
\pgfsetstrokecolor{currentstroke}%
\pgfsetdash{}{0pt}%
\pgfpathmoveto{\pgfqpoint{1.409258in}{3.495952in}}%
\pgfpathlineto{\pgfqpoint{1.433599in}{2.007862in}}%
\pgfpathlineto{\pgfqpoint{3.408698in}{1.424382in}}%
\pgfusepath{stroke}%
\end{pgfscope}%
\begin{pgfscope}%
\pgfsetrectcap%
\pgfsetroundjoin%
\pgfsetlinewidth{0.803000pt}%
\definecolor{currentstroke}{rgb}{0.000000,0.000000,0.000000}%
\pgfsetstrokecolor{currentstroke}%
\pgfsetdash{}{0pt}%
\pgfpathmoveto{\pgfqpoint{2.488270in}{0.503380in}}%
\pgfpathlineto{\pgfqpoint{2.540678in}{0.486352in}}%
\pgfusepath{stroke}%
\end{pgfscope}%
\begin{pgfscope}%
\definecolor{textcolor}{rgb}{0.000000,0.000000,0.000000}%
\pgfsetstrokecolor{textcolor}%
\pgfsetfillcolor{textcolor}%
\pgftext[x=2.684968in,y=0.310882in,,top]{\color{textcolor}\rmfamily\fontsize{10.000000}{12.000000}\selectfont \(\displaystyle {-0.5}\)}%
\end{pgfscope}%
\begin{pgfscope}%
\pgfsetrectcap%
\pgfsetroundjoin%
\pgfsetlinewidth{0.803000pt}%
\definecolor{currentstroke}{rgb}{0.000000,0.000000,0.000000}%
\pgfsetstrokecolor{currentstroke}%
\pgfsetdash{}{0pt}%
\pgfpathmoveto{\pgfqpoint{2.679836in}{0.699621in}}%
\pgfpathlineto{\pgfqpoint{2.731704in}{0.683098in}}%
\pgfusepath{stroke}%
\end{pgfscope}%
\begin{pgfscope}%
\definecolor{textcolor}{rgb}{0.000000,0.000000,0.000000}%
\pgfsetstrokecolor{textcolor}%
\pgfsetfillcolor{textcolor}%
\pgftext[x=2.873788in,y=0.510200in,,top]{\color{textcolor}\rmfamily\fontsize{10.000000}{12.000000}\selectfont \(\displaystyle {-0.4}\)}%
\end{pgfscope}%
\begin{pgfscope}%
\pgfsetrectcap%
\pgfsetroundjoin%
\pgfsetlinewidth{0.803000pt}%
\definecolor{currentstroke}{rgb}{0.000000,0.000000,0.000000}%
\pgfsetstrokecolor{currentstroke}%
\pgfsetdash{}{0pt}%
\pgfpathmoveto{\pgfqpoint{2.865765in}{0.890088in}}%
\pgfpathlineto{\pgfqpoint{2.917103in}{0.874048in}}%
\pgfusepath{stroke}%
\end{pgfscope}%
\begin{pgfscope}%
\definecolor{textcolor}{rgb}{0.000000,0.000000,0.000000}%
\pgfsetstrokecolor{textcolor}%
\pgfsetfillcolor{textcolor}%
\pgftext[x=3.057046in,y=0.703647in,,top]{\color{textcolor}\rmfamily\fontsize{10.000000}{12.000000}\selectfont \(\displaystyle {-0.3}\)}%
\end{pgfscope}%
\begin{pgfscope}%
\pgfsetrectcap%
\pgfsetroundjoin%
\pgfsetlinewidth{0.803000pt}%
\definecolor{currentstroke}{rgb}{0.000000,0.000000,0.000000}%
\pgfsetstrokecolor{currentstroke}%
\pgfsetdash{}{0pt}%
\pgfpathmoveto{\pgfqpoint{3.046302in}{1.075031in}}%
\pgfpathlineto{\pgfqpoint{3.097118in}{1.059454in}}%
\pgfusepath{stroke}%
\end{pgfscope}%
\begin{pgfscope}%
\definecolor{textcolor}{rgb}{0.000000,0.000000,0.000000}%
\pgfsetstrokecolor{textcolor}%
\pgfsetfillcolor{textcolor}%
\pgftext[x=3.234985in,y=0.891479in,,top]{\color{textcolor}\rmfamily\fontsize{10.000000}{12.000000}\selectfont \(\displaystyle {-0.2}\)}%
\end{pgfscope}%
\begin{pgfscope}%
\pgfsetrectcap%
\pgfsetroundjoin%
\pgfsetlinewidth{0.803000pt}%
\definecolor{currentstroke}{rgb}{0.000000,0.000000,0.000000}%
\pgfsetstrokecolor{currentstroke}%
\pgfsetdash{}{0pt}%
\pgfpathmoveto{\pgfqpoint{3.221678in}{1.254688in}}%
\pgfpathlineto{\pgfqpoint{3.271981in}{1.239553in}}%
\pgfusepath{stroke}%
\end{pgfscope}%
\begin{pgfscope}%
\definecolor{textcolor}{rgb}{0.000000,0.000000,0.000000}%
\pgfsetstrokecolor{textcolor}%
\pgfsetfillcolor{textcolor}%
\pgftext[x=3.407833in,y=1.073937in,,top]{\color{textcolor}\rmfamily\fontsize{10.000000}{12.000000}\selectfont \(\displaystyle {-0.1}\)}%
\end{pgfscope}%
\begin{pgfscope}%
\pgfsetrectcap%
\pgfsetroundjoin%
\pgfsetlinewidth{0.803000pt}%
\definecolor{currentstroke}{rgb}{0.000000,0.000000,0.000000}%
\pgfsetstrokecolor{currentstroke}%
\pgfsetdash{}{0pt}%
\pgfpathmoveto{\pgfqpoint{3.392112in}{1.429282in}}%
\pgfpathlineto{\pgfqpoint{3.441911in}{1.414570in}}%
\pgfusepath{stroke}%
\end{pgfscope}%
\begin{pgfscope}%
\definecolor{textcolor}{rgb}{0.000000,0.000000,0.000000}%
\pgfsetstrokecolor{textcolor}%
\pgfsetfillcolor{textcolor}%
\pgftext[x=3.575804in,y=1.251247in,,top]{\color{textcolor}\rmfamily\fontsize{10.000000}{12.000000}\selectfont \(\displaystyle {0.0}\)}%
\end{pgfscope}%
\begin{pgfscope}%
\pgfsetrectcap%
\pgfsetroundjoin%
\pgfsetlinewidth{0.803000pt}%
\definecolor{currentstroke}{rgb}{0.000000,0.000000,0.000000}%
\pgfsetstrokecolor{currentstroke}%
\pgfsetdash{}{0pt}%
\pgfpathmoveto{\pgfqpoint{3.558144in}{1.577751in}}%
\pgfpathlineto{\pgfqpoint{3.628038in}{3.104037in}}%
\pgfusepath{stroke}%
\end{pgfscope}%
\begin{pgfscope}%
\definecolor{textcolor}{rgb}{0.000000,0.000000,0.000000}%
\pgfsetstrokecolor{textcolor}%
\pgfsetfillcolor{textcolor}%
\pgftext[x=4.167903in, y=1.963517in, left, base,rotate=87.378092]{\color{textcolor}\rmfamily\fontsize{10.000000}{12.000000}\selectfont Position Z [\(\displaystyle m\)]}%
\end{pgfscope}%
\begin{pgfscope}%
\pgfsetbuttcap%
\pgfsetroundjoin%
\pgfsetlinewidth{0.803000pt}%
\definecolor{currentstroke}{rgb}{0.690196,0.690196,0.690196}%
\pgfsetstrokecolor{currentstroke}%
\pgfsetdash{}{0pt}%
\pgfpathmoveto{\pgfqpoint{3.562415in}{1.671009in}}%
\pgfpathlineto{\pgfqpoint{1.598575in}{2.237351in}}%
\pgfpathlineto{\pgfqpoint{0.374475in}{1.219424in}}%
\pgfusepath{stroke}%
\end{pgfscope}%
\begin{pgfscope}%
\pgfsetbuttcap%
\pgfsetroundjoin%
\pgfsetlinewidth{0.803000pt}%
\definecolor{currentstroke}{rgb}{0.690196,0.690196,0.690196}%
\pgfsetstrokecolor{currentstroke}%
\pgfsetdash{}{0pt}%
\pgfpathmoveto{\pgfqpoint{3.571642in}{1.872503in}}%
\pgfpathlineto{\pgfqpoint{1.596331in}{2.432442in}}%
\pgfpathlineto{\pgfqpoint{0.364537in}{1.425799in}}%
\pgfusepath{stroke}%
\end{pgfscope}%
\begin{pgfscope}%
\pgfsetbuttcap%
\pgfsetroundjoin%
\pgfsetlinewidth{0.803000pt}%
\definecolor{currentstroke}{rgb}{0.690196,0.690196,0.690196}%
\pgfsetstrokecolor{currentstroke}%
\pgfsetdash{}{0pt}%
\pgfpathmoveto{\pgfqpoint{3.580979in}{2.076388in}}%
\pgfpathlineto{\pgfqpoint{1.594063in}{2.629738in}}%
\pgfpathlineto{\pgfqpoint{0.354476in}{1.634717in}}%
\pgfusepath{stroke}%
\end{pgfscope}%
\begin{pgfscope}%
\pgfsetbuttcap%
\pgfsetroundjoin%
\pgfsetlinewidth{0.803000pt}%
\definecolor{currentstroke}{rgb}{0.690196,0.690196,0.690196}%
\pgfsetstrokecolor{currentstroke}%
\pgfsetdash{}{0pt}%
\pgfpathmoveto{\pgfqpoint{3.590426in}{2.282707in}}%
\pgfpathlineto{\pgfqpoint{1.591768in}{2.829276in}}%
\pgfpathlineto{\pgfqpoint{0.344291in}{1.846224in}}%
\pgfusepath{stroke}%
\end{pgfscope}%
\begin{pgfscope}%
\pgfsetbuttcap%
\pgfsetroundjoin%
\pgfsetlinewidth{0.803000pt}%
\definecolor{currentstroke}{rgb}{0.690196,0.690196,0.690196}%
\pgfsetstrokecolor{currentstroke}%
\pgfsetdash{}{0pt}%
\pgfpathmoveto{\pgfqpoint{3.599988in}{2.491503in}}%
\pgfpathlineto{\pgfqpoint{1.589447in}{3.031093in}}%
\pgfpathlineto{\pgfqpoint{0.333978in}{2.060369in}}%
\pgfusepath{stroke}%
\end{pgfscope}%
\begin{pgfscope}%
\pgfsetbuttcap%
\pgfsetroundjoin%
\pgfsetlinewidth{0.803000pt}%
\definecolor{currentstroke}{rgb}{0.690196,0.690196,0.690196}%
\pgfsetstrokecolor{currentstroke}%
\pgfsetdash{}{0pt}%
\pgfpathmoveto{\pgfqpoint{3.609665in}{2.702821in}}%
\pgfpathlineto{\pgfqpoint{1.587100in}{3.235230in}}%
\pgfpathlineto{\pgfqpoint{0.323536in}{2.277201in}}%
\pgfusepath{stroke}%
\end{pgfscope}%
\begin{pgfscope}%
\pgfsetbuttcap%
\pgfsetroundjoin%
\pgfsetlinewidth{0.803000pt}%
\definecolor{currentstroke}{rgb}{0.690196,0.690196,0.690196}%
\pgfsetstrokecolor{currentstroke}%
\pgfsetdash{}{0pt}%
\pgfpathmoveto{\pgfqpoint{3.619459in}{2.916708in}}%
\pgfpathlineto{\pgfqpoint{1.584725in}{3.441727in}}%
\pgfpathlineto{\pgfqpoint{0.312963in}{2.496772in}}%
\pgfusepath{stroke}%
\end{pgfscope}%
\begin{pgfscope}%
\pgfsetrectcap%
\pgfsetroundjoin%
\pgfsetlinewidth{0.803000pt}%
\definecolor{currentstroke}{rgb}{0.000000,0.000000,0.000000}%
\pgfsetstrokecolor{currentstroke}%
\pgfsetdash{}{0pt}%
\pgfpathmoveto{\pgfqpoint{3.545931in}{1.675763in}}%
\pgfpathlineto{\pgfqpoint{3.595423in}{1.661491in}}%
\pgfusepath{stroke}%
\end{pgfscope}%
\begin{pgfscope}%
\definecolor{textcolor}{rgb}{0.000000,0.000000,0.000000}%
\pgfsetstrokecolor{textcolor}%
\pgfsetfillcolor{textcolor}%
\pgftext[x=3.816547in,y=1.707008in,,top]{\color{textcolor}\rmfamily\fontsize{10.000000}{12.000000}\selectfont \(\displaystyle {0.00}\)}%
\end{pgfscope}%
\begin{pgfscope}%
\pgfsetrectcap%
\pgfsetroundjoin%
\pgfsetlinewidth{0.803000pt}%
\definecolor{currentstroke}{rgb}{0.000000,0.000000,0.000000}%
\pgfsetstrokecolor{currentstroke}%
\pgfsetdash{}{0pt}%
\pgfpathmoveto{\pgfqpoint{3.555057in}{1.877205in}}%
\pgfpathlineto{\pgfqpoint{3.604852in}{1.863089in}}%
\pgfusepath{stroke}%
\end{pgfscope}%
\begin{pgfscope}%
\definecolor{textcolor}{rgb}{0.000000,0.000000,0.000000}%
\pgfsetstrokecolor{textcolor}%
\pgfsetfillcolor{textcolor}%
\pgftext[x=3.827241in,y=1.908105in,,top]{\color{textcolor}\rmfamily\fontsize{10.000000}{12.000000}\selectfont \(\displaystyle {0.02}\)}%
\end{pgfscope}%
\begin{pgfscope}%
\pgfsetrectcap%
\pgfsetroundjoin%
\pgfsetlinewidth{0.803000pt}%
\definecolor{currentstroke}{rgb}{0.000000,0.000000,0.000000}%
\pgfsetstrokecolor{currentstroke}%
\pgfsetdash{}{0pt}%
\pgfpathmoveto{\pgfqpoint{3.564291in}{2.081035in}}%
\pgfpathlineto{\pgfqpoint{3.614393in}{2.067082in}}%
\pgfusepath{stroke}%
\end{pgfscope}%
\begin{pgfscope}%
\definecolor{textcolor}{rgb}{0.000000,0.000000,0.000000}%
\pgfsetstrokecolor{textcolor}%
\pgfsetfillcolor{textcolor}%
\pgftext[x=3.838060in,y=2.111580in,,top]{\color{textcolor}\rmfamily\fontsize{10.000000}{12.000000}\selectfont \(\displaystyle {0.04}\)}%
\end{pgfscope}%
\begin{pgfscope}%
\pgfsetrectcap%
\pgfsetroundjoin%
\pgfsetlinewidth{0.803000pt}%
\definecolor{currentstroke}{rgb}{0.000000,0.000000,0.000000}%
\pgfsetstrokecolor{currentstroke}%
\pgfsetdash{}{0pt}%
\pgfpathmoveto{\pgfqpoint{3.573636in}{2.287298in}}%
\pgfpathlineto{\pgfqpoint{3.624048in}{2.273512in}}%
\pgfusepath{stroke}%
\end{pgfscope}%
\begin{pgfscope}%
\definecolor{textcolor}{rgb}{0.000000,0.000000,0.000000}%
\pgfsetstrokecolor{textcolor}%
\pgfsetfillcolor{textcolor}%
\pgftext[x=3.849009in,y=2.317476in,,top]{\color{textcolor}\rmfamily\fontsize{10.000000}{12.000000}\selectfont \(\displaystyle {0.06}\)}%
\end{pgfscope}%
\begin{pgfscope}%
\pgfsetrectcap%
\pgfsetroundjoin%
\pgfsetlinewidth{0.803000pt}%
\definecolor{currentstroke}{rgb}{0.000000,0.000000,0.000000}%
\pgfsetstrokecolor{currentstroke}%
\pgfsetdash{}{0pt}%
\pgfpathmoveto{\pgfqpoint{3.583093in}{2.496037in}}%
\pgfpathlineto{\pgfqpoint{3.633819in}{2.482423in}}%
\pgfusepath{stroke}%
\end{pgfscope}%
\begin{pgfscope}%
\definecolor{textcolor}{rgb}{0.000000,0.000000,0.000000}%
\pgfsetstrokecolor{textcolor}%
\pgfsetfillcolor{textcolor}%
\pgftext[x=3.860089in,y=2.525838in,,top]{\color{textcolor}\rmfamily\fontsize{10.000000}{12.000000}\selectfont \(\displaystyle {0.08}\)}%
\end{pgfscope}%
\begin{pgfscope}%
\pgfsetrectcap%
\pgfsetroundjoin%
\pgfsetlinewidth{0.803000pt}%
\definecolor{currentstroke}{rgb}{0.000000,0.000000,0.000000}%
\pgfsetstrokecolor{currentstroke}%
\pgfsetdash{}{0pt}%
\pgfpathmoveto{\pgfqpoint{3.592664in}{2.707296in}}%
\pgfpathlineto{\pgfqpoint{3.643709in}{2.693859in}}%
\pgfusepath{stroke}%
\end{pgfscope}%
\begin{pgfscope}%
\definecolor{textcolor}{rgb}{0.000000,0.000000,0.000000}%
\pgfsetstrokecolor{textcolor}%
\pgfsetfillcolor{textcolor}%
\pgftext[x=3.871302in,y=2.736708in,,top]{\color{textcolor}\rmfamily\fontsize{10.000000}{12.000000}\selectfont \(\displaystyle {0.10}\)}%
\end{pgfscope}%
\begin{pgfscope}%
\pgfsetrectcap%
\pgfsetroundjoin%
\pgfsetlinewidth{0.803000pt}%
\definecolor{currentstroke}{rgb}{0.000000,0.000000,0.000000}%
\pgfsetstrokecolor{currentstroke}%
\pgfsetdash{}{0pt}%
\pgfpathmoveto{\pgfqpoint{3.602351in}{2.921122in}}%
\pgfpathlineto{\pgfqpoint{3.653718in}{2.907868in}}%
\pgfusepath{stroke}%
\end{pgfscope}%
\begin{pgfscope}%
\definecolor{textcolor}{rgb}{0.000000,0.000000,0.000000}%
\pgfsetstrokecolor{textcolor}%
\pgfsetfillcolor{textcolor}%
\pgftext[x=3.882651in,y=2.950134in,,top]{\color{textcolor}\rmfamily\fontsize{10.000000}{12.000000}\selectfont \(\displaystyle {0.12}\)}%
\end{pgfscope}%
\begin{pgfscope}%
\pgfpathrectangle{\pgfqpoint{0.100000in}{0.212622in}}{\pgfqpoint{3.696000in}{3.696000in}}%
\pgfusepath{clip}%
\pgfsetrectcap%
\pgfsetroundjoin%
\pgfsetlinewidth{1.505625pt}%
\definecolor{currentstroke}{rgb}{0.121569,0.466667,0.705882}%
\pgfsetstrokecolor{currentstroke}%
\pgfsetdash{}{0pt}%
\pgfpathmoveto{\pgfqpoint{1.552475in}{2.063642in}}%
\pgfpathlineto{\pgfqpoint{2.869038in}{1.677734in}}%
\pgfusepath{stroke}%
\end{pgfscope}%
\begin{pgfscope}%
\pgfpathrectangle{\pgfqpoint{0.100000in}{0.212622in}}{\pgfqpoint{3.696000in}{3.696000in}}%
\pgfusepath{clip}%
\pgfsetbuttcap%
\pgfsetroundjoin%
\definecolor{currentfill}{rgb}{0.121569,0.466667,0.705882}%
\pgfsetfillcolor{currentfill}%
\pgfsetfillopacity{0.300000}%
\pgfsetlinewidth{1.003750pt}%
\definecolor{currentstroke}{rgb}{0.121569,0.466667,0.705882}%
\pgfsetstrokecolor{currentstroke}%
\pgfsetstrokeopacity{0.300000}%
\pgfsetdash{}{0pt}%
\pgfpathmoveto{\pgfqpoint{1.558889in}{2.036882in}}%
\pgfpathcurveto{\pgfqpoint{1.567126in}{2.036882in}}{\pgfqpoint{1.575026in}{2.040154in}}{\pgfqpoint{1.580850in}{2.045978in}}%
\pgfpathcurveto{\pgfqpoint{1.586673in}{2.051802in}}{\pgfqpoint{1.589946in}{2.059702in}}{\pgfqpoint{1.589946in}{2.067938in}}%
\pgfpathcurveto{\pgfqpoint{1.589946in}{2.076175in}}{\pgfqpoint{1.586673in}{2.084075in}}{\pgfqpoint{1.580850in}{2.089899in}}%
\pgfpathcurveto{\pgfqpoint{1.575026in}{2.095722in}}{\pgfqpoint{1.567126in}{2.098995in}}{\pgfqpoint{1.558889in}{2.098995in}}%
\pgfpathcurveto{\pgfqpoint{1.550653in}{2.098995in}}{\pgfqpoint{1.542753in}{2.095722in}}{\pgfqpoint{1.536929in}{2.089899in}}%
\pgfpathcurveto{\pgfqpoint{1.531105in}{2.084075in}}{\pgfqpoint{1.527833in}{2.076175in}}{\pgfqpoint{1.527833in}{2.067938in}}%
\pgfpathcurveto{\pgfqpoint{1.527833in}{2.059702in}}{\pgfqpoint{1.531105in}{2.051802in}}{\pgfqpoint{1.536929in}{2.045978in}}%
\pgfpathcurveto{\pgfqpoint{1.542753in}{2.040154in}}{\pgfqpoint{1.550653in}{2.036882in}}{\pgfqpoint{1.558889in}{2.036882in}}%
\pgfpathclose%
\pgfusepath{stroke,fill}%
\end{pgfscope}%
\begin{pgfscope}%
\pgfpathrectangle{\pgfqpoint{0.100000in}{0.212622in}}{\pgfqpoint{3.696000in}{3.696000in}}%
\pgfusepath{clip}%
\pgfsetbuttcap%
\pgfsetroundjoin%
\definecolor{currentfill}{rgb}{0.121569,0.466667,0.705882}%
\pgfsetfillcolor{currentfill}%
\pgfsetfillopacity{0.300000}%
\pgfsetlinewidth{1.003750pt}%
\definecolor{currentstroke}{rgb}{0.121569,0.466667,0.705882}%
\pgfsetstrokecolor{currentstroke}%
\pgfsetstrokeopacity{0.300000}%
\pgfsetdash{}{0pt}%
\pgfpathmoveto{\pgfqpoint{1.558889in}{2.036882in}}%
\pgfpathcurveto{\pgfqpoint{1.567126in}{2.036882in}}{\pgfqpoint{1.575026in}{2.040154in}}{\pgfqpoint{1.580850in}{2.045978in}}%
\pgfpathcurveto{\pgfqpoint{1.586674in}{2.051802in}}{\pgfqpoint{1.589946in}{2.059702in}}{\pgfqpoint{1.589946in}{2.067938in}}%
\pgfpathcurveto{\pgfqpoint{1.589946in}{2.076175in}}{\pgfqpoint{1.586674in}{2.084075in}}{\pgfqpoint{1.580850in}{2.089899in}}%
\pgfpathcurveto{\pgfqpoint{1.575026in}{2.095722in}}{\pgfqpoint{1.567126in}{2.098995in}}{\pgfqpoint{1.558889in}{2.098995in}}%
\pgfpathcurveto{\pgfqpoint{1.550653in}{2.098995in}}{\pgfqpoint{1.542753in}{2.095722in}}{\pgfqpoint{1.536929in}{2.089899in}}%
\pgfpathcurveto{\pgfqpoint{1.531105in}{2.084075in}}{\pgfqpoint{1.527833in}{2.076175in}}{\pgfqpoint{1.527833in}{2.067938in}}%
\pgfpathcurveto{\pgfqpoint{1.527833in}{2.059702in}}{\pgfqpoint{1.531105in}{2.051802in}}{\pgfqpoint{1.536929in}{2.045978in}}%
\pgfpathcurveto{\pgfqpoint{1.542753in}{2.040154in}}{\pgfqpoint{1.550653in}{2.036882in}}{\pgfqpoint{1.558889in}{2.036882in}}%
\pgfpathclose%
\pgfusepath{stroke,fill}%
\end{pgfscope}%
\begin{pgfscope}%
\pgfpathrectangle{\pgfqpoint{0.100000in}{0.212622in}}{\pgfqpoint{3.696000in}{3.696000in}}%
\pgfusepath{clip}%
\pgfsetbuttcap%
\pgfsetroundjoin%
\definecolor{currentfill}{rgb}{0.121569,0.466667,0.705882}%
\pgfsetfillcolor{currentfill}%
\pgfsetfillopacity{0.300000}%
\pgfsetlinewidth{1.003750pt}%
\definecolor{currentstroke}{rgb}{0.121569,0.466667,0.705882}%
\pgfsetstrokecolor{currentstroke}%
\pgfsetstrokeopacity{0.300000}%
\pgfsetdash{}{0pt}%
\pgfpathmoveto{\pgfqpoint{1.558889in}{2.036882in}}%
\pgfpathcurveto{\pgfqpoint{1.567126in}{2.036882in}}{\pgfqpoint{1.575026in}{2.040154in}}{\pgfqpoint{1.580850in}{2.045978in}}%
\pgfpathcurveto{\pgfqpoint{1.586674in}{2.051802in}}{\pgfqpoint{1.589946in}{2.059702in}}{\pgfqpoint{1.589946in}{2.067938in}}%
\pgfpathcurveto{\pgfqpoint{1.589946in}{2.076175in}}{\pgfqpoint{1.586674in}{2.084075in}}{\pgfqpoint{1.580850in}{2.089899in}}%
\pgfpathcurveto{\pgfqpoint{1.575026in}{2.095722in}}{\pgfqpoint{1.567126in}{2.098995in}}{\pgfqpoint{1.558889in}{2.098995in}}%
\pgfpathcurveto{\pgfqpoint{1.550653in}{2.098995in}}{\pgfqpoint{1.542753in}{2.095722in}}{\pgfqpoint{1.536929in}{2.089899in}}%
\pgfpathcurveto{\pgfqpoint{1.531105in}{2.084075in}}{\pgfqpoint{1.527833in}{2.076175in}}{\pgfqpoint{1.527833in}{2.067938in}}%
\pgfpathcurveto{\pgfqpoint{1.527833in}{2.059702in}}{\pgfqpoint{1.531105in}{2.051802in}}{\pgfqpoint{1.536929in}{2.045978in}}%
\pgfpathcurveto{\pgfqpoint{1.542753in}{2.040154in}}{\pgfqpoint{1.550653in}{2.036882in}}{\pgfqpoint{1.558889in}{2.036882in}}%
\pgfpathclose%
\pgfusepath{stroke,fill}%
\end{pgfscope}%
\begin{pgfscope}%
\pgfpathrectangle{\pgfqpoint{0.100000in}{0.212622in}}{\pgfqpoint{3.696000in}{3.696000in}}%
\pgfusepath{clip}%
\pgfsetbuttcap%
\pgfsetroundjoin%
\definecolor{currentfill}{rgb}{0.121569,0.466667,0.705882}%
\pgfsetfillcolor{currentfill}%
\pgfsetfillopacity{0.300000}%
\pgfsetlinewidth{1.003750pt}%
\definecolor{currentstroke}{rgb}{0.121569,0.466667,0.705882}%
\pgfsetstrokecolor{currentstroke}%
\pgfsetstrokeopacity{0.300000}%
\pgfsetdash{}{0pt}%
\pgfpathmoveto{\pgfqpoint{1.558889in}{2.036882in}}%
\pgfpathcurveto{\pgfqpoint{1.567126in}{2.036882in}}{\pgfqpoint{1.575026in}{2.040154in}}{\pgfqpoint{1.580850in}{2.045978in}}%
\pgfpathcurveto{\pgfqpoint{1.586674in}{2.051802in}}{\pgfqpoint{1.589946in}{2.059702in}}{\pgfqpoint{1.589946in}{2.067938in}}%
\pgfpathcurveto{\pgfqpoint{1.589946in}{2.076175in}}{\pgfqpoint{1.586674in}{2.084075in}}{\pgfqpoint{1.580850in}{2.089899in}}%
\pgfpathcurveto{\pgfqpoint{1.575026in}{2.095722in}}{\pgfqpoint{1.567126in}{2.098995in}}{\pgfqpoint{1.558889in}{2.098995in}}%
\pgfpathcurveto{\pgfqpoint{1.550653in}{2.098995in}}{\pgfqpoint{1.542753in}{2.095722in}}{\pgfqpoint{1.536929in}{2.089899in}}%
\pgfpathcurveto{\pgfqpoint{1.531105in}{2.084075in}}{\pgfqpoint{1.527833in}{2.076175in}}{\pgfqpoint{1.527833in}{2.067938in}}%
\pgfpathcurveto{\pgfqpoint{1.527833in}{2.059702in}}{\pgfqpoint{1.531105in}{2.051802in}}{\pgfqpoint{1.536929in}{2.045978in}}%
\pgfpathcurveto{\pgfqpoint{1.542753in}{2.040154in}}{\pgfqpoint{1.550653in}{2.036882in}}{\pgfqpoint{1.558889in}{2.036882in}}%
\pgfpathclose%
\pgfusepath{stroke,fill}%
\end{pgfscope}%
\begin{pgfscope}%
\pgfpathrectangle{\pgfqpoint{0.100000in}{0.212622in}}{\pgfqpoint{3.696000in}{3.696000in}}%
\pgfusepath{clip}%
\pgfsetbuttcap%
\pgfsetroundjoin%
\definecolor{currentfill}{rgb}{0.121569,0.466667,0.705882}%
\pgfsetfillcolor{currentfill}%
\pgfsetfillopacity{0.300000}%
\pgfsetlinewidth{1.003750pt}%
\definecolor{currentstroke}{rgb}{0.121569,0.466667,0.705882}%
\pgfsetstrokecolor{currentstroke}%
\pgfsetstrokeopacity{0.300000}%
\pgfsetdash{}{0pt}%
\pgfpathmoveto{\pgfqpoint{1.558889in}{2.036882in}}%
\pgfpathcurveto{\pgfqpoint{1.567126in}{2.036882in}}{\pgfqpoint{1.575026in}{2.040154in}}{\pgfqpoint{1.580850in}{2.045978in}}%
\pgfpathcurveto{\pgfqpoint{1.586674in}{2.051802in}}{\pgfqpoint{1.589946in}{2.059702in}}{\pgfqpoint{1.589946in}{2.067938in}}%
\pgfpathcurveto{\pgfqpoint{1.589946in}{2.076175in}}{\pgfqpoint{1.586674in}{2.084075in}}{\pgfqpoint{1.580850in}{2.089899in}}%
\pgfpathcurveto{\pgfqpoint{1.575026in}{2.095722in}}{\pgfqpoint{1.567126in}{2.098995in}}{\pgfqpoint{1.558889in}{2.098995in}}%
\pgfpathcurveto{\pgfqpoint{1.550653in}{2.098995in}}{\pgfqpoint{1.542753in}{2.095722in}}{\pgfqpoint{1.536929in}{2.089899in}}%
\pgfpathcurveto{\pgfqpoint{1.531105in}{2.084075in}}{\pgfqpoint{1.527833in}{2.076175in}}{\pgfqpoint{1.527833in}{2.067938in}}%
\pgfpathcurveto{\pgfqpoint{1.527833in}{2.059702in}}{\pgfqpoint{1.531105in}{2.051802in}}{\pgfqpoint{1.536929in}{2.045978in}}%
\pgfpathcurveto{\pgfqpoint{1.542753in}{2.040154in}}{\pgfqpoint{1.550653in}{2.036882in}}{\pgfqpoint{1.558889in}{2.036882in}}%
\pgfpathclose%
\pgfusepath{stroke,fill}%
\end{pgfscope}%
\begin{pgfscope}%
\pgfpathrectangle{\pgfqpoint{0.100000in}{0.212622in}}{\pgfqpoint{3.696000in}{3.696000in}}%
\pgfusepath{clip}%
\pgfsetbuttcap%
\pgfsetroundjoin%
\definecolor{currentfill}{rgb}{0.121569,0.466667,0.705882}%
\pgfsetfillcolor{currentfill}%
\pgfsetfillopacity{0.300000}%
\pgfsetlinewidth{1.003750pt}%
\definecolor{currentstroke}{rgb}{0.121569,0.466667,0.705882}%
\pgfsetstrokecolor{currentstroke}%
\pgfsetstrokeopacity{0.300000}%
\pgfsetdash{}{0pt}%
\pgfpathmoveto{\pgfqpoint{1.558889in}{2.036882in}}%
\pgfpathcurveto{\pgfqpoint{1.567126in}{2.036882in}}{\pgfqpoint{1.575026in}{2.040154in}}{\pgfqpoint{1.580850in}{2.045978in}}%
\pgfpathcurveto{\pgfqpoint{1.586674in}{2.051802in}}{\pgfqpoint{1.589946in}{2.059702in}}{\pgfqpoint{1.589946in}{2.067938in}}%
\pgfpathcurveto{\pgfqpoint{1.589946in}{2.076175in}}{\pgfqpoint{1.586674in}{2.084075in}}{\pgfqpoint{1.580850in}{2.089899in}}%
\pgfpathcurveto{\pgfqpoint{1.575026in}{2.095722in}}{\pgfqpoint{1.567126in}{2.098995in}}{\pgfqpoint{1.558889in}{2.098995in}}%
\pgfpathcurveto{\pgfqpoint{1.550653in}{2.098995in}}{\pgfqpoint{1.542753in}{2.095722in}}{\pgfqpoint{1.536929in}{2.089899in}}%
\pgfpathcurveto{\pgfqpoint{1.531105in}{2.084075in}}{\pgfqpoint{1.527833in}{2.076175in}}{\pgfqpoint{1.527833in}{2.067938in}}%
\pgfpathcurveto{\pgfqpoint{1.527833in}{2.059702in}}{\pgfqpoint{1.531105in}{2.051802in}}{\pgfqpoint{1.536929in}{2.045978in}}%
\pgfpathcurveto{\pgfqpoint{1.542753in}{2.040154in}}{\pgfqpoint{1.550653in}{2.036882in}}{\pgfqpoint{1.558889in}{2.036882in}}%
\pgfpathclose%
\pgfusepath{stroke,fill}%
\end{pgfscope}%
\begin{pgfscope}%
\pgfpathrectangle{\pgfqpoint{0.100000in}{0.212622in}}{\pgfqpoint{3.696000in}{3.696000in}}%
\pgfusepath{clip}%
\pgfsetbuttcap%
\pgfsetroundjoin%
\definecolor{currentfill}{rgb}{0.121569,0.466667,0.705882}%
\pgfsetfillcolor{currentfill}%
\pgfsetfillopacity{0.300000}%
\pgfsetlinewidth{1.003750pt}%
\definecolor{currentstroke}{rgb}{0.121569,0.466667,0.705882}%
\pgfsetstrokecolor{currentstroke}%
\pgfsetstrokeopacity{0.300000}%
\pgfsetdash{}{0pt}%
\pgfpathmoveto{\pgfqpoint{1.558889in}{2.036882in}}%
\pgfpathcurveto{\pgfqpoint{1.567126in}{2.036882in}}{\pgfqpoint{1.575026in}{2.040154in}}{\pgfqpoint{1.580850in}{2.045978in}}%
\pgfpathcurveto{\pgfqpoint{1.586674in}{2.051802in}}{\pgfqpoint{1.589946in}{2.059702in}}{\pgfqpoint{1.589946in}{2.067938in}}%
\pgfpathcurveto{\pgfqpoint{1.589946in}{2.076175in}}{\pgfqpoint{1.586674in}{2.084075in}}{\pgfqpoint{1.580850in}{2.089899in}}%
\pgfpathcurveto{\pgfqpoint{1.575026in}{2.095722in}}{\pgfqpoint{1.567126in}{2.098995in}}{\pgfqpoint{1.558889in}{2.098995in}}%
\pgfpathcurveto{\pgfqpoint{1.550653in}{2.098995in}}{\pgfqpoint{1.542753in}{2.095722in}}{\pgfqpoint{1.536929in}{2.089899in}}%
\pgfpathcurveto{\pgfqpoint{1.531105in}{2.084075in}}{\pgfqpoint{1.527833in}{2.076175in}}{\pgfqpoint{1.527833in}{2.067938in}}%
\pgfpathcurveto{\pgfqpoint{1.527833in}{2.059702in}}{\pgfqpoint{1.531105in}{2.051802in}}{\pgfqpoint{1.536929in}{2.045978in}}%
\pgfpathcurveto{\pgfqpoint{1.542753in}{2.040154in}}{\pgfqpoint{1.550653in}{2.036882in}}{\pgfqpoint{1.558889in}{2.036882in}}%
\pgfpathclose%
\pgfusepath{stroke,fill}%
\end{pgfscope}%
\begin{pgfscope}%
\pgfpathrectangle{\pgfqpoint{0.100000in}{0.212622in}}{\pgfqpoint{3.696000in}{3.696000in}}%
\pgfusepath{clip}%
\pgfsetbuttcap%
\pgfsetroundjoin%
\definecolor{currentfill}{rgb}{0.121569,0.466667,0.705882}%
\pgfsetfillcolor{currentfill}%
\pgfsetfillopacity{0.300000}%
\pgfsetlinewidth{1.003750pt}%
\definecolor{currentstroke}{rgb}{0.121569,0.466667,0.705882}%
\pgfsetstrokecolor{currentstroke}%
\pgfsetstrokeopacity{0.300000}%
\pgfsetdash{}{0pt}%
\pgfpathmoveto{\pgfqpoint{1.558889in}{2.036882in}}%
\pgfpathcurveto{\pgfqpoint{1.567126in}{2.036882in}}{\pgfqpoint{1.575026in}{2.040154in}}{\pgfqpoint{1.580850in}{2.045978in}}%
\pgfpathcurveto{\pgfqpoint{1.586674in}{2.051802in}}{\pgfqpoint{1.589946in}{2.059702in}}{\pgfqpoint{1.589946in}{2.067938in}}%
\pgfpathcurveto{\pgfqpoint{1.589946in}{2.076175in}}{\pgfqpoint{1.586674in}{2.084075in}}{\pgfqpoint{1.580850in}{2.089899in}}%
\pgfpathcurveto{\pgfqpoint{1.575026in}{2.095722in}}{\pgfqpoint{1.567126in}{2.098995in}}{\pgfqpoint{1.558889in}{2.098995in}}%
\pgfpathcurveto{\pgfqpoint{1.550653in}{2.098995in}}{\pgfqpoint{1.542753in}{2.095722in}}{\pgfqpoint{1.536929in}{2.089899in}}%
\pgfpathcurveto{\pgfqpoint{1.531105in}{2.084075in}}{\pgfqpoint{1.527833in}{2.076175in}}{\pgfqpoint{1.527833in}{2.067938in}}%
\pgfpathcurveto{\pgfqpoint{1.527833in}{2.059702in}}{\pgfqpoint{1.531105in}{2.051802in}}{\pgfqpoint{1.536929in}{2.045978in}}%
\pgfpathcurveto{\pgfqpoint{1.542753in}{2.040154in}}{\pgfqpoint{1.550653in}{2.036882in}}{\pgfqpoint{1.558889in}{2.036882in}}%
\pgfpathclose%
\pgfusepath{stroke,fill}%
\end{pgfscope}%
\begin{pgfscope}%
\pgfpathrectangle{\pgfqpoint{0.100000in}{0.212622in}}{\pgfqpoint{3.696000in}{3.696000in}}%
\pgfusepath{clip}%
\pgfsetbuttcap%
\pgfsetroundjoin%
\definecolor{currentfill}{rgb}{0.121569,0.466667,0.705882}%
\pgfsetfillcolor{currentfill}%
\pgfsetfillopacity{0.300000}%
\pgfsetlinewidth{1.003750pt}%
\definecolor{currentstroke}{rgb}{0.121569,0.466667,0.705882}%
\pgfsetstrokecolor{currentstroke}%
\pgfsetstrokeopacity{0.300000}%
\pgfsetdash{}{0pt}%
\pgfpathmoveto{\pgfqpoint{1.558889in}{2.036882in}}%
\pgfpathcurveto{\pgfqpoint{1.567126in}{2.036882in}}{\pgfqpoint{1.575026in}{2.040154in}}{\pgfqpoint{1.580850in}{2.045978in}}%
\pgfpathcurveto{\pgfqpoint{1.586674in}{2.051802in}}{\pgfqpoint{1.589946in}{2.059702in}}{\pgfqpoint{1.589946in}{2.067938in}}%
\pgfpathcurveto{\pgfqpoint{1.589946in}{2.076175in}}{\pgfqpoint{1.586674in}{2.084075in}}{\pgfqpoint{1.580850in}{2.089899in}}%
\pgfpathcurveto{\pgfqpoint{1.575026in}{2.095722in}}{\pgfqpoint{1.567126in}{2.098995in}}{\pgfqpoint{1.558889in}{2.098995in}}%
\pgfpathcurveto{\pgfqpoint{1.550653in}{2.098995in}}{\pgfqpoint{1.542753in}{2.095722in}}{\pgfqpoint{1.536929in}{2.089899in}}%
\pgfpathcurveto{\pgfqpoint{1.531105in}{2.084075in}}{\pgfqpoint{1.527833in}{2.076175in}}{\pgfqpoint{1.527833in}{2.067938in}}%
\pgfpathcurveto{\pgfqpoint{1.527833in}{2.059702in}}{\pgfqpoint{1.531105in}{2.051802in}}{\pgfqpoint{1.536929in}{2.045978in}}%
\pgfpathcurveto{\pgfqpoint{1.542753in}{2.040154in}}{\pgfqpoint{1.550653in}{2.036882in}}{\pgfqpoint{1.558889in}{2.036882in}}%
\pgfpathclose%
\pgfusepath{stroke,fill}%
\end{pgfscope}%
\begin{pgfscope}%
\pgfpathrectangle{\pgfqpoint{0.100000in}{0.212622in}}{\pgfqpoint{3.696000in}{3.696000in}}%
\pgfusepath{clip}%
\pgfsetbuttcap%
\pgfsetroundjoin%
\definecolor{currentfill}{rgb}{0.121569,0.466667,0.705882}%
\pgfsetfillcolor{currentfill}%
\pgfsetfillopacity{0.300000}%
\pgfsetlinewidth{1.003750pt}%
\definecolor{currentstroke}{rgb}{0.121569,0.466667,0.705882}%
\pgfsetstrokecolor{currentstroke}%
\pgfsetstrokeopacity{0.300000}%
\pgfsetdash{}{0pt}%
\pgfpathmoveto{\pgfqpoint{1.558889in}{2.036882in}}%
\pgfpathcurveto{\pgfqpoint{1.567126in}{2.036882in}}{\pgfqpoint{1.575026in}{2.040154in}}{\pgfqpoint{1.580850in}{2.045978in}}%
\pgfpathcurveto{\pgfqpoint{1.586674in}{2.051802in}}{\pgfqpoint{1.589946in}{2.059702in}}{\pgfqpoint{1.589946in}{2.067938in}}%
\pgfpathcurveto{\pgfqpoint{1.589946in}{2.076175in}}{\pgfqpoint{1.586674in}{2.084075in}}{\pgfqpoint{1.580850in}{2.089899in}}%
\pgfpathcurveto{\pgfqpoint{1.575026in}{2.095722in}}{\pgfqpoint{1.567126in}{2.098995in}}{\pgfqpoint{1.558889in}{2.098995in}}%
\pgfpathcurveto{\pgfqpoint{1.550653in}{2.098995in}}{\pgfqpoint{1.542753in}{2.095722in}}{\pgfqpoint{1.536929in}{2.089899in}}%
\pgfpathcurveto{\pgfqpoint{1.531105in}{2.084075in}}{\pgfqpoint{1.527833in}{2.076175in}}{\pgfqpoint{1.527833in}{2.067938in}}%
\pgfpathcurveto{\pgfqpoint{1.527833in}{2.059702in}}{\pgfqpoint{1.531105in}{2.051802in}}{\pgfqpoint{1.536929in}{2.045978in}}%
\pgfpathcurveto{\pgfqpoint{1.542753in}{2.040154in}}{\pgfqpoint{1.550653in}{2.036882in}}{\pgfqpoint{1.558889in}{2.036882in}}%
\pgfpathclose%
\pgfusepath{stroke,fill}%
\end{pgfscope}%
\begin{pgfscope}%
\pgfpathrectangle{\pgfqpoint{0.100000in}{0.212622in}}{\pgfqpoint{3.696000in}{3.696000in}}%
\pgfusepath{clip}%
\pgfsetbuttcap%
\pgfsetroundjoin%
\definecolor{currentfill}{rgb}{0.121569,0.466667,0.705882}%
\pgfsetfillcolor{currentfill}%
\pgfsetfillopacity{0.300000}%
\pgfsetlinewidth{1.003750pt}%
\definecolor{currentstroke}{rgb}{0.121569,0.466667,0.705882}%
\pgfsetstrokecolor{currentstroke}%
\pgfsetstrokeopacity{0.300000}%
\pgfsetdash{}{0pt}%
\pgfpathmoveto{\pgfqpoint{1.558889in}{2.036882in}}%
\pgfpathcurveto{\pgfqpoint{1.567126in}{2.036882in}}{\pgfqpoint{1.575026in}{2.040154in}}{\pgfqpoint{1.580850in}{2.045978in}}%
\pgfpathcurveto{\pgfqpoint{1.586674in}{2.051802in}}{\pgfqpoint{1.589946in}{2.059702in}}{\pgfqpoint{1.589946in}{2.067938in}}%
\pgfpathcurveto{\pgfqpoint{1.589946in}{2.076175in}}{\pgfqpoint{1.586674in}{2.084075in}}{\pgfqpoint{1.580850in}{2.089899in}}%
\pgfpathcurveto{\pgfqpoint{1.575026in}{2.095722in}}{\pgfqpoint{1.567126in}{2.098995in}}{\pgfqpoint{1.558889in}{2.098995in}}%
\pgfpathcurveto{\pgfqpoint{1.550653in}{2.098995in}}{\pgfqpoint{1.542753in}{2.095722in}}{\pgfqpoint{1.536929in}{2.089899in}}%
\pgfpathcurveto{\pgfqpoint{1.531105in}{2.084075in}}{\pgfqpoint{1.527833in}{2.076175in}}{\pgfqpoint{1.527833in}{2.067938in}}%
\pgfpathcurveto{\pgfqpoint{1.527833in}{2.059702in}}{\pgfqpoint{1.531105in}{2.051802in}}{\pgfqpoint{1.536929in}{2.045978in}}%
\pgfpathcurveto{\pgfqpoint{1.542753in}{2.040154in}}{\pgfqpoint{1.550653in}{2.036882in}}{\pgfqpoint{1.558889in}{2.036882in}}%
\pgfpathclose%
\pgfusepath{stroke,fill}%
\end{pgfscope}%
\begin{pgfscope}%
\pgfpathrectangle{\pgfqpoint{0.100000in}{0.212622in}}{\pgfqpoint{3.696000in}{3.696000in}}%
\pgfusepath{clip}%
\pgfsetbuttcap%
\pgfsetroundjoin%
\definecolor{currentfill}{rgb}{0.121569,0.466667,0.705882}%
\pgfsetfillcolor{currentfill}%
\pgfsetfillopacity{0.300000}%
\pgfsetlinewidth{1.003750pt}%
\definecolor{currentstroke}{rgb}{0.121569,0.466667,0.705882}%
\pgfsetstrokecolor{currentstroke}%
\pgfsetstrokeopacity{0.300000}%
\pgfsetdash{}{0pt}%
\pgfpathmoveto{\pgfqpoint{1.558889in}{2.036882in}}%
\pgfpathcurveto{\pgfqpoint{1.567126in}{2.036882in}}{\pgfqpoint{1.575026in}{2.040154in}}{\pgfqpoint{1.580850in}{2.045978in}}%
\pgfpathcurveto{\pgfqpoint{1.586674in}{2.051802in}}{\pgfqpoint{1.589946in}{2.059702in}}{\pgfqpoint{1.589946in}{2.067938in}}%
\pgfpathcurveto{\pgfqpoint{1.589946in}{2.076175in}}{\pgfqpoint{1.586674in}{2.084075in}}{\pgfqpoint{1.580850in}{2.089899in}}%
\pgfpathcurveto{\pgfqpoint{1.575026in}{2.095722in}}{\pgfqpoint{1.567126in}{2.098995in}}{\pgfqpoint{1.558889in}{2.098995in}}%
\pgfpathcurveto{\pgfqpoint{1.550653in}{2.098995in}}{\pgfqpoint{1.542753in}{2.095722in}}{\pgfqpoint{1.536929in}{2.089899in}}%
\pgfpathcurveto{\pgfqpoint{1.531105in}{2.084075in}}{\pgfqpoint{1.527833in}{2.076175in}}{\pgfqpoint{1.527833in}{2.067938in}}%
\pgfpathcurveto{\pgfqpoint{1.527833in}{2.059702in}}{\pgfqpoint{1.531105in}{2.051802in}}{\pgfqpoint{1.536929in}{2.045978in}}%
\pgfpathcurveto{\pgfqpoint{1.542753in}{2.040154in}}{\pgfqpoint{1.550653in}{2.036882in}}{\pgfqpoint{1.558889in}{2.036882in}}%
\pgfpathclose%
\pgfusepath{stroke,fill}%
\end{pgfscope}%
\begin{pgfscope}%
\pgfpathrectangle{\pgfqpoint{0.100000in}{0.212622in}}{\pgfqpoint{3.696000in}{3.696000in}}%
\pgfusepath{clip}%
\pgfsetbuttcap%
\pgfsetroundjoin%
\definecolor{currentfill}{rgb}{0.121569,0.466667,0.705882}%
\pgfsetfillcolor{currentfill}%
\pgfsetfillopacity{0.300000}%
\pgfsetlinewidth{1.003750pt}%
\definecolor{currentstroke}{rgb}{0.121569,0.466667,0.705882}%
\pgfsetstrokecolor{currentstroke}%
\pgfsetstrokeopacity{0.300000}%
\pgfsetdash{}{0pt}%
\pgfpathmoveto{\pgfqpoint{1.558889in}{2.036882in}}%
\pgfpathcurveto{\pgfqpoint{1.567126in}{2.036882in}}{\pgfqpoint{1.575026in}{2.040154in}}{\pgfqpoint{1.580850in}{2.045978in}}%
\pgfpathcurveto{\pgfqpoint{1.586674in}{2.051802in}}{\pgfqpoint{1.589946in}{2.059702in}}{\pgfqpoint{1.589946in}{2.067938in}}%
\pgfpathcurveto{\pgfqpoint{1.589946in}{2.076175in}}{\pgfqpoint{1.586674in}{2.084075in}}{\pgfqpoint{1.580850in}{2.089899in}}%
\pgfpathcurveto{\pgfqpoint{1.575026in}{2.095722in}}{\pgfqpoint{1.567126in}{2.098995in}}{\pgfqpoint{1.558889in}{2.098995in}}%
\pgfpathcurveto{\pgfqpoint{1.550653in}{2.098995in}}{\pgfqpoint{1.542753in}{2.095722in}}{\pgfqpoint{1.536929in}{2.089899in}}%
\pgfpathcurveto{\pgfqpoint{1.531105in}{2.084075in}}{\pgfqpoint{1.527833in}{2.076175in}}{\pgfqpoint{1.527833in}{2.067938in}}%
\pgfpathcurveto{\pgfqpoint{1.527833in}{2.059702in}}{\pgfqpoint{1.531105in}{2.051802in}}{\pgfqpoint{1.536929in}{2.045978in}}%
\pgfpathcurveto{\pgfqpoint{1.542753in}{2.040154in}}{\pgfqpoint{1.550653in}{2.036882in}}{\pgfqpoint{1.558889in}{2.036882in}}%
\pgfpathclose%
\pgfusepath{stroke,fill}%
\end{pgfscope}%
\begin{pgfscope}%
\pgfpathrectangle{\pgfqpoint{0.100000in}{0.212622in}}{\pgfqpoint{3.696000in}{3.696000in}}%
\pgfusepath{clip}%
\pgfsetbuttcap%
\pgfsetroundjoin%
\definecolor{currentfill}{rgb}{0.121569,0.466667,0.705882}%
\pgfsetfillcolor{currentfill}%
\pgfsetfillopacity{0.300000}%
\pgfsetlinewidth{1.003750pt}%
\definecolor{currentstroke}{rgb}{0.121569,0.466667,0.705882}%
\pgfsetstrokecolor{currentstroke}%
\pgfsetstrokeopacity{0.300000}%
\pgfsetdash{}{0pt}%
\pgfpathmoveto{\pgfqpoint{1.558889in}{2.036882in}}%
\pgfpathcurveto{\pgfqpoint{1.567126in}{2.036882in}}{\pgfqpoint{1.575026in}{2.040154in}}{\pgfqpoint{1.580850in}{2.045978in}}%
\pgfpathcurveto{\pgfqpoint{1.586674in}{2.051802in}}{\pgfqpoint{1.589946in}{2.059702in}}{\pgfqpoint{1.589946in}{2.067938in}}%
\pgfpathcurveto{\pgfqpoint{1.589946in}{2.076175in}}{\pgfqpoint{1.586674in}{2.084075in}}{\pgfqpoint{1.580850in}{2.089899in}}%
\pgfpathcurveto{\pgfqpoint{1.575026in}{2.095722in}}{\pgfqpoint{1.567126in}{2.098995in}}{\pgfqpoint{1.558889in}{2.098995in}}%
\pgfpathcurveto{\pgfqpoint{1.550653in}{2.098995in}}{\pgfqpoint{1.542753in}{2.095722in}}{\pgfqpoint{1.536929in}{2.089899in}}%
\pgfpathcurveto{\pgfqpoint{1.531105in}{2.084075in}}{\pgfqpoint{1.527833in}{2.076175in}}{\pgfqpoint{1.527833in}{2.067938in}}%
\pgfpathcurveto{\pgfqpoint{1.527833in}{2.059702in}}{\pgfqpoint{1.531105in}{2.051802in}}{\pgfqpoint{1.536929in}{2.045978in}}%
\pgfpathcurveto{\pgfqpoint{1.542753in}{2.040154in}}{\pgfqpoint{1.550653in}{2.036882in}}{\pgfqpoint{1.558889in}{2.036882in}}%
\pgfpathclose%
\pgfusepath{stroke,fill}%
\end{pgfscope}%
\begin{pgfscope}%
\pgfpathrectangle{\pgfqpoint{0.100000in}{0.212622in}}{\pgfqpoint{3.696000in}{3.696000in}}%
\pgfusepath{clip}%
\pgfsetbuttcap%
\pgfsetroundjoin%
\definecolor{currentfill}{rgb}{0.121569,0.466667,0.705882}%
\pgfsetfillcolor{currentfill}%
\pgfsetfillopacity{0.300000}%
\pgfsetlinewidth{1.003750pt}%
\definecolor{currentstroke}{rgb}{0.121569,0.466667,0.705882}%
\pgfsetstrokecolor{currentstroke}%
\pgfsetstrokeopacity{0.300000}%
\pgfsetdash{}{0pt}%
\pgfpathmoveto{\pgfqpoint{1.558889in}{2.036882in}}%
\pgfpathcurveto{\pgfqpoint{1.567126in}{2.036882in}}{\pgfqpoint{1.575026in}{2.040154in}}{\pgfqpoint{1.580850in}{2.045978in}}%
\pgfpathcurveto{\pgfqpoint{1.586674in}{2.051802in}}{\pgfqpoint{1.589946in}{2.059702in}}{\pgfqpoint{1.589946in}{2.067938in}}%
\pgfpathcurveto{\pgfqpoint{1.589946in}{2.076175in}}{\pgfqpoint{1.586674in}{2.084075in}}{\pgfqpoint{1.580850in}{2.089899in}}%
\pgfpathcurveto{\pgfqpoint{1.575026in}{2.095722in}}{\pgfqpoint{1.567126in}{2.098995in}}{\pgfqpoint{1.558889in}{2.098995in}}%
\pgfpathcurveto{\pgfqpoint{1.550653in}{2.098995in}}{\pgfqpoint{1.542753in}{2.095722in}}{\pgfqpoint{1.536929in}{2.089899in}}%
\pgfpathcurveto{\pgfqpoint{1.531105in}{2.084075in}}{\pgfqpoint{1.527833in}{2.076175in}}{\pgfqpoint{1.527833in}{2.067938in}}%
\pgfpathcurveto{\pgfqpoint{1.527833in}{2.059702in}}{\pgfqpoint{1.531105in}{2.051802in}}{\pgfqpoint{1.536929in}{2.045978in}}%
\pgfpathcurveto{\pgfqpoint{1.542753in}{2.040154in}}{\pgfqpoint{1.550653in}{2.036882in}}{\pgfqpoint{1.558889in}{2.036882in}}%
\pgfpathclose%
\pgfusepath{stroke,fill}%
\end{pgfscope}%
\begin{pgfscope}%
\pgfpathrectangle{\pgfqpoint{0.100000in}{0.212622in}}{\pgfqpoint{3.696000in}{3.696000in}}%
\pgfusepath{clip}%
\pgfsetbuttcap%
\pgfsetroundjoin%
\definecolor{currentfill}{rgb}{0.121569,0.466667,0.705882}%
\pgfsetfillcolor{currentfill}%
\pgfsetfillopacity{0.300000}%
\pgfsetlinewidth{1.003750pt}%
\definecolor{currentstroke}{rgb}{0.121569,0.466667,0.705882}%
\pgfsetstrokecolor{currentstroke}%
\pgfsetstrokeopacity{0.300000}%
\pgfsetdash{}{0pt}%
\pgfpathmoveto{\pgfqpoint{1.558889in}{2.036882in}}%
\pgfpathcurveto{\pgfqpoint{1.567126in}{2.036882in}}{\pgfqpoint{1.575026in}{2.040154in}}{\pgfqpoint{1.580850in}{2.045978in}}%
\pgfpathcurveto{\pgfqpoint{1.586674in}{2.051802in}}{\pgfqpoint{1.589946in}{2.059702in}}{\pgfqpoint{1.589946in}{2.067938in}}%
\pgfpathcurveto{\pgfqpoint{1.589946in}{2.076175in}}{\pgfqpoint{1.586674in}{2.084075in}}{\pgfqpoint{1.580850in}{2.089899in}}%
\pgfpathcurveto{\pgfqpoint{1.575026in}{2.095722in}}{\pgfqpoint{1.567126in}{2.098995in}}{\pgfqpoint{1.558889in}{2.098995in}}%
\pgfpathcurveto{\pgfqpoint{1.550653in}{2.098995in}}{\pgfqpoint{1.542753in}{2.095722in}}{\pgfqpoint{1.536929in}{2.089899in}}%
\pgfpathcurveto{\pgfqpoint{1.531105in}{2.084075in}}{\pgfqpoint{1.527833in}{2.076175in}}{\pgfqpoint{1.527833in}{2.067938in}}%
\pgfpathcurveto{\pgfqpoint{1.527833in}{2.059702in}}{\pgfqpoint{1.531105in}{2.051802in}}{\pgfqpoint{1.536929in}{2.045978in}}%
\pgfpathcurveto{\pgfqpoint{1.542753in}{2.040154in}}{\pgfqpoint{1.550653in}{2.036882in}}{\pgfqpoint{1.558889in}{2.036882in}}%
\pgfpathclose%
\pgfusepath{stroke,fill}%
\end{pgfscope}%
\begin{pgfscope}%
\pgfpathrectangle{\pgfqpoint{0.100000in}{0.212622in}}{\pgfqpoint{3.696000in}{3.696000in}}%
\pgfusepath{clip}%
\pgfsetbuttcap%
\pgfsetroundjoin%
\definecolor{currentfill}{rgb}{0.121569,0.466667,0.705882}%
\pgfsetfillcolor{currentfill}%
\pgfsetfillopacity{0.300000}%
\pgfsetlinewidth{1.003750pt}%
\definecolor{currentstroke}{rgb}{0.121569,0.466667,0.705882}%
\pgfsetstrokecolor{currentstroke}%
\pgfsetstrokeopacity{0.300000}%
\pgfsetdash{}{0pt}%
\pgfpathmoveto{\pgfqpoint{1.558889in}{2.036882in}}%
\pgfpathcurveto{\pgfqpoint{1.567126in}{2.036882in}}{\pgfqpoint{1.575026in}{2.040154in}}{\pgfqpoint{1.580850in}{2.045978in}}%
\pgfpathcurveto{\pgfqpoint{1.586674in}{2.051802in}}{\pgfqpoint{1.589946in}{2.059702in}}{\pgfqpoint{1.589946in}{2.067938in}}%
\pgfpathcurveto{\pgfqpoint{1.589946in}{2.076175in}}{\pgfqpoint{1.586674in}{2.084075in}}{\pgfqpoint{1.580850in}{2.089899in}}%
\pgfpathcurveto{\pgfqpoint{1.575026in}{2.095722in}}{\pgfqpoint{1.567126in}{2.098995in}}{\pgfqpoint{1.558889in}{2.098995in}}%
\pgfpathcurveto{\pgfqpoint{1.550653in}{2.098995in}}{\pgfqpoint{1.542753in}{2.095722in}}{\pgfqpoint{1.536929in}{2.089899in}}%
\pgfpathcurveto{\pgfqpoint{1.531105in}{2.084075in}}{\pgfqpoint{1.527833in}{2.076175in}}{\pgfqpoint{1.527833in}{2.067938in}}%
\pgfpathcurveto{\pgfqpoint{1.527833in}{2.059702in}}{\pgfqpoint{1.531105in}{2.051802in}}{\pgfqpoint{1.536929in}{2.045978in}}%
\pgfpathcurveto{\pgfqpoint{1.542753in}{2.040154in}}{\pgfqpoint{1.550653in}{2.036882in}}{\pgfqpoint{1.558889in}{2.036882in}}%
\pgfpathclose%
\pgfusepath{stroke,fill}%
\end{pgfscope}%
\begin{pgfscope}%
\pgfpathrectangle{\pgfqpoint{0.100000in}{0.212622in}}{\pgfqpoint{3.696000in}{3.696000in}}%
\pgfusepath{clip}%
\pgfsetbuttcap%
\pgfsetroundjoin%
\definecolor{currentfill}{rgb}{0.121569,0.466667,0.705882}%
\pgfsetfillcolor{currentfill}%
\pgfsetfillopacity{0.300000}%
\pgfsetlinewidth{1.003750pt}%
\definecolor{currentstroke}{rgb}{0.121569,0.466667,0.705882}%
\pgfsetstrokecolor{currentstroke}%
\pgfsetstrokeopacity{0.300000}%
\pgfsetdash{}{0pt}%
\pgfpathmoveto{\pgfqpoint{1.558889in}{2.036882in}}%
\pgfpathcurveto{\pgfqpoint{1.567126in}{2.036882in}}{\pgfqpoint{1.575026in}{2.040154in}}{\pgfqpoint{1.580850in}{2.045978in}}%
\pgfpathcurveto{\pgfqpoint{1.586674in}{2.051802in}}{\pgfqpoint{1.589946in}{2.059702in}}{\pgfqpoint{1.589946in}{2.067938in}}%
\pgfpathcurveto{\pgfqpoint{1.589946in}{2.076175in}}{\pgfqpoint{1.586674in}{2.084075in}}{\pgfqpoint{1.580850in}{2.089899in}}%
\pgfpathcurveto{\pgfqpoint{1.575026in}{2.095722in}}{\pgfqpoint{1.567126in}{2.098995in}}{\pgfqpoint{1.558889in}{2.098995in}}%
\pgfpathcurveto{\pgfqpoint{1.550653in}{2.098995in}}{\pgfqpoint{1.542753in}{2.095722in}}{\pgfqpoint{1.536929in}{2.089899in}}%
\pgfpathcurveto{\pgfqpoint{1.531105in}{2.084075in}}{\pgfqpoint{1.527833in}{2.076175in}}{\pgfqpoint{1.527833in}{2.067938in}}%
\pgfpathcurveto{\pgfqpoint{1.527833in}{2.059702in}}{\pgfqpoint{1.531105in}{2.051802in}}{\pgfqpoint{1.536929in}{2.045978in}}%
\pgfpathcurveto{\pgfqpoint{1.542753in}{2.040154in}}{\pgfqpoint{1.550653in}{2.036882in}}{\pgfqpoint{1.558889in}{2.036882in}}%
\pgfpathclose%
\pgfusepath{stroke,fill}%
\end{pgfscope}%
\begin{pgfscope}%
\pgfpathrectangle{\pgfqpoint{0.100000in}{0.212622in}}{\pgfqpoint{3.696000in}{3.696000in}}%
\pgfusepath{clip}%
\pgfsetbuttcap%
\pgfsetroundjoin%
\definecolor{currentfill}{rgb}{0.121569,0.466667,0.705882}%
\pgfsetfillcolor{currentfill}%
\pgfsetfillopacity{0.300000}%
\pgfsetlinewidth{1.003750pt}%
\definecolor{currentstroke}{rgb}{0.121569,0.466667,0.705882}%
\pgfsetstrokecolor{currentstroke}%
\pgfsetstrokeopacity{0.300000}%
\pgfsetdash{}{0pt}%
\pgfpathmoveto{\pgfqpoint{1.558889in}{2.036882in}}%
\pgfpathcurveto{\pgfqpoint{1.567126in}{2.036882in}}{\pgfqpoint{1.575026in}{2.040154in}}{\pgfqpoint{1.580850in}{2.045978in}}%
\pgfpathcurveto{\pgfqpoint{1.586674in}{2.051802in}}{\pgfqpoint{1.589946in}{2.059702in}}{\pgfqpoint{1.589946in}{2.067938in}}%
\pgfpathcurveto{\pgfqpoint{1.589946in}{2.076175in}}{\pgfqpoint{1.586674in}{2.084075in}}{\pgfqpoint{1.580850in}{2.089899in}}%
\pgfpathcurveto{\pgfqpoint{1.575026in}{2.095722in}}{\pgfqpoint{1.567126in}{2.098995in}}{\pgfqpoint{1.558889in}{2.098995in}}%
\pgfpathcurveto{\pgfqpoint{1.550653in}{2.098995in}}{\pgfqpoint{1.542753in}{2.095722in}}{\pgfqpoint{1.536929in}{2.089899in}}%
\pgfpathcurveto{\pgfqpoint{1.531105in}{2.084075in}}{\pgfqpoint{1.527833in}{2.076175in}}{\pgfqpoint{1.527833in}{2.067938in}}%
\pgfpathcurveto{\pgfqpoint{1.527833in}{2.059702in}}{\pgfqpoint{1.531105in}{2.051802in}}{\pgfqpoint{1.536929in}{2.045978in}}%
\pgfpathcurveto{\pgfqpoint{1.542753in}{2.040154in}}{\pgfqpoint{1.550653in}{2.036882in}}{\pgfqpoint{1.558889in}{2.036882in}}%
\pgfpathclose%
\pgfusepath{stroke,fill}%
\end{pgfscope}%
\begin{pgfscope}%
\pgfpathrectangle{\pgfqpoint{0.100000in}{0.212622in}}{\pgfqpoint{3.696000in}{3.696000in}}%
\pgfusepath{clip}%
\pgfsetbuttcap%
\pgfsetroundjoin%
\definecolor{currentfill}{rgb}{0.121569,0.466667,0.705882}%
\pgfsetfillcolor{currentfill}%
\pgfsetfillopacity{0.300000}%
\pgfsetlinewidth{1.003750pt}%
\definecolor{currentstroke}{rgb}{0.121569,0.466667,0.705882}%
\pgfsetstrokecolor{currentstroke}%
\pgfsetstrokeopacity{0.300000}%
\pgfsetdash{}{0pt}%
\pgfpathmoveto{\pgfqpoint{1.558889in}{2.036882in}}%
\pgfpathcurveto{\pgfqpoint{1.567126in}{2.036882in}}{\pgfqpoint{1.575026in}{2.040154in}}{\pgfqpoint{1.580850in}{2.045978in}}%
\pgfpathcurveto{\pgfqpoint{1.586674in}{2.051802in}}{\pgfqpoint{1.589946in}{2.059702in}}{\pgfqpoint{1.589946in}{2.067938in}}%
\pgfpathcurveto{\pgfqpoint{1.589946in}{2.076175in}}{\pgfqpoint{1.586674in}{2.084075in}}{\pgfqpoint{1.580850in}{2.089899in}}%
\pgfpathcurveto{\pgfqpoint{1.575026in}{2.095722in}}{\pgfqpoint{1.567126in}{2.098995in}}{\pgfqpoint{1.558889in}{2.098995in}}%
\pgfpathcurveto{\pgfqpoint{1.550653in}{2.098995in}}{\pgfqpoint{1.542753in}{2.095722in}}{\pgfqpoint{1.536929in}{2.089899in}}%
\pgfpathcurveto{\pgfqpoint{1.531105in}{2.084075in}}{\pgfqpoint{1.527833in}{2.076175in}}{\pgfqpoint{1.527833in}{2.067938in}}%
\pgfpathcurveto{\pgfqpoint{1.527833in}{2.059702in}}{\pgfqpoint{1.531105in}{2.051802in}}{\pgfqpoint{1.536929in}{2.045978in}}%
\pgfpathcurveto{\pgfqpoint{1.542753in}{2.040154in}}{\pgfqpoint{1.550653in}{2.036882in}}{\pgfqpoint{1.558889in}{2.036882in}}%
\pgfpathclose%
\pgfusepath{stroke,fill}%
\end{pgfscope}%
\begin{pgfscope}%
\pgfpathrectangle{\pgfqpoint{0.100000in}{0.212622in}}{\pgfqpoint{3.696000in}{3.696000in}}%
\pgfusepath{clip}%
\pgfsetbuttcap%
\pgfsetroundjoin%
\definecolor{currentfill}{rgb}{0.121569,0.466667,0.705882}%
\pgfsetfillcolor{currentfill}%
\pgfsetfillopacity{0.300000}%
\pgfsetlinewidth{1.003750pt}%
\definecolor{currentstroke}{rgb}{0.121569,0.466667,0.705882}%
\pgfsetstrokecolor{currentstroke}%
\pgfsetstrokeopacity{0.300000}%
\pgfsetdash{}{0pt}%
\pgfpathmoveto{\pgfqpoint{1.558889in}{2.036882in}}%
\pgfpathcurveto{\pgfqpoint{1.567126in}{2.036882in}}{\pgfqpoint{1.575026in}{2.040154in}}{\pgfqpoint{1.580850in}{2.045978in}}%
\pgfpathcurveto{\pgfqpoint{1.586674in}{2.051802in}}{\pgfqpoint{1.589946in}{2.059702in}}{\pgfqpoint{1.589946in}{2.067938in}}%
\pgfpathcurveto{\pgfqpoint{1.589946in}{2.076175in}}{\pgfqpoint{1.586674in}{2.084075in}}{\pgfqpoint{1.580850in}{2.089899in}}%
\pgfpathcurveto{\pgfqpoint{1.575026in}{2.095722in}}{\pgfqpoint{1.567126in}{2.098995in}}{\pgfqpoint{1.558889in}{2.098995in}}%
\pgfpathcurveto{\pgfqpoint{1.550653in}{2.098995in}}{\pgfqpoint{1.542753in}{2.095722in}}{\pgfqpoint{1.536929in}{2.089899in}}%
\pgfpathcurveto{\pgfqpoint{1.531105in}{2.084075in}}{\pgfqpoint{1.527833in}{2.076175in}}{\pgfqpoint{1.527833in}{2.067938in}}%
\pgfpathcurveto{\pgfqpoint{1.527833in}{2.059702in}}{\pgfqpoint{1.531105in}{2.051802in}}{\pgfqpoint{1.536929in}{2.045978in}}%
\pgfpathcurveto{\pgfqpoint{1.542753in}{2.040154in}}{\pgfqpoint{1.550653in}{2.036882in}}{\pgfqpoint{1.558889in}{2.036882in}}%
\pgfpathclose%
\pgfusepath{stroke,fill}%
\end{pgfscope}%
\begin{pgfscope}%
\pgfpathrectangle{\pgfqpoint{0.100000in}{0.212622in}}{\pgfqpoint{3.696000in}{3.696000in}}%
\pgfusepath{clip}%
\pgfsetbuttcap%
\pgfsetroundjoin%
\definecolor{currentfill}{rgb}{0.121569,0.466667,0.705882}%
\pgfsetfillcolor{currentfill}%
\pgfsetfillopacity{0.300000}%
\pgfsetlinewidth{1.003750pt}%
\definecolor{currentstroke}{rgb}{0.121569,0.466667,0.705882}%
\pgfsetstrokecolor{currentstroke}%
\pgfsetstrokeopacity{0.300000}%
\pgfsetdash{}{0pt}%
\pgfpathmoveto{\pgfqpoint{1.558889in}{2.036882in}}%
\pgfpathcurveto{\pgfqpoint{1.567126in}{2.036882in}}{\pgfqpoint{1.575026in}{2.040154in}}{\pgfqpoint{1.580850in}{2.045978in}}%
\pgfpathcurveto{\pgfqpoint{1.586674in}{2.051802in}}{\pgfqpoint{1.589946in}{2.059702in}}{\pgfqpoint{1.589946in}{2.067938in}}%
\pgfpathcurveto{\pgfqpoint{1.589946in}{2.076175in}}{\pgfqpoint{1.586674in}{2.084075in}}{\pgfqpoint{1.580850in}{2.089899in}}%
\pgfpathcurveto{\pgfqpoint{1.575026in}{2.095722in}}{\pgfqpoint{1.567126in}{2.098995in}}{\pgfqpoint{1.558889in}{2.098995in}}%
\pgfpathcurveto{\pgfqpoint{1.550653in}{2.098995in}}{\pgfqpoint{1.542753in}{2.095722in}}{\pgfqpoint{1.536929in}{2.089899in}}%
\pgfpathcurveto{\pgfqpoint{1.531105in}{2.084075in}}{\pgfqpoint{1.527833in}{2.076175in}}{\pgfqpoint{1.527833in}{2.067938in}}%
\pgfpathcurveto{\pgfqpoint{1.527833in}{2.059702in}}{\pgfqpoint{1.531105in}{2.051802in}}{\pgfqpoint{1.536929in}{2.045978in}}%
\pgfpathcurveto{\pgfqpoint{1.542753in}{2.040154in}}{\pgfqpoint{1.550653in}{2.036882in}}{\pgfqpoint{1.558889in}{2.036882in}}%
\pgfpathclose%
\pgfusepath{stroke,fill}%
\end{pgfscope}%
\begin{pgfscope}%
\pgfpathrectangle{\pgfqpoint{0.100000in}{0.212622in}}{\pgfqpoint{3.696000in}{3.696000in}}%
\pgfusepath{clip}%
\pgfsetbuttcap%
\pgfsetroundjoin%
\definecolor{currentfill}{rgb}{0.121569,0.466667,0.705882}%
\pgfsetfillcolor{currentfill}%
\pgfsetfillopacity{0.300000}%
\pgfsetlinewidth{1.003750pt}%
\definecolor{currentstroke}{rgb}{0.121569,0.466667,0.705882}%
\pgfsetstrokecolor{currentstroke}%
\pgfsetstrokeopacity{0.300000}%
\pgfsetdash{}{0pt}%
\pgfpathmoveto{\pgfqpoint{1.558889in}{2.036882in}}%
\pgfpathcurveto{\pgfqpoint{1.567126in}{2.036882in}}{\pgfqpoint{1.575026in}{2.040154in}}{\pgfqpoint{1.580850in}{2.045978in}}%
\pgfpathcurveto{\pgfqpoint{1.586674in}{2.051802in}}{\pgfqpoint{1.589946in}{2.059702in}}{\pgfqpoint{1.589946in}{2.067938in}}%
\pgfpathcurveto{\pgfqpoint{1.589946in}{2.076175in}}{\pgfqpoint{1.586674in}{2.084075in}}{\pgfqpoint{1.580850in}{2.089899in}}%
\pgfpathcurveto{\pgfqpoint{1.575026in}{2.095722in}}{\pgfqpoint{1.567126in}{2.098995in}}{\pgfqpoint{1.558889in}{2.098995in}}%
\pgfpathcurveto{\pgfqpoint{1.550653in}{2.098995in}}{\pgfqpoint{1.542753in}{2.095722in}}{\pgfqpoint{1.536929in}{2.089899in}}%
\pgfpathcurveto{\pgfqpoint{1.531105in}{2.084075in}}{\pgfqpoint{1.527833in}{2.076175in}}{\pgfqpoint{1.527833in}{2.067938in}}%
\pgfpathcurveto{\pgfqpoint{1.527833in}{2.059702in}}{\pgfqpoint{1.531105in}{2.051802in}}{\pgfqpoint{1.536929in}{2.045978in}}%
\pgfpathcurveto{\pgfqpoint{1.542753in}{2.040154in}}{\pgfqpoint{1.550653in}{2.036882in}}{\pgfqpoint{1.558889in}{2.036882in}}%
\pgfpathclose%
\pgfusepath{stroke,fill}%
\end{pgfscope}%
\begin{pgfscope}%
\pgfpathrectangle{\pgfqpoint{0.100000in}{0.212622in}}{\pgfqpoint{3.696000in}{3.696000in}}%
\pgfusepath{clip}%
\pgfsetbuttcap%
\pgfsetroundjoin%
\definecolor{currentfill}{rgb}{0.121569,0.466667,0.705882}%
\pgfsetfillcolor{currentfill}%
\pgfsetfillopacity{0.300000}%
\pgfsetlinewidth{1.003750pt}%
\definecolor{currentstroke}{rgb}{0.121569,0.466667,0.705882}%
\pgfsetstrokecolor{currentstroke}%
\pgfsetstrokeopacity{0.300000}%
\pgfsetdash{}{0pt}%
\pgfpathmoveto{\pgfqpoint{1.558889in}{2.036882in}}%
\pgfpathcurveto{\pgfqpoint{1.567126in}{2.036882in}}{\pgfqpoint{1.575026in}{2.040154in}}{\pgfqpoint{1.580850in}{2.045978in}}%
\pgfpathcurveto{\pgfqpoint{1.586674in}{2.051802in}}{\pgfqpoint{1.589946in}{2.059702in}}{\pgfqpoint{1.589946in}{2.067938in}}%
\pgfpathcurveto{\pgfqpoint{1.589946in}{2.076175in}}{\pgfqpoint{1.586674in}{2.084075in}}{\pgfqpoint{1.580850in}{2.089899in}}%
\pgfpathcurveto{\pgfqpoint{1.575026in}{2.095722in}}{\pgfqpoint{1.567126in}{2.098995in}}{\pgfqpoint{1.558889in}{2.098995in}}%
\pgfpathcurveto{\pgfqpoint{1.550653in}{2.098995in}}{\pgfqpoint{1.542753in}{2.095722in}}{\pgfqpoint{1.536929in}{2.089899in}}%
\pgfpathcurveto{\pgfqpoint{1.531105in}{2.084075in}}{\pgfqpoint{1.527833in}{2.076175in}}{\pgfqpoint{1.527833in}{2.067938in}}%
\pgfpathcurveto{\pgfqpoint{1.527833in}{2.059702in}}{\pgfqpoint{1.531105in}{2.051802in}}{\pgfqpoint{1.536929in}{2.045978in}}%
\pgfpathcurveto{\pgfqpoint{1.542753in}{2.040154in}}{\pgfqpoint{1.550653in}{2.036882in}}{\pgfqpoint{1.558889in}{2.036882in}}%
\pgfpathclose%
\pgfusepath{stroke,fill}%
\end{pgfscope}%
\begin{pgfscope}%
\pgfpathrectangle{\pgfqpoint{0.100000in}{0.212622in}}{\pgfqpoint{3.696000in}{3.696000in}}%
\pgfusepath{clip}%
\pgfsetbuttcap%
\pgfsetroundjoin%
\definecolor{currentfill}{rgb}{0.121569,0.466667,0.705882}%
\pgfsetfillcolor{currentfill}%
\pgfsetfillopacity{0.300000}%
\pgfsetlinewidth{1.003750pt}%
\definecolor{currentstroke}{rgb}{0.121569,0.466667,0.705882}%
\pgfsetstrokecolor{currentstroke}%
\pgfsetstrokeopacity{0.300000}%
\pgfsetdash{}{0pt}%
\pgfpathmoveto{\pgfqpoint{1.558889in}{2.036882in}}%
\pgfpathcurveto{\pgfqpoint{1.567126in}{2.036882in}}{\pgfqpoint{1.575026in}{2.040154in}}{\pgfqpoint{1.580850in}{2.045978in}}%
\pgfpathcurveto{\pgfqpoint{1.586674in}{2.051802in}}{\pgfqpoint{1.589946in}{2.059702in}}{\pgfqpoint{1.589946in}{2.067938in}}%
\pgfpathcurveto{\pgfqpoint{1.589946in}{2.076175in}}{\pgfqpoint{1.586674in}{2.084075in}}{\pgfqpoint{1.580850in}{2.089899in}}%
\pgfpathcurveto{\pgfqpoint{1.575026in}{2.095722in}}{\pgfqpoint{1.567126in}{2.098995in}}{\pgfqpoint{1.558889in}{2.098995in}}%
\pgfpathcurveto{\pgfqpoint{1.550653in}{2.098995in}}{\pgfqpoint{1.542753in}{2.095722in}}{\pgfqpoint{1.536929in}{2.089899in}}%
\pgfpathcurveto{\pgfqpoint{1.531105in}{2.084075in}}{\pgfqpoint{1.527833in}{2.076175in}}{\pgfqpoint{1.527833in}{2.067938in}}%
\pgfpathcurveto{\pgfqpoint{1.527833in}{2.059702in}}{\pgfqpoint{1.531105in}{2.051802in}}{\pgfqpoint{1.536929in}{2.045978in}}%
\pgfpathcurveto{\pgfqpoint{1.542753in}{2.040154in}}{\pgfqpoint{1.550653in}{2.036882in}}{\pgfqpoint{1.558889in}{2.036882in}}%
\pgfpathclose%
\pgfusepath{stroke,fill}%
\end{pgfscope}%
\begin{pgfscope}%
\pgfpathrectangle{\pgfqpoint{0.100000in}{0.212622in}}{\pgfqpoint{3.696000in}{3.696000in}}%
\pgfusepath{clip}%
\pgfsetbuttcap%
\pgfsetroundjoin%
\definecolor{currentfill}{rgb}{0.121569,0.466667,0.705882}%
\pgfsetfillcolor{currentfill}%
\pgfsetfillopacity{0.300000}%
\pgfsetlinewidth{1.003750pt}%
\definecolor{currentstroke}{rgb}{0.121569,0.466667,0.705882}%
\pgfsetstrokecolor{currentstroke}%
\pgfsetstrokeopacity{0.300000}%
\pgfsetdash{}{0pt}%
\pgfpathmoveto{\pgfqpoint{1.558889in}{2.036882in}}%
\pgfpathcurveto{\pgfqpoint{1.567126in}{2.036882in}}{\pgfqpoint{1.575026in}{2.040154in}}{\pgfqpoint{1.580850in}{2.045978in}}%
\pgfpathcurveto{\pgfqpoint{1.586674in}{2.051802in}}{\pgfqpoint{1.589946in}{2.059702in}}{\pgfqpoint{1.589946in}{2.067938in}}%
\pgfpathcurveto{\pgfqpoint{1.589946in}{2.076175in}}{\pgfqpoint{1.586674in}{2.084075in}}{\pgfqpoint{1.580850in}{2.089899in}}%
\pgfpathcurveto{\pgfqpoint{1.575026in}{2.095722in}}{\pgfqpoint{1.567126in}{2.098995in}}{\pgfqpoint{1.558889in}{2.098995in}}%
\pgfpathcurveto{\pgfqpoint{1.550653in}{2.098995in}}{\pgfqpoint{1.542753in}{2.095722in}}{\pgfqpoint{1.536929in}{2.089899in}}%
\pgfpathcurveto{\pgfqpoint{1.531105in}{2.084075in}}{\pgfqpoint{1.527833in}{2.076175in}}{\pgfqpoint{1.527833in}{2.067938in}}%
\pgfpathcurveto{\pgfqpoint{1.527833in}{2.059702in}}{\pgfqpoint{1.531105in}{2.051802in}}{\pgfqpoint{1.536929in}{2.045978in}}%
\pgfpathcurveto{\pgfqpoint{1.542753in}{2.040154in}}{\pgfqpoint{1.550653in}{2.036882in}}{\pgfqpoint{1.558889in}{2.036882in}}%
\pgfpathclose%
\pgfusepath{stroke,fill}%
\end{pgfscope}%
\begin{pgfscope}%
\pgfpathrectangle{\pgfqpoint{0.100000in}{0.212622in}}{\pgfqpoint{3.696000in}{3.696000in}}%
\pgfusepath{clip}%
\pgfsetbuttcap%
\pgfsetroundjoin%
\definecolor{currentfill}{rgb}{0.121569,0.466667,0.705882}%
\pgfsetfillcolor{currentfill}%
\pgfsetfillopacity{0.300000}%
\pgfsetlinewidth{1.003750pt}%
\definecolor{currentstroke}{rgb}{0.121569,0.466667,0.705882}%
\pgfsetstrokecolor{currentstroke}%
\pgfsetstrokeopacity{0.300000}%
\pgfsetdash{}{0pt}%
\pgfpathmoveto{\pgfqpoint{1.558889in}{2.036882in}}%
\pgfpathcurveto{\pgfqpoint{1.567126in}{2.036882in}}{\pgfqpoint{1.575026in}{2.040154in}}{\pgfqpoint{1.580850in}{2.045978in}}%
\pgfpathcurveto{\pgfqpoint{1.586674in}{2.051802in}}{\pgfqpoint{1.589946in}{2.059702in}}{\pgfqpoint{1.589946in}{2.067938in}}%
\pgfpathcurveto{\pgfqpoint{1.589946in}{2.076175in}}{\pgfqpoint{1.586674in}{2.084075in}}{\pgfqpoint{1.580850in}{2.089899in}}%
\pgfpathcurveto{\pgfqpoint{1.575026in}{2.095722in}}{\pgfqpoint{1.567126in}{2.098995in}}{\pgfqpoint{1.558889in}{2.098995in}}%
\pgfpathcurveto{\pgfqpoint{1.550653in}{2.098995in}}{\pgfqpoint{1.542753in}{2.095722in}}{\pgfqpoint{1.536929in}{2.089899in}}%
\pgfpathcurveto{\pgfqpoint{1.531105in}{2.084075in}}{\pgfqpoint{1.527833in}{2.076175in}}{\pgfqpoint{1.527833in}{2.067938in}}%
\pgfpathcurveto{\pgfqpoint{1.527833in}{2.059702in}}{\pgfqpoint{1.531105in}{2.051802in}}{\pgfqpoint{1.536929in}{2.045978in}}%
\pgfpathcurveto{\pgfqpoint{1.542753in}{2.040154in}}{\pgfqpoint{1.550653in}{2.036882in}}{\pgfqpoint{1.558889in}{2.036882in}}%
\pgfpathclose%
\pgfusepath{stroke,fill}%
\end{pgfscope}%
\begin{pgfscope}%
\pgfpathrectangle{\pgfqpoint{0.100000in}{0.212622in}}{\pgfqpoint{3.696000in}{3.696000in}}%
\pgfusepath{clip}%
\pgfsetbuttcap%
\pgfsetroundjoin%
\definecolor{currentfill}{rgb}{0.121569,0.466667,0.705882}%
\pgfsetfillcolor{currentfill}%
\pgfsetfillopacity{0.300000}%
\pgfsetlinewidth{1.003750pt}%
\definecolor{currentstroke}{rgb}{0.121569,0.466667,0.705882}%
\pgfsetstrokecolor{currentstroke}%
\pgfsetstrokeopacity{0.300000}%
\pgfsetdash{}{0pt}%
\pgfpathmoveto{\pgfqpoint{1.558889in}{2.036882in}}%
\pgfpathcurveto{\pgfqpoint{1.567126in}{2.036882in}}{\pgfqpoint{1.575026in}{2.040154in}}{\pgfqpoint{1.580850in}{2.045978in}}%
\pgfpathcurveto{\pgfqpoint{1.586674in}{2.051802in}}{\pgfqpoint{1.589946in}{2.059702in}}{\pgfqpoint{1.589946in}{2.067938in}}%
\pgfpathcurveto{\pgfqpoint{1.589946in}{2.076175in}}{\pgfqpoint{1.586674in}{2.084075in}}{\pgfqpoint{1.580850in}{2.089899in}}%
\pgfpathcurveto{\pgfqpoint{1.575026in}{2.095722in}}{\pgfqpoint{1.567126in}{2.098995in}}{\pgfqpoint{1.558889in}{2.098995in}}%
\pgfpathcurveto{\pgfqpoint{1.550653in}{2.098995in}}{\pgfqpoint{1.542753in}{2.095722in}}{\pgfqpoint{1.536929in}{2.089899in}}%
\pgfpathcurveto{\pgfqpoint{1.531105in}{2.084075in}}{\pgfqpoint{1.527833in}{2.076175in}}{\pgfqpoint{1.527833in}{2.067938in}}%
\pgfpathcurveto{\pgfqpoint{1.527833in}{2.059702in}}{\pgfqpoint{1.531105in}{2.051802in}}{\pgfqpoint{1.536929in}{2.045978in}}%
\pgfpathcurveto{\pgfqpoint{1.542753in}{2.040154in}}{\pgfqpoint{1.550653in}{2.036882in}}{\pgfqpoint{1.558889in}{2.036882in}}%
\pgfpathclose%
\pgfusepath{stroke,fill}%
\end{pgfscope}%
\begin{pgfscope}%
\pgfpathrectangle{\pgfqpoint{0.100000in}{0.212622in}}{\pgfqpoint{3.696000in}{3.696000in}}%
\pgfusepath{clip}%
\pgfsetbuttcap%
\pgfsetroundjoin%
\definecolor{currentfill}{rgb}{0.121569,0.466667,0.705882}%
\pgfsetfillcolor{currentfill}%
\pgfsetfillopacity{0.300000}%
\pgfsetlinewidth{1.003750pt}%
\definecolor{currentstroke}{rgb}{0.121569,0.466667,0.705882}%
\pgfsetstrokecolor{currentstroke}%
\pgfsetstrokeopacity{0.300000}%
\pgfsetdash{}{0pt}%
\pgfpathmoveto{\pgfqpoint{1.558889in}{2.036882in}}%
\pgfpathcurveto{\pgfqpoint{1.567126in}{2.036882in}}{\pgfqpoint{1.575026in}{2.040154in}}{\pgfqpoint{1.580850in}{2.045978in}}%
\pgfpathcurveto{\pgfqpoint{1.586674in}{2.051802in}}{\pgfqpoint{1.589946in}{2.059702in}}{\pgfqpoint{1.589946in}{2.067938in}}%
\pgfpathcurveto{\pgfqpoint{1.589946in}{2.076175in}}{\pgfqpoint{1.586674in}{2.084075in}}{\pgfqpoint{1.580850in}{2.089899in}}%
\pgfpathcurveto{\pgfqpoint{1.575026in}{2.095722in}}{\pgfqpoint{1.567126in}{2.098995in}}{\pgfqpoint{1.558889in}{2.098995in}}%
\pgfpathcurveto{\pgfqpoint{1.550653in}{2.098995in}}{\pgfqpoint{1.542753in}{2.095722in}}{\pgfqpoint{1.536929in}{2.089899in}}%
\pgfpathcurveto{\pgfqpoint{1.531105in}{2.084075in}}{\pgfqpoint{1.527833in}{2.076175in}}{\pgfqpoint{1.527833in}{2.067938in}}%
\pgfpathcurveto{\pgfqpoint{1.527833in}{2.059702in}}{\pgfqpoint{1.531105in}{2.051802in}}{\pgfqpoint{1.536929in}{2.045978in}}%
\pgfpathcurveto{\pgfqpoint{1.542753in}{2.040154in}}{\pgfqpoint{1.550653in}{2.036882in}}{\pgfqpoint{1.558889in}{2.036882in}}%
\pgfpathclose%
\pgfusepath{stroke,fill}%
\end{pgfscope}%
\begin{pgfscope}%
\pgfpathrectangle{\pgfqpoint{0.100000in}{0.212622in}}{\pgfqpoint{3.696000in}{3.696000in}}%
\pgfusepath{clip}%
\pgfsetbuttcap%
\pgfsetroundjoin%
\definecolor{currentfill}{rgb}{0.121569,0.466667,0.705882}%
\pgfsetfillcolor{currentfill}%
\pgfsetfillopacity{0.300000}%
\pgfsetlinewidth{1.003750pt}%
\definecolor{currentstroke}{rgb}{0.121569,0.466667,0.705882}%
\pgfsetstrokecolor{currentstroke}%
\pgfsetstrokeopacity{0.300000}%
\pgfsetdash{}{0pt}%
\pgfpathmoveto{\pgfqpoint{1.558889in}{2.036882in}}%
\pgfpathcurveto{\pgfqpoint{1.567126in}{2.036882in}}{\pgfqpoint{1.575026in}{2.040154in}}{\pgfqpoint{1.580850in}{2.045978in}}%
\pgfpathcurveto{\pgfqpoint{1.586674in}{2.051802in}}{\pgfqpoint{1.589946in}{2.059702in}}{\pgfqpoint{1.589946in}{2.067938in}}%
\pgfpathcurveto{\pgfqpoint{1.589946in}{2.076175in}}{\pgfqpoint{1.586674in}{2.084075in}}{\pgfqpoint{1.580850in}{2.089899in}}%
\pgfpathcurveto{\pgfqpoint{1.575026in}{2.095722in}}{\pgfqpoint{1.567126in}{2.098995in}}{\pgfqpoint{1.558889in}{2.098995in}}%
\pgfpathcurveto{\pgfqpoint{1.550653in}{2.098995in}}{\pgfqpoint{1.542753in}{2.095722in}}{\pgfqpoint{1.536929in}{2.089899in}}%
\pgfpathcurveto{\pgfqpoint{1.531105in}{2.084075in}}{\pgfqpoint{1.527833in}{2.076175in}}{\pgfqpoint{1.527833in}{2.067938in}}%
\pgfpathcurveto{\pgfqpoint{1.527833in}{2.059702in}}{\pgfqpoint{1.531105in}{2.051802in}}{\pgfqpoint{1.536929in}{2.045978in}}%
\pgfpathcurveto{\pgfqpoint{1.542753in}{2.040154in}}{\pgfqpoint{1.550653in}{2.036882in}}{\pgfqpoint{1.558889in}{2.036882in}}%
\pgfpathclose%
\pgfusepath{stroke,fill}%
\end{pgfscope}%
\begin{pgfscope}%
\pgfpathrectangle{\pgfqpoint{0.100000in}{0.212622in}}{\pgfqpoint{3.696000in}{3.696000in}}%
\pgfusepath{clip}%
\pgfsetbuttcap%
\pgfsetroundjoin%
\definecolor{currentfill}{rgb}{0.121569,0.466667,0.705882}%
\pgfsetfillcolor{currentfill}%
\pgfsetfillopacity{0.300000}%
\pgfsetlinewidth{1.003750pt}%
\definecolor{currentstroke}{rgb}{0.121569,0.466667,0.705882}%
\pgfsetstrokecolor{currentstroke}%
\pgfsetstrokeopacity{0.300000}%
\pgfsetdash{}{0pt}%
\pgfpathmoveto{\pgfqpoint{1.558889in}{2.036882in}}%
\pgfpathcurveto{\pgfqpoint{1.567126in}{2.036882in}}{\pgfqpoint{1.575026in}{2.040154in}}{\pgfqpoint{1.580850in}{2.045978in}}%
\pgfpathcurveto{\pgfqpoint{1.586674in}{2.051802in}}{\pgfqpoint{1.589946in}{2.059702in}}{\pgfqpoint{1.589946in}{2.067938in}}%
\pgfpathcurveto{\pgfqpoint{1.589946in}{2.076175in}}{\pgfqpoint{1.586674in}{2.084075in}}{\pgfqpoint{1.580850in}{2.089899in}}%
\pgfpathcurveto{\pgfqpoint{1.575026in}{2.095722in}}{\pgfqpoint{1.567126in}{2.098995in}}{\pgfqpoint{1.558889in}{2.098995in}}%
\pgfpathcurveto{\pgfqpoint{1.550653in}{2.098995in}}{\pgfqpoint{1.542753in}{2.095722in}}{\pgfqpoint{1.536929in}{2.089899in}}%
\pgfpathcurveto{\pgfqpoint{1.531105in}{2.084075in}}{\pgfqpoint{1.527833in}{2.076175in}}{\pgfqpoint{1.527833in}{2.067938in}}%
\pgfpathcurveto{\pgfqpoint{1.527833in}{2.059702in}}{\pgfqpoint{1.531105in}{2.051802in}}{\pgfqpoint{1.536929in}{2.045978in}}%
\pgfpathcurveto{\pgfqpoint{1.542753in}{2.040154in}}{\pgfqpoint{1.550653in}{2.036882in}}{\pgfqpoint{1.558889in}{2.036882in}}%
\pgfpathclose%
\pgfusepath{stroke,fill}%
\end{pgfscope}%
\begin{pgfscope}%
\pgfpathrectangle{\pgfqpoint{0.100000in}{0.212622in}}{\pgfqpoint{3.696000in}{3.696000in}}%
\pgfusepath{clip}%
\pgfsetbuttcap%
\pgfsetroundjoin%
\definecolor{currentfill}{rgb}{0.121569,0.466667,0.705882}%
\pgfsetfillcolor{currentfill}%
\pgfsetfillopacity{0.300000}%
\pgfsetlinewidth{1.003750pt}%
\definecolor{currentstroke}{rgb}{0.121569,0.466667,0.705882}%
\pgfsetstrokecolor{currentstroke}%
\pgfsetstrokeopacity{0.300000}%
\pgfsetdash{}{0pt}%
\pgfpathmoveto{\pgfqpoint{1.558889in}{2.036882in}}%
\pgfpathcurveto{\pgfqpoint{1.567126in}{2.036882in}}{\pgfqpoint{1.575026in}{2.040154in}}{\pgfqpoint{1.580850in}{2.045978in}}%
\pgfpathcurveto{\pgfqpoint{1.586674in}{2.051802in}}{\pgfqpoint{1.589946in}{2.059702in}}{\pgfqpoint{1.589946in}{2.067938in}}%
\pgfpathcurveto{\pgfqpoint{1.589946in}{2.076175in}}{\pgfqpoint{1.586674in}{2.084075in}}{\pgfqpoint{1.580850in}{2.089899in}}%
\pgfpathcurveto{\pgfqpoint{1.575026in}{2.095722in}}{\pgfqpoint{1.567126in}{2.098995in}}{\pgfqpoint{1.558889in}{2.098995in}}%
\pgfpathcurveto{\pgfqpoint{1.550653in}{2.098995in}}{\pgfqpoint{1.542753in}{2.095722in}}{\pgfqpoint{1.536929in}{2.089899in}}%
\pgfpathcurveto{\pgfqpoint{1.531105in}{2.084075in}}{\pgfqpoint{1.527833in}{2.076175in}}{\pgfqpoint{1.527833in}{2.067938in}}%
\pgfpathcurveto{\pgfqpoint{1.527833in}{2.059702in}}{\pgfqpoint{1.531105in}{2.051802in}}{\pgfqpoint{1.536929in}{2.045978in}}%
\pgfpathcurveto{\pgfqpoint{1.542753in}{2.040154in}}{\pgfqpoint{1.550653in}{2.036882in}}{\pgfqpoint{1.558889in}{2.036882in}}%
\pgfpathclose%
\pgfusepath{stroke,fill}%
\end{pgfscope}%
\begin{pgfscope}%
\pgfpathrectangle{\pgfqpoint{0.100000in}{0.212622in}}{\pgfqpoint{3.696000in}{3.696000in}}%
\pgfusepath{clip}%
\pgfsetbuttcap%
\pgfsetroundjoin%
\definecolor{currentfill}{rgb}{0.121569,0.466667,0.705882}%
\pgfsetfillcolor{currentfill}%
\pgfsetfillopacity{0.300000}%
\pgfsetlinewidth{1.003750pt}%
\definecolor{currentstroke}{rgb}{0.121569,0.466667,0.705882}%
\pgfsetstrokecolor{currentstroke}%
\pgfsetstrokeopacity{0.300000}%
\pgfsetdash{}{0pt}%
\pgfpathmoveto{\pgfqpoint{1.558889in}{2.036882in}}%
\pgfpathcurveto{\pgfqpoint{1.567126in}{2.036882in}}{\pgfqpoint{1.575026in}{2.040154in}}{\pgfqpoint{1.580850in}{2.045978in}}%
\pgfpathcurveto{\pgfqpoint{1.586674in}{2.051802in}}{\pgfqpoint{1.589946in}{2.059702in}}{\pgfqpoint{1.589946in}{2.067938in}}%
\pgfpathcurveto{\pgfqpoint{1.589946in}{2.076175in}}{\pgfqpoint{1.586674in}{2.084075in}}{\pgfqpoint{1.580850in}{2.089899in}}%
\pgfpathcurveto{\pgfqpoint{1.575026in}{2.095722in}}{\pgfqpoint{1.567126in}{2.098995in}}{\pgfqpoint{1.558889in}{2.098995in}}%
\pgfpathcurveto{\pgfqpoint{1.550653in}{2.098995in}}{\pgfqpoint{1.542753in}{2.095722in}}{\pgfqpoint{1.536929in}{2.089899in}}%
\pgfpathcurveto{\pgfqpoint{1.531105in}{2.084075in}}{\pgfqpoint{1.527833in}{2.076175in}}{\pgfqpoint{1.527833in}{2.067938in}}%
\pgfpathcurveto{\pgfqpoint{1.527833in}{2.059702in}}{\pgfqpoint{1.531105in}{2.051802in}}{\pgfqpoint{1.536929in}{2.045978in}}%
\pgfpathcurveto{\pgfqpoint{1.542753in}{2.040154in}}{\pgfqpoint{1.550653in}{2.036882in}}{\pgfqpoint{1.558889in}{2.036882in}}%
\pgfpathclose%
\pgfusepath{stroke,fill}%
\end{pgfscope}%
\begin{pgfscope}%
\pgfpathrectangle{\pgfqpoint{0.100000in}{0.212622in}}{\pgfqpoint{3.696000in}{3.696000in}}%
\pgfusepath{clip}%
\pgfsetbuttcap%
\pgfsetroundjoin%
\definecolor{currentfill}{rgb}{0.121569,0.466667,0.705882}%
\pgfsetfillcolor{currentfill}%
\pgfsetfillopacity{0.300000}%
\pgfsetlinewidth{1.003750pt}%
\definecolor{currentstroke}{rgb}{0.121569,0.466667,0.705882}%
\pgfsetstrokecolor{currentstroke}%
\pgfsetstrokeopacity{0.300000}%
\pgfsetdash{}{0pt}%
\pgfpathmoveto{\pgfqpoint{1.558889in}{2.036882in}}%
\pgfpathcurveto{\pgfqpoint{1.567126in}{2.036882in}}{\pgfqpoint{1.575026in}{2.040154in}}{\pgfqpoint{1.580850in}{2.045978in}}%
\pgfpathcurveto{\pgfqpoint{1.586674in}{2.051802in}}{\pgfqpoint{1.589946in}{2.059702in}}{\pgfqpoint{1.589946in}{2.067938in}}%
\pgfpathcurveto{\pgfqpoint{1.589946in}{2.076175in}}{\pgfqpoint{1.586674in}{2.084075in}}{\pgfqpoint{1.580850in}{2.089899in}}%
\pgfpathcurveto{\pgfqpoint{1.575026in}{2.095722in}}{\pgfqpoint{1.567126in}{2.098995in}}{\pgfqpoint{1.558889in}{2.098995in}}%
\pgfpathcurveto{\pgfqpoint{1.550653in}{2.098995in}}{\pgfqpoint{1.542753in}{2.095722in}}{\pgfqpoint{1.536929in}{2.089899in}}%
\pgfpathcurveto{\pgfqpoint{1.531105in}{2.084075in}}{\pgfqpoint{1.527833in}{2.076175in}}{\pgfqpoint{1.527833in}{2.067938in}}%
\pgfpathcurveto{\pgfqpoint{1.527833in}{2.059702in}}{\pgfqpoint{1.531105in}{2.051802in}}{\pgfqpoint{1.536929in}{2.045978in}}%
\pgfpathcurveto{\pgfqpoint{1.542753in}{2.040154in}}{\pgfqpoint{1.550653in}{2.036882in}}{\pgfqpoint{1.558889in}{2.036882in}}%
\pgfpathclose%
\pgfusepath{stroke,fill}%
\end{pgfscope}%
\begin{pgfscope}%
\pgfpathrectangle{\pgfqpoint{0.100000in}{0.212622in}}{\pgfqpoint{3.696000in}{3.696000in}}%
\pgfusepath{clip}%
\pgfsetbuttcap%
\pgfsetroundjoin%
\definecolor{currentfill}{rgb}{0.121569,0.466667,0.705882}%
\pgfsetfillcolor{currentfill}%
\pgfsetfillopacity{0.300000}%
\pgfsetlinewidth{1.003750pt}%
\definecolor{currentstroke}{rgb}{0.121569,0.466667,0.705882}%
\pgfsetstrokecolor{currentstroke}%
\pgfsetstrokeopacity{0.300000}%
\pgfsetdash{}{0pt}%
\pgfpathmoveto{\pgfqpoint{1.558889in}{2.036882in}}%
\pgfpathcurveto{\pgfqpoint{1.567126in}{2.036882in}}{\pgfqpoint{1.575026in}{2.040154in}}{\pgfqpoint{1.580850in}{2.045978in}}%
\pgfpathcurveto{\pgfqpoint{1.586674in}{2.051802in}}{\pgfqpoint{1.589946in}{2.059702in}}{\pgfqpoint{1.589946in}{2.067938in}}%
\pgfpathcurveto{\pgfqpoint{1.589946in}{2.076175in}}{\pgfqpoint{1.586674in}{2.084075in}}{\pgfqpoint{1.580850in}{2.089899in}}%
\pgfpathcurveto{\pgfqpoint{1.575026in}{2.095722in}}{\pgfqpoint{1.567126in}{2.098995in}}{\pgfqpoint{1.558889in}{2.098995in}}%
\pgfpathcurveto{\pgfqpoint{1.550653in}{2.098995in}}{\pgfqpoint{1.542753in}{2.095722in}}{\pgfqpoint{1.536929in}{2.089899in}}%
\pgfpathcurveto{\pgfqpoint{1.531105in}{2.084075in}}{\pgfqpoint{1.527833in}{2.076175in}}{\pgfqpoint{1.527833in}{2.067938in}}%
\pgfpathcurveto{\pgfqpoint{1.527833in}{2.059702in}}{\pgfqpoint{1.531105in}{2.051802in}}{\pgfqpoint{1.536929in}{2.045978in}}%
\pgfpathcurveto{\pgfqpoint{1.542753in}{2.040154in}}{\pgfqpoint{1.550653in}{2.036882in}}{\pgfqpoint{1.558889in}{2.036882in}}%
\pgfpathclose%
\pgfusepath{stroke,fill}%
\end{pgfscope}%
\begin{pgfscope}%
\pgfpathrectangle{\pgfqpoint{0.100000in}{0.212622in}}{\pgfqpoint{3.696000in}{3.696000in}}%
\pgfusepath{clip}%
\pgfsetbuttcap%
\pgfsetroundjoin%
\definecolor{currentfill}{rgb}{0.121569,0.466667,0.705882}%
\pgfsetfillcolor{currentfill}%
\pgfsetfillopacity{0.300000}%
\pgfsetlinewidth{1.003750pt}%
\definecolor{currentstroke}{rgb}{0.121569,0.466667,0.705882}%
\pgfsetstrokecolor{currentstroke}%
\pgfsetstrokeopacity{0.300000}%
\pgfsetdash{}{0pt}%
\pgfpathmoveto{\pgfqpoint{1.558889in}{2.036882in}}%
\pgfpathcurveto{\pgfqpoint{1.567126in}{2.036882in}}{\pgfqpoint{1.575026in}{2.040154in}}{\pgfqpoint{1.580850in}{2.045978in}}%
\pgfpathcurveto{\pgfqpoint{1.586674in}{2.051802in}}{\pgfqpoint{1.589946in}{2.059702in}}{\pgfqpoint{1.589946in}{2.067938in}}%
\pgfpathcurveto{\pgfqpoint{1.589946in}{2.076175in}}{\pgfqpoint{1.586674in}{2.084075in}}{\pgfqpoint{1.580850in}{2.089899in}}%
\pgfpathcurveto{\pgfqpoint{1.575026in}{2.095722in}}{\pgfqpoint{1.567126in}{2.098995in}}{\pgfqpoint{1.558889in}{2.098995in}}%
\pgfpathcurveto{\pgfqpoint{1.550653in}{2.098995in}}{\pgfqpoint{1.542753in}{2.095722in}}{\pgfqpoint{1.536929in}{2.089899in}}%
\pgfpathcurveto{\pgfqpoint{1.531105in}{2.084075in}}{\pgfqpoint{1.527833in}{2.076175in}}{\pgfqpoint{1.527833in}{2.067938in}}%
\pgfpathcurveto{\pgfqpoint{1.527833in}{2.059702in}}{\pgfqpoint{1.531105in}{2.051802in}}{\pgfqpoint{1.536929in}{2.045978in}}%
\pgfpathcurveto{\pgfqpoint{1.542753in}{2.040154in}}{\pgfqpoint{1.550653in}{2.036882in}}{\pgfqpoint{1.558889in}{2.036882in}}%
\pgfpathclose%
\pgfusepath{stroke,fill}%
\end{pgfscope}%
\begin{pgfscope}%
\pgfpathrectangle{\pgfqpoint{0.100000in}{0.212622in}}{\pgfqpoint{3.696000in}{3.696000in}}%
\pgfusepath{clip}%
\pgfsetbuttcap%
\pgfsetroundjoin%
\definecolor{currentfill}{rgb}{0.121569,0.466667,0.705882}%
\pgfsetfillcolor{currentfill}%
\pgfsetfillopacity{0.300000}%
\pgfsetlinewidth{1.003750pt}%
\definecolor{currentstroke}{rgb}{0.121569,0.466667,0.705882}%
\pgfsetstrokecolor{currentstroke}%
\pgfsetstrokeopacity{0.300000}%
\pgfsetdash{}{0pt}%
\pgfpathmoveto{\pgfqpoint{1.558889in}{2.036882in}}%
\pgfpathcurveto{\pgfqpoint{1.567126in}{2.036882in}}{\pgfqpoint{1.575026in}{2.040154in}}{\pgfqpoint{1.580850in}{2.045978in}}%
\pgfpathcurveto{\pgfqpoint{1.586674in}{2.051802in}}{\pgfqpoint{1.589946in}{2.059702in}}{\pgfqpoint{1.589946in}{2.067938in}}%
\pgfpathcurveto{\pgfqpoint{1.589946in}{2.076175in}}{\pgfqpoint{1.586674in}{2.084075in}}{\pgfqpoint{1.580850in}{2.089899in}}%
\pgfpathcurveto{\pgfqpoint{1.575026in}{2.095722in}}{\pgfqpoint{1.567126in}{2.098995in}}{\pgfqpoint{1.558889in}{2.098995in}}%
\pgfpathcurveto{\pgfqpoint{1.550653in}{2.098995in}}{\pgfqpoint{1.542753in}{2.095722in}}{\pgfqpoint{1.536929in}{2.089899in}}%
\pgfpathcurveto{\pgfqpoint{1.531105in}{2.084075in}}{\pgfqpoint{1.527833in}{2.076175in}}{\pgfqpoint{1.527833in}{2.067938in}}%
\pgfpathcurveto{\pgfqpoint{1.527833in}{2.059702in}}{\pgfqpoint{1.531105in}{2.051802in}}{\pgfqpoint{1.536929in}{2.045978in}}%
\pgfpathcurveto{\pgfqpoint{1.542753in}{2.040154in}}{\pgfqpoint{1.550653in}{2.036882in}}{\pgfqpoint{1.558889in}{2.036882in}}%
\pgfpathclose%
\pgfusepath{stroke,fill}%
\end{pgfscope}%
\begin{pgfscope}%
\pgfpathrectangle{\pgfqpoint{0.100000in}{0.212622in}}{\pgfqpoint{3.696000in}{3.696000in}}%
\pgfusepath{clip}%
\pgfsetbuttcap%
\pgfsetroundjoin%
\definecolor{currentfill}{rgb}{0.121569,0.466667,0.705882}%
\pgfsetfillcolor{currentfill}%
\pgfsetfillopacity{0.300000}%
\pgfsetlinewidth{1.003750pt}%
\definecolor{currentstroke}{rgb}{0.121569,0.466667,0.705882}%
\pgfsetstrokecolor{currentstroke}%
\pgfsetstrokeopacity{0.300000}%
\pgfsetdash{}{0pt}%
\pgfpathmoveto{\pgfqpoint{1.558889in}{2.036882in}}%
\pgfpathcurveto{\pgfqpoint{1.567126in}{2.036882in}}{\pgfqpoint{1.575026in}{2.040154in}}{\pgfqpoint{1.580850in}{2.045978in}}%
\pgfpathcurveto{\pgfqpoint{1.586674in}{2.051802in}}{\pgfqpoint{1.589946in}{2.059702in}}{\pgfqpoint{1.589946in}{2.067938in}}%
\pgfpathcurveto{\pgfqpoint{1.589946in}{2.076175in}}{\pgfqpoint{1.586674in}{2.084075in}}{\pgfqpoint{1.580850in}{2.089899in}}%
\pgfpathcurveto{\pgfqpoint{1.575026in}{2.095722in}}{\pgfqpoint{1.567126in}{2.098995in}}{\pgfqpoint{1.558889in}{2.098995in}}%
\pgfpathcurveto{\pgfqpoint{1.550653in}{2.098995in}}{\pgfqpoint{1.542753in}{2.095722in}}{\pgfqpoint{1.536929in}{2.089899in}}%
\pgfpathcurveto{\pgfqpoint{1.531105in}{2.084075in}}{\pgfqpoint{1.527833in}{2.076175in}}{\pgfqpoint{1.527833in}{2.067938in}}%
\pgfpathcurveto{\pgfqpoint{1.527833in}{2.059702in}}{\pgfqpoint{1.531105in}{2.051802in}}{\pgfqpoint{1.536929in}{2.045978in}}%
\pgfpathcurveto{\pgfqpoint{1.542753in}{2.040154in}}{\pgfqpoint{1.550653in}{2.036882in}}{\pgfqpoint{1.558889in}{2.036882in}}%
\pgfpathclose%
\pgfusepath{stroke,fill}%
\end{pgfscope}%
\begin{pgfscope}%
\pgfpathrectangle{\pgfqpoint{0.100000in}{0.212622in}}{\pgfqpoint{3.696000in}{3.696000in}}%
\pgfusepath{clip}%
\pgfsetbuttcap%
\pgfsetroundjoin%
\definecolor{currentfill}{rgb}{0.121569,0.466667,0.705882}%
\pgfsetfillcolor{currentfill}%
\pgfsetfillopacity{0.300000}%
\pgfsetlinewidth{1.003750pt}%
\definecolor{currentstroke}{rgb}{0.121569,0.466667,0.705882}%
\pgfsetstrokecolor{currentstroke}%
\pgfsetstrokeopacity{0.300000}%
\pgfsetdash{}{0pt}%
\pgfpathmoveto{\pgfqpoint{1.558889in}{2.036882in}}%
\pgfpathcurveto{\pgfqpoint{1.567126in}{2.036882in}}{\pgfqpoint{1.575026in}{2.040154in}}{\pgfqpoint{1.580850in}{2.045978in}}%
\pgfpathcurveto{\pgfqpoint{1.586674in}{2.051802in}}{\pgfqpoint{1.589946in}{2.059702in}}{\pgfqpoint{1.589946in}{2.067938in}}%
\pgfpathcurveto{\pgfqpoint{1.589946in}{2.076175in}}{\pgfqpoint{1.586674in}{2.084075in}}{\pgfqpoint{1.580850in}{2.089899in}}%
\pgfpathcurveto{\pgfqpoint{1.575026in}{2.095722in}}{\pgfqpoint{1.567126in}{2.098995in}}{\pgfqpoint{1.558889in}{2.098995in}}%
\pgfpathcurveto{\pgfqpoint{1.550653in}{2.098995in}}{\pgfqpoint{1.542753in}{2.095722in}}{\pgfqpoint{1.536929in}{2.089899in}}%
\pgfpathcurveto{\pgfqpoint{1.531105in}{2.084075in}}{\pgfqpoint{1.527833in}{2.076175in}}{\pgfqpoint{1.527833in}{2.067938in}}%
\pgfpathcurveto{\pgfqpoint{1.527833in}{2.059702in}}{\pgfqpoint{1.531105in}{2.051802in}}{\pgfqpoint{1.536929in}{2.045978in}}%
\pgfpathcurveto{\pgfqpoint{1.542753in}{2.040154in}}{\pgfqpoint{1.550653in}{2.036882in}}{\pgfqpoint{1.558889in}{2.036882in}}%
\pgfpathclose%
\pgfusepath{stroke,fill}%
\end{pgfscope}%
\begin{pgfscope}%
\pgfpathrectangle{\pgfqpoint{0.100000in}{0.212622in}}{\pgfqpoint{3.696000in}{3.696000in}}%
\pgfusepath{clip}%
\pgfsetbuttcap%
\pgfsetroundjoin%
\definecolor{currentfill}{rgb}{0.121569,0.466667,0.705882}%
\pgfsetfillcolor{currentfill}%
\pgfsetfillopacity{0.300000}%
\pgfsetlinewidth{1.003750pt}%
\definecolor{currentstroke}{rgb}{0.121569,0.466667,0.705882}%
\pgfsetstrokecolor{currentstroke}%
\pgfsetstrokeopacity{0.300000}%
\pgfsetdash{}{0pt}%
\pgfpathmoveto{\pgfqpoint{1.558889in}{2.036882in}}%
\pgfpathcurveto{\pgfqpoint{1.567126in}{2.036882in}}{\pgfqpoint{1.575026in}{2.040154in}}{\pgfqpoint{1.580850in}{2.045978in}}%
\pgfpathcurveto{\pgfqpoint{1.586674in}{2.051802in}}{\pgfqpoint{1.589946in}{2.059702in}}{\pgfqpoint{1.589946in}{2.067938in}}%
\pgfpathcurveto{\pgfqpoint{1.589946in}{2.076175in}}{\pgfqpoint{1.586674in}{2.084075in}}{\pgfqpoint{1.580850in}{2.089899in}}%
\pgfpathcurveto{\pgfqpoint{1.575026in}{2.095722in}}{\pgfqpoint{1.567126in}{2.098995in}}{\pgfqpoint{1.558889in}{2.098995in}}%
\pgfpathcurveto{\pgfqpoint{1.550653in}{2.098995in}}{\pgfqpoint{1.542753in}{2.095722in}}{\pgfqpoint{1.536929in}{2.089899in}}%
\pgfpathcurveto{\pgfqpoint{1.531105in}{2.084075in}}{\pgfqpoint{1.527833in}{2.076175in}}{\pgfqpoint{1.527833in}{2.067938in}}%
\pgfpathcurveto{\pgfqpoint{1.527833in}{2.059702in}}{\pgfqpoint{1.531105in}{2.051802in}}{\pgfqpoint{1.536929in}{2.045978in}}%
\pgfpathcurveto{\pgfqpoint{1.542753in}{2.040154in}}{\pgfqpoint{1.550653in}{2.036882in}}{\pgfqpoint{1.558889in}{2.036882in}}%
\pgfpathclose%
\pgfusepath{stroke,fill}%
\end{pgfscope}%
\begin{pgfscope}%
\pgfpathrectangle{\pgfqpoint{0.100000in}{0.212622in}}{\pgfqpoint{3.696000in}{3.696000in}}%
\pgfusepath{clip}%
\pgfsetbuttcap%
\pgfsetroundjoin%
\definecolor{currentfill}{rgb}{0.121569,0.466667,0.705882}%
\pgfsetfillcolor{currentfill}%
\pgfsetfillopacity{0.300000}%
\pgfsetlinewidth{1.003750pt}%
\definecolor{currentstroke}{rgb}{0.121569,0.466667,0.705882}%
\pgfsetstrokecolor{currentstroke}%
\pgfsetstrokeopacity{0.300000}%
\pgfsetdash{}{0pt}%
\pgfpathmoveto{\pgfqpoint{1.558889in}{2.036882in}}%
\pgfpathcurveto{\pgfqpoint{1.567126in}{2.036882in}}{\pgfqpoint{1.575026in}{2.040154in}}{\pgfqpoint{1.580850in}{2.045978in}}%
\pgfpathcurveto{\pgfqpoint{1.586674in}{2.051802in}}{\pgfqpoint{1.589946in}{2.059702in}}{\pgfqpoint{1.589946in}{2.067938in}}%
\pgfpathcurveto{\pgfqpoint{1.589946in}{2.076175in}}{\pgfqpoint{1.586674in}{2.084075in}}{\pgfqpoint{1.580850in}{2.089899in}}%
\pgfpathcurveto{\pgfqpoint{1.575026in}{2.095722in}}{\pgfqpoint{1.567126in}{2.098995in}}{\pgfqpoint{1.558889in}{2.098995in}}%
\pgfpathcurveto{\pgfqpoint{1.550653in}{2.098995in}}{\pgfqpoint{1.542753in}{2.095722in}}{\pgfqpoint{1.536929in}{2.089899in}}%
\pgfpathcurveto{\pgfqpoint{1.531105in}{2.084075in}}{\pgfqpoint{1.527833in}{2.076175in}}{\pgfqpoint{1.527833in}{2.067938in}}%
\pgfpathcurveto{\pgfqpoint{1.527833in}{2.059702in}}{\pgfqpoint{1.531105in}{2.051802in}}{\pgfqpoint{1.536929in}{2.045978in}}%
\pgfpathcurveto{\pgfqpoint{1.542753in}{2.040154in}}{\pgfqpoint{1.550653in}{2.036882in}}{\pgfqpoint{1.558889in}{2.036882in}}%
\pgfpathclose%
\pgfusepath{stroke,fill}%
\end{pgfscope}%
\begin{pgfscope}%
\pgfpathrectangle{\pgfqpoint{0.100000in}{0.212622in}}{\pgfqpoint{3.696000in}{3.696000in}}%
\pgfusepath{clip}%
\pgfsetbuttcap%
\pgfsetroundjoin%
\definecolor{currentfill}{rgb}{0.121569,0.466667,0.705882}%
\pgfsetfillcolor{currentfill}%
\pgfsetfillopacity{0.300000}%
\pgfsetlinewidth{1.003750pt}%
\definecolor{currentstroke}{rgb}{0.121569,0.466667,0.705882}%
\pgfsetstrokecolor{currentstroke}%
\pgfsetstrokeopacity{0.300000}%
\pgfsetdash{}{0pt}%
\pgfpathmoveto{\pgfqpoint{1.558889in}{2.036882in}}%
\pgfpathcurveto{\pgfqpoint{1.567126in}{2.036882in}}{\pgfqpoint{1.575026in}{2.040154in}}{\pgfqpoint{1.580850in}{2.045978in}}%
\pgfpathcurveto{\pgfqpoint{1.586674in}{2.051802in}}{\pgfqpoint{1.589946in}{2.059702in}}{\pgfqpoint{1.589946in}{2.067938in}}%
\pgfpathcurveto{\pgfqpoint{1.589946in}{2.076175in}}{\pgfqpoint{1.586674in}{2.084075in}}{\pgfqpoint{1.580850in}{2.089899in}}%
\pgfpathcurveto{\pgfqpoint{1.575026in}{2.095722in}}{\pgfqpoint{1.567126in}{2.098995in}}{\pgfqpoint{1.558889in}{2.098995in}}%
\pgfpathcurveto{\pgfqpoint{1.550653in}{2.098995in}}{\pgfqpoint{1.542753in}{2.095722in}}{\pgfqpoint{1.536929in}{2.089899in}}%
\pgfpathcurveto{\pgfqpoint{1.531105in}{2.084075in}}{\pgfqpoint{1.527833in}{2.076175in}}{\pgfqpoint{1.527833in}{2.067938in}}%
\pgfpathcurveto{\pgfqpoint{1.527833in}{2.059702in}}{\pgfqpoint{1.531105in}{2.051802in}}{\pgfqpoint{1.536929in}{2.045978in}}%
\pgfpathcurveto{\pgfqpoint{1.542753in}{2.040154in}}{\pgfqpoint{1.550653in}{2.036882in}}{\pgfqpoint{1.558889in}{2.036882in}}%
\pgfpathclose%
\pgfusepath{stroke,fill}%
\end{pgfscope}%
\begin{pgfscope}%
\pgfpathrectangle{\pgfqpoint{0.100000in}{0.212622in}}{\pgfqpoint{3.696000in}{3.696000in}}%
\pgfusepath{clip}%
\pgfsetbuttcap%
\pgfsetroundjoin%
\definecolor{currentfill}{rgb}{0.121569,0.466667,0.705882}%
\pgfsetfillcolor{currentfill}%
\pgfsetfillopacity{0.300000}%
\pgfsetlinewidth{1.003750pt}%
\definecolor{currentstroke}{rgb}{0.121569,0.466667,0.705882}%
\pgfsetstrokecolor{currentstroke}%
\pgfsetstrokeopacity{0.300000}%
\pgfsetdash{}{0pt}%
\pgfpathmoveto{\pgfqpoint{1.558889in}{2.036882in}}%
\pgfpathcurveto{\pgfqpoint{1.567126in}{2.036882in}}{\pgfqpoint{1.575026in}{2.040154in}}{\pgfqpoint{1.580850in}{2.045978in}}%
\pgfpathcurveto{\pgfqpoint{1.586674in}{2.051802in}}{\pgfqpoint{1.589946in}{2.059702in}}{\pgfqpoint{1.589946in}{2.067938in}}%
\pgfpathcurveto{\pgfqpoint{1.589946in}{2.076175in}}{\pgfqpoint{1.586674in}{2.084075in}}{\pgfqpoint{1.580850in}{2.089899in}}%
\pgfpathcurveto{\pgfqpoint{1.575026in}{2.095722in}}{\pgfqpoint{1.567126in}{2.098995in}}{\pgfqpoint{1.558889in}{2.098995in}}%
\pgfpathcurveto{\pgfqpoint{1.550653in}{2.098995in}}{\pgfqpoint{1.542753in}{2.095722in}}{\pgfqpoint{1.536929in}{2.089899in}}%
\pgfpathcurveto{\pgfqpoint{1.531105in}{2.084075in}}{\pgfqpoint{1.527833in}{2.076175in}}{\pgfqpoint{1.527833in}{2.067938in}}%
\pgfpathcurveto{\pgfqpoint{1.527833in}{2.059702in}}{\pgfqpoint{1.531105in}{2.051802in}}{\pgfqpoint{1.536929in}{2.045978in}}%
\pgfpathcurveto{\pgfqpoint{1.542753in}{2.040154in}}{\pgfqpoint{1.550653in}{2.036882in}}{\pgfqpoint{1.558889in}{2.036882in}}%
\pgfpathclose%
\pgfusepath{stroke,fill}%
\end{pgfscope}%
\begin{pgfscope}%
\pgfpathrectangle{\pgfqpoint{0.100000in}{0.212622in}}{\pgfqpoint{3.696000in}{3.696000in}}%
\pgfusepath{clip}%
\pgfsetbuttcap%
\pgfsetroundjoin%
\definecolor{currentfill}{rgb}{0.121569,0.466667,0.705882}%
\pgfsetfillcolor{currentfill}%
\pgfsetfillopacity{0.300000}%
\pgfsetlinewidth{1.003750pt}%
\definecolor{currentstroke}{rgb}{0.121569,0.466667,0.705882}%
\pgfsetstrokecolor{currentstroke}%
\pgfsetstrokeopacity{0.300000}%
\pgfsetdash{}{0pt}%
\pgfpathmoveto{\pgfqpoint{1.558889in}{2.036882in}}%
\pgfpathcurveto{\pgfqpoint{1.567126in}{2.036882in}}{\pgfqpoint{1.575026in}{2.040154in}}{\pgfqpoint{1.580850in}{2.045978in}}%
\pgfpathcurveto{\pgfqpoint{1.586674in}{2.051802in}}{\pgfqpoint{1.589946in}{2.059702in}}{\pgfqpoint{1.589946in}{2.067938in}}%
\pgfpathcurveto{\pgfqpoint{1.589946in}{2.076175in}}{\pgfqpoint{1.586674in}{2.084075in}}{\pgfqpoint{1.580850in}{2.089899in}}%
\pgfpathcurveto{\pgfqpoint{1.575026in}{2.095722in}}{\pgfqpoint{1.567126in}{2.098995in}}{\pgfqpoint{1.558889in}{2.098995in}}%
\pgfpathcurveto{\pgfqpoint{1.550653in}{2.098995in}}{\pgfqpoint{1.542753in}{2.095722in}}{\pgfqpoint{1.536929in}{2.089899in}}%
\pgfpathcurveto{\pgfqpoint{1.531105in}{2.084075in}}{\pgfqpoint{1.527833in}{2.076175in}}{\pgfqpoint{1.527833in}{2.067938in}}%
\pgfpathcurveto{\pgfqpoint{1.527833in}{2.059702in}}{\pgfqpoint{1.531105in}{2.051802in}}{\pgfqpoint{1.536929in}{2.045978in}}%
\pgfpathcurveto{\pgfqpoint{1.542753in}{2.040154in}}{\pgfqpoint{1.550653in}{2.036882in}}{\pgfqpoint{1.558889in}{2.036882in}}%
\pgfpathclose%
\pgfusepath{stroke,fill}%
\end{pgfscope}%
\begin{pgfscope}%
\pgfpathrectangle{\pgfqpoint{0.100000in}{0.212622in}}{\pgfqpoint{3.696000in}{3.696000in}}%
\pgfusepath{clip}%
\pgfsetbuttcap%
\pgfsetroundjoin%
\definecolor{currentfill}{rgb}{0.121569,0.466667,0.705882}%
\pgfsetfillcolor{currentfill}%
\pgfsetfillopacity{0.300000}%
\pgfsetlinewidth{1.003750pt}%
\definecolor{currentstroke}{rgb}{0.121569,0.466667,0.705882}%
\pgfsetstrokecolor{currentstroke}%
\pgfsetstrokeopacity{0.300000}%
\pgfsetdash{}{0pt}%
\pgfpathmoveto{\pgfqpoint{1.558889in}{2.036882in}}%
\pgfpathcurveto{\pgfqpoint{1.567126in}{2.036882in}}{\pgfqpoint{1.575026in}{2.040154in}}{\pgfqpoint{1.580850in}{2.045978in}}%
\pgfpathcurveto{\pgfqpoint{1.586674in}{2.051802in}}{\pgfqpoint{1.589946in}{2.059702in}}{\pgfqpoint{1.589946in}{2.067938in}}%
\pgfpathcurveto{\pgfqpoint{1.589946in}{2.076175in}}{\pgfqpoint{1.586674in}{2.084075in}}{\pgfqpoint{1.580850in}{2.089899in}}%
\pgfpathcurveto{\pgfqpoint{1.575026in}{2.095722in}}{\pgfqpoint{1.567126in}{2.098995in}}{\pgfqpoint{1.558889in}{2.098995in}}%
\pgfpathcurveto{\pgfqpoint{1.550653in}{2.098995in}}{\pgfqpoint{1.542753in}{2.095722in}}{\pgfqpoint{1.536929in}{2.089899in}}%
\pgfpathcurveto{\pgfqpoint{1.531105in}{2.084075in}}{\pgfqpoint{1.527833in}{2.076175in}}{\pgfqpoint{1.527833in}{2.067938in}}%
\pgfpathcurveto{\pgfqpoint{1.527833in}{2.059702in}}{\pgfqpoint{1.531105in}{2.051802in}}{\pgfqpoint{1.536929in}{2.045978in}}%
\pgfpathcurveto{\pgfqpoint{1.542753in}{2.040154in}}{\pgfqpoint{1.550653in}{2.036882in}}{\pgfqpoint{1.558889in}{2.036882in}}%
\pgfpathclose%
\pgfusepath{stroke,fill}%
\end{pgfscope}%
\begin{pgfscope}%
\pgfpathrectangle{\pgfqpoint{0.100000in}{0.212622in}}{\pgfqpoint{3.696000in}{3.696000in}}%
\pgfusepath{clip}%
\pgfsetbuttcap%
\pgfsetroundjoin%
\definecolor{currentfill}{rgb}{0.121569,0.466667,0.705882}%
\pgfsetfillcolor{currentfill}%
\pgfsetfillopacity{0.300000}%
\pgfsetlinewidth{1.003750pt}%
\definecolor{currentstroke}{rgb}{0.121569,0.466667,0.705882}%
\pgfsetstrokecolor{currentstroke}%
\pgfsetstrokeopacity{0.300000}%
\pgfsetdash{}{0pt}%
\pgfpathmoveto{\pgfqpoint{1.558889in}{2.036882in}}%
\pgfpathcurveto{\pgfqpoint{1.567126in}{2.036882in}}{\pgfqpoint{1.575026in}{2.040154in}}{\pgfqpoint{1.580850in}{2.045978in}}%
\pgfpathcurveto{\pgfqpoint{1.586674in}{2.051802in}}{\pgfqpoint{1.589946in}{2.059702in}}{\pgfqpoint{1.589946in}{2.067938in}}%
\pgfpathcurveto{\pgfqpoint{1.589946in}{2.076175in}}{\pgfqpoint{1.586674in}{2.084075in}}{\pgfqpoint{1.580850in}{2.089899in}}%
\pgfpathcurveto{\pgfqpoint{1.575026in}{2.095722in}}{\pgfqpoint{1.567126in}{2.098995in}}{\pgfqpoint{1.558889in}{2.098995in}}%
\pgfpathcurveto{\pgfqpoint{1.550653in}{2.098995in}}{\pgfqpoint{1.542753in}{2.095722in}}{\pgfqpoint{1.536929in}{2.089899in}}%
\pgfpathcurveto{\pgfqpoint{1.531105in}{2.084075in}}{\pgfqpoint{1.527833in}{2.076175in}}{\pgfqpoint{1.527833in}{2.067938in}}%
\pgfpathcurveto{\pgfqpoint{1.527833in}{2.059702in}}{\pgfqpoint{1.531105in}{2.051802in}}{\pgfqpoint{1.536929in}{2.045978in}}%
\pgfpathcurveto{\pgfqpoint{1.542753in}{2.040154in}}{\pgfqpoint{1.550653in}{2.036882in}}{\pgfqpoint{1.558889in}{2.036882in}}%
\pgfpathclose%
\pgfusepath{stroke,fill}%
\end{pgfscope}%
\begin{pgfscope}%
\pgfpathrectangle{\pgfqpoint{0.100000in}{0.212622in}}{\pgfqpoint{3.696000in}{3.696000in}}%
\pgfusepath{clip}%
\pgfsetbuttcap%
\pgfsetroundjoin%
\definecolor{currentfill}{rgb}{0.121569,0.466667,0.705882}%
\pgfsetfillcolor{currentfill}%
\pgfsetfillopacity{0.300000}%
\pgfsetlinewidth{1.003750pt}%
\definecolor{currentstroke}{rgb}{0.121569,0.466667,0.705882}%
\pgfsetstrokecolor{currentstroke}%
\pgfsetstrokeopacity{0.300000}%
\pgfsetdash{}{0pt}%
\pgfpathmoveto{\pgfqpoint{1.558889in}{2.036882in}}%
\pgfpathcurveto{\pgfqpoint{1.567126in}{2.036882in}}{\pgfqpoint{1.575026in}{2.040154in}}{\pgfqpoint{1.580850in}{2.045978in}}%
\pgfpathcurveto{\pgfqpoint{1.586674in}{2.051802in}}{\pgfqpoint{1.589946in}{2.059702in}}{\pgfqpoint{1.589946in}{2.067938in}}%
\pgfpathcurveto{\pgfqpoint{1.589946in}{2.076175in}}{\pgfqpoint{1.586674in}{2.084075in}}{\pgfqpoint{1.580850in}{2.089899in}}%
\pgfpathcurveto{\pgfqpoint{1.575026in}{2.095722in}}{\pgfqpoint{1.567126in}{2.098995in}}{\pgfqpoint{1.558889in}{2.098995in}}%
\pgfpathcurveto{\pgfqpoint{1.550653in}{2.098995in}}{\pgfqpoint{1.542753in}{2.095722in}}{\pgfqpoint{1.536929in}{2.089899in}}%
\pgfpathcurveto{\pgfqpoint{1.531105in}{2.084075in}}{\pgfqpoint{1.527833in}{2.076175in}}{\pgfqpoint{1.527833in}{2.067938in}}%
\pgfpathcurveto{\pgfqpoint{1.527833in}{2.059702in}}{\pgfqpoint{1.531105in}{2.051802in}}{\pgfqpoint{1.536929in}{2.045978in}}%
\pgfpathcurveto{\pgfqpoint{1.542753in}{2.040154in}}{\pgfqpoint{1.550653in}{2.036882in}}{\pgfqpoint{1.558889in}{2.036882in}}%
\pgfpathclose%
\pgfusepath{stroke,fill}%
\end{pgfscope}%
\begin{pgfscope}%
\pgfpathrectangle{\pgfqpoint{0.100000in}{0.212622in}}{\pgfqpoint{3.696000in}{3.696000in}}%
\pgfusepath{clip}%
\pgfsetbuttcap%
\pgfsetroundjoin%
\definecolor{currentfill}{rgb}{0.121569,0.466667,0.705882}%
\pgfsetfillcolor{currentfill}%
\pgfsetfillopacity{0.300000}%
\pgfsetlinewidth{1.003750pt}%
\definecolor{currentstroke}{rgb}{0.121569,0.466667,0.705882}%
\pgfsetstrokecolor{currentstroke}%
\pgfsetstrokeopacity{0.300000}%
\pgfsetdash{}{0pt}%
\pgfpathmoveto{\pgfqpoint{1.558889in}{2.036882in}}%
\pgfpathcurveto{\pgfqpoint{1.567126in}{2.036882in}}{\pgfqpoint{1.575026in}{2.040154in}}{\pgfqpoint{1.580850in}{2.045978in}}%
\pgfpathcurveto{\pgfqpoint{1.586674in}{2.051802in}}{\pgfqpoint{1.589946in}{2.059702in}}{\pgfqpoint{1.589946in}{2.067938in}}%
\pgfpathcurveto{\pgfqpoint{1.589946in}{2.076175in}}{\pgfqpoint{1.586674in}{2.084075in}}{\pgfqpoint{1.580850in}{2.089899in}}%
\pgfpathcurveto{\pgfqpoint{1.575026in}{2.095722in}}{\pgfqpoint{1.567126in}{2.098995in}}{\pgfqpoint{1.558889in}{2.098995in}}%
\pgfpathcurveto{\pgfqpoint{1.550653in}{2.098995in}}{\pgfqpoint{1.542753in}{2.095722in}}{\pgfqpoint{1.536929in}{2.089899in}}%
\pgfpathcurveto{\pgfqpoint{1.531105in}{2.084075in}}{\pgfqpoint{1.527833in}{2.076175in}}{\pgfqpoint{1.527833in}{2.067938in}}%
\pgfpathcurveto{\pgfqpoint{1.527833in}{2.059702in}}{\pgfqpoint{1.531105in}{2.051802in}}{\pgfqpoint{1.536929in}{2.045978in}}%
\pgfpathcurveto{\pgfqpoint{1.542753in}{2.040154in}}{\pgfqpoint{1.550653in}{2.036882in}}{\pgfqpoint{1.558889in}{2.036882in}}%
\pgfpathclose%
\pgfusepath{stroke,fill}%
\end{pgfscope}%
\begin{pgfscope}%
\pgfpathrectangle{\pgfqpoint{0.100000in}{0.212622in}}{\pgfqpoint{3.696000in}{3.696000in}}%
\pgfusepath{clip}%
\pgfsetbuttcap%
\pgfsetroundjoin%
\definecolor{currentfill}{rgb}{0.121569,0.466667,0.705882}%
\pgfsetfillcolor{currentfill}%
\pgfsetfillopacity{0.300000}%
\pgfsetlinewidth{1.003750pt}%
\definecolor{currentstroke}{rgb}{0.121569,0.466667,0.705882}%
\pgfsetstrokecolor{currentstroke}%
\pgfsetstrokeopacity{0.300000}%
\pgfsetdash{}{0pt}%
\pgfpathmoveto{\pgfqpoint{1.558889in}{2.036882in}}%
\pgfpathcurveto{\pgfqpoint{1.567126in}{2.036882in}}{\pgfqpoint{1.575026in}{2.040154in}}{\pgfqpoint{1.580850in}{2.045978in}}%
\pgfpathcurveto{\pgfqpoint{1.586674in}{2.051802in}}{\pgfqpoint{1.589946in}{2.059702in}}{\pgfqpoint{1.589946in}{2.067938in}}%
\pgfpathcurveto{\pgfqpoint{1.589946in}{2.076175in}}{\pgfqpoint{1.586674in}{2.084075in}}{\pgfqpoint{1.580850in}{2.089899in}}%
\pgfpathcurveto{\pgfqpoint{1.575026in}{2.095722in}}{\pgfqpoint{1.567126in}{2.098995in}}{\pgfqpoint{1.558889in}{2.098995in}}%
\pgfpathcurveto{\pgfqpoint{1.550653in}{2.098995in}}{\pgfqpoint{1.542753in}{2.095722in}}{\pgfqpoint{1.536929in}{2.089899in}}%
\pgfpathcurveto{\pgfqpoint{1.531105in}{2.084075in}}{\pgfqpoint{1.527833in}{2.076175in}}{\pgfqpoint{1.527833in}{2.067938in}}%
\pgfpathcurveto{\pgfqpoint{1.527833in}{2.059702in}}{\pgfqpoint{1.531105in}{2.051802in}}{\pgfqpoint{1.536929in}{2.045978in}}%
\pgfpathcurveto{\pgfqpoint{1.542753in}{2.040154in}}{\pgfqpoint{1.550653in}{2.036882in}}{\pgfqpoint{1.558889in}{2.036882in}}%
\pgfpathclose%
\pgfusepath{stroke,fill}%
\end{pgfscope}%
\begin{pgfscope}%
\pgfpathrectangle{\pgfqpoint{0.100000in}{0.212622in}}{\pgfqpoint{3.696000in}{3.696000in}}%
\pgfusepath{clip}%
\pgfsetbuttcap%
\pgfsetroundjoin%
\definecolor{currentfill}{rgb}{0.121569,0.466667,0.705882}%
\pgfsetfillcolor{currentfill}%
\pgfsetfillopacity{0.300000}%
\pgfsetlinewidth{1.003750pt}%
\definecolor{currentstroke}{rgb}{0.121569,0.466667,0.705882}%
\pgfsetstrokecolor{currentstroke}%
\pgfsetstrokeopacity{0.300000}%
\pgfsetdash{}{0pt}%
\pgfpathmoveto{\pgfqpoint{1.558889in}{2.036882in}}%
\pgfpathcurveto{\pgfqpoint{1.567126in}{2.036882in}}{\pgfqpoint{1.575026in}{2.040154in}}{\pgfqpoint{1.580850in}{2.045978in}}%
\pgfpathcurveto{\pgfqpoint{1.586674in}{2.051802in}}{\pgfqpoint{1.589946in}{2.059702in}}{\pgfqpoint{1.589946in}{2.067938in}}%
\pgfpathcurveto{\pgfqpoint{1.589946in}{2.076175in}}{\pgfqpoint{1.586674in}{2.084075in}}{\pgfqpoint{1.580850in}{2.089899in}}%
\pgfpathcurveto{\pgfqpoint{1.575026in}{2.095722in}}{\pgfqpoint{1.567126in}{2.098995in}}{\pgfqpoint{1.558889in}{2.098995in}}%
\pgfpathcurveto{\pgfqpoint{1.550653in}{2.098995in}}{\pgfqpoint{1.542753in}{2.095722in}}{\pgfqpoint{1.536929in}{2.089899in}}%
\pgfpathcurveto{\pgfqpoint{1.531105in}{2.084075in}}{\pgfqpoint{1.527833in}{2.076175in}}{\pgfqpoint{1.527833in}{2.067938in}}%
\pgfpathcurveto{\pgfqpoint{1.527833in}{2.059702in}}{\pgfqpoint{1.531105in}{2.051802in}}{\pgfqpoint{1.536929in}{2.045978in}}%
\pgfpathcurveto{\pgfqpoint{1.542753in}{2.040154in}}{\pgfqpoint{1.550653in}{2.036882in}}{\pgfqpoint{1.558889in}{2.036882in}}%
\pgfpathclose%
\pgfusepath{stroke,fill}%
\end{pgfscope}%
\begin{pgfscope}%
\pgfpathrectangle{\pgfqpoint{0.100000in}{0.212622in}}{\pgfqpoint{3.696000in}{3.696000in}}%
\pgfusepath{clip}%
\pgfsetbuttcap%
\pgfsetroundjoin%
\definecolor{currentfill}{rgb}{0.121569,0.466667,0.705882}%
\pgfsetfillcolor{currentfill}%
\pgfsetfillopacity{0.300000}%
\pgfsetlinewidth{1.003750pt}%
\definecolor{currentstroke}{rgb}{0.121569,0.466667,0.705882}%
\pgfsetstrokecolor{currentstroke}%
\pgfsetstrokeopacity{0.300000}%
\pgfsetdash{}{0pt}%
\pgfpathmoveto{\pgfqpoint{1.558889in}{2.036882in}}%
\pgfpathcurveto{\pgfqpoint{1.567126in}{2.036882in}}{\pgfqpoint{1.575026in}{2.040154in}}{\pgfqpoint{1.580850in}{2.045978in}}%
\pgfpathcurveto{\pgfqpoint{1.586674in}{2.051802in}}{\pgfqpoint{1.589946in}{2.059702in}}{\pgfqpoint{1.589946in}{2.067938in}}%
\pgfpathcurveto{\pgfqpoint{1.589946in}{2.076175in}}{\pgfqpoint{1.586674in}{2.084075in}}{\pgfqpoint{1.580850in}{2.089899in}}%
\pgfpathcurveto{\pgfqpoint{1.575026in}{2.095722in}}{\pgfqpoint{1.567126in}{2.098995in}}{\pgfqpoint{1.558889in}{2.098995in}}%
\pgfpathcurveto{\pgfqpoint{1.550653in}{2.098995in}}{\pgfqpoint{1.542753in}{2.095722in}}{\pgfqpoint{1.536929in}{2.089899in}}%
\pgfpathcurveto{\pgfqpoint{1.531105in}{2.084075in}}{\pgfqpoint{1.527833in}{2.076175in}}{\pgfqpoint{1.527833in}{2.067938in}}%
\pgfpathcurveto{\pgfqpoint{1.527833in}{2.059702in}}{\pgfqpoint{1.531105in}{2.051802in}}{\pgfqpoint{1.536929in}{2.045978in}}%
\pgfpathcurveto{\pgfqpoint{1.542753in}{2.040154in}}{\pgfqpoint{1.550653in}{2.036882in}}{\pgfqpoint{1.558889in}{2.036882in}}%
\pgfpathclose%
\pgfusepath{stroke,fill}%
\end{pgfscope}%
\begin{pgfscope}%
\pgfpathrectangle{\pgfqpoint{0.100000in}{0.212622in}}{\pgfqpoint{3.696000in}{3.696000in}}%
\pgfusepath{clip}%
\pgfsetbuttcap%
\pgfsetroundjoin%
\definecolor{currentfill}{rgb}{0.121569,0.466667,0.705882}%
\pgfsetfillcolor{currentfill}%
\pgfsetfillopacity{0.300000}%
\pgfsetlinewidth{1.003750pt}%
\definecolor{currentstroke}{rgb}{0.121569,0.466667,0.705882}%
\pgfsetstrokecolor{currentstroke}%
\pgfsetstrokeopacity{0.300000}%
\pgfsetdash{}{0pt}%
\pgfpathmoveto{\pgfqpoint{1.558889in}{2.036882in}}%
\pgfpathcurveto{\pgfqpoint{1.567126in}{2.036882in}}{\pgfqpoint{1.575026in}{2.040154in}}{\pgfqpoint{1.580850in}{2.045978in}}%
\pgfpathcurveto{\pgfqpoint{1.586674in}{2.051802in}}{\pgfqpoint{1.589946in}{2.059702in}}{\pgfqpoint{1.589946in}{2.067938in}}%
\pgfpathcurveto{\pgfqpoint{1.589946in}{2.076175in}}{\pgfqpoint{1.586674in}{2.084075in}}{\pgfqpoint{1.580850in}{2.089899in}}%
\pgfpathcurveto{\pgfqpoint{1.575026in}{2.095722in}}{\pgfqpoint{1.567126in}{2.098995in}}{\pgfqpoint{1.558889in}{2.098995in}}%
\pgfpathcurveto{\pgfqpoint{1.550653in}{2.098995in}}{\pgfqpoint{1.542753in}{2.095722in}}{\pgfqpoint{1.536929in}{2.089899in}}%
\pgfpathcurveto{\pgfqpoint{1.531105in}{2.084075in}}{\pgfqpoint{1.527833in}{2.076175in}}{\pgfqpoint{1.527833in}{2.067938in}}%
\pgfpathcurveto{\pgfqpoint{1.527833in}{2.059702in}}{\pgfqpoint{1.531105in}{2.051802in}}{\pgfqpoint{1.536929in}{2.045978in}}%
\pgfpathcurveto{\pgfqpoint{1.542753in}{2.040154in}}{\pgfqpoint{1.550653in}{2.036882in}}{\pgfqpoint{1.558889in}{2.036882in}}%
\pgfpathclose%
\pgfusepath{stroke,fill}%
\end{pgfscope}%
\begin{pgfscope}%
\pgfpathrectangle{\pgfqpoint{0.100000in}{0.212622in}}{\pgfqpoint{3.696000in}{3.696000in}}%
\pgfusepath{clip}%
\pgfsetbuttcap%
\pgfsetroundjoin%
\definecolor{currentfill}{rgb}{0.121569,0.466667,0.705882}%
\pgfsetfillcolor{currentfill}%
\pgfsetfillopacity{0.300000}%
\pgfsetlinewidth{1.003750pt}%
\definecolor{currentstroke}{rgb}{0.121569,0.466667,0.705882}%
\pgfsetstrokecolor{currentstroke}%
\pgfsetstrokeopacity{0.300000}%
\pgfsetdash{}{0pt}%
\pgfpathmoveto{\pgfqpoint{1.558889in}{2.036882in}}%
\pgfpathcurveto{\pgfqpoint{1.567126in}{2.036882in}}{\pgfqpoint{1.575026in}{2.040154in}}{\pgfqpoint{1.580850in}{2.045978in}}%
\pgfpathcurveto{\pgfqpoint{1.586674in}{2.051802in}}{\pgfqpoint{1.589946in}{2.059702in}}{\pgfqpoint{1.589946in}{2.067938in}}%
\pgfpathcurveto{\pgfqpoint{1.589946in}{2.076175in}}{\pgfqpoint{1.586674in}{2.084075in}}{\pgfqpoint{1.580850in}{2.089899in}}%
\pgfpathcurveto{\pgfqpoint{1.575026in}{2.095722in}}{\pgfqpoint{1.567126in}{2.098995in}}{\pgfqpoint{1.558889in}{2.098995in}}%
\pgfpathcurveto{\pgfqpoint{1.550653in}{2.098995in}}{\pgfqpoint{1.542753in}{2.095722in}}{\pgfqpoint{1.536929in}{2.089899in}}%
\pgfpathcurveto{\pgfqpoint{1.531105in}{2.084075in}}{\pgfqpoint{1.527833in}{2.076175in}}{\pgfqpoint{1.527833in}{2.067938in}}%
\pgfpathcurveto{\pgfqpoint{1.527833in}{2.059702in}}{\pgfqpoint{1.531105in}{2.051802in}}{\pgfqpoint{1.536929in}{2.045978in}}%
\pgfpathcurveto{\pgfqpoint{1.542753in}{2.040154in}}{\pgfqpoint{1.550653in}{2.036882in}}{\pgfqpoint{1.558889in}{2.036882in}}%
\pgfpathclose%
\pgfusepath{stroke,fill}%
\end{pgfscope}%
\begin{pgfscope}%
\pgfpathrectangle{\pgfqpoint{0.100000in}{0.212622in}}{\pgfqpoint{3.696000in}{3.696000in}}%
\pgfusepath{clip}%
\pgfsetbuttcap%
\pgfsetroundjoin%
\definecolor{currentfill}{rgb}{0.121569,0.466667,0.705882}%
\pgfsetfillcolor{currentfill}%
\pgfsetfillopacity{0.300000}%
\pgfsetlinewidth{1.003750pt}%
\definecolor{currentstroke}{rgb}{0.121569,0.466667,0.705882}%
\pgfsetstrokecolor{currentstroke}%
\pgfsetstrokeopacity{0.300000}%
\pgfsetdash{}{0pt}%
\pgfpathmoveto{\pgfqpoint{1.558889in}{2.036882in}}%
\pgfpathcurveto{\pgfqpoint{1.567126in}{2.036882in}}{\pgfqpoint{1.575026in}{2.040154in}}{\pgfqpoint{1.580850in}{2.045978in}}%
\pgfpathcurveto{\pgfqpoint{1.586674in}{2.051802in}}{\pgfqpoint{1.589946in}{2.059702in}}{\pgfqpoint{1.589946in}{2.067938in}}%
\pgfpathcurveto{\pgfqpoint{1.589946in}{2.076175in}}{\pgfqpoint{1.586674in}{2.084075in}}{\pgfqpoint{1.580850in}{2.089899in}}%
\pgfpathcurveto{\pgfqpoint{1.575026in}{2.095722in}}{\pgfqpoint{1.567126in}{2.098995in}}{\pgfqpoint{1.558889in}{2.098995in}}%
\pgfpathcurveto{\pgfqpoint{1.550653in}{2.098995in}}{\pgfqpoint{1.542753in}{2.095722in}}{\pgfqpoint{1.536929in}{2.089899in}}%
\pgfpathcurveto{\pgfqpoint{1.531105in}{2.084075in}}{\pgfqpoint{1.527833in}{2.076175in}}{\pgfqpoint{1.527833in}{2.067938in}}%
\pgfpathcurveto{\pgfqpoint{1.527833in}{2.059702in}}{\pgfqpoint{1.531105in}{2.051802in}}{\pgfqpoint{1.536929in}{2.045978in}}%
\pgfpathcurveto{\pgfqpoint{1.542753in}{2.040154in}}{\pgfqpoint{1.550653in}{2.036882in}}{\pgfqpoint{1.558889in}{2.036882in}}%
\pgfpathclose%
\pgfusepath{stroke,fill}%
\end{pgfscope}%
\begin{pgfscope}%
\pgfpathrectangle{\pgfqpoint{0.100000in}{0.212622in}}{\pgfqpoint{3.696000in}{3.696000in}}%
\pgfusepath{clip}%
\pgfsetbuttcap%
\pgfsetroundjoin%
\definecolor{currentfill}{rgb}{0.121569,0.466667,0.705882}%
\pgfsetfillcolor{currentfill}%
\pgfsetfillopacity{0.300000}%
\pgfsetlinewidth{1.003750pt}%
\definecolor{currentstroke}{rgb}{0.121569,0.466667,0.705882}%
\pgfsetstrokecolor{currentstroke}%
\pgfsetstrokeopacity{0.300000}%
\pgfsetdash{}{0pt}%
\pgfpathmoveto{\pgfqpoint{1.558889in}{2.036882in}}%
\pgfpathcurveto{\pgfqpoint{1.567126in}{2.036882in}}{\pgfqpoint{1.575026in}{2.040154in}}{\pgfqpoint{1.580850in}{2.045978in}}%
\pgfpathcurveto{\pgfqpoint{1.586674in}{2.051802in}}{\pgfqpoint{1.589946in}{2.059702in}}{\pgfqpoint{1.589946in}{2.067938in}}%
\pgfpathcurveto{\pgfqpoint{1.589946in}{2.076175in}}{\pgfqpoint{1.586674in}{2.084075in}}{\pgfqpoint{1.580850in}{2.089899in}}%
\pgfpathcurveto{\pgfqpoint{1.575026in}{2.095722in}}{\pgfqpoint{1.567126in}{2.098995in}}{\pgfqpoint{1.558889in}{2.098995in}}%
\pgfpathcurveto{\pgfqpoint{1.550653in}{2.098995in}}{\pgfqpoint{1.542753in}{2.095722in}}{\pgfqpoint{1.536929in}{2.089899in}}%
\pgfpathcurveto{\pgfqpoint{1.531105in}{2.084075in}}{\pgfqpoint{1.527833in}{2.076175in}}{\pgfqpoint{1.527833in}{2.067938in}}%
\pgfpathcurveto{\pgfqpoint{1.527833in}{2.059702in}}{\pgfqpoint{1.531105in}{2.051802in}}{\pgfqpoint{1.536929in}{2.045978in}}%
\pgfpathcurveto{\pgfqpoint{1.542753in}{2.040154in}}{\pgfqpoint{1.550653in}{2.036882in}}{\pgfqpoint{1.558889in}{2.036882in}}%
\pgfpathclose%
\pgfusepath{stroke,fill}%
\end{pgfscope}%
\begin{pgfscope}%
\pgfpathrectangle{\pgfqpoint{0.100000in}{0.212622in}}{\pgfqpoint{3.696000in}{3.696000in}}%
\pgfusepath{clip}%
\pgfsetbuttcap%
\pgfsetroundjoin%
\definecolor{currentfill}{rgb}{0.121569,0.466667,0.705882}%
\pgfsetfillcolor{currentfill}%
\pgfsetfillopacity{0.300000}%
\pgfsetlinewidth{1.003750pt}%
\definecolor{currentstroke}{rgb}{0.121569,0.466667,0.705882}%
\pgfsetstrokecolor{currentstroke}%
\pgfsetstrokeopacity{0.300000}%
\pgfsetdash{}{0pt}%
\pgfpathmoveto{\pgfqpoint{1.558889in}{2.036882in}}%
\pgfpathcurveto{\pgfqpoint{1.567126in}{2.036882in}}{\pgfqpoint{1.575026in}{2.040154in}}{\pgfqpoint{1.580850in}{2.045978in}}%
\pgfpathcurveto{\pgfqpoint{1.586674in}{2.051802in}}{\pgfqpoint{1.589946in}{2.059702in}}{\pgfqpoint{1.589946in}{2.067938in}}%
\pgfpathcurveto{\pgfqpoint{1.589946in}{2.076175in}}{\pgfqpoint{1.586674in}{2.084075in}}{\pgfqpoint{1.580850in}{2.089899in}}%
\pgfpathcurveto{\pgfqpoint{1.575026in}{2.095722in}}{\pgfqpoint{1.567126in}{2.098995in}}{\pgfqpoint{1.558889in}{2.098995in}}%
\pgfpathcurveto{\pgfqpoint{1.550653in}{2.098995in}}{\pgfqpoint{1.542753in}{2.095722in}}{\pgfqpoint{1.536929in}{2.089899in}}%
\pgfpathcurveto{\pgfqpoint{1.531105in}{2.084075in}}{\pgfqpoint{1.527833in}{2.076175in}}{\pgfqpoint{1.527833in}{2.067938in}}%
\pgfpathcurveto{\pgfqpoint{1.527833in}{2.059702in}}{\pgfqpoint{1.531105in}{2.051802in}}{\pgfqpoint{1.536929in}{2.045978in}}%
\pgfpathcurveto{\pgfqpoint{1.542753in}{2.040154in}}{\pgfqpoint{1.550653in}{2.036882in}}{\pgfqpoint{1.558889in}{2.036882in}}%
\pgfpathclose%
\pgfusepath{stroke,fill}%
\end{pgfscope}%
\begin{pgfscope}%
\pgfpathrectangle{\pgfqpoint{0.100000in}{0.212622in}}{\pgfqpoint{3.696000in}{3.696000in}}%
\pgfusepath{clip}%
\pgfsetbuttcap%
\pgfsetroundjoin%
\definecolor{currentfill}{rgb}{0.121569,0.466667,0.705882}%
\pgfsetfillcolor{currentfill}%
\pgfsetfillopacity{0.300000}%
\pgfsetlinewidth{1.003750pt}%
\definecolor{currentstroke}{rgb}{0.121569,0.466667,0.705882}%
\pgfsetstrokecolor{currentstroke}%
\pgfsetstrokeopacity{0.300000}%
\pgfsetdash{}{0pt}%
\pgfpathmoveto{\pgfqpoint{1.558889in}{2.036882in}}%
\pgfpathcurveto{\pgfqpoint{1.567126in}{2.036882in}}{\pgfqpoint{1.575026in}{2.040154in}}{\pgfqpoint{1.580850in}{2.045978in}}%
\pgfpathcurveto{\pgfqpoint{1.586674in}{2.051802in}}{\pgfqpoint{1.589946in}{2.059702in}}{\pgfqpoint{1.589946in}{2.067938in}}%
\pgfpathcurveto{\pgfqpoint{1.589946in}{2.076175in}}{\pgfqpoint{1.586674in}{2.084075in}}{\pgfqpoint{1.580850in}{2.089899in}}%
\pgfpathcurveto{\pgfqpoint{1.575026in}{2.095722in}}{\pgfqpoint{1.567126in}{2.098995in}}{\pgfqpoint{1.558889in}{2.098995in}}%
\pgfpathcurveto{\pgfqpoint{1.550653in}{2.098995in}}{\pgfqpoint{1.542753in}{2.095722in}}{\pgfqpoint{1.536929in}{2.089899in}}%
\pgfpathcurveto{\pgfqpoint{1.531105in}{2.084075in}}{\pgfqpoint{1.527833in}{2.076175in}}{\pgfqpoint{1.527833in}{2.067938in}}%
\pgfpathcurveto{\pgfqpoint{1.527833in}{2.059702in}}{\pgfqpoint{1.531105in}{2.051802in}}{\pgfqpoint{1.536929in}{2.045978in}}%
\pgfpathcurveto{\pgfqpoint{1.542753in}{2.040154in}}{\pgfqpoint{1.550653in}{2.036882in}}{\pgfqpoint{1.558889in}{2.036882in}}%
\pgfpathclose%
\pgfusepath{stroke,fill}%
\end{pgfscope}%
\begin{pgfscope}%
\pgfpathrectangle{\pgfqpoint{0.100000in}{0.212622in}}{\pgfqpoint{3.696000in}{3.696000in}}%
\pgfusepath{clip}%
\pgfsetbuttcap%
\pgfsetroundjoin%
\definecolor{currentfill}{rgb}{0.121569,0.466667,0.705882}%
\pgfsetfillcolor{currentfill}%
\pgfsetfillopacity{0.300000}%
\pgfsetlinewidth{1.003750pt}%
\definecolor{currentstroke}{rgb}{0.121569,0.466667,0.705882}%
\pgfsetstrokecolor{currentstroke}%
\pgfsetstrokeopacity{0.300000}%
\pgfsetdash{}{0pt}%
\pgfpathmoveto{\pgfqpoint{1.558889in}{2.036882in}}%
\pgfpathcurveto{\pgfqpoint{1.567126in}{2.036882in}}{\pgfqpoint{1.575026in}{2.040154in}}{\pgfqpoint{1.580850in}{2.045978in}}%
\pgfpathcurveto{\pgfqpoint{1.586674in}{2.051802in}}{\pgfqpoint{1.589946in}{2.059702in}}{\pgfqpoint{1.589946in}{2.067938in}}%
\pgfpathcurveto{\pgfqpoint{1.589946in}{2.076175in}}{\pgfqpoint{1.586674in}{2.084075in}}{\pgfqpoint{1.580850in}{2.089899in}}%
\pgfpathcurveto{\pgfqpoint{1.575026in}{2.095722in}}{\pgfqpoint{1.567126in}{2.098995in}}{\pgfqpoint{1.558889in}{2.098995in}}%
\pgfpathcurveto{\pgfqpoint{1.550653in}{2.098995in}}{\pgfqpoint{1.542753in}{2.095722in}}{\pgfqpoint{1.536929in}{2.089899in}}%
\pgfpathcurveto{\pgfqpoint{1.531105in}{2.084075in}}{\pgfqpoint{1.527833in}{2.076175in}}{\pgfqpoint{1.527833in}{2.067938in}}%
\pgfpathcurveto{\pgfqpoint{1.527833in}{2.059702in}}{\pgfqpoint{1.531105in}{2.051802in}}{\pgfqpoint{1.536929in}{2.045978in}}%
\pgfpathcurveto{\pgfqpoint{1.542753in}{2.040154in}}{\pgfqpoint{1.550653in}{2.036882in}}{\pgfqpoint{1.558889in}{2.036882in}}%
\pgfpathclose%
\pgfusepath{stroke,fill}%
\end{pgfscope}%
\begin{pgfscope}%
\pgfpathrectangle{\pgfqpoint{0.100000in}{0.212622in}}{\pgfqpoint{3.696000in}{3.696000in}}%
\pgfusepath{clip}%
\pgfsetbuttcap%
\pgfsetroundjoin%
\definecolor{currentfill}{rgb}{0.121569,0.466667,0.705882}%
\pgfsetfillcolor{currentfill}%
\pgfsetfillopacity{0.300000}%
\pgfsetlinewidth{1.003750pt}%
\definecolor{currentstroke}{rgb}{0.121569,0.466667,0.705882}%
\pgfsetstrokecolor{currentstroke}%
\pgfsetstrokeopacity{0.300000}%
\pgfsetdash{}{0pt}%
\pgfpathmoveto{\pgfqpoint{1.558889in}{2.036882in}}%
\pgfpathcurveto{\pgfqpoint{1.567126in}{2.036882in}}{\pgfqpoint{1.575026in}{2.040154in}}{\pgfqpoint{1.580850in}{2.045978in}}%
\pgfpathcurveto{\pgfqpoint{1.586674in}{2.051802in}}{\pgfqpoint{1.589946in}{2.059702in}}{\pgfqpoint{1.589946in}{2.067938in}}%
\pgfpathcurveto{\pgfqpoint{1.589946in}{2.076175in}}{\pgfqpoint{1.586674in}{2.084075in}}{\pgfqpoint{1.580850in}{2.089899in}}%
\pgfpathcurveto{\pgfqpoint{1.575026in}{2.095722in}}{\pgfqpoint{1.567126in}{2.098995in}}{\pgfqpoint{1.558889in}{2.098995in}}%
\pgfpathcurveto{\pgfqpoint{1.550653in}{2.098995in}}{\pgfqpoint{1.542753in}{2.095722in}}{\pgfqpoint{1.536929in}{2.089899in}}%
\pgfpathcurveto{\pgfqpoint{1.531105in}{2.084075in}}{\pgfqpoint{1.527833in}{2.076175in}}{\pgfqpoint{1.527833in}{2.067938in}}%
\pgfpathcurveto{\pgfqpoint{1.527833in}{2.059702in}}{\pgfqpoint{1.531105in}{2.051802in}}{\pgfqpoint{1.536929in}{2.045978in}}%
\pgfpathcurveto{\pgfqpoint{1.542753in}{2.040154in}}{\pgfqpoint{1.550653in}{2.036882in}}{\pgfqpoint{1.558889in}{2.036882in}}%
\pgfpathclose%
\pgfusepath{stroke,fill}%
\end{pgfscope}%
\begin{pgfscope}%
\pgfpathrectangle{\pgfqpoint{0.100000in}{0.212622in}}{\pgfqpoint{3.696000in}{3.696000in}}%
\pgfusepath{clip}%
\pgfsetbuttcap%
\pgfsetroundjoin%
\definecolor{currentfill}{rgb}{0.121569,0.466667,0.705882}%
\pgfsetfillcolor{currentfill}%
\pgfsetfillopacity{0.300000}%
\pgfsetlinewidth{1.003750pt}%
\definecolor{currentstroke}{rgb}{0.121569,0.466667,0.705882}%
\pgfsetstrokecolor{currentstroke}%
\pgfsetstrokeopacity{0.300000}%
\pgfsetdash{}{0pt}%
\pgfpathmoveto{\pgfqpoint{1.558889in}{2.036882in}}%
\pgfpathcurveto{\pgfqpoint{1.567126in}{2.036882in}}{\pgfqpoint{1.575026in}{2.040154in}}{\pgfqpoint{1.580850in}{2.045978in}}%
\pgfpathcurveto{\pgfqpoint{1.586674in}{2.051802in}}{\pgfqpoint{1.589946in}{2.059702in}}{\pgfqpoint{1.589946in}{2.067938in}}%
\pgfpathcurveto{\pgfqpoint{1.589946in}{2.076175in}}{\pgfqpoint{1.586674in}{2.084075in}}{\pgfqpoint{1.580850in}{2.089899in}}%
\pgfpathcurveto{\pgfqpoint{1.575026in}{2.095722in}}{\pgfqpoint{1.567126in}{2.098995in}}{\pgfqpoint{1.558889in}{2.098995in}}%
\pgfpathcurveto{\pgfqpoint{1.550653in}{2.098995in}}{\pgfqpoint{1.542753in}{2.095722in}}{\pgfqpoint{1.536929in}{2.089899in}}%
\pgfpathcurveto{\pgfqpoint{1.531105in}{2.084075in}}{\pgfqpoint{1.527833in}{2.076175in}}{\pgfqpoint{1.527833in}{2.067938in}}%
\pgfpathcurveto{\pgfqpoint{1.527833in}{2.059702in}}{\pgfqpoint{1.531105in}{2.051802in}}{\pgfqpoint{1.536929in}{2.045978in}}%
\pgfpathcurveto{\pgfqpoint{1.542753in}{2.040154in}}{\pgfqpoint{1.550653in}{2.036882in}}{\pgfqpoint{1.558889in}{2.036882in}}%
\pgfpathclose%
\pgfusepath{stroke,fill}%
\end{pgfscope}%
\begin{pgfscope}%
\pgfpathrectangle{\pgfqpoint{0.100000in}{0.212622in}}{\pgfqpoint{3.696000in}{3.696000in}}%
\pgfusepath{clip}%
\pgfsetbuttcap%
\pgfsetroundjoin%
\definecolor{currentfill}{rgb}{0.121569,0.466667,0.705882}%
\pgfsetfillcolor{currentfill}%
\pgfsetfillopacity{0.300000}%
\pgfsetlinewidth{1.003750pt}%
\definecolor{currentstroke}{rgb}{0.121569,0.466667,0.705882}%
\pgfsetstrokecolor{currentstroke}%
\pgfsetstrokeopacity{0.300000}%
\pgfsetdash{}{0pt}%
\pgfpathmoveto{\pgfqpoint{1.558889in}{2.036882in}}%
\pgfpathcurveto{\pgfqpoint{1.567126in}{2.036882in}}{\pgfqpoint{1.575026in}{2.040154in}}{\pgfqpoint{1.580850in}{2.045978in}}%
\pgfpathcurveto{\pgfqpoint{1.586674in}{2.051802in}}{\pgfqpoint{1.589946in}{2.059702in}}{\pgfqpoint{1.589946in}{2.067938in}}%
\pgfpathcurveto{\pgfqpoint{1.589946in}{2.076175in}}{\pgfqpoint{1.586674in}{2.084075in}}{\pgfqpoint{1.580850in}{2.089899in}}%
\pgfpathcurveto{\pgfqpoint{1.575026in}{2.095722in}}{\pgfqpoint{1.567126in}{2.098995in}}{\pgfqpoint{1.558889in}{2.098995in}}%
\pgfpathcurveto{\pgfqpoint{1.550653in}{2.098995in}}{\pgfqpoint{1.542753in}{2.095722in}}{\pgfqpoint{1.536929in}{2.089899in}}%
\pgfpathcurveto{\pgfqpoint{1.531105in}{2.084075in}}{\pgfqpoint{1.527833in}{2.076175in}}{\pgfqpoint{1.527833in}{2.067938in}}%
\pgfpathcurveto{\pgfqpoint{1.527833in}{2.059702in}}{\pgfqpoint{1.531105in}{2.051802in}}{\pgfqpoint{1.536929in}{2.045978in}}%
\pgfpathcurveto{\pgfqpoint{1.542753in}{2.040154in}}{\pgfqpoint{1.550653in}{2.036882in}}{\pgfqpoint{1.558889in}{2.036882in}}%
\pgfpathclose%
\pgfusepath{stroke,fill}%
\end{pgfscope}%
\begin{pgfscope}%
\pgfpathrectangle{\pgfqpoint{0.100000in}{0.212622in}}{\pgfqpoint{3.696000in}{3.696000in}}%
\pgfusepath{clip}%
\pgfsetbuttcap%
\pgfsetroundjoin%
\definecolor{currentfill}{rgb}{0.121569,0.466667,0.705882}%
\pgfsetfillcolor{currentfill}%
\pgfsetfillopacity{0.300000}%
\pgfsetlinewidth{1.003750pt}%
\definecolor{currentstroke}{rgb}{0.121569,0.466667,0.705882}%
\pgfsetstrokecolor{currentstroke}%
\pgfsetstrokeopacity{0.300000}%
\pgfsetdash{}{0pt}%
\pgfpathmoveto{\pgfqpoint{1.558889in}{2.036882in}}%
\pgfpathcurveto{\pgfqpoint{1.567126in}{2.036882in}}{\pgfqpoint{1.575026in}{2.040154in}}{\pgfqpoint{1.580850in}{2.045978in}}%
\pgfpathcurveto{\pgfqpoint{1.586674in}{2.051802in}}{\pgfqpoint{1.589946in}{2.059702in}}{\pgfqpoint{1.589946in}{2.067938in}}%
\pgfpathcurveto{\pgfqpoint{1.589946in}{2.076175in}}{\pgfqpoint{1.586674in}{2.084075in}}{\pgfqpoint{1.580850in}{2.089899in}}%
\pgfpathcurveto{\pgfqpoint{1.575026in}{2.095722in}}{\pgfqpoint{1.567126in}{2.098995in}}{\pgfqpoint{1.558889in}{2.098995in}}%
\pgfpathcurveto{\pgfqpoint{1.550653in}{2.098995in}}{\pgfqpoint{1.542753in}{2.095722in}}{\pgfqpoint{1.536929in}{2.089899in}}%
\pgfpathcurveto{\pgfqpoint{1.531105in}{2.084075in}}{\pgfqpoint{1.527833in}{2.076175in}}{\pgfqpoint{1.527833in}{2.067938in}}%
\pgfpathcurveto{\pgfqpoint{1.527833in}{2.059702in}}{\pgfqpoint{1.531105in}{2.051802in}}{\pgfqpoint{1.536929in}{2.045978in}}%
\pgfpathcurveto{\pgfqpoint{1.542753in}{2.040154in}}{\pgfqpoint{1.550653in}{2.036882in}}{\pgfqpoint{1.558889in}{2.036882in}}%
\pgfpathclose%
\pgfusepath{stroke,fill}%
\end{pgfscope}%
\begin{pgfscope}%
\pgfpathrectangle{\pgfqpoint{0.100000in}{0.212622in}}{\pgfqpoint{3.696000in}{3.696000in}}%
\pgfusepath{clip}%
\pgfsetbuttcap%
\pgfsetroundjoin%
\definecolor{currentfill}{rgb}{0.121569,0.466667,0.705882}%
\pgfsetfillcolor{currentfill}%
\pgfsetfillopacity{0.300000}%
\pgfsetlinewidth{1.003750pt}%
\definecolor{currentstroke}{rgb}{0.121569,0.466667,0.705882}%
\pgfsetstrokecolor{currentstroke}%
\pgfsetstrokeopacity{0.300000}%
\pgfsetdash{}{0pt}%
\pgfpathmoveto{\pgfqpoint{1.558889in}{2.036882in}}%
\pgfpathcurveto{\pgfqpoint{1.567126in}{2.036882in}}{\pgfqpoint{1.575026in}{2.040154in}}{\pgfqpoint{1.580850in}{2.045978in}}%
\pgfpathcurveto{\pgfqpoint{1.586674in}{2.051802in}}{\pgfqpoint{1.589946in}{2.059702in}}{\pgfqpoint{1.589946in}{2.067938in}}%
\pgfpathcurveto{\pgfqpoint{1.589946in}{2.076175in}}{\pgfqpoint{1.586674in}{2.084075in}}{\pgfqpoint{1.580850in}{2.089899in}}%
\pgfpathcurveto{\pgfqpoint{1.575026in}{2.095722in}}{\pgfqpoint{1.567126in}{2.098995in}}{\pgfqpoint{1.558889in}{2.098995in}}%
\pgfpathcurveto{\pgfqpoint{1.550653in}{2.098995in}}{\pgfqpoint{1.542753in}{2.095722in}}{\pgfqpoint{1.536929in}{2.089899in}}%
\pgfpathcurveto{\pgfqpoint{1.531105in}{2.084075in}}{\pgfqpoint{1.527833in}{2.076175in}}{\pgfqpoint{1.527833in}{2.067938in}}%
\pgfpathcurveto{\pgfqpoint{1.527833in}{2.059702in}}{\pgfqpoint{1.531105in}{2.051802in}}{\pgfqpoint{1.536929in}{2.045978in}}%
\pgfpathcurveto{\pgfqpoint{1.542753in}{2.040154in}}{\pgfqpoint{1.550653in}{2.036882in}}{\pgfqpoint{1.558889in}{2.036882in}}%
\pgfpathclose%
\pgfusepath{stroke,fill}%
\end{pgfscope}%
\begin{pgfscope}%
\pgfpathrectangle{\pgfqpoint{0.100000in}{0.212622in}}{\pgfqpoint{3.696000in}{3.696000in}}%
\pgfusepath{clip}%
\pgfsetbuttcap%
\pgfsetroundjoin%
\definecolor{currentfill}{rgb}{0.121569,0.466667,0.705882}%
\pgfsetfillcolor{currentfill}%
\pgfsetfillopacity{0.300000}%
\pgfsetlinewidth{1.003750pt}%
\definecolor{currentstroke}{rgb}{0.121569,0.466667,0.705882}%
\pgfsetstrokecolor{currentstroke}%
\pgfsetstrokeopacity{0.300000}%
\pgfsetdash{}{0pt}%
\pgfpathmoveto{\pgfqpoint{1.558889in}{2.036882in}}%
\pgfpathcurveto{\pgfqpoint{1.567126in}{2.036882in}}{\pgfqpoint{1.575026in}{2.040154in}}{\pgfqpoint{1.580850in}{2.045978in}}%
\pgfpathcurveto{\pgfqpoint{1.586674in}{2.051802in}}{\pgfqpoint{1.589946in}{2.059702in}}{\pgfqpoint{1.589946in}{2.067938in}}%
\pgfpathcurveto{\pgfqpoint{1.589946in}{2.076175in}}{\pgfqpoint{1.586674in}{2.084075in}}{\pgfqpoint{1.580850in}{2.089899in}}%
\pgfpathcurveto{\pgfqpoint{1.575026in}{2.095722in}}{\pgfqpoint{1.567126in}{2.098995in}}{\pgfqpoint{1.558889in}{2.098995in}}%
\pgfpathcurveto{\pgfqpoint{1.550653in}{2.098995in}}{\pgfqpoint{1.542753in}{2.095722in}}{\pgfqpoint{1.536929in}{2.089899in}}%
\pgfpathcurveto{\pgfqpoint{1.531105in}{2.084075in}}{\pgfqpoint{1.527833in}{2.076175in}}{\pgfqpoint{1.527833in}{2.067938in}}%
\pgfpathcurveto{\pgfqpoint{1.527833in}{2.059702in}}{\pgfqpoint{1.531105in}{2.051802in}}{\pgfqpoint{1.536929in}{2.045978in}}%
\pgfpathcurveto{\pgfqpoint{1.542753in}{2.040154in}}{\pgfqpoint{1.550653in}{2.036882in}}{\pgfqpoint{1.558889in}{2.036882in}}%
\pgfpathclose%
\pgfusepath{stroke,fill}%
\end{pgfscope}%
\begin{pgfscope}%
\pgfpathrectangle{\pgfqpoint{0.100000in}{0.212622in}}{\pgfqpoint{3.696000in}{3.696000in}}%
\pgfusepath{clip}%
\pgfsetbuttcap%
\pgfsetroundjoin%
\definecolor{currentfill}{rgb}{0.121569,0.466667,0.705882}%
\pgfsetfillcolor{currentfill}%
\pgfsetfillopacity{0.300000}%
\pgfsetlinewidth{1.003750pt}%
\definecolor{currentstroke}{rgb}{0.121569,0.466667,0.705882}%
\pgfsetstrokecolor{currentstroke}%
\pgfsetstrokeopacity{0.300000}%
\pgfsetdash{}{0pt}%
\pgfpathmoveto{\pgfqpoint{1.558889in}{2.036882in}}%
\pgfpathcurveto{\pgfqpoint{1.567126in}{2.036882in}}{\pgfqpoint{1.575026in}{2.040154in}}{\pgfqpoint{1.580850in}{2.045978in}}%
\pgfpathcurveto{\pgfqpoint{1.586674in}{2.051802in}}{\pgfqpoint{1.589946in}{2.059702in}}{\pgfqpoint{1.589946in}{2.067938in}}%
\pgfpathcurveto{\pgfqpoint{1.589946in}{2.076175in}}{\pgfqpoint{1.586674in}{2.084075in}}{\pgfqpoint{1.580850in}{2.089899in}}%
\pgfpathcurveto{\pgfqpoint{1.575026in}{2.095722in}}{\pgfqpoint{1.567126in}{2.098995in}}{\pgfqpoint{1.558889in}{2.098995in}}%
\pgfpathcurveto{\pgfqpoint{1.550653in}{2.098995in}}{\pgfqpoint{1.542753in}{2.095722in}}{\pgfqpoint{1.536929in}{2.089899in}}%
\pgfpathcurveto{\pgfqpoint{1.531105in}{2.084075in}}{\pgfqpoint{1.527833in}{2.076175in}}{\pgfqpoint{1.527833in}{2.067938in}}%
\pgfpathcurveto{\pgfqpoint{1.527833in}{2.059702in}}{\pgfqpoint{1.531105in}{2.051802in}}{\pgfqpoint{1.536929in}{2.045978in}}%
\pgfpathcurveto{\pgfqpoint{1.542753in}{2.040154in}}{\pgfqpoint{1.550653in}{2.036882in}}{\pgfqpoint{1.558889in}{2.036882in}}%
\pgfpathclose%
\pgfusepath{stroke,fill}%
\end{pgfscope}%
\begin{pgfscope}%
\pgfpathrectangle{\pgfqpoint{0.100000in}{0.212622in}}{\pgfqpoint{3.696000in}{3.696000in}}%
\pgfusepath{clip}%
\pgfsetbuttcap%
\pgfsetroundjoin%
\definecolor{currentfill}{rgb}{0.121569,0.466667,0.705882}%
\pgfsetfillcolor{currentfill}%
\pgfsetfillopacity{0.300000}%
\pgfsetlinewidth{1.003750pt}%
\definecolor{currentstroke}{rgb}{0.121569,0.466667,0.705882}%
\pgfsetstrokecolor{currentstroke}%
\pgfsetstrokeopacity{0.300000}%
\pgfsetdash{}{0pt}%
\pgfpathmoveto{\pgfqpoint{1.558889in}{2.036882in}}%
\pgfpathcurveto{\pgfqpoint{1.567126in}{2.036882in}}{\pgfqpoint{1.575026in}{2.040154in}}{\pgfqpoint{1.580850in}{2.045978in}}%
\pgfpathcurveto{\pgfqpoint{1.586674in}{2.051802in}}{\pgfqpoint{1.589946in}{2.059702in}}{\pgfqpoint{1.589946in}{2.067938in}}%
\pgfpathcurveto{\pgfqpoint{1.589946in}{2.076175in}}{\pgfqpoint{1.586674in}{2.084075in}}{\pgfqpoint{1.580850in}{2.089899in}}%
\pgfpathcurveto{\pgfqpoint{1.575026in}{2.095722in}}{\pgfqpoint{1.567126in}{2.098995in}}{\pgfqpoint{1.558889in}{2.098995in}}%
\pgfpathcurveto{\pgfqpoint{1.550653in}{2.098995in}}{\pgfqpoint{1.542753in}{2.095722in}}{\pgfqpoint{1.536929in}{2.089899in}}%
\pgfpathcurveto{\pgfqpoint{1.531105in}{2.084075in}}{\pgfqpoint{1.527833in}{2.076175in}}{\pgfqpoint{1.527833in}{2.067938in}}%
\pgfpathcurveto{\pgfqpoint{1.527833in}{2.059702in}}{\pgfqpoint{1.531105in}{2.051802in}}{\pgfqpoint{1.536929in}{2.045978in}}%
\pgfpathcurveto{\pgfqpoint{1.542753in}{2.040154in}}{\pgfqpoint{1.550653in}{2.036882in}}{\pgfqpoint{1.558889in}{2.036882in}}%
\pgfpathclose%
\pgfusepath{stroke,fill}%
\end{pgfscope}%
\begin{pgfscope}%
\pgfpathrectangle{\pgfqpoint{0.100000in}{0.212622in}}{\pgfqpoint{3.696000in}{3.696000in}}%
\pgfusepath{clip}%
\pgfsetbuttcap%
\pgfsetroundjoin%
\definecolor{currentfill}{rgb}{0.121569,0.466667,0.705882}%
\pgfsetfillcolor{currentfill}%
\pgfsetfillopacity{0.300000}%
\pgfsetlinewidth{1.003750pt}%
\definecolor{currentstroke}{rgb}{0.121569,0.466667,0.705882}%
\pgfsetstrokecolor{currentstroke}%
\pgfsetstrokeopacity{0.300000}%
\pgfsetdash{}{0pt}%
\pgfpathmoveto{\pgfqpoint{1.558889in}{2.036882in}}%
\pgfpathcurveto{\pgfqpoint{1.567126in}{2.036882in}}{\pgfqpoint{1.575026in}{2.040154in}}{\pgfqpoint{1.580850in}{2.045978in}}%
\pgfpathcurveto{\pgfqpoint{1.586674in}{2.051802in}}{\pgfqpoint{1.589946in}{2.059702in}}{\pgfqpoint{1.589946in}{2.067938in}}%
\pgfpathcurveto{\pgfqpoint{1.589946in}{2.076175in}}{\pgfqpoint{1.586674in}{2.084075in}}{\pgfqpoint{1.580850in}{2.089899in}}%
\pgfpathcurveto{\pgfqpoint{1.575026in}{2.095722in}}{\pgfqpoint{1.567126in}{2.098995in}}{\pgfqpoint{1.558889in}{2.098995in}}%
\pgfpathcurveto{\pgfqpoint{1.550653in}{2.098995in}}{\pgfqpoint{1.542753in}{2.095722in}}{\pgfqpoint{1.536929in}{2.089899in}}%
\pgfpathcurveto{\pgfqpoint{1.531105in}{2.084075in}}{\pgfqpoint{1.527833in}{2.076175in}}{\pgfqpoint{1.527833in}{2.067938in}}%
\pgfpathcurveto{\pgfqpoint{1.527833in}{2.059702in}}{\pgfqpoint{1.531105in}{2.051802in}}{\pgfqpoint{1.536929in}{2.045978in}}%
\pgfpathcurveto{\pgfqpoint{1.542753in}{2.040154in}}{\pgfqpoint{1.550653in}{2.036882in}}{\pgfqpoint{1.558889in}{2.036882in}}%
\pgfpathclose%
\pgfusepath{stroke,fill}%
\end{pgfscope}%
\begin{pgfscope}%
\pgfpathrectangle{\pgfqpoint{0.100000in}{0.212622in}}{\pgfqpoint{3.696000in}{3.696000in}}%
\pgfusepath{clip}%
\pgfsetbuttcap%
\pgfsetroundjoin%
\definecolor{currentfill}{rgb}{0.121569,0.466667,0.705882}%
\pgfsetfillcolor{currentfill}%
\pgfsetfillopacity{0.300000}%
\pgfsetlinewidth{1.003750pt}%
\definecolor{currentstroke}{rgb}{0.121569,0.466667,0.705882}%
\pgfsetstrokecolor{currentstroke}%
\pgfsetstrokeopacity{0.300000}%
\pgfsetdash{}{0pt}%
\pgfpathmoveto{\pgfqpoint{1.558889in}{2.036882in}}%
\pgfpathcurveto{\pgfqpoint{1.567126in}{2.036882in}}{\pgfqpoint{1.575026in}{2.040154in}}{\pgfqpoint{1.580850in}{2.045978in}}%
\pgfpathcurveto{\pgfqpoint{1.586674in}{2.051802in}}{\pgfqpoint{1.589946in}{2.059702in}}{\pgfqpoint{1.589946in}{2.067938in}}%
\pgfpathcurveto{\pgfqpoint{1.589946in}{2.076175in}}{\pgfqpoint{1.586674in}{2.084075in}}{\pgfqpoint{1.580850in}{2.089899in}}%
\pgfpathcurveto{\pgfqpoint{1.575026in}{2.095722in}}{\pgfqpoint{1.567126in}{2.098995in}}{\pgfqpoint{1.558889in}{2.098995in}}%
\pgfpathcurveto{\pgfqpoint{1.550653in}{2.098995in}}{\pgfqpoint{1.542753in}{2.095722in}}{\pgfqpoint{1.536929in}{2.089899in}}%
\pgfpathcurveto{\pgfqpoint{1.531105in}{2.084075in}}{\pgfqpoint{1.527833in}{2.076175in}}{\pgfqpoint{1.527833in}{2.067938in}}%
\pgfpathcurveto{\pgfqpoint{1.527833in}{2.059702in}}{\pgfqpoint{1.531105in}{2.051802in}}{\pgfqpoint{1.536929in}{2.045978in}}%
\pgfpathcurveto{\pgfqpoint{1.542753in}{2.040154in}}{\pgfqpoint{1.550653in}{2.036882in}}{\pgfqpoint{1.558889in}{2.036882in}}%
\pgfpathclose%
\pgfusepath{stroke,fill}%
\end{pgfscope}%
\begin{pgfscope}%
\pgfpathrectangle{\pgfqpoint{0.100000in}{0.212622in}}{\pgfqpoint{3.696000in}{3.696000in}}%
\pgfusepath{clip}%
\pgfsetbuttcap%
\pgfsetroundjoin%
\definecolor{currentfill}{rgb}{0.121569,0.466667,0.705882}%
\pgfsetfillcolor{currentfill}%
\pgfsetfillopacity{0.300000}%
\pgfsetlinewidth{1.003750pt}%
\definecolor{currentstroke}{rgb}{0.121569,0.466667,0.705882}%
\pgfsetstrokecolor{currentstroke}%
\pgfsetstrokeopacity{0.300000}%
\pgfsetdash{}{0pt}%
\pgfpathmoveto{\pgfqpoint{1.558889in}{2.036882in}}%
\pgfpathcurveto{\pgfqpoint{1.567126in}{2.036882in}}{\pgfqpoint{1.575026in}{2.040154in}}{\pgfqpoint{1.580850in}{2.045978in}}%
\pgfpathcurveto{\pgfqpoint{1.586674in}{2.051802in}}{\pgfqpoint{1.589946in}{2.059702in}}{\pgfqpoint{1.589946in}{2.067938in}}%
\pgfpathcurveto{\pgfqpoint{1.589946in}{2.076175in}}{\pgfqpoint{1.586674in}{2.084075in}}{\pgfqpoint{1.580850in}{2.089899in}}%
\pgfpathcurveto{\pgfqpoint{1.575026in}{2.095722in}}{\pgfqpoint{1.567126in}{2.098995in}}{\pgfqpoint{1.558889in}{2.098995in}}%
\pgfpathcurveto{\pgfqpoint{1.550653in}{2.098995in}}{\pgfqpoint{1.542753in}{2.095722in}}{\pgfqpoint{1.536929in}{2.089899in}}%
\pgfpathcurveto{\pgfqpoint{1.531105in}{2.084075in}}{\pgfqpoint{1.527833in}{2.076175in}}{\pgfqpoint{1.527833in}{2.067938in}}%
\pgfpathcurveto{\pgfqpoint{1.527833in}{2.059702in}}{\pgfqpoint{1.531105in}{2.051802in}}{\pgfqpoint{1.536929in}{2.045978in}}%
\pgfpathcurveto{\pgfqpoint{1.542753in}{2.040154in}}{\pgfqpoint{1.550653in}{2.036882in}}{\pgfqpoint{1.558889in}{2.036882in}}%
\pgfpathclose%
\pgfusepath{stroke,fill}%
\end{pgfscope}%
\begin{pgfscope}%
\pgfpathrectangle{\pgfqpoint{0.100000in}{0.212622in}}{\pgfqpoint{3.696000in}{3.696000in}}%
\pgfusepath{clip}%
\pgfsetbuttcap%
\pgfsetroundjoin%
\definecolor{currentfill}{rgb}{0.121569,0.466667,0.705882}%
\pgfsetfillcolor{currentfill}%
\pgfsetfillopacity{0.300000}%
\pgfsetlinewidth{1.003750pt}%
\definecolor{currentstroke}{rgb}{0.121569,0.466667,0.705882}%
\pgfsetstrokecolor{currentstroke}%
\pgfsetstrokeopacity{0.300000}%
\pgfsetdash{}{0pt}%
\pgfpathmoveto{\pgfqpoint{1.558889in}{2.036882in}}%
\pgfpathcurveto{\pgfqpoint{1.567126in}{2.036882in}}{\pgfqpoint{1.575026in}{2.040154in}}{\pgfqpoint{1.580850in}{2.045978in}}%
\pgfpathcurveto{\pgfqpoint{1.586674in}{2.051802in}}{\pgfqpoint{1.589946in}{2.059702in}}{\pgfqpoint{1.589946in}{2.067938in}}%
\pgfpathcurveto{\pgfqpoint{1.589946in}{2.076175in}}{\pgfqpoint{1.586674in}{2.084075in}}{\pgfqpoint{1.580850in}{2.089899in}}%
\pgfpathcurveto{\pgfqpoint{1.575026in}{2.095722in}}{\pgfqpoint{1.567126in}{2.098995in}}{\pgfqpoint{1.558889in}{2.098995in}}%
\pgfpathcurveto{\pgfqpoint{1.550653in}{2.098995in}}{\pgfqpoint{1.542753in}{2.095722in}}{\pgfqpoint{1.536929in}{2.089899in}}%
\pgfpathcurveto{\pgfqpoint{1.531105in}{2.084075in}}{\pgfqpoint{1.527833in}{2.076175in}}{\pgfqpoint{1.527833in}{2.067938in}}%
\pgfpathcurveto{\pgfqpoint{1.527833in}{2.059702in}}{\pgfqpoint{1.531105in}{2.051802in}}{\pgfqpoint{1.536929in}{2.045978in}}%
\pgfpathcurveto{\pgfqpoint{1.542753in}{2.040154in}}{\pgfqpoint{1.550653in}{2.036882in}}{\pgfqpoint{1.558889in}{2.036882in}}%
\pgfpathclose%
\pgfusepath{stroke,fill}%
\end{pgfscope}%
\begin{pgfscope}%
\pgfpathrectangle{\pgfqpoint{0.100000in}{0.212622in}}{\pgfqpoint{3.696000in}{3.696000in}}%
\pgfusepath{clip}%
\pgfsetbuttcap%
\pgfsetroundjoin%
\definecolor{currentfill}{rgb}{0.121569,0.466667,0.705882}%
\pgfsetfillcolor{currentfill}%
\pgfsetfillopacity{0.300000}%
\pgfsetlinewidth{1.003750pt}%
\definecolor{currentstroke}{rgb}{0.121569,0.466667,0.705882}%
\pgfsetstrokecolor{currentstroke}%
\pgfsetstrokeopacity{0.300000}%
\pgfsetdash{}{0pt}%
\pgfpathmoveto{\pgfqpoint{1.558889in}{2.036882in}}%
\pgfpathcurveto{\pgfqpoint{1.567126in}{2.036882in}}{\pgfqpoint{1.575026in}{2.040154in}}{\pgfqpoint{1.580850in}{2.045978in}}%
\pgfpathcurveto{\pgfqpoint{1.586674in}{2.051802in}}{\pgfqpoint{1.589946in}{2.059702in}}{\pgfqpoint{1.589946in}{2.067938in}}%
\pgfpathcurveto{\pgfqpoint{1.589946in}{2.076175in}}{\pgfqpoint{1.586674in}{2.084075in}}{\pgfqpoint{1.580850in}{2.089899in}}%
\pgfpathcurveto{\pgfqpoint{1.575026in}{2.095722in}}{\pgfqpoint{1.567126in}{2.098995in}}{\pgfqpoint{1.558889in}{2.098995in}}%
\pgfpathcurveto{\pgfqpoint{1.550653in}{2.098995in}}{\pgfqpoint{1.542753in}{2.095722in}}{\pgfqpoint{1.536929in}{2.089899in}}%
\pgfpathcurveto{\pgfqpoint{1.531105in}{2.084075in}}{\pgfqpoint{1.527833in}{2.076175in}}{\pgfqpoint{1.527833in}{2.067938in}}%
\pgfpathcurveto{\pgfqpoint{1.527833in}{2.059702in}}{\pgfqpoint{1.531105in}{2.051802in}}{\pgfqpoint{1.536929in}{2.045978in}}%
\pgfpathcurveto{\pgfqpoint{1.542753in}{2.040154in}}{\pgfqpoint{1.550653in}{2.036882in}}{\pgfqpoint{1.558889in}{2.036882in}}%
\pgfpathclose%
\pgfusepath{stroke,fill}%
\end{pgfscope}%
\begin{pgfscope}%
\pgfpathrectangle{\pgfqpoint{0.100000in}{0.212622in}}{\pgfqpoint{3.696000in}{3.696000in}}%
\pgfusepath{clip}%
\pgfsetbuttcap%
\pgfsetroundjoin%
\definecolor{currentfill}{rgb}{0.121569,0.466667,0.705882}%
\pgfsetfillcolor{currentfill}%
\pgfsetfillopacity{0.300000}%
\pgfsetlinewidth{1.003750pt}%
\definecolor{currentstroke}{rgb}{0.121569,0.466667,0.705882}%
\pgfsetstrokecolor{currentstroke}%
\pgfsetstrokeopacity{0.300000}%
\pgfsetdash{}{0pt}%
\pgfpathmoveto{\pgfqpoint{1.558889in}{2.036882in}}%
\pgfpathcurveto{\pgfqpoint{1.567126in}{2.036882in}}{\pgfqpoint{1.575026in}{2.040154in}}{\pgfqpoint{1.580850in}{2.045978in}}%
\pgfpathcurveto{\pgfqpoint{1.586674in}{2.051802in}}{\pgfqpoint{1.589946in}{2.059702in}}{\pgfqpoint{1.589946in}{2.067938in}}%
\pgfpathcurveto{\pgfqpoint{1.589946in}{2.076175in}}{\pgfqpoint{1.586674in}{2.084075in}}{\pgfqpoint{1.580850in}{2.089899in}}%
\pgfpathcurveto{\pgfqpoint{1.575026in}{2.095722in}}{\pgfqpoint{1.567126in}{2.098995in}}{\pgfqpoint{1.558889in}{2.098995in}}%
\pgfpathcurveto{\pgfqpoint{1.550653in}{2.098995in}}{\pgfqpoint{1.542753in}{2.095722in}}{\pgfqpoint{1.536929in}{2.089899in}}%
\pgfpathcurveto{\pgfqpoint{1.531105in}{2.084075in}}{\pgfqpoint{1.527833in}{2.076175in}}{\pgfqpoint{1.527833in}{2.067938in}}%
\pgfpathcurveto{\pgfqpoint{1.527833in}{2.059702in}}{\pgfqpoint{1.531105in}{2.051802in}}{\pgfqpoint{1.536929in}{2.045978in}}%
\pgfpathcurveto{\pgfqpoint{1.542753in}{2.040154in}}{\pgfqpoint{1.550653in}{2.036882in}}{\pgfqpoint{1.558889in}{2.036882in}}%
\pgfpathclose%
\pgfusepath{stroke,fill}%
\end{pgfscope}%
\begin{pgfscope}%
\pgfpathrectangle{\pgfqpoint{0.100000in}{0.212622in}}{\pgfqpoint{3.696000in}{3.696000in}}%
\pgfusepath{clip}%
\pgfsetbuttcap%
\pgfsetroundjoin%
\definecolor{currentfill}{rgb}{0.121569,0.466667,0.705882}%
\pgfsetfillcolor{currentfill}%
\pgfsetfillopacity{0.300000}%
\pgfsetlinewidth{1.003750pt}%
\definecolor{currentstroke}{rgb}{0.121569,0.466667,0.705882}%
\pgfsetstrokecolor{currentstroke}%
\pgfsetstrokeopacity{0.300000}%
\pgfsetdash{}{0pt}%
\pgfpathmoveto{\pgfqpoint{1.558889in}{2.036882in}}%
\pgfpathcurveto{\pgfqpoint{1.567126in}{2.036882in}}{\pgfqpoint{1.575026in}{2.040154in}}{\pgfqpoint{1.580850in}{2.045978in}}%
\pgfpathcurveto{\pgfqpoint{1.586674in}{2.051802in}}{\pgfqpoint{1.589946in}{2.059702in}}{\pgfqpoint{1.589946in}{2.067938in}}%
\pgfpathcurveto{\pgfqpoint{1.589946in}{2.076175in}}{\pgfqpoint{1.586674in}{2.084075in}}{\pgfqpoint{1.580850in}{2.089899in}}%
\pgfpathcurveto{\pgfqpoint{1.575026in}{2.095722in}}{\pgfqpoint{1.567126in}{2.098995in}}{\pgfqpoint{1.558889in}{2.098995in}}%
\pgfpathcurveto{\pgfqpoint{1.550653in}{2.098995in}}{\pgfqpoint{1.542753in}{2.095722in}}{\pgfqpoint{1.536929in}{2.089899in}}%
\pgfpathcurveto{\pgfqpoint{1.531105in}{2.084075in}}{\pgfqpoint{1.527833in}{2.076175in}}{\pgfqpoint{1.527833in}{2.067938in}}%
\pgfpathcurveto{\pgfqpoint{1.527833in}{2.059702in}}{\pgfqpoint{1.531105in}{2.051802in}}{\pgfqpoint{1.536929in}{2.045978in}}%
\pgfpathcurveto{\pgfqpoint{1.542753in}{2.040154in}}{\pgfqpoint{1.550653in}{2.036882in}}{\pgfqpoint{1.558889in}{2.036882in}}%
\pgfpathclose%
\pgfusepath{stroke,fill}%
\end{pgfscope}%
\begin{pgfscope}%
\pgfpathrectangle{\pgfqpoint{0.100000in}{0.212622in}}{\pgfqpoint{3.696000in}{3.696000in}}%
\pgfusepath{clip}%
\pgfsetbuttcap%
\pgfsetroundjoin%
\definecolor{currentfill}{rgb}{0.121569,0.466667,0.705882}%
\pgfsetfillcolor{currentfill}%
\pgfsetfillopacity{0.300000}%
\pgfsetlinewidth{1.003750pt}%
\definecolor{currentstroke}{rgb}{0.121569,0.466667,0.705882}%
\pgfsetstrokecolor{currentstroke}%
\pgfsetstrokeopacity{0.300000}%
\pgfsetdash{}{0pt}%
\pgfpathmoveto{\pgfqpoint{1.558889in}{2.036882in}}%
\pgfpathcurveto{\pgfqpoint{1.567126in}{2.036882in}}{\pgfqpoint{1.575026in}{2.040154in}}{\pgfqpoint{1.580850in}{2.045978in}}%
\pgfpathcurveto{\pgfqpoint{1.586674in}{2.051802in}}{\pgfqpoint{1.589946in}{2.059702in}}{\pgfqpoint{1.589946in}{2.067938in}}%
\pgfpathcurveto{\pgfqpoint{1.589946in}{2.076175in}}{\pgfqpoint{1.586674in}{2.084075in}}{\pgfqpoint{1.580850in}{2.089899in}}%
\pgfpathcurveto{\pgfqpoint{1.575026in}{2.095722in}}{\pgfqpoint{1.567126in}{2.098995in}}{\pgfqpoint{1.558889in}{2.098995in}}%
\pgfpathcurveto{\pgfqpoint{1.550653in}{2.098995in}}{\pgfqpoint{1.542753in}{2.095722in}}{\pgfqpoint{1.536929in}{2.089899in}}%
\pgfpathcurveto{\pgfqpoint{1.531105in}{2.084075in}}{\pgfqpoint{1.527833in}{2.076175in}}{\pgfqpoint{1.527833in}{2.067938in}}%
\pgfpathcurveto{\pgfqpoint{1.527833in}{2.059702in}}{\pgfqpoint{1.531105in}{2.051802in}}{\pgfqpoint{1.536929in}{2.045978in}}%
\pgfpathcurveto{\pgfqpoint{1.542753in}{2.040154in}}{\pgfqpoint{1.550653in}{2.036882in}}{\pgfqpoint{1.558889in}{2.036882in}}%
\pgfpathclose%
\pgfusepath{stroke,fill}%
\end{pgfscope}%
\begin{pgfscope}%
\pgfpathrectangle{\pgfqpoint{0.100000in}{0.212622in}}{\pgfqpoint{3.696000in}{3.696000in}}%
\pgfusepath{clip}%
\pgfsetbuttcap%
\pgfsetroundjoin%
\definecolor{currentfill}{rgb}{0.121569,0.466667,0.705882}%
\pgfsetfillcolor{currentfill}%
\pgfsetfillopacity{0.300000}%
\pgfsetlinewidth{1.003750pt}%
\definecolor{currentstroke}{rgb}{0.121569,0.466667,0.705882}%
\pgfsetstrokecolor{currentstroke}%
\pgfsetstrokeopacity{0.300000}%
\pgfsetdash{}{0pt}%
\pgfpathmoveto{\pgfqpoint{1.558889in}{2.036882in}}%
\pgfpathcurveto{\pgfqpoint{1.567126in}{2.036882in}}{\pgfqpoint{1.575026in}{2.040154in}}{\pgfqpoint{1.580850in}{2.045978in}}%
\pgfpathcurveto{\pgfqpoint{1.586674in}{2.051802in}}{\pgfqpoint{1.589946in}{2.059702in}}{\pgfqpoint{1.589946in}{2.067938in}}%
\pgfpathcurveto{\pgfqpoint{1.589946in}{2.076175in}}{\pgfqpoint{1.586674in}{2.084075in}}{\pgfqpoint{1.580850in}{2.089899in}}%
\pgfpathcurveto{\pgfqpoint{1.575026in}{2.095722in}}{\pgfqpoint{1.567126in}{2.098995in}}{\pgfqpoint{1.558889in}{2.098995in}}%
\pgfpathcurveto{\pgfqpoint{1.550653in}{2.098995in}}{\pgfqpoint{1.542753in}{2.095722in}}{\pgfqpoint{1.536929in}{2.089899in}}%
\pgfpathcurveto{\pgfqpoint{1.531105in}{2.084075in}}{\pgfqpoint{1.527833in}{2.076175in}}{\pgfqpoint{1.527833in}{2.067938in}}%
\pgfpathcurveto{\pgfqpoint{1.527833in}{2.059702in}}{\pgfqpoint{1.531105in}{2.051802in}}{\pgfqpoint{1.536929in}{2.045978in}}%
\pgfpathcurveto{\pgfqpoint{1.542753in}{2.040154in}}{\pgfqpoint{1.550653in}{2.036882in}}{\pgfqpoint{1.558889in}{2.036882in}}%
\pgfpathclose%
\pgfusepath{stroke,fill}%
\end{pgfscope}%
\begin{pgfscope}%
\pgfpathrectangle{\pgfqpoint{0.100000in}{0.212622in}}{\pgfqpoint{3.696000in}{3.696000in}}%
\pgfusepath{clip}%
\pgfsetbuttcap%
\pgfsetroundjoin%
\definecolor{currentfill}{rgb}{0.121569,0.466667,0.705882}%
\pgfsetfillcolor{currentfill}%
\pgfsetfillopacity{0.300000}%
\pgfsetlinewidth{1.003750pt}%
\definecolor{currentstroke}{rgb}{0.121569,0.466667,0.705882}%
\pgfsetstrokecolor{currentstroke}%
\pgfsetstrokeopacity{0.300000}%
\pgfsetdash{}{0pt}%
\pgfpathmoveto{\pgfqpoint{1.558889in}{2.036882in}}%
\pgfpathcurveto{\pgfqpoint{1.567126in}{2.036882in}}{\pgfqpoint{1.575026in}{2.040154in}}{\pgfqpoint{1.580850in}{2.045978in}}%
\pgfpathcurveto{\pgfqpoint{1.586674in}{2.051802in}}{\pgfqpoint{1.589946in}{2.059702in}}{\pgfqpoint{1.589946in}{2.067938in}}%
\pgfpathcurveto{\pgfqpoint{1.589946in}{2.076175in}}{\pgfqpoint{1.586674in}{2.084075in}}{\pgfqpoint{1.580850in}{2.089899in}}%
\pgfpathcurveto{\pgfqpoint{1.575026in}{2.095722in}}{\pgfqpoint{1.567126in}{2.098995in}}{\pgfqpoint{1.558889in}{2.098995in}}%
\pgfpathcurveto{\pgfqpoint{1.550653in}{2.098995in}}{\pgfqpoint{1.542753in}{2.095722in}}{\pgfqpoint{1.536929in}{2.089899in}}%
\pgfpathcurveto{\pgfqpoint{1.531105in}{2.084075in}}{\pgfqpoint{1.527833in}{2.076175in}}{\pgfqpoint{1.527833in}{2.067938in}}%
\pgfpathcurveto{\pgfqpoint{1.527833in}{2.059702in}}{\pgfqpoint{1.531105in}{2.051802in}}{\pgfqpoint{1.536929in}{2.045978in}}%
\pgfpathcurveto{\pgfqpoint{1.542753in}{2.040154in}}{\pgfqpoint{1.550653in}{2.036882in}}{\pgfqpoint{1.558889in}{2.036882in}}%
\pgfpathclose%
\pgfusepath{stroke,fill}%
\end{pgfscope}%
\begin{pgfscope}%
\pgfpathrectangle{\pgfqpoint{0.100000in}{0.212622in}}{\pgfqpoint{3.696000in}{3.696000in}}%
\pgfusepath{clip}%
\pgfsetbuttcap%
\pgfsetroundjoin%
\definecolor{currentfill}{rgb}{0.121569,0.466667,0.705882}%
\pgfsetfillcolor{currentfill}%
\pgfsetfillopacity{0.300000}%
\pgfsetlinewidth{1.003750pt}%
\definecolor{currentstroke}{rgb}{0.121569,0.466667,0.705882}%
\pgfsetstrokecolor{currentstroke}%
\pgfsetstrokeopacity{0.300000}%
\pgfsetdash{}{0pt}%
\pgfpathmoveto{\pgfqpoint{1.558889in}{2.036882in}}%
\pgfpathcurveto{\pgfqpoint{1.567126in}{2.036882in}}{\pgfqpoint{1.575026in}{2.040154in}}{\pgfqpoint{1.580850in}{2.045978in}}%
\pgfpathcurveto{\pgfqpoint{1.586674in}{2.051802in}}{\pgfqpoint{1.589946in}{2.059702in}}{\pgfqpoint{1.589946in}{2.067938in}}%
\pgfpathcurveto{\pgfqpoint{1.589946in}{2.076175in}}{\pgfqpoint{1.586674in}{2.084075in}}{\pgfqpoint{1.580850in}{2.089899in}}%
\pgfpathcurveto{\pgfqpoint{1.575026in}{2.095722in}}{\pgfqpoint{1.567126in}{2.098995in}}{\pgfqpoint{1.558889in}{2.098995in}}%
\pgfpathcurveto{\pgfqpoint{1.550653in}{2.098995in}}{\pgfqpoint{1.542753in}{2.095722in}}{\pgfqpoint{1.536929in}{2.089899in}}%
\pgfpathcurveto{\pgfqpoint{1.531105in}{2.084075in}}{\pgfqpoint{1.527833in}{2.076175in}}{\pgfqpoint{1.527833in}{2.067938in}}%
\pgfpathcurveto{\pgfqpoint{1.527833in}{2.059702in}}{\pgfqpoint{1.531105in}{2.051802in}}{\pgfqpoint{1.536929in}{2.045978in}}%
\pgfpathcurveto{\pgfqpoint{1.542753in}{2.040154in}}{\pgfqpoint{1.550653in}{2.036882in}}{\pgfqpoint{1.558889in}{2.036882in}}%
\pgfpathclose%
\pgfusepath{stroke,fill}%
\end{pgfscope}%
\begin{pgfscope}%
\pgfpathrectangle{\pgfqpoint{0.100000in}{0.212622in}}{\pgfqpoint{3.696000in}{3.696000in}}%
\pgfusepath{clip}%
\pgfsetbuttcap%
\pgfsetroundjoin%
\definecolor{currentfill}{rgb}{0.121569,0.466667,0.705882}%
\pgfsetfillcolor{currentfill}%
\pgfsetfillopacity{0.300000}%
\pgfsetlinewidth{1.003750pt}%
\definecolor{currentstroke}{rgb}{0.121569,0.466667,0.705882}%
\pgfsetstrokecolor{currentstroke}%
\pgfsetstrokeopacity{0.300000}%
\pgfsetdash{}{0pt}%
\pgfpathmoveto{\pgfqpoint{1.558889in}{2.036882in}}%
\pgfpathcurveto{\pgfqpoint{1.567126in}{2.036882in}}{\pgfqpoint{1.575026in}{2.040154in}}{\pgfqpoint{1.580850in}{2.045978in}}%
\pgfpathcurveto{\pgfqpoint{1.586674in}{2.051802in}}{\pgfqpoint{1.589946in}{2.059702in}}{\pgfqpoint{1.589946in}{2.067938in}}%
\pgfpathcurveto{\pgfqpoint{1.589946in}{2.076175in}}{\pgfqpoint{1.586674in}{2.084075in}}{\pgfqpoint{1.580850in}{2.089899in}}%
\pgfpathcurveto{\pgfqpoint{1.575026in}{2.095722in}}{\pgfqpoint{1.567126in}{2.098995in}}{\pgfqpoint{1.558889in}{2.098995in}}%
\pgfpathcurveto{\pgfqpoint{1.550653in}{2.098995in}}{\pgfqpoint{1.542753in}{2.095722in}}{\pgfqpoint{1.536929in}{2.089899in}}%
\pgfpathcurveto{\pgfqpoint{1.531105in}{2.084075in}}{\pgfqpoint{1.527833in}{2.076175in}}{\pgfqpoint{1.527833in}{2.067938in}}%
\pgfpathcurveto{\pgfqpoint{1.527833in}{2.059702in}}{\pgfqpoint{1.531105in}{2.051802in}}{\pgfqpoint{1.536929in}{2.045978in}}%
\pgfpathcurveto{\pgfqpoint{1.542753in}{2.040154in}}{\pgfqpoint{1.550653in}{2.036882in}}{\pgfqpoint{1.558889in}{2.036882in}}%
\pgfpathclose%
\pgfusepath{stroke,fill}%
\end{pgfscope}%
\begin{pgfscope}%
\pgfpathrectangle{\pgfqpoint{0.100000in}{0.212622in}}{\pgfqpoint{3.696000in}{3.696000in}}%
\pgfusepath{clip}%
\pgfsetbuttcap%
\pgfsetroundjoin%
\definecolor{currentfill}{rgb}{0.121569,0.466667,0.705882}%
\pgfsetfillcolor{currentfill}%
\pgfsetfillopacity{0.300000}%
\pgfsetlinewidth{1.003750pt}%
\definecolor{currentstroke}{rgb}{0.121569,0.466667,0.705882}%
\pgfsetstrokecolor{currentstroke}%
\pgfsetstrokeopacity{0.300000}%
\pgfsetdash{}{0pt}%
\pgfpathmoveto{\pgfqpoint{1.558889in}{2.036882in}}%
\pgfpathcurveto{\pgfqpoint{1.567126in}{2.036882in}}{\pgfqpoint{1.575026in}{2.040154in}}{\pgfqpoint{1.580850in}{2.045978in}}%
\pgfpathcurveto{\pgfqpoint{1.586674in}{2.051802in}}{\pgfqpoint{1.589946in}{2.059702in}}{\pgfqpoint{1.589946in}{2.067938in}}%
\pgfpathcurveto{\pgfqpoint{1.589946in}{2.076175in}}{\pgfqpoint{1.586674in}{2.084075in}}{\pgfqpoint{1.580850in}{2.089899in}}%
\pgfpathcurveto{\pgfqpoint{1.575026in}{2.095722in}}{\pgfqpoint{1.567126in}{2.098995in}}{\pgfqpoint{1.558889in}{2.098995in}}%
\pgfpathcurveto{\pgfqpoint{1.550653in}{2.098995in}}{\pgfqpoint{1.542753in}{2.095722in}}{\pgfqpoint{1.536929in}{2.089899in}}%
\pgfpathcurveto{\pgfqpoint{1.531105in}{2.084075in}}{\pgfqpoint{1.527833in}{2.076175in}}{\pgfqpoint{1.527833in}{2.067938in}}%
\pgfpathcurveto{\pgfqpoint{1.527833in}{2.059702in}}{\pgfqpoint{1.531105in}{2.051802in}}{\pgfqpoint{1.536929in}{2.045978in}}%
\pgfpathcurveto{\pgfqpoint{1.542753in}{2.040154in}}{\pgfqpoint{1.550653in}{2.036882in}}{\pgfqpoint{1.558889in}{2.036882in}}%
\pgfpathclose%
\pgfusepath{stroke,fill}%
\end{pgfscope}%
\begin{pgfscope}%
\pgfpathrectangle{\pgfqpoint{0.100000in}{0.212622in}}{\pgfqpoint{3.696000in}{3.696000in}}%
\pgfusepath{clip}%
\pgfsetbuttcap%
\pgfsetroundjoin%
\definecolor{currentfill}{rgb}{0.121569,0.466667,0.705882}%
\pgfsetfillcolor{currentfill}%
\pgfsetfillopacity{0.300000}%
\pgfsetlinewidth{1.003750pt}%
\definecolor{currentstroke}{rgb}{0.121569,0.466667,0.705882}%
\pgfsetstrokecolor{currentstroke}%
\pgfsetstrokeopacity{0.300000}%
\pgfsetdash{}{0pt}%
\pgfpathmoveto{\pgfqpoint{1.558889in}{2.036882in}}%
\pgfpathcurveto{\pgfqpoint{1.567126in}{2.036882in}}{\pgfqpoint{1.575026in}{2.040154in}}{\pgfqpoint{1.580850in}{2.045978in}}%
\pgfpathcurveto{\pgfqpoint{1.586674in}{2.051802in}}{\pgfqpoint{1.589946in}{2.059702in}}{\pgfqpoint{1.589946in}{2.067938in}}%
\pgfpathcurveto{\pgfqpoint{1.589946in}{2.076175in}}{\pgfqpoint{1.586674in}{2.084075in}}{\pgfqpoint{1.580850in}{2.089899in}}%
\pgfpathcurveto{\pgfqpoint{1.575026in}{2.095722in}}{\pgfqpoint{1.567126in}{2.098995in}}{\pgfqpoint{1.558889in}{2.098995in}}%
\pgfpathcurveto{\pgfqpoint{1.550653in}{2.098995in}}{\pgfqpoint{1.542753in}{2.095722in}}{\pgfqpoint{1.536929in}{2.089899in}}%
\pgfpathcurveto{\pgfqpoint{1.531105in}{2.084075in}}{\pgfqpoint{1.527833in}{2.076175in}}{\pgfqpoint{1.527833in}{2.067938in}}%
\pgfpathcurveto{\pgfqpoint{1.527833in}{2.059702in}}{\pgfqpoint{1.531105in}{2.051802in}}{\pgfqpoint{1.536929in}{2.045978in}}%
\pgfpathcurveto{\pgfqpoint{1.542753in}{2.040154in}}{\pgfqpoint{1.550653in}{2.036882in}}{\pgfqpoint{1.558889in}{2.036882in}}%
\pgfpathclose%
\pgfusepath{stroke,fill}%
\end{pgfscope}%
\begin{pgfscope}%
\pgfpathrectangle{\pgfqpoint{0.100000in}{0.212622in}}{\pgfqpoint{3.696000in}{3.696000in}}%
\pgfusepath{clip}%
\pgfsetbuttcap%
\pgfsetroundjoin%
\definecolor{currentfill}{rgb}{0.121569,0.466667,0.705882}%
\pgfsetfillcolor{currentfill}%
\pgfsetfillopacity{0.300000}%
\pgfsetlinewidth{1.003750pt}%
\definecolor{currentstroke}{rgb}{0.121569,0.466667,0.705882}%
\pgfsetstrokecolor{currentstroke}%
\pgfsetstrokeopacity{0.300000}%
\pgfsetdash{}{0pt}%
\pgfpathmoveto{\pgfqpoint{1.558889in}{2.036882in}}%
\pgfpathcurveto{\pgfqpoint{1.567126in}{2.036882in}}{\pgfqpoint{1.575026in}{2.040154in}}{\pgfqpoint{1.580850in}{2.045978in}}%
\pgfpathcurveto{\pgfqpoint{1.586674in}{2.051802in}}{\pgfqpoint{1.589946in}{2.059702in}}{\pgfqpoint{1.589946in}{2.067938in}}%
\pgfpathcurveto{\pgfqpoint{1.589946in}{2.076175in}}{\pgfqpoint{1.586674in}{2.084075in}}{\pgfqpoint{1.580850in}{2.089899in}}%
\pgfpathcurveto{\pgfqpoint{1.575026in}{2.095722in}}{\pgfqpoint{1.567126in}{2.098995in}}{\pgfqpoint{1.558889in}{2.098995in}}%
\pgfpathcurveto{\pgfqpoint{1.550653in}{2.098995in}}{\pgfqpoint{1.542753in}{2.095722in}}{\pgfqpoint{1.536929in}{2.089899in}}%
\pgfpathcurveto{\pgfqpoint{1.531105in}{2.084075in}}{\pgfqpoint{1.527833in}{2.076175in}}{\pgfqpoint{1.527833in}{2.067938in}}%
\pgfpathcurveto{\pgfqpoint{1.527833in}{2.059702in}}{\pgfqpoint{1.531105in}{2.051802in}}{\pgfqpoint{1.536929in}{2.045978in}}%
\pgfpathcurveto{\pgfqpoint{1.542753in}{2.040154in}}{\pgfqpoint{1.550653in}{2.036882in}}{\pgfqpoint{1.558889in}{2.036882in}}%
\pgfpathclose%
\pgfusepath{stroke,fill}%
\end{pgfscope}%
\begin{pgfscope}%
\pgfpathrectangle{\pgfqpoint{0.100000in}{0.212622in}}{\pgfqpoint{3.696000in}{3.696000in}}%
\pgfusepath{clip}%
\pgfsetbuttcap%
\pgfsetroundjoin%
\definecolor{currentfill}{rgb}{0.121569,0.466667,0.705882}%
\pgfsetfillcolor{currentfill}%
\pgfsetfillopacity{0.300000}%
\pgfsetlinewidth{1.003750pt}%
\definecolor{currentstroke}{rgb}{0.121569,0.466667,0.705882}%
\pgfsetstrokecolor{currentstroke}%
\pgfsetstrokeopacity{0.300000}%
\pgfsetdash{}{0pt}%
\pgfpathmoveto{\pgfqpoint{1.558889in}{2.036882in}}%
\pgfpathcurveto{\pgfqpoint{1.567126in}{2.036882in}}{\pgfqpoint{1.575026in}{2.040154in}}{\pgfqpoint{1.580850in}{2.045978in}}%
\pgfpathcurveto{\pgfqpoint{1.586674in}{2.051802in}}{\pgfqpoint{1.589946in}{2.059702in}}{\pgfqpoint{1.589946in}{2.067938in}}%
\pgfpathcurveto{\pgfqpoint{1.589946in}{2.076175in}}{\pgfqpoint{1.586674in}{2.084075in}}{\pgfqpoint{1.580850in}{2.089899in}}%
\pgfpathcurveto{\pgfqpoint{1.575026in}{2.095722in}}{\pgfqpoint{1.567126in}{2.098995in}}{\pgfqpoint{1.558889in}{2.098995in}}%
\pgfpathcurveto{\pgfqpoint{1.550653in}{2.098995in}}{\pgfqpoint{1.542753in}{2.095722in}}{\pgfqpoint{1.536929in}{2.089899in}}%
\pgfpathcurveto{\pgfqpoint{1.531105in}{2.084075in}}{\pgfqpoint{1.527833in}{2.076175in}}{\pgfqpoint{1.527833in}{2.067938in}}%
\pgfpathcurveto{\pgfqpoint{1.527833in}{2.059702in}}{\pgfqpoint{1.531105in}{2.051802in}}{\pgfqpoint{1.536929in}{2.045978in}}%
\pgfpathcurveto{\pgfqpoint{1.542753in}{2.040154in}}{\pgfqpoint{1.550653in}{2.036882in}}{\pgfqpoint{1.558889in}{2.036882in}}%
\pgfpathclose%
\pgfusepath{stroke,fill}%
\end{pgfscope}%
\begin{pgfscope}%
\pgfpathrectangle{\pgfqpoint{0.100000in}{0.212622in}}{\pgfqpoint{3.696000in}{3.696000in}}%
\pgfusepath{clip}%
\pgfsetbuttcap%
\pgfsetroundjoin%
\definecolor{currentfill}{rgb}{0.121569,0.466667,0.705882}%
\pgfsetfillcolor{currentfill}%
\pgfsetfillopacity{0.300000}%
\pgfsetlinewidth{1.003750pt}%
\definecolor{currentstroke}{rgb}{0.121569,0.466667,0.705882}%
\pgfsetstrokecolor{currentstroke}%
\pgfsetstrokeopacity{0.300000}%
\pgfsetdash{}{0pt}%
\pgfpathmoveto{\pgfqpoint{1.558889in}{2.036882in}}%
\pgfpathcurveto{\pgfqpoint{1.567126in}{2.036882in}}{\pgfqpoint{1.575026in}{2.040154in}}{\pgfqpoint{1.580850in}{2.045978in}}%
\pgfpathcurveto{\pgfqpoint{1.586674in}{2.051802in}}{\pgfqpoint{1.589946in}{2.059702in}}{\pgfqpoint{1.589946in}{2.067938in}}%
\pgfpathcurveto{\pgfqpoint{1.589946in}{2.076175in}}{\pgfqpoint{1.586674in}{2.084075in}}{\pgfqpoint{1.580850in}{2.089899in}}%
\pgfpathcurveto{\pgfqpoint{1.575026in}{2.095722in}}{\pgfqpoint{1.567126in}{2.098995in}}{\pgfqpoint{1.558889in}{2.098995in}}%
\pgfpathcurveto{\pgfqpoint{1.550653in}{2.098995in}}{\pgfqpoint{1.542753in}{2.095722in}}{\pgfqpoint{1.536929in}{2.089899in}}%
\pgfpathcurveto{\pgfqpoint{1.531105in}{2.084075in}}{\pgfqpoint{1.527833in}{2.076175in}}{\pgfqpoint{1.527833in}{2.067938in}}%
\pgfpathcurveto{\pgfqpoint{1.527833in}{2.059702in}}{\pgfqpoint{1.531105in}{2.051802in}}{\pgfqpoint{1.536929in}{2.045978in}}%
\pgfpathcurveto{\pgfqpoint{1.542753in}{2.040154in}}{\pgfqpoint{1.550653in}{2.036882in}}{\pgfqpoint{1.558889in}{2.036882in}}%
\pgfpathclose%
\pgfusepath{stroke,fill}%
\end{pgfscope}%
\begin{pgfscope}%
\pgfpathrectangle{\pgfqpoint{0.100000in}{0.212622in}}{\pgfqpoint{3.696000in}{3.696000in}}%
\pgfusepath{clip}%
\pgfsetbuttcap%
\pgfsetroundjoin%
\definecolor{currentfill}{rgb}{0.121569,0.466667,0.705882}%
\pgfsetfillcolor{currentfill}%
\pgfsetfillopacity{0.300000}%
\pgfsetlinewidth{1.003750pt}%
\definecolor{currentstroke}{rgb}{0.121569,0.466667,0.705882}%
\pgfsetstrokecolor{currentstroke}%
\pgfsetstrokeopacity{0.300000}%
\pgfsetdash{}{0pt}%
\pgfpathmoveto{\pgfqpoint{1.558889in}{2.036882in}}%
\pgfpathcurveto{\pgfqpoint{1.567126in}{2.036882in}}{\pgfqpoint{1.575026in}{2.040154in}}{\pgfqpoint{1.580850in}{2.045978in}}%
\pgfpathcurveto{\pgfqpoint{1.586674in}{2.051802in}}{\pgfqpoint{1.589946in}{2.059702in}}{\pgfqpoint{1.589946in}{2.067938in}}%
\pgfpathcurveto{\pgfqpoint{1.589946in}{2.076175in}}{\pgfqpoint{1.586674in}{2.084075in}}{\pgfqpoint{1.580850in}{2.089899in}}%
\pgfpathcurveto{\pgfqpoint{1.575026in}{2.095722in}}{\pgfqpoint{1.567126in}{2.098995in}}{\pgfqpoint{1.558889in}{2.098995in}}%
\pgfpathcurveto{\pgfqpoint{1.550653in}{2.098995in}}{\pgfqpoint{1.542753in}{2.095722in}}{\pgfqpoint{1.536929in}{2.089899in}}%
\pgfpathcurveto{\pgfqpoint{1.531105in}{2.084075in}}{\pgfqpoint{1.527833in}{2.076175in}}{\pgfqpoint{1.527833in}{2.067938in}}%
\pgfpathcurveto{\pgfqpoint{1.527833in}{2.059702in}}{\pgfqpoint{1.531105in}{2.051802in}}{\pgfqpoint{1.536929in}{2.045978in}}%
\pgfpathcurveto{\pgfqpoint{1.542753in}{2.040154in}}{\pgfqpoint{1.550653in}{2.036882in}}{\pgfqpoint{1.558889in}{2.036882in}}%
\pgfpathclose%
\pgfusepath{stroke,fill}%
\end{pgfscope}%
\begin{pgfscope}%
\pgfpathrectangle{\pgfqpoint{0.100000in}{0.212622in}}{\pgfqpoint{3.696000in}{3.696000in}}%
\pgfusepath{clip}%
\pgfsetbuttcap%
\pgfsetroundjoin%
\definecolor{currentfill}{rgb}{0.121569,0.466667,0.705882}%
\pgfsetfillcolor{currentfill}%
\pgfsetfillopacity{0.300000}%
\pgfsetlinewidth{1.003750pt}%
\definecolor{currentstroke}{rgb}{0.121569,0.466667,0.705882}%
\pgfsetstrokecolor{currentstroke}%
\pgfsetstrokeopacity{0.300000}%
\pgfsetdash{}{0pt}%
\pgfpathmoveto{\pgfqpoint{1.558889in}{2.036882in}}%
\pgfpathcurveto{\pgfqpoint{1.567126in}{2.036882in}}{\pgfqpoint{1.575026in}{2.040154in}}{\pgfqpoint{1.580850in}{2.045978in}}%
\pgfpathcurveto{\pgfqpoint{1.586674in}{2.051802in}}{\pgfqpoint{1.589946in}{2.059702in}}{\pgfqpoint{1.589946in}{2.067938in}}%
\pgfpathcurveto{\pgfqpoint{1.589946in}{2.076175in}}{\pgfqpoint{1.586674in}{2.084075in}}{\pgfqpoint{1.580850in}{2.089899in}}%
\pgfpathcurveto{\pgfqpoint{1.575026in}{2.095722in}}{\pgfqpoint{1.567126in}{2.098995in}}{\pgfqpoint{1.558889in}{2.098995in}}%
\pgfpathcurveto{\pgfqpoint{1.550653in}{2.098995in}}{\pgfqpoint{1.542753in}{2.095722in}}{\pgfqpoint{1.536929in}{2.089899in}}%
\pgfpathcurveto{\pgfqpoint{1.531105in}{2.084075in}}{\pgfqpoint{1.527833in}{2.076175in}}{\pgfqpoint{1.527833in}{2.067938in}}%
\pgfpathcurveto{\pgfqpoint{1.527833in}{2.059702in}}{\pgfqpoint{1.531105in}{2.051802in}}{\pgfqpoint{1.536929in}{2.045978in}}%
\pgfpathcurveto{\pgfqpoint{1.542753in}{2.040154in}}{\pgfqpoint{1.550653in}{2.036882in}}{\pgfqpoint{1.558889in}{2.036882in}}%
\pgfpathclose%
\pgfusepath{stroke,fill}%
\end{pgfscope}%
\begin{pgfscope}%
\pgfpathrectangle{\pgfqpoint{0.100000in}{0.212622in}}{\pgfqpoint{3.696000in}{3.696000in}}%
\pgfusepath{clip}%
\pgfsetbuttcap%
\pgfsetroundjoin%
\definecolor{currentfill}{rgb}{0.121569,0.466667,0.705882}%
\pgfsetfillcolor{currentfill}%
\pgfsetfillopacity{0.300000}%
\pgfsetlinewidth{1.003750pt}%
\definecolor{currentstroke}{rgb}{0.121569,0.466667,0.705882}%
\pgfsetstrokecolor{currentstroke}%
\pgfsetstrokeopacity{0.300000}%
\pgfsetdash{}{0pt}%
\pgfpathmoveto{\pgfqpoint{1.558889in}{2.036882in}}%
\pgfpathcurveto{\pgfqpoint{1.567126in}{2.036882in}}{\pgfqpoint{1.575026in}{2.040154in}}{\pgfqpoint{1.580850in}{2.045978in}}%
\pgfpathcurveto{\pgfqpoint{1.586674in}{2.051802in}}{\pgfqpoint{1.589946in}{2.059702in}}{\pgfqpoint{1.589946in}{2.067938in}}%
\pgfpathcurveto{\pgfqpoint{1.589946in}{2.076175in}}{\pgfqpoint{1.586674in}{2.084075in}}{\pgfqpoint{1.580850in}{2.089899in}}%
\pgfpathcurveto{\pgfqpoint{1.575026in}{2.095722in}}{\pgfqpoint{1.567126in}{2.098995in}}{\pgfqpoint{1.558889in}{2.098995in}}%
\pgfpathcurveto{\pgfqpoint{1.550653in}{2.098995in}}{\pgfqpoint{1.542753in}{2.095722in}}{\pgfqpoint{1.536929in}{2.089899in}}%
\pgfpathcurveto{\pgfqpoint{1.531105in}{2.084075in}}{\pgfqpoint{1.527833in}{2.076175in}}{\pgfqpoint{1.527833in}{2.067938in}}%
\pgfpathcurveto{\pgfqpoint{1.527833in}{2.059702in}}{\pgfqpoint{1.531105in}{2.051802in}}{\pgfqpoint{1.536929in}{2.045978in}}%
\pgfpathcurveto{\pgfqpoint{1.542753in}{2.040154in}}{\pgfqpoint{1.550653in}{2.036882in}}{\pgfqpoint{1.558889in}{2.036882in}}%
\pgfpathclose%
\pgfusepath{stroke,fill}%
\end{pgfscope}%
\begin{pgfscope}%
\pgfpathrectangle{\pgfqpoint{0.100000in}{0.212622in}}{\pgfqpoint{3.696000in}{3.696000in}}%
\pgfusepath{clip}%
\pgfsetbuttcap%
\pgfsetroundjoin%
\definecolor{currentfill}{rgb}{0.121569,0.466667,0.705882}%
\pgfsetfillcolor{currentfill}%
\pgfsetfillopacity{0.300000}%
\pgfsetlinewidth{1.003750pt}%
\definecolor{currentstroke}{rgb}{0.121569,0.466667,0.705882}%
\pgfsetstrokecolor{currentstroke}%
\pgfsetstrokeopacity{0.300000}%
\pgfsetdash{}{0pt}%
\pgfpathmoveto{\pgfqpoint{1.558889in}{2.036882in}}%
\pgfpathcurveto{\pgfqpoint{1.567126in}{2.036882in}}{\pgfqpoint{1.575026in}{2.040154in}}{\pgfqpoint{1.580850in}{2.045978in}}%
\pgfpathcurveto{\pgfqpoint{1.586674in}{2.051802in}}{\pgfqpoint{1.589946in}{2.059702in}}{\pgfqpoint{1.589946in}{2.067938in}}%
\pgfpathcurveto{\pgfqpoint{1.589946in}{2.076175in}}{\pgfqpoint{1.586674in}{2.084075in}}{\pgfqpoint{1.580850in}{2.089899in}}%
\pgfpathcurveto{\pgfqpoint{1.575026in}{2.095722in}}{\pgfqpoint{1.567126in}{2.098995in}}{\pgfqpoint{1.558889in}{2.098995in}}%
\pgfpathcurveto{\pgfqpoint{1.550653in}{2.098995in}}{\pgfqpoint{1.542753in}{2.095722in}}{\pgfqpoint{1.536929in}{2.089899in}}%
\pgfpathcurveto{\pgfqpoint{1.531105in}{2.084075in}}{\pgfqpoint{1.527833in}{2.076175in}}{\pgfqpoint{1.527833in}{2.067938in}}%
\pgfpathcurveto{\pgfqpoint{1.527833in}{2.059702in}}{\pgfqpoint{1.531105in}{2.051802in}}{\pgfqpoint{1.536929in}{2.045978in}}%
\pgfpathcurveto{\pgfqpoint{1.542753in}{2.040154in}}{\pgfqpoint{1.550653in}{2.036882in}}{\pgfqpoint{1.558889in}{2.036882in}}%
\pgfpathclose%
\pgfusepath{stroke,fill}%
\end{pgfscope}%
\begin{pgfscope}%
\pgfpathrectangle{\pgfqpoint{0.100000in}{0.212622in}}{\pgfqpoint{3.696000in}{3.696000in}}%
\pgfusepath{clip}%
\pgfsetbuttcap%
\pgfsetroundjoin%
\definecolor{currentfill}{rgb}{0.121569,0.466667,0.705882}%
\pgfsetfillcolor{currentfill}%
\pgfsetfillopacity{0.300000}%
\pgfsetlinewidth{1.003750pt}%
\definecolor{currentstroke}{rgb}{0.121569,0.466667,0.705882}%
\pgfsetstrokecolor{currentstroke}%
\pgfsetstrokeopacity{0.300000}%
\pgfsetdash{}{0pt}%
\pgfpathmoveto{\pgfqpoint{1.558889in}{2.036882in}}%
\pgfpathcurveto{\pgfqpoint{1.567126in}{2.036882in}}{\pgfqpoint{1.575026in}{2.040154in}}{\pgfqpoint{1.580850in}{2.045978in}}%
\pgfpathcurveto{\pgfqpoint{1.586674in}{2.051802in}}{\pgfqpoint{1.589946in}{2.059702in}}{\pgfqpoint{1.589946in}{2.067938in}}%
\pgfpathcurveto{\pgfqpoint{1.589946in}{2.076175in}}{\pgfqpoint{1.586674in}{2.084075in}}{\pgfqpoint{1.580850in}{2.089899in}}%
\pgfpathcurveto{\pgfqpoint{1.575026in}{2.095722in}}{\pgfqpoint{1.567126in}{2.098995in}}{\pgfqpoint{1.558889in}{2.098995in}}%
\pgfpathcurveto{\pgfqpoint{1.550653in}{2.098995in}}{\pgfqpoint{1.542753in}{2.095722in}}{\pgfqpoint{1.536929in}{2.089899in}}%
\pgfpathcurveto{\pgfqpoint{1.531105in}{2.084075in}}{\pgfqpoint{1.527833in}{2.076175in}}{\pgfqpoint{1.527833in}{2.067938in}}%
\pgfpathcurveto{\pgfqpoint{1.527833in}{2.059702in}}{\pgfqpoint{1.531105in}{2.051802in}}{\pgfqpoint{1.536929in}{2.045978in}}%
\pgfpathcurveto{\pgfqpoint{1.542753in}{2.040154in}}{\pgfqpoint{1.550653in}{2.036882in}}{\pgfqpoint{1.558889in}{2.036882in}}%
\pgfpathclose%
\pgfusepath{stroke,fill}%
\end{pgfscope}%
\begin{pgfscope}%
\pgfpathrectangle{\pgfqpoint{0.100000in}{0.212622in}}{\pgfqpoint{3.696000in}{3.696000in}}%
\pgfusepath{clip}%
\pgfsetbuttcap%
\pgfsetroundjoin%
\definecolor{currentfill}{rgb}{0.121569,0.466667,0.705882}%
\pgfsetfillcolor{currentfill}%
\pgfsetfillopacity{0.300000}%
\pgfsetlinewidth{1.003750pt}%
\definecolor{currentstroke}{rgb}{0.121569,0.466667,0.705882}%
\pgfsetstrokecolor{currentstroke}%
\pgfsetstrokeopacity{0.300000}%
\pgfsetdash{}{0pt}%
\pgfpathmoveto{\pgfqpoint{1.558889in}{2.036882in}}%
\pgfpathcurveto{\pgfqpoint{1.567126in}{2.036882in}}{\pgfqpoint{1.575026in}{2.040154in}}{\pgfqpoint{1.580850in}{2.045978in}}%
\pgfpathcurveto{\pgfqpoint{1.586674in}{2.051802in}}{\pgfqpoint{1.589946in}{2.059702in}}{\pgfqpoint{1.589946in}{2.067938in}}%
\pgfpathcurveto{\pgfqpoint{1.589946in}{2.076175in}}{\pgfqpoint{1.586674in}{2.084075in}}{\pgfqpoint{1.580850in}{2.089899in}}%
\pgfpathcurveto{\pgfqpoint{1.575026in}{2.095722in}}{\pgfqpoint{1.567126in}{2.098995in}}{\pgfqpoint{1.558889in}{2.098995in}}%
\pgfpathcurveto{\pgfqpoint{1.550653in}{2.098995in}}{\pgfqpoint{1.542753in}{2.095722in}}{\pgfqpoint{1.536929in}{2.089899in}}%
\pgfpathcurveto{\pgfqpoint{1.531105in}{2.084075in}}{\pgfqpoint{1.527833in}{2.076175in}}{\pgfqpoint{1.527833in}{2.067938in}}%
\pgfpathcurveto{\pgfqpoint{1.527833in}{2.059702in}}{\pgfqpoint{1.531105in}{2.051802in}}{\pgfqpoint{1.536929in}{2.045978in}}%
\pgfpathcurveto{\pgfqpoint{1.542753in}{2.040154in}}{\pgfqpoint{1.550653in}{2.036882in}}{\pgfqpoint{1.558889in}{2.036882in}}%
\pgfpathclose%
\pgfusepath{stroke,fill}%
\end{pgfscope}%
\begin{pgfscope}%
\pgfpathrectangle{\pgfqpoint{0.100000in}{0.212622in}}{\pgfqpoint{3.696000in}{3.696000in}}%
\pgfusepath{clip}%
\pgfsetbuttcap%
\pgfsetroundjoin%
\definecolor{currentfill}{rgb}{0.121569,0.466667,0.705882}%
\pgfsetfillcolor{currentfill}%
\pgfsetfillopacity{0.300000}%
\pgfsetlinewidth{1.003750pt}%
\definecolor{currentstroke}{rgb}{0.121569,0.466667,0.705882}%
\pgfsetstrokecolor{currentstroke}%
\pgfsetstrokeopacity{0.300000}%
\pgfsetdash{}{0pt}%
\pgfpathmoveto{\pgfqpoint{1.558889in}{2.036882in}}%
\pgfpathcurveto{\pgfqpoint{1.567126in}{2.036882in}}{\pgfqpoint{1.575026in}{2.040154in}}{\pgfqpoint{1.580850in}{2.045978in}}%
\pgfpathcurveto{\pgfqpoint{1.586674in}{2.051802in}}{\pgfqpoint{1.589946in}{2.059702in}}{\pgfqpoint{1.589946in}{2.067938in}}%
\pgfpathcurveto{\pgfqpoint{1.589946in}{2.076175in}}{\pgfqpoint{1.586674in}{2.084075in}}{\pgfqpoint{1.580850in}{2.089899in}}%
\pgfpathcurveto{\pgfqpoint{1.575026in}{2.095722in}}{\pgfqpoint{1.567126in}{2.098995in}}{\pgfqpoint{1.558889in}{2.098995in}}%
\pgfpathcurveto{\pgfqpoint{1.550653in}{2.098995in}}{\pgfqpoint{1.542753in}{2.095722in}}{\pgfqpoint{1.536929in}{2.089899in}}%
\pgfpathcurveto{\pgfqpoint{1.531105in}{2.084075in}}{\pgfqpoint{1.527833in}{2.076175in}}{\pgfqpoint{1.527833in}{2.067938in}}%
\pgfpathcurveto{\pgfqpoint{1.527833in}{2.059702in}}{\pgfqpoint{1.531105in}{2.051802in}}{\pgfqpoint{1.536929in}{2.045978in}}%
\pgfpathcurveto{\pgfqpoint{1.542753in}{2.040154in}}{\pgfqpoint{1.550653in}{2.036882in}}{\pgfqpoint{1.558889in}{2.036882in}}%
\pgfpathclose%
\pgfusepath{stroke,fill}%
\end{pgfscope}%
\begin{pgfscope}%
\pgfpathrectangle{\pgfqpoint{0.100000in}{0.212622in}}{\pgfqpoint{3.696000in}{3.696000in}}%
\pgfusepath{clip}%
\pgfsetbuttcap%
\pgfsetroundjoin%
\definecolor{currentfill}{rgb}{0.121569,0.466667,0.705882}%
\pgfsetfillcolor{currentfill}%
\pgfsetfillopacity{0.300000}%
\pgfsetlinewidth{1.003750pt}%
\definecolor{currentstroke}{rgb}{0.121569,0.466667,0.705882}%
\pgfsetstrokecolor{currentstroke}%
\pgfsetstrokeopacity{0.300000}%
\pgfsetdash{}{0pt}%
\pgfpathmoveto{\pgfqpoint{1.558889in}{2.036882in}}%
\pgfpathcurveto{\pgfqpoint{1.567126in}{2.036882in}}{\pgfqpoint{1.575026in}{2.040154in}}{\pgfqpoint{1.580850in}{2.045978in}}%
\pgfpathcurveto{\pgfqpoint{1.586674in}{2.051802in}}{\pgfqpoint{1.589946in}{2.059702in}}{\pgfqpoint{1.589946in}{2.067938in}}%
\pgfpathcurveto{\pgfqpoint{1.589946in}{2.076175in}}{\pgfqpoint{1.586674in}{2.084075in}}{\pgfqpoint{1.580850in}{2.089899in}}%
\pgfpathcurveto{\pgfqpoint{1.575026in}{2.095722in}}{\pgfqpoint{1.567126in}{2.098995in}}{\pgfqpoint{1.558889in}{2.098995in}}%
\pgfpathcurveto{\pgfqpoint{1.550653in}{2.098995in}}{\pgfqpoint{1.542753in}{2.095722in}}{\pgfqpoint{1.536929in}{2.089899in}}%
\pgfpathcurveto{\pgfqpoint{1.531105in}{2.084075in}}{\pgfqpoint{1.527833in}{2.076175in}}{\pgfqpoint{1.527833in}{2.067938in}}%
\pgfpathcurveto{\pgfqpoint{1.527833in}{2.059702in}}{\pgfqpoint{1.531105in}{2.051802in}}{\pgfqpoint{1.536929in}{2.045978in}}%
\pgfpathcurveto{\pgfqpoint{1.542753in}{2.040154in}}{\pgfqpoint{1.550653in}{2.036882in}}{\pgfqpoint{1.558889in}{2.036882in}}%
\pgfpathclose%
\pgfusepath{stroke,fill}%
\end{pgfscope}%
\begin{pgfscope}%
\pgfpathrectangle{\pgfqpoint{0.100000in}{0.212622in}}{\pgfqpoint{3.696000in}{3.696000in}}%
\pgfusepath{clip}%
\pgfsetbuttcap%
\pgfsetroundjoin%
\definecolor{currentfill}{rgb}{0.121569,0.466667,0.705882}%
\pgfsetfillcolor{currentfill}%
\pgfsetfillopacity{0.300000}%
\pgfsetlinewidth{1.003750pt}%
\definecolor{currentstroke}{rgb}{0.121569,0.466667,0.705882}%
\pgfsetstrokecolor{currentstroke}%
\pgfsetstrokeopacity{0.300000}%
\pgfsetdash{}{0pt}%
\pgfpathmoveto{\pgfqpoint{1.558889in}{2.036882in}}%
\pgfpathcurveto{\pgfqpoint{1.567126in}{2.036882in}}{\pgfqpoint{1.575026in}{2.040154in}}{\pgfqpoint{1.580850in}{2.045978in}}%
\pgfpathcurveto{\pgfqpoint{1.586674in}{2.051802in}}{\pgfqpoint{1.589946in}{2.059702in}}{\pgfqpoint{1.589946in}{2.067938in}}%
\pgfpathcurveto{\pgfqpoint{1.589946in}{2.076175in}}{\pgfqpoint{1.586674in}{2.084075in}}{\pgfqpoint{1.580850in}{2.089899in}}%
\pgfpathcurveto{\pgfqpoint{1.575026in}{2.095722in}}{\pgfqpoint{1.567126in}{2.098995in}}{\pgfqpoint{1.558889in}{2.098995in}}%
\pgfpathcurveto{\pgfqpoint{1.550653in}{2.098995in}}{\pgfqpoint{1.542753in}{2.095722in}}{\pgfqpoint{1.536929in}{2.089899in}}%
\pgfpathcurveto{\pgfqpoint{1.531105in}{2.084075in}}{\pgfqpoint{1.527833in}{2.076175in}}{\pgfqpoint{1.527833in}{2.067938in}}%
\pgfpathcurveto{\pgfqpoint{1.527833in}{2.059702in}}{\pgfqpoint{1.531105in}{2.051802in}}{\pgfqpoint{1.536929in}{2.045978in}}%
\pgfpathcurveto{\pgfqpoint{1.542753in}{2.040154in}}{\pgfqpoint{1.550653in}{2.036882in}}{\pgfqpoint{1.558889in}{2.036882in}}%
\pgfpathclose%
\pgfusepath{stroke,fill}%
\end{pgfscope}%
\begin{pgfscope}%
\pgfpathrectangle{\pgfqpoint{0.100000in}{0.212622in}}{\pgfqpoint{3.696000in}{3.696000in}}%
\pgfusepath{clip}%
\pgfsetbuttcap%
\pgfsetroundjoin%
\definecolor{currentfill}{rgb}{0.121569,0.466667,0.705882}%
\pgfsetfillcolor{currentfill}%
\pgfsetfillopacity{0.300000}%
\pgfsetlinewidth{1.003750pt}%
\definecolor{currentstroke}{rgb}{0.121569,0.466667,0.705882}%
\pgfsetstrokecolor{currentstroke}%
\pgfsetstrokeopacity{0.300000}%
\pgfsetdash{}{0pt}%
\pgfpathmoveto{\pgfqpoint{1.558889in}{2.036882in}}%
\pgfpathcurveto{\pgfqpoint{1.567126in}{2.036882in}}{\pgfqpoint{1.575026in}{2.040154in}}{\pgfqpoint{1.580850in}{2.045978in}}%
\pgfpathcurveto{\pgfqpoint{1.586674in}{2.051802in}}{\pgfqpoint{1.589946in}{2.059702in}}{\pgfqpoint{1.589946in}{2.067938in}}%
\pgfpathcurveto{\pgfqpoint{1.589946in}{2.076175in}}{\pgfqpoint{1.586674in}{2.084075in}}{\pgfqpoint{1.580850in}{2.089899in}}%
\pgfpathcurveto{\pgfqpoint{1.575026in}{2.095722in}}{\pgfqpoint{1.567126in}{2.098995in}}{\pgfqpoint{1.558889in}{2.098995in}}%
\pgfpathcurveto{\pgfqpoint{1.550653in}{2.098995in}}{\pgfqpoint{1.542753in}{2.095722in}}{\pgfqpoint{1.536929in}{2.089899in}}%
\pgfpathcurveto{\pgfqpoint{1.531105in}{2.084075in}}{\pgfqpoint{1.527833in}{2.076175in}}{\pgfqpoint{1.527833in}{2.067938in}}%
\pgfpathcurveto{\pgfqpoint{1.527833in}{2.059702in}}{\pgfqpoint{1.531105in}{2.051802in}}{\pgfqpoint{1.536929in}{2.045978in}}%
\pgfpathcurveto{\pgfqpoint{1.542753in}{2.040154in}}{\pgfqpoint{1.550653in}{2.036882in}}{\pgfqpoint{1.558889in}{2.036882in}}%
\pgfpathclose%
\pgfusepath{stroke,fill}%
\end{pgfscope}%
\begin{pgfscope}%
\pgfpathrectangle{\pgfqpoint{0.100000in}{0.212622in}}{\pgfqpoint{3.696000in}{3.696000in}}%
\pgfusepath{clip}%
\pgfsetbuttcap%
\pgfsetroundjoin%
\definecolor{currentfill}{rgb}{0.121569,0.466667,0.705882}%
\pgfsetfillcolor{currentfill}%
\pgfsetfillopacity{0.300000}%
\pgfsetlinewidth{1.003750pt}%
\definecolor{currentstroke}{rgb}{0.121569,0.466667,0.705882}%
\pgfsetstrokecolor{currentstroke}%
\pgfsetstrokeopacity{0.300000}%
\pgfsetdash{}{0pt}%
\pgfpathmoveto{\pgfqpoint{1.558889in}{2.036882in}}%
\pgfpathcurveto{\pgfqpoint{1.567126in}{2.036882in}}{\pgfqpoint{1.575026in}{2.040154in}}{\pgfqpoint{1.580850in}{2.045978in}}%
\pgfpathcurveto{\pgfqpoint{1.586674in}{2.051802in}}{\pgfqpoint{1.589946in}{2.059702in}}{\pgfqpoint{1.589946in}{2.067938in}}%
\pgfpathcurveto{\pgfqpoint{1.589946in}{2.076175in}}{\pgfqpoint{1.586674in}{2.084075in}}{\pgfqpoint{1.580850in}{2.089899in}}%
\pgfpathcurveto{\pgfqpoint{1.575026in}{2.095722in}}{\pgfqpoint{1.567126in}{2.098995in}}{\pgfqpoint{1.558889in}{2.098995in}}%
\pgfpathcurveto{\pgfqpoint{1.550653in}{2.098995in}}{\pgfqpoint{1.542753in}{2.095722in}}{\pgfqpoint{1.536929in}{2.089899in}}%
\pgfpathcurveto{\pgfqpoint{1.531105in}{2.084075in}}{\pgfqpoint{1.527833in}{2.076175in}}{\pgfqpoint{1.527833in}{2.067938in}}%
\pgfpathcurveto{\pgfqpoint{1.527833in}{2.059702in}}{\pgfqpoint{1.531105in}{2.051802in}}{\pgfqpoint{1.536929in}{2.045978in}}%
\pgfpathcurveto{\pgfqpoint{1.542753in}{2.040154in}}{\pgfqpoint{1.550653in}{2.036882in}}{\pgfqpoint{1.558889in}{2.036882in}}%
\pgfpathclose%
\pgfusepath{stroke,fill}%
\end{pgfscope}%
\begin{pgfscope}%
\pgfpathrectangle{\pgfqpoint{0.100000in}{0.212622in}}{\pgfqpoint{3.696000in}{3.696000in}}%
\pgfusepath{clip}%
\pgfsetbuttcap%
\pgfsetroundjoin%
\definecolor{currentfill}{rgb}{0.121569,0.466667,0.705882}%
\pgfsetfillcolor{currentfill}%
\pgfsetfillopacity{0.300000}%
\pgfsetlinewidth{1.003750pt}%
\definecolor{currentstroke}{rgb}{0.121569,0.466667,0.705882}%
\pgfsetstrokecolor{currentstroke}%
\pgfsetstrokeopacity{0.300000}%
\pgfsetdash{}{0pt}%
\pgfpathmoveto{\pgfqpoint{1.558889in}{2.036882in}}%
\pgfpathcurveto{\pgfqpoint{1.567126in}{2.036882in}}{\pgfqpoint{1.575026in}{2.040154in}}{\pgfqpoint{1.580850in}{2.045978in}}%
\pgfpathcurveto{\pgfqpoint{1.586674in}{2.051802in}}{\pgfqpoint{1.589946in}{2.059702in}}{\pgfqpoint{1.589946in}{2.067938in}}%
\pgfpathcurveto{\pgfqpoint{1.589946in}{2.076175in}}{\pgfqpoint{1.586674in}{2.084075in}}{\pgfqpoint{1.580850in}{2.089899in}}%
\pgfpathcurveto{\pgfqpoint{1.575026in}{2.095722in}}{\pgfqpoint{1.567126in}{2.098995in}}{\pgfqpoint{1.558889in}{2.098995in}}%
\pgfpathcurveto{\pgfqpoint{1.550653in}{2.098995in}}{\pgfqpoint{1.542753in}{2.095722in}}{\pgfqpoint{1.536929in}{2.089899in}}%
\pgfpathcurveto{\pgfqpoint{1.531105in}{2.084075in}}{\pgfqpoint{1.527833in}{2.076175in}}{\pgfqpoint{1.527833in}{2.067938in}}%
\pgfpathcurveto{\pgfqpoint{1.527833in}{2.059702in}}{\pgfqpoint{1.531105in}{2.051802in}}{\pgfqpoint{1.536929in}{2.045978in}}%
\pgfpathcurveto{\pgfqpoint{1.542753in}{2.040154in}}{\pgfqpoint{1.550653in}{2.036882in}}{\pgfqpoint{1.558889in}{2.036882in}}%
\pgfpathclose%
\pgfusepath{stroke,fill}%
\end{pgfscope}%
\begin{pgfscope}%
\pgfpathrectangle{\pgfqpoint{0.100000in}{0.212622in}}{\pgfqpoint{3.696000in}{3.696000in}}%
\pgfusepath{clip}%
\pgfsetbuttcap%
\pgfsetroundjoin%
\definecolor{currentfill}{rgb}{0.121569,0.466667,0.705882}%
\pgfsetfillcolor{currentfill}%
\pgfsetfillopacity{0.300000}%
\pgfsetlinewidth{1.003750pt}%
\definecolor{currentstroke}{rgb}{0.121569,0.466667,0.705882}%
\pgfsetstrokecolor{currentstroke}%
\pgfsetstrokeopacity{0.300000}%
\pgfsetdash{}{0pt}%
\pgfpathmoveto{\pgfqpoint{1.558889in}{2.036882in}}%
\pgfpathcurveto{\pgfqpoint{1.567126in}{2.036882in}}{\pgfqpoint{1.575026in}{2.040154in}}{\pgfqpoint{1.580850in}{2.045978in}}%
\pgfpathcurveto{\pgfqpoint{1.586674in}{2.051802in}}{\pgfqpoint{1.589946in}{2.059702in}}{\pgfqpoint{1.589946in}{2.067938in}}%
\pgfpathcurveto{\pgfqpoint{1.589946in}{2.076175in}}{\pgfqpoint{1.586674in}{2.084075in}}{\pgfqpoint{1.580850in}{2.089899in}}%
\pgfpathcurveto{\pgfqpoint{1.575026in}{2.095722in}}{\pgfqpoint{1.567126in}{2.098995in}}{\pgfqpoint{1.558889in}{2.098995in}}%
\pgfpathcurveto{\pgfqpoint{1.550653in}{2.098995in}}{\pgfqpoint{1.542753in}{2.095722in}}{\pgfqpoint{1.536929in}{2.089899in}}%
\pgfpathcurveto{\pgfqpoint{1.531105in}{2.084075in}}{\pgfqpoint{1.527833in}{2.076175in}}{\pgfqpoint{1.527833in}{2.067938in}}%
\pgfpathcurveto{\pgfqpoint{1.527833in}{2.059702in}}{\pgfqpoint{1.531105in}{2.051802in}}{\pgfqpoint{1.536929in}{2.045978in}}%
\pgfpathcurveto{\pgfqpoint{1.542753in}{2.040154in}}{\pgfqpoint{1.550653in}{2.036882in}}{\pgfqpoint{1.558889in}{2.036882in}}%
\pgfpathclose%
\pgfusepath{stroke,fill}%
\end{pgfscope}%
\begin{pgfscope}%
\pgfpathrectangle{\pgfqpoint{0.100000in}{0.212622in}}{\pgfqpoint{3.696000in}{3.696000in}}%
\pgfusepath{clip}%
\pgfsetbuttcap%
\pgfsetroundjoin%
\definecolor{currentfill}{rgb}{0.121569,0.466667,0.705882}%
\pgfsetfillcolor{currentfill}%
\pgfsetfillopacity{0.300000}%
\pgfsetlinewidth{1.003750pt}%
\definecolor{currentstroke}{rgb}{0.121569,0.466667,0.705882}%
\pgfsetstrokecolor{currentstroke}%
\pgfsetstrokeopacity{0.300000}%
\pgfsetdash{}{0pt}%
\pgfpathmoveto{\pgfqpoint{1.558889in}{2.036882in}}%
\pgfpathcurveto{\pgfqpoint{1.567126in}{2.036882in}}{\pgfqpoint{1.575026in}{2.040154in}}{\pgfqpoint{1.580850in}{2.045978in}}%
\pgfpathcurveto{\pgfqpoint{1.586674in}{2.051802in}}{\pgfqpoint{1.589946in}{2.059702in}}{\pgfqpoint{1.589946in}{2.067938in}}%
\pgfpathcurveto{\pgfqpoint{1.589946in}{2.076175in}}{\pgfqpoint{1.586674in}{2.084075in}}{\pgfqpoint{1.580850in}{2.089899in}}%
\pgfpathcurveto{\pgfqpoint{1.575026in}{2.095722in}}{\pgfqpoint{1.567126in}{2.098995in}}{\pgfqpoint{1.558889in}{2.098995in}}%
\pgfpathcurveto{\pgfqpoint{1.550653in}{2.098995in}}{\pgfqpoint{1.542753in}{2.095722in}}{\pgfqpoint{1.536929in}{2.089899in}}%
\pgfpathcurveto{\pgfqpoint{1.531105in}{2.084075in}}{\pgfqpoint{1.527833in}{2.076175in}}{\pgfqpoint{1.527833in}{2.067938in}}%
\pgfpathcurveto{\pgfqpoint{1.527833in}{2.059702in}}{\pgfqpoint{1.531105in}{2.051802in}}{\pgfqpoint{1.536929in}{2.045978in}}%
\pgfpathcurveto{\pgfqpoint{1.542753in}{2.040154in}}{\pgfqpoint{1.550653in}{2.036882in}}{\pgfqpoint{1.558889in}{2.036882in}}%
\pgfpathclose%
\pgfusepath{stroke,fill}%
\end{pgfscope}%
\begin{pgfscope}%
\pgfpathrectangle{\pgfqpoint{0.100000in}{0.212622in}}{\pgfqpoint{3.696000in}{3.696000in}}%
\pgfusepath{clip}%
\pgfsetbuttcap%
\pgfsetroundjoin%
\definecolor{currentfill}{rgb}{0.121569,0.466667,0.705882}%
\pgfsetfillcolor{currentfill}%
\pgfsetfillopacity{0.300000}%
\pgfsetlinewidth{1.003750pt}%
\definecolor{currentstroke}{rgb}{0.121569,0.466667,0.705882}%
\pgfsetstrokecolor{currentstroke}%
\pgfsetstrokeopacity{0.300000}%
\pgfsetdash{}{0pt}%
\pgfpathmoveto{\pgfqpoint{1.558889in}{2.036882in}}%
\pgfpathcurveto{\pgfqpoint{1.567126in}{2.036882in}}{\pgfqpoint{1.575026in}{2.040154in}}{\pgfqpoint{1.580850in}{2.045978in}}%
\pgfpathcurveto{\pgfqpoint{1.586674in}{2.051802in}}{\pgfqpoint{1.589946in}{2.059702in}}{\pgfqpoint{1.589946in}{2.067938in}}%
\pgfpathcurveto{\pgfqpoint{1.589946in}{2.076175in}}{\pgfqpoint{1.586674in}{2.084075in}}{\pgfqpoint{1.580850in}{2.089899in}}%
\pgfpathcurveto{\pgfqpoint{1.575026in}{2.095722in}}{\pgfqpoint{1.567126in}{2.098995in}}{\pgfqpoint{1.558889in}{2.098995in}}%
\pgfpathcurveto{\pgfqpoint{1.550653in}{2.098995in}}{\pgfqpoint{1.542753in}{2.095722in}}{\pgfqpoint{1.536929in}{2.089899in}}%
\pgfpathcurveto{\pgfqpoint{1.531105in}{2.084075in}}{\pgfqpoint{1.527833in}{2.076175in}}{\pgfqpoint{1.527833in}{2.067938in}}%
\pgfpathcurveto{\pgfqpoint{1.527833in}{2.059702in}}{\pgfqpoint{1.531105in}{2.051802in}}{\pgfqpoint{1.536929in}{2.045978in}}%
\pgfpathcurveto{\pgfqpoint{1.542753in}{2.040154in}}{\pgfqpoint{1.550653in}{2.036882in}}{\pgfqpoint{1.558889in}{2.036882in}}%
\pgfpathclose%
\pgfusepath{stroke,fill}%
\end{pgfscope}%
\begin{pgfscope}%
\pgfpathrectangle{\pgfqpoint{0.100000in}{0.212622in}}{\pgfqpoint{3.696000in}{3.696000in}}%
\pgfusepath{clip}%
\pgfsetbuttcap%
\pgfsetroundjoin%
\definecolor{currentfill}{rgb}{0.121569,0.466667,0.705882}%
\pgfsetfillcolor{currentfill}%
\pgfsetfillopacity{0.300000}%
\pgfsetlinewidth{1.003750pt}%
\definecolor{currentstroke}{rgb}{0.121569,0.466667,0.705882}%
\pgfsetstrokecolor{currentstroke}%
\pgfsetstrokeopacity{0.300000}%
\pgfsetdash{}{0pt}%
\pgfpathmoveto{\pgfqpoint{1.558889in}{2.036882in}}%
\pgfpathcurveto{\pgfqpoint{1.567126in}{2.036882in}}{\pgfqpoint{1.575026in}{2.040154in}}{\pgfqpoint{1.580850in}{2.045978in}}%
\pgfpathcurveto{\pgfqpoint{1.586674in}{2.051802in}}{\pgfqpoint{1.589946in}{2.059702in}}{\pgfqpoint{1.589946in}{2.067938in}}%
\pgfpathcurveto{\pgfqpoint{1.589946in}{2.076175in}}{\pgfqpoint{1.586674in}{2.084075in}}{\pgfqpoint{1.580850in}{2.089899in}}%
\pgfpathcurveto{\pgfqpoint{1.575026in}{2.095722in}}{\pgfqpoint{1.567126in}{2.098995in}}{\pgfqpoint{1.558889in}{2.098995in}}%
\pgfpathcurveto{\pgfqpoint{1.550653in}{2.098995in}}{\pgfqpoint{1.542753in}{2.095722in}}{\pgfqpoint{1.536929in}{2.089899in}}%
\pgfpathcurveto{\pgfqpoint{1.531105in}{2.084075in}}{\pgfqpoint{1.527833in}{2.076175in}}{\pgfqpoint{1.527833in}{2.067938in}}%
\pgfpathcurveto{\pgfqpoint{1.527833in}{2.059702in}}{\pgfqpoint{1.531105in}{2.051802in}}{\pgfqpoint{1.536929in}{2.045978in}}%
\pgfpathcurveto{\pgfqpoint{1.542753in}{2.040154in}}{\pgfqpoint{1.550653in}{2.036882in}}{\pgfqpoint{1.558889in}{2.036882in}}%
\pgfpathclose%
\pgfusepath{stroke,fill}%
\end{pgfscope}%
\begin{pgfscope}%
\pgfpathrectangle{\pgfqpoint{0.100000in}{0.212622in}}{\pgfqpoint{3.696000in}{3.696000in}}%
\pgfusepath{clip}%
\pgfsetbuttcap%
\pgfsetroundjoin%
\definecolor{currentfill}{rgb}{0.121569,0.466667,0.705882}%
\pgfsetfillcolor{currentfill}%
\pgfsetfillopacity{0.300000}%
\pgfsetlinewidth{1.003750pt}%
\definecolor{currentstroke}{rgb}{0.121569,0.466667,0.705882}%
\pgfsetstrokecolor{currentstroke}%
\pgfsetstrokeopacity{0.300000}%
\pgfsetdash{}{0pt}%
\pgfpathmoveto{\pgfqpoint{1.558889in}{2.036882in}}%
\pgfpathcurveto{\pgfqpoint{1.567126in}{2.036882in}}{\pgfqpoint{1.575026in}{2.040154in}}{\pgfqpoint{1.580850in}{2.045978in}}%
\pgfpathcurveto{\pgfqpoint{1.586674in}{2.051802in}}{\pgfqpoint{1.589946in}{2.059702in}}{\pgfqpoint{1.589946in}{2.067938in}}%
\pgfpathcurveto{\pgfqpoint{1.589946in}{2.076175in}}{\pgfqpoint{1.586674in}{2.084075in}}{\pgfqpoint{1.580850in}{2.089899in}}%
\pgfpathcurveto{\pgfqpoint{1.575026in}{2.095722in}}{\pgfqpoint{1.567126in}{2.098995in}}{\pgfqpoint{1.558889in}{2.098995in}}%
\pgfpathcurveto{\pgfqpoint{1.550653in}{2.098995in}}{\pgfqpoint{1.542753in}{2.095722in}}{\pgfqpoint{1.536929in}{2.089899in}}%
\pgfpathcurveto{\pgfqpoint{1.531105in}{2.084075in}}{\pgfqpoint{1.527833in}{2.076175in}}{\pgfqpoint{1.527833in}{2.067938in}}%
\pgfpathcurveto{\pgfqpoint{1.527833in}{2.059702in}}{\pgfqpoint{1.531105in}{2.051802in}}{\pgfqpoint{1.536929in}{2.045978in}}%
\pgfpathcurveto{\pgfqpoint{1.542753in}{2.040154in}}{\pgfqpoint{1.550653in}{2.036882in}}{\pgfqpoint{1.558889in}{2.036882in}}%
\pgfpathclose%
\pgfusepath{stroke,fill}%
\end{pgfscope}%
\begin{pgfscope}%
\pgfpathrectangle{\pgfqpoint{0.100000in}{0.212622in}}{\pgfqpoint{3.696000in}{3.696000in}}%
\pgfusepath{clip}%
\pgfsetbuttcap%
\pgfsetroundjoin%
\definecolor{currentfill}{rgb}{0.121569,0.466667,0.705882}%
\pgfsetfillcolor{currentfill}%
\pgfsetfillopacity{0.300000}%
\pgfsetlinewidth{1.003750pt}%
\definecolor{currentstroke}{rgb}{0.121569,0.466667,0.705882}%
\pgfsetstrokecolor{currentstroke}%
\pgfsetstrokeopacity{0.300000}%
\pgfsetdash{}{0pt}%
\pgfpathmoveto{\pgfqpoint{1.558889in}{2.036882in}}%
\pgfpathcurveto{\pgfqpoint{1.567126in}{2.036882in}}{\pgfqpoint{1.575026in}{2.040154in}}{\pgfqpoint{1.580850in}{2.045978in}}%
\pgfpathcurveto{\pgfqpoint{1.586674in}{2.051802in}}{\pgfqpoint{1.589946in}{2.059702in}}{\pgfqpoint{1.589946in}{2.067938in}}%
\pgfpathcurveto{\pgfqpoint{1.589946in}{2.076175in}}{\pgfqpoint{1.586674in}{2.084075in}}{\pgfqpoint{1.580850in}{2.089899in}}%
\pgfpathcurveto{\pgfqpoint{1.575026in}{2.095722in}}{\pgfqpoint{1.567126in}{2.098995in}}{\pgfqpoint{1.558889in}{2.098995in}}%
\pgfpathcurveto{\pgfqpoint{1.550653in}{2.098995in}}{\pgfqpoint{1.542753in}{2.095722in}}{\pgfqpoint{1.536929in}{2.089899in}}%
\pgfpathcurveto{\pgfqpoint{1.531105in}{2.084075in}}{\pgfqpoint{1.527833in}{2.076175in}}{\pgfqpoint{1.527833in}{2.067938in}}%
\pgfpathcurveto{\pgfqpoint{1.527833in}{2.059702in}}{\pgfqpoint{1.531105in}{2.051802in}}{\pgfqpoint{1.536929in}{2.045978in}}%
\pgfpathcurveto{\pgfqpoint{1.542753in}{2.040154in}}{\pgfqpoint{1.550653in}{2.036882in}}{\pgfqpoint{1.558889in}{2.036882in}}%
\pgfpathclose%
\pgfusepath{stroke,fill}%
\end{pgfscope}%
\begin{pgfscope}%
\pgfpathrectangle{\pgfqpoint{0.100000in}{0.212622in}}{\pgfqpoint{3.696000in}{3.696000in}}%
\pgfusepath{clip}%
\pgfsetbuttcap%
\pgfsetroundjoin%
\definecolor{currentfill}{rgb}{0.121569,0.466667,0.705882}%
\pgfsetfillcolor{currentfill}%
\pgfsetfillopacity{0.300000}%
\pgfsetlinewidth{1.003750pt}%
\definecolor{currentstroke}{rgb}{0.121569,0.466667,0.705882}%
\pgfsetstrokecolor{currentstroke}%
\pgfsetstrokeopacity{0.300000}%
\pgfsetdash{}{0pt}%
\pgfpathmoveto{\pgfqpoint{1.558889in}{2.036882in}}%
\pgfpathcurveto{\pgfqpoint{1.567126in}{2.036882in}}{\pgfqpoint{1.575026in}{2.040154in}}{\pgfqpoint{1.580850in}{2.045978in}}%
\pgfpathcurveto{\pgfqpoint{1.586674in}{2.051802in}}{\pgfqpoint{1.589946in}{2.059702in}}{\pgfqpoint{1.589946in}{2.067938in}}%
\pgfpathcurveto{\pgfqpoint{1.589946in}{2.076175in}}{\pgfqpoint{1.586674in}{2.084075in}}{\pgfqpoint{1.580850in}{2.089899in}}%
\pgfpathcurveto{\pgfqpoint{1.575026in}{2.095722in}}{\pgfqpoint{1.567126in}{2.098995in}}{\pgfqpoint{1.558889in}{2.098995in}}%
\pgfpathcurveto{\pgfqpoint{1.550653in}{2.098995in}}{\pgfqpoint{1.542753in}{2.095722in}}{\pgfqpoint{1.536929in}{2.089899in}}%
\pgfpathcurveto{\pgfqpoint{1.531105in}{2.084075in}}{\pgfqpoint{1.527833in}{2.076175in}}{\pgfqpoint{1.527833in}{2.067938in}}%
\pgfpathcurveto{\pgfqpoint{1.527833in}{2.059702in}}{\pgfqpoint{1.531105in}{2.051802in}}{\pgfqpoint{1.536929in}{2.045978in}}%
\pgfpathcurveto{\pgfqpoint{1.542753in}{2.040154in}}{\pgfqpoint{1.550653in}{2.036882in}}{\pgfqpoint{1.558889in}{2.036882in}}%
\pgfpathclose%
\pgfusepath{stroke,fill}%
\end{pgfscope}%
\begin{pgfscope}%
\pgfpathrectangle{\pgfqpoint{0.100000in}{0.212622in}}{\pgfqpoint{3.696000in}{3.696000in}}%
\pgfusepath{clip}%
\pgfsetbuttcap%
\pgfsetroundjoin%
\definecolor{currentfill}{rgb}{0.121569,0.466667,0.705882}%
\pgfsetfillcolor{currentfill}%
\pgfsetfillopacity{0.300000}%
\pgfsetlinewidth{1.003750pt}%
\definecolor{currentstroke}{rgb}{0.121569,0.466667,0.705882}%
\pgfsetstrokecolor{currentstroke}%
\pgfsetstrokeopacity{0.300000}%
\pgfsetdash{}{0pt}%
\pgfpathmoveto{\pgfqpoint{1.558889in}{2.036882in}}%
\pgfpathcurveto{\pgfqpoint{1.567126in}{2.036882in}}{\pgfqpoint{1.575026in}{2.040154in}}{\pgfqpoint{1.580850in}{2.045978in}}%
\pgfpathcurveto{\pgfqpoint{1.586674in}{2.051802in}}{\pgfqpoint{1.589946in}{2.059702in}}{\pgfqpoint{1.589946in}{2.067938in}}%
\pgfpathcurveto{\pgfqpoint{1.589946in}{2.076175in}}{\pgfqpoint{1.586674in}{2.084075in}}{\pgfqpoint{1.580850in}{2.089899in}}%
\pgfpathcurveto{\pgfqpoint{1.575026in}{2.095722in}}{\pgfqpoint{1.567126in}{2.098995in}}{\pgfqpoint{1.558889in}{2.098995in}}%
\pgfpathcurveto{\pgfqpoint{1.550653in}{2.098995in}}{\pgfqpoint{1.542753in}{2.095722in}}{\pgfqpoint{1.536929in}{2.089899in}}%
\pgfpathcurveto{\pgfqpoint{1.531105in}{2.084075in}}{\pgfqpoint{1.527833in}{2.076175in}}{\pgfqpoint{1.527833in}{2.067938in}}%
\pgfpathcurveto{\pgfqpoint{1.527833in}{2.059702in}}{\pgfqpoint{1.531105in}{2.051802in}}{\pgfqpoint{1.536929in}{2.045978in}}%
\pgfpathcurveto{\pgfqpoint{1.542753in}{2.040154in}}{\pgfqpoint{1.550653in}{2.036882in}}{\pgfqpoint{1.558889in}{2.036882in}}%
\pgfpathclose%
\pgfusepath{stroke,fill}%
\end{pgfscope}%
\begin{pgfscope}%
\pgfpathrectangle{\pgfqpoint{0.100000in}{0.212622in}}{\pgfqpoint{3.696000in}{3.696000in}}%
\pgfusepath{clip}%
\pgfsetbuttcap%
\pgfsetroundjoin%
\definecolor{currentfill}{rgb}{0.121569,0.466667,0.705882}%
\pgfsetfillcolor{currentfill}%
\pgfsetfillopacity{0.300000}%
\pgfsetlinewidth{1.003750pt}%
\definecolor{currentstroke}{rgb}{0.121569,0.466667,0.705882}%
\pgfsetstrokecolor{currentstroke}%
\pgfsetstrokeopacity{0.300000}%
\pgfsetdash{}{0pt}%
\pgfpathmoveto{\pgfqpoint{1.558889in}{2.036882in}}%
\pgfpathcurveto{\pgfqpoint{1.567126in}{2.036882in}}{\pgfqpoint{1.575026in}{2.040154in}}{\pgfqpoint{1.580850in}{2.045978in}}%
\pgfpathcurveto{\pgfqpoint{1.586674in}{2.051802in}}{\pgfqpoint{1.589946in}{2.059702in}}{\pgfqpoint{1.589946in}{2.067938in}}%
\pgfpathcurveto{\pgfqpoint{1.589946in}{2.076175in}}{\pgfqpoint{1.586674in}{2.084075in}}{\pgfqpoint{1.580850in}{2.089899in}}%
\pgfpathcurveto{\pgfqpoint{1.575026in}{2.095722in}}{\pgfqpoint{1.567126in}{2.098995in}}{\pgfqpoint{1.558889in}{2.098995in}}%
\pgfpathcurveto{\pgfqpoint{1.550653in}{2.098995in}}{\pgfqpoint{1.542753in}{2.095722in}}{\pgfqpoint{1.536929in}{2.089899in}}%
\pgfpathcurveto{\pgfqpoint{1.531105in}{2.084075in}}{\pgfqpoint{1.527833in}{2.076175in}}{\pgfqpoint{1.527833in}{2.067938in}}%
\pgfpathcurveto{\pgfqpoint{1.527833in}{2.059702in}}{\pgfqpoint{1.531105in}{2.051802in}}{\pgfqpoint{1.536929in}{2.045978in}}%
\pgfpathcurveto{\pgfqpoint{1.542753in}{2.040154in}}{\pgfqpoint{1.550653in}{2.036882in}}{\pgfqpoint{1.558889in}{2.036882in}}%
\pgfpathclose%
\pgfusepath{stroke,fill}%
\end{pgfscope}%
\begin{pgfscope}%
\pgfpathrectangle{\pgfqpoint{0.100000in}{0.212622in}}{\pgfqpoint{3.696000in}{3.696000in}}%
\pgfusepath{clip}%
\pgfsetbuttcap%
\pgfsetroundjoin%
\definecolor{currentfill}{rgb}{0.121569,0.466667,0.705882}%
\pgfsetfillcolor{currentfill}%
\pgfsetfillopacity{0.300000}%
\pgfsetlinewidth{1.003750pt}%
\definecolor{currentstroke}{rgb}{0.121569,0.466667,0.705882}%
\pgfsetstrokecolor{currentstroke}%
\pgfsetstrokeopacity{0.300000}%
\pgfsetdash{}{0pt}%
\pgfpathmoveto{\pgfqpoint{1.558889in}{2.036882in}}%
\pgfpathcurveto{\pgfqpoint{1.567126in}{2.036882in}}{\pgfqpoint{1.575026in}{2.040154in}}{\pgfqpoint{1.580850in}{2.045978in}}%
\pgfpathcurveto{\pgfqpoint{1.586674in}{2.051802in}}{\pgfqpoint{1.589946in}{2.059702in}}{\pgfqpoint{1.589946in}{2.067938in}}%
\pgfpathcurveto{\pgfqpoint{1.589946in}{2.076175in}}{\pgfqpoint{1.586674in}{2.084075in}}{\pgfqpoint{1.580850in}{2.089899in}}%
\pgfpathcurveto{\pgfqpoint{1.575026in}{2.095722in}}{\pgfqpoint{1.567126in}{2.098995in}}{\pgfqpoint{1.558889in}{2.098995in}}%
\pgfpathcurveto{\pgfqpoint{1.550653in}{2.098995in}}{\pgfqpoint{1.542753in}{2.095722in}}{\pgfqpoint{1.536929in}{2.089899in}}%
\pgfpathcurveto{\pgfqpoint{1.531105in}{2.084075in}}{\pgfqpoint{1.527833in}{2.076175in}}{\pgfqpoint{1.527833in}{2.067938in}}%
\pgfpathcurveto{\pgfqpoint{1.527833in}{2.059702in}}{\pgfqpoint{1.531105in}{2.051802in}}{\pgfqpoint{1.536929in}{2.045978in}}%
\pgfpathcurveto{\pgfqpoint{1.542753in}{2.040154in}}{\pgfqpoint{1.550653in}{2.036882in}}{\pgfqpoint{1.558889in}{2.036882in}}%
\pgfpathclose%
\pgfusepath{stroke,fill}%
\end{pgfscope}%
\begin{pgfscope}%
\pgfpathrectangle{\pgfqpoint{0.100000in}{0.212622in}}{\pgfqpoint{3.696000in}{3.696000in}}%
\pgfusepath{clip}%
\pgfsetbuttcap%
\pgfsetroundjoin%
\definecolor{currentfill}{rgb}{0.121569,0.466667,0.705882}%
\pgfsetfillcolor{currentfill}%
\pgfsetfillopacity{0.300000}%
\pgfsetlinewidth{1.003750pt}%
\definecolor{currentstroke}{rgb}{0.121569,0.466667,0.705882}%
\pgfsetstrokecolor{currentstroke}%
\pgfsetstrokeopacity{0.300000}%
\pgfsetdash{}{0pt}%
\pgfpathmoveto{\pgfqpoint{1.558889in}{2.036882in}}%
\pgfpathcurveto{\pgfqpoint{1.567126in}{2.036882in}}{\pgfqpoint{1.575026in}{2.040154in}}{\pgfqpoint{1.580850in}{2.045978in}}%
\pgfpathcurveto{\pgfqpoint{1.586674in}{2.051802in}}{\pgfqpoint{1.589946in}{2.059702in}}{\pgfqpoint{1.589946in}{2.067938in}}%
\pgfpathcurveto{\pgfqpoint{1.589946in}{2.076175in}}{\pgfqpoint{1.586674in}{2.084075in}}{\pgfqpoint{1.580850in}{2.089899in}}%
\pgfpathcurveto{\pgfqpoint{1.575026in}{2.095722in}}{\pgfqpoint{1.567126in}{2.098995in}}{\pgfqpoint{1.558889in}{2.098995in}}%
\pgfpathcurveto{\pgfqpoint{1.550653in}{2.098995in}}{\pgfqpoint{1.542753in}{2.095722in}}{\pgfqpoint{1.536929in}{2.089899in}}%
\pgfpathcurveto{\pgfqpoint{1.531105in}{2.084075in}}{\pgfqpoint{1.527833in}{2.076175in}}{\pgfqpoint{1.527833in}{2.067938in}}%
\pgfpathcurveto{\pgfqpoint{1.527833in}{2.059702in}}{\pgfqpoint{1.531105in}{2.051802in}}{\pgfqpoint{1.536929in}{2.045978in}}%
\pgfpathcurveto{\pgfqpoint{1.542753in}{2.040154in}}{\pgfqpoint{1.550653in}{2.036882in}}{\pgfqpoint{1.558889in}{2.036882in}}%
\pgfpathclose%
\pgfusepath{stroke,fill}%
\end{pgfscope}%
\begin{pgfscope}%
\pgfpathrectangle{\pgfqpoint{0.100000in}{0.212622in}}{\pgfqpoint{3.696000in}{3.696000in}}%
\pgfusepath{clip}%
\pgfsetbuttcap%
\pgfsetroundjoin%
\definecolor{currentfill}{rgb}{0.121569,0.466667,0.705882}%
\pgfsetfillcolor{currentfill}%
\pgfsetfillopacity{0.300000}%
\pgfsetlinewidth{1.003750pt}%
\definecolor{currentstroke}{rgb}{0.121569,0.466667,0.705882}%
\pgfsetstrokecolor{currentstroke}%
\pgfsetstrokeopacity{0.300000}%
\pgfsetdash{}{0pt}%
\pgfpathmoveto{\pgfqpoint{1.558889in}{2.036882in}}%
\pgfpathcurveto{\pgfqpoint{1.567126in}{2.036882in}}{\pgfqpoint{1.575026in}{2.040154in}}{\pgfqpoint{1.580850in}{2.045978in}}%
\pgfpathcurveto{\pgfqpoint{1.586674in}{2.051802in}}{\pgfqpoint{1.589946in}{2.059702in}}{\pgfqpoint{1.589946in}{2.067938in}}%
\pgfpathcurveto{\pgfqpoint{1.589946in}{2.076175in}}{\pgfqpoint{1.586674in}{2.084075in}}{\pgfqpoint{1.580850in}{2.089899in}}%
\pgfpathcurveto{\pgfqpoint{1.575026in}{2.095722in}}{\pgfqpoint{1.567126in}{2.098995in}}{\pgfqpoint{1.558889in}{2.098995in}}%
\pgfpathcurveto{\pgfqpoint{1.550653in}{2.098995in}}{\pgfqpoint{1.542753in}{2.095722in}}{\pgfqpoint{1.536929in}{2.089899in}}%
\pgfpathcurveto{\pgfqpoint{1.531105in}{2.084075in}}{\pgfqpoint{1.527833in}{2.076175in}}{\pgfqpoint{1.527833in}{2.067938in}}%
\pgfpathcurveto{\pgfqpoint{1.527833in}{2.059702in}}{\pgfqpoint{1.531105in}{2.051802in}}{\pgfqpoint{1.536929in}{2.045978in}}%
\pgfpathcurveto{\pgfqpoint{1.542753in}{2.040154in}}{\pgfqpoint{1.550653in}{2.036882in}}{\pgfqpoint{1.558889in}{2.036882in}}%
\pgfpathclose%
\pgfusepath{stroke,fill}%
\end{pgfscope}%
\begin{pgfscope}%
\pgfpathrectangle{\pgfqpoint{0.100000in}{0.212622in}}{\pgfqpoint{3.696000in}{3.696000in}}%
\pgfusepath{clip}%
\pgfsetbuttcap%
\pgfsetroundjoin%
\definecolor{currentfill}{rgb}{0.121569,0.466667,0.705882}%
\pgfsetfillcolor{currentfill}%
\pgfsetfillopacity{0.300000}%
\pgfsetlinewidth{1.003750pt}%
\definecolor{currentstroke}{rgb}{0.121569,0.466667,0.705882}%
\pgfsetstrokecolor{currentstroke}%
\pgfsetstrokeopacity{0.300000}%
\pgfsetdash{}{0pt}%
\pgfpathmoveto{\pgfqpoint{1.558889in}{2.036882in}}%
\pgfpathcurveto{\pgfqpoint{1.567126in}{2.036882in}}{\pgfqpoint{1.575026in}{2.040154in}}{\pgfqpoint{1.580850in}{2.045978in}}%
\pgfpathcurveto{\pgfqpoint{1.586674in}{2.051802in}}{\pgfqpoint{1.589946in}{2.059702in}}{\pgfqpoint{1.589946in}{2.067938in}}%
\pgfpathcurveto{\pgfqpoint{1.589946in}{2.076175in}}{\pgfqpoint{1.586674in}{2.084075in}}{\pgfqpoint{1.580850in}{2.089899in}}%
\pgfpathcurveto{\pgfqpoint{1.575026in}{2.095722in}}{\pgfqpoint{1.567126in}{2.098995in}}{\pgfqpoint{1.558889in}{2.098995in}}%
\pgfpathcurveto{\pgfqpoint{1.550653in}{2.098995in}}{\pgfqpoint{1.542753in}{2.095722in}}{\pgfqpoint{1.536929in}{2.089899in}}%
\pgfpathcurveto{\pgfqpoint{1.531105in}{2.084075in}}{\pgfqpoint{1.527833in}{2.076175in}}{\pgfqpoint{1.527833in}{2.067938in}}%
\pgfpathcurveto{\pgfqpoint{1.527833in}{2.059702in}}{\pgfqpoint{1.531105in}{2.051802in}}{\pgfqpoint{1.536929in}{2.045978in}}%
\pgfpathcurveto{\pgfqpoint{1.542753in}{2.040154in}}{\pgfqpoint{1.550653in}{2.036882in}}{\pgfqpoint{1.558889in}{2.036882in}}%
\pgfpathclose%
\pgfusepath{stroke,fill}%
\end{pgfscope}%
\begin{pgfscope}%
\pgfpathrectangle{\pgfqpoint{0.100000in}{0.212622in}}{\pgfqpoint{3.696000in}{3.696000in}}%
\pgfusepath{clip}%
\pgfsetbuttcap%
\pgfsetroundjoin%
\definecolor{currentfill}{rgb}{0.121569,0.466667,0.705882}%
\pgfsetfillcolor{currentfill}%
\pgfsetfillopacity{0.300000}%
\pgfsetlinewidth{1.003750pt}%
\definecolor{currentstroke}{rgb}{0.121569,0.466667,0.705882}%
\pgfsetstrokecolor{currentstroke}%
\pgfsetstrokeopacity{0.300000}%
\pgfsetdash{}{0pt}%
\pgfpathmoveto{\pgfqpoint{1.558889in}{2.036882in}}%
\pgfpathcurveto{\pgfqpoint{1.567126in}{2.036882in}}{\pgfqpoint{1.575026in}{2.040154in}}{\pgfqpoint{1.580850in}{2.045978in}}%
\pgfpathcurveto{\pgfqpoint{1.586673in}{2.051802in}}{\pgfqpoint{1.589946in}{2.059702in}}{\pgfqpoint{1.589946in}{2.067938in}}%
\pgfpathcurveto{\pgfqpoint{1.589946in}{2.076174in}}{\pgfqpoint{1.586673in}{2.084075in}}{\pgfqpoint{1.580850in}{2.089898in}}%
\pgfpathcurveto{\pgfqpoint{1.575026in}{2.095722in}}{\pgfqpoint{1.567126in}{2.098995in}}{\pgfqpoint{1.558889in}{2.098995in}}%
\pgfpathcurveto{\pgfqpoint{1.550653in}{2.098995in}}{\pgfqpoint{1.542753in}{2.095722in}}{\pgfqpoint{1.536929in}{2.089898in}}%
\pgfpathcurveto{\pgfqpoint{1.531105in}{2.084075in}}{\pgfqpoint{1.527833in}{2.076174in}}{\pgfqpoint{1.527833in}{2.067938in}}%
\pgfpathcurveto{\pgfqpoint{1.527833in}{2.059702in}}{\pgfqpoint{1.531105in}{2.051802in}}{\pgfqpoint{1.536929in}{2.045978in}}%
\pgfpathcurveto{\pgfqpoint{1.542753in}{2.040154in}}{\pgfqpoint{1.550653in}{2.036882in}}{\pgfqpoint{1.558889in}{2.036882in}}%
\pgfpathclose%
\pgfusepath{stroke,fill}%
\end{pgfscope}%
\begin{pgfscope}%
\pgfpathrectangle{\pgfqpoint{0.100000in}{0.212622in}}{\pgfqpoint{3.696000in}{3.696000in}}%
\pgfusepath{clip}%
\pgfsetbuttcap%
\pgfsetroundjoin%
\definecolor{currentfill}{rgb}{0.121569,0.466667,0.705882}%
\pgfsetfillcolor{currentfill}%
\pgfsetfillopacity{0.300000}%
\pgfsetlinewidth{1.003750pt}%
\definecolor{currentstroke}{rgb}{0.121569,0.466667,0.705882}%
\pgfsetstrokecolor{currentstroke}%
\pgfsetstrokeopacity{0.300000}%
\pgfsetdash{}{0pt}%
\pgfpathmoveto{\pgfqpoint{1.558889in}{2.036881in}}%
\pgfpathcurveto{\pgfqpoint{1.567125in}{2.036881in}}{\pgfqpoint{1.575025in}{2.040154in}}{\pgfqpoint{1.580849in}{2.045978in}}%
\pgfpathcurveto{\pgfqpoint{1.586673in}{2.051802in}}{\pgfqpoint{1.589945in}{2.059702in}}{\pgfqpoint{1.589945in}{2.067938in}}%
\pgfpathcurveto{\pgfqpoint{1.589945in}{2.076174in}}{\pgfqpoint{1.586673in}{2.084074in}}{\pgfqpoint{1.580849in}{2.089898in}}%
\pgfpathcurveto{\pgfqpoint{1.575025in}{2.095722in}}{\pgfqpoint{1.567125in}{2.098994in}}{\pgfqpoint{1.558889in}{2.098994in}}%
\pgfpathcurveto{\pgfqpoint{1.550653in}{2.098994in}}{\pgfqpoint{1.542753in}{2.095722in}}{\pgfqpoint{1.536929in}{2.089898in}}%
\pgfpathcurveto{\pgfqpoint{1.531105in}{2.084074in}}{\pgfqpoint{1.527832in}{2.076174in}}{\pgfqpoint{1.527832in}{2.067938in}}%
\pgfpathcurveto{\pgfqpoint{1.527832in}{2.059702in}}{\pgfqpoint{1.531105in}{2.051802in}}{\pgfqpoint{1.536929in}{2.045978in}}%
\pgfpathcurveto{\pgfqpoint{1.542753in}{2.040154in}}{\pgfqpoint{1.550653in}{2.036881in}}{\pgfqpoint{1.558889in}{2.036881in}}%
\pgfpathclose%
\pgfusepath{stroke,fill}%
\end{pgfscope}%
\begin{pgfscope}%
\pgfpathrectangle{\pgfqpoint{0.100000in}{0.212622in}}{\pgfqpoint{3.696000in}{3.696000in}}%
\pgfusepath{clip}%
\pgfsetbuttcap%
\pgfsetroundjoin%
\definecolor{currentfill}{rgb}{0.121569,0.466667,0.705882}%
\pgfsetfillcolor{currentfill}%
\pgfsetfillopacity{0.300000}%
\pgfsetlinewidth{1.003750pt}%
\definecolor{currentstroke}{rgb}{0.121569,0.466667,0.705882}%
\pgfsetstrokecolor{currentstroke}%
\pgfsetstrokeopacity{0.300000}%
\pgfsetdash{}{0pt}%
\pgfpathmoveto{\pgfqpoint{1.558889in}{2.036881in}}%
\pgfpathcurveto{\pgfqpoint{1.567125in}{2.036881in}}{\pgfqpoint{1.575025in}{2.040154in}}{\pgfqpoint{1.580849in}{2.045978in}}%
\pgfpathcurveto{\pgfqpoint{1.586673in}{2.051802in}}{\pgfqpoint{1.589945in}{2.059702in}}{\pgfqpoint{1.589945in}{2.067938in}}%
\pgfpathcurveto{\pgfqpoint{1.589945in}{2.076174in}}{\pgfqpoint{1.586673in}{2.084074in}}{\pgfqpoint{1.580849in}{2.089898in}}%
\pgfpathcurveto{\pgfqpoint{1.575025in}{2.095722in}}{\pgfqpoint{1.567125in}{2.098994in}}{\pgfqpoint{1.558889in}{2.098994in}}%
\pgfpathcurveto{\pgfqpoint{1.550652in}{2.098994in}}{\pgfqpoint{1.542752in}{2.095722in}}{\pgfqpoint{1.536928in}{2.089898in}}%
\pgfpathcurveto{\pgfqpoint{1.531104in}{2.084074in}}{\pgfqpoint{1.527832in}{2.076174in}}{\pgfqpoint{1.527832in}{2.067938in}}%
\pgfpathcurveto{\pgfqpoint{1.527832in}{2.059702in}}{\pgfqpoint{1.531104in}{2.051802in}}{\pgfqpoint{1.536928in}{2.045978in}}%
\pgfpathcurveto{\pgfqpoint{1.542752in}{2.040154in}}{\pgfqpoint{1.550652in}{2.036881in}}{\pgfqpoint{1.558889in}{2.036881in}}%
\pgfpathclose%
\pgfusepath{stroke,fill}%
\end{pgfscope}%
\begin{pgfscope}%
\pgfpathrectangle{\pgfqpoint{0.100000in}{0.212622in}}{\pgfqpoint{3.696000in}{3.696000in}}%
\pgfusepath{clip}%
\pgfsetbuttcap%
\pgfsetroundjoin%
\definecolor{currentfill}{rgb}{0.121569,0.466667,0.705882}%
\pgfsetfillcolor{currentfill}%
\pgfsetfillopacity{0.300001}%
\pgfsetlinewidth{1.003750pt}%
\definecolor{currentstroke}{rgb}{0.121569,0.466667,0.705882}%
\pgfsetstrokecolor{currentstroke}%
\pgfsetstrokeopacity{0.300001}%
\pgfsetdash{}{0pt}%
\pgfpathmoveto{\pgfqpoint{1.558887in}{2.036880in}}%
\pgfpathcurveto{\pgfqpoint{1.567123in}{2.036880in}}{\pgfqpoint{1.575023in}{2.040152in}}{\pgfqpoint{1.580847in}{2.045976in}}%
\pgfpathcurveto{\pgfqpoint{1.586671in}{2.051800in}}{\pgfqpoint{1.589943in}{2.059700in}}{\pgfqpoint{1.589943in}{2.067936in}}%
\pgfpathcurveto{\pgfqpoint{1.589943in}{2.076173in}}{\pgfqpoint{1.586671in}{2.084073in}}{\pgfqpoint{1.580847in}{2.089897in}}%
\pgfpathcurveto{\pgfqpoint{1.575023in}{2.095721in}}{\pgfqpoint{1.567123in}{2.098993in}}{\pgfqpoint{1.558887in}{2.098993in}}%
\pgfpathcurveto{\pgfqpoint{1.550651in}{2.098993in}}{\pgfqpoint{1.542751in}{2.095721in}}{\pgfqpoint{1.536927in}{2.089897in}}%
\pgfpathcurveto{\pgfqpoint{1.531103in}{2.084073in}}{\pgfqpoint{1.527830in}{2.076173in}}{\pgfqpoint{1.527830in}{2.067936in}}%
\pgfpathcurveto{\pgfqpoint{1.527830in}{2.059700in}}{\pgfqpoint{1.531103in}{2.051800in}}{\pgfqpoint{1.536927in}{2.045976in}}%
\pgfpathcurveto{\pgfqpoint{1.542751in}{2.040152in}}{\pgfqpoint{1.550651in}{2.036880in}}{\pgfqpoint{1.558887in}{2.036880in}}%
\pgfpathclose%
\pgfusepath{stroke,fill}%
\end{pgfscope}%
\begin{pgfscope}%
\pgfpathrectangle{\pgfqpoint{0.100000in}{0.212622in}}{\pgfqpoint{3.696000in}{3.696000in}}%
\pgfusepath{clip}%
\pgfsetbuttcap%
\pgfsetroundjoin%
\definecolor{currentfill}{rgb}{0.121569,0.466667,0.705882}%
\pgfsetfillcolor{currentfill}%
\pgfsetfillopacity{0.300001}%
\pgfsetlinewidth{1.003750pt}%
\definecolor{currentstroke}{rgb}{0.121569,0.466667,0.705882}%
\pgfsetstrokecolor{currentstroke}%
\pgfsetstrokeopacity{0.300001}%
\pgfsetdash{}{0pt}%
\pgfpathmoveto{\pgfqpoint{1.558886in}{2.036880in}}%
\pgfpathcurveto{\pgfqpoint{1.567123in}{2.036880in}}{\pgfqpoint{1.575023in}{2.040152in}}{\pgfqpoint{1.580847in}{2.045976in}}%
\pgfpathcurveto{\pgfqpoint{1.586670in}{2.051800in}}{\pgfqpoint{1.589943in}{2.059700in}}{\pgfqpoint{1.589943in}{2.067936in}}%
\pgfpathcurveto{\pgfqpoint{1.589943in}{2.076173in}}{\pgfqpoint{1.586670in}{2.084073in}}{\pgfqpoint{1.580847in}{2.089897in}}%
\pgfpathcurveto{\pgfqpoint{1.575023in}{2.095721in}}{\pgfqpoint{1.567123in}{2.098993in}}{\pgfqpoint{1.558886in}{2.098993in}}%
\pgfpathcurveto{\pgfqpoint{1.550650in}{2.098993in}}{\pgfqpoint{1.542750in}{2.095721in}}{\pgfqpoint{1.536926in}{2.089897in}}%
\pgfpathcurveto{\pgfqpoint{1.531102in}{2.084073in}}{\pgfqpoint{1.527830in}{2.076173in}}{\pgfqpoint{1.527830in}{2.067936in}}%
\pgfpathcurveto{\pgfqpoint{1.527830in}{2.059700in}}{\pgfqpoint{1.531102in}{2.051800in}}{\pgfqpoint{1.536926in}{2.045976in}}%
\pgfpathcurveto{\pgfqpoint{1.542750in}{2.040152in}}{\pgfqpoint{1.550650in}{2.036880in}}{\pgfqpoint{1.558886in}{2.036880in}}%
\pgfpathclose%
\pgfusepath{stroke,fill}%
\end{pgfscope}%
\begin{pgfscope}%
\pgfpathrectangle{\pgfqpoint{0.100000in}{0.212622in}}{\pgfqpoint{3.696000in}{3.696000in}}%
\pgfusepath{clip}%
\pgfsetbuttcap%
\pgfsetroundjoin%
\definecolor{currentfill}{rgb}{0.121569,0.466667,0.705882}%
\pgfsetfillcolor{currentfill}%
\pgfsetfillopacity{0.300002}%
\pgfsetlinewidth{1.003750pt}%
\definecolor{currentstroke}{rgb}{0.121569,0.466667,0.705882}%
\pgfsetstrokecolor{currentstroke}%
\pgfsetstrokeopacity{0.300002}%
\pgfsetdash{}{0pt}%
\pgfpathmoveto{\pgfqpoint{1.558880in}{2.036876in}}%
\pgfpathcurveto{\pgfqpoint{1.567116in}{2.036876in}}{\pgfqpoint{1.575016in}{2.040148in}}{\pgfqpoint{1.580840in}{2.045972in}}%
\pgfpathcurveto{\pgfqpoint{1.586664in}{2.051796in}}{\pgfqpoint{1.589936in}{2.059696in}}{\pgfqpoint{1.589936in}{2.067932in}}%
\pgfpathcurveto{\pgfqpoint{1.589936in}{2.076169in}}{\pgfqpoint{1.586664in}{2.084069in}}{\pgfqpoint{1.580840in}{2.089893in}}%
\pgfpathcurveto{\pgfqpoint{1.575016in}{2.095716in}}{\pgfqpoint{1.567116in}{2.098989in}}{\pgfqpoint{1.558880in}{2.098989in}}%
\pgfpathcurveto{\pgfqpoint{1.550643in}{2.098989in}}{\pgfqpoint{1.542743in}{2.095716in}}{\pgfqpoint{1.536919in}{2.089893in}}%
\pgfpathcurveto{\pgfqpoint{1.531096in}{2.084069in}}{\pgfqpoint{1.527823in}{2.076169in}}{\pgfqpoint{1.527823in}{2.067932in}}%
\pgfpathcurveto{\pgfqpoint{1.527823in}{2.059696in}}{\pgfqpoint{1.531096in}{2.051796in}}{\pgfqpoint{1.536919in}{2.045972in}}%
\pgfpathcurveto{\pgfqpoint{1.542743in}{2.040148in}}{\pgfqpoint{1.550643in}{2.036876in}}{\pgfqpoint{1.558880in}{2.036876in}}%
\pgfpathclose%
\pgfusepath{stroke,fill}%
\end{pgfscope}%
\begin{pgfscope}%
\pgfpathrectangle{\pgfqpoint{0.100000in}{0.212622in}}{\pgfqpoint{3.696000in}{3.696000in}}%
\pgfusepath{clip}%
\pgfsetbuttcap%
\pgfsetroundjoin%
\definecolor{currentfill}{rgb}{0.121569,0.466667,0.705882}%
\pgfsetfillcolor{currentfill}%
\pgfsetfillopacity{0.300002}%
\pgfsetlinewidth{1.003750pt}%
\definecolor{currentstroke}{rgb}{0.121569,0.466667,0.705882}%
\pgfsetstrokecolor{currentstroke}%
\pgfsetstrokeopacity{0.300002}%
\pgfsetdash{}{0pt}%
\pgfpathmoveto{\pgfqpoint{1.558880in}{2.036874in}}%
\pgfpathcurveto{\pgfqpoint{1.567116in}{2.036874in}}{\pgfqpoint{1.575016in}{2.040147in}}{\pgfqpoint{1.580840in}{2.045971in}}%
\pgfpathcurveto{\pgfqpoint{1.586664in}{2.051794in}}{\pgfqpoint{1.589936in}{2.059694in}}{\pgfqpoint{1.589936in}{2.067931in}}%
\pgfpathcurveto{\pgfqpoint{1.589936in}{2.076167in}}{\pgfqpoint{1.586664in}{2.084067in}}{\pgfqpoint{1.580840in}{2.089891in}}%
\pgfpathcurveto{\pgfqpoint{1.575016in}{2.095715in}}{\pgfqpoint{1.567116in}{2.098987in}}{\pgfqpoint{1.558880in}{2.098987in}}%
\pgfpathcurveto{\pgfqpoint{1.550644in}{2.098987in}}{\pgfqpoint{1.542744in}{2.095715in}}{\pgfqpoint{1.536920in}{2.089891in}}%
\pgfpathcurveto{\pgfqpoint{1.531096in}{2.084067in}}{\pgfqpoint{1.527823in}{2.076167in}}{\pgfqpoint{1.527823in}{2.067931in}}%
\pgfpathcurveto{\pgfqpoint{1.527823in}{2.059694in}}{\pgfqpoint{1.531096in}{2.051794in}}{\pgfqpoint{1.536920in}{2.045971in}}%
\pgfpathcurveto{\pgfqpoint{1.542744in}{2.040147in}}{\pgfqpoint{1.550644in}{2.036874in}}{\pgfqpoint{1.558880in}{2.036874in}}%
\pgfpathclose%
\pgfusepath{stroke,fill}%
\end{pgfscope}%
\begin{pgfscope}%
\pgfpathrectangle{\pgfqpoint{0.100000in}{0.212622in}}{\pgfqpoint{3.696000in}{3.696000in}}%
\pgfusepath{clip}%
\pgfsetbuttcap%
\pgfsetroundjoin%
\definecolor{currentfill}{rgb}{0.121569,0.466667,0.705882}%
\pgfsetfillcolor{currentfill}%
\pgfsetfillopacity{0.300004}%
\pgfsetlinewidth{1.003750pt}%
\definecolor{currentstroke}{rgb}{0.121569,0.466667,0.705882}%
\pgfsetstrokecolor{currentstroke}%
\pgfsetstrokeopacity{0.300004}%
\pgfsetdash{}{0pt}%
\pgfpathmoveto{\pgfqpoint{1.558864in}{2.036868in}}%
\pgfpathcurveto{\pgfqpoint{1.567100in}{2.036868in}}{\pgfqpoint{1.575000in}{2.040140in}}{\pgfqpoint{1.580824in}{2.045964in}}%
\pgfpathcurveto{\pgfqpoint{1.586648in}{2.051788in}}{\pgfqpoint{1.589920in}{2.059688in}}{\pgfqpoint{1.589920in}{2.067924in}}%
\pgfpathcurveto{\pgfqpoint{1.589920in}{2.076160in}}{\pgfqpoint{1.586648in}{2.084060in}}{\pgfqpoint{1.580824in}{2.089884in}}%
\pgfpathcurveto{\pgfqpoint{1.575000in}{2.095708in}}{\pgfqpoint{1.567100in}{2.098981in}}{\pgfqpoint{1.558864in}{2.098981in}}%
\pgfpathcurveto{\pgfqpoint{1.550628in}{2.098981in}}{\pgfqpoint{1.542728in}{2.095708in}}{\pgfqpoint{1.536904in}{2.089884in}}%
\pgfpathcurveto{\pgfqpoint{1.531080in}{2.084060in}}{\pgfqpoint{1.527807in}{2.076160in}}{\pgfqpoint{1.527807in}{2.067924in}}%
\pgfpathcurveto{\pgfqpoint{1.527807in}{2.059688in}}{\pgfqpoint{1.531080in}{2.051788in}}{\pgfqpoint{1.536904in}{2.045964in}}%
\pgfpathcurveto{\pgfqpoint{1.542728in}{2.040140in}}{\pgfqpoint{1.550628in}{2.036868in}}{\pgfqpoint{1.558864in}{2.036868in}}%
\pgfpathclose%
\pgfusepath{stroke,fill}%
\end{pgfscope}%
\begin{pgfscope}%
\pgfpathrectangle{\pgfqpoint{0.100000in}{0.212622in}}{\pgfqpoint{3.696000in}{3.696000in}}%
\pgfusepath{clip}%
\pgfsetbuttcap%
\pgfsetroundjoin%
\definecolor{currentfill}{rgb}{0.121569,0.466667,0.705882}%
\pgfsetfillcolor{currentfill}%
\pgfsetfillopacity{0.300005}%
\pgfsetlinewidth{1.003750pt}%
\definecolor{currentstroke}{rgb}{0.121569,0.466667,0.705882}%
\pgfsetstrokecolor{currentstroke}%
\pgfsetstrokeopacity{0.300005}%
\pgfsetdash{}{0pt}%
\pgfpathmoveto{\pgfqpoint{1.558865in}{2.036863in}}%
\pgfpathcurveto{\pgfqpoint{1.567101in}{2.036863in}}{\pgfqpoint{1.575001in}{2.040135in}}{\pgfqpoint{1.580825in}{2.045959in}}%
\pgfpathcurveto{\pgfqpoint{1.586649in}{2.051783in}}{\pgfqpoint{1.589921in}{2.059683in}}{\pgfqpoint{1.589921in}{2.067919in}}%
\pgfpathcurveto{\pgfqpoint{1.589921in}{2.076155in}}{\pgfqpoint{1.586649in}{2.084055in}}{\pgfqpoint{1.580825in}{2.089879in}}%
\pgfpathcurveto{\pgfqpoint{1.575001in}{2.095703in}}{\pgfqpoint{1.567101in}{2.098976in}}{\pgfqpoint{1.558865in}{2.098976in}}%
\pgfpathcurveto{\pgfqpoint{1.550628in}{2.098976in}}{\pgfqpoint{1.542728in}{2.095703in}}{\pgfqpoint{1.536904in}{2.089879in}}%
\pgfpathcurveto{\pgfqpoint{1.531080in}{2.084055in}}{\pgfqpoint{1.527808in}{2.076155in}}{\pgfqpoint{1.527808in}{2.067919in}}%
\pgfpathcurveto{\pgfqpoint{1.527808in}{2.059683in}}{\pgfqpoint{1.531080in}{2.051783in}}{\pgfqpoint{1.536904in}{2.045959in}}%
\pgfpathcurveto{\pgfqpoint{1.542728in}{2.040135in}}{\pgfqpoint{1.550628in}{2.036863in}}{\pgfqpoint{1.558865in}{2.036863in}}%
\pgfpathclose%
\pgfusepath{stroke,fill}%
\end{pgfscope}%
\begin{pgfscope}%
\pgfpathrectangle{\pgfqpoint{0.100000in}{0.212622in}}{\pgfqpoint{3.696000in}{3.696000in}}%
\pgfusepath{clip}%
\pgfsetbuttcap%
\pgfsetroundjoin%
\definecolor{currentfill}{rgb}{0.121569,0.466667,0.705882}%
\pgfsetfillcolor{currentfill}%
\pgfsetfillopacity{0.300007}%
\pgfsetlinewidth{1.003750pt}%
\definecolor{currentstroke}{rgb}{0.121569,0.466667,0.705882}%
\pgfsetstrokecolor{currentstroke}%
\pgfsetstrokeopacity{0.300007}%
\pgfsetdash{}{0pt}%
\pgfpathmoveto{\pgfqpoint{1.558823in}{2.036854in}}%
\pgfpathcurveto{\pgfqpoint{1.567059in}{2.036854in}}{\pgfqpoint{1.574959in}{2.040126in}}{\pgfqpoint{1.580783in}{2.045950in}}%
\pgfpathcurveto{\pgfqpoint{1.586607in}{2.051774in}}{\pgfqpoint{1.589880in}{2.059674in}}{\pgfqpoint{1.589880in}{2.067910in}}%
\pgfpathcurveto{\pgfqpoint{1.589880in}{2.076146in}}{\pgfqpoint{1.586607in}{2.084046in}}{\pgfqpoint{1.580783in}{2.089870in}}%
\pgfpathcurveto{\pgfqpoint{1.574959in}{2.095694in}}{\pgfqpoint{1.567059in}{2.098967in}}{\pgfqpoint{1.558823in}{2.098967in}}%
\pgfpathcurveto{\pgfqpoint{1.550587in}{2.098967in}}{\pgfqpoint{1.542687in}{2.095694in}}{\pgfqpoint{1.536863in}{2.089870in}}%
\pgfpathcurveto{\pgfqpoint{1.531039in}{2.084046in}}{\pgfqpoint{1.527767in}{2.076146in}}{\pgfqpoint{1.527767in}{2.067910in}}%
\pgfpathcurveto{\pgfqpoint{1.527767in}{2.059674in}}{\pgfqpoint{1.531039in}{2.051774in}}{\pgfqpoint{1.536863in}{2.045950in}}%
\pgfpathcurveto{\pgfqpoint{1.542687in}{2.040126in}}{\pgfqpoint{1.550587in}{2.036854in}}{\pgfqpoint{1.558823in}{2.036854in}}%
\pgfpathclose%
\pgfusepath{stroke,fill}%
\end{pgfscope}%
\begin{pgfscope}%
\pgfpathrectangle{\pgfqpoint{0.100000in}{0.212622in}}{\pgfqpoint{3.696000in}{3.696000in}}%
\pgfusepath{clip}%
\pgfsetbuttcap%
\pgfsetroundjoin%
\definecolor{currentfill}{rgb}{0.121569,0.466667,0.705882}%
\pgfsetfillcolor{currentfill}%
\pgfsetfillopacity{0.300012}%
\pgfsetlinewidth{1.003750pt}%
\definecolor{currentstroke}{rgb}{0.121569,0.466667,0.705882}%
\pgfsetstrokecolor{currentstroke}%
\pgfsetstrokeopacity{0.300012}%
\pgfsetdash{}{0pt}%
\pgfpathmoveto{\pgfqpoint{1.558706in}{2.036820in}}%
\pgfpathcurveto{\pgfqpoint{1.566942in}{2.036820in}}{\pgfqpoint{1.574842in}{2.040093in}}{\pgfqpoint{1.580666in}{2.045916in}}%
\pgfpathcurveto{\pgfqpoint{1.586490in}{2.051740in}}{\pgfqpoint{1.589762in}{2.059640in}}{\pgfqpoint{1.589762in}{2.067877in}}%
\pgfpathcurveto{\pgfqpoint{1.589762in}{2.076113in}}{\pgfqpoint{1.586490in}{2.084013in}}{\pgfqpoint{1.580666in}{2.089837in}}%
\pgfpathcurveto{\pgfqpoint{1.574842in}{2.095661in}}{\pgfqpoint{1.566942in}{2.098933in}}{\pgfqpoint{1.558706in}{2.098933in}}%
\pgfpathcurveto{\pgfqpoint{1.550469in}{2.098933in}}{\pgfqpoint{1.542569in}{2.095661in}}{\pgfqpoint{1.536745in}{2.089837in}}%
\pgfpathcurveto{\pgfqpoint{1.530921in}{2.084013in}}{\pgfqpoint{1.527649in}{2.076113in}}{\pgfqpoint{1.527649in}{2.067877in}}%
\pgfpathcurveto{\pgfqpoint{1.527649in}{2.059640in}}{\pgfqpoint{1.530921in}{2.051740in}}{\pgfqpoint{1.536745in}{2.045916in}}%
\pgfpathcurveto{\pgfqpoint{1.542569in}{2.040093in}}{\pgfqpoint{1.550469in}{2.036820in}}{\pgfqpoint{1.558706in}{2.036820in}}%
\pgfpathclose%
\pgfusepath{stroke,fill}%
\end{pgfscope}%
\begin{pgfscope}%
\pgfpathrectangle{\pgfqpoint{0.100000in}{0.212622in}}{\pgfqpoint{3.696000in}{3.696000in}}%
\pgfusepath{clip}%
\pgfsetbuttcap%
\pgfsetroundjoin%
\definecolor{currentfill}{rgb}{0.121569,0.466667,0.705882}%
\pgfsetfillcolor{currentfill}%
\pgfsetfillopacity{0.300013}%
\pgfsetlinewidth{1.003750pt}%
\definecolor{currentstroke}{rgb}{0.121569,0.466667,0.705882}%
\pgfsetstrokecolor{currentstroke}%
\pgfsetstrokeopacity{0.300013}%
\pgfsetdash{}{0pt}%
\pgfpathmoveto{\pgfqpoint{1.558819in}{2.036828in}}%
\pgfpathcurveto{\pgfqpoint{1.567055in}{2.036828in}}{\pgfqpoint{1.574955in}{2.040100in}}{\pgfqpoint{1.580779in}{2.045924in}}%
\pgfpathcurveto{\pgfqpoint{1.586603in}{2.051748in}}{\pgfqpoint{1.589876in}{2.059648in}}{\pgfqpoint{1.589876in}{2.067884in}}%
\pgfpathcurveto{\pgfqpoint{1.589876in}{2.076121in}}{\pgfqpoint{1.586603in}{2.084021in}}{\pgfqpoint{1.580779in}{2.089845in}}%
\pgfpathcurveto{\pgfqpoint{1.574955in}{2.095668in}}{\pgfqpoint{1.567055in}{2.098941in}}{\pgfqpoint{1.558819in}{2.098941in}}%
\pgfpathcurveto{\pgfqpoint{1.550583in}{2.098941in}}{\pgfqpoint{1.542683in}{2.095668in}}{\pgfqpoint{1.536859in}{2.089845in}}%
\pgfpathcurveto{\pgfqpoint{1.531035in}{2.084021in}}{\pgfqpoint{1.527763in}{2.076121in}}{\pgfqpoint{1.527763in}{2.067884in}}%
\pgfpathcurveto{\pgfqpoint{1.527763in}{2.059648in}}{\pgfqpoint{1.531035in}{2.051748in}}{\pgfqpoint{1.536859in}{2.045924in}}%
\pgfpathcurveto{\pgfqpoint{1.542683in}{2.040100in}}{\pgfqpoint{1.550583in}{2.036828in}}{\pgfqpoint{1.558819in}{2.036828in}}%
\pgfpathclose%
\pgfusepath{stroke,fill}%
\end{pgfscope}%
\begin{pgfscope}%
\pgfpathrectangle{\pgfqpoint{0.100000in}{0.212622in}}{\pgfqpoint{3.696000in}{3.696000in}}%
\pgfusepath{clip}%
\pgfsetbuttcap%
\pgfsetroundjoin%
\definecolor{currentfill}{rgb}{0.121569,0.466667,0.705882}%
\pgfsetfillcolor{currentfill}%
\pgfsetfillopacity{0.300044}%
\pgfsetlinewidth{1.003750pt}%
\definecolor{currentstroke}{rgb}{0.121569,0.466667,0.705882}%
\pgfsetstrokecolor{currentstroke}%
\pgfsetstrokeopacity{0.300044}%
\pgfsetdash{}{0pt}%
\pgfpathmoveto{\pgfqpoint{1.558273in}{2.036683in}}%
\pgfpathcurveto{\pgfqpoint{1.566509in}{2.036683in}}{\pgfqpoint{1.574409in}{2.039955in}}{\pgfqpoint{1.580233in}{2.045779in}}%
\pgfpathcurveto{\pgfqpoint{1.586057in}{2.051603in}}{\pgfqpoint{1.589329in}{2.059503in}}{\pgfqpoint{1.589329in}{2.067740in}}%
\pgfpathcurveto{\pgfqpoint{1.589329in}{2.075976in}}{\pgfqpoint{1.586057in}{2.083876in}}{\pgfqpoint{1.580233in}{2.089700in}}%
\pgfpathcurveto{\pgfqpoint{1.574409in}{2.095524in}}{\pgfqpoint{1.566509in}{2.098796in}}{\pgfqpoint{1.558273in}{2.098796in}}%
\pgfpathcurveto{\pgfqpoint{1.550036in}{2.098796in}}{\pgfqpoint{1.542136in}{2.095524in}}{\pgfqpoint{1.536312in}{2.089700in}}%
\pgfpathcurveto{\pgfqpoint{1.530488in}{2.083876in}}{\pgfqpoint{1.527216in}{2.075976in}}{\pgfqpoint{1.527216in}{2.067740in}}%
\pgfpathcurveto{\pgfqpoint{1.527216in}{2.059503in}}{\pgfqpoint{1.530488in}{2.051603in}}{\pgfqpoint{1.536312in}{2.045779in}}%
\pgfpathcurveto{\pgfqpoint{1.542136in}{2.039955in}}{\pgfqpoint{1.550036in}{2.036683in}}{\pgfqpoint{1.558273in}{2.036683in}}%
\pgfpathclose%
\pgfusepath{stroke,fill}%
\end{pgfscope}%
\begin{pgfscope}%
\pgfpathrectangle{\pgfqpoint{0.100000in}{0.212622in}}{\pgfqpoint{3.696000in}{3.696000in}}%
\pgfusepath{clip}%
\pgfsetbuttcap%
\pgfsetroundjoin%
\definecolor{currentfill}{rgb}{0.121569,0.466667,0.705882}%
\pgfsetfillcolor{currentfill}%
\pgfsetfillopacity{0.300053}%
\pgfsetlinewidth{1.003750pt}%
\definecolor{currentstroke}{rgb}{0.121569,0.466667,0.705882}%
\pgfsetstrokecolor{currentstroke}%
\pgfsetstrokeopacity{0.300053}%
\pgfsetdash{}{0pt}%
\pgfpathmoveto{\pgfqpoint{1.558623in}{2.036667in}}%
\pgfpathcurveto{\pgfqpoint{1.566859in}{2.036667in}}{\pgfqpoint{1.574759in}{2.039940in}}{\pgfqpoint{1.580583in}{2.045764in}}%
\pgfpathcurveto{\pgfqpoint{1.586407in}{2.051588in}}{\pgfqpoint{1.589679in}{2.059488in}}{\pgfqpoint{1.589679in}{2.067724in}}%
\pgfpathcurveto{\pgfqpoint{1.589679in}{2.075960in}}{\pgfqpoint{1.586407in}{2.083860in}}{\pgfqpoint{1.580583in}{2.089684in}}%
\pgfpathcurveto{\pgfqpoint{1.574759in}{2.095508in}}{\pgfqpoint{1.566859in}{2.098780in}}{\pgfqpoint{1.558623in}{2.098780in}}%
\pgfpathcurveto{\pgfqpoint{1.550386in}{2.098780in}}{\pgfqpoint{1.542486in}{2.095508in}}{\pgfqpoint{1.536662in}{2.089684in}}%
\pgfpathcurveto{\pgfqpoint{1.530839in}{2.083860in}}{\pgfqpoint{1.527566in}{2.075960in}}{\pgfqpoint{1.527566in}{2.067724in}}%
\pgfpathcurveto{\pgfqpoint{1.527566in}{2.059488in}}{\pgfqpoint{1.530839in}{2.051588in}}{\pgfqpoint{1.536662in}{2.045764in}}%
\pgfpathcurveto{\pgfqpoint{1.542486in}{2.039940in}}{\pgfqpoint{1.550386in}{2.036667in}}{\pgfqpoint{1.558623in}{2.036667in}}%
\pgfpathclose%
\pgfusepath{stroke,fill}%
\end{pgfscope}%
\begin{pgfscope}%
\pgfpathrectangle{\pgfqpoint{0.100000in}{0.212622in}}{\pgfqpoint{3.696000in}{3.696000in}}%
\pgfusepath{clip}%
\pgfsetbuttcap%
\pgfsetroundjoin%
\definecolor{currentfill}{rgb}{0.121569,0.466667,0.705882}%
\pgfsetfillcolor{currentfill}%
\pgfsetfillopacity{0.300158}%
\pgfsetlinewidth{1.003750pt}%
\definecolor{currentstroke}{rgb}{0.121569,0.466667,0.705882}%
\pgfsetstrokecolor{currentstroke}%
\pgfsetstrokeopacity{0.300158}%
\pgfsetdash{}{0pt}%
\pgfpathmoveto{\pgfqpoint{1.558065in}{2.036232in}}%
\pgfpathcurveto{\pgfqpoint{1.566301in}{2.036232in}}{\pgfqpoint{1.574201in}{2.039505in}}{\pgfqpoint{1.580025in}{2.045328in}}%
\pgfpathcurveto{\pgfqpoint{1.585849in}{2.051152in}}{\pgfqpoint{1.589121in}{2.059052in}}{\pgfqpoint{1.589121in}{2.067289in}}%
\pgfpathcurveto{\pgfqpoint{1.589121in}{2.075525in}}{\pgfqpoint{1.585849in}{2.083425in}}{\pgfqpoint{1.580025in}{2.089249in}}%
\pgfpathcurveto{\pgfqpoint{1.574201in}{2.095073in}}{\pgfqpoint{1.566301in}{2.098345in}}{\pgfqpoint{1.558065in}{2.098345in}}%
\pgfpathcurveto{\pgfqpoint{1.549829in}{2.098345in}}{\pgfqpoint{1.541929in}{2.095073in}}{\pgfqpoint{1.536105in}{2.089249in}}%
\pgfpathcurveto{\pgfqpoint{1.530281in}{2.083425in}}{\pgfqpoint{1.527008in}{2.075525in}}{\pgfqpoint{1.527008in}{2.067289in}}%
\pgfpathcurveto{\pgfqpoint{1.527008in}{2.059052in}}{\pgfqpoint{1.530281in}{2.051152in}}{\pgfqpoint{1.536105in}{2.045328in}}%
\pgfpathcurveto{\pgfqpoint{1.541929in}{2.039505in}}{\pgfqpoint{1.549829in}{2.036232in}}{\pgfqpoint{1.558065in}{2.036232in}}%
\pgfpathclose%
\pgfusepath{stroke,fill}%
\end{pgfscope}%
\begin{pgfscope}%
\pgfpathrectangle{\pgfqpoint{0.100000in}{0.212622in}}{\pgfqpoint{3.696000in}{3.696000in}}%
\pgfusepath{clip}%
\pgfsetbuttcap%
\pgfsetroundjoin%
\definecolor{currentfill}{rgb}{0.121569,0.466667,0.705882}%
\pgfsetfillcolor{currentfill}%
\pgfsetfillopacity{0.300610}%
\pgfsetlinewidth{1.003750pt}%
\definecolor{currentstroke}{rgb}{0.121569,0.466667,0.705882}%
\pgfsetstrokecolor{currentstroke}%
\pgfsetstrokeopacity{0.300610}%
\pgfsetdash{}{0pt}%
\pgfpathmoveto{\pgfqpoint{1.555889in}{2.034538in}}%
\pgfpathcurveto{\pgfqpoint{1.564126in}{2.034538in}}{\pgfqpoint{1.572026in}{2.037811in}}{\pgfqpoint{1.577850in}{2.043635in}}%
\pgfpathcurveto{\pgfqpoint{1.583674in}{2.049459in}}{\pgfqpoint{1.586946in}{2.057359in}}{\pgfqpoint{1.586946in}{2.065595in}}%
\pgfpathcurveto{\pgfqpoint{1.586946in}{2.073831in}}{\pgfqpoint{1.583674in}{2.081731in}}{\pgfqpoint{1.577850in}{2.087555in}}%
\pgfpathcurveto{\pgfqpoint{1.572026in}{2.093379in}}{\pgfqpoint{1.564126in}{2.096651in}}{\pgfqpoint{1.555889in}{2.096651in}}%
\pgfpathcurveto{\pgfqpoint{1.547653in}{2.096651in}}{\pgfqpoint{1.539753in}{2.093379in}}{\pgfqpoint{1.533929in}{2.087555in}}%
\pgfpathcurveto{\pgfqpoint{1.528105in}{2.081731in}}{\pgfqpoint{1.524833in}{2.073831in}}{\pgfqpoint{1.524833in}{2.065595in}}%
\pgfpathcurveto{\pgfqpoint{1.524833in}{2.057359in}}{\pgfqpoint{1.528105in}{2.049459in}}{\pgfqpoint{1.533929in}{2.043635in}}%
\pgfpathcurveto{\pgfqpoint{1.539753in}{2.037811in}}{\pgfqpoint{1.547653in}{2.034538in}}{\pgfqpoint{1.555889in}{2.034538in}}%
\pgfpathclose%
\pgfusepath{stroke,fill}%
\end{pgfscope}%
\begin{pgfscope}%
\pgfpathrectangle{\pgfqpoint{0.100000in}{0.212622in}}{\pgfqpoint{3.696000in}{3.696000in}}%
\pgfusepath{clip}%
\pgfsetbuttcap%
\pgfsetroundjoin%
\definecolor{currentfill}{rgb}{0.121569,0.466667,0.705882}%
\pgfsetfillcolor{currentfill}%
\pgfsetfillopacity{0.300610}%
\pgfsetlinewidth{1.003750pt}%
\definecolor{currentstroke}{rgb}{0.121569,0.466667,0.705882}%
\pgfsetstrokecolor{currentstroke}%
\pgfsetstrokeopacity{0.300610}%
\pgfsetdash{}{0pt}%
\pgfpathmoveto{\pgfqpoint{1.555889in}{2.034538in}}%
\pgfpathcurveto{\pgfqpoint{1.564126in}{2.034538in}}{\pgfqpoint{1.572026in}{2.037811in}}{\pgfqpoint{1.577850in}{2.043635in}}%
\pgfpathcurveto{\pgfqpoint{1.583674in}{2.049459in}}{\pgfqpoint{1.586946in}{2.057359in}}{\pgfqpoint{1.586946in}{2.065595in}}%
\pgfpathcurveto{\pgfqpoint{1.586946in}{2.073831in}}{\pgfqpoint{1.583674in}{2.081731in}}{\pgfqpoint{1.577850in}{2.087555in}}%
\pgfpathcurveto{\pgfqpoint{1.572026in}{2.093379in}}{\pgfqpoint{1.564126in}{2.096651in}}{\pgfqpoint{1.555889in}{2.096651in}}%
\pgfpathcurveto{\pgfqpoint{1.547653in}{2.096651in}}{\pgfqpoint{1.539753in}{2.093379in}}{\pgfqpoint{1.533929in}{2.087555in}}%
\pgfpathcurveto{\pgfqpoint{1.528105in}{2.081731in}}{\pgfqpoint{1.524833in}{2.073831in}}{\pgfqpoint{1.524833in}{2.065595in}}%
\pgfpathcurveto{\pgfqpoint{1.524833in}{2.057359in}}{\pgfqpoint{1.528105in}{2.049459in}}{\pgfqpoint{1.533929in}{2.043635in}}%
\pgfpathcurveto{\pgfqpoint{1.539753in}{2.037811in}}{\pgfqpoint{1.547653in}{2.034538in}}{\pgfqpoint{1.555889in}{2.034538in}}%
\pgfpathclose%
\pgfusepath{stroke,fill}%
\end{pgfscope}%
\begin{pgfscope}%
\pgfpathrectangle{\pgfqpoint{0.100000in}{0.212622in}}{\pgfqpoint{3.696000in}{3.696000in}}%
\pgfusepath{clip}%
\pgfsetbuttcap%
\pgfsetroundjoin%
\definecolor{currentfill}{rgb}{0.121569,0.466667,0.705882}%
\pgfsetfillcolor{currentfill}%
\pgfsetfillopacity{0.300610}%
\pgfsetlinewidth{1.003750pt}%
\definecolor{currentstroke}{rgb}{0.121569,0.466667,0.705882}%
\pgfsetstrokecolor{currentstroke}%
\pgfsetstrokeopacity{0.300610}%
\pgfsetdash{}{0pt}%
\pgfpathmoveto{\pgfqpoint{1.555889in}{2.034538in}}%
\pgfpathcurveto{\pgfqpoint{1.564126in}{2.034538in}}{\pgfqpoint{1.572026in}{2.037811in}}{\pgfqpoint{1.577850in}{2.043635in}}%
\pgfpathcurveto{\pgfqpoint{1.583674in}{2.049459in}}{\pgfqpoint{1.586946in}{2.057359in}}{\pgfqpoint{1.586946in}{2.065595in}}%
\pgfpathcurveto{\pgfqpoint{1.586946in}{2.073831in}}{\pgfqpoint{1.583674in}{2.081731in}}{\pgfqpoint{1.577850in}{2.087555in}}%
\pgfpathcurveto{\pgfqpoint{1.572026in}{2.093379in}}{\pgfqpoint{1.564126in}{2.096651in}}{\pgfqpoint{1.555889in}{2.096651in}}%
\pgfpathcurveto{\pgfqpoint{1.547653in}{2.096651in}}{\pgfqpoint{1.539753in}{2.093379in}}{\pgfqpoint{1.533929in}{2.087555in}}%
\pgfpathcurveto{\pgfqpoint{1.528105in}{2.081731in}}{\pgfqpoint{1.524833in}{2.073831in}}{\pgfqpoint{1.524833in}{2.065595in}}%
\pgfpathcurveto{\pgfqpoint{1.524833in}{2.057359in}}{\pgfqpoint{1.528105in}{2.049459in}}{\pgfqpoint{1.533929in}{2.043635in}}%
\pgfpathcurveto{\pgfqpoint{1.539753in}{2.037811in}}{\pgfqpoint{1.547653in}{2.034538in}}{\pgfqpoint{1.555889in}{2.034538in}}%
\pgfpathclose%
\pgfusepath{stroke,fill}%
\end{pgfscope}%
\begin{pgfscope}%
\pgfpathrectangle{\pgfqpoint{0.100000in}{0.212622in}}{\pgfqpoint{3.696000in}{3.696000in}}%
\pgfusepath{clip}%
\pgfsetbuttcap%
\pgfsetroundjoin%
\definecolor{currentfill}{rgb}{0.121569,0.466667,0.705882}%
\pgfsetfillcolor{currentfill}%
\pgfsetfillopacity{0.300610}%
\pgfsetlinewidth{1.003750pt}%
\definecolor{currentstroke}{rgb}{0.121569,0.466667,0.705882}%
\pgfsetstrokecolor{currentstroke}%
\pgfsetstrokeopacity{0.300610}%
\pgfsetdash{}{0pt}%
\pgfpathmoveto{\pgfqpoint{1.555889in}{2.034538in}}%
\pgfpathcurveto{\pgfqpoint{1.564126in}{2.034538in}}{\pgfqpoint{1.572026in}{2.037811in}}{\pgfqpoint{1.577850in}{2.043635in}}%
\pgfpathcurveto{\pgfqpoint{1.583674in}{2.049459in}}{\pgfqpoint{1.586946in}{2.057359in}}{\pgfqpoint{1.586946in}{2.065595in}}%
\pgfpathcurveto{\pgfqpoint{1.586946in}{2.073831in}}{\pgfqpoint{1.583674in}{2.081731in}}{\pgfqpoint{1.577850in}{2.087555in}}%
\pgfpathcurveto{\pgfqpoint{1.572026in}{2.093379in}}{\pgfqpoint{1.564126in}{2.096651in}}{\pgfqpoint{1.555889in}{2.096651in}}%
\pgfpathcurveto{\pgfqpoint{1.547653in}{2.096651in}}{\pgfqpoint{1.539753in}{2.093379in}}{\pgfqpoint{1.533929in}{2.087555in}}%
\pgfpathcurveto{\pgfqpoint{1.528105in}{2.081731in}}{\pgfqpoint{1.524833in}{2.073831in}}{\pgfqpoint{1.524833in}{2.065595in}}%
\pgfpathcurveto{\pgfqpoint{1.524833in}{2.057359in}}{\pgfqpoint{1.528105in}{2.049459in}}{\pgfqpoint{1.533929in}{2.043635in}}%
\pgfpathcurveto{\pgfqpoint{1.539753in}{2.037811in}}{\pgfqpoint{1.547653in}{2.034538in}}{\pgfqpoint{1.555889in}{2.034538in}}%
\pgfpathclose%
\pgfusepath{stroke,fill}%
\end{pgfscope}%
\begin{pgfscope}%
\pgfpathrectangle{\pgfqpoint{0.100000in}{0.212622in}}{\pgfqpoint{3.696000in}{3.696000in}}%
\pgfusepath{clip}%
\pgfsetbuttcap%
\pgfsetroundjoin%
\definecolor{currentfill}{rgb}{0.121569,0.466667,0.705882}%
\pgfsetfillcolor{currentfill}%
\pgfsetfillopacity{0.300610}%
\pgfsetlinewidth{1.003750pt}%
\definecolor{currentstroke}{rgb}{0.121569,0.466667,0.705882}%
\pgfsetstrokecolor{currentstroke}%
\pgfsetstrokeopacity{0.300610}%
\pgfsetdash{}{0pt}%
\pgfpathmoveto{\pgfqpoint{1.555889in}{2.034538in}}%
\pgfpathcurveto{\pgfqpoint{1.564126in}{2.034538in}}{\pgfqpoint{1.572026in}{2.037811in}}{\pgfqpoint{1.577850in}{2.043635in}}%
\pgfpathcurveto{\pgfqpoint{1.583674in}{2.049459in}}{\pgfqpoint{1.586946in}{2.057359in}}{\pgfqpoint{1.586946in}{2.065595in}}%
\pgfpathcurveto{\pgfqpoint{1.586946in}{2.073831in}}{\pgfqpoint{1.583674in}{2.081731in}}{\pgfqpoint{1.577850in}{2.087555in}}%
\pgfpathcurveto{\pgfqpoint{1.572026in}{2.093379in}}{\pgfqpoint{1.564126in}{2.096651in}}{\pgfqpoint{1.555889in}{2.096651in}}%
\pgfpathcurveto{\pgfqpoint{1.547653in}{2.096651in}}{\pgfqpoint{1.539753in}{2.093379in}}{\pgfqpoint{1.533929in}{2.087555in}}%
\pgfpathcurveto{\pgfqpoint{1.528105in}{2.081731in}}{\pgfqpoint{1.524833in}{2.073831in}}{\pgfqpoint{1.524833in}{2.065595in}}%
\pgfpathcurveto{\pgfqpoint{1.524833in}{2.057359in}}{\pgfqpoint{1.528105in}{2.049459in}}{\pgfqpoint{1.533929in}{2.043635in}}%
\pgfpathcurveto{\pgfqpoint{1.539753in}{2.037811in}}{\pgfqpoint{1.547653in}{2.034538in}}{\pgfqpoint{1.555889in}{2.034538in}}%
\pgfpathclose%
\pgfusepath{stroke,fill}%
\end{pgfscope}%
\begin{pgfscope}%
\pgfpathrectangle{\pgfqpoint{0.100000in}{0.212622in}}{\pgfqpoint{3.696000in}{3.696000in}}%
\pgfusepath{clip}%
\pgfsetbuttcap%
\pgfsetroundjoin%
\definecolor{currentfill}{rgb}{0.121569,0.466667,0.705882}%
\pgfsetfillcolor{currentfill}%
\pgfsetfillopacity{0.300610}%
\pgfsetlinewidth{1.003750pt}%
\definecolor{currentstroke}{rgb}{0.121569,0.466667,0.705882}%
\pgfsetstrokecolor{currentstroke}%
\pgfsetstrokeopacity{0.300610}%
\pgfsetdash{}{0pt}%
\pgfpathmoveto{\pgfqpoint{1.555889in}{2.034538in}}%
\pgfpathcurveto{\pgfqpoint{1.564126in}{2.034538in}}{\pgfqpoint{1.572026in}{2.037811in}}{\pgfqpoint{1.577850in}{2.043635in}}%
\pgfpathcurveto{\pgfqpoint{1.583674in}{2.049459in}}{\pgfqpoint{1.586946in}{2.057359in}}{\pgfqpoint{1.586946in}{2.065595in}}%
\pgfpathcurveto{\pgfqpoint{1.586946in}{2.073831in}}{\pgfqpoint{1.583674in}{2.081731in}}{\pgfqpoint{1.577850in}{2.087555in}}%
\pgfpathcurveto{\pgfqpoint{1.572026in}{2.093379in}}{\pgfqpoint{1.564126in}{2.096651in}}{\pgfqpoint{1.555889in}{2.096651in}}%
\pgfpathcurveto{\pgfqpoint{1.547653in}{2.096651in}}{\pgfqpoint{1.539753in}{2.093379in}}{\pgfqpoint{1.533929in}{2.087555in}}%
\pgfpathcurveto{\pgfqpoint{1.528105in}{2.081731in}}{\pgfqpoint{1.524833in}{2.073831in}}{\pgfqpoint{1.524833in}{2.065595in}}%
\pgfpathcurveto{\pgfqpoint{1.524833in}{2.057359in}}{\pgfqpoint{1.528105in}{2.049459in}}{\pgfqpoint{1.533929in}{2.043635in}}%
\pgfpathcurveto{\pgfqpoint{1.539753in}{2.037811in}}{\pgfqpoint{1.547653in}{2.034538in}}{\pgfqpoint{1.555889in}{2.034538in}}%
\pgfpathclose%
\pgfusepath{stroke,fill}%
\end{pgfscope}%
\begin{pgfscope}%
\pgfpathrectangle{\pgfqpoint{0.100000in}{0.212622in}}{\pgfqpoint{3.696000in}{3.696000in}}%
\pgfusepath{clip}%
\pgfsetbuttcap%
\pgfsetroundjoin%
\definecolor{currentfill}{rgb}{0.121569,0.466667,0.705882}%
\pgfsetfillcolor{currentfill}%
\pgfsetfillopacity{0.300610}%
\pgfsetlinewidth{1.003750pt}%
\definecolor{currentstroke}{rgb}{0.121569,0.466667,0.705882}%
\pgfsetstrokecolor{currentstroke}%
\pgfsetstrokeopacity{0.300610}%
\pgfsetdash{}{0pt}%
\pgfpathmoveto{\pgfqpoint{1.555889in}{2.034538in}}%
\pgfpathcurveto{\pgfqpoint{1.564126in}{2.034538in}}{\pgfqpoint{1.572026in}{2.037811in}}{\pgfqpoint{1.577850in}{2.043635in}}%
\pgfpathcurveto{\pgfqpoint{1.583674in}{2.049459in}}{\pgfqpoint{1.586946in}{2.057359in}}{\pgfqpoint{1.586946in}{2.065595in}}%
\pgfpathcurveto{\pgfqpoint{1.586946in}{2.073831in}}{\pgfqpoint{1.583674in}{2.081731in}}{\pgfqpoint{1.577850in}{2.087555in}}%
\pgfpathcurveto{\pgfqpoint{1.572026in}{2.093379in}}{\pgfqpoint{1.564126in}{2.096651in}}{\pgfqpoint{1.555889in}{2.096651in}}%
\pgfpathcurveto{\pgfqpoint{1.547653in}{2.096651in}}{\pgfqpoint{1.539753in}{2.093379in}}{\pgfqpoint{1.533929in}{2.087555in}}%
\pgfpathcurveto{\pgfqpoint{1.528105in}{2.081731in}}{\pgfqpoint{1.524833in}{2.073831in}}{\pgfqpoint{1.524833in}{2.065595in}}%
\pgfpathcurveto{\pgfqpoint{1.524833in}{2.057359in}}{\pgfqpoint{1.528105in}{2.049459in}}{\pgfqpoint{1.533929in}{2.043635in}}%
\pgfpathcurveto{\pgfqpoint{1.539753in}{2.037811in}}{\pgfqpoint{1.547653in}{2.034538in}}{\pgfqpoint{1.555889in}{2.034538in}}%
\pgfpathclose%
\pgfusepath{stroke,fill}%
\end{pgfscope}%
\begin{pgfscope}%
\pgfpathrectangle{\pgfqpoint{0.100000in}{0.212622in}}{\pgfqpoint{3.696000in}{3.696000in}}%
\pgfusepath{clip}%
\pgfsetbuttcap%
\pgfsetroundjoin%
\definecolor{currentfill}{rgb}{0.121569,0.466667,0.705882}%
\pgfsetfillcolor{currentfill}%
\pgfsetfillopacity{0.300610}%
\pgfsetlinewidth{1.003750pt}%
\definecolor{currentstroke}{rgb}{0.121569,0.466667,0.705882}%
\pgfsetstrokecolor{currentstroke}%
\pgfsetstrokeopacity{0.300610}%
\pgfsetdash{}{0pt}%
\pgfpathmoveto{\pgfqpoint{1.555889in}{2.034538in}}%
\pgfpathcurveto{\pgfqpoint{1.564126in}{2.034538in}}{\pgfqpoint{1.572026in}{2.037811in}}{\pgfqpoint{1.577850in}{2.043635in}}%
\pgfpathcurveto{\pgfqpoint{1.583674in}{2.049459in}}{\pgfqpoint{1.586946in}{2.057359in}}{\pgfqpoint{1.586946in}{2.065595in}}%
\pgfpathcurveto{\pgfqpoint{1.586946in}{2.073831in}}{\pgfqpoint{1.583674in}{2.081731in}}{\pgfqpoint{1.577850in}{2.087555in}}%
\pgfpathcurveto{\pgfqpoint{1.572026in}{2.093379in}}{\pgfqpoint{1.564126in}{2.096651in}}{\pgfqpoint{1.555889in}{2.096651in}}%
\pgfpathcurveto{\pgfqpoint{1.547653in}{2.096651in}}{\pgfqpoint{1.539753in}{2.093379in}}{\pgfqpoint{1.533929in}{2.087555in}}%
\pgfpathcurveto{\pgfqpoint{1.528105in}{2.081731in}}{\pgfqpoint{1.524833in}{2.073831in}}{\pgfqpoint{1.524833in}{2.065595in}}%
\pgfpathcurveto{\pgfqpoint{1.524833in}{2.057359in}}{\pgfqpoint{1.528105in}{2.049459in}}{\pgfqpoint{1.533929in}{2.043635in}}%
\pgfpathcurveto{\pgfqpoint{1.539753in}{2.037811in}}{\pgfqpoint{1.547653in}{2.034538in}}{\pgfqpoint{1.555889in}{2.034538in}}%
\pgfpathclose%
\pgfusepath{stroke,fill}%
\end{pgfscope}%
\begin{pgfscope}%
\pgfpathrectangle{\pgfqpoint{0.100000in}{0.212622in}}{\pgfqpoint{3.696000in}{3.696000in}}%
\pgfusepath{clip}%
\pgfsetbuttcap%
\pgfsetroundjoin%
\definecolor{currentfill}{rgb}{0.121569,0.466667,0.705882}%
\pgfsetfillcolor{currentfill}%
\pgfsetfillopacity{0.300610}%
\pgfsetlinewidth{1.003750pt}%
\definecolor{currentstroke}{rgb}{0.121569,0.466667,0.705882}%
\pgfsetstrokecolor{currentstroke}%
\pgfsetstrokeopacity{0.300610}%
\pgfsetdash{}{0pt}%
\pgfpathmoveto{\pgfqpoint{1.555889in}{2.034538in}}%
\pgfpathcurveto{\pgfqpoint{1.564126in}{2.034538in}}{\pgfqpoint{1.572026in}{2.037811in}}{\pgfqpoint{1.577850in}{2.043635in}}%
\pgfpathcurveto{\pgfqpoint{1.583674in}{2.049459in}}{\pgfqpoint{1.586946in}{2.057359in}}{\pgfqpoint{1.586946in}{2.065595in}}%
\pgfpathcurveto{\pgfqpoint{1.586946in}{2.073831in}}{\pgfqpoint{1.583674in}{2.081731in}}{\pgfqpoint{1.577850in}{2.087555in}}%
\pgfpathcurveto{\pgfqpoint{1.572026in}{2.093379in}}{\pgfqpoint{1.564126in}{2.096651in}}{\pgfqpoint{1.555889in}{2.096651in}}%
\pgfpathcurveto{\pgfqpoint{1.547653in}{2.096651in}}{\pgfqpoint{1.539753in}{2.093379in}}{\pgfqpoint{1.533929in}{2.087555in}}%
\pgfpathcurveto{\pgfqpoint{1.528105in}{2.081731in}}{\pgfqpoint{1.524833in}{2.073831in}}{\pgfqpoint{1.524833in}{2.065595in}}%
\pgfpathcurveto{\pgfqpoint{1.524833in}{2.057359in}}{\pgfqpoint{1.528105in}{2.049459in}}{\pgfqpoint{1.533929in}{2.043635in}}%
\pgfpathcurveto{\pgfqpoint{1.539753in}{2.037811in}}{\pgfqpoint{1.547653in}{2.034538in}}{\pgfqpoint{1.555889in}{2.034538in}}%
\pgfpathclose%
\pgfusepath{stroke,fill}%
\end{pgfscope}%
\begin{pgfscope}%
\pgfpathrectangle{\pgfqpoint{0.100000in}{0.212622in}}{\pgfqpoint{3.696000in}{3.696000in}}%
\pgfusepath{clip}%
\pgfsetbuttcap%
\pgfsetroundjoin%
\definecolor{currentfill}{rgb}{0.121569,0.466667,0.705882}%
\pgfsetfillcolor{currentfill}%
\pgfsetfillopacity{0.300610}%
\pgfsetlinewidth{1.003750pt}%
\definecolor{currentstroke}{rgb}{0.121569,0.466667,0.705882}%
\pgfsetstrokecolor{currentstroke}%
\pgfsetstrokeopacity{0.300610}%
\pgfsetdash{}{0pt}%
\pgfpathmoveto{\pgfqpoint{1.555889in}{2.034538in}}%
\pgfpathcurveto{\pgfqpoint{1.564126in}{2.034538in}}{\pgfqpoint{1.572026in}{2.037811in}}{\pgfqpoint{1.577850in}{2.043635in}}%
\pgfpathcurveto{\pgfqpoint{1.583674in}{2.049459in}}{\pgfqpoint{1.586946in}{2.057359in}}{\pgfqpoint{1.586946in}{2.065595in}}%
\pgfpathcurveto{\pgfqpoint{1.586946in}{2.073831in}}{\pgfqpoint{1.583674in}{2.081731in}}{\pgfqpoint{1.577850in}{2.087555in}}%
\pgfpathcurveto{\pgfqpoint{1.572026in}{2.093379in}}{\pgfqpoint{1.564126in}{2.096651in}}{\pgfqpoint{1.555889in}{2.096651in}}%
\pgfpathcurveto{\pgfqpoint{1.547653in}{2.096651in}}{\pgfqpoint{1.539753in}{2.093379in}}{\pgfqpoint{1.533929in}{2.087555in}}%
\pgfpathcurveto{\pgfqpoint{1.528105in}{2.081731in}}{\pgfqpoint{1.524833in}{2.073831in}}{\pgfqpoint{1.524833in}{2.065595in}}%
\pgfpathcurveto{\pgfqpoint{1.524833in}{2.057359in}}{\pgfqpoint{1.528105in}{2.049459in}}{\pgfqpoint{1.533929in}{2.043635in}}%
\pgfpathcurveto{\pgfqpoint{1.539753in}{2.037811in}}{\pgfqpoint{1.547653in}{2.034538in}}{\pgfqpoint{1.555889in}{2.034538in}}%
\pgfpathclose%
\pgfusepath{stroke,fill}%
\end{pgfscope}%
\begin{pgfscope}%
\pgfpathrectangle{\pgfqpoint{0.100000in}{0.212622in}}{\pgfqpoint{3.696000in}{3.696000in}}%
\pgfusepath{clip}%
\pgfsetbuttcap%
\pgfsetroundjoin%
\definecolor{currentfill}{rgb}{0.121569,0.466667,0.705882}%
\pgfsetfillcolor{currentfill}%
\pgfsetfillopacity{0.300610}%
\pgfsetlinewidth{1.003750pt}%
\definecolor{currentstroke}{rgb}{0.121569,0.466667,0.705882}%
\pgfsetstrokecolor{currentstroke}%
\pgfsetstrokeopacity{0.300610}%
\pgfsetdash{}{0pt}%
\pgfpathmoveto{\pgfqpoint{1.555889in}{2.034538in}}%
\pgfpathcurveto{\pgfqpoint{1.564126in}{2.034538in}}{\pgfqpoint{1.572026in}{2.037811in}}{\pgfqpoint{1.577850in}{2.043635in}}%
\pgfpathcurveto{\pgfqpoint{1.583674in}{2.049459in}}{\pgfqpoint{1.586946in}{2.057359in}}{\pgfqpoint{1.586946in}{2.065595in}}%
\pgfpathcurveto{\pgfqpoint{1.586946in}{2.073831in}}{\pgfqpoint{1.583674in}{2.081731in}}{\pgfqpoint{1.577850in}{2.087555in}}%
\pgfpathcurveto{\pgfqpoint{1.572026in}{2.093379in}}{\pgfqpoint{1.564126in}{2.096651in}}{\pgfqpoint{1.555889in}{2.096651in}}%
\pgfpathcurveto{\pgfqpoint{1.547653in}{2.096651in}}{\pgfqpoint{1.539753in}{2.093379in}}{\pgfqpoint{1.533929in}{2.087555in}}%
\pgfpathcurveto{\pgfqpoint{1.528105in}{2.081731in}}{\pgfqpoint{1.524833in}{2.073831in}}{\pgfqpoint{1.524833in}{2.065595in}}%
\pgfpathcurveto{\pgfqpoint{1.524833in}{2.057359in}}{\pgfqpoint{1.528105in}{2.049459in}}{\pgfqpoint{1.533929in}{2.043635in}}%
\pgfpathcurveto{\pgfqpoint{1.539753in}{2.037811in}}{\pgfqpoint{1.547653in}{2.034538in}}{\pgfqpoint{1.555889in}{2.034538in}}%
\pgfpathclose%
\pgfusepath{stroke,fill}%
\end{pgfscope}%
\begin{pgfscope}%
\pgfpathrectangle{\pgfqpoint{0.100000in}{0.212622in}}{\pgfqpoint{3.696000in}{3.696000in}}%
\pgfusepath{clip}%
\pgfsetbuttcap%
\pgfsetroundjoin%
\definecolor{currentfill}{rgb}{0.121569,0.466667,0.705882}%
\pgfsetfillcolor{currentfill}%
\pgfsetfillopacity{0.300610}%
\pgfsetlinewidth{1.003750pt}%
\definecolor{currentstroke}{rgb}{0.121569,0.466667,0.705882}%
\pgfsetstrokecolor{currentstroke}%
\pgfsetstrokeopacity{0.300610}%
\pgfsetdash{}{0pt}%
\pgfpathmoveto{\pgfqpoint{1.555889in}{2.034538in}}%
\pgfpathcurveto{\pgfqpoint{1.564126in}{2.034538in}}{\pgfqpoint{1.572026in}{2.037811in}}{\pgfqpoint{1.577850in}{2.043635in}}%
\pgfpathcurveto{\pgfqpoint{1.583674in}{2.049459in}}{\pgfqpoint{1.586946in}{2.057359in}}{\pgfqpoint{1.586946in}{2.065595in}}%
\pgfpathcurveto{\pgfqpoint{1.586946in}{2.073831in}}{\pgfqpoint{1.583674in}{2.081731in}}{\pgfqpoint{1.577850in}{2.087555in}}%
\pgfpathcurveto{\pgfqpoint{1.572026in}{2.093379in}}{\pgfqpoint{1.564126in}{2.096651in}}{\pgfqpoint{1.555889in}{2.096651in}}%
\pgfpathcurveto{\pgfqpoint{1.547653in}{2.096651in}}{\pgfqpoint{1.539753in}{2.093379in}}{\pgfqpoint{1.533929in}{2.087555in}}%
\pgfpathcurveto{\pgfqpoint{1.528105in}{2.081731in}}{\pgfqpoint{1.524833in}{2.073831in}}{\pgfqpoint{1.524833in}{2.065595in}}%
\pgfpathcurveto{\pgfqpoint{1.524833in}{2.057359in}}{\pgfqpoint{1.528105in}{2.049459in}}{\pgfqpoint{1.533929in}{2.043635in}}%
\pgfpathcurveto{\pgfqpoint{1.539753in}{2.037811in}}{\pgfqpoint{1.547653in}{2.034538in}}{\pgfqpoint{1.555889in}{2.034538in}}%
\pgfpathclose%
\pgfusepath{stroke,fill}%
\end{pgfscope}%
\begin{pgfscope}%
\pgfpathrectangle{\pgfqpoint{0.100000in}{0.212622in}}{\pgfqpoint{3.696000in}{3.696000in}}%
\pgfusepath{clip}%
\pgfsetbuttcap%
\pgfsetroundjoin%
\definecolor{currentfill}{rgb}{0.121569,0.466667,0.705882}%
\pgfsetfillcolor{currentfill}%
\pgfsetfillopacity{0.300610}%
\pgfsetlinewidth{1.003750pt}%
\definecolor{currentstroke}{rgb}{0.121569,0.466667,0.705882}%
\pgfsetstrokecolor{currentstroke}%
\pgfsetstrokeopacity{0.300610}%
\pgfsetdash{}{0pt}%
\pgfpathmoveto{\pgfqpoint{1.555889in}{2.034538in}}%
\pgfpathcurveto{\pgfqpoint{1.564126in}{2.034538in}}{\pgfqpoint{1.572026in}{2.037811in}}{\pgfqpoint{1.577850in}{2.043635in}}%
\pgfpathcurveto{\pgfqpoint{1.583674in}{2.049459in}}{\pgfqpoint{1.586946in}{2.057359in}}{\pgfqpoint{1.586946in}{2.065595in}}%
\pgfpathcurveto{\pgfqpoint{1.586946in}{2.073831in}}{\pgfqpoint{1.583674in}{2.081731in}}{\pgfqpoint{1.577850in}{2.087555in}}%
\pgfpathcurveto{\pgfqpoint{1.572026in}{2.093379in}}{\pgfqpoint{1.564126in}{2.096651in}}{\pgfqpoint{1.555889in}{2.096651in}}%
\pgfpathcurveto{\pgfqpoint{1.547653in}{2.096651in}}{\pgfqpoint{1.539753in}{2.093379in}}{\pgfqpoint{1.533929in}{2.087555in}}%
\pgfpathcurveto{\pgfqpoint{1.528105in}{2.081731in}}{\pgfqpoint{1.524833in}{2.073831in}}{\pgfqpoint{1.524833in}{2.065595in}}%
\pgfpathcurveto{\pgfqpoint{1.524833in}{2.057359in}}{\pgfqpoint{1.528105in}{2.049459in}}{\pgfqpoint{1.533929in}{2.043635in}}%
\pgfpathcurveto{\pgfqpoint{1.539753in}{2.037811in}}{\pgfqpoint{1.547653in}{2.034538in}}{\pgfqpoint{1.555889in}{2.034538in}}%
\pgfpathclose%
\pgfusepath{stroke,fill}%
\end{pgfscope}%
\begin{pgfscope}%
\pgfpathrectangle{\pgfqpoint{0.100000in}{0.212622in}}{\pgfqpoint{3.696000in}{3.696000in}}%
\pgfusepath{clip}%
\pgfsetbuttcap%
\pgfsetroundjoin%
\definecolor{currentfill}{rgb}{0.121569,0.466667,0.705882}%
\pgfsetfillcolor{currentfill}%
\pgfsetfillopacity{0.300610}%
\pgfsetlinewidth{1.003750pt}%
\definecolor{currentstroke}{rgb}{0.121569,0.466667,0.705882}%
\pgfsetstrokecolor{currentstroke}%
\pgfsetstrokeopacity{0.300610}%
\pgfsetdash{}{0pt}%
\pgfpathmoveto{\pgfqpoint{1.555889in}{2.034538in}}%
\pgfpathcurveto{\pgfqpoint{1.564126in}{2.034538in}}{\pgfqpoint{1.572026in}{2.037811in}}{\pgfqpoint{1.577850in}{2.043635in}}%
\pgfpathcurveto{\pgfqpoint{1.583674in}{2.049459in}}{\pgfqpoint{1.586946in}{2.057359in}}{\pgfqpoint{1.586946in}{2.065595in}}%
\pgfpathcurveto{\pgfqpoint{1.586946in}{2.073831in}}{\pgfqpoint{1.583674in}{2.081731in}}{\pgfqpoint{1.577850in}{2.087555in}}%
\pgfpathcurveto{\pgfqpoint{1.572026in}{2.093379in}}{\pgfqpoint{1.564126in}{2.096651in}}{\pgfqpoint{1.555889in}{2.096651in}}%
\pgfpathcurveto{\pgfqpoint{1.547653in}{2.096651in}}{\pgfqpoint{1.539753in}{2.093379in}}{\pgfqpoint{1.533929in}{2.087555in}}%
\pgfpathcurveto{\pgfqpoint{1.528105in}{2.081731in}}{\pgfqpoint{1.524833in}{2.073831in}}{\pgfqpoint{1.524833in}{2.065595in}}%
\pgfpathcurveto{\pgfqpoint{1.524833in}{2.057359in}}{\pgfqpoint{1.528105in}{2.049459in}}{\pgfqpoint{1.533929in}{2.043635in}}%
\pgfpathcurveto{\pgfqpoint{1.539753in}{2.037811in}}{\pgfqpoint{1.547653in}{2.034538in}}{\pgfqpoint{1.555889in}{2.034538in}}%
\pgfpathclose%
\pgfusepath{stroke,fill}%
\end{pgfscope}%
\begin{pgfscope}%
\pgfpathrectangle{\pgfqpoint{0.100000in}{0.212622in}}{\pgfqpoint{3.696000in}{3.696000in}}%
\pgfusepath{clip}%
\pgfsetbuttcap%
\pgfsetroundjoin%
\definecolor{currentfill}{rgb}{0.121569,0.466667,0.705882}%
\pgfsetfillcolor{currentfill}%
\pgfsetfillopacity{0.300610}%
\pgfsetlinewidth{1.003750pt}%
\definecolor{currentstroke}{rgb}{0.121569,0.466667,0.705882}%
\pgfsetstrokecolor{currentstroke}%
\pgfsetstrokeopacity{0.300610}%
\pgfsetdash{}{0pt}%
\pgfpathmoveto{\pgfqpoint{1.555889in}{2.034538in}}%
\pgfpathcurveto{\pgfqpoint{1.564126in}{2.034538in}}{\pgfqpoint{1.572026in}{2.037811in}}{\pgfqpoint{1.577850in}{2.043635in}}%
\pgfpathcurveto{\pgfqpoint{1.583674in}{2.049459in}}{\pgfqpoint{1.586946in}{2.057359in}}{\pgfqpoint{1.586946in}{2.065595in}}%
\pgfpathcurveto{\pgfqpoint{1.586946in}{2.073831in}}{\pgfqpoint{1.583674in}{2.081731in}}{\pgfqpoint{1.577850in}{2.087555in}}%
\pgfpathcurveto{\pgfqpoint{1.572026in}{2.093379in}}{\pgfqpoint{1.564126in}{2.096651in}}{\pgfqpoint{1.555889in}{2.096651in}}%
\pgfpathcurveto{\pgfqpoint{1.547653in}{2.096651in}}{\pgfqpoint{1.539753in}{2.093379in}}{\pgfqpoint{1.533929in}{2.087555in}}%
\pgfpathcurveto{\pgfqpoint{1.528105in}{2.081731in}}{\pgfqpoint{1.524833in}{2.073831in}}{\pgfqpoint{1.524833in}{2.065595in}}%
\pgfpathcurveto{\pgfqpoint{1.524833in}{2.057359in}}{\pgfqpoint{1.528105in}{2.049459in}}{\pgfqpoint{1.533929in}{2.043635in}}%
\pgfpathcurveto{\pgfqpoint{1.539753in}{2.037811in}}{\pgfqpoint{1.547653in}{2.034538in}}{\pgfqpoint{1.555889in}{2.034538in}}%
\pgfpathclose%
\pgfusepath{stroke,fill}%
\end{pgfscope}%
\begin{pgfscope}%
\pgfpathrectangle{\pgfqpoint{0.100000in}{0.212622in}}{\pgfqpoint{3.696000in}{3.696000in}}%
\pgfusepath{clip}%
\pgfsetbuttcap%
\pgfsetroundjoin%
\definecolor{currentfill}{rgb}{0.121569,0.466667,0.705882}%
\pgfsetfillcolor{currentfill}%
\pgfsetfillopacity{0.300610}%
\pgfsetlinewidth{1.003750pt}%
\definecolor{currentstroke}{rgb}{0.121569,0.466667,0.705882}%
\pgfsetstrokecolor{currentstroke}%
\pgfsetstrokeopacity{0.300610}%
\pgfsetdash{}{0pt}%
\pgfpathmoveto{\pgfqpoint{1.555889in}{2.034538in}}%
\pgfpathcurveto{\pgfqpoint{1.564126in}{2.034538in}}{\pgfqpoint{1.572026in}{2.037811in}}{\pgfqpoint{1.577850in}{2.043635in}}%
\pgfpathcurveto{\pgfqpoint{1.583674in}{2.049459in}}{\pgfqpoint{1.586946in}{2.057359in}}{\pgfqpoint{1.586946in}{2.065595in}}%
\pgfpathcurveto{\pgfqpoint{1.586946in}{2.073831in}}{\pgfqpoint{1.583674in}{2.081731in}}{\pgfqpoint{1.577850in}{2.087555in}}%
\pgfpathcurveto{\pgfqpoint{1.572026in}{2.093379in}}{\pgfqpoint{1.564126in}{2.096651in}}{\pgfqpoint{1.555889in}{2.096651in}}%
\pgfpathcurveto{\pgfqpoint{1.547653in}{2.096651in}}{\pgfqpoint{1.539753in}{2.093379in}}{\pgfqpoint{1.533929in}{2.087555in}}%
\pgfpathcurveto{\pgfqpoint{1.528105in}{2.081731in}}{\pgfqpoint{1.524833in}{2.073831in}}{\pgfqpoint{1.524833in}{2.065595in}}%
\pgfpathcurveto{\pgfqpoint{1.524833in}{2.057359in}}{\pgfqpoint{1.528105in}{2.049459in}}{\pgfqpoint{1.533929in}{2.043635in}}%
\pgfpathcurveto{\pgfqpoint{1.539753in}{2.037811in}}{\pgfqpoint{1.547653in}{2.034538in}}{\pgfqpoint{1.555889in}{2.034538in}}%
\pgfpathclose%
\pgfusepath{stroke,fill}%
\end{pgfscope}%
\begin{pgfscope}%
\pgfpathrectangle{\pgfqpoint{0.100000in}{0.212622in}}{\pgfqpoint{3.696000in}{3.696000in}}%
\pgfusepath{clip}%
\pgfsetbuttcap%
\pgfsetroundjoin%
\definecolor{currentfill}{rgb}{0.121569,0.466667,0.705882}%
\pgfsetfillcolor{currentfill}%
\pgfsetfillopacity{0.300610}%
\pgfsetlinewidth{1.003750pt}%
\definecolor{currentstroke}{rgb}{0.121569,0.466667,0.705882}%
\pgfsetstrokecolor{currentstroke}%
\pgfsetstrokeopacity{0.300610}%
\pgfsetdash{}{0pt}%
\pgfpathmoveto{\pgfqpoint{1.555889in}{2.034538in}}%
\pgfpathcurveto{\pgfqpoint{1.564126in}{2.034538in}}{\pgfqpoint{1.572026in}{2.037811in}}{\pgfqpoint{1.577850in}{2.043635in}}%
\pgfpathcurveto{\pgfqpoint{1.583674in}{2.049459in}}{\pgfqpoint{1.586946in}{2.057359in}}{\pgfqpoint{1.586946in}{2.065595in}}%
\pgfpathcurveto{\pgfqpoint{1.586946in}{2.073831in}}{\pgfqpoint{1.583674in}{2.081731in}}{\pgfqpoint{1.577850in}{2.087555in}}%
\pgfpathcurveto{\pgfqpoint{1.572026in}{2.093379in}}{\pgfqpoint{1.564126in}{2.096651in}}{\pgfqpoint{1.555889in}{2.096651in}}%
\pgfpathcurveto{\pgfqpoint{1.547653in}{2.096651in}}{\pgfqpoint{1.539753in}{2.093379in}}{\pgfqpoint{1.533929in}{2.087555in}}%
\pgfpathcurveto{\pgfqpoint{1.528105in}{2.081731in}}{\pgfqpoint{1.524833in}{2.073831in}}{\pgfqpoint{1.524833in}{2.065595in}}%
\pgfpathcurveto{\pgfqpoint{1.524833in}{2.057359in}}{\pgfqpoint{1.528105in}{2.049459in}}{\pgfqpoint{1.533929in}{2.043635in}}%
\pgfpathcurveto{\pgfqpoint{1.539753in}{2.037811in}}{\pgfqpoint{1.547653in}{2.034538in}}{\pgfqpoint{1.555889in}{2.034538in}}%
\pgfpathclose%
\pgfusepath{stroke,fill}%
\end{pgfscope}%
\begin{pgfscope}%
\pgfpathrectangle{\pgfqpoint{0.100000in}{0.212622in}}{\pgfqpoint{3.696000in}{3.696000in}}%
\pgfusepath{clip}%
\pgfsetbuttcap%
\pgfsetroundjoin%
\definecolor{currentfill}{rgb}{0.121569,0.466667,0.705882}%
\pgfsetfillcolor{currentfill}%
\pgfsetfillopacity{0.300610}%
\pgfsetlinewidth{1.003750pt}%
\definecolor{currentstroke}{rgb}{0.121569,0.466667,0.705882}%
\pgfsetstrokecolor{currentstroke}%
\pgfsetstrokeopacity{0.300610}%
\pgfsetdash{}{0pt}%
\pgfpathmoveto{\pgfqpoint{1.555889in}{2.034538in}}%
\pgfpathcurveto{\pgfqpoint{1.564126in}{2.034538in}}{\pgfqpoint{1.572026in}{2.037811in}}{\pgfqpoint{1.577850in}{2.043635in}}%
\pgfpathcurveto{\pgfqpoint{1.583674in}{2.049459in}}{\pgfqpoint{1.586946in}{2.057359in}}{\pgfqpoint{1.586946in}{2.065595in}}%
\pgfpathcurveto{\pgfqpoint{1.586946in}{2.073831in}}{\pgfqpoint{1.583674in}{2.081731in}}{\pgfqpoint{1.577850in}{2.087555in}}%
\pgfpathcurveto{\pgfqpoint{1.572026in}{2.093379in}}{\pgfqpoint{1.564126in}{2.096651in}}{\pgfqpoint{1.555889in}{2.096651in}}%
\pgfpathcurveto{\pgfqpoint{1.547653in}{2.096651in}}{\pgfqpoint{1.539753in}{2.093379in}}{\pgfqpoint{1.533929in}{2.087555in}}%
\pgfpathcurveto{\pgfqpoint{1.528105in}{2.081731in}}{\pgfqpoint{1.524833in}{2.073831in}}{\pgfqpoint{1.524833in}{2.065595in}}%
\pgfpathcurveto{\pgfqpoint{1.524833in}{2.057359in}}{\pgfqpoint{1.528105in}{2.049459in}}{\pgfqpoint{1.533929in}{2.043635in}}%
\pgfpathcurveto{\pgfqpoint{1.539753in}{2.037811in}}{\pgfqpoint{1.547653in}{2.034538in}}{\pgfqpoint{1.555889in}{2.034538in}}%
\pgfpathclose%
\pgfusepath{stroke,fill}%
\end{pgfscope}%
\begin{pgfscope}%
\pgfpathrectangle{\pgfqpoint{0.100000in}{0.212622in}}{\pgfqpoint{3.696000in}{3.696000in}}%
\pgfusepath{clip}%
\pgfsetbuttcap%
\pgfsetroundjoin%
\definecolor{currentfill}{rgb}{0.121569,0.466667,0.705882}%
\pgfsetfillcolor{currentfill}%
\pgfsetfillopacity{0.300610}%
\pgfsetlinewidth{1.003750pt}%
\definecolor{currentstroke}{rgb}{0.121569,0.466667,0.705882}%
\pgfsetstrokecolor{currentstroke}%
\pgfsetstrokeopacity{0.300610}%
\pgfsetdash{}{0pt}%
\pgfpathmoveto{\pgfqpoint{1.555889in}{2.034538in}}%
\pgfpathcurveto{\pgfqpoint{1.564126in}{2.034538in}}{\pgfqpoint{1.572026in}{2.037811in}}{\pgfqpoint{1.577850in}{2.043635in}}%
\pgfpathcurveto{\pgfqpoint{1.583674in}{2.049459in}}{\pgfqpoint{1.586946in}{2.057359in}}{\pgfqpoint{1.586946in}{2.065595in}}%
\pgfpathcurveto{\pgfqpoint{1.586946in}{2.073831in}}{\pgfqpoint{1.583674in}{2.081731in}}{\pgfqpoint{1.577850in}{2.087555in}}%
\pgfpathcurveto{\pgfqpoint{1.572026in}{2.093379in}}{\pgfqpoint{1.564126in}{2.096651in}}{\pgfqpoint{1.555889in}{2.096651in}}%
\pgfpathcurveto{\pgfqpoint{1.547653in}{2.096651in}}{\pgfqpoint{1.539753in}{2.093379in}}{\pgfqpoint{1.533929in}{2.087555in}}%
\pgfpathcurveto{\pgfqpoint{1.528105in}{2.081731in}}{\pgfqpoint{1.524833in}{2.073831in}}{\pgfqpoint{1.524833in}{2.065595in}}%
\pgfpathcurveto{\pgfqpoint{1.524833in}{2.057359in}}{\pgfqpoint{1.528105in}{2.049459in}}{\pgfqpoint{1.533929in}{2.043635in}}%
\pgfpathcurveto{\pgfqpoint{1.539753in}{2.037811in}}{\pgfqpoint{1.547653in}{2.034538in}}{\pgfqpoint{1.555889in}{2.034538in}}%
\pgfpathclose%
\pgfusepath{stroke,fill}%
\end{pgfscope}%
\begin{pgfscope}%
\pgfpathrectangle{\pgfqpoint{0.100000in}{0.212622in}}{\pgfqpoint{3.696000in}{3.696000in}}%
\pgfusepath{clip}%
\pgfsetbuttcap%
\pgfsetroundjoin%
\definecolor{currentfill}{rgb}{0.121569,0.466667,0.705882}%
\pgfsetfillcolor{currentfill}%
\pgfsetfillopacity{0.300610}%
\pgfsetlinewidth{1.003750pt}%
\definecolor{currentstroke}{rgb}{0.121569,0.466667,0.705882}%
\pgfsetstrokecolor{currentstroke}%
\pgfsetstrokeopacity{0.300610}%
\pgfsetdash{}{0pt}%
\pgfpathmoveto{\pgfqpoint{1.555889in}{2.034538in}}%
\pgfpathcurveto{\pgfqpoint{1.564126in}{2.034538in}}{\pgfqpoint{1.572026in}{2.037811in}}{\pgfqpoint{1.577850in}{2.043635in}}%
\pgfpathcurveto{\pgfqpoint{1.583674in}{2.049459in}}{\pgfqpoint{1.586946in}{2.057359in}}{\pgfqpoint{1.586946in}{2.065595in}}%
\pgfpathcurveto{\pgfqpoint{1.586946in}{2.073831in}}{\pgfqpoint{1.583674in}{2.081731in}}{\pgfqpoint{1.577850in}{2.087555in}}%
\pgfpathcurveto{\pgfqpoint{1.572026in}{2.093379in}}{\pgfqpoint{1.564126in}{2.096651in}}{\pgfqpoint{1.555889in}{2.096651in}}%
\pgfpathcurveto{\pgfqpoint{1.547653in}{2.096651in}}{\pgfqpoint{1.539753in}{2.093379in}}{\pgfqpoint{1.533929in}{2.087555in}}%
\pgfpathcurveto{\pgfqpoint{1.528105in}{2.081731in}}{\pgfqpoint{1.524833in}{2.073831in}}{\pgfqpoint{1.524833in}{2.065595in}}%
\pgfpathcurveto{\pgfqpoint{1.524833in}{2.057359in}}{\pgfqpoint{1.528105in}{2.049459in}}{\pgfqpoint{1.533929in}{2.043635in}}%
\pgfpathcurveto{\pgfqpoint{1.539753in}{2.037811in}}{\pgfqpoint{1.547653in}{2.034538in}}{\pgfqpoint{1.555889in}{2.034538in}}%
\pgfpathclose%
\pgfusepath{stroke,fill}%
\end{pgfscope}%
\begin{pgfscope}%
\pgfpathrectangle{\pgfqpoint{0.100000in}{0.212622in}}{\pgfqpoint{3.696000in}{3.696000in}}%
\pgfusepath{clip}%
\pgfsetbuttcap%
\pgfsetroundjoin%
\definecolor{currentfill}{rgb}{0.121569,0.466667,0.705882}%
\pgfsetfillcolor{currentfill}%
\pgfsetfillopacity{0.300610}%
\pgfsetlinewidth{1.003750pt}%
\definecolor{currentstroke}{rgb}{0.121569,0.466667,0.705882}%
\pgfsetstrokecolor{currentstroke}%
\pgfsetstrokeopacity{0.300610}%
\pgfsetdash{}{0pt}%
\pgfpathmoveto{\pgfqpoint{1.555889in}{2.034538in}}%
\pgfpathcurveto{\pgfqpoint{1.564126in}{2.034538in}}{\pgfqpoint{1.572026in}{2.037811in}}{\pgfqpoint{1.577850in}{2.043635in}}%
\pgfpathcurveto{\pgfqpoint{1.583674in}{2.049459in}}{\pgfqpoint{1.586946in}{2.057359in}}{\pgfqpoint{1.586946in}{2.065595in}}%
\pgfpathcurveto{\pgfqpoint{1.586946in}{2.073831in}}{\pgfqpoint{1.583674in}{2.081731in}}{\pgfqpoint{1.577850in}{2.087555in}}%
\pgfpathcurveto{\pgfqpoint{1.572026in}{2.093379in}}{\pgfqpoint{1.564126in}{2.096651in}}{\pgfqpoint{1.555889in}{2.096651in}}%
\pgfpathcurveto{\pgfqpoint{1.547653in}{2.096651in}}{\pgfqpoint{1.539753in}{2.093379in}}{\pgfqpoint{1.533929in}{2.087555in}}%
\pgfpathcurveto{\pgfqpoint{1.528105in}{2.081731in}}{\pgfqpoint{1.524833in}{2.073831in}}{\pgfqpoint{1.524833in}{2.065595in}}%
\pgfpathcurveto{\pgfqpoint{1.524833in}{2.057359in}}{\pgfqpoint{1.528105in}{2.049459in}}{\pgfqpoint{1.533929in}{2.043635in}}%
\pgfpathcurveto{\pgfqpoint{1.539753in}{2.037811in}}{\pgfqpoint{1.547653in}{2.034538in}}{\pgfqpoint{1.555889in}{2.034538in}}%
\pgfpathclose%
\pgfusepath{stroke,fill}%
\end{pgfscope}%
\begin{pgfscope}%
\pgfpathrectangle{\pgfqpoint{0.100000in}{0.212622in}}{\pgfqpoint{3.696000in}{3.696000in}}%
\pgfusepath{clip}%
\pgfsetbuttcap%
\pgfsetroundjoin%
\definecolor{currentfill}{rgb}{0.121569,0.466667,0.705882}%
\pgfsetfillcolor{currentfill}%
\pgfsetfillopacity{0.300610}%
\pgfsetlinewidth{1.003750pt}%
\definecolor{currentstroke}{rgb}{0.121569,0.466667,0.705882}%
\pgfsetstrokecolor{currentstroke}%
\pgfsetstrokeopacity{0.300610}%
\pgfsetdash{}{0pt}%
\pgfpathmoveto{\pgfqpoint{1.555889in}{2.034538in}}%
\pgfpathcurveto{\pgfqpoint{1.564126in}{2.034538in}}{\pgfqpoint{1.572026in}{2.037811in}}{\pgfqpoint{1.577850in}{2.043635in}}%
\pgfpathcurveto{\pgfqpoint{1.583674in}{2.049459in}}{\pgfqpoint{1.586946in}{2.057359in}}{\pgfqpoint{1.586946in}{2.065595in}}%
\pgfpathcurveto{\pgfqpoint{1.586946in}{2.073831in}}{\pgfqpoint{1.583674in}{2.081731in}}{\pgfqpoint{1.577850in}{2.087555in}}%
\pgfpathcurveto{\pgfqpoint{1.572026in}{2.093379in}}{\pgfqpoint{1.564126in}{2.096651in}}{\pgfqpoint{1.555889in}{2.096651in}}%
\pgfpathcurveto{\pgfqpoint{1.547653in}{2.096651in}}{\pgfqpoint{1.539753in}{2.093379in}}{\pgfqpoint{1.533929in}{2.087555in}}%
\pgfpathcurveto{\pgfqpoint{1.528105in}{2.081731in}}{\pgfqpoint{1.524833in}{2.073831in}}{\pgfqpoint{1.524833in}{2.065595in}}%
\pgfpathcurveto{\pgfqpoint{1.524833in}{2.057359in}}{\pgfqpoint{1.528105in}{2.049459in}}{\pgfqpoint{1.533929in}{2.043635in}}%
\pgfpathcurveto{\pgfqpoint{1.539753in}{2.037811in}}{\pgfqpoint{1.547653in}{2.034538in}}{\pgfqpoint{1.555889in}{2.034538in}}%
\pgfpathclose%
\pgfusepath{stroke,fill}%
\end{pgfscope}%
\begin{pgfscope}%
\pgfpathrectangle{\pgfqpoint{0.100000in}{0.212622in}}{\pgfqpoint{3.696000in}{3.696000in}}%
\pgfusepath{clip}%
\pgfsetbuttcap%
\pgfsetroundjoin%
\definecolor{currentfill}{rgb}{0.121569,0.466667,0.705882}%
\pgfsetfillcolor{currentfill}%
\pgfsetfillopacity{0.300610}%
\pgfsetlinewidth{1.003750pt}%
\definecolor{currentstroke}{rgb}{0.121569,0.466667,0.705882}%
\pgfsetstrokecolor{currentstroke}%
\pgfsetstrokeopacity{0.300610}%
\pgfsetdash{}{0pt}%
\pgfpathmoveto{\pgfqpoint{1.555889in}{2.034538in}}%
\pgfpathcurveto{\pgfqpoint{1.564126in}{2.034538in}}{\pgfqpoint{1.572026in}{2.037811in}}{\pgfqpoint{1.577850in}{2.043635in}}%
\pgfpathcurveto{\pgfqpoint{1.583674in}{2.049459in}}{\pgfqpoint{1.586946in}{2.057359in}}{\pgfqpoint{1.586946in}{2.065595in}}%
\pgfpathcurveto{\pgfqpoint{1.586946in}{2.073831in}}{\pgfqpoint{1.583674in}{2.081731in}}{\pgfqpoint{1.577850in}{2.087555in}}%
\pgfpathcurveto{\pgfqpoint{1.572026in}{2.093379in}}{\pgfqpoint{1.564126in}{2.096651in}}{\pgfqpoint{1.555889in}{2.096651in}}%
\pgfpathcurveto{\pgfqpoint{1.547653in}{2.096651in}}{\pgfqpoint{1.539753in}{2.093379in}}{\pgfqpoint{1.533929in}{2.087555in}}%
\pgfpathcurveto{\pgfqpoint{1.528105in}{2.081731in}}{\pgfqpoint{1.524833in}{2.073831in}}{\pgfqpoint{1.524833in}{2.065595in}}%
\pgfpathcurveto{\pgfqpoint{1.524833in}{2.057359in}}{\pgfqpoint{1.528105in}{2.049459in}}{\pgfqpoint{1.533929in}{2.043635in}}%
\pgfpathcurveto{\pgfqpoint{1.539753in}{2.037811in}}{\pgfqpoint{1.547653in}{2.034538in}}{\pgfqpoint{1.555889in}{2.034538in}}%
\pgfpathclose%
\pgfusepath{stroke,fill}%
\end{pgfscope}%
\begin{pgfscope}%
\pgfpathrectangle{\pgfqpoint{0.100000in}{0.212622in}}{\pgfqpoint{3.696000in}{3.696000in}}%
\pgfusepath{clip}%
\pgfsetbuttcap%
\pgfsetroundjoin%
\definecolor{currentfill}{rgb}{0.121569,0.466667,0.705882}%
\pgfsetfillcolor{currentfill}%
\pgfsetfillopacity{0.300610}%
\pgfsetlinewidth{1.003750pt}%
\definecolor{currentstroke}{rgb}{0.121569,0.466667,0.705882}%
\pgfsetstrokecolor{currentstroke}%
\pgfsetstrokeopacity{0.300610}%
\pgfsetdash{}{0pt}%
\pgfpathmoveto{\pgfqpoint{1.555889in}{2.034538in}}%
\pgfpathcurveto{\pgfqpoint{1.564126in}{2.034538in}}{\pgfqpoint{1.572026in}{2.037811in}}{\pgfqpoint{1.577850in}{2.043635in}}%
\pgfpathcurveto{\pgfqpoint{1.583674in}{2.049459in}}{\pgfqpoint{1.586946in}{2.057359in}}{\pgfqpoint{1.586946in}{2.065595in}}%
\pgfpathcurveto{\pgfqpoint{1.586946in}{2.073831in}}{\pgfqpoint{1.583674in}{2.081731in}}{\pgfqpoint{1.577850in}{2.087555in}}%
\pgfpathcurveto{\pgfqpoint{1.572026in}{2.093379in}}{\pgfqpoint{1.564126in}{2.096651in}}{\pgfqpoint{1.555889in}{2.096651in}}%
\pgfpathcurveto{\pgfqpoint{1.547653in}{2.096651in}}{\pgfqpoint{1.539753in}{2.093379in}}{\pgfqpoint{1.533929in}{2.087555in}}%
\pgfpathcurveto{\pgfqpoint{1.528105in}{2.081731in}}{\pgfqpoint{1.524833in}{2.073831in}}{\pgfqpoint{1.524833in}{2.065595in}}%
\pgfpathcurveto{\pgfqpoint{1.524833in}{2.057359in}}{\pgfqpoint{1.528105in}{2.049459in}}{\pgfqpoint{1.533929in}{2.043635in}}%
\pgfpathcurveto{\pgfqpoint{1.539753in}{2.037811in}}{\pgfqpoint{1.547653in}{2.034538in}}{\pgfqpoint{1.555889in}{2.034538in}}%
\pgfpathclose%
\pgfusepath{stroke,fill}%
\end{pgfscope}%
\begin{pgfscope}%
\pgfpathrectangle{\pgfqpoint{0.100000in}{0.212622in}}{\pgfqpoint{3.696000in}{3.696000in}}%
\pgfusepath{clip}%
\pgfsetbuttcap%
\pgfsetroundjoin%
\definecolor{currentfill}{rgb}{0.121569,0.466667,0.705882}%
\pgfsetfillcolor{currentfill}%
\pgfsetfillopacity{0.300610}%
\pgfsetlinewidth{1.003750pt}%
\definecolor{currentstroke}{rgb}{0.121569,0.466667,0.705882}%
\pgfsetstrokecolor{currentstroke}%
\pgfsetstrokeopacity{0.300610}%
\pgfsetdash{}{0pt}%
\pgfpathmoveto{\pgfqpoint{1.555889in}{2.034538in}}%
\pgfpathcurveto{\pgfqpoint{1.564126in}{2.034538in}}{\pgfqpoint{1.572026in}{2.037811in}}{\pgfqpoint{1.577850in}{2.043635in}}%
\pgfpathcurveto{\pgfqpoint{1.583674in}{2.049459in}}{\pgfqpoint{1.586946in}{2.057359in}}{\pgfqpoint{1.586946in}{2.065595in}}%
\pgfpathcurveto{\pgfqpoint{1.586946in}{2.073831in}}{\pgfqpoint{1.583674in}{2.081731in}}{\pgfqpoint{1.577850in}{2.087555in}}%
\pgfpathcurveto{\pgfqpoint{1.572026in}{2.093379in}}{\pgfqpoint{1.564126in}{2.096651in}}{\pgfqpoint{1.555889in}{2.096651in}}%
\pgfpathcurveto{\pgfqpoint{1.547653in}{2.096651in}}{\pgfqpoint{1.539753in}{2.093379in}}{\pgfqpoint{1.533929in}{2.087555in}}%
\pgfpathcurveto{\pgfqpoint{1.528105in}{2.081731in}}{\pgfqpoint{1.524833in}{2.073831in}}{\pgfqpoint{1.524833in}{2.065595in}}%
\pgfpathcurveto{\pgfqpoint{1.524833in}{2.057359in}}{\pgfqpoint{1.528105in}{2.049459in}}{\pgfqpoint{1.533929in}{2.043635in}}%
\pgfpathcurveto{\pgfqpoint{1.539753in}{2.037811in}}{\pgfqpoint{1.547653in}{2.034538in}}{\pgfqpoint{1.555889in}{2.034538in}}%
\pgfpathclose%
\pgfusepath{stroke,fill}%
\end{pgfscope}%
\begin{pgfscope}%
\pgfpathrectangle{\pgfqpoint{0.100000in}{0.212622in}}{\pgfqpoint{3.696000in}{3.696000in}}%
\pgfusepath{clip}%
\pgfsetbuttcap%
\pgfsetroundjoin%
\definecolor{currentfill}{rgb}{0.121569,0.466667,0.705882}%
\pgfsetfillcolor{currentfill}%
\pgfsetfillopacity{0.300610}%
\pgfsetlinewidth{1.003750pt}%
\definecolor{currentstroke}{rgb}{0.121569,0.466667,0.705882}%
\pgfsetstrokecolor{currentstroke}%
\pgfsetstrokeopacity{0.300610}%
\pgfsetdash{}{0pt}%
\pgfpathmoveto{\pgfqpoint{1.555889in}{2.034538in}}%
\pgfpathcurveto{\pgfqpoint{1.564126in}{2.034538in}}{\pgfqpoint{1.572026in}{2.037811in}}{\pgfqpoint{1.577850in}{2.043635in}}%
\pgfpathcurveto{\pgfqpoint{1.583674in}{2.049459in}}{\pgfqpoint{1.586946in}{2.057359in}}{\pgfqpoint{1.586946in}{2.065595in}}%
\pgfpathcurveto{\pgfqpoint{1.586946in}{2.073831in}}{\pgfqpoint{1.583674in}{2.081731in}}{\pgfqpoint{1.577850in}{2.087555in}}%
\pgfpathcurveto{\pgfqpoint{1.572026in}{2.093379in}}{\pgfqpoint{1.564126in}{2.096651in}}{\pgfqpoint{1.555889in}{2.096651in}}%
\pgfpathcurveto{\pgfqpoint{1.547653in}{2.096651in}}{\pgfqpoint{1.539753in}{2.093379in}}{\pgfqpoint{1.533929in}{2.087555in}}%
\pgfpathcurveto{\pgfqpoint{1.528105in}{2.081731in}}{\pgfqpoint{1.524833in}{2.073831in}}{\pgfqpoint{1.524833in}{2.065595in}}%
\pgfpathcurveto{\pgfqpoint{1.524833in}{2.057359in}}{\pgfqpoint{1.528105in}{2.049459in}}{\pgfqpoint{1.533929in}{2.043635in}}%
\pgfpathcurveto{\pgfqpoint{1.539753in}{2.037811in}}{\pgfqpoint{1.547653in}{2.034538in}}{\pgfqpoint{1.555889in}{2.034538in}}%
\pgfpathclose%
\pgfusepath{stroke,fill}%
\end{pgfscope}%
\begin{pgfscope}%
\pgfpathrectangle{\pgfqpoint{0.100000in}{0.212622in}}{\pgfqpoint{3.696000in}{3.696000in}}%
\pgfusepath{clip}%
\pgfsetbuttcap%
\pgfsetroundjoin%
\definecolor{currentfill}{rgb}{0.121569,0.466667,0.705882}%
\pgfsetfillcolor{currentfill}%
\pgfsetfillopacity{0.300610}%
\pgfsetlinewidth{1.003750pt}%
\definecolor{currentstroke}{rgb}{0.121569,0.466667,0.705882}%
\pgfsetstrokecolor{currentstroke}%
\pgfsetstrokeopacity{0.300610}%
\pgfsetdash{}{0pt}%
\pgfpathmoveto{\pgfqpoint{1.555889in}{2.034538in}}%
\pgfpathcurveto{\pgfqpoint{1.564126in}{2.034538in}}{\pgfqpoint{1.572026in}{2.037811in}}{\pgfqpoint{1.577850in}{2.043635in}}%
\pgfpathcurveto{\pgfqpoint{1.583674in}{2.049459in}}{\pgfqpoint{1.586946in}{2.057359in}}{\pgfqpoint{1.586946in}{2.065595in}}%
\pgfpathcurveto{\pgfqpoint{1.586946in}{2.073831in}}{\pgfqpoint{1.583674in}{2.081731in}}{\pgfqpoint{1.577850in}{2.087555in}}%
\pgfpathcurveto{\pgfqpoint{1.572026in}{2.093379in}}{\pgfqpoint{1.564126in}{2.096651in}}{\pgfqpoint{1.555889in}{2.096651in}}%
\pgfpathcurveto{\pgfqpoint{1.547653in}{2.096651in}}{\pgfqpoint{1.539753in}{2.093379in}}{\pgfqpoint{1.533929in}{2.087555in}}%
\pgfpathcurveto{\pgfqpoint{1.528105in}{2.081731in}}{\pgfqpoint{1.524833in}{2.073831in}}{\pgfqpoint{1.524833in}{2.065595in}}%
\pgfpathcurveto{\pgfqpoint{1.524833in}{2.057359in}}{\pgfqpoint{1.528105in}{2.049459in}}{\pgfqpoint{1.533929in}{2.043635in}}%
\pgfpathcurveto{\pgfqpoint{1.539753in}{2.037811in}}{\pgfqpoint{1.547653in}{2.034538in}}{\pgfqpoint{1.555889in}{2.034538in}}%
\pgfpathclose%
\pgfusepath{stroke,fill}%
\end{pgfscope}%
\begin{pgfscope}%
\pgfpathrectangle{\pgfqpoint{0.100000in}{0.212622in}}{\pgfqpoint{3.696000in}{3.696000in}}%
\pgfusepath{clip}%
\pgfsetbuttcap%
\pgfsetroundjoin%
\definecolor{currentfill}{rgb}{0.121569,0.466667,0.705882}%
\pgfsetfillcolor{currentfill}%
\pgfsetfillopacity{0.300610}%
\pgfsetlinewidth{1.003750pt}%
\definecolor{currentstroke}{rgb}{0.121569,0.466667,0.705882}%
\pgfsetstrokecolor{currentstroke}%
\pgfsetstrokeopacity{0.300610}%
\pgfsetdash{}{0pt}%
\pgfpathmoveto{\pgfqpoint{1.555889in}{2.034538in}}%
\pgfpathcurveto{\pgfqpoint{1.564126in}{2.034538in}}{\pgfqpoint{1.572026in}{2.037811in}}{\pgfqpoint{1.577850in}{2.043635in}}%
\pgfpathcurveto{\pgfqpoint{1.583674in}{2.049459in}}{\pgfqpoint{1.586946in}{2.057359in}}{\pgfqpoint{1.586946in}{2.065595in}}%
\pgfpathcurveto{\pgfqpoint{1.586946in}{2.073831in}}{\pgfqpoint{1.583674in}{2.081731in}}{\pgfqpoint{1.577850in}{2.087555in}}%
\pgfpathcurveto{\pgfqpoint{1.572026in}{2.093379in}}{\pgfqpoint{1.564126in}{2.096651in}}{\pgfqpoint{1.555889in}{2.096651in}}%
\pgfpathcurveto{\pgfqpoint{1.547653in}{2.096651in}}{\pgfqpoint{1.539753in}{2.093379in}}{\pgfqpoint{1.533929in}{2.087555in}}%
\pgfpathcurveto{\pgfqpoint{1.528105in}{2.081731in}}{\pgfqpoint{1.524833in}{2.073831in}}{\pgfqpoint{1.524833in}{2.065595in}}%
\pgfpathcurveto{\pgfqpoint{1.524833in}{2.057359in}}{\pgfqpoint{1.528105in}{2.049459in}}{\pgfqpoint{1.533929in}{2.043635in}}%
\pgfpathcurveto{\pgfqpoint{1.539753in}{2.037811in}}{\pgfqpoint{1.547653in}{2.034538in}}{\pgfqpoint{1.555889in}{2.034538in}}%
\pgfpathclose%
\pgfusepath{stroke,fill}%
\end{pgfscope}%
\begin{pgfscope}%
\pgfpathrectangle{\pgfqpoint{0.100000in}{0.212622in}}{\pgfqpoint{3.696000in}{3.696000in}}%
\pgfusepath{clip}%
\pgfsetbuttcap%
\pgfsetroundjoin%
\definecolor{currentfill}{rgb}{0.121569,0.466667,0.705882}%
\pgfsetfillcolor{currentfill}%
\pgfsetfillopacity{0.300610}%
\pgfsetlinewidth{1.003750pt}%
\definecolor{currentstroke}{rgb}{0.121569,0.466667,0.705882}%
\pgfsetstrokecolor{currentstroke}%
\pgfsetstrokeopacity{0.300610}%
\pgfsetdash{}{0pt}%
\pgfpathmoveto{\pgfqpoint{1.555889in}{2.034538in}}%
\pgfpathcurveto{\pgfqpoint{1.564126in}{2.034538in}}{\pgfqpoint{1.572026in}{2.037811in}}{\pgfqpoint{1.577850in}{2.043635in}}%
\pgfpathcurveto{\pgfqpoint{1.583674in}{2.049459in}}{\pgfqpoint{1.586946in}{2.057359in}}{\pgfqpoint{1.586946in}{2.065595in}}%
\pgfpathcurveto{\pgfqpoint{1.586946in}{2.073831in}}{\pgfqpoint{1.583674in}{2.081731in}}{\pgfqpoint{1.577850in}{2.087555in}}%
\pgfpathcurveto{\pgfqpoint{1.572026in}{2.093379in}}{\pgfqpoint{1.564126in}{2.096651in}}{\pgfqpoint{1.555889in}{2.096651in}}%
\pgfpathcurveto{\pgfqpoint{1.547653in}{2.096651in}}{\pgfqpoint{1.539753in}{2.093379in}}{\pgfqpoint{1.533929in}{2.087555in}}%
\pgfpathcurveto{\pgfqpoint{1.528105in}{2.081731in}}{\pgfqpoint{1.524833in}{2.073831in}}{\pgfqpoint{1.524833in}{2.065595in}}%
\pgfpathcurveto{\pgfqpoint{1.524833in}{2.057359in}}{\pgfqpoint{1.528105in}{2.049459in}}{\pgfqpoint{1.533929in}{2.043635in}}%
\pgfpathcurveto{\pgfqpoint{1.539753in}{2.037811in}}{\pgfqpoint{1.547653in}{2.034538in}}{\pgfqpoint{1.555889in}{2.034538in}}%
\pgfpathclose%
\pgfusepath{stroke,fill}%
\end{pgfscope}%
\begin{pgfscope}%
\pgfpathrectangle{\pgfqpoint{0.100000in}{0.212622in}}{\pgfqpoint{3.696000in}{3.696000in}}%
\pgfusepath{clip}%
\pgfsetbuttcap%
\pgfsetroundjoin%
\definecolor{currentfill}{rgb}{0.121569,0.466667,0.705882}%
\pgfsetfillcolor{currentfill}%
\pgfsetfillopacity{0.300610}%
\pgfsetlinewidth{1.003750pt}%
\definecolor{currentstroke}{rgb}{0.121569,0.466667,0.705882}%
\pgfsetstrokecolor{currentstroke}%
\pgfsetstrokeopacity{0.300610}%
\pgfsetdash{}{0pt}%
\pgfpathmoveto{\pgfqpoint{1.555889in}{2.034538in}}%
\pgfpathcurveto{\pgfqpoint{1.564126in}{2.034538in}}{\pgfqpoint{1.572026in}{2.037811in}}{\pgfqpoint{1.577850in}{2.043635in}}%
\pgfpathcurveto{\pgfqpoint{1.583674in}{2.049459in}}{\pgfqpoint{1.586946in}{2.057359in}}{\pgfqpoint{1.586946in}{2.065595in}}%
\pgfpathcurveto{\pgfqpoint{1.586946in}{2.073831in}}{\pgfqpoint{1.583674in}{2.081731in}}{\pgfqpoint{1.577850in}{2.087555in}}%
\pgfpathcurveto{\pgfqpoint{1.572026in}{2.093379in}}{\pgfqpoint{1.564126in}{2.096651in}}{\pgfqpoint{1.555889in}{2.096651in}}%
\pgfpathcurveto{\pgfqpoint{1.547653in}{2.096651in}}{\pgfqpoint{1.539753in}{2.093379in}}{\pgfqpoint{1.533929in}{2.087555in}}%
\pgfpathcurveto{\pgfqpoint{1.528105in}{2.081731in}}{\pgfqpoint{1.524833in}{2.073831in}}{\pgfqpoint{1.524833in}{2.065595in}}%
\pgfpathcurveto{\pgfqpoint{1.524833in}{2.057359in}}{\pgfqpoint{1.528105in}{2.049459in}}{\pgfqpoint{1.533929in}{2.043635in}}%
\pgfpathcurveto{\pgfqpoint{1.539753in}{2.037811in}}{\pgfqpoint{1.547653in}{2.034538in}}{\pgfqpoint{1.555889in}{2.034538in}}%
\pgfpathclose%
\pgfusepath{stroke,fill}%
\end{pgfscope}%
\begin{pgfscope}%
\pgfpathrectangle{\pgfqpoint{0.100000in}{0.212622in}}{\pgfqpoint{3.696000in}{3.696000in}}%
\pgfusepath{clip}%
\pgfsetbuttcap%
\pgfsetroundjoin%
\definecolor{currentfill}{rgb}{0.121569,0.466667,0.705882}%
\pgfsetfillcolor{currentfill}%
\pgfsetfillopacity{0.300610}%
\pgfsetlinewidth{1.003750pt}%
\definecolor{currentstroke}{rgb}{0.121569,0.466667,0.705882}%
\pgfsetstrokecolor{currentstroke}%
\pgfsetstrokeopacity{0.300610}%
\pgfsetdash{}{0pt}%
\pgfpathmoveto{\pgfqpoint{1.555889in}{2.034538in}}%
\pgfpathcurveto{\pgfqpoint{1.564126in}{2.034538in}}{\pgfqpoint{1.572026in}{2.037811in}}{\pgfqpoint{1.577850in}{2.043635in}}%
\pgfpathcurveto{\pgfqpoint{1.583674in}{2.049459in}}{\pgfqpoint{1.586946in}{2.057359in}}{\pgfqpoint{1.586946in}{2.065595in}}%
\pgfpathcurveto{\pgfqpoint{1.586946in}{2.073831in}}{\pgfqpoint{1.583674in}{2.081731in}}{\pgfqpoint{1.577850in}{2.087555in}}%
\pgfpathcurveto{\pgfqpoint{1.572026in}{2.093379in}}{\pgfqpoint{1.564126in}{2.096651in}}{\pgfqpoint{1.555889in}{2.096651in}}%
\pgfpathcurveto{\pgfqpoint{1.547653in}{2.096651in}}{\pgfqpoint{1.539753in}{2.093379in}}{\pgfqpoint{1.533929in}{2.087555in}}%
\pgfpathcurveto{\pgfqpoint{1.528105in}{2.081731in}}{\pgfqpoint{1.524833in}{2.073831in}}{\pgfqpoint{1.524833in}{2.065595in}}%
\pgfpathcurveto{\pgfqpoint{1.524833in}{2.057359in}}{\pgfqpoint{1.528105in}{2.049459in}}{\pgfqpoint{1.533929in}{2.043635in}}%
\pgfpathcurveto{\pgfqpoint{1.539753in}{2.037811in}}{\pgfqpoint{1.547653in}{2.034538in}}{\pgfqpoint{1.555889in}{2.034538in}}%
\pgfpathclose%
\pgfusepath{stroke,fill}%
\end{pgfscope}%
\begin{pgfscope}%
\pgfpathrectangle{\pgfqpoint{0.100000in}{0.212622in}}{\pgfqpoint{3.696000in}{3.696000in}}%
\pgfusepath{clip}%
\pgfsetbuttcap%
\pgfsetroundjoin%
\definecolor{currentfill}{rgb}{0.121569,0.466667,0.705882}%
\pgfsetfillcolor{currentfill}%
\pgfsetfillopacity{0.300610}%
\pgfsetlinewidth{1.003750pt}%
\definecolor{currentstroke}{rgb}{0.121569,0.466667,0.705882}%
\pgfsetstrokecolor{currentstroke}%
\pgfsetstrokeopacity{0.300610}%
\pgfsetdash{}{0pt}%
\pgfpathmoveto{\pgfqpoint{1.555889in}{2.034538in}}%
\pgfpathcurveto{\pgfqpoint{1.564126in}{2.034538in}}{\pgfqpoint{1.572026in}{2.037811in}}{\pgfqpoint{1.577850in}{2.043635in}}%
\pgfpathcurveto{\pgfqpoint{1.583674in}{2.049459in}}{\pgfqpoint{1.586946in}{2.057359in}}{\pgfqpoint{1.586946in}{2.065595in}}%
\pgfpathcurveto{\pgfqpoint{1.586946in}{2.073831in}}{\pgfqpoint{1.583674in}{2.081731in}}{\pgfqpoint{1.577850in}{2.087555in}}%
\pgfpathcurveto{\pgfqpoint{1.572026in}{2.093379in}}{\pgfqpoint{1.564126in}{2.096651in}}{\pgfqpoint{1.555889in}{2.096651in}}%
\pgfpathcurveto{\pgfqpoint{1.547653in}{2.096651in}}{\pgfqpoint{1.539753in}{2.093379in}}{\pgfqpoint{1.533929in}{2.087555in}}%
\pgfpathcurveto{\pgfqpoint{1.528105in}{2.081731in}}{\pgfqpoint{1.524833in}{2.073831in}}{\pgfqpoint{1.524833in}{2.065595in}}%
\pgfpathcurveto{\pgfqpoint{1.524833in}{2.057359in}}{\pgfqpoint{1.528105in}{2.049459in}}{\pgfqpoint{1.533929in}{2.043635in}}%
\pgfpathcurveto{\pgfqpoint{1.539753in}{2.037811in}}{\pgfqpoint{1.547653in}{2.034538in}}{\pgfqpoint{1.555889in}{2.034538in}}%
\pgfpathclose%
\pgfusepath{stroke,fill}%
\end{pgfscope}%
\begin{pgfscope}%
\pgfpathrectangle{\pgfqpoint{0.100000in}{0.212622in}}{\pgfqpoint{3.696000in}{3.696000in}}%
\pgfusepath{clip}%
\pgfsetbuttcap%
\pgfsetroundjoin%
\definecolor{currentfill}{rgb}{0.121569,0.466667,0.705882}%
\pgfsetfillcolor{currentfill}%
\pgfsetfillopacity{0.300610}%
\pgfsetlinewidth{1.003750pt}%
\definecolor{currentstroke}{rgb}{0.121569,0.466667,0.705882}%
\pgfsetstrokecolor{currentstroke}%
\pgfsetstrokeopacity{0.300610}%
\pgfsetdash{}{0pt}%
\pgfpathmoveto{\pgfqpoint{1.555889in}{2.034538in}}%
\pgfpathcurveto{\pgfqpoint{1.564126in}{2.034538in}}{\pgfqpoint{1.572026in}{2.037811in}}{\pgfqpoint{1.577850in}{2.043635in}}%
\pgfpathcurveto{\pgfqpoint{1.583674in}{2.049459in}}{\pgfqpoint{1.586946in}{2.057359in}}{\pgfqpoint{1.586946in}{2.065595in}}%
\pgfpathcurveto{\pgfqpoint{1.586946in}{2.073831in}}{\pgfqpoint{1.583674in}{2.081731in}}{\pgfqpoint{1.577850in}{2.087555in}}%
\pgfpathcurveto{\pgfqpoint{1.572026in}{2.093379in}}{\pgfqpoint{1.564126in}{2.096651in}}{\pgfqpoint{1.555889in}{2.096651in}}%
\pgfpathcurveto{\pgfqpoint{1.547653in}{2.096651in}}{\pgfqpoint{1.539753in}{2.093379in}}{\pgfqpoint{1.533929in}{2.087555in}}%
\pgfpathcurveto{\pgfqpoint{1.528105in}{2.081731in}}{\pgfqpoint{1.524833in}{2.073831in}}{\pgfqpoint{1.524833in}{2.065595in}}%
\pgfpathcurveto{\pgfqpoint{1.524833in}{2.057359in}}{\pgfqpoint{1.528105in}{2.049459in}}{\pgfqpoint{1.533929in}{2.043635in}}%
\pgfpathcurveto{\pgfqpoint{1.539753in}{2.037811in}}{\pgfqpoint{1.547653in}{2.034538in}}{\pgfqpoint{1.555889in}{2.034538in}}%
\pgfpathclose%
\pgfusepath{stroke,fill}%
\end{pgfscope}%
\begin{pgfscope}%
\pgfpathrectangle{\pgfqpoint{0.100000in}{0.212622in}}{\pgfqpoint{3.696000in}{3.696000in}}%
\pgfusepath{clip}%
\pgfsetbuttcap%
\pgfsetroundjoin%
\definecolor{currentfill}{rgb}{0.121569,0.466667,0.705882}%
\pgfsetfillcolor{currentfill}%
\pgfsetfillopacity{0.300610}%
\pgfsetlinewidth{1.003750pt}%
\definecolor{currentstroke}{rgb}{0.121569,0.466667,0.705882}%
\pgfsetstrokecolor{currentstroke}%
\pgfsetstrokeopacity{0.300610}%
\pgfsetdash{}{0pt}%
\pgfpathmoveto{\pgfqpoint{1.555889in}{2.034538in}}%
\pgfpathcurveto{\pgfqpoint{1.564126in}{2.034538in}}{\pgfqpoint{1.572026in}{2.037811in}}{\pgfqpoint{1.577850in}{2.043635in}}%
\pgfpathcurveto{\pgfqpoint{1.583674in}{2.049459in}}{\pgfqpoint{1.586946in}{2.057359in}}{\pgfqpoint{1.586946in}{2.065595in}}%
\pgfpathcurveto{\pgfqpoint{1.586946in}{2.073831in}}{\pgfqpoint{1.583674in}{2.081731in}}{\pgfqpoint{1.577850in}{2.087555in}}%
\pgfpathcurveto{\pgfqpoint{1.572026in}{2.093379in}}{\pgfqpoint{1.564126in}{2.096651in}}{\pgfqpoint{1.555889in}{2.096651in}}%
\pgfpathcurveto{\pgfqpoint{1.547653in}{2.096651in}}{\pgfqpoint{1.539753in}{2.093379in}}{\pgfqpoint{1.533929in}{2.087555in}}%
\pgfpathcurveto{\pgfqpoint{1.528105in}{2.081731in}}{\pgfqpoint{1.524833in}{2.073831in}}{\pgfqpoint{1.524833in}{2.065595in}}%
\pgfpathcurveto{\pgfqpoint{1.524833in}{2.057359in}}{\pgfqpoint{1.528105in}{2.049459in}}{\pgfqpoint{1.533929in}{2.043635in}}%
\pgfpathcurveto{\pgfqpoint{1.539753in}{2.037811in}}{\pgfqpoint{1.547653in}{2.034538in}}{\pgfqpoint{1.555889in}{2.034538in}}%
\pgfpathclose%
\pgfusepath{stroke,fill}%
\end{pgfscope}%
\begin{pgfscope}%
\pgfpathrectangle{\pgfqpoint{0.100000in}{0.212622in}}{\pgfqpoint{3.696000in}{3.696000in}}%
\pgfusepath{clip}%
\pgfsetbuttcap%
\pgfsetroundjoin%
\definecolor{currentfill}{rgb}{0.121569,0.466667,0.705882}%
\pgfsetfillcolor{currentfill}%
\pgfsetfillopacity{0.300610}%
\pgfsetlinewidth{1.003750pt}%
\definecolor{currentstroke}{rgb}{0.121569,0.466667,0.705882}%
\pgfsetstrokecolor{currentstroke}%
\pgfsetstrokeopacity{0.300610}%
\pgfsetdash{}{0pt}%
\pgfpathmoveto{\pgfqpoint{1.555889in}{2.034538in}}%
\pgfpathcurveto{\pgfqpoint{1.564126in}{2.034538in}}{\pgfqpoint{1.572026in}{2.037811in}}{\pgfqpoint{1.577850in}{2.043635in}}%
\pgfpathcurveto{\pgfqpoint{1.583674in}{2.049459in}}{\pgfqpoint{1.586946in}{2.057359in}}{\pgfqpoint{1.586946in}{2.065595in}}%
\pgfpathcurveto{\pgfqpoint{1.586946in}{2.073831in}}{\pgfqpoint{1.583674in}{2.081731in}}{\pgfqpoint{1.577850in}{2.087555in}}%
\pgfpathcurveto{\pgfqpoint{1.572026in}{2.093379in}}{\pgfqpoint{1.564126in}{2.096651in}}{\pgfqpoint{1.555889in}{2.096651in}}%
\pgfpathcurveto{\pgfqpoint{1.547653in}{2.096651in}}{\pgfqpoint{1.539753in}{2.093379in}}{\pgfqpoint{1.533929in}{2.087555in}}%
\pgfpathcurveto{\pgfqpoint{1.528105in}{2.081731in}}{\pgfqpoint{1.524833in}{2.073831in}}{\pgfqpoint{1.524833in}{2.065595in}}%
\pgfpathcurveto{\pgfqpoint{1.524833in}{2.057359in}}{\pgfqpoint{1.528105in}{2.049459in}}{\pgfqpoint{1.533929in}{2.043635in}}%
\pgfpathcurveto{\pgfqpoint{1.539753in}{2.037811in}}{\pgfqpoint{1.547653in}{2.034538in}}{\pgfqpoint{1.555889in}{2.034538in}}%
\pgfpathclose%
\pgfusepath{stroke,fill}%
\end{pgfscope}%
\begin{pgfscope}%
\pgfpathrectangle{\pgfqpoint{0.100000in}{0.212622in}}{\pgfqpoint{3.696000in}{3.696000in}}%
\pgfusepath{clip}%
\pgfsetbuttcap%
\pgfsetroundjoin%
\definecolor{currentfill}{rgb}{0.121569,0.466667,0.705882}%
\pgfsetfillcolor{currentfill}%
\pgfsetfillopacity{0.300610}%
\pgfsetlinewidth{1.003750pt}%
\definecolor{currentstroke}{rgb}{0.121569,0.466667,0.705882}%
\pgfsetstrokecolor{currentstroke}%
\pgfsetstrokeopacity{0.300610}%
\pgfsetdash{}{0pt}%
\pgfpathmoveto{\pgfqpoint{1.555889in}{2.034538in}}%
\pgfpathcurveto{\pgfqpoint{1.564126in}{2.034538in}}{\pgfqpoint{1.572026in}{2.037811in}}{\pgfqpoint{1.577850in}{2.043635in}}%
\pgfpathcurveto{\pgfqpoint{1.583674in}{2.049459in}}{\pgfqpoint{1.586946in}{2.057359in}}{\pgfqpoint{1.586946in}{2.065595in}}%
\pgfpathcurveto{\pgfqpoint{1.586946in}{2.073831in}}{\pgfqpoint{1.583674in}{2.081731in}}{\pgfqpoint{1.577850in}{2.087555in}}%
\pgfpathcurveto{\pgfqpoint{1.572026in}{2.093379in}}{\pgfqpoint{1.564126in}{2.096651in}}{\pgfqpoint{1.555889in}{2.096651in}}%
\pgfpathcurveto{\pgfqpoint{1.547653in}{2.096651in}}{\pgfqpoint{1.539753in}{2.093379in}}{\pgfqpoint{1.533929in}{2.087555in}}%
\pgfpathcurveto{\pgfqpoint{1.528105in}{2.081731in}}{\pgfqpoint{1.524833in}{2.073831in}}{\pgfqpoint{1.524833in}{2.065595in}}%
\pgfpathcurveto{\pgfqpoint{1.524833in}{2.057359in}}{\pgfqpoint{1.528105in}{2.049459in}}{\pgfqpoint{1.533929in}{2.043635in}}%
\pgfpathcurveto{\pgfqpoint{1.539753in}{2.037811in}}{\pgfqpoint{1.547653in}{2.034538in}}{\pgfqpoint{1.555889in}{2.034538in}}%
\pgfpathclose%
\pgfusepath{stroke,fill}%
\end{pgfscope}%
\begin{pgfscope}%
\pgfpathrectangle{\pgfqpoint{0.100000in}{0.212622in}}{\pgfqpoint{3.696000in}{3.696000in}}%
\pgfusepath{clip}%
\pgfsetbuttcap%
\pgfsetroundjoin%
\definecolor{currentfill}{rgb}{0.121569,0.466667,0.705882}%
\pgfsetfillcolor{currentfill}%
\pgfsetfillopacity{0.300610}%
\pgfsetlinewidth{1.003750pt}%
\definecolor{currentstroke}{rgb}{0.121569,0.466667,0.705882}%
\pgfsetstrokecolor{currentstroke}%
\pgfsetstrokeopacity{0.300610}%
\pgfsetdash{}{0pt}%
\pgfpathmoveto{\pgfqpoint{1.555889in}{2.034538in}}%
\pgfpathcurveto{\pgfqpoint{1.564126in}{2.034538in}}{\pgfqpoint{1.572026in}{2.037811in}}{\pgfqpoint{1.577850in}{2.043635in}}%
\pgfpathcurveto{\pgfqpoint{1.583674in}{2.049459in}}{\pgfqpoint{1.586946in}{2.057359in}}{\pgfqpoint{1.586946in}{2.065595in}}%
\pgfpathcurveto{\pgfqpoint{1.586946in}{2.073831in}}{\pgfqpoint{1.583674in}{2.081731in}}{\pgfqpoint{1.577850in}{2.087555in}}%
\pgfpathcurveto{\pgfqpoint{1.572026in}{2.093379in}}{\pgfqpoint{1.564126in}{2.096651in}}{\pgfqpoint{1.555889in}{2.096651in}}%
\pgfpathcurveto{\pgfqpoint{1.547653in}{2.096651in}}{\pgfqpoint{1.539753in}{2.093379in}}{\pgfqpoint{1.533929in}{2.087555in}}%
\pgfpathcurveto{\pgfqpoint{1.528105in}{2.081731in}}{\pgfqpoint{1.524833in}{2.073831in}}{\pgfqpoint{1.524833in}{2.065595in}}%
\pgfpathcurveto{\pgfqpoint{1.524833in}{2.057359in}}{\pgfqpoint{1.528105in}{2.049459in}}{\pgfqpoint{1.533929in}{2.043635in}}%
\pgfpathcurveto{\pgfqpoint{1.539753in}{2.037811in}}{\pgfqpoint{1.547653in}{2.034538in}}{\pgfqpoint{1.555889in}{2.034538in}}%
\pgfpathclose%
\pgfusepath{stroke,fill}%
\end{pgfscope}%
\begin{pgfscope}%
\pgfpathrectangle{\pgfqpoint{0.100000in}{0.212622in}}{\pgfqpoint{3.696000in}{3.696000in}}%
\pgfusepath{clip}%
\pgfsetbuttcap%
\pgfsetroundjoin%
\definecolor{currentfill}{rgb}{0.121569,0.466667,0.705882}%
\pgfsetfillcolor{currentfill}%
\pgfsetfillopacity{0.300610}%
\pgfsetlinewidth{1.003750pt}%
\definecolor{currentstroke}{rgb}{0.121569,0.466667,0.705882}%
\pgfsetstrokecolor{currentstroke}%
\pgfsetstrokeopacity{0.300610}%
\pgfsetdash{}{0pt}%
\pgfpathmoveto{\pgfqpoint{1.555889in}{2.034538in}}%
\pgfpathcurveto{\pgfqpoint{1.564126in}{2.034538in}}{\pgfqpoint{1.572026in}{2.037811in}}{\pgfqpoint{1.577850in}{2.043635in}}%
\pgfpathcurveto{\pgfqpoint{1.583674in}{2.049459in}}{\pgfqpoint{1.586946in}{2.057359in}}{\pgfqpoint{1.586946in}{2.065595in}}%
\pgfpathcurveto{\pgfqpoint{1.586946in}{2.073831in}}{\pgfqpoint{1.583674in}{2.081731in}}{\pgfqpoint{1.577850in}{2.087555in}}%
\pgfpathcurveto{\pgfqpoint{1.572026in}{2.093379in}}{\pgfqpoint{1.564126in}{2.096651in}}{\pgfqpoint{1.555889in}{2.096651in}}%
\pgfpathcurveto{\pgfqpoint{1.547653in}{2.096651in}}{\pgfqpoint{1.539753in}{2.093379in}}{\pgfqpoint{1.533929in}{2.087555in}}%
\pgfpathcurveto{\pgfqpoint{1.528105in}{2.081731in}}{\pgfqpoint{1.524833in}{2.073831in}}{\pgfqpoint{1.524833in}{2.065595in}}%
\pgfpathcurveto{\pgfqpoint{1.524833in}{2.057359in}}{\pgfqpoint{1.528105in}{2.049459in}}{\pgfqpoint{1.533929in}{2.043635in}}%
\pgfpathcurveto{\pgfqpoint{1.539753in}{2.037811in}}{\pgfqpoint{1.547653in}{2.034538in}}{\pgfqpoint{1.555889in}{2.034538in}}%
\pgfpathclose%
\pgfusepath{stroke,fill}%
\end{pgfscope}%
\begin{pgfscope}%
\pgfpathrectangle{\pgfqpoint{0.100000in}{0.212622in}}{\pgfqpoint{3.696000in}{3.696000in}}%
\pgfusepath{clip}%
\pgfsetbuttcap%
\pgfsetroundjoin%
\definecolor{currentfill}{rgb}{0.121569,0.466667,0.705882}%
\pgfsetfillcolor{currentfill}%
\pgfsetfillopacity{0.300610}%
\pgfsetlinewidth{1.003750pt}%
\definecolor{currentstroke}{rgb}{0.121569,0.466667,0.705882}%
\pgfsetstrokecolor{currentstroke}%
\pgfsetstrokeopacity{0.300610}%
\pgfsetdash{}{0pt}%
\pgfpathmoveto{\pgfqpoint{1.555889in}{2.034538in}}%
\pgfpathcurveto{\pgfqpoint{1.564126in}{2.034538in}}{\pgfqpoint{1.572026in}{2.037811in}}{\pgfqpoint{1.577850in}{2.043635in}}%
\pgfpathcurveto{\pgfqpoint{1.583674in}{2.049459in}}{\pgfqpoint{1.586946in}{2.057359in}}{\pgfqpoint{1.586946in}{2.065595in}}%
\pgfpathcurveto{\pgfqpoint{1.586946in}{2.073831in}}{\pgfqpoint{1.583674in}{2.081731in}}{\pgfqpoint{1.577850in}{2.087555in}}%
\pgfpathcurveto{\pgfqpoint{1.572026in}{2.093379in}}{\pgfqpoint{1.564126in}{2.096651in}}{\pgfqpoint{1.555889in}{2.096651in}}%
\pgfpathcurveto{\pgfqpoint{1.547653in}{2.096651in}}{\pgfqpoint{1.539753in}{2.093379in}}{\pgfqpoint{1.533929in}{2.087555in}}%
\pgfpathcurveto{\pgfqpoint{1.528105in}{2.081731in}}{\pgfqpoint{1.524833in}{2.073831in}}{\pgfqpoint{1.524833in}{2.065595in}}%
\pgfpathcurveto{\pgfqpoint{1.524833in}{2.057359in}}{\pgfqpoint{1.528105in}{2.049459in}}{\pgfqpoint{1.533929in}{2.043635in}}%
\pgfpathcurveto{\pgfqpoint{1.539753in}{2.037811in}}{\pgfqpoint{1.547653in}{2.034538in}}{\pgfqpoint{1.555889in}{2.034538in}}%
\pgfpathclose%
\pgfusepath{stroke,fill}%
\end{pgfscope}%
\begin{pgfscope}%
\pgfpathrectangle{\pgfqpoint{0.100000in}{0.212622in}}{\pgfqpoint{3.696000in}{3.696000in}}%
\pgfusepath{clip}%
\pgfsetbuttcap%
\pgfsetroundjoin%
\definecolor{currentfill}{rgb}{0.121569,0.466667,0.705882}%
\pgfsetfillcolor{currentfill}%
\pgfsetfillopacity{0.300610}%
\pgfsetlinewidth{1.003750pt}%
\definecolor{currentstroke}{rgb}{0.121569,0.466667,0.705882}%
\pgfsetstrokecolor{currentstroke}%
\pgfsetstrokeopacity{0.300610}%
\pgfsetdash{}{0pt}%
\pgfpathmoveto{\pgfqpoint{1.555889in}{2.034538in}}%
\pgfpathcurveto{\pgfqpoint{1.564126in}{2.034538in}}{\pgfqpoint{1.572026in}{2.037811in}}{\pgfqpoint{1.577850in}{2.043635in}}%
\pgfpathcurveto{\pgfqpoint{1.583674in}{2.049459in}}{\pgfqpoint{1.586946in}{2.057359in}}{\pgfqpoint{1.586946in}{2.065595in}}%
\pgfpathcurveto{\pgfqpoint{1.586946in}{2.073831in}}{\pgfqpoint{1.583674in}{2.081731in}}{\pgfqpoint{1.577850in}{2.087555in}}%
\pgfpathcurveto{\pgfqpoint{1.572026in}{2.093379in}}{\pgfqpoint{1.564126in}{2.096651in}}{\pgfqpoint{1.555889in}{2.096651in}}%
\pgfpathcurveto{\pgfqpoint{1.547653in}{2.096651in}}{\pgfqpoint{1.539753in}{2.093379in}}{\pgfqpoint{1.533929in}{2.087555in}}%
\pgfpathcurveto{\pgfqpoint{1.528105in}{2.081731in}}{\pgfqpoint{1.524833in}{2.073831in}}{\pgfqpoint{1.524833in}{2.065595in}}%
\pgfpathcurveto{\pgfqpoint{1.524833in}{2.057359in}}{\pgfqpoint{1.528105in}{2.049459in}}{\pgfqpoint{1.533929in}{2.043635in}}%
\pgfpathcurveto{\pgfqpoint{1.539753in}{2.037811in}}{\pgfqpoint{1.547653in}{2.034538in}}{\pgfqpoint{1.555889in}{2.034538in}}%
\pgfpathclose%
\pgfusepath{stroke,fill}%
\end{pgfscope}%
\begin{pgfscope}%
\pgfpathrectangle{\pgfqpoint{0.100000in}{0.212622in}}{\pgfqpoint{3.696000in}{3.696000in}}%
\pgfusepath{clip}%
\pgfsetbuttcap%
\pgfsetroundjoin%
\definecolor{currentfill}{rgb}{0.121569,0.466667,0.705882}%
\pgfsetfillcolor{currentfill}%
\pgfsetfillopacity{0.300610}%
\pgfsetlinewidth{1.003750pt}%
\definecolor{currentstroke}{rgb}{0.121569,0.466667,0.705882}%
\pgfsetstrokecolor{currentstroke}%
\pgfsetstrokeopacity{0.300610}%
\pgfsetdash{}{0pt}%
\pgfpathmoveto{\pgfqpoint{1.555889in}{2.034538in}}%
\pgfpathcurveto{\pgfqpoint{1.564126in}{2.034538in}}{\pgfqpoint{1.572026in}{2.037811in}}{\pgfqpoint{1.577850in}{2.043635in}}%
\pgfpathcurveto{\pgfqpoint{1.583674in}{2.049459in}}{\pgfqpoint{1.586946in}{2.057359in}}{\pgfqpoint{1.586946in}{2.065595in}}%
\pgfpathcurveto{\pgfqpoint{1.586946in}{2.073831in}}{\pgfqpoint{1.583674in}{2.081731in}}{\pgfqpoint{1.577850in}{2.087555in}}%
\pgfpathcurveto{\pgfqpoint{1.572026in}{2.093379in}}{\pgfqpoint{1.564126in}{2.096651in}}{\pgfqpoint{1.555889in}{2.096651in}}%
\pgfpathcurveto{\pgfqpoint{1.547653in}{2.096651in}}{\pgfqpoint{1.539753in}{2.093379in}}{\pgfqpoint{1.533929in}{2.087555in}}%
\pgfpathcurveto{\pgfqpoint{1.528105in}{2.081731in}}{\pgfqpoint{1.524833in}{2.073831in}}{\pgfqpoint{1.524833in}{2.065595in}}%
\pgfpathcurveto{\pgfqpoint{1.524833in}{2.057359in}}{\pgfqpoint{1.528105in}{2.049459in}}{\pgfqpoint{1.533929in}{2.043635in}}%
\pgfpathcurveto{\pgfqpoint{1.539753in}{2.037811in}}{\pgfqpoint{1.547653in}{2.034538in}}{\pgfqpoint{1.555889in}{2.034538in}}%
\pgfpathclose%
\pgfusepath{stroke,fill}%
\end{pgfscope}%
\begin{pgfscope}%
\pgfpathrectangle{\pgfqpoint{0.100000in}{0.212622in}}{\pgfqpoint{3.696000in}{3.696000in}}%
\pgfusepath{clip}%
\pgfsetbuttcap%
\pgfsetroundjoin%
\definecolor{currentfill}{rgb}{0.121569,0.466667,0.705882}%
\pgfsetfillcolor{currentfill}%
\pgfsetfillopacity{0.300610}%
\pgfsetlinewidth{1.003750pt}%
\definecolor{currentstroke}{rgb}{0.121569,0.466667,0.705882}%
\pgfsetstrokecolor{currentstroke}%
\pgfsetstrokeopacity{0.300610}%
\pgfsetdash{}{0pt}%
\pgfpathmoveto{\pgfqpoint{1.555889in}{2.034538in}}%
\pgfpathcurveto{\pgfqpoint{1.564126in}{2.034538in}}{\pgfqpoint{1.572026in}{2.037811in}}{\pgfqpoint{1.577850in}{2.043635in}}%
\pgfpathcurveto{\pgfqpoint{1.583674in}{2.049459in}}{\pgfqpoint{1.586946in}{2.057359in}}{\pgfqpoint{1.586946in}{2.065595in}}%
\pgfpathcurveto{\pgfqpoint{1.586946in}{2.073831in}}{\pgfqpoint{1.583674in}{2.081731in}}{\pgfqpoint{1.577850in}{2.087555in}}%
\pgfpathcurveto{\pgfqpoint{1.572026in}{2.093379in}}{\pgfqpoint{1.564126in}{2.096651in}}{\pgfqpoint{1.555889in}{2.096651in}}%
\pgfpathcurveto{\pgfqpoint{1.547653in}{2.096651in}}{\pgfqpoint{1.539753in}{2.093379in}}{\pgfqpoint{1.533929in}{2.087555in}}%
\pgfpathcurveto{\pgfqpoint{1.528105in}{2.081731in}}{\pgfqpoint{1.524833in}{2.073831in}}{\pgfqpoint{1.524833in}{2.065595in}}%
\pgfpathcurveto{\pgfqpoint{1.524833in}{2.057359in}}{\pgfqpoint{1.528105in}{2.049459in}}{\pgfqpoint{1.533929in}{2.043635in}}%
\pgfpathcurveto{\pgfqpoint{1.539753in}{2.037811in}}{\pgfqpoint{1.547653in}{2.034538in}}{\pgfqpoint{1.555889in}{2.034538in}}%
\pgfpathclose%
\pgfusepath{stroke,fill}%
\end{pgfscope}%
\begin{pgfscope}%
\pgfpathrectangle{\pgfqpoint{0.100000in}{0.212622in}}{\pgfqpoint{3.696000in}{3.696000in}}%
\pgfusepath{clip}%
\pgfsetbuttcap%
\pgfsetroundjoin%
\definecolor{currentfill}{rgb}{0.121569,0.466667,0.705882}%
\pgfsetfillcolor{currentfill}%
\pgfsetfillopacity{0.300610}%
\pgfsetlinewidth{1.003750pt}%
\definecolor{currentstroke}{rgb}{0.121569,0.466667,0.705882}%
\pgfsetstrokecolor{currentstroke}%
\pgfsetstrokeopacity{0.300610}%
\pgfsetdash{}{0pt}%
\pgfpathmoveto{\pgfqpoint{1.555889in}{2.034538in}}%
\pgfpathcurveto{\pgfqpoint{1.564126in}{2.034538in}}{\pgfqpoint{1.572026in}{2.037811in}}{\pgfqpoint{1.577850in}{2.043635in}}%
\pgfpathcurveto{\pgfqpoint{1.583674in}{2.049459in}}{\pgfqpoint{1.586946in}{2.057359in}}{\pgfqpoint{1.586946in}{2.065595in}}%
\pgfpathcurveto{\pgfqpoint{1.586946in}{2.073831in}}{\pgfqpoint{1.583674in}{2.081731in}}{\pgfqpoint{1.577850in}{2.087555in}}%
\pgfpathcurveto{\pgfqpoint{1.572026in}{2.093379in}}{\pgfqpoint{1.564126in}{2.096651in}}{\pgfqpoint{1.555889in}{2.096651in}}%
\pgfpathcurveto{\pgfqpoint{1.547653in}{2.096651in}}{\pgfqpoint{1.539753in}{2.093379in}}{\pgfqpoint{1.533929in}{2.087555in}}%
\pgfpathcurveto{\pgfqpoint{1.528105in}{2.081731in}}{\pgfqpoint{1.524833in}{2.073831in}}{\pgfqpoint{1.524833in}{2.065595in}}%
\pgfpathcurveto{\pgfqpoint{1.524833in}{2.057359in}}{\pgfqpoint{1.528105in}{2.049459in}}{\pgfqpoint{1.533929in}{2.043635in}}%
\pgfpathcurveto{\pgfqpoint{1.539753in}{2.037811in}}{\pgfqpoint{1.547653in}{2.034538in}}{\pgfqpoint{1.555889in}{2.034538in}}%
\pgfpathclose%
\pgfusepath{stroke,fill}%
\end{pgfscope}%
\begin{pgfscope}%
\pgfpathrectangle{\pgfqpoint{0.100000in}{0.212622in}}{\pgfqpoint{3.696000in}{3.696000in}}%
\pgfusepath{clip}%
\pgfsetbuttcap%
\pgfsetroundjoin%
\definecolor{currentfill}{rgb}{0.121569,0.466667,0.705882}%
\pgfsetfillcolor{currentfill}%
\pgfsetfillopacity{0.300610}%
\pgfsetlinewidth{1.003750pt}%
\definecolor{currentstroke}{rgb}{0.121569,0.466667,0.705882}%
\pgfsetstrokecolor{currentstroke}%
\pgfsetstrokeopacity{0.300610}%
\pgfsetdash{}{0pt}%
\pgfpathmoveto{\pgfqpoint{1.555889in}{2.034538in}}%
\pgfpathcurveto{\pgfqpoint{1.564126in}{2.034538in}}{\pgfqpoint{1.572026in}{2.037811in}}{\pgfqpoint{1.577850in}{2.043635in}}%
\pgfpathcurveto{\pgfqpoint{1.583674in}{2.049459in}}{\pgfqpoint{1.586946in}{2.057359in}}{\pgfqpoint{1.586946in}{2.065595in}}%
\pgfpathcurveto{\pgfqpoint{1.586946in}{2.073831in}}{\pgfqpoint{1.583674in}{2.081731in}}{\pgfqpoint{1.577850in}{2.087555in}}%
\pgfpathcurveto{\pgfqpoint{1.572026in}{2.093379in}}{\pgfqpoint{1.564126in}{2.096651in}}{\pgfqpoint{1.555889in}{2.096651in}}%
\pgfpathcurveto{\pgfqpoint{1.547653in}{2.096651in}}{\pgfqpoint{1.539753in}{2.093379in}}{\pgfqpoint{1.533929in}{2.087555in}}%
\pgfpathcurveto{\pgfqpoint{1.528105in}{2.081731in}}{\pgfqpoint{1.524833in}{2.073831in}}{\pgfqpoint{1.524833in}{2.065595in}}%
\pgfpathcurveto{\pgfqpoint{1.524833in}{2.057359in}}{\pgfqpoint{1.528105in}{2.049459in}}{\pgfqpoint{1.533929in}{2.043635in}}%
\pgfpathcurveto{\pgfqpoint{1.539753in}{2.037811in}}{\pgfqpoint{1.547653in}{2.034538in}}{\pgfqpoint{1.555889in}{2.034538in}}%
\pgfpathclose%
\pgfusepath{stroke,fill}%
\end{pgfscope}%
\begin{pgfscope}%
\pgfpathrectangle{\pgfqpoint{0.100000in}{0.212622in}}{\pgfqpoint{3.696000in}{3.696000in}}%
\pgfusepath{clip}%
\pgfsetbuttcap%
\pgfsetroundjoin%
\definecolor{currentfill}{rgb}{0.121569,0.466667,0.705882}%
\pgfsetfillcolor{currentfill}%
\pgfsetfillopacity{0.300610}%
\pgfsetlinewidth{1.003750pt}%
\definecolor{currentstroke}{rgb}{0.121569,0.466667,0.705882}%
\pgfsetstrokecolor{currentstroke}%
\pgfsetstrokeopacity{0.300610}%
\pgfsetdash{}{0pt}%
\pgfpathmoveto{\pgfqpoint{1.555889in}{2.034538in}}%
\pgfpathcurveto{\pgfqpoint{1.564126in}{2.034538in}}{\pgfqpoint{1.572026in}{2.037811in}}{\pgfqpoint{1.577850in}{2.043635in}}%
\pgfpathcurveto{\pgfqpoint{1.583674in}{2.049459in}}{\pgfqpoint{1.586946in}{2.057359in}}{\pgfqpoint{1.586946in}{2.065595in}}%
\pgfpathcurveto{\pgfqpoint{1.586946in}{2.073831in}}{\pgfqpoint{1.583674in}{2.081731in}}{\pgfqpoint{1.577850in}{2.087555in}}%
\pgfpathcurveto{\pgfqpoint{1.572026in}{2.093379in}}{\pgfqpoint{1.564126in}{2.096651in}}{\pgfqpoint{1.555889in}{2.096651in}}%
\pgfpathcurveto{\pgfqpoint{1.547653in}{2.096651in}}{\pgfqpoint{1.539753in}{2.093379in}}{\pgfqpoint{1.533929in}{2.087555in}}%
\pgfpathcurveto{\pgfqpoint{1.528105in}{2.081731in}}{\pgfqpoint{1.524833in}{2.073831in}}{\pgfqpoint{1.524833in}{2.065595in}}%
\pgfpathcurveto{\pgfqpoint{1.524833in}{2.057359in}}{\pgfqpoint{1.528105in}{2.049459in}}{\pgfqpoint{1.533929in}{2.043635in}}%
\pgfpathcurveto{\pgfqpoint{1.539753in}{2.037811in}}{\pgfqpoint{1.547653in}{2.034538in}}{\pgfqpoint{1.555889in}{2.034538in}}%
\pgfpathclose%
\pgfusepath{stroke,fill}%
\end{pgfscope}%
\begin{pgfscope}%
\pgfpathrectangle{\pgfqpoint{0.100000in}{0.212622in}}{\pgfqpoint{3.696000in}{3.696000in}}%
\pgfusepath{clip}%
\pgfsetbuttcap%
\pgfsetroundjoin%
\definecolor{currentfill}{rgb}{0.121569,0.466667,0.705882}%
\pgfsetfillcolor{currentfill}%
\pgfsetfillopacity{0.300610}%
\pgfsetlinewidth{1.003750pt}%
\definecolor{currentstroke}{rgb}{0.121569,0.466667,0.705882}%
\pgfsetstrokecolor{currentstroke}%
\pgfsetstrokeopacity{0.300610}%
\pgfsetdash{}{0pt}%
\pgfpathmoveto{\pgfqpoint{1.555889in}{2.034538in}}%
\pgfpathcurveto{\pgfqpoint{1.564126in}{2.034538in}}{\pgfqpoint{1.572026in}{2.037811in}}{\pgfqpoint{1.577850in}{2.043635in}}%
\pgfpathcurveto{\pgfqpoint{1.583674in}{2.049459in}}{\pgfqpoint{1.586946in}{2.057359in}}{\pgfqpoint{1.586946in}{2.065595in}}%
\pgfpathcurveto{\pgfqpoint{1.586946in}{2.073831in}}{\pgfqpoint{1.583674in}{2.081731in}}{\pgfqpoint{1.577850in}{2.087555in}}%
\pgfpathcurveto{\pgfqpoint{1.572026in}{2.093379in}}{\pgfqpoint{1.564126in}{2.096651in}}{\pgfqpoint{1.555889in}{2.096651in}}%
\pgfpathcurveto{\pgfqpoint{1.547653in}{2.096651in}}{\pgfqpoint{1.539753in}{2.093379in}}{\pgfqpoint{1.533929in}{2.087555in}}%
\pgfpathcurveto{\pgfqpoint{1.528105in}{2.081731in}}{\pgfqpoint{1.524833in}{2.073831in}}{\pgfqpoint{1.524833in}{2.065595in}}%
\pgfpathcurveto{\pgfqpoint{1.524833in}{2.057359in}}{\pgfqpoint{1.528105in}{2.049459in}}{\pgfqpoint{1.533929in}{2.043635in}}%
\pgfpathcurveto{\pgfqpoint{1.539753in}{2.037811in}}{\pgfqpoint{1.547653in}{2.034538in}}{\pgfqpoint{1.555889in}{2.034538in}}%
\pgfpathclose%
\pgfusepath{stroke,fill}%
\end{pgfscope}%
\begin{pgfscope}%
\pgfpathrectangle{\pgfqpoint{0.100000in}{0.212622in}}{\pgfqpoint{3.696000in}{3.696000in}}%
\pgfusepath{clip}%
\pgfsetbuttcap%
\pgfsetroundjoin%
\definecolor{currentfill}{rgb}{0.121569,0.466667,0.705882}%
\pgfsetfillcolor{currentfill}%
\pgfsetfillopacity{0.300610}%
\pgfsetlinewidth{1.003750pt}%
\definecolor{currentstroke}{rgb}{0.121569,0.466667,0.705882}%
\pgfsetstrokecolor{currentstroke}%
\pgfsetstrokeopacity{0.300610}%
\pgfsetdash{}{0pt}%
\pgfpathmoveto{\pgfqpoint{1.555889in}{2.034538in}}%
\pgfpathcurveto{\pgfqpoint{1.564126in}{2.034538in}}{\pgfqpoint{1.572026in}{2.037811in}}{\pgfqpoint{1.577850in}{2.043635in}}%
\pgfpathcurveto{\pgfqpoint{1.583674in}{2.049459in}}{\pgfqpoint{1.586946in}{2.057359in}}{\pgfqpoint{1.586946in}{2.065595in}}%
\pgfpathcurveto{\pgfqpoint{1.586946in}{2.073831in}}{\pgfqpoint{1.583674in}{2.081731in}}{\pgfqpoint{1.577850in}{2.087555in}}%
\pgfpathcurveto{\pgfqpoint{1.572026in}{2.093379in}}{\pgfqpoint{1.564126in}{2.096651in}}{\pgfqpoint{1.555889in}{2.096651in}}%
\pgfpathcurveto{\pgfqpoint{1.547653in}{2.096651in}}{\pgfqpoint{1.539753in}{2.093379in}}{\pgfqpoint{1.533929in}{2.087555in}}%
\pgfpathcurveto{\pgfqpoint{1.528105in}{2.081731in}}{\pgfqpoint{1.524833in}{2.073831in}}{\pgfqpoint{1.524833in}{2.065595in}}%
\pgfpathcurveto{\pgfqpoint{1.524833in}{2.057359in}}{\pgfqpoint{1.528105in}{2.049459in}}{\pgfqpoint{1.533929in}{2.043635in}}%
\pgfpathcurveto{\pgfqpoint{1.539753in}{2.037811in}}{\pgfqpoint{1.547653in}{2.034538in}}{\pgfqpoint{1.555889in}{2.034538in}}%
\pgfpathclose%
\pgfusepath{stroke,fill}%
\end{pgfscope}%
\begin{pgfscope}%
\pgfpathrectangle{\pgfqpoint{0.100000in}{0.212622in}}{\pgfqpoint{3.696000in}{3.696000in}}%
\pgfusepath{clip}%
\pgfsetbuttcap%
\pgfsetroundjoin%
\definecolor{currentfill}{rgb}{0.121569,0.466667,0.705882}%
\pgfsetfillcolor{currentfill}%
\pgfsetfillopacity{0.300610}%
\pgfsetlinewidth{1.003750pt}%
\definecolor{currentstroke}{rgb}{0.121569,0.466667,0.705882}%
\pgfsetstrokecolor{currentstroke}%
\pgfsetstrokeopacity{0.300610}%
\pgfsetdash{}{0pt}%
\pgfpathmoveto{\pgfqpoint{1.555889in}{2.034538in}}%
\pgfpathcurveto{\pgfqpoint{1.564126in}{2.034538in}}{\pgfqpoint{1.572026in}{2.037811in}}{\pgfqpoint{1.577850in}{2.043635in}}%
\pgfpathcurveto{\pgfqpoint{1.583674in}{2.049459in}}{\pgfqpoint{1.586946in}{2.057359in}}{\pgfqpoint{1.586946in}{2.065595in}}%
\pgfpathcurveto{\pgfqpoint{1.586946in}{2.073831in}}{\pgfqpoint{1.583674in}{2.081731in}}{\pgfqpoint{1.577850in}{2.087555in}}%
\pgfpathcurveto{\pgfqpoint{1.572026in}{2.093379in}}{\pgfqpoint{1.564126in}{2.096651in}}{\pgfqpoint{1.555889in}{2.096651in}}%
\pgfpathcurveto{\pgfqpoint{1.547653in}{2.096651in}}{\pgfqpoint{1.539753in}{2.093379in}}{\pgfqpoint{1.533929in}{2.087555in}}%
\pgfpathcurveto{\pgfqpoint{1.528105in}{2.081731in}}{\pgfqpoint{1.524833in}{2.073831in}}{\pgfqpoint{1.524833in}{2.065595in}}%
\pgfpathcurveto{\pgfqpoint{1.524833in}{2.057359in}}{\pgfqpoint{1.528105in}{2.049459in}}{\pgfqpoint{1.533929in}{2.043635in}}%
\pgfpathcurveto{\pgfqpoint{1.539753in}{2.037811in}}{\pgfqpoint{1.547653in}{2.034538in}}{\pgfqpoint{1.555889in}{2.034538in}}%
\pgfpathclose%
\pgfusepath{stroke,fill}%
\end{pgfscope}%
\begin{pgfscope}%
\pgfpathrectangle{\pgfqpoint{0.100000in}{0.212622in}}{\pgfqpoint{3.696000in}{3.696000in}}%
\pgfusepath{clip}%
\pgfsetbuttcap%
\pgfsetroundjoin%
\definecolor{currentfill}{rgb}{0.121569,0.466667,0.705882}%
\pgfsetfillcolor{currentfill}%
\pgfsetfillopacity{0.300610}%
\pgfsetlinewidth{1.003750pt}%
\definecolor{currentstroke}{rgb}{0.121569,0.466667,0.705882}%
\pgfsetstrokecolor{currentstroke}%
\pgfsetstrokeopacity{0.300610}%
\pgfsetdash{}{0pt}%
\pgfpathmoveto{\pgfqpoint{1.555889in}{2.034538in}}%
\pgfpathcurveto{\pgfqpoint{1.564126in}{2.034538in}}{\pgfqpoint{1.572026in}{2.037811in}}{\pgfqpoint{1.577850in}{2.043635in}}%
\pgfpathcurveto{\pgfqpoint{1.583674in}{2.049459in}}{\pgfqpoint{1.586946in}{2.057359in}}{\pgfqpoint{1.586946in}{2.065595in}}%
\pgfpathcurveto{\pgfqpoint{1.586946in}{2.073831in}}{\pgfqpoint{1.583674in}{2.081731in}}{\pgfqpoint{1.577850in}{2.087555in}}%
\pgfpathcurveto{\pgfqpoint{1.572026in}{2.093379in}}{\pgfqpoint{1.564126in}{2.096651in}}{\pgfqpoint{1.555889in}{2.096651in}}%
\pgfpathcurveto{\pgfqpoint{1.547653in}{2.096651in}}{\pgfqpoint{1.539753in}{2.093379in}}{\pgfqpoint{1.533929in}{2.087555in}}%
\pgfpathcurveto{\pgfqpoint{1.528105in}{2.081731in}}{\pgfqpoint{1.524833in}{2.073831in}}{\pgfqpoint{1.524833in}{2.065595in}}%
\pgfpathcurveto{\pgfqpoint{1.524833in}{2.057359in}}{\pgfqpoint{1.528105in}{2.049459in}}{\pgfqpoint{1.533929in}{2.043635in}}%
\pgfpathcurveto{\pgfqpoint{1.539753in}{2.037811in}}{\pgfqpoint{1.547653in}{2.034538in}}{\pgfqpoint{1.555889in}{2.034538in}}%
\pgfpathclose%
\pgfusepath{stroke,fill}%
\end{pgfscope}%
\begin{pgfscope}%
\pgfpathrectangle{\pgfqpoint{0.100000in}{0.212622in}}{\pgfqpoint{3.696000in}{3.696000in}}%
\pgfusepath{clip}%
\pgfsetbuttcap%
\pgfsetroundjoin%
\definecolor{currentfill}{rgb}{0.121569,0.466667,0.705882}%
\pgfsetfillcolor{currentfill}%
\pgfsetfillopacity{0.300610}%
\pgfsetlinewidth{1.003750pt}%
\definecolor{currentstroke}{rgb}{0.121569,0.466667,0.705882}%
\pgfsetstrokecolor{currentstroke}%
\pgfsetstrokeopacity{0.300610}%
\pgfsetdash{}{0pt}%
\pgfpathmoveto{\pgfqpoint{1.555889in}{2.034538in}}%
\pgfpathcurveto{\pgfqpoint{1.564126in}{2.034538in}}{\pgfqpoint{1.572026in}{2.037811in}}{\pgfqpoint{1.577850in}{2.043635in}}%
\pgfpathcurveto{\pgfqpoint{1.583674in}{2.049459in}}{\pgfqpoint{1.586946in}{2.057359in}}{\pgfqpoint{1.586946in}{2.065595in}}%
\pgfpathcurveto{\pgfqpoint{1.586946in}{2.073831in}}{\pgfqpoint{1.583674in}{2.081731in}}{\pgfqpoint{1.577850in}{2.087555in}}%
\pgfpathcurveto{\pgfqpoint{1.572026in}{2.093379in}}{\pgfqpoint{1.564126in}{2.096651in}}{\pgfqpoint{1.555889in}{2.096651in}}%
\pgfpathcurveto{\pgfqpoint{1.547653in}{2.096651in}}{\pgfqpoint{1.539753in}{2.093379in}}{\pgfqpoint{1.533929in}{2.087555in}}%
\pgfpathcurveto{\pgfqpoint{1.528105in}{2.081731in}}{\pgfqpoint{1.524833in}{2.073831in}}{\pgfqpoint{1.524833in}{2.065595in}}%
\pgfpathcurveto{\pgfqpoint{1.524833in}{2.057359in}}{\pgfqpoint{1.528105in}{2.049459in}}{\pgfqpoint{1.533929in}{2.043635in}}%
\pgfpathcurveto{\pgfqpoint{1.539753in}{2.037811in}}{\pgfqpoint{1.547653in}{2.034538in}}{\pgfqpoint{1.555889in}{2.034538in}}%
\pgfpathclose%
\pgfusepath{stroke,fill}%
\end{pgfscope}%
\begin{pgfscope}%
\pgfpathrectangle{\pgfqpoint{0.100000in}{0.212622in}}{\pgfqpoint{3.696000in}{3.696000in}}%
\pgfusepath{clip}%
\pgfsetbuttcap%
\pgfsetroundjoin%
\definecolor{currentfill}{rgb}{0.121569,0.466667,0.705882}%
\pgfsetfillcolor{currentfill}%
\pgfsetfillopacity{0.300610}%
\pgfsetlinewidth{1.003750pt}%
\definecolor{currentstroke}{rgb}{0.121569,0.466667,0.705882}%
\pgfsetstrokecolor{currentstroke}%
\pgfsetstrokeopacity{0.300610}%
\pgfsetdash{}{0pt}%
\pgfpathmoveto{\pgfqpoint{1.555889in}{2.034538in}}%
\pgfpathcurveto{\pgfqpoint{1.564126in}{2.034538in}}{\pgfqpoint{1.572026in}{2.037811in}}{\pgfqpoint{1.577850in}{2.043635in}}%
\pgfpathcurveto{\pgfqpoint{1.583674in}{2.049459in}}{\pgfqpoint{1.586946in}{2.057359in}}{\pgfqpoint{1.586946in}{2.065595in}}%
\pgfpathcurveto{\pgfqpoint{1.586946in}{2.073831in}}{\pgfqpoint{1.583674in}{2.081731in}}{\pgfqpoint{1.577850in}{2.087555in}}%
\pgfpathcurveto{\pgfqpoint{1.572026in}{2.093379in}}{\pgfqpoint{1.564126in}{2.096651in}}{\pgfqpoint{1.555889in}{2.096651in}}%
\pgfpathcurveto{\pgfqpoint{1.547653in}{2.096651in}}{\pgfqpoint{1.539753in}{2.093379in}}{\pgfqpoint{1.533929in}{2.087555in}}%
\pgfpathcurveto{\pgfqpoint{1.528105in}{2.081731in}}{\pgfqpoint{1.524833in}{2.073831in}}{\pgfqpoint{1.524833in}{2.065595in}}%
\pgfpathcurveto{\pgfqpoint{1.524833in}{2.057359in}}{\pgfqpoint{1.528105in}{2.049459in}}{\pgfqpoint{1.533929in}{2.043635in}}%
\pgfpathcurveto{\pgfqpoint{1.539753in}{2.037811in}}{\pgfqpoint{1.547653in}{2.034538in}}{\pgfqpoint{1.555889in}{2.034538in}}%
\pgfpathclose%
\pgfusepath{stroke,fill}%
\end{pgfscope}%
\begin{pgfscope}%
\pgfpathrectangle{\pgfqpoint{0.100000in}{0.212622in}}{\pgfqpoint{3.696000in}{3.696000in}}%
\pgfusepath{clip}%
\pgfsetbuttcap%
\pgfsetroundjoin%
\definecolor{currentfill}{rgb}{0.121569,0.466667,0.705882}%
\pgfsetfillcolor{currentfill}%
\pgfsetfillopacity{0.300610}%
\pgfsetlinewidth{1.003750pt}%
\definecolor{currentstroke}{rgb}{0.121569,0.466667,0.705882}%
\pgfsetstrokecolor{currentstroke}%
\pgfsetstrokeopacity{0.300610}%
\pgfsetdash{}{0pt}%
\pgfpathmoveto{\pgfqpoint{1.555889in}{2.034538in}}%
\pgfpathcurveto{\pgfqpoint{1.564126in}{2.034538in}}{\pgfqpoint{1.572026in}{2.037811in}}{\pgfqpoint{1.577850in}{2.043635in}}%
\pgfpathcurveto{\pgfqpoint{1.583674in}{2.049459in}}{\pgfqpoint{1.586946in}{2.057359in}}{\pgfqpoint{1.586946in}{2.065595in}}%
\pgfpathcurveto{\pgfqpoint{1.586946in}{2.073831in}}{\pgfqpoint{1.583674in}{2.081731in}}{\pgfqpoint{1.577850in}{2.087555in}}%
\pgfpathcurveto{\pgfqpoint{1.572026in}{2.093379in}}{\pgfqpoint{1.564126in}{2.096651in}}{\pgfqpoint{1.555889in}{2.096651in}}%
\pgfpathcurveto{\pgfqpoint{1.547653in}{2.096651in}}{\pgfqpoint{1.539753in}{2.093379in}}{\pgfqpoint{1.533929in}{2.087555in}}%
\pgfpathcurveto{\pgfqpoint{1.528105in}{2.081731in}}{\pgfqpoint{1.524833in}{2.073831in}}{\pgfqpoint{1.524833in}{2.065595in}}%
\pgfpathcurveto{\pgfqpoint{1.524833in}{2.057359in}}{\pgfqpoint{1.528105in}{2.049459in}}{\pgfqpoint{1.533929in}{2.043635in}}%
\pgfpathcurveto{\pgfqpoint{1.539753in}{2.037811in}}{\pgfqpoint{1.547653in}{2.034538in}}{\pgfqpoint{1.555889in}{2.034538in}}%
\pgfpathclose%
\pgfusepath{stroke,fill}%
\end{pgfscope}%
\begin{pgfscope}%
\pgfpathrectangle{\pgfqpoint{0.100000in}{0.212622in}}{\pgfqpoint{3.696000in}{3.696000in}}%
\pgfusepath{clip}%
\pgfsetbuttcap%
\pgfsetroundjoin%
\definecolor{currentfill}{rgb}{0.121569,0.466667,0.705882}%
\pgfsetfillcolor{currentfill}%
\pgfsetfillopacity{0.300610}%
\pgfsetlinewidth{1.003750pt}%
\definecolor{currentstroke}{rgb}{0.121569,0.466667,0.705882}%
\pgfsetstrokecolor{currentstroke}%
\pgfsetstrokeopacity{0.300610}%
\pgfsetdash{}{0pt}%
\pgfpathmoveto{\pgfqpoint{1.555889in}{2.034538in}}%
\pgfpathcurveto{\pgfqpoint{1.564126in}{2.034538in}}{\pgfqpoint{1.572026in}{2.037811in}}{\pgfqpoint{1.577850in}{2.043635in}}%
\pgfpathcurveto{\pgfqpoint{1.583674in}{2.049459in}}{\pgfqpoint{1.586946in}{2.057359in}}{\pgfqpoint{1.586946in}{2.065595in}}%
\pgfpathcurveto{\pgfqpoint{1.586946in}{2.073831in}}{\pgfqpoint{1.583674in}{2.081731in}}{\pgfqpoint{1.577850in}{2.087555in}}%
\pgfpathcurveto{\pgfqpoint{1.572026in}{2.093379in}}{\pgfqpoint{1.564126in}{2.096651in}}{\pgfqpoint{1.555889in}{2.096651in}}%
\pgfpathcurveto{\pgfqpoint{1.547653in}{2.096651in}}{\pgfqpoint{1.539753in}{2.093379in}}{\pgfqpoint{1.533929in}{2.087555in}}%
\pgfpathcurveto{\pgfqpoint{1.528105in}{2.081731in}}{\pgfqpoint{1.524833in}{2.073831in}}{\pgfqpoint{1.524833in}{2.065595in}}%
\pgfpathcurveto{\pgfqpoint{1.524833in}{2.057359in}}{\pgfqpoint{1.528105in}{2.049459in}}{\pgfqpoint{1.533929in}{2.043635in}}%
\pgfpathcurveto{\pgfqpoint{1.539753in}{2.037811in}}{\pgfqpoint{1.547653in}{2.034538in}}{\pgfqpoint{1.555889in}{2.034538in}}%
\pgfpathclose%
\pgfusepath{stroke,fill}%
\end{pgfscope}%
\begin{pgfscope}%
\pgfpathrectangle{\pgfqpoint{0.100000in}{0.212622in}}{\pgfqpoint{3.696000in}{3.696000in}}%
\pgfusepath{clip}%
\pgfsetbuttcap%
\pgfsetroundjoin%
\definecolor{currentfill}{rgb}{0.121569,0.466667,0.705882}%
\pgfsetfillcolor{currentfill}%
\pgfsetfillopacity{0.300610}%
\pgfsetlinewidth{1.003750pt}%
\definecolor{currentstroke}{rgb}{0.121569,0.466667,0.705882}%
\pgfsetstrokecolor{currentstroke}%
\pgfsetstrokeopacity{0.300610}%
\pgfsetdash{}{0pt}%
\pgfpathmoveto{\pgfqpoint{1.555889in}{2.034538in}}%
\pgfpathcurveto{\pgfqpoint{1.564126in}{2.034538in}}{\pgfqpoint{1.572026in}{2.037811in}}{\pgfqpoint{1.577850in}{2.043635in}}%
\pgfpathcurveto{\pgfqpoint{1.583674in}{2.049459in}}{\pgfqpoint{1.586946in}{2.057359in}}{\pgfqpoint{1.586946in}{2.065595in}}%
\pgfpathcurveto{\pgfqpoint{1.586946in}{2.073831in}}{\pgfqpoint{1.583674in}{2.081731in}}{\pgfqpoint{1.577850in}{2.087555in}}%
\pgfpathcurveto{\pgfqpoint{1.572026in}{2.093379in}}{\pgfqpoint{1.564126in}{2.096651in}}{\pgfqpoint{1.555889in}{2.096651in}}%
\pgfpathcurveto{\pgfqpoint{1.547653in}{2.096651in}}{\pgfqpoint{1.539753in}{2.093379in}}{\pgfqpoint{1.533929in}{2.087555in}}%
\pgfpathcurveto{\pgfqpoint{1.528105in}{2.081731in}}{\pgfqpoint{1.524833in}{2.073831in}}{\pgfqpoint{1.524833in}{2.065595in}}%
\pgfpathcurveto{\pgfqpoint{1.524833in}{2.057359in}}{\pgfqpoint{1.528105in}{2.049459in}}{\pgfqpoint{1.533929in}{2.043635in}}%
\pgfpathcurveto{\pgfqpoint{1.539753in}{2.037811in}}{\pgfqpoint{1.547653in}{2.034538in}}{\pgfqpoint{1.555889in}{2.034538in}}%
\pgfpathclose%
\pgfusepath{stroke,fill}%
\end{pgfscope}%
\begin{pgfscope}%
\pgfpathrectangle{\pgfqpoint{0.100000in}{0.212622in}}{\pgfqpoint{3.696000in}{3.696000in}}%
\pgfusepath{clip}%
\pgfsetbuttcap%
\pgfsetroundjoin%
\definecolor{currentfill}{rgb}{0.121569,0.466667,0.705882}%
\pgfsetfillcolor{currentfill}%
\pgfsetfillopacity{0.300610}%
\pgfsetlinewidth{1.003750pt}%
\definecolor{currentstroke}{rgb}{0.121569,0.466667,0.705882}%
\pgfsetstrokecolor{currentstroke}%
\pgfsetstrokeopacity{0.300610}%
\pgfsetdash{}{0pt}%
\pgfpathmoveto{\pgfqpoint{1.555889in}{2.034538in}}%
\pgfpathcurveto{\pgfqpoint{1.564126in}{2.034538in}}{\pgfqpoint{1.572026in}{2.037811in}}{\pgfqpoint{1.577850in}{2.043635in}}%
\pgfpathcurveto{\pgfqpoint{1.583674in}{2.049459in}}{\pgfqpoint{1.586946in}{2.057359in}}{\pgfqpoint{1.586946in}{2.065595in}}%
\pgfpathcurveto{\pgfqpoint{1.586946in}{2.073831in}}{\pgfqpoint{1.583674in}{2.081731in}}{\pgfqpoint{1.577850in}{2.087555in}}%
\pgfpathcurveto{\pgfqpoint{1.572026in}{2.093379in}}{\pgfqpoint{1.564126in}{2.096651in}}{\pgfqpoint{1.555889in}{2.096651in}}%
\pgfpathcurveto{\pgfqpoint{1.547653in}{2.096651in}}{\pgfqpoint{1.539753in}{2.093379in}}{\pgfqpoint{1.533929in}{2.087555in}}%
\pgfpathcurveto{\pgfqpoint{1.528105in}{2.081731in}}{\pgfqpoint{1.524833in}{2.073831in}}{\pgfqpoint{1.524833in}{2.065595in}}%
\pgfpathcurveto{\pgfqpoint{1.524833in}{2.057359in}}{\pgfqpoint{1.528105in}{2.049459in}}{\pgfqpoint{1.533929in}{2.043635in}}%
\pgfpathcurveto{\pgfqpoint{1.539753in}{2.037811in}}{\pgfqpoint{1.547653in}{2.034538in}}{\pgfqpoint{1.555889in}{2.034538in}}%
\pgfpathclose%
\pgfusepath{stroke,fill}%
\end{pgfscope}%
\begin{pgfscope}%
\pgfpathrectangle{\pgfqpoint{0.100000in}{0.212622in}}{\pgfqpoint{3.696000in}{3.696000in}}%
\pgfusepath{clip}%
\pgfsetbuttcap%
\pgfsetroundjoin%
\definecolor{currentfill}{rgb}{0.121569,0.466667,0.705882}%
\pgfsetfillcolor{currentfill}%
\pgfsetfillopacity{0.300610}%
\pgfsetlinewidth{1.003750pt}%
\definecolor{currentstroke}{rgb}{0.121569,0.466667,0.705882}%
\pgfsetstrokecolor{currentstroke}%
\pgfsetstrokeopacity{0.300610}%
\pgfsetdash{}{0pt}%
\pgfpathmoveto{\pgfqpoint{1.555889in}{2.034538in}}%
\pgfpathcurveto{\pgfqpoint{1.564126in}{2.034538in}}{\pgfqpoint{1.572026in}{2.037811in}}{\pgfqpoint{1.577850in}{2.043635in}}%
\pgfpathcurveto{\pgfqpoint{1.583674in}{2.049459in}}{\pgfqpoint{1.586946in}{2.057359in}}{\pgfqpoint{1.586946in}{2.065595in}}%
\pgfpathcurveto{\pgfqpoint{1.586946in}{2.073831in}}{\pgfqpoint{1.583674in}{2.081731in}}{\pgfqpoint{1.577850in}{2.087555in}}%
\pgfpathcurveto{\pgfqpoint{1.572026in}{2.093379in}}{\pgfqpoint{1.564126in}{2.096651in}}{\pgfqpoint{1.555889in}{2.096651in}}%
\pgfpathcurveto{\pgfqpoint{1.547653in}{2.096651in}}{\pgfqpoint{1.539753in}{2.093379in}}{\pgfqpoint{1.533929in}{2.087555in}}%
\pgfpathcurveto{\pgfqpoint{1.528105in}{2.081731in}}{\pgfqpoint{1.524833in}{2.073831in}}{\pgfqpoint{1.524833in}{2.065595in}}%
\pgfpathcurveto{\pgfqpoint{1.524833in}{2.057359in}}{\pgfqpoint{1.528105in}{2.049459in}}{\pgfqpoint{1.533929in}{2.043635in}}%
\pgfpathcurveto{\pgfqpoint{1.539753in}{2.037811in}}{\pgfqpoint{1.547653in}{2.034538in}}{\pgfqpoint{1.555889in}{2.034538in}}%
\pgfpathclose%
\pgfusepath{stroke,fill}%
\end{pgfscope}%
\begin{pgfscope}%
\pgfpathrectangle{\pgfqpoint{0.100000in}{0.212622in}}{\pgfqpoint{3.696000in}{3.696000in}}%
\pgfusepath{clip}%
\pgfsetbuttcap%
\pgfsetroundjoin%
\definecolor{currentfill}{rgb}{0.121569,0.466667,0.705882}%
\pgfsetfillcolor{currentfill}%
\pgfsetfillopacity{0.300610}%
\pgfsetlinewidth{1.003750pt}%
\definecolor{currentstroke}{rgb}{0.121569,0.466667,0.705882}%
\pgfsetstrokecolor{currentstroke}%
\pgfsetstrokeopacity{0.300610}%
\pgfsetdash{}{0pt}%
\pgfpathmoveto{\pgfqpoint{1.555889in}{2.034538in}}%
\pgfpathcurveto{\pgfqpoint{1.564126in}{2.034538in}}{\pgfqpoint{1.572026in}{2.037811in}}{\pgfqpoint{1.577850in}{2.043635in}}%
\pgfpathcurveto{\pgfqpoint{1.583674in}{2.049459in}}{\pgfqpoint{1.586946in}{2.057359in}}{\pgfqpoint{1.586946in}{2.065595in}}%
\pgfpathcurveto{\pgfqpoint{1.586946in}{2.073831in}}{\pgfqpoint{1.583674in}{2.081731in}}{\pgfqpoint{1.577850in}{2.087555in}}%
\pgfpathcurveto{\pgfqpoint{1.572026in}{2.093379in}}{\pgfqpoint{1.564126in}{2.096651in}}{\pgfqpoint{1.555889in}{2.096651in}}%
\pgfpathcurveto{\pgfqpoint{1.547653in}{2.096651in}}{\pgfqpoint{1.539753in}{2.093379in}}{\pgfqpoint{1.533929in}{2.087555in}}%
\pgfpathcurveto{\pgfqpoint{1.528105in}{2.081731in}}{\pgfqpoint{1.524833in}{2.073831in}}{\pgfqpoint{1.524833in}{2.065595in}}%
\pgfpathcurveto{\pgfqpoint{1.524833in}{2.057359in}}{\pgfqpoint{1.528105in}{2.049459in}}{\pgfqpoint{1.533929in}{2.043635in}}%
\pgfpathcurveto{\pgfqpoint{1.539753in}{2.037811in}}{\pgfqpoint{1.547653in}{2.034538in}}{\pgfqpoint{1.555889in}{2.034538in}}%
\pgfpathclose%
\pgfusepath{stroke,fill}%
\end{pgfscope}%
\begin{pgfscope}%
\pgfpathrectangle{\pgfqpoint{0.100000in}{0.212622in}}{\pgfqpoint{3.696000in}{3.696000in}}%
\pgfusepath{clip}%
\pgfsetbuttcap%
\pgfsetroundjoin%
\definecolor{currentfill}{rgb}{0.121569,0.466667,0.705882}%
\pgfsetfillcolor{currentfill}%
\pgfsetfillopacity{0.300610}%
\pgfsetlinewidth{1.003750pt}%
\definecolor{currentstroke}{rgb}{0.121569,0.466667,0.705882}%
\pgfsetstrokecolor{currentstroke}%
\pgfsetstrokeopacity{0.300610}%
\pgfsetdash{}{0pt}%
\pgfpathmoveto{\pgfqpoint{1.555889in}{2.034538in}}%
\pgfpathcurveto{\pgfqpoint{1.564126in}{2.034538in}}{\pgfqpoint{1.572026in}{2.037811in}}{\pgfqpoint{1.577850in}{2.043635in}}%
\pgfpathcurveto{\pgfqpoint{1.583674in}{2.049459in}}{\pgfqpoint{1.586946in}{2.057359in}}{\pgfqpoint{1.586946in}{2.065595in}}%
\pgfpathcurveto{\pgfqpoint{1.586946in}{2.073831in}}{\pgfqpoint{1.583674in}{2.081731in}}{\pgfqpoint{1.577850in}{2.087555in}}%
\pgfpathcurveto{\pgfqpoint{1.572026in}{2.093379in}}{\pgfqpoint{1.564126in}{2.096651in}}{\pgfqpoint{1.555889in}{2.096651in}}%
\pgfpathcurveto{\pgfqpoint{1.547653in}{2.096651in}}{\pgfqpoint{1.539753in}{2.093379in}}{\pgfqpoint{1.533929in}{2.087555in}}%
\pgfpathcurveto{\pgfqpoint{1.528105in}{2.081731in}}{\pgfqpoint{1.524833in}{2.073831in}}{\pgfqpoint{1.524833in}{2.065595in}}%
\pgfpathcurveto{\pgfqpoint{1.524833in}{2.057359in}}{\pgfqpoint{1.528105in}{2.049459in}}{\pgfqpoint{1.533929in}{2.043635in}}%
\pgfpathcurveto{\pgfqpoint{1.539753in}{2.037811in}}{\pgfqpoint{1.547653in}{2.034538in}}{\pgfqpoint{1.555889in}{2.034538in}}%
\pgfpathclose%
\pgfusepath{stroke,fill}%
\end{pgfscope}%
\begin{pgfscope}%
\pgfpathrectangle{\pgfqpoint{0.100000in}{0.212622in}}{\pgfqpoint{3.696000in}{3.696000in}}%
\pgfusepath{clip}%
\pgfsetbuttcap%
\pgfsetroundjoin%
\definecolor{currentfill}{rgb}{0.121569,0.466667,0.705882}%
\pgfsetfillcolor{currentfill}%
\pgfsetfillopacity{0.300610}%
\pgfsetlinewidth{1.003750pt}%
\definecolor{currentstroke}{rgb}{0.121569,0.466667,0.705882}%
\pgfsetstrokecolor{currentstroke}%
\pgfsetstrokeopacity{0.300610}%
\pgfsetdash{}{0pt}%
\pgfpathmoveto{\pgfqpoint{1.555889in}{2.034538in}}%
\pgfpathcurveto{\pgfqpoint{1.564126in}{2.034538in}}{\pgfqpoint{1.572026in}{2.037811in}}{\pgfqpoint{1.577850in}{2.043635in}}%
\pgfpathcurveto{\pgfqpoint{1.583674in}{2.049459in}}{\pgfqpoint{1.586946in}{2.057359in}}{\pgfqpoint{1.586946in}{2.065595in}}%
\pgfpathcurveto{\pgfqpoint{1.586946in}{2.073831in}}{\pgfqpoint{1.583674in}{2.081731in}}{\pgfqpoint{1.577850in}{2.087555in}}%
\pgfpathcurveto{\pgfqpoint{1.572026in}{2.093379in}}{\pgfqpoint{1.564126in}{2.096651in}}{\pgfqpoint{1.555889in}{2.096651in}}%
\pgfpathcurveto{\pgfqpoint{1.547653in}{2.096651in}}{\pgfqpoint{1.539753in}{2.093379in}}{\pgfqpoint{1.533929in}{2.087555in}}%
\pgfpathcurveto{\pgfqpoint{1.528105in}{2.081731in}}{\pgfqpoint{1.524833in}{2.073831in}}{\pgfqpoint{1.524833in}{2.065595in}}%
\pgfpathcurveto{\pgfqpoint{1.524833in}{2.057359in}}{\pgfqpoint{1.528105in}{2.049459in}}{\pgfqpoint{1.533929in}{2.043635in}}%
\pgfpathcurveto{\pgfqpoint{1.539753in}{2.037811in}}{\pgfqpoint{1.547653in}{2.034538in}}{\pgfqpoint{1.555889in}{2.034538in}}%
\pgfpathclose%
\pgfusepath{stroke,fill}%
\end{pgfscope}%
\begin{pgfscope}%
\pgfpathrectangle{\pgfqpoint{0.100000in}{0.212622in}}{\pgfqpoint{3.696000in}{3.696000in}}%
\pgfusepath{clip}%
\pgfsetbuttcap%
\pgfsetroundjoin%
\definecolor{currentfill}{rgb}{0.121569,0.466667,0.705882}%
\pgfsetfillcolor{currentfill}%
\pgfsetfillopacity{0.300610}%
\pgfsetlinewidth{1.003750pt}%
\definecolor{currentstroke}{rgb}{0.121569,0.466667,0.705882}%
\pgfsetstrokecolor{currentstroke}%
\pgfsetstrokeopacity{0.300610}%
\pgfsetdash{}{0pt}%
\pgfpathmoveto{\pgfqpoint{1.555889in}{2.034538in}}%
\pgfpathcurveto{\pgfqpoint{1.564126in}{2.034538in}}{\pgfqpoint{1.572026in}{2.037811in}}{\pgfqpoint{1.577850in}{2.043635in}}%
\pgfpathcurveto{\pgfqpoint{1.583674in}{2.049459in}}{\pgfqpoint{1.586946in}{2.057359in}}{\pgfqpoint{1.586946in}{2.065595in}}%
\pgfpathcurveto{\pgfqpoint{1.586946in}{2.073831in}}{\pgfqpoint{1.583674in}{2.081731in}}{\pgfqpoint{1.577850in}{2.087555in}}%
\pgfpathcurveto{\pgfqpoint{1.572026in}{2.093379in}}{\pgfqpoint{1.564126in}{2.096651in}}{\pgfqpoint{1.555889in}{2.096651in}}%
\pgfpathcurveto{\pgfqpoint{1.547653in}{2.096651in}}{\pgfqpoint{1.539753in}{2.093379in}}{\pgfqpoint{1.533929in}{2.087555in}}%
\pgfpathcurveto{\pgfqpoint{1.528105in}{2.081731in}}{\pgfqpoint{1.524833in}{2.073831in}}{\pgfqpoint{1.524833in}{2.065595in}}%
\pgfpathcurveto{\pgfqpoint{1.524833in}{2.057359in}}{\pgfqpoint{1.528105in}{2.049459in}}{\pgfqpoint{1.533929in}{2.043635in}}%
\pgfpathcurveto{\pgfqpoint{1.539753in}{2.037811in}}{\pgfqpoint{1.547653in}{2.034538in}}{\pgfqpoint{1.555889in}{2.034538in}}%
\pgfpathclose%
\pgfusepath{stroke,fill}%
\end{pgfscope}%
\begin{pgfscope}%
\pgfpathrectangle{\pgfqpoint{0.100000in}{0.212622in}}{\pgfqpoint{3.696000in}{3.696000in}}%
\pgfusepath{clip}%
\pgfsetbuttcap%
\pgfsetroundjoin%
\definecolor{currentfill}{rgb}{0.121569,0.466667,0.705882}%
\pgfsetfillcolor{currentfill}%
\pgfsetfillopacity{0.300610}%
\pgfsetlinewidth{1.003750pt}%
\definecolor{currentstroke}{rgb}{0.121569,0.466667,0.705882}%
\pgfsetstrokecolor{currentstroke}%
\pgfsetstrokeopacity{0.300610}%
\pgfsetdash{}{0pt}%
\pgfpathmoveto{\pgfqpoint{1.555889in}{2.034538in}}%
\pgfpathcurveto{\pgfqpoint{1.564126in}{2.034538in}}{\pgfqpoint{1.572026in}{2.037811in}}{\pgfqpoint{1.577850in}{2.043635in}}%
\pgfpathcurveto{\pgfqpoint{1.583674in}{2.049459in}}{\pgfqpoint{1.586946in}{2.057359in}}{\pgfqpoint{1.586946in}{2.065595in}}%
\pgfpathcurveto{\pgfqpoint{1.586946in}{2.073831in}}{\pgfqpoint{1.583674in}{2.081731in}}{\pgfqpoint{1.577850in}{2.087555in}}%
\pgfpathcurveto{\pgfqpoint{1.572026in}{2.093379in}}{\pgfqpoint{1.564126in}{2.096651in}}{\pgfqpoint{1.555889in}{2.096651in}}%
\pgfpathcurveto{\pgfqpoint{1.547653in}{2.096651in}}{\pgfqpoint{1.539753in}{2.093379in}}{\pgfqpoint{1.533929in}{2.087555in}}%
\pgfpathcurveto{\pgfqpoint{1.528105in}{2.081731in}}{\pgfqpoint{1.524833in}{2.073831in}}{\pgfqpoint{1.524833in}{2.065595in}}%
\pgfpathcurveto{\pgfqpoint{1.524833in}{2.057359in}}{\pgfqpoint{1.528105in}{2.049459in}}{\pgfqpoint{1.533929in}{2.043635in}}%
\pgfpathcurveto{\pgfqpoint{1.539753in}{2.037811in}}{\pgfqpoint{1.547653in}{2.034538in}}{\pgfqpoint{1.555889in}{2.034538in}}%
\pgfpathclose%
\pgfusepath{stroke,fill}%
\end{pgfscope}%
\begin{pgfscope}%
\pgfpathrectangle{\pgfqpoint{0.100000in}{0.212622in}}{\pgfqpoint{3.696000in}{3.696000in}}%
\pgfusepath{clip}%
\pgfsetbuttcap%
\pgfsetroundjoin%
\definecolor{currentfill}{rgb}{0.121569,0.466667,0.705882}%
\pgfsetfillcolor{currentfill}%
\pgfsetfillopacity{0.300610}%
\pgfsetlinewidth{1.003750pt}%
\definecolor{currentstroke}{rgb}{0.121569,0.466667,0.705882}%
\pgfsetstrokecolor{currentstroke}%
\pgfsetstrokeopacity{0.300610}%
\pgfsetdash{}{0pt}%
\pgfpathmoveto{\pgfqpoint{1.555889in}{2.034538in}}%
\pgfpathcurveto{\pgfqpoint{1.564126in}{2.034538in}}{\pgfqpoint{1.572026in}{2.037811in}}{\pgfqpoint{1.577850in}{2.043635in}}%
\pgfpathcurveto{\pgfqpoint{1.583674in}{2.049459in}}{\pgfqpoint{1.586946in}{2.057359in}}{\pgfqpoint{1.586946in}{2.065595in}}%
\pgfpathcurveto{\pgfqpoint{1.586946in}{2.073831in}}{\pgfqpoint{1.583674in}{2.081731in}}{\pgfqpoint{1.577850in}{2.087555in}}%
\pgfpathcurveto{\pgfqpoint{1.572026in}{2.093379in}}{\pgfqpoint{1.564126in}{2.096651in}}{\pgfqpoint{1.555889in}{2.096651in}}%
\pgfpathcurveto{\pgfqpoint{1.547653in}{2.096651in}}{\pgfqpoint{1.539753in}{2.093379in}}{\pgfqpoint{1.533929in}{2.087555in}}%
\pgfpathcurveto{\pgfqpoint{1.528105in}{2.081731in}}{\pgfqpoint{1.524833in}{2.073831in}}{\pgfqpoint{1.524833in}{2.065595in}}%
\pgfpathcurveto{\pgfqpoint{1.524833in}{2.057359in}}{\pgfqpoint{1.528105in}{2.049459in}}{\pgfqpoint{1.533929in}{2.043635in}}%
\pgfpathcurveto{\pgfqpoint{1.539753in}{2.037811in}}{\pgfqpoint{1.547653in}{2.034538in}}{\pgfqpoint{1.555889in}{2.034538in}}%
\pgfpathclose%
\pgfusepath{stroke,fill}%
\end{pgfscope}%
\begin{pgfscope}%
\pgfpathrectangle{\pgfqpoint{0.100000in}{0.212622in}}{\pgfqpoint{3.696000in}{3.696000in}}%
\pgfusepath{clip}%
\pgfsetbuttcap%
\pgfsetroundjoin%
\definecolor{currentfill}{rgb}{0.121569,0.466667,0.705882}%
\pgfsetfillcolor{currentfill}%
\pgfsetfillopacity{0.300610}%
\pgfsetlinewidth{1.003750pt}%
\definecolor{currentstroke}{rgb}{0.121569,0.466667,0.705882}%
\pgfsetstrokecolor{currentstroke}%
\pgfsetstrokeopacity{0.300610}%
\pgfsetdash{}{0pt}%
\pgfpathmoveto{\pgfqpoint{1.555889in}{2.034538in}}%
\pgfpathcurveto{\pgfqpoint{1.564126in}{2.034538in}}{\pgfqpoint{1.572026in}{2.037811in}}{\pgfqpoint{1.577850in}{2.043635in}}%
\pgfpathcurveto{\pgfqpoint{1.583674in}{2.049459in}}{\pgfqpoint{1.586946in}{2.057359in}}{\pgfqpoint{1.586946in}{2.065595in}}%
\pgfpathcurveto{\pgfqpoint{1.586946in}{2.073831in}}{\pgfqpoint{1.583674in}{2.081731in}}{\pgfqpoint{1.577850in}{2.087555in}}%
\pgfpathcurveto{\pgfqpoint{1.572026in}{2.093379in}}{\pgfqpoint{1.564126in}{2.096651in}}{\pgfqpoint{1.555889in}{2.096651in}}%
\pgfpathcurveto{\pgfqpoint{1.547653in}{2.096651in}}{\pgfqpoint{1.539753in}{2.093379in}}{\pgfqpoint{1.533929in}{2.087555in}}%
\pgfpathcurveto{\pgfqpoint{1.528105in}{2.081731in}}{\pgfqpoint{1.524833in}{2.073831in}}{\pgfqpoint{1.524833in}{2.065595in}}%
\pgfpathcurveto{\pgfqpoint{1.524833in}{2.057359in}}{\pgfqpoint{1.528105in}{2.049459in}}{\pgfqpoint{1.533929in}{2.043635in}}%
\pgfpathcurveto{\pgfqpoint{1.539753in}{2.037811in}}{\pgfqpoint{1.547653in}{2.034538in}}{\pgfqpoint{1.555889in}{2.034538in}}%
\pgfpathclose%
\pgfusepath{stroke,fill}%
\end{pgfscope}%
\begin{pgfscope}%
\pgfpathrectangle{\pgfqpoint{0.100000in}{0.212622in}}{\pgfqpoint{3.696000in}{3.696000in}}%
\pgfusepath{clip}%
\pgfsetbuttcap%
\pgfsetroundjoin%
\definecolor{currentfill}{rgb}{0.121569,0.466667,0.705882}%
\pgfsetfillcolor{currentfill}%
\pgfsetfillopacity{0.300610}%
\pgfsetlinewidth{1.003750pt}%
\definecolor{currentstroke}{rgb}{0.121569,0.466667,0.705882}%
\pgfsetstrokecolor{currentstroke}%
\pgfsetstrokeopacity{0.300610}%
\pgfsetdash{}{0pt}%
\pgfpathmoveto{\pgfqpoint{1.555889in}{2.034538in}}%
\pgfpathcurveto{\pgfqpoint{1.564126in}{2.034538in}}{\pgfqpoint{1.572026in}{2.037811in}}{\pgfqpoint{1.577850in}{2.043635in}}%
\pgfpathcurveto{\pgfqpoint{1.583674in}{2.049459in}}{\pgfqpoint{1.586946in}{2.057359in}}{\pgfqpoint{1.586946in}{2.065595in}}%
\pgfpathcurveto{\pgfqpoint{1.586946in}{2.073831in}}{\pgfqpoint{1.583674in}{2.081731in}}{\pgfqpoint{1.577850in}{2.087555in}}%
\pgfpathcurveto{\pgfqpoint{1.572026in}{2.093379in}}{\pgfqpoint{1.564126in}{2.096651in}}{\pgfqpoint{1.555889in}{2.096651in}}%
\pgfpathcurveto{\pgfqpoint{1.547653in}{2.096651in}}{\pgfqpoint{1.539753in}{2.093379in}}{\pgfqpoint{1.533929in}{2.087555in}}%
\pgfpathcurveto{\pgfqpoint{1.528105in}{2.081731in}}{\pgfqpoint{1.524833in}{2.073831in}}{\pgfqpoint{1.524833in}{2.065595in}}%
\pgfpathcurveto{\pgfqpoint{1.524833in}{2.057359in}}{\pgfqpoint{1.528105in}{2.049459in}}{\pgfqpoint{1.533929in}{2.043635in}}%
\pgfpathcurveto{\pgfqpoint{1.539753in}{2.037811in}}{\pgfqpoint{1.547653in}{2.034538in}}{\pgfqpoint{1.555889in}{2.034538in}}%
\pgfpathclose%
\pgfusepath{stroke,fill}%
\end{pgfscope}%
\begin{pgfscope}%
\pgfpathrectangle{\pgfqpoint{0.100000in}{0.212622in}}{\pgfqpoint{3.696000in}{3.696000in}}%
\pgfusepath{clip}%
\pgfsetbuttcap%
\pgfsetroundjoin%
\definecolor{currentfill}{rgb}{0.121569,0.466667,0.705882}%
\pgfsetfillcolor{currentfill}%
\pgfsetfillopacity{0.300610}%
\pgfsetlinewidth{1.003750pt}%
\definecolor{currentstroke}{rgb}{0.121569,0.466667,0.705882}%
\pgfsetstrokecolor{currentstroke}%
\pgfsetstrokeopacity{0.300610}%
\pgfsetdash{}{0pt}%
\pgfpathmoveto{\pgfqpoint{1.555889in}{2.034538in}}%
\pgfpathcurveto{\pgfqpoint{1.564126in}{2.034538in}}{\pgfqpoint{1.572026in}{2.037811in}}{\pgfqpoint{1.577850in}{2.043635in}}%
\pgfpathcurveto{\pgfqpoint{1.583674in}{2.049459in}}{\pgfqpoint{1.586946in}{2.057359in}}{\pgfqpoint{1.586946in}{2.065595in}}%
\pgfpathcurveto{\pgfqpoint{1.586946in}{2.073831in}}{\pgfqpoint{1.583674in}{2.081731in}}{\pgfqpoint{1.577850in}{2.087555in}}%
\pgfpathcurveto{\pgfqpoint{1.572026in}{2.093379in}}{\pgfqpoint{1.564126in}{2.096651in}}{\pgfqpoint{1.555889in}{2.096651in}}%
\pgfpathcurveto{\pgfqpoint{1.547653in}{2.096651in}}{\pgfqpoint{1.539753in}{2.093379in}}{\pgfqpoint{1.533929in}{2.087555in}}%
\pgfpathcurveto{\pgfqpoint{1.528105in}{2.081731in}}{\pgfqpoint{1.524833in}{2.073831in}}{\pgfqpoint{1.524833in}{2.065595in}}%
\pgfpathcurveto{\pgfqpoint{1.524833in}{2.057359in}}{\pgfqpoint{1.528105in}{2.049459in}}{\pgfqpoint{1.533929in}{2.043635in}}%
\pgfpathcurveto{\pgfqpoint{1.539753in}{2.037811in}}{\pgfqpoint{1.547653in}{2.034538in}}{\pgfqpoint{1.555889in}{2.034538in}}%
\pgfpathclose%
\pgfusepath{stroke,fill}%
\end{pgfscope}%
\begin{pgfscope}%
\pgfpathrectangle{\pgfqpoint{0.100000in}{0.212622in}}{\pgfqpoint{3.696000in}{3.696000in}}%
\pgfusepath{clip}%
\pgfsetbuttcap%
\pgfsetroundjoin%
\definecolor{currentfill}{rgb}{0.121569,0.466667,0.705882}%
\pgfsetfillcolor{currentfill}%
\pgfsetfillopacity{0.300610}%
\pgfsetlinewidth{1.003750pt}%
\definecolor{currentstroke}{rgb}{0.121569,0.466667,0.705882}%
\pgfsetstrokecolor{currentstroke}%
\pgfsetstrokeopacity{0.300610}%
\pgfsetdash{}{0pt}%
\pgfpathmoveto{\pgfqpoint{1.555889in}{2.034538in}}%
\pgfpathcurveto{\pgfqpoint{1.564126in}{2.034538in}}{\pgfqpoint{1.572026in}{2.037811in}}{\pgfqpoint{1.577850in}{2.043635in}}%
\pgfpathcurveto{\pgfqpoint{1.583674in}{2.049459in}}{\pgfqpoint{1.586946in}{2.057359in}}{\pgfqpoint{1.586946in}{2.065595in}}%
\pgfpathcurveto{\pgfqpoint{1.586946in}{2.073831in}}{\pgfqpoint{1.583674in}{2.081731in}}{\pgfqpoint{1.577850in}{2.087555in}}%
\pgfpathcurveto{\pgfqpoint{1.572026in}{2.093379in}}{\pgfqpoint{1.564126in}{2.096651in}}{\pgfqpoint{1.555889in}{2.096651in}}%
\pgfpathcurveto{\pgfqpoint{1.547653in}{2.096651in}}{\pgfqpoint{1.539753in}{2.093379in}}{\pgfqpoint{1.533929in}{2.087555in}}%
\pgfpathcurveto{\pgfqpoint{1.528105in}{2.081731in}}{\pgfqpoint{1.524833in}{2.073831in}}{\pgfqpoint{1.524833in}{2.065595in}}%
\pgfpathcurveto{\pgfqpoint{1.524833in}{2.057359in}}{\pgfqpoint{1.528105in}{2.049459in}}{\pgfqpoint{1.533929in}{2.043635in}}%
\pgfpathcurveto{\pgfqpoint{1.539753in}{2.037811in}}{\pgfqpoint{1.547653in}{2.034538in}}{\pgfqpoint{1.555889in}{2.034538in}}%
\pgfpathclose%
\pgfusepath{stroke,fill}%
\end{pgfscope}%
\begin{pgfscope}%
\pgfpathrectangle{\pgfqpoint{0.100000in}{0.212622in}}{\pgfqpoint{3.696000in}{3.696000in}}%
\pgfusepath{clip}%
\pgfsetbuttcap%
\pgfsetroundjoin%
\definecolor{currentfill}{rgb}{0.121569,0.466667,0.705882}%
\pgfsetfillcolor{currentfill}%
\pgfsetfillopacity{0.300610}%
\pgfsetlinewidth{1.003750pt}%
\definecolor{currentstroke}{rgb}{0.121569,0.466667,0.705882}%
\pgfsetstrokecolor{currentstroke}%
\pgfsetstrokeopacity{0.300610}%
\pgfsetdash{}{0pt}%
\pgfpathmoveto{\pgfqpoint{1.555889in}{2.034538in}}%
\pgfpathcurveto{\pgfqpoint{1.564126in}{2.034538in}}{\pgfqpoint{1.572026in}{2.037811in}}{\pgfqpoint{1.577850in}{2.043635in}}%
\pgfpathcurveto{\pgfqpoint{1.583674in}{2.049459in}}{\pgfqpoint{1.586946in}{2.057359in}}{\pgfqpoint{1.586946in}{2.065595in}}%
\pgfpathcurveto{\pgfqpoint{1.586946in}{2.073831in}}{\pgfqpoint{1.583674in}{2.081731in}}{\pgfqpoint{1.577850in}{2.087555in}}%
\pgfpathcurveto{\pgfqpoint{1.572026in}{2.093379in}}{\pgfqpoint{1.564126in}{2.096651in}}{\pgfqpoint{1.555889in}{2.096651in}}%
\pgfpathcurveto{\pgfqpoint{1.547653in}{2.096651in}}{\pgfqpoint{1.539753in}{2.093379in}}{\pgfqpoint{1.533929in}{2.087555in}}%
\pgfpathcurveto{\pgfqpoint{1.528105in}{2.081731in}}{\pgfqpoint{1.524833in}{2.073831in}}{\pgfqpoint{1.524833in}{2.065595in}}%
\pgfpathcurveto{\pgfqpoint{1.524833in}{2.057359in}}{\pgfqpoint{1.528105in}{2.049459in}}{\pgfqpoint{1.533929in}{2.043635in}}%
\pgfpathcurveto{\pgfqpoint{1.539753in}{2.037811in}}{\pgfqpoint{1.547653in}{2.034538in}}{\pgfqpoint{1.555889in}{2.034538in}}%
\pgfpathclose%
\pgfusepath{stroke,fill}%
\end{pgfscope}%
\begin{pgfscope}%
\pgfpathrectangle{\pgfqpoint{0.100000in}{0.212622in}}{\pgfqpoint{3.696000in}{3.696000in}}%
\pgfusepath{clip}%
\pgfsetbuttcap%
\pgfsetroundjoin%
\definecolor{currentfill}{rgb}{0.121569,0.466667,0.705882}%
\pgfsetfillcolor{currentfill}%
\pgfsetfillopacity{0.300610}%
\pgfsetlinewidth{1.003750pt}%
\definecolor{currentstroke}{rgb}{0.121569,0.466667,0.705882}%
\pgfsetstrokecolor{currentstroke}%
\pgfsetstrokeopacity{0.300610}%
\pgfsetdash{}{0pt}%
\pgfpathmoveto{\pgfqpoint{1.555889in}{2.034538in}}%
\pgfpathcurveto{\pgfqpoint{1.564126in}{2.034538in}}{\pgfqpoint{1.572026in}{2.037811in}}{\pgfqpoint{1.577850in}{2.043635in}}%
\pgfpathcurveto{\pgfqpoint{1.583674in}{2.049459in}}{\pgfqpoint{1.586946in}{2.057359in}}{\pgfqpoint{1.586946in}{2.065595in}}%
\pgfpathcurveto{\pgfqpoint{1.586946in}{2.073831in}}{\pgfqpoint{1.583674in}{2.081731in}}{\pgfqpoint{1.577850in}{2.087555in}}%
\pgfpathcurveto{\pgfqpoint{1.572026in}{2.093379in}}{\pgfqpoint{1.564126in}{2.096651in}}{\pgfqpoint{1.555889in}{2.096651in}}%
\pgfpathcurveto{\pgfqpoint{1.547653in}{2.096651in}}{\pgfqpoint{1.539753in}{2.093379in}}{\pgfqpoint{1.533929in}{2.087555in}}%
\pgfpathcurveto{\pgfqpoint{1.528105in}{2.081731in}}{\pgfqpoint{1.524833in}{2.073831in}}{\pgfqpoint{1.524833in}{2.065595in}}%
\pgfpathcurveto{\pgfqpoint{1.524833in}{2.057359in}}{\pgfqpoint{1.528105in}{2.049459in}}{\pgfqpoint{1.533929in}{2.043635in}}%
\pgfpathcurveto{\pgfqpoint{1.539753in}{2.037811in}}{\pgfqpoint{1.547653in}{2.034538in}}{\pgfqpoint{1.555889in}{2.034538in}}%
\pgfpathclose%
\pgfusepath{stroke,fill}%
\end{pgfscope}%
\begin{pgfscope}%
\pgfpathrectangle{\pgfqpoint{0.100000in}{0.212622in}}{\pgfqpoint{3.696000in}{3.696000in}}%
\pgfusepath{clip}%
\pgfsetbuttcap%
\pgfsetroundjoin%
\definecolor{currentfill}{rgb}{0.121569,0.466667,0.705882}%
\pgfsetfillcolor{currentfill}%
\pgfsetfillopacity{0.300610}%
\pgfsetlinewidth{1.003750pt}%
\definecolor{currentstroke}{rgb}{0.121569,0.466667,0.705882}%
\pgfsetstrokecolor{currentstroke}%
\pgfsetstrokeopacity{0.300610}%
\pgfsetdash{}{0pt}%
\pgfpathmoveto{\pgfqpoint{1.555889in}{2.034538in}}%
\pgfpathcurveto{\pgfqpoint{1.564126in}{2.034538in}}{\pgfqpoint{1.572026in}{2.037811in}}{\pgfqpoint{1.577850in}{2.043635in}}%
\pgfpathcurveto{\pgfqpoint{1.583674in}{2.049459in}}{\pgfqpoint{1.586946in}{2.057359in}}{\pgfqpoint{1.586946in}{2.065595in}}%
\pgfpathcurveto{\pgfqpoint{1.586946in}{2.073831in}}{\pgfqpoint{1.583674in}{2.081731in}}{\pgfqpoint{1.577850in}{2.087555in}}%
\pgfpathcurveto{\pgfqpoint{1.572026in}{2.093379in}}{\pgfqpoint{1.564126in}{2.096651in}}{\pgfqpoint{1.555889in}{2.096651in}}%
\pgfpathcurveto{\pgfqpoint{1.547653in}{2.096651in}}{\pgfqpoint{1.539753in}{2.093379in}}{\pgfqpoint{1.533929in}{2.087555in}}%
\pgfpathcurveto{\pgfqpoint{1.528105in}{2.081731in}}{\pgfqpoint{1.524833in}{2.073831in}}{\pgfqpoint{1.524833in}{2.065595in}}%
\pgfpathcurveto{\pgfqpoint{1.524833in}{2.057359in}}{\pgfqpoint{1.528105in}{2.049459in}}{\pgfqpoint{1.533929in}{2.043635in}}%
\pgfpathcurveto{\pgfqpoint{1.539753in}{2.037811in}}{\pgfqpoint{1.547653in}{2.034538in}}{\pgfqpoint{1.555889in}{2.034538in}}%
\pgfpathclose%
\pgfusepath{stroke,fill}%
\end{pgfscope}%
\begin{pgfscope}%
\pgfpathrectangle{\pgfqpoint{0.100000in}{0.212622in}}{\pgfqpoint{3.696000in}{3.696000in}}%
\pgfusepath{clip}%
\pgfsetbuttcap%
\pgfsetroundjoin%
\definecolor{currentfill}{rgb}{0.121569,0.466667,0.705882}%
\pgfsetfillcolor{currentfill}%
\pgfsetfillopacity{0.300610}%
\pgfsetlinewidth{1.003750pt}%
\definecolor{currentstroke}{rgb}{0.121569,0.466667,0.705882}%
\pgfsetstrokecolor{currentstroke}%
\pgfsetstrokeopacity{0.300610}%
\pgfsetdash{}{0pt}%
\pgfpathmoveto{\pgfqpoint{1.555889in}{2.034538in}}%
\pgfpathcurveto{\pgfqpoint{1.564126in}{2.034538in}}{\pgfqpoint{1.572026in}{2.037811in}}{\pgfqpoint{1.577850in}{2.043635in}}%
\pgfpathcurveto{\pgfqpoint{1.583674in}{2.049459in}}{\pgfqpoint{1.586946in}{2.057359in}}{\pgfqpoint{1.586946in}{2.065595in}}%
\pgfpathcurveto{\pgfqpoint{1.586946in}{2.073831in}}{\pgfqpoint{1.583674in}{2.081731in}}{\pgfqpoint{1.577850in}{2.087555in}}%
\pgfpathcurveto{\pgfqpoint{1.572026in}{2.093379in}}{\pgfqpoint{1.564126in}{2.096651in}}{\pgfqpoint{1.555889in}{2.096651in}}%
\pgfpathcurveto{\pgfqpoint{1.547653in}{2.096651in}}{\pgfqpoint{1.539753in}{2.093379in}}{\pgfqpoint{1.533929in}{2.087555in}}%
\pgfpathcurveto{\pgfqpoint{1.528105in}{2.081731in}}{\pgfqpoint{1.524833in}{2.073831in}}{\pgfqpoint{1.524833in}{2.065595in}}%
\pgfpathcurveto{\pgfqpoint{1.524833in}{2.057359in}}{\pgfqpoint{1.528105in}{2.049459in}}{\pgfqpoint{1.533929in}{2.043635in}}%
\pgfpathcurveto{\pgfqpoint{1.539753in}{2.037811in}}{\pgfqpoint{1.547653in}{2.034538in}}{\pgfqpoint{1.555889in}{2.034538in}}%
\pgfpathclose%
\pgfusepath{stroke,fill}%
\end{pgfscope}%
\begin{pgfscope}%
\pgfpathrectangle{\pgfqpoint{0.100000in}{0.212622in}}{\pgfqpoint{3.696000in}{3.696000in}}%
\pgfusepath{clip}%
\pgfsetbuttcap%
\pgfsetroundjoin%
\definecolor{currentfill}{rgb}{0.121569,0.466667,0.705882}%
\pgfsetfillcolor{currentfill}%
\pgfsetfillopacity{0.300610}%
\pgfsetlinewidth{1.003750pt}%
\definecolor{currentstroke}{rgb}{0.121569,0.466667,0.705882}%
\pgfsetstrokecolor{currentstroke}%
\pgfsetstrokeopacity{0.300610}%
\pgfsetdash{}{0pt}%
\pgfpathmoveto{\pgfqpoint{1.555889in}{2.034538in}}%
\pgfpathcurveto{\pgfqpoint{1.564126in}{2.034538in}}{\pgfqpoint{1.572026in}{2.037811in}}{\pgfqpoint{1.577850in}{2.043635in}}%
\pgfpathcurveto{\pgfqpoint{1.583674in}{2.049459in}}{\pgfqpoint{1.586946in}{2.057359in}}{\pgfqpoint{1.586946in}{2.065595in}}%
\pgfpathcurveto{\pgfqpoint{1.586946in}{2.073831in}}{\pgfqpoint{1.583674in}{2.081731in}}{\pgfqpoint{1.577850in}{2.087555in}}%
\pgfpathcurveto{\pgfqpoint{1.572026in}{2.093379in}}{\pgfqpoint{1.564126in}{2.096651in}}{\pgfqpoint{1.555889in}{2.096651in}}%
\pgfpathcurveto{\pgfqpoint{1.547653in}{2.096651in}}{\pgfqpoint{1.539753in}{2.093379in}}{\pgfqpoint{1.533929in}{2.087555in}}%
\pgfpathcurveto{\pgfqpoint{1.528105in}{2.081731in}}{\pgfqpoint{1.524833in}{2.073831in}}{\pgfqpoint{1.524833in}{2.065595in}}%
\pgfpathcurveto{\pgfqpoint{1.524833in}{2.057359in}}{\pgfqpoint{1.528105in}{2.049459in}}{\pgfqpoint{1.533929in}{2.043635in}}%
\pgfpathcurveto{\pgfqpoint{1.539753in}{2.037811in}}{\pgfqpoint{1.547653in}{2.034538in}}{\pgfqpoint{1.555889in}{2.034538in}}%
\pgfpathclose%
\pgfusepath{stroke,fill}%
\end{pgfscope}%
\begin{pgfscope}%
\pgfpathrectangle{\pgfqpoint{0.100000in}{0.212622in}}{\pgfqpoint{3.696000in}{3.696000in}}%
\pgfusepath{clip}%
\pgfsetbuttcap%
\pgfsetroundjoin%
\definecolor{currentfill}{rgb}{0.121569,0.466667,0.705882}%
\pgfsetfillcolor{currentfill}%
\pgfsetfillopacity{0.300610}%
\pgfsetlinewidth{1.003750pt}%
\definecolor{currentstroke}{rgb}{0.121569,0.466667,0.705882}%
\pgfsetstrokecolor{currentstroke}%
\pgfsetstrokeopacity{0.300610}%
\pgfsetdash{}{0pt}%
\pgfpathmoveto{\pgfqpoint{1.555889in}{2.034538in}}%
\pgfpathcurveto{\pgfqpoint{1.564126in}{2.034538in}}{\pgfqpoint{1.572026in}{2.037811in}}{\pgfqpoint{1.577850in}{2.043635in}}%
\pgfpathcurveto{\pgfqpoint{1.583674in}{2.049459in}}{\pgfqpoint{1.586946in}{2.057359in}}{\pgfqpoint{1.586946in}{2.065595in}}%
\pgfpathcurveto{\pgfqpoint{1.586946in}{2.073831in}}{\pgfqpoint{1.583674in}{2.081731in}}{\pgfqpoint{1.577850in}{2.087555in}}%
\pgfpathcurveto{\pgfqpoint{1.572026in}{2.093379in}}{\pgfqpoint{1.564126in}{2.096651in}}{\pgfqpoint{1.555889in}{2.096651in}}%
\pgfpathcurveto{\pgfqpoint{1.547653in}{2.096651in}}{\pgfqpoint{1.539753in}{2.093379in}}{\pgfqpoint{1.533929in}{2.087555in}}%
\pgfpathcurveto{\pgfqpoint{1.528105in}{2.081731in}}{\pgfqpoint{1.524833in}{2.073831in}}{\pgfqpoint{1.524833in}{2.065595in}}%
\pgfpathcurveto{\pgfqpoint{1.524833in}{2.057359in}}{\pgfqpoint{1.528105in}{2.049459in}}{\pgfqpoint{1.533929in}{2.043635in}}%
\pgfpathcurveto{\pgfqpoint{1.539753in}{2.037811in}}{\pgfqpoint{1.547653in}{2.034538in}}{\pgfqpoint{1.555889in}{2.034538in}}%
\pgfpathclose%
\pgfusepath{stroke,fill}%
\end{pgfscope}%
\begin{pgfscope}%
\pgfpathrectangle{\pgfqpoint{0.100000in}{0.212622in}}{\pgfqpoint{3.696000in}{3.696000in}}%
\pgfusepath{clip}%
\pgfsetbuttcap%
\pgfsetroundjoin%
\definecolor{currentfill}{rgb}{0.121569,0.466667,0.705882}%
\pgfsetfillcolor{currentfill}%
\pgfsetfillopacity{0.300610}%
\pgfsetlinewidth{1.003750pt}%
\definecolor{currentstroke}{rgb}{0.121569,0.466667,0.705882}%
\pgfsetstrokecolor{currentstroke}%
\pgfsetstrokeopacity{0.300610}%
\pgfsetdash{}{0pt}%
\pgfpathmoveto{\pgfqpoint{1.555889in}{2.034538in}}%
\pgfpathcurveto{\pgfqpoint{1.564126in}{2.034538in}}{\pgfqpoint{1.572026in}{2.037811in}}{\pgfqpoint{1.577850in}{2.043635in}}%
\pgfpathcurveto{\pgfqpoint{1.583674in}{2.049459in}}{\pgfqpoint{1.586946in}{2.057359in}}{\pgfqpoint{1.586946in}{2.065595in}}%
\pgfpathcurveto{\pgfqpoint{1.586946in}{2.073831in}}{\pgfqpoint{1.583674in}{2.081731in}}{\pgfqpoint{1.577850in}{2.087555in}}%
\pgfpathcurveto{\pgfqpoint{1.572026in}{2.093379in}}{\pgfqpoint{1.564126in}{2.096651in}}{\pgfqpoint{1.555889in}{2.096651in}}%
\pgfpathcurveto{\pgfqpoint{1.547653in}{2.096651in}}{\pgfqpoint{1.539753in}{2.093379in}}{\pgfqpoint{1.533929in}{2.087555in}}%
\pgfpathcurveto{\pgfqpoint{1.528105in}{2.081731in}}{\pgfqpoint{1.524833in}{2.073831in}}{\pgfqpoint{1.524833in}{2.065595in}}%
\pgfpathcurveto{\pgfqpoint{1.524833in}{2.057359in}}{\pgfqpoint{1.528105in}{2.049459in}}{\pgfqpoint{1.533929in}{2.043635in}}%
\pgfpathcurveto{\pgfqpoint{1.539753in}{2.037811in}}{\pgfqpoint{1.547653in}{2.034538in}}{\pgfqpoint{1.555889in}{2.034538in}}%
\pgfpathclose%
\pgfusepath{stroke,fill}%
\end{pgfscope}%
\begin{pgfscope}%
\pgfpathrectangle{\pgfqpoint{0.100000in}{0.212622in}}{\pgfqpoint{3.696000in}{3.696000in}}%
\pgfusepath{clip}%
\pgfsetbuttcap%
\pgfsetroundjoin%
\definecolor{currentfill}{rgb}{0.121569,0.466667,0.705882}%
\pgfsetfillcolor{currentfill}%
\pgfsetfillopacity{0.300610}%
\pgfsetlinewidth{1.003750pt}%
\definecolor{currentstroke}{rgb}{0.121569,0.466667,0.705882}%
\pgfsetstrokecolor{currentstroke}%
\pgfsetstrokeopacity{0.300610}%
\pgfsetdash{}{0pt}%
\pgfpathmoveto{\pgfqpoint{1.555889in}{2.034538in}}%
\pgfpathcurveto{\pgfqpoint{1.564126in}{2.034538in}}{\pgfqpoint{1.572026in}{2.037811in}}{\pgfqpoint{1.577850in}{2.043635in}}%
\pgfpathcurveto{\pgfqpoint{1.583674in}{2.049459in}}{\pgfqpoint{1.586946in}{2.057359in}}{\pgfqpoint{1.586946in}{2.065595in}}%
\pgfpathcurveto{\pgfqpoint{1.586946in}{2.073831in}}{\pgfqpoint{1.583674in}{2.081731in}}{\pgfqpoint{1.577850in}{2.087555in}}%
\pgfpathcurveto{\pgfqpoint{1.572026in}{2.093379in}}{\pgfqpoint{1.564126in}{2.096651in}}{\pgfqpoint{1.555889in}{2.096651in}}%
\pgfpathcurveto{\pgfqpoint{1.547653in}{2.096651in}}{\pgfqpoint{1.539753in}{2.093379in}}{\pgfqpoint{1.533929in}{2.087555in}}%
\pgfpathcurveto{\pgfqpoint{1.528105in}{2.081731in}}{\pgfqpoint{1.524833in}{2.073831in}}{\pgfqpoint{1.524833in}{2.065595in}}%
\pgfpathcurveto{\pgfqpoint{1.524833in}{2.057359in}}{\pgfqpoint{1.528105in}{2.049459in}}{\pgfqpoint{1.533929in}{2.043635in}}%
\pgfpathcurveto{\pgfqpoint{1.539753in}{2.037811in}}{\pgfqpoint{1.547653in}{2.034538in}}{\pgfqpoint{1.555889in}{2.034538in}}%
\pgfpathclose%
\pgfusepath{stroke,fill}%
\end{pgfscope}%
\begin{pgfscope}%
\pgfpathrectangle{\pgfqpoint{0.100000in}{0.212622in}}{\pgfqpoint{3.696000in}{3.696000in}}%
\pgfusepath{clip}%
\pgfsetbuttcap%
\pgfsetroundjoin%
\definecolor{currentfill}{rgb}{0.121569,0.466667,0.705882}%
\pgfsetfillcolor{currentfill}%
\pgfsetfillopacity{0.300610}%
\pgfsetlinewidth{1.003750pt}%
\definecolor{currentstroke}{rgb}{0.121569,0.466667,0.705882}%
\pgfsetstrokecolor{currentstroke}%
\pgfsetstrokeopacity{0.300610}%
\pgfsetdash{}{0pt}%
\pgfpathmoveto{\pgfqpoint{1.555889in}{2.034538in}}%
\pgfpathcurveto{\pgfqpoint{1.564126in}{2.034538in}}{\pgfqpoint{1.572026in}{2.037811in}}{\pgfqpoint{1.577850in}{2.043635in}}%
\pgfpathcurveto{\pgfqpoint{1.583674in}{2.049459in}}{\pgfqpoint{1.586946in}{2.057359in}}{\pgfqpoint{1.586946in}{2.065595in}}%
\pgfpathcurveto{\pgfqpoint{1.586946in}{2.073831in}}{\pgfqpoint{1.583674in}{2.081731in}}{\pgfqpoint{1.577850in}{2.087555in}}%
\pgfpathcurveto{\pgfqpoint{1.572026in}{2.093379in}}{\pgfqpoint{1.564126in}{2.096651in}}{\pgfqpoint{1.555889in}{2.096651in}}%
\pgfpathcurveto{\pgfqpoint{1.547653in}{2.096651in}}{\pgfqpoint{1.539753in}{2.093379in}}{\pgfqpoint{1.533929in}{2.087555in}}%
\pgfpathcurveto{\pgfqpoint{1.528105in}{2.081731in}}{\pgfqpoint{1.524833in}{2.073831in}}{\pgfqpoint{1.524833in}{2.065595in}}%
\pgfpathcurveto{\pgfqpoint{1.524833in}{2.057359in}}{\pgfqpoint{1.528105in}{2.049459in}}{\pgfqpoint{1.533929in}{2.043635in}}%
\pgfpathcurveto{\pgfqpoint{1.539753in}{2.037811in}}{\pgfqpoint{1.547653in}{2.034538in}}{\pgfqpoint{1.555889in}{2.034538in}}%
\pgfpathclose%
\pgfusepath{stroke,fill}%
\end{pgfscope}%
\begin{pgfscope}%
\pgfpathrectangle{\pgfqpoint{0.100000in}{0.212622in}}{\pgfqpoint{3.696000in}{3.696000in}}%
\pgfusepath{clip}%
\pgfsetbuttcap%
\pgfsetroundjoin%
\definecolor{currentfill}{rgb}{0.121569,0.466667,0.705882}%
\pgfsetfillcolor{currentfill}%
\pgfsetfillopacity{0.300610}%
\pgfsetlinewidth{1.003750pt}%
\definecolor{currentstroke}{rgb}{0.121569,0.466667,0.705882}%
\pgfsetstrokecolor{currentstroke}%
\pgfsetstrokeopacity{0.300610}%
\pgfsetdash{}{0pt}%
\pgfpathmoveto{\pgfqpoint{1.555889in}{2.034538in}}%
\pgfpathcurveto{\pgfqpoint{1.564126in}{2.034538in}}{\pgfqpoint{1.572026in}{2.037811in}}{\pgfqpoint{1.577850in}{2.043635in}}%
\pgfpathcurveto{\pgfqpoint{1.583674in}{2.049459in}}{\pgfqpoint{1.586946in}{2.057359in}}{\pgfqpoint{1.586946in}{2.065595in}}%
\pgfpathcurveto{\pgfqpoint{1.586946in}{2.073831in}}{\pgfqpoint{1.583674in}{2.081731in}}{\pgfqpoint{1.577850in}{2.087555in}}%
\pgfpathcurveto{\pgfqpoint{1.572026in}{2.093379in}}{\pgfqpoint{1.564126in}{2.096651in}}{\pgfqpoint{1.555889in}{2.096651in}}%
\pgfpathcurveto{\pgfqpoint{1.547653in}{2.096651in}}{\pgfqpoint{1.539753in}{2.093379in}}{\pgfqpoint{1.533929in}{2.087555in}}%
\pgfpathcurveto{\pgfqpoint{1.528105in}{2.081731in}}{\pgfqpoint{1.524833in}{2.073831in}}{\pgfqpoint{1.524833in}{2.065595in}}%
\pgfpathcurveto{\pgfqpoint{1.524833in}{2.057359in}}{\pgfqpoint{1.528105in}{2.049459in}}{\pgfqpoint{1.533929in}{2.043635in}}%
\pgfpathcurveto{\pgfqpoint{1.539753in}{2.037811in}}{\pgfqpoint{1.547653in}{2.034538in}}{\pgfqpoint{1.555889in}{2.034538in}}%
\pgfpathclose%
\pgfusepath{stroke,fill}%
\end{pgfscope}%
\begin{pgfscope}%
\pgfpathrectangle{\pgfqpoint{0.100000in}{0.212622in}}{\pgfqpoint{3.696000in}{3.696000in}}%
\pgfusepath{clip}%
\pgfsetbuttcap%
\pgfsetroundjoin%
\definecolor{currentfill}{rgb}{0.121569,0.466667,0.705882}%
\pgfsetfillcolor{currentfill}%
\pgfsetfillopacity{0.300610}%
\pgfsetlinewidth{1.003750pt}%
\definecolor{currentstroke}{rgb}{0.121569,0.466667,0.705882}%
\pgfsetstrokecolor{currentstroke}%
\pgfsetstrokeopacity{0.300610}%
\pgfsetdash{}{0pt}%
\pgfpathmoveto{\pgfqpoint{1.555889in}{2.034538in}}%
\pgfpathcurveto{\pgfqpoint{1.564126in}{2.034538in}}{\pgfqpoint{1.572026in}{2.037811in}}{\pgfqpoint{1.577850in}{2.043635in}}%
\pgfpathcurveto{\pgfqpoint{1.583674in}{2.049459in}}{\pgfqpoint{1.586946in}{2.057359in}}{\pgfqpoint{1.586946in}{2.065595in}}%
\pgfpathcurveto{\pgfqpoint{1.586946in}{2.073831in}}{\pgfqpoint{1.583674in}{2.081731in}}{\pgfqpoint{1.577850in}{2.087555in}}%
\pgfpathcurveto{\pgfqpoint{1.572026in}{2.093379in}}{\pgfqpoint{1.564126in}{2.096651in}}{\pgfqpoint{1.555889in}{2.096651in}}%
\pgfpathcurveto{\pgfqpoint{1.547653in}{2.096651in}}{\pgfqpoint{1.539753in}{2.093379in}}{\pgfqpoint{1.533929in}{2.087555in}}%
\pgfpathcurveto{\pgfqpoint{1.528105in}{2.081731in}}{\pgfqpoint{1.524833in}{2.073831in}}{\pgfqpoint{1.524833in}{2.065595in}}%
\pgfpathcurveto{\pgfqpoint{1.524833in}{2.057359in}}{\pgfqpoint{1.528105in}{2.049459in}}{\pgfqpoint{1.533929in}{2.043635in}}%
\pgfpathcurveto{\pgfqpoint{1.539753in}{2.037811in}}{\pgfqpoint{1.547653in}{2.034538in}}{\pgfqpoint{1.555889in}{2.034538in}}%
\pgfpathclose%
\pgfusepath{stroke,fill}%
\end{pgfscope}%
\begin{pgfscope}%
\pgfpathrectangle{\pgfqpoint{0.100000in}{0.212622in}}{\pgfqpoint{3.696000in}{3.696000in}}%
\pgfusepath{clip}%
\pgfsetbuttcap%
\pgfsetroundjoin%
\definecolor{currentfill}{rgb}{0.121569,0.466667,0.705882}%
\pgfsetfillcolor{currentfill}%
\pgfsetfillopacity{0.300610}%
\pgfsetlinewidth{1.003750pt}%
\definecolor{currentstroke}{rgb}{0.121569,0.466667,0.705882}%
\pgfsetstrokecolor{currentstroke}%
\pgfsetstrokeopacity{0.300610}%
\pgfsetdash{}{0pt}%
\pgfpathmoveto{\pgfqpoint{1.555889in}{2.034538in}}%
\pgfpathcurveto{\pgfqpoint{1.564126in}{2.034538in}}{\pgfqpoint{1.572026in}{2.037811in}}{\pgfqpoint{1.577850in}{2.043635in}}%
\pgfpathcurveto{\pgfqpoint{1.583674in}{2.049459in}}{\pgfqpoint{1.586946in}{2.057359in}}{\pgfqpoint{1.586946in}{2.065595in}}%
\pgfpathcurveto{\pgfqpoint{1.586946in}{2.073831in}}{\pgfqpoint{1.583674in}{2.081731in}}{\pgfqpoint{1.577850in}{2.087555in}}%
\pgfpathcurveto{\pgfqpoint{1.572026in}{2.093379in}}{\pgfqpoint{1.564126in}{2.096651in}}{\pgfqpoint{1.555889in}{2.096651in}}%
\pgfpathcurveto{\pgfqpoint{1.547653in}{2.096651in}}{\pgfqpoint{1.539753in}{2.093379in}}{\pgfqpoint{1.533929in}{2.087555in}}%
\pgfpathcurveto{\pgfqpoint{1.528105in}{2.081731in}}{\pgfqpoint{1.524833in}{2.073831in}}{\pgfqpoint{1.524833in}{2.065595in}}%
\pgfpathcurveto{\pgfqpoint{1.524833in}{2.057359in}}{\pgfqpoint{1.528105in}{2.049459in}}{\pgfqpoint{1.533929in}{2.043635in}}%
\pgfpathcurveto{\pgfqpoint{1.539753in}{2.037811in}}{\pgfqpoint{1.547653in}{2.034538in}}{\pgfqpoint{1.555889in}{2.034538in}}%
\pgfpathclose%
\pgfusepath{stroke,fill}%
\end{pgfscope}%
\begin{pgfscope}%
\pgfpathrectangle{\pgfqpoint{0.100000in}{0.212622in}}{\pgfqpoint{3.696000in}{3.696000in}}%
\pgfusepath{clip}%
\pgfsetbuttcap%
\pgfsetroundjoin%
\definecolor{currentfill}{rgb}{0.121569,0.466667,0.705882}%
\pgfsetfillcolor{currentfill}%
\pgfsetfillopacity{0.300610}%
\pgfsetlinewidth{1.003750pt}%
\definecolor{currentstroke}{rgb}{0.121569,0.466667,0.705882}%
\pgfsetstrokecolor{currentstroke}%
\pgfsetstrokeopacity{0.300610}%
\pgfsetdash{}{0pt}%
\pgfpathmoveto{\pgfqpoint{1.555889in}{2.034538in}}%
\pgfpathcurveto{\pgfqpoint{1.564126in}{2.034538in}}{\pgfqpoint{1.572026in}{2.037811in}}{\pgfqpoint{1.577850in}{2.043635in}}%
\pgfpathcurveto{\pgfqpoint{1.583674in}{2.049459in}}{\pgfqpoint{1.586946in}{2.057359in}}{\pgfqpoint{1.586946in}{2.065595in}}%
\pgfpathcurveto{\pgfqpoint{1.586946in}{2.073831in}}{\pgfqpoint{1.583674in}{2.081731in}}{\pgfqpoint{1.577850in}{2.087555in}}%
\pgfpathcurveto{\pgfqpoint{1.572026in}{2.093379in}}{\pgfqpoint{1.564126in}{2.096651in}}{\pgfqpoint{1.555889in}{2.096651in}}%
\pgfpathcurveto{\pgfqpoint{1.547653in}{2.096651in}}{\pgfqpoint{1.539753in}{2.093379in}}{\pgfqpoint{1.533929in}{2.087555in}}%
\pgfpathcurveto{\pgfqpoint{1.528105in}{2.081731in}}{\pgfqpoint{1.524833in}{2.073831in}}{\pgfqpoint{1.524833in}{2.065595in}}%
\pgfpathcurveto{\pgfqpoint{1.524833in}{2.057359in}}{\pgfqpoint{1.528105in}{2.049459in}}{\pgfqpoint{1.533929in}{2.043635in}}%
\pgfpathcurveto{\pgfqpoint{1.539753in}{2.037811in}}{\pgfqpoint{1.547653in}{2.034538in}}{\pgfqpoint{1.555889in}{2.034538in}}%
\pgfpathclose%
\pgfusepath{stroke,fill}%
\end{pgfscope}%
\begin{pgfscope}%
\pgfpathrectangle{\pgfqpoint{0.100000in}{0.212622in}}{\pgfqpoint{3.696000in}{3.696000in}}%
\pgfusepath{clip}%
\pgfsetbuttcap%
\pgfsetroundjoin%
\definecolor{currentfill}{rgb}{0.121569,0.466667,0.705882}%
\pgfsetfillcolor{currentfill}%
\pgfsetfillopacity{0.300610}%
\pgfsetlinewidth{1.003750pt}%
\definecolor{currentstroke}{rgb}{0.121569,0.466667,0.705882}%
\pgfsetstrokecolor{currentstroke}%
\pgfsetstrokeopacity{0.300610}%
\pgfsetdash{}{0pt}%
\pgfpathmoveto{\pgfqpoint{1.555889in}{2.034538in}}%
\pgfpathcurveto{\pgfqpoint{1.564126in}{2.034538in}}{\pgfqpoint{1.572026in}{2.037811in}}{\pgfqpoint{1.577850in}{2.043635in}}%
\pgfpathcurveto{\pgfqpoint{1.583674in}{2.049459in}}{\pgfqpoint{1.586946in}{2.057359in}}{\pgfqpoint{1.586946in}{2.065595in}}%
\pgfpathcurveto{\pgfqpoint{1.586946in}{2.073831in}}{\pgfqpoint{1.583674in}{2.081731in}}{\pgfqpoint{1.577850in}{2.087555in}}%
\pgfpathcurveto{\pgfqpoint{1.572026in}{2.093379in}}{\pgfqpoint{1.564126in}{2.096651in}}{\pgfqpoint{1.555889in}{2.096651in}}%
\pgfpathcurveto{\pgfqpoint{1.547653in}{2.096651in}}{\pgfqpoint{1.539753in}{2.093379in}}{\pgfqpoint{1.533929in}{2.087555in}}%
\pgfpathcurveto{\pgfqpoint{1.528105in}{2.081731in}}{\pgfqpoint{1.524833in}{2.073831in}}{\pgfqpoint{1.524833in}{2.065595in}}%
\pgfpathcurveto{\pgfqpoint{1.524833in}{2.057359in}}{\pgfqpoint{1.528105in}{2.049459in}}{\pgfqpoint{1.533929in}{2.043635in}}%
\pgfpathcurveto{\pgfqpoint{1.539753in}{2.037811in}}{\pgfqpoint{1.547653in}{2.034538in}}{\pgfqpoint{1.555889in}{2.034538in}}%
\pgfpathclose%
\pgfusepath{stroke,fill}%
\end{pgfscope}%
\begin{pgfscope}%
\pgfpathrectangle{\pgfqpoint{0.100000in}{0.212622in}}{\pgfqpoint{3.696000in}{3.696000in}}%
\pgfusepath{clip}%
\pgfsetbuttcap%
\pgfsetroundjoin%
\definecolor{currentfill}{rgb}{0.121569,0.466667,0.705882}%
\pgfsetfillcolor{currentfill}%
\pgfsetfillopacity{0.300610}%
\pgfsetlinewidth{1.003750pt}%
\definecolor{currentstroke}{rgb}{0.121569,0.466667,0.705882}%
\pgfsetstrokecolor{currentstroke}%
\pgfsetstrokeopacity{0.300610}%
\pgfsetdash{}{0pt}%
\pgfpathmoveto{\pgfqpoint{1.555889in}{2.034538in}}%
\pgfpathcurveto{\pgfqpoint{1.564126in}{2.034538in}}{\pgfqpoint{1.572026in}{2.037811in}}{\pgfqpoint{1.577850in}{2.043635in}}%
\pgfpathcurveto{\pgfqpoint{1.583674in}{2.049459in}}{\pgfqpoint{1.586946in}{2.057359in}}{\pgfqpoint{1.586946in}{2.065595in}}%
\pgfpathcurveto{\pgfqpoint{1.586946in}{2.073831in}}{\pgfqpoint{1.583674in}{2.081731in}}{\pgfqpoint{1.577850in}{2.087555in}}%
\pgfpathcurveto{\pgfqpoint{1.572026in}{2.093379in}}{\pgfqpoint{1.564126in}{2.096651in}}{\pgfqpoint{1.555889in}{2.096651in}}%
\pgfpathcurveto{\pgfqpoint{1.547653in}{2.096651in}}{\pgfqpoint{1.539753in}{2.093379in}}{\pgfqpoint{1.533929in}{2.087555in}}%
\pgfpathcurveto{\pgfqpoint{1.528105in}{2.081731in}}{\pgfqpoint{1.524833in}{2.073831in}}{\pgfqpoint{1.524833in}{2.065595in}}%
\pgfpathcurveto{\pgfqpoint{1.524833in}{2.057359in}}{\pgfqpoint{1.528105in}{2.049459in}}{\pgfqpoint{1.533929in}{2.043635in}}%
\pgfpathcurveto{\pgfqpoint{1.539753in}{2.037811in}}{\pgfqpoint{1.547653in}{2.034538in}}{\pgfqpoint{1.555889in}{2.034538in}}%
\pgfpathclose%
\pgfusepath{stroke,fill}%
\end{pgfscope}%
\begin{pgfscope}%
\pgfpathrectangle{\pgfqpoint{0.100000in}{0.212622in}}{\pgfqpoint{3.696000in}{3.696000in}}%
\pgfusepath{clip}%
\pgfsetbuttcap%
\pgfsetroundjoin%
\definecolor{currentfill}{rgb}{0.121569,0.466667,0.705882}%
\pgfsetfillcolor{currentfill}%
\pgfsetfillopacity{0.300610}%
\pgfsetlinewidth{1.003750pt}%
\definecolor{currentstroke}{rgb}{0.121569,0.466667,0.705882}%
\pgfsetstrokecolor{currentstroke}%
\pgfsetstrokeopacity{0.300610}%
\pgfsetdash{}{0pt}%
\pgfpathmoveto{\pgfqpoint{1.555889in}{2.034538in}}%
\pgfpathcurveto{\pgfqpoint{1.564126in}{2.034538in}}{\pgfqpoint{1.572026in}{2.037811in}}{\pgfqpoint{1.577850in}{2.043635in}}%
\pgfpathcurveto{\pgfqpoint{1.583674in}{2.049459in}}{\pgfqpoint{1.586946in}{2.057359in}}{\pgfqpoint{1.586946in}{2.065595in}}%
\pgfpathcurveto{\pgfqpoint{1.586946in}{2.073831in}}{\pgfqpoint{1.583674in}{2.081731in}}{\pgfqpoint{1.577850in}{2.087555in}}%
\pgfpathcurveto{\pgfqpoint{1.572026in}{2.093379in}}{\pgfqpoint{1.564126in}{2.096651in}}{\pgfqpoint{1.555889in}{2.096651in}}%
\pgfpathcurveto{\pgfqpoint{1.547653in}{2.096651in}}{\pgfqpoint{1.539753in}{2.093379in}}{\pgfqpoint{1.533929in}{2.087555in}}%
\pgfpathcurveto{\pgfqpoint{1.528105in}{2.081731in}}{\pgfqpoint{1.524833in}{2.073831in}}{\pgfqpoint{1.524833in}{2.065595in}}%
\pgfpathcurveto{\pgfqpoint{1.524833in}{2.057359in}}{\pgfqpoint{1.528105in}{2.049459in}}{\pgfqpoint{1.533929in}{2.043635in}}%
\pgfpathcurveto{\pgfqpoint{1.539753in}{2.037811in}}{\pgfqpoint{1.547653in}{2.034538in}}{\pgfqpoint{1.555889in}{2.034538in}}%
\pgfpathclose%
\pgfusepath{stroke,fill}%
\end{pgfscope}%
\begin{pgfscope}%
\pgfpathrectangle{\pgfqpoint{0.100000in}{0.212622in}}{\pgfqpoint{3.696000in}{3.696000in}}%
\pgfusepath{clip}%
\pgfsetbuttcap%
\pgfsetroundjoin%
\definecolor{currentfill}{rgb}{0.121569,0.466667,0.705882}%
\pgfsetfillcolor{currentfill}%
\pgfsetfillopacity{0.300610}%
\pgfsetlinewidth{1.003750pt}%
\definecolor{currentstroke}{rgb}{0.121569,0.466667,0.705882}%
\pgfsetstrokecolor{currentstroke}%
\pgfsetstrokeopacity{0.300610}%
\pgfsetdash{}{0pt}%
\pgfpathmoveto{\pgfqpoint{1.555889in}{2.034538in}}%
\pgfpathcurveto{\pgfqpoint{1.564126in}{2.034538in}}{\pgfqpoint{1.572026in}{2.037811in}}{\pgfqpoint{1.577850in}{2.043635in}}%
\pgfpathcurveto{\pgfqpoint{1.583674in}{2.049459in}}{\pgfqpoint{1.586946in}{2.057359in}}{\pgfqpoint{1.586946in}{2.065595in}}%
\pgfpathcurveto{\pgfqpoint{1.586946in}{2.073831in}}{\pgfqpoint{1.583674in}{2.081731in}}{\pgfqpoint{1.577850in}{2.087555in}}%
\pgfpathcurveto{\pgfqpoint{1.572026in}{2.093379in}}{\pgfqpoint{1.564126in}{2.096651in}}{\pgfqpoint{1.555889in}{2.096651in}}%
\pgfpathcurveto{\pgfqpoint{1.547653in}{2.096651in}}{\pgfqpoint{1.539753in}{2.093379in}}{\pgfqpoint{1.533929in}{2.087555in}}%
\pgfpathcurveto{\pgfqpoint{1.528105in}{2.081731in}}{\pgfqpoint{1.524833in}{2.073831in}}{\pgfqpoint{1.524833in}{2.065595in}}%
\pgfpathcurveto{\pgfqpoint{1.524833in}{2.057359in}}{\pgfqpoint{1.528105in}{2.049459in}}{\pgfqpoint{1.533929in}{2.043635in}}%
\pgfpathcurveto{\pgfqpoint{1.539753in}{2.037811in}}{\pgfqpoint{1.547653in}{2.034538in}}{\pgfqpoint{1.555889in}{2.034538in}}%
\pgfpathclose%
\pgfusepath{stroke,fill}%
\end{pgfscope}%
\begin{pgfscope}%
\pgfpathrectangle{\pgfqpoint{0.100000in}{0.212622in}}{\pgfqpoint{3.696000in}{3.696000in}}%
\pgfusepath{clip}%
\pgfsetbuttcap%
\pgfsetroundjoin%
\definecolor{currentfill}{rgb}{0.121569,0.466667,0.705882}%
\pgfsetfillcolor{currentfill}%
\pgfsetfillopacity{0.300610}%
\pgfsetlinewidth{1.003750pt}%
\definecolor{currentstroke}{rgb}{0.121569,0.466667,0.705882}%
\pgfsetstrokecolor{currentstroke}%
\pgfsetstrokeopacity{0.300610}%
\pgfsetdash{}{0pt}%
\pgfpathmoveto{\pgfqpoint{1.555889in}{2.034538in}}%
\pgfpathcurveto{\pgfqpoint{1.564126in}{2.034538in}}{\pgfqpoint{1.572026in}{2.037811in}}{\pgfqpoint{1.577850in}{2.043635in}}%
\pgfpathcurveto{\pgfqpoint{1.583674in}{2.049459in}}{\pgfqpoint{1.586946in}{2.057359in}}{\pgfqpoint{1.586946in}{2.065595in}}%
\pgfpathcurveto{\pgfqpoint{1.586946in}{2.073831in}}{\pgfqpoint{1.583674in}{2.081731in}}{\pgfqpoint{1.577850in}{2.087555in}}%
\pgfpathcurveto{\pgfqpoint{1.572026in}{2.093379in}}{\pgfqpoint{1.564126in}{2.096651in}}{\pgfqpoint{1.555889in}{2.096651in}}%
\pgfpathcurveto{\pgfqpoint{1.547653in}{2.096651in}}{\pgfqpoint{1.539753in}{2.093379in}}{\pgfqpoint{1.533929in}{2.087555in}}%
\pgfpathcurveto{\pgfqpoint{1.528105in}{2.081731in}}{\pgfqpoint{1.524833in}{2.073831in}}{\pgfqpoint{1.524833in}{2.065595in}}%
\pgfpathcurveto{\pgfqpoint{1.524833in}{2.057359in}}{\pgfqpoint{1.528105in}{2.049459in}}{\pgfqpoint{1.533929in}{2.043635in}}%
\pgfpathcurveto{\pgfqpoint{1.539753in}{2.037811in}}{\pgfqpoint{1.547653in}{2.034538in}}{\pgfqpoint{1.555889in}{2.034538in}}%
\pgfpathclose%
\pgfusepath{stroke,fill}%
\end{pgfscope}%
\begin{pgfscope}%
\pgfpathrectangle{\pgfqpoint{0.100000in}{0.212622in}}{\pgfqpoint{3.696000in}{3.696000in}}%
\pgfusepath{clip}%
\pgfsetbuttcap%
\pgfsetroundjoin%
\definecolor{currentfill}{rgb}{0.121569,0.466667,0.705882}%
\pgfsetfillcolor{currentfill}%
\pgfsetfillopacity{0.300610}%
\pgfsetlinewidth{1.003750pt}%
\definecolor{currentstroke}{rgb}{0.121569,0.466667,0.705882}%
\pgfsetstrokecolor{currentstroke}%
\pgfsetstrokeopacity{0.300610}%
\pgfsetdash{}{0pt}%
\pgfpathmoveto{\pgfqpoint{1.555889in}{2.034538in}}%
\pgfpathcurveto{\pgfqpoint{1.564126in}{2.034538in}}{\pgfqpoint{1.572026in}{2.037811in}}{\pgfqpoint{1.577850in}{2.043635in}}%
\pgfpathcurveto{\pgfqpoint{1.583674in}{2.049459in}}{\pgfqpoint{1.586946in}{2.057359in}}{\pgfqpoint{1.586946in}{2.065595in}}%
\pgfpathcurveto{\pgfqpoint{1.586946in}{2.073831in}}{\pgfqpoint{1.583674in}{2.081731in}}{\pgfqpoint{1.577850in}{2.087555in}}%
\pgfpathcurveto{\pgfqpoint{1.572026in}{2.093379in}}{\pgfqpoint{1.564126in}{2.096651in}}{\pgfqpoint{1.555889in}{2.096651in}}%
\pgfpathcurveto{\pgfqpoint{1.547653in}{2.096651in}}{\pgfqpoint{1.539753in}{2.093379in}}{\pgfqpoint{1.533929in}{2.087555in}}%
\pgfpathcurveto{\pgfqpoint{1.528105in}{2.081731in}}{\pgfqpoint{1.524833in}{2.073831in}}{\pgfqpoint{1.524833in}{2.065595in}}%
\pgfpathcurveto{\pgfqpoint{1.524833in}{2.057359in}}{\pgfqpoint{1.528105in}{2.049459in}}{\pgfqpoint{1.533929in}{2.043635in}}%
\pgfpathcurveto{\pgfqpoint{1.539753in}{2.037811in}}{\pgfqpoint{1.547653in}{2.034538in}}{\pgfqpoint{1.555889in}{2.034538in}}%
\pgfpathclose%
\pgfusepath{stroke,fill}%
\end{pgfscope}%
\begin{pgfscope}%
\pgfpathrectangle{\pgfqpoint{0.100000in}{0.212622in}}{\pgfqpoint{3.696000in}{3.696000in}}%
\pgfusepath{clip}%
\pgfsetbuttcap%
\pgfsetroundjoin%
\definecolor{currentfill}{rgb}{0.121569,0.466667,0.705882}%
\pgfsetfillcolor{currentfill}%
\pgfsetfillopacity{0.300610}%
\pgfsetlinewidth{1.003750pt}%
\definecolor{currentstroke}{rgb}{0.121569,0.466667,0.705882}%
\pgfsetstrokecolor{currentstroke}%
\pgfsetstrokeopacity{0.300610}%
\pgfsetdash{}{0pt}%
\pgfpathmoveto{\pgfqpoint{1.555889in}{2.034538in}}%
\pgfpathcurveto{\pgfqpoint{1.564126in}{2.034538in}}{\pgfqpoint{1.572026in}{2.037811in}}{\pgfqpoint{1.577850in}{2.043635in}}%
\pgfpathcurveto{\pgfqpoint{1.583674in}{2.049459in}}{\pgfqpoint{1.586946in}{2.057359in}}{\pgfqpoint{1.586946in}{2.065595in}}%
\pgfpathcurveto{\pgfqpoint{1.586946in}{2.073831in}}{\pgfqpoint{1.583674in}{2.081731in}}{\pgfqpoint{1.577850in}{2.087555in}}%
\pgfpathcurveto{\pgfqpoint{1.572026in}{2.093379in}}{\pgfqpoint{1.564126in}{2.096651in}}{\pgfqpoint{1.555889in}{2.096651in}}%
\pgfpathcurveto{\pgfqpoint{1.547653in}{2.096651in}}{\pgfqpoint{1.539753in}{2.093379in}}{\pgfqpoint{1.533929in}{2.087555in}}%
\pgfpathcurveto{\pgfqpoint{1.528105in}{2.081731in}}{\pgfqpoint{1.524833in}{2.073831in}}{\pgfqpoint{1.524833in}{2.065595in}}%
\pgfpathcurveto{\pgfqpoint{1.524833in}{2.057359in}}{\pgfqpoint{1.528105in}{2.049459in}}{\pgfqpoint{1.533929in}{2.043635in}}%
\pgfpathcurveto{\pgfqpoint{1.539753in}{2.037811in}}{\pgfqpoint{1.547653in}{2.034538in}}{\pgfqpoint{1.555889in}{2.034538in}}%
\pgfpathclose%
\pgfusepath{stroke,fill}%
\end{pgfscope}%
\begin{pgfscope}%
\pgfpathrectangle{\pgfqpoint{0.100000in}{0.212622in}}{\pgfqpoint{3.696000in}{3.696000in}}%
\pgfusepath{clip}%
\pgfsetbuttcap%
\pgfsetroundjoin%
\definecolor{currentfill}{rgb}{0.121569,0.466667,0.705882}%
\pgfsetfillcolor{currentfill}%
\pgfsetfillopacity{0.300610}%
\pgfsetlinewidth{1.003750pt}%
\definecolor{currentstroke}{rgb}{0.121569,0.466667,0.705882}%
\pgfsetstrokecolor{currentstroke}%
\pgfsetstrokeopacity{0.300610}%
\pgfsetdash{}{0pt}%
\pgfpathmoveto{\pgfqpoint{1.555889in}{2.034538in}}%
\pgfpathcurveto{\pgfqpoint{1.564126in}{2.034538in}}{\pgfqpoint{1.572026in}{2.037811in}}{\pgfqpoint{1.577850in}{2.043635in}}%
\pgfpathcurveto{\pgfqpoint{1.583674in}{2.049459in}}{\pgfqpoint{1.586946in}{2.057359in}}{\pgfqpoint{1.586946in}{2.065595in}}%
\pgfpathcurveto{\pgfqpoint{1.586946in}{2.073831in}}{\pgfqpoint{1.583674in}{2.081731in}}{\pgfqpoint{1.577850in}{2.087555in}}%
\pgfpathcurveto{\pgfqpoint{1.572026in}{2.093379in}}{\pgfqpoint{1.564126in}{2.096651in}}{\pgfqpoint{1.555889in}{2.096651in}}%
\pgfpathcurveto{\pgfqpoint{1.547653in}{2.096651in}}{\pgfqpoint{1.539753in}{2.093379in}}{\pgfqpoint{1.533929in}{2.087555in}}%
\pgfpathcurveto{\pgfqpoint{1.528105in}{2.081731in}}{\pgfqpoint{1.524833in}{2.073831in}}{\pgfqpoint{1.524833in}{2.065595in}}%
\pgfpathcurveto{\pgfqpoint{1.524833in}{2.057359in}}{\pgfqpoint{1.528105in}{2.049459in}}{\pgfqpoint{1.533929in}{2.043635in}}%
\pgfpathcurveto{\pgfqpoint{1.539753in}{2.037811in}}{\pgfqpoint{1.547653in}{2.034538in}}{\pgfqpoint{1.555889in}{2.034538in}}%
\pgfpathclose%
\pgfusepath{stroke,fill}%
\end{pgfscope}%
\begin{pgfscope}%
\pgfpathrectangle{\pgfqpoint{0.100000in}{0.212622in}}{\pgfqpoint{3.696000in}{3.696000in}}%
\pgfusepath{clip}%
\pgfsetbuttcap%
\pgfsetroundjoin%
\definecolor{currentfill}{rgb}{0.121569,0.466667,0.705882}%
\pgfsetfillcolor{currentfill}%
\pgfsetfillopacity{0.300610}%
\pgfsetlinewidth{1.003750pt}%
\definecolor{currentstroke}{rgb}{0.121569,0.466667,0.705882}%
\pgfsetstrokecolor{currentstroke}%
\pgfsetstrokeopacity{0.300610}%
\pgfsetdash{}{0pt}%
\pgfpathmoveto{\pgfqpoint{1.555889in}{2.034538in}}%
\pgfpathcurveto{\pgfqpoint{1.564126in}{2.034538in}}{\pgfqpoint{1.572026in}{2.037811in}}{\pgfqpoint{1.577850in}{2.043635in}}%
\pgfpathcurveto{\pgfqpoint{1.583674in}{2.049459in}}{\pgfqpoint{1.586946in}{2.057359in}}{\pgfqpoint{1.586946in}{2.065595in}}%
\pgfpathcurveto{\pgfqpoint{1.586946in}{2.073831in}}{\pgfqpoint{1.583674in}{2.081731in}}{\pgfqpoint{1.577850in}{2.087555in}}%
\pgfpathcurveto{\pgfqpoint{1.572026in}{2.093379in}}{\pgfqpoint{1.564126in}{2.096651in}}{\pgfqpoint{1.555889in}{2.096651in}}%
\pgfpathcurveto{\pgfqpoint{1.547653in}{2.096651in}}{\pgfqpoint{1.539753in}{2.093379in}}{\pgfqpoint{1.533929in}{2.087555in}}%
\pgfpathcurveto{\pgfqpoint{1.528105in}{2.081731in}}{\pgfqpoint{1.524833in}{2.073831in}}{\pgfqpoint{1.524833in}{2.065595in}}%
\pgfpathcurveto{\pgfqpoint{1.524833in}{2.057359in}}{\pgfqpoint{1.528105in}{2.049459in}}{\pgfqpoint{1.533929in}{2.043635in}}%
\pgfpathcurveto{\pgfqpoint{1.539753in}{2.037811in}}{\pgfqpoint{1.547653in}{2.034538in}}{\pgfqpoint{1.555889in}{2.034538in}}%
\pgfpathclose%
\pgfusepath{stroke,fill}%
\end{pgfscope}%
\begin{pgfscope}%
\pgfpathrectangle{\pgfqpoint{0.100000in}{0.212622in}}{\pgfqpoint{3.696000in}{3.696000in}}%
\pgfusepath{clip}%
\pgfsetbuttcap%
\pgfsetroundjoin%
\definecolor{currentfill}{rgb}{0.121569,0.466667,0.705882}%
\pgfsetfillcolor{currentfill}%
\pgfsetfillopacity{0.300610}%
\pgfsetlinewidth{1.003750pt}%
\definecolor{currentstroke}{rgb}{0.121569,0.466667,0.705882}%
\pgfsetstrokecolor{currentstroke}%
\pgfsetstrokeopacity{0.300610}%
\pgfsetdash{}{0pt}%
\pgfpathmoveto{\pgfqpoint{1.555889in}{2.034538in}}%
\pgfpathcurveto{\pgfqpoint{1.564126in}{2.034538in}}{\pgfqpoint{1.572026in}{2.037811in}}{\pgfqpoint{1.577850in}{2.043635in}}%
\pgfpathcurveto{\pgfqpoint{1.583674in}{2.049459in}}{\pgfqpoint{1.586946in}{2.057359in}}{\pgfqpoint{1.586946in}{2.065595in}}%
\pgfpathcurveto{\pgfqpoint{1.586946in}{2.073831in}}{\pgfqpoint{1.583674in}{2.081731in}}{\pgfqpoint{1.577850in}{2.087555in}}%
\pgfpathcurveto{\pgfqpoint{1.572026in}{2.093379in}}{\pgfqpoint{1.564126in}{2.096651in}}{\pgfqpoint{1.555889in}{2.096651in}}%
\pgfpathcurveto{\pgfqpoint{1.547653in}{2.096651in}}{\pgfqpoint{1.539753in}{2.093379in}}{\pgfqpoint{1.533929in}{2.087555in}}%
\pgfpathcurveto{\pgfqpoint{1.528105in}{2.081731in}}{\pgfqpoint{1.524833in}{2.073831in}}{\pgfqpoint{1.524833in}{2.065595in}}%
\pgfpathcurveto{\pgfqpoint{1.524833in}{2.057359in}}{\pgfqpoint{1.528105in}{2.049459in}}{\pgfqpoint{1.533929in}{2.043635in}}%
\pgfpathcurveto{\pgfqpoint{1.539753in}{2.037811in}}{\pgfqpoint{1.547653in}{2.034538in}}{\pgfqpoint{1.555889in}{2.034538in}}%
\pgfpathclose%
\pgfusepath{stroke,fill}%
\end{pgfscope}%
\begin{pgfscope}%
\pgfpathrectangle{\pgfqpoint{0.100000in}{0.212622in}}{\pgfqpoint{3.696000in}{3.696000in}}%
\pgfusepath{clip}%
\pgfsetbuttcap%
\pgfsetroundjoin%
\definecolor{currentfill}{rgb}{0.121569,0.466667,0.705882}%
\pgfsetfillcolor{currentfill}%
\pgfsetfillopacity{0.300610}%
\pgfsetlinewidth{1.003750pt}%
\definecolor{currentstroke}{rgb}{0.121569,0.466667,0.705882}%
\pgfsetstrokecolor{currentstroke}%
\pgfsetstrokeopacity{0.300610}%
\pgfsetdash{}{0pt}%
\pgfpathmoveto{\pgfqpoint{1.555889in}{2.034538in}}%
\pgfpathcurveto{\pgfqpoint{1.564126in}{2.034538in}}{\pgfqpoint{1.572026in}{2.037811in}}{\pgfqpoint{1.577850in}{2.043635in}}%
\pgfpathcurveto{\pgfqpoint{1.583674in}{2.049459in}}{\pgfqpoint{1.586946in}{2.057359in}}{\pgfqpoint{1.586946in}{2.065595in}}%
\pgfpathcurveto{\pgfqpoint{1.586946in}{2.073831in}}{\pgfqpoint{1.583674in}{2.081731in}}{\pgfqpoint{1.577850in}{2.087555in}}%
\pgfpathcurveto{\pgfqpoint{1.572026in}{2.093379in}}{\pgfqpoint{1.564126in}{2.096651in}}{\pgfqpoint{1.555889in}{2.096651in}}%
\pgfpathcurveto{\pgfqpoint{1.547653in}{2.096651in}}{\pgfqpoint{1.539753in}{2.093379in}}{\pgfqpoint{1.533929in}{2.087555in}}%
\pgfpathcurveto{\pgfqpoint{1.528105in}{2.081731in}}{\pgfqpoint{1.524833in}{2.073831in}}{\pgfqpoint{1.524833in}{2.065595in}}%
\pgfpathcurveto{\pgfqpoint{1.524833in}{2.057359in}}{\pgfqpoint{1.528105in}{2.049459in}}{\pgfqpoint{1.533929in}{2.043635in}}%
\pgfpathcurveto{\pgfqpoint{1.539753in}{2.037811in}}{\pgfqpoint{1.547653in}{2.034538in}}{\pgfqpoint{1.555889in}{2.034538in}}%
\pgfpathclose%
\pgfusepath{stroke,fill}%
\end{pgfscope}%
\begin{pgfscope}%
\pgfpathrectangle{\pgfqpoint{0.100000in}{0.212622in}}{\pgfqpoint{3.696000in}{3.696000in}}%
\pgfusepath{clip}%
\pgfsetbuttcap%
\pgfsetroundjoin%
\definecolor{currentfill}{rgb}{0.121569,0.466667,0.705882}%
\pgfsetfillcolor{currentfill}%
\pgfsetfillopacity{0.300610}%
\pgfsetlinewidth{1.003750pt}%
\definecolor{currentstroke}{rgb}{0.121569,0.466667,0.705882}%
\pgfsetstrokecolor{currentstroke}%
\pgfsetstrokeopacity{0.300610}%
\pgfsetdash{}{0pt}%
\pgfpathmoveto{\pgfqpoint{1.555889in}{2.034538in}}%
\pgfpathcurveto{\pgfqpoint{1.564126in}{2.034538in}}{\pgfqpoint{1.572026in}{2.037811in}}{\pgfqpoint{1.577850in}{2.043635in}}%
\pgfpathcurveto{\pgfqpoint{1.583674in}{2.049459in}}{\pgfqpoint{1.586946in}{2.057359in}}{\pgfqpoint{1.586946in}{2.065595in}}%
\pgfpathcurveto{\pgfqpoint{1.586946in}{2.073831in}}{\pgfqpoint{1.583674in}{2.081731in}}{\pgfqpoint{1.577850in}{2.087555in}}%
\pgfpathcurveto{\pgfqpoint{1.572026in}{2.093379in}}{\pgfqpoint{1.564126in}{2.096651in}}{\pgfqpoint{1.555889in}{2.096651in}}%
\pgfpathcurveto{\pgfqpoint{1.547653in}{2.096651in}}{\pgfqpoint{1.539753in}{2.093379in}}{\pgfqpoint{1.533929in}{2.087555in}}%
\pgfpathcurveto{\pgfqpoint{1.528105in}{2.081731in}}{\pgfqpoint{1.524833in}{2.073831in}}{\pgfqpoint{1.524833in}{2.065595in}}%
\pgfpathcurveto{\pgfqpoint{1.524833in}{2.057359in}}{\pgfqpoint{1.528105in}{2.049459in}}{\pgfqpoint{1.533929in}{2.043635in}}%
\pgfpathcurveto{\pgfqpoint{1.539753in}{2.037811in}}{\pgfqpoint{1.547653in}{2.034538in}}{\pgfqpoint{1.555889in}{2.034538in}}%
\pgfpathclose%
\pgfusepath{stroke,fill}%
\end{pgfscope}%
\begin{pgfscope}%
\pgfpathrectangle{\pgfqpoint{0.100000in}{0.212622in}}{\pgfqpoint{3.696000in}{3.696000in}}%
\pgfusepath{clip}%
\pgfsetbuttcap%
\pgfsetroundjoin%
\definecolor{currentfill}{rgb}{0.121569,0.466667,0.705882}%
\pgfsetfillcolor{currentfill}%
\pgfsetfillopacity{0.300610}%
\pgfsetlinewidth{1.003750pt}%
\definecolor{currentstroke}{rgb}{0.121569,0.466667,0.705882}%
\pgfsetstrokecolor{currentstroke}%
\pgfsetstrokeopacity{0.300610}%
\pgfsetdash{}{0pt}%
\pgfpathmoveto{\pgfqpoint{1.555889in}{2.034538in}}%
\pgfpathcurveto{\pgfqpoint{1.564126in}{2.034538in}}{\pgfqpoint{1.572026in}{2.037811in}}{\pgfqpoint{1.577850in}{2.043635in}}%
\pgfpathcurveto{\pgfqpoint{1.583674in}{2.049459in}}{\pgfqpoint{1.586946in}{2.057359in}}{\pgfqpoint{1.586946in}{2.065595in}}%
\pgfpathcurveto{\pgfqpoint{1.586946in}{2.073831in}}{\pgfqpoint{1.583674in}{2.081731in}}{\pgfqpoint{1.577850in}{2.087555in}}%
\pgfpathcurveto{\pgfqpoint{1.572026in}{2.093379in}}{\pgfqpoint{1.564126in}{2.096651in}}{\pgfqpoint{1.555889in}{2.096651in}}%
\pgfpathcurveto{\pgfqpoint{1.547653in}{2.096651in}}{\pgfqpoint{1.539753in}{2.093379in}}{\pgfqpoint{1.533929in}{2.087555in}}%
\pgfpathcurveto{\pgfqpoint{1.528105in}{2.081731in}}{\pgfqpoint{1.524833in}{2.073831in}}{\pgfqpoint{1.524833in}{2.065595in}}%
\pgfpathcurveto{\pgfqpoint{1.524833in}{2.057359in}}{\pgfqpoint{1.528105in}{2.049459in}}{\pgfqpoint{1.533929in}{2.043635in}}%
\pgfpathcurveto{\pgfqpoint{1.539753in}{2.037811in}}{\pgfqpoint{1.547653in}{2.034538in}}{\pgfqpoint{1.555889in}{2.034538in}}%
\pgfpathclose%
\pgfusepath{stroke,fill}%
\end{pgfscope}%
\begin{pgfscope}%
\pgfpathrectangle{\pgfqpoint{0.100000in}{0.212622in}}{\pgfqpoint{3.696000in}{3.696000in}}%
\pgfusepath{clip}%
\pgfsetbuttcap%
\pgfsetroundjoin%
\definecolor{currentfill}{rgb}{0.121569,0.466667,0.705882}%
\pgfsetfillcolor{currentfill}%
\pgfsetfillopacity{0.300610}%
\pgfsetlinewidth{1.003750pt}%
\definecolor{currentstroke}{rgb}{0.121569,0.466667,0.705882}%
\pgfsetstrokecolor{currentstroke}%
\pgfsetstrokeopacity{0.300610}%
\pgfsetdash{}{0pt}%
\pgfpathmoveto{\pgfqpoint{1.555889in}{2.034538in}}%
\pgfpathcurveto{\pgfqpoint{1.564126in}{2.034538in}}{\pgfqpoint{1.572026in}{2.037811in}}{\pgfqpoint{1.577850in}{2.043635in}}%
\pgfpathcurveto{\pgfqpoint{1.583674in}{2.049459in}}{\pgfqpoint{1.586946in}{2.057359in}}{\pgfqpoint{1.586946in}{2.065595in}}%
\pgfpathcurveto{\pgfqpoint{1.586946in}{2.073831in}}{\pgfqpoint{1.583674in}{2.081731in}}{\pgfqpoint{1.577850in}{2.087555in}}%
\pgfpathcurveto{\pgfqpoint{1.572026in}{2.093379in}}{\pgfqpoint{1.564126in}{2.096651in}}{\pgfqpoint{1.555889in}{2.096651in}}%
\pgfpathcurveto{\pgfqpoint{1.547653in}{2.096651in}}{\pgfqpoint{1.539753in}{2.093379in}}{\pgfqpoint{1.533929in}{2.087555in}}%
\pgfpathcurveto{\pgfqpoint{1.528105in}{2.081731in}}{\pgfqpoint{1.524833in}{2.073831in}}{\pgfqpoint{1.524833in}{2.065595in}}%
\pgfpathcurveto{\pgfqpoint{1.524833in}{2.057359in}}{\pgfqpoint{1.528105in}{2.049459in}}{\pgfqpoint{1.533929in}{2.043635in}}%
\pgfpathcurveto{\pgfqpoint{1.539753in}{2.037811in}}{\pgfqpoint{1.547653in}{2.034538in}}{\pgfqpoint{1.555889in}{2.034538in}}%
\pgfpathclose%
\pgfusepath{stroke,fill}%
\end{pgfscope}%
\begin{pgfscope}%
\pgfpathrectangle{\pgfqpoint{0.100000in}{0.212622in}}{\pgfqpoint{3.696000in}{3.696000in}}%
\pgfusepath{clip}%
\pgfsetbuttcap%
\pgfsetroundjoin%
\definecolor{currentfill}{rgb}{0.121569,0.466667,0.705882}%
\pgfsetfillcolor{currentfill}%
\pgfsetfillopacity{0.300610}%
\pgfsetlinewidth{1.003750pt}%
\definecolor{currentstroke}{rgb}{0.121569,0.466667,0.705882}%
\pgfsetstrokecolor{currentstroke}%
\pgfsetstrokeopacity{0.300610}%
\pgfsetdash{}{0pt}%
\pgfpathmoveto{\pgfqpoint{1.555889in}{2.034538in}}%
\pgfpathcurveto{\pgfqpoint{1.564126in}{2.034538in}}{\pgfqpoint{1.572026in}{2.037811in}}{\pgfqpoint{1.577850in}{2.043635in}}%
\pgfpathcurveto{\pgfqpoint{1.583673in}{2.049459in}}{\pgfqpoint{1.586946in}{2.057359in}}{\pgfqpoint{1.586946in}{2.065595in}}%
\pgfpathcurveto{\pgfqpoint{1.586946in}{2.073831in}}{\pgfqpoint{1.583673in}{2.081731in}}{\pgfqpoint{1.577850in}{2.087555in}}%
\pgfpathcurveto{\pgfqpoint{1.572026in}{2.093379in}}{\pgfqpoint{1.564126in}{2.096651in}}{\pgfqpoint{1.555889in}{2.096651in}}%
\pgfpathcurveto{\pgfqpoint{1.547653in}{2.096651in}}{\pgfqpoint{1.539753in}{2.093379in}}{\pgfqpoint{1.533929in}{2.087555in}}%
\pgfpathcurveto{\pgfqpoint{1.528105in}{2.081731in}}{\pgfqpoint{1.524833in}{2.073831in}}{\pgfqpoint{1.524833in}{2.065595in}}%
\pgfpathcurveto{\pgfqpoint{1.524833in}{2.057359in}}{\pgfqpoint{1.528105in}{2.049459in}}{\pgfqpoint{1.533929in}{2.043635in}}%
\pgfpathcurveto{\pgfqpoint{1.539753in}{2.037811in}}{\pgfqpoint{1.547653in}{2.034538in}}{\pgfqpoint{1.555889in}{2.034538in}}%
\pgfpathclose%
\pgfusepath{stroke,fill}%
\end{pgfscope}%
\begin{pgfscope}%
\pgfpathrectangle{\pgfqpoint{0.100000in}{0.212622in}}{\pgfqpoint{3.696000in}{3.696000in}}%
\pgfusepath{clip}%
\pgfsetbuttcap%
\pgfsetroundjoin%
\definecolor{currentfill}{rgb}{0.121569,0.466667,0.705882}%
\pgfsetfillcolor{currentfill}%
\pgfsetfillopacity{0.300610}%
\pgfsetlinewidth{1.003750pt}%
\definecolor{currentstroke}{rgb}{0.121569,0.466667,0.705882}%
\pgfsetstrokecolor{currentstroke}%
\pgfsetstrokeopacity{0.300610}%
\pgfsetdash{}{0pt}%
\pgfpathmoveto{\pgfqpoint{1.555889in}{2.034538in}}%
\pgfpathcurveto{\pgfqpoint{1.564125in}{2.034538in}}{\pgfqpoint{1.572025in}{2.037811in}}{\pgfqpoint{1.577849in}{2.043635in}}%
\pgfpathcurveto{\pgfqpoint{1.583673in}{2.049458in}}{\pgfqpoint{1.586946in}{2.057359in}}{\pgfqpoint{1.586946in}{2.065595in}}%
\pgfpathcurveto{\pgfqpoint{1.586946in}{2.073831in}}{\pgfqpoint{1.583673in}{2.081731in}}{\pgfqpoint{1.577849in}{2.087555in}}%
\pgfpathcurveto{\pgfqpoint{1.572025in}{2.093379in}}{\pgfqpoint{1.564125in}{2.096651in}}{\pgfqpoint{1.555889in}{2.096651in}}%
\pgfpathcurveto{\pgfqpoint{1.547653in}{2.096651in}}{\pgfqpoint{1.539753in}{2.093379in}}{\pgfqpoint{1.533929in}{2.087555in}}%
\pgfpathcurveto{\pgfqpoint{1.528105in}{2.081731in}}{\pgfqpoint{1.524833in}{2.073831in}}{\pgfqpoint{1.524833in}{2.065595in}}%
\pgfpathcurveto{\pgfqpoint{1.524833in}{2.057359in}}{\pgfqpoint{1.528105in}{2.049458in}}{\pgfqpoint{1.533929in}{2.043635in}}%
\pgfpathcurveto{\pgfqpoint{1.539753in}{2.037811in}}{\pgfqpoint{1.547653in}{2.034538in}}{\pgfqpoint{1.555889in}{2.034538in}}%
\pgfpathclose%
\pgfusepath{stroke,fill}%
\end{pgfscope}%
\begin{pgfscope}%
\pgfpathrectangle{\pgfqpoint{0.100000in}{0.212622in}}{\pgfqpoint{3.696000in}{3.696000in}}%
\pgfusepath{clip}%
\pgfsetbuttcap%
\pgfsetroundjoin%
\definecolor{currentfill}{rgb}{0.121569,0.466667,0.705882}%
\pgfsetfillcolor{currentfill}%
\pgfsetfillopacity{0.300610}%
\pgfsetlinewidth{1.003750pt}%
\definecolor{currentstroke}{rgb}{0.121569,0.466667,0.705882}%
\pgfsetstrokecolor{currentstroke}%
\pgfsetstrokeopacity{0.300610}%
\pgfsetdash{}{0pt}%
\pgfpathmoveto{\pgfqpoint{1.555889in}{2.034538in}}%
\pgfpathcurveto{\pgfqpoint{1.564125in}{2.034538in}}{\pgfqpoint{1.572025in}{2.037810in}}{\pgfqpoint{1.577849in}{2.043634in}}%
\pgfpathcurveto{\pgfqpoint{1.583673in}{2.049458in}}{\pgfqpoint{1.586945in}{2.057358in}}{\pgfqpoint{1.586945in}{2.065594in}}%
\pgfpathcurveto{\pgfqpoint{1.586945in}{2.073831in}}{\pgfqpoint{1.583673in}{2.081731in}}{\pgfqpoint{1.577849in}{2.087555in}}%
\pgfpathcurveto{\pgfqpoint{1.572025in}{2.093379in}}{\pgfqpoint{1.564125in}{2.096651in}}{\pgfqpoint{1.555889in}{2.096651in}}%
\pgfpathcurveto{\pgfqpoint{1.547652in}{2.096651in}}{\pgfqpoint{1.539752in}{2.093379in}}{\pgfqpoint{1.533928in}{2.087555in}}%
\pgfpathcurveto{\pgfqpoint{1.528105in}{2.081731in}}{\pgfqpoint{1.524832in}{2.073831in}}{\pgfqpoint{1.524832in}{2.065594in}}%
\pgfpathcurveto{\pgfqpoint{1.524832in}{2.057358in}}{\pgfqpoint{1.528105in}{2.049458in}}{\pgfqpoint{1.533928in}{2.043634in}}%
\pgfpathcurveto{\pgfqpoint{1.539752in}{2.037810in}}{\pgfqpoint{1.547652in}{2.034538in}}{\pgfqpoint{1.555889in}{2.034538in}}%
\pgfpathclose%
\pgfusepath{stroke,fill}%
\end{pgfscope}%
\begin{pgfscope}%
\pgfpathrectangle{\pgfqpoint{0.100000in}{0.212622in}}{\pgfqpoint{3.696000in}{3.696000in}}%
\pgfusepath{clip}%
\pgfsetbuttcap%
\pgfsetroundjoin%
\definecolor{currentfill}{rgb}{0.121569,0.466667,0.705882}%
\pgfsetfillcolor{currentfill}%
\pgfsetfillopacity{0.300611}%
\pgfsetlinewidth{1.003750pt}%
\definecolor{currentstroke}{rgb}{0.121569,0.466667,0.705882}%
\pgfsetstrokecolor{currentstroke}%
\pgfsetstrokeopacity{0.300611}%
\pgfsetdash{}{0pt}%
\pgfpathmoveto{\pgfqpoint{1.555888in}{2.034538in}}%
\pgfpathcurveto{\pgfqpoint{1.564125in}{2.034538in}}{\pgfqpoint{1.572025in}{2.037810in}}{\pgfqpoint{1.577849in}{2.043634in}}%
\pgfpathcurveto{\pgfqpoint{1.583672in}{2.049458in}}{\pgfqpoint{1.586945in}{2.057358in}}{\pgfqpoint{1.586945in}{2.065594in}}%
\pgfpathcurveto{\pgfqpoint{1.586945in}{2.073830in}}{\pgfqpoint{1.583672in}{2.081731in}}{\pgfqpoint{1.577849in}{2.087554in}}%
\pgfpathcurveto{\pgfqpoint{1.572025in}{2.093378in}}{\pgfqpoint{1.564125in}{2.096651in}}{\pgfqpoint{1.555888in}{2.096651in}}%
\pgfpathcurveto{\pgfqpoint{1.547652in}{2.096651in}}{\pgfqpoint{1.539752in}{2.093378in}}{\pgfqpoint{1.533928in}{2.087554in}}%
\pgfpathcurveto{\pgfqpoint{1.528104in}{2.081731in}}{\pgfqpoint{1.524832in}{2.073830in}}{\pgfqpoint{1.524832in}{2.065594in}}%
\pgfpathcurveto{\pgfqpoint{1.524832in}{2.057358in}}{\pgfqpoint{1.528104in}{2.049458in}}{\pgfqpoint{1.533928in}{2.043634in}}%
\pgfpathcurveto{\pgfqpoint{1.539752in}{2.037810in}}{\pgfqpoint{1.547652in}{2.034538in}}{\pgfqpoint{1.555888in}{2.034538in}}%
\pgfpathclose%
\pgfusepath{stroke,fill}%
\end{pgfscope}%
\begin{pgfscope}%
\pgfpathrectangle{\pgfqpoint{0.100000in}{0.212622in}}{\pgfqpoint{3.696000in}{3.696000in}}%
\pgfusepath{clip}%
\pgfsetbuttcap%
\pgfsetroundjoin%
\definecolor{currentfill}{rgb}{0.121569,0.466667,0.705882}%
\pgfsetfillcolor{currentfill}%
\pgfsetfillopacity{0.300611}%
\pgfsetlinewidth{1.003750pt}%
\definecolor{currentstroke}{rgb}{0.121569,0.466667,0.705882}%
\pgfsetstrokecolor{currentstroke}%
\pgfsetstrokeopacity{0.300611}%
\pgfsetdash{}{0pt}%
\pgfpathmoveto{\pgfqpoint{1.555887in}{2.034537in}}%
\pgfpathcurveto{\pgfqpoint{1.564124in}{2.034537in}}{\pgfqpoint{1.572024in}{2.037809in}}{\pgfqpoint{1.577848in}{2.043633in}}%
\pgfpathcurveto{\pgfqpoint{1.583671in}{2.049457in}}{\pgfqpoint{1.586944in}{2.057357in}}{\pgfqpoint{1.586944in}{2.065593in}}%
\pgfpathcurveto{\pgfqpoint{1.586944in}{2.073830in}}{\pgfqpoint{1.583671in}{2.081730in}}{\pgfqpoint{1.577848in}{2.087554in}}%
\pgfpathcurveto{\pgfqpoint{1.572024in}{2.093378in}}{\pgfqpoint{1.564124in}{2.096650in}}{\pgfqpoint{1.555887in}{2.096650in}}%
\pgfpathcurveto{\pgfqpoint{1.547651in}{2.096650in}}{\pgfqpoint{1.539751in}{2.093378in}}{\pgfqpoint{1.533927in}{2.087554in}}%
\pgfpathcurveto{\pgfqpoint{1.528103in}{2.081730in}}{\pgfqpoint{1.524831in}{2.073830in}}{\pgfqpoint{1.524831in}{2.065593in}}%
\pgfpathcurveto{\pgfqpoint{1.524831in}{2.057357in}}{\pgfqpoint{1.528103in}{2.049457in}}{\pgfqpoint{1.533927in}{2.043633in}}%
\pgfpathcurveto{\pgfqpoint{1.539751in}{2.037809in}}{\pgfqpoint{1.547651in}{2.034537in}}{\pgfqpoint{1.555887in}{2.034537in}}%
\pgfpathclose%
\pgfusepath{stroke,fill}%
\end{pgfscope}%
\begin{pgfscope}%
\pgfpathrectangle{\pgfqpoint{0.100000in}{0.212622in}}{\pgfqpoint{3.696000in}{3.696000in}}%
\pgfusepath{clip}%
\pgfsetbuttcap%
\pgfsetroundjoin%
\definecolor{currentfill}{rgb}{0.121569,0.466667,0.705882}%
\pgfsetfillcolor{currentfill}%
\pgfsetfillopacity{0.300611}%
\pgfsetlinewidth{1.003750pt}%
\definecolor{currentstroke}{rgb}{0.121569,0.466667,0.705882}%
\pgfsetstrokecolor{currentstroke}%
\pgfsetstrokeopacity{0.300611}%
\pgfsetdash{}{0pt}%
\pgfpathmoveto{\pgfqpoint{1.555886in}{2.034536in}}%
\pgfpathcurveto{\pgfqpoint{1.564122in}{2.034536in}}{\pgfqpoint{1.572022in}{2.037808in}}{\pgfqpoint{1.577846in}{2.043632in}}%
\pgfpathcurveto{\pgfqpoint{1.583670in}{2.049456in}}{\pgfqpoint{1.586943in}{2.057356in}}{\pgfqpoint{1.586943in}{2.065593in}}%
\pgfpathcurveto{\pgfqpoint{1.586943in}{2.073829in}}{\pgfqpoint{1.583670in}{2.081729in}}{\pgfqpoint{1.577846in}{2.087553in}}%
\pgfpathcurveto{\pgfqpoint{1.572022in}{2.093377in}}{\pgfqpoint{1.564122in}{2.096649in}}{\pgfqpoint{1.555886in}{2.096649in}}%
\pgfpathcurveto{\pgfqpoint{1.547650in}{2.096649in}}{\pgfqpoint{1.539750in}{2.093377in}}{\pgfqpoint{1.533926in}{2.087553in}}%
\pgfpathcurveto{\pgfqpoint{1.528102in}{2.081729in}}{\pgfqpoint{1.524830in}{2.073829in}}{\pgfqpoint{1.524830in}{2.065593in}}%
\pgfpathcurveto{\pgfqpoint{1.524830in}{2.057356in}}{\pgfqpoint{1.528102in}{2.049456in}}{\pgfqpoint{1.533926in}{2.043632in}}%
\pgfpathcurveto{\pgfqpoint{1.539750in}{2.037808in}}{\pgfqpoint{1.547650in}{2.034536in}}{\pgfqpoint{1.555886in}{2.034536in}}%
\pgfpathclose%
\pgfusepath{stroke,fill}%
\end{pgfscope}%
\begin{pgfscope}%
\pgfpathrectangle{\pgfqpoint{0.100000in}{0.212622in}}{\pgfqpoint{3.696000in}{3.696000in}}%
\pgfusepath{clip}%
\pgfsetbuttcap%
\pgfsetroundjoin%
\definecolor{currentfill}{rgb}{0.121569,0.466667,0.705882}%
\pgfsetfillcolor{currentfill}%
\pgfsetfillopacity{0.300611}%
\pgfsetlinewidth{1.003750pt}%
\definecolor{currentstroke}{rgb}{0.121569,0.466667,0.705882}%
\pgfsetstrokecolor{currentstroke}%
\pgfsetstrokeopacity{0.300611}%
\pgfsetdash{}{0pt}%
\pgfpathmoveto{\pgfqpoint{1.555884in}{2.034535in}}%
\pgfpathcurveto{\pgfqpoint{1.564120in}{2.034535in}}{\pgfqpoint{1.572020in}{2.037807in}}{\pgfqpoint{1.577844in}{2.043631in}}%
\pgfpathcurveto{\pgfqpoint{1.583668in}{2.049455in}}{\pgfqpoint{1.586940in}{2.057355in}}{\pgfqpoint{1.586940in}{2.065591in}}%
\pgfpathcurveto{\pgfqpoint{1.586940in}{2.073827in}}{\pgfqpoint{1.583668in}{2.081727in}}{\pgfqpoint{1.577844in}{2.087551in}}%
\pgfpathcurveto{\pgfqpoint{1.572020in}{2.093375in}}{\pgfqpoint{1.564120in}{2.096648in}}{\pgfqpoint{1.555884in}{2.096648in}}%
\pgfpathcurveto{\pgfqpoint{1.547648in}{2.096648in}}{\pgfqpoint{1.539748in}{2.093375in}}{\pgfqpoint{1.533924in}{2.087551in}}%
\pgfpathcurveto{\pgfqpoint{1.528100in}{2.081727in}}{\pgfqpoint{1.524827in}{2.073827in}}{\pgfqpoint{1.524827in}{2.065591in}}%
\pgfpathcurveto{\pgfqpoint{1.524827in}{2.057355in}}{\pgfqpoint{1.528100in}{2.049455in}}{\pgfqpoint{1.533924in}{2.043631in}}%
\pgfpathcurveto{\pgfqpoint{1.539748in}{2.037807in}}{\pgfqpoint{1.547648in}{2.034535in}}{\pgfqpoint{1.555884in}{2.034535in}}%
\pgfpathclose%
\pgfusepath{stroke,fill}%
\end{pgfscope}%
\begin{pgfscope}%
\pgfpathrectangle{\pgfqpoint{0.100000in}{0.212622in}}{\pgfqpoint{3.696000in}{3.696000in}}%
\pgfusepath{clip}%
\pgfsetbuttcap%
\pgfsetroundjoin%
\definecolor{currentfill}{rgb}{0.121569,0.466667,0.705882}%
\pgfsetfillcolor{currentfill}%
\pgfsetfillopacity{0.300611}%
\pgfsetlinewidth{1.003750pt}%
\definecolor{currentstroke}{rgb}{0.121569,0.466667,0.705882}%
\pgfsetstrokecolor{currentstroke}%
\pgfsetstrokeopacity{0.300611}%
\pgfsetdash{}{0pt}%
\pgfpathmoveto{\pgfqpoint{1.555882in}{2.034534in}}%
\pgfpathcurveto{\pgfqpoint{1.564118in}{2.034534in}}{\pgfqpoint{1.572018in}{2.037807in}}{\pgfqpoint{1.577842in}{2.043630in}}%
\pgfpathcurveto{\pgfqpoint{1.583666in}{2.049454in}}{\pgfqpoint{1.586938in}{2.057354in}}{\pgfqpoint{1.586938in}{2.065591in}}%
\pgfpathcurveto{\pgfqpoint{1.586938in}{2.073827in}}{\pgfqpoint{1.583666in}{2.081727in}}{\pgfqpoint{1.577842in}{2.087551in}}%
\pgfpathcurveto{\pgfqpoint{1.572018in}{2.093375in}}{\pgfqpoint{1.564118in}{2.096647in}}{\pgfqpoint{1.555882in}{2.096647in}}%
\pgfpathcurveto{\pgfqpoint{1.547645in}{2.096647in}}{\pgfqpoint{1.539745in}{2.093375in}}{\pgfqpoint{1.533921in}{2.087551in}}%
\pgfpathcurveto{\pgfqpoint{1.528097in}{2.081727in}}{\pgfqpoint{1.524825in}{2.073827in}}{\pgfqpoint{1.524825in}{2.065591in}}%
\pgfpathcurveto{\pgfqpoint{1.524825in}{2.057354in}}{\pgfqpoint{1.528097in}{2.049454in}}{\pgfqpoint{1.533921in}{2.043630in}}%
\pgfpathcurveto{\pgfqpoint{1.539745in}{2.037807in}}{\pgfqpoint{1.547645in}{2.034534in}}{\pgfqpoint{1.555882in}{2.034534in}}%
\pgfpathclose%
\pgfusepath{stroke,fill}%
\end{pgfscope}%
\begin{pgfscope}%
\pgfpathrectangle{\pgfqpoint{0.100000in}{0.212622in}}{\pgfqpoint{3.696000in}{3.696000in}}%
\pgfusepath{clip}%
\pgfsetbuttcap%
\pgfsetroundjoin%
\definecolor{currentfill}{rgb}{0.121569,0.466667,0.705882}%
\pgfsetfillcolor{currentfill}%
\pgfsetfillopacity{0.300613}%
\pgfsetlinewidth{1.003750pt}%
\definecolor{currentstroke}{rgb}{0.121569,0.466667,0.705882}%
\pgfsetstrokecolor{currentstroke}%
\pgfsetstrokeopacity{0.300613}%
\pgfsetdash{}{0pt}%
\pgfpathmoveto{\pgfqpoint{1.555866in}{2.034525in}}%
\pgfpathcurveto{\pgfqpoint{1.564102in}{2.034525in}}{\pgfqpoint{1.572002in}{2.037798in}}{\pgfqpoint{1.577826in}{2.043622in}}%
\pgfpathcurveto{\pgfqpoint{1.583650in}{2.049446in}}{\pgfqpoint{1.586923in}{2.057346in}}{\pgfqpoint{1.586923in}{2.065582in}}%
\pgfpathcurveto{\pgfqpoint{1.586923in}{2.073818in}}{\pgfqpoint{1.583650in}{2.081718in}}{\pgfqpoint{1.577826in}{2.087542in}}%
\pgfpathcurveto{\pgfqpoint{1.572002in}{2.093366in}}{\pgfqpoint{1.564102in}{2.096638in}}{\pgfqpoint{1.555866in}{2.096638in}}%
\pgfpathcurveto{\pgfqpoint{1.547630in}{2.096638in}}{\pgfqpoint{1.539730in}{2.093366in}}{\pgfqpoint{1.533906in}{2.087542in}}%
\pgfpathcurveto{\pgfqpoint{1.528082in}{2.081718in}}{\pgfqpoint{1.524810in}{2.073818in}}{\pgfqpoint{1.524810in}{2.065582in}}%
\pgfpathcurveto{\pgfqpoint{1.524810in}{2.057346in}}{\pgfqpoint{1.528082in}{2.049446in}}{\pgfqpoint{1.533906in}{2.043622in}}%
\pgfpathcurveto{\pgfqpoint{1.539730in}{2.037798in}}{\pgfqpoint{1.547630in}{2.034525in}}{\pgfqpoint{1.555866in}{2.034525in}}%
\pgfpathclose%
\pgfusepath{stroke,fill}%
\end{pgfscope}%
\begin{pgfscope}%
\pgfpathrectangle{\pgfqpoint{0.100000in}{0.212622in}}{\pgfqpoint{3.696000in}{3.696000in}}%
\pgfusepath{clip}%
\pgfsetbuttcap%
\pgfsetroundjoin%
\definecolor{currentfill}{rgb}{0.121569,0.466667,0.705882}%
\pgfsetfillcolor{currentfill}%
\pgfsetfillopacity{0.300615}%
\pgfsetlinewidth{1.003750pt}%
\definecolor{currentstroke}{rgb}{0.121569,0.466667,0.705882}%
\pgfsetstrokecolor{currentstroke}%
\pgfsetstrokeopacity{0.300615}%
\pgfsetdash{}{0pt}%
\pgfpathmoveto{\pgfqpoint{1.555867in}{2.034521in}}%
\pgfpathcurveto{\pgfqpoint{1.564103in}{2.034521in}}{\pgfqpoint{1.572004in}{2.037794in}}{\pgfqpoint{1.577827in}{2.043618in}}%
\pgfpathcurveto{\pgfqpoint{1.583651in}{2.049442in}}{\pgfqpoint{1.586924in}{2.057342in}}{\pgfqpoint{1.586924in}{2.065578in}}%
\pgfpathcurveto{\pgfqpoint{1.586924in}{2.073814in}}{\pgfqpoint{1.583651in}{2.081714in}}{\pgfqpoint{1.577827in}{2.087538in}}%
\pgfpathcurveto{\pgfqpoint{1.572004in}{2.093362in}}{\pgfqpoint{1.564103in}{2.096634in}}{\pgfqpoint{1.555867in}{2.096634in}}%
\pgfpathcurveto{\pgfqpoint{1.547631in}{2.096634in}}{\pgfqpoint{1.539731in}{2.093362in}}{\pgfqpoint{1.533907in}{2.087538in}}%
\pgfpathcurveto{\pgfqpoint{1.528083in}{2.081714in}}{\pgfqpoint{1.524811in}{2.073814in}}{\pgfqpoint{1.524811in}{2.065578in}}%
\pgfpathcurveto{\pgfqpoint{1.524811in}{2.057342in}}{\pgfqpoint{1.528083in}{2.049442in}}{\pgfqpoint{1.533907in}{2.043618in}}%
\pgfpathcurveto{\pgfqpoint{1.539731in}{2.037794in}}{\pgfqpoint{1.547631in}{2.034521in}}{\pgfqpoint{1.555867in}{2.034521in}}%
\pgfpathclose%
\pgfusepath{stroke,fill}%
\end{pgfscope}%
\begin{pgfscope}%
\pgfpathrectangle{\pgfqpoint{0.100000in}{0.212622in}}{\pgfqpoint{3.696000in}{3.696000in}}%
\pgfusepath{clip}%
\pgfsetbuttcap%
\pgfsetroundjoin%
\definecolor{currentfill}{rgb}{0.121569,0.466667,0.705882}%
\pgfsetfillcolor{currentfill}%
\pgfsetfillopacity{0.300623}%
\pgfsetlinewidth{1.003750pt}%
\definecolor{currentstroke}{rgb}{0.121569,0.466667,0.705882}%
\pgfsetstrokecolor{currentstroke}%
\pgfsetstrokeopacity{0.300623}%
\pgfsetdash{}{0pt}%
\pgfpathmoveto{\pgfqpoint{1.555822in}{2.034492in}}%
\pgfpathcurveto{\pgfqpoint{1.564059in}{2.034492in}}{\pgfqpoint{1.571959in}{2.037764in}}{\pgfqpoint{1.577783in}{2.043588in}}%
\pgfpathcurveto{\pgfqpoint{1.583607in}{2.049412in}}{\pgfqpoint{1.586879in}{2.057312in}}{\pgfqpoint{1.586879in}{2.065548in}}%
\pgfpathcurveto{\pgfqpoint{1.586879in}{2.073784in}}{\pgfqpoint{1.583607in}{2.081684in}}{\pgfqpoint{1.577783in}{2.087508in}}%
\pgfpathcurveto{\pgfqpoint{1.571959in}{2.093332in}}{\pgfqpoint{1.564059in}{2.096605in}}{\pgfqpoint{1.555822in}{2.096605in}}%
\pgfpathcurveto{\pgfqpoint{1.547586in}{2.096605in}}{\pgfqpoint{1.539686in}{2.093332in}}{\pgfqpoint{1.533862in}{2.087508in}}%
\pgfpathcurveto{\pgfqpoint{1.528038in}{2.081684in}}{\pgfqpoint{1.524766in}{2.073784in}}{\pgfqpoint{1.524766in}{2.065548in}}%
\pgfpathcurveto{\pgfqpoint{1.524766in}{2.057312in}}{\pgfqpoint{1.528038in}{2.049412in}}{\pgfqpoint{1.533862in}{2.043588in}}%
\pgfpathcurveto{\pgfqpoint{1.539686in}{2.037764in}}{\pgfqpoint{1.547586in}{2.034492in}}{\pgfqpoint{1.555822in}{2.034492in}}%
\pgfpathclose%
\pgfusepath{stroke,fill}%
\end{pgfscope}%
\begin{pgfscope}%
\pgfpathrectangle{\pgfqpoint{0.100000in}{0.212622in}}{\pgfqpoint{3.696000in}{3.696000in}}%
\pgfusepath{clip}%
\pgfsetbuttcap%
\pgfsetroundjoin%
\definecolor{currentfill}{rgb}{0.121569,0.466667,0.705882}%
\pgfsetfillcolor{currentfill}%
\pgfsetfillopacity{0.300633}%
\pgfsetlinewidth{1.003750pt}%
\definecolor{currentstroke}{rgb}{0.121569,0.466667,0.705882}%
\pgfsetstrokecolor{currentstroke}%
\pgfsetstrokeopacity{0.300633}%
\pgfsetdash{}{0pt}%
\pgfpathmoveto{\pgfqpoint{1.555767in}{2.034451in}}%
\pgfpathcurveto{\pgfqpoint{1.564003in}{2.034451in}}{\pgfqpoint{1.571904in}{2.037723in}}{\pgfqpoint{1.577727in}{2.043547in}}%
\pgfpathcurveto{\pgfqpoint{1.583551in}{2.049371in}}{\pgfqpoint{1.586824in}{2.057271in}}{\pgfqpoint{1.586824in}{2.065507in}}%
\pgfpathcurveto{\pgfqpoint{1.586824in}{2.073743in}}{\pgfqpoint{1.583551in}{2.081643in}}{\pgfqpoint{1.577727in}{2.087467in}}%
\pgfpathcurveto{\pgfqpoint{1.571904in}{2.093291in}}{\pgfqpoint{1.564003in}{2.096564in}}{\pgfqpoint{1.555767in}{2.096564in}}%
\pgfpathcurveto{\pgfqpoint{1.547531in}{2.096564in}}{\pgfqpoint{1.539631in}{2.093291in}}{\pgfqpoint{1.533807in}{2.087467in}}%
\pgfpathcurveto{\pgfqpoint{1.527983in}{2.081643in}}{\pgfqpoint{1.524711in}{2.073743in}}{\pgfqpoint{1.524711in}{2.065507in}}%
\pgfpathcurveto{\pgfqpoint{1.524711in}{2.057271in}}{\pgfqpoint{1.527983in}{2.049371in}}{\pgfqpoint{1.533807in}{2.043547in}}%
\pgfpathcurveto{\pgfqpoint{1.539631in}{2.037723in}}{\pgfqpoint{1.547531in}{2.034451in}}{\pgfqpoint{1.555767in}{2.034451in}}%
\pgfpathclose%
\pgfusepath{stroke,fill}%
\end{pgfscope}%
\begin{pgfscope}%
\pgfpathrectangle{\pgfqpoint{0.100000in}{0.212622in}}{\pgfqpoint{3.696000in}{3.696000in}}%
\pgfusepath{clip}%
\pgfsetbuttcap%
\pgfsetroundjoin%
\definecolor{currentfill}{rgb}{0.121569,0.466667,0.705882}%
\pgfsetfillcolor{currentfill}%
\pgfsetfillopacity{0.300642}%
\pgfsetlinewidth{1.003750pt}%
\definecolor{currentstroke}{rgb}{0.121569,0.466667,0.705882}%
\pgfsetstrokecolor{currentstroke}%
\pgfsetstrokeopacity{0.300642}%
\pgfsetdash{}{0pt}%
\pgfpathmoveto{\pgfqpoint{1.555699in}{2.034411in}}%
\pgfpathcurveto{\pgfqpoint{1.563935in}{2.034411in}}{\pgfqpoint{1.571835in}{2.037684in}}{\pgfqpoint{1.577659in}{2.043507in}}%
\pgfpathcurveto{\pgfqpoint{1.583483in}{2.049331in}}{\pgfqpoint{1.586755in}{2.057231in}}{\pgfqpoint{1.586755in}{2.065468in}}%
\pgfpathcurveto{\pgfqpoint{1.586755in}{2.073704in}}{\pgfqpoint{1.583483in}{2.081604in}}{\pgfqpoint{1.577659in}{2.087428in}}%
\pgfpathcurveto{\pgfqpoint{1.571835in}{2.093252in}}{\pgfqpoint{1.563935in}{2.096524in}}{\pgfqpoint{1.555699in}{2.096524in}}%
\pgfpathcurveto{\pgfqpoint{1.547463in}{2.096524in}}{\pgfqpoint{1.539562in}{2.093252in}}{\pgfqpoint{1.533739in}{2.087428in}}%
\pgfpathcurveto{\pgfqpoint{1.527915in}{2.081604in}}{\pgfqpoint{1.524642in}{2.073704in}}{\pgfqpoint{1.524642in}{2.065468in}}%
\pgfpathcurveto{\pgfqpoint{1.524642in}{2.057231in}}{\pgfqpoint{1.527915in}{2.049331in}}{\pgfqpoint{1.533739in}{2.043507in}}%
\pgfpathcurveto{\pgfqpoint{1.539562in}{2.037684in}}{\pgfqpoint{1.547463in}{2.034411in}}{\pgfqpoint{1.555699in}{2.034411in}}%
\pgfpathclose%
\pgfusepath{stroke,fill}%
\end{pgfscope}%
\begin{pgfscope}%
\pgfpathrectangle{\pgfqpoint{0.100000in}{0.212622in}}{\pgfqpoint{3.696000in}{3.696000in}}%
\pgfusepath{clip}%
\pgfsetbuttcap%
\pgfsetroundjoin%
\definecolor{currentfill}{rgb}{0.121569,0.466667,0.705882}%
\pgfsetfillcolor{currentfill}%
\pgfsetfillopacity{0.300675}%
\pgfsetlinewidth{1.003750pt}%
\definecolor{currentstroke}{rgb}{0.121569,0.466667,0.705882}%
\pgfsetstrokecolor{currentstroke}%
\pgfsetstrokeopacity{0.300675}%
\pgfsetdash{}{0pt}%
\pgfpathmoveto{\pgfqpoint{1.555517in}{2.034276in}}%
\pgfpathcurveto{\pgfqpoint{1.563754in}{2.034276in}}{\pgfqpoint{1.571654in}{2.037549in}}{\pgfqpoint{1.577478in}{2.043373in}}%
\pgfpathcurveto{\pgfqpoint{1.583302in}{2.049197in}}{\pgfqpoint{1.586574in}{2.057097in}}{\pgfqpoint{1.586574in}{2.065333in}}%
\pgfpathcurveto{\pgfqpoint{1.586574in}{2.073569in}}{\pgfqpoint{1.583302in}{2.081469in}}{\pgfqpoint{1.577478in}{2.087293in}}%
\pgfpathcurveto{\pgfqpoint{1.571654in}{2.093117in}}{\pgfqpoint{1.563754in}{2.096389in}}{\pgfqpoint{1.555517in}{2.096389in}}%
\pgfpathcurveto{\pgfqpoint{1.547281in}{2.096389in}}{\pgfqpoint{1.539381in}{2.093117in}}{\pgfqpoint{1.533557in}{2.087293in}}%
\pgfpathcurveto{\pgfqpoint{1.527733in}{2.081469in}}{\pgfqpoint{1.524461in}{2.073569in}}{\pgfqpoint{1.524461in}{2.065333in}}%
\pgfpathcurveto{\pgfqpoint{1.524461in}{2.057097in}}{\pgfqpoint{1.527733in}{2.049197in}}{\pgfqpoint{1.533557in}{2.043373in}}%
\pgfpathcurveto{\pgfqpoint{1.539381in}{2.037549in}}{\pgfqpoint{1.547281in}{2.034276in}}{\pgfqpoint{1.555517in}{2.034276in}}%
\pgfpathclose%
\pgfusepath{stroke,fill}%
\end{pgfscope}%
\begin{pgfscope}%
\pgfpathrectangle{\pgfqpoint{0.100000in}{0.212622in}}{\pgfqpoint{3.696000in}{3.696000in}}%
\pgfusepath{clip}%
\pgfsetbuttcap%
\pgfsetroundjoin%
\definecolor{currentfill}{rgb}{0.121569,0.466667,0.705882}%
\pgfsetfillcolor{currentfill}%
\pgfsetfillopacity{0.300685}%
\pgfsetlinewidth{1.003750pt}%
\definecolor{currentstroke}{rgb}{0.121569,0.466667,0.705882}%
\pgfsetstrokecolor{currentstroke}%
\pgfsetstrokeopacity{0.300685}%
\pgfsetdash{}{0pt}%
\pgfpathmoveto{\pgfqpoint{1.555349in}{2.034168in}}%
\pgfpathcurveto{\pgfqpoint{1.563586in}{2.034168in}}{\pgfqpoint{1.571486in}{2.037440in}}{\pgfqpoint{1.577310in}{2.043264in}}%
\pgfpathcurveto{\pgfqpoint{1.583133in}{2.049088in}}{\pgfqpoint{1.586406in}{2.056988in}}{\pgfqpoint{1.586406in}{2.065224in}}%
\pgfpathcurveto{\pgfqpoint{1.586406in}{2.073461in}}{\pgfqpoint{1.583133in}{2.081361in}}{\pgfqpoint{1.577310in}{2.087185in}}%
\pgfpathcurveto{\pgfqpoint{1.571486in}{2.093009in}}{\pgfqpoint{1.563586in}{2.096281in}}{\pgfqpoint{1.555349in}{2.096281in}}%
\pgfpathcurveto{\pgfqpoint{1.547113in}{2.096281in}}{\pgfqpoint{1.539213in}{2.093009in}}{\pgfqpoint{1.533389in}{2.087185in}}%
\pgfpathcurveto{\pgfqpoint{1.527565in}{2.081361in}}{\pgfqpoint{1.524293in}{2.073461in}}{\pgfqpoint{1.524293in}{2.065224in}}%
\pgfpathcurveto{\pgfqpoint{1.524293in}{2.056988in}}{\pgfqpoint{1.527565in}{2.049088in}}{\pgfqpoint{1.533389in}{2.043264in}}%
\pgfpathcurveto{\pgfqpoint{1.539213in}{2.037440in}}{\pgfqpoint{1.547113in}{2.034168in}}{\pgfqpoint{1.555349in}{2.034168in}}%
\pgfpathclose%
\pgfusepath{stroke,fill}%
\end{pgfscope}%
\begin{pgfscope}%
\pgfpathrectangle{\pgfqpoint{0.100000in}{0.212622in}}{\pgfqpoint{3.696000in}{3.696000in}}%
\pgfusepath{clip}%
\pgfsetbuttcap%
\pgfsetroundjoin%
\definecolor{currentfill}{rgb}{0.121569,0.466667,0.705882}%
\pgfsetfillcolor{currentfill}%
\pgfsetfillopacity{0.300749}%
\pgfsetlinewidth{1.003750pt}%
\definecolor{currentstroke}{rgb}{0.121569,0.466667,0.705882}%
\pgfsetstrokecolor{currentstroke}%
\pgfsetstrokeopacity{0.300749}%
\pgfsetdash{}{0pt}%
\pgfpathmoveto{\pgfqpoint{1.554895in}{2.033844in}}%
\pgfpathcurveto{\pgfqpoint{1.563131in}{2.033844in}}{\pgfqpoint{1.571031in}{2.037116in}}{\pgfqpoint{1.576855in}{2.042940in}}%
\pgfpathcurveto{\pgfqpoint{1.582679in}{2.048764in}}{\pgfqpoint{1.585952in}{2.056664in}}{\pgfqpoint{1.585952in}{2.064901in}}%
\pgfpathcurveto{\pgfqpoint{1.585952in}{2.073137in}}{\pgfqpoint{1.582679in}{2.081037in}}{\pgfqpoint{1.576855in}{2.086861in}}%
\pgfpathcurveto{\pgfqpoint{1.571031in}{2.092685in}}{\pgfqpoint{1.563131in}{2.095957in}}{\pgfqpoint{1.554895in}{2.095957in}}%
\pgfpathcurveto{\pgfqpoint{1.546659in}{2.095957in}}{\pgfqpoint{1.538759in}{2.092685in}}{\pgfqpoint{1.532935in}{2.086861in}}%
\pgfpathcurveto{\pgfqpoint{1.527111in}{2.081037in}}{\pgfqpoint{1.523839in}{2.073137in}}{\pgfqpoint{1.523839in}{2.064901in}}%
\pgfpathcurveto{\pgfqpoint{1.523839in}{2.056664in}}{\pgfqpoint{1.527111in}{2.048764in}}{\pgfqpoint{1.532935in}{2.042940in}}%
\pgfpathcurveto{\pgfqpoint{1.538759in}{2.037116in}}{\pgfqpoint{1.546659in}{2.033844in}}{\pgfqpoint{1.554895in}{2.033844in}}%
\pgfpathclose%
\pgfusepath{stroke,fill}%
\end{pgfscope}%
\begin{pgfscope}%
\pgfpathrectangle{\pgfqpoint{0.100000in}{0.212622in}}{\pgfqpoint{3.696000in}{3.696000in}}%
\pgfusepath{clip}%
\pgfsetbuttcap%
\pgfsetroundjoin%
\definecolor{currentfill}{rgb}{0.121569,0.466667,0.705882}%
\pgfsetfillcolor{currentfill}%
\pgfsetfillopacity{0.300817}%
\pgfsetlinewidth{1.003750pt}%
\definecolor{currentstroke}{rgb}{0.121569,0.466667,0.705882}%
\pgfsetstrokecolor{currentstroke}%
\pgfsetstrokeopacity{0.300817}%
\pgfsetdash{}{0pt}%
\pgfpathmoveto{\pgfqpoint{1.554271in}{2.033541in}}%
\pgfpathcurveto{\pgfqpoint{1.562508in}{2.033541in}}{\pgfqpoint{1.570408in}{2.036814in}}{\pgfqpoint{1.576232in}{2.042638in}}%
\pgfpathcurveto{\pgfqpoint{1.582056in}{2.048462in}}{\pgfqpoint{1.585328in}{2.056362in}}{\pgfqpoint{1.585328in}{2.064598in}}%
\pgfpathcurveto{\pgfqpoint{1.585328in}{2.072834in}}{\pgfqpoint{1.582056in}{2.080734in}}{\pgfqpoint{1.576232in}{2.086558in}}%
\pgfpathcurveto{\pgfqpoint{1.570408in}{2.092382in}}{\pgfqpoint{1.562508in}{2.095654in}}{\pgfqpoint{1.554271in}{2.095654in}}%
\pgfpathcurveto{\pgfqpoint{1.546035in}{2.095654in}}{\pgfqpoint{1.538135in}{2.092382in}}{\pgfqpoint{1.532311in}{2.086558in}}%
\pgfpathcurveto{\pgfqpoint{1.526487in}{2.080734in}}{\pgfqpoint{1.523215in}{2.072834in}}{\pgfqpoint{1.523215in}{2.064598in}}%
\pgfpathcurveto{\pgfqpoint{1.523215in}{2.056362in}}{\pgfqpoint{1.526487in}{2.048462in}}{\pgfqpoint{1.532311in}{2.042638in}}%
\pgfpathcurveto{\pgfqpoint{1.538135in}{2.036814in}}{\pgfqpoint{1.546035in}{2.033541in}}{\pgfqpoint{1.554271in}{2.033541in}}%
\pgfpathclose%
\pgfusepath{stroke,fill}%
\end{pgfscope}%
\begin{pgfscope}%
\pgfpathrectangle{\pgfqpoint{0.100000in}{0.212622in}}{\pgfqpoint{3.696000in}{3.696000in}}%
\pgfusepath{clip}%
\pgfsetbuttcap%
\pgfsetroundjoin%
\definecolor{currentfill}{rgb}{0.121569,0.466667,0.705882}%
\pgfsetfillcolor{currentfill}%
\pgfsetfillopacity{0.300913}%
\pgfsetlinewidth{1.003750pt}%
\definecolor{currentstroke}{rgb}{0.121569,0.466667,0.705882}%
\pgfsetstrokecolor{currentstroke}%
\pgfsetstrokeopacity{0.300913}%
\pgfsetdash{}{0pt}%
\pgfpathmoveto{\pgfqpoint{1.553148in}{2.032799in}}%
\pgfpathcurveto{\pgfqpoint{1.561385in}{2.032799in}}{\pgfqpoint{1.569285in}{2.036071in}}{\pgfqpoint{1.575109in}{2.041895in}}%
\pgfpathcurveto{\pgfqpoint{1.580933in}{2.047719in}}{\pgfqpoint{1.584205in}{2.055619in}}{\pgfqpoint{1.584205in}{2.063855in}}%
\pgfpathcurveto{\pgfqpoint{1.584205in}{2.072092in}}{\pgfqpoint{1.580933in}{2.079992in}}{\pgfqpoint{1.575109in}{2.085816in}}%
\pgfpathcurveto{\pgfqpoint{1.569285in}{2.091640in}}{\pgfqpoint{1.561385in}{2.094912in}}{\pgfqpoint{1.553148in}{2.094912in}}%
\pgfpathcurveto{\pgfqpoint{1.544912in}{2.094912in}}{\pgfqpoint{1.537012in}{2.091640in}}{\pgfqpoint{1.531188in}{2.085816in}}%
\pgfpathcurveto{\pgfqpoint{1.525364in}{2.079992in}}{\pgfqpoint{1.522092in}{2.072092in}}{\pgfqpoint{1.522092in}{2.063855in}}%
\pgfpathcurveto{\pgfqpoint{1.522092in}{2.055619in}}{\pgfqpoint{1.525364in}{2.047719in}}{\pgfqpoint{1.531188in}{2.041895in}}%
\pgfpathcurveto{\pgfqpoint{1.537012in}{2.036071in}}{\pgfqpoint{1.544912in}{2.032799in}}{\pgfqpoint{1.553148in}{2.032799in}}%
\pgfpathclose%
\pgfusepath{stroke,fill}%
\end{pgfscope}%
\begin{pgfscope}%
\pgfpathrectangle{\pgfqpoint{0.100000in}{0.212622in}}{\pgfqpoint{3.696000in}{3.696000in}}%
\pgfusepath{clip}%
\pgfsetbuttcap%
\pgfsetroundjoin%
\definecolor{currentfill}{rgb}{0.121569,0.466667,0.705882}%
\pgfsetfillcolor{currentfill}%
\pgfsetfillopacity{0.300988}%
\pgfsetlinewidth{1.003750pt}%
\definecolor{currentstroke}{rgb}{0.121569,0.466667,0.705882}%
\pgfsetstrokecolor{currentstroke}%
\pgfsetstrokeopacity{0.300988}%
\pgfsetdash{}{0pt}%
\pgfpathmoveto{\pgfqpoint{1.552553in}{2.032657in}}%
\pgfpathcurveto{\pgfqpoint{1.560789in}{2.032657in}}{\pgfqpoint{1.568689in}{2.035929in}}{\pgfqpoint{1.574513in}{2.041753in}}%
\pgfpathcurveto{\pgfqpoint{1.580337in}{2.047577in}}{\pgfqpoint{1.583610in}{2.055477in}}{\pgfqpoint{1.583610in}{2.063713in}}%
\pgfpathcurveto{\pgfqpoint{1.583610in}{2.071949in}}{\pgfqpoint{1.580337in}{2.079849in}}{\pgfqpoint{1.574513in}{2.085673in}}%
\pgfpathcurveto{\pgfqpoint{1.568689in}{2.091497in}}{\pgfqpoint{1.560789in}{2.094770in}}{\pgfqpoint{1.552553in}{2.094770in}}%
\pgfpathcurveto{\pgfqpoint{1.544317in}{2.094770in}}{\pgfqpoint{1.536417in}{2.091497in}}{\pgfqpoint{1.530593in}{2.085673in}}%
\pgfpathcurveto{\pgfqpoint{1.524769in}{2.079849in}}{\pgfqpoint{1.521497in}{2.071949in}}{\pgfqpoint{1.521497in}{2.063713in}}%
\pgfpathcurveto{\pgfqpoint{1.521497in}{2.055477in}}{\pgfqpoint{1.524769in}{2.047577in}}{\pgfqpoint{1.530593in}{2.041753in}}%
\pgfpathcurveto{\pgfqpoint{1.536417in}{2.035929in}}{\pgfqpoint{1.544317in}{2.032657in}}{\pgfqpoint{1.552553in}{2.032657in}}%
\pgfpathclose%
\pgfusepath{stroke,fill}%
\end{pgfscope}%
\begin{pgfscope}%
\pgfpathrectangle{\pgfqpoint{0.100000in}{0.212622in}}{\pgfqpoint{3.696000in}{3.696000in}}%
\pgfusepath{clip}%
\pgfsetbuttcap%
\pgfsetroundjoin%
\definecolor{currentfill}{rgb}{0.121569,0.466667,0.705882}%
\pgfsetfillcolor{currentfill}%
\pgfsetfillopacity{0.301000}%
\pgfsetlinewidth{1.003750pt}%
\definecolor{currentstroke}{rgb}{0.121569,0.466667,0.705882}%
\pgfsetstrokecolor{currentstroke}%
\pgfsetstrokeopacity{0.301000}%
\pgfsetdash{}{0pt}%
\pgfpathmoveto{\pgfqpoint{1.552517in}{2.032625in}}%
\pgfpathcurveto{\pgfqpoint{1.560753in}{2.032625in}}{\pgfqpoint{1.568653in}{2.035898in}}{\pgfqpoint{1.574477in}{2.041722in}}%
\pgfpathcurveto{\pgfqpoint{1.580301in}{2.047546in}}{\pgfqpoint{1.583573in}{2.055446in}}{\pgfqpoint{1.583573in}{2.063682in}}%
\pgfpathcurveto{\pgfqpoint{1.583573in}{2.071918in}}{\pgfqpoint{1.580301in}{2.079818in}}{\pgfqpoint{1.574477in}{2.085642in}}%
\pgfpathcurveto{\pgfqpoint{1.568653in}{2.091466in}}{\pgfqpoint{1.560753in}{2.094738in}}{\pgfqpoint{1.552517in}{2.094738in}}%
\pgfpathcurveto{\pgfqpoint{1.544281in}{2.094738in}}{\pgfqpoint{1.536381in}{2.091466in}}{\pgfqpoint{1.530557in}{2.085642in}}%
\pgfpathcurveto{\pgfqpoint{1.524733in}{2.079818in}}{\pgfqpoint{1.521460in}{2.071918in}}{\pgfqpoint{1.521460in}{2.063682in}}%
\pgfpathcurveto{\pgfqpoint{1.521460in}{2.055446in}}{\pgfqpoint{1.524733in}{2.047546in}}{\pgfqpoint{1.530557in}{2.041722in}}%
\pgfpathcurveto{\pgfqpoint{1.536381in}{2.035898in}}{\pgfqpoint{1.544281in}{2.032625in}}{\pgfqpoint{1.552517in}{2.032625in}}%
\pgfpathclose%
\pgfusepath{stroke,fill}%
\end{pgfscope}%
\begin{pgfscope}%
\pgfpathrectangle{\pgfqpoint{0.100000in}{0.212622in}}{\pgfqpoint{3.696000in}{3.696000in}}%
\pgfusepath{clip}%
\pgfsetbuttcap%
\pgfsetroundjoin%
\definecolor{currentfill}{rgb}{0.121569,0.466667,0.705882}%
\pgfsetfillcolor{currentfill}%
\pgfsetfillopacity{0.301001}%
\pgfsetlinewidth{1.003750pt}%
\definecolor{currentstroke}{rgb}{0.121569,0.466667,0.705882}%
\pgfsetstrokecolor{currentstroke}%
\pgfsetstrokeopacity{0.301001}%
\pgfsetdash{}{0pt}%
\pgfpathmoveto{\pgfqpoint{1.552517in}{2.032627in}}%
\pgfpathcurveto{\pgfqpoint{1.560753in}{2.032627in}}{\pgfqpoint{1.568653in}{2.035899in}}{\pgfqpoint{1.574477in}{2.041723in}}%
\pgfpathcurveto{\pgfqpoint{1.580301in}{2.047547in}}{\pgfqpoint{1.583574in}{2.055447in}}{\pgfqpoint{1.583574in}{2.063684in}}%
\pgfpathcurveto{\pgfqpoint{1.583574in}{2.071920in}}{\pgfqpoint{1.580301in}{2.079820in}}{\pgfqpoint{1.574477in}{2.085644in}}%
\pgfpathcurveto{\pgfqpoint{1.568653in}{2.091468in}}{\pgfqpoint{1.560753in}{2.094740in}}{\pgfqpoint{1.552517in}{2.094740in}}%
\pgfpathcurveto{\pgfqpoint{1.544281in}{2.094740in}}{\pgfqpoint{1.536381in}{2.091468in}}{\pgfqpoint{1.530557in}{2.085644in}}%
\pgfpathcurveto{\pgfqpoint{1.524733in}{2.079820in}}{\pgfqpoint{1.521461in}{2.071920in}}{\pgfqpoint{1.521461in}{2.063684in}}%
\pgfpathcurveto{\pgfqpoint{1.521461in}{2.055447in}}{\pgfqpoint{1.524733in}{2.047547in}}{\pgfqpoint{1.530557in}{2.041723in}}%
\pgfpathcurveto{\pgfqpoint{1.536381in}{2.035899in}}{\pgfqpoint{1.544281in}{2.032627in}}{\pgfqpoint{1.552517in}{2.032627in}}%
\pgfpathclose%
\pgfusepath{stroke,fill}%
\end{pgfscope}%
\begin{pgfscope}%
\pgfpathrectangle{\pgfqpoint{0.100000in}{0.212622in}}{\pgfqpoint{3.696000in}{3.696000in}}%
\pgfusepath{clip}%
\pgfsetbuttcap%
\pgfsetroundjoin%
\definecolor{currentfill}{rgb}{0.121569,0.466667,0.705882}%
\pgfsetfillcolor{currentfill}%
\pgfsetfillopacity{0.301001}%
\pgfsetlinewidth{1.003750pt}%
\definecolor{currentstroke}{rgb}{0.121569,0.466667,0.705882}%
\pgfsetstrokecolor{currentstroke}%
\pgfsetstrokeopacity{0.301001}%
\pgfsetdash{}{0pt}%
\pgfpathmoveto{\pgfqpoint{1.552517in}{2.032626in}}%
\pgfpathcurveto{\pgfqpoint{1.560753in}{2.032626in}}{\pgfqpoint{1.568653in}{2.035899in}}{\pgfqpoint{1.574477in}{2.041722in}}%
\pgfpathcurveto{\pgfqpoint{1.580301in}{2.047546in}}{\pgfqpoint{1.583573in}{2.055446in}}{\pgfqpoint{1.583573in}{2.063683in}}%
\pgfpathcurveto{\pgfqpoint{1.583573in}{2.071919in}}{\pgfqpoint{1.580301in}{2.079819in}}{\pgfqpoint{1.574477in}{2.085643in}}%
\pgfpathcurveto{\pgfqpoint{1.568653in}{2.091467in}}{\pgfqpoint{1.560753in}{2.094739in}}{\pgfqpoint{1.552517in}{2.094739in}}%
\pgfpathcurveto{\pgfqpoint{1.544281in}{2.094739in}}{\pgfqpoint{1.536381in}{2.091467in}}{\pgfqpoint{1.530557in}{2.085643in}}%
\pgfpathcurveto{\pgfqpoint{1.524733in}{2.079819in}}{\pgfqpoint{1.521460in}{2.071919in}}{\pgfqpoint{1.521460in}{2.063683in}}%
\pgfpathcurveto{\pgfqpoint{1.521460in}{2.055446in}}{\pgfqpoint{1.524733in}{2.047546in}}{\pgfqpoint{1.530557in}{2.041722in}}%
\pgfpathcurveto{\pgfqpoint{1.536381in}{2.035899in}}{\pgfqpoint{1.544281in}{2.032626in}}{\pgfqpoint{1.552517in}{2.032626in}}%
\pgfpathclose%
\pgfusepath{stroke,fill}%
\end{pgfscope}%
\begin{pgfscope}%
\pgfpathrectangle{\pgfqpoint{0.100000in}{0.212622in}}{\pgfqpoint{3.696000in}{3.696000in}}%
\pgfusepath{clip}%
\pgfsetbuttcap%
\pgfsetroundjoin%
\definecolor{currentfill}{rgb}{0.121569,0.466667,0.705882}%
\pgfsetfillcolor{currentfill}%
\pgfsetfillopacity{0.301001}%
\pgfsetlinewidth{1.003750pt}%
\definecolor{currentstroke}{rgb}{0.121569,0.466667,0.705882}%
\pgfsetstrokecolor{currentstroke}%
\pgfsetstrokeopacity{0.301001}%
\pgfsetdash{}{0pt}%
\pgfpathmoveto{\pgfqpoint{1.552516in}{2.032626in}}%
\pgfpathcurveto{\pgfqpoint{1.560753in}{2.032626in}}{\pgfqpoint{1.568653in}{2.035898in}}{\pgfqpoint{1.574477in}{2.041722in}}%
\pgfpathcurveto{\pgfqpoint{1.580301in}{2.047546in}}{\pgfqpoint{1.583573in}{2.055446in}}{\pgfqpoint{1.583573in}{2.063682in}}%
\pgfpathcurveto{\pgfqpoint{1.583573in}{2.071919in}}{\pgfqpoint{1.580301in}{2.079819in}}{\pgfqpoint{1.574477in}{2.085643in}}%
\pgfpathcurveto{\pgfqpoint{1.568653in}{2.091467in}}{\pgfqpoint{1.560753in}{2.094739in}}{\pgfqpoint{1.552516in}{2.094739in}}%
\pgfpathcurveto{\pgfqpoint{1.544280in}{2.094739in}}{\pgfqpoint{1.536380in}{2.091467in}}{\pgfqpoint{1.530556in}{2.085643in}}%
\pgfpathcurveto{\pgfqpoint{1.524732in}{2.079819in}}{\pgfqpoint{1.521460in}{2.071919in}}{\pgfqpoint{1.521460in}{2.063682in}}%
\pgfpathcurveto{\pgfqpoint{1.521460in}{2.055446in}}{\pgfqpoint{1.524732in}{2.047546in}}{\pgfqpoint{1.530556in}{2.041722in}}%
\pgfpathcurveto{\pgfqpoint{1.536380in}{2.035898in}}{\pgfqpoint{1.544280in}{2.032626in}}{\pgfqpoint{1.552516in}{2.032626in}}%
\pgfpathclose%
\pgfusepath{stroke,fill}%
\end{pgfscope}%
\begin{pgfscope}%
\pgfpathrectangle{\pgfqpoint{0.100000in}{0.212622in}}{\pgfqpoint{3.696000in}{3.696000in}}%
\pgfusepath{clip}%
\pgfsetbuttcap%
\pgfsetroundjoin%
\definecolor{currentfill}{rgb}{0.121569,0.466667,0.705882}%
\pgfsetfillcolor{currentfill}%
\pgfsetfillopacity{0.301002}%
\pgfsetlinewidth{1.003750pt}%
\definecolor{currentstroke}{rgb}{0.121569,0.466667,0.705882}%
\pgfsetstrokecolor{currentstroke}%
\pgfsetstrokeopacity{0.301002}%
\pgfsetdash{}{0pt}%
\pgfpathmoveto{\pgfqpoint{1.552516in}{2.032626in}}%
\pgfpathcurveto{\pgfqpoint{1.560753in}{2.032626in}}{\pgfqpoint{1.568653in}{2.035898in}}{\pgfqpoint{1.574477in}{2.041722in}}%
\pgfpathcurveto{\pgfqpoint{1.580300in}{2.047546in}}{\pgfqpoint{1.583573in}{2.055446in}}{\pgfqpoint{1.583573in}{2.063682in}}%
\pgfpathcurveto{\pgfqpoint{1.583573in}{2.071918in}}{\pgfqpoint{1.580300in}{2.079818in}}{\pgfqpoint{1.574477in}{2.085642in}}%
\pgfpathcurveto{\pgfqpoint{1.568653in}{2.091466in}}{\pgfqpoint{1.560753in}{2.094739in}}{\pgfqpoint{1.552516in}{2.094739in}}%
\pgfpathcurveto{\pgfqpoint{1.544280in}{2.094739in}}{\pgfqpoint{1.536380in}{2.091466in}}{\pgfqpoint{1.530556in}{2.085642in}}%
\pgfpathcurveto{\pgfqpoint{1.524732in}{2.079818in}}{\pgfqpoint{1.521460in}{2.071918in}}{\pgfqpoint{1.521460in}{2.063682in}}%
\pgfpathcurveto{\pgfqpoint{1.521460in}{2.055446in}}{\pgfqpoint{1.524732in}{2.047546in}}{\pgfqpoint{1.530556in}{2.041722in}}%
\pgfpathcurveto{\pgfqpoint{1.536380in}{2.035898in}}{\pgfqpoint{1.544280in}{2.032626in}}{\pgfqpoint{1.552516in}{2.032626in}}%
\pgfpathclose%
\pgfusepath{stroke,fill}%
\end{pgfscope}%
\begin{pgfscope}%
\pgfpathrectangle{\pgfqpoint{0.100000in}{0.212622in}}{\pgfqpoint{3.696000in}{3.696000in}}%
\pgfusepath{clip}%
\pgfsetbuttcap%
\pgfsetroundjoin%
\definecolor{currentfill}{rgb}{0.121569,0.466667,0.705882}%
\pgfsetfillcolor{currentfill}%
\pgfsetfillopacity{0.301002}%
\pgfsetlinewidth{1.003750pt}%
\definecolor{currentstroke}{rgb}{0.121569,0.466667,0.705882}%
\pgfsetstrokecolor{currentstroke}%
\pgfsetstrokeopacity{0.301002}%
\pgfsetdash{}{0pt}%
\pgfpathmoveto{\pgfqpoint{1.552515in}{2.032625in}}%
\pgfpathcurveto{\pgfqpoint{1.560752in}{2.032625in}}{\pgfqpoint{1.568652in}{2.035897in}}{\pgfqpoint{1.574476in}{2.041721in}}%
\pgfpathcurveto{\pgfqpoint{1.580300in}{2.047545in}}{\pgfqpoint{1.583572in}{2.055445in}}{\pgfqpoint{1.583572in}{2.063681in}}%
\pgfpathcurveto{\pgfqpoint{1.583572in}{2.071918in}}{\pgfqpoint{1.580300in}{2.079818in}}{\pgfqpoint{1.574476in}{2.085642in}}%
\pgfpathcurveto{\pgfqpoint{1.568652in}{2.091466in}}{\pgfqpoint{1.560752in}{2.094738in}}{\pgfqpoint{1.552515in}{2.094738in}}%
\pgfpathcurveto{\pgfqpoint{1.544279in}{2.094738in}}{\pgfqpoint{1.536379in}{2.091466in}}{\pgfqpoint{1.530555in}{2.085642in}}%
\pgfpathcurveto{\pgfqpoint{1.524731in}{2.079818in}}{\pgfqpoint{1.521459in}{2.071918in}}{\pgfqpoint{1.521459in}{2.063681in}}%
\pgfpathcurveto{\pgfqpoint{1.521459in}{2.055445in}}{\pgfqpoint{1.524731in}{2.047545in}}{\pgfqpoint{1.530555in}{2.041721in}}%
\pgfpathcurveto{\pgfqpoint{1.536379in}{2.035897in}}{\pgfqpoint{1.544279in}{2.032625in}}{\pgfqpoint{1.552515in}{2.032625in}}%
\pgfpathclose%
\pgfusepath{stroke,fill}%
\end{pgfscope}%
\begin{pgfscope}%
\pgfpathrectangle{\pgfqpoint{0.100000in}{0.212622in}}{\pgfqpoint{3.696000in}{3.696000in}}%
\pgfusepath{clip}%
\pgfsetbuttcap%
\pgfsetroundjoin%
\definecolor{currentfill}{rgb}{0.121569,0.466667,0.705882}%
\pgfsetfillcolor{currentfill}%
\pgfsetfillopacity{0.308555}%
\pgfsetlinewidth{1.003750pt}%
\definecolor{currentstroke}{rgb}{0.121569,0.466667,0.705882}%
\pgfsetstrokecolor{currentstroke}%
\pgfsetstrokeopacity{0.308555}%
\pgfsetdash{}{0pt}%
\pgfpathmoveto{\pgfqpoint{1.535956in}{2.018161in}}%
\pgfpathcurveto{\pgfqpoint{1.544192in}{2.018161in}}{\pgfqpoint{1.552092in}{2.021433in}}{\pgfqpoint{1.557916in}{2.027257in}}%
\pgfpathcurveto{\pgfqpoint{1.563740in}{2.033081in}}{\pgfqpoint{1.567013in}{2.040981in}}{\pgfqpoint{1.567013in}{2.049217in}}%
\pgfpathcurveto{\pgfqpoint{1.567013in}{2.057454in}}{\pgfqpoint{1.563740in}{2.065354in}}{\pgfqpoint{1.557916in}{2.071178in}}%
\pgfpathcurveto{\pgfqpoint{1.552092in}{2.077002in}}{\pgfqpoint{1.544192in}{2.080274in}}{\pgfqpoint{1.535956in}{2.080274in}}%
\pgfpathcurveto{\pgfqpoint{1.527720in}{2.080274in}}{\pgfqpoint{1.519820in}{2.077002in}}{\pgfqpoint{1.513996in}{2.071178in}}%
\pgfpathcurveto{\pgfqpoint{1.508172in}{2.065354in}}{\pgfqpoint{1.504900in}{2.057454in}}{\pgfqpoint{1.504900in}{2.049217in}}%
\pgfpathcurveto{\pgfqpoint{1.504900in}{2.040981in}}{\pgfqpoint{1.508172in}{2.033081in}}{\pgfqpoint{1.513996in}{2.027257in}}%
\pgfpathcurveto{\pgfqpoint{1.519820in}{2.021433in}}{\pgfqpoint{1.527720in}{2.018161in}}{\pgfqpoint{1.535956in}{2.018161in}}%
\pgfpathclose%
\pgfusepath{stroke,fill}%
\end{pgfscope}%
\begin{pgfscope}%
\pgfpathrectangle{\pgfqpoint{0.100000in}{0.212622in}}{\pgfqpoint{3.696000in}{3.696000in}}%
\pgfusepath{clip}%
\pgfsetbuttcap%
\pgfsetroundjoin%
\definecolor{currentfill}{rgb}{0.121569,0.466667,0.705882}%
\pgfsetfillcolor{currentfill}%
\pgfsetfillopacity{0.321725}%
\pgfsetlinewidth{1.003750pt}%
\definecolor{currentstroke}{rgb}{0.121569,0.466667,0.705882}%
\pgfsetstrokecolor{currentstroke}%
\pgfsetstrokeopacity{0.321725}%
\pgfsetdash{}{0pt}%
\pgfpathmoveto{\pgfqpoint{1.497947in}{1.985458in}}%
\pgfpathcurveto{\pgfqpoint{1.506183in}{1.985458in}}{\pgfqpoint{1.514083in}{1.988730in}}{\pgfqpoint{1.519907in}{1.994554in}}%
\pgfpathcurveto{\pgfqpoint{1.525731in}{2.000378in}}{\pgfqpoint{1.529003in}{2.008278in}}{\pgfqpoint{1.529003in}{2.016514in}}%
\pgfpathcurveto{\pgfqpoint{1.529003in}{2.024751in}}{\pgfqpoint{1.525731in}{2.032651in}}{\pgfqpoint{1.519907in}{2.038475in}}%
\pgfpathcurveto{\pgfqpoint{1.514083in}{2.044299in}}{\pgfqpoint{1.506183in}{2.047571in}}{\pgfqpoint{1.497947in}{2.047571in}}%
\pgfpathcurveto{\pgfqpoint{1.489711in}{2.047571in}}{\pgfqpoint{1.481811in}{2.044299in}}{\pgfqpoint{1.475987in}{2.038475in}}%
\pgfpathcurveto{\pgfqpoint{1.470163in}{2.032651in}}{\pgfqpoint{1.466890in}{2.024751in}}{\pgfqpoint{1.466890in}{2.016514in}}%
\pgfpathcurveto{\pgfqpoint{1.466890in}{2.008278in}}{\pgfqpoint{1.470163in}{2.000378in}}{\pgfqpoint{1.475987in}{1.994554in}}%
\pgfpathcurveto{\pgfqpoint{1.481811in}{1.988730in}}{\pgfqpoint{1.489711in}{1.985458in}}{\pgfqpoint{1.497947in}{1.985458in}}%
\pgfpathclose%
\pgfusepath{stroke,fill}%
\end{pgfscope}%
\begin{pgfscope}%
\pgfpathrectangle{\pgfqpoint{0.100000in}{0.212622in}}{\pgfqpoint{3.696000in}{3.696000in}}%
\pgfusepath{clip}%
\pgfsetbuttcap%
\pgfsetroundjoin%
\definecolor{currentfill}{rgb}{0.121569,0.466667,0.705882}%
\pgfsetfillcolor{currentfill}%
\pgfsetfillopacity{0.326363}%
\pgfsetlinewidth{1.003750pt}%
\definecolor{currentstroke}{rgb}{0.121569,0.466667,0.705882}%
\pgfsetstrokecolor{currentstroke}%
\pgfsetstrokeopacity{0.326363}%
\pgfsetdash{}{0pt}%
\pgfpathmoveto{\pgfqpoint{1.485487in}{1.974132in}}%
\pgfpathcurveto{\pgfqpoint{1.493723in}{1.974132in}}{\pgfqpoint{1.501623in}{1.977404in}}{\pgfqpoint{1.507447in}{1.983228in}}%
\pgfpathcurveto{\pgfqpoint{1.513271in}{1.989052in}}{\pgfqpoint{1.516543in}{1.996952in}}{\pgfqpoint{1.516543in}{2.005189in}}%
\pgfpathcurveto{\pgfqpoint{1.516543in}{2.013425in}}{\pgfqpoint{1.513271in}{2.021325in}}{\pgfqpoint{1.507447in}{2.027149in}}%
\pgfpathcurveto{\pgfqpoint{1.501623in}{2.032973in}}{\pgfqpoint{1.493723in}{2.036245in}}{\pgfqpoint{1.485487in}{2.036245in}}%
\pgfpathcurveto{\pgfqpoint{1.477251in}{2.036245in}}{\pgfqpoint{1.469351in}{2.032973in}}{\pgfqpoint{1.463527in}{2.027149in}}%
\pgfpathcurveto{\pgfqpoint{1.457703in}{2.021325in}}{\pgfqpoint{1.454430in}{2.013425in}}{\pgfqpoint{1.454430in}{2.005189in}}%
\pgfpathcurveto{\pgfqpoint{1.454430in}{1.996952in}}{\pgfqpoint{1.457703in}{1.989052in}}{\pgfqpoint{1.463527in}{1.983228in}}%
\pgfpathcurveto{\pgfqpoint{1.469351in}{1.977404in}}{\pgfqpoint{1.477251in}{1.974132in}}{\pgfqpoint{1.485487in}{1.974132in}}%
\pgfpathclose%
\pgfusepath{stroke,fill}%
\end{pgfscope}%
\begin{pgfscope}%
\pgfpathrectangle{\pgfqpoint{0.100000in}{0.212622in}}{\pgfqpoint{3.696000in}{3.696000in}}%
\pgfusepath{clip}%
\pgfsetbuttcap%
\pgfsetroundjoin%
\definecolor{currentfill}{rgb}{0.121569,0.466667,0.705882}%
\pgfsetfillcolor{currentfill}%
\pgfsetfillopacity{0.327185}%
\pgfsetlinewidth{1.003750pt}%
\definecolor{currentstroke}{rgb}{0.121569,0.466667,0.705882}%
\pgfsetstrokecolor{currentstroke}%
\pgfsetstrokeopacity{0.327185}%
\pgfsetdash{}{0pt}%
\pgfpathmoveto{\pgfqpoint{1.486739in}{1.976371in}}%
\pgfpathcurveto{\pgfqpoint{1.494976in}{1.976371in}}{\pgfqpoint{1.502876in}{1.979643in}}{\pgfqpoint{1.508700in}{1.985467in}}%
\pgfpathcurveto{\pgfqpoint{1.514524in}{1.991291in}}{\pgfqpoint{1.517796in}{1.999191in}}{\pgfqpoint{1.517796in}{2.007428in}}%
\pgfpathcurveto{\pgfqpoint{1.517796in}{2.015664in}}{\pgfqpoint{1.514524in}{2.023564in}}{\pgfqpoint{1.508700in}{2.029388in}}%
\pgfpathcurveto{\pgfqpoint{1.502876in}{2.035212in}}{\pgfqpoint{1.494976in}{2.038484in}}{\pgfqpoint{1.486739in}{2.038484in}}%
\pgfpathcurveto{\pgfqpoint{1.478503in}{2.038484in}}{\pgfqpoint{1.470603in}{2.035212in}}{\pgfqpoint{1.464779in}{2.029388in}}%
\pgfpathcurveto{\pgfqpoint{1.458955in}{2.023564in}}{\pgfqpoint{1.455683in}{2.015664in}}{\pgfqpoint{1.455683in}{2.007428in}}%
\pgfpathcurveto{\pgfqpoint{1.455683in}{1.999191in}}{\pgfqpoint{1.458955in}{1.991291in}}{\pgfqpoint{1.464779in}{1.985467in}}%
\pgfpathcurveto{\pgfqpoint{1.470603in}{1.979643in}}{\pgfqpoint{1.478503in}{1.976371in}}{\pgfqpoint{1.486739in}{1.976371in}}%
\pgfpathclose%
\pgfusepath{stroke,fill}%
\end{pgfscope}%
\begin{pgfscope}%
\pgfpathrectangle{\pgfqpoint{0.100000in}{0.212622in}}{\pgfqpoint{3.696000in}{3.696000in}}%
\pgfusepath{clip}%
\pgfsetbuttcap%
\pgfsetroundjoin%
\definecolor{currentfill}{rgb}{0.121569,0.466667,0.705882}%
\pgfsetfillcolor{currentfill}%
\pgfsetfillopacity{0.337411}%
\pgfsetlinewidth{1.003750pt}%
\definecolor{currentstroke}{rgb}{0.121569,0.466667,0.705882}%
\pgfsetstrokecolor{currentstroke}%
\pgfsetstrokeopacity{0.337411}%
\pgfsetdash{}{0pt}%
\pgfpathmoveto{\pgfqpoint{1.457202in}{1.945880in}}%
\pgfpathcurveto{\pgfqpoint{1.465438in}{1.945880in}}{\pgfqpoint{1.473338in}{1.949152in}}{\pgfqpoint{1.479162in}{1.954976in}}%
\pgfpathcurveto{\pgfqpoint{1.484986in}{1.960800in}}{\pgfqpoint{1.488258in}{1.968700in}}{\pgfqpoint{1.488258in}{1.976936in}}%
\pgfpathcurveto{\pgfqpoint{1.488258in}{1.985173in}}{\pgfqpoint{1.484986in}{1.993073in}}{\pgfqpoint{1.479162in}{1.998897in}}%
\pgfpathcurveto{\pgfqpoint{1.473338in}{2.004721in}}{\pgfqpoint{1.465438in}{2.007993in}}{\pgfqpoint{1.457202in}{2.007993in}}%
\pgfpathcurveto{\pgfqpoint{1.448965in}{2.007993in}}{\pgfqpoint{1.441065in}{2.004721in}}{\pgfqpoint{1.435241in}{1.998897in}}%
\pgfpathcurveto{\pgfqpoint{1.429417in}{1.993073in}}{\pgfqpoint{1.426145in}{1.985173in}}{\pgfqpoint{1.426145in}{1.976936in}}%
\pgfpathcurveto{\pgfqpoint{1.426145in}{1.968700in}}{\pgfqpoint{1.429417in}{1.960800in}}{\pgfqpoint{1.435241in}{1.954976in}}%
\pgfpathcurveto{\pgfqpoint{1.441065in}{1.949152in}}{\pgfqpoint{1.448965in}{1.945880in}}{\pgfqpoint{1.457202in}{1.945880in}}%
\pgfpathclose%
\pgfusepath{stroke,fill}%
\end{pgfscope}%
\begin{pgfscope}%
\pgfpathrectangle{\pgfqpoint{0.100000in}{0.212622in}}{\pgfqpoint{3.696000in}{3.696000in}}%
\pgfusepath{clip}%
\pgfsetbuttcap%
\pgfsetroundjoin%
\definecolor{currentfill}{rgb}{0.121569,0.466667,0.705882}%
\pgfsetfillcolor{currentfill}%
\pgfsetfillopacity{0.338667}%
\pgfsetlinewidth{1.003750pt}%
\definecolor{currentstroke}{rgb}{0.121569,0.466667,0.705882}%
\pgfsetstrokecolor{currentstroke}%
\pgfsetstrokeopacity{0.338667}%
\pgfsetdash{}{0pt}%
\pgfpathmoveto{\pgfqpoint{1.455366in}{1.941431in}}%
\pgfpathcurveto{\pgfqpoint{1.463602in}{1.941431in}}{\pgfqpoint{1.471502in}{1.944703in}}{\pgfqpoint{1.477326in}{1.950527in}}%
\pgfpathcurveto{\pgfqpoint{1.483150in}{1.956351in}}{\pgfqpoint{1.486422in}{1.964251in}}{\pgfqpoint{1.486422in}{1.972488in}}%
\pgfpathcurveto{\pgfqpoint{1.486422in}{1.980724in}}{\pgfqpoint{1.483150in}{1.988624in}}{\pgfqpoint{1.477326in}{1.994448in}}%
\pgfpathcurveto{\pgfqpoint{1.471502in}{2.000272in}}{\pgfqpoint{1.463602in}{2.003544in}}{\pgfqpoint{1.455366in}{2.003544in}}%
\pgfpathcurveto{\pgfqpoint{1.447129in}{2.003544in}}{\pgfqpoint{1.439229in}{2.000272in}}{\pgfqpoint{1.433405in}{1.994448in}}%
\pgfpathcurveto{\pgfqpoint{1.427581in}{1.988624in}}{\pgfqpoint{1.424309in}{1.980724in}}{\pgfqpoint{1.424309in}{1.972488in}}%
\pgfpathcurveto{\pgfqpoint{1.424309in}{1.964251in}}{\pgfqpoint{1.427581in}{1.956351in}}{\pgfqpoint{1.433405in}{1.950527in}}%
\pgfpathcurveto{\pgfqpoint{1.439229in}{1.944703in}}{\pgfqpoint{1.447129in}{1.941431in}}{\pgfqpoint{1.455366in}{1.941431in}}%
\pgfpathclose%
\pgfusepath{stroke,fill}%
\end{pgfscope}%
\begin{pgfscope}%
\pgfpathrectangle{\pgfqpoint{0.100000in}{0.212622in}}{\pgfqpoint{3.696000in}{3.696000in}}%
\pgfusepath{clip}%
\pgfsetbuttcap%
\pgfsetroundjoin%
\definecolor{currentfill}{rgb}{0.121569,0.466667,0.705882}%
\pgfsetfillcolor{currentfill}%
\pgfsetfillopacity{0.342340}%
\pgfsetlinewidth{1.003750pt}%
\definecolor{currentstroke}{rgb}{0.121569,0.466667,0.705882}%
\pgfsetstrokecolor{currentstroke}%
\pgfsetstrokeopacity{0.342340}%
\pgfsetdash{}{0pt}%
\pgfpathmoveto{\pgfqpoint{1.448575in}{1.937436in}}%
\pgfpathcurveto{\pgfqpoint{1.456812in}{1.937436in}}{\pgfqpoint{1.464712in}{1.940709in}}{\pgfqpoint{1.470536in}{1.946532in}}%
\pgfpathcurveto{\pgfqpoint{1.476360in}{1.952356in}}{\pgfqpoint{1.479632in}{1.960256in}}{\pgfqpoint{1.479632in}{1.968493in}}%
\pgfpathcurveto{\pgfqpoint{1.479632in}{1.976729in}}{\pgfqpoint{1.476360in}{1.984629in}}{\pgfqpoint{1.470536in}{1.990453in}}%
\pgfpathcurveto{\pgfqpoint{1.464712in}{1.996277in}}{\pgfqpoint{1.456812in}{1.999549in}}{\pgfqpoint{1.448575in}{1.999549in}}%
\pgfpathcurveto{\pgfqpoint{1.440339in}{1.999549in}}{\pgfqpoint{1.432439in}{1.996277in}}{\pgfqpoint{1.426615in}{1.990453in}}%
\pgfpathcurveto{\pgfqpoint{1.420791in}{1.984629in}}{\pgfqpoint{1.417519in}{1.976729in}}{\pgfqpoint{1.417519in}{1.968493in}}%
\pgfpathcurveto{\pgfqpoint{1.417519in}{1.960256in}}{\pgfqpoint{1.420791in}{1.952356in}}{\pgfqpoint{1.426615in}{1.946532in}}%
\pgfpathcurveto{\pgfqpoint{1.432439in}{1.940709in}}{\pgfqpoint{1.440339in}{1.937436in}}{\pgfqpoint{1.448575in}{1.937436in}}%
\pgfpathclose%
\pgfusepath{stroke,fill}%
\end{pgfscope}%
\begin{pgfscope}%
\pgfpathrectangle{\pgfqpoint{0.100000in}{0.212622in}}{\pgfqpoint{3.696000in}{3.696000in}}%
\pgfusepath{clip}%
\pgfsetbuttcap%
\pgfsetroundjoin%
\definecolor{currentfill}{rgb}{0.121569,0.466667,0.705882}%
\pgfsetfillcolor{currentfill}%
\pgfsetfillopacity{0.351366}%
\pgfsetlinewidth{1.003750pt}%
\definecolor{currentstroke}{rgb}{0.121569,0.466667,0.705882}%
\pgfsetstrokecolor{currentstroke}%
\pgfsetstrokeopacity{0.351366}%
\pgfsetdash{}{0pt}%
\pgfpathmoveto{\pgfqpoint{1.422368in}{1.910108in}}%
\pgfpathcurveto{\pgfqpoint{1.430604in}{1.910108in}}{\pgfqpoint{1.438504in}{1.913381in}}{\pgfqpoint{1.444328in}{1.919205in}}%
\pgfpathcurveto{\pgfqpoint{1.450152in}{1.925028in}}{\pgfqpoint{1.453424in}{1.932929in}}{\pgfqpoint{1.453424in}{1.941165in}}%
\pgfpathcurveto{\pgfqpoint{1.453424in}{1.949401in}}{\pgfqpoint{1.450152in}{1.957301in}}{\pgfqpoint{1.444328in}{1.963125in}}%
\pgfpathcurveto{\pgfqpoint{1.438504in}{1.968949in}}{\pgfqpoint{1.430604in}{1.972221in}}{\pgfqpoint{1.422368in}{1.972221in}}%
\pgfpathcurveto{\pgfqpoint{1.414132in}{1.972221in}}{\pgfqpoint{1.406232in}{1.968949in}}{\pgfqpoint{1.400408in}{1.963125in}}%
\pgfpathcurveto{\pgfqpoint{1.394584in}{1.957301in}}{\pgfqpoint{1.391311in}{1.949401in}}{\pgfqpoint{1.391311in}{1.941165in}}%
\pgfpathcurveto{\pgfqpoint{1.391311in}{1.932929in}}{\pgfqpoint{1.394584in}{1.925028in}}{\pgfqpoint{1.400408in}{1.919205in}}%
\pgfpathcurveto{\pgfqpoint{1.406232in}{1.913381in}}{\pgfqpoint{1.414132in}{1.910108in}}{\pgfqpoint{1.422368in}{1.910108in}}%
\pgfpathclose%
\pgfusepath{stroke,fill}%
\end{pgfscope}%
\begin{pgfscope}%
\pgfpathrectangle{\pgfqpoint{0.100000in}{0.212622in}}{\pgfqpoint{3.696000in}{3.696000in}}%
\pgfusepath{clip}%
\pgfsetbuttcap%
\pgfsetroundjoin%
\definecolor{currentfill}{rgb}{0.121569,0.466667,0.705882}%
\pgfsetfillcolor{currentfill}%
\pgfsetfillopacity{0.352948}%
\pgfsetlinewidth{1.003750pt}%
\definecolor{currentstroke}{rgb}{0.121569,0.466667,0.705882}%
\pgfsetstrokecolor{currentstroke}%
\pgfsetstrokeopacity{0.352948}%
\pgfsetdash{}{0pt}%
\pgfpathmoveto{\pgfqpoint{1.420604in}{1.909780in}}%
\pgfpathcurveto{\pgfqpoint{1.428841in}{1.909780in}}{\pgfqpoint{1.436741in}{1.913053in}}{\pgfqpoint{1.442565in}{1.918877in}}%
\pgfpathcurveto{\pgfqpoint{1.448388in}{1.924700in}}{\pgfqpoint{1.451661in}{1.932601in}}{\pgfqpoint{1.451661in}{1.940837in}}%
\pgfpathcurveto{\pgfqpoint{1.451661in}{1.949073in}}{\pgfqpoint{1.448388in}{1.956973in}}{\pgfqpoint{1.442565in}{1.962797in}}%
\pgfpathcurveto{\pgfqpoint{1.436741in}{1.968621in}}{\pgfqpoint{1.428841in}{1.971893in}}{\pgfqpoint{1.420604in}{1.971893in}}%
\pgfpathcurveto{\pgfqpoint{1.412368in}{1.971893in}}{\pgfqpoint{1.404468in}{1.968621in}}{\pgfqpoint{1.398644in}{1.962797in}}%
\pgfpathcurveto{\pgfqpoint{1.392820in}{1.956973in}}{\pgfqpoint{1.389548in}{1.949073in}}{\pgfqpoint{1.389548in}{1.940837in}}%
\pgfpathcurveto{\pgfqpoint{1.389548in}{1.932601in}}{\pgfqpoint{1.392820in}{1.924700in}}{\pgfqpoint{1.398644in}{1.918877in}}%
\pgfpathcurveto{\pgfqpoint{1.404468in}{1.913053in}}{\pgfqpoint{1.412368in}{1.909780in}}{\pgfqpoint{1.420604in}{1.909780in}}%
\pgfpathclose%
\pgfusepath{stroke,fill}%
\end{pgfscope}%
\begin{pgfscope}%
\pgfpathrectangle{\pgfqpoint{0.100000in}{0.212622in}}{\pgfqpoint{3.696000in}{3.696000in}}%
\pgfusepath{clip}%
\pgfsetbuttcap%
\pgfsetroundjoin%
\definecolor{currentfill}{rgb}{0.121569,0.466667,0.705882}%
\pgfsetfillcolor{currentfill}%
\pgfsetfillopacity{0.356993}%
\pgfsetlinewidth{1.003750pt}%
\definecolor{currentstroke}{rgb}{0.121569,0.466667,0.705882}%
\pgfsetstrokecolor{currentstroke}%
\pgfsetstrokeopacity{0.356993}%
\pgfsetdash{}{0pt}%
\pgfpathmoveto{\pgfqpoint{1.424313in}{1.905164in}}%
\pgfpathcurveto{\pgfqpoint{1.432550in}{1.905164in}}{\pgfqpoint{1.440450in}{1.908437in}}{\pgfqpoint{1.446274in}{1.914261in}}%
\pgfpathcurveto{\pgfqpoint{1.452098in}{1.920085in}}{\pgfqpoint{1.455370in}{1.927985in}}{\pgfqpoint{1.455370in}{1.936221in}}%
\pgfpathcurveto{\pgfqpoint{1.455370in}{1.944457in}}{\pgfqpoint{1.452098in}{1.952357in}}{\pgfqpoint{1.446274in}{1.958181in}}%
\pgfpathcurveto{\pgfqpoint{1.440450in}{1.964005in}}{\pgfqpoint{1.432550in}{1.967277in}}{\pgfqpoint{1.424313in}{1.967277in}}%
\pgfpathcurveto{\pgfqpoint{1.416077in}{1.967277in}}{\pgfqpoint{1.408177in}{1.964005in}}{\pgfqpoint{1.402353in}{1.958181in}}%
\pgfpathcurveto{\pgfqpoint{1.396529in}{1.952357in}}{\pgfqpoint{1.393257in}{1.944457in}}{\pgfqpoint{1.393257in}{1.936221in}}%
\pgfpathcurveto{\pgfqpoint{1.393257in}{1.927985in}}{\pgfqpoint{1.396529in}{1.920085in}}{\pgfqpoint{1.402353in}{1.914261in}}%
\pgfpathcurveto{\pgfqpoint{1.408177in}{1.908437in}}{\pgfqpoint{1.416077in}{1.905164in}}{\pgfqpoint{1.424313in}{1.905164in}}%
\pgfpathclose%
\pgfusepath{stroke,fill}%
\end{pgfscope}%
\begin{pgfscope}%
\pgfpathrectangle{\pgfqpoint{0.100000in}{0.212622in}}{\pgfqpoint{3.696000in}{3.696000in}}%
\pgfusepath{clip}%
\pgfsetbuttcap%
\pgfsetroundjoin%
\definecolor{currentfill}{rgb}{0.121569,0.466667,0.705882}%
\pgfsetfillcolor{currentfill}%
\pgfsetfillopacity{0.358545}%
\pgfsetlinewidth{1.003750pt}%
\definecolor{currentstroke}{rgb}{0.121569,0.466667,0.705882}%
\pgfsetstrokecolor{currentstroke}%
\pgfsetstrokeopacity{0.358545}%
\pgfsetdash{}{0pt}%
\pgfpathmoveto{\pgfqpoint{1.405493in}{1.895589in}}%
\pgfpathcurveto{\pgfqpoint{1.413729in}{1.895589in}}{\pgfqpoint{1.421629in}{1.898861in}}{\pgfqpoint{1.427453in}{1.904685in}}%
\pgfpathcurveto{\pgfqpoint{1.433277in}{1.910509in}}{\pgfqpoint{1.436550in}{1.918409in}}{\pgfqpoint{1.436550in}{1.926645in}}%
\pgfpathcurveto{\pgfqpoint{1.436550in}{1.934882in}}{\pgfqpoint{1.433277in}{1.942782in}}{\pgfqpoint{1.427453in}{1.948606in}}%
\pgfpathcurveto{\pgfqpoint{1.421629in}{1.954430in}}{\pgfqpoint{1.413729in}{1.957702in}}{\pgfqpoint{1.405493in}{1.957702in}}%
\pgfpathcurveto{\pgfqpoint{1.397257in}{1.957702in}}{\pgfqpoint{1.389357in}{1.954430in}}{\pgfqpoint{1.383533in}{1.948606in}}%
\pgfpathcurveto{\pgfqpoint{1.377709in}{1.942782in}}{\pgfqpoint{1.374437in}{1.934882in}}{\pgfqpoint{1.374437in}{1.926645in}}%
\pgfpathcurveto{\pgfqpoint{1.374437in}{1.918409in}}{\pgfqpoint{1.377709in}{1.910509in}}{\pgfqpoint{1.383533in}{1.904685in}}%
\pgfpathcurveto{\pgfqpoint{1.389357in}{1.898861in}}{\pgfqpoint{1.397257in}{1.895589in}}{\pgfqpoint{1.405493in}{1.895589in}}%
\pgfpathclose%
\pgfusepath{stroke,fill}%
\end{pgfscope}%
\begin{pgfscope}%
\pgfpathrectangle{\pgfqpoint{0.100000in}{0.212622in}}{\pgfqpoint{3.696000in}{3.696000in}}%
\pgfusepath{clip}%
\pgfsetbuttcap%
\pgfsetroundjoin%
\definecolor{currentfill}{rgb}{0.121569,0.466667,0.705882}%
\pgfsetfillcolor{currentfill}%
\pgfsetfillopacity{0.358619}%
\pgfsetlinewidth{1.003750pt}%
\definecolor{currentstroke}{rgb}{0.121569,0.466667,0.705882}%
\pgfsetstrokecolor{currentstroke}%
\pgfsetstrokeopacity{0.358619}%
\pgfsetdash{}{0pt}%
\pgfpathmoveto{\pgfqpoint{1.421371in}{1.900783in}}%
\pgfpathcurveto{\pgfqpoint{1.429607in}{1.900783in}}{\pgfqpoint{1.437507in}{1.904055in}}{\pgfqpoint{1.443331in}{1.909879in}}%
\pgfpathcurveto{\pgfqpoint{1.449155in}{1.915703in}}{\pgfqpoint{1.452427in}{1.923603in}}{\pgfqpoint{1.452427in}{1.931839in}}%
\pgfpathcurveto{\pgfqpoint{1.452427in}{1.940076in}}{\pgfqpoint{1.449155in}{1.947976in}}{\pgfqpoint{1.443331in}{1.953800in}}%
\pgfpathcurveto{\pgfqpoint{1.437507in}{1.959624in}}{\pgfqpoint{1.429607in}{1.962896in}}{\pgfqpoint{1.421371in}{1.962896in}}%
\pgfpathcurveto{\pgfqpoint{1.413135in}{1.962896in}}{\pgfqpoint{1.405234in}{1.959624in}}{\pgfqpoint{1.399411in}{1.953800in}}%
\pgfpathcurveto{\pgfqpoint{1.393587in}{1.947976in}}{\pgfqpoint{1.390314in}{1.940076in}}{\pgfqpoint{1.390314in}{1.931839in}}%
\pgfpathcurveto{\pgfqpoint{1.390314in}{1.923603in}}{\pgfqpoint{1.393587in}{1.915703in}}{\pgfqpoint{1.399411in}{1.909879in}}%
\pgfpathcurveto{\pgfqpoint{1.405234in}{1.904055in}}{\pgfqpoint{1.413135in}{1.900783in}}{\pgfqpoint{1.421371in}{1.900783in}}%
\pgfpathclose%
\pgfusepath{stroke,fill}%
\end{pgfscope}%
\begin{pgfscope}%
\pgfpathrectangle{\pgfqpoint{0.100000in}{0.212622in}}{\pgfqpoint{3.696000in}{3.696000in}}%
\pgfusepath{clip}%
\pgfsetbuttcap%
\pgfsetroundjoin%
\definecolor{currentfill}{rgb}{0.121569,0.466667,0.705882}%
\pgfsetfillcolor{currentfill}%
\pgfsetfillopacity{0.358656}%
\pgfsetlinewidth{1.003750pt}%
\definecolor{currentstroke}{rgb}{0.121569,0.466667,0.705882}%
\pgfsetstrokecolor{currentstroke}%
\pgfsetstrokeopacity{0.358656}%
\pgfsetdash{}{0pt}%
\pgfpathmoveto{\pgfqpoint{1.416694in}{1.898639in}}%
\pgfpathcurveto{\pgfqpoint{1.424930in}{1.898639in}}{\pgfqpoint{1.432830in}{1.901912in}}{\pgfqpoint{1.438654in}{1.907736in}}%
\pgfpathcurveto{\pgfqpoint{1.444478in}{1.913559in}}{\pgfqpoint{1.447750in}{1.921459in}}{\pgfqpoint{1.447750in}{1.929696in}}%
\pgfpathcurveto{\pgfqpoint{1.447750in}{1.937932in}}{\pgfqpoint{1.444478in}{1.945832in}}{\pgfqpoint{1.438654in}{1.951656in}}%
\pgfpathcurveto{\pgfqpoint{1.432830in}{1.957480in}}{\pgfqpoint{1.424930in}{1.960752in}}{\pgfqpoint{1.416694in}{1.960752in}}%
\pgfpathcurveto{\pgfqpoint{1.408457in}{1.960752in}}{\pgfqpoint{1.400557in}{1.957480in}}{\pgfqpoint{1.394733in}{1.951656in}}%
\pgfpathcurveto{\pgfqpoint{1.388909in}{1.945832in}}{\pgfqpoint{1.385637in}{1.937932in}}{\pgfqpoint{1.385637in}{1.929696in}}%
\pgfpathcurveto{\pgfqpoint{1.385637in}{1.921459in}}{\pgfqpoint{1.388909in}{1.913559in}}{\pgfqpoint{1.394733in}{1.907736in}}%
\pgfpathcurveto{\pgfqpoint{1.400557in}{1.901912in}}{\pgfqpoint{1.408457in}{1.898639in}}{\pgfqpoint{1.416694in}{1.898639in}}%
\pgfpathclose%
\pgfusepath{stroke,fill}%
\end{pgfscope}%
\begin{pgfscope}%
\pgfpathrectangle{\pgfqpoint{0.100000in}{0.212622in}}{\pgfqpoint{3.696000in}{3.696000in}}%
\pgfusepath{clip}%
\pgfsetbuttcap%
\pgfsetroundjoin%
\definecolor{currentfill}{rgb}{0.121569,0.466667,0.705882}%
\pgfsetfillcolor{currentfill}%
\pgfsetfillopacity{0.358856}%
\pgfsetlinewidth{1.003750pt}%
\definecolor{currentstroke}{rgb}{0.121569,0.466667,0.705882}%
\pgfsetstrokecolor{currentstroke}%
\pgfsetstrokeopacity{0.358856}%
\pgfsetdash{}{0pt}%
\pgfpathmoveto{\pgfqpoint{1.424854in}{1.902914in}}%
\pgfpathcurveto{\pgfqpoint{1.433090in}{1.902914in}}{\pgfqpoint{1.440990in}{1.906186in}}{\pgfqpoint{1.446814in}{1.912010in}}%
\pgfpathcurveto{\pgfqpoint{1.452638in}{1.917834in}}{\pgfqpoint{1.455910in}{1.925734in}}{\pgfqpoint{1.455910in}{1.933970in}}%
\pgfpathcurveto{\pgfqpoint{1.455910in}{1.942207in}}{\pgfqpoint{1.452638in}{1.950107in}}{\pgfqpoint{1.446814in}{1.955931in}}%
\pgfpathcurveto{\pgfqpoint{1.440990in}{1.961755in}}{\pgfqpoint{1.433090in}{1.965027in}}{\pgfqpoint{1.424854in}{1.965027in}}%
\pgfpathcurveto{\pgfqpoint{1.416617in}{1.965027in}}{\pgfqpoint{1.408717in}{1.961755in}}{\pgfqpoint{1.402893in}{1.955931in}}%
\pgfpathcurveto{\pgfqpoint{1.397069in}{1.950107in}}{\pgfqpoint{1.393797in}{1.942207in}}{\pgfqpoint{1.393797in}{1.933970in}}%
\pgfpathcurveto{\pgfqpoint{1.393797in}{1.925734in}}{\pgfqpoint{1.397069in}{1.917834in}}{\pgfqpoint{1.402893in}{1.912010in}}%
\pgfpathcurveto{\pgfqpoint{1.408717in}{1.906186in}}{\pgfqpoint{1.416617in}{1.902914in}}{\pgfqpoint{1.424854in}{1.902914in}}%
\pgfpathclose%
\pgfusepath{stroke,fill}%
\end{pgfscope}%
\begin{pgfscope}%
\pgfpathrectangle{\pgfqpoint{0.100000in}{0.212622in}}{\pgfqpoint{3.696000in}{3.696000in}}%
\pgfusepath{clip}%
\pgfsetbuttcap%
\pgfsetroundjoin%
\definecolor{currentfill}{rgb}{0.121569,0.466667,0.705882}%
\pgfsetfillcolor{currentfill}%
\pgfsetfillopacity{0.359276}%
\pgfsetlinewidth{1.003750pt}%
\definecolor{currentstroke}{rgb}{0.121569,0.466667,0.705882}%
\pgfsetstrokecolor{currentstroke}%
\pgfsetstrokeopacity{0.359276}%
\pgfsetdash{}{0pt}%
\pgfpathmoveto{\pgfqpoint{1.412062in}{1.898314in}}%
\pgfpathcurveto{\pgfqpoint{1.420298in}{1.898314in}}{\pgfqpoint{1.428198in}{1.901586in}}{\pgfqpoint{1.434022in}{1.907410in}}%
\pgfpathcurveto{\pgfqpoint{1.439846in}{1.913234in}}{\pgfqpoint{1.443119in}{1.921134in}}{\pgfqpoint{1.443119in}{1.929370in}}%
\pgfpathcurveto{\pgfqpoint{1.443119in}{1.937606in}}{\pgfqpoint{1.439846in}{1.945506in}}{\pgfqpoint{1.434022in}{1.951330in}}%
\pgfpathcurveto{\pgfqpoint{1.428198in}{1.957154in}}{\pgfqpoint{1.420298in}{1.960427in}}{\pgfqpoint{1.412062in}{1.960427in}}%
\pgfpathcurveto{\pgfqpoint{1.403826in}{1.960427in}}{\pgfqpoint{1.395926in}{1.957154in}}{\pgfqpoint{1.390102in}{1.951330in}}%
\pgfpathcurveto{\pgfqpoint{1.384278in}{1.945506in}}{\pgfqpoint{1.381006in}{1.937606in}}{\pgfqpoint{1.381006in}{1.929370in}}%
\pgfpathcurveto{\pgfqpoint{1.381006in}{1.921134in}}{\pgfqpoint{1.384278in}{1.913234in}}{\pgfqpoint{1.390102in}{1.907410in}}%
\pgfpathcurveto{\pgfqpoint{1.395926in}{1.901586in}}{\pgfqpoint{1.403826in}{1.898314in}}{\pgfqpoint{1.412062in}{1.898314in}}%
\pgfpathclose%
\pgfusepath{stroke,fill}%
\end{pgfscope}%
\begin{pgfscope}%
\pgfpathrectangle{\pgfqpoint{0.100000in}{0.212622in}}{\pgfqpoint{3.696000in}{3.696000in}}%
\pgfusepath{clip}%
\pgfsetbuttcap%
\pgfsetroundjoin%
\definecolor{currentfill}{rgb}{0.121569,0.466667,0.705882}%
\pgfsetfillcolor{currentfill}%
\pgfsetfillopacity{0.359543}%
\pgfsetlinewidth{1.003750pt}%
\definecolor{currentstroke}{rgb}{0.121569,0.466667,0.705882}%
\pgfsetstrokecolor{currentstroke}%
\pgfsetstrokeopacity{0.359543}%
\pgfsetdash{}{0pt}%
\pgfpathmoveto{\pgfqpoint{1.425327in}{1.903820in}}%
\pgfpathcurveto{\pgfqpoint{1.433563in}{1.903820in}}{\pgfqpoint{1.441463in}{1.907092in}}{\pgfqpoint{1.447287in}{1.912916in}}%
\pgfpathcurveto{\pgfqpoint{1.453111in}{1.918740in}}{\pgfqpoint{1.456383in}{1.926640in}}{\pgfqpoint{1.456383in}{1.934876in}}%
\pgfpathcurveto{\pgfqpoint{1.456383in}{1.943113in}}{\pgfqpoint{1.453111in}{1.951013in}}{\pgfqpoint{1.447287in}{1.956837in}}%
\pgfpathcurveto{\pgfqpoint{1.441463in}{1.962661in}}{\pgfqpoint{1.433563in}{1.965933in}}{\pgfqpoint{1.425327in}{1.965933in}}%
\pgfpathcurveto{\pgfqpoint{1.417090in}{1.965933in}}{\pgfqpoint{1.409190in}{1.962661in}}{\pgfqpoint{1.403366in}{1.956837in}}%
\pgfpathcurveto{\pgfqpoint{1.397543in}{1.951013in}}{\pgfqpoint{1.394270in}{1.943113in}}{\pgfqpoint{1.394270in}{1.934876in}}%
\pgfpathcurveto{\pgfqpoint{1.394270in}{1.926640in}}{\pgfqpoint{1.397543in}{1.918740in}}{\pgfqpoint{1.403366in}{1.912916in}}%
\pgfpathcurveto{\pgfqpoint{1.409190in}{1.907092in}}{\pgfqpoint{1.417090in}{1.903820in}}{\pgfqpoint{1.425327in}{1.903820in}}%
\pgfpathclose%
\pgfusepath{stroke,fill}%
\end{pgfscope}%
\begin{pgfscope}%
\pgfpathrectangle{\pgfqpoint{0.100000in}{0.212622in}}{\pgfqpoint{3.696000in}{3.696000in}}%
\pgfusepath{clip}%
\pgfsetbuttcap%
\pgfsetroundjoin%
\definecolor{currentfill}{rgb}{0.121569,0.466667,0.705882}%
\pgfsetfillcolor{currentfill}%
\pgfsetfillopacity{0.359792}%
\pgfsetlinewidth{1.003750pt}%
\definecolor{currentstroke}{rgb}{0.121569,0.466667,0.705882}%
\pgfsetstrokecolor{currentstroke}%
\pgfsetstrokeopacity{0.359792}%
\pgfsetdash{}{0pt}%
\pgfpathmoveto{\pgfqpoint{1.788068in}{2.157409in}}%
\pgfpathcurveto{\pgfqpoint{1.796304in}{2.157409in}}{\pgfqpoint{1.804204in}{2.160681in}}{\pgfqpoint{1.810028in}{2.166505in}}%
\pgfpathcurveto{\pgfqpoint{1.815852in}{2.172329in}}{\pgfqpoint{1.819125in}{2.180229in}}{\pgfqpoint{1.819125in}{2.188466in}}%
\pgfpathcurveto{\pgfqpoint{1.819125in}{2.196702in}}{\pgfqpoint{1.815852in}{2.204602in}}{\pgfqpoint{1.810028in}{2.210426in}}%
\pgfpathcurveto{\pgfqpoint{1.804204in}{2.216250in}}{\pgfqpoint{1.796304in}{2.219522in}}{\pgfqpoint{1.788068in}{2.219522in}}%
\pgfpathcurveto{\pgfqpoint{1.779832in}{2.219522in}}{\pgfqpoint{1.771932in}{2.216250in}}{\pgfqpoint{1.766108in}{2.210426in}}%
\pgfpathcurveto{\pgfqpoint{1.760284in}{2.204602in}}{\pgfqpoint{1.757012in}{2.196702in}}{\pgfqpoint{1.757012in}{2.188466in}}%
\pgfpathcurveto{\pgfqpoint{1.757012in}{2.180229in}}{\pgfqpoint{1.760284in}{2.172329in}}{\pgfqpoint{1.766108in}{2.166505in}}%
\pgfpathcurveto{\pgfqpoint{1.771932in}{2.160681in}}{\pgfqpoint{1.779832in}{2.157409in}}{\pgfqpoint{1.788068in}{2.157409in}}%
\pgfpathclose%
\pgfusepath{stroke,fill}%
\end{pgfscope}%
\begin{pgfscope}%
\pgfpathrectangle{\pgfqpoint{0.100000in}{0.212622in}}{\pgfqpoint{3.696000in}{3.696000in}}%
\pgfusepath{clip}%
\pgfsetbuttcap%
\pgfsetroundjoin%
\definecolor{currentfill}{rgb}{0.121569,0.466667,0.705882}%
\pgfsetfillcolor{currentfill}%
\pgfsetfillopacity{0.360815}%
\pgfsetlinewidth{1.003750pt}%
\definecolor{currentstroke}{rgb}{0.121569,0.466667,0.705882}%
\pgfsetstrokecolor{currentstroke}%
\pgfsetstrokeopacity{0.360815}%
\pgfsetdash{}{0pt}%
\pgfpathmoveto{\pgfqpoint{1.403735in}{1.891546in}}%
\pgfpathcurveto{\pgfqpoint{1.411971in}{1.891546in}}{\pgfqpoint{1.419871in}{1.894818in}}{\pgfqpoint{1.425695in}{1.900642in}}%
\pgfpathcurveto{\pgfqpoint{1.431519in}{1.906466in}}{\pgfqpoint{1.434791in}{1.914366in}}{\pgfqpoint{1.434791in}{1.922602in}}%
\pgfpathcurveto{\pgfqpoint{1.434791in}{1.930839in}}{\pgfqpoint{1.431519in}{1.938739in}}{\pgfqpoint{1.425695in}{1.944563in}}%
\pgfpathcurveto{\pgfqpoint{1.419871in}{1.950387in}}{\pgfqpoint{1.411971in}{1.953659in}}{\pgfqpoint{1.403735in}{1.953659in}}%
\pgfpathcurveto{\pgfqpoint{1.395498in}{1.953659in}}{\pgfqpoint{1.387598in}{1.950387in}}{\pgfqpoint{1.381774in}{1.944563in}}%
\pgfpathcurveto{\pgfqpoint{1.375951in}{1.938739in}}{\pgfqpoint{1.372678in}{1.930839in}}{\pgfqpoint{1.372678in}{1.922602in}}%
\pgfpathcurveto{\pgfqpoint{1.372678in}{1.914366in}}{\pgfqpoint{1.375951in}{1.906466in}}{\pgfqpoint{1.381774in}{1.900642in}}%
\pgfpathcurveto{\pgfqpoint{1.387598in}{1.894818in}}{\pgfqpoint{1.395498in}{1.891546in}}{\pgfqpoint{1.403735in}{1.891546in}}%
\pgfpathclose%
\pgfusepath{stroke,fill}%
\end{pgfscope}%
\begin{pgfscope}%
\pgfpathrectangle{\pgfqpoint{0.100000in}{0.212622in}}{\pgfqpoint{3.696000in}{3.696000in}}%
\pgfusepath{clip}%
\pgfsetbuttcap%
\pgfsetroundjoin%
\definecolor{currentfill}{rgb}{0.121569,0.466667,0.705882}%
\pgfsetfillcolor{currentfill}%
\pgfsetfillopacity{0.360904}%
\pgfsetlinewidth{1.003750pt}%
\definecolor{currentstroke}{rgb}{0.121569,0.466667,0.705882}%
\pgfsetstrokecolor{currentstroke}%
\pgfsetstrokeopacity{0.360904}%
\pgfsetdash{}{0pt}%
\pgfpathmoveto{\pgfqpoint{1.945540in}{2.246627in}}%
\pgfpathcurveto{\pgfqpoint{1.953776in}{2.246627in}}{\pgfqpoint{1.961676in}{2.249899in}}{\pgfqpoint{1.967500in}{2.255723in}}%
\pgfpathcurveto{\pgfqpoint{1.973324in}{2.261547in}}{\pgfqpoint{1.976597in}{2.269447in}}{\pgfqpoint{1.976597in}{2.277683in}}%
\pgfpathcurveto{\pgfqpoint{1.976597in}{2.285919in}}{\pgfqpoint{1.973324in}{2.293820in}}{\pgfqpoint{1.967500in}{2.299643in}}%
\pgfpathcurveto{\pgfqpoint{1.961676in}{2.305467in}}{\pgfqpoint{1.953776in}{2.308740in}}{\pgfqpoint{1.945540in}{2.308740in}}%
\pgfpathcurveto{\pgfqpoint{1.937304in}{2.308740in}}{\pgfqpoint{1.929404in}{2.305467in}}{\pgfqpoint{1.923580in}{2.299643in}}%
\pgfpathcurveto{\pgfqpoint{1.917756in}{2.293820in}}{\pgfqpoint{1.914484in}{2.285919in}}{\pgfqpoint{1.914484in}{2.277683in}}%
\pgfpathcurveto{\pgfqpoint{1.914484in}{2.269447in}}{\pgfqpoint{1.917756in}{2.261547in}}{\pgfqpoint{1.923580in}{2.255723in}}%
\pgfpathcurveto{\pgfqpoint{1.929404in}{2.249899in}}{\pgfqpoint{1.937304in}{2.246627in}}{\pgfqpoint{1.945540in}{2.246627in}}%
\pgfpathclose%
\pgfusepath{stroke,fill}%
\end{pgfscope}%
\begin{pgfscope}%
\pgfpathrectangle{\pgfqpoint{0.100000in}{0.212622in}}{\pgfqpoint{3.696000in}{3.696000in}}%
\pgfusepath{clip}%
\pgfsetbuttcap%
\pgfsetroundjoin%
\definecolor{currentfill}{rgb}{0.121569,0.466667,0.705882}%
\pgfsetfillcolor{currentfill}%
\pgfsetfillopacity{0.361303}%
\pgfsetlinewidth{1.003750pt}%
\definecolor{currentstroke}{rgb}{0.121569,0.466667,0.705882}%
\pgfsetstrokecolor{currentstroke}%
\pgfsetstrokeopacity{0.361303}%
\pgfsetdash{}{0pt}%
\pgfpathmoveto{\pgfqpoint{1.947852in}{2.249360in}}%
\pgfpathcurveto{\pgfqpoint{1.956088in}{2.249360in}}{\pgfqpoint{1.963988in}{2.252633in}}{\pgfqpoint{1.969812in}{2.258457in}}%
\pgfpathcurveto{\pgfqpoint{1.975636in}{2.264281in}}{\pgfqpoint{1.978908in}{2.272181in}}{\pgfqpoint{1.978908in}{2.280417in}}%
\pgfpathcurveto{\pgfqpoint{1.978908in}{2.288653in}}{\pgfqpoint{1.975636in}{2.296553in}}{\pgfqpoint{1.969812in}{2.302377in}}%
\pgfpathcurveto{\pgfqpoint{1.963988in}{2.308201in}}{\pgfqpoint{1.956088in}{2.311473in}}{\pgfqpoint{1.947852in}{2.311473in}}%
\pgfpathcurveto{\pgfqpoint{1.939615in}{2.311473in}}{\pgfqpoint{1.931715in}{2.308201in}}{\pgfqpoint{1.925891in}{2.302377in}}%
\pgfpathcurveto{\pgfqpoint{1.920067in}{2.296553in}}{\pgfqpoint{1.916795in}{2.288653in}}{\pgfqpoint{1.916795in}{2.280417in}}%
\pgfpathcurveto{\pgfqpoint{1.916795in}{2.272181in}}{\pgfqpoint{1.920067in}{2.264281in}}{\pgfqpoint{1.925891in}{2.258457in}}%
\pgfpathcurveto{\pgfqpoint{1.931715in}{2.252633in}}{\pgfqpoint{1.939615in}{2.249360in}}{\pgfqpoint{1.947852in}{2.249360in}}%
\pgfpathclose%
\pgfusepath{stroke,fill}%
\end{pgfscope}%
\begin{pgfscope}%
\pgfpathrectangle{\pgfqpoint{0.100000in}{0.212622in}}{\pgfqpoint{3.696000in}{3.696000in}}%
\pgfusepath{clip}%
\pgfsetbuttcap%
\pgfsetroundjoin%
\definecolor{currentfill}{rgb}{0.121569,0.466667,0.705882}%
\pgfsetfillcolor{currentfill}%
\pgfsetfillopacity{0.361515}%
\pgfsetlinewidth{1.003750pt}%
\definecolor{currentstroke}{rgb}{0.121569,0.466667,0.705882}%
\pgfsetstrokecolor{currentstroke}%
\pgfsetstrokeopacity{0.361515}%
\pgfsetdash{}{0pt}%
\pgfpathmoveto{\pgfqpoint{1.399094in}{1.887601in}}%
\pgfpathcurveto{\pgfqpoint{1.407330in}{1.887601in}}{\pgfqpoint{1.415230in}{1.890874in}}{\pgfqpoint{1.421054in}{1.896698in}}%
\pgfpathcurveto{\pgfqpoint{1.426878in}{1.902521in}}{\pgfqpoint{1.430150in}{1.910422in}}{\pgfqpoint{1.430150in}{1.918658in}}%
\pgfpathcurveto{\pgfqpoint{1.430150in}{1.926894in}}{\pgfqpoint{1.426878in}{1.934794in}}{\pgfqpoint{1.421054in}{1.940618in}}%
\pgfpathcurveto{\pgfqpoint{1.415230in}{1.946442in}}{\pgfqpoint{1.407330in}{1.949714in}}{\pgfqpoint{1.399094in}{1.949714in}}%
\pgfpathcurveto{\pgfqpoint{1.390858in}{1.949714in}}{\pgfqpoint{1.382957in}{1.946442in}}{\pgfqpoint{1.377134in}{1.940618in}}%
\pgfpathcurveto{\pgfqpoint{1.371310in}{1.934794in}}{\pgfqpoint{1.368037in}{1.926894in}}{\pgfqpoint{1.368037in}{1.918658in}}%
\pgfpathcurveto{\pgfqpoint{1.368037in}{1.910422in}}{\pgfqpoint{1.371310in}{1.902521in}}{\pgfqpoint{1.377134in}{1.896698in}}%
\pgfpathcurveto{\pgfqpoint{1.382957in}{1.890874in}}{\pgfqpoint{1.390858in}{1.887601in}}{\pgfqpoint{1.399094in}{1.887601in}}%
\pgfpathclose%
\pgfusepath{stroke,fill}%
\end{pgfscope}%
\begin{pgfscope}%
\pgfpathrectangle{\pgfqpoint{0.100000in}{0.212622in}}{\pgfqpoint{3.696000in}{3.696000in}}%
\pgfusepath{clip}%
\pgfsetbuttcap%
\pgfsetroundjoin%
\definecolor{currentfill}{rgb}{0.121569,0.466667,0.705882}%
\pgfsetfillcolor{currentfill}%
\pgfsetfillopacity{0.361770}%
\pgfsetlinewidth{1.003750pt}%
\definecolor{currentstroke}{rgb}{0.121569,0.466667,0.705882}%
\pgfsetstrokecolor{currentstroke}%
\pgfsetstrokeopacity{0.361770}%
\pgfsetdash{}{0pt}%
\pgfpathmoveto{\pgfqpoint{1.911598in}{2.224991in}}%
\pgfpathcurveto{\pgfqpoint{1.919835in}{2.224991in}}{\pgfqpoint{1.927735in}{2.228263in}}{\pgfqpoint{1.933559in}{2.234087in}}%
\pgfpathcurveto{\pgfqpoint{1.939383in}{2.239911in}}{\pgfqpoint{1.942655in}{2.247811in}}{\pgfqpoint{1.942655in}{2.256048in}}%
\pgfpathcurveto{\pgfqpoint{1.942655in}{2.264284in}}{\pgfqpoint{1.939383in}{2.272184in}}{\pgfqpoint{1.933559in}{2.278008in}}%
\pgfpathcurveto{\pgfqpoint{1.927735in}{2.283832in}}{\pgfqpoint{1.919835in}{2.287104in}}{\pgfqpoint{1.911598in}{2.287104in}}%
\pgfpathcurveto{\pgfqpoint{1.903362in}{2.287104in}}{\pgfqpoint{1.895462in}{2.283832in}}{\pgfqpoint{1.889638in}{2.278008in}}%
\pgfpathcurveto{\pgfqpoint{1.883814in}{2.272184in}}{\pgfqpoint{1.880542in}{2.264284in}}{\pgfqpoint{1.880542in}{2.256048in}}%
\pgfpathcurveto{\pgfqpoint{1.880542in}{2.247811in}}{\pgfqpoint{1.883814in}{2.239911in}}{\pgfqpoint{1.889638in}{2.234087in}}%
\pgfpathcurveto{\pgfqpoint{1.895462in}{2.228263in}}{\pgfqpoint{1.903362in}{2.224991in}}{\pgfqpoint{1.911598in}{2.224991in}}%
\pgfpathclose%
\pgfusepath{stroke,fill}%
\end{pgfscope}%
\begin{pgfscope}%
\pgfpathrectangle{\pgfqpoint{0.100000in}{0.212622in}}{\pgfqpoint{3.696000in}{3.696000in}}%
\pgfusepath{clip}%
\pgfsetbuttcap%
\pgfsetroundjoin%
\definecolor{currentfill}{rgb}{0.121569,0.466667,0.705882}%
\pgfsetfillcolor{currentfill}%
\pgfsetfillopacity{0.362107}%
\pgfsetlinewidth{1.003750pt}%
\definecolor{currentstroke}{rgb}{0.121569,0.466667,0.705882}%
\pgfsetstrokecolor{currentstroke}%
\pgfsetstrokeopacity{0.362107}%
\pgfsetdash{}{0pt}%
\pgfpathmoveto{\pgfqpoint{1.903664in}{2.216052in}}%
\pgfpathcurveto{\pgfqpoint{1.911900in}{2.216052in}}{\pgfqpoint{1.919800in}{2.219325in}}{\pgfqpoint{1.925624in}{2.225149in}}%
\pgfpathcurveto{\pgfqpoint{1.931448in}{2.230973in}}{\pgfqpoint{1.934720in}{2.238873in}}{\pgfqpoint{1.934720in}{2.247109in}}%
\pgfpathcurveto{\pgfqpoint{1.934720in}{2.255345in}}{\pgfqpoint{1.931448in}{2.263245in}}{\pgfqpoint{1.925624in}{2.269069in}}%
\pgfpathcurveto{\pgfqpoint{1.919800in}{2.274893in}}{\pgfqpoint{1.911900in}{2.278165in}}{\pgfqpoint{1.903664in}{2.278165in}}%
\pgfpathcurveto{\pgfqpoint{1.895428in}{2.278165in}}{\pgfqpoint{1.887527in}{2.274893in}}{\pgfqpoint{1.881704in}{2.269069in}}%
\pgfpathcurveto{\pgfqpoint{1.875880in}{2.263245in}}{\pgfqpoint{1.872607in}{2.255345in}}{\pgfqpoint{1.872607in}{2.247109in}}%
\pgfpathcurveto{\pgfqpoint{1.872607in}{2.238873in}}{\pgfqpoint{1.875880in}{2.230973in}}{\pgfqpoint{1.881704in}{2.225149in}}%
\pgfpathcurveto{\pgfqpoint{1.887527in}{2.219325in}}{\pgfqpoint{1.895428in}{2.216052in}}{\pgfqpoint{1.903664in}{2.216052in}}%
\pgfpathclose%
\pgfusepath{stroke,fill}%
\end{pgfscope}%
\begin{pgfscope}%
\pgfpathrectangle{\pgfqpoint{0.100000in}{0.212622in}}{\pgfqpoint{3.696000in}{3.696000in}}%
\pgfusepath{clip}%
\pgfsetbuttcap%
\pgfsetroundjoin%
\definecolor{currentfill}{rgb}{0.121569,0.466667,0.705882}%
\pgfsetfillcolor{currentfill}%
\pgfsetfillopacity{0.362378}%
\pgfsetlinewidth{1.003750pt}%
\definecolor{currentstroke}{rgb}{0.121569,0.466667,0.705882}%
\pgfsetstrokecolor{currentstroke}%
\pgfsetstrokeopacity{0.362378}%
\pgfsetdash{}{0pt}%
\pgfpathmoveto{\pgfqpoint{1.901204in}{2.213758in}}%
\pgfpathcurveto{\pgfqpoint{1.909441in}{2.213758in}}{\pgfqpoint{1.917341in}{2.217030in}}{\pgfqpoint{1.923165in}{2.222854in}}%
\pgfpathcurveto{\pgfqpoint{1.928988in}{2.228678in}}{\pgfqpoint{1.932261in}{2.236578in}}{\pgfqpoint{1.932261in}{2.244814in}}%
\pgfpathcurveto{\pgfqpoint{1.932261in}{2.253051in}}{\pgfqpoint{1.928988in}{2.260951in}}{\pgfqpoint{1.923165in}{2.266775in}}%
\pgfpathcurveto{\pgfqpoint{1.917341in}{2.272599in}}{\pgfqpoint{1.909441in}{2.275871in}}{\pgfqpoint{1.901204in}{2.275871in}}%
\pgfpathcurveto{\pgfqpoint{1.892968in}{2.275871in}}{\pgfqpoint{1.885068in}{2.272599in}}{\pgfqpoint{1.879244in}{2.266775in}}%
\pgfpathcurveto{\pgfqpoint{1.873420in}{2.260951in}}{\pgfqpoint{1.870148in}{2.253051in}}{\pgfqpoint{1.870148in}{2.244814in}}%
\pgfpathcurveto{\pgfqpoint{1.870148in}{2.236578in}}{\pgfqpoint{1.873420in}{2.228678in}}{\pgfqpoint{1.879244in}{2.222854in}}%
\pgfpathcurveto{\pgfqpoint{1.885068in}{2.217030in}}{\pgfqpoint{1.892968in}{2.213758in}}{\pgfqpoint{1.901204in}{2.213758in}}%
\pgfpathclose%
\pgfusepath{stroke,fill}%
\end{pgfscope}%
\begin{pgfscope}%
\pgfpathrectangle{\pgfqpoint{0.100000in}{0.212622in}}{\pgfqpoint{3.696000in}{3.696000in}}%
\pgfusepath{clip}%
\pgfsetbuttcap%
\pgfsetroundjoin%
\definecolor{currentfill}{rgb}{0.121569,0.466667,0.705882}%
\pgfsetfillcolor{currentfill}%
\pgfsetfillopacity{0.362426}%
\pgfsetlinewidth{1.003750pt}%
\definecolor{currentstroke}{rgb}{0.121569,0.466667,0.705882}%
\pgfsetstrokecolor{currentstroke}%
\pgfsetstrokeopacity{0.362426}%
\pgfsetdash{}{0pt}%
\pgfpathmoveto{\pgfqpoint{1.454478in}{1.921459in}}%
\pgfpathcurveto{\pgfqpoint{1.462714in}{1.921459in}}{\pgfqpoint{1.470614in}{1.924732in}}{\pgfqpoint{1.476438in}{1.930555in}}%
\pgfpathcurveto{\pgfqpoint{1.482262in}{1.936379in}}{\pgfqpoint{1.485534in}{1.944279in}}{\pgfqpoint{1.485534in}{1.952516in}}%
\pgfpathcurveto{\pgfqpoint{1.485534in}{1.960752in}}{\pgfqpoint{1.482262in}{1.968652in}}{\pgfqpoint{1.476438in}{1.974476in}}%
\pgfpathcurveto{\pgfqpoint{1.470614in}{1.980300in}}{\pgfqpoint{1.462714in}{1.983572in}}{\pgfqpoint{1.454478in}{1.983572in}}%
\pgfpathcurveto{\pgfqpoint{1.446241in}{1.983572in}}{\pgfqpoint{1.438341in}{1.980300in}}{\pgfqpoint{1.432517in}{1.974476in}}%
\pgfpathcurveto{\pgfqpoint{1.426693in}{1.968652in}}{\pgfqpoint{1.423421in}{1.960752in}}{\pgfqpoint{1.423421in}{1.952516in}}%
\pgfpathcurveto{\pgfqpoint{1.423421in}{1.944279in}}{\pgfqpoint{1.426693in}{1.936379in}}{\pgfqpoint{1.432517in}{1.930555in}}%
\pgfpathcurveto{\pgfqpoint{1.438341in}{1.924732in}}{\pgfqpoint{1.446241in}{1.921459in}}{\pgfqpoint{1.454478in}{1.921459in}}%
\pgfpathclose%
\pgfusepath{stroke,fill}%
\end{pgfscope}%
\begin{pgfscope}%
\pgfpathrectangle{\pgfqpoint{0.100000in}{0.212622in}}{\pgfqpoint{3.696000in}{3.696000in}}%
\pgfusepath{clip}%
\pgfsetbuttcap%
\pgfsetroundjoin%
\definecolor{currentfill}{rgb}{0.121569,0.466667,0.705882}%
\pgfsetfillcolor{currentfill}%
\pgfsetfillopacity{0.362442}%
\pgfsetlinewidth{1.003750pt}%
\definecolor{currentstroke}{rgb}{0.121569,0.466667,0.705882}%
\pgfsetstrokecolor{currentstroke}%
\pgfsetstrokeopacity{0.362442}%
\pgfsetdash{}{0pt}%
\pgfpathmoveto{\pgfqpoint{1.898690in}{2.212872in}}%
\pgfpathcurveto{\pgfqpoint{1.906926in}{2.212872in}}{\pgfqpoint{1.914826in}{2.216144in}}{\pgfqpoint{1.920650in}{2.221968in}}%
\pgfpathcurveto{\pgfqpoint{1.926474in}{2.227792in}}{\pgfqpoint{1.929746in}{2.235692in}}{\pgfqpoint{1.929746in}{2.243929in}}%
\pgfpathcurveto{\pgfqpoint{1.929746in}{2.252165in}}{\pgfqpoint{1.926474in}{2.260065in}}{\pgfqpoint{1.920650in}{2.265889in}}%
\pgfpathcurveto{\pgfqpoint{1.914826in}{2.271713in}}{\pgfqpoint{1.906926in}{2.274985in}}{\pgfqpoint{1.898690in}{2.274985in}}%
\pgfpathcurveto{\pgfqpoint{1.890453in}{2.274985in}}{\pgfqpoint{1.882553in}{2.271713in}}{\pgfqpoint{1.876729in}{2.265889in}}%
\pgfpathcurveto{\pgfqpoint{1.870905in}{2.260065in}}{\pgfqpoint{1.867633in}{2.252165in}}{\pgfqpoint{1.867633in}{2.243929in}}%
\pgfpathcurveto{\pgfqpoint{1.867633in}{2.235692in}}{\pgfqpoint{1.870905in}{2.227792in}}{\pgfqpoint{1.876729in}{2.221968in}}%
\pgfpathcurveto{\pgfqpoint{1.882553in}{2.216144in}}{\pgfqpoint{1.890453in}{2.212872in}}{\pgfqpoint{1.898690in}{2.212872in}}%
\pgfpathclose%
\pgfusepath{stroke,fill}%
\end{pgfscope}%
\begin{pgfscope}%
\pgfpathrectangle{\pgfqpoint{0.100000in}{0.212622in}}{\pgfqpoint{3.696000in}{3.696000in}}%
\pgfusepath{clip}%
\pgfsetbuttcap%
\pgfsetroundjoin%
\definecolor{currentfill}{rgb}{0.121569,0.466667,0.705882}%
\pgfsetfillcolor{currentfill}%
\pgfsetfillopacity{0.362525}%
\pgfsetlinewidth{1.003750pt}%
\definecolor{currentstroke}{rgb}{0.121569,0.466667,0.705882}%
\pgfsetstrokecolor{currentstroke}%
\pgfsetstrokeopacity{0.362525}%
\pgfsetdash{}{0pt}%
\pgfpathmoveto{\pgfqpoint{1.894083in}{2.210867in}}%
\pgfpathcurveto{\pgfqpoint{1.902320in}{2.210867in}}{\pgfqpoint{1.910220in}{2.214139in}}{\pgfqpoint{1.916044in}{2.219963in}}%
\pgfpathcurveto{\pgfqpoint{1.921867in}{2.225787in}}{\pgfqpoint{1.925140in}{2.233687in}}{\pgfqpoint{1.925140in}{2.241923in}}%
\pgfpathcurveto{\pgfqpoint{1.925140in}{2.250159in}}{\pgfqpoint{1.921867in}{2.258059in}}{\pgfqpoint{1.916044in}{2.263883in}}%
\pgfpathcurveto{\pgfqpoint{1.910220in}{2.269707in}}{\pgfqpoint{1.902320in}{2.272980in}}{\pgfqpoint{1.894083in}{2.272980in}}%
\pgfpathcurveto{\pgfqpoint{1.885847in}{2.272980in}}{\pgfqpoint{1.877947in}{2.269707in}}{\pgfqpoint{1.872123in}{2.263883in}}%
\pgfpathcurveto{\pgfqpoint{1.866299in}{2.258059in}}{\pgfqpoint{1.863027in}{2.250159in}}{\pgfqpoint{1.863027in}{2.241923in}}%
\pgfpathcurveto{\pgfqpoint{1.863027in}{2.233687in}}{\pgfqpoint{1.866299in}{2.225787in}}{\pgfqpoint{1.872123in}{2.219963in}}%
\pgfpathcurveto{\pgfqpoint{1.877947in}{2.214139in}}{\pgfqpoint{1.885847in}{2.210867in}}{\pgfqpoint{1.894083in}{2.210867in}}%
\pgfpathclose%
\pgfusepath{stroke,fill}%
\end{pgfscope}%
\begin{pgfscope}%
\pgfpathrectangle{\pgfqpoint{0.100000in}{0.212622in}}{\pgfqpoint{3.696000in}{3.696000in}}%
\pgfusepath{clip}%
\pgfsetbuttcap%
\pgfsetroundjoin%
\definecolor{currentfill}{rgb}{0.121569,0.466667,0.705882}%
\pgfsetfillcolor{currentfill}%
\pgfsetfillopacity{0.363050}%
\pgfsetlinewidth{1.003750pt}%
\definecolor{currentstroke}{rgb}{0.121569,0.466667,0.705882}%
\pgfsetstrokecolor{currentstroke}%
\pgfsetstrokeopacity{0.363050}%
\pgfsetdash{}{0pt}%
\pgfpathmoveto{\pgfqpoint{1.416517in}{1.894443in}}%
\pgfpathcurveto{\pgfqpoint{1.424753in}{1.894443in}}{\pgfqpoint{1.432653in}{1.897716in}}{\pgfqpoint{1.438477in}{1.903540in}}%
\pgfpathcurveto{\pgfqpoint{1.444301in}{1.909364in}}{\pgfqpoint{1.447573in}{1.917264in}}{\pgfqpoint{1.447573in}{1.925500in}}%
\pgfpathcurveto{\pgfqpoint{1.447573in}{1.933736in}}{\pgfqpoint{1.444301in}{1.941636in}}{\pgfqpoint{1.438477in}{1.947460in}}%
\pgfpathcurveto{\pgfqpoint{1.432653in}{1.953284in}}{\pgfqpoint{1.424753in}{1.956556in}}{\pgfqpoint{1.416517in}{1.956556in}}%
\pgfpathcurveto{\pgfqpoint{1.408281in}{1.956556in}}{\pgfqpoint{1.400381in}{1.953284in}}{\pgfqpoint{1.394557in}{1.947460in}}%
\pgfpathcurveto{\pgfqpoint{1.388733in}{1.941636in}}{\pgfqpoint{1.385460in}{1.933736in}}{\pgfqpoint{1.385460in}{1.925500in}}%
\pgfpathcurveto{\pgfqpoint{1.385460in}{1.917264in}}{\pgfqpoint{1.388733in}{1.909364in}}{\pgfqpoint{1.394557in}{1.903540in}}%
\pgfpathcurveto{\pgfqpoint{1.400381in}{1.897716in}}{\pgfqpoint{1.408281in}{1.894443in}}{\pgfqpoint{1.416517in}{1.894443in}}%
\pgfpathclose%
\pgfusepath{stroke,fill}%
\end{pgfscope}%
\begin{pgfscope}%
\pgfpathrectangle{\pgfqpoint{0.100000in}{0.212622in}}{\pgfqpoint{3.696000in}{3.696000in}}%
\pgfusepath{clip}%
\pgfsetbuttcap%
\pgfsetroundjoin%
\definecolor{currentfill}{rgb}{0.121569,0.466667,0.705882}%
\pgfsetfillcolor{currentfill}%
\pgfsetfillopacity{0.363353}%
\pgfsetlinewidth{1.003750pt}%
\definecolor{currentstroke}{rgb}{0.121569,0.466667,0.705882}%
\pgfsetstrokecolor{currentstroke}%
\pgfsetstrokeopacity{0.363353}%
\pgfsetdash{}{0pt}%
\pgfpathmoveto{\pgfqpoint{1.901862in}{2.213728in}}%
\pgfpathcurveto{\pgfqpoint{1.910099in}{2.213728in}}{\pgfqpoint{1.917999in}{2.217000in}}{\pgfqpoint{1.923823in}{2.222824in}}%
\pgfpathcurveto{\pgfqpoint{1.929646in}{2.228648in}}{\pgfqpoint{1.932919in}{2.236548in}}{\pgfqpoint{1.932919in}{2.244784in}}%
\pgfpathcurveto{\pgfqpoint{1.932919in}{2.253021in}}{\pgfqpoint{1.929646in}{2.260921in}}{\pgfqpoint{1.923823in}{2.266745in}}%
\pgfpathcurveto{\pgfqpoint{1.917999in}{2.272569in}}{\pgfqpoint{1.910099in}{2.275841in}}{\pgfqpoint{1.901862in}{2.275841in}}%
\pgfpathcurveto{\pgfqpoint{1.893626in}{2.275841in}}{\pgfqpoint{1.885726in}{2.272569in}}{\pgfqpoint{1.879902in}{2.266745in}}%
\pgfpathcurveto{\pgfqpoint{1.874078in}{2.260921in}}{\pgfqpoint{1.870806in}{2.253021in}}{\pgfqpoint{1.870806in}{2.244784in}}%
\pgfpathcurveto{\pgfqpoint{1.870806in}{2.236548in}}{\pgfqpoint{1.874078in}{2.228648in}}{\pgfqpoint{1.879902in}{2.222824in}}%
\pgfpathcurveto{\pgfqpoint{1.885726in}{2.217000in}}{\pgfqpoint{1.893626in}{2.213728in}}{\pgfqpoint{1.901862in}{2.213728in}}%
\pgfpathclose%
\pgfusepath{stroke,fill}%
\end{pgfscope}%
\begin{pgfscope}%
\pgfpathrectangle{\pgfqpoint{0.100000in}{0.212622in}}{\pgfqpoint{3.696000in}{3.696000in}}%
\pgfusepath{clip}%
\pgfsetbuttcap%
\pgfsetroundjoin%
\definecolor{currentfill}{rgb}{0.121569,0.466667,0.705882}%
\pgfsetfillcolor{currentfill}%
\pgfsetfillopacity{0.363788}%
\pgfsetlinewidth{1.003750pt}%
\definecolor{currentstroke}{rgb}{0.121569,0.466667,0.705882}%
\pgfsetstrokecolor{currentstroke}%
\pgfsetstrokeopacity{0.363788}%
\pgfsetdash{}{0pt}%
\pgfpathmoveto{\pgfqpoint{1.419688in}{1.899859in}}%
\pgfpathcurveto{\pgfqpoint{1.427924in}{1.899859in}}{\pgfqpoint{1.435824in}{1.903132in}}{\pgfqpoint{1.441648in}{1.908956in}}%
\pgfpathcurveto{\pgfqpoint{1.447472in}{1.914780in}}{\pgfqpoint{1.450745in}{1.922680in}}{\pgfqpoint{1.450745in}{1.930916in}}%
\pgfpathcurveto{\pgfqpoint{1.450745in}{1.939152in}}{\pgfqpoint{1.447472in}{1.947052in}}{\pgfqpoint{1.441648in}{1.952876in}}%
\pgfpathcurveto{\pgfqpoint{1.435824in}{1.958700in}}{\pgfqpoint{1.427924in}{1.961972in}}{\pgfqpoint{1.419688in}{1.961972in}}%
\pgfpathcurveto{\pgfqpoint{1.411452in}{1.961972in}}{\pgfqpoint{1.403552in}{1.958700in}}{\pgfqpoint{1.397728in}{1.952876in}}%
\pgfpathcurveto{\pgfqpoint{1.391904in}{1.947052in}}{\pgfqpoint{1.388632in}{1.939152in}}{\pgfqpoint{1.388632in}{1.930916in}}%
\pgfpathcurveto{\pgfqpoint{1.388632in}{1.922680in}}{\pgfqpoint{1.391904in}{1.914780in}}{\pgfqpoint{1.397728in}{1.908956in}}%
\pgfpathcurveto{\pgfqpoint{1.403552in}{1.903132in}}{\pgfqpoint{1.411452in}{1.899859in}}{\pgfqpoint{1.419688in}{1.899859in}}%
\pgfpathclose%
\pgfusepath{stroke,fill}%
\end{pgfscope}%
\begin{pgfscope}%
\pgfpathrectangle{\pgfqpoint{0.100000in}{0.212622in}}{\pgfqpoint{3.696000in}{3.696000in}}%
\pgfusepath{clip}%
\pgfsetbuttcap%
\pgfsetroundjoin%
\definecolor{currentfill}{rgb}{0.121569,0.466667,0.705882}%
\pgfsetfillcolor{currentfill}%
\pgfsetfillopacity{0.364690}%
\pgfsetlinewidth{1.003750pt}%
\definecolor{currentstroke}{rgb}{0.121569,0.466667,0.705882}%
\pgfsetstrokecolor{currentstroke}%
\pgfsetstrokeopacity{0.364690}%
\pgfsetdash{}{0pt}%
\pgfpathmoveto{\pgfqpoint{1.865816in}{2.190764in}}%
\pgfpathcurveto{\pgfqpoint{1.874052in}{2.190764in}}{\pgfqpoint{1.881952in}{2.194036in}}{\pgfqpoint{1.887776in}{2.199860in}}%
\pgfpathcurveto{\pgfqpoint{1.893600in}{2.205684in}}{\pgfqpoint{1.896872in}{2.213584in}}{\pgfqpoint{1.896872in}{2.221820in}}%
\pgfpathcurveto{\pgfqpoint{1.896872in}{2.230057in}}{\pgfqpoint{1.893600in}{2.237957in}}{\pgfqpoint{1.887776in}{2.243781in}}%
\pgfpathcurveto{\pgfqpoint{1.881952in}{2.249605in}}{\pgfqpoint{1.874052in}{2.252877in}}{\pgfqpoint{1.865816in}{2.252877in}}%
\pgfpathcurveto{\pgfqpoint{1.857579in}{2.252877in}}{\pgfqpoint{1.849679in}{2.249605in}}{\pgfqpoint{1.843855in}{2.243781in}}%
\pgfpathcurveto{\pgfqpoint{1.838031in}{2.237957in}}{\pgfqpoint{1.834759in}{2.230057in}}{\pgfqpoint{1.834759in}{2.221820in}}%
\pgfpathcurveto{\pgfqpoint{1.834759in}{2.213584in}}{\pgfqpoint{1.838031in}{2.205684in}}{\pgfqpoint{1.843855in}{2.199860in}}%
\pgfpathcurveto{\pgfqpoint{1.849679in}{2.194036in}}{\pgfqpoint{1.857579in}{2.190764in}}{\pgfqpoint{1.865816in}{2.190764in}}%
\pgfpathclose%
\pgfusepath{stroke,fill}%
\end{pgfscope}%
\begin{pgfscope}%
\pgfpathrectangle{\pgfqpoint{0.100000in}{0.212622in}}{\pgfqpoint{3.696000in}{3.696000in}}%
\pgfusepath{clip}%
\pgfsetbuttcap%
\pgfsetroundjoin%
\definecolor{currentfill}{rgb}{0.121569,0.466667,0.705882}%
\pgfsetfillcolor{currentfill}%
\pgfsetfillopacity{0.365191}%
\pgfsetlinewidth{1.003750pt}%
\definecolor{currentstroke}{rgb}{0.121569,0.466667,0.705882}%
\pgfsetstrokecolor{currentstroke}%
\pgfsetstrokeopacity{0.365191}%
\pgfsetdash{}{0pt}%
\pgfpathmoveto{\pgfqpoint{1.871854in}{2.195197in}}%
\pgfpathcurveto{\pgfqpoint{1.880090in}{2.195197in}}{\pgfqpoint{1.887990in}{2.198469in}}{\pgfqpoint{1.893814in}{2.204293in}}%
\pgfpathcurveto{\pgfqpoint{1.899638in}{2.210117in}}{\pgfqpoint{1.902910in}{2.218017in}}{\pgfqpoint{1.902910in}{2.226253in}}%
\pgfpathcurveto{\pgfqpoint{1.902910in}{2.234489in}}{\pgfqpoint{1.899638in}{2.242389in}}{\pgfqpoint{1.893814in}{2.248213in}}%
\pgfpathcurveto{\pgfqpoint{1.887990in}{2.254037in}}{\pgfqpoint{1.880090in}{2.257310in}}{\pgfqpoint{1.871854in}{2.257310in}}%
\pgfpathcurveto{\pgfqpoint{1.863617in}{2.257310in}}{\pgfqpoint{1.855717in}{2.254037in}}{\pgfqpoint{1.849893in}{2.248213in}}%
\pgfpathcurveto{\pgfqpoint{1.844069in}{2.242389in}}{\pgfqpoint{1.840797in}{2.234489in}}{\pgfqpoint{1.840797in}{2.226253in}}%
\pgfpathcurveto{\pgfqpoint{1.840797in}{2.218017in}}{\pgfqpoint{1.844069in}{2.210117in}}{\pgfqpoint{1.849893in}{2.204293in}}%
\pgfpathcurveto{\pgfqpoint{1.855717in}{2.198469in}}{\pgfqpoint{1.863617in}{2.195197in}}{\pgfqpoint{1.871854in}{2.195197in}}%
\pgfpathclose%
\pgfusepath{stroke,fill}%
\end{pgfscope}%
\begin{pgfscope}%
\pgfpathrectangle{\pgfqpoint{0.100000in}{0.212622in}}{\pgfqpoint{3.696000in}{3.696000in}}%
\pgfusepath{clip}%
\pgfsetbuttcap%
\pgfsetroundjoin%
\definecolor{currentfill}{rgb}{0.121569,0.466667,0.705882}%
\pgfsetfillcolor{currentfill}%
\pgfsetfillopacity{0.365210}%
\pgfsetlinewidth{1.003750pt}%
\definecolor{currentstroke}{rgb}{0.121569,0.466667,0.705882}%
\pgfsetstrokecolor{currentstroke}%
\pgfsetstrokeopacity{0.365210}%
\pgfsetdash{}{0pt}%
\pgfpathmoveto{\pgfqpoint{1.881419in}{2.200344in}}%
\pgfpathcurveto{\pgfqpoint{1.889655in}{2.200344in}}{\pgfqpoint{1.897556in}{2.203616in}}{\pgfqpoint{1.903379in}{2.209440in}}%
\pgfpathcurveto{\pgfqpoint{1.909203in}{2.215264in}}{\pgfqpoint{1.912476in}{2.223164in}}{\pgfqpoint{1.912476in}{2.231400in}}%
\pgfpathcurveto{\pgfqpoint{1.912476in}{2.239636in}}{\pgfqpoint{1.909203in}{2.247536in}}{\pgfqpoint{1.903379in}{2.253360in}}%
\pgfpathcurveto{\pgfqpoint{1.897556in}{2.259184in}}{\pgfqpoint{1.889655in}{2.262457in}}{\pgfqpoint{1.881419in}{2.262457in}}%
\pgfpathcurveto{\pgfqpoint{1.873183in}{2.262457in}}{\pgfqpoint{1.865283in}{2.259184in}}{\pgfqpoint{1.859459in}{2.253360in}}%
\pgfpathcurveto{\pgfqpoint{1.853635in}{2.247536in}}{\pgfqpoint{1.850363in}{2.239636in}}{\pgfqpoint{1.850363in}{2.231400in}}%
\pgfpathcurveto{\pgfqpoint{1.850363in}{2.223164in}}{\pgfqpoint{1.853635in}{2.215264in}}{\pgfqpoint{1.859459in}{2.209440in}}%
\pgfpathcurveto{\pgfqpoint{1.865283in}{2.203616in}}{\pgfqpoint{1.873183in}{2.200344in}}{\pgfqpoint{1.881419in}{2.200344in}}%
\pgfpathclose%
\pgfusepath{stroke,fill}%
\end{pgfscope}%
\begin{pgfscope}%
\pgfpathrectangle{\pgfqpoint{0.100000in}{0.212622in}}{\pgfqpoint{3.696000in}{3.696000in}}%
\pgfusepath{clip}%
\pgfsetbuttcap%
\pgfsetroundjoin%
\definecolor{currentfill}{rgb}{0.121569,0.466667,0.705882}%
\pgfsetfillcolor{currentfill}%
\pgfsetfillopacity{0.365374}%
\pgfsetlinewidth{1.003750pt}%
\definecolor{currentstroke}{rgb}{0.121569,0.466667,0.705882}%
\pgfsetstrokecolor{currentstroke}%
\pgfsetstrokeopacity{0.365374}%
\pgfsetdash{}{0pt}%
\pgfpathmoveto{\pgfqpoint{1.912238in}{2.226764in}}%
\pgfpathcurveto{\pgfqpoint{1.920475in}{2.226764in}}{\pgfqpoint{1.928375in}{2.230036in}}{\pgfqpoint{1.934199in}{2.235860in}}%
\pgfpathcurveto{\pgfqpoint{1.940023in}{2.241684in}}{\pgfqpoint{1.943295in}{2.249584in}}{\pgfqpoint{1.943295in}{2.257821in}}%
\pgfpathcurveto{\pgfqpoint{1.943295in}{2.266057in}}{\pgfqpoint{1.940023in}{2.273957in}}{\pgfqpoint{1.934199in}{2.279781in}}%
\pgfpathcurveto{\pgfqpoint{1.928375in}{2.285605in}}{\pgfqpoint{1.920475in}{2.288877in}}{\pgfqpoint{1.912238in}{2.288877in}}%
\pgfpathcurveto{\pgfqpoint{1.904002in}{2.288877in}}{\pgfqpoint{1.896102in}{2.285605in}}{\pgfqpoint{1.890278in}{2.279781in}}%
\pgfpathcurveto{\pgfqpoint{1.884454in}{2.273957in}}{\pgfqpoint{1.881182in}{2.266057in}}{\pgfqpoint{1.881182in}{2.257821in}}%
\pgfpathcurveto{\pgfqpoint{1.881182in}{2.249584in}}{\pgfqpoint{1.884454in}{2.241684in}}{\pgfqpoint{1.890278in}{2.235860in}}%
\pgfpathcurveto{\pgfqpoint{1.896102in}{2.230036in}}{\pgfqpoint{1.904002in}{2.226764in}}{\pgfqpoint{1.912238in}{2.226764in}}%
\pgfpathclose%
\pgfusepath{stroke,fill}%
\end{pgfscope}%
\begin{pgfscope}%
\pgfpathrectangle{\pgfqpoint{0.100000in}{0.212622in}}{\pgfqpoint{3.696000in}{3.696000in}}%
\pgfusepath{clip}%
\pgfsetbuttcap%
\pgfsetroundjoin%
\definecolor{currentfill}{rgb}{0.121569,0.466667,0.705882}%
\pgfsetfillcolor{currentfill}%
\pgfsetfillopacity{0.365430}%
\pgfsetlinewidth{1.003750pt}%
\definecolor{currentstroke}{rgb}{0.121569,0.466667,0.705882}%
\pgfsetstrokecolor{currentstroke}%
\pgfsetstrokeopacity{0.365430}%
\pgfsetdash{}{0pt}%
\pgfpathmoveto{\pgfqpoint{1.751533in}{2.126976in}}%
\pgfpathcurveto{\pgfqpoint{1.759769in}{2.126976in}}{\pgfqpoint{1.767669in}{2.130248in}}{\pgfqpoint{1.773493in}{2.136072in}}%
\pgfpathcurveto{\pgfqpoint{1.779317in}{2.141896in}}{\pgfqpoint{1.782590in}{2.149796in}}{\pgfqpoint{1.782590in}{2.158032in}}%
\pgfpathcurveto{\pgfqpoint{1.782590in}{2.166268in}}{\pgfqpoint{1.779317in}{2.174168in}}{\pgfqpoint{1.773493in}{2.179992in}}%
\pgfpathcurveto{\pgfqpoint{1.767669in}{2.185816in}}{\pgfqpoint{1.759769in}{2.189089in}}{\pgfqpoint{1.751533in}{2.189089in}}%
\pgfpathcurveto{\pgfqpoint{1.743297in}{2.189089in}}{\pgfqpoint{1.735397in}{2.185816in}}{\pgfqpoint{1.729573in}{2.179992in}}%
\pgfpathcurveto{\pgfqpoint{1.723749in}{2.174168in}}{\pgfqpoint{1.720477in}{2.166268in}}{\pgfqpoint{1.720477in}{2.158032in}}%
\pgfpathcurveto{\pgfqpoint{1.720477in}{2.149796in}}{\pgfqpoint{1.723749in}{2.141896in}}{\pgfqpoint{1.729573in}{2.136072in}}%
\pgfpathcurveto{\pgfqpoint{1.735397in}{2.130248in}}{\pgfqpoint{1.743297in}{2.126976in}}{\pgfqpoint{1.751533in}{2.126976in}}%
\pgfpathclose%
\pgfusepath{stroke,fill}%
\end{pgfscope}%
\begin{pgfscope}%
\pgfpathrectangle{\pgfqpoint{0.100000in}{0.212622in}}{\pgfqpoint{3.696000in}{3.696000in}}%
\pgfusepath{clip}%
\pgfsetbuttcap%
\pgfsetroundjoin%
\definecolor{currentfill}{rgb}{0.121569,0.466667,0.705882}%
\pgfsetfillcolor{currentfill}%
\pgfsetfillopacity{0.365666}%
\pgfsetlinewidth{1.003750pt}%
\definecolor{currentstroke}{rgb}{0.121569,0.466667,0.705882}%
\pgfsetstrokecolor{currentstroke}%
\pgfsetstrokeopacity{0.365666}%
\pgfsetdash{}{0pt}%
\pgfpathmoveto{\pgfqpoint{1.867183in}{2.190228in}}%
\pgfpathcurveto{\pgfqpoint{1.875419in}{2.190228in}}{\pgfqpoint{1.883319in}{2.193501in}}{\pgfqpoint{1.889143in}{2.199325in}}%
\pgfpathcurveto{\pgfqpoint{1.894967in}{2.205148in}}{\pgfqpoint{1.898239in}{2.213048in}}{\pgfqpoint{1.898239in}{2.221285in}}%
\pgfpathcurveto{\pgfqpoint{1.898239in}{2.229521in}}{\pgfqpoint{1.894967in}{2.237421in}}{\pgfqpoint{1.889143in}{2.243245in}}%
\pgfpathcurveto{\pgfqpoint{1.883319in}{2.249069in}}{\pgfqpoint{1.875419in}{2.252341in}}{\pgfqpoint{1.867183in}{2.252341in}}%
\pgfpathcurveto{\pgfqpoint{1.858947in}{2.252341in}}{\pgfqpoint{1.851047in}{2.249069in}}{\pgfqpoint{1.845223in}{2.243245in}}%
\pgfpathcurveto{\pgfqpoint{1.839399in}{2.237421in}}{\pgfqpoint{1.836126in}{2.229521in}}{\pgfqpoint{1.836126in}{2.221285in}}%
\pgfpathcurveto{\pgfqpoint{1.836126in}{2.213048in}}{\pgfqpoint{1.839399in}{2.205148in}}{\pgfqpoint{1.845223in}{2.199325in}}%
\pgfpathcurveto{\pgfqpoint{1.851047in}{2.193501in}}{\pgfqpoint{1.858947in}{2.190228in}}{\pgfqpoint{1.867183in}{2.190228in}}%
\pgfpathclose%
\pgfusepath{stroke,fill}%
\end{pgfscope}%
\begin{pgfscope}%
\pgfpathrectangle{\pgfqpoint{0.100000in}{0.212622in}}{\pgfqpoint{3.696000in}{3.696000in}}%
\pgfusepath{clip}%
\pgfsetbuttcap%
\pgfsetroundjoin%
\definecolor{currentfill}{rgb}{0.121569,0.466667,0.705882}%
\pgfsetfillcolor{currentfill}%
\pgfsetfillopacity{0.365789}%
\pgfsetlinewidth{1.003750pt}%
\definecolor{currentstroke}{rgb}{0.121569,0.466667,0.705882}%
\pgfsetstrokecolor{currentstroke}%
\pgfsetstrokeopacity{0.365789}%
\pgfsetdash{}{0pt}%
\pgfpathmoveto{\pgfqpoint{1.924169in}{2.227919in}}%
\pgfpathcurveto{\pgfqpoint{1.932406in}{2.227919in}}{\pgfqpoint{1.940306in}{2.231191in}}{\pgfqpoint{1.946130in}{2.237015in}}%
\pgfpathcurveto{\pgfqpoint{1.951954in}{2.242839in}}{\pgfqpoint{1.955226in}{2.250739in}}{\pgfqpoint{1.955226in}{2.258976in}}%
\pgfpathcurveto{\pgfqpoint{1.955226in}{2.267212in}}{\pgfqpoint{1.951954in}{2.275112in}}{\pgfqpoint{1.946130in}{2.280936in}}%
\pgfpathcurveto{\pgfqpoint{1.940306in}{2.286760in}}{\pgfqpoint{1.932406in}{2.290032in}}{\pgfqpoint{1.924169in}{2.290032in}}%
\pgfpathcurveto{\pgfqpoint{1.915933in}{2.290032in}}{\pgfqpoint{1.908033in}{2.286760in}}{\pgfqpoint{1.902209in}{2.280936in}}%
\pgfpathcurveto{\pgfqpoint{1.896385in}{2.275112in}}{\pgfqpoint{1.893113in}{2.267212in}}{\pgfqpoint{1.893113in}{2.258976in}}%
\pgfpathcurveto{\pgfqpoint{1.893113in}{2.250739in}}{\pgfqpoint{1.896385in}{2.242839in}}{\pgfqpoint{1.902209in}{2.237015in}}%
\pgfpathcurveto{\pgfqpoint{1.908033in}{2.231191in}}{\pgfqpoint{1.915933in}{2.227919in}}{\pgfqpoint{1.924169in}{2.227919in}}%
\pgfpathclose%
\pgfusepath{stroke,fill}%
\end{pgfscope}%
\begin{pgfscope}%
\pgfpathrectangle{\pgfqpoint{0.100000in}{0.212622in}}{\pgfqpoint{3.696000in}{3.696000in}}%
\pgfusepath{clip}%
\pgfsetbuttcap%
\pgfsetroundjoin%
\definecolor{currentfill}{rgb}{0.121569,0.466667,0.705882}%
\pgfsetfillcolor{currentfill}%
\pgfsetfillopacity{0.365839}%
\pgfsetlinewidth{1.003750pt}%
\definecolor{currentstroke}{rgb}{0.121569,0.466667,0.705882}%
\pgfsetstrokecolor{currentstroke}%
\pgfsetstrokeopacity{0.365839}%
\pgfsetdash{}{0pt}%
\pgfpathmoveto{\pgfqpoint{1.858332in}{2.185230in}}%
\pgfpathcurveto{\pgfqpoint{1.866568in}{2.185230in}}{\pgfqpoint{1.874468in}{2.188502in}}{\pgfqpoint{1.880292in}{2.194326in}}%
\pgfpathcurveto{\pgfqpoint{1.886116in}{2.200150in}}{\pgfqpoint{1.889388in}{2.208050in}}{\pgfqpoint{1.889388in}{2.216287in}}%
\pgfpathcurveto{\pgfqpoint{1.889388in}{2.224523in}}{\pgfqpoint{1.886116in}{2.232423in}}{\pgfqpoint{1.880292in}{2.238247in}}%
\pgfpathcurveto{\pgfqpoint{1.874468in}{2.244071in}}{\pgfqpoint{1.866568in}{2.247343in}}{\pgfqpoint{1.858332in}{2.247343in}}%
\pgfpathcurveto{\pgfqpoint{1.850096in}{2.247343in}}{\pgfqpoint{1.842196in}{2.244071in}}{\pgfqpoint{1.836372in}{2.238247in}}%
\pgfpathcurveto{\pgfqpoint{1.830548in}{2.232423in}}{\pgfqpoint{1.827275in}{2.224523in}}{\pgfqpoint{1.827275in}{2.216287in}}%
\pgfpathcurveto{\pgfqpoint{1.827275in}{2.208050in}}{\pgfqpoint{1.830548in}{2.200150in}}{\pgfqpoint{1.836372in}{2.194326in}}%
\pgfpathcurveto{\pgfqpoint{1.842196in}{2.188502in}}{\pgfqpoint{1.850096in}{2.185230in}}{\pgfqpoint{1.858332in}{2.185230in}}%
\pgfpathclose%
\pgfusepath{stroke,fill}%
\end{pgfscope}%
\begin{pgfscope}%
\pgfpathrectangle{\pgfqpoint{0.100000in}{0.212622in}}{\pgfqpoint{3.696000in}{3.696000in}}%
\pgfusepath{clip}%
\pgfsetbuttcap%
\pgfsetroundjoin%
\definecolor{currentfill}{rgb}{0.121569,0.466667,0.705882}%
\pgfsetfillcolor{currentfill}%
\pgfsetfillopacity{0.365907}%
\pgfsetlinewidth{1.003750pt}%
\definecolor{currentstroke}{rgb}{0.121569,0.466667,0.705882}%
\pgfsetstrokecolor{currentstroke}%
\pgfsetstrokeopacity{0.365907}%
\pgfsetdash{}{0pt}%
\pgfpathmoveto{\pgfqpoint{1.920879in}{2.230890in}}%
\pgfpathcurveto{\pgfqpoint{1.929115in}{2.230890in}}{\pgfqpoint{1.937015in}{2.234162in}}{\pgfqpoint{1.942839in}{2.239986in}}%
\pgfpathcurveto{\pgfqpoint{1.948663in}{2.245810in}}{\pgfqpoint{1.951935in}{2.253710in}}{\pgfqpoint{1.951935in}{2.261946in}}%
\pgfpathcurveto{\pgfqpoint{1.951935in}{2.270182in}}{\pgfqpoint{1.948663in}{2.278082in}}{\pgfqpoint{1.942839in}{2.283906in}}%
\pgfpathcurveto{\pgfqpoint{1.937015in}{2.289730in}}{\pgfqpoint{1.929115in}{2.293003in}}{\pgfqpoint{1.920879in}{2.293003in}}%
\pgfpathcurveto{\pgfqpoint{1.912642in}{2.293003in}}{\pgfqpoint{1.904742in}{2.289730in}}{\pgfqpoint{1.898918in}{2.283906in}}%
\pgfpathcurveto{\pgfqpoint{1.893094in}{2.278082in}}{\pgfqpoint{1.889822in}{2.270182in}}{\pgfqpoint{1.889822in}{2.261946in}}%
\pgfpathcurveto{\pgfqpoint{1.889822in}{2.253710in}}{\pgfqpoint{1.893094in}{2.245810in}}{\pgfqpoint{1.898918in}{2.239986in}}%
\pgfpathcurveto{\pgfqpoint{1.904742in}{2.234162in}}{\pgfqpoint{1.912642in}{2.230890in}}{\pgfqpoint{1.920879in}{2.230890in}}%
\pgfpathclose%
\pgfusepath{stroke,fill}%
\end{pgfscope}%
\begin{pgfscope}%
\pgfpathrectangle{\pgfqpoint{0.100000in}{0.212622in}}{\pgfqpoint{3.696000in}{3.696000in}}%
\pgfusepath{clip}%
\pgfsetbuttcap%
\pgfsetroundjoin%
\definecolor{currentfill}{rgb}{0.121569,0.466667,0.705882}%
\pgfsetfillcolor{currentfill}%
\pgfsetfillopacity{0.366362}%
\pgfsetlinewidth{1.003750pt}%
\definecolor{currentstroke}{rgb}{0.121569,0.466667,0.705882}%
\pgfsetstrokecolor{currentstroke}%
\pgfsetstrokeopacity{0.366362}%
\pgfsetdash{}{0pt}%
\pgfpathmoveto{\pgfqpoint{1.851779in}{2.182318in}}%
\pgfpathcurveto{\pgfqpoint{1.860015in}{2.182318in}}{\pgfqpoint{1.867915in}{2.185591in}}{\pgfqpoint{1.873739in}{2.191415in}}%
\pgfpathcurveto{\pgfqpoint{1.879563in}{2.197239in}}{\pgfqpoint{1.882835in}{2.205139in}}{\pgfqpoint{1.882835in}{2.213375in}}%
\pgfpathcurveto{\pgfqpoint{1.882835in}{2.221611in}}{\pgfqpoint{1.879563in}{2.229511in}}{\pgfqpoint{1.873739in}{2.235335in}}%
\pgfpathcurveto{\pgfqpoint{1.867915in}{2.241159in}}{\pgfqpoint{1.860015in}{2.244431in}}{\pgfqpoint{1.851779in}{2.244431in}}%
\pgfpathcurveto{\pgfqpoint{1.843542in}{2.244431in}}{\pgfqpoint{1.835642in}{2.241159in}}{\pgfqpoint{1.829818in}{2.235335in}}%
\pgfpathcurveto{\pgfqpoint{1.823994in}{2.229511in}}{\pgfqpoint{1.820722in}{2.221611in}}{\pgfqpoint{1.820722in}{2.213375in}}%
\pgfpathcurveto{\pgfqpoint{1.820722in}{2.205139in}}{\pgfqpoint{1.823994in}{2.197239in}}{\pgfqpoint{1.829818in}{2.191415in}}%
\pgfpathcurveto{\pgfqpoint{1.835642in}{2.185591in}}{\pgfqpoint{1.843542in}{2.182318in}}{\pgfqpoint{1.851779in}{2.182318in}}%
\pgfpathclose%
\pgfusepath{stroke,fill}%
\end{pgfscope}%
\begin{pgfscope}%
\pgfpathrectangle{\pgfqpoint{0.100000in}{0.212622in}}{\pgfqpoint{3.696000in}{3.696000in}}%
\pgfusepath{clip}%
\pgfsetbuttcap%
\pgfsetroundjoin%
\definecolor{currentfill}{rgb}{0.121569,0.466667,0.705882}%
\pgfsetfillcolor{currentfill}%
\pgfsetfillopacity{0.366501}%
\pgfsetlinewidth{1.003750pt}%
\definecolor{currentstroke}{rgb}{0.121569,0.466667,0.705882}%
\pgfsetstrokecolor{currentstroke}%
\pgfsetstrokeopacity{0.366501}%
\pgfsetdash{}{0pt}%
\pgfpathmoveto{\pgfqpoint{1.461401in}{1.925477in}}%
\pgfpathcurveto{\pgfqpoint{1.469638in}{1.925477in}}{\pgfqpoint{1.477538in}{1.928749in}}{\pgfqpoint{1.483362in}{1.934573in}}%
\pgfpathcurveto{\pgfqpoint{1.489186in}{1.940397in}}{\pgfqpoint{1.492458in}{1.948297in}}{\pgfqpoint{1.492458in}{1.956533in}}%
\pgfpathcurveto{\pgfqpoint{1.492458in}{1.964769in}}{\pgfqpoint{1.489186in}{1.972669in}}{\pgfqpoint{1.483362in}{1.978493in}}%
\pgfpathcurveto{\pgfqpoint{1.477538in}{1.984317in}}{\pgfqpoint{1.469638in}{1.987590in}}{\pgfqpoint{1.461401in}{1.987590in}}%
\pgfpathcurveto{\pgfqpoint{1.453165in}{1.987590in}}{\pgfqpoint{1.445265in}{1.984317in}}{\pgfqpoint{1.439441in}{1.978493in}}%
\pgfpathcurveto{\pgfqpoint{1.433617in}{1.972669in}}{\pgfqpoint{1.430345in}{1.964769in}}{\pgfqpoint{1.430345in}{1.956533in}}%
\pgfpathcurveto{\pgfqpoint{1.430345in}{1.948297in}}{\pgfqpoint{1.433617in}{1.940397in}}{\pgfqpoint{1.439441in}{1.934573in}}%
\pgfpathcurveto{\pgfqpoint{1.445265in}{1.928749in}}{\pgfqpoint{1.453165in}{1.925477in}}{\pgfqpoint{1.461401in}{1.925477in}}%
\pgfpathclose%
\pgfusepath{stroke,fill}%
\end{pgfscope}%
\begin{pgfscope}%
\pgfpathrectangle{\pgfqpoint{0.100000in}{0.212622in}}{\pgfqpoint{3.696000in}{3.696000in}}%
\pgfusepath{clip}%
\pgfsetbuttcap%
\pgfsetroundjoin%
\definecolor{currentfill}{rgb}{0.121569,0.466667,0.705882}%
\pgfsetfillcolor{currentfill}%
\pgfsetfillopacity{0.366696}%
\pgfsetlinewidth{1.003750pt}%
\definecolor{currentstroke}{rgb}{0.121569,0.466667,0.705882}%
\pgfsetstrokecolor{currentstroke}%
\pgfsetstrokeopacity{0.366696}%
\pgfsetdash{}{0pt}%
\pgfpathmoveto{\pgfqpoint{1.852757in}{2.182328in}}%
\pgfpathcurveto{\pgfqpoint{1.860993in}{2.182328in}}{\pgfqpoint{1.868893in}{2.185600in}}{\pgfqpoint{1.874717in}{2.191424in}}%
\pgfpathcurveto{\pgfqpoint{1.880541in}{2.197248in}}{\pgfqpoint{1.883814in}{2.205148in}}{\pgfqpoint{1.883814in}{2.213384in}}%
\pgfpathcurveto{\pgfqpoint{1.883814in}{2.221621in}}{\pgfqpoint{1.880541in}{2.229521in}}{\pgfqpoint{1.874717in}{2.235345in}}%
\pgfpathcurveto{\pgfqpoint{1.868893in}{2.241168in}}{\pgfqpoint{1.860993in}{2.244441in}}{\pgfqpoint{1.852757in}{2.244441in}}%
\pgfpathcurveto{\pgfqpoint{1.844521in}{2.244441in}}{\pgfqpoint{1.836621in}{2.241168in}}{\pgfqpoint{1.830797in}{2.235345in}}%
\pgfpathcurveto{\pgfqpoint{1.824973in}{2.229521in}}{\pgfqpoint{1.821701in}{2.221621in}}{\pgfqpoint{1.821701in}{2.213384in}}%
\pgfpathcurveto{\pgfqpoint{1.821701in}{2.205148in}}{\pgfqpoint{1.824973in}{2.197248in}}{\pgfqpoint{1.830797in}{2.191424in}}%
\pgfpathcurveto{\pgfqpoint{1.836621in}{2.185600in}}{\pgfqpoint{1.844521in}{2.182328in}}{\pgfqpoint{1.852757in}{2.182328in}}%
\pgfpathclose%
\pgfusepath{stroke,fill}%
\end{pgfscope}%
\begin{pgfscope}%
\pgfpathrectangle{\pgfqpoint{0.100000in}{0.212622in}}{\pgfqpoint{3.696000in}{3.696000in}}%
\pgfusepath{clip}%
\pgfsetbuttcap%
\pgfsetroundjoin%
\definecolor{currentfill}{rgb}{0.121569,0.466667,0.705882}%
\pgfsetfillcolor{currentfill}%
\pgfsetfillopacity{0.366931}%
\pgfsetlinewidth{1.003750pt}%
\definecolor{currentstroke}{rgb}{0.121569,0.466667,0.705882}%
\pgfsetstrokecolor{currentstroke}%
\pgfsetstrokeopacity{0.366931}%
\pgfsetdash{}{0pt}%
\pgfpathmoveto{\pgfqpoint{1.753796in}{2.122987in}}%
\pgfpathcurveto{\pgfqpoint{1.762032in}{2.122987in}}{\pgfqpoint{1.769933in}{2.126260in}}{\pgfqpoint{1.775756in}{2.132083in}}%
\pgfpathcurveto{\pgfqpoint{1.781580in}{2.137907in}}{\pgfqpoint{1.784853in}{2.145807in}}{\pgfqpoint{1.784853in}{2.154044in}}%
\pgfpathcurveto{\pgfqpoint{1.784853in}{2.162280in}}{\pgfqpoint{1.781580in}{2.170180in}}{\pgfqpoint{1.775756in}{2.176004in}}%
\pgfpathcurveto{\pgfqpoint{1.769933in}{2.181828in}}{\pgfqpoint{1.762032in}{2.185100in}}{\pgfqpoint{1.753796in}{2.185100in}}%
\pgfpathcurveto{\pgfqpoint{1.745560in}{2.185100in}}{\pgfqpoint{1.737660in}{2.181828in}}{\pgfqpoint{1.731836in}{2.176004in}}%
\pgfpathcurveto{\pgfqpoint{1.726012in}{2.170180in}}{\pgfqpoint{1.722740in}{2.162280in}}{\pgfqpoint{1.722740in}{2.154044in}}%
\pgfpathcurveto{\pgfqpoint{1.722740in}{2.145807in}}{\pgfqpoint{1.726012in}{2.137907in}}{\pgfqpoint{1.731836in}{2.132083in}}%
\pgfpathcurveto{\pgfqpoint{1.737660in}{2.126260in}}{\pgfqpoint{1.745560in}{2.122987in}}{\pgfqpoint{1.753796in}{2.122987in}}%
\pgfpathclose%
\pgfusepath{stroke,fill}%
\end{pgfscope}%
\begin{pgfscope}%
\pgfpathrectangle{\pgfqpoint{0.100000in}{0.212622in}}{\pgfqpoint{3.696000in}{3.696000in}}%
\pgfusepath{clip}%
\pgfsetbuttcap%
\pgfsetroundjoin%
\definecolor{currentfill}{rgb}{0.121569,0.466667,0.705882}%
\pgfsetfillcolor{currentfill}%
\pgfsetfillopacity{0.367386}%
\pgfsetlinewidth{1.003750pt}%
\definecolor{currentstroke}{rgb}{0.121569,0.466667,0.705882}%
\pgfsetstrokecolor{currentstroke}%
\pgfsetstrokeopacity{0.367386}%
\pgfsetdash{}{0pt}%
\pgfpathmoveto{\pgfqpoint{1.431715in}{1.901707in}}%
\pgfpathcurveto{\pgfqpoint{1.439951in}{1.901707in}}{\pgfqpoint{1.447851in}{1.904980in}}{\pgfqpoint{1.453675in}{1.910804in}}%
\pgfpathcurveto{\pgfqpoint{1.459499in}{1.916628in}}{\pgfqpoint{1.462771in}{1.924528in}}{\pgfqpoint{1.462771in}{1.932764in}}%
\pgfpathcurveto{\pgfqpoint{1.462771in}{1.941000in}}{\pgfqpoint{1.459499in}{1.948900in}}{\pgfqpoint{1.453675in}{1.954724in}}%
\pgfpathcurveto{\pgfqpoint{1.447851in}{1.960548in}}{\pgfqpoint{1.439951in}{1.963820in}}{\pgfqpoint{1.431715in}{1.963820in}}%
\pgfpathcurveto{\pgfqpoint{1.423479in}{1.963820in}}{\pgfqpoint{1.415579in}{1.960548in}}{\pgfqpoint{1.409755in}{1.954724in}}%
\pgfpathcurveto{\pgfqpoint{1.403931in}{1.948900in}}{\pgfqpoint{1.400658in}{1.941000in}}{\pgfqpoint{1.400658in}{1.932764in}}%
\pgfpathcurveto{\pgfqpoint{1.400658in}{1.924528in}}{\pgfqpoint{1.403931in}{1.916628in}}{\pgfqpoint{1.409755in}{1.910804in}}%
\pgfpathcurveto{\pgfqpoint{1.415579in}{1.904980in}}{\pgfqpoint{1.423479in}{1.901707in}}{\pgfqpoint{1.431715in}{1.901707in}}%
\pgfpathclose%
\pgfusepath{stroke,fill}%
\end{pgfscope}%
\begin{pgfscope}%
\pgfpathrectangle{\pgfqpoint{0.100000in}{0.212622in}}{\pgfqpoint{3.696000in}{3.696000in}}%
\pgfusepath{clip}%
\pgfsetbuttcap%
\pgfsetroundjoin%
\definecolor{currentfill}{rgb}{0.121569,0.466667,0.705882}%
\pgfsetfillcolor{currentfill}%
\pgfsetfillopacity{0.367431}%
\pgfsetlinewidth{1.003750pt}%
\definecolor{currentstroke}{rgb}{0.121569,0.466667,0.705882}%
\pgfsetstrokecolor{currentstroke}%
\pgfsetstrokeopacity{0.367431}%
\pgfsetdash{}{0pt}%
\pgfpathmoveto{\pgfqpoint{1.869657in}{2.192537in}}%
\pgfpathcurveto{\pgfqpoint{1.877894in}{2.192537in}}{\pgfqpoint{1.885794in}{2.195809in}}{\pgfqpoint{1.891617in}{2.201633in}}%
\pgfpathcurveto{\pgfqpoint{1.897441in}{2.207457in}}{\pgfqpoint{1.900714in}{2.215357in}}{\pgfqpoint{1.900714in}{2.223593in}}%
\pgfpathcurveto{\pgfqpoint{1.900714in}{2.231830in}}{\pgfqpoint{1.897441in}{2.239730in}}{\pgfqpoint{1.891617in}{2.245554in}}%
\pgfpathcurveto{\pgfqpoint{1.885794in}{2.251378in}}{\pgfqpoint{1.877894in}{2.254650in}}{\pgfqpoint{1.869657in}{2.254650in}}%
\pgfpathcurveto{\pgfqpoint{1.861421in}{2.254650in}}{\pgfqpoint{1.853521in}{2.251378in}}{\pgfqpoint{1.847697in}{2.245554in}}%
\pgfpathcurveto{\pgfqpoint{1.841873in}{2.239730in}}{\pgfqpoint{1.838601in}{2.231830in}}{\pgfqpoint{1.838601in}{2.223593in}}%
\pgfpathcurveto{\pgfqpoint{1.838601in}{2.215357in}}{\pgfqpoint{1.841873in}{2.207457in}}{\pgfqpoint{1.847697in}{2.201633in}}%
\pgfpathcurveto{\pgfqpoint{1.853521in}{2.195809in}}{\pgfqpoint{1.861421in}{2.192537in}}{\pgfqpoint{1.869657in}{2.192537in}}%
\pgfpathclose%
\pgfusepath{stroke,fill}%
\end{pgfscope}%
\begin{pgfscope}%
\pgfpathrectangle{\pgfqpoint{0.100000in}{0.212622in}}{\pgfqpoint{3.696000in}{3.696000in}}%
\pgfusepath{clip}%
\pgfsetbuttcap%
\pgfsetroundjoin%
\definecolor{currentfill}{rgb}{0.121569,0.466667,0.705882}%
\pgfsetfillcolor{currentfill}%
\pgfsetfillopacity{0.367462}%
\pgfsetlinewidth{1.003750pt}%
\definecolor{currentstroke}{rgb}{0.121569,0.466667,0.705882}%
\pgfsetstrokecolor{currentstroke}%
\pgfsetstrokeopacity{0.367462}%
\pgfsetdash{}{0pt}%
\pgfpathmoveto{\pgfqpoint{1.463546in}{1.927121in}}%
\pgfpathcurveto{\pgfqpoint{1.471782in}{1.927121in}}{\pgfqpoint{1.479682in}{1.930393in}}{\pgfqpoint{1.485506in}{1.936217in}}%
\pgfpathcurveto{\pgfqpoint{1.491330in}{1.942041in}}{\pgfqpoint{1.494603in}{1.949941in}}{\pgfqpoint{1.494603in}{1.958177in}}%
\pgfpathcurveto{\pgfqpoint{1.494603in}{1.966413in}}{\pgfqpoint{1.491330in}{1.974313in}}{\pgfqpoint{1.485506in}{1.980137in}}%
\pgfpathcurveto{\pgfqpoint{1.479682in}{1.985961in}}{\pgfqpoint{1.471782in}{1.989234in}}{\pgfqpoint{1.463546in}{1.989234in}}%
\pgfpathcurveto{\pgfqpoint{1.455310in}{1.989234in}}{\pgfqpoint{1.447410in}{1.985961in}}{\pgfqpoint{1.441586in}{1.980137in}}%
\pgfpathcurveto{\pgfqpoint{1.435762in}{1.974313in}}{\pgfqpoint{1.432490in}{1.966413in}}{\pgfqpoint{1.432490in}{1.958177in}}%
\pgfpathcurveto{\pgfqpoint{1.432490in}{1.949941in}}{\pgfqpoint{1.435762in}{1.942041in}}{\pgfqpoint{1.441586in}{1.936217in}}%
\pgfpathcurveto{\pgfqpoint{1.447410in}{1.930393in}}{\pgfqpoint{1.455310in}{1.927121in}}{\pgfqpoint{1.463546in}{1.927121in}}%
\pgfpathclose%
\pgfusepath{stroke,fill}%
\end{pgfscope}%
\begin{pgfscope}%
\pgfpathrectangle{\pgfqpoint{0.100000in}{0.212622in}}{\pgfqpoint{3.696000in}{3.696000in}}%
\pgfusepath{clip}%
\pgfsetbuttcap%
\pgfsetroundjoin%
\definecolor{currentfill}{rgb}{0.121569,0.466667,0.705882}%
\pgfsetfillcolor{currentfill}%
\pgfsetfillopacity{0.367703}%
\pgfsetlinewidth{1.003750pt}%
\definecolor{currentstroke}{rgb}{0.121569,0.466667,0.705882}%
\pgfsetstrokecolor{currentstroke}%
\pgfsetstrokeopacity{0.367703}%
\pgfsetdash{}{0pt}%
\pgfpathmoveto{\pgfqpoint{1.867149in}{2.190872in}}%
\pgfpathcurveto{\pgfqpoint{1.875386in}{2.190872in}}{\pgfqpoint{1.883286in}{2.194144in}}{\pgfqpoint{1.889110in}{2.199968in}}%
\pgfpathcurveto{\pgfqpoint{1.894934in}{2.205792in}}{\pgfqpoint{1.898206in}{2.213692in}}{\pgfqpoint{1.898206in}{2.221928in}}%
\pgfpathcurveto{\pgfqpoint{1.898206in}{2.230164in}}{\pgfqpoint{1.894934in}{2.238064in}}{\pgfqpoint{1.889110in}{2.243888in}}%
\pgfpathcurveto{\pgfqpoint{1.883286in}{2.249712in}}{\pgfqpoint{1.875386in}{2.252985in}}{\pgfqpoint{1.867149in}{2.252985in}}%
\pgfpathcurveto{\pgfqpoint{1.858913in}{2.252985in}}{\pgfqpoint{1.851013in}{2.249712in}}{\pgfqpoint{1.845189in}{2.243888in}}%
\pgfpathcurveto{\pgfqpoint{1.839365in}{2.238064in}}{\pgfqpoint{1.836093in}{2.230164in}}{\pgfqpoint{1.836093in}{2.221928in}}%
\pgfpathcurveto{\pgfqpoint{1.836093in}{2.213692in}}{\pgfqpoint{1.839365in}{2.205792in}}{\pgfqpoint{1.845189in}{2.199968in}}%
\pgfpathcurveto{\pgfqpoint{1.851013in}{2.194144in}}{\pgfqpoint{1.858913in}{2.190872in}}{\pgfqpoint{1.867149in}{2.190872in}}%
\pgfpathclose%
\pgfusepath{stroke,fill}%
\end{pgfscope}%
\begin{pgfscope}%
\pgfpathrectangle{\pgfqpoint{0.100000in}{0.212622in}}{\pgfqpoint{3.696000in}{3.696000in}}%
\pgfusepath{clip}%
\pgfsetbuttcap%
\pgfsetroundjoin%
\definecolor{currentfill}{rgb}{0.121569,0.466667,0.705882}%
\pgfsetfillcolor{currentfill}%
\pgfsetfillopacity{0.367719}%
\pgfsetlinewidth{1.003750pt}%
\definecolor{currentstroke}{rgb}{0.121569,0.466667,0.705882}%
\pgfsetstrokecolor{currentstroke}%
\pgfsetstrokeopacity{0.367719}%
\pgfsetdash{}{0pt}%
\pgfpathmoveto{\pgfqpoint{1.449593in}{1.914911in}}%
\pgfpathcurveto{\pgfqpoint{1.457829in}{1.914911in}}{\pgfqpoint{1.465730in}{1.918183in}}{\pgfqpoint{1.471553in}{1.924007in}}%
\pgfpathcurveto{\pgfqpoint{1.477377in}{1.929831in}}{\pgfqpoint{1.480650in}{1.937731in}}{\pgfqpoint{1.480650in}{1.945967in}}%
\pgfpathcurveto{\pgfqpoint{1.480650in}{1.954203in}}{\pgfqpoint{1.477377in}{1.962103in}}{\pgfqpoint{1.471553in}{1.967927in}}%
\pgfpathcurveto{\pgfqpoint{1.465730in}{1.973751in}}{\pgfqpoint{1.457829in}{1.977024in}}{\pgfqpoint{1.449593in}{1.977024in}}%
\pgfpathcurveto{\pgfqpoint{1.441357in}{1.977024in}}{\pgfqpoint{1.433457in}{1.973751in}}{\pgfqpoint{1.427633in}{1.967927in}}%
\pgfpathcurveto{\pgfqpoint{1.421809in}{1.962103in}}{\pgfqpoint{1.418537in}{1.954203in}}{\pgfqpoint{1.418537in}{1.945967in}}%
\pgfpathcurveto{\pgfqpoint{1.418537in}{1.937731in}}{\pgfqpoint{1.421809in}{1.929831in}}{\pgfqpoint{1.427633in}{1.924007in}}%
\pgfpathcurveto{\pgfqpoint{1.433457in}{1.918183in}}{\pgfqpoint{1.441357in}{1.914911in}}{\pgfqpoint{1.449593in}{1.914911in}}%
\pgfpathclose%
\pgfusepath{stroke,fill}%
\end{pgfscope}%
\begin{pgfscope}%
\pgfpathrectangle{\pgfqpoint{0.100000in}{0.212622in}}{\pgfqpoint{3.696000in}{3.696000in}}%
\pgfusepath{clip}%
\pgfsetbuttcap%
\pgfsetroundjoin%
\definecolor{currentfill}{rgb}{0.121569,0.466667,0.705882}%
\pgfsetfillcolor{currentfill}%
\pgfsetfillopacity{0.367766}%
\pgfsetlinewidth{1.003750pt}%
\definecolor{currentstroke}{rgb}{0.121569,0.466667,0.705882}%
\pgfsetstrokecolor{currentstroke}%
\pgfsetstrokeopacity{0.367766}%
\pgfsetdash{}{0pt}%
\pgfpathmoveto{\pgfqpoint{1.930175in}{2.230792in}}%
\pgfpathcurveto{\pgfqpoint{1.938412in}{2.230792in}}{\pgfqpoint{1.946312in}{2.234065in}}{\pgfqpoint{1.952136in}{2.239889in}}%
\pgfpathcurveto{\pgfqpoint{1.957959in}{2.245713in}}{\pgfqpoint{1.961232in}{2.253613in}}{\pgfqpoint{1.961232in}{2.261849in}}%
\pgfpathcurveto{\pgfqpoint{1.961232in}{2.270085in}}{\pgfqpoint{1.957959in}{2.277985in}}{\pgfqpoint{1.952136in}{2.283809in}}%
\pgfpathcurveto{\pgfqpoint{1.946312in}{2.289633in}}{\pgfqpoint{1.938412in}{2.292905in}}{\pgfqpoint{1.930175in}{2.292905in}}%
\pgfpathcurveto{\pgfqpoint{1.921939in}{2.292905in}}{\pgfqpoint{1.914039in}{2.289633in}}{\pgfqpoint{1.908215in}{2.283809in}}%
\pgfpathcurveto{\pgfqpoint{1.902391in}{2.277985in}}{\pgfqpoint{1.899119in}{2.270085in}}{\pgfqpoint{1.899119in}{2.261849in}}%
\pgfpathcurveto{\pgfqpoint{1.899119in}{2.253613in}}{\pgfqpoint{1.902391in}{2.245713in}}{\pgfqpoint{1.908215in}{2.239889in}}%
\pgfpathcurveto{\pgfqpoint{1.914039in}{2.234065in}}{\pgfqpoint{1.921939in}{2.230792in}}{\pgfqpoint{1.930175in}{2.230792in}}%
\pgfpathclose%
\pgfusepath{stroke,fill}%
\end{pgfscope}%
\begin{pgfscope}%
\pgfpathrectangle{\pgfqpoint{0.100000in}{0.212622in}}{\pgfqpoint{3.696000in}{3.696000in}}%
\pgfusepath{clip}%
\pgfsetbuttcap%
\pgfsetroundjoin%
\definecolor{currentfill}{rgb}{0.121569,0.466667,0.705882}%
\pgfsetfillcolor{currentfill}%
\pgfsetfillopacity{0.367849}%
\pgfsetlinewidth{1.003750pt}%
\definecolor{currentstroke}{rgb}{0.121569,0.466667,0.705882}%
\pgfsetstrokecolor{currentstroke}%
\pgfsetstrokeopacity{0.367849}%
\pgfsetdash{}{0pt}%
\pgfpathmoveto{\pgfqpoint{1.466605in}{1.929325in}}%
\pgfpathcurveto{\pgfqpoint{1.474841in}{1.929325in}}{\pgfqpoint{1.482741in}{1.932597in}}{\pgfqpoint{1.488565in}{1.938421in}}%
\pgfpathcurveto{\pgfqpoint{1.494389in}{1.944245in}}{\pgfqpoint{1.497661in}{1.952145in}}{\pgfqpoint{1.497661in}{1.960381in}}%
\pgfpathcurveto{\pgfqpoint{1.497661in}{1.968618in}}{\pgfqpoint{1.494389in}{1.976518in}}{\pgfqpoint{1.488565in}{1.982342in}}%
\pgfpathcurveto{\pgfqpoint{1.482741in}{1.988166in}}{\pgfqpoint{1.474841in}{1.991438in}}{\pgfqpoint{1.466605in}{1.991438in}}%
\pgfpathcurveto{\pgfqpoint{1.458369in}{1.991438in}}{\pgfqpoint{1.450469in}{1.988166in}}{\pgfqpoint{1.444645in}{1.982342in}}%
\pgfpathcurveto{\pgfqpoint{1.438821in}{1.976518in}}{\pgfqpoint{1.435548in}{1.968618in}}{\pgfqpoint{1.435548in}{1.960381in}}%
\pgfpathcurveto{\pgfqpoint{1.435548in}{1.952145in}}{\pgfqpoint{1.438821in}{1.944245in}}{\pgfqpoint{1.444645in}{1.938421in}}%
\pgfpathcurveto{\pgfqpoint{1.450469in}{1.932597in}}{\pgfqpoint{1.458369in}{1.929325in}}{\pgfqpoint{1.466605in}{1.929325in}}%
\pgfpathclose%
\pgfusepath{stroke,fill}%
\end{pgfscope}%
\begin{pgfscope}%
\pgfpathrectangle{\pgfqpoint{0.100000in}{0.212622in}}{\pgfqpoint{3.696000in}{3.696000in}}%
\pgfusepath{clip}%
\pgfsetbuttcap%
\pgfsetroundjoin%
\definecolor{currentfill}{rgb}{0.121569,0.466667,0.705882}%
\pgfsetfillcolor{currentfill}%
\pgfsetfillopacity{0.367890}%
\pgfsetlinewidth{1.003750pt}%
\definecolor{currentstroke}{rgb}{0.121569,0.466667,0.705882}%
\pgfsetstrokecolor{currentstroke}%
\pgfsetstrokeopacity{0.367890}%
\pgfsetdash{}{0pt}%
\pgfpathmoveto{\pgfqpoint{1.870157in}{2.191728in}}%
\pgfpathcurveto{\pgfqpoint{1.878394in}{2.191728in}}{\pgfqpoint{1.886294in}{2.195001in}}{\pgfqpoint{1.892118in}{2.200825in}}%
\pgfpathcurveto{\pgfqpoint{1.897941in}{2.206648in}}{\pgfqpoint{1.901214in}{2.214549in}}{\pgfqpoint{1.901214in}{2.222785in}}%
\pgfpathcurveto{\pgfqpoint{1.901214in}{2.231021in}}{\pgfqpoint{1.897941in}{2.238921in}}{\pgfqpoint{1.892118in}{2.244745in}}%
\pgfpathcurveto{\pgfqpoint{1.886294in}{2.250569in}}{\pgfqpoint{1.878394in}{2.253841in}}{\pgfqpoint{1.870157in}{2.253841in}}%
\pgfpathcurveto{\pgfqpoint{1.861921in}{2.253841in}}{\pgfqpoint{1.854021in}{2.250569in}}{\pgfqpoint{1.848197in}{2.244745in}}%
\pgfpathcurveto{\pgfqpoint{1.842373in}{2.238921in}}{\pgfqpoint{1.839101in}{2.231021in}}{\pgfqpoint{1.839101in}{2.222785in}}%
\pgfpathcurveto{\pgfqpoint{1.839101in}{2.214549in}}{\pgfqpoint{1.842373in}{2.206648in}}{\pgfqpoint{1.848197in}{2.200825in}}%
\pgfpathcurveto{\pgfqpoint{1.854021in}{2.195001in}}{\pgfqpoint{1.861921in}{2.191728in}}{\pgfqpoint{1.870157in}{2.191728in}}%
\pgfpathclose%
\pgfusepath{stroke,fill}%
\end{pgfscope}%
\begin{pgfscope}%
\pgfpathrectangle{\pgfqpoint{0.100000in}{0.212622in}}{\pgfqpoint{3.696000in}{3.696000in}}%
\pgfusepath{clip}%
\pgfsetbuttcap%
\pgfsetroundjoin%
\definecolor{currentfill}{rgb}{0.121569,0.466667,0.705882}%
\pgfsetfillcolor{currentfill}%
\pgfsetfillopacity{0.368041}%
\pgfsetlinewidth{1.003750pt}%
\definecolor{currentstroke}{rgb}{0.121569,0.466667,0.705882}%
\pgfsetstrokecolor{currentstroke}%
\pgfsetstrokeopacity{0.368041}%
\pgfsetdash{}{0pt}%
\pgfpathmoveto{\pgfqpoint{1.458409in}{1.922005in}}%
\pgfpathcurveto{\pgfqpoint{1.466645in}{1.922005in}}{\pgfqpoint{1.474545in}{1.925278in}}{\pgfqpoint{1.480369in}{1.931102in}}%
\pgfpathcurveto{\pgfqpoint{1.486193in}{1.936926in}}{\pgfqpoint{1.489465in}{1.944826in}}{\pgfqpoint{1.489465in}{1.953062in}}%
\pgfpathcurveto{\pgfqpoint{1.489465in}{1.961298in}}{\pgfqpoint{1.486193in}{1.969198in}}{\pgfqpoint{1.480369in}{1.975022in}}%
\pgfpathcurveto{\pgfqpoint{1.474545in}{1.980846in}}{\pgfqpoint{1.466645in}{1.984118in}}{\pgfqpoint{1.458409in}{1.984118in}}%
\pgfpathcurveto{\pgfqpoint{1.450172in}{1.984118in}}{\pgfqpoint{1.442272in}{1.980846in}}{\pgfqpoint{1.436448in}{1.975022in}}%
\pgfpathcurveto{\pgfqpoint{1.430624in}{1.969198in}}{\pgfqpoint{1.427352in}{1.961298in}}{\pgfqpoint{1.427352in}{1.953062in}}%
\pgfpathcurveto{\pgfqpoint{1.427352in}{1.944826in}}{\pgfqpoint{1.430624in}{1.936926in}}{\pgfqpoint{1.436448in}{1.931102in}}%
\pgfpathcurveto{\pgfqpoint{1.442272in}{1.925278in}}{\pgfqpoint{1.450172in}{1.922005in}}{\pgfqpoint{1.458409in}{1.922005in}}%
\pgfpathclose%
\pgfusepath{stroke,fill}%
\end{pgfscope}%
\begin{pgfscope}%
\pgfpathrectangle{\pgfqpoint{0.100000in}{0.212622in}}{\pgfqpoint{3.696000in}{3.696000in}}%
\pgfusepath{clip}%
\pgfsetbuttcap%
\pgfsetroundjoin%
\definecolor{currentfill}{rgb}{0.121569,0.466667,0.705882}%
\pgfsetfillcolor{currentfill}%
\pgfsetfillopacity{0.368226}%
\pgfsetlinewidth{1.003750pt}%
\definecolor{currentstroke}{rgb}{0.121569,0.466667,0.705882}%
\pgfsetstrokecolor{currentstroke}%
\pgfsetstrokeopacity{0.368226}%
\pgfsetdash{}{0pt}%
\pgfpathmoveto{\pgfqpoint{1.441797in}{1.908165in}}%
\pgfpathcurveto{\pgfqpoint{1.450033in}{1.908165in}}{\pgfqpoint{1.457933in}{1.911437in}}{\pgfqpoint{1.463757in}{1.917261in}}%
\pgfpathcurveto{\pgfqpoint{1.469581in}{1.923085in}}{\pgfqpoint{1.472853in}{1.930985in}}{\pgfqpoint{1.472853in}{1.939221in}}%
\pgfpathcurveto{\pgfqpoint{1.472853in}{1.947457in}}{\pgfqpoint{1.469581in}{1.955357in}}{\pgfqpoint{1.463757in}{1.961181in}}%
\pgfpathcurveto{\pgfqpoint{1.457933in}{1.967005in}}{\pgfqpoint{1.450033in}{1.970278in}}{\pgfqpoint{1.441797in}{1.970278in}}%
\pgfpathcurveto{\pgfqpoint{1.433560in}{1.970278in}}{\pgfqpoint{1.425660in}{1.967005in}}{\pgfqpoint{1.419836in}{1.961181in}}%
\pgfpathcurveto{\pgfqpoint{1.414012in}{1.955357in}}{\pgfqpoint{1.410740in}{1.947457in}}{\pgfqpoint{1.410740in}{1.939221in}}%
\pgfpathcurveto{\pgfqpoint{1.410740in}{1.930985in}}{\pgfqpoint{1.414012in}{1.923085in}}{\pgfqpoint{1.419836in}{1.917261in}}%
\pgfpathcurveto{\pgfqpoint{1.425660in}{1.911437in}}{\pgfqpoint{1.433560in}{1.908165in}}{\pgfqpoint{1.441797in}{1.908165in}}%
\pgfpathclose%
\pgfusepath{stroke,fill}%
\end{pgfscope}%
\begin{pgfscope}%
\pgfpathrectangle{\pgfqpoint{0.100000in}{0.212622in}}{\pgfqpoint{3.696000in}{3.696000in}}%
\pgfusepath{clip}%
\pgfsetbuttcap%
\pgfsetroundjoin%
\definecolor{currentfill}{rgb}{0.121569,0.466667,0.705882}%
\pgfsetfillcolor{currentfill}%
\pgfsetfillopacity{0.368231}%
\pgfsetlinewidth{1.003750pt}%
\definecolor{currentstroke}{rgb}{0.121569,0.466667,0.705882}%
\pgfsetstrokecolor{currentstroke}%
\pgfsetstrokeopacity{0.368231}%
\pgfsetdash{}{0pt}%
\pgfpathmoveto{\pgfqpoint{1.445876in}{1.912046in}}%
\pgfpathcurveto{\pgfqpoint{1.454113in}{1.912046in}}{\pgfqpoint{1.462013in}{1.915318in}}{\pgfqpoint{1.467837in}{1.921142in}}%
\pgfpathcurveto{\pgfqpoint{1.473661in}{1.926966in}}{\pgfqpoint{1.476933in}{1.934866in}}{\pgfqpoint{1.476933in}{1.943102in}}%
\pgfpathcurveto{\pgfqpoint{1.476933in}{1.951338in}}{\pgfqpoint{1.473661in}{1.959239in}}{\pgfqpoint{1.467837in}{1.965062in}}%
\pgfpathcurveto{\pgfqpoint{1.462013in}{1.970886in}}{\pgfqpoint{1.454113in}{1.974159in}}{\pgfqpoint{1.445876in}{1.974159in}}%
\pgfpathcurveto{\pgfqpoint{1.437640in}{1.974159in}}{\pgfqpoint{1.429740in}{1.970886in}}{\pgfqpoint{1.423916in}{1.965062in}}%
\pgfpathcurveto{\pgfqpoint{1.418092in}{1.959239in}}{\pgfqpoint{1.414820in}{1.951338in}}{\pgfqpoint{1.414820in}{1.943102in}}%
\pgfpathcurveto{\pgfqpoint{1.414820in}{1.934866in}}{\pgfqpoint{1.418092in}{1.926966in}}{\pgfqpoint{1.423916in}{1.921142in}}%
\pgfpathcurveto{\pgfqpoint{1.429740in}{1.915318in}}{\pgfqpoint{1.437640in}{1.912046in}}{\pgfqpoint{1.445876in}{1.912046in}}%
\pgfpathclose%
\pgfusepath{stroke,fill}%
\end{pgfscope}%
\begin{pgfscope}%
\pgfpathrectangle{\pgfqpoint{0.100000in}{0.212622in}}{\pgfqpoint{3.696000in}{3.696000in}}%
\pgfusepath{clip}%
\pgfsetbuttcap%
\pgfsetroundjoin%
\definecolor{currentfill}{rgb}{0.121569,0.466667,0.705882}%
\pgfsetfillcolor{currentfill}%
\pgfsetfillopacity{0.368421}%
\pgfsetlinewidth{1.003750pt}%
\definecolor{currentstroke}{rgb}{0.121569,0.466667,0.705882}%
\pgfsetstrokecolor{currentstroke}%
\pgfsetstrokeopacity{0.368421}%
\pgfsetdash{}{0pt}%
\pgfpathmoveto{\pgfqpoint{1.450051in}{1.913659in}}%
\pgfpathcurveto{\pgfqpoint{1.458288in}{1.913659in}}{\pgfqpoint{1.466188in}{1.916931in}}{\pgfqpoint{1.472012in}{1.922755in}}%
\pgfpathcurveto{\pgfqpoint{1.477836in}{1.928579in}}{\pgfqpoint{1.481108in}{1.936479in}}{\pgfqpoint{1.481108in}{1.944715in}}%
\pgfpathcurveto{\pgfqpoint{1.481108in}{1.952951in}}{\pgfqpoint{1.477836in}{1.960852in}}{\pgfqpoint{1.472012in}{1.966675in}}%
\pgfpathcurveto{\pgfqpoint{1.466188in}{1.972499in}}{\pgfqpoint{1.458288in}{1.975772in}}{\pgfqpoint{1.450051in}{1.975772in}}%
\pgfpathcurveto{\pgfqpoint{1.441815in}{1.975772in}}{\pgfqpoint{1.433915in}{1.972499in}}{\pgfqpoint{1.428091in}{1.966675in}}%
\pgfpathcurveto{\pgfqpoint{1.422267in}{1.960852in}}{\pgfqpoint{1.418995in}{1.952951in}}{\pgfqpoint{1.418995in}{1.944715in}}%
\pgfpathcurveto{\pgfqpoint{1.418995in}{1.936479in}}{\pgfqpoint{1.422267in}{1.928579in}}{\pgfqpoint{1.428091in}{1.922755in}}%
\pgfpathcurveto{\pgfqpoint{1.433915in}{1.916931in}}{\pgfqpoint{1.441815in}{1.913659in}}{\pgfqpoint{1.450051in}{1.913659in}}%
\pgfpathclose%
\pgfusepath{stroke,fill}%
\end{pgfscope}%
\begin{pgfscope}%
\pgfpathrectangle{\pgfqpoint{0.100000in}{0.212622in}}{\pgfqpoint{3.696000in}{3.696000in}}%
\pgfusepath{clip}%
\pgfsetbuttcap%
\pgfsetroundjoin%
\definecolor{currentfill}{rgb}{0.121569,0.466667,0.705882}%
\pgfsetfillcolor{currentfill}%
\pgfsetfillopacity{0.368812}%
\pgfsetlinewidth{1.003750pt}%
\definecolor{currentstroke}{rgb}{0.121569,0.466667,0.705882}%
\pgfsetstrokecolor{currentstroke}%
\pgfsetstrokeopacity{0.368812}%
\pgfsetdash{}{0pt}%
\pgfpathmoveto{\pgfqpoint{1.716533in}{2.101561in}}%
\pgfpathcurveto{\pgfqpoint{1.724770in}{2.101561in}}{\pgfqpoint{1.732670in}{2.104833in}}{\pgfqpoint{1.738494in}{2.110657in}}%
\pgfpathcurveto{\pgfqpoint{1.744318in}{2.116481in}}{\pgfqpoint{1.747590in}{2.124381in}}{\pgfqpoint{1.747590in}{2.132617in}}%
\pgfpathcurveto{\pgfqpoint{1.747590in}{2.140853in}}{\pgfqpoint{1.744318in}{2.148753in}}{\pgfqpoint{1.738494in}{2.154577in}}%
\pgfpathcurveto{\pgfqpoint{1.732670in}{2.160401in}}{\pgfqpoint{1.724770in}{2.163674in}}{\pgfqpoint{1.716533in}{2.163674in}}%
\pgfpathcurveto{\pgfqpoint{1.708297in}{2.163674in}}{\pgfqpoint{1.700397in}{2.160401in}}{\pgfqpoint{1.694573in}{2.154577in}}%
\pgfpathcurveto{\pgfqpoint{1.688749in}{2.148753in}}{\pgfqpoint{1.685477in}{2.140853in}}{\pgfqpoint{1.685477in}{2.132617in}}%
\pgfpathcurveto{\pgfqpoint{1.685477in}{2.124381in}}{\pgfqpoint{1.688749in}{2.116481in}}{\pgfqpoint{1.694573in}{2.110657in}}%
\pgfpathcurveto{\pgfqpoint{1.700397in}{2.104833in}}{\pgfqpoint{1.708297in}{2.101561in}}{\pgfqpoint{1.716533in}{2.101561in}}%
\pgfpathclose%
\pgfusepath{stroke,fill}%
\end{pgfscope}%
\begin{pgfscope}%
\pgfpathrectangle{\pgfqpoint{0.100000in}{0.212622in}}{\pgfqpoint{3.696000in}{3.696000in}}%
\pgfusepath{clip}%
\pgfsetbuttcap%
\pgfsetroundjoin%
\definecolor{currentfill}{rgb}{0.121569,0.466667,0.705882}%
\pgfsetfillcolor{currentfill}%
\pgfsetfillopacity{0.368884}%
\pgfsetlinewidth{1.003750pt}%
\definecolor{currentstroke}{rgb}{0.121569,0.466667,0.705882}%
\pgfsetstrokecolor{currentstroke}%
\pgfsetstrokeopacity{0.368884}%
\pgfsetdash{}{0pt}%
\pgfpathmoveto{\pgfqpoint{1.929173in}{2.230730in}}%
\pgfpathcurveto{\pgfqpoint{1.937410in}{2.230730in}}{\pgfqpoint{1.945310in}{2.234002in}}{\pgfqpoint{1.951134in}{2.239826in}}%
\pgfpathcurveto{\pgfqpoint{1.956958in}{2.245650in}}{\pgfqpoint{1.960230in}{2.253550in}}{\pgfqpoint{1.960230in}{2.261786in}}%
\pgfpathcurveto{\pgfqpoint{1.960230in}{2.270023in}}{\pgfqpoint{1.956958in}{2.277923in}}{\pgfqpoint{1.951134in}{2.283747in}}%
\pgfpathcurveto{\pgfqpoint{1.945310in}{2.289571in}}{\pgfqpoint{1.937410in}{2.292843in}}{\pgfqpoint{1.929173in}{2.292843in}}%
\pgfpathcurveto{\pgfqpoint{1.920937in}{2.292843in}}{\pgfqpoint{1.913037in}{2.289571in}}{\pgfqpoint{1.907213in}{2.283747in}}%
\pgfpathcurveto{\pgfqpoint{1.901389in}{2.277923in}}{\pgfqpoint{1.898117in}{2.270023in}}{\pgfqpoint{1.898117in}{2.261786in}}%
\pgfpathcurveto{\pgfqpoint{1.898117in}{2.253550in}}{\pgfqpoint{1.901389in}{2.245650in}}{\pgfqpoint{1.907213in}{2.239826in}}%
\pgfpathcurveto{\pgfqpoint{1.913037in}{2.234002in}}{\pgfqpoint{1.920937in}{2.230730in}}{\pgfqpoint{1.929173in}{2.230730in}}%
\pgfpathclose%
\pgfusepath{stroke,fill}%
\end{pgfscope}%
\begin{pgfscope}%
\pgfpathrectangle{\pgfqpoint{0.100000in}{0.212622in}}{\pgfqpoint{3.696000in}{3.696000in}}%
\pgfusepath{clip}%
\pgfsetbuttcap%
\pgfsetroundjoin%
\definecolor{currentfill}{rgb}{0.121569,0.466667,0.705882}%
\pgfsetfillcolor{currentfill}%
\pgfsetfillopacity{0.368913}%
\pgfsetlinewidth{1.003750pt}%
\definecolor{currentstroke}{rgb}{0.121569,0.466667,0.705882}%
\pgfsetstrokecolor{currentstroke}%
\pgfsetstrokeopacity{0.368913}%
\pgfsetdash{}{0pt}%
\pgfpathmoveto{\pgfqpoint{1.764944in}{2.126439in}}%
\pgfpathcurveto{\pgfqpoint{1.773180in}{2.126439in}}{\pgfqpoint{1.781080in}{2.129711in}}{\pgfqpoint{1.786904in}{2.135535in}}%
\pgfpathcurveto{\pgfqpoint{1.792728in}{2.141359in}}{\pgfqpoint{1.796000in}{2.149259in}}{\pgfqpoint{1.796000in}{2.157495in}}%
\pgfpathcurveto{\pgfqpoint{1.796000in}{2.165731in}}{\pgfqpoint{1.792728in}{2.173631in}}{\pgfqpoint{1.786904in}{2.179455in}}%
\pgfpathcurveto{\pgfqpoint{1.781080in}{2.185279in}}{\pgfqpoint{1.773180in}{2.188552in}}{\pgfqpoint{1.764944in}{2.188552in}}%
\pgfpathcurveto{\pgfqpoint{1.756707in}{2.188552in}}{\pgfqpoint{1.748807in}{2.185279in}}{\pgfqpoint{1.742983in}{2.179455in}}%
\pgfpathcurveto{\pgfqpoint{1.737159in}{2.173631in}}{\pgfqpoint{1.733887in}{2.165731in}}{\pgfqpoint{1.733887in}{2.157495in}}%
\pgfpathcurveto{\pgfqpoint{1.733887in}{2.149259in}}{\pgfqpoint{1.737159in}{2.141359in}}{\pgfqpoint{1.742983in}{2.135535in}}%
\pgfpathcurveto{\pgfqpoint{1.748807in}{2.129711in}}{\pgfqpoint{1.756707in}{2.126439in}}{\pgfqpoint{1.764944in}{2.126439in}}%
\pgfpathclose%
\pgfusepath{stroke,fill}%
\end{pgfscope}%
\begin{pgfscope}%
\pgfpathrectangle{\pgfqpoint{0.100000in}{0.212622in}}{\pgfqpoint{3.696000in}{3.696000in}}%
\pgfusepath{clip}%
\pgfsetbuttcap%
\pgfsetroundjoin%
\definecolor{currentfill}{rgb}{0.121569,0.466667,0.705882}%
\pgfsetfillcolor{currentfill}%
\pgfsetfillopacity{0.368951}%
\pgfsetlinewidth{1.003750pt}%
\definecolor{currentstroke}{rgb}{0.121569,0.466667,0.705882}%
\pgfsetstrokecolor{currentstroke}%
\pgfsetstrokeopacity{0.368951}%
\pgfsetdash{}{0pt}%
\pgfpathmoveto{\pgfqpoint{1.412525in}{1.891185in}}%
\pgfpathcurveto{\pgfqpoint{1.420762in}{1.891185in}}{\pgfqpoint{1.428662in}{1.894458in}}{\pgfqpoint{1.434486in}{1.900282in}}%
\pgfpathcurveto{\pgfqpoint{1.440310in}{1.906106in}}{\pgfqpoint{1.443582in}{1.914006in}}{\pgfqpoint{1.443582in}{1.922242in}}%
\pgfpathcurveto{\pgfqpoint{1.443582in}{1.930478in}}{\pgfqpoint{1.440310in}{1.938378in}}{\pgfqpoint{1.434486in}{1.944202in}}%
\pgfpathcurveto{\pgfqpoint{1.428662in}{1.950026in}}{\pgfqpoint{1.420762in}{1.953298in}}{\pgfqpoint{1.412525in}{1.953298in}}%
\pgfpathcurveto{\pgfqpoint{1.404289in}{1.953298in}}{\pgfqpoint{1.396389in}{1.950026in}}{\pgfqpoint{1.390565in}{1.944202in}}%
\pgfpathcurveto{\pgfqpoint{1.384741in}{1.938378in}}{\pgfqpoint{1.381469in}{1.930478in}}{\pgfqpoint{1.381469in}{1.922242in}}%
\pgfpathcurveto{\pgfqpoint{1.381469in}{1.914006in}}{\pgfqpoint{1.384741in}{1.906106in}}{\pgfqpoint{1.390565in}{1.900282in}}%
\pgfpathcurveto{\pgfqpoint{1.396389in}{1.894458in}}{\pgfqpoint{1.404289in}{1.891185in}}{\pgfqpoint{1.412525in}{1.891185in}}%
\pgfpathclose%
\pgfusepath{stroke,fill}%
\end{pgfscope}%
\begin{pgfscope}%
\pgfpathrectangle{\pgfqpoint{0.100000in}{0.212622in}}{\pgfqpoint{3.696000in}{3.696000in}}%
\pgfusepath{clip}%
\pgfsetbuttcap%
\pgfsetroundjoin%
\definecolor{currentfill}{rgb}{0.121569,0.466667,0.705882}%
\pgfsetfillcolor{currentfill}%
\pgfsetfillopacity{0.368990}%
\pgfsetlinewidth{1.003750pt}%
\definecolor{currentstroke}{rgb}{0.121569,0.466667,0.705882}%
\pgfsetstrokecolor{currentstroke}%
\pgfsetstrokeopacity{0.368990}%
\pgfsetdash{}{0pt}%
\pgfpathmoveto{\pgfqpoint{1.408086in}{1.888993in}}%
\pgfpathcurveto{\pgfqpoint{1.416323in}{1.888993in}}{\pgfqpoint{1.424223in}{1.892265in}}{\pgfqpoint{1.430047in}{1.898089in}}%
\pgfpathcurveto{\pgfqpoint{1.435871in}{1.903913in}}{\pgfqpoint{1.439143in}{1.911813in}}{\pgfqpoint{1.439143in}{1.920049in}}%
\pgfpathcurveto{\pgfqpoint{1.439143in}{1.928286in}}{\pgfqpoint{1.435871in}{1.936186in}}{\pgfqpoint{1.430047in}{1.942010in}}%
\pgfpathcurveto{\pgfqpoint{1.424223in}{1.947834in}}{\pgfqpoint{1.416323in}{1.951106in}}{\pgfqpoint{1.408086in}{1.951106in}}%
\pgfpathcurveto{\pgfqpoint{1.399850in}{1.951106in}}{\pgfqpoint{1.391950in}{1.947834in}}{\pgfqpoint{1.386126in}{1.942010in}}%
\pgfpathcurveto{\pgfqpoint{1.380302in}{1.936186in}}{\pgfqpoint{1.377030in}{1.928286in}}{\pgfqpoint{1.377030in}{1.920049in}}%
\pgfpathcurveto{\pgfqpoint{1.377030in}{1.911813in}}{\pgfqpoint{1.380302in}{1.903913in}}{\pgfqpoint{1.386126in}{1.898089in}}%
\pgfpathcurveto{\pgfqpoint{1.391950in}{1.892265in}}{\pgfqpoint{1.399850in}{1.888993in}}{\pgfqpoint{1.408086in}{1.888993in}}%
\pgfpathclose%
\pgfusepath{stroke,fill}%
\end{pgfscope}%
\begin{pgfscope}%
\pgfpathrectangle{\pgfqpoint{0.100000in}{0.212622in}}{\pgfqpoint{3.696000in}{3.696000in}}%
\pgfusepath{clip}%
\pgfsetbuttcap%
\pgfsetroundjoin%
\definecolor{currentfill}{rgb}{0.121569,0.466667,0.705882}%
\pgfsetfillcolor{currentfill}%
\pgfsetfillopacity{0.369005}%
\pgfsetlinewidth{1.003750pt}%
\definecolor{currentstroke}{rgb}{0.121569,0.466667,0.705882}%
\pgfsetstrokecolor{currentstroke}%
\pgfsetstrokeopacity{0.369005}%
\pgfsetdash{}{0pt}%
\pgfpathmoveto{\pgfqpoint{1.410894in}{1.890553in}}%
\pgfpathcurveto{\pgfqpoint{1.419130in}{1.890553in}}{\pgfqpoint{1.427030in}{1.893825in}}{\pgfqpoint{1.432854in}{1.899649in}}%
\pgfpathcurveto{\pgfqpoint{1.438678in}{1.905473in}}{\pgfqpoint{1.441950in}{1.913373in}}{\pgfqpoint{1.441950in}{1.921609in}}%
\pgfpathcurveto{\pgfqpoint{1.441950in}{1.929845in}}{\pgfqpoint{1.438678in}{1.937745in}}{\pgfqpoint{1.432854in}{1.943569in}}%
\pgfpathcurveto{\pgfqpoint{1.427030in}{1.949393in}}{\pgfqpoint{1.419130in}{1.952666in}}{\pgfqpoint{1.410894in}{1.952666in}}%
\pgfpathcurveto{\pgfqpoint{1.402657in}{1.952666in}}{\pgfqpoint{1.394757in}{1.949393in}}{\pgfqpoint{1.388933in}{1.943569in}}%
\pgfpathcurveto{\pgfqpoint{1.383110in}{1.937745in}}{\pgfqpoint{1.379837in}{1.929845in}}{\pgfqpoint{1.379837in}{1.921609in}}%
\pgfpathcurveto{\pgfqpoint{1.379837in}{1.913373in}}{\pgfqpoint{1.383110in}{1.905473in}}{\pgfqpoint{1.388933in}{1.899649in}}%
\pgfpathcurveto{\pgfqpoint{1.394757in}{1.893825in}}{\pgfqpoint{1.402657in}{1.890553in}}{\pgfqpoint{1.410894in}{1.890553in}}%
\pgfpathclose%
\pgfusepath{stroke,fill}%
\end{pgfscope}%
\begin{pgfscope}%
\pgfpathrectangle{\pgfqpoint{0.100000in}{0.212622in}}{\pgfqpoint{3.696000in}{3.696000in}}%
\pgfusepath{clip}%
\pgfsetbuttcap%
\pgfsetroundjoin%
\definecolor{currentfill}{rgb}{0.121569,0.466667,0.705882}%
\pgfsetfillcolor{currentfill}%
\pgfsetfillopacity{0.369024}%
\pgfsetlinewidth{1.003750pt}%
\definecolor{currentstroke}{rgb}{0.121569,0.466667,0.705882}%
\pgfsetstrokecolor{currentstroke}%
\pgfsetstrokeopacity{0.369024}%
\pgfsetdash{}{0pt}%
\pgfpathmoveto{\pgfqpoint{1.838960in}{2.170891in}}%
\pgfpathcurveto{\pgfqpoint{1.847196in}{2.170891in}}{\pgfqpoint{1.855096in}{2.174163in}}{\pgfqpoint{1.860920in}{2.179987in}}%
\pgfpathcurveto{\pgfqpoint{1.866744in}{2.185811in}}{\pgfqpoint{1.870016in}{2.193711in}}{\pgfqpoint{1.870016in}{2.201947in}}%
\pgfpathcurveto{\pgfqpoint{1.870016in}{2.210184in}}{\pgfqpoint{1.866744in}{2.218084in}}{\pgfqpoint{1.860920in}{2.223908in}}%
\pgfpathcurveto{\pgfqpoint{1.855096in}{2.229731in}}{\pgfqpoint{1.847196in}{2.233004in}}{\pgfqpoint{1.838960in}{2.233004in}}%
\pgfpathcurveto{\pgfqpoint{1.830724in}{2.233004in}}{\pgfqpoint{1.822824in}{2.229731in}}{\pgfqpoint{1.817000in}{2.223908in}}%
\pgfpathcurveto{\pgfqpoint{1.811176in}{2.218084in}}{\pgfqpoint{1.807903in}{2.210184in}}{\pgfqpoint{1.807903in}{2.201947in}}%
\pgfpathcurveto{\pgfqpoint{1.807903in}{2.193711in}}{\pgfqpoint{1.811176in}{2.185811in}}{\pgfqpoint{1.817000in}{2.179987in}}%
\pgfpathcurveto{\pgfqpoint{1.822824in}{2.174163in}}{\pgfqpoint{1.830724in}{2.170891in}}{\pgfqpoint{1.838960in}{2.170891in}}%
\pgfpathclose%
\pgfusepath{stroke,fill}%
\end{pgfscope}%
\begin{pgfscope}%
\pgfpathrectangle{\pgfqpoint{0.100000in}{0.212622in}}{\pgfqpoint{3.696000in}{3.696000in}}%
\pgfusepath{clip}%
\pgfsetbuttcap%
\pgfsetroundjoin%
\definecolor{currentfill}{rgb}{0.121569,0.466667,0.705882}%
\pgfsetfillcolor{currentfill}%
\pgfsetfillopacity{0.369172}%
\pgfsetlinewidth{1.003750pt}%
\definecolor{currentstroke}{rgb}{0.121569,0.466667,0.705882}%
\pgfsetstrokecolor{currentstroke}%
\pgfsetstrokeopacity{0.369172}%
\pgfsetdash{}{0pt}%
\pgfpathmoveto{\pgfqpoint{1.692509in}{2.087751in}}%
\pgfpathcurveto{\pgfqpoint{1.700745in}{2.087751in}}{\pgfqpoint{1.708645in}{2.091023in}}{\pgfqpoint{1.714469in}{2.096847in}}%
\pgfpathcurveto{\pgfqpoint{1.720293in}{2.102671in}}{\pgfqpoint{1.723565in}{2.110571in}}{\pgfqpoint{1.723565in}{2.118807in}}%
\pgfpathcurveto{\pgfqpoint{1.723565in}{2.127044in}}{\pgfqpoint{1.720293in}{2.134944in}}{\pgfqpoint{1.714469in}{2.140768in}}%
\pgfpathcurveto{\pgfqpoint{1.708645in}{2.146592in}}{\pgfqpoint{1.700745in}{2.149864in}}{\pgfqpoint{1.692509in}{2.149864in}}%
\pgfpathcurveto{\pgfqpoint{1.684272in}{2.149864in}}{\pgfqpoint{1.676372in}{2.146592in}}{\pgfqpoint{1.670548in}{2.140768in}}%
\pgfpathcurveto{\pgfqpoint{1.664724in}{2.134944in}}{\pgfqpoint{1.661452in}{2.127044in}}{\pgfqpoint{1.661452in}{2.118807in}}%
\pgfpathcurveto{\pgfqpoint{1.661452in}{2.110571in}}{\pgfqpoint{1.664724in}{2.102671in}}{\pgfqpoint{1.670548in}{2.096847in}}%
\pgfpathcurveto{\pgfqpoint{1.676372in}{2.091023in}}{\pgfqpoint{1.684272in}{2.087751in}}{\pgfqpoint{1.692509in}{2.087751in}}%
\pgfpathclose%
\pgfusepath{stroke,fill}%
\end{pgfscope}%
\begin{pgfscope}%
\pgfpathrectangle{\pgfqpoint{0.100000in}{0.212622in}}{\pgfqpoint{3.696000in}{3.696000in}}%
\pgfusepath{clip}%
\pgfsetbuttcap%
\pgfsetroundjoin%
\definecolor{currentfill}{rgb}{0.121569,0.466667,0.705882}%
\pgfsetfillcolor{currentfill}%
\pgfsetfillopacity{0.369433}%
\pgfsetlinewidth{1.003750pt}%
\definecolor{currentstroke}{rgb}{0.121569,0.466667,0.705882}%
\pgfsetstrokecolor{currentstroke}%
\pgfsetstrokeopacity{0.369433}%
\pgfsetdash{}{0pt}%
\pgfpathmoveto{\pgfqpoint{1.417595in}{1.895165in}}%
\pgfpathcurveto{\pgfqpoint{1.425831in}{1.895165in}}{\pgfqpoint{1.433731in}{1.898437in}}{\pgfqpoint{1.439555in}{1.904261in}}%
\pgfpathcurveto{\pgfqpoint{1.445379in}{1.910085in}}{\pgfqpoint{1.448651in}{1.917985in}}{\pgfqpoint{1.448651in}{1.926222in}}%
\pgfpathcurveto{\pgfqpoint{1.448651in}{1.934458in}}{\pgfqpoint{1.445379in}{1.942358in}}{\pgfqpoint{1.439555in}{1.948182in}}%
\pgfpathcurveto{\pgfqpoint{1.433731in}{1.954006in}}{\pgfqpoint{1.425831in}{1.957278in}}{\pgfqpoint{1.417595in}{1.957278in}}%
\pgfpathcurveto{\pgfqpoint{1.409358in}{1.957278in}}{\pgfqpoint{1.401458in}{1.954006in}}{\pgfqpoint{1.395634in}{1.948182in}}%
\pgfpathcurveto{\pgfqpoint{1.389810in}{1.942358in}}{\pgfqpoint{1.386538in}{1.934458in}}{\pgfqpoint{1.386538in}{1.926222in}}%
\pgfpathcurveto{\pgfqpoint{1.386538in}{1.917985in}}{\pgfqpoint{1.389810in}{1.910085in}}{\pgfqpoint{1.395634in}{1.904261in}}%
\pgfpathcurveto{\pgfqpoint{1.401458in}{1.898437in}}{\pgfqpoint{1.409358in}{1.895165in}}{\pgfqpoint{1.417595in}{1.895165in}}%
\pgfpathclose%
\pgfusepath{stroke,fill}%
\end{pgfscope}%
\begin{pgfscope}%
\pgfpathrectangle{\pgfqpoint{0.100000in}{0.212622in}}{\pgfqpoint{3.696000in}{3.696000in}}%
\pgfusepath{clip}%
\pgfsetbuttcap%
\pgfsetroundjoin%
\definecolor{currentfill}{rgb}{0.121569,0.466667,0.705882}%
\pgfsetfillcolor{currentfill}%
\pgfsetfillopacity{0.369486}%
\pgfsetlinewidth{1.003750pt}%
\definecolor{currentstroke}{rgb}{0.121569,0.466667,0.705882}%
\pgfsetstrokecolor{currentstroke}%
\pgfsetstrokeopacity{0.369486}%
\pgfsetdash{}{0pt}%
\pgfpathmoveto{\pgfqpoint{1.903807in}{2.218028in}}%
\pgfpathcurveto{\pgfqpoint{1.912043in}{2.218028in}}{\pgfqpoint{1.919943in}{2.221300in}}{\pgfqpoint{1.925767in}{2.227124in}}%
\pgfpathcurveto{\pgfqpoint{1.931591in}{2.232948in}}{\pgfqpoint{1.934863in}{2.240848in}}{\pgfqpoint{1.934863in}{2.249084in}}%
\pgfpathcurveto{\pgfqpoint{1.934863in}{2.257321in}}{\pgfqpoint{1.931591in}{2.265221in}}{\pgfqpoint{1.925767in}{2.271045in}}%
\pgfpathcurveto{\pgfqpoint{1.919943in}{2.276868in}}{\pgfqpoint{1.912043in}{2.280141in}}{\pgfqpoint{1.903807in}{2.280141in}}%
\pgfpathcurveto{\pgfqpoint{1.895571in}{2.280141in}}{\pgfqpoint{1.887671in}{2.276868in}}{\pgfqpoint{1.881847in}{2.271045in}}%
\pgfpathcurveto{\pgfqpoint{1.876023in}{2.265221in}}{\pgfqpoint{1.872750in}{2.257321in}}{\pgfqpoint{1.872750in}{2.249084in}}%
\pgfpathcurveto{\pgfqpoint{1.872750in}{2.240848in}}{\pgfqpoint{1.876023in}{2.232948in}}{\pgfqpoint{1.881847in}{2.227124in}}%
\pgfpathcurveto{\pgfqpoint{1.887671in}{2.221300in}}{\pgfqpoint{1.895571in}{2.218028in}}{\pgfqpoint{1.903807in}{2.218028in}}%
\pgfpathclose%
\pgfusepath{stroke,fill}%
\end{pgfscope}%
\begin{pgfscope}%
\pgfpathrectangle{\pgfqpoint{0.100000in}{0.212622in}}{\pgfqpoint{3.696000in}{3.696000in}}%
\pgfusepath{clip}%
\pgfsetbuttcap%
\pgfsetroundjoin%
\definecolor{currentfill}{rgb}{0.121569,0.466667,0.705882}%
\pgfsetfillcolor{currentfill}%
\pgfsetfillopacity{0.369508}%
\pgfsetlinewidth{1.003750pt}%
\definecolor{currentstroke}{rgb}{0.121569,0.466667,0.705882}%
\pgfsetstrokecolor{currentstroke}%
\pgfsetstrokeopacity{0.369508}%
\pgfsetdash{}{0pt}%
\pgfpathmoveto{\pgfqpoint{1.785647in}{2.142649in}}%
\pgfpathcurveto{\pgfqpoint{1.793884in}{2.142649in}}{\pgfqpoint{1.801784in}{2.145921in}}{\pgfqpoint{1.807608in}{2.151745in}}%
\pgfpathcurveto{\pgfqpoint{1.813432in}{2.157569in}}{\pgfqpoint{1.816704in}{2.165469in}}{\pgfqpoint{1.816704in}{2.173705in}}%
\pgfpathcurveto{\pgfqpoint{1.816704in}{2.181942in}}{\pgfqpoint{1.813432in}{2.189842in}}{\pgfqpoint{1.807608in}{2.195666in}}%
\pgfpathcurveto{\pgfqpoint{1.801784in}{2.201490in}}{\pgfqpoint{1.793884in}{2.204762in}}{\pgfqpoint{1.785647in}{2.204762in}}%
\pgfpathcurveto{\pgfqpoint{1.777411in}{2.204762in}}{\pgfqpoint{1.769511in}{2.201490in}}{\pgfqpoint{1.763687in}{2.195666in}}%
\pgfpathcurveto{\pgfqpoint{1.757863in}{2.189842in}}{\pgfqpoint{1.754591in}{2.181942in}}{\pgfqpoint{1.754591in}{2.173705in}}%
\pgfpathcurveto{\pgfqpoint{1.754591in}{2.165469in}}{\pgfqpoint{1.757863in}{2.157569in}}{\pgfqpoint{1.763687in}{2.151745in}}%
\pgfpathcurveto{\pgfqpoint{1.769511in}{2.145921in}}{\pgfqpoint{1.777411in}{2.142649in}}{\pgfqpoint{1.785647in}{2.142649in}}%
\pgfpathclose%
\pgfusepath{stroke,fill}%
\end{pgfscope}%
\begin{pgfscope}%
\pgfpathrectangle{\pgfqpoint{0.100000in}{0.212622in}}{\pgfqpoint{3.696000in}{3.696000in}}%
\pgfusepath{clip}%
\pgfsetbuttcap%
\pgfsetroundjoin%
\definecolor{currentfill}{rgb}{0.121569,0.466667,0.705882}%
\pgfsetfillcolor{currentfill}%
\pgfsetfillopacity{0.369737}%
\pgfsetlinewidth{1.003750pt}%
\definecolor{currentstroke}{rgb}{0.121569,0.466667,0.705882}%
\pgfsetstrokecolor{currentstroke}%
\pgfsetstrokeopacity{0.369737}%
\pgfsetdash{}{0pt}%
\pgfpathmoveto{\pgfqpoint{1.412675in}{1.890956in}}%
\pgfpathcurveto{\pgfqpoint{1.420911in}{1.890956in}}{\pgfqpoint{1.428811in}{1.894228in}}{\pgfqpoint{1.434635in}{1.900052in}}%
\pgfpathcurveto{\pgfqpoint{1.440459in}{1.905876in}}{\pgfqpoint{1.443731in}{1.913776in}}{\pgfqpoint{1.443731in}{1.922012in}}%
\pgfpathcurveto{\pgfqpoint{1.443731in}{1.930248in}}{\pgfqpoint{1.440459in}{1.938148in}}{\pgfqpoint{1.434635in}{1.943972in}}%
\pgfpathcurveto{\pgfqpoint{1.428811in}{1.949796in}}{\pgfqpoint{1.420911in}{1.953069in}}{\pgfqpoint{1.412675in}{1.953069in}}%
\pgfpathcurveto{\pgfqpoint{1.404439in}{1.953069in}}{\pgfqpoint{1.396539in}{1.949796in}}{\pgfqpoint{1.390715in}{1.943972in}}%
\pgfpathcurveto{\pgfqpoint{1.384891in}{1.938148in}}{\pgfqpoint{1.381618in}{1.930248in}}{\pgfqpoint{1.381618in}{1.922012in}}%
\pgfpathcurveto{\pgfqpoint{1.381618in}{1.913776in}}{\pgfqpoint{1.384891in}{1.905876in}}{\pgfqpoint{1.390715in}{1.900052in}}%
\pgfpathcurveto{\pgfqpoint{1.396539in}{1.894228in}}{\pgfqpoint{1.404439in}{1.890956in}}{\pgfqpoint{1.412675in}{1.890956in}}%
\pgfpathclose%
\pgfusepath{stroke,fill}%
\end{pgfscope}%
\begin{pgfscope}%
\pgfpathrectangle{\pgfqpoint{0.100000in}{0.212622in}}{\pgfqpoint{3.696000in}{3.696000in}}%
\pgfusepath{clip}%
\pgfsetbuttcap%
\pgfsetroundjoin%
\definecolor{currentfill}{rgb}{0.121569,0.466667,0.705882}%
\pgfsetfillcolor{currentfill}%
\pgfsetfillopacity{0.370066}%
\pgfsetlinewidth{1.003750pt}%
\definecolor{currentstroke}{rgb}{0.121569,0.466667,0.705882}%
\pgfsetstrokecolor{currentstroke}%
\pgfsetstrokeopacity{0.370066}%
\pgfsetdash{}{0pt}%
\pgfpathmoveto{\pgfqpoint{1.723430in}{2.107504in}}%
\pgfpathcurveto{\pgfqpoint{1.731666in}{2.107504in}}{\pgfqpoint{1.739566in}{2.110776in}}{\pgfqpoint{1.745390in}{2.116600in}}%
\pgfpathcurveto{\pgfqpoint{1.751214in}{2.122424in}}{\pgfqpoint{1.754487in}{2.130324in}}{\pgfqpoint{1.754487in}{2.138560in}}%
\pgfpathcurveto{\pgfqpoint{1.754487in}{2.146797in}}{\pgfqpoint{1.751214in}{2.154697in}}{\pgfqpoint{1.745390in}{2.160521in}}%
\pgfpathcurveto{\pgfqpoint{1.739566in}{2.166344in}}{\pgfqpoint{1.731666in}{2.169617in}}{\pgfqpoint{1.723430in}{2.169617in}}%
\pgfpathcurveto{\pgfqpoint{1.715194in}{2.169617in}}{\pgfqpoint{1.707294in}{2.166344in}}{\pgfqpoint{1.701470in}{2.160521in}}%
\pgfpathcurveto{\pgfqpoint{1.695646in}{2.154697in}}{\pgfqpoint{1.692374in}{2.146797in}}{\pgfqpoint{1.692374in}{2.138560in}}%
\pgfpathcurveto{\pgfqpoint{1.692374in}{2.130324in}}{\pgfqpoint{1.695646in}{2.122424in}}{\pgfqpoint{1.701470in}{2.116600in}}%
\pgfpathcurveto{\pgfqpoint{1.707294in}{2.110776in}}{\pgfqpoint{1.715194in}{2.107504in}}{\pgfqpoint{1.723430in}{2.107504in}}%
\pgfpathclose%
\pgfusepath{stroke,fill}%
\end{pgfscope}%
\begin{pgfscope}%
\pgfpathrectangle{\pgfqpoint{0.100000in}{0.212622in}}{\pgfqpoint{3.696000in}{3.696000in}}%
\pgfusepath{clip}%
\pgfsetbuttcap%
\pgfsetroundjoin%
\definecolor{currentfill}{rgb}{0.121569,0.466667,0.705882}%
\pgfsetfillcolor{currentfill}%
\pgfsetfillopacity{0.370220}%
\pgfsetlinewidth{1.003750pt}%
\definecolor{currentstroke}{rgb}{0.121569,0.466667,0.705882}%
\pgfsetstrokecolor{currentstroke}%
\pgfsetstrokeopacity{0.370220}%
\pgfsetdash{}{0pt}%
\pgfpathmoveto{\pgfqpoint{1.771858in}{2.132827in}}%
\pgfpathcurveto{\pgfqpoint{1.780094in}{2.132827in}}{\pgfqpoint{1.787994in}{2.136099in}}{\pgfqpoint{1.793818in}{2.141923in}}%
\pgfpathcurveto{\pgfqpoint{1.799642in}{2.147747in}}{\pgfqpoint{1.802914in}{2.155647in}}{\pgfqpoint{1.802914in}{2.163883in}}%
\pgfpathcurveto{\pgfqpoint{1.802914in}{2.172120in}}{\pgfqpoint{1.799642in}{2.180020in}}{\pgfqpoint{1.793818in}{2.185844in}}%
\pgfpathcurveto{\pgfqpoint{1.787994in}{2.191668in}}{\pgfqpoint{1.780094in}{2.194940in}}{\pgfqpoint{1.771858in}{2.194940in}}%
\pgfpathcurveto{\pgfqpoint{1.763622in}{2.194940in}}{\pgfqpoint{1.755721in}{2.191668in}}{\pgfqpoint{1.749898in}{2.185844in}}%
\pgfpathcurveto{\pgfqpoint{1.744074in}{2.180020in}}{\pgfqpoint{1.740801in}{2.172120in}}{\pgfqpoint{1.740801in}{2.163883in}}%
\pgfpathcurveto{\pgfqpoint{1.740801in}{2.155647in}}{\pgfqpoint{1.744074in}{2.147747in}}{\pgfqpoint{1.749898in}{2.141923in}}%
\pgfpathcurveto{\pgfqpoint{1.755721in}{2.136099in}}{\pgfqpoint{1.763622in}{2.132827in}}{\pgfqpoint{1.771858in}{2.132827in}}%
\pgfpathclose%
\pgfusepath{stroke,fill}%
\end{pgfscope}%
\begin{pgfscope}%
\pgfpathrectangle{\pgfqpoint{0.100000in}{0.212622in}}{\pgfqpoint{3.696000in}{3.696000in}}%
\pgfusepath{clip}%
\pgfsetbuttcap%
\pgfsetroundjoin%
\definecolor{currentfill}{rgb}{0.121569,0.466667,0.705882}%
\pgfsetfillcolor{currentfill}%
\pgfsetfillopacity{0.370314}%
\pgfsetlinewidth{1.003750pt}%
\definecolor{currentstroke}{rgb}{0.121569,0.466667,0.705882}%
\pgfsetstrokecolor{currentstroke}%
\pgfsetstrokeopacity{0.370314}%
\pgfsetdash{}{0pt}%
\pgfpathmoveto{\pgfqpoint{1.887943in}{2.198823in}}%
\pgfpathcurveto{\pgfqpoint{1.896179in}{2.198823in}}{\pgfqpoint{1.904079in}{2.202096in}}{\pgfqpoint{1.909903in}{2.207919in}}%
\pgfpathcurveto{\pgfqpoint{1.915727in}{2.213743in}}{\pgfqpoint{1.918999in}{2.221643in}}{\pgfqpoint{1.918999in}{2.229880in}}%
\pgfpathcurveto{\pgfqpoint{1.918999in}{2.238116in}}{\pgfqpoint{1.915727in}{2.246016in}}{\pgfqpoint{1.909903in}{2.251840in}}%
\pgfpathcurveto{\pgfqpoint{1.904079in}{2.257664in}}{\pgfqpoint{1.896179in}{2.260936in}}{\pgfqpoint{1.887943in}{2.260936in}}%
\pgfpathcurveto{\pgfqpoint{1.879706in}{2.260936in}}{\pgfqpoint{1.871806in}{2.257664in}}{\pgfqpoint{1.865982in}{2.251840in}}%
\pgfpathcurveto{\pgfqpoint{1.860158in}{2.246016in}}{\pgfqpoint{1.856886in}{2.238116in}}{\pgfqpoint{1.856886in}{2.229880in}}%
\pgfpathcurveto{\pgfqpoint{1.856886in}{2.221643in}}{\pgfqpoint{1.860158in}{2.213743in}}{\pgfqpoint{1.865982in}{2.207919in}}%
\pgfpathcurveto{\pgfqpoint{1.871806in}{2.202096in}}{\pgfqpoint{1.879706in}{2.198823in}}{\pgfqpoint{1.887943in}{2.198823in}}%
\pgfpathclose%
\pgfusepath{stroke,fill}%
\end{pgfscope}%
\begin{pgfscope}%
\pgfpathrectangle{\pgfqpoint{0.100000in}{0.212622in}}{\pgfqpoint{3.696000in}{3.696000in}}%
\pgfusepath{clip}%
\pgfsetbuttcap%
\pgfsetroundjoin%
\definecolor{currentfill}{rgb}{0.121569,0.466667,0.705882}%
\pgfsetfillcolor{currentfill}%
\pgfsetfillopacity{0.370456}%
\pgfsetlinewidth{1.003750pt}%
\definecolor{currentstroke}{rgb}{0.121569,0.466667,0.705882}%
\pgfsetstrokecolor{currentstroke}%
\pgfsetstrokeopacity{0.370456}%
\pgfsetdash{}{0pt}%
\pgfpathmoveto{\pgfqpoint{1.794896in}{2.148776in}}%
\pgfpathcurveto{\pgfqpoint{1.803132in}{2.148776in}}{\pgfqpoint{1.811032in}{2.152048in}}{\pgfqpoint{1.816856in}{2.157872in}}%
\pgfpathcurveto{\pgfqpoint{1.822680in}{2.163696in}}{\pgfqpoint{1.825953in}{2.171596in}}{\pgfqpoint{1.825953in}{2.179832in}}%
\pgfpathcurveto{\pgfqpoint{1.825953in}{2.188069in}}{\pgfqpoint{1.822680in}{2.195969in}}{\pgfqpoint{1.816856in}{2.201793in}}%
\pgfpathcurveto{\pgfqpoint{1.811032in}{2.207617in}}{\pgfqpoint{1.803132in}{2.210889in}}{\pgfqpoint{1.794896in}{2.210889in}}%
\pgfpathcurveto{\pgfqpoint{1.786660in}{2.210889in}}{\pgfqpoint{1.778760in}{2.207617in}}{\pgfqpoint{1.772936in}{2.201793in}}%
\pgfpathcurveto{\pgfqpoint{1.767112in}{2.195969in}}{\pgfqpoint{1.763840in}{2.188069in}}{\pgfqpoint{1.763840in}{2.179832in}}%
\pgfpathcurveto{\pgfqpoint{1.763840in}{2.171596in}}{\pgfqpoint{1.767112in}{2.163696in}}{\pgfqpoint{1.772936in}{2.157872in}}%
\pgfpathcurveto{\pgfqpoint{1.778760in}{2.152048in}}{\pgfqpoint{1.786660in}{2.148776in}}{\pgfqpoint{1.794896in}{2.148776in}}%
\pgfpathclose%
\pgfusepath{stroke,fill}%
\end{pgfscope}%
\begin{pgfscope}%
\pgfpathrectangle{\pgfqpoint{0.100000in}{0.212622in}}{\pgfqpoint{3.696000in}{3.696000in}}%
\pgfusepath{clip}%
\pgfsetbuttcap%
\pgfsetroundjoin%
\definecolor{currentfill}{rgb}{0.121569,0.466667,0.705882}%
\pgfsetfillcolor{currentfill}%
\pgfsetfillopacity{0.370501}%
\pgfsetlinewidth{1.003750pt}%
\definecolor{currentstroke}{rgb}{0.121569,0.466667,0.705882}%
\pgfsetstrokecolor{currentstroke}%
\pgfsetstrokeopacity{0.370501}%
\pgfsetdash{}{0pt}%
\pgfpathmoveto{\pgfqpoint{1.726802in}{2.104820in}}%
\pgfpathcurveto{\pgfqpoint{1.735038in}{2.104820in}}{\pgfqpoint{1.742938in}{2.108092in}}{\pgfqpoint{1.748762in}{2.113916in}}%
\pgfpathcurveto{\pgfqpoint{1.754586in}{2.119740in}}{\pgfqpoint{1.757859in}{2.127640in}}{\pgfqpoint{1.757859in}{2.135877in}}%
\pgfpathcurveto{\pgfqpoint{1.757859in}{2.144113in}}{\pgfqpoint{1.754586in}{2.152013in}}{\pgfqpoint{1.748762in}{2.157837in}}%
\pgfpathcurveto{\pgfqpoint{1.742938in}{2.163661in}}{\pgfqpoint{1.735038in}{2.166933in}}{\pgfqpoint{1.726802in}{2.166933in}}%
\pgfpathcurveto{\pgfqpoint{1.718566in}{2.166933in}}{\pgfqpoint{1.710666in}{2.163661in}}{\pgfqpoint{1.704842in}{2.157837in}}%
\pgfpathcurveto{\pgfqpoint{1.699018in}{2.152013in}}{\pgfqpoint{1.695746in}{2.144113in}}{\pgfqpoint{1.695746in}{2.135877in}}%
\pgfpathcurveto{\pgfqpoint{1.695746in}{2.127640in}}{\pgfqpoint{1.699018in}{2.119740in}}{\pgfqpoint{1.704842in}{2.113916in}}%
\pgfpathcurveto{\pgfqpoint{1.710666in}{2.108092in}}{\pgfqpoint{1.718566in}{2.104820in}}{\pgfqpoint{1.726802in}{2.104820in}}%
\pgfpathclose%
\pgfusepath{stroke,fill}%
\end{pgfscope}%
\begin{pgfscope}%
\pgfpathrectangle{\pgfqpoint{0.100000in}{0.212622in}}{\pgfqpoint{3.696000in}{3.696000in}}%
\pgfusepath{clip}%
\pgfsetbuttcap%
\pgfsetroundjoin%
\definecolor{currentfill}{rgb}{0.121569,0.466667,0.705882}%
\pgfsetfillcolor{currentfill}%
\pgfsetfillopacity{0.370773}%
\pgfsetlinewidth{1.003750pt}%
\definecolor{currentstroke}{rgb}{0.121569,0.466667,0.705882}%
\pgfsetstrokecolor{currentstroke}%
\pgfsetstrokeopacity{0.370773}%
\pgfsetdash{}{0pt}%
\pgfpathmoveto{\pgfqpoint{1.461778in}{1.923962in}}%
\pgfpathcurveto{\pgfqpoint{1.470014in}{1.923962in}}{\pgfqpoint{1.477914in}{1.927234in}}{\pgfqpoint{1.483738in}{1.933058in}}%
\pgfpathcurveto{\pgfqpoint{1.489562in}{1.938882in}}{\pgfqpoint{1.492835in}{1.946782in}}{\pgfqpoint{1.492835in}{1.955018in}}%
\pgfpathcurveto{\pgfqpoint{1.492835in}{1.963254in}}{\pgfqpoint{1.489562in}{1.971155in}}{\pgfqpoint{1.483738in}{1.976978in}}%
\pgfpathcurveto{\pgfqpoint{1.477914in}{1.982802in}}{\pgfqpoint{1.470014in}{1.986075in}}{\pgfqpoint{1.461778in}{1.986075in}}%
\pgfpathcurveto{\pgfqpoint{1.453542in}{1.986075in}}{\pgfqpoint{1.445642in}{1.982802in}}{\pgfqpoint{1.439818in}{1.976978in}}%
\pgfpathcurveto{\pgfqpoint{1.433994in}{1.971155in}}{\pgfqpoint{1.430722in}{1.963254in}}{\pgfqpoint{1.430722in}{1.955018in}}%
\pgfpathcurveto{\pgfqpoint{1.430722in}{1.946782in}}{\pgfqpoint{1.433994in}{1.938882in}}{\pgfqpoint{1.439818in}{1.933058in}}%
\pgfpathcurveto{\pgfqpoint{1.445642in}{1.927234in}}{\pgfqpoint{1.453542in}{1.923962in}}{\pgfqpoint{1.461778in}{1.923962in}}%
\pgfpathclose%
\pgfusepath{stroke,fill}%
\end{pgfscope}%
\begin{pgfscope}%
\pgfpathrectangle{\pgfqpoint{0.100000in}{0.212622in}}{\pgfqpoint{3.696000in}{3.696000in}}%
\pgfusepath{clip}%
\pgfsetbuttcap%
\pgfsetroundjoin%
\definecolor{currentfill}{rgb}{0.121569,0.466667,0.705882}%
\pgfsetfillcolor{currentfill}%
\pgfsetfillopacity{0.370797}%
\pgfsetlinewidth{1.003750pt}%
\definecolor{currentstroke}{rgb}{0.121569,0.466667,0.705882}%
\pgfsetstrokecolor{currentstroke}%
\pgfsetstrokeopacity{0.370797}%
\pgfsetdash{}{0pt}%
\pgfpathmoveto{\pgfqpoint{1.901120in}{2.212514in}}%
\pgfpathcurveto{\pgfqpoint{1.909357in}{2.212514in}}{\pgfqpoint{1.917257in}{2.215786in}}{\pgfqpoint{1.923081in}{2.221610in}}%
\pgfpathcurveto{\pgfqpoint{1.928904in}{2.227434in}}{\pgfqpoint{1.932177in}{2.235334in}}{\pgfqpoint{1.932177in}{2.243570in}}%
\pgfpathcurveto{\pgfqpoint{1.932177in}{2.251806in}}{\pgfqpoint{1.928904in}{2.259706in}}{\pgfqpoint{1.923081in}{2.265530in}}%
\pgfpathcurveto{\pgfqpoint{1.917257in}{2.271354in}}{\pgfqpoint{1.909357in}{2.274627in}}{\pgfqpoint{1.901120in}{2.274627in}}%
\pgfpathcurveto{\pgfqpoint{1.892884in}{2.274627in}}{\pgfqpoint{1.884984in}{2.271354in}}{\pgfqpoint{1.879160in}{2.265530in}}%
\pgfpathcurveto{\pgfqpoint{1.873336in}{2.259706in}}{\pgfqpoint{1.870064in}{2.251806in}}{\pgfqpoint{1.870064in}{2.243570in}}%
\pgfpathcurveto{\pgfqpoint{1.870064in}{2.235334in}}{\pgfqpoint{1.873336in}{2.227434in}}{\pgfqpoint{1.879160in}{2.221610in}}%
\pgfpathcurveto{\pgfqpoint{1.884984in}{2.215786in}}{\pgfqpoint{1.892884in}{2.212514in}}{\pgfqpoint{1.901120in}{2.212514in}}%
\pgfpathclose%
\pgfusepath{stroke,fill}%
\end{pgfscope}%
\begin{pgfscope}%
\pgfpathrectangle{\pgfqpoint{0.100000in}{0.212622in}}{\pgfqpoint{3.696000in}{3.696000in}}%
\pgfusepath{clip}%
\pgfsetbuttcap%
\pgfsetroundjoin%
\definecolor{currentfill}{rgb}{0.121569,0.466667,0.705882}%
\pgfsetfillcolor{currentfill}%
\pgfsetfillopacity{0.371165}%
\pgfsetlinewidth{1.003750pt}%
\definecolor{currentstroke}{rgb}{0.121569,0.466667,0.705882}%
\pgfsetstrokecolor{currentstroke}%
\pgfsetstrokeopacity{0.371165}%
\pgfsetdash{}{0pt}%
\pgfpathmoveto{\pgfqpoint{1.415407in}{1.890725in}}%
\pgfpathcurveto{\pgfqpoint{1.423644in}{1.890725in}}{\pgfqpoint{1.431544in}{1.893998in}}{\pgfqpoint{1.437367in}{1.899822in}}%
\pgfpathcurveto{\pgfqpoint{1.443191in}{1.905646in}}{\pgfqpoint{1.446464in}{1.913546in}}{\pgfqpoint{1.446464in}{1.921782in}}%
\pgfpathcurveto{\pgfqpoint{1.446464in}{1.930018in}}{\pgfqpoint{1.443191in}{1.937918in}}{\pgfqpoint{1.437367in}{1.943742in}}%
\pgfpathcurveto{\pgfqpoint{1.431544in}{1.949566in}}{\pgfqpoint{1.423644in}{1.952838in}}{\pgfqpoint{1.415407in}{1.952838in}}%
\pgfpathcurveto{\pgfqpoint{1.407171in}{1.952838in}}{\pgfqpoint{1.399271in}{1.949566in}}{\pgfqpoint{1.393447in}{1.943742in}}%
\pgfpathcurveto{\pgfqpoint{1.387623in}{1.937918in}}{\pgfqpoint{1.384351in}{1.930018in}}{\pgfqpoint{1.384351in}{1.921782in}}%
\pgfpathcurveto{\pgfqpoint{1.384351in}{1.913546in}}{\pgfqpoint{1.387623in}{1.905646in}}{\pgfqpoint{1.393447in}{1.899822in}}%
\pgfpathcurveto{\pgfqpoint{1.399271in}{1.893998in}}{\pgfqpoint{1.407171in}{1.890725in}}{\pgfqpoint{1.415407in}{1.890725in}}%
\pgfpathclose%
\pgfusepath{stroke,fill}%
\end{pgfscope}%
\begin{pgfscope}%
\pgfpathrectangle{\pgfqpoint{0.100000in}{0.212622in}}{\pgfqpoint{3.696000in}{3.696000in}}%
\pgfusepath{clip}%
\pgfsetbuttcap%
\pgfsetroundjoin%
\definecolor{currentfill}{rgb}{0.121569,0.466667,0.705882}%
\pgfsetfillcolor{currentfill}%
\pgfsetfillopacity{0.371387}%
\pgfsetlinewidth{1.003750pt}%
\definecolor{currentstroke}{rgb}{0.121569,0.466667,0.705882}%
\pgfsetstrokecolor{currentstroke}%
\pgfsetstrokeopacity{0.371387}%
\pgfsetdash{}{0pt}%
\pgfpathmoveto{\pgfqpoint{1.923960in}{2.225977in}}%
\pgfpathcurveto{\pgfqpoint{1.932196in}{2.225977in}}{\pgfqpoint{1.940096in}{2.229249in}}{\pgfqpoint{1.945920in}{2.235073in}}%
\pgfpathcurveto{\pgfqpoint{1.951744in}{2.240897in}}{\pgfqpoint{1.955016in}{2.248797in}}{\pgfqpoint{1.955016in}{2.257033in}}%
\pgfpathcurveto{\pgfqpoint{1.955016in}{2.265270in}}{\pgfqpoint{1.951744in}{2.273170in}}{\pgfqpoint{1.945920in}{2.278994in}}%
\pgfpathcurveto{\pgfqpoint{1.940096in}{2.284818in}}{\pgfqpoint{1.932196in}{2.288090in}}{\pgfqpoint{1.923960in}{2.288090in}}%
\pgfpathcurveto{\pgfqpoint{1.915723in}{2.288090in}}{\pgfqpoint{1.907823in}{2.284818in}}{\pgfqpoint{1.901999in}{2.278994in}}%
\pgfpathcurveto{\pgfqpoint{1.896176in}{2.273170in}}{\pgfqpoint{1.892903in}{2.265270in}}{\pgfqpoint{1.892903in}{2.257033in}}%
\pgfpathcurveto{\pgfqpoint{1.892903in}{2.248797in}}{\pgfqpoint{1.896176in}{2.240897in}}{\pgfqpoint{1.901999in}{2.235073in}}%
\pgfpathcurveto{\pgfqpoint{1.907823in}{2.229249in}}{\pgfqpoint{1.915723in}{2.225977in}}{\pgfqpoint{1.923960in}{2.225977in}}%
\pgfpathclose%
\pgfusepath{stroke,fill}%
\end{pgfscope}%
\begin{pgfscope}%
\pgfpathrectangle{\pgfqpoint{0.100000in}{0.212622in}}{\pgfqpoint{3.696000in}{3.696000in}}%
\pgfusepath{clip}%
\pgfsetbuttcap%
\pgfsetroundjoin%
\definecolor{currentfill}{rgb}{0.121569,0.466667,0.705882}%
\pgfsetfillcolor{currentfill}%
\pgfsetfillopacity{0.371603}%
\pgfsetlinewidth{1.003750pt}%
\definecolor{currentstroke}{rgb}{0.121569,0.466667,0.705882}%
\pgfsetstrokecolor{currentstroke}%
\pgfsetstrokeopacity{0.371603}%
\pgfsetdash{}{0pt}%
\pgfpathmoveto{\pgfqpoint{1.894776in}{2.208804in}}%
\pgfpathcurveto{\pgfqpoint{1.903012in}{2.208804in}}{\pgfqpoint{1.910912in}{2.212077in}}{\pgfqpoint{1.916736in}{2.217901in}}%
\pgfpathcurveto{\pgfqpoint{1.922560in}{2.223724in}}{\pgfqpoint{1.925832in}{2.231625in}}{\pgfqpoint{1.925832in}{2.239861in}}%
\pgfpathcurveto{\pgfqpoint{1.925832in}{2.248097in}}{\pgfqpoint{1.922560in}{2.255997in}}{\pgfqpoint{1.916736in}{2.261821in}}%
\pgfpathcurveto{\pgfqpoint{1.910912in}{2.267645in}}{\pgfqpoint{1.903012in}{2.270917in}}{\pgfqpoint{1.894776in}{2.270917in}}%
\pgfpathcurveto{\pgfqpoint{1.886539in}{2.270917in}}{\pgfqpoint{1.878639in}{2.267645in}}{\pgfqpoint{1.872815in}{2.261821in}}%
\pgfpathcurveto{\pgfqpoint{1.866992in}{2.255997in}}{\pgfqpoint{1.863719in}{2.248097in}}{\pgfqpoint{1.863719in}{2.239861in}}%
\pgfpathcurveto{\pgfqpoint{1.863719in}{2.231625in}}{\pgfqpoint{1.866992in}{2.223724in}}{\pgfqpoint{1.872815in}{2.217901in}}%
\pgfpathcurveto{\pgfqpoint{1.878639in}{2.212077in}}{\pgfqpoint{1.886539in}{2.208804in}}{\pgfqpoint{1.894776in}{2.208804in}}%
\pgfpathclose%
\pgfusepath{stroke,fill}%
\end{pgfscope}%
\begin{pgfscope}%
\pgfpathrectangle{\pgfqpoint{0.100000in}{0.212622in}}{\pgfqpoint{3.696000in}{3.696000in}}%
\pgfusepath{clip}%
\pgfsetbuttcap%
\pgfsetroundjoin%
\definecolor{currentfill}{rgb}{0.121569,0.466667,0.705882}%
\pgfsetfillcolor{currentfill}%
\pgfsetfillopacity{0.371905}%
\pgfsetlinewidth{1.003750pt}%
\definecolor{currentstroke}{rgb}{0.121569,0.466667,0.705882}%
\pgfsetstrokecolor{currentstroke}%
\pgfsetstrokeopacity{0.371905}%
\pgfsetdash{}{0pt}%
\pgfpathmoveto{\pgfqpoint{1.460687in}{1.922378in}}%
\pgfpathcurveto{\pgfqpoint{1.468923in}{1.922378in}}{\pgfqpoint{1.476823in}{1.925650in}}{\pgfqpoint{1.482647in}{1.931474in}}%
\pgfpathcurveto{\pgfqpoint{1.488471in}{1.937298in}}{\pgfqpoint{1.491743in}{1.945198in}}{\pgfqpoint{1.491743in}{1.953435in}}%
\pgfpathcurveto{\pgfqpoint{1.491743in}{1.961671in}}{\pgfqpoint{1.488471in}{1.969571in}}{\pgfqpoint{1.482647in}{1.975395in}}%
\pgfpathcurveto{\pgfqpoint{1.476823in}{1.981219in}}{\pgfqpoint{1.468923in}{1.984491in}}{\pgfqpoint{1.460687in}{1.984491in}}%
\pgfpathcurveto{\pgfqpoint{1.452450in}{1.984491in}}{\pgfqpoint{1.444550in}{1.981219in}}{\pgfqpoint{1.438726in}{1.975395in}}%
\pgfpathcurveto{\pgfqpoint{1.432902in}{1.969571in}}{\pgfqpoint{1.429630in}{1.961671in}}{\pgfqpoint{1.429630in}{1.953435in}}%
\pgfpathcurveto{\pgfqpoint{1.429630in}{1.945198in}}{\pgfqpoint{1.432902in}{1.937298in}}{\pgfqpoint{1.438726in}{1.931474in}}%
\pgfpathcurveto{\pgfqpoint{1.444550in}{1.925650in}}{\pgfqpoint{1.452450in}{1.922378in}}{\pgfqpoint{1.460687in}{1.922378in}}%
\pgfpathclose%
\pgfusepath{stroke,fill}%
\end{pgfscope}%
\begin{pgfscope}%
\pgfpathrectangle{\pgfqpoint{0.100000in}{0.212622in}}{\pgfqpoint{3.696000in}{3.696000in}}%
\pgfusepath{clip}%
\pgfsetbuttcap%
\pgfsetroundjoin%
\definecolor{currentfill}{rgb}{0.121569,0.466667,0.705882}%
\pgfsetfillcolor{currentfill}%
\pgfsetfillopacity{0.371933}%
\pgfsetlinewidth{1.003750pt}%
\definecolor{currentstroke}{rgb}{0.121569,0.466667,0.705882}%
\pgfsetstrokecolor{currentstroke}%
\pgfsetstrokeopacity{0.371933}%
\pgfsetdash{}{0pt}%
\pgfpathmoveto{\pgfqpoint{1.462486in}{1.924777in}}%
\pgfpathcurveto{\pgfqpoint{1.470722in}{1.924777in}}{\pgfqpoint{1.478622in}{1.928049in}}{\pgfqpoint{1.484446in}{1.933873in}}%
\pgfpathcurveto{\pgfqpoint{1.490270in}{1.939697in}}{\pgfqpoint{1.493542in}{1.947597in}}{\pgfqpoint{1.493542in}{1.955834in}}%
\pgfpathcurveto{\pgfqpoint{1.493542in}{1.964070in}}{\pgfqpoint{1.490270in}{1.971970in}}{\pgfqpoint{1.484446in}{1.977794in}}%
\pgfpathcurveto{\pgfqpoint{1.478622in}{1.983618in}}{\pgfqpoint{1.470722in}{1.986890in}}{\pgfqpoint{1.462486in}{1.986890in}}%
\pgfpathcurveto{\pgfqpoint{1.454249in}{1.986890in}}{\pgfqpoint{1.446349in}{1.983618in}}{\pgfqpoint{1.440525in}{1.977794in}}%
\pgfpathcurveto{\pgfqpoint{1.434701in}{1.971970in}}{\pgfqpoint{1.431429in}{1.964070in}}{\pgfqpoint{1.431429in}{1.955834in}}%
\pgfpathcurveto{\pgfqpoint{1.431429in}{1.947597in}}{\pgfqpoint{1.434701in}{1.939697in}}{\pgfqpoint{1.440525in}{1.933873in}}%
\pgfpathcurveto{\pgfqpoint{1.446349in}{1.928049in}}{\pgfqpoint{1.454249in}{1.924777in}}{\pgfqpoint{1.462486in}{1.924777in}}%
\pgfpathclose%
\pgfusepath{stroke,fill}%
\end{pgfscope}%
\begin{pgfscope}%
\pgfpathrectangle{\pgfqpoint{0.100000in}{0.212622in}}{\pgfqpoint{3.696000in}{3.696000in}}%
\pgfusepath{clip}%
\pgfsetbuttcap%
\pgfsetroundjoin%
\definecolor{currentfill}{rgb}{0.121569,0.466667,0.705882}%
\pgfsetfillcolor{currentfill}%
\pgfsetfillopacity{0.372115}%
\pgfsetlinewidth{1.003750pt}%
\definecolor{currentstroke}{rgb}{0.121569,0.466667,0.705882}%
\pgfsetstrokecolor{currentstroke}%
\pgfsetstrokeopacity{0.372115}%
\pgfsetdash{}{0pt}%
\pgfpathmoveto{\pgfqpoint{1.923385in}{2.225980in}}%
\pgfpathcurveto{\pgfqpoint{1.931622in}{2.225980in}}{\pgfqpoint{1.939522in}{2.229252in}}{\pgfqpoint{1.945346in}{2.235076in}}%
\pgfpathcurveto{\pgfqpoint{1.951170in}{2.240900in}}{\pgfqpoint{1.954442in}{2.248800in}}{\pgfqpoint{1.954442in}{2.257037in}}%
\pgfpathcurveto{\pgfqpoint{1.954442in}{2.265273in}}{\pgfqpoint{1.951170in}{2.273173in}}{\pgfqpoint{1.945346in}{2.278997in}}%
\pgfpathcurveto{\pgfqpoint{1.939522in}{2.284821in}}{\pgfqpoint{1.931622in}{2.288093in}}{\pgfqpoint{1.923385in}{2.288093in}}%
\pgfpathcurveto{\pgfqpoint{1.915149in}{2.288093in}}{\pgfqpoint{1.907249in}{2.284821in}}{\pgfqpoint{1.901425in}{2.278997in}}%
\pgfpathcurveto{\pgfqpoint{1.895601in}{2.273173in}}{\pgfqpoint{1.892329in}{2.265273in}}{\pgfqpoint{1.892329in}{2.257037in}}%
\pgfpathcurveto{\pgfqpoint{1.892329in}{2.248800in}}{\pgfqpoint{1.895601in}{2.240900in}}{\pgfqpoint{1.901425in}{2.235076in}}%
\pgfpathcurveto{\pgfqpoint{1.907249in}{2.229252in}}{\pgfqpoint{1.915149in}{2.225980in}}{\pgfqpoint{1.923385in}{2.225980in}}%
\pgfpathclose%
\pgfusepath{stroke,fill}%
\end{pgfscope}%
\begin{pgfscope}%
\pgfpathrectangle{\pgfqpoint{0.100000in}{0.212622in}}{\pgfqpoint{3.696000in}{3.696000in}}%
\pgfusepath{clip}%
\pgfsetbuttcap%
\pgfsetroundjoin%
\definecolor{currentfill}{rgb}{0.121569,0.466667,0.705882}%
\pgfsetfillcolor{currentfill}%
\pgfsetfillopacity{0.372404}%
\pgfsetlinewidth{1.003750pt}%
\definecolor{currentstroke}{rgb}{0.121569,0.466667,0.705882}%
\pgfsetstrokecolor{currentstroke}%
\pgfsetstrokeopacity{0.372404}%
\pgfsetdash{}{0pt}%
\pgfpathmoveto{\pgfqpoint{1.821782in}{2.159421in}}%
\pgfpathcurveto{\pgfqpoint{1.830019in}{2.159421in}}{\pgfqpoint{1.837919in}{2.162694in}}{\pgfqpoint{1.843743in}{2.168517in}}%
\pgfpathcurveto{\pgfqpoint{1.849566in}{2.174341in}}{\pgfqpoint{1.852839in}{2.182241in}}{\pgfqpoint{1.852839in}{2.190478in}}%
\pgfpathcurveto{\pgfqpoint{1.852839in}{2.198714in}}{\pgfqpoint{1.849566in}{2.206614in}}{\pgfqpoint{1.843743in}{2.212438in}}%
\pgfpathcurveto{\pgfqpoint{1.837919in}{2.218262in}}{\pgfqpoint{1.830019in}{2.221534in}}{\pgfqpoint{1.821782in}{2.221534in}}%
\pgfpathcurveto{\pgfqpoint{1.813546in}{2.221534in}}{\pgfqpoint{1.805646in}{2.218262in}}{\pgfqpoint{1.799822in}{2.212438in}}%
\pgfpathcurveto{\pgfqpoint{1.793998in}{2.206614in}}{\pgfqpoint{1.790726in}{2.198714in}}{\pgfqpoint{1.790726in}{2.190478in}}%
\pgfpathcurveto{\pgfqpoint{1.790726in}{2.182241in}}{\pgfqpoint{1.793998in}{2.174341in}}{\pgfqpoint{1.799822in}{2.168517in}}%
\pgfpathcurveto{\pgfqpoint{1.805646in}{2.162694in}}{\pgfqpoint{1.813546in}{2.159421in}}{\pgfqpoint{1.821782in}{2.159421in}}%
\pgfpathclose%
\pgfusepath{stroke,fill}%
\end{pgfscope}%
\begin{pgfscope}%
\pgfpathrectangle{\pgfqpoint{0.100000in}{0.212622in}}{\pgfqpoint{3.696000in}{3.696000in}}%
\pgfusepath{clip}%
\pgfsetbuttcap%
\pgfsetroundjoin%
\definecolor{currentfill}{rgb}{0.121569,0.466667,0.705882}%
\pgfsetfillcolor{currentfill}%
\pgfsetfillopacity{0.373035}%
\pgfsetlinewidth{1.003750pt}%
\definecolor{currentstroke}{rgb}{0.121569,0.466667,0.705882}%
\pgfsetstrokecolor{currentstroke}%
\pgfsetstrokeopacity{0.373035}%
\pgfsetdash{}{0pt}%
\pgfpathmoveto{\pgfqpoint{1.814758in}{2.158483in}}%
\pgfpathcurveto{\pgfqpoint{1.822994in}{2.158483in}}{\pgfqpoint{1.830894in}{2.161755in}}{\pgfqpoint{1.836718in}{2.167579in}}%
\pgfpathcurveto{\pgfqpoint{1.842542in}{2.173403in}}{\pgfqpoint{1.845814in}{2.181303in}}{\pgfqpoint{1.845814in}{2.189539in}}%
\pgfpathcurveto{\pgfqpoint{1.845814in}{2.197775in}}{\pgfqpoint{1.842542in}{2.205675in}}{\pgfqpoint{1.836718in}{2.211499in}}%
\pgfpathcurveto{\pgfqpoint{1.830894in}{2.217323in}}{\pgfqpoint{1.822994in}{2.220596in}}{\pgfqpoint{1.814758in}{2.220596in}}%
\pgfpathcurveto{\pgfqpoint{1.806522in}{2.220596in}}{\pgfqpoint{1.798621in}{2.217323in}}{\pgfqpoint{1.792798in}{2.211499in}}%
\pgfpathcurveto{\pgfqpoint{1.786974in}{2.205675in}}{\pgfqpoint{1.783701in}{2.197775in}}{\pgfqpoint{1.783701in}{2.189539in}}%
\pgfpathcurveto{\pgfqpoint{1.783701in}{2.181303in}}{\pgfqpoint{1.786974in}{2.173403in}}{\pgfqpoint{1.792798in}{2.167579in}}%
\pgfpathcurveto{\pgfqpoint{1.798621in}{2.161755in}}{\pgfqpoint{1.806522in}{2.158483in}}{\pgfqpoint{1.814758in}{2.158483in}}%
\pgfpathclose%
\pgfusepath{stroke,fill}%
\end{pgfscope}%
\begin{pgfscope}%
\pgfpathrectangle{\pgfqpoint{0.100000in}{0.212622in}}{\pgfqpoint{3.696000in}{3.696000in}}%
\pgfusepath{clip}%
\pgfsetbuttcap%
\pgfsetroundjoin%
\definecolor{currentfill}{rgb}{0.121569,0.466667,0.705882}%
\pgfsetfillcolor{currentfill}%
\pgfsetfillopacity{0.373064}%
\pgfsetlinewidth{1.003750pt}%
\definecolor{currentstroke}{rgb}{0.121569,0.466667,0.705882}%
\pgfsetstrokecolor{currentstroke}%
\pgfsetstrokeopacity{0.373064}%
\pgfsetdash{}{0pt}%
\pgfpathmoveto{\pgfqpoint{1.466524in}{1.928368in}}%
\pgfpathcurveto{\pgfqpoint{1.474761in}{1.928368in}}{\pgfqpoint{1.482661in}{1.931640in}}{\pgfqpoint{1.488485in}{1.937464in}}%
\pgfpathcurveto{\pgfqpoint{1.494309in}{1.943288in}}{\pgfqpoint{1.497581in}{1.951188in}}{\pgfqpoint{1.497581in}{1.959424in}}%
\pgfpathcurveto{\pgfqpoint{1.497581in}{1.967660in}}{\pgfqpoint{1.494309in}{1.975560in}}{\pgfqpoint{1.488485in}{1.981384in}}%
\pgfpathcurveto{\pgfqpoint{1.482661in}{1.987208in}}{\pgfqpoint{1.474761in}{1.990481in}}{\pgfqpoint{1.466524in}{1.990481in}}%
\pgfpathcurveto{\pgfqpoint{1.458288in}{1.990481in}}{\pgfqpoint{1.450388in}{1.987208in}}{\pgfqpoint{1.444564in}{1.981384in}}%
\pgfpathcurveto{\pgfqpoint{1.438740in}{1.975560in}}{\pgfqpoint{1.435468in}{1.967660in}}{\pgfqpoint{1.435468in}{1.959424in}}%
\pgfpathcurveto{\pgfqpoint{1.435468in}{1.951188in}}{\pgfqpoint{1.438740in}{1.943288in}}{\pgfqpoint{1.444564in}{1.937464in}}%
\pgfpathcurveto{\pgfqpoint{1.450388in}{1.931640in}}{\pgfqpoint{1.458288in}{1.928368in}}{\pgfqpoint{1.466524in}{1.928368in}}%
\pgfpathclose%
\pgfusepath{stroke,fill}%
\end{pgfscope}%
\begin{pgfscope}%
\pgfpathrectangle{\pgfqpoint{0.100000in}{0.212622in}}{\pgfqpoint{3.696000in}{3.696000in}}%
\pgfusepath{clip}%
\pgfsetbuttcap%
\pgfsetroundjoin%
\definecolor{currentfill}{rgb}{0.121569,0.466667,0.705882}%
\pgfsetfillcolor{currentfill}%
\pgfsetfillopacity{0.373389}%
\pgfsetlinewidth{1.003750pt}%
\definecolor{currentstroke}{rgb}{0.121569,0.466667,0.705882}%
\pgfsetstrokecolor{currentstroke}%
\pgfsetstrokeopacity{0.373389}%
\pgfsetdash{}{0pt}%
\pgfpathmoveto{\pgfqpoint{1.920104in}{2.222893in}}%
\pgfpathcurveto{\pgfqpoint{1.928341in}{2.222893in}}{\pgfqpoint{1.936241in}{2.226165in}}{\pgfqpoint{1.942065in}{2.231989in}}%
\pgfpathcurveto{\pgfqpoint{1.947889in}{2.237813in}}{\pgfqpoint{1.951161in}{2.245713in}}{\pgfqpoint{1.951161in}{2.253949in}}%
\pgfpathcurveto{\pgfqpoint{1.951161in}{2.262186in}}{\pgfqpoint{1.947889in}{2.270086in}}{\pgfqpoint{1.942065in}{2.275910in}}%
\pgfpathcurveto{\pgfqpoint{1.936241in}{2.281734in}}{\pgfqpoint{1.928341in}{2.285006in}}{\pgfqpoint{1.920104in}{2.285006in}}%
\pgfpathcurveto{\pgfqpoint{1.911868in}{2.285006in}}{\pgfqpoint{1.903968in}{2.281734in}}{\pgfqpoint{1.898144in}{2.275910in}}%
\pgfpathcurveto{\pgfqpoint{1.892320in}{2.270086in}}{\pgfqpoint{1.889048in}{2.262186in}}{\pgfqpoint{1.889048in}{2.253949in}}%
\pgfpathcurveto{\pgfqpoint{1.889048in}{2.245713in}}{\pgfqpoint{1.892320in}{2.237813in}}{\pgfqpoint{1.898144in}{2.231989in}}%
\pgfpathcurveto{\pgfqpoint{1.903968in}{2.226165in}}{\pgfqpoint{1.911868in}{2.222893in}}{\pgfqpoint{1.920104in}{2.222893in}}%
\pgfpathclose%
\pgfusepath{stroke,fill}%
\end{pgfscope}%
\begin{pgfscope}%
\pgfpathrectangle{\pgfqpoint{0.100000in}{0.212622in}}{\pgfqpoint{3.696000in}{3.696000in}}%
\pgfusepath{clip}%
\pgfsetbuttcap%
\pgfsetroundjoin%
\definecolor{currentfill}{rgb}{0.121569,0.466667,0.705882}%
\pgfsetfillcolor{currentfill}%
\pgfsetfillopacity{0.373494}%
\pgfsetlinewidth{1.003750pt}%
\definecolor{currentstroke}{rgb}{0.121569,0.466667,0.705882}%
\pgfsetstrokecolor{currentstroke}%
\pgfsetstrokeopacity{0.373494}%
\pgfsetdash{}{0pt}%
\pgfpathmoveto{\pgfqpoint{1.921133in}{2.223731in}}%
\pgfpathcurveto{\pgfqpoint{1.929369in}{2.223731in}}{\pgfqpoint{1.937270in}{2.227003in}}{\pgfqpoint{1.943093in}{2.232827in}}%
\pgfpathcurveto{\pgfqpoint{1.948917in}{2.238651in}}{\pgfqpoint{1.952190in}{2.246551in}}{\pgfqpoint{1.952190in}{2.254787in}}%
\pgfpathcurveto{\pgfqpoint{1.952190in}{2.263024in}}{\pgfqpoint{1.948917in}{2.270924in}}{\pgfqpoint{1.943093in}{2.276748in}}%
\pgfpathcurveto{\pgfqpoint{1.937270in}{2.282572in}}{\pgfqpoint{1.929369in}{2.285844in}}{\pgfqpoint{1.921133in}{2.285844in}}%
\pgfpathcurveto{\pgfqpoint{1.912897in}{2.285844in}}{\pgfqpoint{1.904997in}{2.282572in}}{\pgfqpoint{1.899173in}{2.276748in}}%
\pgfpathcurveto{\pgfqpoint{1.893349in}{2.270924in}}{\pgfqpoint{1.890077in}{2.263024in}}{\pgfqpoint{1.890077in}{2.254787in}}%
\pgfpathcurveto{\pgfqpoint{1.890077in}{2.246551in}}{\pgfqpoint{1.893349in}{2.238651in}}{\pgfqpoint{1.899173in}{2.232827in}}%
\pgfpathcurveto{\pgfqpoint{1.904997in}{2.227003in}}{\pgfqpoint{1.912897in}{2.223731in}}{\pgfqpoint{1.921133in}{2.223731in}}%
\pgfpathclose%
\pgfusepath{stroke,fill}%
\end{pgfscope}%
\begin{pgfscope}%
\pgfpathrectangle{\pgfqpoint{0.100000in}{0.212622in}}{\pgfqpoint{3.696000in}{3.696000in}}%
\pgfusepath{clip}%
\pgfsetbuttcap%
\pgfsetroundjoin%
\definecolor{currentfill}{rgb}{0.121569,0.466667,0.705882}%
\pgfsetfillcolor{currentfill}%
\pgfsetfillopacity{0.373866}%
\pgfsetlinewidth{1.003750pt}%
\definecolor{currentstroke}{rgb}{0.121569,0.466667,0.705882}%
\pgfsetstrokecolor{currentstroke}%
\pgfsetstrokeopacity{0.373866}%
\pgfsetdash{}{0pt}%
\pgfpathmoveto{\pgfqpoint{1.924248in}{2.224796in}}%
\pgfpathcurveto{\pgfqpoint{1.932485in}{2.224796in}}{\pgfqpoint{1.940385in}{2.228069in}}{\pgfqpoint{1.946209in}{2.233893in}}%
\pgfpathcurveto{\pgfqpoint{1.952032in}{2.239717in}}{\pgfqpoint{1.955305in}{2.247617in}}{\pgfqpoint{1.955305in}{2.255853in}}%
\pgfpathcurveto{\pgfqpoint{1.955305in}{2.264089in}}{\pgfqpoint{1.952032in}{2.271989in}}{\pgfqpoint{1.946209in}{2.277813in}}%
\pgfpathcurveto{\pgfqpoint{1.940385in}{2.283637in}}{\pgfqpoint{1.932485in}{2.286909in}}{\pgfqpoint{1.924248in}{2.286909in}}%
\pgfpathcurveto{\pgfqpoint{1.916012in}{2.286909in}}{\pgfqpoint{1.908112in}{2.283637in}}{\pgfqpoint{1.902288in}{2.277813in}}%
\pgfpathcurveto{\pgfqpoint{1.896464in}{2.271989in}}{\pgfqpoint{1.893192in}{2.264089in}}{\pgfqpoint{1.893192in}{2.255853in}}%
\pgfpathcurveto{\pgfqpoint{1.893192in}{2.247617in}}{\pgfqpoint{1.896464in}{2.239717in}}{\pgfqpoint{1.902288in}{2.233893in}}%
\pgfpathcurveto{\pgfqpoint{1.908112in}{2.228069in}}{\pgfqpoint{1.916012in}{2.224796in}}{\pgfqpoint{1.924248in}{2.224796in}}%
\pgfpathclose%
\pgfusepath{stroke,fill}%
\end{pgfscope}%
\begin{pgfscope}%
\pgfpathrectangle{\pgfqpoint{0.100000in}{0.212622in}}{\pgfqpoint{3.696000in}{3.696000in}}%
\pgfusepath{clip}%
\pgfsetbuttcap%
\pgfsetroundjoin%
\definecolor{currentfill}{rgb}{0.121569,0.466667,0.705882}%
\pgfsetfillcolor{currentfill}%
\pgfsetfillopacity{0.374393}%
\pgfsetlinewidth{1.003750pt}%
\definecolor{currentstroke}{rgb}{0.121569,0.466667,0.705882}%
\pgfsetstrokecolor{currentstroke}%
\pgfsetstrokeopacity{0.374393}%
\pgfsetdash{}{0pt}%
\pgfpathmoveto{\pgfqpoint{1.934256in}{2.224456in}}%
\pgfpathcurveto{\pgfqpoint{1.942492in}{2.224456in}}{\pgfqpoint{1.950392in}{2.227729in}}{\pgfqpoint{1.956216in}{2.233553in}}%
\pgfpathcurveto{\pgfqpoint{1.962040in}{2.239376in}}{\pgfqpoint{1.965312in}{2.247276in}}{\pgfqpoint{1.965312in}{2.255513in}}%
\pgfpathcurveto{\pgfqpoint{1.965312in}{2.263749in}}{\pgfqpoint{1.962040in}{2.271649in}}{\pgfqpoint{1.956216in}{2.277473in}}%
\pgfpathcurveto{\pgfqpoint{1.950392in}{2.283297in}}{\pgfqpoint{1.942492in}{2.286569in}}{\pgfqpoint{1.934256in}{2.286569in}}%
\pgfpathcurveto{\pgfqpoint{1.926019in}{2.286569in}}{\pgfqpoint{1.918119in}{2.283297in}}{\pgfqpoint{1.912295in}{2.277473in}}%
\pgfpathcurveto{\pgfqpoint{1.906471in}{2.271649in}}{\pgfqpoint{1.903199in}{2.263749in}}{\pgfqpoint{1.903199in}{2.255513in}}%
\pgfpathcurveto{\pgfqpoint{1.903199in}{2.247276in}}{\pgfqpoint{1.906471in}{2.239376in}}{\pgfqpoint{1.912295in}{2.233553in}}%
\pgfpathcurveto{\pgfqpoint{1.918119in}{2.227729in}}{\pgfqpoint{1.926019in}{2.224456in}}{\pgfqpoint{1.934256in}{2.224456in}}%
\pgfpathclose%
\pgfusepath{stroke,fill}%
\end{pgfscope}%
\begin{pgfscope}%
\pgfpathrectangle{\pgfqpoint{0.100000in}{0.212622in}}{\pgfqpoint{3.696000in}{3.696000in}}%
\pgfusepath{clip}%
\pgfsetbuttcap%
\pgfsetroundjoin%
\definecolor{currentfill}{rgb}{0.121569,0.466667,0.705882}%
\pgfsetfillcolor{currentfill}%
\pgfsetfillopacity{0.374425}%
\pgfsetlinewidth{1.003750pt}%
\definecolor{currentstroke}{rgb}{0.121569,0.466667,0.705882}%
\pgfsetstrokecolor{currentstroke}%
\pgfsetstrokeopacity{0.374425}%
\pgfsetdash{}{0pt}%
\pgfpathmoveto{\pgfqpoint{1.801847in}{2.151164in}}%
\pgfpathcurveto{\pgfqpoint{1.810083in}{2.151164in}}{\pgfqpoint{1.817983in}{2.154436in}}{\pgfqpoint{1.823807in}{2.160260in}}%
\pgfpathcurveto{\pgfqpoint{1.829631in}{2.166084in}}{\pgfqpoint{1.832903in}{2.173984in}}{\pgfqpoint{1.832903in}{2.182220in}}%
\pgfpathcurveto{\pgfqpoint{1.832903in}{2.190457in}}{\pgfqpoint{1.829631in}{2.198357in}}{\pgfqpoint{1.823807in}{2.204180in}}%
\pgfpathcurveto{\pgfqpoint{1.817983in}{2.210004in}}{\pgfqpoint{1.810083in}{2.213277in}}{\pgfqpoint{1.801847in}{2.213277in}}%
\pgfpathcurveto{\pgfqpoint{1.793611in}{2.213277in}}{\pgfqpoint{1.785710in}{2.210004in}}{\pgfqpoint{1.779887in}{2.204180in}}%
\pgfpathcurveto{\pgfqpoint{1.774063in}{2.198357in}}{\pgfqpoint{1.770790in}{2.190457in}}{\pgfqpoint{1.770790in}{2.182220in}}%
\pgfpathcurveto{\pgfqpoint{1.770790in}{2.173984in}}{\pgfqpoint{1.774063in}{2.166084in}}{\pgfqpoint{1.779887in}{2.160260in}}%
\pgfpathcurveto{\pgfqpoint{1.785710in}{2.154436in}}{\pgfqpoint{1.793611in}{2.151164in}}{\pgfqpoint{1.801847in}{2.151164in}}%
\pgfpathclose%
\pgfusepath{stroke,fill}%
\end{pgfscope}%
\begin{pgfscope}%
\pgfpathrectangle{\pgfqpoint{0.100000in}{0.212622in}}{\pgfqpoint{3.696000in}{3.696000in}}%
\pgfusepath{clip}%
\pgfsetbuttcap%
\pgfsetroundjoin%
\definecolor{currentfill}{rgb}{0.121569,0.466667,0.705882}%
\pgfsetfillcolor{currentfill}%
\pgfsetfillopacity{0.374490}%
\pgfsetlinewidth{1.003750pt}%
\definecolor{currentstroke}{rgb}{0.121569,0.466667,0.705882}%
\pgfsetstrokecolor{currentstroke}%
\pgfsetstrokeopacity{0.374490}%
\pgfsetdash{}{0pt}%
\pgfpathmoveto{\pgfqpoint{1.939601in}{2.225500in}}%
\pgfpathcurveto{\pgfqpoint{1.947837in}{2.225500in}}{\pgfqpoint{1.955737in}{2.228772in}}{\pgfqpoint{1.961561in}{2.234596in}}%
\pgfpathcurveto{\pgfqpoint{1.967385in}{2.240420in}}{\pgfqpoint{1.970657in}{2.248320in}}{\pgfqpoint{1.970657in}{2.256556in}}%
\pgfpathcurveto{\pgfqpoint{1.970657in}{2.264793in}}{\pgfqpoint{1.967385in}{2.272693in}}{\pgfqpoint{1.961561in}{2.278517in}}%
\pgfpathcurveto{\pgfqpoint{1.955737in}{2.284340in}}{\pgfqpoint{1.947837in}{2.287613in}}{\pgfqpoint{1.939601in}{2.287613in}}%
\pgfpathcurveto{\pgfqpoint{1.931364in}{2.287613in}}{\pgfqpoint{1.923464in}{2.284340in}}{\pgfqpoint{1.917640in}{2.278517in}}%
\pgfpathcurveto{\pgfqpoint{1.911816in}{2.272693in}}{\pgfqpoint{1.908544in}{2.264793in}}{\pgfqpoint{1.908544in}{2.256556in}}%
\pgfpathcurveto{\pgfqpoint{1.908544in}{2.248320in}}{\pgfqpoint{1.911816in}{2.240420in}}{\pgfqpoint{1.917640in}{2.234596in}}%
\pgfpathcurveto{\pgfqpoint{1.923464in}{2.228772in}}{\pgfqpoint{1.931364in}{2.225500in}}{\pgfqpoint{1.939601in}{2.225500in}}%
\pgfpathclose%
\pgfusepath{stroke,fill}%
\end{pgfscope}%
\begin{pgfscope}%
\pgfpathrectangle{\pgfqpoint{0.100000in}{0.212622in}}{\pgfqpoint{3.696000in}{3.696000in}}%
\pgfusepath{clip}%
\pgfsetbuttcap%
\pgfsetroundjoin%
\definecolor{currentfill}{rgb}{0.121569,0.466667,0.705882}%
\pgfsetfillcolor{currentfill}%
\pgfsetfillopacity{0.374743}%
\pgfsetlinewidth{1.003750pt}%
\definecolor{currentstroke}{rgb}{0.121569,0.466667,0.705882}%
\pgfsetstrokecolor{currentstroke}%
\pgfsetstrokeopacity{0.374743}%
\pgfsetdash{}{0pt}%
\pgfpathmoveto{\pgfqpoint{1.455736in}{1.916729in}}%
\pgfpathcurveto{\pgfqpoint{1.463973in}{1.916729in}}{\pgfqpoint{1.471873in}{1.920001in}}{\pgfqpoint{1.477697in}{1.925825in}}%
\pgfpathcurveto{\pgfqpoint{1.483521in}{1.931649in}}{\pgfqpoint{1.486793in}{1.939549in}}{\pgfqpoint{1.486793in}{1.947785in}}%
\pgfpathcurveto{\pgfqpoint{1.486793in}{1.956022in}}{\pgfqpoint{1.483521in}{1.963922in}}{\pgfqpoint{1.477697in}{1.969746in}}%
\pgfpathcurveto{\pgfqpoint{1.471873in}{1.975570in}}{\pgfqpoint{1.463973in}{1.978842in}}{\pgfqpoint{1.455736in}{1.978842in}}%
\pgfpathcurveto{\pgfqpoint{1.447500in}{1.978842in}}{\pgfqpoint{1.439600in}{1.975570in}}{\pgfqpoint{1.433776in}{1.969746in}}%
\pgfpathcurveto{\pgfqpoint{1.427952in}{1.963922in}}{\pgfqpoint{1.424680in}{1.956022in}}{\pgfqpoint{1.424680in}{1.947785in}}%
\pgfpathcurveto{\pgfqpoint{1.424680in}{1.939549in}}{\pgfqpoint{1.427952in}{1.931649in}}{\pgfqpoint{1.433776in}{1.925825in}}%
\pgfpathcurveto{\pgfqpoint{1.439600in}{1.920001in}}{\pgfqpoint{1.447500in}{1.916729in}}{\pgfqpoint{1.455736in}{1.916729in}}%
\pgfpathclose%
\pgfusepath{stroke,fill}%
\end{pgfscope}%
\begin{pgfscope}%
\pgfpathrectangle{\pgfqpoint{0.100000in}{0.212622in}}{\pgfqpoint{3.696000in}{3.696000in}}%
\pgfusepath{clip}%
\pgfsetbuttcap%
\pgfsetroundjoin%
\definecolor{currentfill}{rgb}{0.121569,0.466667,0.705882}%
\pgfsetfillcolor{currentfill}%
\pgfsetfillopacity{0.374801}%
\pgfsetlinewidth{1.003750pt}%
\definecolor{currentstroke}{rgb}{0.121569,0.466667,0.705882}%
\pgfsetstrokecolor{currentstroke}%
\pgfsetstrokeopacity{0.374801}%
\pgfsetdash{}{0pt}%
\pgfpathmoveto{\pgfqpoint{1.924670in}{2.224144in}}%
\pgfpathcurveto{\pgfqpoint{1.932907in}{2.224144in}}{\pgfqpoint{1.940807in}{2.227417in}}{\pgfqpoint{1.946631in}{2.233241in}}%
\pgfpathcurveto{\pgfqpoint{1.952455in}{2.239065in}}{\pgfqpoint{1.955727in}{2.246965in}}{\pgfqpoint{1.955727in}{2.255201in}}%
\pgfpathcurveto{\pgfqpoint{1.955727in}{2.263437in}}{\pgfqpoint{1.952455in}{2.271337in}}{\pgfqpoint{1.946631in}{2.277161in}}%
\pgfpathcurveto{\pgfqpoint{1.940807in}{2.282985in}}{\pgfqpoint{1.932907in}{2.286257in}}{\pgfqpoint{1.924670in}{2.286257in}}%
\pgfpathcurveto{\pgfqpoint{1.916434in}{2.286257in}}{\pgfqpoint{1.908534in}{2.282985in}}{\pgfqpoint{1.902710in}{2.277161in}}%
\pgfpathcurveto{\pgfqpoint{1.896886in}{2.271337in}}{\pgfqpoint{1.893614in}{2.263437in}}{\pgfqpoint{1.893614in}{2.255201in}}%
\pgfpathcurveto{\pgfqpoint{1.893614in}{2.246965in}}{\pgfqpoint{1.896886in}{2.239065in}}{\pgfqpoint{1.902710in}{2.233241in}}%
\pgfpathcurveto{\pgfqpoint{1.908534in}{2.227417in}}{\pgfqpoint{1.916434in}{2.224144in}}{\pgfqpoint{1.924670in}{2.224144in}}%
\pgfpathclose%
\pgfusepath{stroke,fill}%
\end{pgfscope}%
\begin{pgfscope}%
\pgfpathrectangle{\pgfqpoint{0.100000in}{0.212622in}}{\pgfqpoint{3.696000in}{3.696000in}}%
\pgfusepath{clip}%
\pgfsetbuttcap%
\pgfsetroundjoin%
\definecolor{currentfill}{rgb}{0.121569,0.466667,0.705882}%
\pgfsetfillcolor{currentfill}%
\pgfsetfillopacity{0.374816}%
\pgfsetlinewidth{1.003750pt}%
\definecolor{currentstroke}{rgb}{0.121569,0.466667,0.705882}%
\pgfsetstrokecolor{currentstroke}%
\pgfsetstrokeopacity{0.374816}%
\pgfsetdash{}{0pt}%
\pgfpathmoveto{\pgfqpoint{1.918197in}{2.219171in}}%
\pgfpathcurveto{\pgfqpoint{1.926433in}{2.219171in}}{\pgfqpoint{1.934333in}{2.222443in}}{\pgfqpoint{1.940157in}{2.228267in}}%
\pgfpathcurveto{\pgfqpoint{1.945981in}{2.234091in}}{\pgfqpoint{1.949254in}{2.241991in}}{\pgfqpoint{1.949254in}{2.250228in}}%
\pgfpathcurveto{\pgfqpoint{1.949254in}{2.258464in}}{\pgfqpoint{1.945981in}{2.266364in}}{\pgfqpoint{1.940157in}{2.272188in}}%
\pgfpathcurveto{\pgfqpoint{1.934333in}{2.278012in}}{\pgfqpoint{1.926433in}{2.281284in}}{\pgfqpoint{1.918197in}{2.281284in}}%
\pgfpathcurveto{\pgfqpoint{1.909961in}{2.281284in}}{\pgfqpoint{1.902061in}{2.278012in}}{\pgfqpoint{1.896237in}{2.272188in}}%
\pgfpathcurveto{\pgfqpoint{1.890413in}{2.266364in}}{\pgfqpoint{1.887141in}{2.258464in}}{\pgfqpoint{1.887141in}{2.250228in}}%
\pgfpathcurveto{\pgfqpoint{1.887141in}{2.241991in}}{\pgfqpoint{1.890413in}{2.234091in}}{\pgfqpoint{1.896237in}{2.228267in}}%
\pgfpathcurveto{\pgfqpoint{1.902061in}{2.222443in}}{\pgfqpoint{1.909961in}{2.219171in}}{\pgfqpoint{1.918197in}{2.219171in}}%
\pgfpathclose%
\pgfusepath{stroke,fill}%
\end{pgfscope}%
\begin{pgfscope}%
\pgfpathrectangle{\pgfqpoint{0.100000in}{0.212622in}}{\pgfqpoint{3.696000in}{3.696000in}}%
\pgfusepath{clip}%
\pgfsetbuttcap%
\pgfsetroundjoin%
\definecolor{currentfill}{rgb}{0.121569,0.466667,0.705882}%
\pgfsetfillcolor{currentfill}%
\pgfsetfillopacity{0.374826}%
\pgfsetlinewidth{1.003750pt}%
\definecolor{currentstroke}{rgb}{0.121569,0.466667,0.705882}%
\pgfsetstrokecolor{currentstroke}%
\pgfsetstrokeopacity{0.374826}%
\pgfsetdash{}{0pt}%
\pgfpathmoveto{\pgfqpoint{1.701436in}{2.086370in}}%
\pgfpathcurveto{\pgfqpoint{1.709673in}{2.086370in}}{\pgfqpoint{1.717573in}{2.089643in}}{\pgfqpoint{1.723397in}{2.095467in}}%
\pgfpathcurveto{\pgfqpoint{1.729221in}{2.101291in}}{\pgfqpoint{1.732493in}{2.109191in}}{\pgfqpoint{1.732493in}{2.117427in}}%
\pgfpathcurveto{\pgfqpoint{1.732493in}{2.125663in}}{\pgfqpoint{1.729221in}{2.133563in}}{\pgfqpoint{1.723397in}{2.139387in}}%
\pgfpathcurveto{\pgfqpoint{1.717573in}{2.145211in}}{\pgfqpoint{1.709673in}{2.148483in}}{\pgfqpoint{1.701436in}{2.148483in}}%
\pgfpathcurveto{\pgfqpoint{1.693200in}{2.148483in}}{\pgfqpoint{1.685300in}{2.145211in}}{\pgfqpoint{1.679476in}{2.139387in}}%
\pgfpathcurveto{\pgfqpoint{1.673652in}{2.133563in}}{\pgfqpoint{1.670380in}{2.125663in}}{\pgfqpoint{1.670380in}{2.117427in}}%
\pgfpathcurveto{\pgfqpoint{1.670380in}{2.109191in}}{\pgfqpoint{1.673652in}{2.101291in}}{\pgfqpoint{1.679476in}{2.095467in}}%
\pgfpathcurveto{\pgfqpoint{1.685300in}{2.089643in}}{\pgfqpoint{1.693200in}{2.086370in}}{\pgfqpoint{1.701436in}{2.086370in}}%
\pgfpathclose%
\pgfusepath{stroke,fill}%
\end{pgfscope}%
\begin{pgfscope}%
\pgfpathrectangle{\pgfqpoint{0.100000in}{0.212622in}}{\pgfqpoint{3.696000in}{3.696000in}}%
\pgfusepath{clip}%
\pgfsetbuttcap%
\pgfsetroundjoin%
\definecolor{currentfill}{rgb}{0.121569,0.466667,0.705882}%
\pgfsetfillcolor{currentfill}%
\pgfsetfillopacity{0.374912}%
\pgfsetlinewidth{1.003750pt}%
\definecolor{currentstroke}{rgb}{0.121569,0.466667,0.705882}%
\pgfsetstrokecolor{currentstroke}%
\pgfsetstrokeopacity{0.374912}%
\pgfsetdash{}{0pt}%
\pgfpathmoveto{\pgfqpoint{1.803827in}{2.151376in}}%
\pgfpathcurveto{\pgfqpoint{1.812063in}{2.151376in}}{\pgfqpoint{1.819963in}{2.154649in}}{\pgfqpoint{1.825787in}{2.160473in}}%
\pgfpathcurveto{\pgfqpoint{1.831611in}{2.166296in}}{\pgfqpoint{1.834883in}{2.174197in}}{\pgfqpoint{1.834883in}{2.182433in}}%
\pgfpathcurveto{\pgfqpoint{1.834883in}{2.190669in}}{\pgfqpoint{1.831611in}{2.198569in}}{\pgfqpoint{1.825787in}{2.204393in}}%
\pgfpathcurveto{\pgfqpoint{1.819963in}{2.210217in}}{\pgfqpoint{1.812063in}{2.213489in}}{\pgfqpoint{1.803827in}{2.213489in}}%
\pgfpathcurveto{\pgfqpoint{1.795590in}{2.213489in}}{\pgfqpoint{1.787690in}{2.210217in}}{\pgfqpoint{1.781866in}{2.204393in}}%
\pgfpathcurveto{\pgfqpoint{1.776042in}{2.198569in}}{\pgfqpoint{1.772770in}{2.190669in}}{\pgfqpoint{1.772770in}{2.182433in}}%
\pgfpathcurveto{\pgfqpoint{1.772770in}{2.174197in}}{\pgfqpoint{1.776042in}{2.166296in}}{\pgfqpoint{1.781866in}{2.160473in}}%
\pgfpathcurveto{\pgfqpoint{1.787690in}{2.154649in}}{\pgfqpoint{1.795590in}{2.151376in}}{\pgfqpoint{1.803827in}{2.151376in}}%
\pgfpathclose%
\pgfusepath{stroke,fill}%
\end{pgfscope}%
\begin{pgfscope}%
\pgfpathrectangle{\pgfqpoint{0.100000in}{0.212622in}}{\pgfqpoint{3.696000in}{3.696000in}}%
\pgfusepath{clip}%
\pgfsetbuttcap%
\pgfsetroundjoin%
\definecolor{currentfill}{rgb}{0.121569,0.466667,0.705882}%
\pgfsetfillcolor{currentfill}%
\pgfsetfillopacity{0.374912}%
\pgfsetlinewidth{1.003750pt}%
\definecolor{currentstroke}{rgb}{0.121569,0.466667,0.705882}%
\pgfsetstrokecolor{currentstroke}%
\pgfsetstrokeopacity{0.374912}%
\pgfsetdash{}{0pt}%
\pgfpathmoveto{\pgfqpoint{1.946451in}{2.232650in}}%
\pgfpathcurveto{\pgfqpoint{1.954688in}{2.232650in}}{\pgfqpoint{1.962588in}{2.235923in}}{\pgfqpoint{1.968412in}{2.241746in}}%
\pgfpathcurveto{\pgfqpoint{1.974236in}{2.247570in}}{\pgfqpoint{1.977508in}{2.255470in}}{\pgfqpoint{1.977508in}{2.263707in}}%
\pgfpathcurveto{\pgfqpoint{1.977508in}{2.271943in}}{\pgfqpoint{1.974236in}{2.279843in}}{\pgfqpoint{1.968412in}{2.285667in}}%
\pgfpathcurveto{\pgfqpoint{1.962588in}{2.291491in}}{\pgfqpoint{1.954688in}{2.294763in}}{\pgfqpoint{1.946451in}{2.294763in}}%
\pgfpathcurveto{\pgfqpoint{1.938215in}{2.294763in}}{\pgfqpoint{1.930315in}{2.291491in}}{\pgfqpoint{1.924491in}{2.285667in}}%
\pgfpathcurveto{\pgfqpoint{1.918667in}{2.279843in}}{\pgfqpoint{1.915395in}{2.271943in}}{\pgfqpoint{1.915395in}{2.263707in}}%
\pgfpathcurveto{\pgfqpoint{1.915395in}{2.255470in}}{\pgfqpoint{1.918667in}{2.247570in}}{\pgfqpoint{1.924491in}{2.241746in}}%
\pgfpathcurveto{\pgfqpoint{1.930315in}{2.235923in}}{\pgfqpoint{1.938215in}{2.232650in}}{\pgfqpoint{1.946451in}{2.232650in}}%
\pgfpathclose%
\pgfusepath{stroke,fill}%
\end{pgfscope}%
\begin{pgfscope}%
\pgfpathrectangle{\pgfqpoint{0.100000in}{0.212622in}}{\pgfqpoint{3.696000in}{3.696000in}}%
\pgfusepath{clip}%
\pgfsetbuttcap%
\pgfsetroundjoin%
\definecolor{currentfill}{rgb}{0.121569,0.466667,0.705882}%
\pgfsetfillcolor{currentfill}%
\pgfsetfillopacity{0.375155}%
\pgfsetlinewidth{1.003750pt}%
\definecolor{currentstroke}{rgb}{0.121569,0.466667,0.705882}%
\pgfsetstrokecolor{currentstroke}%
\pgfsetstrokeopacity{0.375155}%
\pgfsetdash{}{0pt}%
\pgfpathmoveto{\pgfqpoint{1.927012in}{2.222328in}}%
\pgfpathcurveto{\pgfqpoint{1.935248in}{2.222328in}}{\pgfqpoint{1.943148in}{2.225600in}}{\pgfqpoint{1.948972in}{2.231424in}}%
\pgfpathcurveto{\pgfqpoint{1.954796in}{2.237248in}}{\pgfqpoint{1.958069in}{2.245148in}}{\pgfqpoint{1.958069in}{2.253384in}}%
\pgfpathcurveto{\pgfqpoint{1.958069in}{2.261620in}}{\pgfqpoint{1.954796in}{2.269520in}}{\pgfqpoint{1.948972in}{2.275344in}}%
\pgfpathcurveto{\pgfqpoint{1.943148in}{2.281168in}}{\pgfqpoint{1.935248in}{2.284441in}}{\pgfqpoint{1.927012in}{2.284441in}}%
\pgfpathcurveto{\pgfqpoint{1.918776in}{2.284441in}}{\pgfqpoint{1.910876in}{2.281168in}}{\pgfqpoint{1.905052in}{2.275344in}}%
\pgfpathcurveto{\pgfqpoint{1.899228in}{2.269520in}}{\pgfqpoint{1.895956in}{2.261620in}}{\pgfqpoint{1.895956in}{2.253384in}}%
\pgfpathcurveto{\pgfqpoint{1.895956in}{2.245148in}}{\pgfqpoint{1.899228in}{2.237248in}}{\pgfqpoint{1.905052in}{2.231424in}}%
\pgfpathcurveto{\pgfqpoint{1.910876in}{2.225600in}}{\pgfqpoint{1.918776in}{2.222328in}}{\pgfqpoint{1.927012in}{2.222328in}}%
\pgfpathclose%
\pgfusepath{stroke,fill}%
\end{pgfscope}%
\begin{pgfscope}%
\pgfpathrectangle{\pgfqpoint{0.100000in}{0.212622in}}{\pgfqpoint{3.696000in}{3.696000in}}%
\pgfusepath{clip}%
\pgfsetbuttcap%
\pgfsetroundjoin%
\definecolor{currentfill}{rgb}{0.121569,0.466667,0.705882}%
\pgfsetfillcolor{currentfill}%
\pgfsetfillopacity{0.376747}%
\pgfsetlinewidth{1.003750pt}%
\definecolor{currentstroke}{rgb}{0.121569,0.466667,0.705882}%
\pgfsetstrokecolor{currentstroke}%
\pgfsetstrokeopacity{0.376747}%
\pgfsetdash{}{0pt}%
\pgfpathmoveto{\pgfqpoint{1.459254in}{1.921335in}}%
\pgfpathcurveto{\pgfqpoint{1.467490in}{1.921335in}}{\pgfqpoint{1.475390in}{1.924607in}}{\pgfqpoint{1.481214in}{1.930431in}}%
\pgfpathcurveto{\pgfqpoint{1.487038in}{1.936255in}}{\pgfqpoint{1.490310in}{1.944155in}}{\pgfqpoint{1.490310in}{1.952391in}}%
\pgfpathcurveto{\pgfqpoint{1.490310in}{1.960628in}}{\pgfqpoint{1.487038in}{1.968528in}}{\pgfqpoint{1.481214in}{1.974352in}}%
\pgfpathcurveto{\pgfqpoint{1.475390in}{1.980176in}}{\pgfqpoint{1.467490in}{1.983448in}}{\pgfqpoint{1.459254in}{1.983448in}}%
\pgfpathcurveto{\pgfqpoint{1.451018in}{1.983448in}}{\pgfqpoint{1.443117in}{1.980176in}}{\pgfqpoint{1.437294in}{1.974352in}}%
\pgfpathcurveto{\pgfqpoint{1.431470in}{1.968528in}}{\pgfqpoint{1.428197in}{1.960628in}}{\pgfqpoint{1.428197in}{1.952391in}}%
\pgfpathcurveto{\pgfqpoint{1.428197in}{1.944155in}}{\pgfqpoint{1.431470in}{1.936255in}}{\pgfqpoint{1.437294in}{1.930431in}}%
\pgfpathcurveto{\pgfqpoint{1.443117in}{1.924607in}}{\pgfqpoint{1.451018in}{1.921335in}}{\pgfqpoint{1.459254in}{1.921335in}}%
\pgfpathclose%
\pgfusepath{stroke,fill}%
\end{pgfscope}%
\begin{pgfscope}%
\pgfpathrectangle{\pgfqpoint{0.100000in}{0.212622in}}{\pgfqpoint{3.696000in}{3.696000in}}%
\pgfusepath{clip}%
\pgfsetbuttcap%
\pgfsetroundjoin%
\definecolor{currentfill}{rgb}{0.121569,0.466667,0.705882}%
\pgfsetfillcolor{currentfill}%
\pgfsetfillopacity{0.376771}%
\pgfsetlinewidth{1.003750pt}%
\definecolor{currentstroke}{rgb}{0.121569,0.466667,0.705882}%
\pgfsetstrokecolor{currentstroke}%
\pgfsetstrokeopacity{0.376771}%
\pgfsetdash{}{0pt}%
\pgfpathmoveto{\pgfqpoint{1.680550in}{2.066527in}}%
\pgfpathcurveto{\pgfqpoint{1.688786in}{2.066527in}}{\pgfqpoint{1.696686in}{2.069800in}}{\pgfqpoint{1.702510in}{2.075624in}}%
\pgfpathcurveto{\pgfqpoint{1.708334in}{2.081448in}}{\pgfqpoint{1.711606in}{2.089348in}}{\pgfqpoint{1.711606in}{2.097584in}}%
\pgfpathcurveto{\pgfqpoint{1.711606in}{2.105820in}}{\pgfqpoint{1.708334in}{2.113720in}}{\pgfqpoint{1.702510in}{2.119544in}}%
\pgfpathcurveto{\pgfqpoint{1.696686in}{2.125368in}}{\pgfqpoint{1.688786in}{2.128640in}}{\pgfqpoint{1.680550in}{2.128640in}}%
\pgfpathcurveto{\pgfqpoint{1.672314in}{2.128640in}}{\pgfqpoint{1.664414in}{2.125368in}}{\pgfqpoint{1.658590in}{2.119544in}}%
\pgfpathcurveto{\pgfqpoint{1.652766in}{2.113720in}}{\pgfqpoint{1.649493in}{2.105820in}}{\pgfqpoint{1.649493in}{2.097584in}}%
\pgfpathcurveto{\pgfqpoint{1.649493in}{2.089348in}}{\pgfqpoint{1.652766in}{2.081448in}}{\pgfqpoint{1.658590in}{2.075624in}}%
\pgfpathcurveto{\pgfqpoint{1.664414in}{2.069800in}}{\pgfqpoint{1.672314in}{2.066527in}}{\pgfqpoint{1.680550in}{2.066527in}}%
\pgfpathclose%
\pgfusepath{stroke,fill}%
\end{pgfscope}%
\begin{pgfscope}%
\pgfpathrectangle{\pgfqpoint{0.100000in}{0.212622in}}{\pgfqpoint{3.696000in}{3.696000in}}%
\pgfusepath{clip}%
\pgfsetbuttcap%
\pgfsetroundjoin%
\definecolor{currentfill}{rgb}{0.121569,0.466667,0.705882}%
\pgfsetfillcolor{currentfill}%
\pgfsetfillopacity{0.377353}%
\pgfsetlinewidth{1.003750pt}%
\definecolor{currentstroke}{rgb}{0.121569,0.466667,0.705882}%
\pgfsetstrokecolor{currentstroke}%
\pgfsetstrokeopacity{0.377353}%
\pgfsetdash{}{0pt}%
\pgfpathmoveto{\pgfqpoint{1.784882in}{2.135851in}}%
\pgfpathcurveto{\pgfqpoint{1.793118in}{2.135851in}}{\pgfqpoint{1.801018in}{2.139123in}}{\pgfqpoint{1.806842in}{2.144947in}}%
\pgfpathcurveto{\pgfqpoint{1.812666in}{2.150771in}}{\pgfqpoint{1.815938in}{2.158671in}}{\pgfqpoint{1.815938in}{2.166907in}}%
\pgfpathcurveto{\pgfqpoint{1.815938in}{2.175144in}}{\pgfqpoint{1.812666in}{2.183044in}}{\pgfqpoint{1.806842in}{2.188868in}}%
\pgfpathcurveto{\pgfqpoint{1.801018in}{2.194692in}}{\pgfqpoint{1.793118in}{2.197964in}}{\pgfqpoint{1.784882in}{2.197964in}}%
\pgfpathcurveto{\pgfqpoint{1.776645in}{2.197964in}}{\pgfqpoint{1.768745in}{2.194692in}}{\pgfqpoint{1.762921in}{2.188868in}}%
\pgfpathcurveto{\pgfqpoint{1.757097in}{2.183044in}}{\pgfqpoint{1.753825in}{2.175144in}}{\pgfqpoint{1.753825in}{2.166907in}}%
\pgfpathcurveto{\pgfqpoint{1.753825in}{2.158671in}}{\pgfqpoint{1.757097in}{2.150771in}}{\pgfqpoint{1.762921in}{2.144947in}}%
\pgfpathcurveto{\pgfqpoint{1.768745in}{2.139123in}}{\pgfqpoint{1.776645in}{2.135851in}}{\pgfqpoint{1.784882in}{2.135851in}}%
\pgfpathclose%
\pgfusepath{stroke,fill}%
\end{pgfscope}%
\begin{pgfscope}%
\pgfpathrectangle{\pgfqpoint{0.100000in}{0.212622in}}{\pgfqpoint{3.696000in}{3.696000in}}%
\pgfusepath{clip}%
\pgfsetbuttcap%
\pgfsetroundjoin%
\definecolor{currentfill}{rgb}{0.121569,0.466667,0.705882}%
\pgfsetfillcolor{currentfill}%
\pgfsetfillopacity{0.378385}%
\pgfsetlinewidth{1.003750pt}%
\definecolor{currentstroke}{rgb}{0.121569,0.466667,0.705882}%
\pgfsetstrokecolor{currentstroke}%
\pgfsetstrokeopacity{0.378385}%
\pgfsetdash{}{0pt}%
\pgfpathmoveto{\pgfqpoint{1.456448in}{1.918304in}}%
\pgfpathcurveto{\pgfqpoint{1.464685in}{1.918304in}}{\pgfqpoint{1.472585in}{1.921576in}}{\pgfqpoint{1.478409in}{1.927400in}}%
\pgfpathcurveto{\pgfqpoint{1.484232in}{1.933224in}}{\pgfqpoint{1.487505in}{1.941124in}}{\pgfqpoint{1.487505in}{1.949360in}}%
\pgfpathcurveto{\pgfqpoint{1.487505in}{1.957596in}}{\pgfqpoint{1.484232in}{1.965496in}}{\pgfqpoint{1.478409in}{1.971320in}}%
\pgfpathcurveto{\pgfqpoint{1.472585in}{1.977144in}}{\pgfqpoint{1.464685in}{1.980417in}}{\pgfqpoint{1.456448in}{1.980417in}}%
\pgfpathcurveto{\pgfqpoint{1.448212in}{1.980417in}}{\pgfqpoint{1.440312in}{1.977144in}}{\pgfqpoint{1.434488in}{1.971320in}}%
\pgfpathcurveto{\pgfqpoint{1.428664in}{1.965496in}}{\pgfqpoint{1.425392in}{1.957596in}}{\pgfqpoint{1.425392in}{1.949360in}}%
\pgfpathcurveto{\pgfqpoint{1.425392in}{1.941124in}}{\pgfqpoint{1.428664in}{1.933224in}}{\pgfqpoint{1.434488in}{1.927400in}}%
\pgfpathcurveto{\pgfqpoint{1.440312in}{1.921576in}}{\pgfqpoint{1.448212in}{1.918304in}}{\pgfqpoint{1.456448in}{1.918304in}}%
\pgfpathclose%
\pgfusepath{stroke,fill}%
\end{pgfscope}%
\begin{pgfscope}%
\pgfpathrectangle{\pgfqpoint{0.100000in}{0.212622in}}{\pgfqpoint{3.696000in}{3.696000in}}%
\pgfusepath{clip}%
\pgfsetbuttcap%
\pgfsetroundjoin%
\definecolor{currentfill}{rgb}{0.121569,0.466667,0.705882}%
\pgfsetfillcolor{currentfill}%
\pgfsetfillopacity{0.378852}%
\pgfsetlinewidth{1.003750pt}%
\definecolor{currentstroke}{rgb}{0.121569,0.466667,0.705882}%
\pgfsetstrokecolor{currentstroke}%
\pgfsetstrokeopacity{0.378852}%
\pgfsetdash{}{0pt}%
\pgfpathmoveto{\pgfqpoint{2.183951in}{2.356403in}}%
\pgfpathcurveto{\pgfqpoint{2.192188in}{2.356403in}}{\pgfqpoint{2.200088in}{2.359675in}}{\pgfqpoint{2.205912in}{2.365499in}}%
\pgfpathcurveto{\pgfqpoint{2.211736in}{2.371323in}}{\pgfqpoint{2.215008in}{2.379223in}}{\pgfqpoint{2.215008in}{2.387460in}}%
\pgfpathcurveto{\pgfqpoint{2.215008in}{2.395696in}}{\pgfqpoint{2.211736in}{2.403596in}}{\pgfqpoint{2.205912in}{2.409420in}}%
\pgfpathcurveto{\pgfqpoint{2.200088in}{2.415244in}}{\pgfqpoint{2.192188in}{2.418516in}}{\pgfqpoint{2.183951in}{2.418516in}}%
\pgfpathcurveto{\pgfqpoint{2.175715in}{2.418516in}}{\pgfqpoint{2.167815in}{2.415244in}}{\pgfqpoint{2.161991in}{2.409420in}}%
\pgfpathcurveto{\pgfqpoint{2.156167in}{2.403596in}}{\pgfqpoint{2.152895in}{2.395696in}}{\pgfqpoint{2.152895in}{2.387460in}}%
\pgfpathcurveto{\pgfqpoint{2.152895in}{2.379223in}}{\pgfqpoint{2.156167in}{2.371323in}}{\pgfqpoint{2.161991in}{2.365499in}}%
\pgfpathcurveto{\pgfqpoint{2.167815in}{2.359675in}}{\pgfqpoint{2.175715in}{2.356403in}}{\pgfqpoint{2.183951in}{2.356403in}}%
\pgfpathclose%
\pgfusepath{stroke,fill}%
\end{pgfscope}%
\begin{pgfscope}%
\pgfpathrectangle{\pgfqpoint{0.100000in}{0.212622in}}{\pgfqpoint{3.696000in}{3.696000in}}%
\pgfusepath{clip}%
\pgfsetbuttcap%
\pgfsetroundjoin%
\definecolor{currentfill}{rgb}{0.121569,0.466667,0.705882}%
\pgfsetfillcolor{currentfill}%
\pgfsetfillopacity{0.379201}%
\pgfsetlinewidth{1.003750pt}%
\definecolor{currentstroke}{rgb}{0.121569,0.466667,0.705882}%
\pgfsetstrokecolor{currentstroke}%
\pgfsetstrokeopacity{0.379201}%
\pgfsetdash{}{0pt}%
\pgfpathmoveto{\pgfqpoint{2.188171in}{2.361112in}}%
\pgfpathcurveto{\pgfqpoint{2.196407in}{2.361112in}}{\pgfqpoint{2.204307in}{2.364385in}}{\pgfqpoint{2.210131in}{2.370208in}}%
\pgfpathcurveto{\pgfqpoint{2.215955in}{2.376032in}}{\pgfqpoint{2.219227in}{2.383932in}}{\pgfqpoint{2.219227in}{2.392169in}}%
\pgfpathcurveto{\pgfqpoint{2.219227in}{2.400405in}}{\pgfqpoint{2.215955in}{2.408305in}}{\pgfqpoint{2.210131in}{2.414129in}}%
\pgfpathcurveto{\pgfqpoint{2.204307in}{2.419953in}}{\pgfqpoint{2.196407in}{2.423225in}}{\pgfqpoint{2.188171in}{2.423225in}}%
\pgfpathcurveto{\pgfqpoint{2.179934in}{2.423225in}}{\pgfqpoint{2.172034in}{2.419953in}}{\pgfqpoint{2.166210in}{2.414129in}}%
\pgfpathcurveto{\pgfqpoint{2.160386in}{2.408305in}}{\pgfqpoint{2.157114in}{2.400405in}}{\pgfqpoint{2.157114in}{2.392169in}}%
\pgfpathcurveto{\pgfqpoint{2.157114in}{2.383932in}}{\pgfqpoint{2.160386in}{2.376032in}}{\pgfqpoint{2.166210in}{2.370208in}}%
\pgfpathcurveto{\pgfqpoint{2.172034in}{2.364385in}}{\pgfqpoint{2.179934in}{2.361112in}}{\pgfqpoint{2.188171in}{2.361112in}}%
\pgfpathclose%
\pgfusepath{stroke,fill}%
\end{pgfscope}%
\begin{pgfscope}%
\pgfpathrectangle{\pgfqpoint{0.100000in}{0.212622in}}{\pgfqpoint{3.696000in}{3.696000in}}%
\pgfusepath{clip}%
\pgfsetbuttcap%
\pgfsetroundjoin%
\definecolor{currentfill}{rgb}{0.121569,0.466667,0.705882}%
\pgfsetfillcolor{currentfill}%
\pgfsetfillopacity{0.379292}%
\pgfsetlinewidth{1.003750pt}%
\definecolor{currentstroke}{rgb}{0.121569,0.466667,0.705882}%
\pgfsetstrokecolor{currentstroke}%
\pgfsetstrokeopacity{0.379292}%
\pgfsetdash{}{0pt}%
\pgfpathmoveto{\pgfqpoint{1.457216in}{1.919393in}}%
\pgfpathcurveto{\pgfqpoint{1.465452in}{1.919393in}}{\pgfqpoint{1.473352in}{1.922665in}}{\pgfqpoint{1.479176in}{1.928489in}}%
\pgfpathcurveto{\pgfqpoint{1.485000in}{1.934313in}}{\pgfqpoint{1.488273in}{1.942213in}}{\pgfqpoint{1.488273in}{1.950449in}}%
\pgfpathcurveto{\pgfqpoint{1.488273in}{1.958686in}}{\pgfqpoint{1.485000in}{1.966586in}}{\pgfqpoint{1.479176in}{1.972410in}}%
\pgfpathcurveto{\pgfqpoint{1.473352in}{1.978233in}}{\pgfqpoint{1.465452in}{1.981506in}}{\pgfqpoint{1.457216in}{1.981506in}}%
\pgfpathcurveto{\pgfqpoint{1.448980in}{1.981506in}}{\pgfqpoint{1.441080in}{1.978233in}}{\pgfqpoint{1.435256in}{1.972410in}}%
\pgfpathcurveto{\pgfqpoint{1.429432in}{1.966586in}}{\pgfqpoint{1.426160in}{1.958686in}}{\pgfqpoint{1.426160in}{1.950449in}}%
\pgfpathcurveto{\pgfqpoint{1.426160in}{1.942213in}}{\pgfqpoint{1.429432in}{1.934313in}}{\pgfqpoint{1.435256in}{1.928489in}}%
\pgfpathcurveto{\pgfqpoint{1.441080in}{1.922665in}}{\pgfqpoint{1.448980in}{1.919393in}}{\pgfqpoint{1.457216in}{1.919393in}}%
\pgfpathclose%
\pgfusepath{stroke,fill}%
\end{pgfscope}%
\begin{pgfscope}%
\pgfpathrectangle{\pgfqpoint{0.100000in}{0.212622in}}{\pgfqpoint{3.696000in}{3.696000in}}%
\pgfusepath{clip}%
\pgfsetbuttcap%
\pgfsetroundjoin%
\definecolor{currentfill}{rgb}{0.121569,0.466667,0.705882}%
\pgfsetfillcolor{currentfill}%
\pgfsetfillopacity{0.380912}%
\pgfsetlinewidth{1.003750pt}%
\definecolor{currentstroke}{rgb}{0.121569,0.466667,0.705882}%
\pgfsetstrokecolor{currentstroke}%
\pgfsetstrokeopacity{0.380912}%
\pgfsetdash{}{0pt}%
\pgfpathmoveto{\pgfqpoint{2.138333in}{2.326775in}}%
\pgfpathcurveto{\pgfqpoint{2.146570in}{2.326775in}}{\pgfqpoint{2.154470in}{2.330047in}}{\pgfqpoint{2.160294in}{2.335871in}}%
\pgfpathcurveto{\pgfqpoint{2.166117in}{2.341695in}}{\pgfqpoint{2.169390in}{2.349595in}}{\pgfqpoint{2.169390in}{2.357831in}}%
\pgfpathcurveto{\pgfqpoint{2.169390in}{2.366068in}}{\pgfqpoint{2.166117in}{2.373968in}}{\pgfqpoint{2.160294in}{2.379792in}}%
\pgfpathcurveto{\pgfqpoint{2.154470in}{2.385616in}}{\pgfqpoint{2.146570in}{2.388888in}}{\pgfqpoint{2.138333in}{2.388888in}}%
\pgfpathcurveto{\pgfqpoint{2.130097in}{2.388888in}}{\pgfqpoint{2.122197in}{2.385616in}}{\pgfqpoint{2.116373in}{2.379792in}}%
\pgfpathcurveto{\pgfqpoint{2.110549in}{2.373968in}}{\pgfqpoint{2.107277in}{2.366068in}}{\pgfqpoint{2.107277in}{2.357831in}}%
\pgfpathcurveto{\pgfqpoint{2.107277in}{2.349595in}}{\pgfqpoint{2.110549in}{2.341695in}}{\pgfqpoint{2.116373in}{2.335871in}}%
\pgfpathcurveto{\pgfqpoint{2.122197in}{2.330047in}}{\pgfqpoint{2.130097in}{2.326775in}}{\pgfqpoint{2.138333in}{2.326775in}}%
\pgfpathclose%
\pgfusepath{stroke,fill}%
\end{pgfscope}%
\begin{pgfscope}%
\pgfpathrectangle{\pgfqpoint{0.100000in}{0.212622in}}{\pgfqpoint{3.696000in}{3.696000in}}%
\pgfusepath{clip}%
\pgfsetbuttcap%
\pgfsetroundjoin%
\definecolor{currentfill}{rgb}{0.121569,0.466667,0.705882}%
\pgfsetfillcolor{currentfill}%
\pgfsetfillopacity{0.381313}%
\pgfsetlinewidth{1.003750pt}%
\definecolor{currentstroke}{rgb}{0.121569,0.466667,0.705882}%
\pgfsetstrokecolor{currentstroke}%
\pgfsetstrokeopacity{0.381313}%
\pgfsetdash{}{0pt}%
\pgfpathmoveto{\pgfqpoint{2.160620in}{2.339466in}}%
\pgfpathcurveto{\pgfqpoint{2.168856in}{2.339466in}}{\pgfqpoint{2.176757in}{2.342739in}}{\pgfqpoint{2.182580in}{2.348562in}}%
\pgfpathcurveto{\pgfqpoint{2.188404in}{2.354386in}}{\pgfqpoint{2.191677in}{2.362286in}}{\pgfqpoint{2.191677in}{2.370523in}}%
\pgfpathcurveto{\pgfqpoint{2.191677in}{2.378759in}}{\pgfqpoint{2.188404in}{2.386659in}}{\pgfqpoint{2.182580in}{2.392483in}}%
\pgfpathcurveto{\pgfqpoint{2.176757in}{2.398307in}}{\pgfqpoint{2.168856in}{2.401579in}}{\pgfqpoint{2.160620in}{2.401579in}}%
\pgfpathcurveto{\pgfqpoint{2.152384in}{2.401579in}}{\pgfqpoint{2.144484in}{2.398307in}}{\pgfqpoint{2.138660in}{2.392483in}}%
\pgfpathcurveto{\pgfqpoint{2.132836in}{2.386659in}}{\pgfqpoint{2.129564in}{2.378759in}}{\pgfqpoint{2.129564in}{2.370523in}}%
\pgfpathcurveto{\pgfqpoint{2.129564in}{2.362286in}}{\pgfqpoint{2.132836in}{2.354386in}}{\pgfqpoint{2.138660in}{2.348562in}}%
\pgfpathcurveto{\pgfqpoint{2.144484in}{2.342739in}}{\pgfqpoint{2.152384in}{2.339466in}}{\pgfqpoint{2.160620in}{2.339466in}}%
\pgfpathclose%
\pgfusepath{stroke,fill}%
\end{pgfscope}%
\begin{pgfscope}%
\pgfpathrectangle{\pgfqpoint{0.100000in}{0.212622in}}{\pgfqpoint{3.696000in}{3.696000in}}%
\pgfusepath{clip}%
\pgfsetbuttcap%
\pgfsetroundjoin%
\definecolor{currentfill}{rgb}{0.121569,0.466667,0.705882}%
\pgfsetfillcolor{currentfill}%
\pgfsetfillopacity{0.381493}%
\pgfsetlinewidth{1.003750pt}%
\definecolor{currentstroke}{rgb}{0.121569,0.466667,0.705882}%
\pgfsetstrokecolor{currentstroke}%
\pgfsetstrokeopacity{0.381493}%
\pgfsetdash{}{0pt}%
\pgfpathmoveto{\pgfqpoint{1.932440in}{2.216334in}}%
\pgfpathcurveto{\pgfqpoint{1.940676in}{2.216334in}}{\pgfqpoint{1.948576in}{2.219606in}}{\pgfqpoint{1.954400in}{2.225430in}}%
\pgfpathcurveto{\pgfqpoint{1.960224in}{2.231254in}}{\pgfqpoint{1.963497in}{2.239154in}}{\pgfqpoint{1.963497in}{2.247391in}}%
\pgfpathcurveto{\pgfqpoint{1.963497in}{2.255627in}}{\pgfqpoint{1.960224in}{2.263527in}}{\pgfqpoint{1.954400in}{2.269351in}}%
\pgfpathcurveto{\pgfqpoint{1.948576in}{2.275175in}}{\pgfqpoint{1.940676in}{2.278447in}}{\pgfqpoint{1.932440in}{2.278447in}}%
\pgfpathcurveto{\pgfqpoint{1.924204in}{2.278447in}}{\pgfqpoint{1.916304in}{2.275175in}}{\pgfqpoint{1.910480in}{2.269351in}}%
\pgfpathcurveto{\pgfqpoint{1.904656in}{2.263527in}}{\pgfqpoint{1.901384in}{2.255627in}}{\pgfqpoint{1.901384in}{2.247391in}}%
\pgfpathcurveto{\pgfqpoint{1.901384in}{2.239154in}}{\pgfqpoint{1.904656in}{2.231254in}}{\pgfqpoint{1.910480in}{2.225430in}}%
\pgfpathcurveto{\pgfqpoint{1.916304in}{2.219606in}}{\pgfqpoint{1.924204in}{2.216334in}}{\pgfqpoint{1.932440in}{2.216334in}}%
\pgfpathclose%
\pgfusepath{stroke,fill}%
\end{pgfscope}%
\begin{pgfscope}%
\pgfpathrectangle{\pgfqpoint{0.100000in}{0.212622in}}{\pgfqpoint{3.696000in}{3.696000in}}%
\pgfusepath{clip}%
\pgfsetbuttcap%
\pgfsetroundjoin%
\definecolor{currentfill}{rgb}{0.121569,0.466667,0.705882}%
\pgfsetfillcolor{currentfill}%
\pgfsetfillopacity{0.381692}%
\pgfsetlinewidth{1.003750pt}%
\definecolor{currentstroke}{rgb}{0.121569,0.466667,0.705882}%
\pgfsetstrokecolor{currentstroke}%
\pgfsetstrokeopacity{0.381692}%
\pgfsetdash{}{0pt}%
\pgfpathmoveto{\pgfqpoint{1.462614in}{1.925674in}}%
\pgfpathcurveto{\pgfqpoint{1.470850in}{1.925674in}}{\pgfqpoint{1.478750in}{1.928946in}}{\pgfqpoint{1.484574in}{1.934770in}}%
\pgfpathcurveto{\pgfqpoint{1.490398in}{1.940594in}}{\pgfqpoint{1.493670in}{1.948494in}}{\pgfqpoint{1.493670in}{1.956731in}}%
\pgfpathcurveto{\pgfqpoint{1.493670in}{1.964967in}}{\pgfqpoint{1.490398in}{1.972867in}}{\pgfqpoint{1.484574in}{1.978691in}}%
\pgfpathcurveto{\pgfqpoint{1.478750in}{1.984515in}}{\pgfqpoint{1.470850in}{1.987787in}}{\pgfqpoint{1.462614in}{1.987787in}}%
\pgfpathcurveto{\pgfqpoint{1.454378in}{1.987787in}}{\pgfqpoint{1.446478in}{1.984515in}}{\pgfqpoint{1.440654in}{1.978691in}}%
\pgfpathcurveto{\pgfqpoint{1.434830in}{1.972867in}}{\pgfqpoint{1.431557in}{1.964967in}}{\pgfqpoint{1.431557in}{1.956731in}}%
\pgfpathcurveto{\pgfqpoint{1.431557in}{1.948494in}}{\pgfqpoint{1.434830in}{1.940594in}}{\pgfqpoint{1.440654in}{1.934770in}}%
\pgfpathcurveto{\pgfqpoint{1.446478in}{1.928946in}}{\pgfqpoint{1.454378in}{1.925674in}}{\pgfqpoint{1.462614in}{1.925674in}}%
\pgfpathclose%
\pgfusepath{stroke,fill}%
\end{pgfscope}%
\begin{pgfscope}%
\pgfpathrectangle{\pgfqpoint{0.100000in}{0.212622in}}{\pgfqpoint{3.696000in}{3.696000in}}%
\pgfusepath{clip}%
\pgfsetbuttcap%
\pgfsetroundjoin%
\definecolor{currentfill}{rgb}{0.121569,0.466667,0.705882}%
\pgfsetfillcolor{currentfill}%
\pgfsetfillopacity{0.381831}%
\pgfsetlinewidth{1.003750pt}%
\definecolor{currentstroke}{rgb}{0.121569,0.466667,0.705882}%
\pgfsetstrokecolor{currentstroke}%
\pgfsetstrokeopacity{0.381831}%
\pgfsetdash{}{0pt}%
\pgfpathmoveto{\pgfqpoint{2.165315in}{2.338637in}}%
\pgfpathcurveto{\pgfqpoint{2.173551in}{2.338637in}}{\pgfqpoint{2.181451in}{2.341910in}}{\pgfqpoint{2.187275in}{2.347733in}}%
\pgfpathcurveto{\pgfqpoint{2.193099in}{2.353557in}}{\pgfqpoint{2.196371in}{2.361457in}}{\pgfqpoint{2.196371in}{2.369694in}}%
\pgfpathcurveto{\pgfqpoint{2.196371in}{2.377930in}}{\pgfqpoint{2.193099in}{2.385830in}}{\pgfqpoint{2.187275in}{2.391654in}}%
\pgfpathcurveto{\pgfqpoint{2.181451in}{2.397478in}}{\pgfqpoint{2.173551in}{2.400750in}}{\pgfqpoint{2.165315in}{2.400750in}}%
\pgfpathcurveto{\pgfqpoint{2.157079in}{2.400750in}}{\pgfqpoint{2.149179in}{2.397478in}}{\pgfqpoint{2.143355in}{2.391654in}}%
\pgfpathcurveto{\pgfqpoint{2.137531in}{2.385830in}}{\pgfqpoint{2.134258in}{2.377930in}}{\pgfqpoint{2.134258in}{2.369694in}}%
\pgfpathcurveto{\pgfqpoint{2.134258in}{2.361457in}}{\pgfqpoint{2.137531in}{2.353557in}}{\pgfqpoint{2.143355in}{2.347733in}}%
\pgfpathcurveto{\pgfqpoint{2.149179in}{2.341910in}}{\pgfqpoint{2.157079in}{2.338637in}}{\pgfqpoint{2.165315in}{2.338637in}}%
\pgfpathclose%
\pgfusepath{stroke,fill}%
\end{pgfscope}%
\begin{pgfscope}%
\pgfpathrectangle{\pgfqpoint{0.100000in}{0.212622in}}{\pgfqpoint{3.696000in}{3.696000in}}%
\pgfusepath{clip}%
\pgfsetbuttcap%
\pgfsetroundjoin%
\definecolor{currentfill}{rgb}{0.121569,0.466667,0.705882}%
\pgfsetfillcolor{currentfill}%
\pgfsetfillopacity{0.382101}%
\pgfsetlinewidth{1.003750pt}%
\definecolor{currentstroke}{rgb}{0.121569,0.466667,0.705882}%
\pgfsetstrokecolor{currentstroke}%
\pgfsetstrokeopacity{0.382101}%
\pgfsetdash{}{0pt}%
\pgfpathmoveto{\pgfqpoint{1.933833in}{2.214284in}}%
\pgfpathcurveto{\pgfqpoint{1.942069in}{2.214284in}}{\pgfqpoint{1.949969in}{2.217556in}}{\pgfqpoint{1.955793in}{2.223380in}}%
\pgfpathcurveto{\pgfqpoint{1.961617in}{2.229204in}}{\pgfqpoint{1.964890in}{2.237104in}}{\pgfqpoint{1.964890in}{2.245340in}}%
\pgfpathcurveto{\pgfqpoint{1.964890in}{2.253576in}}{\pgfqpoint{1.961617in}{2.261477in}}{\pgfqpoint{1.955793in}{2.267300in}}%
\pgfpathcurveto{\pgfqpoint{1.949969in}{2.273124in}}{\pgfqpoint{1.942069in}{2.276397in}}{\pgfqpoint{1.933833in}{2.276397in}}%
\pgfpathcurveto{\pgfqpoint{1.925597in}{2.276397in}}{\pgfqpoint{1.917697in}{2.273124in}}{\pgfqpoint{1.911873in}{2.267300in}}%
\pgfpathcurveto{\pgfqpoint{1.906049in}{2.261477in}}{\pgfqpoint{1.902777in}{2.253576in}}{\pgfqpoint{1.902777in}{2.245340in}}%
\pgfpathcurveto{\pgfqpoint{1.902777in}{2.237104in}}{\pgfqpoint{1.906049in}{2.229204in}}{\pgfqpoint{1.911873in}{2.223380in}}%
\pgfpathcurveto{\pgfqpoint{1.917697in}{2.217556in}}{\pgfqpoint{1.925597in}{2.214284in}}{\pgfqpoint{1.933833in}{2.214284in}}%
\pgfpathclose%
\pgfusepath{stroke,fill}%
\end{pgfscope}%
\begin{pgfscope}%
\pgfpathrectangle{\pgfqpoint{0.100000in}{0.212622in}}{\pgfqpoint{3.696000in}{3.696000in}}%
\pgfusepath{clip}%
\pgfsetbuttcap%
\pgfsetroundjoin%
\definecolor{currentfill}{rgb}{0.121569,0.466667,0.705882}%
\pgfsetfillcolor{currentfill}%
\pgfsetfillopacity{0.382282}%
\pgfsetlinewidth{1.003750pt}%
\definecolor{currentstroke}{rgb}{0.121569,0.466667,0.705882}%
\pgfsetstrokecolor{currentstroke}%
\pgfsetstrokeopacity{0.382282}%
\pgfsetdash{}{0pt}%
\pgfpathmoveto{\pgfqpoint{1.451978in}{1.912206in}}%
\pgfpathcurveto{\pgfqpoint{1.460214in}{1.912206in}}{\pgfqpoint{1.468114in}{1.915479in}}{\pgfqpoint{1.473938in}{1.921303in}}%
\pgfpathcurveto{\pgfqpoint{1.479762in}{1.927127in}}{\pgfqpoint{1.483034in}{1.935027in}}{\pgfqpoint{1.483034in}{1.943263in}}%
\pgfpathcurveto{\pgfqpoint{1.483034in}{1.951499in}}{\pgfqpoint{1.479762in}{1.959399in}}{\pgfqpoint{1.473938in}{1.965223in}}%
\pgfpathcurveto{\pgfqpoint{1.468114in}{1.971047in}}{\pgfqpoint{1.460214in}{1.974319in}}{\pgfqpoint{1.451978in}{1.974319in}}%
\pgfpathcurveto{\pgfqpoint{1.443741in}{1.974319in}}{\pgfqpoint{1.435841in}{1.971047in}}{\pgfqpoint{1.430017in}{1.965223in}}%
\pgfpathcurveto{\pgfqpoint{1.424193in}{1.959399in}}{\pgfqpoint{1.420921in}{1.951499in}}{\pgfqpoint{1.420921in}{1.943263in}}%
\pgfpathcurveto{\pgfqpoint{1.420921in}{1.935027in}}{\pgfqpoint{1.424193in}{1.927127in}}{\pgfqpoint{1.430017in}{1.921303in}}%
\pgfpathcurveto{\pgfqpoint{1.435841in}{1.915479in}}{\pgfqpoint{1.443741in}{1.912206in}}{\pgfqpoint{1.451978in}{1.912206in}}%
\pgfpathclose%
\pgfusepath{stroke,fill}%
\end{pgfscope}%
\begin{pgfscope}%
\pgfpathrectangle{\pgfqpoint{0.100000in}{0.212622in}}{\pgfqpoint{3.696000in}{3.696000in}}%
\pgfusepath{clip}%
\pgfsetbuttcap%
\pgfsetroundjoin%
\definecolor{currentfill}{rgb}{0.121569,0.466667,0.705882}%
\pgfsetfillcolor{currentfill}%
\pgfsetfillopacity{0.383169}%
\pgfsetlinewidth{1.003750pt}%
\definecolor{currentstroke}{rgb}{0.121569,0.466667,0.705882}%
\pgfsetstrokecolor{currentstroke}%
\pgfsetstrokeopacity{0.383169}%
\pgfsetdash{}{0pt}%
\pgfpathmoveto{\pgfqpoint{1.935713in}{2.216329in}}%
\pgfpathcurveto{\pgfqpoint{1.943950in}{2.216329in}}{\pgfqpoint{1.951850in}{2.219601in}}{\pgfqpoint{1.957673in}{2.225425in}}%
\pgfpathcurveto{\pgfqpoint{1.963497in}{2.231249in}}{\pgfqpoint{1.966770in}{2.239149in}}{\pgfqpoint{1.966770in}{2.247385in}}%
\pgfpathcurveto{\pgfqpoint{1.966770in}{2.255622in}}{\pgfqpoint{1.963497in}{2.263522in}}{\pgfqpoint{1.957673in}{2.269346in}}%
\pgfpathcurveto{\pgfqpoint{1.951850in}{2.275170in}}{\pgfqpoint{1.943950in}{2.278442in}}{\pgfqpoint{1.935713in}{2.278442in}}%
\pgfpathcurveto{\pgfqpoint{1.927477in}{2.278442in}}{\pgfqpoint{1.919577in}{2.275170in}}{\pgfqpoint{1.913753in}{2.269346in}}%
\pgfpathcurveto{\pgfqpoint{1.907929in}{2.263522in}}{\pgfqpoint{1.904657in}{2.255622in}}{\pgfqpoint{1.904657in}{2.247385in}}%
\pgfpathcurveto{\pgfqpoint{1.904657in}{2.239149in}}{\pgfqpoint{1.907929in}{2.231249in}}{\pgfqpoint{1.913753in}{2.225425in}}%
\pgfpathcurveto{\pgfqpoint{1.919577in}{2.219601in}}{\pgfqpoint{1.927477in}{2.216329in}}{\pgfqpoint{1.935713in}{2.216329in}}%
\pgfpathclose%
\pgfusepath{stroke,fill}%
\end{pgfscope}%
\begin{pgfscope}%
\pgfpathrectangle{\pgfqpoint{0.100000in}{0.212622in}}{\pgfqpoint{3.696000in}{3.696000in}}%
\pgfusepath{clip}%
\pgfsetbuttcap%
\pgfsetroundjoin%
\definecolor{currentfill}{rgb}{0.121569,0.466667,0.705882}%
\pgfsetfillcolor{currentfill}%
\pgfsetfillopacity{0.384253}%
\pgfsetlinewidth{1.003750pt}%
\definecolor{currentstroke}{rgb}{0.121569,0.466667,0.705882}%
\pgfsetstrokecolor{currentstroke}%
\pgfsetstrokeopacity{0.384253}%
\pgfsetdash{}{0pt}%
\pgfpathmoveto{\pgfqpoint{2.134719in}{2.320581in}}%
\pgfpathcurveto{\pgfqpoint{2.142955in}{2.320581in}}{\pgfqpoint{2.150855in}{2.323854in}}{\pgfqpoint{2.156679in}{2.329678in}}%
\pgfpathcurveto{\pgfqpoint{2.162503in}{2.335501in}}{\pgfqpoint{2.165775in}{2.343402in}}{\pgfqpoint{2.165775in}{2.351638in}}%
\pgfpathcurveto{\pgfqpoint{2.165775in}{2.359874in}}{\pgfqpoint{2.162503in}{2.367774in}}{\pgfqpoint{2.156679in}{2.373598in}}%
\pgfpathcurveto{\pgfqpoint{2.150855in}{2.379422in}}{\pgfqpoint{2.142955in}{2.382694in}}{\pgfqpoint{2.134719in}{2.382694in}}%
\pgfpathcurveto{\pgfqpoint{2.126482in}{2.382694in}}{\pgfqpoint{2.118582in}{2.379422in}}{\pgfqpoint{2.112758in}{2.373598in}}%
\pgfpathcurveto{\pgfqpoint{2.106934in}{2.367774in}}{\pgfqpoint{2.103662in}{2.359874in}}{\pgfqpoint{2.103662in}{2.351638in}}%
\pgfpathcurveto{\pgfqpoint{2.103662in}{2.343402in}}{\pgfqpoint{2.106934in}{2.335501in}}{\pgfqpoint{2.112758in}{2.329678in}}%
\pgfpathcurveto{\pgfqpoint{2.118582in}{2.323854in}}{\pgfqpoint{2.126482in}{2.320581in}}{\pgfqpoint{2.134719in}{2.320581in}}%
\pgfpathclose%
\pgfusepath{stroke,fill}%
\end{pgfscope}%
\begin{pgfscope}%
\pgfpathrectangle{\pgfqpoint{0.100000in}{0.212622in}}{\pgfqpoint{3.696000in}{3.696000in}}%
\pgfusepath{clip}%
\pgfsetbuttcap%
\pgfsetroundjoin%
\definecolor{currentfill}{rgb}{0.121569,0.466667,0.705882}%
\pgfsetfillcolor{currentfill}%
\pgfsetfillopacity{0.384859}%
\pgfsetlinewidth{1.003750pt}%
\definecolor{currentstroke}{rgb}{0.121569,0.466667,0.705882}%
\pgfsetstrokecolor{currentstroke}%
\pgfsetstrokeopacity{0.384859}%
\pgfsetdash{}{0pt}%
\pgfpathmoveto{\pgfqpoint{2.143751in}{2.325086in}}%
\pgfpathcurveto{\pgfqpoint{2.151987in}{2.325086in}}{\pgfqpoint{2.159887in}{2.328358in}}{\pgfqpoint{2.165711in}{2.334182in}}%
\pgfpathcurveto{\pgfqpoint{2.171535in}{2.340006in}}{\pgfqpoint{2.174807in}{2.347906in}}{\pgfqpoint{2.174807in}{2.356142in}}%
\pgfpathcurveto{\pgfqpoint{2.174807in}{2.364378in}}{\pgfqpoint{2.171535in}{2.372278in}}{\pgfqpoint{2.165711in}{2.378102in}}%
\pgfpathcurveto{\pgfqpoint{2.159887in}{2.383926in}}{\pgfqpoint{2.151987in}{2.387199in}}{\pgfqpoint{2.143751in}{2.387199in}}%
\pgfpathcurveto{\pgfqpoint{2.135514in}{2.387199in}}{\pgfqpoint{2.127614in}{2.383926in}}{\pgfqpoint{2.121790in}{2.378102in}}%
\pgfpathcurveto{\pgfqpoint{2.115966in}{2.372278in}}{\pgfqpoint{2.112694in}{2.364378in}}{\pgfqpoint{2.112694in}{2.356142in}}%
\pgfpathcurveto{\pgfqpoint{2.112694in}{2.347906in}}{\pgfqpoint{2.115966in}{2.340006in}}{\pgfqpoint{2.121790in}{2.334182in}}%
\pgfpathcurveto{\pgfqpoint{2.127614in}{2.328358in}}{\pgfqpoint{2.135514in}{2.325086in}}{\pgfqpoint{2.143751in}{2.325086in}}%
\pgfpathclose%
\pgfusepath{stroke,fill}%
\end{pgfscope}%
\begin{pgfscope}%
\pgfpathrectangle{\pgfqpoint{0.100000in}{0.212622in}}{\pgfqpoint{3.696000in}{3.696000in}}%
\pgfusepath{clip}%
\pgfsetbuttcap%
\pgfsetroundjoin%
\definecolor{currentfill}{rgb}{0.121569,0.466667,0.705882}%
\pgfsetfillcolor{currentfill}%
\pgfsetfillopacity{0.385068}%
\pgfsetlinewidth{1.003750pt}%
\definecolor{currentstroke}{rgb}{0.121569,0.466667,0.705882}%
\pgfsetstrokecolor{currentstroke}%
\pgfsetstrokeopacity{0.385068}%
\pgfsetdash{}{0pt}%
\pgfpathmoveto{\pgfqpoint{2.140429in}{2.327733in}}%
\pgfpathcurveto{\pgfqpoint{2.148666in}{2.327733in}}{\pgfqpoint{2.156566in}{2.331005in}}{\pgfqpoint{2.162390in}{2.336829in}}%
\pgfpathcurveto{\pgfqpoint{2.168214in}{2.342653in}}{\pgfqpoint{2.171486in}{2.350553in}}{\pgfqpoint{2.171486in}{2.358790in}}%
\pgfpathcurveto{\pgfqpoint{2.171486in}{2.367026in}}{\pgfqpoint{2.168214in}{2.374926in}}{\pgfqpoint{2.162390in}{2.380750in}}%
\pgfpathcurveto{\pgfqpoint{2.156566in}{2.386574in}}{\pgfqpoint{2.148666in}{2.389846in}}{\pgfqpoint{2.140429in}{2.389846in}}%
\pgfpathcurveto{\pgfqpoint{2.132193in}{2.389846in}}{\pgfqpoint{2.124293in}{2.386574in}}{\pgfqpoint{2.118469in}{2.380750in}}%
\pgfpathcurveto{\pgfqpoint{2.112645in}{2.374926in}}{\pgfqpoint{2.109373in}{2.367026in}}{\pgfqpoint{2.109373in}{2.358790in}}%
\pgfpathcurveto{\pgfqpoint{2.109373in}{2.350553in}}{\pgfqpoint{2.112645in}{2.342653in}}{\pgfqpoint{2.118469in}{2.336829in}}%
\pgfpathcurveto{\pgfqpoint{2.124293in}{2.331005in}}{\pgfqpoint{2.132193in}{2.327733in}}{\pgfqpoint{2.140429in}{2.327733in}}%
\pgfpathclose%
\pgfusepath{stroke,fill}%
\end{pgfscope}%
\begin{pgfscope}%
\pgfpathrectangle{\pgfqpoint{0.100000in}{0.212622in}}{\pgfqpoint{3.696000in}{3.696000in}}%
\pgfusepath{clip}%
\pgfsetbuttcap%
\pgfsetroundjoin%
\definecolor{currentfill}{rgb}{0.121569,0.466667,0.705882}%
\pgfsetfillcolor{currentfill}%
\pgfsetfillopacity{0.385428}%
\pgfsetlinewidth{1.003750pt}%
\definecolor{currentstroke}{rgb}{0.121569,0.466667,0.705882}%
\pgfsetstrokecolor{currentstroke}%
\pgfsetstrokeopacity{0.385428}%
\pgfsetdash{}{0pt}%
\pgfpathmoveto{\pgfqpoint{2.176620in}{2.348271in}}%
\pgfpathcurveto{\pgfqpoint{2.184857in}{2.348271in}}{\pgfqpoint{2.192757in}{2.351544in}}{\pgfqpoint{2.198581in}{2.357367in}}%
\pgfpathcurveto{\pgfqpoint{2.204404in}{2.363191in}}{\pgfqpoint{2.207677in}{2.371091in}}{\pgfqpoint{2.207677in}{2.379328in}}%
\pgfpathcurveto{\pgfqpoint{2.207677in}{2.387564in}}{\pgfqpoint{2.204404in}{2.395464in}}{\pgfqpoint{2.198581in}{2.401288in}}%
\pgfpathcurveto{\pgfqpoint{2.192757in}{2.407112in}}{\pgfqpoint{2.184857in}{2.410384in}}{\pgfqpoint{2.176620in}{2.410384in}}%
\pgfpathcurveto{\pgfqpoint{2.168384in}{2.410384in}}{\pgfqpoint{2.160484in}{2.407112in}}{\pgfqpoint{2.154660in}{2.401288in}}%
\pgfpathcurveto{\pgfqpoint{2.148836in}{2.395464in}}{\pgfqpoint{2.145564in}{2.387564in}}{\pgfqpoint{2.145564in}{2.379328in}}%
\pgfpathcurveto{\pgfqpoint{2.145564in}{2.371091in}}{\pgfqpoint{2.148836in}{2.363191in}}{\pgfqpoint{2.154660in}{2.357367in}}%
\pgfpathcurveto{\pgfqpoint{2.160484in}{2.351544in}}{\pgfqpoint{2.168384in}{2.348271in}}{\pgfqpoint{2.176620in}{2.348271in}}%
\pgfpathclose%
\pgfusepath{stroke,fill}%
\end{pgfscope}%
\begin{pgfscope}%
\pgfpathrectangle{\pgfqpoint{0.100000in}{0.212622in}}{\pgfqpoint{3.696000in}{3.696000in}}%
\pgfusepath{clip}%
\pgfsetbuttcap%
\pgfsetroundjoin%
\definecolor{currentfill}{rgb}{0.121569,0.466667,0.705882}%
\pgfsetfillcolor{currentfill}%
\pgfsetfillopacity{0.385529}%
\pgfsetlinewidth{1.003750pt}%
\definecolor{currentstroke}{rgb}{0.121569,0.466667,0.705882}%
\pgfsetstrokecolor{currentstroke}%
\pgfsetstrokeopacity{0.385529}%
\pgfsetdash{}{0pt}%
\pgfpathmoveto{\pgfqpoint{1.933462in}{2.212786in}}%
\pgfpathcurveto{\pgfqpoint{1.941698in}{2.212786in}}{\pgfqpoint{1.949598in}{2.216058in}}{\pgfqpoint{1.955422in}{2.221882in}}%
\pgfpathcurveto{\pgfqpoint{1.961246in}{2.227706in}}{\pgfqpoint{1.964518in}{2.235606in}}{\pgfqpoint{1.964518in}{2.243842in}}%
\pgfpathcurveto{\pgfqpoint{1.964518in}{2.252078in}}{\pgfqpoint{1.961246in}{2.259978in}}{\pgfqpoint{1.955422in}{2.265802in}}%
\pgfpathcurveto{\pgfqpoint{1.949598in}{2.271626in}}{\pgfqpoint{1.941698in}{2.274899in}}{\pgfqpoint{1.933462in}{2.274899in}}%
\pgfpathcurveto{\pgfqpoint{1.925226in}{2.274899in}}{\pgfqpoint{1.917325in}{2.271626in}}{\pgfqpoint{1.911502in}{2.265802in}}%
\pgfpathcurveto{\pgfqpoint{1.905678in}{2.259978in}}{\pgfqpoint{1.902405in}{2.252078in}}{\pgfqpoint{1.902405in}{2.243842in}}%
\pgfpathcurveto{\pgfqpoint{1.902405in}{2.235606in}}{\pgfqpoint{1.905678in}{2.227706in}}{\pgfqpoint{1.911502in}{2.221882in}}%
\pgfpathcurveto{\pgfqpoint{1.917325in}{2.216058in}}{\pgfqpoint{1.925226in}{2.212786in}}{\pgfqpoint{1.933462in}{2.212786in}}%
\pgfpathclose%
\pgfusepath{stroke,fill}%
\end{pgfscope}%
\begin{pgfscope}%
\pgfpathrectangle{\pgfqpoint{0.100000in}{0.212622in}}{\pgfqpoint{3.696000in}{3.696000in}}%
\pgfusepath{clip}%
\pgfsetbuttcap%
\pgfsetroundjoin%
\definecolor{currentfill}{rgb}{0.121569,0.466667,0.705882}%
\pgfsetfillcolor{currentfill}%
\pgfsetfillopacity{0.385936}%
\pgfsetlinewidth{1.003750pt}%
\definecolor{currentstroke}{rgb}{0.121569,0.466667,0.705882}%
\pgfsetstrokecolor{currentstroke}%
\pgfsetstrokeopacity{0.385936}%
\pgfsetdash{}{0pt}%
\pgfpathmoveto{\pgfqpoint{1.956072in}{2.227759in}}%
\pgfpathcurveto{\pgfqpoint{1.964308in}{2.227759in}}{\pgfqpoint{1.972208in}{2.231031in}}{\pgfqpoint{1.978032in}{2.236855in}}%
\pgfpathcurveto{\pgfqpoint{1.983856in}{2.242679in}}{\pgfqpoint{1.987128in}{2.250579in}}{\pgfqpoint{1.987128in}{2.258815in}}%
\pgfpathcurveto{\pgfqpoint{1.987128in}{2.267052in}}{\pgfqpoint{1.983856in}{2.274952in}}{\pgfqpoint{1.978032in}{2.280776in}}%
\pgfpathcurveto{\pgfqpoint{1.972208in}{2.286600in}}{\pgfqpoint{1.964308in}{2.289872in}}{\pgfqpoint{1.956072in}{2.289872in}}%
\pgfpathcurveto{\pgfqpoint{1.947836in}{2.289872in}}{\pgfqpoint{1.939936in}{2.286600in}}{\pgfqpoint{1.934112in}{2.280776in}}%
\pgfpathcurveto{\pgfqpoint{1.928288in}{2.274952in}}{\pgfqpoint{1.925015in}{2.267052in}}{\pgfqpoint{1.925015in}{2.258815in}}%
\pgfpathcurveto{\pgfqpoint{1.925015in}{2.250579in}}{\pgfqpoint{1.928288in}{2.242679in}}{\pgfqpoint{1.934112in}{2.236855in}}%
\pgfpathcurveto{\pgfqpoint{1.939936in}{2.231031in}}{\pgfqpoint{1.947836in}{2.227759in}}{\pgfqpoint{1.956072in}{2.227759in}}%
\pgfpathclose%
\pgfusepath{stroke,fill}%
\end{pgfscope}%
\begin{pgfscope}%
\pgfpathrectangle{\pgfqpoint{0.100000in}{0.212622in}}{\pgfqpoint{3.696000in}{3.696000in}}%
\pgfusepath{clip}%
\pgfsetbuttcap%
\pgfsetroundjoin%
\definecolor{currentfill}{rgb}{0.121569,0.466667,0.705882}%
\pgfsetfillcolor{currentfill}%
\pgfsetfillopacity{0.385945}%
\pgfsetlinewidth{1.003750pt}%
\definecolor{currentstroke}{rgb}{0.121569,0.466667,0.705882}%
\pgfsetstrokecolor{currentstroke}%
\pgfsetstrokeopacity{0.385945}%
\pgfsetdash{}{0pt}%
\pgfpathmoveto{\pgfqpoint{2.176610in}{2.347932in}}%
\pgfpathcurveto{\pgfqpoint{2.184847in}{2.347932in}}{\pgfqpoint{2.192747in}{2.351204in}}{\pgfqpoint{2.198571in}{2.357028in}}%
\pgfpathcurveto{\pgfqpoint{2.204395in}{2.362852in}}{\pgfqpoint{2.207667in}{2.370752in}}{\pgfqpoint{2.207667in}{2.378988in}}%
\pgfpathcurveto{\pgfqpoint{2.207667in}{2.387225in}}{\pgfqpoint{2.204395in}{2.395125in}}{\pgfqpoint{2.198571in}{2.400949in}}%
\pgfpathcurveto{\pgfqpoint{2.192747in}{2.406772in}}{\pgfqpoint{2.184847in}{2.410045in}}{\pgfqpoint{2.176610in}{2.410045in}}%
\pgfpathcurveto{\pgfqpoint{2.168374in}{2.410045in}}{\pgfqpoint{2.160474in}{2.406772in}}{\pgfqpoint{2.154650in}{2.400949in}}%
\pgfpathcurveto{\pgfqpoint{2.148826in}{2.395125in}}{\pgfqpoint{2.145554in}{2.387225in}}{\pgfqpoint{2.145554in}{2.378988in}}%
\pgfpathcurveto{\pgfqpoint{2.145554in}{2.370752in}}{\pgfqpoint{2.148826in}{2.362852in}}{\pgfqpoint{2.154650in}{2.357028in}}%
\pgfpathcurveto{\pgfqpoint{2.160474in}{2.351204in}}{\pgfqpoint{2.168374in}{2.347932in}}{\pgfqpoint{2.176610in}{2.347932in}}%
\pgfpathclose%
\pgfusepath{stroke,fill}%
\end{pgfscope}%
\begin{pgfscope}%
\pgfpathrectangle{\pgfqpoint{0.100000in}{0.212622in}}{\pgfqpoint{3.696000in}{3.696000in}}%
\pgfusepath{clip}%
\pgfsetbuttcap%
\pgfsetroundjoin%
\definecolor{currentfill}{rgb}{0.121569,0.466667,0.705882}%
\pgfsetfillcolor{currentfill}%
\pgfsetfillopacity{0.386072}%
\pgfsetlinewidth{1.003750pt}%
\definecolor{currentstroke}{rgb}{0.121569,0.466667,0.705882}%
\pgfsetstrokecolor{currentstroke}%
\pgfsetstrokeopacity{0.386072}%
\pgfsetdash{}{0pt}%
\pgfpathmoveto{\pgfqpoint{2.171476in}{2.343626in}}%
\pgfpathcurveto{\pgfqpoint{2.179712in}{2.343626in}}{\pgfqpoint{2.187612in}{2.346899in}}{\pgfqpoint{2.193436in}{2.352723in}}%
\pgfpathcurveto{\pgfqpoint{2.199260in}{2.358547in}}{\pgfqpoint{2.202532in}{2.366447in}}{\pgfqpoint{2.202532in}{2.374683in}}%
\pgfpathcurveto{\pgfqpoint{2.202532in}{2.382919in}}{\pgfqpoint{2.199260in}{2.390819in}}{\pgfqpoint{2.193436in}{2.396643in}}%
\pgfpathcurveto{\pgfqpoint{2.187612in}{2.402467in}}{\pgfqpoint{2.179712in}{2.405739in}}{\pgfqpoint{2.171476in}{2.405739in}}%
\pgfpathcurveto{\pgfqpoint{2.163240in}{2.405739in}}{\pgfqpoint{2.155339in}{2.402467in}}{\pgfqpoint{2.149516in}{2.396643in}}%
\pgfpathcurveto{\pgfqpoint{2.143692in}{2.390819in}}{\pgfqpoint{2.140419in}{2.382919in}}{\pgfqpoint{2.140419in}{2.374683in}}%
\pgfpathcurveto{\pgfqpoint{2.140419in}{2.366447in}}{\pgfqpoint{2.143692in}{2.358547in}}{\pgfqpoint{2.149516in}{2.352723in}}%
\pgfpathcurveto{\pgfqpoint{2.155339in}{2.346899in}}{\pgfqpoint{2.163240in}{2.343626in}}{\pgfqpoint{2.171476in}{2.343626in}}%
\pgfpathclose%
\pgfusepath{stroke,fill}%
\end{pgfscope}%
\begin{pgfscope}%
\pgfpathrectangle{\pgfqpoint{0.100000in}{0.212622in}}{\pgfqpoint{3.696000in}{3.696000in}}%
\pgfusepath{clip}%
\pgfsetbuttcap%
\pgfsetroundjoin%
\definecolor{currentfill}{rgb}{0.121569,0.466667,0.705882}%
\pgfsetfillcolor{currentfill}%
\pgfsetfillopacity{0.386728}%
\pgfsetlinewidth{1.003750pt}%
\definecolor{currentstroke}{rgb}{0.121569,0.466667,0.705882}%
\pgfsetstrokecolor{currentstroke}%
\pgfsetstrokeopacity{0.386728}%
\pgfsetdash{}{0pt}%
\pgfpathmoveto{\pgfqpoint{2.176331in}{2.347236in}}%
\pgfpathcurveto{\pgfqpoint{2.184567in}{2.347236in}}{\pgfqpoint{2.192467in}{2.350508in}}{\pgfqpoint{2.198291in}{2.356332in}}%
\pgfpathcurveto{\pgfqpoint{2.204115in}{2.362156in}}{\pgfqpoint{2.207387in}{2.370056in}}{\pgfqpoint{2.207387in}{2.378292in}}%
\pgfpathcurveto{\pgfqpoint{2.207387in}{2.386529in}}{\pgfqpoint{2.204115in}{2.394429in}}{\pgfqpoint{2.198291in}{2.400253in}}%
\pgfpathcurveto{\pgfqpoint{2.192467in}{2.406077in}}{\pgfqpoint{2.184567in}{2.409349in}}{\pgfqpoint{2.176331in}{2.409349in}}%
\pgfpathcurveto{\pgfqpoint{2.168094in}{2.409349in}}{\pgfqpoint{2.160194in}{2.406077in}}{\pgfqpoint{2.154371in}{2.400253in}}%
\pgfpathcurveto{\pgfqpoint{2.148547in}{2.394429in}}{\pgfqpoint{2.145274in}{2.386529in}}{\pgfqpoint{2.145274in}{2.378292in}}%
\pgfpathcurveto{\pgfqpoint{2.145274in}{2.370056in}}{\pgfqpoint{2.148547in}{2.362156in}}{\pgfqpoint{2.154371in}{2.356332in}}%
\pgfpathcurveto{\pgfqpoint{2.160194in}{2.350508in}}{\pgfqpoint{2.168094in}{2.347236in}}{\pgfqpoint{2.176331in}{2.347236in}}%
\pgfpathclose%
\pgfusepath{stroke,fill}%
\end{pgfscope}%
\begin{pgfscope}%
\pgfpathrectangle{\pgfqpoint{0.100000in}{0.212622in}}{\pgfqpoint{3.696000in}{3.696000in}}%
\pgfusepath{clip}%
\pgfsetbuttcap%
\pgfsetroundjoin%
\definecolor{currentfill}{rgb}{0.121569,0.466667,0.705882}%
\pgfsetfillcolor{currentfill}%
\pgfsetfillopacity{0.386865}%
\pgfsetlinewidth{1.003750pt}%
\definecolor{currentstroke}{rgb}{0.121569,0.466667,0.705882}%
\pgfsetstrokecolor{currentstroke}%
\pgfsetstrokeopacity{0.386865}%
\pgfsetdash{}{0pt}%
\pgfpathmoveto{\pgfqpoint{2.177128in}{2.347807in}}%
\pgfpathcurveto{\pgfqpoint{2.185364in}{2.347807in}}{\pgfqpoint{2.193264in}{2.351080in}}{\pgfqpoint{2.199088in}{2.356903in}}%
\pgfpathcurveto{\pgfqpoint{2.204912in}{2.362727in}}{\pgfqpoint{2.208185in}{2.370627in}}{\pgfqpoint{2.208185in}{2.378864in}}%
\pgfpathcurveto{\pgfqpoint{2.208185in}{2.387100in}}{\pgfqpoint{2.204912in}{2.395000in}}{\pgfqpoint{2.199088in}{2.400824in}}%
\pgfpathcurveto{\pgfqpoint{2.193264in}{2.406648in}}{\pgfqpoint{2.185364in}{2.409920in}}{\pgfqpoint{2.177128in}{2.409920in}}%
\pgfpathcurveto{\pgfqpoint{2.168892in}{2.409920in}}{\pgfqpoint{2.160992in}{2.406648in}}{\pgfqpoint{2.155168in}{2.400824in}}%
\pgfpathcurveto{\pgfqpoint{2.149344in}{2.395000in}}{\pgfqpoint{2.146072in}{2.387100in}}{\pgfqpoint{2.146072in}{2.378864in}}%
\pgfpathcurveto{\pgfqpoint{2.146072in}{2.370627in}}{\pgfqpoint{2.149344in}{2.362727in}}{\pgfqpoint{2.155168in}{2.356903in}}%
\pgfpathcurveto{\pgfqpoint{2.160992in}{2.351080in}}{\pgfqpoint{2.168892in}{2.347807in}}{\pgfqpoint{2.177128in}{2.347807in}}%
\pgfpathclose%
\pgfusepath{stroke,fill}%
\end{pgfscope}%
\begin{pgfscope}%
\pgfpathrectangle{\pgfqpoint{0.100000in}{0.212622in}}{\pgfqpoint{3.696000in}{3.696000in}}%
\pgfusepath{clip}%
\pgfsetbuttcap%
\pgfsetroundjoin%
\definecolor{currentfill}{rgb}{0.121569,0.466667,0.705882}%
\pgfsetfillcolor{currentfill}%
\pgfsetfillopacity{0.386915}%
\pgfsetlinewidth{1.003750pt}%
\definecolor{currentstroke}{rgb}{0.121569,0.466667,0.705882}%
\pgfsetstrokecolor{currentstroke}%
\pgfsetstrokeopacity{0.386915}%
\pgfsetdash{}{0pt}%
\pgfpathmoveto{\pgfqpoint{1.619969in}{2.025072in}}%
\pgfpathcurveto{\pgfqpoint{1.628205in}{2.025072in}}{\pgfqpoint{1.636105in}{2.028344in}}{\pgfqpoint{1.641929in}{2.034168in}}%
\pgfpathcurveto{\pgfqpoint{1.647753in}{2.039992in}}{\pgfqpoint{1.651025in}{2.047892in}}{\pgfqpoint{1.651025in}{2.056129in}}%
\pgfpathcurveto{\pgfqpoint{1.651025in}{2.064365in}}{\pgfqpoint{1.647753in}{2.072265in}}{\pgfqpoint{1.641929in}{2.078089in}}%
\pgfpathcurveto{\pgfqpoint{1.636105in}{2.083913in}}{\pgfqpoint{1.628205in}{2.087185in}}{\pgfqpoint{1.619969in}{2.087185in}}%
\pgfpathcurveto{\pgfqpoint{1.611733in}{2.087185in}}{\pgfqpoint{1.603833in}{2.083913in}}{\pgfqpoint{1.598009in}{2.078089in}}%
\pgfpathcurveto{\pgfqpoint{1.592185in}{2.072265in}}{\pgfqpoint{1.588912in}{2.064365in}}{\pgfqpoint{1.588912in}{2.056129in}}%
\pgfpathcurveto{\pgfqpoint{1.588912in}{2.047892in}}{\pgfqpoint{1.592185in}{2.039992in}}{\pgfqpoint{1.598009in}{2.034168in}}%
\pgfpathcurveto{\pgfqpoint{1.603833in}{2.028344in}}{\pgfqpoint{1.611733in}{2.025072in}}{\pgfqpoint{1.619969in}{2.025072in}}%
\pgfpathclose%
\pgfusepath{stroke,fill}%
\end{pgfscope}%
\begin{pgfscope}%
\pgfpathrectangle{\pgfqpoint{0.100000in}{0.212622in}}{\pgfqpoint{3.696000in}{3.696000in}}%
\pgfusepath{clip}%
\pgfsetbuttcap%
\pgfsetroundjoin%
\definecolor{currentfill}{rgb}{0.121569,0.466667,0.705882}%
\pgfsetfillcolor{currentfill}%
\pgfsetfillopacity{0.386993}%
\pgfsetlinewidth{1.003750pt}%
\definecolor{currentstroke}{rgb}{0.121569,0.466667,0.705882}%
\pgfsetstrokecolor{currentstroke}%
\pgfsetstrokeopacity{0.386993}%
\pgfsetdash{}{0pt}%
\pgfpathmoveto{\pgfqpoint{2.122254in}{2.306559in}}%
\pgfpathcurveto{\pgfqpoint{2.130490in}{2.306559in}}{\pgfqpoint{2.138390in}{2.309831in}}{\pgfqpoint{2.144214in}{2.315655in}}%
\pgfpathcurveto{\pgfqpoint{2.150038in}{2.321479in}}{\pgfqpoint{2.153310in}{2.329379in}}{\pgfqpoint{2.153310in}{2.337616in}}%
\pgfpathcurveto{\pgfqpoint{2.153310in}{2.345852in}}{\pgfqpoint{2.150038in}{2.353752in}}{\pgfqpoint{2.144214in}{2.359576in}}%
\pgfpathcurveto{\pgfqpoint{2.138390in}{2.365400in}}{\pgfqpoint{2.130490in}{2.368672in}}{\pgfqpoint{2.122254in}{2.368672in}}%
\pgfpathcurveto{\pgfqpoint{2.114018in}{2.368672in}}{\pgfqpoint{2.106118in}{2.365400in}}{\pgfqpoint{2.100294in}{2.359576in}}%
\pgfpathcurveto{\pgfqpoint{2.094470in}{2.353752in}}{\pgfqpoint{2.091197in}{2.345852in}}{\pgfqpoint{2.091197in}{2.337616in}}%
\pgfpathcurveto{\pgfqpoint{2.091197in}{2.329379in}}{\pgfqpoint{2.094470in}{2.321479in}}{\pgfqpoint{2.100294in}{2.315655in}}%
\pgfpathcurveto{\pgfqpoint{2.106118in}{2.309831in}}{\pgfqpoint{2.114018in}{2.306559in}}{\pgfqpoint{2.122254in}{2.306559in}}%
\pgfpathclose%
\pgfusepath{stroke,fill}%
\end{pgfscope}%
\begin{pgfscope}%
\pgfpathrectangle{\pgfqpoint{0.100000in}{0.212622in}}{\pgfqpoint{3.696000in}{3.696000in}}%
\pgfusepath{clip}%
\pgfsetbuttcap%
\pgfsetroundjoin%
\definecolor{currentfill}{rgb}{0.121569,0.466667,0.705882}%
\pgfsetfillcolor{currentfill}%
\pgfsetfillopacity{0.387940}%
\pgfsetlinewidth{1.003750pt}%
\definecolor{currentstroke}{rgb}{0.121569,0.466667,0.705882}%
\pgfsetstrokecolor{currentstroke}%
\pgfsetstrokeopacity{0.387940}%
\pgfsetdash{}{0pt}%
\pgfpathmoveto{\pgfqpoint{2.175587in}{2.345597in}}%
\pgfpathcurveto{\pgfqpoint{2.183824in}{2.345597in}}{\pgfqpoint{2.191724in}{2.348869in}}{\pgfqpoint{2.197547in}{2.354693in}}%
\pgfpathcurveto{\pgfqpoint{2.203371in}{2.360517in}}{\pgfqpoint{2.206644in}{2.368417in}}{\pgfqpoint{2.206644in}{2.376654in}}%
\pgfpathcurveto{\pgfqpoint{2.206644in}{2.384890in}}{\pgfqpoint{2.203371in}{2.392790in}}{\pgfqpoint{2.197547in}{2.398614in}}%
\pgfpathcurveto{\pgfqpoint{2.191724in}{2.404438in}}{\pgfqpoint{2.183824in}{2.407710in}}{\pgfqpoint{2.175587in}{2.407710in}}%
\pgfpathcurveto{\pgfqpoint{2.167351in}{2.407710in}}{\pgfqpoint{2.159451in}{2.404438in}}{\pgfqpoint{2.153627in}{2.398614in}}%
\pgfpathcurveto{\pgfqpoint{2.147803in}{2.392790in}}{\pgfqpoint{2.144531in}{2.384890in}}{\pgfqpoint{2.144531in}{2.376654in}}%
\pgfpathcurveto{\pgfqpoint{2.144531in}{2.368417in}}{\pgfqpoint{2.147803in}{2.360517in}}{\pgfqpoint{2.153627in}{2.354693in}}%
\pgfpathcurveto{\pgfqpoint{2.159451in}{2.348869in}}{\pgfqpoint{2.167351in}{2.345597in}}{\pgfqpoint{2.175587in}{2.345597in}}%
\pgfpathclose%
\pgfusepath{stroke,fill}%
\end{pgfscope}%
\begin{pgfscope}%
\pgfpathrectangle{\pgfqpoint{0.100000in}{0.212622in}}{\pgfqpoint{3.696000in}{3.696000in}}%
\pgfusepath{clip}%
\pgfsetbuttcap%
\pgfsetroundjoin%
\definecolor{currentfill}{rgb}{0.121569,0.466667,0.705882}%
\pgfsetfillcolor{currentfill}%
\pgfsetfillopacity{0.388407}%
\pgfsetlinewidth{1.003750pt}%
\definecolor{currentstroke}{rgb}{0.121569,0.466667,0.705882}%
\pgfsetstrokecolor{currentstroke}%
\pgfsetstrokeopacity{0.388407}%
\pgfsetdash{}{0pt}%
\pgfpathmoveto{\pgfqpoint{1.447795in}{1.911418in}}%
\pgfpathcurveto{\pgfqpoint{1.456031in}{1.911418in}}{\pgfqpoint{1.463931in}{1.914690in}}{\pgfqpoint{1.469755in}{1.920514in}}%
\pgfpathcurveto{\pgfqpoint{1.475579in}{1.926338in}}{\pgfqpoint{1.478851in}{1.934238in}}{\pgfqpoint{1.478851in}{1.942474in}}%
\pgfpathcurveto{\pgfqpoint{1.478851in}{1.950710in}}{\pgfqpoint{1.475579in}{1.958610in}}{\pgfqpoint{1.469755in}{1.964434in}}%
\pgfpathcurveto{\pgfqpoint{1.463931in}{1.970258in}}{\pgfqpoint{1.456031in}{1.973531in}}{\pgfqpoint{1.447795in}{1.973531in}}%
\pgfpathcurveto{\pgfqpoint{1.439559in}{1.973531in}}{\pgfqpoint{1.431659in}{1.970258in}}{\pgfqpoint{1.425835in}{1.964434in}}%
\pgfpathcurveto{\pgfqpoint{1.420011in}{1.958610in}}{\pgfqpoint{1.416738in}{1.950710in}}{\pgfqpoint{1.416738in}{1.942474in}}%
\pgfpathcurveto{\pgfqpoint{1.416738in}{1.934238in}}{\pgfqpoint{1.420011in}{1.926338in}}{\pgfqpoint{1.425835in}{1.920514in}}%
\pgfpathcurveto{\pgfqpoint{1.431659in}{1.914690in}}{\pgfqpoint{1.439559in}{1.911418in}}{\pgfqpoint{1.447795in}{1.911418in}}%
\pgfpathclose%
\pgfusepath{stroke,fill}%
\end{pgfscope}%
\begin{pgfscope}%
\pgfpathrectangle{\pgfqpoint{0.100000in}{0.212622in}}{\pgfqpoint{3.696000in}{3.696000in}}%
\pgfusepath{clip}%
\pgfsetbuttcap%
\pgfsetroundjoin%
\definecolor{currentfill}{rgb}{0.121569,0.466667,0.705882}%
\pgfsetfillcolor{currentfill}%
\pgfsetfillopacity{0.388789}%
\pgfsetlinewidth{1.003750pt}%
\definecolor{currentstroke}{rgb}{0.121569,0.466667,0.705882}%
\pgfsetstrokecolor{currentstroke}%
\pgfsetstrokeopacity{0.388789}%
\pgfsetdash{}{0pt}%
\pgfpathmoveto{\pgfqpoint{2.123260in}{2.308760in}}%
\pgfpathcurveto{\pgfqpoint{2.131496in}{2.308760in}}{\pgfqpoint{2.139396in}{2.312033in}}{\pgfqpoint{2.145220in}{2.317856in}}%
\pgfpathcurveto{\pgfqpoint{2.151044in}{2.323680in}}{\pgfqpoint{2.154316in}{2.331580in}}{\pgfqpoint{2.154316in}{2.339817in}}%
\pgfpathcurveto{\pgfqpoint{2.154316in}{2.348053in}}{\pgfqpoint{2.151044in}{2.355953in}}{\pgfqpoint{2.145220in}{2.361777in}}%
\pgfpathcurveto{\pgfqpoint{2.139396in}{2.367601in}}{\pgfqpoint{2.131496in}{2.370873in}}{\pgfqpoint{2.123260in}{2.370873in}}%
\pgfpathcurveto{\pgfqpoint{2.115023in}{2.370873in}}{\pgfqpoint{2.107123in}{2.367601in}}{\pgfqpoint{2.101299in}{2.361777in}}%
\pgfpathcurveto{\pgfqpoint{2.095475in}{2.355953in}}{\pgfqpoint{2.092203in}{2.348053in}}{\pgfqpoint{2.092203in}{2.339817in}}%
\pgfpathcurveto{\pgfqpoint{2.092203in}{2.331580in}}{\pgfqpoint{2.095475in}{2.323680in}}{\pgfqpoint{2.101299in}{2.317856in}}%
\pgfpathcurveto{\pgfqpoint{2.107123in}{2.312033in}}{\pgfqpoint{2.115023in}{2.308760in}}{\pgfqpoint{2.123260in}{2.308760in}}%
\pgfpathclose%
\pgfusepath{stroke,fill}%
\end{pgfscope}%
\begin{pgfscope}%
\pgfpathrectangle{\pgfqpoint{0.100000in}{0.212622in}}{\pgfqpoint{3.696000in}{3.696000in}}%
\pgfusepath{clip}%
\pgfsetbuttcap%
\pgfsetroundjoin%
\definecolor{currentfill}{rgb}{0.121569,0.466667,0.705882}%
\pgfsetfillcolor{currentfill}%
\pgfsetfillopacity{0.388988}%
\pgfsetlinewidth{1.003750pt}%
\definecolor{currentstroke}{rgb}{0.121569,0.466667,0.705882}%
\pgfsetstrokecolor{currentstroke}%
\pgfsetstrokeopacity{0.388988}%
\pgfsetdash{}{0pt}%
\pgfpathmoveto{\pgfqpoint{2.175889in}{2.345791in}}%
\pgfpathcurveto{\pgfqpoint{2.184126in}{2.345791in}}{\pgfqpoint{2.192026in}{2.349063in}}{\pgfqpoint{2.197850in}{2.354887in}}%
\pgfpathcurveto{\pgfqpoint{2.203673in}{2.360711in}}{\pgfqpoint{2.206946in}{2.368611in}}{\pgfqpoint{2.206946in}{2.376847in}}%
\pgfpathcurveto{\pgfqpoint{2.206946in}{2.385083in}}{\pgfqpoint{2.203673in}{2.392983in}}{\pgfqpoint{2.197850in}{2.398807in}}%
\pgfpathcurveto{\pgfqpoint{2.192026in}{2.404631in}}{\pgfqpoint{2.184126in}{2.407904in}}{\pgfqpoint{2.175889in}{2.407904in}}%
\pgfpathcurveto{\pgfqpoint{2.167653in}{2.407904in}}{\pgfqpoint{2.159753in}{2.404631in}}{\pgfqpoint{2.153929in}{2.398807in}}%
\pgfpathcurveto{\pgfqpoint{2.148105in}{2.392983in}}{\pgfqpoint{2.144833in}{2.385083in}}{\pgfqpoint{2.144833in}{2.376847in}}%
\pgfpathcurveto{\pgfqpoint{2.144833in}{2.368611in}}{\pgfqpoint{2.148105in}{2.360711in}}{\pgfqpoint{2.153929in}{2.354887in}}%
\pgfpathcurveto{\pgfqpoint{2.159753in}{2.349063in}}{\pgfqpoint{2.167653in}{2.345791in}}{\pgfqpoint{2.175889in}{2.345791in}}%
\pgfpathclose%
\pgfusepath{stroke,fill}%
\end{pgfscope}%
\begin{pgfscope}%
\pgfpathrectangle{\pgfqpoint{0.100000in}{0.212622in}}{\pgfqpoint{3.696000in}{3.696000in}}%
\pgfusepath{clip}%
\pgfsetbuttcap%
\pgfsetroundjoin%
\definecolor{currentfill}{rgb}{0.121569,0.466667,0.705882}%
\pgfsetfillcolor{currentfill}%
\pgfsetfillopacity{0.389410}%
\pgfsetlinewidth{1.003750pt}%
\definecolor{currentstroke}{rgb}{0.121569,0.466667,0.705882}%
\pgfsetstrokecolor{currentstroke}%
\pgfsetstrokeopacity{0.389410}%
\pgfsetdash{}{0pt}%
\pgfpathmoveto{\pgfqpoint{1.933603in}{2.206108in}}%
\pgfpathcurveto{\pgfqpoint{1.941839in}{2.206108in}}{\pgfqpoint{1.949739in}{2.209380in}}{\pgfqpoint{1.955563in}{2.215204in}}%
\pgfpathcurveto{\pgfqpoint{1.961387in}{2.221028in}}{\pgfqpoint{1.964660in}{2.228928in}}{\pgfqpoint{1.964660in}{2.237164in}}%
\pgfpathcurveto{\pgfqpoint{1.964660in}{2.245401in}}{\pgfqpoint{1.961387in}{2.253301in}}{\pgfqpoint{1.955563in}{2.259124in}}%
\pgfpathcurveto{\pgfqpoint{1.949739in}{2.264948in}}{\pgfqpoint{1.941839in}{2.268221in}}{\pgfqpoint{1.933603in}{2.268221in}}%
\pgfpathcurveto{\pgfqpoint{1.925367in}{2.268221in}}{\pgfqpoint{1.917467in}{2.264948in}}{\pgfqpoint{1.911643in}{2.259124in}}%
\pgfpathcurveto{\pgfqpoint{1.905819in}{2.253301in}}{\pgfqpoint{1.902547in}{2.245401in}}{\pgfqpoint{1.902547in}{2.237164in}}%
\pgfpathcurveto{\pgfqpoint{1.902547in}{2.228928in}}{\pgfqpoint{1.905819in}{2.221028in}}{\pgfqpoint{1.911643in}{2.215204in}}%
\pgfpathcurveto{\pgfqpoint{1.917467in}{2.209380in}}{\pgfqpoint{1.925367in}{2.206108in}}{\pgfqpoint{1.933603in}{2.206108in}}%
\pgfpathclose%
\pgfusepath{stroke,fill}%
\end{pgfscope}%
\begin{pgfscope}%
\pgfpathrectangle{\pgfqpoint{0.100000in}{0.212622in}}{\pgfqpoint{3.696000in}{3.696000in}}%
\pgfusepath{clip}%
\pgfsetbuttcap%
\pgfsetroundjoin%
\definecolor{currentfill}{rgb}{0.121569,0.466667,0.705882}%
\pgfsetfillcolor{currentfill}%
\pgfsetfillopacity{0.389615}%
\pgfsetlinewidth{1.003750pt}%
\definecolor{currentstroke}{rgb}{0.121569,0.466667,0.705882}%
\pgfsetstrokecolor{currentstroke}%
\pgfsetstrokeopacity{0.389615}%
\pgfsetdash{}{0pt}%
\pgfpathmoveto{\pgfqpoint{1.926536in}{2.204832in}}%
\pgfpathcurveto{\pgfqpoint{1.934772in}{2.204832in}}{\pgfqpoint{1.942672in}{2.208105in}}{\pgfqpoint{1.948496in}{2.213929in}}%
\pgfpathcurveto{\pgfqpoint{1.954320in}{2.219753in}}{\pgfqpoint{1.957592in}{2.227653in}}{\pgfqpoint{1.957592in}{2.235889in}}%
\pgfpathcurveto{\pgfqpoint{1.957592in}{2.244125in}}{\pgfqpoint{1.954320in}{2.252025in}}{\pgfqpoint{1.948496in}{2.257849in}}%
\pgfpathcurveto{\pgfqpoint{1.942672in}{2.263673in}}{\pgfqpoint{1.934772in}{2.266945in}}{\pgfqpoint{1.926536in}{2.266945in}}%
\pgfpathcurveto{\pgfqpoint{1.918300in}{2.266945in}}{\pgfqpoint{1.910400in}{2.263673in}}{\pgfqpoint{1.904576in}{2.257849in}}%
\pgfpathcurveto{\pgfqpoint{1.898752in}{2.252025in}}{\pgfqpoint{1.895479in}{2.244125in}}{\pgfqpoint{1.895479in}{2.235889in}}%
\pgfpathcurveto{\pgfqpoint{1.895479in}{2.227653in}}{\pgfqpoint{1.898752in}{2.219753in}}{\pgfqpoint{1.904576in}{2.213929in}}%
\pgfpathcurveto{\pgfqpoint{1.910400in}{2.208105in}}{\pgfqpoint{1.918300in}{2.204832in}}{\pgfqpoint{1.926536in}{2.204832in}}%
\pgfpathclose%
\pgfusepath{stroke,fill}%
\end{pgfscope}%
\begin{pgfscope}%
\pgfpathrectangle{\pgfqpoint{0.100000in}{0.212622in}}{\pgfqpoint{3.696000in}{3.696000in}}%
\pgfusepath{clip}%
\pgfsetbuttcap%
\pgfsetroundjoin%
\definecolor{currentfill}{rgb}{0.121569,0.466667,0.705882}%
\pgfsetfillcolor{currentfill}%
\pgfsetfillopacity{0.390466}%
\pgfsetlinewidth{1.003750pt}%
\definecolor{currentstroke}{rgb}{0.121569,0.466667,0.705882}%
\pgfsetstrokecolor{currentstroke}%
\pgfsetstrokeopacity{0.390466}%
\pgfsetdash{}{0pt}%
\pgfpathmoveto{\pgfqpoint{1.566531in}{2.006866in}}%
\pgfpathcurveto{\pgfqpoint{1.574767in}{2.006866in}}{\pgfqpoint{1.582667in}{2.010139in}}{\pgfqpoint{1.588491in}{2.015962in}}%
\pgfpathcurveto{\pgfqpoint{1.594315in}{2.021786in}}{\pgfqpoint{1.597588in}{2.029686in}}{\pgfqpoint{1.597588in}{2.037923in}}%
\pgfpathcurveto{\pgfqpoint{1.597588in}{2.046159in}}{\pgfqpoint{1.594315in}{2.054059in}}{\pgfqpoint{1.588491in}{2.059883in}}%
\pgfpathcurveto{\pgfqpoint{1.582667in}{2.065707in}}{\pgfqpoint{1.574767in}{2.068979in}}{\pgfqpoint{1.566531in}{2.068979in}}%
\pgfpathcurveto{\pgfqpoint{1.558295in}{2.068979in}}{\pgfqpoint{1.550395in}{2.065707in}}{\pgfqpoint{1.544571in}{2.059883in}}%
\pgfpathcurveto{\pgfqpoint{1.538747in}{2.054059in}}{\pgfqpoint{1.535475in}{2.046159in}}{\pgfqpoint{1.535475in}{2.037923in}}%
\pgfpathcurveto{\pgfqpoint{1.535475in}{2.029686in}}{\pgfqpoint{1.538747in}{2.021786in}}{\pgfqpoint{1.544571in}{2.015962in}}%
\pgfpathcurveto{\pgfqpoint{1.550395in}{2.010139in}}{\pgfqpoint{1.558295in}{2.006866in}}{\pgfqpoint{1.566531in}{2.006866in}}%
\pgfpathclose%
\pgfusepath{stroke,fill}%
\end{pgfscope}%
\begin{pgfscope}%
\pgfpathrectangle{\pgfqpoint{0.100000in}{0.212622in}}{\pgfqpoint{3.696000in}{3.696000in}}%
\pgfusepath{clip}%
\pgfsetbuttcap%
\pgfsetroundjoin%
\definecolor{currentfill}{rgb}{0.121569,0.466667,0.705882}%
\pgfsetfillcolor{currentfill}%
\pgfsetfillopacity{0.392192}%
\pgfsetlinewidth{1.003750pt}%
\definecolor{currentstroke}{rgb}{0.121569,0.466667,0.705882}%
\pgfsetstrokecolor{currentstroke}%
\pgfsetstrokeopacity{0.392192}%
\pgfsetdash{}{0pt}%
\pgfpathmoveto{\pgfqpoint{1.442006in}{1.902360in}}%
\pgfpathcurveto{\pgfqpoint{1.450242in}{1.902360in}}{\pgfqpoint{1.458142in}{1.905632in}}{\pgfqpoint{1.463966in}{1.911456in}}%
\pgfpathcurveto{\pgfqpoint{1.469790in}{1.917280in}}{\pgfqpoint{1.473062in}{1.925180in}}{\pgfqpoint{1.473062in}{1.933416in}}%
\pgfpathcurveto{\pgfqpoint{1.473062in}{1.941652in}}{\pgfqpoint{1.469790in}{1.949552in}}{\pgfqpoint{1.463966in}{1.955376in}}%
\pgfpathcurveto{\pgfqpoint{1.458142in}{1.961200in}}{\pgfqpoint{1.450242in}{1.964473in}}{\pgfqpoint{1.442006in}{1.964473in}}%
\pgfpathcurveto{\pgfqpoint{1.433769in}{1.964473in}}{\pgfqpoint{1.425869in}{1.961200in}}{\pgfqpoint{1.420045in}{1.955376in}}%
\pgfpathcurveto{\pgfqpoint{1.414221in}{1.949552in}}{\pgfqpoint{1.410949in}{1.941652in}}{\pgfqpoint{1.410949in}{1.933416in}}%
\pgfpathcurveto{\pgfqpoint{1.410949in}{1.925180in}}{\pgfqpoint{1.414221in}{1.917280in}}{\pgfqpoint{1.420045in}{1.911456in}}%
\pgfpathcurveto{\pgfqpoint{1.425869in}{1.905632in}}{\pgfqpoint{1.433769in}{1.902360in}}{\pgfqpoint{1.442006in}{1.902360in}}%
\pgfpathclose%
\pgfusepath{stroke,fill}%
\end{pgfscope}%
\begin{pgfscope}%
\pgfpathrectangle{\pgfqpoint{0.100000in}{0.212622in}}{\pgfqpoint{3.696000in}{3.696000in}}%
\pgfusepath{clip}%
\pgfsetbuttcap%
\pgfsetroundjoin%
\definecolor{currentfill}{rgb}{0.121569,0.466667,0.705882}%
\pgfsetfillcolor{currentfill}%
\pgfsetfillopacity{0.392593}%
\pgfsetlinewidth{1.003750pt}%
\definecolor{currentstroke}{rgb}{0.121569,0.466667,0.705882}%
\pgfsetstrokecolor{currentstroke}%
\pgfsetstrokeopacity{0.392593}%
\pgfsetdash{}{0pt}%
\pgfpathmoveto{\pgfqpoint{2.167954in}{2.334228in}}%
\pgfpathcurveto{\pgfqpoint{2.176190in}{2.334228in}}{\pgfqpoint{2.184090in}{2.337501in}}{\pgfqpoint{2.189914in}{2.343324in}}%
\pgfpathcurveto{\pgfqpoint{2.195738in}{2.349148in}}{\pgfqpoint{2.199011in}{2.357048in}}{\pgfqpoint{2.199011in}{2.365285in}}%
\pgfpathcurveto{\pgfqpoint{2.199011in}{2.373521in}}{\pgfqpoint{2.195738in}{2.381421in}}{\pgfqpoint{2.189914in}{2.387245in}}%
\pgfpathcurveto{\pgfqpoint{2.184090in}{2.393069in}}{\pgfqpoint{2.176190in}{2.396341in}}{\pgfqpoint{2.167954in}{2.396341in}}%
\pgfpathcurveto{\pgfqpoint{2.159718in}{2.396341in}}{\pgfqpoint{2.151818in}{2.393069in}}{\pgfqpoint{2.145994in}{2.387245in}}%
\pgfpathcurveto{\pgfqpoint{2.140170in}{2.381421in}}{\pgfqpoint{2.136898in}{2.373521in}}{\pgfqpoint{2.136898in}{2.365285in}}%
\pgfpathcurveto{\pgfqpoint{2.136898in}{2.357048in}}{\pgfqpoint{2.140170in}{2.349148in}}{\pgfqpoint{2.145994in}{2.343324in}}%
\pgfpathcurveto{\pgfqpoint{2.151818in}{2.337501in}}{\pgfqpoint{2.159718in}{2.334228in}}{\pgfqpoint{2.167954in}{2.334228in}}%
\pgfpathclose%
\pgfusepath{stroke,fill}%
\end{pgfscope}%
\begin{pgfscope}%
\pgfpathrectangle{\pgfqpoint{0.100000in}{0.212622in}}{\pgfqpoint{3.696000in}{3.696000in}}%
\pgfusepath{clip}%
\pgfsetbuttcap%
\pgfsetroundjoin%
\definecolor{currentfill}{rgb}{0.121569,0.466667,0.705882}%
\pgfsetfillcolor{currentfill}%
\pgfsetfillopacity{0.392722}%
\pgfsetlinewidth{1.003750pt}%
\definecolor{currentstroke}{rgb}{0.121569,0.466667,0.705882}%
\pgfsetstrokecolor{currentstroke}%
\pgfsetstrokeopacity{0.392722}%
\pgfsetdash{}{0pt}%
\pgfpathmoveto{\pgfqpoint{2.174765in}{2.341619in}}%
\pgfpathcurveto{\pgfqpoint{2.183001in}{2.341619in}}{\pgfqpoint{2.190901in}{2.344891in}}{\pgfqpoint{2.196725in}{2.350715in}}%
\pgfpathcurveto{\pgfqpoint{2.202549in}{2.356539in}}{\pgfqpoint{2.205821in}{2.364439in}}{\pgfqpoint{2.205821in}{2.372676in}}%
\pgfpathcurveto{\pgfqpoint{2.205821in}{2.380912in}}{\pgfqpoint{2.202549in}{2.388812in}}{\pgfqpoint{2.196725in}{2.394636in}}%
\pgfpathcurveto{\pgfqpoint{2.190901in}{2.400460in}}{\pgfqpoint{2.183001in}{2.403732in}}{\pgfqpoint{2.174765in}{2.403732in}}%
\pgfpathcurveto{\pgfqpoint{2.166528in}{2.403732in}}{\pgfqpoint{2.158628in}{2.400460in}}{\pgfqpoint{2.152804in}{2.394636in}}%
\pgfpathcurveto{\pgfqpoint{2.146980in}{2.388812in}}{\pgfqpoint{2.143708in}{2.380912in}}{\pgfqpoint{2.143708in}{2.372676in}}%
\pgfpathcurveto{\pgfqpoint{2.143708in}{2.364439in}}{\pgfqpoint{2.146980in}{2.356539in}}{\pgfqpoint{2.152804in}{2.350715in}}%
\pgfpathcurveto{\pgfqpoint{2.158628in}{2.344891in}}{\pgfqpoint{2.166528in}{2.341619in}}{\pgfqpoint{2.174765in}{2.341619in}}%
\pgfpathclose%
\pgfusepath{stroke,fill}%
\end{pgfscope}%
\begin{pgfscope}%
\pgfpathrectangle{\pgfqpoint{0.100000in}{0.212622in}}{\pgfqpoint{3.696000in}{3.696000in}}%
\pgfusepath{clip}%
\pgfsetbuttcap%
\pgfsetroundjoin%
\definecolor{currentfill}{rgb}{0.121569,0.466667,0.705882}%
\pgfsetfillcolor{currentfill}%
\pgfsetfillopacity{0.392900}%
\pgfsetlinewidth{1.003750pt}%
\definecolor{currentstroke}{rgb}{0.121569,0.466667,0.705882}%
\pgfsetstrokecolor{currentstroke}%
\pgfsetstrokeopacity{0.392900}%
\pgfsetdash{}{0pt}%
\pgfpathmoveto{\pgfqpoint{1.582800in}{2.020354in}}%
\pgfpathcurveto{\pgfqpoint{1.591037in}{2.020354in}}{\pgfqpoint{1.598937in}{2.023626in}}{\pgfqpoint{1.604761in}{2.029450in}}%
\pgfpathcurveto{\pgfqpoint{1.610585in}{2.035274in}}{\pgfqpoint{1.613857in}{2.043174in}}{\pgfqpoint{1.613857in}{2.051410in}}%
\pgfpathcurveto{\pgfqpoint{1.613857in}{2.059647in}}{\pgfqpoint{1.610585in}{2.067547in}}{\pgfqpoint{1.604761in}{2.073371in}}%
\pgfpathcurveto{\pgfqpoint{1.598937in}{2.079194in}}{\pgfqpoint{1.591037in}{2.082467in}}{\pgfqpoint{1.582800in}{2.082467in}}%
\pgfpathcurveto{\pgfqpoint{1.574564in}{2.082467in}}{\pgfqpoint{1.566664in}{2.079194in}}{\pgfqpoint{1.560840in}{2.073371in}}%
\pgfpathcurveto{\pgfqpoint{1.555016in}{2.067547in}}{\pgfqpoint{1.551744in}{2.059647in}}{\pgfqpoint{1.551744in}{2.051410in}}%
\pgfpathcurveto{\pgfqpoint{1.551744in}{2.043174in}}{\pgfqpoint{1.555016in}{2.035274in}}{\pgfqpoint{1.560840in}{2.029450in}}%
\pgfpathcurveto{\pgfqpoint{1.566664in}{2.023626in}}{\pgfqpoint{1.574564in}{2.020354in}}{\pgfqpoint{1.582800in}{2.020354in}}%
\pgfpathclose%
\pgfusepath{stroke,fill}%
\end{pgfscope}%
\begin{pgfscope}%
\pgfpathrectangle{\pgfqpoint{0.100000in}{0.212622in}}{\pgfqpoint{3.696000in}{3.696000in}}%
\pgfusepath{clip}%
\pgfsetbuttcap%
\pgfsetroundjoin%
\definecolor{currentfill}{rgb}{0.121569,0.466667,0.705882}%
\pgfsetfillcolor{currentfill}%
\pgfsetfillopacity{0.393869}%
\pgfsetlinewidth{1.003750pt}%
\definecolor{currentstroke}{rgb}{0.121569,0.466667,0.705882}%
\pgfsetstrokecolor{currentstroke}%
\pgfsetstrokeopacity{0.393869}%
\pgfsetdash{}{0pt}%
\pgfpathmoveto{\pgfqpoint{1.588091in}{2.010959in}}%
\pgfpathcurveto{\pgfqpoint{1.596327in}{2.010959in}}{\pgfqpoint{1.604227in}{2.014231in}}{\pgfqpoint{1.610051in}{2.020055in}}%
\pgfpathcurveto{\pgfqpoint{1.615875in}{2.025879in}}{\pgfqpoint{1.619148in}{2.033779in}}{\pgfqpoint{1.619148in}{2.042015in}}%
\pgfpathcurveto{\pgfqpoint{1.619148in}{2.050252in}}{\pgfqpoint{1.615875in}{2.058152in}}{\pgfqpoint{1.610051in}{2.063976in}}%
\pgfpathcurveto{\pgfqpoint{1.604227in}{2.069800in}}{\pgfqpoint{1.596327in}{2.073072in}}{\pgfqpoint{1.588091in}{2.073072in}}%
\pgfpathcurveto{\pgfqpoint{1.579855in}{2.073072in}}{\pgfqpoint{1.571955in}{2.069800in}}{\pgfqpoint{1.566131in}{2.063976in}}%
\pgfpathcurveto{\pgfqpoint{1.560307in}{2.058152in}}{\pgfqpoint{1.557035in}{2.050252in}}{\pgfqpoint{1.557035in}{2.042015in}}%
\pgfpathcurveto{\pgfqpoint{1.557035in}{2.033779in}}{\pgfqpoint{1.560307in}{2.025879in}}{\pgfqpoint{1.566131in}{2.020055in}}%
\pgfpathcurveto{\pgfqpoint{1.571955in}{2.014231in}}{\pgfqpoint{1.579855in}{2.010959in}}{\pgfqpoint{1.588091in}{2.010959in}}%
\pgfpathclose%
\pgfusepath{stroke,fill}%
\end{pgfscope}%
\begin{pgfscope}%
\pgfpathrectangle{\pgfqpoint{0.100000in}{0.212622in}}{\pgfqpoint{3.696000in}{3.696000in}}%
\pgfusepath{clip}%
\pgfsetbuttcap%
\pgfsetroundjoin%
\definecolor{currentfill}{rgb}{0.121569,0.466667,0.705882}%
\pgfsetfillcolor{currentfill}%
\pgfsetfillopacity{0.394303}%
\pgfsetlinewidth{1.003750pt}%
\definecolor{currentstroke}{rgb}{0.121569,0.466667,0.705882}%
\pgfsetstrokecolor{currentstroke}%
\pgfsetstrokeopacity{0.394303}%
\pgfsetdash{}{0pt}%
\pgfpathmoveto{\pgfqpoint{2.081020in}{2.281362in}}%
\pgfpathcurveto{\pgfqpoint{2.089256in}{2.281362in}}{\pgfqpoint{2.097156in}{2.284635in}}{\pgfqpoint{2.102980in}{2.290459in}}%
\pgfpathcurveto{\pgfqpoint{2.108804in}{2.296282in}}{\pgfqpoint{2.112076in}{2.304183in}}{\pgfqpoint{2.112076in}{2.312419in}}%
\pgfpathcurveto{\pgfqpoint{2.112076in}{2.320655in}}{\pgfqpoint{2.108804in}{2.328555in}}{\pgfqpoint{2.102980in}{2.334379in}}%
\pgfpathcurveto{\pgfqpoint{2.097156in}{2.340203in}}{\pgfqpoint{2.089256in}{2.343475in}}{\pgfqpoint{2.081020in}{2.343475in}}%
\pgfpathcurveto{\pgfqpoint{2.072783in}{2.343475in}}{\pgfqpoint{2.064883in}{2.340203in}}{\pgfqpoint{2.059059in}{2.334379in}}%
\pgfpathcurveto{\pgfqpoint{2.053235in}{2.328555in}}{\pgfqpoint{2.049963in}{2.320655in}}{\pgfqpoint{2.049963in}{2.312419in}}%
\pgfpathcurveto{\pgfqpoint{2.049963in}{2.304183in}}{\pgfqpoint{2.053235in}{2.296282in}}{\pgfqpoint{2.059059in}{2.290459in}}%
\pgfpathcurveto{\pgfqpoint{2.064883in}{2.284635in}}{\pgfqpoint{2.072783in}{2.281362in}}{\pgfqpoint{2.081020in}{2.281362in}}%
\pgfpathclose%
\pgfusepath{stroke,fill}%
\end{pgfscope}%
\begin{pgfscope}%
\pgfpathrectangle{\pgfqpoint{0.100000in}{0.212622in}}{\pgfqpoint{3.696000in}{3.696000in}}%
\pgfusepath{clip}%
\pgfsetbuttcap%
\pgfsetroundjoin%
\definecolor{currentfill}{rgb}{0.121569,0.466667,0.705882}%
\pgfsetfillcolor{currentfill}%
\pgfsetfillopacity{0.395409}%
\pgfsetlinewidth{1.003750pt}%
\definecolor{currentstroke}{rgb}{0.121569,0.466667,0.705882}%
\pgfsetstrokecolor{currentstroke}%
\pgfsetstrokeopacity{0.395409}%
\pgfsetdash{}{0pt}%
\pgfpathmoveto{\pgfqpoint{2.081288in}{2.268731in}}%
\pgfpathcurveto{\pgfqpoint{2.089524in}{2.268731in}}{\pgfqpoint{2.097424in}{2.272003in}}{\pgfqpoint{2.103248in}{2.277827in}}%
\pgfpathcurveto{\pgfqpoint{2.109072in}{2.283651in}}{\pgfqpoint{2.112344in}{2.291551in}}{\pgfqpoint{2.112344in}{2.299787in}}%
\pgfpathcurveto{\pgfqpoint{2.112344in}{2.308023in}}{\pgfqpoint{2.109072in}{2.315923in}}{\pgfqpoint{2.103248in}{2.321747in}}%
\pgfpathcurveto{\pgfqpoint{2.097424in}{2.327571in}}{\pgfqpoint{2.089524in}{2.330844in}}{\pgfqpoint{2.081288in}{2.330844in}}%
\pgfpathcurveto{\pgfqpoint{2.073052in}{2.330844in}}{\pgfqpoint{2.065152in}{2.327571in}}{\pgfqpoint{2.059328in}{2.321747in}}%
\pgfpathcurveto{\pgfqpoint{2.053504in}{2.315923in}}{\pgfqpoint{2.050231in}{2.308023in}}{\pgfqpoint{2.050231in}{2.299787in}}%
\pgfpathcurveto{\pgfqpoint{2.050231in}{2.291551in}}{\pgfqpoint{2.053504in}{2.283651in}}{\pgfqpoint{2.059328in}{2.277827in}}%
\pgfpathcurveto{\pgfqpoint{2.065152in}{2.272003in}}{\pgfqpoint{2.073052in}{2.268731in}}{\pgfqpoint{2.081288in}{2.268731in}}%
\pgfpathclose%
\pgfusepath{stroke,fill}%
\end{pgfscope}%
\begin{pgfscope}%
\pgfpathrectangle{\pgfqpoint{0.100000in}{0.212622in}}{\pgfqpoint{3.696000in}{3.696000in}}%
\pgfusepath{clip}%
\pgfsetbuttcap%
\pgfsetroundjoin%
\definecolor{currentfill}{rgb}{0.121569,0.466667,0.705882}%
\pgfsetfillcolor{currentfill}%
\pgfsetfillopacity{0.395411}%
\pgfsetlinewidth{1.003750pt}%
\definecolor{currentstroke}{rgb}{0.121569,0.466667,0.705882}%
\pgfsetstrokecolor{currentstroke}%
\pgfsetstrokeopacity{0.395411}%
\pgfsetdash{}{0pt}%
\pgfpathmoveto{\pgfqpoint{2.184265in}{2.348193in}}%
\pgfpathcurveto{\pgfqpoint{2.192502in}{2.348193in}}{\pgfqpoint{2.200402in}{2.351466in}}{\pgfqpoint{2.206225in}{2.357290in}}%
\pgfpathcurveto{\pgfqpoint{2.212049in}{2.363113in}}{\pgfqpoint{2.215322in}{2.371013in}}{\pgfqpoint{2.215322in}{2.379250in}}%
\pgfpathcurveto{\pgfqpoint{2.215322in}{2.387486in}}{\pgfqpoint{2.212049in}{2.395386in}}{\pgfqpoint{2.206225in}{2.401210in}}%
\pgfpathcurveto{\pgfqpoint{2.200402in}{2.407034in}}{\pgfqpoint{2.192502in}{2.410306in}}{\pgfqpoint{2.184265in}{2.410306in}}%
\pgfpathcurveto{\pgfqpoint{2.176029in}{2.410306in}}{\pgfqpoint{2.168129in}{2.407034in}}{\pgfqpoint{2.162305in}{2.401210in}}%
\pgfpathcurveto{\pgfqpoint{2.156481in}{2.395386in}}{\pgfqpoint{2.153209in}{2.387486in}}{\pgfqpoint{2.153209in}{2.379250in}}%
\pgfpathcurveto{\pgfqpoint{2.153209in}{2.371013in}}{\pgfqpoint{2.156481in}{2.363113in}}{\pgfqpoint{2.162305in}{2.357290in}}%
\pgfpathcurveto{\pgfqpoint{2.168129in}{2.351466in}}{\pgfqpoint{2.176029in}{2.348193in}}{\pgfqpoint{2.184265in}{2.348193in}}%
\pgfpathclose%
\pgfusepath{stroke,fill}%
\end{pgfscope}%
\begin{pgfscope}%
\pgfpathrectangle{\pgfqpoint{0.100000in}{0.212622in}}{\pgfqpoint{3.696000in}{3.696000in}}%
\pgfusepath{clip}%
\pgfsetbuttcap%
\pgfsetroundjoin%
\definecolor{currentfill}{rgb}{0.121569,0.466667,0.705882}%
\pgfsetfillcolor{currentfill}%
\pgfsetfillopacity{0.395469}%
\pgfsetlinewidth{1.003750pt}%
\definecolor{currentstroke}{rgb}{0.121569,0.466667,0.705882}%
\pgfsetstrokecolor{currentstroke}%
\pgfsetstrokeopacity{0.395469}%
\pgfsetdash{}{0pt}%
\pgfpathmoveto{\pgfqpoint{1.441599in}{1.905342in}}%
\pgfpathcurveto{\pgfqpoint{1.449835in}{1.905342in}}{\pgfqpoint{1.457735in}{1.908615in}}{\pgfqpoint{1.463559in}{1.914439in}}%
\pgfpathcurveto{\pgfqpoint{1.469383in}{1.920262in}}{\pgfqpoint{1.472655in}{1.928162in}}{\pgfqpoint{1.472655in}{1.936399in}}%
\pgfpathcurveto{\pgfqpoint{1.472655in}{1.944635in}}{\pgfqpoint{1.469383in}{1.952535in}}{\pgfqpoint{1.463559in}{1.958359in}}%
\pgfpathcurveto{\pgfqpoint{1.457735in}{1.964183in}}{\pgfqpoint{1.449835in}{1.967455in}}{\pgfqpoint{1.441599in}{1.967455in}}%
\pgfpathcurveto{\pgfqpoint{1.433362in}{1.967455in}}{\pgfqpoint{1.425462in}{1.964183in}}{\pgfqpoint{1.419638in}{1.958359in}}%
\pgfpathcurveto{\pgfqpoint{1.413815in}{1.952535in}}{\pgfqpoint{1.410542in}{1.944635in}}{\pgfqpoint{1.410542in}{1.936399in}}%
\pgfpathcurveto{\pgfqpoint{1.410542in}{1.928162in}}{\pgfqpoint{1.413815in}{1.920262in}}{\pgfqpoint{1.419638in}{1.914439in}}%
\pgfpathcurveto{\pgfqpoint{1.425462in}{1.908615in}}{\pgfqpoint{1.433362in}{1.905342in}}{\pgfqpoint{1.441599in}{1.905342in}}%
\pgfpathclose%
\pgfusepath{stroke,fill}%
\end{pgfscope}%
\begin{pgfscope}%
\pgfpathrectangle{\pgfqpoint{0.100000in}{0.212622in}}{\pgfqpoint{3.696000in}{3.696000in}}%
\pgfusepath{clip}%
\pgfsetbuttcap%
\pgfsetroundjoin%
\definecolor{currentfill}{rgb}{0.121569,0.466667,0.705882}%
\pgfsetfillcolor{currentfill}%
\pgfsetfillopacity{0.398290}%
\pgfsetlinewidth{1.003750pt}%
\definecolor{currentstroke}{rgb}{0.121569,0.466667,0.705882}%
\pgfsetstrokecolor{currentstroke}%
\pgfsetstrokeopacity{0.398290}%
\pgfsetdash{}{0pt}%
\pgfpathmoveto{\pgfqpoint{2.163568in}{2.325122in}}%
\pgfpathcurveto{\pgfqpoint{2.171804in}{2.325122in}}{\pgfqpoint{2.179704in}{2.328395in}}{\pgfqpoint{2.185528in}{2.334218in}}%
\pgfpathcurveto{\pgfqpoint{2.191352in}{2.340042in}}{\pgfqpoint{2.194625in}{2.347942in}}{\pgfqpoint{2.194625in}{2.356179in}}%
\pgfpathcurveto{\pgfqpoint{2.194625in}{2.364415in}}{\pgfqpoint{2.191352in}{2.372315in}}{\pgfqpoint{2.185528in}{2.378139in}}%
\pgfpathcurveto{\pgfqpoint{2.179704in}{2.383963in}}{\pgfqpoint{2.171804in}{2.387235in}}{\pgfqpoint{2.163568in}{2.387235in}}%
\pgfpathcurveto{\pgfqpoint{2.155332in}{2.387235in}}{\pgfqpoint{2.147432in}{2.383963in}}{\pgfqpoint{2.141608in}{2.378139in}}%
\pgfpathcurveto{\pgfqpoint{2.135784in}{2.372315in}}{\pgfqpoint{2.132512in}{2.364415in}}{\pgfqpoint{2.132512in}{2.356179in}}%
\pgfpathcurveto{\pgfqpoint{2.132512in}{2.347942in}}{\pgfqpoint{2.135784in}{2.340042in}}{\pgfqpoint{2.141608in}{2.334218in}}%
\pgfpathcurveto{\pgfqpoint{2.147432in}{2.328395in}}{\pgfqpoint{2.155332in}{2.325122in}}{\pgfqpoint{2.163568in}{2.325122in}}%
\pgfpathclose%
\pgfusepath{stroke,fill}%
\end{pgfscope}%
\begin{pgfscope}%
\pgfpathrectangle{\pgfqpoint{0.100000in}{0.212622in}}{\pgfqpoint{3.696000in}{3.696000in}}%
\pgfusepath{clip}%
\pgfsetbuttcap%
\pgfsetroundjoin%
\definecolor{currentfill}{rgb}{0.121569,0.466667,0.705882}%
\pgfsetfillcolor{currentfill}%
\pgfsetfillopacity{0.398980}%
\pgfsetlinewidth{1.003750pt}%
\definecolor{currentstroke}{rgb}{0.121569,0.466667,0.705882}%
\pgfsetstrokecolor{currentstroke}%
\pgfsetstrokeopacity{0.398980}%
\pgfsetdash{}{0pt}%
\pgfpathmoveto{\pgfqpoint{1.941516in}{2.213192in}}%
\pgfpathcurveto{\pgfqpoint{1.949752in}{2.213192in}}{\pgfqpoint{1.957652in}{2.216464in}}{\pgfqpoint{1.963476in}{2.222288in}}%
\pgfpathcurveto{\pgfqpoint{1.969300in}{2.228112in}}{\pgfqpoint{1.972572in}{2.236012in}}{\pgfqpoint{1.972572in}{2.244249in}}%
\pgfpathcurveto{\pgfqpoint{1.972572in}{2.252485in}}{\pgfqpoint{1.969300in}{2.260385in}}{\pgfqpoint{1.963476in}{2.266209in}}%
\pgfpathcurveto{\pgfqpoint{1.957652in}{2.272033in}}{\pgfqpoint{1.949752in}{2.275305in}}{\pgfqpoint{1.941516in}{2.275305in}}%
\pgfpathcurveto{\pgfqpoint{1.933279in}{2.275305in}}{\pgfqpoint{1.925379in}{2.272033in}}{\pgfqpoint{1.919555in}{2.266209in}}%
\pgfpathcurveto{\pgfqpoint{1.913732in}{2.260385in}}{\pgfqpoint{1.910459in}{2.252485in}}{\pgfqpoint{1.910459in}{2.244249in}}%
\pgfpathcurveto{\pgfqpoint{1.910459in}{2.236012in}}{\pgfqpoint{1.913732in}{2.228112in}}{\pgfqpoint{1.919555in}{2.222288in}}%
\pgfpathcurveto{\pgfqpoint{1.925379in}{2.216464in}}{\pgfqpoint{1.933279in}{2.213192in}}{\pgfqpoint{1.941516in}{2.213192in}}%
\pgfpathclose%
\pgfusepath{stroke,fill}%
\end{pgfscope}%
\begin{pgfscope}%
\pgfpathrectangle{\pgfqpoint{0.100000in}{0.212622in}}{\pgfqpoint{3.696000in}{3.696000in}}%
\pgfusepath{clip}%
\pgfsetbuttcap%
\pgfsetroundjoin%
\definecolor{currentfill}{rgb}{0.121569,0.466667,0.705882}%
\pgfsetfillcolor{currentfill}%
\pgfsetfillopacity{0.399161}%
\pgfsetlinewidth{1.003750pt}%
\definecolor{currentstroke}{rgb}{0.121569,0.466667,0.705882}%
\pgfsetstrokecolor{currentstroke}%
\pgfsetstrokeopacity{0.399161}%
\pgfsetdash{}{0pt}%
\pgfpathmoveto{\pgfqpoint{1.516529in}{1.963773in}}%
\pgfpathcurveto{\pgfqpoint{1.524765in}{1.963773in}}{\pgfqpoint{1.532665in}{1.967045in}}{\pgfqpoint{1.538489in}{1.972869in}}%
\pgfpathcurveto{\pgfqpoint{1.544313in}{1.978693in}}{\pgfqpoint{1.547585in}{1.986593in}}{\pgfqpoint{1.547585in}{1.994829in}}%
\pgfpathcurveto{\pgfqpoint{1.547585in}{2.003066in}}{\pgfqpoint{1.544313in}{2.010966in}}{\pgfqpoint{1.538489in}{2.016790in}}%
\pgfpathcurveto{\pgfqpoint{1.532665in}{2.022614in}}{\pgfqpoint{1.524765in}{2.025886in}}{\pgfqpoint{1.516529in}{2.025886in}}%
\pgfpathcurveto{\pgfqpoint{1.508292in}{2.025886in}}{\pgfqpoint{1.500392in}{2.022614in}}{\pgfqpoint{1.494568in}{2.016790in}}%
\pgfpathcurveto{\pgfqpoint{1.488744in}{2.010966in}}{\pgfqpoint{1.485472in}{2.003066in}}{\pgfqpoint{1.485472in}{1.994829in}}%
\pgfpathcurveto{\pgfqpoint{1.485472in}{1.986593in}}{\pgfqpoint{1.488744in}{1.978693in}}{\pgfqpoint{1.494568in}{1.972869in}}%
\pgfpathcurveto{\pgfqpoint{1.500392in}{1.967045in}}{\pgfqpoint{1.508292in}{1.963773in}}{\pgfqpoint{1.516529in}{1.963773in}}%
\pgfpathclose%
\pgfusepath{stroke,fill}%
\end{pgfscope}%
\begin{pgfscope}%
\pgfpathrectangle{\pgfqpoint{0.100000in}{0.212622in}}{\pgfqpoint{3.696000in}{3.696000in}}%
\pgfusepath{clip}%
\pgfsetbuttcap%
\pgfsetroundjoin%
\definecolor{currentfill}{rgb}{0.121569,0.466667,0.705882}%
\pgfsetfillcolor{currentfill}%
\pgfsetfillopacity{0.399780}%
\pgfsetlinewidth{1.003750pt}%
\definecolor{currentstroke}{rgb}{0.121569,0.466667,0.705882}%
\pgfsetstrokecolor{currentstroke}%
\pgfsetstrokeopacity{0.399780}%
\pgfsetdash{}{0pt}%
\pgfpathmoveto{\pgfqpoint{1.923026in}{2.192451in}}%
\pgfpathcurveto{\pgfqpoint{1.931262in}{2.192451in}}{\pgfqpoint{1.939162in}{2.195723in}}{\pgfqpoint{1.944986in}{2.201547in}}%
\pgfpathcurveto{\pgfqpoint{1.950810in}{2.207371in}}{\pgfqpoint{1.954083in}{2.215271in}}{\pgfqpoint{1.954083in}{2.223507in}}%
\pgfpathcurveto{\pgfqpoint{1.954083in}{2.231743in}}{\pgfqpoint{1.950810in}{2.239643in}}{\pgfqpoint{1.944986in}{2.245467in}}%
\pgfpathcurveto{\pgfqpoint{1.939162in}{2.251291in}}{\pgfqpoint{1.931262in}{2.254564in}}{\pgfqpoint{1.923026in}{2.254564in}}%
\pgfpathcurveto{\pgfqpoint{1.914790in}{2.254564in}}{\pgfqpoint{1.906890in}{2.251291in}}{\pgfqpoint{1.901066in}{2.245467in}}%
\pgfpathcurveto{\pgfqpoint{1.895242in}{2.239643in}}{\pgfqpoint{1.891970in}{2.231743in}}{\pgfqpoint{1.891970in}{2.223507in}}%
\pgfpathcurveto{\pgfqpoint{1.891970in}{2.215271in}}{\pgfqpoint{1.895242in}{2.207371in}}{\pgfqpoint{1.901066in}{2.201547in}}%
\pgfpathcurveto{\pgfqpoint{1.906890in}{2.195723in}}{\pgfqpoint{1.914790in}{2.192451in}}{\pgfqpoint{1.923026in}{2.192451in}}%
\pgfpathclose%
\pgfusepath{stroke,fill}%
\end{pgfscope}%
\begin{pgfscope}%
\pgfpathrectangle{\pgfqpoint{0.100000in}{0.212622in}}{\pgfqpoint{3.696000in}{3.696000in}}%
\pgfusepath{clip}%
\pgfsetbuttcap%
\pgfsetroundjoin%
\definecolor{currentfill}{rgb}{0.121569,0.466667,0.705882}%
\pgfsetfillcolor{currentfill}%
\pgfsetfillopacity{0.401545}%
\pgfsetlinewidth{1.003750pt}%
\definecolor{currentstroke}{rgb}{0.121569,0.466667,0.705882}%
\pgfsetstrokecolor{currentstroke}%
\pgfsetstrokeopacity{0.401545}%
\pgfsetdash{}{0pt}%
\pgfpathmoveto{\pgfqpoint{2.185757in}{2.345889in}}%
\pgfpathcurveto{\pgfqpoint{2.193994in}{2.345889in}}{\pgfqpoint{2.201894in}{2.349161in}}{\pgfqpoint{2.207718in}{2.354985in}}%
\pgfpathcurveto{\pgfqpoint{2.213541in}{2.360809in}}{\pgfqpoint{2.216814in}{2.368709in}}{\pgfqpoint{2.216814in}{2.376945in}}%
\pgfpathcurveto{\pgfqpoint{2.216814in}{2.385182in}}{\pgfqpoint{2.213541in}{2.393082in}}{\pgfqpoint{2.207718in}{2.398906in}}%
\pgfpathcurveto{\pgfqpoint{2.201894in}{2.404729in}}{\pgfqpoint{2.193994in}{2.408002in}}{\pgfqpoint{2.185757in}{2.408002in}}%
\pgfpathcurveto{\pgfqpoint{2.177521in}{2.408002in}}{\pgfqpoint{2.169621in}{2.404729in}}{\pgfqpoint{2.163797in}{2.398906in}}%
\pgfpathcurveto{\pgfqpoint{2.157973in}{2.393082in}}{\pgfqpoint{2.154701in}{2.385182in}}{\pgfqpoint{2.154701in}{2.376945in}}%
\pgfpathcurveto{\pgfqpoint{2.154701in}{2.368709in}}{\pgfqpoint{2.157973in}{2.360809in}}{\pgfqpoint{2.163797in}{2.354985in}}%
\pgfpathcurveto{\pgfqpoint{2.169621in}{2.349161in}}{\pgfqpoint{2.177521in}{2.345889in}}{\pgfqpoint{2.185757in}{2.345889in}}%
\pgfpathclose%
\pgfusepath{stroke,fill}%
\end{pgfscope}%
\begin{pgfscope}%
\pgfpathrectangle{\pgfqpoint{0.100000in}{0.212622in}}{\pgfqpoint{3.696000in}{3.696000in}}%
\pgfusepath{clip}%
\pgfsetbuttcap%
\pgfsetroundjoin%
\definecolor{currentfill}{rgb}{0.121569,0.466667,0.705882}%
\pgfsetfillcolor{currentfill}%
\pgfsetfillopacity{0.402300}%
\pgfsetlinewidth{1.003750pt}%
\definecolor{currentstroke}{rgb}{0.121569,0.466667,0.705882}%
\pgfsetstrokecolor{currentstroke}%
\pgfsetstrokeopacity{0.402300}%
\pgfsetdash{}{0pt}%
\pgfpathmoveto{\pgfqpoint{1.429527in}{1.893720in}}%
\pgfpathcurveto{\pgfqpoint{1.437764in}{1.893720in}}{\pgfqpoint{1.445664in}{1.896993in}}{\pgfqpoint{1.451487in}{1.902817in}}%
\pgfpathcurveto{\pgfqpoint{1.457311in}{1.908641in}}{\pgfqpoint{1.460584in}{1.916541in}}{\pgfqpoint{1.460584in}{1.924777in}}%
\pgfpathcurveto{\pgfqpoint{1.460584in}{1.933013in}}{\pgfqpoint{1.457311in}{1.940913in}}{\pgfqpoint{1.451487in}{1.946737in}}%
\pgfpathcurveto{\pgfqpoint{1.445664in}{1.952561in}}{\pgfqpoint{1.437764in}{1.955833in}}{\pgfqpoint{1.429527in}{1.955833in}}%
\pgfpathcurveto{\pgfqpoint{1.421291in}{1.955833in}}{\pgfqpoint{1.413391in}{1.952561in}}{\pgfqpoint{1.407567in}{1.946737in}}%
\pgfpathcurveto{\pgfqpoint{1.401743in}{1.940913in}}{\pgfqpoint{1.398471in}{1.933013in}}{\pgfqpoint{1.398471in}{1.924777in}}%
\pgfpathcurveto{\pgfqpoint{1.398471in}{1.916541in}}{\pgfqpoint{1.401743in}{1.908641in}}{\pgfqpoint{1.407567in}{1.902817in}}%
\pgfpathcurveto{\pgfqpoint{1.413391in}{1.896993in}}{\pgfqpoint{1.421291in}{1.893720in}}{\pgfqpoint{1.429527in}{1.893720in}}%
\pgfpathclose%
\pgfusepath{stroke,fill}%
\end{pgfscope}%
\begin{pgfscope}%
\pgfpathrectangle{\pgfqpoint{0.100000in}{0.212622in}}{\pgfqpoint{3.696000in}{3.696000in}}%
\pgfusepath{clip}%
\pgfsetbuttcap%
\pgfsetroundjoin%
\definecolor{currentfill}{rgb}{0.121569,0.466667,0.705882}%
\pgfsetfillcolor{currentfill}%
\pgfsetfillopacity{0.402482}%
\pgfsetlinewidth{1.003750pt}%
\definecolor{currentstroke}{rgb}{0.121569,0.466667,0.705882}%
\pgfsetstrokecolor{currentstroke}%
\pgfsetstrokeopacity{0.402482}%
\pgfsetdash{}{0pt}%
\pgfpathmoveto{\pgfqpoint{1.513610in}{1.953210in}}%
\pgfpathcurveto{\pgfqpoint{1.521846in}{1.953210in}}{\pgfqpoint{1.529746in}{1.956482in}}{\pgfqpoint{1.535570in}{1.962306in}}%
\pgfpathcurveto{\pgfqpoint{1.541394in}{1.968130in}}{\pgfqpoint{1.544666in}{1.976030in}}{\pgfqpoint{1.544666in}{1.984266in}}%
\pgfpathcurveto{\pgfqpoint{1.544666in}{1.992503in}}{\pgfqpoint{1.541394in}{2.000403in}}{\pgfqpoint{1.535570in}{2.006227in}}%
\pgfpathcurveto{\pgfqpoint{1.529746in}{2.012050in}}{\pgfqpoint{1.521846in}{2.015323in}}{\pgfqpoint{1.513610in}{2.015323in}}%
\pgfpathcurveto{\pgfqpoint{1.505374in}{2.015323in}}{\pgfqpoint{1.497473in}{2.012050in}}{\pgfqpoint{1.491650in}{2.006227in}}%
\pgfpathcurveto{\pgfqpoint{1.485826in}{2.000403in}}{\pgfqpoint{1.482553in}{1.992503in}}{\pgfqpoint{1.482553in}{1.984266in}}%
\pgfpathcurveto{\pgfqpoint{1.482553in}{1.976030in}}{\pgfqpoint{1.485826in}{1.968130in}}{\pgfqpoint{1.491650in}{1.962306in}}%
\pgfpathcurveto{\pgfqpoint{1.497473in}{1.956482in}}{\pgfqpoint{1.505374in}{1.953210in}}{\pgfqpoint{1.513610in}{1.953210in}}%
\pgfpathclose%
\pgfusepath{stroke,fill}%
\end{pgfscope}%
\begin{pgfscope}%
\pgfpathrectangle{\pgfqpoint{0.100000in}{0.212622in}}{\pgfqpoint{3.696000in}{3.696000in}}%
\pgfusepath{clip}%
\pgfsetbuttcap%
\pgfsetroundjoin%
\definecolor{currentfill}{rgb}{0.121569,0.466667,0.705882}%
\pgfsetfillcolor{currentfill}%
\pgfsetfillopacity{0.402799}%
\pgfsetlinewidth{1.003750pt}%
\definecolor{currentstroke}{rgb}{0.121569,0.466667,0.705882}%
\pgfsetstrokecolor{currentstroke}%
\pgfsetstrokeopacity{0.402799}%
\pgfsetdash{}{0pt}%
\pgfpathmoveto{\pgfqpoint{1.494292in}{1.942873in}}%
\pgfpathcurveto{\pgfqpoint{1.502528in}{1.942873in}}{\pgfqpoint{1.510428in}{1.946146in}}{\pgfqpoint{1.516252in}{1.951970in}}%
\pgfpathcurveto{\pgfqpoint{1.522076in}{1.957794in}}{\pgfqpoint{1.525348in}{1.965694in}}{\pgfqpoint{1.525348in}{1.973930in}}%
\pgfpathcurveto{\pgfqpoint{1.525348in}{1.982166in}}{\pgfqpoint{1.522076in}{1.990066in}}{\pgfqpoint{1.516252in}{1.995890in}}%
\pgfpathcurveto{\pgfqpoint{1.510428in}{2.001714in}}{\pgfqpoint{1.502528in}{2.004986in}}{\pgfqpoint{1.494292in}{2.004986in}}%
\pgfpathcurveto{\pgfqpoint{1.486055in}{2.004986in}}{\pgfqpoint{1.478155in}{2.001714in}}{\pgfqpoint{1.472331in}{1.995890in}}%
\pgfpathcurveto{\pgfqpoint{1.466507in}{1.990066in}}{\pgfqpoint{1.463235in}{1.982166in}}{\pgfqpoint{1.463235in}{1.973930in}}%
\pgfpathcurveto{\pgfqpoint{1.463235in}{1.965694in}}{\pgfqpoint{1.466507in}{1.957794in}}{\pgfqpoint{1.472331in}{1.951970in}}%
\pgfpathcurveto{\pgfqpoint{1.478155in}{1.946146in}}{\pgfqpoint{1.486055in}{1.942873in}}{\pgfqpoint{1.494292in}{1.942873in}}%
\pgfpathclose%
\pgfusepath{stroke,fill}%
\end{pgfscope}%
\begin{pgfscope}%
\pgfpathrectangle{\pgfqpoint{0.100000in}{0.212622in}}{\pgfqpoint{3.696000in}{3.696000in}}%
\pgfusepath{clip}%
\pgfsetbuttcap%
\pgfsetroundjoin%
\definecolor{currentfill}{rgb}{0.121569,0.466667,0.705882}%
\pgfsetfillcolor{currentfill}%
\pgfsetfillopacity{0.403109}%
\pgfsetlinewidth{1.003750pt}%
\definecolor{currentstroke}{rgb}{0.121569,0.466667,0.705882}%
\pgfsetstrokecolor{currentstroke}%
\pgfsetstrokeopacity{0.403109}%
\pgfsetdash{}{0pt}%
\pgfpathmoveto{\pgfqpoint{1.940284in}{2.204216in}}%
\pgfpathcurveto{\pgfqpoint{1.948520in}{2.204216in}}{\pgfqpoint{1.956420in}{2.207489in}}{\pgfqpoint{1.962244in}{2.213313in}}%
\pgfpathcurveto{\pgfqpoint{1.968068in}{2.219136in}}{\pgfqpoint{1.971341in}{2.227037in}}{\pgfqpoint{1.971341in}{2.235273in}}%
\pgfpathcurveto{\pgfqpoint{1.971341in}{2.243509in}}{\pgfqpoint{1.968068in}{2.251409in}}{\pgfqpoint{1.962244in}{2.257233in}}%
\pgfpathcurveto{\pgfqpoint{1.956420in}{2.263057in}}{\pgfqpoint{1.948520in}{2.266329in}}{\pgfqpoint{1.940284in}{2.266329in}}%
\pgfpathcurveto{\pgfqpoint{1.932048in}{2.266329in}}{\pgfqpoint{1.924148in}{2.263057in}}{\pgfqpoint{1.918324in}{2.257233in}}%
\pgfpathcurveto{\pgfqpoint{1.912500in}{2.251409in}}{\pgfqpoint{1.909228in}{2.243509in}}{\pgfqpoint{1.909228in}{2.235273in}}%
\pgfpathcurveto{\pgfqpoint{1.909228in}{2.227037in}}{\pgfqpoint{1.912500in}{2.219136in}}{\pgfqpoint{1.918324in}{2.213313in}}%
\pgfpathcurveto{\pgfqpoint{1.924148in}{2.207489in}}{\pgfqpoint{1.932048in}{2.204216in}}{\pgfqpoint{1.940284in}{2.204216in}}%
\pgfpathclose%
\pgfusepath{stroke,fill}%
\end{pgfscope}%
\begin{pgfscope}%
\pgfpathrectangle{\pgfqpoint{0.100000in}{0.212622in}}{\pgfqpoint{3.696000in}{3.696000in}}%
\pgfusepath{clip}%
\pgfsetbuttcap%
\pgfsetroundjoin%
\definecolor{currentfill}{rgb}{0.121569,0.466667,0.705882}%
\pgfsetfillcolor{currentfill}%
\pgfsetfillopacity{0.403925}%
\pgfsetlinewidth{1.003750pt}%
\definecolor{currentstroke}{rgb}{0.121569,0.466667,0.705882}%
\pgfsetstrokecolor{currentstroke}%
\pgfsetstrokeopacity{0.403925}%
\pgfsetdash{}{0pt}%
\pgfpathmoveto{\pgfqpoint{1.435634in}{1.899108in}}%
\pgfpathcurveto{\pgfqpoint{1.443871in}{1.899108in}}{\pgfqpoint{1.451771in}{1.902380in}}{\pgfqpoint{1.457595in}{1.908204in}}%
\pgfpathcurveto{\pgfqpoint{1.463419in}{1.914028in}}{\pgfqpoint{1.466691in}{1.921928in}}{\pgfqpoint{1.466691in}{1.930165in}}%
\pgfpathcurveto{\pgfqpoint{1.466691in}{1.938401in}}{\pgfqpoint{1.463419in}{1.946301in}}{\pgfqpoint{1.457595in}{1.952125in}}%
\pgfpathcurveto{\pgfqpoint{1.451771in}{1.957949in}}{\pgfqpoint{1.443871in}{1.961221in}}{\pgfqpoint{1.435634in}{1.961221in}}%
\pgfpathcurveto{\pgfqpoint{1.427398in}{1.961221in}}{\pgfqpoint{1.419498in}{1.957949in}}{\pgfqpoint{1.413674in}{1.952125in}}%
\pgfpathcurveto{\pgfqpoint{1.407850in}{1.946301in}}{\pgfqpoint{1.404578in}{1.938401in}}{\pgfqpoint{1.404578in}{1.930165in}}%
\pgfpathcurveto{\pgfqpoint{1.404578in}{1.921928in}}{\pgfqpoint{1.407850in}{1.914028in}}{\pgfqpoint{1.413674in}{1.908204in}}%
\pgfpathcurveto{\pgfqpoint{1.419498in}{1.902380in}}{\pgfqpoint{1.427398in}{1.899108in}}{\pgfqpoint{1.435634in}{1.899108in}}%
\pgfpathclose%
\pgfusepath{stroke,fill}%
\end{pgfscope}%
\begin{pgfscope}%
\pgfpathrectangle{\pgfqpoint{0.100000in}{0.212622in}}{\pgfqpoint{3.696000in}{3.696000in}}%
\pgfusepath{clip}%
\pgfsetbuttcap%
\pgfsetroundjoin%
\definecolor{currentfill}{rgb}{0.121569,0.466667,0.705882}%
\pgfsetfillcolor{currentfill}%
\pgfsetfillopacity{0.404534}%
\pgfsetlinewidth{1.003750pt}%
\definecolor{currentstroke}{rgb}{0.121569,0.466667,0.705882}%
\pgfsetstrokecolor{currentstroke}%
\pgfsetstrokeopacity{0.404534}%
\pgfsetdash{}{0pt}%
\pgfpathmoveto{\pgfqpoint{1.529998in}{1.967357in}}%
\pgfpathcurveto{\pgfqpoint{1.538234in}{1.967357in}}{\pgfqpoint{1.546134in}{1.970629in}}{\pgfqpoint{1.551958in}{1.976453in}}%
\pgfpathcurveto{\pgfqpoint{1.557782in}{1.982277in}}{\pgfqpoint{1.561054in}{1.990177in}}{\pgfqpoint{1.561054in}{1.998414in}}%
\pgfpathcurveto{\pgfqpoint{1.561054in}{2.006650in}}{\pgfqpoint{1.557782in}{2.014550in}}{\pgfqpoint{1.551958in}{2.020374in}}%
\pgfpathcurveto{\pgfqpoint{1.546134in}{2.026198in}}{\pgfqpoint{1.538234in}{2.029470in}}{\pgfqpoint{1.529998in}{2.029470in}}%
\pgfpathcurveto{\pgfqpoint{1.521761in}{2.029470in}}{\pgfqpoint{1.513861in}{2.026198in}}{\pgfqpoint{1.508037in}{2.020374in}}%
\pgfpathcurveto{\pgfqpoint{1.502213in}{2.014550in}}{\pgfqpoint{1.498941in}{2.006650in}}{\pgfqpoint{1.498941in}{1.998414in}}%
\pgfpathcurveto{\pgfqpoint{1.498941in}{1.990177in}}{\pgfqpoint{1.502213in}{1.982277in}}{\pgfqpoint{1.508037in}{1.976453in}}%
\pgfpathcurveto{\pgfqpoint{1.513861in}{1.970629in}}{\pgfqpoint{1.521761in}{1.967357in}}{\pgfqpoint{1.529998in}{1.967357in}}%
\pgfpathclose%
\pgfusepath{stroke,fill}%
\end{pgfscope}%
\begin{pgfscope}%
\pgfpathrectangle{\pgfqpoint{0.100000in}{0.212622in}}{\pgfqpoint{3.696000in}{3.696000in}}%
\pgfusepath{clip}%
\pgfsetbuttcap%
\pgfsetroundjoin%
\definecolor{currentfill}{rgb}{0.121569,0.466667,0.705882}%
\pgfsetfillcolor{currentfill}%
\pgfsetfillopacity{0.405163}%
\pgfsetlinewidth{1.003750pt}%
\definecolor{currentstroke}{rgb}{0.121569,0.466667,0.705882}%
\pgfsetstrokecolor{currentstroke}%
\pgfsetstrokeopacity{0.405163}%
\pgfsetdash{}{0pt}%
\pgfpathmoveto{\pgfqpoint{1.472127in}{1.928088in}}%
\pgfpathcurveto{\pgfqpoint{1.480363in}{1.928088in}}{\pgfqpoint{1.488263in}{1.931360in}}{\pgfqpoint{1.494087in}{1.937184in}}%
\pgfpathcurveto{\pgfqpoint{1.499911in}{1.943008in}}{\pgfqpoint{1.503183in}{1.950908in}}{\pgfqpoint{1.503183in}{1.959144in}}%
\pgfpathcurveto{\pgfqpoint{1.503183in}{1.967381in}}{\pgfqpoint{1.499911in}{1.975281in}}{\pgfqpoint{1.494087in}{1.981105in}}%
\pgfpathcurveto{\pgfqpoint{1.488263in}{1.986929in}}{\pgfqpoint{1.480363in}{1.990201in}}{\pgfqpoint{1.472127in}{1.990201in}}%
\pgfpathcurveto{\pgfqpoint{1.463891in}{1.990201in}}{\pgfqpoint{1.455991in}{1.986929in}}{\pgfqpoint{1.450167in}{1.981105in}}%
\pgfpathcurveto{\pgfqpoint{1.444343in}{1.975281in}}{\pgfqpoint{1.441070in}{1.967381in}}{\pgfqpoint{1.441070in}{1.959144in}}%
\pgfpathcurveto{\pgfqpoint{1.441070in}{1.950908in}}{\pgfqpoint{1.444343in}{1.943008in}}{\pgfqpoint{1.450167in}{1.937184in}}%
\pgfpathcurveto{\pgfqpoint{1.455991in}{1.931360in}}{\pgfqpoint{1.463891in}{1.928088in}}{\pgfqpoint{1.472127in}{1.928088in}}%
\pgfpathclose%
\pgfusepath{stroke,fill}%
\end{pgfscope}%
\begin{pgfscope}%
\pgfpathrectangle{\pgfqpoint{0.100000in}{0.212622in}}{\pgfqpoint{3.696000in}{3.696000in}}%
\pgfusepath{clip}%
\pgfsetbuttcap%
\pgfsetroundjoin%
\definecolor{currentfill}{rgb}{0.121569,0.466667,0.705882}%
\pgfsetfillcolor{currentfill}%
\pgfsetfillopacity{0.405251}%
\pgfsetlinewidth{1.003750pt}%
\definecolor{currentstroke}{rgb}{0.121569,0.466667,0.705882}%
\pgfsetstrokecolor{currentstroke}%
\pgfsetstrokeopacity{0.405251}%
\pgfsetdash{}{0pt}%
\pgfpathmoveto{\pgfqpoint{1.941225in}{2.203809in}}%
\pgfpathcurveto{\pgfqpoint{1.949462in}{2.203809in}}{\pgfqpoint{1.957362in}{2.207081in}}{\pgfqpoint{1.963185in}{2.212905in}}%
\pgfpathcurveto{\pgfqpoint{1.969009in}{2.218729in}}{\pgfqpoint{1.972282in}{2.226629in}}{\pgfqpoint{1.972282in}{2.234866in}}%
\pgfpathcurveto{\pgfqpoint{1.972282in}{2.243102in}}{\pgfqpoint{1.969009in}{2.251002in}}{\pgfqpoint{1.963185in}{2.256826in}}%
\pgfpathcurveto{\pgfqpoint{1.957362in}{2.262650in}}{\pgfqpoint{1.949462in}{2.265922in}}{\pgfqpoint{1.941225in}{2.265922in}}%
\pgfpathcurveto{\pgfqpoint{1.932989in}{2.265922in}}{\pgfqpoint{1.925089in}{2.262650in}}{\pgfqpoint{1.919265in}{2.256826in}}%
\pgfpathcurveto{\pgfqpoint{1.913441in}{2.251002in}}{\pgfqpoint{1.910169in}{2.243102in}}{\pgfqpoint{1.910169in}{2.234866in}}%
\pgfpathcurveto{\pgfqpoint{1.910169in}{2.226629in}}{\pgfqpoint{1.913441in}{2.218729in}}{\pgfqpoint{1.919265in}{2.212905in}}%
\pgfpathcurveto{\pgfqpoint{1.925089in}{2.207081in}}{\pgfqpoint{1.932989in}{2.203809in}}{\pgfqpoint{1.941225in}{2.203809in}}%
\pgfpathclose%
\pgfusepath{stroke,fill}%
\end{pgfscope}%
\begin{pgfscope}%
\pgfpathrectangle{\pgfqpoint{0.100000in}{0.212622in}}{\pgfqpoint{3.696000in}{3.696000in}}%
\pgfusepath{clip}%
\pgfsetbuttcap%
\pgfsetroundjoin%
\definecolor{currentfill}{rgb}{0.121569,0.466667,0.705882}%
\pgfsetfillcolor{currentfill}%
\pgfsetfillopacity{0.407160}%
\pgfsetlinewidth{1.003750pt}%
\definecolor{currentstroke}{rgb}{0.121569,0.466667,0.705882}%
\pgfsetstrokecolor{currentstroke}%
\pgfsetstrokeopacity{0.407160}%
\pgfsetdash{}{0pt}%
\pgfpathmoveto{\pgfqpoint{1.431298in}{1.895137in}}%
\pgfpathcurveto{\pgfqpoint{1.439534in}{1.895137in}}{\pgfqpoint{1.447434in}{1.898409in}}{\pgfqpoint{1.453258in}{1.904233in}}%
\pgfpathcurveto{\pgfqpoint{1.459082in}{1.910057in}}{\pgfqpoint{1.462354in}{1.917957in}}{\pgfqpoint{1.462354in}{1.926193in}}%
\pgfpathcurveto{\pgfqpoint{1.462354in}{1.934430in}}{\pgfqpoint{1.459082in}{1.942330in}}{\pgfqpoint{1.453258in}{1.948154in}}%
\pgfpathcurveto{\pgfqpoint{1.447434in}{1.953977in}}{\pgfqpoint{1.439534in}{1.957250in}}{\pgfqpoint{1.431298in}{1.957250in}}%
\pgfpathcurveto{\pgfqpoint{1.423061in}{1.957250in}}{\pgfqpoint{1.415161in}{1.953977in}}{\pgfqpoint{1.409337in}{1.948154in}}%
\pgfpathcurveto{\pgfqpoint{1.403513in}{1.942330in}}{\pgfqpoint{1.400241in}{1.934430in}}{\pgfqpoint{1.400241in}{1.926193in}}%
\pgfpathcurveto{\pgfqpoint{1.400241in}{1.917957in}}{\pgfqpoint{1.403513in}{1.910057in}}{\pgfqpoint{1.409337in}{1.904233in}}%
\pgfpathcurveto{\pgfqpoint{1.415161in}{1.898409in}}{\pgfqpoint{1.423061in}{1.895137in}}{\pgfqpoint{1.431298in}{1.895137in}}%
\pgfpathclose%
\pgfusepath{stroke,fill}%
\end{pgfscope}%
\begin{pgfscope}%
\pgfpathrectangle{\pgfqpoint{0.100000in}{0.212622in}}{\pgfqpoint{3.696000in}{3.696000in}}%
\pgfusepath{clip}%
\pgfsetbuttcap%
\pgfsetroundjoin%
\definecolor{currentfill}{rgb}{0.121569,0.466667,0.705882}%
\pgfsetfillcolor{currentfill}%
\pgfsetfillopacity{0.407200}%
\pgfsetlinewidth{1.003750pt}%
\definecolor{currentstroke}{rgb}{0.121569,0.466667,0.705882}%
\pgfsetstrokecolor{currentstroke}%
\pgfsetstrokeopacity{0.407200}%
\pgfsetdash{}{0pt}%
\pgfpathmoveto{\pgfqpoint{1.438504in}{1.902973in}}%
\pgfpathcurveto{\pgfqpoint{1.446741in}{1.902973in}}{\pgfqpoint{1.454641in}{1.906245in}}{\pgfqpoint{1.460465in}{1.912069in}}%
\pgfpathcurveto{\pgfqpoint{1.466288in}{1.917893in}}{\pgfqpoint{1.469561in}{1.925793in}}{\pgfqpoint{1.469561in}{1.934029in}}%
\pgfpathcurveto{\pgfqpoint{1.469561in}{1.942266in}}{\pgfqpoint{1.466288in}{1.950166in}}{\pgfqpoint{1.460465in}{1.955990in}}%
\pgfpathcurveto{\pgfqpoint{1.454641in}{1.961814in}}{\pgfqpoint{1.446741in}{1.965086in}}{\pgfqpoint{1.438504in}{1.965086in}}%
\pgfpathcurveto{\pgfqpoint{1.430268in}{1.965086in}}{\pgfqpoint{1.422368in}{1.961814in}}{\pgfqpoint{1.416544in}{1.955990in}}%
\pgfpathcurveto{\pgfqpoint{1.410720in}{1.950166in}}{\pgfqpoint{1.407448in}{1.942266in}}{\pgfqpoint{1.407448in}{1.934029in}}%
\pgfpathcurveto{\pgfqpoint{1.407448in}{1.925793in}}{\pgfqpoint{1.410720in}{1.917893in}}{\pgfqpoint{1.416544in}{1.912069in}}%
\pgfpathcurveto{\pgfqpoint{1.422368in}{1.906245in}}{\pgfqpoint{1.430268in}{1.902973in}}{\pgfqpoint{1.438504in}{1.902973in}}%
\pgfpathclose%
\pgfusepath{stroke,fill}%
\end{pgfscope}%
\begin{pgfscope}%
\pgfpathrectangle{\pgfqpoint{0.100000in}{0.212622in}}{\pgfqpoint{3.696000in}{3.696000in}}%
\pgfusepath{clip}%
\pgfsetbuttcap%
\pgfsetroundjoin%
\definecolor{currentfill}{rgb}{0.121569,0.466667,0.705882}%
\pgfsetfillcolor{currentfill}%
\pgfsetfillopacity{0.407806}%
\pgfsetlinewidth{1.003750pt}%
\definecolor{currentstroke}{rgb}{0.121569,0.466667,0.705882}%
\pgfsetstrokecolor{currentstroke}%
\pgfsetstrokeopacity{0.407806}%
\pgfsetdash{}{0pt}%
\pgfpathmoveto{\pgfqpoint{1.469563in}{1.921161in}}%
\pgfpathcurveto{\pgfqpoint{1.477799in}{1.921161in}}{\pgfqpoint{1.485699in}{1.924433in}}{\pgfqpoint{1.491523in}{1.930257in}}%
\pgfpathcurveto{\pgfqpoint{1.497347in}{1.936081in}}{\pgfqpoint{1.500620in}{1.943981in}}{\pgfqpoint{1.500620in}{1.952217in}}%
\pgfpathcurveto{\pgfqpoint{1.500620in}{1.960454in}}{\pgfqpoint{1.497347in}{1.968354in}}{\pgfqpoint{1.491523in}{1.974178in}}%
\pgfpathcurveto{\pgfqpoint{1.485699in}{1.980002in}}{\pgfqpoint{1.477799in}{1.983274in}}{\pgfqpoint{1.469563in}{1.983274in}}%
\pgfpathcurveto{\pgfqpoint{1.461327in}{1.983274in}}{\pgfqpoint{1.453427in}{1.980002in}}{\pgfqpoint{1.447603in}{1.974178in}}%
\pgfpathcurveto{\pgfqpoint{1.441779in}{1.968354in}}{\pgfqpoint{1.438507in}{1.960454in}}{\pgfqpoint{1.438507in}{1.952217in}}%
\pgfpathcurveto{\pgfqpoint{1.438507in}{1.943981in}}{\pgfqpoint{1.441779in}{1.936081in}}{\pgfqpoint{1.447603in}{1.930257in}}%
\pgfpathcurveto{\pgfqpoint{1.453427in}{1.924433in}}{\pgfqpoint{1.461327in}{1.921161in}}{\pgfqpoint{1.469563in}{1.921161in}}%
\pgfpathclose%
\pgfusepath{stroke,fill}%
\end{pgfscope}%
\begin{pgfscope}%
\pgfpathrectangle{\pgfqpoint{0.100000in}{0.212622in}}{\pgfqpoint{3.696000in}{3.696000in}}%
\pgfusepath{clip}%
\pgfsetbuttcap%
\pgfsetroundjoin%
\definecolor{currentfill}{rgb}{0.121569,0.466667,0.705882}%
\pgfsetfillcolor{currentfill}%
\pgfsetfillopacity{0.408188}%
\pgfsetlinewidth{1.003750pt}%
\definecolor{currentstroke}{rgb}{0.121569,0.466667,0.705882}%
\pgfsetstrokecolor{currentstroke}%
\pgfsetstrokeopacity{0.408188}%
\pgfsetdash{}{0pt}%
\pgfpathmoveto{\pgfqpoint{1.448248in}{1.907731in}}%
\pgfpathcurveto{\pgfqpoint{1.456485in}{1.907731in}}{\pgfqpoint{1.464385in}{1.911004in}}{\pgfqpoint{1.470209in}{1.916827in}}%
\pgfpathcurveto{\pgfqpoint{1.476033in}{1.922651in}}{\pgfqpoint{1.479305in}{1.930551in}}{\pgfqpoint{1.479305in}{1.938788in}}%
\pgfpathcurveto{\pgfqpoint{1.479305in}{1.947024in}}{\pgfqpoint{1.476033in}{1.954924in}}{\pgfqpoint{1.470209in}{1.960748in}}%
\pgfpathcurveto{\pgfqpoint{1.464385in}{1.966572in}}{\pgfqpoint{1.456485in}{1.969844in}}{\pgfqpoint{1.448248in}{1.969844in}}%
\pgfpathcurveto{\pgfqpoint{1.440012in}{1.969844in}}{\pgfqpoint{1.432112in}{1.966572in}}{\pgfqpoint{1.426288in}{1.960748in}}%
\pgfpathcurveto{\pgfqpoint{1.420464in}{1.954924in}}{\pgfqpoint{1.417192in}{1.947024in}}{\pgfqpoint{1.417192in}{1.938788in}}%
\pgfpathcurveto{\pgfqpoint{1.417192in}{1.930551in}}{\pgfqpoint{1.420464in}{1.922651in}}{\pgfqpoint{1.426288in}{1.916827in}}%
\pgfpathcurveto{\pgfqpoint{1.432112in}{1.911004in}}{\pgfqpoint{1.440012in}{1.907731in}}{\pgfqpoint{1.448248in}{1.907731in}}%
\pgfpathclose%
\pgfusepath{stroke,fill}%
\end{pgfscope}%
\begin{pgfscope}%
\pgfpathrectangle{\pgfqpoint{0.100000in}{0.212622in}}{\pgfqpoint{3.696000in}{3.696000in}}%
\pgfusepath{clip}%
\pgfsetbuttcap%
\pgfsetroundjoin%
\definecolor{currentfill}{rgb}{0.121569,0.466667,0.705882}%
\pgfsetfillcolor{currentfill}%
\pgfsetfillopacity{0.408482}%
\pgfsetlinewidth{1.003750pt}%
\definecolor{currentstroke}{rgb}{0.121569,0.466667,0.705882}%
\pgfsetstrokecolor{currentstroke}%
\pgfsetstrokeopacity{0.408482}%
\pgfsetdash{}{0pt}%
\pgfpathmoveto{\pgfqpoint{2.149497in}{2.304798in}}%
\pgfpathcurveto{\pgfqpoint{2.157733in}{2.304798in}}{\pgfqpoint{2.165633in}{2.308070in}}{\pgfqpoint{2.171457in}{2.313894in}}%
\pgfpathcurveto{\pgfqpoint{2.177281in}{2.319718in}}{\pgfqpoint{2.180553in}{2.327618in}}{\pgfqpoint{2.180553in}{2.335854in}}%
\pgfpathcurveto{\pgfqpoint{2.180553in}{2.344090in}}{\pgfqpoint{2.177281in}{2.351990in}}{\pgfqpoint{2.171457in}{2.357814in}}%
\pgfpathcurveto{\pgfqpoint{2.165633in}{2.363638in}}{\pgfqpoint{2.157733in}{2.366911in}}{\pgfqpoint{2.149497in}{2.366911in}}%
\pgfpathcurveto{\pgfqpoint{2.141261in}{2.366911in}}{\pgfqpoint{2.133361in}{2.363638in}}{\pgfqpoint{2.127537in}{2.357814in}}%
\pgfpathcurveto{\pgfqpoint{2.121713in}{2.351990in}}{\pgfqpoint{2.118440in}{2.344090in}}{\pgfqpoint{2.118440in}{2.335854in}}%
\pgfpathcurveto{\pgfqpoint{2.118440in}{2.327618in}}{\pgfqpoint{2.121713in}{2.319718in}}{\pgfqpoint{2.127537in}{2.313894in}}%
\pgfpathcurveto{\pgfqpoint{2.133361in}{2.308070in}}{\pgfqpoint{2.141261in}{2.304798in}}{\pgfqpoint{2.149497in}{2.304798in}}%
\pgfpathclose%
\pgfusepath{stroke,fill}%
\end{pgfscope}%
\begin{pgfscope}%
\pgfpathrectangle{\pgfqpoint{0.100000in}{0.212622in}}{\pgfqpoint{3.696000in}{3.696000in}}%
\pgfusepath{clip}%
\pgfsetbuttcap%
\pgfsetroundjoin%
\definecolor{currentfill}{rgb}{0.121569,0.466667,0.705882}%
\pgfsetfillcolor{currentfill}%
\pgfsetfillopacity{0.408497}%
\pgfsetlinewidth{1.003750pt}%
\definecolor{currentstroke}{rgb}{0.121569,0.466667,0.705882}%
\pgfsetstrokecolor{currentstroke}%
\pgfsetstrokeopacity{0.408497}%
\pgfsetdash{}{0pt}%
\pgfpathmoveto{\pgfqpoint{1.442460in}{1.905470in}}%
\pgfpathcurveto{\pgfqpoint{1.450696in}{1.905470in}}{\pgfqpoint{1.458596in}{1.908742in}}{\pgfqpoint{1.464420in}{1.914566in}}%
\pgfpathcurveto{\pgfqpoint{1.470244in}{1.920390in}}{\pgfqpoint{1.473516in}{1.928290in}}{\pgfqpoint{1.473516in}{1.936526in}}%
\pgfpathcurveto{\pgfqpoint{1.473516in}{1.944763in}}{\pgfqpoint{1.470244in}{1.952663in}}{\pgfqpoint{1.464420in}{1.958487in}}%
\pgfpathcurveto{\pgfqpoint{1.458596in}{1.964310in}}{\pgfqpoint{1.450696in}{1.967583in}}{\pgfqpoint{1.442460in}{1.967583in}}%
\pgfpathcurveto{\pgfqpoint{1.434224in}{1.967583in}}{\pgfqpoint{1.426324in}{1.964310in}}{\pgfqpoint{1.420500in}{1.958487in}}%
\pgfpathcurveto{\pgfqpoint{1.414676in}{1.952663in}}{\pgfqpoint{1.411403in}{1.944763in}}{\pgfqpoint{1.411403in}{1.936526in}}%
\pgfpathcurveto{\pgfqpoint{1.411403in}{1.928290in}}{\pgfqpoint{1.414676in}{1.920390in}}{\pgfqpoint{1.420500in}{1.914566in}}%
\pgfpathcurveto{\pgfqpoint{1.426324in}{1.908742in}}{\pgfqpoint{1.434224in}{1.905470in}}{\pgfqpoint{1.442460in}{1.905470in}}%
\pgfpathclose%
\pgfusepath{stroke,fill}%
\end{pgfscope}%
\begin{pgfscope}%
\pgfpathrectangle{\pgfqpoint{0.100000in}{0.212622in}}{\pgfqpoint{3.696000in}{3.696000in}}%
\pgfusepath{clip}%
\pgfsetbuttcap%
\pgfsetroundjoin%
\definecolor{currentfill}{rgb}{0.121569,0.466667,0.705882}%
\pgfsetfillcolor{currentfill}%
\pgfsetfillopacity{0.408504}%
\pgfsetlinewidth{1.003750pt}%
\definecolor{currentstroke}{rgb}{0.121569,0.466667,0.705882}%
\pgfsetstrokecolor{currentstroke}%
\pgfsetstrokeopacity{0.408504}%
\pgfsetdash{}{0pt}%
\pgfpathmoveto{\pgfqpoint{2.030461in}{2.239995in}}%
\pgfpathcurveto{\pgfqpoint{2.038698in}{2.239995in}}{\pgfqpoint{2.046598in}{2.243268in}}{\pgfqpoint{2.052422in}{2.249092in}}%
\pgfpathcurveto{\pgfqpoint{2.058245in}{2.254915in}}{\pgfqpoint{2.061518in}{2.262816in}}{\pgfqpoint{2.061518in}{2.271052in}}%
\pgfpathcurveto{\pgfqpoint{2.061518in}{2.279288in}}{\pgfqpoint{2.058245in}{2.287188in}}{\pgfqpoint{2.052422in}{2.293012in}}%
\pgfpathcurveto{\pgfqpoint{2.046598in}{2.298836in}}{\pgfqpoint{2.038698in}{2.302108in}}{\pgfqpoint{2.030461in}{2.302108in}}%
\pgfpathcurveto{\pgfqpoint{2.022225in}{2.302108in}}{\pgfqpoint{2.014325in}{2.298836in}}{\pgfqpoint{2.008501in}{2.293012in}}%
\pgfpathcurveto{\pgfqpoint{2.002677in}{2.287188in}}{\pgfqpoint{1.999405in}{2.279288in}}{\pgfqpoint{1.999405in}{2.271052in}}%
\pgfpathcurveto{\pgfqpoint{1.999405in}{2.262816in}}{\pgfqpoint{2.002677in}{2.254915in}}{\pgfqpoint{2.008501in}{2.249092in}}%
\pgfpathcurveto{\pgfqpoint{2.014325in}{2.243268in}}{\pgfqpoint{2.022225in}{2.239995in}}{\pgfqpoint{2.030461in}{2.239995in}}%
\pgfpathclose%
\pgfusepath{stroke,fill}%
\end{pgfscope}%
\begin{pgfscope}%
\pgfpathrectangle{\pgfqpoint{0.100000in}{0.212622in}}{\pgfqpoint{3.696000in}{3.696000in}}%
\pgfusepath{clip}%
\pgfsetbuttcap%
\pgfsetroundjoin%
\definecolor{currentfill}{rgb}{0.121569,0.466667,0.705882}%
\pgfsetfillcolor{currentfill}%
\pgfsetfillopacity{0.409462}%
\pgfsetlinewidth{1.003750pt}%
\definecolor{currentstroke}{rgb}{0.121569,0.466667,0.705882}%
\pgfsetstrokecolor{currentstroke}%
\pgfsetstrokeopacity{0.409462}%
\pgfsetdash{}{0pt}%
\pgfpathmoveto{\pgfqpoint{1.434272in}{1.897756in}}%
\pgfpathcurveto{\pgfqpoint{1.442509in}{1.897756in}}{\pgfqpoint{1.450409in}{1.901029in}}{\pgfqpoint{1.456233in}{1.906853in}}%
\pgfpathcurveto{\pgfqpoint{1.462057in}{1.912676in}}{\pgfqpoint{1.465329in}{1.920577in}}{\pgfqpoint{1.465329in}{1.928813in}}%
\pgfpathcurveto{\pgfqpoint{1.465329in}{1.937049in}}{\pgfqpoint{1.462057in}{1.944949in}}{\pgfqpoint{1.456233in}{1.950773in}}%
\pgfpathcurveto{\pgfqpoint{1.450409in}{1.956597in}}{\pgfqpoint{1.442509in}{1.959869in}}{\pgfqpoint{1.434272in}{1.959869in}}%
\pgfpathcurveto{\pgfqpoint{1.426036in}{1.959869in}}{\pgfqpoint{1.418136in}{1.956597in}}{\pgfqpoint{1.412312in}{1.950773in}}%
\pgfpathcurveto{\pgfqpoint{1.406488in}{1.944949in}}{\pgfqpoint{1.403216in}{1.937049in}}{\pgfqpoint{1.403216in}{1.928813in}}%
\pgfpathcurveto{\pgfqpoint{1.403216in}{1.920577in}}{\pgfqpoint{1.406488in}{1.912676in}}{\pgfqpoint{1.412312in}{1.906853in}}%
\pgfpathcurveto{\pgfqpoint{1.418136in}{1.901029in}}{\pgfqpoint{1.426036in}{1.897756in}}{\pgfqpoint{1.434272in}{1.897756in}}%
\pgfpathclose%
\pgfusepath{stroke,fill}%
\end{pgfscope}%
\begin{pgfscope}%
\pgfpathrectangle{\pgfqpoint{0.100000in}{0.212622in}}{\pgfqpoint{3.696000in}{3.696000in}}%
\pgfusepath{clip}%
\pgfsetbuttcap%
\pgfsetroundjoin%
\definecolor{currentfill}{rgb}{0.121569,0.466667,0.705882}%
\pgfsetfillcolor{currentfill}%
\pgfsetfillopacity{0.410557}%
\pgfsetlinewidth{1.003750pt}%
\definecolor{currentstroke}{rgb}{0.121569,0.466667,0.705882}%
\pgfsetstrokecolor{currentstroke}%
\pgfsetstrokeopacity{0.410557}%
\pgfsetdash{}{0pt}%
\pgfpathmoveto{\pgfqpoint{1.445743in}{1.900966in}}%
\pgfpathcurveto{\pgfqpoint{1.453979in}{1.900966in}}{\pgfqpoint{1.461879in}{1.904239in}}{\pgfqpoint{1.467703in}{1.910062in}}%
\pgfpathcurveto{\pgfqpoint{1.473527in}{1.915886in}}{\pgfqpoint{1.476800in}{1.923786in}}{\pgfqpoint{1.476800in}{1.932023in}}%
\pgfpathcurveto{\pgfqpoint{1.476800in}{1.940259in}}{\pgfqpoint{1.473527in}{1.948159in}}{\pgfqpoint{1.467703in}{1.953983in}}%
\pgfpathcurveto{\pgfqpoint{1.461879in}{1.959807in}}{\pgfqpoint{1.453979in}{1.963079in}}{\pgfqpoint{1.445743in}{1.963079in}}%
\pgfpathcurveto{\pgfqpoint{1.437507in}{1.963079in}}{\pgfqpoint{1.429607in}{1.959807in}}{\pgfqpoint{1.423783in}{1.953983in}}%
\pgfpathcurveto{\pgfqpoint{1.417959in}{1.948159in}}{\pgfqpoint{1.414687in}{1.940259in}}{\pgfqpoint{1.414687in}{1.932023in}}%
\pgfpathcurveto{\pgfqpoint{1.414687in}{1.923786in}}{\pgfqpoint{1.417959in}{1.915886in}}{\pgfqpoint{1.423783in}{1.910062in}}%
\pgfpathcurveto{\pgfqpoint{1.429607in}{1.904239in}}{\pgfqpoint{1.437507in}{1.900966in}}{\pgfqpoint{1.445743in}{1.900966in}}%
\pgfpathclose%
\pgfusepath{stroke,fill}%
\end{pgfscope}%
\begin{pgfscope}%
\pgfpathrectangle{\pgfqpoint{0.100000in}{0.212622in}}{\pgfqpoint{3.696000in}{3.696000in}}%
\pgfusepath{clip}%
\pgfsetbuttcap%
\pgfsetroundjoin%
\definecolor{currentfill}{rgb}{0.121569,0.466667,0.705882}%
\pgfsetfillcolor{currentfill}%
\pgfsetfillopacity{0.413766}%
\pgfsetlinewidth{1.003750pt}%
\definecolor{currentstroke}{rgb}{0.121569,0.466667,0.705882}%
\pgfsetstrokecolor{currentstroke}%
\pgfsetstrokeopacity{0.413766}%
\pgfsetdash{}{0pt}%
\pgfpathmoveto{\pgfqpoint{1.923044in}{2.187048in}}%
\pgfpathcurveto{\pgfqpoint{1.931280in}{2.187048in}}{\pgfqpoint{1.939180in}{2.190321in}}{\pgfqpoint{1.945004in}{2.196145in}}%
\pgfpathcurveto{\pgfqpoint{1.950828in}{2.201968in}}{\pgfqpoint{1.954100in}{2.209869in}}{\pgfqpoint{1.954100in}{2.218105in}}%
\pgfpathcurveto{\pgfqpoint{1.954100in}{2.226341in}}{\pgfqpoint{1.950828in}{2.234241in}}{\pgfqpoint{1.945004in}{2.240065in}}%
\pgfpathcurveto{\pgfqpoint{1.939180in}{2.245889in}}{\pgfqpoint{1.931280in}{2.249161in}}{\pgfqpoint{1.923044in}{2.249161in}}%
\pgfpathcurveto{\pgfqpoint{1.914807in}{2.249161in}}{\pgfqpoint{1.906907in}{2.245889in}}{\pgfqpoint{1.901083in}{2.240065in}}%
\pgfpathcurveto{\pgfqpoint{1.895259in}{2.234241in}}{\pgfqpoint{1.891987in}{2.226341in}}{\pgfqpoint{1.891987in}{2.218105in}}%
\pgfpathcurveto{\pgfqpoint{1.891987in}{2.209869in}}{\pgfqpoint{1.895259in}{2.201968in}}{\pgfqpoint{1.901083in}{2.196145in}}%
\pgfpathcurveto{\pgfqpoint{1.906907in}{2.190321in}}{\pgfqpoint{1.914807in}{2.187048in}}{\pgfqpoint{1.923044in}{2.187048in}}%
\pgfpathclose%
\pgfusepath{stroke,fill}%
\end{pgfscope}%
\begin{pgfscope}%
\pgfpathrectangle{\pgfqpoint{0.100000in}{0.212622in}}{\pgfqpoint{3.696000in}{3.696000in}}%
\pgfusepath{clip}%
\pgfsetbuttcap%
\pgfsetroundjoin%
\definecolor{currentfill}{rgb}{0.121569,0.466667,0.705882}%
\pgfsetfillcolor{currentfill}%
\pgfsetfillopacity{0.418116}%
\pgfsetlinewidth{1.003750pt}%
\definecolor{currentstroke}{rgb}{0.121569,0.466667,0.705882}%
\pgfsetstrokecolor{currentstroke}%
\pgfsetstrokeopacity{0.418116}%
\pgfsetdash{}{0pt}%
\pgfpathmoveto{\pgfqpoint{1.915359in}{2.183091in}}%
\pgfpathcurveto{\pgfqpoint{1.923595in}{2.183091in}}{\pgfqpoint{1.931495in}{2.186363in}}{\pgfqpoint{1.937319in}{2.192187in}}%
\pgfpathcurveto{\pgfqpoint{1.943143in}{2.198011in}}{\pgfqpoint{1.946416in}{2.205911in}}{\pgfqpoint{1.946416in}{2.214147in}}%
\pgfpathcurveto{\pgfqpoint{1.946416in}{2.222383in}}{\pgfqpoint{1.943143in}{2.230283in}}{\pgfqpoint{1.937319in}{2.236107in}}%
\pgfpathcurveto{\pgfqpoint{1.931495in}{2.241931in}}{\pgfqpoint{1.923595in}{2.245204in}}{\pgfqpoint{1.915359in}{2.245204in}}%
\pgfpathcurveto{\pgfqpoint{1.907123in}{2.245204in}}{\pgfqpoint{1.899223in}{2.241931in}}{\pgfqpoint{1.893399in}{2.236107in}}%
\pgfpathcurveto{\pgfqpoint{1.887575in}{2.230283in}}{\pgfqpoint{1.884303in}{2.222383in}}{\pgfqpoint{1.884303in}{2.214147in}}%
\pgfpathcurveto{\pgfqpoint{1.884303in}{2.205911in}}{\pgfqpoint{1.887575in}{2.198011in}}{\pgfqpoint{1.893399in}{2.192187in}}%
\pgfpathcurveto{\pgfqpoint{1.899223in}{2.186363in}}{\pgfqpoint{1.907123in}{2.183091in}}{\pgfqpoint{1.915359in}{2.183091in}}%
\pgfpathclose%
\pgfusepath{stroke,fill}%
\end{pgfscope}%
\begin{pgfscope}%
\pgfpathrectangle{\pgfqpoint{0.100000in}{0.212622in}}{\pgfqpoint{3.696000in}{3.696000in}}%
\pgfusepath{clip}%
\pgfsetbuttcap%
\pgfsetroundjoin%
\definecolor{currentfill}{rgb}{0.121569,0.466667,0.705882}%
\pgfsetfillcolor{currentfill}%
\pgfsetfillopacity{0.418667}%
\pgfsetlinewidth{1.003750pt}%
\definecolor{currentstroke}{rgb}{0.121569,0.466667,0.705882}%
\pgfsetstrokecolor{currentstroke}%
\pgfsetstrokeopacity{0.418667}%
\pgfsetdash{}{0pt}%
\pgfpathmoveto{\pgfqpoint{1.996083in}{2.217946in}}%
\pgfpathcurveto{\pgfqpoint{2.004320in}{2.217946in}}{\pgfqpoint{2.012220in}{2.221218in}}{\pgfqpoint{2.018044in}{2.227042in}}%
\pgfpathcurveto{\pgfqpoint{2.023868in}{2.232866in}}{\pgfqpoint{2.027140in}{2.240766in}}{\pgfqpoint{2.027140in}{2.249003in}}%
\pgfpathcurveto{\pgfqpoint{2.027140in}{2.257239in}}{\pgfqpoint{2.023868in}{2.265139in}}{\pgfqpoint{2.018044in}{2.270963in}}%
\pgfpathcurveto{\pgfqpoint{2.012220in}{2.276787in}}{\pgfqpoint{2.004320in}{2.280059in}}{\pgfqpoint{1.996083in}{2.280059in}}%
\pgfpathcurveto{\pgfqpoint{1.987847in}{2.280059in}}{\pgfqpoint{1.979947in}{2.276787in}}{\pgfqpoint{1.974123in}{2.270963in}}%
\pgfpathcurveto{\pgfqpoint{1.968299in}{2.265139in}}{\pgfqpoint{1.965027in}{2.257239in}}{\pgfqpoint{1.965027in}{2.249003in}}%
\pgfpathcurveto{\pgfqpoint{1.965027in}{2.240766in}}{\pgfqpoint{1.968299in}{2.232866in}}{\pgfqpoint{1.974123in}{2.227042in}}%
\pgfpathcurveto{\pgfqpoint{1.979947in}{2.221218in}}{\pgfqpoint{1.987847in}{2.217946in}}{\pgfqpoint{1.996083in}{2.217946in}}%
\pgfpathclose%
\pgfusepath{stroke,fill}%
\end{pgfscope}%
\begin{pgfscope}%
\pgfpathrectangle{\pgfqpoint{0.100000in}{0.212622in}}{\pgfqpoint{3.696000in}{3.696000in}}%
\pgfusepath{clip}%
\pgfsetbuttcap%
\pgfsetroundjoin%
\definecolor{currentfill}{rgb}{0.121569,0.466667,0.705882}%
\pgfsetfillcolor{currentfill}%
\pgfsetfillopacity{0.419556}%
\pgfsetlinewidth{1.003750pt}%
\definecolor{currentstroke}{rgb}{0.121569,0.466667,0.705882}%
\pgfsetstrokecolor{currentstroke}%
\pgfsetstrokeopacity{0.419556}%
\pgfsetdash{}{0pt}%
\pgfpathmoveto{\pgfqpoint{2.145700in}{2.302800in}}%
\pgfpathcurveto{\pgfqpoint{2.153936in}{2.302800in}}{\pgfqpoint{2.161836in}{2.306073in}}{\pgfqpoint{2.167660in}{2.311897in}}%
\pgfpathcurveto{\pgfqpoint{2.173484in}{2.317721in}}{\pgfqpoint{2.176756in}{2.325621in}}{\pgfqpoint{2.176756in}{2.333857in}}%
\pgfpathcurveto{\pgfqpoint{2.176756in}{2.342093in}}{\pgfqpoint{2.173484in}{2.349993in}}{\pgfqpoint{2.167660in}{2.355817in}}%
\pgfpathcurveto{\pgfqpoint{2.161836in}{2.361641in}}{\pgfqpoint{2.153936in}{2.364913in}}{\pgfqpoint{2.145700in}{2.364913in}}%
\pgfpathcurveto{\pgfqpoint{2.137463in}{2.364913in}}{\pgfqpoint{2.129563in}{2.361641in}}{\pgfqpoint{2.123739in}{2.355817in}}%
\pgfpathcurveto{\pgfqpoint{2.117915in}{2.349993in}}{\pgfqpoint{2.114643in}{2.342093in}}{\pgfqpoint{2.114643in}{2.333857in}}%
\pgfpathcurveto{\pgfqpoint{2.114643in}{2.325621in}}{\pgfqpoint{2.117915in}{2.317721in}}{\pgfqpoint{2.123739in}{2.311897in}}%
\pgfpathcurveto{\pgfqpoint{2.129563in}{2.306073in}}{\pgfqpoint{2.137463in}{2.302800in}}{\pgfqpoint{2.145700in}{2.302800in}}%
\pgfpathclose%
\pgfusepath{stroke,fill}%
\end{pgfscope}%
\begin{pgfscope}%
\pgfpathrectangle{\pgfqpoint{0.100000in}{0.212622in}}{\pgfqpoint{3.696000in}{3.696000in}}%
\pgfusepath{clip}%
\pgfsetbuttcap%
\pgfsetroundjoin%
\definecolor{currentfill}{rgb}{0.121569,0.466667,0.705882}%
\pgfsetfillcolor{currentfill}%
\pgfsetfillopacity{0.420594}%
\pgfsetlinewidth{1.003750pt}%
\definecolor{currentstroke}{rgb}{0.121569,0.466667,0.705882}%
\pgfsetstrokecolor{currentstroke}%
\pgfsetstrokeopacity{0.420594}%
\pgfsetdash{}{0pt}%
\pgfpathmoveto{\pgfqpoint{1.940487in}{2.199832in}}%
\pgfpathcurveto{\pgfqpoint{1.948723in}{2.199832in}}{\pgfqpoint{1.956623in}{2.203105in}}{\pgfqpoint{1.962447in}{2.208929in}}%
\pgfpathcurveto{\pgfqpoint{1.968271in}{2.214753in}}{\pgfqpoint{1.971543in}{2.222653in}}{\pgfqpoint{1.971543in}{2.230889in}}%
\pgfpathcurveto{\pgfqpoint{1.971543in}{2.239125in}}{\pgfqpoint{1.968271in}{2.247025in}}{\pgfqpoint{1.962447in}{2.252849in}}%
\pgfpathcurveto{\pgfqpoint{1.956623in}{2.258673in}}{\pgfqpoint{1.948723in}{2.261945in}}{\pgfqpoint{1.940487in}{2.261945in}}%
\pgfpathcurveto{\pgfqpoint{1.932251in}{2.261945in}}{\pgfqpoint{1.924351in}{2.258673in}}{\pgfqpoint{1.918527in}{2.252849in}}%
\pgfpathcurveto{\pgfqpoint{1.912703in}{2.247025in}}{\pgfqpoint{1.909430in}{2.239125in}}{\pgfqpoint{1.909430in}{2.230889in}}%
\pgfpathcurveto{\pgfqpoint{1.909430in}{2.222653in}}{\pgfqpoint{1.912703in}{2.214753in}}{\pgfqpoint{1.918527in}{2.208929in}}%
\pgfpathcurveto{\pgfqpoint{1.924351in}{2.203105in}}{\pgfqpoint{1.932251in}{2.199832in}}{\pgfqpoint{1.940487in}{2.199832in}}%
\pgfpathclose%
\pgfusepath{stroke,fill}%
\end{pgfscope}%
\begin{pgfscope}%
\pgfpathrectangle{\pgfqpoint{0.100000in}{0.212622in}}{\pgfqpoint{3.696000in}{3.696000in}}%
\pgfusepath{clip}%
\pgfsetbuttcap%
\pgfsetroundjoin%
\definecolor{currentfill}{rgb}{0.121569,0.466667,0.705882}%
\pgfsetfillcolor{currentfill}%
\pgfsetfillopacity{0.420621}%
\pgfsetlinewidth{1.003750pt}%
\definecolor{currentstroke}{rgb}{0.121569,0.466667,0.705882}%
\pgfsetstrokecolor{currentstroke}%
\pgfsetstrokeopacity{0.420621}%
\pgfsetdash{}{0pt}%
\pgfpathmoveto{\pgfqpoint{1.910350in}{2.179105in}}%
\pgfpathcurveto{\pgfqpoint{1.918586in}{2.179105in}}{\pgfqpoint{1.926486in}{2.182377in}}{\pgfqpoint{1.932310in}{2.188201in}}%
\pgfpathcurveto{\pgfqpoint{1.938134in}{2.194025in}}{\pgfqpoint{1.941406in}{2.201925in}}{\pgfqpoint{1.941406in}{2.210161in}}%
\pgfpathcurveto{\pgfqpoint{1.941406in}{2.218397in}}{\pgfqpoint{1.938134in}{2.226298in}}{\pgfqpoint{1.932310in}{2.232121in}}%
\pgfpathcurveto{\pgfqpoint{1.926486in}{2.237945in}}{\pgfqpoint{1.918586in}{2.241218in}}{\pgfqpoint{1.910350in}{2.241218in}}%
\pgfpathcurveto{\pgfqpoint{1.902113in}{2.241218in}}{\pgfqpoint{1.894213in}{2.237945in}}{\pgfqpoint{1.888389in}{2.232121in}}%
\pgfpathcurveto{\pgfqpoint{1.882565in}{2.226298in}}{\pgfqpoint{1.879293in}{2.218397in}}{\pgfqpoint{1.879293in}{2.210161in}}%
\pgfpathcurveto{\pgfqpoint{1.879293in}{2.201925in}}{\pgfqpoint{1.882565in}{2.194025in}}{\pgfqpoint{1.888389in}{2.188201in}}%
\pgfpathcurveto{\pgfqpoint{1.894213in}{2.182377in}}{\pgfqpoint{1.902113in}{2.179105in}}{\pgfqpoint{1.910350in}{2.179105in}}%
\pgfpathclose%
\pgfusepath{stroke,fill}%
\end{pgfscope}%
\begin{pgfscope}%
\pgfpathrectangle{\pgfqpoint{0.100000in}{0.212622in}}{\pgfqpoint{3.696000in}{3.696000in}}%
\pgfusepath{clip}%
\pgfsetbuttcap%
\pgfsetroundjoin%
\definecolor{currentfill}{rgb}{0.121569,0.466667,0.705882}%
\pgfsetfillcolor{currentfill}%
\pgfsetfillopacity{0.421334}%
\pgfsetlinewidth{1.003750pt}%
\definecolor{currentstroke}{rgb}{0.121569,0.466667,0.705882}%
\pgfsetstrokecolor{currentstroke}%
\pgfsetstrokeopacity{0.421334}%
\pgfsetdash{}{0pt}%
\pgfpathmoveto{\pgfqpoint{1.984967in}{2.214222in}}%
\pgfpathcurveto{\pgfqpoint{1.993203in}{2.214222in}}{\pgfqpoint{2.001103in}{2.217495in}}{\pgfqpoint{2.006927in}{2.223319in}}%
\pgfpathcurveto{\pgfqpoint{2.012751in}{2.229143in}}{\pgfqpoint{2.016023in}{2.237043in}}{\pgfqpoint{2.016023in}{2.245279in}}%
\pgfpathcurveto{\pgfqpoint{2.016023in}{2.253515in}}{\pgfqpoint{2.012751in}{2.261415in}}{\pgfqpoint{2.006927in}{2.267239in}}%
\pgfpathcurveto{\pgfqpoint{2.001103in}{2.273063in}}{\pgfqpoint{1.993203in}{2.276335in}}{\pgfqpoint{1.984967in}{2.276335in}}%
\pgfpathcurveto{\pgfqpoint{1.976730in}{2.276335in}}{\pgfqpoint{1.968830in}{2.273063in}}{\pgfqpoint{1.963006in}{2.267239in}}%
\pgfpathcurveto{\pgfqpoint{1.957182in}{2.261415in}}{\pgfqpoint{1.953910in}{2.253515in}}{\pgfqpoint{1.953910in}{2.245279in}}%
\pgfpathcurveto{\pgfqpoint{1.953910in}{2.237043in}}{\pgfqpoint{1.957182in}{2.229143in}}{\pgfqpoint{1.963006in}{2.223319in}}%
\pgfpathcurveto{\pgfqpoint{1.968830in}{2.217495in}}{\pgfqpoint{1.976730in}{2.214222in}}{\pgfqpoint{1.984967in}{2.214222in}}%
\pgfpathclose%
\pgfusepath{stroke,fill}%
\end{pgfscope}%
\begin{pgfscope}%
\pgfpathrectangle{\pgfqpoint{0.100000in}{0.212622in}}{\pgfqpoint{3.696000in}{3.696000in}}%
\pgfusepath{clip}%
\pgfsetbuttcap%
\pgfsetroundjoin%
\definecolor{currentfill}{rgb}{0.121569,0.466667,0.705882}%
\pgfsetfillcolor{currentfill}%
\pgfsetfillopacity{0.421813}%
\pgfsetlinewidth{1.003750pt}%
\definecolor{currentstroke}{rgb}{0.121569,0.466667,0.705882}%
\pgfsetstrokecolor{currentstroke}%
\pgfsetstrokeopacity{0.421813}%
\pgfsetdash{}{0pt}%
\pgfpathmoveto{\pgfqpoint{1.913318in}{2.179539in}}%
\pgfpathcurveto{\pgfqpoint{1.921554in}{2.179539in}}{\pgfqpoint{1.929454in}{2.182811in}}{\pgfqpoint{1.935278in}{2.188635in}}%
\pgfpathcurveto{\pgfqpoint{1.941102in}{2.194459in}}{\pgfqpoint{1.944374in}{2.202359in}}{\pgfqpoint{1.944374in}{2.210595in}}%
\pgfpathcurveto{\pgfqpoint{1.944374in}{2.218831in}}{\pgfqpoint{1.941102in}{2.226731in}}{\pgfqpoint{1.935278in}{2.232555in}}%
\pgfpathcurveto{\pgfqpoint{1.929454in}{2.238379in}}{\pgfqpoint{1.921554in}{2.241652in}}{\pgfqpoint{1.913318in}{2.241652in}}%
\pgfpathcurveto{\pgfqpoint{1.905081in}{2.241652in}}{\pgfqpoint{1.897181in}{2.238379in}}{\pgfqpoint{1.891358in}{2.232555in}}%
\pgfpathcurveto{\pgfqpoint{1.885534in}{2.226731in}}{\pgfqpoint{1.882261in}{2.218831in}}{\pgfqpoint{1.882261in}{2.210595in}}%
\pgfpathcurveto{\pgfqpoint{1.882261in}{2.202359in}}{\pgfqpoint{1.885534in}{2.194459in}}{\pgfqpoint{1.891358in}{2.188635in}}%
\pgfpathcurveto{\pgfqpoint{1.897181in}{2.182811in}}{\pgfqpoint{1.905081in}{2.179539in}}{\pgfqpoint{1.913318in}{2.179539in}}%
\pgfpathclose%
\pgfusepath{stroke,fill}%
\end{pgfscope}%
\begin{pgfscope}%
\pgfpathrectangle{\pgfqpoint{0.100000in}{0.212622in}}{\pgfqpoint{3.696000in}{3.696000in}}%
\pgfusepath{clip}%
\pgfsetbuttcap%
\pgfsetroundjoin%
\definecolor{currentfill}{rgb}{0.121569,0.466667,0.705882}%
\pgfsetfillcolor{currentfill}%
\pgfsetfillopacity{0.421955}%
\pgfsetlinewidth{1.003750pt}%
\definecolor{currentstroke}{rgb}{0.121569,0.466667,0.705882}%
\pgfsetstrokecolor{currentstroke}%
\pgfsetstrokeopacity{0.421955}%
\pgfsetdash{}{0pt}%
\pgfpathmoveto{\pgfqpoint{1.909002in}{2.176501in}}%
\pgfpathcurveto{\pgfqpoint{1.917238in}{2.176501in}}{\pgfqpoint{1.925138in}{2.179773in}}{\pgfqpoint{1.930962in}{2.185597in}}%
\pgfpathcurveto{\pgfqpoint{1.936786in}{2.191421in}}{\pgfqpoint{1.940058in}{2.199321in}}{\pgfqpoint{1.940058in}{2.207557in}}%
\pgfpathcurveto{\pgfqpoint{1.940058in}{2.215794in}}{\pgfqpoint{1.936786in}{2.223694in}}{\pgfqpoint{1.930962in}{2.229518in}}%
\pgfpathcurveto{\pgfqpoint{1.925138in}{2.235342in}}{\pgfqpoint{1.917238in}{2.238614in}}{\pgfqpoint{1.909002in}{2.238614in}}%
\pgfpathcurveto{\pgfqpoint{1.900766in}{2.238614in}}{\pgfqpoint{1.892866in}{2.235342in}}{\pgfqpoint{1.887042in}{2.229518in}}%
\pgfpathcurveto{\pgfqpoint{1.881218in}{2.223694in}}{\pgfqpoint{1.877945in}{2.215794in}}{\pgfqpoint{1.877945in}{2.207557in}}%
\pgfpathcurveto{\pgfqpoint{1.877945in}{2.199321in}}{\pgfqpoint{1.881218in}{2.191421in}}{\pgfqpoint{1.887042in}{2.185597in}}%
\pgfpathcurveto{\pgfqpoint{1.892866in}{2.179773in}}{\pgfqpoint{1.900766in}{2.176501in}}{\pgfqpoint{1.909002in}{2.176501in}}%
\pgfpathclose%
\pgfusepath{stroke,fill}%
\end{pgfscope}%
\begin{pgfscope}%
\pgfpathrectangle{\pgfqpoint{0.100000in}{0.212622in}}{\pgfqpoint{3.696000in}{3.696000in}}%
\pgfusepath{clip}%
\pgfsetbuttcap%
\pgfsetroundjoin%
\definecolor{currentfill}{rgb}{0.121569,0.466667,0.705882}%
\pgfsetfillcolor{currentfill}%
\pgfsetfillopacity{0.422334}%
\pgfsetlinewidth{1.003750pt}%
\definecolor{currentstroke}{rgb}{0.121569,0.466667,0.705882}%
\pgfsetstrokecolor{currentstroke}%
\pgfsetstrokeopacity{0.422334}%
\pgfsetdash{}{0pt}%
\pgfpathmoveto{\pgfqpoint{2.127996in}{2.280507in}}%
\pgfpathcurveto{\pgfqpoint{2.136232in}{2.280507in}}{\pgfqpoint{2.144132in}{2.283780in}}{\pgfqpoint{2.149956in}{2.289603in}}%
\pgfpathcurveto{\pgfqpoint{2.155780in}{2.295427in}}{\pgfqpoint{2.159053in}{2.303327in}}{\pgfqpoint{2.159053in}{2.311564in}}%
\pgfpathcurveto{\pgfqpoint{2.159053in}{2.319800in}}{\pgfqpoint{2.155780in}{2.327700in}}{\pgfqpoint{2.149956in}{2.333524in}}%
\pgfpathcurveto{\pgfqpoint{2.144132in}{2.339348in}}{\pgfqpoint{2.136232in}{2.342620in}}{\pgfqpoint{2.127996in}{2.342620in}}%
\pgfpathcurveto{\pgfqpoint{2.119760in}{2.342620in}}{\pgfqpoint{2.111860in}{2.339348in}}{\pgfqpoint{2.106036in}{2.333524in}}%
\pgfpathcurveto{\pgfqpoint{2.100212in}{2.327700in}}{\pgfqpoint{2.096940in}{2.319800in}}{\pgfqpoint{2.096940in}{2.311564in}}%
\pgfpathcurveto{\pgfqpoint{2.096940in}{2.303327in}}{\pgfqpoint{2.100212in}{2.295427in}}{\pgfqpoint{2.106036in}{2.289603in}}%
\pgfpathcurveto{\pgfqpoint{2.111860in}{2.283780in}}{\pgfqpoint{2.119760in}{2.280507in}}{\pgfqpoint{2.127996in}{2.280507in}}%
\pgfpathclose%
\pgfusepath{stroke,fill}%
\end{pgfscope}%
\begin{pgfscope}%
\pgfpathrectangle{\pgfqpoint{0.100000in}{0.212622in}}{\pgfqpoint{3.696000in}{3.696000in}}%
\pgfusepath{clip}%
\pgfsetbuttcap%
\pgfsetroundjoin%
\definecolor{currentfill}{rgb}{0.121569,0.466667,0.705882}%
\pgfsetfillcolor{currentfill}%
\pgfsetfillopacity{0.424783}%
\pgfsetlinewidth{1.003750pt}%
\definecolor{currentstroke}{rgb}{0.121569,0.466667,0.705882}%
\pgfsetstrokecolor{currentstroke}%
\pgfsetstrokeopacity{0.424783}%
\pgfsetdash{}{0pt}%
\pgfpathmoveto{\pgfqpoint{1.913244in}{2.176083in}}%
\pgfpathcurveto{\pgfqpoint{1.921480in}{2.176083in}}{\pgfqpoint{1.929380in}{2.179356in}}{\pgfqpoint{1.935204in}{2.185180in}}%
\pgfpathcurveto{\pgfqpoint{1.941028in}{2.191003in}}{\pgfqpoint{1.944300in}{2.198904in}}{\pgfqpoint{1.944300in}{2.207140in}}%
\pgfpathcurveto{\pgfqpoint{1.944300in}{2.215376in}}{\pgfqpoint{1.941028in}{2.223276in}}{\pgfqpoint{1.935204in}{2.229100in}}%
\pgfpathcurveto{\pgfqpoint{1.929380in}{2.234924in}}{\pgfqpoint{1.921480in}{2.238196in}}{\pgfqpoint{1.913244in}{2.238196in}}%
\pgfpathcurveto{\pgfqpoint{1.905007in}{2.238196in}}{\pgfqpoint{1.897107in}{2.234924in}}{\pgfqpoint{1.891283in}{2.229100in}}%
\pgfpathcurveto{\pgfqpoint{1.885459in}{2.223276in}}{\pgfqpoint{1.882187in}{2.215376in}}{\pgfqpoint{1.882187in}{2.207140in}}%
\pgfpathcurveto{\pgfqpoint{1.882187in}{2.198904in}}{\pgfqpoint{1.885459in}{2.191003in}}{\pgfqpoint{1.891283in}{2.185180in}}%
\pgfpathcurveto{\pgfqpoint{1.897107in}{2.179356in}}{\pgfqpoint{1.905007in}{2.176083in}}{\pgfqpoint{1.913244in}{2.176083in}}%
\pgfpathclose%
\pgfusepath{stroke,fill}%
\end{pgfscope}%
\begin{pgfscope}%
\pgfpathrectangle{\pgfqpoint{0.100000in}{0.212622in}}{\pgfqpoint{3.696000in}{3.696000in}}%
\pgfusepath{clip}%
\pgfsetbuttcap%
\pgfsetroundjoin%
\definecolor{currentfill}{rgb}{0.121569,0.466667,0.705882}%
\pgfsetfillcolor{currentfill}%
\pgfsetfillopacity{0.424880}%
\pgfsetlinewidth{1.003750pt}%
\definecolor{currentstroke}{rgb}{0.121569,0.466667,0.705882}%
\pgfsetstrokecolor{currentstroke}%
\pgfsetstrokeopacity{0.424880}%
\pgfsetdash{}{0pt}%
\pgfpathmoveto{\pgfqpoint{1.907448in}{2.171543in}}%
\pgfpathcurveto{\pgfqpoint{1.915684in}{2.171543in}}{\pgfqpoint{1.923584in}{2.174816in}}{\pgfqpoint{1.929408in}{2.180640in}}%
\pgfpathcurveto{\pgfqpoint{1.935232in}{2.186464in}}{\pgfqpoint{1.938504in}{2.194364in}}{\pgfqpoint{1.938504in}{2.202600in}}%
\pgfpathcurveto{\pgfqpoint{1.938504in}{2.210836in}}{\pgfqpoint{1.935232in}{2.218736in}}{\pgfqpoint{1.929408in}{2.224560in}}%
\pgfpathcurveto{\pgfqpoint{1.923584in}{2.230384in}}{\pgfqpoint{1.915684in}{2.233656in}}{\pgfqpoint{1.907448in}{2.233656in}}%
\pgfpathcurveto{\pgfqpoint{1.899212in}{2.233656in}}{\pgfqpoint{1.891312in}{2.230384in}}{\pgfqpoint{1.885488in}{2.224560in}}%
\pgfpathcurveto{\pgfqpoint{1.879664in}{2.218736in}}{\pgfqpoint{1.876391in}{2.210836in}}{\pgfqpoint{1.876391in}{2.202600in}}%
\pgfpathcurveto{\pgfqpoint{1.876391in}{2.194364in}}{\pgfqpoint{1.879664in}{2.186464in}}{\pgfqpoint{1.885488in}{2.180640in}}%
\pgfpathcurveto{\pgfqpoint{1.891312in}{2.174816in}}{\pgfqpoint{1.899212in}{2.171543in}}{\pgfqpoint{1.907448in}{2.171543in}}%
\pgfpathclose%
\pgfusepath{stroke,fill}%
\end{pgfscope}%
\begin{pgfscope}%
\pgfpathrectangle{\pgfqpoint{0.100000in}{0.212622in}}{\pgfqpoint{3.696000in}{3.696000in}}%
\pgfusepath{clip}%
\pgfsetbuttcap%
\pgfsetroundjoin%
\definecolor{currentfill}{rgb}{0.121569,0.466667,0.705882}%
\pgfsetfillcolor{currentfill}%
\pgfsetfillopacity{0.425245}%
\pgfsetlinewidth{1.003750pt}%
\definecolor{currentstroke}{rgb}{0.121569,0.466667,0.705882}%
\pgfsetstrokecolor{currentstroke}%
\pgfsetstrokeopacity{0.425245}%
\pgfsetdash{}{0pt}%
\pgfpathmoveto{\pgfqpoint{1.917799in}{2.178643in}}%
\pgfpathcurveto{\pgfqpoint{1.926035in}{2.178643in}}{\pgfqpoint{1.933935in}{2.181915in}}{\pgfqpoint{1.939759in}{2.187739in}}%
\pgfpathcurveto{\pgfqpoint{1.945583in}{2.193563in}}{\pgfqpoint{1.948856in}{2.201463in}}{\pgfqpoint{1.948856in}{2.209699in}}%
\pgfpathcurveto{\pgfqpoint{1.948856in}{2.217936in}}{\pgfqpoint{1.945583in}{2.225836in}}{\pgfqpoint{1.939759in}{2.231660in}}%
\pgfpathcurveto{\pgfqpoint{1.933935in}{2.237484in}}{\pgfqpoint{1.926035in}{2.240756in}}{\pgfqpoint{1.917799in}{2.240756in}}%
\pgfpathcurveto{\pgfqpoint{1.909563in}{2.240756in}}{\pgfqpoint{1.901663in}{2.237484in}}{\pgfqpoint{1.895839in}{2.231660in}}%
\pgfpathcurveto{\pgfqpoint{1.890015in}{2.225836in}}{\pgfqpoint{1.886743in}{2.217936in}}{\pgfqpoint{1.886743in}{2.209699in}}%
\pgfpathcurveto{\pgfqpoint{1.886743in}{2.201463in}}{\pgfqpoint{1.890015in}{2.193563in}}{\pgfqpoint{1.895839in}{2.187739in}}%
\pgfpathcurveto{\pgfqpoint{1.901663in}{2.181915in}}{\pgfqpoint{1.909563in}{2.178643in}}{\pgfqpoint{1.917799in}{2.178643in}}%
\pgfpathclose%
\pgfusepath{stroke,fill}%
\end{pgfscope}%
\begin{pgfscope}%
\pgfpathrectangle{\pgfqpoint{0.100000in}{0.212622in}}{\pgfqpoint{3.696000in}{3.696000in}}%
\pgfusepath{clip}%
\pgfsetbuttcap%
\pgfsetroundjoin%
\definecolor{currentfill}{rgb}{0.121569,0.466667,0.705882}%
\pgfsetfillcolor{currentfill}%
\pgfsetfillopacity{0.426084}%
\pgfsetlinewidth{1.003750pt}%
\definecolor{currentstroke}{rgb}{0.121569,0.466667,0.705882}%
\pgfsetstrokecolor{currentstroke}%
\pgfsetstrokeopacity{0.426084}%
\pgfsetdash{}{0pt}%
\pgfpathmoveto{\pgfqpoint{1.967023in}{2.200086in}}%
\pgfpathcurveto{\pgfqpoint{1.975259in}{2.200086in}}{\pgfqpoint{1.983159in}{2.203358in}}{\pgfqpoint{1.988983in}{2.209182in}}%
\pgfpathcurveto{\pgfqpoint{1.994807in}{2.215006in}}{\pgfqpoint{1.998079in}{2.222906in}}{\pgfqpoint{1.998079in}{2.231142in}}%
\pgfpathcurveto{\pgfqpoint{1.998079in}{2.239379in}}{\pgfqpoint{1.994807in}{2.247279in}}{\pgfqpoint{1.988983in}{2.253103in}}%
\pgfpathcurveto{\pgfqpoint{1.983159in}{2.258927in}}{\pgfqpoint{1.975259in}{2.262199in}}{\pgfqpoint{1.967023in}{2.262199in}}%
\pgfpathcurveto{\pgfqpoint{1.958787in}{2.262199in}}{\pgfqpoint{1.950887in}{2.258927in}}{\pgfqpoint{1.945063in}{2.253103in}}%
\pgfpathcurveto{\pgfqpoint{1.939239in}{2.247279in}}{\pgfqpoint{1.935966in}{2.239379in}}{\pgfqpoint{1.935966in}{2.231142in}}%
\pgfpathcurveto{\pgfqpoint{1.935966in}{2.222906in}}{\pgfqpoint{1.939239in}{2.215006in}}{\pgfqpoint{1.945063in}{2.209182in}}%
\pgfpathcurveto{\pgfqpoint{1.950887in}{2.203358in}}{\pgfqpoint{1.958787in}{2.200086in}}{\pgfqpoint{1.967023in}{2.200086in}}%
\pgfpathclose%
\pgfusepath{stroke,fill}%
\end{pgfscope}%
\begin{pgfscope}%
\pgfpathrectangle{\pgfqpoint{0.100000in}{0.212622in}}{\pgfqpoint{3.696000in}{3.696000in}}%
\pgfusepath{clip}%
\pgfsetbuttcap%
\pgfsetroundjoin%
\definecolor{currentfill}{rgb}{0.121569,0.466667,0.705882}%
\pgfsetfillcolor{currentfill}%
\pgfsetfillopacity{0.428649}%
\pgfsetlinewidth{1.003750pt}%
\definecolor{currentstroke}{rgb}{0.121569,0.466667,0.705882}%
\pgfsetstrokecolor{currentstroke}%
\pgfsetstrokeopacity{0.428649}%
\pgfsetdash{}{0pt}%
\pgfpathmoveto{\pgfqpoint{1.958117in}{2.197044in}}%
\pgfpathcurveto{\pgfqpoint{1.966353in}{2.197044in}}{\pgfqpoint{1.974253in}{2.200316in}}{\pgfqpoint{1.980077in}{2.206140in}}%
\pgfpathcurveto{\pgfqpoint{1.985901in}{2.211964in}}{\pgfqpoint{1.989173in}{2.219864in}}{\pgfqpoint{1.989173in}{2.228101in}}%
\pgfpathcurveto{\pgfqpoint{1.989173in}{2.236337in}}{\pgfqpoint{1.985901in}{2.244237in}}{\pgfqpoint{1.980077in}{2.250061in}}%
\pgfpathcurveto{\pgfqpoint{1.974253in}{2.255885in}}{\pgfqpoint{1.966353in}{2.259157in}}{\pgfqpoint{1.958117in}{2.259157in}}%
\pgfpathcurveto{\pgfqpoint{1.949880in}{2.259157in}}{\pgfqpoint{1.941980in}{2.255885in}}{\pgfqpoint{1.936156in}{2.250061in}}%
\pgfpathcurveto{\pgfqpoint{1.930332in}{2.244237in}}{\pgfqpoint{1.927060in}{2.236337in}}{\pgfqpoint{1.927060in}{2.228101in}}%
\pgfpathcurveto{\pgfqpoint{1.927060in}{2.219864in}}{\pgfqpoint{1.930332in}{2.211964in}}{\pgfqpoint{1.936156in}{2.206140in}}%
\pgfpathcurveto{\pgfqpoint{1.941980in}{2.200316in}}{\pgfqpoint{1.949880in}{2.197044in}}{\pgfqpoint{1.958117in}{2.197044in}}%
\pgfpathclose%
\pgfusepath{stroke,fill}%
\end{pgfscope}%
\begin{pgfscope}%
\pgfpathrectangle{\pgfqpoint{0.100000in}{0.212622in}}{\pgfqpoint{3.696000in}{3.696000in}}%
\pgfusepath{clip}%
\pgfsetbuttcap%
\pgfsetroundjoin%
\definecolor{currentfill}{rgb}{0.121569,0.466667,0.705882}%
\pgfsetfillcolor{currentfill}%
\pgfsetfillopacity{0.428830}%
\pgfsetlinewidth{1.003750pt}%
\definecolor{currentstroke}{rgb}{0.121569,0.466667,0.705882}%
\pgfsetstrokecolor{currentstroke}%
\pgfsetstrokeopacity{0.428830}%
\pgfsetdash{}{0pt}%
\pgfpathmoveto{\pgfqpoint{1.923346in}{2.184266in}}%
\pgfpathcurveto{\pgfqpoint{1.931583in}{2.184266in}}{\pgfqpoint{1.939483in}{2.187539in}}{\pgfqpoint{1.945307in}{2.193363in}}%
\pgfpathcurveto{\pgfqpoint{1.951131in}{2.199187in}}{\pgfqpoint{1.954403in}{2.207087in}}{\pgfqpoint{1.954403in}{2.215323in}}%
\pgfpathcurveto{\pgfqpoint{1.954403in}{2.223559in}}{\pgfqpoint{1.951131in}{2.231459in}}{\pgfqpoint{1.945307in}{2.237283in}}%
\pgfpathcurveto{\pgfqpoint{1.939483in}{2.243107in}}{\pgfqpoint{1.931583in}{2.246379in}}{\pgfqpoint{1.923346in}{2.246379in}}%
\pgfpathcurveto{\pgfqpoint{1.915110in}{2.246379in}}{\pgfqpoint{1.907210in}{2.243107in}}{\pgfqpoint{1.901386in}{2.237283in}}%
\pgfpathcurveto{\pgfqpoint{1.895562in}{2.231459in}}{\pgfqpoint{1.892290in}{2.223559in}}{\pgfqpoint{1.892290in}{2.215323in}}%
\pgfpathcurveto{\pgfqpoint{1.892290in}{2.207087in}}{\pgfqpoint{1.895562in}{2.199187in}}{\pgfqpoint{1.901386in}{2.193363in}}%
\pgfpathcurveto{\pgfqpoint{1.907210in}{2.187539in}}{\pgfqpoint{1.915110in}{2.184266in}}{\pgfqpoint{1.923346in}{2.184266in}}%
\pgfpathclose%
\pgfusepath{stroke,fill}%
\end{pgfscope}%
\begin{pgfscope}%
\pgfpathrectangle{\pgfqpoint{0.100000in}{0.212622in}}{\pgfqpoint{3.696000in}{3.696000in}}%
\pgfusepath{clip}%
\pgfsetbuttcap%
\pgfsetroundjoin%
\definecolor{currentfill}{rgb}{0.121569,0.466667,0.705882}%
\pgfsetfillcolor{currentfill}%
\pgfsetfillopacity{0.429358}%
\pgfsetlinewidth{1.003750pt}%
\definecolor{currentstroke}{rgb}{0.121569,0.466667,0.705882}%
\pgfsetstrokecolor{currentstroke}%
\pgfsetstrokeopacity{0.429358}%
\pgfsetdash{}{0pt}%
\pgfpathmoveto{\pgfqpoint{2.121730in}{2.277997in}}%
\pgfpathcurveto{\pgfqpoint{2.129966in}{2.277997in}}{\pgfqpoint{2.137866in}{2.281270in}}{\pgfqpoint{2.143690in}{2.287093in}}%
\pgfpathcurveto{\pgfqpoint{2.149514in}{2.292917in}}{\pgfqpoint{2.152787in}{2.300817in}}{\pgfqpoint{2.152787in}{2.309054in}}%
\pgfpathcurveto{\pgfqpoint{2.152787in}{2.317290in}}{\pgfqpoint{2.149514in}{2.325190in}}{\pgfqpoint{2.143690in}{2.331014in}}%
\pgfpathcurveto{\pgfqpoint{2.137866in}{2.336838in}}{\pgfqpoint{2.129966in}{2.340110in}}{\pgfqpoint{2.121730in}{2.340110in}}%
\pgfpathcurveto{\pgfqpoint{2.113494in}{2.340110in}}{\pgfqpoint{2.105594in}{2.336838in}}{\pgfqpoint{2.099770in}{2.331014in}}%
\pgfpathcurveto{\pgfqpoint{2.093946in}{2.325190in}}{\pgfqpoint{2.090674in}{2.317290in}}{\pgfqpoint{2.090674in}{2.309054in}}%
\pgfpathcurveto{\pgfqpoint{2.090674in}{2.300817in}}{\pgfqpoint{2.093946in}{2.292917in}}{\pgfqpoint{2.099770in}{2.287093in}}%
\pgfpathcurveto{\pgfqpoint{2.105594in}{2.281270in}}{\pgfqpoint{2.113494in}{2.277997in}}{\pgfqpoint{2.121730in}{2.277997in}}%
\pgfpathclose%
\pgfusepath{stroke,fill}%
\end{pgfscope}%
\begin{pgfscope}%
\pgfpathrectangle{\pgfqpoint{0.100000in}{0.212622in}}{\pgfqpoint{3.696000in}{3.696000in}}%
\pgfusepath{clip}%
\pgfsetbuttcap%
\pgfsetroundjoin%
\definecolor{currentfill}{rgb}{0.121569,0.466667,0.705882}%
\pgfsetfillcolor{currentfill}%
\pgfsetfillopacity{0.430074}%
\pgfsetlinewidth{1.003750pt}%
\definecolor{currentstroke}{rgb}{0.121569,0.466667,0.705882}%
\pgfsetstrokecolor{currentstroke}%
\pgfsetstrokeopacity{0.430074}%
\pgfsetdash{}{0pt}%
\pgfpathmoveto{\pgfqpoint{1.942247in}{2.193270in}}%
\pgfpathcurveto{\pgfqpoint{1.950484in}{2.193270in}}{\pgfqpoint{1.958384in}{2.196542in}}{\pgfqpoint{1.964208in}{2.202366in}}%
\pgfpathcurveto{\pgfqpoint{1.970032in}{2.208190in}}{\pgfqpoint{1.973304in}{2.216090in}}{\pgfqpoint{1.973304in}{2.224326in}}%
\pgfpathcurveto{\pgfqpoint{1.973304in}{2.232563in}}{\pgfqpoint{1.970032in}{2.240463in}}{\pgfqpoint{1.964208in}{2.246287in}}%
\pgfpathcurveto{\pgfqpoint{1.958384in}{2.252111in}}{\pgfqpoint{1.950484in}{2.255383in}}{\pgfqpoint{1.942247in}{2.255383in}}%
\pgfpathcurveto{\pgfqpoint{1.934011in}{2.255383in}}{\pgfqpoint{1.926111in}{2.252111in}}{\pgfqpoint{1.920287in}{2.246287in}}%
\pgfpathcurveto{\pgfqpoint{1.914463in}{2.240463in}}{\pgfqpoint{1.911191in}{2.232563in}}{\pgfqpoint{1.911191in}{2.224326in}}%
\pgfpathcurveto{\pgfqpoint{1.911191in}{2.216090in}}{\pgfqpoint{1.914463in}{2.208190in}}{\pgfqpoint{1.920287in}{2.202366in}}%
\pgfpathcurveto{\pgfqpoint{1.926111in}{2.196542in}}{\pgfqpoint{1.934011in}{2.193270in}}{\pgfqpoint{1.942247in}{2.193270in}}%
\pgfpathclose%
\pgfusepath{stroke,fill}%
\end{pgfscope}%
\begin{pgfscope}%
\pgfpathrectangle{\pgfqpoint{0.100000in}{0.212622in}}{\pgfqpoint{3.696000in}{3.696000in}}%
\pgfusepath{clip}%
\pgfsetbuttcap%
\pgfsetroundjoin%
\definecolor{currentfill}{rgb}{0.121569,0.466667,0.705882}%
\pgfsetfillcolor{currentfill}%
\pgfsetfillopacity{0.430287}%
\pgfsetlinewidth{1.003750pt}%
\definecolor{currentstroke}{rgb}{0.121569,0.466667,0.705882}%
\pgfsetstrokecolor{currentstroke}%
\pgfsetstrokeopacity{0.430287}%
\pgfsetdash{}{0pt}%
\pgfpathmoveto{\pgfqpoint{2.123883in}{2.281981in}}%
\pgfpathcurveto{\pgfqpoint{2.132119in}{2.281981in}}{\pgfqpoint{2.140019in}{2.285253in}}{\pgfqpoint{2.145843in}{2.291077in}}%
\pgfpathcurveto{\pgfqpoint{2.151667in}{2.296901in}}{\pgfqpoint{2.154940in}{2.304801in}}{\pgfqpoint{2.154940in}{2.313037in}}%
\pgfpathcurveto{\pgfqpoint{2.154940in}{2.321273in}}{\pgfqpoint{2.151667in}{2.329173in}}{\pgfqpoint{2.145843in}{2.334997in}}%
\pgfpathcurveto{\pgfqpoint{2.140019in}{2.340821in}}{\pgfqpoint{2.132119in}{2.344094in}}{\pgfqpoint{2.123883in}{2.344094in}}%
\pgfpathcurveto{\pgfqpoint{2.115647in}{2.344094in}}{\pgfqpoint{2.107747in}{2.340821in}}{\pgfqpoint{2.101923in}{2.334997in}}%
\pgfpathcurveto{\pgfqpoint{2.096099in}{2.329173in}}{\pgfqpoint{2.092827in}{2.321273in}}{\pgfqpoint{2.092827in}{2.313037in}}%
\pgfpathcurveto{\pgfqpoint{2.092827in}{2.304801in}}{\pgfqpoint{2.096099in}{2.296901in}}{\pgfqpoint{2.101923in}{2.291077in}}%
\pgfpathcurveto{\pgfqpoint{2.107747in}{2.285253in}}{\pgfqpoint{2.115647in}{2.281981in}}{\pgfqpoint{2.123883in}{2.281981in}}%
\pgfpathclose%
\pgfusepath{stroke,fill}%
\end{pgfscope}%
\begin{pgfscope}%
\pgfpathrectangle{\pgfqpoint{0.100000in}{0.212622in}}{\pgfqpoint{3.696000in}{3.696000in}}%
\pgfusepath{clip}%
\pgfsetbuttcap%
\pgfsetroundjoin%
\definecolor{currentfill}{rgb}{0.121569,0.466667,0.705882}%
\pgfsetfillcolor{currentfill}%
\pgfsetfillopacity{0.430518}%
\pgfsetlinewidth{1.003750pt}%
\definecolor{currentstroke}{rgb}{0.121569,0.466667,0.705882}%
\pgfsetstrokecolor{currentstroke}%
\pgfsetstrokeopacity{0.430518}%
\pgfsetdash{}{0pt}%
\pgfpathmoveto{\pgfqpoint{1.948396in}{2.193644in}}%
\pgfpathcurveto{\pgfqpoint{1.956633in}{2.193644in}}{\pgfqpoint{1.964533in}{2.196916in}}{\pgfqpoint{1.970357in}{2.202740in}}%
\pgfpathcurveto{\pgfqpoint{1.976181in}{2.208564in}}{\pgfqpoint{1.979453in}{2.216464in}}{\pgfqpoint{1.979453in}{2.224701in}}%
\pgfpathcurveto{\pgfqpoint{1.979453in}{2.232937in}}{\pgfqpoint{1.976181in}{2.240837in}}{\pgfqpoint{1.970357in}{2.246661in}}%
\pgfpathcurveto{\pgfqpoint{1.964533in}{2.252485in}}{\pgfqpoint{1.956633in}{2.255757in}}{\pgfqpoint{1.948396in}{2.255757in}}%
\pgfpathcurveto{\pgfqpoint{1.940160in}{2.255757in}}{\pgfqpoint{1.932260in}{2.252485in}}{\pgfqpoint{1.926436in}{2.246661in}}%
\pgfpathcurveto{\pgfqpoint{1.920612in}{2.240837in}}{\pgfqpoint{1.917340in}{2.232937in}}{\pgfqpoint{1.917340in}{2.224701in}}%
\pgfpathcurveto{\pgfqpoint{1.917340in}{2.216464in}}{\pgfqpoint{1.920612in}{2.208564in}}{\pgfqpoint{1.926436in}{2.202740in}}%
\pgfpathcurveto{\pgfqpoint{1.932260in}{2.196916in}}{\pgfqpoint{1.940160in}{2.193644in}}{\pgfqpoint{1.948396in}{2.193644in}}%
\pgfpathclose%
\pgfusepath{stroke,fill}%
\end{pgfscope}%
\begin{pgfscope}%
\pgfpathrectangle{\pgfqpoint{0.100000in}{0.212622in}}{\pgfqpoint{3.696000in}{3.696000in}}%
\pgfusepath{clip}%
\pgfsetbuttcap%
\pgfsetroundjoin%
\definecolor{currentfill}{rgb}{0.121569,0.466667,0.705882}%
\pgfsetfillcolor{currentfill}%
\pgfsetfillopacity{0.431286}%
\pgfsetlinewidth{1.003750pt}%
\definecolor{currentstroke}{rgb}{0.121569,0.466667,0.705882}%
\pgfsetstrokecolor{currentstroke}%
\pgfsetstrokeopacity{0.431286}%
\pgfsetdash{}{0pt}%
\pgfpathmoveto{\pgfqpoint{1.947894in}{2.188145in}}%
\pgfpathcurveto{\pgfqpoint{1.956130in}{2.188145in}}{\pgfqpoint{1.964030in}{2.191417in}}{\pgfqpoint{1.969854in}{2.197241in}}%
\pgfpathcurveto{\pgfqpoint{1.975678in}{2.203065in}}{\pgfqpoint{1.978950in}{2.210965in}}{\pgfqpoint{1.978950in}{2.219202in}}%
\pgfpathcurveto{\pgfqpoint{1.978950in}{2.227438in}}{\pgfqpoint{1.975678in}{2.235338in}}{\pgfqpoint{1.969854in}{2.241162in}}%
\pgfpathcurveto{\pgfqpoint{1.964030in}{2.246986in}}{\pgfqpoint{1.956130in}{2.250258in}}{\pgfqpoint{1.947894in}{2.250258in}}%
\pgfpathcurveto{\pgfqpoint{1.939657in}{2.250258in}}{\pgfqpoint{1.931757in}{2.246986in}}{\pgfqpoint{1.925933in}{2.241162in}}%
\pgfpathcurveto{\pgfqpoint{1.920109in}{2.235338in}}{\pgfqpoint{1.916837in}{2.227438in}}{\pgfqpoint{1.916837in}{2.219202in}}%
\pgfpathcurveto{\pgfqpoint{1.916837in}{2.210965in}}{\pgfqpoint{1.920109in}{2.203065in}}{\pgfqpoint{1.925933in}{2.197241in}}%
\pgfpathcurveto{\pgfqpoint{1.931757in}{2.191417in}}{\pgfqpoint{1.939657in}{2.188145in}}{\pgfqpoint{1.947894in}{2.188145in}}%
\pgfpathclose%
\pgfusepath{stroke,fill}%
\end{pgfscope}%
\begin{pgfscope}%
\pgfpathrectangle{\pgfqpoint{0.100000in}{0.212622in}}{\pgfqpoint{3.696000in}{3.696000in}}%
\pgfusepath{clip}%
\pgfsetbuttcap%
\pgfsetroundjoin%
\definecolor{currentfill}{rgb}{0.121569,0.466667,0.705882}%
\pgfsetfillcolor{currentfill}%
\pgfsetfillopacity{0.431610}%
\pgfsetlinewidth{1.003750pt}%
\definecolor{currentstroke}{rgb}{0.121569,0.466667,0.705882}%
\pgfsetstrokecolor{currentstroke}%
\pgfsetstrokeopacity{0.431610}%
\pgfsetdash{}{0pt}%
\pgfpathmoveto{\pgfqpoint{2.122427in}{2.281452in}}%
\pgfpathcurveto{\pgfqpoint{2.130663in}{2.281452in}}{\pgfqpoint{2.138563in}{2.284725in}}{\pgfqpoint{2.144387in}{2.290548in}}%
\pgfpathcurveto{\pgfqpoint{2.150211in}{2.296372in}}{\pgfqpoint{2.153483in}{2.304272in}}{\pgfqpoint{2.153483in}{2.312509in}}%
\pgfpathcurveto{\pgfqpoint{2.153483in}{2.320745in}}{\pgfqpoint{2.150211in}{2.328645in}}{\pgfqpoint{2.144387in}{2.334469in}}%
\pgfpathcurveto{\pgfqpoint{2.138563in}{2.340293in}}{\pgfqpoint{2.130663in}{2.343565in}}{\pgfqpoint{2.122427in}{2.343565in}}%
\pgfpathcurveto{\pgfqpoint{2.114191in}{2.343565in}}{\pgfqpoint{2.106291in}{2.340293in}}{\pgfqpoint{2.100467in}{2.334469in}}%
\pgfpathcurveto{\pgfqpoint{2.094643in}{2.328645in}}{\pgfqpoint{2.091370in}{2.320745in}}{\pgfqpoint{2.091370in}{2.312509in}}%
\pgfpathcurveto{\pgfqpoint{2.091370in}{2.304272in}}{\pgfqpoint{2.094643in}{2.296372in}}{\pgfqpoint{2.100467in}{2.290548in}}%
\pgfpathcurveto{\pgfqpoint{2.106291in}{2.284725in}}{\pgfqpoint{2.114191in}{2.281452in}}{\pgfqpoint{2.122427in}{2.281452in}}%
\pgfpathclose%
\pgfusepath{stroke,fill}%
\end{pgfscope}%
\begin{pgfscope}%
\pgfpathrectangle{\pgfqpoint{0.100000in}{0.212622in}}{\pgfqpoint{3.696000in}{3.696000in}}%
\pgfusepath{clip}%
\pgfsetbuttcap%
\pgfsetroundjoin%
\definecolor{currentfill}{rgb}{0.121569,0.466667,0.705882}%
\pgfsetfillcolor{currentfill}%
\pgfsetfillopacity{0.432074}%
\pgfsetlinewidth{1.003750pt}%
\definecolor{currentstroke}{rgb}{0.121569,0.466667,0.705882}%
\pgfsetstrokecolor{currentstroke}%
\pgfsetstrokeopacity{0.432074}%
\pgfsetdash{}{0pt}%
\pgfpathmoveto{\pgfqpoint{1.930808in}{2.184390in}}%
\pgfpathcurveto{\pgfqpoint{1.939044in}{2.184390in}}{\pgfqpoint{1.946944in}{2.187662in}}{\pgfqpoint{1.952768in}{2.193486in}}%
\pgfpathcurveto{\pgfqpoint{1.958592in}{2.199310in}}{\pgfqpoint{1.961864in}{2.207210in}}{\pgfqpoint{1.961864in}{2.215446in}}%
\pgfpathcurveto{\pgfqpoint{1.961864in}{2.223683in}}{\pgfqpoint{1.958592in}{2.231583in}}{\pgfqpoint{1.952768in}{2.237406in}}%
\pgfpathcurveto{\pgfqpoint{1.946944in}{2.243230in}}{\pgfqpoint{1.939044in}{2.246503in}}{\pgfqpoint{1.930808in}{2.246503in}}%
\pgfpathcurveto{\pgfqpoint{1.922571in}{2.246503in}}{\pgfqpoint{1.914671in}{2.243230in}}{\pgfqpoint{1.908847in}{2.237406in}}%
\pgfpathcurveto{\pgfqpoint{1.903023in}{2.231583in}}{\pgfqpoint{1.899751in}{2.223683in}}{\pgfqpoint{1.899751in}{2.215446in}}%
\pgfpathcurveto{\pgfqpoint{1.899751in}{2.207210in}}{\pgfqpoint{1.903023in}{2.199310in}}{\pgfqpoint{1.908847in}{2.193486in}}%
\pgfpathcurveto{\pgfqpoint{1.914671in}{2.187662in}}{\pgfqpoint{1.922571in}{2.184390in}}{\pgfqpoint{1.930808in}{2.184390in}}%
\pgfpathclose%
\pgfusepath{stroke,fill}%
\end{pgfscope}%
\begin{pgfscope}%
\pgfpathrectangle{\pgfqpoint{0.100000in}{0.212622in}}{\pgfqpoint{3.696000in}{3.696000in}}%
\pgfusepath{clip}%
\pgfsetbuttcap%
\pgfsetroundjoin%
\definecolor{currentfill}{rgb}{0.121569,0.466667,0.705882}%
\pgfsetfillcolor{currentfill}%
\pgfsetfillopacity{0.433548}%
\pgfsetlinewidth{1.003750pt}%
\definecolor{currentstroke}{rgb}{0.121569,0.466667,0.705882}%
\pgfsetstrokecolor{currentstroke}%
\pgfsetstrokeopacity{0.433548}%
\pgfsetdash{}{0pt}%
\pgfpathmoveto{\pgfqpoint{1.935028in}{2.180310in}}%
\pgfpathcurveto{\pgfqpoint{1.943264in}{2.180310in}}{\pgfqpoint{1.951164in}{2.183582in}}{\pgfqpoint{1.956988in}{2.189406in}}%
\pgfpathcurveto{\pgfqpoint{1.962812in}{2.195230in}}{\pgfqpoint{1.966084in}{2.203130in}}{\pgfqpoint{1.966084in}{2.211367in}}%
\pgfpathcurveto{\pgfqpoint{1.966084in}{2.219603in}}{\pgfqpoint{1.962812in}{2.227503in}}{\pgfqpoint{1.956988in}{2.233327in}}%
\pgfpathcurveto{\pgfqpoint{1.951164in}{2.239151in}}{\pgfqpoint{1.943264in}{2.242423in}}{\pgfqpoint{1.935028in}{2.242423in}}%
\pgfpathcurveto{\pgfqpoint{1.926792in}{2.242423in}}{\pgfqpoint{1.918891in}{2.239151in}}{\pgfqpoint{1.913068in}{2.233327in}}%
\pgfpathcurveto{\pgfqpoint{1.907244in}{2.227503in}}{\pgfqpoint{1.903971in}{2.219603in}}{\pgfqpoint{1.903971in}{2.211367in}}%
\pgfpathcurveto{\pgfqpoint{1.903971in}{2.203130in}}{\pgfqpoint{1.907244in}{2.195230in}}{\pgfqpoint{1.913068in}{2.189406in}}%
\pgfpathcurveto{\pgfqpoint{1.918891in}{2.183582in}}{\pgfqpoint{1.926792in}{2.180310in}}{\pgfqpoint{1.935028in}{2.180310in}}%
\pgfpathclose%
\pgfusepath{stroke,fill}%
\end{pgfscope}%
\begin{pgfscope}%
\pgfpathrectangle{\pgfqpoint{0.100000in}{0.212622in}}{\pgfqpoint{3.696000in}{3.696000in}}%
\pgfusepath{clip}%
\pgfsetbuttcap%
\pgfsetroundjoin%
\definecolor{currentfill}{rgb}{0.121569,0.466667,0.705882}%
\pgfsetfillcolor{currentfill}%
\pgfsetfillopacity{0.433812}%
\pgfsetlinewidth{1.003750pt}%
\definecolor{currentstroke}{rgb}{0.121569,0.466667,0.705882}%
\pgfsetstrokecolor{currentstroke}%
\pgfsetstrokeopacity{0.433812}%
\pgfsetdash{}{0pt}%
\pgfpathmoveto{\pgfqpoint{2.119869in}{2.277853in}}%
\pgfpathcurveto{\pgfqpoint{2.128105in}{2.277853in}}{\pgfqpoint{2.136005in}{2.281125in}}{\pgfqpoint{2.141829in}{2.286949in}}%
\pgfpathcurveto{\pgfqpoint{2.147653in}{2.292773in}}{\pgfqpoint{2.150925in}{2.300673in}}{\pgfqpoint{2.150925in}{2.308909in}}%
\pgfpathcurveto{\pgfqpoint{2.150925in}{2.317146in}}{\pgfqpoint{2.147653in}{2.325046in}}{\pgfqpoint{2.141829in}{2.330870in}}%
\pgfpathcurveto{\pgfqpoint{2.136005in}{2.336694in}}{\pgfqpoint{2.128105in}{2.339966in}}{\pgfqpoint{2.119869in}{2.339966in}}%
\pgfpathcurveto{\pgfqpoint{2.111632in}{2.339966in}}{\pgfqpoint{2.103732in}{2.336694in}}{\pgfqpoint{2.097908in}{2.330870in}}%
\pgfpathcurveto{\pgfqpoint{2.092084in}{2.325046in}}{\pgfqpoint{2.088812in}{2.317146in}}{\pgfqpoint{2.088812in}{2.308909in}}%
\pgfpathcurveto{\pgfqpoint{2.088812in}{2.300673in}}{\pgfqpoint{2.092084in}{2.292773in}}{\pgfqpoint{2.097908in}{2.286949in}}%
\pgfpathcurveto{\pgfqpoint{2.103732in}{2.281125in}}{\pgfqpoint{2.111632in}{2.277853in}}{\pgfqpoint{2.119869in}{2.277853in}}%
\pgfpathclose%
\pgfusepath{stroke,fill}%
\end{pgfscope}%
\begin{pgfscope}%
\pgfpathrectangle{\pgfqpoint{0.100000in}{0.212622in}}{\pgfqpoint{3.696000in}{3.696000in}}%
\pgfusepath{clip}%
\pgfsetbuttcap%
\pgfsetroundjoin%
\definecolor{currentfill}{rgb}{0.121569,0.466667,0.705882}%
\pgfsetfillcolor{currentfill}%
\pgfsetfillopacity{0.433843}%
\pgfsetlinewidth{1.003750pt}%
\definecolor{currentstroke}{rgb}{0.121569,0.466667,0.705882}%
\pgfsetstrokecolor{currentstroke}%
\pgfsetstrokeopacity{0.433843}%
\pgfsetdash{}{0pt}%
\pgfpathmoveto{\pgfqpoint{2.117865in}{2.275484in}}%
\pgfpathcurveto{\pgfqpoint{2.126102in}{2.275484in}}{\pgfqpoint{2.134002in}{2.278757in}}{\pgfqpoint{2.139826in}{2.284581in}}%
\pgfpathcurveto{\pgfqpoint{2.145649in}{2.290405in}}{\pgfqpoint{2.148922in}{2.298305in}}{\pgfqpoint{2.148922in}{2.306541in}}%
\pgfpathcurveto{\pgfqpoint{2.148922in}{2.314777in}}{\pgfqpoint{2.145649in}{2.322677in}}{\pgfqpoint{2.139826in}{2.328501in}}%
\pgfpathcurveto{\pgfqpoint{2.134002in}{2.334325in}}{\pgfqpoint{2.126102in}{2.337597in}}{\pgfqpoint{2.117865in}{2.337597in}}%
\pgfpathcurveto{\pgfqpoint{2.109629in}{2.337597in}}{\pgfqpoint{2.101729in}{2.334325in}}{\pgfqpoint{2.095905in}{2.328501in}}%
\pgfpathcurveto{\pgfqpoint{2.090081in}{2.322677in}}{\pgfqpoint{2.086809in}{2.314777in}}{\pgfqpoint{2.086809in}{2.306541in}}%
\pgfpathcurveto{\pgfqpoint{2.086809in}{2.298305in}}{\pgfqpoint{2.090081in}{2.290405in}}{\pgfqpoint{2.095905in}{2.284581in}}%
\pgfpathcurveto{\pgfqpoint{2.101729in}{2.278757in}}{\pgfqpoint{2.109629in}{2.275484in}}{\pgfqpoint{2.117865in}{2.275484in}}%
\pgfpathclose%
\pgfusepath{stroke,fill}%
\end{pgfscope}%
\begin{pgfscope}%
\pgfpathrectangle{\pgfqpoint{0.100000in}{0.212622in}}{\pgfqpoint{3.696000in}{3.696000in}}%
\pgfusepath{clip}%
\pgfsetbuttcap%
\pgfsetroundjoin%
\definecolor{currentfill}{rgb}{0.121569,0.466667,0.705882}%
\pgfsetfillcolor{currentfill}%
\pgfsetfillopacity{0.434181}%
\pgfsetlinewidth{1.003750pt}%
\definecolor{currentstroke}{rgb}{0.121569,0.466667,0.705882}%
\pgfsetstrokecolor{currentstroke}%
\pgfsetstrokeopacity{0.434181}%
\pgfsetdash{}{0pt}%
\pgfpathmoveto{\pgfqpoint{1.914341in}{2.171024in}}%
\pgfpathcurveto{\pgfqpoint{1.922577in}{2.171024in}}{\pgfqpoint{1.930477in}{2.174296in}}{\pgfqpoint{1.936301in}{2.180120in}}%
\pgfpathcurveto{\pgfqpoint{1.942125in}{2.185944in}}{\pgfqpoint{1.945397in}{2.193844in}}{\pgfqpoint{1.945397in}{2.202080in}}%
\pgfpathcurveto{\pgfqpoint{1.945397in}{2.210317in}}{\pgfqpoint{1.942125in}{2.218217in}}{\pgfqpoint{1.936301in}{2.224040in}}%
\pgfpathcurveto{\pgfqpoint{1.930477in}{2.229864in}}{\pgfqpoint{1.922577in}{2.233137in}}{\pgfqpoint{1.914341in}{2.233137in}}%
\pgfpathcurveto{\pgfqpoint{1.906104in}{2.233137in}}{\pgfqpoint{1.898204in}{2.229864in}}{\pgfqpoint{1.892380in}{2.224040in}}%
\pgfpathcurveto{\pgfqpoint{1.886556in}{2.218217in}}{\pgfqpoint{1.883284in}{2.210317in}}{\pgfqpoint{1.883284in}{2.202080in}}%
\pgfpathcurveto{\pgfqpoint{1.883284in}{2.193844in}}{\pgfqpoint{1.886556in}{2.185944in}}{\pgfqpoint{1.892380in}{2.180120in}}%
\pgfpathcurveto{\pgfqpoint{1.898204in}{2.174296in}}{\pgfqpoint{1.906104in}{2.171024in}}{\pgfqpoint{1.914341in}{2.171024in}}%
\pgfpathclose%
\pgfusepath{stroke,fill}%
\end{pgfscope}%
\begin{pgfscope}%
\pgfpathrectangle{\pgfqpoint{0.100000in}{0.212622in}}{\pgfqpoint{3.696000in}{3.696000in}}%
\pgfusepath{clip}%
\pgfsetbuttcap%
\pgfsetroundjoin%
\definecolor{currentfill}{rgb}{0.121569,0.466667,0.705882}%
\pgfsetfillcolor{currentfill}%
\pgfsetfillopacity{0.434563}%
\pgfsetlinewidth{1.003750pt}%
\definecolor{currentstroke}{rgb}{0.121569,0.466667,0.705882}%
\pgfsetstrokecolor{currentstroke}%
\pgfsetstrokeopacity{0.434563}%
\pgfsetdash{}{0pt}%
\pgfpathmoveto{\pgfqpoint{2.120461in}{2.278759in}}%
\pgfpathcurveto{\pgfqpoint{2.128698in}{2.278759in}}{\pgfqpoint{2.136598in}{2.282031in}}{\pgfqpoint{2.142422in}{2.287855in}}%
\pgfpathcurveto{\pgfqpoint{2.148245in}{2.293679in}}{\pgfqpoint{2.151518in}{2.301579in}}{\pgfqpoint{2.151518in}{2.309815in}}%
\pgfpathcurveto{\pgfqpoint{2.151518in}{2.318052in}}{\pgfqpoint{2.148245in}{2.325952in}}{\pgfqpoint{2.142422in}{2.331776in}}%
\pgfpathcurveto{\pgfqpoint{2.136598in}{2.337599in}}{\pgfqpoint{2.128698in}{2.340872in}}{\pgfqpoint{2.120461in}{2.340872in}}%
\pgfpathcurveto{\pgfqpoint{2.112225in}{2.340872in}}{\pgfqpoint{2.104325in}{2.337599in}}{\pgfqpoint{2.098501in}{2.331776in}}%
\pgfpathcurveto{\pgfqpoint{2.092677in}{2.325952in}}{\pgfqpoint{2.089405in}{2.318052in}}{\pgfqpoint{2.089405in}{2.309815in}}%
\pgfpathcurveto{\pgfqpoint{2.089405in}{2.301579in}}{\pgfqpoint{2.092677in}{2.293679in}}{\pgfqpoint{2.098501in}{2.287855in}}%
\pgfpathcurveto{\pgfqpoint{2.104325in}{2.282031in}}{\pgfqpoint{2.112225in}{2.278759in}}{\pgfqpoint{2.120461in}{2.278759in}}%
\pgfpathclose%
\pgfusepath{stroke,fill}%
\end{pgfscope}%
\begin{pgfscope}%
\pgfpathrectangle{\pgfqpoint{0.100000in}{0.212622in}}{\pgfqpoint{3.696000in}{3.696000in}}%
\pgfusepath{clip}%
\pgfsetbuttcap%
\pgfsetroundjoin%
\definecolor{currentfill}{rgb}{0.121569,0.466667,0.705882}%
\pgfsetfillcolor{currentfill}%
\pgfsetfillopacity{0.436923}%
\pgfsetlinewidth{1.003750pt}%
\definecolor{currentstroke}{rgb}{0.121569,0.466667,0.705882}%
\pgfsetstrokecolor{currentstroke}%
\pgfsetstrokeopacity{0.436923}%
\pgfsetdash{}{0pt}%
\pgfpathmoveto{\pgfqpoint{2.116856in}{2.274818in}}%
\pgfpathcurveto{\pgfqpoint{2.125093in}{2.274818in}}{\pgfqpoint{2.132993in}{2.278090in}}{\pgfqpoint{2.138817in}{2.283914in}}%
\pgfpathcurveto{\pgfqpoint{2.144641in}{2.289738in}}{\pgfqpoint{2.147913in}{2.297638in}}{\pgfqpoint{2.147913in}{2.305874in}}%
\pgfpathcurveto{\pgfqpoint{2.147913in}{2.314110in}}{\pgfqpoint{2.144641in}{2.322010in}}{\pgfqpoint{2.138817in}{2.327834in}}%
\pgfpathcurveto{\pgfqpoint{2.132993in}{2.333658in}}{\pgfqpoint{2.125093in}{2.336931in}}{\pgfqpoint{2.116856in}{2.336931in}}%
\pgfpathcurveto{\pgfqpoint{2.108620in}{2.336931in}}{\pgfqpoint{2.100720in}{2.333658in}}{\pgfqpoint{2.094896in}{2.327834in}}%
\pgfpathcurveto{\pgfqpoint{2.089072in}{2.322010in}}{\pgfqpoint{2.085800in}{2.314110in}}{\pgfqpoint{2.085800in}{2.305874in}}%
\pgfpathcurveto{\pgfqpoint{2.085800in}{2.297638in}}{\pgfqpoint{2.089072in}{2.289738in}}{\pgfqpoint{2.094896in}{2.283914in}}%
\pgfpathcurveto{\pgfqpoint{2.100720in}{2.278090in}}{\pgfqpoint{2.108620in}{2.274818in}}{\pgfqpoint{2.116856in}{2.274818in}}%
\pgfpathclose%
\pgfusepath{stroke,fill}%
\end{pgfscope}%
\begin{pgfscope}%
\pgfpathrectangle{\pgfqpoint{0.100000in}{0.212622in}}{\pgfqpoint{3.696000in}{3.696000in}}%
\pgfusepath{clip}%
\pgfsetbuttcap%
\pgfsetroundjoin%
\definecolor{currentfill}{rgb}{0.121569,0.466667,0.705882}%
\pgfsetfillcolor{currentfill}%
\pgfsetfillopacity{0.438124}%
\pgfsetlinewidth{1.003750pt}%
\definecolor{currentstroke}{rgb}{0.121569,0.466667,0.705882}%
\pgfsetstrokecolor{currentstroke}%
\pgfsetstrokeopacity{0.438124}%
\pgfsetdash{}{0pt}%
\pgfpathmoveto{\pgfqpoint{2.117269in}{2.272006in}}%
\pgfpathcurveto{\pgfqpoint{2.125506in}{2.272006in}}{\pgfqpoint{2.133406in}{2.275278in}}{\pgfqpoint{2.139230in}{2.281102in}}%
\pgfpathcurveto{\pgfqpoint{2.145054in}{2.286926in}}{\pgfqpoint{2.148326in}{2.294826in}}{\pgfqpoint{2.148326in}{2.303062in}}%
\pgfpathcurveto{\pgfqpoint{2.148326in}{2.311298in}}{\pgfqpoint{2.145054in}{2.319199in}}{\pgfqpoint{2.139230in}{2.325022in}}%
\pgfpathcurveto{\pgfqpoint{2.133406in}{2.330846in}}{\pgfqpoint{2.125506in}{2.334119in}}{\pgfqpoint{2.117269in}{2.334119in}}%
\pgfpathcurveto{\pgfqpoint{2.109033in}{2.334119in}}{\pgfqpoint{2.101133in}{2.330846in}}{\pgfqpoint{2.095309in}{2.325022in}}%
\pgfpathcurveto{\pgfqpoint{2.089485in}{2.319199in}}{\pgfqpoint{2.086213in}{2.311298in}}{\pgfqpoint{2.086213in}{2.303062in}}%
\pgfpathcurveto{\pgfqpoint{2.086213in}{2.294826in}}{\pgfqpoint{2.089485in}{2.286926in}}{\pgfqpoint{2.095309in}{2.281102in}}%
\pgfpathcurveto{\pgfqpoint{2.101133in}{2.275278in}}{\pgfqpoint{2.109033in}{2.272006in}}{\pgfqpoint{2.117269in}{2.272006in}}%
\pgfpathclose%
\pgfusepath{stroke,fill}%
\end{pgfscope}%
\begin{pgfscope}%
\pgfpathrectangle{\pgfqpoint{0.100000in}{0.212622in}}{\pgfqpoint{3.696000in}{3.696000in}}%
\pgfusepath{clip}%
\pgfsetbuttcap%
\pgfsetroundjoin%
\definecolor{currentfill}{rgb}{0.121569,0.466667,0.705882}%
\pgfsetfillcolor{currentfill}%
\pgfsetfillopacity{0.439286}%
\pgfsetlinewidth{1.003750pt}%
\definecolor{currentstroke}{rgb}{0.121569,0.466667,0.705882}%
\pgfsetstrokecolor{currentstroke}%
\pgfsetstrokeopacity{0.439286}%
\pgfsetdash{}{0pt}%
\pgfpathmoveto{\pgfqpoint{2.121314in}{2.277635in}}%
\pgfpathcurveto{\pgfqpoint{2.129550in}{2.277635in}}{\pgfqpoint{2.137450in}{2.280907in}}{\pgfqpoint{2.143274in}{2.286731in}}%
\pgfpathcurveto{\pgfqpoint{2.149098in}{2.292555in}}{\pgfqpoint{2.152370in}{2.300455in}}{\pgfqpoint{2.152370in}{2.308691in}}%
\pgfpathcurveto{\pgfqpoint{2.152370in}{2.316928in}}{\pgfqpoint{2.149098in}{2.324828in}}{\pgfqpoint{2.143274in}{2.330652in}}%
\pgfpathcurveto{\pgfqpoint{2.137450in}{2.336476in}}{\pgfqpoint{2.129550in}{2.339748in}}{\pgfqpoint{2.121314in}{2.339748in}}%
\pgfpathcurveto{\pgfqpoint{2.113078in}{2.339748in}}{\pgfqpoint{2.105178in}{2.336476in}}{\pgfqpoint{2.099354in}{2.330652in}}%
\pgfpathcurveto{\pgfqpoint{2.093530in}{2.324828in}}{\pgfqpoint{2.090257in}{2.316928in}}{\pgfqpoint{2.090257in}{2.308691in}}%
\pgfpathcurveto{\pgfqpoint{2.090257in}{2.300455in}}{\pgfqpoint{2.093530in}{2.292555in}}{\pgfqpoint{2.099354in}{2.286731in}}%
\pgfpathcurveto{\pgfqpoint{2.105178in}{2.280907in}}{\pgfqpoint{2.113078in}{2.277635in}}{\pgfqpoint{2.121314in}{2.277635in}}%
\pgfpathclose%
\pgfusepath{stroke,fill}%
\end{pgfscope}%
\begin{pgfscope}%
\pgfpathrectangle{\pgfqpoint{0.100000in}{0.212622in}}{\pgfqpoint{3.696000in}{3.696000in}}%
\pgfusepath{clip}%
\pgfsetbuttcap%
\pgfsetroundjoin%
\definecolor{currentfill}{rgb}{0.121569,0.466667,0.705882}%
\pgfsetfillcolor{currentfill}%
\pgfsetfillopacity{0.450422}%
\pgfsetlinewidth{1.003750pt}%
\definecolor{currentstroke}{rgb}{0.121569,0.466667,0.705882}%
\pgfsetstrokecolor{currentstroke}%
\pgfsetstrokeopacity{0.450422}%
\pgfsetdash{}{0pt}%
\pgfpathmoveto{\pgfqpoint{2.094833in}{2.252420in}}%
\pgfpathcurveto{\pgfqpoint{2.103069in}{2.252420in}}{\pgfqpoint{2.110969in}{2.255692in}}{\pgfqpoint{2.116793in}{2.261516in}}%
\pgfpathcurveto{\pgfqpoint{2.122617in}{2.267340in}}{\pgfqpoint{2.125889in}{2.275240in}}{\pgfqpoint{2.125889in}{2.283476in}}%
\pgfpathcurveto{\pgfqpoint{2.125889in}{2.291712in}}{\pgfqpoint{2.122617in}{2.299612in}}{\pgfqpoint{2.116793in}{2.305436in}}%
\pgfpathcurveto{\pgfqpoint{2.110969in}{2.311260in}}{\pgfqpoint{2.103069in}{2.314533in}}{\pgfqpoint{2.094833in}{2.314533in}}%
\pgfpathcurveto{\pgfqpoint{2.086597in}{2.314533in}}{\pgfqpoint{2.078697in}{2.311260in}}{\pgfqpoint{2.072873in}{2.305436in}}%
\pgfpathcurveto{\pgfqpoint{2.067049in}{2.299612in}}{\pgfqpoint{2.063776in}{2.291712in}}{\pgfqpoint{2.063776in}{2.283476in}}%
\pgfpathcurveto{\pgfqpoint{2.063776in}{2.275240in}}{\pgfqpoint{2.067049in}{2.267340in}}{\pgfqpoint{2.072873in}{2.261516in}}%
\pgfpathcurveto{\pgfqpoint{2.078697in}{2.255692in}}{\pgfqpoint{2.086597in}{2.252420in}}{\pgfqpoint{2.094833in}{2.252420in}}%
\pgfpathclose%
\pgfusepath{stroke,fill}%
\end{pgfscope}%
\begin{pgfscope}%
\pgfpathrectangle{\pgfqpoint{0.100000in}{0.212622in}}{\pgfqpoint{3.696000in}{3.696000in}}%
\pgfusepath{clip}%
\pgfsetbuttcap%
\pgfsetroundjoin%
\definecolor{currentfill}{rgb}{0.121569,0.466667,0.705882}%
\pgfsetfillcolor{currentfill}%
\pgfsetfillopacity{0.456893}%
\pgfsetlinewidth{1.003750pt}%
\definecolor{currentstroke}{rgb}{0.121569,0.466667,0.705882}%
\pgfsetstrokecolor{currentstroke}%
\pgfsetstrokeopacity{0.456893}%
\pgfsetdash{}{0pt}%
\pgfpathmoveto{\pgfqpoint{2.085938in}{2.247596in}}%
\pgfpathcurveto{\pgfqpoint{2.094174in}{2.247596in}}{\pgfqpoint{2.102074in}{2.250869in}}{\pgfqpoint{2.107898in}{2.256693in}}%
\pgfpathcurveto{\pgfqpoint{2.113722in}{2.262517in}}{\pgfqpoint{2.116994in}{2.270417in}}{\pgfqpoint{2.116994in}{2.278653in}}%
\pgfpathcurveto{\pgfqpoint{2.116994in}{2.286889in}}{\pgfqpoint{2.113722in}{2.294789in}}{\pgfqpoint{2.107898in}{2.300613in}}%
\pgfpathcurveto{\pgfqpoint{2.102074in}{2.306437in}}{\pgfqpoint{2.094174in}{2.309709in}}{\pgfqpoint{2.085938in}{2.309709in}}%
\pgfpathcurveto{\pgfqpoint{2.077701in}{2.309709in}}{\pgfqpoint{2.069801in}{2.306437in}}{\pgfqpoint{2.063977in}{2.300613in}}%
\pgfpathcurveto{\pgfqpoint{2.058153in}{2.294789in}}{\pgfqpoint{2.054881in}{2.286889in}}{\pgfqpoint{2.054881in}{2.278653in}}%
\pgfpathcurveto{\pgfqpoint{2.054881in}{2.270417in}}{\pgfqpoint{2.058153in}{2.262517in}}{\pgfqpoint{2.063977in}{2.256693in}}%
\pgfpathcurveto{\pgfqpoint{2.069801in}{2.250869in}}{\pgfqpoint{2.077701in}{2.247596in}}{\pgfqpoint{2.085938in}{2.247596in}}%
\pgfpathclose%
\pgfusepath{stroke,fill}%
\end{pgfscope}%
\begin{pgfscope}%
\pgfpathrectangle{\pgfqpoint{0.100000in}{0.212622in}}{\pgfqpoint{3.696000in}{3.696000in}}%
\pgfusepath{clip}%
\pgfsetbuttcap%
\pgfsetroundjoin%
\definecolor{currentfill}{rgb}{0.121569,0.466667,0.705882}%
\pgfsetfillcolor{currentfill}%
\pgfsetfillopacity{0.469336}%
\pgfsetlinewidth{1.003750pt}%
\definecolor{currentstroke}{rgb}{0.121569,0.466667,0.705882}%
\pgfsetstrokecolor{currentstroke}%
\pgfsetstrokeopacity{0.469336}%
\pgfsetdash{}{0pt}%
\pgfpathmoveto{\pgfqpoint{2.058255in}{2.222430in}}%
\pgfpathcurveto{\pgfqpoint{2.066491in}{2.222430in}}{\pgfqpoint{2.074391in}{2.225702in}}{\pgfqpoint{2.080215in}{2.231526in}}%
\pgfpathcurveto{\pgfqpoint{2.086039in}{2.237350in}}{\pgfqpoint{2.089311in}{2.245250in}}{\pgfqpoint{2.089311in}{2.253486in}}%
\pgfpathcurveto{\pgfqpoint{2.089311in}{2.261722in}}{\pgfqpoint{2.086039in}{2.269622in}}{\pgfqpoint{2.080215in}{2.275446in}}%
\pgfpathcurveto{\pgfqpoint{2.074391in}{2.281270in}}{\pgfqpoint{2.066491in}{2.284543in}}{\pgfqpoint{2.058255in}{2.284543in}}%
\pgfpathcurveto{\pgfqpoint{2.050018in}{2.284543in}}{\pgfqpoint{2.042118in}{2.281270in}}{\pgfqpoint{2.036294in}{2.275446in}}%
\pgfpathcurveto{\pgfqpoint{2.030470in}{2.269622in}}{\pgfqpoint{2.027198in}{2.261722in}}{\pgfqpoint{2.027198in}{2.253486in}}%
\pgfpathcurveto{\pgfqpoint{2.027198in}{2.245250in}}{\pgfqpoint{2.030470in}{2.237350in}}{\pgfqpoint{2.036294in}{2.231526in}}%
\pgfpathcurveto{\pgfqpoint{2.042118in}{2.225702in}}{\pgfqpoint{2.050018in}{2.222430in}}{\pgfqpoint{2.058255in}{2.222430in}}%
\pgfpathclose%
\pgfusepath{stroke,fill}%
\end{pgfscope}%
\begin{pgfscope}%
\pgfpathrectangle{\pgfqpoint{0.100000in}{0.212622in}}{\pgfqpoint{3.696000in}{3.696000in}}%
\pgfusepath{clip}%
\pgfsetbuttcap%
\pgfsetroundjoin%
\definecolor{currentfill}{rgb}{0.121569,0.466667,0.705882}%
\pgfsetfillcolor{currentfill}%
\pgfsetfillopacity{0.481072}%
\pgfsetlinewidth{1.003750pt}%
\definecolor{currentstroke}{rgb}{0.121569,0.466667,0.705882}%
\pgfsetstrokecolor{currentstroke}%
\pgfsetstrokeopacity{0.481072}%
\pgfsetdash{}{0pt}%
\pgfpathmoveto{\pgfqpoint{2.036016in}{2.205287in}}%
\pgfpathcurveto{\pgfqpoint{2.044252in}{2.205287in}}{\pgfqpoint{2.052152in}{2.208559in}}{\pgfqpoint{2.057976in}{2.214383in}}%
\pgfpathcurveto{\pgfqpoint{2.063800in}{2.220207in}}{\pgfqpoint{2.067072in}{2.228107in}}{\pgfqpoint{2.067072in}{2.236344in}}%
\pgfpathcurveto{\pgfqpoint{2.067072in}{2.244580in}}{\pgfqpoint{2.063800in}{2.252480in}}{\pgfqpoint{2.057976in}{2.258304in}}%
\pgfpathcurveto{\pgfqpoint{2.052152in}{2.264128in}}{\pgfqpoint{2.044252in}{2.267400in}}{\pgfqpoint{2.036016in}{2.267400in}}%
\pgfpathcurveto{\pgfqpoint{2.027779in}{2.267400in}}{\pgfqpoint{2.019879in}{2.264128in}}{\pgfqpoint{2.014055in}{2.258304in}}%
\pgfpathcurveto{\pgfqpoint{2.008231in}{2.252480in}}{\pgfqpoint{2.004959in}{2.244580in}}{\pgfqpoint{2.004959in}{2.236344in}}%
\pgfpathcurveto{\pgfqpoint{2.004959in}{2.228107in}}{\pgfqpoint{2.008231in}{2.220207in}}{\pgfqpoint{2.014055in}{2.214383in}}%
\pgfpathcurveto{\pgfqpoint{2.019879in}{2.208559in}}{\pgfqpoint{2.027779in}{2.205287in}}{\pgfqpoint{2.036016in}{2.205287in}}%
\pgfpathclose%
\pgfusepath{stroke,fill}%
\end{pgfscope}%
\begin{pgfscope}%
\pgfpathrectangle{\pgfqpoint{0.100000in}{0.212622in}}{\pgfqpoint{3.696000in}{3.696000in}}%
\pgfusepath{clip}%
\pgfsetbuttcap%
\pgfsetroundjoin%
\definecolor{currentfill}{rgb}{0.121569,0.466667,0.705882}%
\pgfsetfillcolor{currentfill}%
\pgfsetfillopacity{0.494760}%
\pgfsetlinewidth{1.003750pt}%
\definecolor{currentstroke}{rgb}{0.121569,0.466667,0.705882}%
\pgfsetstrokecolor{currentstroke}%
\pgfsetstrokeopacity{0.494760}%
\pgfsetdash{}{0pt}%
\pgfpathmoveto{\pgfqpoint{2.006545in}{2.175218in}}%
\pgfpathcurveto{\pgfqpoint{2.014781in}{2.175218in}}{\pgfqpoint{2.022681in}{2.178490in}}{\pgfqpoint{2.028505in}{2.184314in}}%
\pgfpathcurveto{\pgfqpoint{2.034329in}{2.190138in}}{\pgfqpoint{2.037601in}{2.198038in}}{\pgfqpoint{2.037601in}{2.206275in}}%
\pgfpathcurveto{\pgfqpoint{2.037601in}{2.214511in}}{\pgfqpoint{2.034329in}{2.222411in}}{\pgfqpoint{2.028505in}{2.228235in}}%
\pgfpathcurveto{\pgfqpoint{2.022681in}{2.234059in}}{\pgfqpoint{2.014781in}{2.237331in}}{\pgfqpoint{2.006545in}{2.237331in}}%
\pgfpathcurveto{\pgfqpoint{1.998308in}{2.237331in}}{\pgfqpoint{1.990408in}{2.234059in}}{\pgfqpoint{1.984584in}{2.228235in}}%
\pgfpathcurveto{\pgfqpoint{1.978760in}{2.222411in}}{\pgfqpoint{1.975488in}{2.214511in}}{\pgfqpoint{1.975488in}{2.206275in}}%
\pgfpathcurveto{\pgfqpoint{1.975488in}{2.198038in}}{\pgfqpoint{1.978760in}{2.190138in}}{\pgfqpoint{1.984584in}{2.184314in}}%
\pgfpathcurveto{\pgfqpoint{1.990408in}{2.178490in}}{\pgfqpoint{1.998308in}{2.175218in}}{\pgfqpoint{2.006545in}{2.175218in}}%
\pgfpathclose%
\pgfusepath{stroke,fill}%
\end{pgfscope}%
\begin{pgfscope}%
\pgfpathrectangle{\pgfqpoint{0.100000in}{0.212622in}}{\pgfqpoint{3.696000in}{3.696000in}}%
\pgfusepath{clip}%
\pgfsetbuttcap%
\pgfsetroundjoin%
\definecolor{currentfill}{rgb}{0.121569,0.466667,0.705882}%
\pgfsetfillcolor{currentfill}%
\pgfsetfillopacity{0.504768}%
\pgfsetlinewidth{1.003750pt}%
\definecolor{currentstroke}{rgb}{0.121569,0.466667,0.705882}%
\pgfsetstrokecolor{currentstroke}%
\pgfsetstrokeopacity{0.504768}%
\pgfsetdash{}{0pt}%
\pgfpathmoveto{\pgfqpoint{2.005273in}{2.161740in}}%
\pgfpathcurveto{\pgfqpoint{2.013510in}{2.161740in}}{\pgfqpoint{2.021410in}{2.165012in}}{\pgfqpoint{2.027234in}{2.170836in}}%
\pgfpathcurveto{\pgfqpoint{2.033058in}{2.176660in}}{\pgfqpoint{2.036330in}{2.184560in}}{\pgfqpoint{2.036330in}{2.192796in}}%
\pgfpathcurveto{\pgfqpoint{2.036330in}{2.201033in}}{\pgfqpoint{2.033058in}{2.208933in}}{\pgfqpoint{2.027234in}{2.214757in}}%
\pgfpathcurveto{\pgfqpoint{2.021410in}{2.220580in}}{\pgfqpoint{2.013510in}{2.223853in}}{\pgfqpoint{2.005273in}{2.223853in}}%
\pgfpathcurveto{\pgfqpoint{1.997037in}{2.223853in}}{\pgfqpoint{1.989137in}{2.220580in}}{\pgfqpoint{1.983313in}{2.214757in}}%
\pgfpathcurveto{\pgfqpoint{1.977489in}{2.208933in}}{\pgfqpoint{1.974217in}{2.201033in}}{\pgfqpoint{1.974217in}{2.192796in}}%
\pgfpathcurveto{\pgfqpoint{1.974217in}{2.184560in}}{\pgfqpoint{1.977489in}{2.176660in}}{\pgfqpoint{1.983313in}{2.170836in}}%
\pgfpathcurveto{\pgfqpoint{1.989137in}{2.165012in}}{\pgfqpoint{1.997037in}{2.161740in}}{\pgfqpoint{2.005273in}{2.161740in}}%
\pgfpathclose%
\pgfusepath{stroke,fill}%
\end{pgfscope}%
\begin{pgfscope}%
\pgfpathrectangle{\pgfqpoint{0.100000in}{0.212622in}}{\pgfqpoint{3.696000in}{3.696000in}}%
\pgfusepath{clip}%
\pgfsetbuttcap%
\pgfsetroundjoin%
\definecolor{currentfill}{rgb}{0.121569,0.466667,0.705882}%
\pgfsetfillcolor{currentfill}%
\pgfsetfillopacity{0.505999}%
\pgfsetlinewidth{1.003750pt}%
\definecolor{currentstroke}{rgb}{0.121569,0.466667,0.705882}%
\pgfsetstrokecolor{currentstroke}%
\pgfsetstrokeopacity{0.505999}%
\pgfsetdash{}{0pt}%
\pgfpathmoveto{\pgfqpoint{2.017080in}{2.158627in}}%
\pgfpathcurveto{\pgfqpoint{2.025316in}{2.158627in}}{\pgfqpoint{2.033216in}{2.161900in}}{\pgfqpoint{2.039040in}{2.167723in}}%
\pgfpathcurveto{\pgfqpoint{2.044864in}{2.173547in}}{\pgfqpoint{2.048136in}{2.181447in}}{\pgfqpoint{2.048136in}{2.189684in}}%
\pgfpathcurveto{\pgfqpoint{2.048136in}{2.197920in}}{\pgfqpoint{2.044864in}{2.205820in}}{\pgfqpoint{2.039040in}{2.211644in}}%
\pgfpathcurveto{\pgfqpoint{2.033216in}{2.217468in}}{\pgfqpoint{2.025316in}{2.220740in}}{\pgfqpoint{2.017080in}{2.220740in}}%
\pgfpathcurveto{\pgfqpoint{2.008843in}{2.220740in}}{\pgfqpoint{2.000943in}{2.217468in}}{\pgfqpoint{1.995119in}{2.211644in}}%
\pgfpathcurveto{\pgfqpoint{1.989295in}{2.205820in}}{\pgfqpoint{1.986023in}{2.197920in}}{\pgfqpoint{1.986023in}{2.189684in}}%
\pgfpathcurveto{\pgfqpoint{1.986023in}{2.181447in}}{\pgfqpoint{1.989295in}{2.173547in}}{\pgfqpoint{1.995119in}{2.167723in}}%
\pgfpathcurveto{\pgfqpoint{2.000943in}{2.161900in}}{\pgfqpoint{2.008843in}{2.158627in}}{\pgfqpoint{2.017080in}{2.158627in}}%
\pgfpathclose%
\pgfusepath{stroke,fill}%
\end{pgfscope}%
\begin{pgfscope}%
\pgfpathrectangle{\pgfqpoint{0.100000in}{0.212622in}}{\pgfqpoint{3.696000in}{3.696000in}}%
\pgfusepath{clip}%
\pgfsetbuttcap%
\pgfsetroundjoin%
\definecolor{currentfill}{rgb}{0.121569,0.466667,0.705882}%
\pgfsetfillcolor{currentfill}%
\pgfsetfillopacity{0.511419}%
\pgfsetlinewidth{1.003750pt}%
\definecolor{currentstroke}{rgb}{0.121569,0.466667,0.705882}%
\pgfsetstrokecolor{currentstroke}%
\pgfsetstrokeopacity{0.511419}%
\pgfsetdash{}{0pt}%
\pgfpathmoveto{\pgfqpoint{1.968713in}{2.134630in}}%
\pgfpathcurveto{\pgfqpoint{1.976950in}{2.134630in}}{\pgfqpoint{1.984850in}{2.137903in}}{\pgfqpoint{1.990674in}{2.143726in}}%
\pgfpathcurveto{\pgfqpoint{1.996498in}{2.149550in}}{\pgfqpoint{1.999770in}{2.157450in}}{\pgfqpoint{1.999770in}{2.165687in}}%
\pgfpathcurveto{\pgfqpoint{1.999770in}{2.173923in}}{\pgfqpoint{1.996498in}{2.181823in}}{\pgfqpoint{1.990674in}{2.187647in}}%
\pgfpathcurveto{\pgfqpoint{1.984850in}{2.193471in}}{\pgfqpoint{1.976950in}{2.196743in}}{\pgfqpoint{1.968713in}{2.196743in}}%
\pgfpathcurveto{\pgfqpoint{1.960477in}{2.196743in}}{\pgfqpoint{1.952577in}{2.193471in}}{\pgfqpoint{1.946753in}{2.187647in}}%
\pgfpathcurveto{\pgfqpoint{1.940929in}{2.181823in}}{\pgfqpoint{1.937657in}{2.173923in}}{\pgfqpoint{1.937657in}{2.165687in}}%
\pgfpathcurveto{\pgfqpoint{1.937657in}{2.157450in}}{\pgfqpoint{1.940929in}{2.149550in}}{\pgfqpoint{1.946753in}{2.143726in}}%
\pgfpathcurveto{\pgfqpoint{1.952577in}{2.137903in}}{\pgfqpoint{1.960477in}{2.134630in}}{\pgfqpoint{1.968713in}{2.134630in}}%
\pgfpathclose%
\pgfusepath{stroke,fill}%
\end{pgfscope}%
\begin{pgfscope}%
\pgfpathrectangle{\pgfqpoint{0.100000in}{0.212622in}}{\pgfqpoint{3.696000in}{3.696000in}}%
\pgfusepath{clip}%
\pgfsetbuttcap%
\pgfsetroundjoin%
\definecolor{currentfill}{rgb}{0.121569,0.466667,0.705882}%
\pgfsetfillcolor{currentfill}%
\pgfsetfillopacity{0.515782}%
\pgfsetlinewidth{1.003750pt}%
\definecolor{currentstroke}{rgb}{0.121569,0.466667,0.705882}%
\pgfsetstrokecolor{currentstroke}%
\pgfsetstrokeopacity{0.515782}%
\pgfsetdash{}{0pt}%
\pgfpathmoveto{\pgfqpoint{2.005921in}{2.137030in}}%
\pgfpathcurveto{\pgfqpoint{2.014157in}{2.137030in}}{\pgfqpoint{2.022057in}{2.140302in}}{\pgfqpoint{2.027881in}{2.146126in}}%
\pgfpathcurveto{\pgfqpoint{2.033705in}{2.151950in}}{\pgfqpoint{2.036977in}{2.159850in}}{\pgfqpoint{2.036977in}{2.168087in}}%
\pgfpathcurveto{\pgfqpoint{2.036977in}{2.176323in}}{\pgfqpoint{2.033705in}{2.184223in}}{\pgfqpoint{2.027881in}{2.190047in}}%
\pgfpathcurveto{\pgfqpoint{2.022057in}{2.195871in}}{\pgfqpoint{2.014157in}{2.199143in}}{\pgfqpoint{2.005921in}{2.199143in}}%
\pgfpathcurveto{\pgfqpoint{1.997685in}{2.199143in}}{\pgfqpoint{1.989785in}{2.195871in}}{\pgfqpoint{1.983961in}{2.190047in}}%
\pgfpathcurveto{\pgfqpoint{1.978137in}{2.184223in}}{\pgfqpoint{1.974864in}{2.176323in}}{\pgfqpoint{1.974864in}{2.168087in}}%
\pgfpathcurveto{\pgfqpoint{1.974864in}{2.159850in}}{\pgfqpoint{1.978137in}{2.151950in}}{\pgfqpoint{1.983961in}{2.146126in}}%
\pgfpathcurveto{\pgfqpoint{1.989785in}{2.140302in}}{\pgfqpoint{1.997685in}{2.137030in}}{\pgfqpoint{2.005921in}{2.137030in}}%
\pgfpathclose%
\pgfusepath{stroke,fill}%
\end{pgfscope}%
\begin{pgfscope}%
\pgfpathrectangle{\pgfqpoint{0.100000in}{0.212622in}}{\pgfqpoint{3.696000in}{3.696000in}}%
\pgfusepath{clip}%
\pgfsetbuttcap%
\pgfsetroundjoin%
\definecolor{currentfill}{rgb}{0.121569,0.466667,0.705882}%
\pgfsetfillcolor{currentfill}%
\pgfsetfillopacity{0.521322}%
\pgfsetlinewidth{1.003750pt}%
\definecolor{currentstroke}{rgb}{0.121569,0.466667,0.705882}%
\pgfsetstrokecolor{currentstroke}%
\pgfsetstrokeopacity{0.521322}%
\pgfsetdash{}{0pt}%
\pgfpathmoveto{\pgfqpoint{2.013796in}{2.142973in}}%
\pgfpathcurveto{\pgfqpoint{2.022033in}{2.142973in}}{\pgfqpoint{2.029933in}{2.146245in}}{\pgfqpoint{2.035757in}{2.152069in}}%
\pgfpathcurveto{\pgfqpoint{2.041581in}{2.157893in}}{\pgfqpoint{2.044853in}{2.165793in}}{\pgfqpoint{2.044853in}{2.174030in}}%
\pgfpathcurveto{\pgfqpoint{2.044853in}{2.182266in}}{\pgfqpoint{2.041581in}{2.190166in}}{\pgfqpoint{2.035757in}{2.195990in}}%
\pgfpathcurveto{\pgfqpoint{2.029933in}{2.201814in}}{\pgfqpoint{2.022033in}{2.205086in}}{\pgfqpoint{2.013796in}{2.205086in}}%
\pgfpathcurveto{\pgfqpoint{2.005560in}{2.205086in}}{\pgfqpoint{1.997660in}{2.201814in}}{\pgfqpoint{1.991836in}{2.195990in}}%
\pgfpathcurveto{\pgfqpoint{1.986012in}{2.190166in}}{\pgfqpoint{1.982740in}{2.182266in}}{\pgfqpoint{1.982740in}{2.174030in}}%
\pgfpathcurveto{\pgfqpoint{1.982740in}{2.165793in}}{\pgfqpoint{1.986012in}{2.157893in}}{\pgfqpoint{1.991836in}{2.152069in}}%
\pgfpathcurveto{\pgfqpoint{1.997660in}{2.146245in}}{\pgfqpoint{2.005560in}{2.142973in}}{\pgfqpoint{2.013796in}{2.142973in}}%
\pgfpathclose%
\pgfusepath{stroke,fill}%
\end{pgfscope}%
\begin{pgfscope}%
\pgfpathrectangle{\pgfqpoint{0.100000in}{0.212622in}}{\pgfqpoint{3.696000in}{3.696000in}}%
\pgfusepath{clip}%
\pgfsetbuttcap%
\pgfsetroundjoin%
\definecolor{currentfill}{rgb}{0.121569,0.466667,0.705882}%
\pgfsetfillcolor{currentfill}%
\pgfsetfillopacity{0.535801}%
\pgfsetlinewidth{1.003750pt}%
\definecolor{currentstroke}{rgb}{0.121569,0.466667,0.705882}%
\pgfsetstrokecolor{currentstroke}%
\pgfsetstrokeopacity{0.535801}%
\pgfsetdash{}{0pt}%
\pgfpathmoveto{\pgfqpoint{2.008501in}{2.099546in}}%
\pgfpathcurveto{\pgfqpoint{2.016737in}{2.099546in}}{\pgfqpoint{2.024637in}{2.102819in}}{\pgfqpoint{2.030461in}{2.108643in}}%
\pgfpathcurveto{\pgfqpoint{2.036285in}{2.114466in}}{\pgfqpoint{2.039558in}{2.122367in}}{\pgfqpoint{2.039558in}{2.130603in}}%
\pgfpathcurveto{\pgfqpoint{2.039558in}{2.138839in}}{\pgfqpoint{2.036285in}{2.146739in}}{\pgfqpoint{2.030461in}{2.152563in}}%
\pgfpathcurveto{\pgfqpoint{2.024637in}{2.158387in}}{\pgfqpoint{2.016737in}{2.161659in}}{\pgfqpoint{2.008501in}{2.161659in}}%
\pgfpathcurveto{\pgfqpoint{2.000265in}{2.161659in}}{\pgfqpoint{1.992365in}{2.158387in}}{\pgfqpoint{1.986541in}{2.152563in}}%
\pgfpathcurveto{\pgfqpoint{1.980717in}{2.146739in}}{\pgfqpoint{1.977445in}{2.138839in}}{\pgfqpoint{1.977445in}{2.130603in}}%
\pgfpathcurveto{\pgfqpoint{1.977445in}{2.122367in}}{\pgfqpoint{1.980717in}{2.114466in}}{\pgfqpoint{1.986541in}{2.108643in}}%
\pgfpathcurveto{\pgfqpoint{1.992365in}{2.102819in}}{\pgfqpoint{2.000265in}{2.099546in}}{\pgfqpoint{2.008501in}{2.099546in}}%
\pgfpathclose%
\pgfusepath{stroke,fill}%
\end{pgfscope}%
\begin{pgfscope}%
\pgfpathrectangle{\pgfqpoint{0.100000in}{0.212622in}}{\pgfqpoint{3.696000in}{3.696000in}}%
\pgfusepath{clip}%
\pgfsetbuttcap%
\pgfsetroundjoin%
\definecolor{currentfill}{rgb}{0.121569,0.466667,0.705882}%
\pgfsetfillcolor{currentfill}%
\pgfsetfillopacity{0.537783}%
\pgfsetlinewidth{1.003750pt}%
\definecolor{currentstroke}{rgb}{0.121569,0.466667,0.705882}%
\pgfsetstrokecolor{currentstroke}%
\pgfsetstrokeopacity{0.537783}%
\pgfsetdash{}{0pt}%
\pgfpathmoveto{\pgfqpoint{1.981344in}{2.092052in}}%
\pgfpathcurveto{\pgfqpoint{1.989580in}{2.092052in}}{\pgfqpoint{1.997480in}{2.095324in}}{\pgfqpoint{2.003304in}{2.101148in}}%
\pgfpathcurveto{\pgfqpoint{2.009128in}{2.106972in}}{\pgfqpoint{2.012401in}{2.114872in}}{\pgfqpoint{2.012401in}{2.123108in}}%
\pgfpathcurveto{\pgfqpoint{2.012401in}{2.131345in}}{\pgfqpoint{2.009128in}{2.139245in}}{\pgfqpoint{2.003304in}{2.145069in}}%
\pgfpathcurveto{\pgfqpoint{1.997480in}{2.150892in}}{\pgfqpoint{1.989580in}{2.154165in}}{\pgfqpoint{1.981344in}{2.154165in}}%
\pgfpathcurveto{\pgfqpoint{1.973108in}{2.154165in}}{\pgfqpoint{1.965208in}{2.150892in}}{\pgfqpoint{1.959384in}{2.145069in}}%
\pgfpathcurveto{\pgfqpoint{1.953560in}{2.139245in}}{\pgfqpoint{1.950288in}{2.131345in}}{\pgfqpoint{1.950288in}{2.123108in}}%
\pgfpathcurveto{\pgfqpoint{1.950288in}{2.114872in}}{\pgfqpoint{1.953560in}{2.106972in}}{\pgfqpoint{1.959384in}{2.101148in}}%
\pgfpathcurveto{\pgfqpoint{1.965208in}{2.095324in}}{\pgfqpoint{1.973108in}{2.092052in}}{\pgfqpoint{1.981344in}{2.092052in}}%
\pgfpathclose%
\pgfusepath{stroke,fill}%
\end{pgfscope}%
\begin{pgfscope}%
\pgfpathrectangle{\pgfqpoint{0.100000in}{0.212622in}}{\pgfqpoint{3.696000in}{3.696000in}}%
\pgfusepath{clip}%
\pgfsetbuttcap%
\pgfsetroundjoin%
\definecolor{currentfill}{rgb}{0.121569,0.466667,0.705882}%
\pgfsetfillcolor{currentfill}%
\pgfsetfillopacity{0.545688}%
\pgfsetlinewidth{1.003750pt}%
\definecolor{currentstroke}{rgb}{0.121569,0.466667,0.705882}%
\pgfsetstrokecolor{currentstroke}%
\pgfsetstrokeopacity{0.545688}%
\pgfsetdash{}{0pt}%
\pgfpathmoveto{\pgfqpoint{2.004297in}{2.088151in}}%
\pgfpathcurveto{\pgfqpoint{2.012534in}{2.088151in}}{\pgfqpoint{2.020434in}{2.091424in}}{\pgfqpoint{2.026258in}{2.097248in}}%
\pgfpathcurveto{\pgfqpoint{2.032082in}{2.103072in}}{\pgfqpoint{2.035354in}{2.110972in}}{\pgfqpoint{2.035354in}{2.119208in}}%
\pgfpathcurveto{\pgfqpoint{2.035354in}{2.127444in}}{\pgfqpoint{2.032082in}{2.135344in}}{\pgfqpoint{2.026258in}{2.141168in}}%
\pgfpathcurveto{\pgfqpoint{2.020434in}{2.146992in}}{\pgfqpoint{2.012534in}{2.150264in}}{\pgfqpoint{2.004297in}{2.150264in}}%
\pgfpathcurveto{\pgfqpoint{1.996061in}{2.150264in}}{\pgfqpoint{1.988161in}{2.146992in}}{\pgfqpoint{1.982337in}{2.141168in}}%
\pgfpathcurveto{\pgfqpoint{1.976513in}{2.135344in}}{\pgfqpoint{1.973241in}{2.127444in}}{\pgfqpoint{1.973241in}{2.119208in}}%
\pgfpathcurveto{\pgfqpoint{1.973241in}{2.110972in}}{\pgfqpoint{1.976513in}{2.103072in}}{\pgfqpoint{1.982337in}{2.097248in}}%
\pgfpathcurveto{\pgfqpoint{1.988161in}{2.091424in}}{\pgfqpoint{1.996061in}{2.088151in}}{\pgfqpoint{2.004297in}{2.088151in}}%
\pgfpathclose%
\pgfusepath{stroke,fill}%
\end{pgfscope}%
\begin{pgfscope}%
\pgfpathrectangle{\pgfqpoint{0.100000in}{0.212622in}}{\pgfqpoint{3.696000in}{3.696000in}}%
\pgfusepath{clip}%
\pgfsetbuttcap%
\pgfsetroundjoin%
\definecolor{currentfill}{rgb}{0.121569,0.466667,0.705882}%
\pgfsetfillcolor{currentfill}%
\pgfsetfillopacity{0.560509}%
\pgfsetlinewidth{1.003750pt}%
\definecolor{currentstroke}{rgb}{0.121569,0.466667,0.705882}%
\pgfsetstrokecolor{currentstroke}%
\pgfsetstrokeopacity{0.560509}%
\pgfsetdash{}{0pt}%
\pgfpathmoveto{\pgfqpoint{1.982736in}{2.052233in}}%
\pgfpathcurveto{\pgfqpoint{1.990972in}{2.052233in}}{\pgfqpoint{1.998872in}{2.055506in}}{\pgfqpoint{2.004696in}{2.061330in}}%
\pgfpathcurveto{\pgfqpoint{2.010520in}{2.067154in}}{\pgfqpoint{2.013793in}{2.075054in}}{\pgfqpoint{2.013793in}{2.083290in}}%
\pgfpathcurveto{\pgfqpoint{2.013793in}{2.091526in}}{\pgfqpoint{2.010520in}{2.099426in}}{\pgfqpoint{2.004696in}{2.105250in}}%
\pgfpathcurveto{\pgfqpoint{1.998872in}{2.111074in}}{\pgfqpoint{1.990972in}{2.114346in}}{\pgfqpoint{1.982736in}{2.114346in}}%
\pgfpathcurveto{\pgfqpoint{1.974500in}{2.114346in}}{\pgfqpoint{1.966600in}{2.111074in}}{\pgfqpoint{1.960776in}{2.105250in}}%
\pgfpathcurveto{\pgfqpoint{1.954952in}{2.099426in}}{\pgfqpoint{1.951680in}{2.091526in}}{\pgfqpoint{1.951680in}{2.083290in}}%
\pgfpathcurveto{\pgfqpoint{1.951680in}{2.075054in}}{\pgfqpoint{1.954952in}{2.067154in}}{\pgfqpoint{1.960776in}{2.061330in}}%
\pgfpathcurveto{\pgfqpoint{1.966600in}{2.055506in}}{\pgfqpoint{1.974500in}{2.052233in}}{\pgfqpoint{1.982736in}{2.052233in}}%
\pgfpathclose%
\pgfusepath{stroke,fill}%
\end{pgfscope}%
\begin{pgfscope}%
\pgfpathrectangle{\pgfqpoint{0.100000in}{0.212622in}}{\pgfqpoint{3.696000in}{3.696000in}}%
\pgfusepath{clip}%
\pgfsetbuttcap%
\pgfsetroundjoin%
\definecolor{currentfill}{rgb}{0.121569,0.466667,0.705882}%
\pgfsetfillcolor{currentfill}%
\pgfsetfillopacity{0.569889}%
\pgfsetlinewidth{1.003750pt}%
\definecolor{currentstroke}{rgb}{0.121569,0.466667,0.705882}%
\pgfsetstrokecolor{currentstroke}%
\pgfsetstrokeopacity{0.569889}%
\pgfsetdash{}{0pt}%
\pgfpathmoveto{\pgfqpoint{1.982398in}{2.045868in}}%
\pgfpathcurveto{\pgfqpoint{1.990634in}{2.045868in}}{\pgfqpoint{1.998534in}{2.049140in}}{\pgfqpoint{2.004358in}{2.054964in}}%
\pgfpathcurveto{\pgfqpoint{2.010182in}{2.060788in}}{\pgfqpoint{2.013454in}{2.068688in}}{\pgfqpoint{2.013454in}{2.076924in}}%
\pgfpathcurveto{\pgfqpoint{2.013454in}{2.085160in}}{\pgfqpoint{2.010182in}{2.093060in}}{\pgfqpoint{2.004358in}{2.098884in}}%
\pgfpathcurveto{\pgfqpoint{1.998534in}{2.104708in}}{\pgfqpoint{1.990634in}{2.107981in}}{\pgfqpoint{1.982398in}{2.107981in}}%
\pgfpathcurveto{\pgfqpoint{1.974161in}{2.107981in}}{\pgfqpoint{1.966261in}{2.104708in}}{\pgfqpoint{1.960437in}{2.098884in}}%
\pgfpathcurveto{\pgfqpoint{1.954613in}{2.093060in}}{\pgfqpoint{1.951341in}{2.085160in}}{\pgfqpoint{1.951341in}{2.076924in}}%
\pgfpathcurveto{\pgfqpoint{1.951341in}{2.068688in}}{\pgfqpoint{1.954613in}{2.060788in}}{\pgfqpoint{1.960437in}{2.054964in}}%
\pgfpathcurveto{\pgfqpoint{1.966261in}{2.049140in}}{\pgfqpoint{1.974161in}{2.045868in}}{\pgfqpoint{1.982398in}{2.045868in}}%
\pgfpathclose%
\pgfusepath{stroke,fill}%
\end{pgfscope}%
\begin{pgfscope}%
\pgfpathrectangle{\pgfqpoint{0.100000in}{0.212622in}}{\pgfqpoint{3.696000in}{3.696000in}}%
\pgfusepath{clip}%
\pgfsetbuttcap%
\pgfsetroundjoin%
\definecolor{currentfill}{rgb}{0.121569,0.466667,0.705882}%
\pgfsetfillcolor{currentfill}%
\pgfsetfillopacity{0.582299}%
\pgfsetlinewidth{1.003750pt}%
\definecolor{currentstroke}{rgb}{0.121569,0.466667,0.705882}%
\pgfsetstrokecolor{currentstroke}%
\pgfsetstrokeopacity{0.582299}%
\pgfsetdash{}{0pt}%
\pgfpathmoveto{\pgfqpoint{1.976006in}{2.040393in}}%
\pgfpathcurveto{\pgfqpoint{1.984242in}{2.040393in}}{\pgfqpoint{1.992142in}{2.043665in}}{\pgfqpoint{1.997966in}{2.049489in}}%
\pgfpathcurveto{\pgfqpoint{2.003790in}{2.055313in}}{\pgfqpoint{2.007062in}{2.063213in}}{\pgfqpoint{2.007062in}{2.071450in}}%
\pgfpathcurveto{\pgfqpoint{2.007062in}{2.079686in}}{\pgfqpoint{2.003790in}{2.087586in}}{\pgfqpoint{1.997966in}{2.093410in}}%
\pgfpathcurveto{\pgfqpoint{1.992142in}{2.099234in}}{\pgfqpoint{1.984242in}{2.102506in}}{\pgfqpoint{1.976006in}{2.102506in}}%
\pgfpathcurveto{\pgfqpoint{1.967769in}{2.102506in}}{\pgfqpoint{1.959869in}{2.099234in}}{\pgfqpoint{1.954045in}{2.093410in}}%
\pgfpathcurveto{\pgfqpoint{1.948222in}{2.087586in}}{\pgfqpoint{1.944949in}{2.079686in}}{\pgfqpoint{1.944949in}{2.071450in}}%
\pgfpathcurveto{\pgfqpoint{1.944949in}{2.063213in}}{\pgfqpoint{1.948222in}{2.055313in}}{\pgfqpoint{1.954045in}{2.049489in}}%
\pgfpathcurveto{\pgfqpoint{1.959869in}{2.043665in}}{\pgfqpoint{1.967769in}{2.040393in}}{\pgfqpoint{1.976006in}{2.040393in}}%
\pgfpathclose%
\pgfusepath{stroke,fill}%
\end{pgfscope}%
\begin{pgfscope}%
\pgfpathrectangle{\pgfqpoint{0.100000in}{0.212622in}}{\pgfqpoint{3.696000in}{3.696000in}}%
\pgfusepath{clip}%
\pgfsetbuttcap%
\pgfsetroundjoin%
\definecolor{currentfill}{rgb}{0.121569,0.466667,0.705882}%
\pgfsetfillcolor{currentfill}%
\pgfsetfillopacity{0.613456}%
\pgfsetlinewidth{1.003750pt}%
\definecolor{currentstroke}{rgb}{0.121569,0.466667,0.705882}%
\pgfsetstrokecolor{currentstroke}%
\pgfsetstrokeopacity{0.613456}%
\pgfsetdash{}{0pt}%
\pgfpathmoveto{\pgfqpoint{1.908178in}{1.963937in}}%
\pgfpathcurveto{\pgfqpoint{1.916414in}{1.963937in}}{\pgfqpoint{1.924314in}{1.967210in}}{\pgfqpoint{1.930138in}{1.973033in}}%
\pgfpathcurveto{\pgfqpoint{1.935962in}{1.978857in}}{\pgfqpoint{1.939234in}{1.986757in}}{\pgfqpoint{1.939234in}{1.994994in}}%
\pgfpathcurveto{\pgfqpoint{1.939234in}{2.003230in}}{\pgfqpoint{1.935962in}{2.011130in}}{\pgfqpoint{1.930138in}{2.016954in}}%
\pgfpathcurveto{\pgfqpoint{1.924314in}{2.022778in}}{\pgfqpoint{1.916414in}{2.026050in}}{\pgfqpoint{1.908178in}{2.026050in}}%
\pgfpathcurveto{\pgfqpoint{1.899941in}{2.026050in}}{\pgfqpoint{1.892041in}{2.022778in}}{\pgfqpoint{1.886217in}{2.016954in}}%
\pgfpathcurveto{\pgfqpoint{1.880393in}{2.011130in}}{\pgfqpoint{1.877121in}{2.003230in}}{\pgfqpoint{1.877121in}{1.994994in}}%
\pgfpathcurveto{\pgfqpoint{1.877121in}{1.986757in}}{\pgfqpoint{1.880393in}{1.978857in}}{\pgfqpoint{1.886217in}{1.973033in}}%
\pgfpathcurveto{\pgfqpoint{1.892041in}{1.967210in}}{\pgfqpoint{1.899941in}{1.963937in}}{\pgfqpoint{1.908178in}{1.963937in}}%
\pgfpathclose%
\pgfusepath{stroke,fill}%
\end{pgfscope}%
\begin{pgfscope}%
\pgfpathrectangle{\pgfqpoint{0.100000in}{0.212622in}}{\pgfqpoint{3.696000in}{3.696000in}}%
\pgfusepath{clip}%
\pgfsetbuttcap%
\pgfsetroundjoin%
\definecolor{currentfill}{rgb}{0.121569,0.466667,0.705882}%
\pgfsetfillcolor{currentfill}%
\pgfsetfillopacity{0.616782}%
\pgfsetlinewidth{1.003750pt}%
\definecolor{currentstroke}{rgb}{0.121569,0.466667,0.705882}%
\pgfsetstrokecolor{currentstroke}%
\pgfsetstrokeopacity{0.616782}%
\pgfsetdash{}{0pt}%
\pgfpathmoveto{\pgfqpoint{1.921472in}{1.990334in}}%
\pgfpathcurveto{\pgfqpoint{1.929709in}{1.990334in}}{\pgfqpoint{1.937609in}{1.993606in}}{\pgfqpoint{1.943433in}{1.999430in}}%
\pgfpathcurveto{\pgfqpoint{1.949256in}{2.005254in}}{\pgfqpoint{1.952529in}{2.013154in}}{\pgfqpoint{1.952529in}{2.021390in}}%
\pgfpathcurveto{\pgfqpoint{1.952529in}{2.029627in}}{\pgfqpoint{1.949256in}{2.037527in}}{\pgfqpoint{1.943433in}{2.043350in}}%
\pgfpathcurveto{\pgfqpoint{1.937609in}{2.049174in}}{\pgfqpoint{1.929709in}{2.052447in}}{\pgfqpoint{1.921472in}{2.052447in}}%
\pgfpathcurveto{\pgfqpoint{1.913236in}{2.052447in}}{\pgfqpoint{1.905336in}{2.049174in}}{\pgfqpoint{1.899512in}{2.043350in}}%
\pgfpathcurveto{\pgfqpoint{1.893688in}{2.037527in}}{\pgfqpoint{1.890416in}{2.029627in}}{\pgfqpoint{1.890416in}{2.021390in}}%
\pgfpathcurveto{\pgfqpoint{1.890416in}{2.013154in}}{\pgfqpoint{1.893688in}{2.005254in}}{\pgfqpoint{1.899512in}{1.999430in}}%
\pgfpathcurveto{\pgfqpoint{1.905336in}{1.993606in}}{\pgfqpoint{1.913236in}{1.990334in}}{\pgfqpoint{1.921472in}{1.990334in}}%
\pgfpathclose%
\pgfusepath{stroke,fill}%
\end{pgfscope}%
\begin{pgfscope}%
\pgfpathrectangle{\pgfqpoint{0.100000in}{0.212622in}}{\pgfqpoint{3.696000in}{3.696000in}}%
\pgfusepath{clip}%
\pgfsetbuttcap%
\pgfsetroundjoin%
\definecolor{currentfill}{rgb}{0.121569,0.466667,0.705882}%
\pgfsetfillcolor{currentfill}%
\pgfsetfillopacity{0.647733}%
\pgfsetlinewidth{1.003750pt}%
\definecolor{currentstroke}{rgb}{0.121569,0.466667,0.705882}%
\pgfsetstrokecolor{currentstroke}%
\pgfsetstrokeopacity{0.647733}%
\pgfsetdash{}{0pt}%
\pgfpathmoveto{\pgfqpoint{1.856458in}{1.931096in}}%
\pgfpathcurveto{\pgfqpoint{1.864695in}{1.931096in}}{\pgfqpoint{1.872595in}{1.934369in}}{\pgfqpoint{1.878418in}{1.940193in}}%
\pgfpathcurveto{\pgfqpoint{1.884242in}{1.946017in}}{\pgfqpoint{1.887515in}{1.953917in}}{\pgfqpoint{1.887515in}{1.962153in}}%
\pgfpathcurveto{\pgfqpoint{1.887515in}{1.970389in}}{\pgfqpoint{1.884242in}{1.978289in}}{\pgfqpoint{1.878418in}{1.984113in}}%
\pgfpathcurveto{\pgfqpoint{1.872595in}{1.989937in}}{\pgfqpoint{1.864695in}{1.993209in}}{\pgfqpoint{1.856458in}{1.993209in}}%
\pgfpathcurveto{\pgfqpoint{1.848222in}{1.993209in}}{\pgfqpoint{1.840322in}{1.989937in}}{\pgfqpoint{1.834498in}{1.984113in}}%
\pgfpathcurveto{\pgfqpoint{1.828674in}{1.978289in}}{\pgfqpoint{1.825402in}{1.970389in}}{\pgfqpoint{1.825402in}{1.962153in}}%
\pgfpathcurveto{\pgfqpoint{1.825402in}{1.953917in}}{\pgfqpoint{1.828674in}{1.946017in}}{\pgfqpoint{1.834498in}{1.940193in}}%
\pgfpathcurveto{\pgfqpoint{1.840322in}{1.934369in}}{\pgfqpoint{1.848222in}{1.931096in}}{\pgfqpoint{1.856458in}{1.931096in}}%
\pgfpathclose%
\pgfusepath{stroke,fill}%
\end{pgfscope}%
\begin{pgfscope}%
\pgfpathrectangle{\pgfqpoint{0.100000in}{0.212622in}}{\pgfqpoint{3.696000in}{3.696000in}}%
\pgfusepath{clip}%
\pgfsetbuttcap%
\pgfsetroundjoin%
\definecolor{currentfill}{rgb}{0.121569,0.466667,0.705882}%
\pgfsetfillcolor{currentfill}%
\pgfsetfillopacity{0.648648}%
\pgfsetlinewidth{1.003750pt}%
\definecolor{currentstroke}{rgb}{0.121569,0.466667,0.705882}%
\pgfsetstrokecolor{currentstroke}%
\pgfsetstrokeopacity{0.648648}%
\pgfsetdash{}{0pt}%
\pgfpathmoveto{\pgfqpoint{1.840471in}{1.908562in}}%
\pgfpathcurveto{\pgfqpoint{1.848708in}{1.908562in}}{\pgfqpoint{1.856608in}{1.911834in}}{\pgfqpoint{1.862432in}{1.917658in}}%
\pgfpathcurveto{\pgfqpoint{1.868256in}{1.923482in}}{\pgfqpoint{1.871528in}{1.931382in}}{\pgfqpoint{1.871528in}{1.939619in}}%
\pgfpathcurveto{\pgfqpoint{1.871528in}{1.947855in}}{\pgfqpoint{1.868256in}{1.955755in}}{\pgfqpoint{1.862432in}{1.961579in}}%
\pgfpathcurveto{\pgfqpoint{1.856608in}{1.967403in}}{\pgfqpoint{1.848708in}{1.970675in}}{\pgfqpoint{1.840471in}{1.970675in}}%
\pgfpathcurveto{\pgfqpoint{1.832235in}{1.970675in}}{\pgfqpoint{1.824335in}{1.967403in}}{\pgfqpoint{1.818511in}{1.961579in}}%
\pgfpathcurveto{\pgfqpoint{1.812687in}{1.955755in}}{\pgfqpoint{1.809415in}{1.947855in}}{\pgfqpoint{1.809415in}{1.939619in}}%
\pgfpathcurveto{\pgfqpoint{1.809415in}{1.931382in}}{\pgfqpoint{1.812687in}{1.923482in}}{\pgfqpoint{1.818511in}{1.917658in}}%
\pgfpathcurveto{\pgfqpoint{1.824335in}{1.911834in}}{\pgfqpoint{1.832235in}{1.908562in}}{\pgfqpoint{1.840471in}{1.908562in}}%
\pgfpathclose%
\pgfusepath{stroke,fill}%
\end{pgfscope}%
\begin{pgfscope}%
\pgfpathrectangle{\pgfqpoint{0.100000in}{0.212622in}}{\pgfqpoint{3.696000in}{3.696000in}}%
\pgfusepath{clip}%
\pgfsetbuttcap%
\pgfsetroundjoin%
\definecolor{currentfill}{rgb}{0.121569,0.466667,0.705882}%
\pgfsetfillcolor{currentfill}%
\pgfsetfillopacity{0.652610}%
\pgfsetlinewidth{1.003750pt}%
\definecolor{currentstroke}{rgb}{0.121569,0.466667,0.705882}%
\pgfsetstrokecolor{currentstroke}%
\pgfsetstrokeopacity{0.652610}%
\pgfsetdash{}{0pt}%
\pgfpathmoveto{\pgfqpoint{1.847580in}{1.923538in}}%
\pgfpathcurveto{\pgfqpoint{1.855816in}{1.923538in}}{\pgfqpoint{1.863716in}{1.926810in}}{\pgfqpoint{1.869540in}{1.932634in}}%
\pgfpathcurveto{\pgfqpoint{1.875364in}{1.938458in}}{\pgfqpoint{1.878636in}{1.946358in}}{\pgfqpoint{1.878636in}{1.954594in}}%
\pgfpathcurveto{\pgfqpoint{1.878636in}{1.962831in}}{\pgfqpoint{1.875364in}{1.970731in}}{\pgfqpoint{1.869540in}{1.976555in}}%
\pgfpathcurveto{\pgfqpoint{1.863716in}{1.982379in}}{\pgfqpoint{1.855816in}{1.985651in}}{\pgfqpoint{1.847580in}{1.985651in}}%
\pgfpathcurveto{\pgfqpoint{1.839343in}{1.985651in}}{\pgfqpoint{1.831443in}{1.982379in}}{\pgfqpoint{1.825619in}{1.976555in}}%
\pgfpathcurveto{\pgfqpoint{1.819795in}{1.970731in}}{\pgfqpoint{1.816523in}{1.962831in}}{\pgfqpoint{1.816523in}{1.954594in}}%
\pgfpathcurveto{\pgfqpoint{1.816523in}{1.946358in}}{\pgfqpoint{1.819795in}{1.938458in}}{\pgfqpoint{1.825619in}{1.932634in}}%
\pgfpathcurveto{\pgfqpoint{1.831443in}{1.926810in}}{\pgfqpoint{1.839343in}{1.923538in}}{\pgfqpoint{1.847580in}{1.923538in}}%
\pgfpathclose%
\pgfusepath{stroke,fill}%
\end{pgfscope}%
\begin{pgfscope}%
\pgfpathrectangle{\pgfqpoint{0.100000in}{0.212622in}}{\pgfqpoint{3.696000in}{3.696000in}}%
\pgfusepath{clip}%
\pgfsetbuttcap%
\pgfsetroundjoin%
\definecolor{currentfill}{rgb}{0.121569,0.466667,0.705882}%
\pgfsetfillcolor{currentfill}%
\pgfsetfillopacity{0.658885}%
\pgfsetlinewidth{1.003750pt}%
\definecolor{currentstroke}{rgb}{0.121569,0.466667,0.705882}%
\pgfsetstrokecolor{currentstroke}%
\pgfsetstrokeopacity{0.658885}%
\pgfsetdash{}{0pt}%
\pgfpathmoveto{\pgfqpoint{1.833681in}{1.907783in}}%
\pgfpathcurveto{\pgfqpoint{1.841917in}{1.907783in}}{\pgfqpoint{1.849817in}{1.911055in}}{\pgfqpoint{1.855641in}{1.916879in}}%
\pgfpathcurveto{\pgfqpoint{1.861465in}{1.922703in}}{\pgfqpoint{1.864737in}{1.930603in}}{\pgfqpoint{1.864737in}{1.938839in}}%
\pgfpathcurveto{\pgfqpoint{1.864737in}{1.947076in}}{\pgfqpoint{1.861465in}{1.954976in}}{\pgfqpoint{1.855641in}{1.960800in}}%
\pgfpathcurveto{\pgfqpoint{1.849817in}{1.966624in}}{\pgfqpoint{1.841917in}{1.969896in}}{\pgfqpoint{1.833681in}{1.969896in}}%
\pgfpathcurveto{\pgfqpoint{1.825444in}{1.969896in}}{\pgfqpoint{1.817544in}{1.966624in}}{\pgfqpoint{1.811720in}{1.960800in}}%
\pgfpathcurveto{\pgfqpoint{1.805896in}{1.954976in}}{\pgfqpoint{1.802624in}{1.947076in}}{\pgfqpoint{1.802624in}{1.938839in}}%
\pgfpathcurveto{\pgfqpoint{1.802624in}{1.930603in}}{\pgfqpoint{1.805896in}{1.922703in}}{\pgfqpoint{1.811720in}{1.916879in}}%
\pgfpathcurveto{\pgfqpoint{1.817544in}{1.911055in}}{\pgfqpoint{1.825444in}{1.907783in}}{\pgfqpoint{1.833681in}{1.907783in}}%
\pgfpathclose%
\pgfusepath{stroke,fill}%
\end{pgfscope}%
\begin{pgfscope}%
\pgfpathrectangle{\pgfqpoint{0.100000in}{0.212622in}}{\pgfqpoint{3.696000in}{3.696000in}}%
\pgfusepath{clip}%
\pgfsetbuttcap%
\pgfsetroundjoin%
\definecolor{currentfill}{rgb}{0.121569,0.466667,0.705882}%
\pgfsetfillcolor{currentfill}%
\pgfsetfillopacity{0.660883}%
\pgfsetlinewidth{1.003750pt}%
\definecolor{currentstroke}{rgb}{0.121569,0.466667,0.705882}%
\pgfsetstrokecolor{currentstroke}%
\pgfsetstrokeopacity{0.660883}%
\pgfsetdash{}{0pt}%
\pgfpathmoveto{\pgfqpoint{1.834251in}{1.907392in}}%
\pgfpathcurveto{\pgfqpoint{1.842488in}{1.907392in}}{\pgfqpoint{1.850388in}{1.910665in}}{\pgfqpoint{1.856212in}{1.916489in}}%
\pgfpathcurveto{\pgfqpoint{1.862036in}{1.922313in}}{\pgfqpoint{1.865308in}{1.930213in}}{\pgfqpoint{1.865308in}{1.938449in}}%
\pgfpathcurveto{\pgfqpoint{1.865308in}{1.946685in}}{\pgfqpoint{1.862036in}{1.954585in}}{\pgfqpoint{1.856212in}{1.960409in}}%
\pgfpathcurveto{\pgfqpoint{1.850388in}{1.966233in}}{\pgfqpoint{1.842488in}{1.969505in}}{\pgfqpoint{1.834251in}{1.969505in}}%
\pgfpathcurveto{\pgfqpoint{1.826015in}{1.969505in}}{\pgfqpoint{1.818115in}{1.966233in}}{\pgfqpoint{1.812291in}{1.960409in}}%
\pgfpathcurveto{\pgfqpoint{1.806467in}{1.954585in}}{\pgfqpoint{1.803195in}{1.946685in}}{\pgfqpoint{1.803195in}{1.938449in}}%
\pgfpathcurveto{\pgfqpoint{1.803195in}{1.930213in}}{\pgfqpoint{1.806467in}{1.922313in}}{\pgfqpoint{1.812291in}{1.916489in}}%
\pgfpathcurveto{\pgfqpoint{1.818115in}{1.910665in}}{\pgfqpoint{1.826015in}{1.907392in}}{\pgfqpoint{1.834251in}{1.907392in}}%
\pgfpathclose%
\pgfusepath{stroke,fill}%
\end{pgfscope}%
\begin{pgfscope}%
\pgfpathrectangle{\pgfqpoint{0.100000in}{0.212622in}}{\pgfqpoint{3.696000in}{3.696000in}}%
\pgfusepath{clip}%
\pgfsetbuttcap%
\pgfsetroundjoin%
\definecolor{currentfill}{rgb}{0.121569,0.466667,0.705882}%
\pgfsetfillcolor{currentfill}%
\pgfsetfillopacity{0.662003}%
\pgfsetlinewidth{1.003750pt}%
\definecolor{currentstroke}{rgb}{0.121569,0.466667,0.705882}%
\pgfsetstrokecolor{currentstroke}%
\pgfsetstrokeopacity{0.662003}%
\pgfsetdash{}{0pt}%
\pgfpathmoveto{\pgfqpoint{1.839755in}{1.914924in}}%
\pgfpathcurveto{\pgfqpoint{1.847991in}{1.914924in}}{\pgfqpoint{1.855891in}{1.918196in}}{\pgfqpoint{1.861715in}{1.924020in}}%
\pgfpathcurveto{\pgfqpoint{1.867539in}{1.929844in}}{\pgfqpoint{1.870811in}{1.937744in}}{\pgfqpoint{1.870811in}{1.945980in}}%
\pgfpathcurveto{\pgfqpoint{1.870811in}{1.954216in}}{\pgfqpoint{1.867539in}{1.962116in}}{\pgfqpoint{1.861715in}{1.967940in}}%
\pgfpathcurveto{\pgfqpoint{1.855891in}{1.973764in}}{\pgfqpoint{1.847991in}{1.977037in}}{\pgfqpoint{1.839755in}{1.977037in}}%
\pgfpathcurveto{\pgfqpoint{1.831518in}{1.977037in}}{\pgfqpoint{1.823618in}{1.973764in}}{\pgfqpoint{1.817794in}{1.967940in}}%
\pgfpathcurveto{\pgfqpoint{1.811970in}{1.962116in}}{\pgfqpoint{1.808698in}{1.954216in}}{\pgfqpoint{1.808698in}{1.945980in}}%
\pgfpathcurveto{\pgfqpoint{1.808698in}{1.937744in}}{\pgfqpoint{1.811970in}{1.929844in}}{\pgfqpoint{1.817794in}{1.924020in}}%
\pgfpathcurveto{\pgfqpoint{1.823618in}{1.918196in}}{\pgfqpoint{1.831518in}{1.914924in}}{\pgfqpoint{1.839755in}{1.914924in}}%
\pgfpathclose%
\pgfusepath{stroke,fill}%
\end{pgfscope}%
\begin{pgfscope}%
\pgfpathrectangle{\pgfqpoint{0.100000in}{0.212622in}}{\pgfqpoint{3.696000in}{3.696000in}}%
\pgfusepath{clip}%
\pgfsetbuttcap%
\pgfsetroundjoin%
\definecolor{currentfill}{rgb}{0.121569,0.466667,0.705882}%
\pgfsetfillcolor{currentfill}%
\pgfsetfillopacity{0.668098}%
\pgfsetlinewidth{1.003750pt}%
\definecolor{currentstroke}{rgb}{0.121569,0.466667,0.705882}%
\pgfsetstrokecolor{currentstroke}%
\pgfsetstrokeopacity{0.668098}%
\pgfsetdash{}{0pt}%
\pgfpathmoveto{\pgfqpoint{1.830223in}{1.905702in}}%
\pgfpathcurveto{\pgfqpoint{1.838459in}{1.905702in}}{\pgfqpoint{1.846359in}{1.908974in}}{\pgfqpoint{1.852183in}{1.914798in}}%
\pgfpathcurveto{\pgfqpoint{1.858007in}{1.920622in}}{\pgfqpoint{1.861280in}{1.928522in}}{\pgfqpoint{1.861280in}{1.936758in}}%
\pgfpathcurveto{\pgfqpoint{1.861280in}{1.944995in}}{\pgfqpoint{1.858007in}{1.952895in}}{\pgfqpoint{1.852183in}{1.958719in}}%
\pgfpathcurveto{\pgfqpoint{1.846359in}{1.964543in}}{\pgfqpoint{1.838459in}{1.967815in}}{\pgfqpoint{1.830223in}{1.967815in}}%
\pgfpathcurveto{\pgfqpoint{1.821987in}{1.967815in}}{\pgfqpoint{1.814087in}{1.964543in}}{\pgfqpoint{1.808263in}{1.958719in}}%
\pgfpathcurveto{\pgfqpoint{1.802439in}{1.952895in}}{\pgfqpoint{1.799167in}{1.944995in}}{\pgfqpoint{1.799167in}{1.936758in}}%
\pgfpathcurveto{\pgfqpoint{1.799167in}{1.928522in}}{\pgfqpoint{1.802439in}{1.920622in}}{\pgfqpoint{1.808263in}{1.914798in}}%
\pgfpathcurveto{\pgfqpoint{1.814087in}{1.908974in}}{\pgfqpoint{1.821987in}{1.905702in}}{\pgfqpoint{1.830223in}{1.905702in}}%
\pgfpathclose%
\pgfusepath{stroke,fill}%
\end{pgfscope}%
\begin{pgfscope}%
\pgfpathrectangle{\pgfqpoint{0.100000in}{0.212622in}}{\pgfqpoint{3.696000in}{3.696000in}}%
\pgfusepath{clip}%
\pgfsetbuttcap%
\pgfsetroundjoin%
\definecolor{currentfill}{rgb}{0.121569,0.466667,0.705882}%
\pgfsetfillcolor{currentfill}%
\pgfsetfillopacity{0.670825}%
\pgfsetlinewidth{1.003750pt}%
\definecolor{currentstroke}{rgb}{0.121569,0.466667,0.705882}%
\pgfsetstrokecolor{currentstroke}%
\pgfsetstrokeopacity{0.670825}%
\pgfsetdash{}{0pt}%
\pgfpathmoveto{\pgfqpoint{1.864290in}{1.923808in}}%
\pgfpathcurveto{\pgfqpoint{1.872526in}{1.923808in}}{\pgfqpoint{1.880427in}{1.927081in}}{\pgfqpoint{1.886250in}{1.932905in}}%
\pgfpathcurveto{\pgfqpoint{1.892074in}{1.938728in}}{\pgfqpoint{1.895347in}{1.946629in}}{\pgfqpoint{1.895347in}{1.954865in}}%
\pgfpathcurveto{\pgfqpoint{1.895347in}{1.963101in}}{\pgfqpoint{1.892074in}{1.971001in}}{\pgfqpoint{1.886250in}{1.976825in}}%
\pgfpathcurveto{\pgfqpoint{1.880427in}{1.982649in}}{\pgfqpoint{1.872526in}{1.985921in}}{\pgfqpoint{1.864290in}{1.985921in}}%
\pgfpathcurveto{\pgfqpoint{1.856054in}{1.985921in}}{\pgfqpoint{1.848154in}{1.982649in}}{\pgfqpoint{1.842330in}{1.976825in}}%
\pgfpathcurveto{\pgfqpoint{1.836506in}{1.971001in}}{\pgfqpoint{1.833234in}{1.963101in}}{\pgfqpoint{1.833234in}{1.954865in}}%
\pgfpathcurveto{\pgfqpoint{1.833234in}{1.946629in}}{\pgfqpoint{1.836506in}{1.938728in}}{\pgfqpoint{1.842330in}{1.932905in}}%
\pgfpathcurveto{\pgfqpoint{1.848154in}{1.927081in}}{\pgfqpoint{1.856054in}{1.923808in}}{\pgfqpoint{1.864290in}{1.923808in}}%
\pgfpathclose%
\pgfusepath{stroke,fill}%
\end{pgfscope}%
\begin{pgfscope}%
\pgfpathrectangle{\pgfqpoint{0.100000in}{0.212622in}}{\pgfqpoint{3.696000in}{3.696000in}}%
\pgfusepath{clip}%
\pgfsetbuttcap%
\pgfsetroundjoin%
\definecolor{currentfill}{rgb}{0.121569,0.466667,0.705882}%
\pgfsetfillcolor{currentfill}%
\pgfsetfillopacity{0.671049}%
\pgfsetlinewidth{1.003750pt}%
\definecolor{currentstroke}{rgb}{0.121569,0.466667,0.705882}%
\pgfsetstrokecolor{currentstroke}%
\pgfsetstrokeopacity{0.671049}%
\pgfsetdash{}{0pt}%
\pgfpathmoveto{\pgfqpoint{1.891994in}{1.944789in}}%
\pgfpathcurveto{\pgfqpoint{1.900230in}{1.944789in}}{\pgfqpoint{1.908130in}{1.948062in}}{\pgfqpoint{1.913954in}{1.953885in}}%
\pgfpathcurveto{\pgfqpoint{1.919778in}{1.959709in}}{\pgfqpoint{1.923051in}{1.967609in}}{\pgfqpoint{1.923051in}{1.975846in}}%
\pgfpathcurveto{\pgfqpoint{1.923051in}{1.984082in}}{\pgfqpoint{1.919778in}{1.991982in}}{\pgfqpoint{1.913954in}{1.997806in}}%
\pgfpathcurveto{\pgfqpoint{1.908130in}{2.003630in}}{\pgfqpoint{1.900230in}{2.006902in}}{\pgfqpoint{1.891994in}{2.006902in}}%
\pgfpathcurveto{\pgfqpoint{1.883758in}{2.006902in}}{\pgfqpoint{1.875858in}{2.003630in}}{\pgfqpoint{1.870034in}{1.997806in}}%
\pgfpathcurveto{\pgfqpoint{1.864210in}{1.991982in}}{\pgfqpoint{1.860938in}{1.984082in}}{\pgfqpoint{1.860938in}{1.975846in}}%
\pgfpathcurveto{\pgfqpoint{1.860938in}{1.967609in}}{\pgfqpoint{1.864210in}{1.959709in}}{\pgfqpoint{1.870034in}{1.953885in}}%
\pgfpathcurveto{\pgfqpoint{1.875858in}{1.948062in}}{\pgfqpoint{1.883758in}{1.944789in}}{\pgfqpoint{1.891994in}{1.944789in}}%
\pgfpathclose%
\pgfusepath{stroke,fill}%
\end{pgfscope}%
\begin{pgfscope}%
\pgfpathrectangle{\pgfqpoint{0.100000in}{0.212622in}}{\pgfqpoint{3.696000in}{3.696000in}}%
\pgfusepath{clip}%
\pgfsetbuttcap%
\pgfsetroundjoin%
\definecolor{currentfill}{rgb}{0.121569,0.466667,0.705882}%
\pgfsetfillcolor{currentfill}%
\pgfsetfillopacity{0.673963}%
\pgfsetlinewidth{1.003750pt}%
\definecolor{currentstroke}{rgb}{0.121569,0.466667,0.705882}%
\pgfsetstrokecolor{currentstroke}%
\pgfsetstrokeopacity{0.673963}%
\pgfsetdash{}{0pt}%
\pgfpathmoveto{\pgfqpoint{1.844006in}{1.913089in}}%
\pgfpathcurveto{\pgfqpoint{1.852242in}{1.913089in}}{\pgfqpoint{1.860142in}{1.916361in}}{\pgfqpoint{1.865966in}{1.922185in}}%
\pgfpathcurveto{\pgfqpoint{1.871790in}{1.928009in}}{\pgfqpoint{1.875062in}{1.935909in}}{\pgfqpoint{1.875062in}{1.944146in}}%
\pgfpathcurveto{\pgfqpoint{1.875062in}{1.952382in}}{\pgfqpoint{1.871790in}{1.960282in}}{\pgfqpoint{1.865966in}{1.966106in}}%
\pgfpathcurveto{\pgfqpoint{1.860142in}{1.971930in}}{\pgfqpoint{1.852242in}{1.975202in}}{\pgfqpoint{1.844006in}{1.975202in}}%
\pgfpathcurveto{\pgfqpoint{1.835770in}{1.975202in}}{\pgfqpoint{1.827870in}{1.971930in}}{\pgfqpoint{1.822046in}{1.966106in}}%
\pgfpathcurveto{\pgfqpoint{1.816222in}{1.960282in}}{\pgfqpoint{1.812949in}{1.952382in}}{\pgfqpoint{1.812949in}{1.944146in}}%
\pgfpathcurveto{\pgfqpoint{1.812949in}{1.935909in}}{\pgfqpoint{1.816222in}{1.928009in}}{\pgfqpoint{1.822046in}{1.922185in}}%
\pgfpathcurveto{\pgfqpoint{1.827870in}{1.916361in}}{\pgfqpoint{1.835770in}{1.913089in}}{\pgfqpoint{1.844006in}{1.913089in}}%
\pgfpathclose%
\pgfusepath{stroke,fill}%
\end{pgfscope}%
\begin{pgfscope}%
\pgfpathrectangle{\pgfqpoint{0.100000in}{0.212622in}}{\pgfqpoint{3.696000in}{3.696000in}}%
\pgfusepath{clip}%
\pgfsetbuttcap%
\pgfsetroundjoin%
\definecolor{currentfill}{rgb}{0.121569,0.466667,0.705882}%
\pgfsetfillcolor{currentfill}%
\pgfsetfillopacity{0.674542}%
\pgfsetlinewidth{1.003750pt}%
\definecolor{currentstroke}{rgb}{0.121569,0.466667,0.705882}%
\pgfsetstrokecolor{currentstroke}%
\pgfsetstrokeopacity{0.674542}%
\pgfsetdash{}{0pt}%
\pgfpathmoveto{\pgfqpoint{1.820795in}{1.897541in}}%
\pgfpathcurveto{\pgfqpoint{1.829032in}{1.897541in}}{\pgfqpoint{1.836932in}{1.900813in}}{\pgfqpoint{1.842756in}{1.906637in}}%
\pgfpathcurveto{\pgfqpoint{1.848580in}{1.912461in}}{\pgfqpoint{1.851852in}{1.920361in}}{\pgfqpoint{1.851852in}{1.928597in}}%
\pgfpathcurveto{\pgfqpoint{1.851852in}{1.936834in}}{\pgfqpoint{1.848580in}{1.944734in}}{\pgfqpoint{1.842756in}{1.950558in}}%
\pgfpathcurveto{\pgfqpoint{1.836932in}{1.956382in}}{\pgfqpoint{1.829032in}{1.959654in}}{\pgfqpoint{1.820795in}{1.959654in}}%
\pgfpathcurveto{\pgfqpoint{1.812559in}{1.959654in}}{\pgfqpoint{1.804659in}{1.956382in}}{\pgfqpoint{1.798835in}{1.950558in}}%
\pgfpathcurveto{\pgfqpoint{1.793011in}{1.944734in}}{\pgfqpoint{1.789739in}{1.936834in}}{\pgfqpoint{1.789739in}{1.928597in}}%
\pgfpathcurveto{\pgfqpoint{1.789739in}{1.920361in}}{\pgfqpoint{1.793011in}{1.912461in}}{\pgfqpoint{1.798835in}{1.906637in}}%
\pgfpathcurveto{\pgfqpoint{1.804659in}{1.900813in}}{\pgfqpoint{1.812559in}{1.897541in}}{\pgfqpoint{1.820795in}{1.897541in}}%
\pgfpathclose%
\pgfusepath{stroke,fill}%
\end{pgfscope}%
\begin{pgfscope}%
\pgfpathrectangle{\pgfqpoint{0.100000in}{0.212622in}}{\pgfqpoint{3.696000in}{3.696000in}}%
\pgfusepath{clip}%
\pgfsetbuttcap%
\pgfsetroundjoin%
\definecolor{currentfill}{rgb}{0.121569,0.466667,0.705882}%
\pgfsetfillcolor{currentfill}%
\pgfsetfillopacity{0.675382}%
\pgfsetlinewidth{1.003750pt}%
\definecolor{currentstroke}{rgb}{0.121569,0.466667,0.705882}%
\pgfsetstrokecolor{currentstroke}%
\pgfsetstrokeopacity{0.675382}%
\pgfsetdash{}{0pt}%
\pgfpathmoveto{\pgfqpoint{1.863457in}{1.918116in}}%
\pgfpathcurveto{\pgfqpoint{1.871693in}{1.918116in}}{\pgfqpoint{1.879593in}{1.921388in}}{\pgfqpoint{1.885417in}{1.927212in}}%
\pgfpathcurveto{\pgfqpoint{1.891241in}{1.933036in}}{\pgfqpoint{1.894513in}{1.940936in}}{\pgfqpoint{1.894513in}{1.949172in}}%
\pgfpathcurveto{\pgfqpoint{1.894513in}{1.957409in}}{\pgfqpoint{1.891241in}{1.965309in}}{\pgfqpoint{1.885417in}{1.971133in}}%
\pgfpathcurveto{\pgfqpoint{1.879593in}{1.976956in}}{\pgfqpoint{1.871693in}{1.980229in}}{\pgfqpoint{1.863457in}{1.980229in}}%
\pgfpathcurveto{\pgfqpoint{1.855220in}{1.980229in}}{\pgfqpoint{1.847320in}{1.976956in}}{\pgfqpoint{1.841496in}{1.971133in}}%
\pgfpathcurveto{\pgfqpoint{1.835673in}{1.965309in}}{\pgfqpoint{1.832400in}{1.957409in}}{\pgfqpoint{1.832400in}{1.949172in}}%
\pgfpathcurveto{\pgfqpoint{1.832400in}{1.940936in}}{\pgfqpoint{1.835673in}{1.933036in}}{\pgfqpoint{1.841496in}{1.927212in}}%
\pgfpathcurveto{\pgfqpoint{1.847320in}{1.921388in}}{\pgfqpoint{1.855220in}{1.918116in}}{\pgfqpoint{1.863457in}{1.918116in}}%
\pgfpathclose%
\pgfusepath{stroke,fill}%
\end{pgfscope}%
\begin{pgfscope}%
\pgfpathrectangle{\pgfqpoint{0.100000in}{0.212622in}}{\pgfqpoint{3.696000in}{3.696000in}}%
\pgfusepath{clip}%
\pgfsetbuttcap%
\pgfsetroundjoin%
\definecolor{currentfill}{rgb}{0.121569,0.466667,0.705882}%
\pgfsetfillcolor{currentfill}%
\pgfsetfillopacity{0.676218}%
\pgfsetlinewidth{1.003750pt}%
\definecolor{currentstroke}{rgb}{0.121569,0.466667,0.705882}%
\pgfsetstrokecolor{currentstroke}%
\pgfsetstrokeopacity{0.676218}%
\pgfsetdash{}{0pt}%
\pgfpathmoveto{\pgfqpoint{1.907063in}{1.952665in}}%
\pgfpathcurveto{\pgfqpoint{1.915300in}{1.952665in}}{\pgfqpoint{1.923200in}{1.955938in}}{\pgfqpoint{1.929024in}{1.961761in}}%
\pgfpathcurveto{\pgfqpoint{1.934848in}{1.967585in}}{\pgfqpoint{1.938120in}{1.975485in}}{\pgfqpoint{1.938120in}{1.983722in}}%
\pgfpathcurveto{\pgfqpoint{1.938120in}{1.991958in}}{\pgfqpoint{1.934848in}{1.999858in}}{\pgfqpoint{1.929024in}{2.005682in}}%
\pgfpathcurveto{\pgfqpoint{1.923200in}{2.011506in}}{\pgfqpoint{1.915300in}{2.014778in}}{\pgfqpoint{1.907063in}{2.014778in}}%
\pgfpathcurveto{\pgfqpoint{1.898827in}{2.014778in}}{\pgfqpoint{1.890927in}{2.011506in}}{\pgfqpoint{1.885103in}{2.005682in}}%
\pgfpathcurveto{\pgfqpoint{1.879279in}{1.999858in}}{\pgfqpoint{1.876007in}{1.991958in}}{\pgfqpoint{1.876007in}{1.983722in}}%
\pgfpathcurveto{\pgfqpoint{1.876007in}{1.975485in}}{\pgfqpoint{1.879279in}{1.967585in}}{\pgfqpoint{1.885103in}{1.961761in}}%
\pgfpathcurveto{\pgfqpoint{1.890927in}{1.955938in}}{\pgfqpoint{1.898827in}{1.952665in}}{\pgfqpoint{1.907063in}{1.952665in}}%
\pgfpathclose%
\pgfusepath{stroke,fill}%
\end{pgfscope}%
\begin{pgfscope}%
\pgfpathrectangle{\pgfqpoint{0.100000in}{0.212622in}}{\pgfqpoint{3.696000in}{3.696000in}}%
\pgfusepath{clip}%
\pgfsetbuttcap%
\pgfsetroundjoin%
\definecolor{currentfill}{rgb}{0.121569,0.466667,0.705882}%
\pgfsetfillcolor{currentfill}%
\pgfsetfillopacity{0.677599}%
\pgfsetlinewidth{1.003750pt}%
\definecolor{currentstroke}{rgb}{0.121569,0.466667,0.705882}%
\pgfsetstrokecolor{currentstroke}%
\pgfsetstrokeopacity{0.677599}%
\pgfsetdash{}{0pt}%
\pgfpathmoveto{\pgfqpoint{1.821334in}{1.894995in}}%
\pgfpathcurveto{\pgfqpoint{1.829570in}{1.894995in}}{\pgfqpoint{1.837470in}{1.898267in}}{\pgfqpoint{1.843294in}{1.904091in}}%
\pgfpathcurveto{\pgfqpoint{1.849118in}{1.909915in}}{\pgfqpoint{1.852390in}{1.917815in}}{\pgfqpoint{1.852390in}{1.926051in}}%
\pgfpathcurveto{\pgfqpoint{1.852390in}{1.934288in}}{\pgfqpoint{1.849118in}{1.942188in}}{\pgfqpoint{1.843294in}{1.948012in}}%
\pgfpathcurveto{\pgfqpoint{1.837470in}{1.953835in}}{\pgfqpoint{1.829570in}{1.957108in}}{\pgfqpoint{1.821334in}{1.957108in}}%
\pgfpathcurveto{\pgfqpoint{1.813097in}{1.957108in}}{\pgfqpoint{1.805197in}{1.953835in}}{\pgfqpoint{1.799373in}{1.948012in}}%
\pgfpathcurveto{\pgfqpoint{1.793549in}{1.942188in}}{\pgfqpoint{1.790277in}{1.934288in}}{\pgfqpoint{1.790277in}{1.926051in}}%
\pgfpathcurveto{\pgfqpoint{1.790277in}{1.917815in}}{\pgfqpoint{1.793549in}{1.909915in}}{\pgfqpoint{1.799373in}{1.904091in}}%
\pgfpathcurveto{\pgfqpoint{1.805197in}{1.898267in}}{\pgfqpoint{1.813097in}{1.894995in}}{\pgfqpoint{1.821334in}{1.894995in}}%
\pgfpathclose%
\pgfusepath{stroke,fill}%
\end{pgfscope}%
\begin{pgfscope}%
\pgfpathrectangle{\pgfqpoint{0.100000in}{0.212622in}}{\pgfqpoint{3.696000in}{3.696000in}}%
\pgfusepath{clip}%
\pgfsetbuttcap%
\pgfsetroundjoin%
\definecolor{currentfill}{rgb}{0.121569,0.466667,0.705882}%
\pgfsetfillcolor{currentfill}%
\pgfsetfillopacity{0.678834}%
\pgfsetlinewidth{1.003750pt}%
\definecolor{currentstroke}{rgb}{0.121569,0.466667,0.705882}%
\pgfsetstrokecolor{currentstroke}%
\pgfsetstrokeopacity{0.678834}%
\pgfsetdash{}{0pt}%
\pgfpathmoveto{\pgfqpoint{1.881285in}{1.925857in}}%
\pgfpathcurveto{\pgfqpoint{1.889522in}{1.925857in}}{\pgfqpoint{1.897422in}{1.929129in}}{\pgfqpoint{1.903246in}{1.934953in}}%
\pgfpathcurveto{\pgfqpoint{1.909070in}{1.940777in}}{\pgfqpoint{1.912342in}{1.948677in}}{\pgfqpoint{1.912342in}{1.956913in}}%
\pgfpathcurveto{\pgfqpoint{1.912342in}{1.965149in}}{\pgfqpoint{1.909070in}{1.973050in}}{\pgfqpoint{1.903246in}{1.978873in}}%
\pgfpathcurveto{\pgfqpoint{1.897422in}{1.984697in}}{\pgfqpoint{1.889522in}{1.987970in}}{\pgfqpoint{1.881285in}{1.987970in}}%
\pgfpathcurveto{\pgfqpoint{1.873049in}{1.987970in}}{\pgfqpoint{1.865149in}{1.984697in}}{\pgfqpoint{1.859325in}{1.978873in}}%
\pgfpathcurveto{\pgfqpoint{1.853501in}{1.973050in}}{\pgfqpoint{1.850229in}{1.965149in}}{\pgfqpoint{1.850229in}{1.956913in}}%
\pgfpathcurveto{\pgfqpoint{1.850229in}{1.948677in}}{\pgfqpoint{1.853501in}{1.940777in}}{\pgfqpoint{1.859325in}{1.934953in}}%
\pgfpathcurveto{\pgfqpoint{1.865149in}{1.929129in}}{\pgfqpoint{1.873049in}{1.925857in}}{\pgfqpoint{1.881285in}{1.925857in}}%
\pgfpathclose%
\pgfusepath{stroke,fill}%
\end{pgfscope}%
\begin{pgfscope}%
\pgfpathrectangle{\pgfqpoint{0.100000in}{0.212622in}}{\pgfqpoint{3.696000in}{3.696000in}}%
\pgfusepath{clip}%
\pgfsetbuttcap%
\pgfsetroundjoin%
\definecolor{currentfill}{rgb}{0.121569,0.466667,0.705882}%
\pgfsetfillcolor{currentfill}%
\pgfsetfillopacity{0.682849}%
\pgfsetlinewidth{1.003750pt}%
\definecolor{currentstroke}{rgb}{0.121569,0.466667,0.705882}%
\pgfsetstrokecolor{currentstroke}%
\pgfsetstrokeopacity{0.682849}%
\pgfsetdash{}{0pt}%
\pgfpathmoveto{\pgfqpoint{1.913173in}{1.955242in}}%
\pgfpathcurveto{\pgfqpoint{1.921410in}{1.955242in}}{\pgfqpoint{1.929310in}{1.958514in}}{\pgfqpoint{1.935133in}{1.964338in}}%
\pgfpathcurveto{\pgfqpoint{1.940957in}{1.970162in}}{\pgfqpoint{1.944230in}{1.978062in}}{\pgfqpoint{1.944230in}{1.986299in}}%
\pgfpathcurveto{\pgfqpoint{1.944230in}{1.994535in}}{\pgfqpoint{1.940957in}{2.002435in}}{\pgfqpoint{1.935133in}{2.008259in}}%
\pgfpathcurveto{\pgfqpoint{1.929310in}{2.014083in}}{\pgfqpoint{1.921410in}{2.017355in}}{\pgfqpoint{1.913173in}{2.017355in}}%
\pgfpathcurveto{\pgfqpoint{1.904937in}{2.017355in}}{\pgfqpoint{1.897037in}{2.014083in}}{\pgfqpoint{1.891213in}{2.008259in}}%
\pgfpathcurveto{\pgfqpoint{1.885389in}{2.002435in}}{\pgfqpoint{1.882117in}{1.994535in}}{\pgfqpoint{1.882117in}{1.986299in}}%
\pgfpathcurveto{\pgfqpoint{1.882117in}{1.978062in}}{\pgfqpoint{1.885389in}{1.970162in}}{\pgfqpoint{1.891213in}{1.964338in}}%
\pgfpathcurveto{\pgfqpoint{1.897037in}{1.958514in}}{\pgfqpoint{1.904937in}{1.955242in}}{\pgfqpoint{1.913173in}{1.955242in}}%
\pgfpathclose%
\pgfusepath{stroke,fill}%
\end{pgfscope}%
\begin{pgfscope}%
\pgfpathrectangle{\pgfqpoint{0.100000in}{0.212622in}}{\pgfqpoint{3.696000in}{3.696000in}}%
\pgfusepath{clip}%
\pgfsetbuttcap%
\pgfsetroundjoin%
\definecolor{currentfill}{rgb}{0.121569,0.466667,0.705882}%
\pgfsetfillcolor{currentfill}%
\pgfsetfillopacity{0.691840}%
\pgfsetlinewidth{1.003750pt}%
\definecolor{currentstroke}{rgb}{0.121569,0.466667,0.705882}%
\pgfsetstrokecolor{currentstroke}%
\pgfsetstrokeopacity{0.691840}%
\pgfsetdash{}{0pt}%
\pgfpathmoveto{\pgfqpoint{1.912709in}{1.944785in}}%
\pgfpathcurveto{\pgfqpoint{1.920945in}{1.944785in}}{\pgfqpoint{1.928845in}{1.948058in}}{\pgfqpoint{1.934669in}{1.953882in}}%
\pgfpathcurveto{\pgfqpoint{1.940493in}{1.959705in}}{\pgfqpoint{1.943766in}{1.967606in}}{\pgfqpoint{1.943766in}{1.975842in}}%
\pgfpathcurveto{\pgfqpoint{1.943766in}{1.984078in}}{\pgfqpoint{1.940493in}{1.991978in}}{\pgfqpoint{1.934669in}{1.997802in}}%
\pgfpathcurveto{\pgfqpoint{1.928845in}{2.003626in}}{\pgfqpoint{1.920945in}{2.006898in}}{\pgfqpoint{1.912709in}{2.006898in}}%
\pgfpathcurveto{\pgfqpoint{1.904473in}{2.006898in}}{\pgfqpoint{1.896573in}{2.003626in}}{\pgfqpoint{1.890749in}{1.997802in}}%
\pgfpathcurveto{\pgfqpoint{1.884925in}{1.991978in}}{\pgfqpoint{1.881653in}{1.984078in}}{\pgfqpoint{1.881653in}{1.975842in}}%
\pgfpathcurveto{\pgfqpoint{1.881653in}{1.967606in}}{\pgfqpoint{1.884925in}{1.959705in}}{\pgfqpoint{1.890749in}{1.953882in}}%
\pgfpathcurveto{\pgfqpoint{1.896573in}{1.948058in}}{\pgfqpoint{1.904473in}{1.944785in}}{\pgfqpoint{1.912709in}{1.944785in}}%
\pgfpathclose%
\pgfusepath{stroke,fill}%
\end{pgfscope}%
\begin{pgfscope}%
\pgfpathrectangle{\pgfqpoint{0.100000in}{0.212622in}}{\pgfqpoint{3.696000in}{3.696000in}}%
\pgfusepath{clip}%
\pgfsetbuttcap%
\pgfsetroundjoin%
\definecolor{currentfill}{rgb}{0.121569,0.466667,0.705882}%
\pgfsetfillcolor{currentfill}%
\pgfsetfillopacity{0.691929}%
\pgfsetlinewidth{1.003750pt}%
\definecolor{currentstroke}{rgb}{0.121569,0.466667,0.705882}%
\pgfsetstrokecolor{currentstroke}%
\pgfsetstrokeopacity{0.691929}%
\pgfsetdash{}{0pt}%
\pgfpathmoveto{\pgfqpoint{1.900269in}{1.939167in}}%
\pgfpathcurveto{\pgfqpoint{1.908505in}{1.939167in}}{\pgfqpoint{1.916406in}{1.942439in}}{\pgfqpoint{1.922229in}{1.948263in}}%
\pgfpathcurveto{\pgfqpoint{1.928053in}{1.954087in}}{\pgfqpoint{1.931326in}{1.961987in}}{\pgfqpoint{1.931326in}{1.970223in}}%
\pgfpathcurveto{\pgfqpoint{1.931326in}{1.978459in}}{\pgfqpoint{1.928053in}{1.986360in}}{\pgfqpoint{1.922229in}{1.992183in}}%
\pgfpathcurveto{\pgfqpoint{1.916406in}{1.998007in}}{\pgfqpoint{1.908505in}{2.001280in}}{\pgfqpoint{1.900269in}{2.001280in}}%
\pgfpathcurveto{\pgfqpoint{1.892033in}{2.001280in}}{\pgfqpoint{1.884133in}{1.998007in}}{\pgfqpoint{1.878309in}{1.992183in}}%
\pgfpathcurveto{\pgfqpoint{1.872485in}{1.986360in}}{\pgfqpoint{1.869213in}{1.978459in}}{\pgfqpoint{1.869213in}{1.970223in}}%
\pgfpathcurveto{\pgfqpoint{1.869213in}{1.961987in}}{\pgfqpoint{1.872485in}{1.954087in}}{\pgfqpoint{1.878309in}{1.948263in}}%
\pgfpathcurveto{\pgfqpoint{1.884133in}{1.942439in}}{\pgfqpoint{1.892033in}{1.939167in}}{\pgfqpoint{1.900269in}{1.939167in}}%
\pgfpathclose%
\pgfusepath{stroke,fill}%
\end{pgfscope}%
\begin{pgfscope}%
\pgfpathrectangle{\pgfqpoint{0.100000in}{0.212622in}}{\pgfqpoint{3.696000in}{3.696000in}}%
\pgfusepath{clip}%
\pgfsetbuttcap%
\pgfsetroundjoin%
\definecolor{currentfill}{rgb}{0.121569,0.466667,0.705882}%
\pgfsetfillcolor{currentfill}%
\pgfsetfillopacity{0.692261}%
\pgfsetlinewidth{1.003750pt}%
\definecolor{currentstroke}{rgb}{0.121569,0.466667,0.705882}%
\pgfsetstrokecolor{currentstroke}%
\pgfsetstrokeopacity{0.692261}%
\pgfsetdash{}{0pt}%
\pgfpathmoveto{\pgfqpoint{1.873929in}{1.914869in}}%
\pgfpathcurveto{\pgfqpoint{1.882166in}{1.914869in}}{\pgfqpoint{1.890066in}{1.918141in}}{\pgfqpoint{1.895890in}{1.923965in}}%
\pgfpathcurveto{\pgfqpoint{1.901714in}{1.929789in}}{\pgfqpoint{1.904986in}{1.937689in}}{\pgfqpoint{1.904986in}{1.945925in}}%
\pgfpathcurveto{\pgfqpoint{1.904986in}{1.954162in}}{\pgfqpoint{1.901714in}{1.962062in}}{\pgfqpoint{1.895890in}{1.967886in}}%
\pgfpathcurveto{\pgfqpoint{1.890066in}{1.973709in}}{\pgfqpoint{1.882166in}{1.976982in}}{\pgfqpoint{1.873929in}{1.976982in}}%
\pgfpathcurveto{\pgfqpoint{1.865693in}{1.976982in}}{\pgfqpoint{1.857793in}{1.973709in}}{\pgfqpoint{1.851969in}{1.967886in}}%
\pgfpathcurveto{\pgfqpoint{1.846145in}{1.962062in}}{\pgfqpoint{1.842873in}{1.954162in}}{\pgfqpoint{1.842873in}{1.945925in}}%
\pgfpathcurveto{\pgfqpoint{1.842873in}{1.937689in}}{\pgfqpoint{1.846145in}{1.929789in}}{\pgfqpoint{1.851969in}{1.923965in}}%
\pgfpathcurveto{\pgfqpoint{1.857793in}{1.918141in}}{\pgfqpoint{1.865693in}{1.914869in}}{\pgfqpoint{1.873929in}{1.914869in}}%
\pgfpathclose%
\pgfusepath{stroke,fill}%
\end{pgfscope}%
\begin{pgfscope}%
\pgfpathrectangle{\pgfqpoint{0.100000in}{0.212622in}}{\pgfqpoint{3.696000in}{3.696000in}}%
\pgfusepath{clip}%
\pgfsetbuttcap%
\pgfsetroundjoin%
\definecolor{currentfill}{rgb}{0.121569,0.466667,0.705882}%
\pgfsetfillcolor{currentfill}%
\pgfsetfillopacity{0.692328}%
\pgfsetlinewidth{1.003750pt}%
\definecolor{currentstroke}{rgb}{0.121569,0.466667,0.705882}%
\pgfsetstrokecolor{currentstroke}%
\pgfsetstrokeopacity{0.692328}%
\pgfsetdash{}{0pt}%
\pgfpathmoveto{\pgfqpoint{1.904814in}{1.940230in}}%
\pgfpathcurveto{\pgfqpoint{1.913050in}{1.940230in}}{\pgfqpoint{1.920950in}{1.943502in}}{\pgfqpoint{1.926774in}{1.949326in}}%
\pgfpathcurveto{\pgfqpoint{1.932598in}{1.955150in}}{\pgfqpoint{1.935871in}{1.963050in}}{\pgfqpoint{1.935871in}{1.971286in}}%
\pgfpathcurveto{\pgfqpoint{1.935871in}{1.979523in}}{\pgfqpoint{1.932598in}{1.987423in}}{\pgfqpoint{1.926774in}{1.993247in}}%
\pgfpathcurveto{\pgfqpoint{1.920950in}{1.999070in}}{\pgfqpoint{1.913050in}{2.002343in}}{\pgfqpoint{1.904814in}{2.002343in}}%
\pgfpathcurveto{\pgfqpoint{1.896578in}{2.002343in}}{\pgfqpoint{1.888678in}{1.999070in}}{\pgfqpoint{1.882854in}{1.993247in}}%
\pgfpathcurveto{\pgfqpoint{1.877030in}{1.987423in}}{\pgfqpoint{1.873758in}{1.979523in}}{\pgfqpoint{1.873758in}{1.971286in}}%
\pgfpathcurveto{\pgfqpoint{1.873758in}{1.963050in}}{\pgfqpoint{1.877030in}{1.955150in}}{\pgfqpoint{1.882854in}{1.949326in}}%
\pgfpathcurveto{\pgfqpoint{1.888678in}{1.943502in}}{\pgfqpoint{1.896578in}{1.940230in}}{\pgfqpoint{1.904814in}{1.940230in}}%
\pgfpathclose%
\pgfusepath{stroke,fill}%
\end{pgfscope}%
\begin{pgfscope}%
\pgfpathrectangle{\pgfqpoint{0.100000in}{0.212622in}}{\pgfqpoint{3.696000in}{3.696000in}}%
\pgfusepath{clip}%
\pgfsetbuttcap%
\pgfsetroundjoin%
\definecolor{currentfill}{rgb}{0.121569,0.466667,0.705882}%
\pgfsetfillcolor{currentfill}%
\pgfsetfillopacity{0.692994}%
\pgfsetlinewidth{1.003750pt}%
\definecolor{currentstroke}{rgb}{0.121569,0.466667,0.705882}%
\pgfsetstrokecolor{currentstroke}%
\pgfsetstrokeopacity{0.692994}%
\pgfsetdash{}{0pt}%
\pgfpathmoveto{\pgfqpoint{1.906278in}{1.937980in}}%
\pgfpathcurveto{\pgfqpoint{1.914514in}{1.937980in}}{\pgfqpoint{1.922414in}{1.941252in}}{\pgfqpoint{1.928238in}{1.947076in}}%
\pgfpathcurveto{\pgfqpoint{1.934062in}{1.952900in}}{\pgfqpoint{1.937334in}{1.960800in}}{\pgfqpoint{1.937334in}{1.969036in}}%
\pgfpathcurveto{\pgfqpoint{1.937334in}{1.977273in}}{\pgfqpoint{1.934062in}{1.985173in}}{\pgfqpoint{1.928238in}{1.990997in}}%
\pgfpathcurveto{\pgfqpoint{1.922414in}{1.996821in}}{\pgfqpoint{1.914514in}{2.000093in}}{\pgfqpoint{1.906278in}{2.000093in}}%
\pgfpathcurveto{\pgfqpoint{1.898041in}{2.000093in}}{\pgfqpoint{1.890141in}{1.996821in}}{\pgfqpoint{1.884317in}{1.990997in}}%
\pgfpathcurveto{\pgfqpoint{1.878493in}{1.985173in}}{\pgfqpoint{1.875221in}{1.977273in}}{\pgfqpoint{1.875221in}{1.969036in}}%
\pgfpathcurveto{\pgfqpoint{1.875221in}{1.960800in}}{\pgfqpoint{1.878493in}{1.952900in}}{\pgfqpoint{1.884317in}{1.947076in}}%
\pgfpathcurveto{\pgfqpoint{1.890141in}{1.941252in}}{\pgfqpoint{1.898041in}{1.937980in}}{\pgfqpoint{1.906278in}{1.937980in}}%
\pgfpathclose%
\pgfusepath{stroke,fill}%
\end{pgfscope}%
\begin{pgfscope}%
\pgfpathrectangle{\pgfqpoint{0.100000in}{0.212622in}}{\pgfqpoint{3.696000in}{3.696000in}}%
\pgfusepath{clip}%
\pgfsetbuttcap%
\pgfsetroundjoin%
\definecolor{currentfill}{rgb}{0.121569,0.466667,0.705882}%
\pgfsetfillcolor{currentfill}%
\pgfsetfillopacity{0.694023}%
\pgfsetlinewidth{1.003750pt}%
\definecolor{currentstroke}{rgb}{0.121569,0.466667,0.705882}%
\pgfsetstrokecolor{currentstroke}%
\pgfsetstrokeopacity{0.694023}%
\pgfsetdash{}{0pt}%
\pgfpathmoveto{\pgfqpoint{1.895928in}{1.932104in}}%
\pgfpathcurveto{\pgfqpoint{1.904164in}{1.932104in}}{\pgfqpoint{1.912064in}{1.935377in}}{\pgfqpoint{1.917888in}{1.941201in}}%
\pgfpathcurveto{\pgfqpoint{1.923712in}{1.947025in}}{\pgfqpoint{1.926985in}{1.954925in}}{\pgfqpoint{1.926985in}{1.963161in}}%
\pgfpathcurveto{\pgfqpoint{1.926985in}{1.971397in}}{\pgfqpoint{1.923712in}{1.979297in}}{\pgfqpoint{1.917888in}{1.985121in}}%
\pgfpathcurveto{\pgfqpoint{1.912064in}{1.990945in}}{\pgfqpoint{1.904164in}{1.994217in}}{\pgfqpoint{1.895928in}{1.994217in}}%
\pgfpathcurveto{\pgfqpoint{1.887692in}{1.994217in}}{\pgfqpoint{1.879792in}{1.990945in}}{\pgfqpoint{1.873968in}{1.985121in}}%
\pgfpathcurveto{\pgfqpoint{1.868144in}{1.979297in}}{\pgfqpoint{1.864872in}{1.971397in}}{\pgfqpoint{1.864872in}{1.963161in}}%
\pgfpathcurveto{\pgfqpoint{1.864872in}{1.954925in}}{\pgfqpoint{1.868144in}{1.947025in}}{\pgfqpoint{1.873968in}{1.941201in}}%
\pgfpathcurveto{\pgfqpoint{1.879792in}{1.935377in}}{\pgfqpoint{1.887692in}{1.932104in}}{\pgfqpoint{1.895928in}{1.932104in}}%
\pgfpathclose%
\pgfusepath{stroke,fill}%
\end{pgfscope}%
\begin{pgfscope}%
\pgfpathrectangle{\pgfqpoint{0.100000in}{0.212622in}}{\pgfqpoint{3.696000in}{3.696000in}}%
\pgfusepath{clip}%
\pgfsetbuttcap%
\pgfsetroundjoin%
\definecolor{currentfill}{rgb}{0.121569,0.466667,0.705882}%
\pgfsetfillcolor{currentfill}%
\pgfsetfillopacity{0.694125}%
\pgfsetlinewidth{1.003750pt}%
\definecolor{currentstroke}{rgb}{0.121569,0.466667,0.705882}%
\pgfsetstrokecolor{currentstroke}%
\pgfsetstrokeopacity{0.694125}%
\pgfsetdash{}{0pt}%
\pgfpathmoveto{\pgfqpoint{1.887063in}{1.924298in}}%
\pgfpathcurveto{\pgfqpoint{1.895299in}{1.924298in}}{\pgfqpoint{1.903199in}{1.927570in}}{\pgfqpoint{1.909023in}{1.933394in}}%
\pgfpathcurveto{\pgfqpoint{1.914847in}{1.939218in}}{\pgfqpoint{1.918120in}{1.947118in}}{\pgfqpoint{1.918120in}{1.955354in}}%
\pgfpathcurveto{\pgfqpoint{1.918120in}{1.963590in}}{\pgfqpoint{1.914847in}{1.971490in}}{\pgfqpoint{1.909023in}{1.977314in}}%
\pgfpathcurveto{\pgfqpoint{1.903199in}{1.983138in}}{\pgfqpoint{1.895299in}{1.986411in}}{\pgfqpoint{1.887063in}{1.986411in}}%
\pgfpathcurveto{\pgfqpoint{1.878827in}{1.986411in}}{\pgfqpoint{1.870927in}{1.983138in}}{\pgfqpoint{1.865103in}{1.977314in}}%
\pgfpathcurveto{\pgfqpoint{1.859279in}{1.971490in}}{\pgfqpoint{1.856007in}{1.963590in}}{\pgfqpoint{1.856007in}{1.955354in}}%
\pgfpathcurveto{\pgfqpoint{1.856007in}{1.947118in}}{\pgfqpoint{1.859279in}{1.939218in}}{\pgfqpoint{1.865103in}{1.933394in}}%
\pgfpathcurveto{\pgfqpoint{1.870927in}{1.927570in}}{\pgfqpoint{1.878827in}{1.924298in}}{\pgfqpoint{1.887063in}{1.924298in}}%
\pgfpathclose%
\pgfusepath{stroke,fill}%
\end{pgfscope}%
\begin{pgfscope}%
\pgfpathrectangle{\pgfqpoint{0.100000in}{0.212622in}}{\pgfqpoint{3.696000in}{3.696000in}}%
\pgfusepath{clip}%
\pgfsetbuttcap%
\pgfsetroundjoin%
\definecolor{currentfill}{rgb}{0.121569,0.466667,0.705882}%
\pgfsetfillcolor{currentfill}%
\pgfsetfillopacity{0.696324}%
\pgfsetlinewidth{1.003750pt}%
\definecolor{currentstroke}{rgb}{0.121569,0.466667,0.705882}%
\pgfsetstrokecolor{currentstroke}%
\pgfsetstrokeopacity{0.696324}%
\pgfsetdash{}{0pt}%
\pgfpathmoveto{\pgfqpoint{1.909196in}{1.939857in}}%
\pgfpathcurveto{\pgfqpoint{1.917432in}{1.939857in}}{\pgfqpoint{1.925332in}{1.943129in}}{\pgfqpoint{1.931156in}{1.948953in}}%
\pgfpathcurveto{\pgfqpoint{1.936980in}{1.954777in}}{\pgfqpoint{1.940252in}{1.962677in}}{\pgfqpoint{1.940252in}{1.970913in}}%
\pgfpathcurveto{\pgfqpoint{1.940252in}{1.979150in}}{\pgfqpoint{1.936980in}{1.987050in}}{\pgfqpoint{1.931156in}{1.992874in}}%
\pgfpathcurveto{\pgfqpoint{1.925332in}{1.998697in}}{\pgfqpoint{1.917432in}{2.001970in}}{\pgfqpoint{1.909196in}{2.001970in}}%
\pgfpathcurveto{\pgfqpoint{1.900960in}{2.001970in}}{\pgfqpoint{1.893060in}{1.998697in}}{\pgfqpoint{1.887236in}{1.992874in}}%
\pgfpathcurveto{\pgfqpoint{1.881412in}{1.987050in}}{\pgfqpoint{1.878139in}{1.979150in}}{\pgfqpoint{1.878139in}{1.970913in}}%
\pgfpathcurveto{\pgfqpoint{1.878139in}{1.962677in}}{\pgfqpoint{1.881412in}{1.954777in}}{\pgfqpoint{1.887236in}{1.948953in}}%
\pgfpathcurveto{\pgfqpoint{1.893060in}{1.943129in}}{\pgfqpoint{1.900960in}{1.939857in}}{\pgfqpoint{1.909196in}{1.939857in}}%
\pgfpathclose%
\pgfusepath{stroke,fill}%
\end{pgfscope}%
\begin{pgfscope}%
\pgfpathrectangle{\pgfqpoint{0.100000in}{0.212622in}}{\pgfqpoint{3.696000in}{3.696000in}}%
\pgfusepath{clip}%
\pgfsetbuttcap%
\pgfsetroundjoin%
\definecolor{currentfill}{rgb}{0.121569,0.466667,0.705882}%
\pgfsetfillcolor{currentfill}%
\pgfsetfillopacity{0.697846}%
\pgfsetlinewidth{1.003750pt}%
\definecolor{currentstroke}{rgb}{0.121569,0.466667,0.705882}%
\pgfsetstrokecolor{currentstroke}%
\pgfsetstrokeopacity{0.697846}%
\pgfsetdash{}{0pt}%
\pgfpathmoveto{\pgfqpoint{1.898554in}{1.929278in}}%
\pgfpathcurveto{\pgfqpoint{1.906790in}{1.929278in}}{\pgfqpoint{1.914690in}{1.932550in}}{\pgfqpoint{1.920514in}{1.938374in}}%
\pgfpathcurveto{\pgfqpoint{1.926338in}{1.944198in}}{\pgfqpoint{1.929610in}{1.952098in}}{\pgfqpoint{1.929610in}{1.960334in}}%
\pgfpathcurveto{\pgfqpoint{1.929610in}{1.968570in}}{\pgfqpoint{1.926338in}{1.976471in}}{\pgfqpoint{1.920514in}{1.982294in}}%
\pgfpathcurveto{\pgfqpoint{1.914690in}{1.988118in}}{\pgfqpoint{1.906790in}{1.991391in}}{\pgfqpoint{1.898554in}{1.991391in}}%
\pgfpathcurveto{\pgfqpoint{1.890317in}{1.991391in}}{\pgfqpoint{1.882417in}{1.988118in}}{\pgfqpoint{1.876593in}{1.982294in}}%
\pgfpathcurveto{\pgfqpoint{1.870769in}{1.976471in}}{\pgfqpoint{1.867497in}{1.968570in}}{\pgfqpoint{1.867497in}{1.960334in}}%
\pgfpathcurveto{\pgfqpoint{1.867497in}{1.952098in}}{\pgfqpoint{1.870769in}{1.944198in}}{\pgfqpoint{1.876593in}{1.938374in}}%
\pgfpathcurveto{\pgfqpoint{1.882417in}{1.932550in}}{\pgfqpoint{1.890317in}{1.929278in}}{\pgfqpoint{1.898554in}{1.929278in}}%
\pgfpathclose%
\pgfusepath{stroke,fill}%
\end{pgfscope}%
\begin{pgfscope}%
\pgfpathrectangle{\pgfqpoint{0.100000in}{0.212622in}}{\pgfqpoint{3.696000in}{3.696000in}}%
\pgfusepath{clip}%
\pgfsetbuttcap%
\pgfsetroundjoin%
\definecolor{currentfill}{rgb}{0.121569,0.466667,0.705882}%
\pgfsetfillcolor{currentfill}%
\pgfsetfillopacity{0.701720}%
\pgfsetlinewidth{1.003750pt}%
\definecolor{currentstroke}{rgb}{0.121569,0.466667,0.705882}%
\pgfsetstrokecolor{currentstroke}%
\pgfsetstrokeopacity{0.701720}%
\pgfsetdash{}{0pt}%
\pgfpathmoveto{\pgfqpoint{1.898497in}{1.927108in}}%
\pgfpathcurveto{\pgfqpoint{1.906734in}{1.927108in}}{\pgfqpoint{1.914634in}{1.930380in}}{\pgfqpoint{1.920458in}{1.936204in}}%
\pgfpathcurveto{\pgfqpoint{1.926281in}{1.942028in}}{\pgfqpoint{1.929554in}{1.949928in}}{\pgfqpoint{1.929554in}{1.958165in}}%
\pgfpathcurveto{\pgfqpoint{1.929554in}{1.966401in}}{\pgfqpoint{1.926281in}{1.974301in}}{\pgfqpoint{1.920458in}{1.980125in}}%
\pgfpathcurveto{\pgfqpoint{1.914634in}{1.985949in}}{\pgfqpoint{1.906734in}{1.989221in}}{\pgfqpoint{1.898497in}{1.989221in}}%
\pgfpathcurveto{\pgfqpoint{1.890261in}{1.989221in}}{\pgfqpoint{1.882361in}{1.985949in}}{\pgfqpoint{1.876537in}{1.980125in}}%
\pgfpathcurveto{\pgfqpoint{1.870713in}{1.974301in}}{\pgfqpoint{1.867441in}{1.966401in}}{\pgfqpoint{1.867441in}{1.958165in}}%
\pgfpathcurveto{\pgfqpoint{1.867441in}{1.949928in}}{\pgfqpoint{1.870713in}{1.942028in}}{\pgfqpoint{1.876537in}{1.936204in}}%
\pgfpathcurveto{\pgfqpoint{1.882361in}{1.930380in}}{\pgfqpoint{1.890261in}{1.927108in}}{\pgfqpoint{1.898497in}{1.927108in}}%
\pgfpathclose%
\pgfusepath{stroke,fill}%
\end{pgfscope}%
\begin{pgfscope}%
\pgfpathrectangle{\pgfqpoint{0.100000in}{0.212622in}}{\pgfqpoint{3.696000in}{3.696000in}}%
\pgfusepath{clip}%
\pgfsetbuttcap%
\pgfsetroundjoin%
\definecolor{currentfill}{rgb}{0.121569,0.466667,0.705882}%
\pgfsetfillcolor{currentfill}%
\pgfsetfillopacity{0.702463}%
\pgfsetlinewidth{1.003750pt}%
\definecolor{currentstroke}{rgb}{0.121569,0.466667,0.705882}%
\pgfsetstrokecolor{currentstroke}%
\pgfsetstrokeopacity{0.702463}%
\pgfsetdash{}{0pt}%
\pgfpathmoveto{\pgfqpoint{1.903970in}{1.932960in}}%
\pgfpathcurveto{\pgfqpoint{1.912206in}{1.932960in}}{\pgfqpoint{1.920106in}{1.936233in}}{\pgfqpoint{1.925930in}{1.942057in}}%
\pgfpathcurveto{\pgfqpoint{1.931754in}{1.947881in}}{\pgfqpoint{1.935026in}{1.955781in}}{\pgfqpoint{1.935026in}{1.964017in}}%
\pgfpathcurveto{\pgfqpoint{1.935026in}{1.972253in}}{\pgfqpoint{1.931754in}{1.980153in}}{\pgfqpoint{1.925930in}{1.985977in}}%
\pgfpathcurveto{\pgfqpoint{1.920106in}{1.991801in}}{\pgfqpoint{1.912206in}{1.995073in}}{\pgfqpoint{1.903970in}{1.995073in}}%
\pgfpathcurveto{\pgfqpoint{1.895734in}{1.995073in}}{\pgfqpoint{1.887834in}{1.991801in}}{\pgfqpoint{1.882010in}{1.985977in}}%
\pgfpathcurveto{\pgfqpoint{1.876186in}{1.980153in}}{\pgfqpoint{1.872913in}{1.972253in}}{\pgfqpoint{1.872913in}{1.964017in}}%
\pgfpathcurveto{\pgfqpoint{1.872913in}{1.955781in}}{\pgfqpoint{1.876186in}{1.947881in}}{\pgfqpoint{1.882010in}{1.942057in}}%
\pgfpathcurveto{\pgfqpoint{1.887834in}{1.936233in}}{\pgfqpoint{1.895734in}{1.932960in}}{\pgfqpoint{1.903970in}{1.932960in}}%
\pgfpathclose%
\pgfusepath{stroke,fill}%
\end{pgfscope}%
\begin{pgfscope}%
\pgfpathrectangle{\pgfqpoint{0.100000in}{0.212622in}}{\pgfqpoint{3.696000in}{3.696000in}}%
\pgfusepath{clip}%
\pgfsetbuttcap%
\pgfsetroundjoin%
\definecolor{currentfill}{rgb}{0.121569,0.466667,0.705882}%
\pgfsetfillcolor{currentfill}%
\pgfsetfillopacity{0.704914}%
\pgfsetlinewidth{1.003750pt}%
\definecolor{currentstroke}{rgb}{0.121569,0.466667,0.705882}%
\pgfsetstrokecolor{currentstroke}%
\pgfsetstrokeopacity{0.704914}%
\pgfsetdash{}{0pt}%
\pgfpathmoveto{\pgfqpoint{1.907736in}{1.932640in}}%
\pgfpathcurveto{\pgfqpoint{1.915972in}{1.932640in}}{\pgfqpoint{1.923872in}{1.935912in}}{\pgfqpoint{1.929696in}{1.941736in}}%
\pgfpathcurveto{\pgfqpoint{1.935520in}{1.947560in}}{\pgfqpoint{1.938793in}{1.955460in}}{\pgfqpoint{1.938793in}{1.963696in}}%
\pgfpathcurveto{\pgfqpoint{1.938793in}{1.971933in}}{\pgfqpoint{1.935520in}{1.979833in}}{\pgfqpoint{1.929696in}{1.985657in}}%
\pgfpathcurveto{\pgfqpoint{1.923872in}{1.991481in}}{\pgfqpoint{1.915972in}{1.994753in}}{\pgfqpoint{1.907736in}{1.994753in}}%
\pgfpathcurveto{\pgfqpoint{1.899500in}{1.994753in}}{\pgfqpoint{1.891600in}{1.991481in}}{\pgfqpoint{1.885776in}{1.985657in}}%
\pgfpathcurveto{\pgfqpoint{1.879952in}{1.979833in}}{\pgfqpoint{1.876680in}{1.971933in}}{\pgfqpoint{1.876680in}{1.963696in}}%
\pgfpathcurveto{\pgfqpoint{1.876680in}{1.955460in}}{\pgfqpoint{1.879952in}{1.947560in}}{\pgfqpoint{1.885776in}{1.941736in}}%
\pgfpathcurveto{\pgfqpoint{1.891600in}{1.935912in}}{\pgfqpoint{1.899500in}{1.932640in}}{\pgfqpoint{1.907736in}{1.932640in}}%
\pgfpathclose%
\pgfusepath{stroke,fill}%
\end{pgfscope}%
\begin{pgfscope}%
\pgfpathrectangle{\pgfqpoint{0.100000in}{0.212622in}}{\pgfqpoint{3.696000in}{3.696000in}}%
\pgfusepath{clip}%
\pgfsetbuttcap%
\pgfsetroundjoin%
\definecolor{currentfill}{rgb}{0.121569,0.466667,0.705882}%
\pgfsetfillcolor{currentfill}%
\pgfsetfillopacity{0.706275}%
\pgfsetlinewidth{1.003750pt}%
\definecolor{currentstroke}{rgb}{0.121569,0.466667,0.705882}%
\pgfsetstrokecolor{currentstroke}%
\pgfsetstrokeopacity{0.706275}%
\pgfsetdash{}{0pt}%
\pgfpathmoveto{\pgfqpoint{1.898584in}{1.922471in}}%
\pgfpathcurveto{\pgfqpoint{1.906821in}{1.922471in}}{\pgfqpoint{1.914721in}{1.925743in}}{\pgfqpoint{1.920545in}{1.931567in}}%
\pgfpathcurveto{\pgfqpoint{1.926369in}{1.937391in}}{\pgfqpoint{1.929641in}{1.945291in}}{\pgfqpoint{1.929641in}{1.953527in}}%
\pgfpathcurveto{\pgfqpoint{1.929641in}{1.961764in}}{\pgfqpoint{1.926369in}{1.969664in}}{\pgfqpoint{1.920545in}{1.975488in}}%
\pgfpathcurveto{\pgfqpoint{1.914721in}{1.981312in}}{\pgfqpoint{1.906821in}{1.984584in}}{\pgfqpoint{1.898584in}{1.984584in}}%
\pgfpathcurveto{\pgfqpoint{1.890348in}{1.984584in}}{\pgfqpoint{1.882448in}{1.981312in}}{\pgfqpoint{1.876624in}{1.975488in}}%
\pgfpathcurveto{\pgfqpoint{1.870800in}{1.969664in}}{\pgfqpoint{1.867528in}{1.961764in}}{\pgfqpoint{1.867528in}{1.953527in}}%
\pgfpathcurveto{\pgfqpoint{1.867528in}{1.945291in}}{\pgfqpoint{1.870800in}{1.937391in}}{\pgfqpoint{1.876624in}{1.931567in}}%
\pgfpathcurveto{\pgfqpoint{1.882448in}{1.925743in}}{\pgfqpoint{1.890348in}{1.922471in}}{\pgfqpoint{1.898584in}{1.922471in}}%
\pgfpathclose%
\pgfusepath{stroke,fill}%
\end{pgfscope}%
\begin{pgfscope}%
\pgfpathrectangle{\pgfqpoint{0.100000in}{0.212622in}}{\pgfqpoint{3.696000in}{3.696000in}}%
\pgfusepath{clip}%
\pgfsetbuttcap%
\pgfsetroundjoin%
\definecolor{currentfill}{rgb}{0.121569,0.466667,0.705882}%
\pgfsetfillcolor{currentfill}%
\pgfsetfillopacity{0.709611}%
\pgfsetlinewidth{1.003750pt}%
\definecolor{currentstroke}{rgb}{0.121569,0.466667,0.705882}%
\pgfsetstrokecolor{currentstroke}%
\pgfsetstrokeopacity{0.709611}%
\pgfsetdash{}{0pt}%
\pgfpathmoveto{\pgfqpoint{1.904933in}{1.927130in}}%
\pgfpathcurveto{\pgfqpoint{1.913170in}{1.927130in}}{\pgfqpoint{1.921070in}{1.930402in}}{\pgfqpoint{1.926894in}{1.936226in}}%
\pgfpathcurveto{\pgfqpoint{1.932717in}{1.942050in}}{\pgfqpoint{1.935990in}{1.949950in}}{\pgfqpoint{1.935990in}{1.958186in}}%
\pgfpathcurveto{\pgfqpoint{1.935990in}{1.966422in}}{\pgfqpoint{1.932717in}{1.974322in}}{\pgfqpoint{1.926894in}{1.980146in}}%
\pgfpathcurveto{\pgfqpoint{1.921070in}{1.985970in}}{\pgfqpoint{1.913170in}{1.989243in}}{\pgfqpoint{1.904933in}{1.989243in}}%
\pgfpathcurveto{\pgfqpoint{1.896697in}{1.989243in}}{\pgfqpoint{1.888797in}{1.985970in}}{\pgfqpoint{1.882973in}{1.980146in}}%
\pgfpathcurveto{\pgfqpoint{1.877149in}{1.974322in}}{\pgfqpoint{1.873877in}{1.966422in}}{\pgfqpoint{1.873877in}{1.958186in}}%
\pgfpathcurveto{\pgfqpoint{1.873877in}{1.949950in}}{\pgfqpoint{1.877149in}{1.942050in}}{\pgfqpoint{1.882973in}{1.936226in}}%
\pgfpathcurveto{\pgfqpoint{1.888797in}{1.930402in}}{\pgfqpoint{1.896697in}{1.927130in}}{\pgfqpoint{1.904933in}{1.927130in}}%
\pgfpathclose%
\pgfusepath{stroke,fill}%
\end{pgfscope}%
\begin{pgfscope}%
\pgfpathrectangle{\pgfqpoint{0.100000in}{0.212622in}}{\pgfqpoint{3.696000in}{3.696000in}}%
\pgfusepath{clip}%
\pgfsetbuttcap%
\pgfsetroundjoin%
\definecolor{currentfill}{rgb}{0.121569,0.466667,0.705882}%
\pgfsetfillcolor{currentfill}%
\pgfsetfillopacity{0.710781}%
\pgfsetlinewidth{1.003750pt}%
\definecolor{currentstroke}{rgb}{0.121569,0.466667,0.705882}%
\pgfsetstrokecolor{currentstroke}%
\pgfsetstrokeopacity{0.710781}%
\pgfsetdash{}{0pt}%
\pgfpathmoveto{\pgfqpoint{1.894679in}{1.917048in}}%
\pgfpathcurveto{\pgfqpoint{1.902915in}{1.917048in}}{\pgfqpoint{1.910815in}{1.920320in}}{\pgfqpoint{1.916639in}{1.926144in}}%
\pgfpathcurveto{\pgfqpoint{1.922463in}{1.931968in}}{\pgfqpoint{1.925735in}{1.939868in}}{\pgfqpoint{1.925735in}{1.948104in}}%
\pgfpathcurveto{\pgfqpoint{1.925735in}{1.956341in}}{\pgfqpoint{1.922463in}{1.964241in}}{\pgfqpoint{1.916639in}{1.970065in}}%
\pgfpathcurveto{\pgfqpoint{1.910815in}{1.975889in}}{\pgfqpoint{1.902915in}{1.979161in}}{\pgfqpoint{1.894679in}{1.979161in}}%
\pgfpathcurveto{\pgfqpoint{1.886442in}{1.979161in}}{\pgfqpoint{1.878542in}{1.975889in}}{\pgfqpoint{1.872718in}{1.970065in}}%
\pgfpathcurveto{\pgfqpoint{1.866895in}{1.964241in}}{\pgfqpoint{1.863622in}{1.956341in}}{\pgfqpoint{1.863622in}{1.948104in}}%
\pgfpathcurveto{\pgfqpoint{1.863622in}{1.939868in}}{\pgfqpoint{1.866895in}{1.931968in}}{\pgfqpoint{1.872718in}{1.926144in}}%
\pgfpathcurveto{\pgfqpoint{1.878542in}{1.920320in}}{\pgfqpoint{1.886442in}{1.917048in}}{\pgfqpoint{1.894679in}{1.917048in}}%
\pgfpathclose%
\pgfusepath{stroke,fill}%
\end{pgfscope}%
\begin{pgfscope}%
\pgfpathrectangle{\pgfqpoint{0.100000in}{0.212622in}}{\pgfqpoint{3.696000in}{3.696000in}}%
\pgfusepath{clip}%
\pgfsetbuttcap%
\pgfsetroundjoin%
\definecolor{currentfill}{rgb}{0.121569,0.466667,0.705882}%
\pgfsetfillcolor{currentfill}%
\pgfsetfillopacity{0.711262}%
\pgfsetlinewidth{1.003750pt}%
\definecolor{currentstroke}{rgb}{0.121569,0.466667,0.705882}%
\pgfsetstrokecolor{currentstroke}%
\pgfsetstrokeopacity{0.711262}%
\pgfsetdash{}{0pt}%
\pgfpathmoveto{\pgfqpoint{2.029282in}{2.001217in}}%
\pgfpathcurveto{\pgfqpoint{2.037518in}{2.001217in}}{\pgfqpoint{2.045418in}{2.004490in}}{\pgfqpoint{2.051242in}{2.010314in}}%
\pgfpathcurveto{\pgfqpoint{2.057066in}{2.016138in}}{\pgfqpoint{2.060338in}{2.024038in}}{\pgfqpoint{2.060338in}{2.032274in}}%
\pgfpathcurveto{\pgfqpoint{2.060338in}{2.040510in}}{\pgfqpoint{2.057066in}{2.048410in}}{\pgfqpoint{2.051242in}{2.054234in}}%
\pgfpathcurveto{\pgfqpoint{2.045418in}{2.060058in}}{\pgfqpoint{2.037518in}{2.063330in}}{\pgfqpoint{2.029282in}{2.063330in}}%
\pgfpathcurveto{\pgfqpoint{2.021045in}{2.063330in}}{\pgfqpoint{2.013145in}{2.060058in}}{\pgfqpoint{2.007321in}{2.054234in}}%
\pgfpathcurveto{\pgfqpoint{2.001497in}{2.048410in}}{\pgfqpoint{1.998225in}{2.040510in}}{\pgfqpoint{1.998225in}{2.032274in}}%
\pgfpathcurveto{\pgfqpoint{1.998225in}{2.024038in}}{\pgfqpoint{2.001497in}{2.016138in}}{\pgfqpoint{2.007321in}{2.010314in}}%
\pgfpathcurveto{\pgfqpoint{2.013145in}{2.004490in}}{\pgfqpoint{2.021045in}{2.001217in}}{\pgfqpoint{2.029282in}{2.001217in}}%
\pgfpathclose%
\pgfusepath{stroke,fill}%
\end{pgfscope}%
\begin{pgfscope}%
\pgfpathrectangle{\pgfqpoint{0.100000in}{0.212622in}}{\pgfqpoint{3.696000in}{3.696000in}}%
\pgfusepath{clip}%
\pgfsetbuttcap%
\pgfsetroundjoin%
\definecolor{currentfill}{rgb}{0.121569,0.466667,0.705882}%
\pgfsetfillcolor{currentfill}%
\pgfsetfillopacity{0.712051}%
\pgfsetlinewidth{1.003750pt}%
\definecolor{currentstroke}{rgb}{0.121569,0.466667,0.705882}%
\pgfsetstrokecolor{currentstroke}%
\pgfsetstrokeopacity{0.712051}%
\pgfsetdash{}{0pt}%
\pgfpathmoveto{\pgfqpoint{2.016450in}{1.985923in}}%
\pgfpathcurveto{\pgfqpoint{2.024686in}{1.985923in}}{\pgfqpoint{2.032586in}{1.989195in}}{\pgfqpoint{2.038410in}{1.995019in}}%
\pgfpathcurveto{\pgfqpoint{2.044234in}{2.000843in}}{\pgfqpoint{2.047507in}{2.008743in}}{\pgfqpoint{2.047507in}{2.016979in}}%
\pgfpathcurveto{\pgfqpoint{2.047507in}{2.025215in}}{\pgfqpoint{2.044234in}{2.033115in}}{\pgfqpoint{2.038410in}{2.038939in}}%
\pgfpathcurveto{\pgfqpoint{2.032586in}{2.044763in}}{\pgfqpoint{2.024686in}{2.048036in}}{\pgfqpoint{2.016450in}{2.048036in}}%
\pgfpathcurveto{\pgfqpoint{2.008214in}{2.048036in}}{\pgfqpoint{2.000314in}{2.044763in}}{\pgfqpoint{1.994490in}{2.038939in}}%
\pgfpathcurveto{\pgfqpoint{1.988666in}{2.033115in}}{\pgfqpoint{1.985394in}{2.025215in}}{\pgfqpoint{1.985394in}{2.016979in}}%
\pgfpathcurveto{\pgfqpoint{1.985394in}{2.008743in}}{\pgfqpoint{1.988666in}{2.000843in}}{\pgfqpoint{1.994490in}{1.995019in}}%
\pgfpathcurveto{\pgfqpoint{2.000314in}{1.989195in}}{\pgfqpoint{2.008214in}{1.985923in}}{\pgfqpoint{2.016450in}{1.985923in}}%
\pgfpathclose%
\pgfusepath{stroke,fill}%
\end{pgfscope}%
\begin{pgfscope}%
\pgfpathrectangle{\pgfqpoint{0.100000in}{0.212622in}}{\pgfqpoint{3.696000in}{3.696000in}}%
\pgfusepath{clip}%
\pgfsetbuttcap%
\pgfsetroundjoin%
\definecolor{currentfill}{rgb}{0.121569,0.466667,0.705882}%
\pgfsetfillcolor{currentfill}%
\pgfsetfillopacity{0.714851}%
\pgfsetlinewidth{1.003750pt}%
\definecolor{currentstroke}{rgb}{0.121569,0.466667,0.705882}%
\pgfsetstrokecolor{currentstroke}%
\pgfsetstrokeopacity{0.714851}%
\pgfsetdash{}{0pt}%
\pgfpathmoveto{\pgfqpoint{1.894832in}{1.912377in}}%
\pgfpathcurveto{\pgfqpoint{1.903068in}{1.912377in}}{\pgfqpoint{1.910968in}{1.915649in}}{\pgfqpoint{1.916792in}{1.921473in}}%
\pgfpathcurveto{\pgfqpoint{1.922616in}{1.927297in}}{\pgfqpoint{1.925889in}{1.935197in}}{\pgfqpoint{1.925889in}{1.943434in}}%
\pgfpathcurveto{\pgfqpoint{1.925889in}{1.951670in}}{\pgfqpoint{1.922616in}{1.959570in}}{\pgfqpoint{1.916792in}{1.965394in}}%
\pgfpathcurveto{\pgfqpoint{1.910968in}{1.971218in}}{\pgfqpoint{1.903068in}{1.974490in}}{\pgfqpoint{1.894832in}{1.974490in}}%
\pgfpathcurveto{\pgfqpoint{1.886596in}{1.974490in}}{\pgfqpoint{1.878696in}{1.971218in}}{\pgfqpoint{1.872872in}{1.965394in}}%
\pgfpathcurveto{\pgfqpoint{1.867048in}{1.959570in}}{\pgfqpoint{1.863776in}{1.951670in}}{\pgfqpoint{1.863776in}{1.943434in}}%
\pgfpathcurveto{\pgfqpoint{1.863776in}{1.935197in}}{\pgfqpoint{1.867048in}{1.927297in}}{\pgfqpoint{1.872872in}{1.921473in}}%
\pgfpathcurveto{\pgfqpoint{1.878696in}{1.915649in}}{\pgfqpoint{1.886596in}{1.912377in}}{\pgfqpoint{1.894832in}{1.912377in}}%
\pgfpathclose%
\pgfusepath{stroke,fill}%
\end{pgfscope}%
\begin{pgfscope}%
\pgfpathrectangle{\pgfqpoint{0.100000in}{0.212622in}}{\pgfqpoint{3.696000in}{3.696000in}}%
\pgfusepath{clip}%
\pgfsetbuttcap%
\pgfsetroundjoin%
\definecolor{currentfill}{rgb}{0.121569,0.466667,0.705882}%
\pgfsetfillcolor{currentfill}%
\pgfsetfillopacity{0.715200}%
\pgfsetlinewidth{1.003750pt}%
\definecolor{currentstroke}{rgb}{0.121569,0.466667,0.705882}%
\pgfsetstrokecolor{currentstroke}%
\pgfsetstrokeopacity{0.715200}%
\pgfsetdash{}{0pt}%
\pgfpathmoveto{\pgfqpoint{1.920942in}{1.931142in}}%
\pgfpathcurveto{\pgfqpoint{1.929178in}{1.931142in}}{\pgfqpoint{1.937079in}{1.934415in}}{\pgfqpoint{1.942902in}{1.940239in}}%
\pgfpathcurveto{\pgfqpoint{1.948726in}{1.946063in}}{\pgfqpoint{1.951999in}{1.953963in}}{\pgfqpoint{1.951999in}{1.962199in}}%
\pgfpathcurveto{\pgfqpoint{1.951999in}{1.970435in}}{\pgfqpoint{1.948726in}{1.978335in}}{\pgfqpoint{1.942902in}{1.984159in}}%
\pgfpathcurveto{\pgfqpoint{1.937079in}{1.989983in}}{\pgfqpoint{1.929178in}{1.993255in}}{\pgfqpoint{1.920942in}{1.993255in}}%
\pgfpathcurveto{\pgfqpoint{1.912706in}{1.993255in}}{\pgfqpoint{1.904806in}{1.989983in}}{\pgfqpoint{1.898982in}{1.984159in}}%
\pgfpathcurveto{\pgfqpoint{1.893158in}{1.978335in}}{\pgfqpoint{1.889886in}{1.970435in}}{\pgfqpoint{1.889886in}{1.962199in}}%
\pgfpathcurveto{\pgfqpoint{1.889886in}{1.953963in}}{\pgfqpoint{1.893158in}{1.946063in}}{\pgfqpoint{1.898982in}{1.940239in}}%
\pgfpathcurveto{\pgfqpoint{1.904806in}{1.934415in}}{\pgfqpoint{1.912706in}{1.931142in}}{\pgfqpoint{1.920942in}{1.931142in}}%
\pgfpathclose%
\pgfusepath{stroke,fill}%
\end{pgfscope}%
\begin{pgfscope}%
\pgfpathrectangle{\pgfqpoint{0.100000in}{0.212622in}}{\pgfqpoint{3.696000in}{3.696000in}}%
\pgfusepath{clip}%
\pgfsetbuttcap%
\pgfsetroundjoin%
\definecolor{currentfill}{rgb}{0.121569,0.466667,0.705882}%
\pgfsetfillcolor{currentfill}%
\pgfsetfillopacity{0.715204}%
\pgfsetlinewidth{1.003750pt}%
\definecolor{currentstroke}{rgb}{0.121569,0.466667,0.705882}%
\pgfsetstrokecolor{currentstroke}%
\pgfsetstrokeopacity{0.715204}%
\pgfsetdash{}{0pt}%
\pgfpathmoveto{\pgfqpoint{1.992190in}{1.967725in}}%
\pgfpathcurveto{\pgfqpoint{2.000426in}{1.967725in}}{\pgfqpoint{2.008326in}{1.970997in}}{\pgfqpoint{2.014150in}{1.976821in}}%
\pgfpathcurveto{\pgfqpoint{2.019974in}{1.982645in}}{\pgfqpoint{2.023246in}{1.990545in}}{\pgfqpoint{2.023246in}{1.998781in}}%
\pgfpathcurveto{\pgfqpoint{2.023246in}{2.007017in}}{\pgfqpoint{2.019974in}{2.014917in}}{\pgfqpoint{2.014150in}{2.020741in}}%
\pgfpathcurveto{\pgfqpoint{2.008326in}{2.026565in}}{\pgfqpoint{2.000426in}{2.029838in}}{\pgfqpoint{1.992190in}{2.029838in}}%
\pgfpathcurveto{\pgfqpoint{1.983953in}{2.029838in}}{\pgfqpoint{1.976053in}{2.026565in}}{\pgfqpoint{1.970229in}{2.020741in}}%
\pgfpathcurveto{\pgfqpoint{1.964405in}{2.014917in}}{\pgfqpoint{1.961133in}{2.007017in}}{\pgfqpoint{1.961133in}{1.998781in}}%
\pgfpathcurveto{\pgfqpoint{1.961133in}{1.990545in}}{\pgfqpoint{1.964405in}{1.982645in}}{\pgfqpoint{1.970229in}{1.976821in}}%
\pgfpathcurveto{\pgfqpoint{1.976053in}{1.970997in}}{\pgfqpoint{1.983953in}{1.967725in}}{\pgfqpoint{1.992190in}{1.967725in}}%
\pgfpathclose%
\pgfusepath{stroke,fill}%
\end{pgfscope}%
\begin{pgfscope}%
\pgfpathrectangle{\pgfqpoint{0.100000in}{0.212622in}}{\pgfqpoint{3.696000in}{3.696000in}}%
\pgfusepath{clip}%
\pgfsetbuttcap%
\pgfsetroundjoin%
\definecolor{currentfill}{rgb}{0.121569,0.466667,0.705882}%
\pgfsetfillcolor{currentfill}%
\pgfsetfillopacity{0.715821}%
\pgfsetlinewidth{1.003750pt}%
\definecolor{currentstroke}{rgb}{0.121569,0.466667,0.705882}%
\pgfsetstrokecolor{currentstroke}%
\pgfsetstrokeopacity{0.715821}%
\pgfsetdash{}{0pt}%
\pgfpathmoveto{\pgfqpoint{1.900411in}{1.916597in}}%
\pgfpathcurveto{\pgfqpoint{1.908648in}{1.916597in}}{\pgfqpoint{1.916548in}{1.919870in}}{\pgfqpoint{1.922372in}{1.925694in}}%
\pgfpathcurveto{\pgfqpoint{1.928196in}{1.931517in}}{\pgfqpoint{1.931468in}{1.939418in}}{\pgfqpoint{1.931468in}{1.947654in}}%
\pgfpathcurveto{\pgfqpoint{1.931468in}{1.955890in}}{\pgfqpoint{1.928196in}{1.963790in}}{\pgfqpoint{1.922372in}{1.969614in}}%
\pgfpathcurveto{\pgfqpoint{1.916548in}{1.975438in}}{\pgfqpoint{1.908648in}{1.978710in}}{\pgfqpoint{1.900411in}{1.978710in}}%
\pgfpathcurveto{\pgfqpoint{1.892175in}{1.978710in}}{\pgfqpoint{1.884275in}{1.975438in}}{\pgfqpoint{1.878451in}{1.969614in}}%
\pgfpathcurveto{\pgfqpoint{1.872627in}{1.963790in}}{\pgfqpoint{1.869355in}{1.955890in}}{\pgfqpoint{1.869355in}{1.947654in}}%
\pgfpathcurveto{\pgfqpoint{1.869355in}{1.939418in}}{\pgfqpoint{1.872627in}{1.931517in}}{\pgfqpoint{1.878451in}{1.925694in}}%
\pgfpathcurveto{\pgfqpoint{1.884275in}{1.919870in}}{\pgfqpoint{1.892175in}{1.916597in}}{\pgfqpoint{1.900411in}{1.916597in}}%
\pgfpathclose%
\pgfusepath{stroke,fill}%
\end{pgfscope}%
\begin{pgfscope}%
\pgfpathrectangle{\pgfqpoint{0.100000in}{0.212622in}}{\pgfqpoint{3.696000in}{3.696000in}}%
\pgfusepath{clip}%
\pgfsetbuttcap%
\pgfsetroundjoin%
\definecolor{currentfill}{rgb}{0.121569,0.466667,0.705882}%
\pgfsetfillcolor{currentfill}%
\pgfsetfillopacity{0.717679}%
\pgfsetlinewidth{1.003750pt}%
\definecolor{currentstroke}{rgb}{0.121569,0.466667,0.705882}%
\pgfsetstrokecolor{currentstroke}%
\pgfsetstrokeopacity{0.717679}%
\pgfsetdash{}{0pt}%
\pgfpathmoveto{\pgfqpoint{1.902644in}{1.913099in}}%
\pgfpathcurveto{\pgfqpoint{1.910880in}{1.913099in}}{\pgfqpoint{1.918780in}{1.916372in}}{\pgfqpoint{1.924604in}{1.922196in}}%
\pgfpathcurveto{\pgfqpoint{1.930428in}{1.928019in}}{\pgfqpoint{1.933700in}{1.935919in}}{\pgfqpoint{1.933700in}{1.944156in}}%
\pgfpathcurveto{\pgfqpoint{1.933700in}{1.952392in}}{\pgfqpoint{1.930428in}{1.960292in}}{\pgfqpoint{1.924604in}{1.966116in}}%
\pgfpathcurveto{\pgfqpoint{1.918780in}{1.971940in}}{\pgfqpoint{1.910880in}{1.975212in}}{\pgfqpoint{1.902644in}{1.975212in}}%
\pgfpathcurveto{\pgfqpoint{1.894408in}{1.975212in}}{\pgfqpoint{1.886508in}{1.971940in}}{\pgfqpoint{1.880684in}{1.966116in}}%
\pgfpathcurveto{\pgfqpoint{1.874860in}{1.960292in}}{\pgfqpoint{1.871587in}{1.952392in}}{\pgfqpoint{1.871587in}{1.944156in}}%
\pgfpathcurveto{\pgfqpoint{1.871587in}{1.935919in}}{\pgfqpoint{1.874860in}{1.928019in}}{\pgfqpoint{1.880684in}{1.922196in}}%
\pgfpathcurveto{\pgfqpoint{1.886508in}{1.916372in}}{\pgfqpoint{1.894408in}{1.913099in}}{\pgfqpoint{1.902644in}{1.913099in}}%
\pgfpathclose%
\pgfusepath{stroke,fill}%
\end{pgfscope}%
\begin{pgfscope}%
\pgfpathrectangle{\pgfqpoint{0.100000in}{0.212622in}}{\pgfqpoint{3.696000in}{3.696000in}}%
\pgfusepath{clip}%
\pgfsetbuttcap%
\pgfsetroundjoin%
\definecolor{currentfill}{rgb}{0.121569,0.466667,0.705882}%
\pgfsetfillcolor{currentfill}%
\pgfsetfillopacity{0.720890}%
\pgfsetlinewidth{1.003750pt}%
\definecolor{currentstroke}{rgb}{0.121569,0.466667,0.705882}%
\pgfsetstrokecolor{currentstroke}%
\pgfsetstrokeopacity{0.720890}%
\pgfsetdash{}{0pt}%
\pgfpathmoveto{\pgfqpoint{1.928290in}{1.936726in}}%
\pgfpathcurveto{\pgfqpoint{1.936527in}{1.936726in}}{\pgfqpoint{1.944427in}{1.939999in}}{\pgfqpoint{1.950251in}{1.945822in}}%
\pgfpathcurveto{\pgfqpoint{1.956075in}{1.951646in}}{\pgfqpoint{1.959347in}{1.959546in}}{\pgfqpoint{1.959347in}{1.967783in}}%
\pgfpathcurveto{\pgfqpoint{1.959347in}{1.976019in}}{\pgfqpoint{1.956075in}{1.983919in}}{\pgfqpoint{1.950251in}{1.989743in}}%
\pgfpathcurveto{\pgfqpoint{1.944427in}{1.995567in}}{\pgfqpoint{1.936527in}{1.998839in}}{\pgfqpoint{1.928290in}{1.998839in}}%
\pgfpathcurveto{\pgfqpoint{1.920054in}{1.998839in}}{\pgfqpoint{1.912154in}{1.995567in}}{\pgfqpoint{1.906330in}{1.989743in}}%
\pgfpathcurveto{\pgfqpoint{1.900506in}{1.983919in}}{\pgfqpoint{1.897234in}{1.976019in}}{\pgfqpoint{1.897234in}{1.967783in}}%
\pgfpathcurveto{\pgfqpoint{1.897234in}{1.959546in}}{\pgfqpoint{1.900506in}{1.951646in}}{\pgfqpoint{1.906330in}{1.945822in}}%
\pgfpathcurveto{\pgfqpoint{1.912154in}{1.939999in}}{\pgfqpoint{1.920054in}{1.936726in}}{\pgfqpoint{1.928290in}{1.936726in}}%
\pgfpathclose%
\pgfusepath{stroke,fill}%
\end{pgfscope}%
\begin{pgfscope}%
\pgfpathrectangle{\pgfqpoint{0.100000in}{0.212622in}}{\pgfqpoint{3.696000in}{3.696000in}}%
\pgfusepath{clip}%
\pgfsetbuttcap%
\pgfsetroundjoin%
\definecolor{currentfill}{rgb}{0.121569,0.466667,0.705882}%
\pgfsetfillcolor{currentfill}%
\pgfsetfillopacity{0.721477}%
\pgfsetlinewidth{1.003750pt}%
\definecolor{currentstroke}{rgb}{0.121569,0.466667,0.705882}%
\pgfsetstrokecolor{currentstroke}%
\pgfsetstrokeopacity{0.721477}%
\pgfsetdash{}{0pt}%
\pgfpathmoveto{\pgfqpoint{1.961456in}{1.947413in}}%
\pgfpathcurveto{\pgfqpoint{1.969692in}{1.947413in}}{\pgfqpoint{1.977592in}{1.950685in}}{\pgfqpoint{1.983416in}{1.956509in}}%
\pgfpathcurveto{\pgfqpoint{1.989240in}{1.962333in}}{\pgfqpoint{1.992512in}{1.970233in}}{\pgfqpoint{1.992512in}{1.978469in}}%
\pgfpathcurveto{\pgfqpoint{1.992512in}{1.986706in}}{\pgfqpoint{1.989240in}{1.994606in}}{\pgfqpoint{1.983416in}{2.000430in}}%
\pgfpathcurveto{\pgfqpoint{1.977592in}{2.006253in}}{\pgfqpoint{1.969692in}{2.009526in}}{\pgfqpoint{1.961456in}{2.009526in}}%
\pgfpathcurveto{\pgfqpoint{1.953220in}{2.009526in}}{\pgfqpoint{1.945320in}{2.006253in}}{\pgfqpoint{1.939496in}{2.000430in}}%
\pgfpathcurveto{\pgfqpoint{1.933672in}{1.994606in}}{\pgfqpoint{1.930399in}{1.986706in}}{\pgfqpoint{1.930399in}{1.978469in}}%
\pgfpathcurveto{\pgfqpoint{1.930399in}{1.970233in}}{\pgfqpoint{1.933672in}{1.962333in}}{\pgfqpoint{1.939496in}{1.956509in}}%
\pgfpathcurveto{\pgfqpoint{1.945320in}{1.950685in}}{\pgfqpoint{1.953220in}{1.947413in}}{\pgfqpoint{1.961456in}{1.947413in}}%
\pgfpathclose%
\pgfusepath{stroke,fill}%
\end{pgfscope}%
\begin{pgfscope}%
\pgfpathrectangle{\pgfqpoint{0.100000in}{0.212622in}}{\pgfqpoint{3.696000in}{3.696000in}}%
\pgfusepath{clip}%
\pgfsetbuttcap%
\pgfsetroundjoin%
\definecolor{currentfill}{rgb}{0.121569,0.466667,0.705882}%
\pgfsetfillcolor{currentfill}%
\pgfsetfillopacity{0.723245}%
\pgfsetlinewidth{1.003750pt}%
\definecolor{currentstroke}{rgb}{0.121569,0.466667,0.705882}%
\pgfsetstrokecolor{currentstroke}%
\pgfsetstrokeopacity{0.723245}%
\pgfsetdash{}{0pt}%
\pgfpathmoveto{\pgfqpoint{2.009348in}{1.983462in}}%
\pgfpathcurveto{\pgfqpoint{2.017584in}{1.983462in}}{\pgfqpoint{2.025484in}{1.986734in}}{\pgfqpoint{2.031308in}{1.992558in}}%
\pgfpathcurveto{\pgfqpoint{2.037132in}{1.998382in}}{\pgfqpoint{2.040404in}{2.006282in}}{\pgfqpoint{2.040404in}{2.014518in}}%
\pgfpathcurveto{\pgfqpoint{2.040404in}{2.022755in}}{\pgfqpoint{2.037132in}{2.030655in}}{\pgfqpoint{2.031308in}{2.036479in}}%
\pgfpathcurveto{\pgfqpoint{2.025484in}{2.042303in}}{\pgfqpoint{2.017584in}{2.045575in}}{\pgfqpoint{2.009348in}{2.045575in}}%
\pgfpathcurveto{\pgfqpoint{2.001111in}{2.045575in}}{\pgfqpoint{1.993211in}{2.042303in}}{\pgfqpoint{1.987387in}{2.036479in}}%
\pgfpathcurveto{\pgfqpoint{1.981564in}{2.030655in}}{\pgfqpoint{1.978291in}{2.022755in}}{\pgfqpoint{1.978291in}{2.014518in}}%
\pgfpathcurveto{\pgfqpoint{1.978291in}{2.006282in}}{\pgfqpoint{1.981564in}{1.998382in}}{\pgfqpoint{1.987387in}{1.992558in}}%
\pgfpathcurveto{\pgfqpoint{1.993211in}{1.986734in}}{\pgfqpoint{2.001111in}{1.983462in}}{\pgfqpoint{2.009348in}{1.983462in}}%
\pgfpathclose%
\pgfusepath{stroke,fill}%
\end{pgfscope}%
\begin{pgfscope}%
\pgfpathrectangle{\pgfqpoint{0.100000in}{0.212622in}}{\pgfqpoint{3.696000in}{3.696000in}}%
\pgfusepath{clip}%
\pgfsetbuttcap%
\pgfsetroundjoin%
\definecolor{currentfill}{rgb}{0.121569,0.466667,0.705882}%
\pgfsetfillcolor{currentfill}%
\pgfsetfillopacity{0.724710}%
\pgfsetlinewidth{1.003750pt}%
\definecolor{currentstroke}{rgb}{0.121569,0.466667,0.705882}%
\pgfsetstrokecolor{currentstroke}%
\pgfsetstrokeopacity{0.724710}%
\pgfsetdash{}{0pt}%
\pgfpathmoveto{\pgfqpoint{1.997558in}{1.967067in}}%
\pgfpathcurveto{\pgfqpoint{2.005794in}{1.967067in}}{\pgfqpoint{2.013694in}{1.970340in}}{\pgfqpoint{2.019518in}{1.976164in}}%
\pgfpathcurveto{\pgfqpoint{2.025342in}{1.981988in}}{\pgfqpoint{2.028615in}{1.989888in}}{\pgfqpoint{2.028615in}{1.998124in}}%
\pgfpathcurveto{\pgfqpoint{2.028615in}{2.006360in}}{\pgfqpoint{2.025342in}{2.014260in}}{\pgfqpoint{2.019518in}{2.020084in}}%
\pgfpathcurveto{\pgfqpoint{2.013694in}{2.025908in}}{\pgfqpoint{2.005794in}{2.029180in}}{\pgfqpoint{1.997558in}{2.029180in}}%
\pgfpathcurveto{\pgfqpoint{1.989322in}{2.029180in}}{\pgfqpoint{1.981422in}{2.025908in}}{\pgfqpoint{1.975598in}{2.020084in}}%
\pgfpathcurveto{\pgfqpoint{1.969774in}{2.014260in}}{\pgfqpoint{1.966502in}{2.006360in}}{\pgfqpoint{1.966502in}{1.998124in}}%
\pgfpathcurveto{\pgfqpoint{1.966502in}{1.989888in}}{\pgfqpoint{1.969774in}{1.981988in}}{\pgfqpoint{1.975598in}{1.976164in}}%
\pgfpathcurveto{\pgfqpoint{1.981422in}{1.970340in}}{\pgfqpoint{1.989322in}{1.967067in}}{\pgfqpoint{1.997558in}{1.967067in}}%
\pgfpathclose%
\pgfusepath{stroke,fill}%
\end{pgfscope}%
\begin{pgfscope}%
\pgfpathrectangle{\pgfqpoint{0.100000in}{0.212622in}}{\pgfqpoint{3.696000in}{3.696000in}}%
\pgfusepath{clip}%
\pgfsetbuttcap%
\pgfsetroundjoin%
\definecolor{currentfill}{rgb}{0.121569,0.466667,0.705882}%
\pgfsetfillcolor{currentfill}%
\pgfsetfillopacity{0.725013}%
\pgfsetlinewidth{1.003750pt}%
\definecolor{currentstroke}{rgb}{0.121569,0.466667,0.705882}%
\pgfsetstrokecolor{currentstroke}%
\pgfsetstrokeopacity{0.725013}%
\pgfsetdash{}{0pt}%
\pgfpathmoveto{\pgfqpoint{3.156939in}{2.710322in}}%
\pgfpathcurveto{\pgfqpoint{3.165175in}{2.710322in}}{\pgfqpoint{3.173075in}{2.713594in}}{\pgfqpoint{3.178899in}{2.719418in}}%
\pgfpathcurveto{\pgfqpoint{3.184723in}{2.725242in}}{\pgfqpoint{3.187995in}{2.733142in}}{\pgfqpoint{3.187995in}{2.741378in}}%
\pgfpathcurveto{\pgfqpoint{3.187995in}{2.749614in}}{\pgfqpoint{3.184723in}{2.757514in}}{\pgfqpoint{3.178899in}{2.763338in}}%
\pgfpathcurveto{\pgfqpoint{3.173075in}{2.769162in}}{\pgfqpoint{3.165175in}{2.772435in}}{\pgfqpoint{3.156939in}{2.772435in}}%
\pgfpathcurveto{\pgfqpoint{3.148702in}{2.772435in}}{\pgfqpoint{3.140802in}{2.769162in}}{\pgfqpoint{3.134979in}{2.763338in}}%
\pgfpathcurveto{\pgfqpoint{3.129155in}{2.757514in}}{\pgfqpoint{3.125882in}{2.749614in}}{\pgfqpoint{3.125882in}{2.741378in}}%
\pgfpathcurveto{\pgfqpoint{3.125882in}{2.733142in}}{\pgfqpoint{3.129155in}{2.725242in}}{\pgfqpoint{3.134979in}{2.719418in}}%
\pgfpathcurveto{\pgfqpoint{3.140802in}{2.713594in}}{\pgfqpoint{3.148702in}{2.710322in}}{\pgfqpoint{3.156939in}{2.710322in}}%
\pgfpathclose%
\pgfusepath{stroke,fill}%
\end{pgfscope}%
\begin{pgfscope}%
\pgfpathrectangle{\pgfqpoint{0.100000in}{0.212622in}}{\pgfqpoint{3.696000in}{3.696000in}}%
\pgfusepath{clip}%
\pgfsetbuttcap%
\pgfsetroundjoin%
\definecolor{currentfill}{rgb}{0.121569,0.466667,0.705882}%
\pgfsetfillcolor{currentfill}%
\pgfsetfillopacity{0.727037}%
\pgfsetlinewidth{1.003750pt}%
\definecolor{currentstroke}{rgb}{0.121569,0.466667,0.705882}%
\pgfsetstrokecolor{currentstroke}%
\pgfsetstrokeopacity{0.727037}%
\pgfsetdash{}{0pt}%
\pgfpathmoveto{\pgfqpoint{1.894541in}{1.898197in}}%
\pgfpathcurveto{\pgfqpoint{1.902778in}{1.898197in}}{\pgfqpoint{1.910678in}{1.901469in}}{\pgfqpoint{1.916502in}{1.907293in}}%
\pgfpathcurveto{\pgfqpoint{1.922326in}{1.913117in}}{\pgfqpoint{1.925598in}{1.921017in}}{\pgfqpoint{1.925598in}{1.929253in}}%
\pgfpathcurveto{\pgfqpoint{1.925598in}{1.937489in}}{\pgfqpoint{1.922326in}{1.945389in}}{\pgfqpoint{1.916502in}{1.951213in}}%
\pgfpathcurveto{\pgfqpoint{1.910678in}{1.957037in}}{\pgfqpoint{1.902778in}{1.960310in}}{\pgfqpoint{1.894541in}{1.960310in}}%
\pgfpathcurveto{\pgfqpoint{1.886305in}{1.960310in}}{\pgfqpoint{1.878405in}{1.957037in}}{\pgfqpoint{1.872581in}{1.951213in}}%
\pgfpathcurveto{\pgfqpoint{1.866757in}{1.945389in}}{\pgfqpoint{1.863485in}{1.937489in}}{\pgfqpoint{1.863485in}{1.929253in}}%
\pgfpathcurveto{\pgfqpoint{1.863485in}{1.921017in}}{\pgfqpoint{1.866757in}{1.913117in}}{\pgfqpoint{1.872581in}{1.907293in}}%
\pgfpathcurveto{\pgfqpoint{1.878405in}{1.901469in}}{\pgfqpoint{1.886305in}{1.898197in}}{\pgfqpoint{1.894541in}{1.898197in}}%
\pgfpathclose%
\pgfusepath{stroke,fill}%
\end{pgfscope}%
\begin{pgfscope}%
\pgfpathrectangle{\pgfqpoint{0.100000in}{0.212622in}}{\pgfqpoint{3.696000in}{3.696000in}}%
\pgfusepath{clip}%
\pgfsetbuttcap%
\pgfsetroundjoin%
\definecolor{currentfill}{rgb}{0.121569,0.466667,0.705882}%
\pgfsetfillcolor{currentfill}%
\pgfsetfillopacity{0.729929}%
\pgfsetlinewidth{1.003750pt}%
\definecolor{currentstroke}{rgb}{0.121569,0.466667,0.705882}%
\pgfsetstrokecolor{currentstroke}%
\pgfsetstrokeopacity{0.729929}%
\pgfsetdash{}{0pt}%
\pgfpathmoveto{\pgfqpoint{3.142770in}{2.696637in}}%
\pgfpathcurveto{\pgfqpoint{3.151006in}{2.696637in}}{\pgfqpoint{3.158906in}{2.699909in}}{\pgfqpoint{3.164730in}{2.705733in}}%
\pgfpathcurveto{\pgfqpoint{3.170554in}{2.711557in}}{\pgfqpoint{3.173826in}{2.719457in}}{\pgfqpoint{3.173826in}{2.727693in}}%
\pgfpathcurveto{\pgfqpoint{3.173826in}{2.735930in}}{\pgfqpoint{3.170554in}{2.743830in}}{\pgfqpoint{3.164730in}{2.749654in}}%
\pgfpathcurveto{\pgfqpoint{3.158906in}{2.755478in}}{\pgfqpoint{3.151006in}{2.758750in}}{\pgfqpoint{3.142770in}{2.758750in}}%
\pgfpathcurveto{\pgfqpoint{3.134533in}{2.758750in}}{\pgfqpoint{3.126633in}{2.755478in}}{\pgfqpoint{3.120809in}{2.749654in}}%
\pgfpathcurveto{\pgfqpoint{3.114985in}{2.743830in}}{\pgfqpoint{3.111713in}{2.735930in}}{\pgfqpoint{3.111713in}{2.727693in}}%
\pgfpathcurveto{\pgfqpoint{3.111713in}{2.719457in}}{\pgfqpoint{3.114985in}{2.711557in}}{\pgfqpoint{3.120809in}{2.705733in}}%
\pgfpathcurveto{\pgfqpoint{3.126633in}{2.699909in}}{\pgfqpoint{3.134533in}{2.696637in}}{\pgfqpoint{3.142770in}{2.696637in}}%
\pgfpathclose%
\pgfusepath{stroke,fill}%
\end{pgfscope}%
\begin{pgfscope}%
\pgfpathrectangle{\pgfqpoint{0.100000in}{0.212622in}}{\pgfqpoint{3.696000in}{3.696000in}}%
\pgfusepath{clip}%
\pgfsetbuttcap%
\pgfsetroundjoin%
\definecolor{currentfill}{rgb}{0.121569,0.466667,0.705882}%
\pgfsetfillcolor{currentfill}%
\pgfsetfillopacity{0.730099}%
\pgfsetlinewidth{1.003750pt}%
\definecolor{currentstroke}{rgb}{0.121569,0.466667,0.705882}%
\pgfsetstrokecolor{currentstroke}%
\pgfsetstrokeopacity{0.730099}%
\pgfsetdash{}{0pt}%
\pgfpathmoveto{\pgfqpoint{1.923368in}{1.917831in}}%
\pgfpathcurveto{\pgfqpoint{1.931604in}{1.917831in}}{\pgfqpoint{1.939504in}{1.921104in}}{\pgfqpoint{1.945328in}{1.926928in}}%
\pgfpathcurveto{\pgfqpoint{1.951152in}{1.932752in}}{\pgfqpoint{1.954424in}{1.940652in}}{\pgfqpoint{1.954424in}{1.948888in}}%
\pgfpathcurveto{\pgfqpoint{1.954424in}{1.957124in}}{\pgfqpoint{1.951152in}{1.965024in}}{\pgfqpoint{1.945328in}{1.970848in}}%
\pgfpathcurveto{\pgfqpoint{1.939504in}{1.976672in}}{\pgfqpoint{1.931604in}{1.979944in}}{\pgfqpoint{1.923368in}{1.979944in}}%
\pgfpathcurveto{\pgfqpoint{1.915131in}{1.979944in}}{\pgfqpoint{1.907231in}{1.976672in}}{\pgfqpoint{1.901407in}{1.970848in}}%
\pgfpathcurveto{\pgfqpoint{1.895583in}{1.965024in}}{\pgfqpoint{1.892311in}{1.957124in}}{\pgfqpoint{1.892311in}{1.948888in}}%
\pgfpathcurveto{\pgfqpoint{1.892311in}{1.940652in}}{\pgfqpoint{1.895583in}{1.932752in}}{\pgfqpoint{1.901407in}{1.926928in}}%
\pgfpathcurveto{\pgfqpoint{1.907231in}{1.921104in}}{\pgfqpoint{1.915131in}{1.917831in}}{\pgfqpoint{1.923368in}{1.917831in}}%
\pgfpathclose%
\pgfusepath{stroke,fill}%
\end{pgfscope}%
\begin{pgfscope}%
\pgfpathrectangle{\pgfqpoint{0.100000in}{0.212622in}}{\pgfqpoint{3.696000in}{3.696000in}}%
\pgfusepath{clip}%
\pgfsetbuttcap%
\pgfsetroundjoin%
\definecolor{currentfill}{rgb}{0.121569,0.466667,0.705882}%
\pgfsetfillcolor{currentfill}%
\pgfsetfillopacity{0.730432}%
\pgfsetlinewidth{1.003750pt}%
\definecolor{currentstroke}{rgb}{0.121569,0.466667,0.705882}%
\pgfsetstrokecolor{currentstroke}%
\pgfsetstrokeopacity{0.730432}%
\pgfsetdash{}{0pt}%
\pgfpathmoveto{\pgfqpoint{1.911348in}{1.915059in}}%
\pgfpathcurveto{\pgfqpoint{1.919584in}{1.915059in}}{\pgfqpoint{1.927484in}{1.918332in}}{\pgfqpoint{1.933308in}{1.924156in}}%
\pgfpathcurveto{\pgfqpoint{1.939132in}{1.929979in}}{\pgfqpoint{1.942405in}{1.937879in}}{\pgfqpoint{1.942405in}{1.946116in}}%
\pgfpathcurveto{\pgfqpoint{1.942405in}{1.954352in}}{\pgfqpoint{1.939132in}{1.962252in}}{\pgfqpoint{1.933308in}{1.968076in}}%
\pgfpathcurveto{\pgfqpoint{1.927484in}{1.973900in}}{\pgfqpoint{1.919584in}{1.977172in}}{\pgfqpoint{1.911348in}{1.977172in}}%
\pgfpathcurveto{\pgfqpoint{1.903112in}{1.977172in}}{\pgfqpoint{1.895212in}{1.973900in}}{\pgfqpoint{1.889388in}{1.968076in}}%
\pgfpathcurveto{\pgfqpoint{1.883564in}{1.962252in}}{\pgfqpoint{1.880292in}{1.954352in}}{\pgfqpoint{1.880292in}{1.946116in}}%
\pgfpathcurveto{\pgfqpoint{1.880292in}{1.937879in}}{\pgfqpoint{1.883564in}{1.929979in}}{\pgfqpoint{1.889388in}{1.924156in}}%
\pgfpathcurveto{\pgfqpoint{1.895212in}{1.918332in}}{\pgfqpoint{1.903112in}{1.915059in}}{\pgfqpoint{1.911348in}{1.915059in}}%
\pgfpathclose%
\pgfusepath{stroke,fill}%
\end{pgfscope}%
\begin{pgfscope}%
\pgfpathrectangle{\pgfqpoint{0.100000in}{0.212622in}}{\pgfqpoint{3.696000in}{3.696000in}}%
\pgfusepath{clip}%
\pgfsetbuttcap%
\pgfsetroundjoin%
\definecolor{currentfill}{rgb}{0.121569,0.466667,0.705882}%
\pgfsetfillcolor{currentfill}%
\pgfsetfillopacity{0.734770}%
\pgfsetlinewidth{1.003750pt}%
\definecolor{currentstroke}{rgb}{0.121569,0.466667,0.705882}%
\pgfsetstrokecolor{currentstroke}%
\pgfsetstrokeopacity{0.734770}%
\pgfsetdash{}{0pt}%
\pgfpathmoveto{\pgfqpoint{1.980516in}{1.954551in}}%
\pgfpathcurveto{\pgfqpoint{1.988752in}{1.954551in}}{\pgfqpoint{1.996652in}{1.957824in}}{\pgfqpoint{2.002476in}{1.963648in}}%
\pgfpathcurveto{\pgfqpoint{2.008300in}{1.969472in}}{\pgfqpoint{2.011572in}{1.977372in}}{\pgfqpoint{2.011572in}{1.985608in}}%
\pgfpathcurveto{\pgfqpoint{2.011572in}{1.993844in}}{\pgfqpoint{2.008300in}{2.001744in}}{\pgfqpoint{2.002476in}{2.007568in}}%
\pgfpathcurveto{\pgfqpoint{1.996652in}{2.013392in}}{\pgfqpoint{1.988752in}{2.016664in}}{\pgfqpoint{1.980516in}{2.016664in}}%
\pgfpathcurveto{\pgfqpoint{1.972279in}{2.016664in}}{\pgfqpoint{1.964379in}{2.013392in}}{\pgfqpoint{1.958555in}{2.007568in}}%
\pgfpathcurveto{\pgfqpoint{1.952731in}{2.001744in}}{\pgfqpoint{1.949459in}{1.993844in}}{\pgfqpoint{1.949459in}{1.985608in}}%
\pgfpathcurveto{\pgfqpoint{1.949459in}{1.977372in}}{\pgfqpoint{1.952731in}{1.969472in}}{\pgfqpoint{1.958555in}{1.963648in}}%
\pgfpathcurveto{\pgfqpoint{1.964379in}{1.957824in}}{\pgfqpoint{1.972279in}{1.954551in}}{\pgfqpoint{1.980516in}{1.954551in}}%
\pgfpathclose%
\pgfusepath{stroke,fill}%
\end{pgfscope}%
\begin{pgfscope}%
\pgfpathrectangle{\pgfqpoint{0.100000in}{0.212622in}}{\pgfqpoint{3.696000in}{3.696000in}}%
\pgfusepath{clip}%
\pgfsetbuttcap%
\pgfsetroundjoin%
\definecolor{currentfill}{rgb}{0.121569,0.466667,0.705882}%
\pgfsetfillcolor{currentfill}%
\pgfsetfillopacity{0.735076}%
\pgfsetlinewidth{1.003750pt}%
\definecolor{currentstroke}{rgb}{0.121569,0.466667,0.705882}%
\pgfsetstrokecolor{currentstroke}%
\pgfsetstrokeopacity{0.735076}%
\pgfsetdash{}{0pt}%
\pgfpathmoveto{\pgfqpoint{1.984169in}{1.961138in}}%
\pgfpathcurveto{\pgfqpoint{1.992406in}{1.961138in}}{\pgfqpoint{2.000306in}{1.964410in}}{\pgfqpoint{2.006130in}{1.970234in}}%
\pgfpathcurveto{\pgfqpoint{2.011953in}{1.976058in}}{\pgfqpoint{2.015226in}{1.983958in}}{\pgfqpoint{2.015226in}{1.992194in}}%
\pgfpathcurveto{\pgfqpoint{2.015226in}{2.000430in}}{\pgfqpoint{2.011953in}{2.008331in}}{\pgfqpoint{2.006130in}{2.014154in}}%
\pgfpathcurveto{\pgfqpoint{2.000306in}{2.019978in}}{\pgfqpoint{1.992406in}{2.023251in}}{\pgfqpoint{1.984169in}{2.023251in}}%
\pgfpathcurveto{\pgfqpoint{1.975933in}{2.023251in}}{\pgfqpoint{1.968033in}{2.019978in}}{\pgfqpoint{1.962209in}{2.014154in}}%
\pgfpathcurveto{\pgfqpoint{1.956385in}{2.008331in}}{\pgfqpoint{1.953113in}{2.000430in}}{\pgfqpoint{1.953113in}{1.992194in}}%
\pgfpathcurveto{\pgfqpoint{1.953113in}{1.983958in}}{\pgfqpoint{1.956385in}{1.976058in}}{\pgfqpoint{1.962209in}{1.970234in}}%
\pgfpathcurveto{\pgfqpoint{1.968033in}{1.964410in}}{\pgfqpoint{1.975933in}{1.961138in}}{\pgfqpoint{1.984169in}{1.961138in}}%
\pgfpathclose%
\pgfusepath{stroke,fill}%
\end{pgfscope}%
\begin{pgfscope}%
\pgfpathrectangle{\pgfqpoint{0.100000in}{0.212622in}}{\pgfqpoint{3.696000in}{3.696000in}}%
\pgfusepath{clip}%
\pgfsetbuttcap%
\pgfsetroundjoin%
\definecolor{currentfill}{rgb}{0.121569,0.466667,0.705882}%
\pgfsetfillcolor{currentfill}%
\pgfsetfillopacity{0.736932}%
\pgfsetlinewidth{1.003750pt}%
\definecolor{currentstroke}{rgb}{0.121569,0.466667,0.705882}%
\pgfsetstrokecolor{currentstroke}%
\pgfsetstrokeopacity{0.736932}%
\pgfsetdash{}{0pt}%
\pgfpathmoveto{\pgfqpoint{1.982849in}{1.959924in}}%
\pgfpathcurveto{\pgfqpoint{1.991085in}{1.959924in}}{\pgfqpoint{1.998985in}{1.963197in}}{\pgfqpoint{2.004809in}{1.969021in}}%
\pgfpathcurveto{\pgfqpoint{2.010633in}{1.974844in}}{\pgfqpoint{2.013906in}{1.982745in}}{\pgfqpoint{2.013906in}{1.990981in}}%
\pgfpathcurveto{\pgfqpoint{2.013906in}{1.999217in}}{\pgfqpoint{2.010633in}{2.007117in}}{\pgfqpoint{2.004809in}{2.012941in}}%
\pgfpathcurveto{\pgfqpoint{1.998985in}{2.018765in}}{\pgfqpoint{1.991085in}{2.022037in}}{\pgfqpoint{1.982849in}{2.022037in}}%
\pgfpathcurveto{\pgfqpoint{1.974613in}{2.022037in}}{\pgfqpoint{1.966713in}{2.018765in}}{\pgfqpoint{1.960889in}{2.012941in}}%
\pgfpathcurveto{\pgfqpoint{1.955065in}{2.007117in}}{\pgfqpoint{1.951793in}{1.999217in}}{\pgfqpoint{1.951793in}{1.990981in}}%
\pgfpathcurveto{\pgfqpoint{1.951793in}{1.982745in}}{\pgfqpoint{1.955065in}{1.974844in}}{\pgfqpoint{1.960889in}{1.969021in}}%
\pgfpathcurveto{\pgfqpoint{1.966713in}{1.963197in}}{\pgfqpoint{1.974613in}{1.959924in}}{\pgfqpoint{1.982849in}{1.959924in}}%
\pgfpathclose%
\pgfusepath{stroke,fill}%
\end{pgfscope}%
\begin{pgfscope}%
\pgfpathrectangle{\pgfqpoint{0.100000in}{0.212622in}}{\pgfqpoint{3.696000in}{3.696000in}}%
\pgfusepath{clip}%
\pgfsetbuttcap%
\pgfsetroundjoin%
\definecolor{currentfill}{rgb}{0.121569,0.466667,0.705882}%
\pgfsetfillcolor{currentfill}%
\pgfsetfillopacity{0.737864}%
\pgfsetlinewidth{1.003750pt}%
\definecolor{currentstroke}{rgb}{0.121569,0.466667,0.705882}%
\pgfsetstrokecolor{currentstroke}%
\pgfsetstrokeopacity{0.737864}%
\pgfsetdash{}{0pt}%
\pgfpathmoveto{\pgfqpoint{1.977435in}{1.954019in}}%
\pgfpathcurveto{\pgfqpoint{1.985672in}{1.954019in}}{\pgfqpoint{1.993572in}{1.957291in}}{\pgfqpoint{1.999396in}{1.963115in}}%
\pgfpathcurveto{\pgfqpoint{2.005219in}{1.968939in}}{\pgfqpoint{2.008492in}{1.976839in}}{\pgfqpoint{2.008492in}{1.985075in}}%
\pgfpathcurveto{\pgfqpoint{2.008492in}{1.993311in}}{\pgfqpoint{2.005219in}{2.001211in}}{\pgfqpoint{1.999396in}{2.007035in}}%
\pgfpathcurveto{\pgfqpoint{1.993572in}{2.012859in}}{\pgfqpoint{1.985672in}{2.016132in}}{\pgfqpoint{1.977435in}{2.016132in}}%
\pgfpathcurveto{\pgfqpoint{1.969199in}{2.016132in}}{\pgfqpoint{1.961299in}{2.012859in}}{\pgfqpoint{1.955475in}{2.007035in}}%
\pgfpathcurveto{\pgfqpoint{1.949651in}{2.001211in}}{\pgfqpoint{1.946379in}{1.993311in}}{\pgfqpoint{1.946379in}{1.985075in}}%
\pgfpathcurveto{\pgfqpoint{1.946379in}{1.976839in}}{\pgfqpoint{1.949651in}{1.968939in}}{\pgfqpoint{1.955475in}{1.963115in}}%
\pgfpathcurveto{\pgfqpoint{1.961299in}{1.957291in}}{\pgfqpoint{1.969199in}{1.954019in}}{\pgfqpoint{1.977435in}{1.954019in}}%
\pgfpathclose%
\pgfusepath{stroke,fill}%
\end{pgfscope}%
\begin{pgfscope}%
\pgfpathrectangle{\pgfqpoint{0.100000in}{0.212622in}}{\pgfqpoint{3.696000in}{3.696000in}}%
\pgfusepath{clip}%
\pgfsetbuttcap%
\pgfsetroundjoin%
\definecolor{currentfill}{rgb}{0.121569,0.466667,0.705882}%
\pgfsetfillcolor{currentfill}%
\pgfsetfillopacity{0.738290}%
\pgfsetlinewidth{1.003750pt}%
\definecolor{currentstroke}{rgb}{0.121569,0.466667,0.705882}%
\pgfsetstrokecolor{currentstroke}%
\pgfsetstrokeopacity{0.738290}%
\pgfsetdash{}{0pt}%
\pgfpathmoveto{\pgfqpoint{1.979966in}{1.956782in}}%
\pgfpathcurveto{\pgfqpoint{1.988203in}{1.956782in}}{\pgfqpoint{1.996103in}{1.960054in}}{\pgfqpoint{2.001927in}{1.965878in}}%
\pgfpathcurveto{\pgfqpoint{2.007751in}{1.971702in}}{\pgfqpoint{2.011023in}{1.979602in}}{\pgfqpoint{2.011023in}{1.987838in}}%
\pgfpathcurveto{\pgfqpoint{2.011023in}{1.996074in}}{\pgfqpoint{2.007751in}{2.003974in}}{\pgfqpoint{2.001927in}{2.009798in}}%
\pgfpathcurveto{\pgfqpoint{1.996103in}{2.015622in}}{\pgfqpoint{1.988203in}{2.018895in}}{\pgfqpoint{1.979966in}{2.018895in}}%
\pgfpathcurveto{\pgfqpoint{1.971730in}{2.018895in}}{\pgfqpoint{1.963830in}{2.015622in}}{\pgfqpoint{1.958006in}{2.009798in}}%
\pgfpathcurveto{\pgfqpoint{1.952182in}{2.003974in}}{\pgfqpoint{1.948910in}{1.996074in}}{\pgfqpoint{1.948910in}{1.987838in}}%
\pgfpathcurveto{\pgfqpoint{1.948910in}{1.979602in}}{\pgfqpoint{1.952182in}{1.971702in}}{\pgfqpoint{1.958006in}{1.965878in}}%
\pgfpathcurveto{\pgfqpoint{1.963830in}{1.960054in}}{\pgfqpoint{1.971730in}{1.956782in}}{\pgfqpoint{1.979966in}{1.956782in}}%
\pgfpathclose%
\pgfusepath{stroke,fill}%
\end{pgfscope}%
\begin{pgfscope}%
\pgfpathrectangle{\pgfqpoint{0.100000in}{0.212622in}}{\pgfqpoint{3.696000in}{3.696000in}}%
\pgfusepath{clip}%
\pgfsetbuttcap%
\pgfsetroundjoin%
\definecolor{currentfill}{rgb}{0.121569,0.466667,0.705882}%
\pgfsetfillcolor{currentfill}%
\pgfsetfillopacity{0.738365}%
\pgfsetlinewidth{1.003750pt}%
\definecolor{currentstroke}{rgb}{0.121569,0.466667,0.705882}%
\pgfsetstrokecolor{currentstroke}%
\pgfsetstrokeopacity{0.738365}%
\pgfsetdash{}{0pt}%
\pgfpathmoveto{\pgfqpoint{1.981548in}{1.957543in}}%
\pgfpathcurveto{\pgfqpoint{1.989784in}{1.957543in}}{\pgfqpoint{1.997684in}{1.960815in}}{\pgfqpoint{2.003508in}{1.966639in}}%
\pgfpathcurveto{\pgfqpoint{2.009332in}{1.972463in}}{\pgfqpoint{2.012604in}{1.980363in}}{\pgfqpoint{2.012604in}{1.988599in}}%
\pgfpathcurveto{\pgfqpoint{2.012604in}{1.996835in}}{\pgfqpoint{2.009332in}{2.004735in}}{\pgfqpoint{2.003508in}{2.010559in}}%
\pgfpathcurveto{\pgfqpoint{1.997684in}{2.016383in}}{\pgfqpoint{1.989784in}{2.019656in}}{\pgfqpoint{1.981548in}{2.019656in}}%
\pgfpathcurveto{\pgfqpoint{1.973312in}{2.019656in}}{\pgfqpoint{1.965411in}{2.016383in}}{\pgfqpoint{1.959588in}{2.010559in}}%
\pgfpathcurveto{\pgfqpoint{1.953764in}{2.004735in}}{\pgfqpoint{1.950491in}{1.996835in}}{\pgfqpoint{1.950491in}{1.988599in}}%
\pgfpathcurveto{\pgfqpoint{1.950491in}{1.980363in}}{\pgfqpoint{1.953764in}{1.972463in}}{\pgfqpoint{1.959588in}{1.966639in}}%
\pgfpathcurveto{\pgfqpoint{1.965411in}{1.960815in}}{\pgfqpoint{1.973312in}{1.957543in}}{\pgfqpoint{1.981548in}{1.957543in}}%
\pgfpathclose%
\pgfusepath{stroke,fill}%
\end{pgfscope}%
\begin{pgfscope}%
\pgfpathrectangle{\pgfqpoint{0.100000in}{0.212622in}}{\pgfqpoint{3.696000in}{3.696000in}}%
\pgfusepath{clip}%
\pgfsetbuttcap%
\pgfsetroundjoin%
\definecolor{currentfill}{rgb}{0.121569,0.466667,0.705882}%
\pgfsetfillcolor{currentfill}%
\pgfsetfillopacity{0.738893}%
\pgfsetlinewidth{1.003750pt}%
\definecolor{currentstroke}{rgb}{0.121569,0.466667,0.705882}%
\pgfsetstrokecolor{currentstroke}%
\pgfsetstrokeopacity{0.738893}%
\pgfsetdash{}{0pt}%
\pgfpathmoveto{\pgfqpoint{3.116931in}{2.671681in}}%
\pgfpathcurveto{\pgfqpoint{3.125167in}{2.671681in}}{\pgfqpoint{3.133067in}{2.674953in}}{\pgfqpoint{3.138891in}{2.680777in}}%
\pgfpathcurveto{\pgfqpoint{3.144715in}{2.686601in}}{\pgfqpoint{3.147987in}{2.694501in}}{\pgfqpoint{3.147987in}{2.702737in}}%
\pgfpathcurveto{\pgfqpoint{3.147987in}{2.710974in}}{\pgfqpoint{3.144715in}{2.718874in}}{\pgfqpoint{3.138891in}{2.724698in}}%
\pgfpathcurveto{\pgfqpoint{3.133067in}{2.730522in}}{\pgfqpoint{3.125167in}{2.733794in}}{\pgfqpoint{3.116931in}{2.733794in}}%
\pgfpathcurveto{\pgfqpoint{3.108694in}{2.733794in}}{\pgfqpoint{3.100794in}{2.730522in}}{\pgfqpoint{3.094970in}{2.724698in}}%
\pgfpathcurveto{\pgfqpoint{3.089146in}{2.718874in}}{\pgfqpoint{3.085874in}{2.710974in}}{\pgfqpoint{3.085874in}{2.702737in}}%
\pgfpathcurveto{\pgfqpoint{3.085874in}{2.694501in}}{\pgfqpoint{3.089146in}{2.686601in}}{\pgfqpoint{3.094970in}{2.680777in}}%
\pgfpathcurveto{\pgfqpoint{3.100794in}{2.674953in}}{\pgfqpoint{3.108694in}{2.671681in}}{\pgfqpoint{3.116931in}{2.671681in}}%
\pgfpathclose%
\pgfusepath{stroke,fill}%
\end{pgfscope}%
\begin{pgfscope}%
\pgfpathrectangle{\pgfqpoint{0.100000in}{0.212622in}}{\pgfqpoint{3.696000in}{3.696000in}}%
\pgfusepath{clip}%
\pgfsetbuttcap%
\pgfsetroundjoin%
\definecolor{currentfill}{rgb}{0.121569,0.466667,0.705882}%
\pgfsetfillcolor{currentfill}%
\pgfsetfillopacity{0.739121}%
\pgfsetlinewidth{1.003750pt}%
\definecolor{currentstroke}{rgb}{0.121569,0.466667,0.705882}%
\pgfsetstrokecolor{currentstroke}%
\pgfsetstrokeopacity{0.739121}%
\pgfsetdash{}{0pt}%
\pgfpathmoveto{\pgfqpoint{1.989138in}{1.963841in}}%
\pgfpathcurveto{\pgfqpoint{1.997374in}{1.963841in}}{\pgfqpoint{2.005274in}{1.967113in}}{\pgfqpoint{2.011098in}{1.972937in}}%
\pgfpathcurveto{\pgfqpoint{2.016922in}{1.978761in}}{\pgfqpoint{2.020194in}{1.986661in}}{\pgfqpoint{2.020194in}{1.994898in}}%
\pgfpathcurveto{\pgfqpoint{2.020194in}{2.003134in}}{\pgfqpoint{2.016922in}{2.011034in}}{\pgfqpoint{2.011098in}{2.016858in}}%
\pgfpathcurveto{\pgfqpoint{2.005274in}{2.022682in}}{\pgfqpoint{1.997374in}{2.025954in}}{\pgfqpoint{1.989138in}{2.025954in}}%
\pgfpathcurveto{\pgfqpoint{1.980902in}{2.025954in}}{\pgfqpoint{1.973002in}{2.022682in}}{\pgfqpoint{1.967178in}{2.016858in}}%
\pgfpathcurveto{\pgfqpoint{1.961354in}{2.011034in}}{\pgfqpoint{1.958081in}{2.003134in}}{\pgfqpoint{1.958081in}{1.994898in}}%
\pgfpathcurveto{\pgfqpoint{1.958081in}{1.986661in}}{\pgfqpoint{1.961354in}{1.978761in}}{\pgfqpoint{1.967178in}{1.972937in}}%
\pgfpathcurveto{\pgfqpoint{1.973002in}{1.967113in}}{\pgfqpoint{1.980902in}{1.963841in}}{\pgfqpoint{1.989138in}{1.963841in}}%
\pgfpathclose%
\pgfusepath{stroke,fill}%
\end{pgfscope}%
\begin{pgfscope}%
\pgfpathrectangle{\pgfqpoint{0.100000in}{0.212622in}}{\pgfqpoint{3.696000in}{3.696000in}}%
\pgfusepath{clip}%
\pgfsetbuttcap%
\pgfsetroundjoin%
\definecolor{currentfill}{rgb}{0.121569,0.466667,0.705882}%
\pgfsetfillcolor{currentfill}%
\pgfsetfillopacity{0.739407}%
\pgfsetlinewidth{1.003750pt}%
\definecolor{currentstroke}{rgb}{0.121569,0.466667,0.705882}%
\pgfsetstrokecolor{currentstroke}%
\pgfsetstrokeopacity{0.739407}%
\pgfsetdash{}{0pt}%
\pgfpathmoveto{\pgfqpoint{1.981343in}{1.957641in}}%
\pgfpathcurveto{\pgfqpoint{1.989579in}{1.957641in}}{\pgfqpoint{1.997479in}{1.960913in}}{\pgfqpoint{2.003303in}{1.966737in}}%
\pgfpathcurveto{\pgfqpoint{2.009127in}{1.972561in}}{\pgfqpoint{2.012399in}{1.980461in}}{\pgfqpoint{2.012399in}{1.988697in}}%
\pgfpathcurveto{\pgfqpoint{2.012399in}{1.996934in}}{\pgfqpoint{2.009127in}{2.004834in}}{\pgfqpoint{2.003303in}{2.010658in}}%
\pgfpathcurveto{\pgfqpoint{1.997479in}{2.016482in}}{\pgfqpoint{1.989579in}{2.019754in}}{\pgfqpoint{1.981343in}{2.019754in}}%
\pgfpathcurveto{\pgfqpoint{1.973106in}{2.019754in}}{\pgfqpoint{1.965206in}{2.016482in}}{\pgfqpoint{1.959382in}{2.010658in}}%
\pgfpathcurveto{\pgfqpoint{1.953558in}{2.004834in}}{\pgfqpoint{1.950286in}{1.996934in}}{\pgfqpoint{1.950286in}{1.988697in}}%
\pgfpathcurveto{\pgfqpoint{1.950286in}{1.980461in}}{\pgfqpoint{1.953558in}{1.972561in}}{\pgfqpoint{1.959382in}{1.966737in}}%
\pgfpathcurveto{\pgfqpoint{1.965206in}{1.960913in}}{\pgfqpoint{1.973106in}{1.957641in}}{\pgfqpoint{1.981343in}{1.957641in}}%
\pgfpathclose%
\pgfusepath{stroke,fill}%
\end{pgfscope}%
\begin{pgfscope}%
\pgfpathrectangle{\pgfqpoint{0.100000in}{0.212622in}}{\pgfqpoint{3.696000in}{3.696000in}}%
\pgfusepath{clip}%
\pgfsetbuttcap%
\pgfsetroundjoin%
\definecolor{currentfill}{rgb}{0.121569,0.466667,0.705882}%
\pgfsetfillcolor{currentfill}%
\pgfsetfillopacity{0.741739}%
\pgfsetlinewidth{1.003750pt}%
\definecolor{currentstroke}{rgb}{0.121569,0.466667,0.705882}%
\pgfsetstrokecolor{currentstroke}%
\pgfsetstrokeopacity{0.741739}%
\pgfsetdash{}{0pt}%
\pgfpathmoveto{\pgfqpoint{1.976897in}{1.951328in}}%
\pgfpathcurveto{\pgfqpoint{1.985133in}{1.951328in}}{\pgfqpoint{1.993033in}{1.954600in}}{\pgfqpoint{1.998857in}{1.960424in}}%
\pgfpathcurveto{\pgfqpoint{2.004681in}{1.966248in}}{\pgfqpoint{2.007953in}{1.974148in}}{\pgfqpoint{2.007953in}{1.982384in}}%
\pgfpathcurveto{\pgfqpoint{2.007953in}{1.990621in}}{\pgfqpoint{2.004681in}{1.998521in}}{\pgfqpoint{1.998857in}{2.004345in}}%
\pgfpathcurveto{\pgfqpoint{1.993033in}{2.010169in}}{\pgfqpoint{1.985133in}{2.013441in}}{\pgfqpoint{1.976897in}{2.013441in}}%
\pgfpathcurveto{\pgfqpoint{1.968660in}{2.013441in}}{\pgfqpoint{1.960760in}{2.010169in}}{\pgfqpoint{1.954936in}{2.004345in}}%
\pgfpathcurveto{\pgfqpoint{1.949113in}{1.998521in}}{\pgfqpoint{1.945840in}{1.990621in}}{\pgfqpoint{1.945840in}{1.982384in}}%
\pgfpathcurveto{\pgfqpoint{1.945840in}{1.974148in}}{\pgfqpoint{1.949113in}{1.966248in}}{\pgfqpoint{1.954936in}{1.960424in}}%
\pgfpathcurveto{\pgfqpoint{1.960760in}{1.954600in}}{\pgfqpoint{1.968660in}{1.951328in}}{\pgfqpoint{1.976897in}{1.951328in}}%
\pgfpathclose%
\pgfusepath{stroke,fill}%
\end{pgfscope}%
\begin{pgfscope}%
\pgfpathrectangle{\pgfqpoint{0.100000in}{0.212622in}}{\pgfqpoint{3.696000in}{3.696000in}}%
\pgfusepath{clip}%
\pgfsetbuttcap%
\pgfsetroundjoin%
\definecolor{currentfill}{rgb}{0.121569,0.466667,0.705882}%
\pgfsetfillcolor{currentfill}%
\pgfsetfillopacity{0.742454}%
\pgfsetlinewidth{1.003750pt}%
\definecolor{currentstroke}{rgb}{0.121569,0.466667,0.705882}%
\pgfsetstrokecolor{currentstroke}%
\pgfsetstrokeopacity{0.742454}%
\pgfsetdash{}{0pt}%
\pgfpathmoveto{\pgfqpoint{1.990845in}{1.964322in}}%
\pgfpathcurveto{\pgfqpoint{1.999082in}{1.964322in}}{\pgfqpoint{2.006982in}{1.967595in}}{\pgfqpoint{2.012806in}{1.973419in}}%
\pgfpathcurveto{\pgfqpoint{2.018629in}{1.979243in}}{\pgfqpoint{2.021902in}{1.987143in}}{\pgfqpoint{2.021902in}{1.995379in}}%
\pgfpathcurveto{\pgfqpoint{2.021902in}{2.003615in}}{\pgfqpoint{2.018629in}{2.011515in}}{\pgfqpoint{2.012806in}{2.017339in}}%
\pgfpathcurveto{\pgfqpoint{2.006982in}{2.023163in}}{\pgfqpoint{1.999082in}{2.026435in}}{\pgfqpoint{1.990845in}{2.026435in}}%
\pgfpathcurveto{\pgfqpoint{1.982609in}{2.026435in}}{\pgfqpoint{1.974709in}{2.023163in}}{\pgfqpoint{1.968885in}{2.017339in}}%
\pgfpathcurveto{\pgfqpoint{1.963061in}{2.011515in}}{\pgfqpoint{1.959789in}{2.003615in}}{\pgfqpoint{1.959789in}{1.995379in}}%
\pgfpathcurveto{\pgfqpoint{1.959789in}{1.987143in}}{\pgfqpoint{1.963061in}{1.979243in}}{\pgfqpoint{1.968885in}{1.973419in}}%
\pgfpathcurveto{\pgfqpoint{1.974709in}{1.967595in}}{\pgfqpoint{1.982609in}{1.964322in}}{\pgfqpoint{1.990845in}{1.964322in}}%
\pgfpathclose%
\pgfusepath{stroke,fill}%
\end{pgfscope}%
\begin{pgfscope}%
\pgfpathrectangle{\pgfqpoint{0.100000in}{0.212622in}}{\pgfqpoint{3.696000in}{3.696000in}}%
\pgfusepath{clip}%
\pgfsetbuttcap%
\pgfsetroundjoin%
\definecolor{currentfill}{rgb}{0.121569,0.466667,0.705882}%
\pgfsetfillcolor{currentfill}%
\pgfsetfillopacity{0.743595}%
\pgfsetlinewidth{1.003750pt}%
\definecolor{currentstroke}{rgb}{0.121569,0.466667,0.705882}%
\pgfsetstrokecolor{currentstroke}%
\pgfsetstrokeopacity{0.743595}%
\pgfsetdash{}{0pt}%
\pgfpathmoveto{\pgfqpoint{1.979234in}{1.949726in}}%
\pgfpathcurveto{\pgfqpoint{1.987471in}{1.949726in}}{\pgfqpoint{1.995371in}{1.952999in}}{\pgfqpoint{2.001194in}{1.958823in}}%
\pgfpathcurveto{\pgfqpoint{2.007018in}{1.964647in}}{\pgfqpoint{2.010291in}{1.972547in}}{\pgfqpoint{2.010291in}{1.980783in}}%
\pgfpathcurveto{\pgfqpoint{2.010291in}{1.989019in}}{\pgfqpoint{2.007018in}{1.996919in}}{\pgfqpoint{2.001194in}{2.002743in}}%
\pgfpathcurveto{\pgfqpoint{1.995371in}{2.008567in}}{\pgfqpoint{1.987471in}{2.011839in}}{\pgfqpoint{1.979234in}{2.011839in}}%
\pgfpathcurveto{\pgfqpoint{1.970998in}{2.011839in}}{\pgfqpoint{1.963098in}{2.008567in}}{\pgfqpoint{1.957274in}{2.002743in}}%
\pgfpathcurveto{\pgfqpoint{1.951450in}{1.996919in}}{\pgfqpoint{1.948178in}{1.989019in}}{\pgfqpoint{1.948178in}{1.980783in}}%
\pgfpathcurveto{\pgfqpoint{1.948178in}{1.972547in}}{\pgfqpoint{1.951450in}{1.964647in}}{\pgfqpoint{1.957274in}{1.958823in}}%
\pgfpathcurveto{\pgfqpoint{1.963098in}{1.952999in}}{\pgfqpoint{1.970998in}{1.949726in}}{\pgfqpoint{1.979234in}{1.949726in}}%
\pgfpathclose%
\pgfusepath{stroke,fill}%
\end{pgfscope}%
\begin{pgfscope}%
\pgfpathrectangle{\pgfqpoint{0.100000in}{0.212622in}}{\pgfqpoint{3.696000in}{3.696000in}}%
\pgfusepath{clip}%
\pgfsetbuttcap%
\pgfsetroundjoin%
\definecolor{currentfill}{rgb}{0.121569,0.466667,0.705882}%
\pgfsetfillcolor{currentfill}%
\pgfsetfillopacity{0.743932}%
\pgfsetlinewidth{1.003750pt}%
\definecolor{currentstroke}{rgb}{0.121569,0.466667,0.705882}%
\pgfsetstrokecolor{currentstroke}%
\pgfsetstrokeopacity{0.743932}%
\pgfsetdash{}{0pt}%
\pgfpathmoveto{\pgfqpoint{3.102418in}{2.658820in}}%
\pgfpathcurveto{\pgfqpoint{3.110654in}{2.658820in}}{\pgfqpoint{3.118554in}{2.662092in}}{\pgfqpoint{3.124378in}{2.667916in}}%
\pgfpathcurveto{\pgfqpoint{3.130202in}{2.673740in}}{\pgfqpoint{3.133475in}{2.681640in}}{\pgfqpoint{3.133475in}{2.689876in}}%
\pgfpathcurveto{\pgfqpoint{3.133475in}{2.698112in}}{\pgfqpoint{3.130202in}{2.706012in}}{\pgfqpoint{3.124378in}{2.711836in}}%
\pgfpathcurveto{\pgfqpoint{3.118554in}{2.717660in}}{\pgfqpoint{3.110654in}{2.720933in}}{\pgfqpoint{3.102418in}{2.720933in}}%
\pgfpathcurveto{\pgfqpoint{3.094182in}{2.720933in}}{\pgfqpoint{3.086282in}{2.717660in}}{\pgfqpoint{3.080458in}{2.711836in}}%
\pgfpathcurveto{\pgfqpoint{3.074634in}{2.706012in}}{\pgfqpoint{3.071362in}{2.698112in}}{\pgfqpoint{3.071362in}{2.689876in}}%
\pgfpathcurveto{\pgfqpoint{3.071362in}{2.681640in}}{\pgfqpoint{3.074634in}{2.673740in}}{\pgfqpoint{3.080458in}{2.667916in}}%
\pgfpathcurveto{\pgfqpoint{3.086282in}{2.662092in}}{\pgfqpoint{3.094182in}{2.658820in}}{\pgfqpoint{3.102418in}{2.658820in}}%
\pgfpathclose%
\pgfusepath{stroke,fill}%
\end{pgfscope}%
\begin{pgfscope}%
\pgfpathrectangle{\pgfqpoint{0.100000in}{0.212622in}}{\pgfqpoint{3.696000in}{3.696000in}}%
\pgfusepath{clip}%
\pgfsetbuttcap%
\pgfsetroundjoin%
\definecolor{currentfill}{rgb}{0.121569,0.466667,0.705882}%
\pgfsetfillcolor{currentfill}%
\pgfsetfillopacity{0.748722}%
\pgfsetlinewidth{1.003750pt}%
\definecolor{currentstroke}{rgb}{0.121569,0.466667,0.705882}%
\pgfsetstrokecolor{currentstroke}%
\pgfsetstrokeopacity{0.748722}%
\pgfsetdash{}{0pt}%
\pgfpathmoveto{\pgfqpoint{1.980699in}{1.955314in}}%
\pgfpathcurveto{\pgfqpoint{1.988936in}{1.955314in}}{\pgfqpoint{1.996836in}{1.958587in}}{\pgfqpoint{2.002660in}{1.964410in}}%
\pgfpathcurveto{\pgfqpoint{2.008484in}{1.970234in}}{\pgfqpoint{2.011756in}{1.978134in}}{\pgfqpoint{2.011756in}{1.986371in}}%
\pgfpathcurveto{\pgfqpoint{2.011756in}{1.994607in}}{\pgfqpoint{2.008484in}{2.002507in}}{\pgfqpoint{2.002660in}{2.008331in}}%
\pgfpathcurveto{\pgfqpoint{1.996836in}{2.014155in}}{\pgfqpoint{1.988936in}{2.017427in}}{\pgfqpoint{1.980699in}{2.017427in}}%
\pgfpathcurveto{\pgfqpoint{1.972463in}{2.017427in}}{\pgfqpoint{1.964563in}{2.014155in}}{\pgfqpoint{1.958739in}{2.008331in}}%
\pgfpathcurveto{\pgfqpoint{1.952915in}{2.002507in}}{\pgfqpoint{1.949643in}{1.994607in}}{\pgfqpoint{1.949643in}{1.986371in}}%
\pgfpathcurveto{\pgfqpoint{1.949643in}{1.978134in}}{\pgfqpoint{1.952915in}{1.970234in}}{\pgfqpoint{1.958739in}{1.964410in}}%
\pgfpathcurveto{\pgfqpoint{1.964563in}{1.958587in}}{\pgfqpoint{1.972463in}{1.955314in}}{\pgfqpoint{1.980699in}{1.955314in}}%
\pgfpathclose%
\pgfusepath{stroke,fill}%
\end{pgfscope}%
\begin{pgfscope}%
\pgfpathrectangle{\pgfqpoint{0.100000in}{0.212622in}}{\pgfqpoint{3.696000in}{3.696000in}}%
\pgfusepath{clip}%
\pgfsetbuttcap%
\pgfsetroundjoin%
\definecolor{currentfill}{rgb}{0.121569,0.466667,0.705882}%
\pgfsetfillcolor{currentfill}%
\pgfsetfillopacity{0.750709}%
\pgfsetlinewidth{1.003750pt}%
\definecolor{currentstroke}{rgb}{0.121569,0.466667,0.705882}%
\pgfsetstrokecolor{currentstroke}%
\pgfsetstrokeopacity{0.750709}%
\pgfsetdash{}{0pt}%
\pgfpathmoveto{\pgfqpoint{1.984429in}{1.959442in}}%
\pgfpathcurveto{\pgfqpoint{1.992665in}{1.959442in}}{\pgfqpoint{2.000566in}{1.962714in}}{\pgfqpoint{2.006389in}{1.968538in}}%
\pgfpathcurveto{\pgfqpoint{2.012213in}{1.974362in}}{\pgfqpoint{2.015486in}{1.982262in}}{\pgfqpoint{2.015486in}{1.990499in}}%
\pgfpathcurveto{\pgfqpoint{2.015486in}{1.998735in}}{\pgfqpoint{2.012213in}{2.006635in}}{\pgfqpoint{2.006389in}{2.012459in}}%
\pgfpathcurveto{\pgfqpoint{2.000566in}{2.018283in}}{\pgfqpoint{1.992665in}{2.021555in}}{\pgfqpoint{1.984429in}{2.021555in}}%
\pgfpathcurveto{\pgfqpoint{1.976193in}{2.021555in}}{\pgfqpoint{1.968293in}{2.018283in}}{\pgfqpoint{1.962469in}{2.012459in}}%
\pgfpathcurveto{\pgfqpoint{1.956645in}{2.006635in}}{\pgfqpoint{1.953373in}{1.998735in}}{\pgfqpoint{1.953373in}{1.990499in}}%
\pgfpathcurveto{\pgfqpoint{1.953373in}{1.982262in}}{\pgfqpoint{1.956645in}{1.974362in}}{\pgfqpoint{1.962469in}{1.968538in}}%
\pgfpathcurveto{\pgfqpoint{1.968293in}{1.962714in}}{\pgfqpoint{1.976193in}{1.959442in}}{\pgfqpoint{1.984429in}{1.959442in}}%
\pgfpathclose%
\pgfusepath{stroke,fill}%
\end{pgfscope}%
\begin{pgfscope}%
\pgfpathrectangle{\pgfqpoint{0.100000in}{0.212622in}}{\pgfqpoint{3.696000in}{3.696000in}}%
\pgfusepath{clip}%
\pgfsetbuttcap%
\pgfsetroundjoin%
\definecolor{currentfill}{rgb}{0.121569,0.466667,0.705882}%
\pgfsetfillcolor{currentfill}%
\pgfsetfillopacity{0.754429}%
\pgfsetlinewidth{1.003750pt}%
\definecolor{currentstroke}{rgb}{0.121569,0.466667,0.705882}%
\pgfsetstrokecolor{currentstroke}%
\pgfsetstrokeopacity{0.754429}%
\pgfsetdash{}{0pt}%
\pgfpathmoveto{\pgfqpoint{3.072194in}{2.630894in}}%
\pgfpathcurveto{\pgfqpoint{3.080430in}{2.630894in}}{\pgfqpoint{3.088331in}{2.634167in}}{\pgfqpoint{3.094154in}{2.639991in}}%
\pgfpathcurveto{\pgfqpoint{3.099978in}{2.645815in}}{\pgfqpoint{3.103251in}{2.653715in}}{\pgfqpoint{3.103251in}{2.661951in}}%
\pgfpathcurveto{\pgfqpoint{3.103251in}{2.670187in}}{\pgfqpoint{3.099978in}{2.678087in}}{\pgfqpoint{3.094154in}{2.683911in}}%
\pgfpathcurveto{\pgfqpoint{3.088331in}{2.689735in}}{\pgfqpoint{3.080430in}{2.693007in}}{\pgfqpoint{3.072194in}{2.693007in}}%
\pgfpathcurveto{\pgfqpoint{3.063958in}{2.693007in}}{\pgfqpoint{3.056058in}{2.689735in}}{\pgfqpoint{3.050234in}{2.683911in}}%
\pgfpathcurveto{\pgfqpoint{3.044410in}{2.678087in}}{\pgfqpoint{3.041138in}{2.670187in}}{\pgfqpoint{3.041138in}{2.661951in}}%
\pgfpathcurveto{\pgfqpoint{3.041138in}{2.653715in}}{\pgfqpoint{3.044410in}{2.645815in}}{\pgfqpoint{3.050234in}{2.639991in}}%
\pgfpathcurveto{\pgfqpoint{3.056058in}{2.634167in}}{\pgfqpoint{3.063958in}{2.630894in}}{\pgfqpoint{3.072194in}{2.630894in}}%
\pgfpathclose%
\pgfusepath{stroke,fill}%
\end{pgfscope}%
\begin{pgfscope}%
\pgfpathrectangle{\pgfqpoint{0.100000in}{0.212622in}}{\pgfqpoint{3.696000in}{3.696000in}}%
\pgfusepath{clip}%
\pgfsetbuttcap%
\pgfsetroundjoin%
\definecolor{currentfill}{rgb}{0.121569,0.466667,0.705882}%
\pgfsetfillcolor{currentfill}%
\pgfsetfillopacity{0.758728}%
\pgfsetlinewidth{1.003750pt}%
\definecolor{currentstroke}{rgb}{0.121569,0.466667,0.705882}%
\pgfsetstrokecolor{currentstroke}%
\pgfsetstrokeopacity{0.758728}%
\pgfsetdash{}{0pt}%
\pgfpathmoveto{\pgfqpoint{1.970383in}{1.943157in}}%
\pgfpathcurveto{\pgfqpoint{1.978620in}{1.943157in}}{\pgfqpoint{1.986520in}{1.946429in}}{\pgfqpoint{1.992344in}{1.952253in}}%
\pgfpathcurveto{\pgfqpoint{1.998167in}{1.958077in}}{\pgfqpoint{2.001440in}{1.965977in}}{\pgfqpoint{2.001440in}{1.974213in}}%
\pgfpathcurveto{\pgfqpoint{2.001440in}{1.982450in}}{\pgfqpoint{1.998167in}{1.990350in}}{\pgfqpoint{1.992344in}{1.996174in}}%
\pgfpathcurveto{\pgfqpoint{1.986520in}{2.001997in}}{\pgfqpoint{1.978620in}{2.005270in}}{\pgfqpoint{1.970383in}{2.005270in}}%
\pgfpathcurveto{\pgfqpoint{1.962147in}{2.005270in}}{\pgfqpoint{1.954247in}{2.001997in}}{\pgfqpoint{1.948423in}{1.996174in}}%
\pgfpathcurveto{\pgfqpoint{1.942599in}{1.990350in}}{\pgfqpoint{1.939327in}{1.982450in}}{\pgfqpoint{1.939327in}{1.974213in}}%
\pgfpathcurveto{\pgfqpoint{1.939327in}{1.965977in}}{\pgfqpoint{1.942599in}{1.958077in}}{\pgfqpoint{1.948423in}{1.952253in}}%
\pgfpathcurveto{\pgfqpoint{1.954247in}{1.946429in}}{\pgfqpoint{1.962147in}{1.943157in}}{\pgfqpoint{1.970383in}{1.943157in}}%
\pgfpathclose%
\pgfusepath{stroke,fill}%
\end{pgfscope}%
\begin{pgfscope}%
\pgfpathrectangle{\pgfqpoint{0.100000in}{0.212622in}}{\pgfqpoint{3.696000in}{3.696000in}}%
\pgfusepath{clip}%
\pgfsetbuttcap%
\pgfsetroundjoin%
\definecolor{currentfill}{rgb}{0.121569,0.466667,0.705882}%
\pgfsetfillcolor{currentfill}%
\pgfsetfillopacity{0.761299}%
\pgfsetlinewidth{1.003750pt}%
\definecolor{currentstroke}{rgb}{0.121569,0.466667,0.705882}%
\pgfsetstrokecolor{currentstroke}%
\pgfsetstrokeopacity{0.761299}%
\pgfsetdash{}{0pt}%
\pgfpathmoveto{\pgfqpoint{1.976227in}{1.950513in}}%
\pgfpathcurveto{\pgfqpoint{1.984463in}{1.950513in}}{\pgfqpoint{1.992363in}{1.953785in}}{\pgfqpoint{1.998187in}{1.959609in}}%
\pgfpathcurveto{\pgfqpoint{2.004011in}{1.965433in}}{\pgfqpoint{2.007283in}{1.973333in}}{\pgfqpoint{2.007283in}{1.981569in}}%
\pgfpathcurveto{\pgfqpoint{2.007283in}{1.989806in}}{\pgfqpoint{2.004011in}{1.997706in}}{\pgfqpoint{1.998187in}{2.003530in}}%
\pgfpathcurveto{\pgfqpoint{1.992363in}{2.009354in}}{\pgfqpoint{1.984463in}{2.012626in}}{\pgfqpoint{1.976227in}{2.012626in}}%
\pgfpathcurveto{\pgfqpoint{1.967990in}{2.012626in}}{\pgfqpoint{1.960090in}{2.009354in}}{\pgfqpoint{1.954266in}{2.003530in}}%
\pgfpathcurveto{\pgfqpoint{1.948442in}{1.997706in}}{\pgfqpoint{1.945170in}{1.989806in}}{\pgfqpoint{1.945170in}{1.981569in}}%
\pgfpathcurveto{\pgfqpoint{1.945170in}{1.973333in}}{\pgfqpoint{1.948442in}{1.965433in}}{\pgfqpoint{1.954266in}{1.959609in}}%
\pgfpathcurveto{\pgfqpoint{1.960090in}{1.953785in}}{\pgfqpoint{1.967990in}{1.950513in}}{\pgfqpoint{1.976227in}{1.950513in}}%
\pgfpathclose%
\pgfusepath{stroke,fill}%
\end{pgfscope}%
\begin{pgfscope}%
\pgfpathrectangle{\pgfqpoint{0.100000in}{0.212622in}}{\pgfqpoint{3.696000in}{3.696000in}}%
\pgfusepath{clip}%
\pgfsetbuttcap%
\pgfsetroundjoin%
\definecolor{currentfill}{rgb}{0.121569,0.466667,0.705882}%
\pgfsetfillcolor{currentfill}%
\pgfsetfillopacity{0.773186}%
\pgfsetlinewidth{1.003750pt}%
\definecolor{currentstroke}{rgb}{0.121569,0.466667,0.705882}%
\pgfsetstrokecolor{currentstroke}%
\pgfsetstrokeopacity{0.773186}%
\pgfsetdash{}{0pt}%
\pgfpathmoveto{\pgfqpoint{3.018770in}{2.583402in}}%
\pgfpathcurveto{\pgfqpoint{3.027007in}{2.583402in}}{\pgfqpoint{3.034907in}{2.586675in}}{\pgfqpoint{3.040731in}{2.592498in}}%
\pgfpathcurveto{\pgfqpoint{3.046554in}{2.598322in}}{\pgfqpoint{3.049827in}{2.606222in}}{\pgfqpoint{3.049827in}{2.614459in}}%
\pgfpathcurveto{\pgfqpoint{3.049827in}{2.622695in}}{\pgfqpoint{3.046554in}{2.630595in}}{\pgfqpoint{3.040731in}{2.636419in}}%
\pgfpathcurveto{\pgfqpoint{3.034907in}{2.642243in}}{\pgfqpoint{3.027007in}{2.645515in}}{\pgfqpoint{3.018770in}{2.645515in}}%
\pgfpathcurveto{\pgfqpoint{3.010534in}{2.645515in}}{\pgfqpoint{3.002634in}{2.642243in}}{\pgfqpoint{2.996810in}{2.636419in}}%
\pgfpathcurveto{\pgfqpoint{2.990986in}{2.630595in}}{\pgfqpoint{2.987714in}{2.622695in}}{\pgfqpoint{2.987714in}{2.614459in}}%
\pgfpathcurveto{\pgfqpoint{2.987714in}{2.606222in}}{\pgfqpoint{2.990986in}{2.598322in}}{\pgfqpoint{2.996810in}{2.592498in}}%
\pgfpathcurveto{\pgfqpoint{3.002634in}{2.586675in}}{\pgfqpoint{3.010534in}{2.583402in}}{\pgfqpoint{3.018770in}{2.583402in}}%
\pgfpathclose%
\pgfusepath{stroke,fill}%
\end{pgfscope}%
\begin{pgfscope}%
\pgfpathrectangle{\pgfqpoint{0.100000in}{0.212622in}}{\pgfqpoint{3.696000in}{3.696000in}}%
\pgfusepath{clip}%
\pgfsetbuttcap%
\pgfsetroundjoin%
\definecolor{currentfill}{rgb}{0.121569,0.466667,0.705882}%
\pgfsetfillcolor{currentfill}%
\pgfsetfillopacity{0.775217}%
\pgfsetlinewidth{1.003750pt}%
\definecolor{currentstroke}{rgb}{0.121569,0.466667,0.705882}%
\pgfsetstrokecolor{currentstroke}%
\pgfsetstrokeopacity{0.775217}%
\pgfsetdash{}{0pt}%
\pgfpathmoveto{\pgfqpoint{1.948607in}{1.923041in}}%
\pgfpathcurveto{\pgfqpoint{1.956843in}{1.923041in}}{\pgfqpoint{1.964743in}{1.926314in}}{\pgfqpoint{1.970567in}{1.932138in}}%
\pgfpathcurveto{\pgfqpoint{1.976391in}{1.937962in}}{\pgfqpoint{1.979663in}{1.945862in}}{\pgfqpoint{1.979663in}{1.954098in}}%
\pgfpathcurveto{\pgfqpoint{1.979663in}{1.962334in}}{\pgfqpoint{1.976391in}{1.970234in}}{\pgfqpoint{1.970567in}{1.976058in}}%
\pgfpathcurveto{\pgfqpoint{1.964743in}{1.981882in}}{\pgfqpoint{1.956843in}{1.985154in}}{\pgfqpoint{1.948607in}{1.985154in}}%
\pgfpathcurveto{\pgfqpoint{1.940371in}{1.985154in}}{\pgfqpoint{1.932471in}{1.981882in}}{\pgfqpoint{1.926647in}{1.976058in}}%
\pgfpathcurveto{\pgfqpoint{1.920823in}{1.970234in}}{\pgfqpoint{1.917550in}{1.962334in}}{\pgfqpoint{1.917550in}{1.954098in}}%
\pgfpathcurveto{\pgfqpoint{1.917550in}{1.945862in}}{\pgfqpoint{1.920823in}{1.937962in}}{\pgfqpoint{1.926647in}{1.932138in}}%
\pgfpathcurveto{\pgfqpoint{1.932471in}{1.926314in}}{\pgfqpoint{1.940371in}{1.923041in}}{\pgfqpoint{1.948607in}{1.923041in}}%
\pgfpathclose%
\pgfusepath{stroke,fill}%
\end{pgfscope}%
\begin{pgfscope}%
\pgfpathrectangle{\pgfqpoint{0.100000in}{0.212622in}}{\pgfqpoint{3.696000in}{3.696000in}}%
\pgfusepath{clip}%
\pgfsetbuttcap%
\pgfsetroundjoin%
\definecolor{currentfill}{rgb}{0.121569,0.466667,0.705882}%
\pgfsetfillcolor{currentfill}%
\pgfsetfillopacity{0.777093}%
\pgfsetlinewidth{1.003750pt}%
\definecolor{currentstroke}{rgb}{0.121569,0.466667,0.705882}%
\pgfsetstrokecolor{currentstroke}%
\pgfsetstrokeopacity{0.777093}%
\pgfsetdash{}{0pt}%
\pgfpathmoveto{\pgfqpoint{2.047939in}{1.986077in}}%
\pgfpathcurveto{\pgfqpoint{2.056175in}{1.986077in}}{\pgfqpoint{2.064075in}{1.989349in}}{\pgfqpoint{2.069899in}{1.995173in}}%
\pgfpathcurveto{\pgfqpoint{2.075723in}{2.000997in}}{\pgfqpoint{2.078995in}{2.008897in}}{\pgfqpoint{2.078995in}{2.017134in}}%
\pgfpathcurveto{\pgfqpoint{2.078995in}{2.025370in}}{\pgfqpoint{2.075723in}{2.033270in}}{\pgfqpoint{2.069899in}{2.039094in}}%
\pgfpathcurveto{\pgfqpoint{2.064075in}{2.044918in}}{\pgfqpoint{2.056175in}{2.048190in}}{\pgfqpoint{2.047939in}{2.048190in}}%
\pgfpathcurveto{\pgfqpoint{2.039702in}{2.048190in}}{\pgfqpoint{2.031802in}{2.044918in}}{\pgfqpoint{2.025978in}{2.039094in}}%
\pgfpathcurveto{\pgfqpoint{2.020154in}{2.033270in}}{\pgfqpoint{2.016882in}{2.025370in}}{\pgfqpoint{2.016882in}{2.017134in}}%
\pgfpathcurveto{\pgfqpoint{2.016882in}{2.008897in}}{\pgfqpoint{2.020154in}{2.000997in}}{\pgfqpoint{2.025978in}{1.995173in}}%
\pgfpathcurveto{\pgfqpoint{2.031802in}{1.989349in}}{\pgfqpoint{2.039702in}{1.986077in}}{\pgfqpoint{2.047939in}{1.986077in}}%
\pgfpathclose%
\pgfusepath{stroke,fill}%
\end{pgfscope}%
\begin{pgfscope}%
\pgfpathrectangle{\pgfqpoint{0.100000in}{0.212622in}}{\pgfqpoint{3.696000in}{3.696000in}}%
\pgfusepath{clip}%
\pgfsetbuttcap%
\pgfsetroundjoin%
\definecolor{currentfill}{rgb}{0.121569,0.466667,0.705882}%
\pgfsetfillcolor{currentfill}%
\pgfsetfillopacity{0.780327}%
\pgfsetlinewidth{1.003750pt}%
\definecolor{currentstroke}{rgb}{0.121569,0.466667,0.705882}%
\pgfsetstrokecolor{currentstroke}%
\pgfsetstrokeopacity{0.780327}%
\pgfsetdash{}{0pt}%
\pgfpathmoveto{\pgfqpoint{2.083310in}{2.007952in}}%
\pgfpathcurveto{\pgfqpoint{2.091546in}{2.007952in}}{\pgfqpoint{2.099446in}{2.011224in}}{\pgfqpoint{2.105270in}{2.017048in}}%
\pgfpathcurveto{\pgfqpoint{2.111094in}{2.022872in}}{\pgfqpoint{2.114366in}{2.030772in}}{\pgfqpoint{2.114366in}{2.039008in}}%
\pgfpathcurveto{\pgfqpoint{2.114366in}{2.047244in}}{\pgfqpoint{2.111094in}{2.055144in}}{\pgfqpoint{2.105270in}{2.060968in}}%
\pgfpathcurveto{\pgfqpoint{2.099446in}{2.066792in}}{\pgfqpoint{2.091546in}{2.070065in}}{\pgfqpoint{2.083310in}{2.070065in}}%
\pgfpathcurveto{\pgfqpoint{2.075073in}{2.070065in}}{\pgfqpoint{2.067173in}{2.066792in}}{\pgfqpoint{2.061349in}{2.060968in}}%
\pgfpathcurveto{\pgfqpoint{2.055525in}{2.055144in}}{\pgfqpoint{2.052253in}{2.047244in}}{\pgfqpoint{2.052253in}{2.039008in}}%
\pgfpathcurveto{\pgfqpoint{2.052253in}{2.030772in}}{\pgfqpoint{2.055525in}{2.022872in}}{\pgfqpoint{2.061349in}{2.017048in}}%
\pgfpathcurveto{\pgfqpoint{2.067173in}{2.011224in}}{\pgfqpoint{2.075073in}{2.007952in}}{\pgfqpoint{2.083310in}{2.007952in}}%
\pgfpathclose%
\pgfusepath{stroke,fill}%
\end{pgfscope}%
\begin{pgfscope}%
\pgfpathrectangle{\pgfqpoint{0.100000in}{0.212622in}}{\pgfqpoint{3.696000in}{3.696000in}}%
\pgfusepath{clip}%
\pgfsetbuttcap%
\pgfsetroundjoin%
\definecolor{currentfill}{rgb}{0.121569,0.466667,0.705882}%
\pgfsetfillcolor{currentfill}%
\pgfsetfillopacity{0.780378}%
\pgfsetlinewidth{1.003750pt}%
\definecolor{currentstroke}{rgb}{0.121569,0.466667,0.705882}%
\pgfsetstrokecolor{currentstroke}%
\pgfsetstrokeopacity{0.780378}%
\pgfsetdash{}{0pt}%
\pgfpathmoveto{\pgfqpoint{1.949769in}{1.926130in}}%
\pgfpathcurveto{\pgfqpoint{1.958005in}{1.926130in}}{\pgfqpoint{1.965906in}{1.929402in}}{\pgfqpoint{1.971729in}{1.935226in}}%
\pgfpathcurveto{\pgfqpoint{1.977553in}{1.941050in}}{\pgfqpoint{1.980826in}{1.948950in}}{\pgfqpoint{1.980826in}{1.957186in}}%
\pgfpathcurveto{\pgfqpoint{1.980826in}{1.965423in}}{\pgfqpoint{1.977553in}{1.973323in}}{\pgfqpoint{1.971729in}{1.979147in}}%
\pgfpathcurveto{\pgfqpoint{1.965906in}{1.984971in}}{\pgfqpoint{1.958005in}{1.988243in}}{\pgfqpoint{1.949769in}{1.988243in}}%
\pgfpathcurveto{\pgfqpoint{1.941533in}{1.988243in}}{\pgfqpoint{1.933633in}{1.984971in}}{\pgfqpoint{1.927809in}{1.979147in}}%
\pgfpathcurveto{\pgfqpoint{1.921985in}{1.973323in}}{\pgfqpoint{1.918713in}{1.965423in}}{\pgfqpoint{1.918713in}{1.957186in}}%
\pgfpathcurveto{\pgfqpoint{1.918713in}{1.948950in}}{\pgfqpoint{1.921985in}{1.941050in}}{\pgfqpoint{1.927809in}{1.935226in}}%
\pgfpathcurveto{\pgfqpoint{1.933633in}{1.929402in}}{\pgfqpoint{1.941533in}{1.926130in}}{\pgfqpoint{1.949769in}{1.926130in}}%
\pgfpathclose%
\pgfusepath{stroke,fill}%
\end{pgfscope}%
\begin{pgfscope}%
\pgfpathrectangle{\pgfqpoint{0.100000in}{0.212622in}}{\pgfqpoint{3.696000in}{3.696000in}}%
\pgfusepath{clip}%
\pgfsetbuttcap%
\pgfsetroundjoin%
\definecolor{currentfill}{rgb}{0.121569,0.466667,0.705882}%
\pgfsetfillcolor{currentfill}%
\pgfsetfillopacity{0.781920}%
\pgfsetlinewidth{1.003750pt}%
\definecolor{currentstroke}{rgb}{0.121569,0.466667,0.705882}%
\pgfsetstrokecolor{currentstroke}%
\pgfsetstrokeopacity{0.781920}%
\pgfsetdash{}{0pt}%
\pgfpathmoveto{\pgfqpoint{1.961916in}{1.936257in}}%
\pgfpathcurveto{\pgfqpoint{1.970153in}{1.936257in}}{\pgfqpoint{1.978053in}{1.939529in}}{\pgfqpoint{1.983877in}{1.945353in}}%
\pgfpathcurveto{\pgfqpoint{1.989701in}{1.951177in}}{\pgfqpoint{1.992973in}{1.959077in}}{\pgfqpoint{1.992973in}{1.967313in}}%
\pgfpathcurveto{\pgfqpoint{1.992973in}{1.975549in}}{\pgfqpoint{1.989701in}{1.983449in}}{\pgfqpoint{1.983877in}{1.989273in}}%
\pgfpathcurveto{\pgfqpoint{1.978053in}{1.995097in}}{\pgfqpoint{1.970153in}{1.998370in}}{\pgfqpoint{1.961916in}{1.998370in}}%
\pgfpathcurveto{\pgfqpoint{1.953680in}{1.998370in}}{\pgfqpoint{1.945780in}{1.995097in}}{\pgfqpoint{1.939956in}{1.989273in}}%
\pgfpathcurveto{\pgfqpoint{1.934132in}{1.983449in}}{\pgfqpoint{1.930860in}{1.975549in}}{\pgfqpoint{1.930860in}{1.967313in}}%
\pgfpathcurveto{\pgfqpoint{1.930860in}{1.959077in}}{\pgfqpoint{1.934132in}{1.951177in}}{\pgfqpoint{1.939956in}{1.945353in}}%
\pgfpathcurveto{\pgfqpoint{1.945780in}{1.939529in}}{\pgfqpoint{1.953680in}{1.936257in}}{\pgfqpoint{1.961916in}{1.936257in}}%
\pgfpathclose%
\pgfusepath{stroke,fill}%
\end{pgfscope}%
\begin{pgfscope}%
\pgfpathrectangle{\pgfqpoint{0.100000in}{0.212622in}}{\pgfqpoint{3.696000in}{3.696000in}}%
\pgfusepath{clip}%
\pgfsetbuttcap%
\pgfsetroundjoin%
\definecolor{currentfill}{rgb}{0.121569,0.466667,0.705882}%
\pgfsetfillcolor{currentfill}%
\pgfsetfillopacity{0.782817}%
\pgfsetlinewidth{1.003750pt}%
\definecolor{currentstroke}{rgb}{0.121569,0.466667,0.705882}%
\pgfsetstrokecolor{currentstroke}%
\pgfsetstrokeopacity{0.782817}%
\pgfsetdash{}{0pt}%
\pgfpathmoveto{\pgfqpoint{2.052109in}{1.981491in}}%
\pgfpathcurveto{\pgfqpoint{2.060345in}{1.981491in}}{\pgfqpoint{2.068245in}{1.984763in}}{\pgfqpoint{2.074069in}{1.990587in}}%
\pgfpathcurveto{\pgfqpoint{2.079893in}{1.996411in}}{\pgfqpoint{2.083165in}{2.004311in}}{\pgfqpoint{2.083165in}{2.012548in}}%
\pgfpathcurveto{\pgfqpoint{2.083165in}{2.020784in}}{\pgfqpoint{2.079893in}{2.028684in}}{\pgfqpoint{2.074069in}{2.034508in}}%
\pgfpathcurveto{\pgfqpoint{2.068245in}{2.040332in}}{\pgfqpoint{2.060345in}{2.043604in}}{\pgfqpoint{2.052109in}{2.043604in}}%
\pgfpathcurveto{\pgfqpoint{2.043873in}{2.043604in}}{\pgfqpoint{2.035972in}{2.040332in}}{\pgfqpoint{2.030149in}{2.034508in}}%
\pgfpathcurveto{\pgfqpoint{2.024325in}{2.028684in}}{\pgfqpoint{2.021052in}{2.020784in}}{\pgfqpoint{2.021052in}{2.012548in}}%
\pgfpathcurveto{\pgfqpoint{2.021052in}{2.004311in}}{\pgfqpoint{2.024325in}{1.996411in}}{\pgfqpoint{2.030149in}{1.990587in}}%
\pgfpathcurveto{\pgfqpoint{2.035972in}{1.984763in}}{\pgfqpoint{2.043873in}{1.981491in}}{\pgfqpoint{2.052109in}{1.981491in}}%
\pgfpathclose%
\pgfusepath{stroke,fill}%
\end{pgfscope}%
\begin{pgfscope}%
\pgfpathrectangle{\pgfqpoint{0.100000in}{0.212622in}}{\pgfqpoint{3.696000in}{3.696000in}}%
\pgfusepath{clip}%
\pgfsetbuttcap%
\pgfsetroundjoin%
\definecolor{currentfill}{rgb}{0.121569,0.466667,0.705882}%
\pgfsetfillcolor{currentfill}%
\pgfsetfillopacity{0.785065}%
\pgfsetlinewidth{1.003750pt}%
\definecolor{currentstroke}{rgb}{0.121569,0.466667,0.705882}%
\pgfsetstrokecolor{currentstroke}%
\pgfsetstrokeopacity{0.785065}%
\pgfsetdash{}{0pt}%
\pgfpathmoveto{\pgfqpoint{2.006042in}{1.956740in}}%
\pgfpathcurveto{\pgfqpoint{2.014278in}{1.956740in}}{\pgfqpoint{2.022178in}{1.960012in}}{\pgfqpoint{2.028002in}{1.965836in}}%
\pgfpathcurveto{\pgfqpoint{2.033826in}{1.971660in}}{\pgfqpoint{2.037098in}{1.979560in}}{\pgfqpoint{2.037098in}{1.987797in}}%
\pgfpathcurveto{\pgfqpoint{2.037098in}{1.996033in}}{\pgfqpoint{2.033826in}{2.003933in}}{\pgfqpoint{2.028002in}{2.009757in}}%
\pgfpathcurveto{\pgfqpoint{2.022178in}{2.015581in}}{\pgfqpoint{2.014278in}{2.018853in}}{\pgfqpoint{2.006042in}{2.018853in}}%
\pgfpathcurveto{\pgfqpoint{1.997806in}{2.018853in}}{\pgfqpoint{1.989906in}{2.015581in}}{\pgfqpoint{1.984082in}{2.009757in}}%
\pgfpathcurveto{\pgfqpoint{1.978258in}{2.003933in}}{\pgfqpoint{1.974985in}{1.996033in}}{\pgfqpoint{1.974985in}{1.987797in}}%
\pgfpathcurveto{\pgfqpoint{1.974985in}{1.979560in}}{\pgfqpoint{1.978258in}{1.971660in}}{\pgfqpoint{1.984082in}{1.965836in}}%
\pgfpathcurveto{\pgfqpoint{1.989906in}{1.960012in}}{\pgfqpoint{1.997806in}{1.956740in}}{\pgfqpoint{2.006042in}{1.956740in}}%
\pgfpathclose%
\pgfusepath{stroke,fill}%
\end{pgfscope}%
\begin{pgfscope}%
\pgfpathrectangle{\pgfqpoint{0.100000in}{0.212622in}}{\pgfqpoint{3.696000in}{3.696000in}}%
\pgfusepath{clip}%
\pgfsetbuttcap%
\pgfsetroundjoin%
\definecolor{currentfill}{rgb}{0.121569,0.466667,0.705882}%
\pgfsetfillcolor{currentfill}%
\pgfsetfillopacity{0.785270}%
\pgfsetlinewidth{1.003750pt}%
\definecolor{currentstroke}{rgb}{0.121569,0.466667,0.705882}%
\pgfsetstrokecolor{currentstroke}%
\pgfsetstrokeopacity{0.785270}%
\pgfsetdash{}{0pt}%
\pgfpathmoveto{\pgfqpoint{1.988269in}{1.951471in}}%
\pgfpathcurveto{\pgfqpoint{1.996505in}{1.951471in}}{\pgfqpoint{2.004405in}{1.954743in}}{\pgfqpoint{2.010229in}{1.960567in}}%
\pgfpathcurveto{\pgfqpoint{2.016053in}{1.966391in}}{\pgfqpoint{2.019325in}{1.974291in}}{\pgfqpoint{2.019325in}{1.982527in}}%
\pgfpathcurveto{\pgfqpoint{2.019325in}{1.990763in}}{\pgfqpoint{2.016053in}{1.998664in}}{\pgfqpoint{2.010229in}{2.004487in}}%
\pgfpathcurveto{\pgfqpoint{2.004405in}{2.010311in}}{\pgfqpoint{1.996505in}{2.013584in}}{\pgfqpoint{1.988269in}{2.013584in}}%
\pgfpathcurveto{\pgfqpoint{1.980032in}{2.013584in}}{\pgfqpoint{1.972132in}{2.010311in}}{\pgfqpoint{1.966309in}{2.004487in}}%
\pgfpathcurveto{\pgfqpoint{1.960485in}{1.998664in}}{\pgfqpoint{1.957212in}{1.990763in}}{\pgfqpoint{1.957212in}{1.982527in}}%
\pgfpathcurveto{\pgfqpoint{1.957212in}{1.974291in}}{\pgfqpoint{1.960485in}{1.966391in}}{\pgfqpoint{1.966309in}{1.960567in}}%
\pgfpathcurveto{\pgfqpoint{1.972132in}{1.954743in}}{\pgfqpoint{1.980032in}{1.951471in}}{\pgfqpoint{1.988269in}{1.951471in}}%
\pgfpathclose%
\pgfusepath{stroke,fill}%
\end{pgfscope}%
\begin{pgfscope}%
\pgfpathrectangle{\pgfqpoint{0.100000in}{0.212622in}}{\pgfqpoint{3.696000in}{3.696000in}}%
\pgfusepath{clip}%
\pgfsetbuttcap%
\pgfsetroundjoin%
\definecolor{currentfill}{rgb}{0.121569,0.466667,0.705882}%
\pgfsetfillcolor{currentfill}%
\pgfsetfillopacity{0.786134}%
\pgfsetlinewidth{1.003750pt}%
\definecolor{currentstroke}{rgb}{0.121569,0.466667,0.705882}%
\pgfsetstrokecolor{currentstroke}%
\pgfsetstrokeopacity{0.786134}%
\pgfsetdash{}{0pt}%
\pgfpathmoveto{\pgfqpoint{2.261552in}{2.140435in}}%
\pgfpathcurveto{\pgfqpoint{2.269789in}{2.140435in}}{\pgfqpoint{2.277689in}{2.143707in}}{\pgfqpoint{2.283513in}{2.149531in}}%
\pgfpathcurveto{\pgfqpoint{2.289337in}{2.155355in}}{\pgfqpoint{2.292609in}{2.163255in}}{\pgfqpoint{2.292609in}{2.171491in}}%
\pgfpathcurveto{\pgfqpoint{2.292609in}{2.179727in}}{\pgfqpoint{2.289337in}{2.187628in}}{\pgfqpoint{2.283513in}{2.193451in}}%
\pgfpathcurveto{\pgfqpoint{2.277689in}{2.199275in}}{\pgfqpoint{2.269789in}{2.202548in}}{\pgfqpoint{2.261552in}{2.202548in}}%
\pgfpathcurveto{\pgfqpoint{2.253316in}{2.202548in}}{\pgfqpoint{2.245416in}{2.199275in}}{\pgfqpoint{2.239592in}{2.193451in}}%
\pgfpathcurveto{\pgfqpoint{2.233768in}{2.187628in}}{\pgfqpoint{2.230496in}{2.179727in}}{\pgfqpoint{2.230496in}{2.171491in}}%
\pgfpathcurveto{\pgfqpoint{2.230496in}{2.163255in}}{\pgfqpoint{2.233768in}{2.155355in}}{\pgfqpoint{2.239592in}{2.149531in}}%
\pgfpathcurveto{\pgfqpoint{2.245416in}{2.143707in}}{\pgfqpoint{2.253316in}{2.140435in}}{\pgfqpoint{2.261552in}{2.140435in}}%
\pgfpathclose%
\pgfusepath{stroke,fill}%
\end{pgfscope}%
\begin{pgfscope}%
\pgfpathrectangle{\pgfqpoint{0.100000in}{0.212622in}}{\pgfqpoint{3.696000in}{3.696000in}}%
\pgfusepath{clip}%
\pgfsetbuttcap%
\pgfsetroundjoin%
\definecolor{currentfill}{rgb}{0.121569,0.466667,0.705882}%
\pgfsetfillcolor{currentfill}%
\pgfsetfillopacity{0.788176}%
\pgfsetlinewidth{1.003750pt}%
\definecolor{currentstroke}{rgb}{0.121569,0.466667,0.705882}%
\pgfsetstrokecolor{currentstroke}%
\pgfsetstrokeopacity{0.788176}%
\pgfsetdash{}{0pt}%
\pgfpathmoveto{\pgfqpoint{2.345807in}{2.174053in}}%
\pgfpathcurveto{\pgfqpoint{2.354044in}{2.174053in}}{\pgfqpoint{2.361944in}{2.177326in}}{\pgfqpoint{2.367768in}{2.183150in}}%
\pgfpathcurveto{\pgfqpoint{2.373592in}{2.188974in}}{\pgfqpoint{2.376864in}{2.196874in}}{\pgfqpoint{2.376864in}{2.205110in}}%
\pgfpathcurveto{\pgfqpoint{2.376864in}{2.213346in}}{\pgfqpoint{2.373592in}{2.221246in}}{\pgfqpoint{2.367768in}{2.227070in}}%
\pgfpathcurveto{\pgfqpoint{2.361944in}{2.232894in}}{\pgfqpoint{2.354044in}{2.236166in}}{\pgfqpoint{2.345807in}{2.236166in}}%
\pgfpathcurveto{\pgfqpoint{2.337571in}{2.236166in}}{\pgfqpoint{2.329671in}{2.232894in}}{\pgfqpoint{2.323847in}{2.227070in}}%
\pgfpathcurveto{\pgfqpoint{2.318023in}{2.221246in}}{\pgfqpoint{2.314751in}{2.213346in}}{\pgfqpoint{2.314751in}{2.205110in}}%
\pgfpathcurveto{\pgfqpoint{2.314751in}{2.196874in}}{\pgfqpoint{2.318023in}{2.188974in}}{\pgfqpoint{2.323847in}{2.183150in}}%
\pgfpathcurveto{\pgfqpoint{2.329671in}{2.177326in}}{\pgfqpoint{2.337571in}{2.174053in}}{\pgfqpoint{2.345807in}{2.174053in}}%
\pgfpathclose%
\pgfusepath{stroke,fill}%
\end{pgfscope}%
\begin{pgfscope}%
\pgfpathrectangle{\pgfqpoint{0.100000in}{0.212622in}}{\pgfqpoint{3.696000in}{3.696000in}}%
\pgfusepath{clip}%
\pgfsetbuttcap%
\pgfsetroundjoin%
\definecolor{currentfill}{rgb}{0.121569,0.466667,0.705882}%
\pgfsetfillcolor{currentfill}%
\pgfsetfillopacity{0.789277}%
\pgfsetlinewidth{1.003750pt}%
\definecolor{currentstroke}{rgb}{0.121569,0.466667,0.705882}%
\pgfsetstrokecolor{currentstroke}%
\pgfsetstrokeopacity{0.789277}%
\pgfsetdash{}{0pt}%
\pgfpathmoveto{\pgfqpoint{2.340558in}{2.168048in}}%
\pgfpathcurveto{\pgfqpoint{2.348795in}{2.168048in}}{\pgfqpoint{2.356695in}{2.171320in}}{\pgfqpoint{2.362519in}{2.177144in}}%
\pgfpathcurveto{\pgfqpoint{2.368343in}{2.182968in}}{\pgfqpoint{2.371615in}{2.190868in}}{\pgfqpoint{2.371615in}{2.199105in}}%
\pgfpathcurveto{\pgfqpoint{2.371615in}{2.207341in}}{\pgfqpoint{2.368343in}{2.215241in}}{\pgfqpoint{2.362519in}{2.221065in}}%
\pgfpathcurveto{\pgfqpoint{2.356695in}{2.226889in}}{\pgfqpoint{2.348795in}{2.230161in}}{\pgfqpoint{2.340558in}{2.230161in}}%
\pgfpathcurveto{\pgfqpoint{2.332322in}{2.230161in}}{\pgfqpoint{2.324422in}{2.226889in}}{\pgfqpoint{2.318598in}{2.221065in}}%
\pgfpathcurveto{\pgfqpoint{2.312774in}{2.215241in}}{\pgfqpoint{2.309502in}{2.207341in}}{\pgfqpoint{2.309502in}{2.199105in}}%
\pgfpathcurveto{\pgfqpoint{2.309502in}{2.190868in}}{\pgfqpoint{2.312774in}{2.182968in}}{\pgfqpoint{2.318598in}{2.177144in}}%
\pgfpathcurveto{\pgfqpoint{2.324422in}{2.171320in}}{\pgfqpoint{2.332322in}{2.168048in}}{\pgfqpoint{2.340558in}{2.168048in}}%
\pgfpathclose%
\pgfusepath{stroke,fill}%
\end{pgfscope}%
\begin{pgfscope}%
\pgfpathrectangle{\pgfqpoint{0.100000in}{0.212622in}}{\pgfqpoint{3.696000in}{3.696000in}}%
\pgfusepath{clip}%
\pgfsetbuttcap%
\pgfsetroundjoin%
\definecolor{currentfill}{rgb}{0.121569,0.466667,0.705882}%
\pgfsetfillcolor{currentfill}%
\pgfsetfillopacity{0.789550}%
\pgfsetlinewidth{1.003750pt}%
\definecolor{currentstroke}{rgb}{0.121569,0.466667,0.705882}%
\pgfsetstrokecolor{currentstroke}%
\pgfsetstrokeopacity{0.789550}%
\pgfsetdash{}{0pt}%
\pgfpathmoveto{\pgfqpoint{1.956973in}{1.926490in}}%
\pgfpathcurveto{\pgfqpoint{1.965210in}{1.926490in}}{\pgfqpoint{1.973110in}{1.929762in}}{\pgfqpoint{1.978933in}{1.935586in}}%
\pgfpathcurveto{\pgfqpoint{1.984757in}{1.941410in}}{\pgfqpoint{1.988030in}{1.949310in}}{\pgfqpoint{1.988030in}{1.957546in}}%
\pgfpathcurveto{\pgfqpoint{1.988030in}{1.965783in}}{\pgfqpoint{1.984757in}{1.973683in}}{\pgfqpoint{1.978933in}{1.979507in}}%
\pgfpathcurveto{\pgfqpoint{1.973110in}{1.985331in}}{\pgfqpoint{1.965210in}{1.988603in}}{\pgfqpoint{1.956973in}{1.988603in}}%
\pgfpathcurveto{\pgfqpoint{1.948737in}{1.988603in}}{\pgfqpoint{1.940837in}{1.985331in}}{\pgfqpoint{1.935013in}{1.979507in}}%
\pgfpathcurveto{\pgfqpoint{1.929189in}{1.973683in}}{\pgfqpoint{1.925917in}{1.965783in}}{\pgfqpoint{1.925917in}{1.957546in}}%
\pgfpathcurveto{\pgfqpoint{1.925917in}{1.949310in}}{\pgfqpoint{1.929189in}{1.941410in}}{\pgfqpoint{1.935013in}{1.935586in}}%
\pgfpathcurveto{\pgfqpoint{1.940837in}{1.929762in}}{\pgfqpoint{1.948737in}{1.926490in}}{\pgfqpoint{1.956973in}{1.926490in}}%
\pgfpathclose%
\pgfusepath{stroke,fill}%
\end{pgfscope}%
\begin{pgfscope}%
\pgfpathrectangle{\pgfqpoint{0.100000in}{0.212622in}}{\pgfqpoint{3.696000in}{3.696000in}}%
\pgfusepath{clip}%
\pgfsetbuttcap%
\pgfsetroundjoin%
\definecolor{currentfill}{rgb}{0.121569,0.466667,0.705882}%
\pgfsetfillcolor{currentfill}%
\pgfsetfillopacity{0.789727}%
\pgfsetlinewidth{1.003750pt}%
\definecolor{currentstroke}{rgb}{0.121569,0.466667,0.705882}%
\pgfsetstrokecolor{currentstroke}%
\pgfsetstrokeopacity{0.789727}%
\pgfsetdash{}{0pt}%
\pgfpathmoveto{\pgfqpoint{2.337274in}{2.167823in}}%
\pgfpathcurveto{\pgfqpoint{2.345510in}{2.167823in}}{\pgfqpoint{2.353410in}{2.171095in}}{\pgfqpoint{2.359234in}{2.176919in}}%
\pgfpathcurveto{\pgfqpoint{2.365058in}{2.182743in}}{\pgfqpoint{2.368330in}{2.190643in}}{\pgfqpoint{2.368330in}{2.198880in}}%
\pgfpathcurveto{\pgfqpoint{2.368330in}{2.207116in}}{\pgfqpoint{2.365058in}{2.215016in}}{\pgfqpoint{2.359234in}{2.220840in}}%
\pgfpathcurveto{\pgfqpoint{2.353410in}{2.226664in}}{\pgfqpoint{2.345510in}{2.229936in}}{\pgfqpoint{2.337274in}{2.229936in}}%
\pgfpathcurveto{\pgfqpoint{2.329038in}{2.229936in}}{\pgfqpoint{2.321138in}{2.226664in}}{\pgfqpoint{2.315314in}{2.220840in}}%
\pgfpathcurveto{\pgfqpoint{2.309490in}{2.215016in}}{\pgfqpoint{2.306217in}{2.207116in}}{\pgfqpoint{2.306217in}{2.198880in}}%
\pgfpathcurveto{\pgfqpoint{2.306217in}{2.190643in}}{\pgfqpoint{2.309490in}{2.182743in}}{\pgfqpoint{2.315314in}{2.176919in}}%
\pgfpathcurveto{\pgfqpoint{2.321138in}{2.171095in}}{\pgfqpoint{2.329038in}{2.167823in}}{\pgfqpoint{2.337274in}{2.167823in}}%
\pgfpathclose%
\pgfusepath{stroke,fill}%
\end{pgfscope}%
\begin{pgfscope}%
\pgfpathrectangle{\pgfqpoint{0.100000in}{0.212622in}}{\pgfqpoint{3.696000in}{3.696000in}}%
\pgfusepath{clip}%
\pgfsetbuttcap%
\pgfsetroundjoin%
\definecolor{currentfill}{rgb}{0.121569,0.466667,0.705882}%
\pgfsetfillcolor{currentfill}%
\pgfsetfillopacity{0.790205}%
\pgfsetlinewidth{1.003750pt}%
\definecolor{currentstroke}{rgb}{0.121569,0.466667,0.705882}%
\pgfsetstrokecolor{currentstroke}%
\pgfsetstrokeopacity{0.790205}%
\pgfsetdash{}{0pt}%
\pgfpathmoveto{\pgfqpoint{2.333845in}{2.165344in}}%
\pgfpathcurveto{\pgfqpoint{2.342082in}{2.165344in}}{\pgfqpoint{2.349982in}{2.168616in}}{\pgfqpoint{2.355806in}{2.174440in}}%
\pgfpathcurveto{\pgfqpoint{2.361630in}{2.180264in}}{\pgfqpoint{2.364902in}{2.188164in}}{\pgfqpoint{2.364902in}{2.196401in}}%
\pgfpathcurveto{\pgfqpoint{2.364902in}{2.204637in}}{\pgfqpoint{2.361630in}{2.212537in}}{\pgfqpoint{2.355806in}{2.218361in}}%
\pgfpathcurveto{\pgfqpoint{2.349982in}{2.224185in}}{\pgfqpoint{2.342082in}{2.227457in}}{\pgfqpoint{2.333845in}{2.227457in}}%
\pgfpathcurveto{\pgfqpoint{2.325609in}{2.227457in}}{\pgfqpoint{2.317709in}{2.224185in}}{\pgfqpoint{2.311885in}{2.218361in}}%
\pgfpathcurveto{\pgfqpoint{2.306061in}{2.212537in}}{\pgfqpoint{2.302789in}{2.204637in}}{\pgfqpoint{2.302789in}{2.196401in}}%
\pgfpathcurveto{\pgfqpoint{2.302789in}{2.188164in}}{\pgfqpoint{2.306061in}{2.180264in}}{\pgfqpoint{2.311885in}{2.174440in}}%
\pgfpathcurveto{\pgfqpoint{2.317709in}{2.168616in}}{\pgfqpoint{2.325609in}{2.165344in}}{\pgfqpoint{2.333845in}{2.165344in}}%
\pgfpathclose%
\pgfusepath{stroke,fill}%
\end{pgfscope}%
\begin{pgfscope}%
\pgfpathrectangle{\pgfqpoint{0.100000in}{0.212622in}}{\pgfqpoint{3.696000in}{3.696000in}}%
\pgfusepath{clip}%
\pgfsetbuttcap%
\pgfsetroundjoin%
\definecolor{currentfill}{rgb}{0.121569,0.466667,0.705882}%
\pgfsetfillcolor{currentfill}%
\pgfsetfillopacity{0.790246}%
\pgfsetlinewidth{1.003750pt}%
\definecolor{currentstroke}{rgb}{0.121569,0.466667,0.705882}%
\pgfsetstrokecolor{currentstroke}%
\pgfsetstrokeopacity{0.790246}%
\pgfsetdash{}{0pt}%
\pgfpathmoveto{\pgfqpoint{2.186738in}{2.075454in}}%
\pgfpathcurveto{\pgfqpoint{2.194975in}{2.075454in}}{\pgfqpoint{2.202875in}{2.078726in}}{\pgfqpoint{2.208699in}{2.084550in}}%
\pgfpathcurveto{\pgfqpoint{2.214523in}{2.090374in}}{\pgfqpoint{2.217795in}{2.098274in}}{\pgfqpoint{2.217795in}{2.106510in}}%
\pgfpathcurveto{\pgfqpoint{2.217795in}{2.114746in}}{\pgfqpoint{2.214523in}{2.122646in}}{\pgfqpoint{2.208699in}{2.128470in}}%
\pgfpathcurveto{\pgfqpoint{2.202875in}{2.134294in}}{\pgfqpoint{2.194975in}{2.137567in}}{\pgfqpoint{2.186738in}{2.137567in}}%
\pgfpathcurveto{\pgfqpoint{2.178502in}{2.137567in}}{\pgfqpoint{2.170602in}{2.134294in}}{\pgfqpoint{2.164778in}{2.128470in}}%
\pgfpathcurveto{\pgfqpoint{2.158954in}{2.122646in}}{\pgfqpoint{2.155682in}{2.114746in}}{\pgfqpoint{2.155682in}{2.106510in}}%
\pgfpathcurveto{\pgfqpoint{2.155682in}{2.098274in}}{\pgfqpoint{2.158954in}{2.090374in}}{\pgfqpoint{2.164778in}{2.084550in}}%
\pgfpathcurveto{\pgfqpoint{2.170602in}{2.078726in}}{\pgfqpoint{2.178502in}{2.075454in}}{\pgfqpoint{2.186738in}{2.075454in}}%
\pgfpathclose%
\pgfusepath{stroke,fill}%
\end{pgfscope}%
\begin{pgfscope}%
\pgfpathrectangle{\pgfqpoint{0.100000in}{0.212622in}}{\pgfqpoint{3.696000in}{3.696000in}}%
\pgfusepath{clip}%
\pgfsetbuttcap%
\pgfsetroundjoin%
\definecolor{currentfill}{rgb}{0.121569,0.466667,0.705882}%
\pgfsetfillcolor{currentfill}%
\pgfsetfillopacity{0.790371}%
\pgfsetlinewidth{1.003750pt}%
\definecolor{currentstroke}{rgb}{0.121569,0.466667,0.705882}%
\pgfsetstrokecolor{currentstroke}%
\pgfsetstrokeopacity{0.790371}%
\pgfsetdash{}{0pt}%
\pgfpathmoveto{\pgfqpoint{2.329848in}{2.163592in}}%
\pgfpathcurveto{\pgfqpoint{2.338084in}{2.163592in}}{\pgfqpoint{2.345984in}{2.166864in}}{\pgfqpoint{2.351808in}{2.172688in}}%
\pgfpathcurveto{\pgfqpoint{2.357632in}{2.178512in}}{\pgfqpoint{2.360905in}{2.186412in}}{\pgfqpoint{2.360905in}{2.194649in}}%
\pgfpathcurveto{\pgfqpoint{2.360905in}{2.202885in}}{\pgfqpoint{2.357632in}{2.210785in}}{\pgfqpoint{2.351808in}{2.216609in}}%
\pgfpathcurveto{\pgfqpoint{2.345984in}{2.222433in}}{\pgfqpoint{2.338084in}{2.225705in}}{\pgfqpoint{2.329848in}{2.225705in}}%
\pgfpathcurveto{\pgfqpoint{2.321612in}{2.225705in}}{\pgfqpoint{2.313712in}{2.222433in}}{\pgfqpoint{2.307888in}{2.216609in}}%
\pgfpathcurveto{\pgfqpoint{2.302064in}{2.210785in}}{\pgfqpoint{2.298792in}{2.202885in}}{\pgfqpoint{2.298792in}{2.194649in}}%
\pgfpathcurveto{\pgfqpoint{2.298792in}{2.186412in}}{\pgfqpoint{2.302064in}{2.178512in}}{\pgfqpoint{2.307888in}{2.172688in}}%
\pgfpathcurveto{\pgfqpoint{2.313712in}{2.166864in}}{\pgfqpoint{2.321612in}{2.163592in}}{\pgfqpoint{2.329848in}{2.163592in}}%
\pgfpathclose%
\pgfusepath{stroke,fill}%
\end{pgfscope}%
\begin{pgfscope}%
\pgfpathrectangle{\pgfqpoint{0.100000in}{0.212622in}}{\pgfqpoint{3.696000in}{3.696000in}}%
\pgfusepath{clip}%
\pgfsetbuttcap%
\pgfsetroundjoin%
\definecolor{currentfill}{rgb}{0.121569,0.466667,0.705882}%
\pgfsetfillcolor{currentfill}%
\pgfsetfillopacity{0.790415}%
\pgfsetlinewidth{1.003750pt}%
\definecolor{currentstroke}{rgb}{0.121569,0.466667,0.705882}%
\pgfsetstrokecolor{currentstroke}%
\pgfsetstrokeopacity{0.790415}%
\pgfsetdash{}{0pt}%
\pgfpathmoveto{\pgfqpoint{2.327081in}{2.162994in}}%
\pgfpathcurveto{\pgfqpoint{2.335317in}{2.162994in}}{\pgfqpoint{2.343217in}{2.166266in}}{\pgfqpoint{2.349041in}{2.172090in}}%
\pgfpathcurveto{\pgfqpoint{2.354865in}{2.177914in}}{\pgfqpoint{2.358138in}{2.185814in}}{\pgfqpoint{2.358138in}{2.194050in}}%
\pgfpathcurveto{\pgfqpoint{2.358138in}{2.202287in}}{\pgfqpoint{2.354865in}{2.210187in}}{\pgfqpoint{2.349041in}{2.216011in}}%
\pgfpathcurveto{\pgfqpoint{2.343217in}{2.221835in}}{\pgfqpoint{2.335317in}{2.225107in}}{\pgfqpoint{2.327081in}{2.225107in}}%
\pgfpathcurveto{\pgfqpoint{2.318845in}{2.225107in}}{\pgfqpoint{2.310945in}{2.221835in}}{\pgfqpoint{2.305121in}{2.216011in}}%
\pgfpathcurveto{\pgfqpoint{2.299297in}{2.210187in}}{\pgfqpoint{2.296025in}{2.202287in}}{\pgfqpoint{2.296025in}{2.194050in}}%
\pgfpathcurveto{\pgfqpoint{2.296025in}{2.185814in}}{\pgfqpoint{2.299297in}{2.177914in}}{\pgfqpoint{2.305121in}{2.172090in}}%
\pgfpathcurveto{\pgfqpoint{2.310945in}{2.166266in}}{\pgfqpoint{2.318845in}{2.162994in}}{\pgfqpoint{2.327081in}{2.162994in}}%
\pgfpathclose%
\pgfusepath{stroke,fill}%
\end{pgfscope}%
\begin{pgfscope}%
\pgfpathrectangle{\pgfqpoint{0.100000in}{0.212622in}}{\pgfqpoint{3.696000in}{3.696000in}}%
\pgfusepath{clip}%
\pgfsetbuttcap%
\pgfsetroundjoin%
\definecolor{currentfill}{rgb}{0.121569,0.466667,0.705882}%
\pgfsetfillcolor{currentfill}%
\pgfsetfillopacity{0.791629}%
\pgfsetlinewidth{1.003750pt}%
\definecolor{currentstroke}{rgb}{0.121569,0.466667,0.705882}%
\pgfsetstrokecolor{currentstroke}%
\pgfsetstrokeopacity{0.791629}%
\pgfsetdash{}{0pt}%
\pgfpathmoveto{\pgfqpoint{2.342903in}{2.168922in}}%
\pgfpathcurveto{\pgfqpoint{2.351139in}{2.168922in}}{\pgfqpoint{2.359039in}{2.172195in}}{\pgfqpoint{2.364863in}{2.178019in}}%
\pgfpathcurveto{\pgfqpoint{2.370687in}{2.183843in}}{\pgfqpoint{2.373959in}{2.191743in}}{\pgfqpoint{2.373959in}{2.199979in}}%
\pgfpathcurveto{\pgfqpoint{2.373959in}{2.208215in}}{\pgfqpoint{2.370687in}{2.216115in}}{\pgfqpoint{2.364863in}{2.221939in}}%
\pgfpathcurveto{\pgfqpoint{2.359039in}{2.227763in}}{\pgfqpoint{2.351139in}{2.231035in}}{\pgfqpoint{2.342903in}{2.231035in}}%
\pgfpathcurveto{\pgfqpoint{2.334667in}{2.231035in}}{\pgfqpoint{2.326767in}{2.227763in}}{\pgfqpoint{2.320943in}{2.221939in}}%
\pgfpathcurveto{\pgfqpoint{2.315119in}{2.216115in}}{\pgfqpoint{2.311846in}{2.208215in}}{\pgfqpoint{2.311846in}{2.199979in}}%
\pgfpathcurveto{\pgfqpoint{2.311846in}{2.191743in}}{\pgfqpoint{2.315119in}{2.183843in}}{\pgfqpoint{2.320943in}{2.178019in}}%
\pgfpathcurveto{\pgfqpoint{2.326767in}{2.172195in}}{\pgfqpoint{2.334667in}{2.168922in}}{\pgfqpoint{2.342903in}{2.168922in}}%
\pgfpathclose%
\pgfusepath{stroke,fill}%
\end{pgfscope}%
\begin{pgfscope}%
\pgfpathrectangle{\pgfqpoint{0.100000in}{0.212622in}}{\pgfqpoint{3.696000in}{3.696000in}}%
\pgfusepath{clip}%
\pgfsetbuttcap%
\pgfsetroundjoin%
\definecolor{currentfill}{rgb}{0.121569,0.466667,0.705882}%
\pgfsetfillcolor{currentfill}%
\pgfsetfillopacity{0.791964}%
\pgfsetlinewidth{1.003750pt}%
\definecolor{currentstroke}{rgb}{0.121569,0.466667,0.705882}%
\pgfsetstrokecolor{currentstroke}%
\pgfsetstrokeopacity{0.791964}%
\pgfsetdash{}{0pt}%
\pgfpathmoveto{\pgfqpoint{2.965298in}{2.533854in}}%
\pgfpathcurveto{\pgfqpoint{2.973534in}{2.533854in}}{\pgfqpoint{2.981434in}{2.537126in}}{\pgfqpoint{2.987258in}{2.542950in}}%
\pgfpathcurveto{\pgfqpoint{2.993082in}{2.548774in}}{\pgfqpoint{2.996354in}{2.556674in}}{\pgfqpoint{2.996354in}{2.564911in}}%
\pgfpathcurveto{\pgfqpoint{2.996354in}{2.573147in}}{\pgfqpoint{2.993082in}{2.581047in}}{\pgfqpoint{2.987258in}{2.586871in}}%
\pgfpathcurveto{\pgfqpoint{2.981434in}{2.592695in}}{\pgfqpoint{2.973534in}{2.595967in}}{\pgfqpoint{2.965298in}{2.595967in}}%
\pgfpathcurveto{\pgfqpoint{2.957062in}{2.595967in}}{\pgfqpoint{2.949162in}{2.592695in}}{\pgfqpoint{2.943338in}{2.586871in}}%
\pgfpathcurveto{\pgfqpoint{2.937514in}{2.581047in}}{\pgfqpoint{2.934241in}{2.573147in}}{\pgfqpoint{2.934241in}{2.564911in}}%
\pgfpathcurveto{\pgfqpoint{2.934241in}{2.556674in}}{\pgfqpoint{2.937514in}{2.548774in}}{\pgfqpoint{2.943338in}{2.542950in}}%
\pgfpathcurveto{\pgfqpoint{2.949162in}{2.537126in}}{\pgfqpoint{2.957062in}{2.533854in}}{\pgfqpoint{2.965298in}{2.533854in}}%
\pgfpathclose%
\pgfusepath{stroke,fill}%
\end{pgfscope}%
\begin{pgfscope}%
\pgfpathrectangle{\pgfqpoint{0.100000in}{0.212622in}}{\pgfqpoint{3.696000in}{3.696000in}}%
\pgfusepath{clip}%
\pgfsetbuttcap%
\pgfsetroundjoin%
\definecolor{currentfill}{rgb}{0.121569,0.466667,0.705882}%
\pgfsetfillcolor{currentfill}%
\pgfsetfillopacity{0.793143}%
\pgfsetlinewidth{1.003750pt}%
\definecolor{currentstroke}{rgb}{0.121569,0.466667,0.705882}%
\pgfsetstrokecolor{currentstroke}%
\pgfsetstrokeopacity{0.793143}%
\pgfsetdash{}{0pt}%
\pgfpathmoveto{\pgfqpoint{2.333715in}{2.159384in}}%
\pgfpathcurveto{\pgfqpoint{2.341951in}{2.159384in}}{\pgfqpoint{2.349851in}{2.162657in}}{\pgfqpoint{2.355675in}{2.168481in}}%
\pgfpathcurveto{\pgfqpoint{2.361499in}{2.174305in}}{\pgfqpoint{2.364772in}{2.182205in}}{\pgfqpoint{2.364772in}{2.190441in}}%
\pgfpathcurveto{\pgfqpoint{2.364772in}{2.198677in}}{\pgfqpoint{2.361499in}{2.206577in}}{\pgfqpoint{2.355675in}{2.212401in}}%
\pgfpathcurveto{\pgfqpoint{2.349851in}{2.218225in}}{\pgfqpoint{2.341951in}{2.221497in}}{\pgfqpoint{2.333715in}{2.221497in}}%
\pgfpathcurveto{\pgfqpoint{2.325479in}{2.221497in}}{\pgfqpoint{2.317579in}{2.218225in}}{\pgfqpoint{2.311755in}{2.212401in}}%
\pgfpathcurveto{\pgfqpoint{2.305931in}{2.206577in}}{\pgfqpoint{2.302659in}{2.198677in}}{\pgfqpoint{2.302659in}{2.190441in}}%
\pgfpathcurveto{\pgfqpoint{2.302659in}{2.182205in}}{\pgfqpoint{2.305931in}{2.174305in}}{\pgfqpoint{2.311755in}{2.168481in}}%
\pgfpathcurveto{\pgfqpoint{2.317579in}{2.162657in}}{\pgfqpoint{2.325479in}{2.159384in}}{\pgfqpoint{2.333715in}{2.159384in}}%
\pgfpathclose%
\pgfusepath{stroke,fill}%
\end{pgfscope}%
\begin{pgfscope}%
\pgfpathrectangle{\pgfqpoint{0.100000in}{0.212622in}}{\pgfqpoint{3.696000in}{3.696000in}}%
\pgfusepath{clip}%
\pgfsetbuttcap%
\pgfsetroundjoin%
\definecolor{currentfill}{rgb}{0.121569,0.466667,0.705882}%
\pgfsetfillcolor{currentfill}%
\pgfsetfillopacity{0.794308}%
\pgfsetlinewidth{1.003750pt}%
\definecolor{currentstroke}{rgb}{0.121569,0.466667,0.705882}%
\pgfsetstrokecolor{currentstroke}%
\pgfsetstrokeopacity{0.794308}%
\pgfsetdash{}{0pt}%
\pgfpathmoveto{\pgfqpoint{2.093371in}{2.008216in}}%
\pgfpathcurveto{\pgfqpoint{2.101607in}{2.008216in}}{\pgfqpoint{2.109507in}{2.011488in}}{\pgfqpoint{2.115331in}{2.017312in}}%
\pgfpathcurveto{\pgfqpoint{2.121155in}{2.023136in}}{\pgfqpoint{2.124427in}{2.031036in}}{\pgfqpoint{2.124427in}{2.039272in}}%
\pgfpathcurveto{\pgfqpoint{2.124427in}{2.047509in}}{\pgfqpoint{2.121155in}{2.055409in}}{\pgfqpoint{2.115331in}{2.061233in}}%
\pgfpathcurveto{\pgfqpoint{2.109507in}{2.067057in}}{\pgfqpoint{2.101607in}{2.070329in}}{\pgfqpoint{2.093371in}{2.070329in}}%
\pgfpathcurveto{\pgfqpoint{2.085135in}{2.070329in}}{\pgfqpoint{2.077234in}{2.067057in}}{\pgfqpoint{2.071411in}{2.061233in}}%
\pgfpathcurveto{\pgfqpoint{2.065587in}{2.055409in}}{\pgfqpoint{2.062314in}{2.047509in}}{\pgfqpoint{2.062314in}{2.039272in}}%
\pgfpathcurveto{\pgfqpoint{2.062314in}{2.031036in}}{\pgfqpoint{2.065587in}{2.023136in}}{\pgfqpoint{2.071411in}{2.017312in}}%
\pgfpathcurveto{\pgfqpoint{2.077234in}{2.011488in}}{\pgfqpoint{2.085135in}{2.008216in}}{\pgfqpoint{2.093371in}{2.008216in}}%
\pgfpathclose%
\pgfusepath{stroke,fill}%
\end{pgfscope}%
\begin{pgfscope}%
\pgfpathrectangle{\pgfqpoint{0.100000in}{0.212622in}}{\pgfqpoint{3.696000in}{3.696000in}}%
\pgfusepath{clip}%
\pgfsetbuttcap%
\pgfsetroundjoin%
\definecolor{currentfill}{rgb}{0.121569,0.466667,0.705882}%
\pgfsetfillcolor{currentfill}%
\pgfsetfillopacity{0.795200}%
\pgfsetlinewidth{1.003750pt}%
\definecolor{currentstroke}{rgb}{0.121569,0.466667,0.705882}%
\pgfsetstrokecolor{currentstroke}%
\pgfsetstrokeopacity{0.795200}%
\pgfsetdash{}{0pt}%
\pgfpathmoveto{\pgfqpoint{2.278554in}{2.133184in}}%
\pgfpathcurveto{\pgfqpoint{2.286791in}{2.133184in}}{\pgfqpoint{2.294691in}{2.136456in}}{\pgfqpoint{2.300515in}{2.142280in}}%
\pgfpathcurveto{\pgfqpoint{2.306339in}{2.148104in}}{\pgfqpoint{2.309611in}{2.156004in}}{\pgfqpoint{2.309611in}{2.164240in}}%
\pgfpathcurveto{\pgfqpoint{2.309611in}{2.172476in}}{\pgfqpoint{2.306339in}{2.180376in}}{\pgfqpoint{2.300515in}{2.186200in}}%
\pgfpathcurveto{\pgfqpoint{2.294691in}{2.192024in}}{\pgfqpoint{2.286791in}{2.195297in}}{\pgfqpoint{2.278554in}{2.195297in}}%
\pgfpathcurveto{\pgfqpoint{2.270318in}{2.195297in}}{\pgfqpoint{2.262418in}{2.192024in}}{\pgfqpoint{2.256594in}{2.186200in}}%
\pgfpathcurveto{\pgfqpoint{2.250770in}{2.180376in}}{\pgfqpoint{2.247498in}{2.172476in}}{\pgfqpoint{2.247498in}{2.164240in}}%
\pgfpathcurveto{\pgfqpoint{2.247498in}{2.156004in}}{\pgfqpoint{2.250770in}{2.148104in}}{\pgfqpoint{2.256594in}{2.142280in}}%
\pgfpathcurveto{\pgfqpoint{2.262418in}{2.136456in}}{\pgfqpoint{2.270318in}{2.133184in}}{\pgfqpoint{2.278554in}{2.133184in}}%
\pgfpathclose%
\pgfusepath{stroke,fill}%
\end{pgfscope}%
\begin{pgfscope}%
\pgfpathrectangle{\pgfqpoint{0.100000in}{0.212622in}}{\pgfqpoint{3.696000in}{3.696000in}}%
\pgfusepath{clip}%
\pgfsetbuttcap%
\pgfsetroundjoin%
\definecolor{currentfill}{rgb}{0.121569,0.466667,0.705882}%
\pgfsetfillcolor{currentfill}%
\pgfsetfillopacity{0.795494}%
\pgfsetlinewidth{1.003750pt}%
\definecolor{currentstroke}{rgb}{0.121569,0.466667,0.705882}%
\pgfsetstrokecolor{currentstroke}%
\pgfsetstrokeopacity{0.795494}%
\pgfsetdash{}{0pt}%
\pgfpathmoveto{\pgfqpoint{2.303129in}{2.146407in}}%
\pgfpathcurveto{\pgfqpoint{2.311365in}{2.146407in}}{\pgfqpoint{2.319265in}{2.149679in}}{\pgfqpoint{2.325089in}{2.155503in}}%
\pgfpathcurveto{\pgfqpoint{2.330913in}{2.161327in}}{\pgfqpoint{2.334185in}{2.169227in}}{\pgfqpoint{2.334185in}{2.177463in}}%
\pgfpathcurveto{\pgfqpoint{2.334185in}{2.185700in}}{\pgfqpoint{2.330913in}{2.193600in}}{\pgfqpoint{2.325089in}{2.199424in}}%
\pgfpathcurveto{\pgfqpoint{2.319265in}{2.205247in}}{\pgfqpoint{2.311365in}{2.208520in}}{\pgfqpoint{2.303129in}{2.208520in}}%
\pgfpathcurveto{\pgfqpoint{2.294893in}{2.208520in}}{\pgfqpoint{2.286993in}{2.205247in}}{\pgfqpoint{2.281169in}{2.199424in}}%
\pgfpathcurveto{\pgfqpoint{2.275345in}{2.193600in}}{\pgfqpoint{2.272072in}{2.185700in}}{\pgfqpoint{2.272072in}{2.177463in}}%
\pgfpathcurveto{\pgfqpoint{2.272072in}{2.169227in}}{\pgfqpoint{2.275345in}{2.161327in}}{\pgfqpoint{2.281169in}{2.155503in}}%
\pgfpathcurveto{\pgfqpoint{2.286993in}{2.149679in}}{\pgfqpoint{2.294893in}{2.146407in}}{\pgfqpoint{2.303129in}{2.146407in}}%
\pgfpathclose%
\pgfusepath{stroke,fill}%
\end{pgfscope}%
\begin{pgfscope}%
\pgfpathrectangle{\pgfqpoint{0.100000in}{0.212622in}}{\pgfqpoint{3.696000in}{3.696000in}}%
\pgfusepath{clip}%
\pgfsetbuttcap%
\pgfsetroundjoin%
\definecolor{currentfill}{rgb}{0.121569,0.466667,0.705882}%
\pgfsetfillcolor{currentfill}%
\pgfsetfillopacity{0.795536}%
\pgfsetlinewidth{1.003750pt}%
\definecolor{currentstroke}{rgb}{0.121569,0.466667,0.705882}%
\pgfsetstrokecolor{currentstroke}%
\pgfsetstrokeopacity{0.795536}%
\pgfsetdash{}{0pt}%
\pgfpathmoveto{\pgfqpoint{2.141895in}{2.036517in}}%
\pgfpathcurveto{\pgfqpoint{2.150131in}{2.036517in}}{\pgfqpoint{2.158031in}{2.039789in}}{\pgfqpoint{2.163855in}{2.045613in}}%
\pgfpathcurveto{\pgfqpoint{2.169679in}{2.051437in}}{\pgfqpoint{2.172951in}{2.059337in}}{\pgfqpoint{2.172951in}{2.067574in}}%
\pgfpathcurveto{\pgfqpoint{2.172951in}{2.075810in}}{\pgfqpoint{2.169679in}{2.083710in}}{\pgfqpoint{2.163855in}{2.089534in}}%
\pgfpathcurveto{\pgfqpoint{2.158031in}{2.095358in}}{\pgfqpoint{2.150131in}{2.098630in}}{\pgfqpoint{2.141895in}{2.098630in}}%
\pgfpathcurveto{\pgfqpoint{2.133658in}{2.098630in}}{\pgfqpoint{2.125758in}{2.095358in}}{\pgfqpoint{2.119934in}{2.089534in}}%
\pgfpathcurveto{\pgfqpoint{2.114110in}{2.083710in}}{\pgfqpoint{2.110838in}{2.075810in}}{\pgfqpoint{2.110838in}{2.067574in}}%
\pgfpathcurveto{\pgfqpoint{2.110838in}{2.059337in}}{\pgfqpoint{2.114110in}{2.051437in}}{\pgfqpoint{2.119934in}{2.045613in}}%
\pgfpathcurveto{\pgfqpoint{2.125758in}{2.039789in}}{\pgfqpoint{2.133658in}{2.036517in}}{\pgfqpoint{2.141895in}{2.036517in}}%
\pgfpathclose%
\pgfusepath{stroke,fill}%
\end{pgfscope}%
\begin{pgfscope}%
\pgfpathrectangle{\pgfqpoint{0.100000in}{0.212622in}}{\pgfqpoint{3.696000in}{3.696000in}}%
\pgfusepath{clip}%
\pgfsetbuttcap%
\pgfsetroundjoin%
\definecolor{currentfill}{rgb}{0.121569,0.466667,0.705882}%
\pgfsetfillcolor{currentfill}%
\pgfsetfillopacity{0.795959}%
\pgfsetlinewidth{1.003750pt}%
\definecolor{currentstroke}{rgb}{0.121569,0.466667,0.705882}%
\pgfsetstrokecolor{currentstroke}%
\pgfsetstrokeopacity{0.795959}%
\pgfsetdash{}{0pt}%
\pgfpathmoveto{\pgfqpoint{2.340495in}{2.162616in}}%
\pgfpathcurveto{\pgfqpoint{2.348731in}{2.162616in}}{\pgfqpoint{2.356631in}{2.165889in}}{\pgfqpoint{2.362455in}{2.171713in}}%
\pgfpathcurveto{\pgfqpoint{2.368279in}{2.177537in}}{\pgfqpoint{2.371551in}{2.185437in}}{\pgfqpoint{2.371551in}{2.193673in}}%
\pgfpathcurveto{\pgfqpoint{2.371551in}{2.201909in}}{\pgfqpoint{2.368279in}{2.209809in}}{\pgfqpoint{2.362455in}{2.215633in}}%
\pgfpathcurveto{\pgfqpoint{2.356631in}{2.221457in}}{\pgfqpoint{2.348731in}{2.224729in}}{\pgfqpoint{2.340495in}{2.224729in}}%
\pgfpathcurveto{\pgfqpoint{2.332258in}{2.224729in}}{\pgfqpoint{2.324358in}{2.221457in}}{\pgfqpoint{2.318534in}{2.215633in}}%
\pgfpathcurveto{\pgfqpoint{2.312710in}{2.209809in}}{\pgfqpoint{2.309438in}{2.201909in}}{\pgfqpoint{2.309438in}{2.193673in}}%
\pgfpathcurveto{\pgfqpoint{2.309438in}{2.185437in}}{\pgfqpoint{2.312710in}{2.177537in}}{\pgfqpoint{2.318534in}{2.171713in}}%
\pgfpathcurveto{\pgfqpoint{2.324358in}{2.165889in}}{\pgfqpoint{2.332258in}{2.162616in}}{\pgfqpoint{2.340495in}{2.162616in}}%
\pgfpathclose%
\pgfusepath{stroke,fill}%
\end{pgfscope}%
\begin{pgfscope}%
\pgfpathrectangle{\pgfqpoint{0.100000in}{0.212622in}}{\pgfqpoint{3.696000in}{3.696000in}}%
\pgfusepath{clip}%
\pgfsetbuttcap%
\pgfsetroundjoin%
\definecolor{currentfill}{rgb}{0.121569,0.466667,0.705882}%
\pgfsetfillcolor{currentfill}%
\pgfsetfillopacity{0.797055}%
\pgfsetlinewidth{1.003750pt}%
\definecolor{currentstroke}{rgb}{0.121569,0.466667,0.705882}%
\pgfsetstrokecolor{currentstroke}%
\pgfsetstrokeopacity{0.797055}%
\pgfsetdash{}{0pt}%
\pgfpathmoveto{\pgfqpoint{2.302099in}{2.139395in}}%
\pgfpathcurveto{\pgfqpoint{2.310335in}{2.139395in}}{\pgfqpoint{2.318235in}{2.142667in}}{\pgfqpoint{2.324059in}{2.148491in}}%
\pgfpathcurveto{\pgfqpoint{2.329883in}{2.154315in}}{\pgfqpoint{2.333155in}{2.162215in}}{\pgfqpoint{2.333155in}{2.170451in}}%
\pgfpathcurveto{\pgfqpoint{2.333155in}{2.178688in}}{\pgfqpoint{2.329883in}{2.186588in}}{\pgfqpoint{2.324059in}{2.192412in}}%
\pgfpathcurveto{\pgfqpoint{2.318235in}{2.198236in}}{\pgfqpoint{2.310335in}{2.201508in}}{\pgfqpoint{2.302099in}{2.201508in}}%
\pgfpathcurveto{\pgfqpoint{2.293863in}{2.201508in}}{\pgfqpoint{2.285962in}{2.198236in}}{\pgfqpoint{2.280139in}{2.192412in}}%
\pgfpathcurveto{\pgfqpoint{2.274315in}{2.186588in}}{\pgfqpoint{2.271042in}{2.178688in}}{\pgfqpoint{2.271042in}{2.170451in}}%
\pgfpathcurveto{\pgfqpoint{2.271042in}{2.162215in}}{\pgfqpoint{2.274315in}{2.154315in}}{\pgfqpoint{2.280139in}{2.148491in}}%
\pgfpathcurveto{\pgfqpoint{2.285962in}{2.142667in}}{\pgfqpoint{2.293863in}{2.139395in}}{\pgfqpoint{2.302099in}{2.139395in}}%
\pgfpathclose%
\pgfusepath{stroke,fill}%
\end{pgfscope}%
\begin{pgfscope}%
\pgfpathrectangle{\pgfqpoint{0.100000in}{0.212622in}}{\pgfqpoint{3.696000in}{3.696000in}}%
\pgfusepath{clip}%
\pgfsetbuttcap%
\pgfsetroundjoin%
\definecolor{currentfill}{rgb}{0.121569,0.466667,0.705882}%
\pgfsetfillcolor{currentfill}%
\pgfsetfillopacity{0.797083}%
\pgfsetlinewidth{1.003750pt}%
\definecolor{currentstroke}{rgb}{0.121569,0.466667,0.705882}%
\pgfsetstrokecolor{currentstroke}%
\pgfsetstrokeopacity{0.797083}%
\pgfsetdash{}{0pt}%
\pgfpathmoveto{\pgfqpoint{2.193362in}{2.066208in}}%
\pgfpathcurveto{\pgfqpoint{2.201598in}{2.066208in}}{\pgfqpoint{2.209498in}{2.069480in}}{\pgfqpoint{2.215322in}{2.075304in}}%
\pgfpathcurveto{\pgfqpoint{2.221146in}{2.081128in}}{\pgfqpoint{2.224418in}{2.089028in}}{\pgfqpoint{2.224418in}{2.097265in}}%
\pgfpathcurveto{\pgfqpoint{2.224418in}{2.105501in}}{\pgfqpoint{2.221146in}{2.113401in}}{\pgfqpoint{2.215322in}{2.119225in}}%
\pgfpathcurveto{\pgfqpoint{2.209498in}{2.125049in}}{\pgfqpoint{2.201598in}{2.128321in}}{\pgfqpoint{2.193362in}{2.128321in}}%
\pgfpathcurveto{\pgfqpoint{2.185125in}{2.128321in}}{\pgfqpoint{2.177225in}{2.125049in}}{\pgfqpoint{2.171401in}{2.119225in}}%
\pgfpathcurveto{\pgfqpoint{2.165578in}{2.113401in}}{\pgfqpoint{2.162305in}{2.105501in}}{\pgfqpoint{2.162305in}{2.097265in}}%
\pgfpathcurveto{\pgfqpoint{2.162305in}{2.089028in}}{\pgfqpoint{2.165578in}{2.081128in}}{\pgfqpoint{2.171401in}{2.075304in}}%
\pgfpathcurveto{\pgfqpoint{2.177225in}{2.069480in}}{\pgfqpoint{2.185125in}{2.066208in}}{\pgfqpoint{2.193362in}{2.066208in}}%
\pgfpathclose%
\pgfusepath{stroke,fill}%
\end{pgfscope}%
\begin{pgfscope}%
\pgfpathrectangle{\pgfqpoint{0.100000in}{0.212622in}}{\pgfqpoint{3.696000in}{3.696000in}}%
\pgfusepath{clip}%
\pgfsetbuttcap%
\pgfsetroundjoin%
\definecolor{currentfill}{rgb}{0.121569,0.466667,0.705882}%
\pgfsetfillcolor{currentfill}%
\pgfsetfillopacity{0.797130}%
\pgfsetlinewidth{1.003750pt}%
\definecolor{currentstroke}{rgb}{0.121569,0.466667,0.705882}%
\pgfsetstrokecolor{currentstroke}%
\pgfsetstrokeopacity{0.797130}%
\pgfsetdash{}{0pt}%
\pgfpathmoveto{\pgfqpoint{2.330820in}{2.152488in}}%
\pgfpathcurveto{\pgfqpoint{2.339056in}{2.152488in}}{\pgfqpoint{2.346957in}{2.155761in}}{\pgfqpoint{2.352780in}{2.161585in}}%
\pgfpathcurveto{\pgfqpoint{2.358604in}{2.167409in}}{\pgfqpoint{2.361877in}{2.175309in}}{\pgfqpoint{2.361877in}{2.183545in}}%
\pgfpathcurveto{\pgfqpoint{2.361877in}{2.191781in}}{\pgfqpoint{2.358604in}{2.199681in}}{\pgfqpoint{2.352780in}{2.205505in}}%
\pgfpathcurveto{\pgfqpoint{2.346957in}{2.211329in}}{\pgfqpoint{2.339056in}{2.214601in}}{\pgfqpoint{2.330820in}{2.214601in}}%
\pgfpathcurveto{\pgfqpoint{2.322584in}{2.214601in}}{\pgfqpoint{2.314684in}{2.211329in}}{\pgfqpoint{2.308860in}{2.205505in}}%
\pgfpathcurveto{\pgfqpoint{2.303036in}{2.199681in}}{\pgfqpoint{2.299764in}{2.191781in}}{\pgfqpoint{2.299764in}{2.183545in}}%
\pgfpathcurveto{\pgfqpoint{2.299764in}{2.175309in}}{\pgfqpoint{2.303036in}{2.167409in}}{\pgfqpoint{2.308860in}{2.161585in}}%
\pgfpathcurveto{\pgfqpoint{2.314684in}{2.155761in}}{\pgfqpoint{2.322584in}{2.152488in}}{\pgfqpoint{2.330820in}{2.152488in}}%
\pgfpathclose%
\pgfusepath{stroke,fill}%
\end{pgfscope}%
\begin{pgfscope}%
\pgfpathrectangle{\pgfqpoint{0.100000in}{0.212622in}}{\pgfqpoint{3.696000in}{3.696000in}}%
\pgfusepath{clip}%
\pgfsetbuttcap%
\pgfsetroundjoin%
\definecolor{currentfill}{rgb}{0.121569,0.466667,0.705882}%
\pgfsetfillcolor{currentfill}%
\pgfsetfillopacity{0.798487}%
\pgfsetlinewidth{1.003750pt}%
\definecolor{currentstroke}{rgb}{0.121569,0.466667,0.705882}%
\pgfsetstrokecolor{currentstroke}%
\pgfsetstrokeopacity{0.798487}%
\pgfsetdash{}{0pt}%
\pgfpathmoveto{\pgfqpoint{2.053224in}{1.969629in}}%
\pgfpathcurveto{\pgfqpoint{2.061460in}{1.969629in}}{\pgfqpoint{2.069360in}{1.972902in}}{\pgfqpoint{2.075184in}{1.978726in}}%
\pgfpathcurveto{\pgfqpoint{2.081008in}{1.984550in}}{\pgfqpoint{2.084281in}{1.992450in}}{\pgfqpoint{2.084281in}{2.000686in}}%
\pgfpathcurveto{\pgfqpoint{2.084281in}{2.008922in}}{\pgfqpoint{2.081008in}{2.016822in}}{\pgfqpoint{2.075184in}{2.022646in}}%
\pgfpathcurveto{\pgfqpoint{2.069360in}{2.028470in}}{\pgfqpoint{2.061460in}{2.031742in}}{\pgfqpoint{2.053224in}{2.031742in}}%
\pgfpathcurveto{\pgfqpoint{2.044988in}{2.031742in}}{\pgfqpoint{2.037088in}{2.028470in}}{\pgfqpoint{2.031264in}{2.022646in}}%
\pgfpathcurveto{\pgfqpoint{2.025440in}{2.016822in}}{\pgfqpoint{2.022168in}{2.008922in}}{\pgfqpoint{2.022168in}{2.000686in}}%
\pgfpathcurveto{\pgfqpoint{2.022168in}{1.992450in}}{\pgfqpoint{2.025440in}{1.984550in}}{\pgfqpoint{2.031264in}{1.978726in}}%
\pgfpathcurveto{\pgfqpoint{2.037088in}{1.972902in}}{\pgfqpoint{2.044988in}{1.969629in}}{\pgfqpoint{2.053224in}{1.969629in}}%
\pgfpathclose%
\pgfusepath{stroke,fill}%
\end{pgfscope}%
\begin{pgfscope}%
\pgfpathrectangle{\pgfqpoint{0.100000in}{0.212622in}}{\pgfqpoint{3.696000in}{3.696000in}}%
\pgfusepath{clip}%
\pgfsetbuttcap%
\pgfsetroundjoin%
\definecolor{currentfill}{rgb}{0.121569,0.466667,0.705882}%
\pgfsetfillcolor{currentfill}%
\pgfsetfillopacity{0.804461}%
\pgfsetlinewidth{1.003750pt}%
\definecolor{currentstroke}{rgb}{0.121569,0.466667,0.705882}%
\pgfsetstrokecolor{currentstroke}%
\pgfsetstrokeopacity{0.804461}%
\pgfsetdash{}{0pt}%
\pgfpathmoveto{\pgfqpoint{2.929759in}{2.502772in}}%
\pgfpathcurveto{\pgfqpoint{2.937995in}{2.502772in}}{\pgfqpoint{2.945895in}{2.506044in}}{\pgfqpoint{2.951719in}{2.511868in}}%
\pgfpathcurveto{\pgfqpoint{2.957543in}{2.517692in}}{\pgfqpoint{2.960815in}{2.525592in}}{\pgfqpoint{2.960815in}{2.533829in}}%
\pgfpathcurveto{\pgfqpoint{2.960815in}{2.542065in}}{\pgfqpoint{2.957543in}{2.549965in}}{\pgfqpoint{2.951719in}{2.555789in}}%
\pgfpathcurveto{\pgfqpoint{2.945895in}{2.561613in}}{\pgfqpoint{2.937995in}{2.564885in}}{\pgfqpoint{2.929759in}{2.564885in}}%
\pgfpathcurveto{\pgfqpoint{2.921523in}{2.564885in}}{\pgfqpoint{2.913623in}{2.561613in}}{\pgfqpoint{2.907799in}{2.555789in}}%
\pgfpathcurveto{\pgfqpoint{2.901975in}{2.549965in}}{\pgfqpoint{2.898702in}{2.542065in}}{\pgfqpoint{2.898702in}{2.533829in}}%
\pgfpathcurveto{\pgfqpoint{2.898702in}{2.525592in}}{\pgfqpoint{2.901975in}{2.517692in}}{\pgfqpoint{2.907799in}{2.511868in}}%
\pgfpathcurveto{\pgfqpoint{2.913623in}{2.506044in}}{\pgfqpoint{2.921523in}{2.502772in}}{\pgfqpoint{2.929759in}{2.502772in}}%
\pgfpathclose%
\pgfusepath{stroke,fill}%
\end{pgfscope}%
\begin{pgfscope}%
\pgfpathrectangle{\pgfqpoint{0.100000in}{0.212622in}}{\pgfqpoint{3.696000in}{3.696000in}}%
\pgfusepath{clip}%
\pgfsetbuttcap%
\pgfsetroundjoin%
\definecolor{currentfill}{rgb}{0.121569,0.466667,0.705882}%
\pgfsetfillcolor{currentfill}%
\pgfsetfillopacity{0.804508}%
\pgfsetlinewidth{1.003750pt}%
\definecolor{currentstroke}{rgb}{0.121569,0.466667,0.705882}%
\pgfsetstrokecolor{currentstroke}%
\pgfsetstrokeopacity{0.804508}%
\pgfsetdash{}{0pt}%
\pgfpathmoveto{\pgfqpoint{2.091928in}{2.004062in}}%
\pgfpathcurveto{\pgfqpoint{2.100165in}{2.004062in}}{\pgfqpoint{2.108065in}{2.007334in}}{\pgfqpoint{2.113889in}{2.013158in}}%
\pgfpathcurveto{\pgfqpoint{2.119713in}{2.018982in}}{\pgfqpoint{2.122985in}{2.026882in}}{\pgfqpoint{2.122985in}{2.035119in}}%
\pgfpathcurveto{\pgfqpoint{2.122985in}{2.043355in}}{\pgfqpoint{2.119713in}{2.051255in}}{\pgfqpoint{2.113889in}{2.057079in}}%
\pgfpathcurveto{\pgfqpoint{2.108065in}{2.062903in}}{\pgfqpoint{2.100165in}{2.066175in}}{\pgfqpoint{2.091928in}{2.066175in}}%
\pgfpathcurveto{\pgfqpoint{2.083692in}{2.066175in}}{\pgfqpoint{2.075792in}{2.062903in}}{\pgfqpoint{2.069968in}{2.057079in}}%
\pgfpathcurveto{\pgfqpoint{2.064144in}{2.051255in}}{\pgfqpoint{2.060872in}{2.043355in}}{\pgfqpoint{2.060872in}{2.035119in}}%
\pgfpathcurveto{\pgfqpoint{2.060872in}{2.026882in}}{\pgfqpoint{2.064144in}{2.018982in}}{\pgfqpoint{2.069968in}{2.013158in}}%
\pgfpathcurveto{\pgfqpoint{2.075792in}{2.007334in}}{\pgfqpoint{2.083692in}{2.004062in}}{\pgfqpoint{2.091928in}{2.004062in}}%
\pgfpathclose%
\pgfusepath{stroke,fill}%
\end{pgfscope}%
\begin{pgfscope}%
\pgfpathrectangle{\pgfqpoint{0.100000in}{0.212622in}}{\pgfqpoint{3.696000in}{3.696000in}}%
\pgfusepath{clip}%
\pgfsetbuttcap%
\pgfsetroundjoin%
\definecolor{currentfill}{rgb}{0.121569,0.466667,0.705882}%
\pgfsetfillcolor{currentfill}%
\pgfsetfillopacity{0.805236}%
\pgfsetlinewidth{1.003750pt}%
\definecolor{currentstroke}{rgb}{0.121569,0.466667,0.705882}%
\pgfsetstrokecolor{currentstroke}%
\pgfsetstrokeopacity{0.805236}%
\pgfsetdash{}{0pt}%
\pgfpathmoveto{\pgfqpoint{2.320224in}{2.138282in}}%
\pgfpathcurveto{\pgfqpoint{2.328461in}{2.138282in}}{\pgfqpoint{2.336361in}{2.141554in}}{\pgfqpoint{2.342185in}{2.147378in}}%
\pgfpathcurveto{\pgfqpoint{2.348008in}{2.153202in}}{\pgfqpoint{2.351281in}{2.161102in}}{\pgfqpoint{2.351281in}{2.169339in}}%
\pgfpathcurveto{\pgfqpoint{2.351281in}{2.177575in}}{\pgfqpoint{2.348008in}{2.185475in}}{\pgfqpoint{2.342185in}{2.191299in}}%
\pgfpathcurveto{\pgfqpoint{2.336361in}{2.197123in}}{\pgfqpoint{2.328461in}{2.200395in}}{\pgfqpoint{2.320224in}{2.200395in}}%
\pgfpathcurveto{\pgfqpoint{2.311988in}{2.200395in}}{\pgfqpoint{2.304088in}{2.197123in}}{\pgfqpoint{2.298264in}{2.191299in}}%
\pgfpathcurveto{\pgfqpoint{2.292440in}{2.185475in}}{\pgfqpoint{2.289168in}{2.177575in}}{\pgfqpoint{2.289168in}{2.169339in}}%
\pgfpathcurveto{\pgfqpoint{2.289168in}{2.161102in}}{\pgfqpoint{2.292440in}{2.153202in}}{\pgfqpoint{2.298264in}{2.147378in}}%
\pgfpathcurveto{\pgfqpoint{2.304088in}{2.141554in}}{\pgfqpoint{2.311988in}{2.138282in}}{\pgfqpoint{2.320224in}{2.138282in}}%
\pgfpathclose%
\pgfusepath{stroke,fill}%
\end{pgfscope}%
\begin{pgfscope}%
\pgfpathrectangle{\pgfqpoint{0.100000in}{0.212622in}}{\pgfqpoint{3.696000in}{3.696000in}}%
\pgfusepath{clip}%
\pgfsetbuttcap%
\pgfsetroundjoin%
\definecolor{currentfill}{rgb}{0.121569,0.466667,0.705882}%
\pgfsetfillcolor{currentfill}%
\pgfsetfillopacity{0.806926}%
\pgfsetlinewidth{1.003750pt}%
\definecolor{currentstroke}{rgb}{0.121569,0.466667,0.705882}%
\pgfsetstrokecolor{currentstroke}%
\pgfsetstrokeopacity{0.806926}%
\pgfsetdash{}{0pt}%
\pgfpathmoveto{\pgfqpoint{2.327464in}{2.151144in}}%
\pgfpathcurveto{\pgfqpoint{2.335700in}{2.151144in}}{\pgfqpoint{2.343600in}{2.154416in}}{\pgfqpoint{2.349424in}{2.160240in}}%
\pgfpathcurveto{\pgfqpoint{2.355248in}{2.166064in}}{\pgfqpoint{2.358520in}{2.173964in}}{\pgfqpoint{2.358520in}{2.182200in}}%
\pgfpathcurveto{\pgfqpoint{2.358520in}{2.190436in}}{\pgfqpoint{2.355248in}{2.198336in}}{\pgfqpoint{2.349424in}{2.204160in}}%
\pgfpathcurveto{\pgfqpoint{2.343600in}{2.209984in}}{\pgfqpoint{2.335700in}{2.213257in}}{\pgfqpoint{2.327464in}{2.213257in}}%
\pgfpathcurveto{\pgfqpoint{2.319227in}{2.213257in}}{\pgfqpoint{2.311327in}{2.209984in}}{\pgfqpoint{2.305503in}{2.204160in}}%
\pgfpathcurveto{\pgfqpoint{2.299680in}{2.198336in}}{\pgfqpoint{2.296407in}{2.190436in}}{\pgfqpoint{2.296407in}{2.182200in}}%
\pgfpathcurveto{\pgfqpoint{2.296407in}{2.173964in}}{\pgfqpoint{2.299680in}{2.166064in}}{\pgfqpoint{2.305503in}{2.160240in}}%
\pgfpathcurveto{\pgfqpoint{2.311327in}{2.154416in}}{\pgfqpoint{2.319227in}{2.151144in}}{\pgfqpoint{2.327464in}{2.151144in}}%
\pgfpathclose%
\pgfusepath{stroke,fill}%
\end{pgfscope}%
\begin{pgfscope}%
\pgfpathrectangle{\pgfqpoint{0.100000in}{0.212622in}}{\pgfqpoint{3.696000in}{3.696000in}}%
\pgfusepath{clip}%
\pgfsetbuttcap%
\pgfsetroundjoin%
\definecolor{currentfill}{rgb}{0.121569,0.466667,0.705882}%
\pgfsetfillcolor{currentfill}%
\pgfsetfillopacity{0.809293}%
\pgfsetlinewidth{1.003750pt}%
\definecolor{currentstroke}{rgb}{0.121569,0.466667,0.705882}%
\pgfsetstrokecolor{currentstroke}%
\pgfsetstrokeopacity{0.809293}%
\pgfsetdash{}{0pt}%
\pgfpathmoveto{\pgfqpoint{2.252690in}{2.096226in}}%
\pgfpathcurveto{\pgfqpoint{2.260926in}{2.096226in}}{\pgfqpoint{2.268826in}{2.099499in}}{\pgfqpoint{2.274650in}{2.105323in}}%
\pgfpathcurveto{\pgfqpoint{2.280474in}{2.111147in}}{\pgfqpoint{2.283747in}{2.119047in}}{\pgfqpoint{2.283747in}{2.127283in}}%
\pgfpathcurveto{\pgfqpoint{2.283747in}{2.135519in}}{\pgfqpoint{2.280474in}{2.143419in}}{\pgfqpoint{2.274650in}{2.149243in}}%
\pgfpathcurveto{\pgfqpoint{2.268826in}{2.155067in}}{\pgfqpoint{2.260926in}{2.158339in}}{\pgfqpoint{2.252690in}{2.158339in}}%
\pgfpathcurveto{\pgfqpoint{2.244454in}{2.158339in}}{\pgfqpoint{2.236554in}{2.155067in}}{\pgfqpoint{2.230730in}{2.149243in}}%
\pgfpathcurveto{\pgfqpoint{2.224906in}{2.143419in}}{\pgfqpoint{2.221634in}{2.135519in}}{\pgfqpoint{2.221634in}{2.127283in}}%
\pgfpathcurveto{\pgfqpoint{2.221634in}{2.119047in}}{\pgfqpoint{2.224906in}{2.111147in}}{\pgfqpoint{2.230730in}{2.105323in}}%
\pgfpathcurveto{\pgfqpoint{2.236554in}{2.099499in}}{\pgfqpoint{2.244454in}{2.096226in}}{\pgfqpoint{2.252690in}{2.096226in}}%
\pgfpathclose%
\pgfusepath{stroke,fill}%
\end{pgfscope}%
\begin{pgfscope}%
\pgfpathrectangle{\pgfqpoint{0.100000in}{0.212622in}}{\pgfqpoint{3.696000in}{3.696000in}}%
\pgfusepath{clip}%
\pgfsetbuttcap%
\pgfsetroundjoin%
\definecolor{currentfill}{rgb}{0.121569,0.466667,0.705882}%
\pgfsetfillcolor{currentfill}%
\pgfsetfillopacity{0.810734}%
\pgfsetlinewidth{1.003750pt}%
\definecolor{currentstroke}{rgb}{0.121569,0.466667,0.705882}%
\pgfsetstrokecolor{currentstroke}%
\pgfsetstrokeopacity{0.810734}%
\pgfsetdash{}{0pt}%
\pgfpathmoveto{\pgfqpoint{2.212862in}{2.063301in}}%
\pgfpathcurveto{\pgfqpoint{2.221099in}{2.063301in}}{\pgfqpoint{2.228999in}{2.066574in}}{\pgfqpoint{2.234823in}{2.072397in}}%
\pgfpathcurveto{\pgfqpoint{2.240647in}{2.078221in}}{\pgfqpoint{2.243919in}{2.086121in}}{\pgfqpoint{2.243919in}{2.094358in}}%
\pgfpathcurveto{\pgfqpoint{2.243919in}{2.102594in}}{\pgfqpoint{2.240647in}{2.110494in}}{\pgfqpoint{2.234823in}{2.116318in}}%
\pgfpathcurveto{\pgfqpoint{2.228999in}{2.122142in}}{\pgfqpoint{2.221099in}{2.125414in}}{\pgfqpoint{2.212862in}{2.125414in}}%
\pgfpathcurveto{\pgfqpoint{2.204626in}{2.125414in}}{\pgfqpoint{2.196726in}{2.122142in}}{\pgfqpoint{2.190902in}{2.116318in}}%
\pgfpathcurveto{\pgfqpoint{2.185078in}{2.110494in}}{\pgfqpoint{2.181806in}{2.102594in}}{\pgfqpoint{2.181806in}{2.094358in}}%
\pgfpathcurveto{\pgfqpoint{2.181806in}{2.086121in}}{\pgfqpoint{2.185078in}{2.078221in}}{\pgfqpoint{2.190902in}{2.072397in}}%
\pgfpathcurveto{\pgfqpoint{2.196726in}{2.066574in}}{\pgfqpoint{2.204626in}{2.063301in}}{\pgfqpoint{2.212862in}{2.063301in}}%
\pgfpathclose%
\pgfusepath{stroke,fill}%
\end{pgfscope}%
\begin{pgfscope}%
\pgfpathrectangle{\pgfqpoint{0.100000in}{0.212622in}}{\pgfqpoint{3.696000in}{3.696000in}}%
\pgfusepath{clip}%
\pgfsetbuttcap%
\pgfsetroundjoin%
\definecolor{currentfill}{rgb}{0.121569,0.466667,0.705882}%
\pgfsetfillcolor{currentfill}%
\pgfsetfillopacity{0.818059}%
\pgfsetlinewidth{1.003750pt}%
\definecolor{currentstroke}{rgb}{0.121569,0.466667,0.705882}%
\pgfsetstrokecolor{currentstroke}%
\pgfsetstrokeopacity{0.818059}%
\pgfsetdash{}{0pt}%
\pgfpathmoveto{\pgfqpoint{2.316469in}{2.132425in}}%
\pgfpathcurveto{\pgfqpoint{2.324705in}{2.132425in}}{\pgfqpoint{2.332605in}{2.135698in}}{\pgfqpoint{2.338429in}{2.141521in}}%
\pgfpathcurveto{\pgfqpoint{2.344253in}{2.147345in}}{\pgfqpoint{2.347526in}{2.155245in}}{\pgfqpoint{2.347526in}{2.163482in}}%
\pgfpathcurveto{\pgfqpoint{2.347526in}{2.171718in}}{\pgfqpoint{2.344253in}{2.179618in}}{\pgfqpoint{2.338429in}{2.185442in}}%
\pgfpathcurveto{\pgfqpoint{2.332605in}{2.191266in}}{\pgfqpoint{2.324705in}{2.194538in}}{\pgfqpoint{2.316469in}{2.194538in}}%
\pgfpathcurveto{\pgfqpoint{2.308233in}{2.194538in}}{\pgfqpoint{2.300333in}{2.191266in}}{\pgfqpoint{2.294509in}{2.185442in}}%
\pgfpathcurveto{\pgfqpoint{2.288685in}{2.179618in}}{\pgfqpoint{2.285413in}{2.171718in}}{\pgfqpoint{2.285413in}{2.163482in}}%
\pgfpathcurveto{\pgfqpoint{2.285413in}{2.155245in}}{\pgfqpoint{2.288685in}{2.147345in}}{\pgfqpoint{2.294509in}{2.141521in}}%
\pgfpathcurveto{\pgfqpoint{2.300333in}{2.135698in}}{\pgfqpoint{2.308233in}{2.132425in}}{\pgfqpoint{2.316469in}{2.132425in}}%
\pgfpathclose%
\pgfusepath{stroke,fill}%
\end{pgfscope}%
\begin{pgfscope}%
\pgfpathrectangle{\pgfqpoint{0.100000in}{0.212622in}}{\pgfqpoint{3.696000in}{3.696000in}}%
\pgfusepath{clip}%
\pgfsetbuttcap%
\pgfsetroundjoin%
\definecolor{currentfill}{rgb}{0.121569,0.466667,0.705882}%
\pgfsetfillcolor{currentfill}%
\pgfsetfillopacity{0.820644}%
\pgfsetlinewidth{1.003750pt}%
\definecolor{currentstroke}{rgb}{0.121569,0.466667,0.705882}%
\pgfsetstrokecolor{currentstroke}%
\pgfsetstrokeopacity{0.820644}%
\pgfsetdash{}{0pt}%
\pgfpathmoveto{\pgfqpoint{2.295839in}{2.111590in}}%
\pgfpathcurveto{\pgfqpoint{2.304075in}{2.111590in}}{\pgfqpoint{2.311975in}{2.114862in}}{\pgfqpoint{2.317799in}{2.120686in}}%
\pgfpathcurveto{\pgfqpoint{2.323623in}{2.126510in}}{\pgfqpoint{2.326895in}{2.134410in}}{\pgfqpoint{2.326895in}{2.142646in}}%
\pgfpathcurveto{\pgfqpoint{2.326895in}{2.150883in}}{\pgfqpoint{2.323623in}{2.158783in}}{\pgfqpoint{2.317799in}{2.164607in}}%
\pgfpathcurveto{\pgfqpoint{2.311975in}{2.170431in}}{\pgfqpoint{2.304075in}{2.173703in}}{\pgfqpoint{2.295839in}{2.173703in}}%
\pgfpathcurveto{\pgfqpoint{2.287602in}{2.173703in}}{\pgfqpoint{2.279702in}{2.170431in}}{\pgfqpoint{2.273878in}{2.164607in}}%
\pgfpathcurveto{\pgfqpoint{2.268054in}{2.158783in}}{\pgfqpoint{2.264782in}{2.150883in}}{\pgfqpoint{2.264782in}{2.142646in}}%
\pgfpathcurveto{\pgfqpoint{2.264782in}{2.134410in}}{\pgfqpoint{2.268054in}{2.126510in}}{\pgfqpoint{2.273878in}{2.120686in}}%
\pgfpathcurveto{\pgfqpoint{2.279702in}{2.114862in}}{\pgfqpoint{2.287602in}{2.111590in}}{\pgfqpoint{2.295839in}{2.111590in}}%
\pgfpathclose%
\pgfusepath{stroke,fill}%
\end{pgfscope}%
\begin{pgfscope}%
\pgfpathrectangle{\pgfqpoint{0.100000in}{0.212622in}}{\pgfqpoint{3.696000in}{3.696000in}}%
\pgfusepath{clip}%
\pgfsetbuttcap%
\pgfsetroundjoin%
\definecolor{currentfill}{rgb}{0.121569,0.466667,0.705882}%
\pgfsetfillcolor{currentfill}%
\pgfsetfillopacity{0.824782}%
\pgfsetlinewidth{1.003750pt}%
\definecolor{currentstroke}{rgb}{0.121569,0.466667,0.705882}%
\pgfsetstrokecolor{currentstroke}%
\pgfsetstrokeopacity{0.824782}%
\pgfsetdash{}{0pt}%
\pgfpathmoveto{\pgfqpoint{2.309043in}{2.119293in}}%
\pgfpathcurveto{\pgfqpoint{2.317279in}{2.119293in}}{\pgfqpoint{2.325179in}{2.122565in}}{\pgfqpoint{2.331003in}{2.128389in}}%
\pgfpathcurveto{\pgfqpoint{2.336827in}{2.134213in}}{\pgfqpoint{2.340099in}{2.142113in}}{\pgfqpoint{2.340099in}{2.150349in}}%
\pgfpathcurveto{\pgfqpoint{2.340099in}{2.158586in}}{\pgfqpoint{2.336827in}{2.166486in}}{\pgfqpoint{2.331003in}{2.172310in}}%
\pgfpathcurveto{\pgfqpoint{2.325179in}{2.178134in}}{\pgfqpoint{2.317279in}{2.181406in}}{\pgfqpoint{2.309043in}{2.181406in}}%
\pgfpathcurveto{\pgfqpoint{2.300806in}{2.181406in}}{\pgfqpoint{2.292906in}{2.178134in}}{\pgfqpoint{2.287082in}{2.172310in}}%
\pgfpathcurveto{\pgfqpoint{2.281258in}{2.166486in}}{\pgfqpoint{2.277986in}{2.158586in}}{\pgfqpoint{2.277986in}{2.150349in}}%
\pgfpathcurveto{\pgfqpoint{2.277986in}{2.142113in}}{\pgfqpoint{2.281258in}{2.134213in}}{\pgfqpoint{2.287082in}{2.128389in}}%
\pgfpathcurveto{\pgfqpoint{2.292906in}{2.122565in}}{\pgfqpoint{2.300806in}{2.119293in}}{\pgfqpoint{2.309043in}{2.119293in}}%
\pgfpathclose%
\pgfusepath{stroke,fill}%
\end{pgfscope}%
\begin{pgfscope}%
\pgfpathrectangle{\pgfqpoint{0.100000in}{0.212622in}}{\pgfqpoint{3.696000in}{3.696000in}}%
\pgfusepath{clip}%
\pgfsetbuttcap%
\pgfsetroundjoin%
\definecolor{currentfill}{rgb}{0.121569,0.466667,0.705882}%
\pgfsetfillcolor{currentfill}%
\pgfsetfillopacity{0.828990}%
\pgfsetlinewidth{1.003750pt}%
\definecolor{currentstroke}{rgb}{0.121569,0.466667,0.705882}%
\pgfsetstrokecolor{currentstroke}%
\pgfsetstrokeopacity{0.828990}%
\pgfsetdash{}{0pt}%
\pgfpathmoveto{\pgfqpoint{2.312517in}{2.123850in}}%
\pgfpathcurveto{\pgfqpoint{2.320754in}{2.123850in}}{\pgfqpoint{2.328654in}{2.127122in}}{\pgfqpoint{2.334478in}{2.132946in}}%
\pgfpathcurveto{\pgfqpoint{2.340302in}{2.138770in}}{\pgfqpoint{2.343574in}{2.146670in}}{\pgfqpoint{2.343574in}{2.154906in}}%
\pgfpathcurveto{\pgfqpoint{2.343574in}{2.163143in}}{\pgfqpoint{2.340302in}{2.171043in}}{\pgfqpoint{2.334478in}{2.176867in}}%
\pgfpathcurveto{\pgfqpoint{2.328654in}{2.182690in}}{\pgfqpoint{2.320754in}{2.185963in}}{\pgfqpoint{2.312517in}{2.185963in}}%
\pgfpathcurveto{\pgfqpoint{2.304281in}{2.185963in}}{\pgfqpoint{2.296381in}{2.182690in}}{\pgfqpoint{2.290557in}{2.176867in}}%
\pgfpathcurveto{\pgfqpoint{2.284733in}{2.171043in}}{\pgfqpoint{2.281461in}{2.163143in}}{\pgfqpoint{2.281461in}{2.154906in}}%
\pgfpathcurveto{\pgfqpoint{2.281461in}{2.146670in}}{\pgfqpoint{2.284733in}{2.138770in}}{\pgfqpoint{2.290557in}{2.132946in}}%
\pgfpathcurveto{\pgfqpoint{2.296381in}{2.127122in}}{\pgfqpoint{2.304281in}{2.123850in}}{\pgfqpoint{2.312517in}{2.123850in}}%
\pgfpathclose%
\pgfusepath{stroke,fill}%
\end{pgfscope}%
\begin{pgfscope}%
\pgfpathrectangle{\pgfqpoint{0.100000in}{0.212622in}}{\pgfqpoint{3.696000in}{3.696000in}}%
\pgfusepath{clip}%
\pgfsetbuttcap%
\pgfsetroundjoin%
\definecolor{currentfill}{rgb}{0.121569,0.466667,0.705882}%
\pgfsetfillcolor{currentfill}%
\pgfsetfillopacity{0.832553}%
\pgfsetlinewidth{1.003750pt}%
\definecolor{currentstroke}{rgb}{0.121569,0.466667,0.705882}%
\pgfsetstrokecolor{currentstroke}%
\pgfsetstrokeopacity{0.832553}%
\pgfsetdash{}{0pt}%
\pgfpathmoveto{\pgfqpoint{2.849909in}{2.428573in}}%
\pgfpathcurveto{\pgfqpoint{2.858145in}{2.428573in}}{\pgfqpoint{2.866045in}{2.431845in}}{\pgfqpoint{2.871869in}{2.437669in}}%
\pgfpathcurveto{\pgfqpoint{2.877693in}{2.443493in}}{\pgfqpoint{2.880965in}{2.451393in}}{\pgfqpoint{2.880965in}{2.459629in}}%
\pgfpathcurveto{\pgfqpoint{2.880965in}{2.467866in}}{\pgfqpoint{2.877693in}{2.475766in}}{\pgfqpoint{2.871869in}{2.481590in}}%
\pgfpathcurveto{\pgfqpoint{2.866045in}{2.487413in}}{\pgfqpoint{2.858145in}{2.490686in}}{\pgfqpoint{2.849909in}{2.490686in}}%
\pgfpathcurveto{\pgfqpoint{2.841672in}{2.490686in}}{\pgfqpoint{2.833772in}{2.487413in}}{\pgfqpoint{2.827948in}{2.481590in}}%
\pgfpathcurveto{\pgfqpoint{2.822124in}{2.475766in}}{\pgfqpoint{2.818852in}{2.467866in}}{\pgfqpoint{2.818852in}{2.459629in}}%
\pgfpathcurveto{\pgfqpoint{2.818852in}{2.451393in}}{\pgfqpoint{2.822124in}{2.443493in}}{\pgfqpoint{2.827948in}{2.437669in}}%
\pgfpathcurveto{\pgfqpoint{2.833772in}{2.431845in}}{\pgfqpoint{2.841672in}{2.428573in}}{\pgfqpoint{2.849909in}{2.428573in}}%
\pgfpathclose%
\pgfusepath{stroke,fill}%
\end{pgfscope}%
\begin{pgfscope}%
\pgfpathrectangle{\pgfqpoint{0.100000in}{0.212622in}}{\pgfqpoint{3.696000in}{3.696000in}}%
\pgfusepath{clip}%
\pgfsetbuttcap%
\pgfsetroundjoin%
\definecolor{currentfill}{rgb}{0.121569,0.466667,0.705882}%
\pgfsetfillcolor{currentfill}%
\pgfsetfillopacity{0.836717}%
\pgfsetlinewidth{1.003750pt}%
\definecolor{currentstroke}{rgb}{0.121569,0.466667,0.705882}%
\pgfsetstrokecolor{currentstroke}%
\pgfsetstrokeopacity{0.836717}%
\pgfsetdash{}{0pt}%
\pgfpathmoveto{\pgfqpoint{2.306738in}{2.113676in}}%
\pgfpathcurveto{\pgfqpoint{2.314974in}{2.113676in}}{\pgfqpoint{2.322874in}{2.116948in}}{\pgfqpoint{2.328698in}{2.122772in}}%
\pgfpathcurveto{\pgfqpoint{2.334522in}{2.128596in}}{\pgfqpoint{2.337794in}{2.136496in}}{\pgfqpoint{2.337794in}{2.144733in}}%
\pgfpathcurveto{\pgfqpoint{2.337794in}{2.152969in}}{\pgfqpoint{2.334522in}{2.160869in}}{\pgfqpoint{2.328698in}{2.166693in}}%
\pgfpathcurveto{\pgfqpoint{2.322874in}{2.172517in}}{\pgfqpoint{2.314974in}{2.175789in}}{\pgfqpoint{2.306738in}{2.175789in}}%
\pgfpathcurveto{\pgfqpoint{2.298502in}{2.175789in}}{\pgfqpoint{2.290602in}{2.172517in}}{\pgfqpoint{2.284778in}{2.166693in}}%
\pgfpathcurveto{\pgfqpoint{2.278954in}{2.160869in}}{\pgfqpoint{2.275681in}{2.152969in}}{\pgfqpoint{2.275681in}{2.144733in}}%
\pgfpathcurveto{\pgfqpoint{2.275681in}{2.136496in}}{\pgfqpoint{2.278954in}{2.128596in}}{\pgfqpoint{2.284778in}{2.122772in}}%
\pgfpathcurveto{\pgfqpoint{2.290602in}{2.116948in}}{\pgfqpoint{2.298502in}{2.113676in}}{\pgfqpoint{2.306738in}{2.113676in}}%
\pgfpathclose%
\pgfusepath{stroke,fill}%
\end{pgfscope}%
\begin{pgfscope}%
\pgfpathrectangle{\pgfqpoint{0.100000in}{0.212622in}}{\pgfqpoint{3.696000in}{3.696000in}}%
\pgfusepath{clip}%
\pgfsetbuttcap%
\pgfsetroundjoin%
\definecolor{currentfill}{rgb}{0.121569,0.466667,0.705882}%
\pgfsetfillcolor{currentfill}%
\pgfsetfillopacity{0.838335}%
\pgfsetlinewidth{1.003750pt}%
\definecolor{currentstroke}{rgb}{0.121569,0.466667,0.705882}%
\pgfsetstrokecolor{currentstroke}%
\pgfsetstrokeopacity{0.838335}%
\pgfsetdash{}{0pt}%
\pgfpathmoveto{\pgfqpoint{2.362710in}{2.156979in}}%
\pgfpathcurveto{\pgfqpoint{2.370946in}{2.156979in}}{\pgfqpoint{2.378847in}{2.160251in}}{\pgfqpoint{2.384670in}{2.166075in}}%
\pgfpathcurveto{\pgfqpoint{2.390494in}{2.171899in}}{\pgfqpoint{2.393767in}{2.179799in}}{\pgfqpoint{2.393767in}{2.188035in}}%
\pgfpathcurveto{\pgfqpoint{2.393767in}{2.196271in}}{\pgfqpoint{2.390494in}{2.204171in}}{\pgfqpoint{2.384670in}{2.209995in}}%
\pgfpathcurveto{\pgfqpoint{2.378847in}{2.215819in}}{\pgfqpoint{2.370946in}{2.219092in}}{\pgfqpoint{2.362710in}{2.219092in}}%
\pgfpathcurveto{\pgfqpoint{2.354474in}{2.219092in}}{\pgfqpoint{2.346574in}{2.215819in}}{\pgfqpoint{2.340750in}{2.209995in}}%
\pgfpathcurveto{\pgfqpoint{2.334926in}{2.204171in}}{\pgfqpoint{2.331654in}{2.196271in}}{\pgfqpoint{2.331654in}{2.188035in}}%
\pgfpathcurveto{\pgfqpoint{2.331654in}{2.179799in}}{\pgfqpoint{2.334926in}{2.171899in}}{\pgfqpoint{2.340750in}{2.166075in}}%
\pgfpathcurveto{\pgfqpoint{2.346574in}{2.160251in}}{\pgfqpoint{2.354474in}{2.156979in}}{\pgfqpoint{2.362710in}{2.156979in}}%
\pgfpathclose%
\pgfusepath{stroke,fill}%
\end{pgfscope}%
\begin{pgfscope}%
\pgfpathrectangle{\pgfqpoint{0.100000in}{0.212622in}}{\pgfqpoint{3.696000in}{3.696000in}}%
\pgfusepath{clip}%
\pgfsetbuttcap%
\pgfsetroundjoin%
\definecolor{currentfill}{rgb}{0.121569,0.466667,0.705882}%
\pgfsetfillcolor{currentfill}%
\pgfsetfillopacity{0.839798}%
\pgfsetlinewidth{1.003750pt}%
\definecolor{currentstroke}{rgb}{0.121569,0.466667,0.705882}%
\pgfsetstrokecolor{currentstroke}%
\pgfsetstrokeopacity{0.839798}%
\pgfsetdash{}{0pt}%
\pgfpathmoveto{\pgfqpoint{2.335231in}{2.130822in}}%
\pgfpathcurveto{\pgfqpoint{2.343467in}{2.130822in}}{\pgfqpoint{2.351367in}{2.134095in}}{\pgfqpoint{2.357191in}{2.139918in}}%
\pgfpathcurveto{\pgfqpoint{2.363015in}{2.145742in}}{\pgfqpoint{2.366287in}{2.153642in}}{\pgfqpoint{2.366287in}{2.161879in}}%
\pgfpathcurveto{\pgfqpoint{2.366287in}{2.170115in}}{\pgfqpoint{2.363015in}{2.178015in}}{\pgfqpoint{2.357191in}{2.183839in}}%
\pgfpathcurveto{\pgfqpoint{2.351367in}{2.189663in}}{\pgfqpoint{2.343467in}{2.192935in}}{\pgfqpoint{2.335231in}{2.192935in}}%
\pgfpathcurveto{\pgfqpoint{2.326994in}{2.192935in}}{\pgfqpoint{2.319094in}{2.189663in}}{\pgfqpoint{2.313270in}{2.183839in}}%
\pgfpathcurveto{\pgfqpoint{2.307446in}{2.178015in}}{\pgfqpoint{2.304174in}{2.170115in}}{\pgfqpoint{2.304174in}{2.161879in}}%
\pgfpathcurveto{\pgfqpoint{2.304174in}{2.153642in}}{\pgfqpoint{2.307446in}{2.145742in}}{\pgfqpoint{2.313270in}{2.139918in}}%
\pgfpathcurveto{\pgfqpoint{2.319094in}{2.134095in}}{\pgfqpoint{2.326994in}{2.130822in}}{\pgfqpoint{2.335231in}{2.130822in}}%
\pgfpathclose%
\pgfusepath{stroke,fill}%
\end{pgfscope}%
\begin{pgfscope}%
\pgfpathrectangle{\pgfqpoint{0.100000in}{0.212622in}}{\pgfqpoint{3.696000in}{3.696000in}}%
\pgfusepath{clip}%
\pgfsetbuttcap%
\pgfsetroundjoin%
\definecolor{currentfill}{rgb}{0.121569,0.466667,0.705882}%
\pgfsetfillcolor{currentfill}%
\pgfsetfillopacity{0.841645}%
\pgfsetlinewidth{1.003750pt}%
\definecolor{currentstroke}{rgb}{0.121569,0.466667,0.705882}%
\pgfsetstrokecolor{currentstroke}%
\pgfsetstrokeopacity{0.841645}%
\pgfsetdash{}{0pt}%
\pgfpathmoveto{\pgfqpoint{2.311530in}{2.115932in}}%
\pgfpathcurveto{\pgfqpoint{2.319766in}{2.115932in}}{\pgfqpoint{2.327666in}{2.119204in}}{\pgfqpoint{2.333490in}{2.125028in}}%
\pgfpathcurveto{\pgfqpoint{2.339314in}{2.130852in}}{\pgfqpoint{2.342586in}{2.138752in}}{\pgfqpoint{2.342586in}{2.146988in}}%
\pgfpathcurveto{\pgfqpoint{2.342586in}{2.155224in}}{\pgfqpoint{2.339314in}{2.163124in}}{\pgfqpoint{2.333490in}{2.168948in}}%
\pgfpathcurveto{\pgfqpoint{2.327666in}{2.174772in}}{\pgfqpoint{2.319766in}{2.178045in}}{\pgfqpoint{2.311530in}{2.178045in}}%
\pgfpathcurveto{\pgfqpoint{2.303294in}{2.178045in}}{\pgfqpoint{2.295394in}{2.174772in}}{\pgfqpoint{2.289570in}{2.168948in}}%
\pgfpathcurveto{\pgfqpoint{2.283746in}{2.163124in}}{\pgfqpoint{2.280473in}{2.155224in}}{\pgfqpoint{2.280473in}{2.146988in}}%
\pgfpathcurveto{\pgfqpoint{2.280473in}{2.138752in}}{\pgfqpoint{2.283746in}{2.130852in}}{\pgfqpoint{2.289570in}{2.125028in}}%
\pgfpathcurveto{\pgfqpoint{2.295394in}{2.119204in}}{\pgfqpoint{2.303294in}{2.115932in}}{\pgfqpoint{2.311530in}{2.115932in}}%
\pgfpathclose%
\pgfusepath{stroke,fill}%
\end{pgfscope}%
\begin{pgfscope}%
\pgfpathrectangle{\pgfqpoint{0.100000in}{0.212622in}}{\pgfqpoint{3.696000in}{3.696000in}}%
\pgfusepath{clip}%
\pgfsetbuttcap%
\pgfsetroundjoin%
\definecolor{currentfill}{rgb}{0.121569,0.466667,0.705882}%
\pgfsetfillcolor{currentfill}%
\pgfsetfillopacity{0.846106}%
\pgfsetlinewidth{1.003750pt}%
\definecolor{currentstroke}{rgb}{0.121569,0.466667,0.705882}%
\pgfsetstrokecolor{currentstroke}%
\pgfsetstrokeopacity{0.846106}%
\pgfsetdash{}{0pt}%
\pgfpathmoveto{\pgfqpoint{2.389558in}{2.174374in}}%
\pgfpathcurveto{\pgfqpoint{2.397794in}{2.174374in}}{\pgfqpoint{2.405694in}{2.177646in}}{\pgfqpoint{2.411518in}{2.183470in}}%
\pgfpathcurveto{\pgfqpoint{2.417342in}{2.189294in}}{\pgfqpoint{2.420614in}{2.197194in}}{\pgfqpoint{2.420614in}{2.205430in}}%
\pgfpathcurveto{\pgfqpoint{2.420614in}{2.213667in}}{\pgfqpoint{2.417342in}{2.221567in}}{\pgfqpoint{2.411518in}{2.227391in}}%
\pgfpathcurveto{\pgfqpoint{2.405694in}{2.233214in}}{\pgfqpoint{2.397794in}{2.236487in}}{\pgfqpoint{2.389558in}{2.236487in}}%
\pgfpathcurveto{\pgfqpoint{2.381321in}{2.236487in}}{\pgfqpoint{2.373421in}{2.233214in}}{\pgfqpoint{2.367597in}{2.227391in}}%
\pgfpathcurveto{\pgfqpoint{2.361773in}{2.221567in}}{\pgfqpoint{2.358501in}{2.213667in}}{\pgfqpoint{2.358501in}{2.205430in}}%
\pgfpathcurveto{\pgfqpoint{2.358501in}{2.197194in}}{\pgfqpoint{2.361773in}{2.189294in}}{\pgfqpoint{2.367597in}{2.183470in}}%
\pgfpathcurveto{\pgfqpoint{2.373421in}{2.177646in}}{\pgfqpoint{2.381321in}{2.174374in}}{\pgfqpoint{2.389558in}{2.174374in}}%
\pgfpathclose%
\pgfusepath{stroke,fill}%
\end{pgfscope}%
\begin{pgfscope}%
\pgfpathrectangle{\pgfqpoint{0.100000in}{0.212622in}}{\pgfqpoint{3.696000in}{3.696000in}}%
\pgfusepath{clip}%
\pgfsetbuttcap%
\pgfsetroundjoin%
\definecolor{currentfill}{rgb}{0.121569,0.466667,0.705882}%
\pgfsetfillcolor{currentfill}%
\pgfsetfillopacity{0.848894}%
\pgfsetlinewidth{1.003750pt}%
\definecolor{currentstroke}{rgb}{0.121569,0.466667,0.705882}%
\pgfsetstrokecolor{currentstroke}%
\pgfsetstrokeopacity{0.848894}%
\pgfsetdash{}{0pt}%
\pgfpathmoveto{\pgfqpoint{2.353268in}{2.136252in}}%
\pgfpathcurveto{\pgfqpoint{2.361504in}{2.136252in}}{\pgfqpoint{2.369404in}{2.139524in}}{\pgfqpoint{2.375228in}{2.145348in}}%
\pgfpathcurveto{\pgfqpoint{2.381052in}{2.151172in}}{\pgfqpoint{2.384324in}{2.159072in}}{\pgfqpoint{2.384324in}{2.167308in}}%
\pgfpathcurveto{\pgfqpoint{2.384324in}{2.175545in}}{\pgfqpoint{2.381052in}{2.183445in}}{\pgfqpoint{2.375228in}{2.189269in}}%
\pgfpathcurveto{\pgfqpoint{2.369404in}{2.195093in}}{\pgfqpoint{2.361504in}{2.198365in}}{\pgfqpoint{2.353268in}{2.198365in}}%
\pgfpathcurveto{\pgfqpoint{2.345032in}{2.198365in}}{\pgfqpoint{2.337131in}{2.195093in}}{\pgfqpoint{2.331308in}{2.189269in}}%
\pgfpathcurveto{\pgfqpoint{2.325484in}{2.183445in}}{\pgfqpoint{2.322211in}{2.175545in}}{\pgfqpoint{2.322211in}{2.167308in}}%
\pgfpathcurveto{\pgfqpoint{2.322211in}{2.159072in}}{\pgfqpoint{2.325484in}{2.151172in}}{\pgfqpoint{2.331308in}{2.145348in}}%
\pgfpathcurveto{\pgfqpoint{2.337131in}{2.139524in}}{\pgfqpoint{2.345032in}{2.136252in}}{\pgfqpoint{2.353268in}{2.136252in}}%
\pgfpathclose%
\pgfusepath{stroke,fill}%
\end{pgfscope}%
\begin{pgfscope}%
\pgfpathrectangle{\pgfqpoint{0.100000in}{0.212622in}}{\pgfqpoint{3.696000in}{3.696000in}}%
\pgfusepath{clip}%
\pgfsetbuttcap%
\pgfsetroundjoin%
\definecolor{currentfill}{rgb}{0.121569,0.466667,0.705882}%
\pgfsetfillcolor{currentfill}%
\pgfsetfillopacity{0.854276}%
\pgfsetlinewidth{1.003750pt}%
\definecolor{currentstroke}{rgb}{0.121569,0.466667,0.705882}%
\pgfsetstrokecolor{currentstroke}%
\pgfsetstrokeopacity{0.854276}%
\pgfsetdash{}{0pt}%
\pgfpathmoveto{\pgfqpoint{2.421335in}{2.193496in}}%
\pgfpathcurveto{\pgfqpoint{2.429571in}{2.193496in}}{\pgfqpoint{2.437471in}{2.196768in}}{\pgfqpoint{2.443295in}{2.202592in}}%
\pgfpathcurveto{\pgfqpoint{2.449119in}{2.208416in}}{\pgfqpoint{2.452391in}{2.216316in}}{\pgfqpoint{2.452391in}{2.224552in}}%
\pgfpathcurveto{\pgfqpoint{2.452391in}{2.232788in}}{\pgfqpoint{2.449119in}{2.240688in}}{\pgfqpoint{2.443295in}{2.246512in}}%
\pgfpathcurveto{\pgfqpoint{2.437471in}{2.252336in}}{\pgfqpoint{2.429571in}{2.255609in}}{\pgfqpoint{2.421335in}{2.255609in}}%
\pgfpathcurveto{\pgfqpoint{2.413099in}{2.255609in}}{\pgfqpoint{2.405199in}{2.252336in}}{\pgfqpoint{2.399375in}{2.246512in}}%
\pgfpathcurveto{\pgfqpoint{2.393551in}{2.240688in}}{\pgfqpoint{2.390278in}{2.232788in}}{\pgfqpoint{2.390278in}{2.224552in}}%
\pgfpathcurveto{\pgfqpoint{2.390278in}{2.216316in}}{\pgfqpoint{2.393551in}{2.208416in}}{\pgfqpoint{2.399375in}{2.202592in}}%
\pgfpathcurveto{\pgfqpoint{2.405199in}{2.196768in}}{\pgfqpoint{2.413099in}{2.193496in}}{\pgfqpoint{2.421335in}{2.193496in}}%
\pgfpathclose%
\pgfusepath{stroke,fill}%
\end{pgfscope}%
\begin{pgfscope}%
\pgfpathrectangle{\pgfqpoint{0.100000in}{0.212622in}}{\pgfqpoint{3.696000in}{3.696000in}}%
\pgfusepath{clip}%
\pgfsetbuttcap%
\pgfsetroundjoin%
\definecolor{currentfill}{rgb}{0.121569,0.466667,0.705882}%
\pgfsetfillcolor{currentfill}%
\pgfsetfillopacity{0.860802}%
\pgfsetlinewidth{1.003750pt}%
\definecolor{currentstroke}{rgb}{0.121569,0.466667,0.705882}%
\pgfsetstrokecolor{currentstroke}%
\pgfsetstrokeopacity{0.860802}%
\pgfsetdash{}{0pt}%
\pgfpathmoveto{\pgfqpoint{2.370062in}{2.137848in}}%
\pgfpathcurveto{\pgfqpoint{2.378298in}{2.137848in}}{\pgfqpoint{2.386198in}{2.141121in}}{\pgfqpoint{2.392022in}{2.146945in}}%
\pgfpathcurveto{\pgfqpoint{2.397846in}{2.152769in}}{\pgfqpoint{2.401118in}{2.160669in}}{\pgfqpoint{2.401118in}{2.168905in}}%
\pgfpathcurveto{\pgfqpoint{2.401118in}{2.177141in}}{\pgfqpoint{2.397846in}{2.185041in}}{\pgfqpoint{2.392022in}{2.190865in}}%
\pgfpathcurveto{\pgfqpoint{2.386198in}{2.196689in}}{\pgfqpoint{2.378298in}{2.199961in}}{\pgfqpoint{2.370062in}{2.199961in}}%
\pgfpathcurveto{\pgfqpoint{2.361825in}{2.199961in}}{\pgfqpoint{2.353925in}{2.196689in}}{\pgfqpoint{2.348101in}{2.190865in}}%
\pgfpathcurveto{\pgfqpoint{2.342277in}{2.185041in}}{\pgfqpoint{2.339005in}{2.177141in}}{\pgfqpoint{2.339005in}{2.168905in}}%
\pgfpathcurveto{\pgfqpoint{2.339005in}{2.160669in}}{\pgfqpoint{2.342277in}{2.152769in}}{\pgfqpoint{2.348101in}{2.146945in}}%
\pgfpathcurveto{\pgfqpoint{2.353925in}{2.141121in}}{\pgfqpoint{2.361825in}{2.137848in}}{\pgfqpoint{2.370062in}{2.137848in}}%
\pgfpathclose%
\pgfusepath{stroke,fill}%
\end{pgfscope}%
\begin{pgfscope}%
\pgfpathrectangle{\pgfqpoint{0.100000in}{0.212622in}}{\pgfqpoint{3.696000in}{3.696000in}}%
\pgfusepath{clip}%
\pgfsetbuttcap%
\pgfsetroundjoin%
\definecolor{currentfill}{rgb}{0.121569,0.466667,0.705882}%
\pgfsetfillcolor{currentfill}%
\pgfsetfillopacity{0.865728}%
\pgfsetlinewidth{1.003750pt}%
\definecolor{currentstroke}{rgb}{0.121569,0.466667,0.705882}%
\pgfsetstrokecolor{currentstroke}%
\pgfsetstrokeopacity{0.865728}%
\pgfsetdash{}{0pt}%
\pgfpathmoveto{\pgfqpoint{2.753711in}{2.346659in}}%
\pgfpathcurveto{\pgfqpoint{2.761947in}{2.346659in}}{\pgfqpoint{2.769847in}{2.349932in}}{\pgfqpoint{2.775671in}{2.355755in}}%
\pgfpathcurveto{\pgfqpoint{2.781495in}{2.361579in}}{\pgfqpoint{2.784767in}{2.369479in}}{\pgfqpoint{2.784767in}{2.377716in}}%
\pgfpathcurveto{\pgfqpoint{2.784767in}{2.385952in}}{\pgfqpoint{2.781495in}{2.393852in}}{\pgfqpoint{2.775671in}{2.399676in}}%
\pgfpathcurveto{\pgfqpoint{2.769847in}{2.405500in}}{\pgfqpoint{2.761947in}{2.408772in}}{\pgfqpoint{2.753711in}{2.408772in}}%
\pgfpathcurveto{\pgfqpoint{2.745474in}{2.408772in}}{\pgfqpoint{2.737574in}{2.405500in}}{\pgfqpoint{2.731750in}{2.399676in}}%
\pgfpathcurveto{\pgfqpoint{2.725926in}{2.393852in}}{\pgfqpoint{2.722654in}{2.385952in}}{\pgfqpoint{2.722654in}{2.377716in}}%
\pgfpathcurveto{\pgfqpoint{2.722654in}{2.369479in}}{\pgfqpoint{2.725926in}{2.361579in}}{\pgfqpoint{2.731750in}{2.355755in}}%
\pgfpathcurveto{\pgfqpoint{2.737574in}{2.349932in}}{\pgfqpoint{2.745474in}{2.346659in}}{\pgfqpoint{2.753711in}{2.346659in}}%
\pgfpathclose%
\pgfusepath{stroke,fill}%
\end{pgfscope}%
\begin{pgfscope}%
\pgfpathrectangle{\pgfqpoint{0.100000in}{0.212622in}}{\pgfqpoint{3.696000in}{3.696000in}}%
\pgfusepath{clip}%
\pgfsetbuttcap%
\pgfsetroundjoin%
\definecolor{currentfill}{rgb}{0.121569,0.466667,0.705882}%
\pgfsetfillcolor{currentfill}%
\pgfsetfillopacity{0.872394}%
\pgfsetlinewidth{1.003750pt}%
\definecolor{currentstroke}{rgb}{0.121569,0.466667,0.705882}%
\pgfsetstrokecolor{currentstroke}%
\pgfsetstrokeopacity{0.872394}%
\pgfsetdash{}{0pt}%
\pgfpathmoveto{\pgfqpoint{2.427344in}{2.176286in}}%
\pgfpathcurveto{\pgfqpoint{2.435580in}{2.176286in}}{\pgfqpoint{2.443480in}{2.179558in}}{\pgfqpoint{2.449304in}{2.185382in}}%
\pgfpathcurveto{\pgfqpoint{2.455128in}{2.191206in}}{\pgfqpoint{2.458400in}{2.199106in}}{\pgfqpoint{2.458400in}{2.207342in}}%
\pgfpathcurveto{\pgfqpoint{2.458400in}{2.215578in}}{\pgfqpoint{2.455128in}{2.223478in}}{\pgfqpoint{2.449304in}{2.229302in}}%
\pgfpathcurveto{\pgfqpoint{2.443480in}{2.235126in}}{\pgfqpoint{2.435580in}{2.238399in}}{\pgfqpoint{2.427344in}{2.238399in}}%
\pgfpathcurveto{\pgfqpoint{2.419108in}{2.238399in}}{\pgfqpoint{2.411208in}{2.235126in}}{\pgfqpoint{2.405384in}{2.229302in}}%
\pgfpathcurveto{\pgfqpoint{2.399560in}{2.223478in}}{\pgfqpoint{2.396287in}{2.215578in}}{\pgfqpoint{2.396287in}{2.207342in}}%
\pgfpathcurveto{\pgfqpoint{2.396287in}{2.199106in}}{\pgfqpoint{2.399560in}{2.191206in}}{\pgfqpoint{2.405384in}{2.185382in}}%
\pgfpathcurveto{\pgfqpoint{2.411208in}{2.179558in}}{\pgfqpoint{2.419108in}{2.176286in}}{\pgfqpoint{2.427344in}{2.176286in}}%
\pgfpathclose%
\pgfusepath{stroke,fill}%
\end{pgfscope}%
\begin{pgfscope}%
\pgfpathrectangle{\pgfqpoint{0.100000in}{0.212622in}}{\pgfqpoint{3.696000in}{3.696000in}}%
\pgfusepath{clip}%
\pgfsetbuttcap%
\pgfsetroundjoin%
\definecolor{currentfill}{rgb}{0.121569,0.466667,0.705882}%
\pgfsetfillcolor{currentfill}%
\pgfsetfillopacity{0.883287}%
\pgfsetlinewidth{1.003750pt}%
\definecolor{currentstroke}{rgb}{0.121569,0.466667,0.705882}%
\pgfsetstrokecolor{currentstroke}%
\pgfsetstrokeopacity{0.883287}%
\pgfsetdash{}{0pt}%
\pgfpathmoveto{\pgfqpoint{2.362822in}{2.113205in}}%
\pgfpathcurveto{\pgfqpoint{2.371059in}{2.113205in}}{\pgfqpoint{2.378959in}{2.116477in}}{\pgfqpoint{2.384783in}{2.122301in}}%
\pgfpathcurveto{\pgfqpoint{2.390606in}{2.128125in}}{\pgfqpoint{2.393879in}{2.136025in}}{\pgfqpoint{2.393879in}{2.144261in}}%
\pgfpathcurveto{\pgfqpoint{2.393879in}{2.152498in}}{\pgfqpoint{2.390606in}{2.160398in}}{\pgfqpoint{2.384783in}{2.166222in}}%
\pgfpathcurveto{\pgfqpoint{2.378959in}{2.172046in}}{\pgfqpoint{2.371059in}{2.175318in}}{\pgfqpoint{2.362822in}{2.175318in}}%
\pgfpathcurveto{\pgfqpoint{2.354586in}{2.175318in}}{\pgfqpoint{2.346686in}{2.172046in}}{\pgfqpoint{2.340862in}{2.166222in}}%
\pgfpathcurveto{\pgfqpoint{2.335038in}{2.160398in}}{\pgfqpoint{2.331766in}{2.152498in}}{\pgfqpoint{2.331766in}{2.144261in}}%
\pgfpathcurveto{\pgfqpoint{2.331766in}{2.136025in}}{\pgfqpoint{2.335038in}{2.128125in}}{\pgfqpoint{2.340862in}{2.122301in}}%
\pgfpathcurveto{\pgfqpoint{2.346686in}{2.116477in}}{\pgfqpoint{2.354586in}{2.113205in}}{\pgfqpoint{2.362822in}{2.113205in}}%
\pgfpathclose%
\pgfusepath{stroke,fill}%
\end{pgfscope}%
\begin{pgfscope}%
\pgfpathrectangle{\pgfqpoint{0.100000in}{0.212622in}}{\pgfqpoint{3.696000in}{3.696000in}}%
\pgfusepath{clip}%
\pgfsetbuttcap%
\pgfsetroundjoin%
\definecolor{currentfill}{rgb}{0.121569,0.466667,0.705882}%
\pgfsetfillcolor{currentfill}%
\pgfsetfillopacity{0.896525}%
\pgfsetlinewidth{1.003750pt}%
\definecolor{currentstroke}{rgb}{0.121569,0.466667,0.705882}%
\pgfsetstrokecolor{currentstroke}%
\pgfsetstrokeopacity{0.896525}%
\pgfsetdash{}{0pt}%
\pgfpathmoveto{\pgfqpoint{2.425031in}{2.159048in}}%
\pgfpathcurveto{\pgfqpoint{2.433268in}{2.159048in}}{\pgfqpoint{2.441168in}{2.162320in}}{\pgfqpoint{2.446992in}{2.168144in}}%
\pgfpathcurveto{\pgfqpoint{2.452816in}{2.173968in}}{\pgfqpoint{2.456088in}{2.181868in}}{\pgfqpoint{2.456088in}{2.190105in}}%
\pgfpathcurveto{\pgfqpoint{2.456088in}{2.198341in}}{\pgfqpoint{2.452816in}{2.206241in}}{\pgfqpoint{2.446992in}{2.212065in}}%
\pgfpathcurveto{\pgfqpoint{2.441168in}{2.217889in}}{\pgfqpoint{2.433268in}{2.221161in}}{\pgfqpoint{2.425031in}{2.221161in}}%
\pgfpathcurveto{\pgfqpoint{2.416795in}{2.221161in}}{\pgfqpoint{2.408895in}{2.217889in}}{\pgfqpoint{2.403071in}{2.212065in}}%
\pgfpathcurveto{\pgfqpoint{2.397247in}{2.206241in}}{\pgfqpoint{2.393975in}{2.198341in}}{\pgfqpoint{2.393975in}{2.190105in}}%
\pgfpathcurveto{\pgfqpoint{2.393975in}{2.181868in}}{\pgfqpoint{2.397247in}{2.173968in}}{\pgfqpoint{2.403071in}{2.168144in}}%
\pgfpathcurveto{\pgfqpoint{2.408895in}{2.162320in}}{\pgfqpoint{2.416795in}{2.159048in}}{\pgfqpoint{2.425031in}{2.159048in}}%
\pgfpathclose%
\pgfusepath{stroke,fill}%
\end{pgfscope}%
\begin{pgfscope}%
\pgfpathrectangle{\pgfqpoint{0.100000in}{0.212622in}}{\pgfqpoint{3.696000in}{3.696000in}}%
\pgfusepath{clip}%
\pgfsetbuttcap%
\pgfsetroundjoin%
\definecolor{currentfill}{rgb}{0.121569,0.466667,0.705882}%
\pgfsetfillcolor{currentfill}%
\pgfsetfillopacity{0.900093}%
\pgfsetlinewidth{1.003750pt}%
\definecolor{currentstroke}{rgb}{0.121569,0.466667,0.705882}%
\pgfsetstrokecolor{currentstroke}%
\pgfsetstrokeopacity{0.900093}%
\pgfsetdash{}{0pt}%
\pgfpathmoveto{\pgfqpoint{2.377922in}{2.110295in}}%
\pgfpathcurveto{\pgfqpoint{2.386158in}{2.110295in}}{\pgfqpoint{2.394058in}{2.113567in}}{\pgfqpoint{2.399882in}{2.119391in}}%
\pgfpathcurveto{\pgfqpoint{2.405706in}{2.125215in}}{\pgfqpoint{2.408978in}{2.133115in}}{\pgfqpoint{2.408978in}{2.141351in}}%
\pgfpathcurveto{\pgfqpoint{2.408978in}{2.149587in}}{\pgfqpoint{2.405706in}{2.157487in}}{\pgfqpoint{2.399882in}{2.163311in}}%
\pgfpathcurveto{\pgfqpoint{2.394058in}{2.169135in}}{\pgfqpoint{2.386158in}{2.172408in}}{\pgfqpoint{2.377922in}{2.172408in}}%
\pgfpathcurveto{\pgfqpoint{2.369686in}{2.172408in}}{\pgfqpoint{2.361786in}{2.169135in}}{\pgfqpoint{2.355962in}{2.163311in}}%
\pgfpathcurveto{\pgfqpoint{2.350138in}{2.157487in}}{\pgfqpoint{2.346865in}{2.149587in}}{\pgfqpoint{2.346865in}{2.141351in}}%
\pgfpathcurveto{\pgfqpoint{2.346865in}{2.133115in}}{\pgfqpoint{2.350138in}{2.125215in}}{\pgfqpoint{2.355962in}{2.119391in}}%
\pgfpathcurveto{\pgfqpoint{2.361786in}{2.113567in}}{\pgfqpoint{2.369686in}{2.110295in}}{\pgfqpoint{2.377922in}{2.110295in}}%
\pgfpathclose%
\pgfusepath{stroke,fill}%
\end{pgfscope}%
\begin{pgfscope}%
\pgfpathrectangle{\pgfqpoint{0.100000in}{0.212622in}}{\pgfqpoint{3.696000in}{3.696000in}}%
\pgfusepath{clip}%
\pgfsetbuttcap%
\pgfsetroundjoin%
\definecolor{currentfill}{rgb}{0.121569,0.466667,0.705882}%
\pgfsetfillcolor{currentfill}%
\pgfsetfillopacity{0.916183}%
\pgfsetlinewidth{1.003750pt}%
\definecolor{currentstroke}{rgb}{0.121569,0.466667,0.705882}%
\pgfsetstrokecolor{currentstroke}%
\pgfsetstrokeopacity{0.916183}%
\pgfsetdash{}{0pt}%
\pgfpathmoveto{\pgfqpoint{2.433525in}{2.137855in}}%
\pgfpathcurveto{\pgfqpoint{2.441761in}{2.137855in}}{\pgfqpoint{2.449661in}{2.141127in}}{\pgfqpoint{2.455485in}{2.146951in}}%
\pgfpathcurveto{\pgfqpoint{2.461309in}{2.152775in}}{\pgfqpoint{2.464581in}{2.160675in}}{\pgfqpoint{2.464581in}{2.168911in}}%
\pgfpathcurveto{\pgfqpoint{2.464581in}{2.177147in}}{\pgfqpoint{2.461309in}{2.185047in}}{\pgfqpoint{2.455485in}{2.190871in}}%
\pgfpathcurveto{\pgfqpoint{2.449661in}{2.196695in}}{\pgfqpoint{2.441761in}{2.199968in}}{\pgfqpoint{2.433525in}{2.199968in}}%
\pgfpathcurveto{\pgfqpoint{2.425288in}{2.199968in}}{\pgfqpoint{2.417388in}{2.196695in}}{\pgfqpoint{2.411564in}{2.190871in}}%
\pgfpathcurveto{\pgfqpoint{2.405741in}{2.185047in}}{\pgfqpoint{2.402468in}{2.177147in}}{\pgfqpoint{2.402468in}{2.168911in}}%
\pgfpathcurveto{\pgfqpoint{2.402468in}{2.160675in}}{\pgfqpoint{2.405741in}{2.152775in}}{\pgfqpoint{2.411564in}{2.146951in}}%
\pgfpathcurveto{\pgfqpoint{2.417388in}{2.141127in}}{\pgfqpoint{2.425288in}{2.137855in}}{\pgfqpoint{2.433525in}{2.137855in}}%
\pgfpathclose%
\pgfusepath{stroke,fill}%
\end{pgfscope}%
\begin{pgfscope}%
\pgfpathrectangle{\pgfqpoint{0.100000in}{0.212622in}}{\pgfqpoint{3.696000in}{3.696000in}}%
\pgfusepath{clip}%
\pgfsetbuttcap%
\pgfsetroundjoin%
\definecolor{currentfill}{rgb}{0.121569,0.466667,0.705882}%
\pgfsetfillcolor{currentfill}%
\pgfsetfillopacity{0.920511}%
\pgfsetlinewidth{1.003750pt}%
\definecolor{currentstroke}{rgb}{0.121569,0.466667,0.705882}%
\pgfsetstrokecolor{currentstroke}%
\pgfsetstrokeopacity{0.920511}%
\pgfsetdash{}{0pt}%
\pgfpathmoveto{\pgfqpoint{2.605264in}{2.213747in}}%
\pgfpathcurveto{\pgfqpoint{2.613501in}{2.213747in}}{\pgfqpoint{2.621401in}{2.217019in}}{\pgfqpoint{2.627225in}{2.222843in}}%
\pgfpathcurveto{\pgfqpoint{2.633049in}{2.228667in}}{\pgfqpoint{2.636321in}{2.236567in}}{\pgfqpoint{2.636321in}{2.244803in}}%
\pgfpathcurveto{\pgfqpoint{2.636321in}{2.253039in}}{\pgfqpoint{2.633049in}{2.260940in}}{\pgfqpoint{2.627225in}{2.266763in}}%
\pgfpathcurveto{\pgfqpoint{2.621401in}{2.272587in}}{\pgfqpoint{2.613501in}{2.275860in}}{\pgfqpoint{2.605264in}{2.275860in}}%
\pgfpathcurveto{\pgfqpoint{2.597028in}{2.275860in}}{\pgfqpoint{2.589128in}{2.272587in}}{\pgfqpoint{2.583304in}{2.266763in}}%
\pgfpathcurveto{\pgfqpoint{2.577480in}{2.260940in}}{\pgfqpoint{2.574208in}{2.253039in}}{\pgfqpoint{2.574208in}{2.244803in}}%
\pgfpathcurveto{\pgfqpoint{2.574208in}{2.236567in}}{\pgfqpoint{2.577480in}{2.228667in}}{\pgfqpoint{2.583304in}{2.222843in}}%
\pgfpathcurveto{\pgfqpoint{2.589128in}{2.217019in}}{\pgfqpoint{2.597028in}{2.213747in}}{\pgfqpoint{2.605264in}{2.213747in}}%
\pgfpathclose%
\pgfusepath{stroke,fill}%
\end{pgfscope}%
\begin{pgfscope}%
\pgfpathrectangle{\pgfqpoint{0.100000in}{0.212622in}}{\pgfqpoint{3.696000in}{3.696000in}}%
\pgfusepath{clip}%
\pgfsetbuttcap%
\pgfsetroundjoin%
\definecolor{currentfill}{rgb}{0.121569,0.466667,0.705882}%
\pgfsetfillcolor{currentfill}%
\pgfsetfillopacity{0.925943}%
\pgfsetlinewidth{1.003750pt}%
\definecolor{currentstroke}{rgb}{0.121569,0.466667,0.705882}%
\pgfsetstrokecolor{currentstroke}%
\pgfsetstrokeopacity{0.925943}%
\pgfsetdash{}{0pt}%
\pgfpathmoveto{\pgfqpoint{2.367470in}{2.071000in}}%
\pgfpathcurveto{\pgfqpoint{2.375706in}{2.071000in}}{\pgfqpoint{2.383606in}{2.074272in}}{\pgfqpoint{2.389430in}{2.080096in}}%
\pgfpathcurveto{\pgfqpoint{2.395254in}{2.085920in}}{\pgfqpoint{2.398527in}{2.093820in}}{\pgfqpoint{2.398527in}{2.102056in}}%
\pgfpathcurveto{\pgfqpoint{2.398527in}{2.110292in}}{\pgfqpoint{2.395254in}{2.118192in}}{\pgfqpoint{2.389430in}{2.124016in}}%
\pgfpathcurveto{\pgfqpoint{2.383606in}{2.129840in}}{\pgfqpoint{2.375706in}{2.133113in}}{\pgfqpoint{2.367470in}{2.133113in}}%
\pgfpathcurveto{\pgfqpoint{2.359234in}{2.133113in}}{\pgfqpoint{2.351334in}{2.129840in}}{\pgfqpoint{2.345510in}{2.124016in}}%
\pgfpathcurveto{\pgfqpoint{2.339686in}{2.118192in}}{\pgfqpoint{2.336414in}{2.110292in}}{\pgfqpoint{2.336414in}{2.102056in}}%
\pgfpathcurveto{\pgfqpoint{2.336414in}{2.093820in}}{\pgfqpoint{2.339686in}{2.085920in}}{\pgfqpoint{2.345510in}{2.080096in}}%
\pgfpathcurveto{\pgfqpoint{2.351334in}{2.074272in}}{\pgfqpoint{2.359234in}{2.071000in}}{\pgfqpoint{2.367470in}{2.071000in}}%
\pgfpathclose%
\pgfusepath{stroke,fill}%
\end{pgfscope}%
\begin{pgfscope}%
\pgfpathrectangle{\pgfqpoint{0.100000in}{0.212622in}}{\pgfqpoint{3.696000in}{3.696000in}}%
\pgfusepath{clip}%
\pgfsetbuttcap%
\pgfsetroundjoin%
\definecolor{currentfill}{rgb}{0.121569,0.466667,0.705882}%
\pgfsetfillcolor{currentfill}%
\pgfsetfillopacity{0.940817}%
\pgfsetlinewidth{1.003750pt}%
\definecolor{currentstroke}{rgb}{0.121569,0.466667,0.705882}%
\pgfsetstrokecolor{currentstroke}%
\pgfsetstrokeopacity{0.940817}%
\pgfsetdash{}{0pt}%
\pgfpathmoveto{\pgfqpoint{2.398546in}{2.087675in}}%
\pgfpathcurveto{\pgfqpoint{2.406783in}{2.087675in}}{\pgfqpoint{2.414683in}{2.090947in}}{\pgfqpoint{2.420507in}{2.096771in}}%
\pgfpathcurveto{\pgfqpoint{2.426330in}{2.102595in}}{\pgfqpoint{2.429603in}{2.110495in}}{\pgfqpoint{2.429603in}{2.118732in}}%
\pgfpathcurveto{\pgfqpoint{2.429603in}{2.126968in}}{\pgfqpoint{2.426330in}{2.134868in}}{\pgfqpoint{2.420507in}{2.140692in}}%
\pgfpathcurveto{\pgfqpoint{2.414683in}{2.146516in}}{\pgfqpoint{2.406783in}{2.149788in}}{\pgfqpoint{2.398546in}{2.149788in}}%
\pgfpathcurveto{\pgfqpoint{2.390310in}{2.149788in}}{\pgfqpoint{2.382410in}{2.146516in}}{\pgfqpoint{2.376586in}{2.140692in}}%
\pgfpathcurveto{\pgfqpoint{2.370762in}{2.134868in}}{\pgfqpoint{2.367490in}{2.126968in}}{\pgfqpoint{2.367490in}{2.118732in}}%
\pgfpathcurveto{\pgfqpoint{2.367490in}{2.110495in}}{\pgfqpoint{2.370762in}{2.102595in}}{\pgfqpoint{2.376586in}{2.096771in}}%
\pgfpathcurveto{\pgfqpoint{2.382410in}{2.090947in}}{\pgfqpoint{2.390310in}{2.087675in}}{\pgfqpoint{2.398546in}{2.087675in}}%
\pgfpathclose%
\pgfusepath{stroke,fill}%
\end{pgfscope}%
\begin{pgfscope}%
\pgfpathrectangle{\pgfqpoint{0.100000in}{0.212622in}}{\pgfqpoint{3.696000in}{3.696000in}}%
\pgfusepath{clip}%
\pgfsetbuttcap%
\pgfsetroundjoin%
\definecolor{currentfill}{rgb}{0.121569,0.466667,0.705882}%
\pgfsetfillcolor{currentfill}%
\pgfsetfillopacity{0.950611}%
\pgfsetlinewidth{1.003750pt}%
\definecolor{currentstroke}{rgb}{0.121569,0.466667,0.705882}%
\pgfsetstrokecolor{currentstroke}%
\pgfsetstrokeopacity{0.950611}%
\pgfsetdash{}{0pt}%
\pgfpathmoveto{\pgfqpoint{2.523426in}{2.146379in}}%
\pgfpathcurveto{\pgfqpoint{2.531663in}{2.146379in}}{\pgfqpoint{2.539563in}{2.149651in}}{\pgfqpoint{2.545387in}{2.155475in}}%
\pgfpathcurveto{\pgfqpoint{2.551211in}{2.161299in}}{\pgfqpoint{2.554483in}{2.169199in}}{\pgfqpoint{2.554483in}{2.177436in}}%
\pgfpathcurveto{\pgfqpoint{2.554483in}{2.185672in}}{\pgfqpoint{2.551211in}{2.193572in}}{\pgfqpoint{2.545387in}{2.199396in}}%
\pgfpathcurveto{\pgfqpoint{2.539563in}{2.205220in}}{\pgfqpoint{2.531663in}{2.208492in}}{\pgfqpoint{2.523426in}{2.208492in}}%
\pgfpathcurveto{\pgfqpoint{2.515190in}{2.208492in}}{\pgfqpoint{2.507290in}{2.205220in}}{\pgfqpoint{2.501466in}{2.199396in}}%
\pgfpathcurveto{\pgfqpoint{2.495642in}{2.193572in}}{\pgfqpoint{2.492370in}{2.185672in}}{\pgfqpoint{2.492370in}{2.177436in}}%
\pgfpathcurveto{\pgfqpoint{2.492370in}{2.169199in}}{\pgfqpoint{2.495642in}{2.161299in}}{\pgfqpoint{2.501466in}{2.155475in}}%
\pgfpathcurveto{\pgfqpoint{2.507290in}{2.149651in}}{\pgfqpoint{2.515190in}{2.146379in}}{\pgfqpoint{2.523426in}{2.146379in}}%
\pgfpathclose%
\pgfusepath{stroke,fill}%
\end{pgfscope}%
\begin{pgfscope}%
\pgfpathrectangle{\pgfqpoint{0.100000in}{0.212622in}}{\pgfqpoint{3.696000in}{3.696000in}}%
\pgfusepath{clip}%
\pgfsetbuttcap%
\pgfsetroundjoin%
\definecolor{currentfill}{rgb}{0.121569,0.466667,0.705882}%
\pgfsetfillcolor{currentfill}%
\pgfsetfillopacity{0.967138}%
\pgfsetlinewidth{1.003750pt}%
\definecolor{currentstroke}{rgb}{0.121569,0.466667,0.705882}%
\pgfsetstrokecolor{currentstroke}%
\pgfsetstrokeopacity{0.967138}%
\pgfsetdash{}{0pt}%
\pgfpathmoveto{\pgfqpoint{2.363981in}{2.048075in}}%
\pgfpathcurveto{\pgfqpoint{2.372217in}{2.048075in}}{\pgfqpoint{2.380117in}{2.051347in}}{\pgfqpoint{2.385941in}{2.057171in}}%
\pgfpathcurveto{\pgfqpoint{2.391765in}{2.062995in}}{\pgfqpoint{2.395038in}{2.070895in}}{\pgfqpoint{2.395038in}{2.079131in}}%
\pgfpathcurveto{\pgfqpoint{2.395038in}{2.087368in}}{\pgfqpoint{2.391765in}{2.095268in}}{\pgfqpoint{2.385941in}{2.101092in}}%
\pgfpathcurveto{\pgfqpoint{2.380117in}{2.106915in}}{\pgfqpoint{2.372217in}{2.110188in}}{\pgfqpoint{2.363981in}{2.110188in}}%
\pgfpathcurveto{\pgfqpoint{2.355745in}{2.110188in}}{\pgfqpoint{2.347845in}{2.106915in}}{\pgfqpoint{2.342021in}{2.101092in}}%
\pgfpathcurveto{\pgfqpoint{2.336197in}{2.095268in}}{\pgfqpoint{2.332925in}{2.087368in}}{\pgfqpoint{2.332925in}{2.079131in}}%
\pgfpathcurveto{\pgfqpoint{2.332925in}{2.070895in}}{\pgfqpoint{2.336197in}{2.062995in}}{\pgfqpoint{2.342021in}{2.057171in}}%
\pgfpathcurveto{\pgfqpoint{2.347845in}{2.051347in}}{\pgfqpoint{2.355745in}{2.048075in}}{\pgfqpoint{2.363981in}{2.048075in}}%
\pgfpathclose%
\pgfusepath{stroke,fill}%
\end{pgfscope}%
\begin{pgfscope}%
\pgfpathrectangle{\pgfqpoint{0.100000in}{0.212622in}}{\pgfqpoint{3.696000in}{3.696000in}}%
\pgfusepath{clip}%
\pgfsetbuttcap%
\pgfsetroundjoin%
\definecolor{currentfill}{rgb}{0.121569,0.466667,0.705882}%
\pgfsetfillcolor{currentfill}%
\pgfsetfillopacity{0.980993}%
\pgfsetlinewidth{1.003750pt}%
\definecolor{currentstroke}{rgb}{0.121569,0.466667,0.705882}%
\pgfsetstrokecolor{currentstroke}%
\pgfsetstrokeopacity{0.980993}%
\pgfsetdash{}{0pt}%
\pgfpathmoveto{\pgfqpoint{2.440360in}{2.072328in}}%
\pgfpathcurveto{\pgfqpoint{2.448596in}{2.072328in}}{\pgfqpoint{2.456496in}{2.075600in}}{\pgfqpoint{2.462320in}{2.081424in}}%
\pgfpathcurveto{\pgfqpoint{2.468144in}{2.087248in}}{\pgfqpoint{2.471416in}{2.095148in}}{\pgfqpoint{2.471416in}{2.103384in}}%
\pgfpathcurveto{\pgfqpoint{2.471416in}{2.111621in}}{\pgfqpoint{2.468144in}{2.119521in}}{\pgfqpoint{2.462320in}{2.125344in}}%
\pgfpathcurveto{\pgfqpoint{2.456496in}{2.131168in}}{\pgfqpoint{2.448596in}{2.134441in}}{\pgfqpoint{2.440360in}{2.134441in}}%
\pgfpathcurveto{\pgfqpoint{2.432123in}{2.134441in}}{\pgfqpoint{2.424223in}{2.131168in}}{\pgfqpoint{2.418399in}{2.125344in}}%
\pgfpathcurveto{\pgfqpoint{2.412575in}{2.119521in}}{\pgfqpoint{2.409303in}{2.111621in}}{\pgfqpoint{2.409303in}{2.103384in}}%
\pgfpathcurveto{\pgfqpoint{2.409303in}{2.095148in}}{\pgfqpoint{2.412575in}{2.087248in}}{\pgfqpoint{2.418399in}{2.081424in}}%
\pgfpathcurveto{\pgfqpoint{2.424223in}{2.075600in}}{\pgfqpoint{2.432123in}{2.072328in}}{\pgfqpoint{2.440360in}{2.072328in}}%
\pgfpathclose%
\pgfusepath{stroke,fill}%
\end{pgfscope}%
\begin{pgfscope}%
\pgfpathrectangle{\pgfqpoint{0.100000in}{0.212622in}}{\pgfqpoint{3.696000in}{3.696000in}}%
\pgfusepath{clip}%
\pgfsetbuttcap%
\pgfsetroundjoin%
\definecolor{currentfill}{rgb}{0.121569,0.466667,0.705882}%
\pgfsetfillcolor{currentfill}%
\pgfsetfillopacity{0.982385}%
\pgfsetlinewidth{1.003750pt}%
\definecolor{currentstroke}{rgb}{0.121569,0.466667,0.705882}%
\pgfsetstrokecolor{currentstroke}%
\pgfsetstrokeopacity{0.982385}%
\pgfsetdash{}{0pt}%
\pgfpathmoveto{\pgfqpoint{2.362701in}{2.044494in}}%
\pgfpathcurveto{\pgfqpoint{2.370938in}{2.044494in}}{\pgfqpoint{2.378838in}{2.047766in}}{\pgfqpoint{2.384662in}{2.053590in}}%
\pgfpathcurveto{\pgfqpoint{2.390486in}{2.059414in}}{\pgfqpoint{2.393758in}{2.067314in}}{\pgfqpoint{2.393758in}{2.075551in}}%
\pgfpathcurveto{\pgfqpoint{2.393758in}{2.083787in}}{\pgfqpoint{2.390486in}{2.091687in}}{\pgfqpoint{2.384662in}{2.097511in}}%
\pgfpathcurveto{\pgfqpoint{2.378838in}{2.103335in}}{\pgfqpoint{2.370938in}{2.106607in}}{\pgfqpoint{2.362701in}{2.106607in}}%
\pgfpathcurveto{\pgfqpoint{2.354465in}{2.106607in}}{\pgfqpoint{2.346565in}{2.103335in}}{\pgfqpoint{2.340741in}{2.097511in}}%
\pgfpathcurveto{\pgfqpoint{2.334917in}{2.091687in}}{\pgfqpoint{2.331645in}{2.083787in}}{\pgfqpoint{2.331645in}{2.075551in}}%
\pgfpathcurveto{\pgfqpoint{2.331645in}{2.067314in}}{\pgfqpoint{2.334917in}{2.059414in}}{\pgfqpoint{2.340741in}{2.053590in}}%
\pgfpathcurveto{\pgfqpoint{2.346565in}{2.047766in}}{\pgfqpoint{2.354465in}{2.044494in}}{\pgfqpoint{2.362701in}{2.044494in}}%
\pgfpathclose%
\pgfusepath{stroke,fill}%
\end{pgfscope}%
\begin{pgfscope}%
\pgfpathrectangle{\pgfqpoint{0.100000in}{0.212622in}}{\pgfqpoint{3.696000in}{3.696000in}}%
\pgfusepath{clip}%
\pgfsetbuttcap%
\pgfsetroundjoin%
\definecolor{currentfill}{rgb}{0.121569,0.466667,0.705882}%
\pgfsetfillcolor{currentfill}%
\pgfsetfillopacity{0.984214}%
\pgfsetlinewidth{1.003750pt}%
\definecolor{currentstroke}{rgb}{0.121569,0.466667,0.705882}%
\pgfsetstrokecolor{currentstroke}%
\pgfsetstrokeopacity{0.984214}%
\pgfsetdash{}{0pt}%
\pgfpathmoveto{\pgfqpoint{2.380276in}{2.061025in}}%
\pgfpathcurveto{\pgfqpoint{2.388512in}{2.061025in}}{\pgfqpoint{2.396412in}{2.064297in}}{\pgfqpoint{2.402236in}{2.070121in}}%
\pgfpathcurveto{\pgfqpoint{2.408060in}{2.075945in}}{\pgfqpoint{2.411332in}{2.083845in}}{\pgfqpoint{2.411332in}{2.092081in}}%
\pgfpathcurveto{\pgfqpoint{2.411332in}{2.100318in}}{\pgfqpoint{2.408060in}{2.108218in}}{\pgfqpoint{2.402236in}{2.114042in}}%
\pgfpathcurveto{\pgfqpoint{2.396412in}{2.119866in}}{\pgfqpoint{2.388512in}{2.123138in}}{\pgfqpoint{2.380276in}{2.123138in}}%
\pgfpathcurveto{\pgfqpoint{2.372039in}{2.123138in}}{\pgfqpoint{2.364139in}{2.119866in}}{\pgfqpoint{2.358315in}{2.114042in}}%
\pgfpathcurveto{\pgfqpoint{2.352492in}{2.108218in}}{\pgfqpoint{2.349219in}{2.100318in}}{\pgfqpoint{2.349219in}{2.092081in}}%
\pgfpathcurveto{\pgfqpoint{2.349219in}{2.083845in}}{\pgfqpoint{2.352492in}{2.075945in}}{\pgfqpoint{2.358315in}{2.070121in}}%
\pgfpathcurveto{\pgfqpoint{2.364139in}{2.064297in}}{\pgfqpoint{2.372039in}{2.061025in}}{\pgfqpoint{2.380276in}{2.061025in}}%
\pgfpathclose%
\pgfusepath{stroke,fill}%
\end{pgfscope}%
\begin{pgfscope}%
\pgfpathrectangle{\pgfqpoint{0.100000in}{0.212622in}}{\pgfqpoint{3.696000in}{3.696000in}}%
\pgfusepath{clip}%
\pgfsetbuttcap%
\pgfsetroundjoin%
\definecolor{currentfill}{rgb}{0.121569,0.466667,0.705882}%
\pgfsetfillcolor{currentfill}%
\pgfsetfillopacity{0.985732}%
\pgfsetlinewidth{1.003750pt}%
\definecolor{currentstroke}{rgb}{0.121569,0.466667,0.705882}%
\pgfsetstrokecolor{currentstroke}%
\pgfsetstrokeopacity{0.985732}%
\pgfsetdash{}{0pt}%
\pgfpathmoveto{\pgfqpoint{2.428350in}{2.066244in}}%
\pgfpathcurveto{\pgfqpoint{2.436586in}{2.066244in}}{\pgfqpoint{2.444486in}{2.069516in}}{\pgfqpoint{2.450310in}{2.075340in}}%
\pgfpathcurveto{\pgfqpoint{2.456134in}{2.081164in}}{\pgfqpoint{2.459406in}{2.089064in}}{\pgfqpoint{2.459406in}{2.097301in}}%
\pgfpathcurveto{\pgfqpoint{2.459406in}{2.105537in}}{\pgfqpoint{2.456134in}{2.113437in}}{\pgfqpoint{2.450310in}{2.119261in}}%
\pgfpathcurveto{\pgfqpoint{2.444486in}{2.125085in}}{\pgfqpoint{2.436586in}{2.128357in}}{\pgfqpoint{2.428350in}{2.128357in}}%
\pgfpathcurveto{\pgfqpoint{2.420113in}{2.128357in}}{\pgfqpoint{2.412213in}{2.125085in}}{\pgfqpoint{2.406389in}{2.119261in}}%
\pgfpathcurveto{\pgfqpoint{2.400566in}{2.113437in}}{\pgfqpoint{2.397293in}{2.105537in}}{\pgfqpoint{2.397293in}{2.097301in}}%
\pgfpathcurveto{\pgfqpoint{2.397293in}{2.089064in}}{\pgfqpoint{2.400566in}{2.081164in}}{\pgfqpoint{2.406389in}{2.075340in}}%
\pgfpathcurveto{\pgfqpoint{2.412213in}{2.069516in}}{\pgfqpoint{2.420113in}{2.066244in}}{\pgfqpoint{2.428350in}{2.066244in}}%
\pgfpathclose%
\pgfusepath{stroke,fill}%
\end{pgfscope}%
\begin{pgfscope}%
\pgfpathrectangle{\pgfqpoint{0.100000in}{0.212622in}}{\pgfqpoint{3.696000in}{3.696000in}}%
\pgfusepath{clip}%
\pgfsetbuttcap%
\pgfsetroundjoin%
\definecolor{currentfill}{rgb}{0.121569,0.466667,0.705882}%
\pgfsetfillcolor{currentfill}%
\pgfsetfillopacity{0.987345}%
\pgfsetlinewidth{1.003750pt}%
\definecolor{currentstroke}{rgb}{0.121569,0.466667,0.705882}%
\pgfsetstrokecolor{currentstroke}%
\pgfsetstrokeopacity{0.987345}%
\pgfsetdash{}{0pt}%
\pgfpathmoveto{\pgfqpoint{2.389357in}{2.054468in}}%
\pgfpathcurveto{\pgfqpoint{2.397594in}{2.054468in}}{\pgfqpoint{2.405494in}{2.057740in}}{\pgfqpoint{2.411318in}{2.063564in}}%
\pgfpathcurveto{\pgfqpoint{2.417141in}{2.069388in}}{\pgfqpoint{2.420414in}{2.077288in}}{\pgfqpoint{2.420414in}{2.085524in}}%
\pgfpathcurveto{\pgfqpoint{2.420414in}{2.093760in}}{\pgfqpoint{2.417141in}{2.101660in}}{\pgfqpoint{2.411318in}{2.107484in}}%
\pgfpathcurveto{\pgfqpoint{2.405494in}{2.113308in}}{\pgfqpoint{2.397594in}{2.116581in}}{\pgfqpoint{2.389357in}{2.116581in}}%
\pgfpathcurveto{\pgfqpoint{2.381121in}{2.116581in}}{\pgfqpoint{2.373221in}{2.113308in}}{\pgfqpoint{2.367397in}{2.107484in}}%
\pgfpathcurveto{\pgfqpoint{2.361573in}{2.101660in}}{\pgfqpoint{2.358301in}{2.093760in}}{\pgfqpoint{2.358301in}{2.085524in}}%
\pgfpathcurveto{\pgfqpoint{2.358301in}{2.077288in}}{\pgfqpoint{2.361573in}{2.069388in}}{\pgfqpoint{2.367397in}{2.063564in}}%
\pgfpathcurveto{\pgfqpoint{2.373221in}{2.057740in}}{\pgfqpoint{2.381121in}{2.054468in}}{\pgfqpoint{2.389357in}{2.054468in}}%
\pgfpathclose%
\pgfusepath{stroke,fill}%
\end{pgfscope}%
\begin{pgfscope}%
\pgfpathrectangle{\pgfqpoint{0.100000in}{0.212622in}}{\pgfqpoint{3.696000in}{3.696000in}}%
\pgfusepath{clip}%
\pgfsetbuttcap%
\pgfsetroundjoin%
\definecolor{currentfill}{rgb}{0.121569,0.466667,0.705882}%
\pgfsetfillcolor{currentfill}%
\pgfsetfillopacity{0.988372}%
\pgfsetlinewidth{1.003750pt}%
\definecolor{currentstroke}{rgb}{0.121569,0.466667,0.705882}%
\pgfsetstrokecolor{currentstroke}%
\pgfsetstrokeopacity{0.988372}%
\pgfsetdash{}{0pt}%
\pgfpathmoveto{\pgfqpoint{2.419122in}{2.058898in}}%
\pgfpathcurveto{\pgfqpoint{2.427359in}{2.058898in}}{\pgfqpoint{2.435259in}{2.062170in}}{\pgfqpoint{2.441083in}{2.067994in}}%
\pgfpathcurveto{\pgfqpoint{2.446907in}{2.073818in}}{\pgfqpoint{2.450179in}{2.081718in}}{\pgfqpoint{2.450179in}{2.089954in}}%
\pgfpathcurveto{\pgfqpoint{2.450179in}{2.098190in}}{\pgfqpoint{2.446907in}{2.106090in}}{\pgfqpoint{2.441083in}{2.111914in}}%
\pgfpathcurveto{\pgfqpoint{2.435259in}{2.117738in}}{\pgfqpoint{2.427359in}{2.121011in}}{\pgfqpoint{2.419122in}{2.121011in}}%
\pgfpathcurveto{\pgfqpoint{2.410886in}{2.121011in}}{\pgfqpoint{2.402986in}{2.117738in}}{\pgfqpoint{2.397162in}{2.111914in}}%
\pgfpathcurveto{\pgfqpoint{2.391338in}{2.106090in}}{\pgfqpoint{2.388066in}{2.098190in}}{\pgfqpoint{2.388066in}{2.089954in}}%
\pgfpathcurveto{\pgfqpoint{2.388066in}{2.081718in}}{\pgfqpoint{2.391338in}{2.073818in}}{\pgfqpoint{2.397162in}{2.067994in}}%
\pgfpathcurveto{\pgfqpoint{2.402986in}{2.062170in}}{\pgfqpoint{2.410886in}{2.058898in}}{\pgfqpoint{2.419122in}{2.058898in}}%
\pgfpathclose%
\pgfusepath{stroke,fill}%
\end{pgfscope}%
\begin{pgfscope}%
\pgfpathrectangle{\pgfqpoint{0.100000in}{0.212622in}}{\pgfqpoint{3.696000in}{3.696000in}}%
\pgfusepath{clip}%
\pgfsetbuttcap%
\pgfsetroundjoin%
\definecolor{currentfill}{rgb}{0.121569,0.466667,0.705882}%
\pgfsetfillcolor{currentfill}%
\pgfsetfillopacity{0.990311}%
\pgfsetlinewidth{1.003750pt}%
\definecolor{currentstroke}{rgb}{0.121569,0.466667,0.705882}%
\pgfsetstrokecolor{currentstroke}%
\pgfsetstrokeopacity{0.990311}%
\pgfsetdash{}{0pt}%
\pgfpathmoveto{\pgfqpoint{2.410454in}{2.054411in}}%
\pgfpathcurveto{\pgfqpoint{2.418690in}{2.054411in}}{\pgfqpoint{2.426590in}{2.057683in}}{\pgfqpoint{2.432414in}{2.063507in}}%
\pgfpathcurveto{\pgfqpoint{2.438238in}{2.069331in}}{\pgfqpoint{2.441510in}{2.077231in}}{\pgfqpoint{2.441510in}{2.085468in}}%
\pgfpathcurveto{\pgfqpoint{2.441510in}{2.093704in}}{\pgfqpoint{2.438238in}{2.101604in}}{\pgfqpoint{2.432414in}{2.107428in}}%
\pgfpathcurveto{\pgfqpoint{2.426590in}{2.113252in}}{\pgfqpoint{2.418690in}{2.116524in}}{\pgfqpoint{2.410454in}{2.116524in}}%
\pgfpathcurveto{\pgfqpoint{2.402218in}{2.116524in}}{\pgfqpoint{2.394318in}{2.113252in}}{\pgfqpoint{2.388494in}{2.107428in}}%
\pgfpathcurveto{\pgfqpoint{2.382670in}{2.101604in}}{\pgfqpoint{2.379397in}{2.093704in}}{\pgfqpoint{2.379397in}{2.085468in}}%
\pgfpathcurveto{\pgfqpoint{2.379397in}{2.077231in}}{\pgfqpoint{2.382670in}{2.069331in}}{\pgfqpoint{2.388494in}{2.063507in}}%
\pgfpathcurveto{\pgfqpoint{2.394318in}{2.057683in}}{\pgfqpoint{2.402218in}{2.054411in}}{\pgfqpoint{2.410454in}{2.054411in}}%
\pgfpathclose%
\pgfusepath{stroke,fill}%
\end{pgfscope}%
\begin{pgfscope}%
\pgfpathrectangle{\pgfqpoint{0.100000in}{0.212622in}}{\pgfqpoint{3.696000in}{3.696000in}}%
\pgfusepath{clip}%
\pgfsetbuttcap%
\pgfsetroundjoin%
\definecolor{currentfill}{rgb}{0.121569,0.466667,0.705882}%
\pgfsetfillcolor{currentfill}%
\pgfsetfillopacity{0.998636}%
\pgfsetlinewidth{1.003750pt}%
\definecolor{currentstroke}{rgb}{0.121569,0.466667,0.705882}%
\pgfsetstrokecolor{currentstroke}%
\pgfsetstrokeopacity{0.998636}%
\pgfsetdash{}{0pt}%
\pgfpathmoveto{\pgfqpoint{2.376462in}{2.035915in}}%
\pgfpathcurveto{\pgfqpoint{2.384698in}{2.035915in}}{\pgfqpoint{2.392598in}{2.039187in}}{\pgfqpoint{2.398422in}{2.045011in}}%
\pgfpathcurveto{\pgfqpoint{2.404246in}{2.050835in}}{\pgfqpoint{2.407518in}{2.058735in}}{\pgfqpoint{2.407518in}{2.066971in}}%
\pgfpathcurveto{\pgfqpoint{2.407518in}{2.075208in}}{\pgfqpoint{2.404246in}{2.083108in}}{\pgfqpoint{2.398422in}{2.088931in}}%
\pgfpathcurveto{\pgfqpoint{2.392598in}{2.094755in}}{\pgfqpoint{2.384698in}{2.098028in}}{\pgfqpoint{2.376462in}{2.098028in}}%
\pgfpathcurveto{\pgfqpoint{2.368226in}{2.098028in}}{\pgfqpoint{2.360325in}{2.094755in}}{\pgfqpoint{2.354502in}{2.088931in}}%
\pgfpathcurveto{\pgfqpoint{2.348678in}{2.083108in}}{\pgfqpoint{2.345405in}{2.075208in}}{\pgfqpoint{2.345405in}{2.066971in}}%
\pgfpathcurveto{\pgfqpoint{2.345405in}{2.058735in}}{\pgfqpoint{2.348678in}{2.050835in}}{\pgfqpoint{2.354502in}{2.045011in}}%
\pgfpathcurveto{\pgfqpoint{2.360325in}{2.039187in}}{\pgfqpoint{2.368226in}{2.035915in}}{\pgfqpoint{2.376462in}{2.035915in}}%
\pgfpathclose%
\pgfusepath{stroke,fill}%
\end{pgfscope}%
\begin{pgfscope}%
\pgfpathrectangle{\pgfqpoint{0.100000in}{0.212622in}}{\pgfqpoint{3.696000in}{3.696000in}}%
\pgfusepath{clip}%
\pgfsetbuttcap%
\pgfsetroundjoin%
\definecolor{currentfill}{rgb}{0.121569,0.466667,0.705882}%
\pgfsetfillcolor{currentfill}%
\pgfsetfillopacity{0.998643}%
\pgfsetlinewidth{1.003750pt}%
\definecolor{currentstroke}{rgb}{0.121569,0.466667,0.705882}%
\pgfsetstrokecolor{currentstroke}%
\pgfsetstrokeopacity{0.998643}%
\pgfsetdash{}{0pt}%
\pgfpathmoveto{\pgfqpoint{2.380436in}{2.029230in}}%
\pgfpathcurveto{\pgfqpoint{2.388672in}{2.029230in}}{\pgfqpoint{2.396572in}{2.032502in}}{\pgfqpoint{2.402396in}{2.038326in}}%
\pgfpathcurveto{\pgfqpoint{2.408220in}{2.044150in}}{\pgfqpoint{2.411492in}{2.052050in}}{\pgfqpoint{2.411492in}{2.060286in}}%
\pgfpathcurveto{\pgfqpoint{2.411492in}{2.068523in}}{\pgfqpoint{2.408220in}{2.076423in}}{\pgfqpoint{2.402396in}{2.082247in}}%
\pgfpathcurveto{\pgfqpoint{2.396572in}{2.088071in}}{\pgfqpoint{2.388672in}{2.091343in}}{\pgfqpoint{2.380436in}{2.091343in}}%
\pgfpathcurveto{\pgfqpoint{2.372200in}{2.091343in}}{\pgfqpoint{2.364300in}{2.088071in}}{\pgfqpoint{2.358476in}{2.082247in}}%
\pgfpathcurveto{\pgfqpoint{2.352652in}{2.076423in}}{\pgfqpoint{2.349379in}{2.068523in}}{\pgfqpoint{2.349379in}{2.060286in}}%
\pgfpathcurveto{\pgfqpoint{2.349379in}{2.052050in}}{\pgfqpoint{2.352652in}{2.044150in}}{\pgfqpoint{2.358476in}{2.038326in}}%
\pgfpathcurveto{\pgfqpoint{2.364300in}{2.032502in}}{\pgfqpoint{2.372200in}{2.029230in}}{\pgfqpoint{2.380436in}{2.029230in}}%
\pgfpathclose%
\pgfusepath{stroke,fill}%
\end{pgfscope}%
\begin{pgfscope}%
\pgfpathrectangle{\pgfqpoint{0.100000in}{0.212622in}}{\pgfqpoint{3.696000in}{3.696000in}}%
\pgfusepath{clip}%
\pgfsetbuttcap%
\pgfsetroundjoin%
\definecolor{currentfill}{rgb}{0.121569,0.466667,0.705882}%
\pgfsetfillcolor{currentfill}%
\pgfsetlinewidth{1.003750pt}%
\definecolor{currentstroke}{rgb}{0.121569,0.466667,0.705882}%
\pgfsetstrokecolor{currentstroke}%
\pgfsetdash{}{0pt}%
\pgfpathmoveto{\pgfqpoint{2.365935in}{2.026639in}}%
\pgfpathcurveto{\pgfqpoint{2.374171in}{2.026639in}}{\pgfqpoint{2.382071in}{2.029911in}}{\pgfqpoint{2.387895in}{2.035735in}}%
\pgfpathcurveto{\pgfqpoint{2.393719in}{2.041559in}}{\pgfqpoint{2.396991in}{2.049459in}}{\pgfqpoint{2.396991in}{2.057695in}}%
\pgfpathcurveto{\pgfqpoint{2.396991in}{2.065932in}}{\pgfqpoint{2.393719in}{2.073832in}}{\pgfqpoint{2.387895in}{2.079656in}}%
\pgfpathcurveto{\pgfqpoint{2.382071in}{2.085480in}}{\pgfqpoint{2.374171in}{2.088752in}}{\pgfqpoint{2.365935in}{2.088752in}}%
\pgfpathcurveto{\pgfqpoint{2.357698in}{2.088752in}}{\pgfqpoint{2.349798in}{2.085480in}}{\pgfqpoint{2.343974in}{2.079656in}}%
\pgfpathcurveto{\pgfqpoint{2.338151in}{2.073832in}}{\pgfqpoint{2.334878in}{2.065932in}}{\pgfqpoint{2.334878in}{2.057695in}}%
\pgfpathcurveto{\pgfqpoint{2.334878in}{2.049459in}}{\pgfqpoint{2.338151in}{2.041559in}}{\pgfqpoint{2.343974in}{2.035735in}}%
\pgfpathcurveto{\pgfqpoint{2.349798in}{2.029911in}}{\pgfqpoint{2.357698in}{2.026639in}}{\pgfqpoint{2.365935in}{2.026639in}}%
\pgfpathclose%
\pgfusepath{stroke,fill}%
\end{pgfscope}%
\begin{pgfscope}%
\pgfsetbuttcap%
\pgfsetmiterjoin%
\definecolor{currentfill}{rgb}{1.000000,1.000000,1.000000}%
\pgfsetfillcolor{currentfill}%
\pgfsetfillopacity{0.800000}%
\pgfsetlinewidth{1.003750pt}%
\definecolor{currentstroke}{rgb}{0.800000,0.800000,0.800000}%
\pgfsetstrokecolor{currentstroke}%
\pgfsetstrokeopacity{0.800000}%
\pgfsetdash{}{0pt}%
\pgfpathmoveto{\pgfqpoint{2.104889in}{3.410289in}}%
\pgfpathlineto{\pgfqpoint{3.698778in}{3.410289in}}%
\pgfpathquadraticcurveto{\pgfqpoint{3.726556in}{3.410289in}}{\pgfqpoint{3.726556in}{3.438067in}}%
\pgfpathlineto{\pgfqpoint{3.726556in}{3.811400in}}%
\pgfpathquadraticcurveto{\pgfqpoint{3.726556in}{3.839178in}}{\pgfqpoint{3.698778in}{3.839178in}}%
\pgfpathlineto{\pgfqpoint{2.104889in}{3.839178in}}%
\pgfpathquadraticcurveto{\pgfqpoint{2.077111in}{3.839178in}}{\pgfqpoint{2.077111in}{3.811400in}}%
\pgfpathlineto{\pgfqpoint{2.077111in}{3.438067in}}%
\pgfpathquadraticcurveto{\pgfqpoint{2.077111in}{3.410289in}}{\pgfqpoint{2.104889in}{3.410289in}}%
\pgfpathclose%
\pgfusepath{stroke,fill}%
\end{pgfscope}%
\begin{pgfscope}%
\pgfsetrectcap%
\pgfsetroundjoin%
\pgfsetlinewidth{1.505625pt}%
\definecolor{currentstroke}{rgb}{0.121569,0.466667,0.705882}%
\pgfsetstrokecolor{currentstroke}%
\pgfsetdash{}{0pt}%
\pgfpathmoveto{\pgfqpoint{2.132667in}{3.735011in}}%
\pgfpathlineto{\pgfqpoint{2.410444in}{3.735011in}}%
\pgfusepath{stroke}%
\end{pgfscope}%
\begin{pgfscope}%
\definecolor{textcolor}{rgb}{0.000000,0.000000,0.000000}%
\pgfsetstrokecolor{textcolor}%
\pgfsetfillcolor{textcolor}%
\pgftext[x=2.521555in,y=3.686400in,left,base]{\color{textcolor}\rmfamily\fontsize{10.000000}{12.000000}\selectfont Ground truth}%
\end{pgfscope}%
\begin{pgfscope}%
\pgfsetbuttcap%
\pgfsetroundjoin%
\definecolor{currentfill}{rgb}{0.121569,0.466667,0.705882}%
\pgfsetfillcolor{currentfill}%
\pgfsetlinewidth{1.003750pt}%
\definecolor{currentstroke}{rgb}{0.121569,0.466667,0.705882}%
\pgfsetstrokecolor{currentstroke}%
\pgfsetdash{}{0pt}%
\pgfsys@defobject{currentmarker}{\pgfqpoint{-0.031056in}{-0.031056in}}{\pgfqpoint{0.031056in}{0.031056in}}{%
\pgfpathmoveto{\pgfqpoint{0.000000in}{-0.031056in}}%
\pgfpathcurveto{\pgfqpoint{0.008236in}{-0.031056in}}{\pgfqpoint{0.016136in}{-0.027784in}}{\pgfqpoint{0.021960in}{-0.021960in}}%
\pgfpathcurveto{\pgfqpoint{0.027784in}{-0.016136in}}{\pgfqpoint{0.031056in}{-0.008236in}}{\pgfqpoint{0.031056in}{0.000000in}}%
\pgfpathcurveto{\pgfqpoint{0.031056in}{0.008236in}}{\pgfqpoint{0.027784in}{0.016136in}}{\pgfqpoint{0.021960in}{0.021960in}}%
\pgfpathcurveto{\pgfqpoint{0.016136in}{0.027784in}}{\pgfqpoint{0.008236in}{0.031056in}}{\pgfqpoint{0.000000in}{0.031056in}}%
\pgfpathcurveto{\pgfqpoint{-0.008236in}{0.031056in}}{\pgfqpoint{-0.016136in}{0.027784in}}{\pgfqpoint{-0.021960in}{0.021960in}}%
\pgfpathcurveto{\pgfqpoint{-0.027784in}{0.016136in}}{\pgfqpoint{-0.031056in}{0.008236in}}{\pgfqpoint{-0.031056in}{0.000000in}}%
\pgfpathcurveto{\pgfqpoint{-0.031056in}{-0.008236in}}{\pgfqpoint{-0.027784in}{-0.016136in}}{\pgfqpoint{-0.021960in}{-0.021960in}}%
\pgfpathcurveto{\pgfqpoint{-0.016136in}{-0.027784in}}{\pgfqpoint{-0.008236in}{-0.031056in}}{\pgfqpoint{0.000000in}{-0.031056in}}%
\pgfpathclose%
\pgfusepath{stroke,fill}%
}%
\begin{pgfscope}%
\pgfsys@transformshift{2.271555in}{3.529248in}%
\pgfsys@useobject{currentmarker}{}%
\end{pgfscope}%
\end{pgfscope}%
\begin{pgfscope}%
\definecolor{textcolor}{rgb}{0.000000,0.000000,0.000000}%
\pgfsetstrokecolor{textcolor}%
\pgfsetfillcolor{textcolor}%
\pgftext[x=2.521555in,y=3.492789in,left,base]{\color{textcolor}\rmfamily\fontsize{10.000000}{12.000000}\selectfont Estimated position}%
\end{pgfscope}%
\end{pgfpicture}%
\makeatother%
\endgroup%
}
%         \caption{SAAM's 3D position estimation had the lowest turn error for the 4-meter line experiment.}
%         \label{fig:line28_3D}
%     \end{subfigure}
%     \caption{Position estimation by the best performing algorithms in the 28-meter line experiment.}
%     \label{fig:line28}
% \end{figure}

% \subsection{Triangle}

% The triangle shape consisted of moving the inertial system in a triangle patter for a determined distance. 3 distances were tested: 4, 16, and 28 meter. The results are shown below:

% \subsubsection{4 meter}

% For the 4-meter triangle experiment, the ROLEQ algorithm which had the lowest displacement error with an average of 0.78 meters (6.54\% of error margin), and EKF with an average of 0.49 meters of turn error (4.09\% of error margin).

% \begin{figure}[!h]
%     \centering
%     \begin{table}[H]
    \begin{center}
        \resizebox{1\linewidth}{!}{

            \begin{tabular}[t]{lcccc}
                \hline
                Algorithm     & Displacement Error[$m$] & Displacement Error[\%] & Turn Error[$m$] & Turn Error[\%] \\
                \hline
                AngularRate   & 4.53                    & 37.72                  & 7.35            & 61.28          \\
                AQUA          & 3.10                    & 25.82                  & 4.27            & 35.61          \\
                Complementary & 3.85                    & 32.08                  & 3.43            & 28.59          \\
                Davenport     & 1.12                    & 9.30                   & 1.39            & 11.57          \\
                EKF           & 1.24                    & 10.31                  & 0.49            & 4.09           \\
                FAMC          & 4.43                    & 36.90                  & 7.12            & 59.37          \\
                FLAE          & 1.09                    & 9.08                   & 1.32            & 11.00          \\
                Fourati       & 7.36                    & 61.33                  & 8.47            & 70.60          \\
                Madgwick      & 1.94                    & 16.15                  & 1.02            & 8.51           \\
                Mahony        & 0.90                    & 7.50                   & 0.99            & 8.28           \\
                OLEQ          & 1.18                    & 9.81                   & 1.03            & 8.58           \\
                QUEST         & 2.84                    & 23.65                  & 2.58            & 21.50          \\
                ROLEQ         & 1.17                    & 9.71                   & 0.85            & 7.06           \\
                SAAM          & 1.05                    & 8.71                   & 1.08            & 9.01           \\
                Tilt          & 1.05                    & 8.71                   & 1.08            & 9.01           \\

                \hline
                Average       & 2.45                    & 20.45                  & 2.83            & 23.60
            \end{tabular}
        }
        \caption{4 meter triangle position estimation error (displacement and turn) of the sensor fusion algorithms. }
        \label{tab:4_triangle}
    \end{center}
\end{table}
% \end{figure}

% \begin{figure}[!h]
%     \centering
%     \begin{subfigure}{0.49\textwidth}
%         \centering
%         \resizebox{1\linewidth}{!}{%% Creator: Matplotlib, PGF backend
%%
%% To include the figure in your LaTeX document, write
%%   \input{<filename>.pgf}
%%
%% Make sure the required packages are loaded in your preamble
%%   \usepackage{pgf}
%%
%% and, on pdftex
%%   \usepackage[utf8]{inputenc}\DeclareUnicodeCharacter{2212}{-}
%%
%% or, on luatex and xetex
%%   \usepackage{unicode-math}
%%
%% Figures using additional raster images can only be included by \input if
%% they are in the same directory as the main LaTeX file. For loading figures
%% from other directories you can use the `import` package
%%   \usepackage{import}
%%
%% and then include the figures with
%%   \import{<path to file>}{<filename>.pgf}
%%
%% Matplotlib used the following preamble
%%   \usepackage{fontspec}
%%
\begingroup%
\makeatletter%
\begin{pgfpicture}%
\pgfpathrectangle{\pgfpointorigin}{\pgfqpoint{5.629167in}{4.311000in}}%
\pgfusepath{use as bounding box, clip}%
\begin{pgfscope}%
\pgfsetbuttcap%
\pgfsetmiterjoin%
\definecolor{currentfill}{rgb}{1.000000,1.000000,1.000000}%
\pgfsetfillcolor{currentfill}%
\pgfsetlinewidth{0.000000pt}%
\definecolor{currentstroke}{rgb}{1.000000,1.000000,1.000000}%
\pgfsetstrokecolor{currentstroke}%
\pgfsetdash{}{0pt}%
\pgfpathmoveto{\pgfqpoint{0.000000in}{0.000000in}}%
\pgfpathlineto{\pgfqpoint{5.629167in}{0.000000in}}%
\pgfpathlineto{\pgfqpoint{5.629167in}{4.311000in}}%
\pgfpathlineto{\pgfqpoint{0.000000in}{4.311000in}}%
\pgfpathclose%
\pgfusepath{fill}%
\end{pgfscope}%
\begin{pgfscope}%
\pgfsetbuttcap%
\pgfsetmiterjoin%
\definecolor{currentfill}{rgb}{1.000000,1.000000,1.000000}%
\pgfsetfillcolor{currentfill}%
\pgfsetlinewidth{0.000000pt}%
\definecolor{currentstroke}{rgb}{0.000000,0.000000,0.000000}%
\pgfsetstrokecolor{currentstroke}%
\pgfsetstrokeopacity{0.000000}%
\pgfsetdash{}{0pt}%
\pgfpathmoveto{\pgfqpoint{0.569167in}{0.515000in}}%
\pgfpathlineto{\pgfqpoint{5.529167in}{0.515000in}}%
\pgfpathlineto{\pgfqpoint{5.529167in}{4.211000in}}%
\pgfpathlineto{\pgfqpoint{0.569167in}{4.211000in}}%
\pgfpathclose%
\pgfusepath{fill}%
\end{pgfscope}%
\begin{pgfscope}%
\pgfpathrectangle{\pgfqpoint{0.569167in}{0.515000in}}{\pgfqpoint{4.960000in}{3.696000in}}%
\pgfusepath{clip}%
\pgfsetbuttcap%
\pgfsetroundjoin%
\definecolor{currentfill}{rgb}{0.121569,0.466667,0.705882}%
\pgfsetfillcolor{currentfill}%
\pgfsetlinewidth{1.003750pt}%
\definecolor{currentstroke}{rgb}{0.121569,0.466667,0.705882}%
\pgfsetstrokecolor{currentstroke}%
\pgfsetdash{}{0pt}%
\pgfsys@defobject{currentmarker}{\pgfqpoint{-0.041667in}{-0.041667in}}{\pgfqpoint{0.041667in}{0.041667in}}{%
\pgfpathmoveto{\pgfqpoint{0.000000in}{-0.041667in}}%
\pgfpathcurveto{\pgfqpoint{0.011050in}{-0.041667in}}{\pgfqpoint{0.021649in}{-0.037276in}}{\pgfqpoint{0.029463in}{-0.029463in}}%
\pgfpathcurveto{\pgfqpoint{0.037276in}{-0.021649in}}{\pgfqpoint{0.041667in}{-0.011050in}}{\pgfqpoint{0.041667in}{0.000000in}}%
\pgfpathcurveto{\pgfqpoint{0.041667in}{0.011050in}}{\pgfqpoint{0.037276in}{0.021649in}}{\pgfqpoint{0.029463in}{0.029463in}}%
\pgfpathcurveto{\pgfqpoint{0.021649in}{0.037276in}}{\pgfqpoint{0.011050in}{0.041667in}}{\pgfqpoint{0.000000in}{0.041667in}}%
\pgfpathcurveto{\pgfqpoint{-0.011050in}{0.041667in}}{\pgfqpoint{-0.021649in}{0.037276in}}{\pgfqpoint{-0.029463in}{0.029463in}}%
\pgfpathcurveto{\pgfqpoint{-0.037276in}{0.021649in}}{\pgfqpoint{-0.041667in}{0.011050in}}{\pgfqpoint{-0.041667in}{0.000000in}}%
\pgfpathcurveto{\pgfqpoint{-0.041667in}{-0.011050in}}{\pgfqpoint{-0.037276in}{-0.021649in}}{\pgfqpoint{-0.029463in}{-0.029463in}}%
\pgfpathcurveto{\pgfqpoint{-0.021649in}{-0.037276in}}{\pgfqpoint{-0.011050in}{-0.041667in}}{\pgfqpoint{0.000000in}{-0.041667in}}%
\pgfpathclose%
\pgfusepath{stroke,fill}%
}%
\begin{pgfscope}%
\pgfsys@transformshift{1.922680in}{1.493534in}%
\pgfsys@useobject{currentmarker}{}%
\end{pgfscope}%
\begin{pgfscope}%
\pgfsys@transformshift{1.923376in}{1.479705in}%
\pgfsys@useobject{currentmarker}{}%
\end{pgfscope}%
\begin{pgfscope}%
\pgfsys@transformshift{1.924822in}{1.455994in}%
\pgfsys@useobject{currentmarker}{}%
\end{pgfscope}%
\begin{pgfscope}%
\pgfsys@transformshift{1.927766in}{1.424039in}%
\pgfsys@useobject{currentmarker}{}%
\end{pgfscope}%
\begin{pgfscope}%
\pgfsys@transformshift{1.928625in}{1.385533in}%
\pgfsys@useobject{currentmarker}{}%
\end{pgfscope}%
\begin{pgfscope}%
\pgfsys@transformshift{1.930503in}{1.343472in}%
\pgfsys@useobject{currentmarker}{}%
\end{pgfscope}%
\begin{pgfscope}%
\pgfsys@transformshift{1.931221in}{1.320325in}%
\pgfsys@useobject{currentmarker}{}%
\end{pgfscope}%
\begin{pgfscope}%
\pgfsys@transformshift{1.930588in}{1.307605in}%
\pgfsys@useobject{currentmarker}{}%
\end{pgfscope}%
\begin{pgfscope}%
\pgfsys@transformshift{1.931149in}{1.300622in}%
\pgfsys@useobject{currentmarker}{}%
\end{pgfscope}%
\begin{pgfscope}%
\pgfsys@transformshift{1.931181in}{1.296770in}%
\pgfsys@useobject{currentmarker}{}%
\end{pgfscope}%
\begin{pgfscope}%
\pgfsys@transformshift{1.931238in}{1.294651in}%
\pgfsys@useobject{currentmarker}{}%
\end{pgfscope}%
\begin{pgfscope}%
\pgfsys@transformshift{1.931394in}{1.293496in}%
\pgfsys@useobject{currentmarker}{}%
\end{pgfscope}%
\begin{pgfscope}%
\pgfsys@transformshift{1.931409in}{1.292856in}%
\pgfsys@useobject{currentmarker}{}%
\end{pgfscope}%
\begin{pgfscope}%
\pgfsys@transformshift{1.931414in}{1.292503in}%
\pgfsys@useobject{currentmarker}{}%
\end{pgfscope}%
\begin{pgfscope}%
\pgfsys@transformshift{1.931417in}{1.292309in}%
\pgfsys@useobject{currentmarker}{}%
\end{pgfscope}%
\begin{pgfscope}%
\pgfsys@transformshift{1.931425in}{1.292203in}%
\pgfsys@useobject{currentmarker}{}%
\end{pgfscope}%
\begin{pgfscope}%
\pgfsys@transformshift{1.931428in}{1.292144in}%
\pgfsys@useobject{currentmarker}{}%
\end{pgfscope}%
\begin{pgfscope}%
\pgfsys@transformshift{1.931428in}{1.292112in}%
\pgfsys@useobject{currentmarker}{}%
\end{pgfscope}%
\begin{pgfscope}%
\pgfsys@transformshift{1.931429in}{1.292094in}%
\pgfsys@useobject{currentmarker}{}%
\end{pgfscope}%
\begin{pgfscope}%
\pgfsys@transformshift{1.931429in}{1.292085in}%
\pgfsys@useobject{currentmarker}{}%
\end{pgfscope}%
\begin{pgfscope}%
\pgfsys@transformshift{1.931430in}{1.292079in}%
\pgfsys@useobject{currentmarker}{}%
\end{pgfscope}%
\begin{pgfscope}%
\pgfsys@transformshift{1.931429in}{1.292076in}%
\pgfsys@useobject{currentmarker}{}%
\end{pgfscope}%
\begin{pgfscope}%
\pgfsys@transformshift{1.931753in}{1.289058in}%
\pgfsys@useobject{currentmarker}{}%
\end{pgfscope}%
\begin{pgfscope}%
\pgfsys@transformshift{1.931694in}{1.287390in}%
\pgfsys@useobject{currentmarker}{}%
\end{pgfscope}%
\begin{pgfscope}%
\pgfsys@transformshift{1.931774in}{1.286475in}%
\pgfsys@useobject{currentmarker}{}%
\end{pgfscope}%
\begin{pgfscope}%
\pgfsys@transformshift{1.931817in}{1.285972in}%
\pgfsys@useobject{currentmarker}{}%
\end{pgfscope}%
\begin{pgfscope}%
\pgfsys@transformshift{1.931821in}{1.285694in}%
\pgfsys@useobject{currentmarker}{}%
\end{pgfscope}%
\begin{pgfscope}%
\pgfsys@transformshift{1.931833in}{1.285542in}%
\pgfsys@useobject{currentmarker}{}%
\end{pgfscope}%
\begin{pgfscope}%
\pgfsys@transformshift{1.931835in}{1.285458in}%
\pgfsys@useobject{currentmarker}{}%
\end{pgfscope}%
\begin{pgfscope}%
\pgfsys@transformshift{1.931836in}{1.285412in}%
\pgfsys@useobject{currentmarker}{}%
\end{pgfscope}%
\begin{pgfscope}%
\pgfsys@transformshift{1.931839in}{1.285387in}%
\pgfsys@useobject{currentmarker}{}%
\end{pgfscope}%
\begin{pgfscope}%
\pgfsys@transformshift{1.932028in}{1.282129in}%
\pgfsys@useobject{currentmarker}{}%
\end{pgfscope}%
\begin{pgfscope}%
\pgfsys@transformshift{1.932323in}{1.280358in}%
\pgfsys@useobject{currentmarker}{}%
\end{pgfscope}%
\begin{pgfscope}%
\pgfsys@transformshift{1.932514in}{1.279390in}%
\pgfsys@useobject{currentmarker}{}%
\end{pgfscope}%
\begin{pgfscope}%
\pgfsys@transformshift{1.932635in}{1.278860in}%
\pgfsys@useobject{currentmarker}{}%
\end{pgfscope}%
\begin{pgfscope}%
\pgfsys@transformshift{1.932751in}{1.278585in}%
\pgfsys@useobject{currentmarker}{}%
\end{pgfscope}%
\begin{pgfscope}%
\pgfsys@transformshift{1.932839in}{1.278446in}%
\pgfsys@useobject{currentmarker}{}%
\end{pgfscope}%
\begin{pgfscope}%
\pgfsys@transformshift{1.932907in}{1.278386in}%
\pgfsys@useobject{currentmarker}{}%
\end{pgfscope}%
\begin{pgfscope}%
\pgfsys@transformshift{1.932950in}{1.278362in}%
\pgfsys@useobject{currentmarker}{}%
\end{pgfscope}%
\begin{pgfscope}%
\pgfsys@transformshift{1.938616in}{1.276051in}%
\pgfsys@useobject{currentmarker}{}%
\end{pgfscope}%
\begin{pgfscope}%
\pgfsys@transformshift{1.941838in}{1.275078in}%
\pgfsys@useobject{currentmarker}{}%
\end{pgfscope}%
\begin{pgfscope}%
\pgfsys@transformshift{1.948790in}{1.274742in}%
\pgfsys@useobject{currentmarker}{}%
\end{pgfscope}%
\begin{pgfscope}%
\pgfsys@transformshift{1.962899in}{1.276779in}%
\pgfsys@useobject{currentmarker}{}%
\end{pgfscope}%
\begin{pgfscope}%
\pgfsys@transformshift{1.970064in}{1.279962in}%
\pgfsys@useobject{currentmarker}{}%
\end{pgfscope}%
\begin{pgfscope}%
\pgfsys@transformshift{1.973766in}{1.282173in}%
\pgfsys@useobject{currentmarker}{}%
\end{pgfscope}%
\begin{pgfscope}%
\pgfsys@transformshift{1.985697in}{1.298250in}%
\pgfsys@useobject{currentmarker}{}%
\end{pgfscope}%
\begin{pgfscope}%
\pgfsys@transformshift{2.000362in}{1.316983in}%
\pgfsys@useobject{currentmarker}{}%
\end{pgfscope}%
\begin{pgfscope}%
\pgfsys@transformshift{2.006081in}{1.346204in}%
\pgfsys@useobject{currentmarker}{}%
\end{pgfscope}%
\begin{pgfscope}%
\pgfsys@transformshift{2.018143in}{1.381124in}%
\pgfsys@useobject{currentmarker}{}%
\end{pgfscope}%
\begin{pgfscope}%
\pgfsys@transformshift{2.020036in}{1.426981in}%
\pgfsys@useobject{currentmarker}{}%
\end{pgfscope}%
\begin{pgfscope}%
\pgfsys@transformshift{2.025380in}{1.451651in}%
\pgfsys@useobject{currentmarker}{}%
\end{pgfscope}%
\begin{pgfscope}%
\pgfsys@transformshift{2.024485in}{1.484477in}%
\pgfsys@useobject{currentmarker}{}%
\end{pgfscope}%
\begin{pgfscope}%
\pgfsys@transformshift{2.026143in}{1.502462in}%
\pgfsys@useobject{currentmarker}{}%
\end{pgfscope}%
\begin{pgfscope}%
\pgfsys@transformshift{2.024376in}{1.523989in}%
\pgfsys@useobject{currentmarker}{}%
\end{pgfscope}%
\begin{pgfscope}%
\pgfsys@transformshift{2.025664in}{1.548469in}%
\pgfsys@useobject{currentmarker}{}%
\end{pgfscope}%
\begin{pgfscope}%
\pgfsys@transformshift{2.025046in}{1.561938in}%
\pgfsys@useobject{currentmarker}{}%
\end{pgfscope}%
\begin{pgfscope}%
\pgfsys@transformshift{2.025519in}{1.569338in}%
\pgfsys@useobject{currentmarker}{}%
\end{pgfscope}%
\begin{pgfscope}%
\pgfsys@transformshift{2.025364in}{1.573414in}%
\pgfsys@useobject{currentmarker}{}%
\end{pgfscope}%
\begin{pgfscope}%
\pgfsys@transformshift{2.025550in}{1.575649in}%
\pgfsys@useobject{currentmarker}{}%
\end{pgfscope}%
\begin{pgfscope}%
\pgfsys@transformshift{2.025444in}{1.576878in}%
\pgfsys@useobject{currentmarker}{}%
\end{pgfscope}%
\begin{pgfscope}%
\pgfsys@transformshift{2.025959in}{1.581064in}%
\pgfsys@useobject{currentmarker}{}%
\end{pgfscope}%
\begin{pgfscope}%
\pgfsys@transformshift{2.025815in}{1.583379in}%
\pgfsys@useobject{currentmarker}{}%
\end{pgfscope}%
\begin{pgfscope}%
\pgfsys@transformshift{2.026035in}{1.584637in}%
\pgfsys@useobject{currentmarker}{}%
\end{pgfscope}%
\begin{pgfscope}%
\pgfsys@transformshift{2.026031in}{1.585339in}%
\pgfsys@useobject{currentmarker}{}%
\end{pgfscope}%
\begin{pgfscope}%
\pgfsys@transformshift{2.026091in}{1.585720in}%
\pgfsys@useobject{currentmarker}{}%
\end{pgfscope}%
\begin{pgfscope}%
\pgfsys@transformshift{2.025984in}{1.590493in}%
\pgfsys@useobject{currentmarker}{}%
\end{pgfscope}%
\begin{pgfscope}%
\pgfsys@transformshift{2.026314in}{1.593098in}%
\pgfsys@useobject{currentmarker}{}%
\end{pgfscope}%
\begin{pgfscope}%
\pgfsys@transformshift{2.026220in}{1.594539in}%
\pgfsys@useobject{currentmarker}{}%
\end{pgfscope}%
\begin{pgfscope}%
\pgfsys@transformshift{2.027181in}{1.599420in}%
\pgfsys@useobject{currentmarker}{}%
\end{pgfscope}%
\begin{pgfscope}%
\pgfsys@transformshift{2.025934in}{1.607682in}%
\pgfsys@useobject{currentmarker}{}%
\end{pgfscope}%
\begin{pgfscope}%
\pgfsys@transformshift{2.034668in}{1.630516in}%
\pgfsys@useobject{currentmarker}{}%
\end{pgfscope}%
\begin{pgfscope}%
\pgfsys@transformshift{2.032561in}{1.643796in}%
\pgfsys@useobject{currentmarker}{}%
\end{pgfscope}%
\begin{pgfscope}%
\pgfsys@transformshift{2.035004in}{1.650776in}%
\pgfsys@useobject{currentmarker}{}%
\end{pgfscope}%
\begin{pgfscope}%
\pgfsys@transformshift{2.030785in}{1.674677in}%
\pgfsys@useobject{currentmarker}{}%
\end{pgfscope}%
\begin{pgfscope}%
\pgfsys@transformshift{2.042678in}{1.706093in}%
\pgfsys@useobject{currentmarker}{}%
\end{pgfscope}%
\begin{pgfscope}%
\pgfsys@transformshift{2.035800in}{1.758838in}%
\pgfsys@useobject{currentmarker}{}%
\end{pgfscope}%
\begin{pgfscope}%
\pgfsys@transformshift{2.054514in}{1.819555in}%
\pgfsys@useobject{currentmarker}{}%
\end{pgfscope}%
\begin{pgfscope}%
\pgfsys@transformshift{2.050221in}{1.854236in}%
\pgfsys@useobject{currentmarker}{}%
\end{pgfscope}%
\begin{pgfscope}%
\pgfsys@transformshift{2.056581in}{1.872373in}%
\pgfsys@useobject{currentmarker}{}%
\end{pgfscope}%
\begin{pgfscope}%
\pgfsys@transformshift{2.051157in}{1.894194in}%
\pgfsys@useobject{currentmarker}{}%
\end{pgfscope}%
\begin{pgfscope}%
\pgfsys@transformshift{2.058909in}{1.919509in}%
\pgfsys@useobject{currentmarker}{}%
\end{pgfscope}%
\begin{pgfscope}%
\pgfsys@transformshift{2.055482in}{1.933662in}%
\pgfsys@useobject{currentmarker}{}%
\end{pgfscope}%
\begin{pgfscope}%
\pgfsys@transformshift{2.061859in}{1.954300in}%
\pgfsys@useobject{currentmarker}{}%
\end{pgfscope}%
\begin{pgfscope}%
\pgfsys@transformshift{2.057220in}{1.980145in}%
\pgfsys@useobject{currentmarker}{}%
\end{pgfscope}%
\begin{pgfscope}%
\pgfsys@transformshift{2.065499in}{2.011267in}%
\pgfsys@useobject{currentmarker}{}%
\end{pgfscope}%
\begin{pgfscope}%
\pgfsys@transformshift{2.054476in}{2.048048in}%
\pgfsys@useobject{currentmarker}{}%
\end{pgfscope}%
\begin{pgfscope}%
\pgfsys@transformshift{2.064097in}{2.090109in}%
\pgfsys@useobject{currentmarker}{}%
\end{pgfscope}%
\begin{pgfscope}%
\pgfsys@transformshift{2.051380in}{2.143002in}%
\pgfsys@useobject{currentmarker}{}%
\end{pgfscope}%
\begin{pgfscope}%
\pgfsys@transformshift{2.071541in}{2.201848in}%
\pgfsys@useobject{currentmarker}{}%
\end{pgfscope}%
\begin{pgfscope}%
\pgfsys@transformshift{2.064530in}{2.235219in}%
\pgfsys@useobject{currentmarker}{}%
\end{pgfscope}%
\begin{pgfscope}%
\pgfsys@transformshift{2.079210in}{2.272102in}%
\pgfsys@useobject{currentmarker}{}%
\end{pgfscope}%
\begin{pgfscope}%
\pgfsys@transformshift{2.075298in}{2.293576in}%
\pgfsys@useobject{currentmarker}{}%
\end{pgfscope}%
\begin{pgfscope}%
\pgfsys@transformshift{2.079746in}{2.304709in}%
\pgfsys@useobject{currentmarker}{}%
\end{pgfscope}%
\begin{pgfscope}%
\pgfsys@transformshift{2.075243in}{2.325431in}%
\pgfsys@useobject{currentmarker}{}%
\end{pgfscope}%
\begin{pgfscope}%
\pgfsys@transformshift{2.086307in}{2.355103in}%
\pgfsys@useobject{currentmarker}{}%
\end{pgfscope}%
\begin{pgfscope}%
\pgfsys@transformshift{2.080273in}{2.395083in}%
\pgfsys@useobject{currentmarker}{}%
\end{pgfscope}%
\begin{pgfscope}%
\pgfsys@transformshift{2.099394in}{2.439983in}%
\pgfsys@useobject{currentmarker}{}%
\end{pgfscope}%
\begin{pgfscope}%
\pgfsys@transformshift{2.091724in}{2.498862in}%
\pgfsys@useobject{currentmarker}{}%
\end{pgfscope}%
\begin{pgfscope}%
\pgfsys@transformshift{2.123931in}{2.564316in}%
\pgfsys@useobject{currentmarker}{}%
\end{pgfscope}%
\begin{pgfscope}%
\pgfsys@transformshift{2.116845in}{2.645334in}%
\pgfsys@useobject{currentmarker}{}%
\end{pgfscope}%
\begin{pgfscope}%
\pgfsys@transformshift{2.148818in}{2.733273in}%
\pgfsys@useobject{currentmarker}{}%
\end{pgfscope}%
\begin{pgfscope}%
\pgfsys@transformshift{2.136003in}{2.831173in}%
\pgfsys@useobject{currentmarker}{}%
\end{pgfscope}%
\begin{pgfscope}%
\pgfsys@transformshift{2.177133in}{2.928714in}%
\pgfsys@useobject{currentmarker}{}%
\end{pgfscope}%
\begin{pgfscope}%
\pgfsys@transformshift{2.171276in}{2.986640in}%
\pgfsys@useobject{currentmarker}{}%
\end{pgfscope}%
\begin{pgfscope}%
\pgfsys@transformshift{2.191594in}{3.044568in}%
\pgfsys@useobject{currentmarker}{}%
\end{pgfscope}%
\begin{pgfscope}%
\pgfsys@transformshift{2.190218in}{3.078303in}%
\pgfsys@useobject{currentmarker}{}%
\end{pgfscope}%
\begin{pgfscope}%
\pgfsys@transformshift{2.203060in}{3.116383in}%
\pgfsys@useobject{currentmarker}{}%
\end{pgfscope}%
\begin{pgfscope}%
\pgfsys@transformshift{2.203054in}{3.138486in}%
\pgfsys@useobject{currentmarker}{}%
\end{pgfscope}%
\begin{pgfscope}%
\pgfsys@transformshift{2.212078in}{3.164518in}%
\pgfsys@useobject{currentmarker}{}%
\end{pgfscope}%
\begin{pgfscope}%
\pgfsys@transformshift{2.210909in}{3.179627in}%
\pgfsys@useobject{currentmarker}{}%
\end{pgfscope}%
\begin{pgfscope}%
\pgfsys@transformshift{2.214043in}{3.187350in}%
\pgfsys@useobject{currentmarker}{}%
\end{pgfscope}%
\begin{pgfscope}%
\pgfsys@transformshift{2.213598in}{3.191912in}%
\pgfsys@useobject{currentmarker}{}%
\end{pgfscope}%
\begin{pgfscope}%
\pgfsys@transformshift{2.216536in}{3.199563in}%
\pgfsys@useobject{currentmarker}{}%
\end{pgfscope}%
\begin{pgfscope}%
\pgfsys@transformshift{2.216120in}{3.204051in}%
\pgfsys@useobject{currentmarker}{}%
\end{pgfscope}%
\begin{pgfscope}%
\pgfsys@transformshift{2.219233in}{3.212023in}%
\pgfsys@useobject{currentmarker}{}%
\end{pgfscope}%
\begin{pgfscope}%
\pgfsys@transformshift{2.218631in}{3.216691in}%
\pgfsys@useobject{currentmarker}{}%
\end{pgfscope}%
\begin{pgfscope}%
\pgfsys@transformshift{2.221562in}{3.223959in}%
\pgfsys@useobject{currentmarker}{}%
\end{pgfscope}%
\begin{pgfscope}%
\pgfsys@transformshift{2.219016in}{3.234487in}%
\pgfsys@useobject{currentmarker}{}%
\end{pgfscope}%
\begin{pgfscope}%
\pgfsys@transformshift{2.225257in}{3.249039in}%
\pgfsys@useobject{currentmarker}{}%
\end{pgfscope}%
\begin{pgfscope}%
\pgfsys@transformshift{2.226881in}{3.257595in}%
\pgfsys@useobject{currentmarker}{}%
\end{pgfscope}%
\begin{pgfscope}%
\pgfsys@transformshift{2.231172in}{3.270172in}%
\pgfsys@useobject{currentmarker}{}%
\end{pgfscope}%
\begin{pgfscope}%
\pgfsys@transformshift{2.233611in}{3.287101in}%
\pgfsys@useobject{currentmarker}{}%
\end{pgfscope}%
\begin{pgfscope}%
\pgfsys@transformshift{2.235364in}{3.296343in}%
\pgfsys@useobject{currentmarker}{}%
\end{pgfscope}%
\begin{pgfscope}%
\pgfsys@transformshift{2.236440in}{3.301404in}%
\pgfsys@useobject{currentmarker}{}%
\end{pgfscope}%
\begin{pgfscope}%
\pgfsys@transformshift{2.236365in}{3.304248in}%
\pgfsys@useobject{currentmarker}{}%
\end{pgfscope}%
\begin{pgfscope}%
\pgfsys@transformshift{2.236907in}{3.305717in}%
\pgfsys@useobject{currentmarker}{}%
\end{pgfscope}%
\begin{pgfscope}%
\pgfsys@transformshift{2.236847in}{3.306575in}%
\pgfsys@useobject{currentmarker}{}%
\end{pgfscope}%
\begin{pgfscope}%
\pgfsys@transformshift{2.238772in}{3.311545in}%
\pgfsys@useobject{currentmarker}{}%
\end{pgfscope}%
\begin{pgfscope}%
\pgfsys@transformshift{2.238491in}{3.314463in}%
\pgfsys@useobject{currentmarker}{}%
\end{pgfscope}%
\begin{pgfscope}%
\pgfsys@transformshift{2.239064in}{3.315969in}%
\pgfsys@useobject{currentmarker}{}%
\end{pgfscope}%
\begin{pgfscope}%
\pgfsys@transformshift{2.237963in}{3.324985in}%
\pgfsys@useobject{currentmarker}{}%
\end{pgfscope}%
\begin{pgfscope}%
\pgfsys@transformshift{2.239528in}{3.329729in}%
\pgfsys@useobject{currentmarker}{}%
\end{pgfscope}%
\begin{pgfscope}%
\pgfsys@transformshift{2.239715in}{3.332470in}%
\pgfsys@useobject{currentmarker}{}%
\end{pgfscope}%
\begin{pgfscope}%
\pgfsys@transformshift{2.241032in}{3.338319in}%
\pgfsys@useobject{currentmarker}{}%
\end{pgfscope}%
\begin{pgfscope}%
\pgfsys@transformshift{2.243086in}{3.348854in}%
\pgfsys@useobject{currentmarker}{}%
\end{pgfscope}%
\begin{pgfscope}%
\pgfsys@transformshift{2.243562in}{3.354738in}%
\pgfsys@useobject{currentmarker}{}%
\end{pgfscope}%
\begin{pgfscope}%
\pgfsys@transformshift{2.246947in}{3.364363in}%
\pgfsys@useobject{currentmarker}{}%
\end{pgfscope}%
\begin{pgfscope}%
\pgfsys@transformshift{2.246456in}{3.369954in}%
\pgfsys@useobject{currentmarker}{}%
\end{pgfscope}%
\begin{pgfscope}%
\pgfsys@transformshift{2.249215in}{3.379755in}%
\pgfsys@useobject{currentmarker}{}%
\end{pgfscope}%
\begin{pgfscope}%
\pgfsys@transformshift{2.247080in}{3.435429in}%
\pgfsys@useobject{currentmarker}{}%
\end{pgfscope}%
\begin{pgfscope}%
\pgfsys@transformshift{2.255844in}{3.464792in}%
\pgfsys@useobject{currentmarker}{}%
\end{pgfscope}%
\begin{pgfscope}%
\pgfsys@transformshift{2.255442in}{3.481641in}%
\pgfsys@useobject{currentmarker}{}%
\end{pgfscope}%
\begin{pgfscope}%
\pgfsys@transformshift{2.257715in}{3.490627in}%
\pgfsys@useobject{currentmarker}{}%
\end{pgfscope}%
\begin{pgfscope}%
\pgfsys@transformshift{2.257919in}{3.495721in}%
\pgfsys@useobject{currentmarker}{}%
\end{pgfscope}%
\begin{pgfscope}%
\pgfsys@transformshift{2.258418in}{3.498480in}%
\pgfsys@useobject{currentmarker}{}%
\end{pgfscope}%
\begin{pgfscope}%
\pgfsys@transformshift{2.258787in}{3.506217in}%
\pgfsys@useobject{currentmarker}{}%
\end{pgfscope}%
\begin{pgfscope}%
\pgfsys@transformshift{2.257969in}{3.510397in}%
\pgfsys@useobject{currentmarker}{}%
\end{pgfscope}%
\begin{pgfscope}%
\pgfsys@transformshift{2.260103in}{3.520416in}%
\pgfsys@useobject{currentmarker}{}%
\end{pgfscope}%
\begin{pgfscope}%
\pgfsys@transformshift{2.260216in}{3.526048in}%
\pgfsys@useobject{currentmarker}{}%
\end{pgfscope}%
\begin{pgfscope}%
\pgfsys@transformshift{2.260888in}{3.529073in}%
\pgfsys@useobject{currentmarker}{}%
\end{pgfscope}%
\begin{pgfscope}%
\pgfsys@transformshift{2.261128in}{3.530760in}%
\pgfsys@useobject{currentmarker}{}%
\end{pgfscope}%
\begin{pgfscope}%
\pgfsys@transformshift{2.261185in}{3.531695in}%
\pgfsys@useobject{currentmarker}{}%
\end{pgfscope}%
\begin{pgfscope}%
\pgfsys@transformshift{2.261324in}{3.532192in}%
\pgfsys@useobject{currentmarker}{}%
\end{pgfscope}%
\begin{pgfscope}%
\pgfsys@transformshift{2.261303in}{3.532475in}%
\pgfsys@useobject{currentmarker}{}%
\end{pgfscope}%
\begin{pgfscope}%
\pgfsys@transformshift{2.261353in}{3.532622in}%
\pgfsys@useobject{currentmarker}{}%
\end{pgfscope}%
\begin{pgfscope}%
\pgfsys@transformshift{2.259229in}{3.549357in}%
\pgfsys@useobject{currentmarker}{}%
\end{pgfscope}%
\begin{pgfscope}%
\pgfsys@transformshift{2.262765in}{3.557934in}%
\pgfsys@useobject{currentmarker}{}%
\end{pgfscope}%
\begin{pgfscope}%
\pgfsys@transformshift{2.261569in}{3.562895in}%
\pgfsys@useobject{currentmarker}{}%
\end{pgfscope}%
\begin{pgfscope}%
\pgfsys@transformshift{2.264423in}{3.571361in}%
\pgfsys@useobject{currentmarker}{}%
\end{pgfscope}%
\begin{pgfscope}%
\pgfsys@transformshift{2.264170in}{3.576269in}%
\pgfsys@useobject{currentmarker}{}%
\end{pgfscope}%
\begin{pgfscope}%
\pgfsys@transformshift{2.265019in}{3.578834in}%
\pgfsys@useobject{currentmarker}{}%
\end{pgfscope}%
\begin{pgfscope}%
\pgfsys@transformshift{2.265754in}{3.587417in}%
\pgfsys@useobject{currentmarker}{}%
\end{pgfscope}%
\begin{pgfscope}%
\pgfsys@transformshift{2.267101in}{3.599879in}%
\pgfsys@useobject{currentmarker}{}%
\end{pgfscope}%
\begin{pgfscope}%
\pgfsys@transformshift{2.271695in}{3.614803in}%
\pgfsys@useobject{currentmarker}{}%
\end{pgfscope}%
\begin{pgfscope}%
\pgfsys@transformshift{2.272655in}{3.623337in}%
\pgfsys@useobject{currentmarker}{}%
\end{pgfscope}%
\begin{pgfscope}%
\pgfsys@transformshift{2.273977in}{3.627872in}%
\pgfsys@useobject{currentmarker}{}%
\end{pgfscope}%
\begin{pgfscope}%
\pgfsys@transformshift{2.275950in}{3.635549in}%
\pgfsys@useobject{currentmarker}{}%
\end{pgfscope}%
\begin{pgfscope}%
\pgfsys@transformshift{2.276501in}{3.639874in}%
\pgfsys@useobject{currentmarker}{}%
\end{pgfscope}%
\begin{pgfscope}%
\pgfsys@transformshift{2.277785in}{3.641899in}%
\pgfsys@useobject{currentmarker}{}%
\end{pgfscope}%
\begin{pgfscope}%
\pgfsys@transformshift{2.277594in}{3.643204in}%
\pgfsys@useobject{currentmarker}{}%
\end{pgfscope}%
\begin{pgfscope}%
\pgfsys@transformshift{2.277776in}{3.643907in}%
\pgfsys@useobject{currentmarker}{}%
\end{pgfscope}%
\begin{pgfscope}%
\pgfsys@transformshift{2.278670in}{3.649398in}%
\pgfsys@useobject{currentmarker}{}%
\end{pgfscope}%
\begin{pgfscope}%
\pgfsys@transformshift{2.279001in}{3.652441in}%
\pgfsys@useobject{currentmarker}{}%
\end{pgfscope}%
\begin{pgfscope}%
\pgfsys@transformshift{2.282080in}{3.658117in}%
\pgfsys@useobject{currentmarker}{}%
\end{pgfscope}%
\begin{pgfscope}%
\pgfsys@transformshift{2.281250in}{3.661570in}%
\pgfsys@useobject{currentmarker}{}%
\end{pgfscope}%
\begin{pgfscope}%
\pgfsys@transformshift{2.283113in}{3.669468in}%
\pgfsys@useobject{currentmarker}{}%
\end{pgfscope}%
\begin{pgfscope}%
\pgfsys@transformshift{2.283699in}{3.683601in}%
\pgfsys@useobject{currentmarker}{}%
\end{pgfscope}%
\begin{pgfscope}%
\pgfsys@transformshift{2.285183in}{3.691237in}%
\pgfsys@useobject{currentmarker}{}%
\end{pgfscope}%
\begin{pgfscope}%
\pgfsys@transformshift{2.288888in}{3.704357in}%
\pgfsys@useobject{currentmarker}{}%
\end{pgfscope}%
\begin{pgfscope}%
\pgfsys@transformshift{2.288866in}{3.711856in}%
\pgfsys@useobject{currentmarker}{}%
\end{pgfscope}%
\begin{pgfscope}%
\pgfsys@transformshift{2.293035in}{3.723377in}%
\pgfsys@useobject{currentmarker}{}%
\end{pgfscope}%
\begin{pgfscope}%
\pgfsys@transformshift{2.293478in}{3.730101in}%
\pgfsys@useobject{currentmarker}{}%
\end{pgfscope}%
\begin{pgfscope}%
\pgfsys@transformshift{2.294483in}{3.733668in}%
\pgfsys@useobject{currentmarker}{}%
\end{pgfscope}%
\begin{pgfscope}%
\pgfsys@transformshift{2.294706in}{3.735694in}%
\pgfsys@useobject{currentmarker}{}%
\end{pgfscope}%
\begin{pgfscope}%
\pgfsys@transformshift{2.294970in}{3.736784in}%
\pgfsys@useobject{currentmarker}{}%
\end{pgfscope}%
\begin{pgfscope}%
\pgfsys@transformshift{2.295780in}{3.741487in}%
\pgfsys@useobject{currentmarker}{}%
\end{pgfscope}%
\begin{pgfscope}%
\pgfsys@transformshift{2.296398in}{3.744039in}%
\pgfsys@useobject{currentmarker}{}%
\end{pgfscope}%
\begin{pgfscope}%
\pgfsys@transformshift{2.296670in}{3.745456in}%
\pgfsys@useobject{currentmarker}{}%
\end{pgfscope}%
\begin{pgfscope}%
\pgfsys@transformshift{2.298199in}{3.750347in}%
\pgfsys@useobject{currentmarker}{}%
\end{pgfscope}%
\begin{pgfscope}%
\pgfsys@transformshift{2.298318in}{3.753163in}%
\pgfsys@useobject{currentmarker}{}%
\end{pgfscope}%
\begin{pgfscope}%
\pgfsys@transformshift{2.302166in}{3.761306in}%
\pgfsys@useobject{currentmarker}{}%
\end{pgfscope}%
\begin{pgfscope}%
\pgfsys@transformshift{2.302412in}{3.766254in}%
\pgfsys@useobject{currentmarker}{}%
\end{pgfscope}%
\begin{pgfscope}%
\pgfsys@transformshift{2.303361in}{3.768808in}%
\pgfsys@useobject{currentmarker}{}%
\end{pgfscope}%
\begin{pgfscope}%
\pgfsys@transformshift{2.304589in}{3.776475in}%
\pgfsys@useobject{currentmarker}{}%
\end{pgfscope}%
\begin{pgfscope}%
\pgfsys@transformshift{2.305685in}{3.780603in}%
\pgfsys@useobject{currentmarker}{}%
\end{pgfscope}%
\begin{pgfscope}%
\pgfsys@transformshift{2.308356in}{3.790614in}%
\pgfsys@useobject{currentmarker}{}%
\end{pgfscope}%
\begin{pgfscope}%
\pgfsys@transformshift{2.309448in}{3.796208in}%
\pgfsys@useobject{currentmarker}{}%
\end{pgfscope}%
\begin{pgfscope}%
\pgfsys@transformshift{2.310261in}{3.799235in}%
\pgfsys@useobject{currentmarker}{}%
\end{pgfscope}%
\begin{pgfscope}%
\pgfsys@transformshift{2.312168in}{3.805872in}%
\pgfsys@useobject{currentmarker}{}%
\end{pgfscope}%
\begin{pgfscope}%
\pgfsys@transformshift{2.312678in}{3.809636in}%
\pgfsys@useobject{currentmarker}{}%
\end{pgfscope}%
\begin{pgfscope}%
\pgfsys@transformshift{2.315648in}{3.818270in}%
\pgfsys@useobject{currentmarker}{}%
\end{pgfscope}%
\begin{pgfscope}%
\pgfsys@transformshift{2.315972in}{3.823281in}%
\pgfsys@useobject{currentmarker}{}%
\end{pgfscope}%
\begin{pgfscope}%
\pgfsys@transformshift{2.316726in}{3.825938in}%
\pgfsys@useobject{currentmarker}{}%
\end{pgfscope}%
\begin{pgfscope}%
\pgfsys@transformshift{2.317421in}{3.835333in}%
\pgfsys@useobject{currentmarker}{}%
\end{pgfscope}%
\begin{pgfscope}%
\pgfsys@transformshift{2.320783in}{3.847450in}%
\pgfsys@useobject{currentmarker}{}%
\end{pgfscope}%
\begin{pgfscope}%
\pgfsys@transformshift{2.322612in}{3.854120in}%
\pgfsys@useobject{currentmarker}{}%
\end{pgfscope}%
\begin{pgfscope}%
\pgfsys@transformshift{2.322259in}{3.857908in}%
\pgfsys@useobject{currentmarker}{}%
\end{pgfscope}%
\begin{pgfscope}%
\pgfsys@transformshift{2.326043in}{3.867785in}%
\pgfsys@useobject{currentmarker}{}%
\end{pgfscope}%
\begin{pgfscope}%
\pgfsys@transformshift{2.326780in}{3.873555in}%
\pgfsys@useobject{currentmarker}{}%
\end{pgfscope}%
\begin{pgfscope}%
\pgfsys@transformshift{2.328051in}{3.882988in}%
\pgfsys@useobject{currentmarker}{}%
\end{pgfscope}%
\begin{pgfscope}%
\pgfsys@transformshift{2.329954in}{3.887865in}%
\pgfsys@useobject{currentmarker}{}%
\end{pgfscope}%
\begin{pgfscope}%
\pgfsys@transformshift{2.329926in}{3.890744in}%
\pgfsys@useobject{currentmarker}{}%
\end{pgfscope}%
\begin{pgfscope}%
\pgfsys@transformshift{2.332264in}{3.897975in}%
\pgfsys@useobject{currentmarker}{}%
\end{pgfscope}%
\begin{pgfscope}%
\pgfsys@transformshift{2.331909in}{3.908530in}%
\pgfsys@useobject{currentmarker}{}%
\end{pgfscope}%
\begin{pgfscope}%
\pgfsys@transformshift{2.336303in}{3.922563in}%
\pgfsys@useobject{currentmarker}{}%
\end{pgfscope}%
\begin{pgfscope}%
\pgfsys@transformshift{2.336735in}{3.930638in}%
\pgfsys@useobject{currentmarker}{}%
\end{pgfscope}%
\begin{pgfscope}%
\pgfsys@transformshift{2.339860in}{3.941595in}%
\pgfsys@useobject{currentmarker}{}%
\end{pgfscope}%
\begin{pgfscope}%
\pgfsys@transformshift{2.340426in}{3.947836in}%
\pgfsys@useobject{currentmarker}{}%
\end{pgfscope}%
\begin{pgfscope}%
\pgfsys@transformshift{2.342567in}{3.956782in}%
\pgfsys@useobject{currentmarker}{}%
\end{pgfscope}%
\begin{pgfscope}%
\pgfsys@transformshift{2.342992in}{3.961823in}%
\pgfsys@useobject{currentmarker}{}%
\end{pgfscope}%
\begin{pgfscope}%
\pgfsys@transformshift{2.343629in}{3.964532in}%
\pgfsys@useobject{currentmarker}{}%
\end{pgfscope}%
\begin{pgfscope}%
\pgfsys@transformshift{2.343726in}{3.966059in}%
\pgfsys@useobject{currentmarker}{}%
\end{pgfscope}%
\begin{pgfscope}%
\pgfsys@transformshift{2.343943in}{3.966873in}%
\pgfsys@useobject{currentmarker}{}%
\end{pgfscope}%
\begin{pgfscope}%
\pgfsys@transformshift{2.343983in}{3.967334in}%
\pgfsys@useobject{currentmarker}{}%
\end{pgfscope}%
\begin{pgfscope}%
\pgfsys@transformshift{2.344042in}{3.967582in}%
\pgfsys@useobject{currentmarker}{}%
\end{pgfscope}%
\begin{pgfscope}%
\pgfsys@transformshift{2.344052in}{3.967721in}%
\pgfsys@useobject{currentmarker}{}%
\end{pgfscope}%
\begin{pgfscope}%
\pgfsys@transformshift{2.344069in}{3.967797in}%
\pgfsys@useobject{currentmarker}{}%
\end{pgfscope}%
\begin{pgfscope}%
\pgfsys@transformshift{2.344072in}{3.967839in}%
\pgfsys@useobject{currentmarker}{}%
\end{pgfscope}%
\begin{pgfscope}%
\pgfsys@transformshift{2.345027in}{3.971486in}%
\pgfsys@useobject{currentmarker}{}%
\end{pgfscope}%
\begin{pgfscope}%
\pgfsys@transformshift{2.345242in}{3.973549in}%
\pgfsys@useobject{currentmarker}{}%
\end{pgfscope}%
\begin{pgfscope}%
\pgfsys@transformshift{2.346467in}{3.978712in}%
\pgfsys@useobject{currentmarker}{}%
\end{pgfscope}%
\begin{pgfscope}%
\pgfsys@transformshift{2.346762in}{3.981616in}%
\pgfsys@useobject{currentmarker}{}%
\end{pgfscope}%
\begin{pgfscope}%
\pgfsys@transformshift{2.347745in}{3.987488in}%
\pgfsys@useobject{currentmarker}{}%
\end{pgfscope}%
\begin{pgfscope}%
\pgfsys@transformshift{2.347770in}{3.990762in}%
\pgfsys@useobject{currentmarker}{}%
\end{pgfscope}%
\begin{pgfscope}%
\pgfsys@transformshift{2.347221in}{4.000397in}%
\pgfsys@useobject{currentmarker}{}%
\end{pgfscope}%
\begin{pgfscope}%
\pgfsys@transformshift{2.344180in}{4.018466in}%
\pgfsys@useobject{currentmarker}{}%
\end{pgfscope}%
\begin{pgfscope}%
\pgfsys@transformshift{2.338080in}{4.026488in}%
\pgfsys@useobject{currentmarker}{}%
\end{pgfscope}%
\begin{pgfscope}%
\pgfsys@transformshift{2.326996in}{4.038894in}%
\pgfsys@useobject{currentmarker}{}%
\end{pgfscope}%
\begin{pgfscope}%
\pgfsys@transformshift{2.304908in}{4.040685in}%
\pgfsys@useobject{currentmarker}{}%
\end{pgfscope}%
\begin{pgfscope}%
\pgfsys@transformshift{2.292941in}{4.043000in}%
\pgfsys@useobject{currentmarker}{}%
\end{pgfscope}%
\begin{pgfscope}%
\pgfsys@transformshift{2.286591in}{4.040849in}%
\pgfsys@useobject{currentmarker}{}%
\end{pgfscope}%
\begin{pgfscope}%
\pgfsys@transformshift{2.282908in}{4.040679in}%
\pgfsys@useobject{currentmarker}{}%
\end{pgfscope}%
\begin{pgfscope}%
\pgfsys@transformshift{2.281056in}{4.039853in}%
\pgfsys@useobject{currentmarker}{}%
\end{pgfscope}%
\begin{pgfscope}%
\pgfsys@transformshift{2.280004in}{4.039483in}%
\pgfsys@useobject{currentmarker}{}%
\end{pgfscope}%
\begin{pgfscope}%
\pgfsys@transformshift{2.273366in}{4.035445in}%
\pgfsys@useobject{currentmarker}{}%
\end{pgfscope}%
\begin{pgfscope}%
\pgfsys@transformshift{2.261597in}{4.027414in}%
\pgfsys@useobject{currentmarker}{}%
\end{pgfscope}%
\begin{pgfscope}%
\pgfsys@transformshift{2.256257in}{4.021679in}%
\pgfsys@useobject{currentmarker}{}%
\end{pgfscope}%
\begin{pgfscope}%
\pgfsys@transformshift{2.252849in}{4.019041in}%
\pgfsys@useobject{currentmarker}{}%
\end{pgfscope}%
\begin{pgfscope}%
\pgfsys@transformshift{2.246946in}{4.011437in}%
\pgfsys@useobject{currentmarker}{}%
\end{pgfscope}%
\begin{pgfscope}%
\pgfsys@transformshift{2.234504in}{4.001782in}%
\pgfsys@useobject{currentmarker}{}%
\end{pgfscope}%
\begin{pgfscope}%
\pgfsys@transformshift{2.229466in}{3.994735in}%
\pgfsys@useobject{currentmarker}{}%
\end{pgfscope}%
\begin{pgfscope}%
\pgfsys@transformshift{2.226245in}{3.991225in}%
\pgfsys@useobject{currentmarker}{}%
\end{pgfscope}%
\begin{pgfscope}%
\pgfsys@transformshift{2.224427in}{3.989338in}%
\pgfsys@useobject{currentmarker}{}%
\end{pgfscope}%
\begin{pgfscope}%
\pgfsys@transformshift{2.220759in}{3.983075in}%
\pgfsys@useobject{currentmarker}{}%
\end{pgfscope}%
\begin{pgfscope}%
\pgfsys@transformshift{2.211257in}{3.976498in}%
\pgfsys@useobject{currentmarker}{}%
\end{pgfscope}%
\begin{pgfscope}%
\pgfsys@transformshift{2.202158in}{3.959948in}%
\pgfsys@useobject{currentmarker}{}%
\end{pgfscope}%
\begin{pgfscope}%
\pgfsys@transformshift{2.185125in}{3.945566in}%
\pgfsys@useobject{currentmarker}{}%
\end{pgfscope}%
\begin{pgfscope}%
\pgfsys@transformshift{2.178745in}{3.935095in}%
\pgfsys@useobject{currentmarker}{}%
\end{pgfscope}%
\begin{pgfscope}%
\pgfsys@transformshift{2.174595in}{3.929779in}%
\pgfsys@useobject{currentmarker}{}%
\end{pgfscope}%
\begin{pgfscope}%
\pgfsys@transformshift{2.168433in}{3.922076in}%
\pgfsys@useobject{currentmarker}{}%
\end{pgfscope}%
\begin{pgfscope}%
\pgfsys@transformshift{2.165584in}{3.917458in}%
\pgfsys@useobject{currentmarker}{}%
\end{pgfscope}%
\begin{pgfscope}%
\pgfsys@transformshift{2.159398in}{3.911344in}%
\pgfsys@useobject{currentmarker}{}%
\end{pgfscope}%
\begin{pgfscope}%
\pgfsys@transformshift{2.153199in}{3.898082in}%
\pgfsys@useobject{currentmarker}{}%
\end{pgfscope}%
\begin{pgfscope}%
\pgfsys@transformshift{2.146989in}{3.892957in}%
\pgfsys@useobject{currentmarker}{}%
\end{pgfscope}%
\begin{pgfscope}%
\pgfsys@transformshift{2.140724in}{3.881788in}%
\pgfsys@useobject{currentmarker}{}%
\end{pgfscope}%
\begin{pgfscope}%
\pgfsys@transformshift{2.135463in}{3.877104in}%
\pgfsys@useobject{currentmarker}{}%
\end{pgfscope}%
\begin{pgfscope}%
\pgfsys@transformshift{2.133596in}{3.873709in}%
\pgfsys@useobject{currentmarker}{}%
\end{pgfscope}%
\begin{pgfscope}%
\pgfsys@transformshift{2.127552in}{3.868529in}%
\pgfsys@useobject{currentmarker}{}%
\end{pgfscope}%
\begin{pgfscope}%
\pgfsys@transformshift{2.125538in}{3.864641in}%
\pgfsys@useobject{currentmarker}{}%
\end{pgfscope}%
\begin{pgfscope}%
\pgfsys@transformshift{2.116178in}{3.854721in}%
\pgfsys@useobject{currentmarker}{}%
\end{pgfscope}%
\begin{pgfscope}%
\pgfsys@transformshift{2.107767in}{3.837303in}%
\pgfsys@useobject{currentmarker}{}%
\end{pgfscope}%
\begin{pgfscope}%
\pgfsys@transformshift{2.089589in}{3.818567in}%
\pgfsys@useobject{currentmarker}{}%
\end{pgfscope}%
\begin{pgfscope}%
\pgfsys@transformshift{2.071867in}{3.787099in}%
\pgfsys@useobject{currentmarker}{}%
\end{pgfscope}%
\begin{pgfscope}%
\pgfsys@transformshift{2.039047in}{3.757335in}%
\pgfsys@useobject{currentmarker}{}%
\end{pgfscope}%
\begin{pgfscope}%
\pgfsys@transformshift{2.012872in}{3.713514in}%
\pgfsys@useobject{currentmarker}{}%
\end{pgfscope}%
\begin{pgfscope}%
\pgfsys@transformshift{1.992645in}{3.694046in}%
\pgfsys@useobject{currentmarker}{}%
\end{pgfscope}%
\begin{pgfscope}%
\pgfsys@transformshift{1.974991in}{3.667153in}%
\pgfsys@useobject{currentmarker}{}%
\end{pgfscope}%
\begin{pgfscope}%
\pgfsys@transformshift{1.946868in}{3.640093in}%
\pgfsys@useobject{currentmarker}{}%
\end{pgfscope}%
\begin{pgfscope}%
\pgfsys@transformshift{1.923818in}{3.599647in}%
\pgfsys@useobject{currentmarker}{}%
\end{pgfscope}%
\begin{pgfscope}%
\pgfsys@transformshift{1.953040in}{3.566386in}%
\pgfsys@useobject{currentmarker}{}%
\end{pgfscope}%
\begin{pgfscope}%
\pgfsys@transformshift{1.985063in}{3.514856in}%
\pgfsys@useobject{currentmarker}{}%
\end{pgfscope}%
\begin{pgfscope}%
\pgfsys@transformshift{2.037287in}{3.473911in}%
\pgfsys@useobject{currentmarker}{}%
\end{pgfscope}%
\begin{pgfscope}%
\pgfsys@transformshift{2.078248in}{3.408875in}%
\pgfsys@useobject{currentmarker}{}%
\end{pgfscope}%
\begin{pgfscope}%
\pgfsys@transformshift{2.114338in}{3.386862in}%
\pgfsys@useobject{currentmarker}{}%
\end{pgfscope}%
\begin{pgfscope}%
\pgfsys@transformshift{2.143892in}{3.343743in}%
\pgfsys@useobject{currentmarker}{}%
\end{pgfscope}%
\begin{pgfscope}%
\pgfsys@transformshift{2.192739in}{3.311672in}%
\pgfsys@useobject{currentmarker}{}%
\end{pgfscope}%
\begin{pgfscope}%
\pgfsys@transformshift{2.229244in}{3.259474in}%
\pgfsys@useobject{currentmarker}{}%
\end{pgfscope}%
\begin{pgfscope}%
\pgfsys@transformshift{2.288453in}{3.219154in}%
\pgfsys@useobject{currentmarker}{}%
\end{pgfscope}%
\begin{pgfscope}%
\pgfsys@transformshift{2.339161in}{3.158610in}%
\pgfsys@useobject{currentmarker}{}%
\end{pgfscope}%
\begin{pgfscope}%
\pgfsys@transformshift{2.416426in}{3.117347in}%
\pgfsys@useobject{currentmarker}{}%
\end{pgfscope}%
\begin{pgfscope}%
\pgfsys@transformshift{2.477797in}{3.048195in}%
\pgfsys@useobject{currentmarker}{}%
\end{pgfscope}%
\begin{pgfscope}%
\pgfsys@transformshift{2.522320in}{3.023626in}%
\pgfsys@useobject{currentmarker}{}%
\end{pgfscope}%
\begin{pgfscope}%
\pgfsys@transformshift{2.557495in}{2.981402in}%
\pgfsys@useobject{currentmarker}{}%
\end{pgfscope}%
\begin{pgfscope}%
\pgfsys@transformshift{2.605959in}{2.946552in}%
\pgfsys@useobject{currentmarker}{}%
\end{pgfscope}%
\begin{pgfscope}%
\pgfsys@transformshift{2.624387in}{2.919380in}%
\pgfsys@useobject{currentmarker}{}%
\end{pgfscope}%
\begin{pgfscope}%
\pgfsys@transformshift{2.651433in}{2.894998in}%
\pgfsys@useobject{currentmarker}{}%
\end{pgfscope}%
\begin{pgfscope}%
\pgfsys@transformshift{2.672896in}{2.861579in}%
\pgfsys@useobject{currentmarker}{}%
\end{pgfscope}%
\begin{pgfscope}%
\pgfsys@transformshift{2.708882in}{2.836031in}%
\pgfsys@useobject{currentmarker}{}%
\end{pgfscope}%
\begin{pgfscope}%
\pgfsys@transformshift{2.733231in}{2.791622in}%
\pgfsys@useobject{currentmarker}{}%
\end{pgfscope}%
\begin{pgfscope}%
\pgfsys@transformshift{2.754406in}{2.773524in}%
\pgfsys@useobject{currentmarker}{}%
\end{pgfscope}%
\begin{pgfscope}%
\pgfsys@transformshift{2.772340in}{2.742978in}%
\pgfsys@useobject{currentmarker}{}%
\end{pgfscope}%
\begin{pgfscope}%
\pgfsys@transformshift{2.787527in}{2.730775in}%
\pgfsys@useobject{currentmarker}{}%
\end{pgfscope}%
\begin{pgfscope}%
\pgfsys@transformshift{2.801656in}{2.706938in}%
\pgfsys@useobject{currentmarker}{}%
\end{pgfscope}%
\begin{pgfscope}%
\pgfsys@transformshift{2.814060in}{2.698082in}%
\pgfsys@useobject{currentmarker}{}%
\end{pgfscope}%
\begin{pgfscope}%
\pgfsys@transformshift{2.828479in}{2.674615in}%
\pgfsys@useobject{currentmarker}{}%
\end{pgfscope}%
\begin{pgfscope}%
\pgfsys@transformshift{2.852287in}{2.655096in}%
\pgfsys@useobject{currentmarker}{}%
\end{pgfscope}%
\begin{pgfscope}%
\pgfsys@transformshift{2.871508in}{2.623781in}%
\pgfsys@useobject{currentmarker}{}%
\end{pgfscope}%
\begin{pgfscope}%
\pgfsys@transformshift{2.903910in}{2.598398in}%
\pgfsys@useobject{currentmarker}{}%
\end{pgfscope}%
\begin{pgfscope}%
\pgfsys@transformshift{2.931023in}{2.555747in}%
\pgfsys@useobject{currentmarker}{}%
\end{pgfscope}%
\begin{pgfscope}%
\pgfsys@transformshift{2.974818in}{2.520355in}%
\pgfsys@useobject{currentmarker}{}%
\end{pgfscope}%
\begin{pgfscope}%
\pgfsys@transformshift{3.009058in}{2.461554in}%
\pgfsys@useobject{currentmarker}{}%
\end{pgfscope}%
\begin{pgfscope}%
\pgfsys@transformshift{3.066649in}{2.413526in}%
\pgfsys@useobject{currentmarker}{}%
\end{pgfscope}%
\begin{pgfscope}%
\pgfsys@transformshift{3.109372in}{2.343459in}%
\pgfsys@useobject{currentmarker}{}%
\end{pgfscope}%
\begin{pgfscope}%
\pgfsys@transformshift{3.142323in}{2.312615in}%
\pgfsys@useobject{currentmarker}{}%
\end{pgfscope}%
\begin{pgfscope}%
\pgfsys@transformshift{3.169487in}{2.263277in}%
\pgfsys@useobject{currentmarker}{}%
\end{pgfscope}%
\begin{pgfscope}%
\pgfsys@transformshift{3.193443in}{2.243638in}%
\pgfsys@useobject{currentmarker}{}%
\end{pgfscope}%
\begin{pgfscope}%
\pgfsys@transformshift{3.213672in}{2.204347in}%
\pgfsys@useobject{currentmarker}{}%
\end{pgfscope}%
\begin{pgfscope}%
\pgfsys@transformshift{3.253538in}{2.170528in}%
\pgfsys@useobject{currentmarker}{}%
\end{pgfscope}%
\begin{pgfscope}%
\pgfsys@transformshift{3.285400in}{2.116476in}%
\pgfsys@useobject{currentmarker}{}%
\end{pgfscope}%
\begin{pgfscope}%
\pgfsys@transformshift{3.338449in}{2.075316in}%
\pgfsys@useobject{currentmarker}{}%
\end{pgfscope}%
\begin{pgfscope}%
\pgfsys@transformshift{3.375597in}{2.011225in}%
\pgfsys@useobject{currentmarker}{}%
\end{pgfscope}%
\begin{pgfscope}%
\pgfsys@transformshift{3.405529in}{1.983585in}%
\pgfsys@useobject{currentmarker}{}%
\end{pgfscope}%
\begin{pgfscope}%
\pgfsys@transformshift{3.430960in}{1.942354in}%
\pgfsys@useobject{currentmarker}{}%
\end{pgfscope}%
\begin{pgfscope}%
\pgfsys@transformshift{3.451256in}{1.925092in}%
\pgfsys@useobject{currentmarker}{}%
\end{pgfscope}%
\begin{pgfscope}%
\pgfsys@transformshift{3.467998in}{1.893252in}%
\pgfsys@useobject{currentmarker}{}%
\end{pgfscope}%
\begin{pgfscope}%
\pgfsys@transformshift{3.499578in}{1.864556in}%
\pgfsys@useobject{currentmarker}{}%
\end{pgfscope}%
\begin{pgfscope}%
\pgfsys@transformshift{3.523037in}{1.820736in}%
\pgfsys@useobject{currentmarker}{}%
\end{pgfscope}%
\begin{pgfscope}%
\pgfsys@transformshift{3.566135in}{1.779679in}%
\pgfsys@useobject{currentmarker}{}%
\end{pgfscope}%
\begin{pgfscope}%
\pgfsys@transformshift{3.581973in}{1.751027in}%
\pgfsys@useobject{currentmarker}{}%
\end{pgfscope}%
\begin{pgfscope}%
\pgfsys@transformshift{3.603864in}{1.721734in}%
\pgfsys@useobject{currentmarker}{}%
\end{pgfscope}%
\begin{pgfscope}%
\pgfsys@transformshift{3.615998in}{1.705694in}%
\pgfsys@useobject{currentmarker}{}%
\end{pgfscope}%
\begin{pgfscope}%
\pgfsys@transformshift{3.626486in}{1.681765in}%
\pgfsys@useobject{currentmarker}{}%
\end{pgfscope}%
\begin{pgfscope}%
\pgfsys@transformshift{3.654294in}{1.656558in}%
\pgfsys@useobject{currentmarker}{}%
\end{pgfscope}%
\begin{pgfscope}%
\pgfsys@transformshift{3.676073in}{1.620466in}%
\pgfsys@useobject{currentmarker}{}%
\end{pgfscope}%
\begin{pgfscope}%
\pgfsys@transformshift{3.692882in}{1.604498in}%
\pgfsys@useobject{currentmarker}{}%
\end{pgfscope}%
\begin{pgfscope}%
\pgfsys@transformshift{3.707484in}{1.578945in}%
\pgfsys@useobject{currentmarker}{}%
\end{pgfscope}%
\begin{pgfscope}%
\pgfsys@transformshift{3.727762in}{1.549867in}%
\pgfsys@useobject{currentmarker}{}%
\end{pgfscope}%
\begin{pgfscope}%
\pgfsys@transformshift{3.755465in}{1.518557in}%
\pgfsys@useobject{currentmarker}{}%
\end{pgfscope}%
\begin{pgfscope}%
\pgfsys@transformshift{3.775702in}{1.478419in}%
\pgfsys@useobject{currentmarker}{}%
\end{pgfscope}%
\begin{pgfscope}%
\pgfsys@transformshift{3.817747in}{1.446501in}%
\pgfsys@useobject{currentmarker}{}%
\end{pgfscope}%
\begin{pgfscope}%
\pgfsys@transformshift{3.844192in}{1.396680in}%
\pgfsys@useobject{currentmarker}{}%
\end{pgfscope}%
\begin{pgfscope}%
\pgfsys@transformshift{3.888380in}{1.355371in}%
\pgfsys@useobject{currentmarker}{}%
\end{pgfscope}%
\begin{pgfscope}%
\pgfsys@transformshift{3.926554in}{1.297566in}%
\pgfsys@useobject{currentmarker}{}%
\end{pgfscope}%
\begin{pgfscope}%
\pgfsys@transformshift{3.981088in}{1.247318in}%
\pgfsys@useobject{currentmarker}{}%
\end{pgfscope}%
\begin{pgfscope}%
\pgfsys@transformshift{4.024752in}{1.179000in}%
\pgfsys@useobject{currentmarker}{}%
\end{pgfscope}%
\begin{pgfscope}%
\pgfsys@transformshift{4.080539in}{1.111127in}%
\pgfsys@useobject{currentmarker}{}%
\end{pgfscope}%
\begin{pgfscope}%
\pgfsys@transformshift{4.133639in}{1.031547in}%
\pgfsys@useobject{currentmarker}{}%
\end{pgfscope}%
\begin{pgfscope}%
\pgfsys@transformshift{4.168460in}{0.934970in}%
\pgfsys@useobject{currentmarker}{}%
\end{pgfscope}%
\begin{pgfscope}%
\pgfsys@transformshift{4.158581in}{0.879376in}%
\pgfsys@useobject{currentmarker}{}%
\end{pgfscope}%
\begin{pgfscope}%
\pgfsys@transformshift{4.116375in}{0.828877in}%
\pgfsys@useobject{currentmarker}{}%
\end{pgfscope}%
\begin{pgfscope}%
\pgfsys@transformshift{4.041038in}{0.798816in}%
\pgfsys@useobject{currentmarker}{}%
\end{pgfscope}%
\begin{pgfscope}%
\pgfsys@transformshift{3.957333in}{0.763133in}%
\pgfsys@useobject{currentmarker}{}%
\end{pgfscope}%
\begin{pgfscope}%
\pgfsys@transformshift{3.908030in}{0.754536in}%
\pgfsys@useobject{currentmarker}{}%
\end{pgfscope}%
\begin{pgfscope}%
\pgfsys@transformshift{3.852638in}{0.750085in}%
\pgfsys@useobject{currentmarker}{}%
\end{pgfscope}%
\begin{pgfscope}%
\pgfsys@transformshift{3.788631in}{0.747274in}%
\pgfsys@useobject{currentmarker}{}%
\end{pgfscope}%
\begin{pgfscope}%
\pgfsys@transformshift{3.721042in}{0.742337in}%
\pgfsys@useobject{currentmarker}{}%
\end{pgfscope}%
\begin{pgfscope}%
\pgfsys@transformshift{3.683903in}{0.739182in}%
\pgfsys@useobject{currentmarker}{}%
\end{pgfscope}%
\begin{pgfscope}%
\pgfsys@transformshift{3.643646in}{0.736587in}%
\pgfsys@useobject{currentmarker}{}%
\end{pgfscope}%
\begin{pgfscope}%
\pgfsys@transformshift{3.621541in}{0.734678in}%
\pgfsys@useobject{currentmarker}{}%
\end{pgfscope}%
\begin{pgfscope}%
\pgfsys@transformshift{3.609343in}{0.734352in}%
\pgfsys@useobject{currentmarker}{}%
\end{pgfscope}%
\begin{pgfscope}%
\pgfsys@transformshift{3.602655in}{0.733784in}%
\pgfsys@useobject{currentmarker}{}%
\end{pgfscope}%
\begin{pgfscope}%
\pgfsys@transformshift{3.598964in}{0.733809in}%
\pgfsys@useobject{currentmarker}{}%
\end{pgfscope}%
\begin{pgfscope}%
\pgfsys@transformshift{3.596937in}{0.733691in}%
\pgfsys@useobject{currentmarker}{}%
\end{pgfscope}%
\begin{pgfscope}%
\pgfsys@transformshift{3.595820in}{0.733680in}%
\pgfsys@useobject{currentmarker}{}%
\end{pgfscope}%
\begin{pgfscope}%
\pgfsys@transformshift{3.595206in}{0.733665in}%
\pgfsys@useobject{currentmarker}{}%
\end{pgfscope}%
\begin{pgfscope}%
\pgfsys@transformshift{3.589149in}{0.734071in}%
\pgfsys@useobject{currentmarker}{}%
\end{pgfscope}%
\begin{pgfscope}%
\pgfsys@transformshift{3.585810in}{0.734020in}%
\pgfsys@useobject{currentmarker}{}%
\end{pgfscope}%
\begin{pgfscope}%
\pgfsys@transformshift{3.572902in}{0.733660in}%
\pgfsys@useobject{currentmarker}{}%
\end{pgfscope}%
\begin{pgfscope}%
\pgfsys@transformshift{3.553268in}{0.733736in}%
\pgfsys@useobject{currentmarker}{}%
\end{pgfscope}%
\begin{pgfscope}%
\pgfsys@transformshift{3.524468in}{0.732205in}%
\pgfsys@useobject{currentmarker}{}%
\end{pgfscope}%
\begin{pgfscope}%
\pgfsys@transformshift{3.486127in}{0.732712in}%
\pgfsys@useobject{currentmarker}{}%
\end{pgfscope}%
\begin{pgfscope}%
\pgfsys@transformshift{3.443876in}{0.729817in}%
\pgfsys@useobject{currentmarker}{}%
\end{pgfscope}%
\begin{pgfscope}%
\pgfsys@transformshift{3.395830in}{0.728662in}%
\pgfsys@useobject{currentmarker}{}%
\end{pgfscope}%
\begin{pgfscope}%
\pgfsys@transformshift{3.344655in}{0.726243in}%
\pgfsys@useobject{currentmarker}{}%
\end{pgfscope}%
\begin{pgfscope}%
\pgfsys@transformshift{3.289711in}{0.726448in}%
\pgfsys@useobject{currentmarker}{}%
\end{pgfscope}%
\begin{pgfscope}%
\pgfsys@transformshift{3.259514in}{0.725308in}%
\pgfsys@useobject{currentmarker}{}%
\end{pgfscope}%
\begin{pgfscope}%
\pgfsys@transformshift{3.242895in}{0.725074in}%
\pgfsys@useobject{currentmarker}{}%
\end{pgfscope}%
\begin{pgfscope}%
\pgfsys@transformshift{3.233758in}{0.724782in}%
\pgfsys@useobject{currentmarker}{}%
\end{pgfscope}%
\begin{pgfscope}%
\pgfsys@transformshift{3.228730in}{0.724746in}%
\pgfsys@useobject{currentmarker}{}%
\end{pgfscope}%
\begin{pgfscope}%
\pgfsys@transformshift{3.225965in}{0.724738in}%
\pgfsys@useobject{currentmarker}{}%
\end{pgfscope}%
\begin{pgfscope}%
\pgfsys@transformshift{3.224444in}{0.724746in}%
\pgfsys@useobject{currentmarker}{}%
\end{pgfscope}%
\begin{pgfscope}%
\pgfsys@transformshift{3.219609in}{0.724688in}%
\pgfsys@useobject{currentmarker}{}%
\end{pgfscope}%
\begin{pgfscope}%
\pgfsys@transformshift{3.211898in}{0.724439in}%
\pgfsys@useobject{currentmarker}{}%
\end{pgfscope}%
\begin{pgfscope}%
\pgfsys@transformshift{3.198864in}{0.723584in}%
\pgfsys@useobject{currentmarker}{}%
\end{pgfscope}%
\begin{pgfscope}%
\pgfsys@transformshift{3.180201in}{0.725209in}%
\pgfsys@useobject{currentmarker}{}%
\end{pgfscope}%
\begin{pgfscope}%
\pgfsys@transformshift{3.156279in}{0.724376in}%
\pgfsys@useobject{currentmarker}{}%
\end{pgfscope}%
\begin{pgfscope}%
\pgfsys@transformshift{3.143115in}{0.724358in}%
\pgfsys@useobject{currentmarker}{}%
\end{pgfscope}%
\begin{pgfscope}%
\pgfsys@transformshift{3.135896in}{0.723799in}%
\pgfsys@useobject{currentmarker}{}%
\end{pgfscope}%
\begin{pgfscope}%
\pgfsys@transformshift{3.131915in}{0.723910in}%
\pgfsys@useobject{currentmarker}{}%
\end{pgfscope}%
\begin{pgfscope}%
\pgfsys@transformshift{3.123760in}{0.723458in}%
\pgfsys@useobject{currentmarker}{}%
\end{pgfscope}%
\begin{pgfscope}%
\pgfsys@transformshift{3.109677in}{0.722839in}%
\pgfsys@useobject{currentmarker}{}%
\end{pgfscope}%
\begin{pgfscope}%
\pgfsys@transformshift{3.087905in}{0.723106in}%
\pgfsys@useobject{currentmarker}{}%
\end{pgfscope}%
\begin{pgfscope}%
\pgfsys@transformshift{3.058248in}{0.722211in}%
\pgfsys@useobject{currentmarker}{}%
\end{pgfscope}%
\begin{pgfscope}%
\pgfsys@transformshift{3.025096in}{0.720542in}%
\pgfsys@useobject{currentmarker}{}%
\end{pgfscope}%
\begin{pgfscope}%
\pgfsys@transformshift{3.006842in}{0.720230in}%
\pgfsys@useobject{currentmarker}{}%
\end{pgfscope}%
\begin{pgfscope}%
\pgfsys@transformshift{2.996810in}{0.719815in}%
\pgfsys@useobject{currentmarker}{}%
\end{pgfscope}%
\begin{pgfscope}%
\pgfsys@transformshift{2.979599in}{0.718581in}%
\pgfsys@useobject{currentmarker}{}%
\end{pgfscope}%
\begin{pgfscope}%
\pgfsys@transformshift{2.954587in}{0.717423in}%
\pgfsys@useobject{currentmarker}{}%
\end{pgfscope}%
\begin{pgfscope}%
\pgfsys@transformshift{2.940829in}{0.716827in}%
\pgfsys@useobject{currentmarker}{}%
\end{pgfscope}%
\begin{pgfscope}%
\pgfsys@transformshift{2.933264in}{0.717186in}%
\pgfsys@useobject{currentmarker}{}%
\end{pgfscope}%
\begin{pgfscope}%
\pgfsys@transformshift{2.929098in}{0.717192in}%
\pgfsys@useobject{currentmarker}{}%
\end{pgfscope}%
\begin{pgfscope}%
\pgfsys@transformshift{2.917093in}{0.717525in}%
\pgfsys@useobject{currentmarker}{}%
\end{pgfscope}%
\begin{pgfscope}%
\pgfsys@transformshift{2.895079in}{0.717100in}%
\pgfsys@useobject{currentmarker}{}%
\end{pgfscope}%
\begin{pgfscope}%
\pgfsys@transformshift{2.870134in}{0.716639in}%
\pgfsys@useobject{currentmarker}{}%
\end{pgfscope}%
\begin{pgfscope}%
\pgfsys@transformshift{2.856439in}{0.715783in}%
\pgfsys@useobject{currentmarker}{}%
\end{pgfscope}%
\begin{pgfscope}%
\pgfsys@transformshift{2.848920in}{0.715131in}%
\pgfsys@useobject{currentmarker}{}%
\end{pgfscope}%
\begin{pgfscope}%
\pgfsys@transformshift{2.829369in}{0.715254in}%
\pgfsys@useobject{currentmarker}{}%
\end{pgfscope}%
\begin{pgfscope}%
\pgfsys@transformshift{2.799319in}{0.712608in}%
\pgfsys@useobject{currentmarker}{}%
\end{pgfscope}%
\begin{pgfscope}%
\pgfsys@transformshift{2.764552in}{0.713151in}%
\pgfsys@useobject{currentmarker}{}%
\end{pgfscope}%
\begin{pgfscope}%
\pgfsys@transformshift{2.724025in}{0.707821in}%
\pgfsys@useobject{currentmarker}{}%
\end{pgfscope}%
\begin{pgfscope}%
\pgfsys@transformshift{2.701569in}{0.706770in}%
\pgfsys@useobject{currentmarker}{}%
\end{pgfscope}%
\begin{pgfscope}%
\pgfsys@transformshift{2.689206in}{0.706530in}%
\pgfsys@useobject{currentmarker}{}%
\end{pgfscope}%
\begin{pgfscope}%
\pgfsys@transformshift{2.667388in}{0.704694in}%
\pgfsys@useobject{currentmarker}{}%
\end{pgfscope}%
\begin{pgfscope}%
\pgfsys@transformshift{2.634257in}{0.704331in}%
\pgfsys@useobject{currentmarker}{}%
\end{pgfscope}%
\begin{pgfscope}%
\pgfsys@transformshift{2.593779in}{0.701169in}%
\pgfsys@useobject{currentmarker}{}%
\end{pgfscope}%
\begin{pgfscope}%
\pgfsys@transformshift{2.571488in}{0.699846in}%
\pgfsys@useobject{currentmarker}{}%
\end{pgfscope}%
\begin{pgfscope}%
\pgfsys@transformshift{2.559211in}{0.699504in}%
\pgfsys@useobject{currentmarker}{}%
\end{pgfscope}%
\begin{pgfscope}%
\pgfsys@transformshift{2.543725in}{0.698224in}%
\pgfsys@useobject{currentmarker}{}%
\end{pgfscope}%
\begin{pgfscope}%
\pgfsys@transformshift{2.517102in}{0.698083in}%
\pgfsys@useobject{currentmarker}{}%
\end{pgfscope}%
\begin{pgfscope}%
\pgfsys@transformshift{2.485021in}{0.693923in}%
\pgfsys@useobject{currentmarker}{}%
\end{pgfscope}%
\begin{pgfscope}%
\pgfsys@transformshift{2.448759in}{0.693545in}%
\pgfsys@useobject{currentmarker}{}%
\end{pgfscope}%
\begin{pgfscope}%
\pgfsys@transformshift{2.428817in}{0.693897in}%
\pgfsys@useobject{currentmarker}{}%
\end{pgfscope}%
\begin{pgfscope}%
\pgfsys@transformshift{2.405221in}{0.693243in}%
\pgfsys@useobject{currentmarker}{}%
\end{pgfscope}%
\begin{pgfscope}%
\pgfsys@transformshift{2.374434in}{0.693680in}%
\pgfsys@useobject{currentmarker}{}%
\end{pgfscope}%
\begin{pgfscope}%
\pgfsys@transformshift{2.333978in}{0.692953in}%
\pgfsys@useobject{currentmarker}{}%
\end{pgfscope}%
\begin{pgfscope}%
\pgfsys@transformshift{2.290653in}{0.692448in}%
\pgfsys@useobject{currentmarker}{}%
\end{pgfscope}%
\begin{pgfscope}%
\pgfsys@transformshift{2.266832in}{0.693135in}%
\pgfsys@useobject{currentmarker}{}%
\end{pgfscope}%
\begin{pgfscope}%
\pgfsys@transformshift{2.239372in}{0.691948in}%
\pgfsys@useobject{currentmarker}{}%
\end{pgfscope}%
\begin{pgfscope}%
\pgfsys@transformshift{2.203855in}{0.694109in}%
\pgfsys@useobject{currentmarker}{}%
\end{pgfscope}%
\begin{pgfscope}%
\pgfsys@transformshift{2.161128in}{0.690697in}%
\pgfsys@useobject{currentmarker}{}%
\end{pgfscope}%
\begin{pgfscope}%
\pgfsys@transformshift{2.113925in}{0.688994in}%
\pgfsys@useobject{currentmarker}{}%
\end{pgfscope}%
\begin{pgfscope}%
\pgfsys@transformshift{2.087955in}{0.689646in}%
\pgfsys@useobject{currentmarker}{}%
\end{pgfscope}%
\begin{pgfscope}%
\pgfsys@transformshift{2.058368in}{0.688347in}%
\pgfsys@useobject{currentmarker}{}%
\end{pgfscope}%
\begin{pgfscope}%
\pgfsys@transformshift{2.021499in}{0.691086in}%
\pgfsys@useobject{currentmarker}{}%
\end{pgfscope}%
\begin{pgfscope}%
\pgfsys@transformshift{1.975125in}{0.691268in}%
\pgfsys@useobject{currentmarker}{}%
\end{pgfscope}%
\begin{pgfscope}%
\pgfsys@transformshift{1.924857in}{0.693211in}%
\pgfsys@useobject{currentmarker}{}%
\end{pgfscope}%
\begin{pgfscope}%
\pgfsys@transformshift{1.939105in}{0.691224in}%
\pgfsys@useobject{currentmarker}{}%
\end{pgfscope}%
\begin{pgfscope}%
\pgfsys@transformshift{1.970389in}{0.689540in}%
\pgfsys@useobject{currentmarker}{}%
\end{pgfscope}%
\begin{pgfscope}%
\pgfsys@transformshift{2.011384in}{0.691154in}%
\pgfsys@useobject{currentmarker}{}%
\end{pgfscope}%
\begin{pgfscope}%
\pgfsys@transformshift{2.057116in}{0.691500in}%
\pgfsys@useobject{currentmarker}{}%
\end{pgfscope}%
\begin{pgfscope}%
\pgfsys@transformshift{2.106303in}{0.691268in}%
\pgfsys@useobject{currentmarker}{}%
\end{pgfscope}%
\begin{pgfscope}%
\pgfsys@transformshift{2.158639in}{0.690456in}%
\pgfsys@useobject{currentmarker}{}%
\end{pgfscope}%
\begin{pgfscope}%
\pgfsys@transformshift{2.187376in}{0.688736in}%
\pgfsys@useobject{currentmarker}{}%
\end{pgfscope}%
\begin{pgfscope}%
\pgfsys@transformshift{2.203181in}{0.687786in}%
\pgfsys@useobject{currentmarker}{}%
\end{pgfscope}%
\begin{pgfscope}%
\pgfsys@transformshift{2.222462in}{0.687290in}%
\pgfsys@useobject{currentmarker}{}%
\end{pgfscope}%
\begin{pgfscope}%
\pgfsys@transformshift{2.251513in}{0.685981in}%
\pgfsys@useobject{currentmarker}{}%
\end{pgfscope}%
\begin{pgfscope}%
\pgfsys@transformshift{2.287787in}{0.684599in}%
\pgfsys@useobject{currentmarker}{}%
\end{pgfscope}%
\begin{pgfscope}%
\pgfsys@transformshift{2.327668in}{0.683000in}%
\pgfsys@useobject{currentmarker}{}%
\end{pgfscope}%
\begin{pgfscope}%
\pgfsys@transformshift{2.349524in}{0.685048in}%
\pgfsys@useobject{currentmarker}{}%
\end{pgfscope}%
\begin{pgfscope}%
\pgfsys@transformshift{2.375688in}{0.683575in}%
\pgfsys@useobject{currentmarker}{}%
\end{pgfscope}%
\begin{pgfscope}%
\pgfsys@transformshift{2.409266in}{0.685762in}%
\pgfsys@useobject{currentmarker}{}%
\end{pgfscope}%
\begin{pgfscope}%
\pgfsys@transformshift{2.448741in}{0.687888in}%
\pgfsys@useobject{currentmarker}{}%
\end{pgfscope}%
\begin{pgfscope}%
\pgfsys@transformshift{2.497977in}{0.686448in}%
\pgfsys@useobject{currentmarker}{}%
\end{pgfscope}%
\begin{pgfscope}%
\pgfsys@transformshift{2.550930in}{0.688775in}%
\pgfsys@useobject{currentmarker}{}%
\end{pgfscope}%
\begin{pgfscope}%
\pgfsys@transformshift{2.580038in}{0.690364in}%
\pgfsys@useobject{currentmarker}{}%
\end{pgfscope}%
\begin{pgfscope}%
\pgfsys@transformshift{2.595848in}{0.693036in}%
\pgfsys@useobject{currentmarker}{}%
\end{pgfscope}%
\begin{pgfscope}%
\pgfsys@transformshift{2.614448in}{0.696356in}%
\pgfsys@useobject{currentmarker}{}%
\end{pgfscope}%
\begin{pgfscope}%
\pgfsys@transformshift{2.624827in}{0.696879in}%
\pgfsys@useobject{currentmarker}{}%
\end{pgfscope}%
\begin{pgfscope}%
\pgfsys@transformshift{2.630488in}{0.697668in}%
\pgfsys@useobject{currentmarker}{}%
\end{pgfscope}%
\begin{pgfscope}%
\pgfsys@transformshift{2.640646in}{0.698780in}%
\pgfsys@useobject{currentmarker}{}%
\end{pgfscope}%
\begin{pgfscope}%
\pgfsys@transformshift{2.655497in}{0.701928in}%
\pgfsys@useobject{currentmarker}{}%
\end{pgfscope}%
\begin{pgfscope}%
\pgfsys@transformshift{2.674796in}{0.703027in}%
\pgfsys@useobject{currentmarker}{}%
\end{pgfscope}%
\begin{pgfscope}%
\pgfsys@transformshift{2.696905in}{0.708238in}%
\pgfsys@useobject{currentmarker}{}%
\end{pgfscope}%
\begin{pgfscope}%
\pgfsys@transformshift{2.721958in}{0.714144in}%
\pgfsys@useobject{currentmarker}{}%
\end{pgfscope}%
\end{pgfscope}%
\begin{pgfscope}%
\pgfsetbuttcap%
\pgfsetroundjoin%
\definecolor{currentfill}{rgb}{0.000000,0.000000,0.000000}%
\pgfsetfillcolor{currentfill}%
\pgfsetlinewidth{0.803000pt}%
\definecolor{currentstroke}{rgb}{0.000000,0.000000,0.000000}%
\pgfsetstrokecolor{currentstroke}%
\pgfsetdash{}{0pt}%
\pgfsys@defobject{currentmarker}{\pgfqpoint{0.000000in}{-0.048611in}}{\pgfqpoint{0.000000in}{0.000000in}}{%
\pgfpathmoveto{\pgfqpoint{0.000000in}{0.000000in}}%
\pgfpathlineto{\pgfqpoint{0.000000in}{-0.048611in}}%
\pgfusepath{stroke,fill}%
}%
\begin{pgfscope}%
\pgfsys@transformshift{0.787200in}{0.515000in}%
\pgfsys@useobject{currentmarker}{}%
\end{pgfscope}%
\end{pgfscope}%
\begin{pgfscope}%
\definecolor{textcolor}{rgb}{0.000000,0.000000,0.000000}%
\pgfsetstrokecolor{textcolor}%
\pgfsetfillcolor{textcolor}%
\pgftext[x=0.787200in,y=0.417777in,,top]{\color{textcolor}\rmfamily\fontsize{10.000000}{12.000000}\selectfont \(\displaystyle {−2}\)}%
\end{pgfscope}%
\begin{pgfscope}%
\pgfsetbuttcap%
\pgfsetroundjoin%
\definecolor{currentfill}{rgb}{0.000000,0.000000,0.000000}%
\pgfsetfillcolor{currentfill}%
\pgfsetlinewidth{0.803000pt}%
\definecolor{currentstroke}{rgb}{0.000000,0.000000,0.000000}%
\pgfsetstrokecolor{currentstroke}%
\pgfsetdash{}{0pt}%
\pgfsys@defobject{currentmarker}{\pgfqpoint{0.000000in}{-0.048611in}}{\pgfqpoint{0.000000in}{0.000000in}}{%
\pgfpathmoveto{\pgfqpoint{0.000000in}{0.000000in}}%
\pgfpathlineto{\pgfqpoint{0.000000in}{-0.048611in}}%
\pgfusepath{stroke,fill}%
}%
\begin{pgfscope}%
\pgfsys@transformshift{1.352691in}{0.515000in}%
\pgfsys@useobject{currentmarker}{}%
\end{pgfscope}%
\end{pgfscope}%
\begin{pgfscope}%
\definecolor{textcolor}{rgb}{0.000000,0.000000,0.000000}%
\pgfsetstrokecolor{textcolor}%
\pgfsetfillcolor{textcolor}%
\pgftext[x=1.352691in,y=0.417777in,,top]{\color{textcolor}\rmfamily\fontsize{10.000000}{12.000000}\selectfont \(\displaystyle {−1}\)}%
\end{pgfscope}%
\begin{pgfscope}%
\pgfsetbuttcap%
\pgfsetroundjoin%
\definecolor{currentfill}{rgb}{0.000000,0.000000,0.000000}%
\pgfsetfillcolor{currentfill}%
\pgfsetlinewidth{0.803000pt}%
\definecolor{currentstroke}{rgb}{0.000000,0.000000,0.000000}%
\pgfsetstrokecolor{currentstroke}%
\pgfsetdash{}{0pt}%
\pgfsys@defobject{currentmarker}{\pgfqpoint{0.000000in}{-0.048611in}}{\pgfqpoint{0.000000in}{0.000000in}}{%
\pgfpathmoveto{\pgfqpoint{0.000000in}{0.000000in}}%
\pgfpathlineto{\pgfqpoint{0.000000in}{-0.048611in}}%
\pgfusepath{stroke,fill}%
}%
\begin{pgfscope}%
\pgfsys@transformshift{1.918183in}{0.515000in}%
\pgfsys@useobject{currentmarker}{}%
\end{pgfscope}%
\end{pgfscope}%
\begin{pgfscope}%
\definecolor{textcolor}{rgb}{0.000000,0.000000,0.000000}%
\pgfsetstrokecolor{textcolor}%
\pgfsetfillcolor{textcolor}%
\pgftext[x=1.918183in,y=0.417777in,,top]{\color{textcolor}\rmfamily\fontsize{10.000000}{12.000000}\selectfont \(\displaystyle {0}\)}%
\end{pgfscope}%
\begin{pgfscope}%
\pgfsetbuttcap%
\pgfsetroundjoin%
\definecolor{currentfill}{rgb}{0.000000,0.000000,0.000000}%
\pgfsetfillcolor{currentfill}%
\pgfsetlinewidth{0.803000pt}%
\definecolor{currentstroke}{rgb}{0.000000,0.000000,0.000000}%
\pgfsetstrokecolor{currentstroke}%
\pgfsetdash{}{0pt}%
\pgfsys@defobject{currentmarker}{\pgfqpoint{0.000000in}{-0.048611in}}{\pgfqpoint{0.000000in}{0.000000in}}{%
\pgfpathmoveto{\pgfqpoint{0.000000in}{0.000000in}}%
\pgfpathlineto{\pgfqpoint{0.000000in}{-0.048611in}}%
\pgfusepath{stroke,fill}%
}%
\begin{pgfscope}%
\pgfsys@transformshift{2.483675in}{0.515000in}%
\pgfsys@useobject{currentmarker}{}%
\end{pgfscope}%
\end{pgfscope}%
\begin{pgfscope}%
\definecolor{textcolor}{rgb}{0.000000,0.000000,0.000000}%
\pgfsetstrokecolor{textcolor}%
\pgfsetfillcolor{textcolor}%
\pgftext[x=2.483675in,y=0.417777in,,top]{\color{textcolor}\rmfamily\fontsize{10.000000}{12.000000}\selectfont \(\displaystyle {1}\)}%
\end{pgfscope}%
\begin{pgfscope}%
\pgfsetbuttcap%
\pgfsetroundjoin%
\definecolor{currentfill}{rgb}{0.000000,0.000000,0.000000}%
\pgfsetfillcolor{currentfill}%
\pgfsetlinewidth{0.803000pt}%
\definecolor{currentstroke}{rgb}{0.000000,0.000000,0.000000}%
\pgfsetstrokecolor{currentstroke}%
\pgfsetdash{}{0pt}%
\pgfsys@defobject{currentmarker}{\pgfqpoint{0.000000in}{-0.048611in}}{\pgfqpoint{0.000000in}{0.000000in}}{%
\pgfpathmoveto{\pgfqpoint{0.000000in}{0.000000in}}%
\pgfpathlineto{\pgfqpoint{0.000000in}{-0.048611in}}%
\pgfusepath{stroke,fill}%
}%
\begin{pgfscope}%
\pgfsys@transformshift{3.049167in}{0.515000in}%
\pgfsys@useobject{currentmarker}{}%
\end{pgfscope}%
\end{pgfscope}%
\begin{pgfscope}%
\definecolor{textcolor}{rgb}{0.000000,0.000000,0.000000}%
\pgfsetstrokecolor{textcolor}%
\pgfsetfillcolor{textcolor}%
\pgftext[x=3.049167in,y=0.417777in,,top]{\color{textcolor}\rmfamily\fontsize{10.000000}{12.000000}\selectfont \(\displaystyle {2}\)}%
\end{pgfscope}%
\begin{pgfscope}%
\pgfsetbuttcap%
\pgfsetroundjoin%
\definecolor{currentfill}{rgb}{0.000000,0.000000,0.000000}%
\pgfsetfillcolor{currentfill}%
\pgfsetlinewidth{0.803000pt}%
\definecolor{currentstroke}{rgb}{0.000000,0.000000,0.000000}%
\pgfsetstrokecolor{currentstroke}%
\pgfsetdash{}{0pt}%
\pgfsys@defobject{currentmarker}{\pgfqpoint{0.000000in}{-0.048611in}}{\pgfqpoint{0.000000in}{0.000000in}}{%
\pgfpathmoveto{\pgfqpoint{0.000000in}{0.000000in}}%
\pgfpathlineto{\pgfqpoint{0.000000in}{-0.048611in}}%
\pgfusepath{stroke,fill}%
}%
\begin{pgfscope}%
\pgfsys@transformshift{3.614658in}{0.515000in}%
\pgfsys@useobject{currentmarker}{}%
\end{pgfscope}%
\end{pgfscope}%
\begin{pgfscope}%
\definecolor{textcolor}{rgb}{0.000000,0.000000,0.000000}%
\pgfsetstrokecolor{textcolor}%
\pgfsetfillcolor{textcolor}%
\pgftext[x=3.614658in,y=0.417777in,,top]{\color{textcolor}\rmfamily\fontsize{10.000000}{12.000000}\selectfont \(\displaystyle {3}\)}%
\end{pgfscope}%
\begin{pgfscope}%
\pgfsetbuttcap%
\pgfsetroundjoin%
\definecolor{currentfill}{rgb}{0.000000,0.000000,0.000000}%
\pgfsetfillcolor{currentfill}%
\pgfsetlinewidth{0.803000pt}%
\definecolor{currentstroke}{rgb}{0.000000,0.000000,0.000000}%
\pgfsetstrokecolor{currentstroke}%
\pgfsetdash{}{0pt}%
\pgfsys@defobject{currentmarker}{\pgfqpoint{0.000000in}{-0.048611in}}{\pgfqpoint{0.000000in}{0.000000in}}{%
\pgfpathmoveto{\pgfqpoint{0.000000in}{0.000000in}}%
\pgfpathlineto{\pgfqpoint{0.000000in}{-0.048611in}}%
\pgfusepath{stroke,fill}%
}%
\begin{pgfscope}%
\pgfsys@transformshift{4.180150in}{0.515000in}%
\pgfsys@useobject{currentmarker}{}%
\end{pgfscope}%
\end{pgfscope}%
\begin{pgfscope}%
\definecolor{textcolor}{rgb}{0.000000,0.000000,0.000000}%
\pgfsetstrokecolor{textcolor}%
\pgfsetfillcolor{textcolor}%
\pgftext[x=4.180150in,y=0.417777in,,top]{\color{textcolor}\rmfamily\fontsize{10.000000}{12.000000}\selectfont \(\displaystyle {4}\)}%
\end{pgfscope}%
\begin{pgfscope}%
\pgfsetbuttcap%
\pgfsetroundjoin%
\definecolor{currentfill}{rgb}{0.000000,0.000000,0.000000}%
\pgfsetfillcolor{currentfill}%
\pgfsetlinewidth{0.803000pt}%
\definecolor{currentstroke}{rgb}{0.000000,0.000000,0.000000}%
\pgfsetstrokecolor{currentstroke}%
\pgfsetdash{}{0pt}%
\pgfsys@defobject{currentmarker}{\pgfqpoint{0.000000in}{-0.048611in}}{\pgfqpoint{0.000000in}{0.000000in}}{%
\pgfpathmoveto{\pgfqpoint{0.000000in}{0.000000in}}%
\pgfpathlineto{\pgfqpoint{0.000000in}{-0.048611in}}%
\pgfusepath{stroke,fill}%
}%
\begin{pgfscope}%
\pgfsys@transformshift{4.745642in}{0.515000in}%
\pgfsys@useobject{currentmarker}{}%
\end{pgfscope}%
\end{pgfscope}%
\begin{pgfscope}%
\definecolor{textcolor}{rgb}{0.000000,0.000000,0.000000}%
\pgfsetstrokecolor{textcolor}%
\pgfsetfillcolor{textcolor}%
\pgftext[x=4.745642in,y=0.417777in,,top]{\color{textcolor}\rmfamily\fontsize{10.000000}{12.000000}\selectfont \(\displaystyle {5}\)}%
\end{pgfscope}%
\begin{pgfscope}%
\pgfsetbuttcap%
\pgfsetroundjoin%
\definecolor{currentfill}{rgb}{0.000000,0.000000,0.000000}%
\pgfsetfillcolor{currentfill}%
\pgfsetlinewidth{0.803000pt}%
\definecolor{currentstroke}{rgb}{0.000000,0.000000,0.000000}%
\pgfsetstrokecolor{currentstroke}%
\pgfsetdash{}{0pt}%
\pgfsys@defobject{currentmarker}{\pgfqpoint{0.000000in}{-0.048611in}}{\pgfqpoint{0.000000in}{0.000000in}}{%
\pgfpathmoveto{\pgfqpoint{0.000000in}{0.000000in}}%
\pgfpathlineto{\pgfqpoint{0.000000in}{-0.048611in}}%
\pgfusepath{stroke,fill}%
}%
\begin{pgfscope}%
\pgfsys@transformshift{5.311134in}{0.515000in}%
\pgfsys@useobject{currentmarker}{}%
\end{pgfscope}%
\end{pgfscope}%
\begin{pgfscope}%
\definecolor{textcolor}{rgb}{0.000000,0.000000,0.000000}%
\pgfsetstrokecolor{textcolor}%
\pgfsetfillcolor{textcolor}%
\pgftext[x=5.311134in,y=0.417777in,,top]{\color{textcolor}\rmfamily\fontsize{10.000000}{12.000000}\selectfont \(\displaystyle {6}\)}%
\end{pgfscope}%
\begin{pgfscope}%
\definecolor{textcolor}{rgb}{0.000000,0.000000,0.000000}%
\pgfsetstrokecolor{textcolor}%
\pgfsetfillcolor{textcolor}%
\pgftext[x=3.049167in,y=0.238889in,,top]{\color{textcolor}\rmfamily\fontsize{10.000000}{12.000000}\selectfont Position X [\(\displaystyle m\)]}%
\end{pgfscope}%
\begin{pgfscope}%
\pgfsetbuttcap%
\pgfsetroundjoin%
\definecolor{currentfill}{rgb}{0.000000,0.000000,0.000000}%
\pgfsetfillcolor{currentfill}%
\pgfsetlinewidth{0.803000pt}%
\definecolor{currentstroke}{rgb}{0.000000,0.000000,0.000000}%
\pgfsetstrokecolor{currentstroke}%
\pgfsetdash{}{0pt}%
\pgfsys@defobject{currentmarker}{\pgfqpoint{-0.048611in}{0.000000in}}{\pgfqpoint{-0.000000in}{0.000000in}}{%
\pgfpathmoveto{\pgfqpoint{-0.000000in}{0.000000in}}%
\pgfpathlineto{\pgfqpoint{-0.048611in}{0.000000in}}%
\pgfusepath{stroke,fill}%
}%
\begin{pgfscope}%
\pgfsys@transformshift{0.569167in}{0.926983in}%
\pgfsys@useobject{currentmarker}{}%
\end{pgfscope}%
\end{pgfscope}%
\begin{pgfscope}%
\definecolor{textcolor}{rgb}{0.000000,0.000000,0.000000}%
\pgfsetstrokecolor{textcolor}%
\pgfsetfillcolor{textcolor}%
\pgftext[x=0.294444in, y=0.878788in, left, base]{\color{textcolor}\rmfamily\fontsize{10.000000}{12.000000}\selectfont \(\displaystyle {−1}\)}%
\end{pgfscope}%
\begin{pgfscope}%
\pgfsetbuttcap%
\pgfsetroundjoin%
\definecolor{currentfill}{rgb}{0.000000,0.000000,0.000000}%
\pgfsetfillcolor{currentfill}%
\pgfsetlinewidth{0.803000pt}%
\definecolor{currentstroke}{rgb}{0.000000,0.000000,0.000000}%
\pgfsetstrokecolor{currentstroke}%
\pgfsetdash{}{0pt}%
\pgfsys@defobject{currentmarker}{\pgfqpoint{-0.048611in}{0.000000in}}{\pgfqpoint{-0.000000in}{0.000000in}}{%
\pgfpathmoveto{\pgfqpoint{-0.000000in}{0.000000in}}%
\pgfpathlineto{\pgfqpoint{-0.048611in}{0.000000in}}%
\pgfusepath{stroke,fill}%
}%
\begin{pgfscope}%
\pgfsys@transformshift{0.569167in}{1.492474in}%
\pgfsys@useobject{currentmarker}{}%
\end{pgfscope}%
\end{pgfscope}%
\begin{pgfscope}%
\definecolor{textcolor}{rgb}{0.000000,0.000000,0.000000}%
\pgfsetstrokecolor{textcolor}%
\pgfsetfillcolor{textcolor}%
\pgftext[x=0.402500in, y=1.444280in, left, base]{\color{textcolor}\rmfamily\fontsize{10.000000}{12.000000}\selectfont \(\displaystyle {0}\)}%
\end{pgfscope}%
\begin{pgfscope}%
\pgfsetbuttcap%
\pgfsetroundjoin%
\definecolor{currentfill}{rgb}{0.000000,0.000000,0.000000}%
\pgfsetfillcolor{currentfill}%
\pgfsetlinewidth{0.803000pt}%
\definecolor{currentstroke}{rgb}{0.000000,0.000000,0.000000}%
\pgfsetstrokecolor{currentstroke}%
\pgfsetdash{}{0pt}%
\pgfsys@defobject{currentmarker}{\pgfqpoint{-0.048611in}{0.000000in}}{\pgfqpoint{-0.000000in}{0.000000in}}{%
\pgfpathmoveto{\pgfqpoint{-0.000000in}{0.000000in}}%
\pgfpathlineto{\pgfqpoint{-0.048611in}{0.000000in}}%
\pgfusepath{stroke,fill}%
}%
\begin{pgfscope}%
\pgfsys@transformshift{0.569167in}{2.057966in}%
\pgfsys@useobject{currentmarker}{}%
\end{pgfscope}%
\end{pgfscope}%
\begin{pgfscope}%
\definecolor{textcolor}{rgb}{0.000000,0.000000,0.000000}%
\pgfsetstrokecolor{textcolor}%
\pgfsetfillcolor{textcolor}%
\pgftext[x=0.402500in, y=2.009772in, left, base]{\color{textcolor}\rmfamily\fontsize{10.000000}{12.000000}\selectfont \(\displaystyle {1}\)}%
\end{pgfscope}%
\begin{pgfscope}%
\pgfsetbuttcap%
\pgfsetroundjoin%
\definecolor{currentfill}{rgb}{0.000000,0.000000,0.000000}%
\pgfsetfillcolor{currentfill}%
\pgfsetlinewidth{0.803000pt}%
\definecolor{currentstroke}{rgb}{0.000000,0.000000,0.000000}%
\pgfsetstrokecolor{currentstroke}%
\pgfsetdash{}{0pt}%
\pgfsys@defobject{currentmarker}{\pgfqpoint{-0.048611in}{0.000000in}}{\pgfqpoint{-0.000000in}{0.000000in}}{%
\pgfpathmoveto{\pgfqpoint{-0.000000in}{0.000000in}}%
\pgfpathlineto{\pgfqpoint{-0.048611in}{0.000000in}}%
\pgfusepath{stroke,fill}%
}%
\begin{pgfscope}%
\pgfsys@transformshift{0.569167in}{2.623458in}%
\pgfsys@useobject{currentmarker}{}%
\end{pgfscope}%
\end{pgfscope}%
\begin{pgfscope}%
\definecolor{textcolor}{rgb}{0.000000,0.000000,0.000000}%
\pgfsetstrokecolor{textcolor}%
\pgfsetfillcolor{textcolor}%
\pgftext[x=0.402500in, y=2.575263in, left, base]{\color{textcolor}\rmfamily\fontsize{10.000000}{12.000000}\selectfont \(\displaystyle {2}\)}%
\end{pgfscope}%
\begin{pgfscope}%
\pgfsetbuttcap%
\pgfsetroundjoin%
\definecolor{currentfill}{rgb}{0.000000,0.000000,0.000000}%
\pgfsetfillcolor{currentfill}%
\pgfsetlinewidth{0.803000pt}%
\definecolor{currentstroke}{rgb}{0.000000,0.000000,0.000000}%
\pgfsetstrokecolor{currentstroke}%
\pgfsetdash{}{0pt}%
\pgfsys@defobject{currentmarker}{\pgfqpoint{-0.048611in}{0.000000in}}{\pgfqpoint{-0.000000in}{0.000000in}}{%
\pgfpathmoveto{\pgfqpoint{-0.000000in}{0.000000in}}%
\pgfpathlineto{\pgfqpoint{-0.048611in}{0.000000in}}%
\pgfusepath{stroke,fill}%
}%
\begin{pgfscope}%
\pgfsys@transformshift{0.569167in}{3.188950in}%
\pgfsys@useobject{currentmarker}{}%
\end{pgfscope}%
\end{pgfscope}%
\begin{pgfscope}%
\definecolor{textcolor}{rgb}{0.000000,0.000000,0.000000}%
\pgfsetstrokecolor{textcolor}%
\pgfsetfillcolor{textcolor}%
\pgftext[x=0.402500in, y=3.140755in, left, base]{\color{textcolor}\rmfamily\fontsize{10.000000}{12.000000}\selectfont \(\displaystyle {3}\)}%
\end{pgfscope}%
\begin{pgfscope}%
\pgfsetbuttcap%
\pgfsetroundjoin%
\definecolor{currentfill}{rgb}{0.000000,0.000000,0.000000}%
\pgfsetfillcolor{currentfill}%
\pgfsetlinewidth{0.803000pt}%
\definecolor{currentstroke}{rgb}{0.000000,0.000000,0.000000}%
\pgfsetstrokecolor{currentstroke}%
\pgfsetdash{}{0pt}%
\pgfsys@defobject{currentmarker}{\pgfqpoint{-0.048611in}{0.000000in}}{\pgfqpoint{-0.000000in}{0.000000in}}{%
\pgfpathmoveto{\pgfqpoint{-0.000000in}{0.000000in}}%
\pgfpathlineto{\pgfqpoint{-0.048611in}{0.000000in}}%
\pgfusepath{stroke,fill}%
}%
\begin{pgfscope}%
\pgfsys@transformshift{0.569167in}{3.754441in}%
\pgfsys@useobject{currentmarker}{}%
\end{pgfscope}%
\end{pgfscope}%
\begin{pgfscope}%
\definecolor{textcolor}{rgb}{0.000000,0.000000,0.000000}%
\pgfsetstrokecolor{textcolor}%
\pgfsetfillcolor{textcolor}%
\pgftext[x=0.402500in, y=3.706247in, left, base]{\color{textcolor}\rmfamily\fontsize{10.000000}{12.000000}\selectfont \(\displaystyle {4}\)}%
\end{pgfscope}%
\begin{pgfscope}%
\definecolor{textcolor}{rgb}{0.000000,0.000000,0.000000}%
\pgfsetstrokecolor{textcolor}%
\pgfsetfillcolor{textcolor}%
\pgftext[x=0.238889in,y=2.363000in,,bottom,rotate=90.000000]{\color{textcolor}\rmfamily\fontsize{10.000000}{12.000000}\selectfont Position Y [\(\displaystyle m\)]}%
\end{pgfscope}%
\begin{pgfscope}%
\pgfpathrectangle{\pgfqpoint{0.569167in}{0.515000in}}{\pgfqpoint{4.960000in}{3.696000in}}%
\pgfusepath{clip}%
\pgfsetrectcap%
\pgfsetroundjoin%
\pgfsetlinewidth{1.505625pt}%
\definecolor{currentstroke}{rgb}{0.121569,0.466667,0.705882}%
\pgfsetstrokecolor{currentstroke}%
\pgfsetdash{}{0pt}%
\pgfpathmoveto{\pgfqpoint{1.918183in}{1.492474in}}%
\pgfpathlineto{\pgfqpoint{1.918183in}{3.754441in}}%
\pgfpathlineto{\pgfqpoint{4.180150in}{1.492474in}}%
\pgfpathlineto{\pgfqpoint{1.918183in}{1.492474in}}%
\pgfusepath{stroke}%
\end{pgfscope}%
\begin{pgfscope}%
\pgfsetrectcap%
\pgfsetmiterjoin%
\pgfsetlinewidth{0.803000pt}%
\definecolor{currentstroke}{rgb}{0.000000,0.000000,0.000000}%
\pgfsetstrokecolor{currentstroke}%
\pgfsetdash{}{0pt}%
\pgfpathmoveto{\pgfqpoint{0.569167in}{0.515000in}}%
\pgfpathlineto{\pgfqpoint{0.569167in}{4.211000in}}%
\pgfusepath{stroke}%
\end{pgfscope}%
\begin{pgfscope}%
\pgfsetrectcap%
\pgfsetmiterjoin%
\pgfsetlinewidth{0.803000pt}%
\definecolor{currentstroke}{rgb}{0.000000,0.000000,0.000000}%
\pgfsetstrokecolor{currentstroke}%
\pgfsetdash{}{0pt}%
\pgfpathmoveto{\pgfqpoint{5.529167in}{0.515000in}}%
\pgfpathlineto{\pgfqpoint{5.529167in}{4.211000in}}%
\pgfusepath{stroke}%
\end{pgfscope}%
\begin{pgfscope}%
\pgfsetrectcap%
\pgfsetmiterjoin%
\pgfsetlinewidth{0.803000pt}%
\definecolor{currentstroke}{rgb}{0.000000,0.000000,0.000000}%
\pgfsetstrokecolor{currentstroke}%
\pgfsetdash{}{0pt}%
\pgfpathmoveto{\pgfqpoint{0.569167in}{0.515000in}}%
\pgfpathlineto{\pgfqpoint{5.529167in}{0.515000in}}%
\pgfusepath{stroke}%
\end{pgfscope}%
\begin{pgfscope}%
\pgfsetrectcap%
\pgfsetmiterjoin%
\pgfsetlinewidth{0.803000pt}%
\definecolor{currentstroke}{rgb}{0.000000,0.000000,0.000000}%
\pgfsetstrokecolor{currentstroke}%
\pgfsetdash{}{0pt}%
\pgfpathmoveto{\pgfqpoint{0.569167in}{4.211000in}}%
\pgfpathlineto{\pgfqpoint{5.529167in}{4.211000in}}%
\pgfusepath{stroke}%
\end{pgfscope}%
\begin{pgfscope}%
\pgfsetbuttcap%
\pgfsetmiterjoin%
\definecolor{currentfill}{rgb}{1.000000,1.000000,1.000000}%
\pgfsetfillcolor{currentfill}%
\pgfsetfillopacity{0.800000}%
\pgfsetlinewidth{1.003750pt}%
\definecolor{currentstroke}{rgb}{0.800000,0.800000,0.800000}%
\pgfsetstrokecolor{currentstroke}%
\pgfsetstrokeopacity{0.800000}%
\pgfsetdash{}{0pt}%
\pgfpathmoveto{\pgfqpoint{3.838055in}{3.712667in}}%
\pgfpathlineto{\pgfqpoint{5.431944in}{3.712667in}}%
\pgfpathquadraticcurveto{\pgfqpoint{5.459722in}{3.712667in}}{\pgfqpoint{5.459722in}{3.740444in}}%
\pgfpathlineto{\pgfqpoint{5.459722in}{4.113777in}}%
\pgfpathquadraticcurveto{\pgfqpoint{5.459722in}{4.141555in}}{\pgfqpoint{5.431944in}{4.141555in}}%
\pgfpathlineto{\pgfqpoint{3.838055in}{4.141555in}}%
\pgfpathquadraticcurveto{\pgfqpoint{3.810278in}{4.141555in}}{\pgfqpoint{3.810278in}{4.113777in}}%
\pgfpathlineto{\pgfqpoint{3.810278in}{3.740444in}}%
\pgfpathquadraticcurveto{\pgfqpoint{3.810278in}{3.712667in}}{\pgfqpoint{3.838055in}{3.712667in}}%
\pgfpathclose%
\pgfusepath{stroke,fill}%
\end{pgfscope}%
\begin{pgfscope}%
\pgfsetrectcap%
\pgfsetroundjoin%
\pgfsetlinewidth{1.505625pt}%
\definecolor{currentstroke}{rgb}{0.121569,0.466667,0.705882}%
\pgfsetstrokecolor{currentstroke}%
\pgfsetdash{}{0pt}%
\pgfpathmoveto{\pgfqpoint{3.865833in}{4.037388in}}%
\pgfpathlineto{\pgfqpoint{4.143611in}{4.037388in}}%
\pgfusepath{stroke}%
\end{pgfscope}%
\begin{pgfscope}%
\definecolor{textcolor}{rgb}{0.000000,0.000000,0.000000}%
\pgfsetstrokecolor{textcolor}%
\pgfsetfillcolor{textcolor}%
\pgftext[x=4.254722in,y=3.988777in,left,base]{\color{textcolor}\rmfamily\fontsize{10.000000}{12.000000}\selectfont Ground truth}%
\end{pgfscope}%
\begin{pgfscope}%
\pgfsetbuttcap%
\pgfsetroundjoin%
\definecolor{currentfill}{rgb}{0.121569,0.466667,0.705882}%
\pgfsetfillcolor{currentfill}%
\pgfsetlinewidth{1.003750pt}%
\definecolor{currentstroke}{rgb}{0.121569,0.466667,0.705882}%
\pgfsetstrokecolor{currentstroke}%
\pgfsetdash{}{0pt}%
\pgfsys@defobject{currentmarker}{\pgfqpoint{-0.041667in}{-0.041667in}}{\pgfqpoint{0.041667in}{0.041667in}}{%
\pgfpathmoveto{\pgfqpoint{0.000000in}{-0.041667in}}%
\pgfpathcurveto{\pgfqpoint{0.011050in}{-0.041667in}}{\pgfqpoint{0.021649in}{-0.037276in}}{\pgfqpoint{0.029463in}{-0.029463in}}%
\pgfpathcurveto{\pgfqpoint{0.037276in}{-0.021649in}}{\pgfqpoint{0.041667in}{-0.011050in}}{\pgfqpoint{0.041667in}{0.000000in}}%
\pgfpathcurveto{\pgfqpoint{0.041667in}{0.011050in}}{\pgfqpoint{0.037276in}{0.021649in}}{\pgfqpoint{0.029463in}{0.029463in}}%
\pgfpathcurveto{\pgfqpoint{0.021649in}{0.037276in}}{\pgfqpoint{0.011050in}{0.041667in}}{\pgfqpoint{0.000000in}{0.041667in}}%
\pgfpathcurveto{\pgfqpoint{-0.011050in}{0.041667in}}{\pgfqpoint{-0.021649in}{0.037276in}}{\pgfqpoint{-0.029463in}{0.029463in}}%
\pgfpathcurveto{\pgfqpoint{-0.037276in}{0.021649in}}{\pgfqpoint{-0.041667in}{0.011050in}}{\pgfqpoint{-0.041667in}{0.000000in}}%
\pgfpathcurveto{\pgfqpoint{-0.041667in}{-0.011050in}}{\pgfqpoint{-0.037276in}{-0.021649in}}{\pgfqpoint{-0.029463in}{-0.029463in}}%
\pgfpathcurveto{\pgfqpoint{-0.021649in}{-0.037276in}}{\pgfqpoint{-0.011050in}{-0.041667in}}{\pgfqpoint{0.000000in}{-0.041667in}}%
\pgfpathclose%
\pgfusepath{stroke,fill}%
}%
\begin{pgfscope}%
\pgfsys@transformshift{4.004722in}{3.831625in}%
\pgfsys@useobject{currentmarker}{}%
\end{pgfscope}%
\end{pgfscope}%
\begin{pgfscope}%
\definecolor{textcolor}{rgb}{0.000000,0.000000,0.000000}%
\pgfsetstrokecolor{textcolor}%
\pgfsetfillcolor{textcolor}%
\pgftext[x=4.254722in,y=3.795166in,left,base]{\color{textcolor}\rmfamily\fontsize{10.000000}{12.000000}\selectfont Estimated position}%
\end{pgfscope}%
\end{pgfpicture}%
\makeatother%
\endgroup%
}
%         \caption{ROLEQ's 3D position estimation had the lowest turn displacement for the 4-meter side triangle experiment.}
%         \label{fig:triangle4_2D}
%     \end{subfigure}
%     \begin{subfigure}{0.49\textwidth}
%         \centering
%         \resizebox{1\linewidth}{!}{%% Creator: Matplotlib, PGF backend
%%
%% To include the figure in your LaTeX document, write
%%   \input{<filename>.pgf}
%%
%% Make sure the required packages are loaded in your preamble
%%   \usepackage{pgf}
%%
%% and, on pdftex
%%   \usepackage[utf8]{inputenc}\DeclareUnicodeCharacter{2212}{-}
%%
%% or, on luatex and xetex
%%   \usepackage{unicode-math}
%%
%% Figures using additional raster images can only be included by \input if
%% they are in the same directory as the main LaTeX file. For loading figures
%% from other directories you can use the `import` package
%%   \usepackage{import}
%%
%% and then include the figures with
%%   \import{<path to file>}{<filename>.pgf}
%%
%% Matplotlib used the following preamble
%%   \usepackage{fontspec}
%%
\begingroup%
\makeatletter%
\begin{pgfpicture}%
\pgfpathrectangle{\pgfpointorigin}{\pgfqpoint{4.342355in}{4.207622in}}%
\pgfusepath{use as bounding box, clip}%
\begin{pgfscope}%
\pgfsetbuttcap%
\pgfsetmiterjoin%
\definecolor{currentfill}{rgb}{1.000000,1.000000,1.000000}%
\pgfsetfillcolor{currentfill}%
\pgfsetlinewidth{0.000000pt}%
\definecolor{currentstroke}{rgb}{1.000000,1.000000,1.000000}%
\pgfsetstrokecolor{currentstroke}%
\pgfsetdash{}{0pt}%
\pgfpathmoveto{\pgfqpoint{0.000000in}{0.000000in}}%
\pgfpathlineto{\pgfqpoint{4.342355in}{0.000000in}}%
\pgfpathlineto{\pgfqpoint{4.342355in}{4.207622in}}%
\pgfpathlineto{\pgfqpoint{0.000000in}{4.207622in}}%
\pgfpathclose%
\pgfusepath{fill}%
\end{pgfscope}%
\begin{pgfscope}%
\pgfsetbuttcap%
\pgfsetmiterjoin%
\definecolor{currentfill}{rgb}{1.000000,1.000000,1.000000}%
\pgfsetfillcolor{currentfill}%
\pgfsetlinewidth{0.000000pt}%
\definecolor{currentstroke}{rgb}{0.000000,0.000000,0.000000}%
\pgfsetstrokecolor{currentstroke}%
\pgfsetstrokeopacity{0.000000}%
\pgfsetdash{}{0pt}%
\pgfpathmoveto{\pgfqpoint{0.100000in}{0.212622in}}%
\pgfpathlineto{\pgfqpoint{3.796000in}{0.212622in}}%
\pgfpathlineto{\pgfqpoint{3.796000in}{3.908622in}}%
\pgfpathlineto{\pgfqpoint{0.100000in}{3.908622in}}%
\pgfpathclose%
\pgfusepath{fill}%
\end{pgfscope}%
\begin{pgfscope}%
\pgfsetbuttcap%
\pgfsetmiterjoin%
\definecolor{currentfill}{rgb}{0.950000,0.950000,0.950000}%
\pgfsetfillcolor{currentfill}%
\pgfsetfillopacity{0.500000}%
\pgfsetlinewidth{1.003750pt}%
\definecolor{currentstroke}{rgb}{0.950000,0.950000,0.950000}%
\pgfsetstrokecolor{currentstroke}%
\pgfsetstrokeopacity{0.500000}%
\pgfsetdash{}{0pt}%
\pgfpathmoveto{\pgfqpoint{0.379073in}{1.123938in}}%
\pgfpathlineto{\pgfqpoint{1.599613in}{2.147018in}}%
\pgfpathlineto{\pgfqpoint{1.582647in}{3.622484in}}%
\pgfpathlineto{\pgfqpoint{0.303698in}{2.689165in}}%
\pgfusepath{stroke,fill}%
\end{pgfscope}%
\begin{pgfscope}%
\pgfsetbuttcap%
\pgfsetmiterjoin%
\definecolor{currentfill}{rgb}{0.900000,0.900000,0.900000}%
\pgfsetfillcolor{currentfill}%
\pgfsetfillopacity{0.500000}%
\pgfsetlinewidth{1.003750pt}%
\definecolor{currentstroke}{rgb}{0.900000,0.900000,0.900000}%
\pgfsetstrokecolor{currentstroke}%
\pgfsetstrokeopacity{0.500000}%
\pgfsetdash{}{0pt}%
\pgfpathmoveto{\pgfqpoint{1.599613in}{2.147018in}}%
\pgfpathlineto{\pgfqpoint{3.558144in}{1.577751in}}%
\pgfpathlineto{\pgfqpoint{3.628038in}{3.104037in}}%
\pgfpathlineto{\pgfqpoint{1.582647in}{3.622484in}}%
\pgfusepath{stroke,fill}%
\end{pgfscope}%
\begin{pgfscope}%
\pgfsetbuttcap%
\pgfsetmiterjoin%
\definecolor{currentfill}{rgb}{0.925000,0.925000,0.925000}%
\pgfsetfillcolor{currentfill}%
\pgfsetfillopacity{0.500000}%
\pgfsetlinewidth{1.003750pt}%
\definecolor{currentstroke}{rgb}{0.925000,0.925000,0.925000}%
\pgfsetstrokecolor{currentstroke}%
\pgfsetstrokeopacity{0.500000}%
\pgfsetdash{}{0pt}%
\pgfpathmoveto{\pgfqpoint{0.379073in}{1.123938in}}%
\pgfpathlineto{\pgfqpoint{2.455212in}{0.445871in}}%
\pgfpathlineto{\pgfqpoint{3.558144in}{1.577751in}}%
\pgfpathlineto{\pgfqpoint{1.599613in}{2.147018in}}%
\pgfusepath{stroke,fill}%
\end{pgfscope}%
\begin{pgfscope}%
\pgfsetrectcap%
\pgfsetroundjoin%
\pgfsetlinewidth{0.803000pt}%
\definecolor{currentstroke}{rgb}{0.000000,0.000000,0.000000}%
\pgfsetstrokecolor{currentstroke}%
\pgfsetdash{}{0pt}%
\pgfpathmoveto{\pgfqpoint{0.379073in}{1.123938in}}%
\pgfpathlineto{\pgfqpoint{2.455212in}{0.445871in}}%
\pgfusepath{stroke}%
\end{pgfscope}%
\begin{pgfscope}%
\definecolor{textcolor}{rgb}{0.000000,0.000000,0.000000}%
\pgfsetstrokecolor{textcolor}%
\pgfsetfillcolor{textcolor}%
\pgftext[x=0.730374in, y=0.408886in, left, base,rotate=341.912962]{\color{textcolor}\rmfamily\fontsize{10.000000}{12.000000}\selectfont Position X [\(\displaystyle m\)]}%
\end{pgfscope}%
\begin{pgfscope}%
\pgfsetbuttcap%
\pgfsetroundjoin%
\pgfsetlinewidth{0.803000pt}%
\definecolor{currentstroke}{rgb}{0.690196,0.690196,0.690196}%
\pgfsetstrokecolor{currentstroke}%
\pgfsetdash{}{0pt}%
\pgfpathmoveto{\pgfqpoint{0.709705in}{1.015953in}}%
\pgfpathlineto{\pgfqpoint{1.912675in}{2.056024in}}%
\pgfpathlineto{\pgfqpoint{1.909013in}{3.539759in}}%
\pgfusepath{stroke}%
\end{pgfscope}%
\begin{pgfscope}%
\pgfsetbuttcap%
\pgfsetroundjoin%
\pgfsetlinewidth{0.803000pt}%
\definecolor{currentstroke}{rgb}{0.690196,0.690196,0.690196}%
\pgfsetstrokecolor{currentstroke}%
\pgfsetdash{}{0pt}%
\pgfpathmoveto{\pgfqpoint{1.012382in}{0.917099in}}%
\pgfpathlineto{\pgfqpoint{2.198881in}{1.972835in}}%
\pgfpathlineto{\pgfqpoint{2.207575in}{3.464083in}}%
\pgfusepath{stroke}%
\end{pgfscope}%
\begin{pgfscope}%
\pgfsetbuttcap%
\pgfsetroundjoin%
\pgfsetlinewidth{0.803000pt}%
\definecolor{currentstroke}{rgb}{0.690196,0.690196,0.690196}%
\pgfsetstrokecolor{currentstroke}%
\pgfsetdash{}{0pt}%
\pgfpathmoveto{\pgfqpoint{1.319834in}{0.816685in}}%
\pgfpathlineto{\pgfqpoint{2.489226in}{1.888443in}}%
\pgfpathlineto{\pgfqpoint{2.510643in}{3.387264in}}%
\pgfusepath{stroke}%
\end{pgfscope}%
\begin{pgfscope}%
\pgfsetbuttcap%
\pgfsetroundjoin%
\pgfsetlinewidth{0.803000pt}%
\definecolor{currentstroke}{rgb}{0.690196,0.690196,0.690196}%
\pgfsetstrokecolor{currentstroke}%
\pgfsetdash{}{0pt}%
\pgfpathmoveto{\pgfqpoint{1.632176in}{0.714674in}}%
\pgfpathlineto{\pgfqpoint{2.783799in}{1.802822in}}%
\pgfpathlineto{\pgfqpoint{2.818318in}{3.309277in}}%
\pgfusepath{stroke}%
\end{pgfscope}%
\begin{pgfscope}%
\pgfsetbuttcap%
\pgfsetroundjoin%
\pgfsetlinewidth{0.803000pt}%
\definecolor{currentstroke}{rgb}{0.690196,0.690196,0.690196}%
\pgfsetstrokecolor{currentstroke}%
\pgfsetdash{}{0pt}%
\pgfpathmoveto{\pgfqpoint{1.949526in}{0.611028in}}%
\pgfpathlineto{\pgfqpoint{3.082695in}{1.715945in}}%
\pgfpathlineto{\pgfqpoint{3.130708in}{3.230095in}}%
\pgfusepath{stroke}%
\end{pgfscope}%
\begin{pgfscope}%
\pgfsetbuttcap%
\pgfsetroundjoin%
\pgfsetlinewidth{0.803000pt}%
\definecolor{currentstroke}{rgb}{0.690196,0.690196,0.690196}%
\pgfsetstrokecolor{currentstroke}%
\pgfsetdash{}{0pt}%
\pgfpathmoveto{\pgfqpoint{2.272004in}{0.505707in}}%
\pgfpathlineto{\pgfqpoint{3.386009in}{1.627784in}}%
\pgfpathlineto{\pgfqpoint{3.447921in}{3.149691in}}%
\pgfusepath{stroke}%
\end{pgfscope}%
\begin{pgfscope}%
\pgfsetrectcap%
\pgfsetroundjoin%
\pgfsetlinewidth{0.803000pt}%
\definecolor{currentstroke}{rgb}{0.000000,0.000000,0.000000}%
\pgfsetstrokecolor{currentstroke}%
\pgfsetdash{}{0pt}%
\pgfpathmoveto{\pgfqpoint{0.720185in}{1.025014in}}%
\pgfpathlineto{\pgfqpoint{0.688700in}{0.997793in}}%
\pgfusepath{stroke}%
\end{pgfscope}%
\begin{pgfscope}%
\definecolor{textcolor}{rgb}{0.000000,0.000000,0.000000}%
\pgfsetstrokecolor{textcolor}%
\pgfsetfillcolor{textcolor}%
\pgftext[x=0.605342in,y=0.796332in,,top]{\color{textcolor}\rmfamily\fontsize{10.000000}{12.000000}\selectfont \(\displaystyle {−1}\)}%
\end{pgfscope}%
\begin{pgfscope}%
\pgfsetrectcap%
\pgfsetroundjoin%
\pgfsetlinewidth{0.803000pt}%
\definecolor{currentstroke}{rgb}{0.000000,0.000000,0.000000}%
\pgfsetstrokecolor{currentstroke}%
\pgfsetdash{}{0pt}%
\pgfpathmoveto{\pgfqpoint{1.022725in}{0.926302in}}%
\pgfpathlineto{\pgfqpoint{0.991651in}{0.898653in}}%
\pgfusepath{stroke}%
\end{pgfscope}%
\begin{pgfscope}%
\definecolor{textcolor}{rgb}{0.000000,0.000000,0.000000}%
\pgfsetstrokecolor{textcolor}%
\pgfsetfillcolor{textcolor}%
\pgftext[x=0.908338in,y=0.695380in,,top]{\color{textcolor}\rmfamily\fontsize{10.000000}{12.000000}\selectfont \(\displaystyle {0}\)}%
\end{pgfscope}%
\begin{pgfscope}%
\pgfsetrectcap%
\pgfsetroundjoin%
\pgfsetlinewidth{0.803000pt}%
\definecolor{currentstroke}{rgb}{0.000000,0.000000,0.000000}%
\pgfsetstrokecolor{currentstroke}%
\pgfsetdash{}{0pt}%
\pgfpathmoveto{\pgfqpoint{1.330035in}{0.826034in}}%
\pgfpathlineto{\pgfqpoint{1.299388in}{0.797946in}}%
\pgfusepath{stroke}%
\end{pgfscope}%
\begin{pgfscope}%
\definecolor{textcolor}{rgb}{0.000000,0.000000,0.000000}%
\pgfsetstrokecolor{textcolor}%
\pgfsetfillcolor{textcolor}%
\pgftext[x=1.216134in,y=0.592829in,,top]{\color{textcolor}\rmfamily\fontsize{10.000000}{12.000000}\selectfont \(\displaystyle {1}\)}%
\end{pgfscope}%
\begin{pgfscope}%
\pgfsetrectcap%
\pgfsetroundjoin%
\pgfsetlinewidth{0.803000pt}%
\definecolor{currentstroke}{rgb}{0.000000,0.000000,0.000000}%
\pgfsetstrokecolor{currentstroke}%
\pgfsetdash{}{0pt}%
\pgfpathmoveto{\pgfqpoint{1.642228in}{0.724172in}}%
\pgfpathlineto{\pgfqpoint{1.612028in}{0.695636in}}%
\pgfusepath{stroke}%
\end{pgfscope}%
\begin{pgfscope}%
\definecolor{textcolor}{rgb}{0.000000,0.000000,0.000000}%
\pgfsetstrokecolor{textcolor}%
\pgfsetfillcolor{textcolor}%
\pgftext[x=1.528843in,y=0.488641in,,top]{\color{textcolor}\rmfamily\fontsize{10.000000}{12.000000}\selectfont \(\displaystyle {2}\)}%
\end{pgfscope}%
\begin{pgfscope}%
\pgfsetrectcap%
\pgfsetroundjoin%
\pgfsetlinewidth{0.803000pt}%
\definecolor{currentstroke}{rgb}{0.000000,0.000000,0.000000}%
\pgfsetstrokecolor{currentstroke}%
\pgfsetdash{}{0pt}%
\pgfpathmoveto{\pgfqpoint{1.959423in}{0.620679in}}%
\pgfpathlineto{\pgfqpoint{1.929686in}{0.591683in}}%
\pgfusepath{stroke}%
\end{pgfscope}%
\begin{pgfscope}%
\definecolor{textcolor}{rgb}{0.000000,0.000000,0.000000}%
\pgfsetstrokecolor{textcolor}%
\pgfsetfillcolor{textcolor}%
\pgftext[x=1.846585in,y=0.382776in,,top]{\color{textcolor}\rmfamily\fontsize{10.000000}{12.000000}\selectfont \(\displaystyle {3}\)}%
\end{pgfscope}%
\begin{pgfscope}%
\pgfsetrectcap%
\pgfsetroundjoin%
\pgfsetlinewidth{0.803000pt}%
\definecolor{currentstroke}{rgb}{0.000000,0.000000,0.000000}%
\pgfsetstrokecolor{currentstroke}%
\pgfsetdash{}{0pt}%
\pgfpathmoveto{\pgfqpoint{2.281740in}{0.515514in}}%
\pgfpathlineto{\pgfqpoint{2.252486in}{0.486048in}}%
\pgfusepath{stroke}%
\end{pgfscope}%
\begin{pgfscope}%
\definecolor{textcolor}{rgb}{0.000000,0.000000,0.000000}%
\pgfsetstrokecolor{textcolor}%
\pgfsetfillcolor{textcolor}%
\pgftext[x=2.169482in,y=0.275194in,,top]{\color{textcolor}\rmfamily\fontsize{10.000000}{12.000000}\selectfont \(\displaystyle {4}\)}%
\end{pgfscope}%
\begin{pgfscope}%
\pgfsetrectcap%
\pgfsetroundjoin%
\pgfsetlinewidth{0.803000pt}%
\definecolor{currentstroke}{rgb}{0.000000,0.000000,0.000000}%
\pgfsetstrokecolor{currentstroke}%
\pgfsetdash{}{0pt}%
\pgfpathmoveto{\pgfqpoint{3.558144in}{1.577751in}}%
\pgfpathlineto{\pgfqpoint{2.455212in}{0.445871in}}%
\pgfusepath{stroke}%
\end{pgfscope}%
\begin{pgfscope}%
\definecolor{textcolor}{rgb}{0.000000,0.000000,0.000000}%
\pgfsetstrokecolor{textcolor}%
\pgfsetfillcolor{textcolor}%
\pgftext[x=3.120747in, y=0.305657in, left, base,rotate=45.742112]{\color{textcolor}\rmfamily\fontsize{10.000000}{12.000000}\selectfont Position Y [\(\displaystyle m\)]}%
\end{pgfscope}%
\begin{pgfscope}%
\pgfsetbuttcap%
\pgfsetroundjoin%
\pgfsetlinewidth{0.803000pt}%
\definecolor{currentstroke}{rgb}{0.690196,0.690196,0.690196}%
\pgfsetstrokecolor{currentstroke}%
\pgfsetdash{}{0pt}%
\pgfpathmoveto{\pgfqpoint{0.382945in}{2.746996in}}%
\pgfpathlineto{\pgfqpoint{0.454433in}{1.187105in}}%
\pgfpathlineto{\pgfqpoint{2.523593in}{0.516046in}}%
\pgfusepath{stroke}%
\end{pgfscope}%
\begin{pgfscope}%
\pgfsetbuttcap%
\pgfsetroundjoin%
\pgfsetlinewidth{0.803000pt}%
\definecolor{currentstroke}{rgb}{0.690196,0.690196,0.690196}%
\pgfsetstrokecolor{currentstroke}%
\pgfsetdash{}{0pt}%
\pgfpathmoveto{\pgfqpoint{0.575101in}{2.887223in}}%
\pgfpathlineto{\pgfqpoint{0.637308in}{1.340395in}}%
\pgfpathlineto{\pgfqpoint{2.689377in}{0.686182in}}%
\pgfusepath{stroke}%
\end{pgfscope}%
\begin{pgfscope}%
\pgfsetbuttcap%
\pgfsetroundjoin%
\pgfsetlinewidth{0.803000pt}%
\definecolor{currentstroke}{rgb}{0.690196,0.690196,0.690196}%
\pgfsetstrokecolor{currentstroke}%
\pgfsetdash{}{0pt}%
\pgfpathmoveto{\pgfqpoint{0.762375in}{3.023887in}}%
\pgfpathlineto{\pgfqpoint{0.815738in}{1.489958in}}%
\pgfpathlineto{\pgfqpoint{2.850920in}{0.851965in}}%
\pgfusepath{stroke}%
\end{pgfscope}%
\begin{pgfscope}%
\pgfsetbuttcap%
\pgfsetroundjoin%
\pgfsetlinewidth{0.803000pt}%
\definecolor{currentstroke}{rgb}{0.690196,0.690196,0.690196}%
\pgfsetstrokecolor{currentstroke}%
\pgfsetdash{}{0pt}%
\pgfpathmoveto{\pgfqpoint{0.944950in}{3.157122in}}%
\pgfpathlineto{\pgfqpoint{0.989882in}{1.635929in}}%
\pgfpathlineto{\pgfqpoint{3.008382in}{1.013559in}}%
\pgfusepath{stroke}%
\end{pgfscope}%
\begin{pgfscope}%
\pgfsetbuttcap%
\pgfsetroundjoin%
\pgfsetlinewidth{0.803000pt}%
\definecolor{currentstroke}{rgb}{0.690196,0.690196,0.690196}%
\pgfsetstrokecolor{currentstroke}%
\pgfsetdash{}{0pt}%
\pgfpathmoveto{\pgfqpoint{1.123002in}{3.287056in}}%
\pgfpathlineto{\pgfqpoint{1.159893in}{1.778436in}}%
\pgfpathlineto{\pgfqpoint{3.161916in}{1.171123in}}%
\pgfusepath{stroke}%
\end{pgfscope}%
\begin{pgfscope}%
\pgfsetbuttcap%
\pgfsetroundjoin%
\pgfsetlinewidth{0.803000pt}%
\definecolor{currentstroke}{rgb}{0.690196,0.690196,0.690196}%
\pgfsetstrokecolor{currentstroke}%
\pgfsetdash{}{0pt}%
\pgfpathmoveto{\pgfqpoint{1.296696in}{3.413810in}}%
\pgfpathlineto{\pgfqpoint{1.325917in}{1.917600in}}%
\pgfpathlineto{\pgfqpoint{3.311666in}{1.324803in}}%
\pgfusepath{stroke}%
\end{pgfscope}%
\begin{pgfscope}%
\pgfsetbuttcap%
\pgfsetroundjoin%
\pgfsetlinewidth{0.803000pt}%
\definecolor{currentstroke}{rgb}{0.690196,0.690196,0.690196}%
\pgfsetstrokecolor{currentstroke}%
\pgfsetdash{}{0pt}%
\pgfpathmoveto{\pgfqpoint{1.466190in}{3.537499in}}%
\pgfpathlineto{\pgfqpoint{1.488091in}{2.053538in}}%
\pgfpathlineto{\pgfqpoint{3.457772in}{1.474744in}}%
\pgfusepath{stroke}%
\end{pgfscope}%
\begin{pgfscope}%
\pgfsetrectcap%
\pgfsetroundjoin%
\pgfsetlinewidth{0.803000pt}%
\definecolor{currentstroke}{rgb}{0.000000,0.000000,0.000000}%
\pgfsetstrokecolor{currentstroke}%
\pgfsetdash{}{0pt}%
\pgfpathmoveto{\pgfqpoint{2.506155in}{0.521701in}}%
\pgfpathlineto{\pgfqpoint{2.558513in}{0.504721in}}%
\pgfusepath{stroke}%
\end{pgfscope}%
\begin{pgfscope}%
\definecolor{textcolor}{rgb}{0.000000,0.000000,0.000000}%
\pgfsetstrokecolor{textcolor}%
\pgfsetfillcolor{textcolor}%
\pgftext[x=2.702597in,y=0.329491in,,top]{\color{textcolor}\rmfamily\fontsize{10.000000}{12.000000}\selectfont \(\displaystyle {−1}\)}%
\end{pgfscope}%
\begin{pgfscope}%
\pgfsetrectcap%
\pgfsetroundjoin%
\pgfsetlinewidth{0.803000pt}%
\definecolor{currentstroke}{rgb}{0.000000,0.000000,0.000000}%
\pgfsetstrokecolor{currentstroke}%
\pgfsetdash{}{0pt}%
\pgfpathmoveto{\pgfqpoint{2.672095in}{0.691691in}}%
\pgfpathlineto{\pgfqpoint{2.723985in}{0.675148in}}%
\pgfusepath{stroke}%
\end{pgfscope}%
\begin{pgfscope}%
\definecolor{textcolor}{rgb}{0.000000,0.000000,0.000000}%
\pgfsetstrokecolor{textcolor}%
\pgfsetfillcolor{textcolor}%
\pgftext[x=2.866158in,y=0.502146in,,top]{\color{textcolor}\rmfamily\fontsize{10.000000}{12.000000}\selectfont \(\displaystyle {0}\)}%
\end{pgfscope}%
\begin{pgfscope}%
\pgfsetrectcap%
\pgfsetroundjoin%
\pgfsetlinewidth{0.803000pt}%
\definecolor{currentstroke}{rgb}{0.000000,0.000000,0.000000}%
\pgfsetstrokecolor{currentstroke}%
\pgfsetdash{}{0pt}%
\pgfpathmoveto{\pgfqpoint{2.833791in}{0.857334in}}%
\pgfpathlineto{\pgfqpoint{2.885221in}{0.841212in}}%
\pgfusepath{stroke}%
\end{pgfscope}%
\begin{pgfscope}%
\definecolor{textcolor}{rgb}{0.000000,0.000000,0.000000}%
\pgfsetstrokecolor{textcolor}%
\pgfsetfillcolor{textcolor}%
\pgftext[x=3.025533in,y=0.670381in,,top]{\color{textcolor}\rmfamily\fontsize{10.000000}{12.000000}\selectfont \(\displaystyle {1}\)}%
\end{pgfscope}%
\begin{pgfscope}%
\pgfsetrectcap%
\pgfsetroundjoin%
\pgfsetlinewidth{0.803000pt}%
\definecolor{currentstroke}{rgb}{0.000000,0.000000,0.000000}%
\pgfsetstrokecolor{currentstroke}%
\pgfsetdash{}{0pt}%
\pgfpathmoveto{\pgfqpoint{2.991404in}{1.018794in}}%
\pgfpathlineto{\pgfqpoint{3.042380in}{1.003077in}}%
\pgfusepath{stroke}%
\end{pgfscope}%
\begin{pgfscope}%
\definecolor{textcolor}{rgb}{0.000000,0.000000,0.000000}%
\pgfsetstrokecolor{textcolor}%
\pgfsetfillcolor{textcolor}%
\pgftext[x=3.180878in,y=0.834364in,,top]{\color{textcolor}\rmfamily\fontsize{10.000000}{12.000000}\selectfont \(\displaystyle {2}\)}%
\end{pgfscope}%
\begin{pgfscope}%
\pgfsetrectcap%
\pgfsetroundjoin%
\pgfsetlinewidth{0.803000pt}%
\definecolor{currentstroke}{rgb}{0.000000,0.000000,0.000000}%
\pgfsetstrokecolor{currentstroke}%
\pgfsetdash{}{0pt}%
\pgfpathmoveto{\pgfqpoint{3.145087in}{1.176228in}}%
\pgfpathlineto{\pgfqpoint{3.195615in}{1.160900in}}%
\pgfusepath{stroke}%
\end{pgfscope}%
\begin{pgfscope}%
\definecolor{textcolor}{rgb}{0.000000,0.000000,0.000000}%
\pgfsetstrokecolor{textcolor}%
\pgfsetfillcolor{textcolor}%
\pgftext[x=3.332347in,y=0.994254in,,top]{\color{textcolor}\rmfamily\fontsize{10.000000}{12.000000}\selectfont \(\displaystyle {3}\)}%
\end{pgfscope}%
\begin{pgfscope}%
\pgfsetrectcap%
\pgfsetroundjoin%
\pgfsetlinewidth{0.803000pt}%
\definecolor{currentstroke}{rgb}{0.000000,0.000000,0.000000}%
\pgfsetstrokecolor{currentstroke}%
\pgfsetdash{}{0pt}%
\pgfpathmoveto{\pgfqpoint{3.294984in}{1.329783in}}%
\pgfpathlineto{\pgfqpoint{3.345071in}{1.314831in}}%
\pgfusepath{stroke}%
\end{pgfscope}%
\begin{pgfscope}%
\definecolor{textcolor}{rgb}{0.000000,0.000000,0.000000}%
\pgfsetstrokecolor{textcolor}%
\pgfsetfillcolor{textcolor}%
\pgftext[x=3.480080in,y=1.150201in,,top]{\color{textcolor}\rmfamily\fontsize{10.000000}{12.000000}\selectfont \(\displaystyle {4}\)}%
\end{pgfscope}%
\begin{pgfscope}%
\pgfsetrectcap%
\pgfsetroundjoin%
\pgfsetlinewidth{0.803000pt}%
\definecolor{currentstroke}{rgb}{0.000000,0.000000,0.000000}%
\pgfsetstrokecolor{currentstroke}%
\pgfsetdash{}{0pt}%
\pgfpathmoveto{\pgfqpoint{3.441234in}{1.479603in}}%
\pgfpathlineto{\pgfqpoint{3.490887in}{1.465013in}}%
\pgfusepath{stroke}%
\end{pgfscope}%
\begin{pgfscope}%
\definecolor{textcolor}{rgb}{0.000000,0.000000,0.000000}%
\pgfsetstrokecolor{textcolor}%
\pgfsetfillcolor{textcolor}%
\pgftext[x=3.624216in,y=1.302351in,,top]{\color{textcolor}\rmfamily\fontsize{10.000000}{12.000000}\selectfont \(\displaystyle {5}\)}%
\end{pgfscope}%
\begin{pgfscope}%
\pgfsetrectcap%
\pgfsetroundjoin%
\pgfsetlinewidth{0.803000pt}%
\definecolor{currentstroke}{rgb}{0.000000,0.000000,0.000000}%
\pgfsetstrokecolor{currentstroke}%
\pgfsetdash{}{0pt}%
\pgfpathmoveto{\pgfqpoint{3.558144in}{1.577751in}}%
\pgfpathlineto{\pgfqpoint{3.628038in}{3.104037in}}%
\pgfusepath{stroke}%
\end{pgfscope}%
\begin{pgfscope}%
\definecolor{textcolor}{rgb}{0.000000,0.000000,0.000000}%
\pgfsetstrokecolor{textcolor}%
\pgfsetfillcolor{textcolor}%
\pgftext[x=4.167903in, y=1.963517in, left, base,rotate=87.378092]{\color{textcolor}\rmfamily\fontsize{10.000000}{12.000000}\selectfont Position Z [\(\displaystyle m\)]}%
\end{pgfscope}%
\begin{pgfscope}%
\pgfsetbuttcap%
\pgfsetroundjoin%
\pgfsetlinewidth{0.803000pt}%
\definecolor{currentstroke}{rgb}{0.690196,0.690196,0.690196}%
\pgfsetstrokecolor{currentstroke}%
\pgfsetdash{}{0pt}%
\pgfpathmoveto{\pgfqpoint{3.571285in}{1.864714in}}%
\pgfpathlineto{\pgfqpoint{1.596418in}{2.424902in}}%
\pgfpathlineto{\pgfqpoint{0.364921in}{1.417819in}}%
\pgfusepath{stroke}%
\end{pgfscope}%
\begin{pgfscope}%
\pgfsetbuttcap%
\pgfsetroundjoin%
\pgfsetlinewidth{0.803000pt}%
\definecolor{currentstroke}{rgb}{0.690196,0.690196,0.690196}%
\pgfsetstrokecolor{currentstroke}%
\pgfsetdash{}{0pt}%
\pgfpathmoveto{\pgfqpoint{3.587137in}{2.210869in}}%
\pgfpathlineto{\pgfqpoint{1.592567in}{2.759812in}}%
\pgfpathlineto{\pgfqpoint{0.347838in}{1.772568in}}%
\pgfusepath{stroke}%
\end{pgfscope}%
\begin{pgfscope}%
\pgfsetbuttcap%
\pgfsetroundjoin%
\pgfsetlinewidth{0.803000pt}%
\definecolor{currentstroke}{rgb}{0.690196,0.690196,0.690196}%
\pgfsetstrokecolor{currentstroke}%
\pgfsetdash{}{0pt}%
\pgfpathmoveto{\pgfqpoint{3.603311in}{2.564070in}}%
\pgfpathlineto{\pgfqpoint{1.588641in}{3.101208in}}%
\pgfpathlineto{\pgfqpoint{0.330393in}{2.134818in}}%
\pgfusepath{stroke}%
\end{pgfscope}%
\begin{pgfscope}%
\pgfsetbuttcap%
\pgfsetroundjoin%
\pgfsetlinewidth{0.803000pt}%
\definecolor{currentstroke}{rgb}{0.690196,0.690196,0.690196}%
\pgfsetstrokecolor{currentstroke}%
\pgfsetdash{}{0pt}%
\pgfpathmoveto{\pgfqpoint{3.619818in}{2.924534in}}%
\pgfpathlineto{\pgfqpoint{1.584638in}{3.449281in}}%
\pgfpathlineto{\pgfqpoint{0.312576in}{2.504809in}}%
\pgfusepath{stroke}%
\end{pgfscope}%
\begin{pgfscope}%
\pgfsetrectcap%
\pgfsetroundjoin%
\pgfsetlinewidth{0.803000pt}%
\definecolor{currentstroke}{rgb}{0.000000,0.000000,0.000000}%
\pgfsetstrokecolor{currentstroke}%
\pgfsetdash{}{0pt}%
\pgfpathmoveto{\pgfqpoint{3.554704in}{1.869417in}}%
\pgfpathlineto{\pgfqpoint{3.604487in}{1.855296in}}%
\pgfusepath{stroke}%
\end{pgfscope}%
\begin{pgfscope}%
\definecolor{textcolor}{rgb}{0.000000,0.000000,0.000000}%
\pgfsetstrokecolor{textcolor}%
\pgfsetfillcolor{textcolor}%
\pgftext[x=3.826827in,y=1.900330in,,top]{\color{textcolor}\rmfamily\fontsize{10.000000}{12.000000}\selectfont \(\displaystyle {−2}\)}%
\end{pgfscope}%
\begin{pgfscope}%
\pgfsetrectcap%
\pgfsetroundjoin%
\pgfsetlinewidth{0.803000pt}%
\definecolor{currentstroke}{rgb}{0.000000,0.000000,0.000000}%
\pgfsetstrokecolor{currentstroke}%
\pgfsetdash{}{0pt}%
\pgfpathmoveto{\pgfqpoint{3.570382in}{2.215480in}}%
\pgfpathlineto{\pgfqpoint{3.620686in}{2.201635in}}%
\pgfusepath{stroke}%
\end{pgfscope}%
\begin{pgfscope}%
\definecolor{textcolor}{rgb}{0.000000,0.000000,0.000000}%
\pgfsetstrokecolor{textcolor}%
\pgfsetfillcolor{textcolor}%
\pgftext[x=3.845197in,y=2.245786in,,top]{\color{textcolor}\rmfamily\fontsize{10.000000}{12.000000}\selectfont \(\displaystyle {−1}\)}%
\end{pgfscope}%
\begin{pgfscope}%
\pgfsetrectcap%
\pgfsetroundjoin%
\pgfsetlinewidth{0.803000pt}%
\definecolor{currentstroke}{rgb}{0.000000,0.000000,0.000000}%
\pgfsetstrokecolor{currentstroke}%
\pgfsetdash{}{0pt}%
\pgfpathmoveto{\pgfqpoint{3.586379in}{2.568584in}}%
\pgfpathlineto{\pgfqpoint{3.637215in}{2.555030in}}%
\pgfusepath{stroke}%
\end{pgfscope}%
\begin{pgfscope}%
\definecolor{textcolor}{rgb}{0.000000,0.000000,0.000000}%
\pgfsetstrokecolor{textcolor}%
\pgfsetfillcolor{textcolor}%
\pgftext[x=3.863939in,y=2.598252in,,top]{\color{textcolor}\rmfamily\fontsize{10.000000}{12.000000}\selectfont \(\displaystyle {0}\)}%
\end{pgfscope}%
\begin{pgfscope}%
\pgfsetrectcap%
\pgfsetroundjoin%
\pgfsetlinewidth{0.803000pt}%
\definecolor{currentstroke}{rgb}{0.000000,0.000000,0.000000}%
\pgfsetstrokecolor{currentstroke}%
\pgfsetdash{}{0pt}%
\pgfpathmoveto{\pgfqpoint{3.602705in}{2.928946in}}%
\pgfpathlineto{\pgfqpoint{3.654085in}{2.915699in}}%
\pgfusepath{stroke}%
\end{pgfscope}%
\begin{pgfscope}%
\definecolor{textcolor}{rgb}{0.000000,0.000000,0.000000}%
\pgfsetstrokecolor{textcolor}%
\pgfsetfillcolor{textcolor}%
\pgftext[x=3.883066in,y=2.957943in,,top]{\color{textcolor}\rmfamily\fontsize{10.000000}{12.000000}\selectfont \(\displaystyle {1}\)}%
\end{pgfscope}%
\begin{pgfscope}%
\pgfpathrectangle{\pgfqpoint{0.100000in}{0.212622in}}{\pgfqpoint{3.696000in}{3.696000in}}%
\pgfusepath{clip}%
\pgfsetrectcap%
\pgfsetroundjoin%
\pgfsetlinewidth{1.505625pt}%
\definecolor{currentstroke}{rgb}{0.121569,0.466667,0.705882}%
\pgfsetstrokecolor{currentstroke}%
\pgfsetdash{}{0pt}%
\pgfpathmoveto{\pgfqpoint{1.241632in}{2.150693in}}%
\pgfpathlineto{\pgfqpoint{1.931244in}{2.713857in}}%
\pgfpathlineto{\pgfqpoint{2.524279in}{1.774594in}}%
\pgfpathlineto{\pgfqpoint{1.241632in}{2.150693in}}%
\pgfusepath{stroke}%
\end{pgfscope}%
\begin{pgfscope}%
\pgfpathrectangle{\pgfqpoint{0.100000in}{0.212622in}}{\pgfqpoint{3.696000in}{3.696000in}}%
\pgfusepath{clip}%
\pgfsetrectcap%
\pgfsetroundjoin%
\pgfsetlinewidth{1.505625pt}%
\definecolor{currentstroke}{rgb}{1.000000,0.000000,0.000000}%
\pgfsetstrokecolor{currentstroke}%
\pgfsetdash{}{0pt}%
\pgfpathmoveto{\pgfqpoint{1.239117in}{2.150603in}}%
\pgfpathlineto{\pgfqpoint{1.241632in}{2.150693in}}%
\pgfusepath{stroke}%
\end{pgfscope}%
\begin{pgfscope}%
\pgfpathrectangle{\pgfqpoint{0.100000in}{0.212622in}}{\pgfqpoint{3.696000in}{3.696000in}}%
\pgfusepath{clip}%
\pgfsetrectcap%
\pgfsetroundjoin%
\pgfsetlinewidth{1.505625pt}%
\definecolor{currentstroke}{rgb}{1.000000,0.000000,0.000000}%
\pgfsetstrokecolor{currentstroke}%
\pgfsetdash{}{0pt}%
\pgfpathmoveto{\pgfqpoint{1.164194in}{2.386280in}}%
\pgfpathlineto{\pgfqpoint{1.241632in}{2.150693in}}%
\pgfusepath{stroke}%
\end{pgfscope}%
\begin{pgfscope}%
\pgfpathrectangle{\pgfqpoint{0.100000in}{0.212622in}}{\pgfqpoint{3.696000in}{3.696000in}}%
\pgfusepath{clip}%
\pgfsetrectcap%
\pgfsetroundjoin%
\pgfsetlinewidth{1.505625pt}%
\definecolor{currentstroke}{rgb}{1.000000,0.000000,0.000000}%
\pgfsetstrokecolor{currentstroke}%
\pgfsetdash{}{0pt}%
\pgfpathmoveto{\pgfqpoint{1.688882in}{3.396388in}}%
\pgfpathlineto{\pgfqpoint{1.931244in}{2.713857in}}%
\pgfusepath{stroke}%
\end{pgfscope}%
\begin{pgfscope}%
\pgfpathrectangle{\pgfqpoint{0.100000in}{0.212622in}}{\pgfqpoint{3.696000in}{3.696000in}}%
\pgfusepath{clip}%
\pgfsetrectcap%
\pgfsetroundjoin%
\pgfsetlinewidth{1.505625pt}%
\definecolor{currentstroke}{rgb}{1.000000,0.000000,0.000000}%
\pgfsetstrokecolor{currentstroke}%
\pgfsetdash{}{0pt}%
\pgfpathmoveto{\pgfqpoint{2.208324in}{2.571828in}}%
\pgfpathlineto{\pgfqpoint{1.931244in}{2.713857in}}%
\pgfusepath{stroke}%
\end{pgfscope}%
\begin{pgfscope}%
\pgfpathrectangle{\pgfqpoint{0.100000in}{0.212622in}}{\pgfqpoint{3.696000in}{3.696000in}}%
\pgfusepath{clip}%
\pgfsetrectcap%
\pgfsetroundjoin%
\pgfsetlinewidth{1.505625pt}%
\definecolor{currentstroke}{rgb}{1.000000,0.000000,0.000000}%
\pgfsetstrokecolor{currentstroke}%
\pgfsetdash{}{0pt}%
\pgfpathmoveto{\pgfqpoint{2.583411in}{1.064377in}}%
\pgfpathlineto{\pgfqpoint{2.524279in}{1.774594in}}%
\pgfusepath{stroke}%
\end{pgfscope}%
\begin{pgfscope}%
\pgfpathrectangle{\pgfqpoint{0.100000in}{0.212622in}}{\pgfqpoint{3.696000in}{3.696000in}}%
\pgfusepath{clip}%
\pgfsetrectcap%
\pgfsetroundjoin%
\pgfsetlinewidth{1.505625pt}%
\definecolor{currentstroke}{rgb}{1.000000,0.000000,0.000000}%
\pgfsetstrokecolor{currentstroke}%
\pgfsetdash{}{0pt}%
\pgfpathmoveto{\pgfqpoint{2.258658in}{0.941491in}}%
\pgfpathlineto{\pgfqpoint{2.524279in}{1.774594in}}%
\pgfusepath{stroke}%
\end{pgfscope}%
\begin{pgfscope}%
\pgfpathrectangle{\pgfqpoint{0.100000in}{0.212622in}}{\pgfqpoint{3.696000in}{3.696000in}}%
\pgfusepath{clip}%
\pgfsetrectcap%
\pgfsetroundjoin%
\pgfsetlinewidth{1.505625pt}%
\definecolor{currentstroke}{rgb}{1.000000,0.000000,0.000000}%
\pgfsetstrokecolor{currentstroke}%
\pgfsetdash{}{0pt}%
\pgfpathmoveto{\pgfqpoint{1.030831in}{1.105606in}}%
\pgfpathlineto{\pgfqpoint{1.241632in}{2.150693in}}%
\pgfusepath{stroke}%
\end{pgfscope}%
\begin{pgfscope}%
\pgfpathrectangle{\pgfqpoint{0.100000in}{0.212622in}}{\pgfqpoint{3.696000in}{3.696000in}}%
\pgfusepath{clip}%
\pgfsetbuttcap%
\pgfsetroundjoin%
\definecolor{currentfill}{rgb}{1.000000,0.498039,0.054902}%
\pgfsetfillcolor{currentfill}%
\pgfsetfillopacity{0.300000}%
\pgfsetlinewidth{1.003750pt}%
\definecolor{currentstroke}{rgb}{1.000000,0.498039,0.054902}%
\pgfsetstrokecolor{currentstroke}%
\pgfsetstrokeopacity{0.300000}%
\pgfsetdash{}{0pt}%
\pgfpathmoveto{\pgfqpoint{1.688882in}{3.365331in}}%
\pgfpathcurveto{\pgfqpoint{1.697118in}{3.365331in}}{\pgfqpoint{1.705018in}{3.368604in}}{\pgfqpoint{1.710842in}{3.374428in}}%
\pgfpathcurveto{\pgfqpoint{1.716666in}{3.380252in}}{\pgfqpoint{1.719938in}{3.388152in}}{\pgfqpoint{1.719938in}{3.396388in}}%
\pgfpathcurveto{\pgfqpoint{1.719938in}{3.404624in}}{\pgfqpoint{1.716666in}{3.412524in}}{\pgfqpoint{1.710842in}{3.418348in}}%
\pgfpathcurveto{\pgfqpoint{1.705018in}{3.424172in}}{\pgfqpoint{1.697118in}{3.427444in}}{\pgfqpoint{1.688882in}{3.427444in}}%
\pgfpathcurveto{\pgfqpoint{1.680645in}{3.427444in}}{\pgfqpoint{1.672745in}{3.424172in}}{\pgfqpoint{1.666922in}{3.418348in}}%
\pgfpathcurveto{\pgfqpoint{1.661098in}{3.412524in}}{\pgfqpoint{1.657825in}{3.404624in}}{\pgfqpoint{1.657825in}{3.396388in}}%
\pgfpathcurveto{\pgfqpoint{1.657825in}{3.388152in}}{\pgfqpoint{1.661098in}{3.380252in}}{\pgfqpoint{1.666922in}{3.374428in}}%
\pgfpathcurveto{\pgfqpoint{1.672745in}{3.368604in}}{\pgfqpoint{1.680645in}{3.365331in}}{\pgfqpoint{1.688882in}{3.365331in}}%
\pgfpathclose%
\pgfusepath{stroke,fill}%
\end{pgfscope}%
\begin{pgfscope}%
\pgfpathrectangle{\pgfqpoint{0.100000in}{0.212622in}}{\pgfqpoint{3.696000in}{3.696000in}}%
\pgfusepath{clip}%
\pgfsetbuttcap%
\pgfsetroundjoin%
\definecolor{currentfill}{rgb}{1.000000,0.498039,0.054902}%
\pgfsetfillcolor{currentfill}%
\pgfsetfillopacity{0.537620}%
\pgfsetlinewidth{1.003750pt}%
\definecolor{currentstroke}{rgb}{1.000000,0.498039,0.054902}%
\pgfsetstrokecolor{currentstroke}%
\pgfsetstrokeopacity{0.537620}%
\pgfsetdash{}{0pt}%
\pgfpathmoveto{\pgfqpoint{2.208324in}{2.540772in}}%
\pgfpathcurveto{\pgfqpoint{2.216560in}{2.540772in}}{\pgfqpoint{2.224460in}{2.544044in}}{\pgfqpoint{2.230284in}{2.549868in}}%
\pgfpathcurveto{\pgfqpoint{2.236108in}{2.555692in}}{\pgfqpoint{2.239380in}{2.563592in}}{\pgfqpoint{2.239380in}{2.571828in}}%
\pgfpathcurveto{\pgfqpoint{2.239380in}{2.580065in}}{\pgfqpoint{2.236108in}{2.587965in}}{\pgfqpoint{2.230284in}{2.593789in}}%
\pgfpathcurveto{\pgfqpoint{2.224460in}{2.599613in}}{\pgfqpoint{2.216560in}{2.602885in}}{\pgfqpoint{2.208324in}{2.602885in}}%
\pgfpathcurveto{\pgfqpoint{2.200087in}{2.602885in}}{\pgfqpoint{2.192187in}{2.599613in}}{\pgfqpoint{2.186363in}{2.593789in}}%
\pgfpathcurveto{\pgfqpoint{2.180540in}{2.587965in}}{\pgfqpoint{2.177267in}{2.580065in}}{\pgfqpoint{2.177267in}{2.571828in}}%
\pgfpathcurveto{\pgfqpoint{2.177267in}{2.563592in}}{\pgfqpoint{2.180540in}{2.555692in}}{\pgfqpoint{2.186363in}{2.549868in}}%
\pgfpathcurveto{\pgfqpoint{2.192187in}{2.544044in}}{\pgfqpoint{2.200087in}{2.540772in}}{\pgfqpoint{2.208324in}{2.540772in}}%
\pgfpathclose%
\pgfusepath{stroke,fill}%
\end{pgfscope}%
\begin{pgfscope}%
\pgfpathrectangle{\pgfqpoint{0.100000in}{0.212622in}}{\pgfqpoint{3.696000in}{3.696000in}}%
\pgfusepath{clip}%
\pgfsetbuttcap%
\pgfsetroundjoin%
\definecolor{currentfill}{rgb}{1.000000,0.498039,0.054902}%
\pgfsetfillcolor{currentfill}%
\pgfsetfillopacity{0.693706}%
\pgfsetlinewidth{1.003750pt}%
\definecolor{currentstroke}{rgb}{1.000000,0.498039,0.054902}%
\pgfsetstrokecolor{currentstroke}%
\pgfsetstrokeopacity{0.693706}%
\pgfsetdash{}{0pt}%
\pgfpathmoveto{\pgfqpoint{1.164194in}{2.355223in}}%
\pgfpathcurveto{\pgfqpoint{1.172430in}{2.355223in}}{\pgfqpoint{1.180330in}{2.358496in}}{\pgfqpoint{1.186154in}{2.364319in}}%
\pgfpathcurveto{\pgfqpoint{1.191978in}{2.370143in}}{\pgfqpoint{1.195250in}{2.378043in}}{\pgfqpoint{1.195250in}{2.386280in}}%
\pgfpathcurveto{\pgfqpoint{1.195250in}{2.394516in}}{\pgfqpoint{1.191978in}{2.402416in}}{\pgfqpoint{1.186154in}{2.408240in}}%
\pgfpathcurveto{\pgfqpoint{1.180330in}{2.414064in}}{\pgfqpoint{1.172430in}{2.417336in}}{\pgfqpoint{1.164194in}{2.417336in}}%
\pgfpathcurveto{\pgfqpoint{1.155957in}{2.417336in}}{\pgfqpoint{1.148057in}{2.414064in}}{\pgfqpoint{1.142233in}{2.408240in}}%
\pgfpathcurveto{\pgfqpoint{1.136409in}{2.402416in}}{\pgfqpoint{1.133137in}{2.394516in}}{\pgfqpoint{1.133137in}{2.386280in}}%
\pgfpathcurveto{\pgfqpoint{1.133137in}{2.378043in}}{\pgfqpoint{1.136409in}{2.370143in}}{\pgfqpoint{1.142233in}{2.364319in}}%
\pgfpathcurveto{\pgfqpoint{1.148057in}{2.358496in}}{\pgfqpoint{1.155957in}{2.355223in}}{\pgfqpoint{1.164194in}{2.355223in}}%
\pgfpathclose%
\pgfusepath{stroke,fill}%
\end{pgfscope}%
\begin{pgfscope}%
\pgfpathrectangle{\pgfqpoint{0.100000in}{0.212622in}}{\pgfqpoint{3.696000in}{3.696000in}}%
\pgfusepath{clip}%
\pgfsetbuttcap%
\pgfsetroundjoin%
\definecolor{currentfill}{rgb}{1.000000,0.498039,0.054902}%
\pgfsetfillcolor{currentfill}%
\pgfsetfillopacity{0.735388}%
\pgfsetlinewidth{1.003750pt}%
\definecolor{currentstroke}{rgb}{1.000000,0.498039,0.054902}%
\pgfsetstrokecolor{currentstroke}%
\pgfsetstrokeopacity{0.735388}%
\pgfsetdash{}{0pt}%
\pgfpathmoveto{\pgfqpoint{1.030831in}{1.074549in}}%
\pgfpathcurveto{\pgfqpoint{1.039067in}{1.074549in}}{\pgfqpoint{1.046967in}{1.077822in}}{\pgfqpoint{1.052791in}{1.083646in}}%
\pgfpathcurveto{\pgfqpoint{1.058615in}{1.089470in}}{\pgfqpoint{1.061887in}{1.097370in}}{\pgfqpoint{1.061887in}{1.105606in}}%
\pgfpathcurveto{\pgfqpoint{1.061887in}{1.113842in}}{\pgfqpoint{1.058615in}{1.121742in}}{\pgfqpoint{1.052791in}{1.127566in}}%
\pgfpathcurveto{\pgfqpoint{1.046967in}{1.133390in}}{\pgfqpoint{1.039067in}{1.136662in}}{\pgfqpoint{1.030831in}{1.136662in}}%
\pgfpathcurveto{\pgfqpoint{1.022594in}{1.136662in}}{\pgfqpoint{1.014694in}{1.133390in}}{\pgfqpoint{1.008870in}{1.127566in}}%
\pgfpathcurveto{\pgfqpoint{1.003046in}{1.121742in}}{\pgfqpoint{0.999774in}{1.113842in}}{\pgfqpoint{0.999774in}{1.105606in}}%
\pgfpathcurveto{\pgfqpoint{0.999774in}{1.097370in}}{\pgfqpoint{1.003046in}{1.089470in}}{\pgfqpoint{1.008870in}{1.083646in}}%
\pgfpathcurveto{\pgfqpoint{1.014694in}{1.077822in}}{\pgfqpoint{1.022594in}{1.074549in}}{\pgfqpoint{1.030831in}{1.074549in}}%
\pgfpathclose%
\pgfusepath{stroke,fill}%
\end{pgfscope}%
\begin{pgfscope}%
\pgfpathrectangle{\pgfqpoint{0.100000in}{0.212622in}}{\pgfqpoint{3.696000in}{3.696000in}}%
\pgfusepath{clip}%
\pgfsetbuttcap%
\pgfsetroundjoin%
\definecolor{currentfill}{rgb}{1.000000,0.498039,0.054902}%
\pgfsetfillcolor{currentfill}%
\pgfsetfillopacity{0.885061}%
\pgfsetlinewidth{1.003750pt}%
\definecolor{currentstroke}{rgb}{1.000000,0.498039,0.054902}%
\pgfsetstrokecolor{currentstroke}%
\pgfsetstrokeopacity{0.885061}%
\pgfsetdash{}{0pt}%
\pgfpathmoveto{\pgfqpoint{1.239117in}{2.119546in}}%
\pgfpathcurveto{\pgfqpoint{1.247353in}{2.119546in}}{\pgfqpoint{1.255253in}{2.122819in}}{\pgfqpoint{1.261077in}{2.128642in}}%
\pgfpathcurveto{\pgfqpoint{1.266901in}{2.134466in}}{\pgfqpoint{1.270173in}{2.142366in}}{\pgfqpoint{1.270173in}{2.150603in}}%
\pgfpathcurveto{\pgfqpoint{1.270173in}{2.158839in}}{\pgfqpoint{1.266901in}{2.166739in}}{\pgfqpoint{1.261077in}{2.172563in}}%
\pgfpathcurveto{\pgfqpoint{1.255253in}{2.178387in}}{\pgfqpoint{1.247353in}{2.181659in}}{\pgfqpoint{1.239117in}{2.181659in}}%
\pgfpathcurveto{\pgfqpoint{1.230881in}{2.181659in}}{\pgfqpoint{1.222981in}{2.178387in}}{\pgfqpoint{1.217157in}{2.172563in}}%
\pgfpathcurveto{\pgfqpoint{1.211333in}{2.166739in}}{\pgfqpoint{1.208060in}{2.158839in}}{\pgfqpoint{1.208060in}{2.150603in}}%
\pgfpathcurveto{\pgfqpoint{1.208060in}{2.142366in}}{\pgfqpoint{1.211333in}{2.134466in}}{\pgfqpoint{1.217157in}{2.128642in}}%
\pgfpathcurveto{\pgfqpoint{1.222981in}{2.122819in}}{\pgfqpoint{1.230881in}{2.119546in}}{\pgfqpoint{1.239117in}{2.119546in}}%
\pgfpathclose%
\pgfusepath{stroke,fill}%
\end{pgfscope}%
\begin{pgfscope}%
\pgfpathrectangle{\pgfqpoint{0.100000in}{0.212622in}}{\pgfqpoint{3.696000in}{3.696000in}}%
\pgfusepath{clip}%
\pgfsetbuttcap%
\pgfsetroundjoin%
\definecolor{currentfill}{rgb}{1.000000,0.498039,0.054902}%
\pgfsetfillcolor{currentfill}%
\pgfsetfillopacity{0.970083}%
\pgfsetlinewidth{1.003750pt}%
\definecolor{currentstroke}{rgb}{1.000000,0.498039,0.054902}%
\pgfsetstrokecolor{currentstroke}%
\pgfsetstrokeopacity{0.970083}%
\pgfsetdash{}{0pt}%
\pgfpathmoveto{\pgfqpoint{2.583411in}{1.033320in}}%
\pgfpathcurveto{\pgfqpoint{2.591647in}{1.033320in}}{\pgfqpoint{2.599547in}{1.036593in}}{\pgfqpoint{2.605371in}{1.042417in}}%
\pgfpathcurveto{\pgfqpoint{2.611195in}{1.048240in}}{\pgfqpoint{2.614467in}{1.056141in}}{\pgfqpoint{2.614467in}{1.064377in}}%
\pgfpathcurveto{\pgfqpoint{2.614467in}{1.072613in}}{\pgfqpoint{2.611195in}{1.080513in}}{\pgfqpoint{2.605371in}{1.086337in}}%
\pgfpathcurveto{\pgfqpoint{2.599547in}{1.092161in}}{\pgfqpoint{2.591647in}{1.095433in}}{\pgfqpoint{2.583411in}{1.095433in}}%
\pgfpathcurveto{\pgfqpoint{2.575175in}{1.095433in}}{\pgfqpoint{2.567275in}{1.092161in}}{\pgfqpoint{2.561451in}{1.086337in}}%
\pgfpathcurveto{\pgfqpoint{2.555627in}{1.080513in}}{\pgfqpoint{2.552354in}{1.072613in}}{\pgfqpoint{2.552354in}{1.064377in}}%
\pgfpathcurveto{\pgfqpoint{2.552354in}{1.056141in}}{\pgfqpoint{2.555627in}{1.048240in}}{\pgfqpoint{2.561451in}{1.042417in}}%
\pgfpathcurveto{\pgfqpoint{2.567275in}{1.036593in}}{\pgfqpoint{2.575175in}{1.033320in}}{\pgfqpoint{2.583411in}{1.033320in}}%
\pgfpathclose%
\pgfusepath{stroke,fill}%
\end{pgfscope}%
\begin{pgfscope}%
\pgfpathrectangle{\pgfqpoint{0.100000in}{0.212622in}}{\pgfqpoint{3.696000in}{3.696000in}}%
\pgfusepath{clip}%
\pgfsetbuttcap%
\pgfsetroundjoin%
\definecolor{currentfill}{rgb}{1.000000,0.498039,0.054902}%
\pgfsetfillcolor{currentfill}%
\pgfsetlinewidth{1.003750pt}%
\definecolor{currentstroke}{rgb}{1.000000,0.498039,0.054902}%
\pgfsetstrokecolor{currentstroke}%
\pgfsetdash{}{0pt}%
\pgfpathmoveto{\pgfqpoint{2.258658in}{0.910434in}}%
\pgfpathcurveto{\pgfqpoint{2.266895in}{0.910434in}}{\pgfqpoint{2.274795in}{0.913707in}}{\pgfqpoint{2.280619in}{0.919531in}}%
\pgfpathcurveto{\pgfqpoint{2.286443in}{0.925354in}}{\pgfqpoint{2.289715in}{0.933255in}}{\pgfqpoint{2.289715in}{0.941491in}}%
\pgfpathcurveto{\pgfqpoint{2.289715in}{0.949727in}}{\pgfqpoint{2.286443in}{0.957627in}}{\pgfqpoint{2.280619in}{0.963451in}}%
\pgfpathcurveto{\pgfqpoint{2.274795in}{0.969275in}}{\pgfqpoint{2.266895in}{0.972547in}}{\pgfqpoint{2.258658in}{0.972547in}}%
\pgfpathcurveto{\pgfqpoint{2.250422in}{0.972547in}}{\pgfqpoint{2.242522in}{0.969275in}}{\pgfqpoint{2.236698in}{0.963451in}}%
\pgfpathcurveto{\pgfqpoint{2.230874in}{0.957627in}}{\pgfqpoint{2.227602in}{0.949727in}}{\pgfqpoint{2.227602in}{0.941491in}}%
\pgfpathcurveto{\pgfqpoint{2.227602in}{0.933255in}}{\pgfqpoint{2.230874in}{0.925354in}}{\pgfqpoint{2.236698in}{0.919531in}}%
\pgfpathcurveto{\pgfqpoint{2.242522in}{0.913707in}}{\pgfqpoint{2.250422in}{0.910434in}}{\pgfqpoint{2.258658in}{0.910434in}}%
\pgfpathclose%
\pgfusepath{stroke,fill}%
\end{pgfscope}%
\begin{pgfscope}%
\definecolor{textcolor}{rgb}{0.000000,0.000000,0.000000}%
\pgfsetstrokecolor{textcolor}%
\pgfsetfillcolor{textcolor}%
\pgftext[x=1.948000in,y=3.991956in,,base]{\color{textcolor}\rmfamily\fontsize{12.000000}{14.400000}\selectfont EKF}%
\end{pgfscope}%
\begin{pgfscope}%
\pgfpathrectangle{\pgfqpoint{0.100000in}{0.212622in}}{\pgfqpoint{3.696000in}{3.696000in}}%
\pgfusepath{clip}%
\pgfsetbuttcap%
\pgfsetroundjoin%
\definecolor{currentfill}{rgb}{0.121569,0.466667,0.705882}%
\pgfsetfillcolor{currentfill}%
\pgfsetfillopacity{0.300000}%
\pgfsetlinewidth{1.003750pt}%
\definecolor{currentstroke}{rgb}{0.121569,0.466667,0.705882}%
\pgfsetstrokecolor{currentstroke}%
\pgfsetstrokeopacity{0.300000}%
\pgfsetdash{}{0pt}%
\pgfpathmoveto{\pgfqpoint{1.688882in}{3.365331in}}%
\pgfpathcurveto{\pgfqpoint{1.697118in}{3.365331in}}{\pgfqpoint{1.705018in}{3.368604in}}{\pgfqpoint{1.710842in}{3.374428in}}%
\pgfpathcurveto{\pgfqpoint{1.716666in}{3.380252in}}{\pgfqpoint{1.719938in}{3.388152in}}{\pgfqpoint{1.719938in}{3.396388in}}%
\pgfpathcurveto{\pgfqpoint{1.719938in}{3.404624in}}{\pgfqpoint{1.716666in}{3.412524in}}{\pgfqpoint{1.710842in}{3.418348in}}%
\pgfpathcurveto{\pgfqpoint{1.705018in}{3.424172in}}{\pgfqpoint{1.697118in}{3.427444in}}{\pgfqpoint{1.688882in}{3.427444in}}%
\pgfpathcurveto{\pgfqpoint{1.680645in}{3.427444in}}{\pgfqpoint{1.672745in}{3.424172in}}{\pgfqpoint{1.666922in}{3.418348in}}%
\pgfpathcurveto{\pgfqpoint{1.661098in}{3.412524in}}{\pgfqpoint{1.657825in}{3.404624in}}{\pgfqpoint{1.657825in}{3.396388in}}%
\pgfpathcurveto{\pgfqpoint{1.657825in}{3.388152in}}{\pgfqpoint{1.661098in}{3.380252in}}{\pgfqpoint{1.666922in}{3.374428in}}%
\pgfpathcurveto{\pgfqpoint{1.672745in}{3.368604in}}{\pgfqpoint{1.680645in}{3.365331in}}{\pgfqpoint{1.688882in}{3.365331in}}%
\pgfpathclose%
\pgfusepath{stroke,fill}%
\end{pgfscope}%
\begin{pgfscope}%
\pgfpathrectangle{\pgfqpoint{0.100000in}{0.212622in}}{\pgfqpoint{3.696000in}{3.696000in}}%
\pgfusepath{clip}%
\pgfsetbuttcap%
\pgfsetroundjoin%
\definecolor{currentfill}{rgb}{0.121569,0.466667,0.705882}%
\pgfsetfillcolor{currentfill}%
\pgfsetfillopacity{0.300030}%
\pgfsetlinewidth{1.003750pt}%
\definecolor{currentstroke}{rgb}{0.121569,0.466667,0.705882}%
\pgfsetstrokecolor{currentstroke}%
\pgfsetstrokeopacity{0.300030}%
\pgfsetdash{}{0pt}%
\pgfpathmoveto{\pgfqpoint{1.685685in}{3.366792in}}%
\pgfpathcurveto{\pgfqpoint{1.693921in}{3.366792in}}{\pgfqpoint{1.701822in}{3.370065in}}{\pgfqpoint{1.707645in}{3.375889in}}%
\pgfpathcurveto{\pgfqpoint{1.713469in}{3.381712in}}{\pgfqpoint{1.716742in}{3.389612in}}{\pgfqpoint{1.716742in}{3.397849in}}%
\pgfpathcurveto{\pgfqpoint{1.716742in}{3.406085in}}{\pgfqpoint{1.713469in}{3.413985in}}{\pgfqpoint{1.707645in}{3.419809in}}%
\pgfpathcurveto{\pgfqpoint{1.701822in}{3.425633in}}{\pgfqpoint{1.693921in}{3.428905in}}{\pgfqpoint{1.685685in}{3.428905in}}%
\pgfpathcurveto{\pgfqpoint{1.677449in}{3.428905in}}{\pgfqpoint{1.669549in}{3.425633in}}{\pgfqpoint{1.663725in}{3.419809in}}%
\pgfpathcurveto{\pgfqpoint{1.657901in}{3.413985in}}{\pgfqpoint{1.654629in}{3.406085in}}{\pgfqpoint{1.654629in}{3.397849in}}%
\pgfpathcurveto{\pgfqpoint{1.654629in}{3.389612in}}{\pgfqpoint{1.657901in}{3.381712in}}{\pgfqpoint{1.663725in}{3.375889in}}%
\pgfpathcurveto{\pgfqpoint{1.669549in}{3.370065in}}{\pgfqpoint{1.677449in}{3.366792in}}{\pgfqpoint{1.685685in}{3.366792in}}%
\pgfpathclose%
\pgfusepath{stroke,fill}%
\end{pgfscope}%
\begin{pgfscope}%
\pgfpathrectangle{\pgfqpoint{0.100000in}{0.212622in}}{\pgfqpoint{3.696000in}{3.696000in}}%
\pgfusepath{clip}%
\pgfsetbuttcap%
\pgfsetroundjoin%
\definecolor{currentfill}{rgb}{0.121569,0.466667,0.705882}%
\pgfsetfillcolor{currentfill}%
\pgfsetfillopacity{0.300099}%
\pgfsetlinewidth{1.003750pt}%
\definecolor{currentstroke}{rgb}{0.121569,0.466667,0.705882}%
\pgfsetstrokecolor{currentstroke}%
\pgfsetstrokeopacity{0.300099}%
\pgfsetdash{}{0pt}%
\pgfpathmoveto{\pgfqpoint{1.693927in}{3.362904in}}%
\pgfpathcurveto{\pgfqpoint{1.702164in}{3.362904in}}{\pgfqpoint{1.710064in}{3.366177in}}{\pgfqpoint{1.715888in}{3.372001in}}%
\pgfpathcurveto{\pgfqpoint{1.721712in}{3.377824in}}{\pgfqpoint{1.724984in}{3.385725in}}{\pgfqpoint{1.724984in}{3.393961in}}%
\pgfpathcurveto{\pgfqpoint{1.724984in}{3.402197in}}{\pgfqpoint{1.721712in}{3.410097in}}{\pgfqpoint{1.715888in}{3.415921in}}%
\pgfpathcurveto{\pgfqpoint{1.710064in}{3.421745in}}{\pgfqpoint{1.702164in}{3.425017in}}{\pgfqpoint{1.693927in}{3.425017in}}%
\pgfpathcurveto{\pgfqpoint{1.685691in}{3.425017in}}{\pgfqpoint{1.677791in}{3.421745in}}{\pgfqpoint{1.671967in}{3.415921in}}%
\pgfpathcurveto{\pgfqpoint{1.666143in}{3.410097in}}{\pgfqpoint{1.662871in}{3.402197in}}{\pgfqpoint{1.662871in}{3.393961in}}%
\pgfpathcurveto{\pgfqpoint{1.662871in}{3.385725in}}{\pgfqpoint{1.666143in}{3.377824in}}{\pgfqpoint{1.671967in}{3.372001in}}%
\pgfpathcurveto{\pgfqpoint{1.677791in}{3.366177in}}{\pgfqpoint{1.685691in}{3.362904in}}{\pgfqpoint{1.693927in}{3.362904in}}%
\pgfpathclose%
\pgfusepath{stroke,fill}%
\end{pgfscope}%
\begin{pgfscope}%
\pgfpathrectangle{\pgfqpoint{0.100000in}{0.212622in}}{\pgfqpoint{3.696000in}{3.696000in}}%
\pgfusepath{clip}%
\pgfsetbuttcap%
\pgfsetroundjoin%
\definecolor{currentfill}{rgb}{0.121569,0.466667,0.705882}%
\pgfsetfillcolor{currentfill}%
\pgfsetfillopacity{0.300176}%
\pgfsetlinewidth{1.003750pt}%
\definecolor{currentstroke}{rgb}{0.121569,0.466667,0.705882}%
\pgfsetstrokecolor{currentstroke}%
\pgfsetstrokeopacity{0.300176}%
\pgfsetdash{}{0pt}%
\pgfpathmoveto{\pgfqpoint{1.679823in}{3.369146in}}%
\pgfpathcurveto{\pgfqpoint{1.688060in}{3.369146in}}{\pgfqpoint{1.695960in}{3.372419in}}{\pgfqpoint{1.701784in}{3.378243in}}%
\pgfpathcurveto{\pgfqpoint{1.707608in}{3.384067in}}{\pgfqpoint{1.710880in}{3.391967in}}{\pgfqpoint{1.710880in}{3.400203in}}%
\pgfpathcurveto{\pgfqpoint{1.710880in}{3.408439in}}{\pgfqpoint{1.707608in}{3.416339in}}{\pgfqpoint{1.701784in}{3.422163in}}%
\pgfpathcurveto{\pgfqpoint{1.695960in}{3.427987in}}{\pgfqpoint{1.688060in}{3.431259in}}{\pgfqpoint{1.679823in}{3.431259in}}%
\pgfpathcurveto{\pgfqpoint{1.671587in}{3.431259in}}{\pgfqpoint{1.663687in}{3.427987in}}{\pgfqpoint{1.657863in}{3.422163in}}%
\pgfpathcurveto{\pgfqpoint{1.652039in}{3.416339in}}{\pgfqpoint{1.648767in}{3.408439in}}{\pgfqpoint{1.648767in}{3.400203in}}%
\pgfpathcurveto{\pgfqpoint{1.648767in}{3.391967in}}{\pgfqpoint{1.652039in}{3.384067in}}{\pgfqpoint{1.657863in}{3.378243in}}%
\pgfpathcurveto{\pgfqpoint{1.663687in}{3.372419in}}{\pgfqpoint{1.671587in}{3.369146in}}{\pgfqpoint{1.679823in}{3.369146in}}%
\pgfpathclose%
\pgfusepath{stroke,fill}%
\end{pgfscope}%
\begin{pgfscope}%
\pgfpathrectangle{\pgfqpoint{0.100000in}{0.212622in}}{\pgfqpoint{3.696000in}{3.696000in}}%
\pgfusepath{clip}%
\pgfsetbuttcap%
\pgfsetroundjoin%
\definecolor{currentfill}{rgb}{0.121569,0.466667,0.705882}%
\pgfsetfillcolor{currentfill}%
\pgfsetfillopacity{0.300347}%
\pgfsetlinewidth{1.003750pt}%
\definecolor{currentstroke}{rgb}{0.121569,0.466667,0.705882}%
\pgfsetstrokecolor{currentstroke}%
\pgfsetstrokeopacity{0.300347}%
\pgfsetdash{}{0pt}%
\pgfpathmoveto{\pgfqpoint{1.702301in}{3.358315in}}%
\pgfpathcurveto{\pgfqpoint{1.710538in}{3.358315in}}{\pgfqpoint{1.718438in}{3.361587in}}{\pgfqpoint{1.724262in}{3.367411in}}%
\pgfpathcurveto{\pgfqpoint{1.730086in}{3.373235in}}{\pgfqpoint{1.733358in}{3.381135in}}{\pgfqpoint{1.733358in}{3.389371in}}%
\pgfpathcurveto{\pgfqpoint{1.733358in}{3.397607in}}{\pgfqpoint{1.730086in}{3.405507in}}{\pgfqpoint{1.724262in}{3.411331in}}%
\pgfpathcurveto{\pgfqpoint{1.718438in}{3.417155in}}{\pgfqpoint{1.710538in}{3.420428in}}{\pgfqpoint{1.702301in}{3.420428in}}%
\pgfpathcurveto{\pgfqpoint{1.694065in}{3.420428in}}{\pgfqpoint{1.686165in}{3.417155in}}{\pgfqpoint{1.680341in}{3.411331in}}%
\pgfpathcurveto{\pgfqpoint{1.674517in}{3.405507in}}{\pgfqpoint{1.671245in}{3.397607in}}{\pgfqpoint{1.671245in}{3.389371in}}%
\pgfpathcurveto{\pgfqpoint{1.671245in}{3.381135in}}{\pgfqpoint{1.674517in}{3.373235in}}{\pgfqpoint{1.680341in}{3.367411in}}%
\pgfpathcurveto{\pgfqpoint{1.686165in}{3.361587in}}{\pgfqpoint{1.694065in}{3.358315in}}{\pgfqpoint{1.702301in}{3.358315in}}%
\pgfpathclose%
\pgfusepath{stroke,fill}%
\end{pgfscope}%
\begin{pgfscope}%
\pgfpathrectangle{\pgfqpoint{0.100000in}{0.212622in}}{\pgfqpoint{3.696000in}{3.696000in}}%
\pgfusepath{clip}%
\pgfsetbuttcap%
\pgfsetroundjoin%
\definecolor{currentfill}{rgb}{0.121569,0.466667,0.705882}%
\pgfsetfillcolor{currentfill}%
\pgfsetfillopacity{0.300394}%
\pgfsetlinewidth{1.003750pt}%
\definecolor{currentstroke}{rgb}{0.121569,0.466667,0.705882}%
\pgfsetstrokecolor{currentstroke}%
\pgfsetstrokeopacity{0.300394}%
\pgfsetdash{}{0pt}%
\pgfpathmoveto{\pgfqpoint{1.675794in}{3.370675in}}%
\pgfpathcurveto{\pgfqpoint{1.684030in}{3.370675in}}{\pgfqpoint{1.691930in}{3.373947in}}{\pgfqpoint{1.697754in}{3.379771in}}%
\pgfpathcurveto{\pgfqpoint{1.703578in}{3.385595in}}{\pgfqpoint{1.706850in}{3.393495in}}{\pgfqpoint{1.706850in}{3.401731in}}%
\pgfpathcurveto{\pgfqpoint{1.706850in}{3.409967in}}{\pgfqpoint{1.703578in}{3.417867in}}{\pgfqpoint{1.697754in}{3.423691in}}%
\pgfpathcurveto{\pgfqpoint{1.691930in}{3.429515in}}{\pgfqpoint{1.684030in}{3.432788in}}{\pgfqpoint{1.675794in}{3.432788in}}%
\pgfpathcurveto{\pgfqpoint{1.667557in}{3.432788in}}{\pgfqpoint{1.659657in}{3.429515in}}{\pgfqpoint{1.653833in}{3.423691in}}%
\pgfpathcurveto{\pgfqpoint{1.648009in}{3.417867in}}{\pgfqpoint{1.644737in}{3.409967in}}{\pgfqpoint{1.644737in}{3.401731in}}%
\pgfpathcurveto{\pgfqpoint{1.644737in}{3.393495in}}{\pgfqpoint{1.648009in}{3.385595in}}{\pgfqpoint{1.653833in}{3.379771in}}%
\pgfpathcurveto{\pgfqpoint{1.659657in}{3.373947in}}{\pgfqpoint{1.667557in}{3.370675in}}{\pgfqpoint{1.675794in}{3.370675in}}%
\pgfpathclose%
\pgfusepath{stroke,fill}%
\end{pgfscope}%
\begin{pgfscope}%
\pgfpathrectangle{\pgfqpoint{0.100000in}{0.212622in}}{\pgfqpoint{3.696000in}{3.696000in}}%
\pgfusepath{clip}%
\pgfsetbuttcap%
\pgfsetroundjoin%
\definecolor{currentfill}{rgb}{0.121569,0.466667,0.705882}%
\pgfsetfillcolor{currentfill}%
\pgfsetfillopacity{0.300628}%
\pgfsetlinewidth{1.003750pt}%
\definecolor{currentstroke}{rgb}{0.121569,0.466667,0.705882}%
\pgfsetstrokecolor{currentstroke}%
\pgfsetstrokeopacity{0.300628}%
\pgfsetdash{}{0pt}%
\pgfpathmoveto{\pgfqpoint{1.706775in}{3.355681in}}%
\pgfpathcurveto{\pgfqpoint{1.715012in}{3.355681in}}{\pgfqpoint{1.722912in}{3.358953in}}{\pgfqpoint{1.728735in}{3.364777in}}%
\pgfpathcurveto{\pgfqpoint{1.734559in}{3.370601in}}{\pgfqpoint{1.737832in}{3.378501in}}{\pgfqpoint{1.737832in}{3.386737in}}%
\pgfpathcurveto{\pgfqpoint{1.737832in}{3.394974in}}{\pgfqpoint{1.734559in}{3.402874in}}{\pgfqpoint{1.728735in}{3.408698in}}%
\pgfpathcurveto{\pgfqpoint{1.722912in}{3.414522in}}{\pgfqpoint{1.715012in}{3.417794in}}{\pgfqpoint{1.706775in}{3.417794in}}%
\pgfpathcurveto{\pgfqpoint{1.698539in}{3.417794in}}{\pgfqpoint{1.690639in}{3.414522in}}{\pgfqpoint{1.684815in}{3.408698in}}%
\pgfpathcurveto{\pgfqpoint{1.678991in}{3.402874in}}{\pgfqpoint{1.675719in}{3.394974in}}{\pgfqpoint{1.675719in}{3.386737in}}%
\pgfpathcurveto{\pgfqpoint{1.675719in}{3.378501in}}{\pgfqpoint{1.678991in}{3.370601in}}{\pgfqpoint{1.684815in}{3.364777in}}%
\pgfpathcurveto{\pgfqpoint{1.690639in}{3.358953in}}{\pgfqpoint{1.698539in}{3.355681in}}{\pgfqpoint{1.706775in}{3.355681in}}%
\pgfpathclose%
\pgfusepath{stroke,fill}%
\end{pgfscope}%
\begin{pgfscope}%
\pgfpathrectangle{\pgfqpoint{0.100000in}{0.212622in}}{\pgfqpoint{3.696000in}{3.696000in}}%
\pgfusepath{clip}%
\pgfsetbuttcap%
\pgfsetroundjoin%
\definecolor{currentfill}{rgb}{0.121569,0.466667,0.705882}%
\pgfsetfillcolor{currentfill}%
\pgfsetfillopacity{0.300925}%
\pgfsetlinewidth{1.003750pt}%
\definecolor{currentstroke}{rgb}{0.121569,0.466667,0.705882}%
\pgfsetstrokecolor{currentstroke}%
\pgfsetstrokeopacity{0.300925}%
\pgfsetdash{}{0pt}%
\pgfpathmoveto{\pgfqpoint{1.668465in}{3.373090in}}%
\pgfpathcurveto{\pgfqpoint{1.676701in}{3.373090in}}{\pgfqpoint{1.684601in}{3.376363in}}{\pgfqpoint{1.690425in}{3.382187in}}%
\pgfpathcurveto{\pgfqpoint{1.696249in}{3.388011in}}{\pgfqpoint{1.699521in}{3.395911in}}{\pgfqpoint{1.699521in}{3.404147in}}%
\pgfpathcurveto{\pgfqpoint{1.699521in}{3.412383in}}{\pgfqpoint{1.696249in}{3.420283in}}{\pgfqpoint{1.690425in}{3.426107in}}%
\pgfpathcurveto{\pgfqpoint{1.684601in}{3.431931in}}{\pgfqpoint{1.676701in}{3.435203in}}{\pgfqpoint{1.668465in}{3.435203in}}%
\pgfpathcurveto{\pgfqpoint{1.660228in}{3.435203in}}{\pgfqpoint{1.652328in}{3.431931in}}{\pgfqpoint{1.646504in}{3.426107in}}%
\pgfpathcurveto{\pgfqpoint{1.640681in}{3.420283in}}{\pgfqpoint{1.637408in}{3.412383in}}{\pgfqpoint{1.637408in}{3.404147in}}%
\pgfpathcurveto{\pgfqpoint{1.637408in}{3.395911in}}{\pgfqpoint{1.640681in}{3.388011in}}{\pgfqpoint{1.646504in}{3.382187in}}%
\pgfpathcurveto{\pgfqpoint{1.652328in}{3.376363in}}{\pgfqpoint{1.660228in}{3.373090in}}{\pgfqpoint{1.668465in}{3.373090in}}%
\pgfpathclose%
\pgfusepath{stroke,fill}%
\end{pgfscope}%
\begin{pgfscope}%
\pgfpathrectangle{\pgfqpoint{0.100000in}{0.212622in}}{\pgfqpoint{3.696000in}{3.696000in}}%
\pgfusepath{clip}%
\pgfsetbuttcap%
\pgfsetroundjoin%
\definecolor{currentfill}{rgb}{0.121569,0.466667,0.705882}%
\pgfsetfillcolor{currentfill}%
\pgfsetfillopacity{0.301102}%
\pgfsetlinewidth{1.003750pt}%
\definecolor{currentstroke}{rgb}{0.121569,0.466667,0.705882}%
\pgfsetstrokecolor{currentstroke}%
\pgfsetstrokeopacity{0.301102}%
\pgfsetdash{}{0pt}%
\pgfpathmoveto{\pgfqpoint{1.713779in}{3.351085in}}%
\pgfpathcurveto{\pgfqpoint{1.722016in}{3.351085in}}{\pgfqpoint{1.729916in}{3.354357in}}{\pgfqpoint{1.735740in}{3.360181in}}%
\pgfpathcurveto{\pgfqpoint{1.741563in}{3.366005in}}{\pgfqpoint{1.744836in}{3.373905in}}{\pgfqpoint{1.744836in}{3.382141in}}%
\pgfpathcurveto{\pgfqpoint{1.744836in}{3.390377in}}{\pgfqpoint{1.741563in}{3.398277in}}{\pgfqpoint{1.735740in}{3.404101in}}%
\pgfpathcurveto{\pgfqpoint{1.729916in}{3.409925in}}{\pgfqpoint{1.722016in}{3.413198in}}{\pgfqpoint{1.713779in}{3.413198in}}%
\pgfpathcurveto{\pgfqpoint{1.705543in}{3.413198in}}{\pgfqpoint{1.697643in}{3.409925in}}{\pgfqpoint{1.691819in}{3.404101in}}%
\pgfpathcurveto{\pgfqpoint{1.685995in}{3.398277in}}{\pgfqpoint{1.682723in}{3.390377in}}{\pgfqpoint{1.682723in}{3.382141in}}%
\pgfpathcurveto{\pgfqpoint{1.682723in}{3.373905in}}{\pgfqpoint{1.685995in}{3.366005in}}{\pgfqpoint{1.691819in}{3.360181in}}%
\pgfpathcurveto{\pgfqpoint{1.697643in}{3.354357in}}{\pgfqpoint{1.705543in}{3.351085in}}{\pgfqpoint{1.713779in}{3.351085in}}%
\pgfpathclose%
\pgfusepath{stroke,fill}%
\end{pgfscope}%
\begin{pgfscope}%
\pgfpathrectangle{\pgfqpoint{0.100000in}{0.212622in}}{\pgfqpoint{3.696000in}{3.696000in}}%
\pgfusepath{clip}%
\pgfsetbuttcap%
\pgfsetroundjoin%
\definecolor{currentfill}{rgb}{0.121569,0.466667,0.705882}%
\pgfsetfillcolor{currentfill}%
\pgfsetfillopacity{0.301486}%
\pgfsetlinewidth{1.003750pt}%
\definecolor{currentstroke}{rgb}{0.121569,0.466667,0.705882}%
\pgfsetstrokecolor{currentstroke}%
\pgfsetstrokeopacity{0.301486}%
\pgfsetdash{}{0pt}%
\pgfpathmoveto{\pgfqpoint{1.717490in}{3.348524in}}%
\pgfpathcurveto{\pgfqpoint{1.725726in}{3.348524in}}{\pgfqpoint{1.733626in}{3.351796in}}{\pgfqpoint{1.739450in}{3.357620in}}%
\pgfpathcurveto{\pgfqpoint{1.745274in}{3.363444in}}{\pgfqpoint{1.748546in}{3.371344in}}{\pgfqpoint{1.748546in}{3.379580in}}%
\pgfpathcurveto{\pgfqpoint{1.748546in}{3.387816in}}{\pgfqpoint{1.745274in}{3.395717in}}{\pgfqpoint{1.739450in}{3.401540in}}%
\pgfpathcurveto{\pgfqpoint{1.733626in}{3.407364in}}{\pgfqpoint{1.725726in}{3.410637in}}{\pgfqpoint{1.717490in}{3.410637in}}%
\pgfpathcurveto{\pgfqpoint{1.709254in}{3.410637in}}{\pgfqpoint{1.701353in}{3.407364in}}{\pgfqpoint{1.695530in}{3.401540in}}%
\pgfpathcurveto{\pgfqpoint{1.689706in}{3.395717in}}{\pgfqpoint{1.686433in}{3.387816in}}{\pgfqpoint{1.686433in}{3.379580in}}%
\pgfpathcurveto{\pgfqpoint{1.686433in}{3.371344in}}{\pgfqpoint{1.689706in}{3.363444in}}{\pgfqpoint{1.695530in}{3.357620in}}%
\pgfpathcurveto{\pgfqpoint{1.701353in}{3.351796in}}{\pgfqpoint{1.709254in}{3.348524in}}{\pgfqpoint{1.717490in}{3.348524in}}%
\pgfpathclose%
\pgfusepath{stroke,fill}%
\end{pgfscope}%
\begin{pgfscope}%
\pgfpathrectangle{\pgfqpoint{0.100000in}{0.212622in}}{\pgfqpoint{3.696000in}{3.696000in}}%
\pgfusepath{clip}%
\pgfsetbuttcap%
\pgfsetroundjoin%
\definecolor{currentfill}{rgb}{0.121569,0.466667,0.705882}%
\pgfsetfillcolor{currentfill}%
\pgfsetfillopacity{0.301711}%
\pgfsetlinewidth{1.003750pt}%
\definecolor{currentstroke}{rgb}{0.121569,0.466667,0.705882}%
\pgfsetstrokecolor{currentstroke}%
\pgfsetstrokeopacity{0.301711}%
\pgfsetdash{}{0pt}%
\pgfpathmoveto{\pgfqpoint{1.719486in}{3.347024in}}%
\pgfpathcurveto{\pgfqpoint{1.727722in}{3.347024in}}{\pgfqpoint{1.735622in}{3.350297in}}{\pgfqpoint{1.741446in}{3.356121in}}%
\pgfpathcurveto{\pgfqpoint{1.747270in}{3.361944in}}{\pgfqpoint{1.750542in}{3.369845in}}{\pgfqpoint{1.750542in}{3.378081in}}%
\pgfpathcurveto{\pgfqpoint{1.750542in}{3.386317in}}{\pgfqpoint{1.747270in}{3.394217in}}{\pgfqpoint{1.741446in}{3.400041in}}%
\pgfpathcurveto{\pgfqpoint{1.735622in}{3.405865in}}{\pgfqpoint{1.727722in}{3.409137in}}{\pgfqpoint{1.719486in}{3.409137in}}%
\pgfpathcurveto{\pgfqpoint{1.711249in}{3.409137in}}{\pgfqpoint{1.703349in}{3.405865in}}{\pgfqpoint{1.697525in}{3.400041in}}%
\pgfpathcurveto{\pgfqpoint{1.691701in}{3.394217in}}{\pgfqpoint{1.688429in}{3.386317in}}{\pgfqpoint{1.688429in}{3.378081in}}%
\pgfpathcurveto{\pgfqpoint{1.688429in}{3.369845in}}{\pgfqpoint{1.691701in}{3.361944in}}{\pgfqpoint{1.697525in}{3.356121in}}%
\pgfpathcurveto{\pgfqpoint{1.703349in}{3.350297in}}{\pgfqpoint{1.711249in}{3.347024in}}{\pgfqpoint{1.719486in}{3.347024in}}%
\pgfpathclose%
\pgfusepath{stroke,fill}%
\end{pgfscope}%
\begin{pgfscope}%
\pgfpathrectangle{\pgfqpoint{0.100000in}{0.212622in}}{\pgfqpoint{3.696000in}{3.696000in}}%
\pgfusepath{clip}%
\pgfsetbuttcap%
\pgfsetroundjoin%
\definecolor{currentfill}{rgb}{0.121569,0.466667,0.705882}%
\pgfsetfillcolor{currentfill}%
\pgfsetfillopacity{0.302202}%
\pgfsetlinewidth{1.003750pt}%
\definecolor{currentstroke}{rgb}{0.121569,0.466667,0.705882}%
\pgfsetstrokecolor{currentstroke}%
\pgfsetstrokeopacity{0.302202}%
\pgfsetdash{}{0pt}%
\pgfpathmoveto{\pgfqpoint{1.655198in}{3.376944in}}%
\pgfpathcurveto{\pgfqpoint{1.663434in}{3.376944in}}{\pgfqpoint{1.671334in}{3.380216in}}{\pgfqpoint{1.677158in}{3.386040in}}%
\pgfpathcurveto{\pgfqpoint{1.682982in}{3.391864in}}{\pgfqpoint{1.686254in}{3.399764in}}{\pgfqpoint{1.686254in}{3.408000in}}%
\pgfpathcurveto{\pgfqpoint{1.686254in}{3.416236in}}{\pgfqpoint{1.682982in}{3.424137in}}{\pgfqpoint{1.677158in}{3.429960in}}%
\pgfpathcurveto{\pgfqpoint{1.671334in}{3.435784in}}{\pgfqpoint{1.663434in}{3.439057in}}{\pgfqpoint{1.655198in}{3.439057in}}%
\pgfpathcurveto{\pgfqpoint{1.646962in}{3.439057in}}{\pgfqpoint{1.639062in}{3.435784in}}{\pgfqpoint{1.633238in}{3.429960in}}%
\pgfpathcurveto{\pgfqpoint{1.627414in}{3.424137in}}{\pgfqpoint{1.624141in}{3.416236in}}{\pgfqpoint{1.624141in}{3.408000in}}%
\pgfpathcurveto{\pgfqpoint{1.624141in}{3.399764in}}{\pgfqpoint{1.627414in}{3.391864in}}{\pgfqpoint{1.633238in}{3.386040in}}%
\pgfpathcurveto{\pgfqpoint{1.639062in}{3.380216in}}{\pgfqpoint{1.646962in}{3.376944in}}{\pgfqpoint{1.655198in}{3.376944in}}%
\pgfpathclose%
\pgfusepath{stroke,fill}%
\end{pgfscope}%
\begin{pgfscope}%
\pgfpathrectangle{\pgfqpoint{0.100000in}{0.212622in}}{\pgfqpoint{3.696000in}{3.696000in}}%
\pgfusepath{clip}%
\pgfsetbuttcap%
\pgfsetroundjoin%
\definecolor{currentfill}{rgb}{0.121569,0.466667,0.705882}%
\pgfsetfillcolor{currentfill}%
\pgfsetfillopacity{0.302309}%
\pgfsetlinewidth{1.003750pt}%
\definecolor{currentstroke}{rgb}{0.121569,0.466667,0.705882}%
\pgfsetstrokecolor{currentstroke}%
\pgfsetstrokeopacity{0.302309}%
\pgfsetdash{}{0pt}%
\pgfpathmoveto{\pgfqpoint{1.723393in}{3.343938in}}%
\pgfpathcurveto{\pgfqpoint{1.731630in}{3.343938in}}{\pgfqpoint{1.739530in}{3.347210in}}{\pgfqpoint{1.745353in}{3.353034in}}%
\pgfpathcurveto{\pgfqpoint{1.751177in}{3.358858in}}{\pgfqpoint{1.754450in}{3.366758in}}{\pgfqpoint{1.754450in}{3.374995in}}%
\pgfpathcurveto{\pgfqpoint{1.754450in}{3.383231in}}{\pgfqpoint{1.751177in}{3.391131in}}{\pgfqpoint{1.745353in}{3.396955in}}%
\pgfpathcurveto{\pgfqpoint{1.739530in}{3.402779in}}{\pgfqpoint{1.731630in}{3.406051in}}{\pgfqpoint{1.723393in}{3.406051in}}%
\pgfpathcurveto{\pgfqpoint{1.715157in}{3.406051in}}{\pgfqpoint{1.707257in}{3.402779in}}{\pgfqpoint{1.701433in}{3.396955in}}%
\pgfpathcurveto{\pgfqpoint{1.695609in}{3.391131in}}{\pgfqpoint{1.692337in}{3.383231in}}{\pgfqpoint{1.692337in}{3.374995in}}%
\pgfpathcurveto{\pgfqpoint{1.692337in}{3.366758in}}{\pgfqpoint{1.695609in}{3.358858in}}{\pgfqpoint{1.701433in}{3.353034in}}%
\pgfpathcurveto{\pgfqpoint{1.707257in}{3.347210in}}{\pgfqpoint{1.715157in}{3.343938in}}{\pgfqpoint{1.723393in}{3.343938in}}%
\pgfpathclose%
\pgfusepath{stroke,fill}%
\end{pgfscope}%
\begin{pgfscope}%
\pgfpathrectangle{\pgfqpoint{0.100000in}{0.212622in}}{\pgfqpoint{3.696000in}{3.696000in}}%
\pgfusepath{clip}%
\pgfsetbuttcap%
\pgfsetroundjoin%
\definecolor{currentfill}{rgb}{0.121569,0.466667,0.705882}%
\pgfsetfillcolor{currentfill}%
\pgfsetfillopacity{0.302648}%
\pgfsetlinewidth{1.003750pt}%
\definecolor{currentstroke}{rgb}{0.121569,0.466667,0.705882}%
\pgfsetstrokecolor{currentstroke}%
\pgfsetstrokeopacity{0.302648}%
\pgfsetdash{}{0pt}%
\pgfpathmoveto{\pgfqpoint{1.725495in}{3.342160in}}%
\pgfpathcurveto{\pgfqpoint{1.733732in}{3.342160in}}{\pgfqpoint{1.741632in}{3.345432in}}{\pgfqpoint{1.747456in}{3.351256in}}%
\pgfpathcurveto{\pgfqpoint{1.753280in}{3.357080in}}{\pgfqpoint{1.756552in}{3.364980in}}{\pgfqpoint{1.756552in}{3.373216in}}%
\pgfpathcurveto{\pgfqpoint{1.756552in}{3.381452in}}{\pgfqpoint{1.753280in}{3.389352in}}{\pgfqpoint{1.747456in}{3.395176in}}%
\pgfpathcurveto{\pgfqpoint{1.741632in}{3.401000in}}{\pgfqpoint{1.733732in}{3.404273in}}{\pgfqpoint{1.725495in}{3.404273in}}%
\pgfpathcurveto{\pgfqpoint{1.717259in}{3.404273in}}{\pgfqpoint{1.709359in}{3.401000in}}{\pgfqpoint{1.703535in}{3.395176in}}%
\pgfpathcurveto{\pgfqpoint{1.697711in}{3.389352in}}{\pgfqpoint{1.694439in}{3.381452in}}{\pgfqpoint{1.694439in}{3.373216in}}%
\pgfpathcurveto{\pgfqpoint{1.694439in}{3.364980in}}{\pgfqpoint{1.697711in}{3.357080in}}{\pgfqpoint{1.703535in}{3.351256in}}%
\pgfpathcurveto{\pgfqpoint{1.709359in}{3.345432in}}{\pgfqpoint{1.717259in}{3.342160in}}{\pgfqpoint{1.725495in}{3.342160in}}%
\pgfpathclose%
\pgfusepath{stroke,fill}%
\end{pgfscope}%
\begin{pgfscope}%
\pgfpathrectangle{\pgfqpoint{0.100000in}{0.212622in}}{\pgfqpoint{3.696000in}{3.696000in}}%
\pgfusepath{clip}%
\pgfsetbuttcap%
\pgfsetroundjoin%
\definecolor{currentfill}{rgb}{0.121569,0.466667,0.705882}%
\pgfsetfillcolor{currentfill}%
\pgfsetfillopacity{0.303566}%
\pgfsetlinewidth{1.003750pt}%
\definecolor{currentstroke}{rgb}{0.121569,0.466667,0.705882}%
\pgfsetstrokecolor{currentstroke}%
\pgfsetstrokeopacity{0.303566}%
\pgfsetdash{}{0pt}%
\pgfpathmoveto{\pgfqpoint{1.644064in}{3.379820in}}%
\pgfpathcurveto{\pgfqpoint{1.652300in}{3.379820in}}{\pgfqpoint{1.660200in}{3.383092in}}{\pgfqpoint{1.666024in}{3.388916in}}%
\pgfpathcurveto{\pgfqpoint{1.671848in}{3.394740in}}{\pgfqpoint{1.675121in}{3.402640in}}{\pgfqpoint{1.675121in}{3.410877in}}%
\pgfpathcurveto{\pgfqpoint{1.675121in}{3.419113in}}{\pgfqpoint{1.671848in}{3.427013in}}{\pgfqpoint{1.666024in}{3.432837in}}%
\pgfpathcurveto{\pgfqpoint{1.660200in}{3.438661in}}{\pgfqpoint{1.652300in}{3.441933in}}{\pgfqpoint{1.644064in}{3.441933in}}%
\pgfpathcurveto{\pgfqpoint{1.635828in}{3.441933in}}{\pgfqpoint{1.627928in}{3.438661in}}{\pgfqpoint{1.622104in}{3.432837in}}%
\pgfpathcurveto{\pgfqpoint{1.616280in}{3.427013in}}{\pgfqpoint{1.613008in}{3.419113in}}{\pgfqpoint{1.613008in}{3.410877in}}%
\pgfpathcurveto{\pgfqpoint{1.613008in}{3.402640in}}{\pgfqpoint{1.616280in}{3.394740in}}{\pgfqpoint{1.622104in}{3.388916in}}%
\pgfpathcurveto{\pgfqpoint{1.627928in}{3.383092in}}{\pgfqpoint{1.635828in}{3.379820in}}{\pgfqpoint{1.644064in}{3.379820in}}%
\pgfpathclose%
\pgfusepath{stroke,fill}%
\end{pgfscope}%
\begin{pgfscope}%
\pgfpathrectangle{\pgfqpoint{0.100000in}{0.212622in}}{\pgfqpoint{3.696000in}{3.696000in}}%
\pgfusepath{clip}%
\pgfsetbuttcap%
\pgfsetroundjoin%
\definecolor{currentfill}{rgb}{0.121569,0.466667,0.705882}%
\pgfsetfillcolor{currentfill}%
\pgfsetfillopacity{0.303899}%
\pgfsetlinewidth{1.003750pt}%
\definecolor{currentstroke}{rgb}{0.121569,0.466667,0.705882}%
\pgfsetstrokecolor{currentstroke}%
\pgfsetstrokeopacity{0.303899}%
\pgfsetdash{}{0pt}%
\pgfpathmoveto{\pgfqpoint{1.731701in}{3.336633in}}%
\pgfpathcurveto{\pgfqpoint{1.739937in}{3.336633in}}{\pgfqpoint{1.747837in}{3.339905in}}{\pgfqpoint{1.753661in}{3.345729in}}%
\pgfpathcurveto{\pgfqpoint{1.759485in}{3.351553in}}{\pgfqpoint{1.762758in}{3.359453in}}{\pgfqpoint{1.762758in}{3.367690in}}%
\pgfpathcurveto{\pgfqpoint{1.762758in}{3.375926in}}{\pgfqpoint{1.759485in}{3.383826in}}{\pgfqpoint{1.753661in}{3.389650in}}%
\pgfpathcurveto{\pgfqpoint{1.747837in}{3.395474in}}{\pgfqpoint{1.739937in}{3.398746in}}{\pgfqpoint{1.731701in}{3.398746in}}%
\pgfpathcurveto{\pgfqpoint{1.723465in}{3.398746in}}{\pgfqpoint{1.715565in}{3.395474in}}{\pgfqpoint{1.709741in}{3.389650in}}%
\pgfpathcurveto{\pgfqpoint{1.703917in}{3.383826in}}{\pgfqpoint{1.700645in}{3.375926in}}{\pgfqpoint{1.700645in}{3.367690in}}%
\pgfpathcurveto{\pgfqpoint{1.700645in}{3.359453in}}{\pgfqpoint{1.703917in}{3.351553in}}{\pgfqpoint{1.709741in}{3.345729in}}%
\pgfpathcurveto{\pgfqpoint{1.715565in}{3.339905in}}{\pgfqpoint{1.723465in}{3.336633in}}{\pgfqpoint{1.731701in}{3.336633in}}%
\pgfpathclose%
\pgfusepath{stroke,fill}%
\end{pgfscope}%
\begin{pgfscope}%
\pgfpathrectangle{\pgfqpoint{0.100000in}{0.212622in}}{\pgfqpoint{3.696000in}{3.696000in}}%
\pgfusepath{clip}%
\pgfsetbuttcap%
\pgfsetroundjoin%
\definecolor{currentfill}{rgb}{0.121569,0.466667,0.705882}%
\pgfsetfillcolor{currentfill}%
\pgfsetfillopacity{0.304525}%
\pgfsetlinewidth{1.003750pt}%
\definecolor{currentstroke}{rgb}{0.121569,0.466667,0.705882}%
\pgfsetstrokecolor{currentstroke}%
\pgfsetstrokeopacity{0.304525}%
\pgfsetdash{}{0pt}%
\pgfpathmoveto{\pgfqpoint{1.637348in}{3.381198in}}%
\pgfpathcurveto{\pgfqpoint{1.645584in}{3.381198in}}{\pgfqpoint{1.653484in}{3.384470in}}{\pgfqpoint{1.659308in}{3.390294in}}%
\pgfpathcurveto{\pgfqpoint{1.665132in}{3.396118in}}{\pgfqpoint{1.668404in}{3.404018in}}{\pgfqpoint{1.668404in}{3.412254in}}%
\pgfpathcurveto{\pgfqpoint{1.668404in}{3.420490in}}{\pgfqpoint{1.665132in}{3.428391in}}{\pgfqpoint{1.659308in}{3.434214in}}%
\pgfpathcurveto{\pgfqpoint{1.653484in}{3.440038in}}{\pgfqpoint{1.645584in}{3.443311in}}{\pgfqpoint{1.637348in}{3.443311in}}%
\pgfpathcurveto{\pgfqpoint{1.629112in}{3.443311in}}{\pgfqpoint{1.621212in}{3.440038in}}{\pgfqpoint{1.615388in}{3.434214in}}%
\pgfpathcurveto{\pgfqpoint{1.609564in}{3.428391in}}{\pgfqpoint{1.606291in}{3.420490in}}{\pgfqpoint{1.606291in}{3.412254in}}%
\pgfpathcurveto{\pgfqpoint{1.606291in}{3.404018in}}{\pgfqpoint{1.609564in}{3.396118in}}{\pgfqpoint{1.615388in}{3.390294in}}%
\pgfpathcurveto{\pgfqpoint{1.621212in}{3.384470in}}{\pgfqpoint{1.629112in}{3.381198in}}{\pgfqpoint{1.637348in}{3.381198in}}%
\pgfpathclose%
\pgfusepath{stroke,fill}%
\end{pgfscope}%
\begin{pgfscope}%
\pgfpathrectangle{\pgfqpoint{0.100000in}{0.212622in}}{\pgfqpoint{3.696000in}{3.696000in}}%
\pgfusepath{clip}%
\pgfsetbuttcap%
\pgfsetroundjoin%
\definecolor{currentfill}{rgb}{0.121569,0.466667,0.705882}%
\pgfsetfillcolor{currentfill}%
\pgfsetfillopacity{0.305242}%
\pgfsetlinewidth{1.003750pt}%
\definecolor{currentstroke}{rgb}{0.121569,0.466667,0.705882}%
\pgfsetstrokecolor{currentstroke}%
\pgfsetstrokeopacity{0.305242}%
\pgfsetdash{}{0pt}%
\pgfpathmoveto{\pgfqpoint{1.633194in}{3.382061in}}%
\pgfpathcurveto{\pgfqpoint{1.641430in}{3.382061in}}{\pgfqpoint{1.649330in}{3.385334in}}{\pgfqpoint{1.655154in}{3.391158in}}%
\pgfpathcurveto{\pgfqpoint{1.660978in}{3.396982in}}{\pgfqpoint{1.664251in}{3.404882in}}{\pgfqpoint{1.664251in}{3.413118in}}%
\pgfpathcurveto{\pgfqpoint{1.664251in}{3.421354in}}{\pgfqpoint{1.660978in}{3.429254in}}{\pgfqpoint{1.655154in}{3.435078in}}%
\pgfpathcurveto{\pgfqpoint{1.649330in}{3.440902in}}{\pgfqpoint{1.641430in}{3.444174in}}{\pgfqpoint{1.633194in}{3.444174in}}%
\pgfpathcurveto{\pgfqpoint{1.624958in}{3.444174in}}{\pgfqpoint{1.617058in}{3.440902in}}{\pgfqpoint{1.611234in}{3.435078in}}%
\pgfpathcurveto{\pgfqpoint{1.605410in}{3.429254in}}{\pgfqpoint{1.602138in}{3.421354in}}{\pgfqpoint{1.602138in}{3.413118in}}%
\pgfpathcurveto{\pgfqpoint{1.602138in}{3.404882in}}{\pgfqpoint{1.605410in}{3.396982in}}{\pgfqpoint{1.611234in}{3.391158in}}%
\pgfpathcurveto{\pgfqpoint{1.617058in}{3.385334in}}{\pgfqpoint{1.624958in}{3.382061in}}{\pgfqpoint{1.633194in}{3.382061in}}%
\pgfpathclose%
\pgfusepath{stroke,fill}%
\end{pgfscope}%
\begin{pgfscope}%
\pgfpathrectangle{\pgfqpoint{0.100000in}{0.212622in}}{\pgfqpoint{3.696000in}{3.696000in}}%
\pgfusepath{clip}%
\pgfsetbuttcap%
\pgfsetroundjoin%
\definecolor{currentfill}{rgb}{0.121569,0.466667,0.705882}%
\pgfsetfillcolor{currentfill}%
\pgfsetfillopacity{0.305522}%
\pgfsetlinewidth{1.003750pt}%
\definecolor{currentstroke}{rgb}{0.121569,0.466667,0.705882}%
\pgfsetstrokecolor{currentstroke}%
\pgfsetstrokeopacity{0.305522}%
\pgfsetdash{}{0pt}%
\pgfpathmoveto{\pgfqpoint{1.631731in}{3.382263in}}%
\pgfpathcurveto{\pgfqpoint{1.639967in}{3.382263in}}{\pgfqpoint{1.647867in}{3.385535in}}{\pgfqpoint{1.653691in}{3.391359in}}%
\pgfpathcurveto{\pgfqpoint{1.659515in}{3.397183in}}{\pgfqpoint{1.662787in}{3.405083in}}{\pgfqpoint{1.662787in}{3.413320in}}%
\pgfpathcurveto{\pgfqpoint{1.662787in}{3.421556in}}{\pgfqpoint{1.659515in}{3.429456in}}{\pgfqpoint{1.653691in}{3.435280in}}%
\pgfpathcurveto{\pgfqpoint{1.647867in}{3.441104in}}{\pgfqpoint{1.639967in}{3.444376in}}{\pgfqpoint{1.631731in}{3.444376in}}%
\pgfpathcurveto{\pgfqpoint{1.623494in}{3.444376in}}{\pgfqpoint{1.615594in}{3.441104in}}{\pgfqpoint{1.609770in}{3.435280in}}%
\pgfpathcurveto{\pgfqpoint{1.603947in}{3.429456in}}{\pgfqpoint{1.600674in}{3.421556in}}{\pgfqpoint{1.600674in}{3.413320in}}%
\pgfpathcurveto{\pgfqpoint{1.600674in}{3.405083in}}{\pgfqpoint{1.603947in}{3.397183in}}{\pgfqpoint{1.609770in}{3.391359in}}%
\pgfpathcurveto{\pgfqpoint{1.615594in}{3.385535in}}{\pgfqpoint{1.623494in}{3.382263in}}{\pgfqpoint{1.631731in}{3.382263in}}%
\pgfpathclose%
\pgfusepath{stroke,fill}%
\end{pgfscope}%
\begin{pgfscope}%
\pgfpathrectangle{\pgfqpoint{0.100000in}{0.212622in}}{\pgfqpoint{3.696000in}{3.696000in}}%
\pgfusepath{clip}%
\pgfsetbuttcap%
\pgfsetroundjoin%
\definecolor{currentfill}{rgb}{0.121569,0.466667,0.705882}%
\pgfsetfillcolor{currentfill}%
\pgfsetfillopacity{0.305723}%
\pgfsetlinewidth{1.003750pt}%
\definecolor{currentstroke}{rgb}{0.121569,0.466667,0.705882}%
\pgfsetstrokecolor{currentstroke}%
\pgfsetstrokeopacity{0.305723}%
\pgfsetdash{}{0pt}%
\pgfpathmoveto{\pgfqpoint{1.740203in}{3.328360in}}%
\pgfpathcurveto{\pgfqpoint{1.748439in}{3.328360in}}{\pgfqpoint{1.756339in}{3.331632in}}{\pgfqpoint{1.762163in}{3.337456in}}%
\pgfpathcurveto{\pgfqpoint{1.767987in}{3.343280in}}{\pgfqpoint{1.771259in}{3.351180in}}{\pgfqpoint{1.771259in}{3.359416in}}%
\pgfpathcurveto{\pgfqpoint{1.771259in}{3.367653in}}{\pgfqpoint{1.767987in}{3.375553in}}{\pgfqpoint{1.762163in}{3.381377in}}%
\pgfpathcurveto{\pgfqpoint{1.756339in}{3.387201in}}{\pgfqpoint{1.748439in}{3.390473in}}{\pgfqpoint{1.740203in}{3.390473in}}%
\pgfpathcurveto{\pgfqpoint{1.731967in}{3.390473in}}{\pgfqpoint{1.724067in}{3.387201in}}{\pgfqpoint{1.718243in}{3.381377in}}%
\pgfpathcurveto{\pgfqpoint{1.712419in}{3.375553in}}{\pgfqpoint{1.709146in}{3.367653in}}{\pgfqpoint{1.709146in}{3.359416in}}%
\pgfpathcurveto{\pgfqpoint{1.709146in}{3.351180in}}{\pgfqpoint{1.712419in}{3.343280in}}{\pgfqpoint{1.718243in}{3.337456in}}%
\pgfpathcurveto{\pgfqpoint{1.724067in}{3.331632in}}{\pgfqpoint{1.731967in}{3.328360in}}{\pgfqpoint{1.740203in}{3.328360in}}%
\pgfpathclose%
\pgfusepath{stroke,fill}%
\end{pgfscope}%
\begin{pgfscope}%
\pgfpathrectangle{\pgfqpoint{0.100000in}{0.212622in}}{\pgfqpoint{3.696000in}{3.696000in}}%
\pgfusepath{clip}%
\pgfsetbuttcap%
\pgfsetroundjoin%
\definecolor{currentfill}{rgb}{0.121569,0.466667,0.705882}%
\pgfsetfillcolor{currentfill}%
\pgfsetfillopacity{0.306094}%
\pgfsetlinewidth{1.003750pt}%
\definecolor{currentstroke}{rgb}{0.121569,0.466667,0.705882}%
\pgfsetstrokecolor{currentstroke}%
\pgfsetstrokeopacity{0.306094}%
\pgfsetdash{}{0pt}%
\pgfpathmoveto{\pgfqpoint{1.629142in}{3.382548in}}%
\pgfpathcurveto{\pgfqpoint{1.637378in}{3.382548in}}{\pgfqpoint{1.645278in}{3.385821in}}{\pgfqpoint{1.651102in}{3.391644in}}%
\pgfpathcurveto{\pgfqpoint{1.656926in}{3.397468in}}{\pgfqpoint{1.660198in}{3.405368in}}{\pgfqpoint{1.660198in}{3.413605in}}%
\pgfpathcurveto{\pgfqpoint{1.660198in}{3.421841in}}{\pgfqpoint{1.656926in}{3.429741in}}{\pgfqpoint{1.651102in}{3.435565in}}%
\pgfpathcurveto{\pgfqpoint{1.645278in}{3.441389in}}{\pgfqpoint{1.637378in}{3.444661in}}{\pgfqpoint{1.629142in}{3.444661in}}%
\pgfpathcurveto{\pgfqpoint{1.620905in}{3.444661in}}{\pgfqpoint{1.613005in}{3.441389in}}{\pgfqpoint{1.607181in}{3.435565in}}%
\pgfpathcurveto{\pgfqpoint{1.601357in}{3.429741in}}{\pgfqpoint{1.598085in}{3.421841in}}{\pgfqpoint{1.598085in}{3.413605in}}%
\pgfpathcurveto{\pgfqpoint{1.598085in}{3.405368in}}{\pgfqpoint{1.601357in}{3.397468in}}{\pgfqpoint{1.607181in}{3.391644in}}%
\pgfpathcurveto{\pgfqpoint{1.613005in}{3.385821in}}{\pgfqpoint{1.620905in}{3.382548in}}{\pgfqpoint{1.629142in}{3.382548in}}%
\pgfpathclose%
\pgfusepath{stroke,fill}%
\end{pgfscope}%
\begin{pgfscope}%
\pgfpathrectangle{\pgfqpoint{0.100000in}{0.212622in}}{\pgfqpoint{3.696000in}{3.696000in}}%
\pgfusepath{clip}%
\pgfsetbuttcap%
\pgfsetroundjoin%
\definecolor{currentfill}{rgb}{0.121569,0.466667,0.705882}%
\pgfsetfillcolor{currentfill}%
\pgfsetfillopacity{0.307183}%
\pgfsetlinewidth{1.003750pt}%
\definecolor{currentstroke}{rgb}{0.121569,0.466667,0.705882}%
\pgfsetstrokecolor{currentstroke}%
\pgfsetstrokeopacity{0.307183}%
\pgfsetdash{}{0pt}%
\pgfpathmoveto{\pgfqpoint{1.624573in}{3.382732in}}%
\pgfpathcurveto{\pgfqpoint{1.632810in}{3.382732in}}{\pgfqpoint{1.640710in}{3.386005in}}{\pgfqpoint{1.646534in}{3.391829in}}%
\pgfpathcurveto{\pgfqpoint{1.652357in}{3.397653in}}{\pgfqpoint{1.655630in}{3.405553in}}{\pgfqpoint{1.655630in}{3.413789in}}%
\pgfpathcurveto{\pgfqpoint{1.655630in}{3.422025in}}{\pgfqpoint{1.652357in}{3.429925in}}{\pgfqpoint{1.646534in}{3.435749in}}%
\pgfpathcurveto{\pgfqpoint{1.640710in}{3.441573in}}{\pgfqpoint{1.632810in}{3.444845in}}{\pgfqpoint{1.624573in}{3.444845in}}%
\pgfpathcurveto{\pgfqpoint{1.616337in}{3.444845in}}{\pgfqpoint{1.608437in}{3.441573in}}{\pgfqpoint{1.602613in}{3.435749in}}%
\pgfpathcurveto{\pgfqpoint{1.596789in}{3.429925in}}{\pgfqpoint{1.593517in}{3.422025in}}{\pgfqpoint{1.593517in}{3.413789in}}%
\pgfpathcurveto{\pgfqpoint{1.593517in}{3.405553in}}{\pgfqpoint{1.596789in}{3.397653in}}{\pgfqpoint{1.602613in}{3.391829in}}%
\pgfpathcurveto{\pgfqpoint{1.608437in}{3.386005in}}{\pgfqpoint{1.616337in}{3.382732in}}{\pgfqpoint{1.624573in}{3.382732in}}%
\pgfpathclose%
\pgfusepath{stroke,fill}%
\end{pgfscope}%
\begin{pgfscope}%
\pgfpathrectangle{\pgfqpoint{0.100000in}{0.212622in}}{\pgfqpoint{3.696000in}{3.696000in}}%
\pgfusepath{clip}%
\pgfsetbuttcap%
\pgfsetroundjoin%
\definecolor{currentfill}{rgb}{0.121569,0.466667,0.705882}%
\pgfsetfillcolor{currentfill}%
\pgfsetfillopacity{0.308461}%
\pgfsetlinewidth{1.003750pt}%
\definecolor{currentstroke}{rgb}{0.121569,0.466667,0.705882}%
\pgfsetstrokecolor{currentstroke}%
\pgfsetstrokeopacity{0.308461}%
\pgfsetdash{}{0pt}%
\pgfpathmoveto{\pgfqpoint{1.751080in}{3.317195in}}%
\pgfpathcurveto{\pgfqpoint{1.759317in}{3.317195in}}{\pgfqpoint{1.767217in}{3.320468in}}{\pgfqpoint{1.773041in}{3.326292in}}%
\pgfpathcurveto{\pgfqpoint{1.778864in}{3.332116in}}{\pgfqpoint{1.782137in}{3.340016in}}{\pgfqpoint{1.782137in}{3.348252in}}%
\pgfpathcurveto{\pgfqpoint{1.782137in}{3.356488in}}{\pgfqpoint{1.778864in}{3.364388in}}{\pgfqpoint{1.773041in}{3.370212in}}%
\pgfpathcurveto{\pgfqpoint{1.767217in}{3.376036in}}{\pgfqpoint{1.759317in}{3.379308in}}{\pgfqpoint{1.751080in}{3.379308in}}%
\pgfpathcurveto{\pgfqpoint{1.742844in}{3.379308in}}{\pgfqpoint{1.734944in}{3.376036in}}{\pgfqpoint{1.729120in}{3.370212in}}%
\pgfpathcurveto{\pgfqpoint{1.723296in}{3.364388in}}{\pgfqpoint{1.720024in}{3.356488in}}{\pgfqpoint{1.720024in}{3.348252in}}%
\pgfpathcurveto{\pgfqpoint{1.720024in}{3.340016in}}{\pgfqpoint{1.723296in}{3.332116in}}{\pgfqpoint{1.729120in}{3.326292in}}%
\pgfpathcurveto{\pgfqpoint{1.734944in}{3.320468in}}{\pgfqpoint{1.742844in}{3.317195in}}{\pgfqpoint{1.751080in}{3.317195in}}%
\pgfpathclose%
\pgfusepath{stroke,fill}%
\end{pgfscope}%
\begin{pgfscope}%
\pgfpathrectangle{\pgfqpoint{0.100000in}{0.212622in}}{\pgfqpoint{3.696000in}{3.696000in}}%
\pgfusepath{clip}%
\pgfsetbuttcap%
\pgfsetroundjoin%
\definecolor{currentfill}{rgb}{0.121569,0.466667,0.705882}%
\pgfsetfillcolor{currentfill}%
\pgfsetfillopacity{0.309316}%
\pgfsetlinewidth{1.003750pt}%
\definecolor{currentstroke}{rgb}{0.121569,0.466667,0.705882}%
\pgfsetstrokecolor{currentstroke}%
\pgfsetstrokeopacity{0.309316}%
\pgfsetdash{}{0pt}%
\pgfpathmoveto{\pgfqpoint{1.616545in}{3.382731in}}%
\pgfpathcurveto{\pgfqpoint{1.624782in}{3.382731in}}{\pgfqpoint{1.632682in}{3.386004in}}{\pgfqpoint{1.638506in}{3.391828in}}%
\pgfpathcurveto{\pgfqpoint{1.644330in}{3.397652in}}{\pgfqpoint{1.647602in}{3.405552in}}{\pgfqpoint{1.647602in}{3.413788in}}%
\pgfpathcurveto{\pgfqpoint{1.647602in}{3.422024in}}{\pgfqpoint{1.644330in}{3.429924in}}{\pgfqpoint{1.638506in}{3.435748in}}%
\pgfpathcurveto{\pgfqpoint{1.632682in}{3.441572in}}{\pgfqpoint{1.624782in}{3.444844in}}{\pgfqpoint{1.616545in}{3.444844in}}%
\pgfpathcurveto{\pgfqpoint{1.608309in}{3.444844in}}{\pgfqpoint{1.600409in}{3.441572in}}{\pgfqpoint{1.594585in}{3.435748in}}%
\pgfpathcurveto{\pgfqpoint{1.588761in}{3.429924in}}{\pgfqpoint{1.585489in}{3.422024in}}{\pgfqpoint{1.585489in}{3.413788in}}%
\pgfpathcurveto{\pgfqpoint{1.585489in}{3.405552in}}{\pgfqpoint{1.588761in}{3.397652in}}{\pgfqpoint{1.594585in}{3.391828in}}%
\pgfpathcurveto{\pgfqpoint{1.600409in}{3.386004in}}{\pgfqpoint{1.608309in}{3.382731in}}{\pgfqpoint{1.616545in}{3.382731in}}%
\pgfpathclose%
\pgfusepath{stroke,fill}%
\end{pgfscope}%
\begin{pgfscope}%
\pgfpathrectangle{\pgfqpoint{0.100000in}{0.212622in}}{\pgfqpoint{3.696000in}{3.696000in}}%
\pgfusepath{clip}%
\pgfsetbuttcap%
\pgfsetroundjoin%
\definecolor{currentfill}{rgb}{0.121569,0.466667,0.705882}%
\pgfsetfillcolor{currentfill}%
\pgfsetfillopacity{0.310670}%
\pgfsetlinewidth{1.003750pt}%
\definecolor{currentstroke}{rgb}{0.121569,0.466667,0.705882}%
\pgfsetstrokecolor{currentstroke}%
\pgfsetstrokeopacity{0.310670}%
\pgfsetdash{}{0pt}%
\pgfpathmoveto{\pgfqpoint{1.611796in}{3.382459in}}%
\pgfpathcurveto{\pgfqpoint{1.620032in}{3.382459in}}{\pgfqpoint{1.627932in}{3.385731in}}{\pgfqpoint{1.633756in}{3.391555in}}%
\pgfpathcurveto{\pgfqpoint{1.639580in}{3.397379in}}{\pgfqpoint{1.642852in}{3.405279in}}{\pgfqpoint{1.642852in}{3.413515in}}%
\pgfpathcurveto{\pgfqpoint{1.642852in}{3.421752in}}{\pgfqpoint{1.639580in}{3.429652in}}{\pgfqpoint{1.633756in}{3.435476in}}%
\pgfpathcurveto{\pgfqpoint{1.627932in}{3.441300in}}{\pgfqpoint{1.620032in}{3.444572in}}{\pgfqpoint{1.611796in}{3.444572in}}%
\pgfpathcurveto{\pgfqpoint{1.603559in}{3.444572in}}{\pgfqpoint{1.595659in}{3.441300in}}{\pgfqpoint{1.589835in}{3.435476in}}%
\pgfpathcurveto{\pgfqpoint{1.584011in}{3.429652in}}{\pgfqpoint{1.580739in}{3.421752in}}{\pgfqpoint{1.580739in}{3.413515in}}%
\pgfpathcurveto{\pgfqpoint{1.580739in}{3.405279in}}{\pgfqpoint{1.584011in}{3.397379in}}{\pgfqpoint{1.589835in}{3.391555in}}%
\pgfpathcurveto{\pgfqpoint{1.595659in}{3.385731in}}{\pgfqpoint{1.603559in}{3.382459in}}{\pgfqpoint{1.611796in}{3.382459in}}%
\pgfpathclose%
\pgfusepath{stroke,fill}%
\end{pgfscope}%
\begin{pgfscope}%
\pgfpathrectangle{\pgfqpoint{0.100000in}{0.212622in}}{\pgfqpoint{3.696000in}{3.696000in}}%
\pgfusepath{clip}%
\pgfsetbuttcap%
\pgfsetroundjoin%
\definecolor{currentfill}{rgb}{0.121569,0.466667,0.705882}%
\pgfsetfillcolor{currentfill}%
\pgfsetfillopacity{0.311318}%
\pgfsetlinewidth{1.003750pt}%
\definecolor{currentstroke}{rgb}{0.121569,0.466667,0.705882}%
\pgfsetstrokecolor{currentstroke}%
\pgfsetstrokeopacity{0.311318}%
\pgfsetdash{}{0pt}%
\pgfpathmoveto{\pgfqpoint{1.609779in}{3.382215in}}%
\pgfpathcurveto{\pgfqpoint{1.618015in}{3.382215in}}{\pgfqpoint{1.625915in}{3.385487in}}{\pgfqpoint{1.631739in}{3.391311in}}%
\pgfpathcurveto{\pgfqpoint{1.637563in}{3.397135in}}{\pgfqpoint{1.640836in}{3.405035in}}{\pgfqpoint{1.640836in}{3.413271in}}%
\pgfpathcurveto{\pgfqpoint{1.640836in}{3.421507in}}{\pgfqpoint{1.637563in}{3.429408in}}{\pgfqpoint{1.631739in}{3.435231in}}%
\pgfpathcurveto{\pgfqpoint{1.625915in}{3.441055in}}{\pgfqpoint{1.618015in}{3.444328in}}{\pgfqpoint{1.609779in}{3.444328in}}%
\pgfpathcurveto{\pgfqpoint{1.601543in}{3.444328in}}{\pgfqpoint{1.593643in}{3.441055in}}{\pgfqpoint{1.587819in}{3.435231in}}%
\pgfpathcurveto{\pgfqpoint{1.581995in}{3.429408in}}{\pgfqpoint{1.578723in}{3.421507in}}{\pgfqpoint{1.578723in}{3.413271in}}%
\pgfpathcurveto{\pgfqpoint{1.578723in}{3.405035in}}{\pgfqpoint{1.581995in}{3.397135in}}{\pgfqpoint{1.587819in}{3.391311in}}%
\pgfpathcurveto{\pgfqpoint{1.593643in}{3.385487in}}{\pgfqpoint{1.601543in}{3.382215in}}{\pgfqpoint{1.609779in}{3.382215in}}%
\pgfpathclose%
\pgfusepath{stroke,fill}%
\end{pgfscope}%
\begin{pgfscope}%
\pgfpathrectangle{\pgfqpoint{0.100000in}{0.212622in}}{\pgfqpoint{3.696000in}{3.696000in}}%
\pgfusepath{clip}%
\pgfsetbuttcap%
\pgfsetroundjoin%
\definecolor{currentfill}{rgb}{0.121569,0.466667,0.705882}%
\pgfsetfillcolor{currentfill}%
\pgfsetfillopacity{0.312381}%
\pgfsetlinewidth{1.003750pt}%
\definecolor{currentstroke}{rgb}{0.121569,0.466667,0.705882}%
\pgfsetstrokecolor{currentstroke}%
\pgfsetstrokeopacity{0.312381}%
\pgfsetdash{}{0pt}%
\pgfpathmoveto{\pgfqpoint{1.765553in}{3.301268in}}%
\pgfpathcurveto{\pgfqpoint{1.773789in}{3.301268in}}{\pgfqpoint{1.781689in}{3.304541in}}{\pgfqpoint{1.787513in}{3.310364in}}%
\pgfpathcurveto{\pgfqpoint{1.793337in}{3.316188in}}{\pgfqpoint{1.796609in}{3.324088in}}{\pgfqpoint{1.796609in}{3.332325in}}%
\pgfpathcurveto{\pgfqpoint{1.796609in}{3.340561in}}{\pgfqpoint{1.793337in}{3.348461in}}{\pgfqpoint{1.787513in}{3.354285in}}%
\pgfpathcurveto{\pgfqpoint{1.781689in}{3.360109in}}{\pgfqpoint{1.773789in}{3.363381in}}{\pgfqpoint{1.765553in}{3.363381in}}%
\pgfpathcurveto{\pgfqpoint{1.757316in}{3.363381in}}{\pgfqpoint{1.749416in}{3.360109in}}{\pgfqpoint{1.743592in}{3.354285in}}%
\pgfpathcurveto{\pgfqpoint{1.737768in}{3.348461in}}{\pgfqpoint{1.734496in}{3.340561in}}{\pgfqpoint{1.734496in}{3.332325in}}%
\pgfpathcurveto{\pgfqpoint{1.734496in}{3.324088in}}{\pgfqpoint{1.737768in}{3.316188in}}{\pgfqpoint{1.743592in}{3.310364in}}%
\pgfpathcurveto{\pgfqpoint{1.749416in}{3.304541in}}{\pgfqpoint{1.757316in}{3.301268in}}{\pgfqpoint{1.765553in}{3.301268in}}%
\pgfpathclose%
\pgfusepath{stroke,fill}%
\end{pgfscope}%
\begin{pgfscope}%
\pgfpathrectangle{\pgfqpoint{0.100000in}{0.212622in}}{\pgfqpoint{3.696000in}{3.696000in}}%
\pgfusepath{clip}%
\pgfsetbuttcap%
\pgfsetroundjoin%
\definecolor{currentfill}{rgb}{0.121569,0.466667,0.705882}%
\pgfsetfillcolor{currentfill}%
\pgfsetfillopacity{0.312509}%
\pgfsetlinewidth{1.003750pt}%
\definecolor{currentstroke}{rgb}{0.121569,0.466667,0.705882}%
\pgfsetstrokecolor{currentstroke}%
\pgfsetstrokeopacity{0.312509}%
\pgfsetdash{}{0pt}%
\pgfpathmoveto{\pgfqpoint{1.606236in}{3.381528in}}%
\pgfpathcurveto{\pgfqpoint{1.614473in}{3.381528in}}{\pgfqpoint{1.622373in}{3.384801in}}{\pgfqpoint{1.628197in}{3.390625in}}%
\pgfpathcurveto{\pgfqpoint{1.634021in}{3.396449in}}{\pgfqpoint{1.637293in}{3.404349in}}{\pgfqpoint{1.637293in}{3.412585in}}%
\pgfpathcurveto{\pgfqpoint{1.637293in}{3.420821in}}{\pgfqpoint{1.634021in}{3.428721in}}{\pgfqpoint{1.628197in}{3.434545in}}%
\pgfpathcurveto{\pgfqpoint{1.622373in}{3.440369in}}{\pgfqpoint{1.614473in}{3.443641in}}{\pgfqpoint{1.606236in}{3.443641in}}%
\pgfpathcurveto{\pgfqpoint{1.598000in}{3.443641in}}{\pgfqpoint{1.590100in}{3.440369in}}{\pgfqpoint{1.584276in}{3.434545in}}%
\pgfpathcurveto{\pgfqpoint{1.578452in}{3.428721in}}{\pgfqpoint{1.575180in}{3.420821in}}{\pgfqpoint{1.575180in}{3.412585in}}%
\pgfpathcurveto{\pgfqpoint{1.575180in}{3.404349in}}{\pgfqpoint{1.578452in}{3.396449in}}{\pgfqpoint{1.584276in}{3.390625in}}%
\pgfpathcurveto{\pgfqpoint{1.590100in}{3.384801in}}{\pgfqpoint{1.598000in}{3.381528in}}{\pgfqpoint{1.606236in}{3.381528in}}%
\pgfpathclose%
\pgfusepath{stroke,fill}%
\end{pgfscope}%
\begin{pgfscope}%
\pgfpathrectangle{\pgfqpoint{0.100000in}{0.212622in}}{\pgfqpoint{3.696000in}{3.696000in}}%
\pgfusepath{clip}%
\pgfsetbuttcap%
\pgfsetroundjoin%
\definecolor{currentfill}{rgb}{0.121569,0.466667,0.705882}%
\pgfsetfillcolor{currentfill}%
\pgfsetfillopacity{0.314669}%
\pgfsetlinewidth{1.003750pt}%
\definecolor{currentstroke}{rgb}{0.121569,0.466667,0.705882}%
\pgfsetstrokecolor{currentstroke}%
\pgfsetstrokeopacity{0.314669}%
\pgfsetdash{}{0pt}%
\pgfpathmoveto{\pgfqpoint{1.600030in}{3.379734in}}%
\pgfpathcurveto{\pgfqpoint{1.608267in}{3.379734in}}{\pgfqpoint{1.616167in}{3.383006in}}{\pgfqpoint{1.621991in}{3.388830in}}%
\pgfpathcurveto{\pgfqpoint{1.627815in}{3.394654in}}{\pgfqpoint{1.631087in}{3.402554in}}{\pgfqpoint{1.631087in}{3.410791in}}%
\pgfpathcurveto{\pgfqpoint{1.631087in}{3.419027in}}{\pgfqpoint{1.627815in}{3.426927in}}{\pgfqpoint{1.621991in}{3.432751in}}%
\pgfpathcurveto{\pgfqpoint{1.616167in}{3.438575in}}{\pgfqpoint{1.608267in}{3.441847in}}{\pgfqpoint{1.600030in}{3.441847in}}%
\pgfpathcurveto{\pgfqpoint{1.591794in}{3.441847in}}{\pgfqpoint{1.583894in}{3.438575in}}{\pgfqpoint{1.578070in}{3.432751in}}%
\pgfpathcurveto{\pgfqpoint{1.572246in}{3.426927in}}{\pgfqpoint{1.568974in}{3.419027in}}{\pgfqpoint{1.568974in}{3.410791in}}%
\pgfpathcurveto{\pgfqpoint{1.568974in}{3.402554in}}{\pgfqpoint{1.572246in}{3.394654in}}{\pgfqpoint{1.578070in}{3.388830in}}%
\pgfpathcurveto{\pgfqpoint{1.583894in}{3.383006in}}{\pgfqpoint{1.591794in}{3.379734in}}{\pgfqpoint{1.600030in}{3.379734in}}%
\pgfpathclose%
\pgfusepath{stroke,fill}%
\end{pgfscope}%
\begin{pgfscope}%
\pgfpathrectangle{\pgfqpoint{0.100000in}{0.212622in}}{\pgfqpoint{3.696000in}{3.696000in}}%
\pgfusepath{clip}%
\pgfsetbuttcap%
\pgfsetroundjoin%
\definecolor{currentfill}{rgb}{0.121569,0.466667,0.705882}%
\pgfsetfillcolor{currentfill}%
\pgfsetfillopacity{0.315848}%
\pgfsetlinewidth{1.003750pt}%
\definecolor{currentstroke}{rgb}{0.121569,0.466667,0.705882}%
\pgfsetstrokecolor{currentstroke}%
\pgfsetstrokeopacity{0.315848}%
\pgfsetdash{}{0pt}%
\pgfpathmoveto{\pgfqpoint{1.596799in}{3.378448in}}%
\pgfpathcurveto{\pgfqpoint{1.605035in}{3.378448in}}{\pgfqpoint{1.612935in}{3.381721in}}{\pgfqpoint{1.618759in}{3.387545in}}%
\pgfpathcurveto{\pgfqpoint{1.624583in}{3.393368in}}{\pgfqpoint{1.627856in}{3.401268in}}{\pgfqpoint{1.627856in}{3.409505in}}%
\pgfpathcurveto{\pgfqpoint{1.627856in}{3.417741in}}{\pgfqpoint{1.624583in}{3.425641in}}{\pgfqpoint{1.618759in}{3.431465in}}%
\pgfpathcurveto{\pgfqpoint{1.612935in}{3.437289in}}{\pgfqpoint{1.605035in}{3.440561in}}{\pgfqpoint{1.596799in}{3.440561in}}%
\pgfpathcurveto{\pgfqpoint{1.588563in}{3.440561in}}{\pgfqpoint{1.580663in}{3.437289in}}{\pgfqpoint{1.574839in}{3.431465in}}%
\pgfpathcurveto{\pgfqpoint{1.569015in}{3.425641in}}{\pgfqpoint{1.565743in}{3.417741in}}{\pgfqpoint{1.565743in}{3.409505in}}%
\pgfpathcurveto{\pgfqpoint{1.565743in}{3.401268in}}{\pgfqpoint{1.569015in}{3.393368in}}{\pgfqpoint{1.574839in}{3.387545in}}%
\pgfpathcurveto{\pgfqpoint{1.580663in}{3.381721in}}{\pgfqpoint{1.588563in}{3.378448in}}{\pgfqpoint{1.596799in}{3.378448in}}%
\pgfpathclose%
\pgfusepath{stroke,fill}%
\end{pgfscope}%
\begin{pgfscope}%
\pgfpathrectangle{\pgfqpoint{0.100000in}{0.212622in}}{\pgfqpoint{3.696000in}{3.696000in}}%
\pgfusepath{clip}%
\pgfsetbuttcap%
\pgfsetroundjoin%
\definecolor{currentfill}{rgb}{0.121569,0.466667,0.705882}%
\pgfsetfillcolor{currentfill}%
\pgfsetfillopacity{0.316024}%
\pgfsetlinewidth{1.003750pt}%
\definecolor{currentstroke}{rgb}{0.121569,0.466667,0.705882}%
\pgfsetstrokecolor{currentstroke}%
\pgfsetstrokeopacity{0.316024}%
\pgfsetdash{}{0pt}%
\pgfpathmoveto{\pgfqpoint{1.596372in}{3.378226in}}%
\pgfpathcurveto{\pgfqpoint{1.604608in}{3.378226in}}{\pgfqpoint{1.612508in}{3.381498in}}{\pgfqpoint{1.618332in}{3.387322in}}%
\pgfpathcurveto{\pgfqpoint{1.624156in}{3.393146in}}{\pgfqpoint{1.627428in}{3.401046in}}{\pgfqpoint{1.627428in}{3.409282in}}%
\pgfpathcurveto{\pgfqpoint{1.627428in}{3.417518in}}{\pgfqpoint{1.624156in}{3.425419in}}{\pgfqpoint{1.618332in}{3.431242in}}%
\pgfpathcurveto{\pgfqpoint{1.612508in}{3.437066in}}{\pgfqpoint{1.604608in}{3.440339in}}{\pgfqpoint{1.596372in}{3.440339in}}%
\pgfpathcurveto{\pgfqpoint{1.588135in}{3.440339in}}{\pgfqpoint{1.580235in}{3.437066in}}{\pgfqpoint{1.574411in}{3.431242in}}%
\pgfpathcurveto{\pgfqpoint{1.568587in}{3.425419in}}{\pgfqpoint{1.565315in}{3.417518in}}{\pgfqpoint{1.565315in}{3.409282in}}%
\pgfpathcurveto{\pgfqpoint{1.565315in}{3.401046in}}{\pgfqpoint{1.568587in}{3.393146in}}{\pgfqpoint{1.574411in}{3.387322in}}%
\pgfpathcurveto{\pgfqpoint{1.580235in}{3.381498in}}{\pgfqpoint{1.588135in}{3.378226in}}{\pgfqpoint{1.596372in}{3.378226in}}%
\pgfpathclose%
\pgfusepath{stroke,fill}%
\end{pgfscope}%
\begin{pgfscope}%
\pgfpathrectangle{\pgfqpoint{0.100000in}{0.212622in}}{\pgfqpoint{3.696000in}{3.696000in}}%
\pgfusepath{clip}%
\pgfsetbuttcap%
\pgfsetroundjoin%
\definecolor{currentfill}{rgb}{0.121569,0.466667,0.705882}%
\pgfsetfillcolor{currentfill}%
\pgfsetfillopacity{0.316327}%
\pgfsetlinewidth{1.003750pt}%
\definecolor{currentstroke}{rgb}{0.121569,0.466667,0.705882}%
\pgfsetstrokecolor{currentstroke}%
\pgfsetstrokeopacity{0.316327}%
\pgfsetdash{}{0pt}%
\pgfpathmoveto{\pgfqpoint{1.595620in}{3.377710in}}%
\pgfpathcurveto{\pgfqpoint{1.603856in}{3.377710in}}{\pgfqpoint{1.611757in}{3.380982in}}{\pgfqpoint{1.617580in}{3.386806in}}%
\pgfpathcurveto{\pgfqpoint{1.623404in}{3.392630in}}{\pgfqpoint{1.626677in}{3.400530in}}{\pgfqpoint{1.626677in}{3.408767in}}%
\pgfpathcurveto{\pgfqpoint{1.626677in}{3.417003in}}{\pgfqpoint{1.623404in}{3.424903in}}{\pgfqpoint{1.617580in}{3.430727in}}%
\pgfpathcurveto{\pgfqpoint{1.611757in}{3.436551in}}{\pgfqpoint{1.603856in}{3.439823in}}{\pgfqpoint{1.595620in}{3.439823in}}%
\pgfpathcurveto{\pgfqpoint{1.587384in}{3.439823in}}{\pgfqpoint{1.579484in}{3.436551in}}{\pgfqpoint{1.573660in}{3.430727in}}%
\pgfpathcurveto{\pgfqpoint{1.567836in}{3.424903in}}{\pgfqpoint{1.564564in}{3.417003in}}{\pgfqpoint{1.564564in}{3.408767in}}%
\pgfpathcurveto{\pgfqpoint{1.564564in}{3.400530in}}{\pgfqpoint{1.567836in}{3.392630in}}{\pgfqpoint{1.573660in}{3.386806in}}%
\pgfpathcurveto{\pgfqpoint{1.579484in}{3.380982in}}{\pgfqpoint{1.587384in}{3.377710in}}{\pgfqpoint{1.595620in}{3.377710in}}%
\pgfpathclose%
\pgfusepath{stroke,fill}%
\end{pgfscope}%
\begin{pgfscope}%
\pgfpathrectangle{\pgfqpoint{0.100000in}{0.212622in}}{\pgfqpoint{3.696000in}{3.696000in}}%
\pgfusepath{clip}%
\pgfsetbuttcap%
\pgfsetroundjoin%
\definecolor{currentfill}{rgb}{0.121569,0.466667,0.705882}%
\pgfsetfillcolor{currentfill}%
\pgfsetfillopacity{0.316883}%
\pgfsetlinewidth{1.003750pt}%
\definecolor{currentstroke}{rgb}{0.121569,0.466667,0.705882}%
\pgfsetstrokecolor{currentstroke}%
\pgfsetstrokeopacity{0.316883}%
\pgfsetdash{}{0pt}%
\pgfpathmoveto{\pgfqpoint{1.594316in}{3.376694in}}%
\pgfpathcurveto{\pgfqpoint{1.602553in}{3.376694in}}{\pgfqpoint{1.610453in}{3.379966in}}{\pgfqpoint{1.616277in}{3.385790in}}%
\pgfpathcurveto{\pgfqpoint{1.622101in}{3.391614in}}{\pgfqpoint{1.625373in}{3.399514in}}{\pgfqpoint{1.625373in}{3.407750in}}%
\pgfpathcurveto{\pgfqpoint{1.625373in}{3.415987in}}{\pgfqpoint{1.622101in}{3.423887in}}{\pgfqpoint{1.616277in}{3.429711in}}%
\pgfpathcurveto{\pgfqpoint{1.610453in}{3.435535in}}{\pgfqpoint{1.602553in}{3.438807in}}{\pgfqpoint{1.594316in}{3.438807in}}%
\pgfpathcurveto{\pgfqpoint{1.586080in}{3.438807in}}{\pgfqpoint{1.578180in}{3.435535in}}{\pgfqpoint{1.572356in}{3.429711in}}%
\pgfpathcurveto{\pgfqpoint{1.566532in}{3.423887in}}{\pgfqpoint{1.563260in}{3.415987in}}{\pgfqpoint{1.563260in}{3.407750in}}%
\pgfpathcurveto{\pgfqpoint{1.563260in}{3.399514in}}{\pgfqpoint{1.566532in}{3.391614in}}{\pgfqpoint{1.572356in}{3.385790in}}%
\pgfpathcurveto{\pgfqpoint{1.578180in}{3.379966in}}{\pgfqpoint{1.586080in}{3.376694in}}{\pgfqpoint{1.594316in}{3.376694in}}%
\pgfpathclose%
\pgfusepath{stroke,fill}%
\end{pgfscope}%
\begin{pgfscope}%
\pgfpathrectangle{\pgfqpoint{0.100000in}{0.212622in}}{\pgfqpoint{3.696000in}{3.696000in}}%
\pgfusepath{clip}%
\pgfsetbuttcap%
\pgfsetroundjoin%
\definecolor{currentfill}{rgb}{0.121569,0.466667,0.705882}%
\pgfsetfillcolor{currentfill}%
\pgfsetfillopacity{0.317497}%
\pgfsetlinewidth{1.003750pt}%
\definecolor{currentstroke}{rgb}{0.121569,0.466667,0.705882}%
\pgfsetstrokecolor{currentstroke}%
\pgfsetstrokeopacity{0.317497}%
\pgfsetdash{}{0pt}%
\pgfpathmoveto{\pgfqpoint{1.782602in}{3.281813in}}%
\pgfpathcurveto{\pgfqpoint{1.790838in}{3.281813in}}{\pgfqpoint{1.798738in}{3.285085in}}{\pgfqpoint{1.804562in}{3.290909in}}%
\pgfpathcurveto{\pgfqpoint{1.810386in}{3.296733in}}{\pgfqpoint{1.813658in}{3.304633in}}{\pgfqpoint{1.813658in}{3.312869in}}%
\pgfpathcurveto{\pgfqpoint{1.813658in}{3.321106in}}{\pgfqpoint{1.810386in}{3.329006in}}{\pgfqpoint{1.804562in}{3.334829in}}%
\pgfpathcurveto{\pgfqpoint{1.798738in}{3.340653in}}{\pgfqpoint{1.790838in}{3.343926in}}{\pgfqpoint{1.782602in}{3.343926in}}%
\pgfpathcurveto{\pgfqpoint{1.774366in}{3.343926in}}{\pgfqpoint{1.766465in}{3.340653in}}{\pgfqpoint{1.760642in}{3.334829in}}%
\pgfpathcurveto{\pgfqpoint{1.754818in}{3.329006in}}{\pgfqpoint{1.751545in}{3.321106in}}{\pgfqpoint{1.751545in}{3.312869in}}%
\pgfpathcurveto{\pgfqpoint{1.751545in}{3.304633in}}{\pgfqpoint{1.754818in}{3.296733in}}{\pgfqpoint{1.760642in}{3.290909in}}%
\pgfpathcurveto{\pgfqpoint{1.766465in}{3.285085in}}{\pgfqpoint{1.774366in}{3.281813in}}{\pgfqpoint{1.782602in}{3.281813in}}%
\pgfpathclose%
\pgfusepath{stroke,fill}%
\end{pgfscope}%
\begin{pgfscope}%
\pgfpathrectangle{\pgfqpoint{0.100000in}{0.212622in}}{\pgfqpoint{3.696000in}{3.696000in}}%
\pgfusepath{clip}%
\pgfsetbuttcap%
\pgfsetroundjoin%
\definecolor{currentfill}{rgb}{0.121569,0.466667,0.705882}%
\pgfsetfillcolor{currentfill}%
\pgfsetfillopacity{0.317881}%
\pgfsetlinewidth{1.003750pt}%
\definecolor{currentstroke}{rgb}{0.121569,0.466667,0.705882}%
\pgfsetstrokecolor{currentstroke}%
\pgfsetstrokeopacity{0.317881}%
\pgfsetdash{}{0pt}%
\pgfpathmoveto{\pgfqpoint{1.592116in}{3.374554in}}%
\pgfpathcurveto{\pgfqpoint{1.600353in}{3.374554in}}{\pgfqpoint{1.608253in}{3.377826in}}{\pgfqpoint{1.614077in}{3.383650in}}%
\pgfpathcurveto{\pgfqpoint{1.619901in}{3.389474in}}{\pgfqpoint{1.623173in}{3.397374in}}{\pgfqpoint{1.623173in}{3.405611in}}%
\pgfpathcurveto{\pgfqpoint{1.623173in}{3.413847in}}{\pgfqpoint{1.619901in}{3.421747in}}{\pgfqpoint{1.614077in}{3.427571in}}%
\pgfpathcurveto{\pgfqpoint{1.608253in}{3.433395in}}{\pgfqpoint{1.600353in}{3.436667in}}{\pgfqpoint{1.592116in}{3.436667in}}%
\pgfpathcurveto{\pgfqpoint{1.583880in}{3.436667in}}{\pgfqpoint{1.575980in}{3.433395in}}{\pgfqpoint{1.570156in}{3.427571in}}%
\pgfpathcurveto{\pgfqpoint{1.564332in}{3.421747in}}{\pgfqpoint{1.561060in}{3.413847in}}{\pgfqpoint{1.561060in}{3.405611in}}%
\pgfpathcurveto{\pgfqpoint{1.561060in}{3.397374in}}{\pgfqpoint{1.564332in}{3.389474in}}{\pgfqpoint{1.570156in}{3.383650in}}%
\pgfpathcurveto{\pgfqpoint{1.575980in}{3.377826in}}{\pgfqpoint{1.583880in}{3.374554in}}{\pgfqpoint{1.592116in}{3.374554in}}%
\pgfpathclose%
\pgfusepath{stroke,fill}%
\end{pgfscope}%
\begin{pgfscope}%
\pgfpathrectangle{\pgfqpoint{0.100000in}{0.212622in}}{\pgfqpoint{3.696000in}{3.696000in}}%
\pgfusepath{clip}%
\pgfsetbuttcap%
\pgfsetroundjoin%
\definecolor{currentfill}{rgb}{0.121569,0.466667,0.705882}%
\pgfsetfillcolor{currentfill}%
\pgfsetfillopacity{0.319641}%
\pgfsetlinewidth{1.003750pt}%
\definecolor{currentstroke}{rgb}{0.121569,0.466667,0.705882}%
\pgfsetstrokecolor{currentstroke}%
\pgfsetstrokeopacity{0.319641}%
\pgfsetdash{}{0pt}%
\pgfpathmoveto{\pgfqpoint{1.588293in}{3.370205in}}%
\pgfpathcurveto{\pgfqpoint{1.596530in}{3.370205in}}{\pgfqpoint{1.604430in}{3.373477in}}{\pgfqpoint{1.610253in}{3.379301in}}%
\pgfpathcurveto{\pgfqpoint{1.616077in}{3.385125in}}{\pgfqpoint{1.619350in}{3.393025in}}{\pgfqpoint{1.619350in}{3.401261in}}%
\pgfpathcurveto{\pgfqpoint{1.619350in}{3.409497in}}{\pgfqpoint{1.616077in}{3.417398in}}{\pgfqpoint{1.610253in}{3.423221in}}%
\pgfpathcurveto{\pgfqpoint{1.604430in}{3.429045in}}{\pgfqpoint{1.596530in}{3.432318in}}{\pgfqpoint{1.588293in}{3.432318in}}%
\pgfpathcurveto{\pgfqpoint{1.580057in}{3.432318in}}{\pgfqpoint{1.572157in}{3.429045in}}{\pgfqpoint{1.566333in}{3.423221in}}%
\pgfpathcurveto{\pgfqpoint{1.560509in}{3.417398in}}{\pgfqpoint{1.557237in}{3.409497in}}{\pgfqpoint{1.557237in}{3.401261in}}%
\pgfpathcurveto{\pgfqpoint{1.557237in}{3.393025in}}{\pgfqpoint{1.560509in}{3.385125in}}{\pgfqpoint{1.566333in}{3.379301in}}%
\pgfpathcurveto{\pgfqpoint{1.572157in}{3.373477in}}{\pgfqpoint{1.580057in}{3.370205in}}{\pgfqpoint{1.588293in}{3.370205in}}%
\pgfpathclose%
\pgfusepath{stroke,fill}%
\end{pgfscope}%
\begin{pgfscope}%
\pgfpathrectangle{\pgfqpoint{0.100000in}{0.212622in}}{\pgfqpoint{3.696000in}{3.696000in}}%
\pgfusepath{clip}%
\pgfsetbuttcap%
\pgfsetroundjoin%
\definecolor{currentfill}{rgb}{0.121569,0.466667,0.705882}%
\pgfsetfillcolor{currentfill}%
\pgfsetfillopacity{0.322852}%
\pgfsetlinewidth{1.003750pt}%
\definecolor{currentstroke}{rgb}{0.121569,0.466667,0.705882}%
\pgfsetstrokecolor{currentstroke}%
\pgfsetstrokeopacity{0.322852}%
\pgfsetdash{}{0pt}%
\pgfpathmoveto{\pgfqpoint{1.582344in}{3.361171in}}%
\pgfpathcurveto{\pgfqpoint{1.590580in}{3.361171in}}{\pgfqpoint{1.598480in}{3.364443in}}{\pgfqpoint{1.604304in}{3.370267in}}%
\pgfpathcurveto{\pgfqpoint{1.610128in}{3.376091in}}{\pgfqpoint{1.613400in}{3.383991in}}{\pgfqpoint{1.613400in}{3.392228in}}%
\pgfpathcurveto{\pgfqpoint{1.613400in}{3.400464in}}{\pgfqpoint{1.610128in}{3.408364in}}{\pgfqpoint{1.604304in}{3.414188in}}%
\pgfpathcurveto{\pgfqpoint{1.598480in}{3.420012in}}{\pgfqpoint{1.590580in}{3.423284in}}{\pgfqpoint{1.582344in}{3.423284in}}%
\pgfpathcurveto{\pgfqpoint{1.574107in}{3.423284in}}{\pgfqpoint{1.566207in}{3.420012in}}{\pgfqpoint{1.560383in}{3.414188in}}%
\pgfpathcurveto{\pgfqpoint{1.554560in}{3.408364in}}{\pgfqpoint{1.551287in}{3.400464in}}{\pgfqpoint{1.551287in}{3.392228in}}%
\pgfpathcurveto{\pgfqpoint{1.551287in}{3.383991in}}{\pgfqpoint{1.554560in}{3.376091in}}{\pgfqpoint{1.560383in}{3.370267in}}%
\pgfpathcurveto{\pgfqpoint{1.566207in}{3.364443in}}{\pgfqpoint{1.574107in}{3.361171in}}{\pgfqpoint{1.582344in}{3.361171in}}%
\pgfpathclose%
\pgfusepath{stroke,fill}%
\end{pgfscope}%
\begin{pgfscope}%
\pgfpathrectangle{\pgfqpoint{0.100000in}{0.212622in}}{\pgfqpoint{3.696000in}{3.696000in}}%
\pgfusepath{clip}%
\pgfsetbuttcap%
\pgfsetroundjoin%
\definecolor{currentfill}{rgb}{0.121569,0.466667,0.705882}%
\pgfsetfillcolor{currentfill}%
\pgfsetfillopacity{0.323579}%
\pgfsetlinewidth{1.003750pt}%
\definecolor{currentstroke}{rgb}{0.121569,0.466667,0.705882}%
\pgfsetstrokecolor{currentstroke}%
\pgfsetstrokeopacity{0.323579}%
\pgfsetdash{}{0pt}%
\pgfpathmoveto{\pgfqpoint{1.801691in}{3.259262in}}%
\pgfpathcurveto{\pgfqpoint{1.809927in}{3.259262in}}{\pgfqpoint{1.817827in}{3.262535in}}{\pgfqpoint{1.823651in}{3.268359in}}%
\pgfpathcurveto{\pgfqpoint{1.829475in}{3.274183in}}{\pgfqpoint{1.832748in}{3.282083in}}{\pgfqpoint{1.832748in}{3.290319in}}%
\pgfpathcurveto{\pgfqpoint{1.832748in}{3.298555in}}{\pgfqpoint{1.829475in}{3.306455in}}{\pgfqpoint{1.823651in}{3.312279in}}%
\pgfpathcurveto{\pgfqpoint{1.817827in}{3.318103in}}{\pgfqpoint{1.809927in}{3.321375in}}{\pgfqpoint{1.801691in}{3.321375in}}%
\pgfpathcurveto{\pgfqpoint{1.793455in}{3.321375in}}{\pgfqpoint{1.785555in}{3.318103in}}{\pgfqpoint{1.779731in}{3.312279in}}%
\pgfpathcurveto{\pgfqpoint{1.773907in}{3.306455in}}{\pgfqpoint{1.770635in}{3.298555in}}{\pgfqpoint{1.770635in}{3.290319in}}%
\pgfpathcurveto{\pgfqpoint{1.770635in}{3.282083in}}{\pgfqpoint{1.773907in}{3.274183in}}{\pgfqpoint{1.779731in}{3.268359in}}%
\pgfpathcurveto{\pgfqpoint{1.785555in}{3.262535in}}{\pgfqpoint{1.793455in}{3.259262in}}{\pgfqpoint{1.801691in}{3.259262in}}%
\pgfpathclose%
\pgfusepath{stroke,fill}%
\end{pgfscope}%
\begin{pgfscope}%
\pgfpathrectangle{\pgfqpoint{0.100000in}{0.212622in}}{\pgfqpoint{3.696000in}{3.696000in}}%
\pgfusepath{clip}%
\pgfsetbuttcap%
\pgfsetroundjoin%
\definecolor{currentfill}{rgb}{0.121569,0.466667,0.705882}%
\pgfsetfillcolor{currentfill}%
\pgfsetfillopacity{0.325152}%
\pgfsetlinewidth{1.003750pt}%
\definecolor{currentstroke}{rgb}{0.121569,0.466667,0.705882}%
\pgfsetstrokecolor{currentstroke}%
\pgfsetstrokeopacity{0.325152}%
\pgfsetdash{}{0pt}%
\pgfpathmoveto{\pgfqpoint{1.577862in}{3.353938in}}%
\pgfpathcurveto{\pgfqpoint{1.586098in}{3.353938in}}{\pgfqpoint{1.593999in}{3.357211in}}{\pgfqpoint{1.599822in}{3.363035in}}%
\pgfpathcurveto{\pgfqpoint{1.605646in}{3.368859in}}{\pgfqpoint{1.608919in}{3.376759in}}{\pgfqpoint{1.608919in}{3.384995in}}%
\pgfpathcurveto{\pgfqpoint{1.608919in}{3.393231in}}{\pgfqpoint{1.605646in}{3.401131in}}{\pgfqpoint{1.599822in}{3.406955in}}%
\pgfpathcurveto{\pgfqpoint{1.593999in}{3.412779in}}{\pgfqpoint{1.586098in}{3.416051in}}{\pgfqpoint{1.577862in}{3.416051in}}%
\pgfpathcurveto{\pgfqpoint{1.569626in}{3.416051in}}{\pgfqpoint{1.561726in}{3.412779in}}{\pgfqpoint{1.555902in}{3.406955in}}%
\pgfpathcurveto{\pgfqpoint{1.550078in}{3.401131in}}{\pgfqpoint{1.546806in}{3.393231in}}{\pgfqpoint{1.546806in}{3.384995in}}%
\pgfpathcurveto{\pgfqpoint{1.546806in}{3.376759in}}{\pgfqpoint{1.550078in}{3.368859in}}{\pgfqpoint{1.555902in}{3.363035in}}%
\pgfpathcurveto{\pgfqpoint{1.561726in}{3.357211in}}{\pgfqpoint{1.569626in}{3.353938in}}{\pgfqpoint{1.577862in}{3.353938in}}%
\pgfpathclose%
\pgfusepath{stroke,fill}%
\end{pgfscope}%
\begin{pgfscope}%
\pgfpathrectangle{\pgfqpoint{0.100000in}{0.212622in}}{\pgfqpoint{3.696000in}{3.696000in}}%
\pgfusepath{clip}%
\pgfsetbuttcap%
\pgfsetroundjoin%
\definecolor{currentfill}{rgb}{0.121569,0.466667,0.705882}%
\pgfsetfillcolor{currentfill}%
\pgfsetfillopacity{0.326651}%
\pgfsetlinewidth{1.003750pt}%
\definecolor{currentstroke}{rgb}{0.121569,0.466667,0.705882}%
\pgfsetstrokecolor{currentstroke}%
\pgfsetstrokeopacity{0.326651}%
\pgfsetdash{}{0pt}%
\pgfpathmoveto{\pgfqpoint{1.575704in}{3.349467in}}%
\pgfpathcurveto{\pgfqpoint{1.583940in}{3.349467in}}{\pgfqpoint{1.591840in}{3.352740in}}{\pgfqpoint{1.597664in}{3.358564in}}%
\pgfpathcurveto{\pgfqpoint{1.603488in}{3.364387in}}{\pgfqpoint{1.606760in}{3.372287in}}{\pgfqpoint{1.606760in}{3.380524in}}%
\pgfpathcurveto{\pgfqpoint{1.606760in}{3.388760in}}{\pgfqpoint{1.603488in}{3.396660in}}{\pgfqpoint{1.597664in}{3.402484in}}%
\pgfpathcurveto{\pgfqpoint{1.591840in}{3.408308in}}{\pgfqpoint{1.583940in}{3.411580in}}{\pgfqpoint{1.575704in}{3.411580in}}%
\pgfpathcurveto{\pgfqpoint{1.567468in}{3.411580in}}{\pgfqpoint{1.559568in}{3.408308in}}{\pgfqpoint{1.553744in}{3.402484in}}%
\pgfpathcurveto{\pgfqpoint{1.547920in}{3.396660in}}{\pgfqpoint{1.544647in}{3.388760in}}{\pgfqpoint{1.544647in}{3.380524in}}%
\pgfpathcurveto{\pgfqpoint{1.544647in}{3.372287in}}{\pgfqpoint{1.547920in}{3.364387in}}{\pgfqpoint{1.553744in}{3.358564in}}%
\pgfpathcurveto{\pgfqpoint{1.559568in}{3.352740in}}{\pgfqpoint{1.567468in}{3.349467in}}{\pgfqpoint{1.575704in}{3.349467in}}%
\pgfpathclose%
\pgfusepath{stroke,fill}%
\end{pgfscope}%
\begin{pgfscope}%
\pgfpathrectangle{\pgfqpoint{0.100000in}{0.212622in}}{\pgfqpoint{3.696000in}{3.696000in}}%
\pgfusepath{clip}%
\pgfsetbuttcap%
\pgfsetroundjoin%
\definecolor{currentfill}{rgb}{0.121569,0.466667,0.705882}%
\pgfsetfillcolor{currentfill}%
\pgfsetfillopacity{0.327092}%
\pgfsetlinewidth{1.003750pt}%
\definecolor{currentstroke}{rgb}{0.121569,0.466667,0.705882}%
\pgfsetstrokecolor{currentstroke}%
\pgfsetstrokeopacity{0.327092}%
\pgfsetdash{}{0pt}%
\pgfpathmoveto{\pgfqpoint{1.811674in}{3.246793in}}%
\pgfpathcurveto{\pgfqpoint{1.819910in}{3.246793in}}{\pgfqpoint{1.827810in}{3.250066in}}{\pgfqpoint{1.833634in}{3.255890in}}%
\pgfpathcurveto{\pgfqpoint{1.839458in}{3.261714in}}{\pgfqpoint{1.842730in}{3.269614in}}{\pgfqpoint{1.842730in}{3.277850in}}%
\pgfpathcurveto{\pgfqpoint{1.842730in}{3.286086in}}{\pgfqpoint{1.839458in}{3.293986in}}{\pgfqpoint{1.833634in}{3.299810in}}%
\pgfpathcurveto{\pgfqpoint{1.827810in}{3.305634in}}{\pgfqpoint{1.819910in}{3.308906in}}{\pgfqpoint{1.811674in}{3.308906in}}%
\pgfpathcurveto{\pgfqpoint{1.803437in}{3.308906in}}{\pgfqpoint{1.795537in}{3.305634in}}{\pgfqpoint{1.789713in}{3.299810in}}%
\pgfpathcurveto{\pgfqpoint{1.783889in}{3.293986in}}{\pgfqpoint{1.780617in}{3.286086in}}{\pgfqpoint{1.780617in}{3.277850in}}%
\pgfpathcurveto{\pgfqpoint{1.780617in}{3.269614in}}{\pgfqpoint{1.783889in}{3.261714in}}{\pgfqpoint{1.789713in}{3.255890in}}%
\pgfpathcurveto{\pgfqpoint{1.795537in}{3.250066in}}{\pgfqpoint{1.803437in}{3.246793in}}{\pgfqpoint{1.811674in}{3.246793in}}%
\pgfpathclose%
\pgfusepath{stroke,fill}%
\end{pgfscope}%
\begin{pgfscope}%
\pgfpathrectangle{\pgfqpoint{0.100000in}{0.212622in}}{\pgfqpoint{3.696000in}{3.696000in}}%
\pgfusepath{clip}%
\pgfsetbuttcap%
\pgfsetroundjoin%
\definecolor{currentfill}{rgb}{0.121569,0.466667,0.705882}%
\pgfsetfillcolor{currentfill}%
\pgfsetfillopacity{0.329367}%
\pgfsetlinewidth{1.003750pt}%
\definecolor{currentstroke}{rgb}{0.121569,0.466667,0.705882}%
\pgfsetstrokecolor{currentstroke}%
\pgfsetstrokeopacity{0.329367}%
\pgfsetdash{}{0pt}%
\pgfpathmoveto{\pgfqpoint{1.571747in}{3.341312in}}%
\pgfpathcurveto{\pgfqpoint{1.579983in}{3.341312in}}{\pgfqpoint{1.587883in}{3.344584in}}{\pgfqpoint{1.593707in}{3.350408in}}%
\pgfpathcurveto{\pgfqpoint{1.599531in}{3.356232in}}{\pgfqpoint{1.602803in}{3.364132in}}{\pgfqpoint{1.602803in}{3.372368in}}%
\pgfpathcurveto{\pgfqpoint{1.602803in}{3.380605in}}{\pgfqpoint{1.599531in}{3.388505in}}{\pgfqpoint{1.593707in}{3.394329in}}%
\pgfpathcurveto{\pgfqpoint{1.587883in}{3.400153in}}{\pgfqpoint{1.579983in}{3.403425in}}{\pgfqpoint{1.571747in}{3.403425in}}%
\pgfpathcurveto{\pgfqpoint{1.563510in}{3.403425in}}{\pgfqpoint{1.555610in}{3.400153in}}{\pgfqpoint{1.549786in}{3.394329in}}%
\pgfpathcurveto{\pgfqpoint{1.543962in}{3.388505in}}{\pgfqpoint{1.540690in}{3.380605in}}{\pgfqpoint{1.540690in}{3.372368in}}%
\pgfpathcurveto{\pgfqpoint{1.540690in}{3.364132in}}{\pgfqpoint{1.543962in}{3.356232in}}{\pgfqpoint{1.549786in}{3.350408in}}%
\pgfpathcurveto{\pgfqpoint{1.555610in}{3.344584in}}{\pgfqpoint{1.563510in}{3.341312in}}{\pgfqpoint{1.571747in}{3.341312in}}%
\pgfpathclose%
\pgfusepath{stroke,fill}%
\end{pgfscope}%
\begin{pgfscope}%
\pgfpathrectangle{\pgfqpoint{0.100000in}{0.212622in}}{\pgfqpoint{3.696000in}{3.696000in}}%
\pgfusepath{clip}%
\pgfsetbuttcap%
\pgfsetroundjoin%
\definecolor{currentfill}{rgb}{0.121569,0.466667,0.705882}%
\pgfsetfillcolor{currentfill}%
\pgfsetfillopacity{0.330835}%
\pgfsetlinewidth{1.003750pt}%
\definecolor{currentstroke}{rgb}{0.121569,0.466667,0.705882}%
\pgfsetstrokecolor{currentstroke}%
\pgfsetstrokeopacity{0.330835}%
\pgfsetdash{}{0pt}%
\pgfpathmoveto{\pgfqpoint{1.569858in}{3.337017in}}%
\pgfpathcurveto{\pgfqpoint{1.578094in}{3.337017in}}{\pgfqpoint{1.585994in}{3.340289in}}{\pgfqpoint{1.591818in}{3.346113in}}%
\pgfpathcurveto{\pgfqpoint{1.597642in}{3.351937in}}{\pgfqpoint{1.600914in}{3.359837in}}{\pgfqpoint{1.600914in}{3.368073in}}%
\pgfpathcurveto{\pgfqpoint{1.600914in}{3.376310in}}{\pgfqpoint{1.597642in}{3.384210in}}{\pgfqpoint{1.591818in}{3.390034in}}%
\pgfpathcurveto{\pgfqpoint{1.585994in}{3.395857in}}{\pgfqpoint{1.578094in}{3.399130in}}{\pgfqpoint{1.569858in}{3.399130in}}%
\pgfpathcurveto{\pgfqpoint{1.561621in}{3.399130in}}{\pgfqpoint{1.553721in}{3.395857in}}{\pgfqpoint{1.547897in}{3.390034in}}%
\pgfpathcurveto{\pgfqpoint{1.542073in}{3.384210in}}{\pgfqpoint{1.538801in}{3.376310in}}{\pgfqpoint{1.538801in}{3.368073in}}%
\pgfpathcurveto{\pgfqpoint{1.538801in}{3.359837in}}{\pgfqpoint{1.542073in}{3.351937in}}{\pgfqpoint{1.547897in}{3.346113in}}%
\pgfpathcurveto{\pgfqpoint{1.553721in}{3.340289in}}{\pgfqpoint{1.561621in}{3.337017in}}{\pgfqpoint{1.569858in}{3.337017in}}%
\pgfpathclose%
\pgfusepath{stroke,fill}%
\end{pgfscope}%
\begin{pgfscope}%
\pgfpathrectangle{\pgfqpoint{0.100000in}{0.212622in}}{\pgfqpoint{3.696000in}{3.696000in}}%
\pgfusepath{clip}%
\pgfsetbuttcap%
\pgfsetroundjoin%
\definecolor{currentfill}{rgb}{0.121569,0.466667,0.705882}%
\pgfsetfillcolor{currentfill}%
\pgfsetfillopacity{0.331192}%
\pgfsetlinewidth{1.003750pt}%
\definecolor{currentstroke}{rgb}{0.121569,0.466667,0.705882}%
\pgfsetstrokecolor{currentstroke}%
\pgfsetstrokeopacity{0.331192}%
\pgfsetdash{}{0pt}%
\pgfpathmoveto{\pgfqpoint{1.822790in}{3.232356in}}%
\pgfpathcurveto{\pgfqpoint{1.831027in}{3.232356in}}{\pgfqpoint{1.838927in}{3.235628in}}{\pgfqpoint{1.844751in}{3.241452in}}%
\pgfpathcurveto{\pgfqpoint{1.850575in}{3.247276in}}{\pgfqpoint{1.853847in}{3.255176in}}{\pgfqpoint{1.853847in}{3.263412in}}%
\pgfpathcurveto{\pgfqpoint{1.853847in}{3.271648in}}{\pgfqpoint{1.850575in}{3.279548in}}{\pgfqpoint{1.844751in}{3.285372in}}%
\pgfpathcurveto{\pgfqpoint{1.838927in}{3.291196in}}{\pgfqpoint{1.831027in}{3.294469in}}{\pgfqpoint{1.822790in}{3.294469in}}%
\pgfpathcurveto{\pgfqpoint{1.814554in}{3.294469in}}{\pgfqpoint{1.806654in}{3.291196in}}{\pgfqpoint{1.800830in}{3.285372in}}%
\pgfpathcurveto{\pgfqpoint{1.795006in}{3.279548in}}{\pgfqpoint{1.791734in}{3.271648in}}{\pgfqpoint{1.791734in}{3.263412in}}%
\pgfpathcurveto{\pgfqpoint{1.791734in}{3.255176in}}{\pgfqpoint{1.795006in}{3.247276in}}{\pgfqpoint{1.800830in}{3.241452in}}%
\pgfpathcurveto{\pgfqpoint{1.806654in}{3.235628in}}{\pgfqpoint{1.814554in}{3.232356in}}{\pgfqpoint{1.822790in}{3.232356in}}%
\pgfpathclose%
\pgfusepath{stroke,fill}%
\end{pgfscope}%
\begin{pgfscope}%
\pgfpathrectangle{\pgfqpoint{0.100000in}{0.212622in}}{\pgfqpoint{3.696000in}{3.696000in}}%
\pgfusepath{clip}%
\pgfsetbuttcap%
\pgfsetroundjoin%
\definecolor{currentfill}{rgb}{0.121569,0.466667,0.705882}%
\pgfsetfillcolor{currentfill}%
\pgfsetfillopacity{0.331337}%
\pgfsetlinewidth{1.003750pt}%
\definecolor{currentstroke}{rgb}{0.121569,0.466667,0.705882}%
\pgfsetstrokecolor{currentstroke}%
\pgfsetstrokeopacity{0.331337}%
\pgfsetdash{}{0pt}%
\pgfpathmoveto{\pgfqpoint{1.569218in}{3.335574in}}%
\pgfpathcurveto{\pgfqpoint{1.577455in}{3.335574in}}{\pgfqpoint{1.585355in}{3.338846in}}{\pgfqpoint{1.591179in}{3.344670in}}%
\pgfpathcurveto{\pgfqpoint{1.597003in}{3.350494in}}{\pgfqpoint{1.600275in}{3.358394in}}{\pgfqpoint{1.600275in}{3.366630in}}%
\pgfpathcurveto{\pgfqpoint{1.600275in}{3.374866in}}{\pgfqpoint{1.597003in}{3.382766in}}{\pgfqpoint{1.591179in}{3.388590in}}%
\pgfpathcurveto{\pgfqpoint{1.585355in}{3.394414in}}{\pgfqpoint{1.577455in}{3.397687in}}{\pgfqpoint{1.569218in}{3.397687in}}%
\pgfpathcurveto{\pgfqpoint{1.560982in}{3.397687in}}{\pgfqpoint{1.553082in}{3.394414in}}{\pgfqpoint{1.547258in}{3.388590in}}%
\pgfpathcurveto{\pgfqpoint{1.541434in}{3.382766in}}{\pgfqpoint{1.538162in}{3.374866in}}{\pgfqpoint{1.538162in}{3.366630in}}%
\pgfpathcurveto{\pgfqpoint{1.538162in}{3.358394in}}{\pgfqpoint{1.541434in}{3.350494in}}{\pgfqpoint{1.547258in}{3.344670in}}%
\pgfpathcurveto{\pgfqpoint{1.553082in}{3.338846in}}{\pgfqpoint{1.560982in}{3.335574in}}{\pgfqpoint{1.569218in}{3.335574in}}%
\pgfpathclose%
\pgfusepath{stroke,fill}%
\end{pgfscope}%
\begin{pgfscope}%
\pgfpathrectangle{\pgfqpoint{0.100000in}{0.212622in}}{\pgfqpoint{3.696000in}{3.696000in}}%
\pgfusepath{clip}%
\pgfsetbuttcap%
\pgfsetroundjoin%
\definecolor{currentfill}{rgb}{0.121569,0.466667,0.705882}%
\pgfsetfillcolor{currentfill}%
\pgfsetfillopacity{0.332267}%
\pgfsetlinewidth{1.003750pt}%
\definecolor{currentstroke}{rgb}{0.121569,0.466667,0.705882}%
\pgfsetstrokecolor{currentstroke}%
\pgfsetstrokeopacity{0.332267}%
\pgfsetdash{}{0pt}%
\pgfpathmoveto{\pgfqpoint{1.568091in}{3.332989in}}%
\pgfpathcurveto{\pgfqpoint{1.576327in}{3.332989in}}{\pgfqpoint{1.584227in}{3.336262in}}{\pgfqpoint{1.590051in}{3.342086in}}%
\pgfpathcurveto{\pgfqpoint{1.595875in}{3.347910in}}{\pgfqpoint{1.599148in}{3.355810in}}{\pgfqpoint{1.599148in}{3.364046in}}%
\pgfpathcurveto{\pgfqpoint{1.599148in}{3.372282in}}{\pgfqpoint{1.595875in}{3.380182in}}{\pgfqpoint{1.590051in}{3.386006in}}%
\pgfpathcurveto{\pgfqpoint{1.584227in}{3.391830in}}{\pgfqpoint{1.576327in}{3.395102in}}{\pgfqpoint{1.568091in}{3.395102in}}%
\pgfpathcurveto{\pgfqpoint{1.559855in}{3.395102in}}{\pgfqpoint{1.551955in}{3.391830in}}{\pgfqpoint{1.546131in}{3.386006in}}%
\pgfpathcurveto{\pgfqpoint{1.540307in}{3.380182in}}{\pgfqpoint{1.537035in}{3.372282in}}{\pgfqpoint{1.537035in}{3.364046in}}%
\pgfpathcurveto{\pgfqpoint{1.537035in}{3.355810in}}{\pgfqpoint{1.540307in}{3.347910in}}{\pgfqpoint{1.546131in}{3.342086in}}%
\pgfpathcurveto{\pgfqpoint{1.551955in}{3.336262in}}{\pgfqpoint{1.559855in}{3.332989in}}{\pgfqpoint{1.568091in}{3.332989in}}%
\pgfpathclose%
\pgfusepath{stroke,fill}%
\end{pgfscope}%
\begin{pgfscope}%
\pgfpathrectangle{\pgfqpoint{0.100000in}{0.212622in}}{\pgfqpoint{3.696000in}{3.696000in}}%
\pgfusepath{clip}%
\pgfsetbuttcap%
\pgfsetroundjoin%
\definecolor{currentfill}{rgb}{0.121569,0.466667,0.705882}%
\pgfsetfillcolor{currentfill}%
\pgfsetfillopacity{0.332722}%
\pgfsetlinewidth{1.003750pt}%
\definecolor{currentstroke}{rgb}{0.121569,0.466667,0.705882}%
\pgfsetstrokecolor{currentstroke}%
\pgfsetstrokeopacity{0.332722}%
\pgfsetdash{}{0pt}%
\pgfpathmoveto{\pgfqpoint{1.567524in}{3.331733in}}%
\pgfpathcurveto{\pgfqpoint{1.575761in}{3.331733in}}{\pgfqpoint{1.583661in}{3.335005in}}{\pgfqpoint{1.589485in}{3.340829in}}%
\pgfpathcurveto{\pgfqpoint{1.595308in}{3.346653in}}{\pgfqpoint{1.598581in}{3.354553in}}{\pgfqpoint{1.598581in}{3.362789in}}%
\pgfpathcurveto{\pgfqpoint{1.598581in}{3.371026in}}{\pgfqpoint{1.595308in}{3.378926in}}{\pgfqpoint{1.589485in}{3.384750in}}%
\pgfpathcurveto{\pgfqpoint{1.583661in}{3.390574in}}{\pgfqpoint{1.575761in}{3.393846in}}{\pgfqpoint{1.567524in}{3.393846in}}%
\pgfpathcurveto{\pgfqpoint{1.559288in}{3.393846in}}{\pgfqpoint{1.551388in}{3.390574in}}{\pgfqpoint{1.545564in}{3.384750in}}%
\pgfpathcurveto{\pgfqpoint{1.539740in}{3.378926in}}{\pgfqpoint{1.536468in}{3.371026in}}{\pgfqpoint{1.536468in}{3.362789in}}%
\pgfpathcurveto{\pgfqpoint{1.536468in}{3.354553in}}{\pgfqpoint{1.539740in}{3.346653in}}{\pgfqpoint{1.545564in}{3.340829in}}%
\pgfpathcurveto{\pgfqpoint{1.551388in}{3.335005in}}{\pgfqpoint{1.559288in}{3.331733in}}{\pgfqpoint{1.567524in}{3.331733in}}%
\pgfpathclose%
\pgfusepath{stroke,fill}%
\end{pgfscope}%
\begin{pgfscope}%
\pgfpathrectangle{\pgfqpoint{0.100000in}{0.212622in}}{\pgfqpoint{3.696000in}{3.696000in}}%
\pgfusepath{clip}%
\pgfsetbuttcap%
\pgfsetroundjoin%
\definecolor{currentfill}{rgb}{0.121569,0.466667,0.705882}%
\pgfsetfillcolor{currentfill}%
\pgfsetfillopacity{0.333555}%
\pgfsetlinewidth{1.003750pt}%
\definecolor{currentstroke}{rgb}{0.121569,0.466667,0.705882}%
\pgfsetstrokecolor{currentstroke}%
\pgfsetstrokeopacity{0.333555}%
\pgfsetdash{}{0pt}%
\pgfpathmoveto{\pgfqpoint{1.566511in}{3.329452in}}%
\pgfpathcurveto{\pgfqpoint{1.574747in}{3.329452in}}{\pgfqpoint{1.582647in}{3.332724in}}{\pgfqpoint{1.588471in}{3.338548in}}%
\pgfpathcurveto{\pgfqpoint{1.594295in}{3.344372in}}{\pgfqpoint{1.597567in}{3.352272in}}{\pgfqpoint{1.597567in}{3.360509in}}%
\pgfpathcurveto{\pgfqpoint{1.597567in}{3.368745in}}{\pgfqpoint{1.594295in}{3.376645in}}{\pgfqpoint{1.588471in}{3.382469in}}%
\pgfpathcurveto{\pgfqpoint{1.582647in}{3.388293in}}{\pgfqpoint{1.574747in}{3.391565in}}{\pgfqpoint{1.566511in}{3.391565in}}%
\pgfpathcurveto{\pgfqpoint{1.558275in}{3.391565in}}{\pgfqpoint{1.550374in}{3.388293in}}{\pgfqpoint{1.544551in}{3.382469in}}%
\pgfpathcurveto{\pgfqpoint{1.538727in}{3.376645in}}{\pgfqpoint{1.535454in}{3.368745in}}{\pgfqpoint{1.535454in}{3.360509in}}%
\pgfpathcurveto{\pgfqpoint{1.535454in}{3.352272in}}{\pgfqpoint{1.538727in}{3.344372in}}{\pgfqpoint{1.544551in}{3.338548in}}%
\pgfpathcurveto{\pgfqpoint{1.550374in}{3.332724in}}{\pgfqpoint{1.558275in}{3.329452in}}{\pgfqpoint{1.566511in}{3.329452in}}%
\pgfpathclose%
\pgfusepath{stroke,fill}%
\end{pgfscope}%
\begin{pgfscope}%
\pgfpathrectangle{\pgfqpoint{0.100000in}{0.212622in}}{\pgfqpoint{3.696000in}{3.696000in}}%
\pgfusepath{clip}%
\pgfsetbuttcap%
\pgfsetroundjoin%
\definecolor{currentfill}{rgb}{0.121569,0.466667,0.705882}%
\pgfsetfillcolor{currentfill}%
\pgfsetfillopacity{0.333880}%
\pgfsetlinewidth{1.003750pt}%
\definecolor{currentstroke}{rgb}{0.121569,0.466667,0.705882}%
\pgfsetstrokecolor{currentstroke}%
\pgfsetstrokeopacity{0.333880}%
\pgfsetdash{}{0pt}%
\pgfpathmoveto{\pgfqpoint{1.566113in}{3.328560in}}%
\pgfpathcurveto{\pgfqpoint{1.574349in}{3.328560in}}{\pgfqpoint{1.582249in}{3.331832in}}{\pgfqpoint{1.588073in}{3.337656in}}%
\pgfpathcurveto{\pgfqpoint{1.593897in}{3.343480in}}{\pgfqpoint{1.597169in}{3.351380in}}{\pgfqpoint{1.597169in}{3.359616in}}%
\pgfpathcurveto{\pgfqpoint{1.597169in}{3.367853in}}{\pgfqpoint{1.593897in}{3.375753in}}{\pgfqpoint{1.588073in}{3.381576in}}%
\pgfpathcurveto{\pgfqpoint{1.582249in}{3.387400in}}{\pgfqpoint{1.574349in}{3.390673in}}{\pgfqpoint{1.566113in}{3.390673in}}%
\pgfpathcurveto{\pgfqpoint{1.557876in}{3.390673in}}{\pgfqpoint{1.549976in}{3.387400in}}{\pgfqpoint{1.544152in}{3.381576in}}%
\pgfpathcurveto{\pgfqpoint{1.538328in}{3.375753in}}{\pgfqpoint{1.535056in}{3.367853in}}{\pgfqpoint{1.535056in}{3.359616in}}%
\pgfpathcurveto{\pgfqpoint{1.535056in}{3.351380in}}{\pgfqpoint{1.538328in}{3.343480in}}{\pgfqpoint{1.544152in}{3.337656in}}%
\pgfpathcurveto{\pgfqpoint{1.549976in}{3.331832in}}{\pgfqpoint{1.557876in}{3.328560in}}{\pgfqpoint{1.566113in}{3.328560in}}%
\pgfpathclose%
\pgfusepath{stroke,fill}%
\end{pgfscope}%
\begin{pgfscope}%
\pgfpathrectangle{\pgfqpoint{0.100000in}{0.212622in}}{\pgfqpoint{3.696000in}{3.696000in}}%
\pgfusepath{clip}%
\pgfsetbuttcap%
\pgfsetroundjoin%
\definecolor{currentfill}{rgb}{0.121569,0.466667,0.705882}%
\pgfsetfillcolor{currentfill}%
\pgfsetfillopacity{0.334470}%
\pgfsetlinewidth{1.003750pt}%
\definecolor{currentstroke}{rgb}{0.121569,0.466667,0.705882}%
\pgfsetstrokecolor{currentstroke}%
\pgfsetstrokeopacity{0.334470}%
\pgfsetdash{}{0pt}%
\pgfpathmoveto{\pgfqpoint{1.565381in}{3.326942in}}%
\pgfpathcurveto{\pgfqpoint{1.573617in}{3.326942in}}{\pgfqpoint{1.581517in}{3.330214in}}{\pgfqpoint{1.587341in}{3.336038in}}%
\pgfpathcurveto{\pgfqpoint{1.593165in}{3.341862in}}{\pgfqpoint{1.596437in}{3.349762in}}{\pgfqpoint{1.596437in}{3.357998in}}%
\pgfpathcurveto{\pgfqpoint{1.596437in}{3.366234in}}{\pgfqpoint{1.593165in}{3.374134in}}{\pgfqpoint{1.587341in}{3.379958in}}%
\pgfpathcurveto{\pgfqpoint{1.581517in}{3.385782in}}{\pgfqpoint{1.573617in}{3.389055in}}{\pgfqpoint{1.565381in}{3.389055in}}%
\pgfpathcurveto{\pgfqpoint{1.557145in}{3.389055in}}{\pgfqpoint{1.549245in}{3.385782in}}{\pgfqpoint{1.543421in}{3.379958in}}%
\pgfpathcurveto{\pgfqpoint{1.537597in}{3.374134in}}{\pgfqpoint{1.534324in}{3.366234in}}{\pgfqpoint{1.534324in}{3.357998in}}%
\pgfpathcurveto{\pgfqpoint{1.534324in}{3.349762in}}{\pgfqpoint{1.537597in}{3.341862in}}{\pgfqpoint{1.543421in}{3.336038in}}%
\pgfpathcurveto{\pgfqpoint{1.549245in}{3.330214in}}{\pgfqpoint{1.557145in}{3.326942in}}{\pgfqpoint{1.565381in}{3.326942in}}%
\pgfpathclose%
\pgfusepath{stroke,fill}%
\end{pgfscope}%
\begin{pgfscope}%
\pgfpathrectangle{\pgfqpoint{0.100000in}{0.212622in}}{\pgfqpoint{3.696000in}{3.696000in}}%
\pgfusepath{clip}%
\pgfsetbuttcap%
\pgfsetroundjoin%
\definecolor{currentfill}{rgb}{0.121569,0.466667,0.705882}%
\pgfsetfillcolor{currentfill}%
\pgfsetfillopacity{0.334477}%
\pgfsetlinewidth{1.003750pt}%
\definecolor{currentstroke}{rgb}{0.121569,0.466667,0.705882}%
\pgfsetstrokecolor{currentstroke}%
\pgfsetstrokeopacity{0.334477}%
\pgfsetdash{}{0pt}%
\pgfpathmoveto{\pgfqpoint{1.565373in}{3.326923in}}%
\pgfpathcurveto{\pgfqpoint{1.573609in}{3.326923in}}{\pgfqpoint{1.581509in}{3.330196in}}{\pgfqpoint{1.587333in}{3.336020in}}%
\pgfpathcurveto{\pgfqpoint{1.593157in}{3.341844in}}{\pgfqpoint{1.596429in}{3.349744in}}{\pgfqpoint{1.596429in}{3.357980in}}%
\pgfpathcurveto{\pgfqpoint{1.596429in}{3.366216in}}{\pgfqpoint{1.593157in}{3.374116in}}{\pgfqpoint{1.587333in}{3.379940in}}%
\pgfpathcurveto{\pgfqpoint{1.581509in}{3.385764in}}{\pgfqpoint{1.573609in}{3.389036in}}{\pgfqpoint{1.565373in}{3.389036in}}%
\pgfpathcurveto{\pgfqpoint{1.557136in}{3.389036in}}{\pgfqpoint{1.549236in}{3.385764in}}{\pgfqpoint{1.543412in}{3.379940in}}%
\pgfpathcurveto{\pgfqpoint{1.537588in}{3.374116in}}{\pgfqpoint{1.534316in}{3.366216in}}{\pgfqpoint{1.534316in}{3.357980in}}%
\pgfpathcurveto{\pgfqpoint{1.534316in}{3.349744in}}{\pgfqpoint{1.537588in}{3.341844in}}{\pgfqpoint{1.543412in}{3.336020in}}%
\pgfpathcurveto{\pgfqpoint{1.549236in}{3.330196in}}{\pgfqpoint{1.557136in}{3.326923in}}{\pgfqpoint{1.565373in}{3.326923in}}%
\pgfpathclose%
\pgfusepath{stroke,fill}%
\end{pgfscope}%
\begin{pgfscope}%
\pgfpathrectangle{\pgfqpoint{0.100000in}{0.212622in}}{\pgfqpoint{3.696000in}{3.696000in}}%
\pgfusepath{clip}%
\pgfsetbuttcap%
\pgfsetroundjoin%
\definecolor{currentfill}{rgb}{0.121569,0.466667,0.705882}%
\pgfsetfillcolor{currentfill}%
\pgfsetfillopacity{0.334489}%
\pgfsetlinewidth{1.003750pt}%
\definecolor{currentstroke}{rgb}{0.121569,0.466667,0.705882}%
\pgfsetstrokecolor{currentstroke}%
\pgfsetstrokeopacity{0.334489}%
\pgfsetdash{}{0pt}%
\pgfpathmoveto{\pgfqpoint{1.565357in}{3.326890in}}%
\pgfpathcurveto{\pgfqpoint{1.573594in}{3.326890in}}{\pgfqpoint{1.581494in}{3.330163in}}{\pgfqpoint{1.587318in}{3.335987in}}%
\pgfpathcurveto{\pgfqpoint{1.593142in}{3.341811in}}{\pgfqpoint{1.596414in}{3.349711in}}{\pgfqpoint{1.596414in}{3.357947in}}%
\pgfpathcurveto{\pgfqpoint{1.596414in}{3.366183in}}{\pgfqpoint{1.593142in}{3.374083in}}{\pgfqpoint{1.587318in}{3.379907in}}%
\pgfpathcurveto{\pgfqpoint{1.581494in}{3.385731in}}{\pgfqpoint{1.573594in}{3.389003in}}{\pgfqpoint{1.565357in}{3.389003in}}%
\pgfpathcurveto{\pgfqpoint{1.557121in}{3.389003in}}{\pgfqpoint{1.549221in}{3.385731in}}{\pgfqpoint{1.543397in}{3.379907in}}%
\pgfpathcurveto{\pgfqpoint{1.537573in}{3.374083in}}{\pgfqpoint{1.534301in}{3.366183in}}{\pgfqpoint{1.534301in}{3.357947in}}%
\pgfpathcurveto{\pgfqpoint{1.534301in}{3.349711in}}{\pgfqpoint{1.537573in}{3.341811in}}{\pgfqpoint{1.543397in}{3.335987in}}%
\pgfpathcurveto{\pgfqpoint{1.549221in}{3.330163in}}{\pgfqpoint{1.557121in}{3.326890in}}{\pgfqpoint{1.565357in}{3.326890in}}%
\pgfpathclose%
\pgfusepath{stroke,fill}%
\end{pgfscope}%
\begin{pgfscope}%
\pgfpathrectangle{\pgfqpoint{0.100000in}{0.212622in}}{\pgfqpoint{3.696000in}{3.696000in}}%
\pgfusepath{clip}%
\pgfsetbuttcap%
\pgfsetroundjoin%
\definecolor{currentfill}{rgb}{0.121569,0.466667,0.705882}%
\pgfsetfillcolor{currentfill}%
\pgfsetfillopacity{0.334511}%
\pgfsetlinewidth{1.003750pt}%
\definecolor{currentstroke}{rgb}{0.121569,0.466667,0.705882}%
\pgfsetstrokecolor{currentstroke}%
\pgfsetstrokeopacity{0.334511}%
\pgfsetdash{}{0pt}%
\pgfpathmoveto{\pgfqpoint{1.565330in}{3.326831in}}%
\pgfpathcurveto{\pgfqpoint{1.573566in}{3.326831in}}{\pgfqpoint{1.581466in}{3.330103in}}{\pgfqpoint{1.587290in}{3.335927in}}%
\pgfpathcurveto{\pgfqpoint{1.593114in}{3.341751in}}{\pgfqpoint{1.596386in}{3.349651in}}{\pgfqpoint{1.596386in}{3.357887in}}%
\pgfpathcurveto{\pgfqpoint{1.596386in}{3.366123in}}{\pgfqpoint{1.593114in}{3.374023in}}{\pgfqpoint{1.587290in}{3.379847in}}%
\pgfpathcurveto{\pgfqpoint{1.581466in}{3.385671in}}{\pgfqpoint{1.573566in}{3.388944in}}{\pgfqpoint{1.565330in}{3.388944in}}%
\pgfpathcurveto{\pgfqpoint{1.557094in}{3.388944in}}{\pgfqpoint{1.549194in}{3.385671in}}{\pgfqpoint{1.543370in}{3.379847in}}%
\pgfpathcurveto{\pgfqpoint{1.537546in}{3.374023in}}{\pgfqpoint{1.534273in}{3.366123in}}{\pgfqpoint{1.534273in}{3.357887in}}%
\pgfpathcurveto{\pgfqpoint{1.534273in}{3.349651in}}{\pgfqpoint{1.537546in}{3.341751in}}{\pgfqpoint{1.543370in}{3.335927in}}%
\pgfpathcurveto{\pgfqpoint{1.549194in}{3.330103in}}{\pgfqpoint{1.557094in}{3.326831in}}{\pgfqpoint{1.565330in}{3.326831in}}%
\pgfpathclose%
\pgfusepath{stroke,fill}%
\end{pgfscope}%
\begin{pgfscope}%
\pgfpathrectangle{\pgfqpoint{0.100000in}{0.212622in}}{\pgfqpoint{3.696000in}{3.696000in}}%
\pgfusepath{clip}%
\pgfsetbuttcap%
\pgfsetroundjoin%
\definecolor{currentfill}{rgb}{0.121569,0.466667,0.705882}%
\pgfsetfillcolor{currentfill}%
\pgfsetfillopacity{0.334551}%
\pgfsetlinewidth{1.003750pt}%
\definecolor{currentstroke}{rgb}{0.121569,0.466667,0.705882}%
\pgfsetstrokecolor{currentstroke}%
\pgfsetstrokeopacity{0.334551}%
\pgfsetdash{}{0pt}%
\pgfpathmoveto{\pgfqpoint{1.565281in}{3.326721in}}%
\pgfpathcurveto{\pgfqpoint{1.573517in}{3.326721in}}{\pgfqpoint{1.581418in}{3.329994in}}{\pgfqpoint{1.587241in}{3.335818in}}%
\pgfpathcurveto{\pgfqpoint{1.593065in}{3.341642in}}{\pgfqpoint{1.596338in}{3.349542in}}{\pgfqpoint{1.596338in}{3.357778in}}%
\pgfpathcurveto{\pgfqpoint{1.596338in}{3.366014in}}{\pgfqpoint{1.593065in}{3.373914in}}{\pgfqpoint{1.587241in}{3.379738in}}%
\pgfpathcurveto{\pgfqpoint{1.581418in}{3.385562in}}{\pgfqpoint{1.573517in}{3.388834in}}{\pgfqpoint{1.565281in}{3.388834in}}%
\pgfpathcurveto{\pgfqpoint{1.557045in}{3.388834in}}{\pgfqpoint{1.549145in}{3.385562in}}{\pgfqpoint{1.543321in}{3.379738in}}%
\pgfpathcurveto{\pgfqpoint{1.537497in}{3.373914in}}{\pgfqpoint{1.534225in}{3.366014in}}{\pgfqpoint{1.534225in}{3.357778in}}%
\pgfpathcurveto{\pgfqpoint{1.534225in}{3.349542in}}{\pgfqpoint{1.537497in}{3.341642in}}{\pgfqpoint{1.543321in}{3.335818in}}%
\pgfpathcurveto{\pgfqpoint{1.549145in}{3.329994in}}{\pgfqpoint{1.557045in}{3.326721in}}{\pgfqpoint{1.565281in}{3.326721in}}%
\pgfpathclose%
\pgfusepath{stroke,fill}%
\end{pgfscope}%
\begin{pgfscope}%
\pgfpathrectangle{\pgfqpoint{0.100000in}{0.212622in}}{\pgfqpoint{3.696000in}{3.696000in}}%
\pgfusepath{clip}%
\pgfsetbuttcap%
\pgfsetroundjoin%
\definecolor{currentfill}{rgb}{0.121569,0.466667,0.705882}%
\pgfsetfillcolor{currentfill}%
\pgfsetfillopacity{0.334624}%
\pgfsetlinewidth{1.003750pt}%
\definecolor{currentstroke}{rgb}{0.121569,0.466667,0.705882}%
\pgfsetstrokecolor{currentstroke}%
\pgfsetstrokeopacity{0.334624}%
\pgfsetdash{}{0pt}%
\pgfpathmoveto{\pgfqpoint{1.565191in}{3.326524in}}%
\pgfpathcurveto{\pgfqpoint{1.573427in}{3.326524in}}{\pgfqpoint{1.581327in}{3.329796in}}{\pgfqpoint{1.587151in}{3.335620in}}%
\pgfpathcurveto{\pgfqpoint{1.592975in}{3.341444in}}{\pgfqpoint{1.596247in}{3.349344in}}{\pgfqpoint{1.596247in}{3.357580in}}%
\pgfpathcurveto{\pgfqpoint{1.596247in}{3.365816in}}{\pgfqpoint{1.592975in}{3.373716in}}{\pgfqpoint{1.587151in}{3.379540in}}%
\pgfpathcurveto{\pgfqpoint{1.581327in}{3.385364in}}{\pgfqpoint{1.573427in}{3.388637in}}{\pgfqpoint{1.565191in}{3.388637in}}%
\pgfpathcurveto{\pgfqpoint{1.556954in}{3.388637in}}{\pgfqpoint{1.549054in}{3.385364in}}{\pgfqpoint{1.543230in}{3.379540in}}%
\pgfpathcurveto{\pgfqpoint{1.537406in}{3.373716in}}{\pgfqpoint{1.534134in}{3.365816in}}{\pgfqpoint{1.534134in}{3.357580in}}%
\pgfpathcurveto{\pgfqpoint{1.534134in}{3.349344in}}{\pgfqpoint{1.537406in}{3.341444in}}{\pgfqpoint{1.543230in}{3.335620in}}%
\pgfpathcurveto{\pgfqpoint{1.549054in}{3.329796in}}{\pgfqpoint{1.556954in}{3.326524in}}{\pgfqpoint{1.565191in}{3.326524in}}%
\pgfpathclose%
\pgfusepath{stroke,fill}%
\end{pgfscope}%
\begin{pgfscope}%
\pgfpathrectangle{\pgfqpoint{0.100000in}{0.212622in}}{\pgfqpoint{3.696000in}{3.696000in}}%
\pgfusepath{clip}%
\pgfsetbuttcap%
\pgfsetroundjoin%
\definecolor{currentfill}{rgb}{0.121569,0.466667,0.705882}%
\pgfsetfillcolor{currentfill}%
\pgfsetfillopacity{0.334756}%
\pgfsetlinewidth{1.003750pt}%
\definecolor{currentstroke}{rgb}{0.121569,0.466667,0.705882}%
\pgfsetstrokecolor{currentstroke}%
\pgfsetstrokeopacity{0.334756}%
\pgfsetdash{}{0pt}%
\pgfpathmoveto{\pgfqpoint{1.565023in}{3.326166in}}%
\pgfpathcurveto{\pgfqpoint{1.573260in}{3.326166in}}{\pgfqpoint{1.581160in}{3.329438in}}{\pgfqpoint{1.586984in}{3.335262in}}%
\pgfpathcurveto{\pgfqpoint{1.592808in}{3.341086in}}{\pgfqpoint{1.596080in}{3.348986in}}{\pgfqpoint{1.596080in}{3.357222in}}%
\pgfpathcurveto{\pgfqpoint{1.596080in}{3.365459in}}{\pgfqpoint{1.592808in}{3.373359in}}{\pgfqpoint{1.586984in}{3.379182in}}%
\pgfpathcurveto{\pgfqpoint{1.581160in}{3.385006in}}{\pgfqpoint{1.573260in}{3.388279in}}{\pgfqpoint{1.565023in}{3.388279in}}%
\pgfpathcurveto{\pgfqpoint{1.556787in}{3.388279in}}{\pgfqpoint{1.548887in}{3.385006in}}{\pgfqpoint{1.543063in}{3.379182in}}%
\pgfpathcurveto{\pgfqpoint{1.537239in}{3.373359in}}{\pgfqpoint{1.533967in}{3.365459in}}{\pgfqpoint{1.533967in}{3.357222in}}%
\pgfpathcurveto{\pgfqpoint{1.533967in}{3.348986in}}{\pgfqpoint{1.537239in}{3.341086in}}{\pgfqpoint{1.543063in}{3.335262in}}%
\pgfpathcurveto{\pgfqpoint{1.548887in}{3.329438in}}{\pgfqpoint{1.556787in}{3.326166in}}{\pgfqpoint{1.565023in}{3.326166in}}%
\pgfpathclose%
\pgfusepath{stroke,fill}%
\end{pgfscope}%
\begin{pgfscope}%
\pgfpathrectangle{\pgfqpoint{0.100000in}{0.212622in}}{\pgfqpoint{3.696000in}{3.696000in}}%
\pgfusepath{clip}%
\pgfsetbuttcap%
\pgfsetroundjoin%
\definecolor{currentfill}{rgb}{0.121569,0.466667,0.705882}%
\pgfsetfillcolor{currentfill}%
\pgfsetfillopacity{0.334996}%
\pgfsetlinewidth{1.003750pt}%
\definecolor{currentstroke}{rgb}{0.121569,0.466667,0.705882}%
\pgfsetstrokecolor{currentstroke}%
\pgfsetstrokeopacity{0.334996}%
\pgfsetdash{}{0pt}%
\pgfpathmoveto{\pgfqpoint{1.564728in}{3.325510in}}%
\pgfpathcurveto{\pgfqpoint{1.572964in}{3.325510in}}{\pgfqpoint{1.580864in}{3.328783in}}{\pgfqpoint{1.586688in}{3.334607in}}%
\pgfpathcurveto{\pgfqpoint{1.592512in}{3.340431in}}{\pgfqpoint{1.595784in}{3.348331in}}{\pgfqpoint{1.595784in}{3.356567in}}%
\pgfpathcurveto{\pgfqpoint{1.595784in}{3.364803in}}{\pgfqpoint{1.592512in}{3.372703in}}{\pgfqpoint{1.586688in}{3.378527in}}%
\pgfpathcurveto{\pgfqpoint{1.580864in}{3.384351in}}{\pgfqpoint{1.572964in}{3.387623in}}{\pgfqpoint{1.564728in}{3.387623in}}%
\pgfpathcurveto{\pgfqpoint{1.556491in}{3.387623in}}{\pgfqpoint{1.548591in}{3.384351in}}{\pgfqpoint{1.542767in}{3.378527in}}%
\pgfpathcurveto{\pgfqpoint{1.536943in}{3.372703in}}{\pgfqpoint{1.533671in}{3.364803in}}{\pgfqpoint{1.533671in}{3.356567in}}%
\pgfpathcurveto{\pgfqpoint{1.533671in}{3.348331in}}{\pgfqpoint{1.536943in}{3.340431in}}{\pgfqpoint{1.542767in}{3.334607in}}%
\pgfpathcurveto{\pgfqpoint{1.548591in}{3.328783in}}{\pgfqpoint{1.556491in}{3.325510in}}{\pgfqpoint{1.564728in}{3.325510in}}%
\pgfpathclose%
\pgfusepath{stroke,fill}%
\end{pgfscope}%
\begin{pgfscope}%
\pgfpathrectangle{\pgfqpoint{0.100000in}{0.212622in}}{\pgfqpoint{3.696000in}{3.696000in}}%
\pgfusepath{clip}%
\pgfsetbuttcap%
\pgfsetroundjoin%
\definecolor{currentfill}{rgb}{0.121569,0.466667,0.705882}%
\pgfsetfillcolor{currentfill}%
\pgfsetfillopacity{0.335433}%
\pgfsetlinewidth{1.003750pt}%
\definecolor{currentstroke}{rgb}{0.121569,0.466667,0.705882}%
\pgfsetstrokecolor{currentstroke}%
\pgfsetstrokeopacity{0.335433}%
\pgfsetdash{}{0pt}%
\pgfpathmoveto{\pgfqpoint{1.564189in}{3.324320in}}%
\pgfpathcurveto{\pgfqpoint{1.572425in}{3.324320in}}{\pgfqpoint{1.580325in}{3.327593in}}{\pgfqpoint{1.586149in}{3.333417in}}%
\pgfpathcurveto{\pgfqpoint{1.591973in}{3.339241in}}{\pgfqpoint{1.595245in}{3.347141in}}{\pgfqpoint{1.595245in}{3.355377in}}%
\pgfpathcurveto{\pgfqpoint{1.595245in}{3.363613in}}{\pgfqpoint{1.591973in}{3.371513in}}{\pgfqpoint{1.586149in}{3.377337in}}%
\pgfpathcurveto{\pgfqpoint{1.580325in}{3.383161in}}{\pgfqpoint{1.572425in}{3.386433in}}{\pgfqpoint{1.564189in}{3.386433in}}%
\pgfpathcurveto{\pgfqpoint{1.555952in}{3.386433in}}{\pgfqpoint{1.548052in}{3.383161in}}{\pgfqpoint{1.542228in}{3.377337in}}%
\pgfpathcurveto{\pgfqpoint{1.536404in}{3.371513in}}{\pgfqpoint{1.533132in}{3.363613in}}{\pgfqpoint{1.533132in}{3.355377in}}%
\pgfpathcurveto{\pgfqpoint{1.533132in}{3.347141in}}{\pgfqpoint{1.536404in}{3.339241in}}{\pgfqpoint{1.542228in}{3.333417in}}%
\pgfpathcurveto{\pgfqpoint{1.548052in}{3.327593in}}{\pgfqpoint{1.555952in}{3.324320in}}{\pgfqpoint{1.564189in}{3.324320in}}%
\pgfpathclose%
\pgfusepath{stroke,fill}%
\end{pgfscope}%
\begin{pgfscope}%
\pgfpathrectangle{\pgfqpoint{0.100000in}{0.212622in}}{\pgfqpoint{3.696000in}{3.696000in}}%
\pgfusepath{clip}%
\pgfsetbuttcap%
\pgfsetroundjoin%
\definecolor{currentfill}{rgb}{0.121569,0.466667,0.705882}%
\pgfsetfillcolor{currentfill}%
\pgfsetfillopacity{0.336229}%
\pgfsetlinewidth{1.003750pt}%
\definecolor{currentstroke}{rgb}{0.121569,0.466667,0.705882}%
\pgfsetstrokecolor{currentstroke}%
\pgfsetstrokeopacity{0.336229}%
\pgfsetdash{}{0pt}%
\pgfpathmoveto{\pgfqpoint{1.563199in}{3.322168in}}%
\pgfpathcurveto{\pgfqpoint{1.571435in}{3.322168in}}{\pgfqpoint{1.579335in}{3.325440in}}{\pgfqpoint{1.585159in}{3.331264in}}%
\pgfpathcurveto{\pgfqpoint{1.590983in}{3.337088in}}{\pgfqpoint{1.594255in}{3.344988in}}{\pgfqpoint{1.594255in}{3.353224in}}%
\pgfpathcurveto{\pgfqpoint{1.594255in}{3.361461in}}{\pgfqpoint{1.590983in}{3.369361in}}{\pgfqpoint{1.585159in}{3.375185in}}%
\pgfpathcurveto{\pgfqpoint{1.579335in}{3.381009in}}{\pgfqpoint{1.571435in}{3.384281in}}{\pgfqpoint{1.563199in}{3.384281in}}%
\pgfpathcurveto{\pgfqpoint{1.554963in}{3.384281in}}{\pgfqpoint{1.547063in}{3.381009in}}{\pgfqpoint{1.541239in}{3.375185in}}%
\pgfpathcurveto{\pgfqpoint{1.535415in}{3.369361in}}{\pgfqpoint{1.532142in}{3.361461in}}{\pgfqpoint{1.532142in}{3.353224in}}%
\pgfpathcurveto{\pgfqpoint{1.532142in}{3.344988in}}{\pgfqpoint{1.535415in}{3.337088in}}{\pgfqpoint{1.541239in}{3.331264in}}%
\pgfpathcurveto{\pgfqpoint{1.547063in}{3.325440in}}{\pgfqpoint{1.554963in}{3.322168in}}{\pgfqpoint{1.563199in}{3.322168in}}%
\pgfpathclose%
\pgfusepath{stroke,fill}%
\end{pgfscope}%
\begin{pgfscope}%
\pgfpathrectangle{\pgfqpoint{0.100000in}{0.212622in}}{\pgfqpoint{3.696000in}{3.696000in}}%
\pgfusepath{clip}%
\pgfsetbuttcap%
\pgfsetroundjoin%
\definecolor{currentfill}{rgb}{0.121569,0.466667,0.705882}%
\pgfsetfillcolor{currentfill}%
\pgfsetfillopacity{0.336325}%
\pgfsetlinewidth{1.003750pt}%
\definecolor{currentstroke}{rgb}{0.121569,0.466667,0.705882}%
\pgfsetstrokecolor{currentstroke}%
\pgfsetstrokeopacity{0.336325}%
\pgfsetdash{}{0pt}%
\pgfpathmoveto{\pgfqpoint{1.835778in}{3.214827in}}%
\pgfpathcurveto{\pgfqpoint{1.844014in}{3.214827in}}{\pgfqpoint{1.851914in}{3.218100in}}{\pgfqpoint{1.857738in}{3.223924in}}%
\pgfpathcurveto{\pgfqpoint{1.863562in}{3.229747in}}{\pgfqpoint{1.866834in}{3.237648in}}{\pgfqpoint{1.866834in}{3.245884in}}%
\pgfpathcurveto{\pgfqpoint{1.866834in}{3.254120in}}{\pgfqpoint{1.863562in}{3.262020in}}{\pgfqpoint{1.857738in}{3.267844in}}%
\pgfpathcurveto{\pgfqpoint{1.851914in}{3.273668in}}{\pgfqpoint{1.844014in}{3.276940in}}{\pgfqpoint{1.835778in}{3.276940in}}%
\pgfpathcurveto{\pgfqpoint{1.827542in}{3.276940in}}{\pgfqpoint{1.819642in}{3.273668in}}{\pgfqpoint{1.813818in}{3.267844in}}%
\pgfpathcurveto{\pgfqpoint{1.807994in}{3.262020in}}{\pgfqpoint{1.804721in}{3.254120in}}{\pgfqpoint{1.804721in}{3.245884in}}%
\pgfpathcurveto{\pgfqpoint{1.804721in}{3.237648in}}{\pgfqpoint{1.807994in}{3.229747in}}{\pgfqpoint{1.813818in}{3.223924in}}%
\pgfpathcurveto{\pgfqpoint{1.819642in}{3.218100in}}{\pgfqpoint{1.827542in}{3.214827in}}{\pgfqpoint{1.835778in}{3.214827in}}%
\pgfpathclose%
\pgfusepath{stroke,fill}%
\end{pgfscope}%
\begin{pgfscope}%
\pgfpathrectangle{\pgfqpoint{0.100000in}{0.212622in}}{\pgfqpoint{3.696000in}{3.696000in}}%
\pgfusepath{clip}%
\pgfsetbuttcap%
\pgfsetroundjoin%
\definecolor{currentfill}{rgb}{0.121569,0.466667,0.705882}%
\pgfsetfillcolor{currentfill}%
\pgfsetfillopacity{0.337685}%
\pgfsetlinewidth{1.003750pt}%
\definecolor{currentstroke}{rgb}{0.121569,0.466667,0.705882}%
\pgfsetstrokecolor{currentstroke}%
\pgfsetstrokeopacity{0.337685}%
\pgfsetdash{}{0pt}%
\pgfpathmoveto{\pgfqpoint{1.561476in}{3.318231in}}%
\pgfpathcurveto{\pgfqpoint{1.569713in}{3.318231in}}{\pgfqpoint{1.577613in}{3.321503in}}{\pgfqpoint{1.583437in}{3.327327in}}%
\pgfpathcurveto{\pgfqpoint{1.589261in}{3.333151in}}{\pgfqpoint{1.592533in}{3.341051in}}{\pgfqpoint{1.592533in}{3.349288in}}%
\pgfpathcurveto{\pgfqpoint{1.592533in}{3.357524in}}{\pgfqpoint{1.589261in}{3.365424in}}{\pgfqpoint{1.583437in}{3.371248in}}%
\pgfpathcurveto{\pgfqpoint{1.577613in}{3.377072in}}{\pgfqpoint{1.569713in}{3.380344in}}{\pgfqpoint{1.561476in}{3.380344in}}%
\pgfpathcurveto{\pgfqpoint{1.553240in}{3.380344in}}{\pgfqpoint{1.545340in}{3.377072in}}{\pgfqpoint{1.539516in}{3.371248in}}%
\pgfpathcurveto{\pgfqpoint{1.533692in}{3.365424in}}{\pgfqpoint{1.530420in}{3.357524in}}{\pgfqpoint{1.530420in}{3.349288in}}%
\pgfpathcurveto{\pgfqpoint{1.530420in}{3.341051in}}{\pgfqpoint{1.533692in}{3.333151in}}{\pgfqpoint{1.539516in}{3.327327in}}%
\pgfpathcurveto{\pgfqpoint{1.545340in}{3.321503in}}{\pgfqpoint{1.553240in}{3.318231in}}{\pgfqpoint{1.561476in}{3.318231in}}%
\pgfpathclose%
\pgfusepath{stroke,fill}%
\end{pgfscope}%
\begin{pgfscope}%
\pgfpathrectangle{\pgfqpoint{0.100000in}{0.212622in}}{\pgfqpoint{3.696000in}{3.696000in}}%
\pgfusepath{clip}%
\pgfsetbuttcap%
\pgfsetroundjoin%
\definecolor{currentfill}{rgb}{0.121569,0.466667,0.705882}%
\pgfsetfillcolor{currentfill}%
\pgfsetfillopacity{0.338679}%
\pgfsetlinewidth{1.003750pt}%
\definecolor{currentstroke}{rgb}{0.121569,0.466667,0.705882}%
\pgfsetstrokecolor{currentstroke}%
\pgfsetstrokeopacity{0.338679}%
\pgfsetdash{}{0pt}%
\pgfpathmoveto{\pgfqpoint{1.560266in}{3.315581in}}%
\pgfpathcurveto{\pgfqpoint{1.568503in}{3.315581in}}{\pgfqpoint{1.576403in}{3.318853in}}{\pgfqpoint{1.582227in}{3.324677in}}%
\pgfpathcurveto{\pgfqpoint{1.588051in}{3.330501in}}{\pgfqpoint{1.591323in}{3.338401in}}{\pgfqpoint{1.591323in}{3.346637in}}%
\pgfpathcurveto{\pgfqpoint{1.591323in}{3.354873in}}{\pgfqpoint{1.588051in}{3.362773in}}{\pgfqpoint{1.582227in}{3.368597in}}%
\pgfpathcurveto{\pgfqpoint{1.576403in}{3.374421in}}{\pgfqpoint{1.568503in}{3.377694in}}{\pgfqpoint{1.560266in}{3.377694in}}%
\pgfpathcurveto{\pgfqpoint{1.552030in}{3.377694in}}{\pgfqpoint{1.544130in}{3.374421in}}{\pgfqpoint{1.538306in}{3.368597in}}%
\pgfpathcurveto{\pgfqpoint{1.532482in}{3.362773in}}{\pgfqpoint{1.529210in}{3.354873in}}{\pgfqpoint{1.529210in}{3.346637in}}%
\pgfpathcurveto{\pgfqpoint{1.529210in}{3.338401in}}{\pgfqpoint{1.532482in}{3.330501in}}{\pgfqpoint{1.538306in}{3.324677in}}%
\pgfpathcurveto{\pgfqpoint{1.544130in}{3.318853in}}{\pgfqpoint{1.552030in}{3.315581in}}{\pgfqpoint{1.560266in}{3.315581in}}%
\pgfpathclose%
\pgfusepath{stroke,fill}%
\end{pgfscope}%
\begin{pgfscope}%
\pgfpathrectangle{\pgfqpoint{0.100000in}{0.212622in}}{\pgfqpoint{3.696000in}{3.696000in}}%
\pgfusepath{clip}%
\pgfsetbuttcap%
\pgfsetroundjoin%
\definecolor{currentfill}{rgb}{0.121569,0.466667,0.705882}%
\pgfsetfillcolor{currentfill}%
\pgfsetfillopacity{0.340486}%
\pgfsetlinewidth{1.003750pt}%
\definecolor{currentstroke}{rgb}{0.121569,0.466667,0.705882}%
\pgfsetstrokecolor{currentstroke}%
\pgfsetstrokeopacity{0.340486}%
\pgfsetdash{}{0pt}%
\pgfpathmoveto{\pgfqpoint{1.558122in}{3.310720in}}%
\pgfpathcurveto{\pgfqpoint{1.566358in}{3.310720in}}{\pgfqpoint{1.574258in}{3.313992in}}{\pgfqpoint{1.580082in}{3.319816in}}%
\pgfpathcurveto{\pgfqpoint{1.585906in}{3.325640in}}{\pgfqpoint{1.589178in}{3.333540in}}{\pgfqpoint{1.589178in}{3.341776in}}%
\pgfpathcurveto{\pgfqpoint{1.589178in}{3.350013in}}{\pgfqpoint{1.585906in}{3.357913in}}{\pgfqpoint{1.580082in}{3.363737in}}%
\pgfpathcurveto{\pgfqpoint{1.574258in}{3.369561in}}{\pgfqpoint{1.566358in}{3.372833in}}{\pgfqpoint{1.558122in}{3.372833in}}%
\pgfpathcurveto{\pgfqpoint{1.549886in}{3.372833in}}{\pgfqpoint{1.541986in}{3.369561in}}{\pgfqpoint{1.536162in}{3.363737in}}%
\pgfpathcurveto{\pgfqpoint{1.530338in}{3.357913in}}{\pgfqpoint{1.527065in}{3.350013in}}{\pgfqpoint{1.527065in}{3.341776in}}%
\pgfpathcurveto{\pgfqpoint{1.527065in}{3.333540in}}{\pgfqpoint{1.530338in}{3.325640in}}{\pgfqpoint{1.536162in}{3.319816in}}%
\pgfpathcurveto{\pgfqpoint{1.541986in}{3.313992in}}{\pgfqpoint{1.549886in}{3.310720in}}{\pgfqpoint{1.558122in}{3.310720in}}%
\pgfpathclose%
\pgfusepath{stroke,fill}%
\end{pgfscope}%
\begin{pgfscope}%
\pgfpathrectangle{\pgfqpoint{0.100000in}{0.212622in}}{\pgfqpoint{3.696000in}{3.696000in}}%
\pgfusepath{clip}%
\pgfsetbuttcap%
\pgfsetroundjoin%
\definecolor{currentfill}{rgb}{0.121569,0.466667,0.705882}%
\pgfsetfillcolor{currentfill}%
\pgfsetfillopacity{0.341770}%
\pgfsetlinewidth{1.003750pt}%
\definecolor{currentstroke}{rgb}{0.121569,0.466667,0.705882}%
\pgfsetstrokecolor{currentstroke}%
\pgfsetstrokeopacity{0.341770}%
\pgfsetdash{}{0pt}%
\pgfpathmoveto{\pgfqpoint{1.556545in}{3.307311in}}%
\pgfpathcurveto{\pgfqpoint{1.564782in}{3.307311in}}{\pgfqpoint{1.572682in}{3.310584in}}{\pgfqpoint{1.578506in}{3.316407in}}%
\pgfpathcurveto{\pgfqpoint{1.584329in}{3.322231in}}{\pgfqpoint{1.587602in}{3.330131in}}{\pgfqpoint{1.587602in}{3.338368in}}%
\pgfpathcurveto{\pgfqpoint{1.587602in}{3.346604in}}{\pgfqpoint{1.584329in}{3.354504in}}{\pgfqpoint{1.578506in}{3.360328in}}%
\pgfpathcurveto{\pgfqpoint{1.572682in}{3.366152in}}{\pgfqpoint{1.564782in}{3.369424in}}{\pgfqpoint{1.556545in}{3.369424in}}%
\pgfpathcurveto{\pgfqpoint{1.548309in}{3.369424in}}{\pgfqpoint{1.540409in}{3.366152in}}{\pgfqpoint{1.534585in}{3.360328in}}%
\pgfpathcurveto{\pgfqpoint{1.528761in}{3.354504in}}{\pgfqpoint{1.525489in}{3.346604in}}{\pgfqpoint{1.525489in}{3.338368in}}%
\pgfpathcurveto{\pgfqpoint{1.525489in}{3.330131in}}{\pgfqpoint{1.528761in}{3.322231in}}{\pgfqpoint{1.534585in}{3.316407in}}%
\pgfpathcurveto{\pgfqpoint{1.540409in}{3.310584in}}{\pgfqpoint{1.548309in}{3.307311in}}{\pgfqpoint{1.556545in}{3.307311in}}%
\pgfpathclose%
\pgfusepath{stroke,fill}%
\end{pgfscope}%
\begin{pgfscope}%
\pgfpathrectangle{\pgfqpoint{0.100000in}{0.212622in}}{\pgfqpoint{3.696000in}{3.696000in}}%
\pgfusepath{clip}%
\pgfsetbuttcap%
\pgfsetroundjoin%
\definecolor{currentfill}{rgb}{0.121569,0.466667,0.705882}%
\pgfsetfillcolor{currentfill}%
\pgfsetfillopacity{0.342397}%
\pgfsetlinewidth{1.003750pt}%
\definecolor{currentstroke}{rgb}{0.121569,0.466667,0.705882}%
\pgfsetstrokecolor{currentstroke}%
\pgfsetstrokeopacity{0.342397}%
\pgfsetdash{}{0pt}%
\pgfpathmoveto{\pgfqpoint{1.851075in}{3.193389in}}%
\pgfpathcurveto{\pgfqpoint{1.859311in}{3.193389in}}{\pgfqpoint{1.867211in}{3.196661in}}{\pgfqpoint{1.873035in}{3.202485in}}%
\pgfpathcurveto{\pgfqpoint{1.878859in}{3.208309in}}{\pgfqpoint{1.882131in}{3.216209in}}{\pgfqpoint{1.882131in}{3.224446in}}%
\pgfpathcurveto{\pgfqpoint{1.882131in}{3.232682in}}{\pgfqpoint{1.878859in}{3.240582in}}{\pgfqpoint{1.873035in}{3.246406in}}%
\pgfpathcurveto{\pgfqpoint{1.867211in}{3.252230in}}{\pgfqpoint{1.859311in}{3.255502in}}{\pgfqpoint{1.851075in}{3.255502in}}%
\pgfpathcurveto{\pgfqpoint{1.842838in}{3.255502in}}{\pgfqpoint{1.834938in}{3.252230in}}{\pgfqpoint{1.829114in}{3.246406in}}%
\pgfpathcurveto{\pgfqpoint{1.823290in}{3.240582in}}{\pgfqpoint{1.820018in}{3.232682in}}{\pgfqpoint{1.820018in}{3.224446in}}%
\pgfpathcurveto{\pgfqpoint{1.820018in}{3.216209in}}{\pgfqpoint{1.823290in}{3.208309in}}{\pgfqpoint{1.829114in}{3.202485in}}%
\pgfpathcurveto{\pgfqpoint{1.834938in}{3.196661in}}{\pgfqpoint{1.842838in}{3.193389in}}{\pgfqpoint{1.851075in}{3.193389in}}%
\pgfpathclose%
\pgfusepath{stroke,fill}%
\end{pgfscope}%
\begin{pgfscope}%
\pgfpathrectangle{\pgfqpoint{0.100000in}{0.212622in}}{\pgfqpoint{3.696000in}{3.696000in}}%
\pgfusepath{clip}%
\pgfsetbuttcap%
\pgfsetroundjoin%
\definecolor{currentfill}{rgb}{0.121569,0.466667,0.705882}%
\pgfsetfillcolor{currentfill}%
\pgfsetfillopacity{0.344104}%
\pgfsetlinewidth{1.003750pt}%
\definecolor{currentstroke}{rgb}{0.121569,0.466667,0.705882}%
\pgfsetstrokecolor{currentstroke}%
\pgfsetstrokeopacity{0.344104}%
\pgfsetdash{}{0pt}%
\pgfpathmoveto{\pgfqpoint{1.553624in}{3.301141in}}%
\pgfpathcurveto{\pgfqpoint{1.561860in}{3.301141in}}{\pgfqpoint{1.569760in}{3.304413in}}{\pgfqpoint{1.575584in}{3.310237in}}%
\pgfpathcurveto{\pgfqpoint{1.581408in}{3.316061in}}{\pgfqpoint{1.584680in}{3.323961in}}{\pgfqpoint{1.584680in}{3.332197in}}%
\pgfpathcurveto{\pgfqpoint{1.584680in}{3.340434in}}{\pgfqpoint{1.581408in}{3.348334in}}{\pgfqpoint{1.575584in}{3.354158in}}%
\pgfpathcurveto{\pgfqpoint{1.569760in}{3.359982in}}{\pgfqpoint{1.561860in}{3.363254in}}{\pgfqpoint{1.553624in}{3.363254in}}%
\pgfpathcurveto{\pgfqpoint{1.545388in}{3.363254in}}{\pgfqpoint{1.537488in}{3.359982in}}{\pgfqpoint{1.531664in}{3.354158in}}%
\pgfpathcurveto{\pgfqpoint{1.525840in}{3.348334in}}{\pgfqpoint{1.522567in}{3.340434in}}{\pgfqpoint{1.522567in}{3.332197in}}%
\pgfpathcurveto{\pgfqpoint{1.522567in}{3.323961in}}{\pgfqpoint{1.525840in}{3.316061in}}{\pgfqpoint{1.531664in}{3.310237in}}%
\pgfpathcurveto{\pgfqpoint{1.537488in}{3.304413in}}{\pgfqpoint{1.545388in}{3.301141in}}{\pgfqpoint{1.553624in}{3.301141in}}%
\pgfpathclose%
\pgfusepath{stroke,fill}%
\end{pgfscope}%
\begin{pgfscope}%
\pgfpathrectangle{\pgfqpoint{0.100000in}{0.212622in}}{\pgfqpoint{3.696000in}{3.696000in}}%
\pgfusepath{clip}%
\pgfsetbuttcap%
\pgfsetroundjoin%
\definecolor{currentfill}{rgb}{0.121569,0.466667,0.705882}%
\pgfsetfillcolor{currentfill}%
\pgfsetfillopacity{0.345792}%
\pgfsetlinewidth{1.003750pt}%
\definecolor{currentstroke}{rgb}{0.121569,0.466667,0.705882}%
\pgfsetstrokecolor{currentstroke}%
\pgfsetstrokeopacity{0.345792}%
\pgfsetdash{}{0pt}%
\pgfpathmoveto{\pgfqpoint{1.551575in}{3.296717in}}%
\pgfpathcurveto{\pgfqpoint{1.559811in}{3.296717in}}{\pgfqpoint{1.567711in}{3.299989in}}{\pgfqpoint{1.573535in}{3.305813in}}%
\pgfpathcurveto{\pgfqpoint{1.579359in}{3.311637in}}{\pgfqpoint{1.582631in}{3.319537in}}{\pgfqpoint{1.582631in}{3.327773in}}%
\pgfpathcurveto{\pgfqpoint{1.582631in}{3.336010in}}{\pgfqpoint{1.579359in}{3.343910in}}{\pgfqpoint{1.573535in}{3.349734in}}%
\pgfpathcurveto{\pgfqpoint{1.567711in}{3.355557in}}{\pgfqpoint{1.559811in}{3.358830in}}{\pgfqpoint{1.551575in}{3.358830in}}%
\pgfpathcurveto{\pgfqpoint{1.543338in}{3.358830in}}{\pgfqpoint{1.535438in}{3.355557in}}{\pgfqpoint{1.529614in}{3.349734in}}%
\pgfpathcurveto{\pgfqpoint{1.523790in}{3.343910in}}{\pgfqpoint{1.520518in}{3.336010in}}{\pgfqpoint{1.520518in}{3.327773in}}%
\pgfpathcurveto{\pgfqpoint{1.520518in}{3.319537in}}{\pgfqpoint{1.523790in}{3.311637in}}{\pgfqpoint{1.529614in}{3.305813in}}%
\pgfpathcurveto{\pgfqpoint{1.535438in}{3.299989in}}{\pgfqpoint{1.543338in}{3.296717in}}{\pgfqpoint{1.551575in}{3.296717in}}%
\pgfpathclose%
\pgfusepath{stroke,fill}%
\end{pgfscope}%
\begin{pgfscope}%
\pgfpathrectangle{\pgfqpoint{0.100000in}{0.212622in}}{\pgfqpoint{3.696000in}{3.696000in}}%
\pgfusepath{clip}%
\pgfsetbuttcap%
\pgfsetroundjoin%
\definecolor{currentfill}{rgb}{0.121569,0.466667,0.705882}%
\pgfsetfillcolor{currentfill}%
\pgfsetfillopacity{0.347003}%
\pgfsetlinewidth{1.003750pt}%
\definecolor{currentstroke}{rgb}{0.121569,0.466667,0.705882}%
\pgfsetstrokecolor{currentstroke}%
\pgfsetstrokeopacity{0.347003}%
\pgfsetdash{}{0pt}%
\pgfpathmoveto{\pgfqpoint{1.550069in}{3.293540in}}%
\pgfpathcurveto{\pgfqpoint{1.558305in}{3.293540in}}{\pgfqpoint{1.566205in}{3.296812in}}{\pgfqpoint{1.572029in}{3.302636in}}%
\pgfpathcurveto{\pgfqpoint{1.577853in}{3.308460in}}{\pgfqpoint{1.581125in}{3.316360in}}{\pgfqpoint{1.581125in}{3.324596in}}%
\pgfpathcurveto{\pgfqpoint{1.581125in}{3.332832in}}{\pgfqpoint{1.577853in}{3.340733in}}{\pgfqpoint{1.572029in}{3.346556in}}%
\pgfpathcurveto{\pgfqpoint{1.566205in}{3.352380in}}{\pgfqpoint{1.558305in}{3.355653in}}{\pgfqpoint{1.550069in}{3.355653in}}%
\pgfpathcurveto{\pgfqpoint{1.541832in}{3.355653in}}{\pgfqpoint{1.533932in}{3.352380in}}{\pgfqpoint{1.528109in}{3.346556in}}%
\pgfpathcurveto{\pgfqpoint{1.522285in}{3.340733in}}{\pgfqpoint{1.519012in}{3.332832in}}{\pgfqpoint{1.519012in}{3.324596in}}%
\pgfpathcurveto{\pgfqpoint{1.519012in}{3.316360in}}{\pgfqpoint{1.522285in}{3.308460in}}{\pgfqpoint{1.528109in}{3.302636in}}%
\pgfpathcurveto{\pgfqpoint{1.533932in}{3.296812in}}{\pgfqpoint{1.541832in}{3.293540in}}{\pgfqpoint{1.550069in}{3.293540in}}%
\pgfpathclose%
\pgfusepath{stroke,fill}%
\end{pgfscope}%
\begin{pgfscope}%
\pgfpathrectangle{\pgfqpoint{0.100000in}{0.212622in}}{\pgfqpoint{3.696000in}{3.696000in}}%
\pgfusepath{clip}%
\pgfsetbuttcap%
\pgfsetroundjoin%
\definecolor{currentfill}{rgb}{0.121569,0.466667,0.705882}%
\pgfsetfillcolor{currentfill}%
\pgfsetfillopacity{0.347467}%
\pgfsetlinewidth{1.003750pt}%
\definecolor{currentstroke}{rgb}{0.121569,0.466667,0.705882}%
\pgfsetstrokecolor{currentstroke}%
\pgfsetstrokeopacity{0.347467}%
\pgfsetdash{}{0pt}%
\pgfpathmoveto{\pgfqpoint{1.549533in}{3.292329in}}%
\pgfpathcurveto{\pgfqpoint{1.557769in}{3.292329in}}{\pgfqpoint{1.565669in}{3.295602in}}{\pgfqpoint{1.571493in}{3.301426in}}%
\pgfpathcurveto{\pgfqpoint{1.577317in}{3.307250in}}{\pgfqpoint{1.580589in}{3.315150in}}{\pgfqpoint{1.580589in}{3.323386in}}%
\pgfpathcurveto{\pgfqpoint{1.580589in}{3.331622in}}{\pgfqpoint{1.577317in}{3.339522in}}{\pgfqpoint{1.571493in}{3.345346in}}%
\pgfpathcurveto{\pgfqpoint{1.565669in}{3.351170in}}{\pgfqpoint{1.557769in}{3.354442in}}{\pgfqpoint{1.549533in}{3.354442in}}%
\pgfpathcurveto{\pgfqpoint{1.541297in}{3.354442in}}{\pgfqpoint{1.533396in}{3.351170in}}{\pgfqpoint{1.527573in}{3.345346in}}%
\pgfpathcurveto{\pgfqpoint{1.521749in}{3.339522in}}{\pgfqpoint{1.518476in}{3.331622in}}{\pgfqpoint{1.518476in}{3.323386in}}%
\pgfpathcurveto{\pgfqpoint{1.518476in}{3.315150in}}{\pgfqpoint{1.521749in}{3.307250in}}{\pgfqpoint{1.527573in}{3.301426in}}%
\pgfpathcurveto{\pgfqpoint{1.533396in}{3.295602in}}{\pgfqpoint{1.541297in}{3.292329in}}{\pgfqpoint{1.549533in}{3.292329in}}%
\pgfpathclose%
\pgfusepath{stroke,fill}%
\end{pgfscope}%
\begin{pgfscope}%
\pgfpathrectangle{\pgfqpoint{0.100000in}{0.212622in}}{\pgfqpoint{3.696000in}{3.696000in}}%
\pgfusepath{clip}%
\pgfsetbuttcap%
\pgfsetroundjoin%
\definecolor{currentfill}{rgb}{0.121569,0.466667,0.705882}%
\pgfsetfillcolor{currentfill}%
\pgfsetfillopacity{0.348285}%
\pgfsetlinewidth{1.003750pt}%
\definecolor{currentstroke}{rgb}{0.121569,0.466667,0.705882}%
\pgfsetstrokecolor{currentstroke}%
\pgfsetstrokeopacity{0.348285}%
\pgfsetdash{}{0pt}%
\pgfpathmoveto{\pgfqpoint{1.548372in}{3.290171in}}%
\pgfpathcurveto{\pgfqpoint{1.556608in}{3.290171in}}{\pgfqpoint{1.564508in}{3.293444in}}{\pgfqpoint{1.570332in}{3.299268in}}%
\pgfpathcurveto{\pgfqpoint{1.576156in}{3.305092in}}{\pgfqpoint{1.579428in}{3.312992in}}{\pgfqpoint{1.579428in}{3.321228in}}%
\pgfpathcurveto{\pgfqpoint{1.579428in}{3.329464in}}{\pgfqpoint{1.576156in}{3.337364in}}{\pgfqpoint{1.570332in}{3.343188in}}%
\pgfpathcurveto{\pgfqpoint{1.564508in}{3.349012in}}{\pgfqpoint{1.556608in}{3.352284in}}{\pgfqpoint{1.548372in}{3.352284in}}%
\pgfpathcurveto{\pgfqpoint{1.540135in}{3.352284in}}{\pgfqpoint{1.532235in}{3.349012in}}{\pgfqpoint{1.526411in}{3.343188in}}%
\pgfpathcurveto{\pgfqpoint{1.520587in}{3.337364in}}{\pgfqpoint{1.517315in}{3.329464in}}{\pgfqpoint{1.517315in}{3.321228in}}%
\pgfpathcurveto{\pgfqpoint{1.517315in}{3.312992in}}{\pgfqpoint{1.520587in}{3.305092in}}{\pgfqpoint{1.526411in}{3.299268in}}%
\pgfpathcurveto{\pgfqpoint{1.532235in}{3.293444in}}{\pgfqpoint{1.540135in}{3.290171in}}{\pgfqpoint{1.548372in}{3.290171in}}%
\pgfpathclose%
\pgfusepath{stroke,fill}%
\end{pgfscope}%
\begin{pgfscope}%
\pgfpathrectangle{\pgfqpoint{0.100000in}{0.212622in}}{\pgfqpoint{3.696000in}{3.696000in}}%
\pgfusepath{clip}%
\pgfsetbuttcap%
\pgfsetroundjoin%
\definecolor{currentfill}{rgb}{0.121569,0.466667,0.705882}%
\pgfsetfillcolor{currentfill}%
\pgfsetfillopacity{0.349587}%
\pgfsetlinewidth{1.003750pt}%
\definecolor{currentstroke}{rgb}{0.121569,0.466667,0.705882}%
\pgfsetstrokecolor{currentstroke}%
\pgfsetstrokeopacity{0.349587}%
\pgfsetdash{}{0pt}%
\pgfpathmoveto{\pgfqpoint{1.867265in}{3.169476in}}%
\pgfpathcurveto{\pgfqpoint{1.875501in}{3.169476in}}{\pgfqpoint{1.883401in}{3.172748in}}{\pgfqpoint{1.889225in}{3.178572in}}%
\pgfpathcurveto{\pgfqpoint{1.895049in}{3.184396in}}{\pgfqpoint{1.898321in}{3.192296in}}{\pgfqpoint{1.898321in}{3.200533in}}%
\pgfpathcurveto{\pgfqpoint{1.898321in}{3.208769in}}{\pgfqpoint{1.895049in}{3.216669in}}{\pgfqpoint{1.889225in}{3.222493in}}%
\pgfpathcurveto{\pgfqpoint{1.883401in}{3.228317in}}{\pgfqpoint{1.875501in}{3.231589in}}{\pgfqpoint{1.867265in}{3.231589in}}%
\pgfpathcurveto{\pgfqpoint{1.859028in}{3.231589in}}{\pgfqpoint{1.851128in}{3.228317in}}{\pgfqpoint{1.845304in}{3.222493in}}%
\pgfpathcurveto{\pgfqpoint{1.839481in}{3.216669in}}{\pgfqpoint{1.836208in}{3.208769in}}{\pgfqpoint{1.836208in}{3.200533in}}%
\pgfpathcurveto{\pgfqpoint{1.836208in}{3.192296in}}{\pgfqpoint{1.839481in}{3.184396in}}{\pgfqpoint{1.845304in}{3.178572in}}%
\pgfpathcurveto{\pgfqpoint{1.851128in}{3.172748in}}{\pgfqpoint{1.859028in}{3.169476in}}{\pgfqpoint{1.867265in}{3.169476in}}%
\pgfpathclose%
\pgfusepath{stroke,fill}%
\end{pgfscope}%
\begin{pgfscope}%
\pgfpathrectangle{\pgfqpoint{0.100000in}{0.212622in}}{\pgfqpoint{3.696000in}{3.696000in}}%
\pgfusepath{clip}%
\pgfsetbuttcap%
\pgfsetroundjoin%
\definecolor{currentfill}{rgb}{0.121569,0.466667,0.705882}%
\pgfsetfillcolor{currentfill}%
\pgfsetfillopacity{0.349823}%
\pgfsetlinewidth{1.003750pt}%
\definecolor{currentstroke}{rgb}{0.121569,0.466667,0.705882}%
\pgfsetstrokecolor{currentstroke}%
\pgfsetstrokeopacity{0.349823}%
\pgfsetdash{}{0pt}%
\pgfpathmoveto{\pgfqpoint{1.546802in}{3.286053in}}%
\pgfpathcurveto{\pgfqpoint{1.555038in}{3.286053in}}{\pgfqpoint{1.562938in}{3.289326in}}{\pgfqpoint{1.568762in}{3.295150in}}%
\pgfpathcurveto{\pgfqpoint{1.574586in}{3.300974in}}{\pgfqpoint{1.577858in}{3.308874in}}{\pgfqpoint{1.577858in}{3.317110in}}%
\pgfpathcurveto{\pgfqpoint{1.577858in}{3.325346in}}{\pgfqpoint{1.574586in}{3.333246in}}{\pgfqpoint{1.568762in}{3.339070in}}%
\pgfpathcurveto{\pgfqpoint{1.562938in}{3.344894in}}{\pgfqpoint{1.555038in}{3.348166in}}{\pgfqpoint{1.546802in}{3.348166in}}%
\pgfpathcurveto{\pgfqpoint{1.538565in}{3.348166in}}{\pgfqpoint{1.530665in}{3.344894in}}{\pgfqpoint{1.524841in}{3.339070in}}%
\pgfpathcurveto{\pgfqpoint{1.519018in}{3.333246in}}{\pgfqpoint{1.515745in}{3.325346in}}{\pgfqpoint{1.515745in}{3.317110in}}%
\pgfpathcurveto{\pgfqpoint{1.515745in}{3.308874in}}{\pgfqpoint{1.519018in}{3.300974in}}{\pgfqpoint{1.524841in}{3.295150in}}%
\pgfpathcurveto{\pgfqpoint{1.530665in}{3.289326in}}{\pgfqpoint{1.538565in}{3.286053in}}{\pgfqpoint{1.546802in}{3.286053in}}%
\pgfpathclose%
\pgfusepath{stroke,fill}%
\end{pgfscope}%
\begin{pgfscope}%
\pgfpathrectangle{\pgfqpoint{0.100000in}{0.212622in}}{\pgfqpoint{3.696000in}{3.696000in}}%
\pgfusepath{clip}%
\pgfsetbuttcap%
\pgfsetroundjoin%
\definecolor{currentfill}{rgb}{0.121569,0.466667,0.705882}%
\pgfsetfillcolor{currentfill}%
\pgfsetfillopacity{0.350763}%
\pgfsetlinewidth{1.003750pt}%
\definecolor{currentstroke}{rgb}{0.121569,0.466667,0.705882}%
\pgfsetstrokecolor{currentstroke}%
\pgfsetstrokeopacity{0.350763}%
\pgfsetdash{}{0pt}%
\pgfpathmoveto{\pgfqpoint{1.545604in}{3.283699in}}%
\pgfpathcurveto{\pgfqpoint{1.553841in}{3.283699in}}{\pgfqpoint{1.561741in}{3.286971in}}{\pgfqpoint{1.567565in}{3.292795in}}%
\pgfpathcurveto{\pgfqpoint{1.573388in}{3.298619in}}{\pgfqpoint{1.576661in}{3.306519in}}{\pgfqpoint{1.576661in}{3.314755in}}%
\pgfpathcurveto{\pgfqpoint{1.576661in}{3.322991in}}{\pgfqpoint{1.573388in}{3.330891in}}{\pgfqpoint{1.567565in}{3.336715in}}%
\pgfpathcurveto{\pgfqpoint{1.561741in}{3.342539in}}{\pgfqpoint{1.553841in}{3.345812in}}{\pgfqpoint{1.545604in}{3.345812in}}%
\pgfpathcurveto{\pgfqpoint{1.537368in}{3.345812in}}{\pgfqpoint{1.529468in}{3.342539in}}{\pgfqpoint{1.523644in}{3.336715in}}%
\pgfpathcurveto{\pgfqpoint{1.517820in}{3.330891in}}{\pgfqpoint{1.514548in}{3.322991in}}{\pgfqpoint{1.514548in}{3.314755in}}%
\pgfpathcurveto{\pgfqpoint{1.514548in}{3.306519in}}{\pgfqpoint{1.517820in}{3.298619in}}{\pgfqpoint{1.523644in}{3.292795in}}%
\pgfpathcurveto{\pgfqpoint{1.529468in}{3.286971in}}{\pgfqpoint{1.537368in}{3.283699in}}{\pgfqpoint{1.545604in}{3.283699in}}%
\pgfpathclose%
\pgfusepath{stroke,fill}%
\end{pgfscope}%
\begin{pgfscope}%
\pgfpathrectangle{\pgfqpoint{0.100000in}{0.212622in}}{\pgfqpoint{3.696000in}{3.696000in}}%
\pgfusepath{clip}%
\pgfsetbuttcap%
\pgfsetroundjoin%
\definecolor{currentfill}{rgb}{0.121569,0.466667,0.705882}%
\pgfsetfillcolor{currentfill}%
\pgfsetfillopacity{0.352465}%
\pgfsetlinewidth{1.003750pt}%
\definecolor{currentstroke}{rgb}{0.121569,0.466667,0.705882}%
\pgfsetstrokecolor{currentstroke}%
\pgfsetstrokeopacity{0.352465}%
\pgfsetdash{}{0pt}%
\pgfpathmoveto{\pgfqpoint{1.543428in}{3.279389in}}%
\pgfpathcurveto{\pgfqpoint{1.551665in}{3.279389in}}{\pgfqpoint{1.559565in}{3.282662in}}{\pgfqpoint{1.565389in}{3.288486in}}%
\pgfpathcurveto{\pgfqpoint{1.571213in}{3.294310in}}{\pgfqpoint{1.574485in}{3.302210in}}{\pgfqpoint{1.574485in}{3.310446in}}%
\pgfpathcurveto{\pgfqpoint{1.574485in}{3.318682in}}{\pgfqpoint{1.571213in}{3.326582in}}{\pgfqpoint{1.565389in}{3.332406in}}%
\pgfpathcurveto{\pgfqpoint{1.559565in}{3.338230in}}{\pgfqpoint{1.551665in}{3.341502in}}{\pgfqpoint{1.543428in}{3.341502in}}%
\pgfpathcurveto{\pgfqpoint{1.535192in}{3.341502in}}{\pgfqpoint{1.527292in}{3.338230in}}{\pgfqpoint{1.521468in}{3.332406in}}%
\pgfpathcurveto{\pgfqpoint{1.515644in}{3.326582in}}{\pgfqpoint{1.512372in}{3.318682in}}{\pgfqpoint{1.512372in}{3.310446in}}%
\pgfpathcurveto{\pgfqpoint{1.512372in}{3.302210in}}{\pgfqpoint{1.515644in}{3.294310in}}{\pgfqpoint{1.521468in}{3.288486in}}%
\pgfpathcurveto{\pgfqpoint{1.527292in}{3.282662in}}{\pgfqpoint{1.535192in}{3.279389in}}{\pgfqpoint{1.543428in}{3.279389in}}%
\pgfpathclose%
\pgfusepath{stroke,fill}%
\end{pgfscope}%
\begin{pgfscope}%
\pgfpathrectangle{\pgfqpoint{0.100000in}{0.212622in}}{\pgfqpoint{3.696000in}{3.696000in}}%
\pgfusepath{clip}%
\pgfsetbuttcap%
\pgfsetroundjoin%
\definecolor{currentfill}{rgb}{0.121569,0.466667,0.705882}%
\pgfsetfillcolor{currentfill}%
\pgfsetfillopacity{0.353099}%
\pgfsetlinewidth{1.003750pt}%
\definecolor{currentstroke}{rgb}{0.121569,0.466667,0.705882}%
\pgfsetstrokecolor{currentstroke}%
\pgfsetstrokeopacity{0.353099}%
\pgfsetdash{}{0pt}%
\pgfpathmoveto{\pgfqpoint{1.542854in}{3.277788in}}%
\pgfpathcurveto{\pgfqpoint{1.551090in}{3.277788in}}{\pgfqpoint{1.558990in}{3.281061in}}{\pgfqpoint{1.564814in}{3.286885in}}%
\pgfpathcurveto{\pgfqpoint{1.570638in}{3.292709in}}{\pgfqpoint{1.573910in}{3.300609in}}{\pgfqpoint{1.573910in}{3.308845in}}%
\pgfpathcurveto{\pgfqpoint{1.573910in}{3.317081in}}{\pgfqpoint{1.570638in}{3.324981in}}{\pgfqpoint{1.564814in}{3.330805in}}%
\pgfpathcurveto{\pgfqpoint{1.558990in}{3.336629in}}{\pgfqpoint{1.551090in}{3.339901in}}{\pgfqpoint{1.542854in}{3.339901in}}%
\pgfpathcurveto{\pgfqpoint{1.534618in}{3.339901in}}{\pgfqpoint{1.526718in}{3.336629in}}{\pgfqpoint{1.520894in}{3.330805in}}%
\pgfpathcurveto{\pgfqpoint{1.515070in}{3.324981in}}{\pgfqpoint{1.511797in}{3.317081in}}{\pgfqpoint{1.511797in}{3.308845in}}%
\pgfpathcurveto{\pgfqpoint{1.511797in}{3.300609in}}{\pgfqpoint{1.515070in}{3.292709in}}{\pgfqpoint{1.520894in}{3.286885in}}%
\pgfpathcurveto{\pgfqpoint{1.526718in}{3.281061in}}{\pgfqpoint{1.534618in}{3.277788in}}{\pgfqpoint{1.542854in}{3.277788in}}%
\pgfpathclose%
\pgfusepath{stroke,fill}%
\end{pgfscope}%
\begin{pgfscope}%
\pgfpathrectangle{\pgfqpoint{0.100000in}{0.212622in}}{\pgfqpoint{3.696000in}{3.696000in}}%
\pgfusepath{clip}%
\pgfsetbuttcap%
\pgfsetroundjoin%
\definecolor{currentfill}{rgb}{0.121569,0.466667,0.705882}%
\pgfsetfillcolor{currentfill}%
\pgfsetfillopacity{0.354198}%
\pgfsetlinewidth{1.003750pt}%
\definecolor{currentstroke}{rgb}{0.121569,0.466667,0.705882}%
\pgfsetstrokecolor{currentstroke}%
\pgfsetstrokeopacity{0.354198}%
\pgfsetdash{}{0pt}%
\pgfpathmoveto{\pgfqpoint{1.541379in}{3.274950in}}%
\pgfpathcurveto{\pgfqpoint{1.549615in}{3.274950in}}{\pgfqpoint{1.557515in}{3.278222in}}{\pgfqpoint{1.563339in}{3.284046in}}%
\pgfpathcurveto{\pgfqpoint{1.569163in}{3.289870in}}{\pgfqpoint{1.572435in}{3.297770in}}{\pgfqpoint{1.572435in}{3.306006in}}%
\pgfpathcurveto{\pgfqpoint{1.572435in}{3.314242in}}{\pgfqpoint{1.569163in}{3.322143in}}{\pgfqpoint{1.563339in}{3.327966in}}%
\pgfpathcurveto{\pgfqpoint{1.557515in}{3.333790in}}{\pgfqpoint{1.549615in}{3.337063in}}{\pgfqpoint{1.541379in}{3.337063in}}%
\pgfpathcurveto{\pgfqpoint{1.533143in}{3.337063in}}{\pgfqpoint{1.525243in}{3.333790in}}{\pgfqpoint{1.519419in}{3.327966in}}%
\pgfpathcurveto{\pgfqpoint{1.513595in}{3.322143in}}{\pgfqpoint{1.510322in}{3.314242in}}{\pgfqpoint{1.510322in}{3.306006in}}%
\pgfpathcurveto{\pgfqpoint{1.510322in}{3.297770in}}{\pgfqpoint{1.513595in}{3.289870in}}{\pgfqpoint{1.519419in}{3.284046in}}%
\pgfpathcurveto{\pgfqpoint{1.525243in}{3.278222in}}{\pgfqpoint{1.533143in}{3.274950in}}{\pgfqpoint{1.541379in}{3.274950in}}%
\pgfpathclose%
\pgfusepath{stroke,fill}%
\end{pgfscope}%
\begin{pgfscope}%
\pgfpathrectangle{\pgfqpoint{0.100000in}{0.212622in}}{\pgfqpoint{3.696000in}{3.696000in}}%
\pgfusepath{clip}%
\pgfsetbuttcap%
\pgfsetroundjoin%
\definecolor{currentfill}{rgb}{0.121569,0.466667,0.705882}%
\pgfsetfillcolor{currentfill}%
\pgfsetfillopacity{0.356213}%
\pgfsetlinewidth{1.003750pt}%
\definecolor{currentstroke}{rgb}{0.121569,0.466667,0.705882}%
\pgfsetstrokecolor{currentstroke}%
\pgfsetstrokeopacity{0.356213}%
\pgfsetdash{}{0pt}%
\pgfpathmoveto{\pgfqpoint{1.539001in}{3.269619in}}%
\pgfpathcurveto{\pgfqpoint{1.547238in}{3.269619in}}{\pgfqpoint{1.555138in}{3.272891in}}{\pgfqpoint{1.560962in}{3.278715in}}%
\pgfpathcurveto{\pgfqpoint{1.566786in}{3.284539in}}{\pgfqpoint{1.570058in}{3.292439in}}{\pgfqpoint{1.570058in}{3.300675in}}%
\pgfpathcurveto{\pgfqpoint{1.570058in}{3.308912in}}{\pgfqpoint{1.566786in}{3.316812in}}{\pgfqpoint{1.560962in}{3.322636in}}%
\pgfpathcurveto{\pgfqpoint{1.555138in}{3.328460in}}{\pgfqpoint{1.547238in}{3.331732in}}{\pgfqpoint{1.539001in}{3.331732in}}%
\pgfpathcurveto{\pgfqpoint{1.530765in}{3.331732in}}{\pgfqpoint{1.522865in}{3.328460in}}{\pgfqpoint{1.517041in}{3.322636in}}%
\pgfpathcurveto{\pgfqpoint{1.511217in}{3.316812in}}{\pgfqpoint{1.507945in}{3.308912in}}{\pgfqpoint{1.507945in}{3.300675in}}%
\pgfpathcurveto{\pgfqpoint{1.507945in}{3.292439in}}{\pgfqpoint{1.511217in}{3.284539in}}{\pgfqpoint{1.517041in}{3.278715in}}%
\pgfpathcurveto{\pgfqpoint{1.522865in}{3.272891in}}{\pgfqpoint{1.530765in}{3.269619in}}{\pgfqpoint{1.539001in}{3.269619in}}%
\pgfpathclose%
\pgfusepath{stroke,fill}%
\end{pgfscope}%
\begin{pgfscope}%
\pgfpathrectangle{\pgfqpoint{0.100000in}{0.212622in}}{\pgfqpoint{3.696000in}{3.696000in}}%
\pgfusepath{clip}%
\pgfsetbuttcap%
\pgfsetroundjoin%
\definecolor{currentfill}{rgb}{0.121569,0.466667,0.705882}%
\pgfsetfillcolor{currentfill}%
\pgfsetfillopacity{0.357702}%
\pgfsetlinewidth{1.003750pt}%
\definecolor{currentstroke}{rgb}{0.121569,0.466667,0.705882}%
\pgfsetstrokecolor{currentstroke}%
\pgfsetstrokeopacity{0.357702}%
\pgfsetdash{}{0pt}%
\pgfpathmoveto{\pgfqpoint{1.886286in}{3.141067in}}%
\pgfpathcurveto{\pgfqpoint{1.894522in}{3.141067in}}{\pgfqpoint{1.902422in}{3.144339in}}{\pgfqpoint{1.908246in}{3.150163in}}%
\pgfpathcurveto{\pgfqpoint{1.914070in}{3.155987in}}{\pgfqpoint{1.917342in}{3.163887in}}{\pgfqpoint{1.917342in}{3.172123in}}%
\pgfpathcurveto{\pgfqpoint{1.917342in}{3.180359in}}{\pgfqpoint{1.914070in}{3.188259in}}{\pgfqpoint{1.908246in}{3.194083in}}%
\pgfpathcurveto{\pgfqpoint{1.902422in}{3.199907in}}{\pgfqpoint{1.894522in}{3.203180in}}{\pgfqpoint{1.886286in}{3.203180in}}%
\pgfpathcurveto{\pgfqpoint{1.878049in}{3.203180in}}{\pgfqpoint{1.870149in}{3.199907in}}{\pgfqpoint{1.864325in}{3.194083in}}%
\pgfpathcurveto{\pgfqpoint{1.858501in}{3.188259in}}{\pgfqpoint{1.855229in}{3.180359in}}{\pgfqpoint{1.855229in}{3.172123in}}%
\pgfpathcurveto{\pgfqpoint{1.855229in}{3.163887in}}{\pgfqpoint{1.858501in}{3.155987in}}{\pgfqpoint{1.864325in}{3.150163in}}%
\pgfpathcurveto{\pgfqpoint{1.870149in}{3.144339in}}{\pgfqpoint{1.878049in}{3.141067in}}{\pgfqpoint{1.886286in}{3.141067in}}%
\pgfpathclose%
\pgfusepath{stroke,fill}%
\end{pgfscope}%
\begin{pgfscope}%
\pgfpathrectangle{\pgfqpoint{0.100000in}{0.212622in}}{\pgfqpoint{3.696000in}{3.696000in}}%
\pgfusepath{clip}%
\pgfsetbuttcap%
\pgfsetroundjoin%
\definecolor{currentfill}{rgb}{0.121569,0.466667,0.705882}%
\pgfsetfillcolor{currentfill}%
\pgfsetfillopacity{0.357730}%
\pgfsetlinewidth{1.003750pt}%
\definecolor{currentstroke}{rgb}{0.121569,0.466667,0.705882}%
\pgfsetstrokecolor{currentstroke}%
\pgfsetstrokeopacity{0.357730}%
\pgfsetdash{}{0pt}%
\pgfpathmoveto{\pgfqpoint{1.537148in}{3.265706in}}%
\pgfpathcurveto{\pgfqpoint{1.545384in}{3.265706in}}{\pgfqpoint{1.553284in}{3.268978in}}{\pgfqpoint{1.559108in}{3.274802in}}%
\pgfpathcurveto{\pgfqpoint{1.564932in}{3.280626in}}{\pgfqpoint{1.568204in}{3.288526in}}{\pgfqpoint{1.568204in}{3.296762in}}%
\pgfpathcurveto{\pgfqpoint{1.568204in}{3.304999in}}{\pgfqpoint{1.564932in}{3.312899in}}{\pgfqpoint{1.559108in}{3.318723in}}%
\pgfpathcurveto{\pgfqpoint{1.553284in}{3.324547in}}{\pgfqpoint{1.545384in}{3.327819in}}{\pgfqpoint{1.537148in}{3.327819in}}%
\pgfpathcurveto{\pgfqpoint{1.528911in}{3.327819in}}{\pgfqpoint{1.521011in}{3.324547in}}{\pgfqpoint{1.515188in}{3.318723in}}%
\pgfpathcurveto{\pgfqpoint{1.509364in}{3.312899in}}{\pgfqpoint{1.506091in}{3.304999in}}{\pgfqpoint{1.506091in}{3.296762in}}%
\pgfpathcurveto{\pgfqpoint{1.506091in}{3.288526in}}{\pgfqpoint{1.509364in}{3.280626in}}{\pgfqpoint{1.515188in}{3.274802in}}%
\pgfpathcurveto{\pgfqpoint{1.521011in}{3.268978in}}{\pgfqpoint{1.528911in}{3.265706in}}{\pgfqpoint{1.537148in}{3.265706in}}%
\pgfpathclose%
\pgfusepath{stroke,fill}%
\end{pgfscope}%
\begin{pgfscope}%
\pgfpathrectangle{\pgfqpoint{0.100000in}{0.212622in}}{\pgfqpoint{3.696000in}{3.696000in}}%
\pgfusepath{clip}%
\pgfsetbuttcap%
\pgfsetroundjoin%
\definecolor{currentfill}{rgb}{0.121569,0.466667,0.705882}%
\pgfsetfillcolor{currentfill}%
\pgfsetfillopacity{0.358175}%
\pgfsetlinewidth{1.003750pt}%
\definecolor{currentstroke}{rgb}{0.121569,0.466667,0.705882}%
\pgfsetstrokecolor{currentstroke}%
\pgfsetstrokeopacity{0.358175}%
\pgfsetdash{}{0pt}%
\pgfpathmoveto{\pgfqpoint{1.536627in}{3.264541in}}%
\pgfpathcurveto{\pgfqpoint{1.544863in}{3.264541in}}{\pgfqpoint{1.552763in}{3.267813in}}{\pgfqpoint{1.558587in}{3.273637in}}%
\pgfpathcurveto{\pgfqpoint{1.564411in}{3.279461in}}{\pgfqpoint{1.567683in}{3.287361in}}{\pgfqpoint{1.567683in}{3.295598in}}%
\pgfpathcurveto{\pgfqpoint{1.567683in}{3.303834in}}{\pgfqpoint{1.564411in}{3.311734in}}{\pgfqpoint{1.558587in}{3.317558in}}%
\pgfpathcurveto{\pgfqpoint{1.552763in}{3.323382in}}{\pgfqpoint{1.544863in}{3.326654in}}{\pgfqpoint{1.536627in}{3.326654in}}%
\pgfpathcurveto{\pgfqpoint{1.528391in}{3.326654in}}{\pgfqpoint{1.520491in}{3.323382in}}{\pgfqpoint{1.514667in}{3.317558in}}%
\pgfpathcurveto{\pgfqpoint{1.508843in}{3.311734in}}{\pgfqpoint{1.505570in}{3.303834in}}{\pgfqpoint{1.505570in}{3.295598in}}%
\pgfpathcurveto{\pgfqpoint{1.505570in}{3.287361in}}{\pgfqpoint{1.508843in}{3.279461in}}{\pgfqpoint{1.514667in}{3.273637in}}%
\pgfpathcurveto{\pgfqpoint{1.520491in}{3.267813in}}{\pgfqpoint{1.528391in}{3.264541in}}{\pgfqpoint{1.536627in}{3.264541in}}%
\pgfpathclose%
\pgfusepath{stroke,fill}%
\end{pgfscope}%
\begin{pgfscope}%
\pgfpathrectangle{\pgfqpoint{0.100000in}{0.212622in}}{\pgfqpoint{3.696000in}{3.696000in}}%
\pgfusepath{clip}%
\pgfsetbuttcap%
\pgfsetroundjoin%
\definecolor{currentfill}{rgb}{0.121569,0.466667,0.705882}%
\pgfsetfillcolor{currentfill}%
\pgfsetfillopacity{0.358993}%
\pgfsetlinewidth{1.003750pt}%
\definecolor{currentstroke}{rgb}{0.121569,0.466667,0.705882}%
\pgfsetstrokecolor{currentstroke}%
\pgfsetstrokeopacity{0.358993}%
\pgfsetdash{}{0pt}%
\pgfpathmoveto{\pgfqpoint{1.535715in}{3.262437in}}%
\pgfpathcurveto{\pgfqpoint{1.543951in}{3.262437in}}{\pgfqpoint{1.551851in}{3.265709in}}{\pgfqpoint{1.557675in}{3.271533in}}%
\pgfpathcurveto{\pgfqpoint{1.563499in}{3.277357in}}{\pgfqpoint{1.566772in}{3.285257in}}{\pgfqpoint{1.566772in}{3.293493in}}%
\pgfpathcurveto{\pgfqpoint{1.566772in}{3.301730in}}{\pgfqpoint{1.563499in}{3.309630in}}{\pgfqpoint{1.557675in}{3.315454in}}%
\pgfpathcurveto{\pgfqpoint{1.551851in}{3.321278in}}{\pgfqpoint{1.543951in}{3.324550in}}{\pgfqpoint{1.535715in}{3.324550in}}%
\pgfpathcurveto{\pgfqpoint{1.527479in}{3.324550in}}{\pgfqpoint{1.519579in}{3.321278in}}{\pgfqpoint{1.513755in}{3.315454in}}%
\pgfpathcurveto{\pgfqpoint{1.507931in}{3.309630in}}{\pgfqpoint{1.504659in}{3.301730in}}{\pgfqpoint{1.504659in}{3.293493in}}%
\pgfpathcurveto{\pgfqpoint{1.504659in}{3.285257in}}{\pgfqpoint{1.507931in}{3.277357in}}{\pgfqpoint{1.513755in}{3.271533in}}%
\pgfpathcurveto{\pgfqpoint{1.519579in}{3.265709in}}{\pgfqpoint{1.527479in}{3.262437in}}{\pgfqpoint{1.535715in}{3.262437in}}%
\pgfpathclose%
\pgfusepath{stroke,fill}%
\end{pgfscope}%
\begin{pgfscope}%
\pgfpathrectangle{\pgfqpoint{0.100000in}{0.212622in}}{\pgfqpoint{3.696000in}{3.696000in}}%
\pgfusepath{clip}%
\pgfsetbuttcap%
\pgfsetroundjoin%
\definecolor{currentfill}{rgb}{0.121569,0.466667,0.705882}%
\pgfsetfillcolor{currentfill}%
\pgfsetfillopacity{0.360475}%
\pgfsetlinewidth{1.003750pt}%
\definecolor{currentstroke}{rgb}{0.121569,0.466667,0.705882}%
\pgfsetstrokecolor{currentstroke}%
\pgfsetstrokeopacity{0.360475}%
\pgfsetdash{}{0pt}%
\pgfpathmoveto{\pgfqpoint{1.533965in}{3.258654in}}%
\pgfpathcurveto{\pgfqpoint{1.542201in}{3.258654in}}{\pgfqpoint{1.550101in}{3.261927in}}{\pgfqpoint{1.555925in}{3.267750in}}%
\pgfpathcurveto{\pgfqpoint{1.561749in}{3.273574in}}{\pgfqpoint{1.565021in}{3.281474in}}{\pgfqpoint{1.565021in}{3.289711in}}%
\pgfpathcurveto{\pgfqpoint{1.565021in}{3.297947in}}{\pgfqpoint{1.561749in}{3.305847in}}{\pgfqpoint{1.555925in}{3.311671in}}%
\pgfpathcurveto{\pgfqpoint{1.550101in}{3.317495in}}{\pgfqpoint{1.542201in}{3.320767in}}{\pgfqpoint{1.533965in}{3.320767in}}%
\pgfpathcurveto{\pgfqpoint{1.525729in}{3.320767in}}{\pgfqpoint{1.517829in}{3.317495in}}{\pgfqpoint{1.512005in}{3.311671in}}%
\pgfpathcurveto{\pgfqpoint{1.506181in}{3.305847in}}{\pgfqpoint{1.502908in}{3.297947in}}{\pgfqpoint{1.502908in}{3.289711in}}%
\pgfpathcurveto{\pgfqpoint{1.502908in}{3.281474in}}{\pgfqpoint{1.506181in}{3.273574in}}{\pgfqpoint{1.512005in}{3.267750in}}%
\pgfpathcurveto{\pgfqpoint{1.517829in}{3.261927in}}{\pgfqpoint{1.525729in}{3.258654in}}{\pgfqpoint{1.533965in}{3.258654in}}%
\pgfpathclose%
\pgfusepath{stroke,fill}%
\end{pgfscope}%
\begin{pgfscope}%
\pgfpathrectangle{\pgfqpoint{0.100000in}{0.212622in}}{\pgfqpoint{3.696000in}{3.696000in}}%
\pgfusepath{clip}%
\pgfsetbuttcap%
\pgfsetroundjoin%
\definecolor{currentfill}{rgb}{0.121569,0.466667,0.705882}%
\pgfsetfillcolor{currentfill}%
\pgfsetfillopacity{0.361102}%
\pgfsetlinewidth{1.003750pt}%
\definecolor{currentstroke}{rgb}{0.121569,0.466667,0.705882}%
\pgfsetstrokecolor{currentstroke}%
\pgfsetstrokeopacity{0.361102}%
\pgfsetdash{}{0pt}%
\pgfpathmoveto{\pgfqpoint{1.533318in}{3.257064in}}%
\pgfpathcurveto{\pgfqpoint{1.541555in}{3.257064in}}{\pgfqpoint{1.549455in}{3.260337in}}{\pgfqpoint{1.555279in}{3.266161in}}%
\pgfpathcurveto{\pgfqpoint{1.561103in}{3.271985in}}{\pgfqpoint{1.564375in}{3.279885in}}{\pgfqpoint{1.564375in}{3.288121in}}%
\pgfpathcurveto{\pgfqpoint{1.564375in}{3.296357in}}{\pgfqpoint{1.561103in}{3.304257in}}{\pgfqpoint{1.555279in}{3.310081in}}%
\pgfpathcurveto{\pgfqpoint{1.549455in}{3.315905in}}{\pgfqpoint{1.541555in}{3.319177in}}{\pgfqpoint{1.533318in}{3.319177in}}%
\pgfpathcurveto{\pgfqpoint{1.525082in}{3.319177in}}{\pgfqpoint{1.517182in}{3.315905in}}{\pgfqpoint{1.511358in}{3.310081in}}%
\pgfpathcurveto{\pgfqpoint{1.505534in}{3.304257in}}{\pgfqpoint{1.502262in}{3.296357in}}{\pgfqpoint{1.502262in}{3.288121in}}%
\pgfpathcurveto{\pgfqpoint{1.502262in}{3.279885in}}{\pgfqpoint{1.505534in}{3.271985in}}{\pgfqpoint{1.511358in}{3.266161in}}%
\pgfpathcurveto{\pgfqpoint{1.517182in}{3.260337in}}{\pgfqpoint{1.525082in}{3.257064in}}{\pgfqpoint{1.533318in}{3.257064in}}%
\pgfpathclose%
\pgfusepath{stroke,fill}%
\end{pgfscope}%
\begin{pgfscope}%
\pgfpathrectangle{\pgfqpoint{0.100000in}{0.212622in}}{\pgfqpoint{3.696000in}{3.696000in}}%
\pgfusepath{clip}%
\pgfsetbuttcap%
\pgfsetroundjoin%
\definecolor{currentfill}{rgb}{0.121569,0.466667,0.705882}%
\pgfsetfillcolor{currentfill}%
\pgfsetfillopacity{0.362222}%
\pgfsetlinewidth{1.003750pt}%
\definecolor{currentstroke}{rgb}{0.121569,0.466667,0.705882}%
\pgfsetstrokecolor{currentstroke}%
\pgfsetstrokeopacity{0.362222}%
\pgfsetdash{}{0pt}%
\pgfpathmoveto{\pgfqpoint{1.531961in}{3.254220in}}%
\pgfpathcurveto{\pgfqpoint{1.540197in}{3.254220in}}{\pgfqpoint{1.548097in}{3.257492in}}{\pgfqpoint{1.553921in}{3.263316in}}%
\pgfpathcurveto{\pgfqpoint{1.559745in}{3.269140in}}{\pgfqpoint{1.563017in}{3.277040in}}{\pgfqpoint{1.563017in}{3.285276in}}%
\pgfpathcurveto{\pgfqpoint{1.563017in}{3.293513in}}{\pgfqpoint{1.559745in}{3.301413in}}{\pgfqpoint{1.553921in}{3.307237in}}%
\pgfpathcurveto{\pgfqpoint{1.548097in}{3.313061in}}{\pgfqpoint{1.540197in}{3.316333in}}{\pgfqpoint{1.531961in}{3.316333in}}%
\pgfpathcurveto{\pgfqpoint{1.523725in}{3.316333in}}{\pgfqpoint{1.515825in}{3.313061in}}{\pgfqpoint{1.510001in}{3.307237in}}%
\pgfpathcurveto{\pgfqpoint{1.504177in}{3.301413in}}{\pgfqpoint{1.500904in}{3.293513in}}{\pgfqpoint{1.500904in}{3.285276in}}%
\pgfpathcurveto{\pgfqpoint{1.500904in}{3.277040in}}{\pgfqpoint{1.504177in}{3.269140in}}{\pgfqpoint{1.510001in}{3.263316in}}%
\pgfpathcurveto{\pgfqpoint{1.515825in}{3.257492in}}{\pgfqpoint{1.523725in}{3.254220in}}{\pgfqpoint{1.531961in}{3.254220in}}%
\pgfpathclose%
\pgfusepath{stroke,fill}%
\end{pgfscope}%
\begin{pgfscope}%
\pgfpathrectangle{\pgfqpoint{0.100000in}{0.212622in}}{\pgfqpoint{3.696000in}{3.696000in}}%
\pgfusepath{clip}%
\pgfsetbuttcap%
\pgfsetroundjoin%
\definecolor{currentfill}{rgb}{0.121569,0.466667,0.705882}%
\pgfsetfillcolor{currentfill}%
\pgfsetfillopacity{0.362731}%
\pgfsetlinewidth{1.003750pt}%
\definecolor{currentstroke}{rgb}{0.121569,0.466667,0.705882}%
\pgfsetstrokecolor{currentstroke}%
\pgfsetstrokeopacity{0.362731}%
\pgfsetdash{}{0pt}%
\pgfpathmoveto{\pgfqpoint{1.531384in}{3.252899in}}%
\pgfpathcurveto{\pgfqpoint{1.539621in}{3.252899in}}{\pgfqpoint{1.547521in}{3.256171in}}{\pgfqpoint{1.553345in}{3.261995in}}%
\pgfpathcurveto{\pgfqpoint{1.559169in}{3.267819in}}{\pgfqpoint{1.562441in}{3.275719in}}{\pgfqpoint{1.562441in}{3.283955in}}%
\pgfpathcurveto{\pgfqpoint{1.562441in}{3.292191in}}{\pgfqpoint{1.559169in}{3.300092in}}{\pgfqpoint{1.553345in}{3.305915in}}%
\pgfpathcurveto{\pgfqpoint{1.547521in}{3.311739in}}{\pgfqpoint{1.539621in}{3.315012in}}{\pgfqpoint{1.531384in}{3.315012in}}%
\pgfpathcurveto{\pgfqpoint{1.523148in}{3.315012in}}{\pgfqpoint{1.515248in}{3.311739in}}{\pgfqpoint{1.509424in}{3.305915in}}%
\pgfpathcurveto{\pgfqpoint{1.503600in}{3.300092in}}{\pgfqpoint{1.500328in}{3.292191in}}{\pgfqpoint{1.500328in}{3.283955in}}%
\pgfpathcurveto{\pgfqpoint{1.500328in}{3.275719in}}{\pgfqpoint{1.503600in}{3.267819in}}{\pgfqpoint{1.509424in}{3.261995in}}%
\pgfpathcurveto{\pgfqpoint{1.515248in}{3.256171in}}{\pgfqpoint{1.523148in}{3.252899in}}{\pgfqpoint{1.531384in}{3.252899in}}%
\pgfpathclose%
\pgfusepath{stroke,fill}%
\end{pgfscope}%
\begin{pgfscope}%
\pgfpathrectangle{\pgfqpoint{0.100000in}{0.212622in}}{\pgfqpoint{3.696000in}{3.696000in}}%
\pgfusepath{clip}%
\pgfsetbuttcap%
\pgfsetroundjoin%
\definecolor{currentfill}{rgb}{0.121569,0.466667,0.705882}%
\pgfsetfillcolor{currentfill}%
\pgfsetfillopacity{0.363656}%
\pgfsetlinewidth{1.003750pt}%
\definecolor{currentstroke}{rgb}{0.121569,0.466667,0.705882}%
\pgfsetstrokecolor{currentstroke}%
\pgfsetstrokeopacity{0.363656}%
\pgfsetdash{}{0pt}%
\pgfpathmoveto{\pgfqpoint{1.530361in}{3.250484in}}%
\pgfpathcurveto{\pgfqpoint{1.538597in}{3.250484in}}{\pgfqpoint{1.546497in}{3.253756in}}{\pgfqpoint{1.552321in}{3.259580in}}%
\pgfpathcurveto{\pgfqpoint{1.558145in}{3.265404in}}{\pgfqpoint{1.561417in}{3.273304in}}{\pgfqpoint{1.561417in}{3.281540in}}%
\pgfpathcurveto{\pgfqpoint{1.561417in}{3.289777in}}{\pgfqpoint{1.558145in}{3.297677in}}{\pgfqpoint{1.552321in}{3.303501in}}%
\pgfpathcurveto{\pgfqpoint{1.546497in}{3.309325in}}{\pgfqpoint{1.538597in}{3.312597in}}{\pgfqpoint{1.530361in}{3.312597in}}%
\pgfpathcurveto{\pgfqpoint{1.522124in}{3.312597in}}{\pgfqpoint{1.514224in}{3.309325in}}{\pgfqpoint{1.508400in}{3.303501in}}%
\pgfpathcurveto{\pgfqpoint{1.502576in}{3.297677in}}{\pgfqpoint{1.499304in}{3.289777in}}{\pgfqpoint{1.499304in}{3.281540in}}%
\pgfpathcurveto{\pgfqpoint{1.499304in}{3.273304in}}{\pgfqpoint{1.502576in}{3.265404in}}{\pgfqpoint{1.508400in}{3.259580in}}%
\pgfpathcurveto{\pgfqpoint{1.514224in}{3.253756in}}{\pgfqpoint{1.522124in}{3.250484in}}{\pgfqpoint{1.530361in}{3.250484in}}%
\pgfpathclose%
\pgfusepath{stroke,fill}%
\end{pgfscope}%
\begin{pgfscope}%
\pgfpathrectangle{\pgfqpoint{0.100000in}{0.212622in}}{\pgfqpoint{3.696000in}{3.696000in}}%
\pgfusepath{clip}%
\pgfsetbuttcap%
\pgfsetroundjoin%
\definecolor{currentfill}{rgb}{0.121569,0.466667,0.705882}%
\pgfsetfillcolor{currentfill}%
\pgfsetfillopacity{0.365332}%
\pgfsetlinewidth{1.003750pt}%
\definecolor{currentstroke}{rgb}{0.121569,0.466667,0.705882}%
\pgfsetstrokecolor{currentstroke}%
\pgfsetstrokeopacity{0.365332}%
\pgfsetdash{}{0pt}%
\pgfpathmoveto{\pgfqpoint{1.528296in}{3.246206in}}%
\pgfpathcurveto{\pgfqpoint{1.536532in}{3.246206in}}{\pgfqpoint{1.544432in}{3.249478in}}{\pgfqpoint{1.550256in}{3.255302in}}%
\pgfpathcurveto{\pgfqpoint{1.556080in}{3.261126in}}{\pgfqpoint{1.559352in}{3.269026in}}{\pgfqpoint{1.559352in}{3.277262in}}%
\pgfpathcurveto{\pgfqpoint{1.559352in}{3.285499in}}{\pgfqpoint{1.556080in}{3.293399in}}{\pgfqpoint{1.550256in}{3.299223in}}%
\pgfpathcurveto{\pgfqpoint{1.544432in}{3.305047in}}{\pgfqpoint{1.536532in}{3.308319in}}{\pgfqpoint{1.528296in}{3.308319in}}%
\pgfpathcurveto{\pgfqpoint{1.520059in}{3.308319in}}{\pgfqpoint{1.512159in}{3.305047in}}{\pgfqpoint{1.506335in}{3.299223in}}%
\pgfpathcurveto{\pgfqpoint{1.500512in}{3.293399in}}{\pgfqpoint{1.497239in}{3.285499in}}{\pgfqpoint{1.497239in}{3.277262in}}%
\pgfpathcurveto{\pgfqpoint{1.497239in}{3.269026in}}{\pgfqpoint{1.500512in}{3.261126in}}{\pgfqpoint{1.506335in}{3.255302in}}%
\pgfpathcurveto{\pgfqpoint{1.512159in}{3.249478in}}{\pgfqpoint{1.520059in}{3.246206in}}{\pgfqpoint{1.528296in}{3.246206in}}%
\pgfpathclose%
\pgfusepath{stroke,fill}%
\end{pgfscope}%
\begin{pgfscope}%
\pgfpathrectangle{\pgfqpoint{0.100000in}{0.212622in}}{\pgfqpoint{3.696000in}{3.696000in}}%
\pgfusepath{clip}%
\pgfsetbuttcap%
\pgfsetroundjoin%
\definecolor{currentfill}{rgb}{0.121569,0.466667,0.705882}%
\pgfsetfillcolor{currentfill}%
\pgfsetfillopacity{0.366026}%
\pgfsetlinewidth{1.003750pt}%
\definecolor{currentstroke}{rgb}{0.121569,0.466667,0.705882}%
\pgfsetstrokecolor{currentstroke}%
\pgfsetstrokeopacity{0.366026}%
\pgfsetdash{}{0pt}%
\pgfpathmoveto{\pgfqpoint{1.527549in}{3.244383in}}%
\pgfpathcurveto{\pgfqpoint{1.535786in}{3.244383in}}{\pgfqpoint{1.543686in}{3.247656in}}{\pgfqpoint{1.549510in}{3.253479in}}%
\pgfpathcurveto{\pgfqpoint{1.555333in}{3.259303in}}{\pgfqpoint{1.558606in}{3.267203in}}{\pgfqpoint{1.558606in}{3.275440in}}%
\pgfpathcurveto{\pgfqpoint{1.558606in}{3.283676in}}{\pgfqpoint{1.555333in}{3.291576in}}{\pgfqpoint{1.549510in}{3.297400in}}%
\pgfpathcurveto{\pgfqpoint{1.543686in}{3.303224in}}{\pgfqpoint{1.535786in}{3.306496in}}{\pgfqpoint{1.527549in}{3.306496in}}%
\pgfpathcurveto{\pgfqpoint{1.519313in}{3.306496in}}{\pgfqpoint{1.511413in}{3.303224in}}{\pgfqpoint{1.505589in}{3.297400in}}%
\pgfpathcurveto{\pgfqpoint{1.499765in}{3.291576in}}{\pgfqpoint{1.496493in}{3.283676in}}{\pgfqpoint{1.496493in}{3.275440in}}%
\pgfpathcurveto{\pgfqpoint{1.496493in}{3.267203in}}{\pgfqpoint{1.499765in}{3.259303in}}{\pgfqpoint{1.505589in}{3.253479in}}%
\pgfpathcurveto{\pgfqpoint{1.511413in}{3.247656in}}{\pgfqpoint{1.519313in}{3.244383in}}{\pgfqpoint{1.527549in}{3.244383in}}%
\pgfpathclose%
\pgfusepath{stroke,fill}%
\end{pgfscope}%
\begin{pgfscope}%
\pgfpathrectangle{\pgfqpoint{0.100000in}{0.212622in}}{\pgfqpoint{3.696000in}{3.696000in}}%
\pgfusepath{clip}%
\pgfsetbuttcap%
\pgfsetroundjoin%
\definecolor{currentfill}{rgb}{0.121569,0.466667,0.705882}%
\pgfsetfillcolor{currentfill}%
\pgfsetfillopacity{0.366943}%
\pgfsetlinewidth{1.003750pt}%
\definecolor{currentstroke}{rgb}{0.121569,0.466667,0.705882}%
\pgfsetstrokecolor{currentstroke}%
\pgfsetstrokeopacity{0.366943}%
\pgfsetdash{}{0pt}%
\pgfpathmoveto{\pgfqpoint{1.906081in}{3.110265in}}%
\pgfpathcurveto{\pgfqpoint{1.914317in}{3.110265in}}{\pgfqpoint{1.922217in}{3.113537in}}{\pgfqpoint{1.928041in}{3.119361in}}%
\pgfpathcurveto{\pgfqpoint{1.933865in}{3.125185in}}{\pgfqpoint{1.937137in}{3.133085in}}{\pgfqpoint{1.937137in}{3.141321in}}%
\pgfpathcurveto{\pgfqpoint{1.937137in}{3.149558in}}{\pgfqpoint{1.933865in}{3.157458in}}{\pgfqpoint{1.928041in}{3.163282in}}%
\pgfpathcurveto{\pgfqpoint{1.922217in}{3.169106in}}{\pgfqpoint{1.914317in}{3.172378in}}{\pgfqpoint{1.906081in}{3.172378in}}%
\pgfpathcurveto{\pgfqpoint{1.897845in}{3.172378in}}{\pgfqpoint{1.889945in}{3.169106in}}{\pgfqpoint{1.884121in}{3.163282in}}%
\pgfpathcurveto{\pgfqpoint{1.878297in}{3.157458in}}{\pgfqpoint{1.875024in}{3.149558in}}{\pgfqpoint{1.875024in}{3.141321in}}%
\pgfpathcurveto{\pgfqpoint{1.875024in}{3.133085in}}{\pgfqpoint{1.878297in}{3.125185in}}{\pgfqpoint{1.884121in}{3.119361in}}%
\pgfpathcurveto{\pgfqpoint{1.889945in}{3.113537in}}{\pgfqpoint{1.897845in}{3.110265in}}{\pgfqpoint{1.906081in}{3.110265in}}%
\pgfpathclose%
\pgfusepath{stroke,fill}%
\end{pgfscope}%
\begin{pgfscope}%
\pgfpathrectangle{\pgfqpoint{0.100000in}{0.212622in}}{\pgfqpoint{3.696000in}{3.696000in}}%
\pgfusepath{clip}%
\pgfsetbuttcap%
\pgfsetroundjoin%
\definecolor{currentfill}{rgb}{0.121569,0.466667,0.705882}%
\pgfsetfillcolor{currentfill}%
\pgfsetfillopacity{0.367283}%
\pgfsetlinewidth{1.003750pt}%
\definecolor{currentstroke}{rgb}{0.121569,0.466667,0.705882}%
\pgfsetstrokecolor{currentstroke}%
\pgfsetstrokeopacity{0.367283}%
\pgfsetdash{}{0pt}%
\pgfpathmoveto{\pgfqpoint{1.525989in}{3.241189in}}%
\pgfpathcurveto{\pgfqpoint{1.534225in}{3.241189in}}{\pgfqpoint{1.542125in}{3.244461in}}{\pgfqpoint{1.547949in}{3.250285in}}%
\pgfpathcurveto{\pgfqpoint{1.553773in}{3.256109in}}{\pgfqpoint{1.557045in}{3.264009in}}{\pgfqpoint{1.557045in}{3.272246in}}%
\pgfpathcurveto{\pgfqpoint{1.557045in}{3.280482in}}{\pgfqpoint{1.553773in}{3.288382in}}{\pgfqpoint{1.547949in}{3.294206in}}%
\pgfpathcurveto{\pgfqpoint{1.542125in}{3.300030in}}{\pgfqpoint{1.534225in}{3.303302in}}{\pgfqpoint{1.525989in}{3.303302in}}%
\pgfpathcurveto{\pgfqpoint{1.517752in}{3.303302in}}{\pgfqpoint{1.509852in}{3.300030in}}{\pgfqpoint{1.504028in}{3.294206in}}%
\pgfpathcurveto{\pgfqpoint{1.498205in}{3.288382in}}{\pgfqpoint{1.494932in}{3.280482in}}{\pgfqpoint{1.494932in}{3.272246in}}%
\pgfpathcurveto{\pgfqpoint{1.494932in}{3.264009in}}{\pgfqpoint{1.498205in}{3.256109in}}{\pgfqpoint{1.504028in}{3.250285in}}%
\pgfpathcurveto{\pgfqpoint{1.509852in}{3.244461in}}{\pgfqpoint{1.517752in}{3.241189in}}{\pgfqpoint{1.525989in}{3.241189in}}%
\pgfpathclose%
\pgfusepath{stroke,fill}%
\end{pgfscope}%
\begin{pgfscope}%
\pgfpathrectangle{\pgfqpoint{0.100000in}{0.212622in}}{\pgfqpoint{3.696000in}{3.696000in}}%
\pgfusepath{clip}%
\pgfsetbuttcap%
\pgfsetroundjoin%
\definecolor{currentfill}{rgb}{0.121569,0.466667,0.705882}%
\pgfsetfillcolor{currentfill}%
\pgfsetfillopacity{0.367724}%
\pgfsetlinewidth{1.003750pt}%
\definecolor{currentstroke}{rgb}{0.121569,0.466667,0.705882}%
\pgfsetstrokecolor{currentstroke}%
\pgfsetstrokeopacity{0.367724}%
\pgfsetdash{}{0pt}%
\pgfpathmoveto{\pgfqpoint{1.525475in}{3.240040in}}%
\pgfpathcurveto{\pgfqpoint{1.533711in}{3.240040in}}{\pgfqpoint{1.541611in}{3.243313in}}{\pgfqpoint{1.547435in}{3.249136in}}%
\pgfpathcurveto{\pgfqpoint{1.553259in}{3.254960in}}{\pgfqpoint{1.556531in}{3.262860in}}{\pgfqpoint{1.556531in}{3.271097in}}%
\pgfpathcurveto{\pgfqpoint{1.556531in}{3.279333in}}{\pgfqpoint{1.553259in}{3.287233in}}{\pgfqpoint{1.547435in}{3.293057in}}%
\pgfpathcurveto{\pgfqpoint{1.541611in}{3.298881in}}{\pgfqpoint{1.533711in}{3.302153in}}{\pgfqpoint{1.525475in}{3.302153in}}%
\pgfpathcurveto{\pgfqpoint{1.517238in}{3.302153in}}{\pgfqpoint{1.509338in}{3.298881in}}{\pgfqpoint{1.503514in}{3.293057in}}%
\pgfpathcurveto{\pgfqpoint{1.497690in}{3.287233in}}{\pgfqpoint{1.494418in}{3.279333in}}{\pgfqpoint{1.494418in}{3.271097in}}%
\pgfpathcurveto{\pgfqpoint{1.494418in}{3.262860in}}{\pgfqpoint{1.497690in}{3.254960in}}{\pgfqpoint{1.503514in}{3.249136in}}%
\pgfpathcurveto{\pgfqpoint{1.509338in}{3.243313in}}{\pgfqpoint{1.517238in}{3.240040in}}{\pgfqpoint{1.525475in}{3.240040in}}%
\pgfpathclose%
\pgfusepath{stroke,fill}%
\end{pgfscope}%
\begin{pgfscope}%
\pgfpathrectangle{\pgfqpoint{0.100000in}{0.212622in}}{\pgfqpoint{3.696000in}{3.696000in}}%
\pgfusepath{clip}%
\pgfsetbuttcap%
\pgfsetroundjoin%
\definecolor{currentfill}{rgb}{0.121569,0.466667,0.705882}%
\pgfsetfillcolor{currentfill}%
\pgfsetfillopacity{0.368535}%
\pgfsetlinewidth{1.003750pt}%
\definecolor{currentstroke}{rgb}{0.121569,0.466667,0.705882}%
\pgfsetstrokecolor{currentstroke}%
\pgfsetstrokeopacity{0.368535}%
\pgfsetdash{}{0pt}%
\pgfpathmoveto{\pgfqpoint{1.524539in}{3.237994in}}%
\pgfpathcurveto{\pgfqpoint{1.532775in}{3.237994in}}{\pgfqpoint{1.540675in}{3.241266in}}{\pgfqpoint{1.546499in}{3.247090in}}%
\pgfpathcurveto{\pgfqpoint{1.552323in}{3.252914in}}{\pgfqpoint{1.555595in}{3.260814in}}{\pgfqpoint{1.555595in}{3.269050in}}%
\pgfpathcurveto{\pgfqpoint{1.555595in}{3.277287in}}{\pgfqpoint{1.552323in}{3.285187in}}{\pgfqpoint{1.546499in}{3.291011in}}%
\pgfpathcurveto{\pgfqpoint{1.540675in}{3.296835in}}{\pgfqpoint{1.532775in}{3.300107in}}{\pgfqpoint{1.524539in}{3.300107in}}%
\pgfpathcurveto{\pgfqpoint{1.516303in}{3.300107in}}{\pgfqpoint{1.508403in}{3.296835in}}{\pgfqpoint{1.502579in}{3.291011in}}%
\pgfpathcurveto{\pgfqpoint{1.496755in}{3.285187in}}{\pgfqpoint{1.493482in}{3.277287in}}{\pgfqpoint{1.493482in}{3.269050in}}%
\pgfpathcurveto{\pgfqpoint{1.493482in}{3.260814in}}{\pgfqpoint{1.496755in}{3.252914in}}{\pgfqpoint{1.502579in}{3.247090in}}%
\pgfpathcurveto{\pgfqpoint{1.508403in}{3.241266in}}{\pgfqpoint{1.516303in}{3.237994in}}{\pgfqpoint{1.524539in}{3.237994in}}%
\pgfpathclose%
\pgfusepath{stroke,fill}%
\end{pgfscope}%
\begin{pgfscope}%
\pgfpathrectangle{\pgfqpoint{0.100000in}{0.212622in}}{\pgfqpoint{3.696000in}{3.696000in}}%
\pgfusepath{clip}%
\pgfsetbuttcap%
\pgfsetroundjoin%
\definecolor{currentfill}{rgb}{0.121569,0.466667,0.705882}%
\pgfsetfillcolor{currentfill}%
\pgfsetfillopacity{0.369972}%
\pgfsetlinewidth{1.003750pt}%
\definecolor{currentstroke}{rgb}{0.121569,0.466667,0.705882}%
\pgfsetstrokecolor{currentstroke}%
\pgfsetstrokeopacity{0.369972}%
\pgfsetdash{}{0pt}%
\pgfpathmoveto{\pgfqpoint{1.522670in}{3.234240in}}%
\pgfpathcurveto{\pgfqpoint{1.530906in}{3.234240in}}{\pgfqpoint{1.538806in}{3.237512in}}{\pgfqpoint{1.544630in}{3.243336in}}%
\pgfpathcurveto{\pgfqpoint{1.550454in}{3.249160in}}{\pgfqpoint{1.553726in}{3.257060in}}{\pgfqpoint{1.553726in}{3.265296in}}%
\pgfpathcurveto{\pgfqpoint{1.553726in}{3.273532in}}{\pgfqpoint{1.550454in}{3.281433in}}{\pgfqpoint{1.544630in}{3.287256in}}%
\pgfpathcurveto{\pgfqpoint{1.538806in}{3.293080in}}{\pgfqpoint{1.530906in}{3.296353in}}{\pgfqpoint{1.522670in}{3.296353in}}%
\pgfpathcurveto{\pgfqpoint{1.514433in}{3.296353in}}{\pgfqpoint{1.506533in}{3.293080in}}{\pgfqpoint{1.500709in}{3.287256in}}%
\pgfpathcurveto{\pgfqpoint{1.494885in}{3.281433in}}{\pgfqpoint{1.491613in}{3.273532in}}{\pgfqpoint{1.491613in}{3.265296in}}%
\pgfpathcurveto{\pgfqpoint{1.491613in}{3.257060in}}{\pgfqpoint{1.494885in}{3.249160in}}{\pgfqpoint{1.500709in}{3.243336in}}%
\pgfpathcurveto{\pgfqpoint{1.506533in}{3.237512in}}{\pgfqpoint{1.514433in}{3.234240in}}{\pgfqpoint{1.522670in}{3.234240in}}%
\pgfpathclose%
\pgfusepath{stroke,fill}%
\end{pgfscope}%
\begin{pgfscope}%
\pgfpathrectangle{\pgfqpoint{0.100000in}{0.212622in}}{\pgfqpoint{3.696000in}{3.696000in}}%
\pgfusepath{clip}%
\pgfsetbuttcap%
\pgfsetroundjoin%
\definecolor{currentfill}{rgb}{0.121569,0.466667,0.705882}%
\pgfsetfillcolor{currentfill}%
\pgfsetfillopacity{0.370428}%
\pgfsetlinewidth{1.003750pt}%
\definecolor{currentstroke}{rgb}{0.121569,0.466667,0.705882}%
\pgfsetstrokecolor{currentstroke}%
\pgfsetstrokeopacity{0.370428}%
\pgfsetdash{}{0pt}%
\pgfpathmoveto{\pgfqpoint{1.522159in}{3.233036in}}%
\pgfpathcurveto{\pgfqpoint{1.530395in}{3.233036in}}{\pgfqpoint{1.538295in}{3.236308in}}{\pgfqpoint{1.544119in}{3.242132in}}%
\pgfpathcurveto{\pgfqpoint{1.549943in}{3.247956in}}{\pgfqpoint{1.553215in}{3.255856in}}{\pgfqpoint{1.553215in}{3.264092in}}%
\pgfpathcurveto{\pgfqpoint{1.553215in}{3.272328in}}{\pgfqpoint{1.549943in}{3.280228in}}{\pgfqpoint{1.544119in}{3.286052in}}%
\pgfpathcurveto{\pgfqpoint{1.538295in}{3.291876in}}{\pgfqpoint{1.530395in}{3.295149in}}{\pgfqpoint{1.522159in}{3.295149in}}%
\pgfpathcurveto{\pgfqpoint{1.513922in}{3.295149in}}{\pgfqpoint{1.506022in}{3.291876in}}{\pgfqpoint{1.500199in}{3.286052in}}%
\pgfpathcurveto{\pgfqpoint{1.494375in}{3.280228in}}{\pgfqpoint{1.491102in}{3.272328in}}{\pgfqpoint{1.491102in}{3.264092in}}%
\pgfpathcurveto{\pgfqpoint{1.491102in}{3.255856in}}{\pgfqpoint{1.494375in}{3.247956in}}{\pgfqpoint{1.500199in}{3.242132in}}%
\pgfpathcurveto{\pgfqpoint{1.506022in}{3.236308in}}{\pgfqpoint{1.513922in}{3.233036in}}{\pgfqpoint{1.522159in}{3.233036in}}%
\pgfpathclose%
\pgfusepath{stroke,fill}%
\end{pgfscope}%
\begin{pgfscope}%
\pgfpathrectangle{\pgfqpoint{0.100000in}{0.212622in}}{\pgfqpoint{3.696000in}{3.696000in}}%
\pgfusepath{clip}%
\pgfsetbuttcap%
\pgfsetroundjoin%
\definecolor{currentfill}{rgb}{0.121569,0.466667,0.705882}%
\pgfsetfillcolor{currentfill}%
\pgfsetfillopacity{0.371263}%
\pgfsetlinewidth{1.003750pt}%
\definecolor{currentstroke}{rgb}{0.121569,0.466667,0.705882}%
\pgfsetstrokecolor{currentstroke}%
\pgfsetstrokeopacity{0.371263}%
\pgfsetdash{}{0pt}%
\pgfpathmoveto{\pgfqpoint{1.521115in}{3.230956in}}%
\pgfpathcurveto{\pgfqpoint{1.529351in}{3.230956in}}{\pgfqpoint{1.537251in}{3.234228in}}{\pgfqpoint{1.543075in}{3.240052in}}%
\pgfpathcurveto{\pgfqpoint{1.548899in}{3.245876in}}{\pgfqpoint{1.552171in}{3.253776in}}{\pgfqpoint{1.552171in}{3.262012in}}%
\pgfpathcurveto{\pgfqpoint{1.552171in}{3.270248in}}{\pgfqpoint{1.548899in}{3.278148in}}{\pgfqpoint{1.543075in}{3.283972in}}%
\pgfpathcurveto{\pgfqpoint{1.537251in}{3.289796in}}{\pgfqpoint{1.529351in}{3.293069in}}{\pgfqpoint{1.521115in}{3.293069in}}%
\pgfpathcurveto{\pgfqpoint{1.512878in}{3.293069in}}{\pgfqpoint{1.504978in}{3.289796in}}{\pgfqpoint{1.499154in}{3.283972in}}%
\pgfpathcurveto{\pgfqpoint{1.493330in}{3.278148in}}{\pgfqpoint{1.490058in}{3.270248in}}{\pgfqpoint{1.490058in}{3.262012in}}%
\pgfpathcurveto{\pgfqpoint{1.490058in}{3.253776in}}{\pgfqpoint{1.493330in}{3.245876in}}{\pgfqpoint{1.499154in}{3.240052in}}%
\pgfpathcurveto{\pgfqpoint{1.504978in}{3.234228in}}{\pgfqpoint{1.512878in}{3.230956in}}{\pgfqpoint{1.521115in}{3.230956in}}%
\pgfpathclose%
\pgfusepath{stroke,fill}%
\end{pgfscope}%
\begin{pgfscope}%
\pgfpathrectangle{\pgfqpoint{0.100000in}{0.212622in}}{\pgfqpoint{3.696000in}{3.696000in}}%
\pgfusepath{clip}%
\pgfsetbuttcap%
\pgfsetroundjoin%
\definecolor{currentfill}{rgb}{0.121569,0.466667,0.705882}%
\pgfsetfillcolor{currentfill}%
\pgfsetfillopacity{0.371497}%
\pgfsetlinewidth{1.003750pt}%
\definecolor{currentstroke}{rgb}{0.121569,0.466667,0.705882}%
\pgfsetstrokecolor{currentstroke}%
\pgfsetstrokeopacity{0.371497}%
\pgfsetdash{}{0pt}%
\pgfpathmoveto{\pgfqpoint{1.520824in}{3.230361in}}%
\pgfpathcurveto{\pgfqpoint{1.529060in}{3.230361in}}{\pgfqpoint{1.536960in}{3.233633in}}{\pgfqpoint{1.542784in}{3.239457in}}%
\pgfpathcurveto{\pgfqpoint{1.548608in}{3.245281in}}{\pgfqpoint{1.551880in}{3.253181in}}{\pgfqpoint{1.551880in}{3.261417in}}%
\pgfpathcurveto{\pgfqpoint{1.551880in}{3.269654in}}{\pgfqpoint{1.548608in}{3.277554in}}{\pgfqpoint{1.542784in}{3.283378in}}%
\pgfpathcurveto{\pgfqpoint{1.536960in}{3.289202in}}{\pgfqpoint{1.529060in}{3.292474in}}{\pgfqpoint{1.520824in}{3.292474in}}%
\pgfpathcurveto{\pgfqpoint{1.512587in}{3.292474in}}{\pgfqpoint{1.504687in}{3.289202in}}{\pgfqpoint{1.498863in}{3.283378in}}%
\pgfpathcurveto{\pgfqpoint{1.493040in}{3.277554in}}{\pgfqpoint{1.489767in}{3.269654in}}{\pgfqpoint{1.489767in}{3.261417in}}%
\pgfpathcurveto{\pgfqpoint{1.489767in}{3.253181in}}{\pgfqpoint{1.493040in}{3.245281in}}{\pgfqpoint{1.498863in}{3.239457in}}%
\pgfpathcurveto{\pgfqpoint{1.504687in}{3.233633in}}{\pgfqpoint{1.512587in}{3.230361in}}{\pgfqpoint{1.520824in}{3.230361in}}%
\pgfpathclose%
\pgfusepath{stroke,fill}%
\end{pgfscope}%
\begin{pgfscope}%
\pgfpathrectangle{\pgfqpoint{0.100000in}{0.212622in}}{\pgfqpoint{3.696000in}{3.696000in}}%
\pgfusepath{clip}%
\pgfsetbuttcap%
\pgfsetroundjoin%
\definecolor{currentfill}{rgb}{0.121569,0.466667,0.705882}%
\pgfsetfillcolor{currentfill}%
\pgfsetfillopacity{0.371923}%
\pgfsetlinewidth{1.003750pt}%
\definecolor{currentstroke}{rgb}{0.121569,0.466667,0.705882}%
\pgfsetstrokecolor{currentstroke}%
\pgfsetstrokeopacity{0.371923}%
\pgfsetdash{}{0pt}%
\pgfpathmoveto{\pgfqpoint{1.520307in}{3.229273in}}%
\pgfpathcurveto{\pgfqpoint{1.528544in}{3.229273in}}{\pgfqpoint{1.536444in}{3.232545in}}{\pgfqpoint{1.542268in}{3.238369in}}%
\pgfpathcurveto{\pgfqpoint{1.548092in}{3.244193in}}{\pgfqpoint{1.551364in}{3.252093in}}{\pgfqpoint{1.551364in}{3.260329in}}%
\pgfpathcurveto{\pgfqpoint{1.551364in}{3.268566in}}{\pgfqpoint{1.548092in}{3.276466in}}{\pgfqpoint{1.542268in}{3.282290in}}%
\pgfpathcurveto{\pgfqpoint{1.536444in}{3.288114in}}{\pgfqpoint{1.528544in}{3.291386in}}{\pgfqpoint{1.520307in}{3.291386in}}%
\pgfpathcurveto{\pgfqpoint{1.512071in}{3.291386in}}{\pgfqpoint{1.504171in}{3.288114in}}{\pgfqpoint{1.498347in}{3.282290in}}%
\pgfpathcurveto{\pgfqpoint{1.492523in}{3.276466in}}{\pgfqpoint{1.489251in}{3.268566in}}{\pgfqpoint{1.489251in}{3.260329in}}%
\pgfpathcurveto{\pgfqpoint{1.489251in}{3.252093in}}{\pgfqpoint{1.492523in}{3.244193in}}{\pgfqpoint{1.498347in}{3.238369in}}%
\pgfpathcurveto{\pgfqpoint{1.504171in}{3.232545in}}{\pgfqpoint{1.512071in}{3.229273in}}{\pgfqpoint{1.520307in}{3.229273in}}%
\pgfpathclose%
\pgfusepath{stroke,fill}%
\end{pgfscope}%
\begin{pgfscope}%
\pgfpathrectangle{\pgfqpoint{0.100000in}{0.212622in}}{\pgfqpoint{3.696000in}{3.696000in}}%
\pgfusepath{clip}%
\pgfsetbuttcap%
\pgfsetroundjoin%
\definecolor{currentfill}{rgb}{0.121569,0.466667,0.705882}%
\pgfsetfillcolor{currentfill}%
\pgfsetfillopacity{0.372692}%
\pgfsetlinewidth{1.003750pt}%
\definecolor{currentstroke}{rgb}{0.121569,0.466667,0.705882}%
\pgfsetstrokecolor{currentstroke}%
\pgfsetstrokeopacity{0.372692}%
\pgfsetdash{}{0pt}%
\pgfpathmoveto{\pgfqpoint{1.519310in}{3.227316in}}%
\pgfpathcurveto{\pgfqpoint{1.527546in}{3.227316in}}{\pgfqpoint{1.535446in}{3.230589in}}{\pgfqpoint{1.541270in}{3.236413in}}%
\pgfpathcurveto{\pgfqpoint{1.547094in}{3.242236in}}{\pgfqpoint{1.550366in}{3.250137in}}{\pgfqpoint{1.550366in}{3.258373in}}%
\pgfpathcurveto{\pgfqpoint{1.550366in}{3.266609in}}{\pgfqpoint{1.547094in}{3.274509in}}{\pgfqpoint{1.541270in}{3.280333in}}%
\pgfpathcurveto{\pgfqpoint{1.535446in}{3.286157in}}{\pgfqpoint{1.527546in}{3.289429in}}{\pgfqpoint{1.519310in}{3.289429in}}%
\pgfpathcurveto{\pgfqpoint{1.511074in}{3.289429in}}{\pgfqpoint{1.503174in}{3.286157in}}{\pgfqpoint{1.497350in}{3.280333in}}%
\pgfpathcurveto{\pgfqpoint{1.491526in}{3.274509in}}{\pgfqpoint{1.488253in}{3.266609in}}{\pgfqpoint{1.488253in}{3.258373in}}%
\pgfpathcurveto{\pgfqpoint{1.488253in}{3.250137in}}{\pgfqpoint{1.491526in}{3.242236in}}{\pgfqpoint{1.497350in}{3.236413in}}%
\pgfpathcurveto{\pgfqpoint{1.503174in}{3.230589in}}{\pgfqpoint{1.511074in}{3.227316in}}{\pgfqpoint{1.519310in}{3.227316in}}%
\pgfpathclose%
\pgfusepath{stroke,fill}%
\end{pgfscope}%
\begin{pgfscope}%
\pgfpathrectangle{\pgfqpoint{0.100000in}{0.212622in}}{\pgfqpoint{3.696000in}{3.696000in}}%
\pgfusepath{clip}%
\pgfsetbuttcap%
\pgfsetroundjoin%
\definecolor{currentfill}{rgb}{0.121569,0.466667,0.705882}%
\pgfsetfillcolor{currentfill}%
\pgfsetfillopacity{0.372873}%
\pgfsetlinewidth{1.003750pt}%
\definecolor{currentstroke}{rgb}{0.121569,0.466667,0.705882}%
\pgfsetstrokecolor{currentstroke}%
\pgfsetstrokeopacity{0.372873}%
\pgfsetdash{}{0pt}%
\pgfpathmoveto{\pgfqpoint{1.519086in}{3.226848in}}%
\pgfpathcurveto{\pgfqpoint{1.527323in}{3.226848in}}{\pgfqpoint{1.535223in}{3.230120in}}{\pgfqpoint{1.541047in}{3.235944in}}%
\pgfpathcurveto{\pgfqpoint{1.546870in}{3.241768in}}{\pgfqpoint{1.550143in}{3.249668in}}{\pgfqpoint{1.550143in}{3.257904in}}%
\pgfpathcurveto{\pgfqpoint{1.550143in}{3.266141in}}{\pgfqpoint{1.546870in}{3.274041in}}{\pgfqpoint{1.541047in}{3.279865in}}%
\pgfpathcurveto{\pgfqpoint{1.535223in}{3.285688in}}{\pgfqpoint{1.527323in}{3.288961in}}{\pgfqpoint{1.519086in}{3.288961in}}%
\pgfpathcurveto{\pgfqpoint{1.510850in}{3.288961in}}{\pgfqpoint{1.502950in}{3.285688in}}{\pgfqpoint{1.497126in}{3.279865in}}%
\pgfpathcurveto{\pgfqpoint{1.491302in}{3.274041in}}{\pgfqpoint{1.488030in}{3.266141in}}{\pgfqpoint{1.488030in}{3.257904in}}%
\pgfpathcurveto{\pgfqpoint{1.488030in}{3.249668in}}{\pgfqpoint{1.491302in}{3.241768in}}{\pgfqpoint{1.497126in}{3.235944in}}%
\pgfpathcurveto{\pgfqpoint{1.502950in}{3.230120in}}{\pgfqpoint{1.510850in}{3.226848in}}{\pgfqpoint{1.519086in}{3.226848in}}%
\pgfpathclose%
\pgfusepath{stroke,fill}%
\end{pgfscope}%
\begin{pgfscope}%
\pgfpathrectangle{\pgfqpoint{0.100000in}{0.212622in}}{\pgfqpoint{3.696000in}{3.696000in}}%
\pgfusepath{clip}%
\pgfsetbuttcap%
\pgfsetroundjoin%
\definecolor{currentfill}{rgb}{0.121569,0.466667,0.705882}%
\pgfsetfillcolor{currentfill}%
\pgfsetfillopacity{0.373194}%
\pgfsetlinewidth{1.003750pt}%
\definecolor{currentstroke}{rgb}{0.121569,0.466667,0.705882}%
\pgfsetstrokecolor{currentstroke}%
\pgfsetstrokeopacity{0.373194}%
\pgfsetdash{}{0pt}%
\pgfpathmoveto{\pgfqpoint{1.518647in}{3.225992in}}%
\pgfpathcurveto{\pgfqpoint{1.526883in}{3.225992in}}{\pgfqpoint{1.534784in}{3.229264in}}{\pgfqpoint{1.540607in}{3.235088in}}%
\pgfpathcurveto{\pgfqpoint{1.546431in}{3.240912in}}{\pgfqpoint{1.549704in}{3.248812in}}{\pgfqpoint{1.549704in}{3.257048in}}%
\pgfpathcurveto{\pgfqpoint{1.549704in}{3.265285in}}{\pgfqpoint{1.546431in}{3.273185in}}{\pgfqpoint{1.540607in}{3.279009in}}%
\pgfpathcurveto{\pgfqpoint{1.534784in}{3.284833in}}{\pgfqpoint{1.526883in}{3.288105in}}{\pgfqpoint{1.518647in}{3.288105in}}%
\pgfpathcurveto{\pgfqpoint{1.510411in}{3.288105in}}{\pgfqpoint{1.502511in}{3.284833in}}{\pgfqpoint{1.496687in}{3.279009in}}%
\pgfpathcurveto{\pgfqpoint{1.490863in}{3.273185in}}{\pgfqpoint{1.487591in}{3.265285in}}{\pgfqpoint{1.487591in}{3.257048in}}%
\pgfpathcurveto{\pgfqpoint{1.487591in}{3.248812in}}{\pgfqpoint{1.490863in}{3.240912in}}{\pgfqpoint{1.496687in}{3.235088in}}%
\pgfpathcurveto{\pgfqpoint{1.502511in}{3.229264in}}{\pgfqpoint{1.510411in}{3.225992in}}{\pgfqpoint{1.518647in}{3.225992in}}%
\pgfpathclose%
\pgfusepath{stroke,fill}%
\end{pgfscope}%
\begin{pgfscope}%
\pgfpathrectangle{\pgfqpoint{0.100000in}{0.212622in}}{\pgfqpoint{3.696000in}{3.696000in}}%
\pgfusepath{clip}%
\pgfsetbuttcap%
\pgfsetroundjoin%
\definecolor{currentfill}{rgb}{0.121569,0.466667,0.705882}%
\pgfsetfillcolor{currentfill}%
\pgfsetfillopacity{0.373781}%
\pgfsetlinewidth{1.003750pt}%
\definecolor{currentstroke}{rgb}{0.121569,0.466667,0.705882}%
\pgfsetstrokecolor{currentstroke}%
\pgfsetstrokeopacity{0.373781}%
\pgfsetdash{}{0pt}%
\pgfpathmoveto{\pgfqpoint{1.517917in}{3.224395in}}%
\pgfpathcurveto{\pgfqpoint{1.526153in}{3.224395in}}{\pgfqpoint{1.534053in}{3.227668in}}{\pgfqpoint{1.539877in}{3.233492in}}%
\pgfpathcurveto{\pgfqpoint{1.545701in}{3.239315in}}{\pgfqpoint{1.548973in}{3.247216in}}{\pgfqpoint{1.548973in}{3.255452in}}%
\pgfpathcurveto{\pgfqpoint{1.548973in}{3.263688in}}{\pgfqpoint{1.545701in}{3.271588in}}{\pgfqpoint{1.539877in}{3.277412in}}%
\pgfpathcurveto{\pgfqpoint{1.534053in}{3.283236in}}{\pgfqpoint{1.526153in}{3.286508in}}{\pgfqpoint{1.517917in}{3.286508in}}%
\pgfpathcurveto{\pgfqpoint{1.509681in}{3.286508in}}{\pgfqpoint{1.501781in}{3.283236in}}{\pgfqpoint{1.495957in}{3.277412in}}%
\pgfpathcurveto{\pgfqpoint{1.490133in}{3.271588in}}{\pgfqpoint{1.486860in}{3.263688in}}{\pgfqpoint{1.486860in}{3.255452in}}%
\pgfpathcurveto{\pgfqpoint{1.486860in}{3.247216in}}{\pgfqpoint{1.490133in}{3.239315in}}{\pgfqpoint{1.495957in}{3.233492in}}%
\pgfpathcurveto{\pgfqpoint{1.501781in}{3.227668in}}{\pgfqpoint{1.509681in}{3.224395in}}{\pgfqpoint{1.517917in}{3.224395in}}%
\pgfpathclose%
\pgfusepath{stroke,fill}%
\end{pgfscope}%
\begin{pgfscope}%
\pgfpathrectangle{\pgfqpoint{0.100000in}{0.212622in}}{\pgfqpoint{3.696000in}{3.696000in}}%
\pgfusepath{clip}%
\pgfsetbuttcap%
\pgfsetroundjoin%
\definecolor{currentfill}{rgb}{0.121569,0.466667,0.705882}%
\pgfsetfillcolor{currentfill}%
\pgfsetfillopacity{0.374832}%
\pgfsetlinewidth{1.003750pt}%
\definecolor{currentstroke}{rgb}{0.121569,0.466667,0.705882}%
\pgfsetstrokecolor{currentstroke}%
\pgfsetstrokeopacity{0.374832}%
\pgfsetdash{}{0pt}%
\pgfpathmoveto{\pgfqpoint{1.516491in}{3.221501in}}%
\pgfpathcurveto{\pgfqpoint{1.524727in}{3.221501in}}{\pgfqpoint{1.532627in}{3.224773in}}{\pgfqpoint{1.538451in}{3.230597in}}%
\pgfpathcurveto{\pgfqpoint{1.544275in}{3.236421in}}{\pgfqpoint{1.547547in}{3.244321in}}{\pgfqpoint{1.547547in}{3.252557in}}%
\pgfpathcurveto{\pgfqpoint{1.547547in}{3.260793in}}{\pgfqpoint{1.544275in}{3.268694in}}{\pgfqpoint{1.538451in}{3.274517in}}%
\pgfpathcurveto{\pgfqpoint{1.532627in}{3.280341in}}{\pgfqpoint{1.524727in}{3.283614in}}{\pgfqpoint{1.516491in}{3.283614in}}%
\pgfpathcurveto{\pgfqpoint{1.508255in}{3.283614in}}{\pgfqpoint{1.500355in}{3.280341in}}{\pgfqpoint{1.494531in}{3.274517in}}%
\pgfpathcurveto{\pgfqpoint{1.488707in}{3.268694in}}{\pgfqpoint{1.485434in}{3.260793in}}{\pgfqpoint{1.485434in}{3.252557in}}%
\pgfpathcurveto{\pgfqpoint{1.485434in}{3.244321in}}{\pgfqpoint{1.488707in}{3.236421in}}{\pgfqpoint{1.494531in}{3.230597in}}%
\pgfpathcurveto{\pgfqpoint{1.500355in}{3.224773in}}{\pgfqpoint{1.508255in}{3.221501in}}{\pgfqpoint{1.516491in}{3.221501in}}%
\pgfpathclose%
\pgfusepath{stroke,fill}%
\end{pgfscope}%
\begin{pgfscope}%
\pgfpathrectangle{\pgfqpoint{0.100000in}{0.212622in}}{\pgfqpoint{3.696000in}{3.696000in}}%
\pgfusepath{clip}%
\pgfsetbuttcap%
\pgfsetroundjoin%
\definecolor{currentfill}{rgb}{0.121569,0.466667,0.705882}%
\pgfsetfillcolor{currentfill}%
\pgfsetfillopacity{0.376722}%
\pgfsetlinewidth{1.003750pt}%
\definecolor{currentstroke}{rgb}{0.121569,0.466667,0.705882}%
\pgfsetstrokecolor{currentstroke}%
\pgfsetstrokeopacity{0.376722}%
\pgfsetdash{}{0pt}%
\pgfpathmoveto{\pgfqpoint{1.513790in}{3.216235in}}%
\pgfpathcurveto{\pgfqpoint{1.522026in}{3.216235in}}{\pgfqpoint{1.529926in}{3.219508in}}{\pgfqpoint{1.535750in}{3.225332in}}%
\pgfpathcurveto{\pgfqpoint{1.541574in}{3.231156in}}{\pgfqpoint{1.544846in}{3.239056in}}{\pgfqpoint{1.544846in}{3.247292in}}%
\pgfpathcurveto{\pgfqpoint{1.544846in}{3.255528in}}{\pgfqpoint{1.541574in}{3.263428in}}{\pgfqpoint{1.535750in}{3.269252in}}%
\pgfpathcurveto{\pgfqpoint{1.529926in}{3.275076in}}{\pgfqpoint{1.522026in}{3.278348in}}{\pgfqpoint{1.513790in}{3.278348in}}%
\pgfpathcurveto{\pgfqpoint{1.505553in}{3.278348in}}{\pgfqpoint{1.497653in}{3.275076in}}{\pgfqpoint{1.491829in}{3.269252in}}%
\pgfpathcurveto{\pgfqpoint{1.486005in}{3.263428in}}{\pgfqpoint{1.482733in}{3.255528in}}{\pgfqpoint{1.482733in}{3.247292in}}%
\pgfpathcurveto{\pgfqpoint{1.482733in}{3.239056in}}{\pgfqpoint{1.486005in}{3.231156in}}{\pgfqpoint{1.491829in}{3.225332in}}%
\pgfpathcurveto{\pgfqpoint{1.497653in}{3.219508in}}{\pgfqpoint{1.505553in}{3.216235in}}{\pgfqpoint{1.513790in}{3.216235in}}%
\pgfpathclose%
\pgfusepath{stroke,fill}%
\end{pgfscope}%
\begin{pgfscope}%
\pgfpathrectangle{\pgfqpoint{0.100000in}{0.212622in}}{\pgfqpoint{3.696000in}{3.696000in}}%
\pgfusepath{clip}%
\pgfsetbuttcap%
\pgfsetroundjoin%
\definecolor{currentfill}{rgb}{0.121569,0.466667,0.705882}%
\pgfsetfillcolor{currentfill}%
\pgfsetfillopacity{0.377061}%
\pgfsetlinewidth{1.003750pt}%
\definecolor{currentstroke}{rgb}{0.121569,0.466667,0.705882}%
\pgfsetstrokecolor{currentstroke}%
\pgfsetstrokeopacity{0.377061}%
\pgfsetdash{}{0pt}%
\pgfpathmoveto{\pgfqpoint{1.929560in}{3.072505in}}%
\pgfpathcurveto{\pgfqpoint{1.937796in}{3.072505in}}{\pgfqpoint{1.945696in}{3.075777in}}{\pgfqpoint{1.951520in}{3.081601in}}%
\pgfpathcurveto{\pgfqpoint{1.957344in}{3.087425in}}{\pgfqpoint{1.960616in}{3.095325in}}{\pgfqpoint{1.960616in}{3.103561in}}%
\pgfpathcurveto{\pgfqpoint{1.960616in}{3.111797in}}{\pgfqpoint{1.957344in}{3.119698in}}{\pgfqpoint{1.951520in}{3.125521in}}%
\pgfpathcurveto{\pgfqpoint{1.945696in}{3.131345in}}{\pgfqpoint{1.937796in}{3.134618in}}{\pgfqpoint{1.929560in}{3.134618in}}%
\pgfpathcurveto{\pgfqpoint{1.921324in}{3.134618in}}{\pgfqpoint{1.913424in}{3.131345in}}{\pgfqpoint{1.907600in}{3.125521in}}%
\pgfpathcurveto{\pgfqpoint{1.901776in}{3.119698in}}{\pgfqpoint{1.898503in}{3.111797in}}{\pgfqpoint{1.898503in}{3.103561in}}%
\pgfpathcurveto{\pgfqpoint{1.898503in}{3.095325in}}{\pgfqpoint{1.901776in}{3.087425in}}{\pgfqpoint{1.907600in}{3.081601in}}%
\pgfpathcurveto{\pgfqpoint{1.913424in}{3.075777in}}{\pgfqpoint{1.921324in}{3.072505in}}{\pgfqpoint{1.929560in}{3.072505in}}%
\pgfpathclose%
\pgfusepath{stroke,fill}%
\end{pgfscope}%
\begin{pgfscope}%
\pgfpathrectangle{\pgfqpoint{0.100000in}{0.212622in}}{\pgfqpoint{3.696000in}{3.696000in}}%
\pgfusepath{clip}%
\pgfsetbuttcap%
\pgfsetroundjoin%
\definecolor{currentfill}{rgb}{0.121569,0.466667,0.705882}%
\pgfsetfillcolor{currentfill}%
\pgfsetfillopacity{0.377886}%
\pgfsetlinewidth{1.003750pt}%
\definecolor{currentstroke}{rgb}{0.121569,0.466667,0.705882}%
\pgfsetstrokecolor{currentstroke}%
\pgfsetstrokeopacity{0.377886}%
\pgfsetdash{}{0pt}%
\pgfpathmoveto{\pgfqpoint{1.512362in}{3.212867in}}%
\pgfpathcurveto{\pgfqpoint{1.520598in}{3.212867in}}{\pgfqpoint{1.528498in}{3.216139in}}{\pgfqpoint{1.534322in}{3.221963in}}%
\pgfpathcurveto{\pgfqpoint{1.540146in}{3.227787in}}{\pgfqpoint{1.543418in}{3.235687in}}{\pgfqpoint{1.543418in}{3.243924in}}%
\pgfpathcurveto{\pgfqpoint{1.543418in}{3.252160in}}{\pgfqpoint{1.540146in}{3.260060in}}{\pgfqpoint{1.534322in}{3.265884in}}%
\pgfpathcurveto{\pgfqpoint{1.528498in}{3.271708in}}{\pgfqpoint{1.520598in}{3.274980in}}{\pgfqpoint{1.512362in}{3.274980in}}%
\pgfpathcurveto{\pgfqpoint{1.504125in}{3.274980in}}{\pgfqpoint{1.496225in}{3.271708in}}{\pgfqpoint{1.490401in}{3.265884in}}%
\pgfpathcurveto{\pgfqpoint{1.484577in}{3.260060in}}{\pgfqpoint{1.481305in}{3.252160in}}{\pgfqpoint{1.481305in}{3.243924in}}%
\pgfpathcurveto{\pgfqpoint{1.481305in}{3.235687in}}{\pgfqpoint{1.484577in}{3.227787in}}{\pgfqpoint{1.490401in}{3.221963in}}%
\pgfpathcurveto{\pgfqpoint{1.496225in}{3.216139in}}{\pgfqpoint{1.504125in}{3.212867in}}{\pgfqpoint{1.512362in}{3.212867in}}%
\pgfpathclose%
\pgfusepath{stroke,fill}%
\end{pgfscope}%
\begin{pgfscope}%
\pgfpathrectangle{\pgfqpoint{0.100000in}{0.212622in}}{\pgfqpoint{3.696000in}{3.696000in}}%
\pgfusepath{clip}%
\pgfsetbuttcap%
\pgfsetroundjoin%
\definecolor{currentfill}{rgb}{0.121569,0.466667,0.705882}%
\pgfsetfillcolor{currentfill}%
\pgfsetfillopacity{0.379983}%
\pgfsetlinewidth{1.003750pt}%
\definecolor{currentstroke}{rgb}{0.121569,0.466667,0.705882}%
\pgfsetstrokecolor{currentstroke}%
\pgfsetstrokeopacity{0.379983}%
\pgfsetdash{}{0pt}%
\pgfpathmoveto{\pgfqpoint{1.509263in}{3.207050in}}%
\pgfpathcurveto{\pgfqpoint{1.517500in}{3.207050in}}{\pgfqpoint{1.525400in}{3.210322in}}{\pgfqpoint{1.531224in}{3.216146in}}%
\pgfpathcurveto{\pgfqpoint{1.537048in}{3.221970in}}{\pgfqpoint{1.540320in}{3.229870in}}{\pgfqpoint{1.540320in}{3.238106in}}%
\pgfpathcurveto{\pgfqpoint{1.540320in}{3.246343in}}{\pgfqpoint{1.537048in}{3.254243in}}{\pgfqpoint{1.531224in}{3.260067in}}%
\pgfpathcurveto{\pgfqpoint{1.525400in}{3.265890in}}{\pgfqpoint{1.517500in}{3.269163in}}{\pgfqpoint{1.509263in}{3.269163in}}%
\pgfpathcurveto{\pgfqpoint{1.501027in}{3.269163in}}{\pgfqpoint{1.493127in}{3.265890in}}{\pgfqpoint{1.487303in}{3.260067in}}%
\pgfpathcurveto{\pgfqpoint{1.481479in}{3.254243in}}{\pgfqpoint{1.478207in}{3.246343in}}{\pgfqpoint{1.478207in}{3.238106in}}%
\pgfpathcurveto{\pgfqpoint{1.478207in}{3.229870in}}{\pgfqpoint{1.481479in}{3.221970in}}{\pgfqpoint{1.487303in}{3.216146in}}%
\pgfpathcurveto{\pgfqpoint{1.493127in}{3.210322in}}{\pgfqpoint{1.501027in}{3.207050in}}{\pgfqpoint{1.509263in}{3.207050in}}%
\pgfpathclose%
\pgfusepath{stroke,fill}%
\end{pgfscope}%
\begin{pgfscope}%
\pgfpathrectangle{\pgfqpoint{0.100000in}{0.212622in}}{\pgfqpoint{3.696000in}{3.696000in}}%
\pgfusepath{clip}%
\pgfsetbuttcap%
\pgfsetroundjoin%
\definecolor{currentfill}{rgb}{0.121569,0.466667,0.705882}%
\pgfsetfillcolor{currentfill}%
\pgfsetfillopacity{0.381171}%
\pgfsetlinewidth{1.003750pt}%
\definecolor{currentstroke}{rgb}{0.121569,0.466667,0.705882}%
\pgfsetstrokecolor{currentstroke}%
\pgfsetstrokeopacity{0.381171}%
\pgfsetdash{}{0pt}%
\pgfpathmoveto{\pgfqpoint{1.507665in}{3.203552in}}%
\pgfpathcurveto{\pgfqpoint{1.515901in}{3.203552in}}{\pgfqpoint{1.523802in}{3.206824in}}{\pgfqpoint{1.529625in}{3.212648in}}%
\pgfpathcurveto{\pgfqpoint{1.535449in}{3.218472in}}{\pgfqpoint{1.538722in}{3.226372in}}{\pgfqpoint{1.538722in}{3.234608in}}%
\pgfpathcurveto{\pgfqpoint{1.538722in}{3.242845in}}{\pgfqpoint{1.535449in}{3.250745in}}{\pgfqpoint{1.529625in}{3.256569in}}%
\pgfpathcurveto{\pgfqpoint{1.523802in}{3.262393in}}{\pgfqpoint{1.515901in}{3.265665in}}{\pgfqpoint{1.507665in}{3.265665in}}%
\pgfpathcurveto{\pgfqpoint{1.499429in}{3.265665in}}{\pgfqpoint{1.491529in}{3.262393in}}{\pgfqpoint{1.485705in}{3.256569in}}%
\pgfpathcurveto{\pgfqpoint{1.479881in}{3.250745in}}{\pgfqpoint{1.476609in}{3.242845in}}{\pgfqpoint{1.476609in}{3.234608in}}%
\pgfpathcurveto{\pgfqpoint{1.476609in}{3.226372in}}{\pgfqpoint{1.479881in}{3.218472in}}{\pgfqpoint{1.485705in}{3.212648in}}%
\pgfpathcurveto{\pgfqpoint{1.491529in}{3.206824in}}{\pgfqpoint{1.499429in}{3.203552in}}{\pgfqpoint{1.507665in}{3.203552in}}%
\pgfpathclose%
\pgfusepath{stroke,fill}%
\end{pgfscope}%
\begin{pgfscope}%
\pgfpathrectangle{\pgfqpoint{0.100000in}{0.212622in}}{\pgfqpoint{3.696000in}{3.696000in}}%
\pgfusepath{clip}%
\pgfsetbuttcap%
\pgfsetroundjoin%
\definecolor{currentfill}{rgb}{0.121569,0.466667,0.705882}%
\pgfsetfillcolor{currentfill}%
\pgfsetfillopacity{0.383164}%
\pgfsetlinewidth{1.003750pt}%
\definecolor{currentstroke}{rgb}{0.121569,0.466667,0.705882}%
\pgfsetstrokecolor{currentstroke}%
\pgfsetstrokeopacity{0.383164}%
\pgfsetdash{}{0pt}%
\pgfpathmoveto{\pgfqpoint{1.941201in}{3.052624in}}%
\pgfpathcurveto{\pgfqpoint{1.949437in}{3.052624in}}{\pgfqpoint{1.957337in}{3.055897in}}{\pgfqpoint{1.963161in}{3.061720in}}%
\pgfpathcurveto{\pgfqpoint{1.968985in}{3.067544in}}{\pgfqpoint{1.972257in}{3.075444in}}{\pgfqpoint{1.972257in}{3.083681in}}%
\pgfpathcurveto{\pgfqpoint{1.972257in}{3.091917in}}{\pgfqpoint{1.968985in}{3.099817in}}{\pgfqpoint{1.963161in}{3.105641in}}%
\pgfpathcurveto{\pgfqpoint{1.957337in}{3.111465in}}{\pgfqpoint{1.949437in}{3.114737in}}{\pgfqpoint{1.941201in}{3.114737in}}%
\pgfpathcurveto{\pgfqpoint{1.932965in}{3.114737in}}{\pgfqpoint{1.925065in}{3.111465in}}{\pgfqpoint{1.919241in}{3.105641in}}%
\pgfpathcurveto{\pgfqpoint{1.913417in}{3.099817in}}{\pgfqpoint{1.910144in}{3.091917in}}{\pgfqpoint{1.910144in}{3.083681in}}%
\pgfpathcurveto{\pgfqpoint{1.910144in}{3.075444in}}{\pgfqpoint{1.913417in}{3.067544in}}{\pgfqpoint{1.919241in}{3.061720in}}%
\pgfpathcurveto{\pgfqpoint{1.925065in}{3.055897in}}{\pgfqpoint{1.932965in}{3.052624in}}{\pgfqpoint{1.941201in}{3.052624in}}%
\pgfpathclose%
\pgfusepath{stroke,fill}%
\end{pgfscope}%
\begin{pgfscope}%
\pgfpathrectangle{\pgfqpoint{0.100000in}{0.212622in}}{\pgfqpoint{3.696000in}{3.696000in}}%
\pgfusepath{clip}%
\pgfsetbuttcap%
\pgfsetroundjoin%
\definecolor{currentfill}{rgb}{0.121569,0.466667,0.705882}%
\pgfsetfillcolor{currentfill}%
\pgfsetfillopacity{0.383334}%
\pgfsetlinewidth{1.003750pt}%
\definecolor{currentstroke}{rgb}{0.121569,0.466667,0.705882}%
\pgfsetstrokecolor{currentstroke}%
\pgfsetstrokeopacity{0.383334}%
\pgfsetdash{}{0pt}%
\pgfpathmoveto{\pgfqpoint{1.504436in}{3.197462in}}%
\pgfpathcurveto{\pgfqpoint{1.512672in}{3.197462in}}{\pgfqpoint{1.520572in}{3.200734in}}{\pgfqpoint{1.526396in}{3.206558in}}%
\pgfpathcurveto{\pgfqpoint{1.532220in}{3.212382in}}{\pgfqpoint{1.535492in}{3.220282in}}{\pgfqpoint{1.535492in}{3.228518in}}%
\pgfpathcurveto{\pgfqpoint{1.535492in}{3.236755in}}{\pgfqpoint{1.532220in}{3.244655in}}{\pgfqpoint{1.526396in}{3.250479in}}%
\pgfpathcurveto{\pgfqpoint{1.520572in}{3.256303in}}{\pgfqpoint{1.512672in}{3.259575in}}{\pgfqpoint{1.504436in}{3.259575in}}%
\pgfpathcurveto{\pgfqpoint{1.496199in}{3.259575in}}{\pgfqpoint{1.488299in}{3.256303in}}{\pgfqpoint{1.482475in}{3.250479in}}%
\pgfpathcurveto{\pgfqpoint{1.476651in}{3.244655in}}{\pgfqpoint{1.473379in}{3.236755in}}{\pgfqpoint{1.473379in}{3.228518in}}%
\pgfpathcurveto{\pgfqpoint{1.473379in}{3.220282in}}{\pgfqpoint{1.476651in}{3.212382in}}{\pgfqpoint{1.482475in}{3.206558in}}%
\pgfpathcurveto{\pgfqpoint{1.488299in}{3.200734in}}{\pgfqpoint{1.496199in}{3.197462in}}{\pgfqpoint{1.504436in}{3.197462in}}%
\pgfpathclose%
\pgfusepath{stroke,fill}%
\end{pgfscope}%
\begin{pgfscope}%
\pgfpathrectangle{\pgfqpoint{0.100000in}{0.212622in}}{\pgfqpoint{3.696000in}{3.696000in}}%
\pgfusepath{clip}%
\pgfsetbuttcap%
\pgfsetroundjoin%
\definecolor{currentfill}{rgb}{0.121569,0.466667,0.705882}%
\pgfsetfillcolor{currentfill}%
\pgfsetfillopacity{0.384576}%
\pgfsetlinewidth{1.003750pt}%
\definecolor{currentstroke}{rgb}{0.121569,0.466667,0.705882}%
\pgfsetstrokecolor{currentstroke}%
\pgfsetstrokeopacity{0.384576}%
\pgfsetdash{}{0pt}%
\pgfpathmoveto{\pgfqpoint{1.502628in}{3.193930in}}%
\pgfpathcurveto{\pgfqpoint{1.510865in}{3.193930in}}{\pgfqpoint{1.518765in}{3.197202in}}{\pgfqpoint{1.524589in}{3.203026in}}%
\pgfpathcurveto{\pgfqpoint{1.530413in}{3.208850in}}{\pgfqpoint{1.533685in}{3.216750in}}{\pgfqpoint{1.533685in}{3.224986in}}%
\pgfpathcurveto{\pgfqpoint{1.533685in}{3.233223in}}{\pgfqpoint{1.530413in}{3.241123in}}{\pgfqpoint{1.524589in}{3.246947in}}%
\pgfpathcurveto{\pgfqpoint{1.518765in}{3.252770in}}{\pgfqpoint{1.510865in}{3.256043in}}{\pgfqpoint{1.502628in}{3.256043in}}%
\pgfpathcurveto{\pgfqpoint{1.494392in}{3.256043in}}{\pgfqpoint{1.486492in}{3.252770in}}{\pgfqpoint{1.480668in}{3.246947in}}%
\pgfpathcurveto{\pgfqpoint{1.474844in}{3.241123in}}{\pgfqpoint{1.471572in}{3.233223in}}{\pgfqpoint{1.471572in}{3.224986in}}%
\pgfpathcurveto{\pgfqpoint{1.471572in}{3.216750in}}{\pgfqpoint{1.474844in}{3.208850in}}{\pgfqpoint{1.480668in}{3.203026in}}%
\pgfpathcurveto{\pgfqpoint{1.486492in}{3.197202in}}{\pgfqpoint{1.494392in}{3.193930in}}{\pgfqpoint{1.502628in}{3.193930in}}%
\pgfpathclose%
\pgfusepath{stroke,fill}%
\end{pgfscope}%
\begin{pgfscope}%
\pgfpathrectangle{\pgfqpoint{0.100000in}{0.212622in}}{\pgfqpoint{3.696000in}{3.696000in}}%
\pgfusepath{clip}%
\pgfsetbuttcap%
\pgfsetroundjoin%
\definecolor{currentfill}{rgb}{0.121569,0.466667,0.705882}%
\pgfsetfillcolor{currentfill}%
\pgfsetfillopacity{0.385148}%
\pgfsetlinewidth{1.003750pt}%
\definecolor{currentstroke}{rgb}{0.121569,0.466667,0.705882}%
\pgfsetstrokecolor{currentstroke}%
\pgfsetstrokeopacity{0.385148}%
\pgfsetdash{}{0pt}%
\pgfpathmoveto{\pgfqpoint{1.502194in}{3.192267in}}%
\pgfpathcurveto{\pgfqpoint{1.510430in}{3.192267in}}{\pgfqpoint{1.518330in}{3.195540in}}{\pgfqpoint{1.524154in}{3.201364in}}%
\pgfpathcurveto{\pgfqpoint{1.529978in}{3.207188in}}{\pgfqpoint{1.533251in}{3.215088in}}{\pgfqpoint{1.533251in}{3.223324in}}%
\pgfpathcurveto{\pgfqpoint{1.533251in}{3.231560in}}{\pgfqpoint{1.529978in}{3.239460in}}{\pgfqpoint{1.524154in}{3.245284in}}%
\pgfpathcurveto{\pgfqpoint{1.518330in}{3.251108in}}{\pgfqpoint{1.510430in}{3.254380in}}{\pgfqpoint{1.502194in}{3.254380in}}%
\pgfpathcurveto{\pgfqpoint{1.493958in}{3.254380in}}{\pgfqpoint{1.486058in}{3.251108in}}{\pgfqpoint{1.480234in}{3.245284in}}%
\pgfpathcurveto{\pgfqpoint{1.474410in}{3.239460in}}{\pgfqpoint{1.471138in}{3.231560in}}{\pgfqpoint{1.471138in}{3.223324in}}%
\pgfpathcurveto{\pgfqpoint{1.471138in}{3.215088in}}{\pgfqpoint{1.474410in}{3.207188in}}{\pgfqpoint{1.480234in}{3.201364in}}%
\pgfpathcurveto{\pgfqpoint{1.486058in}{3.195540in}}{\pgfqpoint{1.493958in}{3.192267in}}{\pgfqpoint{1.502194in}{3.192267in}}%
\pgfpathclose%
\pgfusepath{stroke,fill}%
\end{pgfscope}%
\begin{pgfscope}%
\pgfpathrectangle{\pgfqpoint{0.100000in}{0.212622in}}{\pgfqpoint{3.696000in}{3.696000in}}%
\pgfusepath{clip}%
\pgfsetbuttcap%
\pgfsetroundjoin%
\definecolor{currentfill}{rgb}{0.121569,0.466667,0.705882}%
\pgfsetfillcolor{currentfill}%
\pgfsetfillopacity{0.386162}%
\pgfsetlinewidth{1.003750pt}%
\definecolor{currentstroke}{rgb}{0.121569,0.466667,0.705882}%
\pgfsetstrokecolor{currentstroke}%
\pgfsetstrokeopacity{0.386162}%
\pgfsetdash{}{0pt}%
\pgfpathmoveto{\pgfqpoint{1.500722in}{3.189589in}}%
\pgfpathcurveto{\pgfqpoint{1.508958in}{3.189589in}}{\pgfqpoint{1.516858in}{3.192861in}}{\pgfqpoint{1.522682in}{3.198685in}}%
\pgfpathcurveto{\pgfqpoint{1.528506in}{3.204509in}}{\pgfqpoint{1.531778in}{3.212409in}}{\pgfqpoint{1.531778in}{3.220645in}}%
\pgfpathcurveto{\pgfqpoint{1.531778in}{3.228881in}}{\pgfqpoint{1.528506in}{3.236781in}}{\pgfqpoint{1.522682in}{3.242605in}}%
\pgfpathcurveto{\pgfqpoint{1.516858in}{3.248429in}}{\pgfqpoint{1.508958in}{3.251702in}}{\pgfqpoint{1.500722in}{3.251702in}}%
\pgfpathcurveto{\pgfqpoint{1.492485in}{3.251702in}}{\pgfqpoint{1.484585in}{3.248429in}}{\pgfqpoint{1.478761in}{3.242605in}}%
\pgfpathcurveto{\pgfqpoint{1.472937in}{3.236781in}}{\pgfqpoint{1.469665in}{3.228881in}}{\pgfqpoint{1.469665in}{3.220645in}}%
\pgfpathcurveto{\pgfqpoint{1.469665in}{3.212409in}}{\pgfqpoint{1.472937in}{3.204509in}}{\pgfqpoint{1.478761in}{3.198685in}}%
\pgfpathcurveto{\pgfqpoint{1.484585in}{3.192861in}}{\pgfqpoint{1.492485in}{3.189589in}}{\pgfqpoint{1.500722in}{3.189589in}}%
\pgfpathclose%
\pgfusepath{stroke,fill}%
\end{pgfscope}%
\begin{pgfscope}%
\pgfpathrectangle{\pgfqpoint{0.100000in}{0.212622in}}{\pgfqpoint{3.696000in}{3.696000in}}%
\pgfusepath{clip}%
\pgfsetbuttcap%
\pgfsetroundjoin%
\definecolor{currentfill}{rgb}{0.121569,0.466667,0.705882}%
\pgfsetfillcolor{currentfill}%
\pgfsetfillopacity{0.386651}%
\pgfsetlinewidth{1.003750pt}%
\definecolor{currentstroke}{rgb}{0.121569,0.466667,0.705882}%
\pgfsetstrokecolor{currentstroke}%
\pgfsetstrokeopacity{0.386651}%
\pgfsetdash{}{0pt}%
\pgfpathmoveto{\pgfqpoint{1.500170in}{3.188241in}}%
\pgfpathcurveto{\pgfqpoint{1.508406in}{3.188241in}}{\pgfqpoint{1.516306in}{3.191513in}}{\pgfqpoint{1.522130in}{3.197337in}}%
\pgfpathcurveto{\pgfqpoint{1.527954in}{3.203161in}}{\pgfqpoint{1.531227in}{3.211061in}}{\pgfqpoint{1.531227in}{3.219297in}}%
\pgfpathcurveto{\pgfqpoint{1.531227in}{3.227533in}}{\pgfqpoint{1.527954in}{3.235433in}}{\pgfqpoint{1.522130in}{3.241257in}}%
\pgfpathcurveto{\pgfqpoint{1.516306in}{3.247081in}}{\pgfqpoint{1.508406in}{3.250354in}}{\pgfqpoint{1.500170in}{3.250354in}}%
\pgfpathcurveto{\pgfqpoint{1.491934in}{3.250354in}}{\pgfqpoint{1.484034in}{3.247081in}}{\pgfqpoint{1.478210in}{3.241257in}}%
\pgfpathcurveto{\pgfqpoint{1.472386in}{3.235433in}}{\pgfqpoint{1.469114in}{3.227533in}}{\pgfqpoint{1.469114in}{3.219297in}}%
\pgfpathcurveto{\pgfqpoint{1.469114in}{3.211061in}}{\pgfqpoint{1.472386in}{3.203161in}}{\pgfqpoint{1.478210in}{3.197337in}}%
\pgfpathcurveto{\pgfqpoint{1.484034in}{3.191513in}}{\pgfqpoint{1.491934in}{3.188241in}}{\pgfqpoint{1.500170in}{3.188241in}}%
\pgfpathclose%
\pgfusepath{stroke,fill}%
\end{pgfscope}%
\begin{pgfscope}%
\pgfpathrectangle{\pgfqpoint{0.100000in}{0.212622in}}{\pgfqpoint{3.696000in}{3.696000in}}%
\pgfusepath{clip}%
\pgfsetbuttcap%
\pgfsetroundjoin%
\definecolor{currentfill}{rgb}{0.121569,0.466667,0.705882}%
\pgfsetfillcolor{currentfill}%
\pgfsetfillopacity{0.387515}%
\pgfsetlinewidth{1.003750pt}%
\definecolor{currentstroke}{rgb}{0.121569,0.466667,0.705882}%
\pgfsetstrokecolor{currentstroke}%
\pgfsetstrokeopacity{0.387515}%
\pgfsetdash{}{0pt}%
\pgfpathmoveto{\pgfqpoint{1.498919in}{3.185873in}}%
\pgfpathcurveto{\pgfqpoint{1.507155in}{3.185873in}}{\pgfqpoint{1.515055in}{3.189145in}}{\pgfqpoint{1.520879in}{3.194969in}}%
\pgfpathcurveto{\pgfqpoint{1.526703in}{3.200793in}}{\pgfqpoint{1.529975in}{3.208693in}}{\pgfqpoint{1.529975in}{3.216929in}}%
\pgfpathcurveto{\pgfqpoint{1.529975in}{3.225165in}}{\pgfqpoint{1.526703in}{3.233065in}}{\pgfqpoint{1.520879in}{3.238889in}}%
\pgfpathcurveto{\pgfqpoint{1.515055in}{3.244713in}}{\pgfqpoint{1.507155in}{3.247986in}}{\pgfqpoint{1.498919in}{3.247986in}}%
\pgfpathcurveto{\pgfqpoint{1.490682in}{3.247986in}}{\pgfqpoint{1.482782in}{3.244713in}}{\pgfqpoint{1.476958in}{3.238889in}}%
\pgfpathcurveto{\pgfqpoint{1.471134in}{3.233065in}}{\pgfqpoint{1.467862in}{3.225165in}}{\pgfqpoint{1.467862in}{3.216929in}}%
\pgfpathcurveto{\pgfqpoint{1.467862in}{3.208693in}}{\pgfqpoint{1.471134in}{3.200793in}}{\pgfqpoint{1.476958in}{3.194969in}}%
\pgfpathcurveto{\pgfqpoint{1.482782in}{3.189145in}}{\pgfqpoint{1.490682in}{3.185873in}}{\pgfqpoint{1.498919in}{3.185873in}}%
\pgfpathclose%
\pgfusepath{stroke,fill}%
\end{pgfscope}%
\begin{pgfscope}%
\pgfpathrectangle{\pgfqpoint{0.100000in}{0.212622in}}{\pgfqpoint{3.696000in}{3.696000in}}%
\pgfusepath{clip}%
\pgfsetbuttcap%
\pgfsetroundjoin%
\definecolor{currentfill}{rgb}{0.121569,0.466667,0.705882}%
\pgfsetfillcolor{currentfill}%
\pgfsetfillopacity{0.387627}%
\pgfsetlinewidth{1.003750pt}%
\definecolor{currentstroke}{rgb}{0.121569,0.466667,0.705882}%
\pgfsetstrokecolor{currentstroke}%
\pgfsetstrokeopacity{0.387627}%
\pgfsetdash{}{0pt}%
\pgfpathmoveto{\pgfqpoint{1.498761in}{3.185558in}}%
\pgfpathcurveto{\pgfqpoint{1.506997in}{3.185558in}}{\pgfqpoint{1.514897in}{3.188830in}}{\pgfqpoint{1.520721in}{3.194654in}}%
\pgfpathcurveto{\pgfqpoint{1.526545in}{3.200478in}}{\pgfqpoint{1.529817in}{3.208378in}}{\pgfqpoint{1.529817in}{3.216614in}}%
\pgfpathcurveto{\pgfqpoint{1.529817in}{3.224851in}}{\pgfqpoint{1.526545in}{3.232751in}}{\pgfqpoint{1.520721in}{3.238575in}}%
\pgfpathcurveto{\pgfqpoint{1.514897in}{3.244398in}}{\pgfqpoint{1.506997in}{3.247671in}}{\pgfqpoint{1.498761in}{3.247671in}}%
\pgfpathcurveto{\pgfqpoint{1.490525in}{3.247671in}}{\pgfqpoint{1.482625in}{3.244398in}}{\pgfqpoint{1.476801in}{3.238575in}}%
\pgfpathcurveto{\pgfqpoint{1.470977in}{3.232751in}}{\pgfqpoint{1.467704in}{3.224851in}}{\pgfqpoint{1.467704in}{3.216614in}}%
\pgfpathcurveto{\pgfqpoint{1.467704in}{3.208378in}}{\pgfqpoint{1.470977in}{3.200478in}}{\pgfqpoint{1.476801in}{3.194654in}}%
\pgfpathcurveto{\pgfqpoint{1.482625in}{3.188830in}}{\pgfqpoint{1.490525in}{3.185558in}}{\pgfqpoint{1.498761in}{3.185558in}}%
\pgfpathclose%
\pgfusepath{stroke,fill}%
\end{pgfscope}%
\begin{pgfscope}%
\pgfpathrectangle{\pgfqpoint{0.100000in}{0.212622in}}{\pgfqpoint{3.696000in}{3.696000in}}%
\pgfusepath{clip}%
\pgfsetbuttcap%
\pgfsetroundjoin%
\definecolor{currentfill}{rgb}{0.121569,0.466667,0.705882}%
\pgfsetfillcolor{currentfill}%
\pgfsetfillopacity{0.387838}%
\pgfsetlinewidth{1.003750pt}%
\definecolor{currentstroke}{rgb}{0.121569,0.466667,0.705882}%
\pgfsetstrokecolor{currentstroke}%
\pgfsetstrokeopacity{0.387838}%
\pgfsetdash{}{0pt}%
\pgfpathmoveto{\pgfqpoint{1.498596in}{3.184937in}}%
\pgfpathcurveto{\pgfqpoint{1.506832in}{3.184937in}}{\pgfqpoint{1.514732in}{3.188210in}}{\pgfqpoint{1.520556in}{3.194034in}}%
\pgfpathcurveto{\pgfqpoint{1.526380in}{3.199857in}}{\pgfqpoint{1.529652in}{3.207757in}}{\pgfqpoint{1.529652in}{3.215994in}}%
\pgfpathcurveto{\pgfqpoint{1.529652in}{3.224230in}}{\pgfqpoint{1.526380in}{3.232130in}}{\pgfqpoint{1.520556in}{3.237954in}}%
\pgfpathcurveto{\pgfqpoint{1.514732in}{3.243778in}}{\pgfqpoint{1.506832in}{3.247050in}}{\pgfqpoint{1.498596in}{3.247050in}}%
\pgfpathcurveto{\pgfqpoint{1.490359in}{3.247050in}}{\pgfqpoint{1.482459in}{3.243778in}}{\pgfqpoint{1.476635in}{3.237954in}}%
\pgfpathcurveto{\pgfqpoint{1.470811in}{3.232130in}}{\pgfqpoint{1.467539in}{3.224230in}}{\pgfqpoint{1.467539in}{3.215994in}}%
\pgfpathcurveto{\pgfqpoint{1.467539in}{3.207757in}}{\pgfqpoint{1.470811in}{3.199857in}}{\pgfqpoint{1.476635in}{3.194034in}}%
\pgfpathcurveto{\pgfqpoint{1.482459in}{3.188210in}}{\pgfqpoint{1.490359in}{3.184937in}}{\pgfqpoint{1.498596in}{3.184937in}}%
\pgfpathclose%
\pgfusepath{stroke,fill}%
\end{pgfscope}%
\begin{pgfscope}%
\pgfpathrectangle{\pgfqpoint{0.100000in}{0.212622in}}{\pgfqpoint{3.696000in}{3.696000in}}%
\pgfusepath{clip}%
\pgfsetbuttcap%
\pgfsetroundjoin%
\definecolor{currentfill}{rgb}{0.121569,0.466667,0.705882}%
\pgfsetfillcolor{currentfill}%
\pgfsetfillopacity{0.388219}%
\pgfsetlinewidth{1.003750pt}%
\definecolor{currentstroke}{rgb}{0.121569,0.466667,0.705882}%
\pgfsetstrokecolor{currentstroke}%
\pgfsetstrokeopacity{0.388219}%
\pgfsetdash{}{0pt}%
\pgfpathmoveto{\pgfqpoint{1.498059in}{3.183947in}}%
\pgfpathcurveto{\pgfqpoint{1.506296in}{3.183947in}}{\pgfqpoint{1.514196in}{3.187220in}}{\pgfqpoint{1.520020in}{3.193044in}}%
\pgfpathcurveto{\pgfqpoint{1.525844in}{3.198867in}}{\pgfqpoint{1.529116in}{3.206768in}}{\pgfqpoint{1.529116in}{3.215004in}}%
\pgfpathcurveto{\pgfqpoint{1.529116in}{3.223240in}}{\pgfqpoint{1.525844in}{3.231140in}}{\pgfqpoint{1.520020in}{3.236964in}}%
\pgfpathcurveto{\pgfqpoint{1.514196in}{3.242788in}}{\pgfqpoint{1.506296in}{3.246060in}}{\pgfqpoint{1.498059in}{3.246060in}}%
\pgfpathcurveto{\pgfqpoint{1.489823in}{3.246060in}}{\pgfqpoint{1.481923in}{3.242788in}}{\pgfqpoint{1.476099in}{3.236964in}}%
\pgfpathcurveto{\pgfqpoint{1.470275in}{3.231140in}}{\pgfqpoint{1.467003in}{3.223240in}}{\pgfqpoint{1.467003in}{3.215004in}}%
\pgfpathcurveto{\pgfqpoint{1.467003in}{3.206768in}}{\pgfqpoint{1.470275in}{3.198867in}}{\pgfqpoint{1.476099in}{3.193044in}}%
\pgfpathcurveto{\pgfqpoint{1.481923in}{3.187220in}}{\pgfqpoint{1.489823in}{3.183947in}}{\pgfqpoint{1.498059in}{3.183947in}}%
\pgfpathclose%
\pgfusepath{stroke,fill}%
\end{pgfscope}%
\begin{pgfscope}%
\pgfpathrectangle{\pgfqpoint{0.100000in}{0.212622in}}{\pgfqpoint{3.696000in}{3.696000in}}%
\pgfusepath{clip}%
\pgfsetbuttcap%
\pgfsetroundjoin%
\definecolor{currentfill}{rgb}{0.121569,0.466667,0.705882}%
\pgfsetfillcolor{currentfill}%
\pgfsetfillopacity{0.388917}%
\pgfsetlinewidth{1.003750pt}%
\definecolor{currentstroke}{rgb}{0.121569,0.466667,0.705882}%
\pgfsetstrokecolor{currentstroke}%
\pgfsetstrokeopacity{0.388917}%
\pgfsetdash{}{0pt}%
\pgfpathmoveto{\pgfqpoint{1.497239in}{3.182052in}}%
\pgfpathcurveto{\pgfqpoint{1.505476in}{3.182052in}}{\pgfqpoint{1.513376in}{3.185325in}}{\pgfqpoint{1.519200in}{3.191149in}}%
\pgfpathcurveto{\pgfqpoint{1.525023in}{3.196972in}}{\pgfqpoint{1.528296in}{3.204872in}}{\pgfqpoint{1.528296in}{3.213109in}}%
\pgfpathcurveto{\pgfqpoint{1.528296in}{3.221345in}}{\pgfqpoint{1.525023in}{3.229245in}}{\pgfqpoint{1.519200in}{3.235069in}}%
\pgfpathcurveto{\pgfqpoint{1.513376in}{3.240893in}}{\pgfqpoint{1.505476in}{3.244165in}}{\pgfqpoint{1.497239in}{3.244165in}}%
\pgfpathcurveto{\pgfqpoint{1.489003in}{3.244165in}}{\pgfqpoint{1.481103in}{3.240893in}}{\pgfqpoint{1.475279in}{3.235069in}}%
\pgfpathcurveto{\pgfqpoint{1.469455in}{3.229245in}}{\pgfqpoint{1.466183in}{3.221345in}}{\pgfqpoint{1.466183in}{3.213109in}}%
\pgfpathcurveto{\pgfqpoint{1.466183in}{3.204872in}}{\pgfqpoint{1.469455in}{3.196972in}}{\pgfqpoint{1.475279in}{3.191149in}}%
\pgfpathcurveto{\pgfqpoint{1.481103in}{3.185325in}}{\pgfqpoint{1.489003in}{3.182052in}}{\pgfqpoint{1.497239in}{3.182052in}}%
\pgfpathclose%
\pgfusepath{stroke,fill}%
\end{pgfscope}%
\begin{pgfscope}%
\pgfpathrectangle{\pgfqpoint{0.100000in}{0.212622in}}{\pgfqpoint{3.696000in}{3.696000in}}%
\pgfusepath{clip}%
\pgfsetbuttcap%
\pgfsetroundjoin%
\definecolor{currentfill}{rgb}{0.121569,0.466667,0.705882}%
\pgfsetfillcolor{currentfill}%
\pgfsetfillopacity{0.390187}%
\pgfsetlinewidth{1.003750pt}%
\definecolor{currentstroke}{rgb}{0.121569,0.466667,0.705882}%
\pgfsetstrokecolor{currentstroke}%
\pgfsetstrokeopacity{0.390187}%
\pgfsetdash{}{0pt}%
\pgfpathmoveto{\pgfqpoint{1.495473in}{3.178824in}}%
\pgfpathcurveto{\pgfqpoint{1.503709in}{3.178824in}}{\pgfqpoint{1.511609in}{3.182097in}}{\pgfqpoint{1.517433in}{3.187921in}}%
\pgfpathcurveto{\pgfqpoint{1.523257in}{3.193745in}}{\pgfqpoint{1.526530in}{3.201645in}}{\pgfqpoint{1.526530in}{3.209881in}}%
\pgfpathcurveto{\pgfqpoint{1.526530in}{3.218117in}}{\pgfqpoint{1.523257in}{3.226017in}}{\pgfqpoint{1.517433in}{3.231841in}}%
\pgfpathcurveto{\pgfqpoint{1.511609in}{3.237665in}}{\pgfqpoint{1.503709in}{3.240937in}}{\pgfqpoint{1.495473in}{3.240937in}}%
\pgfpathcurveto{\pgfqpoint{1.487237in}{3.240937in}}{\pgfqpoint{1.479337in}{3.237665in}}{\pgfqpoint{1.473513in}{3.231841in}}%
\pgfpathcurveto{\pgfqpoint{1.467689in}{3.226017in}}{\pgfqpoint{1.464417in}{3.218117in}}{\pgfqpoint{1.464417in}{3.209881in}}%
\pgfpathcurveto{\pgfqpoint{1.464417in}{3.201645in}}{\pgfqpoint{1.467689in}{3.193745in}}{\pgfqpoint{1.473513in}{3.187921in}}%
\pgfpathcurveto{\pgfqpoint{1.479337in}{3.182097in}}{\pgfqpoint{1.487237in}{3.178824in}}{\pgfqpoint{1.495473in}{3.178824in}}%
\pgfpathclose%
\pgfusepath{stroke,fill}%
\end{pgfscope}%
\begin{pgfscope}%
\pgfpathrectangle{\pgfqpoint{0.100000in}{0.212622in}}{\pgfqpoint{3.696000in}{3.696000in}}%
\pgfusepath{clip}%
\pgfsetbuttcap%
\pgfsetroundjoin%
\definecolor{currentfill}{rgb}{0.121569,0.466667,0.705882}%
\pgfsetfillcolor{currentfill}%
\pgfsetfillopacity{0.390231}%
\pgfsetlinewidth{1.003750pt}%
\definecolor{currentstroke}{rgb}{0.121569,0.466667,0.705882}%
\pgfsetstrokecolor{currentstroke}%
\pgfsetstrokeopacity{0.390231}%
\pgfsetdash{}{0pt}%
\pgfpathmoveto{\pgfqpoint{1.956631in}{3.026921in}}%
\pgfpathcurveto{\pgfqpoint{1.964867in}{3.026921in}}{\pgfqpoint{1.972767in}{3.030194in}}{\pgfqpoint{1.978591in}{3.036018in}}%
\pgfpathcurveto{\pgfqpoint{1.984415in}{3.041842in}}{\pgfqpoint{1.987687in}{3.049742in}}{\pgfqpoint{1.987687in}{3.057978in}}%
\pgfpathcurveto{\pgfqpoint{1.987687in}{3.066214in}}{\pgfqpoint{1.984415in}{3.074114in}}{\pgfqpoint{1.978591in}{3.079938in}}%
\pgfpathcurveto{\pgfqpoint{1.972767in}{3.085762in}}{\pgfqpoint{1.964867in}{3.089034in}}{\pgfqpoint{1.956631in}{3.089034in}}%
\pgfpathcurveto{\pgfqpoint{1.948394in}{3.089034in}}{\pgfqpoint{1.940494in}{3.085762in}}{\pgfqpoint{1.934670in}{3.079938in}}%
\pgfpathcurveto{\pgfqpoint{1.928846in}{3.074114in}}{\pgfqpoint{1.925574in}{3.066214in}}{\pgfqpoint{1.925574in}{3.057978in}}%
\pgfpathcurveto{\pgfqpoint{1.925574in}{3.049742in}}{\pgfqpoint{1.928846in}{3.041842in}}{\pgfqpoint{1.934670in}{3.036018in}}%
\pgfpathcurveto{\pgfqpoint{1.940494in}{3.030194in}}{\pgfqpoint{1.948394in}{3.026921in}}{\pgfqpoint{1.956631in}{3.026921in}}%
\pgfpathclose%
\pgfusepath{stroke,fill}%
\end{pgfscope}%
\begin{pgfscope}%
\pgfpathrectangle{\pgfqpoint{0.100000in}{0.212622in}}{\pgfqpoint{3.696000in}{3.696000in}}%
\pgfusepath{clip}%
\pgfsetbuttcap%
\pgfsetroundjoin%
\definecolor{currentfill}{rgb}{0.121569,0.466667,0.705882}%
\pgfsetfillcolor{currentfill}%
\pgfsetfillopacity{0.390947}%
\pgfsetlinewidth{1.003750pt}%
\definecolor{currentstroke}{rgb}{0.121569,0.466667,0.705882}%
\pgfsetstrokecolor{currentstroke}%
\pgfsetstrokeopacity{0.390947}%
\pgfsetdash{}{0pt}%
\pgfpathmoveto{\pgfqpoint{1.494507in}{3.176840in}}%
\pgfpathcurveto{\pgfqpoint{1.502743in}{3.176840in}}{\pgfqpoint{1.510643in}{3.180112in}}{\pgfqpoint{1.516467in}{3.185936in}}%
\pgfpathcurveto{\pgfqpoint{1.522291in}{3.191760in}}{\pgfqpoint{1.525563in}{3.199660in}}{\pgfqpoint{1.525563in}{3.207896in}}%
\pgfpathcurveto{\pgfqpoint{1.525563in}{3.216133in}}{\pgfqpoint{1.522291in}{3.224033in}}{\pgfqpoint{1.516467in}{3.229857in}}%
\pgfpathcurveto{\pgfqpoint{1.510643in}{3.235681in}}{\pgfqpoint{1.502743in}{3.238953in}}{\pgfqpoint{1.494507in}{3.238953in}}%
\pgfpathcurveto{\pgfqpoint{1.486270in}{3.238953in}}{\pgfqpoint{1.478370in}{3.235681in}}{\pgfqpoint{1.472546in}{3.229857in}}%
\pgfpathcurveto{\pgfqpoint{1.466722in}{3.224033in}}{\pgfqpoint{1.463450in}{3.216133in}}{\pgfqpoint{1.463450in}{3.207896in}}%
\pgfpathcurveto{\pgfqpoint{1.463450in}{3.199660in}}{\pgfqpoint{1.466722in}{3.191760in}}{\pgfqpoint{1.472546in}{3.185936in}}%
\pgfpathcurveto{\pgfqpoint{1.478370in}{3.180112in}}{\pgfqpoint{1.486270in}{3.176840in}}{\pgfqpoint{1.494507in}{3.176840in}}%
\pgfpathclose%
\pgfusepath{stroke,fill}%
\end{pgfscope}%
\begin{pgfscope}%
\pgfpathrectangle{\pgfqpoint{0.100000in}{0.212622in}}{\pgfqpoint{3.696000in}{3.696000in}}%
\pgfusepath{clip}%
\pgfsetbuttcap%
\pgfsetroundjoin%
\definecolor{currentfill}{rgb}{0.121569,0.466667,0.705882}%
\pgfsetfillcolor{currentfill}%
\pgfsetfillopacity{0.392338}%
\pgfsetlinewidth{1.003750pt}%
\definecolor{currentstroke}{rgb}{0.121569,0.466667,0.705882}%
\pgfsetstrokecolor{currentstroke}%
\pgfsetstrokeopacity{0.392338}%
\pgfsetdash{}{0pt}%
\pgfpathmoveto{\pgfqpoint{1.492722in}{3.173293in}}%
\pgfpathcurveto{\pgfqpoint{1.500958in}{3.173293in}}{\pgfqpoint{1.508858in}{3.176565in}}{\pgfqpoint{1.514682in}{3.182389in}}%
\pgfpathcurveto{\pgfqpoint{1.520506in}{3.188213in}}{\pgfqpoint{1.523778in}{3.196113in}}{\pgfqpoint{1.523778in}{3.204350in}}%
\pgfpathcurveto{\pgfqpoint{1.523778in}{3.212586in}}{\pgfqpoint{1.520506in}{3.220486in}}{\pgfqpoint{1.514682in}{3.226310in}}%
\pgfpathcurveto{\pgfqpoint{1.508858in}{3.232134in}}{\pgfqpoint{1.500958in}{3.235406in}}{\pgfqpoint{1.492722in}{3.235406in}}%
\pgfpathcurveto{\pgfqpoint{1.484486in}{3.235406in}}{\pgfqpoint{1.476585in}{3.232134in}}{\pgfqpoint{1.470762in}{3.226310in}}%
\pgfpathcurveto{\pgfqpoint{1.464938in}{3.220486in}}{\pgfqpoint{1.461665in}{3.212586in}}{\pgfqpoint{1.461665in}{3.204350in}}%
\pgfpathcurveto{\pgfqpoint{1.461665in}{3.196113in}}{\pgfqpoint{1.464938in}{3.188213in}}{\pgfqpoint{1.470762in}{3.182389in}}%
\pgfpathcurveto{\pgfqpoint{1.476585in}{3.176565in}}{\pgfqpoint{1.484486in}{3.173293in}}{\pgfqpoint{1.492722in}{3.173293in}}%
\pgfpathclose%
\pgfusepath{stroke,fill}%
\end{pgfscope}%
\begin{pgfscope}%
\pgfpathrectangle{\pgfqpoint{0.100000in}{0.212622in}}{\pgfqpoint{3.696000in}{3.696000in}}%
\pgfusepath{clip}%
\pgfsetbuttcap%
\pgfsetroundjoin%
\definecolor{currentfill}{rgb}{0.121569,0.466667,0.705882}%
\pgfsetfillcolor{currentfill}%
\pgfsetfillopacity{0.394740}%
\pgfsetlinewidth{1.003750pt}%
\definecolor{currentstroke}{rgb}{0.121569,0.466667,0.705882}%
\pgfsetstrokecolor{currentstroke}%
\pgfsetstrokeopacity{0.394740}%
\pgfsetdash{}{0pt}%
\pgfpathmoveto{\pgfqpoint{1.489047in}{3.166675in}}%
\pgfpathcurveto{\pgfqpoint{1.497284in}{3.166675in}}{\pgfqpoint{1.505184in}{3.169947in}}{\pgfqpoint{1.511008in}{3.175771in}}%
\pgfpathcurveto{\pgfqpoint{1.516831in}{3.181595in}}{\pgfqpoint{1.520104in}{3.189495in}}{\pgfqpoint{1.520104in}{3.197731in}}%
\pgfpathcurveto{\pgfqpoint{1.520104in}{3.205968in}}{\pgfqpoint{1.516831in}{3.213868in}}{\pgfqpoint{1.511008in}{3.219692in}}%
\pgfpathcurveto{\pgfqpoint{1.505184in}{3.225515in}}{\pgfqpoint{1.497284in}{3.228788in}}{\pgfqpoint{1.489047in}{3.228788in}}%
\pgfpathcurveto{\pgfqpoint{1.480811in}{3.228788in}}{\pgfqpoint{1.472911in}{3.225515in}}{\pgfqpoint{1.467087in}{3.219692in}}%
\pgfpathcurveto{\pgfqpoint{1.461263in}{3.213868in}}{\pgfqpoint{1.457991in}{3.205968in}}{\pgfqpoint{1.457991in}{3.197731in}}%
\pgfpathcurveto{\pgfqpoint{1.457991in}{3.189495in}}{\pgfqpoint{1.461263in}{3.181595in}}{\pgfqpoint{1.467087in}{3.175771in}}%
\pgfpathcurveto{\pgfqpoint{1.472911in}{3.169947in}}{\pgfqpoint{1.480811in}{3.166675in}}{\pgfqpoint{1.489047in}{3.166675in}}%
\pgfpathclose%
\pgfusepath{stroke,fill}%
\end{pgfscope}%
\begin{pgfscope}%
\pgfpathrectangle{\pgfqpoint{0.100000in}{0.212622in}}{\pgfqpoint{3.696000in}{3.696000in}}%
\pgfusepath{clip}%
\pgfsetbuttcap%
\pgfsetroundjoin%
\definecolor{currentfill}{rgb}{0.121569,0.466667,0.705882}%
\pgfsetfillcolor{currentfill}%
\pgfsetfillopacity{0.396737}%
\pgfsetlinewidth{1.003750pt}%
\definecolor{currentstroke}{rgb}{0.121569,0.466667,0.705882}%
\pgfsetstrokecolor{currentstroke}%
\pgfsetstrokeopacity{0.396737}%
\pgfsetdash{}{0pt}%
\pgfpathmoveto{\pgfqpoint{1.486625in}{3.161213in}}%
\pgfpathcurveto{\pgfqpoint{1.494861in}{3.161213in}}{\pgfqpoint{1.502762in}{3.164485in}}{\pgfqpoint{1.508585in}{3.170309in}}%
\pgfpathcurveto{\pgfqpoint{1.514409in}{3.176133in}}{\pgfqpoint{1.517682in}{3.184033in}}{\pgfqpoint{1.517682in}{3.192269in}}%
\pgfpathcurveto{\pgfqpoint{1.517682in}{3.200505in}}{\pgfqpoint{1.514409in}{3.208405in}}{\pgfqpoint{1.508585in}{3.214229in}}%
\pgfpathcurveto{\pgfqpoint{1.502762in}{3.220053in}}{\pgfqpoint{1.494861in}{3.223326in}}{\pgfqpoint{1.486625in}{3.223326in}}%
\pgfpathcurveto{\pgfqpoint{1.478389in}{3.223326in}}{\pgfqpoint{1.470489in}{3.220053in}}{\pgfqpoint{1.464665in}{3.214229in}}%
\pgfpathcurveto{\pgfqpoint{1.458841in}{3.208405in}}{\pgfqpoint{1.455569in}{3.200505in}}{\pgfqpoint{1.455569in}{3.192269in}}%
\pgfpathcurveto{\pgfqpoint{1.455569in}{3.184033in}}{\pgfqpoint{1.458841in}{3.176133in}}{\pgfqpoint{1.464665in}{3.170309in}}%
\pgfpathcurveto{\pgfqpoint{1.470489in}{3.164485in}}{\pgfqpoint{1.478389in}{3.161213in}}{\pgfqpoint{1.486625in}{3.161213in}}%
\pgfpathclose%
\pgfusepath{stroke,fill}%
\end{pgfscope}%
\begin{pgfscope}%
\pgfpathrectangle{\pgfqpoint{0.100000in}{0.212622in}}{\pgfqpoint{3.696000in}{3.696000in}}%
\pgfusepath{clip}%
\pgfsetbuttcap%
\pgfsetroundjoin%
\definecolor{currentfill}{rgb}{0.121569,0.466667,0.705882}%
\pgfsetfillcolor{currentfill}%
\pgfsetfillopacity{0.398104}%
\pgfsetlinewidth{1.003750pt}%
\definecolor{currentstroke}{rgb}{0.121569,0.466667,0.705882}%
\pgfsetstrokecolor{currentstroke}%
\pgfsetstrokeopacity{0.398104}%
\pgfsetdash{}{0pt}%
\pgfpathmoveto{\pgfqpoint{1.484679in}{3.157652in}}%
\pgfpathcurveto{\pgfqpoint{1.492915in}{3.157652in}}{\pgfqpoint{1.500815in}{3.160924in}}{\pgfqpoint{1.506639in}{3.166748in}}%
\pgfpathcurveto{\pgfqpoint{1.512463in}{3.172572in}}{\pgfqpoint{1.515735in}{3.180472in}}{\pgfqpoint{1.515735in}{3.188709in}}%
\pgfpathcurveto{\pgfqpoint{1.515735in}{3.196945in}}{\pgfqpoint{1.512463in}{3.204845in}}{\pgfqpoint{1.506639in}{3.210669in}}%
\pgfpathcurveto{\pgfqpoint{1.500815in}{3.216493in}}{\pgfqpoint{1.492915in}{3.219765in}}{\pgfqpoint{1.484679in}{3.219765in}}%
\pgfpathcurveto{\pgfqpoint{1.476442in}{3.219765in}}{\pgfqpoint{1.468542in}{3.216493in}}{\pgfqpoint{1.462718in}{3.210669in}}%
\pgfpathcurveto{\pgfqpoint{1.456894in}{3.204845in}}{\pgfqpoint{1.453622in}{3.196945in}}{\pgfqpoint{1.453622in}{3.188709in}}%
\pgfpathcurveto{\pgfqpoint{1.453622in}{3.180472in}}{\pgfqpoint{1.456894in}{3.172572in}}{\pgfqpoint{1.462718in}{3.166748in}}%
\pgfpathcurveto{\pgfqpoint{1.468542in}{3.160924in}}{\pgfqpoint{1.476442in}{3.157652in}}{\pgfqpoint{1.484679in}{3.157652in}}%
\pgfpathclose%
\pgfusepath{stroke,fill}%
\end{pgfscope}%
\begin{pgfscope}%
\pgfpathrectangle{\pgfqpoint{0.100000in}{0.212622in}}{\pgfqpoint{3.696000in}{3.696000in}}%
\pgfusepath{clip}%
\pgfsetbuttcap%
\pgfsetroundjoin%
\definecolor{currentfill}{rgb}{0.121569,0.466667,0.705882}%
\pgfsetfillcolor{currentfill}%
\pgfsetfillopacity{0.398533}%
\pgfsetlinewidth{1.003750pt}%
\definecolor{currentstroke}{rgb}{0.121569,0.466667,0.705882}%
\pgfsetstrokecolor{currentstroke}%
\pgfsetstrokeopacity{0.398533}%
\pgfsetdash{}{0pt}%
\pgfpathmoveto{\pgfqpoint{1.484084in}{3.156519in}}%
\pgfpathcurveto{\pgfqpoint{1.492320in}{3.156519in}}{\pgfqpoint{1.500220in}{3.159791in}}{\pgfqpoint{1.506044in}{3.165615in}}%
\pgfpathcurveto{\pgfqpoint{1.511868in}{3.171439in}}{\pgfqpoint{1.515141in}{3.179339in}}{\pgfqpoint{1.515141in}{3.187576in}}%
\pgfpathcurveto{\pgfqpoint{1.515141in}{3.195812in}}{\pgfqpoint{1.511868in}{3.203712in}}{\pgfqpoint{1.506044in}{3.209536in}}%
\pgfpathcurveto{\pgfqpoint{1.500220in}{3.215360in}}{\pgfqpoint{1.492320in}{3.218632in}}{\pgfqpoint{1.484084in}{3.218632in}}%
\pgfpathcurveto{\pgfqpoint{1.475848in}{3.218632in}}{\pgfqpoint{1.467948in}{3.215360in}}{\pgfqpoint{1.462124in}{3.209536in}}%
\pgfpathcurveto{\pgfqpoint{1.456300in}{3.203712in}}{\pgfqpoint{1.453028in}{3.195812in}}{\pgfqpoint{1.453028in}{3.187576in}}%
\pgfpathcurveto{\pgfqpoint{1.453028in}{3.179339in}}{\pgfqpoint{1.456300in}{3.171439in}}{\pgfqpoint{1.462124in}{3.165615in}}%
\pgfpathcurveto{\pgfqpoint{1.467948in}{3.159791in}}{\pgfqpoint{1.475848in}{3.156519in}}{\pgfqpoint{1.484084in}{3.156519in}}%
\pgfpathclose%
\pgfusepath{stroke,fill}%
\end{pgfscope}%
\begin{pgfscope}%
\pgfpathrectangle{\pgfqpoint{0.100000in}{0.212622in}}{\pgfqpoint{3.696000in}{3.696000in}}%
\pgfusepath{clip}%
\pgfsetbuttcap%
\pgfsetroundjoin%
\definecolor{currentfill}{rgb}{0.121569,0.466667,0.705882}%
\pgfsetfillcolor{currentfill}%
\pgfsetfillopacity{0.398599}%
\pgfsetlinewidth{1.003750pt}%
\definecolor{currentstroke}{rgb}{0.121569,0.466667,0.705882}%
\pgfsetstrokecolor{currentstroke}%
\pgfsetstrokeopacity{0.398599}%
\pgfsetdash{}{0pt}%
\pgfpathmoveto{\pgfqpoint{1.972632in}{2.998819in}}%
\pgfpathcurveto{\pgfqpoint{1.980868in}{2.998819in}}{\pgfqpoint{1.988768in}{3.002092in}}{\pgfqpoint{1.994592in}{3.007916in}}%
\pgfpathcurveto{\pgfqpoint{2.000416in}{3.013740in}}{\pgfqpoint{2.003689in}{3.021640in}}{\pgfqpoint{2.003689in}{3.029876in}}%
\pgfpathcurveto{\pgfqpoint{2.003689in}{3.038112in}}{\pgfqpoint{2.000416in}{3.046012in}}{\pgfqpoint{1.994592in}{3.051836in}}%
\pgfpathcurveto{\pgfqpoint{1.988768in}{3.057660in}}{\pgfqpoint{1.980868in}{3.060932in}}{\pgfqpoint{1.972632in}{3.060932in}}%
\pgfpathcurveto{\pgfqpoint{1.964396in}{3.060932in}}{\pgfqpoint{1.956496in}{3.057660in}}{\pgfqpoint{1.950672in}{3.051836in}}%
\pgfpathcurveto{\pgfqpoint{1.944848in}{3.046012in}}{\pgfqpoint{1.941576in}{3.038112in}}{\pgfqpoint{1.941576in}{3.029876in}}%
\pgfpathcurveto{\pgfqpoint{1.941576in}{3.021640in}}{\pgfqpoint{1.944848in}{3.013740in}}{\pgfqpoint{1.950672in}{3.007916in}}%
\pgfpathcurveto{\pgfqpoint{1.956496in}{3.002092in}}{\pgfqpoint{1.964396in}{2.998819in}}{\pgfqpoint{1.972632in}{2.998819in}}%
\pgfpathclose%
\pgfusepath{stroke,fill}%
\end{pgfscope}%
\begin{pgfscope}%
\pgfpathrectangle{\pgfqpoint{0.100000in}{0.212622in}}{\pgfqpoint{3.696000in}{3.696000in}}%
\pgfusepath{clip}%
\pgfsetbuttcap%
\pgfsetroundjoin%
\definecolor{currentfill}{rgb}{0.121569,0.466667,0.705882}%
\pgfsetfillcolor{currentfill}%
\pgfsetfillopacity{0.399316}%
\pgfsetlinewidth{1.003750pt}%
\definecolor{currentstroke}{rgb}{0.121569,0.466667,0.705882}%
\pgfsetstrokecolor{currentstroke}%
\pgfsetstrokeopacity{0.399316}%
\pgfsetdash{}{0pt}%
\pgfpathmoveto{\pgfqpoint{1.483014in}{3.154467in}}%
\pgfpathcurveto{\pgfqpoint{1.491250in}{3.154467in}}{\pgfqpoint{1.499150in}{3.157739in}}{\pgfqpoint{1.504974in}{3.163563in}}%
\pgfpathcurveto{\pgfqpoint{1.510798in}{3.169387in}}{\pgfqpoint{1.514071in}{3.177287in}}{\pgfqpoint{1.514071in}{3.185523in}}%
\pgfpathcurveto{\pgfqpoint{1.514071in}{3.193760in}}{\pgfqpoint{1.510798in}{3.201660in}}{\pgfqpoint{1.504974in}{3.207484in}}%
\pgfpathcurveto{\pgfqpoint{1.499150in}{3.213308in}}{\pgfqpoint{1.491250in}{3.216580in}}{\pgfqpoint{1.483014in}{3.216580in}}%
\pgfpathcurveto{\pgfqpoint{1.474778in}{3.216580in}}{\pgfqpoint{1.466878in}{3.213308in}}{\pgfqpoint{1.461054in}{3.207484in}}%
\pgfpathcurveto{\pgfqpoint{1.455230in}{3.201660in}}{\pgfqpoint{1.451958in}{3.193760in}}{\pgfqpoint{1.451958in}{3.185523in}}%
\pgfpathcurveto{\pgfqpoint{1.451958in}{3.177287in}}{\pgfqpoint{1.455230in}{3.169387in}}{\pgfqpoint{1.461054in}{3.163563in}}%
\pgfpathcurveto{\pgfqpoint{1.466878in}{3.157739in}}{\pgfqpoint{1.474778in}{3.154467in}}{\pgfqpoint{1.483014in}{3.154467in}}%
\pgfpathclose%
\pgfusepath{stroke,fill}%
\end{pgfscope}%
\begin{pgfscope}%
\pgfpathrectangle{\pgfqpoint{0.100000in}{0.212622in}}{\pgfqpoint{3.696000in}{3.696000in}}%
\pgfusepath{clip}%
\pgfsetbuttcap%
\pgfsetroundjoin%
\definecolor{currentfill}{rgb}{0.121569,0.466667,0.705882}%
\pgfsetfillcolor{currentfill}%
\pgfsetfillopacity{0.400699}%
\pgfsetlinewidth{1.003750pt}%
\definecolor{currentstroke}{rgb}{0.121569,0.466667,0.705882}%
\pgfsetstrokecolor{currentstroke}%
\pgfsetstrokeopacity{0.400699}%
\pgfsetdash{}{0pt}%
\pgfpathmoveto{\pgfqpoint{1.480869in}{3.150735in}}%
\pgfpathcurveto{\pgfqpoint{1.489106in}{3.150735in}}{\pgfqpoint{1.497006in}{3.154007in}}{\pgfqpoint{1.502830in}{3.159831in}}%
\pgfpathcurveto{\pgfqpoint{1.508654in}{3.165655in}}{\pgfqpoint{1.511926in}{3.173555in}}{\pgfqpoint{1.511926in}{3.181791in}}%
\pgfpathcurveto{\pgfqpoint{1.511926in}{3.190028in}}{\pgfqpoint{1.508654in}{3.197928in}}{\pgfqpoint{1.502830in}{3.203752in}}%
\pgfpathcurveto{\pgfqpoint{1.497006in}{3.209576in}}{\pgfqpoint{1.489106in}{3.212848in}}{\pgfqpoint{1.480869in}{3.212848in}}%
\pgfpathcurveto{\pgfqpoint{1.472633in}{3.212848in}}{\pgfqpoint{1.464733in}{3.209576in}}{\pgfqpoint{1.458909in}{3.203752in}}%
\pgfpathcurveto{\pgfqpoint{1.453085in}{3.197928in}}{\pgfqpoint{1.449813in}{3.190028in}}{\pgfqpoint{1.449813in}{3.181791in}}%
\pgfpathcurveto{\pgfqpoint{1.449813in}{3.173555in}}{\pgfqpoint{1.453085in}{3.165655in}}{\pgfqpoint{1.458909in}{3.159831in}}%
\pgfpathcurveto{\pgfqpoint{1.464733in}{3.154007in}}{\pgfqpoint{1.472633in}{3.150735in}}{\pgfqpoint{1.480869in}{3.150735in}}%
\pgfpathclose%
\pgfusepath{stroke,fill}%
\end{pgfscope}%
\begin{pgfscope}%
\pgfpathrectangle{\pgfqpoint{0.100000in}{0.212622in}}{\pgfqpoint{3.696000in}{3.696000in}}%
\pgfusepath{clip}%
\pgfsetbuttcap%
\pgfsetroundjoin%
\definecolor{currentfill}{rgb}{0.121569,0.466667,0.705882}%
\pgfsetfillcolor{currentfill}%
\pgfsetfillopacity{0.401544}%
\pgfsetlinewidth{1.003750pt}%
\definecolor{currentstroke}{rgb}{0.121569,0.466667,0.705882}%
\pgfsetstrokecolor{currentstroke}%
\pgfsetstrokeopacity{0.401544}%
\pgfsetdash{}{0pt}%
\pgfpathmoveto{\pgfqpoint{1.480056in}{3.148522in}}%
\pgfpathcurveto{\pgfqpoint{1.488292in}{3.148522in}}{\pgfqpoint{1.496192in}{3.151794in}}{\pgfqpoint{1.502016in}{3.157618in}}%
\pgfpathcurveto{\pgfqpoint{1.507840in}{3.163442in}}{\pgfqpoint{1.511112in}{3.171342in}}{\pgfqpoint{1.511112in}{3.179578in}}%
\pgfpathcurveto{\pgfqpoint{1.511112in}{3.187814in}}{\pgfqpoint{1.507840in}{3.195714in}}{\pgfqpoint{1.502016in}{3.201538in}}%
\pgfpathcurveto{\pgfqpoint{1.496192in}{3.207362in}}{\pgfqpoint{1.488292in}{3.210635in}}{\pgfqpoint{1.480056in}{3.210635in}}%
\pgfpathcurveto{\pgfqpoint{1.471819in}{3.210635in}}{\pgfqpoint{1.463919in}{3.207362in}}{\pgfqpoint{1.458095in}{3.201538in}}%
\pgfpathcurveto{\pgfqpoint{1.452271in}{3.195714in}}{\pgfqpoint{1.448999in}{3.187814in}}{\pgfqpoint{1.448999in}{3.179578in}}%
\pgfpathcurveto{\pgfqpoint{1.448999in}{3.171342in}}{\pgfqpoint{1.452271in}{3.163442in}}{\pgfqpoint{1.458095in}{3.157618in}}%
\pgfpathcurveto{\pgfqpoint{1.463919in}{3.151794in}}{\pgfqpoint{1.471819in}{3.148522in}}{\pgfqpoint{1.480056in}{3.148522in}}%
\pgfpathclose%
\pgfusepath{stroke,fill}%
\end{pgfscope}%
\begin{pgfscope}%
\pgfpathrectangle{\pgfqpoint{0.100000in}{0.212622in}}{\pgfqpoint{3.696000in}{3.696000in}}%
\pgfusepath{clip}%
\pgfsetbuttcap%
\pgfsetroundjoin%
\definecolor{currentfill}{rgb}{0.121569,0.466667,0.705882}%
\pgfsetfillcolor{currentfill}%
\pgfsetfillopacity{0.402992}%
\pgfsetlinewidth{1.003750pt}%
\definecolor{currentstroke}{rgb}{0.121569,0.466667,0.705882}%
\pgfsetstrokecolor{currentstroke}%
\pgfsetstrokeopacity{0.402992}%
\pgfsetdash{}{0pt}%
\pgfpathmoveto{\pgfqpoint{1.477918in}{3.144607in}}%
\pgfpathcurveto{\pgfqpoint{1.486154in}{3.144607in}}{\pgfqpoint{1.494054in}{3.147880in}}{\pgfqpoint{1.499878in}{3.153703in}}%
\pgfpathcurveto{\pgfqpoint{1.505702in}{3.159527in}}{\pgfqpoint{1.508974in}{3.167427in}}{\pgfqpoint{1.508974in}{3.175664in}}%
\pgfpathcurveto{\pgfqpoint{1.508974in}{3.183900in}}{\pgfqpoint{1.505702in}{3.191800in}}{\pgfqpoint{1.499878in}{3.197624in}}%
\pgfpathcurveto{\pgfqpoint{1.494054in}{3.203448in}}{\pgfqpoint{1.486154in}{3.206720in}}{\pgfqpoint{1.477918in}{3.206720in}}%
\pgfpathcurveto{\pgfqpoint{1.469682in}{3.206720in}}{\pgfqpoint{1.461782in}{3.203448in}}{\pgfqpoint{1.455958in}{3.197624in}}%
\pgfpathcurveto{\pgfqpoint{1.450134in}{3.191800in}}{\pgfqpoint{1.446861in}{3.183900in}}{\pgfqpoint{1.446861in}{3.175664in}}%
\pgfpathcurveto{\pgfqpoint{1.446861in}{3.167427in}}{\pgfqpoint{1.450134in}{3.159527in}}{\pgfqpoint{1.455958in}{3.153703in}}%
\pgfpathcurveto{\pgfqpoint{1.461782in}{3.147880in}}{\pgfqpoint{1.469682in}{3.144607in}}{\pgfqpoint{1.477918in}{3.144607in}}%
\pgfpathclose%
\pgfusepath{stroke,fill}%
\end{pgfscope}%
\begin{pgfscope}%
\pgfpathrectangle{\pgfqpoint{0.100000in}{0.212622in}}{\pgfqpoint{3.696000in}{3.696000in}}%
\pgfusepath{clip}%
\pgfsetbuttcap%
\pgfsetroundjoin%
\definecolor{currentfill}{rgb}{0.121569,0.466667,0.705882}%
\pgfsetfillcolor{currentfill}%
\pgfsetfillopacity{0.405703}%
\pgfsetlinewidth{1.003750pt}%
\definecolor{currentstroke}{rgb}{0.121569,0.466667,0.705882}%
\pgfsetstrokecolor{currentstroke}%
\pgfsetstrokeopacity{0.405703}%
\pgfsetdash{}{0pt}%
\pgfpathmoveto{\pgfqpoint{1.474660in}{3.137308in}}%
\pgfpathcurveto{\pgfqpoint{1.482896in}{3.137308in}}{\pgfqpoint{1.490797in}{3.140580in}}{\pgfqpoint{1.496620in}{3.146404in}}%
\pgfpathcurveto{\pgfqpoint{1.502444in}{3.152228in}}{\pgfqpoint{1.505717in}{3.160128in}}{\pgfqpoint{1.505717in}{3.168364in}}%
\pgfpathcurveto{\pgfqpoint{1.505717in}{3.176601in}}{\pgfqpoint{1.502444in}{3.184501in}}{\pgfqpoint{1.496620in}{3.190325in}}%
\pgfpathcurveto{\pgfqpoint{1.490797in}{3.196149in}}{\pgfqpoint{1.482896in}{3.199421in}}{\pgfqpoint{1.474660in}{3.199421in}}%
\pgfpathcurveto{\pgfqpoint{1.466424in}{3.199421in}}{\pgfqpoint{1.458524in}{3.196149in}}{\pgfqpoint{1.452700in}{3.190325in}}%
\pgfpathcurveto{\pgfqpoint{1.446876in}{3.184501in}}{\pgfqpoint{1.443604in}{3.176601in}}{\pgfqpoint{1.443604in}{3.168364in}}%
\pgfpathcurveto{\pgfqpoint{1.443604in}{3.160128in}}{\pgfqpoint{1.446876in}{3.152228in}}{\pgfqpoint{1.452700in}{3.146404in}}%
\pgfpathcurveto{\pgfqpoint{1.458524in}{3.140580in}}{\pgfqpoint{1.466424in}{3.137308in}}{\pgfqpoint{1.474660in}{3.137308in}}%
\pgfpathclose%
\pgfusepath{stroke,fill}%
\end{pgfscope}%
\begin{pgfscope}%
\pgfpathrectangle{\pgfqpoint{0.100000in}{0.212622in}}{\pgfqpoint{3.696000in}{3.696000in}}%
\pgfusepath{clip}%
\pgfsetbuttcap%
\pgfsetroundjoin%
\definecolor{currentfill}{rgb}{0.121569,0.466667,0.705882}%
\pgfsetfillcolor{currentfill}%
\pgfsetfillopacity{0.405728}%
\pgfsetlinewidth{1.003750pt}%
\definecolor{currentstroke}{rgb}{0.121569,0.466667,0.705882}%
\pgfsetstrokecolor{currentstroke}%
\pgfsetstrokeopacity{0.405728}%
\pgfsetdash{}{0pt}%
\pgfpathmoveto{\pgfqpoint{1.474625in}{3.137244in}}%
\pgfpathcurveto{\pgfqpoint{1.482861in}{3.137244in}}{\pgfqpoint{1.490761in}{3.140517in}}{\pgfqpoint{1.496585in}{3.146341in}}%
\pgfpathcurveto{\pgfqpoint{1.502409in}{3.152165in}}{\pgfqpoint{1.505681in}{3.160065in}}{\pgfqpoint{1.505681in}{3.168301in}}%
\pgfpathcurveto{\pgfqpoint{1.505681in}{3.176537in}}{\pgfqpoint{1.502409in}{3.184437in}}{\pgfqpoint{1.496585in}{3.190261in}}%
\pgfpathcurveto{\pgfqpoint{1.490761in}{3.196085in}}{\pgfqpoint{1.482861in}{3.199357in}}{\pgfqpoint{1.474625in}{3.199357in}}%
\pgfpathcurveto{\pgfqpoint{1.466388in}{3.199357in}}{\pgfqpoint{1.458488in}{3.196085in}}{\pgfqpoint{1.452664in}{3.190261in}}%
\pgfpathcurveto{\pgfqpoint{1.446841in}{3.184437in}}{\pgfqpoint{1.443568in}{3.176537in}}{\pgfqpoint{1.443568in}{3.168301in}}%
\pgfpathcurveto{\pgfqpoint{1.443568in}{3.160065in}}{\pgfqpoint{1.446841in}{3.152165in}}{\pgfqpoint{1.452664in}{3.146341in}}%
\pgfpathcurveto{\pgfqpoint{1.458488in}{3.140517in}}{\pgfqpoint{1.466388in}{3.137244in}}{\pgfqpoint{1.474625in}{3.137244in}}%
\pgfpathclose%
\pgfusepath{stroke,fill}%
\end{pgfscope}%
\begin{pgfscope}%
\pgfpathrectangle{\pgfqpoint{0.100000in}{0.212622in}}{\pgfqpoint{3.696000in}{3.696000in}}%
\pgfusepath{clip}%
\pgfsetbuttcap%
\pgfsetroundjoin%
\definecolor{currentfill}{rgb}{0.121569,0.466667,0.705882}%
\pgfsetfillcolor{currentfill}%
\pgfsetfillopacity{0.405774}%
\pgfsetlinewidth{1.003750pt}%
\definecolor{currentstroke}{rgb}{0.121569,0.466667,0.705882}%
\pgfsetstrokecolor{currentstroke}%
\pgfsetstrokeopacity{0.405774}%
\pgfsetdash{}{0pt}%
\pgfpathmoveto{\pgfqpoint{1.474567in}{3.137125in}}%
\pgfpathcurveto{\pgfqpoint{1.482803in}{3.137125in}}{\pgfqpoint{1.490703in}{3.140398in}}{\pgfqpoint{1.496527in}{3.146222in}}%
\pgfpathcurveto{\pgfqpoint{1.502351in}{3.152046in}}{\pgfqpoint{1.505624in}{3.159946in}}{\pgfqpoint{1.505624in}{3.168182in}}%
\pgfpathcurveto{\pgfqpoint{1.505624in}{3.176418in}}{\pgfqpoint{1.502351in}{3.184318in}}{\pgfqpoint{1.496527in}{3.190142in}}%
\pgfpathcurveto{\pgfqpoint{1.490703in}{3.195966in}}{\pgfqpoint{1.482803in}{3.199238in}}{\pgfqpoint{1.474567in}{3.199238in}}%
\pgfpathcurveto{\pgfqpoint{1.466331in}{3.199238in}}{\pgfqpoint{1.458431in}{3.195966in}}{\pgfqpoint{1.452607in}{3.190142in}}%
\pgfpathcurveto{\pgfqpoint{1.446783in}{3.184318in}}{\pgfqpoint{1.443511in}{3.176418in}}{\pgfqpoint{1.443511in}{3.168182in}}%
\pgfpathcurveto{\pgfqpoint{1.443511in}{3.159946in}}{\pgfqpoint{1.446783in}{3.152046in}}{\pgfqpoint{1.452607in}{3.146222in}}%
\pgfpathcurveto{\pgfqpoint{1.458431in}{3.140398in}}{\pgfqpoint{1.466331in}{3.137125in}}{\pgfqpoint{1.474567in}{3.137125in}}%
\pgfpathclose%
\pgfusepath{stroke,fill}%
\end{pgfscope}%
\begin{pgfscope}%
\pgfpathrectangle{\pgfqpoint{0.100000in}{0.212622in}}{\pgfqpoint{3.696000in}{3.696000in}}%
\pgfusepath{clip}%
\pgfsetbuttcap%
\pgfsetroundjoin%
\definecolor{currentfill}{rgb}{0.121569,0.466667,0.705882}%
\pgfsetfillcolor{currentfill}%
\pgfsetfillopacity{0.405856}%
\pgfsetlinewidth{1.003750pt}%
\definecolor{currentstroke}{rgb}{0.121569,0.466667,0.705882}%
\pgfsetstrokecolor{currentstroke}%
\pgfsetstrokeopacity{0.405856}%
\pgfsetdash{}{0pt}%
\pgfpathmoveto{\pgfqpoint{1.474450in}{3.136912in}}%
\pgfpathcurveto{\pgfqpoint{1.482686in}{3.136912in}}{\pgfqpoint{1.490586in}{3.140184in}}{\pgfqpoint{1.496410in}{3.146008in}}%
\pgfpathcurveto{\pgfqpoint{1.502234in}{3.151832in}}{\pgfqpoint{1.505507in}{3.159732in}}{\pgfqpoint{1.505507in}{3.167968in}}%
\pgfpathcurveto{\pgfqpoint{1.505507in}{3.176205in}}{\pgfqpoint{1.502234in}{3.184105in}}{\pgfqpoint{1.496410in}{3.189929in}}%
\pgfpathcurveto{\pgfqpoint{1.490586in}{3.195752in}}{\pgfqpoint{1.482686in}{3.199025in}}{\pgfqpoint{1.474450in}{3.199025in}}%
\pgfpathcurveto{\pgfqpoint{1.466214in}{3.199025in}}{\pgfqpoint{1.458314in}{3.195752in}}{\pgfqpoint{1.452490in}{3.189929in}}%
\pgfpathcurveto{\pgfqpoint{1.446666in}{3.184105in}}{\pgfqpoint{1.443394in}{3.176205in}}{\pgfqpoint{1.443394in}{3.167968in}}%
\pgfpathcurveto{\pgfqpoint{1.443394in}{3.159732in}}{\pgfqpoint{1.446666in}{3.151832in}}{\pgfqpoint{1.452490in}{3.146008in}}%
\pgfpathcurveto{\pgfqpoint{1.458314in}{3.140184in}}{\pgfqpoint{1.466214in}{3.136912in}}{\pgfqpoint{1.474450in}{3.136912in}}%
\pgfpathclose%
\pgfusepath{stroke,fill}%
\end{pgfscope}%
\begin{pgfscope}%
\pgfpathrectangle{\pgfqpoint{0.100000in}{0.212622in}}{\pgfqpoint{3.696000in}{3.696000in}}%
\pgfusepath{clip}%
\pgfsetbuttcap%
\pgfsetroundjoin%
\definecolor{currentfill}{rgb}{0.121569,0.466667,0.705882}%
\pgfsetfillcolor{currentfill}%
\pgfsetfillopacity{0.406008}%
\pgfsetlinewidth{1.003750pt}%
\definecolor{currentstroke}{rgb}{0.121569,0.466667,0.705882}%
\pgfsetstrokecolor{currentstroke}%
\pgfsetstrokeopacity{0.406008}%
\pgfsetdash{}{0pt}%
\pgfpathmoveto{\pgfqpoint{1.474266in}{3.136512in}}%
\pgfpathcurveto{\pgfqpoint{1.482502in}{3.136512in}}{\pgfqpoint{1.490402in}{3.139784in}}{\pgfqpoint{1.496226in}{3.145608in}}%
\pgfpathcurveto{\pgfqpoint{1.502050in}{3.151432in}}{\pgfqpoint{1.505322in}{3.159332in}}{\pgfqpoint{1.505322in}{3.167568in}}%
\pgfpathcurveto{\pgfqpoint{1.505322in}{3.175804in}}{\pgfqpoint{1.502050in}{3.183704in}}{\pgfqpoint{1.496226in}{3.189528in}}%
\pgfpathcurveto{\pgfqpoint{1.490402in}{3.195352in}}{\pgfqpoint{1.482502in}{3.198625in}}{\pgfqpoint{1.474266in}{3.198625in}}%
\pgfpathcurveto{\pgfqpoint{1.466030in}{3.198625in}}{\pgfqpoint{1.458130in}{3.195352in}}{\pgfqpoint{1.452306in}{3.189528in}}%
\pgfpathcurveto{\pgfqpoint{1.446482in}{3.183704in}}{\pgfqpoint{1.443209in}{3.175804in}}{\pgfqpoint{1.443209in}{3.167568in}}%
\pgfpathcurveto{\pgfqpoint{1.443209in}{3.159332in}}{\pgfqpoint{1.446482in}{3.151432in}}{\pgfqpoint{1.452306in}{3.145608in}}%
\pgfpathcurveto{\pgfqpoint{1.458130in}{3.139784in}}{\pgfqpoint{1.466030in}{3.136512in}}{\pgfqpoint{1.474266in}{3.136512in}}%
\pgfpathclose%
\pgfusepath{stroke,fill}%
\end{pgfscope}%
\begin{pgfscope}%
\pgfpathrectangle{\pgfqpoint{0.100000in}{0.212622in}}{\pgfqpoint{3.696000in}{3.696000in}}%
\pgfusepath{clip}%
\pgfsetbuttcap%
\pgfsetroundjoin%
\definecolor{currentfill}{rgb}{0.121569,0.466667,0.705882}%
\pgfsetfillcolor{currentfill}%
\pgfsetfillopacity{0.406282}%
\pgfsetlinewidth{1.003750pt}%
\definecolor{currentstroke}{rgb}{0.121569,0.466667,0.705882}%
\pgfsetstrokecolor{currentstroke}%
\pgfsetstrokeopacity{0.406282}%
\pgfsetdash{}{0pt}%
\pgfpathmoveto{\pgfqpoint{1.473881in}{3.135819in}}%
\pgfpathcurveto{\pgfqpoint{1.482117in}{3.135819in}}{\pgfqpoint{1.490017in}{3.139091in}}{\pgfqpoint{1.495841in}{3.144915in}}%
\pgfpathcurveto{\pgfqpoint{1.501665in}{3.150739in}}{\pgfqpoint{1.504937in}{3.158639in}}{\pgfqpoint{1.504937in}{3.166875in}}%
\pgfpathcurveto{\pgfqpoint{1.504937in}{3.175112in}}{\pgfqpoint{1.501665in}{3.183012in}}{\pgfqpoint{1.495841in}{3.188836in}}%
\pgfpathcurveto{\pgfqpoint{1.490017in}{3.194660in}}{\pgfqpoint{1.482117in}{3.197932in}}{\pgfqpoint{1.473881in}{3.197932in}}%
\pgfpathcurveto{\pgfqpoint{1.465644in}{3.197932in}}{\pgfqpoint{1.457744in}{3.194660in}}{\pgfqpoint{1.451920in}{3.188836in}}%
\pgfpathcurveto{\pgfqpoint{1.446096in}{3.183012in}}{\pgfqpoint{1.442824in}{3.175112in}}{\pgfqpoint{1.442824in}{3.166875in}}%
\pgfpathcurveto{\pgfqpoint{1.442824in}{3.158639in}}{\pgfqpoint{1.446096in}{3.150739in}}{\pgfqpoint{1.451920in}{3.144915in}}%
\pgfpathcurveto{\pgfqpoint{1.457744in}{3.139091in}}{\pgfqpoint{1.465644in}{3.135819in}}{\pgfqpoint{1.473881in}{3.135819in}}%
\pgfpathclose%
\pgfusepath{stroke,fill}%
\end{pgfscope}%
\begin{pgfscope}%
\pgfpathrectangle{\pgfqpoint{0.100000in}{0.212622in}}{\pgfqpoint{3.696000in}{3.696000in}}%
\pgfusepath{clip}%
\pgfsetbuttcap%
\pgfsetroundjoin%
\definecolor{currentfill}{rgb}{0.121569,0.466667,0.705882}%
\pgfsetfillcolor{currentfill}%
\pgfsetfillopacity{0.406784}%
\pgfsetlinewidth{1.003750pt}%
\definecolor{currentstroke}{rgb}{0.121569,0.466667,0.705882}%
\pgfsetstrokecolor{currentstroke}%
\pgfsetstrokeopacity{0.406784}%
\pgfsetdash{}{0pt}%
\pgfpathmoveto{\pgfqpoint{1.473242in}{3.134522in}}%
\pgfpathcurveto{\pgfqpoint{1.481478in}{3.134522in}}{\pgfqpoint{1.489378in}{3.137795in}}{\pgfqpoint{1.495202in}{3.143619in}}%
\pgfpathcurveto{\pgfqpoint{1.501026in}{3.149443in}}{\pgfqpoint{1.504298in}{3.157343in}}{\pgfqpoint{1.504298in}{3.165579in}}%
\pgfpathcurveto{\pgfqpoint{1.504298in}{3.173815in}}{\pgfqpoint{1.501026in}{3.181715in}}{\pgfqpoint{1.495202in}{3.187539in}}%
\pgfpathcurveto{\pgfqpoint{1.489378in}{3.193363in}}{\pgfqpoint{1.481478in}{3.196635in}}{\pgfqpoint{1.473242in}{3.196635in}}%
\pgfpathcurveto{\pgfqpoint{1.465006in}{3.196635in}}{\pgfqpoint{1.457106in}{3.193363in}}{\pgfqpoint{1.451282in}{3.187539in}}%
\pgfpathcurveto{\pgfqpoint{1.445458in}{3.181715in}}{\pgfqpoint{1.442185in}{3.173815in}}{\pgfqpoint{1.442185in}{3.165579in}}%
\pgfpathcurveto{\pgfqpoint{1.442185in}{3.157343in}}{\pgfqpoint{1.445458in}{3.149443in}}{\pgfqpoint{1.451282in}{3.143619in}}%
\pgfpathcurveto{\pgfqpoint{1.457106in}{3.137795in}}{\pgfqpoint{1.465006in}{3.134522in}}{\pgfqpoint{1.473242in}{3.134522in}}%
\pgfpathclose%
\pgfusepath{stroke,fill}%
\end{pgfscope}%
\begin{pgfscope}%
\pgfpathrectangle{\pgfqpoint{0.100000in}{0.212622in}}{\pgfqpoint{3.696000in}{3.696000in}}%
\pgfusepath{clip}%
\pgfsetbuttcap%
\pgfsetroundjoin%
\definecolor{currentfill}{rgb}{0.121569,0.466667,0.705882}%
\pgfsetfillcolor{currentfill}%
\pgfsetfillopacity{0.407344}%
\pgfsetlinewidth{1.003750pt}%
\definecolor{currentstroke}{rgb}{0.121569,0.466667,0.705882}%
\pgfsetstrokecolor{currentstroke}%
\pgfsetstrokeopacity{0.407344}%
\pgfsetdash{}{0pt}%
\pgfpathmoveto{\pgfqpoint{1.990684in}{2.967029in}}%
\pgfpathcurveto{\pgfqpoint{1.998921in}{2.967029in}}{\pgfqpoint{2.006821in}{2.970301in}}{\pgfqpoint{2.012645in}{2.976125in}}%
\pgfpathcurveto{\pgfqpoint{2.018469in}{2.981949in}}{\pgfqpoint{2.021741in}{2.989849in}}{\pgfqpoint{2.021741in}{2.998086in}}%
\pgfpathcurveto{\pgfqpoint{2.021741in}{3.006322in}}{\pgfqpoint{2.018469in}{3.014222in}}{\pgfqpoint{2.012645in}{3.020046in}}%
\pgfpathcurveto{\pgfqpoint{2.006821in}{3.025870in}}{\pgfqpoint{1.998921in}{3.029142in}}{\pgfqpoint{1.990684in}{3.029142in}}%
\pgfpathcurveto{\pgfqpoint{1.982448in}{3.029142in}}{\pgfqpoint{1.974548in}{3.025870in}}{\pgfqpoint{1.968724in}{3.020046in}}%
\pgfpathcurveto{\pgfqpoint{1.962900in}{3.014222in}}{\pgfqpoint{1.959628in}{3.006322in}}{\pgfqpoint{1.959628in}{2.998086in}}%
\pgfpathcurveto{\pgfqpoint{1.959628in}{2.989849in}}{\pgfqpoint{1.962900in}{2.981949in}}{\pgfqpoint{1.968724in}{2.976125in}}%
\pgfpathcurveto{\pgfqpoint{1.974548in}{2.970301in}}{\pgfqpoint{1.982448in}{2.967029in}}{\pgfqpoint{1.990684in}{2.967029in}}%
\pgfpathclose%
\pgfusepath{stroke,fill}%
\end{pgfscope}%
\begin{pgfscope}%
\pgfpathrectangle{\pgfqpoint{0.100000in}{0.212622in}}{\pgfqpoint{3.696000in}{3.696000in}}%
\pgfusepath{clip}%
\pgfsetbuttcap%
\pgfsetroundjoin%
\definecolor{currentfill}{rgb}{0.121569,0.466667,0.705882}%
\pgfsetfillcolor{currentfill}%
\pgfsetfillopacity{0.407687}%
\pgfsetlinewidth{1.003750pt}%
\definecolor{currentstroke}{rgb}{0.121569,0.466667,0.705882}%
\pgfsetstrokecolor{currentstroke}%
\pgfsetstrokeopacity{0.407687}%
\pgfsetdash{}{0pt}%
\pgfpathmoveto{\pgfqpoint{1.471974in}{3.132210in}}%
\pgfpathcurveto{\pgfqpoint{1.480210in}{3.132210in}}{\pgfqpoint{1.488110in}{3.135482in}}{\pgfqpoint{1.493934in}{3.141306in}}%
\pgfpathcurveto{\pgfqpoint{1.499758in}{3.147130in}}{\pgfqpoint{1.503031in}{3.155030in}}{\pgfqpoint{1.503031in}{3.163266in}}%
\pgfpathcurveto{\pgfqpoint{1.503031in}{3.171502in}}{\pgfqpoint{1.499758in}{3.179403in}}{\pgfqpoint{1.493934in}{3.185226in}}%
\pgfpathcurveto{\pgfqpoint{1.488110in}{3.191050in}}{\pgfqpoint{1.480210in}{3.194323in}}{\pgfqpoint{1.471974in}{3.194323in}}%
\pgfpathcurveto{\pgfqpoint{1.463738in}{3.194323in}}{\pgfqpoint{1.455838in}{3.191050in}}{\pgfqpoint{1.450014in}{3.185226in}}%
\pgfpathcurveto{\pgfqpoint{1.444190in}{3.179403in}}{\pgfqpoint{1.440918in}{3.171502in}}{\pgfqpoint{1.440918in}{3.163266in}}%
\pgfpathcurveto{\pgfqpoint{1.440918in}{3.155030in}}{\pgfqpoint{1.444190in}{3.147130in}}{\pgfqpoint{1.450014in}{3.141306in}}%
\pgfpathcurveto{\pgfqpoint{1.455838in}{3.135482in}}{\pgfqpoint{1.463738in}{3.132210in}}{\pgfqpoint{1.471974in}{3.132210in}}%
\pgfpathclose%
\pgfusepath{stroke,fill}%
\end{pgfscope}%
\begin{pgfscope}%
\pgfpathrectangle{\pgfqpoint{0.100000in}{0.212622in}}{\pgfqpoint{3.696000in}{3.696000in}}%
\pgfusepath{clip}%
\pgfsetbuttcap%
\pgfsetroundjoin%
\definecolor{currentfill}{rgb}{0.121569,0.466667,0.705882}%
\pgfsetfillcolor{currentfill}%
\pgfsetfillopacity{0.409331}%
\pgfsetlinewidth{1.003750pt}%
\definecolor{currentstroke}{rgb}{0.121569,0.466667,0.705882}%
\pgfsetstrokecolor{currentstroke}%
\pgfsetstrokeopacity{0.409331}%
\pgfsetdash{}{0pt}%
\pgfpathmoveto{\pgfqpoint{1.469692in}{3.127993in}}%
\pgfpathcurveto{\pgfqpoint{1.477928in}{3.127993in}}{\pgfqpoint{1.485829in}{3.131265in}}{\pgfqpoint{1.491652in}{3.137089in}}%
\pgfpathcurveto{\pgfqpoint{1.497476in}{3.142913in}}{\pgfqpoint{1.500749in}{3.150813in}}{\pgfqpoint{1.500749in}{3.159049in}}%
\pgfpathcurveto{\pgfqpoint{1.500749in}{3.167285in}}{\pgfqpoint{1.497476in}{3.175185in}}{\pgfqpoint{1.491652in}{3.181009in}}%
\pgfpathcurveto{\pgfqpoint{1.485829in}{3.186833in}}{\pgfqpoint{1.477928in}{3.190106in}}{\pgfqpoint{1.469692in}{3.190106in}}%
\pgfpathcurveto{\pgfqpoint{1.461456in}{3.190106in}}{\pgfqpoint{1.453556in}{3.186833in}}{\pgfqpoint{1.447732in}{3.181009in}}%
\pgfpathcurveto{\pgfqpoint{1.441908in}{3.175185in}}{\pgfqpoint{1.438636in}{3.167285in}}{\pgfqpoint{1.438636in}{3.159049in}}%
\pgfpathcurveto{\pgfqpoint{1.438636in}{3.150813in}}{\pgfqpoint{1.441908in}{3.142913in}}{\pgfqpoint{1.447732in}{3.137089in}}%
\pgfpathcurveto{\pgfqpoint{1.453556in}{3.131265in}}{\pgfqpoint{1.461456in}{3.127993in}}{\pgfqpoint{1.469692in}{3.127993in}}%
\pgfpathclose%
\pgfusepath{stroke,fill}%
\end{pgfscope}%
\begin{pgfscope}%
\pgfpathrectangle{\pgfqpoint{0.100000in}{0.212622in}}{\pgfqpoint{3.696000in}{3.696000in}}%
\pgfusepath{clip}%
\pgfsetbuttcap%
\pgfsetroundjoin%
\definecolor{currentfill}{rgb}{0.121569,0.466667,0.705882}%
\pgfsetfillcolor{currentfill}%
\pgfsetfillopacity{0.410051}%
\pgfsetlinewidth{1.003750pt}%
\definecolor{currentstroke}{rgb}{0.121569,0.466667,0.705882}%
\pgfsetstrokecolor{currentstroke}%
\pgfsetstrokeopacity{0.410051}%
\pgfsetdash{}{0pt}%
\pgfpathmoveto{\pgfqpoint{1.468964in}{3.126220in}}%
\pgfpathcurveto{\pgfqpoint{1.477201in}{3.126220in}}{\pgfqpoint{1.485101in}{3.129492in}}{\pgfqpoint{1.490925in}{3.135316in}}%
\pgfpathcurveto{\pgfqpoint{1.496749in}{3.141140in}}{\pgfqpoint{1.500021in}{3.149040in}}{\pgfqpoint{1.500021in}{3.157276in}}%
\pgfpathcurveto{\pgfqpoint{1.500021in}{3.165512in}}{\pgfqpoint{1.496749in}{3.173412in}}{\pgfqpoint{1.490925in}{3.179236in}}%
\pgfpathcurveto{\pgfqpoint{1.485101in}{3.185060in}}{\pgfqpoint{1.477201in}{3.188333in}}{\pgfqpoint{1.468964in}{3.188333in}}%
\pgfpathcurveto{\pgfqpoint{1.460728in}{3.188333in}}{\pgfqpoint{1.452828in}{3.185060in}}{\pgfqpoint{1.447004in}{3.179236in}}%
\pgfpathcurveto{\pgfqpoint{1.441180in}{3.173412in}}{\pgfqpoint{1.437908in}{3.165512in}}{\pgfqpoint{1.437908in}{3.157276in}}%
\pgfpathcurveto{\pgfqpoint{1.437908in}{3.149040in}}{\pgfqpoint{1.441180in}{3.141140in}}{\pgfqpoint{1.447004in}{3.135316in}}%
\pgfpathcurveto{\pgfqpoint{1.452828in}{3.129492in}}{\pgfqpoint{1.460728in}{3.126220in}}{\pgfqpoint{1.468964in}{3.126220in}}%
\pgfpathclose%
\pgfusepath{stroke,fill}%
\end{pgfscope}%
\begin{pgfscope}%
\pgfpathrectangle{\pgfqpoint{0.100000in}{0.212622in}}{\pgfqpoint{3.696000in}{3.696000in}}%
\pgfusepath{clip}%
\pgfsetbuttcap%
\pgfsetroundjoin%
\definecolor{currentfill}{rgb}{0.121569,0.466667,0.705882}%
\pgfsetfillcolor{currentfill}%
\pgfsetfillopacity{0.411305}%
\pgfsetlinewidth{1.003750pt}%
\definecolor{currentstroke}{rgb}{0.121569,0.466667,0.705882}%
\pgfsetstrokecolor{currentstroke}%
\pgfsetstrokeopacity{0.411305}%
\pgfsetdash{}{0pt}%
\pgfpathmoveto{\pgfqpoint{1.467338in}{3.122992in}}%
\pgfpathcurveto{\pgfqpoint{1.475575in}{3.122992in}}{\pgfqpoint{1.483475in}{3.126264in}}{\pgfqpoint{1.489299in}{3.132088in}}%
\pgfpathcurveto{\pgfqpoint{1.495123in}{3.137912in}}{\pgfqpoint{1.498395in}{3.145812in}}{\pgfqpoint{1.498395in}{3.154048in}}%
\pgfpathcurveto{\pgfqpoint{1.498395in}{3.162284in}}{\pgfqpoint{1.495123in}{3.170184in}}{\pgfqpoint{1.489299in}{3.176008in}}%
\pgfpathcurveto{\pgfqpoint{1.483475in}{3.181832in}}{\pgfqpoint{1.475575in}{3.185105in}}{\pgfqpoint{1.467338in}{3.185105in}}%
\pgfpathcurveto{\pgfqpoint{1.459102in}{3.185105in}}{\pgfqpoint{1.451202in}{3.181832in}}{\pgfqpoint{1.445378in}{3.176008in}}%
\pgfpathcurveto{\pgfqpoint{1.439554in}{3.170184in}}{\pgfqpoint{1.436282in}{3.162284in}}{\pgfqpoint{1.436282in}{3.154048in}}%
\pgfpathcurveto{\pgfqpoint{1.436282in}{3.145812in}}{\pgfqpoint{1.439554in}{3.137912in}}{\pgfqpoint{1.445378in}{3.132088in}}%
\pgfpathcurveto{\pgfqpoint{1.451202in}{3.126264in}}{\pgfqpoint{1.459102in}{3.122992in}}{\pgfqpoint{1.467338in}{3.122992in}}%
\pgfpathclose%
\pgfusepath{stroke,fill}%
\end{pgfscope}%
\begin{pgfscope}%
\pgfpathrectangle{\pgfqpoint{0.100000in}{0.212622in}}{\pgfqpoint{3.696000in}{3.696000in}}%
\pgfusepath{clip}%
\pgfsetbuttcap%
\pgfsetroundjoin%
\definecolor{currentfill}{rgb}{0.121569,0.466667,0.705882}%
\pgfsetfillcolor{currentfill}%
\pgfsetfillopacity{0.411763}%
\pgfsetlinewidth{1.003750pt}%
\definecolor{currentstroke}{rgb}{0.121569,0.466667,0.705882}%
\pgfsetstrokecolor{currentstroke}%
\pgfsetstrokeopacity{0.411763}%
\pgfsetdash{}{0pt}%
\pgfpathmoveto{\pgfqpoint{1.466794in}{3.121802in}}%
\pgfpathcurveto{\pgfqpoint{1.475031in}{3.121802in}}{\pgfqpoint{1.482931in}{3.125074in}}{\pgfqpoint{1.488755in}{3.130898in}}%
\pgfpathcurveto{\pgfqpoint{1.494579in}{3.136722in}}{\pgfqpoint{1.497851in}{3.144622in}}{\pgfqpoint{1.497851in}{3.152858in}}%
\pgfpathcurveto{\pgfqpoint{1.497851in}{3.161094in}}{\pgfqpoint{1.494579in}{3.168994in}}{\pgfqpoint{1.488755in}{3.174818in}}%
\pgfpathcurveto{\pgfqpoint{1.482931in}{3.180642in}}{\pgfqpoint{1.475031in}{3.183915in}}{\pgfqpoint{1.466794in}{3.183915in}}%
\pgfpathcurveto{\pgfqpoint{1.458558in}{3.183915in}}{\pgfqpoint{1.450658in}{3.180642in}}{\pgfqpoint{1.444834in}{3.174818in}}%
\pgfpathcurveto{\pgfqpoint{1.439010in}{3.168994in}}{\pgfqpoint{1.435738in}{3.161094in}}{\pgfqpoint{1.435738in}{3.152858in}}%
\pgfpathcurveto{\pgfqpoint{1.435738in}{3.144622in}}{\pgfqpoint{1.439010in}{3.136722in}}{\pgfqpoint{1.444834in}{3.130898in}}%
\pgfpathcurveto{\pgfqpoint{1.450658in}{3.125074in}}{\pgfqpoint{1.458558in}{3.121802in}}{\pgfqpoint{1.466794in}{3.121802in}}%
\pgfpathclose%
\pgfusepath{stroke,fill}%
\end{pgfscope}%
\begin{pgfscope}%
\pgfpathrectangle{\pgfqpoint{0.100000in}{0.212622in}}{\pgfqpoint{3.696000in}{3.696000in}}%
\pgfusepath{clip}%
\pgfsetbuttcap%
\pgfsetroundjoin%
\definecolor{currentfill}{rgb}{0.121569,0.466667,0.705882}%
\pgfsetfillcolor{currentfill}%
\pgfsetfillopacity{0.412583}%
\pgfsetlinewidth{1.003750pt}%
\definecolor{currentstroke}{rgb}{0.121569,0.466667,0.705882}%
\pgfsetstrokecolor{currentstroke}%
\pgfsetstrokeopacity{0.412583}%
\pgfsetdash{}{0pt}%
\pgfpathmoveto{\pgfqpoint{1.465714in}{3.119661in}}%
\pgfpathcurveto{\pgfqpoint{1.473950in}{3.119661in}}{\pgfqpoint{1.481850in}{3.122933in}}{\pgfqpoint{1.487674in}{3.128757in}}%
\pgfpathcurveto{\pgfqpoint{1.493498in}{3.134581in}}{\pgfqpoint{1.496770in}{3.142481in}}{\pgfqpoint{1.496770in}{3.150717in}}%
\pgfpathcurveto{\pgfqpoint{1.496770in}{3.158954in}}{\pgfqpoint{1.493498in}{3.166854in}}{\pgfqpoint{1.487674in}{3.172678in}}%
\pgfpathcurveto{\pgfqpoint{1.481850in}{3.178502in}}{\pgfqpoint{1.473950in}{3.181774in}}{\pgfqpoint{1.465714in}{3.181774in}}%
\pgfpathcurveto{\pgfqpoint{1.457478in}{3.181774in}}{\pgfqpoint{1.449578in}{3.178502in}}{\pgfqpoint{1.443754in}{3.172678in}}%
\pgfpathcurveto{\pgfqpoint{1.437930in}{3.166854in}}{\pgfqpoint{1.434657in}{3.158954in}}{\pgfqpoint{1.434657in}{3.150717in}}%
\pgfpathcurveto{\pgfqpoint{1.434657in}{3.142481in}}{\pgfqpoint{1.437930in}{3.134581in}}{\pgfqpoint{1.443754in}{3.128757in}}%
\pgfpathcurveto{\pgfqpoint{1.449578in}{3.122933in}}{\pgfqpoint{1.457478in}{3.119661in}}{\pgfqpoint{1.465714in}{3.119661in}}%
\pgfpathclose%
\pgfusepath{stroke,fill}%
\end{pgfscope}%
\begin{pgfscope}%
\pgfpathrectangle{\pgfqpoint{0.100000in}{0.212622in}}{\pgfqpoint{3.696000in}{3.696000in}}%
\pgfusepath{clip}%
\pgfsetbuttcap%
\pgfsetroundjoin%
\definecolor{currentfill}{rgb}{0.121569,0.466667,0.705882}%
\pgfsetfillcolor{currentfill}%
\pgfsetfillopacity{0.414082}%
\pgfsetlinewidth{1.003750pt}%
\definecolor{currentstroke}{rgb}{0.121569,0.466667,0.705882}%
\pgfsetstrokecolor{currentstroke}%
\pgfsetstrokeopacity{0.414082}%
\pgfsetdash{}{0pt}%
\pgfpathmoveto{\pgfqpoint{1.463833in}{3.115726in}}%
\pgfpathcurveto{\pgfqpoint{1.472069in}{3.115726in}}{\pgfqpoint{1.479969in}{3.118998in}}{\pgfqpoint{1.485793in}{3.124822in}}%
\pgfpathcurveto{\pgfqpoint{1.491617in}{3.130646in}}{\pgfqpoint{1.494890in}{3.138546in}}{\pgfqpoint{1.494890in}{3.146783in}}%
\pgfpathcurveto{\pgfqpoint{1.494890in}{3.155019in}}{\pgfqpoint{1.491617in}{3.162919in}}{\pgfqpoint{1.485793in}{3.168743in}}%
\pgfpathcurveto{\pgfqpoint{1.479969in}{3.174567in}}{\pgfqpoint{1.472069in}{3.177839in}}{\pgfqpoint{1.463833in}{3.177839in}}%
\pgfpathcurveto{\pgfqpoint{1.455597in}{3.177839in}}{\pgfqpoint{1.447697in}{3.174567in}}{\pgfqpoint{1.441873in}{3.168743in}}%
\pgfpathcurveto{\pgfqpoint{1.436049in}{3.162919in}}{\pgfqpoint{1.432777in}{3.155019in}}{\pgfqpoint{1.432777in}{3.146783in}}%
\pgfpathcurveto{\pgfqpoint{1.432777in}{3.138546in}}{\pgfqpoint{1.436049in}{3.130646in}}{\pgfqpoint{1.441873in}{3.124822in}}%
\pgfpathcurveto{\pgfqpoint{1.447697in}{3.118998in}}{\pgfqpoint{1.455597in}{3.115726in}}{\pgfqpoint{1.463833in}{3.115726in}}%
\pgfpathclose%
\pgfusepath{stroke,fill}%
\end{pgfscope}%
\begin{pgfscope}%
\pgfpathrectangle{\pgfqpoint{0.100000in}{0.212622in}}{\pgfqpoint{3.696000in}{3.696000in}}%
\pgfusepath{clip}%
\pgfsetbuttcap%
\pgfsetroundjoin%
\definecolor{currentfill}{rgb}{0.121569,0.466667,0.705882}%
\pgfsetfillcolor{currentfill}%
\pgfsetfillopacity{0.416765}%
\pgfsetlinewidth{1.003750pt}%
\definecolor{currentstroke}{rgb}{0.121569,0.466667,0.705882}%
\pgfsetstrokecolor{currentstroke}%
\pgfsetstrokeopacity{0.416765}%
\pgfsetdash{}{0pt}%
\pgfpathmoveto{\pgfqpoint{1.460366in}{3.108435in}}%
\pgfpathcurveto{\pgfqpoint{1.468602in}{3.108435in}}{\pgfqpoint{1.476502in}{3.111707in}}{\pgfqpoint{1.482326in}{3.117531in}}%
\pgfpathcurveto{\pgfqpoint{1.488150in}{3.123355in}}{\pgfqpoint{1.491422in}{3.131255in}}{\pgfqpoint{1.491422in}{3.139491in}}%
\pgfpathcurveto{\pgfqpoint{1.491422in}{3.147727in}}{\pgfqpoint{1.488150in}{3.155627in}}{\pgfqpoint{1.482326in}{3.161451in}}%
\pgfpathcurveto{\pgfqpoint{1.476502in}{3.167275in}}{\pgfqpoint{1.468602in}{3.170548in}}{\pgfqpoint{1.460366in}{3.170548in}}%
\pgfpathcurveto{\pgfqpoint{1.452129in}{3.170548in}}{\pgfqpoint{1.444229in}{3.167275in}}{\pgfqpoint{1.438405in}{3.161451in}}%
\pgfpathcurveto{\pgfqpoint{1.432581in}{3.155627in}}{\pgfqpoint{1.429309in}{3.147727in}}{\pgfqpoint{1.429309in}{3.139491in}}%
\pgfpathcurveto{\pgfqpoint{1.429309in}{3.131255in}}{\pgfqpoint{1.432581in}{3.123355in}}{\pgfqpoint{1.438405in}{3.117531in}}%
\pgfpathcurveto{\pgfqpoint{1.444229in}{3.111707in}}{\pgfqpoint{1.452129in}{3.108435in}}{\pgfqpoint{1.460366in}{3.108435in}}%
\pgfpathclose%
\pgfusepath{stroke,fill}%
\end{pgfscope}%
\begin{pgfscope}%
\pgfpathrectangle{\pgfqpoint{0.100000in}{0.212622in}}{\pgfqpoint{3.696000in}{3.696000in}}%
\pgfusepath{clip}%
\pgfsetbuttcap%
\pgfsetroundjoin%
\definecolor{currentfill}{rgb}{0.121569,0.466667,0.705882}%
\pgfsetfillcolor{currentfill}%
\pgfsetfillopacity{0.417745}%
\pgfsetlinewidth{1.003750pt}%
\definecolor{currentstroke}{rgb}{0.121569,0.466667,0.705882}%
\pgfsetstrokecolor{currentstroke}%
\pgfsetstrokeopacity{0.417745}%
\pgfsetdash{}{0pt}%
\pgfpathmoveto{\pgfqpoint{2.009347in}{2.931958in}}%
\pgfpathcurveto{\pgfqpoint{2.017583in}{2.931958in}}{\pgfqpoint{2.025483in}{2.935230in}}{\pgfqpoint{2.031307in}{2.941054in}}%
\pgfpathcurveto{\pgfqpoint{2.037131in}{2.946878in}}{\pgfqpoint{2.040403in}{2.954778in}}{\pgfqpoint{2.040403in}{2.963015in}}%
\pgfpathcurveto{\pgfqpoint{2.040403in}{2.971251in}}{\pgfqpoint{2.037131in}{2.979151in}}{\pgfqpoint{2.031307in}{2.984975in}}%
\pgfpathcurveto{\pgfqpoint{2.025483in}{2.990799in}}{\pgfqpoint{2.017583in}{2.994071in}}{\pgfqpoint{2.009347in}{2.994071in}}%
\pgfpathcurveto{\pgfqpoint{2.001110in}{2.994071in}}{\pgfqpoint{1.993210in}{2.990799in}}{\pgfqpoint{1.987386in}{2.984975in}}%
\pgfpathcurveto{\pgfqpoint{1.981562in}{2.979151in}}{\pgfqpoint{1.978290in}{2.971251in}}{\pgfqpoint{1.978290in}{2.963015in}}%
\pgfpathcurveto{\pgfqpoint{1.978290in}{2.954778in}}{\pgfqpoint{1.981562in}{2.946878in}}{\pgfqpoint{1.987386in}{2.941054in}}%
\pgfpathcurveto{\pgfqpoint{1.993210in}{2.935230in}}{\pgfqpoint{2.001110in}{2.931958in}}{\pgfqpoint{2.009347in}{2.931958in}}%
\pgfpathclose%
\pgfusepath{stroke,fill}%
\end{pgfscope}%
\begin{pgfscope}%
\pgfpathrectangle{\pgfqpoint{0.100000in}{0.212622in}}{\pgfqpoint{3.696000in}{3.696000in}}%
\pgfusepath{clip}%
\pgfsetbuttcap%
\pgfsetroundjoin%
\definecolor{currentfill}{rgb}{0.121569,0.466667,0.705882}%
\pgfsetfillcolor{currentfill}%
\pgfsetfillopacity{0.421621}%
\pgfsetlinewidth{1.003750pt}%
\definecolor{currentstroke}{rgb}{0.121569,0.466667,0.705882}%
\pgfsetstrokecolor{currentstroke}%
\pgfsetstrokeopacity{0.421621}%
\pgfsetdash{}{0pt}%
\pgfpathmoveto{\pgfqpoint{1.453808in}{3.095248in}}%
\pgfpathcurveto{\pgfqpoint{1.462044in}{3.095248in}}{\pgfqpoint{1.469944in}{3.098520in}}{\pgfqpoint{1.475768in}{3.104344in}}%
\pgfpathcurveto{\pgfqpoint{1.481592in}{3.110168in}}{\pgfqpoint{1.484864in}{3.118068in}}{\pgfqpoint{1.484864in}{3.126304in}}%
\pgfpathcurveto{\pgfqpoint{1.484864in}{3.134541in}}{\pgfqpoint{1.481592in}{3.142441in}}{\pgfqpoint{1.475768in}{3.148265in}}%
\pgfpathcurveto{\pgfqpoint{1.469944in}{3.154088in}}{\pgfqpoint{1.462044in}{3.157361in}}{\pgfqpoint{1.453808in}{3.157361in}}%
\pgfpathcurveto{\pgfqpoint{1.445572in}{3.157361in}}{\pgfqpoint{1.437671in}{3.154088in}}{\pgfqpoint{1.431848in}{3.148265in}}%
\pgfpathcurveto{\pgfqpoint{1.426024in}{3.142441in}}{\pgfqpoint{1.422751in}{3.134541in}}{\pgfqpoint{1.422751in}{3.126304in}}%
\pgfpathcurveto{\pgfqpoint{1.422751in}{3.118068in}}{\pgfqpoint{1.426024in}{3.110168in}}{\pgfqpoint{1.431848in}{3.104344in}}%
\pgfpathcurveto{\pgfqpoint{1.437671in}{3.098520in}}{\pgfqpoint{1.445572in}{3.095248in}}{\pgfqpoint{1.453808in}{3.095248in}}%
\pgfpathclose%
\pgfusepath{stroke,fill}%
\end{pgfscope}%
\begin{pgfscope}%
\pgfpathrectangle{\pgfqpoint{0.100000in}{0.212622in}}{\pgfqpoint{3.696000in}{3.696000in}}%
\pgfusepath{clip}%
\pgfsetbuttcap%
\pgfsetroundjoin%
\definecolor{currentfill}{rgb}{0.121569,0.466667,0.705882}%
\pgfsetfillcolor{currentfill}%
\pgfsetfillopacity{0.429142}%
\pgfsetlinewidth{1.003750pt}%
\definecolor{currentstroke}{rgb}{0.121569,0.466667,0.705882}%
\pgfsetstrokecolor{currentstroke}%
\pgfsetstrokeopacity{0.429142}%
\pgfsetdash{}{0pt}%
\pgfpathmoveto{\pgfqpoint{2.030353in}{2.893109in}}%
\pgfpathcurveto{\pgfqpoint{2.038589in}{2.893109in}}{\pgfqpoint{2.046489in}{2.896381in}}{\pgfqpoint{2.052313in}{2.902205in}}%
\pgfpathcurveto{\pgfqpoint{2.058137in}{2.908029in}}{\pgfqpoint{2.061409in}{2.915929in}}{\pgfqpoint{2.061409in}{2.924165in}}%
\pgfpathcurveto{\pgfqpoint{2.061409in}{2.932401in}}{\pgfqpoint{2.058137in}{2.940301in}}{\pgfqpoint{2.052313in}{2.946125in}}%
\pgfpathcurveto{\pgfqpoint{2.046489in}{2.951949in}}{\pgfqpoint{2.038589in}{2.955222in}}{\pgfqpoint{2.030353in}{2.955222in}}%
\pgfpathcurveto{\pgfqpoint{2.022117in}{2.955222in}}{\pgfqpoint{2.014216in}{2.951949in}}{\pgfqpoint{2.008393in}{2.946125in}}%
\pgfpathcurveto{\pgfqpoint{2.002569in}{2.940301in}}{\pgfqpoint{1.999296in}{2.932401in}}{\pgfqpoint{1.999296in}{2.924165in}}%
\pgfpathcurveto{\pgfqpoint{1.999296in}{2.915929in}}{\pgfqpoint{2.002569in}{2.908029in}}{\pgfqpoint{2.008393in}{2.902205in}}%
\pgfpathcurveto{\pgfqpoint{2.014216in}{2.896381in}}{\pgfqpoint{2.022117in}{2.893109in}}{\pgfqpoint{2.030353in}{2.893109in}}%
\pgfpathclose%
\pgfusepath{stroke,fill}%
\end{pgfscope}%
\begin{pgfscope}%
\pgfpathrectangle{\pgfqpoint{0.100000in}{0.212622in}}{\pgfqpoint{3.696000in}{3.696000in}}%
\pgfusepath{clip}%
\pgfsetbuttcap%
\pgfsetroundjoin%
\definecolor{currentfill}{rgb}{0.121569,0.466667,0.705882}%
\pgfsetfillcolor{currentfill}%
\pgfsetfillopacity{0.430510}%
\pgfsetlinewidth{1.003750pt}%
\definecolor{currentstroke}{rgb}{0.121569,0.466667,0.705882}%
\pgfsetstrokecolor{currentstroke}%
\pgfsetstrokeopacity{0.430510}%
\pgfsetdash{}{0pt}%
\pgfpathmoveto{\pgfqpoint{1.442498in}{3.070930in}}%
\pgfpathcurveto{\pgfqpoint{1.450734in}{3.070930in}}{\pgfqpoint{1.458634in}{3.074202in}}{\pgfqpoint{1.464458in}{3.080026in}}%
\pgfpathcurveto{\pgfqpoint{1.470282in}{3.085850in}}{\pgfqpoint{1.473554in}{3.093750in}}{\pgfqpoint{1.473554in}{3.101987in}}%
\pgfpathcurveto{\pgfqpoint{1.473554in}{3.110223in}}{\pgfqpoint{1.470282in}{3.118123in}}{\pgfqpoint{1.464458in}{3.123947in}}%
\pgfpathcurveto{\pgfqpoint{1.458634in}{3.129771in}}{\pgfqpoint{1.450734in}{3.133043in}}{\pgfqpoint{1.442498in}{3.133043in}}%
\pgfpathcurveto{\pgfqpoint{1.434262in}{3.133043in}}{\pgfqpoint{1.426361in}{3.129771in}}{\pgfqpoint{1.420538in}{3.123947in}}%
\pgfpathcurveto{\pgfqpoint{1.414714in}{3.118123in}}{\pgfqpoint{1.411441in}{3.110223in}}{\pgfqpoint{1.411441in}{3.101987in}}%
\pgfpathcurveto{\pgfqpoint{1.411441in}{3.093750in}}{\pgfqpoint{1.414714in}{3.085850in}}{\pgfqpoint{1.420538in}{3.080026in}}%
\pgfpathcurveto{\pgfqpoint{1.426361in}{3.074202in}}{\pgfqpoint{1.434262in}{3.070930in}}{\pgfqpoint{1.442498in}{3.070930in}}%
\pgfpathclose%
\pgfusepath{stroke,fill}%
\end{pgfscope}%
\begin{pgfscope}%
\pgfpathrectangle{\pgfqpoint{0.100000in}{0.212622in}}{\pgfqpoint{3.696000in}{3.696000in}}%
\pgfusepath{clip}%
\pgfsetbuttcap%
\pgfsetroundjoin%
\definecolor{currentfill}{rgb}{0.121569,0.466667,0.705882}%
\pgfsetfillcolor{currentfill}%
\pgfsetfillopacity{0.432141}%
\pgfsetlinewidth{1.003750pt}%
\definecolor{currentstroke}{rgb}{0.121569,0.466667,0.705882}%
\pgfsetstrokecolor{currentstroke}%
\pgfsetstrokeopacity{0.432141}%
\pgfsetdash{}{0pt}%
\pgfpathmoveto{\pgfqpoint{1.440325in}{3.066582in}}%
\pgfpathcurveto{\pgfqpoint{1.448561in}{3.066582in}}{\pgfqpoint{1.456461in}{3.069854in}}{\pgfqpoint{1.462285in}{3.075678in}}%
\pgfpathcurveto{\pgfqpoint{1.468109in}{3.081502in}}{\pgfqpoint{1.471382in}{3.089402in}}{\pgfqpoint{1.471382in}{3.097639in}}%
\pgfpathcurveto{\pgfqpoint{1.471382in}{3.105875in}}{\pgfqpoint{1.468109in}{3.113775in}}{\pgfqpoint{1.462285in}{3.119599in}}%
\pgfpathcurveto{\pgfqpoint{1.456461in}{3.125423in}}{\pgfqpoint{1.448561in}{3.128695in}}{\pgfqpoint{1.440325in}{3.128695in}}%
\pgfpathcurveto{\pgfqpoint{1.432089in}{3.128695in}}{\pgfqpoint{1.424189in}{3.125423in}}{\pgfqpoint{1.418365in}{3.119599in}}%
\pgfpathcurveto{\pgfqpoint{1.412541in}{3.113775in}}{\pgfqpoint{1.409269in}{3.105875in}}{\pgfqpoint{1.409269in}{3.097639in}}%
\pgfpathcurveto{\pgfqpoint{1.409269in}{3.089402in}}{\pgfqpoint{1.412541in}{3.081502in}}{\pgfqpoint{1.418365in}{3.075678in}}%
\pgfpathcurveto{\pgfqpoint{1.424189in}{3.069854in}}{\pgfqpoint{1.432089in}{3.066582in}}{\pgfqpoint{1.440325in}{3.066582in}}%
\pgfpathclose%
\pgfusepath{stroke,fill}%
\end{pgfscope}%
\begin{pgfscope}%
\pgfpathrectangle{\pgfqpoint{0.100000in}{0.212622in}}{\pgfqpoint{3.696000in}{3.696000in}}%
\pgfusepath{clip}%
\pgfsetbuttcap%
\pgfsetroundjoin%
\definecolor{currentfill}{rgb}{0.121569,0.466667,0.705882}%
\pgfsetfillcolor{currentfill}%
\pgfsetfillopacity{0.433056}%
\pgfsetlinewidth{1.003750pt}%
\definecolor{currentstroke}{rgb}{0.121569,0.466667,0.705882}%
\pgfsetstrokecolor{currentstroke}%
\pgfsetstrokeopacity{0.433056}%
\pgfsetdash{}{0pt}%
\pgfpathmoveto{\pgfqpoint{1.439290in}{3.064125in}}%
\pgfpathcurveto{\pgfqpoint{1.447526in}{3.064125in}}{\pgfqpoint{1.455426in}{3.067398in}}{\pgfqpoint{1.461250in}{3.073222in}}%
\pgfpathcurveto{\pgfqpoint{1.467074in}{3.079045in}}{\pgfqpoint{1.470346in}{3.086945in}}{\pgfqpoint{1.470346in}{3.095182in}}%
\pgfpathcurveto{\pgfqpoint{1.470346in}{3.103418in}}{\pgfqpoint{1.467074in}{3.111318in}}{\pgfqpoint{1.461250in}{3.117142in}}%
\pgfpathcurveto{\pgfqpoint{1.455426in}{3.122966in}}{\pgfqpoint{1.447526in}{3.126238in}}{\pgfqpoint{1.439290in}{3.126238in}}%
\pgfpathcurveto{\pgfqpoint{1.431053in}{3.126238in}}{\pgfqpoint{1.423153in}{3.122966in}}{\pgfqpoint{1.417329in}{3.117142in}}%
\pgfpathcurveto{\pgfqpoint{1.411505in}{3.111318in}}{\pgfqpoint{1.408233in}{3.103418in}}{\pgfqpoint{1.408233in}{3.095182in}}%
\pgfpathcurveto{\pgfqpoint{1.408233in}{3.086945in}}{\pgfqpoint{1.411505in}{3.079045in}}{\pgfqpoint{1.417329in}{3.073222in}}%
\pgfpathcurveto{\pgfqpoint{1.423153in}{3.067398in}}{\pgfqpoint{1.431053in}{3.064125in}}{\pgfqpoint{1.439290in}{3.064125in}}%
\pgfpathclose%
\pgfusepath{stroke,fill}%
\end{pgfscope}%
\begin{pgfscope}%
\pgfpathrectangle{\pgfqpoint{0.100000in}{0.212622in}}{\pgfqpoint{3.696000in}{3.696000in}}%
\pgfusepath{clip}%
\pgfsetbuttcap%
\pgfsetroundjoin%
\definecolor{currentfill}{rgb}{0.121569,0.466667,0.705882}%
\pgfsetfillcolor{currentfill}%
\pgfsetfillopacity{0.434686}%
\pgfsetlinewidth{1.003750pt}%
\definecolor{currentstroke}{rgb}{0.121569,0.466667,0.705882}%
\pgfsetstrokecolor{currentstroke}%
\pgfsetstrokeopacity{0.434686}%
\pgfsetdash{}{0pt}%
\pgfpathmoveto{\pgfqpoint{1.437059in}{3.059783in}}%
\pgfpathcurveto{\pgfqpoint{1.445295in}{3.059783in}}{\pgfqpoint{1.453195in}{3.063056in}}{\pgfqpoint{1.459019in}{3.068880in}}%
\pgfpathcurveto{\pgfqpoint{1.464843in}{3.074704in}}{\pgfqpoint{1.468115in}{3.082604in}}{\pgfqpoint{1.468115in}{3.090840in}}%
\pgfpathcurveto{\pgfqpoint{1.468115in}{3.099076in}}{\pgfqpoint{1.464843in}{3.106976in}}{\pgfqpoint{1.459019in}{3.112800in}}%
\pgfpathcurveto{\pgfqpoint{1.453195in}{3.118624in}}{\pgfqpoint{1.445295in}{3.121896in}}{\pgfqpoint{1.437059in}{3.121896in}}%
\pgfpathcurveto{\pgfqpoint{1.428822in}{3.121896in}}{\pgfqpoint{1.420922in}{3.118624in}}{\pgfqpoint{1.415098in}{3.112800in}}%
\pgfpathcurveto{\pgfqpoint{1.409275in}{3.106976in}}{\pgfqpoint{1.406002in}{3.099076in}}{\pgfqpoint{1.406002in}{3.090840in}}%
\pgfpathcurveto{\pgfqpoint{1.406002in}{3.082604in}}{\pgfqpoint{1.409275in}{3.074704in}}{\pgfqpoint{1.415098in}{3.068880in}}%
\pgfpathcurveto{\pgfqpoint{1.420922in}{3.063056in}}{\pgfqpoint{1.428822in}{3.059783in}}{\pgfqpoint{1.437059in}{3.059783in}}%
\pgfpathclose%
\pgfusepath{stroke,fill}%
\end{pgfscope}%
\begin{pgfscope}%
\pgfpathrectangle{\pgfqpoint{0.100000in}{0.212622in}}{\pgfqpoint{3.696000in}{3.696000in}}%
\pgfusepath{clip}%
\pgfsetbuttcap%
\pgfsetroundjoin%
\definecolor{currentfill}{rgb}{0.121569,0.466667,0.705882}%
\pgfsetfillcolor{currentfill}%
\pgfsetfillopacity{0.435638}%
\pgfsetlinewidth{1.003750pt}%
\definecolor{currentstroke}{rgb}{0.121569,0.466667,0.705882}%
\pgfsetstrokecolor{currentstroke}%
\pgfsetstrokeopacity{0.435638}%
\pgfsetdash{}{0pt}%
\pgfpathmoveto{\pgfqpoint{1.435969in}{3.057154in}}%
\pgfpathcurveto{\pgfqpoint{1.444206in}{3.057154in}}{\pgfqpoint{1.452106in}{3.060427in}}{\pgfqpoint{1.457930in}{3.066251in}}%
\pgfpathcurveto{\pgfqpoint{1.463754in}{3.072074in}}{\pgfqpoint{1.467026in}{3.079974in}}{\pgfqpoint{1.467026in}{3.088211in}}%
\pgfpathcurveto{\pgfqpoint{1.467026in}{3.096447in}}{\pgfqpoint{1.463754in}{3.104347in}}{\pgfqpoint{1.457930in}{3.110171in}}%
\pgfpathcurveto{\pgfqpoint{1.452106in}{3.115995in}}{\pgfqpoint{1.444206in}{3.119267in}}{\pgfqpoint{1.435969in}{3.119267in}}%
\pgfpathcurveto{\pgfqpoint{1.427733in}{3.119267in}}{\pgfqpoint{1.419833in}{3.115995in}}{\pgfqpoint{1.414009in}{3.110171in}}%
\pgfpathcurveto{\pgfqpoint{1.408185in}{3.104347in}}{\pgfqpoint{1.404913in}{3.096447in}}{\pgfqpoint{1.404913in}{3.088211in}}%
\pgfpathcurveto{\pgfqpoint{1.404913in}{3.079974in}}{\pgfqpoint{1.408185in}{3.072074in}}{\pgfqpoint{1.414009in}{3.066251in}}%
\pgfpathcurveto{\pgfqpoint{1.419833in}{3.060427in}}{\pgfqpoint{1.427733in}{3.057154in}}{\pgfqpoint{1.435969in}{3.057154in}}%
\pgfpathclose%
\pgfusepath{stroke,fill}%
\end{pgfscope}%
\begin{pgfscope}%
\pgfpathrectangle{\pgfqpoint{0.100000in}{0.212622in}}{\pgfqpoint{3.696000in}{3.696000in}}%
\pgfusepath{clip}%
\pgfsetbuttcap%
\pgfsetroundjoin%
\definecolor{currentfill}{rgb}{0.121569,0.466667,0.705882}%
\pgfsetfillcolor{currentfill}%
\pgfsetfillopacity{0.437402}%
\pgfsetlinewidth{1.003750pt}%
\definecolor{currentstroke}{rgb}{0.121569,0.466667,0.705882}%
\pgfsetstrokecolor{currentstroke}%
\pgfsetstrokeopacity{0.437402}%
\pgfsetdash{}{0pt}%
\pgfpathmoveto{\pgfqpoint{1.433643in}{3.052763in}}%
\pgfpathcurveto{\pgfqpoint{1.441879in}{3.052763in}}{\pgfqpoint{1.449779in}{3.056036in}}{\pgfqpoint{1.455603in}{3.061860in}}%
\pgfpathcurveto{\pgfqpoint{1.461427in}{3.067684in}}{\pgfqpoint{1.464699in}{3.075584in}}{\pgfqpoint{1.464699in}{3.083820in}}%
\pgfpathcurveto{\pgfqpoint{1.464699in}{3.092056in}}{\pgfqpoint{1.461427in}{3.099956in}}{\pgfqpoint{1.455603in}{3.105780in}}%
\pgfpathcurveto{\pgfqpoint{1.449779in}{3.111604in}}{\pgfqpoint{1.441879in}{3.114876in}}{\pgfqpoint{1.433643in}{3.114876in}}%
\pgfpathcurveto{\pgfqpoint{1.425406in}{3.114876in}}{\pgfqpoint{1.417506in}{3.111604in}}{\pgfqpoint{1.411682in}{3.105780in}}%
\pgfpathcurveto{\pgfqpoint{1.405859in}{3.099956in}}{\pgfqpoint{1.402586in}{3.092056in}}{\pgfqpoint{1.402586in}{3.083820in}}%
\pgfpathcurveto{\pgfqpoint{1.402586in}{3.075584in}}{\pgfqpoint{1.405859in}{3.067684in}}{\pgfqpoint{1.411682in}{3.061860in}}%
\pgfpathcurveto{\pgfqpoint{1.417506in}{3.056036in}}{\pgfqpoint{1.425406in}{3.052763in}}{\pgfqpoint{1.433643in}{3.052763in}}%
\pgfpathclose%
\pgfusepath{stroke,fill}%
\end{pgfscope}%
\begin{pgfscope}%
\pgfpathrectangle{\pgfqpoint{0.100000in}{0.212622in}}{\pgfqpoint{3.696000in}{3.696000in}}%
\pgfusepath{clip}%
\pgfsetbuttcap%
\pgfsetroundjoin%
\definecolor{currentfill}{rgb}{0.121569,0.466667,0.705882}%
\pgfsetfillcolor{currentfill}%
\pgfsetfillopacity{0.438398}%
\pgfsetlinewidth{1.003750pt}%
\definecolor{currentstroke}{rgb}{0.121569,0.466667,0.705882}%
\pgfsetstrokecolor{currentstroke}%
\pgfsetstrokeopacity{0.438398}%
\pgfsetdash{}{0pt}%
\pgfpathmoveto{\pgfqpoint{1.432513in}{3.050220in}}%
\pgfpathcurveto{\pgfqpoint{1.440749in}{3.050220in}}{\pgfqpoint{1.448649in}{3.053493in}}{\pgfqpoint{1.454473in}{3.059317in}}%
\pgfpathcurveto{\pgfqpoint{1.460297in}{3.065141in}}{\pgfqpoint{1.463569in}{3.073041in}}{\pgfqpoint{1.463569in}{3.081277in}}%
\pgfpathcurveto{\pgfqpoint{1.463569in}{3.089513in}}{\pgfqpoint{1.460297in}{3.097413in}}{\pgfqpoint{1.454473in}{3.103237in}}%
\pgfpathcurveto{\pgfqpoint{1.448649in}{3.109061in}}{\pgfqpoint{1.440749in}{3.112333in}}{\pgfqpoint{1.432513in}{3.112333in}}%
\pgfpathcurveto{\pgfqpoint{1.424277in}{3.112333in}}{\pgfqpoint{1.416376in}{3.109061in}}{\pgfqpoint{1.410553in}{3.103237in}}%
\pgfpathcurveto{\pgfqpoint{1.404729in}{3.097413in}}{\pgfqpoint{1.401456in}{3.089513in}}{\pgfqpoint{1.401456in}{3.081277in}}%
\pgfpathcurveto{\pgfqpoint{1.401456in}{3.073041in}}{\pgfqpoint{1.404729in}{3.065141in}}{\pgfqpoint{1.410553in}{3.059317in}}%
\pgfpathcurveto{\pgfqpoint{1.416376in}{3.053493in}}{\pgfqpoint{1.424277in}{3.050220in}}{\pgfqpoint{1.432513in}{3.050220in}}%
\pgfpathclose%
\pgfusepath{stroke,fill}%
\end{pgfscope}%
\begin{pgfscope}%
\pgfpathrectangle{\pgfqpoint{0.100000in}{0.212622in}}{\pgfqpoint{3.696000in}{3.696000in}}%
\pgfusepath{clip}%
\pgfsetbuttcap%
\pgfsetroundjoin%
\definecolor{currentfill}{rgb}{0.121569,0.466667,0.705882}%
\pgfsetfillcolor{currentfill}%
\pgfsetfillopacity{0.438851}%
\pgfsetlinewidth{1.003750pt}%
\definecolor{currentstroke}{rgb}{0.121569,0.466667,0.705882}%
\pgfsetstrokecolor{currentstroke}%
\pgfsetstrokeopacity{0.438851}%
\pgfsetdash{}{0pt}%
\pgfpathmoveto{\pgfqpoint{1.431942in}{3.049081in}}%
\pgfpathcurveto{\pgfqpoint{1.440178in}{3.049081in}}{\pgfqpoint{1.448078in}{3.052353in}}{\pgfqpoint{1.453902in}{3.058177in}}%
\pgfpathcurveto{\pgfqpoint{1.459726in}{3.064001in}}{\pgfqpoint{1.462998in}{3.071901in}}{\pgfqpoint{1.462998in}{3.080137in}}%
\pgfpathcurveto{\pgfqpoint{1.462998in}{3.088374in}}{\pgfqpoint{1.459726in}{3.096274in}}{\pgfqpoint{1.453902in}{3.102098in}}%
\pgfpathcurveto{\pgfqpoint{1.448078in}{3.107922in}}{\pgfqpoint{1.440178in}{3.111194in}}{\pgfqpoint{1.431942in}{3.111194in}}%
\pgfpathcurveto{\pgfqpoint{1.423706in}{3.111194in}}{\pgfqpoint{1.415806in}{3.107922in}}{\pgfqpoint{1.409982in}{3.102098in}}%
\pgfpathcurveto{\pgfqpoint{1.404158in}{3.096274in}}{\pgfqpoint{1.400885in}{3.088374in}}{\pgfqpoint{1.400885in}{3.080137in}}%
\pgfpathcurveto{\pgfqpoint{1.400885in}{3.071901in}}{\pgfqpoint{1.404158in}{3.064001in}}{\pgfqpoint{1.409982in}{3.058177in}}%
\pgfpathcurveto{\pgfqpoint{1.415806in}{3.052353in}}{\pgfqpoint{1.423706in}{3.049081in}}{\pgfqpoint{1.431942in}{3.049081in}}%
\pgfpathclose%
\pgfusepath{stroke,fill}%
\end{pgfscope}%
\begin{pgfscope}%
\pgfpathrectangle{\pgfqpoint{0.100000in}{0.212622in}}{\pgfqpoint{3.696000in}{3.696000in}}%
\pgfusepath{clip}%
\pgfsetbuttcap%
\pgfsetroundjoin%
\definecolor{currentfill}{rgb}{0.121569,0.466667,0.705882}%
\pgfsetfillcolor{currentfill}%
\pgfsetfillopacity{0.439667}%
\pgfsetlinewidth{1.003750pt}%
\definecolor{currentstroke}{rgb}{0.121569,0.466667,0.705882}%
\pgfsetstrokecolor{currentstroke}%
\pgfsetstrokeopacity{0.439667}%
\pgfsetdash{}{0pt}%
\pgfpathmoveto{\pgfqpoint{1.430897in}{3.046982in}}%
\pgfpathcurveto{\pgfqpoint{1.439133in}{3.046982in}}{\pgfqpoint{1.447033in}{3.050254in}}{\pgfqpoint{1.452857in}{3.056078in}}%
\pgfpathcurveto{\pgfqpoint{1.458681in}{3.061902in}}{\pgfqpoint{1.461953in}{3.069802in}}{\pgfqpoint{1.461953in}{3.078038in}}%
\pgfpathcurveto{\pgfqpoint{1.461953in}{3.086275in}}{\pgfqpoint{1.458681in}{3.094175in}}{\pgfqpoint{1.452857in}{3.099999in}}%
\pgfpathcurveto{\pgfqpoint{1.447033in}{3.105823in}}{\pgfqpoint{1.439133in}{3.109095in}}{\pgfqpoint{1.430897in}{3.109095in}}%
\pgfpathcurveto{\pgfqpoint{1.422661in}{3.109095in}}{\pgfqpoint{1.414760in}{3.105823in}}{\pgfqpoint{1.408937in}{3.099999in}}%
\pgfpathcurveto{\pgfqpoint{1.403113in}{3.094175in}}{\pgfqpoint{1.399840in}{3.086275in}}{\pgfqpoint{1.399840in}{3.078038in}}%
\pgfpathcurveto{\pgfqpoint{1.399840in}{3.069802in}}{\pgfqpoint{1.403113in}{3.061902in}}{\pgfqpoint{1.408937in}{3.056078in}}%
\pgfpathcurveto{\pgfqpoint{1.414760in}{3.050254in}}{\pgfqpoint{1.422661in}{3.046982in}}{\pgfqpoint{1.430897in}{3.046982in}}%
\pgfpathclose%
\pgfusepath{stroke,fill}%
\end{pgfscope}%
\begin{pgfscope}%
\pgfpathrectangle{\pgfqpoint{0.100000in}{0.212622in}}{\pgfqpoint{3.696000in}{3.696000in}}%
\pgfusepath{clip}%
\pgfsetbuttcap%
\pgfsetroundjoin%
\definecolor{currentfill}{rgb}{0.121569,0.466667,0.705882}%
\pgfsetfillcolor{currentfill}%
\pgfsetfillopacity{0.441194}%
\pgfsetlinewidth{1.003750pt}%
\definecolor{currentstroke}{rgb}{0.121569,0.466667,0.705882}%
\pgfsetstrokecolor{currentstroke}%
\pgfsetstrokeopacity{0.441194}%
\pgfsetdash{}{0pt}%
\pgfpathmoveto{\pgfqpoint{1.429258in}{3.043149in}}%
\pgfpathcurveto{\pgfqpoint{1.437494in}{3.043149in}}{\pgfqpoint{1.445394in}{3.046422in}}{\pgfqpoint{1.451218in}{3.052246in}}%
\pgfpathcurveto{\pgfqpoint{1.457042in}{3.058070in}}{\pgfqpoint{1.460315in}{3.065970in}}{\pgfqpoint{1.460315in}{3.074206in}}%
\pgfpathcurveto{\pgfqpoint{1.460315in}{3.082442in}}{\pgfqpoint{1.457042in}{3.090342in}}{\pgfqpoint{1.451218in}{3.096166in}}%
\pgfpathcurveto{\pgfqpoint{1.445394in}{3.101990in}}{\pgfqpoint{1.437494in}{3.105262in}}{\pgfqpoint{1.429258in}{3.105262in}}%
\pgfpathcurveto{\pgfqpoint{1.421022in}{3.105262in}}{\pgfqpoint{1.413122in}{3.101990in}}{\pgfqpoint{1.407298in}{3.096166in}}%
\pgfpathcurveto{\pgfqpoint{1.401474in}{3.090342in}}{\pgfqpoint{1.398202in}{3.082442in}}{\pgfqpoint{1.398202in}{3.074206in}}%
\pgfpathcurveto{\pgfqpoint{1.398202in}{3.065970in}}{\pgfqpoint{1.401474in}{3.058070in}}{\pgfqpoint{1.407298in}{3.052246in}}%
\pgfpathcurveto{\pgfqpoint{1.413122in}{3.046422in}}{\pgfqpoint{1.421022in}{3.043149in}}{\pgfqpoint{1.429258in}{3.043149in}}%
\pgfpathclose%
\pgfusepath{stroke,fill}%
\end{pgfscope}%
\begin{pgfscope}%
\pgfpathrectangle{\pgfqpoint{0.100000in}{0.212622in}}{\pgfqpoint{3.696000in}{3.696000in}}%
\pgfusepath{clip}%
\pgfsetbuttcap%
\pgfsetroundjoin%
\definecolor{currentfill}{rgb}{0.121569,0.466667,0.705882}%
\pgfsetfillcolor{currentfill}%
\pgfsetfillopacity{0.441456}%
\pgfsetlinewidth{1.003750pt}%
\definecolor{currentstroke}{rgb}{0.121569,0.466667,0.705882}%
\pgfsetstrokecolor{currentstroke}%
\pgfsetstrokeopacity{0.441456}%
\pgfsetdash{}{0pt}%
\pgfpathmoveto{\pgfqpoint{1.428908in}{3.042478in}}%
\pgfpathcurveto{\pgfqpoint{1.437144in}{3.042478in}}{\pgfqpoint{1.445044in}{3.045751in}}{\pgfqpoint{1.450868in}{3.051575in}}%
\pgfpathcurveto{\pgfqpoint{1.456692in}{3.057399in}}{\pgfqpoint{1.459964in}{3.065299in}}{\pgfqpoint{1.459964in}{3.073535in}}%
\pgfpathcurveto{\pgfqpoint{1.459964in}{3.081771in}}{\pgfqpoint{1.456692in}{3.089671in}}{\pgfqpoint{1.450868in}{3.095495in}}%
\pgfpathcurveto{\pgfqpoint{1.445044in}{3.101319in}}{\pgfqpoint{1.437144in}{3.104591in}}{\pgfqpoint{1.428908in}{3.104591in}}%
\pgfpathcurveto{\pgfqpoint{1.420672in}{3.104591in}}{\pgfqpoint{1.412772in}{3.101319in}}{\pgfqpoint{1.406948in}{3.095495in}}%
\pgfpathcurveto{\pgfqpoint{1.401124in}{3.089671in}}{\pgfqpoint{1.397851in}{3.081771in}}{\pgfqpoint{1.397851in}{3.073535in}}%
\pgfpathcurveto{\pgfqpoint{1.397851in}{3.065299in}}{\pgfqpoint{1.401124in}{3.057399in}}{\pgfqpoint{1.406948in}{3.051575in}}%
\pgfpathcurveto{\pgfqpoint{1.412772in}{3.045751in}}{\pgfqpoint{1.420672in}{3.042478in}}{\pgfqpoint{1.428908in}{3.042478in}}%
\pgfpathclose%
\pgfusepath{stroke,fill}%
\end{pgfscope}%
\begin{pgfscope}%
\pgfpathrectangle{\pgfqpoint{0.100000in}{0.212622in}}{\pgfqpoint{3.696000in}{3.696000in}}%
\pgfusepath{clip}%
\pgfsetbuttcap%
\pgfsetroundjoin%
\definecolor{currentfill}{rgb}{0.121569,0.466667,0.705882}%
\pgfsetfillcolor{currentfill}%
\pgfsetfillopacity{0.441943}%
\pgfsetlinewidth{1.003750pt}%
\definecolor{currentstroke}{rgb}{0.121569,0.466667,0.705882}%
\pgfsetstrokecolor{currentstroke}%
\pgfsetstrokeopacity{0.441943}%
\pgfsetdash{}{0pt}%
\pgfpathmoveto{\pgfqpoint{1.428374in}{3.041218in}}%
\pgfpathcurveto{\pgfqpoint{1.436610in}{3.041218in}}{\pgfqpoint{1.444510in}{3.044490in}}{\pgfqpoint{1.450334in}{3.050314in}}%
\pgfpathcurveto{\pgfqpoint{1.456158in}{3.056138in}}{\pgfqpoint{1.459430in}{3.064038in}}{\pgfqpoint{1.459430in}{3.072275in}}%
\pgfpathcurveto{\pgfqpoint{1.459430in}{3.080511in}}{\pgfqpoint{1.456158in}{3.088411in}}{\pgfqpoint{1.450334in}{3.094235in}}%
\pgfpathcurveto{\pgfqpoint{1.444510in}{3.100059in}}{\pgfqpoint{1.436610in}{3.103331in}}{\pgfqpoint{1.428374in}{3.103331in}}%
\pgfpathcurveto{\pgfqpoint{1.420138in}{3.103331in}}{\pgfqpoint{1.412238in}{3.100059in}}{\pgfqpoint{1.406414in}{3.094235in}}%
\pgfpathcurveto{\pgfqpoint{1.400590in}{3.088411in}}{\pgfqpoint{1.397317in}{3.080511in}}{\pgfqpoint{1.397317in}{3.072275in}}%
\pgfpathcurveto{\pgfqpoint{1.397317in}{3.064038in}}{\pgfqpoint{1.400590in}{3.056138in}}{\pgfqpoint{1.406414in}{3.050314in}}%
\pgfpathcurveto{\pgfqpoint{1.412238in}{3.044490in}}{\pgfqpoint{1.420138in}{3.041218in}}{\pgfqpoint{1.428374in}{3.041218in}}%
\pgfpathclose%
\pgfusepath{stroke,fill}%
\end{pgfscope}%
\begin{pgfscope}%
\pgfpathrectangle{\pgfqpoint{0.100000in}{0.212622in}}{\pgfqpoint{3.696000in}{3.696000in}}%
\pgfusepath{clip}%
\pgfsetbuttcap%
\pgfsetroundjoin%
\definecolor{currentfill}{rgb}{0.121569,0.466667,0.705882}%
\pgfsetfillcolor{currentfill}%
\pgfsetfillopacity{0.442040}%
\pgfsetlinewidth{1.003750pt}%
\definecolor{currentstroke}{rgb}{0.121569,0.466667,0.705882}%
\pgfsetstrokecolor{currentstroke}%
\pgfsetstrokeopacity{0.442040}%
\pgfsetdash{}{0pt}%
\pgfpathmoveto{\pgfqpoint{2.053557in}{2.850668in}}%
\pgfpathcurveto{\pgfqpoint{2.061793in}{2.850668in}}{\pgfqpoint{2.069693in}{2.853940in}}{\pgfqpoint{2.075517in}{2.859764in}}%
\pgfpathcurveto{\pgfqpoint{2.081341in}{2.865588in}}{\pgfqpoint{2.084613in}{2.873488in}}{\pgfqpoint{2.084613in}{2.881724in}}%
\pgfpathcurveto{\pgfqpoint{2.084613in}{2.889961in}}{\pgfqpoint{2.081341in}{2.897861in}}{\pgfqpoint{2.075517in}{2.903685in}}%
\pgfpathcurveto{\pgfqpoint{2.069693in}{2.909508in}}{\pgfqpoint{2.061793in}{2.912781in}}{\pgfqpoint{2.053557in}{2.912781in}}%
\pgfpathcurveto{\pgfqpoint{2.045320in}{2.912781in}}{\pgfqpoint{2.037420in}{2.909508in}}{\pgfqpoint{2.031596in}{2.903685in}}%
\pgfpathcurveto{\pgfqpoint{2.025772in}{2.897861in}}{\pgfqpoint{2.022500in}{2.889961in}}{\pgfqpoint{2.022500in}{2.881724in}}%
\pgfpathcurveto{\pgfqpoint{2.022500in}{2.873488in}}{\pgfqpoint{2.025772in}{2.865588in}}{\pgfqpoint{2.031596in}{2.859764in}}%
\pgfpathcurveto{\pgfqpoint{2.037420in}{2.853940in}}{\pgfqpoint{2.045320in}{2.850668in}}{\pgfqpoint{2.053557in}{2.850668in}}%
\pgfpathclose%
\pgfusepath{stroke,fill}%
\end{pgfscope}%
\begin{pgfscope}%
\pgfpathrectangle{\pgfqpoint{0.100000in}{0.212622in}}{\pgfqpoint{3.696000in}{3.696000in}}%
\pgfusepath{clip}%
\pgfsetbuttcap%
\pgfsetroundjoin%
\definecolor{currentfill}{rgb}{0.121569,0.466667,0.705882}%
\pgfsetfillcolor{currentfill}%
\pgfsetfillopacity{0.442814}%
\pgfsetlinewidth{1.003750pt}%
\definecolor{currentstroke}{rgb}{0.121569,0.466667,0.705882}%
\pgfsetstrokecolor{currentstroke}%
\pgfsetstrokeopacity{0.442814}%
\pgfsetdash{}{0pt}%
\pgfpathmoveto{\pgfqpoint{1.427226in}{3.039002in}}%
\pgfpathcurveto{\pgfqpoint{1.435462in}{3.039002in}}{\pgfqpoint{1.443362in}{3.042274in}}{\pgfqpoint{1.449186in}{3.048098in}}%
\pgfpathcurveto{\pgfqpoint{1.455010in}{3.053922in}}{\pgfqpoint{1.458282in}{3.061822in}}{\pgfqpoint{1.458282in}{3.070059in}}%
\pgfpathcurveto{\pgfqpoint{1.458282in}{3.078295in}}{\pgfqpoint{1.455010in}{3.086195in}}{\pgfqpoint{1.449186in}{3.092019in}}%
\pgfpathcurveto{\pgfqpoint{1.443362in}{3.097843in}}{\pgfqpoint{1.435462in}{3.101115in}}{\pgfqpoint{1.427226in}{3.101115in}}%
\pgfpathcurveto{\pgfqpoint{1.418989in}{3.101115in}}{\pgfqpoint{1.411089in}{3.097843in}}{\pgfqpoint{1.405265in}{3.092019in}}%
\pgfpathcurveto{\pgfqpoint{1.399441in}{3.086195in}}{\pgfqpoint{1.396169in}{3.078295in}}{\pgfqpoint{1.396169in}{3.070059in}}%
\pgfpathcurveto{\pgfqpoint{1.396169in}{3.061822in}}{\pgfqpoint{1.399441in}{3.053922in}}{\pgfqpoint{1.405265in}{3.048098in}}%
\pgfpathcurveto{\pgfqpoint{1.411089in}{3.042274in}}{\pgfqpoint{1.418989in}{3.039002in}}{\pgfqpoint{1.427226in}{3.039002in}}%
\pgfpathclose%
\pgfusepath{stroke,fill}%
\end{pgfscope}%
\begin{pgfscope}%
\pgfpathrectangle{\pgfqpoint{0.100000in}{0.212622in}}{\pgfqpoint{3.696000in}{3.696000in}}%
\pgfusepath{clip}%
\pgfsetbuttcap%
\pgfsetroundjoin%
\definecolor{currentfill}{rgb}{0.121569,0.466667,0.705882}%
\pgfsetfillcolor{currentfill}%
\pgfsetfillopacity{0.442956}%
\pgfsetlinewidth{1.003750pt}%
\definecolor{currentstroke}{rgb}{0.121569,0.466667,0.705882}%
\pgfsetstrokecolor{currentstroke}%
\pgfsetstrokeopacity{0.442956}%
\pgfsetdash{}{0pt}%
\pgfpathmoveto{\pgfqpoint{1.427072in}{3.038628in}}%
\pgfpathcurveto{\pgfqpoint{1.435308in}{3.038628in}}{\pgfqpoint{1.443208in}{3.041901in}}{\pgfqpoint{1.449032in}{3.047725in}}%
\pgfpathcurveto{\pgfqpoint{1.454856in}{3.053549in}}{\pgfqpoint{1.458128in}{3.061449in}}{\pgfqpoint{1.458128in}{3.069685in}}%
\pgfpathcurveto{\pgfqpoint{1.458128in}{3.077921in}}{\pgfqpoint{1.454856in}{3.085821in}}{\pgfqpoint{1.449032in}{3.091645in}}%
\pgfpathcurveto{\pgfqpoint{1.443208in}{3.097469in}}{\pgfqpoint{1.435308in}{3.100741in}}{\pgfqpoint{1.427072in}{3.100741in}}%
\pgfpathcurveto{\pgfqpoint{1.418836in}{3.100741in}}{\pgfqpoint{1.410936in}{3.097469in}}{\pgfqpoint{1.405112in}{3.091645in}}%
\pgfpathcurveto{\pgfqpoint{1.399288in}{3.085821in}}{\pgfqpoint{1.396015in}{3.077921in}}{\pgfqpoint{1.396015in}{3.069685in}}%
\pgfpathcurveto{\pgfqpoint{1.396015in}{3.061449in}}{\pgfqpoint{1.399288in}{3.053549in}}{\pgfqpoint{1.405112in}{3.047725in}}%
\pgfpathcurveto{\pgfqpoint{1.410936in}{3.041901in}}{\pgfqpoint{1.418836in}{3.038628in}}{\pgfqpoint{1.427072in}{3.038628in}}%
\pgfpathclose%
\pgfusepath{stroke,fill}%
\end{pgfscope}%
\begin{pgfscope}%
\pgfpathrectangle{\pgfqpoint{0.100000in}{0.212622in}}{\pgfqpoint{3.696000in}{3.696000in}}%
\pgfusepath{clip}%
\pgfsetbuttcap%
\pgfsetroundjoin%
\definecolor{currentfill}{rgb}{0.121569,0.466667,0.705882}%
\pgfsetfillcolor{currentfill}%
\pgfsetfillopacity{0.443213}%
\pgfsetlinewidth{1.003750pt}%
\definecolor{currentstroke}{rgb}{0.121569,0.466667,0.705882}%
\pgfsetstrokecolor{currentstroke}%
\pgfsetstrokeopacity{0.443213}%
\pgfsetdash{}{0pt}%
\pgfpathmoveto{\pgfqpoint{1.426731in}{3.037984in}}%
\pgfpathcurveto{\pgfqpoint{1.434968in}{3.037984in}}{\pgfqpoint{1.442868in}{3.041256in}}{\pgfqpoint{1.448692in}{3.047080in}}%
\pgfpathcurveto{\pgfqpoint{1.454515in}{3.052904in}}{\pgfqpoint{1.457788in}{3.060804in}}{\pgfqpoint{1.457788in}{3.069041in}}%
\pgfpathcurveto{\pgfqpoint{1.457788in}{3.077277in}}{\pgfqpoint{1.454515in}{3.085177in}}{\pgfqpoint{1.448692in}{3.091001in}}%
\pgfpathcurveto{\pgfqpoint{1.442868in}{3.096825in}}{\pgfqpoint{1.434968in}{3.100097in}}{\pgfqpoint{1.426731in}{3.100097in}}%
\pgfpathcurveto{\pgfqpoint{1.418495in}{3.100097in}}{\pgfqpoint{1.410595in}{3.096825in}}{\pgfqpoint{1.404771in}{3.091001in}}%
\pgfpathcurveto{\pgfqpoint{1.398947in}{3.085177in}}{\pgfqpoint{1.395675in}{3.077277in}}{\pgfqpoint{1.395675in}{3.069041in}}%
\pgfpathcurveto{\pgfqpoint{1.395675in}{3.060804in}}{\pgfqpoint{1.398947in}{3.052904in}}{\pgfqpoint{1.404771in}{3.047080in}}%
\pgfpathcurveto{\pgfqpoint{1.410595in}{3.041256in}}{\pgfqpoint{1.418495in}{3.037984in}}{\pgfqpoint{1.426731in}{3.037984in}}%
\pgfpathclose%
\pgfusepath{stroke,fill}%
\end{pgfscope}%
\begin{pgfscope}%
\pgfpathrectangle{\pgfqpoint{0.100000in}{0.212622in}}{\pgfqpoint{3.696000in}{3.696000in}}%
\pgfusepath{clip}%
\pgfsetbuttcap%
\pgfsetroundjoin%
\definecolor{currentfill}{rgb}{0.121569,0.466667,0.705882}%
\pgfsetfillcolor{currentfill}%
\pgfsetfillopacity{0.443693}%
\pgfsetlinewidth{1.003750pt}%
\definecolor{currentstroke}{rgb}{0.121569,0.466667,0.705882}%
\pgfsetstrokecolor{currentstroke}%
\pgfsetstrokeopacity{0.443693}%
\pgfsetdash{}{0pt}%
\pgfpathmoveto{\pgfqpoint{1.426218in}{3.036788in}}%
\pgfpathcurveto{\pgfqpoint{1.434455in}{3.036788in}}{\pgfqpoint{1.442355in}{3.040060in}}{\pgfqpoint{1.448179in}{3.045884in}}%
\pgfpathcurveto{\pgfqpoint{1.454003in}{3.051708in}}{\pgfqpoint{1.457275in}{3.059608in}}{\pgfqpoint{1.457275in}{3.067844in}}%
\pgfpathcurveto{\pgfqpoint{1.457275in}{3.076080in}}{\pgfqpoint{1.454003in}{3.083980in}}{\pgfqpoint{1.448179in}{3.089804in}}%
\pgfpathcurveto{\pgfqpoint{1.442355in}{3.095628in}}{\pgfqpoint{1.434455in}{3.098901in}}{\pgfqpoint{1.426218in}{3.098901in}}%
\pgfpathcurveto{\pgfqpoint{1.417982in}{3.098901in}}{\pgfqpoint{1.410082in}{3.095628in}}{\pgfqpoint{1.404258in}{3.089804in}}%
\pgfpathcurveto{\pgfqpoint{1.398434in}{3.083980in}}{\pgfqpoint{1.395162in}{3.076080in}}{\pgfqpoint{1.395162in}{3.067844in}}%
\pgfpathcurveto{\pgfqpoint{1.395162in}{3.059608in}}{\pgfqpoint{1.398434in}{3.051708in}}{\pgfqpoint{1.404258in}{3.045884in}}%
\pgfpathcurveto{\pgfqpoint{1.410082in}{3.040060in}}{\pgfqpoint{1.417982in}{3.036788in}}{\pgfqpoint{1.426218in}{3.036788in}}%
\pgfpathclose%
\pgfusepath{stroke,fill}%
\end{pgfscope}%
\begin{pgfscope}%
\pgfpathrectangle{\pgfqpoint{0.100000in}{0.212622in}}{\pgfqpoint{3.696000in}{3.696000in}}%
\pgfusepath{clip}%
\pgfsetbuttcap%
\pgfsetroundjoin%
\definecolor{currentfill}{rgb}{0.121569,0.466667,0.705882}%
\pgfsetfillcolor{currentfill}%
\pgfsetfillopacity{0.444562}%
\pgfsetlinewidth{1.003750pt}%
\definecolor{currentstroke}{rgb}{0.121569,0.466667,0.705882}%
\pgfsetstrokecolor{currentstroke}%
\pgfsetstrokeopacity{0.444562}%
\pgfsetdash{}{0pt}%
\pgfpathmoveto{\pgfqpoint{1.425151in}{3.034690in}}%
\pgfpathcurveto{\pgfqpoint{1.433387in}{3.034690in}}{\pgfqpoint{1.441287in}{3.037962in}}{\pgfqpoint{1.447111in}{3.043786in}}%
\pgfpathcurveto{\pgfqpoint{1.452935in}{3.049610in}}{\pgfqpoint{1.456208in}{3.057510in}}{\pgfqpoint{1.456208in}{3.065746in}}%
\pgfpathcurveto{\pgfqpoint{1.456208in}{3.073983in}}{\pgfqpoint{1.452935in}{3.081883in}}{\pgfqpoint{1.447111in}{3.087707in}}%
\pgfpathcurveto{\pgfqpoint{1.441287in}{3.093530in}}{\pgfqpoint{1.433387in}{3.096803in}}{\pgfqpoint{1.425151in}{3.096803in}}%
\pgfpathcurveto{\pgfqpoint{1.416915in}{3.096803in}}{\pgfqpoint{1.409015in}{3.093530in}}{\pgfqpoint{1.403191in}{3.087707in}}%
\pgfpathcurveto{\pgfqpoint{1.397367in}{3.081883in}}{\pgfqpoint{1.394095in}{3.073983in}}{\pgfqpoint{1.394095in}{3.065746in}}%
\pgfpathcurveto{\pgfqpoint{1.394095in}{3.057510in}}{\pgfqpoint{1.397367in}{3.049610in}}{\pgfqpoint{1.403191in}{3.043786in}}%
\pgfpathcurveto{\pgfqpoint{1.409015in}{3.037962in}}{\pgfqpoint{1.416915in}{3.034690in}}{\pgfqpoint{1.425151in}{3.034690in}}%
\pgfpathclose%
\pgfusepath{stroke,fill}%
\end{pgfscope}%
\begin{pgfscope}%
\pgfpathrectangle{\pgfqpoint{0.100000in}{0.212622in}}{\pgfqpoint{3.696000in}{3.696000in}}%
\pgfusepath{clip}%
\pgfsetbuttcap%
\pgfsetroundjoin%
\definecolor{currentfill}{rgb}{0.121569,0.466667,0.705882}%
\pgfsetfillcolor{currentfill}%
\pgfsetfillopacity{0.446175}%
\pgfsetlinewidth{1.003750pt}%
\definecolor{currentstroke}{rgb}{0.121569,0.466667,0.705882}%
\pgfsetstrokecolor{currentstroke}%
\pgfsetstrokeopacity{0.446175}%
\pgfsetdash{}{0pt}%
\pgfpathmoveto{\pgfqpoint{1.423470in}{3.030826in}}%
\pgfpathcurveto{\pgfqpoint{1.431706in}{3.030826in}}{\pgfqpoint{1.439606in}{3.034098in}}{\pgfqpoint{1.445430in}{3.039922in}}%
\pgfpathcurveto{\pgfqpoint{1.451254in}{3.045746in}}{\pgfqpoint{1.454526in}{3.053646in}}{\pgfqpoint{1.454526in}{3.061882in}}%
\pgfpathcurveto{\pgfqpoint{1.454526in}{3.070119in}}{\pgfqpoint{1.451254in}{3.078019in}}{\pgfqpoint{1.445430in}{3.083843in}}%
\pgfpathcurveto{\pgfqpoint{1.439606in}{3.089666in}}{\pgfqpoint{1.431706in}{3.092939in}}{\pgfqpoint{1.423470in}{3.092939in}}%
\pgfpathcurveto{\pgfqpoint{1.415233in}{3.092939in}}{\pgfqpoint{1.407333in}{3.089666in}}{\pgfqpoint{1.401509in}{3.083843in}}%
\pgfpathcurveto{\pgfqpoint{1.395686in}{3.078019in}}{\pgfqpoint{1.392413in}{3.070119in}}{\pgfqpoint{1.392413in}{3.061882in}}%
\pgfpathcurveto{\pgfqpoint{1.392413in}{3.053646in}}{\pgfqpoint{1.395686in}{3.045746in}}{\pgfqpoint{1.401509in}{3.039922in}}%
\pgfpathcurveto{\pgfqpoint{1.407333in}{3.034098in}}{\pgfqpoint{1.415233in}{3.030826in}}{\pgfqpoint{1.423470in}{3.030826in}}%
\pgfpathclose%
\pgfusepath{stroke,fill}%
\end{pgfscope}%
\begin{pgfscope}%
\pgfpathrectangle{\pgfqpoint{0.100000in}{0.212622in}}{\pgfqpoint{3.696000in}{3.696000in}}%
\pgfusepath{clip}%
\pgfsetbuttcap%
\pgfsetroundjoin%
\definecolor{currentfill}{rgb}{0.121569,0.466667,0.705882}%
\pgfsetfillcolor{currentfill}%
\pgfsetfillopacity{0.449112}%
\pgfsetlinewidth{1.003750pt}%
\definecolor{currentstroke}{rgb}{0.121569,0.466667,0.705882}%
\pgfsetstrokecolor{currentstroke}%
\pgfsetstrokeopacity{0.449112}%
\pgfsetdash{}{0pt}%
\pgfpathmoveto{\pgfqpoint{1.420099in}{3.024021in}}%
\pgfpathcurveto{\pgfqpoint{1.428335in}{3.024021in}}{\pgfqpoint{1.436235in}{3.027293in}}{\pgfqpoint{1.442059in}{3.033117in}}%
\pgfpathcurveto{\pgfqpoint{1.447883in}{3.038941in}}{\pgfqpoint{1.451155in}{3.046841in}}{\pgfqpoint{1.451155in}{3.055077in}}%
\pgfpathcurveto{\pgfqpoint{1.451155in}{3.063313in}}{\pgfqpoint{1.447883in}{3.071213in}}{\pgfqpoint{1.442059in}{3.077037in}}%
\pgfpathcurveto{\pgfqpoint{1.436235in}{3.082861in}}{\pgfqpoint{1.428335in}{3.086134in}}{\pgfqpoint{1.420099in}{3.086134in}}%
\pgfpathcurveto{\pgfqpoint{1.411862in}{3.086134in}}{\pgfqpoint{1.403962in}{3.082861in}}{\pgfqpoint{1.398138in}{3.077037in}}%
\pgfpathcurveto{\pgfqpoint{1.392314in}{3.071213in}}{\pgfqpoint{1.389042in}{3.063313in}}{\pgfqpoint{1.389042in}{3.055077in}}%
\pgfpathcurveto{\pgfqpoint{1.389042in}{3.046841in}}{\pgfqpoint{1.392314in}{3.038941in}}{\pgfqpoint{1.398138in}{3.033117in}}%
\pgfpathcurveto{\pgfqpoint{1.403962in}{3.027293in}}{\pgfqpoint{1.411862in}{3.024021in}}{\pgfqpoint{1.420099in}{3.024021in}}%
\pgfpathclose%
\pgfusepath{stroke,fill}%
\end{pgfscope}%
\begin{pgfscope}%
\pgfpathrectangle{\pgfqpoint{0.100000in}{0.212622in}}{\pgfqpoint{3.696000in}{3.696000in}}%
\pgfusepath{clip}%
\pgfsetbuttcap%
\pgfsetroundjoin%
\definecolor{currentfill}{rgb}{0.121569,0.466667,0.705882}%
\pgfsetfillcolor{currentfill}%
\pgfsetfillopacity{0.451447}%
\pgfsetlinewidth{1.003750pt}%
\definecolor{currentstroke}{rgb}{0.121569,0.466667,0.705882}%
\pgfsetstrokecolor{currentstroke}%
\pgfsetstrokeopacity{0.451447}%
\pgfsetdash{}{0pt}%
\pgfpathmoveto{\pgfqpoint{1.417536in}{3.018897in}}%
\pgfpathcurveto{\pgfqpoint{1.425772in}{3.018897in}}{\pgfqpoint{1.433673in}{3.022169in}}{\pgfqpoint{1.439496in}{3.027993in}}%
\pgfpathcurveto{\pgfqpoint{1.445320in}{3.033817in}}{\pgfqpoint{1.448593in}{3.041717in}}{\pgfqpoint{1.448593in}{3.049953in}}%
\pgfpathcurveto{\pgfqpoint{1.448593in}{3.058190in}}{\pgfqpoint{1.445320in}{3.066090in}}{\pgfqpoint{1.439496in}{3.071914in}}%
\pgfpathcurveto{\pgfqpoint{1.433673in}{3.077737in}}{\pgfqpoint{1.425772in}{3.081010in}}{\pgfqpoint{1.417536in}{3.081010in}}%
\pgfpathcurveto{\pgfqpoint{1.409300in}{3.081010in}}{\pgfqpoint{1.401400in}{3.077737in}}{\pgfqpoint{1.395576in}{3.071914in}}%
\pgfpathcurveto{\pgfqpoint{1.389752in}{3.066090in}}{\pgfqpoint{1.386480in}{3.058190in}}{\pgfqpoint{1.386480in}{3.049953in}}%
\pgfpathcurveto{\pgfqpoint{1.386480in}{3.041717in}}{\pgfqpoint{1.389752in}{3.033817in}}{\pgfqpoint{1.395576in}{3.027993in}}%
\pgfpathcurveto{\pgfqpoint{1.401400in}{3.022169in}}{\pgfqpoint{1.409300in}{3.018897in}}{\pgfqpoint{1.417536in}{3.018897in}}%
\pgfpathclose%
\pgfusepath{stroke,fill}%
\end{pgfscope}%
\begin{pgfscope}%
\pgfpathrectangle{\pgfqpoint{0.100000in}{0.212622in}}{\pgfqpoint{3.696000in}{3.696000in}}%
\pgfusepath{clip}%
\pgfsetbuttcap%
\pgfsetroundjoin%
\definecolor{currentfill}{rgb}{0.121569,0.466667,0.705882}%
\pgfsetfillcolor{currentfill}%
\pgfsetfillopacity{0.452991}%
\pgfsetlinewidth{1.003750pt}%
\definecolor{currentstroke}{rgb}{0.121569,0.466667,0.705882}%
\pgfsetstrokecolor{currentstroke}%
\pgfsetstrokeopacity{0.452991}%
\pgfsetdash{}{0pt}%
\pgfpathmoveto{\pgfqpoint{1.415743in}{3.015676in}}%
\pgfpathcurveto{\pgfqpoint{1.423979in}{3.015676in}}{\pgfqpoint{1.431879in}{3.018949in}}{\pgfqpoint{1.437703in}{3.024773in}}%
\pgfpathcurveto{\pgfqpoint{1.443527in}{3.030596in}}{\pgfqpoint{1.446799in}{3.038497in}}{\pgfqpoint{1.446799in}{3.046733in}}%
\pgfpathcurveto{\pgfqpoint{1.446799in}{3.054969in}}{\pgfqpoint{1.443527in}{3.062869in}}{\pgfqpoint{1.437703in}{3.068693in}}%
\pgfpathcurveto{\pgfqpoint{1.431879in}{3.074517in}}{\pgfqpoint{1.423979in}{3.077789in}}{\pgfqpoint{1.415743in}{3.077789in}}%
\pgfpathcurveto{\pgfqpoint{1.407507in}{3.077789in}}{\pgfqpoint{1.399606in}{3.074517in}}{\pgfqpoint{1.393783in}{3.068693in}}%
\pgfpathcurveto{\pgfqpoint{1.387959in}{3.062869in}}{\pgfqpoint{1.384686in}{3.054969in}}{\pgfqpoint{1.384686in}{3.046733in}}%
\pgfpathcurveto{\pgfqpoint{1.384686in}{3.038497in}}{\pgfqpoint{1.387959in}{3.030596in}}{\pgfqpoint{1.393783in}{3.024773in}}%
\pgfpathcurveto{\pgfqpoint{1.399606in}{3.018949in}}{\pgfqpoint{1.407507in}{3.015676in}}{\pgfqpoint{1.415743in}{3.015676in}}%
\pgfpathclose%
\pgfusepath{stroke,fill}%
\end{pgfscope}%
\begin{pgfscope}%
\pgfpathrectangle{\pgfqpoint{0.100000in}{0.212622in}}{\pgfqpoint{3.696000in}{3.696000in}}%
\pgfusepath{clip}%
\pgfsetbuttcap%
\pgfsetroundjoin%
\definecolor{currentfill}{rgb}{0.121569,0.466667,0.705882}%
\pgfsetfillcolor{currentfill}%
\pgfsetfillopacity{0.455424}%
\pgfsetlinewidth{1.003750pt}%
\definecolor{currentstroke}{rgb}{0.121569,0.466667,0.705882}%
\pgfsetstrokecolor{currentstroke}%
\pgfsetstrokeopacity{0.455424}%
\pgfsetdash{}{0pt}%
\pgfpathmoveto{\pgfqpoint{2.079647in}{2.805954in}}%
\pgfpathcurveto{\pgfqpoint{2.087884in}{2.805954in}}{\pgfqpoint{2.095784in}{2.809226in}}{\pgfqpoint{2.101608in}{2.815050in}}%
\pgfpathcurveto{\pgfqpoint{2.107432in}{2.820874in}}{\pgfqpoint{2.110704in}{2.828774in}}{\pgfqpoint{2.110704in}{2.837010in}}%
\pgfpathcurveto{\pgfqpoint{2.110704in}{2.845246in}}{\pgfqpoint{2.107432in}{2.853146in}}{\pgfqpoint{2.101608in}{2.858970in}}%
\pgfpathcurveto{\pgfqpoint{2.095784in}{2.864794in}}{\pgfqpoint{2.087884in}{2.868067in}}{\pgfqpoint{2.079647in}{2.868067in}}%
\pgfpathcurveto{\pgfqpoint{2.071411in}{2.868067in}}{\pgfqpoint{2.063511in}{2.864794in}}{\pgfqpoint{2.057687in}{2.858970in}}%
\pgfpathcurveto{\pgfqpoint{2.051863in}{2.853146in}}{\pgfqpoint{2.048591in}{2.845246in}}{\pgfqpoint{2.048591in}{2.837010in}}%
\pgfpathcurveto{\pgfqpoint{2.048591in}{2.828774in}}{\pgfqpoint{2.051863in}{2.820874in}}{\pgfqpoint{2.057687in}{2.815050in}}%
\pgfpathcurveto{\pgfqpoint{2.063511in}{2.809226in}}{\pgfqpoint{2.071411in}{2.805954in}}{\pgfqpoint{2.079647in}{2.805954in}}%
\pgfpathclose%
\pgfusepath{stroke,fill}%
\end{pgfscope}%
\begin{pgfscope}%
\pgfpathrectangle{\pgfqpoint{0.100000in}{0.212622in}}{\pgfqpoint{3.696000in}{3.696000in}}%
\pgfusepath{clip}%
\pgfsetbuttcap%
\pgfsetroundjoin%
\definecolor{currentfill}{rgb}{0.121569,0.466667,0.705882}%
\pgfsetfillcolor{currentfill}%
\pgfsetfillopacity{0.455541}%
\pgfsetlinewidth{1.003750pt}%
\definecolor{currentstroke}{rgb}{0.121569,0.466667,0.705882}%
\pgfsetstrokecolor{currentstroke}%
\pgfsetstrokeopacity{0.455541}%
\pgfsetdash{}{0pt}%
\pgfpathmoveto{\pgfqpoint{1.412144in}{3.009043in}}%
\pgfpathcurveto{\pgfqpoint{1.420381in}{3.009043in}}{\pgfqpoint{1.428281in}{3.012315in}}{\pgfqpoint{1.434105in}{3.018139in}}%
\pgfpathcurveto{\pgfqpoint{1.439928in}{3.023963in}}{\pgfqpoint{1.443201in}{3.031863in}}{\pgfqpoint{1.443201in}{3.040100in}}%
\pgfpathcurveto{\pgfqpoint{1.443201in}{3.048336in}}{\pgfqpoint{1.439928in}{3.056236in}}{\pgfqpoint{1.434105in}{3.062060in}}%
\pgfpathcurveto{\pgfqpoint{1.428281in}{3.067884in}}{\pgfqpoint{1.420381in}{3.071156in}}{\pgfqpoint{1.412144in}{3.071156in}}%
\pgfpathcurveto{\pgfqpoint{1.403908in}{3.071156in}}{\pgfqpoint{1.396008in}{3.067884in}}{\pgfqpoint{1.390184in}{3.062060in}}%
\pgfpathcurveto{\pgfqpoint{1.384360in}{3.056236in}}{\pgfqpoint{1.381088in}{3.048336in}}{\pgfqpoint{1.381088in}{3.040100in}}%
\pgfpathcurveto{\pgfqpoint{1.381088in}{3.031863in}}{\pgfqpoint{1.384360in}{3.023963in}}{\pgfqpoint{1.390184in}{3.018139in}}%
\pgfpathcurveto{\pgfqpoint{1.396008in}{3.012315in}}{\pgfqpoint{1.403908in}{3.009043in}}{\pgfqpoint{1.412144in}{3.009043in}}%
\pgfpathclose%
\pgfusepath{stroke,fill}%
\end{pgfscope}%
\begin{pgfscope}%
\pgfpathrectangle{\pgfqpoint{0.100000in}{0.212622in}}{\pgfqpoint{3.696000in}{3.696000in}}%
\pgfusepath{clip}%
\pgfsetbuttcap%
\pgfsetroundjoin%
\definecolor{currentfill}{rgb}{0.121569,0.466667,0.705882}%
\pgfsetfillcolor{currentfill}%
\pgfsetfillopacity{0.457364}%
\pgfsetlinewidth{1.003750pt}%
\definecolor{currentstroke}{rgb}{0.121569,0.466667,0.705882}%
\pgfsetstrokecolor{currentstroke}%
\pgfsetstrokeopacity{0.457364}%
\pgfsetdash{}{0pt}%
\pgfpathmoveto{\pgfqpoint{1.410743in}{3.004112in}}%
\pgfpathcurveto{\pgfqpoint{1.418979in}{3.004112in}}{\pgfqpoint{1.426880in}{3.007384in}}{\pgfqpoint{1.432703in}{3.013208in}}%
\pgfpathcurveto{\pgfqpoint{1.438527in}{3.019032in}}{\pgfqpoint{1.441800in}{3.026932in}}{\pgfqpoint{1.441800in}{3.035169in}}%
\pgfpathcurveto{\pgfqpoint{1.441800in}{3.043405in}}{\pgfqpoint{1.438527in}{3.051305in}}{\pgfqpoint{1.432703in}{3.057129in}}%
\pgfpathcurveto{\pgfqpoint{1.426880in}{3.062953in}}{\pgfqpoint{1.418979in}{3.066225in}}{\pgfqpoint{1.410743in}{3.066225in}}%
\pgfpathcurveto{\pgfqpoint{1.402507in}{3.066225in}}{\pgfqpoint{1.394607in}{3.062953in}}{\pgfqpoint{1.388783in}{3.057129in}}%
\pgfpathcurveto{\pgfqpoint{1.382959in}{3.051305in}}{\pgfqpoint{1.379687in}{3.043405in}}{\pgfqpoint{1.379687in}{3.035169in}}%
\pgfpathcurveto{\pgfqpoint{1.379687in}{3.026932in}}{\pgfqpoint{1.382959in}{3.019032in}}{\pgfqpoint{1.388783in}{3.013208in}}%
\pgfpathcurveto{\pgfqpoint{1.394607in}{3.007384in}}{\pgfqpoint{1.402507in}{3.004112in}}{\pgfqpoint{1.410743in}{3.004112in}}%
\pgfpathclose%
\pgfusepath{stroke,fill}%
\end{pgfscope}%
\begin{pgfscope}%
\pgfpathrectangle{\pgfqpoint{0.100000in}{0.212622in}}{\pgfqpoint{3.696000in}{3.696000in}}%
\pgfusepath{clip}%
\pgfsetbuttcap%
\pgfsetroundjoin%
\definecolor{currentfill}{rgb}{0.121569,0.466667,0.705882}%
\pgfsetfillcolor{currentfill}%
\pgfsetfillopacity{0.458635}%
\pgfsetlinewidth{1.003750pt}%
\definecolor{currentstroke}{rgb}{0.121569,0.466667,0.705882}%
\pgfsetstrokecolor{currentstroke}%
\pgfsetstrokeopacity{0.458635}%
\pgfsetdash{}{0pt}%
\pgfpathmoveto{\pgfqpoint{1.409011in}{3.000818in}}%
\pgfpathcurveto{\pgfqpoint{1.417247in}{3.000818in}}{\pgfqpoint{1.425147in}{3.004090in}}{\pgfqpoint{1.430971in}{3.009914in}}%
\pgfpathcurveto{\pgfqpoint{1.436795in}{3.015738in}}{\pgfqpoint{1.440067in}{3.023638in}}{\pgfqpoint{1.440067in}{3.031874in}}%
\pgfpathcurveto{\pgfqpoint{1.440067in}{3.040110in}}{\pgfqpoint{1.436795in}{3.048011in}}{\pgfqpoint{1.430971in}{3.053834in}}%
\pgfpathcurveto{\pgfqpoint{1.425147in}{3.059658in}}{\pgfqpoint{1.417247in}{3.062931in}}{\pgfqpoint{1.409011in}{3.062931in}}%
\pgfpathcurveto{\pgfqpoint{1.400774in}{3.062931in}}{\pgfqpoint{1.392874in}{3.059658in}}{\pgfqpoint{1.387050in}{3.053834in}}%
\pgfpathcurveto{\pgfqpoint{1.381226in}{3.048011in}}{\pgfqpoint{1.377954in}{3.040110in}}{\pgfqpoint{1.377954in}{3.031874in}}%
\pgfpathcurveto{\pgfqpoint{1.377954in}{3.023638in}}{\pgfqpoint{1.381226in}{3.015738in}}{\pgfqpoint{1.387050in}{3.009914in}}%
\pgfpathcurveto{\pgfqpoint{1.392874in}{3.004090in}}{\pgfqpoint{1.400774in}{3.000818in}}{\pgfqpoint{1.409011in}{3.000818in}}%
\pgfpathclose%
\pgfusepath{stroke,fill}%
\end{pgfscope}%
\begin{pgfscope}%
\pgfpathrectangle{\pgfqpoint{0.100000in}{0.212622in}}{\pgfqpoint{3.696000in}{3.696000in}}%
\pgfusepath{clip}%
\pgfsetbuttcap%
\pgfsetroundjoin%
\definecolor{currentfill}{rgb}{0.121569,0.466667,0.705882}%
\pgfsetfillcolor{currentfill}%
\pgfsetfillopacity{0.459424}%
\pgfsetlinewidth{1.003750pt}%
\definecolor{currentstroke}{rgb}{0.121569,0.466667,0.705882}%
\pgfsetstrokecolor{currentstroke}%
\pgfsetstrokeopacity{0.459424}%
\pgfsetdash{}{0pt}%
\pgfpathmoveto{\pgfqpoint{1.408194in}{2.998774in}}%
\pgfpathcurveto{\pgfqpoint{1.416431in}{2.998774in}}{\pgfqpoint{1.424331in}{3.002046in}}{\pgfqpoint{1.430154in}{3.007870in}}%
\pgfpathcurveto{\pgfqpoint{1.435978in}{3.013694in}}{\pgfqpoint{1.439251in}{3.021594in}}{\pgfqpoint{1.439251in}{3.029831in}}%
\pgfpathcurveto{\pgfqpoint{1.439251in}{3.038067in}}{\pgfqpoint{1.435978in}{3.045967in}}{\pgfqpoint{1.430154in}{3.051791in}}%
\pgfpathcurveto{\pgfqpoint{1.424331in}{3.057615in}}{\pgfqpoint{1.416431in}{3.060887in}}{\pgfqpoint{1.408194in}{3.060887in}}%
\pgfpathcurveto{\pgfqpoint{1.399958in}{3.060887in}}{\pgfqpoint{1.392058in}{3.057615in}}{\pgfqpoint{1.386234in}{3.051791in}}%
\pgfpathcurveto{\pgfqpoint{1.380410in}{3.045967in}}{\pgfqpoint{1.377138in}{3.038067in}}{\pgfqpoint{1.377138in}{3.029831in}}%
\pgfpathcurveto{\pgfqpoint{1.377138in}{3.021594in}}{\pgfqpoint{1.380410in}{3.013694in}}{\pgfqpoint{1.386234in}{3.007870in}}%
\pgfpathcurveto{\pgfqpoint{1.392058in}{3.002046in}}{\pgfqpoint{1.399958in}{2.998774in}}{\pgfqpoint{1.408194in}{2.998774in}}%
\pgfpathclose%
\pgfusepath{stroke,fill}%
\end{pgfscope}%
\begin{pgfscope}%
\pgfpathrectangle{\pgfqpoint{0.100000in}{0.212622in}}{\pgfqpoint{3.696000in}{3.696000in}}%
\pgfusepath{clip}%
\pgfsetbuttcap%
\pgfsetroundjoin%
\definecolor{currentfill}{rgb}{0.121569,0.466667,0.705882}%
\pgfsetfillcolor{currentfill}%
\pgfsetfillopacity{0.460830}%
\pgfsetlinewidth{1.003750pt}%
\definecolor{currentstroke}{rgb}{0.121569,0.466667,0.705882}%
\pgfsetstrokecolor{currentstroke}%
\pgfsetstrokeopacity{0.460830}%
\pgfsetdash{}{0pt}%
\pgfpathmoveto{\pgfqpoint{1.406360in}{2.995195in}}%
\pgfpathcurveto{\pgfqpoint{1.414597in}{2.995195in}}{\pgfqpoint{1.422497in}{2.998468in}}{\pgfqpoint{1.428321in}{3.004292in}}%
\pgfpathcurveto{\pgfqpoint{1.434144in}{3.010116in}}{\pgfqpoint{1.437417in}{3.018016in}}{\pgfqpoint{1.437417in}{3.026252in}}%
\pgfpathcurveto{\pgfqpoint{1.437417in}{3.034488in}}{\pgfqpoint{1.434144in}{3.042388in}}{\pgfqpoint{1.428321in}{3.048212in}}%
\pgfpathcurveto{\pgfqpoint{1.422497in}{3.054036in}}{\pgfqpoint{1.414597in}{3.057308in}}{\pgfqpoint{1.406360in}{3.057308in}}%
\pgfpathcurveto{\pgfqpoint{1.398124in}{3.057308in}}{\pgfqpoint{1.390224in}{3.054036in}}{\pgfqpoint{1.384400in}{3.048212in}}%
\pgfpathcurveto{\pgfqpoint{1.378576in}{3.042388in}}{\pgfqpoint{1.375304in}{3.034488in}}{\pgfqpoint{1.375304in}{3.026252in}}%
\pgfpathcurveto{\pgfqpoint{1.375304in}{3.018016in}}{\pgfqpoint{1.378576in}{3.010116in}}{\pgfqpoint{1.384400in}{3.004292in}}%
\pgfpathcurveto{\pgfqpoint{1.390224in}{2.998468in}}{\pgfqpoint{1.398124in}{2.995195in}}{\pgfqpoint{1.406360in}{2.995195in}}%
\pgfpathclose%
\pgfusepath{stroke,fill}%
\end{pgfscope}%
\begin{pgfscope}%
\pgfpathrectangle{\pgfqpoint{0.100000in}{0.212622in}}{\pgfqpoint{3.696000in}{3.696000in}}%
\pgfusepath{clip}%
\pgfsetbuttcap%
\pgfsetroundjoin%
\definecolor{currentfill}{rgb}{0.121569,0.466667,0.705882}%
\pgfsetfillcolor{currentfill}%
\pgfsetfillopacity{0.461590}%
\pgfsetlinewidth{1.003750pt}%
\definecolor{currentstroke}{rgb}{0.121569,0.466667,0.705882}%
\pgfsetstrokecolor{currentstroke}%
\pgfsetstrokeopacity{0.461590}%
\pgfsetdash{}{0pt}%
\pgfpathmoveto{\pgfqpoint{1.405572in}{2.993258in}}%
\pgfpathcurveto{\pgfqpoint{1.413808in}{2.993258in}}{\pgfqpoint{1.421708in}{2.996531in}}{\pgfqpoint{1.427532in}{3.002354in}}%
\pgfpathcurveto{\pgfqpoint{1.433356in}{3.008178in}}{\pgfqpoint{1.436628in}{3.016078in}}{\pgfqpoint{1.436628in}{3.024315in}}%
\pgfpathcurveto{\pgfqpoint{1.436628in}{3.032551in}}{\pgfqpoint{1.433356in}{3.040451in}}{\pgfqpoint{1.427532in}{3.046275in}}%
\pgfpathcurveto{\pgfqpoint{1.421708in}{3.052099in}}{\pgfqpoint{1.413808in}{3.055371in}}{\pgfqpoint{1.405572in}{3.055371in}}%
\pgfpathcurveto{\pgfqpoint{1.397336in}{3.055371in}}{\pgfqpoint{1.389436in}{3.052099in}}{\pgfqpoint{1.383612in}{3.046275in}}%
\pgfpathcurveto{\pgfqpoint{1.377788in}{3.040451in}}{\pgfqpoint{1.374515in}{3.032551in}}{\pgfqpoint{1.374515in}{3.024315in}}%
\pgfpathcurveto{\pgfqpoint{1.374515in}{3.016078in}}{\pgfqpoint{1.377788in}{3.008178in}}{\pgfqpoint{1.383612in}{3.002354in}}%
\pgfpathcurveto{\pgfqpoint{1.389436in}{2.996531in}}{\pgfqpoint{1.397336in}{2.993258in}}{\pgfqpoint{1.405572in}{2.993258in}}%
\pgfpathclose%
\pgfusepath{stroke,fill}%
\end{pgfscope}%
\begin{pgfscope}%
\pgfpathrectangle{\pgfqpoint{0.100000in}{0.212622in}}{\pgfqpoint{3.696000in}{3.696000in}}%
\pgfusepath{clip}%
\pgfsetbuttcap%
\pgfsetroundjoin%
\definecolor{currentfill}{rgb}{0.121569,0.466667,0.705882}%
\pgfsetfillcolor{currentfill}%
\pgfsetfillopacity{0.462919}%
\pgfsetlinewidth{1.003750pt}%
\definecolor{currentstroke}{rgb}{0.121569,0.466667,0.705882}%
\pgfsetstrokecolor{currentstroke}%
\pgfsetstrokeopacity{0.462919}%
\pgfsetdash{}{0pt}%
\pgfpathmoveto{\pgfqpoint{1.403853in}{2.989730in}}%
\pgfpathcurveto{\pgfqpoint{1.412089in}{2.989730in}}{\pgfqpoint{1.419989in}{2.993002in}}{\pgfqpoint{1.425813in}{2.998826in}}%
\pgfpathcurveto{\pgfqpoint{1.431637in}{3.004650in}}{\pgfqpoint{1.434909in}{3.012550in}}{\pgfqpoint{1.434909in}{3.020786in}}%
\pgfpathcurveto{\pgfqpoint{1.434909in}{3.029022in}}{\pgfqpoint{1.431637in}{3.036922in}}{\pgfqpoint{1.425813in}{3.042746in}}%
\pgfpathcurveto{\pgfqpoint{1.419989in}{3.048570in}}{\pgfqpoint{1.412089in}{3.051843in}}{\pgfqpoint{1.403853in}{3.051843in}}%
\pgfpathcurveto{\pgfqpoint{1.395617in}{3.051843in}}{\pgfqpoint{1.387717in}{3.048570in}}{\pgfqpoint{1.381893in}{3.042746in}}%
\pgfpathcurveto{\pgfqpoint{1.376069in}{3.036922in}}{\pgfqpoint{1.372796in}{3.029022in}}{\pgfqpoint{1.372796in}{3.020786in}}%
\pgfpathcurveto{\pgfqpoint{1.372796in}{3.012550in}}{\pgfqpoint{1.376069in}{3.004650in}}{\pgfqpoint{1.381893in}{2.998826in}}%
\pgfpathcurveto{\pgfqpoint{1.387717in}{2.993002in}}{\pgfqpoint{1.395617in}{2.989730in}}{\pgfqpoint{1.403853in}{2.989730in}}%
\pgfpathclose%
\pgfusepath{stroke,fill}%
\end{pgfscope}%
\begin{pgfscope}%
\pgfpathrectangle{\pgfqpoint{0.100000in}{0.212622in}}{\pgfqpoint{3.696000in}{3.696000in}}%
\pgfusepath{clip}%
\pgfsetbuttcap%
\pgfsetroundjoin%
\definecolor{currentfill}{rgb}{0.121569,0.466667,0.705882}%
\pgfsetfillcolor{currentfill}%
\pgfsetfillopacity{0.463110}%
\pgfsetlinewidth{1.003750pt}%
\definecolor{currentstroke}{rgb}{0.121569,0.466667,0.705882}%
\pgfsetstrokecolor{currentstroke}%
\pgfsetstrokeopacity{0.463110}%
\pgfsetdash{}{0pt}%
\pgfpathmoveto{\pgfqpoint{2.092873in}{2.781554in}}%
\pgfpathcurveto{\pgfqpoint{2.101109in}{2.781554in}}{\pgfqpoint{2.109009in}{2.784826in}}{\pgfqpoint{2.114833in}{2.790650in}}%
\pgfpathcurveto{\pgfqpoint{2.120657in}{2.796474in}}{\pgfqpoint{2.123930in}{2.804374in}}{\pgfqpoint{2.123930in}{2.812610in}}%
\pgfpathcurveto{\pgfqpoint{2.123930in}{2.820847in}}{\pgfqpoint{2.120657in}{2.828747in}}{\pgfqpoint{2.114833in}{2.834571in}}%
\pgfpathcurveto{\pgfqpoint{2.109009in}{2.840394in}}{\pgfqpoint{2.101109in}{2.843667in}}{\pgfqpoint{2.092873in}{2.843667in}}%
\pgfpathcurveto{\pgfqpoint{2.084637in}{2.843667in}}{\pgfqpoint{2.076737in}{2.840394in}}{\pgfqpoint{2.070913in}{2.834571in}}%
\pgfpathcurveto{\pgfqpoint{2.065089in}{2.828747in}}{\pgfqpoint{2.061817in}{2.820847in}}{\pgfqpoint{2.061817in}{2.812610in}}%
\pgfpathcurveto{\pgfqpoint{2.061817in}{2.804374in}}{\pgfqpoint{2.065089in}{2.796474in}}{\pgfqpoint{2.070913in}{2.790650in}}%
\pgfpathcurveto{\pgfqpoint{2.076737in}{2.784826in}}{\pgfqpoint{2.084637in}{2.781554in}}{\pgfqpoint{2.092873in}{2.781554in}}%
\pgfpathclose%
\pgfusepath{stroke,fill}%
\end{pgfscope}%
\begin{pgfscope}%
\pgfpathrectangle{\pgfqpoint{0.100000in}{0.212622in}}{\pgfqpoint{3.696000in}{3.696000in}}%
\pgfusepath{clip}%
\pgfsetbuttcap%
\pgfsetroundjoin%
\definecolor{currentfill}{rgb}{0.121569,0.466667,0.705882}%
\pgfsetfillcolor{currentfill}%
\pgfsetfillopacity{0.463680}%
\pgfsetlinewidth{1.003750pt}%
\definecolor{currentstroke}{rgb}{0.121569,0.466667,0.705882}%
\pgfsetstrokecolor{currentstroke}%
\pgfsetstrokeopacity{0.463680}%
\pgfsetdash{}{0pt}%
\pgfpathmoveto{\pgfqpoint{1.403074in}{2.987695in}}%
\pgfpathcurveto{\pgfqpoint{1.411311in}{2.987695in}}{\pgfqpoint{1.419211in}{2.990967in}}{\pgfqpoint{1.425035in}{2.996791in}}%
\pgfpathcurveto{\pgfqpoint{1.430859in}{3.002615in}}{\pgfqpoint{1.434131in}{3.010515in}}{\pgfqpoint{1.434131in}{3.018752in}}%
\pgfpathcurveto{\pgfqpoint{1.434131in}{3.026988in}}{\pgfqpoint{1.430859in}{3.034888in}}{\pgfqpoint{1.425035in}{3.040712in}}%
\pgfpathcurveto{\pgfqpoint{1.419211in}{3.046536in}}{\pgfqpoint{1.411311in}{3.049808in}}{\pgfqpoint{1.403074in}{3.049808in}}%
\pgfpathcurveto{\pgfqpoint{1.394838in}{3.049808in}}{\pgfqpoint{1.386938in}{3.046536in}}{\pgfqpoint{1.381114in}{3.040712in}}%
\pgfpathcurveto{\pgfqpoint{1.375290in}{3.034888in}}{\pgfqpoint{1.372018in}{3.026988in}}{\pgfqpoint{1.372018in}{3.018752in}}%
\pgfpathcurveto{\pgfqpoint{1.372018in}{3.010515in}}{\pgfqpoint{1.375290in}{3.002615in}}{\pgfqpoint{1.381114in}{2.996791in}}%
\pgfpathcurveto{\pgfqpoint{1.386938in}{2.990967in}}{\pgfqpoint{1.394838in}{2.987695in}}{\pgfqpoint{1.403074in}{2.987695in}}%
\pgfpathclose%
\pgfusepath{stroke,fill}%
\end{pgfscope}%
\begin{pgfscope}%
\pgfpathrectangle{\pgfqpoint{0.100000in}{0.212622in}}{\pgfqpoint{3.696000in}{3.696000in}}%
\pgfusepath{clip}%
\pgfsetbuttcap%
\pgfsetroundjoin%
\definecolor{currentfill}{rgb}{0.121569,0.466667,0.705882}%
\pgfsetfillcolor{currentfill}%
\pgfsetfillopacity{0.465023}%
\pgfsetlinewidth{1.003750pt}%
\definecolor{currentstroke}{rgb}{0.121569,0.466667,0.705882}%
\pgfsetstrokecolor{currentstroke}%
\pgfsetstrokeopacity{0.465023}%
\pgfsetdash{}{0pt}%
\pgfpathmoveto{\pgfqpoint{1.401329in}{2.984061in}}%
\pgfpathcurveto{\pgfqpoint{1.409565in}{2.984061in}}{\pgfqpoint{1.417465in}{2.987334in}}{\pgfqpoint{1.423289in}{2.993158in}}%
\pgfpathcurveto{\pgfqpoint{1.429113in}{2.998981in}}{\pgfqpoint{1.432386in}{3.006882in}}{\pgfqpoint{1.432386in}{3.015118in}}%
\pgfpathcurveto{\pgfqpoint{1.432386in}{3.023354in}}{\pgfqpoint{1.429113in}{3.031254in}}{\pgfqpoint{1.423289in}{3.037078in}}%
\pgfpathcurveto{\pgfqpoint{1.417465in}{3.042902in}}{\pgfqpoint{1.409565in}{3.046174in}}{\pgfqpoint{1.401329in}{3.046174in}}%
\pgfpathcurveto{\pgfqpoint{1.393093in}{3.046174in}}{\pgfqpoint{1.385193in}{3.042902in}}{\pgfqpoint{1.379369in}{3.037078in}}%
\pgfpathcurveto{\pgfqpoint{1.373545in}{3.031254in}}{\pgfqpoint{1.370273in}{3.023354in}}{\pgfqpoint{1.370273in}{3.015118in}}%
\pgfpathcurveto{\pgfqpoint{1.370273in}{3.006882in}}{\pgfqpoint{1.373545in}{2.998981in}}{\pgfqpoint{1.379369in}{2.993158in}}%
\pgfpathcurveto{\pgfqpoint{1.385193in}{2.987334in}}{\pgfqpoint{1.393093in}{2.984061in}}{\pgfqpoint{1.401329in}{2.984061in}}%
\pgfpathclose%
\pgfusepath{stroke,fill}%
\end{pgfscope}%
\begin{pgfscope}%
\pgfpathrectangle{\pgfqpoint{0.100000in}{0.212622in}}{\pgfqpoint{3.696000in}{3.696000in}}%
\pgfusepath{clip}%
\pgfsetbuttcap%
\pgfsetroundjoin%
\definecolor{currentfill}{rgb}{0.121569,0.466667,0.705882}%
\pgfsetfillcolor{currentfill}%
\pgfsetfillopacity{0.467537}%
\pgfsetlinewidth{1.003750pt}%
\definecolor{currentstroke}{rgb}{0.121569,0.466667,0.705882}%
\pgfsetstrokecolor{currentstroke}%
\pgfsetstrokeopacity{0.467537}%
\pgfsetdash{}{0pt}%
\pgfpathmoveto{\pgfqpoint{1.398643in}{2.977392in}}%
\pgfpathcurveto{\pgfqpoint{1.406879in}{2.977392in}}{\pgfqpoint{1.414780in}{2.980665in}}{\pgfqpoint{1.420603in}{2.986489in}}%
\pgfpathcurveto{\pgfqpoint{1.426427in}{2.992312in}}{\pgfqpoint{1.429700in}{3.000212in}}{\pgfqpoint{1.429700in}{3.008449in}}%
\pgfpathcurveto{\pgfqpoint{1.429700in}{3.016685in}}{\pgfqpoint{1.426427in}{3.024585in}}{\pgfqpoint{1.420603in}{3.030409in}}%
\pgfpathcurveto{\pgfqpoint{1.414780in}{3.036233in}}{\pgfqpoint{1.406879in}{3.039505in}}{\pgfqpoint{1.398643in}{3.039505in}}%
\pgfpathcurveto{\pgfqpoint{1.390407in}{3.039505in}}{\pgfqpoint{1.382507in}{3.036233in}}{\pgfqpoint{1.376683in}{3.030409in}}%
\pgfpathcurveto{\pgfqpoint{1.370859in}{3.024585in}}{\pgfqpoint{1.367587in}{3.016685in}}{\pgfqpoint{1.367587in}{3.008449in}}%
\pgfpathcurveto{\pgfqpoint{1.367587in}{3.000212in}}{\pgfqpoint{1.370859in}{2.992312in}}{\pgfqpoint{1.376683in}{2.986489in}}%
\pgfpathcurveto{\pgfqpoint{1.382507in}{2.980665in}}{\pgfqpoint{1.390407in}{2.977392in}}{\pgfqpoint{1.398643in}{2.977392in}}%
\pgfpathclose%
\pgfusepath{stroke,fill}%
\end{pgfscope}%
\begin{pgfscope}%
\pgfpathrectangle{\pgfqpoint{0.100000in}{0.212622in}}{\pgfqpoint{3.696000in}{3.696000in}}%
\pgfusepath{clip}%
\pgfsetbuttcap%
\pgfsetroundjoin%
\definecolor{currentfill}{rgb}{0.121569,0.466667,0.705882}%
\pgfsetfillcolor{currentfill}%
\pgfsetfillopacity{0.471147}%
\pgfsetlinewidth{1.003750pt}%
\definecolor{currentstroke}{rgb}{0.121569,0.466667,0.705882}%
\pgfsetstrokecolor{currentstroke}%
\pgfsetstrokeopacity{0.471147}%
\pgfsetdash{}{0pt}%
\pgfpathmoveto{\pgfqpoint{2.108197in}{2.754839in}}%
\pgfpathcurveto{\pgfqpoint{2.116434in}{2.754839in}}{\pgfqpoint{2.124334in}{2.758111in}}{\pgfqpoint{2.130158in}{2.763935in}}%
\pgfpathcurveto{\pgfqpoint{2.135982in}{2.769759in}}{\pgfqpoint{2.139254in}{2.777659in}}{\pgfqpoint{2.139254in}{2.785895in}}%
\pgfpathcurveto{\pgfqpoint{2.139254in}{2.794131in}}{\pgfqpoint{2.135982in}{2.802031in}}{\pgfqpoint{2.130158in}{2.807855in}}%
\pgfpathcurveto{\pgfqpoint{2.124334in}{2.813679in}}{\pgfqpoint{2.116434in}{2.816952in}}{\pgfqpoint{2.108197in}{2.816952in}}%
\pgfpathcurveto{\pgfqpoint{2.099961in}{2.816952in}}{\pgfqpoint{2.092061in}{2.813679in}}{\pgfqpoint{2.086237in}{2.807855in}}%
\pgfpathcurveto{\pgfqpoint{2.080413in}{2.802031in}}{\pgfqpoint{2.077141in}{2.794131in}}{\pgfqpoint{2.077141in}{2.785895in}}%
\pgfpathcurveto{\pgfqpoint{2.077141in}{2.777659in}}{\pgfqpoint{2.080413in}{2.769759in}}{\pgfqpoint{2.086237in}{2.763935in}}%
\pgfpathcurveto{\pgfqpoint{2.092061in}{2.758111in}}{\pgfqpoint{2.099961in}{2.754839in}}{\pgfqpoint{2.108197in}{2.754839in}}%
\pgfpathclose%
\pgfusepath{stroke,fill}%
\end{pgfscope}%
\begin{pgfscope}%
\pgfpathrectangle{\pgfqpoint{0.100000in}{0.212622in}}{\pgfqpoint{3.696000in}{3.696000in}}%
\pgfusepath{clip}%
\pgfsetbuttcap%
\pgfsetroundjoin%
\definecolor{currentfill}{rgb}{0.121569,0.466667,0.705882}%
\pgfsetfillcolor{currentfill}%
\pgfsetfillopacity{0.472071}%
\pgfsetlinewidth{1.003750pt}%
\definecolor{currentstroke}{rgb}{0.121569,0.466667,0.705882}%
\pgfsetstrokecolor{currentstroke}%
\pgfsetstrokeopacity{0.472071}%
\pgfsetdash{}{0pt}%
\pgfpathmoveto{\pgfqpoint{1.393355in}{2.965362in}}%
\pgfpathcurveto{\pgfqpoint{1.401591in}{2.965362in}}{\pgfqpoint{1.409491in}{2.968634in}}{\pgfqpoint{1.415315in}{2.974458in}}%
\pgfpathcurveto{\pgfqpoint{1.421139in}{2.980282in}}{\pgfqpoint{1.424411in}{2.988182in}}{\pgfqpoint{1.424411in}{2.996419in}}%
\pgfpathcurveto{\pgfqpoint{1.424411in}{3.004655in}}{\pgfqpoint{1.421139in}{3.012555in}}{\pgfqpoint{1.415315in}{3.018379in}}%
\pgfpathcurveto{\pgfqpoint{1.409491in}{3.024203in}}{\pgfqpoint{1.401591in}{3.027475in}}{\pgfqpoint{1.393355in}{3.027475in}}%
\pgfpathcurveto{\pgfqpoint{1.385119in}{3.027475in}}{\pgfqpoint{1.377219in}{3.024203in}}{\pgfqpoint{1.371395in}{3.018379in}}%
\pgfpathcurveto{\pgfqpoint{1.365571in}{3.012555in}}{\pgfqpoint{1.362298in}{3.004655in}}{\pgfqpoint{1.362298in}{2.996419in}}%
\pgfpathcurveto{\pgfqpoint{1.362298in}{2.988182in}}{\pgfqpoint{1.365571in}{2.980282in}}{\pgfqpoint{1.371395in}{2.974458in}}%
\pgfpathcurveto{\pgfqpoint{1.377219in}{2.968634in}}{\pgfqpoint{1.385119in}{2.965362in}}{\pgfqpoint{1.393355in}{2.965362in}}%
\pgfpathclose%
\pgfusepath{stroke,fill}%
\end{pgfscope}%
\begin{pgfscope}%
\pgfpathrectangle{\pgfqpoint{0.100000in}{0.212622in}}{\pgfqpoint{3.696000in}{3.696000in}}%
\pgfusepath{clip}%
\pgfsetbuttcap%
\pgfsetroundjoin%
\definecolor{currentfill}{rgb}{0.121569,0.466667,0.705882}%
\pgfsetfillcolor{currentfill}%
\pgfsetfillopacity{0.475715}%
\pgfsetlinewidth{1.003750pt}%
\definecolor{currentstroke}{rgb}{0.121569,0.466667,0.705882}%
\pgfsetstrokecolor{currentstroke}%
\pgfsetstrokeopacity{0.475715}%
\pgfsetdash{}{0pt}%
\pgfpathmoveto{\pgfqpoint{1.389348in}{2.955549in}}%
\pgfpathcurveto{\pgfqpoint{1.397585in}{2.955549in}}{\pgfqpoint{1.405485in}{2.958821in}}{\pgfqpoint{1.411309in}{2.964645in}}%
\pgfpathcurveto{\pgfqpoint{1.417133in}{2.970469in}}{\pgfqpoint{1.420405in}{2.978369in}}{\pgfqpoint{1.420405in}{2.986606in}}%
\pgfpathcurveto{\pgfqpoint{1.420405in}{2.994842in}}{\pgfqpoint{1.417133in}{3.002742in}}{\pgfqpoint{1.411309in}{3.008566in}}%
\pgfpathcurveto{\pgfqpoint{1.405485in}{3.014390in}}{\pgfqpoint{1.397585in}{3.017662in}}{\pgfqpoint{1.389348in}{3.017662in}}%
\pgfpathcurveto{\pgfqpoint{1.381112in}{3.017662in}}{\pgfqpoint{1.373212in}{3.014390in}}{\pgfqpoint{1.367388in}{3.008566in}}%
\pgfpathcurveto{\pgfqpoint{1.361564in}{3.002742in}}{\pgfqpoint{1.358292in}{2.994842in}}{\pgfqpoint{1.358292in}{2.986606in}}%
\pgfpathcurveto{\pgfqpoint{1.358292in}{2.978369in}}{\pgfqpoint{1.361564in}{2.970469in}}{\pgfqpoint{1.367388in}{2.964645in}}%
\pgfpathcurveto{\pgfqpoint{1.373212in}{2.958821in}}{\pgfqpoint{1.381112in}{2.955549in}}{\pgfqpoint{1.389348in}{2.955549in}}%
\pgfpathclose%
\pgfusepath{stroke,fill}%
\end{pgfscope}%
\begin{pgfscope}%
\pgfpathrectangle{\pgfqpoint{0.100000in}{0.212622in}}{\pgfqpoint{3.696000in}{3.696000in}}%
\pgfusepath{clip}%
\pgfsetbuttcap%
\pgfsetroundjoin%
\definecolor{currentfill}{rgb}{0.121569,0.466667,0.705882}%
\pgfsetfillcolor{currentfill}%
\pgfsetfillopacity{0.480349}%
\pgfsetlinewidth{1.003750pt}%
\definecolor{currentstroke}{rgb}{0.121569,0.466667,0.705882}%
\pgfsetstrokecolor{currentstroke}%
\pgfsetstrokeopacity{0.480349}%
\pgfsetdash{}{0pt}%
\pgfpathmoveto{\pgfqpoint{2.123600in}{2.726527in}}%
\pgfpathcurveto{\pgfqpoint{2.131836in}{2.726527in}}{\pgfqpoint{2.139736in}{2.729800in}}{\pgfqpoint{2.145560in}{2.735624in}}%
\pgfpathcurveto{\pgfqpoint{2.151384in}{2.741447in}}{\pgfqpoint{2.154656in}{2.749348in}}{\pgfqpoint{2.154656in}{2.757584in}}%
\pgfpathcurveto{\pgfqpoint{2.154656in}{2.765820in}}{\pgfqpoint{2.151384in}{2.773720in}}{\pgfqpoint{2.145560in}{2.779544in}}%
\pgfpathcurveto{\pgfqpoint{2.139736in}{2.785368in}}{\pgfqpoint{2.131836in}{2.788640in}}{\pgfqpoint{2.123600in}{2.788640in}}%
\pgfpathcurveto{\pgfqpoint{2.115363in}{2.788640in}}{\pgfqpoint{2.107463in}{2.785368in}}{\pgfqpoint{2.101639in}{2.779544in}}%
\pgfpathcurveto{\pgfqpoint{2.095815in}{2.773720in}}{\pgfqpoint{2.092543in}{2.765820in}}{\pgfqpoint{2.092543in}{2.757584in}}%
\pgfpathcurveto{\pgfqpoint{2.092543in}{2.749348in}}{\pgfqpoint{2.095815in}{2.741447in}}{\pgfqpoint{2.101639in}{2.735624in}}%
\pgfpathcurveto{\pgfqpoint{2.107463in}{2.729800in}}{\pgfqpoint{2.115363in}{2.726527in}}{\pgfqpoint{2.123600in}{2.726527in}}%
\pgfpathclose%
\pgfusepath{stroke,fill}%
\end{pgfscope}%
\begin{pgfscope}%
\pgfpathrectangle{\pgfqpoint{0.100000in}{0.212622in}}{\pgfqpoint{3.696000in}{3.696000in}}%
\pgfusepath{clip}%
\pgfsetbuttcap%
\pgfsetroundjoin%
\definecolor{currentfill}{rgb}{0.121569,0.466667,0.705882}%
\pgfsetfillcolor{currentfill}%
\pgfsetfillopacity{0.482351}%
\pgfsetlinewidth{1.003750pt}%
\definecolor{currentstroke}{rgb}{0.121569,0.466667,0.705882}%
\pgfsetstrokecolor{currentstroke}%
\pgfsetstrokeopacity{0.482351}%
\pgfsetdash{}{0pt}%
\pgfpathmoveto{\pgfqpoint{1.381838in}{2.937851in}}%
\pgfpathcurveto{\pgfqpoint{1.390074in}{2.937851in}}{\pgfqpoint{1.397974in}{2.941123in}}{\pgfqpoint{1.403798in}{2.946947in}}%
\pgfpathcurveto{\pgfqpoint{1.409622in}{2.952771in}}{\pgfqpoint{1.412894in}{2.960671in}}{\pgfqpoint{1.412894in}{2.968907in}}%
\pgfpathcurveto{\pgfqpoint{1.412894in}{2.977143in}}{\pgfqpoint{1.409622in}{2.985043in}}{\pgfqpoint{1.403798in}{2.990867in}}%
\pgfpathcurveto{\pgfqpoint{1.397974in}{2.996691in}}{\pgfqpoint{1.390074in}{2.999964in}}{\pgfqpoint{1.381838in}{2.999964in}}%
\pgfpathcurveto{\pgfqpoint{1.373602in}{2.999964in}}{\pgfqpoint{1.365702in}{2.996691in}}{\pgfqpoint{1.359878in}{2.990867in}}%
\pgfpathcurveto{\pgfqpoint{1.354054in}{2.985043in}}{\pgfqpoint{1.350781in}{2.977143in}}{\pgfqpoint{1.350781in}{2.968907in}}%
\pgfpathcurveto{\pgfqpoint{1.350781in}{2.960671in}}{\pgfqpoint{1.354054in}{2.952771in}}{\pgfqpoint{1.359878in}{2.946947in}}%
\pgfpathcurveto{\pgfqpoint{1.365702in}{2.941123in}}{\pgfqpoint{1.373602in}{2.937851in}}{\pgfqpoint{1.381838in}{2.937851in}}%
\pgfpathclose%
\pgfusepath{stroke,fill}%
\end{pgfscope}%
\begin{pgfscope}%
\pgfpathrectangle{\pgfqpoint{0.100000in}{0.212622in}}{\pgfqpoint{3.696000in}{3.696000in}}%
\pgfusepath{clip}%
\pgfsetbuttcap%
\pgfsetroundjoin%
\definecolor{currentfill}{rgb}{0.121569,0.466667,0.705882}%
\pgfsetfillcolor{currentfill}%
\pgfsetfillopacity{0.485117}%
\pgfsetlinewidth{1.003750pt}%
\definecolor{currentstroke}{rgb}{0.121569,0.466667,0.705882}%
\pgfsetstrokecolor{currentstroke}%
\pgfsetstrokeopacity{0.485117}%
\pgfsetdash{}{0pt}%
\pgfpathmoveto{\pgfqpoint{2.132462in}{2.710028in}}%
\pgfpathcurveto{\pgfqpoint{2.140699in}{2.710028in}}{\pgfqpoint{2.148599in}{2.713301in}}{\pgfqpoint{2.154423in}{2.719125in}}%
\pgfpathcurveto{\pgfqpoint{2.160247in}{2.724949in}}{\pgfqpoint{2.163519in}{2.732849in}}{\pgfqpoint{2.163519in}{2.741085in}}%
\pgfpathcurveto{\pgfqpoint{2.163519in}{2.749321in}}{\pgfqpoint{2.160247in}{2.757221in}}{\pgfqpoint{2.154423in}{2.763045in}}%
\pgfpathcurveto{\pgfqpoint{2.148599in}{2.768869in}}{\pgfqpoint{2.140699in}{2.772141in}}{\pgfqpoint{2.132462in}{2.772141in}}%
\pgfpathcurveto{\pgfqpoint{2.124226in}{2.772141in}}{\pgfqpoint{2.116326in}{2.768869in}}{\pgfqpoint{2.110502in}{2.763045in}}%
\pgfpathcurveto{\pgfqpoint{2.104678in}{2.757221in}}{\pgfqpoint{2.101406in}{2.749321in}}{\pgfqpoint{2.101406in}{2.741085in}}%
\pgfpathcurveto{\pgfqpoint{2.101406in}{2.732849in}}{\pgfqpoint{2.104678in}{2.724949in}}{\pgfqpoint{2.110502in}{2.719125in}}%
\pgfpathcurveto{\pgfqpoint{2.116326in}{2.713301in}}{\pgfqpoint{2.124226in}{2.710028in}}{\pgfqpoint{2.132462in}{2.710028in}}%
\pgfpathclose%
\pgfusepath{stroke,fill}%
\end{pgfscope}%
\begin{pgfscope}%
\pgfpathrectangle{\pgfqpoint{0.100000in}{0.212622in}}{\pgfqpoint{3.696000in}{3.696000in}}%
\pgfusepath{clip}%
\pgfsetbuttcap%
\pgfsetroundjoin%
\definecolor{currentfill}{rgb}{0.121569,0.466667,0.705882}%
\pgfsetfillcolor{currentfill}%
\pgfsetfillopacity{0.487956}%
\pgfsetlinewidth{1.003750pt}%
\definecolor{currentstroke}{rgb}{0.121569,0.466667,0.705882}%
\pgfsetstrokecolor{currentstroke}%
\pgfsetstrokeopacity{0.487956}%
\pgfsetdash{}{0pt}%
\pgfpathmoveto{\pgfqpoint{1.375703in}{2.922924in}}%
\pgfpathcurveto{\pgfqpoint{1.383940in}{2.922924in}}{\pgfqpoint{1.391840in}{2.926196in}}{\pgfqpoint{1.397664in}{2.932020in}}%
\pgfpathcurveto{\pgfqpoint{1.403488in}{2.937844in}}{\pgfqpoint{1.406760in}{2.945744in}}{\pgfqpoint{1.406760in}{2.953980in}}%
\pgfpathcurveto{\pgfqpoint{1.406760in}{2.962216in}}{\pgfqpoint{1.403488in}{2.970116in}}{\pgfqpoint{1.397664in}{2.975940in}}%
\pgfpathcurveto{\pgfqpoint{1.391840in}{2.981764in}}{\pgfqpoint{1.383940in}{2.985037in}}{\pgfqpoint{1.375703in}{2.985037in}}%
\pgfpathcurveto{\pgfqpoint{1.367467in}{2.985037in}}{\pgfqpoint{1.359567in}{2.981764in}}{\pgfqpoint{1.353743in}{2.975940in}}%
\pgfpathcurveto{\pgfqpoint{1.347919in}{2.970116in}}{\pgfqpoint{1.344647in}{2.962216in}}{\pgfqpoint{1.344647in}{2.953980in}}%
\pgfpathcurveto{\pgfqpoint{1.344647in}{2.945744in}}{\pgfqpoint{1.347919in}{2.937844in}}{\pgfqpoint{1.353743in}{2.932020in}}%
\pgfpathcurveto{\pgfqpoint{1.359567in}{2.926196in}}{\pgfqpoint{1.367467in}{2.922924in}}{\pgfqpoint{1.375703in}{2.922924in}}%
\pgfpathclose%
\pgfusepath{stroke,fill}%
\end{pgfscope}%
\begin{pgfscope}%
\pgfpathrectangle{\pgfqpoint{0.100000in}{0.212622in}}{\pgfqpoint{3.696000in}{3.696000in}}%
\pgfusepath{clip}%
\pgfsetbuttcap%
\pgfsetroundjoin%
\definecolor{currentfill}{rgb}{0.121569,0.466667,0.705882}%
\pgfsetfillcolor{currentfill}%
\pgfsetfillopacity{0.490753}%
\pgfsetlinewidth{1.003750pt}%
\definecolor{currentstroke}{rgb}{0.121569,0.466667,0.705882}%
\pgfsetstrokecolor{currentstroke}%
\pgfsetstrokeopacity{0.490753}%
\pgfsetdash{}{0pt}%
\pgfpathmoveto{\pgfqpoint{2.141107in}{2.692154in}}%
\pgfpathcurveto{\pgfqpoint{2.149343in}{2.692154in}}{\pgfqpoint{2.157243in}{2.695426in}}{\pgfqpoint{2.163067in}{2.701250in}}%
\pgfpathcurveto{\pgfqpoint{2.168891in}{2.707074in}}{\pgfqpoint{2.172163in}{2.714974in}}{\pgfqpoint{2.172163in}{2.723210in}}%
\pgfpathcurveto{\pgfqpoint{2.172163in}{2.731447in}}{\pgfqpoint{2.168891in}{2.739347in}}{\pgfqpoint{2.163067in}{2.745171in}}%
\pgfpathcurveto{\pgfqpoint{2.157243in}{2.750995in}}{\pgfqpoint{2.149343in}{2.754267in}}{\pgfqpoint{2.141107in}{2.754267in}}%
\pgfpathcurveto{\pgfqpoint{2.132870in}{2.754267in}}{\pgfqpoint{2.124970in}{2.750995in}}{\pgfqpoint{2.119146in}{2.745171in}}%
\pgfpathcurveto{\pgfqpoint{2.113322in}{2.739347in}}{\pgfqpoint{2.110050in}{2.731447in}}{\pgfqpoint{2.110050in}{2.723210in}}%
\pgfpathcurveto{\pgfqpoint{2.110050in}{2.714974in}}{\pgfqpoint{2.113322in}{2.707074in}}{\pgfqpoint{2.119146in}{2.701250in}}%
\pgfpathcurveto{\pgfqpoint{2.124970in}{2.695426in}}{\pgfqpoint{2.132870in}{2.692154in}}{\pgfqpoint{2.141107in}{2.692154in}}%
\pgfpathclose%
\pgfusepath{stroke,fill}%
\end{pgfscope}%
\begin{pgfscope}%
\pgfpathrectangle{\pgfqpoint{0.100000in}{0.212622in}}{\pgfqpoint{3.696000in}{3.696000in}}%
\pgfusepath{clip}%
\pgfsetbuttcap%
\pgfsetroundjoin%
\definecolor{currentfill}{rgb}{0.121569,0.466667,0.705882}%
\pgfsetfillcolor{currentfill}%
\pgfsetfillopacity{0.496558}%
\pgfsetlinewidth{1.003750pt}%
\definecolor{currentstroke}{rgb}{0.121569,0.466667,0.705882}%
\pgfsetstrokecolor{currentstroke}%
\pgfsetstrokeopacity{0.496558}%
\pgfsetdash{}{0pt}%
\pgfpathmoveto{\pgfqpoint{2.151078in}{2.671619in}}%
\pgfpathcurveto{\pgfqpoint{2.159315in}{2.671619in}}{\pgfqpoint{2.167215in}{2.674891in}}{\pgfqpoint{2.173039in}{2.680715in}}%
\pgfpathcurveto{\pgfqpoint{2.178862in}{2.686539in}}{\pgfqpoint{2.182135in}{2.694439in}}{\pgfqpoint{2.182135in}{2.702676in}}%
\pgfpathcurveto{\pgfqpoint{2.182135in}{2.710912in}}{\pgfqpoint{2.178862in}{2.718812in}}{\pgfqpoint{2.173039in}{2.724636in}}%
\pgfpathcurveto{\pgfqpoint{2.167215in}{2.730460in}}{\pgfqpoint{2.159315in}{2.733732in}}{\pgfqpoint{2.151078in}{2.733732in}}%
\pgfpathcurveto{\pgfqpoint{2.142842in}{2.733732in}}{\pgfqpoint{2.134942in}{2.730460in}}{\pgfqpoint{2.129118in}{2.724636in}}%
\pgfpathcurveto{\pgfqpoint{2.123294in}{2.718812in}}{\pgfqpoint{2.120022in}{2.710912in}}{\pgfqpoint{2.120022in}{2.702676in}}%
\pgfpathcurveto{\pgfqpoint{2.120022in}{2.694439in}}{\pgfqpoint{2.123294in}{2.686539in}}{\pgfqpoint{2.129118in}{2.680715in}}%
\pgfpathcurveto{\pgfqpoint{2.134942in}{2.674891in}}{\pgfqpoint{2.142842in}{2.671619in}}{\pgfqpoint{2.151078in}{2.671619in}}%
\pgfpathclose%
\pgfusepath{stroke,fill}%
\end{pgfscope}%
\begin{pgfscope}%
\pgfpathrectangle{\pgfqpoint{0.100000in}{0.212622in}}{\pgfqpoint{3.696000in}{3.696000in}}%
\pgfusepath{clip}%
\pgfsetbuttcap%
\pgfsetroundjoin%
\definecolor{currentfill}{rgb}{0.121569,0.466667,0.705882}%
\pgfsetfillcolor{currentfill}%
\pgfsetfillopacity{0.498120}%
\pgfsetlinewidth{1.003750pt}%
\definecolor{currentstroke}{rgb}{0.121569,0.466667,0.705882}%
\pgfsetstrokecolor{currentstroke}%
\pgfsetstrokeopacity{0.498120}%
\pgfsetdash{}{0pt}%
\pgfpathmoveto{\pgfqpoint{1.363827in}{2.896032in}}%
\pgfpathcurveto{\pgfqpoint{1.372063in}{2.896032in}}{\pgfqpoint{1.379963in}{2.899304in}}{\pgfqpoint{1.385787in}{2.905128in}}%
\pgfpathcurveto{\pgfqpoint{1.391611in}{2.910952in}}{\pgfqpoint{1.394883in}{2.918852in}}{\pgfqpoint{1.394883in}{2.927089in}}%
\pgfpathcurveto{\pgfqpoint{1.394883in}{2.935325in}}{\pgfqpoint{1.391611in}{2.943225in}}{\pgfqpoint{1.385787in}{2.949049in}}%
\pgfpathcurveto{\pgfqpoint{1.379963in}{2.954873in}}{\pgfqpoint{1.372063in}{2.958145in}}{\pgfqpoint{1.363827in}{2.958145in}}%
\pgfpathcurveto{\pgfqpoint{1.355590in}{2.958145in}}{\pgfqpoint{1.347690in}{2.954873in}}{\pgfqpoint{1.341866in}{2.949049in}}%
\pgfpathcurveto{\pgfqpoint{1.336043in}{2.943225in}}{\pgfqpoint{1.332770in}{2.935325in}}{\pgfqpoint{1.332770in}{2.927089in}}%
\pgfpathcurveto{\pgfqpoint{1.332770in}{2.918852in}}{\pgfqpoint{1.336043in}{2.910952in}}{\pgfqpoint{1.341866in}{2.905128in}}%
\pgfpathcurveto{\pgfqpoint{1.347690in}{2.899304in}}{\pgfqpoint{1.355590in}{2.896032in}}{\pgfqpoint{1.363827in}{2.896032in}}%
\pgfpathclose%
\pgfusepath{stroke,fill}%
\end{pgfscope}%
\begin{pgfscope}%
\pgfpathrectangle{\pgfqpoint{0.100000in}{0.212622in}}{\pgfqpoint{3.696000in}{3.696000in}}%
\pgfusepath{clip}%
\pgfsetbuttcap%
\pgfsetroundjoin%
\definecolor{currentfill}{rgb}{0.121569,0.466667,0.705882}%
\pgfsetfillcolor{currentfill}%
\pgfsetfillopacity{0.503426}%
\pgfsetlinewidth{1.003750pt}%
\definecolor{currentstroke}{rgb}{0.121569,0.466667,0.705882}%
\pgfsetstrokecolor{currentstroke}%
\pgfsetstrokeopacity{0.503426}%
\pgfsetdash{}{0pt}%
\pgfpathmoveto{\pgfqpoint{2.160790in}{2.649438in}}%
\pgfpathcurveto{\pgfqpoint{2.169026in}{2.649438in}}{\pgfqpoint{2.176926in}{2.652710in}}{\pgfqpoint{2.182750in}{2.658534in}}%
\pgfpathcurveto{\pgfqpoint{2.188574in}{2.664358in}}{\pgfqpoint{2.191846in}{2.672258in}}{\pgfqpoint{2.191846in}{2.680494in}}%
\pgfpathcurveto{\pgfqpoint{2.191846in}{2.688730in}}{\pgfqpoint{2.188574in}{2.696630in}}{\pgfqpoint{2.182750in}{2.702454in}}%
\pgfpathcurveto{\pgfqpoint{2.176926in}{2.708278in}}{\pgfqpoint{2.169026in}{2.711551in}}{\pgfqpoint{2.160790in}{2.711551in}}%
\pgfpathcurveto{\pgfqpoint{2.152553in}{2.711551in}}{\pgfqpoint{2.144653in}{2.708278in}}{\pgfqpoint{2.138829in}{2.702454in}}%
\pgfpathcurveto{\pgfqpoint{2.133005in}{2.696630in}}{\pgfqpoint{2.129733in}{2.688730in}}{\pgfqpoint{2.129733in}{2.680494in}}%
\pgfpathcurveto{\pgfqpoint{2.129733in}{2.672258in}}{\pgfqpoint{2.133005in}{2.664358in}}{\pgfqpoint{2.138829in}{2.658534in}}%
\pgfpathcurveto{\pgfqpoint{2.144653in}{2.652710in}}{\pgfqpoint{2.152553in}{2.649438in}}{\pgfqpoint{2.160790in}{2.649438in}}%
\pgfpathclose%
\pgfusepath{stroke,fill}%
\end{pgfscope}%
\begin{pgfscope}%
\pgfpathrectangle{\pgfqpoint{0.100000in}{0.212622in}}{\pgfqpoint{3.696000in}{3.696000in}}%
\pgfusepath{clip}%
\pgfsetbuttcap%
\pgfsetroundjoin%
\definecolor{currentfill}{rgb}{0.121569,0.466667,0.705882}%
\pgfsetfillcolor{currentfill}%
\pgfsetfillopacity{0.507927}%
\pgfsetlinewidth{1.003750pt}%
\definecolor{currentstroke}{rgb}{0.121569,0.466667,0.705882}%
\pgfsetstrokecolor{currentstroke}%
\pgfsetstrokeopacity{0.507927}%
\pgfsetdash{}{0pt}%
\pgfpathmoveto{\pgfqpoint{1.353922in}{2.870131in}}%
\pgfpathcurveto{\pgfqpoint{1.362158in}{2.870131in}}{\pgfqpoint{1.370058in}{2.873403in}}{\pgfqpoint{1.375882in}{2.879227in}}%
\pgfpathcurveto{\pgfqpoint{1.381706in}{2.885051in}}{\pgfqpoint{1.384978in}{2.892951in}}{\pgfqpoint{1.384978in}{2.901187in}}%
\pgfpathcurveto{\pgfqpoint{1.384978in}{2.909423in}}{\pgfqpoint{1.381706in}{2.917323in}}{\pgfqpoint{1.375882in}{2.923147in}}%
\pgfpathcurveto{\pgfqpoint{1.370058in}{2.928971in}}{\pgfqpoint{1.362158in}{2.932244in}}{\pgfqpoint{1.353922in}{2.932244in}}%
\pgfpathcurveto{\pgfqpoint{1.345686in}{2.932244in}}{\pgfqpoint{1.337785in}{2.928971in}}{\pgfqpoint{1.331962in}{2.923147in}}%
\pgfpathcurveto{\pgfqpoint{1.326138in}{2.917323in}}{\pgfqpoint{1.322865in}{2.909423in}}{\pgfqpoint{1.322865in}{2.901187in}}%
\pgfpathcurveto{\pgfqpoint{1.322865in}{2.892951in}}{\pgfqpoint{1.326138in}{2.885051in}}{\pgfqpoint{1.331962in}{2.879227in}}%
\pgfpathcurveto{\pgfqpoint{1.337785in}{2.873403in}}{\pgfqpoint{1.345686in}{2.870131in}}{\pgfqpoint{1.353922in}{2.870131in}}%
\pgfpathclose%
\pgfusepath{stroke,fill}%
\end{pgfscope}%
\begin{pgfscope}%
\pgfpathrectangle{\pgfqpoint{0.100000in}{0.212622in}}{\pgfqpoint{3.696000in}{3.696000in}}%
\pgfusepath{clip}%
\pgfsetbuttcap%
\pgfsetroundjoin%
\definecolor{currentfill}{rgb}{0.121569,0.466667,0.705882}%
\pgfsetfillcolor{currentfill}%
\pgfsetfillopacity{0.510679}%
\pgfsetlinewidth{1.003750pt}%
\definecolor{currentstroke}{rgb}{0.121569,0.466667,0.705882}%
\pgfsetstrokecolor{currentstroke}%
\pgfsetstrokeopacity{0.510679}%
\pgfsetdash{}{0pt}%
\pgfpathmoveto{\pgfqpoint{2.173850in}{2.622724in}}%
\pgfpathcurveto{\pgfqpoint{2.182086in}{2.622724in}}{\pgfqpoint{2.189986in}{2.625996in}}{\pgfqpoint{2.195810in}{2.631820in}}%
\pgfpathcurveto{\pgfqpoint{2.201634in}{2.637644in}}{\pgfqpoint{2.204906in}{2.645544in}}{\pgfqpoint{2.204906in}{2.653780in}}%
\pgfpathcurveto{\pgfqpoint{2.204906in}{2.662016in}}{\pgfqpoint{2.201634in}{2.669917in}}{\pgfqpoint{2.195810in}{2.675740in}}%
\pgfpathcurveto{\pgfqpoint{2.189986in}{2.681564in}}{\pgfqpoint{2.182086in}{2.684837in}}{\pgfqpoint{2.173850in}{2.684837in}}%
\pgfpathcurveto{\pgfqpoint{2.165613in}{2.684837in}}{\pgfqpoint{2.157713in}{2.681564in}}{\pgfqpoint{2.151889in}{2.675740in}}%
\pgfpathcurveto{\pgfqpoint{2.146065in}{2.669917in}}{\pgfqpoint{2.142793in}{2.662016in}}{\pgfqpoint{2.142793in}{2.653780in}}%
\pgfpathcurveto{\pgfqpoint{2.142793in}{2.645544in}}{\pgfqpoint{2.146065in}{2.637644in}}{\pgfqpoint{2.151889in}{2.631820in}}%
\pgfpathcurveto{\pgfqpoint{2.157713in}{2.625996in}}{\pgfqpoint{2.165613in}{2.622724in}}{\pgfqpoint{2.173850in}{2.622724in}}%
\pgfpathclose%
\pgfusepath{stroke,fill}%
\end{pgfscope}%
\begin{pgfscope}%
\pgfpathrectangle{\pgfqpoint{0.100000in}{0.212622in}}{\pgfqpoint{3.696000in}{3.696000in}}%
\pgfusepath{clip}%
\pgfsetbuttcap%
\pgfsetroundjoin%
\definecolor{currentfill}{rgb}{0.121569,0.466667,0.705882}%
\pgfsetfillcolor{currentfill}%
\pgfsetfillopacity{0.514954}%
\pgfsetlinewidth{1.003750pt}%
\definecolor{currentstroke}{rgb}{0.121569,0.466667,0.705882}%
\pgfsetstrokecolor{currentstroke}%
\pgfsetstrokeopacity{0.514954}%
\pgfsetdash{}{0pt}%
\pgfpathmoveto{\pgfqpoint{2.179733in}{2.608230in}}%
\pgfpathcurveto{\pgfqpoint{2.187970in}{2.608230in}}{\pgfqpoint{2.195870in}{2.611502in}}{\pgfqpoint{2.201694in}{2.617326in}}%
\pgfpathcurveto{\pgfqpoint{2.207518in}{2.623150in}}{\pgfqpoint{2.210790in}{2.631050in}}{\pgfqpoint{2.210790in}{2.639287in}}%
\pgfpathcurveto{\pgfqpoint{2.210790in}{2.647523in}}{\pgfqpoint{2.207518in}{2.655423in}}{\pgfqpoint{2.201694in}{2.661247in}}%
\pgfpathcurveto{\pgfqpoint{2.195870in}{2.667071in}}{\pgfqpoint{2.187970in}{2.670343in}}{\pgfqpoint{2.179733in}{2.670343in}}%
\pgfpathcurveto{\pgfqpoint{2.171497in}{2.670343in}}{\pgfqpoint{2.163597in}{2.667071in}}{\pgfqpoint{2.157773in}{2.661247in}}%
\pgfpathcurveto{\pgfqpoint{2.151949in}{2.655423in}}{\pgfqpoint{2.148677in}{2.647523in}}{\pgfqpoint{2.148677in}{2.639287in}}%
\pgfpathcurveto{\pgfqpoint{2.148677in}{2.631050in}}{\pgfqpoint{2.151949in}{2.623150in}}{\pgfqpoint{2.157773in}{2.617326in}}%
\pgfpathcurveto{\pgfqpoint{2.163597in}{2.611502in}}{\pgfqpoint{2.171497in}{2.608230in}}{\pgfqpoint{2.179733in}{2.608230in}}%
\pgfpathclose%
\pgfusepath{stroke,fill}%
\end{pgfscope}%
\begin{pgfscope}%
\pgfpathrectangle{\pgfqpoint{0.100000in}{0.212622in}}{\pgfqpoint{3.696000in}{3.696000in}}%
\pgfusepath{clip}%
\pgfsetbuttcap%
\pgfsetroundjoin%
\definecolor{currentfill}{rgb}{0.121569,0.466667,0.705882}%
\pgfsetfillcolor{currentfill}%
\pgfsetfillopacity{0.520039}%
\pgfsetlinewidth{1.003750pt}%
\definecolor{currentstroke}{rgb}{0.121569,0.466667,0.705882}%
\pgfsetstrokecolor{currentstroke}%
\pgfsetstrokeopacity{0.520039}%
\pgfsetdash{}{0pt}%
\pgfpathmoveto{\pgfqpoint{2.188420in}{2.589150in}}%
\pgfpathcurveto{\pgfqpoint{2.196656in}{2.589150in}}{\pgfqpoint{2.204556in}{2.592422in}}{\pgfqpoint{2.210380in}{2.598246in}}%
\pgfpathcurveto{\pgfqpoint{2.216204in}{2.604070in}}{\pgfqpoint{2.219477in}{2.611970in}}{\pgfqpoint{2.219477in}{2.620206in}}%
\pgfpathcurveto{\pgfqpoint{2.219477in}{2.628442in}}{\pgfqpoint{2.216204in}{2.636342in}}{\pgfqpoint{2.210380in}{2.642166in}}%
\pgfpathcurveto{\pgfqpoint{2.204556in}{2.647990in}}{\pgfqpoint{2.196656in}{2.651263in}}{\pgfqpoint{2.188420in}{2.651263in}}%
\pgfpathcurveto{\pgfqpoint{2.180184in}{2.651263in}}{\pgfqpoint{2.172284in}{2.647990in}}{\pgfqpoint{2.166460in}{2.642166in}}%
\pgfpathcurveto{\pgfqpoint{2.160636in}{2.636342in}}{\pgfqpoint{2.157364in}{2.628442in}}{\pgfqpoint{2.157364in}{2.620206in}}%
\pgfpathcurveto{\pgfqpoint{2.157364in}{2.611970in}}{\pgfqpoint{2.160636in}{2.604070in}}{\pgfqpoint{2.166460in}{2.598246in}}%
\pgfpathcurveto{\pgfqpoint{2.172284in}{2.592422in}}{\pgfqpoint{2.180184in}{2.589150in}}{\pgfqpoint{2.188420in}{2.589150in}}%
\pgfpathclose%
\pgfusepath{stroke,fill}%
\end{pgfscope}%
\begin{pgfscope}%
\pgfpathrectangle{\pgfqpoint{0.100000in}{0.212622in}}{\pgfqpoint{3.696000in}{3.696000in}}%
\pgfusepath{clip}%
\pgfsetbuttcap%
\pgfsetroundjoin%
\definecolor{currentfill}{rgb}{0.121569,0.466667,0.705882}%
\pgfsetfillcolor{currentfill}%
\pgfsetfillopacity{0.523036}%
\pgfsetlinewidth{1.003750pt}%
\definecolor{currentstroke}{rgb}{0.121569,0.466667,0.705882}%
\pgfsetstrokecolor{currentstroke}%
\pgfsetstrokeopacity{0.523036}%
\pgfsetdash{}{0pt}%
\pgfpathmoveto{\pgfqpoint{2.192402in}{2.578912in}}%
\pgfpathcurveto{\pgfqpoint{2.200639in}{2.578912in}}{\pgfqpoint{2.208539in}{2.582185in}}{\pgfqpoint{2.214362in}{2.588009in}}%
\pgfpathcurveto{\pgfqpoint{2.220186in}{2.593833in}}{\pgfqpoint{2.223459in}{2.601733in}}{\pgfqpoint{2.223459in}{2.609969in}}%
\pgfpathcurveto{\pgfqpoint{2.223459in}{2.618205in}}{\pgfqpoint{2.220186in}{2.626105in}}{\pgfqpoint{2.214362in}{2.631929in}}%
\pgfpathcurveto{\pgfqpoint{2.208539in}{2.637753in}}{\pgfqpoint{2.200639in}{2.641025in}}{\pgfqpoint{2.192402in}{2.641025in}}%
\pgfpathcurveto{\pgfqpoint{2.184166in}{2.641025in}}{\pgfqpoint{2.176266in}{2.637753in}}{\pgfqpoint{2.170442in}{2.631929in}}%
\pgfpathcurveto{\pgfqpoint{2.164618in}{2.626105in}}{\pgfqpoint{2.161346in}{2.618205in}}{\pgfqpoint{2.161346in}{2.609969in}}%
\pgfpathcurveto{\pgfqpoint{2.161346in}{2.601733in}}{\pgfqpoint{2.164618in}{2.593833in}}{\pgfqpoint{2.170442in}{2.588009in}}%
\pgfpathcurveto{\pgfqpoint{2.176266in}{2.582185in}}{\pgfqpoint{2.184166in}{2.578912in}}{\pgfqpoint{2.192402in}{2.578912in}}%
\pgfpathclose%
\pgfusepath{stroke,fill}%
\end{pgfscope}%
\begin{pgfscope}%
\pgfpathrectangle{\pgfqpoint{0.100000in}{0.212622in}}{\pgfqpoint{3.696000in}{3.696000in}}%
\pgfusepath{clip}%
\pgfsetbuttcap%
\pgfsetroundjoin%
\definecolor{currentfill}{rgb}{0.121569,0.466667,0.705882}%
\pgfsetfillcolor{currentfill}%
\pgfsetfillopacity{0.525217}%
\pgfsetlinewidth{1.003750pt}%
\definecolor{currentstroke}{rgb}{0.121569,0.466667,0.705882}%
\pgfsetstrokecolor{currentstroke}%
\pgfsetstrokeopacity{0.525217}%
\pgfsetdash{}{0pt}%
\pgfpathmoveto{\pgfqpoint{1.331886in}{2.823259in}}%
\pgfpathcurveto{\pgfqpoint{1.340122in}{2.823259in}}{\pgfqpoint{1.348022in}{2.826531in}}{\pgfqpoint{1.353846in}{2.832355in}}%
\pgfpathcurveto{\pgfqpoint{1.359670in}{2.838179in}}{\pgfqpoint{1.362942in}{2.846079in}}{\pgfqpoint{1.362942in}{2.854315in}}%
\pgfpathcurveto{\pgfqpoint{1.362942in}{2.862552in}}{\pgfqpoint{1.359670in}{2.870452in}}{\pgfqpoint{1.353846in}{2.876276in}}%
\pgfpathcurveto{\pgfqpoint{1.348022in}{2.882099in}}{\pgfqpoint{1.340122in}{2.885372in}}{\pgfqpoint{1.331886in}{2.885372in}}%
\pgfpathcurveto{\pgfqpoint{1.323649in}{2.885372in}}{\pgfqpoint{1.315749in}{2.882099in}}{\pgfqpoint{1.309925in}{2.876276in}}%
\pgfpathcurveto{\pgfqpoint{1.304101in}{2.870452in}}{\pgfqpoint{1.300829in}{2.862552in}}{\pgfqpoint{1.300829in}{2.854315in}}%
\pgfpathcurveto{\pgfqpoint{1.300829in}{2.846079in}}{\pgfqpoint{1.304101in}{2.838179in}}{\pgfqpoint{1.309925in}{2.832355in}}%
\pgfpathcurveto{\pgfqpoint{1.315749in}{2.826531in}}{\pgfqpoint{1.323649in}{2.823259in}}{\pgfqpoint{1.331886in}{2.823259in}}%
\pgfpathclose%
\pgfusepath{stroke,fill}%
\end{pgfscope}%
\begin{pgfscope}%
\pgfpathrectangle{\pgfqpoint{0.100000in}{0.212622in}}{\pgfqpoint{3.696000in}{3.696000in}}%
\pgfusepath{clip}%
\pgfsetbuttcap%
\pgfsetroundjoin%
\definecolor{currentfill}{rgb}{0.121569,0.466667,0.705882}%
\pgfsetfillcolor{currentfill}%
\pgfsetfillopacity{0.527088}%
\pgfsetlinewidth{1.003750pt}%
\definecolor{currentstroke}{rgb}{0.121569,0.466667,0.705882}%
\pgfsetstrokecolor{currentstroke}%
\pgfsetstrokeopacity{0.527088}%
\pgfsetdash{}{0pt}%
\pgfpathmoveto{\pgfqpoint{2.198865in}{2.563987in}}%
\pgfpathcurveto{\pgfqpoint{2.207102in}{2.563987in}}{\pgfqpoint{2.215002in}{2.567260in}}{\pgfqpoint{2.220826in}{2.573084in}}%
\pgfpathcurveto{\pgfqpoint{2.226650in}{2.578908in}}{\pgfqpoint{2.229922in}{2.586808in}}{\pgfqpoint{2.229922in}{2.595044in}}%
\pgfpathcurveto{\pgfqpoint{2.229922in}{2.603280in}}{\pgfqpoint{2.226650in}{2.611180in}}{\pgfqpoint{2.220826in}{2.617004in}}%
\pgfpathcurveto{\pgfqpoint{2.215002in}{2.622828in}}{\pgfqpoint{2.207102in}{2.626100in}}{\pgfqpoint{2.198865in}{2.626100in}}%
\pgfpathcurveto{\pgfqpoint{2.190629in}{2.626100in}}{\pgfqpoint{2.182729in}{2.622828in}}{\pgfqpoint{2.176905in}{2.617004in}}%
\pgfpathcurveto{\pgfqpoint{2.171081in}{2.611180in}}{\pgfqpoint{2.167809in}{2.603280in}}{\pgfqpoint{2.167809in}{2.595044in}}%
\pgfpathcurveto{\pgfqpoint{2.167809in}{2.586808in}}{\pgfqpoint{2.171081in}{2.578908in}}{\pgfqpoint{2.176905in}{2.573084in}}%
\pgfpathcurveto{\pgfqpoint{2.182729in}{2.567260in}}{\pgfqpoint{2.190629in}{2.563987in}}{\pgfqpoint{2.198865in}{2.563987in}}%
\pgfpathclose%
\pgfusepath{stroke,fill}%
\end{pgfscope}%
\begin{pgfscope}%
\pgfpathrectangle{\pgfqpoint{0.100000in}{0.212622in}}{\pgfqpoint{3.696000in}{3.696000in}}%
\pgfusepath{clip}%
\pgfsetbuttcap%
\pgfsetroundjoin%
\definecolor{currentfill}{rgb}{0.121569,0.466667,0.705882}%
\pgfsetfillcolor{currentfill}%
\pgfsetfillopacity{0.529408}%
\pgfsetlinewidth{1.003750pt}%
\definecolor{currentstroke}{rgb}{0.121569,0.466667,0.705882}%
\pgfsetstrokecolor{currentstroke}%
\pgfsetstrokeopacity{0.529408}%
\pgfsetdash{}{0pt}%
\pgfpathmoveto{\pgfqpoint{2.201895in}{2.555807in}}%
\pgfpathcurveto{\pgfqpoint{2.210132in}{2.555807in}}{\pgfqpoint{2.218032in}{2.559079in}}{\pgfqpoint{2.223856in}{2.564903in}}%
\pgfpathcurveto{\pgfqpoint{2.229680in}{2.570727in}}{\pgfqpoint{2.232952in}{2.578627in}}{\pgfqpoint{2.232952in}{2.586863in}}%
\pgfpathcurveto{\pgfqpoint{2.232952in}{2.595100in}}{\pgfqpoint{2.229680in}{2.603000in}}{\pgfqpoint{2.223856in}{2.608824in}}%
\pgfpathcurveto{\pgfqpoint{2.218032in}{2.614647in}}{\pgfqpoint{2.210132in}{2.617920in}}{\pgfqpoint{2.201895in}{2.617920in}}%
\pgfpathcurveto{\pgfqpoint{2.193659in}{2.617920in}}{\pgfqpoint{2.185759in}{2.614647in}}{\pgfqpoint{2.179935in}{2.608824in}}%
\pgfpathcurveto{\pgfqpoint{2.174111in}{2.603000in}}{\pgfqpoint{2.170839in}{2.595100in}}{\pgfqpoint{2.170839in}{2.586863in}}%
\pgfpathcurveto{\pgfqpoint{2.170839in}{2.578627in}}{\pgfqpoint{2.174111in}{2.570727in}}{\pgfqpoint{2.179935in}{2.564903in}}%
\pgfpathcurveto{\pgfqpoint{2.185759in}{2.559079in}}{\pgfqpoint{2.193659in}{2.555807in}}{\pgfqpoint{2.201895in}{2.555807in}}%
\pgfpathclose%
\pgfusepath{stroke,fill}%
\end{pgfscope}%
\begin{pgfscope}%
\pgfpathrectangle{\pgfqpoint{0.100000in}{0.212622in}}{\pgfqpoint{3.696000in}{3.696000in}}%
\pgfusepath{clip}%
\pgfsetbuttcap%
\pgfsetroundjoin%
\definecolor{currentfill}{rgb}{0.121569,0.466667,0.705882}%
\pgfsetfillcolor{currentfill}%
\pgfsetfillopacity{0.533398}%
\pgfsetlinewidth{1.003750pt}%
\definecolor{currentstroke}{rgb}{0.121569,0.466667,0.705882}%
\pgfsetstrokecolor{currentstroke}%
\pgfsetstrokeopacity{0.533398}%
\pgfsetdash{}{0pt}%
\pgfpathmoveto{\pgfqpoint{2.208324in}{2.540772in}}%
\pgfpathcurveto{\pgfqpoint{2.216560in}{2.540772in}}{\pgfqpoint{2.224460in}{2.544044in}}{\pgfqpoint{2.230284in}{2.549868in}}%
\pgfpathcurveto{\pgfqpoint{2.236108in}{2.555692in}}{\pgfqpoint{2.239380in}{2.563592in}}{\pgfqpoint{2.239380in}{2.571828in}}%
\pgfpathcurveto{\pgfqpoint{2.239380in}{2.580065in}}{\pgfqpoint{2.236108in}{2.587965in}}{\pgfqpoint{2.230284in}{2.593789in}}%
\pgfpathcurveto{\pgfqpoint{2.224460in}{2.599613in}}{\pgfqpoint{2.216560in}{2.602885in}}{\pgfqpoint{2.208324in}{2.602885in}}%
\pgfpathcurveto{\pgfqpoint{2.200087in}{2.602885in}}{\pgfqpoint{2.192187in}{2.599613in}}{\pgfqpoint{2.186363in}{2.593789in}}%
\pgfpathcurveto{\pgfqpoint{2.180540in}{2.587965in}}{\pgfqpoint{2.177267in}{2.580065in}}{\pgfqpoint{2.177267in}{2.571828in}}%
\pgfpathcurveto{\pgfqpoint{2.177267in}{2.563592in}}{\pgfqpoint{2.180540in}{2.555692in}}{\pgfqpoint{2.186363in}{2.549868in}}%
\pgfpathcurveto{\pgfqpoint{2.192187in}{2.544044in}}{\pgfqpoint{2.200087in}{2.540772in}}{\pgfqpoint{2.208324in}{2.540772in}}%
\pgfpathclose%
\pgfusepath{stroke,fill}%
\end{pgfscope}%
\begin{pgfscope}%
\pgfpathrectangle{\pgfqpoint{0.100000in}{0.212622in}}{\pgfqpoint{3.696000in}{3.696000in}}%
\pgfusepath{clip}%
\pgfsetbuttcap%
\pgfsetroundjoin%
\definecolor{currentfill}{rgb}{0.121569,0.466667,0.705882}%
\pgfsetfillcolor{currentfill}%
\pgfsetfillopacity{0.538158}%
\pgfsetlinewidth{1.003750pt}%
\definecolor{currentstroke}{rgb}{0.121569,0.466667,0.705882}%
\pgfsetstrokecolor{currentstroke}%
\pgfsetstrokeopacity{0.538158}%
\pgfsetdash{}{0pt}%
\pgfpathmoveto{\pgfqpoint{2.214367in}{2.524428in}}%
\pgfpathcurveto{\pgfqpoint{2.222603in}{2.524428in}}{\pgfqpoint{2.230503in}{2.527700in}}{\pgfqpoint{2.236327in}{2.533524in}}%
\pgfpathcurveto{\pgfqpoint{2.242151in}{2.539348in}}{\pgfqpoint{2.245423in}{2.547248in}}{\pgfqpoint{2.245423in}{2.555484in}}%
\pgfpathcurveto{\pgfqpoint{2.245423in}{2.563721in}}{\pgfqpoint{2.242151in}{2.571621in}}{\pgfqpoint{2.236327in}{2.577445in}}%
\pgfpathcurveto{\pgfqpoint{2.230503in}{2.583268in}}{\pgfqpoint{2.222603in}{2.586541in}}{\pgfqpoint{2.214367in}{2.586541in}}%
\pgfpathcurveto{\pgfqpoint{2.206131in}{2.586541in}}{\pgfqpoint{2.198230in}{2.583268in}}{\pgfqpoint{2.192407in}{2.577445in}}%
\pgfpathcurveto{\pgfqpoint{2.186583in}{2.571621in}}{\pgfqpoint{2.183310in}{2.563721in}}{\pgfqpoint{2.183310in}{2.555484in}}%
\pgfpathcurveto{\pgfqpoint{2.183310in}{2.547248in}}{\pgfqpoint{2.186583in}{2.539348in}}{\pgfqpoint{2.192407in}{2.533524in}}%
\pgfpathcurveto{\pgfqpoint{2.198230in}{2.527700in}}{\pgfqpoint{2.206131in}{2.524428in}}{\pgfqpoint{2.214367in}{2.524428in}}%
\pgfpathclose%
\pgfusepath{stroke,fill}%
\end{pgfscope}%
\begin{pgfscope}%
\pgfpathrectangle{\pgfqpoint{0.100000in}{0.212622in}}{\pgfqpoint{3.696000in}{3.696000in}}%
\pgfusepath{clip}%
\pgfsetbuttcap%
\pgfsetroundjoin%
\definecolor{currentfill}{rgb}{0.121569,0.466667,0.705882}%
\pgfsetfillcolor{currentfill}%
\pgfsetfillopacity{0.541933}%
\pgfsetlinewidth{1.003750pt}%
\definecolor{currentstroke}{rgb}{0.121569,0.466667,0.705882}%
\pgfsetstrokecolor{currentstroke}%
\pgfsetstrokeopacity{0.541933}%
\pgfsetdash{}{0pt}%
\pgfpathmoveto{\pgfqpoint{1.316392in}{2.778040in}}%
\pgfpathcurveto{\pgfqpoint{1.324628in}{2.778040in}}{\pgfqpoint{1.332528in}{2.781312in}}{\pgfqpoint{1.338352in}{2.787136in}}%
\pgfpathcurveto{\pgfqpoint{1.344176in}{2.792960in}}{\pgfqpoint{1.347448in}{2.800860in}}{\pgfqpoint{1.347448in}{2.809096in}}%
\pgfpathcurveto{\pgfqpoint{1.347448in}{2.817332in}}{\pgfqpoint{1.344176in}{2.825233in}}{\pgfqpoint{1.338352in}{2.831056in}}%
\pgfpathcurveto{\pgfqpoint{1.332528in}{2.836880in}}{\pgfqpoint{1.324628in}{2.840153in}}{\pgfqpoint{1.316392in}{2.840153in}}%
\pgfpathcurveto{\pgfqpoint{1.308156in}{2.840153in}}{\pgfqpoint{1.300256in}{2.836880in}}{\pgfqpoint{1.294432in}{2.831056in}}%
\pgfpathcurveto{\pgfqpoint{1.288608in}{2.825233in}}{\pgfqpoint{1.285335in}{2.817332in}}{\pgfqpoint{1.285335in}{2.809096in}}%
\pgfpathcurveto{\pgfqpoint{1.285335in}{2.800860in}}{\pgfqpoint{1.288608in}{2.792960in}}{\pgfqpoint{1.294432in}{2.787136in}}%
\pgfpathcurveto{\pgfqpoint{1.300256in}{2.781312in}}{\pgfqpoint{1.308156in}{2.778040in}}{\pgfqpoint{1.316392in}{2.778040in}}%
\pgfpathclose%
\pgfusepath{stroke,fill}%
\end{pgfscope}%
\begin{pgfscope}%
\pgfpathrectangle{\pgfqpoint{0.100000in}{0.212622in}}{\pgfqpoint{3.696000in}{3.696000in}}%
\pgfusepath{clip}%
\pgfsetbuttcap%
\pgfsetroundjoin%
\definecolor{currentfill}{rgb}{0.121569,0.466667,0.705882}%
\pgfsetfillcolor{currentfill}%
\pgfsetfillopacity{0.543555}%
\pgfsetlinewidth{1.003750pt}%
\definecolor{currentstroke}{rgb}{0.121569,0.466667,0.705882}%
\pgfsetstrokecolor{currentstroke}%
\pgfsetstrokeopacity{0.543555}%
\pgfsetdash{}{0pt}%
\pgfpathmoveto{\pgfqpoint{2.222429in}{2.504234in}}%
\pgfpathcurveto{\pgfqpoint{2.230665in}{2.504234in}}{\pgfqpoint{2.238566in}{2.507506in}}{\pgfqpoint{2.244389in}{2.513330in}}%
\pgfpathcurveto{\pgfqpoint{2.250213in}{2.519154in}}{\pgfqpoint{2.253486in}{2.527054in}}{\pgfqpoint{2.253486in}{2.535290in}}%
\pgfpathcurveto{\pgfqpoint{2.253486in}{2.543527in}}{\pgfqpoint{2.250213in}{2.551427in}}{\pgfqpoint{2.244389in}{2.557251in}}%
\pgfpathcurveto{\pgfqpoint{2.238566in}{2.563075in}}{\pgfqpoint{2.230665in}{2.566347in}}{\pgfqpoint{2.222429in}{2.566347in}}%
\pgfpathcurveto{\pgfqpoint{2.214193in}{2.566347in}}{\pgfqpoint{2.206293in}{2.563075in}}{\pgfqpoint{2.200469in}{2.557251in}}%
\pgfpathcurveto{\pgfqpoint{2.194645in}{2.551427in}}{\pgfqpoint{2.191373in}{2.543527in}}{\pgfqpoint{2.191373in}{2.535290in}}%
\pgfpathcurveto{\pgfqpoint{2.191373in}{2.527054in}}{\pgfqpoint{2.194645in}{2.519154in}}{\pgfqpoint{2.200469in}{2.513330in}}%
\pgfpathcurveto{\pgfqpoint{2.206293in}{2.507506in}}{\pgfqpoint{2.214193in}{2.504234in}}{\pgfqpoint{2.222429in}{2.504234in}}%
\pgfpathclose%
\pgfusepath{stroke,fill}%
\end{pgfscope}%
\begin{pgfscope}%
\pgfpathrectangle{\pgfqpoint{0.100000in}{0.212622in}}{\pgfqpoint{3.696000in}{3.696000in}}%
\pgfusepath{clip}%
\pgfsetbuttcap%
\pgfsetroundjoin%
\definecolor{currentfill}{rgb}{0.121569,0.466667,0.705882}%
\pgfsetfillcolor{currentfill}%
\pgfsetfillopacity{0.550021}%
\pgfsetlinewidth{1.003750pt}%
\definecolor{currentstroke}{rgb}{0.121569,0.466667,0.705882}%
\pgfsetstrokecolor{currentstroke}%
\pgfsetstrokeopacity{0.550021}%
\pgfsetdash{}{0pt}%
\pgfpathmoveto{\pgfqpoint{2.229722in}{2.482266in}}%
\pgfpathcurveto{\pgfqpoint{2.237959in}{2.482266in}}{\pgfqpoint{2.245859in}{2.485538in}}{\pgfqpoint{2.251683in}{2.491362in}}%
\pgfpathcurveto{\pgfqpoint{2.257507in}{2.497186in}}{\pgfqpoint{2.260779in}{2.505086in}}{\pgfqpoint{2.260779in}{2.513323in}}%
\pgfpathcurveto{\pgfqpoint{2.260779in}{2.521559in}}{\pgfqpoint{2.257507in}{2.529459in}}{\pgfqpoint{2.251683in}{2.535283in}}%
\pgfpathcurveto{\pgfqpoint{2.245859in}{2.541107in}}{\pgfqpoint{2.237959in}{2.544379in}}{\pgfqpoint{2.229722in}{2.544379in}}%
\pgfpathcurveto{\pgfqpoint{2.221486in}{2.544379in}}{\pgfqpoint{2.213586in}{2.541107in}}{\pgfqpoint{2.207762in}{2.535283in}}%
\pgfpathcurveto{\pgfqpoint{2.201938in}{2.529459in}}{\pgfqpoint{2.198666in}{2.521559in}}{\pgfqpoint{2.198666in}{2.513323in}}%
\pgfpathcurveto{\pgfqpoint{2.198666in}{2.505086in}}{\pgfqpoint{2.201938in}{2.497186in}}{\pgfqpoint{2.207762in}{2.491362in}}%
\pgfpathcurveto{\pgfqpoint{2.213586in}{2.485538in}}{\pgfqpoint{2.221486in}{2.482266in}}{\pgfqpoint{2.229722in}{2.482266in}}%
\pgfpathclose%
\pgfusepath{stroke,fill}%
\end{pgfscope}%
\begin{pgfscope}%
\pgfpathrectangle{\pgfqpoint{0.100000in}{0.212622in}}{\pgfqpoint{3.696000in}{3.696000in}}%
\pgfusepath{clip}%
\pgfsetbuttcap%
\pgfsetroundjoin%
\definecolor{currentfill}{rgb}{0.121569,0.466667,0.705882}%
\pgfsetfillcolor{currentfill}%
\pgfsetfillopacity{0.557034}%
\pgfsetlinewidth{1.003750pt}%
\definecolor{currentstroke}{rgb}{0.121569,0.466667,0.705882}%
\pgfsetstrokecolor{currentstroke}%
\pgfsetstrokeopacity{0.557034}%
\pgfsetdash{}{0pt}%
\pgfpathmoveto{\pgfqpoint{1.295706in}{2.735967in}}%
\pgfpathcurveto{\pgfqpoint{1.303942in}{2.735967in}}{\pgfqpoint{1.311842in}{2.739239in}}{\pgfqpoint{1.317666in}{2.745063in}}%
\pgfpathcurveto{\pgfqpoint{1.323490in}{2.750887in}}{\pgfqpoint{1.326762in}{2.758787in}}{\pgfqpoint{1.326762in}{2.767023in}}%
\pgfpathcurveto{\pgfqpoint{1.326762in}{2.775260in}}{\pgfqpoint{1.323490in}{2.783160in}}{\pgfqpoint{1.317666in}{2.788984in}}%
\pgfpathcurveto{\pgfqpoint{1.311842in}{2.794808in}}{\pgfqpoint{1.303942in}{2.798080in}}{\pgfqpoint{1.295706in}{2.798080in}}%
\pgfpathcurveto{\pgfqpoint{1.287469in}{2.798080in}}{\pgfqpoint{1.279569in}{2.794808in}}{\pgfqpoint{1.273745in}{2.788984in}}%
\pgfpathcurveto{\pgfqpoint{1.267921in}{2.783160in}}{\pgfqpoint{1.264649in}{2.775260in}}{\pgfqpoint{1.264649in}{2.767023in}}%
\pgfpathcurveto{\pgfqpoint{1.264649in}{2.758787in}}{\pgfqpoint{1.267921in}{2.750887in}}{\pgfqpoint{1.273745in}{2.745063in}}%
\pgfpathcurveto{\pgfqpoint{1.279569in}{2.739239in}}{\pgfqpoint{1.287469in}{2.735967in}}{\pgfqpoint{1.295706in}{2.735967in}}%
\pgfpathclose%
\pgfusepath{stroke,fill}%
\end{pgfscope}%
\begin{pgfscope}%
\pgfpathrectangle{\pgfqpoint{0.100000in}{0.212622in}}{\pgfqpoint{3.696000in}{3.696000in}}%
\pgfusepath{clip}%
\pgfsetbuttcap%
\pgfsetroundjoin%
\definecolor{currentfill}{rgb}{0.121569,0.466667,0.705882}%
\pgfsetfillcolor{currentfill}%
\pgfsetfillopacity{0.557554}%
\pgfsetlinewidth{1.003750pt}%
\definecolor{currentstroke}{rgb}{0.121569,0.466667,0.705882}%
\pgfsetstrokecolor{currentstroke}%
\pgfsetstrokeopacity{0.557554}%
\pgfsetdash{}{0pt}%
\pgfpathmoveto{\pgfqpoint{2.240543in}{2.454576in}}%
\pgfpathcurveto{\pgfqpoint{2.248779in}{2.454576in}}{\pgfqpoint{2.256679in}{2.457848in}}{\pgfqpoint{2.262503in}{2.463672in}}%
\pgfpathcurveto{\pgfqpoint{2.268327in}{2.469496in}}{\pgfqpoint{2.271599in}{2.477396in}}{\pgfqpoint{2.271599in}{2.485632in}}%
\pgfpathcurveto{\pgfqpoint{2.271599in}{2.493868in}}{\pgfqpoint{2.268327in}{2.501768in}}{\pgfqpoint{2.262503in}{2.507592in}}%
\pgfpathcurveto{\pgfqpoint{2.256679in}{2.513416in}}{\pgfqpoint{2.248779in}{2.516689in}}{\pgfqpoint{2.240543in}{2.516689in}}%
\pgfpathcurveto{\pgfqpoint{2.232307in}{2.516689in}}{\pgfqpoint{2.224407in}{2.513416in}}{\pgfqpoint{2.218583in}{2.507592in}}%
\pgfpathcurveto{\pgfqpoint{2.212759in}{2.501768in}}{\pgfqpoint{2.209486in}{2.493868in}}{\pgfqpoint{2.209486in}{2.485632in}}%
\pgfpathcurveto{\pgfqpoint{2.209486in}{2.477396in}}{\pgfqpoint{2.212759in}{2.469496in}}{\pgfqpoint{2.218583in}{2.463672in}}%
\pgfpathcurveto{\pgfqpoint{2.224407in}{2.457848in}}{\pgfqpoint{2.232307in}{2.454576in}}{\pgfqpoint{2.240543in}{2.454576in}}%
\pgfpathclose%
\pgfusepath{stroke,fill}%
\end{pgfscope}%
\begin{pgfscope}%
\pgfpathrectangle{\pgfqpoint{0.100000in}{0.212622in}}{\pgfqpoint{3.696000in}{3.696000in}}%
\pgfusepath{clip}%
\pgfsetbuttcap%
\pgfsetroundjoin%
\definecolor{currentfill}{rgb}{0.121569,0.466667,0.705882}%
\pgfsetfillcolor{currentfill}%
\pgfsetfillopacity{0.566455}%
\pgfsetlinewidth{1.003750pt}%
\definecolor{currentstroke}{rgb}{0.121569,0.466667,0.705882}%
\pgfsetstrokecolor{currentstroke}%
\pgfsetstrokeopacity{0.566455}%
\pgfsetdash{}{0pt}%
\pgfpathmoveto{\pgfqpoint{2.250787in}{2.424624in}}%
\pgfpathcurveto{\pgfqpoint{2.259024in}{2.424624in}}{\pgfqpoint{2.266924in}{2.427897in}}{\pgfqpoint{2.272747in}{2.433721in}}%
\pgfpathcurveto{\pgfqpoint{2.278571in}{2.439545in}}{\pgfqpoint{2.281844in}{2.447445in}}{\pgfqpoint{2.281844in}{2.455681in}}%
\pgfpathcurveto{\pgfqpoint{2.281844in}{2.463917in}}{\pgfqpoint{2.278571in}{2.471817in}}{\pgfqpoint{2.272747in}{2.477641in}}%
\pgfpathcurveto{\pgfqpoint{2.266924in}{2.483465in}}{\pgfqpoint{2.259024in}{2.486737in}}{\pgfqpoint{2.250787in}{2.486737in}}%
\pgfpathcurveto{\pgfqpoint{2.242551in}{2.486737in}}{\pgfqpoint{2.234651in}{2.483465in}}{\pgfqpoint{2.228827in}{2.477641in}}%
\pgfpathcurveto{\pgfqpoint{2.223003in}{2.471817in}}{\pgfqpoint{2.219731in}{2.463917in}}{\pgfqpoint{2.219731in}{2.455681in}}%
\pgfpathcurveto{\pgfqpoint{2.219731in}{2.447445in}}{\pgfqpoint{2.223003in}{2.439545in}}{\pgfqpoint{2.228827in}{2.433721in}}%
\pgfpathcurveto{\pgfqpoint{2.234651in}{2.427897in}}{\pgfqpoint{2.242551in}{2.424624in}}{\pgfqpoint{2.250787in}{2.424624in}}%
\pgfpathclose%
\pgfusepath{stroke,fill}%
\end{pgfscope}%
\begin{pgfscope}%
\pgfpathrectangle{\pgfqpoint{0.100000in}{0.212622in}}{\pgfqpoint{3.696000in}{3.696000in}}%
\pgfusepath{clip}%
\pgfsetbuttcap%
\pgfsetroundjoin%
\definecolor{currentfill}{rgb}{0.121569,0.466667,0.705882}%
\pgfsetfillcolor{currentfill}%
\pgfsetfillopacity{0.570923}%
\pgfsetlinewidth{1.003750pt}%
\definecolor{currentstroke}{rgb}{0.121569,0.466667,0.705882}%
\pgfsetstrokecolor{currentstroke}%
\pgfsetstrokeopacity{0.570923}%
\pgfsetdash{}{0pt}%
\pgfpathmoveto{\pgfqpoint{1.284504in}{2.697664in}}%
\pgfpathcurveto{\pgfqpoint{1.292740in}{2.697664in}}{\pgfqpoint{1.300640in}{2.700936in}}{\pgfqpoint{1.306464in}{2.706760in}}%
\pgfpathcurveto{\pgfqpoint{1.312288in}{2.712584in}}{\pgfqpoint{1.315560in}{2.720484in}}{\pgfqpoint{1.315560in}{2.728721in}}%
\pgfpathcurveto{\pgfqpoint{1.315560in}{2.736957in}}{\pgfqpoint{1.312288in}{2.744857in}}{\pgfqpoint{1.306464in}{2.750681in}}%
\pgfpathcurveto{\pgfqpoint{1.300640in}{2.756505in}}{\pgfqpoint{1.292740in}{2.759777in}}{\pgfqpoint{1.284504in}{2.759777in}}%
\pgfpathcurveto{\pgfqpoint{1.276267in}{2.759777in}}{\pgfqpoint{1.268367in}{2.756505in}}{\pgfqpoint{1.262544in}{2.750681in}}%
\pgfpathcurveto{\pgfqpoint{1.256720in}{2.744857in}}{\pgfqpoint{1.253447in}{2.736957in}}{\pgfqpoint{1.253447in}{2.728721in}}%
\pgfpathcurveto{\pgfqpoint{1.253447in}{2.720484in}}{\pgfqpoint{1.256720in}{2.712584in}}{\pgfqpoint{1.262544in}{2.706760in}}%
\pgfpathcurveto{\pgfqpoint{1.268367in}{2.700936in}}{\pgfqpoint{1.276267in}{2.697664in}}{\pgfqpoint{1.284504in}{2.697664in}}%
\pgfpathclose%
\pgfusepath{stroke,fill}%
\end{pgfscope}%
\begin{pgfscope}%
\pgfpathrectangle{\pgfqpoint{0.100000in}{0.212622in}}{\pgfqpoint{3.696000in}{3.696000in}}%
\pgfusepath{clip}%
\pgfsetbuttcap%
\pgfsetroundjoin%
\definecolor{currentfill}{rgb}{0.121569,0.466667,0.705882}%
\pgfsetfillcolor{currentfill}%
\pgfsetfillopacity{0.576664}%
\pgfsetlinewidth{1.003750pt}%
\definecolor{currentstroke}{rgb}{0.121569,0.466667,0.705882}%
\pgfsetstrokecolor{currentstroke}%
\pgfsetstrokeopacity{0.576664}%
\pgfsetdash{}{0pt}%
\pgfpathmoveto{\pgfqpoint{2.265438in}{2.387313in}}%
\pgfpathcurveto{\pgfqpoint{2.273675in}{2.387313in}}{\pgfqpoint{2.281575in}{2.390586in}}{\pgfqpoint{2.287399in}{2.396410in}}%
\pgfpathcurveto{\pgfqpoint{2.293223in}{2.402234in}}{\pgfqpoint{2.296495in}{2.410134in}}{\pgfqpoint{2.296495in}{2.418370in}}%
\pgfpathcurveto{\pgfqpoint{2.296495in}{2.426606in}}{\pgfqpoint{2.293223in}{2.434506in}}{\pgfqpoint{2.287399in}{2.440330in}}%
\pgfpathcurveto{\pgfqpoint{2.281575in}{2.446154in}}{\pgfqpoint{2.273675in}{2.449426in}}{\pgfqpoint{2.265438in}{2.449426in}}%
\pgfpathcurveto{\pgfqpoint{2.257202in}{2.449426in}}{\pgfqpoint{2.249302in}{2.446154in}}{\pgfqpoint{2.243478in}{2.440330in}}%
\pgfpathcurveto{\pgfqpoint{2.237654in}{2.434506in}}{\pgfqpoint{2.234382in}{2.426606in}}{\pgfqpoint{2.234382in}{2.418370in}}%
\pgfpathcurveto{\pgfqpoint{2.234382in}{2.410134in}}{\pgfqpoint{2.237654in}{2.402234in}}{\pgfqpoint{2.243478in}{2.396410in}}%
\pgfpathcurveto{\pgfqpoint{2.249302in}{2.390586in}}{\pgfqpoint{2.257202in}{2.387313in}}{\pgfqpoint{2.265438in}{2.387313in}}%
\pgfpathclose%
\pgfusepath{stroke,fill}%
\end{pgfscope}%
\begin{pgfscope}%
\pgfpathrectangle{\pgfqpoint{0.100000in}{0.212622in}}{\pgfqpoint{3.696000in}{3.696000in}}%
\pgfusepath{clip}%
\pgfsetbuttcap%
\pgfsetroundjoin%
\definecolor{currentfill}{rgb}{0.121569,0.466667,0.705882}%
\pgfsetfillcolor{currentfill}%
\pgfsetfillopacity{0.582576}%
\pgfsetlinewidth{1.003750pt}%
\definecolor{currentstroke}{rgb}{0.121569,0.466667,0.705882}%
\pgfsetstrokecolor{currentstroke}%
\pgfsetstrokeopacity{0.582576}%
\pgfsetdash{}{0pt}%
\pgfpathmoveto{\pgfqpoint{1.267560in}{2.664507in}}%
\pgfpathcurveto{\pgfqpoint{1.275797in}{2.664507in}}{\pgfqpoint{1.283697in}{2.667779in}}{\pgfqpoint{1.289521in}{2.673603in}}%
\pgfpathcurveto{\pgfqpoint{1.295345in}{2.679427in}}{\pgfqpoint{1.298617in}{2.687327in}}{\pgfqpoint{1.298617in}{2.695563in}}%
\pgfpathcurveto{\pgfqpoint{1.298617in}{2.703800in}}{\pgfqpoint{1.295345in}{2.711700in}}{\pgfqpoint{1.289521in}{2.717524in}}%
\pgfpathcurveto{\pgfqpoint{1.283697in}{2.723348in}}{\pgfqpoint{1.275797in}{2.726620in}}{\pgfqpoint{1.267560in}{2.726620in}}%
\pgfpathcurveto{\pgfqpoint{1.259324in}{2.726620in}}{\pgfqpoint{1.251424in}{2.723348in}}{\pgfqpoint{1.245600in}{2.717524in}}%
\pgfpathcurveto{\pgfqpoint{1.239776in}{2.711700in}}{\pgfqpoint{1.236504in}{2.703800in}}{\pgfqpoint{1.236504in}{2.695563in}}%
\pgfpathcurveto{\pgfqpoint{1.236504in}{2.687327in}}{\pgfqpoint{1.239776in}{2.679427in}}{\pgfqpoint{1.245600in}{2.673603in}}%
\pgfpathcurveto{\pgfqpoint{1.251424in}{2.667779in}}{\pgfqpoint{1.259324in}{2.664507in}}{\pgfqpoint{1.267560in}{2.664507in}}%
\pgfpathclose%
\pgfusepath{stroke,fill}%
\end{pgfscope}%
\begin{pgfscope}%
\pgfpathrectangle{\pgfqpoint{0.100000in}{0.212622in}}{\pgfqpoint{3.696000in}{3.696000in}}%
\pgfusepath{clip}%
\pgfsetbuttcap%
\pgfsetroundjoin%
\definecolor{currentfill}{rgb}{0.121569,0.466667,0.705882}%
\pgfsetfillcolor{currentfill}%
\pgfsetfillopacity{0.588518}%
\pgfsetlinewidth{1.003750pt}%
\definecolor{currentstroke}{rgb}{0.121569,0.466667,0.705882}%
\pgfsetstrokecolor{currentstroke}%
\pgfsetstrokeopacity{0.588518}%
\pgfsetdash{}{0pt}%
\pgfpathmoveto{\pgfqpoint{2.278560in}{2.346746in}}%
\pgfpathcurveto{\pgfqpoint{2.286796in}{2.346746in}}{\pgfqpoint{2.294696in}{2.350019in}}{\pgfqpoint{2.300520in}{2.355843in}}%
\pgfpathcurveto{\pgfqpoint{2.306344in}{2.361667in}}{\pgfqpoint{2.309617in}{2.369567in}}{\pgfqpoint{2.309617in}{2.377803in}}%
\pgfpathcurveto{\pgfqpoint{2.309617in}{2.386039in}}{\pgfqpoint{2.306344in}{2.393939in}}{\pgfqpoint{2.300520in}{2.399763in}}%
\pgfpathcurveto{\pgfqpoint{2.294696in}{2.405587in}}{\pgfqpoint{2.286796in}{2.408859in}}{\pgfqpoint{2.278560in}{2.408859in}}%
\pgfpathcurveto{\pgfqpoint{2.270324in}{2.408859in}}{\pgfqpoint{2.262424in}{2.405587in}}{\pgfqpoint{2.256600in}{2.399763in}}%
\pgfpathcurveto{\pgfqpoint{2.250776in}{2.393939in}}{\pgfqpoint{2.247504in}{2.386039in}}{\pgfqpoint{2.247504in}{2.377803in}}%
\pgfpathcurveto{\pgfqpoint{2.247504in}{2.369567in}}{\pgfqpoint{2.250776in}{2.361667in}}{\pgfqpoint{2.256600in}{2.355843in}}%
\pgfpathcurveto{\pgfqpoint{2.262424in}{2.350019in}}{\pgfqpoint{2.270324in}{2.346746in}}{\pgfqpoint{2.278560in}{2.346746in}}%
\pgfpathclose%
\pgfusepath{stroke,fill}%
\end{pgfscope}%
\begin{pgfscope}%
\pgfpathrectangle{\pgfqpoint{0.100000in}{0.212622in}}{\pgfqpoint{3.696000in}{3.696000in}}%
\pgfusepath{clip}%
\pgfsetbuttcap%
\pgfsetroundjoin%
\definecolor{currentfill}{rgb}{0.121569,0.466667,0.705882}%
\pgfsetfillcolor{currentfill}%
\pgfsetfillopacity{0.592551}%
\pgfsetlinewidth{1.003750pt}%
\definecolor{currentstroke}{rgb}{0.121569,0.466667,0.705882}%
\pgfsetstrokecolor{currentstroke}%
\pgfsetstrokeopacity{0.592551}%
\pgfsetdash{}{0pt}%
\pgfpathmoveto{\pgfqpoint{1.259317in}{2.635533in}}%
\pgfpathcurveto{\pgfqpoint{1.267553in}{2.635533in}}{\pgfqpoint{1.275453in}{2.638806in}}{\pgfqpoint{1.281277in}{2.644630in}}%
\pgfpathcurveto{\pgfqpoint{1.287101in}{2.650453in}}{\pgfqpoint{1.290374in}{2.658354in}}{\pgfqpoint{1.290374in}{2.666590in}}%
\pgfpathcurveto{\pgfqpoint{1.290374in}{2.674826in}}{\pgfqpoint{1.287101in}{2.682726in}}{\pgfqpoint{1.281277in}{2.688550in}}%
\pgfpathcurveto{\pgfqpoint{1.275453in}{2.694374in}}{\pgfqpoint{1.267553in}{2.697646in}}{\pgfqpoint{1.259317in}{2.697646in}}%
\pgfpathcurveto{\pgfqpoint{1.251081in}{2.697646in}}{\pgfqpoint{1.243181in}{2.694374in}}{\pgfqpoint{1.237357in}{2.688550in}}%
\pgfpathcurveto{\pgfqpoint{1.231533in}{2.682726in}}{\pgfqpoint{1.228261in}{2.674826in}}{\pgfqpoint{1.228261in}{2.666590in}}%
\pgfpathcurveto{\pgfqpoint{1.228261in}{2.658354in}}{\pgfqpoint{1.231533in}{2.650453in}}{\pgfqpoint{1.237357in}{2.644630in}}%
\pgfpathcurveto{\pgfqpoint{1.243181in}{2.638806in}}{\pgfqpoint{1.251081in}{2.635533in}}{\pgfqpoint{1.259317in}{2.635533in}}%
\pgfpathclose%
\pgfusepath{stroke,fill}%
\end{pgfscope}%
\begin{pgfscope}%
\pgfpathrectangle{\pgfqpoint{0.100000in}{0.212622in}}{\pgfqpoint{3.696000in}{3.696000in}}%
\pgfusepath{clip}%
\pgfsetbuttcap%
\pgfsetroundjoin%
\definecolor{currentfill}{rgb}{0.121569,0.466667,0.705882}%
\pgfsetfillcolor{currentfill}%
\pgfsetfillopacity{0.600398}%
\pgfsetlinewidth{1.003750pt}%
\definecolor{currentstroke}{rgb}{0.121569,0.466667,0.705882}%
\pgfsetstrokecolor{currentstroke}%
\pgfsetstrokeopacity{0.600398}%
\pgfsetdash{}{0pt}%
\pgfpathmoveto{\pgfqpoint{1.248530in}{2.612830in}}%
\pgfpathcurveto{\pgfqpoint{1.256766in}{2.612830in}}{\pgfqpoint{1.264666in}{2.616103in}}{\pgfqpoint{1.270490in}{2.621926in}}%
\pgfpathcurveto{\pgfqpoint{1.276314in}{2.627750in}}{\pgfqpoint{1.279586in}{2.635650in}}{\pgfqpoint{1.279586in}{2.643887in}}%
\pgfpathcurveto{\pgfqpoint{1.279586in}{2.652123in}}{\pgfqpoint{1.276314in}{2.660023in}}{\pgfqpoint{1.270490in}{2.665847in}}%
\pgfpathcurveto{\pgfqpoint{1.264666in}{2.671671in}}{\pgfqpoint{1.256766in}{2.674943in}}{\pgfqpoint{1.248530in}{2.674943in}}%
\pgfpathcurveto{\pgfqpoint{1.240294in}{2.674943in}}{\pgfqpoint{1.232394in}{2.671671in}}{\pgfqpoint{1.226570in}{2.665847in}}%
\pgfpathcurveto{\pgfqpoint{1.220746in}{2.660023in}}{\pgfqpoint{1.217473in}{2.652123in}}{\pgfqpoint{1.217473in}{2.643887in}}%
\pgfpathcurveto{\pgfqpoint{1.217473in}{2.635650in}}{\pgfqpoint{1.220746in}{2.627750in}}{\pgfqpoint{1.226570in}{2.621926in}}%
\pgfpathcurveto{\pgfqpoint{1.232394in}{2.616103in}}{\pgfqpoint{1.240294in}{2.612830in}}{\pgfqpoint{1.248530in}{2.612830in}}%
\pgfpathclose%
\pgfusepath{stroke,fill}%
\end{pgfscope}%
\begin{pgfscope}%
\pgfpathrectangle{\pgfqpoint{0.100000in}{0.212622in}}{\pgfqpoint{3.696000in}{3.696000in}}%
\pgfusepath{clip}%
\pgfsetbuttcap%
\pgfsetroundjoin%
\definecolor{currentfill}{rgb}{0.121569,0.466667,0.705882}%
\pgfsetfillcolor{currentfill}%
\pgfsetfillopacity{0.600772}%
\pgfsetlinewidth{1.003750pt}%
\definecolor{currentstroke}{rgb}{0.121569,0.466667,0.705882}%
\pgfsetstrokecolor{currentstroke}%
\pgfsetstrokeopacity{0.600772}%
\pgfsetdash{}{0pt}%
\pgfpathmoveto{\pgfqpoint{2.296003in}{2.300847in}}%
\pgfpathcurveto{\pgfqpoint{2.304239in}{2.300847in}}{\pgfqpoint{2.312139in}{2.304120in}}{\pgfqpoint{2.317963in}{2.309943in}}%
\pgfpathcurveto{\pgfqpoint{2.323787in}{2.315767in}}{\pgfqpoint{2.327059in}{2.323667in}}{\pgfqpoint{2.327059in}{2.331904in}}%
\pgfpathcurveto{\pgfqpoint{2.327059in}{2.340140in}}{\pgfqpoint{2.323787in}{2.348040in}}{\pgfqpoint{2.317963in}{2.353864in}}%
\pgfpathcurveto{\pgfqpoint{2.312139in}{2.359688in}}{\pgfqpoint{2.304239in}{2.362960in}}{\pgfqpoint{2.296003in}{2.362960in}}%
\pgfpathcurveto{\pgfqpoint{2.287767in}{2.362960in}}{\pgfqpoint{2.279867in}{2.359688in}}{\pgfqpoint{2.274043in}{2.353864in}}%
\pgfpathcurveto{\pgfqpoint{2.268219in}{2.348040in}}{\pgfqpoint{2.264946in}{2.340140in}}{\pgfqpoint{2.264946in}{2.331904in}}%
\pgfpathcurveto{\pgfqpoint{2.264946in}{2.323667in}}{\pgfqpoint{2.268219in}{2.315767in}}{\pgfqpoint{2.274043in}{2.309943in}}%
\pgfpathcurveto{\pgfqpoint{2.279867in}{2.304120in}}{\pgfqpoint{2.287767in}{2.300847in}}{\pgfqpoint{2.296003in}{2.300847in}}%
\pgfpathclose%
\pgfusepath{stroke,fill}%
\end{pgfscope}%
\begin{pgfscope}%
\pgfpathrectangle{\pgfqpoint{0.100000in}{0.212622in}}{\pgfqpoint{3.696000in}{3.696000in}}%
\pgfusepath{clip}%
\pgfsetbuttcap%
\pgfsetroundjoin%
\definecolor{currentfill}{rgb}{0.121569,0.466667,0.705882}%
\pgfsetfillcolor{currentfill}%
\pgfsetfillopacity{0.607241}%
\pgfsetlinewidth{1.003750pt}%
\definecolor{currentstroke}{rgb}{0.121569,0.466667,0.705882}%
\pgfsetstrokecolor{currentstroke}%
\pgfsetstrokeopacity{0.607241}%
\pgfsetdash{}{0pt}%
\pgfpathmoveto{\pgfqpoint{1.242575in}{2.593295in}}%
\pgfpathcurveto{\pgfqpoint{1.250811in}{2.593295in}}{\pgfqpoint{1.258711in}{2.596568in}}{\pgfqpoint{1.264535in}{2.602392in}}%
\pgfpathcurveto{\pgfqpoint{1.270359in}{2.608216in}}{\pgfqpoint{1.273632in}{2.616116in}}{\pgfqpoint{1.273632in}{2.624352in}}%
\pgfpathcurveto{\pgfqpoint{1.273632in}{2.632588in}}{\pgfqpoint{1.270359in}{2.640488in}}{\pgfqpoint{1.264535in}{2.646312in}}%
\pgfpathcurveto{\pgfqpoint{1.258711in}{2.652136in}}{\pgfqpoint{1.250811in}{2.655408in}}{\pgfqpoint{1.242575in}{2.655408in}}%
\pgfpathcurveto{\pgfqpoint{1.234339in}{2.655408in}}{\pgfqpoint{1.226439in}{2.652136in}}{\pgfqpoint{1.220615in}{2.646312in}}%
\pgfpathcurveto{\pgfqpoint{1.214791in}{2.640488in}}{\pgfqpoint{1.211519in}{2.632588in}}{\pgfqpoint{1.211519in}{2.624352in}}%
\pgfpathcurveto{\pgfqpoint{1.211519in}{2.616116in}}{\pgfqpoint{1.214791in}{2.608216in}}{\pgfqpoint{1.220615in}{2.602392in}}%
\pgfpathcurveto{\pgfqpoint{1.226439in}{2.596568in}}{\pgfqpoint{1.234339in}{2.593295in}}{\pgfqpoint{1.242575in}{2.593295in}}%
\pgfpathclose%
\pgfusepath{stroke,fill}%
\end{pgfscope}%
\begin{pgfscope}%
\pgfpathrectangle{\pgfqpoint{0.100000in}{0.212622in}}{\pgfqpoint{3.696000in}{3.696000in}}%
\pgfusepath{clip}%
\pgfsetbuttcap%
\pgfsetroundjoin%
\definecolor{currentfill}{rgb}{0.121569,0.466667,0.705882}%
\pgfsetfillcolor{currentfill}%
\pgfsetfillopacity{0.607926}%
\pgfsetlinewidth{1.003750pt}%
\definecolor{currentstroke}{rgb}{0.121569,0.466667,0.705882}%
\pgfsetstrokecolor{currentstroke}%
\pgfsetstrokeopacity{0.607926}%
\pgfsetdash{}{0pt}%
\pgfpathmoveto{\pgfqpoint{2.303917in}{2.276233in}}%
\pgfpathcurveto{\pgfqpoint{2.312153in}{2.276233in}}{\pgfqpoint{2.320053in}{2.279506in}}{\pgfqpoint{2.325877in}{2.285330in}}%
\pgfpathcurveto{\pgfqpoint{2.331701in}{2.291154in}}{\pgfqpoint{2.334974in}{2.299054in}}{\pgfqpoint{2.334974in}{2.307290in}}%
\pgfpathcurveto{\pgfqpoint{2.334974in}{2.315526in}}{\pgfqpoint{2.331701in}{2.323426in}}{\pgfqpoint{2.325877in}{2.329250in}}%
\pgfpathcurveto{\pgfqpoint{2.320053in}{2.335074in}}{\pgfqpoint{2.312153in}{2.338346in}}{\pgfqpoint{2.303917in}{2.338346in}}%
\pgfpathcurveto{\pgfqpoint{2.295681in}{2.338346in}}{\pgfqpoint{2.287781in}{2.335074in}}{\pgfqpoint{2.281957in}{2.329250in}}%
\pgfpathcurveto{\pgfqpoint{2.276133in}{2.323426in}}{\pgfqpoint{2.272861in}{2.315526in}}{\pgfqpoint{2.272861in}{2.307290in}}%
\pgfpathcurveto{\pgfqpoint{2.272861in}{2.299054in}}{\pgfqpoint{2.276133in}{2.291154in}}{\pgfqpoint{2.281957in}{2.285330in}}%
\pgfpathcurveto{\pgfqpoint{2.287781in}{2.279506in}}{\pgfqpoint{2.295681in}{2.276233in}}{\pgfqpoint{2.303917in}{2.276233in}}%
\pgfpathclose%
\pgfusepath{stroke,fill}%
\end{pgfscope}%
\begin{pgfscope}%
\pgfpathrectangle{\pgfqpoint{0.100000in}{0.212622in}}{\pgfqpoint{3.696000in}{3.696000in}}%
\pgfusepath{clip}%
\pgfsetbuttcap%
\pgfsetroundjoin%
\definecolor{currentfill}{rgb}{0.121569,0.466667,0.705882}%
\pgfsetfillcolor{currentfill}%
\pgfsetfillopacity{0.612319}%
\pgfsetlinewidth{1.003750pt}%
\definecolor{currentstroke}{rgb}{0.121569,0.466667,0.705882}%
\pgfsetstrokecolor{currentstroke}%
\pgfsetstrokeopacity{0.612319}%
\pgfsetdash{}{0pt}%
\pgfpathmoveto{\pgfqpoint{1.235303in}{2.578610in}}%
\pgfpathcurveto{\pgfqpoint{1.243539in}{2.578610in}}{\pgfqpoint{1.251439in}{2.581883in}}{\pgfqpoint{1.257263in}{2.587707in}}%
\pgfpathcurveto{\pgfqpoint{1.263087in}{2.593531in}}{\pgfqpoint{1.266359in}{2.601431in}}{\pgfqpoint{1.266359in}{2.609667in}}%
\pgfpathcurveto{\pgfqpoint{1.266359in}{2.617903in}}{\pgfqpoint{1.263087in}{2.625803in}}{\pgfqpoint{1.257263in}{2.631627in}}%
\pgfpathcurveto{\pgfqpoint{1.251439in}{2.637451in}}{\pgfqpoint{1.243539in}{2.640723in}}{\pgfqpoint{1.235303in}{2.640723in}}%
\pgfpathcurveto{\pgfqpoint{1.227067in}{2.640723in}}{\pgfqpoint{1.219167in}{2.637451in}}{\pgfqpoint{1.213343in}{2.631627in}}%
\pgfpathcurveto{\pgfqpoint{1.207519in}{2.625803in}}{\pgfqpoint{1.204246in}{2.617903in}}{\pgfqpoint{1.204246in}{2.609667in}}%
\pgfpathcurveto{\pgfqpoint{1.204246in}{2.601431in}}{\pgfqpoint{1.207519in}{2.593531in}}{\pgfqpoint{1.213343in}{2.587707in}}%
\pgfpathcurveto{\pgfqpoint{1.219167in}{2.581883in}}{\pgfqpoint{1.227067in}{2.578610in}}{\pgfqpoint{1.235303in}{2.578610in}}%
\pgfpathclose%
\pgfusepath{stroke,fill}%
\end{pgfscope}%
\begin{pgfscope}%
\pgfpathrectangle{\pgfqpoint{0.100000in}{0.212622in}}{\pgfqpoint{3.696000in}{3.696000in}}%
\pgfusepath{clip}%
\pgfsetbuttcap%
\pgfsetroundjoin%
\definecolor{currentfill}{rgb}{0.121569,0.466667,0.705882}%
\pgfsetfillcolor{currentfill}%
\pgfsetfillopacity{0.615927}%
\pgfsetlinewidth{1.003750pt}%
\definecolor{currentstroke}{rgb}{0.121569,0.466667,0.705882}%
\pgfsetstrokecolor{currentstroke}%
\pgfsetstrokeopacity{0.615927}%
\pgfsetdash{}{0pt}%
\pgfpathmoveto{\pgfqpoint{1.232391in}{2.568269in}}%
\pgfpathcurveto{\pgfqpoint{1.240628in}{2.568269in}}{\pgfqpoint{1.248528in}{2.571541in}}{\pgfqpoint{1.254352in}{2.577365in}}%
\pgfpathcurveto{\pgfqpoint{1.260176in}{2.583189in}}{\pgfqpoint{1.263448in}{2.591089in}}{\pgfqpoint{1.263448in}{2.599325in}}%
\pgfpathcurveto{\pgfqpoint{1.263448in}{2.607561in}}{\pgfqpoint{1.260176in}{2.615461in}}{\pgfqpoint{1.254352in}{2.621285in}}%
\pgfpathcurveto{\pgfqpoint{1.248528in}{2.627109in}}{\pgfqpoint{1.240628in}{2.630382in}}{\pgfqpoint{1.232391in}{2.630382in}}%
\pgfpathcurveto{\pgfqpoint{1.224155in}{2.630382in}}{\pgfqpoint{1.216255in}{2.627109in}}{\pgfqpoint{1.210431in}{2.621285in}}%
\pgfpathcurveto{\pgfqpoint{1.204607in}{2.615461in}}{\pgfqpoint{1.201335in}{2.607561in}}{\pgfqpoint{1.201335in}{2.599325in}}%
\pgfpathcurveto{\pgfqpoint{1.201335in}{2.591089in}}{\pgfqpoint{1.204607in}{2.583189in}}{\pgfqpoint{1.210431in}{2.577365in}}%
\pgfpathcurveto{\pgfqpoint{1.216255in}{2.571541in}}{\pgfqpoint{1.224155in}{2.568269in}}{\pgfqpoint{1.232391in}{2.568269in}}%
\pgfpathclose%
\pgfusepath{stroke,fill}%
\end{pgfscope}%
\begin{pgfscope}%
\pgfpathrectangle{\pgfqpoint{0.100000in}{0.212622in}}{\pgfqpoint{3.696000in}{3.696000in}}%
\pgfusepath{clip}%
\pgfsetbuttcap%
\pgfsetroundjoin%
\definecolor{currentfill}{rgb}{0.121569,0.466667,0.705882}%
\pgfsetfillcolor{currentfill}%
\pgfsetfillopacity{0.616567}%
\pgfsetlinewidth{1.003750pt}%
\definecolor{currentstroke}{rgb}{0.121569,0.466667,0.705882}%
\pgfsetstrokecolor{currentstroke}%
\pgfsetstrokeopacity{0.616567}%
\pgfsetdash{}{0pt}%
\pgfpathmoveto{\pgfqpoint{2.315279in}{2.245009in}}%
\pgfpathcurveto{\pgfqpoint{2.323515in}{2.245009in}}{\pgfqpoint{2.331415in}{2.248281in}}{\pgfqpoint{2.337239in}{2.254105in}}%
\pgfpathcurveto{\pgfqpoint{2.343063in}{2.259929in}}{\pgfqpoint{2.346335in}{2.267829in}}{\pgfqpoint{2.346335in}{2.276066in}}%
\pgfpathcurveto{\pgfqpoint{2.346335in}{2.284302in}}{\pgfqpoint{2.343063in}{2.292202in}}{\pgfqpoint{2.337239in}{2.298026in}}%
\pgfpathcurveto{\pgfqpoint{2.331415in}{2.303850in}}{\pgfqpoint{2.323515in}{2.307122in}}{\pgfqpoint{2.315279in}{2.307122in}}%
\pgfpathcurveto{\pgfqpoint{2.307043in}{2.307122in}}{\pgfqpoint{2.299143in}{2.303850in}}{\pgfqpoint{2.293319in}{2.298026in}}%
\pgfpathcurveto{\pgfqpoint{2.287495in}{2.292202in}}{\pgfqpoint{2.284222in}{2.284302in}}{\pgfqpoint{2.284222in}{2.276066in}}%
\pgfpathcurveto{\pgfqpoint{2.284222in}{2.267829in}}{\pgfqpoint{2.287495in}{2.259929in}}{\pgfqpoint{2.293319in}{2.254105in}}%
\pgfpathcurveto{\pgfqpoint{2.299143in}{2.248281in}}{\pgfqpoint{2.307043in}{2.245009in}}{\pgfqpoint{2.315279in}{2.245009in}}%
\pgfpathclose%
\pgfusepath{stroke,fill}%
\end{pgfscope}%
\begin{pgfscope}%
\pgfpathrectangle{\pgfqpoint{0.100000in}{0.212622in}}{\pgfqpoint{3.696000in}{3.696000in}}%
\pgfusepath{clip}%
\pgfsetbuttcap%
\pgfsetroundjoin%
\definecolor{currentfill}{rgb}{0.121569,0.466667,0.705882}%
\pgfsetfillcolor{currentfill}%
\pgfsetfillopacity{0.617873}%
\pgfsetlinewidth{1.003750pt}%
\definecolor{currentstroke}{rgb}{0.121569,0.466667,0.705882}%
\pgfsetstrokecolor{currentstroke}%
\pgfsetstrokeopacity{0.617873}%
\pgfsetdash{}{0pt}%
\pgfpathmoveto{\pgfqpoint{1.229834in}{2.562622in}}%
\pgfpathcurveto{\pgfqpoint{1.238070in}{2.562622in}}{\pgfqpoint{1.245970in}{2.565894in}}{\pgfqpoint{1.251794in}{2.571718in}}%
\pgfpathcurveto{\pgfqpoint{1.257618in}{2.577542in}}{\pgfqpoint{1.260890in}{2.585442in}}{\pgfqpoint{1.260890in}{2.593678in}}%
\pgfpathcurveto{\pgfqpoint{1.260890in}{2.601915in}}{\pgfqpoint{1.257618in}{2.609815in}}{\pgfqpoint{1.251794in}{2.615639in}}%
\pgfpathcurveto{\pgfqpoint{1.245970in}{2.621463in}}{\pgfqpoint{1.238070in}{2.624735in}}{\pgfqpoint{1.229834in}{2.624735in}}%
\pgfpathcurveto{\pgfqpoint{1.221598in}{2.624735in}}{\pgfqpoint{1.213698in}{2.621463in}}{\pgfqpoint{1.207874in}{2.615639in}}%
\pgfpathcurveto{\pgfqpoint{1.202050in}{2.609815in}}{\pgfqpoint{1.198777in}{2.601915in}}{\pgfqpoint{1.198777in}{2.593678in}}%
\pgfpathcurveto{\pgfqpoint{1.198777in}{2.585442in}}{\pgfqpoint{1.202050in}{2.577542in}}{\pgfqpoint{1.207874in}{2.571718in}}%
\pgfpathcurveto{\pgfqpoint{1.213698in}{2.565894in}}{\pgfqpoint{1.221598in}{2.562622in}}{\pgfqpoint{1.229834in}{2.562622in}}%
\pgfpathclose%
\pgfusepath{stroke,fill}%
\end{pgfscope}%
\begin{pgfscope}%
\pgfpathrectangle{\pgfqpoint{0.100000in}{0.212622in}}{\pgfqpoint{3.696000in}{3.696000in}}%
\pgfusepath{clip}%
\pgfsetbuttcap%
\pgfsetroundjoin%
\definecolor{currentfill}{rgb}{0.121569,0.466667,0.705882}%
\pgfsetfillcolor{currentfill}%
\pgfsetfillopacity{0.621563}%
\pgfsetlinewidth{1.003750pt}%
\definecolor{currentstroke}{rgb}{0.121569,0.466667,0.705882}%
\pgfsetstrokecolor{currentstroke}%
\pgfsetstrokeopacity{0.621563}%
\pgfsetdash{}{0pt}%
\pgfpathmoveto{\pgfqpoint{2.320311in}{2.228155in}}%
\pgfpathcurveto{\pgfqpoint{2.328547in}{2.228155in}}{\pgfqpoint{2.336447in}{2.231427in}}{\pgfqpoint{2.342271in}{2.237251in}}%
\pgfpathcurveto{\pgfqpoint{2.348095in}{2.243075in}}{\pgfqpoint{2.351367in}{2.250975in}}{\pgfqpoint{2.351367in}{2.259212in}}%
\pgfpathcurveto{\pgfqpoint{2.351367in}{2.267448in}}{\pgfqpoint{2.348095in}{2.275348in}}{\pgfqpoint{2.342271in}{2.281172in}}%
\pgfpathcurveto{\pgfqpoint{2.336447in}{2.286996in}}{\pgfqpoint{2.328547in}{2.290268in}}{\pgfqpoint{2.320311in}{2.290268in}}%
\pgfpathcurveto{\pgfqpoint{2.312075in}{2.290268in}}{\pgfqpoint{2.304175in}{2.286996in}}{\pgfqpoint{2.298351in}{2.281172in}}%
\pgfpathcurveto{\pgfqpoint{2.292527in}{2.275348in}}{\pgfqpoint{2.289254in}{2.267448in}}{\pgfqpoint{2.289254in}{2.259212in}}%
\pgfpathcurveto{\pgfqpoint{2.289254in}{2.250975in}}{\pgfqpoint{2.292527in}{2.243075in}}{\pgfqpoint{2.298351in}{2.237251in}}%
\pgfpathcurveto{\pgfqpoint{2.304175in}{2.231427in}}{\pgfqpoint{2.312075in}{2.228155in}}{\pgfqpoint{2.320311in}{2.228155in}}%
\pgfpathclose%
\pgfusepath{stroke,fill}%
\end{pgfscope}%
\begin{pgfscope}%
\pgfpathrectangle{\pgfqpoint{0.100000in}{0.212622in}}{\pgfqpoint{3.696000in}{3.696000in}}%
\pgfusepath{clip}%
\pgfsetbuttcap%
\pgfsetroundjoin%
\definecolor{currentfill}{rgb}{0.121569,0.466667,0.705882}%
\pgfsetfillcolor{currentfill}%
\pgfsetfillopacity{0.621573}%
\pgfsetlinewidth{1.003750pt}%
\definecolor{currentstroke}{rgb}{0.121569,0.466667,0.705882}%
\pgfsetstrokecolor{currentstroke}%
\pgfsetstrokeopacity{0.621573}%
\pgfsetdash{}{0pt}%
\pgfpathmoveto{\pgfqpoint{1.226770in}{2.551921in}}%
\pgfpathcurveto{\pgfqpoint{1.235006in}{2.551921in}}{\pgfqpoint{1.242906in}{2.555194in}}{\pgfqpoint{1.248730in}{2.561017in}}%
\pgfpathcurveto{\pgfqpoint{1.254554in}{2.566841in}}{\pgfqpoint{1.257826in}{2.574741in}}{\pgfqpoint{1.257826in}{2.582978in}}%
\pgfpathcurveto{\pgfqpoint{1.257826in}{2.591214in}}{\pgfqpoint{1.254554in}{2.599114in}}{\pgfqpoint{1.248730in}{2.604938in}}%
\pgfpathcurveto{\pgfqpoint{1.242906in}{2.610762in}}{\pgfqpoint{1.235006in}{2.614034in}}{\pgfqpoint{1.226770in}{2.614034in}}%
\pgfpathcurveto{\pgfqpoint{1.218534in}{2.614034in}}{\pgfqpoint{1.210634in}{2.610762in}}{\pgfqpoint{1.204810in}{2.604938in}}%
\pgfpathcurveto{\pgfqpoint{1.198986in}{2.599114in}}{\pgfqpoint{1.195713in}{2.591214in}}{\pgfqpoint{1.195713in}{2.582978in}}%
\pgfpathcurveto{\pgfqpoint{1.195713in}{2.574741in}}{\pgfqpoint{1.198986in}{2.566841in}}{\pgfqpoint{1.204810in}{2.561017in}}%
\pgfpathcurveto{\pgfqpoint{1.210634in}{2.555194in}}{\pgfqpoint{1.218534in}{2.551921in}}{\pgfqpoint{1.226770in}{2.551921in}}%
\pgfpathclose%
\pgfusepath{stroke,fill}%
\end{pgfscope}%
\begin{pgfscope}%
\pgfpathrectangle{\pgfqpoint{0.100000in}{0.212622in}}{\pgfqpoint{3.696000in}{3.696000in}}%
\pgfusepath{clip}%
\pgfsetbuttcap%
\pgfsetroundjoin%
\definecolor{currentfill}{rgb}{0.121569,0.466667,0.705882}%
\pgfsetfillcolor{currentfill}%
\pgfsetfillopacity{0.628102}%
\pgfsetlinewidth{1.003750pt}%
\definecolor{currentstroke}{rgb}{0.121569,0.466667,0.705882}%
\pgfsetstrokecolor{currentstroke}%
\pgfsetstrokeopacity{0.628102}%
\pgfsetdash{}{0pt}%
\pgfpathmoveto{\pgfqpoint{1.218656in}{2.533199in}}%
\pgfpathcurveto{\pgfqpoint{1.226893in}{2.533199in}}{\pgfqpoint{1.234793in}{2.536472in}}{\pgfqpoint{1.240617in}{2.542296in}}%
\pgfpathcurveto{\pgfqpoint{1.246441in}{2.548120in}}{\pgfqpoint{1.249713in}{2.556020in}}{\pgfqpoint{1.249713in}{2.564256in}}%
\pgfpathcurveto{\pgfqpoint{1.249713in}{2.572492in}}{\pgfqpoint{1.246441in}{2.580392in}}{\pgfqpoint{1.240617in}{2.586216in}}%
\pgfpathcurveto{\pgfqpoint{1.234793in}{2.592040in}}{\pgfqpoint{1.226893in}{2.595312in}}{\pgfqpoint{1.218656in}{2.595312in}}%
\pgfpathcurveto{\pgfqpoint{1.210420in}{2.595312in}}{\pgfqpoint{1.202520in}{2.592040in}}{\pgfqpoint{1.196696in}{2.586216in}}%
\pgfpathcurveto{\pgfqpoint{1.190872in}{2.580392in}}{\pgfqpoint{1.187600in}{2.572492in}}{\pgfqpoint{1.187600in}{2.564256in}}%
\pgfpathcurveto{\pgfqpoint{1.187600in}{2.556020in}}{\pgfqpoint{1.190872in}{2.548120in}}{\pgfqpoint{1.196696in}{2.542296in}}%
\pgfpathcurveto{\pgfqpoint{1.202520in}{2.536472in}}{\pgfqpoint{1.210420in}{2.533199in}}{\pgfqpoint{1.218656in}{2.533199in}}%
\pgfpathclose%
\pgfusepath{stroke,fill}%
\end{pgfscope}%
\begin{pgfscope}%
\pgfpathrectangle{\pgfqpoint{0.100000in}{0.212622in}}{\pgfqpoint{3.696000in}{3.696000in}}%
\pgfusepath{clip}%
\pgfsetbuttcap%
\pgfsetroundjoin%
\definecolor{currentfill}{rgb}{0.121569,0.466667,0.705882}%
\pgfsetfillcolor{currentfill}%
\pgfsetfillopacity{0.628205}%
\pgfsetlinewidth{1.003750pt}%
\definecolor{currentstroke}{rgb}{0.121569,0.466667,0.705882}%
\pgfsetstrokecolor{currentstroke}%
\pgfsetstrokeopacity{0.628205}%
\pgfsetdash{}{0pt}%
\pgfpathmoveto{\pgfqpoint{2.329571in}{2.203155in}}%
\pgfpathcurveto{\pgfqpoint{2.337807in}{2.203155in}}{\pgfqpoint{2.345707in}{2.206428in}}{\pgfqpoint{2.351531in}{2.212252in}}%
\pgfpathcurveto{\pgfqpoint{2.357355in}{2.218076in}}{\pgfqpoint{2.360628in}{2.225976in}}{\pgfqpoint{2.360628in}{2.234212in}}%
\pgfpathcurveto{\pgfqpoint{2.360628in}{2.242448in}}{\pgfqpoint{2.357355in}{2.250348in}}{\pgfqpoint{2.351531in}{2.256172in}}%
\pgfpathcurveto{\pgfqpoint{2.345707in}{2.261996in}}{\pgfqpoint{2.337807in}{2.265268in}}{\pgfqpoint{2.329571in}{2.265268in}}%
\pgfpathcurveto{\pgfqpoint{2.321335in}{2.265268in}}{\pgfqpoint{2.313435in}{2.261996in}}{\pgfqpoint{2.307611in}{2.256172in}}%
\pgfpathcurveto{\pgfqpoint{2.301787in}{2.250348in}}{\pgfqpoint{2.298515in}{2.242448in}}{\pgfqpoint{2.298515in}{2.234212in}}%
\pgfpathcurveto{\pgfqpoint{2.298515in}{2.225976in}}{\pgfqpoint{2.301787in}{2.218076in}}{\pgfqpoint{2.307611in}{2.212252in}}%
\pgfpathcurveto{\pgfqpoint{2.313435in}{2.206428in}}{\pgfqpoint{2.321335in}{2.203155in}}{\pgfqpoint{2.329571in}{2.203155in}}%
\pgfpathclose%
\pgfusepath{stroke,fill}%
\end{pgfscope}%
\begin{pgfscope}%
\pgfpathrectangle{\pgfqpoint{0.100000in}{0.212622in}}{\pgfqpoint{3.696000in}{3.696000in}}%
\pgfusepath{clip}%
\pgfsetbuttcap%
\pgfsetroundjoin%
\definecolor{currentfill}{rgb}{0.121569,0.466667,0.705882}%
\pgfsetfillcolor{currentfill}%
\pgfsetfillopacity{0.633954}%
\pgfsetlinewidth{1.003750pt}%
\definecolor{currentstroke}{rgb}{0.121569,0.466667,0.705882}%
\pgfsetstrokecolor{currentstroke}%
\pgfsetstrokeopacity{0.633954}%
\pgfsetdash{}{0pt}%
\pgfpathmoveto{\pgfqpoint{1.213839in}{2.516674in}}%
\pgfpathcurveto{\pgfqpoint{1.222076in}{2.516674in}}{\pgfqpoint{1.229976in}{2.519947in}}{\pgfqpoint{1.235800in}{2.525770in}}%
\pgfpathcurveto{\pgfqpoint{1.241624in}{2.531594in}}{\pgfqpoint{1.244896in}{2.539494in}}{\pgfqpoint{1.244896in}{2.547731in}}%
\pgfpathcurveto{\pgfqpoint{1.244896in}{2.555967in}}{\pgfqpoint{1.241624in}{2.563867in}}{\pgfqpoint{1.235800in}{2.569691in}}%
\pgfpathcurveto{\pgfqpoint{1.229976in}{2.575515in}}{\pgfqpoint{1.222076in}{2.578787in}}{\pgfqpoint{1.213839in}{2.578787in}}%
\pgfpathcurveto{\pgfqpoint{1.205603in}{2.578787in}}{\pgfqpoint{1.197703in}{2.575515in}}{\pgfqpoint{1.191879in}{2.569691in}}%
\pgfpathcurveto{\pgfqpoint{1.186055in}{2.563867in}}{\pgfqpoint{1.182783in}{2.555967in}}{\pgfqpoint{1.182783in}{2.547731in}}%
\pgfpathcurveto{\pgfqpoint{1.182783in}{2.539494in}}{\pgfqpoint{1.186055in}{2.531594in}}{\pgfqpoint{1.191879in}{2.525770in}}%
\pgfpathcurveto{\pgfqpoint{1.197703in}{2.519947in}}{\pgfqpoint{1.205603in}{2.516674in}}{\pgfqpoint{1.213839in}{2.516674in}}%
\pgfpathclose%
\pgfusepath{stroke,fill}%
\end{pgfscope}%
\begin{pgfscope}%
\pgfpathrectangle{\pgfqpoint{0.100000in}{0.212622in}}{\pgfqpoint{3.696000in}{3.696000in}}%
\pgfusepath{clip}%
\pgfsetbuttcap%
\pgfsetroundjoin%
\definecolor{currentfill}{rgb}{0.121569,0.466667,0.705882}%
\pgfsetfillcolor{currentfill}%
\pgfsetfillopacity{0.636403}%
\pgfsetlinewidth{1.003750pt}%
\definecolor{currentstroke}{rgb}{0.121569,0.466667,0.705882}%
\pgfsetstrokecolor{currentstroke}%
\pgfsetstrokeopacity{0.636403}%
\pgfsetdash{}{0pt}%
\pgfpathmoveto{\pgfqpoint{2.337933in}{2.173552in}}%
\pgfpathcurveto{\pgfqpoint{2.346169in}{2.173552in}}{\pgfqpoint{2.354069in}{2.176825in}}{\pgfqpoint{2.359893in}{2.182648in}}%
\pgfpathcurveto{\pgfqpoint{2.365717in}{2.188472in}}{\pgfqpoint{2.368989in}{2.196372in}}{\pgfqpoint{2.368989in}{2.204609in}}%
\pgfpathcurveto{\pgfqpoint{2.368989in}{2.212845in}}{\pgfqpoint{2.365717in}{2.220745in}}{\pgfqpoint{2.359893in}{2.226569in}}%
\pgfpathcurveto{\pgfqpoint{2.354069in}{2.232393in}}{\pgfqpoint{2.346169in}{2.235665in}}{\pgfqpoint{2.337933in}{2.235665in}}%
\pgfpathcurveto{\pgfqpoint{2.329696in}{2.235665in}}{\pgfqpoint{2.321796in}{2.232393in}}{\pgfqpoint{2.315972in}{2.226569in}}%
\pgfpathcurveto{\pgfqpoint{2.310148in}{2.220745in}}{\pgfqpoint{2.306876in}{2.212845in}}{\pgfqpoint{2.306876in}{2.204609in}}%
\pgfpathcurveto{\pgfqpoint{2.306876in}{2.196372in}}{\pgfqpoint{2.310148in}{2.188472in}}{\pgfqpoint{2.315972in}{2.182648in}}%
\pgfpathcurveto{\pgfqpoint{2.321796in}{2.176825in}}{\pgfqpoint{2.329696in}{2.173552in}}{\pgfqpoint{2.337933in}{2.173552in}}%
\pgfpathclose%
\pgfusepath{stroke,fill}%
\end{pgfscope}%
\begin{pgfscope}%
\pgfpathrectangle{\pgfqpoint{0.100000in}{0.212622in}}{\pgfqpoint{3.696000in}{3.696000in}}%
\pgfusepath{clip}%
\pgfsetbuttcap%
\pgfsetroundjoin%
\definecolor{currentfill}{rgb}{0.121569,0.466667,0.705882}%
\pgfsetfillcolor{currentfill}%
\pgfsetfillopacity{0.644132}%
\pgfsetlinewidth{1.003750pt}%
\definecolor{currentstroke}{rgb}{0.121569,0.466667,0.705882}%
\pgfsetstrokecolor{currentstroke}%
\pgfsetstrokeopacity{0.644132}%
\pgfsetdash{}{0pt}%
\pgfpathmoveto{\pgfqpoint{1.201204in}{2.486826in}}%
\pgfpathcurveto{\pgfqpoint{1.209440in}{2.486826in}}{\pgfqpoint{1.217340in}{2.490098in}}{\pgfqpoint{1.223164in}{2.495922in}}%
\pgfpathcurveto{\pgfqpoint{1.228988in}{2.501746in}}{\pgfqpoint{1.232260in}{2.509646in}}{\pgfqpoint{1.232260in}{2.517882in}}%
\pgfpathcurveto{\pgfqpoint{1.232260in}{2.526119in}}{\pgfqpoint{1.228988in}{2.534019in}}{\pgfqpoint{1.223164in}{2.539843in}}%
\pgfpathcurveto{\pgfqpoint{1.217340in}{2.545666in}}{\pgfqpoint{1.209440in}{2.548939in}}{\pgfqpoint{1.201204in}{2.548939in}}%
\pgfpathcurveto{\pgfqpoint{1.192968in}{2.548939in}}{\pgfqpoint{1.185068in}{2.545666in}}{\pgfqpoint{1.179244in}{2.539843in}}%
\pgfpathcurveto{\pgfqpoint{1.173420in}{2.534019in}}{\pgfqpoint{1.170147in}{2.526119in}}{\pgfqpoint{1.170147in}{2.517882in}}%
\pgfpathcurveto{\pgfqpoint{1.170147in}{2.509646in}}{\pgfqpoint{1.173420in}{2.501746in}}{\pgfqpoint{1.179244in}{2.495922in}}%
\pgfpathcurveto{\pgfqpoint{1.185068in}{2.490098in}}{\pgfqpoint{1.192968in}{2.486826in}}{\pgfqpoint{1.201204in}{2.486826in}}%
\pgfpathclose%
\pgfusepath{stroke,fill}%
\end{pgfscope}%
\begin{pgfscope}%
\pgfpathrectangle{\pgfqpoint{0.100000in}{0.212622in}}{\pgfqpoint{3.696000in}{3.696000in}}%
\pgfusepath{clip}%
\pgfsetbuttcap%
\pgfsetroundjoin%
\definecolor{currentfill}{rgb}{0.121569,0.466667,0.705882}%
\pgfsetfillcolor{currentfill}%
\pgfsetfillopacity{0.645748}%
\pgfsetlinewidth{1.003750pt}%
\definecolor{currentstroke}{rgb}{0.121569,0.466667,0.705882}%
\pgfsetstrokecolor{currentstroke}%
\pgfsetstrokeopacity{0.645748}%
\pgfsetdash{}{0pt}%
\pgfpathmoveto{\pgfqpoint{2.350444in}{2.137089in}}%
\pgfpathcurveto{\pgfqpoint{2.358681in}{2.137089in}}{\pgfqpoint{2.366581in}{2.140362in}}{\pgfqpoint{2.372405in}{2.146185in}}%
\pgfpathcurveto{\pgfqpoint{2.378229in}{2.152009in}}{\pgfqpoint{2.381501in}{2.159909in}}{\pgfqpoint{2.381501in}{2.168146in}}%
\pgfpathcurveto{\pgfqpoint{2.381501in}{2.176382in}}{\pgfqpoint{2.378229in}{2.184282in}}{\pgfqpoint{2.372405in}{2.190106in}}%
\pgfpathcurveto{\pgfqpoint{2.366581in}{2.195930in}}{\pgfqpoint{2.358681in}{2.199202in}}{\pgfqpoint{2.350444in}{2.199202in}}%
\pgfpathcurveto{\pgfqpoint{2.342208in}{2.199202in}}{\pgfqpoint{2.334308in}{2.195930in}}{\pgfqpoint{2.328484in}{2.190106in}}%
\pgfpathcurveto{\pgfqpoint{2.322660in}{2.184282in}}{\pgfqpoint{2.319388in}{2.176382in}}{\pgfqpoint{2.319388in}{2.168146in}}%
\pgfpathcurveto{\pgfqpoint{2.319388in}{2.159909in}}{\pgfqpoint{2.322660in}{2.152009in}}{\pgfqpoint{2.328484in}{2.146185in}}%
\pgfpathcurveto{\pgfqpoint{2.334308in}{2.140362in}}{\pgfqpoint{2.342208in}{2.137089in}}{\pgfqpoint{2.350444in}{2.137089in}}%
\pgfpathclose%
\pgfusepath{stroke,fill}%
\end{pgfscope}%
\begin{pgfscope}%
\pgfpathrectangle{\pgfqpoint{0.100000in}{0.212622in}}{\pgfqpoint{3.696000in}{3.696000in}}%
\pgfusepath{clip}%
\pgfsetbuttcap%
\pgfsetroundjoin%
\definecolor{currentfill}{rgb}{0.121569,0.466667,0.705882}%
\pgfsetfillcolor{currentfill}%
\pgfsetfillopacity{0.653262}%
\pgfsetlinewidth{1.003750pt}%
\definecolor{currentstroke}{rgb}{0.121569,0.466667,0.705882}%
\pgfsetstrokecolor{currentstroke}%
\pgfsetstrokeopacity{0.653262}%
\pgfsetdash{}{0pt}%
\pgfpathmoveto{\pgfqpoint{1.192719in}{2.459824in}}%
\pgfpathcurveto{\pgfqpoint{1.200955in}{2.459824in}}{\pgfqpoint{1.208855in}{2.463096in}}{\pgfqpoint{1.214679in}{2.468920in}}%
\pgfpathcurveto{\pgfqpoint{1.220503in}{2.474744in}}{\pgfqpoint{1.223775in}{2.482644in}}{\pgfqpoint{1.223775in}{2.490880in}}%
\pgfpathcurveto{\pgfqpoint{1.223775in}{2.499116in}}{\pgfqpoint{1.220503in}{2.507016in}}{\pgfqpoint{1.214679in}{2.512840in}}%
\pgfpathcurveto{\pgfqpoint{1.208855in}{2.518664in}}{\pgfqpoint{1.200955in}{2.521937in}}{\pgfqpoint{1.192719in}{2.521937in}}%
\pgfpathcurveto{\pgfqpoint{1.184482in}{2.521937in}}{\pgfqpoint{1.176582in}{2.518664in}}{\pgfqpoint{1.170758in}{2.512840in}}%
\pgfpathcurveto{\pgfqpoint{1.164935in}{2.507016in}}{\pgfqpoint{1.161662in}{2.499116in}}{\pgfqpoint{1.161662in}{2.490880in}}%
\pgfpathcurveto{\pgfqpoint{1.161662in}{2.482644in}}{\pgfqpoint{1.164935in}{2.474744in}}{\pgfqpoint{1.170758in}{2.468920in}}%
\pgfpathcurveto{\pgfqpoint{1.176582in}{2.463096in}}{\pgfqpoint{1.184482in}{2.459824in}}{\pgfqpoint{1.192719in}{2.459824in}}%
\pgfpathclose%
\pgfusepath{stroke,fill}%
\end{pgfscope}%
\begin{pgfscope}%
\pgfpathrectangle{\pgfqpoint{0.100000in}{0.212622in}}{\pgfqpoint{3.696000in}{3.696000in}}%
\pgfusepath{clip}%
\pgfsetbuttcap%
\pgfsetroundjoin%
\definecolor{currentfill}{rgb}{0.121569,0.466667,0.705882}%
\pgfsetfillcolor{currentfill}%
\pgfsetfillopacity{0.656483}%
\pgfsetlinewidth{1.003750pt}%
\definecolor{currentstroke}{rgb}{0.121569,0.466667,0.705882}%
\pgfsetstrokecolor{currentstroke}%
\pgfsetstrokeopacity{0.656483}%
\pgfsetdash{}{0pt}%
\pgfpathmoveto{\pgfqpoint{2.360689in}{2.099323in}}%
\pgfpathcurveto{\pgfqpoint{2.368925in}{2.099323in}}{\pgfqpoint{2.376825in}{2.102595in}}{\pgfqpoint{2.382649in}{2.108419in}}%
\pgfpathcurveto{\pgfqpoint{2.388473in}{2.114243in}}{\pgfqpoint{2.391746in}{2.122143in}}{\pgfqpoint{2.391746in}{2.130380in}}%
\pgfpathcurveto{\pgfqpoint{2.391746in}{2.138616in}}{\pgfqpoint{2.388473in}{2.146516in}}{\pgfqpoint{2.382649in}{2.152340in}}%
\pgfpathcurveto{\pgfqpoint{2.376825in}{2.158164in}}{\pgfqpoint{2.368925in}{2.161436in}}{\pgfqpoint{2.360689in}{2.161436in}}%
\pgfpathcurveto{\pgfqpoint{2.352453in}{2.161436in}}{\pgfqpoint{2.344553in}{2.158164in}}{\pgfqpoint{2.338729in}{2.152340in}}%
\pgfpathcurveto{\pgfqpoint{2.332905in}{2.146516in}}{\pgfqpoint{2.329633in}{2.138616in}}{\pgfqpoint{2.329633in}{2.130380in}}%
\pgfpathcurveto{\pgfqpoint{2.329633in}{2.122143in}}{\pgfqpoint{2.332905in}{2.114243in}}{\pgfqpoint{2.338729in}{2.108419in}}%
\pgfpathcurveto{\pgfqpoint{2.344553in}{2.102595in}}{\pgfqpoint{2.352453in}{2.099323in}}{\pgfqpoint{2.360689in}{2.099323in}}%
\pgfpathclose%
\pgfusepath{stroke,fill}%
\end{pgfscope}%
\begin{pgfscope}%
\pgfpathrectangle{\pgfqpoint{0.100000in}{0.212622in}}{\pgfqpoint{3.696000in}{3.696000in}}%
\pgfusepath{clip}%
\pgfsetbuttcap%
\pgfsetroundjoin%
\definecolor{currentfill}{rgb}{0.121569,0.466667,0.705882}%
\pgfsetfillcolor{currentfill}%
\pgfsetfillopacity{0.660185}%
\pgfsetlinewidth{1.003750pt}%
\definecolor{currentstroke}{rgb}{0.121569,0.466667,0.705882}%
\pgfsetstrokecolor{currentstroke}%
\pgfsetstrokeopacity{0.660185}%
\pgfsetdash{}{0pt}%
\pgfpathmoveto{\pgfqpoint{1.184272in}{2.438151in}}%
\pgfpathcurveto{\pgfqpoint{1.192509in}{2.438151in}}{\pgfqpoint{1.200409in}{2.441423in}}{\pgfqpoint{1.206233in}{2.447247in}}%
\pgfpathcurveto{\pgfqpoint{1.212056in}{2.453071in}}{\pgfqpoint{1.215329in}{2.460971in}}{\pgfqpoint{1.215329in}{2.469208in}}%
\pgfpathcurveto{\pgfqpoint{1.215329in}{2.477444in}}{\pgfqpoint{1.212056in}{2.485344in}}{\pgfqpoint{1.206233in}{2.491168in}}%
\pgfpathcurveto{\pgfqpoint{1.200409in}{2.496992in}}{\pgfqpoint{1.192509in}{2.500264in}}{\pgfqpoint{1.184272in}{2.500264in}}%
\pgfpathcurveto{\pgfqpoint{1.176036in}{2.500264in}}{\pgfqpoint{1.168136in}{2.496992in}}{\pgfqpoint{1.162312in}{2.491168in}}%
\pgfpathcurveto{\pgfqpoint{1.156488in}{2.485344in}}{\pgfqpoint{1.153216in}{2.477444in}}{\pgfqpoint{1.153216in}{2.469208in}}%
\pgfpathcurveto{\pgfqpoint{1.153216in}{2.460971in}}{\pgfqpoint{1.156488in}{2.453071in}}{\pgfqpoint{1.162312in}{2.447247in}}%
\pgfpathcurveto{\pgfqpoint{1.168136in}{2.441423in}}{\pgfqpoint{1.176036in}{2.438151in}}{\pgfqpoint{1.184272in}{2.438151in}}%
\pgfpathclose%
\pgfusepath{stroke,fill}%
\end{pgfscope}%
\begin{pgfscope}%
\pgfpathrectangle{\pgfqpoint{0.100000in}{0.212622in}}{\pgfqpoint{3.696000in}{3.696000in}}%
\pgfusepath{clip}%
\pgfsetbuttcap%
\pgfsetroundjoin%
\definecolor{currentfill}{rgb}{0.121569,0.466667,0.705882}%
\pgfsetfillcolor{currentfill}%
\pgfsetfillopacity{0.666567}%
\pgfsetlinewidth{1.003750pt}%
\definecolor{currentstroke}{rgb}{0.121569,0.466667,0.705882}%
\pgfsetstrokecolor{currentstroke}%
\pgfsetstrokeopacity{0.666567}%
\pgfsetdash{}{0pt}%
\pgfpathmoveto{\pgfqpoint{1.181290in}{2.417409in}}%
\pgfpathcurveto{\pgfqpoint{1.189526in}{2.417409in}}{\pgfqpoint{1.197426in}{2.420681in}}{\pgfqpoint{1.203250in}{2.426505in}}%
\pgfpathcurveto{\pgfqpoint{1.209074in}{2.432329in}}{\pgfqpoint{1.212346in}{2.440229in}}{\pgfqpoint{1.212346in}{2.448465in}}%
\pgfpathcurveto{\pgfqpoint{1.212346in}{2.456701in}}{\pgfqpoint{1.209074in}{2.464601in}}{\pgfqpoint{1.203250in}{2.470425in}}%
\pgfpathcurveto{\pgfqpoint{1.197426in}{2.476249in}}{\pgfqpoint{1.189526in}{2.479522in}}{\pgfqpoint{1.181290in}{2.479522in}}%
\pgfpathcurveto{\pgfqpoint{1.173053in}{2.479522in}}{\pgfqpoint{1.165153in}{2.476249in}}{\pgfqpoint{1.159329in}{2.470425in}}%
\pgfpathcurveto{\pgfqpoint{1.153505in}{2.464601in}}{\pgfqpoint{1.150233in}{2.456701in}}{\pgfqpoint{1.150233in}{2.448465in}}%
\pgfpathcurveto{\pgfqpoint{1.150233in}{2.440229in}}{\pgfqpoint{1.153505in}{2.432329in}}{\pgfqpoint{1.159329in}{2.426505in}}%
\pgfpathcurveto{\pgfqpoint{1.165153in}{2.420681in}}{\pgfqpoint{1.173053in}{2.417409in}}{\pgfqpoint{1.181290in}{2.417409in}}%
\pgfpathclose%
\pgfusepath{stroke,fill}%
\end{pgfscope}%
\begin{pgfscope}%
\pgfpathrectangle{\pgfqpoint{0.100000in}{0.212622in}}{\pgfqpoint{3.696000in}{3.696000in}}%
\pgfusepath{clip}%
\pgfsetbuttcap%
\pgfsetroundjoin%
\definecolor{currentfill}{rgb}{0.121569,0.466667,0.705882}%
\pgfsetfillcolor{currentfill}%
\pgfsetfillopacity{0.667787}%
\pgfsetlinewidth{1.003750pt}%
\definecolor{currentstroke}{rgb}{0.121569,0.466667,0.705882}%
\pgfsetstrokecolor{currentstroke}%
\pgfsetstrokeopacity{0.667787}%
\pgfsetdash{}{0pt}%
\pgfpathmoveto{\pgfqpoint{2.375196in}{2.056749in}}%
\pgfpathcurveto{\pgfqpoint{2.383432in}{2.056749in}}{\pgfqpoint{2.391332in}{2.060021in}}{\pgfqpoint{2.397156in}{2.065845in}}%
\pgfpathcurveto{\pgfqpoint{2.402980in}{2.071669in}}{\pgfqpoint{2.406252in}{2.079569in}}{\pgfqpoint{2.406252in}{2.087806in}}%
\pgfpathcurveto{\pgfqpoint{2.406252in}{2.096042in}}{\pgfqpoint{2.402980in}{2.103942in}}{\pgfqpoint{2.397156in}{2.109766in}}%
\pgfpathcurveto{\pgfqpoint{2.391332in}{2.115590in}}{\pgfqpoint{2.383432in}{2.118862in}}{\pgfqpoint{2.375196in}{2.118862in}}%
\pgfpathcurveto{\pgfqpoint{2.366960in}{2.118862in}}{\pgfqpoint{2.359060in}{2.115590in}}{\pgfqpoint{2.353236in}{2.109766in}}%
\pgfpathcurveto{\pgfqpoint{2.347412in}{2.103942in}}{\pgfqpoint{2.344139in}{2.096042in}}{\pgfqpoint{2.344139in}{2.087806in}}%
\pgfpathcurveto{\pgfqpoint{2.344139in}{2.079569in}}{\pgfqpoint{2.347412in}{2.071669in}}{\pgfqpoint{2.353236in}{2.065845in}}%
\pgfpathcurveto{\pgfqpoint{2.359060in}{2.060021in}}{\pgfqpoint{2.366960in}{2.056749in}}{\pgfqpoint{2.375196in}{2.056749in}}%
\pgfpathclose%
\pgfusepath{stroke,fill}%
\end{pgfscope}%
\begin{pgfscope}%
\pgfpathrectangle{\pgfqpoint{0.100000in}{0.212622in}}{\pgfqpoint{3.696000in}{3.696000in}}%
\pgfusepath{clip}%
\pgfsetbuttcap%
\pgfsetroundjoin%
\definecolor{currentfill}{rgb}{0.121569,0.466667,0.705882}%
\pgfsetfillcolor{currentfill}%
\pgfsetfillopacity{0.671717}%
\pgfsetlinewidth{1.003750pt}%
\definecolor{currentstroke}{rgb}{0.121569,0.466667,0.705882}%
\pgfsetstrokecolor{currentstroke}%
\pgfsetstrokeopacity{0.671717}%
\pgfsetdash{}{0pt}%
\pgfpathmoveto{\pgfqpoint{1.175354in}{2.400810in}}%
\pgfpathcurveto{\pgfqpoint{1.183590in}{2.400810in}}{\pgfqpoint{1.191490in}{2.404082in}}{\pgfqpoint{1.197314in}{2.409906in}}%
\pgfpathcurveto{\pgfqpoint{1.203138in}{2.415730in}}{\pgfqpoint{1.206410in}{2.423630in}}{\pgfqpoint{1.206410in}{2.431866in}}%
\pgfpathcurveto{\pgfqpoint{1.206410in}{2.440102in}}{\pgfqpoint{1.203138in}{2.448002in}}{\pgfqpoint{1.197314in}{2.453826in}}%
\pgfpathcurveto{\pgfqpoint{1.191490in}{2.459650in}}{\pgfqpoint{1.183590in}{2.462923in}}{\pgfqpoint{1.175354in}{2.462923in}}%
\pgfpathcurveto{\pgfqpoint{1.167118in}{2.462923in}}{\pgfqpoint{1.159218in}{2.459650in}}{\pgfqpoint{1.153394in}{2.453826in}}%
\pgfpathcurveto{\pgfqpoint{1.147570in}{2.448002in}}{\pgfqpoint{1.144297in}{2.440102in}}{\pgfqpoint{1.144297in}{2.431866in}}%
\pgfpathcurveto{\pgfqpoint{1.144297in}{2.423630in}}{\pgfqpoint{1.147570in}{2.415730in}}{\pgfqpoint{1.153394in}{2.409906in}}%
\pgfpathcurveto{\pgfqpoint{1.159218in}{2.404082in}}{\pgfqpoint{1.167118in}{2.400810in}}{\pgfqpoint{1.175354in}{2.400810in}}%
\pgfpathclose%
\pgfusepath{stroke,fill}%
\end{pgfscope}%
\begin{pgfscope}%
\pgfpathrectangle{\pgfqpoint{0.100000in}{0.212622in}}{\pgfqpoint{3.696000in}{3.696000in}}%
\pgfusepath{clip}%
\pgfsetbuttcap%
\pgfsetroundjoin%
\definecolor{currentfill}{rgb}{0.121569,0.466667,0.705882}%
\pgfsetfillcolor{currentfill}%
\pgfsetfillopacity{0.674300}%
\pgfsetlinewidth{1.003750pt}%
\definecolor{currentstroke}{rgb}{0.121569,0.466667,0.705882}%
\pgfsetstrokecolor{currentstroke}%
\pgfsetstrokeopacity{0.674300}%
\pgfsetdash{}{0pt}%
\pgfpathmoveto{\pgfqpoint{2.381812in}{2.033754in}}%
\pgfpathcurveto{\pgfqpoint{2.390048in}{2.033754in}}{\pgfqpoint{2.397948in}{2.037027in}}{\pgfqpoint{2.403772in}{2.042851in}}%
\pgfpathcurveto{\pgfqpoint{2.409596in}{2.048675in}}{\pgfqpoint{2.412868in}{2.056575in}}{\pgfqpoint{2.412868in}{2.064811in}}%
\pgfpathcurveto{\pgfqpoint{2.412868in}{2.073047in}}{\pgfqpoint{2.409596in}{2.080947in}}{\pgfqpoint{2.403772in}{2.086771in}}%
\pgfpathcurveto{\pgfqpoint{2.397948in}{2.092595in}}{\pgfqpoint{2.390048in}{2.095867in}}{\pgfqpoint{2.381812in}{2.095867in}}%
\pgfpathcurveto{\pgfqpoint{2.373575in}{2.095867in}}{\pgfqpoint{2.365675in}{2.092595in}}{\pgfqpoint{2.359851in}{2.086771in}}%
\pgfpathcurveto{\pgfqpoint{2.354027in}{2.080947in}}{\pgfqpoint{2.350755in}{2.073047in}}{\pgfqpoint{2.350755in}{2.064811in}}%
\pgfpathcurveto{\pgfqpoint{2.350755in}{2.056575in}}{\pgfqpoint{2.354027in}{2.048675in}}{\pgfqpoint{2.359851in}{2.042851in}}%
\pgfpathcurveto{\pgfqpoint{2.365675in}{2.037027in}}{\pgfqpoint{2.373575in}{2.033754in}}{\pgfqpoint{2.381812in}{2.033754in}}%
\pgfpathclose%
\pgfusepath{stroke,fill}%
\end{pgfscope}%
\begin{pgfscope}%
\pgfpathrectangle{\pgfqpoint{0.100000in}{0.212622in}}{\pgfqpoint{3.696000in}{3.696000in}}%
\pgfusepath{clip}%
\pgfsetbuttcap%
\pgfsetroundjoin%
\definecolor{currentfill}{rgb}{0.121569,0.466667,0.705882}%
\pgfsetfillcolor{currentfill}%
\pgfsetfillopacity{0.675969}%
\pgfsetlinewidth{1.003750pt}%
\definecolor{currentstroke}{rgb}{0.121569,0.466667,0.705882}%
\pgfsetstrokecolor{currentstroke}%
\pgfsetstrokeopacity{0.675969}%
\pgfsetdash{}{0pt}%
\pgfpathmoveto{\pgfqpoint{1.173132in}{2.386168in}}%
\pgfpathcurveto{\pgfqpoint{1.181368in}{2.386168in}}{\pgfqpoint{1.189268in}{2.389441in}}{\pgfqpoint{1.195092in}{2.395265in}}%
\pgfpathcurveto{\pgfqpoint{1.200916in}{2.401088in}}{\pgfqpoint{1.204189in}{2.408989in}}{\pgfqpoint{1.204189in}{2.417225in}}%
\pgfpathcurveto{\pgfqpoint{1.204189in}{2.425461in}}{\pgfqpoint{1.200916in}{2.433361in}}{\pgfqpoint{1.195092in}{2.439185in}}%
\pgfpathcurveto{\pgfqpoint{1.189268in}{2.445009in}}{\pgfqpoint{1.181368in}{2.448281in}}{\pgfqpoint{1.173132in}{2.448281in}}%
\pgfpathcurveto{\pgfqpoint{1.164896in}{2.448281in}}{\pgfqpoint{1.156996in}{2.445009in}}{\pgfqpoint{1.151172in}{2.439185in}}%
\pgfpathcurveto{\pgfqpoint{1.145348in}{2.433361in}}{\pgfqpoint{1.142076in}{2.425461in}}{\pgfqpoint{1.142076in}{2.417225in}}%
\pgfpathcurveto{\pgfqpoint{1.142076in}{2.408989in}}{\pgfqpoint{1.145348in}{2.401088in}}{\pgfqpoint{1.151172in}{2.395265in}}%
\pgfpathcurveto{\pgfqpoint{1.156996in}{2.389441in}}{\pgfqpoint{1.164896in}{2.386168in}}{\pgfqpoint{1.173132in}{2.386168in}}%
\pgfpathclose%
\pgfusepath{stroke,fill}%
\end{pgfscope}%
\begin{pgfscope}%
\pgfpathrectangle{\pgfqpoint{0.100000in}{0.212622in}}{\pgfqpoint{3.696000in}{3.696000in}}%
\pgfusepath{clip}%
\pgfsetbuttcap%
\pgfsetroundjoin%
\definecolor{currentfill}{rgb}{0.121569,0.466667,0.705882}%
\pgfsetfillcolor{currentfill}%
\pgfsetfillopacity{0.679645}%
\pgfsetlinewidth{1.003750pt}%
\definecolor{currentstroke}{rgb}{0.121569,0.466667,0.705882}%
\pgfsetstrokecolor{currentstroke}%
\pgfsetstrokeopacity{0.679645}%
\pgfsetdash{}{0pt}%
\pgfpathmoveto{\pgfqpoint{1.169334in}{2.375719in}}%
\pgfpathcurveto{\pgfqpoint{1.177571in}{2.375719in}}{\pgfqpoint{1.185471in}{2.378991in}}{\pgfqpoint{1.191295in}{2.384815in}}%
\pgfpathcurveto{\pgfqpoint{1.197119in}{2.390639in}}{\pgfqpoint{1.200391in}{2.398539in}}{\pgfqpoint{1.200391in}{2.406775in}}%
\pgfpathcurveto{\pgfqpoint{1.200391in}{2.415012in}}{\pgfqpoint{1.197119in}{2.422912in}}{\pgfqpoint{1.191295in}{2.428736in}}%
\pgfpathcurveto{\pgfqpoint{1.185471in}{2.434560in}}{\pgfqpoint{1.177571in}{2.437832in}}{\pgfqpoint{1.169334in}{2.437832in}}%
\pgfpathcurveto{\pgfqpoint{1.161098in}{2.437832in}}{\pgfqpoint{1.153198in}{2.434560in}}{\pgfqpoint{1.147374in}{2.428736in}}%
\pgfpathcurveto{\pgfqpoint{1.141550in}{2.422912in}}{\pgfqpoint{1.138278in}{2.415012in}}{\pgfqpoint{1.138278in}{2.406775in}}%
\pgfpathcurveto{\pgfqpoint{1.138278in}{2.398539in}}{\pgfqpoint{1.141550in}{2.390639in}}{\pgfqpoint{1.147374in}{2.384815in}}%
\pgfpathcurveto{\pgfqpoint{1.153198in}{2.378991in}}{\pgfqpoint{1.161098in}{2.375719in}}{\pgfqpoint{1.169334in}{2.375719in}}%
\pgfpathclose%
\pgfusepath{stroke,fill}%
\end{pgfscope}%
\begin{pgfscope}%
\pgfpathrectangle{\pgfqpoint{0.100000in}{0.212622in}}{\pgfqpoint{3.696000in}{3.696000in}}%
\pgfusepath{clip}%
\pgfsetbuttcap%
\pgfsetroundjoin%
\definecolor{currentfill}{rgb}{0.121569,0.466667,0.705882}%
\pgfsetfillcolor{currentfill}%
\pgfsetfillopacity{0.681834}%
\pgfsetlinewidth{1.003750pt}%
\definecolor{currentstroke}{rgb}{0.121569,0.466667,0.705882}%
\pgfsetstrokecolor{currentstroke}%
\pgfsetstrokeopacity{0.681834}%
\pgfsetdash{}{0pt}%
\pgfpathmoveto{\pgfqpoint{2.390723in}{2.005958in}}%
\pgfpathcurveto{\pgfqpoint{2.398959in}{2.005958in}}{\pgfqpoint{2.406859in}{2.009231in}}{\pgfqpoint{2.412683in}{2.015055in}}%
\pgfpathcurveto{\pgfqpoint{2.418507in}{2.020878in}}{\pgfqpoint{2.421780in}{2.028779in}}{\pgfqpoint{2.421780in}{2.037015in}}%
\pgfpathcurveto{\pgfqpoint{2.421780in}{2.045251in}}{\pgfqpoint{2.418507in}{2.053151in}}{\pgfqpoint{2.412683in}{2.058975in}}%
\pgfpathcurveto{\pgfqpoint{2.406859in}{2.064799in}}{\pgfqpoint{2.398959in}{2.068071in}}{\pgfqpoint{2.390723in}{2.068071in}}%
\pgfpathcurveto{\pgfqpoint{2.382487in}{2.068071in}}{\pgfqpoint{2.374587in}{2.064799in}}{\pgfqpoint{2.368763in}{2.058975in}}%
\pgfpathcurveto{\pgfqpoint{2.362939in}{2.053151in}}{\pgfqpoint{2.359667in}{2.045251in}}{\pgfqpoint{2.359667in}{2.037015in}}%
\pgfpathcurveto{\pgfqpoint{2.359667in}{2.028779in}}{\pgfqpoint{2.362939in}{2.020878in}}{\pgfqpoint{2.368763in}{2.015055in}}%
\pgfpathcurveto{\pgfqpoint{2.374587in}{2.009231in}}{\pgfqpoint{2.382487in}{2.005958in}}{\pgfqpoint{2.390723in}{2.005958in}}%
\pgfpathclose%
\pgfusepath{stroke,fill}%
\end{pgfscope}%
\begin{pgfscope}%
\pgfpathrectangle{\pgfqpoint{0.100000in}{0.212622in}}{\pgfqpoint{3.696000in}{3.696000in}}%
\pgfusepath{clip}%
\pgfsetbuttcap%
\pgfsetroundjoin%
\definecolor{currentfill}{rgb}{0.121569,0.466667,0.705882}%
\pgfsetfillcolor{currentfill}%
\pgfsetfillopacity{0.682185}%
\pgfsetlinewidth{1.003750pt}%
\definecolor{currentstroke}{rgb}{0.121569,0.466667,0.705882}%
\pgfsetstrokecolor{currentstroke}%
\pgfsetstrokeopacity{0.682185}%
\pgfsetdash{}{0pt}%
\pgfpathmoveto{\pgfqpoint{1.168621in}{2.368141in}}%
\pgfpathcurveto{\pgfqpoint{1.176858in}{2.368141in}}{\pgfqpoint{1.184758in}{2.371413in}}{\pgfqpoint{1.190582in}{2.377237in}}%
\pgfpathcurveto{\pgfqpoint{1.196406in}{2.383061in}}{\pgfqpoint{1.199678in}{2.390961in}}{\pgfqpoint{1.199678in}{2.399197in}}%
\pgfpathcurveto{\pgfqpoint{1.199678in}{2.407434in}}{\pgfqpoint{1.196406in}{2.415334in}}{\pgfqpoint{1.190582in}{2.421158in}}%
\pgfpathcurveto{\pgfqpoint{1.184758in}{2.426982in}}{\pgfqpoint{1.176858in}{2.430254in}}{\pgfqpoint{1.168621in}{2.430254in}}%
\pgfpathcurveto{\pgfqpoint{1.160385in}{2.430254in}}{\pgfqpoint{1.152485in}{2.426982in}}{\pgfqpoint{1.146661in}{2.421158in}}%
\pgfpathcurveto{\pgfqpoint{1.140837in}{2.415334in}}{\pgfqpoint{1.137565in}{2.407434in}}{\pgfqpoint{1.137565in}{2.399197in}}%
\pgfpathcurveto{\pgfqpoint{1.137565in}{2.390961in}}{\pgfqpoint{1.140837in}{2.383061in}}{\pgfqpoint{1.146661in}{2.377237in}}%
\pgfpathcurveto{\pgfqpoint{1.152485in}{2.371413in}}{\pgfqpoint{1.160385in}{2.368141in}}{\pgfqpoint{1.168621in}{2.368141in}}%
\pgfpathclose%
\pgfusepath{stroke,fill}%
\end{pgfscope}%
\begin{pgfscope}%
\pgfpathrectangle{\pgfqpoint{0.100000in}{0.212622in}}{\pgfqpoint{3.696000in}{3.696000in}}%
\pgfusepath{clip}%
\pgfsetbuttcap%
\pgfsetroundjoin%
\definecolor{currentfill}{rgb}{0.121569,0.466667,0.705882}%
\pgfsetfillcolor{currentfill}%
\pgfsetfillopacity{0.686140}%
\pgfsetlinewidth{1.003750pt}%
\definecolor{currentstroke}{rgb}{0.121569,0.466667,0.705882}%
\pgfsetstrokecolor{currentstroke}%
\pgfsetstrokeopacity{0.686140}%
\pgfsetdash{}{0pt}%
\pgfpathmoveto{\pgfqpoint{2.394896in}{1.990948in}}%
\pgfpathcurveto{\pgfqpoint{2.403132in}{1.990948in}}{\pgfqpoint{2.411032in}{1.994220in}}{\pgfqpoint{2.416856in}{2.000044in}}%
\pgfpathcurveto{\pgfqpoint{2.422680in}{2.005868in}}{\pgfqpoint{2.425953in}{2.013768in}}{\pgfqpoint{2.425953in}{2.022004in}}%
\pgfpathcurveto{\pgfqpoint{2.425953in}{2.030240in}}{\pgfqpoint{2.422680in}{2.038140in}}{\pgfqpoint{2.416856in}{2.043964in}}%
\pgfpathcurveto{\pgfqpoint{2.411032in}{2.049788in}}{\pgfqpoint{2.403132in}{2.053061in}}{\pgfqpoint{2.394896in}{2.053061in}}%
\pgfpathcurveto{\pgfqpoint{2.386660in}{2.053061in}}{\pgfqpoint{2.378760in}{2.049788in}}{\pgfqpoint{2.372936in}{2.043964in}}%
\pgfpathcurveto{\pgfqpoint{2.367112in}{2.038140in}}{\pgfqpoint{2.363840in}{2.030240in}}{\pgfqpoint{2.363840in}{2.022004in}}%
\pgfpathcurveto{\pgfqpoint{2.363840in}{2.013768in}}{\pgfqpoint{2.367112in}{2.005868in}}{\pgfqpoint{2.372936in}{2.000044in}}%
\pgfpathcurveto{\pgfqpoint{2.378760in}{1.994220in}}{\pgfqpoint{2.386660in}{1.990948in}}{\pgfqpoint{2.394896in}{1.990948in}}%
\pgfpathclose%
\pgfusepath{stroke,fill}%
\end{pgfscope}%
\begin{pgfscope}%
\pgfpathrectangle{\pgfqpoint{0.100000in}{0.212622in}}{\pgfqpoint{3.696000in}{3.696000in}}%
\pgfusepath{clip}%
\pgfsetbuttcap%
\pgfsetroundjoin%
\definecolor{currentfill}{rgb}{0.121569,0.466667,0.705882}%
\pgfsetfillcolor{currentfill}%
\pgfsetfillopacity{0.686711}%
\pgfsetlinewidth{1.003750pt}%
\definecolor{currentstroke}{rgb}{0.121569,0.466667,0.705882}%
\pgfsetstrokecolor{currentstroke}%
\pgfsetstrokeopacity{0.686711}%
\pgfsetdash{}{0pt}%
\pgfpathmoveto{\pgfqpoint{1.164194in}{2.355223in}}%
\pgfpathcurveto{\pgfqpoint{1.172430in}{2.355223in}}{\pgfqpoint{1.180330in}{2.358496in}}{\pgfqpoint{1.186154in}{2.364319in}}%
\pgfpathcurveto{\pgfqpoint{1.191978in}{2.370143in}}{\pgfqpoint{1.195250in}{2.378043in}}{\pgfqpoint{1.195250in}{2.386280in}}%
\pgfpathcurveto{\pgfqpoint{1.195250in}{2.394516in}}{\pgfqpoint{1.191978in}{2.402416in}}{\pgfqpoint{1.186154in}{2.408240in}}%
\pgfpathcurveto{\pgfqpoint{1.180330in}{2.414064in}}{\pgfqpoint{1.172430in}{2.417336in}}{\pgfqpoint{1.164194in}{2.417336in}}%
\pgfpathcurveto{\pgfqpoint{1.155957in}{2.417336in}}{\pgfqpoint{1.148057in}{2.414064in}}{\pgfqpoint{1.142233in}{2.408240in}}%
\pgfpathcurveto{\pgfqpoint{1.136409in}{2.402416in}}{\pgfqpoint{1.133137in}{2.394516in}}{\pgfqpoint{1.133137in}{2.386280in}}%
\pgfpathcurveto{\pgfqpoint{1.133137in}{2.378043in}}{\pgfqpoint{1.136409in}{2.370143in}}{\pgfqpoint{1.142233in}{2.364319in}}%
\pgfpathcurveto{\pgfqpoint{1.148057in}{2.358496in}}{\pgfqpoint{1.155957in}{2.355223in}}{\pgfqpoint{1.164194in}{2.355223in}}%
\pgfpathclose%
\pgfusepath{stroke,fill}%
\end{pgfscope}%
\begin{pgfscope}%
\pgfpathrectangle{\pgfqpoint{0.100000in}{0.212622in}}{\pgfqpoint{3.696000in}{3.696000in}}%
\pgfusepath{clip}%
\pgfsetbuttcap%
\pgfsetroundjoin%
\definecolor{currentfill}{rgb}{0.121569,0.466667,0.705882}%
\pgfsetfillcolor{currentfill}%
\pgfsetfillopacity{0.690648}%
\pgfsetlinewidth{1.003750pt}%
\definecolor{currentstroke}{rgb}{0.121569,0.466667,0.705882}%
\pgfsetstrokecolor{currentstroke}%
\pgfsetstrokeopacity{0.690648}%
\pgfsetdash{}{0pt}%
\pgfpathmoveto{\pgfqpoint{1.163543in}{2.343452in}}%
\pgfpathcurveto{\pgfqpoint{1.171779in}{2.343452in}}{\pgfqpoint{1.179679in}{2.346724in}}{\pgfqpoint{1.185503in}{2.352548in}}%
\pgfpathcurveto{\pgfqpoint{1.191327in}{2.358372in}}{\pgfqpoint{1.194600in}{2.366272in}}{\pgfqpoint{1.194600in}{2.374509in}}%
\pgfpathcurveto{\pgfqpoint{1.194600in}{2.382745in}}{\pgfqpoint{1.191327in}{2.390645in}}{\pgfqpoint{1.185503in}{2.396469in}}%
\pgfpathcurveto{\pgfqpoint{1.179679in}{2.402293in}}{\pgfqpoint{1.171779in}{2.405565in}}{\pgfqpoint{1.163543in}{2.405565in}}%
\pgfpathcurveto{\pgfqpoint{1.155307in}{2.405565in}}{\pgfqpoint{1.147407in}{2.402293in}}{\pgfqpoint{1.141583in}{2.396469in}}%
\pgfpathcurveto{\pgfqpoint{1.135759in}{2.390645in}}{\pgfqpoint{1.132487in}{2.382745in}}{\pgfqpoint{1.132487in}{2.374509in}}%
\pgfpathcurveto{\pgfqpoint{1.132487in}{2.366272in}}{\pgfqpoint{1.135759in}{2.358372in}}{\pgfqpoint{1.141583in}{2.352548in}}%
\pgfpathcurveto{\pgfqpoint{1.147407in}{2.346724in}}{\pgfqpoint{1.155307in}{2.343452in}}{\pgfqpoint{1.163543in}{2.343452in}}%
\pgfpathclose%
\pgfusepath{stroke,fill}%
\end{pgfscope}%
\begin{pgfscope}%
\pgfpathrectangle{\pgfqpoint{0.100000in}{0.212622in}}{\pgfqpoint{3.696000in}{3.696000in}}%
\pgfusepath{clip}%
\pgfsetbuttcap%
\pgfsetroundjoin%
\definecolor{currentfill}{rgb}{0.121569,0.466667,0.705882}%
\pgfsetfillcolor{currentfill}%
\pgfsetfillopacity{0.691737}%
\pgfsetlinewidth{1.003750pt}%
\definecolor{currentstroke}{rgb}{0.121569,0.466667,0.705882}%
\pgfsetstrokecolor{currentstroke}%
\pgfsetstrokeopacity{0.691737}%
\pgfsetdash{}{0pt}%
\pgfpathmoveto{\pgfqpoint{2.401721in}{1.970304in}}%
\pgfpathcurveto{\pgfqpoint{2.409957in}{1.970304in}}{\pgfqpoint{2.417857in}{1.973576in}}{\pgfqpoint{2.423681in}{1.979400in}}%
\pgfpathcurveto{\pgfqpoint{2.429505in}{1.985224in}}{\pgfqpoint{2.432777in}{1.993124in}}{\pgfqpoint{2.432777in}{2.001361in}}%
\pgfpathcurveto{\pgfqpoint{2.432777in}{2.009597in}}{\pgfqpoint{2.429505in}{2.017497in}}{\pgfqpoint{2.423681in}{2.023321in}}%
\pgfpathcurveto{\pgfqpoint{2.417857in}{2.029145in}}{\pgfqpoint{2.409957in}{2.032417in}}{\pgfqpoint{2.401721in}{2.032417in}}%
\pgfpathcurveto{\pgfqpoint{2.393485in}{2.032417in}}{\pgfqpoint{2.385585in}{2.029145in}}{\pgfqpoint{2.379761in}{2.023321in}}%
\pgfpathcurveto{\pgfqpoint{2.373937in}{2.017497in}}{\pgfqpoint{2.370664in}{2.009597in}}{\pgfqpoint{2.370664in}{2.001361in}}%
\pgfpathcurveto{\pgfqpoint{2.370664in}{1.993124in}}{\pgfqpoint{2.373937in}{1.985224in}}{\pgfqpoint{2.379761in}{1.979400in}}%
\pgfpathcurveto{\pgfqpoint{2.385585in}{1.973576in}}{\pgfqpoint{2.393485in}{1.970304in}}{\pgfqpoint{2.401721in}{1.970304in}}%
\pgfpathclose%
\pgfusepath{stroke,fill}%
\end{pgfscope}%
\begin{pgfscope}%
\pgfpathrectangle{\pgfqpoint{0.100000in}{0.212622in}}{\pgfqpoint{3.696000in}{3.696000in}}%
\pgfusepath{clip}%
\pgfsetbuttcap%
\pgfsetroundjoin%
\definecolor{currentfill}{rgb}{0.121569,0.466667,0.705882}%
\pgfsetfillcolor{currentfill}%
\pgfsetfillopacity{0.694047}%
\pgfsetlinewidth{1.003750pt}%
\definecolor{currentstroke}{rgb}{0.121569,0.466667,0.705882}%
\pgfsetstrokecolor{currentstroke}%
\pgfsetstrokeopacity{0.694047}%
\pgfsetdash{}{0pt}%
\pgfpathmoveto{\pgfqpoint{1.160824in}{2.334198in}}%
\pgfpathcurveto{\pgfqpoint{1.169060in}{2.334198in}}{\pgfqpoint{1.176960in}{2.337470in}}{\pgfqpoint{1.182784in}{2.343294in}}%
\pgfpathcurveto{\pgfqpoint{1.188608in}{2.349118in}}{\pgfqpoint{1.191880in}{2.357018in}}{\pgfqpoint{1.191880in}{2.365254in}}%
\pgfpathcurveto{\pgfqpoint{1.191880in}{2.373490in}}{\pgfqpoint{1.188608in}{2.381390in}}{\pgfqpoint{1.182784in}{2.387214in}}%
\pgfpathcurveto{\pgfqpoint{1.176960in}{2.393038in}}{\pgfqpoint{1.169060in}{2.396311in}}{\pgfqpoint{1.160824in}{2.396311in}}%
\pgfpathcurveto{\pgfqpoint{1.152587in}{2.396311in}}{\pgfqpoint{1.144687in}{2.393038in}}{\pgfqpoint{1.138863in}{2.387214in}}%
\pgfpathcurveto{\pgfqpoint{1.133040in}{2.381390in}}{\pgfqpoint{1.129767in}{2.373490in}}{\pgfqpoint{1.129767in}{2.365254in}}%
\pgfpathcurveto{\pgfqpoint{1.129767in}{2.357018in}}{\pgfqpoint{1.133040in}{2.349118in}}{\pgfqpoint{1.138863in}{2.343294in}}%
\pgfpathcurveto{\pgfqpoint{1.144687in}{2.337470in}}{\pgfqpoint{1.152587in}{2.334198in}}{\pgfqpoint{1.160824in}{2.334198in}}%
\pgfpathclose%
\pgfusepath{stroke,fill}%
\end{pgfscope}%
\begin{pgfscope}%
\pgfpathrectangle{\pgfqpoint{0.100000in}{0.212622in}}{\pgfqpoint{3.696000in}{3.696000in}}%
\pgfusepath{clip}%
\pgfsetbuttcap%
\pgfsetroundjoin%
\definecolor{currentfill}{rgb}{0.121569,0.466667,0.705882}%
\pgfsetfillcolor{currentfill}%
\pgfsetfillopacity{0.698629}%
\pgfsetlinewidth{1.003750pt}%
\definecolor{currentstroke}{rgb}{0.121569,0.466667,0.705882}%
\pgfsetstrokecolor{currentstroke}%
\pgfsetstrokeopacity{0.698629}%
\pgfsetdash{}{0pt}%
\pgfpathmoveto{\pgfqpoint{2.408193in}{1.946032in}}%
\pgfpathcurveto{\pgfqpoint{2.416430in}{1.946032in}}{\pgfqpoint{2.424330in}{1.949304in}}{\pgfqpoint{2.430154in}{1.955128in}}%
\pgfpathcurveto{\pgfqpoint{2.435978in}{1.960952in}}{\pgfqpoint{2.439250in}{1.968852in}}{\pgfqpoint{2.439250in}{1.977088in}}%
\pgfpathcurveto{\pgfqpoint{2.439250in}{1.985325in}}{\pgfqpoint{2.435978in}{1.993225in}}{\pgfqpoint{2.430154in}{1.999049in}}%
\pgfpathcurveto{\pgfqpoint{2.424330in}{2.004873in}}{\pgfqpoint{2.416430in}{2.008145in}}{\pgfqpoint{2.408193in}{2.008145in}}%
\pgfpathcurveto{\pgfqpoint{2.399957in}{2.008145in}}{\pgfqpoint{2.392057in}{2.004873in}}{\pgfqpoint{2.386233in}{1.999049in}}%
\pgfpathcurveto{\pgfqpoint{2.380409in}{1.993225in}}{\pgfqpoint{2.377137in}{1.985325in}}{\pgfqpoint{2.377137in}{1.977088in}}%
\pgfpathcurveto{\pgfqpoint{2.377137in}{1.968852in}}{\pgfqpoint{2.380409in}{1.960952in}}{\pgfqpoint{2.386233in}{1.955128in}}%
\pgfpathcurveto{\pgfqpoint{2.392057in}{1.949304in}}{\pgfqpoint{2.399957in}{1.946032in}}{\pgfqpoint{2.408193in}{1.946032in}}%
\pgfpathclose%
\pgfusepath{stroke,fill}%
\end{pgfscope}%
\begin{pgfscope}%
\pgfpathrectangle{\pgfqpoint{0.100000in}{0.212622in}}{\pgfqpoint{3.696000in}{3.696000in}}%
\pgfusepath{clip}%
\pgfsetbuttcap%
\pgfsetroundjoin%
\definecolor{currentfill}{rgb}{0.121569,0.466667,0.705882}%
\pgfsetfillcolor{currentfill}%
\pgfsetfillopacity{0.700236}%
\pgfsetlinewidth{1.003750pt}%
\definecolor{currentstroke}{rgb}{0.121569,0.466667,0.705882}%
\pgfsetstrokecolor{currentstroke}%
\pgfsetstrokeopacity{0.700236}%
\pgfsetdash{}{0pt}%
\pgfpathmoveto{\pgfqpoint{1.157617in}{2.316633in}}%
\pgfpathcurveto{\pgfqpoint{1.165853in}{2.316633in}}{\pgfqpoint{1.173753in}{2.319905in}}{\pgfqpoint{1.179577in}{2.325729in}}%
\pgfpathcurveto{\pgfqpoint{1.185401in}{2.331553in}}{\pgfqpoint{1.188673in}{2.339453in}}{\pgfqpoint{1.188673in}{2.347690in}}%
\pgfpathcurveto{\pgfqpoint{1.188673in}{2.355926in}}{\pgfqpoint{1.185401in}{2.363826in}}{\pgfqpoint{1.179577in}{2.369650in}}%
\pgfpathcurveto{\pgfqpoint{1.173753in}{2.375474in}}{\pgfqpoint{1.165853in}{2.378746in}}{\pgfqpoint{1.157617in}{2.378746in}}%
\pgfpathcurveto{\pgfqpoint{1.149380in}{2.378746in}}{\pgfqpoint{1.141480in}{2.375474in}}{\pgfqpoint{1.135656in}{2.369650in}}%
\pgfpathcurveto{\pgfqpoint{1.129832in}{2.363826in}}{\pgfqpoint{1.126560in}{2.355926in}}{\pgfqpoint{1.126560in}{2.347690in}}%
\pgfpathcurveto{\pgfqpoint{1.126560in}{2.339453in}}{\pgfqpoint{1.129832in}{2.331553in}}{\pgfqpoint{1.135656in}{2.325729in}}%
\pgfpathcurveto{\pgfqpoint{1.141480in}{2.319905in}}{\pgfqpoint{1.149380in}{2.316633in}}{\pgfqpoint{1.157617in}{2.316633in}}%
\pgfpathclose%
\pgfusepath{stroke,fill}%
\end{pgfscope}%
\begin{pgfscope}%
\pgfpathrectangle{\pgfqpoint{0.100000in}{0.212622in}}{\pgfqpoint{3.696000in}{3.696000in}}%
\pgfusepath{clip}%
\pgfsetbuttcap%
\pgfsetroundjoin%
\definecolor{currentfill}{rgb}{0.121569,0.466667,0.705882}%
\pgfsetfillcolor{currentfill}%
\pgfsetfillopacity{0.706477}%
\pgfsetlinewidth{1.003750pt}%
\definecolor{currentstroke}{rgb}{0.121569,0.466667,0.705882}%
\pgfsetstrokecolor{currentstroke}%
\pgfsetstrokeopacity{0.706477}%
\pgfsetdash{}{0pt}%
\pgfpathmoveto{\pgfqpoint{2.416906in}{1.917430in}}%
\pgfpathcurveto{\pgfqpoint{2.425142in}{1.917430in}}{\pgfqpoint{2.433042in}{1.920703in}}{\pgfqpoint{2.438866in}{1.926526in}}%
\pgfpathcurveto{\pgfqpoint{2.444690in}{1.932350in}}{\pgfqpoint{2.447962in}{1.940250in}}{\pgfqpoint{2.447962in}{1.948487in}}%
\pgfpathcurveto{\pgfqpoint{2.447962in}{1.956723in}}{\pgfqpoint{2.444690in}{1.964623in}}{\pgfqpoint{2.438866in}{1.970447in}}%
\pgfpathcurveto{\pgfqpoint{2.433042in}{1.976271in}}{\pgfqpoint{2.425142in}{1.979543in}}{\pgfqpoint{2.416906in}{1.979543in}}%
\pgfpathcurveto{\pgfqpoint{2.408669in}{1.979543in}}{\pgfqpoint{2.400769in}{1.976271in}}{\pgfqpoint{2.394945in}{1.970447in}}%
\pgfpathcurveto{\pgfqpoint{2.389121in}{1.964623in}}{\pgfqpoint{2.385849in}{1.956723in}}{\pgfqpoint{2.385849in}{1.948487in}}%
\pgfpathcurveto{\pgfqpoint{2.385849in}{1.940250in}}{\pgfqpoint{2.389121in}{1.932350in}}{\pgfqpoint{2.394945in}{1.926526in}}%
\pgfpathcurveto{\pgfqpoint{2.400769in}{1.920703in}}{\pgfqpoint{2.408669in}{1.917430in}}{\pgfqpoint{2.416906in}{1.917430in}}%
\pgfpathclose%
\pgfusepath{stroke,fill}%
\end{pgfscope}%
\begin{pgfscope}%
\pgfpathrectangle{\pgfqpoint{0.100000in}{0.212622in}}{\pgfqpoint{3.696000in}{3.696000in}}%
\pgfusepath{clip}%
\pgfsetbuttcap%
\pgfsetroundjoin%
\definecolor{currentfill}{rgb}{0.121569,0.466667,0.705882}%
\pgfsetfillcolor{currentfill}%
\pgfsetfillopacity{0.711688}%
\pgfsetlinewidth{1.003750pt}%
\definecolor{currentstroke}{rgb}{0.121569,0.466667,0.705882}%
\pgfsetstrokecolor{currentstroke}%
\pgfsetstrokeopacity{0.711688}%
\pgfsetdash{}{0pt}%
\pgfpathmoveto{\pgfqpoint{1.149857in}{2.286056in}}%
\pgfpathcurveto{\pgfqpoint{1.158093in}{2.286056in}}{\pgfqpoint{1.165993in}{2.289329in}}{\pgfqpoint{1.171817in}{2.295153in}}%
\pgfpathcurveto{\pgfqpoint{1.177641in}{2.300976in}}{\pgfqpoint{1.180914in}{2.308877in}}{\pgfqpoint{1.180914in}{2.317113in}}%
\pgfpathcurveto{\pgfqpoint{1.180914in}{2.325349in}}{\pgfqpoint{1.177641in}{2.333249in}}{\pgfqpoint{1.171817in}{2.339073in}}%
\pgfpathcurveto{\pgfqpoint{1.165993in}{2.344897in}}{\pgfqpoint{1.158093in}{2.348169in}}{\pgfqpoint{1.149857in}{2.348169in}}%
\pgfpathcurveto{\pgfqpoint{1.141621in}{2.348169in}}{\pgfqpoint{1.133721in}{2.344897in}}{\pgfqpoint{1.127897in}{2.339073in}}%
\pgfpathcurveto{\pgfqpoint{1.122073in}{2.333249in}}{\pgfqpoint{1.118801in}{2.325349in}}{\pgfqpoint{1.118801in}{2.317113in}}%
\pgfpathcurveto{\pgfqpoint{1.118801in}{2.308877in}}{\pgfqpoint{1.122073in}{2.300976in}}{\pgfqpoint{1.127897in}{2.295153in}}%
\pgfpathcurveto{\pgfqpoint{1.133721in}{2.289329in}}{\pgfqpoint{1.141621in}{2.286056in}}{\pgfqpoint{1.149857in}{2.286056in}}%
\pgfpathclose%
\pgfusepath{stroke,fill}%
\end{pgfscope}%
\begin{pgfscope}%
\pgfpathrectangle{\pgfqpoint{0.100000in}{0.212622in}}{\pgfqpoint{3.696000in}{3.696000in}}%
\pgfusepath{clip}%
\pgfsetbuttcap%
\pgfsetroundjoin%
\definecolor{currentfill}{rgb}{0.121569,0.466667,0.705882}%
\pgfsetfillcolor{currentfill}%
\pgfsetfillopacity{0.716149}%
\pgfsetlinewidth{1.003750pt}%
\definecolor{currentstroke}{rgb}{0.121569,0.466667,0.705882}%
\pgfsetstrokecolor{currentstroke}%
\pgfsetstrokeopacity{0.716149}%
\pgfsetdash{}{0pt}%
\pgfpathmoveto{\pgfqpoint{2.426420in}{1.883707in}}%
\pgfpathcurveto{\pgfqpoint{2.434656in}{1.883707in}}{\pgfqpoint{2.442556in}{1.886980in}}{\pgfqpoint{2.448380in}{1.892804in}}%
\pgfpathcurveto{\pgfqpoint{2.454204in}{1.898628in}}{\pgfqpoint{2.457477in}{1.906528in}}{\pgfqpoint{2.457477in}{1.914764in}}%
\pgfpathcurveto{\pgfqpoint{2.457477in}{1.923000in}}{\pgfqpoint{2.454204in}{1.930900in}}{\pgfqpoint{2.448380in}{1.936724in}}%
\pgfpathcurveto{\pgfqpoint{2.442556in}{1.942548in}}{\pgfqpoint{2.434656in}{1.945820in}}{\pgfqpoint{2.426420in}{1.945820in}}%
\pgfpathcurveto{\pgfqpoint{2.418184in}{1.945820in}}{\pgfqpoint{2.410284in}{1.942548in}}{\pgfqpoint{2.404460in}{1.936724in}}%
\pgfpathcurveto{\pgfqpoint{2.398636in}{1.930900in}}{\pgfqpoint{2.395364in}{1.923000in}}{\pgfqpoint{2.395364in}{1.914764in}}%
\pgfpathcurveto{\pgfqpoint{2.395364in}{1.906528in}}{\pgfqpoint{2.398636in}{1.898628in}}{\pgfqpoint{2.404460in}{1.892804in}}%
\pgfpathcurveto{\pgfqpoint{2.410284in}{1.886980in}}{\pgfqpoint{2.418184in}{1.883707in}}{\pgfqpoint{2.426420in}{1.883707in}}%
\pgfpathclose%
\pgfusepath{stroke,fill}%
\end{pgfscope}%
\begin{pgfscope}%
\pgfpathrectangle{\pgfqpoint{0.100000in}{0.212622in}}{\pgfqpoint{3.696000in}{3.696000in}}%
\pgfusepath{clip}%
\pgfsetbuttcap%
\pgfsetroundjoin%
\definecolor{currentfill}{rgb}{0.121569,0.466667,0.705882}%
\pgfsetfillcolor{currentfill}%
\pgfsetfillopacity{0.721487}%
\pgfsetlinewidth{1.003750pt}%
\definecolor{currentstroke}{rgb}{0.121569,0.466667,0.705882}%
\pgfsetstrokecolor{currentstroke}%
\pgfsetstrokeopacity{0.721487}%
\pgfsetdash{}{0pt}%
\pgfpathmoveto{\pgfqpoint{2.431768in}{1.865201in}}%
\pgfpathcurveto{\pgfqpoint{2.440004in}{1.865201in}}{\pgfqpoint{2.447904in}{1.868473in}}{\pgfqpoint{2.453728in}{1.874297in}}%
\pgfpathcurveto{\pgfqpoint{2.459552in}{1.880121in}}{\pgfqpoint{2.462825in}{1.888021in}}{\pgfqpoint{2.462825in}{1.896258in}}%
\pgfpathcurveto{\pgfqpoint{2.462825in}{1.904494in}}{\pgfqpoint{2.459552in}{1.912394in}}{\pgfqpoint{2.453728in}{1.918218in}}%
\pgfpathcurveto{\pgfqpoint{2.447904in}{1.924042in}}{\pgfqpoint{2.440004in}{1.927314in}}{\pgfqpoint{2.431768in}{1.927314in}}%
\pgfpathcurveto{\pgfqpoint{2.423532in}{1.927314in}}{\pgfqpoint{2.415632in}{1.924042in}}{\pgfqpoint{2.409808in}{1.918218in}}%
\pgfpathcurveto{\pgfqpoint{2.403984in}{1.912394in}}{\pgfqpoint{2.400712in}{1.904494in}}{\pgfqpoint{2.400712in}{1.896258in}}%
\pgfpathcurveto{\pgfqpoint{2.400712in}{1.888021in}}{\pgfqpoint{2.403984in}{1.880121in}}{\pgfqpoint{2.409808in}{1.874297in}}%
\pgfpathcurveto{\pgfqpoint{2.415632in}{1.868473in}}{\pgfqpoint{2.423532in}{1.865201in}}{\pgfqpoint{2.431768in}{1.865201in}}%
\pgfpathclose%
\pgfusepath{stroke,fill}%
\end{pgfscope}%
\begin{pgfscope}%
\pgfpathrectangle{\pgfqpoint{0.100000in}{0.212622in}}{\pgfqpoint{3.696000in}{3.696000in}}%
\pgfusepath{clip}%
\pgfsetbuttcap%
\pgfsetroundjoin%
\definecolor{currentfill}{rgb}{0.121569,0.466667,0.705882}%
\pgfsetfillcolor{currentfill}%
\pgfsetfillopacity{0.721503}%
\pgfsetlinewidth{1.003750pt}%
\definecolor{currentstroke}{rgb}{0.121569,0.466667,0.705882}%
\pgfsetstrokecolor{currentstroke}%
\pgfsetstrokeopacity{0.721503}%
\pgfsetdash{}{0pt}%
\pgfpathmoveto{\pgfqpoint{1.146299in}{2.260160in}}%
\pgfpathcurveto{\pgfqpoint{1.154535in}{2.260160in}}{\pgfqpoint{1.162435in}{2.263432in}}{\pgfqpoint{1.168259in}{2.269256in}}%
\pgfpathcurveto{\pgfqpoint{1.174083in}{2.275080in}}{\pgfqpoint{1.177356in}{2.282980in}}{\pgfqpoint{1.177356in}{2.291216in}}%
\pgfpathcurveto{\pgfqpoint{1.177356in}{2.299453in}}{\pgfqpoint{1.174083in}{2.307353in}}{\pgfqpoint{1.168259in}{2.313177in}}%
\pgfpathcurveto{\pgfqpoint{1.162435in}{2.319001in}}{\pgfqpoint{1.154535in}{2.322273in}}{\pgfqpoint{1.146299in}{2.322273in}}%
\pgfpathcurveto{\pgfqpoint{1.138063in}{2.322273in}}{\pgfqpoint{1.130163in}{2.319001in}}{\pgfqpoint{1.124339in}{2.313177in}}%
\pgfpathcurveto{\pgfqpoint{1.118515in}{2.307353in}}{\pgfqpoint{1.115243in}{2.299453in}}{\pgfqpoint{1.115243in}{2.291216in}}%
\pgfpathcurveto{\pgfqpoint{1.115243in}{2.282980in}}{\pgfqpoint{1.118515in}{2.275080in}}{\pgfqpoint{1.124339in}{2.269256in}}%
\pgfpathcurveto{\pgfqpoint{1.130163in}{2.263432in}}{\pgfqpoint{1.138063in}{2.260160in}}{\pgfqpoint{1.146299in}{2.260160in}}%
\pgfpathclose%
\pgfusepath{stroke,fill}%
\end{pgfscope}%
\begin{pgfscope}%
\pgfpathrectangle{\pgfqpoint{0.100000in}{0.212622in}}{\pgfqpoint{3.696000in}{3.696000in}}%
\pgfusepath{clip}%
\pgfsetbuttcap%
\pgfsetroundjoin%
\definecolor{currentfill}{rgb}{0.121569,0.466667,0.705882}%
\pgfsetfillcolor{currentfill}%
\pgfsetfillopacity{0.727411}%
\pgfsetlinewidth{1.003750pt}%
\definecolor{currentstroke}{rgb}{0.121569,0.466667,0.705882}%
\pgfsetstrokecolor{currentstroke}%
\pgfsetstrokeopacity{0.727411}%
\pgfsetdash{}{0pt}%
\pgfpathmoveto{\pgfqpoint{2.438085in}{1.844469in}}%
\pgfpathcurveto{\pgfqpoint{2.446322in}{1.844469in}}{\pgfqpoint{2.454222in}{1.847741in}}{\pgfqpoint{2.460046in}{1.853565in}}%
\pgfpathcurveto{\pgfqpoint{2.465869in}{1.859389in}}{\pgfqpoint{2.469142in}{1.867289in}}{\pgfqpoint{2.469142in}{1.875526in}}%
\pgfpathcurveto{\pgfqpoint{2.469142in}{1.883762in}}{\pgfqpoint{2.465869in}{1.891662in}}{\pgfqpoint{2.460046in}{1.897486in}}%
\pgfpathcurveto{\pgfqpoint{2.454222in}{1.903310in}}{\pgfqpoint{2.446322in}{1.906582in}}{\pgfqpoint{2.438085in}{1.906582in}}%
\pgfpathcurveto{\pgfqpoint{2.429849in}{1.906582in}}{\pgfqpoint{2.421949in}{1.903310in}}{\pgfqpoint{2.416125in}{1.897486in}}%
\pgfpathcurveto{\pgfqpoint{2.410301in}{1.891662in}}{\pgfqpoint{2.407029in}{1.883762in}}{\pgfqpoint{2.407029in}{1.875526in}}%
\pgfpathcurveto{\pgfqpoint{2.407029in}{1.867289in}}{\pgfqpoint{2.410301in}{1.859389in}}{\pgfqpoint{2.416125in}{1.853565in}}%
\pgfpathcurveto{\pgfqpoint{2.421949in}{1.847741in}}{\pgfqpoint{2.429849in}{1.844469in}}{\pgfqpoint{2.438085in}{1.844469in}}%
\pgfpathclose%
\pgfusepath{stroke,fill}%
\end{pgfscope}%
\begin{pgfscope}%
\pgfpathrectangle{\pgfqpoint{0.100000in}{0.212622in}}{\pgfqpoint{3.696000in}{3.696000in}}%
\pgfusepath{clip}%
\pgfsetbuttcap%
\pgfsetroundjoin%
\definecolor{currentfill}{rgb}{0.121569,0.466667,0.705882}%
\pgfsetfillcolor{currentfill}%
\pgfsetfillopacity{0.727619}%
\pgfsetlinewidth{1.003750pt}%
\definecolor{currentstroke}{rgb}{0.121569,0.466667,0.705882}%
\pgfsetstrokecolor{currentstroke}%
\pgfsetstrokeopacity{0.727619}%
\pgfsetdash{}{0pt}%
\pgfpathmoveto{\pgfqpoint{1.142150in}{2.244078in}}%
\pgfpathcurveto{\pgfqpoint{1.150386in}{2.244078in}}{\pgfqpoint{1.158286in}{2.247350in}}{\pgfqpoint{1.164110in}{2.253174in}}%
\pgfpathcurveto{\pgfqpoint{1.169934in}{2.258998in}}{\pgfqpoint{1.173206in}{2.266898in}}{\pgfqpoint{1.173206in}{2.275134in}}%
\pgfpathcurveto{\pgfqpoint{1.173206in}{2.283370in}}{\pgfqpoint{1.169934in}{2.291270in}}{\pgfqpoint{1.164110in}{2.297094in}}%
\pgfpathcurveto{\pgfqpoint{1.158286in}{2.302918in}}{\pgfqpoint{1.150386in}{2.306191in}}{\pgfqpoint{1.142150in}{2.306191in}}%
\pgfpathcurveto{\pgfqpoint{1.133914in}{2.306191in}}{\pgfqpoint{1.126014in}{2.302918in}}{\pgfqpoint{1.120190in}{2.297094in}}%
\pgfpathcurveto{\pgfqpoint{1.114366in}{2.291270in}}{\pgfqpoint{1.111093in}{2.283370in}}{\pgfqpoint{1.111093in}{2.275134in}}%
\pgfpathcurveto{\pgfqpoint{1.111093in}{2.266898in}}{\pgfqpoint{1.114366in}{2.258998in}}{\pgfqpoint{1.120190in}{2.253174in}}%
\pgfpathcurveto{\pgfqpoint{1.126014in}{2.247350in}}{\pgfqpoint{1.133914in}{2.244078in}}{\pgfqpoint{1.142150in}{2.244078in}}%
\pgfpathclose%
\pgfusepath{stroke,fill}%
\end{pgfscope}%
\begin{pgfscope}%
\pgfpathrectangle{\pgfqpoint{0.100000in}{0.212622in}}{\pgfqpoint{3.696000in}{3.696000in}}%
\pgfusepath{clip}%
\pgfsetbuttcap%
\pgfsetroundjoin%
\definecolor{currentfill}{rgb}{0.121569,0.466667,0.705882}%
\pgfsetfillcolor{currentfill}%
\pgfsetfillopacity{0.727652}%
\pgfsetlinewidth{1.003750pt}%
\definecolor{currentstroke}{rgb}{0.121569,0.466667,0.705882}%
\pgfsetstrokecolor{currentstroke}%
\pgfsetstrokeopacity{0.727652}%
\pgfsetdash{}{0pt}%
\pgfpathmoveto{\pgfqpoint{1.030831in}{1.074549in}}%
\pgfpathcurveto{\pgfqpoint{1.039067in}{1.074549in}}{\pgfqpoint{1.046967in}{1.077822in}}{\pgfqpoint{1.052791in}{1.083646in}}%
\pgfpathcurveto{\pgfqpoint{1.058615in}{1.089470in}}{\pgfqpoint{1.061887in}{1.097370in}}{\pgfqpoint{1.061887in}{1.105606in}}%
\pgfpathcurveto{\pgfqpoint{1.061887in}{1.113842in}}{\pgfqpoint{1.058615in}{1.121742in}}{\pgfqpoint{1.052791in}{1.127566in}}%
\pgfpathcurveto{\pgfqpoint{1.046967in}{1.133390in}}{\pgfqpoint{1.039067in}{1.136662in}}{\pgfqpoint{1.030831in}{1.136662in}}%
\pgfpathcurveto{\pgfqpoint{1.022594in}{1.136662in}}{\pgfqpoint{1.014694in}{1.133390in}}{\pgfqpoint{1.008870in}{1.127566in}}%
\pgfpathcurveto{\pgfqpoint{1.003046in}{1.121742in}}{\pgfqpoint{0.999774in}{1.113842in}}{\pgfqpoint{0.999774in}{1.105606in}}%
\pgfpathcurveto{\pgfqpoint{0.999774in}{1.097370in}}{\pgfqpoint{1.003046in}{1.089470in}}{\pgfqpoint{1.008870in}{1.083646in}}%
\pgfpathcurveto{\pgfqpoint{1.014694in}{1.077822in}}{\pgfqpoint{1.022594in}{1.074549in}}{\pgfqpoint{1.030831in}{1.074549in}}%
\pgfpathclose%
\pgfusepath{stroke,fill}%
\end{pgfscope}%
\begin{pgfscope}%
\pgfpathrectangle{\pgfqpoint{0.100000in}{0.212622in}}{\pgfqpoint{3.696000in}{3.696000in}}%
\pgfusepath{clip}%
\pgfsetbuttcap%
\pgfsetroundjoin%
\definecolor{currentfill}{rgb}{0.121569,0.466667,0.705882}%
\pgfsetfillcolor{currentfill}%
\pgfsetfillopacity{0.730742}%
\pgfsetlinewidth{1.003750pt}%
\definecolor{currentstroke}{rgb}{0.121569,0.466667,0.705882}%
\pgfsetstrokecolor{currentstroke}%
\pgfsetstrokeopacity{0.730742}%
\pgfsetdash{}{0pt}%
\pgfpathmoveto{\pgfqpoint{2.441154in}{1.833162in}}%
\pgfpathcurveto{\pgfqpoint{2.449390in}{1.833162in}}{\pgfqpoint{2.457290in}{1.836434in}}{\pgfqpoint{2.463114in}{1.842258in}}%
\pgfpathcurveto{\pgfqpoint{2.468938in}{1.848082in}}{\pgfqpoint{2.472210in}{1.855982in}}{\pgfqpoint{2.472210in}{1.864219in}}%
\pgfpathcurveto{\pgfqpoint{2.472210in}{1.872455in}}{\pgfqpoint{2.468938in}{1.880355in}}{\pgfqpoint{2.463114in}{1.886179in}}%
\pgfpathcurveto{\pgfqpoint{2.457290in}{1.892003in}}{\pgfqpoint{2.449390in}{1.895275in}}{\pgfqpoint{2.441154in}{1.895275in}}%
\pgfpathcurveto{\pgfqpoint{2.432918in}{1.895275in}}{\pgfqpoint{2.425018in}{1.892003in}}{\pgfqpoint{2.419194in}{1.886179in}}%
\pgfpathcurveto{\pgfqpoint{2.413370in}{1.880355in}}{\pgfqpoint{2.410097in}{1.872455in}}{\pgfqpoint{2.410097in}{1.864219in}}%
\pgfpathcurveto{\pgfqpoint{2.410097in}{1.855982in}}{\pgfqpoint{2.413370in}{1.848082in}}{\pgfqpoint{2.419194in}{1.842258in}}%
\pgfpathcurveto{\pgfqpoint{2.425018in}{1.836434in}}{\pgfqpoint{2.432918in}{1.833162in}}{\pgfqpoint{2.441154in}{1.833162in}}%
\pgfpathclose%
\pgfusepath{stroke,fill}%
\end{pgfscope}%
\begin{pgfscope}%
\pgfpathrectangle{\pgfqpoint{0.100000in}{0.212622in}}{\pgfqpoint{3.696000in}{3.696000in}}%
\pgfusepath{clip}%
\pgfsetbuttcap%
\pgfsetroundjoin%
\definecolor{currentfill}{rgb}{0.121569,0.466667,0.705882}%
\pgfsetfillcolor{currentfill}%
\pgfsetfillopacity{0.732018}%
\pgfsetlinewidth{1.003750pt}%
\definecolor{currentstroke}{rgb}{0.121569,0.466667,0.705882}%
\pgfsetstrokecolor{currentstroke}%
\pgfsetstrokeopacity{0.732018}%
\pgfsetdash{}{0pt}%
\pgfpathmoveto{\pgfqpoint{1.140773in}{2.231812in}}%
\pgfpathcurveto{\pgfqpoint{1.149010in}{2.231812in}}{\pgfqpoint{1.156910in}{2.235084in}}{\pgfqpoint{1.162733in}{2.240908in}}%
\pgfpathcurveto{\pgfqpoint{1.168557in}{2.246732in}}{\pgfqpoint{1.171830in}{2.254632in}}{\pgfqpoint{1.171830in}{2.262868in}}%
\pgfpathcurveto{\pgfqpoint{1.171830in}{2.271105in}}{\pgfqpoint{1.168557in}{2.279005in}}{\pgfqpoint{1.162733in}{2.284829in}}%
\pgfpathcurveto{\pgfqpoint{1.156910in}{2.290653in}}{\pgfqpoint{1.149010in}{2.293925in}}{\pgfqpoint{1.140773in}{2.293925in}}%
\pgfpathcurveto{\pgfqpoint{1.132537in}{2.293925in}}{\pgfqpoint{1.124637in}{2.290653in}}{\pgfqpoint{1.118813in}{2.284829in}}%
\pgfpathcurveto{\pgfqpoint{1.112989in}{2.279005in}}{\pgfqpoint{1.109717in}{2.271105in}}{\pgfqpoint{1.109717in}{2.262868in}}%
\pgfpathcurveto{\pgfqpoint{1.109717in}{2.254632in}}{\pgfqpoint{1.112989in}{2.246732in}}{\pgfqpoint{1.118813in}{2.240908in}}%
\pgfpathcurveto{\pgfqpoint{1.124637in}{2.235084in}}{\pgfqpoint{1.132537in}{2.231812in}}{\pgfqpoint{1.140773in}{2.231812in}}%
\pgfpathclose%
\pgfusepath{stroke,fill}%
\end{pgfscope}%
\begin{pgfscope}%
\pgfpathrectangle{\pgfqpoint{0.100000in}{0.212622in}}{\pgfqpoint{3.696000in}{3.696000in}}%
\pgfusepath{clip}%
\pgfsetbuttcap%
\pgfsetroundjoin%
\definecolor{currentfill}{rgb}{0.121569,0.466667,0.705882}%
\pgfsetfillcolor{currentfill}%
\pgfsetfillopacity{0.732048}%
\pgfsetlinewidth{1.003750pt}%
\definecolor{currentstroke}{rgb}{0.121569,0.466667,0.705882}%
\pgfsetstrokecolor{currentstroke}%
\pgfsetstrokeopacity{0.732048}%
\pgfsetdash{}{0pt}%
\pgfpathmoveto{\pgfqpoint{1.044521in}{1.074532in}}%
\pgfpathcurveto{\pgfqpoint{1.052758in}{1.074532in}}{\pgfqpoint{1.060658in}{1.077804in}}{\pgfqpoint{1.066482in}{1.083628in}}%
\pgfpathcurveto{\pgfqpoint{1.072305in}{1.089452in}}{\pgfqpoint{1.075578in}{1.097352in}}{\pgfqpoint{1.075578in}{1.105588in}}%
\pgfpathcurveto{\pgfqpoint{1.075578in}{1.113825in}}{\pgfqpoint{1.072305in}{1.121725in}}{\pgfqpoint{1.066482in}{1.127549in}}%
\pgfpathcurveto{\pgfqpoint{1.060658in}{1.133373in}}{\pgfqpoint{1.052758in}{1.136645in}}{\pgfqpoint{1.044521in}{1.136645in}}%
\pgfpathcurveto{\pgfqpoint{1.036285in}{1.136645in}}{\pgfqpoint{1.028385in}{1.133373in}}{\pgfqpoint{1.022561in}{1.127549in}}%
\pgfpathcurveto{\pgfqpoint{1.016737in}{1.121725in}}{\pgfqpoint{1.013465in}{1.113825in}}{\pgfqpoint{1.013465in}{1.105588in}}%
\pgfpathcurveto{\pgfqpoint{1.013465in}{1.097352in}}{\pgfqpoint{1.016737in}{1.089452in}}{\pgfqpoint{1.022561in}{1.083628in}}%
\pgfpathcurveto{\pgfqpoint{1.028385in}{1.077804in}}{\pgfqpoint{1.036285in}{1.074532in}}{\pgfqpoint{1.044521in}{1.074532in}}%
\pgfpathclose%
\pgfusepath{stroke,fill}%
\end{pgfscope}%
\begin{pgfscope}%
\pgfpathrectangle{\pgfqpoint{0.100000in}{0.212622in}}{\pgfqpoint{3.696000in}{3.696000in}}%
\pgfusepath{clip}%
\pgfsetbuttcap%
\pgfsetroundjoin%
\definecolor{currentfill}{rgb}{0.121569,0.466667,0.705882}%
\pgfsetfillcolor{currentfill}%
\pgfsetfillopacity{0.733341}%
\pgfsetlinewidth{1.003750pt}%
\definecolor{currentstroke}{rgb}{0.121569,0.466667,0.705882}%
\pgfsetstrokecolor{currentstroke}%
\pgfsetstrokeopacity{0.733341}%
\pgfsetdash{}{0pt}%
\pgfpathmoveto{\pgfqpoint{1.140000in}{2.228108in}}%
\pgfpathcurveto{\pgfqpoint{1.148236in}{2.228108in}}{\pgfqpoint{1.156136in}{2.231380in}}{\pgfqpoint{1.161960in}{2.237204in}}%
\pgfpathcurveto{\pgfqpoint{1.167784in}{2.243028in}}{\pgfqpoint{1.171056in}{2.250928in}}{\pgfqpoint{1.171056in}{2.259164in}}%
\pgfpathcurveto{\pgfqpoint{1.171056in}{2.267401in}}{\pgfqpoint{1.167784in}{2.275301in}}{\pgfqpoint{1.161960in}{2.281125in}}%
\pgfpathcurveto{\pgfqpoint{1.156136in}{2.286948in}}{\pgfqpoint{1.148236in}{2.290221in}}{\pgfqpoint{1.140000in}{2.290221in}}%
\pgfpathcurveto{\pgfqpoint{1.131763in}{2.290221in}}{\pgfqpoint{1.123863in}{2.286948in}}{\pgfqpoint{1.118039in}{2.281125in}}%
\pgfpathcurveto{\pgfqpoint{1.112216in}{2.275301in}}{\pgfqpoint{1.108943in}{2.267401in}}{\pgfqpoint{1.108943in}{2.259164in}}%
\pgfpathcurveto{\pgfqpoint{1.108943in}{2.250928in}}{\pgfqpoint{1.112216in}{2.243028in}}{\pgfqpoint{1.118039in}{2.237204in}}%
\pgfpathcurveto{\pgfqpoint{1.123863in}{2.231380in}}{\pgfqpoint{1.131763in}{2.228108in}}{\pgfqpoint{1.140000in}{2.228108in}}%
\pgfpathclose%
\pgfusepath{stroke,fill}%
\end{pgfscope}%
\begin{pgfscope}%
\pgfpathrectangle{\pgfqpoint{0.100000in}{0.212622in}}{\pgfqpoint{3.696000in}{3.696000in}}%
\pgfusepath{clip}%
\pgfsetbuttcap%
\pgfsetroundjoin%
\definecolor{currentfill}{rgb}{0.121569,0.466667,0.705882}%
\pgfsetfillcolor{currentfill}%
\pgfsetfillopacity{0.734848}%
\pgfsetlinewidth{1.003750pt}%
\definecolor{currentstroke}{rgb}{0.121569,0.466667,0.705882}%
\pgfsetstrokecolor{currentstroke}%
\pgfsetstrokeopacity{0.734848}%
\pgfsetdash{}{0pt}%
\pgfpathmoveto{\pgfqpoint{2.446064in}{1.817986in}}%
\pgfpathcurveto{\pgfqpoint{2.454301in}{1.817986in}}{\pgfqpoint{2.462201in}{1.821258in}}{\pgfqpoint{2.468025in}{1.827082in}}%
\pgfpathcurveto{\pgfqpoint{2.473849in}{1.832906in}}{\pgfqpoint{2.477121in}{1.840806in}}{\pgfqpoint{2.477121in}{1.849042in}}%
\pgfpathcurveto{\pgfqpoint{2.477121in}{1.857278in}}{\pgfqpoint{2.473849in}{1.865178in}}{\pgfqpoint{2.468025in}{1.871002in}}%
\pgfpathcurveto{\pgfqpoint{2.462201in}{1.876826in}}{\pgfqpoint{2.454301in}{1.880099in}}{\pgfqpoint{2.446064in}{1.880099in}}%
\pgfpathcurveto{\pgfqpoint{2.437828in}{1.880099in}}{\pgfqpoint{2.429928in}{1.876826in}}{\pgfqpoint{2.424104in}{1.871002in}}%
\pgfpathcurveto{\pgfqpoint{2.418280in}{1.865178in}}{\pgfqpoint{2.415008in}{1.857278in}}{\pgfqpoint{2.415008in}{1.849042in}}%
\pgfpathcurveto{\pgfqpoint{2.415008in}{1.840806in}}{\pgfqpoint{2.418280in}{1.832906in}}{\pgfqpoint{2.424104in}{1.827082in}}%
\pgfpathcurveto{\pgfqpoint{2.429928in}{1.821258in}}{\pgfqpoint{2.437828in}{1.817986in}}{\pgfqpoint{2.446064in}{1.817986in}}%
\pgfpathclose%
\pgfusepath{stroke,fill}%
\end{pgfscope}%
\begin{pgfscope}%
\pgfpathrectangle{\pgfqpoint{0.100000in}{0.212622in}}{\pgfqpoint{3.696000in}{3.696000in}}%
\pgfusepath{clip}%
\pgfsetbuttcap%
\pgfsetroundjoin%
\definecolor{currentfill}{rgb}{0.121569,0.466667,0.705882}%
\pgfsetfillcolor{currentfill}%
\pgfsetfillopacity{0.735754}%
\pgfsetlinewidth{1.003750pt}%
\definecolor{currentstroke}{rgb}{0.121569,0.466667,0.705882}%
\pgfsetstrokecolor{currentstroke}%
\pgfsetstrokeopacity{0.735754}%
\pgfsetdash{}{0pt}%
\pgfpathmoveto{\pgfqpoint{1.139701in}{2.221119in}}%
\pgfpathcurveto{\pgfqpoint{1.147937in}{2.221119in}}{\pgfqpoint{1.155837in}{2.224391in}}{\pgfqpoint{1.161661in}{2.230215in}}%
\pgfpathcurveto{\pgfqpoint{1.167485in}{2.236039in}}{\pgfqpoint{1.170757in}{2.243939in}}{\pgfqpoint{1.170757in}{2.252175in}}%
\pgfpathcurveto{\pgfqpoint{1.170757in}{2.260411in}}{\pgfqpoint{1.167485in}{2.268312in}}{\pgfqpoint{1.161661in}{2.274135in}}%
\pgfpathcurveto{\pgfqpoint{1.155837in}{2.279959in}}{\pgfqpoint{1.147937in}{2.283232in}}{\pgfqpoint{1.139701in}{2.283232in}}%
\pgfpathcurveto{\pgfqpoint{1.131465in}{2.283232in}}{\pgfqpoint{1.123565in}{2.279959in}}{\pgfqpoint{1.117741in}{2.274135in}}%
\pgfpathcurveto{\pgfqpoint{1.111917in}{2.268312in}}{\pgfqpoint{1.108644in}{2.260411in}}{\pgfqpoint{1.108644in}{2.252175in}}%
\pgfpathcurveto{\pgfqpoint{1.108644in}{2.243939in}}{\pgfqpoint{1.111917in}{2.236039in}}{\pgfqpoint{1.117741in}{2.230215in}}%
\pgfpathcurveto{\pgfqpoint{1.123565in}{2.224391in}}{\pgfqpoint{1.131465in}{2.221119in}}{\pgfqpoint{1.139701in}{2.221119in}}%
\pgfpathclose%
\pgfusepath{stroke,fill}%
\end{pgfscope}%
\begin{pgfscope}%
\pgfpathrectangle{\pgfqpoint{0.100000in}{0.212622in}}{\pgfqpoint{3.696000in}{3.696000in}}%
\pgfusepath{clip}%
\pgfsetbuttcap%
\pgfsetroundjoin%
\definecolor{currentfill}{rgb}{0.121569,0.466667,0.705882}%
\pgfsetfillcolor{currentfill}%
\pgfsetfillopacity{0.735755}%
\pgfsetlinewidth{1.003750pt}%
\definecolor{currentstroke}{rgb}{0.121569,0.466667,0.705882}%
\pgfsetstrokecolor{currentstroke}%
\pgfsetstrokeopacity{0.735755}%
\pgfsetdash{}{0pt}%
\pgfpathmoveto{\pgfqpoint{1.056816in}{1.074515in}}%
\pgfpathcurveto{\pgfqpoint{1.065052in}{1.074515in}}{\pgfqpoint{1.072952in}{1.077787in}}{\pgfqpoint{1.078776in}{1.083611in}}%
\pgfpathcurveto{\pgfqpoint{1.084600in}{1.089435in}}{\pgfqpoint{1.087872in}{1.097335in}}{\pgfqpoint{1.087872in}{1.105571in}}%
\pgfpathcurveto{\pgfqpoint{1.087872in}{1.113808in}}{\pgfqpoint{1.084600in}{1.121708in}}{\pgfqpoint{1.078776in}{1.127532in}}%
\pgfpathcurveto{\pgfqpoint{1.072952in}{1.133356in}}{\pgfqpoint{1.065052in}{1.136628in}}{\pgfqpoint{1.056816in}{1.136628in}}%
\pgfpathcurveto{\pgfqpoint{1.048579in}{1.136628in}}{\pgfqpoint{1.040679in}{1.133356in}}{\pgfqpoint{1.034855in}{1.127532in}}%
\pgfpathcurveto{\pgfqpoint{1.029031in}{1.121708in}}{\pgfqpoint{1.025759in}{1.113808in}}{\pgfqpoint{1.025759in}{1.105571in}}%
\pgfpathcurveto{\pgfqpoint{1.025759in}{1.097335in}}{\pgfqpoint{1.029031in}{1.089435in}}{\pgfqpoint{1.034855in}{1.083611in}}%
\pgfpathcurveto{\pgfqpoint{1.040679in}{1.077787in}}{\pgfqpoint{1.048579in}{1.074515in}}{\pgfqpoint{1.056816in}{1.074515in}}%
\pgfpathclose%
\pgfusepath{stroke,fill}%
\end{pgfscope}%
\begin{pgfscope}%
\pgfpathrectangle{\pgfqpoint{0.100000in}{0.212622in}}{\pgfqpoint{3.696000in}{3.696000in}}%
\pgfusepath{clip}%
\pgfsetbuttcap%
\pgfsetroundjoin%
\definecolor{currentfill}{rgb}{0.121569,0.466667,0.705882}%
\pgfsetfillcolor{currentfill}%
\pgfsetfillopacity{0.738858}%
\pgfsetlinewidth{1.003750pt}%
\definecolor{currentstroke}{rgb}{0.121569,0.466667,0.705882}%
\pgfsetstrokecolor{currentstroke}%
\pgfsetstrokeopacity{0.738858}%
\pgfsetdash{}{0pt}%
\pgfpathmoveto{\pgfqpoint{1.067154in}{1.073804in}}%
\pgfpathcurveto{\pgfqpoint{1.075390in}{1.073804in}}{\pgfqpoint{1.083290in}{1.077076in}}{\pgfqpoint{1.089114in}{1.082900in}}%
\pgfpathcurveto{\pgfqpoint{1.094938in}{1.088724in}}{\pgfqpoint{1.098210in}{1.096624in}}{\pgfqpoint{1.098210in}{1.104860in}}%
\pgfpathcurveto{\pgfqpoint{1.098210in}{1.113097in}}{\pgfqpoint{1.094938in}{1.120997in}}{\pgfqpoint{1.089114in}{1.126821in}}%
\pgfpathcurveto{\pgfqpoint{1.083290in}{1.132645in}}{\pgfqpoint{1.075390in}{1.135917in}}{\pgfqpoint{1.067154in}{1.135917in}}%
\pgfpathcurveto{\pgfqpoint{1.058917in}{1.135917in}}{\pgfqpoint{1.051017in}{1.132645in}}{\pgfqpoint{1.045193in}{1.126821in}}%
\pgfpathcurveto{\pgfqpoint{1.039370in}{1.120997in}}{\pgfqpoint{1.036097in}{1.113097in}}{\pgfqpoint{1.036097in}{1.104860in}}%
\pgfpathcurveto{\pgfqpoint{1.036097in}{1.096624in}}{\pgfqpoint{1.039370in}{1.088724in}}{\pgfqpoint{1.045193in}{1.082900in}}%
\pgfpathcurveto{\pgfqpoint{1.051017in}{1.077076in}}{\pgfqpoint{1.058917in}{1.073804in}}{\pgfqpoint{1.067154in}{1.073804in}}%
\pgfpathclose%
\pgfusepath{stroke,fill}%
\end{pgfscope}%
\begin{pgfscope}%
\pgfpathrectangle{\pgfqpoint{0.100000in}{0.212622in}}{\pgfqpoint{3.696000in}{3.696000in}}%
\pgfusepath{clip}%
\pgfsetbuttcap%
\pgfsetroundjoin%
\definecolor{currentfill}{rgb}{0.121569,0.466667,0.705882}%
\pgfsetfillcolor{currentfill}%
\pgfsetfillopacity{0.740279}%
\pgfsetlinewidth{1.003750pt}%
\definecolor{currentstroke}{rgb}{0.121569,0.466667,0.705882}%
\pgfsetstrokecolor{currentstroke}%
\pgfsetstrokeopacity{0.740279}%
\pgfsetdash{}{0pt}%
\pgfpathmoveto{\pgfqpoint{1.137768in}{2.209213in}}%
\pgfpathcurveto{\pgfqpoint{1.146004in}{2.209213in}}{\pgfqpoint{1.153904in}{2.212485in}}{\pgfqpoint{1.159728in}{2.218309in}}%
\pgfpathcurveto{\pgfqpoint{1.165552in}{2.224133in}}{\pgfqpoint{1.168825in}{2.232033in}}{\pgfqpoint{1.168825in}{2.240269in}}%
\pgfpathcurveto{\pgfqpoint{1.168825in}{2.248505in}}{\pgfqpoint{1.165552in}{2.256405in}}{\pgfqpoint{1.159728in}{2.262229in}}%
\pgfpathcurveto{\pgfqpoint{1.153904in}{2.268053in}}{\pgfqpoint{1.146004in}{2.271326in}}{\pgfqpoint{1.137768in}{2.271326in}}%
\pgfpathcurveto{\pgfqpoint{1.129532in}{2.271326in}}{\pgfqpoint{1.121632in}{2.268053in}}{\pgfqpoint{1.115808in}{2.262229in}}%
\pgfpathcurveto{\pgfqpoint{1.109984in}{2.256405in}}{\pgfqpoint{1.106712in}{2.248505in}}{\pgfqpoint{1.106712in}{2.240269in}}%
\pgfpathcurveto{\pgfqpoint{1.106712in}{2.232033in}}{\pgfqpoint{1.109984in}{2.224133in}}{\pgfqpoint{1.115808in}{2.218309in}}%
\pgfpathcurveto{\pgfqpoint{1.121632in}{2.212485in}}{\pgfqpoint{1.129532in}{2.209213in}}{\pgfqpoint{1.137768in}{2.209213in}}%
\pgfpathclose%
\pgfusepath{stroke,fill}%
\end{pgfscope}%
\begin{pgfscope}%
\pgfpathrectangle{\pgfqpoint{0.100000in}{0.212622in}}{\pgfqpoint{3.696000in}{3.696000in}}%
\pgfusepath{clip}%
\pgfsetbuttcap%
\pgfsetroundjoin%
\definecolor{currentfill}{rgb}{0.121569,0.466667,0.705882}%
\pgfsetfillcolor{currentfill}%
\pgfsetfillopacity{0.740994}%
\pgfsetlinewidth{1.003750pt}%
\definecolor{currentstroke}{rgb}{0.121569,0.466667,0.705882}%
\pgfsetstrokecolor{currentstroke}%
\pgfsetstrokeopacity{0.740994}%
\pgfsetdash{}{0pt}%
\pgfpathmoveto{\pgfqpoint{2.450811in}{1.796221in}}%
\pgfpathcurveto{\pgfqpoint{2.459047in}{1.796221in}}{\pgfqpoint{2.466947in}{1.799493in}}{\pgfqpoint{2.472771in}{1.805317in}}%
\pgfpathcurveto{\pgfqpoint{2.478595in}{1.811141in}}{\pgfqpoint{2.481867in}{1.819041in}}{\pgfqpoint{2.481867in}{1.827277in}}%
\pgfpathcurveto{\pgfqpoint{2.481867in}{1.835514in}}{\pgfqpoint{2.478595in}{1.843414in}}{\pgfqpoint{2.472771in}{1.849238in}}%
\pgfpathcurveto{\pgfqpoint{2.466947in}{1.855062in}}{\pgfqpoint{2.459047in}{1.858334in}}{\pgfqpoint{2.450811in}{1.858334in}}%
\pgfpathcurveto{\pgfqpoint{2.442574in}{1.858334in}}{\pgfqpoint{2.434674in}{1.855062in}}{\pgfqpoint{2.428850in}{1.849238in}}%
\pgfpathcurveto{\pgfqpoint{2.423026in}{1.843414in}}{\pgfqpoint{2.419754in}{1.835514in}}{\pgfqpoint{2.419754in}{1.827277in}}%
\pgfpathcurveto{\pgfqpoint{2.419754in}{1.819041in}}{\pgfqpoint{2.423026in}{1.811141in}}{\pgfqpoint{2.428850in}{1.805317in}}%
\pgfpathcurveto{\pgfqpoint{2.434674in}{1.799493in}}{\pgfqpoint{2.442574in}{1.796221in}}{\pgfqpoint{2.450811in}{1.796221in}}%
\pgfpathclose%
\pgfusepath{stroke,fill}%
\end{pgfscope}%
\begin{pgfscope}%
\pgfpathrectangle{\pgfqpoint{0.100000in}{0.212622in}}{\pgfqpoint{3.696000in}{3.696000in}}%
\pgfusepath{clip}%
\pgfsetbuttcap%
\pgfsetroundjoin%
\definecolor{currentfill}{rgb}{0.121569,0.466667,0.705882}%
\pgfsetfillcolor{currentfill}%
\pgfsetfillopacity{0.741277}%
\pgfsetlinewidth{1.003750pt}%
\definecolor{currentstroke}{rgb}{0.121569,0.466667,0.705882}%
\pgfsetstrokecolor{currentstroke}%
\pgfsetstrokeopacity{0.741277}%
\pgfsetdash{}{0pt}%
\pgfpathmoveto{\pgfqpoint{1.075329in}{1.073346in}}%
\pgfpathcurveto{\pgfqpoint{1.083565in}{1.073346in}}{\pgfqpoint{1.091465in}{1.076619in}}{\pgfqpoint{1.097289in}{1.082443in}}%
\pgfpathcurveto{\pgfqpoint{1.103113in}{1.088267in}}{\pgfqpoint{1.106385in}{1.096167in}}{\pgfqpoint{1.106385in}{1.104403in}}%
\pgfpathcurveto{\pgfqpoint{1.106385in}{1.112639in}}{\pgfqpoint{1.103113in}{1.120539in}}{\pgfqpoint{1.097289in}{1.126363in}}%
\pgfpathcurveto{\pgfqpoint{1.091465in}{1.132187in}}{\pgfqpoint{1.083565in}{1.135459in}}{\pgfqpoint{1.075329in}{1.135459in}}%
\pgfpathcurveto{\pgfqpoint{1.067093in}{1.135459in}}{\pgfqpoint{1.059193in}{1.132187in}}{\pgfqpoint{1.053369in}{1.126363in}}%
\pgfpathcurveto{\pgfqpoint{1.047545in}{1.120539in}}{\pgfqpoint{1.044272in}{1.112639in}}{\pgfqpoint{1.044272in}{1.104403in}}%
\pgfpathcurveto{\pgfqpoint{1.044272in}{1.096167in}}{\pgfqpoint{1.047545in}{1.088267in}}{\pgfqpoint{1.053369in}{1.082443in}}%
\pgfpathcurveto{\pgfqpoint{1.059193in}{1.076619in}}{\pgfqpoint{1.067093in}{1.073346in}}{\pgfqpoint{1.075329in}{1.073346in}}%
\pgfpathclose%
\pgfusepath{stroke,fill}%
\end{pgfscope}%
\begin{pgfscope}%
\pgfpathrectangle{\pgfqpoint{0.100000in}{0.212622in}}{\pgfqpoint{3.696000in}{3.696000in}}%
\pgfusepath{clip}%
\pgfsetbuttcap%
\pgfsetroundjoin%
\definecolor{currentfill}{rgb}{0.121569,0.466667,0.705882}%
\pgfsetfillcolor{currentfill}%
\pgfsetfillopacity{0.741844}%
\pgfsetlinewidth{1.003750pt}%
\definecolor{currentstroke}{rgb}{0.121569,0.466667,0.705882}%
\pgfsetstrokecolor{currentstroke}%
\pgfsetstrokeopacity{0.741844}%
\pgfsetdash{}{0pt}%
\pgfpathmoveto{\pgfqpoint{1.137493in}{2.205118in}}%
\pgfpathcurveto{\pgfqpoint{1.145729in}{2.205118in}}{\pgfqpoint{1.153629in}{2.208391in}}{\pgfqpoint{1.159453in}{2.214215in}}%
\pgfpathcurveto{\pgfqpoint{1.165277in}{2.220039in}}{\pgfqpoint{1.168549in}{2.227939in}}{\pgfqpoint{1.168549in}{2.236175in}}%
\pgfpathcurveto{\pgfqpoint{1.168549in}{2.244411in}}{\pgfqpoint{1.165277in}{2.252311in}}{\pgfqpoint{1.159453in}{2.258135in}}%
\pgfpathcurveto{\pgfqpoint{1.153629in}{2.263959in}}{\pgfqpoint{1.145729in}{2.267231in}}{\pgfqpoint{1.137493in}{2.267231in}}%
\pgfpathcurveto{\pgfqpoint{1.129256in}{2.267231in}}{\pgfqpoint{1.121356in}{2.263959in}}{\pgfqpoint{1.115532in}{2.258135in}}%
\pgfpathcurveto{\pgfqpoint{1.109709in}{2.252311in}}{\pgfqpoint{1.106436in}{2.244411in}}{\pgfqpoint{1.106436in}{2.236175in}}%
\pgfpathcurveto{\pgfqpoint{1.106436in}{2.227939in}}{\pgfqpoint{1.109709in}{2.220039in}}{\pgfqpoint{1.115532in}{2.214215in}}%
\pgfpathcurveto{\pgfqpoint{1.121356in}{2.208391in}}{\pgfqpoint{1.129256in}{2.205118in}}{\pgfqpoint{1.137493in}{2.205118in}}%
\pgfpathclose%
\pgfusepath{stroke,fill}%
\end{pgfscope}%
\begin{pgfscope}%
\pgfpathrectangle{\pgfqpoint{0.100000in}{0.212622in}}{\pgfqpoint{3.696000in}{3.696000in}}%
\pgfusepath{clip}%
\pgfsetbuttcap%
\pgfsetroundjoin%
\definecolor{currentfill}{rgb}{0.121569,0.466667,0.705882}%
\pgfsetfillcolor{currentfill}%
\pgfsetfillopacity{0.742767}%
\pgfsetlinewidth{1.003750pt}%
\definecolor{currentstroke}{rgb}{0.121569,0.466667,0.705882}%
\pgfsetstrokecolor{currentstroke}%
\pgfsetstrokeopacity{0.742767}%
\pgfsetdash{}{0pt}%
\pgfpathmoveto{\pgfqpoint{1.137287in}{2.202653in}}%
\pgfpathcurveto{\pgfqpoint{1.145523in}{2.202653in}}{\pgfqpoint{1.153423in}{2.205925in}}{\pgfqpoint{1.159247in}{2.211749in}}%
\pgfpathcurveto{\pgfqpoint{1.165071in}{2.217573in}}{\pgfqpoint{1.168343in}{2.225473in}}{\pgfqpoint{1.168343in}{2.233709in}}%
\pgfpathcurveto{\pgfqpoint{1.168343in}{2.241946in}}{\pgfqpoint{1.165071in}{2.249846in}}{\pgfqpoint{1.159247in}{2.255670in}}%
\pgfpathcurveto{\pgfqpoint{1.153423in}{2.261494in}}{\pgfqpoint{1.145523in}{2.264766in}}{\pgfqpoint{1.137287in}{2.264766in}}%
\pgfpathcurveto{\pgfqpoint{1.129050in}{2.264766in}}{\pgfqpoint{1.121150in}{2.261494in}}{\pgfqpoint{1.115326in}{2.255670in}}%
\pgfpathcurveto{\pgfqpoint{1.109502in}{2.249846in}}{\pgfqpoint{1.106230in}{2.241946in}}{\pgfqpoint{1.106230in}{2.233709in}}%
\pgfpathcurveto{\pgfqpoint{1.106230in}{2.225473in}}{\pgfqpoint{1.109502in}{2.217573in}}{\pgfqpoint{1.115326in}{2.211749in}}%
\pgfpathcurveto{\pgfqpoint{1.121150in}{2.205925in}}{\pgfqpoint{1.129050in}{2.202653in}}{\pgfqpoint{1.137287in}{2.202653in}}%
\pgfpathclose%
\pgfusepath{stroke,fill}%
\end{pgfscope}%
\begin{pgfscope}%
\pgfpathrectangle{\pgfqpoint{0.100000in}{0.212622in}}{\pgfqpoint{3.696000in}{3.696000in}}%
\pgfusepath{clip}%
\pgfsetbuttcap%
\pgfsetroundjoin%
\definecolor{currentfill}{rgb}{0.121569,0.466667,0.705882}%
\pgfsetfillcolor{currentfill}%
\pgfsetfillopacity{0.742909}%
\pgfsetlinewidth{1.003750pt}%
\definecolor{currentstroke}{rgb}{0.121569,0.466667,0.705882}%
\pgfsetstrokecolor{currentstroke}%
\pgfsetstrokeopacity{0.742909}%
\pgfsetdash{}{0pt}%
\pgfpathmoveto{\pgfqpoint{1.080852in}{1.073119in}}%
\pgfpathcurveto{\pgfqpoint{1.089089in}{1.073119in}}{\pgfqpoint{1.096989in}{1.076391in}}{\pgfqpoint{1.102813in}{1.082215in}}%
\pgfpathcurveto{\pgfqpoint{1.108637in}{1.088039in}}{\pgfqpoint{1.111909in}{1.095939in}}{\pgfqpoint{1.111909in}{1.104175in}}%
\pgfpathcurveto{\pgfqpoint{1.111909in}{1.112412in}}{\pgfqpoint{1.108637in}{1.120312in}}{\pgfqpoint{1.102813in}{1.126136in}}%
\pgfpathcurveto{\pgfqpoint{1.096989in}{1.131960in}}{\pgfqpoint{1.089089in}{1.135232in}}{\pgfqpoint{1.080852in}{1.135232in}}%
\pgfpathcurveto{\pgfqpoint{1.072616in}{1.135232in}}{\pgfqpoint{1.064716in}{1.131960in}}{\pgfqpoint{1.058892in}{1.126136in}}%
\pgfpathcurveto{\pgfqpoint{1.053068in}{1.120312in}}{\pgfqpoint{1.049796in}{1.112412in}}{\pgfqpoint{1.049796in}{1.104175in}}%
\pgfpathcurveto{\pgfqpoint{1.049796in}{1.095939in}}{\pgfqpoint{1.053068in}{1.088039in}}{\pgfqpoint{1.058892in}{1.082215in}}%
\pgfpathcurveto{\pgfqpoint{1.064716in}{1.076391in}}{\pgfqpoint{1.072616in}{1.073119in}}{\pgfqpoint{1.080852in}{1.073119in}}%
\pgfpathclose%
\pgfusepath{stroke,fill}%
\end{pgfscope}%
\begin{pgfscope}%
\pgfpathrectangle{\pgfqpoint{0.100000in}{0.212622in}}{\pgfqpoint{3.696000in}{3.696000in}}%
\pgfusepath{clip}%
\pgfsetbuttcap%
\pgfsetroundjoin%
\definecolor{currentfill}{rgb}{0.121569,0.466667,0.705882}%
\pgfsetfillcolor{currentfill}%
\pgfsetfillopacity{0.743037}%
\pgfsetlinewidth{1.003750pt}%
\definecolor{currentstroke}{rgb}{0.121569,0.466667,0.705882}%
\pgfsetstrokecolor{currentstroke}%
\pgfsetstrokeopacity{0.743037}%
\pgfsetdash{}{0pt}%
\pgfpathmoveto{\pgfqpoint{1.137249in}{2.201941in}}%
\pgfpathcurveto{\pgfqpoint{1.145485in}{2.201941in}}{\pgfqpoint{1.153385in}{2.205213in}}{\pgfqpoint{1.159209in}{2.211037in}}%
\pgfpathcurveto{\pgfqpoint{1.165033in}{2.216861in}}{\pgfqpoint{1.168305in}{2.224761in}}{\pgfqpoint{1.168305in}{2.232997in}}%
\pgfpathcurveto{\pgfqpoint{1.168305in}{2.241233in}}{\pgfqpoint{1.165033in}{2.249133in}}{\pgfqpoint{1.159209in}{2.254957in}}%
\pgfpathcurveto{\pgfqpoint{1.153385in}{2.260781in}}{\pgfqpoint{1.145485in}{2.264054in}}{\pgfqpoint{1.137249in}{2.264054in}}%
\pgfpathcurveto{\pgfqpoint{1.129013in}{2.264054in}}{\pgfqpoint{1.121113in}{2.260781in}}{\pgfqpoint{1.115289in}{2.254957in}}%
\pgfpathcurveto{\pgfqpoint{1.109465in}{2.249133in}}{\pgfqpoint{1.106192in}{2.241233in}}{\pgfqpoint{1.106192in}{2.232997in}}%
\pgfpathcurveto{\pgfqpoint{1.106192in}{2.224761in}}{\pgfqpoint{1.109465in}{2.216861in}}{\pgfqpoint{1.115289in}{2.211037in}}%
\pgfpathcurveto{\pgfqpoint{1.121113in}{2.205213in}}{\pgfqpoint{1.129013in}{2.201941in}}{\pgfqpoint{1.137249in}{2.201941in}}%
\pgfpathclose%
\pgfusepath{stroke,fill}%
\end{pgfscope}%
\begin{pgfscope}%
\pgfpathrectangle{\pgfqpoint{0.100000in}{0.212622in}}{\pgfqpoint{3.696000in}{3.696000in}}%
\pgfusepath{clip}%
\pgfsetbuttcap%
\pgfsetroundjoin%
\definecolor{currentfill}{rgb}{0.121569,0.466667,0.705882}%
\pgfsetfillcolor{currentfill}%
\pgfsetfillopacity{0.743525}%
\pgfsetlinewidth{1.003750pt}%
\definecolor{currentstroke}{rgb}{0.121569,0.466667,0.705882}%
\pgfsetstrokecolor{currentstroke}%
\pgfsetstrokeopacity{0.743525}%
\pgfsetdash{}{0pt}%
\pgfpathmoveto{\pgfqpoint{1.137220in}{2.200633in}}%
\pgfpathcurveto{\pgfqpoint{1.145456in}{2.200633in}}{\pgfqpoint{1.153356in}{2.203905in}}{\pgfqpoint{1.159180in}{2.209729in}}%
\pgfpathcurveto{\pgfqpoint{1.165004in}{2.215553in}}{\pgfqpoint{1.168276in}{2.223453in}}{\pgfqpoint{1.168276in}{2.231689in}}%
\pgfpathcurveto{\pgfqpoint{1.168276in}{2.239926in}}{\pgfqpoint{1.165004in}{2.247826in}}{\pgfqpoint{1.159180in}{2.253650in}}%
\pgfpathcurveto{\pgfqpoint{1.153356in}{2.259474in}}{\pgfqpoint{1.145456in}{2.262746in}}{\pgfqpoint{1.137220in}{2.262746in}}%
\pgfpathcurveto{\pgfqpoint{1.128984in}{2.262746in}}{\pgfqpoint{1.121084in}{2.259474in}}{\pgfqpoint{1.115260in}{2.253650in}}%
\pgfpathcurveto{\pgfqpoint{1.109436in}{2.247826in}}{\pgfqpoint{1.106163in}{2.239926in}}{\pgfqpoint{1.106163in}{2.231689in}}%
\pgfpathcurveto{\pgfqpoint{1.106163in}{2.223453in}}{\pgfqpoint{1.109436in}{2.215553in}}{\pgfqpoint{1.115260in}{2.209729in}}%
\pgfpathcurveto{\pgfqpoint{1.121084in}{2.203905in}}{\pgfqpoint{1.128984in}{2.200633in}}{\pgfqpoint{1.137220in}{2.200633in}}%
\pgfpathclose%
\pgfusepath{stroke,fill}%
\end{pgfscope}%
\begin{pgfscope}%
\pgfpathrectangle{\pgfqpoint{0.100000in}{0.212622in}}{\pgfqpoint{3.696000in}{3.696000in}}%
\pgfusepath{clip}%
\pgfsetbuttcap%
\pgfsetroundjoin%
\definecolor{currentfill}{rgb}{0.121569,0.466667,0.705882}%
\pgfsetfillcolor{currentfill}%
\pgfsetfillopacity{0.743819}%
\pgfsetlinewidth{1.003750pt}%
\definecolor{currentstroke}{rgb}{0.121569,0.466667,0.705882}%
\pgfsetstrokecolor{currentstroke}%
\pgfsetstrokeopacity{0.743819}%
\pgfsetdash{}{0pt}%
\pgfpathmoveto{\pgfqpoint{1.083935in}{1.072947in}}%
\pgfpathcurveto{\pgfqpoint{1.092171in}{1.072947in}}{\pgfqpoint{1.100071in}{1.076220in}}{\pgfqpoint{1.105895in}{1.082044in}}%
\pgfpathcurveto{\pgfqpoint{1.111719in}{1.087868in}}{\pgfqpoint{1.114991in}{1.095768in}}{\pgfqpoint{1.114991in}{1.104004in}}%
\pgfpathcurveto{\pgfqpoint{1.114991in}{1.112240in}}{\pgfqpoint{1.111719in}{1.120140in}}{\pgfqpoint{1.105895in}{1.125964in}}%
\pgfpathcurveto{\pgfqpoint{1.100071in}{1.131788in}}{\pgfqpoint{1.092171in}{1.135060in}}{\pgfqpoint{1.083935in}{1.135060in}}%
\pgfpathcurveto{\pgfqpoint{1.075698in}{1.135060in}}{\pgfqpoint{1.067798in}{1.131788in}}{\pgfqpoint{1.061974in}{1.125964in}}%
\pgfpathcurveto{\pgfqpoint{1.056151in}{1.120140in}}{\pgfqpoint{1.052878in}{1.112240in}}{\pgfqpoint{1.052878in}{1.104004in}}%
\pgfpathcurveto{\pgfqpoint{1.052878in}{1.095768in}}{\pgfqpoint{1.056151in}{1.087868in}}{\pgfqpoint{1.061974in}{1.082044in}}%
\pgfpathcurveto{\pgfqpoint{1.067798in}{1.076220in}}{\pgfqpoint{1.075698in}{1.072947in}}{\pgfqpoint{1.083935in}{1.072947in}}%
\pgfpathclose%
\pgfusepath{stroke,fill}%
\end{pgfscope}%
\begin{pgfscope}%
\pgfpathrectangle{\pgfqpoint{0.100000in}{0.212622in}}{\pgfqpoint{3.696000in}{3.696000in}}%
\pgfusepath{clip}%
\pgfsetbuttcap%
\pgfsetroundjoin%
\definecolor{currentfill}{rgb}{0.121569,0.466667,0.705882}%
\pgfsetfillcolor{currentfill}%
\pgfsetfillopacity{0.744417}%
\pgfsetlinewidth{1.003750pt}%
\definecolor{currentstroke}{rgb}{0.121569,0.466667,0.705882}%
\pgfsetstrokecolor{currentstroke}%
\pgfsetstrokeopacity{0.744417}%
\pgfsetdash{}{0pt}%
\pgfpathmoveto{\pgfqpoint{1.137244in}{2.198264in}}%
\pgfpathcurveto{\pgfqpoint{1.145481in}{2.198264in}}{\pgfqpoint{1.153381in}{2.201536in}}{\pgfqpoint{1.159204in}{2.207360in}}%
\pgfpathcurveto{\pgfqpoint{1.165028in}{2.213184in}}{\pgfqpoint{1.168301in}{2.221084in}}{\pgfqpoint{1.168301in}{2.229320in}}%
\pgfpathcurveto{\pgfqpoint{1.168301in}{2.237556in}}{\pgfqpoint{1.165028in}{2.245456in}}{\pgfqpoint{1.159204in}{2.251280in}}%
\pgfpathcurveto{\pgfqpoint{1.153381in}{2.257104in}}{\pgfqpoint{1.145481in}{2.260377in}}{\pgfqpoint{1.137244in}{2.260377in}}%
\pgfpathcurveto{\pgfqpoint{1.129008in}{2.260377in}}{\pgfqpoint{1.121108in}{2.257104in}}{\pgfqpoint{1.115284in}{2.251280in}}%
\pgfpathcurveto{\pgfqpoint{1.109460in}{2.245456in}}{\pgfqpoint{1.106188in}{2.237556in}}{\pgfqpoint{1.106188in}{2.229320in}}%
\pgfpathcurveto{\pgfqpoint{1.106188in}{2.221084in}}{\pgfqpoint{1.109460in}{2.213184in}}{\pgfqpoint{1.115284in}{2.207360in}}%
\pgfpathcurveto{\pgfqpoint{1.121108in}{2.201536in}}{\pgfqpoint{1.129008in}{2.198264in}}{\pgfqpoint{1.137244in}{2.198264in}}%
\pgfpathclose%
\pgfusepath{stroke,fill}%
\end{pgfscope}%
\begin{pgfscope}%
\pgfpathrectangle{\pgfqpoint{0.100000in}{0.212622in}}{\pgfqpoint{3.696000in}{3.696000in}}%
\pgfusepath{clip}%
\pgfsetbuttcap%
\pgfsetroundjoin%
\definecolor{currentfill}{rgb}{0.121569,0.466667,0.705882}%
\pgfsetfillcolor{currentfill}%
\pgfsetfillopacity{0.744489}%
\pgfsetlinewidth{1.003750pt}%
\definecolor{currentstroke}{rgb}{0.121569,0.466667,0.705882}%
\pgfsetstrokecolor{currentstroke}%
\pgfsetstrokeopacity{0.744489}%
\pgfsetdash{}{0pt}%
\pgfpathmoveto{\pgfqpoint{1.137253in}{2.198072in}}%
\pgfpathcurveto{\pgfqpoint{1.145489in}{2.198072in}}{\pgfqpoint{1.153389in}{2.201345in}}{\pgfqpoint{1.159213in}{2.207169in}}%
\pgfpathcurveto{\pgfqpoint{1.165037in}{2.212992in}}{\pgfqpoint{1.168309in}{2.220893in}}{\pgfqpoint{1.168309in}{2.229129in}}%
\pgfpathcurveto{\pgfqpoint{1.168309in}{2.237365in}}{\pgfqpoint{1.165037in}{2.245265in}}{\pgfqpoint{1.159213in}{2.251089in}}%
\pgfpathcurveto{\pgfqpoint{1.153389in}{2.256913in}}{\pgfqpoint{1.145489in}{2.260185in}}{\pgfqpoint{1.137253in}{2.260185in}}%
\pgfpathcurveto{\pgfqpoint{1.129017in}{2.260185in}}{\pgfqpoint{1.121116in}{2.256913in}}{\pgfqpoint{1.115293in}{2.251089in}}%
\pgfpathcurveto{\pgfqpoint{1.109469in}{2.245265in}}{\pgfqpoint{1.106196in}{2.237365in}}{\pgfqpoint{1.106196in}{2.229129in}}%
\pgfpathcurveto{\pgfqpoint{1.106196in}{2.220893in}}{\pgfqpoint{1.109469in}{2.212992in}}{\pgfqpoint{1.115293in}{2.207169in}}%
\pgfpathcurveto{\pgfqpoint{1.121116in}{2.201345in}}{\pgfqpoint{1.129017in}{2.198072in}}{\pgfqpoint{1.137253in}{2.198072in}}%
\pgfpathclose%
\pgfusepath{stroke,fill}%
\end{pgfscope}%
\begin{pgfscope}%
\pgfpathrectangle{\pgfqpoint{0.100000in}{0.212622in}}{\pgfqpoint{3.696000in}{3.696000in}}%
\pgfusepath{clip}%
\pgfsetbuttcap%
\pgfsetroundjoin%
\definecolor{currentfill}{rgb}{0.121569,0.466667,0.705882}%
\pgfsetfillcolor{currentfill}%
\pgfsetfillopacity{0.744620}%
\pgfsetlinewidth{1.003750pt}%
\definecolor{currentstroke}{rgb}{0.121569,0.466667,0.705882}%
\pgfsetstrokecolor{currentstroke}%
\pgfsetstrokeopacity{0.744620}%
\pgfsetdash{}{0pt}%
\pgfpathmoveto{\pgfqpoint{1.137277in}{2.197725in}}%
\pgfpathcurveto{\pgfqpoint{1.145513in}{2.197725in}}{\pgfqpoint{1.153413in}{2.200997in}}{\pgfqpoint{1.159237in}{2.206821in}}%
\pgfpathcurveto{\pgfqpoint{1.165061in}{2.212645in}}{\pgfqpoint{1.168334in}{2.220545in}}{\pgfqpoint{1.168334in}{2.228781in}}%
\pgfpathcurveto{\pgfqpoint{1.168334in}{2.237017in}}{\pgfqpoint{1.165061in}{2.244917in}}{\pgfqpoint{1.159237in}{2.250741in}}%
\pgfpathcurveto{\pgfqpoint{1.153413in}{2.256565in}}{\pgfqpoint{1.145513in}{2.259838in}}{\pgfqpoint{1.137277in}{2.259838in}}%
\pgfpathcurveto{\pgfqpoint{1.129041in}{2.259838in}}{\pgfqpoint{1.121141in}{2.256565in}}{\pgfqpoint{1.115317in}{2.250741in}}%
\pgfpathcurveto{\pgfqpoint{1.109493in}{2.244917in}}{\pgfqpoint{1.106221in}{2.237017in}}{\pgfqpoint{1.106221in}{2.228781in}}%
\pgfpathcurveto{\pgfqpoint{1.106221in}{2.220545in}}{\pgfqpoint{1.109493in}{2.212645in}}{\pgfqpoint{1.115317in}{2.206821in}}%
\pgfpathcurveto{\pgfqpoint{1.121141in}{2.200997in}}{\pgfqpoint{1.129041in}{2.197725in}}{\pgfqpoint{1.137277in}{2.197725in}}%
\pgfpathclose%
\pgfusepath{stroke,fill}%
\end{pgfscope}%
\begin{pgfscope}%
\pgfpathrectangle{\pgfqpoint{0.100000in}{0.212622in}}{\pgfqpoint{3.696000in}{3.696000in}}%
\pgfusepath{clip}%
\pgfsetbuttcap%
\pgfsetroundjoin%
\definecolor{currentfill}{rgb}{0.121569,0.466667,0.705882}%
\pgfsetfillcolor{currentfill}%
\pgfsetfillopacity{0.744857}%
\pgfsetlinewidth{1.003750pt}%
\definecolor{currentstroke}{rgb}{0.121569,0.466667,0.705882}%
\pgfsetstrokecolor{currentstroke}%
\pgfsetstrokeopacity{0.744857}%
\pgfsetdash{}{0pt}%
\pgfpathmoveto{\pgfqpoint{1.137341in}{2.197087in}}%
\pgfpathcurveto{\pgfqpoint{1.145578in}{2.197087in}}{\pgfqpoint{1.153478in}{2.200359in}}{\pgfqpoint{1.159302in}{2.206183in}}%
\pgfpathcurveto{\pgfqpoint{1.165125in}{2.212007in}}{\pgfqpoint{1.168398in}{2.219907in}}{\pgfqpoint{1.168398in}{2.228144in}}%
\pgfpathcurveto{\pgfqpoint{1.168398in}{2.236380in}}{\pgfqpoint{1.165125in}{2.244280in}}{\pgfqpoint{1.159302in}{2.250104in}}%
\pgfpathcurveto{\pgfqpoint{1.153478in}{2.255928in}}{\pgfqpoint{1.145578in}{2.259200in}}{\pgfqpoint{1.137341in}{2.259200in}}%
\pgfpathcurveto{\pgfqpoint{1.129105in}{2.259200in}}{\pgfqpoint{1.121205in}{2.255928in}}{\pgfqpoint{1.115381in}{2.250104in}}%
\pgfpathcurveto{\pgfqpoint{1.109557in}{2.244280in}}{\pgfqpoint{1.106285in}{2.236380in}}{\pgfqpoint{1.106285in}{2.228144in}}%
\pgfpathcurveto{\pgfqpoint{1.106285in}{2.219907in}}{\pgfqpoint{1.109557in}{2.212007in}}{\pgfqpoint{1.115381in}{2.206183in}}%
\pgfpathcurveto{\pgfqpoint{1.121205in}{2.200359in}}{\pgfqpoint{1.129105in}{2.197087in}}{\pgfqpoint{1.137341in}{2.197087in}}%
\pgfpathclose%
\pgfusepath{stroke,fill}%
\end{pgfscope}%
\begin{pgfscope}%
\pgfpathrectangle{\pgfqpoint{0.100000in}{0.212622in}}{\pgfqpoint{3.696000in}{3.696000in}}%
\pgfusepath{clip}%
\pgfsetbuttcap%
\pgfsetroundjoin%
\definecolor{currentfill}{rgb}{0.121569,0.466667,0.705882}%
\pgfsetfillcolor{currentfill}%
\pgfsetfillopacity{0.745284}%
\pgfsetlinewidth{1.003750pt}%
\definecolor{currentstroke}{rgb}{0.121569,0.466667,0.705882}%
\pgfsetstrokecolor{currentstroke}%
\pgfsetstrokeopacity{0.745284}%
\pgfsetdash{}{0pt}%
\pgfpathmoveto{\pgfqpoint{1.137498in}{2.195922in}}%
\pgfpathcurveto{\pgfqpoint{1.145734in}{2.195922in}}{\pgfqpoint{1.153634in}{2.199194in}}{\pgfqpoint{1.159458in}{2.205018in}}%
\pgfpathcurveto{\pgfqpoint{1.165282in}{2.210842in}}{\pgfqpoint{1.168554in}{2.218742in}}{\pgfqpoint{1.168554in}{2.226979in}}%
\pgfpathcurveto{\pgfqpoint{1.168554in}{2.235215in}}{\pgfqpoint{1.165282in}{2.243115in}}{\pgfqpoint{1.159458in}{2.248939in}}%
\pgfpathcurveto{\pgfqpoint{1.153634in}{2.254763in}}{\pgfqpoint{1.145734in}{2.258035in}}{\pgfqpoint{1.137498in}{2.258035in}}%
\pgfpathcurveto{\pgfqpoint{1.129261in}{2.258035in}}{\pgfqpoint{1.121361in}{2.254763in}}{\pgfqpoint{1.115537in}{2.248939in}}%
\pgfpathcurveto{\pgfqpoint{1.109713in}{2.243115in}}{\pgfqpoint{1.106441in}{2.235215in}}{\pgfqpoint{1.106441in}{2.226979in}}%
\pgfpathcurveto{\pgfqpoint{1.106441in}{2.218742in}}{\pgfqpoint{1.109713in}{2.210842in}}{\pgfqpoint{1.115537in}{2.205018in}}%
\pgfpathcurveto{\pgfqpoint{1.121361in}{2.199194in}}{\pgfqpoint{1.129261in}{2.195922in}}{\pgfqpoint{1.137498in}{2.195922in}}%
\pgfpathclose%
\pgfusepath{stroke,fill}%
\end{pgfscope}%
\begin{pgfscope}%
\pgfpathrectangle{\pgfqpoint{0.100000in}{0.212622in}}{\pgfqpoint{3.696000in}{3.696000in}}%
\pgfusepath{clip}%
\pgfsetbuttcap%
\pgfsetroundjoin%
\definecolor{currentfill}{rgb}{0.121569,0.466667,0.705882}%
\pgfsetfillcolor{currentfill}%
\pgfsetfillopacity{0.745481}%
\pgfsetlinewidth{1.003750pt}%
\definecolor{currentstroke}{rgb}{0.121569,0.466667,0.705882}%
\pgfsetstrokecolor{currentstroke}%
\pgfsetstrokeopacity{0.745481}%
\pgfsetdash{}{0pt}%
\pgfpathmoveto{\pgfqpoint{1.089484in}{1.072466in}}%
\pgfpathcurveto{\pgfqpoint{1.097720in}{1.072466in}}{\pgfqpoint{1.105620in}{1.075738in}}{\pgfqpoint{1.111444in}{1.081562in}}%
\pgfpathcurveto{\pgfqpoint{1.117268in}{1.087386in}}{\pgfqpoint{1.120540in}{1.095286in}}{\pgfqpoint{1.120540in}{1.103523in}}%
\pgfpathcurveto{\pgfqpoint{1.120540in}{1.111759in}}{\pgfqpoint{1.117268in}{1.119659in}}{\pgfqpoint{1.111444in}{1.125483in}}%
\pgfpathcurveto{\pgfqpoint{1.105620in}{1.131307in}}{\pgfqpoint{1.097720in}{1.134579in}}{\pgfqpoint{1.089484in}{1.134579in}}%
\pgfpathcurveto{\pgfqpoint{1.081247in}{1.134579in}}{\pgfqpoint{1.073347in}{1.131307in}}{\pgfqpoint{1.067523in}{1.125483in}}%
\pgfpathcurveto{\pgfqpoint{1.061699in}{1.119659in}}{\pgfqpoint{1.058427in}{1.111759in}}{\pgfqpoint{1.058427in}{1.103523in}}%
\pgfpathcurveto{\pgfqpoint{1.058427in}{1.095286in}}{\pgfqpoint{1.061699in}{1.087386in}}{\pgfqpoint{1.067523in}{1.081562in}}%
\pgfpathcurveto{\pgfqpoint{1.073347in}{1.075738in}}{\pgfqpoint{1.081247in}{1.072466in}}{\pgfqpoint{1.089484in}{1.072466in}}%
\pgfpathclose%
\pgfusepath{stroke,fill}%
\end{pgfscope}%
\begin{pgfscope}%
\pgfpathrectangle{\pgfqpoint{0.100000in}{0.212622in}}{\pgfqpoint{3.696000in}{3.696000in}}%
\pgfusepath{clip}%
\pgfsetbuttcap%
\pgfsetroundjoin%
\definecolor{currentfill}{rgb}{0.121569,0.466667,0.705882}%
\pgfsetfillcolor{currentfill}%
\pgfsetfillopacity{0.746058}%
\pgfsetlinewidth{1.003750pt}%
\definecolor{currentstroke}{rgb}{0.121569,0.466667,0.705882}%
\pgfsetstrokecolor{currentstroke}%
\pgfsetstrokeopacity{0.746058}%
\pgfsetdash{}{0pt}%
\pgfpathmoveto{\pgfqpoint{1.137877in}{2.193808in}}%
\pgfpathcurveto{\pgfqpoint{1.146113in}{2.193808in}}{\pgfqpoint{1.154013in}{2.197081in}}{\pgfqpoint{1.159837in}{2.202905in}}%
\pgfpathcurveto{\pgfqpoint{1.165661in}{2.208729in}}{\pgfqpoint{1.168933in}{2.216629in}}{\pgfqpoint{1.168933in}{2.224865in}}%
\pgfpathcurveto{\pgfqpoint{1.168933in}{2.233101in}}{\pgfqpoint{1.165661in}{2.241001in}}{\pgfqpoint{1.159837in}{2.246825in}}%
\pgfpathcurveto{\pgfqpoint{1.154013in}{2.252649in}}{\pgfqpoint{1.146113in}{2.255921in}}{\pgfqpoint{1.137877in}{2.255921in}}%
\pgfpathcurveto{\pgfqpoint{1.129641in}{2.255921in}}{\pgfqpoint{1.121741in}{2.252649in}}{\pgfqpoint{1.115917in}{2.246825in}}%
\pgfpathcurveto{\pgfqpoint{1.110093in}{2.241001in}}{\pgfqpoint{1.106820in}{2.233101in}}{\pgfqpoint{1.106820in}{2.224865in}}%
\pgfpathcurveto{\pgfqpoint{1.106820in}{2.216629in}}{\pgfqpoint{1.110093in}{2.208729in}}{\pgfqpoint{1.115917in}{2.202905in}}%
\pgfpathcurveto{\pgfqpoint{1.121741in}{2.197081in}}{\pgfqpoint{1.129641in}{2.193808in}}{\pgfqpoint{1.137877in}{2.193808in}}%
\pgfpathclose%
\pgfusepath{stroke,fill}%
\end{pgfscope}%
\begin{pgfscope}%
\pgfpathrectangle{\pgfqpoint{0.100000in}{0.212622in}}{\pgfqpoint{3.696000in}{3.696000in}}%
\pgfusepath{clip}%
\pgfsetbuttcap%
\pgfsetroundjoin%
\definecolor{currentfill}{rgb}{0.121569,0.466667,0.705882}%
\pgfsetfillcolor{currentfill}%
\pgfsetfillopacity{0.746284}%
\pgfsetlinewidth{1.003750pt}%
\definecolor{currentstroke}{rgb}{0.121569,0.466667,0.705882}%
\pgfsetstrokecolor{currentstroke}%
\pgfsetstrokeopacity{0.746284}%
\pgfsetdash{}{0pt}%
\pgfpathmoveto{\pgfqpoint{1.138019in}{2.193195in}}%
\pgfpathcurveto{\pgfqpoint{1.146255in}{2.193195in}}{\pgfqpoint{1.154155in}{2.196467in}}{\pgfqpoint{1.159979in}{2.202291in}}%
\pgfpathcurveto{\pgfqpoint{1.165803in}{2.208115in}}{\pgfqpoint{1.169075in}{2.216015in}}{\pgfqpoint{1.169075in}{2.224251in}}%
\pgfpathcurveto{\pgfqpoint{1.169075in}{2.232488in}}{\pgfqpoint{1.165803in}{2.240388in}}{\pgfqpoint{1.159979in}{2.246212in}}%
\pgfpathcurveto{\pgfqpoint{1.154155in}{2.252036in}}{\pgfqpoint{1.146255in}{2.255308in}}{\pgfqpoint{1.138019in}{2.255308in}}%
\pgfpathcurveto{\pgfqpoint{1.129782in}{2.255308in}}{\pgfqpoint{1.121882in}{2.252036in}}{\pgfqpoint{1.116058in}{2.246212in}}%
\pgfpathcurveto{\pgfqpoint{1.110234in}{2.240388in}}{\pgfqpoint{1.106962in}{2.232488in}}{\pgfqpoint{1.106962in}{2.224251in}}%
\pgfpathcurveto{\pgfqpoint{1.106962in}{2.216015in}}{\pgfqpoint{1.110234in}{2.208115in}}{\pgfqpoint{1.116058in}{2.202291in}}%
\pgfpathcurveto{\pgfqpoint{1.121882in}{2.196467in}}{\pgfqpoint{1.129782in}{2.193195in}}{\pgfqpoint{1.138019in}{2.193195in}}%
\pgfpathclose%
\pgfusepath{stroke,fill}%
\end{pgfscope}%
\begin{pgfscope}%
\pgfpathrectangle{\pgfqpoint{0.100000in}{0.212622in}}{\pgfqpoint{3.696000in}{3.696000in}}%
\pgfusepath{clip}%
\pgfsetbuttcap%
\pgfsetroundjoin%
\definecolor{currentfill}{rgb}{0.121569,0.466667,0.705882}%
\pgfsetfillcolor{currentfill}%
\pgfsetfillopacity{0.746693}%
\pgfsetlinewidth{1.003750pt}%
\definecolor{currentstroke}{rgb}{0.121569,0.466667,0.705882}%
\pgfsetstrokecolor{currentstroke}%
\pgfsetstrokeopacity{0.746693}%
\pgfsetdash{}{0pt}%
\pgfpathmoveto{\pgfqpoint{1.138329in}{2.192093in}}%
\pgfpathcurveto{\pgfqpoint{1.146566in}{2.192093in}}{\pgfqpoint{1.154466in}{2.195366in}}{\pgfqpoint{1.160290in}{2.201190in}}%
\pgfpathcurveto{\pgfqpoint{1.166114in}{2.207013in}}{\pgfqpoint{1.169386in}{2.214914in}}{\pgfqpoint{1.169386in}{2.223150in}}%
\pgfpathcurveto{\pgfqpoint{1.169386in}{2.231386in}}{\pgfqpoint{1.166114in}{2.239286in}}{\pgfqpoint{1.160290in}{2.245110in}}%
\pgfpathcurveto{\pgfqpoint{1.154466in}{2.250934in}}{\pgfqpoint{1.146566in}{2.254206in}}{\pgfqpoint{1.138329in}{2.254206in}}%
\pgfpathcurveto{\pgfqpoint{1.130093in}{2.254206in}}{\pgfqpoint{1.122193in}{2.250934in}}{\pgfqpoint{1.116369in}{2.245110in}}%
\pgfpathcurveto{\pgfqpoint{1.110545in}{2.239286in}}{\pgfqpoint{1.107273in}{2.231386in}}{\pgfqpoint{1.107273in}{2.223150in}}%
\pgfpathcurveto{\pgfqpoint{1.107273in}{2.214914in}}{\pgfqpoint{1.110545in}{2.207013in}}{\pgfqpoint{1.116369in}{2.201190in}}%
\pgfpathcurveto{\pgfqpoint{1.122193in}{2.195366in}}{\pgfqpoint{1.130093in}{2.192093in}}{\pgfqpoint{1.138329in}{2.192093in}}%
\pgfpathclose%
\pgfusepath{stroke,fill}%
\end{pgfscope}%
\begin{pgfscope}%
\pgfpathrectangle{\pgfqpoint{0.100000in}{0.212622in}}{\pgfqpoint{3.696000in}{3.696000in}}%
\pgfusepath{clip}%
\pgfsetbuttcap%
\pgfsetroundjoin%
\definecolor{currentfill}{rgb}{0.121569,0.466667,0.705882}%
\pgfsetfillcolor{currentfill}%
\pgfsetfillopacity{0.747433}%
\pgfsetlinewidth{1.003750pt}%
\definecolor{currentstroke}{rgb}{0.121569,0.466667,0.705882}%
\pgfsetstrokecolor{currentstroke}%
\pgfsetstrokeopacity{0.747433}%
\pgfsetdash{}{0pt}%
\pgfpathmoveto{\pgfqpoint{1.139007in}{2.190130in}}%
\pgfpathcurveto{\pgfqpoint{1.147244in}{2.190130in}}{\pgfqpoint{1.155144in}{2.193402in}}{\pgfqpoint{1.160968in}{2.199226in}}%
\pgfpathcurveto{\pgfqpoint{1.166792in}{2.205050in}}{\pgfqpoint{1.170064in}{2.212950in}}{\pgfqpoint{1.170064in}{2.221186in}}%
\pgfpathcurveto{\pgfqpoint{1.170064in}{2.229422in}}{\pgfqpoint{1.166792in}{2.237322in}}{\pgfqpoint{1.160968in}{2.243146in}}%
\pgfpathcurveto{\pgfqpoint{1.155144in}{2.248970in}}{\pgfqpoint{1.147244in}{2.252243in}}{\pgfqpoint{1.139007in}{2.252243in}}%
\pgfpathcurveto{\pgfqpoint{1.130771in}{2.252243in}}{\pgfqpoint{1.122871in}{2.248970in}}{\pgfqpoint{1.117047in}{2.243146in}}%
\pgfpathcurveto{\pgfqpoint{1.111223in}{2.237322in}}{\pgfqpoint{1.107951in}{2.229422in}}{\pgfqpoint{1.107951in}{2.221186in}}%
\pgfpathcurveto{\pgfqpoint{1.107951in}{2.212950in}}{\pgfqpoint{1.111223in}{2.205050in}}{\pgfqpoint{1.117047in}{2.199226in}}%
\pgfpathcurveto{\pgfqpoint{1.122871in}{2.193402in}}{\pgfqpoint{1.130771in}{2.190130in}}{\pgfqpoint{1.139007in}{2.190130in}}%
\pgfpathclose%
\pgfusepath{stroke,fill}%
\end{pgfscope}%
\begin{pgfscope}%
\pgfpathrectangle{\pgfqpoint{0.100000in}{0.212622in}}{\pgfqpoint{3.696000in}{3.696000in}}%
\pgfusepath{clip}%
\pgfsetbuttcap%
\pgfsetroundjoin%
\definecolor{currentfill}{rgb}{0.121569,0.466667,0.705882}%
\pgfsetfillcolor{currentfill}%
\pgfsetfillopacity{0.747777}%
\pgfsetlinewidth{1.003750pt}%
\definecolor{currentstroke}{rgb}{0.121569,0.466667,0.705882}%
\pgfsetstrokecolor{currentstroke}%
\pgfsetstrokeopacity{0.747777}%
\pgfsetdash{}{0pt}%
\pgfpathmoveto{\pgfqpoint{2.457473in}{1.771669in}}%
\pgfpathcurveto{\pgfqpoint{2.465709in}{1.771669in}}{\pgfqpoint{2.473609in}{1.774941in}}{\pgfqpoint{2.479433in}{1.780765in}}%
\pgfpathcurveto{\pgfqpoint{2.485257in}{1.786589in}}{\pgfqpoint{2.488529in}{1.794489in}}{\pgfqpoint{2.488529in}{1.802726in}}%
\pgfpathcurveto{\pgfqpoint{2.488529in}{1.810962in}}{\pgfqpoint{2.485257in}{1.818862in}}{\pgfqpoint{2.479433in}{1.824686in}}%
\pgfpathcurveto{\pgfqpoint{2.473609in}{1.830510in}}{\pgfqpoint{2.465709in}{1.833782in}}{\pgfqpoint{2.457473in}{1.833782in}}%
\pgfpathcurveto{\pgfqpoint{2.449237in}{1.833782in}}{\pgfqpoint{2.441337in}{1.830510in}}{\pgfqpoint{2.435513in}{1.824686in}}%
\pgfpathcurveto{\pgfqpoint{2.429689in}{1.818862in}}{\pgfqpoint{2.426416in}{1.810962in}}{\pgfqpoint{2.426416in}{1.802726in}}%
\pgfpathcurveto{\pgfqpoint{2.426416in}{1.794489in}}{\pgfqpoint{2.429689in}{1.786589in}}{\pgfqpoint{2.435513in}{1.780765in}}%
\pgfpathcurveto{\pgfqpoint{2.441337in}{1.774941in}}{\pgfqpoint{2.449237in}{1.771669in}}{\pgfqpoint{2.457473in}{1.771669in}}%
\pgfpathclose%
\pgfusepath{stroke,fill}%
\end{pgfscope}%
\begin{pgfscope}%
\pgfpathrectangle{\pgfqpoint{0.100000in}{0.212622in}}{\pgfqpoint{3.696000in}{3.696000in}}%
\pgfusepath{clip}%
\pgfsetbuttcap%
\pgfsetroundjoin%
\definecolor{currentfill}{rgb}{0.121569,0.466667,0.705882}%
\pgfsetfillcolor{currentfill}%
\pgfsetfillopacity{0.748358}%
\pgfsetlinewidth{1.003750pt}%
\definecolor{currentstroke}{rgb}{0.121569,0.466667,0.705882}%
\pgfsetstrokecolor{currentstroke}%
\pgfsetstrokeopacity{0.748358}%
\pgfsetdash{}{0pt}%
\pgfpathmoveto{\pgfqpoint{1.099673in}{1.071330in}}%
\pgfpathcurveto{\pgfqpoint{1.107910in}{1.071330in}}{\pgfqpoint{1.115810in}{1.074602in}}{\pgfqpoint{1.121634in}{1.080426in}}%
\pgfpathcurveto{\pgfqpoint{1.127458in}{1.086250in}}{\pgfqpoint{1.130730in}{1.094150in}}{\pgfqpoint{1.130730in}{1.102387in}}%
\pgfpathcurveto{\pgfqpoint{1.130730in}{1.110623in}}{\pgfqpoint{1.127458in}{1.118523in}}{\pgfqpoint{1.121634in}{1.124347in}}%
\pgfpathcurveto{\pgfqpoint{1.115810in}{1.130171in}}{\pgfqpoint{1.107910in}{1.133443in}}{\pgfqpoint{1.099673in}{1.133443in}}%
\pgfpathcurveto{\pgfqpoint{1.091437in}{1.133443in}}{\pgfqpoint{1.083537in}{1.130171in}}{\pgfqpoint{1.077713in}{1.124347in}}%
\pgfpathcurveto{\pgfqpoint{1.071889in}{1.118523in}}{\pgfqpoint{1.068617in}{1.110623in}}{\pgfqpoint{1.068617in}{1.102387in}}%
\pgfpathcurveto{\pgfqpoint{1.068617in}{1.094150in}}{\pgfqpoint{1.071889in}{1.086250in}}{\pgfqpoint{1.077713in}{1.080426in}}%
\pgfpathcurveto{\pgfqpoint{1.083537in}{1.074602in}}{\pgfqpoint{1.091437in}{1.071330in}}{\pgfqpoint{1.099673in}{1.071330in}}%
\pgfpathclose%
\pgfusepath{stroke,fill}%
\end{pgfscope}%
\begin{pgfscope}%
\pgfpathrectangle{\pgfqpoint{0.100000in}{0.212622in}}{\pgfqpoint{3.696000in}{3.696000in}}%
\pgfusepath{clip}%
\pgfsetbuttcap%
\pgfsetroundjoin%
\definecolor{currentfill}{rgb}{0.121569,0.466667,0.705882}%
\pgfsetfillcolor{currentfill}%
\pgfsetfillopacity{0.748769}%
\pgfsetlinewidth{1.003750pt}%
\definecolor{currentstroke}{rgb}{0.121569,0.466667,0.705882}%
\pgfsetstrokecolor{currentstroke}%
\pgfsetstrokeopacity{0.748769}%
\pgfsetdash{}{0pt}%
\pgfpathmoveto{\pgfqpoint{1.140426in}{2.186612in}}%
\pgfpathcurveto{\pgfqpoint{1.148662in}{2.186612in}}{\pgfqpoint{1.156563in}{2.189884in}}{\pgfqpoint{1.162386in}{2.195708in}}%
\pgfpathcurveto{\pgfqpoint{1.168210in}{2.201532in}}{\pgfqpoint{1.171483in}{2.209432in}}{\pgfqpoint{1.171483in}{2.217668in}}%
\pgfpathcurveto{\pgfqpoint{1.171483in}{2.225904in}}{\pgfqpoint{1.168210in}{2.233804in}}{\pgfqpoint{1.162386in}{2.239628in}}%
\pgfpathcurveto{\pgfqpoint{1.156563in}{2.245452in}}{\pgfqpoint{1.148662in}{2.248725in}}{\pgfqpoint{1.140426in}{2.248725in}}%
\pgfpathcurveto{\pgfqpoint{1.132190in}{2.248725in}}{\pgfqpoint{1.124290in}{2.245452in}}{\pgfqpoint{1.118466in}{2.239628in}}%
\pgfpathcurveto{\pgfqpoint{1.112642in}{2.233804in}}{\pgfqpoint{1.109370in}{2.225904in}}{\pgfqpoint{1.109370in}{2.217668in}}%
\pgfpathcurveto{\pgfqpoint{1.109370in}{2.209432in}}{\pgfqpoint{1.112642in}{2.201532in}}{\pgfqpoint{1.118466in}{2.195708in}}%
\pgfpathcurveto{\pgfqpoint{1.124290in}{2.189884in}}{\pgfqpoint{1.132190in}{2.186612in}}{\pgfqpoint{1.140426in}{2.186612in}}%
\pgfpathclose%
\pgfusepath{stroke,fill}%
\end{pgfscope}%
\begin{pgfscope}%
\pgfpathrectangle{\pgfqpoint{0.100000in}{0.212622in}}{\pgfqpoint{3.696000in}{3.696000in}}%
\pgfusepath{clip}%
\pgfsetbuttcap%
\pgfsetroundjoin%
\definecolor{currentfill}{rgb}{0.121569,0.466667,0.705882}%
\pgfsetfillcolor{currentfill}%
\pgfsetfillopacity{0.750639}%
\pgfsetlinewidth{1.003750pt}%
\definecolor{currentstroke}{rgb}{0.121569,0.466667,0.705882}%
\pgfsetstrokecolor{currentstroke}%
\pgfsetstrokeopacity{0.750639}%
\pgfsetdash{}{0pt}%
\pgfpathmoveto{\pgfqpoint{1.108438in}{1.070114in}}%
\pgfpathcurveto{\pgfqpoint{1.116675in}{1.070114in}}{\pgfqpoint{1.124575in}{1.073386in}}{\pgfqpoint{1.130399in}{1.079210in}}%
\pgfpathcurveto{\pgfqpoint{1.136223in}{1.085034in}}{\pgfqpoint{1.139495in}{1.092934in}}{\pgfqpoint{1.139495in}{1.101170in}}%
\pgfpathcurveto{\pgfqpoint{1.139495in}{1.109406in}}{\pgfqpoint{1.136223in}{1.117306in}}{\pgfqpoint{1.130399in}{1.123130in}}%
\pgfpathcurveto{\pgfqpoint{1.124575in}{1.128954in}}{\pgfqpoint{1.116675in}{1.132227in}}{\pgfqpoint{1.108438in}{1.132227in}}%
\pgfpathcurveto{\pgfqpoint{1.100202in}{1.132227in}}{\pgfqpoint{1.092302in}{1.128954in}}{\pgfqpoint{1.086478in}{1.123130in}}%
\pgfpathcurveto{\pgfqpoint{1.080654in}{1.117306in}}{\pgfqpoint{1.077382in}{1.109406in}}{\pgfqpoint{1.077382in}{1.101170in}}%
\pgfpathcurveto{\pgfqpoint{1.077382in}{1.092934in}}{\pgfqpoint{1.080654in}{1.085034in}}{\pgfqpoint{1.086478in}{1.079210in}}%
\pgfpathcurveto{\pgfqpoint{1.092302in}{1.073386in}}{\pgfqpoint{1.100202in}{1.070114in}}{\pgfqpoint{1.108438in}{1.070114in}}%
\pgfpathclose%
\pgfusepath{stroke,fill}%
\end{pgfscope}%
\begin{pgfscope}%
\pgfpathrectangle{\pgfqpoint{0.100000in}{0.212622in}}{\pgfqpoint{3.696000in}{3.696000in}}%
\pgfusepath{clip}%
\pgfsetbuttcap%
\pgfsetroundjoin%
\definecolor{currentfill}{rgb}{0.121569,0.466667,0.705882}%
\pgfsetfillcolor{currentfill}%
\pgfsetfillopacity{0.751191}%
\pgfsetlinewidth{1.003750pt}%
\definecolor{currentstroke}{rgb}{0.121569,0.466667,0.705882}%
\pgfsetstrokecolor{currentstroke}%
\pgfsetstrokeopacity{0.751191}%
\pgfsetdash{}{0pt}%
\pgfpathmoveto{\pgfqpoint{1.143425in}{2.180445in}}%
\pgfpathcurveto{\pgfqpoint{1.151661in}{2.180445in}}{\pgfqpoint{1.159561in}{2.183718in}}{\pgfqpoint{1.165385in}{2.189542in}}%
\pgfpathcurveto{\pgfqpoint{1.171209in}{2.195366in}}{\pgfqpoint{1.174482in}{2.203266in}}{\pgfqpoint{1.174482in}{2.211502in}}%
\pgfpathcurveto{\pgfqpoint{1.174482in}{2.219738in}}{\pgfqpoint{1.171209in}{2.227638in}}{\pgfqpoint{1.165385in}{2.233462in}}%
\pgfpathcurveto{\pgfqpoint{1.159561in}{2.239286in}}{\pgfqpoint{1.151661in}{2.242558in}}{\pgfqpoint{1.143425in}{2.242558in}}%
\pgfpathcurveto{\pgfqpoint{1.135189in}{2.242558in}}{\pgfqpoint{1.127289in}{2.239286in}}{\pgfqpoint{1.121465in}{2.233462in}}%
\pgfpathcurveto{\pgfqpoint{1.115641in}{2.227638in}}{\pgfqpoint{1.112369in}{2.219738in}}{\pgfqpoint{1.112369in}{2.211502in}}%
\pgfpathcurveto{\pgfqpoint{1.112369in}{2.203266in}}{\pgfqpoint{1.115641in}{2.195366in}}{\pgfqpoint{1.121465in}{2.189542in}}%
\pgfpathcurveto{\pgfqpoint{1.127289in}{2.183718in}}{\pgfqpoint{1.135189in}{2.180445in}}{\pgfqpoint{1.143425in}{2.180445in}}%
\pgfpathclose%
\pgfusepath{stroke,fill}%
\end{pgfscope}%
\begin{pgfscope}%
\pgfpathrectangle{\pgfqpoint{0.100000in}{0.212622in}}{\pgfqpoint{3.696000in}{3.696000in}}%
\pgfusepath{clip}%
\pgfsetbuttcap%
\pgfsetroundjoin%
\definecolor{currentfill}{rgb}{0.121569,0.466667,0.705882}%
\pgfsetfillcolor{currentfill}%
\pgfsetfillopacity{0.751537}%
\pgfsetlinewidth{1.003750pt}%
\definecolor{currentstroke}{rgb}{0.121569,0.466667,0.705882}%
\pgfsetstrokecolor{currentstroke}%
\pgfsetstrokeopacity{0.751537}%
\pgfsetdash{}{0pt}%
\pgfpathmoveto{\pgfqpoint{2.460871in}{1.758149in}}%
\pgfpathcurveto{\pgfqpoint{2.469107in}{1.758149in}}{\pgfqpoint{2.477007in}{1.761421in}}{\pgfqpoint{2.482831in}{1.767245in}}%
\pgfpathcurveto{\pgfqpoint{2.488655in}{1.773069in}}{\pgfqpoint{2.491927in}{1.780969in}}{\pgfqpoint{2.491927in}{1.789205in}}%
\pgfpathcurveto{\pgfqpoint{2.491927in}{1.797442in}}{\pgfqpoint{2.488655in}{1.805342in}}{\pgfqpoint{2.482831in}{1.811166in}}%
\pgfpathcurveto{\pgfqpoint{2.477007in}{1.816990in}}{\pgfqpoint{2.469107in}{1.820262in}}{\pgfqpoint{2.460871in}{1.820262in}}%
\pgfpathcurveto{\pgfqpoint{2.452635in}{1.820262in}}{\pgfqpoint{2.444735in}{1.816990in}}{\pgfqpoint{2.438911in}{1.811166in}}%
\pgfpathcurveto{\pgfqpoint{2.433087in}{1.805342in}}{\pgfqpoint{2.429814in}{1.797442in}}{\pgfqpoint{2.429814in}{1.789205in}}%
\pgfpathcurveto{\pgfqpoint{2.429814in}{1.780969in}}{\pgfqpoint{2.433087in}{1.773069in}}{\pgfqpoint{2.438911in}{1.767245in}}%
\pgfpathcurveto{\pgfqpoint{2.444735in}{1.761421in}}{\pgfqpoint{2.452635in}{1.758149in}}{\pgfqpoint{2.460871in}{1.758149in}}%
\pgfpathclose%
\pgfusepath{stroke,fill}%
\end{pgfscope}%
\begin{pgfscope}%
\pgfpathrectangle{\pgfqpoint{0.100000in}{0.212622in}}{\pgfqpoint{3.696000in}{3.696000in}}%
\pgfusepath{clip}%
\pgfsetbuttcap%
\pgfsetroundjoin%
\definecolor{currentfill}{rgb}{0.121569,0.466667,0.705882}%
\pgfsetfillcolor{currentfill}%
\pgfsetfillopacity{0.754509}%
\pgfsetlinewidth{1.003750pt}%
\definecolor{currentstroke}{rgb}{0.121569,0.466667,0.705882}%
\pgfsetstrokecolor{currentstroke}%
\pgfsetstrokeopacity{0.754509}%
\pgfsetdash{}{0pt}%
\pgfpathmoveto{\pgfqpoint{1.124631in}{1.067700in}}%
\pgfpathcurveto{\pgfqpoint{1.132867in}{1.067700in}}{\pgfqpoint{1.140767in}{1.070972in}}{\pgfqpoint{1.146591in}{1.076796in}}%
\pgfpathcurveto{\pgfqpoint{1.152415in}{1.082620in}}{\pgfqpoint{1.155687in}{1.090520in}}{\pgfqpoint{1.155687in}{1.098756in}}%
\pgfpathcurveto{\pgfqpoint{1.155687in}{1.106992in}}{\pgfqpoint{1.152415in}{1.114892in}}{\pgfqpoint{1.146591in}{1.120716in}}%
\pgfpathcurveto{\pgfqpoint{1.140767in}{1.126540in}}{\pgfqpoint{1.132867in}{1.129813in}}{\pgfqpoint{1.124631in}{1.129813in}}%
\pgfpathcurveto{\pgfqpoint{1.116394in}{1.129813in}}{\pgfqpoint{1.108494in}{1.126540in}}{\pgfqpoint{1.102670in}{1.120716in}}%
\pgfpathcurveto{\pgfqpoint{1.096846in}{1.114892in}}{\pgfqpoint{1.093574in}{1.106992in}}{\pgfqpoint{1.093574in}{1.098756in}}%
\pgfpathcurveto{\pgfqpoint{1.093574in}{1.090520in}}{\pgfqpoint{1.096846in}{1.082620in}}{\pgfqpoint{1.102670in}{1.076796in}}%
\pgfpathcurveto{\pgfqpoint{1.108494in}{1.070972in}}{\pgfqpoint{1.116394in}{1.067700in}}{\pgfqpoint{1.124631in}{1.067700in}}%
\pgfpathclose%
\pgfusepath{stroke,fill}%
\end{pgfscope}%
\begin{pgfscope}%
\pgfpathrectangle{\pgfqpoint{0.100000in}{0.212622in}}{\pgfqpoint{3.696000in}{3.696000in}}%
\pgfusepath{clip}%
\pgfsetbuttcap%
\pgfsetroundjoin%
\definecolor{currentfill}{rgb}{0.121569,0.466667,0.705882}%
\pgfsetfillcolor{currentfill}%
\pgfsetfillopacity{0.755576}%
\pgfsetlinewidth{1.003750pt}%
\definecolor{currentstroke}{rgb}{0.121569,0.466667,0.705882}%
\pgfsetstrokecolor{currentstroke}%
\pgfsetstrokeopacity{0.755576}%
\pgfsetdash{}{0pt}%
\pgfpathmoveto{\pgfqpoint{1.149654in}{2.169729in}}%
\pgfpathcurveto{\pgfqpoint{1.157891in}{2.169729in}}{\pgfqpoint{1.165791in}{2.173001in}}{\pgfqpoint{1.171615in}{2.178825in}}%
\pgfpathcurveto{\pgfqpoint{1.177439in}{2.184649in}}{\pgfqpoint{1.180711in}{2.192549in}}{\pgfqpoint{1.180711in}{2.200786in}}%
\pgfpathcurveto{\pgfqpoint{1.180711in}{2.209022in}}{\pgfqpoint{1.177439in}{2.216922in}}{\pgfqpoint{1.171615in}{2.222746in}}%
\pgfpathcurveto{\pgfqpoint{1.165791in}{2.228570in}}{\pgfqpoint{1.157891in}{2.231842in}}{\pgfqpoint{1.149654in}{2.231842in}}%
\pgfpathcurveto{\pgfqpoint{1.141418in}{2.231842in}}{\pgfqpoint{1.133518in}{2.228570in}}{\pgfqpoint{1.127694in}{2.222746in}}%
\pgfpathcurveto{\pgfqpoint{1.121870in}{2.216922in}}{\pgfqpoint{1.118598in}{2.209022in}}{\pgfqpoint{1.118598in}{2.200786in}}%
\pgfpathcurveto{\pgfqpoint{1.118598in}{2.192549in}}{\pgfqpoint{1.121870in}{2.184649in}}{\pgfqpoint{1.127694in}{2.178825in}}%
\pgfpathcurveto{\pgfqpoint{1.133518in}{2.173001in}}{\pgfqpoint{1.141418in}{2.169729in}}{\pgfqpoint{1.149654in}{2.169729in}}%
\pgfpathclose%
\pgfusepath{stroke,fill}%
\end{pgfscope}%
\begin{pgfscope}%
\pgfpathrectangle{\pgfqpoint{0.100000in}{0.212622in}}{\pgfqpoint{3.696000in}{3.696000in}}%
\pgfusepath{clip}%
\pgfsetbuttcap%
\pgfsetroundjoin%
\definecolor{currentfill}{rgb}{0.121569,0.466667,0.705882}%
\pgfsetfillcolor{currentfill}%
\pgfsetfillopacity{0.756275}%
\pgfsetlinewidth{1.003750pt}%
\definecolor{currentstroke}{rgb}{0.121569,0.466667,0.705882}%
\pgfsetstrokecolor{currentstroke}%
\pgfsetstrokeopacity{0.756275}%
\pgfsetdash{}{0pt}%
\pgfpathmoveto{\pgfqpoint{2.465593in}{1.740965in}}%
\pgfpathcurveto{\pgfqpoint{2.473829in}{1.740965in}}{\pgfqpoint{2.481729in}{1.744237in}}{\pgfqpoint{2.487553in}{1.750061in}}%
\pgfpathcurveto{\pgfqpoint{2.493377in}{1.755885in}}{\pgfqpoint{2.496650in}{1.763785in}}{\pgfqpoint{2.496650in}{1.772021in}}%
\pgfpathcurveto{\pgfqpoint{2.496650in}{1.780257in}}{\pgfqpoint{2.493377in}{1.788157in}}{\pgfqpoint{2.487553in}{1.793981in}}%
\pgfpathcurveto{\pgfqpoint{2.481729in}{1.799805in}}{\pgfqpoint{2.473829in}{1.803078in}}{\pgfqpoint{2.465593in}{1.803078in}}%
\pgfpathcurveto{\pgfqpoint{2.457357in}{1.803078in}}{\pgfqpoint{2.449457in}{1.799805in}}{\pgfqpoint{2.443633in}{1.793981in}}%
\pgfpathcurveto{\pgfqpoint{2.437809in}{1.788157in}}{\pgfqpoint{2.434537in}{1.780257in}}{\pgfqpoint{2.434537in}{1.772021in}}%
\pgfpathcurveto{\pgfqpoint{2.434537in}{1.763785in}}{\pgfqpoint{2.437809in}{1.755885in}}{\pgfqpoint{2.443633in}{1.750061in}}%
\pgfpathcurveto{\pgfqpoint{2.449457in}{1.744237in}}{\pgfqpoint{2.457357in}{1.740965in}}{\pgfqpoint{2.465593in}{1.740965in}}%
\pgfpathclose%
\pgfusepath{stroke,fill}%
\end{pgfscope}%
\begin{pgfscope}%
\pgfpathrectangle{\pgfqpoint{0.100000in}{0.212622in}}{\pgfqpoint{3.696000in}{3.696000in}}%
\pgfusepath{clip}%
\pgfsetbuttcap%
\pgfsetroundjoin%
\definecolor{currentfill}{rgb}{0.121569,0.466667,0.705882}%
\pgfsetfillcolor{currentfill}%
\pgfsetfillopacity{0.759397}%
\pgfsetlinewidth{1.003750pt}%
\definecolor{currentstroke}{rgb}{0.121569,0.466667,0.705882}%
\pgfsetstrokecolor{currentstroke}%
\pgfsetstrokeopacity{0.759397}%
\pgfsetdash{}{0pt}%
\pgfpathmoveto{\pgfqpoint{1.155853in}{2.160720in}}%
\pgfpathcurveto{\pgfqpoint{1.164089in}{2.160720in}}{\pgfqpoint{1.171989in}{2.163992in}}{\pgfqpoint{1.177813in}{2.169816in}}%
\pgfpathcurveto{\pgfqpoint{1.183637in}{2.175640in}}{\pgfqpoint{1.186909in}{2.183540in}}{\pgfqpoint{1.186909in}{2.191776in}}%
\pgfpathcurveto{\pgfqpoint{1.186909in}{2.200013in}}{\pgfqpoint{1.183637in}{2.207913in}}{\pgfqpoint{1.177813in}{2.213737in}}%
\pgfpathcurveto{\pgfqpoint{1.171989in}{2.219561in}}{\pgfqpoint{1.164089in}{2.222833in}}{\pgfqpoint{1.155853in}{2.222833in}}%
\pgfpathcurveto{\pgfqpoint{1.147616in}{2.222833in}}{\pgfqpoint{1.139716in}{2.219561in}}{\pgfqpoint{1.133892in}{2.213737in}}%
\pgfpathcurveto{\pgfqpoint{1.128068in}{2.207913in}}{\pgfqpoint{1.124796in}{2.200013in}}{\pgfqpoint{1.124796in}{2.191776in}}%
\pgfpathcurveto{\pgfqpoint{1.124796in}{2.183540in}}{\pgfqpoint{1.128068in}{2.175640in}}{\pgfqpoint{1.133892in}{2.169816in}}%
\pgfpathcurveto{\pgfqpoint{1.139716in}{2.163992in}}{\pgfqpoint{1.147616in}{2.160720in}}{\pgfqpoint{1.155853in}{2.160720in}}%
\pgfpathclose%
\pgfusepath{stroke,fill}%
\end{pgfscope}%
\begin{pgfscope}%
\pgfpathrectangle{\pgfqpoint{0.100000in}{0.212622in}}{\pgfqpoint{3.696000in}{3.696000in}}%
\pgfusepath{clip}%
\pgfsetbuttcap%
\pgfsetroundjoin%
\definecolor{currentfill}{rgb}{0.121569,0.466667,0.705882}%
\pgfsetfillcolor{currentfill}%
\pgfsetfillopacity{0.761425}%
\pgfsetlinewidth{1.003750pt}%
\definecolor{currentstroke}{rgb}{0.121569,0.466667,0.705882}%
\pgfsetstrokecolor{currentstroke}%
\pgfsetstrokeopacity{0.761425}%
\pgfsetdash{}{0pt}%
\pgfpathmoveto{\pgfqpoint{1.154149in}{1.062949in}}%
\pgfpathcurveto{\pgfqpoint{1.162385in}{1.062949in}}{\pgfqpoint{1.170285in}{1.066221in}}{\pgfqpoint{1.176109in}{1.072045in}}%
\pgfpathcurveto{\pgfqpoint{1.181933in}{1.077869in}}{\pgfqpoint{1.185205in}{1.085769in}}{\pgfqpoint{1.185205in}{1.094006in}}%
\pgfpathcurveto{\pgfqpoint{1.185205in}{1.102242in}}{\pgfqpoint{1.181933in}{1.110142in}}{\pgfqpoint{1.176109in}{1.115966in}}%
\pgfpathcurveto{\pgfqpoint{1.170285in}{1.121790in}}{\pgfqpoint{1.162385in}{1.125062in}}{\pgfqpoint{1.154149in}{1.125062in}}%
\pgfpathcurveto{\pgfqpoint{1.145912in}{1.125062in}}{\pgfqpoint{1.138012in}{1.121790in}}{\pgfqpoint{1.132188in}{1.115966in}}%
\pgfpathcurveto{\pgfqpoint{1.126365in}{1.110142in}}{\pgfqpoint{1.123092in}{1.102242in}}{\pgfqpoint{1.123092in}{1.094006in}}%
\pgfpathcurveto{\pgfqpoint{1.123092in}{1.085769in}}{\pgfqpoint{1.126365in}{1.077869in}}{\pgfqpoint{1.132188in}{1.072045in}}%
\pgfpathcurveto{\pgfqpoint{1.138012in}{1.066221in}}{\pgfqpoint{1.145912in}{1.062949in}}{\pgfqpoint{1.154149in}{1.062949in}}%
\pgfpathclose%
\pgfusepath{stroke,fill}%
\end{pgfscope}%
\begin{pgfscope}%
\pgfpathrectangle{\pgfqpoint{0.100000in}{0.212622in}}{\pgfqpoint{3.696000in}{3.696000in}}%
\pgfusepath{clip}%
\pgfsetbuttcap%
\pgfsetroundjoin%
\definecolor{currentfill}{rgb}{0.121569,0.466667,0.705882}%
\pgfsetfillcolor{currentfill}%
\pgfsetfillopacity{0.762016}%
\pgfsetlinewidth{1.003750pt}%
\definecolor{currentstroke}{rgb}{0.121569,0.466667,0.705882}%
\pgfsetstrokecolor{currentstroke}%
\pgfsetstrokeopacity{0.762016}%
\pgfsetdash{}{0pt}%
\pgfpathmoveto{\pgfqpoint{2.471351in}{1.720371in}}%
\pgfpathcurveto{\pgfqpoint{2.479587in}{1.720371in}}{\pgfqpoint{2.487487in}{1.723643in}}{\pgfqpoint{2.493311in}{1.729467in}}%
\pgfpathcurveto{\pgfqpoint{2.499135in}{1.735291in}}{\pgfqpoint{2.502408in}{1.743191in}}{\pgfqpoint{2.502408in}{1.751427in}}%
\pgfpathcurveto{\pgfqpoint{2.502408in}{1.759663in}}{\pgfqpoint{2.499135in}{1.767563in}}{\pgfqpoint{2.493311in}{1.773387in}}%
\pgfpathcurveto{\pgfqpoint{2.487487in}{1.779211in}}{\pgfqpoint{2.479587in}{1.782484in}}{\pgfqpoint{2.471351in}{1.782484in}}%
\pgfpathcurveto{\pgfqpoint{2.463115in}{1.782484in}}{\pgfqpoint{2.455215in}{1.779211in}}{\pgfqpoint{2.449391in}{1.773387in}}%
\pgfpathcurveto{\pgfqpoint{2.443567in}{1.767563in}}{\pgfqpoint{2.440295in}{1.759663in}}{\pgfqpoint{2.440295in}{1.751427in}}%
\pgfpathcurveto{\pgfqpoint{2.440295in}{1.743191in}}{\pgfqpoint{2.443567in}{1.735291in}}{\pgfqpoint{2.449391in}{1.729467in}}%
\pgfpathcurveto{\pgfqpoint{2.455215in}{1.723643in}}{\pgfqpoint{2.463115in}{1.720371in}}{\pgfqpoint{2.471351in}{1.720371in}}%
\pgfpathclose%
\pgfusepath{stroke,fill}%
\end{pgfscope}%
\begin{pgfscope}%
\pgfpathrectangle{\pgfqpoint{0.100000in}{0.212622in}}{\pgfqpoint{3.696000in}{3.696000in}}%
\pgfusepath{clip}%
\pgfsetbuttcap%
\pgfsetroundjoin%
\definecolor{currentfill}{rgb}{0.121569,0.466667,0.705882}%
\pgfsetfillcolor{currentfill}%
\pgfsetfillopacity{0.762603}%
\pgfsetlinewidth{1.003750pt}%
\definecolor{currentstroke}{rgb}{0.121569,0.466667,0.705882}%
\pgfsetstrokecolor{currentstroke}%
\pgfsetstrokeopacity{0.762603}%
\pgfsetdash{}{0pt}%
\pgfpathmoveto{\pgfqpoint{1.161618in}{2.153816in}}%
\pgfpathcurveto{\pgfqpoint{1.169855in}{2.153816in}}{\pgfqpoint{1.177755in}{2.157089in}}{\pgfqpoint{1.183579in}{2.162912in}}%
\pgfpathcurveto{\pgfqpoint{1.189402in}{2.168736in}}{\pgfqpoint{1.192675in}{2.176636in}}{\pgfqpoint{1.192675in}{2.184873in}}%
\pgfpathcurveto{\pgfqpoint{1.192675in}{2.193109in}}{\pgfqpoint{1.189402in}{2.201009in}}{\pgfqpoint{1.183579in}{2.206833in}}%
\pgfpathcurveto{\pgfqpoint{1.177755in}{2.212657in}}{\pgfqpoint{1.169855in}{2.215929in}}{\pgfqpoint{1.161618in}{2.215929in}}%
\pgfpathcurveto{\pgfqpoint{1.153382in}{2.215929in}}{\pgfqpoint{1.145482in}{2.212657in}}{\pgfqpoint{1.139658in}{2.206833in}}%
\pgfpathcurveto{\pgfqpoint{1.133834in}{2.201009in}}{\pgfqpoint{1.130562in}{2.193109in}}{\pgfqpoint{1.130562in}{2.184873in}}%
\pgfpathcurveto{\pgfqpoint{1.130562in}{2.176636in}}{\pgfqpoint{1.133834in}{2.168736in}}{\pgfqpoint{1.139658in}{2.162912in}}%
\pgfpathcurveto{\pgfqpoint{1.145482in}{2.157089in}}{\pgfqpoint{1.153382in}{2.153816in}}{\pgfqpoint{1.161618in}{2.153816in}}%
\pgfpathclose%
\pgfusepath{stroke,fill}%
\end{pgfscope}%
\begin{pgfscope}%
\pgfpathrectangle{\pgfqpoint{0.100000in}{0.212622in}}{\pgfqpoint{3.696000in}{3.696000in}}%
\pgfusepath{clip}%
\pgfsetbuttcap%
\pgfsetroundjoin%
\definecolor{currentfill}{rgb}{0.121569,0.466667,0.705882}%
\pgfsetfillcolor{currentfill}%
\pgfsetfillopacity{0.767764}%
\pgfsetlinewidth{1.003750pt}%
\definecolor{currentstroke}{rgb}{0.121569,0.466667,0.705882}%
\pgfsetstrokecolor{currentstroke}%
\pgfsetstrokeopacity{0.767764}%
\pgfsetdash{}{0pt}%
\pgfpathmoveto{\pgfqpoint{1.181694in}{1.058428in}}%
\pgfpathcurveto{\pgfqpoint{1.189930in}{1.058428in}}{\pgfqpoint{1.197830in}{1.061700in}}{\pgfqpoint{1.203654in}{1.067524in}}%
\pgfpathcurveto{\pgfqpoint{1.209478in}{1.073348in}}{\pgfqpoint{1.212750in}{1.081248in}}{\pgfqpoint{1.212750in}{1.089484in}}%
\pgfpathcurveto{\pgfqpoint{1.212750in}{1.097721in}}{\pgfqpoint{1.209478in}{1.105621in}}{\pgfqpoint{1.203654in}{1.111445in}}%
\pgfpathcurveto{\pgfqpoint{1.197830in}{1.117268in}}{\pgfqpoint{1.189930in}{1.120541in}}{\pgfqpoint{1.181694in}{1.120541in}}%
\pgfpathcurveto{\pgfqpoint{1.173458in}{1.120541in}}{\pgfqpoint{1.165558in}{1.117268in}}{\pgfqpoint{1.159734in}{1.111445in}}%
\pgfpathcurveto{\pgfqpoint{1.153910in}{1.105621in}}{\pgfqpoint{1.150637in}{1.097721in}}{\pgfqpoint{1.150637in}{1.089484in}}%
\pgfpathcurveto{\pgfqpoint{1.150637in}{1.081248in}}{\pgfqpoint{1.153910in}{1.073348in}}{\pgfqpoint{1.159734in}{1.067524in}}%
\pgfpathcurveto{\pgfqpoint{1.165558in}{1.061700in}}{\pgfqpoint{1.173458in}{1.058428in}}{\pgfqpoint{1.181694in}{1.058428in}}%
\pgfpathclose%
\pgfusepath{stroke,fill}%
\end{pgfscope}%
\begin{pgfscope}%
\pgfpathrectangle{\pgfqpoint{0.100000in}{0.212622in}}{\pgfqpoint{3.696000in}{3.696000in}}%
\pgfusepath{clip}%
\pgfsetbuttcap%
\pgfsetroundjoin%
\definecolor{currentfill}{rgb}{0.121569,0.466667,0.705882}%
\pgfsetfillcolor{currentfill}%
\pgfsetfillopacity{0.768496}%
\pgfsetlinewidth{1.003750pt}%
\definecolor{currentstroke}{rgb}{0.121569,0.466667,0.705882}%
\pgfsetstrokecolor{currentstroke}%
\pgfsetstrokeopacity{0.768496}%
\pgfsetdash{}{0pt}%
\pgfpathmoveto{\pgfqpoint{1.172930in}{2.142449in}}%
\pgfpathcurveto{\pgfqpoint{1.181166in}{2.142449in}}{\pgfqpoint{1.189066in}{2.145721in}}{\pgfqpoint{1.194890in}{2.151545in}}%
\pgfpathcurveto{\pgfqpoint{1.200714in}{2.157369in}}{\pgfqpoint{1.203986in}{2.165269in}}{\pgfqpoint{1.203986in}{2.173505in}}%
\pgfpathcurveto{\pgfqpoint{1.203986in}{2.181741in}}{\pgfqpoint{1.200714in}{2.189641in}}{\pgfqpoint{1.194890in}{2.195465in}}%
\pgfpathcurveto{\pgfqpoint{1.189066in}{2.201289in}}{\pgfqpoint{1.181166in}{2.204562in}}{\pgfqpoint{1.172930in}{2.204562in}}%
\pgfpathcurveto{\pgfqpoint{1.164693in}{2.204562in}}{\pgfqpoint{1.156793in}{2.201289in}}{\pgfqpoint{1.150969in}{2.195465in}}%
\pgfpathcurveto{\pgfqpoint{1.145146in}{2.189641in}}{\pgfqpoint{1.141873in}{2.181741in}}{\pgfqpoint{1.141873in}{2.173505in}}%
\pgfpathcurveto{\pgfqpoint{1.141873in}{2.165269in}}{\pgfqpoint{1.145146in}{2.157369in}}{\pgfqpoint{1.150969in}{2.151545in}}%
\pgfpathcurveto{\pgfqpoint{1.156793in}{2.145721in}}{\pgfqpoint{1.164693in}{2.142449in}}{\pgfqpoint{1.172930in}{2.142449in}}%
\pgfpathclose%
\pgfusepath{stroke,fill}%
\end{pgfscope}%
\begin{pgfscope}%
\pgfpathrectangle{\pgfqpoint{0.100000in}{0.212622in}}{\pgfqpoint{3.696000in}{3.696000in}}%
\pgfusepath{clip}%
\pgfsetbuttcap%
\pgfsetroundjoin%
\definecolor{currentfill}{rgb}{0.121569,0.466667,0.705882}%
\pgfsetfillcolor{currentfill}%
\pgfsetfillopacity{0.769039}%
\pgfsetlinewidth{1.003750pt}%
\definecolor{currentstroke}{rgb}{0.121569,0.466667,0.705882}%
\pgfsetstrokecolor{currentstroke}%
\pgfsetstrokeopacity{0.769039}%
\pgfsetdash{}{0pt}%
\pgfpathmoveto{\pgfqpoint{2.476666in}{1.696483in}}%
\pgfpathcurveto{\pgfqpoint{2.484902in}{1.696483in}}{\pgfqpoint{2.492802in}{1.699756in}}{\pgfqpoint{2.498626in}{1.705580in}}%
\pgfpathcurveto{\pgfqpoint{2.504450in}{1.711404in}}{\pgfqpoint{2.507722in}{1.719304in}}{\pgfqpoint{2.507722in}{1.727540in}}%
\pgfpathcurveto{\pgfqpoint{2.507722in}{1.735776in}}{\pgfqpoint{2.504450in}{1.743676in}}{\pgfqpoint{2.498626in}{1.749500in}}%
\pgfpathcurveto{\pgfqpoint{2.492802in}{1.755324in}}{\pgfqpoint{2.484902in}{1.758596in}}{\pgfqpoint{2.476666in}{1.758596in}}%
\pgfpathcurveto{\pgfqpoint{2.468430in}{1.758596in}}{\pgfqpoint{2.460530in}{1.755324in}}{\pgfqpoint{2.454706in}{1.749500in}}%
\pgfpathcurveto{\pgfqpoint{2.448882in}{1.743676in}}{\pgfqpoint{2.445609in}{1.735776in}}{\pgfqpoint{2.445609in}{1.727540in}}%
\pgfpathcurveto{\pgfqpoint{2.445609in}{1.719304in}}{\pgfqpoint{2.448882in}{1.711404in}}{\pgfqpoint{2.454706in}{1.705580in}}%
\pgfpathcurveto{\pgfqpoint{2.460530in}{1.699756in}}{\pgfqpoint{2.468430in}{1.696483in}}{\pgfqpoint{2.476666in}{1.696483in}}%
\pgfpathclose%
\pgfusepath{stroke,fill}%
\end{pgfscope}%
\begin{pgfscope}%
\pgfpathrectangle{\pgfqpoint{0.100000in}{0.212622in}}{\pgfqpoint{3.696000in}{3.696000in}}%
\pgfusepath{clip}%
\pgfsetbuttcap%
\pgfsetroundjoin%
\definecolor{currentfill}{rgb}{0.121569,0.466667,0.705882}%
\pgfsetfillcolor{currentfill}%
\pgfsetfillopacity{0.772514}%
\pgfsetlinewidth{1.003750pt}%
\definecolor{currentstroke}{rgb}{0.121569,0.466667,0.705882}%
\pgfsetstrokecolor{currentstroke}%
\pgfsetstrokeopacity{0.772514}%
\pgfsetdash{}{0pt}%
\pgfpathmoveto{\pgfqpoint{1.204026in}{1.054214in}}%
\pgfpathcurveto{\pgfqpoint{1.212263in}{1.054214in}}{\pgfqpoint{1.220163in}{1.057486in}}{\pgfqpoint{1.225987in}{1.063310in}}%
\pgfpathcurveto{\pgfqpoint{1.231811in}{1.069134in}}{\pgfqpoint{1.235083in}{1.077034in}}{\pgfqpoint{1.235083in}{1.085271in}}%
\pgfpathcurveto{\pgfqpoint{1.235083in}{1.093507in}}{\pgfqpoint{1.231811in}{1.101407in}}{\pgfqpoint{1.225987in}{1.107231in}}%
\pgfpathcurveto{\pgfqpoint{1.220163in}{1.113055in}}{\pgfqpoint{1.212263in}{1.116327in}}{\pgfqpoint{1.204026in}{1.116327in}}%
\pgfpathcurveto{\pgfqpoint{1.195790in}{1.116327in}}{\pgfqpoint{1.187890in}{1.113055in}}{\pgfqpoint{1.182066in}{1.107231in}}%
\pgfpathcurveto{\pgfqpoint{1.176242in}{1.101407in}}{\pgfqpoint{1.172970in}{1.093507in}}{\pgfqpoint{1.172970in}{1.085271in}}%
\pgfpathcurveto{\pgfqpoint{1.172970in}{1.077034in}}{\pgfqpoint{1.176242in}{1.069134in}}{\pgfqpoint{1.182066in}{1.063310in}}%
\pgfpathcurveto{\pgfqpoint{1.187890in}{1.057486in}}{\pgfqpoint{1.195790in}{1.054214in}}{\pgfqpoint{1.204026in}{1.054214in}}%
\pgfpathclose%
\pgfusepath{stroke,fill}%
\end{pgfscope}%
\begin{pgfscope}%
\pgfpathrectangle{\pgfqpoint{0.100000in}{0.212622in}}{\pgfqpoint{3.696000in}{3.696000in}}%
\pgfusepath{clip}%
\pgfsetbuttcap%
\pgfsetroundjoin%
\definecolor{currentfill}{rgb}{0.121569,0.466667,0.705882}%
\pgfsetfillcolor{currentfill}%
\pgfsetfillopacity{0.773083}%
\pgfsetlinewidth{1.003750pt}%
\definecolor{currentstroke}{rgb}{0.121569,0.466667,0.705882}%
\pgfsetstrokecolor{currentstroke}%
\pgfsetstrokeopacity{0.773083}%
\pgfsetdash{}{0pt}%
\pgfpathmoveto{\pgfqpoint{1.182307in}{2.134798in}}%
\pgfpathcurveto{\pgfqpoint{1.190544in}{2.134798in}}{\pgfqpoint{1.198444in}{2.138070in}}{\pgfqpoint{1.204268in}{2.143894in}}%
\pgfpathcurveto{\pgfqpoint{1.210092in}{2.149718in}}{\pgfqpoint{1.213364in}{2.157618in}}{\pgfqpoint{1.213364in}{2.165855in}}%
\pgfpathcurveto{\pgfqpoint{1.213364in}{2.174091in}}{\pgfqpoint{1.210092in}{2.181991in}}{\pgfqpoint{1.204268in}{2.187815in}}%
\pgfpathcurveto{\pgfqpoint{1.198444in}{2.193639in}}{\pgfqpoint{1.190544in}{2.196911in}}{\pgfqpoint{1.182307in}{2.196911in}}%
\pgfpathcurveto{\pgfqpoint{1.174071in}{2.196911in}}{\pgfqpoint{1.166171in}{2.193639in}}{\pgfqpoint{1.160347in}{2.187815in}}%
\pgfpathcurveto{\pgfqpoint{1.154523in}{2.181991in}}{\pgfqpoint{1.151251in}{2.174091in}}{\pgfqpoint{1.151251in}{2.165855in}}%
\pgfpathcurveto{\pgfqpoint{1.151251in}{2.157618in}}{\pgfqpoint{1.154523in}{2.149718in}}{\pgfqpoint{1.160347in}{2.143894in}}%
\pgfpathcurveto{\pgfqpoint{1.166171in}{2.138070in}}{\pgfqpoint{1.174071in}{2.134798in}}{\pgfqpoint{1.182307in}{2.134798in}}%
\pgfpathclose%
\pgfusepath{stroke,fill}%
\end{pgfscope}%
\begin{pgfscope}%
\pgfpathrectangle{\pgfqpoint{0.100000in}{0.212622in}}{\pgfqpoint{3.696000in}{3.696000in}}%
\pgfusepath{clip}%
\pgfsetbuttcap%
\pgfsetroundjoin%
\definecolor{currentfill}{rgb}{0.121569,0.466667,0.705882}%
\pgfsetfillcolor{currentfill}%
\pgfsetfillopacity{0.776227}%
\pgfsetlinewidth{1.003750pt}%
\definecolor{currentstroke}{rgb}{0.121569,0.466667,0.705882}%
\pgfsetstrokecolor{currentstroke}%
\pgfsetstrokeopacity{0.776227}%
\pgfsetdash{}{0pt}%
\pgfpathmoveto{\pgfqpoint{2.484591in}{1.670110in}}%
\pgfpathcurveto{\pgfqpoint{2.492827in}{1.670110in}}{\pgfqpoint{2.500727in}{1.673382in}}{\pgfqpoint{2.506551in}{1.679206in}}%
\pgfpathcurveto{\pgfqpoint{2.512375in}{1.685030in}}{\pgfqpoint{2.515648in}{1.692930in}}{\pgfqpoint{2.515648in}{1.701166in}}%
\pgfpathcurveto{\pgfqpoint{2.515648in}{1.709402in}}{\pgfqpoint{2.512375in}{1.717302in}}{\pgfqpoint{2.506551in}{1.723126in}}%
\pgfpathcurveto{\pgfqpoint{2.500727in}{1.728950in}}{\pgfqpoint{2.492827in}{1.732223in}}{\pgfqpoint{2.484591in}{1.732223in}}%
\pgfpathcurveto{\pgfqpoint{2.476355in}{1.732223in}}{\pgfqpoint{2.468455in}{1.728950in}}{\pgfqpoint{2.462631in}{1.723126in}}%
\pgfpathcurveto{\pgfqpoint{2.456807in}{1.717302in}}{\pgfqpoint{2.453535in}{1.709402in}}{\pgfqpoint{2.453535in}{1.701166in}}%
\pgfpathcurveto{\pgfqpoint{2.453535in}{1.692930in}}{\pgfqpoint{2.456807in}{1.685030in}}{\pgfqpoint{2.462631in}{1.679206in}}%
\pgfpathcurveto{\pgfqpoint{2.468455in}{1.673382in}}{\pgfqpoint{2.476355in}{1.670110in}}{\pgfqpoint{2.484591in}{1.670110in}}%
\pgfpathclose%
\pgfusepath{stroke,fill}%
\end{pgfscope}%
\begin{pgfscope}%
\pgfpathrectangle{\pgfqpoint{0.100000in}{0.212622in}}{\pgfqpoint{3.696000in}{3.696000in}}%
\pgfusepath{clip}%
\pgfsetbuttcap%
\pgfsetroundjoin%
\definecolor{currentfill}{rgb}{0.121569,0.466667,0.705882}%
\pgfsetfillcolor{currentfill}%
\pgfsetfillopacity{0.776772}%
\pgfsetlinewidth{1.003750pt}%
\definecolor{currentstroke}{rgb}{0.121569,0.466667,0.705882}%
\pgfsetstrokecolor{currentstroke}%
\pgfsetstrokeopacity{0.776772}%
\pgfsetdash{}{0pt}%
\pgfpathmoveto{\pgfqpoint{1.223087in}{1.051625in}}%
\pgfpathcurveto{\pgfqpoint{1.231323in}{1.051625in}}{\pgfqpoint{1.239223in}{1.054897in}}{\pgfqpoint{1.245047in}{1.060721in}}%
\pgfpathcurveto{\pgfqpoint{1.250871in}{1.066545in}}{\pgfqpoint{1.254143in}{1.074445in}}{\pgfqpoint{1.254143in}{1.082681in}}%
\pgfpathcurveto{\pgfqpoint{1.254143in}{1.090918in}}{\pgfqpoint{1.250871in}{1.098818in}}{\pgfqpoint{1.245047in}{1.104642in}}%
\pgfpathcurveto{\pgfqpoint{1.239223in}{1.110466in}}{\pgfqpoint{1.231323in}{1.113738in}}{\pgfqpoint{1.223087in}{1.113738in}}%
\pgfpathcurveto{\pgfqpoint{1.214850in}{1.113738in}}{\pgfqpoint{1.206950in}{1.110466in}}{\pgfqpoint{1.201126in}{1.104642in}}%
\pgfpathcurveto{\pgfqpoint{1.195302in}{1.098818in}}{\pgfqpoint{1.192030in}{1.090918in}}{\pgfqpoint{1.192030in}{1.082681in}}%
\pgfpathcurveto{\pgfqpoint{1.192030in}{1.074445in}}{\pgfqpoint{1.195302in}{1.066545in}}{\pgfqpoint{1.201126in}{1.060721in}}%
\pgfpathcurveto{\pgfqpoint{1.206950in}{1.054897in}}{\pgfqpoint{1.214850in}{1.051625in}}{\pgfqpoint{1.223087in}{1.051625in}}%
\pgfpathclose%
\pgfusepath{stroke,fill}%
\end{pgfscope}%
\begin{pgfscope}%
\pgfpathrectangle{\pgfqpoint{0.100000in}{0.212622in}}{\pgfqpoint{3.696000in}{3.696000in}}%
\pgfusepath{clip}%
\pgfsetbuttcap%
\pgfsetroundjoin%
\definecolor{currentfill}{rgb}{0.121569,0.466667,0.705882}%
\pgfsetfillcolor{currentfill}%
\pgfsetfillopacity{0.780146}%
\pgfsetlinewidth{1.003750pt}%
\definecolor{currentstroke}{rgb}{0.121569,0.466667,0.705882}%
\pgfsetstrokecolor{currentstroke}%
\pgfsetstrokeopacity{0.780146}%
\pgfsetdash{}{0pt}%
\pgfpathmoveto{\pgfqpoint{1.237804in}{1.049243in}}%
\pgfpathcurveto{\pgfqpoint{1.246040in}{1.049243in}}{\pgfqpoint{1.253940in}{1.052515in}}{\pgfqpoint{1.259764in}{1.058339in}}%
\pgfpathcurveto{\pgfqpoint{1.265588in}{1.064163in}}{\pgfqpoint{1.268860in}{1.072063in}}{\pgfqpoint{1.268860in}{1.080299in}}%
\pgfpathcurveto{\pgfqpoint{1.268860in}{1.088535in}}{\pgfqpoint{1.265588in}{1.096435in}}{\pgfqpoint{1.259764in}{1.102259in}}%
\pgfpathcurveto{\pgfqpoint{1.253940in}{1.108083in}}{\pgfqpoint{1.246040in}{1.111356in}}{\pgfqpoint{1.237804in}{1.111356in}}%
\pgfpathcurveto{\pgfqpoint{1.229567in}{1.111356in}}{\pgfqpoint{1.221667in}{1.108083in}}{\pgfqpoint{1.215843in}{1.102259in}}%
\pgfpathcurveto{\pgfqpoint{1.210019in}{1.096435in}}{\pgfqpoint{1.206747in}{1.088535in}}{\pgfqpoint{1.206747in}{1.080299in}}%
\pgfpathcurveto{\pgfqpoint{1.206747in}{1.072063in}}{\pgfqpoint{1.210019in}{1.064163in}}{\pgfqpoint{1.215843in}{1.058339in}}%
\pgfpathcurveto{\pgfqpoint{1.221667in}{1.052515in}}{\pgfqpoint{1.229567in}{1.049243in}}{\pgfqpoint{1.237804in}{1.049243in}}%
\pgfpathclose%
\pgfusepath{stroke,fill}%
\end{pgfscope}%
\begin{pgfscope}%
\pgfpathrectangle{\pgfqpoint{0.100000in}{0.212622in}}{\pgfqpoint{3.696000in}{3.696000in}}%
\pgfusepath{clip}%
\pgfsetbuttcap%
\pgfsetroundjoin%
\definecolor{currentfill}{rgb}{0.121569,0.466667,0.705882}%
\pgfsetfillcolor{currentfill}%
\pgfsetfillopacity{0.781151}%
\pgfsetlinewidth{1.003750pt}%
\definecolor{currentstroke}{rgb}{0.121569,0.466667,0.705882}%
\pgfsetstrokecolor{currentstroke}%
\pgfsetstrokeopacity{0.781151}%
\pgfsetdash{}{0pt}%
\pgfpathmoveto{\pgfqpoint{1.200820in}{2.121917in}}%
\pgfpathcurveto{\pgfqpoint{1.209056in}{2.121917in}}{\pgfqpoint{1.216956in}{2.125190in}}{\pgfqpoint{1.222780in}{2.131014in}}%
\pgfpathcurveto{\pgfqpoint{1.228604in}{2.136838in}}{\pgfqpoint{1.231877in}{2.144738in}}{\pgfqpoint{1.231877in}{2.152974in}}%
\pgfpathcurveto{\pgfqpoint{1.231877in}{2.161210in}}{\pgfqpoint{1.228604in}{2.169110in}}{\pgfqpoint{1.222780in}{2.174934in}}%
\pgfpathcurveto{\pgfqpoint{1.216956in}{2.180758in}}{\pgfqpoint{1.209056in}{2.184030in}}{\pgfqpoint{1.200820in}{2.184030in}}%
\pgfpathcurveto{\pgfqpoint{1.192584in}{2.184030in}}{\pgfqpoint{1.184684in}{2.180758in}}{\pgfqpoint{1.178860in}{2.174934in}}%
\pgfpathcurveto{\pgfqpoint{1.173036in}{2.169110in}}{\pgfqpoint{1.169764in}{2.161210in}}{\pgfqpoint{1.169764in}{2.152974in}}%
\pgfpathcurveto{\pgfqpoint{1.169764in}{2.144738in}}{\pgfqpoint{1.173036in}{2.136838in}}{\pgfqpoint{1.178860in}{2.131014in}}%
\pgfpathcurveto{\pgfqpoint{1.184684in}{2.125190in}}{\pgfqpoint{1.192584in}{2.121917in}}{\pgfqpoint{1.200820in}{2.121917in}}%
\pgfpathclose%
\pgfusepath{stroke,fill}%
\end{pgfscope}%
\begin{pgfscope}%
\pgfpathrectangle{\pgfqpoint{0.100000in}{0.212622in}}{\pgfqpoint{3.696000in}{3.696000in}}%
\pgfusepath{clip}%
\pgfsetbuttcap%
\pgfsetroundjoin%
\definecolor{currentfill}{rgb}{0.121569,0.466667,0.705882}%
\pgfsetfillcolor{currentfill}%
\pgfsetfillopacity{0.782869}%
\pgfsetlinewidth{1.003750pt}%
\definecolor{currentstroke}{rgb}{0.121569,0.466667,0.705882}%
\pgfsetstrokecolor{currentstroke}%
\pgfsetstrokeopacity{0.782869}%
\pgfsetdash{}{0pt}%
\pgfpathmoveto{\pgfqpoint{1.250278in}{1.047404in}}%
\pgfpathcurveto{\pgfqpoint{1.258514in}{1.047404in}}{\pgfqpoint{1.266414in}{1.050677in}}{\pgfqpoint{1.272238in}{1.056501in}}%
\pgfpathcurveto{\pgfqpoint{1.278062in}{1.062325in}}{\pgfqpoint{1.281334in}{1.070225in}}{\pgfqpoint{1.281334in}{1.078461in}}%
\pgfpathcurveto{\pgfqpoint{1.281334in}{1.086697in}}{\pgfqpoint{1.278062in}{1.094597in}}{\pgfqpoint{1.272238in}{1.100421in}}%
\pgfpathcurveto{\pgfqpoint{1.266414in}{1.106245in}}{\pgfqpoint{1.258514in}{1.109517in}}{\pgfqpoint{1.250278in}{1.109517in}}%
\pgfpathcurveto{\pgfqpoint{1.242042in}{1.109517in}}{\pgfqpoint{1.234142in}{1.106245in}}{\pgfqpoint{1.228318in}{1.100421in}}%
\pgfpathcurveto{\pgfqpoint{1.222494in}{1.094597in}}{\pgfqpoint{1.219221in}{1.086697in}}{\pgfqpoint{1.219221in}{1.078461in}}%
\pgfpathcurveto{\pgfqpoint{1.219221in}{1.070225in}}{\pgfqpoint{1.222494in}{1.062325in}}{\pgfqpoint{1.228318in}{1.056501in}}%
\pgfpathcurveto{\pgfqpoint{1.234142in}{1.050677in}}{\pgfqpoint{1.242042in}{1.047404in}}{\pgfqpoint{1.250278in}{1.047404in}}%
\pgfpathclose%
\pgfusepath{stroke,fill}%
\end{pgfscope}%
\begin{pgfscope}%
\pgfpathrectangle{\pgfqpoint{0.100000in}{0.212622in}}{\pgfqpoint{3.696000in}{3.696000in}}%
\pgfusepath{clip}%
\pgfsetbuttcap%
\pgfsetroundjoin%
\definecolor{currentfill}{rgb}{0.121569,0.466667,0.705882}%
\pgfsetfillcolor{currentfill}%
\pgfsetfillopacity{0.785114}%
\pgfsetlinewidth{1.003750pt}%
\definecolor{currentstroke}{rgb}{0.121569,0.466667,0.705882}%
\pgfsetstrokecolor{currentstroke}%
\pgfsetstrokeopacity{0.785114}%
\pgfsetdash{}{0pt}%
\pgfpathmoveto{\pgfqpoint{2.491479in}{1.639849in}}%
\pgfpathcurveto{\pgfqpoint{2.499715in}{1.639849in}}{\pgfqpoint{2.507615in}{1.643122in}}{\pgfqpoint{2.513439in}{1.648946in}}%
\pgfpathcurveto{\pgfqpoint{2.519263in}{1.654770in}}{\pgfqpoint{2.522535in}{1.662670in}}{\pgfqpoint{2.522535in}{1.670906in}}%
\pgfpathcurveto{\pgfqpoint{2.522535in}{1.679142in}}{\pgfqpoint{2.519263in}{1.687042in}}{\pgfqpoint{2.513439in}{1.692866in}}%
\pgfpathcurveto{\pgfqpoint{2.507615in}{1.698690in}}{\pgfqpoint{2.499715in}{1.701962in}}{\pgfqpoint{2.491479in}{1.701962in}}%
\pgfpathcurveto{\pgfqpoint{2.483243in}{1.701962in}}{\pgfqpoint{2.475343in}{1.698690in}}{\pgfqpoint{2.469519in}{1.692866in}}%
\pgfpathcurveto{\pgfqpoint{2.463695in}{1.687042in}}{\pgfqpoint{2.460422in}{1.679142in}}{\pgfqpoint{2.460422in}{1.670906in}}%
\pgfpathcurveto{\pgfqpoint{2.460422in}{1.662670in}}{\pgfqpoint{2.463695in}{1.654770in}}{\pgfqpoint{2.469519in}{1.648946in}}%
\pgfpathcurveto{\pgfqpoint{2.475343in}{1.643122in}}{\pgfqpoint{2.483243in}{1.639849in}}{\pgfqpoint{2.491479in}{1.639849in}}%
\pgfpathclose%
\pgfusepath{stroke,fill}%
\end{pgfscope}%
\begin{pgfscope}%
\pgfpathrectangle{\pgfqpoint{0.100000in}{0.212622in}}{\pgfqpoint{3.696000in}{3.696000in}}%
\pgfusepath{clip}%
\pgfsetbuttcap%
\pgfsetroundjoin%
\definecolor{currentfill}{rgb}{0.121569,0.466667,0.705882}%
\pgfsetfillcolor{currentfill}%
\pgfsetfillopacity{0.787590}%
\pgfsetlinewidth{1.003750pt}%
\definecolor{currentstroke}{rgb}{0.121569,0.466667,0.705882}%
\pgfsetstrokecolor{currentstroke}%
\pgfsetstrokeopacity{0.787590}%
\pgfsetdash{}{0pt}%
\pgfpathmoveto{\pgfqpoint{1.216778in}{2.113050in}}%
\pgfpathcurveto{\pgfqpoint{1.225014in}{2.113050in}}{\pgfqpoint{1.232914in}{2.116322in}}{\pgfqpoint{1.238738in}{2.122146in}}%
\pgfpathcurveto{\pgfqpoint{1.244562in}{2.127970in}}{\pgfqpoint{1.247835in}{2.135870in}}{\pgfqpoint{1.247835in}{2.144106in}}%
\pgfpathcurveto{\pgfqpoint{1.247835in}{2.152343in}}{\pgfqpoint{1.244562in}{2.160243in}}{\pgfqpoint{1.238738in}{2.166067in}}%
\pgfpathcurveto{\pgfqpoint{1.232914in}{2.171890in}}{\pgfqpoint{1.225014in}{2.175163in}}{\pgfqpoint{1.216778in}{2.175163in}}%
\pgfpathcurveto{\pgfqpoint{1.208542in}{2.175163in}}{\pgfqpoint{1.200642in}{2.171890in}}{\pgfqpoint{1.194818in}{2.166067in}}%
\pgfpathcurveto{\pgfqpoint{1.188994in}{2.160243in}}{\pgfqpoint{1.185722in}{2.152343in}}{\pgfqpoint{1.185722in}{2.144106in}}%
\pgfpathcurveto{\pgfqpoint{1.185722in}{2.135870in}}{\pgfqpoint{1.188994in}{2.127970in}}{\pgfqpoint{1.194818in}{2.122146in}}%
\pgfpathcurveto{\pgfqpoint{1.200642in}{2.116322in}}{\pgfqpoint{1.208542in}{2.113050in}}{\pgfqpoint{1.216778in}{2.113050in}}%
\pgfpathclose%
\pgfusepath{stroke,fill}%
\end{pgfscope}%
\begin{pgfscope}%
\pgfpathrectangle{\pgfqpoint{0.100000in}{0.212622in}}{\pgfqpoint{3.696000in}{3.696000in}}%
\pgfusepath{clip}%
\pgfsetbuttcap%
\pgfsetroundjoin%
\definecolor{currentfill}{rgb}{0.121569,0.466667,0.705882}%
\pgfsetfillcolor{currentfill}%
\pgfsetfillopacity{0.788010}%
\pgfsetlinewidth{1.003750pt}%
\definecolor{currentstroke}{rgb}{0.121569,0.466667,0.705882}%
\pgfsetstrokecolor{currentstroke}%
\pgfsetstrokeopacity{0.788010}%
\pgfsetdash{}{0pt}%
\pgfpathmoveto{\pgfqpoint{1.272876in}{1.044374in}}%
\pgfpathcurveto{\pgfqpoint{1.281113in}{1.044374in}}{\pgfqpoint{1.289013in}{1.047646in}}{\pgfqpoint{1.294837in}{1.053470in}}%
\pgfpathcurveto{\pgfqpoint{1.300661in}{1.059294in}}{\pgfqpoint{1.303933in}{1.067194in}}{\pgfqpoint{1.303933in}{1.075430in}}%
\pgfpathcurveto{\pgfqpoint{1.303933in}{1.083667in}}{\pgfqpoint{1.300661in}{1.091567in}}{\pgfqpoint{1.294837in}{1.097391in}}%
\pgfpathcurveto{\pgfqpoint{1.289013in}{1.103214in}}{\pgfqpoint{1.281113in}{1.106487in}}{\pgfqpoint{1.272876in}{1.106487in}}%
\pgfpathcurveto{\pgfqpoint{1.264640in}{1.106487in}}{\pgfqpoint{1.256740in}{1.103214in}}{\pgfqpoint{1.250916in}{1.097391in}}%
\pgfpathcurveto{\pgfqpoint{1.245092in}{1.091567in}}{\pgfqpoint{1.241820in}{1.083667in}}{\pgfqpoint{1.241820in}{1.075430in}}%
\pgfpathcurveto{\pgfqpoint{1.241820in}{1.067194in}}{\pgfqpoint{1.245092in}{1.059294in}}{\pgfqpoint{1.250916in}{1.053470in}}%
\pgfpathcurveto{\pgfqpoint{1.256740in}{1.047646in}}{\pgfqpoint{1.264640in}{1.044374in}}{\pgfqpoint{1.272876in}{1.044374in}}%
\pgfpathclose%
\pgfusepath{stroke,fill}%
\end{pgfscope}%
\begin{pgfscope}%
\pgfpathrectangle{\pgfqpoint{0.100000in}{0.212622in}}{\pgfqpoint{3.696000in}{3.696000in}}%
\pgfusepath{clip}%
\pgfsetbuttcap%
\pgfsetroundjoin%
\definecolor{currentfill}{rgb}{0.121569,0.466667,0.705882}%
\pgfsetfillcolor{currentfill}%
\pgfsetfillopacity{0.792650}%
\pgfsetlinewidth{1.003750pt}%
\definecolor{currentstroke}{rgb}{0.121569,0.466667,0.705882}%
\pgfsetstrokecolor{currentstroke}%
\pgfsetstrokeopacity{0.792650}%
\pgfsetdash{}{0pt}%
\pgfpathmoveto{\pgfqpoint{1.293477in}{1.041555in}}%
\pgfpathcurveto{\pgfqpoint{1.301714in}{1.041555in}}{\pgfqpoint{1.309614in}{1.044827in}}{\pgfqpoint{1.315438in}{1.050651in}}%
\pgfpathcurveto{\pgfqpoint{1.321262in}{1.056475in}}{\pgfqpoint{1.324534in}{1.064375in}}{\pgfqpoint{1.324534in}{1.072611in}}%
\pgfpathcurveto{\pgfqpoint{1.324534in}{1.080847in}}{\pgfqpoint{1.321262in}{1.088747in}}{\pgfqpoint{1.315438in}{1.094571in}}%
\pgfpathcurveto{\pgfqpoint{1.309614in}{1.100395in}}{\pgfqpoint{1.301714in}{1.103668in}}{\pgfqpoint{1.293477in}{1.103668in}}%
\pgfpathcurveto{\pgfqpoint{1.285241in}{1.103668in}}{\pgfqpoint{1.277341in}{1.100395in}}{\pgfqpoint{1.271517in}{1.094571in}}%
\pgfpathcurveto{\pgfqpoint{1.265693in}{1.088747in}}{\pgfqpoint{1.262421in}{1.080847in}}{\pgfqpoint{1.262421in}{1.072611in}}%
\pgfpathcurveto{\pgfqpoint{1.262421in}{1.064375in}}{\pgfqpoint{1.265693in}{1.056475in}}{\pgfqpoint{1.271517in}{1.050651in}}%
\pgfpathcurveto{\pgfqpoint{1.277341in}{1.044827in}}{\pgfqpoint{1.285241in}{1.041555in}}{\pgfqpoint{1.293477in}{1.041555in}}%
\pgfpathclose%
\pgfusepath{stroke,fill}%
\end{pgfscope}%
\begin{pgfscope}%
\pgfpathrectangle{\pgfqpoint{0.100000in}{0.212622in}}{\pgfqpoint{3.696000in}{3.696000in}}%
\pgfusepath{clip}%
\pgfsetbuttcap%
\pgfsetroundjoin%
\definecolor{currentfill}{rgb}{0.121569,0.466667,0.705882}%
\pgfsetfillcolor{currentfill}%
\pgfsetfillopacity{0.792687}%
\pgfsetlinewidth{1.003750pt}%
\definecolor{currentstroke}{rgb}{0.121569,0.466667,0.705882}%
\pgfsetstrokecolor{currentstroke}%
\pgfsetstrokeopacity{0.792687}%
\pgfsetdash{}{0pt}%
\pgfpathmoveto{\pgfqpoint{1.230486in}{2.107137in}}%
\pgfpathcurveto{\pgfqpoint{1.238723in}{2.107137in}}{\pgfqpoint{1.246623in}{2.110409in}}{\pgfqpoint{1.252447in}{2.116233in}}%
\pgfpathcurveto{\pgfqpoint{1.258271in}{2.122057in}}{\pgfqpoint{1.261543in}{2.129957in}}{\pgfqpoint{1.261543in}{2.138194in}}%
\pgfpathcurveto{\pgfqpoint{1.261543in}{2.146430in}}{\pgfqpoint{1.258271in}{2.154330in}}{\pgfqpoint{1.252447in}{2.160154in}}%
\pgfpathcurveto{\pgfqpoint{1.246623in}{2.165978in}}{\pgfqpoint{1.238723in}{2.169250in}}{\pgfqpoint{1.230486in}{2.169250in}}%
\pgfpathcurveto{\pgfqpoint{1.222250in}{2.169250in}}{\pgfqpoint{1.214350in}{2.165978in}}{\pgfqpoint{1.208526in}{2.160154in}}%
\pgfpathcurveto{\pgfqpoint{1.202702in}{2.154330in}}{\pgfqpoint{1.199430in}{2.146430in}}{\pgfqpoint{1.199430in}{2.138194in}}%
\pgfpathcurveto{\pgfqpoint{1.199430in}{2.129957in}}{\pgfqpoint{1.202702in}{2.122057in}}{\pgfqpoint{1.208526in}{2.116233in}}%
\pgfpathcurveto{\pgfqpoint{1.214350in}{2.110409in}}{\pgfqpoint{1.222250in}{2.107137in}}{\pgfqpoint{1.230486in}{2.107137in}}%
\pgfpathclose%
\pgfusepath{stroke,fill}%
\end{pgfscope}%
\begin{pgfscope}%
\pgfpathrectangle{\pgfqpoint{0.100000in}{0.212622in}}{\pgfqpoint{3.696000in}{3.696000in}}%
\pgfusepath{clip}%
\pgfsetbuttcap%
\pgfsetroundjoin%
\definecolor{currentfill}{rgb}{0.121569,0.466667,0.705882}%
\pgfsetfillcolor{currentfill}%
\pgfsetfillopacity{0.794127}%
\pgfsetlinewidth{1.003750pt}%
\definecolor{currentstroke}{rgb}{0.121569,0.466667,0.705882}%
\pgfsetstrokecolor{currentstroke}%
\pgfsetstrokeopacity{0.794127}%
\pgfsetdash{}{0pt}%
\pgfpathmoveto{\pgfqpoint{2.501624in}{1.606553in}}%
\pgfpathcurveto{\pgfqpoint{2.509861in}{1.606553in}}{\pgfqpoint{2.517761in}{1.609826in}}{\pgfqpoint{2.523585in}{1.615650in}}%
\pgfpathcurveto{\pgfqpoint{2.529409in}{1.621474in}}{\pgfqpoint{2.532681in}{1.629374in}}{\pgfqpoint{2.532681in}{1.637610in}}%
\pgfpathcurveto{\pgfqpoint{2.532681in}{1.645846in}}{\pgfqpoint{2.529409in}{1.653746in}}{\pgfqpoint{2.523585in}{1.659570in}}%
\pgfpathcurveto{\pgfqpoint{2.517761in}{1.665394in}}{\pgfqpoint{2.509861in}{1.668666in}}{\pgfqpoint{2.501624in}{1.668666in}}%
\pgfpathcurveto{\pgfqpoint{2.493388in}{1.668666in}}{\pgfqpoint{2.485488in}{1.665394in}}{\pgfqpoint{2.479664in}{1.659570in}}%
\pgfpathcurveto{\pgfqpoint{2.473840in}{1.653746in}}{\pgfqpoint{2.470568in}{1.645846in}}{\pgfqpoint{2.470568in}{1.637610in}}%
\pgfpathcurveto{\pgfqpoint{2.470568in}{1.629374in}}{\pgfqpoint{2.473840in}{1.621474in}}{\pgfqpoint{2.479664in}{1.615650in}}%
\pgfpathcurveto{\pgfqpoint{2.485488in}{1.609826in}}{\pgfqpoint{2.493388in}{1.606553in}}{\pgfqpoint{2.501624in}{1.606553in}}%
\pgfpathclose%
\pgfusepath{stroke,fill}%
\end{pgfscope}%
\begin{pgfscope}%
\pgfpathrectangle{\pgfqpoint{0.100000in}{0.212622in}}{\pgfqpoint{3.696000in}{3.696000in}}%
\pgfusepath{clip}%
\pgfsetbuttcap%
\pgfsetroundjoin%
\definecolor{currentfill}{rgb}{0.121569,0.466667,0.705882}%
\pgfsetfillcolor{currentfill}%
\pgfsetfillopacity{0.796246}%
\pgfsetlinewidth{1.003750pt}%
\definecolor{currentstroke}{rgb}{0.121569,0.466667,0.705882}%
\pgfsetstrokecolor{currentstroke}%
\pgfsetstrokeopacity{0.796246}%
\pgfsetdash{}{0pt}%
\pgfpathmoveto{\pgfqpoint{1.310051in}{1.039029in}}%
\pgfpathcurveto{\pgfqpoint{1.318287in}{1.039029in}}{\pgfqpoint{1.326187in}{1.042301in}}{\pgfqpoint{1.332011in}{1.048125in}}%
\pgfpathcurveto{\pgfqpoint{1.337835in}{1.053949in}}{\pgfqpoint{1.341107in}{1.061849in}}{\pgfqpoint{1.341107in}{1.070086in}}%
\pgfpathcurveto{\pgfqpoint{1.341107in}{1.078322in}}{\pgfqpoint{1.337835in}{1.086222in}}{\pgfqpoint{1.332011in}{1.092046in}}%
\pgfpathcurveto{\pgfqpoint{1.326187in}{1.097870in}}{\pgfqpoint{1.318287in}{1.101142in}}{\pgfqpoint{1.310051in}{1.101142in}}%
\pgfpathcurveto{\pgfqpoint{1.301814in}{1.101142in}}{\pgfqpoint{1.293914in}{1.097870in}}{\pgfqpoint{1.288090in}{1.092046in}}%
\pgfpathcurveto{\pgfqpoint{1.282267in}{1.086222in}}{\pgfqpoint{1.278994in}{1.078322in}}{\pgfqpoint{1.278994in}{1.070086in}}%
\pgfpathcurveto{\pgfqpoint{1.278994in}{1.061849in}}{\pgfqpoint{1.282267in}{1.053949in}}{\pgfqpoint{1.288090in}{1.048125in}}%
\pgfpathcurveto{\pgfqpoint{1.293914in}{1.042301in}}{\pgfqpoint{1.301814in}{1.039029in}}{\pgfqpoint{1.310051in}{1.039029in}}%
\pgfpathclose%
\pgfusepath{stroke,fill}%
\end{pgfscope}%
\begin{pgfscope}%
\pgfpathrectangle{\pgfqpoint{0.100000in}{0.212622in}}{\pgfqpoint{3.696000in}{3.696000in}}%
\pgfusepath{clip}%
\pgfsetbuttcap%
\pgfsetroundjoin%
\definecolor{currentfill}{rgb}{0.121569,0.466667,0.705882}%
\pgfsetfillcolor{currentfill}%
\pgfsetfillopacity{0.796635}%
\pgfsetlinewidth{1.003750pt}%
\definecolor{currentstroke}{rgb}{0.121569,0.466667,0.705882}%
\pgfsetstrokecolor{currentstroke}%
\pgfsetstrokeopacity{0.796635}%
\pgfsetdash{}{0pt}%
\pgfpathmoveto{\pgfqpoint{1.242128in}{2.103435in}}%
\pgfpathcurveto{\pgfqpoint{1.250365in}{2.103435in}}{\pgfqpoint{1.258265in}{2.106707in}}{\pgfqpoint{1.264089in}{2.112531in}}%
\pgfpathcurveto{\pgfqpoint{1.269913in}{2.118355in}}{\pgfqpoint{1.273185in}{2.126255in}}{\pgfqpoint{1.273185in}{2.134491in}}%
\pgfpathcurveto{\pgfqpoint{1.273185in}{2.142728in}}{\pgfqpoint{1.269913in}{2.150628in}}{\pgfqpoint{1.264089in}{2.156452in}}%
\pgfpathcurveto{\pgfqpoint{1.258265in}{2.162276in}}{\pgfqpoint{1.250365in}{2.165548in}}{\pgfqpoint{1.242128in}{2.165548in}}%
\pgfpathcurveto{\pgfqpoint{1.233892in}{2.165548in}}{\pgfqpoint{1.225992in}{2.162276in}}{\pgfqpoint{1.220168in}{2.156452in}}%
\pgfpathcurveto{\pgfqpoint{1.214344in}{2.150628in}}{\pgfqpoint{1.211072in}{2.142728in}}{\pgfqpoint{1.211072in}{2.134491in}}%
\pgfpathcurveto{\pgfqpoint{1.211072in}{2.126255in}}{\pgfqpoint{1.214344in}{2.118355in}}{\pgfqpoint{1.220168in}{2.112531in}}%
\pgfpathcurveto{\pgfqpoint{1.225992in}{2.106707in}}{\pgfqpoint{1.233892in}{2.103435in}}{\pgfqpoint{1.242128in}{2.103435in}}%
\pgfpathclose%
\pgfusepath{stroke,fill}%
\end{pgfscope}%
\begin{pgfscope}%
\pgfpathrectangle{\pgfqpoint{0.100000in}{0.212622in}}{\pgfqpoint{3.696000in}{3.696000in}}%
\pgfusepath{clip}%
\pgfsetbuttcap%
\pgfsetroundjoin%
\definecolor{currentfill}{rgb}{0.121569,0.466667,0.705882}%
\pgfsetfillcolor{currentfill}%
\pgfsetfillopacity{0.798646}%
\pgfsetlinewidth{1.003750pt}%
\definecolor{currentstroke}{rgb}{0.121569,0.466667,0.705882}%
\pgfsetstrokecolor{currentstroke}%
\pgfsetstrokeopacity{0.798646}%
\pgfsetdash{}{0pt}%
\pgfpathmoveto{\pgfqpoint{1.321074in}{1.037517in}}%
\pgfpathcurveto{\pgfqpoint{1.329311in}{1.037517in}}{\pgfqpoint{1.337211in}{1.040789in}}{\pgfqpoint{1.343035in}{1.046613in}}%
\pgfpathcurveto{\pgfqpoint{1.348859in}{1.052437in}}{\pgfqpoint{1.352131in}{1.060337in}}{\pgfqpoint{1.352131in}{1.068573in}}%
\pgfpathcurveto{\pgfqpoint{1.352131in}{1.076810in}}{\pgfqpoint{1.348859in}{1.084710in}}{\pgfqpoint{1.343035in}{1.090534in}}%
\pgfpathcurveto{\pgfqpoint{1.337211in}{1.096358in}}{\pgfqpoint{1.329311in}{1.099630in}}{\pgfqpoint{1.321074in}{1.099630in}}%
\pgfpathcurveto{\pgfqpoint{1.312838in}{1.099630in}}{\pgfqpoint{1.304938in}{1.096358in}}{\pgfqpoint{1.299114in}{1.090534in}}%
\pgfpathcurveto{\pgfqpoint{1.293290in}{1.084710in}}{\pgfqpoint{1.290018in}{1.076810in}}{\pgfqpoint{1.290018in}{1.068573in}}%
\pgfpathcurveto{\pgfqpoint{1.290018in}{1.060337in}}{\pgfqpoint{1.293290in}{1.052437in}}{\pgfqpoint{1.299114in}{1.046613in}}%
\pgfpathcurveto{\pgfqpoint{1.304938in}{1.040789in}}{\pgfqpoint{1.312838in}{1.037517in}}{\pgfqpoint{1.321074in}{1.037517in}}%
\pgfpathclose%
\pgfusepath{stroke,fill}%
\end{pgfscope}%
\begin{pgfscope}%
\pgfpathrectangle{\pgfqpoint{0.100000in}{0.212622in}}{\pgfqpoint{3.696000in}{3.696000in}}%
\pgfusepath{clip}%
\pgfsetbuttcap%
\pgfsetroundjoin%
\definecolor{currentfill}{rgb}{0.121569,0.466667,0.705882}%
\pgfsetfillcolor{currentfill}%
\pgfsetfillopacity{0.799905}%
\pgfsetlinewidth{1.003750pt}%
\definecolor{currentstroke}{rgb}{0.121569,0.466667,0.705882}%
\pgfsetstrokecolor{currentstroke}%
\pgfsetstrokeopacity{0.799905}%
\pgfsetdash{}{0pt}%
\pgfpathmoveto{\pgfqpoint{1.252453in}{2.101468in}}%
\pgfpathcurveto{\pgfqpoint{1.260689in}{2.101468in}}{\pgfqpoint{1.268589in}{2.104740in}}{\pgfqpoint{1.274413in}{2.110564in}}%
\pgfpathcurveto{\pgfqpoint{1.280237in}{2.116388in}}{\pgfqpoint{1.283509in}{2.124288in}}{\pgfqpoint{1.283509in}{2.132524in}}%
\pgfpathcurveto{\pgfqpoint{1.283509in}{2.140761in}}{\pgfqpoint{1.280237in}{2.148661in}}{\pgfqpoint{1.274413in}{2.154485in}}%
\pgfpathcurveto{\pgfqpoint{1.268589in}{2.160309in}}{\pgfqpoint{1.260689in}{2.163581in}}{\pgfqpoint{1.252453in}{2.163581in}}%
\pgfpathcurveto{\pgfqpoint{1.244216in}{2.163581in}}{\pgfqpoint{1.236316in}{2.160309in}}{\pgfqpoint{1.230492in}{2.154485in}}%
\pgfpathcurveto{\pgfqpoint{1.224668in}{2.148661in}}{\pgfqpoint{1.221396in}{2.140761in}}{\pgfqpoint{1.221396in}{2.132524in}}%
\pgfpathcurveto{\pgfqpoint{1.221396in}{2.124288in}}{\pgfqpoint{1.224668in}{2.116388in}}{\pgfqpoint{1.230492in}{2.110564in}}%
\pgfpathcurveto{\pgfqpoint{1.236316in}{2.104740in}}{\pgfqpoint{1.244216in}{2.101468in}}{\pgfqpoint{1.252453in}{2.101468in}}%
\pgfpathclose%
\pgfusepath{stroke,fill}%
\end{pgfscope}%
\begin{pgfscope}%
\pgfpathrectangle{\pgfqpoint{0.100000in}{0.212622in}}{\pgfqpoint{3.696000in}{3.696000in}}%
\pgfusepath{clip}%
\pgfsetbuttcap%
\pgfsetroundjoin%
\definecolor{currentfill}{rgb}{0.121569,0.466667,0.705882}%
\pgfsetfillcolor{currentfill}%
\pgfsetfillopacity{0.800572}%
\pgfsetlinewidth{1.003750pt}%
\definecolor{currentstroke}{rgb}{0.121569,0.466667,0.705882}%
\pgfsetstrokecolor{currentstroke}%
\pgfsetstrokeopacity{0.800572}%
\pgfsetdash{}{0pt}%
\pgfpathmoveto{\pgfqpoint{1.254789in}{2.101244in}}%
\pgfpathcurveto{\pgfqpoint{1.263026in}{2.101244in}}{\pgfqpoint{1.270926in}{2.104516in}}{\pgfqpoint{1.276750in}{2.110340in}}%
\pgfpathcurveto{\pgfqpoint{1.282574in}{2.116164in}}{\pgfqpoint{1.285846in}{2.124064in}}{\pgfqpoint{1.285846in}{2.132301in}}%
\pgfpathcurveto{\pgfqpoint{1.285846in}{2.140537in}}{\pgfqpoint{1.282574in}{2.148437in}}{\pgfqpoint{1.276750in}{2.154261in}}%
\pgfpathcurveto{\pgfqpoint{1.270926in}{2.160085in}}{\pgfqpoint{1.263026in}{2.163357in}}{\pgfqpoint{1.254789in}{2.163357in}}%
\pgfpathcurveto{\pgfqpoint{1.246553in}{2.163357in}}{\pgfqpoint{1.238653in}{2.160085in}}{\pgfqpoint{1.232829in}{2.154261in}}%
\pgfpathcurveto{\pgfqpoint{1.227005in}{2.148437in}}{\pgfqpoint{1.223733in}{2.140537in}}{\pgfqpoint{1.223733in}{2.132301in}}%
\pgfpathcurveto{\pgfqpoint{1.223733in}{2.124064in}}{\pgfqpoint{1.227005in}{2.116164in}}{\pgfqpoint{1.232829in}{2.110340in}}%
\pgfpathcurveto{\pgfqpoint{1.238653in}{2.104516in}}{\pgfqpoint{1.246553in}{2.101244in}}{\pgfqpoint{1.254789in}{2.101244in}}%
\pgfpathclose%
\pgfusepath{stroke,fill}%
\end{pgfscope}%
\begin{pgfscope}%
\pgfpathrectangle{\pgfqpoint{0.100000in}{0.212622in}}{\pgfqpoint{3.696000in}{3.696000in}}%
\pgfusepath{clip}%
\pgfsetbuttcap%
\pgfsetroundjoin%
\definecolor{currentfill}{rgb}{0.121569,0.466667,0.705882}%
\pgfsetfillcolor{currentfill}%
\pgfsetfillopacity{0.800686}%
\pgfsetlinewidth{1.003750pt}%
\definecolor{currentstroke}{rgb}{0.121569,0.466667,0.705882}%
\pgfsetstrokecolor{currentstroke}%
\pgfsetstrokeopacity{0.800686}%
\pgfsetdash{}{0pt}%
\pgfpathmoveto{\pgfqpoint{1.330064in}{1.036270in}}%
\pgfpathcurveto{\pgfqpoint{1.338301in}{1.036270in}}{\pgfqpoint{1.346201in}{1.039542in}}{\pgfqpoint{1.352025in}{1.045366in}}%
\pgfpathcurveto{\pgfqpoint{1.357849in}{1.051190in}}{\pgfqpoint{1.361121in}{1.059090in}}{\pgfqpoint{1.361121in}{1.067327in}}%
\pgfpathcurveto{\pgfqpoint{1.361121in}{1.075563in}}{\pgfqpoint{1.357849in}{1.083463in}}{\pgfqpoint{1.352025in}{1.089287in}}%
\pgfpathcurveto{\pgfqpoint{1.346201in}{1.095111in}}{\pgfqpoint{1.338301in}{1.098383in}}{\pgfqpoint{1.330064in}{1.098383in}}%
\pgfpathcurveto{\pgfqpoint{1.321828in}{1.098383in}}{\pgfqpoint{1.313928in}{1.095111in}}{\pgfqpoint{1.308104in}{1.089287in}}%
\pgfpathcurveto{\pgfqpoint{1.302280in}{1.083463in}}{\pgfqpoint{1.299008in}{1.075563in}}{\pgfqpoint{1.299008in}{1.067327in}}%
\pgfpathcurveto{\pgfqpoint{1.299008in}{1.059090in}}{\pgfqpoint{1.302280in}{1.051190in}}{\pgfqpoint{1.308104in}{1.045366in}}%
\pgfpathcurveto{\pgfqpoint{1.313928in}{1.039542in}}{\pgfqpoint{1.321828in}{1.036270in}}{\pgfqpoint{1.330064in}{1.036270in}}%
\pgfpathclose%
\pgfusepath{stroke,fill}%
\end{pgfscope}%
\begin{pgfscope}%
\pgfpathrectangle{\pgfqpoint{0.100000in}{0.212622in}}{\pgfqpoint{3.696000in}{3.696000in}}%
\pgfusepath{clip}%
\pgfsetbuttcap%
\pgfsetroundjoin%
\definecolor{currentfill}{rgb}{0.121569,0.466667,0.705882}%
\pgfsetfillcolor{currentfill}%
\pgfsetfillopacity{0.801693}%
\pgfsetlinewidth{1.003750pt}%
\definecolor{currentstroke}{rgb}{0.121569,0.466667,0.705882}%
\pgfsetstrokecolor{currentstroke}%
\pgfsetstrokeopacity{0.801693}%
\pgfsetdash{}{0pt}%
\pgfpathmoveto{\pgfqpoint{1.259188in}{2.101016in}}%
\pgfpathcurveto{\pgfqpoint{1.267424in}{2.101016in}}{\pgfqpoint{1.275324in}{2.104289in}}{\pgfqpoint{1.281148in}{2.110112in}}%
\pgfpathcurveto{\pgfqpoint{1.286972in}{2.115936in}}{\pgfqpoint{1.290245in}{2.123836in}}{\pgfqpoint{1.290245in}{2.132073in}}%
\pgfpathcurveto{\pgfqpoint{1.290245in}{2.140309in}}{\pgfqpoint{1.286972in}{2.148209in}}{\pgfqpoint{1.281148in}{2.154033in}}%
\pgfpathcurveto{\pgfqpoint{1.275324in}{2.159857in}}{\pgfqpoint{1.267424in}{2.163129in}}{\pgfqpoint{1.259188in}{2.163129in}}%
\pgfpathcurveto{\pgfqpoint{1.250952in}{2.163129in}}{\pgfqpoint{1.243052in}{2.159857in}}{\pgfqpoint{1.237228in}{2.154033in}}%
\pgfpathcurveto{\pgfqpoint{1.231404in}{2.148209in}}{\pgfqpoint{1.228132in}{2.140309in}}{\pgfqpoint{1.228132in}{2.132073in}}%
\pgfpathcurveto{\pgfqpoint{1.228132in}{2.123836in}}{\pgfqpoint{1.231404in}{2.115936in}}{\pgfqpoint{1.237228in}{2.110112in}}%
\pgfpathcurveto{\pgfqpoint{1.243052in}{2.104289in}}{\pgfqpoint{1.250952in}{2.101016in}}{\pgfqpoint{1.259188in}{2.101016in}}%
\pgfpathclose%
\pgfusepath{stroke,fill}%
\end{pgfscope}%
\begin{pgfscope}%
\pgfpathrectangle{\pgfqpoint{0.100000in}{0.212622in}}{\pgfqpoint{3.696000in}{3.696000in}}%
\pgfusepath{clip}%
\pgfsetbuttcap%
\pgfsetroundjoin%
\definecolor{currentfill}{rgb}{0.121569,0.466667,0.705882}%
\pgfsetfillcolor{currentfill}%
\pgfsetfillopacity{0.803597}%
\pgfsetlinewidth{1.003750pt}%
\definecolor{currentstroke}{rgb}{0.121569,0.466667,0.705882}%
\pgfsetstrokecolor{currentstroke}%
\pgfsetstrokeopacity{0.803597}%
\pgfsetdash{}{0pt}%
\pgfpathmoveto{\pgfqpoint{1.267490in}{2.101408in}}%
\pgfpathcurveto{\pgfqpoint{1.275726in}{2.101408in}}{\pgfqpoint{1.283626in}{2.104680in}}{\pgfqpoint{1.289450in}{2.110504in}}%
\pgfpathcurveto{\pgfqpoint{1.295274in}{2.116328in}}{\pgfqpoint{1.298547in}{2.124228in}}{\pgfqpoint{1.298547in}{2.132464in}}%
\pgfpathcurveto{\pgfqpoint{1.298547in}{2.140701in}}{\pgfqpoint{1.295274in}{2.148601in}}{\pgfqpoint{1.289450in}{2.154425in}}%
\pgfpathcurveto{\pgfqpoint{1.283626in}{2.160249in}}{\pgfqpoint{1.275726in}{2.163521in}}{\pgfqpoint{1.267490in}{2.163521in}}%
\pgfpathcurveto{\pgfqpoint{1.259254in}{2.163521in}}{\pgfqpoint{1.251354in}{2.160249in}}{\pgfqpoint{1.245530in}{2.154425in}}%
\pgfpathcurveto{\pgfqpoint{1.239706in}{2.148601in}}{\pgfqpoint{1.236434in}{2.140701in}}{\pgfqpoint{1.236434in}{2.132464in}}%
\pgfpathcurveto{\pgfqpoint{1.236434in}{2.124228in}}{\pgfqpoint{1.239706in}{2.116328in}}{\pgfqpoint{1.245530in}{2.110504in}}%
\pgfpathcurveto{\pgfqpoint{1.251354in}{2.104680in}}{\pgfqpoint{1.259254in}{2.101408in}}{\pgfqpoint{1.267490in}{2.101408in}}%
\pgfpathclose%
\pgfusepath{stroke,fill}%
\end{pgfscope}%
\begin{pgfscope}%
\pgfpathrectangle{\pgfqpoint{0.100000in}{0.212622in}}{\pgfqpoint{3.696000in}{3.696000in}}%
\pgfusepath{clip}%
\pgfsetbuttcap%
\pgfsetroundjoin%
\definecolor{currentfill}{rgb}{0.121569,0.466667,0.705882}%
\pgfsetfillcolor{currentfill}%
\pgfsetfillopacity{0.804121}%
\pgfsetlinewidth{1.003750pt}%
\definecolor{currentstroke}{rgb}{0.121569,0.466667,0.705882}%
\pgfsetstrokecolor{currentstroke}%
\pgfsetstrokeopacity{0.804121}%
\pgfsetdash{}{0pt}%
\pgfpathmoveto{\pgfqpoint{2.510232in}{1.571066in}}%
\pgfpathcurveto{\pgfqpoint{2.518469in}{1.571066in}}{\pgfqpoint{2.526369in}{1.574338in}}{\pgfqpoint{2.532193in}{1.580162in}}%
\pgfpathcurveto{\pgfqpoint{2.538017in}{1.585986in}}{\pgfqpoint{2.541289in}{1.593886in}}{\pgfqpoint{2.541289in}{1.602122in}}%
\pgfpathcurveto{\pgfqpoint{2.541289in}{1.610358in}}{\pgfqpoint{2.538017in}{1.618258in}}{\pgfqpoint{2.532193in}{1.624082in}}%
\pgfpathcurveto{\pgfqpoint{2.526369in}{1.629906in}}{\pgfqpoint{2.518469in}{1.633179in}}{\pgfqpoint{2.510232in}{1.633179in}}%
\pgfpathcurveto{\pgfqpoint{2.501996in}{1.633179in}}{\pgfqpoint{2.494096in}{1.629906in}}{\pgfqpoint{2.488272in}{1.624082in}}%
\pgfpathcurveto{\pgfqpoint{2.482448in}{1.618258in}}{\pgfqpoint{2.479176in}{1.610358in}}{\pgfqpoint{2.479176in}{1.602122in}}%
\pgfpathcurveto{\pgfqpoint{2.479176in}{1.593886in}}{\pgfqpoint{2.482448in}{1.585986in}}{\pgfqpoint{2.488272in}{1.580162in}}%
\pgfpathcurveto{\pgfqpoint{2.494096in}{1.574338in}}{\pgfqpoint{2.501996in}{1.571066in}}{\pgfqpoint{2.510232in}{1.571066in}}%
\pgfpathclose%
\pgfusepath{stroke,fill}%
\end{pgfscope}%
\begin{pgfscope}%
\pgfpathrectangle{\pgfqpoint{0.100000in}{0.212622in}}{\pgfqpoint{3.696000in}{3.696000in}}%
\pgfusepath{clip}%
\pgfsetbuttcap%
\pgfsetroundjoin%
\definecolor{currentfill}{rgb}{0.121569,0.466667,0.705882}%
\pgfsetfillcolor{currentfill}%
\pgfsetfillopacity{0.804373}%
\pgfsetlinewidth{1.003750pt}%
\definecolor{currentstroke}{rgb}{0.121569,0.466667,0.705882}%
\pgfsetstrokecolor{currentstroke}%
\pgfsetstrokeopacity{0.804373}%
\pgfsetdash{}{0pt}%
\pgfpathmoveto{\pgfqpoint{1.346378in}{1.033728in}}%
\pgfpathcurveto{\pgfqpoint{1.354614in}{1.033728in}}{\pgfqpoint{1.362514in}{1.037001in}}{\pgfqpoint{1.368338in}{1.042824in}}%
\pgfpathcurveto{\pgfqpoint{1.374162in}{1.048648in}}{\pgfqpoint{1.377434in}{1.056548in}}{\pgfqpoint{1.377434in}{1.064785in}}%
\pgfpathcurveto{\pgfqpoint{1.377434in}{1.073021in}}{\pgfqpoint{1.374162in}{1.080921in}}{\pgfqpoint{1.368338in}{1.086745in}}%
\pgfpathcurveto{\pgfqpoint{1.362514in}{1.092569in}}{\pgfqpoint{1.354614in}{1.095841in}}{\pgfqpoint{1.346378in}{1.095841in}}%
\pgfpathcurveto{\pgfqpoint{1.338141in}{1.095841in}}{\pgfqpoint{1.330241in}{1.092569in}}{\pgfqpoint{1.324417in}{1.086745in}}%
\pgfpathcurveto{\pgfqpoint{1.318593in}{1.080921in}}{\pgfqpoint{1.315321in}{1.073021in}}{\pgfqpoint{1.315321in}{1.064785in}}%
\pgfpathcurveto{\pgfqpoint{1.315321in}{1.056548in}}{\pgfqpoint{1.318593in}{1.048648in}}{\pgfqpoint{1.324417in}{1.042824in}}%
\pgfpathcurveto{\pgfqpoint{1.330241in}{1.037001in}}{\pgfqpoint{1.338141in}{1.033728in}}{\pgfqpoint{1.346378in}{1.033728in}}%
\pgfpathclose%
\pgfusepath{stroke,fill}%
\end{pgfscope}%
\begin{pgfscope}%
\pgfpathrectangle{\pgfqpoint{0.100000in}{0.212622in}}{\pgfqpoint{3.696000in}{3.696000in}}%
\pgfusepath{clip}%
\pgfsetbuttcap%
\pgfsetroundjoin%
\definecolor{currentfill}{rgb}{0.121569,0.466667,0.705882}%
\pgfsetfillcolor{currentfill}%
\pgfsetfillopacity{0.804560}%
\pgfsetlinewidth{1.003750pt}%
\definecolor{currentstroke}{rgb}{0.121569,0.466667,0.705882}%
\pgfsetstrokecolor{currentstroke}%
\pgfsetstrokeopacity{0.804560}%
\pgfsetdash{}{0pt}%
\pgfpathmoveto{\pgfqpoint{1.271578in}{2.101924in}}%
\pgfpathcurveto{\pgfqpoint{1.279814in}{2.101924in}}{\pgfqpoint{1.287714in}{2.105196in}}{\pgfqpoint{1.293538in}{2.111020in}}%
\pgfpathcurveto{\pgfqpoint{1.299362in}{2.116844in}}{\pgfqpoint{1.302634in}{2.124744in}}{\pgfqpoint{1.302634in}{2.132980in}}%
\pgfpathcurveto{\pgfqpoint{1.302634in}{2.141217in}}{\pgfqpoint{1.299362in}{2.149117in}}{\pgfqpoint{1.293538in}{2.154941in}}%
\pgfpathcurveto{\pgfqpoint{1.287714in}{2.160765in}}{\pgfqpoint{1.279814in}{2.164037in}}{\pgfqpoint{1.271578in}{2.164037in}}%
\pgfpathcurveto{\pgfqpoint{1.263342in}{2.164037in}}{\pgfqpoint{1.255442in}{2.160765in}}{\pgfqpoint{1.249618in}{2.154941in}}%
\pgfpathcurveto{\pgfqpoint{1.243794in}{2.149117in}}{\pgfqpoint{1.240521in}{2.141217in}}{\pgfqpoint{1.240521in}{2.132980in}}%
\pgfpathcurveto{\pgfqpoint{1.240521in}{2.124744in}}{\pgfqpoint{1.243794in}{2.116844in}}{\pgfqpoint{1.249618in}{2.111020in}}%
\pgfpathcurveto{\pgfqpoint{1.255442in}{2.105196in}}{\pgfqpoint{1.263342in}{2.101924in}}{\pgfqpoint{1.271578in}{2.101924in}}%
\pgfpathclose%
\pgfusepath{stroke,fill}%
\end{pgfscope}%
\begin{pgfscope}%
\pgfpathrectangle{\pgfqpoint{0.100000in}{0.212622in}}{\pgfqpoint{3.696000in}{3.696000in}}%
\pgfusepath{clip}%
\pgfsetbuttcap%
\pgfsetroundjoin%
\definecolor{currentfill}{rgb}{0.121569,0.466667,0.705882}%
\pgfsetfillcolor{currentfill}%
\pgfsetfillopacity{0.805057}%
\pgfsetlinewidth{1.003750pt}%
\definecolor{currentstroke}{rgb}{0.121569,0.466667,0.705882}%
\pgfsetstrokecolor{currentstroke}%
\pgfsetstrokeopacity{0.805057}%
\pgfsetdash{}{0pt}%
\pgfpathmoveto{\pgfqpoint{1.273572in}{2.102408in}}%
\pgfpathcurveto{\pgfqpoint{1.281809in}{2.102408in}}{\pgfqpoint{1.289709in}{2.105680in}}{\pgfqpoint{1.295533in}{2.111504in}}%
\pgfpathcurveto{\pgfqpoint{1.301357in}{2.117328in}}{\pgfqpoint{1.304629in}{2.125228in}}{\pgfqpoint{1.304629in}{2.133464in}}%
\pgfpathcurveto{\pgfqpoint{1.304629in}{2.141701in}}{\pgfqpoint{1.301357in}{2.149601in}}{\pgfqpoint{1.295533in}{2.155425in}}%
\pgfpathcurveto{\pgfqpoint{1.289709in}{2.161249in}}{\pgfqpoint{1.281809in}{2.164521in}}{\pgfqpoint{1.273572in}{2.164521in}}%
\pgfpathcurveto{\pgfqpoint{1.265336in}{2.164521in}}{\pgfqpoint{1.257436in}{2.161249in}}{\pgfqpoint{1.251612in}{2.155425in}}%
\pgfpathcurveto{\pgfqpoint{1.245788in}{2.149601in}}{\pgfqpoint{1.242516in}{2.141701in}}{\pgfqpoint{1.242516in}{2.133464in}}%
\pgfpathcurveto{\pgfqpoint{1.242516in}{2.125228in}}{\pgfqpoint{1.245788in}{2.117328in}}{\pgfqpoint{1.251612in}{2.111504in}}%
\pgfpathcurveto{\pgfqpoint{1.257436in}{2.105680in}}{\pgfqpoint{1.265336in}{2.102408in}}{\pgfqpoint{1.273572in}{2.102408in}}%
\pgfpathclose%
\pgfusepath{stroke,fill}%
\end{pgfscope}%
\begin{pgfscope}%
\pgfpathrectangle{\pgfqpoint{0.100000in}{0.212622in}}{\pgfqpoint{3.696000in}{3.696000in}}%
\pgfusepath{clip}%
\pgfsetbuttcap%
\pgfsetroundjoin%
\definecolor{currentfill}{rgb}{0.121569,0.466667,0.705882}%
\pgfsetfillcolor{currentfill}%
\pgfsetfillopacity{0.805996}%
\pgfsetlinewidth{1.003750pt}%
\definecolor{currentstroke}{rgb}{0.121569,0.466667,0.705882}%
\pgfsetstrokecolor{currentstroke}%
\pgfsetstrokeopacity{0.805996}%
\pgfsetdash{}{0pt}%
\pgfpathmoveto{\pgfqpoint{1.277174in}{2.103289in}}%
\pgfpathcurveto{\pgfqpoint{1.285410in}{2.103289in}}{\pgfqpoint{1.293310in}{2.106561in}}{\pgfqpoint{1.299134in}{2.112385in}}%
\pgfpathcurveto{\pgfqpoint{1.304958in}{2.118209in}}{\pgfqpoint{1.308231in}{2.126109in}}{\pgfqpoint{1.308231in}{2.134346in}}%
\pgfpathcurveto{\pgfqpoint{1.308231in}{2.142582in}}{\pgfqpoint{1.304958in}{2.150482in}}{\pgfqpoint{1.299134in}{2.156306in}}%
\pgfpathcurveto{\pgfqpoint{1.293310in}{2.162130in}}{\pgfqpoint{1.285410in}{2.165402in}}{\pgfqpoint{1.277174in}{2.165402in}}%
\pgfpathcurveto{\pgfqpoint{1.268938in}{2.165402in}}{\pgfqpoint{1.261038in}{2.162130in}}{\pgfqpoint{1.255214in}{2.156306in}}%
\pgfpathcurveto{\pgfqpoint{1.249390in}{2.150482in}}{\pgfqpoint{1.246118in}{2.142582in}}{\pgfqpoint{1.246118in}{2.134346in}}%
\pgfpathcurveto{\pgfqpoint{1.246118in}{2.126109in}}{\pgfqpoint{1.249390in}{2.118209in}}{\pgfqpoint{1.255214in}{2.112385in}}%
\pgfpathcurveto{\pgfqpoint{1.261038in}{2.106561in}}{\pgfqpoint{1.268938in}{2.103289in}}{\pgfqpoint{1.277174in}{2.103289in}}%
\pgfpathclose%
\pgfusepath{stroke,fill}%
\end{pgfscope}%
\begin{pgfscope}%
\pgfpathrectangle{\pgfqpoint{0.100000in}{0.212622in}}{\pgfqpoint{3.696000in}{3.696000in}}%
\pgfusepath{clip}%
\pgfsetbuttcap%
\pgfsetroundjoin%
\definecolor{currentfill}{rgb}{0.121569,0.466667,0.705882}%
\pgfsetfillcolor{currentfill}%
\pgfsetfillopacity{0.806005}%
\pgfsetlinewidth{1.003750pt}%
\definecolor{currentstroke}{rgb}{0.121569,0.466667,0.705882}%
\pgfsetstrokecolor{currentstroke}%
\pgfsetstrokeopacity{0.806005}%
\pgfsetdash{}{0pt}%
\pgfpathmoveto{\pgfqpoint{1.277203in}{2.103298in}}%
\pgfpathcurveto{\pgfqpoint{1.285439in}{2.103298in}}{\pgfqpoint{1.293339in}{2.106570in}}{\pgfqpoint{1.299163in}{2.112394in}}%
\pgfpathcurveto{\pgfqpoint{1.304987in}{2.118218in}}{\pgfqpoint{1.308260in}{2.126118in}}{\pgfqpoint{1.308260in}{2.134354in}}%
\pgfpathcurveto{\pgfqpoint{1.308260in}{2.142590in}}{\pgfqpoint{1.304987in}{2.150490in}}{\pgfqpoint{1.299163in}{2.156314in}}%
\pgfpathcurveto{\pgfqpoint{1.293339in}{2.162138in}}{\pgfqpoint{1.285439in}{2.165411in}}{\pgfqpoint{1.277203in}{2.165411in}}%
\pgfpathcurveto{\pgfqpoint{1.268967in}{2.165411in}}{\pgfqpoint{1.261067in}{2.162138in}}{\pgfqpoint{1.255243in}{2.156314in}}%
\pgfpathcurveto{\pgfqpoint{1.249419in}{2.150490in}}{\pgfqpoint{1.246147in}{2.142590in}}{\pgfqpoint{1.246147in}{2.134354in}}%
\pgfpathcurveto{\pgfqpoint{1.246147in}{2.126118in}}{\pgfqpoint{1.249419in}{2.118218in}}{\pgfqpoint{1.255243in}{2.112394in}}%
\pgfpathcurveto{\pgfqpoint{1.261067in}{2.106570in}}{\pgfqpoint{1.268967in}{2.103298in}}{\pgfqpoint{1.277203in}{2.103298in}}%
\pgfpathclose%
\pgfusepath{stroke,fill}%
\end{pgfscope}%
\begin{pgfscope}%
\pgfpathrectangle{\pgfqpoint{0.100000in}{0.212622in}}{\pgfqpoint{3.696000in}{3.696000in}}%
\pgfusepath{clip}%
\pgfsetbuttcap%
\pgfsetroundjoin%
\definecolor{currentfill}{rgb}{0.121569,0.466667,0.705882}%
\pgfsetfillcolor{currentfill}%
\pgfsetfillopacity{0.806022}%
\pgfsetlinewidth{1.003750pt}%
\definecolor{currentstroke}{rgb}{0.121569,0.466667,0.705882}%
\pgfsetstrokecolor{currentstroke}%
\pgfsetstrokeopacity{0.806022}%
\pgfsetdash{}{0pt}%
\pgfpathmoveto{\pgfqpoint{1.277254in}{2.103314in}}%
\pgfpathcurveto{\pgfqpoint{1.285490in}{2.103314in}}{\pgfqpoint{1.293390in}{2.106586in}}{\pgfqpoint{1.299214in}{2.112410in}}%
\pgfpathcurveto{\pgfqpoint{1.305038in}{2.118234in}}{\pgfqpoint{1.308310in}{2.126134in}}{\pgfqpoint{1.308310in}{2.134371in}}%
\pgfpathcurveto{\pgfqpoint{1.308310in}{2.142607in}}{\pgfqpoint{1.305038in}{2.150507in}}{\pgfqpoint{1.299214in}{2.156331in}}%
\pgfpathcurveto{\pgfqpoint{1.293390in}{2.162155in}}{\pgfqpoint{1.285490in}{2.165427in}}{\pgfqpoint{1.277254in}{2.165427in}}%
\pgfpathcurveto{\pgfqpoint{1.269018in}{2.165427in}}{\pgfqpoint{1.261118in}{2.162155in}}{\pgfqpoint{1.255294in}{2.156331in}}%
\pgfpathcurveto{\pgfqpoint{1.249470in}{2.150507in}}{\pgfqpoint{1.246197in}{2.142607in}}{\pgfqpoint{1.246197in}{2.134371in}}%
\pgfpathcurveto{\pgfqpoint{1.246197in}{2.126134in}}{\pgfqpoint{1.249470in}{2.118234in}}{\pgfqpoint{1.255294in}{2.112410in}}%
\pgfpathcurveto{\pgfqpoint{1.261118in}{2.106586in}}{\pgfqpoint{1.269018in}{2.103314in}}{\pgfqpoint{1.277254in}{2.103314in}}%
\pgfpathclose%
\pgfusepath{stroke,fill}%
\end{pgfscope}%
\begin{pgfscope}%
\pgfpathrectangle{\pgfqpoint{0.100000in}{0.212622in}}{\pgfqpoint{3.696000in}{3.696000in}}%
\pgfusepath{clip}%
\pgfsetbuttcap%
\pgfsetroundjoin%
\definecolor{currentfill}{rgb}{0.121569,0.466667,0.705882}%
\pgfsetfillcolor{currentfill}%
\pgfsetfillopacity{0.806057}%
\pgfsetlinewidth{1.003750pt}%
\definecolor{currentstroke}{rgb}{0.121569,0.466667,0.705882}%
\pgfsetstrokecolor{currentstroke}%
\pgfsetstrokeopacity{0.806057}%
\pgfsetdash{}{0pt}%
\pgfpathmoveto{\pgfqpoint{1.277342in}{2.103340in}}%
\pgfpathcurveto{\pgfqpoint{1.285578in}{2.103340in}}{\pgfqpoint{1.293478in}{2.106612in}}{\pgfqpoint{1.299302in}{2.112436in}}%
\pgfpathcurveto{\pgfqpoint{1.305126in}{2.118260in}}{\pgfqpoint{1.308398in}{2.126160in}}{\pgfqpoint{1.308398in}{2.134396in}}%
\pgfpathcurveto{\pgfqpoint{1.308398in}{2.142633in}}{\pgfqpoint{1.305126in}{2.150533in}}{\pgfqpoint{1.299302in}{2.156357in}}%
\pgfpathcurveto{\pgfqpoint{1.293478in}{2.162181in}}{\pgfqpoint{1.285578in}{2.165453in}}{\pgfqpoint{1.277342in}{2.165453in}}%
\pgfpathcurveto{\pgfqpoint{1.269105in}{2.165453in}}{\pgfqpoint{1.261205in}{2.162181in}}{\pgfqpoint{1.255382in}{2.156357in}}%
\pgfpathcurveto{\pgfqpoint{1.249558in}{2.150533in}}{\pgfqpoint{1.246285in}{2.142633in}}{\pgfqpoint{1.246285in}{2.134396in}}%
\pgfpathcurveto{\pgfqpoint{1.246285in}{2.126160in}}{\pgfqpoint{1.249558in}{2.118260in}}{\pgfqpoint{1.255382in}{2.112436in}}%
\pgfpathcurveto{\pgfqpoint{1.261205in}{2.106612in}}{\pgfqpoint{1.269105in}{2.103340in}}{\pgfqpoint{1.277342in}{2.103340in}}%
\pgfpathclose%
\pgfusepath{stroke,fill}%
\end{pgfscope}%
\begin{pgfscope}%
\pgfpathrectangle{\pgfqpoint{0.100000in}{0.212622in}}{\pgfqpoint{3.696000in}{3.696000in}}%
\pgfusepath{clip}%
\pgfsetbuttcap%
\pgfsetroundjoin%
\definecolor{currentfill}{rgb}{0.121569,0.466667,0.705882}%
\pgfsetfillcolor{currentfill}%
\pgfsetfillopacity{0.806127}%
\pgfsetlinewidth{1.003750pt}%
\definecolor{currentstroke}{rgb}{0.121569,0.466667,0.705882}%
\pgfsetstrokecolor{currentstroke}%
\pgfsetstrokeopacity{0.806127}%
\pgfsetdash{}{0pt}%
\pgfpathmoveto{\pgfqpoint{1.277490in}{2.103382in}}%
\pgfpathcurveto{\pgfqpoint{1.285727in}{2.103382in}}{\pgfqpoint{1.293627in}{2.106654in}}{\pgfqpoint{1.299451in}{2.112478in}}%
\pgfpathcurveto{\pgfqpoint{1.305275in}{2.118302in}}{\pgfqpoint{1.308547in}{2.126202in}}{\pgfqpoint{1.308547in}{2.134438in}}%
\pgfpathcurveto{\pgfqpoint{1.308547in}{2.142674in}}{\pgfqpoint{1.305275in}{2.150574in}}{\pgfqpoint{1.299451in}{2.156398in}}%
\pgfpathcurveto{\pgfqpoint{1.293627in}{2.162222in}}{\pgfqpoint{1.285727in}{2.165495in}}{\pgfqpoint{1.277490in}{2.165495in}}%
\pgfpathcurveto{\pgfqpoint{1.269254in}{2.165495in}}{\pgfqpoint{1.261354in}{2.162222in}}{\pgfqpoint{1.255530in}{2.156398in}}%
\pgfpathcurveto{\pgfqpoint{1.249706in}{2.150574in}}{\pgfqpoint{1.246434in}{2.142674in}}{\pgfqpoint{1.246434in}{2.134438in}}%
\pgfpathcurveto{\pgfqpoint{1.246434in}{2.126202in}}{\pgfqpoint{1.249706in}{2.118302in}}{\pgfqpoint{1.255530in}{2.112478in}}%
\pgfpathcurveto{\pgfqpoint{1.261354in}{2.106654in}}{\pgfqpoint{1.269254in}{2.103382in}}{\pgfqpoint{1.277490in}{2.103382in}}%
\pgfpathclose%
\pgfusepath{stroke,fill}%
\end{pgfscope}%
\begin{pgfscope}%
\pgfpathrectangle{\pgfqpoint{0.100000in}{0.212622in}}{\pgfqpoint{3.696000in}{3.696000in}}%
\pgfusepath{clip}%
\pgfsetbuttcap%
\pgfsetroundjoin%
\definecolor{currentfill}{rgb}{0.121569,0.466667,0.705882}%
\pgfsetfillcolor{currentfill}%
\pgfsetfillopacity{0.806264}%
\pgfsetlinewidth{1.003750pt}%
\definecolor{currentstroke}{rgb}{0.121569,0.466667,0.705882}%
\pgfsetstrokecolor{currentstroke}%
\pgfsetstrokeopacity{0.806264}%
\pgfsetdash{}{0pt}%
\pgfpathmoveto{\pgfqpoint{1.277741in}{2.103444in}}%
\pgfpathcurveto{\pgfqpoint{1.285977in}{2.103444in}}{\pgfqpoint{1.293877in}{2.106716in}}{\pgfqpoint{1.299701in}{2.112540in}}%
\pgfpathcurveto{\pgfqpoint{1.305525in}{2.118364in}}{\pgfqpoint{1.308797in}{2.126264in}}{\pgfqpoint{1.308797in}{2.134500in}}%
\pgfpathcurveto{\pgfqpoint{1.308797in}{2.142737in}}{\pgfqpoint{1.305525in}{2.150637in}}{\pgfqpoint{1.299701in}{2.156460in}}%
\pgfpathcurveto{\pgfqpoint{1.293877in}{2.162284in}}{\pgfqpoint{1.285977in}{2.165557in}}{\pgfqpoint{1.277741in}{2.165557in}}%
\pgfpathcurveto{\pgfqpoint{1.269504in}{2.165557in}}{\pgfqpoint{1.261604in}{2.162284in}}{\pgfqpoint{1.255780in}{2.156460in}}%
\pgfpathcurveto{\pgfqpoint{1.249956in}{2.150637in}}{\pgfqpoint{1.246684in}{2.142737in}}{\pgfqpoint{1.246684in}{2.134500in}}%
\pgfpathcurveto{\pgfqpoint{1.246684in}{2.126264in}}{\pgfqpoint{1.249956in}{2.118364in}}{\pgfqpoint{1.255780in}{2.112540in}}%
\pgfpathcurveto{\pgfqpoint{1.261604in}{2.106716in}}{\pgfqpoint{1.269504in}{2.103444in}}{\pgfqpoint{1.277741in}{2.103444in}}%
\pgfpathclose%
\pgfusepath{stroke,fill}%
\end{pgfscope}%
\begin{pgfscope}%
\pgfpathrectangle{\pgfqpoint{0.100000in}{0.212622in}}{\pgfqpoint{3.696000in}{3.696000in}}%
\pgfusepath{clip}%
\pgfsetbuttcap%
\pgfsetroundjoin%
\definecolor{currentfill}{rgb}{0.121569,0.466667,0.705882}%
\pgfsetfillcolor{currentfill}%
\pgfsetfillopacity{0.806527}%
\pgfsetlinewidth{1.003750pt}%
\definecolor{currentstroke}{rgb}{0.121569,0.466667,0.705882}%
\pgfsetstrokecolor{currentstroke}%
\pgfsetstrokeopacity{0.806527}%
\pgfsetdash{}{0pt}%
\pgfpathmoveto{\pgfqpoint{1.278158in}{2.103542in}}%
\pgfpathcurveto{\pgfqpoint{1.286394in}{2.103542in}}{\pgfqpoint{1.294294in}{2.106814in}}{\pgfqpoint{1.300118in}{2.112638in}}%
\pgfpathcurveto{\pgfqpoint{1.305942in}{2.118462in}}{\pgfqpoint{1.309215in}{2.126362in}}{\pgfqpoint{1.309215in}{2.134598in}}%
\pgfpathcurveto{\pgfqpoint{1.309215in}{2.142835in}}{\pgfqpoint{1.305942in}{2.150735in}}{\pgfqpoint{1.300118in}{2.156559in}}%
\pgfpathcurveto{\pgfqpoint{1.294294in}{2.162382in}}{\pgfqpoint{1.286394in}{2.165655in}}{\pgfqpoint{1.278158in}{2.165655in}}%
\pgfpathcurveto{\pgfqpoint{1.269922in}{2.165655in}}{\pgfqpoint{1.262022in}{2.162382in}}{\pgfqpoint{1.256198in}{2.156559in}}%
\pgfpathcurveto{\pgfqpoint{1.250374in}{2.150735in}}{\pgfqpoint{1.247102in}{2.142835in}}{\pgfqpoint{1.247102in}{2.134598in}}%
\pgfpathcurveto{\pgfqpoint{1.247102in}{2.126362in}}{\pgfqpoint{1.250374in}{2.118462in}}{\pgfqpoint{1.256198in}{2.112638in}}%
\pgfpathcurveto{\pgfqpoint{1.262022in}{2.106814in}}{\pgfqpoint{1.269922in}{2.103542in}}{\pgfqpoint{1.278158in}{2.103542in}}%
\pgfpathclose%
\pgfusepath{stroke,fill}%
\end{pgfscope}%
\begin{pgfscope}%
\pgfpathrectangle{\pgfqpoint{0.100000in}{0.212622in}}{\pgfqpoint{3.696000in}{3.696000in}}%
\pgfusepath{clip}%
\pgfsetbuttcap%
\pgfsetroundjoin%
\definecolor{currentfill}{rgb}{0.121569,0.466667,0.705882}%
\pgfsetfillcolor{currentfill}%
\pgfsetfillopacity{0.807027}%
\pgfsetlinewidth{1.003750pt}%
\definecolor{currentstroke}{rgb}{0.121569,0.466667,0.705882}%
\pgfsetstrokecolor{currentstroke}%
\pgfsetstrokeopacity{0.807027}%
\pgfsetdash{}{0pt}%
\pgfpathmoveto{\pgfqpoint{1.278848in}{2.103694in}}%
\pgfpathcurveto{\pgfqpoint{1.287084in}{2.103694in}}{\pgfqpoint{1.294984in}{2.106967in}}{\pgfqpoint{1.300808in}{2.112790in}}%
\pgfpathcurveto{\pgfqpoint{1.306632in}{2.118614in}}{\pgfqpoint{1.309905in}{2.126514in}}{\pgfqpoint{1.309905in}{2.134751in}}%
\pgfpathcurveto{\pgfqpoint{1.309905in}{2.142987in}}{\pgfqpoint{1.306632in}{2.150887in}}{\pgfqpoint{1.300808in}{2.156711in}}%
\pgfpathcurveto{\pgfqpoint{1.294984in}{2.162535in}}{\pgfqpoint{1.287084in}{2.165807in}}{\pgfqpoint{1.278848in}{2.165807in}}%
\pgfpathcurveto{\pgfqpoint{1.270612in}{2.165807in}}{\pgfqpoint{1.262712in}{2.162535in}}{\pgfqpoint{1.256888in}{2.156711in}}%
\pgfpathcurveto{\pgfqpoint{1.251064in}{2.150887in}}{\pgfqpoint{1.247792in}{2.142987in}}{\pgfqpoint{1.247792in}{2.134751in}}%
\pgfpathcurveto{\pgfqpoint{1.247792in}{2.126514in}}{\pgfqpoint{1.251064in}{2.118614in}}{\pgfqpoint{1.256888in}{2.112790in}}%
\pgfpathcurveto{\pgfqpoint{1.262712in}{2.106967in}}{\pgfqpoint{1.270612in}{2.103694in}}{\pgfqpoint{1.278848in}{2.103694in}}%
\pgfpathclose%
\pgfusepath{stroke,fill}%
\end{pgfscope}%
\begin{pgfscope}%
\pgfpathrectangle{\pgfqpoint{0.100000in}{0.212622in}}{\pgfqpoint{3.696000in}{3.696000in}}%
\pgfusepath{clip}%
\pgfsetbuttcap%
\pgfsetroundjoin%
\definecolor{currentfill}{rgb}{0.121569,0.466667,0.705882}%
\pgfsetfillcolor{currentfill}%
\pgfsetfillopacity{0.807971}%
\pgfsetlinewidth{1.003750pt}%
\definecolor{currentstroke}{rgb}{0.121569,0.466667,0.705882}%
\pgfsetstrokecolor{currentstroke}%
\pgfsetstrokeopacity{0.807971}%
\pgfsetdash{}{0pt}%
\pgfpathmoveto{\pgfqpoint{1.279978in}{2.103930in}}%
\pgfpathcurveto{\pgfqpoint{1.288214in}{2.103930in}}{\pgfqpoint{1.296115in}{2.107203in}}{\pgfqpoint{1.301938in}{2.113026in}}%
\pgfpathcurveto{\pgfqpoint{1.307762in}{2.118850in}}{\pgfqpoint{1.311035in}{2.126750in}}{\pgfqpoint{1.311035in}{2.134987in}}%
\pgfpathcurveto{\pgfqpoint{1.311035in}{2.143223in}}{\pgfqpoint{1.307762in}{2.151123in}}{\pgfqpoint{1.301938in}{2.156947in}}%
\pgfpathcurveto{\pgfqpoint{1.296115in}{2.162771in}}{\pgfqpoint{1.288214in}{2.166043in}}{\pgfqpoint{1.279978in}{2.166043in}}%
\pgfpathcurveto{\pgfqpoint{1.271742in}{2.166043in}}{\pgfqpoint{1.263842in}{2.162771in}}{\pgfqpoint{1.258018in}{2.156947in}}%
\pgfpathcurveto{\pgfqpoint{1.252194in}{2.151123in}}{\pgfqpoint{1.248922in}{2.143223in}}{\pgfqpoint{1.248922in}{2.134987in}}%
\pgfpathcurveto{\pgfqpoint{1.248922in}{2.126750in}}{\pgfqpoint{1.252194in}{2.118850in}}{\pgfqpoint{1.258018in}{2.113026in}}%
\pgfpathcurveto{\pgfqpoint{1.263842in}{2.107203in}}{\pgfqpoint{1.271742in}{2.103930in}}{\pgfqpoint{1.279978in}{2.103930in}}%
\pgfpathclose%
\pgfusepath{stroke,fill}%
\end{pgfscope}%
\begin{pgfscope}%
\pgfpathrectangle{\pgfqpoint{0.100000in}{0.212622in}}{\pgfqpoint{3.696000in}{3.696000in}}%
\pgfusepath{clip}%
\pgfsetbuttcap%
\pgfsetroundjoin%
\definecolor{currentfill}{rgb}{0.121569,0.466667,0.705882}%
\pgfsetfillcolor{currentfill}%
\pgfsetfillopacity{0.807979}%
\pgfsetlinewidth{1.003750pt}%
\definecolor{currentstroke}{rgb}{0.121569,0.466667,0.705882}%
\pgfsetstrokecolor{currentstroke}%
\pgfsetstrokeopacity{0.807979}%
\pgfsetdash{}{0pt}%
\pgfpathmoveto{\pgfqpoint{1.279986in}{2.103932in}}%
\pgfpathcurveto{\pgfqpoint{1.288222in}{2.103932in}}{\pgfqpoint{1.296122in}{2.107204in}}{\pgfqpoint{1.301946in}{2.113028in}}%
\pgfpathcurveto{\pgfqpoint{1.307770in}{2.118852in}}{\pgfqpoint{1.311042in}{2.126752in}}{\pgfqpoint{1.311042in}{2.134988in}}%
\pgfpathcurveto{\pgfqpoint{1.311042in}{2.143225in}}{\pgfqpoint{1.307770in}{2.151125in}}{\pgfqpoint{1.301946in}{2.156949in}}%
\pgfpathcurveto{\pgfqpoint{1.296122in}{2.162773in}}{\pgfqpoint{1.288222in}{2.166045in}}{\pgfqpoint{1.279986in}{2.166045in}}%
\pgfpathcurveto{\pgfqpoint{1.271750in}{2.166045in}}{\pgfqpoint{1.263850in}{2.162773in}}{\pgfqpoint{1.258026in}{2.156949in}}%
\pgfpathcurveto{\pgfqpoint{1.252202in}{2.151125in}}{\pgfqpoint{1.248929in}{2.143225in}}{\pgfqpoint{1.248929in}{2.134988in}}%
\pgfpathcurveto{\pgfqpoint{1.248929in}{2.126752in}}{\pgfqpoint{1.252202in}{2.118852in}}{\pgfqpoint{1.258026in}{2.113028in}}%
\pgfpathcurveto{\pgfqpoint{1.263850in}{2.107204in}}{\pgfqpoint{1.271750in}{2.103932in}}{\pgfqpoint{1.279986in}{2.103932in}}%
\pgfpathclose%
\pgfusepath{stroke,fill}%
\end{pgfscope}%
\begin{pgfscope}%
\pgfpathrectangle{\pgfqpoint{0.100000in}{0.212622in}}{\pgfqpoint{3.696000in}{3.696000in}}%
\pgfusepath{clip}%
\pgfsetbuttcap%
\pgfsetroundjoin%
\definecolor{currentfill}{rgb}{0.121569,0.466667,0.705882}%
\pgfsetfillcolor{currentfill}%
\pgfsetfillopacity{0.807993}%
\pgfsetlinewidth{1.003750pt}%
\definecolor{currentstroke}{rgb}{0.121569,0.466667,0.705882}%
\pgfsetstrokecolor{currentstroke}%
\pgfsetstrokeopacity{0.807993}%
\pgfsetdash{}{0pt}%
\pgfpathmoveto{\pgfqpoint{1.279998in}{2.103934in}}%
\pgfpathcurveto{\pgfqpoint{1.288235in}{2.103934in}}{\pgfqpoint{1.296135in}{2.107207in}}{\pgfqpoint{1.301959in}{2.113031in}}%
\pgfpathcurveto{\pgfqpoint{1.307783in}{2.118854in}}{\pgfqpoint{1.311055in}{2.126755in}}{\pgfqpoint{1.311055in}{2.134991in}}%
\pgfpathcurveto{\pgfqpoint{1.311055in}{2.143227in}}{\pgfqpoint{1.307783in}{2.151127in}}{\pgfqpoint{1.301959in}{2.156951in}}%
\pgfpathcurveto{\pgfqpoint{1.296135in}{2.162775in}}{\pgfqpoint{1.288235in}{2.166047in}}{\pgfqpoint{1.279998in}{2.166047in}}%
\pgfpathcurveto{\pgfqpoint{1.271762in}{2.166047in}}{\pgfqpoint{1.263862in}{2.162775in}}{\pgfqpoint{1.258038in}{2.156951in}}%
\pgfpathcurveto{\pgfqpoint{1.252214in}{2.151127in}}{\pgfqpoint{1.248942in}{2.143227in}}{\pgfqpoint{1.248942in}{2.134991in}}%
\pgfpathcurveto{\pgfqpoint{1.248942in}{2.126755in}}{\pgfqpoint{1.252214in}{2.118854in}}{\pgfqpoint{1.258038in}{2.113031in}}%
\pgfpathcurveto{\pgfqpoint{1.263862in}{2.107207in}}{\pgfqpoint{1.271762in}{2.103934in}}{\pgfqpoint{1.279998in}{2.103934in}}%
\pgfpathclose%
\pgfusepath{stroke,fill}%
\end{pgfscope}%
\begin{pgfscope}%
\pgfpathrectangle{\pgfqpoint{0.100000in}{0.212622in}}{\pgfqpoint{3.696000in}{3.696000in}}%
\pgfusepath{clip}%
\pgfsetbuttcap%
\pgfsetroundjoin%
\definecolor{currentfill}{rgb}{0.121569,0.466667,0.705882}%
\pgfsetfillcolor{currentfill}%
\pgfsetfillopacity{0.808019}%
\pgfsetlinewidth{1.003750pt}%
\definecolor{currentstroke}{rgb}{0.121569,0.466667,0.705882}%
\pgfsetstrokecolor{currentstroke}%
\pgfsetstrokeopacity{0.808019}%
\pgfsetdash{}{0pt}%
\pgfpathmoveto{\pgfqpoint{1.280018in}{2.103938in}}%
\pgfpathcurveto{\pgfqpoint{1.288255in}{2.103938in}}{\pgfqpoint{1.296155in}{2.107210in}}{\pgfqpoint{1.301979in}{2.113034in}}%
\pgfpathcurveto{\pgfqpoint{1.307803in}{2.118858in}}{\pgfqpoint{1.311075in}{2.126758in}}{\pgfqpoint{1.311075in}{2.134995in}}%
\pgfpathcurveto{\pgfqpoint{1.311075in}{2.143231in}}{\pgfqpoint{1.307803in}{2.151131in}}{\pgfqpoint{1.301979in}{2.156955in}}%
\pgfpathcurveto{\pgfqpoint{1.296155in}{2.162779in}}{\pgfqpoint{1.288255in}{2.166051in}}{\pgfqpoint{1.280018in}{2.166051in}}%
\pgfpathcurveto{\pgfqpoint{1.271782in}{2.166051in}}{\pgfqpoint{1.263882in}{2.162779in}}{\pgfqpoint{1.258058in}{2.156955in}}%
\pgfpathcurveto{\pgfqpoint{1.252234in}{2.151131in}}{\pgfqpoint{1.248962in}{2.143231in}}{\pgfqpoint{1.248962in}{2.134995in}}%
\pgfpathcurveto{\pgfqpoint{1.248962in}{2.126758in}}{\pgfqpoint{1.252234in}{2.118858in}}{\pgfqpoint{1.258058in}{2.113034in}}%
\pgfpathcurveto{\pgfqpoint{1.263882in}{2.107210in}}{\pgfqpoint{1.271782in}{2.103938in}}{\pgfqpoint{1.280018in}{2.103938in}}%
\pgfpathclose%
\pgfusepath{stroke,fill}%
\end{pgfscope}%
\begin{pgfscope}%
\pgfpathrectangle{\pgfqpoint{0.100000in}{0.212622in}}{\pgfqpoint{3.696000in}{3.696000in}}%
\pgfusepath{clip}%
\pgfsetbuttcap%
\pgfsetroundjoin%
\definecolor{currentfill}{rgb}{0.121569,0.466667,0.705882}%
\pgfsetfillcolor{currentfill}%
\pgfsetfillopacity{0.808068}%
\pgfsetlinewidth{1.003750pt}%
\definecolor{currentstroke}{rgb}{0.121569,0.466667,0.705882}%
\pgfsetstrokecolor{currentstroke}%
\pgfsetstrokeopacity{0.808068}%
\pgfsetdash{}{0pt}%
\pgfpathmoveto{\pgfqpoint{1.280049in}{2.103945in}}%
\pgfpathcurveto{\pgfqpoint{1.288285in}{2.103945in}}{\pgfqpoint{1.296185in}{2.107217in}}{\pgfqpoint{1.302009in}{2.113041in}}%
\pgfpathcurveto{\pgfqpoint{1.307833in}{2.118865in}}{\pgfqpoint{1.311106in}{2.126765in}}{\pgfqpoint{1.311106in}{2.135001in}}%
\pgfpathcurveto{\pgfqpoint{1.311106in}{2.143237in}}{\pgfqpoint{1.307833in}{2.151137in}}{\pgfqpoint{1.302009in}{2.156961in}}%
\pgfpathcurveto{\pgfqpoint{1.296185in}{2.162785in}}{\pgfqpoint{1.288285in}{2.166058in}}{\pgfqpoint{1.280049in}{2.166058in}}%
\pgfpathcurveto{\pgfqpoint{1.271813in}{2.166058in}}{\pgfqpoint{1.263913in}{2.162785in}}{\pgfqpoint{1.258089in}{2.156961in}}%
\pgfpathcurveto{\pgfqpoint{1.252265in}{2.151137in}}{\pgfqpoint{1.248993in}{2.143237in}}{\pgfqpoint{1.248993in}{2.135001in}}%
\pgfpathcurveto{\pgfqpoint{1.248993in}{2.126765in}}{\pgfqpoint{1.252265in}{2.118865in}}{\pgfqpoint{1.258089in}{2.113041in}}%
\pgfpathcurveto{\pgfqpoint{1.263913in}{2.107217in}}{\pgfqpoint{1.271813in}{2.103945in}}{\pgfqpoint{1.280049in}{2.103945in}}%
\pgfpathclose%
\pgfusepath{stroke,fill}%
\end{pgfscope}%
\begin{pgfscope}%
\pgfpathrectangle{\pgfqpoint{0.100000in}{0.212622in}}{\pgfqpoint{3.696000in}{3.696000in}}%
\pgfusepath{clip}%
\pgfsetbuttcap%
\pgfsetroundjoin%
\definecolor{currentfill}{rgb}{0.121569,0.466667,0.705882}%
\pgfsetfillcolor{currentfill}%
\pgfsetfillopacity{0.808156}%
\pgfsetlinewidth{1.003750pt}%
\definecolor{currentstroke}{rgb}{0.121569,0.466667,0.705882}%
\pgfsetstrokecolor{currentstroke}%
\pgfsetstrokeopacity{0.808156}%
\pgfsetdash{}{0pt}%
\pgfpathmoveto{\pgfqpoint{1.280097in}{2.103955in}}%
\pgfpathcurveto{\pgfqpoint{1.288333in}{2.103955in}}{\pgfqpoint{1.296233in}{2.107227in}}{\pgfqpoint{1.302057in}{2.113051in}}%
\pgfpathcurveto{\pgfqpoint{1.307881in}{2.118875in}}{\pgfqpoint{1.311153in}{2.126775in}}{\pgfqpoint{1.311153in}{2.135011in}}%
\pgfpathcurveto{\pgfqpoint{1.311153in}{2.143248in}}{\pgfqpoint{1.307881in}{2.151148in}}{\pgfqpoint{1.302057in}{2.156972in}}%
\pgfpathcurveto{\pgfqpoint{1.296233in}{2.162795in}}{\pgfqpoint{1.288333in}{2.166068in}}{\pgfqpoint{1.280097in}{2.166068in}}%
\pgfpathcurveto{\pgfqpoint{1.271861in}{2.166068in}}{\pgfqpoint{1.263961in}{2.162795in}}{\pgfqpoint{1.258137in}{2.156972in}}%
\pgfpathcurveto{\pgfqpoint{1.252313in}{2.151148in}}{\pgfqpoint{1.249040in}{2.143248in}}{\pgfqpoint{1.249040in}{2.135011in}}%
\pgfpathcurveto{\pgfqpoint{1.249040in}{2.126775in}}{\pgfqpoint{1.252313in}{2.118875in}}{\pgfqpoint{1.258137in}{2.113051in}}%
\pgfpathcurveto{\pgfqpoint{1.263961in}{2.107227in}}{\pgfqpoint{1.271861in}{2.103955in}}{\pgfqpoint{1.280097in}{2.103955in}}%
\pgfpathclose%
\pgfusepath{stroke,fill}%
\end{pgfscope}%
\begin{pgfscope}%
\pgfpathrectangle{\pgfqpoint{0.100000in}{0.212622in}}{\pgfqpoint{3.696000in}{3.696000in}}%
\pgfusepath{clip}%
\pgfsetbuttcap%
\pgfsetroundjoin%
\definecolor{currentfill}{rgb}{0.121569,0.466667,0.705882}%
\pgfsetfillcolor{currentfill}%
\pgfsetfillopacity{0.808319}%
\pgfsetlinewidth{1.003750pt}%
\definecolor{currentstroke}{rgb}{0.121569,0.466667,0.705882}%
\pgfsetstrokecolor{currentstroke}%
\pgfsetstrokeopacity{0.808319}%
\pgfsetdash{}{0pt}%
\pgfpathmoveto{\pgfqpoint{1.280165in}{2.103971in}}%
\pgfpathcurveto{\pgfqpoint{1.288401in}{2.103971in}}{\pgfqpoint{1.296302in}{2.107243in}}{\pgfqpoint{1.302125in}{2.113067in}}%
\pgfpathcurveto{\pgfqpoint{1.307949in}{2.118891in}}{\pgfqpoint{1.311222in}{2.126791in}}{\pgfqpoint{1.311222in}{2.135027in}}%
\pgfpathcurveto{\pgfqpoint{1.311222in}{2.143264in}}{\pgfqpoint{1.307949in}{2.151164in}}{\pgfqpoint{1.302125in}{2.156988in}}%
\pgfpathcurveto{\pgfqpoint{1.296302in}{2.162812in}}{\pgfqpoint{1.288401in}{2.166084in}}{\pgfqpoint{1.280165in}{2.166084in}}%
\pgfpathcurveto{\pgfqpoint{1.271929in}{2.166084in}}{\pgfqpoint{1.264029in}{2.162812in}}{\pgfqpoint{1.258205in}{2.156988in}}%
\pgfpathcurveto{\pgfqpoint{1.252381in}{2.151164in}}{\pgfqpoint{1.249109in}{2.143264in}}{\pgfqpoint{1.249109in}{2.135027in}}%
\pgfpathcurveto{\pgfqpoint{1.249109in}{2.126791in}}{\pgfqpoint{1.252381in}{2.118891in}}{\pgfqpoint{1.258205in}{2.113067in}}%
\pgfpathcurveto{\pgfqpoint{1.264029in}{2.107243in}}{\pgfqpoint{1.271929in}{2.103971in}}{\pgfqpoint{1.280165in}{2.103971in}}%
\pgfpathclose%
\pgfusepath{stroke,fill}%
\end{pgfscope}%
\begin{pgfscope}%
\pgfpathrectangle{\pgfqpoint{0.100000in}{0.212622in}}{\pgfqpoint{3.696000in}{3.696000in}}%
\pgfusepath{clip}%
\pgfsetbuttcap%
\pgfsetroundjoin%
\definecolor{currentfill}{rgb}{0.121569,0.466667,0.705882}%
\pgfsetfillcolor{currentfill}%
\pgfsetfillopacity{0.808618}%
\pgfsetlinewidth{1.003750pt}%
\definecolor{currentstroke}{rgb}{0.121569,0.466667,0.705882}%
\pgfsetstrokecolor{currentstroke}%
\pgfsetstrokeopacity{0.808618}%
\pgfsetdash{}{0pt}%
\pgfpathmoveto{\pgfqpoint{1.280263in}{2.103999in}}%
\pgfpathcurveto{\pgfqpoint{1.288499in}{2.103999in}}{\pgfqpoint{1.296399in}{2.107271in}}{\pgfqpoint{1.302223in}{2.113095in}}%
\pgfpathcurveto{\pgfqpoint{1.308047in}{2.118919in}}{\pgfqpoint{1.311319in}{2.126819in}}{\pgfqpoint{1.311319in}{2.135056in}}%
\pgfpathcurveto{\pgfqpoint{1.311319in}{2.143292in}}{\pgfqpoint{1.308047in}{2.151192in}}{\pgfqpoint{1.302223in}{2.157016in}}%
\pgfpathcurveto{\pgfqpoint{1.296399in}{2.162840in}}{\pgfqpoint{1.288499in}{2.166112in}}{\pgfqpoint{1.280263in}{2.166112in}}%
\pgfpathcurveto{\pgfqpoint{1.272026in}{2.166112in}}{\pgfqpoint{1.264126in}{2.162840in}}{\pgfqpoint{1.258302in}{2.157016in}}%
\pgfpathcurveto{\pgfqpoint{1.252479in}{2.151192in}}{\pgfqpoint{1.249206in}{2.143292in}}{\pgfqpoint{1.249206in}{2.135056in}}%
\pgfpathcurveto{\pgfqpoint{1.249206in}{2.126819in}}{\pgfqpoint{1.252479in}{2.118919in}}{\pgfqpoint{1.258302in}{2.113095in}}%
\pgfpathcurveto{\pgfqpoint{1.264126in}{2.107271in}}{\pgfqpoint{1.272026in}{2.103999in}}{\pgfqpoint{1.280263in}{2.103999in}}%
\pgfpathclose%
\pgfusepath{stroke,fill}%
\end{pgfscope}%
\begin{pgfscope}%
\pgfpathrectangle{\pgfqpoint{0.100000in}{0.212622in}}{\pgfqpoint{3.696000in}{3.696000in}}%
\pgfusepath{clip}%
\pgfsetbuttcap%
\pgfsetroundjoin%
\definecolor{currentfill}{rgb}{0.121569,0.466667,0.705882}%
\pgfsetfillcolor{currentfill}%
\pgfsetfillopacity{0.809162}%
\pgfsetlinewidth{1.003750pt}%
\definecolor{currentstroke}{rgb}{0.121569,0.466667,0.705882}%
\pgfsetstrokecolor{currentstroke}%
\pgfsetstrokeopacity{0.809162}%
\pgfsetdash{}{0pt}%
\pgfpathmoveto{\pgfqpoint{1.280405in}{2.104046in}}%
\pgfpathcurveto{\pgfqpoint{1.288641in}{2.104046in}}{\pgfqpoint{1.296541in}{2.107318in}}{\pgfqpoint{1.302365in}{2.113142in}}%
\pgfpathcurveto{\pgfqpoint{1.308189in}{2.118966in}}{\pgfqpoint{1.311461in}{2.126866in}}{\pgfqpoint{1.311461in}{2.135102in}}%
\pgfpathcurveto{\pgfqpoint{1.311461in}{2.143338in}}{\pgfqpoint{1.308189in}{2.151238in}}{\pgfqpoint{1.302365in}{2.157062in}}%
\pgfpathcurveto{\pgfqpoint{1.296541in}{2.162886in}}{\pgfqpoint{1.288641in}{2.166159in}}{\pgfqpoint{1.280405in}{2.166159in}}%
\pgfpathcurveto{\pgfqpoint{1.272168in}{2.166159in}}{\pgfqpoint{1.264268in}{2.162886in}}{\pgfqpoint{1.258444in}{2.157062in}}%
\pgfpathcurveto{\pgfqpoint{1.252620in}{2.151238in}}{\pgfqpoint{1.249348in}{2.143338in}}{\pgfqpoint{1.249348in}{2.135102in}}%
\pgfpathcurveto{\pgfqpoint{1.249348in}{2.126866in}}{\pgfqpoint{1.252620in}{2.118966in}}{\pgfqpoint{1.258444in}{2.113142in}}%
\pgfpathcurveto{\pgfqpoint{1.264268in}{2.107318in}}{\pgfqpoint{1.272168in}{2.104046in}}{\pgfqpoint{1.280405in}{2.104046in}}%
\pgfpathclose%
\pgfusepath{stroke,fill}%
\end{pgfscope}%
\begin{pgfscope}%
\pgfpathrectangle{\pgfqpoint{0.100000in}{0.212622in}}{\pgfqpoint{3.696000in}{3.696000in}}%
\pgfusepath{clip}%
\pgfsetbuttcap%
\pgfsetroundjoin%
\definecolor{currentfill}{rgb}{0.121569,0.466667,0.705882}%
\pgfsetfillcolor{currentfill}%
\pgfsetfillopacity{0.810161}%
\pgfsetlinewidth{1.003750pt}%
\definecolor{currentstroke}{rgb}{0.121569,0.466667,0.705882}%
\pgfsetstrokecolor{currentstroke}%
\pgfsetstrokeopacity{0.810161}%
\pgfsetdash{}{0pt}%
\pgfpathmoveto{\pgfqpoint{1.280566in}{2.104146in}}%
\pgfpathcurveto{\pgfqpoint{1.288803in}{2.104146in}}{\pgfqpoint{1.296703in}{2.107418in}}{\pgfqpoint{1.302527in}{2.113242in}}%
\pgfpathcurveto{\pgfqpoint{1.308351in}{2.119066in}}{\pgfqpoint{1.311623in}{2.126966in}}{\pgfqpoint{1.311623in}{2.135202in}}%
\pgfpathcurveto{\pgfqpoint{1.311623in}{2.143439in}}{\pgfqpoint{1.308351in}{2.151339in}}{\pgfqpoint{1.302527in}{2.157162in}}%
\pgfpathcurveto{\pgfqpoint{1.296703in}{2.162986in}}{\pgfqpoint{1.288803in}{2.166259in}}{\pgfqpoint{1.280566in}{2.166259in}}%
\pgfpathcurveto{\pgfqpoint{1.272330in}{2.166259in}}{\pgfqpoint{1.264430in}{2.162986in}}{\pgfqpoint{1.258606in}{2.157162in}}%
\pgfpathcurveto{\pgfqpoint{1.252782in}{2.151339in}}{\pgfqpoint{1.249510in}{2.143439in}}{\pgfqpoint{1.249510in}{2.135202in}}%
\pgfpathcurveto{\pgfqpoint{1.249510in}{2.126966in}}{\pgfqpoint{1.252782in}{2.119066in}}{\pgfqpoint{1.258606in}{2.113242in}}%
\pgfpathcurveto{\pgfqpoint{1.264430in}{2.107418in}}{\pgfqpoint{1.272330in}{2.104146in}}{\pgfqpoint{1.280566in}{2.104146in}}%
\pgfpathclose%
\pgfusepath{stroke,fill}%
\end{pgfscope}%
\begin{pgfscope}%
\pgfpathrectangle{\pgfqpoint{0.100000in}{0.212622in}}{\pgfqpoint{3.696000in}{3.696000in}}%
\pgfusepath{clip}%
\pgfsetbuttcap%
\pgfsetroundjoin%
\definecolor{currentfill}{rgb}{0.121569,0.466667,0.705882}%
\pgfsetfillcolor{currentfill}%
\pgfsetfillopacity{0.810162}%
\pgfsetlinewidth{1.003750pt}%
\definecolor{currentstroke}{rgb}{0.121569,0.466667,0.705882}%
\pgfsetstrokecolor{currentstroke}%
\pgfsetstrokeopacity{0.810162}%
\pgfsetdash{}{0pt}%
\pgfpathmoveto{\pgfqpoint{1.280566in}{2.104146in}}%
\pgfpathcurveto{\pgfqpoint{1.288803in}{2.104146in}}{\pgfqpoint{1.296703in}{2.107418in}}{\pgfqpoint{1.302527in}{2.113242in}}%
\pgfpathcurveto{\pgfqpoint{1.308351in}{2.119066in}}{\pgfqpoint{1.311623in}{2.126966in}}{\pgfqpoint{1.311623in}{2.135202in}}%
\pgfpathcurveto{\pgfqpoint{1.311623in}{2.143439in}}{\pgfqpoint{1.308351in}{2.151339in}}{\pgfqpoint{1.302527in}{2.157163in}}%
\pgfpathcurveto{\pgfqpoint{1.296703in}{2.162987in}}{\pgfqpoint{1.288803in}{2.166259in}}{\pgfqpoint{1.280566in}{2.166259in}}%
\pgfpathcurveto{\pgfqpoint{1.272330in}{2.166259in}}{\pgfqpoint{1.264430in}{2.162987in}}{\pgfqpoint{1.258606in}{2.157163in}}%
\pgfpathcurveto{\pgfqpoint{1.252782in}{2.151339in}}{\pgfqpoint{1.249510in}{2.143439in}}{\pgfqpoint{1.249510in}{2.135202in}}%
\pgfpathcurveto{\pgfqpoint{1.249510in}{2.126966in}}{\pgfqpoint{1.252782in}{2.119066in}}{\pgfqpoint{1.258606in}{2.113242in}}%
\pgfpathcurveto{\pgfqpoint{1.264430in}{2.107418in}}{\pgfqpoint{1.272330in}{2.104146in}}{\pgfqpoint{1.280566in}{2.104146in}}%
\pgfpathclose%
\pgfusepath{stroke,fill}%
\end{pgfscope}%
\begin{pgfscope}%
\pgfpathrectangle{\pgfqpoint{0.100000in}{0.212622in}}{\pgfqpoint{3.696000in}{3.696000in}}%
\pgfusepath{clip}%
\pgfsetbuttcap%
\pgfsetroundjoin%
\definecolor{currentfill}{rgb}{0.121569,0.466667,0.705882}%
\pgfsetfillcolor{currentfill}%
\pgfsetfillopacity{0.810163}%
\pgfsetlinewidth{1.003750pt}%
\definecolor{currentstroke}{rgb}{0.121569,0.466667,0.705882}%
\pgfsetstrokecolor{currentstroke}%
\pgfsetstrokeopacity{0.810163}%
\pgfsetdash{}{0pt}%
\pgfpathmoveto{\pgfqpoint{1.280566in}{2.104146in}}%
\pgfpathcurveto{\pgfqpoint{1.288803in}{2.104146in}}{\pgfqpoint{1.296703in}{2.107418in}}{\pgfqpoint{1.302527in}{2.113242in}}%
\pgfpathcurveto{\pgfqpoint{1.308351in}{2.119066in}}{\pgfqpoint{1.311623in}{2.126966in}}{\pgfqpoint{1.311623in}{2.135203in}}%
\pgfpathcurveto{\pgfqpoint{1.311623in}{2.143439in}}{\pgfqpoint{1.308351in}{2.151339in}}{\pgfqpoint{1.302527in}{2.157163in}}%
\pgfpathcurveto{\pgfqpoint{1.296703in}{2.162987in}}{\pgfqpoint{1.288803in}{2.166259in}}{\pgfqpoint{1.280566in}{2.166259in}}%
\pgfpathcurveto{\pgfqpoint{1.272330in}{2.166259in}}{\pgfqpoint{1.264430in}{2.162987in}}{\pgfqpoint{1.258606in}{2.157163in}}%
\pgfpathcurveto{\pgfqpoint{1.252782in}{2.151339in}}{\pgfqpoint{1.249510in}{2.143439in}}{\pgfqpoint{1.249510in}{2.135203in}}%
\pgfpathcurveto{\pgfqpoint{1.249510in}{2.126966in}}{\pgfqpoint{1.252782in}{2.119066in}}{\pgfqpoint{1.258606in}{2.113242in}}%
\pgfpathcurveto{\pgfqpoint{1.264430in}{2.107418in}}{\pgfqpoint{1.272330in}{2.104146in}}{\pgfqpoint{1.280566in}{2.104146in}}%
\pgfpathclose%
\pgfusepath{stroke,fill}%
\end{pgfscope}%
\begin{pgfscope}%
\pgfpathrectangle{\pgfqpoint{0.100000in}{0.212622in}}{\pgfqpoint{3.696000in}{3.696000in}}%
\pgfusepath{clip}%
\pgfsetbuttcap%
\pgfsetroundjoin%
\definecolor{currentfill}{rgb}{0.121569,0.466667,0.705882}%
\pgfsetfillcolor{currentfill}%
\pgfsetfillopacity{0.810167}%
\pgfsetlinewidth{1.003750pt}%
\definecolor{currentstroke}{rgb}{0.121569,0.466667,0.705882}%
\pgfsetstrokecolor{currentstroke}%
\pgfsetstrokeopacity{0.810167}%
\pgfsetdash{}{0pt}%
\pgfpathmoveto{\pgfqpoint{1.280566in}{2.104146in}}%
\pgfpathcurveto{\pgfqpoint{1.288803in}{2.104146in}}{\pgfqpoint{1.296703in}{2.107419in}}{\pgfqpoint{1.302527in}{2.113243in}}%
\pgfpathcurveto{\pgfqpoint{1.308351in}{2.119067in}}{\pgfqpoint{1.311623in}{2.126967in}}{\pgfqpoint{1.311623in}{2.135203in}}%
\pgfpathcurveto{\pgfqpoint{1.311623in}{2.143439in}}{\pgfqpoint{1.308351in}{2.151339in}}{\pgfqpoint{1.302527in}{2.157163in}}%
\pgfpathcurveto{\pgfqpoint{1.296703in}{2.162987in}}{\pgfqpoint{1.288803in}{2.166259in}}{\pgfqpoint{1.280566in}{2.166259in}}%
\pgfpathcurveto{\pgfqpoint{1.272330in}{2.166259in}}{\pgfqpoint{1.264430in}{2.162987in}}{\pgfqpoint{1.258606in}{2.157163in}}%
\pgfpathcurveto{\pgfqpoint{1.252782in}{2.151339in}}{\pgfqpoint{1.249510in}{2.143439in}}{\pgfqpoint{1.249510in}{2.135203in}}%
\pgfpathcurveto{\pgfqpoint{1.249510in}{2.126967in}}{\pgfqpoint{1.252782in}{2.119067in}}{\pgfqpoint{1.258606in}{2.113243in}}%
\pgfpathcurveto{\pgfqpoint{1.264430in}{2.107419in}}{\pgfqpoint{1.272330in}{2.104146in}}{\pgfqpoint{1.280566in}{2.104146in}}%
\pgfpathclose%
\pgfusepath{stroke,fill}%
\end{pgfscope}%
\begin{pgfscope}%
\pgfpathrectangle{\pgfqpoint{0.100000in}{0.212622in}}{\pgfqpoint{3.696000in}{3.696000in}}%
\pgfusepath{clip}%
\pgfsetbuttcap%
\pgfsetroundjoin%
\definecolor{currentfill}{rgb}{0.121569,0.466667,0.705882}%
\pgfsetfillcolor{currentfill}%
\pgfsetfillopacity{0.810172}%
\pgfsetlinewidth{1.003750pt}%
\definecolor{currentstroke}{rgb}{0.121569,0.466667,0.705882}%
\pgfsetstrokecolor{currentstroke}%
\pgfsetstrokeopacity{0.810172}%
\pgfsetdash{}{0pt}%
\pgfpathmoveto{\pgfqpoint{1.280566in}{2.104147in}}%
\pgfpathcurveto{\pgfqpoint{1.288802in}{2.104147in}}{\pgfqpoint{1.296702in}{2.107419in}}{\pgfqpoint{1.302526in}{2.113243in}}%
\pgfpathcurveto{\pgfqpoint{1.308350in}{2.119067in}}{\pgfqpoint{1.311622in}{2.126967in}}{\pgfqpoint{1.311622in}{2.135203in}}%
\pgfpathcurveto{\pgfqpoint{1.311622in}{2.143440in}}{\pgfqpoint{1.308350in}{2.151340in}}{\pgfqpoint{1.302526in}{2.157164in}}%
\pgfpathcurveto{\pgfqpoint{1.296702in}{2.162988in}}{\pgfqpoint{1.288802in}{2.166260in}}{\pgfqpoint{1.280566in}{2.166260in}}%
\pgfpathcurveto{\pgfqpoint{1.272329in}{2.166260in}}{\pgfqpoint{1.264429in}{2.162988in}}{\pgfqpoint{1.258605in}{2.157164in}}%
\pgfpathcurveto{\pgfqpoint{1.252781in}{2.151340in}}{\pgfqpoint{1.249509in}{2.143440in}}{\pgfqpoint{1.249509in}{2.135203in}}%
\pgfpathcurveto{\pgfqpoint{1.249509in}{2.126967in}}{\pgfqpoint{1.252781in}{2.119067in}}{\pgfqpoint{1.258605in}{2.113243in}}%
\pgfpathcurveto{\pgfqpoint{1.264429in}{2.107419in}}{\pgfqpoint{1.272329in}{2.104147in}}{\pgfqpoint{1.280566in}{2.104147in}}%
\pgfpathclose%
\pgfusepath{stroke,fill}%
\end{pgfscope}%
\begin{pgfscope}%
\pgfpathrectangle{\pgfqpoint{0.100000in}{0.212622in}}{\pgfqpoint{3.696000in}{3.696000in}}%
\pgfusepath{clip}%
\pgfsetbuttcap%
\pgfsetroundjoin%
\definecolor{currentfill}{rgb}{0.121569,0.466667,0.705882}%
\pgfsetfillcolor{currentfill}%
\pgfsetfillopacity{0.810183}%
\pgfsetlinewidth{1.003750pt}%
\definecolor{currentstroke}{rgb}{0.121569,0.466667,0.705882}%
\pgfsetstrokecolor{currentstroke}%
\pgfsetstrokeopacity{0.810183}%
\pgfsetdash{}{0pt}%
\pgfpathmoveto{\pgfqpoint{1.280564in}{2.104148in}}%
\pgfpathcurveto{\pgfqpoint{1.288800in}{2.104148in}}{\pgfqpoint{1.296700in}{2.107421in}}{\pgfqpoint{1.302524in}{2.113244in}}%
\pgfpathcurveto{\pgfqpoint{1.308348in}{2.119068in}}{\pgfqpoint{1.311620in}{2.126968in}}{\pgfqpoint{1.311620in}{2.135205in}}%
\pgfpathcurveto{\pgfqpoint{1.311620in}{2.143441in}}{\pgfqpoint{1.308348in}{2.151341in}}{\pgfqpoint{1.302524in}{2.157165in}}%
\pgfpathcurveto{\pgfqpoint{1.296700in}{2.162989in}}{\pgfqpoint{1.288800in}{2.166261in}}{\pgfqpoint{1.280564in}{2.166261in}}%
\pgfpathcurveto{\pgfqpoint{1.272327in}{2.166261in}}{\pgfqpoint{1.264427in}{2.162989in}}{\pgfqpoint{1.258603in}{2.157165in}}%
\pgfpathcurveto{\pgfqpoint{1.252779in}{2.151341in}}{\pgfqpoint{1.249507in}{2.143441in}}{\pgfqpoint{1.249507in}{2.135205in}}%
\pgfpathcurveto{\pgfqpoint{1.249507in}{2.126968in}}{\pgfqpoint{1.252779in}{2.119068in}}{\pgfqpoint{1.258603in}{2.113244in}}%
\pgfpathcurveto{\pgfqpoint{1.264427in}{2.107421in}}{\pgfqpoint{1.272327in}{2.104148in}}{\pgfqpoint{1.280564in}{2.104148in}}%
\pgfpathclose%
\pgfusepath{stroke,fill}%
\end{pgfscope}%
\begin{pgfscope}%
\pgfpathrectangle{\pgfqpoint{0.100000in}{0.212622in}}{\pgfqpoint{3.696000in}{3.696000in}}%
\pgfusepath{clip}%
\pgfsetbuttcap%
\pgfsetroundjoin%
\definecolor{currentfill}{rgb}{0.121569,0.466667,0.705882}%
\pgfsetfillcolor{currentfill}%
\pgfsetfillopacity{0.810202}%
\pgfsetlinewidth{1.003750pt}%
\definecolor{currentstroke}{rgb}{0.121569,0.466667,0.705882}%
\pgfsetstrokecolor{currentstroke}%
\pgfsetstrokeopacity{0.810202}%
\pgfsetdash{}{0pt}%
\pgfpathmoveto{\pgfqpoint{1.280559in}{2.104151in}}%
\pgfpathcurveto{\pgfqpoint{1.288795in}{2.104151in}}{\pgfqpoint{1.296695in}{2.107423in}}{\pgfqpoint{1.302519in}{2.113247in}}%
\pgfpathcurveto{\pgfqpoint{1.308343in}{2.119071in}}{\pgfqpoint{1.311615in}{2.126971in}}{\pgfqpoint{1.311615in}{2.135207in}}%
\pgfpathcurveto{\pgfqpoint{1.311615in}{2.143443in}}{\pgfqpoint{1.308343in}{2.151344in}}{\pgfqpoint{1.302519in}{2.157167in}}%
\pgfpathcurveto{\pgfqpoint{1.296695in}{2.162991in}}{\pgfqpoint{1.288795in}{2.166264in}}{\pgfqpoint{1.280559in}{2.166264in}}%
\pgfpathcurveto{\pgfqpoint{1.272323in}{2.166264in}}{\pgfqpoint{1.264423in}{2.162991in}}{\pgfqpoint{1.258599in}{2.157167in}}%
\pgfpathcurveto{\pgfqpoint{1.252775in}{2.151344in}}{\pgfqpoint{1.249502in}{2.143443in}}{\pgfqpoint{1.249502in}{2.135207in}}%
\pgfpathcurveto{\pgfqpoint{1.249502in}{2.126971in}}{\pgfqpoint{1.252775in}{2.119071in}}{\pgfqpoint{1.258599in}{2.113247in}}%
\pgfpathcurveto{\pgfqpoint{1.264423in}{2.107423in}}{\pgfqpoint{1.272323in}{2.104151in}}{\pgfqpoint{1.280559in}{2.104151in}}%
\pgfpathclose%
\pgfusepath{stroke,fill}%
\end{pgfscope}%
\begin{pgfscope}%
\pgfpathrectangle{\pgfqpoint{0.100000in}{0.212622in}}{\pgfqpoint{3.696000in}{3.696000in}}%
\pgfusepath{clip}%
\pgfsetbuttcap%
\pgfsetroundjoin%
\definecolor{currentfill}{rgb}{0.121569,0.466667,0.705882}%
\pgfsetfillcolor{currentfill}%
\pgfsetfillopacity{0.810238}%
\pgfsetlinewidth{1.003750pt}%
\definecolor{currentstroke}{rgb}{0.121569,0.466667,0.705882}%
\pgfsetstrokecolor{currentstroke}%
\pgfsetstrokeopacity{0.810238}%
\pgfsetdash{}{0pt}%
\pgfpathmoveto{\pgfqpoint{1.280548in}{2.104156in}}%
\pgfpathcurveto{\pgfqpoint{1.288784in}{2.104156in}}{\pgfqpoint{1.296685in}{2.107428in}}{\pgfqpoint{1.302508in}{2.113252in}}%
\pgfpathcurveto{\pgfqpoint{1.308332in}{2.119076in}}{\pgfqpoint{1.311605in}{2.126976in}}{\pgfqpoint{1.311605in}{2.135212in}}%
\pgfpathcurveto{\pgfqpoint{1.311605in}{2.143449in}}{\pgfqpoint{1.308332in}{2.151349in}}{\pgfqpoint{1.302508in}{2.157173in}}%
\pgfpathcurveto{\pgfqpoint{1.296685in}{2.162997in}}{\pgfqpoint{1.288784in}{2.166269in}}{\pgfqpoint{1.280548in}{2.166269in}}%
\pgfpathcurveto{\pgfqpoint{1.272312in}{2.166269in}}{\pgfqpoint{1.264412in}{2.162997in}}{\pgfqpoint{1.258588in}{2.157173in}}%
\pgfpathcurveto{\pgfqpoint{1.252764in}{2.151349in}}{\pgfqpoint{1.249492in}{2.143449in}}{\pgfqpoint{1.249492in}{2.135212in}}%
\pgfpathcurveto{\pgfqpoint{1.249492in}{2.126976in}}{\pgfqpoint{1.252764in}{2.119076in}}{\pgfqpoint{1.258588in}{2.113252in}}%
\pgfpathcurveto{\pgfqpoint{1.264412in}{2.107428in}}{\pgfqpoint{1.272312in}{2.104156in}}{\pgfqpoint{1.280548in}{2.104156in}}%
\pgfpathclose%
\pgfusepath{stroke,fill}%
\end{pgfscope}%
\begin{pgfscope}%
\pgfpathrectangle{\pgfqpoint{0.100000in}{0.212622in}}{\pgfqpoint{3.696000in}{3.696000in}}%
\pgfusepath{clip}%
\pgfsetbuttcap%
\pgfsetroundjoin%
\definecolor{currentfill}{rgb}{0.121569,0.466667,0.705882}%
\pgfsetfillcolor{currentfill}%
\pgfsetfillopacity{0.810302}%
\pgfsetlinewidth{1.003750pt}%
\definecolor{currentstroke}{rgb}{0.121569,0.466667,0.705882}%
\pgfsetstrokecolor{currentstroke}%
\pgfsetstrokeopacity{0.810302}%
\pgfsetdash{}{0pt}%
\pgfpathmoveto{\pgfqpoint{1.280525in}{2.104168in}}%
\pgfpathcurveto{\pgfqpoint{1.288762in}{2.104168in}}{\pgfqpoint{1.296662in}{2.107440in}}{\pgfqpoint{1.302486in}{2.113264in}}%
\pgfpathcurveto{\pgfqpoint{1.308310in}{2.119088in}}{\pgfqpoint{1.311582in}{2.126988in}}{\pgfqpoint{1.311582in}{2.135224in}}%
\pgfpathcurveto{\pgfqpoint{1.311582in}{2.143460in}}{\pgfqpoint{1.308310in}{2.151360in}}{\pgfqpoint{1.302486in}{2.157184in}}%
\pgfpathcurveto{\pgfqpoint{1.296662in}{2.163008in}}{\pgfqpoint{1.288762in}{2.166281in}}{\pgfqpoint{1.280525in}{2.166281in}}%
\pgfpathcurveto{\pgfqpoint{1.272289in}{2.166281in}}{\pgfqpoint{1.264389in}{2.163008in}}{\pgfqpoint{1.258565in}{2.157184in}}%
\pgfpathcurveto{\pgfqpoint{1.252741in}{2.151360in}}{\pgfqpoint{1.249469in}{2.143460in}}{\pgfqpoint{1.249469in}{2.135224in}}%
\pgfpathcurveto{\pgfqpoint{1.249469in}{2.126988in}}{\pgfqpoint{1.252741in}{2.119088in}}{\pgfqpoint{1.258565in}{2.113264in}}%
\pgfpathcurveto{\pgfqpoint{1.264389in}{2.107440in}}{\pgfqpoint{1.272289in}{2.104168in}}{\pgfqpoint{1.280525in}{2.104168in}}%
\pgfpathclose%
\pgfusepath{stroke,fill}%
\end{pgfscope}%
\begin{pgfscope}%
\pgfpathrectangle{\pgfqpoint{0.100000in}{0.212622in}}{\pgfqpoint{3.696000in}{3.696000in}}%
\pgfusepath{clip}%
\pgfsetbuttcap%
\pgfsetroundjoin%
\definecolor{currentfill}{rgb}{0.121569,0.466667,0.705882}%
\pgfsetfillcolor{currentfill}%
\pgfsetfillopacity{0.810418}%
\pgfsetlinewidth{1.003750pt}%
\definecolor{currentstroke}{rgb}{0.121569,0.466667,0.705882}%
\pgfsetstrokecolor{currentstroke}%
\pgfsetstrokeopacity{0.810418}%
\pgfsetdash{}{0pt}%
\pgfpathmoveto{\pgfqpoint{1.280479in}{2.104192in}}%
\pgfpathcurveto{\pgfqpoint{1.288716in}{2.104192in}}{\pgfqpoint{1.296616in}{2.107464in}}{\pgfqpoint{1.302440in}{2.113288in}}%
\pgfpathcurveto{\pgfqpoint{1.308263in}{2.119112in}}{\pgfqpoint{1.311536in}{2.127012in}}{\pgfqpoint{1.311536in}{2.135248in}}%
\pgfpathcurveto{\pgfqpoint{1.311536in}{2.143484in}}{\pgfqpoint{1.308263in}{2.151384in}}{\pgfqpoint{1.302440in}{2.157208in}}%
\pgfpathcurveto{\pgfqpoint{1.296616in}{2.163032in}}{\pgfqpoint{1.288716in}{2.166305in}}{\pgfqpoint{1.280479in}{2.166305in}}%
\pgfpathcurveto{\pgfqpoint{1.272243in}{2.166305in}}{\pgfqpoint{1.264343in}{2.163032in}}{\pgfqpoint{1.258519in}{2.157208in}}%
\pgfpathcurveto{\pgfqpoint{1.252695in}{2.151384in}}{\pgfqpoint{1.249423in}{2.143484in}}{\pgfqpoint{1.249423in}{2.135248in}}%
\pgfpathcurveto{\pgfqpoint{1.249423in}{2.127012in}}{\pgfqpoint{1.252695in}{2.119112in}}{\pgfqpoint{1.258519in}{2.113288in}}%
\pgfpathcurveto{\pgfqpoint{1.264343in}{2.107464in}}{\pgfqpoint{1.272243in}{2.104192in}}{\pgfqpoint{1.280479in}{2.104192in}}%
\pgfpathclose%
\pgfusepath{stroke,fill}%
\end{pgfscope}%
\begin{pgfscope}%
\pgfpathrectangle{\pgfqpoint{0.100000in}{0.212622in}}{\pgfqpoint{3.696000in}{3.696000in}}%
\pgfusepath{clip}%
\pgfsetbuttcap%
\pgfsetroundjoin%
\definecolor{currentfill}{rgb}{0.121569,0.466667,0.705882}%
\pgfsetfillcolor{currentfill}%
\pgfsetfillopacity{0.810632}%
\pgfsetlinewidth{1.003750pt}%
\definecolor{currentstroke}{rgb}{0.121569,0.466667,0.705882}%
\pgfsetstrokecolor{currentstroke}%
\pgfsetstrokeopacity{0.810632}%
\pgfsetdash{}{0pt}%
\pgfpathmoveto{\pgfqpoint{1.280387in}{2.104245in}}%
\pgfpathcurveto{\pgfqpoint{1.288623in}{2.104245in}}{\pgfqpoint{1.296523in}{2.107517in}}{\pgfqpoint{1.302347in}{2.113341in}}%
\pgfpathcurveto{\pgfqpoint{1.308171in}{2.119165in}}{\pgfqpoint{1.311443in}{2.127065in}}{\pgfqpoint{1.311443in}{2.135301in}}%
\pgfpathcurveto{\pgfqpoint{1.311443in}{2.143537in}}{\pgfqpoint{1.308171in}{2.151437in}}{\pgfqpoint{1.302347in}{2.157261in}}%
\pgfpathcurveto{\pgfqpoint{1.296523in}{2.163085in}}{\pgfqpoint{1.288623in}{2.166358in}}{\pgfqpoint{1.280387in}{2.166358in}}%
\pgfpathcurveto{\pgfqpoint{1.272150in}{2.166358in}}{\pgfqpoint{1.264250in}{2.163085in}}{\pgfqpoint{1.258426in}{2.157261in}}%
\pgfpathcurveto{\pgfqpoint{1.252602in}{2.151437in}}{\pgfqpoint{1.249330in}{2.143537in}}{\pgfqpoint{1.249330in}{2.135301in}}%
\pgfpathcurveto{\pgfqpoint{1.249330in}{2.127065in}}{\pgfqpoint{1.252602in}{2.119165in}}{\pgfqpoint{1.258426in}{2.113341in}}%
\pgfpathcurveto{\pgfqpoint{1.264250in}{2.107517in}}{\pgfqpoint{1.272150in}{2.104245in}}{\pgfqpoint{1.280387in}{2.104245in}}%
\pgfpathclose%
\pgfusepath{stroke,fill}%
\end{pgfscope}%
\begin{pgfscope}%
\pgfpathrectangle{\pgfqpoint{0.100000in}{0.212622in}}{\pgfqpoint{3.696000in}{3.696000in}}%
\pgfusepath{clip}%
\pgfsetbuttcap%
\pgfsetroundjoin%
\definecolor{currentfill}{rgb}{0.121569,0.466667,0.705882}%
\pgfsetfillcolor{currentfill}%
\pgfsetfillopacity{0.810953}%
\pgfsetlinewidth{1.003750pt}%
\definecolor{currentstroke}{rgb}{0.121569,0.466667,0.705882}%
\pgfsetstrokecolor{currentstroke}%
\pgfsetstrokeopacity{0.810953}%
\pgfsetdash{}{0pt}%
\pgfpathmoveto{\pgfqpoint{1.376076in}{1.028592in}}%
\pgfpathcurveto{\pgfqpoint{1.384312in}{1.028592in}}{\pgfqpoint{1.392212in}{1.031864in}}{\pgfqpoint{1.398036in}{1.037688in}}%
\pgfpathcurveto{\pgfqpoint{1.403860in}{1.043512in}}{\pgfqpoint{1.407132in}{1.051412in}}{\pgfqpoint{1.407132in}{1.059648in}}%
\pgfpathcurveto{\pgfqpoint{1.407132in}{1.067884in}}{\pgfqpoint{1.403860in}{1.075784in}}{\pgfqpoint{1.398036in}{1.081608in}}%
\pgfpathcurveto{\pgfqpoint{1.392212in}{1.087432in}}{\pgfqpoint{1.384312in}{1.090705in}}{\pgfqpoint{1.376076in}{1.090705in}}%
\pgfpathcurveto{\pgfqpoint{1.367840in}{1.090705in}}{\pgfqpoint{1.359940in}{1.087432in}}{\pgfqpoint{1.354116in}{1.081608in}}%
\pgfpathcurveto{\pgfqpoint{1.348292in}{1.075784in}}{\pgfqpoint{1.345019in}{1.067884in}}{\pgfqpoint{1.345019in}{1.059648in}}%
\pgfpathcurveto{\pgfqpoint{1.345019in}{1.051412in}}{\pgfqpoint{1.348292in}{1.043512in}}{\pgfqpoint{1.354116in}{1.037688in}}%
\pgfpathcurveto{\pgfqpoint{1.359940in}{1.031864in}}{\pgfqpoint{1.367840in}{1.028592in}}{\pgfqpoint{1.376076in}{1.028592in}}%
\pgfpathclose%
\pgfusepath{stroke,fill}%
\end{pgfscope}%
\begin{pgfscope}%
\pgfpathrectangle{\pgfqpoint{0.100000in}{0.212622in}}{\pgfqpoint{3.696000in}{3.696000in}}%
\pgfusepath{clip}%
\pgfsetbuttcap%
\pgfsetroundjoin%
\definecolor{currentfill}{rgb}{0.121569,0.466667,0.705882}%
\pgfsetfillcolor{currentfill}%
\pgfsetfillopacity{0.811023}%
\pgfsetlinewidth{1.003750pt}%
\definecolor{currentstroke}{rgb}{0.121569,0.466667,0.705882}%
\pgfsetstrokecolor{currentstroke}%
\pgfsetstrokeopacity{0.811023}%
\pgfsetdash{}{0pt}%
\pgfpathmoveto{\pgfqpoint{1.280190in}{2.104364in}}%
\pgfpathcurveto{\pgfqpoint{1.288426in}{2.104364in}}{\pgfqpoint{1.296326in}{2.107636in}}{\pgfqpoint{1.302150in}{2.113460in}}%
\pgfpathcurveto{\pgfqpoint{1.307974in}{2.119284in}}{\pgfqpoint{1.311247in}{2.127184in}}{\pgfqpoint{1.311247in}{2.135421in}}%
\pgfpathcurveto{\pgfqpoint{1.311247in}{2.143657in}}{\pgfqpoint{1.307974in}{2.151557in}}{\pgfqpoint{1.302150in}{2.157381in}}%
\pgfpathcurveto{\pgfqpoint{1.296326in}{2.163205in}}{\pgfqpoint{1.288426in}{2.166477in}}{\pgfqpoint{1.280190in}{2.166477in}}%
\pgfpathcurveto{\pgfqpoint{1.271954in}{2.166477in}}{\pgfqpoint{1.264054in}{2.163205in}}{\pgfqpoint{1.258230in}{2.157381in}}%
\pgfpathcurveto{\pgfqpoint{1.252406in}{2.151557in}}{\pgfqpoint{1.249134in}{2.143657in}}{\pgfqpoint{1.249134in}{2.135421in}}%
\pgfpathcurveto{\pgfqpoint{1.249134in}{2.127184in}}{\pgfqpoint{1.252406in}{2.119284in}}{\pgfqpoint{1.258230in}{2.113460in}}%
\pgfpathcurveto{\pgfqpoint{1.264054in}{2.107636in}}{\pgfqpoint{1.271954in}{2.104364in}}{\pgfqpoint{1.280190in}{2.104364in}}%
\pgfpathclose%
\pgfusepath{stroke,fill}%
\end{pgfscope}%
\begin{pgfscope}%
\pgfpathrectangle{\pgfqpoint{0.100000in}{0.212622in}}{\pgfqpoint{3.696000in}{3.696000in}}%
\pgfusepath{clip}%
\pgfsetbuttcap%
\pgfsetroundjoin%
\definecolor{currentfill}{rgb}{0.121569,0.466667,0.705882}%
\pgfsetfillcolor{currentfill}%
\pgfsetfillopacity{0.811744}%
\pgfsetlinewidth{1.003750pt}%
\definecolor{currentstroke}{rgb}{0.121569,0.466667,0.705882}%
\pgfsetstrokecolor{currentstroke}%
\pgfsetstrokeopacity{0.811744}%
\pgfsetdash{}{0pt}%
\pgfpathmoveto{\pgfqpoint{1.279810in}{2.104631in}}%
\pgfpathcurveto{\pgfqpoint{1.288046in}{2.104631in}}{\pgfqpoint{1.295946in}{2.107903in}}{\pgfqpoint{1.301770in}{2.113727in}}%
\pgfpathcurveto{\pgfqpoint{1.307594in}{2.119551in}}{\pgfqpoint{1.310866in}{2.127451in}}{\pgfqpoint{1.310866in}{2.135688in}}%
\pgfpathcurveto{\pgfqpoint{1.310866in}{2.143924in}}{\pgfqpoint{1.307594in}{2.151824in}}{\pgfqpoint{1.301770in}{2.157648in}}%
\pgfpathcurveto{\pgfqpoint{1.295946in}{2.163472in}}{\pgfqpoint{1.288046in}{2.166744in}}{\pgfqpoint{1.279810in}{2.166744in}}%
\pgfpathcurveto{\pgfqpoint{1.271574in}{2.166744in}}{\pgfqpoint{1.263674in}{2.163472in}}{\pgfqpoint{1.257850in}{2.157648in}}%
\pgfpathcurveto{\pgfqpoint{1.252026in}{2.151824in}}{\pgfqpoint{1.248753in}{2.143924in}}{\pgfqpoint{1.248753in}{2.135688in}}%
\pgfpathcurveto{\pgfqpoint{1.248753in}{2.127451in}}{\pgfqpoint{1.252026in}{2.119551in}}{\pgfqpoint{1.257850in}{2.113727in}}%
\pgfpathcurveto{\pgfqpoint{1.263674in}{2.107903in}}{\pgfqpoint{1.271574in}{2.104631in}}{\pgfqpoint{1.279810in}{2.104631in}}%
\pgfpathclose%
\pgfusepath{stroke,fill}%
\end{pgfscope}%
\begin{pgfscope}%
\pgfpathrectangle{\pgfqpoint{0.100000in}{0.212622in}}{\pgfqpoint{3.696000in}{3.696000in}}%
\pgfusepath{clip}%
\pgfsetbuttcap%
\pgfsetroundjoin%
\definecolor{currentfill}{rgb}{0.121569,0.466667,0.705882}%
\pgfsetfillcolor{currentfill}%
\pgfsetfillopacity{0.813085}%
\pgfsetlinewidth{1.003750pt}%
\definecolor{currentstroke}{rgb}{0.121569,0.466667,0.705882}%
\pgfsetstrokecolor{currentstroke}%
\pgfsetstrokeopacity{0.813085}%
\pgfsetdash{}{0pt}%
\pgfpathmoveto{\pgfqpoint{1.279075in}{2.105260in}}%
\pgfpathcurveto{\pgfqpoint{1.287311in}{2.105260in}}{\pgfqpoint{1.295211in}{2.108532in}}{\pgfqpoint{1.301035in}{2.114356in}}%
\pgfpathcurveto{\pgfqpoint{1.306859in}{2.120180in}}{\pgfqpoint{1.310131in}{2.128080in}}{\pgfqpoint{1.310131in}{2.136316in}}%
\pgfpathcurveto{\pgfqpoint{1.310131in}{2.144553in}}{\pgfqpoint{1.306859in}{2.152453in}}{\pgfqpoint{1.301035in}{2.158277in}}%
\pgfpathcurveto{\pgfqpoint{1.295211in}{2.164100in}}{\pgfqpoint{1.287311in}{2.167373in}}{\pgfqpoint{1.279075in}{2.167373in}}%
\pgfpathcurveto{\pgfqpoint{1.270839in}{2.167373in}}{\pgfqpoint{1.262939in}{2.164100in}}{\pgfqpoint{1.257115in}{2.158277in}}%
\pgfpathcurveto{\pgfqpoint{1.251291in}{2.152453in}}{\pgfqpoint{1.248018in}{2.144553in}}{\pgfqpoint{1.248018in}{2.136316in}}%
\pgfpathcurveto{\pgfqpoint{1.248018in}{2.128080in}}{\pgfqpoint{1.251291in}{2.120180in}}{\pgfqpoint{1.257115in}{2.114356in}}%
\pgfpathcurveto{\pgfqpoint{1.262939in}{2.108532in}}{\pgfqpoint{1.270839in}{2.105260in}}{\pgfqpoint{1.279075in}{2.105260in}}%
\pgfpathclose%
\pgfusepath{stroke,fill}%
\end{pgfscope}%
\begin{pgfscope}%
\pgfpathrectangle{\pgfqpoint{0.100000in}{0.212622in}}{\pgfqpoint{3.696000in}{3.696000in}}%
\pgfusepath{clip}%
\pgfsetbuttcap%
\pgfsetroundjoin%
\definecolor{currentfill}{rgb}{0.121569,0.466667,0.705882}%
\pgfsetfillcolor{currentfill}%
\pgfsetfillopacity{0.815601}%
\pgfsetlinewidth{1.003750pt}%
\definecolor{currentstroke}{rgb}{0.121569,0.466667,0.705882}%
\pgfsetstrokecolor{currentstroke}%
\pgfsetstrokeopacity{0.815601}%
\pgfsetdash{}{0pt}%
\pgfpathmoveto{\pgfqpoint{1.277726in}{2.106694in}}%
\pgfpathcurveto{\pgfqpoint{1.285963in}{2.106694in}}{\pgfqpoint{1.293863in}{2.109966in}}{\pgfqpoint{1.299687in}{2.115790in}}%
\pgfpathcurveto{\pgfqpoint{1.305511in}{2.121614in}}{\pgfqpoint{1.308783in}{2.129514in}}{\pgfqpoint{1.308783in}{2.137750in}}%
\pgfpathcurveto{\pgfqpoint{1.308783in}{2.145987in}}{\pgfqpoint{1.305511in}{2.153887in}}{\pgfqpoint{1.299687in}{2.159711in}}%
\pgfpathcurveto{\pgfqpoint{1.293863in}{2.165535in}}{\pgfqpoint{1.285963in}{2.168807in}}{\pgfqpoint{1.277726in}{2.168807in}}%
\pgfpathcurveto{\pgfqpoint{1.269490in}{2.168807in}}{\pgfqpoint{1.261590in}{2.165535in}}{\pgfqpoint{1.255766in}{2.159711in}}%
\pgfpathcurveto{\pgfqpoint{1.249942in}{2.153887in}}{\pgfqpoint{1.246670in}{2.145987in}}{\pgfqpoint{1.246670in}{2.137750in}}%
\pgfpathcurveto{\pgfqpoint{1.246670in}{2.129514in}}{\pgfqpoint{1.249942in}{2.121614in}}{\pgfqpoint{1.255766in}{2.115790in}}%
\pgfpathcurveto{\pgfqpoint{1.261590in}{2.109966in}}{\pgfqpoint{1.269490in}{2.106694in}}{\pgfqpoint{1.277726in}{2.106694in}}%
\pgfpathclose%
\pgfusepath{stroke,fill}%
\end{pgfscope}%
\begin{pgfscope}%
\pgfpathrectangle{\pgfqpoint{0.100000in}{0.212622in}}{\pgfqpoint{3.696000in}{3.696000in}}%
\pgfusepath{clip}%
\pgfsetbuttcap%
\pgfsetroundjoin%
\definecolor{currentfill}{rgb}{0.121569,0.466667,0.705882}%
\pgfsetfillcolor{currentfill}%
\pgfsetfillopacity{0.815844}%
\pgfsetlinewidth{1.003750pt}%
\definecolor{currentstroke}{rgb}{0.121569,0.466667,0.705882}%
\pgfsetstrokecolor{currentstroke}%
\pgfsetstrokeopacity{0.815844}%
\pgfsetdash{}{0pt}%
\pgfpathmoveto{\pgfqpoint{2.520082in}{1.531251in}}%
\pgfpathcurveto{\pgfqpoint{2.528318in}{1.531251in}}{\pgfqpoint{2.536219in}{1.534523in}}{\pgfqpoint{2.542042in}{1.540347in}}%
\pgfpathcurveto{\pgfqpoint{2.547866in}{1.546171in}}{\pgfqpoint{2.551139in}{1.554071in}}{\pgfqpoint{2.551139in}{1.562308in}}%
\pgfpathcurveto{\pgfqpoint{2.551139in}{1.570544in}}{\pgfqpoint{2.547866in}{1.578444in}}{\pgfqpoint{2.542042in}{1.584268in}}%
\pgfpathcurveto{\pgfqpoint{2.536219in}{1.590092in}}{\pgfqpoint{2.528318in}{1.593364in}}{\pgfqpoint{2.520082in}{1.593364in}}%
\pgfpathcurveto{\pgfqpoint{2.511846in}{1.593364in}}{\pgfqpoint{2.503946in}{1.590092in}}{\pgfqpoint{2.498122in}{1.584268in}}%
\pgfpathcurveto{\pgfqpoint{2.492298in}{1.578444in}}{\pgfqpoint{2.489026in}{1.570544in}}{\pgfqpoint{2.489026in}{1.562308in}}%
\pgfpathcurveto{\pgfqpoint{2.489026in}{1.554071in}}{\pgfqpoint{2.492298in}{1.546171in}}{\pgfqpoint{2.498122in}{1.540347in}}%
\pgfpathcurveto{\pgfqpoint{2.503946in}{1.534523in}}{\pgfqpoint{2.511846in}{1.531251in}}{\pgfqpoint{2.520082in}{1.531251in}}%
\pgfpathclose%
\pgfusepath{stroke,fill}%
\end{pgfscope}%
\begin{pgfscope}%
\pgfpathrectangle{\pgfqpoint{0.100000in}{0.212622in}}{\pgfqpoint{3.696000in}{3.696000in}}%
\pgfusepath{clip}%
\pgfsetbuttcap%
\pgfsetroundjoin%
\definecolor{currentfill}{rgb}{0.121569,0.466667,0.705882}%
\pgfsetfillcolor{currentfill}%
\pgfsetfillopacity{0.817149}%
\pgfsetlinewidth{1.003750pt}%
\definecolor{currentstroke}{rgb}{0.121569,0.466667,0.705882}%
\pgfsetstrokecolor{currentstroke}%
\pgfsetstrokeopacity{0.817149}%
\pgfsetdash{}{0pt}%
\pgfpathmoveto{\pgfqpoint{1.404054in}{1.023900in}}%
\pgfpathcurveto{\pgfqpoint{1.412290in}{1.023900in}}{\pgfqpoint{1.420190in}{1.027172in}}{\pgfqpoint{1.426014in}{1.032996in}}%
\pgfpathcurveto{\pgfqpoint{1.431838in}{1.038820in}}{\pgfqpoint{1.435111in}{1.046720in}}{\pgfqpoint{1.435111in}{1.054956in}}%
\pgfpathcurveto{\pgfqpoint{1.435111in}{1.063193in}}{\pgfqpoint{1.431838in}{1.071093in}}{\pgfqpoint{1.426014in}{1.076917in}}%
\pgfpathcurveto{\pgfqpoint{1.420190in}{1.082741in}}{\pgfqpoint{1.412290in}{1.086013in}}{\pgfqpoint{1.404054in}{1.086013in}}%
\pgfpathcurveto{\pgfqpoint{1.395818in}{1.086013in}}{\pgfqpoint{1.387918in}{1.082741in}}{\pgfqpoint{1.382094in}{1.076917in}}%
\pgfpathcurveto{\pgfqpoint{1.376270in}{1.071093in}}{\pgfqpoint{1.372998in}{1.063193in}}{\pgfqpoint{1.372998in}{1.054956in}}%
\pgfpathcurveto{\pgfqpoint{1.372998in}{1.046720in}}{\pgfqpoint{1.376270in}{1.038820in}}{\pgfqpoint{1.382094in}{1.032996in}}%
\pgfpathcurveto{\pgfqpoint{1.387918in}{1.027172in}}{\pgfqpoint{1.395818in}{1.023900in}}{\pgfqpoint{1.404054in}{1.023900in}}%
\pgfpathclose%
\pgfusepath{stroke,fill}%
\end{pgfscope}%
\begin{pgfscope}%
\pgfpathrectangle{\pgfqpoint{0.100000in}{0.212622in}}{\pgfqpoint{3.696000in}{3.696000in}}%
\pgfusepath{clip}%
\pgfsetbuttcap%
\pgfsetroundjoin%
\definecolor{currentfill}{rgb}{0.121569,0.466667,0.705882}%
\pgfsetfillcolor{currentfill}%
\pgfsetfillopacity{0.820500}%
\pgfsetlinewidth{1.003750pt}%
\definecolor{currentstroke}{rgb}{0.121569,0.466667,0.705882}%
\pgfsetstrokecolor{currentstroke}%
\pgfsetstrokeopacity{0.820500}%
\pgfsetdash{}{0pt}%
\pgfpathmoveto{\pgfqpoint{1.275795in}{2.110226in}}%
\pgfpathcurveto{\pgfqpoint{1.284031in}{2.110226in}}{\pgfqpoint{1.291931in}{2.113498in}}{\pgfqpoint{1.297755in}{2.119322in}}%
\pgfpathcurveto{\pgfqpoint{1.303579in}{2.125146in}}{\pgfqpoint{1.306851in}{2.133046in}}{\pgfqpoint{1.306851in}{2.141283in}}%
\pgfpathcurveto{\pgfqpoint{1.306851in}{2.149519in}}{\pgfqpoint{1.303579in}{2.157419in}}{\pgfqpoint{1.297755in}{2.163243in}}%
\pgfpathcurveto{\pgfqpoint{1.291931in}{2.169067in}}{\pgfqpoint{1.284031in}{2.172339in}}{\pgfqpoint{1.275795in}{2.172339in}}%
\pgfpathcurveto{\pgfqpoint{1.267558in}{2.172339in}}{\pgfqpoint{1.259658in}{2.169067in}}{\pgfqpoint{1.253834in}{2.163243in}}%
\pgfpathcurveto{\pgfqpoint{1.248011in}{2.157419in}}{\pgfqpoint{1.244738in}{2.149519in}}{\pgfqpoint{1.244738in}{2.141283in}}%
\pgfpathcurveto{\pgfqpoint{1.244738in}{2.133046in}}{\pgfqpoint{1.248011in}{2.125146in}}{\pgfqpoint{1.253834in}{2.119322in}}%
\pgfpathcurveto{\pgfqpoint{1.259658in}{2.113498in}}{\pgfqpoint{1.267558in}{2.110226in}}{\pgfqpoint{1.275795in}{2.110226in}}%
\pgfpathclose%
\pgfusepath{stroke,fill}%
\end{pgfscope}%
\begin{pgfscope}%
\pgfpathrectangle{\pgfqpoint{0.100000in}{0.212622in}}{\pgfqpoint{3.696000in}{3.696000in}}%
\pgfusepath{clip}%
\pgfsetbuttcap%
\pgfsetroundjoin%
\definecolor{currentfill}{rgb}{0.121569,0.466667,0.705882}%
\pgfsetfillcolor{currentfill}%
\pgfsetfillopacity{0.822863}%
\pgfsetlinewidth{1.003750pt}%
\definecolor{currentstroke}{rgb}{0.121569,0.466667,0.705882}%
\pgfsetstrokecolor{currentstroke}%
\pgfsetstrokeopacity{0.822863}%
\pgfsetdash{}{0pt}%
\pgfpathmoveto{\pgfqpoint{1.430131in}{1.019438in}}%
\pgfpathcurveto{\pgfqpoint{1.438367in}{1.019438in}}{\pgfqpoint{1.446267in}{1.022711in}}{\pgfqpoint{1.452091in}{1.028534in}}%
\pgfpathcurveto{\pgfqpoint{1.457915in}{1.034358in}}{\pgfqpoint{1.461187in}{1.042258in}}{\pgfqpoint{1.461187in}{1.050495in}}%
\pgfpathcurveto{\pgfqpoint{1.461187in}{1.058731in}}{\pgfqpoint{1.457915in}{1.066631in}}{\pgfqpoint{1.452091in}{1.072455in}}%
\pgfpathcurveto{\pgfqpoint{1.446267in}{1.078279in}}{\pgfqpoint{1.438367in}{1.081551in}}{\pgfqpoint{1.430131in}{1.081551in}}%
\pgfpathcurveto{\pgfqpoint{1.421894in}{1.081551in}}{\pgfqpoint{1.413994in}{1.078279in}}{\pgfqpoint{1.408170in}{1.072455in}}%
\pgfpathcurveto{\pgfqpoint{1.402347in}{1.066631in}}{\pgfqpoint{1.399074in}{1.058731in}}{\pgfqpoint{1.399074in}{1.050495in}}%
\pgfpathcurveto{\pgfqpoint{1.399074in}{1.042258in}}{\pgfqpoint{1.402347in}{1.034358in}}{\pgfqpoint{1.408170in}{1.028534in}}%
\pgfpathcurveto{\pgfqpoint{1.413994in}{1.022711in}}{\pgfqpoint{1.421894in}{1.019438in}}{\pgfqpoint{1.430131in}{1.019438in}}%
\pgfpathclose%
\pgfusepath{stroke,fill}%
\end{pgfscope}%
\begin{pgfscope}%
\pgfpathrectangle{\pgfqpoint{0.100000in}{0.212622in}}{\pgfqpoint{3.696000in}{3.696000in}}%
\pgfusepath{clip}%
\pgfsetbuttcap%
\pgfsetroundjoin%
\definecolor{currentfill}{rgb}{0.121569,0.466667,0.705882}%
\pgfsetfillcolor{currentfill}%
\pgfsetfillopacity{0.827923}%
\pgfsetlinewidth{1.003750pt}%
\definecolor{currentstroke}{rgb}{0.121569,0.466667,0.705882}%
\pgfsetstrokecolor{currentstroke}%
\pgfsetstrokeopacity{0.827923}%
\pgfsetdash{}{0pt}%
\pgfpathmoveto{\pgfqpoint{1.453772in}{1.016117in}}%
\pgfpathcurveto{\pgfqpoint{1.462009in}{1.016117in}}{\pgfqpoint{1.469909in}{1.019389in}}{\pgfqpoint{1.475733in}{1.025213in}}%
\pgfpathcurveto{\pgfqpoint{1.481557in}{1.031037in}}{\pgfqpoint{1.484829in}{1.038937in}}{\pgfqpoint{1.484829in}{1.047174in}}%
\pgfpathcurveto{\pgfqpoint{1.484829in}{1.055410in}}{\pgfqpoint{1.481557in}{1.063310in}}{\pgfqpoint{1.475733in}{1.069134in}}%
\pgfpathcurveto{\pgfqpoint{1.469909in}{1.074958in}}{\pgfqpoint{1.462009in}{1.078230in}}{\pgfqpoint{1.453772in}{1.078230in}}%
\pgfpathcurveto{\pgfqpoint{1.445536in}{1.078230in}}{\pgfqpoint{1.437636in}{1.074958in}}{\pgfqpoint{1.431812in}{1.069134in}}%
\pgfpathcurveto{\pgfqpoint{1.425988in}{1.063310in}}{\pgfqpoint{1.422716in}{1.055410in}}{\pgfqpoint{1.422716in}{1.047174in}}%
\pgfpathcurveto{\pgfqpoint{1.422716in}{1.038937in}}{\pgfqpoint{1.425988in}{1.031037in}}{\pgfqpoint{1.431812in}{1.025213in}}%
\pgfpathcurveto{\pgfqpoint{1.437636in}{1.019389in}}{\pgfqpoint{1.445536in}{1.016117in}}{\pgfqpoint{1.453772in}{1.016117in}}%
\pgfpathclose%
\pgfusepath{stroke,fill}%
\end{pgfscope}%
\begin{pgfscope}%
\pgfpathrectangle{\pgfqpoint{0.100000in}{0.212622in}}{\pgfqpoint{3.696000in}{3.696000in}}%
\pgfusepath{clip}%
\pgfsetbuttcap%
\pgfsetroundjoin%
\definecolor{currentfill}{rgb}{0.121569,0.466667,0.705882}%
\pgfsetfillcolor{currentfill}%
\pgfsetfillopacity{0.828404}%
\pgfsetlinewidth{1.003750pt}%
\definecolor{currentstroke}{rgb}{0.121569,0.466667,0.705882}%
\pgfsetstrokecolor{currentstroke}%
\pgfsetstrokeopacity{0.828404}%
\pgfsetdash{}{0pt}%
\pgfpathmoveto{\pgfqpoint{2.531058in}{1.488563in}}%
\pgfpathcurveto{\pgfqpoint{2.539295in}{1.488563in}}{\pgfqpoint{2.547195in}{1.491836in}}{\pgfqpoint{2.553019in}{1.497660in}}%
\pgfpathcurveto{\pgfqpoint{2.558843in}{1.503483in}}{\pgfqpoint{2.562115in}{1.511384in}}{\pgfqpoint{2.562115in}{1.519620in}}%
\pgfpathcurveto{\pgfqpoint{2.562115in}{1.527856in}}{\pgfqpoint{2.558843in}{1.535756in}}{\pgfqpoint{2.553019in}{1.541580in}}%
\pgfpathcurveto{\pgfqpoint{2.547195in}{1.547404in}}{\pgfqpoint{2.539295in}{1.550676in}}{\pgfqpoint{2.531058in}{1.550676in}}%
\pgfpathcurveto{\pgfqpoint{2.522822in}{1.550676in}}{\pgfqpoint{2.514922in}{1.547404in}}{\pgfqpoint{2.509098in}{1.541580in}}%
\pgfpathcurveto{\pgfqpoint{2.503274in}{1.535756in}}{\pgfqpoint{2.500002in}{1.527856in}}{\pgfqpoint{2.500002in}{1.519620in}}%
\pgfpathcurveto{\pgfqpoint{2.500002in}{1.511384in}}{\pgfqpoint{2.503274in}{1.503483in}}{\pgfqpoint{2.509098in}{1.497660in}}%
\pgfpathcurveto{\pgfqpoint{2.514922in}{1.491836in}}{\pgfqpoint{2.522822in}{1.488563in}}{\pgfqpoint{2.531058in}{1.488563in}}%
\pgfpathclose%
\pgfusepath{stroke,fill}%
\end{pgfscope}%
\begin{pgfscope}%
\pgfpathrectangle{\pgfqpoint{0.100000in}{0.212622in}}{\pgfqpoint{3.696000in}{3.696000in}}%
\pgfusepath{clip}%
\pgfsetbuttcap%
\pgfsetroundjoin%
\definecolor{currentfill}{rgb}{0.121569,0.466667,0.705882}%
\pgfsetfillcolor{currentfill}%
\pgfsetfillopacity{0.829985}%
\pgfsetlinewidth{1.003750pt}%
\definecolor{currentstroke}{rgb}{0.121569,0.466667,0.705882}%
\pgfsetstrokecolor{currentstroke}%
\pgfsetstrokeopacity{0.829985}%
\pgfsetdash{}{0pt}%
\pgfpathmoveto{\pgfqpoint{1.274428in}{2.118070in}}%
\pgfpathcurveto{\pgfqpoint{1.282665in}{2.118070in}}{\pgfqpoint{1.290565in}{2.121343in}}{\pgfqpoint{1.296389in}{2.127166in}}%
\pgfpathcurveto{\pgfqpoint{1.302213in}{2.132990in}}{\pgfqpoint{1.305485in}{2.140890in}}{\pgfqpoint{1.305485in}{2.149127in}}%
\pgfpathcurveto{\pgfqpoint{1.305485in}{2.157363in}}{\pgfqpoint{1.302213in}{2.165263in}}{\pgfqpoint{1.296389in}{2.171087in}}%
\pgfpathcurveto{\pgfqpoint{1.290565in}{2.176911in}}{\pgfqpoint{1.282665in}{2.180183in}}{\pgfqpoint{1.274428in}{2.180183in}}%
\pgfpathcurveto{\pgfqpoint{1.266192in}{2.180183in}}{\pgfqpoint{1.258292in}{2.176911in}}{\pgfqpoint{1.252468in}{2.171087in}}%
\pgfpathcurveto{\pgfqpoint{1.246644in}{2.165263in}}{\pgfqpoint{1.243372in}{2.157363in}}{\pgfqpoint{1.243372in}{2.149127in}}%
\pgfpathcurveto{\pgfqpoint{1.243372in}{2.140890in}}{\pgfqpoint{1.246644in}{2.132990in}}{\pgfqpoint{1.252468in}{2.127166in}}%
\pgfpathcurveto{\pgfqpoint{1.258292in}{2.121343in}}{\pgfqpoint{1.266192in}{2.118070in}}{\pgfqpoint{1.274428in}{2.118070in}}%
\pgfpathclose%
\pgfusepath{stroke,fill}%
\end{pgfscope}%
\begin{pgfscope}%
\pgfpathrectangle{\pgfqpoint{0.100000in}{0.212622in}}{\pgfqpoint{3.696000in}{3.696000in}}%
\pgfusepath{clip}%
\pgfsetbuttcap%
\pgfsetroundjoin%
\definecolor{currentfill}{rgb}{0.121569,0.466667,0.705882}%
\pgfsetfillcolor{currentfill}%
\pgfsetfillopacity{0.831851}%
\pgfsetlinewidth{1.003750pt}%
\definecolor{currentstroke}{rgb}{0.121569,0.466667,0.705882}%
\pgfsetstrokecolor{currentstroke}%
\pgfsetstrokeopacity{0.831851}%
\pgfsetdash{}{0pt}%
\pgfpathmoveto{\pgfqpoint{1.471778in}{1.013520in}}%
\pgfpathcurveto{\pgfqpoint{1.480014in}{1.013520in}}{\pgfqpoint{1.487914in}{1.016792in}}{\pgfqpoint{1.493738in}{1.022616in}}%
\pgfpathcurveto{\pgfqpoint{1.499562in}{1.028440in}}{\pgfqpoint{1.502834in}{1.036340in}}{\pgfqpoint{1.502834in}{1.044576in}}%
\pgfpathcurveto{\pgfqpoint{1.502834in}{1.052813in}}{\pgfqpoint{1.499562in}{1.060713in}}{\pgfqpoint{1.493738in}{1.066537in}}%
\pgfpathcurveto{\pgfqpoint{1.487914in}{1.072361in}}{\pgfqpoint{1.480014in}{1.075633in}}{\pgfqpoint{1.471778in}{1.075633in}}%
\pgfpathcurveto{\pgfqpoint{1.463541in}{1.075633in}}{\pgfqpoint{1.455641in}{1.072361in}}{\pgfqpoint{1.449818in}{1.066537in}}%
\pgfpathcurveto{\pgfqpoint{1.443994in}{1.060713in}}{\pgfqpoint{1.440721in}{1.052813in}}{\pgfqpoint{1.440721in}{1.044576in}}%
\pgfpathcurveto{\pgfqpoint{1.440721in}{1.036340in}}{\pgfqpoint{1.443994in}{1.028440in}}{\pgfqpoint{1.449818in}{1.022616in}}%
\pgfpathcurveto{\pgfqpoint{1.455641in}{1.016792in}}{\pgfqpoint{1.463541in}{1.013520in}}{\pgfqpoint{1.471778in}{1.013520in}}%
\pgfpathclose%
\pgfusepath{stroke,fill}%
\end{pgfscope}%
\begin{pgfscope}%
\pgfpathrectangle{\pgfqpoint{0.100000in}{0.212622in}}{\pgfqpoint{3.696000in}{3.696000in}}%
\pgfusepath{clip}%
\pgfsetbuttcap%
\pgfsetroundjoin%
\definecolor{currentfill}{rgb}{0.121569,0.466667,0.705882}%
\pgfsetfillcolor{currentfill}%
\pgfsetfillopacity{0.835385}%
\pgfsetlinewidth{1.003750pt}%
\definecolor{currentstroke}{rgb}{0.121569,0.466667,0.705882}%
\pgfsetstrokecolor{currentstroke}%
\pgfsetstrokeopacity{0.835385}%
\pgfsetdash{}{0pt}%
\pgfpathmoveto{\pgfqpoint{1.487576in}{1.010970in}}%
\pgfpathcurveto{\pgfqpoint{1.495812in}{1.010970in}}{\pgfqpoint{1.503712in}{1.014242in}}{\pgfqpoint{1.509536in}{1.020066in}}%
\pgfpathcurveto{\pgfqpoint{1.515360in}{1.025890in}}{\pgfqpoint{1.518632in}{1.033790in}}{\pgfqpoint{1.518632in}{1.042026in}}%
\pgfpathcurveto{\pgfqpoint{1.518632in}{1.050262in}}{\pgfqpoint{1.515360in}{1.058162in}}{\pgfqpoint{1.509536in}{1.063986in}}%
\pgfpathcurveto{\pgfqpoint{1.503712in}{1.069810in}}{\pgfqpoint{1.495812in}{1.073083in}}{\pgfqpoint{1.487576in}{1.073083in}}%
\pgfpathcurveto{\pgfqpoint{1.479340in}{1.073083in}}{\pgfqpoint{1.471440in}{1.069810in}}{\pgfqpoint{1.465616in}{1.063986in}}%
\pgfpathcurveto{\pgfqpoint{1.459792in}{1.058162in}}{\pgfqpoint{1.456519in}{1.050262in}}{\pgfqpoint{1.456519in}{1.042026in}}%
\pgfpathcurveto{\pgfqpoint{1.456519in}{1.033790in}}{\pgfqpoint{1.459792in}{1.025890in}}{\pgfqpoint{1.465616in}{1.020066in}}%
\pgfpathcurveto{\pgfqpoint{1.471440in}{1.014242in}}{\pgfqpoint{1.479340in}{1.010970in}}{\pgfqpoint{1.487576in}{1.010970in}}%
\pgfpathclose%
\pgfusepath{stroke,fill}%
\end{pgfscope}%
\begin{pgfscope}%
\pgfpathrectangle{\pgfqpoint{0.100000in}{0.212622in}}{\pgfqpoint{3.696000in}{3.696000in}}%
\pgfusepath{clip}%
\pgfsetbuttcap%
\pgfsetroundjoin%
\definecolor{currentfill}{rgb}{0.121569,0.466667,0.705882}%
\pgfsetfillcolor{currentfill}%
\pgfsetfillopacity{0.841777}%
\pgfsetlinewidth{1.003750pt}%
\definecolor{currentstroke}{rgb}{0.121569,0.466667,0.705882}%
\pgfsetstrokecolor{currentstroke}%
\pgfsetstrokeopacity{0.841777}%
\pgfsetdash{}{0pt}%
\pgfpathmoveto{\pgfqpoint{1.516399in}{1.006459in}}%
\pgfpathcurveto{\pgfqpoint{1.524636in}{1.006459in}}{\pgfqpoint{1.532536in}{1.009732in}}{\pgfqpoint{1.538360in}{1.015555in}}%
\pgfpathcurveto{\pgfqpoint{1.544184in}{1.021379in}}{\pgfqpoint{1.547456in}{1.029279in}}{\pgfqpoint{1.547456in}{1.037516in}}%
\pgfpathcurveto{\pgfqpoint{1.547456in}{1.045752in}}{\pgfqpoint{1.544184in}{1.053652in}}{\pgfqpoint{1.538360in}{1.059476in}}%
\pgfpathcurveto{\pgfqpoint{1.532536in}{1.065300in}}{\pgfqpoint{1.524636in}{1.068572in}}{\pgfqpoint{1.516399in}{1.068572in}}%
\pgfpathcurveto{\pgfqpoint{1.508163in}{1.068572in}}{\pgfqpoint{1.500263in}{1.065300in}}{\pgfqpoint{1.494439in}{1.059476in}}%
\pgfpathcurveto{\pgfqpoint{1.488615in}{1.053652in}}{\pgfqpoint{1.485343in}{1.045752in}}{\pgfqpoint{1.485343in}{1.037516in}}%
\pgfpathcurveto{\pgfqpoint{1.485343in}{1.029279in}}{\pgfqpoint{1.488615in}{1.021379in}}{\pgfqpoint{1.494439in}{1.015555in}}%
\pgfpathcurveto{\pgfqpoint{1.500263in}{1.009732in}}{\pgfqpoint{1.508163in}{1.006459in}}{\pgfqpoint{1.516399in}{1.006459in}}%
\pgfpathclose%
\pgfusepath{stroke,fill}%
\end{pgfscope}%
\begin{pgfscope}%
\pgfpathrectangle{\pgfqpoint{0.100000in}{0.212622in}}{\pgfqpoint{3.696000in}{3.696000in}}%
\pgfusepath{clip}%
\pgfsetbuttcap%
\pgfsetroundjoin%
\definecolor{currentfill}{rgb}{0.121569,0.466667,0.705882}%
\pgfsetfillcolor{currentfill}%
\pgfsetfillopacity{0.842154}%
\pgfsetlinewidth{1.003750pt}%
\definecolor{currentstroke}{rgb}{0.121569,0.466667,0.705882}%
\pgfsetstrokecolor{currentstroke}%
\pgfsetstrokeopacity{0.842154}%
\pgfsetdash{}{0pt}%
\pgfpathmoveto{\pgfqpoint{2.543996in}{1.441982in}}%
\pgfpathcurveto{\pgfqpoint{2.552232in}{1.441982in}}{\pgfqpoint{2.560132in}{1.445254in}}{\pgfqpoint{2.565956in}{1.451078in}}%
\pgfpathcurveto{\pgfqpoint{2.571780in}{1.456902in}}{\pgfqpoint{2.575052in}{1.464802in}}{\pgfqpoint{2.575052in}{1.473039in}}%
\pgfpathcurveto{\pgfqpoint{2.575052in}{1.481275in}}{\pgfqpoint{2.571780in}{1.489175in}}{\pgfqpoint{2.565956in}{1.494999in}}%
\pgfpathcurveto{\pgfqpoint{2.560132in}{1.500823in}}{\pgfqpoint{2.552232in}{1.504095in}}{\pgfqpoint{2.543996in}{1.504095in}}%
\pgfpathcurveto{\pgfqpoint{2.535759in}{1.504095in}}{\pgfqpoint{2.527859in}{1.500823in}}{\pgfqpoint{2.522035in}{1.494999in}}%
\pgfpathcurveto{\pgfqpoint{2.516212in}{1.489175in}}{\pgfqpoint{2.512939in}{1.481275in}}{\pgfqpoint{2.512939in}{1.473039in}}%
\pgfpathcurveto{\pgfqpoint{2.512939in}{1.464802in}}{\pgfqpoint{2.516212in}{1.456902in}}{\pgfqpoint{2.522035in}{1.451078in}}%
\pgfpathcurveto{\pgfqpoint{2.527859in}{1.445254in}}{\pgfqpoint{2.535759in}{1.441982in}}{\pgfqpoint{2.543996in}{1.441982in}}%
\pgfpathclose%
\pgfusepath{stroke,fill}%
\end{pgfscope}%
\begin{pgfscope}%
\pgfpathrectangle{\pgfqpoint{0.100000in}{0.212622in}}{\pgfqpoint{3.696000in}{3.696000in}}%
\pgfusepath{clip}%
\pgfsetbuttcap%
\pgfsetroundjoin%
\definecolor{currentfill}{rgb}{0.121569,0.466667,0.705882}%
\pgfsetfillcolor{currentfill}%
\pgfsetfillopacity{0.845914}%
\pgfsetlinewidth{1.003750pt}%
\definecolor{currentstroke}{rgb}{0.121569,0.466667,0.705882}%
\pgfsetstrokecolor{currentstroke}%
\pgfsetstrokeopacity{0.845914}%
\pgfsetdash{}{0pt}%
\pgfpathmoveto{\pgfqpoint{1.270339in}{2.127605in}}%
\pgfpathcurveto{\pgfqpoint{1.278576in}{2.127605in}}{\pgfqpoint{1.286476in}{2.130878in}}{\pgfqpoint{1.292300in}{2.136702in}}%
\pgfpathcurveto{\pgfqpoint{1.298124in}{2.142525in}}{\pgfqpoint{1.301396in}{2.150426in}}{\pgfqpoint{1.301396in}{2.158662in}}%
\pgfpathcurveto{\pgfqpoint{1.301396in}{2.166898in}}{\pgfqpoint{1.298124in}{2.174798in}}{\pgfqpoint{1.292300in}{2.180622in}}%
\pgfpathcurveto{\pgfqpoint{1.286476in}{2.186446in}}{\pgfqpoint{1.278576in}{2.189718in}}{\pgfqpoint{1.270339in}{2.189718in}}%
\pgfpathcurveto{\pgfqpoint{1.262103in}{2.189718in}}{\pgfqpoint{1.254203in}{2.186446in}}{\pgfqpoint{1.248379in}{2.180622in}}%
\pgfpathcurveto{\pgfqpoint{1.242555in}{2.174798in}}{\pgfqpoint{1.239283in}{2.166898in}}{\pgfqpoint{1.239283in}{2.158662in}}%
\pgfpathcurveto{\pgfqpoint{1.239283in}{2.150426in}}{\pgfqpoint{1.242555in}{2.142525in}}{\pgfqpoint{1.248379in}{2.136702in}}%
\pgfpathcurveto{\pgfqpoint{1.254203in}{2.130878in}}{\pgfqpoint{1.262103in}{2.127605in}}{\pgfqpoint{1.270339in}{2.127605in}}%
\pgfpathclose%
\pgfusepath{stroke,fill}%
\end{pgfscope}%
\begin{pgfscope}%
\pgfpathrectangle{\pgfqpoint{0.100000in}{0.212622in}}{\pgfqpoint{3.696000in}{3.696000in}}%
\pgfusepath{clip}%
\pgfsetbuttcap%
\pgfsetroundjoin%
\definecolor{currentfill}{rgb}{0.121569,0.466667,0.705882}%
\pgfsetfillcolor{currentfill}%
\pgfsetfillopacity{0.847607}%
\pgfsetlinewidth{1.003750pt}%
\definecolor{currentstroke}{rgb}{0.121569,0.466667,0.705882}%
\pgfsetstrokecolor{currentstroke}%
\pgfsetstrokeopacity{0.847607}%
\pgfsetdash{}{0pt}%
\pgfpathmoveto{\pgfqpoint{1.543055in}{1.002229in}}%
\pgfpathcurveto{\pgfqpoint{1.551292in}{1.002229in}}{\pgfqpoint{1.559192in}{1.005501in}}{\pgfqpoint{1.565016in}{1.011325in}}%
\pgfpathcurveto{\pgfqpoint{1.570839in}{1.017149in}}{\pgfqpoint{1.574112in}{1.025049in}}{\pgfqpoint{1.574112in}{1.033286in}}%
\pgfpathcurveto{\pgfqpoint{1.574112in}{1.041522in}}{\pgfqpoint{1.570839in}{1.049422in}}{\pgfqpoint{1.565016in}{1.055246in}}%
\pgfpathcurveto{\pgfqpoint{1.559192in}{1.061070in}}{\pgfqpoint{1.551292in}{1.064342in}}{\pgfqpoint{1.543055in}{1.064342in}}%
\pgfpathcurveto{\pgfqpoint{1.534819in}{1.064342in}}{\pgfqpoint{1.526919in}{1.061070in}}{\pgfqpoint{1.521095in}{1.055246in}}%
\pgfpathcurveto{\pgfqpoint{1.515271in}{1.049422in}}{\pgfqpoint{1.511999in}{1.041522in}}{\pgfqpoint{1.511999in}{1.033286in}}%
\pgfpathcurveto{\pgfqpoint{1.511999in}{1.025049in}}{\pgfqpoint{1.515271in}{1.017149in}}{\pgfqpoint{1.521095in}{1.011325in}}%
\pgfpathcurveto{\pgfqpoint{1.526919in}{1.005501in}}{\pgfqpoint{1.534819in}{1.002229in}}{\pgfqpoint{1.543055in}{1.002229in}}%
\pgfpathclose%
\pgfusepath{stroke,fill}%
\end{pgfscope}%
\begin{pgfscope}%
\pgfpathrectangle{\pgfqpoint{0.100000in}{0.212622in}}{\pgfqpoint{3.696000in}{3.696000in}}%
\pgfusepath{clip}%
\pgfsetbuttcap%
\pgfsetroundjoin%
\definecolor{currentfill}{rgb}{0.121569,0.466667,0.705882}%
\pgfsetfillcolor{currentfill}%
\pgfsetfillopacity{0.852062}%
\pgfsetlinewidth{1.003750pt}%
\definecolor{currentstroke}{rgb}{0.121569,0.466667,0.705882}%
\pgfsetstrokecolor{currentstroke}%
\pgfsetstrokeopacity{0.852062}%
\pgfsetdash{}{0pt}%
\pgfpathmoveto{\pgfqpoint{1.564535in}{0.998990in}}%
\pgfpathcurveto{\pgfqpoint{1.572771in}{0.998990in}}{\pgfqpoint{1.580671in}{1.002263in}}{\pgfqpoint{1.586495in}{1.008087in}}%
\pgfpathcurveto{\pgfqpoint{1.592319in}{1.013911in}}{\pgfqpoint{1.595591in}{1.021811in}}{\pgfqpoint{1.595591in}{1.030047in}}%
\pgfpathcurveto{\pgfqpoint{1.595591in}{1.038283in}}{\pgfqpoint{1.592319in}{1.046183in}}{\pgfqpoint{1.586495in}{1.052007in}}%
\pgfpathcurveto{\pgfqpoint{1.580671in}{1.057831in}}{\pgfqpoint{1.572771in}{1.061103in}}{\pgfqpoint{1.564535in}{1.061103in}}%
\pgfpathcurveto{\pgfqpoint{1.556298in}{1.061103in}}{\pgfqpoint{1.548398in}{1.057831in}}{\pgfqpoint{1.542574in}{1.052007in}}%
\pgfpathcurveto{\pgfqpoint{1.536750in}{1.046183in}}{\pgfqpoint{1.533478in}{1.038283in}}{\pgfqpoint{1.533478in}{1.030047in}}%
\pgfpathcurveto{\pgfqpoint{1.533478in}{1.021811in}}{\pgfqpoint{1.536750in}{1.013911in}}{\pgfqpoint{1.542574in}{1.008087in}}%
\pgfpathcurveto{\pgfqpoint{1.548398in}{1.002263in}}{\pgfqpoint{1.556298in}{0.998990in}}{\pgfqpoint{1.564535in}{0.998990in}}%
\pgfpathclose%
\pgfusepath{stroke,fill}%
\end{pgfscope}%
\begin{pgfscope}%
\pgfpathrectangle{\pgfqpoint{0.100000in}{0.212622in}}{\pgfqpoint{3.696000in}{3.696000in}}%
\pgfusepath{clip}%
\pgfsetbuttcap%
\pgfsetroundjoin%
\definecolor{currentfill}{rgb}{0.121569,0.466667,0.705882}%
\pgfsetfillcolor{currentfill}%
\pgfsetfillopacity{0.855808}%
\pgfsetlinewidth{1.003750pt}%
\definecolor{currentstroke}{rgb}{0.121569,0.466667,0.705882}%
\pgfsetstrokecolor{currentstroke}%
\pgfsetstrokeopacity{0.855808}%
\pgfsetdash{}{0pt}%
\pgfpathmoveto{\pgfqpoint{1.581649in}{0.996591in}}%
\pgfpathcurveto{\pgfqpoint{1.589885in}{0.996591in}}{\pgfqpoint{1.597785in}{0.999863in}}{\pgfqpoint{1.603609in}{1.005687in}}%
\pgfpathcurveto{\pgfqpoint{1.609433in}{1.011511in}}{\pgfqpoint{1.612706in}{1.019411in}}{\pgfqpoint{1.612706in}{1.027648in}}%
\pgfpathcurveto{\pgfqpoint{1.612706in}{1.035884in}}{\pgfqpoint{1.609433in}{1.043784in}}{\pgfqpoint{1.603609in}{1.049608in}}%
\pgfpathcurveto{\pgfqpoint{1.597785in}{1.055432in}}{\pgfqpoint{1.589885in}{1.058704in}}{\pgfqpoint{1.581649in}{1.058704in}}%
\pgfpathcurveto{\pgfqpoint{1.573413in}{1.058704in}}{\pgfqpoint{1.565513in}{1.055432in}}{\pgfqpoint{1.559689in}{1.049608in}}%
\pgfpathcurveto{\pgfqpoint{1.553865in}{1.043784in}}{\pgfqpoint{1.550593in}{1.035884in}}{\pgfqpoint{1.550593in}{1.027648in}}%
\pgfpathcurveto{\pgfqpoint{1.550593in}{1.019411in}}{\pgfqpoint{1.553865in}{1.011511in}}{\pgfqpoint{1.559689in}{1.005687in}}%
\pgfpathcurveto{\pgfqpoint{1.565513in}{0.999863in}}{\pgfqpoint{1.573413in}{0.996591in}}{\pgfqpoint{1.581649in}{0.996591in}}%
\pgfpathclose%
\pgfusepath{stroke,fill}%
\end{pgfscope}%
\begin{pgfscope}%
\pgfpathrectangle{\pgfqpoint{0.100000in}{0.212622in}}{\pgfqpoint{3.696000in}{3.696000in}}%
\pgfusepath{clip}%
\pgfsetbuttcap%
\pgfsetroundjoin%
\definecolor{currentfill}{rgb}{0.121569,0.466667,0.705882}%
\pgfsetfillcolor{currentfill}%
\pgfsetfillopacity{0.857114}%
\pgfsetlinewidth{1.003750pt}%
\definecolor{currentstroke}{rgb}{0.121569,0.466667,0.705882}%
\pgfsetstrokecolor{currentstroke}%
\pgfsetstrokeopacity{0.857114}%
\pgfsetdash{}{0pt}%
\pgfpathmoveto{\pgfqpoint{2.557550in}{1.391215in}}%
\pgfpathcurveto{\pgfqpoint{2.565786in}{1.391215in}}{\pgfqpoint{2.573686in}{1.394487in}}{\pgfqpoint{2.579510in}{1.400311in}}%
\pgfpathcurveto{\pgfqpoint{2.585334in}{1.406135in}}{\pgfqpoint{2.588606in}{1.414035in}}{\pgfqpoint{2.588606in}{1.422271in}}%
\pgfpathcurveto{\pgfqpoint{2.588606in}{1.430507in}}{\pgfqpoint{2.585334in}{1.438408in}}{\pgfqpoint{2.579510in}{1.444231in}}%
\pgfpathcurveto{\pgfqpoint{2.573686in}{1.450055in}}{\pgfqpoint{2.565786in}{1.453328in}}{\pgfqpoint{2.557550in}{1.453328in}}%
\pgfpathcurveto{\pgfqpoint{2.549313in}{1.453328in}}{\pgfqpoint{2.541413in}{1.450055in}}{\pgfqpoint{2.535590in}{1.444231in}}%
\pgfpathcurveto{\pgfqpoint{2.529766in}{1.438408in}}{\pgfqpoint{2.526493in}{1.430507in}}{\pgfqpoint{2.526493in}{1.422271in}}%
\pgfpathcurveto{\pgfqpoint{2.526493in}{1.414035in}}{\pgfqpoint{2.529766in}{1.406135in}}{\pgfqpoint{2.535590in}{1.400311in}}%
\pgfpathcurveto{\pgfqpoint{2.541413in}{1.394487in}}{\pgfqpoint{2.549313in}{1.391215in}}{\pgfqpoint{2.557550in}{1.391215in}}%
\pgfpathclose%
\pgfusepath{stroke,fill}%
\end{pgfscope}%
\begin{pgfscope}%
\pgfpathrectangle{\pgfqpoint{0.100000in}{0.212622in}}{\pgfqpoint{3.696000in}{3.696000in}}%
\pgfusepath{clip}%
\pgfsetbuttcap%
\pgfsetroundjoin%
\definecolor{currentfill}{rgb}{0.121569,0.466667,0.705882}%
\pgfsetfillcolor{currentfill}%
\pgfsetfillopacity{0.858009}%
\pgfsetlinewidth{1.003750pt}%
\definecolor{currentstroke}{rgb}{0.121569,0.466667,0.705882}%
\pgfsetstrokecolor{currentstroke}%
\pgfsetstrokeopacity{0.858009}%
\pgfsetdash{}{0pt}%
\pgfpathmoveto{\pgfqpoint{1.261621in}{2.129491in}}%
\pgfpathcurveto{\pgfqpoint{1.269858in}{2.129491in}}{\pgfqpoint{1.277758in}{2.132763in}}{\pgfqpoint{1.283582in}{2.138587in}}%
\pgfpathcurveto{\pgfqpoint{1.289406in}{2.144411in}}{\pgfqpoint{1.292678in}{2.152311in}}{\pgfqpoint{1.292678in}{2.160547in}}%
\pgfpathcurveto{\pgfqpoint{1.292678in}{2.168784in}}{\pgfqpoint{1.289406in}{2.176684in}}{\pgfqpoint{1.283582in}{2.182508in}}%
\pgfpathcurveto{\pgfqpoint{1.277758in}{2.188331in}}{\pgfqpoint{1.269858in}{2.191604in}}{\pgfqpoint{1.261621in}{2.191604in}}%
\pgfpathcurveto{\pgfqpoint{1.253385in}{2.191604in}}{\pgfqpoint{1.245485in}{2.188331in}}{\pgfqpoint{1.239661in}{2.182508in}}%
\pgfpathcurveto{\pgfqpoint{1.233837in}{2.176684in}}{\pgfqpoint{1.230565in}{2.168784in}}{\pgfqpoint{1.230565in}{2.160547in}}%
\pgfpathcurveto{\pgfqpoint{1.230565in}{2.152311in}}{\pgfqpoint{1.233837in}{2.144411in}}{\pgfqpoint{1.239661in}{2.138587in}}%
\pgfpathcurveto{\pgfqpoint{1.245485in}{2.132763in}}{\pgfqpoint{1.253385in}{2.129491in}}{\pgfqpoint{1.261621in}{2.129491in}}%
\pgfpathclose%
\pgfusepath{stroke,fill}%
\end{pgfscope}%
\begin{pgfscope}%
\pgfpathrectangle{\pgfqpoint{0.100000in}{0.212622in}}{\pgfqpoint{3.696000in}{3.696000in}}%
\pgfusepath{clip}%
\pgfsetbuttcap%
\pgfsetroundjoin%
\definecolor{currentfill}{rgb}{0.121569,0.466667,0.705882}%
\pgfsetfillcolor{currentfill}%
\pgfsetfillopacity{0.858925}%
\pgfsetlinewidth{1.003750pt}%
\definecolor{currentstroke}{rgb}{0.121569,0.466667,0.705882}%
\pgfsetstrokecolor{currentstroke}%
\pgfsetstrokeopacity{0.858925}%
\pgfsetdash{}{0pt}%
\pgfpathmoveto{\pgfqpoint{1.596823in}{0.994505in}}%
\pgfpathcurveto{\pgfqpoint{1.605060in}{0.994505in}}{\pgfqpoint{1.612960in}{0.997777in}}{\pgfqpoint{1.618784in}{1.003601in}}%
\pgfpathcurveto{\pgfqpoint{1.624608in}{1.009425in}}{\pgfqpoint{1.627880in}{1.017325in}}{\pgfqpoint{1.627880in}{1.025561in}}%
\pgfpathcurveto{\pgfqpoint{1.627880in}{1.033797in}}{\pgfqpoint{1.624608in}{1.041697in}}{\pgfqpoint{1.618784in}{1.047521in}}%
\pgfpathcurveto{\pgfqpoint{1.612960in}{1.053345in}}{\pgfqpoint{1.605060in}{1.056618in}}{\pgfqpoint{1.596823in}{1.056618in}}%
\pgfpathcurveto{\pgfqpoint{1.588587in}{1.056618in}}{\pgfqpoint{1.580687in}{1.053345in}}{\pgfqpoint{1.574863in}{1.047521in}}%
\pgfpathcurveto{\pgfqpoint{1.569039in}{1.041697in}}{\pgfqpoint{1.565767in}{1.033797in}}{\pgfqpoint{1.565767in}{1.025561in}}%
\pgfpathcurveto{\pgfqpoint{1.565767in}{1.017325in}}{\pgfqpoint{1.569039in}{1.009425in}}{\pgfqpoint{1.574863in}{1.003601in}}%
\pgfpathcurveto{\pgfqpoint{1.580687in}{0.997777in}}{\pgfqpoint{1.588587in}{0.994505in}}{\pgfqpoint{1.596823in}{0.994505in}}%
\pgfpathclose%
\pgfusepath{stroke,fill}%
\end{pgfscope}%
\begin{pgfscope}%
\pgfpathrectangle{\pgfqpoint{0.100000in}{0.212622in}}{\pgfqpoint{3.696000in}{3.696000in}}%
\pgfusepath{clip}%
\pgfsetbuttcap%
\pgfsetroundjoin%
\definecolor{currentfill}{rgb}{0.121569,0.466667,0.705882}%
\pgfsetfillcolor{currentfill}%
\pgfsetfillopacity{0.864769}%
\pgfsetlinewidth{1.003750pt}%
\definecolor{currentstroke}{rgb}{0.121569,0.466667,0.705882}%
\pgfsetstrokecolor{currentstroke}%
\pgfsetstrokeopacity{0.864769}%
\pgfsetdash{}{0pt}%
\pgfpathmoveto{\pgfqpoint{1.624297in}{0.990708in}}%
\pgfpathcurveto{\pgfqpoint{1.632533in}{0.990708in}}{\pgfqpoint{1.640433in}{0.993981in}}{\pgfqpoint{1.646257in}{0.999805in}}%
\pgfpathcurveto{\pgfqpoint{1.652081in}{1.005629in}}{\pgfqpoint{1.655354in}{1.013529in}}{\pgfqpoint{1.655354in}{1.021765in}}%
\pgfpathcurveto{\pgfqpoint{1.655354in}{1.030001in}}{\pgfqpoint{1.652081in}{1.037901in}}{\pgfqpoint{1.646257in}{1.043725in}}%
\pgfpathcurveto{\pgfqpoint{1.640433in}{1.049549in}}{\pgfqpoint{1.632533in}{1.052821in}}{\pgfqpoint{1.624297in}{1.052821in}}%
\pgfpathcurveto{\pgfqpoint{1.616061in}{1.052821in}}{\pgfqpoint{1.608161in}{1.049549in}}{\pgfqpoint{1.602337in}{1.043725in}}%
\pgfpathcurveto{\pgfqpoint{1.596513in}{1.037901in}}{\pgfqpoint{1.593241in}{1.030001in}}{\pgfqpoint{1.593241in}{1.021765in}}%
\pgfpathcurveto{\pgfqpoint{1.593241in}{1.013529in}}{\pgfqpoint{1.596513in}{1.005629in}}{\pgfqpoint{1.602337in}{0.999805in}}%
\pgfpathcurveto{\pgfqpoint{1.608161in}{0.993981in}}{\pgfqpoint{1.616061in}{0.990708in}}{\pgfqpoint{1.624297in}{0.990708in}}%
\pgfpathclose%
\pgfusepath{stroke,fill}%
\end{pgfscope}%
\begin{pgfscope}%
\pgfpathrectangle{\pgfqpoint{0.100000in}{0.212622in}}{\pgfqpoint{3.696000in}{3.696000in}}%
\pgfusepath{clip}%
\pgfsetbuttcap%
\pgfsetroundjoin%
\definecolor{currentfill}{rgb}{0.121569,0.466667,0.705882}%
\pgfsetfillcolor{currentfill}%
\pgfsetfillopacity{0.866342}%
\pgfsetlinewidth{1.003750pt}%
\definecolor{currentstroke}{rgb}{0.121569,0.466667,0.705882}%
\pgfsetstrokecolor{currentstroke}%
\pgfsetstrokeopacity{0.866342}%
\pgfsetdash{}{0pt}%
\pgfpathmoveto{\pgfqpoint{1.251539in}{2.127084in}}%
\pgfpathcurveto{\pgfqpoint{1.259775in}{2.127084in}}{\pgfqpoint{1.267676in}{2.130357in}}{\pgfqpoint{1.273499in}{2.136181in}}%
\pgfpathcurveto{\pgfqpoint{1.279323in}{2.142005in}}{\pgfqpoint{1.282596in}{2.149905in}}{\pgfqpoint{1.282596in}{2.158141in}}%
\pgfpathcurveto{\pgfqpoint{1.282596in}{2.166377in}}{\pgfqpoint{1.279323in}{2.174277in}}{\pgfqpoint{1.273499in}{2.180101in}}%
\pgfpathcurveto{\pgfqpoint{1.267676in}{2.185925in}}{\pgfqpoint{1.259775in}{2.189197in}}{\pgfqpoint{1.251539in}{2.189197in}}%
\pgfpathcurveto{\pgfqpoint{1.243303in}{2.189197in}}{\pgfqpoint{1.235403in}{2.185925in}}{\pgfqpoint{1.229579in}{2.180101in}}%
\pgfpathcurveto{\pgfqpoint{1.223755in}{2.174277in}}{\pgfqpoint{1.220483in}{2.166377in}}{\pgfqpoint{1.220483in}{2.158141in}}%
\pgfpathcurveto{\pgfqpoint{1.220483in}{2.149905in}}{\pgfqpoint{1.223755in}{2.142005in}}{\pgfqpoint{1.229579in}{2.136181in}}%
\pgfpathcurveto{\pgfqpoint{1.235403in}{2.130357in}}{\pgfqpoint{1.243303in}{2.127084in}}{\pgfqpoint{1.251539in}{2.127084in}}%
\pgfpathclose%
\pgfusepath{stroke,fill}%
\end{pgfscope}%
\begin{pgfscope}%
\pgfpathrectangle{\pgfqpoint{0.100000in}{0.212622in}}{\pgfqpoint{3.696000in}{3.696000in}}%
\pgfusepath{clip}%
\pgfsetbuttcap%
\pgfsetroundjoin%
\definecolor{currentfill}{rgb}{0.121569,0.466667,0.705882}%
\pgfsetfillcolor{currentfill}%
\pgfsetfillopacity{0.870151}%
\pgfsetlinewidth{1.003750pt}%
\definecolor{currentstroke}{rgb}{0.121569,0.466667,0.705882}%
\pgfsetstrokecolor{currentstroke}%
\pgfsetstrokeopacity{0.870151}%
\pgfsetdash{}{0pt}%
\pgfpathmoveto{\pgfqpoint{1.649101in}{0.986833in}}%
\pgfpathcurveto{\pgfqpoint{1.657338in}{0.986833in}}{\pgfqpoint{1.665238in}{0.990106in}}{\pgfqpoint{1.671062in}{0.995930in}}%
\pgfpathcurveto{\pgfqpoint{1.676886in}{1.001754in}}{\pgfqpoint{1.680158in}{1.009654in}}{\pgfqpoint{1.680158in}{1.017890in}}%
\pgfpathcurveto{\pgfqpoint{1.680158in}{1.026126in}}{\pgfqpoint{1.676886in}{1.034026in}}{\pgfqpoint{1.671062in}{1.039850in}}%
\pgfpathcurveto{\pgfqpoint{1.665238in}{1.045674in}}{\pgfqpoint{1.657338in}{1.048946in}}{\pgfqpoint{1.649101in}{1.048946in}}%
\pgfpathcurveto{\pgfqpoint{1.640865in}{1.048946in}}{\pgfqpoint{1.632965in}{1.045674in}}{\pgfqpoint{1.627141in}{1.039850in}}%
\pgfpathcurveto{\pgfqpoint{1.621317in}{1.034026in}}{\pgfqpoint{1.618045in}{1.026126in}}{\pgfqpoint{1.618045in}{1.017890in}}%
\pgfpathcurveto{\pgfqpoint{1.618045in}{1.009654in}}{\pgfqpoint{1.621317in}{1.001754in}}{\pgfqpoint{1.627141in}{0.995930in}}%
\pgfpathcurveto{\pgfqpoint{1.632965in}{0.990106in}}{\pgfqpoint{1.640865in}{0.986833in}}{\pgfqpoint{1.649101in}{0.986833in}}%
\pgfpathclose%
\pgfusepath{stroke,fill}%
\end{pgfscope}%
\begin{pgfscope}%
\pgfpathrectangle{\pgfqpoint{0.100000in}{0.212622in}}{\pgfqpoint{3.696000in}{3.696000in}}%
\pgfusepath{clip}%
\pgfsetbuttcap%
\pgfsetroundjoin%
\definecolor{currentfill}{rgb}{0.121569,0.466667,0.705882}%
\pgfsetfillcolor{currentfill}%
\pgfsetfillopacity{0.871797}%
\pgfsetlinewidth{1.003750pt}%
\definecolor{currentstroke}{rgb}{0.121569,0.466667,0.705882}%
\pgfsetstrokecolor{currentstroke}%
\pgfsetstrokeopacity{0.871797}%
\pgfsetdash{}{0pt}%
\pgfpathmoveto{\pgfqpoint{1.243563in}{2.123206in}}%
\pgfpathcurveto{\pgfqpoint{1.251800in}{2.123206in}}{\pgfqpoint{1.259700in}{2.126479in}}{\pgfqpoint{1.265524in}{2.132303in}}%
\pgfpathcurveto{\pgfqpoint{1.271348in}{2.138126in}}{\pgfqpoint{1.274620in}{2.146026in}}{\pgfqpoint{1.274620in}{2.154263in}}%
\pgfpathcurveto{\pgfqpoint{1.274620in}{2.162499in}}{\pgfqpoint{1.271348in}{2.170399in}}{\pgfqpoint{1.265524in}{2.176223in}}%
\pgfpathcurveto{\pgfqpoint{1.259700in}{2.182047in}}{\pgfqpoint{1.251800in}{2.185319in}}{\pgfqpoint{1.243563in}{2.185319in}}%
\pgfpathcurveto{\pgfqpoint{1.235327in}{2.185319in}}{\pgfqpoint{1.227427in}{2.182047in}}{\pgfqpoint{1.221603in}{2.176223in}}%
\pgfpathcurveto{\pgfqpoint{1.215779in}{2.170399in}}{\pgfqpoint{1.212507in}{2.162499in}}{\pgfqpoint{1.212507in}{2.154263in}}%
\pgfpathcurveto{\pgfqpoint{1.212507in}{2.146026in}}{\pgfqpoint{1.215779in}{2.138126in}}{\pgfqpoint{1.221603in}{2.132303in}}%
\pgfpathcurveto{\pgfqpoint{1.227427in}{2.126479in}}{\pgfqpoint{1.235327in}{2.123206in}}{\pgfqpoint{1.243563in}{2.123206in}}%
\pgfpathclose%
\pgfusepath{stroke,fill}%
\end{pgfscope}%
\begin{pgfscope}%
\pgfpathrectangle{\pgfqpoint{0.100000in}{0.212622in}}{\pgfqpoint{3.696000in}{3.696000in}}%
\pgfusepath{clip}%
\pgfsetbuttcap%
\pgfsetroundjoin%
\definecolor{currentfill}{rgb}{0.121569,0.466667,0.705882}%
\pgfsetfillcolor{currentfill}%
\pgfsetfillopacity{0.873469}%
\pgfsetlinewidth{1.003750pt}%
\definecolor{currentstroke}{rgb}{0.121569,0.466667,0.705882}%
\pgfsetstrokecolor{currentstroke}%
\pgfsetstrokeopacity{0.873469}%
\pgfsetdash{}{0pt}%
\pgfpathmoveto{\pgfqpoint{2.571599in}{1.335527in}}%
\pgfpathcurveto{\pgfqpoint{2.579835in}{1.335527in}}{\pgfqpoint{2.587735in}{1.338799in}}{\pgfqpoint{2.593559in}{1.344623in}}%
\pgfpathcurveto{\pgfqpoint{2.599383in}{1.350447in}}{\pgfqpoint{2.602655in}{1.358347in}}{\pgfqpoint{2.602655in}{1.366584in}}%
\pgfpathcurveto{\pgfqpoint{2.602655in}{1.374820in}}{\pgfqpoint{2.599383in}{1.382720in}}{\pgfqpoint{2.593559in}{1.388544in}}%
\pgfpathcurveto{\pgfqpoint{2.587735in}{1.394368in}}{\pgfqpoint{2.579835in}{1.397640in}}{\pgfqpoint{2.571599in}{1.397640in}}%
\pgfpathcurveto{\pgfqpoint{2.563362in}{1.397640in}}{\pgfqpoint{2.555462in}{1.394368in}}{\pgfqpoint{2.549638in}{1.388544in}}%
\pgfpathcurveto{\pgfqpoint{2.543815in}{1.382720in}}{\pgfqpoint{2.540542in}{1.374820in}}{\pgfqpoint{2.540542in}{1.366584in}}%
\pgfpathcurveto{\pgfqpoint{2.540542in}{1.358347in}}{\pgfqpoint{2.543815in}{1.350447in}}{\pgfqpoint{2.549638in}{1.344623in}}%
\pgfpathcurveto{\pgfqpoint{2.555462in}{1.338799in}}{\pgfqpoint{2.563362in}{1.335527in}}{\pgfqpoint{2.571599in}{1.335527in}}%
\pgfpathclose%
\pgfusepath{stroke,fill}%
\end{pgfscope}%
\begin{pgfscope}%
\pgfpathrectangle{\pgfqpoint{0.100000in}{0.212622in}}{\pgfqpoint{3.696000in}{3.696000in}}%
\pgfusepath{clip}%
\pgfsetbuttcap%
\pgfsetroundjoin%
\definecolor{currentfill}{rgb}{0.121569,0.466667,0.705882}%
\pgfsetfillcolor{currentfill}%
\pgfsetfillopacity{0.874439}%
\pgfsetlinewidth{1.003750pt}%
\definecolor{currentstroke}{rgb}{0.121569,0.466667,0.705882}%
\pgfsetstrokecolor{currentstroke}%
\pgfsetstrokeopacity{0.874439}%
\pgfsetdash{}{0pt}%
\pgfpathmoveto{\pgfqpoint{1.669881in}{0.983646in}}%
\pgfpathcurveto{\pgfqpoint{1.678117in}{0.983646in}}{\pgfqpoint{1.686017in}{0.986919in}}{\pgfqpoint{1.691841in}{0.992743in}}%
\pgfpathcurveto{\pgfqpoint{1.697665in}{0.998566in}}{\pgfqpoint{1.700937in}{1.006467in}}{\pgfqpoint{1.700937in}{1.014703in}}%
\pgfpathcurveto{\pgfqpoint{1.700937in}{1.022939in}}{\pgfqpoint{1.697665in}{1.030839in}}{\pgfqpoint{1.691841in}{1.036663in}}%
\pgfpathcurveto{\pgfqpoint{1.686017in}{1.042487in}}{\pgfqpoint{1.678117in}{1.045759in}}{\pgfqpoint{1.669881in}{1.045759in}}%
\pgfpathcurveto{\pgfqpoint{1.661644in}{1.045759in}}{\pgfqpoint{1.653744in}{1.042487in}}{\pgfqpoint{1.647920in}{1.036663in}}%
\pgfpathcurveto{\pgfqpoint{1.642096in}{1.030839in}}{\pgfqpoint{1.638824in}{1.022939in}}{\pgfqpoint{1.638824in}{1.014703in}}%
\pgfpathcurveto{\pgfqpoint{1.638824in}{1.006467in}}{\pgfqpoint{1.642096in}{0.998566in}}{\pgfqpoint{1.647920in}{0.992743in}}%
\pgfpathcurveto{\pgfqpoint{1.653744in}{0.986919in}}{\pgfqpoint{1.661644in}{0.983646in}}{\pgfqpoint{1.669881in}{0.983646in}}%
\pgfpathclose%
\pgfusepath{stroke,fill}%
\end{pgfscope}%
\begin{pgfscope}%
\pgfpathrectangle{\pgfqpoint{0.100000in}{0.212622in}}{\pgfqpoint{3.696000in}{3.696000in}}%
\pgfusepath{clip}%
\pgfsetbuttcap%
\pgfsetroundjoin%
\definecolor{currentfill}{rgb}{0.121569,0.466667,0.705882}%
\pgfsetfillcolor{currentfill}%
\pgfsetfillopacity{0.874667}%
\pgfsetlinewidth{1.003750pt}%
\definecolor{currentstroke}{rgb}{0.121569,0.466667,0.705882}%
\pgfsetstrokecolor{currentstroke}%
\pgfsetstrokeopacity{0.874667}%
\pgfsetdash{}{0pt}%
\pgfpathmoveto{\pgfqpoint{1.239117in}{2.119546in}}%
\pgfpathcurveto{\pgfqpoint{1.247353in}{2.119546in}}{\pgfqpoint{1.255253in}{2.122819in}}{\pgfqpoint{1.261077in}{2.128642in}}%
\pgfpathcurveto{\pgfqpoint{1.266901in}{2.134466in}}{\pgfqpoint{1.270173in}{2.142366in}}{\pgfqpoint{1.270173in}{2.150603in}}%
\pgfpathcurveto{\pgfqpoint{1.270173in}{2.158839in}}{\pgfqpoint{1.266901in}{2.166739in}}{\pgfqpoint{1.261077in}{2.172563in}}%
\pgfpathcurveto{\pgfqpoint{1.255253in}{2.178387in}}{\pgfqpoint{1.247353in}{2.181659in}}{\pgfqpoint{1.239117in}{2.181659in}}%
\pgfpathcurveto{\pgfqpoint{1.230881in}{2.181659in}}{\pgfqpoint{1.222981in}{2.178387in}}{\pgfqpoint{1.217157in}{2.172563in}}%
\pgfpathcurveto{\pgfqpoint{1.211333in}{2.166739in}}{\pgfqpoint{1.208060in}{2.158839in}}{\pgfqpoint{1.208060in}{2.150603in}}%
\pgfpathcurveto{\pgfqpoint{1.208060in}{2.142366in}}{\pgfqpoint{1.211333in}{2.134466in}}{\pgfqpoint{1.217157in}{2.128642in}}%
\pgfpathcurveto{\pgfqpoint{1.222981in}{2.122819in}}{\pgfqpoint{1.230881in}{2.119546in}}{\pgfqpoint{1.239117in}{2.119546in}}%
\pgfpathclose%
\pgfusepath{stroke,fill}%
\end{pgfscope}%
\begin{pgfscope}%
\pgfpathrectangle{\pgfqpoint{0.100000in}{0.212622in}}{\pgfqpoint{3.696000in}{3.696000in}}%
\pgfusepath{clip}%
\pgfsetbuttcap%
\pgfsetroundjoin%
\definecolor{currentfill}{rgb}{0.121569,0.466667,0.705882}%
\pgfsetfillcolor{currentfill}%
\pgfsetfillopacity{0.877864}%
\pgfsetlinewidth{1.003750pt}%
\definecolor{currentstroke}{rgb}{0.121569,0.466667,0.705882}%
\pgfsetstrokecolor{currentstroke}%
\pgfsetstrokeopacity{0.877864}%
\pgfsetdash{}{0pt}%
\pgfpathmoveto{\pgfqpoint{1.685858in}{0.981214in}}%
\pgfpathcurveto{\pgfqpoint{1.694095in}{0.981214in}}{\pgfqpoint{1.701995in}{0.984487in}}{\pgfqpoint{1.707819in}{0.990311in}}%
\pgfpathcurveto{\pgfqpoint{1.713642in}{0.996135in}}{\pgfqpoint{1.716915in}{1.004035in}}{\pgfqpoint{1.716915in}{1.012271in}}%
\pgfpathcurveto{\pgfqpoint{1.716915in}{1.020507in}}{\pgfqpoint{1.713642in}{1.028407in}}{\pgfqpoint{1.707819in}{1.034231in}}%
\pgfpathcurveto{\pgfqpoint{1.701995in}{1.040055in}}{\pgfqpoint{1.694095in}{1.043327in}}{\pgfqpoint{1.685858in}{1.043327in}}%
\pgfpathcurveto{\pgfqpoint{1.677622in}{1.043327in}}{\pgfqpoint{1.669722in}{1.040055in}}{\pgfqpoint{1.663898in}{1.034231in}}%
\pgfpathcurveto{\pgfqpoint{1.658074in}{1.028407in}}{\pgfqpoint{1.654802in}{1.020507in}}{\pgfqpoint{1.654802in}{1.012271in}}%
\pgfpathcurveto{\pgfqpoint{1.654802in}{1.004035in}}{\pgfqpoint{1.658074in}{0.996135in}}{\pgfqpoint{1.663898in}{0.990311in}}%
\pgfpathcurveto{\pgfqpoint{1.669722in}{0.984487in}}{\pgfqpoint{1.677622in}{0.981214in}}{\pgfqpoint{1.685858in}{0.981214in}}%
\pgfpathclose%
\pgfusepath{stroke,fill}%
\end{pgfscope}%
\begin{pgfscope}%
\pgfpathrectangle{\pgfqpoint{0.100000in}{0.212622in}}{\pgfqpoint{3.696000in}{3.696000in}}%
\pgfusepath{clip}%
\pgfsetbuttcap%
\pgfsetroundjoin%
\definecolor{currentfill}{rgb}{0.121569,0.466667,0.705882}%
\pgfsetfillcolor{currentfill}%
\pgfsetfillopacity{0.880711}%
\pgfsetlinewidth{1.003750pt}%
\definecolor{currentstroke}{rgb}{0.121569,0.466667,0.705882}%
\pgfsetstrokecolor{currentstroke}%
\pgfsetstrokeopacity{0.880711}%
\pgfsetdash{}{0pt}%
\pgfpathmoveto{\pgfqpoint{1.699798in}{0.978977in}}%
\pgfpathcurveto{\pgfqpoint{1.708035in}{0.978977in}}{\pgfqpoint{1.715935in}{0.982249in}}{\pgfqpoint{1.721758in}{0.988073in}}%
\pgfpathcurveto{\pgfqpoint{1.727582in}{0.993897in}}{\pgfqpoint{1.730855in}{1.001797in}}{\pgfqpoint{1.730855in}{1.010034in}}%
\pgfpathcurveto{\pgfqpoint{1.730855in}{1.018270in}}{\pgfqpoint{1.727582in}{1.026170in}}{\pgfqpoint{1.721758in}{1.031994in}}%
\pgfpathcurveto{\pgfqpoint{1.715935in}{1.037818in}}{\pgfqpoint{1.708035in}{1.041090in}}{\pgfqpoint{1.699798in}{1.041090in}}%
\pgfpathcurveto{\pgfqpoint{1.691562in}{1.041090in}}{\pgfqpoint{1.683662in}{1.037818in}}{\pgfqpoint{1.677838in}{1.031994in}}%
\pgfpathcurveto{\pgfqpoint{1.672014in}{1.026170in}}{\pgfqpoint{1.668742in}{1.018270in}}{\pgfqpoint{1.668742in}{1.010034in}}%
\pgfpathcurveto{\pgfqpoint{1.668742in}{1.001797in}}{\pgfqpoint{1.672014in}{0.993897in}}{\pgfqpoint{1.677838in}{0.988073in}}%
\pgfpathcurveto{\pgfqpoint{1.683662in}{0.982249in}}{\pgfqpoint{1.691562in}{0.978977in}}{\pgfqpoint{1.699798in}{0.978977in}}%
\pgfpathclose%
\pgfusepath{stroke,fill}%
\end{pgfscope}%
\begin{pgfscope}%
\pgfpathrectangle{\pgfqpoint{0.100000in}{0.212622in}}{\pgfqpoint{3.696000in}{3.696000in}}%
\pgfusepath{clip}%
\pgfsetbuttcap%
\pgfsetroundjoin%
\definecolor{currentfill}{rgb}{0.121569,0.466667,0.705882}%
\pgfsetfillcolor{currentfill}%
\pgfsetfillopacity{0.886065}%
\pgfsetlinewidth{1.003750pt}%
\definecolor{currentstroke}{rgb}{0.121569,0.466667,0.705882}%
\pgfsetstrokecolor{currentstroke}%
\pgfsetstrokeopacity{0.886065}%
\pgfsetdash{}{0pt}%
\pgfpathmoveto{\pgfqpoint{1.725097in}{0.975254in}}%
\pgfpathcurveto{\pgfqpoint{1.733334in}{0.975254in}}{\pgfqpoint{1.741234in}{0.978526in}}{\pgfqpoint{1.747058in}{0.984350in}}%
\pgfpathcurveto{\pgfqpoint{1.752882in}{0.990174in}}{\pgfqpoint{1.756154in}{0.998074in}}{\pgfqpoint{1.756154in}{1.006310in}}%
\pgfpathcurveto{\pgfqpoint{1.756154in}{1.014547in}}{\pgfqpoint{1.752882in}{1.022447in}}{\pgfqpoint{1.747058in}{1.028271in}}%
\pgfpathcurveto{\pgfqpoint{1.741234in}{1.034094in}}{\pgfqpoint{1.733334in}{1.037367in}}{\pgfqpoint{1.725097in}{1.037367in}}%
\pgfpathcurveto{\pgfqpoint{1.716861in}{1.037367in}}{\pgfqpoint{1.708961in}{1.034094in}}{\pgfqpoint{1.703137in}{1.028271in}}%
\pgfpathcurveto{\pgfqpoint{1.697313in}{1.022447in}}{\pgfqpoint{1.694041in}{1.014547in}}{\pgfqpoint{1.694041in}{1.006310in}}%
\pgfpathcurveto{\pgfqpoint{1.694041in}{0.998074in}}{\pgfqpoint{1.697313in}{0.990174in}}{\pgfqpoint{1.703137in}{0.984350in}}%
\pgfpathcurveto{\pgfqpoint{1.708961in}{0.978526in}}{\pgfqpoint{1.716861in}{0.975254in}}{\pgfqpoint{1.725097in}{0.975254in}}%
\pgfpathclose%
\pgfusepath{stroke,fill}%
\end{pgfscope}%
\begin{pgfscope}%
\pgfpathrectangle{\pgfqpoint{0.100000in}{0.212622in}}{\pgfqpoint{3.696000in}{3.696000in}}%
\pgfusepath{clip}%
\pgfsetbuttcap%
\pgfsetroundjoin%
\definecolor{currentfill}{rgb}{0.121569,0.466667,0.705882}%
\pgfsetfillcolor{currentfill}%
\pgfsetfillopacity{0.890894}%
\pgfsetlinewidth{1.003750pt}%
\definecolor{currentstroke}{rgb}{0.121569,0.466667,0.705882}%
\pgfsetstrokecolor{currentstroke}%
\pgfsetstrokeopacity{0.890894}%
\pgfsetdash{}{0pt}%
\pgfpathmoveto{\pgfqpoint{2.586999in}{1.274928in}}%
\pgfpathcurveto{\pgfqpoint{2.595235in}{1.274928in}}{\pgfqpoint{2.603135in}{1.278200in}}{\pgfqpoint{2.608959in}{1.284024in}}%
\pgfpathcurveto{\pgfqpoint{2.614783in}{1.289848in}}{\pgfqpoint{2.618055in}{1.297748in}}{\pgfqpoint{2.618055in}{1.305984in}}%
\pgfpathcurveto{\pgfqpoint{2.618055in}{1.314220in}}{\pgfqpoint{2.614783in}{1.322120in}}{\pgfqpoint{2.608959in}{1.327944in}}%
\pgfpathcurveto{\pgfqpoint{2.603135in}{1.333768in}}{\pgfqpoint{2.595235in}{1.337041in}}{\pgfqpoint{2.586999in}{1.337041in}}%
\pgfpathcurveto{\pgfqpoint{2.578762in}{1.337041in}}{\pgfqpoint{2.570862in}{1.333768in}}{\pgfqpoint{2.565038in}{1.327944in}}%
\pgfpathcurveto{\pgfqpoint{2.559215in}{1.322120in}}{\pgfqpoint{2.555942in}{1.314220in}}{\pgfqpoint{2.555942in}{1.305984in}}%
\pgfpathcurveto{\pgfqpoint{2.555942in}{1.297748in}}{\pgfqpoint{2.559215in}{1.289848in}}{\pgfqpoint{2.565038in}{1.284024in}}%
\pgfpathcurveto{\pgfqpoint{2.570862in}{1.278200in}}{\pgfqpoint{2.578762in}{1.274928in}}{\pgfqpoint{2.586999in}{1.274928in}}%
\pgfpathclose%
\pgfusepath{stroke,fill}%
\end{pgfscope}%
\begin{pgfscope}%
\pgfpathrectangle{\pgfqpoint{0.100000in}{0.212622in}}{\pgfqpoint{3.696000in}{3.696000in}}%
\pgfusepath{clip}%
\pgfsetbuttcap%
\pgfsetroundjoin%
\definecolor{currentfill}{rgb}{0.121569,0.466667,0.705882}%
\pgfsetfillcolor{currentfill}%
\pgfsetfillopacity{0.891051}%
\pgfsetlinewidth{1.003750pt}%
\definecolor{currentstroke}{rgb}{0.121569,0.466667,0.705882}%
\pgfsetstrokecolor{currentstroke}%
\pgfsetstrokeopacity{0.891051}%
\pgfsetdash{}{0pt}%
\pgfpathmoveto{\pgfqpoint{1.748794in}{0.971896in}}%
\pgfpathcurveto{\pgfqpoint{1.757030in}{0.971896in}}{\pgfqpoint{1.764930in}{0.975168in}}{\pgfqpoint{1.770754in}{0.980992in}}%
\pgfpathcurveto{\pgfqpoint{1.776578in}{0.986816in}}{\pgfqpoint{1.779850in}{0.994716in}}{\pgfqpoint{1.779850in}{1.002953in}}%
\pgfpathcurveto{\pgfqpoint{1.779850in}{1.011189in}}{\pgfqpoint{1.776578in}{1.019089in}}{\pgfqpoint{1.770754in}{1.024913in}}%
\pgfpathcurveto{\pgfqpoint{1.764930in}{1.030737in}}{\pgfqpoint{1.757030in}{1.034009in}}{\pgfqpoint{1.748794in}{1.034009in}}%
\pgfpathcurveto{\pgfqpoint{1.740557in}{1.034009in}}{\pgfqpoint{1.732657in}{1.030737in}}{\pgfqpoint{1.726833in}{1.024913in}}%
\pgfpathcurveto{\pgfqpoint{1.721010in}{1.019089in}}{\pgfqpoint{1.717737in}{1.011189in}}{\pgfqpoint{1.717737in}{1.002953in}}%
\pgfpathcurveto{\pgfqpoint{1.717737in}{0.994716in}}{\pgfqpoint{1.721010in}{0.986816in}}{\pgfqpoint{1.726833in}{0.980992in}}%
\pgfpathcurveto{\pgfqpoint{1.732657in}{0.975168in}}{\pgfqpoint{1.740557in}{0.971896in}}{\pgfqpoint{1.748794in}{0.971896in}}%
\pgfpathclose%
\pgfusepath{stroke,fill}%
\end{pgfscope}%
\begin{pgfscope}%
\pgfpathrectangle{\pgfqpoint{0.100000in}{0.212622in}}{\pgfqpoint{3.696000in}{3.696000in}}%
\pgfusepath{clip}%
\pgfsetbuttcap%
\pgfsetroundjoin%
\definecolor{currentfill}{rgb}{0.121569,0.466667,0.705882}%
\pgfsetfillcolor{currentfill}%
\pgfsetfillopacity{0.894763}%
\pgfsetlinewidth{1.003750pt}%
\definecolor{currentstroke}{rgb}{0.121569,0.466667,0.705882}%
\pgfsetstrokecolor{currentstroke}%
\pgfsetstrokeopacity{0.894763}%
\pgfsetdash{}{0pt}%
\pgfpathmoveto{\pgfqpoint{1.766900in}{0.969280in}}%
\pgfpathcurveto{\pgfqpoint{1.775137in}{0.969280in}}{\pgfqpoint{1.783037in}{0.972552in}}{\pgfqpoint{1.788861in}{0.978376in}}%
\pgfpathcurveto{\pgfqpoint{1.794685in}{0.984200in}}{\pgfqpoint{1.797957in}{0.992100in}}{\pgfqpoint{1.797957in}{1.000337in}}%
\pgfpathcurveto{\pgfqpoint{1.797957in}{1.008573in}}{\pgfqpoint{1.794685in}{1.016473in}}{\pgfqpoint{1.788861in}{1.022297in}}%
\pgfpathcurveto{\pgfqpoint{1.783037in}{1.028121in}}{\pgfqpoint{1.775137in}{1.031393in}}{\pgfqpoint{1.766900in}{1.031393in}}%
\pgfpathcurveto{\pgfqpoint{1.758664in}{1.031393in}}{\pgfqpoint{1.750764in}{1.028121in}}{\pgfqpoint{1.744940in}{1.022297in}}%
\pgfpathcurveto{\pgfqpoint{1.739116in}{1.016473in}}{\pgfqpoint{1.735844in}{1.008573in}}{\pgfqpoint{1.735844in}{1.000337in}}%
\pgfpathcurveto{\pgfqpoint{1.735844in}{0.992100in}}{\pgfqpoint{1.739116in}{0.984200in}}{\pgfqpoint{1.744940in}{0.978376in}}%
\pgfpathcurveto{\pgfqpoint{1.750764in}{0.972552in}}{\pgfqpoint{1.758664in}{0.969280in}}{\pgfqpoint{1.766900in}{0.969280in}}%
\pgfpathclose%
\pgfusepath{stroke,fill}%
\end{pgfscope}%
\begin{pgfscope}%
\pgfpathrectangle{\pgfqpoint{0.100000in}{0.212622in}}{\pgfqpoint{3.696000in}{3.696000in}}%
\pgfusepath{clip}%
\pgfsetbuttcap%
\pgfsetroundjoin%
\definecolor{currentfill}{rgb}{0.121569,0.466667,0.705882}%
\pgfsetfillcolor{currentfill}%
\pgfsetfillopacity{0.897720}%
\pgfsetlinewidth{1.003750pt}%
\definecolor{currentstroke}{rgb}{0.121569,0.466667,0.705882}%
\pgfsetstrokecolor{currentstroke}%
\pgfsetstrokeopacity{0.897720}%
\pgfsetdash{}{0pt}%
\pgfpathmoveto{\pgfqpoint{1.780739in}{0.967443in}}%
\pgfpathcurveto{\pgfqpoint{1.788975in}{0.967443in}}{\pgfqpoint{1.796875in}{0.970715in}}{\pgfqpoint{1.802699in}{0.976539in}}%
\pgfpathcurveto{\pgfqpoint{1.808523in}{0.982363in}}{\pgfqpoint{1.811795in}{0.990263in}}{\pgfqpoint{1.811795in}{0.998499in}}%
\pgfpathcurveto{\pgfqpoint{1.811795in}{1.006735in}}{\pgfqpoint{1.808523in}{1.014636in}}{\pgfqpoint{1.802699in}{1.020459in}}%
\pgfpathcurveto{\pgfqpoint{1.796875in}{1.026283in}}{\pgfqpoint{1.788975in}{1.029556in}}{\pgfqpoint{1.780739in}{1.029556in}}%
\pgfpathcurveto{\pgfqpoint{1.772502in}{1.029556in}}{\pgfqpoint{1.764602in}{1.026283in}}{\pgfqpoint{1.758778in}{1.020459in}}%
\pgfpathcurveto{\pgfqpoint{1.752955in}{1.014636in}}{\pgfqpoint{1.749682in}{1.006735in}}{\pgfqpoint{1.749682in}{0.998499in}}%
\pgfpathcurveto{\pgfqpoint{1.749682in}{0.990263in}}{\pgfqpoint{1.752955in}{0.982363in}}{\pgfqpoint{1.758778in}{0.976539in}}%
\pgfpathcurveto{\pgfqpoint{1.764602in}{0.970715in}}{\pgfqpoint{1.772502in}{0.967443in}}{\pgfqpoint{1.780739in}{0.967443in}}%
\pgfpathclose%
\pgfusepath{stroke,fill}%
\end{pgfscope}%
\begin{pgfscope}%
\pgfpathrectangle{\pgfqpoint{0.100000in}{0.212622in}}{\pgfqpoint{3.696000in}{3.696000in}}%
\pgfusepath{clip}%
\pgfsetbuttcap%
\pgfsetroundjoin%
\definecolor{currentfill}{rgb}{0.121569,0.466667,0.705882}%
\pgfsetfillcolor{currentfill}%
\pgfsetfillopacity{0.900005}%
\pgfsetlinewidth{1.003750pt}%
\definecolor{currentstroke}{rgb}{0.121569,0.466667,0.705882}%
\pgfsetstrokecolor{currentstroke}%
\pgfsetstrokeopacity{0.900005}%
\pgfsetdash{}{0pt}%
\pgfpathmoveto{\pgfqpoint{1.792592in}{0.965811in}}%
\pgfpathcurveto{\pgfqpoint{1.800828in}{0.965811in}}{\pgfqpoint{1.808728in}{0.969084in}}{\pgfqpoint{1.814552in}{0.974908in}}%
\pgfpathcurveto{\pgfqpoint{1.820376in}{0.980732in}}{\pgfqpoint{1.823649in}{0.988632in}}{\pgfqpoint{1.823649in}{0.996868in}}%
\pgfpathcurveto{\pgfqpoint{1.823649in}{1.005104in}}{\pgfqpoint{1.820376in}{1.013004in}}{\pgfqpoint{1.814552in}{1.018828in}}%
\pgfpathcurveto{\pgfqpoint{1.808728in}{1.024652in}}{\pgfqpoint{1.800828in}{1.027924in}}{\pgfqpoint{1.792592in}{1.027924in}}%
\pgfpathcurveto{\pgfqpoint{1.784356in}{1.027924in}}{\pgfqpoint{1.776456in}{1.024652in}}{\pgfqpoint{1.770632in}{1.018828in}}%
\pgfpathcurveto{\pgfqpoint{1.764808in}{1.013004in}}{\pgfqpoint{1.761536in}{1.005104in}}{\pgfqpoint{1.761536in}{0.996868in}}%
\pgfpathcurveto{\pgfqpoint{1.761536in}{0.988632in}}{\pgfqpoint{1.764808in}{0.980732in}}{\pgfqpoint{1.770632in}{0.974908in}}%
\pgfpathcurveto{\pgfqpoint{1.776456in}{0.969084in}}{\pgfqpoint{1.784356in}{0.965811in}}{\pgfqpoint{1.792592in}{0.965811in}}%
\pgfpathclose%
\pgfusepath{stroke,fill}%
\end{pgfscope}%
\begin{pgfscope}%
\pgfpathrectangle{\pgfqpoint{0.100000in}{0.212622in}}{\pgfqpoint{3.696000in}{3.696000in}}%
\pgfusepath{clip}%
\pgfsetbuttcap%
\pgfsetroundjoin%
\definecolor{currentfill}{rgb}{0.121569,0.466667,0.705882}%
\pgfsetfillcolor{currentfill}%
\pgfsetfillopacity{0.900880}%
\pgfsetlinewidth{1.003750pt}%
\definecolor{currentstroke}{rgb}{0.121569,0.466667,0.705882}%
\pgfsetstrokecolor{currentstroke}%
\pgfsetstrokeopacity{0.900880}%
\pgfsetdash{}{0pt}%
\pgfpathmoveto{\pgfqpoint{2.594434in}{1.242555in}}%
\pgfpathcurveto{\pgfqpoint{2.602670in}{1.242555in}}{\pgfqpoint{2.610571in}{1.245827in}}{\pgfqpoint{2.616394in}{1.251651in}}%
\pgfpathcurveto{\pgfqpoint{2.622218in}{1.257475in}}{\pgfqpoint{2.625491in}{1.265375in}}{\pgfqpoint{2.625491in}{1.273611in}}%
\pgfpathcurveto{\pgfqpoint{2.625491in}{1.281847in}}{\pgfqpoint{2.622218in}{1.289747in}}{\pgfqpoint{2.616394in}{1.295571in}}%
\pgfpathcurveto{\pgfqpoint{2.610571in}{1.301395in}}{\pgfqpoint{2.602670in}{1.304668in}}{\pgfqpoint{2.594434in}{1.304668in}}%
\pgfpathcurveto{\pgfqpoint{2.586198in}{1.304668in}}{\pgfqpoint{2.578298in}{1.301395in}}{\pgfqpoint{2.572474in}{1.295571in}}%
\pgfpathcurveto{\pgfqpoint{2.566650in}{1.289747in}}{\pgfqpoint{2.563378in}{1.281847in}}{\pgfqpoint{2.563378in}{1.273611in}}%
\pgfpathcurveto{\pgfqpoint{2.563378in}{1.265375in}}{\pgfqpoint{2.566650in}{1.257475in}}{\pgfqpoint{2.572474in}{1.251651in}}%
\pgfpathcurveto{\pgfqpoint{2.578298in}{1.245827in}}{\pgfqpoint{2.586198in}{1.242555in}}{\pgfqpoint{2.594434in}{1.242555in}}%
\pgfpathclose%
\pgfusepath{stroke,fill}%
\end{pgfscope}%
\begin{pgfscope}%
\pgfpathrectangle{\pgfqpoint{0.100000in}{0.212622in}}{\pgfqpoint{3.696000in}{3.696000in}}%
\pgfusepath{clip}%
\pgfsetbuttcap%
\pgfsetroundjoin%
\definecolor{currentfill}{rgb}{0.121569,0.466667,0.705882}%
\pgfsetfillcolor{currentfill}%
\pgfsetfillopacity{0.904325}%
\pgfsetlinewidth{1.003750pt}%
\definecolor{currentstroke}{rgb}{0.121569,0.466667,0.705882}%
\pgfsetstrokecolor{currentstroke}%
\pgfsetstrokeopacity{0.904325}%
\pgfsetdash{}{0pt}%
\pgfpathmoveto{\pgfqpoint{1.814145in}{0.963386in}}%
\pgfpathcurveto{\pgfqpoint{1.822381in}{0.963386in}}{\pgfqpoint{1.830281in}{0.966658in}}{\pgfqpoint{1.836105in}{0.972482in}}%
\pgfpathcurveto{\pgfqpoint{1.841929in}{0.978306in}}{\pgfqpoint{1.845201in}{0.986206in}}{\pgfqpoint{1.845201in}{0.994442in}}%
\pgfpathcurveto{\pgfqpoint{1.845201in}{1.002679in}}{\pgfqpoint{1.841929in}{1.010579in}}{\pgfqpoint{1.836105in}{1.016403in}}%
\pgfpathcurveto{\pgfqpoint{1.830281in}{1.022227in}}{\pgfqpoint{1.822381in}{1.025499in}}{\pgfqpoint{1.814145in}{1.025499in}}%
\pgfpathcurveto{\pgfqpoint{1.805909in}{1.025499in}}{\pgfqpoint{1.798009in}{1.022227in}}{\pgfqpoint{1.792185in}{1.016403in}}%
\pgfpathcurveto{\pgfqpoint{1.786361in}{1.010579in}}{\pgfqpoint{1.783088in}{1.002679in}}{\pgfqpoint{1.783088in}{0.994442in}}%
\pgfpathcurveto{\pgfqpoint{1.783088in}{0.986206in}}{\pgfqpoint{1.786361in}{0.978306in}}{\pgfqpoint{1.792185in}{0.972482in}}%
\pgfpathcurveto{\pgfqpoint{1.798009in}{0.966658in}}{\pgfqpoint{1.805909in}{0.963386in}}{\pgfqpoint{1.814145in}{0.963386in}}%
\pgfpathclose%
\pgfusepath{stroke,fill}%
\end{pgfscope}%
\begin{pgfscope}%
\pgfpathrectangle{\pgfqpoint{0.100000in}{0.212622in}}{\pgfqpoint{3.696000in}{3.696000in}}%
\pgfusepath{clip}%
\pgfsetbuttcap%
\pgfsetroundjoin%
\definecolor{currentfill}{rgb}{0.121569,0.466667,0.705882}%
\pgfsetfillcolor{currentfill}%
\pgfsetfillopacity{0.908257}%
\pgfsetlinewidth{1.003750pt}%
\definecolor{currentstroke}{rgb}{0.121569,0.466667,0.705882}%
\pgfsetstrokecolor{currentstroke}%
\pgfsetstrokeopacity{0.908257}%
\pgfsetdash{}{0pt}%
\pgfpathmoveto{\pgfqpoint{1.833298in}{0.961057in}}%
\pgfpathcurveto{\pgfqpoint{1.841534in}{0.961057in}}{\pgfqpoint{1.849434in}{0.964329in}}{\pgfqpoint{1.855258in}{0.970153in}}%
\pgfpathcurveto{\pgfqpoint{1.861082in}{0.975977in}}{\pgfqpoint{1.864354in}{0.983877in}}{\pgfqpoint{1.864354in}{0.992114in}}%
\pgfpathcurveto{\pgfqpoint{1.864354in}{1.000350in}}{\pgfqpoint{1.861082in}{1.008250in}}{\pgfqpoint{1.855258in}{1.014074in}}%
\pgfpathcurveto{\pgfqpoint{1.849434in}{1.019898in}}{\pgfqpoint{1.841534in}{1.023170in}}{\pgfqpoint{1.833298in}{1.023170in}}%
\pgfpathcurveto{\pgfqpoint{1.825062in}{1.023170in}}{\pgfqpoint{1.817161in}{1.019898in}}{\pgfqpoint{1.811338in}{1.014074in}}%
\pgfpathcurveto{\pgfqpoint{1.805514in}{1.008250in}}{\pgfqpoint{1.802241in}{1.000350in}}{\pgfqpoint{1.802241in}{0.992114in}}%
\pgfpathcurveto{\pgfqpoint{1.802241in}{0.983877in}}{\pgfqpoint{1.805514in}{0.975977in}}{\pgfqpoint{1.811338in}{0.970153in}}%
\pgfpathcurveto{\pgfqpoint{1.817161in}{0.964329in}}{\pgfqpoint{1.825062in}{0.961057in}}{\pgfqpoint{1.833298in}{0.961057in}}%
\pgfpathclose%
\pgfusepath{stroke,fill}%
\end{pgfscope}%
\begin{pgfscope}%
\pgfpathrectangle{\pgfqpoint{0.100000in}{0.212622in}}{\pgfqpoint{3.696000in}{3.696000in}}%
\pgfusepath{clip}%
\pgfsetbuttcap%
\pgfsetroundjoin%
\definecolor{currentfill}{rgb}{0.121569,0.466667,0.705882}%
\pgfsetfillcolor{currentfill}%
\pgfsetfillopacity{0.911366}%
\pgfsetlinewidth{1.003750pt}%
\definecolor{currentstroke}{rgb}{0.121569,0.466667,0.705882}%
\pgfsetstrokecolor{currentstroke}%
\pgfsetstrokeopacity{0.911366}%
\pgfsetdash{}{0pt}%
\pgfpathmoveto{\pgfqpoint{1.849164in}{0.959103in}}%
\pgfpathcurveto{\pgfqpoint{1.857400in}{0.959103in}}{\pgfqpoint{1.865300in}{0.962376in}}{\pgfqpoint{1.871124in}{0.968199in}}%
\pgfpathcurveto{\pgfqpoint{1.876948in}{0.974023in}}{\pgfqpoint{1.880221in}{0.981923in}}{\pgfqpoint{1.880221in}{0.990160in}}%
\pgfpathcurveto{\pgfqpoint{1.880221in}{0.998396in}}{\pgfqpoint{1.876948in}{1.006296in}}{\pgfqpoint{1.871124in}{1.012120in}}%
\pgfpathcurveto{\pgfqpoint{1.865300in}{1.017944in}}{\pgfqpoint{1.857400in}{1.021216in}}{\pgfqpoint{1.849164in}{1.021216in}}%
\pgfpathcurveto{\pgfqpoint{1.840928in}{1.021216in}}{\pgfqpoint{1.833028in}{1.017944in}}{\pgfqpoint{1.827204in}{1.012120in}}%
\pgfpathcurveto{\pgfqpoint{1.821380in}{1.006296in}}{\pgfqpoint{1.818108in}{0.998396in}}{\pgfqpoint{1.818108in}{0.990160in}}%
\pgfpathcurveto{\pgfqpoint{1.818108in}{0.981923in}}{\pgfqpoint{1.821380in}{0.974023in}}{\pgfqpoint{1.827204in}{0.968199in}}%
\pgfpathcurveto{\pgfqpoint{1.833028in}{0.962376in}}{\pgfqpoint{1.840928in}{0.959103in}}{\pgfqpoint{1.849164in}{0.959103in}}%
\pgfpathclose%
\pgfusepath{stroke,fill}%
\end{pgfscope}%
\begin{pgfscope}%
\pgfpathrectangle{\pgfqpoint{0.100000in}{0.212622in}}{\pgfqpoint{3.696000in}{3.696000in}}%
\pgfusepath{clip}%
\pgfsetbuttcap%
\pgfsetroundjoin%
\definecolor{currentfill}{rgb}{0.121569,0.466667,0.705882}%
\pgfsetfillcolor{currentfill}%
\pgfsetfillopacity{0.912116}%
\pgfsetlinewidth{1.003750pt}%
\definecolor{currentstroke}{rgb}{0.121569,0.466667,0.705882}%
\pgfsetstrokecolor{currentstroke}%
\pgfsetstrokeopacity{0.912116}%
\pgfsetdash{}{0pt}%
\pgfpathmoveto{\pgfqpoint{2.599974in}{1.202556in}}%
\pgfpathcurveto{\pgfqpoint{2.608211in}{1.202556in}}{\pgfqpoint{2.616111in}{1.205828in}}{\pgfqpoint{2.621935in}{1.211652in}}%
\pgfpathcurveto{\pgfqpoint{2.627759in}{1.217476in}}{\pgfqpoint{2.631031in}{1.225376in}}{\pgfqpoint{2.631031in}{1.233612in}}%
\pgfpathcurveto{\pgfqpoint{2.631031in}{1.241848in}}{\pgfqpoint{2.627759in}{1.249748in}}{\pgfqpoint{2.621935in}{1.255572in}}%
\pgfpathcurveto{\pgfqpoint{2.616111in}{1.261396in}}{\pgfqpoint{2.608211in}{1.264669in}}{\pgfqpoint{2.599974in}{1.264669in}}%
\pgfpathcurveto{\pgfqpoint{2.591738in}{1.264669in}}{\pgfqpoint{2.583838in}{1.261396in}}{\pgfqpoint{2.578014in}{1.255572in}}%
\pgfpathcurveto{\pgfqpoint{2.572190in}{1.249748in}}{\pgfqpoint{2.568918in}{1.241848in}}{\pgfqpoint{2.568918in}{1.233612in}}%
\pgfpathcurveto{\pgfqpoint{2.568918in}{1.225376in}}{\pgfqpoint{2.572190in}{1.217476in}}{\pgfqpoint{2.578014in}{1.211652in}}%
\pgfpathcurveto{\pgfqpoint{2.583838in}{1.205828in}}{\pgfqpoint{2.591738in}{1.202556in}}{\pgfqpoint{2.599974in}{1.202556in}}%
\pgfpathclose%
\pgfusepath{stroke,fill}%
\end{pgfscope}%
\begin{pgfscope}%
\pgfpathrectangle{\pgfqpoint{0.100000in}{0.212622in}}{\pgfqpoint{3.696000in}{3.696000in}}%
\pgfusepath{clip}%
\pgfsetbuttcap%
\pgfsetroundjoin%
\definecolor{currentfill}{rgb}{0.121569,0.466667,0.705882}%
\pgfsetfillcolor{currentfill}%
\pgfsetfillopacity{0.913291}%
\pgfsetlinewidth{1.003750pt}%
\definecolor{currentstroke}{rgb}{0.121569,0.466667,0.705882}%
\pgfsetstrokecolor{currentstroke}%
\pgfsetstrokeopacity{0.913291}%
\pgfsetdash{}{0pt}%
\pgfpathmoveto{\pgfqpoint{1.858381in}{0.958132in}}%
\pgfpathcurveto{\pgfqpoint{1.866617in}{0.958132in}}{\pgfqpoint{1.874517in}{0.961405in}}{\pgfqpoint{1.880341in}{0.967229in}}%
\pgfpathcurveto{\pgfqpoint{1.886165in}{0.973052in}}{\pgfqpoint{1.889437in}{0.980953in}}{\pgfqpoint{1.889437in}{0.989189in}}%
\pgfpathcurveto{\pgfqpoint{1.889437in}{0.997425in}}{\pgfqpoint{1.886165in}{1.005325in}}{\pgfqpoint{1.880341in}{1.011149in}}%
\pgfpathcurveto{\pgfqpoint{1.874517in}{1.016973in}}{\pgfqpoint{1.866617in}{1.020245in}}{\pgfqpoint{1.858381in}{1.020245in}}%
\pgfpathcurveto{\pgfqpoint{1.850144in}{1.020245in}}{\pgfqpoint{1.842244in}{1.016973in}}{\pgfqpoint{1.836420in}{1.011149in}}%
\pgfpathcurveto{\pgfqpoint{1.830596in}{1.005325in}}{\pgfqpoint{1.827324in}{0.997425in}}{\pgfqpoint{1.827324in}{0.989189in}}%
\pgfpathcurveto{\pgfqpoint{1.827324in}{0.980953in}}{\pgfqpoint{1.830596in}{0.973052in}}{\pgfqpoint{1.836420in}{0.967229in}}%
\pgfpathcurveto{\pgfqpoint{1.842244in}{0.961405in}}{\pgfqpoint{1.850144in}{0.958132in}}{\pgfqpoint{1.858381in}{0.958132in}}%
\pgfpathclose%
\pgfusepath{stroke,fill}%
\end{pgfscope}%
\begin{pgfscope}%
\pgfpathrectangle{\pgfqpoint{0.100000in}{0.212622in}}{\pgfqpoint{3.696000in}{3.696000in}}%
\pgfusepath{clip}%
\pgfsetbuttcap%
\pgfsetroundjoin%
\definecolor{currentfill}{rgb}{0.121569,0.466667,0.705882}%
\pgfsetfillcolor{currentfill}%
\pgfsetfillopacity{0.914734}%
\pgfsetlinewidth{1.003750pt}%
\definecolor{currentstroke}{rgb}{0.121569,0.466667,0.705882}%
\pgfsetstrokecolor{currentstroke}%
\pgfsetstrokeopacity{0.914734}%
\pgfsetdash{}{0pt}%
\pgfpathmoveto{\pgfqpoint{1.865706in}{0.957254in}}%
\pgfpathcurveto{\pgfqpoint{1.873942in}{0.957254in}}{\pgfqpoint{1.881842in}{0.960527in}}{\pgfqpoint{1.887666in}{0.966351in}}%
\pgfpathcurveto{\pgfqpoint{1.893490in}{0.972174in}}{\pgfqpoint{1.896762in}{0.980075in}}{\pgfqpoint{1.896762in}{0.988311in}}%
\pgfpathcurveto{\pgfqpoint{1.896762in}{0.996547in}}{\pgfqpoint{1.893490in}{1.004447in}}{\pgfqpoint{1.887666in}{1.010271in}}%
\pgfpathcurveto{\pgfqpoint{1.881842in}{1.016095in}}{\pgfqpoint{1.873942in}{1.019367in}}{\pgfqpoint{1.865706in}{1.019367in}}%
\pgfpathcurveto{\pgfqpoint{1.857470in}{1.019367in}}{\pgfqpoint{1.849570in}{1.016095in}}{\pgfqpoint{1.843746in}{1.010271in}}%
\pgfpathcurveto{\pgfqpoint{1.837922in}{1.004447in}}{\pgfqpoint{1.834649in}{0.996547in}}{\pgfqpoint{1.834649in}{0.988311in}}%
\pgfpathcurveto{\pgfqpoint{1.834649in}{0.980075in}}{\pgfqpoint{1.837922in}{0.972174in}}{\pgfqpoint{1.843746in}{0.966351in}}%
\pgfpathcurveto{\pgfqpoint{1.849570in}{0.960527in}}{\pgfqpoint{1.857470in}{0.957254in}}{\pgfqpoint{1.865706in}{0.957254in}}%
\pgfpathclose%
\pgfusepath{stroke,fill}%
\end{pgfscope}%
\begin{pgfscope}%
\pgfpathrectangle{\pgfqpoint{0.100000in}{0.212622in}}{\pgfqpoint{3.696000in}{3.696000in}}%
\pgfusepath{clip}%
\pgfsetbuttcap%
\pgfsetroundjoin%
\definecolor{currentfill}{rgb}{0.121569,0.466667,0.705882}%
\pgfsetfillcolor{currentfill}%
\pgfsetfillopacity{0.917491}%
\pgfsetlinewidth{1.003750pt}%
\definecolor{currentstroke}{rgb}{0.121569,0.466667,0.705882}%
\pgfsetstrokecolor{currentstroke}%
\pgfsetstrokeopacity{0.917491}%
\pgfsetdash{}{0pt}%
\pgfpathmoveto{\pgfqpoint{1.878958in}{0.955792in}}%
\pgfpathcurveto{\pgfqpoint{1.887194in}{0.955792in}}{\pgfqpoint{1.895094in}{0.959064in}}{\pgfqpoint{1.900918in}{0.964888in}}%
\pgfpathcurveto{\pgfqpoint{1.906742in}{0.970712in}}{\pgfqpoint{1.910015in}{0.978612in}}{\pgfqpoint{1.910015in}{0.986848in}}%
\pgfpathcurveto{\pgfqpoint{1.910015in}{0.995085in}}{\pgfqpoint{1.906742in}{1.002985in}}{\pgfqpoint{1.900918in}{1.008809in}}%
\pgfpathcurveto{\pgfqpoint{1.895094in}{1.014632in}}{\pgfqpoint{1.887194in}{1.017905in}}{\pgfqpoint{1.878958in}{1.017905in}}%
\pgfpathcurveto{\pgfqpoint{1.870722in}{1.017905in}}{\pgfqpoint{1.862822in}{1.014632in}}{\pgfqpoint{1.856998in}{1.008809in}}%
\pgfpathcurveto{\pgfqpoint{1.851174in}{1.002985in}}{\pgfqpoint{1.847902in}{0.995085in}}{\pgfqpoint{1.847902in}{0.986848in}}%
\pgfpathcurveto{\pgfqpoint{1.847902in}{0.978612in}}{\pgfqpoint{1.851174in}{0.970712in}}{\pgfqpoint{1.856998in}{0.964888in}}%
\pgfpathcurveto{\pgfqpoint{1.862822in}{0.959064in}}{\pgfqpoint{1.870722in}{0.955792in}}{\pgfqpoint{1.878958in}{0.955792in}}%
\pgfpathclose%
\pgfusepath{stroke,fill}%
\end{pgfscope}%
\begin{pgfscope}%
\pgfpathrectangle{\pgfqpoint{0.100000in}{0.212622in}}{\pgfqpoint{3.696000in}{3.696000in}}%
\pgfusepath{clip}%
\pgfsetbuttcap%
\pgfsetroundjoin%
\definecolor{currentfill}{rgb}{0.121569,0.466667,0.705882}%
\pgfsetfillcolor{currentfill}%
\pgfsetfillopacity{0.922477}%
\pgfsetlinewidth{1.003750pt}%
\definecolor{currentstroke}{rgb}{0.121569,0.466667,0.705882}%
\pgfsetstrokecolor{currentstroke}%
\pgfsetstrokeopacity{0.922477}%
\pgfsetdash{}{0pt}%
\pgfpathmoveto{\pgfqpoint{1.903083in}{0.953065in}}%
\pgfpathcurveto{\pgfqpoint{1.911319in}{0.953065in}}{\pgfqpoint{1.919220in}{0.956338in}}{\pgfqpoint{1.925043in}{0.962161in}}%
\pgfpathcurveto{\pgfqpoint{1.930867in}{0.967985in}}{\pgfqpoint{1.934140in}{0.975885in}}{\pgfqpoint{1.934140in}{0.984122in}}%
\pgfpathcurveto{\pgfqpoint{1.934140in}{0.992358in}}{\pgfqpoint{1.930867in}{1.000258in}}{\pgfqpoint{1.925043in}{1.006082in}}%
\pgfpathcurveto{\pgfqpoint{1.919220in}{1.011906in}}{\pgfqpoint{1.911319in}{1.015178in}}{\pgfqpoint{1.903083in}{1.015178in}}%
\pgfpathcurveto{\pgfqpoint{1.894847in}{1.015178in}}{\pgfqpoint{1.886947in}{1.011906in}}{\pgfqpoint{1.881123in}{1.006082in}}%
\pgfpathcurveto{\pgfqpoint{1.875299in}{1.000258in}}{\pgfqpoint{1.872027in}{0.992358in}}{\pgfqpoint{1.872027in}{0.984122in}}%
\pgfpathcurveto{\pgfqpoint{1.872027in}{0.975885in}}{\pgfqpoint{1.875299in}{0.967985in}}{\pgfqpoint{1.881123in}{0.962161in}}%
\pgfpathcurveto{\pgfqpoint{1.886947in}{0.956338in}}{\pgfqpoint{1.894847in}{0.953065in}}{\pgfqpoint{1.903083in}{0.953065in}}%
\pgfpathclose%
\pgfusepath{stroke,fill}%
\end{pgfscope}%
\begin{pgfscope}%
\pgfpathrectangle{\pgfqpoint{0.100000in}{0.212622in}}{\pgfqpoint{3.696000in}{3.696000in}}%
\pgfusepath{clip}%
\pgfsetbuttcap%
\pgfsetroundjoin%
\definecolor{currentfill}{rgb}{0.121569,0.466667,0.705882}%
\pgfsetfillcolor{currentfill}%
\pgfsetfillopacity{0.925365}%
\pgfsetlinewidth{1.003750pt}%
\definecolor{currentstroke}{rgb}{0.121569,0.466667,0.705882}%
\pgfsetstrokecolor{currentstroke}%
\pgfsetstrokeopacity{0.925365}%
\pgfsetdash{}{0pt}%
\pgfpathmoveto{\pgfqpoint{2.603448in}{1.150612in}}%
\pgfpathcurveto{\pgfqpoint{2.611684in}{1.150612in}}{\pgfqpoint{2.619584in}{1.153884in}}{\pgfqpoint{2.625408in}{1.159708in}}%
\pgfpathcurveto{\pgfqpoint{2.631232in}{1.165532in}}{\pgfqpoint{2.634505in}{1.173432in}}{\pgfqpoint{2.634505in}{1.181668in}}%
\pgfpathcurveto{\pgfqpoint{2.634505in}{1.189904in}}{\pgfqpoint{2.631232in}{1.197804in}}{\pgfqpoint{2.625408in}{1.203628in}}%
\pgfpathcurveto{\pgfqpoint{2.619584in}{1.209452in}}{\pgfqpoint{2.611684in}{1.212725in}}{\pgfqpoint{2.603448in}{1.212725in}}%
\pgfpathcurveto{\pgfqpoint{2.595212in}{1.212725in}}{\pgfqpoint{2.587312in}{1.209452in}}{\pgfqpoint{2.581488in}{1.203628in}}%
\pgfpathcurveto{\pgfqpoint{2.575664in}{1.197804in}}{\pgfqpoint{2.572392in}{1.189904in}}{\pgfqpoint{2.572392in}{1.181668in}}%
\pgfpathcurveto{\pgfqpoint{2.572392in}{1.173432in}}{\pgfqpoint{2.575664in}{1.165532in}}{\pgfqpoint{2.581488in}{1.159708in}}%
\pgfpathcurveto{\pgfqpoint{2.587312in}{1.153884in}}{\pgfqpoint{2.595212in}{1.150612in}}{\pgfqpoint{2.603448in}{1.150612in}}%
\pgfpathclose%
\pgfusepath{stroke,fill}%
\end{pgfscope}%
\begin{pgfscope}%
\pgfpathrectangle{\pgfqpoint{0.100000in}{0.212622in}}{\pgfqpoint{3.696000in}{3.696000in}}%
\pgfusepath{clip}%
\pgfsetbuttcap%
\pgfsetroundjoin%
\definecolor{currentfill}{rgb}{0.121569,0.466667,0.705882}%
\pgfsetfillcolor{currentfill}%
\pgfsetfillopacity{0.926417}%
\pgfsetlinewidth{1.003750pt}%
\definecolor{currentstroke}{rgb}{0.121569,0.466667,0.705882}%
\pgfsetstrokecolor{currentstroke}%
\pgfsetstrokeopacity{0.926417}%
\pgfsetdash{}{0pt}%
\pgfpathmoveto{\pgfqpoint{1.922877in}{0.950774in}}%
\pgfpathcurveto{\pgfqpoint{1.931113in}{0.950774in}}{\pgfqpoint{1.939013in}{0.954047in}}{\pgfqpoint{1.944837in}{0.959871in}}%
\pgfpathcurveto{\pgfqpoint{1.950661in}{0.965695in}}{\pgfqpoint{1.953934in}{0.973595in}}{\pgfqpoint{1.953934in}{0.981831in}}%
\pgfpathcurveto{\pgfqpoint{1.953934in}{0.990067in}}{\pgfqpoint{1.950661in}{0.997967in}}{\pgfqpoint{1.944837in}{1.003791in}}%
\pgfpathcurveto{\pgfqpoint{1.939013in}{1.009615in}}{\pgfqpoint{1.931113in}{1.012887in}}{\pgfqpoint{1.922877in}{1.012887in}}%
\pgfpathcurveto{\pgfqpoint{1.914641in}{1.012887in}}{\pgfqpoint{1.906741in}{1.009615in}}{\pgfqpoint{1.900917in}{1.003791in}}%
\pgfpathcurveto{\pgfqpoint{1.895093in}{0.997967in}}{\pgfqpoint{1.891821in}{0.990067in}}{\pgfqpoint{1.891821in}{0.981831in}}%
\pgfpathcurveto{\pgfqpoint{1.891821in}{0.973595in}}{\pgfqpoint{1.895093in}{0.965695in}}{\pgfqpoint{1.900917in}{0.959871in}}%
\pgfpathcurveto{\pgfqpoint{1.906741in}{0.954047in}}{\pgfqpoint{1.914641in}{0.950774in}}{\pgfqpoint{1.922877in}{0.950774in}}%
\pgfpathclose%
\pgfusepath{stroke,fill}%
\end{pgfscope}%
\begin{pgfscope}%
\pgfpathrectangle{\pgfqpoint{0.100000in}{0.212622in}}{\pgfqpoint{3.696000in}{3.696000in}}%
\pgfusepath{clip}%
\pgfsetbuttcap%
\pgfsetroundjoin%
\definecolor{currentfill}{rgb}{0.121569,0.466667,0.705882}%
\pgfsetfillcolor{currentfill}%
\pgfsetfillopacity{0.929158}%
\pgfsetlinewidth{1.003750pt}%
\definecolor{currentstroke}{rgb}{0.121569,0.466667,0.705882}%
\pgfsetstrokecolor{currentstroke}%
\pgfsetstrokeopacity{0.929158}%
\pgfsetdash{}{0pt}%
\pgfpathmoveto{\pgfqpoint{1.935884in}{0.949352in}}%
\pgfpathcurveto{\pgfqpoint{1.944120in}{0.949352in}}{\pgfqpoint{1.952020in}{0.952625in}}{\pgfqpoint{1.957844in}{0.958449in}}%
\pgfpathcurveto{\pgfqpoint{1.963668in}{0.964273in}}{\pgfqpoint{1.966940in}{0.972173in}}{\pgfqpoint{1.966940in}{0.980409in}}%
\pgfpathcurveto{\pgfqpoint{1.966940in}{0.988645in}}{\pgfqpoint{1.963668in}{0.996545in}}{\pgfqpoint{1.957844in}{1.002369in}}%
\pgfpathcurveto{\pgfqpoint{1.952020in}{1.008193in}}{\pgfqpoint{1.944120in}{1.011465in}}{\pgfqpoint{1.935884in}{1.011465in}}%
\pgfpathcurveto{\pgfqpoint{1.927647in}{1.011465in}}{\pgfqpoint{1.919747in}{1.008193in}}{\pgfqpoint{1.913923in}{1.002369in}}%
\pgfpathcurveto{\pgfqpoint{1.908100in}{0.996545in}}{\pgfqpoint{1.904827in}{0.988645in}}{\pgfqpoint{1.904827in}{0.980409in}}%
\pgfpathcurveto{\pgfqpoint{1.904827in}{0.972173in}}{\pgfqpoint{1.908100in}{0.964273in}}{\pgfqpoint{1.913923in}{0.958449in}}%
\pgfpathcurveto{\pgfqpoint{1.919747in}{0.952625in}}{\pgfqpoint{1.927647in}{0.949352in}}{\pgfqpoint{1.935884in}{0.949352in}}%
\pgfpathclose%
\pgfusepath{stroke,fill}%
\end{pgfscope}%
\begin{pgfscope}%
\pgfpathrectangle{\pgfqpoint{0.100000in}{0.212622in}}{\pgfqpoint{3.696000in}{3.696000in}}%
\pgfusepath{clip}%
\pgfsetbuttcap%
\pgfsetroundjoin%
\definecolor{currentfill}{rgb}{0.121569,0.466667,0.705882}%
\pgfsetfillcolor{currentfill}%
\pgfsetfillopacity{0.930619}%
\pgfsetlinewidth{1.003750pt}%
\definecolor{currentstroke}{rgb}{0.121569,0.466667,0.705882}%
\pgfsetstrokecolor{currentstroke}%
\pgfsetstrokeopacity{0.930619}%
\pgfsetdash{}{0pt}%
\pgfpathmoveto{\pgfqpoint{1.943287in}{0.948497in}}%
\pgfpathcurveto{\pgfqpoint{1.951524in}{0.948497in}}{\pgfqpoint{1.959424in}{0.951769in}}{\pgfqpoint{1.965248in}{0.957593in}}%
\pgfpathcurveto{\pgfqpoint{1.971072in}{0.963417in}}{\pgfqpoint{1.974344in}{0.971317in}}{\pgfqpoint{1.974344in}{0.979553in}}%
\pgfpathcurveto{\pgfqpoint{1.974344in}{0.987790in}}{\pgfqpoint{1.971072in}{0.995690in}}{\pgfqpoint{1.965248in}{1.001514in}}%
\pgfpathcurveto{\pgfqpoint{1.959424in}{1.007337in}}{\pgfqpoint{1.951524in}{1.010610in}}{\pgfqpoint{1.943287in}{1.010610in}}%
\pgfpathcurveto{\pgfqpoint{1.935051in}{1.010610in}}{\pgfqpoint{1.927151in}{1.007337in}}{\pgfqpoint{1.921327in}{1.001514in}}%
\pgfpathcurveto{\pgfqpoint{1.915503in}{0.995690in}}{\pgfqpoint{1.912231in}{0.987790in}}{\pgfqpoint{1.912231in}{0.979553in}}%
\pgfpathcurveto{\pgfqpoint{1.912231in}{0.971317in}}{\pgfqpoint{1.915503in}{0.963417in}}{\pgfqpoint{1.921327in}{0.957593in}}%
\pgfpathcurveto{\pgfqpoint{1.927151in}{0.951769in}}{\pgfqpoint{1.935051in}{0.948497in}}{\pgfqpoint{1.943287in}{0.948497in}}%
\pgfpathclose%
\pgfusepath{stroke,fill}%
\end{pgfscope}%
\begin{pgfscope}%
\pgfpathrectangle{\pgfqpoint{0.100000in}{0.212622in}}{\pgfqpoint{3.696000in}{3.696000in}}%
\pgfusepath{clip}%
\pgfsetbuttcap%
\pgfsetroundjoin%
\definecolor{currentfill}{rgb}{0.121569,0.466667,0.705882}%
\pgfsetfillcolor{currentfill}%
\pgfsetfillopacity{0.933335}%
\pgfsetlinewidth{1.003750pt}%
\definecolor{currentstroke}{rgb}{0.121569,0.466667,0.705882}%
\pgfsetstrokecolor{currentstroke}%
\pgfsetstrokeopacity{0.933335}%
\pgfsetdash{}{0pt}%
\pgfpathmoveto{\pgfqpoint{1.956729in}{0.947029in}}%
\pgfpathcurveto{\pgfqpoint{1.964965in}{0.947029in}}{\pgfqpoint{1.972865in}{0.950301in}}{\pgfqpoint{1.978689in}{0.956125in}}%
\pgfpathcurveto{\pgfqpoint{1.984513in}{0.961949in}}{\pgfqpoint{1.987785in}{0.969849in}}{\pgfqpoint{1.987785in}{0.978085in}}%
\pgfpathcurveto{\pgfqpoint{1.987785in}{0.986322in}}{\pgfqpoint{1.984513in}{0.994222in}}{\pgfqpoint{1.978689in}{1.000046in}}%
\pgfpathcurveto{\pgfqpoint{1.972865in}{1.005870in}}{\pgfqpoint{1.964965in}{1.009142in}}{\pgfqpoint{1.956729in}{1.009142in}}%
\pgfpathcurveto{\pgfqpoint{1.948493in}{1.009142in}}{\pgfqpoint{1.940593in}{1.005870in}}{\pgfqpoint{1.934769in}{1.000046in}}%
\pgfpathcurveto{\pgfqpoint{1.928945in}{0.994222in}}{\pgfqpoint{1.925672in}{0.986322in}}{\pgfqpoint{1.925672in}{0.978085in}}%
\pgfpathcurveto{\pgfqpoint{1.925672in}{0.969849in}}{\pgfqpoint{1.928945in}{0.961949in}}{\pgfqpoint{1.934769in}{0.956125in}}%
\pgfpathcurveto{\pgfqpoint{1.940593in}{0.950301in}}{\pgfqpoint{1.948493in}{0.947029in}}{\pgfqpoint{1.956729in}{0.947029in}}%
\pgfpathclose%
\pgfusepath{stroke,fill}%
\end{pgfscope}%
\begin{pgfscope}%
\pgfpathrectangle{\pgfqpoint{0.100000in}{0.212622in}}{\pgfqpoint{3.696000in}{3.696000in}}%
\pgfusepath{clip}%
\pgfsetbuttcap%
\pgfsetroundjoin%
\definecolor{currentfill}{rgb}{0.121569,0.466667,0.705882}%
\pgfsetfillcolor{currentfill}%
\pgfsetfillopacity{0.938346}%
\pgfsetlinewidth{1.003750pt}%
\definecolor{currentstroke}{rgb}{0.121569,0.466667,0.705882}%
\pgfsetstrokecolor{currentstroke}%
\pgfsetstrokeopacity{0.938346}%
\pgfsetdash{}{0pt}%
\pgfpathmoveto{\pgfqpoint{1.981085in}{0.944099in}}%
\pgfpathcurveto{\pgfqpoint{1.989321in}{0.944099in}}{\pgfqpoint{1.997222in}{0.947371in}}{\pgfqpoint{2.003045in}{0.953195in}}%
\pgfpathcurveto{\pgfqpoint{2.008869in}{0.959019in}}{\pgfqpoint{2.012142in}{0.966919in}}{\pgfqpoint{2.012142in}{0.975155in}}%
\pgfpathcurveto{\pgfqpoint{2.012142in}{0.983392in}}{\pgfqpoint{2.008869in}{0.991292in}}{\pgfqpoint{2.003045in}{0.997116in}}%
\pgfpathcurveto{\pgfqpoint{1.997222in}{1.002940in}}{\pgfqpoint{1.989321in}{1.006212in}}{\pgfqpoint{1.981085in}{1.006212in}}%
\pgfpathcurveto{\pgfqpoint{1.972849in}{1.006212in}}{\pgfqpoint{1.964949in}{1.002940in}}{\pgfqpoint{1.959125in}{0.997116in}}%
\pgfpathcurveto{\pgfqpoint{1.953301in}{0.991292in}}{\pgfqpoint{1.950029in}{0.983392in}}{\pgfqpoint{1.950029in}{0.975155in}}%
\pgfpathcurveto{\pgfqpoint{1.950029in}{0.966919in}}{\pgfqpoint{1.953301in}{0.959019in}}{\pgfqpoint{1.959125in}{0.953195in}}%
\pgfpathcurveto{\pgfqpoint{1.964949in}{0.947371in}}{\pgfqpoint{1.972849in}{0.944099in}}{\pgfqpoint{1.981085in}{0.944099in}}%
\pgfpathclose%
\pgfusepath{stroke,fill}%
\end{pgfscope}%
\begin{pgfscope}%
\pgfpathrectangle{\pgfqpoint{0.100000in}{0.212622in}}{\pgfqpoint{3.696000in}{3.696000in}}%
\pgfusepath{clip}%
\pgfsetbuttcap%
\pgfsetroundjoin%
\definecolor{currentfill}{rgb}{0.121569,0.466667,0.705882}%
\pgfsetfillcolor{currentfill}%
\pgfsetfillopacity{0.940334}%
\pgfsetlinewidth{1.003750pt}%
\definecolor{currentstroke}{rgb}{0.121569,0.466667,0.705882}%
\pgfsetstrokecolor{currentstroke}%
\pgfsetstrokeopacity{0.940334}%
\pgfsetdash{}{0pt}%
\pgfpathmoveto{\pgfqpoint{2.598602in}{1.093734in}}%
\pgfpathcurveto{\pgfqpoint{2.606839in}{1.093734in}}{\pgfqpoint{2.614739in}{1.097007in}}{\pgfqpoint{2.620562in}{1.102831in}}%
\pgfpathcurveto{\pgfqpoint{2.626386in}{1.108655in}}{\pgfqpoint{2.629659in}{1.116555in}}{\pgfqpoint{2.629659in}{1.124791in}}%
\pgfpathcurveto{\pgfqpoint{2.629659in}{1.133027in}}{\pgfqpoint{2.626386in}{1.140927in}}{\pgfqpoint{2.620562in}{1.146751in}}%
\pgfpathcurveto{\pgfqpoint{2.614739in}{1.152575in}}{\pgfqpoint{2.606839in}{1.155847in}}{\pgfqpoint{2.598602in}{1.155847in}}%
\pgfpathcurveto{\pgfqpoint{2.590366in}{1.155847in}}{\pgfqpoint{2.582466in}{1.152575in}}{\pgfqpoint{2.576642in}{1.146751in}}%
\pgfpathcurveto{\pgfqpoint{2.570818in}{1.140927in}}{\pgfqpoint{2.567546in}{1.133027in}}{\pgfqpoint{2.567546in}{1.124791in}}%
\pgfpathcurveto{\pgfqpoint{2.567546in}{1.116555in}}{\pgfqpoint{2.570818in}{1.108655in}}{\pgfqpoint{2.576642in}{1.102831in}}%
\pgfpathcurveto{\pgfqpoint{2.582466in}{1.097007in}}{\pgfqpoint{2.590366in}{1.093734in}}{\pgfqpoint{2.598602in}{1.093734in}}%
\pgfpathclose%
\pgfusepath{stroke,fill}%
\end{pgfscope}%
\begin{pgfscope}%
\pgfpathrectangle{\pgfqpoint{0.100000in}{0.212622in}}{\pgfqpoint{3.696000in}{3.696000in}}%
\pgfusepath{clip}%
\pgfsetbuttcap%
\pgfsetroundjoin%
\definecolor{currentfill}{rgb}{0.121569,0.466667,0.705882}%
\pgfsetfillcolor{currentfill}%
\pgfsetfillopacity{0.942514}%
\pgfsetlinewidth{1.003750pt}%
\definecolor{currentstroke}{rgb}{0.121569,0.466667,0.705882}%
\pgfsetstrokecolor{currentstroke}%
\pgfsetstrokeopacity{0.942514}%
\pgfsetdash{}{0pt}%
\pgfpathmoveto{\pgfqpoint{2.001898in}{0.941600in}}%
\pgfpathcurveto{\pgfqpoint{2.010134in}{0.941600in}}{\pgfqpoint{2.018034in}{0.944872in}}{\pgfqpoint{2.023858in}{0.950696in}}%
\pgfpathcurveto{\pgfqpoint{2.029682in}{0.956520in}}{\pgfqpoint{2.032954in}{0.964420in}}{\pgfqpoint{2.032954in}{0.972656in}}%
\pgfpathcurveto{\pgfqpoint{2.032954in}{0.980892in}}{\pgfqpoint{2.029682in}{0.988792in}}{\pgfqpoint{2.023858in}{0.994616in}}%
\pgfpathcurveto{\pgfqpoint{2.018034in}{1.000440in}}{\pgfqpoint{2.010134in}{1.003713in}}{\pgfqpoint{2.001898in}{1.003713in}}%
\pgfpathcurveto{\pgfqpoint{1.993662in}{1.003713in}}{\pgfqpoint{1.985762in}{1.000440in}}{\pgfqpoint{1.979938in}{0.994616in}}%
\pgfpathcurveto{\pgfqpoint{1.974114in}{0.988792in}}{\pgfqpoint{1.970841in}{0.980892in}}{\pgfqpoint{1.970841in}{0.972656in}}%
\pgfpathcurveto{\pgfqpoint{1.970841in}{0.964420in}}{\pgfqpoint{1.974114in}{0.956520in}}{\pgfqpoint{1.979938in}{0.950696in}}%
\pgfpathcurveto{\pgfqpoint{1.985762in}{0.944872in}}{\pgfqpoint{1.993662in}{0.941600in}}{\pgfqpoint{2.001898in}{0.941600in}}%
\pgfpathclose%
\pgfusepath{stroke,fill}%
\end{pgfscope}%
\begin{pgfscope}%
\pgfpathrectangle{\pgfqpoint{0.100000in}{0.212622in}}{\pgfqpoint{3.696000in}{3.696000in}}%
\pgfusepath{clip}%
\pgfsetbuttcap%
\pgfsetroundjoin%
\definecolor{currentfill}{rgb}{0.121569,0.466667,0.705882}%
\pgfsetfillcolor{currentfill}%
\pgfsetfillopacity{0.946176}%
\pgfsetlinewidth{1.003750pt}%
\definecolor{currentstroke}{rgb}{0.121569,0.466667,0.705882}%
\pgfsetstrokecolor{currentstroke}%
\pgfsetstrokeopacity{0.946176}%
\pgfsetdash{}{0pt}%
\pgfpathmoveto{\pgfqpoint{2.019922in}{0.939348in}}%
\pgfpathcurveto{\pgfqpoint{2.028158in}{0.939348in}}{\pgfqpoint{2.036058in}{0.942621in}}{\pgfqpoint{2.041882in}{0.948445in}}%
\pgfpathcurveto{\pgfqpoint{2.047706in}{0.954269in}}{\pgfqpoint{2.050978in}{0.962169in}}{\pgfqpoint{2.050978in}{0.970405in}}%
\pgfpathcurveto{\pgfqpoint{2.050978in}{0.978641in}}{\pgfqpoint{2.047706in}{0.986541in}}{\pgfqpoint{2.041882in}{0.992365in}}%
\pgfpathcurveto{\pgfqpoint{2.036058in}{0.998189in}}{\pgfqpoint{2.028158in}{1.001461in}}{\pgfqpoint{2.019922in}{1.001461in}}%
\pgfpathcurveto{\pgfqpoint{2.011685in}{1.001461in}}{\pgfqpoint{2.003785in}{0.998189in}}{\pgfqpoint{1.997961in}{0.992365in}}%
\pgfpathcurveto{\pgfqpoint{1.992138in}{0.986541in}}{\pgfqpoint{1.988865in}{0.978641in}}{\pgfqpoint{1.988865in}{0.970405in}}%
\pgfpathcurveto{\pgfqpoint{1.988865in}{0.962169in}}{\pgfqpoint{1.992138in}{0.954269in}}{\pgfqpoint{1.997961in}{0.948445in}}%
\pgfpathcurveto{\pgfqpoint{2.003785in}{0.942621in}}{\pgfqpoint{2.011685in}{0.939348in}}{\pgfqpoint{2.019922in}{0.939348in}}%
\pgfpathclose%
\pgfusepath{stroke,fill}%
\end{pgfscope}%
\begin{pgfscope}%
\pgfpathrectangle{\pgfqpoint{0.100000in}{0.212622in}}{\pgfqpoint{3.696000in}{3.696000in}}%
\pgfusepath{clip}%
\pgfsetbuttcap%
\pgfsetroundjoin%
\definecolor{currentfill}{rgb}{0.121569,0.466667,0.705882}%
\pgfsetfillcolor{currentfill}%
\pgfsetfillopacity{0.948389}%
\pgfsetlinewidth{1.003750pt}%
\definecolor{currentstroke}{rgb}{0.121569,0.466667,0.705882}%
\pgfsetstrokecolor{currentstroke}%
\pgfsetstrokeopacity{0.948389}%
\pgfsetdash{}{0pt}%
\pgfpathmoveto{\pgfqpoint{2.031716in}{0.937834in}}%
\pgfpathcurveto{\pgfqpoint{2.039953in}{0.937834in}}{\pgfqpoint{2.047853in}{0.941107in}}{\pgfqpoint{2.053677in}{0.946931in}}%
\pgfpathcurveto{\pgfqpoint{2.059501in}{0.952755in}}{\pgfqpoint{2.062773in}{0.960655in}}{\pgfqpoint{2.062773in}{0.968891in}}%
\pgfpathcurveto{\pgfqpoint{2.062773in}{0.977127in}}{\pgfqpoint{2.059501in}{0.985027in}}{\pgfqpoint{2.053677in}{0.990851in}}%
\pgfpathcurveto{\pgfqpoint{2.047853in}{0.996675in}}{\pgfqpoint{2.039953in}{0.999947in}}{\pgfqpoint{2.031716in}{0.999947in}}%
\pgfpathcurveto{\pgfqpoint{2.023480in}{0.999947in}}{\pgfqpoint{2.015580in}{0.996675in}}{\pgfqpoint{2.009756in}{0.990851in}}%
\pgfpathcurveto{\pgfqpoint{2.003932in}{0.985027in}}{\pgfqpoint{2.000660in}{0.977127in}}{\pgfqpoint{2.000660in}{0.968891in}}%
\pgfpathcurveto{\pgfqpoint{2.000660in}{0.960655in}}{\pgfqpoint{2.003932in}{0.952755in}}{\pgfqpoint{2.009756in}{0.946931in}}%
\pgfpathcurveto{\pgfqpoint{2.015580in}{0.941107in}}{\pgfqpoint{2.023480in}{0.937834in}}{\pgfqpoint{2.031716in}{0.937834in}}%
\pgfpathclose%
\pgfusepath{stroke,fill}%
\end{pgfscope}%
\begin{pgfscope}%
\pgfpathrectangle{\pgfqpoint{0.100000in}{0.212622in}}{\pgfqpoint{3.696000in}{3.696000in}}%
\pgfusepath{clip}%
\pgfsetbuttcap%
\pgfsetroundjoin%
\definecolor{currentfill}{rgb}{0.121569,0.466667,0.705882}%
\pgfsetfillcolor{currentfill}%
\pgfsetfillopacity{0.948711}%
\pgfsetlinewidth{1.003750pt}%
\definecolor{currentstroke}{rgb}{0.121569,0.466667,0.705882}%
\pgfsetstrokecolor{currentstroke}%
\pgfsetstrokeopacity{0.948711}%
\pgfsetdash{}{0pt}%
\pgfpathmoveto{\pgfqpoint{2.592830in}{1.064056in}}%
\pgfpathcurveto{\pgfqpoint{2.601066in}{1.064056in}}{\pgfqpoint{2.608966in}{1.067328in}}{\pgfqpoint{2.614790in}{1.073152in}}%
\pgfpathcurveto{\pgfqpoint{2.620614in}{1.078976in}}{\pgfqpoint{2.623887in}{1.086876in}}{\pgfqpoint{2.623887in}{1.095113in}}%
\pgfpathcurveto{\pgfqpoint{2.623887in}{1.103349in}}{\pgfqpoint{2.620614in}{1.111249in}}{\pgfqpoint{2.614790in}{1.117073in}}%
\pgfpathcurveto{\pgfqpoint{2.608966in}{1.122897in}}{\pgfqpoint{2.601066in}{1.126169in}}{\pgfqpoint{2.592830in}{1.126169in}}%
\pgfpathcurveto{\pgfqpoint{2.584594in}{1.126169in}}{\pgfqpoint{2.576694in}{1.122897in}}{\pgfqpoint{2.570870in}{1.117073in}}%
\pgfpathcurveto{\pgfqpoint{2.565046in}{1.111249in}}{\pgfqpoint{2.561774in}{1.103349in}}{\pgfqpoint{2.561774in}{1.095113in}}%
\pgfpathcurveto{\pgfqpoint{2.561774in}{1.086876in}}{\pgfqpoint{2.565046in}{1.078976in}}{\pgfqpoint{2.570870in}{1.073152in}}%
\pgfpathcurveto{\pgfqpoint{2.576694in}{1.067328in}}{\pgfqpoint{2.584594in}{1.064056in}}{\pgfqpoint{2.592830in}{1.064056in}}%
\pgfpathclose%
\pgfusepath{stroke,fill}%
\end{pgfscope}%
\begin{pgfscope}%
\pgfpathrectangle{\pgfqpoint{0.100000in}{0.212622in}}{\pgfqpoint{3.696000in}{3.696000in}}%
\pgfusepath{clip}%
\pgfsetbuttcap%
\pgfsetroundjoin%
\definecolor{currentfill}{rgb}{0.121569,0.466667,0.705882}%
\pgfsetfillcolor{currentfill}%
\pgfsetfillopacity{0.949297}%
\pgfsetlinewidth{1.003750pt}%
\definecolor{currentstroke}{rgb}{0.121569,0.466667,0.705882}%
\pgfsetstrokecolor{currentstroke}%
\pgfsetstrokeopacity{0.949297}%
\pgfsetdash{}{0pt}%
\pgfpathmoveto{\pgfqpoint{2.036241in}{0.937294in}}%
\pgfpathcurveto{\pgfqpoint{2.044477in}{0.937294in}}{\pgfqpoint{2.052377in}{0.940567in}}{\pgfqpoint{2.058201in}{0.946391in}}%
\pgfpathcurveto{\pgfqpoint{2.064025in}{0.952215in}}{\pgfqpoint{2.067298in}{0.960115in}}{\pgfqpoint{2.067298in}{0.968351in}}%
\pgfpathcurveto{\pgfqpoint{2.067298in}{0.976587in}}{\pgfqpoint{2.064025in}{0.984487in}}{\pgfqpoint{2.058201in}{0.990311in}}%
\pgfpathcurveto{\pgfqpoint{2.052377in}{0.996135in}}{\pgfqpoint{2.044477in}{0.999407in}}{\pgfqpoint{2.036241in}{0.999407in}}%
\pgfpathcurveto{\pgfqpoint{2.028005in}{0.999407in}}{\pgfqpoint{2.020105in}{0.996135in}}{\pgfqpoint{2.014281in}{0.990311in}}%
\pgfpathcurveto{\pgfqpoint{2.008457in}{0.984487in}}{\pgfqpoint{2.005185in}{0.976587in}}{\pgfqpoint{2.005185in}{0.968351in}}%
\pgfpathcurveto{\pgfqpoint{2.005185in}{0.960115in}}{\pgfqpoint{2.008457in}{0.952215in}}{\pgfqpoint{2.014281in}{0.946391in}}%
\pgfpathcurveto{\pgfqpoint{2.020105in}{0.940567in}}{\pgfqpoint{2.028005in}{0.937294in}}{\pgfqpoint{2.036241in}{0.937294in}}%
\pgfpathclose%
\pgfusepath{stroke,fill}%
\end{pgfscope}%
\begin{pgfscope}%
\pgfpathrectangle{\pgfqpoint{0.100000in}{0.212622in}}{\pgfqpoint{3.696000in}{3.696000in}}%
\pgfusepath{clip}%
\pgfsetbuttcap%
\pgfsetroundjoin%
\definecolor{currentfill}{rgb}{0.121569,0.466667,0.705882}%
\pgfsetfillcolor{currentfill}%
\pgfsetfillopacity{0.950965}%
\pgfsetlinewidth{1.003750pt}%
\definecolor{currentstroke}{rgb}{0.121569,0.466667,0.705882}%
\pgfsetstrokecolor{currentstroke}%
\pgfsetstrokeopacity{0.950965}%
\pgfsetdash{}{0pt}%
\pgfpathmoveto{\pgfqpoint{2.044442in}{0.936228in}}%
\pgfpathcurveto{\pgfqpoint{2.052678in}{0.936228in}}{\pgfqpoint{2.060578in}{0.939500in}}{\pgfqpoint{2.066402in}{0.945324in}}%
\pgfpathcurveto{\pgfqpoint{2.072226in}{0.951148in}}{\pgfqpoint{2.075498in}{0.959048in}}{\pgfqpoint{2.075498in}{0.967285in}}%
\pgfpathcurveto{\pgfqpoint{2.075498in}{0.975521in}}{\pgfqpoint{2.072226in}{0.983421in}}{\pgfqpoint{2.066402in}{0.989245in}}%
\pgfpathcurveto{\pgfqpoint{2.060578in}{0.995069in}}{\pgfqpoint{2.052678in}{0.998341in}}{\pgfqpoint{2.044442in}{0.998341in}}%
\pgfpathcurveto{\pgfqpoint{2.036205in}{0.998341in}}{\pgfqpoint{2.028305in}{0.995069in}}{\pgfqpoint{2.022481in}{0.989245in}}%
\pgfpathcurveto{\pgfqpoint{2.016657in}{0.983421in}}{\pgfqpoint{2.013385in}{0.975521in}}{\pgfqpoint{2.013385in}{0.967285in}}%
\pgfpathcurveto{\pgfqpoint{2.013385in}{0.959048in}}{\pgfqpoint{2.016657in}{0.951148in}}{\pgfqpoint{2.022481in}{0.945324in}}%
\pgfpathcurveto{\pgfqpoint{2.028305in}{0.939500in}}{\pgfqpoint{2.036205in}{0.936228in}}{\pgfqpoint{2.044442in}{0.936228in}}%
\pgfpathclose%
\pgfusepath{stroke,fill}%
\end{pgfscope}%
\begin{pgfscope}%
\pgfpathrectangle{\pgfqpoint{0.100000in}{0.212622in}}{\pgfqpoint{3.696000in}{3.696000in}}%
\pgfusepath{clip}%
\pgfsetbuttcap%
\pgfsetroundjoin%
\definecolor{currentfill}{rgb}{0.121569,0.466667,0.705882}%
\pgfsetfillcolor{currentfill}%
\pgfsetfillopacity{0.953943}%
\pgfsetlinewidth{1.003750pt}%
\definecolor{currentstroke}{rgb}{0.121569,0.466667,0.705882}%
\pgfsetstrokecolor{currentstroke}%
\pgfsetstrokeopacity{0.953943}%
\pgfsetdash{}{0pt}%
\pgfpathmoveto{\pgfqpoint{2.059364in}{0.934115in}}%
\pgfpathcurveto{\pgfqpoint{2.067600in}{0.934115in}}{\pgfqpoint{2.075500in}{0.937387in}}{\pgfqpoint{2.081324in}{0.943211in}}%
\pgfpathcurveto{\pgfqpoint{2.087148in}{0.949035in}}{\pgfqpoint{2.090421in}{0.956935in}}{\pgfqpoint{2.090421in}{0.965171in}}%
\pgfpathcurveto{\pgfqpoint{2.090421in}{0.973408in}}{\pgfqpoint{2.087148in}{0.981308in}}{\pgfqpoint{2.081324in}{0.987132in}}%
\pgfpathcurveto{\pgfqpoint{2.075500in}{0.992956in}}{\pgfqpoint{2.067600in}{0.996228in}}{\pgfqpoint{2.059364in}{0.996228in}}%
\pgfpathcurveto{\pgfqpoint{2.051128in}{0.996228in}}{\pgfqpoint{2.043228in}{0.992956in}}{\pgfqpoint{2.037404in}{0.987132in}}%
\pgfpathcurveto{\pgfqpoint{2.031580in}{0.981308in}}{\pgfqpoint{2.028308in}{0.973408in}}{\pgfqpoint{2.028308in}{0.965171in}}%
\pgfpathcurveto{\pgfqpoint{2.028308in}{0.956935in}}{\pgfqpoint{2.031580in}{0.949035in}}{\pgfqpoint{2.037404in}{0.943211in}}%
\pgfpathcurveto{\pgfqpoint{2.043228in}{0.937387in}}{\pgfqpoint{2.051128in}{0.934115in}}{\pgfqpoint{2.059364in}{0.934115in}}%
\pgfpathclose%
\pgfusepath{stroke,fill}%
\end{pgfscope}%
\begin{pgfscope}%
\pgfpathrectangle{\pgfqpoint{0.100000in}{0.212622in}}{\pgfqpoint{3.696000in}{3.696000in}}%
\pgfusepath{clip}%
\pgfsetbuttcap%
\pgfsetroundjoin%
\definecolor{currentfill}{rgb}{0.121569,0.466667,0.705882}%
\pgfsetfillcolor{currentfill}%
\pgfsetfillopacity{0.956569}%
\pgfsetlinewidth{1.003750pt}%
\definecolor{currentstroke}{rgb}{0.121569,0.466667,0.705882}%
\pgfsetstrokecolor{currentstroke}%
\pgfsetstrokeopacity{0.956569}%
\pgfsetdash{}{0pt}%
\pgfpathmoveto{\pgfqpoint{2.072561in}{0.932327in}}%
\pgfpathcurveto{\pgfqpoint{2.080797in}{0.932327in}}{\pgfqpoint{2.088697in}{0.935600in}}{\pgfqpoint{2.094521in}{0.941423in}}%
\pgfpathcurveto{\pgfqpoint{2.100345in}{0.947247in}}{\pgfqpoint{2.103617in}{0.955147in}}{\pgfqpoint{2.103617in}{0.963384in}}%
\pgfpathcurveto{\pgfqpoint{2.103617in}{0.971620in}}{\pgfqpoint{2.100345in}{0.979520in}}{\pgfqpoint{2.094521in}{0.985344in}}%
\pgfpathcurveto{\pgfqpoint{2.088697in}{0.991168in}}{\pgfqpoint{2.080797in}{0.994440in}}{\pgfqpoint{2.072561in}{0.994440in}}%
\pgfpathcurveto{\pgfqpoint{2.064325in}{0.994440in}}{\pgfqpoint{2.056424in}{0.991168in}}{\pgfqpoint{2.050601in}{0.985344in}}%
\pgfpathcurveto{\pgfqpoint{2.044777in}{0.979520in}}{\pgfqpoint{2.041504in}{0.971620in}}{\pgfqpoint{2.041504in}{0.963384in}}%
\pgfpathcurveto{\pgfqpoint{2.041504in}{0.955147in}}{\pgfqpoint{2.044777in}{0.947247in}}{\pgfqpoint{2.050601in}{0.941423in}}%
\pgfpathcurveto{\pgfqpoint{2.056424in}{0.935600in}}{\pgfqpoint{2.064325in}{0.932327in}}{\pgfqpoint{2.072561in}{0.932327in}}%
\pgfpathclose%
\pgfusepath{stroke,fill}%
\end{pgfscope}%
\begin{pgfscope}%
\pgfpathrectangle{\pgfqpoint{0.100000in}{0.212622in}}{\pgfqpoint{3.696000in}{3.696000in}}%
\pgfusepath{clip}%
\pgfsetbuttcap%
\pgfsetroundjoin%
\definecolor{currentfill}{rgb}{0.121569,0.466667,0.705882}%
\pgfsetfillcolor{currentfill}%
\pgfsetfillopacity{0.957947}%
\pgfsetlinewidth{1.003750pt}%
\definecolor{currentstroke}{rgb}{0.121569,0.466667,0.705882}%
\pgfsetstrokecolor{currentstroke}%
\pgfsetstrokeopacity{0.957947}%
\pgfsetdash{}{0pt}%
\pgfpathmoveto{\pgfqpoint{2.079802in}{0.931351in}}%
\pgfpathcurveto{\pgfqpoint{2.088038in}{0.931351in}}{\pgfqpoint{2.095938in}{0.934624in}}{\pgfqpoint{2.101762in}{0.940448in}}%
\pgfpathcurveto{\pgfqpoint{2.107586in}{0.946272in}}{\pgfqpoint{2.110858in}{0.954172in}}{\pgfqpoint{2.110858in}{0.962408in}}%
\pgfpathcurveto{\pgfqpoint{2.110858in}{0.970644in}}{\pgfqpoint{2.107586in}{0.978544in}}{\pgfqpoint{2.101762in}{0.984368in}}%
\pgfpathcurveto{\pgfqpoint{2.095938in}{0.990192in}}{\pgfqpoint{2.088038in}{0.993464in}}{\pgfqpoint{2.079802in}{0.993464in}}%
\pgfpathcurveto{\pgfqpoint{2.071566in}{0.993464in}}{\pgfqpoint{2.063666in}{0.990192in}}{\pgfqpoint{2.057842in}{0.984368in}}%
\pgfpathcurveto{\pgfqpoint{2.052018in}{0.978544in}}{\pgfqpoint{2.048745in}{0.970644in}}{\pgfqpoint{2.048745in}{0.962408in}}%
\pgfpathcurveto{\pgfqpoint{2.048745in}{0.954172in}}{\pgfqpoint{2.052018in}{0.946272in}}{\pgfqpoint{2.057842in}{0.940448in}}%
\pgfpathcurveto{\pgfqpoint{2.063666in}{0.934624in}}{\pgfqpoint{2.071566in}{0.931351in}}{\pgfqpoint{2.079802in}{0.931351in}}%
\pgfpathclose%
\pgfusepath{stroke,fill}%
\end{pgfscope}%
\begin{pgfscope}%
\pgfpathrectangle{\pgfqpoint{0.100000in}{0.212622in}}{\pgfqpoint{3.696000in}{3.696000in}}%
\pgfusepath{clip}%
\pgfsetbuttcap%
\pgfsetroundjoin%
\definecolor{currentfill}{rgb}{0.121569,0.466667,0.705882}%
\pgfsetfillcolor{currentfill}%
\pgfsetfillopacity{0.958178}%
\pgfsetlinewidth{1.003750pt}%
\definecolor{currentstroke}{rgb}{0.121569,0.466667,0.705882}%
\pgfsetstrokecolor{currentstroke}%
\pgfsetstrokeopacity{0.958178}%
\pgfsetdash{}{0pt}%
\pgfpathmoveto{\pgfqpoint{2.583411in}{1.033320in}}%
\pgfpathcurveto{\pgfqpoint{2.591647in}{1.033320in}}{\pgfqpoint{2.599547in}{1.036593in}}{\pgfqpoint{2.605371in}{1.042417in}}%
\pgfpathcurveto{\pgfqpoint{2.611195in}{1.048240in}}{\pgfqpoint{2.614467in}{1.056141in}}{\pgfqpoint{2.614467in}{1.064377in}}%
\pgfpathcurveto{\pgfqpoint{2.614467in}{1.072613in}}{\pgfqpoint{2.611195in}{1.080513in}}{\pgfqpoint{2.605371in}{1.086337in}}%
\pgfpathcurveto{\pgfqpoint{2.599547in}{1.092161in}}{\pgfqpoint{2.591647in}{1.095433in}}{\pgfqpoint{2.583411in}{1.095433in}}%
\pgfpathcurveto{\pgfqpoint{2.575175in}{1.095433in}}{\pgfqpoint{2.567275in}{1.092161in}}{\pgfqpoint{2.561451in}{1.086337in}}%
\pgfpathcurveto{\pgfqpoint{2.555627in}{1.080513in}}{\pgfqpoint{2.552354in}{1.072613in}}{\pgfqpoint{2.552354in}{1.064377in}}%
\pgfpathcurveto{\pgfqpoint{2.552354in}{1.056141in}}{\pgfqpoint{2.555627in}{1.048240in}}{\pgfqpoint{2.561451in}{1.042417in}}%
\pgfpathcurveto{\pgfqpoint{2.567275in}{1.036593in}}{\pgfqpoint{2.575175in}{1.033320in}}{\pgfqpoint{2.583411in}{1.033320in}}%
\pgfpathclose%
\pgfusepath{stroke,fill}%
\end{pgfscope}%
\begin{pgfscope}%
\pgfpathrectangle{\pgfqpoint{0.100000in}{0.212622in}}{\pgfqpoint{3.696000in}{3.696000in}}%
\pgfusepath{clip}%
\pgfsetbuttcap%
\pgfsetroundjoin%
\definecolor{currentfill}{rgb}{0.121569,0.466667,0.705882}%
\pgfsetfillcolor{currentfill}%
\pgfsetfillopacity{0.958426}%
\pgfsetlinewidth{1.003750pt}%
\definecolor{currentstroke}{rgb}{0.121569,0.466667,0.705882}%
\pgfsetstrokecolor{currentstroke}%
\pgfsetstrokeopacity{0.958426}%
\pgfsetdash{}{0pt}%
\pgfpathmoveto{\pgfqpoint{2.082318in}{0.931035in}}%
\pgfpathcurveto{\pgfqpoint{2.090555in}{0.931035in}}{\pgfqpoint{2.098455in}{0.934307in}}{\pgfqpoint{2.104278in}{0.940131in}}%
\pgfpathcurveto{\pgfqpoint{2.110102in}{0.945955in}}{\pgfqpoint{2.113375in}{0.953855in}}{\pgfqpoint{2.113375in}{0.962091in}}%
\pgfpathcurveto{\pgfqpoint{2.113375in}{0.970328in}}{\pgfqpoint{2.110102in}{0.978228in}}{\pgfqpoint{2.104278in}{0.984052in}}%
\pgfpathcurveto{\pgfqpoint{2.098455in}{0.989875in}}{\pgfqpoint{2.090555in}{0.993148in}}{\pgfqpoint{2.082318in}{0.993148in}}%
\pgfpathcurveto{\pgfqpoint{2.074082in}{0.993148in}}{\pgfqpoint{2.066182in}{0.989875in}}{\pgfqpoint{2.060358in}{0.984052in}}%
\pgfpathcurveto{\pgfqpoint{2.054534in}{0.978228in}}{\pgfqpoint{2.051262in}{0.970328in}}{\pgfqpoint{2.051262in}{0.962091in}}%
\pgfpathcurveto{\pgfqpoint{2.051262in}{0.953855in}}{\pgfqpoint{2.054534in}{0.945955in}}{\pgfqpoint{2.060358in}{0.940131in}}%
\pgfpathcurveto{\pgfqpoint{2.066182in}{0.934307in}}{\pgfqpoint{2.074082in}{0.931035in}}{\pgfqpoint{2.082318in}{0.931035in}}%
\pgfpathclose%
\pgfusepath{stroke,fill}%
\end{pgfscope}%
\begin{pgfscope}%
\pgfpathrectangle{\pgfqpoint{0.100000in}{0.212622in}}{\pgfqpoint{3.696000in}{3.696000in}}%
\pgfusepath{clip}%
\pgfsetbuttcap%
\pgfsetroundjoin%
\definecolor{currentfill}{rgb}{0.121569,0.466667,0.705882}%
\pgfsetfillcolor{currentfill}%
\pgfsetfillopacity{0.959271}%
\pgfsetlinewidth{1.003750pt}%
\definecolor{currentstroke}{rgb}{0.121569,0.466667,0.705882}%
\pgfsetstrokecolor{currentstroke}%
\pgfsetstrokeopacity{0.959271}%
\pgfsetdash{}{0pt}%
\pgfpathmoveto{\pgfqpoint{2.086916in}{0.930493in}}%
\pgfpathcurveto{\pgfqpoint{2.095153in}{0.930493in}}{\pgfqpoint{2.103053in}{0.933765in}}{\pgfqpoint{2.108877in}{0.939589in}}%
\pgfpathcurveto{\pgfqpoint{2.114701in}{0.945413in}}{\pgfqpoint{2.117973in}{0.953313in}}{\pgfqpoint{2.117973in}{0.961549in}}%
\pgfpathcurveto{\pgfqpoint{2.117973in}{0.969785in}}{\pgfqpoint{2.114701in}{0.977685in}}{\pgfqpoint{2.108877in}{0.983509in}}%
\pgfpathcurveto{\pgfqpoint{2.103053in}{0.989333in}}{\pgfqpoint{2.095153in}{0.992606in}}{\pgfqpoint{2.086916in}{0.992606in}}%
\pgfpathcurveto{\pgfqpoint{2.078680in}{0.992606in}}{\pgfqpoint{2.070780in}{0.989333in}}{\pgfqpoint{2.064956in}{0.983509in}}%
\pgfpathcurveto{\pgfqpoint{2.059132in}{0.977685in}}{\pgfqpoint{2.055860in}{0.969785in}}{\pgfqpoint{2.055860in}{0.961549in}}%
\pgfpathcurveto{\pgfqpoint{2.055860in}{0.953313in}}{\pgfqpoint{2.059132in}{0.945413in}}{\pgfqpoint{2.064956in}{0.939589in}}%
\pgfpathcurveto{\pgfqpoint{2.070780in}{0.933765in}}{\pgfqpoint{2.078680in}{0.930493in}}{\pgfqpoint{2.086916in}{0.930493in}}%
\pgfpathclose%
\pgfusepath{stroke,fill}%
\end{pgfscope}%
\begin{pgfscope}%
\pgfpathrectangle{\pgfqpoint{0.100000in}{0.212622in}}{\pgfqpoint{3.696000in}{3.696000in}}%
\pgfusepath{clip}%
\pgfsetbuttcap%
\pgfsetroundjoin%
\definecolor{currentfill}{rgb}{0.121569,0.466667,0.705882}%
\pgfsetfillcolor{currentfill}%
\pgfsetfillopacity{0.960837}%
\pgfsetlinewidth{1.003750pt}%
\definecolor{currentstroke}{rgb}{0.121569,0.466667,0.705882}%
\pgfsetstrokecolor{currentstroke}%
\pgfsetstrokeopacity{0.960837}%
\pgfsetdash{}{0pt}%
\pgfpathmoveto{\pgfqpoint{2.095270in}{0.929559in}}%
\pgfpathcurveto{\pgfqpoint{2.103507in}{0.929559in}}{\pgfqpoint{2.111407in}{0.932831in}}{\pgfqpoint{2.117231in}{0.938655in}}%
\pgfpathcurveto{\pgfqpoint{2.123054in}{0.944479in}}{\pgfqpoint{2.126327in}{0.952379in}}{\pgfqpoint{2.126327in}{0.960615in}}%
\pgfpathcurveto{\pgfqpoint{2.126327in}{0.968852in}}{\pgfqpoint{2.123054in}{0.976752in}}{\pgfqpoint{2.117231in}{0.982576in}}%
\pgfpathcurveto{\pgfqpoint{2.111407in}{0.988400in}}{\pgfqpoint{2.103507in}{0.991672in}}{\pgfqpoint{2.095270in}{0.991672in}}%
\pgfpathcurveto{\pgfqpoint{2.087034in}{0.991672in}}{\pgfqpoint{2.079134in}{0.988400in}}{\pgfqpoint{2.073310in}{0.982576in}}%
\pgfpathcurveto{\pgfqpoint{2.067486in}{0.976752in}}{\pgfqpoint{2.064214in}{0.968852in}}{\pgfqpoint{2.064214in}{0.960615in}}%
\pgfpathcurveto{\pgfqpoint{2.064214in}{0.952379in}}{\pgfqpoint{2.067486in}{0.944479in}}{\pgfqpoint{2.073310in}{0.938655in}}%
\pgfpathcurveto{\pgfqpoint{2.079134in}{0.932831in}}{\pgfqpoint{2.087034in}{0.929559in}}{\pgfqpoint{2.095270in}{0.929559in}}%
\pgfpathclose%
\pgfusepath{stroke,fill}%
\end{pgfscope}%
\begin{pgfscope}%
\pgfpathrectangle{\pgfqpoint{0.100000in}{0.212622in}}{\pgfqpoint{3.696000in}{3.696000in}}%
\pgfusepath{clip}%
\pgfsetbuttcap%
\pgfsetroundjoin%
\definecolor{currentfill}{rgb}{0.121569,0.466667,0.705882}%
\pgfsetfillcolor{currentfill}%
\pgfsetfillopacity{0.963626}%
\pgfsetlinewidth{1.003750pt}%
\definecolor{currentstroke}{rgb}{0.121569,0.466667,0.705882}%
\pgfsetstrokecolor{currentstroke}%
\pgfsetstrokeopacity{0.963626}%
\pgfsetdash{}{0pt}%
\pgfpathmoveto{\pgfqpoint{2.110513in}{0.927904in}}%
\pgfpathcurveto{\pgfqpoint{2.118750in}{0.927904in}}{\pgfqpoint{2.126650in}{0.931176in}}{\pgfqpoint{2.132473in}{0.937000in}}%
\pgfpathcurveto{\pgfqpoint{2.138297in}{0.942824in}}{\pgfqpoint{2.141570in}{0.950724in}}{\pgfqpoint{2.141570in}{0.958960in}}%
\pgfpathcurveto{\pgfqpoint{2.141570in}{0.967196in}}{\pgfqpoint{2.138297in}{0.975097in}}{\pgfqpoint{2.132473in}{0.980920in}}%
\pgfpathcurveto{\pgfqpoint{2.126650in}{0.986744in}}{\pgfqpoint{2.118750in}{0.990017in}}{\pgfqpoint{2.110513in}{0.990017in}}%
\pgfpathcurveto{\pgfqpoint{2.102277in}{0.990017in}}{\pgfqpoint{2.094377in}{0.986744in}}{\pgfqpoint{2.088553in}{0.980920in}}%
\pgfpathcurveto{\pgfqpoint{2.082729in}{0.975097in}}{\pgfqpoint{2.079457in}{0.967196in}}{\pgfqpoint{2.079457in}{0.958960in}}%
\pgfpathcurveto{\pgfqpoint{2.079457in}{0.950724in}}{\pgfqpoint{2.082729in}{0.942824in}}{\pgfqpoint{2.088553in}{0.937000in}}%
\pgfpathcurveto{\pgfqpoint{2.094377in}{0.931176in}}{\pgfqpoint{2.102277in}{0.927904in}}{\pgfqpoint{2.110513in}{0.927904in}}%
\pgfpathclose%
\pgfusepath{stroke,fill}%
\end{pgfscope}%
\begin{pgfscope}%
\pgfpathrectangle{\pgfqpoint{0.100000in}{0.212622in}}{\pgfqpoint{3.696000in}{3.696000in}}%
\pgfusepath{clip}%
\pgfsetbuttcap%
\pgfsetroundjoin%
\definecolor{currentfill}{rgb}{0.121569,0.466667,0.705882}%
\pgfsetfillcolor{currentfill}%
\pgfsetfillopacity{0.965510}%
\pgfsetlinewidth{1.003750pt}%
\definecolor{currentstroke}{rgb}{0.121569,0.466667,0.705882}%
\pgfsetstrokecolor{currentstroke}%
\pgfsetstrokeopacity{0.965510}%
\pgfsetdash{}{0pt}%
\pgfpathmoveto{\pgfqpoint{2.121019in}{0.926598in}}%
\pgfpathcurveto{\pgfqpoint{2.129255in}{0.926598in}}{\pgfqpoint{2.137155in}{0.929871in}}{\pgfqpoint{2.142979in}{0.935695in}}%
\pgfpathcurveto{\pgfqpoint{2.148803in}{0.941519in}}{\pgfqpoint{2.152075in}{0.949419in}}{\pgfqpoint{2.152075in}{0.957655in}}%
\pgfpathcurveto{\pgfqpoint{2.152075in}{0.965891in}}{\pgfqpoint{2.148803in}{0.973791in}}{\pgfqpoint{2.142979in}{0.979615in}}%
\pgfpathcurveto{\pgfqpoint{2.137155in}{0.985439in}}{\pgfqpoint{2.129255in}{0.988711in}}{\pgfqpoint{2.121019in}{0.988711in}}%
\pgfpathcurveto{\pgfqpoint{2.112783in}{0.988711in}}{\pgfqpoint{2.104883in}{0.985439in}}{\pgfqpoint{2.099059in}{0.979615in}}%
\pgfpathcurveto{\pgfqpoint{2.093235in}{0.973791in}}{\pgfqpoint{2.089962in}{0.965891in}}{\pgfqpoint{2.089962in}{0.957655in}}%
\pgfpathcurveto{\pgfqpoint{2.089962in}{0.949419in}}{\pgfqpoint{2.093235in}{0.941519in}}{\pgfqpoint{2.099059in}{0.935695in}}%
\pgfpathcurveto{\pgfqpoint{2.104883in}{0.929871in}}{\pgfqpoint{2.112783in}{0.926598in}}{\pgfqpoint{2.121019in}{0.926598in}}%
\pgfpathclose%
\pgfusepath{stroke,fill}%
\end{pgfscope}%
\begin{pgfscope}%
\pgfpathrectangle{\pgfqpoint{0.100000in}{0.212622in}}{\pgfqpoint{3.696000in}{3.696000in}}%
\pgfusepath{clip}%
\pgfsetbuttcap%
\pgfsetroundjoin%
\definecolor{currentfill}{rgb}{0.121569,0.466667,0.705882}%
\pgfsetfillcolor{currentfill}%
\pgfsetfillopacity{0.966585}%
\pgfsetlinewidth{1.003750pt}%
\definecolor{currentstroke}{rgb}{0.121569,0.466667,0.705882}%
\pgfsetstrokecolor{currentstroke}%
\pgfsetstrokeopacity{0.966585}%
\pgfsetdash{}{0pt}%
\pgfpathmoveto{\pgfqpoint{2.127140in}{0.925795in}}%
\pgfpathcurveto{\pgfqpoint{2.135376in}{0.925795in}}{\pgfqpoint{2.143276in}{0.929067in}}{\pgfqpoint{2.149100in}{0.934891in}}%
\pgfpathcurveto{\pgfqpoint{2.154924in}{0.940715in}}{\pgfqpoint{2.158197in}{0.948615in}}{\pgfqpoint{2.158197in}{0.956851in}}%
\pgfpathcurveto{\pgfqpoint{2.158197in}{0.965087in}}{\pgfqpoint{2.154924in}{0.972987in}}{\pgfqpoint{2.149100in}{0.978811in}}%
\pgfpathcurveto{\pgfqpoint{2.143276in}{0.984635in}}{\pgfqpoint{2.135376in}{0.987908in}}{\pgfqpoint{2.127140in}{0.987908in}}%
\pgfpathcurveto{\pgfqpoint{2.118904in}{0.987908in}}{\pgfqpoint{2.111004in}{0.984635in}}{\pgfqpoint{2.105180in}{0.978811in}}%
\pgfpathcurveto{\pgfqpoint{2.099356in}{0.972987in}}{\pgfqpoint{2.096084in}{0.965087in}}{\pgfqpoint{2.096084in}{0.956851in}}%
\pgfpathcurveto{\pgfqpoint{2.096084in}{0.948615in}}{\pgfqpoint{2.099356in}{0.940715in}}{\pgfqpoint{2.105180in}{0.934891in}}%
\pgfpathcurveto{\pgfqpoint{2.111004in}{0.929067in}}{\pgfqpoint{2.118904in}{0.925795in}}{\pgfqpoint{2.127140in}{0.925795in}}%
\pgfpathclose%
\pgfusepath{stroke,fill}%
\end{pgfscope}%
\begin{pgfscope}%
\pgfpathrectangle{\pgfqpoint{0.100000in}{0.212622in}}{\pgfqpoint{3.696000in}{3.696000in}}%
\pgfusepath{clip}%
\pgfsetbuttcap%
\pgfsetroundjoin%
\definecolor{currentfill}{rgb}{0.121569,0.466667,0.705882}%
\pgfsetfillcolor{currentfill}%
\pgfsetfillopacity{0.968497}%
\pgfsetlinewidth{1.003750pt}%
\definecolor{currentstroke}{rgb}{0.121569,0.466667,0.705882}%
\pgfsetstrokecolor{currentstroke}%
\pgfsetstrokeopacity{0.968497}%
\pgfsetdash{}{0pt}%
\pgfpathmoveto{\pgfqpoint{2.138302in}{0.924337in}}%
\pgfpathcurveto{\pgfqpoint{2.146538in}{0.924337in}}{\pgfqpoint{2.154438in}{0.927609in}}{\pgfqpoint{2.160262in}{0.933433in}}%
\pgfpathcurveto{\pgfqpoint{2.166086in}{0.939257in}}{\pgfqpoint{2.169359in}{0.947157in}}{\pgfqpoint{2.169359in}{0.955393in}}%
\pgfpathcurveto{\pgfqpoint{2.169359in}{0.963630in}}{\pgfqpoint{2.166086in}{0.971530in}}{\pgfqpoint{2.160262in}{0.977354in}}%
\pgfpathcurveto{\pgfqpoint{2.154438in}{0.983178in}}{\pgfqpoint{2.146538in}{0.986450in}}{\pgfqpoint{2.138302in}{0.986450in}}%
\pgfpathcurveto{\pgfqpoint{2.130066in}{0.986450in}}{\pgfqpoint{2.122166in}{0.983178in}}{\pgfqpoint{2.116342in}{0.977354in}}%
\pgfpathcurveto{\pgfqpoint{2.110518in}{0.971530in}}{\pgfqpoint{2.107246in}{0.963630in}}{\pgfqpoint{2.107246in}{0.955393in}}%
\pgfpathcurveto{\pgfqpoint{2.107246in}{0.947157in}}{\pgfqpoint{2.110518in}{0.939257in}}{\pgfqpoint{2.116342in}{0.933433in}}%
\pgfpathcurveto{\pgfqpoint{2.122166in}{0.927609in}}{\pgfqpoint{2.130066in}{0.924337in}}{\pgfqpoint{2.138302in}{0.924337in}}%
\pgfpathclose%
\pgfusepath{stroke,fill}%
\end{pgfscope}%
\begin{pgfscope}%
\pgfpathrectangle{\pgfqpoint{0.100000in}{0.212622in}}{\pgfqpoint{3.696000in}{3.696000in}}%
\pgfusepath{clip}%
\pgfsetbuttcap%
\pgfsetroundjoin%
\definecolor{currentfill}{rgb}{0.121569,0.466667,0.705882}%
\pgfsetfillcolor{currentfill}%
\pgfsetfillopacity{0.969262}%
\pgfsetlinewidth{1.003750pt}%
\definecolor{currentstroke}{rgb}{0.121569,0.466667,0.705882}%
\pgfsetstrokecolor{currentstroke}%
\pgfsetstrokeopacity{0.969262}%
\pgfsetdash{}{0pt}%
\pgfpathmoveto{\pgfqpoint{2.569593in}{1.000553in}}%
\pgfpathcurveto{\pgfqpoint{2.577829in}{1.000553in}}{\pgfqpoint{2.585729in}{1.003825in}}{\pgfqpoint{2.591553in}{1.009649in}}%
\pgfpathcurveto{\pgfqpoint{2.597377in}{1.015473in}}{\pgfqpoint{2.600650in}{1.023373in}}{\pgfqpoint{2.600650in}{1.031610in}}%
\pgfpathcurveto{\pgfqpoint{2.600650in}{1.039846in}}{\pgfqpoint{2.597377in}{1.047746in}}{\pgfqpoint{2.591553in}{1.053570in}}%
\pgfpathcurveto{\pgfqpoint{2.585729in}{1.059394in}}{\pgfqpoint{2.577829in}{1.062666in}}{\pgfqpoint{2.569593in}{1.062666in}}%
\pgfpathcurveto{\pgfqpoint{2.561357in}{1.062666in}}{\pgfqpoint{2.553457in}{1.059394in}}{\pgfqpoint{2.547633in}{1.053570in}}%
\pgfpathcurveto{\pgfqpoint{2.541809in}{1.047746in}}{\pgfqpoint{2.538537in}{1.039846in}}{\pgfqpoint{2.538537in}{1.031610in}}%
\pgfpathcurveto{\pgfqpoint{2.538537in}{1.023373in}}{\pgfqpoint{2.541809in}{1.015473in}}{\pgfqpoint{2.547633in}{1.009649in}}%
\pgfpathcurveto{\pgfqpoint{2.553457in}{1.003825in}}{\pgfqpoint{2.561357in}{1.000553in}}{\pgfqpoint{2.569593in}{1.000553in}}%
\pgfpathclose%
\pgfusepath{stroke,fill}%
\end{pgfscope}%
\begin{pgfscope}%
\pgfpathrectangle{\pgfqpoint{0.100000in}{0.212622in}}{\pgfqpoint{3.696000in}{3.696000in}}%
\pgfusepath{clip}%
\pgfsetbuttcap%
\pgfsetroundjoin%
\definecolor{currentfill}{rgb}{0.121569,0.466667,0.705882}%
\pgfsetfillcolor{currentfill}%
\pgfsetfillopacity{0.972033}%
\pgfsetlinewidth{1.003750pt}%
\definecolor{currentstroke}{rgb}{0.121569,0.466667,0.705882}%
\pgfsetstrokecolor{currentstroke}%
\pgfsetstrokeopacity{0.972033}%
\pgfsetdash{}{0pt}%
\pgfpathmoveto{\pgfqpoint{2.158574in}{0.921698in}}%
\pgfpathcurveto{\pgfqpoint{2.166810in}{0.921698in}}{\pgfqpoint{2.174710in}{0.924970in}}{\pgfqpoint{2.180534in}{0.930794in}}%
\pgfpathcurveto{\pgfqpoint{2.186358in}{0.936618in}}{\pgfqpoint{2.189630in}{0.944518in}}{\pgfqpoint{2.189630in}{0.952754in}}%
\pgfpathcurveto{\pgfqpoint{2.189630in}{0.960991in}}{\pgfqpoint{2.186358in}{0.968891in}}{\pgfqpoint{2.180534in}{0.974715in}}%
\pgfpathcurveto{\pgfqpoint{2.174710in}{0.980539in}}{\pgfqpoint{2.166810in}{0.983811in}}{\pgfqpoint{2.158574in}{0.983811in}}%
\pgfpathcurveto{\pgfqpoint{2.150337in}{0.983811in}}{\pgfqpoint{2.142437in}{0.980539in}}{\pgfqpoint{2.136613in}{0.974715in}}%
\pgfpathcurveto{\pgfqpoint{2.130789in}{0.968891in}}{\pgfqpoint{2.127517in}{0.960991in}}{\pgfqpoint{2.127517in}{0.952754in}}%
\pgfpathcurveto{\pgfqpoint{2.127517in}{0.944518in}}{\pgfqpoint{2.130789in}{0.936618in}}{\pgfqpoint{2.136613in}{0.930794in}}%
\pgfpathcurveto{\pgfqpoint{2.142437in}{0.924970in}}{\pgfqpoint{2.150337in}{0.921698in}}{\pgfqpoint{2.158574in}{0.921698in}}%
\pgfpathclose%
\pgfusepath{stroke,fill}%
\end{pgfscope}%
\begin{pgfscope}%
\pgfpathrectangle{\pgfqpoint{0.100000in}{0.212622in}}{\pgfqpoint{3.696000in}{3.696000in}}%
\pgfusepath{clip}%
\pgfsetbuttcap%
\pgfsetroundjoin%
\definecolor{currentfill}{rgb}{0.121569,0.466667,0.705882}%
\pgfsetfillcolor{currentfill}%
\pgfsetfillopacity{0.975126}%
\pgfsetlinewidth{1.003750pt}%
\definecolor{currentstroke}{rgb}{0.121569,0.466667,0.705882}%
\pgfsetstrokecolor{currentstroke}%
\pgfsetstrokeopacity{0.975126}%
\pgfsetdash{}{0pt}%
\pgfpathmoveto{\pgfqpoint{2.176752in}{0.919350in}}%
\pgfpathcurveto{\pgfqpoint{2.184988in}{0.919350in}}{\pgfqpoint{2.192888in}{0.922622in}}{\pgfqpoint{2.198712in}{0.928446in}}%
\pgfpathcurveto{\pgfqpoint{2.204536in}{0.934270in}}{\pgfqpoint{2.207808in}{0.942170in}}{\pgfqpoint{2.207808in}{0.950406in}}%
\pgfpathcurveto{\pgfqpoint{2.207808in}{0.958642in}}{\pgfqpoint{2.204536in}{0.966543in}}{\pgfqpoint{2.198712in}{0.972366in}}%
\pgfpathcurveto{\pgfqpoint{2.192888in}{0.978190in}}{\pgfqpoint{2.184988in}{0.981463in}}{\pgfqpoint{2.176752in}{0.981463in}}%
\pgfpathcurveto{\pgfqpoint{2.168515in}{0.981463in}}{\pgfqpoint{2.160615in}{0.978190in}}{\pgfqpoint{2.154791in}{0.972366in}}%
\pgfpathcurveto{\pgfqpoint{2.148968in}{0.966543in}}{\pgfqpoint{2.145695in}{0.958642in}}{\pgfqpoint{2.145695in}{0.950406in}}%
\pgfpathcurveto{\pgfqpoint{2.145695in}{0.942170in}}{\pgfqpoint{2.148968in}{0.934270in}}{\pgfqpoint{2.154791in}{0.928446in}}%
\pgfpathcurveto{\pgfqpoint{2.160615in}{0.922622in}}{\pgfqpoint{2.168515in}{0.919350in}}{\pgfqpoint{2.176752in}{0.919350in}}%
\pgfpathclose%
\pgfusepath{stroke,fill}%
\end{pgfscope}%
\begin{pgfscope}%
\pgfpathrectangle{\pgfqpoint{0.100000in}{0.212622in}}{\pgfqpoint{3.696000in}{3.696000in}}%
\pgfusepath{clip}%
\pgfsetbuttcap%
\pgfsetroundjoin%
\definecolor{currentfill}{rgb}{0.121569,0.466667,0.705882}%
\pgfsetfillcolor{currentfill}%
\pgfsetfillopacity{0.977319}%
\pgfsetlinewidth{1.003750pt}%
\definecolor{currentstroke}{rgb}{0.121569,0.466667,0.705882}%
\pgfsetstrokecolor{currentstroke}%
\pgfsetstrokeopacity{0.977319}%
\pgfsetdash{}{0pt}%
\pgfpathmoveto{\pgfqpoint{2.190150in}{0.917646in}}%
\pgfpathcurveto{\pgfqpoint{2.198386in}{0.917646in}}{\pgfqpoint{2.206286in}{0.920918in}}{\pgfqpoint{2.212110in}{0.926742in}}%
\pgfpathcurveto{\pgfqpoint{2.217934in}{0.932566in}}{\pgfqpoint{2.221206in}{0.940466in}}{\pgfqpoint{2.221206in}{0.948702in}}%
\pgfpathcurveto{\pgfqpoint{2.221206in}{0.956938in}}{\pgfqpoint{2.217934in}{0.964838in}}{\pgfqpoint{2.212110in}{0.970662in}}%
\pgfpathcurveto{\pgfqpoint{2.206286in}{0.976486in}}{\pgfqpoint{2.198386in}{0.979759in}}{\pgfqpoint{2.190150in}{0.979759in}}%
\pgfpathcurveto{\pgfqpoint{2.181914in}{0.979759in}}{\pgfqpoint{2.174014in}{0.976486in}}{\pgfqpoint{2.168190in}{0.970662in}}%
\pgfpathcurveto{\pgfqpoint{2.162366in}{0.964838in}}{\pgfqpoint{2.159093in}{0.956938in}}{\pgfqpoint{2.159093in}{0.948702in}}%
\pgfpathcurveto{\pgfqpoint{2.159093in}{0.940466in}}{\pgfqpoint{2.162366in}{0.932566in}}{\pgfqpoint{2.168190in}{0.926742in}}%
\pgfpathcurveto{\pgfqpoint{2.174014in}{0.920918in}}{\pgfqpoint{2.181914in}{0.917646in}}{\pgfqpoint{2.190150in}{0.917646in}}%
\pgfpathclose%
\pgfusepath{stroke,fill}%
\end{pgfscope}%
\begin{pgfscope}%
\pgfpathrectangle{\pgfqpoint{0.100000in}{0.212622in}}{\pgfqpoint{3.696000in}{3.696000in}}%
\pgfusepath{clip}%
\pgfsetbuttcap%
\pgfsetroundjoin%
\definecolor{currentfill}{rgb}{0.121569,0.466667,0.705882}%
\pgfsetfillcolor{currentfill}%
\pgfsetfillopacity{0.978766}%
\pgfsetlinewidth{1.003750pt}%
\definecolor{currentstroke}{rgb}{0.121569,0.466667,0.705882}%
\pgfsetstrokecolor{currentstroke}%
\pgfsetstrokeopacity{0.978766}%
\pgfsetdash{}{0pt}%
\pgfpathmoveto{\pgfqpoint{2.198826in}{0.916626in}}%
\pgfpathcurveto{\pgfqpoint{2.207062in}{0.916626in}}{\pgfqpoint{2.214962in}{0.919899in}}{\pgfqpoint{2.220786in}{0.925723in}}%
\pgfpathcurveto{\pgfqpoint{2.226610in}{0.931547in}}{\pgfqpoint{2.229883in}{0.939447in}}{\pgfqpoint{2.229883in}{0.947683in}}%
\pgfpathcurveto{\pgfqpoint{2.229883in}{0.955919in}}{\pgfqpoint{2.226610in}{0.963819in}}{\pgfqpoint{2.220786in}{0.969643in}}%
\pgfpathcurveto{\pgfqpoint{2.214962in}{0.975467in}}{\pgfqpoint{2.207062in}{0.978739in}}{\pgfqpoint{2.198826in}{0.978739in}}%
\pgfpathcurveto{\pgfqpoint{2.190590in}{0.978739in}}{\pgfqpoint{2.182690in}{0.975467in}}{\pgfqpoint{2.176866in}{0.969643in}}%
\pgfpathcurveto{\pgfqpoint{2.171042in}{0.963819in}}{\pgfqpoint{2.167770in}{0.955919in}}{\pgfqpoint{2.167770in}{0.947683in}}%
\pgfpathcurveto{\pgfqpoint{2.167770in}{0.939447in}}{\pgfqpoint{2.171042in}{0.931547in}}{\pgfqpoint{2.176866in}{0.925723in}}%
\pgfpathcurveto{\pgfqpoint{2.182690in}{0.919899in}}{\pgfqpoint{2.190590in}{0.916626in}}{\pgfqpoint{2.198826in}{0.916626in}}%
\pgfpathclose%
\pgfusepath{stroke,fill}%
\end{pgfscope}%
\begin{pgfscope}%
\pgfpathrectangle{\pgfqpoint{0.100000in}{0.212622in}}{\pgfqpoint{3.696000in}{3.696000in}}%
\pgfusepath{clip}%
\pgfsetbuttcap%
\pgfsetroundjoin%
\definecolor{currentfill}{rgb}{0.121569,0.466667,0.705882}%
\pgfsetfillcolor{currentfill}%
\pgfsetfillopacity{0.979588}%
\pgfsetlinewidth{1.003750pt}%
\definecolor{currentstroke}{rgb}{0.121569,0.466667,0.705882}%
\pgfsetstrokecolor{currentstroke}%
\pgfsetstrokeopacity{0.979588}%
\pgfsetdash{}{0pt}%
\pgfpathmoveto{\pgfqpoint{2.203858in}{0.915998in}}%
\pgfpathcurveto{\pgfqpoint{2.212094in}{0.915998in}}{\pgfqpoint{2.219994in}{0.919270in}}{\pgfqpoint{2.225818in}{0.925094in}}%
\pgfpathcurveto{\pgfqpoint{2.231642in}{0.930918in}}{\pgfqpoint{2.234914in}{0.938818in}}{\pgfqpoint{2.234914in}{0.947054in}}%
\pgfpathcurveto{\pgfqpoint{2.234914in}{0.955290in}}{\pgfqpoint{2.231642in}{0.963190in}}{\pgfqpoint{2.225818in}{0.969014in}}%
\pgfpathcurveto{\pgfqpoint{2.219994in}{0.974838in}}{\pgfqpoint{2.212094in}{0.978111in}}{\pgfqpoint{2.203858in}{0.978111in}}%
\pgfpathcurveto{\pgfqpoint{2.195621in}{0.978111in}}{\pgfqpoint{2.187721in}{0.974838in}}{\pgfqpoint{2.181897in}{0.969014in}}%
\pgfpathcurveto{\pgfqpoint{2.176073in}{0.963190in}}{\pgfqpoint{2.172801in}{0.955290in}}{\pgfqpoint{2.172801in}{0.947054in}}%
\pgfpathcurveto{\pgfqpoint{2.172801in}{0.938818in}}{\pgfqpoint{2.176073in}{0.930918in}}{\pgfqpoint{2.181897in}{0.925094in}}%
\pgfpathcurveto{\pgfqpoint{2.187721in}{0.919270in}}{\pgfqpoint{2.195621in}{0.915998in}}{\pgfqpoint{2.203858in}{0.915998in}}%
\pgfpathclose%
\pgfusepath{stroke,fill}%
\end{pgfscope}%
\begin{pgfscope}%
\pgfpathrectangle{\pgfqpoint{0.100000in}{0.212622in}}{\pgfqpoint{3.696000in}{3.696000in}}%
\pgfusepath{clip}%
\pgfsetbuttcap%
\pgfsetroundjoin%
\definecolor{currentfill}{rgb}{0.121569,0.466667,0.705882}%
\pgfsetfillcolor{currentfill}%
\pgfsetfillopacity{0.979979}%
\pgfsetlinewidth{1.003750pt}%
\definecolor{currentstroke}{rgb}{0.121569,0.466667,0.705882}%
\pgfsetstrokecolor{currentstroke}%
\pgfsetstrokeopacity{0.979979}%
\pgfsetdash{}{0pt}%
\pgfpathmoveto{\pgfqpoint{2.206324in}{0.915740in}}%
\pgfpathcurveto{\pgfqpoint{2.214561in}{0.915740in}}{\pgfqpoint{2.222461in}{0.919013in}}{\pgfqpoint{2.228285in}{0.924837in}}%
\pgfpathcurveto{\pgfqpoint{2.234108in}{0.930661in}}{\pgfqpoint{2.237381in}{0.938561in}}{\pgfqpoint{2.237381in}{0.946797in}}%
\pgfpathcurveto{\pgfqpoint{2.237381in}{0.955033in}}{\pgfqpoint{2.234108in}{0.962933in}}{\pgfqpoint{2.228285in}{0.968757in}}%
\pgfpathcurveto{\pgfqpoint{2.222461in}{0.974581in}}{\pgfqpoint{2.214561in}{0.977853in}}{\pgfqpoint{2.206324in}{0.977853in}}%
\pgfpathcurveto{\pgfqpoint{2.198088in}{0.977853in}}{\pgfqpoint{2.190188in}{0.974581in}}{\pgfqpoint{2.184364in}{0.968757in}}%
\pgfpathcurveto{\pgfqpoint{2.178540in}{0.962933in}}{\pgfqpoint{2.175268in}{0.955033in}}{\pgfqpoint{2.175268in}{0.946797in}}%
\pgfpathcurveto{\pgfqpoint{2.175268in}{0.938561in}}{\pgfqpoint{2.178540in}{0.930661in}}{\pgfqpoint{2.184364in}{0.924837in}}%
\pgfpathcurveto{\pgfqpoint{2.190188in}{0.919013in}}{\pgfqpoint{2.198088in}{0.915740in}}{\pgfqpoint{2.206324in}{0.915740in}}%
\pgfpathclose%
\pgfusepath{stroke,fill}%
\end{pgfscope}%
\begin{pgfscope}%
\pgfpathrectangle{\pgfqpoint{0.100000in}{0.212622in}}{\pgfqpoint{3.696000in}{3.696000in}}%
\pgfusepath{clip}%
\pgfsetbuttcap%
\pgfsetroundjoin%
\definecolor{currentfill}{rgb}{0.121569,0.466667,0.705882}%
\pgfsetfillcolor{currentfill}%
\pgfsetfillopacity{0.980680}%
\pgfsetlinewidth{1.003750pt}%
\definecolor{currentstroke}{rgb}{0.121569,0.466667,0.705882}%
\pgfsetstrokecolor{currentstroke}%
\pgfsetstrokeopacity{0.980680}%
\pgfsetdash{}{0pt}%
\pgfpathmoveto{\pgfqpoint{2.210810in}{0.915234in}}%
\pgfpathcurveto{\pgfqpoint{2.219046in}{0.915234in}}{\pgfqpoint{2.226946in}{0.918507in}}{\pgfqpoint{2.232770in}{0.924331in}}%
\pgfpathcurveto{\pgfqpoint{2.238594in}{0.930155in}}{\pgfqpoint{2.241866in}{0.938055in}}{\pgfqpoint{2.241866in}{0.946291in}}%
\pgfpathcurveto{\pgfqpoint{2.241866in}{0.954527in}}{\pgfqpoint{2.238594in}{0.962427in}}{\pgfqpoint{2.232770in}{0.968251in}}%
\pgfpathcurveto{\pgfqpoint{2.226946in}{0.974075in}}{\pgfqpoint{2.219046in}{0.977347in}}{\pgfqpoint{2.210810in}{0.977347in}}%
\pgfpathcurveto{\pgfqpoint{2.202573in}{0.977347in}}{\pgfqpoint{2.194673in}{0.974075in}}{\pgfqpoint{2.188849in}{0.968251in}}%
\pgfpathcurveto{\pgfqpoint{2.183025in}{0.962427in}}{\pgfqpoint{2.179753in}{0.954527in}}{\pgfqpoint{2.179753in}{0.946291in}}%
\pgfpathcurveto{\pgfqpoint{2.179753in}{0.938055in}}{\pgfqpoint{2.183025in}{0.930155in}}{\pgfqpoint{2.188849in}{0.924331in}}%
\pgfpathcurveto{\pgfqpoint{2.194673in}{0.918507in}}{\pgfqpoint{2.202573in}{0.915234in}}{\pgfqpoint{2.210810in}{0.915234in}}%
\pgfpathclose%
\pgfusepath{stroke,fill}%
\end{pgfscope}%
\begin{pgfscope}%
\pgfpathrectangle{\pgfqpoint{0.100000in}{0.212622in}}{\pgfqpoint{3.696000in}{3.696000in}}%
\pgfusepath{clip}%
\pgfsetbuttcap%
\pgfsetroundjoin%
\definecolor{currentfill}{rgb}{0.121569,0.466667,0.705882}%
\pgfsetfillcolor{currentfill}%
\pgfsetfillopacity{0.981049}%
\pgfsetlinewidth{1.003750pt}%
\definecolor{currentstroke}{rgb}{0.121569,0.466667,0.705882}%
\pgfsetstrokecolor{currentstroke}%
\pgfsetstrokeopacity{0.981049}%
\pgfsetdash{}{0pt}%
\pgfpathmoveto{\pgfqpoint{2.551142in}{0.969507in}}%
\pgfpathcurveto{\pgfqpoint{2.559378in}{0.969507in}}{\pgfqpoint{2.567278in}{0.972779in}}{\pgfqpoint{2.573102in}{0.978603in}}%
\pgfpathcurveto{\pgfqpoint{2.578926in}{0.984427in}}{\pgfqpoint{2.582198in}{0.992327in}}{\pgfqpoint{2.582198in}{1.000564in}}%
\pgfpathcurveto{\pgfqpoint{2.582198in}{1.008800in}}{\pgfqpoint{2.578926in}{1.016700in}}{\pgfqpoint{2.573102in}{1.022524in}}%
\pgfpathcurveto{\pgfqpoint{2.567278in}{1.028348in}}{\pgfqpoint{2.559378in}{1.031620in}}{\pgfqpoint{2.551142in}{1.031620in}}%
\pgfpathcurveto{\pgfqpoint{2.542906in}{1.031620in}}{\pgfqpoint{2.535006in}{1.028348in}}{\pgfqpoint{2.529182in}{1.022524in}}%
\pgfpathcurveto{\pgfqpoint{2.523358in}{1.016700in}}{\pgfqpoint{2.520085in}{1.008800in}}{\pgfqpoint{2.520085in}{1.000564in}}%
\pgfpathcurveto{\pgfqpoint{2.520085in}{0.992327in}}{\pgfqpoint{2.523358in}{0.984427in}}{\pgfqpoint{2.529182in}{0.978603in}}%
\pgfpathcurveto{\pgfqpoint{2.535006in}{0.972779in}}{\pgfqpoint{2.542906in}{0.969507in}}{\pgfqpoint{2.551142in}{0.969507in}}%
\pgfpathclose%
\pgfusepath{stroke,fill}%
\end{pgfscope}%
\begin{pgfscope}%
\pgfpathrectangle{\pgfqpoint{0.100000in}{0.212622in}}{\pgfqpoint{3.696000in}{3.696000in}}%
\pgfusepath{clip}%
\pgfsetbuttcap%
\pgfsetroundjoin%
\definecolor{currentfill}{rgb}{0.121569,0.466667,0.705882}%
\pgfsetfillcolor{currentfill}%
\pgfsetfillopacity{0.981948}%
\pgfsetlinewidth{1.003750pt}%
\definecolor{currentstroke}{rgb}{0.121569,0.466667,0.705882}%
\pgfsetstrokecolor{currentstroke}%
\pgfsetstrokeopacity{0.981948}%
\pgfsetdash{}{0pt}%
\pgfpathmoveto{\pgfqpoint{2.218967in}{0.914289in}}%
\pgfpathcurveto{\pgfqpoint{2.227203in}{0.914289in}}{\pgfqpoint{2.235103in}{0.917561in}}{\pgfqpoint{2.240927in}{0.923385in}}%
\pgfpathcurveto{\pgfqpoint{2.246751in}{0.929209in}}{\pgfqpoint{2.250023in}{0.937109in}}{\pgfqpoint{2.250023in}{0.945345in}}%
\pgfpathcurveto{\pgfqpoint{2.250023in}{0.953582in}}{\pgfqpoint{2.246751in}{0.961482in}}{\pgfqpoint{2.240927in}{0.967306in}}%
\pgfpathcurveto{\pgfqpoint{2.235103in}{0.973129in}}{\pgfqpoint{2.227203in}{0.976402in}}{\pgfqpoint{2.218967in}{0.976402in}}%
\pgfpathcurveto{\pgfqpoint{2.210731in}{0.976402in}}{\pgfqpoint{2.202831in}{0.973129in}}{\pgfqpoint{2.197007in}{0.967306in}}%
\pgfpathcurveto{\pgfqpoint{2.191183in}{0.961482in}}{\pgfqpoint{2.187910in}{0.953582in}}{\pgfqpoint{2.187910in}{0.945345in}}%
\pgfpathcurveto{\pgfqpoint{2.187910in}{0.937109in}}{\pgfqpoint{2.191183in}{0.929209in}}{\pgfqpoint{2.197007in}{0.923385in}}%
\pgfpathcurveto{\pgfqpoint{2.202831in}{0.917561in}}{\pgfqpoint{2.210731in}{0.914289in}}{\pgfqpoint{2.218967in}{0.914289in}}%
\pgfpathclose%
\pgfusepath{stroke,fill}%
\end{pgfscope}%
\begin{pgfscope}%
\pgfpathrectangle{\pgfqpoint{0.100000in}{0.212622in}}{\pgfqpoint{3.696000in}{3.696000in}}%
\pgfusepath{clip}%
\pgfsetbuttcap%
\pgfsetroundjoin%
\definecolor{currentfill}{rgb}{0.121569,0.466667,0.705882}%
\pgfsetfillcolor{currentfill}%
\pgfsetfillopacity{0.984175}%
\pgfsetlinewidth{1.003750pt}%
\definecolor{currentstroke}{rgb}{0.121569,0.466667,0.705882}%
\pgfsetstrokecolor{currentstroke}%
\pgfsetstrokeopacity{0.984175}%
\pgfsetdash{}{0pt}%
\pgfpathmoveto{\pgfqpoint{2.233862in}{0.912685in}}%
\pgfpathcurveto{\pgfqpoint{2.242099in}{0.912685in}}{\pgfqpoint{2.249999in}{0.915958in}}{\pgfqpoint{2.255823in}{0.921782in}}%
\pgfpathcurveto{\pgfqpoint{2.261647in}{0.927605in}}{\pgfqpoint{2.264919in}{0.935505in}}{\pgfqpoint{2.264919in}{0.943742in}}%
\pgfpathcurveto{\pgfqpoint{2.264919in}{0.951978in}}{\pgfqpoint{2.261647in}{0.959878in}}{\pgfqpoint{2.255823in}{0.965702in}}%
\pgfpathcurveto{\pgfqpoint{2.249999in}{0.971526in}}{\pgfqpoint{2.242099in}{0.974798in}}{\pgfqpoint{2.233862in}{0.974798in}}%
\pgfpathcurveto{\pgfqpoint{2.225626in}{0.974798in}}{\pgfqpoint{2.217726in}{0.971526in}}{\pgfqpoint{2.211902in}{0.965702in}}%
\pgfpathcurveto{\pgfqpoint{2.206078in}{0.959878in}}{\pgfqpoint{2.202806in}{0.951978in}}{\pgfqpoint{2.202806in}{0.943742in}}%
\pgfpathcurveto{\pgfqpoint{2.202806in}{0.935505in}}{\pgfqpoint{2.206078in}{0.927605in}}{\pgfqpoint{2.211902in}{0.921782in}}%
\pgfpathcurveto{\pgfqpoint{2.217726in}{0.915958in}}{\pgfqpoint{2.225626in}{0.912685in}}{\pgfqpoint{2.233862in}{0.912685in}}%
\pgfpathclose%
\pgfusepath{stroke,fill}%
\end{pgfscope}%
\begin{pgfscope}%
\pgfpathrectangle{\pgfqpoint{0.100000in}{0.212622in}}{\pgfqpoint{3.696000in}{3.696000in}}%
\pgfusepath{clip}%
\pgfsetbuttcap%
\pgfsetroundjoin%
\definecolor{currentfill}{rgb}{0.121569,0.466667,0.705882}%
\pgfsetfillcolor{currentfill}%
\pgfsetfillopacity{0.985792}%
\pgfsetlinewidth{1.003750pt}%
\definecolor{currentstroke}{rgb}{0.121569,0.466667,0.705882}%
\pgfsetstrokecolor{currentstroke}%
\pgfsetstrokeopacity{0.985792}%
\pgfsetdash{}{0pt}%
\pgfpathmoveto{\pgfqpoint{2.245604in}{0.911557in}}%
\pgfpathcurveto{\pgfqpoint{2.253840in}{0.911557in}}{\pgfqpoint{2.261741in}{0.914829in}}{\pgfqpoint{2.267564in}{0.920653in}}%
\pgfpathcurveto{\pgfqpoint{2.273388in}{0.926477in}}{\pgfqpoint{2.276661in}{0.934377in}}{\pgfqpoint{2.276661in}{0.942613in}}%
\pgfpathcurveto{\pgfqpoint{2.276661in}{0.950850in}}{\pgfqpoint{2.273388in}{0.958750in}}{\pgfqpoint{2.267564in}{0.964574in}}%
\pgfpathcurveto{\pgfqpoint{2.261741in}{0.970398in}}{\pgfqpoint{2.253840in}{0.973670in}}{\pgfqpoint{2.245604in}{0.973670in}}%
\pgfpathcurveto{\pgfqpoint{2.237368in}{0.973670in}}{\pgfqpoint{2.229468in}{0.970398in}}{\pgfqpoint{2.223644in}{0.964574in}}%
\pgfpathcurveto{\pgfqpoint{2.217820in}{0.958750in}}{\pgfqpoint{2.214548in}{0.950850in}}{\pgfqpoint{2.214548in}{0.942613in}}%
\pgfpathcurveto{\pgfqpoint{2.214548in}{0.934377in}}{\pgfqpoint{2.217820in}{0.926477in}}{\pgfqpoint{2.223644in}{0.920653in}}%
\pgfpathcurveto{\pgfqpoint{2.229468in}{0.914829in}}{\pgfqpoint{2.237368in}{0.911557in}}{\pgfqpoint{2.245604in}{0.911557in}}%
\pgfpathclose%
\pgfusepath{stroke,fill}%
\end{pgfscope}%
\begin{pgfscope}%
\pgfpathrectangle{\pgfqpoint{0.100000in}{0.212622in}}{\pgfqpoint{3.696000in}{3.696000in}}%
\pgfusepath{clip}%
\pgfsetbuttcap%
\pgfsetroundjoin%
\definecolor{currentfill}{rgb}{0.121569,0.466667,0.705882}%
\pgfsetfillcolor{currentfill}%
\pgfsetfillopacity{0.986926}%
\pgfsetlinewidth{1.003750pt}%
\definecolor{currentstroke}{rgb}{0.121569,0.466667,0.705882}%
\pgfsetstrokecolor{currentstroke}%
\pgfsetstrokeopacity{0.986926}%
\pgfsetdash{}{0pt}%
\pgfpathmoveto{\pgfqpoint{2.253806in}{0.910886in}}%
\pgfpathcurveto{\pgfqpoint{2.262042in}{0.910886in}}{\pgfqpoint{2.269942in}{0.914158in}}{\pgfqpoint{2.275766in}{0.919982in}}%
\pgfpathcurveto{\pgfqpoint{2.281590in}{0.925806in}}{\pgfqpoint{2.284862in}{0.933706in}}{\pgfqpoint{2.284862in}{0.941942in}}%
\pgfpathcurveto{\pgfqpoint{2.284862in}{0.950178in}}{\pgfqpoint{2.281590in}{0.958079in}}{\pgfqpoint{2.275766in}{0.963902in}}%
\pgfpathcurveto{\pgfqpoint{2.269942in}{0.969726in}}{\pgfqpoint{2.262042in}{0.972999in}}{\pgfqpoint{2.253806in}{0.972999in}}%
\pgfpathcurveto{\pgfqpoint{2.245570in}{0.972999in}}{\pgfqpoint{2.237669in}{0.969726in}}{\pgfqpoint{2.231846in}{0.963902in}}%
\pgfpathcurveto{\pgfqpoint{2.226022in}{0.958079in}}{\pgfqpoint{2.222749in}{0.950178in}}{\pgfqpoint{2.222749in}{0.941942in}}%
\pgfpathcurveto{\pgfqpoint{2.222749in}{0.933706in}}{\pgfqpoint{2.226022in}{0.925806in}}{\pgfqpoint{2.231846in}{0.919982in}}%
\pgfpathcurveto{\pgfqpoint{2.237669in}{0.914158in}}{\pgfqpoint{2.245570in}{0.910886in}}{\pgfqpoint{2.253806in}{0.910886in}}%
\pgfpathclose%
\pgfusepath{stroke,fill}%
\end{pgfscope}%
\begin{pgfscope}%
\pgfpathrectangle{\pgfqpoint{0.100000in}{0.212622in}}{\pgfqpoint{3.696000in}{3.696000in}}%
\pgfusepath{clip}%
\pgfsetbuttcap%
\pgfsetroundjoin%
\definecolor{currentfill}{rgb}{0.121569,0.466667,0.705882}%
\pgfsetfillcolor{currentfill}%
\pgfsetfillopacity{0.987204}%
\pgfsetlinewidth{1.003750pt}%
\definecolor{currentstroke}{rgb}{0.121569,0.466667,0.705882}%
\pgfsetstrokecolor{currentstroke}%
\pgfsetstrokeopacity{0.987204}%
\pgfsetdash{}{0pt}%
\pgfpathmoveto{\pgfqpoint{2.538803in}{0.953700in}}%
\pgfpathcurveto{\pgfqpoint{2.547039in}{0.953700in}}{\pgfqpoint{2.554939in}{0.956973in}}{\pgfqpoint{2.560763in}{0.962797in}}%
\pgfpathcurveto{\pgfqpoint{2.566587in}{0.968621in}}{\pgfqpoint{2.569859in}{0.976521in}}{\pgfqpoint{2.569859in}{0.984757in}}%
\pgfpathcurveto{\pgfqpoint{2.569859in}{0.992993in}}{\pgfqpoint{2.566587in}{1.000893in}}{\pgfqpoint{2.560763in}{1.006717in}}%
\pgfpathcurveto{\pgfqpoint{2.554939in}{1.012541in}}{\pgfqpoint{2.547039in}{1.015813in}}{\pgfqpoint{2.538803in}{1.015813in}}%
\pgfpathcurveto{\pgfqpoint{2.530567in}{1.015813in}}{\pgfqpoint{2.522667in}{1.012541in}}{\pgfqpoint{2.516843in}{1.006717in}}%
\pgfpathcurveto{\pgfqpoint{2.511019in}{1.000893in}}{\pgfqpoint{2.507746in}{0.992993in}}{\pgfqpoint{2.507746in}{0.984757in}}%
\pgfpathcurveto{\pgfqpoint{2.507746in}{0.976521in}}{\pgfqpoint{2.511019in}{0.968621in}}{\pgfqpoint{2.516843in}{0.962797in}}%
\pgfpathcurveto{\pgfqpoint{2.522667in}{0.956973in}}{\pgfqpoint{2.530567in}{0.953700in}}{\pgfqpoint{2.538803in}{0.953700in}}%
\pgfpathclose%
\pgfusepath{stroke,fill}%
\end{pgfscope}%
\begin{pgfscope}%
\pgfpathrectangle{\pgfqpoint{0.100000in}{0.212622in}}{\pgfqpoint{3.696000in}{3.696000in}}%
\pgfusepath{clip}%
\pgfsetbuttcap%
\pgfsetroundjoin%
\definecolor{currentfill}{rgb}{0.121569,0.466667,0.705882}%
\pgfsetfillcolor{currentfill}%
\pgfsetfillopacity{0.987563}%
\pgfsetlinewidth{1.003750pt}%
\definecolor{currentstroke}{rgb}{0.121569,0.466667,0.705882}%
\pgfsetstrokecolor{currentstroke}%
\pgfsetstrokeopacity{0.987563}%
\pgfsetdash{}{0pt}%
\pgfpathmoveto{\pgfqpoint{2.258658in}{0.910434in}}%
\pgfpathcurveto{\pgfqpoint{2.266895in}{0.910434in}}{\pgfqpoint{2.274795in}{0.913707in}}{\pgfqpoint{2.280619in}{0.919531in}}%
\pgfpathcurveto{\pgfqpoint{2.286443in}{0.925354in}}{\pgfqpoint{2.289715in}{0.933255in}}{\pgfqpoint{2.289715in}{0.941491in}}%
\pgfpathcurveto{\pgfqpoint{2.289715in}{0.949727in}}{\pgfqpoint{2.286443in}{0.957627in}}{\pgfqpoint{2.280619in}{0.963451in}}%
\pgfpathcurveto{\pgfqpoint{2.274795in}{0.969275in}}{\pgfqpoint{2.266895in}{0.972547in}}{\pgfqpoint{2.258658in}{0.972547in}}%
\pgfpathcurveto{\pgfqpoint{2.250422in}{0.972547in}}{\pgfqpoint{2.242522in}{0.969275in}}{\pgfqpoint{2.236698in}{0.963451in}}%
\pgfpathcurveto{\pgfqpoint{2.230874in}{0.957627in}}{\pgfqpoint{2.227602in}{0.949727in}}{\pgfqpoint{2.227602in}{0.941491in}}%
\pgfpathcurveto{\pgfqpoint{2.227602in}{0.933255in}}{\pgfqpoint{2.230874in}{0.925354in}}{\pgfqpoint{2.236698in}{0.919531in}}%
\pgfpathcurveto{\pgfqpoint{2.242522in}{0.913707in}}{\pgfqpoint{2.250422in}{0.910434in}}{\pgfqpoint{2.258658in}{0.910434in}}%
\pgfpathclose%
\pgfusepath{stroke,fill}%
\end{pgfscope}%
\begin{pgfscope}%
\pgfpathrectangle{\pgfqpoint{0.100000in}{0.212622in}}{\pgfqpoint{3.696000in}{3.696000in}}%
\pgfusepath{clip}%
\pgfsetbuttcap%
\pgfsetroundjoin%
\definecolor{currentfill}{rgb}{0.121569,0.466667,0.705882}%
\pgfsetfillcolor{currentfill}%
\pgfsetfillopacity{0.987959}%
\pgfsetlinewidth{1.003750pt}%
\definecolor{currentstroke}{rgb}{0.121569,0.466667,0.705882}%
\pgfsetstrokecolor{currentstroke}%
\pgfsetstrokeopacity{0.987959}%
\pgfsetdash{}{0pt}%
\pgfpathmoveto{\pgfqpoint{2.261708in}{0.910206in}}%
\pgfpathcurveto{\pgfqpoint{2.269945in}{0.910206in}}{\pgfqpoint{2.277845in}{0.913478in}}{\pgfqpoint{2.283669in}{0.919302in}}%
\pgfpathcurveto{\pgfqpoint{2.289492in}{0.925126in}}{\pgfqpoint{2.292765in}{0.933026in}}{\pgfqpoint{2.292765in}{0.941262in}}%
\pgfpathcurveto{\pgfqpoint{2.292765in}{0.949498in}}{\pgfqpoint{2.289492in}{0.957398in}}{\pgfqpoint{2.283669in}{0.963222in}}%
\pgfpathcurveto{\pgfqpoint{2.277845in}{0.969046in}}{\pgfqpoint{2.269945in}{0.972319in}}{\pgfqpoint{2.261708in}{0.972319in}}%
\pgfpathcurveto{\pgfqpoint{2.253472in}{0.972319in}}{\pgfqpoint{2.245572in}{0.969046in}}{\pgfqpoint{2.239748in}{0.963222in}}%
\pgfpathcurveto{\pgfqpoint{2.233924in}{0.957398in}}{\pgfqpoint{2.230652in}{0.949498in}}{\pgfqpoint{2.230652in}{0.941262in}}%
\pgfpathcurveto{\pgfqpoint{2.230652in}{0.933026in}}{\pgfqpoint{2.233924in}{0.925126in}}{\pgfqpoint{2.239748in}{0.919302in}}%
\pgfpathcurveto{\pgfqpoint{2.245572in}{0.913478in}}{\pgfqpoint{2.253472in}{0.910206in}}{\pgfqpoint{2.261708in}{0.910206in}}%
\pgfpathclose%
\pgfusepath{stroke,fill}%
\end{pgfscope}%
\begin{pgfscope}%
\pgfpathrectangle{\pgfqpoint{0.100000in}{0.212622in}}{\pgfqpoint{3.696000in}{3.696000in}}%
\pgfusepath{clip}%
\pgfsetbuttcap%
\pgfsetroundjoin%
\definecolor{currentfill}{rgb}{0.121569,0.466667,0.705882}%
\pgfsetfillcolor{currentfill}%
\pgfsetfillopacity{0.988076}%
\pgfsetlinewidth{1.003750pt}%
\definecolor{currentstroke}{rgb}{0.121569,0.466667,0.705882}%
\pgfsetstrokecolor{currentstroke}%
\pgfsetstrokeopacity{0.988076}%
\pgfsetdash{}{0pt}%
\pgfpathmoveto{\pgfqpoint{2.262671in}{0.910136in}}%
\pgfpathcurveto{\pgfqpoint{2.270907in}{0.910136in}}{\pgfqpoint{2.278807in}{0.913408in}}{\pgfqpoint{2.284631in}{0.919232in}}%
\pgfpathcurveto{\pgfqpoint{2.290455in}{0.925056in}}{\pgfqpoint{2.293727in}{0.932956in}}{\pgfqpoint{2.293727in}{0.941192in}}%
\pgfpathcurveto{\pgfqpoint{2.293727in}{0.949429in}}{\pgfqpoint{2.290455in}{0.957329in}}{\pgfqpoint{2.284631in}{0.963153in}}%
\pgfpathcurveto{\pgfqpoint{2.278807in}{0.968977in}}{\pgfqpoint{2.270907in}{0.972249in}}{\pgfqpoint{2.262671in}{0.972249in}}%
\pgfpathcurveto{\pgfqpoint{2.254434in}{0.972249in}}{\pgfqpoint{2.246534in}{0.968977in}}{\pgfqpoint{2.240710in}{0.963153in}}%
\pgfpathcurveto{\pgfqpoint{2.234887in}{0.957329in}}{\pgfqpoint{2.231614in}{0.949429in}}{\pgfqpoint{2.231614in}{0.941192in}}%
\pgfpathcurveto{\pgfqpoint{2.231614in}{0.932956in}}{\pgfqpoint{2.234887in}{0.925056in}}{\pgfqpoint{2.240710in}{0.919232in}}%
\pgfpathcurveto{\pgfqpoint{2.246534in}{0.913408in}}{\pgfqpoint{2.254434in}{0.910136in}}{\pgfqpoint{2.262671in}{0.910136in}}%
\pgfpathclose%
\pgfusepath{stroke,fill}%
\end{pgfscope}%
\begin{pgfscope}%
\pgfpathrectangle{\pgfqpoint{0.100000in}{0.212622in}}{\pgfqpoint{3.696000in}{3.696000in}}%
\pgfusepath{clip}%
\pgfsetbuttcap%
\pgfsetroundjoin%
\definecolor{currentfill}{rgb}{0.121569,0.466667,0.705882}%
\pgfsetfillcolor{currentfill}%
\pgfsetfillopacity{0.988280}%
\pgfsetlinewidth{1.003750pt}%
\definecolor{currentstroke}{rgb}{0.121569,0.466667,0.705882}%
\pgfsetstrokecolor{currentstroke}%
\pgfsetstrokeopacity{0.988280}%
\pgfsetdash{}{0pt}%
\pgfpathmoveto{\pgfqpoint{2.264426in}{0.910035in}}%
\pgfpathcurveto{\pgfqpoint{2.272663in}{0.910035in}}{\pgfqpoint{2.280563in}{0.913308in}}{\pgfqpoint{2.286387in}{0.919131in}}%
\pgfpathcurveto{\pgfqpoint{2.292210in}{0.924955in}}{\pgfqpoint{2.295483in}{0.932855in}}{\pgfqpoint{2.295483in}{0.941092in}}%
\pgfpathcurveto{\pgfqpoint{2.295483in}{0.949328in}}{\pgfqpoint{2.292210in}{0.957228in}}{\pgfqpoint{2.286387in}{0.963052in}}%
\pgfpathcurveto{\pgfqpoint{2.280563in}{0.968876in}}{\pgfqpoint{2.272663in}{0.972148in}}{\pgfqpoint{2.264426in}{0.972148in}}%
\pgfpathcurveto{\pgfqpoint{2.256190in}{0.972148in}}{\pgfqpoint{2.248290in}{0.968876in}}{\pgfqpoint{2.242466in}{0.963052in}}%
\pgfpathcurveto{\pgfqpoint{2.236642in}{0.957228in}}{\pgfqpoint{2.233370in}{0.949328in}}{\pgfqpoint{2.233370in}{0.941092in}}%
\pgfpathcurveto{\pgfqpoint{2.233370in}{0.932855in}}{\pgfqpoint{2.236642in}{0.924955in}}{\pgfqpoint{2.242466in}{0.919131in}}%
\pgfpathcurveto{\pgfqpoint{2.248290in}{0.913308in}}{\pgfqpoint{2.256190in}{0.910035in}}{\pgfqpoint{2.264426in}{0.910035in}}%
\pgfpathclose%
\pgfusepath{stroke,fill}%
\end{pgfscope}%
\begin{pgfscope}%
\pgfpathrectangle{\pgfqpoint{0.100000in}{0.212622in}}{\pgfqpoint{3.696000in}{3.696000in}}%
\pgfusepath{clip}%
\pgfsetbuttcap%
\pgfsetroundjoin%
\definecolor{currentfill}{rgb}{0.121569,0.466667,0.705882}%
\pgfsetfillcolor{currentfill}%
\pgfsetfillopacity{0.988630}%
\pgfsetlinewidth{1.003750pt}%
\definecolor{currentstroke}{rgb}{0.121569,0.466667,0.705882}%
\pgfsetstrokecolor{currentstroke}%
\pgfsetstrokeopacity{0.988630}%
\pgfsetdash{}{0pt}%
\pgfpathmoveto{\pgfqpoint{2.267630in}{0.909895in}}%
\pgfpathcurveto{\pgfqpoint{2.275866in}{0.909895in}}{\pgfqpoint{2.283766in}{0.913167in}}{\pgfqpoint{2.289590in}{0.918991in}}%
\pgfpathcurveto{\pgfqpoint{2.295414in}{0.924815in}}{\pgfqpoint{2.298686in}{0.932715in}}{\pgfqpoint{2.298686in}{0.940951in}}%
\pgfpathcurveto{\pgfqpoint{2.298686in}{0.949187in}}{\pgfqpoint{2.295414in}{0.957088in}}{\pgfqpoint{2.289590in}{0.962911in}}%
\pgfpathcurveto{\pgfqpoint{2.283766in}{0.968735in}}{\pgfqpoint{2.275866in}{0.972008in}}{\pgfqpoint{2.267630in}{0.972008in}}%
\pgfpathcurveto{\pgfqpoint{2.259393in}{0.972008in}}{\pgfqpoint{2.251493in}{0.968735in}}{\pgfqpoint{2.245669in}{0.962911in}}%
\pgfpathcurveto{\pgfqpoint{2.239845in}{0.957088in}}{\pgfqpoint{2.236573in}{0.949187in}}{\pgfqpoint{2.236573in}{0.940951in}}%
\pgfpathcurveto{\pgfqpoint{2.236573in}{0.932715in}}{\pgfqpoint{2.239845in}{0.924815in}}{\pgfqpoint{2.245669in}{0.918991in}}%
\pgfpathcurveto{\pgfqpoint{2.251493in}{0.913167in}}{\pgfqpoint{2.259393in}{0.909895in}}{\pgfqpoint{2.267630in}{0.909895in}}%
\pgfpathclose%
\pgfusepath{stroke,fill}%
\end{pgfscope}%
\begin{pgfscope}%
\pgfpathrectangle{\pgfqpoint{0.100000in}{0.212622in}}{\pgfqpoint{3.696000in}{3.696000in}}%
\pgfusepath{clip}%
\pgfsetbuttcap%
\pgfsetroundjoin%
\definecolor{currentfill}{rgb}{0.121569,0.466667,0.705882}%
\pgfsetfillcolor{currentfill}%
\pgfsetfillopacity{0.989244}%
\pgfsetlinewidth{1.003750pt}%
\definecolor{currentstroke}{rgb}{0.121569,0.466667,0.705882}%
\pgfsetstrokecolor{currentstroke}%
\pgfsetstrokeopacity{0.989244}%
\pgfsetdash{}{0pt}%
\pgfpathmoveto{\pgfqpoint{2.273468in}{0.909725in}}%
\pgfpathcurveto{\pgfqpoint{2.281704in}{0.909725in}}{\pgfqpoint{2.289604in}{0.912997in}}{\pgfqpoint{2.295428in}{0.918821in}}%
\pgfpathcurveto{\pgfqpoint{2.301252in}{0.924645in}}{\pgfqpoint{2.304524in}{0.932545in}}{\pgfqpoint{2.304524in}{0.940782in}}%
\pgfpathcurveto{\pgfqpoint{2.304524in}{0.949018in}}{\pgfqpoint{2.301252in}{0.956918in}}{\pgfqpoint{2.295428in}{0.962742in}}%
\pgfpathcurveto{\pgfqpoint{2.289604in}{0.968566in}}{\pgfqpoint{2.281704in}{0.971838in}}{\pgfqpoint{2.273468in}{0.971838in}}%
\pgfpathcurveto{\pgfqpoint{2.265231in}{0.971838in}}{\pgfqpoint{2.257331in}{0.968566in}}{\pgfqpoint{2.251507in}{0.962742in}}%
\pgfpathcurveto{\pgfqpoint{2.245684in}{0.956918in}}{\pgfqpoint{2.242411in}{0.949018in}}{\pgfqpoint{2.242411in}{0.940782in}}%
\pgfpathcurveto{\pgfqpoint{2.242411in}{0.932545in}}{\pgfqpoint{2.245684in}{0.924645in}}{\pgfqpoint{2.251507in}{0.918821in}}%
\pgfpathcurveto{\pgfqpoint{2.257331in}{0.912997in}}{\pgfqpoint{2.265231in}{0.909725in}}{\pgfqpoint{2.273468in}{0.909725in}}%
\pgfpathclose%
\pgfusepath{stroke,fill}%
\end{pgfscope}%
\begin{pgfscope}%
\pgfpathrectangle{\pgfqpoint{0.100000in}{0.212622in}}{\pgfqpoint{3.696000in}{3.696000in}}%
\pgfusepath{clip}%
\pgfsetbuttcap%
\pgfsetroundjoin%
\definecolor{currentfill}{rgb}{0.121569,0.466667,0.705882}%
\pgfsetfillcolor{currentfill}%
\pgfsetfillopacity{0.990255}%
\pgfsetlinewidth{1.003750pt}%
\definecolor{currentstroke}{rgb}{0.121569,0.466667,0.705882}%
\pgfsetstrokecolor{currentstroke}%
\pgfsetstrokeopacity{0.990255}%
\pgfsetdash{}{0pt}%
\pgfpathmoveto{\pgfqpoint{2.284120in}{0.909549in}}%
\pgfpathcurveto{\pgfqpoint{2.292356in}{0.909549in}}{\pgfqpoint{2.300256in}{0.912822in}}{\pgfqpoint{2.306080in}{0.918646in}}%
\pgfpathcurveto{\pgfqpoint{2.311904in}{0.924470in}}{\pgfqpoint{2.315176in}{0.932370in}}{\pgfqpoint{2.315176in}{0.940606in}}%
\pgfpathcurveto{\pgfqpoint{2.315176in}{0.948842in}}{\pgfqpoint{2.311904in}{0.956742in}}{\pgfqpoint{2.306080in}{0.962566in}}%
\pgfpathcurveto{\pgfqpoint{2.300256in}{0.968390in}}{\pgfqpoint{2.292356in}{0.971662in}}{\pgfqpoint{2.284120in}{0.971662in}}%
\pgfpathcurveto{\pgfqpoint{2.275884in}{0.971662in}}{\pgfqpoint{2.267984in}{0.968390in}}{\pgfqpoint{2.262160in}{0.962566in}}%
\pgfpathcurveto{\pgfqpoint{2.256336in}{0.956742in}}{\pgfqpoint{2.253063in}{0.948842in}}{\pgfqpoint{2.253063in}{0.940606in}}%
\pgfpathcurveto{\pgfqpoint{2.253063in}{0.932370in}}{\pgfqpoint{2.256336in}{0.924470in}}{\pgfqpoint{2.262160in}{0.918646in}}%
\pgfpathcurveto{\pgfqpoint{2.267984in}{0.912822in}}{\pgfqpoint{2.275884in}{0.909549in}}{\pgfqpoint{2.284120in}{0.909549in}}%
\pgfpathclose%
\pgfusepath{stroke,fill}%
\end{pgfscope}%
\begin{pgfscope}%
\pgfpathrectangle{\pgfqpoint{0.100000in}{0.212622in}}{\pgfqpoint{3.696000in}{3.696000in}}%
\pgfusepath{clip}%
\pgfsetbuttcap%
\pgfsetroundjoin%
\definecolor{currentfill}{rgb}{0.121569,0.466667,0.705882}%
\pgfsetfillcolor{currentfill}%
\pgfsetfillopacity{0.991952}%
\pgfsetlinewidth{1.003750pt}%
\definecolor{currentstroke}{rgb}{0.121569,0.466667,0.705882}%
\pgfsetstrokecolor{currentstroke}%
\pgfsetstrokeopacity{0.991952}%
\pgfsetdash{}{0pt}%
\pgfpathmoveto{\pgfqpoint{2.303526in}{0.909298in}}%
\pgfpathcurveto{\pgfqpoint{2.311762in}{0.909298in}}{\pgfqpoint{2.319663in}{0.912570in}}{\pgfqpoint{2.325486in}{0.918394in}}%
\pgfpathcurveto{\pgfqpoint{2.331310in}{0.924218in}}{\pgfqpoint{2.334583in}{0.932118in}}{\pgfqpoint{2.334583in}{0.940354in}}%
\pgfpathcurveto{\pgfqpoint{2.334583in}{0.948590in}}{\pgfqpoint{2.331310in}{0.956490in}}{\pgfqpoint{2.325486in}{0.962314in}}%
\pgfpathcurveto{\pgfqpoint{2.319663in}{0.968138in}}{\pgfqpoint{2.311762in}{0.971411in}}{\pgfqpoint{2.303526in}{0.971411in}}%
\pgfpathcurveto{\pgfqpoint{2.295290in}{0.971411in}}{\pgfqpoint{2.287390in}{0.968138in}}{\pgfqpoint{2.281566in}{0.962314in}}%
\pgfpathcurveto{\pgfqpoint{2.275742in}{0.956490in}}{\pgfqpoint{2.272470in}{0.948590in}}{\pgfqpoint{2.272470in}{0.940354in}}%
\pgfpathcurveto{\pgfqpoint{2.272470in}{0.932118in}}{\pgfqpoint{2.275742in}{0.924218in}}{\pgfqpoint{2.281566in}{0.918394in}}%
\pgfpathcurveto{\pgfqpoint{2.287390in}{0.912570in}}{\pgfqpoint{2.295290in}{0.909298in}}{\pgfqpoint{2.303526in}{0.909298in}}%
\pgfpathclose%
\pgfusepath{stroke,fill}%
\end{pgfscope}%
\begin{pgfscope}%
\pgfpathrectangle{\pgfqpoint{0.100000in}{0.212622in}}{\pgfqpoint{3.696000in}{3.696000in}}%
\pgfusepath{clip}%
\pgfsetbuttcap%
\pgfsetroundjoin%
\definecolor{currentfill}{rgb}{0.121569,0.466667,0.705882}%
\pgfsetfillcolor{currentfill}%
\pgfsetfillopacity{0.993419}%
\pgfsetlinewidth{1.003750pt}%
\definecolor{currentstroke}{rgb}{0.121569,0.466667,0.705882}%
\pgfsetstrokecolor{currentstroke}%
\pgfsetstrokeopacity{0.993419}%
\pgfsetdash{}{0pt}%
\pgfpathmoveto{\pgfqpoint{2.523411in}{0.937796in}}%
\pgfpathcurveto{\pgfqpoint{2.531647in}{0.937796in}}{\pgfqpoint{2.539548in}{0.941068in}}{\pgfqpoint{2.545371in}{0.946892in}}%
\pgfpathcurveto{\pgfqpoint{2.551195in}{0.952716in}}{\pgfqpoint{2.554468in}{0.960616in}}{\pgfqpoint{2.554468in}{0.968853in}}%
\pgfpathcurveto{\pgfqpoint{2.554468in}{0.977089in}}{\pgfqpoint{2.551195in}{0.984989in}}{\pgfqpoint{2.545371in}{0.990813in}}%
\pgfpathcurveto{\pgfqpoint{2.539548in}{0.996637in}}{\pgfqpoint{2.531647in}{0.999909in}}{\pgfqpoint{2.523411in}{0.999909in}}%
\pgfpathcurveto{\pgfqpoint{2.515175in}{0.999909in}}{\pgfqpoint{2.507275in}{0.996637in}}{\pgfqpoint{2.501451in}{0.990813in}}%
\pgfpathcurveto{\pgfqpoint{2.495627in}{0.984989in}}{\pgfqpoint{2.492355in}{0.977089in}}{\pgfqpoint{2.492355in}{0.968853in}}%
\pgfpathcurveto{\pgfqpoint{2.492355in}{0.960616in}}{\pgfqpoint{2.495627in}{0.952716in}}{\pgfqpoint{2.501451in}{0.946892in}}%
\pgfpathcurveto{\pgfqpoint{2.507275in}{0.941068in}}{\pgfqpoint{2.515175in}{0.937796in}}{\pgfqpoint{2.523411in}{0.937796in}}%
\pgfpathclose%
\pgfusepath{stroke,fill}%
\end{pgfscope}%
\begin{pgfscope}%
\pgfpathrectangle{\pgfqpoint{0.100000in}{0.212622in}}{\pgfqpoint{3.696000in}{3.696000in}}%
\pgfusepath{clip}%
\pgfsetbuttcap%
\pgfsetroundjoin%
\definecolor{currentfill}{rgb}{0.121569,0.466667,0.705882}%
\pgfsetfillcolor{currentfill}%
\pgfsetfillopacity{0.994600}%
\pgfsetlinewidth{1.003750pt}%
\definecolor{currentstroke}{rgb}{0.121569,0.466667,0.705882}%
\pgfsetstrokecolor{currentstroke}%
\pgfsetstrokeopacity{0.994600}%
\pgfsetdash{}{0pt}%
\pgfpathmoveto{\pgfqpoint{2.338934in}{0.909656in}}%
\pgfpathcurveto{\pgfqpoint{2.347171in}{0.909656in}}{\pgfqpoint{2.355071in}{0.912928in}}{\pgfqpoint{2.360895in}{0.918752in}}%
\pgfpathcurveto{\pgfqpoint{2.366718in}{0.924576in}}{\pgfqpoint{2.369991in}{0.932476in}}{\pgfqpoint{2.369991in}{0.940712in}}%
\pgfpathcurveto{\pgfqpoint{2.369991in}{0.948949in}}{\pgfqpoint{2.366718in}{0.956849in}}{\pgfqpoint{2.360895in}{0.962673in}}%
\pgfpathcurveto{\pgfqpoint{2.355071in}{0.968497in}}{\pgfqpoint{2.347171in}{0.971769in}}{\pgfqpoint{2.338934in}{0.971769in}}%
\pgfpathcurveto{\pgfqpoint{2.330698in}{0.971769in}}{\pgfqpoint{2.322798in}{0.968497in}}{\pgfqpoint{2.316974in}{0.962673in}}%
\pgfpathcurveto{\pgfqpoint{2.311150in}{0.956849in}}{\pgfqpoint{2.307878in}{0.948949in}}{\pgfqpoint{2.307878in}{0.940712in}}%
\pgfpathcurveto{\pgfqpoint{2.307878in}{0.932476in}}{\pgfqpoint{2.311150in}{0.924576in}}{\pgfqpoint{2.316974in}{0.918752in}}%
\pgfpathcurveto{\pgfqpoint{2.322798in}{0.912928in}}{\pgfqpoint{2.330698in}{0.909656in}}{\pgfqpoint{2.338934in}{0.909656in}}%
\pgfpathclose%
\pgfusepath{stroke,fill}%
\end{pgfscope}%
\begin{pgfscope}%
\pgfpathrectangle{\pgfqpoint{0.100000in}{0.212622in}}{\pgfqpoint{3.696000in}{3.696000in}}%
\pgfusepath{clip}%
\pgfsetbuttcap%
\pgfsetroundjoin%
\definecolor{currentfill}{rgb}{0.121569,0.466667,0.705882}%
\pgfsetfillcolor{currentfill}%
\pgfsetfillopacity{0.996507}%
\pgfsetlinewidth{1.003750pt}%
\definecolor{currentstroke}{rgb}{0.121569,0.466667,0.705882}%
\pgfsetstrokecolor{currentstroke}%
\pgfsetstrokeopacity{0.996507}%
\pgfsetdash{}{0pt}%
\pgfpathmoveto{\pgfqpoint{2.513705in}{0.929962in}}%
\pgfpathcurveto{\pgfqpoint{2.521942in}{0.929962in}}{\pgfqpoint{2.529842in}{0.933235in}}{\pgfqpoint{2.535666in}{0.939059in}}%
\pgfpathcurveto{\pgfqpoint{2.541490in}{0.944882in}}{\pgfqpoint{2.544762in}{0.952783in}}{\pgfqpoint{2.544762in}{0.961019in}}%
\pgfpathcurveto{\pgfqpoint{2.544762in}{0.969255in}}{\pgfqpoint{2.541490in}{0.977155in}}{\pgfqpoint{2.535666in}{0.982979in}}%
\pgfpathcurveto{\pgfqpoint{2.529842in}{0.988803in}}{\pgfqpoint{2.521942in}{0.992075in}}{\pgfqpoint{2.513705in}{0.992075in}}%
\pgfpathcurveto{\pgfqpoint{2.505469in}{0.992075in}}{\pgfqpoint{2.497569in}{0.988803in}}{\pgfqpoint{2.491745in}{0.982979in}}%
\pgfpathcurveto{\pgfqpoint{2.485921in}{0.977155in}}{\pgfqpoint{2.482649in}{0.969255in}}{\pgfqpoint{2.482649in}{0.961019in}}%
\pgfpathcurveto{\pgfqpoint{2.482649in}{0.952783in}}{\pgfqpoint{2.485921in}{0.944882in}}{\pgfqpoint{2.491745in}{0.939059in}}%
\pgfpathcurveto{\pgfqpoint{2.497569in}{0.933235in}}{\pgfqpoint{2.505469in}{0.929962in}}{\pgfqpoint{2.513705in}{0.929962in}}%
\pgfpathclose%
\pgfusepath{stroke,fill}%
\end{pgfscope}%
\begin{pgfscope}%
\pgfpathrectangle{\pgfqpoint{0.100000in}{0.212622in}}{\pgfqpoint{3.696000in}{3.696000in}}%
\pgfusepath{clip}%
\pgfsetbuttcap%
\pgfsetroundjoin%
\definecolor{currentfill}{rgb}{0.121569,0.466667,0.705882}%
\pgfsetfillcolor{currentfill}%
\pgfsetfillopacity{0.996703}%
\pgfsetlinewidth{1.003750pt}%
\definecolor{currentstroke}{rgb}{0.121569,0.466667,0.705882}%
\pgfsetstrokecolor{currentstroke}%
\pgfsetstrokeopacity{0.996703}%
\pgfsetdash{}{0pt}%
\pgfpathmoveto{\pgfqpoint{2.372014in}{0.910438in}}%
\pgfpathcurveto{\pgfqpoint{2.380250in}{0.910438in}}{\pgfqpoint{2.388150in}{0.913711in}}{\pgfqpoint{2.393974in}{0.919534in}}%
\pgfpathcurveto{\pgfqpoint{2.399798in}{0.925358in}}{\pgfqpoint{2.403070in}{0.933258in}}{\pgfqpoint{2.403070in}{0.941495in}}%
\pgfpathcurveto{\pgfqpoint{2.403070in}{0.949731in}}{\pgfqpoint{2.399798in}{0.957631in}}{\pgfqpoint{2.393974in}{0.963455in}}%
\pgfpathcurveto{\pgfqpoint{2.388150in}{0.969279in}}{\pgfqpoint{2.380250in}{0.972551in}}{\pgfqpoint{2.372014in}{0.972551in}}%
\pgfpathcurveto{\pgfqpoint{2.363777in}{0.972551in}}{\pgfqpoint{2.355877in}{0.969279in}}{\pgfqpoint{2.350053in}{0.963455in}}%
\pgfpathcurveto{\pgfqpoint{2.344230in}{0.957631in}}{\pgfqpoint{2.340957in}{0.949731in}}{\pgfqpoint{2.340957in}{0.941495in}}%
\pgfpathcurveto{\pgfqpoint{2.340957in}{0.933258in}}{\pgfqpoint{2.344230in}{0.925358in}}{\pgfqpoint{2.350053in}{0.919534in}}%
\pgfpathcurveto{\pgfqpoint{2.355877in}{0.913711in}}{\pgfqpoint{2.363777in}{0.910438in}}{\pgfqpoint{2.372014in}{0.910438in}}%
\pgfpathclose%
\pgfusepath{stroke,fill}%
\end{pgfscope}%
\begin{pgfscope}%
\pgfpathrectangle{\pgfqpoint{0.100000in}{0.212622in}}{\pgfqpoint{3.696000in}{3.696000in}}%
\pgfusepath{clip}%
\pgfsetbuttcap%
\pgfsetroundjoin%
\definecolor{currentfill}{rgb}{0.121569,0.466667,0.705882}%
\pgfsetfillcolor{currentfill}%
\pgfsetfillopacity{0.998010}%
\pgfsetlinewidth{1.003750pt}%
\definecolor{currentstroke}{rgb}{0.121569,0.466667,0.705882}%
\pgfsetstrokecolor{currentstroke}%
\pgfsetstrokeopacity{0.998010}%
\pgfsetdash{}{0pt}%
\pgfpathmoveto{\pgfqpoint{2.507853in}{0.926013in}}%
\pgfpathcurveto{\pgfqpoint{2.516090in}{0.926013in}}{\pgfqpoint{2.523990in}{0.929285in}}{\pgfqpoint{2.529814in}{0.935109in}}%
\pgfpathcurveto{\pgfqpoint{2.535638in}{0.940933in}}{\pgfqpoint{2.538910in}{0.948833in}}{\pgfqpoint{2.538910in}{0.957069in}}%
\pgfpathcurveto{\pgfqpoint{2.538910in}{0.965306in}}{\pgfqpoint{2.535638in}{0.973206in}}{\pgfqpoint{2.529814in}{0.979030in}}%
\pgfpathcurveto{\pgfqpoint{2.523990in}{0.984854in}}{\pgfqpoint{2.516090in}{0.988126in}}{\pgfqpoint{2.507853in}{0.988126in}}%
\pgfpathcurveto{\pgfqpoint{2.499617in}{0.988126in}}{\pgfqpoint{2.491717in}{0.984854in}}{\pgfqpoint{2.485893in}{0.979030in}}%
\pgfpathcurveto{\pgfqpoint{2.480069in}{0.973206in}}{\pgfqpoint{2.476797in}{0.965306in}}{\pgfqpoint{2.476797in}{0.957069in}}%
\pgfpathcurveto{\pgfqpoint{2.476797in}{0.948833in}}{\pgfqpoint{2.480069in}{0.940933in}}{\pgfqpoint{2.485893in}{0.935109in}}%
\pgfpathcurveto{\pgfqpoint{2.491717in}{0.929285in}}{\pgfqpoint{2.499617in}{0.926013in}}{\pgfqpoint{2.507853in}{0.926013in}}%
\pgfpathclose%
\pgfusepath{stroke,fill}%
\end{pgfscope}%
\begin{pgfscope}%
\pgfpathrectangle{\pgfqpoint{0.100000in}{0.212622in}}{\pgfqpoint{3.696000in}{3.696000in}}%
\pgfusepath{clip}%
\pgfsetbuttcap%
\pgfsetroundjoin%
\definecolor{currentfill}{rgb}{0.121569,0.466667,0.705882}%
\pgfsetfillcolor{currentfill}%
\pgfsetfillopacity{0.998318}%
\pgfsetlinewidth{1.003750pt}%
\definecolor{currentstroke}{rgb}{0.121569,0.466667,0.705882}%
\pgfsetstrokecolor{currentstroke}%
\pgfsetstrokeopacity{0.998318}%
\pgfsetdash{}{0pt}%
\pgfpathmoveto{\pgfqpoint{2.403079in}{0.911829in}}%
\pgfpathcurveto{\pgfqpoint{2.411316in}{0.911829in}}{\pgfqpoint{2.419216in}{0.915101in}}{\pgfqpoint{2.425040in}{0.920925in}}%
\pgfpathcurveto{\pgfqpoint{2.430863in}{0.926749in}}{\pgfqpoint{2.434136in}{0.934649in}}{\pgfqpoint{2.434136in}{0.942885in}}%
\pgfpathcurveto{\pgfqpoint{2.434136in}{0.951122in}}{\pgfqpoint{2.430863in}{0.959022in}}{\pgfqpoint{2.425040in}{0.964846in}}%
\pgfpathcurveto{\pgfqpoint{2.419216in}{0.970669in}}{\pgfqpoint{2.411316in}{0.973942in}}{\pgfqpoint{2.403079in}{0.973942in}}%
\pgfpathcurveto{\pgfqpoint{2.394843in}{0.973942in}}{\pgfqpoint{2.386943in}{0.970669in}}{\pgfqpoint{2.381119in}{0.964846in}}%
\pgfpathcurveto{\pgfqpoint{2.375295in}{0.959022in}}{\pgfqpoint{2.372023in}{0.951122in}}{\pgfqpoint{2.372023in}{0.942885in}}%
\pgfpathcurveto{\pgfqpoint{2.372023in}{0.934649in}}{\pgfqpoint{2.375295in}{0.926749in}}{\pgfqpoint{2.381119in}{0.920925in}}%
\pgfpathcurveto{\pgfqpoint{2.386943in}{0.915101in}}{\pgfqpoint{2.394843in}{0.911829in}}{\pgfqpoint{2.403079in}{0.911829in}}%
\pgfpathclose%
\pgfusepath{stroke,fill}%
\end{pgfscope}%
\begin{pgfscope}%
\pgfpathrectangle{\pgfqpoint{0.100000in}{0.212622in}}{\pgfqpoint{3.696000in}{3.696000in}}%
\pgfusepath{clip}%
\pgfsetbuttcap%
\pgfsetroundjoin%
\definecolor{currentfill}{rgb}{0.121569,0.466667,0.705882}%
\pgfsetfillcolor{currentfill}%
\pgfsetfillopacity{0.998734}%
\pgfsetlinewidth{1.003750pt}%
\definecolor{currentstroke}{rgb}{0.121569,0.466667,0.705882}%
\pgfsetstrokecolor{currentstroke}%
\pgfsetstrokeopacity{0.998734}%
\pgfsetdash{}{0pt}%
\pgfpathmoveto{\pgfqpoint{2.504400in}{0.924045in}}%
\pgfpathcurveto{\pgfqpoint{2.512636in}{0.924045in}}{\pgfqpoint{2.520536in}{0.927317in}}{\pgfqpoint{2.526360in}{0.933141in}}%
\pgfpathcurveto{\pgfqpoint{2.532184in}{0.938965in}}{\pgfqpoint{2.535456in}{0.946865in}}{\pgfqpoint{2.535456in}{0.955101in}}%
\pgfpathcurveto{\pgfqpoint{2.535456in}{0.963338in}}{\pgfqpoint{2.532184in}{0.971238in}}{\pgfqpoint{2.526360in}{0.977062in}}%
\pgfpathcurveto{\pgfqpoint{2.520536in}{0.982886in}}{\pgfqpoint{2.512636in}{0.986158in}}{\pgfqpoint{2.504400in}{0.986158in}}%
\pgfpathcurveto{\pgfqpoint{2.496164in}{0.986158in}}{\pgfqpoint{2.488264in}{0.982886in}}{\pgfqpoint{2.482440in}{0.977062in}}%
\pgfpathcurveto{\pgfqpoint{2.476616in}{0.971238in}}{\pgfqpoint{2.473343in}{0.963338in}}{\pgfqpoint{2.473343in}{0.955101in}}%
\pgfpathcurveto{\pgfqpoint{2.473343in}{0.946865in}}{\pgfqpoint{2.476616in}{0.938965in}}{\pgfqpoint{2.482440in}{0.933141in}}%
\pgfpathcurveto{\pgfqpoint{2.488264in}{0.927317in}}{\pgfqpoint{2.496164in}{0.924045in}}{\pgfqpoint{2.504400in}{0.924045in}}%
\pgfpathclose%
\pgfusepath{stroke,fill}%
\end{pgfscope}%
\begin{pgfscope}%
\pgfpathrectangle{\pgfqpoint{0.100000in}{0.212622in}}{\pgfqpoint{3.696000in}{3.696000in}}%
\pgfusepath{clip}%
\pgfsetbuttcap%
\pgfsetroundjoin%
\definecolor{currentfill}{rgb}{0.121569,0.466667,0.705882}%
\pgfsetfillcolor{currentfill}%
\pgfsetfillopacity{0.999077}%
\pgfsetlinewidth{1.003750pt}%
\definecolor{currentstroke}{rgb}{0.121569,0.466667,0.705882}%
\pgfsetstrokecolor{currentstroke}%
\pgfsetstrokeopacity{0.999077}%
\pgfsetdash{}{0pt}%
\pgfpathmoveto{\pgfqpoint{2.502389in}{0.923086in}}%
\pgfpathcurveto{\pgfqpoint{2.510625in}{0.923086in}}{\pgfqpoint{2.518525in}{0.926358in}}{\pgfqpoint{2.524349in}{0.932182in}}%
\pgfpathcurveto{\pgfqpoint{2.530173in}{0.938006in}}{\pgfqpoint{2.533445in}{0.945906in}}{\pgfqpoint{2.533445in}{0.954142in}}%
\pgfpathcurveto{\pgfqpoint{2.533445in}{0.962378in}}{\pgfqpoint{2.530173in}{0.970278in}}{\pgfqpoint{2.524349in}{0.976102in}}%
\pgfpathcurveto{\pgfqpoint{2.518525in}{0.981926in}}{\pgfqpoint{2.510625in}{0.985199in}}{\pgfqpoint{2.502389in}{0.985199in}}%
\pgfpathcurveto{\pgfqpoint{2.494152in}{0.985199in}}{\pgfqpoint{2.486252in}{0.981926in}}{\pgfqpoint{2.480428in}{0.976102in}}%
\pgfpathcurveto{\pgfqpoint{2.474605in}{0.970278in}}{\pgfqpoint{2.471332in}{0.962378in}}{\pgfqpoint{2.471332in}{0.954142in}}%
\pgfpathcurveto{\pgfqpoint{2.471332in}{0.945906in}}{\pgfqpoint{2.474605in}{0.938006in}}{\pgfqpoint{2.480428in}{0.932182in}}%
\pgfpathcurveto{\pgfqpoint{2.486252in}{0.926358in}}{\pgfqpoint{2.494152in}{0.923086in}}{\pgfqpoint{2.502389in}{0.923086in}}%
\pgfpathclose%
\pgfusepath{stroke,fill}%
\end{pgfscope}%
\begin{pgfscope}%
\pgfpathrectangle{\pgfqpoint{0.100000in}{0.212622in}}{\pgfqpoint{3.696000in}{3.696000in}}%
\pgfusepath{clip}%
\pgfsetbuttcap%
\pgfsetroundjoin%
\definecolor{currentfill}{rgb}{0.121569,0.466667,0.705882}%
\pgfsetfillcolor{currentfill}%
\pgfsetfillopacity{0.999233}%
\pgfsetlinewidth{1.003750pt}%
\definecolor{currentstroke}{rgb}{0.121569,0.466667,0.705882}%
\pgfsetstrokecolor{currentstroke}%
\pgfsetstrokeopacity{0.999233}%
\pgfsetdash{}{0pt}%
\pgfpathmoveto{\pgfqpoint{2.501231in}{0.922619in}}%
\pgfpathcurveto{\pgfqpoint{2.509467in}{0.922619in}}{\pgfqpoint{2.517368in}{0.925891in}}{\pgfqpoint{2.523191in}{0.931715in}}%
\pgfpathcurveto{\pgfqpoint{2.529015in}{0.937539in}}{\pgfqpoint{2.532288in}{0.945439in}}{\pgfqpoint{2.532288in}{0.953675in}}%
\pgfpathcurveto{\pgfqpoint{2.532288in}{0.961912in}}{\pgfqpoint{2.529015in}{0.969812in}}{\pgfqpoint{2.523191in}{0.975636in}}%
\pgfpathcurveto{\pgfqpoint{2.517368in}{0.981460in}}{\pgfqpoint{2.509467in}{0.984732in}}{\pgfqpoint{2.501231in}{0.984732in}}%
\pgfpathcurveto{\pgfqpoint{2.492995in}{0.984732in}}{\pgfqpoint{2.485095in}{0.981460in}}{\pgfqpoint{2.479271in}{0.975636in}}%
\pgfpathcurveto{\pgfqpoint{2.473447in}{0.969812in}}{\pgfqpoint{2.470175in}{0.961912in}}{\pgfqpoint{2.470175in}{0.953675in}}%
\pgfpathcurveto{\pgfqpoint{2.470175in}{0.945439in}}{\pgfqpoint{2.473447in}{0.937539in}}{\pgfqpoint{2.479271in}{0.931715in}}%
\pgfpathcurveto{\pgfqpoint{2.485095in}{0.925891in}}{\pgfqpoint{2.492995in}{0.922619in}}{\pgfqpoint{2.501231in}{0.922619in}}%
\pgfpathclose%
\pgfusepath{stroke,fill}%
\end{pgfscope}%
\begin{pgfscope}%
\pgfpathrectangle{\pgfqpoint{0.100000in}{0.212622in}}{\pgfqpoint{3.696000in}{3.696000in}}%
\pgfusepath{clip}%
\pgfsetbuttcap%
\pgfsetroundjoin%
\definecolor{currentfill}{rgb}{0.121569,0.466667,0.705882}%
\pgfsetfillcolor{currentfill}%
\pgfsetfillopacity{0.999303}%
\pgfsetlinewidth{1.003750pt}%
\definecolor{currentstroke}{rgb}{0.121569,0.466667,0.705882}%
\pgfsetstrokecolor{currentstroke}%
\pgfsetstrokeopacity{0.999303}%
\pgfsetdash{}{0pt}%
\pgfpathmoveto{\pgfqpoint{2.500572in}{0.922393in}}%
\pgfpathcurveto{\pgfqpoint{2.508808in}{0.922393in}}{\pgfqpoint{2.516708in}{0.925665in}}{\pgfqpoint{2.522532in}{0.931489in}}%
\pgfpathcurveto{\pgfqpoint{2.528356in}{0.937313in}}{\pgfqpoint{2.531629in}{0.945213in}}{\pgfqpoint{2.531629in}{0.953449in}}%
\pgfpathcurveto{\pgfqpoint{2.531629in}{0.961686in}}{\pgfqpoint{2.528356in}{0.969586in}}{\pgfqpoint{2.522532in}{0.975410in}}%
\pgfpathcurveto{\pgfqpoint{2.516708in}{0.981233in}}{\pgfqpoint{2.508808in}{0.984506in}}{\pgfqpoint{2.500572in}{0.984506in}}%
\pgfpathcurveto{\pgfqpoint{2.492336in}{0.984506in}}{\pgfqpoint{2.484436in}{0.981233in}}{\pgfqpoint{2.478612in}{0.975410in}}%
\pgfpathcurveto{\pgfqpoint{2.472788in}{0.969586in}}{\pgfqpoint{2.469516in}{0.961686in}}{\pgfqpoint{2.469516in}{0.953449in}}%
\pgfpathcurveto{\pgfqpoint{2.469516in}{0.945213in}}{\pgfqpoint{2.472788in}{0.937313in}}{\pgfqpoint{2.478612in}{0.931489in}}%
\pgfpathcurveto{\pgfqpoint{2.484436in}{0.925665in}}{\pgfqpoint{2.492336in}{0.922393in}}{\pgfqpoint{2.500572in}{0.922393in}}%
\pgfpathclose%
\pgfusepath{stroke,fill}%
\end{pgfscope}%
\begin{pgfscope}%
\pgfpathrectangle{\pgfqpoint{0.100000in}{0.212622in}}{\pgfqpoint{3.696000in}{3.696000in}}%
\pgfusepath{clip}%
\pgfsetbuttcap%
\pgfsetroundjoin%
\definecolor{currentfill}{rgb}{0.121569,0.466667,0.705882}%
\pgfsetfillcolor{currentfill}%
\pgfsetfillopacity{0.999318}%
\pgfsetlinewidth{1.003750pt}%
\definecolor{currentstroke}{rgb}{0.121569,0.466667,0.705882}%
\pgfsetstrokecolor{currentstroke}%
\pgfsetstrokeopacity{0.999318}%
\pgfsetdash{}{0pt}%
\pgfpathmoveto{\pgfqpoint{2.430446in}{0.913133in}}%
\pgfpathcurveto{\pgfqpoint{2.438683in}{0.913133in}}{\pgfqpoint{2.446583in}{0.916406in}}{\pgfqpoint{2.452407in}{0.922230in}}%
\pgfpathcurveto{\pgfqpoint{2.458231in}{0.928054in}}{\pgfqpoint{2.461503in}{0.935954in}}{\pgfqpoint{2.461503in}{0.944190in}}%
\pgfpathcurveto{\pgfqpoint{2.461503in}{0.952426in}}{\pgfqpoint{2.458231in}{0.960326in}}{\pgfqpoint{2.452407in}{0.966150in}}%
\pgfpathcurveto{\pgfqpoint{2.446583in}{0.971974in}}{\pgfqpoint{2.438683in}{0.975246in}}{\pgfqpoint{2.430446in}{0.975246in}}%
\pgfpathcurveto{\pgfqpoint{2.422210in}{0.975246in}}{\pgfqpoint{2.414310in}{0.971974in}}{\pgfqpoint{2.408486in}{0.966150in}}%
\pgfpathcurveto{\pgfqpoint{2.402662in}{0.960326in}}{\pgfqpoint{2.399390in}{0.952426in}}{\pgfqpoint{2.399390in}{0.944190in}}%
\pgfpathcurveto{\pgfqpoint{2.399390in}{0.935954in}}{\pgfqpoint{2.402662in}{0.928054in}}{\pgfqpoint{2.408486in}{0.922230in}}%
\pgfpathcurveto{\pgfqpoint{2.414310in}{0.916406in}}{\pgfqpoint{2.422210in}{0.913133in}}{\pgfqpoint{2.430446in}{0.913133in}}%
\pgfpathclose%
\pgfusepath{stroke,fill}%
\end{pgfscope}%
\begin{pgfscope}%
\pgfpathrectangle{\pgfqpoint{0.100000in}{0.212622in}}{\pgfqpoint{3.696000in}{3.696000in}}%
\pgfusepath{clip}%
\pgfsetbuttcap%
\pgfsetroundjoin%
\definecolor{currentfill}{rgb}{0.121569,0.466667,0.705882}%
\pgfsetfillcolor{currentfill}%
\pgfsetfillopacity{0.999332}%
\pgfsetlinewidth{1.003750pt}%
\definecolor{currentstroke}{rgb}{0.121569,0.466667,0.705882}%
\pgfsetstrokecolor{currentstroke}%
\pgfsetstrokeopacity{0.999332}%
\pgfsetdash{}{0pt}%
\pgfpathmoveto{\pgfqpoint{2.500200in}{0.922285in}}%
\pgfpathcurveto{\pgfqpoint{2.508436in}{0.922285in}}{\pgfqpoint{2.516336in}{0.925557in}}{\pgfqpoint{2.522160in}{0.931381in}}%
\pgfpathcurveto{\pgfqpoint{2.527984in}{0.937205in}}{\pgfqpoint{2.531256in}{0.945105in}}{\pgfqpoint{2.531256in}{0.953342in}}%
\pgfpathcurveto{\pgfqpoint{2.531256in}{0.961578in}}{\pgfqpoint{2.527984in}{0.969478in}}{\pgfqpoint{2.522160in}{0.975302in}}%
\pgfpathcurveto{\pgfqpoint{2.516336in}{0.981126in}}{\pgfqpoint{2.508436in}{0.984398in}}{\pgfqpoint{2.500200in}{0.984398in}}%
\pgfpathcurveto{\pgfqpoint{2.491963in}{0.984398in}}{\pgfqpoint{2.484063in}{0.981126in}}{\pgfqpoint{2.478239in}{0.975302in}}%
\pgfpathcurveto{\pgfqpoint{2.472415in}{0.969478in}}{\pgfqpoint{2.469143in}{0.961578in}}{\pgfqpoint{2.469143in}{0.953342in}}%
\pgfpathcurveto{\pgfqpoint{2.469143in}{0.945105in}}{\pgfqpoint{2.472415in}{0.937205in}}{\pgfqpoint{2.478239in}{0.931381in}}%
\pgfpathcurveto{\pgfqpoint{2.484063in}{0.925557in}}{\pgfqpoint{2.491963in}{0.922285in}}{\pgfqpoint{2.500200in}{0.922285in}}%
\pgfpathclose%
\pgfusepath{stroke,fill}%
\end{pgfscope}%
\begin{pgfscope}%
\pgfpathrectangle{\pgfqpoint{0.100000in}{0.212622in}}{\pgfqpoint{3.696000in}{3.696000in}}%
\pgfusepath{clip}%
\pgfsetbuttcap%
\pgfsetroundjoin%
\definecolor{currentfill}{rgb}{0.121569,0.466667,0.705882}%
\pgfsetfillcolor{currentfill}%
\pgfsetfillopacity{0.999573}%
\pgfsetlinewidth{1.003750pt}%
\definecolor{currentstroke}{rgb}{0.121569,0.466667,0.705882}%
\pgfsetstrokecolor{currentstroke}%
\pgfsetstrokeopacity{0.999573}%
\pgfsetdash{}{0pt}%
\pgfpathmoveto{\pgfqpoint{2.496447in}{0.921465in}}%
\pgfpathcurveto{\pgfqpoint{2.504683in}{0.921465in}}{\pgfqpoint{2.512584in}{0.924737in}}{\pgfqpoint{2.518407in}{0.930561in}}%
\pgfpathcurveto{\pgfqpoint{2.524231in}{0.936385in}}{\pgfqpoint{2.527504in}{0.944285in}}{\pgfqpoint{2.527504in}{0.952521in}}%
\pgfpathcurveto{\pgfqpoint{2.527504in}{0.960758in}}{\pgfqpoint{2.524231in}{0.968658in}}{\pgfqpoint{2.518407in}{0.974482in}}%
\pgfpathcurveto{\pgfqpoint{2.512584in}{0.980306in}}{\pgfqpoint{2.504683in}{0.983578in}}{\pgfqpoint{2.496447in}{0.983578in}}%
\pgfpathcurveto{\pgfqpoint{2.488211in}{0.983578in}}{\pgfqpoint{2.480311in}{0.980306in}}{\pgfqpoint{2.474487in}{0.974482in}}%
\pgfpathcurveto{\pgfqpoint{2.468663in}{0.968658in}}{\pgfqpoint{2.465391in}{0.960758in}}{\pgfqpoint{2.465391in}{0.952521in}}%
\pgfpathcurveto{\pgfqpoint{2.465391in}{0.944285in}}{\pgfqpoint{2.468663in}{0.936385in}}{\pgfqpoint{2.474487in}{0.930561in}}%
\pgfpathcurveto{\pgfqpoint{2.480311in}{0.924737in}}{\pgfqpoint{2.488211in}{0.921465in}}{\pgfqpoint{2.496447in}{0.921465in}}%
\pgfpathclose%
\pgfusepath{stroke,fill}%
\end{pgfscope}%
\begin{pgfscope}%
\pgfpathrectangle{\pgfqpoint{0.100000in}{0.212622in}}{\pgfqpoint{3.696000in}{3.696000in}}%
\pgfusepath{clip}%
\pgfsetbuttcap%
\pgfsetroundjoin%
\definecolor{currentfill}{rgb}{0.121569,0.466667,0.705882}%
\pgfsetfillcolor{currentfill}%
\pgfsetfillopacity{0.999660}%
\pgfsetlinewidth{1.003750pt}%
\definecolor{currentstroke}{rgb}{0.121569,0.466667,0.705882}%
\pgfsetstrokecolor{currentstroke}%
\pgfsetstrokeopacity{0.999660}%
\pgfsetdash{}{0pt}%
\pgfpathmoveto{\pgfqpoint{2.494356in}{0.921029in}}%
\pgfpathcurveto{\pgfqpoint{2.502593in}{0.921029in}}{\pgfqpoint{2.510493in}{0.924301in}}{\pgfqpoint{2.516317in}{0.930125in}}%
\pgfpathcurveto{\pgfqpoint{2.522140in}{0.935949in}}{\pgfqpoint{2.525413in}{0.943849in}}{\pgfqpoint{2.525413in}{0.952085in}}%
\pgfpathcurveto{\pgfqpoint{2.525413in}{0.960322in}}{\pgfqpoint{2.522140in}{0.968222in}}{\pgfqpoint{2.516317in}{0.974046in}}%
\pgfpathcurveto{\pgfqpoint{2.510493in}{0.979870in}}{\pgfqpoint{2.502593in}{0.983142in}}{\pgfqpoint{2.494356in}{0.983142in}}%
\pgfpathcurveto{\pgfqpoint{2.486120in}{0.983142in}}{\pgfqpoint{2.478220in}{0.979870in}}{\pgfqpoint{2.472396in}{0.974046in}}%
\pgfpathcurveto{\pgfqpoint{2.466572in}{0.968222in}}{\pgfqpoint{2.463300in}{0.960322in}}{\pgfqpoint{2.463300in}{0.952085in}}%
\pgfpathcurveto{\pgfqpoint{2.463300in}{0.943849in}}{\pgfqpoint{2.466572in}{0.935949in}}{\pgfqpoint{2.472396in}{0.930125in}}%
\pgfpathcurveto{\pgfqpoint{2.478220in}{0.924301in}}{\pgfqpoint{2.486120in}{0.921029in}}{\pgfqpoint{2.494356in}{0.921029in}}%
\pgfpathclose%
\pgfusepath{stroke,fill}%
\end{pgfscope}%
\begin{pgfscope}%
\pgfpathrectangle{\pgfqpoint{0.100000in}{0.212622in}}{\pgfqpoint{3.696000in}{3.696000in}}%
\pgfusepath{clip}%
\pgfsetbuttcap%
\pgfsetroundjoin%
\definecolor{currentfill}{rgb}{0.121569,0.466667,0.705882}%
\pgfsetfillcolor{currentfill}%
\pgfsetfillopacity{0.999931}%
\pgfsetlinewidth{1.003750pt}%
\definecolor{currentstroke}{rgb}{0.121569,0.466667,0.705882}%
\pgfsetstrokecolor{currentstroke}%
\pgfsetstrokeopacity{0.999931}%
\pgfsetdash{}{0pt}%
\pgfpathmoveto{\pgfqpoint{2.486221in}{0.919521in}}%
\pgfpathcurveto{\pgfqpoint{2.494457in}{0.919521in}}{\pgfqpoint{2.502357in}{0.922793in}}{\pgfqpoint{2.508181in}{0.928617in}}%
\pgfpathcurveto{\pgfqpoint{2.514005in}{0.934441in}}{\pgfqpoint{2.517277in}{0.942341in}}{\pgfqpoint{2.517277in}{0.950577in}}%
\pgfpathcurveto{\pgfqpoint{2.517277in}{0.958813in}}{\pgfqpoint{2.514005in}{0.966713in}}{\pgfqpoint{2.508181in}{0.972537in}}%
\pgfpathcurveto{\pgfqpoint{2.502357in}{0.978361in}}{\pgfqpoint{2.494457in}{0.981634in}}{\pgfqpoint{2.486221in}{0.981634in}}%
\pgfpathcurveto{\pgfqpoint{2.477984in}{0.981634in}}{\pgfqpoint{2.470084in}{0.978361in}}{\pgfqpoint{2.464260in}{0.972537in}}%
\pgfpathcurveto{\pgfqpoint{2.458436in}{0.966713in}}{\pgfqpoint{2.455164in}{0.958813in}}{\pgfqpoint{2.455164in}{0.950577in}}%
\pgfpathcurveto{\pgfqpoint{2.455164in}{0.942341in}}{\pgfqpoint{2.458436in}{0.934441in}}{\pgfqpoint{2.464260in}{0.928617in}}%
\pgfpathcurveto{\pgfqpoint{2.470084in}{0.922793in}}{\pgfqpoint{2.477984in}{0.919521in}}{\pgfqpoint{2.486221in}{0.919521in}}%
\pgfpathclose%
\pgfusepath{stroke,fill}%
\end{pgfscope}%
\begin{pgfscope}%
\pgfpathrectangle{\pgfqpoint{0.100000in}{0.212622in}}{\pgfqpoint{3.696000in}{3.696000in}}%
\pgfusepath{clip}%
\pgfsetbuttcap%
\pgfsetroundjoin%
\definecolor{currentfill}{rgb}{0.121569,0.466667,0.705882}%
\pgfsetfillcolor{currentfill}%
\pgfsetfillopacity{0.999950}%
\pgfsetlinewidth{1.003750pt}%
\definecolor{currentstroke}{rgb}{0.121569,0.466667,0.705882}%
\pgfsetstrokecolor{currentstroke}%
\pgfsetstrokeopacity{0.999950}%
\pgfsetdash{}{0pt}%
\pgfpathmoveto{\pgfqpoint{2.455177in}{0.915563in}}%
\pgfpathcurveto{\pgfqpoint{2.463413in}{0.915563in}}{\pgfqpoint{2.471314in}{0.918835in}}{\pgfqpoint{2.477137in}{0.924659in}}%
\pgfpathcurveto{\pgfqpoint{2.482961in}{0.930483in}}{\pgfqpoint{2.486234in}{0.938383in}}{\pgfqpoint{2.486234in}{0.946619in}}%
\pgfpathcurveto{\pgfqpoint{2.486234in}{0.954856in}}{\pgfqpoint{2.482961in}{0.962756in}}{\pgfqpoint{2.477137in}{0.968580in}}%
\pgfpathcurveto{\pgfqpoint{2.471314in}{0.974404in}}{\pgfqpoint{2.463413in}{0.977676in}}{\pgfqpoint{2.455177in}{0.977676in}}%
\pgfpathcurveto{\pgfqpoint{2.446941in}{0.977676in}}{\pgfqpoint{2.439041in}{0.974404in}}{\pgfqpoint{2.433217in}{0.968580in}}%
\pgfpathcurveto{\pgfqpoint{2.427393in}{0.962756in}}{\pgfqpoint{2.424121in}{0.954856in}}{\pgfqpoint{2.424121in}{0.946619in}}%
\pgfpathcurveto{\pgfqpoint{2.424121in}{0.938383in}}{\pgfqpoint{2.427393in}{0.930483in}}{\pgfqpoint{2.433217in}{0.924659in}}%
\pgfpathcurveto{\pgfqpoint{2.439041in}{0.918835in}}{\pgfqpoint{2.446941in}{0.915563in}}{\pgfqpoint{2.455177in}{0.915563in}}%
\pgfpathclose%
\pgfusepath{stroke,fill}%
\end{pgfscope}%
\begin{pgfscope}%
\pgfpathrectangle{\pgfqpoint{0.100000in}{0.212622in}}{\pgfqpoint{3.696000in}{3.696000in}}%
\pgfusepath{clip}%
\pgfsetbuttcap%
\pgfsetroundjoin%
\definecolor{currentfill}{rgb}{0.121569,0.466667,0.705882}%
\pgfsetfillcolor{currentfill}%
\pgfsetlinewidth{1.003750pt}%
\definecolor{currentstroke}{rgb}{0.121569,0.466667,0.705882}%
\pgfsetstrokecolor{currentstroke}%
\pgfsetdash{}{0pt}%
\pgfpathmoveto{\pgfqpoint{2.473685in}{0.917591in}}%
\pgfpathcurveto{\pgfqpoint{2.481921in}{0.917591in}}{\pgfqpoint{2.489821in}{0.920863in}}{\pgfqpoint{2.495645in}{0.926687in}}%
\pgfpathcurveto{\pgfqpoint{2.501469in}{0.932511in}}{\pgfqpoint{2.504741in}{0.940411in}}{\pgfqpoint{2.504741in}{0.948648in}}%
\pgfpathcurveto{\pgfqpoint{2.504741in}{0.956884in}}{\pgfqpoint{2.501469in}{0.964784in}}{\pgfqpoint{2.495645in}{0.970608in}}%
\pgfpathcurveto{\pgfqpoint{2.489821in}{0.976432in}}{\pgfqpoint{2.481921in}{0.979704in}}{\pgfqpoint{2.473685in}{0.979704in}}%
\pgfpathcurveto{\pgfqpoint{2.465449in}{0.979704in}}{\pgfqpoint{2.457549in}{0.976432in}}{\pgfqpoint{2.451725in}{0.970608in}}%
\pgfpathcurveto{\pgfqpoint{2.445901in}{0.964784in}}{\pgfqpoint{2.442628in}{0.956884in}}{\pgfqpoint{2.442628in}{0.948648in}}%
\pgfpathcurveto{\pgfqpoint{2.442628in}{0.940411in}}{\pgfqpoint{2.445901in}{0.932511in}}{\pgfqpoint{2.451725in}{0.926687in}}%
\pgfpathcurveto{\pgfqpoint{2.457549in}{0.920863in}}{\pgfqpoint{2.465449in}{0.917591in}}{\pgfqpoint{2.473685in}{0.917591in}}%
\pgfpathclose%
\pgfusepath{stroke,fill}%
\end{pgfscope}%
\begin{pgfscope}%
\pgfsetbuttcap%
\pgfsetmiterjoin%
\definecolor{currentfill}{rgb}{1.000000,1.000000,1.000000}%
\pgfsetfillcolor{currentfill}%
\pgfsetfillopacity{0.800000}%
\pgfsetlinewidth{1.003750pt}%
\definecolor{currentstroke}{rgb}{0.800000,0.800000,0.800000}%
\pgfsetstrokecolor{currentstroke}%
\pgfsetstrokeopacity{0.800000}%
\pgfsetdash{}{0pt}%
\pgfpathmoveto{\pgfqpoint{2.104889in}{3.216678in}}%
\pgfpathlineto{\pgfqpoint{3.698778in}{3.216678in}}%
\pgfpathquadraticcurveto{\pgfqpoint{3.726556in}{3.216678in}}{\pgfqpoint{3.726556in}{3.244456in}}%
\pgfpathlineto{\pgfqpoint{3.726556in}{3.811400in}}%
\pgfpathquadraticcurveto{\pgfqpoint{3.726556in}{3.839178in}}{\pgfqpoint{3.698778in}{3.839178in}}%
\pgfpathlineto{\pgfqpoint{2.104889in}{3.839178in}}%
\pgfpathquadraticcurveto{\pgfqpoint{2.077111in}{3.839178in}}{\pgfqpoint{2.077111in}{3.811400in}}%
\pgfpathlineto{\pgfqpoint{2.077111in}{3.244456in}}%
\pgfpathquadraticcurveto{\pgfqpoint{2.077111in}{3.216678in}}{\pgfqpoint{2.104889in}{3.216678in}}%
\pgfpathclose%
\pgfusepath{stroke,fill}%
\end{pgfscope}%
\begin{pgfscope}%
\pgfsetrectcap%
\pgfsetroundjoin%
\pgfsetlinewidth{1.505625pt}%
\definecolor{currentstroke}{rgb}{0.121569,0.466667,0.705882}%
\pgfsetstrokecolor{currentstroke}%
\pgfsetdash{}{0pt}%
\pgfpathmoveto{\pgfqpoint{2.132667in}{3.735011in}}%
\pgfpathlineto{\pgfqpoint{2.410444in}{3.735011in}}%
\pgfusepath{stroke}%
\end{pgfscope}%
\begin{pgfscope}%
\definecolor{textcolor}{rgb}{0.000000,0.000000,0.000000}%
\pgfsetstrokecolor{textcolor}%
\pgfsetfillcolor{textcolor}%
\pgftext[x=2.521555in,y=3.686400in,left,base]{\color{textcolor}\rmfamily\fontsize{10.000000}{12.000000}\selectfont Ground truth}%
\end{pgfscope}%
\begin{pgfscope}%
\pgfsetbuttcap%
\pgfsetroundjoin%
\definecolor{currentfill}{rgb}{0.121569,0.466667,0.705882}%
\pgfsetfillcolor{currentfill}%
\pgfsetlinewidth{1.003750pt}%
\definecolor{currentstroke}{rgb}{0.121569,0.466667,0.705882}%
\pgfsetstrokecolor{currentstroke}%
\pgfsetdash{}{0pt}%
\pgfsys@defobject{currentmarker}{\pgfqpoint{-0.031056in}{-0.031056in}}{\pgfqpoint{0.031056in}{0.031056in}}{%
\pgfpathmoveto{\pgfqpoint{0.000000in}{-0.031056in}}%
\pgfpathcurveto{\pgfqpoint{0.008236in}{-0.031056in}}{\pgfqpoint{0.016136in}{-0.027784in}}{\pgfqpoint{0.021960in}{-0.021960in}}%
\pgfpathcurveto{\pgfqpoint{0.027784in}{-0.016136in}}{\pgfqpoint{0.031056in}{-0.008236in}}{\pgfqpoint{0.031056in}{0.000000in}}%
\pgfpathcurveto{\pgfqpoint{0.031056in}{0.008236in}}{\pgfqpoint{0.027784in}{0.016136in}}{\pgfqpoint{0.021960in}{0.021960in}}%
\pgfpathcurveto{\pgfqpoint{0.016136in}{0.027784in}}{\pgfqpoint{0.008236in}{0.031056in}}{\pgfqpoint{0.000000in}{0.031056in}}%
\pgfpathcurveto{\pgfqpoint{-0.008236in}{0.031056in}}{\pgfqpoint{-0.016136in}{0.027784in}}{\pgfqpoint{-0.021960in}{0.021960in}}%
\pgfpathcurveto{\pgfqpoint{-0.027784in}{0.016136in}}{\pgfqpoint{-0.031056in}{0.008236in}}{\pgfqpoint{-0.031056in}{0.000000in}}%
\pgfpathcurveto{\pgfqpoint{-0.031056in}{-0.008236in}}{\pgfqpoint{-0.027784in}{-0.016136in}}{\pgfqpoint{-0.021960in}{-0.021960in}}%
\pgfpathcurveto{\pgfqpoint{-0.016136in}{-0.027784in}}{\pgfqpoint{-0.008236in}{-0.031056in}}{\pgfqpoint{0.000000in}{-0.031056in}}%
\pgfpathclose%
\pgfusepath{stroke,fill}%
}%
\begin{pgfscope}%
\pgfsys@transformshift{2.271555in}{3.529248in}%
\pgfsys@useobject{currentmarker}{}%
\end{pgfscope}%
\end{pgfscope}%
\begin{pgfscope}%
\definecolor{textcolor}{rgb}{0.000000,0.000000,0.000000}%
\pgfsetstrokecolor{textcolor}%
\pgfsetfillcolor{textcolor}%
\pgftext[x=2.521555in,y=3.492789in,left,base]{\color{textcolor}\rmfamily\fontsize{10.000000}{12.000000}\selectfont Estimated position}%
\end{pgfscope}%
\begin{pgfscope}%
\pgfsetbuttcap%
\pgfsetroundjoin%
\definecolor{currentfill}{rgb}{1.000000,0.498039,0.054902}%
\pgfsetfillcolor{currentfill}%
\pgfsetlinewidth{1.003750pt}%
\definecolor{currentstroke}{rgb}{1.000000,0.498039,0.054902}%
\pgfsetstrokecolor{currentstroke}%
\pgfsetdash{}{0pt}%
\pgfsys@defobject{currentmarker}{\pgfqpoint{-0.031056in}{-0.031056in}}{\pgfqpoint{0.031056in}{0.031056in}}{%
\pgfpathmoveto{\pgfqpoint{0.000000in}{-0.031056in}}%
\pgfpathcurveto{\pgfqpoint{0.008236in}{-0.031056in}}{\pgfqpoint{0.016136in}{-0.027784in}}{\pgfqpoint{0.021960in}{-0.021960in}}%
\pgfpathcurveto{\pgfqpoint{0.027784in}{-0.016136in}}{\pgfqpoint{0.031056in}{-0.008236in}}{\pgfqpoint{0.031056in}{0.000000in}}%
\pgfpathcurveto{\pgfqpoint{0.031056in}{0.008236in}}{\pgfqpoint{0.027784in}{0.016136in}}{\pgfqpoint{0.021960in}{0.021960in}}%
\pgfpathcurveto{\pgfqpoint{0.016136in}{0.027784in}}{\pgfqpoint{0.008236in}{0.031056in}}{\pgfqpoint{0.000000in}{0.031056in}}%
\pgfpathcurveto{\pgfqpoint{-0.008236in}{0.031056in}}{\pgfqpoint{-0.016136in}{0.027784in}}{\pgfqpoint{-0.021960in}{0.021960in}}%
\pgfpathcurveto{\pgfqpoint{-0.027784in}{0.016136in}}{\pgfqpoint{-0.031056in}{0.008236in}}{\pgfqpoint{-0.031056in}{0.000000in}}%
\pgfpathcurveto{\pgfqpoint{-0.031056in}{-0.008236in}}{\pgfqpoint{-0.027784in}{-0.016136in}}{\pgfqpoint{-0.021960in}{-0.021960in}}%
\pgfpathcurveto{\pgfqpoint{-0.016136in}{-0.027784in}}{\pgfqpoint{-0.008236in}{-0.031056in}}{\pgfqpoint{0.000000in}{-0.031056in}}%
\pgfpathclose%
\pgfusepath{stroke,fill}%
}%
\begin{pgfscope}%
\pgfsys@transformshift{2.271555in}{3.335637in}%
\pgfsys@useobject{currentmarker}{}%
\end{pgfscope}%
\end{pgfscope}%
\begin{pgfscope}%
\definecolor{textcolor}{rgb}{0.000000,0.000000,0.000000}%
\pgfsetstrokecolor{textcolor}%
\pgfsetfillcolor{textcolor}%
\pgftext[x=2.521555in,y=3.299178in,left,base]{\color{textcolor}\rmfamily\fontsize{10.000000}{12.000000}\selectfont Estimated turn}%
\end{pgfscope}%
\end{pgfpicture}%
\makeatother%
\endgroup%
}
%         \caption{EFK's 3D position estimation had the lowest turn error for the 4-meter side triangle experiment.}
%         \label{fig:triangle4_3D}
%     \end{subfigure}
%     \caption{Position estimation by the best performing algorithms in the 4-meter side triangle experiment.}
%     \label{fig:triangle4}
% \end{figure}

% \subsubsection{16 meter}

% For the 16-meter triangle experiment, the Mahony algorithm which had the lowest displacement error with an average of 1.79 meters (3.74\% of error margin), and Tilt with an average of 2.13 meters of turn error (4.43\% of error margin).

% \begin{figure}[!h]
%     \centering
%     \begin{table}[H]
    \begin{center}
        \resizebox{1\linewidth}{!}{
            \begin{tabular}[t]{lcccc}
                \hline
                Algorithm   & Displacement Error[$m$] & Displacement Error[\%] & Turn Error[$m$] & Turn Error[\%] \\
                \hline
                AngularRate & 12.53                   & 26.10                  & 19.26           & 40.12          \\            AQUA            & 7.48  & 15.58 & 12.02 & 25.04              \\            Complementary            & 8.07  & 16.82 & 9.07 & 18.89              \\            Davenport            & 5.43  & 11.31 & 5.31 & 11.07              \\            EKF            & 1.96  & 4.09 & 1.94 & 4.04              \\            FAMC            & 11.55  & 24.05 & 19.62 & 40.87              \\            FLAE            & 5.41  & 11.27 & 5.17 & 10.76              \\            Fourati            & 10.51  & 21.90 & 19.72 & 41.09              \\            Madgwick            & 3.70  & 7.71 & 7.11 & 14.81              \\            Mahony            & 1.62  & 3.37 & 2.39 & 4.98              \\            OLEQ            & 1.66  & 3.45 & 2.36 & 4.92              \\            QUEST            & 8.39  & 17.48 & 18.56 & 38.66              \\            ROLEQ            & 1.79  & 3.73 & 2.93 & 6.11              \\            SAAM            & 5.76  & 12.01 & 4.60 & 9.59              \\            Tilt            & 5.76  & 12.01 & 4.60 & 9.59              \\
                \hline
                Average     & 6.11                    & 12.72                  & 8.98            & 18.70
            \end{tabular}
        }
        \caption{Accelerometer Specifications. }
        \label{tab:accelerometer_specification}
    \end{center}
\end{table}
% \end{figure}

% \begin{figure}[!h]
%     \centering
%     \begin{subfigure}{0.49\textwidth}
%         \centering
%         \resizebox{1\linewidth}{!}{%% Creator: Matplotlib, PGF backend
%%
%% To include the figure in your LaTeX document, write
%%   \input{<filename>.pgf}
%%
%% Make sure the required packages are loaded in your preamble
%%   \usepackage{pgf}
%%
%% and, on pdftex
%%   \usepackage[utf8]{inputenc}\DeclareUnicodeCharacter{2212}{-}
%%
%% or, on luatex and xetex
%%   \usepackage{unicode-math}
%%
%% Figures using additional raster images can only be included by \input if
%% they are in the same directory as the main LaTeX file. For loading figures
%% from other directories you can use the `import` package
%%   \usepackage{import}
%%
%% and then include the figures with
%%   \import{<path to file>}{<filename>.pgf}
%%
%% Matplotlib used the following preamble
%%   \usepackage{fontspec}
%%   \setmainfont{DejaVuSerif.ttf}[Path=C:/Users/Claudio/AppData/Local/Programs/Python/Python39/Lib/site-packages/matplotlib/mpl-data/fonts/ttf/]
%%   \setsansfont{DejaVuSans.ttf}[Path=C:/Users/Claudio/AppData/Local/Programs/Python/Python39/Lib/site-packages/matplotlib/mpl-data/fonts/ttf/]
%%   \setmonofont{DejaVuSansMono.ttf}[Path=C:/Users/Claudio/AppData/Local/Programs/Python/Python39/Lib/site-packages/matplotlib/mpl-data/fonts/ttf/]
%%
\begingroup%
\makeatletter%
\begin{pgfpicture}%
\pgfpathrectangle{\pgfpointorigin}{\pgfqpoint{4.342069in}{4.226689in}}%
\pgfusepath{use as bounding box, clip}%
\begin{pgfscope}%
\pgfsetbuttcap%
\pgfsetmiterjoin%
\definecolor{currentfill}{rgb}{1.000000,1.000000,1.000000}%
\pgfsetfillcolor{currentfill}%
\pgfsetlinewidth{0.000000pt}%
\definecolor{currentstroke}{rgb}{1.000000,1.000000,1.000000}%
\pgfsetstrokecolor{currentstroke}%
\pgfsetdash{}{0pt}%
\pgfpathmoveto{\pgfqpoint{0.000000in}{0.000000in}}%
\pgfpathlineto{\pgfqpoint{4.342069in}{0.000000in}}%
\pgfpathlineto{\pgfqpoint{4.342069in}{4.226689in}}%
\pgfpathlineto{\pgfqpoint{0.000000in}{4.226689in}}%
\pgfpathclose%
\pgfusepath{fill}%
\end{pgfscope}%
\begin{pgfscope}%
\pgfsetbuttcap%
\pgfsetmiterjoin%
\definecolor{currentfill}{rgb}{1.000000,1.000000,1.000000}%
\pgfsetfillcolor{currentfill}%
\pgfsetlinewidth{0.000000pt}%
\definecolor{currentstroke}{rgb}{0.000000,0.000000,0.000000}%
\pgfsetstrokecolor{currentstroke}%
\pgfsetstrokeopacity{0.000000}%
\pgfsetdash{}{0pt}%
\pgfpathmoveto{\pgfqpoint{0.100000in}{0.220728in}}%
\pgfpathlineto{\pgfqpoint{3.796000in}{0.220728in}}%
\pgfpathlineto{\pgfqpoint{3.796000in}{3.916728in}}%
\pgfpathlineto{\pgfqpoint{0.100000in}{3.916728in}}%
\pgfpathclose%
\pgfusepath{fill}%
\end{pgfscope}%
\begin{pgfscope}%
\pgfsetbuttcap%
\pgfsetmiterjoin%
\definecolor{currentfill}{rgb}{0.950000,0.950000,0.950000}%
\pgfsetfillcolor{currentfill}%
\pgfsetfillopacity{0.500000}%
\pgfsetlinewidth{1.003750pt}%
\definecolor{currentstroke}{rgb}{0.950000,0.950000,0.950000}%
\pgfsetstrokecolor{currentstroke}%
\pgfsetstrokeopacity{0.500000}%
\pgfsetdash{}{0pt}%
\pgfpathmoveto{\pgfqpoint{0.379073in}{1.132043in}}%
\pgfpathlineto{\pgfqpoint{1.599613in}{2.155124in}}%
\pgfpathlineto{\pgfqpoint{1.582647in}{3.630589in}}%
\pgfpathlineto{\pgfqpoint{0.303698in}{2.697271in}}%
\pgfusepath{stroke,fill}%
\end{pgfscope}%
\begin{pgfscope}%
\pgfsetbuttcap%
\pgfsetmiterjoin%
\definecolor{currentfill}{rgb}{0.900000,0.900000,0.900000}%
\pgfsetfillcolor{currentfill}%
\pgfsetfillopacity{0.500000}%
\pgfsetlinewidth{1.003750pt}%
\definecolor{currentstroke}{rgb}{0.900000,0.900000,0.900000}%
\pgfsetstrokecolor{currentstroke}%
\pgfsetstrokeopacity{0.500000}%
\pgfsetdash{}{0pt}%
\pgfpathmoveto{\pgfqpoint{1.599613in}{2.155124in}}%
\pgfpathlineto{\pgfqpoint{3.558144in}{1.585856in}}%
\pgfpathlineto{\pgfqpoint{3.628038in}{3.112142in}}%
\pgfpathlineto{\pgfqpoint{1.582647in}{3.630589in}}%
\pgfusepath{stroke,fill}%
\end{pgfscope}%
\begin{pgfscope}%
\pgfsetbuttcap%
\pgfsetmiterjoin%
\definecolor{currentfill}{rgb}{0.925000,0.925000,0.925000}%
\pgfsetfillcolor{currentfill}%
\pgfsetfillopacity{0.500000}%
\pgfsetlinewidth{1.003750pt}%
\definecolor{currentstroke}{rgb}{0.925000,0.925000,0.925000}%
\pgfsetstrokecolor{currentstroke}%
\pgfsetstrokeopacity{0.500000}%
\pgfsetdash{}{0pt}%
\pgfpathmoveto{\pgfqpoint{0.379073in}{1.132043in}}%
\pgfpathlineto{\pgfqpoint{2.455212in}{0.453976in}}%
\pgfpathlineto{\pgfqpoint{3.558144in}{1.585856in}}%
\pgfpathlineto{\pgfqpoint{1.599613in}{2.155124in}}%
\pgfusepath{stroke,fill}%
\end{pgfscope}%
\begin{pgfscope}%
\pgfsetrectcap%
\pgfsetroundjoin%
\pgfsetlinewidth{0.803000pt}%
\definecolor{currentstroke}{rgb}{0.000000,0.000000,0.000000}%
\pgfsetstrokecolor{currentstroke}%
\pgfsetdash{}{0pt}%
\pgfpathmoveto{\pgfqpoint{0.379073in}{1.132043in}}%
\pgfpathlineto{\pgfqpoint{2.455212in}{0.453976in}}%
\pgfusepath{stroke}%
\end{pgfscope}%
\begin{pgfscope}%
\definecolor{textcolor}{rgb}{0.000000,0.000000,0.000000}%
\pgfsetstrokecolor{textcolor}%
\pgfsetfillcolor{textcolor}%
\pgftext[x=0.697927in, y=0.423808in, left, base,rotate=341.912962]{\color{textcolor}\sffamily\fontsize{10.000000}{12.000000}\selectfont Position X [\(\displaystyle m\)]}%
\end{pgfscope}%
\begin{pgfscope}%
\pgfsetbuttcap%
\pgfsetroundjoin%
\pgfsetlinewidth{0.803000pt}%
\definecolor{currentstroke}{rgb}{0.690196,0.690196,0.690196}%
\pgfsetstrokecolor{currentstroke}%
\pgfsetdash{}{0pt}%
\pgfpathmoveto{\pgfqpoint{0.688752in}{1.030902in}}%
\pgfpathlineto{\pgfqpoint{1.892848in}{2.069892in}}%
\pgfpathlineto{\pgfqpoint{1.888337in}{3.553106in}}%
\pgfusepath{stroke}%
\end{pgfscope}%
\begin{pgfscope}%
\pgfsetbuttcap%
\pgfsetroundjoin%
\pgfsetlinewidth{0.803000pt}%
\definecolor{currentstroke}{rgb}{0.690196,0.690196,0.690196}%
\pgfsetstrokecolor{currentstroke}%
\pgfsetdash{}{0pt}%
\pgfpathmoveto{\pgfqpoint{1.113001in}{0.892342in}}%
\pgfpathlineto{\pgfqpoint{2.293944in}{1.953310in}}%
\pgfpathlineto{\pgfqpoint{2.306782in}{3.447042in}}%
\pgfusepath{stroke}%
\end{pgfscope}%
\begin{pgfscope}%
\pgfsetbuttcap%
\pgfsetroundjoin%
\pgfsetlinewidth{0.803000pt}%
\definecolor{currentstroke}{rgb}{0.690196,0.690196,0.690196}%
\pgfsetstrokecolor{currentstroke}%
\pgfsetdash{}{0pt}%
\pgfpathmoveto{\pgfqpoint{1.546668in}{0.750707in}}%
\pgfpathlineto{\pgfqpoint{2.703195in}{1.834356in}}%
\pgfpathlineto{\pgfqpoint{2.734109in}{3.338727in}}%
\pgfusepath{stroke}%
\end{pgfscope}%
\begin{pgfscope}%
\pgfsetbuttcap%
\pgfsetroundjoin%
\pgfsetlinewidth{0.803000pt}%
\definecolor{currentstroke}{rgb}{0.690196,0.690196,0.690196}%
\pgfsetstrokecolor{currentstroke}%
\pgfsetdash{}{0pt}%
\pgfpathmoveto{\pgfqpoint{1.990070in}{0.605892in}}%
\pgfpathlineto{\pgfqpoint{3.120853in}{1.712960in}}%
\pgfpathlineto{\pgfqpoint{3.170603in}{3.228089in}}%
\pgfusepath{stroke}%
\end{pgfscope}%
\begin{pgfscope}%
\pgfsetrectcap%
\pgfsetroundjoin%
\pgfsetlinewidth{0.803000pt}%
\definecolor{currentstroke}{rgb}{0.000000,0.000000,0.000000}%
\pgfsetstrokecolor{currentstroke}%
\pgfsetdash{}{0pt}%
\pgfpathmoveto{\pgfqpoint{0.699241in}{1.039953in}}%
\pgfpathlineto{\pgfqpoint{0.667728in}{1.012761in}}%
\pgfusepath{stroke}%
\end{pgfscope}%
\begin{pgfscope}%
\definecolor{textcolor}{rgb}{0.000000,0.000000,0.000000}%
\pgfsetstrokecolor{textcolor}%
\pgfsetfillcolor{textcolor}%
\pgftext[x=0.584367in,y=0.811426in,,top]{\color{textcolor}\sffamily\fontsize{10.000000}{12.000000}\selectfont 0}%
\end{pgfscope}%
\begin{pgfscope}%
\pgfsetrectcap%
\pgfsetroundjoin%
\pgfsetlinewidth{0.803000pt}%
\definecolor{currentstroke}{rgb}{0.000000,0.000000,0.000000}%
\pgfsetstrokecolor{currentstroke}%
\pgfsetdash{}{0pt}%
\pgfpathmoveto{\pgfqpoint{1.123298in}{0.901593in}}%
\pgfpathlineto{\pgfqpoint{1.092363in}{0.873801in}}%
\pgfusepath{stroke}%
\end{pgfscope}%
\begin{pgfscope}%
\definecolor{textcolor}{rgb}{0.000000,0.000000,0.000000}%
\pgfsetstrokecolor{textcolor}%
\pgfsetfillcolor{textcolor}%
\pgftext[x=1.009068in,y=0.669925in,,top]{\color{textcolor}\sffamily\fontsize{10.000000}{12.000000}\selectfont 5}%
\end{pgfscope}%
\begin{pgfscope}%
\pgfsetrectcap%
\pgfsetroundjoin%
\pgfsetlinewidth{0.803000pt}%
\definecolor{currentstroke}{rgb}{0.000000,0.000000,0.000000}%
\pgfsetstrokecolor{currentstroke}%
\pgfsetdash{}{0pt}%
\pgfpathmoveto{\pgfqpoint{1.556761in}{0.760164in}}%
\pgfpathlineto{\pgfqpoint{1.526438in}{0.731751in}}%
\pgfusepath{stroke}%
\end{pgfscope}%
\begin{pgfscope}%
\definecolor{textcolor}{rgb}{0.000000,0.000000,0.000000}%
\pgfsetstrokecolor{textcolor}%
\pgfsetfillcolor{textcolor}%
\pgftext[x=1.443233in,y=0.525270in,,top]{\color{textcolor}\sffamily\fontsize{10.000000}{12.000000}\selectfont 10}%
\end{pgfscope}%
\begin{pgfscope}%
\pgfsetrectcap%
\pgfsetroundjoin%
\pgfsetlinewidth{0.803000pt}%
\definecolor{currentstroke}{rgb}{0.000000,0.000000,0.000000}%
\pgfsetstrokecolor{currentstroke}%
\pgfsetdash{}{0pt}%
\pgfpathmoveto{\pgfqpoint{1.999948in}{0.615562in}}%
\pgfpathlineto{\pgfqpoint{1.970271in}{0.586508in}}%
\pgfusepath{stroke}%
\end{pgfscope}%
\begin{pgfscope}%
\definecolor{textcolor}{rgb}{0.000000,0.000000,0.000000}%
\pgfsetstrokecolor{textcolor}%
\pgfsetfillcolor{textcolor}%
\pgftext[x=1.887182in,y=0.377356in,,top]{\color{textcolor}\sffamily\fontsize{10.000000}{12.000000}\selectfont 15}%
\end{pgfscope}%
\begin{pgfscope}%
\pgfsetrectcap%
\pgfsetroundjoin%
\pgfsetlinewidth{0.803000pt}%
\definecolor{currentstroke}{rgb}{0.000000,0.000000,0.000000}%
\pgfsetstrokecolor{currentstroke}%
\pgfsetdash{}{0pt}%
\pgfpathmoveto{\pgfqpoint{3.558144in}{1.585856in}}%
\pgfpathlineto{\pgfqpoint{2.455212in}{0.453976in}}%
\pgfusepath{stroke}%
\end{pgfscope}%
\begin{pgfscope}%
\definecolor{textcolor}{rgb}{0.000000,0.000000,0.000000}%
\pgfsetstrokecolor{textcolor}%
\pgfsetfillcolor{textcolor}%
\pgftext[x=3.103916in, y=0.291339in, left, base,rotate=45.742112]{\color{textcolor}\sffamily\fontsize{10.000000}{12.000000}\selectfont Position Y [\(\displaystyle m\)]}%
\end{pgfscope}%
\begin{pgfscope}%
\pgfsetbuttcap%
\pgfsetroundjoin%
\pgfsetlinewidth{0.803000pt}%
\definecolor{currentstroke}{rgb}{0.690196,0.690196,0.690196}%
\pgfsetstrokecolor{currentstroke}%
\pgfsetdash{}{0pt}%
\pgfpathmoveto{\pgfqpoint{0.519825in}{2.854991in}}%
\pgfpathlineto{\pgfqpoint{0.584681in}{1.304387in}}%
\pgfpathlineto{\pgfqpoint{2.641691in}{0.645349in}}%
\pgfusepath{stroke}%
\end{pgfscope}%
\begin{pgfscope}%
\pgfsetbuttcap%
\pgfsetroundjoin%
\pgfsetlinewidth{0.803000pt}%
\definecolor{currentstroke}{rgb}{0.690196,0.690196,0.690196}%
\pgfsetstrokecolor{currentstroke}%
\pgfsetdash{}{0pt}%
\pgfpathmoveto{\pgfqpoint{0.825695in}{3.078200in}}%
\pgfpathlineto{\pgfqpoint{0.876112in}{1.548671in}}%
\pgfpathlineto{\pgfqpoint{2.905534in}{0.916117in}}%
\pgfusepath{stroke}%
\end{pgfscope}%
\begin{pgfscope}%
\pgfsetbuttcap%
\pgfsetroundjoin%
\pgfsetlinewidth{0.803000pt}%
\definecolor{currentstroke}{rgb}{0.690196,0.690196,0.690196}%
\pgfsetstrokecolor{currentstroke}%
\pgfsetdash{}{0pt}%
\pgfpathmoveto{\pgfqpoint{1.119175in}{3.292369in}}%
\pgfpathlineto{\pgfqpoint{1.156237in}{1.783477in}}%
\pgfpathlineto{\pgfqpoint{3.158616in}{1.175842in}}%
\pgfusepath{stroke}%
\end{pgfscope}%
\begin{pgfscope}%
\pgfsetbuttcap%
\pgfsetroundjoin%
\pgfsetlinewidth{0.803000pt}%
\definecolor{currentstroke}{rgb}{0.690196,0.690196,0.690196}%
\pgfsetstrokecolor{currentstroke}%
\pgfsetdash{}{0pt}%
\pgfpathmoveto{\pgfqpoint{1.401004in}{3.498035in}}%
\pgfpathlineto{\pgfqpoint{1.425701in}{2.009347in}}%
\pgfpathlineto{\pgfqpoint{3.401584in}{1.425187in}}%
\pgfusepath{stroke}%
\end{pgfscope}%
\begin{pgfscope}%
\pgfsetrectcap%
\pgfsetroundjoin%
\pgfsetlinewidth{0.803000pt}%
\definecolor{currentstroke}{rgb}{0.000000,0.000000,0.000000}%
\pgfsetstrokecolor{currentstroke}%
\pgfsetdash{}{0pt}%
\pgfpathmoveto{\pgfqpoint{2.624364in}{0.650901in}}%
\pgfpathlineto{\pgfqpoint{2.676389in}{0.634232in}}%
\pgfusepath{stroke}%
\end{pgfscope}%
\begin{pgfscope}%
\definecolor{textcolor}{rgb}{0.000000,0.000000,0.000000}%
\pgfsetstrokecolor{textcolor}%
\pgfsetfillcolor{textcolor}%
\pgftext[x=2.819111in,y=0.460589in,,top]{\color{textcolor}\sffamily\fontsize{10.000000}{12.000000}\selectfont 0}%
\end{pgfscope}%
\begin{pgfscope}%
\pgfsetrectcap%
\pgfsetroundjoin%
\pgfsetlinewidth{0.803000pt}%
\definecolor{currentstroke}{rgb}{0.000000,0.000000,0.000000}%
\pgfsetstrokecolor{currentstroke}%
\pgfsetdash{}{0pt}%
\pgfpathmoveto{\pgfqpoint{2.888457in}{0.921440in}}%
\pgfpathlineto{\pgfqpoint{2.939730in}{0.905458in}}%
\pgfusepath{stroke}%
\end{pgfscope}%
\begin{pgfscope}%
\definecolor{textcolor}{rgb}{0.000000,0.000000,0.000000}%
\pgfsetstrokecolor{textcolor}%
\pgfsetfillcolor{textcolor}%
\pgftext[x=3.079412in,y=0.735362in,,top]{\color{textcolor}\sffamily\fontsize{10.000000}{12.000000}\selectfont 5}%
\end{pgfscope}%
\begin{pgfscope}%
\pgfsetrectcap%
\pgfsetroundjoin%
\pgfsetlinewidth{0.803000pt}%
\definecolor{currentstroke}{rgb}{0.000000,0.000000,0.000000}%
\pgfsetstrokecolor{currentstroke}%
\pgfsetdash{}{0pt}%
\pgfpathmoveto{\pgfqpoint{3.141784in}{1.180950in}}%
\pgfpathlineto{\pgfqpoint{3.192322in}{1.165614in}}%
\pgfusepath{stroke}%
\end{pgfscope}%
\begin{pgfscope}%
\definecolor{textcolor}{rgb}{0.000000,0.000000,0.000000}%
\pgfsetstrokecolor{textcolor}%
\pgfsetfillcolor{textcolor}%
\pgftext[x=3.329091in,y=0.998923in,,top]{\color{textcolor}\sffamily\fontsize{10.000000}{12.000000}\selectfont 10}%
\end{pgfscope}%
\begin{pgfscope}%
\pgfsetrectcap%
\pgfsetroundjoin%
\pgfsetlinewidth{0.803000pt}%
\definecolor{currentstroke}{rgb}{0.000000,0.000000,0.000000}%
\pgfsetstrokecolor{currentstroke}%
\pgfsetdash{}{0pt}%
\pgfpathmoveto{\pgfqpoint{3.384990in}{1.430092in}}%
\pgfpathlineto{\pgfqpoint{3.434811in}{1.415363in}}%
\pgfusepath{stroke}%
\end{pgfscope}%
\begin{pgfscope}%
\definecolor{textcolor}{rgb}{0.000000,0.000000,0.000000}%
\pgfsetstrokecolor{textcolor}%
\pgfsetfillcolor{textcolor}%
\pgftext[x=3.568786in,y=1.251944in,,top]{\color{textcolor}\sffamily\fontsize{10.000000}{12.000000}\selectfont 15}%
\end{pgfscope}%
\begin{pgfscope}%
\pgfsetrectcap%
\pgfsetroundjoin%
\pgfsetlinewidth{0.803000pt}%
\definecolor{currentstroke}{rgb}{0.000000,0.000000,0.000000}%
\pgfsetstrokecolor{currentstroke}%
\pgfsetdash{}{0pt}%
\pgfpathmoveto{\pgfqpoint{3.558144in}{1.585856in}}%
\pgfpathlineto{\pgfqpoint{3.628038in}{3.112142in}}%
\pgfusepath{stroke}%
\end{pgfscope}%
\begin{pgfscope}%
\definecolor{textcolor}{rgb}{0.000000,0.000000,0.000000}%
\pgfsetstrokecolor{textcolor}%
\pgfsetfillcolor{textcolor}%
\pgftext[x=4.169544in, y=1.928890in, left, base,rotate=87.378092]{\color{textcolor}\sffamily\fontsize{10.000000}{12.000000}\selectfont Position Z [\(\displaystyle m\)]}%
\end{pgfscope}%
\begin{pgfscope}%
\pgfsetbuttcap%
\pgfsetroundjoin%
\pgfsetlinewidth{0.803000pt}%
\definecolor{currentstroke}{rgb}{0.690196,0.690196,0.690196}%
\pgfsetstrokecolor{currentstroke}%
\pgfsetdash{}{0pt}%
\pgfpathmoveto{\pgfqpoint{3.568957in}{1.821984in}}%
\pgfpathlineto{\pgfqpoint{1.596984in}{2.383797in}}%
\pgfpathlineto{\pgfqpoint{0.367429in}{1.373850in}}%
\pgfusepath{stroke}%
\end{pgfscope}%
\begin{pgfscope}%
\pgfsetbuttcap%
\pgfsetroundjoin%
\pgfsetlinewidth{0.803000pt}%
\definecolor{currentstroke}{rgb}{0.690196,0.690196,0.690196}%
\pgfsetstrokecolor{currentstroke}%
\pgfsetdash{}{0pt}%
\pgfpathmoveto{\pgfqpoint{3.582824in}{2.124789in}}%
\pgfpathlineto{\pgfqpoint{1.593614in}{2.676824in}}%
\pgfpathlineto{\pgfqpoint{0.352487in}{1.684124in}}%
\pgfusepath{stroke}%
\end{pgfscope}%
\begin{pgfscope}%
\pgfsetbuttcap%
\pgfsetroundjoin%
\pgfsetlinewidth{0.803000pt}%
\definecolor{currentstroke}{rgb}{0.690196,0.690196,0.690196}%
\pgfsetstrokecolor{currentstroke}%
\pgfsetdash{}{0pt}%
\pgfpathmoveto{\pgfqpoint{3.596937in}{2.432988in}}%
\pgfpathlineto{\pgfqpoint{1.590188in}{2.974818in}}%
\pgfpathlineto{\pgfqpoint{0.337269in}{2.000136in}}%
\pgfusepath{stroke}%
\end{pgfscope}%
\begin{pgfscope}%
\pgfsetbuttcap%
\pgfsetroundjoin%
\pgfsetlinewidth{0.803000pt}%
\definecolor{currentstroke}{rgb}{0.690196,0.690196,0.690196}%
\pgfsetstrokecolor{currentstroke}%
\pgfsetdash{}{0pt}%
\pgfpathmoveto{\pgfqpoint{3.611304in}{2.746724in}}%
\pgfpathlineto{\pgfqpoint{1.586702in}{3.277906in}}%
\pgfpathlineto{\pgfqpoint{0.321767in}{2.322049in}}%
\pgfusepath{stroke}%
\end{pgfscope}%
\begin{pgfscope}%
\pgfsetbuttcap%
\pgfsetroundjoin%
\pgfsetlinewidth{0.803000pt}%
\definecolor{currentstroke}{rgb}{0.690196,0.690196,0.690196}%
\pgfsetstrokecolor{currentstroke}%
\pgfsetdash{}{0pt}%
\pgfpathmoveto{\pgfqpoint{3.625931in}{3.066150in}}%
\pgfpathlineto{\pgfqpoint{1.583157in}{3.586219in}}%
\pgfpathlineto{\pgfqpoint{0.305973in}{2.650028in}}%
\pgfusepath{stroke}%
\end{pgfscope}%
\begin{pgfscope}%
\pgfsetrectcap%
\pgfsetroundjoin%
\pgfsetlinewidth{0.803000pt}%
\definecolor{currentstroke}{rgb}{0.000000,0.000000,0.000000}%
\pgfsetstrokecolor{currentstroke}%
\pgfsetdash{}{0pt}%
\pgfpathmoveto{\pgfqpoint{3.552402in}{1.826701in}}%
\pgfpathlineto{\pgfqpoint{3.602108in}{1.812539in}}%
\pgfusepath{stroke}%
\end{pgfscope}%
\begin{pgfscope}%
\definecolor{textcolor}{rgb}{0.000000,0.000000,0.000000}%
\pgfsetstrokecolor{textcolor}%
\pgfsetfillcolor{textcolor}%
\pgftext[x=3.824129in,y=1.857702in,,top]{\color{textcolor}\sffamily\fontsize{10.000000}{12.000000}\selectfont 0.0}%
\end{pgfscope}%
\begin{pgfscope}%
\pgfsetrectcap%
\pgfsetroundjoin%
\pgfsetlinewidth{0.803000pt}%
\definecolor{currentstroke}{rgb}{0.000000,0.000000,0.000000}%
\pgfsetstrokecolor{currentstroke}%
\pgfsetdash{}{0pt}%
\pgfpathmoveto{\pgfqpoint{3.566116in}{2.129426in}}%
\pgfpathlineto{\pgfqpoint{3.616279in}{2.115505in}}%
\pgfusepath{stroke}%
\end{pgfscope}%
\begin{pgfscope}%
\definecolor{textcolor}{rgb}{0.000000,0.000000,0.000000}%
\pgfsetstrokecolor{textcolor}%
\pgfsetfillcolor{textcolor}%
\pgftext[x=3.840199in,y=2.159899in,,top]{\color{textcolor}\sffamily\fontsize{10.000000}{12.000000}\selectfont 0.2}%
\end{pgfscope}%
\begin{pgfscope}%
\pgfsetrectcap%
\pgfsetroundjoin%
\pgfsetlinewidth{0.803000pt}%
\definecolor{currentstroke}{rgb}{0.000000,0.000000,0.000000}%
\pgfsetstrokecolor{currentstroke}%
\pgfsetdash{}{0pt}%
\pgfpathmoveto{\pgfqpoint{3.580075in}{2.437541in}}%
\pgfpathlineto{\pgfqpoint{3.630702in}{2.423871in}}%
\pgfusepath{stroke}%
\end{pgfscope}%
\begin{pgfscope}%
\definecolor{textcolor}{rgb}{0.000000,0.000000,0.000000}%
\pgfsetstrokecolor{textcolor}%
\pgfsetfillcolor{textcolor}%
\pgftext[x=3.856554in,y=2.467462in,,top]{\color{textcolor}\sffamily\fontsize{10.000000}{12.000000}\selectfont 0.4}%
\end{pgfscope}%
\begin{pgfscope}%
\pgfsetrectcap%
\pgfsetroundjoin%
\pgfsetlinewidth{0.803000pt}%
\definecolor{currentstroke}{rgb}{0.000000,0.000000,0.000000}%
\pgfsetstrokecolor{currentstroke}%
\pgfsetdash{}{0pt}%
\pgfpathmoveto{\pgfqpoint{3.594285in}{2.751190in}}%
\pgfpathlineto{\pgfqpoint{3.645384in}{2.737783in}}%
\pgfusepath{stroke}%
\end{pgfscope}%
\begin{pgfscope}%
\definecolor{textcolor}{rgb}{0.000000,0.000000,0.000000}%
\pgfsetstrokecolor{textcolor}%
\pgfsetfillcolor{textcolor}%
\pgftext[x=3.873201in,y=2.780535in,,top]{\color{textcolor}\sffamily\fontsize{10.000000}{12.000000}\selectfont 0.6}%
\end{pgfscope}%
\begin{pgfscope}%
\pgfsetrectcap%
\pgfsetroundjoin%
\pgfsetlinewidth{0.803000pt}%
\definecolor{currentstroke}{rgb}{0.000000,0.000000,0.000000}%
\pgfsetstrokecolor{currentstroke}%
\pgfsetdash{}{0pt}%
\pgfpathmoveto{\pgfqpoint{3.608752in}{3.070524in}}%
\pgfpathlineto{\pgfqpoint{3.660333in}{3.057392in}}%
\pgfusepath{stroke}%
\end{pgfscope}%
\begin{pgfscope}%
\definecolor{textcolor}{rgb}{0.000000,0.000000,0.000000}%
\pgfsetstrokecolor{textcolor}%
\pgfsetfillcolor{textcolor}%
\pgftext[x=3.890150in,y=3.099267in,,top]{\color{textcolor}\sffamily\fontsize{10.000000}{12.000000}\selectfont 0.8}%
\end{pgfscope}%
\begin{pgfscope}%
\pgfpathrectangle{\pgfqpoint{0.100000in}{0.220728in}}{\pgfqpoint{3.696000in}{3.696000in}}%
\pgfusepath{clip}%
\pgfsetrectcap%
\pgfsetroundjoin%
\pgfsetlinewidth{1.505625pt}%
\definecolor{currentstroke}{rgb}{0.121569,0.466667,0.705882}%
\pgfsetstrokecolor{currentstroke}%
\pgfsetdash{}{0pt}%
\pgfpathmoveto{\pgfqpoint{0.883791in}{1.446920in}}%
\pgfpathlineto{\pgfqpoint{1.771800in}{2.197923in}}%
\pgfpathlineto{\pgfqpoint{2.272126in}{1.010635in}}%
\pgfpathlineto{\pgfqpoint{0.883791in}{1.446920in}}%
\pgfusepath{stroke}%
\end{pgfscope}%
\begin{pgfscope}%
\pgfpathrectangle{\pgfqpoint{0.100000in}{0.220728in}}{\pgfqpoint{3.696000in}{3.696000in}}%
\pgfusepath{clip}%
\pgfsetrectcap%
\pgfsetroundjoin%
\pgfsetlinewidth{1.505625pt}%
\definecolor{currentstroke}{rgb}{1.000000,0.000000,0.000000}%
\pgfsetstrokecolor{currentstroke}%
\pgfsetdash{}{0pt}%
\pgfpathmoveto{\pgfqpoint{0.883216in}{1.446502in}}%
\pgfpathlineto{\pgfqpoint{0.883791in}{1.446920in}}%
\pgfusepath{stroke}%
\end{pgfscope}%
\begin{pgfscope}%
\pgfpathrectangle{\pgfqpoint{0.100000in}{0.220728in}}{\pgfqpoint{3.696000in}{3.696000in}}%
\pgfusepath{clip}%
\pgfsetrectcap%
\pgfsetroundjoin%
\pgfsetlinewidth{1.505625pt}%
\definecolor{currentstroke}{rgb}{1.000000,0.000000,0.000000}%
\pgfsetstrokecolor{currentstroke}%
\pgfsetdash{}{0pt}%
\pgfpathmoveto{\pgfqpoint{0.883030in}{1.446696in}}%
\pgfpathlineto{\pgfqpoint{0.883791in}{1.446920in}}%
\pgfusepath{stroke}%
\end{pgfscope}%
\begin{pgfscope}%
\pgfpathrectangle{\pgfqpoint{0.100000in}{0.220728in}}{\pgfqpoint{3.696000in}{3.696000in}}%
\pgfusepath{clip}%
\pgfsetrectcap%
\pgfsetroundjoin%
\pgfsetlinewidth{1.505625pt}%
\definecolor{currentstroke}{rgb}{1.000000,0.000000,0.000000}%
\pgfsetstrokecolor{currentstroke}%
\pgfsetdash{}{0pt}%
\pgfpathmoveto{\pgfqpoint{0.882371in}{1.446904in}}%
\pgfpathlineto{\pgfqpoint{0.883791in}{1.446920in}}%
\pgfusepath{stroke}%
\end{pgfscope}%
\begin{pgfscope}%
\pgfpathrectangle{\pgfqpoint{0.100000in}{0.220728in}}{\pgfqpoint{3.696000in}{3.696000in}}%
\pgfusepath{clip}%
\pgfsetrectcap%
\pgfsetroundjoin%
\pgfsetlinewidth{1.505625pt}%
\definecolor{currentstroke}{rgb}{1.000000,0.000000,0.000000}%
\pgfsetstrokecolor{currentstroke}%
\pgfsetdash{}{0pt}%
\pgfpathmoveto{\pgfqpoint{0.881305in}{1.447341in}}%
\pgfpathlineto{\pgfqpoint{0.883791in}{1.446920in}}%
\pgfusepath{stroke}%
\end{pgfscope}%
\begin{pgfscope}%
\pgfpathrectangle{\pgfqpoint{0.100000in}{0.220728in}}{\pgfqpoint{3.696000in}{3.696000in}}%
\pgfusepath{clip}%
\pgfsetrectcap%
\pgfsetroundjoin%
\pgfsetlinewidth{1.505625pt}%
\definecolor{currentstroke}{rgb}{1.000000,0.000000,0.000000}%
\pgfsetstrokecolor{currentstroke}%
\pgfsetdash{}{0pt}%
\pgfpathmoveto{\pgfqpoint{0.879009in}{1.448156in}}%
\pgfpathlineto{\pgfqpoint{0.883791in}{1.446920in}}%
\pgfusepath{stroke}%
\end{pgfscope}%
\begin{pgfscope}%
\pgfpathrectangle{\pgfqpoint{0.100000in}{0.220728in}}{\pgfqpoint{3.696000in}{3.696000in}}%
\pgfusepath{clip}%
\pgfsetrectcap%
\pgfsetroundjoin%
\pgfsetlinewidth{1.505625pt}%
\definecolor{currentstroke}{rgb}{1.000000,0.000000,0.000000}%
\pgfsetstrokecolor{currentstroke}%
\pgfsetdash{}{0pt}%
\pgfpathmoveto{\pgfqpoint{0.876546in}{1.449196in}}%
\pgfpathlineto{\pgfqpoint{0.883791in}{1.446920in}}%
\pgfusepath{stroke}%
\end{pgfscope}%
\begin{pgfscope}%
\pgfpathrectangle{\pgfqpoint{0.100000in}{0.220728in}}{\pgfqpoint{3.696000in}{3.696000in}}%
\pgfusepath{clip}%
\pgfsetrectcap%
\pgfsetroundjoin%
\pgfsetlinewidth{1.505625pt}%
\definecolor{currentstroke}{rgb}{1.000000,0.000000,0.000000}%
\pgfsetstrokecolor{currentstroke}%
\pgfsetdash{}{0pt}%
\pgfpathmoveto{\pgfqpoint{0.872935in}{1.450629in}}%
\pgfpathlineto{\pgfqpoint{0.883791in}{1.446920in}}%
\pgfusepath{stroke}%
\end{pgfscope}%
\begin{pgfscope}%
\pgfpathrectangle{\pgfqpoint{0.100000in}{0.220728in}}{\pgfqpoint{3.696000in}{3.696000in}}%
\pgfusepath{clip}%
\pgfsetrectcap%
\pgfsetroundjoin%
\pgfsetlinewidth{1.505625pt}%
\definecolor{currentstroke}{rgb}{1.000000,0.000000,0.000000}%
\pgfsetstrokecolor{currentstroke}%
\pgfsetdash{}{0pt}%
\pgfpathmoveto{\pgfqpoint{0.869345in}{1.451694in}}%
\pgfpathlineto{\pgfqpoint{0.883791in}{1.446920in}}%
\pgfusepath{stroke}%
\end{pgfscope}%
\begin{pgfscope}%
\pgfpathrectangle{\pgfqpoint{0.100000in}{0.220728in}}{\pgfqpoint{3.696000in}{3.696000in}}%
\pgfusepath{clip}%
\pgfsetrectcap%
\pgfsetroundjoin%
\pgfsetlinewidth{1.505625pt}%
\definecolor{currentstroke}{rgb}{1.000000,0.000000,0.000000}%
\pgfsetstrokecolor{currentstroke}%
\pgfsetdash{}{0pt}%
\pgfpathmoveto{\pgfqpoint{0.867054in}{1.452755in}}%
\pgfpathlineto{\pgfqpoint{0.883791in}{1.446920in}}%
\pgfusepath{stroke}%
\end{pgfscope}%
\begin{pgfscope}%
\pgfpathrectangle{\pgfqpoint{0.100000in}{0.220728in}}{\pgfqpoint{3.696000in}{3.696000in}}%
\pgfusepath{clip}%
\pgfsetrectcap%
\pgfsetroundjoin%
\pgfsetlinewidth{1.505625pt}%
\definecolor{currentstroke}{rgb}{1.000000,0.000000,0.000000}%
\pgfsetstrokecolor{currentstroke}%
\pgfsetdash{}{0pt}%
\pgfpathmoveto{\pgfqpoint{0.866012in}{1.453163in}}%
\pgfpathlineto{\pgfqpoint{0.883791in}{1.446920in}}%
\pgfusepath{stroke}%
\end{pgfscope}%
\begin{pgfscope}%
\pgfpathrectangle{\pgfqpoint{0.100000in}{0.220728in}}{\pgfqpoint{3.696000in}{3.696000in}}%
\pgfusepath{clip}%
\pgfsetrectcap%
\pgfsetroundjoin%
\pgfsetlinewidth{1.505625pt}%
\definecolor{currentstroke}{rgb}{1.000000,0.000000,0.000000}%
\pgfsetstrokecolor{currentstroke}%
\pgfsetdash{}{0pt}%
\pgfpathmoveto{\pgfqpoint{0.865352in}{1.453485in}}%
\pgfpathlineto{\pgfqpoint{0.883791in}{1.446920in}}%
\pgfusepath{stroke}%
\end{pgfscope}%
\begin{pgfscope}%
\pgfpathrectangle{\pgfqpoint{0.100000in}{0.220728in}}{\pgfqpoint{3.696000in}{3.696000in}}%
\pgfusepath{clip}%
\pgfsetrectcap%
\pgfsetroundjoin%
\pgfsetlinewidth{1.505625pt}%
\definecolor{currentstroke}{rgb}{1.000000,0.000000,0.000000}%
\pgfsetstrokecolor{currentstroke}%
\pgfsetdash{}{0pt}%
\pgfpathmoveto{\pgfqpoint{0.865009in}{1.453592in}}%
\pgfpathlineto{\pgfqpoint{0.883791in}{1.446920in}}%
\pgfusepath{stroke}%
\end{pgfscope}%
\begin{pgfscope}%
\pgfpathrectangle{\pgfqpoint{0.100000in}{0.220728in}}{\pgfqpoint{3.696000in}{3.696000in}}%
\pgfusepath{clip}%
\pgfsetrectcap%
\pgfsetroundjoin%
\pgfsetlinewidth{1.505625pt}%
\definecolor{currentstroke}{rgb}{1.000000,0.000000,0.000000}%
\pgfsetstrokecolor{currentstroke}%
\pgfsetdash{}{0pt}%
\pgfpathmoveto{\pgfqpoint{0.864820in}{1.453695in}}%
\pgfpathlineto{\pgfqpoint{0.883791in}{1.446920in}}%
\pgfusepath{stroke}%
\end{pgfscope}%
\begin{pgfscope}%
\pgfpathrectangle{\pgfqpoint{0.100000in}{0.220728in}}{\pgfqpoint{3.696000in}{3.696000in}}%
\pgfusepath{clip}%
\pgfsetrectcap%
\pgfsetroundjoin%
\pgfsetlinewidth{1.505625pt}%
\definecolor{currentstroke}{rgb}{1.000000,0.000000,0.000000}%
\pgfsetstrokecolor{currentstroke}%
\pgfsetdash{}{0pt}%
\pgfpathmoveto{\pgfqpoint{0.864719in}{1.453732in}}%
\pgfpathlineto{\pgfqpoint{0.883791in}{1.446920in}}%
\pgfusepath{stroke}%
\end{pgfscope}%
\begin{pgfscope}%
\pgfpathrectangle{\pgfqpoint{0.100000in}{0.220728in}}{\pgfqpoint{3.696000in}{3.696000in}}%
\pgfusepath{clip}%
\pgfsetrectcap%
\pgfsetroundjoin%
\pgfsetlinewidth{1.505625pt}%
\definecolor{currentstroke}{rgb}{1.000000,0.000000,0.000000}%
\pgfsetstrokecolor{currentstroke}%
\pgfsetdash{}{0pt}%
\pgfpathmoveto{\pgfqpoint{0.864659in}{1.453763in}}%
\pgfpathlineto{\pgfqpoint{0.883791in}{1.446920in}}%
\pgfusepath{stroke}%
\end{pgfscope}%
\begin{pgfscope}%
\pgfpathrectangle{\pgfqpoint{0.100000in}{0.220728in}}{\pgfqpoint{3.696000in}{3.696000in}}%
\pgfusepath{clip}%
\pgfsetrectcap%
\pgfsetroundjoin%
\pgfsetlinewidth{1.505625pt}%
\definecolor{currentstroke}{rgb}{1.000000,0.000000,0.000000}%
\pgfsetstrokecolor{currentstroke}%
\pgfsetdash{}{0pt}%
\pgfpathmoveto{\pgfqpoint{0.864628in}{1.453779in}}%
\pgfpathlineto{\pgfqpoint{0.883791in}{1.446920in}}%
\pgfusepath{stroke}%
\end{pgfscope}%
\begin{pgfscope}%
\pgfpathrectangle{\pgfqpoint{0.100000in}{0.220728in}}{\pgfqpoint{3.696000in}{3.696000in}}%
\pgfusepath{clip}%
\pgfsetrectcap%
\pgfsetroundjoin%
\pgfsetlinewidth{1.505625pt}%
\definecolor{currentstroke}{rgb}{1.000000,0.000000,0.000000}%
\pgfsetstrokecolor{currentstroke}%
\pgfsetdash{}{0pt}%
\pgfpathmoveto{\pgfqpoint{0.864610in}{1.453790in}}%
\pgfpathlineto{\pgfqpoint{0.883791in}{1.446920in}}%
\pgfusepath{stroke}%
\end{pgfscope}%
\begin{pgfscope}%
\pgfpathrectangle{\pgfqpoint{0.100000in}{0.220728in}}{\pgfqpoint{3.696000in}{3.696000in}}%
\pgfusepath{clip}%
\pgfsetrectcap%
\pgfsetroundjoin%
\pgfsetlinewidth{1.505625pt}%
\definecolor{currentstroke}{rgb}{1.000000,0.000000,0.000000}%
\pgfsetstrokecolor{currentstroke}%
\pgfsetdash{}{0pt}%
\pgfpathmoveto{\pgfqpoint{0.864600in}{1.453794in}}%
\pgfpathlineto{\pgfqpoint{0.883791in}{1.446920in}}%
\pgfusepath{stroke}%
\end{pgfscope}%
\begin{pgfscope}%
\pgfpathrectangle{\pgfqpoint{0.100000in}{0.220728in}}{\pgfqpoint{3.696000in}{3.696000in}}%
\pgfusepath{clip}%
\pgfsetrectcap%
\pgfsetroundjoin%
\pgfsetlinewidth{1.505625pt}%
\definecolor{currentstroke}{rgb}{1.000000,0.000000,0.000000}%
\pgfsetstrokecolor{currentstroke}%
\pgfsetdash{}{0pt}%
\pgfpathmoveto{\pgfqpoint{0.864595in}{1.453797in}}%
\pgfpathlineto{\pgfqpoint{0.883791in}{1.446920in}}%
\pgfusepath{stroke}%
\end{pgfscope}%
\begin{pgfscope}%
\pgfpathrectangle{\pgfqpoint{0.100000in}{0.220728in}}{\pgfqpoint{3.696000in}{3.696000in}}%
\pgfusepath{clip}%
\pgfsetrectcap%
\pgfsetroundjoin%
\pgfsetlinewidth{1.505625pt}%
\definecolor{currentstroke}{rgb}{1.000000,0.000000,0.000000}%
\pgfsetstrokecolor{currentstroke}%
\pgfsetdash{}{0pt}%
\pgfpathmoveto{\pgfqpoint{0.864592in}{1.453798in}}%
\pgfpathlineto{\pgfqpoint{0.883791in}{1.446920in}}%
\pgfusepath{stroke}%
\end{pgfscope}%
\begin{pgfscope}%
\pgfpathrectangle{\pgfqpoint{0.100000in}{0.220728in}}{\pgfqpoint{3.696000in}{3.696000in}}%
\pgfusepath{clip}%
\pgfsetrectcap%
\pgfsetroundjoin%
\pgfsetlinewidth{1.505625pt}%
\definecolor{currentstroke}{rgb}{1.000000,0.000000,0.000000}%
\pgfsetstrokecolor{currentstroke}%
\pgfsetdash{}{0pt}%
\pgfpathmoveto{\pgfqpoint{0.864590in}{1.453799in}}%
\pgfpathlineto{\pgfqpoint{0.883791in}{1.446920in}}%
\pgfusepath{stroke}%
\end{pgfscope}%
\begin{pgfscope}%
\pgfpathrectangle{\pgfqpoint{0.100000in}{0.220728in}}{\pgfqpoint{3.696000in}{3.696000in}}%
\pgfusepath{clip}%
\pgfsetrectcap%
\pgfsetroundjoin%
\pgfsetlinewidth{1.505625pt}%
\definecolor{currentstroke}{rgb}{1.000000,0.000000,0.000000}%
\pgfsetstrokecolor{currentstroke}%
\pgfsetdash{}{0pt}%
\pgfpathmoveto{\pgfqpoint{0.864589in}{1.453799in}}%
\pgfpathlineto{\pgfqpoint{0.883791in}{1.446920in}}%
\pgfusepath{stroke}%
\end{pgfscope}%
\begin{pgfscope}%
\pgfpathrectangle{\pgfqpoint{0.100000in}{0.220728in}}{\pgfqpoint{3.696000in}{3.696000in}}%
\pgfusepath{clip}%
\pgfsetrectcap%
\pgfsetroundjoin%
\pgfsetlinewidth{1.505625pt}%
\definecolor{currentstroke}{rgb}{1.000000,0.000000,0.000000}%
\pgfsetstrokecolor{currentstroke}%
\pgfsetdash{}{0pt}%
\pgfpathmoveto{\pgfqpoint{0.864589in}{1.453799in}}%
\pgfpathlineto{\pgfqpoint{0.883791in}{1.446920in}}%
\pgfusepath{stroke}%
\end{pgfscope}%
\begin{pgfscope}%
\pgfpathrectangle{\pgfqpoint{0.100000in}{0.220728in}}{\pgfqpoint{3.696000in}{3.696000in}}%
\pgfusepath{clip}%
\pgfsetrectcap%
\pgfsetroundjoin%
\pgfsetlinewidth{1.505625pt}%
\definecolor{currentstroke}{rgb}{1.000000,0.000000,0.000000}%
\pgfsetstrokecolor{currentstroke}%
\pgfsetdash{}{0pt}%
\pgfpathmoveto{\pgfqpoint{0.864589in}{1.453799in}}%
\pgfpathlineto{\pgfqpoint{0.883791in}{1.446920in}}%
\pgfusepath{stroke}%
\end{pgfscope}%
\begin{pgfscope}%
\pgfpathrectangle{\pgfqpoint{0.100000in}{0.220728in}}{\pgfqpoint{3.696000in}{3.696000in}}%
\pgfusepath{clip}%
\pgfsetrectcap%
\pgfsetroundjoin%
\pgfsetlinewidth{1.505625pt}%
\definecolor{currentstroke}{rgb}{1.000000,0.000000,0.000000}%
\pgfsetstrokecolor{currentstroke}%
\pgfsetdash{}{0pt}%
\pgfpathmoveto{\pgfqpoint{0.864588in}{1.453799in}}%
\pgfpathlineto{\pgfqpoint{0.883791in}{1.446920in}}%
\pgfusepath{stroke}%
\end{pgfscope}%
\begin{pgfscope}%
\pgfpathrectangle{\pgfqpoint{0.100000in}{0.220728in}}{\pgfqpoint{3.696000in}{3.696000in}}%
\pgfusepath{clip}%
\pgfsetrectcap%
\pgfsetroundjoin%
\pgfsetlinewidth{1.505625pt}%
\definecolor{currentstroke}{rgb}{1.000000,0.000000,0.000000}%
\pgfsetstrokecolor{currentstroke}%
\pgfsetdash{}{0pt}%
\pgfpathmoveto{\pgfqpoint{0.864588in}{1.453799in}}%
\pgfpathlineto{\pgfqpoint{0.883791in}{1.446920in}}%
\pgfusepath{stroke}%
\end{pgfscope}%
\begin{pgfscope}%
\pgfpathrectangle{\pgfqpoint{0.100000in}{0.220728in}}{\pgfqpoint{3.696000in}{3.696000in}}%
\pgfusepath{clip}%
\pgfsetrectcap%
\pgfsetroundjoin%
\pgfsetlinewidth{1.505625pt}%
\definecolor{currentstroke}{rgb}{1.000000,0.000000,0.000000}%
\pgfsetstrokecolor{currentstroke}%
\pgfsetdash{}{0pt}%
\pgfpathmoveto{\pgfqpoint{0.864588in}{1.453799in}}%
\pgfpathlineto{\pgfqpoint{0.883791in}{1.446920in}}%
\pgfusepath{stroke}%
\end{pgfscope}%
\begin{pgfscope}%
\pgfpathrectangle{\pgfqpoint{0.100000in}{0.220728in}}{\pgfqpoint{3.696000in}{3.696000in}}%
\pgfusepath{clip}%
\pgfsetrectcap%
\pgfsetroundjoin%
\pgfsetlinewidth{1.505625pt}%
\definecolor{currentstroke}{rgb}{1.000000,0.000000,0.000000}%
\pgfsetstrokecolor{currentstroke}%
\pgfsetdash{}{0pt}%
\pgfpathmoveto{\pgfqpoint{0.864588in}{1.453799in}}%
\pgfpathlineto{\pgfqpoint{0.883791in}{1.446920in}}%
\pgfusepath{stroke}%
\end{pgfscope}%
\begin{pgfscope}%
\pgfpathrectangle{\pgfqpoint{0.100000in}{0.220728in}}{\pgfqpoint{3.696000in}{3.696000in}}%
\pgfusepath{clip}%
\pgfsetrectcap%
\pgfsetroundjoin%
\pgfsetlinewidth{1.505625pt}%
\definecolor{currentstroke}{rgb}{1.000000,0.000000,0.000000}%
\pgfsetstrokecolor{currentstroke}%
\pgfsetdash{}{0pt}%
\pgfpathmoveto{\pgfqpoint{0.864588in}{1.453799in}}%
\pgfpathlineto{\pgfqpoint{0.883791in}{1.446920in}}%
\pgfusepath{stroke}%
\end{pgfscope}%
\begin{pgfscope}%
\pgfpathrectangle{\pgfqpoint{0.100000in}{0.220728in}}{\pgfqpoint{3.696000in}{3.696000in}}%
\pgfusepath{clip}%
\pgfsetrectcap%
\pgfsetroundjoin%
\pgfsetlinewidth{1.505625pt}%
\definecolor{currentstroke}{rgb}{1.000000,0.000000,0.000000}%
\pgfsetstrokecolor{currentstroke}%
\pgfsetdash{}{0pt}%
\pgfpathmoveto{\pgfqpoint{0.864588in}{1.453799in}}%
\pgfpathlineto{\pgfqpoint{0.883791in}{1.446920in}}%
\pgfusepath{stroke}%
\end{pgfscope}%
\begin{pgfscope}%
\pgfpathrectangle{\pgfqpoint{0.100000in}{0.220728in}}{\pgfqpoint{3.696000in}{3.696000in}}%
\pgfusepath{clip}%
\pgfsetrectcap%
\pgfsetroundjoin%
\pgfsetlinewidth{1.505625pt}%
\definecolor{currentstroke}{rgb}{1.000000,0.000000,0.000000}%
\pgfsetstrokecolor{currentstroke}%
\pgfsetdash{}{0pt}%
\pgfpathmoveto{\pgfqpoint{0.864588in}{1.453799in}}%
\pgfpathlineto{\pgfqpoint{0.883791in}{1.446920in}}%
\pgfusepath{stroke}%
\end{pgfscope}%
\begin{pgfscope}%
\pgfpathrectangle{\pgfqpoint{0.100000in}{0.220728in}}{\pgfqpoint{3.696000in}{3.696000in}}%
\pgfusepath{clip}%
\pgfsetrectcap%
\pgfsetroundjoin%
\pgfsetlinewidth{1.505625pt}%
\definecolor{currentstroke}{rgb}{1.000000,0.000000,0.000000}%
\pgfsetstrokecolor{currentstroke}%
\pgfsetdash{}{0pt}%
\pgfpathmoveto{\pgfqpoint{0.864588in}{1.453799in}}%
\pgfpathlineto{\pgfqpoint{0.883791in}{1.446920in}}%
\pgfusepath{stroke}%
\end{pgfscope}%
\begin{pgfscope}%
\pgfpathrectangle{\pgfqpoint{0.100000in}{0.220728in}}{\pgfqpoint{3.696000in}{3.696000in}}%
\pgfusepath{clip}%
\pgfsetrectcap%
\pgfsetroundjoin%
\pgfsetlinewidth{1.505625pt}%
\definecolor{currentstroke}{rgb}{1.000000,0.000000,0.000000}%
\pgfsetstrokecolor{currentstroke}%
\pgfsetdash{}{0pt}%
\pgfpathmoveto{\pgfqpoint{0.864588in}{1.453799in}}%
\pgfpathlineto{\pgfqpoint{0.883791in}{1.446920in}}%
\pgfusepath{stroke}%
\end{pgfscope}%
\begin{pgfscope}%
\pgfpathrectangle{\pgfqpoint{0.100000in}{0.220728in}}{\pgfqpoint{3.696000in}{3.696000in}}%
\pgfusepath{clip}%
\pgfsetrectcap%
\pgfsetroundjoin%
\pgfsetlinewidth{1.505625pt}%
\definecolor{currentstroke}{rgb}{1.000000,0.000000,0.000000}%
\pgfsetstrokecolor{currentstroke}%
\pgfsetdash{}{0pt}%
\pgfpathmoveto{\pgfqpoint{0.864588in}{1.453799in}}%
\pgfpathlineto{\pgfqpoint{0.883791in}{1.446920in}}%
\pgfusepath{stroke}%
\end{pgfscope}%
\begin{pgfscope}%
\pgfpathrectangle{\pgfqpoint{0.100000in}{0.220728in}}{\pgfqpoint{3.696000in}{3.696000in}}%
\pgfusepath{clip}%
\pgfsetrectcap%
\pgfsetroundjoin%
\pgfsetlinewidth{1.505625pt}%
\definecolor{currentstroke}{rgb}{1.000000,0.000000,0.000000}%
\pgfsetstrokecolor{currentstroke}%
\pgfsetdash{}{0pt}%
\pgfpathmoveto{\pgfqpoint{0.864588in}{1.453799in}}%
\pgfpathlineto{\pgfqpoint{0.883791in}{1.446920in}}%
\pgfusepath{stroke}%
\end{pgfscope}%
\begin{pgfscope}%
\pgfpathrectangle{\pgfqpoint{0.100000in}{0.220728in}}{\pgfqpoint{3.696000in}{3.696000in}}%
\pgfusepath{clip}%
\pgfsetrectcap%
\pgfsetroundjoin%
\pgfsetlinewidth{1.505625pt}%
\definecolor{currentstroke}{rgb}{1.000000,0.000000,0.000000}%
\pgfsetstrokecolor{currentstroke}%
\pgfsetdash{}{0pt}%
\pgfpathmoveto{\pgfqpoint{0.864588in}{1.453799in}}%
\pgfpathlineto{\pgfqpoint{0.883791in}{1.446920in}}%
\pgfusepath{stroke}%
\end{pgfscope}%
\begin{pgfscope}%
\pgfpathrectangle{\pgfqpoint{0.100000in}{0.220728in}}{\pgfqpoint{3.696000in}{3.696000in}}%
\pgfusepath{clip}%
\pgfsetrectcap%
\pgfsetroundjoin%
\pgfsetlinewidth{1.505625pt}%
\definecolor{currentstroke}{rgb}{1.000000,0.000000,0.000000}%
\pgfsetstrokecolor{currentstroke}%
\pgfsetdash{}{0pt}%
\pgfpathmoveto{\pgfqpoint{0.864588in}{1.453799in}}%
\pgfpathlineto{\pgfqpoint{0.883791in}{1.446920in}}%
\pgfusepath{stroke}%
\end{pgfscope}%
\begin{pgfscope}%
\pgfpathrectangle{\pgfqpoint{0.100000in}{0.220728in}}{\pgfqpoint{3.696000in}{3.696000in}}%
\pgfusepath{clip}%
\pgfsetrectcap%
\pgfsetroundjoin%
\pgfsetlinewidth{1.505625pt}%
\definecolor{currentstroke}{rgb}{1.000000,0.000000,0.000000}%
\pgfsetstrokecolor{currentstroke}%
\pgfsetdash{}{0pt}%
\pgfpathmoveto{\pgfqpoint{0.864588in}{1.453799in}}%
\pgfpathlineto{\pgfqpoint{0.883791in}{1.446920in}}%
\pgfusepath{stroke}%
\end{pgfscope}%
\begin{pgfscope}%
\pgfpathrectangle{\pgfqpoint{0.100000in}{0.220728in}}{\pgfqpoint{3.696000in}{3.696000in}}%
\pgfusepath{clip}%
\pgfsetrectcap%
\pgfsetroundjoin%
\pgfsetlinewidth{1.505625pt}%
\definecolor{currentstroke}{rgb}{1.000000,0.000000,0.000000}%
\pgfsetstrokecolor{currentstroke}%
\pgfsetdash{}{0pt}%
\pgfpathmoveto{\pgfqpoint{0.864588in}{1.453799in}}%
\pgfpathlineto{\pgfqpoint{0.883791in}{1.446920in}}%
\pgfusepath{stroke}%
\end{pgfscope}%
\begin{pgfscope}%
\pgfpathrectangle{\pgfqpoint{0.100000in}{0.220728in}}{\pgfqpoint{3.696000in}{3.696000in}}%
\pgfusepath{clip}%
\pgfsetrectcap%
\pgfsetroundjoin%
\pgfsetlinewidth{1.505625pt}%
\definecolor{currentstroke}{rgb}{1.000000,0.000000,0.000000}%
\pgfsetstrokecolor{currentstroke}%
\pgfsetdash{}{0pt}%
\pgfpathmoveto{\pgfqpoint{0.864588in}{1.453799in}}%
\pgfpathlineto{\pgfqpoint{0.883791in}{1.446920in}}%
\pgfusepath{stroke}%
\end{pgfscope}%
\begin{pgfscope}%
\pgfpathrectangle{\pgfqpoint{0.100000in}{0.220728in}}{\pgfqpoint{3.696000in}{3.696000in}}%
\pgfusepath{clip}%
\pgfsetrectcap%
\pgfsetroundjoin%
\pgfsetlinewidth{1.505625pt}%
\definecolor{currentstroke}{rgb}{1.000000,0.000000,0.000000}%
\pgfsetstrokecolor{currentstroke}%
\pgfsetdash{}{0pt}%
\pgfpathmoveto{\pgfqpoint{0.864588in}{1.453799in}}%
\pgfpathlineto{\pgfqpoint{0.883791in}{1.446920in}}%
\pgfusepath{stroke}%
\end{pgfscope}%
\begin{pgfscope}%
\pgfpathrectangle{\pgfqpoint{0.100000in}{0.220728in}}{\pgfqpoint{3.696000in}{3.696000in}}%
\pgfusepath{clip}%
\pgfsetrectcap%
\pgfsetroundjoin%
\pgfsetlinewidth{1.505625pt}%
\definecolor{currentstroke}{rgb}{1.000000,0.000000,0.000000}%
\pgfsetstrokecolor{currentstroke}%
\pgfsetdash{}{0pt}%
\pgfpathmoveto{\pgfqpoint{0.864588in}{1.453799in}}%
\pgfpathlineto{\pgfqpoint{0.883791in}{1.446920in}}%
\pgfusepath{stroke}%
\end{pgfscope}%
\begin{pgfscope}%
\pgfpathrectangle{\pgfqpoint{0.100000in}{0.220728in}}{\pgfqpoint{3.696000in}{3.696000in}}%
\pgfusepath{clip}%
\pgfsetrectcap%
\pgfsetroundjoin%
\pgfsetlinewidth{1.505625pt}%
\definecolor{currentstroke}{rgb}{1.000000,0.000000,0.000000}%
\pgfsetstrokecolor{currentstroke}%
\pgfsetdash{}{0pt}%
\pgfpathmoveto{\pgfqpoint{0.864588in}{1.453799in}}%
\pgfpathlineto{\pgfqpoint{0.883791in}{1.446920in}}%
\pgfusepath{stroke}%
\end{pgfscope}%
\begin{pgfscope}%
\pgfpathrectangle{\pgfqpoint{0.100000in}{0.220728in}}{\pgfqpoint{3.696000in}{3.696000in}}%
\pgfusepath{clip}%
\pgfsetrectcap%
\pgfsetroundjoin%
\pgfsetlinewidth{1.505625pt}%
\definecolor{currentstroke}{rgb}{1.000000,0.000000,0.000000}%
\pgfsetstrokecolor{currentstroke}%
\pgfsetdash{}{0pt}%
\pgfpathmoveto{\pgfqpoint{0.864588in}{1.453799in}}%
\pgfpathlineto{\pgfqpoint{0.883791in}{1.446920in}}%
\pgfusepath{stroke}%
\end{pgfscope}%
\begin{pgfscope}%
\pgfpathrectangle{\pgfqpoint{0.100000in}{0.220728in}}{\pgfqpoint{3.696000in}{3.696000in}}%
\pgfusepath{clip}%
\pgfsetrectcap%
\pgfsetroundjoin%
\pgfsetlinewidth{1.505625pt}%
\definecolor{currentstroke}{rgb}{1.000000,0.000000,0.000000}%
\pgfsetstrokecolor{currentstroke}%
\pgfsetdash{}{0pt}%
\pgfpathmoveto{\pgfqpoint{0.864588in}{1.453799in}}%
\pgfpathlineto{\pgfqpoint{0.883791in}{1.446920in}}%
\pgfusepath{stroke}%
\end{pgfscope}%
\begin{pgfscope}%
\pgfpathrectangle{\pgfqpoint{0.100000in}{0.220728in}}{\pgfqpoint{3.696000in}{3.696000in}}%
\pgfusepath{clip}%
\pgfsetrectcap%
\pgfsetroundjoin%
\pgfsetlinewidth{1.505625pt}%
\definecolor{currentstroke}{rgb}{1.000000,0.000000,0.000000}%
\pgfsetstrokecolor{currentstroke}%
\pgfsetdash{}{0pt}%
\pgfpathmoveto{\pgfqpoint{0.864588in}{1.453799in}}%
\pgfpathlineto{\pgfqpoint{0.883791in}{1.446920in}}%
\pgfusepath{stroke}%
\end{pgfscope}%
\begin{pgfscope}%
\pgfpathrectangle{\pgfqpoint{0.100000in}{0.220728in}}{\pgfqpoint{3.696000in}{3.696000in}}%
\pgfusepath{clip}%
\pgfsetrectcap%
\pgfsetroundjoin%
\pgfsetlinewidth{1.505625pt}%
\definecolor{currentstroke}{rgb}{1.000000,0.000000,0.000000}%
\pgfsetstrokecolor{currentstroke}%
\pgfsetdash{}{0pt}%
\pgfpathmoveto{\pgfqpoint{0.864588in}{1.453799in}}%
\pgfpathlineto{\pgfqpoint{0.883791in}{1.446920in}}%
\pgfusepath{stroke}%
\end{pgfscope}%
\begin{pgfscope}%
\pgfpathrectangle{\pgfqpoint{0.100000in}{0.220728in}}{\pgfqpoint{3.696000in}{3.696000in}}%
\pgfusepath{clip}%
\pgfsetrectcap%
\pgfsetroundjoin%
\pgfsetlinewidth{1.505625pt}%
\definecolor{currentstroke}{rgb}{1.000000,0.000000,0.000000}%
\pgfsetstrokecolor{currentstroke}%
\pgfsetdash{}{0pt}%
\pgfpathmoveto{\pgfqpoint{0.864588in}{1.453799in}}%
\pgfpathlineto{\pgfqpoint{0.883791in}{1.446920in}}%
\pgfusepath{stroke}%
\end{pgfscope}%
\begin{pgfscope}%
\pgfpathrectangle{\pgfqpoint{0.100000in}{0.220728in}}{\pgfqpoint{3.696000in}{3.696000in}}%
\pgfusepath{clip}%
\pgfsetrectcap%
\pgfsetroundjoin%
\pgfsetlinewidth{1.505625pt}%
\definecolor{currentstroke}{rgb}{1.000000,0.000000,0.000000}%
\pgfsetstrokecolor{currentstroke}%
\pgfsetdash{}{0pt}%
\pgfpathmoveto{\pgfqpoint{0.864588in}{1.453799in}}%
\pgfpathlineto{\pgfqpoint{0.883791in}{1.446920in}}%
\pgfusepath{stroke}%
\end{pgfscope}%
\begin{pgfscope}%
\pgfpathrectangle{\pgfqpoint{0.100000in}{0.220728in}}{\pgfqpoint{3.696000in}{3.696000in}}%
\pgfusepath{clip}%
\pgfsetrectcap%
\pgfsetroundjoin%
\pgfsetlinewidth{1.505625pt}%
\definecolor{currentstroke}{rgb}{1.000000,0.000000,0.000000}%
\pgfsetstrokecolor{currentstroke}%
\pgfsetdash{}{0pt}%
\pgfpathmoveto{\pgfqpoint{0.864588in}{1.453799in}}%
\pgfpathlineto{\pgfqpoint{0.883791in}{1.446920in}}%
\pgfusepath{stroke}%
\end{pgfscope}%
\begin{pgfscope}%
\pgfpathrectangle{\pgfqpoint{0.100000in}{0.220728in}}{\pgfqpoint{3.696000in}{3.696000in}}%
\pgfusepath{clip}%
\pgfsetrectcap%
\pgfsetroundjoin%
\pgfsetlinewidth{1.505625pt}%
\definecolor{currentstroke}{rgb}{1.000000,0.000000,0.000000}%
\pgfsetstrokecolor{currentstroke}%
\pgfsetdash{}{0pt}%
\pgfpathmoveto{\pgfqpoint{0.864588in}{1.453799in}}%
\pgfpathlineto{\pgfqpoint{0.883791in}{1.446920in}}%
\pgfusepath{stroke}%
\end{pgfscope}%
\begin{pgfscope}%
\pgfpathrectangle{\pgfqpoint{0.100000in}{0.220728in}}{\pgfqpoint{3.696000in}{3.696000in}}%
\pgfusepath{clip}%
\pgfsetrectcap%
\pgfsetroundjoin%
\pgfsetlinewidth{1.505625pt}%
\definecolor{currentstroke}{rgb}{1.000000,0.000000,0.000000}%
\pgfsetstrokecolor{currentstroke}%
\pgfsetdash{}{0pt}%
\pgfpathmoveto{\pgfqpoint{0.864588in}{1.453799in}}%
\pgfpathlineto{\pgfqpoint{0.883791in}{1.446920in}}%
\pgfusepath{stroke}%
\end{pgfscope}%
\begin{pgfscope}%
\pgfpathrectangle{\pgfqpoint{0.100000in}{0.220728in}}{\pgfqpoint{3.696000in}{3.696000in}}%
\pgfusepath{clip}%
\pgfsetrectcap%
\pgfsetroundjoin%
\pgfsetlinewidth{1.505625pt}%
\definecolor{currentstroke}{rgb}{1.000000,0.000000,0.000000}%
\pgfsetstrokecolor{currentstroke}%
\pgfsetdash{}{0pt}%
\pgfpathmoveto{\pgfqpoint{0.864588in}{1.453799in}}%
\pgfpathlineto{\pgfqpoint{0.883791in}{1.446920in}}%
\pgfusepath{stroke}%
\end{pgfscope}%
\begin{pgfscope}%
\pgfpathrectangle{\pgfqpoint{0.100000in}{0.220728in}}{\pgfqpoint{3.696000in}{3.696000in}}%
\pgfusepath{clip}%
\pgfsetrectcap%
\pgfsetroundjoin%
\pgfsetlinewidth{1.505625pt}%
\definecolor{currentstroke}{rgb}{1.000000,0.000000,0.000000}%
\pgfsetstrokecolor{currentstroke}%
\pgfsetdash{}{0pt}%
\pgfpathmoveto{\pgfqpoint{0.864588in}{1.453799in}}%
\pgfpathlineto{\pgfqpoint{0.883791in}{1.446920in}}%
\pgfusepath{stroke}%
\end{pgfscope}%
\begin{pgfscope}%
\pgfpathrectangle{\pgfqpoint{0.100000in}{0.220728in}}{\pgfqpoint{3.696000in}{3.696000in}}%
\pgfusepath{clip}%
\pgfsetrectcap%
\pgfsetroundjoin%
\pgfsetlinewidth{1.505625pt}%
\definecolor{currentstroke}{rgb}{1.000000,0.000000,0.000000}%
\pgfsetstrokecolor{currentstroke}%
\pgfsetdash{}{0pt}%
\pgfpathmoveto{\pgfqpoint{0.864588in}{1.453799in}}%
\pgfpathlineto{\pgfqpoint{0.883791in}{1.446920in}}%
\pgfusepath{stroke}%
\end{pgfscope}%
\begin{pgfscope}%
\pgfpathrectangle{\pgfqpoint{0.100000in}{0.220728in}}{\pgfqpoint{3.696000in}{3.696000in}}%
\pgfusepath{clip}%
\pgfsetrectcap%
\pgfsetroundjoin%
\pgfsetlinewidth{1.505625pt}%
\definecolor{currentstroke}{rgb}{1.000000,0.000000,0.000000}%
\pgfsetstrokecolor{currentstroke}%
\pgfsetdash{}{0pt}%
\pgfpathmoveto{\pgfqpoint{0.864588in}{1.453799in}}%
\pgfpathlineto{\pgfqpoint{0.883791in}{1.446920in}}%
\pgfusepath{stroke}%
\end{pgfscope}%
\begin{pgfscope}%
\pgfpathrectangle{\pgfqpoint{0.100000in}{0.220728in}}{\pgfqpoint{3.696000in}{3.696000in}}%
\pgfusepath{clip}%
\pgfsetrectcap%
\pgfsetroundjoin%
\pgfsetlinewidth{1.505625pt}%
\definecolor{currentstroke}{rgb}{1.000000,0.000000,0.000000}%
\pgfsetstrokecolor{currentstroke}%
\pgfsetdash{}{0pt}%
\pgfpathmoveto{\pgfqpoint{0.864588in}{1.453799in}}%
\pgfpathlineto{\pgfqpoint{0.883791in}{1.446920in}}%
\pgfusepath{stroke}%
\end{pgfscope}%
\begin{pgfscope}%
\pgfpathrectangle{\pgfqpoint{0.100000in}{0.220728in}}{\pgfqpoint{3.696000in}{3.696000in}}%
\pgfusepath{clip}%
\pgfsetrectcap%
\pgfsetroundjoin%
\pgfsetlinewidth{1.505625pt}%
\definecolor{currentstroke}{rgb}{1.000000,0.000000,0.000000}%
\pgfsetstrokecolor{currentstroke}%
\pgfsetdash{}{0pt}%
\pgfpathmoveto{\pgfqpoint{0.864588in}{1.453799in}}%
\pgfpathlineto{\pgfqpoint{0.883791in}{1.446920in}}%
\pgfusepath{stroke}%
\end{pgfscope}%
\begin{pgfscope}%
\pgfpathrectangle{\pgfqpoint{0.100000in}{0.220728in}}{\pgfqpoint{3.696000in}{3.696000in}}%
\pgfusepath{clip}%
\pgfsetrectcap%
\pgfsetroundjoin%
\pgfsetlinewidth{1.505625pt}%
\definecolor{currentstroke}{rgb}{1.000000,0.000000,0.000000}%
\pgfsetstrokecolor{currentstroke}%
\pgfsetdash{}{0pt}%
\pgfpathmoveto{\pgfqpoint{0.864588in}{1.453799in}}%
\pgfpathlineto{\pgfqpoint{0.883791in}{1.446920in}}%
\pgfusepath{stroke}%
\end{pgfscope}%
\begin{pgfscope}%
\pgfpathrectangle{\pgfqpoint{0.100000in}{0.220728in}}{\pgfqpoint{3.696000in}{3.696000in}}%
\pgfusepath{clip}%
\pgfsetrectcap%
\pgfsetroundjoin%
\pgfsetlinewidth{1.505625pt}%
\definecolor{currentstroke}{rgb}{1.000000,0.000000,0.000000}%
\pgfsetstrokecolor{currentstroke}%
\pgfsetdash{}{0pt}%
\pgfpathmoveto{\pgfqpoint{0.864588in}{1.453799in}}%
\pgfpathlineto{\pgfqpoint{0.883791in}{1.446920in}}%
\pgfusepath{stroke}%
\end{pgfscope}%
\begin{pgfscope}%
\pgfpathrectangle{\pgfqpoint{0.100000in}{0.220728in}}{\pgfqpoint{3.696000in}{3.696000in}}%
\pgfusepath{clip}%
\pgfsetrectcap%
\pgfsetroundjoin%
\pgfsetlinewidth{1.505625pt}%
\definecolor{currentstroke}{rgb}{1.000000,0.000000,0.000000}%
\pgfsetstrokecolor{currentstroke}%
\pgfsetdash{}{0pt}%
\pgfpathmoveto{\pgfqpoint{0.864588in}{1.453799in}}%
\pgfpathlineto{\pgfqpoint{0.883791in}{1.446920in}}%
\pgfusepath{stroke}%
\end{pgfscope}%
\begin{pgfscope}%
\pgfpathrectangle{\pgfqpoint{0.100000in}{0.220728in}}{\pgfqpoint{3.696000in}{3.696000in}}%
\pgfusepath{clip}%
\pgfsetrectcap%
\pgfsetroundjoin%
\pgfsetlinewidth{1.505625pt}%
\definecolor{currentstroke}{rgb}{1.000000,0.000000,0.000000}%
\pgfsetstrokecolor{currentstroke}%
\pgfsetdash{}{0pt}%
\pgfpathmoveto{\pgfqpoint{0.864588in}{1.453799in}}%
\pgfpathlineto{\pgfqpoint{0.883791in}{1.446920in}}%
\pgfusepath{stroke}%
\end{pgfscope}%
\begin{pgfscope}%
\pgfpathrectangle{\pgfqpoint{0.100000in}{0.220728in}}{\pgfqpoint{3.696000in}{3.696000in}}%
\pgfusepath{clip}%
\pgfsetrectcap%
\pgfsetroundjoin%
\pgfsetlinewidth{1.505625pt}%
\definecolor{currentstroke}{rgb}{1.000000,0.000000,0.000000}%
\pgfsetstrokecolor{currentstroke}%
\pgfsetdash{}{0pt}%
\pgfpathmoveto{\pgfqpoint{0.864588in}{1.453799in}}%
\pgfpathlineto{\pgfqpoint{0.883791in}{1.446920in}}%
\pgfusepath{stroke}%
\end{pgfscope}%
\begin{pgfscope}%
\pgfpathrectangle{\pgfqpoint{0.100000in}{0.220728in}}{\pgfqpoint{3.696000in}{3.696000in}}%
\pgfusepath{clip}%
\pgfsetrectcap%
\pgfsetroundjoin%
\pgfsetlinewidth{1.505625pt}%
\definecolor{currentstroke}{rgb}{1.000000,0.000000,0.000000}%
\pgfsetstrokecolor{currentstroke}%
\pgfsetdash{}{0pt}%
\pgfpathmoveto{\pgfqpoint{0.864588in}{1.453799in}}%
\pgfpathlineto{\pgfqpoint{0.883791in}{1.446920in}}%
\pgfusepath{stroke}%
\end{pgfscope}%
\begin{pgfscope}%
\pgfpathrectangle{\pgfqpoint{0.100000in}{0.220728in}}{\pgfqpoint{3.696000in}{3.696000in}}%
\pgfusepath{clip}%
\pgfsetrectcap%
\pgfsetroundjoin%
\pgfsetlinewidth{1.505625pt}%
\definecolor{currentstroke}{rgb}{1.000000,0.000000,0.000000}%
\pgfsetstrokecolor{currentstroke}%
\pgfsetdash{}{0pt}%
\pgfpathmoveto{\pgfqpoint{0.864588in}{1.453799in}}%
\pgfpathlineto{\pgfqpoint{0.883791in}{1.446920in}}%
\pgfusepath{stroke}%
\end{pgfscope}%
\begin{pgfscope}%
\pgfpathrectangle{\pgfqpoint{0.100000in}{0.220728in}}{\pgfqpoint{3.696000in}{3.696000in}}%
\pgfusepath{clip}%
\pgfsetrectcap%
\pgfsetroundjoin%
\pgfsetlinewidth{1.505625pt}%
\definecolor{currentstroke}{rgb}{1.000000,0.000000,0.000000}%
\pgfsetstrokecolor{currentstroke}%
\pgfsetdash{}{0pt}%
\pgfpathmoveto{\pgfqpoint{0.864588in}{1.453799in}}%
\pgfpathlineto{\pgfqpoint{0.883791in}{1.446920in}}%
\pgfusepath{stroke}%
\end{pgfscope}%
\begin{pgfscope}%
\pgfpathrectangle{\pgfqpoint{0.100000in}{0.220728in}}{\pgfqpoint{3.696000in}{3.696000in}}%
\pgfusepath{clip}%
\pgfsetrectcap%
\pgfsetroundjoin%
\pgfsetlinewidth{1.505625pt}%
\definecolor{currentstroke}{rgb}{1.000000,0.000000,0.000000}%
\pgfsetstrokecolor{currentstroke}%
\pgfsetdash{}{0pt}%
\pgfpathmoveto{\pgfqpoint{0.864588in}{1.453799in}}%
\pgfpathlineto{\pgfqpoint{0.883791in}{1.446920in}}%
\pgfusepath{stroke}%
\end{pgfscope}%
\begin{pgfscope}%
\pgfpathrectangle{\pgfqpoint{0.100000in}{0.220728in}}{\pgfqpoint{3.696000in}{3.696000in}}%
\pgfusepath{clip}%
\pgfsetrectcap%
\pgfsetroundjoin%
\pgfsetlinewidth{1.505625pt}%
\definecolor{currentstroke}{rgb}{1.000000,0.000000,0.000000}%
\pgfsetstrokecolor{currentstroke}%
\pgfsetdash{}{0pt}%
\pgfpathmoveto{\pgfqpoint{0.864588in}{1.453799in}}%
\pgfpathlineto{\pgfqpoint{0.883791in}{1.446920in}}%
\pgfusepath{stroke}%
\end{pgfscope}%
\begin{pgfscope}%
\pgfpathrectangle{\pgfqpoint{0.100000in}{0.220728in}}{\pgfqpoint{3.696000in}{3.696000in}}%
\pgfusepath{clip}%
\pgfsetrectcap%
\pgfsetroundjoin%
\pgfsetlinewidth{1.505625pt}%
\definecolor{currentstroke}{rgb}{1.000000,0.000000,0.000000}%
\pgfsetstrokecolor{currentstroke}%
\pgfsetdash{}{0pt}%
\pgfpathmoveto{\pgfqpoint{0.864588in}{1.453799in}}%
\pgfpathlineto{\pgfqpoint{0.883791in}{1.446920in}}%
\pgfusepath{stroke}%
\end{pgfscope}%
\begin{pgfscope}%
\pgfpathrectangle{\pgfqpoint{0.100000in}{0.220728in}}{\pgfqpoint{3.696000in}{3.696000in}}%
\pgfusepath{clip}%
\pgfsetrectcap%
\pgfsetroundjoin%
\pgfsetlinewidth{1.505625pt}%
\definecolor{currentstroke}{rgb}{1.000000,0.000000,0.000000}%
\pgfsetstrokecolor{currentstroke}%
\pgfsetdash{}{0pt}%
\pgfpathmoveto{\pgfqpoint{0.864588in}{1.453799in}}%
\pgfpathlineto{\pgfqpoint{0.883791in}{1.446920in}}%
\pgfusepath{stroke}%
\end{pgfscope}%
\begin{pgfscope}%
\pgfpathrectangle{\pgfqpoint{0.100000in}{0.220728in}}{\pgfqpoint{3.696000in}{3.696000in}}%
\pgfusepath{clip}%
\pgfsetrectcap%
\pgfsetroundjoin%
\pgfsetlinewidth{1.505625pt}%
\definecolor{currentstroke}{rgb}{1.000000,0.000000,0.000000}%
\pgfsetstrokecolor{currentstroke}%
\pgfsetdash{}{0pt}%
\pgfpathmoveto{\pgfqpoint{0.864588in}{1.453799in}}%
\pgfpathlineto{\pgfqpoint{0.883791in}{1.446920in}}%
\pgfusepath{stroke}%
\end{pgfscope}%
\begin{pgfscope}%
\pgfpathrectangle{\pgfqpoint{0.100000in}{0.220728in}}{\pgfqpoint{3.696000in}{3.696000in}}%
\pgfusepath{clip}%
\pgfsetrectcap%
\pgfsetroundjoin%
\pgfsetlinewidth{1.505625pt}%
\definecolor{currentstroke}{rgb}{1.000000,0.000000,0.000000}%
\pgfsetstrokecolor{currentstroke}%
\pgfsetdash{}{0pt}%
\pgfpathmoveto{\pgfqpoint{0.864588in}{1.453799in}}%
\pgfpathlineto{\pgfqpoint{0.883791in}{1.446920in}}%
\pgfusepath{stroke}%
\end{pgfscope}%
\begin{pgfscope}%
\pgfpathrectangle{\pgfqpoint{0.100000in}{0.220728in}}{\pgfqpoint{3.696000in}{3.696000in}}%
\pgfusepath{clip}%
\pgfsetrectcap%
\pgfsetroundjoin%
\pgfsetlinewidth{1.505625pt}%
\definecolor{currentstroke}{rgb}{1.000000,0.000000,0.000000}%
\pgfsetstrokecolor{currentstroke}%
\pgfsetdash{}{0pt}%
\pgfpathmoveto{\pgfqpoint{0.864588in}{1.453799in}}%
\pgfpathlineto{\pgfqpoint{0.883791in}{1.446920in}}%
\pgfusepath{stroke}%
\end{pgfscope}%
\begin{pgfscope}%
\pgfpathrectangle{\pgfqpoint{0.100000in}{0.220728in}}{\pgfqpoint{3.696000in}{3.696000in}}%
\pgfusepath{clip}%
\pgfsetrectcap%
\pgfsetroundjoin%
\pgfsetlinewidth{1.505625pt}%
\definecolor{currentstroke}{rgb}{1.000000,0.000000,0.000000}%
\pgfsetstrokecolor{currentstroke}%
\pgfsetdash{}{0pt}%
\pgfpathmoveto{\pgfqpoint{0.864588in}{1.453799in}}%
\pgfpathlineto{\pgfqpoint{0.883791in}{1.446920in}}%
\pgfusepath{stroke}%
\end{pgfscope}%
\begin{pgfscope}%
\pgfpathrectangle{\pgfqpoint{0.100000in}{0.220728in}}{\pgfqpoint{3.696000in}{3.696000in}}%
\pgfusepath{clip}%
\pgfsetrectcap%
\pgfsetroundjoin%
\pgfsetlinewidth{1.505625pt}%
\definecolor{currentstroke}{rgb}{1.000000,0.000000,0.000000}%
\pgfsetstrokecolor{currentstroke}%
\pgfsetdash{}{0pt}%
\pgfpathmoveto{\pgfqpoint{0.864588in}{1.453799in}}%
\pgfpathlineto{\pgfqpoint{0.883791in}{1.446920in}}%
\pgfusepath{stroke}%
\end{pgfscope}%
\begin{pgfscope}%
\pgfpathrectangle{\pgfqpoint{0.100000in}{0.220728in}}{\pgfqpoint{3.696000in}{3.696000in}}%
\pgfusepath{clip}%
\pgfsetrectcap%
\pgfsetroundjoin%
\pgfsetlinewidth{1.505625pt}%
\definecolor{currentstroke}{rgb}{1.000000,0.000000,0.000000}%
\pgfsetstrokecolor{currentstroke}%
\pgfsetdash{}{0pt}%
\pgfpathmoveto{\pgfqpoint{0.864588in}{1.453799in}}%
\pgfpathlineto{\pgfqpoint{0.883791in}{1.446920in}}%
\pgfusepath{stroke}%
\end{pgfscope}%
\begin{pgfscope}%
\pgfpathrectangle{\pgfqpoint{0.100000in}{0.220728in}}{\pgfqpoint{3.696000in}{3.696000in}}%
\pgfusepath{clip}%
\pgfsetrectcap%
\pgfsetroundjoin%
\pgfsetlinewidth{1.505625pt}%
\definecolor{currentstroke}{rgb}{1.000000,0.000000,0.000000}%
\pgfsetstrokecolor{currentstroke}%
\pgfsetdash{}{0pt}%
\pgfpathmoveto{\pgfqpoint{0.864588in}{1.453799in}}%
\pgfpathlineto{\pgfqpoint{0.883791in}{1.446920in}}%
\pgfusepath{stroke}%
\end{pgfscope}%
\begin{pgfscope}%
\pgfpathrectangle{\pgfqpoint{0.100000in}{0.220728in}}{\pgfqpoint{3.696000in}{3.696000in}}%
\pgfusepath{clip}%
\pgfsetrectcap%
\pgfsetroundjoin%
\pgfsetlinewidth{1.505625pt}%
\definecolor{currentstroke}{rgb}{1.000000,0.000000,0.000000}%
\pgfsetstrokecolor{currentstroke}%
\pgfsetdash{}{0pt}%
\pgfpathmoveto{\pgfqpoint{0.864588in}{1.453799in}}%
\pgfpathlineto{\pgfqpoint{0.883791in}{1.446920in}}%
\pgfusepath{stroke}%
\end{pgfscope}%
\begin{pgfscope}%
\pgfpathrectangle{\pgfqpoint{0.100000in}{0.220728in}}{\pgfqpoint{3.696000in}{3.696000in}}%
\pgfusepath{clip}%
\pgfsetrectcap%
\pgfsetroundjoin%
\pgfsetlinewidth{1.505625pt}%
\definecolor{currentstroke}{rgb}{1.000000,0.000000,0.000000}%
\pgfsetstrokecolor{currentstroke}%
\pgfsetdash{}{0pt}%
\pgfpathmoveto{\pgfqpoint{0.864588in}{1.453799in}}%
\pgfpathlineto{\pgfqpoint{0.883791in}{1.446920in}}%
\pgfusepath{stroke}%
\end{pgfscope}%
\begin{pgfscope}%
\pgfpathrectangle{\pgfqpoint{0.100000in}{0.220728in}}{\pgfqpoint{3.696000in}{3.696000in}}%
\pgfusepath{clip}%
\pgfsetrectcap%
\pgfsetroundjoin%
\pgfsetlinewidth{1.505625pt}%
\definecolor{currentstroke}{rgb}{1.000000,0.000000,0.000000}%
\pgfsetstrokecolor{currentstroke}%
\pgfsetdash{}{0pt}%
\pgfpathmoveto{\pgfqpoint{0.864588in}{1.453799in}}%
\pgfpathlineto{\pgfqpoint{0.883791in}{1.446920in}}%
\pgfusepath{stroke}%
\end{pgfscope}%
\begin{pgfscope}%
\pgfpathrectangle{\pgfqpoint{0.100000in}{0.220728in}}{\pgfqpoint{3.696000in}{3.696000in}}%
\pgfusepath{clip}%
\pgfsetrectcap%
\pgfsetroundjoin%
\pgfsetlinewidth{1.505625pt}%
\definecolor{currentstroke}{rgb}{1.000000,0.000000,0.000000}%
\pgfsetstrokecolor{currentstroke}%
\pgfsetdash{}{0pt}%
\pgfpathmoveto{\pgfqpoint{0.864588in}{1.453799in}}%
\pgfpathlineto{\pgfqpoint{0.883791in}{1.446920in}}%
\pgfusepath{stroke}%
\end{pgfscope}%
\begin{pgfscope}%
\pgfpathrectangle{\pgfqpoint{0.100000in}{0.220728in}}{\pgfqpoint{3.696000in}{3.696000in}}%
\pgfusepath{clip}%
\pgfsetrectcap%
\pgfsetroundjoin%
\pgfsetlinewidth{1.505625pt}%
\definecolor{currentstroke}{rgb}{1.000000,0.000000,0.000000}%
\pgfsetstrokecolor{currentstroke}%
\pgfsetdash{}{0pt}%
\pgfpathmoveto{\pgfqpoint{0.864588in}{1.453799in}}%
\pgfpathlineto{\pgfqpoint{0.883791in}{1.446920in}}%
\pgfusepath{stroke}%
\end{pgfscope}%
\begin{pgfscope}%
\pgfpathrectangle{\pgfqpoint{0.100000in}{0.220728in}}{\pgfqpoint{3.696000in}{3.696000in}}%
\pgfusepath{clip}%
\pgfsetrectcap%
\pgfsetroundjoin%
\pgfsetlinewidth{1.505625pt}%
\definecolor{currentstroke}{rgb}{1.000000,0.000000,0.000000}%
\pgfsetstrokecolor{currentstroke}%
\pgfsetdash{}{0pt}%
\pgfpathmoveto{\pgfqpoint{0.864588in}{1.453799in}}%
\pgfpathlineto{\pgfqpoint{0.883791in}{1.446920in}}%
\pgfusepath{stroke}%
\end{pgfscope}%
\begin{pgfscope}%
\pgfpathrectangle{\pgfqpoint{0.100000in}{0.220728in}}{\pgfqpoint{3.696000in}{3.696000in}}%
\pgfusepath{clip}%
\pgfsetrectcap%
\pgfsetroundjoin%
\pgfsetlinewidth{1.505625pt}%
\definecolor{currentstroke}{rgb}{1.000000,0.000000,0.000000}%
\pgfsetstrokecolor{currentstroke}%
\pgfsetdash{}{0pt}%
\pgfpathmoveto{\pgfqpoint{0.864588in}{1.453799in}}%
\pgfpathlineto{\pgfqpoint{0.883791in}{1.446920in}}%
\pgfusepath{stroke}%
\end{pgfscope}%
\begin{pgfscope}%
\pgfpathrectangle{\pgfqpoint{0.100000in}{0.220728in}}{\pgfqpoint{3.696000in}{3.696000in}}%
\pgfusepath{clip}%
\pgfsetrectcap%
\pgfsetroundjoin%
\pgfsetlinewidth{1.505625pt}%
\definecolor{currentstroke}{rgb}{1.000000,0.000000,0.000000}%
\pgfsetstrokecolor{currentstroke}%
\pgfsetdash{}{0pt}%
\pgfpathmoveto{\pgfqpoint{0.864588in}{1.453799in}}%
\pgfpathlineto{\pgfqpoint{0.883791in}{1.446920in}}%
\pgfusepath{stroke}%
\end{pgfscope}%
\begin{pgfscope}%
\pgfpathrectangle{\pgfqpoint{0.100000in}{0.220728in}}{\pgfqpoint{3.696000in}{3.696000in}}%
\pgfusepath{clip}%
\pgfsetrectcap%
\pgfsetroundjoin%
\pgfsetlinewidth{1.505625pt}%
\definecolor{currentstroke}{rgb}{1.000000,0.000000,0.000000}%
\pgfsetstrokecolor{currentstroke}%
\pgfsetdash{}{0pt}%
\pgfpathmoveto{\pgfqpoint{0.864588in}{1.453799in}}%
\pgfpathlineto{\pgfqpoint{0.883791in}{1.446920in}}%
\pgfusepath{stroke}%
\end{pgfscope}%
\begin{pgfscope}%
\pgfpathrectangle{\pgfqpoint{0.100000in}{0.220728in}}{\pgfqpoint{3.696000in}{3.696000in}}%
\pgfusepath{clip}%
\pgfsetrectcap%
\pgfsetroundjoin%
\pgfsetlinewidth{1.505625pt}%
\definecolor{currentstroke}{rgb}{1.000000,0.000000,0.000000}%
\pgfsetstrokecolor{currentstroke}%
\pgfsetdash{}{0pt}%
\pgfpathmoveto{\pgfqpoint{0.864588in}{1.453799in}}%
\pgfpathlineto{\pgfqpoint{0.883791in}{1.446920in}}%
\pgfusepath{stroke}%
\end{pgfscope}%
\begin{pgfscope}%
\pgfpathrectangle{\pgfqpoint{0.100000in}{0.220728in}}{\pgfqpoint{3.696000in}{3.696000in}}%
\pgfusepath{clip}%
\pgfsetrectcap%
\pgfsetroundjoin%
\pgfsetlinewidth{1.505625pt}%
\definecolor{currentstroke}{rgb}{1.000000,0.000000,0.000000}%
\pgfsetstrokecolor{currentstroke}%
\pgfsetdash{}{0pt}%
\pgfpathmoveto{\pgfqpoint{0.864588in}{1.453799in}}%
\pgfpathlineto{\pgfqpoint{0.883791in}{1.446920in}}%
\pgfusepath{stroke}%
\end{pgfscope}%
\begin{pgfscope}%
\pgfpathrectangle{\pgfqpoint{0.100000in}{0.220728in}}{\pgfqpoint{3.696000in}{3.696000in}}%
\pgfusepath{clip}%
\pgfsetrectcap%
\pgfsetroundjoin%
\pgfsetlinewidth{1.505625pt}%
\definecolor{currentstroke}{rgb}{1.000000,0.000000,0.000000}%
\pgfsetstrokecolor{currentstroke}%
\pgfsetdash{}{0pt}%
\pgfpathmoveto{\pgfqpoint{0.864588in}{1.453799in}}%
\pgfpathlineto{\pgfqpoint{0.883791in}{1.446920in}}%
\pgfusepath{stroke}%
\end{pgfscope}%
\begin{pgfscope}%
\pgfpathrectangle{\pgfqpoint{0.100000in}{0.220728in}}{\pgfqpoint{3.696000in}{3.696000in}}%
\pgfusepath{clip}%
\pgfsetrectcap%
\pgfsetroundjoin%
\pgfsetlinewidth{1.505625pt}%
\definecolor{currentstroke}{rgb}{1.000000,0.000000,0.000000}%
\pgfsetstrokecolor{currentstroke}%
\pgfsetdash{}{0pt}%
\pgfpathmoveto{\pgfqpoint{0.864588in}{1.453799in}}%
\pgfpathlineto{\pgfqpoint{0.883791in}{1.446920in}}%
\pgfusepath{stroke}%
\end{pgfscope}%
\begin{pgfscope}%
\pgfpathrectangle{\pgfqpoint{0.100000in}{0.220728in}}{\pgfqpoint{3.696000in}{3.696000in}}%
\pgfusepath{clip}%
\pgfsetrectcap%
\pgfsetroundjoin%
\pgfsetlinewidth{1.505625pt}%
\definecolor{currentstroke}{rgb}{1.000000,0.000000,0.000000}%
\pgfsetstrokecolor{currentstroke}%
\pgfsetdash{}{0pt}%
\pgfpathmoveto{\pgfqpoint{0.864588in}{1.453799in}}%
\pgfpathlineto{\pgfqpoint{0.883791in}{1.446920in}}%
\pgfusepath{stroke}%
\end{pgfscope}%
\begin{pgfscope}%
\pgfpathrectangle{\pgfqpoint{0.100000in}{0.220728in}}{\pgfqpoint{3.696000in}{3.696000in}}%
\pgfusepath{clip}%
\pgfsetrectcap%
\pgfsetroundjoin%
\pgfsetlinewidth{1.505625pt}%
\definecolor{currentstroke}{rgb}{1.000000,0.000000,0.000000}%
\pgfsetstrokecolor{currentstroke}%
\pgfsetdash{}{0pt}%
\pgfpathmoveto{\pgfqpoint{0.864588in}{1.453799in}}%
\pgfpathlineto{\pgfqpoint{0.883791in}{1.446920in}}%
\pgfusepath{stroke}%
\end{pgfscope}%
\begin{pgfscope}%
\pgfpathrectangle{\pgfqpoint{0.100000in}{0.220728in}}{\pgfqpoint{3.696000in}{3.696000in}}%
\pgfusepath{clip}%
\pgfsetrectcap%
\pgfsetroundjoin%
\pgfsetlinewidth{1.505625pt}%
\definecolor{currentstroke}{rgb}{1.000000,0.000000,0.000000}%
\pgfsetstrokecolor{currentstroke}%
\pgfsetdash{}{0pt}%
\pgfpathmoveto{\pgfqpoint{0.864588in}{1.453799in}}%
\pgfpathlineto{\pgfqpoint{0.883791in}{1.446920in}}%
\pgfusepath{stroke}%
\end{pgfscope}%
\begin{pgfscope}%
\pgfpathrectangle{\pgfqpoint{0.100000in}{0.220728in}}{\pgfqpoint{3.696000in}{3.696000in}}%
\pgfusepath{clip}%
\pgfsetrectcap%
\pgfsetroundjoin%
\pgfsetlinewidth{1.505625pt}%
\definecolor{currentstroke}{rgb}{1.000000,0.000000,0.000000}%
\pgfsetstrokecolor{currentstroke}%
\pgfsetdash{}{0pt}%
\pgfpathmoveto{\pgfqpoint{0.864588in}{1.453799in}}%
\pgfpathlineto{\pgfqpoint{0.883791in}{1.446920in}}%
\pgfusepath{stroke}%
\end{pgfscope}%
\begin{pgfscope}%
\pgfpathrectangle{\pgfqpoint{0.100000in}{0.220728in}}{\pgfqpoint{3.696000in}{3.696000in}}%
\pgfusepath{clip}%
\pgfsetrectcap%
\pgfsetroundjoin%
\pgfsetlinewidth{1.505625pt}%
\definecolor{currentstroke}{rgb}{1.000000,0.000000,0.000000}%
\pgfsetstrokecolor{currentstroke}%
\pgfsetdash{}{0pt}%
\pgfpathmoveto{\pgfqpoint{0.864588in}{1.453799in}}%
\pgfpathlineto{\pgfqpoint{0.883791in}{1.446920in}}%
\pgfusepath{stroke}%
\end{pgfscope}%
\begin{pgfscope}%
\pgfpathrectangle{\pgfqpoint{0.100000in}{0.220728in}}{\pgfqpoint{3.696000in}{3.696000in}}%
\pgfusepath{clip}%
\pgfsetrectcap%
\pgfsetroundjoin%
\pgfsetlinewidth{1.505625pt}%
\definecolor{currentstroke}{rgb}{1.000000,0.000000,0.000000}%
\pgfsetstrokecolor{currentstroke}%
\pgfsetdash{}{0pt}%
\pgfpathmoveto{\pgfqpoint{0.864588in}{1.453799in}}%
\pgfpathlineto{\pgfqpoint{0.883791in}{1.446920in}}%
\pgfusepath{stroke}%
\end{pgfscope}%
\begin{pgfscope}%
\pgfpathrectangle{\pgfqpoint{0.100000in}{0.220728in}}{\pgfqpoint{3.696000in}{3.696000in}}%
\pgfusepath{clip}%
\pgfsetrectcap%
\pgfsetroundjoin%
\pgfsetlinewidth{1.505625pt}%
\definecolor{currentstroke}{rgb}{1.000000,0.000000,0.000000}%
\pgfsetstrokecolor{currentstroke}%
\pgfsetdash{}{0pt}%
\pgfpathmoveto{\pgfqpoint{0.864588in}{1.453799in}}%
\pgfpathlineto{\pgfqpoint{0.883791in}{1.446920in}}%
\pgfusepath{stroke}%
\end{pgfscope}%
\begin{pgfscope}%
\pgfpathrectangle{\pgfqpoint{0.100000in}{0.220728in}}{\pgfqpoint{3.696000in}{3.696000in}}%
\pgfusepath{clip}%
\pgfsetrectcap%
\pgfsetroundjoin%
\pgfsetlinewidth{1.505625pt}%
\definecolor{currentstroke}{rgb}{1.000000,0.000000,0.000000}%
\pgfsetstrokecolor{currentstroke}%
\pgfsetdash{}{0pt}%
\pgfpathmoveto{\pgfqpoint{0.864588in}{1.453799in}}%
\pgfpathlineto{\pgfqpoint{0.883791in}{1.446920in}}%
\pgfusepath{stroke}%
\end{pgfscope}%
\begin{pgfscope}%
\pgfpathrectangle{\pgfqpoint{0.100000in}{0.220728in}}{\pgfqpoint{3.696000in}{3.696000in}}%
\pgfusepath{clip}%
\pgfsetrectcap%
\pgfsetroundjoin%
\pgfsetlinewidth{1.505625pt}%
\definecolor{currentstroke}{rgb}{1.000000,0.000000,0.000000}%
\pgfsetstrokecolor{currentstroke}%
\pgfsetdash{}{0pt}%
\pgfpathmoveto{\pgfqpoint{0.864035in}{1.453620in}}%
\pgfpathlineto{\pgfqpoint{0.883791in}{1.446920in}}%
\pgfusepath{stroke}%
\end{pgfscope}%
\begin{pgfscope}%
\pgfpathrectangle{\pgfqpoint{0.100000in}{0.220728in}}{\pgfqpoint{3.696000in}{3.696000in}}%
\pgfusepath{clip}%
\pgfsetrectcap%
\pgfsetroundjoin%
\pgfsetlinewidth{1.505625pt}%
\definecolor{currentstroke}{rgb}{1.000000,0.000000,0.000000}%
\pgfsetstrokecolor{currentstroke}%
\pgfsetdash{}{0pt}%
\pgfpathmoveto{\pgfqpoint{0.862639in}{1.453377in}}%
\pgfpathlineto{\pgfqpoint{0.883791in}{1.446920in}}%
\pgfusepath{stroke}%
\end{pgfscope}%
\begin{pgfscope}%
\pgfpathrectangle{\pgfqpoint{0.100000in}{0.220728in}}{\pgfqpoint{3.696000in}{3.696000in}}%
\pgfusepath{clip}%
\pgfsetrectcap%
\pgfsetroundjoin%
\pgfsetlinewidth{1.505625pt}%
\definecolor{currentstroke}{rgb}{1.000000,0.000000,0.000000}%
\pgfsetstrokecolor{currentstroke}%
\pgfsetdash{}{0pt}%
\pgfpathmoveto{\pgfqpoint{0.860032in}{1.453280in}}%
\pgfpathlineto{\pgfqpoint{0.883791in}{1.446920in}}%
\pgfusepath{stroke}%
\end{pgfscope}%
\begin{pgfscope}%
\pgfpathrectangle{\pgfqpoint{0.100000in}{0.220728in}}{\pgfqpoint{3.696000in}{3.696000in}}%
\pgfusepath{clip}%
\pgfsetrectcap%
\pgfsetroundjoin%
\pgfsetlinewidth{1.505625pt}%
\definecolor{currentstroke}{rgb}{1.000000,0.000000,0.000000}%
\pgfsetstrokecolor{currentstroke}%
\pgfsetdash{}{0pt}%
\pgfpathmoveto{\pgfqpoint{0.855887in}{1.453856in}}%
\pgfpathlineto{\pgfqpoint{0.883791in}{1.446920in}}%
\pgfusepath{stroke}%
\end{pgfscope}%
\begin{pgfscope}%
\pgfpathrectangle{\pgfqpoint{0.100000in}{0.220728in}}{\pgfqpoint{3.696000in}{3.696000in}}%
\pgfusepath{clip}%
\pgfsetrectcap%
\pgfsetroundjoin%
\pgfsetlinewidth{1.505625pt}%
\definecolor{currentstroke}{rgb}{1.000000,0.000000,0.000000}%
\pgfsetstrokecolor{currentstroke}%
\pgfsetdash{}{0pt}%
\pgfpathmoveto{\pgfqpoint{0.851847in}{1.455761in}}%
\pgfpathlineto{\pgfqpoint{0.883791in}{1.446920in}}%
\pgfusepath{stroke}%
\end{pgfscope}%
\begin{pgfscope}%
\pgfpathrectangle{\pgfqpoint{0.100000in}{0.220728in}}{\pgfqpoint{3.696000in}{3.696000in}}%
\pgfusepath{clip}%
\pgfsetrectcap%
\pgfsetroundjoin%
\pgfsetlinewidth{1.505625pt}%
\definecolor{currentstroke}{rgb}{1.000000,0.000000,0.000000}%
\pgfsetstrokecolor{currentstroke}%
\pgfsetdash{}{0pt}%
\pgfpathmoveto{\pgfqpoint{0.848760in}{1.462489in}}%
\pgfpathlineto{\pgfqpoint{0.883791in}{1.446920in}}%
\pgfusepath{stroke}%
\end{pgfscope}%
\begin{pgfscope}%
\pgfpathrectangle{\pgfqpoint{0.100000in}{0.220728in}}{\pgfqpoint{3.696000in}{3.696000in}}%
\pgfusepath{clip}%
\pgfsetrectcap%
\pgfsetroundjoin%
\pgfsetlinewidth{1.505625pt}%
\definecolor{currentstroke}{rgb}{1.000000,0.000000,0.000000}%
\pgfsetstrokecolor{currentstroke}%
\pgfsetdash{}{0pt}%
\pgfpathmoveto{\pgfqpoint{0.849354in}{1.469389in}}%
\pgfpathlineto{\pgfqpoint{0.883791in}{1.446920in}}%
\pgfusepath{stroke}%
\end{pgfscope}%
\begin{pgfscope}%
\pgfpathrectangle{\pgfqpoint{0.100000in}{0.220728in}}{\pgfqpoint{3.696000in}{3.696000in}}%
\pgfusepath{clip}%
\pgfsetrectcap%
\pgfsetroundjoin%
\pgfsetlinewidth{1.505625pt}%
\definecolor{currentstroke}{rgb}{1.000000,0.000000,0.000000}%
\pgfsetstrokecolor{currentstroke}%
\pgfsetdash{}{0pt}%
\pgfpathmoveto{\pgfqpoint{0.851099in}{1.473168in}}%
\pgfpathlineto{\pgfqpoint{0.883791in}{1.446920in}}%
\pgfusepath{stroke}%
\end{pgfscope}%
\begin{pgfscope}%
\pgfpathrectangle{\pgfqpoint{0.100000in}{0.220728in}}{\pgfqpoint{3.696000in}{3.696000in}}%
\pgfusepath{clip}%
\pgfsetrectcap%
\pgfsetroundjoin%
\pgfsetlinewidth{1.505625pt}%
\definecolor{currentstroke}{rgb}{1.000000,0.000000,0.000000}%
\pgfsetstrokecolor{currentstroke}%
\pgfsetdash{}{0pt}%
\pgfpathmoveto{\pgfqpoint{0.852889in}{1.473517in}}%
\pgfpathlineto{\pgfqpoint{0.883791in}{1.446920in}}%
\pgfusepath{stroke}%
\end{pgfscope}%
\begin{pgfscope}%
\pgfpathrectangle{\pgfqpoint{0.100000in}{0.220728in}}{\pgfqpoint{3.696000in}{3.696000in}}%
\pgfusepath{clip}%
\pgfsetrectcap%
\pgfsetroundjoin%
\pgfsetlinewidth{1.505625pt}%
\definecolor{currentstroke}{rgb}{1.000000,0.000000,0.000000}%
\pgfsetstrokecolor{currentstroke}%
\pgfsetdash{}{0pt}%
\pgfpathmoveto{\pgfqpoint{0.855668in}{1.474798in}}%
\pgfpathlineto{\pgfqpoint{0.883791in}{1.446920in}}%
\pgfusepath{stroke}%
\end{pgfscope}%
\begin{pgfscope}%
\pgfpathrectangle{\pgfqpoint{0.100000in}{0.220728in}}{\pgfqpoint{3.696000in}{3.696000in}}%
\pgfusepath{clip}%
\pgfsetrectcap%
\pgfsetroundjoin%
\pgfsetlinewidth{1.505625pt}%
\definecolor{currentstroke}{rgb}{1.000000,0.000000,0.000000}%
\pgfsetstrokecolor{currentstroke}%
\pgfsetdash{}{0pt}%
\pgfpathmoveto{\pgfqpoint{0.859232in}{1.476144in}}%
\pgfpathlineto{\pgfqpoint{0.883791in}{1.446920in}}%
\pgfusepath{stroke}%
\end{pgfscope}%
\begin{pgfscope}%
\pgfpathrectangle{\pgfqpoint{0.100000in}{0.220728in}}{\pgfqpoint{3.696000in}{3.696000in}}%
\pgfusepath{clip}%
\pgfsetrectcap%
\pgfsetroundjoin%
\pgfsetlinewidth{1.505625pt}%
\definecolor{currentstroke}{rgb}{1.000000,0.000000,0.000000}%
\pgfsetstrokecolor{currentstroke}%
\pgfsetdash{}{0pt}%
\pgfpathmoveto{\pgfqpoint{0.863184in}{1.480347in}}%
\pgfpathlineto{\pgfqpoint{0.883791in}{1.446920in}}%
\pgfusepath{stroke}%
\end{pgfscope}%
\begin{pgfscope}%
\pgfpathrectangle{\pgfqpoint{0.100000in}{0.220728in}}{\pgfqpoint{3.696000in}{3.696000in}}%
\pgfusepath{clip}%
\pgfsetrectcap%
\pgfsetroundjoin%
\pgfsetlinewidth{1.505625pt}%
\definecolor{currentstroke}{rgb}{1.000000,0.000000,0.000000}%
\pgfsetstrokecolor{currentstroke}%
\pgfsetdash{}{0pt}%
\pgfpathmoveto{\pgfqpoint{0.868006in}{1.481496in}}%
\pgfpathlineto{\pgfqpoint{0.883791in}{1.446920in}}%
\pgfusepath{stroke}%
\end{pgfscope}%
\begin{pgfscope}%
\pgfpathrectangle{\pgfqpoint{0.100000in}{0.220728in}}{\pgfqpoint{3.696000in}{3.696000in}}%
\pgfusepath{clip}%
\pgfsetrectcap%
\pgfsetroundjoin%
\pgfsetlinewidth{1.505625pt}%
\definecolor{currentstroke}{rgb}{1.000000,0.000000,0.000000}%
\pgfsetstrokecolor{currentstroke}%
\pgfsetdash{}{0pt}%
\pgfpathmoveto{\pgfqpoint{0.870640in}{1.483189in}}%
\pgfpathlineto{\pgfqpoint{0.893342in}{1.454997in}}%
\pgfusepath{stroke}%
\end{pgfscope}%
\begin{pgfscope}%
\pgfpathrectangle{\pgfqpoint{0.100000in}{0.220728in}}{\pgfqpoint{3.696000in}{3.696000in}}%
\pgfusepath{clip}%
\pgfsetrectcap%
\pgfsetroundjoin%
\pgfsetlinewidth{1.505625pt}%
\definecolor{currentstroke}{rgb}{1.000000,0.000000,0.000000}%
\pgfsetstrokecolor{currentstroke}%
\pgfsetdash{}{0pt}%
\pgfpathmoveto{\pgfqpoint{0.871974in}{1.484319in}}%
\pgfpathlineto{\pgfqpoint{0.893342in}{1.454997in}}%
\pgfusepath{stroke}%
\end{pgfscope}%
\begin{pgfscope}%
\pgfpathrectangle{\pgfqpoint{0.100000in}{0.220728in}}{\pgfqpoint{3.696000in}{3.696000in}}%
\pgfusepath{clip}%
\pgfsetrectcap%
\pgfsetroundjoin%
\pgfsetlinewidth{1.505625pt}%
\definecolor{currentstroke}{rgb}{1.000000,0.000000,0.000000}%
\pgfsetstrokecolor{currentstroke}%
\pgfsetdash{}{0pt}%
\pgfpathmoveto{\pgfqpoint{0.872717in}{1.484745in}}%
\pgfpathlineto{\pgfqpoint{0.893342in}{1.454997in}}%
\pgfusepath{stroke}%
\end{pgfscope}%
\begin{pgfscope}%
\pgfpathrectangle{\pgfqpoint{0.100000in}{0.220728in}}{\pgfqpoint{3.696000in}{3.696000in}}%
\pgfusepath{clip}%
\pgfsetrectcap%
\pgfsetroundjoin%
\pgfsetlinewidth{1.505625pt}%
\definecolor{currentstroke}{rgb}{1.000000,0.000000,0.000000}%
\pgfsetstrokecolor{currentstroke}%
\pgfsetdash{}{0pt}%
\pgfpathmoveto{\pgfqpoint{0.873152in}{1.485056in}}%
\pgfpathlineto{\pgfqpoint{0.893342in}{1.454997in}}%
\pgfusepath{stroke}%
\end{pgfscope}%
\begin{pgfscope}%
\pgfpathrectangle{\pgfqpoint{0.100000in}{0.220728in}}{\pgfqpoint{3.696000in}{3.696000in}}%
\pgfusepath{clip}%
\pgfsetrectcap%
\pgfsetroundjoin%
\pgfsetlinewidth{1.505625pt}%
\definecolor{currentstroke}{rgb}{1.000000,0.000000,0.000000}%
\pgfsetstrokecolor{currentstroke}%
\pgfsetdash{}{0pt}%
\pgfpathmoveto{\pgfqpoint{0.873380in}{1.485185in}}%
\pgfpathlineto{\pgfqpoint{0.893342in}{1.454997in}}%
\pgfusepath{stroke}%
\end{pgfscope}%
\begin{pgfscope}%
\pgfpathrectangle{\pgfqpoint{0.100000in}{0.220728in}}{\pgfqpoint{3.696000in}{3.696000in}}%
\pgfusepath{clip}%
\pgfsetrectcap%
\pgfsetroundjoin%
\pgfsetlinewidth{1.505625pt}%
\definecolor{currentstroke}{rgb}{1.000000,0.000000,0.000000}%
\pgfsetstrokecolor{currentstroke}%
\pgfsetdash{}{0pt}%
\pgfpathmoveto{\pgfqpoint{0.873504in}{1.485310in}}%
\pgfpathlineto{\pgfqpoint{0.893342in}{1.454997in}}%
\pgfusepath{stroke}%
\end{pgfscope}%
\begin{pgfscope}%
\pgfpathrectangle{\pgfqpoint{0.100000in}{0.220728in}}{\pgfqpoint{3.696000in}{3.696000in}}%
\pgfusepath{clip}%
\pgfsetrectcap%
\pgfsetroundjoin%
\pgfsetlinewidth{1.505625pt}%
\definecolor{currentstroke}{rgb}{1.000000,0.000000,0.000000}%
\pgfsetstrokecolor{currentstroke}%
\pgfsetdash{}{0pt}%
\pgfpathmoveto{\pgfqpoint{0.873588in}{1.485314in}}%
\pgfpathlineto{\pgfqpoint{0.893342in}{1.454997in}}%
\pgfusepath{stroke}%
\end{pgfscope}%
\begin{pgfscope}%
\pgfpathrectangle{\pgfqpoint{0.100000in}{0.220728in}}{\pgfqpoint{3.696000in}{3.696000in}}%
\pgfusepath{clip}%
\pgfsetrectcap%
\pgfsetroundjoin%
\pgfsetlinewidth{1.505625pt}%
\definecolor{currentstroke}{rgb}{1.000000,0.000000,0.000000}%
\pgfsetstrokecolor{currentstroke}%
\pgfsetdash{}{0pt}%
\pgfpathmoveto{\pgfqpoint{0.873623in}{1.485336in}}%
\pgfpathlineto{\pgfqpoint{0.893342in}{1.454997in}}%
\pgfusepath{stroke}%
\end{pgfscope}%
\begin{pgfscope}%
\pgfpathrectangle{\pgfqpoint{0.100000in}{0.220728in}}{\pgfqpoint{3.696000in}{3.696000in}}%
\pgfusepath{clip}%
\pgfsetrectcap%
\pgfsetroundjoin%
\pgfsetlinewidth{1.505625pt}%
\definecolor{currentstroke}{rgb}{1.000000,0.000000,0.000000}%
\pgfsetstrokecolor{currentstroke}%
\pgfsetdash{}{0pt}%
\pgfpathmoveto{\pgfqpoint{0.873645in}{1.485351in}}%
\pgfpathlineto{\pgfqpoint{0.893342in}{1.454997in}}%
\pgfusepath{stroke}%
\end{pgfscope}%
\begin{pgfscope}%
\pgfpathrectangle{\pgfqpoint{0.100000in}{0.220728in}}{\pgfqpoint{3.696000in}{3.696000in}}%
\pgfusepath{clip}%
\pgfsetrectcap%
\pgfsetroundjoin%
\pgfsetlinewidth{1.505625pt}%
\definecolor{currentstroke}{rgb}{1.000000,0.000000,0.000000}%
\pgfsetstrokecolor{currentstroke}%
\pgfsetdash{}{0pt}%
\pgfpathmoveto{\pgfqpoint{0.873657in}{1.485356in}}%
\pgfpathlineto{\pgfqpoint{0.893342in}{1.454997in}}%
\pgfusepath{stroke}%
\end{pgfscope}%
\begin{pgfscope}%
\pgfpathrectangle{\pgfqpoint{0.100000in}{0.220728in}}{\pgfqpoint{3.696000in}{3.696000in}}%
\pgfusepath{clip}%
\pgfsetrectcap%
\pgfsetroundjoin%
\pgfsetlinewidth{1.505625pt}%
\definecolor{currentstroke}{rgb}{1.000000,0.000000,0.000000}%
\pgfsetstrokecolor{currentstroke}%
\pgfsetdash{}{0pt}%
\pgfpathmoveto{\pgfqpoint{0.873663in}{1.485362in}}%
\pgfpathlineto{\pgfqpoint{0.893342in}{1.454997in}}%
\pgfusepath{stroke}%
\end{pgfscope}%
\begin{pgfscope}%
\pgfpathrectangle{\pgfqpoint{0.100000in}{0.220728in}}{\pgfqpoint{3.696000in}{3.696000in}}%
\pgfusepath{clip}%
\pgfsetrectcap%
\pgfsetroundjoin%
\pgfsetlinewidth{1.505625pt}%
\definecolor{currentstroke}{rgb}{1.000000,0.000000,0.000000}%
\pgfsetstrokecolor{currentstroke}%
\pgfsetdash{}{0pt}%
\pgfpathmoveto{\pgfqpoint{0.873667in}{1.485364in}}%
\pgfpathlineto{\pgfqpoint{0.893342in}{1.454997in}}%
\pgfusepath{stroke}%
\end{pgfscope}%
\begin{pgfscope}%
\pgfpathrectangle{\pgfqpoint{0.100000in}{0.220728in}}{\pgfqpoint{3.696000in}{3.696000in}}%
\pgfusepath{clip}%
\pgfsetrectcap%
\pgfsetroundjoin%
\pgfsetlinewidth{1.505625pt}%
\definecolor{currentstroke}{rgb}{1.000000,0.000000,0.000000}%
\pgfsetstrokecolor{currentstroke}%
\pgfsetdash{}{0pt}%
\pgfpathmoveto{\pgfqpoint{0.873669in}{1.485365in}}%
\pgfpathlineto{\pgfqpoint{0.893342in}{1.454997in}}%
\pgfusepath{stroke}%
\end{pgfscope}%
\begin{pgfscope}%
\pgfpathrectangle{\pgfqpoint{0.100000in}{0.220728in}}{\pgfqpoint{3.696000in}{3.696000in}}%
\pgfusepath{clip}%
\pgfsetrectcap%
\pgfsetroundjoin%
\pgfsetlinewidth{1.505625pt}%
\definecolor{currentstroke}{rgb}{1.000000,0.000000,0.000000}%
\pgfsetstrokecolor{currentstroke}%
\pgfsetdash{}{0pt}%
\pgfpathmoveto{\pgfqpoint{0.873670in}{1.485366in}}%
\pgfpathlineto{\pgfqpoint{0.893342in}{1.454997in}}%
\pgfusepath{stroke}%
\end{pgfscope}%
\begin{pgfscope}%
\pgfpathrectangle{\pgfqpoint{0.100000in}{0.220728in}}{\pgfqpoint{3.696000in}{3.696000in}}%
\pgfusepath{clip}%
\pgfsetrectcap%
\pgfsetroundjoin%
\pgfsetlinewidth{1.505625pt}%
\definecolor{currentstroke}{rgb}{1.000000,0.000000,0.000000}%
\pgfsetstrokecolor{currentstroke}%
\pgfsetdash{}{0pt}%
\pgfpathmoveto{\pgfqpoint{0.873671in}{1.485366in}}%
\pgfpathlineto{\pgfqpoint{0.893342in}{1.454997in}}%
\pgfusepath{stroke}%
\end{pgfscope}%
\begin{pgfscope}%
\pgfpathrectangle{\pgfqpoint{0.100000in}{0.220728in}}{\pgfqpoint{3.696000in}{3.696000in}}%
\pgfusepath{clip}%
\pgfsetrectcap%
\pgfsetroundjoin%
\pgfsetlinewidth{1.505625pt}%
\definecolor{currentstroke}{rgb}{1.000000,0.000000,0.000000}%
\pgfsetstrokecolor{currentstroke}%
\pgfsetdash{}{0pt}%
\pgfpathmoveto{\pgfqpoint{0.873671in}{1.485366in}}%
\pgfpathlineto{\pgfqpoint{0.893342in}{1.454997in}}%
\pgfusepath{stroke}%
\end{pgfscope}%
\begin{pgfscope}%
\pgfpathrectangle{\pgfqpoint{0.100000in}{0.220728in}}{\pgfqpoint{3.696000in}{3.696000in}}%
\pgfusepath{clip}%
\pgfsetrectcap%
\pgfsetroundjoin%
\pgfsetlinewidth{1.505625pt}%
\definecolor{currentstroke}{rgb}{1.000000,0.000000,0.000000}%
\pgfsetstrokecolor{currentstroke}%
\pgfsetdash{}{0pt}%
\pgfpathmoveto{\pgfqpoint{0.873671in}{1.485366in}}%
\pgfpathlineto{\pgfqpoint{0.893342in}{1.454997in}}%
\pgfusepath{stroke}%
\end{pgfscope}%
\begin{pgfscope}%
\pgfpathrectangle{\pgfqpoint{0.100000in}{0.220728in}}{\pgfqpoint{3.696000in}{3.696000in}}%
\pgfusepath{clip}%
\pgfsetrectcap%
\pgfsetroundjoin%
\pgfsetlinewidth{1.505625pt}%
\definecolor{currentstroke}{rgb}{1.000000,0.000000,0.000000}%
\pgfsetstrokecolor{currentstroke}%
\pgfsetdash{}{0pt}%
\pgfpathmoveto{\pgfqpoint{0.873671in}{1.485366in}}%
\pgfpathlineto{\pgfqpoint{0.893342in}{1.454997in}}%
\pgfusepath{stroke}%
\end{pgfscope}%
\begin{pgfscope}%
\pgfpathrectangle{\pgfqpoint{0.100000in}{0.220728in}}{\pgfqpoint{3.696000in}{3.696000in}}%
\pgfusepath{clip}%
\pgfsetrectcap%
\pgfsetroundjoin%
\pgfsetlinewidth{1.505625pt}%
\definecolor{currentstroke}{rgb}{1.000000,0.000000,0.000000}%
\pgfsetstrokecolor{currentstroke}%
\pgfsetdash{}{0pt}%
\pgfpathmoveto{\pgfqpoint{0.873671in}{1.485366in}}%
\pgfpathlineto{\pgfqpoint{0.893342in}{1.454997in}}%
\pgfusepath{stroke}%
\end{pgfscope}%
\begin{pgfscope}%
\pgfpathrectangle{\pgfqpoint{0.100000in}{0.220728in}}{\pgfqpoint{3.696000in}{3.696000in}}%
\pgfusepath{clip}%
\pgfsetrectcap%
\pgfsetroundjoin%
\pgfsetlinewidth{1.505625pt}%
\definecolor{currentstroke}{rgb}{1.000000,0.000000,0.000000}%
\pgfsetstrokecolor{currentstroke}%
\pgfsetdash{}{0pt}%
\pgfpathmoveto{\pgfqpoint{0.874753in}{1.484828in}}%
\pgfpathlineto{\pgfqpoint{0.893342in}{1.454997in}}%
\pgfusepath{stroke}%
\end{pgfscope}%
\begin{pgfscope}%
\pgfpathrectangle{\pgfqpoint{0.100000in}{0.220728in}}{\pgfqpoint{3.696000in}{3.696000in}}%
\pgfusepath{clip}%
\pgfsetrectcap%
\pgfsetroundjoin%
\pgfsetlinewidth{1.505625pt}%
\definecolor{currentstroke}{rgb}{1.000000,0.000000,0.000000}%
\pgfsetstrokecolor{currentstroke}%
\pgfsetdash{}{0pt}%
\pgfpathmoveto{\pgfqpoint{0.878188in}{1.484894in}}%
\pgfpathlineto{\pgfqpoint{0.893342in}{1.454997in}}%
\pgfusepath{stroke}%
\end{pgfscope}%
\begin{pgfscope}%
\pgfpathrectangle{\pgfqpoint{0.100000in}{0.220728in}}{\pgfqpoint{3.696000in}{3.696000in}}%
\pgfusepath{clip}%
\pgfsetrectcap%
\pgfsetroundjoin%
\pgfsetlinewidth{1.505625pt}%
\definecolor{currentstroke}{rgb}{1.000000,0.000000,0.000000}%
\pgfsetstrokecolor{currentstroke}%
\pgfsetdash{}{0pt}%
\pgfpathmoveto{\pgfqpoint{0.881627in}{1.485870in}}%
\pgfpathlineto{\pgfqpoint{0.902881in}{1.463065in}}%
\pgfusepath{stroke}%
\end{pgfscope}%
\begin{pgfscope}%
\pgfpathrectangle{\pgfqpoint{0.100000in}{0.220728in}}{\pgfqpoint{3.696000in}{3.696000in}}%
\pgfusepath{clip}%
\pgfsetrectcap%
\pgfsetroundjoin%
\pgfsetlinewidth{1.505625pt}%
\definecolor{currentstroke}{rgb}{1.000000,0.000000,0.000000}%
\pgfsetstrokecolor{currentstroke}%
\pgfsetdash{}{0pt}%
\pgfpathmoveto{\pgfqpoint{0.890019in}{1.486383in}}%
\pgfpathlineto{\pgfqpoint{0.902881in}{1.463065in}}%
\pgfusepath{stroke}%
\end{pgfscope}%
\begin{pgfscope}%
\pgfpathrectangle{\pgfqpoint{0.100000in}{0.220728in}}{\pgfqpoint{3.696000in}{3.696000in}}%
\pgfusepath{clip}%
\pgfsetrectcap%
\pgfsetroundjoin%
\pgfsetlinewidth{1.505625pt}%
\definecolor{currentstroke}{rgb}{1.000000,0.000000,0.000000}%
\pgfsetstrokecolor{currentstroke}%
\pgfsetdash{}{0pt}%
\pgfpathmoveto{\pgfqpoint{0.896622in}{1.493697in}}%
\pgfpathlineto{\pgfqpoint{0.912407in}{1.471121in}}%
\pgfusepath{stroke}%
\end{pgfscope}%
\begin{pgfscope}%
\pgfpathrectangle{\pgfqpoint{0.100000in}{0.220728in}}{\pgfqpoint{3.696000in}{3.696000in}}%
\pgfusepath{clip}%
\pgfsetrectcap%
\pgfsetroundjoin%
\pgfsetlinewidth{1.505625pt}%
\definecolor{currentstroke}{rgb}{1.000000,0.000000,0.000000}%
\pgfsetstrokecolor{currentstroke}%
\pgfsetdash{}{0pt}%
\pgfpathmoveto{\pgfqpoint{0.900798in}{1.496437in}}%
\pgfpathlineto{\pgfqpoint{0.921921in}{1.479167in}}%
\pgfusepath{stroke}%
\end{pgfscope}%
\begin{pgfscope}%
\pgfpathrectangle{\pgfqpoint{0.100000in}{0.220728in}}{\pgfqpoint{3.696000in}{3.696000in}}%
\pgfusepath{clip}%
\pgfsetrectcap%
\pgfsetroundjoin%
\pgfsetlinewidth{1.505625pt}%
\definecolor{currentstroke}{rgb}{1.000000,0.000000,0.000000}%
\pgfsetstrokecolor{currentstroke}%
\pgfsetdash{}{0pt}%
\pgfpathmoveto{\pgfqpoint{0.903552in}{1.498923in}}%
\pgfpathlineto{\pgfqpoint{0.921921in}{1.479167in}}%
\pgfusepath{stroke}%
\end{pgfscope}%
\begin{pgfscope}%
\pgfpathrectangle{\pgfqpoint{0.100000in}{0.220728in}}{\pgfqpoint{3.696000in}{3.696000in}}%
\pgfusepath{clip}%
\pgfsetrectcap%
\pgfsetroundjoin%
\pgfsetlinewidth{1.505625pt}%
\definecolor{currentstroke}{rgb}{1.000000,0.000000,0.000000}%
\pgfsetstrokecolor{currentstroke}%
\pgfsetdash{}{0pt}%
\pgfpathmoveto{\pgfqpoint{0.906328in}{1.498073in}}%
\pgfpathlineto{\pgfqpoint{0.921921in}{1.479167in}}%
\pgfusepath{stroke}%
\end{pgfscope}%
\begin{pgfscope}%
\pgfpathrectangle{\pgfqpoint{0.100000in}{0.220728in}}{\pgfqpoint{3.696000in}{3.696000in}}%
\pgfusepath{clip}%
\pgfsetrectcap%
\pgfsetroundjoin%
\pgfsetlinewidth{1.505625pt}%
\definecolor{currentstroke}{rgb}{1.000000,0.000000,0.000000}%
\pgfsetstrokecolor{currentstroke}%
\pgfsetdash{}{0pt}%
\pgfpathmoveto{\pgfqpoint{0.912270in}{1.499533in}}%
\pgfpathlineto{\pgfqpoint{0.931422in}{1.487202in}}%
\pgfusepath{stroke}%
\end{pgfscope}%
\begin{pgfscope}%
\pgfpathrectangle{\pgfqpoint{0.100000in}{0.220728in}}{\pgfqpoint{3.696000in}{3.696000in}}%
\pgfusepath{clip}%
\pgfsetrectcap%
\pgfsetroundjoin%
\pgfsetlinewidth{1.505625pt}%
\definecolor{currentstroke}{rgb}{1.000000,0.000000,0.000000}%
\pgfsetstrokecolor{currentstroke}%
\pgfsetdash{}{0pt}%
\pgfpathmoveto{\pgfqpoint{0.917662in}{1.501957in}}%
\pgfpathlineto{\pgfqpoint{0.940911in}{1.495227in}}%
\pgfusepath{stroke}%
\end{pgfscope}%
\begin{pgfscope}%
\pgfpathrectangle{\pgfqpoint{0.100000in}{0.220728in}}{\pgfqpoint{3.696000in}{3.696000in}}%
\pgfusepath{clip}%
\pgfsetrectcap%
\pgfsetroundjoin%
\pgfsetlinewidth{1.505625pt}%
\definecolor{currentstroke}{rgb}{1.000000,0.000000,0.000000}%
\pgfsetstrokecolor{currentstroke}%
\pgfsetdash{}{0pt}%
\pgfpathmoveto{\pgfqpoint{0.927578in}{1.503660in}}%
\pgfpathlineto{\pgfqpoint{0.940911in}{1.495227in}}%
\pgfusepath{stroke}%
\end{pgfscope}%
\begin{pgfscope}%
\pgfpathrectangle{\pgfqpoint{0.100000in}{0.220728in}}{\pgfqpoint{3.696000in}{3.696000in}}%
\pgfusepath{clip}%
\pgfsetrectcap%
\pgfsetroundjoin%
\pgfsetlinewidth{1.505625pt}%
\definecolor{currentstroke}{rgb}{1.000000,0.000000,0.000000}%
\pgfsetstrokecolor{currentstroke}%
\pgfsetdash{}{0pt}%
\pgfpathmoveto{\pgfqpoint{0.935004in}{1.513865in}}%
\pgfpathlineto{\pgfqpoint{0.950387in}{1.503241in}}%
\pgfusepath{stroke}%
\end{pgfscope}%
\begin{pgfscope}%
\pgfpathrectangle{\pgfqpoint{0.100000in}{0.220728in}}{\pgfqpoint{3.696000in}{3.696000in}}%
\pgfusepath{clip}%
\pgfsetrectcap%
\pgfsetroundjoin%
\pgfsetlinewidth{1.505625pt}%
\definecolor{currentstroke}{rgb}{1.000000,0.000000,0.000000}%
\pgfsetstrokecolor{currentstroke}%
\pgfsetdash{}{0pt}%
\pgfpathmoveto{\pgfqpoint{0.939828in}{1.516533in}}%
\pgfpathlineto{\pgfqpoint{0.959850in}{1.511245in}}%
\pgfusepath{stroke}%
\end{pgfscope}%
\begin{pgfscope}%
\pgfpathrectangle{\pgfqpoint{0.100000in}{0.220728in}}{\pgfqpoint{3.696000in}{3.696000in}}%
\pgfusepath{clip}%
\pgfsetrectcap%
\pgfsetroundjoin%
\pgfsetlinewidth{1.505625pt}%
\definecolor{currentstroke}{rgb}{1.000000,0.000000,0.000000}%
\pgfsetstrokecolor{currentstroke}%
\pgfsetdash{}{0pt}%
\pgfpathmoveto{\pgfqpoint{0.942860in}{1.519631in}}%
\pgfpathlineto{\pgfqpoint{0.959850in}{1.511245in}}%
\pgfusepath{stroke}%
\end{pgfscope}%
\begin{pgfscope}%
\pgfpathrectangle{\pgfqpoint{0.100000in}{0.220728in}}{\pgfqpoint{3.696000in}{3.696000in}}%
\pgfusepath{clip}%
\pgfsetrectcap%
\pgfsetroundjoin%
\pgfsetlinewidth{1.505625pt}%
\definecolor{currentstroke}{rgb}{1.000000,0.000000,0.000000}%
\pgfsetstrokecolor{currentstroke}%
\pgfsetdash{}{0pt}%
\pgfpathmoveto{\pgfqpoint{0.945090in}{1.519403in}}%
\pgfpathlineto{\pgfqpoint{0.959850in}{1.511245in}}%
\pgfusepath{stroke}%
\end{pgfscope}%
\begin{pgfscope}%
\pgfpathrectangle{\pgfqpoint{0.100000in}{0.220728in}}{\pgfqpoint{3.696000in}{3.696000in}}%
\pgfusepath{clip}%
\pgfsetrectcap%
\pgfsetroundjoin%
\pgfsetlinewidth{1.505625pt}%
\definecolor{currentstroke}{rgb}{1.000000,0.000000,0.000000}%
\pgfsetstrokecolor{currentstroke}%
\pgfsetdash{}{0pt}%
\pgfpathmoveto{\pgfqpoint{0.951508in}{1.520285in}}%
\pgfpathlineto{\pgfqpoint{0.969302in}{1.519238in}}%
\pgfusepath{stroke}%
\end{pgfscope}%
\begin{pgfscope}%
\pgfpathrectangle{\pgfqpoint{0.100000in}{0.220728in}}{\pgfqpoint{3.696000in}{3.696000in}}%
\pgfusepath{clip}%
\pgfsetrectcap%
\pgfsetroundjoin%
\pgfsetlinewidth{1.505625pt}%
\definecolor{currentstroke}{rgb}{1.000000,0.000000,0.000000}%
\pgfsetstrokecolor{currentstroke}%
\pgfsetdash{}{0pt}%
\pgfpathmoveto{\pgfqpoint{0.956692in}{1.523434in}}%
\pgfpathlineto{\pgfqpoint{0.978740in}{1.527220in}}%
\pgfusepath{stroke}%
\end{pgfscope}%
\begin{pgfscope}%
\pgfpathrectangle{\pgfqpoint{0.100000in}{0.220728in}}{\pgfqpoint{3.696000in}{3.696000in}}%
\pgfusepath{clip}%
\pgfsetrectcap%
\pgfsetroundjoin%
\pgfsetlinewidth{1.505625pt}%
\definecolor{currentstroke}{rgb}{1.000000,0.000000,0.000000}%
\pgfsetstrokecolor{currentstroke}%
\pgfsetdash{}{0pt}%
\pgfpathmoveto{\pgfqpoint{0.965392in}{1.529887in}}%
\pgfpathlineto{\pgfqpoint{0.978740in}{1.527220in}}%
\pgfusepath{stroke}%
\end{pgfscope}%
\begin{pgfscope}%
\pgfpathrectangle{\pgfqpoint{0.100000in}{0.220728in}}{\pgfqpoint{3.696000in}{3.696000in}}%
\pgfusepath{clip}%
\pgfsetrectcap%
\pgfsetroundjoin%
\pgfsetlinewidth{1.505625pt}%
\definecolor{currentstroke}{rgb}{1.000000,0.000000,0.000000}%
\pgfsetstrokecolor{currentstroke}%
\pgfsetdash{}{0pt}%
\pgfpathmoveto{\pgfqpoint{0.974112in}{1.542491in}}%
\pgfpathlineto{\pgfqpoint{0.988167in}{1.535192in}}%
\pgfusepath{stroke}%
\end{pgfscope}%
\begin{pgfscope}%
\pgfpathrectangle{\pgfqpoint{0.100000in}{0.220728in}}{\pgfqpoint{3.696000in}{3.696000in}}%
\pgfusepath{clip}%
\pgfsetrectcap%
\pgfsetroundjoin%
\pgfsetlinewidth{1.505625pt}%
\definecolor{currentstroke}{rgb}{1.000000,0.000000,0.000000}%
\pgfsetstrokecolor{currentstroke}%
\pgfsetdash{}{0pt}%
\pgfpathmoveto{\pgfqpoint{0.982120in}{1.545494in}}%
\pgfpathlineto{\pgfqpoint{0.997580in}{1.543154in}}%
\pgfusepath{stroke}%
\end{pgfscope}%
\begin{pgfscope}%
\pgfpathrectangle{\pgfqpoint{0.100000in}{0.220728in}}{\pgfqpoint{3.696000in}{3.696000in}}%
\pgfusepath{clip}%
\pgfsetrectcap%
\pgfsetroundjoin%
\pgfsetlinewidth{1.505625pt}%
\definecolor{currentstroke}{rgb}{1.000000,0.000000,0.000000}%
\pgfsetstrokecolor{currentstroke}%
\pgfsetdash{}{0pt}%
\pgfpathmoveto{\pgfqpoint{0.995565in}{1.552852in}}%
\pgfpathlineto{\pgfqpoint{1.006982in}{1.551105in}}%
\pgfusepath{stroke}%
\end{pgfscope}%
\begin{pgfscope}%
\pgfpathrectangle{\pgfqpoint{0.100000in}{0.220728in}}{\pgfqpoint{3.696000in}{3.696000in}}%
\pgfusepath{clip}%
\pgfsetrectcap%
\pgfsetroundjoin%
\pgfsetlinewidth{1.505625pt}%
\definecolor{currentstroke}{rgb}{1.000000,0.000000,0.000000}%
\pgfsetstrokecolor{currentstroke}%
\pgfsetdash{}{0pt}%
\pgfpathmoveto{\pgfqpoint{1.007838in}{1.560439in}}%
\pgfpathlineto{\pgfqpoint{1.016371in}{1.559045in}}%
\pgfusepath{stroke}%
\end{pgfscope}%
\begin{pgfscope}%
\pgfpathrectangle{\pgfqpoint{0.100000in}{0.220728in}}{\pgfqpoint{3.696000in}{3.696000in}}%
\pgfusepath{clip}%
\pgfsetrectcap%
\pgfsetroundjoin%
\pgfsetlinewidth{1.505625pt}%
\definecolor{currentstroke}{rgb}{1.000000,0.000000,0.000000}%
\pgfsetstrokecolor{currentstroke}%
\pgfsetdash{}{0pt}%
\pgfpathmoveto{\pgfqpoint{1.014355in}{1.564963in}}%
\pgfpathlineto{\pgfqpoint{1.025748in}{1.566975in}}%
\pgfusepath{stroke}%
\end{pgfscope}%
\begin{pgfscope}%
\pgfpathrectangle{\pgfqpoint{0.100000in}{0.220728in}}{\pgfqpoint{3.696000in}{3.696000in}}%
\pgfusepath{clip}%
\pgfsetrectcap%
\pgfsetroundjoin%
\pgfsetlinewidth{1.505625pt}%
\definecolor{currentstroke}{rgb}{1.000000,0.000000,0.000000}%
\pgfsetstrokecolor{currentstroke}%
\pgfsetdash{}{0pt}%
\pgfpathmoveto{\pgfqpoint{1.022715in}{1.572116in}}%
\pgfpathlineto{\pgfqpoint{1.035113in}{1.574895in}}%
\pgfusepath{stroke}%
\end{pgfscope}%
\begin{pgfscope}%
\pgfpathrectangle{\pgfqpoint{0.100000in}{0.220728in}}{\pgfqpoint{3.696000in}{3.696000in}}%
\pgfusepath{clip}%
\pgfsetrectcap%
\pgfsetroundjoin%
\pgfsetlinewidth{1.505625pt}%
\definecolor{currentstroke}{rgb}{1.000000,0.000000,0.000000}%
\pgfsetstrokecolor{currentstroke}%
\pgfsetdash{}{0pt}%
\pgfpathmoveto{\pgfqpoint{1.030863in}{1.576665in}}%
\pgfpathlineto{\pgfqpoint{1.044465in}{1.582805in}}%
\pgfusepath{stroke}%
\end{pgfscope}%
\begin{pgfscope}%
\pgfpathrectangle{\pgfqpoint{0.100000in}{0.220728in}}{\pgfqpoint{3.696000in}{3.696000in}}%
\pgfusepath{clip}%
\pgfsetrectcap%
\pgfsetroundjoin%
\pgfsetlinewidth{1.505625pt}%
\definecolor{currentstroke}{rgb}{1.000000,0.000000,0.000000}%
\pgfsetstrokecolor{currentstroke}%
\pgfsetdash{}{0pt}%
\pgfpathmoveto{\pgfqpoint{1.041499in}{1.585767in}}%
\pgfpathlineto{\pgfqpoint{1.053805in}{1.590704in}}%
\pgfusepath{stroke}%
\end{pgfscope}%
\begin{pgfscope}%
\pgfpathrectangle{\pgfqpoint{0.100000in}{0.220728in}}{\pgfqpoint{3.696000in}{3.696000in}}%
\pgfusepath{clip}%
\pgfsetrectcap%
\pgfsetroundjoin%
\pgfsetlinewidth{1.505625pt}%
\definecolor{currentstroke}{rgb}{1.000000,0.000000,0.000000}%
\pgfsetstrokecolor{currentstroke}%
\pgfsetdash{}{0pt}%
\pgfpathmoveto{\pgfqpoint{1.052356in}{1.599493in}}%
\pgfpathlineto{\pgfqpoint{1.063133in}{1.598592in}}%
\pgfusepath{stroke}%
\end{pgfscope}%
\begin{pgfscope}%
\pgfpathrectangle{\pgfqpoint{0.100000in}{0.220728in}}{\pgfqpoint{3.696000in}{3.696000in}}%
\pgfusepath{clip}%
\pgfsetrectcap%
\pgfsetroundjoin%
\pgfsetlinewidth{1.505625pt}%
\definecolor{currentstroke}{rgb}{1.000000,0.000000,0.000000}%
\pgfsetstrokecolor{currentstroke}%
\pgfsetdash{}{0pt}%
\pgfpathmoveto{\pgfqpoint{1.062834in}{1.605923in}}%
\pgfpathlineto{\pgfqpoint{1.072449in}{1.606471in}}%
\pgfusepath{stroke}%
\end{pgfscope}%
\begin{pgfscope}%
\pgfpathrectangle{\pgfqpoint{0.100000in}{0.220728in}}{\pgfqpoint{3.696000in}{3.696000in}}%
\pgfusepath{clip}%
\pgfsetrectcap%
\pgfsetroundjoin%
\pgfsetlinewidth{1.505625pt}%
\definecolor{currentstroke}{rgb}{1.000000,0.000000,0.000000}%
\pgfsetstrokecolor{currentstroke}%
\pgfsetdash{}{0pt}%
\pgfpathmoveto{\pgfqpoint{1.070147in}{1.614119in}}%
\pgfpathlineto{\pgfqpoint{1.081752in}{1.614339in}}%
\pgfusepath{stroke}%
\end{pgfscope}%
\begin{pgfscope}%
\pgfpathrectangle{\pgfqpoint{0.100000in}{0.220728in}}{\pgfqpoint{3.696000in}{3.696000in}}%
\pgfusepath{clip}%
\pgfsetrectcap%
\pgfsetroundjoin%
\pgfsetlinewidth{1.505625pt}%
\definecolor{currentstroke}{rgb}{1.000000,0.000000,0.000000}%
\pgfsetstrokecolor{currentstroke}%
\pgfsetdash{}{0pt}%
\pgfpathmoveto{\pgfqpoint{1.074572in}{1.618538in}}%
\pgfpathlineto{\pgfqpoint{1.081752in}{1.614339in}}%
\pgfusepath{stroke}%
\end{pgfscope}%
\begin{pgfscope}%
\pgfpathrectangle{\pgfqpoint{0.100000in}{0.220728in}}{\pgfqpoint{3.696000in}{3.696000in}}%
\pgfusepath{clip}%
\pgfsetrectcap%
\pgfsetroundjoin%
\pgfsetlinewidth{1.505625pt}%
\definecolor{currentstroke}{rgb}{1.000000,0.000000,0.000000}%
\pgfsetstrokecolor{currentstroke}%
\pgfsetdash{}{0pt}%
\pgfpathmoveto{\pgfqpoint{1.079664in}{1.620726in}}%
\pgfpathlineto{\pgfqpoint{1.091044in}{1.622197in}}%
\pgfusepath{stroke}%
\end{pgfscope}%
\begin{pgfscope}%
\pgfpathrectangle{\pgfqpoint{0.100000in}{0.220728in}}{\pgfqpoint{3.696000in}{3.696000in}}%
\pgfusepath{clip}%
\pgfsetrectcap%
\pgfsetroundjoin%
\pgfsetlinewidth{1.505625pt}%
\definecolor{currentstroke}{rgb}{1.000000,0.000000,0.000000}%
\pgfsetstrokecolor{currentstroke}%
\pgfsetdash{}{0pt}%
\pgfpathmoveto{\pgfqpoint{1.087622in}{1.633569in}}%
\pgfpathlineto{\pgfqpoint{1.091044in}{1.622197in}}%
\pgfusepath{stroke}%
\end{pgfscope}%
\begin{pgfscope}%
\pgfpathrectangle{\pgfqpoint{0.100000in}{0.220728in}}{\pgfqpoint{3.696000in}{3.696000in}}%
\pgfusepath{clip}%
\pgfsetrectcap%
\pgfsetroundjoin%
\pgfsetlinewidth{1.505625pt}%
\definecolor{currentstroke}{rgb}{1.000000,0.000000,0.000000}%
\pgfsetstrokecolor{currentstroke}%
\pgfsetdash{}{0pt}%
\pgfpathmoveto{\pgfqpoint{1.096668in}{1.640021in}}%
\pgfpathlineto{\pgfqpoint{1.100323in}{1.630045in}}%
\pgfusepath{stroke}%
\end{pgfscope}%
\begin{pgfscope}%
\pgfpathrectangle{\pgfqpoint{0.100000in}{0.220728in}}{\pgfqpoint{3.696000in}{3.696000in}}%
\pgfusepath{clip}%
\pgfsetrectcap%
\pgfsetroundjoin%
\pgfsetlinewidth{1.505625pt}%
\definecolor{currentstroke}{rgb}{1.000000,0.000000,0.000000}%
\pgfsetstrokecolor{currentstroke}%
\pgfsetdash{}{0pt}%
\pgfpathmoveto{\pgfqpoint{1.104023in}{1.640106in}}%
\pgfpathlineto{\pgfqpoint{1.109590in}{1.637882in}}%
\pgfusepath{stroke}%
\end{pgfscope}%
\begin{pgfscope}%
\pgfpathrectangle{\pgfqpoint{0.100000in}{0.220728in}}{\pgfqpoint{3.696000in}{3.696000in}}%
\pgfusepath{clip}%
\pgfsetrectcap%
\pgfsetroundjoin%
\pgfsetlinewidth{1.505625pt}%
\definecolor{currentstroke}{rgb}{1.000000,0.000000,0.000000}%
\pgfsetstrokecolor{currentstroke}%
\pgfsetdash{}{0pt}%
\pgfpathmoveto{\pgfqpoint{1.113944in}{1.649225in}}%
\pgfpathlineto{\pgfqpoint{1.118845in}{1.645709in}}%
\pgfusepath{stroke}%
\end{pgfscope}%
\begin{pgfscope}%
\pgfpathrectangle{\pgfqpoint{0.100000in}{0.220728in}}{\pgfqpoint{3.696000in}{3.696000in}}%
\pgfusepath{clip}%
\pgfsetrectcap%
\pgfsetroundjoin%
\pgfsetlinewidth{1.505625pt}%
\definecolor{currentstroke}{rgb}{1.000000,0.000000,0.000000}%
\pgfsetstrokecolor{currentstroke}%
\pgfsetdash{}{0pt}%
\pgfpathmoveto{\pgfqpoint{1.120203in}{1.654118in}}%
\pgfpathlineto{\pgfqpoint{1.118845in}{1.645709in}}%
\pgfusepath{stroke}%
\end{pgfscope}%
\begin{pgfscope}%
\pgfpathrectangle{\pgfqpoint{0.100000in}{0.220728in}}{\pgfqpoint{3.696000in}{3.696000in}}%
\pgfusepath{clip}%
\pgfsetrectcap%
\pgfsetroundjoin%
\pgfsetlinewidth{1.505625pt}%
\definecolor{currentstroke}{rgb}{1.000000,0.000000,0.000000}%
\pgfsetstrokecolor{currentstroke}%
\pgfsetdash{}{0pt}%
\pgfpathmoveto{\pgfqpoint{1.124330in}{1.651971in}}%
\pgfpathlineto{\pgfqpoint{1.128088in}{1.653526in}}%
\pgfusepath{stroke}%
\end{pgfscope}%
\begin{pgfscope}%
\pgfpathrectangle{\pgfqpoint{0.100000in}{0.220728in}}{\pgfqpoint{3.696000in}{3.696000in}}%
\pgfusepath{clip}%
\pgfsetrectcap%
\pgfsetroundjoin%
\pgfsetlinewidth{1.505625pt}%
\definecolor{currentstroke}{rgb}{1.000000,0.000000,0.000000}%
\pgfsetstrokecolor{currentstroke}%
\pgfsetdash{}{0pt}%
\pgfpathmoveto{\pgfqpoint{1.132686in}{1.656360in}}%
\pgfpathlineto{\pgfqpoint{1.128088in}{1.653526in}}%
\pgfusepath{stroke}%
\end{pgfscope}%
\begin{pgfscope}%
\pgfpathrectangle{\pgfqpoint{0.100000in}{0.220728in}}{\pgfqpoint{3.696000in}{3.696000in}}%
\pgfusepath{clip}%
\pgfsetrectcap%
\pgfsetroundjoin%
\pgfsetlinewidth{1.505625pt}%
\definecolor{currentstroke}{rgb}{1.000000,0.000000,0.000000}%
\pgfsetstrokecolor{currentstroke}%
\pgfsetdash{}{0pt}%
\pgfpathmoveto{\pgfqpoint{1.140665in}{1.663181in}}%
\pgfpathlineto{\pgfqpoint{1.137319in}{1.661333in}}%
\pgfusepath{stroke}%
\end{pgfscope}%
\begin{pgfscope}%
\pgfpathrectangle{\pgfqpoint{0.100000in}{0.220728in}}{\pgfqpoint{3.696000in}{3.696000in}}%
\pgfusepath{clip}%
\pgfsetrectcap%
\pgfsetroundjoin%
\pgfsetlinewidth{1.505625pt}%
\definecolor{currentstroke}{rgb}{1.000000,0.000000,0.000000}%
\pgfsetstrokecolor{currentstroke}%
\pgfsetdash{}{0pt}%
\pgfpathmoveto{\pgfqpoint{1.145164in}{1.667382in}}%
\pgfpathlineto{\pgfqpoint{1.146538in}{1.669130in}}%
\pgfusepath{stroke}%
\end{pgfscope}%
\begin{pgfscope}%
\pgfpathrectangle{\pgfqpoint{0.100000in}{0.220728in}}{\pgfqpoint{3.696000in}{3.696000in}}%
\pgfusepath{clip}%
\pgfsetrectcap%
\pgfsetroundjoin%
\pgfsetlinewidth{1.505625pt}%
\definecolor{currentstroke}{rgb}{1.000000,0.000000,0.000000}%
\pgfsetstrokecolor{currentstroke}%
\pgfsetdash{}{0pt}%
\pgfpathmoveto{\pgfqpoint{1.154185in}{1.670316in}}%
\pgfpathlineto{\pgfqpoint{1.146538in}{1.669130in}}%
\pgfusepath{stroke}%
\end{pgfscope}%
\begin{pgfscope}%
\pgfpathrectangle{\pgfqpoint{0.100000in}{0.220728in}}{\pgfqpoint{3.696000in}{3.696000in}}%
\pgfusepath{clip}%
\pgfsetrectcap%
\pgfsetroundjoin%
\pgfsetlinewidth{1.505625pt}%
\definecolor{currentstroke}{rgb}{1.000000,0.000000,0.000000}%
\pgfsetstrokecolor{currentstroke}%
\pgfsetdash{}{0pt}%
\pgfpathmoveto{\pgfqpoint{1.163162in}{1.672284in}}%
\pgfpathlineto{\pgfqpoint{1.155745in}{1.676916in}}%
\pgfusepath{stroke}%
\end{pgfscope}%
\begin{pgfscope}%
\pgfpathrectangle{\pgfqpoint{0.100000in}{0.220728in}}{\pgfqpoint{3.696000in}{3.696000in}}%
\pgfusepath{clip}%
\pgfsetrectcap%
\pgfsetroundjoin%
\pgfsetlinewidth{1.505625pt}%
\definecolor{currentstroke}{rgb}{1.000000,0.000000,0.000000}%
\pgfsetstrokecolor{currentstroke}%
\pgfsetdash{}{0pt}%
\pgfpathmoveto{\pgfqpoint{1.175789in}{1.690456in}}%
\pgfpathlineto{\pgfqpoint{1.164941in}{1.684693in}}%
\pgfusepath{stroke}%
\end{pgfscope}%
\begin{pgfscope}%
\pgfpathrectangle{\pgfqpoint{0.100000in}{0.220728in}}{\pgfqpoint{3.696000in}{3.696000in}}%
\pgfusepath{clip}%
\pgfsetrectcap%
\pgfsetroundjoin%
\pgfsetlinewidth{1.505625pt}%
\definecolor{currentstroke}{rgb}{1.000000,0.000000,0.000000}%
\pgfsetstrokecolor{currentstroke}%
\pgfsetdash{}{0pt}%
\pgfpathmoveto{\pgfqpoint{1.183241in}{1.694772in}}%
\pgfpathlineto{\pgfqpoint{1.174124in}{1.692459in}}%
\pgfusepath{stroke}%
\end{pgfscope}%
\begin{pgfscope}%
\pgfpathrectangle{\pgfqpoint{0.100000in}{0.220728in}}{\pgfqpoint{3.696000in}{3.696000in}}%
\pgfusepath{clip}%
\pgfsetrectcap%
\pgfsetroundjoin%
\pgfsetlinewidth{1.505625pt}%
\definecolor{currentstroke}{rgb}{1.000000,0.000000,0.000000}%
\pgfsetstrokecolor{currentstroke}%
\pgfsetdash{}{0pt}%
\pgfpathmoveto{\pgfqpoint{1.185873in}{1.694276in}}%
\pgfpathlineto{\pgfqpoint{1.174124in}{1.692459in}}%
\pgfusepath{stroke}%
\end{pgfscope}%
\begin{pgfscope}%
\pgfpathrectangle{\pgfqpoint{0.100000in}{0.220728in}}{\pgfqpoint{3.696000in}{3.696000in}}%
\pgfusepath{clip}%
\pgfsetrectcap%
\pgfsetroundjoin%
\pgfsetlinewidth{1.505625pt}%
\definecolor{currentstroke}{rgb}{1.000000,0.000000,0.000000}%
\pgfsetstrokecolor{currentstroke}%
\pgfsetdash{}{0pt}%
\pgfpathmoveto{\pgfqpoint{1.190662in}{1.698701in}}%
\pgfpathlineto{\pgfqpoint{1.183295in}{1.700215in}}%
\pgfusepath{stroke}%
\end{pgfscope}%
\begin{pgfscope}%
\pgfpathrectangle{\pgfqpoint{0.100000in}{0.220728in}}{\pgfqpoint{3.696000in}{3.696000in}}%
\pgfusepath{clip}%
\pgfsetrectcap%
\pgfsetroundjoin%
\pgfsetlinewidth{1.505625pt}%
\definecolor{currentstroke}{rgb}{1.000000,0.000000,0.000000}%
\pgfsetstrokecolor{currentstroke}%
\pgfsetdash{}{0pt}%
\pgfpathmoveto{\pgfqpoint{1.193685in}{1.700987in}}%
\pgfpathlineto{\pgfqpoint{1.183295in}{1.700215in}}%
\pgfusepath{stroke}%
\end{pgfscope}%
\begin{pgfscope}%
\pgfpathrectangle{\pgfqpoint{0.100000in}{0.220728in}}{\pgfqpoint{3.696000in}{3.696000in}}%
\pgfusepath{clip}%
\pgfsetrectcap%
\pgfsetroundjoin%
\pgfsetlinewidth{1.505625pt}%
\definecolor{currentstroke}{rgb}{1.000000,0.000000,0.000000}%
\pgfsetstrokecolor{currentstroke}%
\pgfsetdash{}{0pt}%
\pgfpathmoveto{\pgfqpoint{1.195053in}{1.701061in}}%
\pgfpathlineto{\pgfqpoint{1.183295in}{1.700215in}}%
\pgfusepath{stroke}%
\end{pgfscope}%
\begin{pgfscope}%
\pgfpathrectangle{\pgfqpoint{0.100000in}{0.220728in}}{\pgfqpoint{3.696000in}{3.696000in}}%
\pgfusepath{clip}%
\pgfsetrectcap%
\pgfsetroundjoin%
\pgfsetlinewidth{1.505625pt}%
\definecolor{currentstroke}{rgb}{1.000000,0.000000,0.000000}%
\pgfsetstrokecolor{currentstroke}%
\pgfsetdash{}{0pt}%
\pgfpathmoveto{\pgfqpoint{1.198879in}{1.706265in}}%
\pgfpathlineto{\pgfqpoint{1.183295in}{1.700215in}}%
\pgfusepath{stroke}%
\end{pgfscope}%
\begin{pgfscope}%
\pgfpathrectangle{\pgfqpoint{0.100000in}{0.220728in}}{\pgfqpoint{3.696000in}{3.696000in}}%
\pgfusepath{clip}%
\pgfsetrectcap%
\pgfsetroundjoin%
\pgfsetlinewidth{1.505625pt}%
\definecolor{currentstroke}{rgb}{1.000000,0.000000,0.000000}%
\pgfsetstrokecolor{currentstroke}%
\pgfsetdash{}{0pt}%
\pgfpathmoveto{\pgfqpoint{1.201039in}{1.708134in}}%
\pgfpathlineto{\pgfqpoint{1.183295in}{1.700215in}}%
\pgfusepath{stroke}%
\end{pgfscope}%
\begin{pgfscope}%
\pgfpathrectangle{\pgfqpoint{0.100000in}{0.220728in}}{\pgfqpoint{3.696000in}{3.696000in}}%
\pgfusepath{clip}%
\pgfsetrectcap%
\pgfsetroundjoin%
\pgfsetlinewidth{1.505625pt}%
\definecolor{currentstroke}{rgb}{1.000000,0.000000,0.000000}%
\pgfsetstrokecolor{currentstroke}%
\pgfsetdash{}{0pt}%
\pgfpathmoveto{\pgfqpoint{1.203488in}{1.709158in}}%
\pgfpathlineto{\pgfqpoint{1.192454in}{1.707961in}}%
\pgfusepath{stroke}%
\end{pgfscope}%
\begin{pgfscope}%
\pgfpathrectangle{\pgfqpoint{0.100000in}{0.220728in}}{\pgfqpoint{3.696000in}{3.696000in}}%
\pgfusepath{clip}%
\pgfsetrectcap%
\pgfsetroundjoin%
\pgfsetlinewidth{1.505625pt}%
\definecolor{currentstroke}{rgb}{1.000000,0.000000,0.000000}%
\pgfsetstrokecolor{currentstroke}%
\pgfsetdash{}{0pt}%
\pgfpathmoveto{\pgfqpoint{1.209360in}{1.716725in}}%
\pgfpathlineto{\pgfqpoint{1.192454in}{1.707961in}}%
\pgfusepath{stroke}%
\end{pgfscope}%
\begin{pgfscope}%
\pgfpathrectangle{\pgfqpoint{0.100000in}{0.220728in}}{\pgfqpoint{3.696000in}{3.696000in}}%
\pgfusepath{clip}%
\pgfsetrectcap%
\pgfsetroundjoin%
\pgfsetlinewidth{1.505625pt}%
\definecolor{currentstroke}{rgb}{1.000000,0.000000,0.000000}%
\pgfsetstrokecolor{currentstroke}%
\pgfsetdash{}{0pt}%
\pgfpathmoveto{\pgfqpoint{1.212305in}{1.720256in}}%
\pgfpathlineto{\pgfqpoint{1.192454in}{1.707961in}}%
\pgfusepath{stroke}%
\end{pgfscope}%
\begin{pgfscope}%
\pgfpathrectangle{\pgfqpoint{0.100000in}{0.220728in}}{\pgfqpoint{3.696000in}{3.696000in}}%
\pgfusepath{clip}%
\pgfsetrectcap%
\pgfsetroundjoin%
\pgfsetlinewidth{1.505625pt}%
\definecolor{currentstroke}{rgb}{1.000000,0.000000,0.000000}%
\pgfsetstrokecolor{currentstroke}%
\pgfsetdash{}{0pt}%
\pgfpathmoveto{\pgfqpoint{1.216164in}{1.722636in}}%
\pgfpathlineto{\pgfqpoint{1.201602in}{1.715698in}}%
\pgfusepath{stroke}%
\end{pgfscope}%
\begin{pgfscope}%
\pgfpathrectangle{\pgfqpoint{0.100000in}{0.220728in}}{\pgfqpoint{3.696000in}{3.696000in}}%
\pgfusepath{clip}%
\pgfsetrectcap%
\pgfsetroundjoin%
\pgfsetlinewidth{1.505625pt}%
\definecolor{currentstroke}{rgb}{1.000000,0.000000,0.000000}%
\pgfsetstrokecolor{currentstroke}%
\pgfsetdash{}{0pt}%
\pgfpathmoveto{\pgfqpoint{1.220841in}{1.726288in}}%
\pgfpathlineto{\pgfqpoint{1.201602in}{1.715698in}}%
\pgfusepath{stroke}%
\end{pgfscope}%
\begin{pgfscope}%
\pgfpathrectangle{\pgfqpoint{0.100000in}{0.220728in}}{\pgfqpoint{3.696000in}{3.696000in}}%
\pgfusepath{clip}%
\pgfsetrectcap%
\pgfsetroundjoin%
\pgfsetlinewidth{1.505625pt}%
\definecolor{currentstroke}{rgb}{1.000000,0.000000,0.000000}%
\pgfsetstrokecolor{currentstroke}%
\pgfsetdash{}{0pt}%
\pgfpathmoveto{\pgfqpoint{1.228427in}{1.732539in}}%
\pgfpathlineto{\pgfqpoint{1.210737in}{1.723424in}}%
\pgfusepath{stroke}%
\end{pgfscope}%
\begin{pgfscope}%
\pgfpathrectangle{\pgfqpoint{0.100000in}{0.220728in}}{\pgfqpoint{3.696000in}{3.696000in}}%
\pgfusepath{clip}%
\pgfsetrectcap%
\pgfsetroundjoin%
\pgfsetlinewidth{1.505625pt}%
\definecolor{currentstroke}{rgb}{1.000000,0.000000,0.000000}%
\pgfsetstrokecolor{currentstroke}%
\pgfsetdash{}{0pt}%
\pgfpathmoveto{\pgfqpoint{1.237089in}{1.735504in}}%
\pgfpathlineto{\pgfqpoint{1.219861in}{1.731140in}}%
\pgfusepath{stroke}%
\end{pgfscope}%
\begin{pgfscope}%
\pgfpathrectangle{\pgfqpoint{0.100000in}{0.220728in}}{\pgfqpoint{3.696000in}{3.696000in}}%
\pgfusepath{clip}%
\pgfsetrectcap%
\pgfsetroundjoin%
\pgfsetlinewidth{1.505625pt}%
\definecolor{currentstroke}{rgb}{1.000000,0.000000,0.000000}%
\pgfsetstrokecolor{currentstroke}%
\pgfsetdash{}{0pt}%
\pgfpathmoveto{\pgfqpoint{1.247559in}{1.741475in}}%
\pgfpathlineto{\pgfqpoint{1.228973in}{1.738846in}}%
\pgfusepath{stroke}%
\end{pgfscope}%
\begin{pgfscope}%
\pgfpathrectangle{\pgfqpoint{0.100000in}{0.220728in}}{\pgfqpoint{3.696000in}{3.696000in}}%
\pgfusepath{clip}%
\pgfsetrectcap%
\pgfsetroundjoin%
\pgfsetlinewidth{1.505625pt}%
\definecolor{currentstroke}{rgb}{1.000000,0.000000,0.000000}%
\pgfsetstrokecolor{currentstroke}%
\pgfsetdash{}{0pt}%
\pgfpathmoveto{\pgfqpoint{1.261178in}{1.748156in}}%
\pgfpathlineto{\pgfqpoint{1.238074in}{1.746542in}}%
\pgfusepath{stroke}%
\end{pgfscope}%
\begin{pgfscope}%
\pgfpathrectangle{\pgfqpoint{0.100000in}{0.220728in}}{\pgfqpoint{3.696000in}{3.696000in}}%
\pgfusepath{clip}%
\pgfsetrectcap%
\pgfsetroundjoin%
\pgfsetlinewidth{1.505625pt}%
\definecolor{currentstroke}{rgb}{1.000000,0.000000,0.000000}%
\pgfsetstrokecolor{currentstroke}%
\pgfsetdash{}{0pt}%
\pgfpathmoveto{\pgfqpoint{1.274407in}{1.752607in}}%
\pgfpathlineto{\pgfqpoint{1.247162in}{1.754229in}}%
\pgfusepath{stroke}%
\end{pgfscope}%
\begin{pgfscope}%
\pgfpathrectangle{\pgfqpoint{0.100000in}{0.220728in}}{\pgfqpoint{3.696000in}{3.696000in}}%
\pgfusepath{clip}%
\pgfsetrectcap%
\pgfsetroundjoin%
\pgfsetlinewidth{1.505625pt}%
\definecolor{currentstroke}{rgb}{1.000000,0.000000,0.000000}%
\pgfsetstrokecolor{currentstroke}%
\pgfsetdash{}{0pt}%
\pgfpathmoveto{\pgfqpoint{1.283576in}{1.756544in}}%
\pgfpathlineto{\pgfqpoint{1.256239in}{1.761905in}}%
\pgfusepath{stroke}%
\end{pgfscope}%
\begin{pgfscope}%
\pgfpathrectangle{\pgfqpoint{0.100000in}{0.220728in}}{\pgfqpoint{3.696000in}{3.696000in}}%
\pgfusepath{clip}%
\pgfsetrectcap%
\pgfsetroundjoin%
\pgfsetlinewidth{1.505625pt}%
\definecolor{currentstroke}{rgb}{1.000000,0.000000,0.000000}%
\pgfsetstrokecolor{currentstroke}%
\pgfsetdash{}{0pt}%
\pgfpathmoveto{\pgfqpoint{1.295085in}{1.764393in}}%
\pgfpathlineto{\pgfqpoint{1.265304in}{1.769571in}}%
\pgfusepath{stroke}%
\end{pgfscope}%
\begin{pgfscope}%
\pgfpathrectangle{\pgfqpoint{0.100000in}{0.220728in}}{\pgfqpoint{3.696000in}{3.696000in}}%
\pgfusepath{clip}%
\pgfsetrectcap%
\pgfsetroundjoin%
\pgfsetlinewidth{1.505625pt}%
\definecolor{currentstroke}{rgb}{1.000000,0.000000,0.000000}%
\pgfsetstrokecolor{currentstroke}%
\pgfsetdash{}{0pt}%
\pgfpathmoveto{\pgfqpoint{1.306094in}{1.769967in}}%
\pgfpathlineto{\pgfqpoint{1.274357in}{1.777228in}}%
\pgfusepath{stroke}%
\end{pgfscope}%
\begin{pgfscope}%
\pgfpathrectangle{\pgfqpoint{0.100000in}{0.220728in}}{\pgfqpoint{3.696000in}{3.696000in}}%
\pgfusepath{clip}%
\pgfsetrectcap%
\pgfsetroundjoin%
\pgfsetlinewidth{1.505625pt}%
\definecolor{currentstroke}{rgb}{1.000000,0.000000,0.000000}%
\pgfsetstrokecolor{currentstroke}%
\pgfsetdash{}{0pt}%
\pgfpathmoveto{\pgfqpoint{1.312772in}{1.772089in}}%
\pgfpathlineto{\pgfqpoint{1.283399in}{1.784875in}}%
\pgfusepath{stroke}%
\end{pgfscope}%
\begin{pgfscope}%
\pgfpathrectangle{\pgfqpoint{0.100000in}{0.220728in}}{\pgfqpoint{3.696000in}{3.696000in}}%
\pgfusepath{clip}%
\pgfsetrectcap%
\pgfsetroundjoin%
\pgfsetlinewidth{1.505625pt}%
\definecolor{currentstroke}{rgb}{1.000000,0.000000,0.000000}%
\pgfsetstrokecolor{currentstroke}%
\pgfsetdash{}{0pt}%
\pgfpathmoveto{\pgfqpoint{1.316653in}{1.774882in}}%
\pgfpathlineto{\pgfqpoint{1.283399in}{1.784875in}}%
\pgfusepath{stroke}%
\end{pgfscope}%
\begin{pgfscope}%
\pgfpathrectangle{\pgfqpoint{0.100000in}{0.220728in}}{\pgfqpoint{3.696000in}{3.696000in}}%
\pgfusepath{clip}%
\pgfsetrectcap%
\pgfsetroundjoin%
\pgfsetlinewidth{1.505625pt}%
\definecolor{currentstroke}{rgb}{1.000000,0.000000,0.000000}%
\pgfsetstrokecolor{currentstroke}%
\pgfsetdash{}{0pt}%
\pgfpathmoveto{\pgfqpoint{1.322089in}{1.776422in}}%
\pgfpathlineto{\pgfqpoint{1.283399in}{1.784875in}}%
\pgfusepath{stroke}%
\end{pgfscope}%
\begin{pgfscope}%
\pgfpathrectangle{\pgfqpoint{0.100000in}{0.220728in}}{\pgfqpoint{3.696000in}{3.696000in}}%
\pgfusepath{clip}%
\pgfsetrectcap%
\pgfsetroundjoin%
\pgfsetlinewidth{1.505625pt}%
\definecolor{currentstroke}{rgb}{1.000000,0.000000,0.000000}%
\pgfsetstrokecolor{currentstroke}%
\pgfsetdash{}{0pt}%
\pgfpathmoveto{\pgfqpoint{1.325218in}{1.779050in}}%
\pgfpathlineto{\pgfqpoint{1.292429in}{1.792512in}}%
\pgfusepath{stroke}%
\end{pgfscope}%
\begin{pgfscope}%
\pgfpathrectangle{\pgfqpoint{0.100000in}{0.220728in}}{\pgfqpoint{3.696000in}{3.696000in}}%
\pgfusepath{clip}%
\pgfsetrectcap%
\pgfsetroundjoin%
\pgfsetlinewidth{1.505625pt}%
\definecolor{currentstroke}{rgb}{1.000000,0.000000,0.000000}%
\pgfsetstrokecolor{currentstroke}%
\pgfsetdash{}{0pt}%
\pgfpathmoveto{\pgfqpoint{1.331351in}{1.782240in}}%
\pgfpathlineto{\pgfqpoint{1.292429in}{1.792512in}}%
\pgfusepath{stroke}%
\end{pgfscope}%
\begin{pgfscope}%
\pgfpathrectangle{\pgfqpoint{0.100000in}{0.220728in}}{\pgfqpoint{3.696000in}{3.696000in}}%
\pgfusepath{clip}%
\pgfsetrectcap%
\pgfsetroundjoin%
\pgfsetlinewidth{1.505625pt}%
\definecolor{currentstroke}{rgb}{1.000000,0.000000,0.000000}%
\pgfsetstrokecolor{currentstroke}%
\pgfsetdash{}{0pt}%
\pgfpathmoveto{\pgfqpoint{1.337973in}{1.783659in}}%
\pgfpathlineto{\pgfqpoint{1.301448in}{1.800139in}}%
\pgfusepath{stroke}%
\end{pgfscope}%
\begin{pgfscope}%
\pgfpathrectangle{\pgfqpoint{0.100000in}{0.220728in}}{\pgfqpoint{3.696000in}{3.696000in}}%
\pgfusepath{clip}%
\pgfsetrectcap%
\pgfsetroundjoin%
\pgfsetlinewidth{1.505625pt}%
\definecolor{currentstroke}{rgb}{1.000000,0.000000,0.000000}%
\pgfsetstrokecolor{currentstroke}%
\pgfsetdash{}{0pt}%
\pgfpathmoveto{\pgfqpoint{1.342213in}{1.787872in}}%
\pgfpathlineto{\pgfqpoint{1.301448in}{1.800139in}}%
\pgfusepath{stroke}%
\end{pgfscope}%
\begin{pgfscope}%
\pgfpathrectangle{\pgfqpoint{0.100000in}{0.220728in}}{\pgfqpoint{3.696000in}{3.696000in}}%
\pgfusepath{clip}%
\pgfsetrectcap%
\pgfsetroundjoin%
\pgfsetlinewidth{1.505625pt}%
\definecolor{currentstroke}{rgb}{1.000000,0.000000,0.000000}%
\pgfsetstrokecolor{currentstroke}%
\pgfsetdash{}{0pt}%
\pgfpathmoveto{\pgfqpoint{1.348715in}{1.790071in}}%
\pgfpathlineto{\pgfqpoint{1.310455in}{1.807756in}}%
\pgfusepath{stroke}%
\end{pgfscope}%
\begin{pgfscope}%
\pgfpathrectangle{\pgfqpoint{0.100000in}{0.220728in}}{\pgfqpoint{3.696000in}{3.696000in}}%
\pgfusepath{clip}%
\pgfsetrectcap%
\pgfsetroundjoin%
\pgfsetlinewidth{1.505625pt}%
\definecolor{currentstroke}{rgb}{1.000000,0.000000,0.000000}%
\pgfsetstrokecolor{currentstroke}%
\pgfsetdash{}{0pt}%
\pgfpathmoveto{\pgfqpoint{1.356464in}{1.792751in}}%
\pgfpathlineto{\pgfqpoint{1.319450in}{1.815363in}}%
\pgfusepath{stroke}%
\end{pgfscope}%
\begin{pgfscope}%
\pgfpathrectangle{\pgfqpoint{0.100000in}{0.220728in}}{\pgfqpoint{3.696000in}{3.696000in}}%
\pgfusepath{clip}%
\pgfsetrectcap%
\pgfsetroundjoin%
\pgfsetlinewidth{1.505625pt}%
\definecolor{currentstroke}{rgb}{1.000000,0.000000,0.000000}%
\pgfsetstrokecolor{currentstroke}%
\pgfsetdash{}{0pt}%
\pgfpathmoveto{\pgfqpoint{1.361612in}{1.795875in}}%
\pgfpathlineto{\pgfqpoint{1.319450in}{1.815363in}}%
\pgfusepath{stroke}%
\end{pgfscope}%
\begin{pgfscope}%
\pgfpathrectangle{\pgfqpoint{0.100000in}{0.220728in}}{\pgfqpoint{3.696000in}{3.696000in}}%
\pgfusepath{clip}%
\pgfsetrectcap%
\pgfsetroundjoin%
\pgfsetlinewidth{1.505625pt}%
\definecolor{currentstroke}{rgb}{1.000000,0.000000,0.000000}%
\pgfsetstrokecolor{currentstroke}%
\pgfsetdash{}{0pt}%
\pgfpathmoveto{\pgfqpoint{1.364436in}{1.798153in}}%
\pgfpathlineto{\pgfqpoint{1.319450in}{1.815363in}}%
\pgfusepath{stroke}%
\end{pgfscope}%
\begin{pgfscope}%
\pgfpathrectangle{\pgfqpoint{0.100000in}{0.220728in}}{\pgfqpoint{3.696000in}{3.696000in}}%
\pgfusepath{clip}%
\pgfsetrectcap%
\pgfsetroundjoin%
\pgfsetlinewidth{1.505625pt}%
\definecolor{currentstroke}{rgb}{1.000000,0.000000,0.000000}%
\pgfsetstrokecolor{currentstroke}%
\pgfsetdash{}{0pt}%
\pgfpathmoveto{\pgfqpoint{1.367126in}{1.799265in}}%
\pgfpathlineto{\pgfqpoint{1.328434in}{1.822961in}}%
\pgfusepath{stroke}%
\end{pgfscope}%
\begin{pgfscope}%
\pgfpathrectangle{\pgfqpoint{0.100000in}{0.220728in}}{\pgfqpoint{3.696000in}{3.696000in}}%
\pgfusepath{clip}%
\pgfsetrectcap%
\pgfsetroundjoin%
\pgfsetlinewidth{1.505625pt}%
\definecolor{currentstroke}{rgb}{1.000000,0.000000,0.000000}%
\pgfsetstrokecolor{currentstroke}%
\pgfsetdash{}{0pt}%
\pgfpathmoveto{\pgfqpoint{1.371037in}{1.802955in}}%
\pgfpathlineto{\pgfqpoint{1.328434in}{1.822961in}}%
\pgfusepath{stroke}%
\end{pgfscope}%
\begin{pgfscope}%
\pgfpathrectangle{\pgfqpoint{0.100000in}{0.220728in}}{\pgfqpoint{3.696000in}{3.696000in}}%
\pgfusepath{clip}%
\pgfsetrectcap%
\pgfsetroundjoin%
\pgfsetlinewidth{1.505625pt}%
\definecolor{currentstroke}{rgb}{1.000000,0.000000,0.000000}%
\pgfsetstrokecolor{currentstroke}%
\pgfsetdash{}{0pt}%
\pgfpathmoveto{\pgfqpoint{1.376489in}{1.806575in}}%
\pgfpathlineto{\pgfqpoint{1.337406in}{1.830549in}}%
\pgfusepath{stroke}%
\end{pgfscope}%
\begin{pgfscope}%
\pgfpathrectangle{\pgfqpoint{0.100000in}{0.220728in}}{\pgfqpoint{3.696000in}{3.696000in}}%
\pgfusepath{clip}%
\pgfsetrectcap%
\pgfsetroundjoin%
\pgfsetlinewidth{1.505625pt}%
\definecolor{currentstroke}{rgb}{1.000000,0.000000,0.000000}%
\pgfsetstrokecolor{currentstroke}%
\pgfsetdash{}{0pt}%
\pgfpathmoveto{\pgfqpoint{1.383144in}{1.810144in}}%
\pgfpathlineto{\pgfqpoint{1.337406in}{1.830549in}}%
\pgfusepath{stroke}%
\end{pgfscope}%
\begin{pgfscope}%
\pgfpathrectangle{\pgfqpoint{0.100000in}{0.220728in}}{\pgfqpoint{3.696000in}{3.696000in}}%
\pgfusepath{clip}%
\pgfsetrectcap%
\pgfsetroundjoin%
\pgfsetlinewidth{1.505625pt}%
\definecolor{currentstroke}{rgb}{1.000000,0.000000,0.000000}%
\pgfsetstrokecolor{currentstroke}%
\pgfsetdash{}{0pt}%
\pgfpathmoveto{\pgfqpoint{1.387627in}{1.812859in}}%
\pgfpathlineto{\pgfqpoint{1.346367in}{1.838127in}}%
\pgfusepath{stroke}%
\end{pgfscope}%
\begin{pgfscope}%
\pgfpathrectangle{\pgfqpoint{0.100000in}{0.220728in}}{\pgfqpoint{3.696000in}{3.696000in}}%
\pgfusepath{clip}%
\pgfsetrectcap%
\pgfsetroundjoin%
\pgfsetlinewidth{1.505625pt}%
\definecolor{currentstroke}{rgb}{1.000000,0.000000,0.000000}%
\pgfsetstrokecolor{currentstroke}%
\pgfsetdash{}{0pt}%
\pgfpathmoveto{\pgfqpoint{1.392960in}{1.819832in}}%
\pgfpathlineto{\pgfqpoint{1.346367in}{1.838127in}}%
\pgfusepath{stroke}%
\end{pgfscope}%
\begin{pgfscope}%
\pgfpathrectangle{\pgfqpoint{0.100000in}{0.220728in}}{\pgfqpoint{3.696000in}{3.696000in}}%
\pgfusepath{clip}%
\pgfsetrectcap%
\pgfsetroundjoin%
\pgfsetlinewidth{1.505625pt}%
\definecolor{currentstroke}{rgb}{1.000000,0.000000,0.000000}%
\pgfsetstrokecolor{currentstroke}%
\pgfsetdash{}{0pt}%
\pgfpathmoveto{\pgfqpoint{1.400928in}{1.823856in}}%
\pgfpathlineto{\pgfqpoint{1.355316in}{1.845696in}}%
\pgfusepath{stroke}%
\end{pgfscope}%
\begin{pgfscope}%
\pgfpathrectangle{\pgfqpoint{0.100000in}{0.220728in}}{\pgfqpoint{3.696000in}{3.696000in}}%
\pgfusepath{clip}%
\pgfsetrectcap%
\pgfsetroundjoin%
\pgfsetlinewidth{1.505625pt}%
\definecolor{currentstroke}{rgb}{1.000000,0.000000,0.000000}%
\pgfsetstrokecolor{currentstroke}%
\pgfsetdash{}{0pt}%
\pgfpathmoveto{\pgfqpoint{1.405181in}{1.827118in}}%
\pgfpathlineto{\pgfqpoint{1.355316in}{1.845696in}}%
\pgfusepath{stroke}%
\end{pgfscope}%
\begin{pgfscope}%
\pgfpathrectangle{\pgfqpoint{0.100000in}{0.220728in}}{\pgfqpoint{3.696000in}{3.696000in}}%
\pgfusepath{clip}%
\pgfsetrectcap%
\pgfsetroundjoin%
\pgfsetlinewidth{1.505625pt}%
\definecolor{currentstroke}{rgb}{1.000000,0.000000,0.000000}%
\pgfsetstrokecolor{currentstroke}%
\pgfsetdash{}{0pt}%
\pgfpathmoveto{\pgfqpoint{1.410496in}{1.830959in}}%
\pgfpathlineto{\pgfqpoint{1.364254in}{1.853255in}}%
\pgfusepath{stroke}%
\end{pgfscope}%
\begin{pgfscope}%
\pgfpathrectangle{\pgfqpoint{0.100000in}{0.220728in}}{\pgfqpoint{3.696000in}{3.696000in}}%
\pgfusepath{clip}%
\pgfsetrectcap%
\pgfsetroundjoin%
\pgfsetlinewidth{1.505625pt}%
\definecolor{currentstroke}{rgb}{1.000000,0.000000,0.000000}%
\pgfsetstrokecolor{currentstroke}%
\pgfsetdash{}{0pt}%
\pgfpathmoveto{\pgfqpoint{1.416246in}{1.832405in}}%
\pgfpathlineto{\pgfqpoint{1.373180in}{1.860804in}}%
\pgfusepath{stroke}%
\end{pgfscope}%
\begin{pgfscope}%
\pgfpathrectangle{\pgfqpoint{0.100000in}{0.220728in}}{\pgfqpoint{3.696000in}{3.696000in}}%
\pgfusepath{clip}%
\pgfsetrectcap%
\pgfsetroundjoin%
\pgfsetlinewidth{1.505625pt}%
\definecolor{currentstroke}{rgb}{1.000000,0.000000,0.000000}%
\pgfsetstrokecolor{currentstroke}%
\pgfsetdash{}{0pt}%
\pgfpathmoveto{\pgfqpoint{1.423673in}{1.836898in}}%
\pgfpathlineto{\pgfqpoint{1.373180in}{1.860804in}}%
\pgfusepath{stroke}%
\end{pgfscope}%
\begin{pgfscope}%
\pgfpathrectangle{\pgfqpoint{0.100000in}{0.220728in}}{\pgfqpoint{3.696000in}{3.696000in}}%
\pgfusepath{clip}%
\pgfsetrectcap%
\pgfsetroundjoin%
\pgfsetlinewidth{1.505625pt}%
\definecolor{currentstroke}{rgb}{1.000000,0.000000,0.000000}%
\pgfsetstrokecolor{currentstroke}%
\pgfsetdash{}{0pt}%
\pgfpathmoveto{\pgfqpoint{1.428003in}{1.841254in}}%
\pgfpathlineto{\pgfqpoint{1.382095in}{1.868344in}}%
\pgfusepath{stroke}%
\end{pgfscope}%
\begin{pgfscope}%
\pgfpathrectangle{\pgfqpoint{0.100000in}{0.220728in}}{\pgfqpoint{3.696000in}{3.696000in}}%
\pgfusepath{clip}%
\pgfsetrectcap%
\pgfsetroundjoin%
\pgfsetlinewidth{1.505625pt}%
\definecolor{currentstroke}{rgb}{1.000000,0.000000,0.000000}%
\pgfsetstrokecolor{currentstroke}%
\pgfsetdash{}{0pt}%
\pgfpathmoveto{\pgfqpoint{1.432886in}{1.843344in}}%
\pgfpathlineto{\pgfqpoint{1.382095in}{1.868344in}}%
\pgfusepath{stroke}%
\end{pgfscope}%
\begin{pgfscope}%
\pgfpathrectangle{\pgfqpoint{0.100000in}{0.220728in}}{\pgfqpoint{3.696000in}{3.696000in}}%
\pgfusepath{clip}%
\pgfsetrectcap%
\pgfsetroundjoin%
\pgfsetlinewidth{1.505625pt}%
\definecolor{currentstroke}{rgb}{1.000000,0.000000,0.000000}%
\pgfsetstrokecolor{currentstroke}%
\pgfsetdash{}{0pt}%
\pgfpathmoveto{\pgfqpoint{1.438737in}{1.847170in}}%
\pgfpathlineto{\pgfqpoint{1.390999in}{1.875873in}}%
\pgfusepath{stroke}%
\end{pgfscope}%
\begin{pgfscope}%
\pgfpathrectangle{\pgfqpoint{0.100000in}{0.220728in}}{\pgfqpoint{3.696000in}{3.696000in}}%
\pgfusepath{clip}%
\pgfsetrectcap%
\pgfsetroundjoin%
\pgfsetlinewidth{1.505625pt}%
\definecolor{currentstroke}{rgb}{1.000000,0.000000,0.000000}%
\pgfsetstrokecolor{currentstroke}%
\pgfsetdash{}{0pt}%
\pgfpathmoveto{\pgfqpoint{1.445456in}{1.852257in}}%
\pgfpathlineto{\pgfqpoint{1.390999in}{1.875873in}}%
\pgfusepath{stroke}%
\end{pgfscope}%
\begin{pgfscope}%
\pgfpathrectangle{\pgfqpoint{0.100000in}{0.220728in}}{\pgfqpoint{3.696000in}{3.696000in}}%
\pgfusepath{clip}%
\pgfsetrectcap%
\pgfsetroundjoin%
\pgfsetlinewidth{1.505625pt}%
\definecolor{currentstroke}{rgb}{1.000000,0.000000,0.000000}%
\pgfsetstrokecolor{currentstroke}%
\pgfsetdash{}{0pt}%
\pgfpathmoveto{\pgfqpoint{1.451956in}{1.854657in}}%
\pgfpathlineto{\pgfqpoint{1.399891in}{1.883394in}}%
\pgfusepath{stroke}%
\end{pgfscope}%
\begin{pgfscope}%
\pgfpathrectangle{\pgfqpoint{0.100000in}{0.220728in}}{\pgfqpoint{3.696000in}{3.696000in}}%
\pgfusepath{clip}%
\pgfsetrectcap%
\pgfsetroundjoin%
\pgfsetlinewidth{1.505625pt}%
\definecolor{currentstroke}{rgb}{1.000000,0.000000,0.000000}%
\pgfsetstrokecolor{currentstroke}%
\pgfsetdash{}{0pt}%
\pgfpathmoveto{\pgfqpoint{1.458261in}{1.838428in}}%
\pgfpathlineto{\pgfqpoint{1.408772in}{1.890904in}}%
\pgfusepath{stroke}%
\end{pgfscope}%
\begin{pgfscope}%
\pgfpathrectangle{\pgfqpoint{0.100000in}{0.220728in}}{\pgfqpoint{3.696000in}{3.696000in}}%
\pgfusepath{clip}%
\pgfsetrectcap%
\pgfsetroundjoin%
\pgfsetlinewidth{1.505625pt}%
\definecolor{currentstroke}{rgb}{1.000000,0.000000,0.000000}%
\pgfsetstrokecolor{currentstroke}%
\pgfsetdash{}{0pt}%
\pgfpathmoveto{\pgfqpoint{1.462021in}{1.842003in}}%
\pgfpathlineto{\pgfqpoint{1.408772in}{1.890904in}}%
\pgfusepath{stroke}%
\end{pgfscope}%
\begin{pgfscope}%
\pgfpathrectangle{\pgfqpoint{0.100000in}{0.220728in}}{\pgfqpoint{3.696000in}{3.696000in}}%
\pgfusepath{clip}%
\pgfsetrectcap%
\pgfsetroundjoin%
\pgfsetlinewidth{1.505625pt}%
\definecolor{currentstroke}{rgb}{1.000000,0.000000,0.000000}%
\pgfsetstrokecolor{currentstroke}%
\pgfsetdash{}{0pt}%
\pgfpathmoveto{\pgfqpoint{1.466766in}{1.842466in}}%
\pgfpathlineto{\pgfqpoint{1.417641in}{1.898406in}}%
\pgfusepath{stroke}%
\end{pgfscope}%
\begin{pgfscope}%
\pgfpathrectangle{\pgfqpoint{0.100000in}{0.220728in}}{\pgfqpoint{3.696000in}{3.696000in}}%
\pgfusepath{clip}%
\pgfsetrectcap%
\pgfsetroundjoin%
\pgfsetlinewidth{1.505625pt}%
\definecolor{currentstroke}{rgb}{1.000000,0.000000,0.000000}%
\pgfsetstrokecolor{currentstroke}%
\pgfsetdash{}{0pt}%
\pgfpathmoveto{\pgfqpoint{1.473660in}{1.837256in}}%
\pgfpathlineto{\pgfqpoint{1.417641in}{1.898406in}}%
\pgfusepath{stroke}%
\end{pgfscope}%
\begin{pgfscope}%
\pgfpathrectangle{\pgfqpoint{0.100000in}{0.220728in}}{\pgfqpoint{3.696000in}{3.696000in}}%
\pgfusepath{clip}%
\pgfsetrectcap%
\pgfsetroundjoin%
\pgfsetlinewidth{1.505625pt}%
\definecolor{currentstroke}{rgb}{1.000000,0.000000,0.000000}%
\pgfsetstrokecolor{currentstroke}%
\pgfsetdash{}{0pt}%
\pgfpathmoveto{\pgfqpoint{1.477105in}{1.841829in}}%
\pgfpathlineto{\pgfqpoint{1.426500in}{1.905897in}}%
\pgfusepath{stroke}%
\end{pgfscope}%
\begin{pgfscope}%
\pgfpathrectangle{\pgfqpoint{0.100000in}{0.220728in}}{\pgfqpoint{3.696000in}{3.696000in}}%
\pgfusepath{clip}%
\pgfsetrectcap%
\pgfsetroundjoin%
\pgfsetlinewidth{1.505625pt}%
\definecolor{currentstroke}{rgb}{1.000000,0.000000,0.000000}%
\pgfsetstrokecolor{currentstroke}%
\pgfsetdash{}{0pt}%
\pgfpathmoveto{\pgfqpoint{1.480823in}{1.840474in}}%
\pgfpathlineto{\pgfqpoint{1.426500in}{1.905897in}}%
\pgfusepath{stroke}%
\end{pgfscope}%
\begin{pgfscope}%
\pgfpathrectangle{\pgfqpoint{0.100000in}{0.220728in}}{\pgfqpoint{3.696000in}{3.696000in}}%
\pgfusepath{clip}%
\pgfsetrectcap%
\pgfsetroundjoin%
\pgfsetlinewidth{1.505625pt}%
\definecolor{currentstroke}{rgb}{1.000000,0.000000,0.000000}%
\pgfsetstrokecolor{currentstroke}%
\pgfsetdash{}{0pt}%
\pgfpathmoveto{\pgfqpoint{1.485682in}{1.845495in}}%
\pgfpathlineto{\pgfqpoint{1.435346in}{1.913379in}}%
\pgfusepath{stroke}%
\end{pgfscope}%
\begin{pgfscope}%
\pgfpathrectangle{\pgfqpoint{0.100000in}{0.220728in}}{\pgfqpoint{3.696000in}{3.696000in}}%
\pgfusepath{clip}%
\pgfsetrectcap%
\pgfsetroundjoin%
\pgfsetlinewidth{1.505625pt}%
\definecolor{currentstroke}{rgb}{1.000000,0.000000,0.000000}%
\pgfsetstrokecolor{currentstroke}%
\pgfsetdash{}{0pt}%
\pgfpathmoveto{\pgfqpoint{1.488511in}{1.846721in}}%
\pgfpathlineto{\pgfqpoint{1.435346in}{1.913379in}}%
\pgfusepath{stroke}%
\end{pgfscope}%
\begin{pgfscope}%
\pgfpathrectangle{\pgfqpoint{0.100000in}{0.220728in}}{\pgfqpoint{3.696000in}{3.696000in}}%
\pgfusepath{clip}%
\pgfsetrectcap%
\pgfsetroundjoin%
\pgfsetlinewidth{1.505625pt}%
\definecolor{currentstroke}{rgb}{1.000000,0.000000,0.000000}%
\pgfsetstrokecolor{currentstroke}%
\pgfsetdash{}{0pt}%
\pgfpathmoveto{\pgfqpoint{1.489927in}{1.846958in}}%
\pgfpathlineto{\pgfqpoint{1.435346in}{1.913379in}}%
\pgfusepath{stroke}%
\end{pgfscope}%
\begin{pgfscope}%
\pgfpathrectangle{\pgfqpoint{0.100000in}{0.220728in}}{\pgfqpoint{3.696000in}{3.696000in}}%
\pgfusepath{clip}%
\pgfsetrectcap%
\pgfsetroundjoin%
\pgfsetlinewidth{1.505625pt}%
\definecolor{currentstroke}{rgb}{1.000000,0.000000,0.000000}%
\pgfsetstrokecolor{currentstroke}%
\pgfsetdash{}{0pt}%
\pgfpathmoveto{\pgfqpoint{1.492317in}{1.848489in}}%
\pgfpathlineto{\pgfqpoint{1.435346in}{1.913379in}}%
\pgfusepath{stroke}%
\end{pgfscope}%
\begin{pgfscope}%
\pgfpathrectangle{\pgfqpoint{0.100000in}{0.220728in}}{\pgfqpoint{3.696000in}{3.696000in}}%
\pgfusepath{clip}%
\pgfsetrectcap%
\pgfsetroundjoin%
\pgfsetlinewidth{1.505625pt}%
\definecolor{currentstroke}{rgb}{1.000000,0.000000,0.000000}%
\pgfsetstrokecolor{currentstroke}%
\pgfsetdash{}{0pt}%
\pgfpathmoveto{\pgfqpoint{1.495338in}{1.849750in}}%
\pgfpathlineto{\pgfqpoint{1.444182in}{1.920852in}}%
\pgfusepath{stroke}%
\end{pgfscope}%
\begin{pgfscope}%
\pgfpathrectangle{\pgfqpoint{0.100000in}{0.220728in}}{\pgfqpoint{3.696000in}{3.696000in}}%
\pgfusepath{clip}%
\pgfsetrectcap%
\pgfsetroundjoin%
\pgfsetlinewidth{1.505625pt}%
\definecolor{currentstroke}{rgb}{1.000000,0.000000,0.000000}%
\pgfsetstrokecolor{currentstroke}%
\pgfsetdash{}{0pt}%
\pgfpathmoveto{\pgfqpoint{1.499083in}{1.852030in}}%
\pgfpathlineto{\pgfqpoint{1.444182in}{1.920852in}}%
\pgfusepath{stroke}%
\end{pgfscope}%
\begin{pgfscope}%
\pgfpathrectangle{\pgfqpoint{0.100000in}{0.220728in}}{\pgfqpoint{3.696000in}{3.696000in}}%
\pgfusepath{clip}%
\pgfsetrectcap%
\pgfsetroundjoin%
\pgfsetlinewidth{1.505625pt}%
\definecolor{currentstroke}{rgb}{1.000000,0.000000,0.000000}%
\pgfsetstrokecolor{currentstroke}%
\pgfsetdash{}{0pt}%
\pgfpathmoveto{\pgfqpoint{1.504335in}{1.855061in}}%
\pgfpathlineto{\pgfqpoint{1.444182in}{1.920852in}}%
\pgfusepath{stroke}%
\end{pgfscope}%
\begin{pgfscope}%
\pgfpathrectangle{\pgfqpoint{0.100000in}{0.220728in}}{\pgfqpoint{3.696000in}{3.696000in}}%
\pgfusepath{clip}%
\pgfsetrectcap%
\pgfsetroundjoin%
\pgfsetlinewidth{1.505625pt}%
\definecolor{currentstroke}{rgb}{1.000000,0.000000,0.000000}%
\pgfsetstrokecolor{currentstroke}%
\pgfsetdash{}{0pt}%
\pgfpathmoveto{\pgfqpoint{1.509741in}{1.855327in}}%
\pgfpathlineto{\pgfqpoint{1.453007in}{1.928314in}}%
\pgfusepath{stroke}%
\end{pgfscope}%
\begin{pgfscope}%
\pgfpathrectangle{\pgfqpoint{0.100000in}{0.220728in}}{\pgfqpoint{3.696000in}{3.696000in}}%
\pgfusepath{clip}%
\pgfsetrectcap%
\pgfsetroundjoin%
\pgfsetlinewidth{1.505625pt}%
\definecolor{currentstroke}{rgb}{1.000000,0.000000,0.000000}%
\pgfsetstrokecolor{currentstroke}%
\pgfsetdash{}{0pt}%
\pgfpathmoveto{\pgfqpoint{1.513151in}{1.856461in}}%
\pgfpathlineto{\pgfqpoint{1.453007in}{1.928314in}}%
\pgfusepath{stroke}%
\end{pgfscope}%
\begin{pgfscope}%
\pgfpathrectangle{\pgfqpoint{0.100000in}{0.220728in}}{\pgfqpoint{3.696000in}{3.696000in}}%
\pgfusepath{clip}%
\pgfsetrectcap%
\pgfsetroundjoin%
\pgfsetlinewidth{1.505625pt}%
\definecolor{currentstroke}{rgb}{1.000000,0.000000,0.000000}%
\pgfsetstrokecolor{currentstroke}%
\pgfsetdash{}{0pt}%
\pgfpathmoveto{\pgfqpoint{1.515105in}{1.858221in}}%
\pgfpathlineto{\pgfqpoint{1.453007in}{1.928314in}}%
\pgfusepath{stroke}%
\end{pgfscope}%
\begin{pgfscope}%
\pgfpathrectangle{\pgfqpoint{0.100000in}{0.220728in}}{\pgfqpoint{3.696000in}{3.696000in}}%
\pgfusepath{clip}%
\pgfsetrectcap%
\pgfsetroundjoin%
\pgfsetlinewidth{1.505625pt}%
\definecolor{currentstroke}{rgb}{1.000000,0.000000,0.000000}%
\pgfsetstrokecolor{currentstroke}%
\pgfsetdash{}{0pt}%
\pgfpathmoveto{\pgfqpoint{1.518726in}{1.858707in}}%
\pgfpathlineto{\pgfqpoint{1.461820in}{1.935768in}}%
\pgfusepath{stroke}%
\end{pgfscope}%
\begin{pgfscope}%
\pgfpathrectangle{\pgfqpoint{0.100000in}{0.220728in}}{\pgfqpoint{3.696000in}{3.696000in}}%
\pgfusepath{clip}%
\pgfsetrectcap%
\pgfsetroundjoin%
\pgfsetlinewidth{1.505625pt}%
\definecolor{currentstroke}{rgb}{1.000000,0.000000,0.000000}%
\pgfsetstrokecolor{currentstroke}%
\pgfsetdash{}{0pt}%
\pgfpathmoveto{\pgfqpoint{1.521007in}{1.861353in}}%
\pgfpathlineto{\pgfqpoint{1.461820in}{1.935768in}}%
\pgfusepath{stroke}%
\end{pgfscope}%
\begin{pgfscope}%
\pgfpathrectangle{\pgfqpoint{0.100000in}{0.220728in}}{\pgfqpoint{3.696000in}{3.696000in}}%
\pgfusepath{clip}%
\pgfsetrectcap%
\pgfsetroundjoin%
\pgfsetlinewidth{1.505625pt}%
\definecolor{currentstroke}{rgb}{1.000000,0.000000,0.000000}%
\pgfsetstrokecolor{currentstroke}%
\pgfsetdash{}{0pt}%
\pgfpathmoveto{\pgfqpoint{1.522436in}{1.861900in}}%
\pgfpathlineto{\pgfqpoint{1.461820in}{1.935768in}}%
\pgfusepath{stroke}%
\end{pgfscope}%
\begin{pgfscope}%
\pgfpathrectangle{\pgfqpoint{0.100000in}{0.220728in}}{\pgfqpoint{3.696000in}{3.696000in}}%
\pgfusepath{clip}%
\pgfsetrectcap%
\pgfsetroundjoin%
\pgfsetlinewidth{1.505625pt}%
\definecolor{currentstroke}{rgb}{1.000000,0.000000,0.000000}%
\pgfsetstrokecolor{currentstroke}%
\pgfsetdash{}{0pt}%
\pgfpathmoveto{\pgfqpoint{1.525417in}{1.862119in}}%
\pgfpathlineto{\pgfqpoint{1.461820in}{1.935768in}}%
\pgfusepath{stroke}%
\end{pgfscope}%
\begin{pgfscope}%
\pgfpathrectangle{\pgfqpoint{0.100000in}{0.220728in}}{\pgfqpoint{3.696000in}{3.696000in}}%
\pgfusepath{clip}%
\pgfsetrectcap%
\pgfsetroundjoin%
\pgfsetlinewidth{1.505625pt}%
\definecolor{currentstroke}{rgb}{1.000000,0.000000,0.000000}%
\pgfsetstrokecolor{currentstroke}%
\pgfsetdash{}{0pt}%
\pgfpathmoveto{\pgfqpoint{1.527285in}{1.862732in}}%
\pgfpathlineto{\pgfqpoint{1.470622in}{1.943212in}}%
\pgfusepath{stroke}%
\end{pgfscope}%
\begin{pgfscope}%
\pgfpathrectangle{\pgfqpoint{0.100000in}{0.220728in}}{\pgfqpoint{3.696000in}{3.696000in}}%
\pgfusepath{clip}%
\pgfsetrectcap%
\pgfsetroundjoin%
\pgfsetlinewidth{1.505625pt}%
\definecolor{currentstroke}{rgb}{1.000000,0.000000,0.000000}%
\pgfsetstrokecolor{currentstroke}%
\pgfsetdash{}{0pt}%
\pgfpathmoveto{\pgfqpoint{1.528426in}{1.863452in}}%
\pgfpathlineto{\pgfqpoint{1.470622in}{1.943212in}}%
\pgfusepath{stroke}%
\end{pgfscope}%
\begin{pgfscope}%
\pgfpathrectangle{\pgfqpoint{0.100000in}{0.220728in}}{\pgfqpoint{3.696000in}{3.696000in}}%
\pgfusepath{clip}%
\pgfsetrectcap%
\pgfsetroundjoin%
\pgfsetlinewidth{1.505625pt}%
\definecolor{currentstroke}{rgb}{1.000000,0.000000,0.000000}%
\pgfsetstrokecolor{currentstroke}%
\pgfsetdash{}{0pt}%
\pgfpathmoveto{\pgfqpoint{1.530796in}{1.863771in}}%
\pgfpathlineto{\pgfqpoint{1.470622in}{1.943212in}}%
\pgfusepath{stroke}%
\end{pgfscope}%
\begin{pgfscope}%
\pgfpathrectangle{\pgfqpoint{0.100000in}{0.220728in}}{\pgfqpoint{3.696000in}{3.696000in}}%
\pgfusepath{clip}%
\pgfsetrectcap%
\pgfsetroundjoin%
\pgfsetlinewidth{1.505625pt}%
\definecolor{currentstroke}{rgb}{1.000000,0.000000,0.000000}%
\pgfsetstrokecolor{currentstroke}%
\pgfsetdash{}{0pt}%
\pgfpathmoveto{\pgfqpoint{1.532297in}{1.864047in}}%
\pgfpathlineto{\pgfqpoint{1.470622in}{1.943212in}}%
\pgfusepath{stroke}%
\end{pgfscope}%
\begin{pgfscope}%
\pgfpathrectangle{\pgfqpoint{0.100000in}{0.220728in}}{\pgfqpoint{3.696000in}{3.696000in}}%
\pgfusepath{clip}%
\pgfsetrectcap%
\pgfsetroundjoin%
\pgfsetlinewidth{1.505625pt}%
\definecolor{currentstroke}{rgb}{1.000000,0.000000,0.000000}%
\pgfsetstrokecolor{currentstroke}%
\pgfsetdash{}{0pt}%
\pgfpathmoveto{\pgfqpoint{1.534550in}{1.865362in}}%
\pgfpathlineto{\pgfqpoint{1.470622in}{1.943212in}}%
\pgfusepath{stroke}%
\end{pgfscope}%
\begin{pgfscope}%
\pgfpathrectangle{\pgfqpoint{0.100000in}{0.220728in}}{\pgfqpoint{3.696000in}{3.696000in}}%
\pgfusepath{clip}%
\pgfsetrectcap%
\pgfsetroundjoin%
\pgfsetlinewidth{1.505625pt}%
\definecolor{currentstroke}{rgb}{1.000000,0.000000,0.000000}%
\pgfsetstrokecolor{currentstroke}%
\pgfsetdash{}{0pt}%
\pgfpathmoveto{\pgfqpoint{1.538025in}{1.867177in}}%
\pgfpathlineto{\pgfqpoint{1.479413in}{1.950647in}}%
\pgfusepath{stroke}%
\end{pgfscope}%
\begin{pgfscope}%
\pgfpathrectangle{\pgfqpoint{0.100000in}{0.220728in}}{\pgfqpoint{3.696000in}{3.696000in}}%
\pgfusepath{clip}%
\pgfsetrectcap%
\pgfsetroundjoin%
\pgfsetlinewidth{1.505625pt}%
\definecolor{currentstroke}{rgb}{1.000000,0.000000,0.000000}%
\pgfsetstrokecolor{currentstroke}%
\pgfsetdash{}{0pt}%
\pgfpathmoveto{\pgfqpoint{1.540130in}{1.867434in}}%
\pgfpathlineto{\pgfqpoint{1.479413in}{1.950647in}}%
\pgfusepath{stroke}%
\end{pgfscope}%
\begin{pgfscope}%
\pgfpathrectangle{\pgfqpoint{0.100000in}{0.220728in}}{\pgfqpoint{3.696000in}{3.696000in}}%
\pgfusepath{clip}%
\pgfsetrectcap%
\pgfsetroundjoin%
\pgfsetlinewidth{1.505625pt}%
\definecolor{currentstroke}{rgb}{1.000000,0.000000,0.000000}%
\pgfsetstrokecolor{currentstroke}%
\pgfsetdash{}{0pt}%
\pgfpathmoveto{\pgfqpoint{1.543274in}{1.868819in}}%
\pgfpathlineto{\pgfqpoint{1.479413in}{1.950647in}}%
\pgfusepath{stroke}%
\end{pgfscope}%
\begin{pgfscope}%
\pgfpathrectangle{\pgfqpoint{0.100000in}{0.220728in}}{\pgfqpoint{3.696000in}{3.696000in}}%
\pgfusepath{clip}%
\pgfsetrectcap%
\pgfsetroundjoin%
\pgfsetlinewidth{1.505625pt}%
\definecolor{currentstroke}{rgb}{1.000000,0.000000,0.000000}%
\pgfsetstrokecolor{currentstroke}%
\pgfsetdash{}{0pt}%
\pgfpathmoveto{\pgfqpoint{1.547186in}{1.869932in}}%
\pgfpathlineto{\pgfqpoint{1.488192in}{1.958072in}}%
\pgfusepath{stroke}%
\end{pgfscope}%
\begin{pgfscope}%
\pgfpathrectangle{\pgfqpoint{0.100000in}{0.220728in}}{\pgfqpoint{3.696000in}{3.696000in}}%
\pgfusepath{clip}%
\pgfsetrectcap%
\pgfsetroundjoin%
\pgfsetlinewidth{1.505625pt}%
\definecolor{currentstroke}{rgb}{1.000000,0.000000,0.000000}%
\pgfsetstrokecolor{currentstroke}%
\pgfsetdash{}{0pt}%
\pgfpathmoveto{\pgfqpoint{1.549667in}{1.870506in}}%
\pgfpathlineto{\pgfqpoint{1.488192in}{1.958072in}}%
\pgfusepath{stroke}%
\end{pgfscope}%
\begin{pgfscope}%
\pgfpathrectangle{\pgfqpoint{0.100000in}{0.220728in}}{\pgfqpoint{3.696000in}{3.696000in}}%
\pgfusepath{clip}%
\pgfsetrectcap%
\pgfsetroundjoin%
\pgfsetlinewidth{1.505625pt}%
\definecolor{currentstroke}{rgb}{1.000000,0.000000,0.000000}%
\pgfsetstrokecolor{currentstroke}%
\pgfsetdash{}{0pt}%
\pgfpathmoveto{\pgfqpoint{1.552667in}{1.872234in}}%
\pgfpathlineto{\pgfqpoint{1.488192in}{1.958072in}}%
\pgfusepath{stroke}%
\end{pgfscope}%
\begin{pgfscope}%
\pgfpathrectangle{\pgfqpoint{0.100000in}{0.220728in}}{\pgfqpoint{3.696000in}{3.696000in}}%
\pgfusepath{clip}%
\pgfsetrectcap%
\pgfsetroundjoin%
\pgfsetlinewidth{1.505625pt}%
\definecolor{currentstroke}{rgb}{1.000000,0.000000,0.000000}%
\pgfsetstrokecolor{currentstroke}%
\pgfsetdash{}{0pt}%
\pgfpathmoveto{\pgfqpoint{1.557232in}{1.872437in}}%
\pgfpathlineto{\pgfqpoint{1.496961in}{1.965487in}}%
\pgfusepath{stroke}%
\end{pgfscope}%
\begin{pgfscope}%
\pgfpathrectangle{\pgfqpoint{0.100000in}{0.220728in}}{\pgfqpoint{3.696000in}{3.696000in}}%
\pgfusepath{clip}%
\pgfsetrectcap%
\pgfsetroundjoin%
\pgfsetlinewidth{1.505625pt}%
\definecolor{currentstroke}{rgb}{1.000000,0.000000,0.000000}%
\pgfsetstrokecolor{currentstroke}%
\pgfsetdash{}{0pt}%
\pgfpathmoveto{\pgfqpoint{1.560704in}{1.876165in}}%
\pgfpathlineto{\pgfqpoint{1.496961in}{1.965487in}}%
\pgfusepath{stroke}%
\end{pgfscope}%
\begin{pgfscope}%
\pgfpathrectangle{\pgfqpoint{0.100000in}{0.220728in}}{\pgfqpoint{3.696000in}{3.696000in}}%
\pgfusepath{clip}%
\pgfsetrectcap%
\pgfsetroundjoin%
\pgfsetlinewidth{1.505625pt}%
\definecolor{currentstroke}{rgb}{1.000000,0.000000,0.000000}%
\pgfsetstrokecolor{currentstroke}%
\pgfsetdash{}{0pt}%
\pgfpathmoveto{\pgfqpoint{1.567635in}{1.880185in}}%
\pgfpathlineto{\pgfqpoint{1.496961in}{1.965487in}}%
\pgfusepath{stroke}%
\end{pgfscope}%
\begin{pgfscope}%
\pgfpathrectangle{\pgfqpoint{0.100000in}{0.220728in}}{\pgfqpoint{3.696000in}{3.696000in}}%
\pgfusepath{clip}%
\pgfsetrectcap%
\pgfsetroundjoin%
\pgfsetlinewidth{1.505625pt}%
\definecolor{currentstroke}{rgb}{1.000000,0.000000,0.000000}%
\pgfsetstrokecolor{currentstroke}%
\pgfsetdash{}{0pt}%
\pgfpathmoveto{\pgfqpoint{1.575400in}{1.878458in}}%
\pgfpathlineto{\pgfqpoint{1.505718in}{1.972894in}}%
\pgfusepath{stroke}%
\end{pgfscope}%
\begin{pgfscope}%
\pgfpathrectangle{\pgfqpoint{0.100000in}{0.220728in}}{\pgfqpoint{3.696000in}{3.696000in}}%
\pgfusepath{clip}%
\pgfsetrectcap%
\pgfsetroundjoin%
\pgfsetlinewidth{1.505625pt}%
\definecolor{currentstroke}{rgb}{1.000000,0.000000,0.000000}%
\pgfsetstrokecolor{currentstroke}%
\pgfsetdash{}{0pt}%
\pgfpathmoveto{\pgfqpoint{1.580538in}{1.883962in}}%
\pgfpathlineto{\pgfqpoint{1.514465in}{1.980291in}}%
\pgfusepath{stroke}%
\end{pgfscope}%
\begin{pgfscope}%
\pgfpathrectangle{\pgfqpoint{0.100000in}{0.220728in}}{\pgfqpoint{3.696000in}{3.696000in}}%
\pgfusepath{clip}%
\pgfsetrectcap%
\pgfsetroundjoin%
\pgfsetlinewidth{1.505625pt}%
\definecolor{currentstroke}{rgb}{1.000000,0.000000,0.000000}%
\pgfsetstrokecolor{currentstroke}%
\pgfsetdash{}{0pt}%
\pgfpathmoveto{\pgfqpoint{1.589172in}{1.887966in}}%
\pgfpathlineto{\pgfqpoint{1.514465in}{1.980291in}}%
\pgfusepath{stroke}%
\end{pgfscope}%
\begin{pgfscope}%
\pgfpathrectangle{\pgfqpoint{0.100000in}{0.220728in}}{\pgfqpoint{3.696000in}{3.696000in}}%
\pgfusepath{clip}%
\pgfsetrectcap%
\pgfsetroundjoin%
\pgfsetlinewidth{1.505625pt}%
\definecolor{currentstroke}{rgb}{1.000000,0.000000,0.000000}%
\pgfsetstrokecolor{currentstroke}%
\pgfsetdash{}{0pt}%
\pgfpathmoveto{\pgfqpoint{1.598023in}{1.886884in}}%
\pgfpathlineto{\pgfqpoint{1.523200in}{1.987678in}}%
\pgfusepath{stroke}%
\end{pgfscope}%
\begin{pgfscope}%
\pgfpathrectangle{\pgfqpoint{0.100000in}{0.220728in}}{\pgfqpoint{3.696000in}{3.696000in}}%
\pgfusepath{clip}%
\pgfsetrectcap%
\pgfsetroundjoin%
\pgfsetlinewidth{1.505625pt}%
\definecolor{currentstroke}{rgb}{1.000000,0.000000,0.000000}%
\pgfsetstrokecolor{currentstroke}%
\pgfsetdash{}{0pt}%
\pgfpathmoveto{\pgfqpoint{1.604664in}{1.890995in}}%
\pgfpathlineto{\pgfqpoint{1.531924in}{1.995056in}}%
\pgfusepath{stroke}%
\end{pgfscope}%
\begin{pgfscope}%
\pgfpathrectangle{\pgfqpoint{0.100000in}{0.220728in}}{\pgfqpoint{3.696000in}{3.696000in}}%
\pgfusepath{clip}%
\pgfsetrectcap%
\pgfsetroundjoin%
\pgfsetlinewidth{1.505625pt}%
\definecolor{currentstroke}{rgb}{1.000000,0.000000,0.000000}%
\pgfsetstrokecolor{currentstroke}%
\pgfsetdash{}{0pt}%
\pgfpathmoveto{\pgfqpoint{1.614589in}{1.898054in}}%
\pgfpathlineto{\pgfqpoint{1.540637in}{2.002425in}}%
\pgfusepath{stroke}%
\end{pgfscope}%
\begin{pgfscope}%
\pgfpathrectangle{\pgfqpoint{0.100000in}{0.220728in}}{\pgfqpoint{3.696000in}{3.696000in}}%
\pgfusepath{clip}%
\pgfsetrectcap%
\pgfsetroundjoin%
\pgfsetlinewidth{1.505625pt}%
\definecolor{currentstroke}{rgb}{1.000000,0.000000,0.000000}%
\pgfsetstrokecolor{currentstroke}%
\pgfsetdash{}{0pt}%
\pgfpathmoveto{\pgfqpoint{1.625033in}{1.896458in}}%
\pgfpathlineto{\pgfqpoint{1.549340in}{2.009785in}}%
\pgfusepath{stroke}%
\end{pgfscope}%
\begin{pgfscope}%
\pgfpathrectangle{\pgfqpoint{0.100000in}{0.220728in}}{\pgfqpoint{3.696000in}{3.696000in}}%
\pgfusepath{clip}%
\pgfsetrectcap%
\pgfsetroundjoin%
\pgfsetlinewidth{1.505625pt}%
\definecolor{currentstroke}{rgb}{1.000000,0.000000,0.000000}%
\pgfsetstrokecolor{currentstroke}%
\pgfsetdash{}{0pt}%
\pgfpathmoveto{\pgfqpoint{1.631726in}{1.900868in}}%
\pgfpathlineto{\pgfqpoint{1.558031in}{2.017135in}}%
\pgfusepath{stroke}%
\end{pgfscope}%
\begin{pgfscope}%
\pgfpathrectangle{\pgfqpoint{0.100000in}{0.220728in}}{\pgfqpoint{3.696000in}{3.696000in}}%
\pgfusepath{clip}%
\pgfsetrectcap%
\pgfsetroundjoin%
\pgfsetlinewidth{1.505625pt}%
\definecolor{currentstroke}{rgb}{1.000000,0.000000,0.000000}%
\pgfsetstrokecolor{currentstroke}%
\pgfsetdash{}{0pt}%
\pgfpathmoveto{\pgfqpoint{1.635309in}{1.904181in}}%
\pgfpathlineto{\pgfqpoint{1.558031in}{2.017135in}}%
\pgfusepath{stroke}%
\end{pgfscope}%
\begin{pgfscope}%
\pgfpathrectangle{\pgfqpoint{0.100000in}{0.220728in}}{\pgfqpoint{3.696000in}{3.696000in}}%
\pgfusepath{clip}%
\pgfsetrectcap%
\pgfsetroundjoin%
\pgfsetlinewidth{1.505625pt}%
\definecolor{currentstroke}{rgb}{1.000000,0.000000,0.000000}%
\pgfsetstrokecolor{currentstroke}%
\pgfsetdash{}{0pt}%
\pgfpathmoveto{\pgfqpoint{1.641416in}{1.907959in}}%
\pgfpathlineto{\pgfqpoint{1.566711in}{2.024476in}}%
\pgfusepath{stroke}%
\end{pgfscope}%
\begin{pgfscope}%
\pgfpathrectangle{\pgfqpoint{0.100000in}{0.220728in}}{\pgfqpoint{3.696000in}{3.696000in}}%
\pgfusepath{clip}%
\pgfsetrectcap%
\pgfsetroundjoin%
\pgfsetlinewidth{1.505625pt}%
\definecolor{currentstroke}{rgb}{1.000000,0.000000,0.000000}%
\pgfsetstrokecolor{currentstroke}%
\pgfsetdash{}{0pt}%
\pgfpathmoveto{\pgfqpoint{1.645501in}{1.908439in}}%
\pgfpathlineto{\pgfqpoint{1.566711in}{2.024476in}}%
\pgfusepath{stroke}%
\end{pgfscope}%
\begin{pgfscope}%
\pgfpathrectangle{\pgfqpoint{0.100000in}{0.220728in}}{\pgfqpoint{3.696000in}{3.696000in}}%
\pgfusepath{clip}%
\pgfsetrectcap%
\pgfsetroundjoin%
\pgfsetlinewidth{1.505625pt}%
\definecolor{currentstroke}{rgb}{1.000000,0.000000,0.000000}%
\pgfsetstrokecolor{currentstroke}%
\pgfsetdash{}{0pt}%
\pgfpathmoveto{\pgfqpoint{1.650340in}{1.912503in}}%
\pgfpathlineto{\pgfqpoint{1.566711in}{2.024476in}}%
\pgfusepath{stroke}%
\end{pgfscope}%
\begin{pgfscope}%
\pgfpathrectangle{\pgfqpoint{0.100000in}{0.220728in}}{\pgfqpoint{3.696000in}{3.696000in}}%
\pgfusepath{clip}%
\pgfsetrectcap%
\pgfsetroundjoin%
\pgfsetlinewidth{1.505625pt}%
\definecolor{currentstroke}{rgb}{1.000000,0.000000,0.000000}%
\pgfsetstrokecolor{currentstroke}%
\pgfsetdash{}{0pt}%
\pgfpathmoveto{\pgfqpoint{1.656350in}{1.913668in}}%
\pgfpathlineto{\pgfqpoint{1.575380in}{2.031808in}}%
\pgfusepath{stroke}%
\end{pgfscope}%
\begin{pgfscope}%
\pgfpathrectangle{\pgfqpoint{0.100000in}{0.220728in}}{\pgfqpoint{3.696000in}{3.696000in}}%
\pgfusepath{clip}%
\pgfsetrectcap%
\pgfsetroundjoin%
\pgfsetlinewidth{1.505625pt}%
\definecolor{currentstroke}{rgb}{1.000000,0.000000,0.000000}%
\pgfsetstrokecolor{currentstroke}%
\pgfsetdash{}{0pt}%
\pgfpathmoveto{\pgfqpoint{1.659197in}{1.913844in}}%
\pgfpathlineto{\pgfqpoint{1.575380in}{2.031808in}}%
\pgfusepath{stroke}%
\end{pgfscope}%
\begin{pgfscope}%
\pgfpathrectangle{\pgfqpoint{0.100000in}{0.220728in}}{\pgfqpoint{3.696000in}{3.696000in}}%
\pgfusepath{clip}%
\pgfsetrectcap%
\pgfsetroundjoin%
\pgfsetlinewidth{1.505625pt}%
\definecolor{currentstroke}{rgb}{1.000000,0.000000,0.000000}%
\pgfsetstrokecolor{currentstroke}%
\pgfsetdash{}{0pt}%
\pgfpathmoveto{\pgfqpoint{1.664162in}{1.916652in}}%
\pgfpathlineto{\pgfqpoint{1.584039in}{2.039130in}}%
\pgfusepath{stroke}%
\end{pgfscope}%
\begin{pgfscope}%
\pgfpathrectangle{\pgfqpoint{0.100000in}{0.220728in}}{\pgfqpoint{3.696000in}{3.696000in}}%
\pgfusepath{clip}%
\pgfsetrectcap%
\pgfsetroundjoin%
\pgfsetlinewidth{1.505625pt}%
\definecolor{currentstroke}{rgb}{1.000000,0.000000,0.000000}%
\pgfsetstrokecolor{currentstroke}%
\pgfsetdash{}{0pt}%
\pgfpathmoveto{\pgfqpoint{1.669587in}{1.921097in}}%
\pgfpathlineto{\pgfqpoint{1.584039in}{2.039130in}}%
\pgfusepath{stroke}%
\end{pgfscope}%
\begin{pgfscope}%
\pgfpathrectangle{\pgfqpoint{0.100000in}{0.220728in}}{\pgfqpoint{3.696000in}{3.696000in}}%
\pgfusepath{clip}%
\pgfsetrectcap%
\pgfsetroundjoin%
\pgfsetlinewidth{1.505625pt}%
\definecolor{currentstroke}{rgb}{1.000000,0.000000,0.000000}%
\pgfsetstrokecolor{currentstroke}%
\pgfsetdash{}{0pt}%
\pgfpathmoveto{\pgfqpoint{1.673323in}{1.924049in}}%
\pgfpathlineto{\pgfqpoint{1.592686in}{2.046444in}}%
\pgfusepath{stroke}%
\end{pgfscope}%
\begin{pgfscope}%
\pgfpathrectangle{\pgfqpoint{0.100000in}{0.220728in}}{\pgfqpoint{3.696000in}{3.696000in}}%
\pgfusepath{clip}%
\pgfsetrectcap%
\pgfsetroundjoin%
\pgfsetlinewidth{1.505625pt}%
\definecolor{currentstroke}{rgb}{1.000000,0.000000,0.000000}%
\pgfsetstrokecolor{currentstroke}%
\pgfsetdash{}{0pt}%
\pgfpathmoveto{\pgfqpoint{1.675596in}{1.925537in}}%
\pgfpathlineto{\pgfqpoint{1.592686in}{2.046444in}}%
\pgfusepath{stroke}%
\end{pgfscope}%
\begin{pgfscope}%
\pgfpathrectangle{\pgfqpoint{0.100000in}{0.220728in}}{\pgfqpoint{3.696000in}{3.696000in}}%
\pgfusepath{clip}%
\pgfsetrectcap%
\pgfsetroundjoin%
\pgfsetlinewidth{1.505625pt}%
\definecolor{currentstroke}{rgb}{1.000000,0.000000,0.000000}%
\pgfsetstrokecolor{currentstroke}%
\pgfsetdash{}{0pt}%
\pgfpathmoveto{\pgfqpoint{1.679244in}{1.926410in}}%
\pgfpathlineto{\pgfqpoint{1.592686in}{2.046444in}}%
\pgfusepath{stroke}%
\end{pgfscope}%
\begin{pgfscope}%
\pgfpathrectangle{\pgfqpoint{0.100000in}{0.220728in}}{\pgfqpoint{3.696000in}{3.696000in}}%
\pgfusepath{clip}%
\pgfsetrectcap%
\pgfsetroundjoin%
\pgfsetlinewidth{1.505625pt}%
\definecolor{currentstroke}{rgb}{1.000000,0.000000,0.000000}%
\pgfsetstrokecolor{currentstroke}%
\pgfsetdash{}{0pt}%
\pgfpathmoveto{\pgfqpoint{1.681783in}{1.928629in}}%
\pgfpathlineto{\pgfqpoint{1.601323in}{2.053748in}}%
\pgfusepath{stroke}%
\end{pgfscope}%
\begin{pgfscope}%
\pgfpathrectangle{\pgfqpoint{0.100000in}{0.220728in}}{\pgfqpoint{3.696000in}{3.696000in}}%
\pgfusepath{clip}%
\pgfsetrectcap%
\pgfsetroundjoin%
\pgfsetlinewidth{1.505625pt}%
\definecolor{currentstroke}{rgb}{1.000000,0.000000,0.000000}%
\pgfsetstrokecolor{currentstroke}%
\pgfsetdash{}{0pt}%
\pgfpathmoveto{\pgfqpoint{1.685058in}{1.931856in}}%
\pgfpathlineto{\pgfqpoint{1.601323in}{2.053748in}}%
\pgfusepath{stroke}%
\end{pgfscope}%
\begin{pgfscope}%
\pgfpathrectangle{\pgfqpoint{0.100000in}{0.220728in}}{\pgfqpoint{3.696000in}{3.696000in}}%
\pgfusepath{clip}%
\pgfsetrectcap%
\pgfsetroundjoin%
\pgfsetlinewidth{1.505625pt}%
\definecolor{currentstroke}{rgb}{1.000000,0.000000,0.000000}%
\pgfsetstrokecolor{currentstroke}%
\pgfsetdash{}{0pt}%
\pgfpathmoveto{\pgfqpoint{1.689212in}{1.932294in}}%
\pgfpathlineto{\pgfqpoint{1.601323in}{2.053748in}}%
\pgfusepath{stroke}%
\end{pgfscope}%
\begin{pgfscope}%
\pgfpathrectangle{\pgfqpoint{0.100000in}{0.220728in}}{\pgfqpoint{3.696000in}{3.696000in}}%
\pgfusepath{clip}%
\pgfsetrectcap%
\pgfsetroundjoin%
\pgfsetlinewidth{1.505625pt}%
\definecolor{currentstroke}{rgb}{1.000000,0.000000,0.000000}%
\pgfsetstrokecolor{currentstroke}%
\pgfsetdash{}{0pt}%
\pgfpathmoveto{\pgfqpoint{1.695785in}{1.935960in}}%
\pgfpathlineto{\pgfqpoint{1.609948in}{2.061043in}}%
\pgfusepath{stroke}%
\end{pgfscope}%
\begin{pgfscope}%
\pgfpathrectangle{\pgfqpoint{0.100000in}{0.220728in}}{\pgfqpoint{3.696000in}{3.696000in}}%
\pgfusepath{clip}%
\pgfsetrectcap%
\pgfsetroundjoin%
\pgfsetlinewidth{1.505625pt}%
\definecolor{currentstroke}{rgb}{1.000000,0.000000,0.000000}%
\pgfsetstrokecolor{currentstroke}%
\pgfsetdash{}{0pt}%
\pgfpathmoveto{\pgfqpoint{1.702842in}{1.942523in}}%
\pgfpathlineto{\pgfqpoint{1.618563in}{2.068328in}}%
\pgfusepath{stroke}%
\end{pgfscope}%
\begin{pgfscope}%
\pgfpathrectangle{\pgfqpoint{0.100000in}{0.220728in}}{\pgfqpoint{3.696000in}{3.696000in}}%
\pgfusepath{clip}%
\pgfsetrectcap%
\pgfsetroundjoin%
\pgfsetlinewidth{1.505625pt}%
\definecolor{currentstroke}{rgb}{1.000000,0.000000,0.000000}%
\pgfsetstrokecolor{currentstroke}%
\pgfsetdash{}{0pt}%
\pgfpathmoveto{\pgfqpoint{1.708694in}{1.940488in}}%
\pgfpathlineto{\pgfqpoint{1.618563in}{2.068328in}}%
\pgfusepath{stroke}%
\end{pgfscope}%
\begin{pgfscope}%
\pgfpathrectangle{\pgfqpoint{0.100000in}{0.220728in}}{\pgfqpoint{3.696000in}{3.696000in}}%
\pgfusepath{clip}%
\pgfsetrectcap%
\pgfsetroundjoin%
\pgfsetlinewidth{1.505625pt}%
\definecolor{currentstroke}{rgb}{1.000000,0.000000,0.000000}%
\pgfsetstrokecolor{currentstroke}%
\pgfsetdash{}{0pt}%
\pgfpathmoveto{\pgfqpoint{1.717553in}{1.946525in}}%
\pgfpathlineto{\pgfqpoint{1.627167in}{2.075605in}}%
\pgfusepath{stroke}%
\end{pgfscope}%
\begin{pgfscope}%
\pgfpathrectangle{\pgfqpoint{0.100000in}{0.220728in}}{\pgfqpoint{3.696000in}{3.696000in}}%
\pgfusepath{clip}%
\pgfsetrectcap%
\pgfsetroundjoin%
\pgfsetlinewidth{1.505625pt}%
\definecolor{currentstroke}{rgb}{1.000000,0.000000,0.000000}%
\pgfsetstrokecolor{currentstroke}%
\pgfsetdash{}{0pt}%
\pgfpathmoveto{\pgfqpoint{1.725939in}{1.949878in}}%
\pgfpathlineto{\pgfqpoint{1.635761in}{2.082872in}}%
\pgfusepath{stroke}%
\end{pgfscope}%
\begin{pgfscope}%
\pgfpathrectangle{\pgfqpoint{0.100000in}{0.220728in}}{\pgfqpoint{3.696000in}{3.696000in}}%
\pgfusepath{clip}%
\pgfsetrectcap%
\pgfsetroundjoin%
\pgfsetlinewidth{1.505625pt}%
\definecolor{currentstroke}{rgb}{1.000000,0.000000,0.000000}%
\pgfsetstrokecolor{currentstroke}%
\pgfsetdash{}{0pt}%
\pgfpathmoveto{\pgfqpoint{1.730406in}{1.952185in}}%
\pgfpathlineto{\pgfqpoint{1.644343in}{2.090131in}}%
\pgfusepath{stroke}%
\end{pgfscope}%
\begin{pgfscope}%
\pgfpathrectangle{\pgfqpoint{0.100000in}{0.220728in}}{\pgfqpoint{3.696000in}{3.696000in}}%
\pgfusepath{clip}%
\pgfsetrectcap%
\pgfsetroundjoin%
\pgfsetlinewidth{1.505625pt}%
\definecolor{currentstroke}{rgb}{1.000000,0.000000,0.000000}%
\pgfsetstrokecolor{currentstroke}%
\pgfsetdash{}{0pt}%
\pgfpathmoveto{\pgfqpoint{1.736954in}{1.953921in}}%
\pgfpathlineto{\pgfqpoint{1.644343in}{2.090131in}}%
\pgfusepath{stroke}%
\end{pgfscope}%
\begin{pgfscope}%
\pgfpathrectangle{\pgfqpoint{0.100000in}{0.220728in}}{\pgfqpoint{3.696000in}{3.696000in}}%
\pgfusepath{clip}%
\pgfsetrectcap%
\pgfsetroundjoin%
\pgfsetlinewidth{1.505625pt}%
\definecolor{currentstroke}{rgb}{1.000000,0.000000,0.000000}%
\pgfsetstrokecolor{currentstroke}%
\pgfsetdash{}{0pt}%
\pgfpathmoveto{\pgfqpoint{1.744053in}{1.957031in}}%
\pgfpathlineto{\pgfqpoint{1.652915in}{2.097380in}}%
\pgfusepath{stroke}%
\end{pgfscope}%
\begin{pgfscope}%
\pgfpathrectangle{\pgfqpoint{0.100000in}{0.220728in}}{\pgfqpoint{3.696000in}{3.696000in}}%
\pgfusepath{clip}%
\pgfsetrectcap%
\pgfsetroundjoin%
\pgfsetlinewidth{1.505625pt}%
\definecolor{currentstroke}{rgb}{1.000000,0.000000,0.000000}%
\pgfsetstrokecolor{currentstroke}%
\pgfsetdash{}{0pt}%
\pgfpathmoveto{\pgfqpoint{1.748243in}{1.958888in}}%
\pgfpathlineto{\pgfqpoint{1.652915in}{2.097380in}}%
\pgfusepath{stroke}%
\end{pgfscope}%
\begin{pgfscope}%
\pgfpathrectangle{\pgfqpoint{0.100000in}{0.220728in}}{\pgfqpoint{3.696000in}{3.696000in}}%
\pgfusepath{clip}%
\pgfsetrectcap%
\pgfsetroundjoin%
\pgfsetlinewidth{1.505625pt}%
\definecolor{currentstroke}{rgb}{1.000000,0.000000,0.000000}%
\pgfsetstrokecolor{currentstroke}%
\pgfsetdash{}{0pt}%
\pgfpathmoveto{\pgfqpoint{1.753328in}{1.961718in}}%
\pgfpathlineto{\pgfqpoint{1.661476in}{2.104620in}}%
\pgfusepath{stroke}%
\end{pgfscope}%
\begin{pgfscope}%
\pgfpathrectangle{\pgfqpoint{0.100000in}{0.220728in}}{\pgfqpoint{3.696000in}{3.696000in}}%
\pgfusepath{clip}%
\pgfsetrectcap%
\pgfsetroundjoin%
\pgfsetlinewidth{1.505625pt}%
\definecolor{currentstroke}{rgb}{1.000000,0.000000,0.000000}%
\pgfsetstrokecolor{currentstroke}%
\pgfsetdash{}{0pt}%
\pgfpathmoveto{\pgfqpoint{1.759566in}{1.963445in}}%
\pgfpathlineto{\pgfqpoint{1.661476in}{2.104620in}}%
\pgfusepath{stroke}%
\end{pgfscope}%
\begin{pgfscope}%
\pgfpathrectangle{\pgfqpoint{0.100000in}{0.220728in}}{\pgfqpoint{3.696000in}{3.696000in}}%
\pgfusepath{clip}%
\pgfsetrectcap%
\pgfsetroundjoin%
\pgfsetlinewidth{1.505625pt}%
\definecolor{currentstroke}{rgb}{1.000000,0.000000,0.000000}%
\pgfsetstrokecolor{currentstroke}%
\pgfsetdash{}{0pt}%
\pgfpathmoveto{\pgfqpoint{1.767351in}{1.971511in}}%
\pgfpathlineto{\pgfqpoint{1.670026in}{2.111851in}}%
\pgfusepath{stroke}%
\end{pgfscope}%
\begin{pgfscope}%
\pgfpathrectangle{\pgfqpoint{0.100000in}{0.220728in}}{\pgfqpoint{3.696000in}{3.696000in}}%
\pgfusepath{clip}%
\pgfsetrectcap%
\pgfsetroundjoin%
\pgfsetlinewidth{1.505625pt}%
\definecolor{currentstroke}{rgb}{1.000000,0.000000,0.000000}%
\pgfsetstrokecolor{currentstroke}%
\pgfsetdash{}{0pt}%
\pgfpathmoveto{\pgfqpoint{1.771818in}{1.974868in}}%
\pgfpathlineto{\pgfqpoint{1.678566in}{2.119073in}}%
\pgfusepath{stroke}%
\end{pgfscope}%
\begin{pgfscope}%
\pgfpathrectangle{\pgfqpoint{0.100000in}{0.220728in}}{\pgfqpoint{3.696000in}{3.696000in}}%
\pgfusepath{clip}%
\pgfsetrectcap%
\pgfsetroundjoin%
\pgfsetlinewidth{1.505625pt}%
\definecolor{currentstroke}{rgb}{1.000000,0.000000,0.000000}%
\pgfsetstrokecolor{currentstroke}%
\pgfsetdash{}{0pt}%
\pgfpathmoveto{\pgfqpoint{1.775567in}{1.974608in}}%
\pgfpathlineto{\pgfqpoint{1.678566in}{2.119073in}}%
\pgfusepath{stroke}%
\end{pgfscope}%
\begin{pgfscope}%
\pgfpathrectangle{\pgfqpoint{0.100000in}{0.220728in}}{\pgfqpoint{3.696000in}{3.696000in}}%
\pgfusepath{clip}%
\pgfsetrectcap%
\pgfsetroundjoin%
\pgfsetlinewidth{1.505625pt}%
\definecolor{currentstroke}{rgb}{1.000000,0.000000,0.000000}%
\pgfsetstrokecolor{currentstroke}%
\pgfsetdash{}{0pt}%
\pgfpathmoveto{\pgfqpoint{1.783806in}{1.985677in}}%
\pgfpathlineto{\pgfqpoint{1.687094in}{2.126286in}}%
\pgfusepath{stroke}%
\end{pgfscope}%
\begin{pgfscope}%
\pgfpathrectangle{\pgfqpoint{0.100000in}{0.220728in}}{\pgfqpoint{3.696000in}{3.696000in}}%
\pgfusepath{clip}%
\pgfsetrectcap%
\pgfsetroundjoin%
\pgfsetlinewidth{1.505625pt}%
\definecolor{currentstroke}{rgb}{1.000000,0.000000,0.000000}%
\pgfsetstrokecolor{currentstroke}%
\pgfsetdash{}{0pt}%
\pgfpathmoveto{\pgfqpoint{1.788171in}{1.988386in}}%
\pgfpathlineto{\pgfqpoint{1.687094in}{2.126286in}}%
\pgfusepath{stroke}%
\end{pgfscope}%
\begin{pgfscope}%
\pgfpathrectangle{\pgfqpoint{0.100000in}{0.220728in}}{\pgfqpoint{3.696000in}{3.696000in}}%
\pgfusepath{clip}%
\pgfsetrectcap%
\pgfsetroundjoin%
\pgfsetlinewidth{1.505625pt}%
\definecolor{currentstroke}{rgb}{1.000000,0.000000,0.000000}%
\pgfsetstrokecolor{currentstroke}%
\pgfsetdash{}{0pt}%
\pgfpathmoveto{\pgfqpoint{1.789883in}{1.989889in}}%
\pgfpathlineto{\pgfqpoint{1.687094in}{2.126286in}}%
\pgfusepath{stroke}%
\end{pgfscope}%
\begin{pgfscope}%
\pgfpathrectangle{\pgfqpoint{0.100000in}{0.220728in}}{\pgfqpoint{3.696000in}{3.696000in}}%
\pgfusepath{clip}%
\pgfsetrectcap%
\pgfsetroundjoin%
\pgfsetlinewidth{1.505625pt}%
\definecolor{currentstroke}{rgb}{1.000000,0.000000,0.000000}%
\pgfsetstrokecolor{currentstroke}%
\pgfsetdash{}{0pt}%
\pgfpathmoveto{\pgfqpoint{1.793173in}{1.991604in}}%
\pgfpathlineto{\pgfqpoint{1.687094in}{2.126286in}}%
\pgfusepath{stroke}%
\end{pgfscope}%
\begin{pgfscope}%
\pgfpathrectangle{\pgfqpoint{0.100000in}{0.220728in}}{\pgfqpoint{3.696000in}{3.696000in}}%
\pgfusepath{clip}%
\pgfsetrectcap%
\pgfsetroundjoin%
\pgfsetlinewidth{1.505625pt}%
\definecolor{currentstroke}{rgb}{1.000000,0.000000,0.000000}%
\pgfsetstrokecolor{currentstroke}%
\pgfsetdash{}{0pt}%
\pgfpathmoveto{\pgfqpoint{1.799501in}{1.999186in}}%
\pgfpathlineto{\pgfqpoint{1.695613in}{2.133490in}}%
\pgfusepath{stroke}%
\end{pgfscope}%
\begin{pgfscope}%
\pgfpathrectangle{\pgfqpoint{0.100000in}{0.220728in}}{\pgfqpoint{3.696000in}{3.696000in}}%
\pgfusepath{clip}%
\pgfsetrectcap%
\pgfsetroundjoin%
\pgfsetlinewidth{1.505625pt}%
\definecolor{currentstroke}{rgb}{1.000000,0.000000,0.000000}%
\pgfsetstrokecolor{currentstroke}%
\pgfsetdash{}{0pt}%
\pgfpathmoveto{\pgfqpoint{1.807253in}{2.002868in}}%
\pgfpathlineto{\pgfqpoint{1.704120in}{2.140685in}}%
\pgfusepath{stroke}%
\end{pgfscope}%
\begin{pgfscope}%
\pgfpathrectangle{\pgfqpoint{0.100000in}{0.220728in}}{\pgfqpoint{3.696000in}{3.696000in}}%
\pgfusepath{clip}%
\pgfsetrectcap%
\pgfsetroundjoin%
\pgfsetlinewidth{1.505625pt}%
\definecolor{currentstroke}{rgb}{1.000000,0.000000,0.000000}%
\pgfsetstrokecolor{currentstroke}%
\pgfsetdash{}{0pt}%
\pgfpathmoveto{\pgfqpoint{1.812166in}{2.004966in}}%
\pgfpathlineto{\pgfqpoint{1.876050in}{2.167865in}}%
\pgfusepath{stroke}%
\end{pgfscope}%
\begin{pgfscope}%
\pgfpathrectangle{\pgfqpoint{0.100000in}{0.220728in}}{\pgfqpoint{3.696000in}{3.696000in}}%
\pgfusepath{clip}%
\pgfsetrectcap%
\pgfsetroundjoin%
\pgfsetlinewidth{1.505625pt}%
\definecolor{currentstroke}{rgb}{1.000000,0.000000,0.000000}%
\pgfsetstrokecolor{currentstroke}%
\pgfsetdash{}{0pt}%
\pgfpathmoveto{\pgfqpoint{1.815151in}{2.006927in}}%
\pgfpathlineto{\pgfqpoint{1.876050in}{2.167865in}}%
\pgfusepath{stroke}%
\end{pgfscope}%
\begin{pgfscope}%
\pgfpathrectangle{\pgfqpoint{0.100000in}{0.220728in}}{\pgfqpoint{3.696000in}{3.696000in}}%
\pgfusepath{clip}%
\pgfsetrectcap%
\pgfsetroundjoin%
\pgfsetlinewidth{1.505625pt}%
\definecolor{currentstroke}{rgb}{1.000000,0.000000,0.000000}%
\pgfsetstrokecolor{currentstroke}%
\pgfsetdash{}{0pt}%
\pgfpathmoveto{\pgfqpoint{1.819047in}{2.008879in}}%
\pgfpathlineto{\pgfqpoint{1.876050in}{2.167865in}}%
\pgfusepath{stroke}%
\end{pgfscope}%
\begin{pgfscope}%
\pgfpathrectangle{\pgfqpoint{0.100000in}{0.220728in}}{\pgfqpoint{3.696000in}{3.696000in}}%
\pgfusepath{clip}%
\pgfsetrectcap%
\pgfsetroundjoin%
\pgfsetlinewidth{1.505625pt}%
\definecolor{currentstroke}{rgb}{1.000000,0.000000,0.000000}%
\pgfsetstrokecolor{currentstroke}%
\pgfsetdash{}{0pt}%
\pgfpathmoveto{\pgfqpoint{1.821664in}{2.010879in}}%
\pgfpathlineto{\pgfqpoint{1.876050in}{2.167865in}}%
\pgfusepath{stroke}%
\end{pgfscope}%
\begin{pgfscope}%
\pgfpathrectangle{\pgfqpoint{0.100000in}{0.220728in}}{\pgfqpoint{3.696000in}{3.696000in}}%
\pgfusepath{clip}%
\pgfsetrectcap%
\pgfsetroundjoin%
\pgfsetlinewidth{1.505625pt}%
\definecolor{currentstroke}{rgb}{1.000000,0.000000,0.000000}%
\pgfsetstrokecolor{currentstroke}%
\pgfsetdash{}{0pt}%
\pgfpathmoveto{\pgfqpoint{1.825639in}{2.013351in}}%
\pgfpathlineto{\pgfqpoint{1.876050in}{2.167865in}}%
\pgfusepath{stroke}%
\end{pgfscope}%
\begin{pgfscope}%
\pgfpathrectangle{\pgfqpoint{0.100000in}{0.220728in}}{\pgfqpoint{3.696000in}{3.696000in}}%
\pgfusepath{clip}%
\pgfsetrectcap%
\pgfsetroundjoin%
\pgfsetlinewidth{1.505625pt}%
\definecolor{currentstroke}{rgb}{1.000000,0.000000,0.000000}%
\pgfsetstrokecolor{currentstroke}%
\pgfsetdash{}{0pt}%
\pgfpathmoveto{\pgfqpoint{1.829734in}{2.013836in}}%
\pgfpathlineto{\pgfqpoint{1.876050in}{2.167865in}}%
\pgfusepath{stroke}%
\end{pgfscope}%
\begin{pgfscope}%
\pgfpathrectangle{\pgfqpoint{0.100000in}{0.220728in}}{\pgfqpoint{3.696000in}{3.696000in}}%
\pgfusepath{clip}%
\pgfsetrectcap%
\pgfsetroundjoin%
\pgfsetlinewidth{1.505625pt}%
\definecolor{currentstroke}{rgb}{1.000000,0.000000,0.000000}%
\pgfsetstrokecolor{currentstroke}%
\pgfsetdash{}{0pt}%
\pgfpathmoveto{\pgfqpoint{1.832781in}{2.015099in}}%
\pgfpathlineto{\pgfqpoint{1.876050in}{2.167865in}}%
\pgfusepath{stroke}%
\end{pgfscope}%
\begin{pgfscope}%
\pgfpathrectangle{\pgfqpoint{0.100000in}{0.220728in}}{\pgfqpoint{3.696000in}{3.696000in}}%
\pgfusepath{clip}%
\pgfsetrectcap%
\pgfsetroundjoin%
\pgfsetlinewidth{1.505625pt}%
\definecolor{currentstroke}{rgb}{1.000000,0.000000,0.000000}%
\pgfsetstrokecolor{currentstroke}%
\pgfsetdash{}{0pt}%
\pgfpathmoveto{\pgfqpoint{1.836431in}{2.018042in}}%
\pgfpathlineto{\pgfqpoint{1.876050in}{2.167865in}}%
\pgfusepath{stroke}%
\end{pgfscope}%
\begin{pgfscope}%
\pgfpathrectangle{\pgfqpoint{0.100000in}{0.220728in}}{\pgfqpoint{3.696000in}{3.696000in}}%
\pgfusepath{clip}%
\pgfsetrectcap%
\pgfsetroundjoin%
\pgfsetlinewidth{1.505625pt}%
\definecolor{currentstroke}{rgb}{1.000000,0.000000,0.000000}%
\pgfsetstrokecolor{currentstroke}%
\pgfsetdash{}{0pt}%
\pgfpathmoveto{\pgfqpoint{1.840946in}{2.018257in}}%
\pgfpathlineto{\pgfqpoint{1.876050in}{2.167865in}}%
\pgfusepath{stroke}%
\end{pgfscope}%
\begin{pgfscope}%
\pgfpathrectangle{\pgfqpoint{0.100000in}{0.220728in}}{\pgfqpoint{3.696000in}{3.696000in}}%
\pgfusepath{clip}%
\pgfsetrectcap%
\pgfsetroundjoin%
\pgfsetlinewidth{1.505625pt}%
\definecolor{currentstroke}{rgb}{1.000000,0.000000,0.000000}%
\pgfsetstrokecolor{currentstroke}%
\pgfsetdash{}{0pt}%
\pgfpathmoveto{\pgfqpoint{1.843704in}{2.020427in}}%
\pgfpathlineto{\pgfqpoint{1.876050in}{2.167865in}}%
\pgfusepath{stroke}%
\end{pgfscope}%
\begin{pgfscope}%
\pgfpathrectangle{\pgfqpoint{0.100000in}{0.220728in}}{\pgfqpoint{3.696000in}{3.696000in}}%
\pgfusepath{clip}%
\pgfsetrectcap%
\pgfsetroundjoin%
\pgfsetlinewidth{1.505625pt}%
\definecolor{currentstroke}{rgb}{1.000000,0.000000,0.000000}%
\pgfsetstrokecolor{currentstroke}%
\pgfsetdash{}{0pt}%
\pgfpathmoveto{\pgfqpoint{1.847604in}{2.021198in}}%
\pgfpathlineto{\pgfqpoint{1.876050in}{2.167865in}}%
\pgfusepath{stroke}%
\end{pgfscope}%
\begin{pgfscope}%
\pgfpathrectangle{\pgfqpoint{0.100000in}{0.220728in}}{\pgfqpoint{3.696000in}{3.696000in}}%
\pgfusepath{clip}%
\pgfsetrectcap%
\pgfsetroundjoin%
\pgfsetlinewidth{1.505625pt}%
\definecolor{currentstroke}{rgb}{1.000000,0.000000,0.000000}%
\pgfsetstrokecolor{currentstroke}%
\pgfsetdash{}{0pt}%
\pgfpathmoveto{\pgfqpoint{1.852696in}{2.023123in}}%
\pgfpathlineto{\pgfqpoint{1.876050in}{2.167865in}}%
\pgfusepath{stroke}%
\end{pgfscope}%
\begin{pgfscope}%
\pgfpathrectangle{\pgfqpoint{0.100000in}{0.220728in}}{\pgfqpoint{3.696000in}{3.696000in}}%
\pgfusepath{clip}%
\pgfsetrectcap%
\pgfsetroundjoin%
\pgfsetlinewidth{1.505625pt}%
\definecolor{currentstroke}{rgb}{1.000000,0.000000,0.000000}%
\pgfsetstrokecolor{currentstroke}%
\pgfsetdash{}{0pt}%
\pgfpathmoveto{\pgfqpoint{1.855854in}{2.025727in}}%
\pgfpathlineto{\pgfqpoint{1.876050in}{2.167865in}}%
\pgfusepath{stroke}%
\end{pgfscope}%
\begin{pgfscope}%
\pgfpathrectangle{\pgfqpoint{0.100000in}{0.220728in}}{\pgfqpoint{3.696000in}{3.696000in}}%
\pgfusepath{clip}%
\pgfsetrectcap%
\pgfsetroundjoin%
\pgfsetlinewidth{1.505625pt}%
\definecolor{currentstroke}{rgb}{1.000000,0.000000,0.000000}%
\pgfsetstrokecolor{currentstroke}%
\pgfsetdash{}{0pt}%
\pgfpathmoveto{\pgfqpoint{1.862630in}{2.025668in}}%
\pgfpathlineto{\pgfqpoint{1.889119in}{2.164097in}}%
\pgfusepath{stroke}%
\end{pgfscope}%
\begin{pgfscope}%
\pgfpathrectangle{\pgfqpoint{0.100000in}{0.220728in}}{\pgfqpoint{3.696000in}{3.696000in}}%
\pgfusepath{clip}%
\pgfsetrectcap%
\pgfsetroundjoin%
\pgfsetlinewidth{1.505625pt}%
\definecolor{currentstroke}{rgb}{1.000000,0.000000,0.000000}%
\pgfsetstrokecolor{currentstroke}%
\pgfsetdash{}{0pt}%
\pgfpathmoveto{\pgfqpoint{1.869277in}{2.027506in}}%
\pgfpathlineto{\pgfqpoint{1.889119in}{2.164097in}}%
\pgfusepath{stroke}%
\end{pgfscope}%
\begin{pgfscope}%
\pgfpathrectangle{\pgfqpoint{0.100000in}{0.220728in}}{\pgfqpoint{3.696000in}{3.696000in}}%
\pgfusepath{clip}%
\pgfsetrectcap%
\pgfsetroundjoin%
\pgfsetlinewidth{1.505625pt}%
\definecolor{currentstroke}{rgb}{1.000000,0.000000,0.000000}%
\pgfsetstrokecolor{currentstroke}%
\pgfsetdash{}{0pt}%
\pgfpathmoveto{\pgfqpoint{1.873954in}{2.030543in}}%
\pgfpathlineto{\pgfqpoint{1.889119in}{2.164097in}}%
\pgfusepath{stroke}%
\end{pgfscope}%
\begin{pgfscope}%
\pgfpathrectangle{\pgfqpoint{0.100000in}{0.220728in}}{\pgfqpoint{3.696000in}{3.696000in}}%
\pgfusepath{clip}%
\pgfsetrectcap%
\pgfsetroundjoin%
\pgfsetlinewidth{1.505625pt}%
\definecolor{currentstroke}{rgb}{1.000000,0.000000,0.000000}%
\pgfsetstrokecolor{currentstroke}%
\pgfsetdash{}{0pt}%
\pgfpathmoveto{\pgfqpoint{1.879478in}{2.034334in}}%
\pgfpathlineto{\pgfqpoint{1.889119in}{2.164097in}}%
\pgfusepath{stroke}%
\end{pgfscope}%
\begin{pgfscope}%
\pgfpathrectangle{\pgfqpoint{0.100000in}{0.220728in}}{\pgfqpoint{3.696000in}{3.696000in}}%
\pgfusepath{clip}%
\pgfsetrectcap%
\pgfsetroundjoin%
\pgfsetlinewidth{1.505625pt}%
\definecolor{currentstroke}{rgb}{1.000000,0.000000,0.000000}%
\pgfsetstrokecolor{currentstroke}%
\pgfsetdash{}{0pt}%
\pgfpathmoveto{\pgfqpoint{1.885657in}{2.037954in}}%
\pgfpathlineto{\pgfqpoint{1.889119in}{2.164097in}}%
\pgfusepath{stroke}%
\end{pgfscope}%
\begin{pgfscope}%
\pgfpathrectangle{\pgfqpoint{0.100000in}{0.220728in}}{\pgfqpoint{3.696000in}{3.696000in}}%
\pgfusepath{clip}%
\pgfsetrectcap%
\pgfsetroundjoin%
\pgfsetlinewidth{1.505625pt}%
\definecolor{currentstroke}{rgb}{1.000000,0.000000,0.000000}%
\pgfsetstrokecolor{currentstroke}%
\pgfsetdash{}{0pt}%
\pgfpathmoveto{\pgfqpoint{1.889775in}{2.039871in}}%
\pgfpathlineto{\pgfqpoint{1.889119in}{2.164097in}}%
\pgfusepath{stroke}%
\end{pgfscope}%
\begin{pgfscope}%
\pgfpathrectangle{\pgfqpoint{0.100000in}{0.220728in}}{\pgfqpoint{3.696000in}{3.696000in}}%
\pgfusepath{clip}%
\pgfsetrectcap%
\pgfsetroundjoin%
\pgfsetlinewidth{1.505625pt}%
\definecolor{currentstroke}{rgb}{1.000000,0.000000,0.000000}%
\pgfsetstrokecolor{currentstroke}%
\pgfsetdash{}{0pt}%
\pgfpathmoveto{\pgfqpoint{1.894540in}{2.043597in}}%
\pgfpathlineto{\pgfqpoint{1.889119in}{2.164097in}}%
\pgfusepath{stroke}%
\end{pgfscope}%
\begin{pgfscope}%
\pgfpathrectangle{\pgfqpoint{0.100000in}{0.220728in}}{\pgfqpoint{3.696000in}{3.696000in}}%
\pgfusepath{clip}%
\pgfsetrectcap%
\pgfsetroundjoin%
\pgfsetlinewidth{1.505625pt}%
\definecolor{currentstroke}{rgb}{1.000000,0.000000,0.000000}%
\pgfsetstrokecolor{currentstroke}%
\pgfsetdash{}{0pt}%
\pgfpathmoveto{\pgfqpoint{1.900587in}{2.047668in}}%
\pgfpathlineto{\pgfqpoint{1.889119in}{2.164097in}}%
\pgfusepath{stroke}%
\end{pgfscope}%
\begin{pgfscope}%
\pgfpathrectangle{\pgfqpoint{0.100000in}{0.220728in}}{\pgfqpoint{3.696000in}{3.696000in}}%
\pgfusepath{clip}%
\pgfsetrectcap%
\pgfsetroundjoin%
\pgfsetlinewidth{1.505625pt}%
\definecolor{currentstroke}{rgb}{1.000000,0.000000,0.000000}%
\pgfsetstrokecolor{currentstroke}%
\pgfsetdash{}{0pt}%
\pgfpathmoveto{\pgfqpoint{1.904493in}{2.050644in}}%
\pgfpathlineto{\pgfqpoint{1.889119in}{2.164097in}}%
\pgfusepath{stroke}%
\end{pgfscope}%
\begin{pgfscope}%
\pgfpathrectangle{\pgfqpoint{0.100000in}{0.220728in}}{\pgfqpoint{3.696000in}{3.696000in}}%
\pgfusepath{clip}%
\pgfsetrectcap%
\pgfsetroundjoin%
\pgfsetlinewidth{1.505625pt}%
\definecolor{currentstroke}{rgb}{1.000000,0.000000,0.000000}%
\pgfsetstrokecolor{currentstroke}%
\pgfsetdash{}{0pt}%
\pgfpathmoveto{\pgfqpoint{1.909042in}{2.053218in}}%
\pgfpathlineto{\pgfqpoint{1.889119in}{2.164097in}}%
\pgfusepath{stroke}%
\end{pgfscope}%
\begin{pgfscope}%
\pgfpathrectangle{\pgfqpoint{0.100000in}{0.220728in}}{\pgfqpoint{3.696000in}{3.696000in}}%
\pgfusepath{clip}%
\pgfsetrectcap%
\pgfsetroundjoin%
\pgfsetlinewidth{1.505625pt}%
\definecolor{currentstroke}{rgb}{1.000000,0.000000,0.000000}%
\pgfsetstrokecolor{currentstroke}%
\pgfsetdash{}{0pt}%
\pgfpathmoveto{\pgfqpoint{1.915862in}{2.054162in}}%
\pgfpathlineto{\pgfqpoint{1.889119in}{2.164097in}}%
\pgfusepath{stroke}%
\end{pgfscope}%
\begin{pgfscope}%
\pgfpathrectangle{\pgfqpoint{0.100000in}{0.220728in}}{\pgfqpoint{3.696000in}{3.696000in}}%
\pgfusepath{clip}%
\pgfsetrectcap%
\pgfsetroundjoin%
\pgfsetlinewidth{1.505625pt}%
\definecolor{currentstroke}{rgb}{1.000000,0.000000,0.000000}%
\pgfsetstrokecolor{currentstroke}%
\pgfsetdash{}{0pt}%
\pgfpathmoveto{\pgfqpoint{1.920009in}{2.056601in}}%
\pgfpathlineto{\pgfqpoint{1.889119in}{2.164097in}}%
\pgfusepath{stroke}%
\end{pgfscope}%
\begin{pgfscope}%
\pgfpathrectangle{\pgfqpoint{0.100000in}{0.220728in}}{\pgfqpoint{3.696000in}{3.696000in}}%
\pgfusepath{clip}%
\pgfsetrectcap%
\pgfsetroundjoin%
\pgfsetlinewidth{1.505625pt}%
\definecolor{currentstroke}{rgb}{1.000000,0.000000,0.000000}%
\pgfsetstrokecolor{currentstroke}%
\pgfsetdash{}{0pt}%
\pgfpathmoveto{\pgfqpoint{1.922368in}{2.057797in}}%
\pgfpathlineto{\pgfqpoint{1.889119in}{2.164097in}}%
\pgfusepath{stroke}%
\end{pgfscope}%
\begin{pgfscope}%
\pgfpathrectangle{\pgfqpoint{0.100000in}{0.220728in}}{\pgfqpoint{3.696000in}{3.696000in}}%
\pgfusepath{clip}%
\pgfsetrectcap%
\pgfsetroundjoin%
\pgfsetlinewidth{1.505625pt}%
\definecolor{currentstroke}{rgb}{1.000000,0.000000,0.000000}%
\pgfsetstrokecolor{currentstroke}%
\pgfsetdash{}{0pt}%
\pgfpathmoveto{\pgfqpoint{1.926896in}{2.059562in}}%
\pgfpathlineto{\pgfqpoint{1.889119in}{2.164097in}}%
\pgfusepath{stroke}%
\end{pgfscope}%
\begin{pgfscope}%
\pgfpathrectangle{\pgfqpoint{0.100000in}{0.220728in}}{\pgfqpoint{3.696000in}{3.696000in}}%
\pgfusepath{clip}%
\pgfsetrectcap%
\pgfsetroundjoin%
\pgfsetlinewidth{1.505625pt}%
\definecolor{currentstroke}{rgb}{1.000000,0.000000,0.000000}%
\pgfsetstrokecolor{currentstroke}%
\pgfsetdash{}{0pt}%
\pgfpathmoveto{\pgfqpoint{1.931388in}{2.060197in}}%
\pgfpathlineto{\pgfqpoint{1.902198in}{2.160326in}}%
\pgfusepath{stroke}%
\end{pgfscope}%
\begin{pgfscope}%
\pgfpathrectangle{\pgfqpoint{0.100000in}{0.220728in}}{\pgfqpoint{3.696000in}{3.696000in}}%
\pgfusepath{clip}%
\pgfsetrectcap%
\pgfsetroundjoin%
\pgfsetlinewidth{1.505625pt}%
\definecolor{currentstroke}{rgb}{1.000000,0.000000,0.000000}%
\pgfsetstrokecolor{currentstroke}%
\pgfsetdash{}{0pt}%
\pgfpathmoveto{\pgfqpoint{1.937012in}{2.061347in}}%
\pgfpathlineto{\pgfqpoint{1.902198in}{2.160326in}}%
\pgfusepath{stroke}%
\end{pgfscope}%
\begin{pgfscope}%
\pgfpathrectangle{\pgfqpoint{0.100000in}{0.220728in}}{\pgfqpoint{3.696000in}{3.696000in}}%
\pgfusepath{clip}%
\pgfsetrectcap%
\pgfsetroundjoin%
\pgfsetlinewidth{1.505625pt}%
\definecolor{currentstroke}{rgb}{1.000000,0.000000,0.000000}%
\pgfsetstrokecolor{currentstroke}%
\pgfsetdash{}{0pt}%
\pgfpathmoveto{\pgfqpoint{1.940183in}{2.060997in}}%
\pgfpathlineto{\pgfqpoint{1.902198in}{2.160326in}}%
\pgfusepath{stroke}%
\end{pgfscope}%
\begin{pgfscope}%
\pgfpathrectangle{\pgfqpoint{0.100000in}{0.220728in}}{\pgfqpoint{3.696000in}{3.696000in}}%
\pgfusepath{clip}%
\pgfsetrectcap%
\pgfsetroundjoin%
\pgfsetlinewidth{1.505625pt}%
\definecolor{currentstroke}{rgb}{1.000000,0.000000,0.000000}%
\pgfsetstrokecolor{currentstroke}%
\pgfsetdash{}{0pt}%
\pgfpathmoveto{\pgfqpoint{1.941953in}{2.062094in}}%
\pgfpathlineto{\pgfqpoint{1.902198in}{2.160326in}}%
\pgfusepath{stroke}%
\end{pgfscope}%
\begin{pgfscope}%
\pgfpathrectangle{\pgfqpoint{0.100000in}{0.220728in}}{\pgfqpoint{3.696000in}{3.696000in}}%
\pgfusepath{clip}%
\pgfsetrectcap%
\pgfsetroundjoin%
\pgfsetlinewidth{1.505625pt}%
\definecolor{currentstroke}{rgb}{1.000000,0.000000,0.000000}%
\pgfsetstrokecolor{currentstroke}%
\pgfsetdash{}{0pt}%
\pgfpathmoveto{\pgfqpoint{1.942972in}{2.062391in}}%
\pgfpathlineto{\pgfqpoint{1.902198in}{2.160326in}}%
\pgfusepath{stroke}%
\end{pgfscope}%
\begin{pgfscope}%
\pgfpathrectangle{\pgfqpoint{0.100000in}{0.220728in}}{\pgfqpoint{3.696000in}{3.696000in}}%
\pgfusepath{clip}%
\pgfsetrectcap%
\pgfsetroundjoin%
\pgfsetlinewidth{1.505625pt}%
\definecolor{currentstroke}{rgb}{1.000000,0.000000,0.000000}%
\pgfsetstrokecolor{currentstroke}%
\pgfsetdash{}{0pt}%
\pgfpathmoveto{\pgfqpoint{1.944486in}{2.062736in}}%
\pgfpathlineto{\pgfqpoint{1.902198in}{2.160326in}}%
\pgfusepath{stroke}%
\end{pgfscope}%
\begin{pgfscope}%
\pgfpathrectangle{\pgfqpoint{0.100000in}{0.220728in}}{\pgfqpoint{3.696000in}{3.696000in}}%
\pgfusepath{clip}%
\pgfsetrectcap%
\pgfsetroundjoin%
\pgfsetlinewidth{1.505625pt}%
\definecolor{currentstroke}{rgb}{1.000000,0.000000,0.000000}%
\pgfsetstrokecolor{currentstroke}%
\pgfsetdash{}{0pt}%
\pgfpathmoveto{\pgfqpoint{1.946313in}{2.063230in}}%
\pgfpathlineto{\pgfqpoint{1.902198in}{2.160326in}}%
\pgfusepath{stroke}%
\end{pgfscope}%
\begin{pgfscope}%
\pgfpathrectangle{\pgfqpoint{0.100000in}{0.220728in}}{\pgfqpoint{3.696000in}{3.696000in}}%
\pgfusepath{clip}%
\pgfsetrectcap%
\pgfsetroundjoin%
\pgfsetlinewidth{1.505625pt}%
\definecolor{currentstroke}{rgb}{1.000000,0.000000,0.000000}%
\pgfsetstrokecolor{currentstroke}%
\pgfsetdash{}{0pt}%
\pgfpathmoveto{\pgfqpoint{1.947408in}{2.063502in}}%
\pgfpathlineto{\pgfqpoint{1.902198in}{2.160326in}}%
\pgfusepath{stroke}%
\end{pgfscope}%
\begin{pgfscope}%
\pgfpathrectangle{\pgfqpoint{0.100000in}{0.220728in}}{\pgfqpoint{3.696000in}{3.696000in}}%
\pgfusepath{clip}%
\pgfsetrectcap%
\pgfsetroundjoin%
\pgfsetlinewidth{1.505625pt}%
\definecolor{currentstroke}{rgb}{1.000000,0.000000,0.000000}%
\pgfsetstrokecolor{currentstroke}%
\pgfsetdash{}{0pt}%
\pgfpathmoveto{\pgfqpoint{1.947990in}{2.063680in}}%
\pgfpathlineto{\pgfqpoint{1.902198in}{2.160326in}}%
\pgfusepath{stroke}%
\end{pgfscope}%
\begin{pgfscope}%
\pgfpathrectangle{\pgfqpoint{0.100000in}{0.220728in}}{\pgfqpoint{3.696000in}{3.696000in}}%
\pgfusepath{clip}%
\pgfsetrectcap%
\pgfsetroundjoin%
\pgfsetlinewidth{1.505625pt}%
\definecolor{currentstroke}{rgb}{1.000000,0.000000,0.000000}%
\pgfsetstrokecolor{currentstroke}%
\pgfsetdash{}{0pt}%
\pgfpathmoveto{\pgfqpoint{1.948313in}{2.063775in}}%
\pgfpathlineto{\pgfqpoint{1.902198in}{2.160326in}}%
\pgfusepath{stroke}%
\end{pgfscope}%
\begin{pgfscope}%
\pgfpathrectangle{\pgfqpoint{0.100000in}{0.220728in}}{\pgfqpoint{3.696000in}{3.696000in}}%
\pgfusepath{clip}%
\pgfsetrectcap%
\pgfsetroundjoin%
\pgfsetlinewidth{1.505625pt}%
\definecolor{currentstroke}{rgb}{1.000000,0.000000,0.000000}%
\pgfsetstrokecolor{currentstroke}%
\pgfsetdash{}{0pt}%
\pgfpathmoveto{\pgfqpoint{1.948492in}{2.063818in}}%
\pgfpathlineto{\pgfqpoint{1.902198in}{2.160326in}}%
\pgfusepath{stroke}%
\end{pgfscope}%
\begin{pgfscope}%
\pgfpathrectangle{\pgfqpoint{0.100000in}{0.220728in}}{\pgfqpoint{3.696000in}{3.696000in}}%
\pgfusepath{clip}%
\pgfsetrectcap%
\pgfsetroundjoin%
\pgfsetlinewidth{1.505625pt}%
\definecolor{currentstroke}{rgb}{1.000000,0.000000,0.000000}%
\pgfsetstrokecolor{currentstroke}%
\pgfsetdash{}{0pt}%
\pgfpathmoveto{\pgfqpoint{1.948583in}{2.063848in}}%
\pgfpathlineto{\pgfqpoint{1.902198in}{2.160326in}}%
\pgfusepath{stroke}%
\end{pgfscope}%
\begin{pgfscope}%
\pgfpathrectangle{\pgfqpoint{0.100000in}{0.220728in}}{\pgfqpoint{3.696000in}{3.696000in}}%
\pgfusepath{clip}%
\pgfsetrectcap%
\pgfsetroundjoin%
\pgfsetlinewidth{1.505625pt}%
\definecolor{currentstroke}{rgb}{1.000000,0.000000,0.000000}%
\pgfsetstrokecolor{currentstroke}%
\pgfsetdash{}{0pt}%
\pgfpathmoveto{\pgfqpoint{1.948642in}{2.063839in}}%
\pgfpathlineto{\pgfqpoint{1.902198in}{2.160326in}}%
\pgfusepath{stroke}%
\end{pgfscope}%
\begin{pgfscope}%
\pgfpathrectangle{\pgfqpoint{0.100000in}{0.220728in}}{\pgfqpoint{3.696000in}{3.696000in}}%
\pgfusepath{clip}%
\pgfsetrectcap%
\pgfsetroundjoin%
\pgfsetlinewidth{1.505625pt}%
\definecolor{currentstroke}{rgb}{1.000000,0.000000,0.000000}%
\pgfsetstrokecolor{currentstroke}%
\pgfsetdash{}{0pt}%
\pgfpathmoveto{\pgfqpoint{1.949252in}{2.063155in}}%
\pgfpathlineto{\pgfqpoint{1.902198in}{2.160326in}}%
\pgfusepath{stroke}%
\end{pgfscope}%
\begin{pgfscope}%
\pgfpathrectangle{\pgfqpoint{0.100000in}{0.220728in}}{\pgfqpoint{3.696000in}{3.696000in}}%
\pgfusepath{clip}%
\pgfsetrectcap%
\pgfsetroundjoin%
\pgfsetlinewidth{1.505625pt}%
\definecolor{currentstroke}{rgb}{1.000000,0.000000,0.000000}%
\pgfsetstrokecolor{currentstroke}%
\pgfsetdash{}{0pt}%
\pgfpathmoveto{\pgfqpoint{1.949612in}{2.063185in}}%
\pgfpathlineto{\pgfqpoint{1.902198in}{2.160326in}}%
\pgfusepath{stroke}%
\end{pgfscope}%
\begin{pgfscope}%
\pgfpathrectangle{\pgfqpoint{0.100000in}{0.220728in}}{\pgfqpoint{3.696000in}{3.696000in}}%
\pgfusepath{clip}%
\pgfsetrectcap%
\pgfsetroundjoin%
\pgfsetlinewidth{1.505625pt}%
\definecolor{currentstroke}{rgb}{1.000000,0.000000,0.000000}%
\pgfsetstrokecolor{currentstroke}%
\pgfsetdash{}{0pt}%
\pgfpathmoveto{\pgfqpoint{1.949818in}{2.063041in}}%
\pgfpathlineto{\pgfqpoint{1.902198in}{2.160326in}}%
\pgfusepath{stroke}%
\end{pgfscope}%
\begin{pgfscope}%
\pgfpathrectangle{\pgfqpoint{0.100000in}{0.220728in}}{\pgfqpoint{3.696000in}{3.696000in}}%
\pgfusepath{clip}%
\pgfsetrectcap%
\pgfsetroundjoin%
\pgfsetlinewidth{1.505625pt}%
\definecolor{currentstroke}{rgb}{1.000000,0.000000,0.000000}%
\pgfsetstrokecolor{currentstroke}%
\pgfsetdash{}{0pt}%
\pgfpathmoveto{\pgfqpoint{1.950563in}{2.062806in}}%
\pgfpathlineto{\pgfqpoint{1.902198in}{2.160326in}}%
\pgfusepath{stroke}%
\end{pgfscope}%
\begin{pgfscope}%
\pgfpathrectangle{\pgfqpoint{0.100000in}{0.220728in}}{\pgfqpoint{3.696000in}{3.696000in}}%
\pgfusepath{clip}%
\pgfsetrectcap%
\pgfsetroundjoin%
\pgfsetlinewidth{1.505625pt}%
\definecolor{currentstroke}{rgb}{1.000000,0.000000,0.000000}%
\pgfsetstrokecolor{currentstroke}%
\pgfsetdash{}{0pt}%
\pgfpathmoveto{\pgfqpoint{1.950916in}{2.062427in}}%
\pgfpathlineto{\pgfqpoint{1.902198in}{2.160326in}}%
\pgfusepath{stroke}%
\end{pgfscope}%
\begin{pgfscope}%
\pgfpathrectangle{\pgfqpoint{0.100000in}{0.220728in}}{\pgfqpoint{3.696000in}{3.696000in}}%
\pgfusepath{clip}%
\pgfsetrectcap%
\pgfsetroundjoin%
\pgfsetlinewidth{1.505625pt}%
\definecolor{currentstroke}{rgb}{1.000000,0.000000,0.000000}%
\pgfsetstrokecolor{currentstroke}%
\pgfsetdash{}{0pt}%
\pgfpathmoveto{\pgfqpoint{1.951094in}{2.062368in}}%
\pgfpathlineto{\pgfqpoint{1.902198in}{2.160326in}}%
\pgfusepath{stroke}%
\end{pgfscope}%
\begin{pgfscope}%
\pgfpathrectangle{\pgfqpoint{0.100000in}{0.220728in}}{\pgfqpoint{3.696000in}{3.696000in}}%
\pgfusepath{clip}%
\pgfsetrectcap%
\pgfsetroundjoin%
\pgfsetlinewidth{1.505625pt}%
\definecolor{currentstroke}{rgb}{1.000000,0.000000,0.000000}%
\pgfsetstrokecolor{currentstroke}%
\pgfsetdash{}{0pt}%
\pgfpathmoveto{\pgfqpoint{1.951168in}{2.062372in}}%
\pgfpathlineto{\pgfqpoint{1.902198in}{2.160326in}}%
\pgfusepath{stroke}%
\end{pgfscope}%
\begin{pgfscope}%
\pgfpathrectangle{\pgfqpoint{0.100000in}{0.220728in}}{\pgfqpoint{3.696000in}{3.696000in}}%
\pgfusepath{clip}%
\pgfsetrectcap%
\pgfsetroundjoin%
\pgfsetlinewidth{1.505625pt}%
\definecolor{currentstroke}{rgb}{1.000000,0.000000,0.000000}%
\pgfsetstrokecolor{currentstroke}%
\pgfsetdash{}{0pt}%
\pgfpathmoveto{\pgfqpoint{1.951195in}{2.062351in}}%
\pgfpathlineto{\pgfqpoint{1.902198in}{2.160326in}}%
\pgfusepath{stroke}%
\end{pgfscope}%
\begin{pgfscope}%
\pgfpathrectangle{\pgfqpoint{0.100000in}{0.220728in}}{\pgfqpoint{3.696000in}{3.696000in}}%
\pgfusepath{clip}%
\pgfsetrectcap%
\pgfsetroundjoin%
\pgfsetlinewidth{1.505625pt}%
\definecolor{currentstroke}{rgb}{1.000000,0.000000,0.000000}%
\pgfsetstrokecolor{currentstroke}%
\pgfsetdash{}{0pt}%
\pgfpathmoveto{\pgfqpoint{1.951206in}{2.062346in}}%
\pgfpathlineto{\pgfqpoint{1.902198in}{2.160326in}}%
\pgfusepath{stroke}%
\end{pgfscope}%
\begin{pgfscope}%
\pgfpathrectangle{\pgfqpoint{0.100000in}{0.220728in}}{\pgfqpoint{3.696000in}{3.696000in}}%
\pgfusepath{clip}%
\pgfsetrectcap%
\pgfsetroundjoin%
\pgfsetlinewidth{1.505625pt}%
\definecolor{currentstroke}{rgb}{1.000000,0.000000,0.000000}%
\pgfsetstrokecolor{currentstroke}%
\pgfsetdash{}{0pt}%
\pgfpathmoveto{\pgfqpoint{1.951269in}{2.062069in}}%
\pgfpathlineto{\pgfqpoint{1.902198in}{2.160326in}}%
\pgfusepath{stroke}%
\end{pgfscope}%
\begin{pgfscope}%
\pgfpathrectangle{\pgfqpoint{0.100000in}{0.220728in}}{\pgfqpoint{3.696000in}{3.696000in}}%
\pgfusepath{clip}%
\pgfsetrectcap%
\pgfsetroundjoin%
\pgfsetlinewidth{1.505625pt}%
\definecolor{currentstroke}{rgb}{1.000000,0.000000,0.000000}%
\pgfsetstrokecolor{currentstroke}%
\pgfsetdash{}{0pt}%
\pgfpathmoveto{\pgfqpoint{1.951334in}{2.062388in}}%
\pgfpathlineto{\pgfqpoint{1.902198in}{2.160326in}}%
\pgfusepath{stroke}%
\end{pgfscope}%
\begin{pgfscope}%
\pgfpathrectangle{\pgfqpoint{0.100000in}{0.220728in}}{\pgfqpoint{3.696000in}{3.696000in}}%
\pgfusepath{clip}%
\pgfsetrectcap%
\pgfsetroundjoin%
\pgfsetlinewidth{1.505625pt}%
\definecolor{currentstroke}{rgb}{1.000000,0.000000,0.000000}%
\pgfsetstrokecolor{currentstroke}%
\pgfsetdash{}{0pt}%
\pgfpathmoveto{\pgfqpoint{1.951334in}{2.062385in}}%
\pgfpathlineto{\pgfqpoint{1.902198in}{2.160326in}}%
\pgfusepath{stroke}%
\end{pgfscope}%
\begin{pgfscope}%
\pgfpathrectangle{\pgfqpoint{0.100000in}{0.220728in}}{\pgfqpoint{3.696000in}{3.696000in}}%
\pgfusepath{clip}%
\pgfsetrectcap%
\pgfsetroundjoin%
\pgfsetlinewidth{1.505625pt}%
\definecolor{currentstroke}{rgb}{1.000000,0.000000,0.000000}%
\pgfsetstrokecolor{currentstroke}%
\pgfsetdash{}{0pt}%
\pgfpathmoveto{\pgfqpoint{1.951315in}{2.062239in}}%
\pgfpathlineto{\pgfqpoint{1.902198in}{2.160326in}}%
\pgfusepath{stroke}%
\end{pgfscope}%
\begin{pgfscope}%
\pgfpathrectangle{\pgfqpoint{0.100000in}{0.220728in}}{\pgfqpoint{3.696000in}{3.696000in}}%
\pgfusepath{clip}%
\pgfsetrectcap%
\pgfsetroundjoin%
\pgfsetlinewidth{1.505625pt}%
\definecolor{currentstroke}{rgb}{1.000000,0.000000,0.000000}%
\pgfsetstrokecolor{currentstroke}%
\pgfsetdash{}{0pt}%
\pgfpathmoveto{\pgfqpoint{1.951257in}{2.062164in}}%
\pgfpathlineto{\pgfqpoint{1.902198in}{2.160326in}}%
\pgfusepath{stroke}%
\end{pgfscope}%
\begin{pgfscope}%
\pgfpathrectangle{\pgfqpoint{0.100000in}{0.220728in}}{\pgfqpoint{3.696000in}{3.696000in}}%
\pgfusepath{clip}%
\pgfsetrectcap%
\pgfsetroundjoin%
\pgfsetlinewidth{1.505625pt}%
\definecolor{currentstroke}{rgb}{1.000000,0.000000,0.000000}%
\pgfsetstrokecolor{currentstroke}%
\pgfsetdash{}{0pt}%
\pgfpathmoveto{\pgfqpoint{1.951178in}{2.062052in}}%
\pgfpathlineto{\pgfqpoint{1.902198in}{2.160326in}}%
\pgfusepath{stroke}%
\end{pgfscope}%
\begin{pgfscope}%
\pgfpathrectangle{\pgfqpoint{0.100000in}{0.220728in}}{\pgfqpoint{3.696000in}{3.696000in}}%
\pgfusepath{clip}%
\pgfsetrectcap%
\pgfsetroundjoin%
\pgfsetlinewidth{1.505625pt}%
\definecolor{currentstroke}{rgb}{1.000000,0.000000,0.000000}%
\pgfsetstrokecolor{currentstroke}%
\pgfsetdash{}{0pt}%
\pgfpathmoveto{\pgfqpoint{1.951026in}{2.062591in}}%
\pgfpathlineto{\pgfqpoint{1.902198in}{2.160326in}}%
\pgfusepath{stroke}%
\end{pgfscope}%
\begin{pgfscope}%
\pgfpathrectangle{\pgfqpoint{0.100000in}{0.220728in}}{\pgfqpoint{3.696000in}{3.696000in}}%
\pgfusepath{clip}%
\pgfsetrectcap%
\pgfsetroundjoin%
\pgfsetlinewidth{1.505625pt}%
\definecolor{currentstroke}{rgb}{1.000000,0.000000,0.000000}%
\pgfsetstrokecolor{currentstroke}%
\pgfsetdash{}{0pt}%
\pgfpathmoveto{\pgfqpoint{1.951405in}{2.059997in}}%
\pgfpathlineto{\pgfqpoint{1.902198in}{2.160326in}}%
\pgfusepath{stroke}%
\end{pgfscope}%
\begin{pgfscope}%
\pgfpathrectangle{\pgfqpoint{0.100000in}{0.220728in}}{\pgfqpoint{3.696000in}{3.696000in}}%
\pgfusepath{clip}%
\pgfsetrectcap%
\pgfsetroundjoin%
\pgfsetlinewidth{1.505625pt}%
\definecolor{currentstroke}{rgb}{1.000000,0.000000,0.000000}%
\pgfsetstrokecolor{currentstroke}%
\pgfsetdash{}{0pt}%
\pgfpathmoveto{\pgfqpoint{1.952205in}{2.057475in}}%
\pgfpathlineto{\pgfqpoint{1.915284in}{2.156553in}}%
\pgfusepath{stroke}%
\end{pgfscope}%
\begin{pgfscope}%
\pgfpathrectangle{\pgfqpoint{0.100000in}{0.220728in}}{\pgfqpoint{3.696000in}{3.696000in}}%
\pgfusepath{clip}%
\pgfsetrectcap%
\pgfsetroundjoin%
\pgfsetlinewidth{1.505625pt}%
\definecolor{currentstroke}{rgb}{1.000000,0.000000,0.000000}%
\pgfsetstrokecolor{currentstroke}%
\pgfsetdash{}{0pt}%
\pgfpathmoveto{\pgfqpoint{1.953631in}{2.059996in}}%
\pgfpathlineto{\pgfqpoint{1.915284in}{2.156553in}}%
\pgfusepath{stroke}%
\end{pgfscope}%
\begin{pgfscope}%
\pgfpathrectangle{\pgfqpoint{0.100000in}{0.220728in}}{\pgfqpoint{3.696000in}{3.696000in}}%
\pgfusepath{clip}%
\pgfsetrectcap%
\pgfsetroundjoin%
\pgfsetlinewidth{1.505625pt}%
\definecolor{currentstroke}{rgb}{1.000000,0.000000,0.000000}%
\pgfsetstrokecolor{currentstroke}%
\pgfsetdash{}{0pt}%
\pgfpathmoveto{\pgfqpoint{1.954148in}{2.056288in}}%
\pgfpathlineto{\pgfqpoint{1.915284in}{2.156553in}}%
\pgfusepath{stroke}%
\end{pgfscope}%
\begin{pgfscope}%
\pgfpathrectangle{\pgfqpoint{0.100000in}{0.220728in}}{\pgfqpoint{3.696000in}{3.696000in}}%
\pgfusepath{clip}%
\pgfsetrectcap%
\pgfsetroundjoin%
\pgfsetlinewidth{1.505625pt}%
\definecolor{currentstroke}{rgb}{1.000000,0.000000,0.000000}%
\pgfsetstrokecolor{currentstroke}%
\pgfsetdash{}{0pt}%
\pgfpathmoveto{\pgfqpoint{1.956741in}{2.052113in}}%
\pgfpathlineto{\pgfqpoint{1.928380in}{2.152777in}}%
\pgfusepath{stroke}%
\end{pgfscope}%
\begin{pgfscope}%
\pgfpathrectangle{\pgfqpoint{0.100000in}{0.220728in}}{\pgfqpoint{3.696000in}{3.696000in}}%
\pgfusepath{clip}%
\pgfsetrectcap%
\pgfsetroundjoin%
\pgfsetlinewidth{1.505625pt}%
\definecolor{currentstroke}{rgb}{1.000000,0.000000,0.000000}%
\pgfsetstrokecolor{currentstroke}%
\pgfsetdash{}{0pt}%
\pgfpathmoveto{\pgfqpoint{1.958161in}{2.052914in}}%
\pgfpathlineto{\pgfqpoint{1.928380in}{2.152777in}}%
\pgfusepath{stroke}%
\end{pgfscope}%
\begin{pgfscope}%
\pgfpathrectangle{\pgfqpoint{0.100000in}{0.220728in}}{\pgfqpoint{3.696000in}{3.696000in}}%
\pgfusepath{clip}%
\pgfsetrectcap%
\pgfsetroundjoin%
\pgfsetlinewidth{1.505625pt}%
\definecolor{currentstroke}{rgb}{1.000000,0.000000,0.000000}%
\pgfsetstrokecolor{currentstroke}%
\pgfsetdash{}{0pt}%
\pgfpathmoveto{\pgfqpoint{1.958574in}{2.049735in}}%
\pgfpathlineto{\pgfqpoint{1.928380in}{2.152777in}}%
\pgfusepath{stroke}%
\end{pgfscope}%
\begin{pgfscope}%
\pgfpathrectangle{\pgfqpoint{0.100000in}{0.220728in}}{\pgfqpoint{3.696000in}{3.696000in}}%
\pgfusepath{clip}%
\pgfsetrectcap%
\pgfsetroundjoin%
\pgfsetlinewidth{1.505625pt}%
\definecolor{currentstroke}{rgb}{1.000000,0.000000,0.000000}%
\pgfsetstrokecolor{currentstroke}%
\pgfsetdash{}{0pt}%
\pgfpathmoveto{\pgfqpoint{1.959619in}{2.048976in}}%
\pgfpathlineto{\pgfqpoint{1.928380in}{2.152777in}}%
\pgfusepath{stroke}%
\end{pgfscope}%
\begin{pgfscope}%
\pgfpathrectangle{\pgfqpoint{0.100000in}{0.220728in}}{\pgfqpoint{3.696000in}{3.696000in}}%
\pgfusepath{clip}%
\pgfsetrectcap%
\pgfsetroundjoin%
\pgfsetlinewidth{1.505625pt}%
\definecolor{currentstroke}{rgb}{1.000000,0.000000,0.000000}%
\pgfsetstrokecolor{currentstroke}%
\pgfsetdash{}{0pt}%
\pgfpathmoveto{\pgfqpoint{1.961435in}{2.050569in}}%
\pgfpathlineto{\pgfqpoint{1.941484in}{2.148999in}}%
\pgfusepath{stroke}%
\end{pgfscope}%
\begin{pgfscope}%
\pgfpathrectangle{\pgfqpoint{0.100000in}{0.220728in}}{\pgfqpoint{3.696000in}{3.696000in}}%
\pgfusepath{clip}%
\pgfsetrectcap%
\pgfsetroundjoin%
\pgfsetlinewidth{1.505625pt}%
\definecolor{currentstroke}{rgb}{1.000000,0.000000,0.000000}%
\pgfsetstrokecolor{currentstroke}%
\pgfsetdash{}{0pt}%
\pgfpathmoveto{\pgfqpoint{1.961618in}{2.049751in}}%
\pgfpathlineto{\pgfqpoint{1.941484in}{2.148999in}}%
\pgfusepath{stroke}%
\end{pgfscope}%
\begin{pgfscope}%
\pgfpathrectangle{\pgfqpoint{0.100000in}{0.220728in}}{\pgfqpoint{3.696000in}{3.696000in}}%
\pgfusepath{clip}%
\pgfsetrectcap%
\pgfsetroundjoin%
\pgfsetlinewidth{1.505625pt}%
\definecolor{currentstroke}{rgb}{1.000000,0.000000,0.000000}%
\pgfsetstrokecolor{currentstroke}%
\pgfsetdash{}{0pt}%
\pgfpathmoveto{\pgfqpoint{1.963072in}{2.048834in}}%
\pgfpathlineto{\pgfqpoint{1.941484in}{2.148999in}}%
\pgfusepath{stroke}%
\end{pgfscope}%
\begin{pgfscope}%
\pgfpathrectangle{\pgfqpoint{0.100000in}{0.220728in}}{\pgfqpoint{3.696000in}{3.696000in}}%
\pgfusepath{clip}%
\pgfsetrectcap%
\pgfsetroundjoin%
\pgfsetlinewidth{1.505625pt}%
\definecolor{currentstroke}{rgb}{1.000000,0.000000,0.000000}%
\pgfsetstrokecolor{currentstroke}%
\pgfsetdash{}{0pt}%
\pgfpathmoveto{\pgfqpoint{1.964962in}{2.048999in}}%
\pgfpathlineto{\pgfqpoint{1.941484in}{2.148999in}}%
\pgfusepath{stroke}%
\end{pgfscope}%
\begin{pgfscope}%
\pgfpathrectangle{\pgfqpoint{0.100000in}{0.220728in}}{\pgfqpoint{3.696000in}{3.696000in}}%
\pgfusepath{clip}%
\pgfsetrectcap%
\pgfsetroundjoin%
\pgfsetlinewidth{1.505625pt}%
\definecolor{currentstroke}{rgb}{1.000000,0.000000,0.000000}%
\pgfsetstrokecolor{currentstroke}%
\pgfsetdash{}{0pt}%
\pgfpathmoveto{\pgfqpoint{1.965670in}{2.045688in}}%
\pgfpathlineto{\pgfqpoint{1.954597in}{2.145218in}}%
\pgfusepath{stroke}%
\end{pgfscope}%
\begin{pgfscope}%
\pgfpathrectangle{\pgfqpoint{0.100000in}{0.220728in}}{\pgfqpoint{3.696000in}{3.696000in}}%
\pgfusepath{clip}%
\pgfsetrectcap%
\pgfsetroundjoin%
\pgfsetlinewidth{1.505625pt}%
\definecolor{currentstroke}{rgb}{1.000000,0.000000,0.000000}%
\pgfsetstrokecolor{currentstroke}%
\pgfsetdash{}{0pt}%
\pgfpathmoveto{\pgfqpoint{1.967987in}{2.045471in}}%
\pgfpathlineto{\pgfqpoint{1.954597in}{2.145218in}}%
\pgfusepath{stroke}%
\end{pgfscope}%
\begin{pgfscope}%
\pgfpathrectangle{\pgfqpoint{0.100000in}{0.220728in}}{\pgfqpoint{3.696000in}{3.696000in}}%
\pgfusepath{clip}%
\pgfsetrectcap%
\pgfsetroundjoin%
\pgfsetlinewidth{1.505625pt}%
\definecolor{currentstroke}{rgb}{1.000000,0.000000,0.000000}%
\pgfsetstrokecolor{currentstroke}%
\pgfsetdash{}{0pt}%
\pgfpathmoveto{\pgfqpoint{1.970741in}{2.046960in}}%
\pgfpathlineto{\pgfqpoint{1.967718in}{2.141435in}}%
\pgfusepath{stroke}%
\end{pgfscope}%
\begin{pgfscope}%
\pgfpathrectangle{\pgfqpoint{0.100000in}{0.220728in}}{\pgfqpoint{3.696000in}{3.696000in}}%
\pgfusepath{clip}%
\pgfsetrectcap%
\pgfsetroundjoin%
\pgfsetlinewidth{1.505625pt}%
\definecolor{currentstroke}{rgb}{1.000000,0.000000,0.000000}%
\pgfsetstrokecolor{currentstroke}%
\pgfsetdash{}{0pt}%
\pgfpathmoveto{\pgfqpoint{1.971625in}{2.045973in}}%
\pgfpathlineto{\pgfqpoint{1.967718in}{2.141435in}}%
\pgfusepath{stroke}%
\end{pgfscope}%
\begin{pgfscope}%
\pgfpathrectangle{\pgfqpoint{0.100000in}{0.220728in}}{\pgfqpoint{3.696000in}{3.696000in}}%
\pgfusepath{clip}%
\pgfsetrectcap%
\pgfsetroundjoin%
\pgfsetlinewidth{1.505625pt}%
\definecolor{currentstroke}{rgb}{1.000000,0.000000,0.000000}%
\pgfsetstrokecolor{currentstroke}%
\pgfsetdash{}{0pt}%
\pgfpathmoveto{\pgfqpoint{1.973457in}{2.045326in}}%
\pgfpathlineto{\pgfqpoint{1.967718in}{2.141435in}}%
\pgfusepath{stroke}%
\end{pgfscope}%
\begin{pgfscope}%
\pgfpathrectangle{\pgfqpoint{0.100000in}{0.220728in}}{\pgfqpoint{3.696000in}{3.696000in}}%
\pgfusepath{clip}%
\pgfsetrectcap%
\pgfsetroundjoin%
\pgfsetlinewidth{1.505625pt}%
\definecolor{currentstroke}{rgb}{1.000000,0.000000,0.000000}%
\pgfsetstrokecolor{currentstroke}%
\pgfsetdash{}{0pt}%
\pgfpathmoveto{\pgfqpoint{1.976093in}{2.047006in}}%
\pgfpathlineto{\pgfqpoint{1.980848in}{2.137650in}}%
\pgfusepath{stroke}%
\end{pgfscope}%
\begin{pgfscope}%
\pgfpathrectangle{\pgfqpoint{0.100000in}{0.220728in}}{\pgfqpoint{3.696000in}{3.696000in}}%
\pgfusepath{clip}%
\pgfsetrectcap%
\pgfsetroundjoin%
\pgfsetlinewidth{1.505625pt}%
\definecolor{currentstroke}{rgb}{1.000000,0.000000,0.000000}%
\pgfsetstrokecolor{currentstroke}%
\pgfsetdash{}{0pt}%
\pgfpathmoveto{\pgfqpoint{1.977742in}{2.044943in}}%
\pgfpathlineto{\pgfqpoint{1.980848in}{2.137650in}}%
\pgfusepath{stroke}%
\end{pgfscope}%
\begin{pgfscope}%
\pgfpathrectangle{\pgfqpoint{0.100000in}{0.220728in}}{\pgfqpoint{3.696000in}{3.696000in}}%
\pgfusepath{clip}%
\pgfsetrectcap%
\pgfsetroundjoin%
\pgfsetlinewidth{1.505625pt}%
\definecolor{currentstroke}{rgb}{1.000000,0.000000,0.000000}%
\pgfsetstrokecolor{currentstroke}%
\pgfsetdash{}{0pt}%
\pgfpathmoveto{\pgfqpoint{1.980262in}{2.039647in}}%
\pgfpathlineto{\pgfqpoint{1.993986in}{2.133861in}}%
\pgfusepath{stroke}%
\end{pgfscope}%
\begin{pgfscope}%
\pgfpathrectangle{\pgfqpoint{0.100000in}{0.220728in}}{\pgfqpoint{3.696000in}{3.696000in}}%
\pgfusepath{clip}%
\pgfsetrectcap%
\pgfsetroundjoin%
\pgfsetlinewidth{1.505625pt}%
\definecolor{currentstroke}{rgb}{1.000000,0.000000,0.000000}%
\pgfsetstrokecolor{currentstroke}%
\pgfsetdash{}{0pt}%
\pgfpathmoveto{\pgfqpoint{1.984160in}{2.041490in}}%
\pgfpathlineto{\pgfqpoint{1.993986in}{2.133861in}}%
\pgfusepath{stroke}%
\end{pgfscope}%
\begin{pgfscope}%
\pgfpathrectangle{\pgfqpoint{0.100000in}{0.220728in}}{\pgfqpoint{3.696000in}{3.696000in}}%
\pgfusepath{clip}%
\pgfsetrectcap%
\pgfsetroundjoin%
\pgfsetlinewidth{1.505625pt}%
\definecolor{currentstroke}{rgb}{1.000000,0.000000,0.000000}%
\pgfsetstrokecolor{currentstroke}%
\pgfsetdash{}{0pt}%
\pgfpathmoveto{\pgfqpoint{1.988033in}{2.029989in}}%
\pgfpathlineto{\pgfqpoint{2.007134in}{2.130071in}}%
\pgfusepath{stroke}%
\end{pgfscope}%
\begin{pgfscope}%
\pgfpathrectangle{\pgfqpoint{0.100000in}{0.220728in}}{\pgfqpoint{3.696000in}{3.696000in}}%
\pgfusepath{clip}%
\pgfsetrectcap%
\pgfsetroundjoin%
\pgfsetlinewidth{1.505625pt}%
\definecolor{currentstroke}{rgb}{1.000000,0.000000,0.000000}%
\pgfsetstrokecolor{currentstroke}%
\pgfsetdash{}{0pt}%
\pgfpathmoveto{\pgfqpoint{1.989544in}{2.025101in}}%
\pgfpathlineto{\pgfqpoint{2.020289in}{2.126278in}}%
\pgfusepath{stroke}%
\end{pgfscope}%
\begin{pgfscope}%
\pgfpathrectangle{\pgfqpoint{0.100000in}{0.220728in}}{\pgfqpoint{3.696000in}{3.696000in}}%
\pgfusepath{clip}%
\pgfsetrectcap%
\pgfsetroundjoin%
\pgfsetlinewidth{1.505625pt}%
\definecolor{currentstroke}{rgb}{1.000000,0.000000,0.000000}%
\pgfsetstrokecolor{currentstroke}%
\pgfsetdash{}{0pt}%
\pgfpathmoveto{\pgfqpoint{1.993204in}{2.021857in}}%
\pgfpathlineto{\pgfqpoint{2.033454in}{2.122482in}}%
\pgfusepath{stroke}%
\end{pgfscope}%
\begin{pgfscope}%
\pgfpathrectangle{\pgfqpoint{0.100000in}{0.220728in}}{\pgfqpoint{3.696000in}{3.696000in}}%
\pgfusepath{clip}%
\pgfsetrectcap%
\pgfsetroundjoin%
\pgfsetlinewidth{1.505625pt}%
\definecolor{currentstroke}{rgb}{1.000000,0.000000,0.000000}%
\pgfsetstrokecolor{currentstroke}%
\pgfsetdash{}{0pt}%
\pgfpathmoveto{\pgfqpoint{1.997161in}{2.017443in}}%
\pgfpathlineto{\pgfqpoint{2.046627in}{2.118684in}}%
\pgfusepath{stroke}%
\end{pgfscope}%
\begin{pgfscope}%
\pgfpathrectangle{\pgfqpoint{0.100000in}{0.220728in}}{\pgfqpoint{3.696000in}{3.696000in}}%
\pgfusepath{clip}%
\pgfsetrectcap%
\pgfsetroundjoin%
\pgfsetlinewidth{1.505625pt}%
\definecolor{currentstroke}{rgb}{1.000000,0.000000,0.000000}%
\pgfsetstrokecolor{currentstroke}%
\pgfsetdash{}{0pt}%
\pgfpathmoveto{\pgfqpoint{1.997183in}{2.018914in}}%
\pgfpathlineto{\pgfqpoint{2.059809in}{2.114883in}}%
\pgfusepath{stroke}%
\end{pgfscope}%
\begin{pgfscope}%
\pgfpathrectangle{\pgfqpoint{0.100000in}{0.220728in}}{\pgfqpoint{3.696000in}{3.696000in}}%
\pgfusepath{clip}%
\pgfsetrectcap%
\pgfsetroundjoin%
\pgfsetlinewidth{1.505625pt}%
\definecolor{currentstroke}{rgb}{1.000000,0.000000,0.000000}%
\pgfsetstrokecolor{currentstroke}%
\pgfsetdash{}{0pt}%
\pgfpathmoveto{\pgfqpoint{1.998458in}{2.008750in}}%
\pgfpathlineto{\pgfqpoint{2.059809in}{2.114883in}}%
\pgfusepath{stroke}%
\end{pgfscope}%
\begin{pgfscope}%
\pgfpathrectangle{\pgfqpoint{0.100000in}{0.220728in}}{\pgfqpoint{3.696000in}{3.696000in}}%
\pgfusepath{clip}%
\pgfsetrectcap%
\pgfsetroundjoin%
\pgfsetlinewidth{1.505625pt}%
\definecolor{currentstroke}{rgb}{1.000000,0.000000,0.000000}%
\pgfsetstrokecolor{currentstroke}%
\pgfsetdash{}{0pt}%
\pgfpathmoveto{\pgfqpoint{2.000258in}{2.000980in}}%
\pgfpathlineto{\pgfqpoint{2.073000in}{2.111080in}}%
\pgfusepath{stroke}%
\end{pgfscope}%
\begin{pgfscope}%
\pgfpathrectangle{\pgfqpoint{0.100000in}{0.220728in}}{\pgfqpoint{3.696000in}{3.696000in}}%
\pgfusepath{clip}%
\pgfsetrectcap%
\pgfsetroundjoin%
\pgfsetlinewidth{1.505625pt}%
\definecolor{currentstroke}{rgb}{1.000000,0.000000,0.000000}%
\pgfsetstrokecolor{currentstroke}%
\pgfsetdash{}{0pt}%
\pgfpathmoveto{\pgfqpoint{1.999286in}{1.992932in}}%
\pgfpathlineto{\pgfqpoint{2.086199in}{2.107274in}}%
\pgfusepath{stroke}%
\end{pgfscope}%
\begin{pgfscope}%
\pgfpathrectangle{\pgfqpoint{0.100000in}{0.220728in}}{\pgfqpoint{3.696000in}{3.696000in}}%
\pgfusepath{clip}%
\pgfsetrectcap%
\pgfsetroundjoin%
\pgfsetlinewidth{1.505625pt}%
\definecolor{currentstroke}{rgb}{1.000000,0.000000,0.000000}%
\pgfsetstrokecolor{currentstroke}%
\pgfsetdash{}{0pt}%
\pgfpathmoveto{\pgfqpoint{2.003842in}{1.989936in}}%
\pgfpathlineto{\pgfqpoint{2.112624in}{2.099656in}}%
\pgfusepath{stroke}%
\end{pgfscope}%
\begin{pgfscope}%
\pgfpathrectangle{\pgfqpoint{0.100000in}{0.220728in}}{\pgfqpoint{3.696000in}{3.696000in}}%
\pgfusepath{clip}%
\pgfsetrectcap%
\pgfsetroundjoin%
\pgfsetlinewidth{1.505625pt}%
\definecolor{currentstroke}{rgb}{1.000000,0.000000,0.000000}%
\pgfsetstrokecolor{currentstroke}%
\pgfsetdash{}{0pt}%
\pgfpathmoveto{\pgfqpoint{2.005560in}{1.986504in}}%
\pgfpathlineto{\pgfqpoint{2.112624in}{2.099656in}}%
\pgfusepath{stroke}%
\end{pgfscope}%
\begin{pgfscope}%
\pgfpathrectangle{\pgfqpoint{0.100000in}{0.220728in}}{\pgfqpoint{3.696000in}{3.696000in}}%
\pgfusepath{clip}%
\pgfsetrectcap%
\pgfsetroundjoin%
\pgfsetlinewidth{1.505625pt}%
\definecolor{currentstroke}{rgb}{1.000000,0.000000,0.000000}%
\pgfsetstrokecolor{currentstroke}%
\pgfsetdash{}{0pt}%
\pgfpathmoveto{\pgfqpoint{2.005466in}{1.981702in}}%
\pgfpathlineto{\pgfqpoint{2.125849in}{2.095842in}}%
\pgfusepath{stroke}%
\end{pgfscope}%
\begin{pgfscope}%
\pgfpathrectangle{\pgfqpoint{0.100000in}{0.220728in}}{\pgfqpoint{3.696000in}{3.696000in}}%
\pgfusepath{clip}%
\pgfsetrectcap%
\pgfsetroundjoin%
\pgfsetlinewidth{1.505625pt}%
\definecolor{currentstroke}{rgb}{1.000000,0.000000,0.000000}%
\pgfsetstrokecolor{currentstroke}%
\pgfsetdash{}{0pt}%
\pgfpathmoveto{\pgfqpoint{2.008475in}{1.978511in}}%
\pgfpathlineto{\pgfqpoint{2.139083in}{2.092027in}}%
\pgfusepath{stroke}%
\end{pgfscope}%
\begin{pgfscope}%
\pgfpathrectangle{\pgfqpoint{0.100000in}{0.220728in}}{\pgfqpoint{3.696000in}{3.696000in}}%
\pgfusepath{clip}%
\pgfsetrectcap%
\pgfsetroundjoin%
\pgfsetlinewidth{1.505625pt}%
\definecolor{currentstroke}{rgb}{1.000000,0.000000,0.000000}%
\pgfsetstrokecolor{currentstroke}%
\pgfsetdash{}{0pt}%
\pgfpathmoveto{\pgfqpoint{2.013129in}{1.975989in}}%
\pgfpathlineto{\pgfqpoint{2.152326in}{2.088208in}}%
\pgfusepath{stroke}%
\end{pgfscope}%
\begin{pgfscope}%
\pgfpathrectangle{\pgfqpoint{0.100000in}{0.220728in}}{\pgfqpoint{3.696000in}{3.696000in}}%
\pgfusepath{clip}%
\pgfsetrectcap%
\pgfsetroundjoin%
\pgfsetlinewidth{1.505625pt}%
\definecolor{currentstroke}{rgb}{1.000000,0.000000,0.000000}%
\pgfsetstrokecolor{currentstroke}%
\pgfsetdash{}{0pt}%
\pgfpathmoveto{\pgfqpoint{2.013597in}{1.972250in}}%
\pgfpathlineto{\pgfqpoint{2.152326in}{2.088208in}}%
\pgfusepath{stroke}%
\end{pgfscope}%
\begin{pgfscope}%
\pgfpathrectangle{\pgfqpoint{0.100000in}{0.220728in}}{\pgfqpoint{3.696000in}{3.696000in}}%
\pgfusepath{clip}%
\pgfsetrectcap%
\pgfsetroundjoin%
\pgfsetlinewidth{1.505625pt}%
\definecolor{currentstroke}{rgb}{1.000000,0.000000,0.000000}%
\pgfsetstrokecolor{currentstroke}%
\pgfsetdash{}{0pt}%
\pgfpathmoveto{\pgfqpoint{2.014848in}{1.970595in}}%
\pgfpathlineto{\pgfqpoint{2.152326in}{2.088208in}}%
\pgfusepath{stroke}%
\end{pgfscope}%
\begin{pgfscope}%
\pgfpathrectangle{\pgfqpoint{0.100000in}{0.220728in}}{\pgfqpoint{3.696000in}{3.696000in}}%
\pgfusepath{clip}%
\pgfsetrectcap%
\pgfsetroundjoin%
\pgfsetlinewidth{1.505625pt}%
\definecolor{currentstroke}{rgb}{1.000000,0.000000,0.000000}%
\pgfsetstrokecolor{currentstroke}%
\pgfsetdash{}{0pt}%
\pgfpathmoveto{\pgfqpoint{2.016367in}{1.969539in}}%
\pgfpathlineto{\pgfqpoint{2.165578in}{2.084388in}}%
\pgfusepath{stroke}%
\end{pgfscope}%
\begin{pgfscope}%
\pgfpathrectangle{\pgfqpoint{0.100000in}{0.220728in}}{\pgfqpoint{3.696000in}{3.696000in}}%
\pgfusepath{clip}%
\pgfsetrectcap%
\pgfsetroundjoin%
\pgfsetlinewidth{1.505625pt}%
\definecolor{currentstroke}{rgb}{1.000000,0.000000,0.000000}%
\pgfsetstrokecolor{currentstroke}%
\pgfsetdash{}{0pt}%
\pgfpathmoveto{\pgfqpoint{2.017032in}{1.964945in}}%
\pgfpathlineto{\pgfqpoint{2.165578in}{2.084388in}}%
\pgfusepath{stroke}%
\end{pgfscope}%
\begin{pgfscope}%
\pgfpathrectangle{\pgfqpoint{0.100000in}{0.220728in}}{\pgfqpoint{3.696000in}{3.696000in}}%
\pgfusepath{clip}%
\pgfsetrectcap%
\pgfsetroundjoin%
\pgfsetlinewidth{1.505625pt}%
\definecolor{currentstroke}{rgb}{1.000000,0.000000,0.000000}%
\pgfsetstrokecolor{currentstroke}%
\pgfsetdash{}{0pt}%
\pgfpathmoveto{\pgfqpoint{2.018394in}{1.961172in}}%
\pgfpathlineto{\pgfqpoint{2.165578in}{2.084388in}}%
\pgfusepath{stroke}%
\end{pgfscope}%
\begin{pgfscope}%
\pgfpathrectangle{\pgfqpoint{0.100000in}{0.220728in}}{\pgfqpoint{3.696000in}{3.696000in}}%
\pgfusepath{clip}%
\pgfsetrectcap%
\pgfsetroundjoin%
\pgfsetlinewidth{1.505625pt}%
\definecolor{currentstroke}{rgb}{1.000000,0.000000,0.000000}%
\pgfsetstrokecolor{currentstroke}%
\pgfsetdash{}{0pt}%
\pgfpathmoveto{\pgfqpoint{2.021019in}{1.957963in}}%
\pgfpathlineto{\pgfqpoint{2.178838in}{2.080564in}}%
\pgfusepath{stroke}%
\end{pgfscope}%
\begin{pgfscope}%
\pgfpathrectangle{\pgfqpoint{0.100000in}{0.220728in}}{\pgfqpoint{3.696000in}{3.696000in}}%
\pgfusepath{clip}%
\pgfsetrectcap%
\pgfsetroundjoin%
\pgfsetlinewidth{1.505625pt}%
\definecolor{currentstroke}{rgb}{1.000000,0.000000,0.000000}%
\pgfsetstrokecolor{currentstroke}%
\pgfsetdash{}{0pt}%
\pgfpathmoveto{\pgfqpoint{2.022096in}{1.958180in}}%
\pgfpathlineto{\pgfqpoint{2.178838in}{2.080564in}}%
\pgfusepath{stroke}%
\end{pgfscope}%
\begin{pgfscope}%
\pgfpathrectangle{\pgfqpoint{0.100000in}{0.220728in}}{\pgfqpoint{3.696000in}{3.696000in}}%
\pgfusepath{clip}%
\pgfsetrectcap%
\pgfsetroundjoin%
\pgfsetlinewidth{1.505625pt}%
\definecolor{currentstroke}{rgb}{1.000000,0.000000,0.000000}%
\pgfsetstrokecolor{currentstroke}%
\pgfsetdash{}{0pt}%
\pgfpathmoveto{\pgfqpoint{2.022231in}{1.956741in}}%
\pgfpathlineto{\pgfqpoint{2.192107in}{2.076739in}}%
\pgfusepath{stroke}%
\end{pgfscope}%
\begin{pgfscope}%
\pgfpathrectangle{\pgfqpoint{0.100000in}{0.220728in}}{\pgfqpoint{3.696000in}{3.696000in}}%
\pgfusepath{clip}%
\pgfsetrectcap%
\pgfsetroundjoin%
\pgfsetlinewidth{1.505625pt}%
\definecolor{currentstroke}{rgb}{1.000000,0.000000,0.000000}%
\pgfsetstrokecolor{currentstroke}%
\pgfsetdash{}{0pt}%
\pgfpathmoveto{\pgfqpoint{2.022693in}{1.953937in}}%
\pgfpathlineto{\pgfqpoint{2.192107in}{2.076739in}}%
\pgfusepath{stroke}%
\end{pgfscope}%
\begin{pgfscope}%
\pgfpathrectangle{\pgfqpoint{0.100000in}{0.220728in}}{\pgfqpoint{3.696000in}{3.696000in}}%
\pgfusepath{clip}%
\pgfsetrectcap%
\pgfsetroundjoin%
\pgfsetlinewidth{1.505625pt}%
\definecolor{currentstroke}{rgb}{1.000000,0.000000,0.000000}%
\pgfsetstrokecolor{currentstroke}%
\pgfsetdash{}{0pt}%
\pgfpathmoveto{\pgfqpoint{2.023742in}{1.953502in}}%
\pgfpathlineto{\pgfqpoint{2.192107in}{2.076739in}}%
\pgfusepath{stroke}%
\end{pgfscope}%
\begin{pgfscope}%
\pgfpathrectangle{\pgfqpoint{0.100000in}{0.220728in}}{\pgfqpoint{3.696000in}{3.696000in}}%
\pgfusepath{clip}%
\pgfsetrectcap%
\pgfsetroundjoin%
\pgfsetlinewidth{1.505625pt}%
\definecolor{currentstroke}{rgb}{1.000000,0.000000,0.000000}%
\pgfsetstrokecolor{currentstroke}%
\pgfsetdash{}{0pt}%
\pgfpathmoveto{\pgfqpoint{2.024469in}{1.951777in}}%
\pgfpathlineto{\pgfqpoint{2.192107in}{2.076739in}}%
\pgfusepath{stroke}%
\end{pgfscope}%
\begin{pgfscope}%
\pgfpathrectangle{\pgfqpoint{0.100000in}{0.220728in}}{\pgfqpoint{3.696000in}{3.696000in}}%
\pgfusepath{clip}%
\pgfsetrectcap%
\pgfsetroundjoin%
\pgfsetlinewidth{1.505625pt}%
\definecolor{currentstroke}{rgb}{1.000000,0.000000,0.000000}%
\pgfsetstrokecolor{currentstroke}%
\pgfsetdash{}{0pt}%
\pgfpathmoveto{\pgfqpoint{2.024819in}{1.950820in}}%
\pgfpathlineto{\pgfqpoint{2.205385in}{2.072910in}}%
\pgfusepath{stroke}%
\end{pgfscope}%
\begin{pgfscope}%
\pgfpathrectangle{\pgfqpoint{0.100000in}{0.220728in}}{\pgfqpoint{3.696000in}{3.696000in}}%
\pgfusepath{clip}%
\pgfsetrectcap%
\pgfsetroundjoin%
\pgfsetlinewidth{1.505625pt}%
\definecolor{currentstroke}{rgb}{1.000000,0.000000,0.000000}%
\pgfsetstrokecolor{currentstroke}%
\pgfsetdash{}{0pt}%
\pgfpathmoveto{\pgfqpoint{2.025201in}{1.950305in}}%
\pgfpathlineto{\pgfqpoint{2.205385in}{2.072910in}}%
\pgfusepath{stroke}%
\end{pgfscope}%
\begin{pgfscope}%
\pgfpathrectangle{\pgfqpoint{0.100000in}{0.220728in}}{\pgfqpoint{3.696000in}{3.696000in}}%
\pgfusepath{clip}%
\pgfsetrectcap%
\pgfsetroundjoin%
\pgfsetlinewidth{1.505625pt}%
\definecolor{currentstroke}{rgb}{1.000000,0.000000,0.000000}%
\pgfsetstrokecolor{currentstroke}%
\pgfsetdash{}{0pt}%
\pgfpathmoveto{\pgfqpoint{2.025367in}{1.950473in}}%
\pgfpathlineto{\pgfqpoint{2.205385in}{2.072910in}}%
\pgfusepath{stroke}%
\end{pgfscope}%
\begin{pgfscope}%
\pgfpathrectangle{\pgfqpoint{0.100000in}{0.220728in}}{\pgfqpoint{3.696000in}{3.696000in}}%
\pgfusepath{clip}%
\pgfsetrectcap%
\pgfsetroundjoin%
\pgfsetlinewidth{1.505625pt}%
\definecolor{currentstroke}{rgb}{1.000000,0.000000,0.000000}%
\pgfsetstrokecolor{currentstroke}%
\pgfsetdash{}{0pt}%
\pgfpathmoveto{\pgfqpoint{2.025424in}{1.950260in}}%
\pgfpathlineto{\pgfqpoint{2.205385in}{2.072910in}}%
\pgfusepath{stroke}%
\end{pgfscope}%
\begin{pgfscope}%
\pgfpathrectangle{\pgfqpoint{0.100000in}{0.220728in}}{\pgfqpoint{3.696000in}{3.696000in}}%
\pgfusepath{clip}%
\pgfsetrectcap%
\pgfsetroundjoin%
\pgfsetlinewidth{1.505625pt}%
\definecolor{currentstroke}{rgb}{1.000000,0.000000,0.000000}%
\pgfsetstrokecolor{currentstroke}%
\pgfsetdash{}{0pt}%
\pgfpathmoveto{\pgfqpoint{2.025830in}{1.949206in}}%
\pgfpathlineto{\pgfqpoint{2.205385in}{2.072910in}}%
\pgfusepath{stroke}%
\end{pgfscope}%
\begin{pgfscope}%
\pgfpathrectangle{\pgfqpoint{0.100000in}{0.220728in}}{\pgfqpoint{3.696000in}{3.696000in}}%
\pgfusepath{clip}%
\pgfsetrectcap%
\pgfsetroundjoin%
\pgfsetlinewidth{1.505625pt}%
\definecolor{currentstroke}{rgb}{1.000000,0.000000,0.000000}%
\pgfsetstrokecolor{currentstroke}%
\pgfsetdash{}{0pt}%
\pgfpathmoveto{\pgfqpoint{2.026151in}{1.949226in}}%
\pgfpathlineto{\pgfqpoint{2.205385in}{2.072910in}}%
\pgfusepath{stroke}%
\end{pgfscope}%
\begin{pgfscope}%
\pgfpathrectangle{\pgfqpoint{0.100000in}{0.220728in}}{\pgfqpoint{3.696000in}{3.696000in}}%
\pgfusepath{clip}%
\pgfsetrectcap%
\pgfsetroundjoin%
\pgfsetlinewidth{1.505625pt}%
\definecolor{currentstroke}{rgb}{1.000000,0.000000,0.000000}%
\pgfsetstrokecolor{currentstroke}%
\pgfsetdash{}{0pt}%
\pgfpathmoveto{\pgfqpoint{2.026606in}{1.948669in}}%
\pgfpathlineto{\pgfqpoint{2.205385in}{2.072910in}}%
\pgfusepath{stroke}%
\end{pgfscope}%
\begin{pgfscope}%
\pgfpathrectangle{\pgfqpoint{0.100000in}{0.220728in}}{\pgfqpoint{3.696000in}{3.696000in}}%
\pgfusepath{clip}%
\pgfsetrectcap%
\pgfsetroundjoin%
\pgfsetlinewidth{1.505625pt}%
\definecolor{currentstroke}{rgb}{1.000000,0.000000,0.000000}%
\pgfsetstrokecolor{currentstroke}%
\pgfsetdash{}{0pt}%
\pgfpathmoveto{\pgfqpoint{2.027046in}{1.947760in}}%
\pgfpathlineto{\pgfqpoint{2.205385in}{2.072910in}}%
\pgfusepath{stroke}%
\end{pgfscope}%
\begin{pgfscope}%
\pgfpathrectangle{\pgfqpoint{0.100000in}{0.220728in}}{\pgfqpoint{3.696000in}{3.696000in}}%
\pgfusepath{clip}%
\pgfsetrectcap%
\pgfsetroundjoin%
\pgfsetlinewidth{1.505625pt}%
\definecolor{currentstroke}{rgb}{1.000000,0.000000,0.000000}%
\pgfsetstrokecolor{currentstroke}%
\pgfsetdash{}{0pt}%
\pgfpathmoveto{\pgfqpoint{2.027649in}{1.946234in}}%
\pgfpathlineto{\pgfqpoint{2.205385in}{2.072910in}}%
\pgfusepath{stroke}%
\end{pgfscope}%
\begin{pgfscope}%
\pgfpathrectangle{\pgfqpoint{0.100000in}{0.220728in}}{\pgfqpoint{3.696000in}{3.696000in}}%
\pgfusepath{clip}%
\pgfsetrectcap%
\pgfsetroundjoin%
\pgfsetlinewidth{1.505625pt}%
\definecolor{currentstroke}{rgb}{1.000000,0.000000,0.000000}%
\pgfsetstrokecolor{currentstroke}%
\pgfsetdash{}{0pt}%
\pgfpathmoveto{\pgfqpoint{2.029074in}{1.945864in}}%
\pgfpathlineto{\pgfqpoint{2.218672in}{2.069079in}}%
\pgfusepath{stroke}%
\end{pgfscope}%
\begin{pgfscope}%
\pgfpathrectangle{\pgfqpoint{0.100000in}{0.220728in}}{\pgfqpoint{3.696000in}{3.696000in}}%
\pgfusepath{clip}%
\pgfsetrectcap%
\pgfsetroundjoin%
\pgfsetlinewidth{1.505625pt}%
\definecolor{currentstroke}{rgb}{1.000000,0.000000,0.000000}%
\pgfsetstrokecolor{currentstroke}%
\pgfsetdash{}{0pt}%
\pgfpathmoveto{\pgfqpoint{2.029495in}{1.943397in}}%
\pgfpathlineto{\pgfqpoint{2.218672in}{2.069079in}}%
\pgfusepath{stroke}%
\end{pgfscope}%
\begin{pgfscope}%
\pgfpathrectangle{\pgfqpoint{0.100000in}{0.220728in}}{\pgfqpoint{3.696000in}{3.696000in}}%
\pgfusepath{clip}%
\pgfsetrectcap%
\pgfsetroundjoin%
\pgfsetlinewidth{1.505625pt}%
\definecolor{currentstroke}{rgb}{1.000000,0.000000,0.000000}%
\pgfsetstrokecolor{currentstroke}%
\pgfsetdash{}{0pt}%
\pgfpathmoveto{\pgfqpoint{2.030248in}{1.944746in}}%
\pgfpathlineto{\pgfqpoint{2.218672in}{2.069079in}}%
\pgfusepath{stroke}%
\end{pgfscope}%
\begin{pgfscope}%
\pgfpathrectangle{\pgfqpoint{0.100000in}{0.220728in}}{\pgfqpoint{3.696000in}{3.696000in}}%
\pgfusepath{clip}%
\pgfsetrectcap%
\pgfsetroundjoin%
\pgfsetlinewidth{1.505625pt}%
\definecolor{currentstroke}{rgb}{1.000000,0.000000,0.000000}%
\pgfsetstrokecolor{currentstroke}%
\pgfsetdash{}{0pt}%
\pgfpathmoveto{\pgfqpoint{2.031504in}{1.943840in}}%
\pgfpathlineto{\pgfqpoint{2.231967in}{2.065246in}}%
\pgfusepath{stroke}%
\end{pgfscope}%
\begin{pgfscope}%
\pgfpathrectangle{\pgfqpoint{0.100000in}{0.220728in}}{\pgfqpoint{3.696000in}{3.696000in}}%
\pgfusepath{clip}%
\pgfsetrectcap%
\pgfsetroundjoin%
\pgfsetlinewidth{1.505625pt}%
\definecolor{currentstroke}{rgb}{1.000000,0.000000,0.000000}%
\pgfsetstrokecolor{currentstroke}%
\pgfsetdash{}{0pt}%
\pgfpathmoveto{\pgfqpoint{2.032441in}{1.943265in}}%
\pgfpathlineto{\pgfqpoint{2.231967in}{2.065246in}}%
\pgfusepath{stroke}%
\end{pgfscope}%
\begin{pgfscope}%
\pgfpathrectangle{\pgfqpoint{0.100000in}{0.220728in}}{\pgfqpoint{3.696000in}{3.696000in}}%
\pgfusepath{clip}%
\pgfsetrectcap%
\pgfsetroundjoin%
\pgfsetlinewidth{1.505625pt}%
\definecolor{currentstroke}{rgb}{1.000000,0.000000,0.000000}%
\pgfsetstrokecolor{currentstroke}%
\pgfsetdash{}{0pt}%
\pgfpathmoveto{\pgfqpoint{2.032780in}{1.942176in}}%
\pgfpathlineto{\pgfqpoint{2.231967in}{2.065246in}}%
\pgfusepath{stroke}%
\end{pgfscope}%
\begin{pgfscope}%
\pgfpathrectangle{\pgfqpoint{0.100000in}{0.220728in}}{\pgfqpoint{3.696000in}{3.696000in}}%
\pgfusepath{clip}%
\pgfsetrectcap%
\pgfsetroundjoin%
\pgfsetlinewidth{1.505625pt}%
\definecolor{currentstroke}{rgb}{1.000000,0.000000,0.000000}%
\pgfsetstrokecolor{currentstroke}%
\pgfsetdash{}{0pt}%
\pgfpathmoveto{\pgfqpoint{2.033392in}{1.941529in}}%
\pgfpathlineto{\pgfqpoint{2.231967in}{2.065246in}}%
\pgfusepath{stroke}%
\end{pgfscope}%
\begin{pgfscope}%
\pgfpathrectangle{\pgfqpoint{0.100000in}{0.220728in}}{\pgfqpoint{3.696000in}{3.696000in}}%
\pgfusepath{clip}%
\pgfsetrectcap%
\pgfsetroundjoin%
\pgfsetlinewidth{1.505625pt}%
\definecolor{currentstroke}{rgb}{1.000000,0.000000,0.000000}%
\pgfsetstrokecolor{currentstroke}%
\pgfsetdash{}{0pt}%
\pgfpathmoveto{\pgfqpoint{2.033933in}{1.941781in}}%
\pgfpathlineto{\pgfqpoint{2.231967in}{2.065246in}}%
\pgfusepath{stroke}%
\end{pgfscope}%
\begin{pgfscope}%
\pgfpathrectangle{\pgfqpoint{0.100000in}{0.220728in}}{\pgfqpoint{3.696000in}{3.696000in}}%
\pgfusepath{clip}%
\pgfsetrectcap%
\pgfsetroundjoin%
\pgfsetlinewidth{1.505625pt}%
\definecolor{currentstroke}{rgb}{1.000000,0.000000,0.000000}%
\pgfsetstrokecolor{currentstroke}%
\pgfsetdash{}{0pt}%
\pgfpathmoveto{\pgfqpoint{2.033824in}{1.941155in}}%
\pgfpathlineto{\pgfqpoint{2.231967in}{2.065246in}}%
\pgfusepath{stroke}%
\end{pgfscope}%
\begin{pgfscope}%
\pgfpathrectangle{\pgfqpoint{0.100000in}{0.220728in}}{\pgfqpoint{3.696000in}{3.696000in}}%
\pgfusepath{clip}%
\pgfsetrectcap%
\pgfsetroundjoin%
\pgfsetlinewidth{1.505625pt}%
\definecolor{currentstroke}{rgb}{1.000000,0.000000,0.000000}%
\pgfsetstrokecolor{currentstroke}%
\pgfsetdash{}{0pt}%
\pgfpathmoveto{\pgfqpoint{2.034928in}{1.938712in}}%
\pgfpathlineto{\pgfqpoint{2.245272in}{2.061410in}}%
\pgfusepath{stroke}%
\end{pgfscope}%
\begin{pgfscope}%
\pgfpathrectangle{\pgfqpoint{0.100000in}{0.220728in}}{\pgfqpoint{3.696000in}{3.696000in}}%
\pgfusepath{clip}%
\pgfsetrectcap%
\pgfsetroundjoin%
\pgfsetlinewidth{1.505625pt}%
\definecolor{currentstroke}{rgb}{1.000000,0.000000,0.000000}%
\pgfsetstrokecolor{currentstroke}%
\pgfsetdash{}{0pt}%
\pgfpathmoveto{\pgfqpoint{2.035517in}{1.938727in}}%
\pgfpathlineto{\pgfqpoint{2.245272in}{2.061410in}}%
\pgfusepath{stroke}%
\end{pgfscope}%
\begin{pgfscope}%
\pgfpathrectangle{\pgfqpoint{0.100000in}{0.220728in}}{\pgfqpoint{3.696000in}{3.696000in}}%
\pgfusepath{clip}%
\pgfsetrectcap%
\pgfsetroundjoin%
\pgfsetlinewidth{1.505625pt}%
\definecolor{currentstroke}{rgb}{1.000000,0.000000,0.000000}%
\pgfsetstrokecolor{currentstroke}%
\pgfsetdash{}{0pt}%
\pgfpathmoveto{\pgfqpoint{2.035876in}{1.937043in}}%
\pgfpathlineto{\pgfqpoint{2.245272in}{2.061410in}}%
\pgfusepath{stroke}%
\end{pgfscope}%
\begin{pgfscope}%
\pgfpathrectangle{\pgfqpoint{0.100000in}{0.220728in}}{\pgfqpoint{3.696000in}{3.696000in}}%
\pgfusepath{clip}%
\pgfsetrectcap%
\pgfsetroundjoin%
\pgfsetlinewidth{1.505625pt}%
\definecolor{currentstroke}{rgb}{1.000000,0.000000,0.000000}%
\pgfsetstrokecolor{currentstroke}%
\pgfsetdash{}{0pt}%
\pgfpathmoveto{\pgfqpoint{2.036493in}{1.936647in}}%
\pgfpathlineto{\pgfqpoint{2.245272in}{2.061410in}}%
\pgfusepath{stroke}%
\end{pgfscope}%
\begin{pgfscope}%
\pgfpathrectangle{\pgfqpoint{0.100000in}{0.220728in}}{\pgfqpoint{3.696000in}{3.696000in}}%
\pgfusepath{clip}%
\pgfsetrectcap%
\pgfsetroundjoin%
\pgfsetlinewidth{1.505625pt}%
\definecolor{currentstroke}{rgb}{1.000000,0.000000,0.000000}%
\pgfsetstrokecolor{currentstroke}%
\pgfsetdash{}{0pt}%
\pgfpathmoveto{\pgfqpoint{2.037402in}{1.934313in}}%
\pgfpathlineto{\pgfqpoint{2.258585in}{2.057571in}}%
\pgfusepath{stroke}%
\end{pgfscope}%
\begin{pgfscope}%
\pgfpathrectangle{\pgfqpoint{0.100000in}{0.220728in}}{\pgfqpoint{3.696000in}{3.696000in}}%
\pgfusepath{clip}%
\pgfsetrectcap%
\pgfsetroundjoin%
\pgfsetlinewidth{1.505625pt}%
\definecolor{currentstroke}{rgb}{1.000000,0.000000,0.000000}%
\pgfsetstrokecolor{currentstroke}%
\pgfsetdash{}{0pt}%
\pgfpathmoveto{\pgfqpoint{2.038903in}{1.932732in}}%
\pgfpathlineto{\pgfqpoint{2.258585in}{2.057571in}}%
\pgfusepath{stroke}%
\end{pgfscope}%
\begin{pgfscope}%
\pgfpathrectangle{\pgfqpoint{0.100000in}{0.220728in}}{\pgfqpoint{3.696000in}{3.696000in}}%
\pgfusepath{clip}%
\pgfsetrectcap%
\pgfsetroundjoin%
\pgfsetlinewidth{1.505625pt}%
\definecolor{currentstroke}{rgb}{1.000000,0.000000,0.000000}%
\pgfsetstrokecolor{currentstroke}%
\pgfsetdash{}{0pt}%
\pgfpathmoveto{\pgfqpoint{2.040740in}{1.927987in}}%
\pgfpathlineto{\pgfqpoint{2.258585in}{2.057571in}}%
\pgfusepath{stroke}%
\end{pgfscope}%
\begin{pgfscope}%
\pgfpathrectangle{\pgfqpoint{0.100000in}{0.220728in}}{\pgfqpoint{3.696000in}{3.696000in}}%
\pgfusepath{clip}%
\pgfsetrectcap%
\pgfsetroundjoin%
\pgfsetlinewidth{1.505625pt}%
\definecolor{currentstroke}{rgb}{1.000000,0.000000,0.000000}%
\pgfsetstrokecolor{currentstroke}%
\pgfsetdash{}{0pt}%
\pgfpathmoveto{\pgfqpoint{2.041325in}{1.927827in}}%
\pgfpathlineto{\pgfqpoint{2.271907in}{2.053730in}}%
\pgfusepath{stroke}%
\end{pgfscope}%
\begin{pgfscope}%
\pgfpathrectangle{\pgfqpoint{0.100000in}{0.220728in}}{\pgfqpoint{3.696000in}{3.696000in}}%
\pgfusepath{clip}%
\pgfsetrectcap%
\pgfsetroundjoin%
\pgfsetlinewidth{1.505625pt}%
\definecolor{currentstroke}{rgb}{1.000000,0.000000,0.000000}%
\pgfsetstrokecolor{currentstroke}%
\pgfsetdash{}{0pt}%
\pgfpathmoveto{\pgfqpoint{2.042365in}{1.925532in}}%
\pgfpathlineto{\pgfqpoint{2.271907in}{2.053730in}}%
\pgfusepath{stroke}%
\end{pgfscope}%
\begin{pgfscope}%
\pgfpathrectangle{\pgfqpoint{0.100000in}{0.220728in}}{\pgfqpoint{3.696000in}{3.696000in}}%
\pgfusepath{clip}%
\pgfsetrectcap%
\pgfsetroundjoin%
\pgfsetlinewidth{1.505625pt}%
\definecolor{currentstroke}{rgb}{1.000000,0.000000,0.000000}%
\pgfsetstrokecolor{currentstroke}%
\pgfsetdash{}{0pt}%
\pgfpathmoveto{\pgfqpoint{2.044323in}{1.924951in}}%
\pgfpathlineto{\pgfqpoint{2.271907in}{2.053730in}}%
\pgfusepath{stroke}%
\end{pgfscope}%
\begin{pgfscope}%
\pgfpathrectangle{\pgfqpoint{0.100000in}{0.220728in}}{\pgfqpoint{3.696000in}{3.696000in}}%
\pgfusepath{clip}%
\pgfsetrectcap%
\pgfsetroundjoin%
\pgfsetlinewidth{1.505625pt}%
\definecolor{currentstroke}{rgb}{1.000000,0.000000,0.000000}%
\pgfsetstrokecolor{currentstroke}%
\pgfsetdash{}{0pt}%
\pgfpathmoveto{\pgfqpoint{2.044834in}{1.925506in}}%
\pgfpathlineto{\pgfqpoint{2.285238in}{2.049887in}}%
\pgfusepath{stroke}%
\end{pgfscope}%
\begin{pgfscope}%
\pgfpathrectangle{\pgfqpoint{0.100000in}{0.220728in}}{\pgfqpoint{3.696000in}{3.696000in}}%
\pgfusepath{clip}%
\pgfsetrectcap%
\pgfsetroundjoin%
\pgfsetlinewidth{1.505625pt}%
\definecolor{currentstroke}{rgb}{1.000000,0.000000,0.000000}%
\pgfsetstrokecolor{currentstroke}%
\pgfsetdash{}{0pt}%
\pgfpathmoveto{\pgfqpoint{2.047003in}{1.920720in}}%
\pgfpathlineto{\pgfqpoint{2.285238in}{2.049887in}}%
\pgfusepath{stroke}%
\end{pgfscope}%
\begin{pgfscope}%
\pgfpathrectangle{\pgfqpoint{0.100000in}{0.220728in}}{\pgfqpoint{3.696000in}{3.696000in}}%
\pgfusepath{clip}%
\pgfsetrectcap%
\pgfsetroundjoin%
\pgfsetlinewidth{1.505625pt}%
\definecolor{currentstroke}{rgb}{1.000000,0.000000,0.000000}%
\pgfsetstrokecolor{currentstroke}%
\pgfsetdash{}{0pt}%
\pgfpathmoveto{\pgfqpoint{2.049123in}{1.916744in}}%
\pgfpathlineto{\pgfqpoint{2.298577in}{2.046041in}}%
\pgfusepath{stroke}%
\end{pgfscope}%
\begin{pgfscope}%
\pgfpathrectangle{\pgfqpoint{0.100000in}{0.220728in}}{\pgfqpoint{3.696000in}{3.696000in}}%
\pgfusepath{clip}%
\pgfsetrectcap%
\pgfsetroundjoin%
\pgfsetlinewidth{1.505625pt}%
\definecolor{currentstroke}{rgb}{1.000000,0.000000,0.000000}%
\pgfsetstrokecolor{currentstroke}%
\pgfsetdash{}{0pt}%
\pgfpathmoveto{\pgfqpoint{2.049857in}{1.911505in}}%
\pgfpathlineto{\pgfqpoint{2.298577in}{2.046041in}}%
\pgfusepath{stroke}%
\end{pgfscope}%
\begin{pgfscope}%
\pgfpathrectangle{\pgfqpoint{0.100000in}{0.220728in}}{\pgfqpoint{3.696000in}{3.696000in}}%
\pgfusepath{clip}%
\pgfsetrectcap%
\pgfsetroundjoin%
\pgfsetlinewidth{1.505625pt}%
\definecolor{currentstroke}{rgb}{1.000000,0.000000,0.000000}%
\pgfsetstrokecolor{currentstroke}%
\pgfsetdash{}{0pt}%
\pgfpathmoveto{\pgfqpoint{2.052617in}{1.905379in}}%
\pgfpathlineto{\pgfqpoint{2.311926in}{2.042192in}}%
\pgfusepath{stroke}%
\end{pgfscope}%
\begin{pgfscope}%
\pgfpathrectangle{\pgfqpoint{0.100000in}{0.220728in}}{\pgfqpoint{3.696000in}{3.696000in}}%
\pgfusepath{clip}%
\pgfsetrectcap%
\pgfsetroundjoin%
\pgfsetlinewidth{1.505625pt}%
\definecolor{currentstroke}{rgb}{1.000000,0.000000,0.000000}%
\pgfsetstrokecolor{currentstroke}%
\pgfsetdash{}{0pt}%
\pgfpathmoveto{\pgfqpoint{2.053904in}{1.903517in}}%
\pgfpathlineto{\pgfqpoint{2.325283in}{2.038341in}}%
\pgfusepath{stroke}%
\end{pgfscope}%
\begin{pgfscope}%
\pgfpathrectangle{\pgfqpoint{0.100000in}{0.220728in}}{\pgfqpoint{3.696000in}{3.696000in}}%
\pgfusepath{clip}%
\pgfsetrectcap%
\pgfsetroundjoin%
\pgfsetlinewidth{1.505625pt}%
\definecolor{currentstroke}{rgb}{1.000000,0.000000,0.000000}%
\pgfsetstrokecolor{currentstroke}%
\pgfsetdash{}{0pt}%
\pgfpathmoveto{\pgfqpoint{2.056221in}{1.902409in}}%
\pgfpathlineto{\pgfqpoint{2.325283in}{2.038341in}}%
\pgfusepath{stroke}%
\end{pgfscope}%
\begin{pgfscope}%
\pgfpathrectangle{\pgfqpoint{0.100000in}{0.220728in}}{\pgfqpoint{3.696000in}{3.696000in}}%
\pgfusepath{clip}%
\pgfsetrectcap%
\pgfsetroundjoin%
\pgfsetlinewidth{1.505625pt}%
\definecolor{currentstroke}{rgb}{1.000000,0.000000,0.000000}%
\pgfsetstrokecolor{currentstroke}%
\pgfsetdash{}{0pt}%
\pgfpathmoveto{\pgfqpoint{2.058053in}{1.899983in}}%
\pgfpathlineto{\pgfqpoint{2.338649in}{2.034487in}}%
\pgfusepath{stroke}%
\end{pgfscope}%
\begin{pgfscope}%
\pgfpathrectangle{\pgfqpoint{0.100000in}{0.220728in}}{\pgfqpoint{3.696000in}{3.696000in}}%
\pgfusepath{clip}%
\pgfsetrectcap%
\pgfsetroundjoin%
\pgfsetlinewidth{1.505625pt}%
\definecolor{currentstroke}{rgb}{1.000000,0.000000,0.000000}%
\pgfsetstrokecolor{currentstroke}%
\pgfsetdash{}{0pt}%
\pgfpathmoveto{\pgfqpoint{2.059874in}{1.898453in}}%
\pgfpathlineto{\pgfqpoint{2.338649in}{2.034487in}}%
\pgfusepath{stroke}%
\end{pgfscope}%
\begin{pgfscope}%
\pgfpathrectangle{\pgfqpoint{0.100000in}{0.220728in}}{\pgfqpoint{3.696000in}{3.696000in}}%
\pgfusepath{clip}%
\pgfsetrectcap%
\pgfsetroundjoin%
\pgfsetlinewidth{1.505625pt}%
\definecolor{currentstroke}{rgb}{1.000000,0.000000,0.000000}%
\pgfsetstrokecolor{currentstroke}%
\pgfsetdash{}{0pt}%
\pgfpathmoveto{\pgfqpoint{2.064514in}{1.898598in}}%
\pgfpathlineto{\pgfqpoint{2.352024in}{2.030631in}}%
\pgfusepath{stroke}%
\end{pgfscope}%
\begin{pgfscope}%
\pgfpathrectangle{\pgfqpoint{0.100000in}{0.220728in}}{\pgfqpoint{3.696000in}{3.696000in}}%
\pgfusepath{clip}%
\pgfsetrectcap%
\pgfsetroundjoin%
\pgfsetlinewidth{1.505625pt}%
\definecolor{currentstroke}{rgb}{1.000000,0.000000,0.000000}%
\pgfsetstrokecolor{currentstroke}%
\pgfsetdash{}{0pt}%
\pgfpathmoveto{\pgfqpoint{2.065242in}{1.893104in}}%
\pgfpathlineto{\pgfqpoint{2.365408in}{2.026772in}}%
\pgfusepath{stroke}%
\end{pgfscope}%
\begin{pgfscope}%
\pgfpathrectangle{\pgfqpoint{0.100000in}{0.220728in}}{\pgfqpoint{3.696000in}{3.696000in}}%
\pgfusepath{clip}%
\pgfsetrectcap%
\pgfsetroundjoin%
\pgfsetlinewidth{1.505625pt}%
\definecolor{currentstroke}{rgb}{1.000000,0.000000,0.000000}%
\pgfsetstrokecolor{currentstroke}%
\pgfsetdash{}{0pt}%
\pgfpathmoveto{\pgfqpoint{2.067658in}{1.885766in}}%
\pgfpathlineto{\pgfqpoint{2.378801in}{2.022910in}}%
\pgfusepath{stroke}%
\end{pgfscope}%
\begin{pgfscope}%
\pgfpathrectangle{\pgfqpoint{0.100000in}{0.220728in}}{\pgfqpoint{3.696000in}{3.696000in}}%
\pgfusepath{clip}%
\pgfsetrectcap%
\pgfsetroundjoin%
\pgfsetlinewidth{1.505625pt}%
\definecolor{currentstroke}{rgb}{1.000000,0.000000,0.000000}%
\pgfsetstrokecolor{currentstroke}%
\pgfsetdash{}{0pt}%
\pgfpathmoveto{\pgfqpoint{2.070187in}{1.885193in}}%
\pgfpathlineto{\pgfqpoint{2.378801in}{2.022910in}}%
\pgfusepath{stroke}%
\end{pgfscope}%
\begin{pgfscope}%
\pgfpathrectangle{\pgfqpoint{0.100000in}{0.220728in}}{\pgfqpoint{3.696000in}{3.696000in}}%
\pgfusepath{clip}%
\pgfsetrectcap%
\pgfsetroundjoin%
\pgfsetlinewidth{1.505625pt}%
\definecolor{currentstroke}{rgb}{1.000000,0.000000,0.000000}%
\pgfsetstrokecolor{currentstroke}%
\pgfsetdash{}{0pt}%
\pgfpathmoveto{\pgfqpoint{2.072852in}{1.883095in}}%
\pgfpathlineto{\pgfqpoint{2.392203in}{2.019046in}}%
\pgfusepath{stroke}%
\end{pgfscope}%
\begin{pgfscope}%
\pgfpathrectangle{\pgfqpoint{0.100000in}{0.220728in}}{\pgfqpoint{3.696000in}{3.696000in}}%
\pgfusepath{clip}%
\pgfsetrectcap%
\pgfsetroundjoin%
\pgfsetlinewidth{1.505625pt}%
\definecolor{currentstroke}{rgb}{1.000000,0.000000,0.000000}%
\pgfsetstrokecolor{currentstroke}%
\pgfsetdash{}{0pt}%
\pgfpathmoveto{\pgfqpoint{2.073935in}{1.878880in}}%
\pgfpathlineto{\pgfqpoint{2.392203in}{2.019046in}}%
\pgfusepath{stroke}%
\end{pgfscope}%
\begin{pgfscope}%
\pgfpathrectangle{\pgfqpoint{0.100000in}{0.220728in}}{\pgfqpoint{3.696000in}{3.696000in}}%
\pgfusepath{clip}%
\pgfsetrectcap%
\pgfsetroundjoin%
\pgfsetlinewidth{1.505625pt}%
\definecolor{currentstroke}{rgb}{1.000000,0.000000,0.000000}%
\pgfsetstrokecolor{currentstroke}%
\pgfsetdash{}{0pt}%
\pgfpathmoveto{\pgfqpoint{2.076355in}{1.871598in}}%
\pgfpathlineto{\pgfqpoint{2.405613in}{2.015180in}}%
\pgfusepath{stroke}%
\end{pgfscope}%
\begin{pgfscope}%
\pgfpathrectangle{\pgfqpoint{0.100000in}{0.220728in}}{\pgfqpoint{3.696000in}{3.696000in}}%
\pgfusepath{clip}%
\pgfsetrectcap%
\pgfsetroundjoin%
\pgfsetlinewidth{1.505625pt}%
\definecolor{currentstroke}{rgb}{1.000000,0.000000,0.000000}%
\pgfsetstrokecolor{currentstroke}%
\pgfsetdash{}{0pt}%
\pgfpathmoveto{\pgfqpoint{2.079652in}{1.869792in}}%
\pgfpathlineto{\pgfqpoint{2.405613in}{2.015180in}}%
\pgfusepath{stroke}%
\end{pgfscope}%
\begin{pgfscope}%
\pgfpathrectangle{\pgfqpoint{0.100000in}{0.220728in}}{\pgfqpoint{3.696000in}{3.696000in}}%
\pgfusepath{clip}%
\pgfsetrectcap%
\pgfsetroundjoin%
\pgfsetlinewidth{1.505625pt}%
\definecolor{currentstroke}{rgb}{1.000000,0.000000,0.000000}%
\pgfsetstrokecolor{currentstroke}%
\pgfsetdash{}{0pt}%
\pgfpathmoveto{\pgfqpoint{2.083238in}{1.866382in}}%
\pgfpathlineto{\pgfqpoint{2.419033in}{2.011311in}}%
\pgfusepath{stroke}%
\end{pgfscope}%
\begin{pgfscope}%
\pgfpathrectangle{\pgfqpoint{0.100000in}{0.220728in}}{\pgfqpoint{3.696000in}{3.696000in}}%
\pgfusepath{clip}%
\pgfsetrectcap%
\pgfsetroundjoin%
\pgfsetlinewidth{1.505625pt}%
\definecolor{currentstroke}{rgb}{1.000000,0.000000,0.000000}%
\pgfsetstrokecolor{currentstroke}%
\pgfsetdash{}{0pt}%
\pgfpathmoveto{\pgfqpoint{2.084304in}{1.863021in}}%
\pgfpathlineto{\pgfqpoint{2.432461in}{2.007439in}}%
\pgfusepath{stroke}%
\end{pgfscope}%
\begin{pgfscope}%
\pgfpathrectangle{\pgfqpoint{0.100000in}{0.220728in}}{\pgfqpoint{3.696000in}{3.696000in}}%
\pgfusepath{clip}%
\pgfsetrectcap%
\pgfsetroundjoin%
\pgfsetlinewidth{1.505625pt}%
\definecolor{currentstroke}{rgb}{1.000000,0.000000,0.000000}%
\pgfsetstrokecolor{currentstroke}%
\pgfsetdash{}{0pt}%
\pgfpathmoveto{\pgfqpoint{2.088749in}{1.858278in}}%
\pgfpathlineto{\pgfqpoint{2.445899in}{2.003564in}}%
\pgfusepath{stroke}%
\end{pgfscope}%
\begin{pgfscope}%
\pgfpathrectangle{\pgfqpoint{0.100000in}{0.220728in}}{\pgfqpoint{3.696000in}{3.696000in}}%
\pgfusepath{clip}%
\pgfsetrectcap%
\pgfsetroundjoin%
\pgfsetlinewidth{1.505625pt}%
\definecolor{currentstroke}{rgb}{1.000000,0.000000,0.000000}%
\pgfsetstrokecolor{currentstroke}%
\pgfsetdash{}{0pt}%
\pgfpathmoveto{\pgfqpoint{2.090641in}{1.855230in}}%
\pgfpathlineto{\pgfqpoint{2.445899in}{2.003564in}}%
\pgfusepath{stroke}%
\end{pgfscope}%
\begin{pgfscope}%
\pgfpathrectangle{\pgfqpoint{0.100000in}{0.220728in}}{\pgfqpoint{3.696000in}{3.696000in}}%
\pgfusepath{clip}%
\pgfsetrectcap%
\pgfsetroundjoin%
\pgfsetlinewidth{1.505625pt}%
\definecolor{currentstroke}{rgb}{1.000000,0.000000,0.000000}%
\pgfsetstrokecolor{currentstroke}%
\pgfsetdash{}{0pt}%
\pgfpathmoveto{\pgfqpoint{2.093075in}{1.852990in}}%
\pgfpathlineto{\pgfqpoint{2.459345in}{1.999688in}}%
\pgfusepath{stroke}%
\end{pgfscope}%
\begin{pgfscope}%
\pgfpathrectangle{\pgfqpoint{0.100000in}{0.220728in}}{\pgfqpoint{3.696000in}{3.696000in}}%
\pgfusepath{clip}%
\pgfsetrectcap%
\pgfsetroundjoin%
\pgfsetlinewidth{1.505625pt}%
\definecolor{currentstroke}{rgb}{1.000000,0.000000,0.000000}%
\pgfsetstrokecolor{currentstroke}%
\pgfsetdash{}{0pt}%
\pgfpathmoveto{\pgfqpoint{2.094220in}{1.851056in}}%
\pgfpathlineto{\pgfqpoint{2.459345in}{1.999688in}}%
\pgfusepath{stroke}%
\end{pgfscope}%
\begin{pgfscope}%
\pgfpathrectangle{\pgfqpoint{0.100000in}{0.220728in}}{\pgfqpoint{3.696000in}{3.696000in}}%
\pgfusepath{clip}%
\pgfsetrectcap%
\pgfsetroundjoin%
\pgfsetlinewidth{1.505625pt}%
\definecolor{currentstroke}{rgb}{1.000000,0.000000,0.000000}%
\pgfsetstrokecolor{currentstroke}%
\pgfsetdash{}{0pt}%
\pgfpathmoveto{\pgfqpoint{2.094828in}{1.849637in}}%
\pgfpathlineto{\pgfqpoint{2.459345in}{1.999688in}}%
\pgfusepath{stroke}%
\end{pgfscope}%
\begin{pgfscope}%
\pgfpathrectangle{\pgfqpoint{0.100000in}{0.220728in}}{\pgfqpoint{3.696000in}{3.696000in}}%
\pgfusepath{clip}%
\pgfsetrectcap%
\pgfsetroundjoin%
\pgfsetlinewidth{1.505625pt}%
\definecolor{currentstroke}{rgb}{1.000000,0.000000,0.000000}%
\pgfsetstrokecolor{currentstroke}%
\pgfsetdash{}{0pt}%
\pgfpathmoveto{\pgfqpoint{2.096143in}{1.848288in}}%
\pgfpathlineto{\pgfqpoint{2.459345in}{1.999688in}}%
\pgfusepath{stroke}%
\end{pgfscope}%
\begin{pgfscope}%
\pgfpathrectangle{\pgfqpoint{0.100000in}{0.220728in}}{\pgfqpoint{3.696000in}{3.696000in}}%
\pgfusepath{clip}%
\pgfsetrectcap%
\pgfsetroundjoin%
\pgfsetlinewidth{1.505625pt}%
\definecolor{currentstroke}{rgb}{1.000000,0.000000,0.000000}%
\pgfsetstrokecolor{currentstroke}%
\pgfsetdash{}{0pt}%
\pgfpathmoveto{\pgfqpoint{2.096511in}{1.846539in}}%
\pgfpathlineto{\pgfqpoint{2.472800in}{1.995808in}}%
\pgfusepath{stroke}%
\end{pgfscope}%
\begin{pgfscope}%
\pgfpathrectangle{\pgfqpoint{0.100000in}{0.220728in}}{\pgfqpoint{3.696000in}{3.696000in}}%
\pgfusepath{clip}%
\pgfsetrectcap%
\pgfsetroundjoin%
\pgfsetlinewidth{1.505625pt}%
\definecolor{currentstroke}{rgb}{1.000000,0.000000,0.000000}%
\pgfsetstrokecolor{currentstroke}%
\pgfsetdash{}{0pt}%
\pgfpathmoveto{\pgfqpoint{2.098445in}{1.844073in}}%
\pgfpathlineto{\pgfqpoint{2.472800in}{1.995808in}}%
\pgfusepath{stroke}%
\end{pgfscope}%
\begin{pgfscope}%
\pgfpathrectangle{\pgfqpoint{0.100000in}{0.220728in}}{\pgfqpoint{3.696000in}{3.696000in}}%
\pgfusepath{clip}%
\pgfsetrectcap%
\pgfsetroundjoin%
\pgfsetlinewidth{1.505625pt}%
\definecolor{currentstroke}{rgb}{1.000000,0.000000,0.000000}%
\pgfsetstrokecolor{currentstroke}%
\pgfsetdash{}{0pt}%
\pgfpathmoveto{\pgfqpoint{2.100254in}{1.842157in}}%
\pgfpathlineto{\pgfqpoint{2.738021in}{1.462166in}}%
\pgfusepath{stroke}%
\end{pgfscope}%
\begin{pgfscope}%
\pgfpathrectangle{\pgfqpoint{0.100000in}{0.220728in}}{\pgfqpoint{3.696000in}{3.696000in}}%
\pgfusepath{clip}%
\pgfsetrectcap%
\pgfsetroundjoin%
\pgfsetlinewidth{1.505625pt}%
\definecolor{currentstroke}{rgb}{1.000000,0.000000,0.000000}%
\pgfsetstrokecolor{currentstroke}%
\pgfsetdash{}{0pt}%
\pgfpathmoveto{\pgfqpoint{2.102170in}{1.838008in}}%
\pgfpathlineto{\pgfqpoint{2.729696in}{1.454098in}}%
\pgfusepath{stroke}%
\end{pgfscope}%
\begin{pgfscope}%
\pgfpathrectangle{\pgfqpoint{0.100000in}{0.220728in}}{\pgfqpoint{3.696000in}{3.696000in}}%
\pgfusepath{clip}%
\pgfsetrectcap%
\pgfsetroundjoin%
\pgfsetlinewidth{1.505625pt}%
\definecolor{currentstroke}{rgb}{1.000000,0.000000,0.000000}%
\pgfsetstrokecolor{currentstroke}%
\pgfsetdash{}{0pt}%
\pgfpathmoveto{\pgfqpoint{2.104591in}{1.833392in}}%
\pgfpathlineto{\pgfqpoint{2.721360in}{1.446019in}}%
\pgfusepath{stroke}%
\end{pgfscope}%
\begin{pgfscope}%
\pgfpathrectangle{\pgfqpoint{0.100000in}{0.220728in}}{\pgfqpoint{3.696000in}{3.696000in}}%
\pgfusepath{clip}%
\pgfsetrectcap%
\pgfsetroundjoin%
\pgfsetlinewidth{1.505625pt}%
\definecolor{currentstroke}{rgb}{1.000000,0.000000,0.000000}%
\pgfsetstrokecolor{currentstroke}%
\pgfsetdash{}{0pt}%
\pgfpathmoveto{\pgfqpoint{2.107115in}{1.827141in}}%
\pgfpathlineto{\pgfqpoint{2.721360in}{1.446019in}}%
\pgfusepath{stroke}%
\end{pgfscope}%
\begin{pgfscope}%
\pgfpathrectangle{\pgfqpoint{0.100000in}{0.220728in}}{\pgfqpoint{3.696000in}{3.696000in}}%
\pgfusepath{clip}%
\pgfsetrectcap%
\pgfsetroundjoin%
\pgfsetlinewidth{1.505625pt}%
\definecolor{currentstroke}{rgb}{1.000000,0.000000,0.000000}%
\pgfsetstrokecolor{currentstroke}%
\pgfsetdash{}{0pt}%
\pgfpathmoveto{\pgfqpoint{2.112626in}{1.826516in}}%
\pgfpathlineto{\pgfqpoint{2.713013in}{1.437929in}}%
\pgfusepath{stroke}%
\end{pgfscope}%
\begin{pgfscope}%
\pgfpathrectangle{\pgfqpoint{0.100000in}{0.220728in}}{\pgfqpoint{3.696000in}{3.696000in}}%
\pgfusepath{clip}%
\pgfsetrectcap%
\pgfsetroundjoin%
\pgfsetlinewidth{1.505625pt}%
\definecolor{currentstroke}{rgb}{1.000000,0.000000,0.000000}%
\pgfsetstrokecolor{currentstroke}%
\pgfsetdash{}{0pt}%
\pgfpathmoveto{\pgfqpoint{2.113001in}{1.823922in}}%
\pgfpathlineto{\pgfqpoint{2.704654in}{1.429828in}}%
\pgfusepath{stroke}%
\end{pgfscope}%
\begin{pgfscope}%
\pgfpathrectangle{\pgfqpoint{0.100000in}{0.220728in}}{\pgfqpoint{3.696000in}{3.696000in}}%
\pgfusepath{clip}%
\pgfsetrectcap%
\pgfsetroundjoin%
\pgfsetlinewidth{1.505625pt}%
\definecolor{currentstroke}{rgb}{1.000000,0.000000,0.000000}%
\pgfsetstrokecolor{currentstroke}%
\pgfsetdash{}{0pt}%
\pgfpathmoveto{\pgfqpoint{2.115098in}{1.820366in}}%
\pgfpathlineto{\pgfqpoint{2.704654in}{1.429828in}}%
\pgfusepath{stroke}%
\end{pgfscope}%
\begin{pgfscope}%
\pgfpathrectangle{\pgfqpoint{0.100000in}{0.220728in}}{\pgfqpoint{3.696000in}{3.696000in}}%
\pgfusepath{clip}%
\pgfsetrectcap%
\pgfsetroundjoin%
\pgfsetlinewidth{1.505625pt}%
\definecolor{currentstroke}{rgb}{1.000000,0.000000,0.000000}%
\pgfsetstrokecolor{currentstroke}%
\pgfsetdash{}{0pt}%
\pgfpathmoveto{\pgfqpoint{2.117839in}{1.816787in}}%
\pgfpathlineto{\pgfqpoint{2.696285in}{1.421717in}}%
\pgfusepath{stroke}%
\end{pgfscope}%
\begin{pgfscope}%
\pgfpathrectangle{\pgfqpoint{0.100000in}{0.220728in}}{\pgfqpoint{3.696000in}{3.696000in}}%
\pgfusepath{clip}%
\pgfsetrectcap%
\pgfsetroundjoin%
\pgfsetlinewidth{1.505625pt}%
\definecolor{currentstroke}{rgb}{1.000000,0.000000,0.000000}%
\pgfsetstrokecolor{currentstroke}%
\pgfsetdash{}{0pt}%
\pgfpathmoveto{\pgfqpoint{2.119775in}{1.813906in}}%
\pgfpathlineto{\pgfqpoint{2.696285in}{1.421717in}}%
\pgfusepath{stroke}%
\end{pgfscope}%
\begin{pgfscope}%
\pgfpathrectangle{\pgfqpoint{0.100000in}{0.220728in}}{\pgfqpoint{3.696000in}{3.696000in}}%
\pgfusepath{clip}%
\pgfsetrectcap%
\pgfsetroundjoin%
\pgfsetlinewidth{1.505625pt}%
\definecolor{currentstroke}{rgb}{1.000000,0.000000,0.000000}%
\pgfsetstrokecolor{currentstroke}%
\pgfsetdash{}{0pt}%
\pgfpathmoveto{\pgfqpoint{2.120884in}{1.810700in}}%
\pgfpathlineto{\pgfqpoint{2.687905in}{1.413595in}}%
\pgfusepath{stroke}%
\end{pgfscope}%
\begin{pgfscope}%
\pgfpathrectangle{\pgfqpoint{0.100000in}{0.220728in}}{\pgfqpoint{3.696000in}{3.696000in}}%
\pgfusepath{clip}%
\pgfsetrectcap%
\pgfsetroundjoin%
\pgfsetlinewidth{1.505625pt}%
\definecolor{currentstroke}{rgb}{1.000000,0.000000,0.000000}%
\pgfsetstrokecolor{currentstroke}%
\pgfsetdash{}{0pt}%
\pgfpathmoveto{\pgfqpoint{2.124490in}{1.809098in}}%
\pgfpathlineto{\pgfqpoint{2.687905in}{1.413595in}}%
\pgfusepath{stroke}%
\end{pgfscope}%
\begin{pgfscope}%
\pgfpathrectangle{\pgfqpoint{0.100000in}{0.220728in}}{\pgfqpoint{3.696000in}{3.696000in}}%
\pgfusepath{clip}%
\pgfsetrectcap%
\pgfsetroundjoin%
\pgfsetlinewidth{1.505625pt}%
\definecolor{currentstroke}{rgb}{1.000000,0.000000,0.000000}%
\pgfsetstrokecolor{currentstroke}%
\pgfsetdash{}{0pt}%
\pgfpathmoveto{\pgfqpoint{2.127846in}{1.806053in}}%
\pgfpathlineto{\pgfqpoint{2.679513in}{1.405462in}}%
\pgfusepath{stroke}%
\end{pgfscope}%
\begin{pgfscope}%
\pgfpathrectangle{\pgfqpoint{0.100000in}{0.220728in}}{\pgfqpoint{3.696000in}{3.696000in}}%
\pgfusepath{clip}%
\pgfsetrectcap%
\pgfsetroundjoin%
\pgfsetlinewidth{1.505625pt}%
\definecolor{currentstroke}{rgb}{1.000000,0.000000,0.000000}%
\pgfsetstrokecolor{currentstroke}%
\pgfsetdash{}{0pt}%
\pgfpathmoveto{\pgfqpoint{2.128955in}{1.802851in}}%
\pgfpathlineto{\pgfqpoint{2.671110in}{1.397319in}}%
\pgfusepath{stroke}%
\end{pgfscope}%
\begin{pgfscope}%
\pgfpathrectangle{\pgfqpoint{0.100000in}{0.220728in}}{\pgfqpoint{3.696000in}{3.696000in}}%
\pgfusepath{clip}%
\pgfsetrectcap%
\pgfsetroundjoin%
\pgfsetlinewidth{1.505625pt}%
\definecolor{currentstroke}{rgb}{1.000000,0.000000,0.000000}%
\pgfsetstrokecolor{currentstroke}%
\pgfsetdash{}{0pt}%
\pgfpathmoveto{\pgfqpoint{2.129900in}{1.800981in}}%
\pgfpathlineto{\pgfqpoint{2.671110in}{1.397319in}}%
\pgfusepath{stroke}%
\end{pgfscope}%
\begin{pgfscope}%
\pgfpathrectangle{\pgfqpoint{0.100000in}{0.220728in}}{\pgfqpoint{3.696000in}{3.696000in}}%
\pgfusepath{clip}%
\pgfsetrectcap%
\pgfsetroundjoin%
\pgfsetlinewidth{1.505625pt}%
\definecolor{currentstroke}{rgb}{1.000000,0.000000,0.000000}%
\pgfsetstrokecolor{currentstroke}%
\pgfsetdash{}{0pt}%
\pgfpathmoveto{\pgfqpoint{2.130922in}{1.799635in}}%
\pgfpathlineto{\pgfqpoint{2.671110in}{1.397319in}}%
\pgfusepath{stroke}%
\end{pgfscope}%
\begin{pgfscope}%
\pgfpathrectangle{\pgfqpoint{0.100000in}{0.220728in}}{\pgfqpoint{3.696000in}{3.696000in}}%
\pgfusepath{clip}%
\pgfsetrectcap%
\pgfsetroundjoin%
\pgfsetlinewidth{1.505625pt}%
\definecolor{currentstroke}{rgb}{1.000000,0.000000,0.000000}%
\pgfsetstrokecolor{currentstroke}%
\pgfsetdash{}{0pt}%
\pgfpathmoveto{\pgfqpoint{2.132402in}{1.798808in}}%
\pgfpathlineto{\pgfqpoint{2.671110in}{1.397319in}}%
\pgfusepath{stroke}%
\end{pgfscope}%
\begin{pgfscope}%
\pgfpathrectangle{\pgfqpoint{0.100000in}{0.220728in}}{\pgfqpoint{3.696000in}{3.696000in}}%
\pgfusepath{clip}%
\pgfsetrectcap%
\pgfsetroundjoin%
\pgfsetlinewidth{1.505625pt}%
\definecolor{currentstroke}{rgb}{1.000000,0.000000,0.000000}%
\pgfsetstrokecolor{currentstroke}%
\pgfsetdash{}{0pt}%
\pgfpathmoveto{\pgfqpoint{2.133065in}{1.797438in}}%
\pgfpathlineto{\pgfqpoint{2.662697in}{1.389164in}}%
\pgfusepath{stroke}%
\end{pgfscope}%
\begin{pgfscope}%
\pgfpathrectangle{\pgfqpoint{0.100000in}{0.220728in}}{\pgfqpoint{3.696000in}{3.696000in}}%
\pgfusepath{clip}%
\pgfsetrectcap%
\pgfsetroundjoin%
\pgfsetlinewidth{1.505625pt}%
\definecolor{currentstroke}{rgb}{1.000000,0.000000,0.000000}%
\pgfsetstrokecolor{currentstroke}%
\pgfsetdash{}{0pt}%
\pgfpathmoveto{\pgfqpoint{2.133907in}{1.795479in}}%
\pgfpathlineto{\pgfqpoint{2.662697in}{1.389164in}}%
\pgfusepath{stroke}%
\end{pgfscope}%
\begin{pgfscope}%
\pgfpathrectangle{\pgfqpoint{0.100000in}{0.220728in}}{\pgfqpoint{3.696000in}{3.696000in}}%
\pgfusepath{clip}%
\pgfsetrectcap%
\pgfsetroundjoin%
\pgfsetlinewidth{1.505625pt}%
\definecolor{currentstroke}{rgb}{1.000000,0.000000,0.000000}%
\pgfsetstrokecolor{currentstroke}%
\pgfsetdash{}{0pt}%
\pgfpathmoveto{\pgfqpoint{2.135227in}{1.794062in}}%
\pgfpathlineto{\pgfqpoint{2.662697in}{1.389164in}}%
\pgfusepath{stroke}%
\end{pgfscope}%
\begin{pgfscope}%
\pgfpathrectangle{\pgfqpoint{0.100000in}{0.220728in}}{\pgfqpoint{3.696000in}{3.696000in}}%
\pgfusepath{clip}%
\pgfsetrectcap%
\pgfsetroundjoin%
\pgfsetlinewidth{1.505625pt}%
\definecolor{currentstroke}{rgb}{1.000000,0.000000,0.000000}%
\pgfsetstrokecolor{currentstroke}%
\pgfsetdash{}{0pt}%
\pgfpathmoveto{\pgfqpoint{2.137615in}{1.794248in}}%
\pgfpathlineto{\pgfqpoint{2.662697in}{1.389164in}}%
\pgfusepath{stroke}%
\end{pgfscope}%
\begin{pgfscope}%
\pgfpathrectangle{\pgfqpoint{0.100000in}{0.220728in}}{\pgfqpoint{3.696000in}{3.696000in}}%
\pgfusepath{clip}%
\pgfsetrectcap%
\pgfsetroundjoin%
\pgfsetlinewidth{1.505625pt}%
\definecolor{currentstroke}{rgb}{1.000000,0.000000,0.000000}%
\pgfsetstrokecolor{currentstroke}%
\pgfsetdash{}{0pt}%
\pgfpathmoveto{\pgfqpoint{2.138653in}{1.791107in}}%
\pgfpathlineto{\pgfqpoint{2.654272in}{1.380999in}}%
\pgfusepath{stroke}%
\end{pgfscope}%
\begin{pgfscope}%
\pgfpathrectangle{\pgfqpoint{0.100000in}{0.220728in}}{\pgfqpoint{3.696000in}{3.696000in}}%
\pgfusepath{clip}%
\pgfsetrectcap%
\pgfsetroundjoin%
\pgfsetlinewidth{1.505625pt}%
\definecolor{currentstroke}{rgb}{1.000000,0.000000,0.000000}%
\pgfsetstrokecolor{currentstroke}%
\pgfsetdash{}{0pt}%
\pgfpathmoveto{\pgfqpoint{2.141081in}{1.785250in}}%
\pgfpathlineto{\pgfqpoint{2.654272in}{1.380999in}}%
\pgfusepath{stroke}%
\end{pgfscope}%
\begin{pgfscope}%
\pgfpathrectangle{\pgfqpoint{0.100000in}{0.220728in}}{\pgfqpoint{3.696000in}{3.696000in}}%
\pgfusepath{clip}%
\pgfsetrectcap%
\pgfsetroundjoin%
\pgfsetlinewidth{1.505625pt}%
\definecolor{currentstroke}{rgb}{1.000000,0.000000,0.000000}%
\pgfsetstrokecolor{currentstroke}%
\pgfsetdash{}{0pt}%
\pgfpathmoveto{\pgfqpoint{2.143094in}{1.783079in}}%
\pgfpathlineto{\pgfqpoint{2.645835in}{1.372823in}}%
\pgfusepath{stroke}%
\end{pgfscope}%
\begin{pgfscope}%
\pgfpathrectangle{\pgfqpoint{0.100000in}{0.220728in}}{\pgfqpoint{3.696000in}{3.696000in}}%
\pgfusepath{clip}%
\pgfsetrectcap%
\pgfsetroundjoin%
\pgfsetlinewidth{1.505625pt}%
\definecolor{currentstroke}{rgb}{1.000000,0.000000,0.000000}%
\pgfsetstrokecolor{currentstroke}%
\pgfsetdash{}{0pt}%
\pgfpathmoveto{\pgfqpoint{2.145438in}{1.781583in}}%
\pgfpathlineto{\pgfqpoint{2.637388in}{1.364636in}}%
\pgfusepath{stroke}%
\end{pgfscope}%
\begin{pgfscope}%
\pgfpathrectangle{\pgfqpoint{0.100000in}{0.220728in}}{\pgfqpoint{3.696000in}{3.696000in}}%
\pgfusepath{clip}%
\pgfsetrectcap%
\pgfsetroundjoin%
\pgfsetlinewidth{1.505625pt}%
\definecolor{currentstroke}{rgb}{1.000000,0.000000,0.000000}%
\pgfsetstrokecolor{currentstroke}%
\pgfsetdash{}{0pt}%
\pgfpathmoveto{\pgfqpoint{2.147696in}{1.776908in}}%
\pgfpathlineto{\pgfqpoint{2.628929in}{1.356438in}}%
\pgfusepath{stroke}%
\end{pgfscope}%
\begin{pgfscope}%
\pgfpathrectangle{\pgfqpoint{0.100000in}{0.220728in}}{\pgfqpoint{3.696000in}{3.696000in}}%
\pgfusepath{clip}%
\pgfsetrectcap%
\pgfsetroundjoin%
\pgfsetlinewidth{1.505625pt}%
\definecolor{currentstroke}{rgb}{1.000000,0.000000,0.000000}%
\pgfsetstrokecolor{currentstroke}%
\pgfsetdash{}{0pt}%
\pgfpathmoveto{\pgfqpoint{2.151038in}{1.769817in}}%
\pgfpathlineto{\pgfqpoint{2.628929in}{1.356438in}}%
\pgfusepath{stroke}%
\end{pgfscope}%
\begin{pgfscope}%
\pgfpathrectangle{\pgfqpoint{0.100000in}{0.220728in}}{\pgfqpoint{3.696000in}{3.696000in}}%
\pgfusepath{clip}%
\pgfsetrectcap%
\pgfsetroundjoin%
\pgfsetlinewidth{1.505625pt}%
\definecolor{currentstroke}{rgb}{1.000000,0.000000,0.000000}%
\pgfsetstrokecolor{currentstroke}%
\pgfsetdash{}{0pt}%
\pgfpathmoveto{\pgfqpoint{2.157478in}{1.767600in}}%
\pgfpathlineto{\pgfqpoint{2.620459in}{1.348229in}}%
\pgfusepath{stroke}%
\end{pgfscope}%
\begin{pgfscope}%
\pgfpathrectangle{\pgfqpoint{0.100000in}{0.220728in}}{\pgfqpoint{3.696000in}{3.696000in}}%
\pgfusepath{clip}%
\pgfsetrectcap%
\pgfsetroundjoin%
\pgfsetlinewidth{1.505625pt}%
\definecolor{currentstroke}{rgb}{1.000000,0.000000,0.000000}%
\pgfsetstrokecolor{currentstroke}%
\pgfsetdash{}{0pt}%
\pgfpathmoveto{\pgfqpoint{2.160594in}{1.760511in}}%
\pgfpathlineto{\pgfqpoint{2.611977in}{1.340009in}}%
\pgfusepath{stroke}%
\end{pgfscope}%
\begin{pgfscope}%
\pgfpathrectangle{\pgfqpoint{0.100000in}{0.220728in}}{\pgfqpoint{3.696000in}{3.696000in}}%
\pgfusepath{clip}%
\pgfsetrectcap%
\pgfsetroundjoin%
\pgfsetlinewidth{1.505625pt}%
\definecolor{currentstroke}{rgb}{1.000000,0.000000,0.000000}%
\pgfsetstrokecolor{currentstroke}%
\pgfsetdash{}{0pt}%
\pgfpathmoveto{\pgfqpoint{2.163139in}{1.748536in}}%
\pgfpathlineto{\pgfqpoint{2.603485in}{1.331778in}}%
\pgfusepath{stroke}%
\end{pgfscope}%
\begin{pgfscope}%
\pgfpathrectangle{\pgfqpoint{0.100000in}{0.220728in}}{\pgfqpoint{3.696000in}{3.696000in}}%
\pgfusepath{clip}%
\pgfsetrectcap%
\pgfsetroundjoin%
\pgfsetlinewidth{1.505625pt}%
\definecolor{currentstroke}{rgb}{1.000000,0.000000,0.000000}%
\pgfsetstrokecolor{currentstroke}%
\pgfsetdash{}{0pt}%
\pgfpathmoveto{\pgfqpoint{2.167626in}{1.736114in}}%
\pgfpathlineto{\pgfqpoint{2.586465in}{1.315283in}}%
\pgfusepath{stroke}%
\end{pgfscope}%
\begin{pgfscope}%
\pgfpathrectangle{\pgfqpoint{0.100000in}{0.220728in}}{\pgfqpoint{3.696000in}{3.696000in}}%
\pgfusepath{clip}%
\pgfsetrectcap%
\pgfsetroundjoin%
\pgfsetlinewidth{1.505625pt}%
\definecolor{currentstroke}{rgb}{1.000000,0.000000,0.000000}%
\pgfsetstrokecolor{currentstroke}%
\pgfsetdash{}{0pt}%
\pgfpathmoveto{\pgfqpoint{2.175747in}{1.736918in}}%
\pgfpathlineto{\pgfqpoint{2.577938in}{1.307019in}}%
\pgfusepath{stroke}%
\end{pgfscope}%
\begin{pgfscope}%
\pgfpathrectangle{\pgfqpoint{0.100000in}{0.220728in}}{\pgfqpoint{3.696000in}{3.696000in}}%
\pgfusepath{clip}%
\pgfsetrectcap%
\pgfsetroundjoin%
\pgfsetlinewidth{1.505625pt}%
\definecolor{currentstroke}{rgb}{1.000000,0.000000,0.000000}%
\pgfsetstrokecolor{currentstroke}%
\pgfsetdash{}{0pt}%
\pgfpathmoveto{\pgfqpoint{2.178160in}{1.732275in}}%
\pgfpathlineto{\pgfqpoint{2.577938in}{1.307019in}}%
\pgfusepath{stroke}%
\end{pgfscope}%
\begin{pgfscope}%
\pgfpathrectangle{\pgfqpoint{0.100000in}{0.220728in}}{\pgfqpoint{3.696000in}{3.696000in}}%
\pgfusepath{clip}%
\pgfsetrectcap%
\pgfsetroundjoin%
\pgfsetlinewidth{1.505625pt}%
\definecolor{currentstroke}{rgb}{1.000000,0.000000,0.000000}%
\pgfsetstrokecolor{currentstroke}%
\pgfsetdash{}{0pt}%
\pgfpathmoveto{\pgfqpoint{2.181012in}{1.722825in}}%
\pgfpathlineto{\pgfqpoint{2.569400in}{1.298744in}}%
\pgfusepath{stroke}%
\end{pgfscope}%
\begin{pgfscope}%
\pgfpathrectangle{\pgfqpoint{0.100000in}{0.220728in}}{\pgfqpoint{3.696000in}{3.696000in}}%
\pgfusepath{clip}%
\pgfsetrectcap%
\pgfsetroundjoin%
\pgfsetlinewidth{1.505625pt}%
\definecolor{currentstroke}{rgb}{1.000000,0.000000,0.000000}%
\pgfsetstrokecolor{currentstroke}%
\pgfsetdash{}{0pt}%
\pgfpathmoveto{\pgfqpoint{2.183147in}{1.719536in}}%
\pgfpathlineto{\pgfqpoint{2.569400in}{1.298744in}}%
\pgfusepath{stroke}%
\end{pgfscope}%
\begin{pgfscope}%
\pgfpathrectangle{\pgfqpoint{0.100000in}{0.220728in}}{\pgfqpoint{3.696000in}{3.696000in}}%
\pgfusepath{clip}%
\pgfsetrectcap%
\pgfsetroundjoin%
\pgfsetlinewidth{1.505625pt}%
\definecolor{currentstroke}{rgb}{1.000000,0.000000,0.000000}%
\pgfsetstrokecolor{currentstroke}%
\pgfsetdash{}{0pt}%
\pgfpathmoveto{\pgfqpoint{2.184324in}{1.719139in}}%
\pgfpathlineto{\pgfqpoint{2.560850in}{1.290457in}}%
\pgfusepath{stroke}%
\end{pgfscope}%
\begin{pgfscope}%
\pgfpathrectangle{\pgfqpoint{0.100000in}{0.220728in}}{\pgfqpoint{3.696000in}{3.696000in}}%
\pgfusepath{clip}%
\pgfsetrectcap%
\pgfsetroundjoin%
\pgfsetlinewidth{1.505625pt}%
\definecolor{currentstroke}{rgb}{1.000000,0.000000,0.000000}%
\pgfsetstrokecolor{currentstroke}%
\pgfsetdash{}{0pt}%
\pgfpathmoveto{\pgfqpoint{2.184590in}{1.717829in}}%
\pgfpathlineto{\pgfqpoint{2.560850in}{1.290457in}}%
\pgfusepath{stroke}%
\end{pgfscope}%
\begin{pgfscope}%
\pgfpathrectangle{\pgfqpoint{0.100000in}{0.220728in}}{\pgfqpoint{3.696000in}{3.696000in}}%
\pgfusepath{clip}%
\pgfsetrectcap%
\pgfsetroundjoin%
\pgfsetlinewidth{1.505625pt}%
\definecolor{currentstroke}{rgb}{1.000000,0.000000,0.000000}%
\pgfsetstrokecolor{currentstroke}%
\pgfsetdash{}{0pt}%
\pgfpathmoveto{\pgfqpoint{2.185240in}{1.715785in}}%
\pgfpathlineto{\pgfqpoint{2.560850in}{1.290457in}}%
\pgfusepath{stroke}%
\end{pgfscope}%
\begin{pgfscope}%
\pgfpathrectangle{\pgfqpoint{0.100000in}{0.220728in}}{\pgfqpoint{3.696000in}{3.696000in}}%
\pgfusepath{clip}%
\pgfsetrectcap%
\pgfsetroundjoin%
\pgfsetlinewidth{1.505625pt}%
\definecolor{currentstroke}{rgb}{1.000000,0.000000,0.000000}%
\pgfsetstrokecolor{currentstroke}%
\pgfsetdash{}{0pt}%
\pgfpathmoveto{\pgfqpoint{2.185626in}{1.715008in}}%
\pgfpathlineto{\pgfqpoint{2.560850in}{1.290457in}}%
\pgfusepath{stroke}%
\end{pgfscope}%
\begin{pgfscope}%
\pgfpathrectangle{\pgfqpoint{0.100000in}{0.220728in}}{\pgfqpoint{3.696000in}{3.696000in}}%
\pgfusepath{clip}%
\pgfsetrectcap%
\pgfsetroundjoin%
\pgfsetlinewidth{1.505625pt}%
\definecolor{currentstroke}{rgb}{1.000000,0.000000,0.000000}%
\pgfsetstrokecolor{currentstroke}%
\pgfsetdash{}{0pt}%
\pgfpathmoveto{\pgfqpoint{2.186279in}{1.715082in}}%
\pgfpathlineto{\pgfqpoint{2.560850in}{1.290457in}}%
\pgfusepath{stroke}%
\end{pgfscope}%
\begin{pgfscope}%
\pgfpathrectangle{\pgfqpoint{0.100000in}{0.220728in}}{\pgfqpoint{3.696000in}{3.696000in}}%
\pgfusepath{clip}%
\pgfsetrectcap%
\pgfsetroundjoin%
\pgfsetlinewidth{1.505625pt}%
\definecolor{currentstroke}{rgb}{1.000000,0.000000,0.000000}%
\pgfsetstrokecolor{currentstroke}%
\pgfsetdash{}{0pt}%
\pgfpathmoveto{\pgfqpoint{2.186446in}{1.714148in}}%
\pgfpathlineto{\pgfqpoint{2.560850in}{1.290457in}}%
\pgfusepath{stroke}%
\end{pgfscope}%
\begin{pgfscope}%
\pgfpathrectangle{\pgfqpoint{0.100000in}{0.220728in}}{\pgfqpoint{3.696000in}{3.696000in}}%
\pgfusepath{clip}%
\pgfsetrectcap%
\pgfsetroundjoin%
\pgfsetlinewidth{1.505625pt}%
\definecolor{currentstroke}{rgb}{1.000000,0.000000,0.000000}%
\pgfsetstrokecolor{currentstroke}%
\pgfsetdash{}{0pt}%
\pgfpathmoveto{\pgfqpoint{2.187341in}{1.712318in}}%
\pgfpathlineto{\pgfqpoint{2.560850in}{1.290457in}}%
\pgfusepath{stroke}%
\end{pgfscope}%
\begin{pgfscope}%
\pgfpathrectangle{\pgfqpoint{0.100000in}{0.220728in}}{\pgfqpoint{3.696000in}{3.696000in}}%
\pgfusepath{clip}%
\pgfsetrectcap%
\pgfsetroundjoin%
\pgfsetlinewidth{1.505625pt}%
\definecolor{currentstroke}{rgb}{1.000000,0.000000,0.000000}%
\pgfsetstrokecolor{currentstroke}%
\pgfsetdash{}{0pt}%
\pgfpathmoveto{\pgfqpoint{2.187629in}{1.711889in}}%
\pgfpathlineto{\pgfqpoint{2.560850in}{1.290457in}}%
\pgfusepath{stroke}%
\end{pgfscope}%
\begin{pgfscope}%
\pgfpathrectangle{\pgfqpoint{0.100000in}{0.220728in}}{\pgfqpoint{3.696000in}{3.696000in}}%
\pgfusepath{clip}%
\pgfsetrectcap%
\pgfsetroundjoin%
\pgfsetlinewidth{1.505625pt}%
\definecolor{currentstroke}{rgb}{1.000000,0.000000,0.000000}%
\pgfsetstrokecolor{currentstroke}%
\pgfsetdash{}{0pt}%
\pgfpathmoveto{\pgfqpoint{2.188565in}{1.712171in}}%
\pgfpathlineto{\pgfqpoint{2.560850in}{1.290457in}}%
\pgfusepath{stroke}%
\end{pgfscope}%
\begin{pgfscope}%
\pgfpathrectangle{\pgfqpoint{0.100000in}{0.220728in}}{\pgfqpoint{3.696000in}{3.696000in}}%
\pgfusepath{clip}%
\pgfsetrectcap%
\pgfsetroundjoin%
\pgfsetlinewidth{1.505625pt}%
\definecolor{currentstroke}{rgb}{1.000000,0.000000,0.000000}%
\pgfsetstrokecolor{currentstroke}%
\pgfsetdash{}{0pt}%
\pgfpathmoveto{\pgfqpoint{2.189278in}{1.710212in}}%
\pgfpathlineto{\pgfqpoint{2.552288in}{1.282160in}}%
\pgfusepath{stroke}%
\end{pgfscope}%
\begin{pgfscope}%
\pgfpathrectangle{\pgfqpoint{0.100000in}{0.220728in}}{\pgfqpoint{3.696000in}{3.696000in}}%
\pgfusepath{clip}%
\pgfsetrectcap%
\pgfsetroundjoin%
\pgfsetlinewidth{1.505625pt}%
\definecolor{currentstroke}{rgb}{1.000000,0.000000,0.000000}%
\pgfsetstrokecolor{currentstroke}%
\pgfsetdash{}{0pt}%
\pgfpathmoveto{\pgfqpoint{2.190480in}{1.706953in}}%
\pgfpathlineto{\pgfqpoint{2.552288in}{1.282160in}}%
\pgfusepath{stroke}%
\end{pgfscope}%
\begin{pgfscope}%
\pgfpathrectangle{\pgfqpoint{0.100000in}{0.220728in}}{\pgfqpoint{3.696000in}{3.696000in}}%
\pgfusepath{clip}%
\pgfsetrectcap%
\pgfsetroundjoin%
\pgfsetlinewidth{1.505625pt}%
\definecolor{currentstroke}{rgb}{1.000000,0.000000,0.000000}%
\pgfsetstrokecolor{currentstroke}%
\pgfsetdash{}{0pt}%
\pgfpathmoveto{\pgfqpoint{2.191485in}{1.707304in}}%
\pgfpathlineto{\pgfqpoint{2.552288in}{1.282160in}}%
\pgfusepath{stroke}%
\end{pgfscope}%
\begin{pgfscope}%
\pgfpathrectangle{\pgfqpoint{0.100000in}{0.220728in}}{\pgfqpoint{3.696000in}{3.696000in}}%
\pgfusepath{clip}%
\pgfsetrectcap%
\pgfsetroundjoin%
\pgfsetlinewidth{1.505625pt}%
\definecolor{currentstroke}{rgb}{1.000000,0.000000,0.000000}%
\pgfsetstrokecolor{currentstroke}%
\pgfsetdash{}{0pt}%
\pgfpathmoveto{\pgfqpoint{2.192849in}{1.705588in}}%
\pgfpathlineto{\pgfqpoint{2.552288in}{1.282160in}}%
\pgfusepath{stroke}%
\end{pgfscope}%
\begin{pgfscope}%
\pgfpathrectangle{\pgfqpoint{0.100000in}{0.220728in}}{\pgfqpoint{3.696000in}{3.696000in}}%
\pgfusepath{clip}%
\pgfsetrectcap%
\pgfsetroundjoin%
\pgfsetlinewidth{1.505625pt}%
\definecolor{currentstroke}{rgb}{1.000000,0.000000,0.000000}%
\pgfsetstrokecolor{currentstroke}%
\pgfsetdash{}{0pt}%
\pgfpathmoveto{\pgfqpoint{2.194345in}{1.700684in}}%
\pgfpathlineto{\pgfqpoint{2.543716in}{1.273852in}}%
\pgfusepath{stroke}%
\end{pgfscope}%
\begin{pgfscope}%
\pgfpathrectangle{\pgfqpoint{0.100000in}{0.220728in}}{\pgfqpoint{3.696000in}{3.696000in}}%
\pgfusepath{clip}%
\pgfsetrectcap%
\pgfsetroundjoin%
\pgfsetlinewidth{1.505625pt}%
\definecolor{currentstroke}{rgb}{1.000000,0.000000,0.000000}%
\pgfsetstrokecolor{currentstroke}%
\pgfsetdash{}{0pt}%
\pgfpathmoveto{\pgfqpoint{2.196074in}{1.695489in}}%
\pgfpathlineto{\pgfqpoint{2.543716in}{1.273852in}}%
\pgfusepath{stroke}%
\end{pgfscope}%
\begin{pgfscope}%
\pgfpathrectangle{\pgfqpoint{0.100000in}{0.220728in}}{\pgfqpoint{3.696000in}{3.696000in}}%
\pgfusepath{clip}%
\pgfsetrectcap%
\pgfsetroundjoin%
\pgfsetlinewidth{1.505625pt}%
\definecolor{currentstroke}{rgb}{1.000000,0.000000,0.000000}%
\pgfsetstrokecolor{currentstroke}%
\pgfsetdash{}{0pt}%
\pgfpathmoveto{\pgfqpoint{2.199476in}{1.697942in}}%
\pgfpathlineto{\pgfqpoint{2.535131in}{1.265532in}}%
\pgfusepath{stroke}%
\end{pgfscope}%
\begin{pgfscope}%
\pgfpathrectangle{\pgfqpoint{0.100000in}{0.220728in}}{\pgfqpoint{3.696000in}{3.696000in}}%
\pgfusepath{clip}%
\pgfsetrectcap%
\pgfsetroundjoin%
\pgfsetlinewidth{1.505625pt}%
\definecolor{currentstroke}{rgb}{1.000000,0.000000,0.000000}%
\pgfsetstrokecolor{currentstroke}%
\pgfsetdash{}{0pt}%
\pgfpathmoveto{\pgfqpoint{2.199916in}{1.695569in}}%
\pgfpathlineto{\pgfqpoint{2.535131in}{1.265532in}}%
\pgfusepath{stroke}%
\end{pgfscope}%
\begin{pgfscope}%
\pgfpathrectangle{\pgfqpoint{0.100000in}{0.220728in}}{\pgfqpoint{3.696000in}{3.696000in}}%
\pgfusepath{clip}%
\pgfsetrectcap%
\pgfsetroundjoin%
\pgfsetlinewidth{1.505625pt}%
\definecolor{currentstroke}{rgb}{1.000000,0.000000,0.000000}%
\pgfsetstrokecolor{currentstroke}%
\pgfsetdash{}{0pt}%
\pgfpathmoveto{\pgfqpoint{2.201375in}{1.691138in}}%
\pgfpathlineto{\pgfqpoint{2.535131in}{1.265532in}}%
\pgfusepath{stroke}%
\end{pgfscope}%
\begin{pgfscope}%
\pgfpathrectangle{\pgfqpoint{0.100000in}{0.220728in}}{\pgfqpoint{3.696000in}{3.696000in}}%
\pgfusepath{clip}%
\pgfsetrectcap%
\pgfsetroundjoin%
\pgfsetlinewidth{1.505625pt}%
\definecolor{currentstroke}{rgb}{1.000000,0.000000,0.000000}%
\pgfsetstrokecolor{currentstroke}%
\pgfsetdash{}{0pt}%
\pgfpathmoveto{\pgfqpoint{2.202028in}{1.689964in}}%
\pgfpathlineto{\pgfqpoint{2.535131in}{1.265532in}}%
\pgfusepath{stroke}%
\end{pgfscope}%
\begin{pgfscope}%
\pgfpathrectangle{\pgfqpoint{0.100000in}{0.220728in}}{\pgfqpoint{3.696000in}{3.696000in}}%
\pgfusepath{clip}%
\pgfsetrectcap%
\pgfsetroundjoin%
\pgfsetlinewidth{1.505625pt}%
\definecolor{currentstroke}{rgb}{1.000000,0.000000,0.000000}%
\pgfsetstrokecolor{currentstroke}%
\pgfsetdash{}{0pt}%
\pgfpathmoveto{\pgfqpoint{2.203514in}{1.690894in}}%
\pgfpathlineto{\pgfqpoint{2.526536in}{1.257201in}}%
\pgfusepath{stroke}%
\end{pgfscope}%
\begin{pgfscope}%
\pgfpathrectangle{\pgfqpoint{0.100000in}{0.220728in}}{\pgfqpoint{3.696000in}{3.696000in}}%
\pgfusepath{clip}%
\pgfsetrectcap%
\pgfsetroundjoin%
\pgfsetlinewidth{1.505625pt}%
\definecolor{currentstroke}{rgb}{1.000000,0.000000,0.000000}%
\pgfsetstrokecolor{currentstroke}%
\pgfsetdash{}{0pt}%
\pgfpathmoveto{\pgfqpoint{2.203769in}{1.689480in}}%
\pgfpathlineto{\pgfqpoint{2.526536in}{1.257201in}}%
\pgfusepath{stroke}%
\end{pgfscope}%
\begin{pgfscope}%
\pgfpathrectangle{\pgfqpoint{0.100000in}{0.220728in}}{\pgfqpoint{3.696000in}{3.696000in}}%
\pgfusepath{clip}%
\pgfsetrectcap%
\pgfsetroundjoin%
\pgfsetlinewidth{1.505625pt}%
\definecolor{currentstroke}{rgb}{1.000000,0.000000,0.000000}%
\pgfsetstrokecolor{currentstroke}%
\pgfsetdash{}{0pt}%
\pgfpathmoveto{\pgfqpoint{2.205052in}{1.686894in}}%
\pgfpathlineto{\pgfqpoint{2.526536in}{1.257201in}}%
\pgfusepath{stroke}%
\end{pgfscope}%
\begin{pgfscope}%
\pgfpathrectangle{\pgfqpoint{0.100000in}{0.220728in}}{\pgfqpoint{3.696000in}{3.696000in}}%
\pgfusepath{clip}%
\pgfsetrectcap%
\pgfsetroundjoin%
\pgfsetlinewidth{1.505625pt}%
\definecolor{currentstroke}{rgb}{1.000000,0.000000,0.000000}%
\pgfsetstrokecolor{currentstroke}%
\pgfsetdash{}{0pt}%
\pgfpathmoveto{\pgfqpoint{2.206081in}{1.683327in}}%
\pgfpathlineto{\pgfqpoint{2.526536in}{1.257201in}}%
\pgfusepath{stroke}%
\end{pgfscope}%
\begin{pgfscope}%
\pgfpathrectangle{\pgfqpoint{0.100000in}{0.220728in}}{\pgfqpoint{3.696000in}{3.696000in}}%
\pgfusepath{clip}%
\pgfsetrectcap%
\pgfsetroundjoin%
\pgfsetlinewidth{1.505625pt}%
\definecolor{currentstroke}{rgb}{1.000000,0.000000,0.000000}%
\pgfsetstrokecolor{currentstroke}%
\pgfsetdash{}{0pt}%
\pgfpathmoveto{\pgfqpoint{2.208383in}{1.684958in}}%
\pgfpathlineto{\pgfqpoint{2.517928in}{1.248859in}}%
\pgfusepath{stroke}%
\end{pgfscope}%
\begin{pgfscope}%
\pgfpathrectangle{\pgfqpoint{0.100000in}{0.220728in}}{\pgfqpoint{3.696000in}{3.696000in}}%
\pgfusepath{clip}%
\pgfsetrectcap%
\pgfsetroundjoin%
\pgfsetlinewidth{1.505625pt}%
\definecolor{currentstroke}{rgb}{1.000000,0.000000,0.000000}%
\pgfsetstrokecolor{currentstroke}%
\pgfsetdash{}{0pt}%
\pgfpathmoveto{\pgfqpoint{2.209400in}{1.683727in}}%
\pgfpathlineto{\pgfqpoint{2.517928in}{1.248859in}}%
\pgfusepath{stroke}%
\end{pgfscope}%
\begin{pgfscope}%
\pgfpathrectangle{\pgfqpoint{0.100000in}{0.220728in}}{\pgfqpoint{3.696000in}{3.696000in}}%
\pgfusepath{clip}%
\pgfsetrectcap%
\pgfsetroundjoin%
\pgfsetlinewidth{1.505625pt}%
\definecolor{currentstroke}{rgb}{1.000000,0.000000,0.000000}%
\pgfsetstrokecolor{currentstroke}%
\pgfsetdash{}{0pt}%
\pgfpathmoveto{\pgfqpoint{2.209915in}{1.682747in}}%
\pgfpathlineto{\pgfqpoint{2.517928in}{1.248859in}}%
\pgfusepath{stroke}%
\end{pgfscope}%
\begin{pgfscope}%
\pgfpathrectangle{\pgfqpoint{0.100000in}{0.220728in}}{\pgfqpoint{3.696000in}{3.696000in}}%
\pgfusepath{clip}%
\pgfsetrectcap%
\pgfsetroundjoin%
\pgfsetlinewidth{1.505625pt}%
\definecolor{currentstroke}{rgb}{1.000000,0.000000,0.000000}%
\pgfsetstrokecolor{currentstroke}%
\pgfsetdash{}{0pt}%
\pgfpathmoveto{\pgfqpoint{2.210584in}{1.681179in}}%
\pgfpathlineto{\pgfqpoint{2.517928in}{1.248859in}}%
\pgfusepath{stroke}%
\end{pgfscope}%
\begin{pgfscope}%
\pgfpathrectangle{\pgfqpoint{0.100000in}{0.220728in}}{\pgfqpoint{3.696000in}{3.696000in}}%
\pgfusepath{clip}%
\pgfsetrectcap%
\pgfsetroundjoin%
\pgfsetlinewidth{1.505625pt}%
\definecolor{currentstroke}{rgb}{1.000000,0.000000,0.000000}%
\pgfsetstrokecolor{currentstroke}%
\pgfsetdash{}{0pt}%
\pgfpathmoveto{\pgfqpoint{2.211616in}{1.681584in}}%
\pgfpathlineto{\pgfqpoint{2.517928in}{1.248859in}}%
\pgfusepath{stroke}%
\end{pgfscope}%
\begin{pgfscope}%
\pgfpathrectangle{\pgfqpoint{0.100000in}{0.220728in}}{\pgfqpoint{3.696000in}{3.696000in}}%
\pgfusepath{clip}%
\pgfsetrectcap%
\pgfsetroundjoin%
\pgfsetlinewidth{1.505625pt}%
\definecolor{currentstroke}{rgb}{1.000000,0.000000,0.000000}%
\pgfsetstrokecolor{currentstroke}%
\pgfsetdash{}{0pt}%
\pgfpathmoveto{\pgfqpoint{2.211869in}{1.680442in}}%
\pgfpathlineto{\pgfqpoint{2.517928in}{1.248859in}}%
\pgfusepath{stroke}%
\end{pgfscope}%
\begin{pgfscope}%
\pgfpathrectangle{\pgfqpoint{0.100000in}{0.220728in}}{\pgfqpoint{3.696000in}{3.696000in}}%
\pgfusepath{clip}%
\pgfsetrectcap%
\pgfsetroundjoin%
\pgfsetlinewidth{1.505625pt}%
\definecolor{currentstroke}{rgb}{1.000000,0.000000,0.000000}%
\pgfsetstrokecolor{currentstroke}%
\pgfsetdash{}{0pt}%
\pgfpathmoveto{\pgfqpoint{2.212511in}{1.679753in}}%
\pgfpathlineto{\pgfqpoint{2.517928in}{1.248859in}}%
\pgfusepath{stroke}%
\end{pgfscope}%
\begin{pgfscope}%
\pgfpathrectangle{\pgfqpoint{0.100000in}{0.220728in}}{\pgfqpoint{3.696000in}{3.696000in}}%
\pgfusepath{clip}%
\pgfsetrectcap%
\pgfsetroundjoin%
\pgfsetlinewidth{1.505625pt}%
\definecolor{currentstroke}{rgb}{1.000000,0.000000,0.000000}%
\pgfsetstrokecolor{currentstroke}%
\pgfsetdash{}{0pt}%
\pgfpathmoveto{\pgfqpoint{2.212753in}{1.679617in}}%
\pgfpathlineto{\pgfqpoint{2.517928in}{1.248859in}}%
\pgfusepath{stroke}%
\end{pgfscope}%
\begin{pgfscope}%
\pgfpathrectangle{\pgfqpoint{0.100000in}{0.220728in}}{\pgfqpoint{3.696000in}{3.696000in}}%
\pgfusepath{clip}%
\pgfsetrectcap%
\pgfsetroundjoin%
\pgfsetlinewidth{1.505625pt}%
\definecolor{currentstroke}{rgb}{1.000000,0.000000,0.000000}%
\pgfsetstrokecolor{currentstroke}%
\pgfsetdash{}{0pt}%
\pgfpathmoveto{\pgfqpoint{2.213606in}{1.679557in}}%
\pgfpathlineto{\pgfqpoint{2.517928in}{1.248859in}}%
\pgfusepath{stroke}%
\end{pgfscope}%
\begin{pgfscope}%
\pgfpathrectangle{\pgfqpoint{0.100000in}{0.220728in}}{\pgfqpoint{3.696000in}{3.696000in}}%
\pgfusepath{clip}%
\pgfsetrectcap%
\pgfsetroundjoin%
\pgfsetlinewidth{1.505625pt}%
\definecolor{currentstroke}{rgb}{1.000000,0.000000,0.000000}%
\pgfsetstrokecolor{currentstroke}%
\pgfsetdash{}{0pt}%
\pgfpathmoveto{\pgfqpoint{2.213985in}{1.678604in}}%
\pgfpathlineto{\pgfqpoint{2.509309in}{1.240506in}}%
\pgfusepath{stroke}%
\end{pgfscope}%
\begin{pgfscope}%
\pgfpathrectangle{\pgfqpoint{0.100000in}{0.220728in}}{\pgfqpoint{3.696000in}{3.696000in}}%
\pgfusepath{clip}%
\pgfsetrectcap%
\pgfsetroundjoin%
\pgfsetlinewidth{1.505625pt}%
\definecolor{currentstroke}{rgb}{1.000000,0.000000,0.000000}%
\pgfsetstrokecolor{currentstroke}%
\pgfsetdash{}{0pt}%
\pgfpathmoveto{\pgfqpoint{2.215392in}{1.674386in}}%
\pgfpathlineto{\pgfqpoint{2.509309in}{1.240506in}}%
\pgfusepath{stroke}%
\end{pgfscope}%
\begin{pgfscope}%
\pgfpathrectangle{\pgfqpoint{0.100000in}{0.220728in}}{\pgfqpoint{3.696000in}{3.696000in}}%
\pgfusepath{clip}%
\pgfsetrectcap%
\pgfsetroundjoin%
\pgfsetlinewidth{1.505625pt}%
\definecolor{currentstroke}{rgb}{1.000000,0.000000,0.000000}%
\pgfsetstrokecolor{currentstroke}%
\pgfsetdash{}{0pt}%
\pgfpathmoveto{\pgfqpoint{2.217742in}{1.674705in}}%
\pgfpathlineto{\pgfqpoint{2.509309in}{1.240506in}}%
\pgfusepath{stroke}%
\end{pgfscope}%
\begin{pgfscope}%
\pgfpathrectangle{\pgfqpoint{0.100000in}{0.220728in}}{\pgfqpoint{3.696000in}{3.696000in}}%
\pgfusepath{clip}%
\pgfsetrectcap%
\pgfsetroundjoin%
\pgfsetlinewidth{1.505625pt}%
\definecolor{currentstroke}{rgb}{1.000000,0.000000,0.000000}%
\pgfsetstrokecolor{currentstroke}%
\pgfsetdash{}{0pt}%
\pgfpathmoveto{\pgfqpoint{2.219308in}{1.672633in}}%
\pgfpathlineto{\pgfqpoint{2.500678in}{1.232141in}}%
\pgfusepath{stroke}%
\end{pgfscope}%
\begin{pgfscope}%
\pgfpathrectangle{\pgfqpoint{0.100000in}{0.220728in}}{\pgfqpoint{3.696000in}{3.696000in}}%
\pgfusepath{clip}%
\pgfsetrectcap%
\pgfsetroundjoin%
\pgfsetlinewidth{1.505625pt}%
\definecolor{currentstroke}{rgb}{1.000000,0.000000,0.000000}%
\pgfsetstrokecolor{currentstroke}%
\pgfsetdash{}{0pt}%
\pgfpathmoveto{\pgfqpoint{2.220832in}{1.670044in}}%
\pgfpathlineto{\pgfqpoint{2.500678in}{1.232141in}}%
\pgfusepath{stroke}%
\end{pgfscope}%
\begin{pgfscope}%
\pgfpathrectangle{\pgfqpoint{0.100000in}{0.220728in}}{\pgfqpoint{3.696000in}{3.696000in}}%
\pgfusepath{clip}%
\pgfsetrectcap%
\pgfsetroundjoin%
\pgfsetlinewidth{1.505625pt}%
\definecolor{currentstroke}{rgb}{1.000000,0.000000,0.000000}%
\pgfsetstrokecolor{currentstroke}%
\pgfsetdash{}{0pt}%
\pgfpathmoveto{\pgfqpoint{2.223806in}{1.664486in}}%
\pgfpathlineto{\pgfqpoint{2.500678in}{1.232141in}}%
\pgfusepath{stroke}%
\end{pgfscope}%
\begin{pgfscope}%
\pgfpathrectangle{\pgfqpoint{0.100000in}{0.220728in}}{\pgfqpoint{3.696000in}{3.696000in}}%
\pgfusepath{clip}%
\pgfsetrectcap%
\pgfsetroundjoin%
\pgfsetlinewidth{1.505625pt}%
\definecolor{currentstroke}{rgb}{1.000000,0.000000,0.000000}%
\pgfsetstrokecolor{currentstroke}%
\pgfsetdash{}{0pt}%
\pgfpathmoveto{\pgfqpoint{2.227786in}{1.665278in}}%
\pgfpathlineto{\pgfqpoint{2.492036in}{1.223765in}}%
\pgfusepath{stroke}%
\end{pgfscope}%
\begin{pgfscope}%
\pgfpathrectangle{\pgfqpoint{0.100000in}{0.220728in}}{\pgfqpoint{3.696000in}{3.696000in}}%
\pgfusepath{clip}%
\pgfsetrectcap%
\pgfsetroundjoin%
\pgfsetlinewidth{1.505625pt}%
\definecolor{currentstroke}{rgb}{1.000000,0.000000,0.000000}%
\pgfsetstrokecolor{currentstroke}%
\pgfsetdash{}{0pt}%
\pgfpathmoveto{\pgfqpoint{2.230498in}{1.660184in}}%
\pgfpathlineto{\pgfqpoint{2.483382in}{1.215378in}}%
\pgfusepath{stroke}%
\end{pgfscope}%
\begin{pgfscope}%
\pgfpathrectangle{\pgfqpoint{0.100000in}{0.220728in}}{\pgfqpoint{3.696000in}{3.696000in}}%
\pgfusepath{clip}%
\pgfsetrectcap%
\pgfsetroundjoin%
\pgfsetlinewidth{1.505625pt}%
\definecolor{currentstroke}{rgb}{1.000000,0.000000,0.000000}%
\pgfsetstrokecolor{currentstroke}%
\pgfsetdash{}{0pt}%
\pgfpathmoveto{\pgfqpoint{2.232320in}{1.656183in}}%
\pgfpathlineto{\pgfqpoint{2.483382in}{1.215378in}}%
\pgfusepath{stroke}%
\end{pgfscope}%
\begin{pgfscope}%
\pgfpathrectangle{\pgfqpoint{0.100000in}{0.220728in}}{\pgfqpoint{3.696000in}{3.696000in}}%
\pgfusepath{clip}%
\pgfsetrectcap%
\pgfsetroundjoin%
\pgfsetlinewidth{1.505625pt}%
\definecolor{currentstroke}{rgb}{1.000000,0.000000,0.000000}%
\pgfsetstrokecolor{currentstroke}%
\pgfsetdash{}{0pt}%
\pgfpathmoveto{\pgfqpoint{2.234178in}{1.653249in}}%
\pgfpathlineto{\pgfqpoint{2.483382in}{1.215378in}}%
\pgfusepath{stroke}%
\end{pgfscope}%
\begin{pgfscope}%
\pgfpathrectangle{\pgfqpoint{0.100000in}{0.220728in}}{\pgfqpoint{3.696000in}{3.696000in}}%
\pgfusepath{clip}%
\pgfsetrectcap%
\pgfsetroundjoin%
\pgfsetlinewidth{1.505625pt}%
\definecolor{currentstroke}{rgb}{1.000000,0.000000,0.000000}%
\pgfsetstrokecolor{currentstroke}%
\pgfsetdash{}{0pt}%
\pgfpathmoveto{\pgfqpoint{2.235518in}{1.653143in}}%
\pgfpathlineto{\pgfqpoint{2.483382in}{1.215378in}}%
\pgfusepath{stroke}%
\end{pgfscope}%
\begin{pgfscope}%
\pgfpathrectangle{\pgfqpoint{0.100000in}{0.220728in}}{\pgfqpoint{3.696000in}{3.696000in}}%
\pgfusepath{clip}%
\pgfsetrectcap%
\pgfsetroundjoin%
\pgfsetlinewidth{1.505625pt}%
\definecolor{currentstroke}{rgb}{1.000000,0.000000,0.000000}%
\pgfsetstrokecolor{currentstroke}%
\pgfsetdash{}{0pt}%
\pgfpathmoveto{\pgfqpoint{2.235853in}{1.651663in}}%
\pgfpathlineto{\pgfqpoint{2.474717in}{1.206980in}}%
\pgfusepath{stroke}%
\end{pgfscope}%
\begin{pgfscope}%
\pgfpathrectangle{\pgfqpoint{0.100000in}{0.220728in}}{\pgfqpoint{3.696000in}{3.696000in}}%
\pgfusepath{clip}%
\pgfsetrectcap%
\pgfsetroundjoin%
\pgfsetlinewidth{1.505625pt}%
\definecolor{currentstroke}{rgb}{1.000000,0.000000,0.000000}%
\pgfsetstrokecolor{currentstroke}%
\pgfsetdash{}{0pt}%
\pgfpathmoveto{\pgfqpoint{2.237022in}{1.649211in}}%
\pgfpathlineto{\pgfqpoint{2.474717in}{1.206980in}}%
\pgfusepath{stroke}%
\end{pgfscope}%
\begin{pgfscope}%
\pgfpathrectangle{\pgfqpoint{0.100000in}{0.220728in}}{\pgfqpoint{3.696000in}{3.696000in}}%
\pgfusepath{clip}%
\pgfsetrectcap%
\pgfsetroundjoin%
\pgfsetlinewidth{1.505625pt}%
\definecolor{currentstroke}{rgb}{1.000000,0.000000,0.000000}%
\pgfsetstrokecolor{currentstroke}%
\pgfsetdash{}{0pt}%
\pgfpathmoveto{\pgfqpoint{2.237494in}{1.648340in}}%
\pgfpathlineto{\pgfqpoint{2.474717in}{1.206980in}}%
\pgfusepath{stroke}%
\end{pgfscope}%
\begin{pgfscope}%
\pgfpathrectangle{\pgfqpoint{0.100000in}{0.220728in}}{\pgfqpoint{3.696000in}{3.696000in}}%
\pgfusepath{clip}%
\pgfsetrectcap%
\pgfsetroundjoin%
\pgfsetlinewidth{1.505625pt}%
\definecolor{currentstroke}{rgb}{1.000000,0.000000,0.000000}%
\pgfsetstrokecolor{currentstroke}%
\pgfsetdash{}{0pt}%
\pgfpathmoveto{\pgfqpoint{2.238420in}{1.648181in}}%
\pgfpathlineto{\pgfqpoint{2.474717in}{1.206980in}}%
\pgfusepath{stroke}%
\end{pgfscope}%
\begin{pgfscope}%
\pgfpathrectangle{\pgfqpoint{0.100000in}{0.220728in}}{\pgfqpoint{3.696000in}{3.696000in}}%
\pgfusepath{clip}%
\pgfsetrectcap%
\pgfsetroundjoin%
\pgfsetlinewidth{1.505625pt}%
\definecolor{currentstroke}{rgb}{1.000000,0.000000,0.000000}%
\pgfsetstrokecolor{currentstroke}%
\pgfsetdash{}{0pt}%
\pgfpathmoveto{\pgfqpoint{2.238742in}{1.647457in}}%
\pgfpathlineto{\pgfqpoint{2.474717in}{1.206980in}}%
\pgfusepath{stroke}%
\end{pgfscope}%
\begin{pgfscope}%
\pgfpathrectangle{\pgfqpoint{0.100000in}{0.220728in}}{\pgfqpoint{3.696000in}{3.696000in}}%
\pgfusepath{clip}%
\pgfsetrectcap%
\pgfsetroundjoin%
\pgfsetlinewidth{1.505625pt}%
\definecolor{currentstroke}{rgb}{1.000000,0.000000,0.000000}%
\pgfsetstrokecolor{currentstroke}%
\pgfsetdash{}{0pt}%
\pgfpathmoveto{\pgfqpoint{2.239690in}{1.644567in}}%
\pgfpathlineto{\pgfqpoint{2.474717in}{1.206980in}}%
\pgfusepath{stroke}%
\end{pgfscope}%
\begin{pgfscope}%
\pgfpathrectangle{\pgfqpoint{0.100000in}{0.220728in}}{\pgfqpoint{3.696000in}{3.696000in}}%
\pgfusepath{clip}%
\pgfsetrectcap%
\pgfsetroundjoin%
\pgfsetlinewidth{1.505625pt}%
\definecolor{currentstroke}{rgb}{1.000000,0.000000,0.000000}%
\pgfsetstrokecolor{currentstroke}%
\pgfsetdash{}{0pt}%
\pgfpathmoveto{\pgfqpoint{2.240744in}{1.643033in}}%
\pgfpathlineto{\pgfqpoint{2.474717in}{1.206980in}}%
\pgfusepath{stroke}%
\end{pgfscope}%
\begin{pgfscope}%
\pgfpathrectangle{\pgfqpoint{0.100000in}{0.220728in}}{\pgfqpoint{3.696000in}{3.696000in}}%
\pgfusepath{clip}%
\pgfsetrectcap%
\pgfsetroundjoin%
\pgfsetlinewidth{1.505625pt}%
\definecolor{currentstroke}{rgb}{1.000000,0.000000,0.000000}%
\pgfsetstrokecolor{currentstroke}%
\pgfsetdash{}{0pt}%
\pgfpathmoveto{\pgfqpoint{2.243162in}{1.641796in}}%
\pgfpathlineto{\pgfqpoint{2.466039in}{1.198570in}}%
\pgfusepath{stroke}%
\end{pgfscope}%
\begin{pgfscope}%
\pgfpathrectangle{\pgfqpoint{0.100000in}{0.220728in}}{\pgfqpoint{3.696000in}{3.696000in}}%
\pgfusepath{clip}%
\pgfsetrectcap%
\pgfsetroundjoin%
\pgfsetlinewidth{1.505625pt}%
\definecolor{currentstroke}{rgb}{1.000000,0.000000,0.000000}%
\pgfsetstrokecolor{currentstroke}%
\pgfsetdash{}{0pt}%
\pgfpathmoveto{\pgfqpoint{2.243441in}{1.640148in}}%
\pgfpathlineto{\pgfqpoint{2.466039in}{1.198570in}}%
\pgfusepath{stroke}%
\end{pgfscope}%
\begin{pgfscope}%
\pgfpathrectangle{\pgfqpoint{0.100000in}{0.220728in}}{\pgfqpoint{3.696000in}{3.696000in}}%
\pgfusepath{clip}%
\pgfsetrectcap%
\pgfsetroundjoin%
\pgfsetlinewidth{1.505625pt}%
\definecolor{currentstroke}{rgb}{1.000000,0.000000,0.000000}%
\pgfsetstrokecolor{currentstroke}%
\pgfsetdash{}{0pt}%
\pgfpathmoveto{\pgfqpoint{2.244831in}{1.634739in}}%
\pgfpathlineto{\pgfqpoint{2.466039in}{1.198570in}}%
\pgfusepath{stroke}%
\end{pgfscope}%
\begin{pgfscope}%
\pgfpathrectangle{\pgfqpoint{0.100000in}{0.220728in}}{\pgfqpoint{3.696000in}{3.696000in}}%
\pgfusepath{clip}%
\pgfsetrectcap%
\pgfsetroundjoin%
\pgfsetlinewidth{1.505625pt}%
\definecolor{currentstroke}{rgb}{1.000000,0.000000,0.000000}%
\pgfsetstrokecolor{currentstroke}%
\pgfsetdash{}{0pt}%
\pgfpathmoveto{\pgfqpoint{2.245780in}{1.634063in}}%
\pgfpathlineto{\pgfqpoint{2.466039in}{1.198570in}}%
\pgfusepath{stroke}%
\end{pgfscope}%
\begin{pgfscope}%
\pgfpathrectangle{\pgfqpoint{0.100000in}{0.220728in}}{\pgfqpoint{3.696000in}{3.696000in}}%
\pgfusepath{clip}%
\pgfsetrectcap%
\pgfsetroundjoin%
\pgfsetlinewidth{1.505625pt}%
\definecolor{currentstroke}{rgb}{1.000000,0.000000,0.000000}%
\pgfsetstrokecolor{currentstroke}%
\pgfsetdash{}{0pt}%
\pgfpathmoveto{\pgfqpoint{2.246790in}{1.633275in}}%
\pgfpathlineto{\pgfqpoint{2.466039in}{1.198570in}}%
\pgfusepath{stroke}%
\end{pgfscope}%
\begin{pgfscope}%
\pgfpathrectangle{\pgfqpoint{0.100000in}{0.220728in}}{\pgfqpoint{3.696000in}{3.696000in}}%
\pgfusepath{clip}%
\pgfsetrectcap%
\pgfsetroundjoin%
\pgfsetlinewidth{1.505625pt}%
\definecolor{currentstroke}{rgb}{1.000000,0.000000,0.000000}%
\pgfsetstrokecolor{currentstroke}%
\pgfsetdash{}{0pt}%
\pgfpathmoveto{\pgfqpoint{2.247088in}{1.632551in}}%
\pgfpathlineto{\pgfqpoint{2.457350in}{1.190149in}}%
\pgfusepath{stroke}%
\end{pgfscope}%
\begin{pgfscope}%
\pgfpathrectangle{\pgfqpoint{0.100000in}{0.220728in}}{\pgfqpoint{3.696000in}{3.696000in}}%
\pgfusepath{clip}%
\pgfsetrectcap%
\pgfsetroundjoin%
\pgfsetlinewidth{1.505625pt}%
\definecolor{currentstroke}{rgb}{1.000000,0.000000,0.000000}%
\pgfsetstrokecolor{currentstroke}%
\pgfsetdash{}{0pt}%
\pgfpathmoveto{\pgfqpoint{2.248118in}{1.629655in}}%
\pgfpathlineto{\pgfqpoint{2.457350in}{1.190149in}}%
\pgfusepath{stroke}%
\end{pgfscope}%
\begin{pgfscope}%
\pgfpathrectangle{\pgfqpoint{0.100000in}{0.220728in}}{\pgfqpoint{3.696000in}{3.696000in}}%
\pgfusepath{clip}%
\pgfsetrectcap%
\pgfsetroundjoin%
\pgfsetlinewidth{1.505625pt}%
\definecolor{currentstroke}{rgb}{1.000000,0.000000,0.000000}%
\pgfsetstrokecolor{currentstroke}%
\pgfsetdash{}{0pt}%
\pgfpathmoveto{\pgfqpoint{2.248782in}{1.628958in}}%
\pgfpathlineto{\pgfqpoint{2.457350in}{1.190149in}}%
\pgfusepath{stroke}%
\end{pgfscope}%
\begin{pgfscope}%
\pgfpathrectangle{\pgfqpoint{0.100000in}{0.220728in}}{\pgfqpoint{3.696000in}{3.696000in}}%
\pgfusepath{clip}%
\pgfsetrectcap%
\pgfsetroundjoin%
\pgfsetlinewidth{1.505625pt}%
\definecolor{currentstroke}{rgb}{1.000000,0.000000,0.000000}%
\pgfsetstrokecolor{currentstroke}%
\pgfsetdash{}{0pt}%
\pgfpathmoveto{\pgfqpoint{2.249587in}{1.628843in}}%
\pgfpathlineto{\pgfqpoint{2.457350in}{1.190149in}}%
\pgfusepath{stroke}%
\end{pgfscope}%
\begin{pgfscope}%
\pgfpathrectangle{\pgfqpoint{0.100000in}{0.220728in}}{\pgfqpoint{3.696000in}{3.696000in}}%
\pgfusepath{clip}%
\pgfsetrectcap%
\pgfsetroundjoin%
\pgfsetlinewidth{1.505625pt}%
\definecolor{currentstroke}{rgb}{1.000000,0.000000,0.000000}%
\pgfsetstrokecolor{currentstroke}%
\pgfsetdash{}{0pt}%
\pgfpathmoveto{\pgfqpoint{2.249830in}{1.628233in}}%
\pgfpathlineto{\pgfqpoint{2.457350in}{1.190149in}}%
\pgfusepath{stroke}%
\end{pgfscope}%
\begin{pgfscope}%
\pgfpathrectangle{\pgfqpoint{0.100000in}{0.220728in}}{\pgfqpoint{3.696000in}{3.696000in}}%
\pgfusepath{clip}%
\pgfsetrectcap%
\pgfsetroundjoin%
\pgfsetlinewidth{1.505625pt}%
\definecolor{currentstroke}{rgb}{1.000000,0.000000,0.000000}%
\pgfsetstrokecolor{currentstroke}%
\pgfsetdash{}{0pt}%
\pgfpathmoveto{\pgfqpoint{2.250458in}{1.627208in}}%
\pgfpathlineto{\pgfqpoint{2.457350in}{1.190149in}}%
\pgfusepath{stroke}%
\end{pgfscope}%
\begin{pgfscope}%
\pgfpathrectangle{\pgfqpoint{0.100000in}{0.220728in}}{\pgfqpoint{3.696000in}{3.696000in}}%
\pgfusepath{clip}%
\pgfsetrectcap%
\pgfsetroundjoin%
\pgfsetlinewidth{1.505625pt}%
\definecolor{currentstroke}{rgb}{1.000000,0.000000,0.000000}%
\pgfsetstrokecolor{currentstroke}%
\pgfsetdash{}{0pt}%
\pgfpathmoveto{\pgfqpoint{2.251459in}{1.624375in}}%
\pgfpathlineto{\pgfqpoint{2.457350in}{1.190149in}}%
\pgfusepath{stroke}%
\end{pgfscope}%
\begin{pgfscope}%
\pgfpathrectangle{\pgfqpoint{0.100000in}{0.220728in}}{\pgfqpoint{3.696000in}{3.696000in}}%
\pgfusepath{clip}%
\pgfsetrectcap%
\pgfsetroundjoin%
\pgfsetlinewidth{1.505625pt}%
\definecolor{currentstroke}{rgb}{1.000000,0.000000,0.000000}%
\pgfsetstrokecolor{currentstroke}%
\pgfsetdash{}{0pt}%
\pgfpathmoveto{\pgfqpoint{2.252061in}{1.624126in}}%
\pgfpathlineto{\pgfqpoint{2.457350in}{1.190149in}}%
\pgfusepath{stroke}%
\end{pgfscope}%
\begin{pgfscope}%
\pgfpathrectangle{\pgfqpoint{0.100000in}{0.220728in}}{\pgfqpoint{3.696000in}{3.696000in}}%
\pgfusepath{clip}%
\pgfsetrectcap%
\pgfsetroundjoin%
\pgfsetlinewidth{1.505625pt}%
\definecolor{currentstroke}{rgb}{1.000000,0.000000,0.000000}%
\pgfsetstrokecolor{currentstroke}%
\pgfsetdash{}{0pt}%
\pgfpathmoveto{\pgfqpoint{2.252438in}{1.623714in}}%
\pgfpathlineto{\pgfqpoint{2.457350in}{1.190149in}}%
\pgfusepath{stroke}%
\end{pgfscope}%
\begin{pgfscope}%
\pgfpathrectangle{\pgfqpoint{0.100000in}{0.220728in}}{\pgfqpoint{3.696000in}{3.696000in}}%
\pgfusepath{clip}%
\pgfsetrectcap%
\pgfsetroundjoin%
\pgfsetlinewidth{1.505625pt}%
\definecolor{currentstroke}{rgb}{1.000000,0.000000,0.000000}%
\pgfsetstrokecolor{currentstroke}%
\pgfsetdash{}{0pt}%
\pgfpathmoveto{\pgfqpoint{2.253016in}{1.622156in}}%
\pgfpathlineto{\pgfqpoint{2.448649in}{1.181716in}}%
\pgfusepath{stroke}%
\end{pgfscope}%
\begin{pgfscope}%
\pgfpathrectangle{\pgfqpoint{0.100000in}{0.220728in}}{\pgfqpoint{3.696000in}{3.696000in}}%
\pgfusepath{clip}%
\pgfsetrectcap%
\pgfsetroundjoin%
\pgfsetlinewidth{1.505625pt}%
\definecolor{currentstroke}{rgb}{1.000000,0.000000,0.000000}%
\pgfsetstrokecolor{currentstroke}%
\pgfsetdash{}{0pt}%
\pgfpathmoveto{\pgfqpoint{2.254201in}{1.618274in}}%
\pgfpathlineto{\pgfqpoint{2.448649in}{1.181716in}}%
\pgfusepath{stroke}%
\end{pgfscope}%
\begin{pgfscope}%
\pgfpathrectangle{\pgfqpoint{0.100000in}{0.220728in}}{\pgfqpoint{3.696000in}{3.696000in}}%
\pgfusepath{clip}%
\pgfsetrectcap%
\pgfsetroundjoin%
\pgfsetlinewidth{1.505625pt}%
\definecolor{currentstroke}{rgb}{1.000000,0.000000,0.000000}%
\pgfsetstrokecolor{currentstroke}%
\pgfsetdash{}{0pt}%
\pgfpathmoveto{\pgfqpoint{2.255289in}{1.618115in}}%
\pgfpathlineto{\pgfqpoint{2.448649in}{1.181716in}}%
\pgfusepath{stroke}%
\end{pgfscope}%
\begin{pgfscope}%
\pgfpathrectangle{\pgfqpoint{0.100000in}{0.220728in}}{\pgfqpoint{3.696000in}{3.696000in}}%
\pgfusepath{clip}%
\pgfsetrectcap%
\pgfsetroundjoin%
\pgfsetlinewidth{1.505625pt}%
\definecolor{currentstroke}{rgb}{1.000000,0.000000,0.000000}%
\pgfsetstrokecolor{currentstroke}%
\pgfsetdash{}{0pt}%
\pgfpathmoveto{\pgfqpoint{2.256377in}{1.617277in}}%
\pgfpathlineto{\pgfqpoint{2.448649in}{1.181716in}}%
\pgfusepath{stroke}%
\end{pgfscope}%
\begin{pgfscope}%
\pgfpathrectangle{\pgfqpoint{0.100000in}{0.220728in}}{\pgfqpoint{3.696000in}{3.696000in}}%
\pgfusepath{clip}%
\pgfsetrectcap%
\pgfsetroundjoin%
\pgfsetlinewidth{1.505625pt}%
\definecolor{currentstroke}{rgb}{1.000000,0.000000,0.000000}%
\pgfsetstrokecolor{currentstroke}%
\pgfsetdash{}{0pt}%
\pgfpathmoveto{\pgfqpoint{2.257240in}{1.614072in}}%
\pgfpathlineto{\pgfqpoint{2.439937in}{1.173272in}}%
\pgfusepath{stroke}%
\end{pgfscope}%
\begin{pgfscope}%
\pgfpathrectangle{\pgfqpoint{0.100000in}{0.220728in}}{\pgfqpoint{3.696000in}{3.696000in}}%
\pgfusepath{clip}%
\pgfsetrectcap%
\pgfsetroundjoin%
\pgfsetlinewidth{1.505625pt}%
\definecolor{currentstroke}{rgb}{1.000000,0.000000,0.000000}%
\pgfsetstrokecolor{currentstroke}%
\pgfsetdash{}{0pt}%
\pgfpathmoveto{\pgfqpoint{2.259598in}{1.607094in}}%
\pgfpathlineto{\pgfqpoint{2.439937in}{1.173272in}}%
\pgfusepath{stroke}%
\end{pgfscope}%
\begin{pgfscope}%
\pgfpathrectangle{\pgfqpoint{0.100000in}{0.220728in}}{\pgfqpoint{3.696000in}{3.696000in}}%
\pgfusepath{clip}%
\pgfsetrectcap%
\pgfsetroundjoin%
\pgfsetlinewidth{1.505625pt}%
\definecolor{currentstroke}{rgb}{1.000000,0.000000,0.000000}%
\pgfsetstrokecolor{currentstroke}%
\pgfsetdash{}{0pt}%
\pgfpathmoveto{\pgfqpoint{2.261588in}{1.601862in}}%
\pgfpathlineto{\pgfqpoint{2.439937in}{1.173272in}}%
\pgfusepath{stroke}%
\end{pgfscope}%
\begin{pgfscope}%
\pgfpathrectangle{\pgfqpoint{0.100000in}{0.220728in}}{\pgfqpoint{3.696000in}{3.696000in}}%
\pgfusepath{clip}%
\pgfsetrectcap%
\pgfsetroundjoin%
\pgfsetlinewidth{1.505625pt}%
\definecolor{currentstroke}{rgb}{1.000000,0.000000,0.000000}%
\pgfsetstrokecolor{currentstroke}%
\pgfsetdash{}{0pt}%
\pgfpathmoveto{\pgfqpoint{2.265258in}{1.603522in}}%
\pgfpathlineto{\pgfqpoint{2.431212in}{1.164817in}}%
\pgfusepath{stroke}%
\end{pgfscope}%
\begin{pgfscope}%
\pgfpathrectangle{\pgfqpoint{0.100000in}{0.220728in}}{\pgfqpoint{3.696000in}{3.696000in}}%
\pgfusepath{clip}%
\pgfsetrectcap%
\pgfsetroundjoin%
\pgfsetlinewidth{1.505625pt}%
\definecolor{currentstroke}{rgb}{1.000000,0.000000,0.000000}%
\pgfsetstrokecolor{currentstroke}%
\pgfsetdash{}{0pt}%
\pgfpathmoveto{\pgfqpoint{2.265793in}{1.600707in}}%
\pgfpathlineto{\pgfqpoint{2.431212in}{1.164817in}}%
\pgfusepath{stroke}%
\end{pgfscope}%
\begin{pgfscope}%
\pgfpathrectangle{\pgfqpoint{0.100000in}{0.220728in}}{\pgfqpoint{3.696000in}{3.696000in}}%
\pgfusepath{clip}%
\pgfsetrectcap%
\pgfsetroundjoin%
\pgfsetlinewidth{1.505625pt}%
\definecolor{currentstroke}{rgb}{1.000000,0.000000,0.000000}%
\pgfsetstrokecolor{currentstroke}%
\pgfsetdash{}{0pt}%
\pgfpathmoveto{\pgfqpoint{2.268032in}{1.596473in}}%
\pgfpathlineto{\pgfqpoint{2.431212in}{1.164817in}}%
\pgfusepath{stroke}%
\end{pgfscope}%
\begin{pgfscope}%
\pgfpathrectangle{\pgfqpoint{0.100000in}{0.220728in}}{\pgfqpoint{3.696000in}{3.696000in}}%
\pgfusepath{clip}%
\pgfsetrectcap%
\pgfsetroundjoin%
\pgfsetlinewidth{1.505625pt}%
\definecolor{currentstroke}{rgb}{1.000000,0.000000,0.000000}%
\pgfsetstrokecolor{currentstroke}%
\pgfsetdash{}{0pt}%
\pgfpathmoveto{\pgfqpoint{2.269340in}{1.591353in}}%
\pgfpathlineto{\pgfqpoint{2.422476in}{1.156350in}}%
\pgfusepath{stroke}%
\end{pgfscope}%
\begin{pgfscope}%
\pgfpathrectangle{\pgfqpoint{0.100000in}{0.220728in}}{\pgfqpoint{3.696000in}{3.696000in}}%
\pgfusepath{clip}%
\pgfsetrectcap%
\pgfsetroundjoin%
\pgfsetlinewidth{1.505625pt}%
\definecolor{currentstroke}{rgb}{1.000000,0.000000,0.000000}%
\pgfsetstrokecolor{currentstroke}%
\pgfsetdash{}{0pt}%
\pgfpathmoveto{\pgfqpoint{2.270736in}{1.591460in}}%
\pgfpathlineto{\pgfqpoint{2.422476in}{1.156350in}}%
\pgfusepath{stroke}%
\end{pgfscope}%
\begin{pgfscope}%
\pgfpathrectangle{\pgfqpoint{0.100000in}{0.220728in}}{\pgfqpoint{3.696000in}{3.696000in}}%
\pgfusepath{clip}%
\pgfsetrectcap%
\pgfsetroundjoin%
\pgfsetlinewidth{1.505625pt}%
\definecolor{currentstroke}{rgb}{1.000000,0.000000,0.000000}%
\pgfsetstrokecolor{currentstroke}%
\pgfsetdash{}{0pt}%
\pgfpathmoveto{\pgfqpoint{2.271106in}{1.590975in}}%
\pgfpathlineto{\pgfqpoint{2.422476in}{1.156350in}}%
\pgfusepath{stroke}%
\end{pgfscope}%
\begin{pgfscope}%
\pgfpathrectangle{\pgfqpoint{0.100000in}{0.220728in}}{\pgfqpoint{3.696000in}{3.696000in}}%
\pgfusepath{clip}%
\pgfsetrectcap%
\pgfsetroundjoin%
\pgfsetlinewidth{1.505625pt}%
\definecolor{currentstroke}{rgb}{1.000000,0.000000,0.000000}%
\pgfsetstrokecolor{currentstroke}%
\pgfsetdash{}{0pt}%
\pgfpathmoveto{\pgfqpoint{2.271930in}{1.589345in}}%
\pgfpathlineto{\pgfqpoint{2.422476in}{1.156350in}}%
\pgfusepath{stroke}%
\end{pgfscope}%
\begin{pgfscope}%
\pgfpathrectangle{\pgfqpoint{0.100000in}{0.220728in}}{\pgfqpoint{3.696000in}{3.696000in}}%
\pgfusepath{clip}%
\pgfsetrectcap%
\pgfsetroundjoin%
\pgfsetlinewidth{1.505625pt}%
\definecolor{currentstroke}{rgb}{1.000000,0.000000,0.000000}%
\pgfsetstrokecolor{currentstroke}%
\pgfsetdash{}{0pt}%
\pgfpathmoveto{\pgfqpoint{2.272762in}{1.586426in}}%
\pgfpathlineto{\pgfqpoint{2.422476in}{1.156350in}}%
\pgfusepath{stroke}%
\end{pgfscope}%
\begin{pgfscope}%
\pgfpathrectangle{\pgfqpoint{0.100000in}{0.220728in}}{\pgfqpoint{3.696000in}{3.696000in}}%
\pgfusepath{clip}%
\pgfsetrectcap%
\pgfsetroundjoin%
\pgfsetlinewidth{1.505625pt}%
\definecolor{currentstroke}{rgb}{1.000000,0.000000,0.000000}%
\pgfsetstrokecolor{currentstroke}%
\pgfsetdash{}{0pt}%
\pgfpathmoveto{\pgfqpoint{2.274403in}{1.585319in}}%
\pgfpathlineto{\pgfqpoint{2.422476in}{1.156350in}}%
\pgfusepath{stroke}%
\end{pgfscope}%
\begin{pgfscope}%
\pgfpathrectangle{\pgfqpoint{0.100000in}{0.220728in}}{\pgfqpoint{3.696000in}{3.696000in}}%
\pgfusepath{clip}%
\pgfsetrectcap%
\pgfsetroundjoin%
\pgfsetlinewidth{1.505625pt}%
\definecolor{currentstroke}{rgb}{1.000000,0.000000,0.000000}%
\pgfsetstrokecolor{currentstroke}%
\pgfsetdash{}{0pt}%
\pgfpathmoveto{\pgfqpoint{2.274976in}{1.584771in}}%
\pgfpathlineto{\pgfqpoint{2.413728in}{1.147872in}}%
\pgfusepath{stroke}%
\end{pgfscope}%
\begin{pgfscope}%
\pgfpathrectangle{\pgfqpoint{0.100000in}{0.220728in}}{\pgfqpoint{3.696000in}{3.696000in}}%
\pgfusepath{clip}%
\pgfsetrectcap%
\pgfsetroundjoin%
\pgfsetlinewidth{1.505625pt}%
\definecolor{currentstroke}{rgb}{1.000000,0.000000,0.000000}%
\pgfsetstrokecolor{currentstroke}%
\pgfsetdash{}{0pt}%
\pgfpathmoveto{\pgfqpoint{2.275368in}{1.583196in}}%
\pgfpathlineto{\pgfqpoint{2.413728in}{1.147872in}}%
\pgfusepath{stroke}%
\end{pgfscope}%
\begin{pgfscope}%
\pgfpathrectangle{\pgfqpoint{0.100000in}{0.220728in}}{\pgfqpoint{3.696000in}{3.696000in}}%
\pgfusepath{clip}%
\pgfsetrectcap%
\pgfsetroundjoin%
\pgfsetlinewidth{1.505625pt}%
\definecolor{currentstroke}{rgb}{1.000000,0.000000,0.000000}%
\pgfsetstrokecolor{currentstroke}%
\pgfsetdash{}{0pt}%
\pgfpathmoveto{\pgfqpoint{2.276454in}{1.577532in}}%
\pgfpathlineto{\pgfqpoint{2.413728in}{1.147872in}}%
\pgfusepath{stroke}%
\end{pgfscope}%
\begin{pgfscope}%
\pgfpathrectangle{\pgfqpoint{0.100000in}{0.220728in}}{\pgfqpoint{3.696000in}{3.696000in}}%
\pgfusepath{clip}%
\pgfsetrectcap%
\pgfsetroundjoin%
\pgfsetlinewidth{1.505625pt}%
\definecolor{currentstroke}{rgb}{1.000000,0.000000,0.000000}%
\pgfsetstrokecolor{currentstroke}%
\pgfsetdash{}{0pt}%
\pgfpathmoveto{\pgfqpoint{2.277374in}{1.576564in}}%
\pgfpathlineto{\pgfqpoint{2.413728in}{1.147872in}}%
\pgfusepath{stroke}%
\end{pgfscope}%
\begin{pgfscope}%
\pgfpathrectangle{\pgfqpoint{0.100000in}{0.220728in}}{\pgfqpoint{3.696000in}{3.696000in}}%
\pgfusepath{clip}%
\pgfsetrectcap%
\pgfsetroundjoin%
\pgfsetlinewidth{1.505625pt}%
\definecolor{currentstroke}{rgb}{1.000000,0.000000,0.000000}%
\pgfsetstrokecolor{currentstroke}%
\pgfsetdash{}{0pt}%
\pgfpathmoveto{\pgfqpoint{2.278694in}{1.576562in}}%
\pgfpathlineto{\pgfqpoint{2.413728in}{1.147872in}}%
\pgfusepath{stroke}%
\end{pgfscope}%
\begin{pgfscope}%
\pgfpathrectangle{\pgfqpoint{0.100000in}{0.220728in}}{\pgfqpoint{3.696000in}{3.696000in}}%
\pgfusepath{clip}%
\pgfsetrectcap%
\pgfsetroundjoin%
\pgfsetlinewidth{1.505625pt}%
\definecolor{currentstroke}{rgb}{1.000000,0.000000,0.000000}%
\pgfsetstrokecolor{currentstroke}%
\pgfsetdash{}{0pt}%
\pgfpathmoveto{\pgfqpoint{2.279011in}{1.575352in}}%
\pgfpathlineto{\pgfqpoint{2.404968in}{1.139382in}}%
\pgfusepath{stroke}%
\end{pgfscope}%
\begin{pgfscope}%
\pgfpathrectangle{\pgfqpoint{0.100000in}{0.220728in}}{\pgfqpoint{3.696000in}{3.696000in}}%
\pgfusepath{clip}%
\pgfsetrectcap%
\pgfsetroundjoin%
\pgfsetlinewidth{1.505625pt}%
\definecolor{currentstroke}{rgb}{1.000000,0.000000,0.000000}%
\pgfsetstrokecolor{currentstroke}%
\pgfsetdash{}{0pt}%
\pgfpathmoveto{\pgfqpoint{2.279444in}{1.574753in}}%
\pgfpathlineto{\pgfqpoint{2.404968in}{1.139382in}}%
\pgfusepath{stroke}%
\end{pgfscope}%
\begin{pgfscope}%
\pgfpathrectangle{\pgfqpoint{0.100000in}{0.220728in}}{\pgfqpoint{3.696000in}{3.696000in}}%
\pgfusepath{clip}%
\pgfsetrectcap%
\pgfsetroundjoin%
\pgfsetlinewidth{1.505625pt}%
\definecolor{currentstroke}{rgb}{1.000000,0.000000,0.000000}%
\pgfsetstrokecolor{currentstroke}%
\pgfsetdash{}{0pt}%
\pgfpathmoveto{\pgfqpoint{2.280062in}{1.573737in}}%
\pgfpathlineto{\pgfqpoint{2.404968in}{1.139382in}}%
\pgfusepath{stroke}%
\end{pgfscope}%
\begin{pgfscope}%
\pgfpathrectangle{\pgfqpoint{0.100000in}{0.220728in}}{\pgfqpoint{3.696000in}{3.696000in}}%
\pgfusepath{clip}%
\pgfsetrectcap%
\pgfsetroundjoin%
\pgfsetlinewidth{1.505625pt}%
\definecolor{currentstroke}{rgb}{1.000000,0.000000,0.000000}%
\pgfsetstrokecolor{currentstroke}%
\pgfsetdash{}{0pt}%
\pgfpathmoveto{\pgfqpoint{2.281044in}{1.574070in}}%
\pgfpathlineto{\pgfqpoint{2.404968in}{1.139382in}}%
\pgfusepath{stroke}%
\end{pgfscope}%
\begin{pgfscope}%
\pgfpathrectangle{\pgfqpoint{0.100000in}{0.220728in}}{\pgfqpoint{3.696000in}{3.696000in}}%
\pgfusepath{clip}%
\pgfsetrectcap%
\pgfsetroundjoin%
\pgfsetlinewidth{1.505625pt}%
\definecolor{currentstroke}{rgb}{1.000000,0.000000,0.000000}%
\pgfsetstrokecolor{currentstroke}%
\pgfsetdash{}{0pt}%
\pgfpathmoveto{\pgfqpoint{2.281249in}{1.573641in}}%
\pgfpathlineto{\pgfqpoint{2.404968in}{1.139382in}}%
\pgfusepath{stroke}%
\end{pgfscope}%
\begin{pgfscope}%
\pgfpathrectangle{\pgfqpoint{0.100000in}{0.220728in}}{\pgfqpoint{3.696000in}{3.696000in}}%
\pgfusepath{clip}%
\pgfsetrectcap%
\pgfsetroundjoin%
\pgfsetlinewidth{1.505625pt}%
\definecolor{currentstroke}{rgb}{1.000000,0.000000,0.000000}%
\pgfsetstrokecolor{currentstroke}%
\pgfsetdash{}{0pt}%
\pgfpathmoveto{\pgfqpoint{2.281684in}{1.572092in}}%
\pgfpathlineto{\pgfqpoint{2.404968in}{1.139382in}}%
\pgfusepath{stroke}%
\end{pgfscope}%
\begin{pgfscope}%
\pgfpathrectangle{\pgfqpoint{0.100000in}{0.220728in}}{\pgfqpoint{3.696000in}{3.696000in}}%
\pgfusepath{clip}%
\pgfsetrectcap%
\pgfsetroundjoin%
\pgfsetlinewidth{1.505625pt}%
\definecolor{currentstroke}{rgb}{1.000000,0.000000,0.000000}%
\pgfsetstrokecolor{currentstroke}%
\pgfsetdash{}{0pt}%
\pgfpathmoveto{\pgfqpoint{2.282685in}{1.569348in}}%
\pgfpathlineto{\pgfqpoint{2.404968in}{1.139382in}}%
\pgfusepath{stroke}%
\end{pgfscope}%
\begin{pgfscope}%
\pgfpathrectangle{\pgfqpoint{0.100000in}{0.220728in}}{\pgfqpoint{3.696000in}{3.696000in}}%
\pgfusepath{clip}%
\pgfsetrectcap%
\pgfsetroundjoin%
\pgfsetlinewidth{1.505625pt}%
\definecolor{currentstroke}{rgb}{1.000000,0.000000,0.000000}%
\pgfsetstrokecolor{currentstroke}%
\pgfsetdash{}{0pt}%
\pgfpathmoveto{\pgfqpoint{2.283736in}{1.567944in}}%
\pgfpathlineto{\pgfqpoint{2.404968in}{1.139382in}}%
\pgfusepath{stroke}%
\end{pgfscope}%
\begin{pgfscope}%
\pgfpathrectangle{\pgfqpoint{0.100000in}{0.220728in}}{\pgfqpoint{3.696000in}{3.696000in}}%
\pgfusepath{clip}%
\pgfsetrectcap%
\pgfsetroundjoin%
\pgfsetlinewidth{1.505625pt}%
\definecolor{currentstroke}{rgb}{1.000000,0.000000,0.000000}%
\pgfsetstrokecolor{currentstroke}%
\pgfsetdash{}{0pt}%
\pgfpathmoveto{\pgfqpoint{2.284900in}{1.566092in}}%
\pgfpathlineto{\pgfqpoint{2.396196in}{1.130880in}}%
\pgfusepath{stroke}%
\end{pgfscope}%
\begin{pgfscope}%
\pgfpathrectangle{\pgfqpoint{0.100000in}{0.220728in}}{\pgfqpoint{3.696000in}{3.696000in}}%
\pgfusepath{clip}%
\pgfsetrectcap%
\pgfsetroundjoin%
\pgfsetlinewidth{1.505625pt}%
\definecolor{currentstroke}{rgb}{1.000000,0.000000,0.000000}%
\pgfsetstrokecolor{currentstroke}%
\pgfsetdash{}{0pt}%
\pgfpathmoveto{\pgfqpoint{2.287026in}{1.563482in}}%
\pgfpathlineto{\pgfqpoint{2.396196in}{1.130880in}}%
\pgfusepath{stroke}%
\end{pgfscope}%
\begin{pgfscope}%
\pgfpathrectangle{\pgfqpoint{0.100000in}{0.220728in}}{\pgfqpoint{3.696000in}{3.696000in}}%
\pgfusepath{clip}%
\pgfsetrectcap%
\pgfsetroundjoin%
\pgfsetlinewidth{1.505625pt}%
\definecolor{currentstroke}{rgb}{1.000000,0.000000,0.000000}%
\pgfsetstrokecolor{currentstroke}%
\pgfsetdash{}{0pt}%
\pgfpathmoveto{\pgfqpoint{2.287872in}{1.561659in}}%
\pgfpathlineto{\pgfqpoint{2.396196in}{1.130880in}}%
\pgfusepath{stroke}%
\end{pgfscope}%
\begin{pgfscope}%
\pgfpathrectangle{\pgfqpoint{0.100000in}{0.220728in}}{\pgfqpoint{3.696000in}{3.696000in}}%
\pgfusepath{clip}%
\pgfsetrectcap%
\pgfsetroundjoin%
\pgfsetlinewidth{1.505625pt}%
\definecolor{currentstroke}{rgb}{1.000000,0.000000,0.000000}%
\pgfsetstrokecolor{currentstroke}%
\pgfsetdash{}{0pt}%
\pgfpathmoveto{\pgfqpoint{2.288891in}{1.557879in}}%
\pgfpathlineto{\pgfqpoint{2.387412in}{1.122367in}}%
\pgfusepath{stroke}%
\end{pgfscope}%
\begin{pgfscope}%
\pgfpathrectangle{\pgfqpoint{0.100000in}{0.220728in}}{\pgfqpoint{3.696000in}{3.696000in}}%
\pgfusepath{clip}%
\pgfsetrectcap%
\pgfsetroundjoin%
\pgfsetlinewidth{1.505625pt}%
\definecolor{currentstroke}{rgb}{1.000000,0.000000,0.000000}%
\pgfsetstrokecolor{currentstroke}%
\pgfsetdash{}{0pt}%
\pgfpathmoveto{\pgfqpoint{2.289574in}{1.557719in}}%
\pgfpathlineto{\pgfqpoint{2.387412in}{1.122367in}}%
\pgfusepath{stroke}%
\end{pgfscope}%
\begin{pgfscope}%
\pgfpathrectangle{\pgfqpoint{0.100000in}{0.220728in}}{\pgfqpoint{3.696000in}{3.696000in}}%
\pgfusepath{clip}%
\pgfsetrectcap%
\pgfsetroundjoin%
\pgfsetlinewidth{1.505625pt}%
\definecolor{currentstroke}{rgb}{1.000000,0.000000,0.000000}%
\pgfsetstrokecolor{currentstroke}%
\pgfsetdash{}{0pt}%
\pgfpathmoveto{\pgfqpoint{2.290350in}{1.556608in}}%
\pgfpathlineto{\pgfqpoint{2.387412in}{1.122367in}}%
\pgfusepath{stroke}%
\end{pgfscope}%
\begin{pgfscope}%
\pgfpathrectangle{\pgfqpoint{0.100000in}{0.220728in}}{\pgfqpoint{3.696000in}{3.696000in}}%
\pgfusepath{clip}%
\pgfsetrectcap%
\pgfsetroundjoin%
\pgfsetlinewidth{1.505625pt}%
\definecolor{currentstroke}{rgb}{1.000000,0.000000,0.000000}%
\pgfsetstrokecolor{currentstroke}%
\pgfsetdash{}{0pt}%
\pgfpathmoveto{\pgfqpoint{2.290615in}{1.555343in}}%
\pgfpathlineto{\pgfqpoint{2.387412in}{1.122367in}}%
\pgfusepath{stroke}%
\end{pgfscope}%
\begin{pgfscope}%
\pgfpathrectangle{\pgfqpoint{0.100000in}{0.220728in}}{\pgfqpoint{3.696000in}{3.696000in}}%
\pgfusepath{clip}%
\pgfsetrectcap%
\pgfsetroundjoin%
\pgfsetlinewidth{1.505625pt}%
\definecolor{currentstroke}{rgb}{1.000000,0.000000,0.000000}%
\pgfsetstrokecolor{currentstroke}%
\pgfsetdash{}{0pt}%
\pgfpathmoveto{\pgfqpoint{2.291312in}{1.552676in}}%
\pgfpathlineto{\pgfqpoint{2.387412in}{1.122367in}}%
\pgfusepath{stroke}%
\end{pgfscope}%
\begin{pgfscope}%
\pgfpathrectangle{\pgfqpoint{0.100000in}{0.220728in}}{\pgfqpoint{3.696000in}{3.696000in}}%
\pgfusepath{clip}%
\pgfsetrectcap%
\pgfsetroundjoin%
\pgfsetlinewidth{1.505625pt}%
\definecolor{currentstroke}{rgb}{1.000000,0.000000,0.000000}%
\pgfsetstrokecolor{currentstroke}%
\pgfsetdash{}{0pt}%
\pgfpathmoveto{\pgfqpoint{2.292506in}{1.552678in}}%
\pgfpathlineto{\pgfqpoint{2.387412in}{1.122367in}}%
\pgfusepath{stroke}%
\end{pgfscope}%
\begin{pgfscope}%
\pgfpathrectangle{\pgfqpoint{0.100000in}{0.220728in}}{\pgfqpoint{3.696000in}{3.696000in}}%
\pgfusepath{clip}%
\pgfsetrectcap%
\pgfsetroundjoin%
\pgfsetlinewidth{1.505625pt}%
\definecolor{currentstroke}{rgb}{1.000000,0.000000,0.000000}%
\pgfsetstrokecolor{currentstroke}%
\pgfsetdash{}{0pt}%
\pgfpathmoveto{\pgfqpoint{2.294080in}{1.551921in}}%
\pgfpathlineto{\pgfqpoint{2.378617in}{1.113843in}}%
\pgfusepath{stroke}%
\end{pgfscope}%
\begin{pgfscope}%
\pgfpathrectangle{\pgfqpoint{0.100000in}{0.220728in}}{\pgfqpoint{3.696000in}{3.696000in}}%
\pgfusepath{clip}%
\pgfsetrectcap%
\pgfsetroundjoin%
\pgfsetlinewidth{1.505625pt}%
\definecolor{currentstroke}{rgb}{1.000000,0.000000,0.000000}%
\pgfsetstrokecolor{currentstroke}%
\pgfsetdash{}{0pt}%
\pgfpathmoveto{\pgfqpoint{2.294766in}{1.548644in}}%
\pgfpathlineto{\pgfqpoint{2.378617in}{1.113843in}}%
\pgfusepath{stroke}%
\end{pgfscope}%
\begin{pgfscope}%
\pgfpathrectangle{\pgfqpoint{0.100000in}{0.220728in}}{\pgfqpoint{3.696000in}{3.696000in}}%
\pgfusepath{clip}%
\pgfsetrectcap%
\pgfsetroundjoin%
\pgfsetlinewidth{1.505625pt}%
\definecolor{currentstroke}{rgb}{1.000000,0.000000,0.000000}%
\pgfsetstrokecolor{currentstroke}%
\pgfsetdash{}{0pt}%
\pgfpathmoveto{\pgfqpoint{2.296969in}{1.545342in}}%
\pgfpathlineto{\pgfqpoint{2.369809in}{1.105306in}}%
\pgfusepath{stroke}%
\end{pgfscope}%
\begin{pgfscope}%
\pgfpathrectangle{\pgfqpoint{0.100000in}{0.220728in}}{\pgfqpoint{3.696000in}{3.696000in}}%
\pgfusepath{clip}%
\pgfsetrectcap%
\pgfsetroundjoin%
\pgfsetlinewidth{1.505625pt}%
\definecolor{currentstroke}{rgb}{1.000000,0.000000,0.000000}%
\pgfsetstrokecolor{currentstroke}%
\pgfsetdash{}{0pt}%
\pgfpathmoveto{\pgfqpoint{2.298970in}{1.540942in}}%
\pgfpathlineto{\pgfqpoint{2.369809in}{1.105306in}}%
\pgfusepath{stroke}%
\end{pgfscope}%
\begin{pgfscope}%
\pgfpathrectangle{\pgfqpoint{0.100000in}{0.220728in}}{\pgfqpoint{3.696000in}{3.696000in}}%
\pgfusepath{clip}%
\pgfsetrectcap%
\pgfsetroundjoin%
\pgfsetlinewidth{1.505625pt}%
\definecolor{currentstroke}{rgb}{1.000000,0.000000,0.000000}%
\pgfsetstrokecolor{currentstroke}%
\pgfsetdash{}{0pt}%
\pgfpathmoveto{\pgfqpoint{2.302895in}{1.543740in}}%
\pgfpathlineto{\pgfqpoint{2.369809in}{1.105306in}}%
\pgfusepath{stroke}%
\end{pgfscope}%
\begin{pgfscope}%
\pgfpathrectangle{\pgfqpoint{0.100000in}{0.220728in}}{\pgfqpoint{3.696000in}{3.696000in}}%
\pgfusepath{clip}%
\pgfsetrectcap%
\pgfsetroundjoin%
\pgfsetlinewidth{1.505625pt}%
\definecolor{currentstroke}{rgb}{1.000000,0.000000,0.000000}%
\pgfsetstrokecolor{currentstroke}%
\pgfsetdash{}{0pt}%
\pgfpathmoveto{\pgfqpoint{2.304367in}{1.539727in}}%
\pgfpathlineto{\pgfqpoint{2.360989in}{1.096759in}}%
\pgfusepath{stroke}%
\end{pgfscope}%
\begin{pgfscope}%
\pgfpathrectangle{\pgfqpoint{0.100000in}{0.220728in}}{\pgfqpoint{3.696000in}{3.696000in}}%
\pgfusepath{clip}%
\pgfsetrectcap%
\pgfsetroundjoin%
\pgfsetlinewidth{1.505625pt}%
\definecolor{currentstroke}{rgb}{1.000000,0.000000,0.000000}%
\pgfsetstrokecolor{currentstroke}%
\pgfsetdash{}{0pt}%
\pgfpathmoveto{\pgfqpoint{2.307500in}{1.535807in}}%
\pgfpathlineto{\pgfqpoint{2.352157in}{1.088199in}}%
\pgfusepath{stroke}%
\end{pgfscope}%
\begin{pgfscope}%
\pgfpathrectangle{\pgfqpoint{0.100000in}{0.220728in}}{\pgfqpoint{3.696000in}{3.696000in}}%
\pgfusepath{clip}%
\pgfsetrectcap%
\pgfsetroundjoin%
\pgfsetlinewidth{1.505625pt}%
\definecolor{currentstroke}{rgb}{1.000000,0.000000,0.000000}%
\pgfsetstrokecolor{currentstroke}%
\pgfsetdash{}{0pt}%
\pgfpathmoveto{\pgfqpoint{2.310315in}{1.522974in}}%
\pgfpathlineto{\pgfqpoint{2.352157in}{1.088199in}}%
\pgfusepath{stroke}%
\end{pgfscope}%
\begin{pgfscope}%
\pgfpathrectangle{\pgfqpoint{0.100000in}{0.220728in}}{\pgfqpoint{3.696000in}{3.696000in}}%
\pgfusepath{clip}%
\pgfsetrectcap%
\pgfsetroundjoin%
\pgfsetlinewidth{1.505625pt}%
\definecolor{currentstroke}{rgb}{1.000000,0.000000,0.000000}%
\pgfsetstrokecolor{currentstroke}%
\pgfsetdash{}{0pt}%
\pgfpathmoveto{\pgfqpoint{2.312279in}{1.521717in}}%
\pgfpathlineto{\pgfqpoint{2.352157in}{1.088199in}}%
\pgfusepath{stroke}%
\end{pgfscope}%
\begin{pgfscope}%
\pgfpathrectangle{\pgfqpoint{0.100000in}{0.220728in}}{\pgfqpoint{3.696000in}{3.696000in}}%
\pgfusepath{clip}%
\pgfsetrectcap%
\pgfsetroundjoin%
\pgfsetlinewidth{1.505625pt}%
\definecolor{currentstroke}{rgb}{1.000000,0.000000,0.000000}%
\pgfsetstrokecolor{currentstroke}%
\pgfsetdash{}{0pt}%
\pgfpathmoveto{\pgfqpoint{2.314949in}{1.520992in}}%
\pgfpathlineto{\pgfqpoint{2.343313in}{1.079628in}}%
\pgfusepath{stroke}%
\end{pgfscope}%
\begin{pgfscope}%
\pgfpathrectangle{\pgfqpoint{0.100000in}{0.220728in}}{\pgfqpoint{3.696000in}{3.696000in}}%
\pgfusepath{clip}%
\pgfsetrectcap%
\pgfsetroundjoin%
\pgfsetlinewidth{1.505625pt}%
\definecolor{currentstroke}{rgb}{1.000000,0.000000,0.000000}%
\pgfsetstrokecolor{currentstroke}%
\pgfsetdash{}{0pt}%
\pgfpathmoveto{\pgfqpoint{2.316292in}{1.517119in}}%
\pgfpathlineto{\pgfqpoint{2.343313in}{1.079628in}}%
\pgfusepath{stroke}%
\end{pgfscope}%
\begin{pgfscope}%
\pgfpathrectangle{\pgfqpoint{0.100000in}{0.220728in}}{\pgfqpoint{3.696000in}{3.696000in}}%
\pgfusepath{clip}%
\pgfsetrectcap%
\pgfsetroundjoin%
\pgfsetlinewidth{1.505625pt}%
\definecolor{currentstroke}{rgb}{1.000000,0.000000,0.000000}%
\pgfsetstrokecolor{currentstroke}%
\pgfsetdash{}{0pt}%
\pgfpathmoveto{\pgfqpoint{2.317302in}{1.515896in}}%
\pgfpathlineto{\pgfqpoint{2.343313in}{1.079628in}}%
\pgfusepath{stroke}%
\end{pgfscope}%
\begin{pgfscope}%
\pgfpathrectangle{\pgfqpoint{0.100000in}{0.220728in}}{\pgfqpoint{3.696000in}{3.696000in}}%
\pgfusepath{clip}%
\pgfsetrectcap%
\pgfsetroundjoin%
\pgfsetlinewidth{1.505625pt}%
\definecolor{currentstroke}{rgb}{1.000000,0.000000,0.000000}%
\pgfsetstrokecolor{currentstroke}%
\pgfsetdash{}{0pt}%
\pgfpathmoveto{\pgfqpoint{2.317976in}{1.513648in}}%
\pgfpathlineto{\pgfqpoint{2.334458in}{1.071045in}}%
\pgfusepath{stroke}%
\end{pgfscope}%
\begin{pgfscope}%
\pgfpathrectangle{\pgfqpoint{0.100000in}{0.220728in}}{\pgfqpoint{3.696000in}{3.696000in}}%
\pgfusepath{clip}%
\pgfsetrectcap%
\pgfsetroundjoin%
\pgfsetlinewidth{1.505625pt}%
\definecolor{currentstroke}{rgb}{1.000000,0.000000,0.000000}%
\pgfsetstrokecolor{currentstroke}%
\pgfsetdash{}{0pt}%
\pgfpathmoveto{\pgfqpoint{2.318333in}{1.513402in}}%
\pgfpathlineto{\pgfqpoint{2.334458in}{1.071045in}}%
\pgfusepath{stroke}%
\end{pgfscope}%
\begin{pgfscope}%
\pgfpathrectangle{\pgfqpoint{0.100000in}{0.220728in}}{\pgfqpoint{3.696000in}{3.696000in}}%
\pgfusepath{clip}%
\pgfsetrectcap%
\pgfsetroundjoin%
\pgfsetlinewidth{1.505625pt}%
\definecolor{currentstroke}{rgb}{1.000000,0.000000,0.000000}%
\pgfsetstrokecolor{currentstroke}%
\pgfsetdash{}{0pt}%
\pgfpathmoveto{\pgfqpoint{2.319073in}{1.512997in}}%
\pgfpathlineto{\pgfqpoint{2.334458in}{1.071045in}}%
\pgfusepath{stroke}%
\end{pgfscope}%
\begin{pgfscope}%
\pgfpathrectangle{\pgfqpoint{0.100000in}{0.220728in}}{\pgfqpoint{3.696000in}{3.696000in}}%
\pgfusepath{clip}%
\pgfsetrectcap%
\pgfsetroundjoin%
\pgfsetlinewidth{1.505625pt}%
\definecolor{currentstroke}{rgb}{1.000000,0.000000,0.000000}%
\pgfsetstrokecolor{currentstroke}%
\pgfsetdash{}{0pt}%
\pgfpathmoveto{\pgfqpoint{2.319857in}{1.510404in}}%
\pgfpathlineto{\pgfqpoint{2.334458in}{1.071045in}}%
\pgfusepath{stroke}%
\end{pgfscope}%
\begin{pgfscope}%
\pgfpathrectangle{\pgfqpoint{0.100000in}{0.220728in}}{\pgfqpoint{3.696000in}{3.696000in}}%
\pgfusepath{clip}%
\pgfsetrectcap%
\pgfsetroundjoin%
\pgfsetlinewidth{1.505625pt}%
\definecolor{currentstroke}{rgb}{1.000000,0.000000,0.000000}%
\pgfsetstrokecolor{currentstroke}%
\pgfsetdash{}{0pt}%
\pgfpathmoveto{\pgfqpoint{2.321087in}{1.506820in}}%
\pgfpathlineto{\pgfqpoint{2.334458in}{1.071045in}}%
\pgfusepath{stroke}%
\end{pgfscope}%
\begin{pgfscope}%
\pgfpathrectangle{\pgfqpoint{0.100000in}{0.220728in}}{\pgfqpoint{3.696000in}{3.696000in}}%
\pgfusepath{clip}%
\pgfsetrectcap%
\pgfsetroundjoin%
\pgfsetlinewidth{1.505625pt}%
\definecolor{currentstroke}{rgb}{1.000000,0.000000,0.000000}%
\pgfsetstrokecolor{currentstroke}%
\pgfsetdash{}{0pt}%
\pgfpathmoveto{\pgfqpoint{2.321834in}{1.505353in}}%
\pgfpathlineto{\pgfqpoint{2.325590in}{1.062451in}}%
\pgfusepath{stroke}%
\end{pgfscope}%
\begin{pgfscope}%
\pgfpathrectangle{\pgfqpoint{0.100000in}{0.220728in}}{\pgfqpoint{3.696000in}{3.696000in}}%
\pgfusepath{clip}%
\pgfsetrectcap%
\pgfsetroundjoin%
\pgfsetlinewidth{1.505625pt}%
\definecolor{currentstroke}{rgb}{1.000000,0.000000,0.000000}%
\pgfsetstrokecolor{currentstroke}%
\pgfsetdash{}{0pt}%
\pgfpathmoveto{\pgfqpoint{2.322369in}{1.504635in}}%
\pgfpathlineto{\pgfqpoint{2.325590in}{1.062451in}}%
\pgfusepath{stroke}%
\end{pgfscope}%
\begin{pgfscope}%
\pgfpathrectangle{\pgfqpoint{0.100000in}{0.220728in}}{\pgfqpoint{3.696000in}{3.696000in}}%
\pgfusepath{clip}%
\pgfsetrectcap%
\pgfsetroundjoin%
\pgfsetlinewidth{1.505625pt}%
\definecolor{currentstroke}{rgb}{1.000000,0.000000,0.000000}%
\pgfsetstrokecolor{currentstroke}%
\pgfsetdash{}{0pt}%
\pgfpathmoveto{\pgfqpoint{2.322606in}{1.504130in}}%
\pgfpathlineto{\pgfqpoint{2.325590in}{1.062451in}}%
\pgfusepath{stroke}%
\end{pgfscope}%
\begin{pgfscope}%
\pgfpathrectangle{\pgfqpoint{0.100000in}{0.220728in}}{\pgfqpoint{3.696000in}{3.696000in}}%
\pgfusepath{clip}%
\pgfsetrectcap%
\pgfsetroundjoin%
\pgfsetlinewidth{1.505625pt}%
\definecolor{currentstroke}{rgb}{1.000000,0.000000,0.000000}%
\pgfsetstrokecolor{currentstroke}%
\pgfsetdash{}{0pt}%
\pgfpathmoveto{\pgfqpoint{2.323297in}{1.503526in}}%
\pgfpathlineto{\pgfqpoint{2.325590in}{1.062451in}}%
\pgfusepath{stroke}%
\end{pgfscope}%
\begin{pgfscope}%
\pgfpathrectangle{\pgfqpoint{0.100000in}{0.220728in}}{\pgfqpoint{3.696000in}{3.696000in}}%
\pgfusepath{clip}%
\pgfsetrectcap%
\pgfsetroundjoin%
\pgfsetlinewidth{1.505625pt}%
\definecolor{currentstroke}{rgb}{1.000000,0.000000,0.000000}%
\pgfsetstrokecolor{currentstroke}%
\pgfsetdash{}{0pt}%
\pgfpathmoveto{\pgfqpoint{2.323487in}{1.502609in}}%
\pgfpathlineto{\pgfqpoint{2.325590in}{1.062451in}}%
\pgfusepath{stroke}%
\end{pgfscope}%
\begin{pgfscope}%
\pgfpathrectangle{\pgfqpoint{0.100000in}{0.220728in}}{\pgfqpoint{3.696000in}{3.696000in}}%
\pgfusepath{clip}%
\pgfsetrectcap%
\pgfsetroundjoin%
\pgfsetlinewidth{1.505625pt}%
\definecolor{currentstroke}{rgb}{1.000000,0.000000,0.000000}%
\pgfsetstrokecolor{currentstroke}%
\pgfsetdash{}{0pt}%
\pgfpathmoveto{\pgfqpoint{2.324079in}{1.502523in}}%
\pgfpathlineto{\pgfqpoint{2.325590in}{1.062451in}}%
\pgfusepath{stroke}%
\end{pgfscope}%
\begin{pgfscope}%
\pgfpathrectangle{\pgfqpoint{0.100000in}{0.220728in}}{\pgfqpoint{3.696000in}{3.696000in}}%
\pgfusepath{clip}%
\pgfsetrectcap%
\pgfsetroundjoin%
\pgfsetlinewidth{1.505625pt}%
\definecolor{currentstroke}{rgb}{1.000000,0.000000,0.000000}%
\pgfsetstrokecolor{currentstroke}%
\pgfsetdash{}{0pt}%
\pgfpathmoveto{\pgfqpoint{2.324347in}{1.502258in}}%
\pgfpathlineto{\pgfqpoint{2.325590in}{1.062451in}}%
\pgfusepath{stroke}%
\end{pgfscope}%
\begin{pgfscope}%
\pgfpathrectangle{\pgfqpoint{0.100000in}{0.220728in}}{\pgfqpoint{3.696000in}{3.696000in}}%
\pgfusepath{clip}%
\pgfsetrectcap%
\pgfsetroundjoin%
\pgfsetlinewidth{1.505625pt}%
\definecolor{currentstroke}{rgb}{1.000000,0.000000,0.000000}%
\pgfsetstrokecolor{currentstroke}%
\pgfsetdash{}{0pt}%
\pgfpathmoveto{\pgfqpoint{2.324409in}{1.501863in}}%
\pgfpathlineto{\pgfqpoint{2.325590in}{1.062451in}}%
\pgfusepath{stroke}%
\end{pgfscope}%
\begin{pgfscope}%
\pgfpathrectangle{\pgfqpoint{0.100000in}{0.220728in}}{\pgfqpoint{3.696000in}{3.696000in}}%
\pgfusepath{clip}%
\pgfsetrectcap%
\pgfsetroundjoin%
\pgfsetlinewidth{1.505625pt}%
\definecolor{currentstroke}{rgb}{1.000000,0.000000,0.000000}%
\pgfsetstrokecolor{currentstroke}%
\pgfsetdash{}{0pt}%
\pgfpathmoveto{\pgfqpoint{2.324998in}{1.499919in}}%
\pgfpathlineto{\pgfqpoint{2.325590in}{1.062451in}}%
\pgfusepath{stroke}%
\end{pgfscope}%
\begin{pgfscope}%
\pgfpathrectangle{\pgfqpoint{0.100000in}{0.220728in}}{\pgfqpoint{3.696000in}{3.696000in}}%
\pgfusepath{clip}%
\pgfsetrectcap%
\pgfsetroundjoin%
\pgfsetlinewidth{1.505625pt}%
\definecolor{currentstroke}{rgb}{1.000000,0.000000,0.000000}%
\pgfsetstrokecolor{currentstroke}%
\pgfsetdash{}{0pt}%
\pgfpathmoveto{\pgfqpoint{2.325615in}{1.499037in}}%
\pgfpathlineto{\pgfqpoint{2.325590in}{1.062451in}}%
\pgfusepath{stroke}%
\end{pgfscope}%
\begin{pgfscope}%
\pgfpathrectangle{\pgfqpoint{0.100000in}{0.220728in}}{\pgfqpoint{3.696000in}{3.696000in}}%
\pgfusepath{clip}%
\pgfsetrectcap%
\pgfsetroundjoin%
\pgfsetlinewidth{1.505625pt}%
\definecolor{currentstroke}{rgb}{1.000000,0.000000,0.000000}%
\pgfsetstrokecolor{currentstroke}%
\pgfsetdash{}{0pt}%
\pgfpathmoveto{\pgfqpoint{2.326783in}{1.497939in}}%
\pgfpathlineto{\pgfqpoint{2.325590in}{1.062451in}}%
\pgfusepath{stroke}%
\end{pgfscope}%
\begin{pgfscope}%
\pgfpathrectangle{\pgfqpoint{0.100000in}{0.220728in}}{\pgfqpoint{3.696000in}{3.696000in}}%
\pgfusepath{clip}%
\pgfsetrectcap%
\pgfsetroundjoin%
\pgfsetlinewidth{1.505625pt}%
\definecolor{currentstroke}{rgb}{1.000000,0.000000,0.000000}%
\pgfsetstrokecolor{currentstroke}%
\pgfsetdash{}{0pt}%
\pgfpathmoveto{\pgfqpoint{2.327377in}{1.496782in}}%
\pgfpathlineto{\pgfqpoint{2.316710in}{1.053844in}}%
\pgfusepath{stroke}%
\end{pgfscope}%
\begin{pgfscope}%
\pgfpathrectangle{\pgfqpoint{0.100000in}{0.220728in}}{\pgfqpoint{3.696000in}{3.696000in}}%
\pgfusepath{clip}%
\pgfsetrectcap%
\pgfsetroundjoin%
\pgfsetlinewidth{1.505625pt}%
\definecolor{currentstroke}{rgb}{1.000000,0.000000,0.000000}%
\pgfsetstrokecolor{currentstroke}%
\pgfsetdash{}{0pt}%
\pgfpathmoveto{\pgfqpoint{2.328035in}{1.495681in}}%
\pgfpathlineto{\pgfqpoint{2.316710in}{1.053844in}}%
\pgfusepath{stroke}%
\end{pgfscope}%
\begin{pgfscope}%
\pgfpathrectangle{\pgfqpoint{0.100000in}{0.220728in}}{\pgfqpoint{3.696000in}{3.696000in}}%
\pgfusepath{clip}%
\pgfsetrectcap%
\pgfsetroundjoin%
\pgfsetlinewidth{1.505625pt}%
\definecolor{currentstroke}{rgb}{1.000000,0.000000,0.000000}%
\pgfsetstrokecolor{currentstroke}%
\pgfsetdash{}{0pt}%
\pgfpathmoveto{\pgfqpoint{2.329131in}{1.493534in}}%
\pgfpathlineto{\pgfqpoint{2.316710in}{1.053844in}}%
\pgfusepath{stroke}%
\end{pgfscope}%
\begin{pgfscope}%
\pgfpathrectangle{\pgfqpoint{0.100000in}{0.220728in}}{\pgfqpoint{3.696000in}{3.696000in}}%
\pgfusepath{clip}%
\pgfsetrectcap%
\pgfsetroundjoin%
\pgfsetlinewidth{1.505625pt}%
\definecolor{currentstroke}{rgb}{1.000000,0.000000,0.000000}%
\pgfsetstrokecolor{currentstroke}%
\pgfsetdash{}{0pt}%
\pgfpathmoveto{\pgfqpoint{2.329724in}{1.493583in}}%
\pgfpathlineto{\pgfqpoint{2.316710in}{1.053844in}}%
\pgfusepath{stroke}%
\end{pgfscope}%
\begin{pgfscope}%
\pgfpathrectangle{\pgfqpoint{0.100000in}{0.220728in}}{\pgfqpoint{3.696000in}{3.696000in}}%
\pgfusepath{clip}%
\pgfsetrectcap%
\pgfsetroundjoin%
\pgfsetlinewidth{1.505625pt}%
\definecolor{currentstroke}{rgb}{1.000000,0.000000,0.000000}%
\pgfsetstrokecolor{currentstroke}%
\pgfsetdash{}{0pt}%
\pgfpathmoveto{\pgfqpoint{2.330262in}{1.493389in}}%
\pgfpathlineto{\pgfqpoint{2.316710in}{1.053844in}}%
\pgfusepath{stroke}%
\end{pgfscope}%
\begin{pgfscope}%
\pgfpathrectangle{\pgfqpoint{0.100000in}{0.220728in}}{\pgfqpoint{3.696000in}{3.696000in}}%
\pgfusepath{clip}%
\pgfsetrectcap%
\pgfsetroundjoin%
\pgfsetlinewidth{1.505625pt}%
\definecolor{currentstroke}{rgb}{1.000000,0.000000,0.000000}%
\pgfsetstrokecolor{currentstroke}%
\pgfsetdash{}{0pt}%
\pgfpathmoveto{\pgfqpoint{2.330660in}{1.492311in}}%
\pgfpathlineto{\pgfqpoint{2.316710in}{1.053844in}}%
\pgfusepath{stroke}%
\end{pgfscope}%
\begin{pgfscope}%
\pgfpathrectangle{\pgfqpoint{0.100000in}{0.220728in}}{\pgfqpoint{3.696000in}{3.696000in}}%
\pgfusepath{clip}%
\pgfsetrectcap%
\pgfsetroundjoin%
\pgfsetlinewidth{1.505625pt}%
\definecolor{currentstroke}{rgb}{1.000000,0.000000,0.000000}%
\pgfsetstrokecolor{currentstroke}%
\pgfsetdash{}{0pt}%
\pgfpathmoveto{\pgfqpoint{2.331350in}{1.490330in}}%
\pgfpathlineto{\pgfqpoint{2.307817in}{1.045226in}}%
\pgfusepath{stroke}%
\end{pgfscope}%
\begin{pgfscope}%
\pgfpathrectangle{\pgfqpoint{0.100000in}{0.220728in}}{\pgfqpoint{3.696000in}{3.696000in}}%
\pgfusepath{clip}%
\pgfsetrectcap%
\pgfsetroundjoin%
\pgfsetlinewidth{1.505625pt}%
\definecolor{currentstroke}{rgb}{1.000000,0.000000,0.000000}%
\pgfsetstrokecolor{currentstroke}%
\pgfsetdash{}{0pt}%
\pgfpathmoveto{\pgfqpoint{2.332179in}{1.489810in}}%
\pgfpathlineto{\pgfqpoint{2.307817in}{1.045226in}}%
\pgfusepath{stroke}%
\end{pgfscope}%
\begin{pgfscope}%
\pgfpathrectangle{\pgfqpoint{0.100000in}{0.220728in}}{\pgfqpoint{3.696000in}{3.696000in}}%
\pgfusepath{clip}%
\pgfsetrectcap%
\pgfsetroundjoin%
\pgfsetlinewidth{1.505625pt}%
\definecolor{currentstroke}{rgb}{1.000000,0.000000,0.000000}%
\pgfsetstrokecolor{currentstroke}%
\pgfsetdash{}{0pt}%
\pgfpathmoveto{\pgfqpoint{2.334592in}{1.489143in}}%
\pgfpathlineto{\pgfqpoint{2.307817in}{1.045226in}}%
\pgfusepath{stroke}%
\end{pgfscope}%
\begin{pgfscope}%
\pgfpathrectangle{\pgfqpoint{0.100000in}{0.220728in}}{\pgfqpoint{3.696000in}{3.696000in}}%
\pgfusepath{clip}%
\pgfsetrectcap%
\pgfsetroundjoin%
\pgfsetlinewidth{1.505625pt}%
\definecolor{currentstroke}{rgb}{1.000000,0.000000,0.000000}%
\pgfsetstrokecolor{currentstroke}%
\pgfsetdash{}{0pt}%
\pgfpathmoveto{\pgfqpoint{2.335717in}{1.486000in}}%
\pgfpathlineto{\pgfqpoint{2.307817in}{1.045226in}}%
\pgfusepath{stroke}%
\end{pgfscope}%
\begin{pgfscope}%
\pgfpathrectangle{\pgfqpoint{0.100000in}{0.220728in}}{\pgfqpoint{3.696000in}{3.696000in}}%
\pgfusepath{clip}%
\pgfsetrectcap%
\pgfsetroundjoin%
\pgfsetlinewidth{1.505625pt}%
\definecolor{currentstroke}{rgb}{1.000000,0.000000,0.000000}%
\pgfsetstrokecolor{currentstroke}%
\pgfsetdash{}{0pt}%
\pgfpathmoveto{\pgfqpoint{2.337674in}{1.481379in}}%
\pgfpathlineto{\pgfqpoint{2.298913in}{1.036596in}}%
\pgfusepath{stroke}%
\end{pgfscope}%
\begin{pgfscope}%
\pgfpathrectangle{\pgfqpoint{0.100000in}{0.220728in}}{\pgfqpoint{3.696000in}{3.696000in}}%
\pgfusepath{clip}%
\pgfsetrectcap%
\pgfsetroundjoin%
\pgfsetlinewidth{1.505625pt}%
\definecolor{currentstroke}{rgb}{1.000000,0.000000,0.000000}%
\pgfsetstrokecolor{currentstroke}%
\pgfsetdash{}{0pt}%
\pgfpathmoveto{\pgfqpoint{2.339564in}{1.477739in}}%
\pgfpathlineto{\pgfqpoint{2.298913in}{1.036596in}}%
\pgfusepath{stroke}%
\end{pgfscope}%
\begin{pgfscope}%
\pgfpathrectangle{\pgfqpoint{0.100000in}{0.220728in}}{\pgfqpoint{3.696000in}{3.696000in}}%
\pgfusepath{clip}%
\pgfsetrectcap%
\pgfsetroundjoin%
\pgfsetlinewidth{1.505625pt}%
\definecolor{currentstroke}{rgb}{1.000000,0.000000,0.000000}%
\pgfsetstrokecolor{currentstroke}%
\pgfsetdash{}{0pt}%
\pgfpathmoveto{\pgfqpoint{2.341170in}{1.477279in}}%
\pgfpathlineto{\pgfqpoint{2.298913in}{1.036596in}}%
\pgfusepath{stroke}%
\end{pgfscope}%
\begin{pgfscope}%
\pgfpathrectangle{\pgfqpoint{0.100000in}{0.220728in}}{\pgfqpoint{3.696000in}{3.696000in}}%
\pgfusepath{clip}%
\pgfsetrectcap%
\pgfsetroundjoin%
\pgfsetlinewidth{1.505625pt}%
\definecolor{currentstroke}{rgb}{1.000000,0.000000,0.000000}%
\pgfsetstrokecolor{currentstroke}%
\pgfsetdash{}{0pt}%
\pgfpathmoveto{\pgfqpoint{2.342353in}{1.475102in}}%
\pgfpathlineto{\pgfqpoint{2.298913in}{1.036596in}}%
\pgfusepath{stroke}%
\end{pgfscope}%
\begin{pgfscope}%
\pgfpathrectangle{\pgfqpoint{0.100000in}{0.220728in}}{\pgfqpoint{3.696000in}{3.696000in}}%
\pgfusepath{clip}%
\pgfsetrectcap%
\pgfsetroundjoin%
\pgfsetlinewidth{1.505625pt}%
\definecolor{currentstroke}{rgb}{1.000000,0.000000,0.000000}%
\pgfsetstrokecolor{currentstroke}%
\pgfsetdash{}{0pt}%
\pgfpathmoveto{\pgfqpoint{2.343943in}{1.471761in}}%
\pgfpathlineto{\pgfqpoint{2.289996in}{1.027954in}}%
\pgfusepath{stroke}%
\end{pgfscope}%
\begin{pgfscope}%
\pgfpathrectangle{\pgfqpoint{0.100000in}{0.220728in}}{\pgfqpoint{3.696000in}{3.696000in}}%
\pgfusepath{clip}%
\pgfsetrectcap%
\pgfsetroundjoin%
\pgfsetlinewidth{1.505625pt}%
\definecolor{currentstroke}{rgb}{1.000000,0.000000,0.000000}%
\pgfsetstrokecolor{currentstroke}%
\pgfsetdash{}{0pt}%
\pgfpathmoveto{\pgfqpoint{2.345791in}{1.465783in}}%
\pgfpathlineto{\pgfqpoint{2.289996in}{1.027954in}}%
\pgfusepath{stroke}%
\end{pgfscope}%
\begin{pgfscope}%
\pgfpathrectangle{\pgfqpoint{0.100000in}{0.220728in}}{\pgfqpoint{3.696000in}{3.696000in}}%
\pgfusepath{clip}%
\pgfsetrectcap%
\pgfsetroundjoin%
\pgfsetlinewidth{1.505625pt}%
\definecolor{currentstroke}{rgb}{1.000000,0.000000,0.000000}%
\pgfsetstrokecolor{currentstroke}%
\pgfsetdash{}{0pt}%
\pgfpathmoveto{\pgfqpoint{2.347844in}{1.463699in}}%
\pgfpathlineto{\pgfqpoint{2.281067in}{1.019301in}}%
\pgfusepath{stroke}%
\end{pgfscope}%
\begin{pgfscope}%
\pgfpathrectangle{\pgfqpoint{0.100000in}{0.220728in}}{\pgfqpoint{3.696000in}{3.696000in}}%
\pgfusepath{clip}%
\pgfsetrectcap%
\pgfsetroundjoin%
\pgfsetlinewidth{1.505625pt}%
\definecolor{currentstroke}{rgb}{1.000000,0.000000,0.000000}%
\pgfsetstrokecolor{currentstroke}%
\pgfsetdash{}{0pt}%
\pgfpathmoveto{\pgfqpoint{2.351536in}{1.461200in}}%
\pgfpathlineto{\pgfqpoint{2.281067in}{1.019301in}}%
\pgfusepath{stroke}%
\end{pgfscope}%
\begin{pgfscope}%
\pgfpathrectangle{\pgfqpoint{0.100000in}{0.220728in}}{\pgfqpoint{3.696000in}{3.696000in}}%
\pgfusepath{clip}%
\pgfsetrectcap%
\pgfsetroundjoin%
\pgfsetlinewidth{1.505625pt}%
\definecolor{currentstroke}{rgb}{1.000000,0.000000,0.000000}%
\pgfsetstrokecolor{currentstroke}%
\pgfsetdash{}{0pt}%
\pgfpathmoveto{\pgfqpoint{2.354394in}{1.458441in}}%
\pgfpathlineto{\pgfqpoint{2.272126in}{1.010635in}}%
\pgfusepath{stroke}%
\end{pgfscope}%
\begin{pgfscope}%
\pgfpathrectangle{\pgfqpoint{0.100000in}{0.220728in}}{\pgfqpoint{3.696000in}{3.696000in}}%
\pgfusepath{clip}%
\pgfsetrectcap%
\pgfsetroundjoin%
\pgfsetlinewidth{1.505625pt}%
\definecolor{currentstroke}{rgb}{1.000000,0.000000,0.000000}%
\pgfsetstrokecolor{currentstroke}%
\pgfsetdash{}{0pt}%
\pgfpathmoveto{\pgfqpoint{2.357660in}{1.452280in}}%
\pgfpathlineto{\pgfqpoint{2.272126in}{1.010635in}}%
\pgfusepath{stroke}%
\end{pgfscope}%
\begin{pgfscope}%
\pgfpathrectangle{\pgfqpoint{0.100000in}{0.220728in}}{\pgfqpoint{3.696000in}{3.696000in}}%
\pgfusepath{clip}%
\pgfsetrectcap%
\pgfsetroundjoin%
\pgfsetlinewidth{1.505625pt}%
\definecolor{currentstroke}{rgb}{1.000000,0.000000,0.000000}%
\pgfsetstrokecolor{currentstroke}%
\pgfsetdash{}{0pt}%
\pgfpathmoveto{\pgfqpoint{2.359937in}{1.441049in}}%
\pgfpathlineto{\pgfqpoint{2.272126in}{1.010635in}}%
\pgfusepath{stroke}%
\end{pgfscope}%
\begin{pgfscope}%
\pgfpathrectangle{\pgfqpoint{0.100000in}{0.220728in}}{\pgfqpoint{3.696000in}{3.696000in}}%
\pgfusepath{clip}%
\pgfsetrectcap%
\pgfsetroundjoin%
\pgfsetlinewidth{1.505625pt}%
\definecolor{currentstroke}{rgb}{1.000000,0.000000,0.000000}%
\pgfsetstrokecolor{currentstroke}%
\pgfsetdash{}{0pt}%
\pgfpathmoveto{\pgfqpoint{2.364362in}{1.435648in}}%
\pgfpathlineto{\pgfqpoint{2.272126in}{1.010635in}}%
\pgfusepath{stroke}%
\end{pgfscope}%
\begin{pgfscope}%
\pgfpathrectangle{\pgfqpoint{0.100000in}{0.220728in}}{\pgfqpoint{3.696000in}{3.696000in}}%
\pgfusepath{clip}%
\pgfsetrectcap%
\pgfsetroundjoin%
\pgfsetlinewidth{1.505625pt}%
\definecolor{currentstroke}{rgb}{1.000000,0.000000,0.000000}%
\pgfsetstrokecolor{currentstroke}%
\pgfsetdash{}{0pt}%
\pgfpathmoveto{\pgfqpoint{2.368911in}{1.430381in}}%
\pgfpathlineto{\pgfqpoint{2.272126in}{1.010635in}}%
\pgfusepath{stroke}%
\end{pgfscope}%
\begin{pgfscope}%
\pgfpathrectangle{\pgfqpoint{0.100000in}{0.220728in}}{\pgfqpoint{3.696000in}{3.696000in}}%
\pgfusepath{clip}%
\pgfsetrectcap%
\pgfsetroundjoin%
\pgfsetlinewidth{1.505625pt}%
\definecolor{currentstroke}{rgb}{1.000000,0.000000,0.000000}%
\pgfsetstrokecolor{currentstroke}%
\pgfsetdash{}{0pt}%
\pgfpathmoveto{\pgfqpoint{2.373013in}{1.428785in}}%
\pgfpathlineto{\pgfqpoint{2.272126in}{1.010635in}}%
\pgfusepath{stroke}%
\end{pgfscope}%
\begin{pgfscope}%
\pgfpathrectangle{\pgfqpoint{0.100000in}{0.220728in}}{\pgfqpoint{3.696000in}{3.696000in}}%
\pgfusepath{clip}%
\pgfsetrectcap%
\pgfsetroundjoin%
\pgfsetlinewidth{1.505625pt}%
\definecolor{currentstroke}{rgb}{1.000000,0.000000,0.000000}%
\pgfsetstrokecolor{currentstroke}%
\pgfsetdash{}{0pt}%
\pgfpathmoveto{\pgfqpoint{2.376083in}{1.417992in}}%
\pgfpathlineto{\pgfqpoint{2.272126in}{1.010635in}}%
\pgfusepath{stroke}%
\end{pgfscope}%
\begin{pgfscope}%
\pgfpathrectangle{\pgfqpoint{0.100000in}{0.220728in}}{\pgfqpoint{3.696000in}{3.696000in}}%
\pgfusepath{clip}%
\pgfsetrectcap%
\pgfsetroundjoin%
\pgfsetlinewidth{1.505625pt}%
\definecolor{currentstroke}{rgb}{1.000000,0.000000,0.000000}%
\pgfsetstrokecolor{currentstroke}%
\pgfsetdash{}{0pt}%
\pgfpathmoveto{\pgfqpoint{2.380541in}{1.408174in}}%
\pgfpathlineto{\pgfqpoint{2.272126in}{1.010635in}}%
\pgfusepath{stroke}%
\end{pgfscope}%
\begin{pgfscope}%
\pgfpathrectangle{\pgfqpoint{0.100000in}{0.220728in}}{\pgfqpoint{3.696000in}{3.696000in}}%
\pgfusepath{clip}%
\pgfsetrectcap%
\pgfsetroundjoin%
\pgfsetlinewidth{1.505625pt}%
\definecolor{currentstroke}{rgb}{1.000000,0.000000,0.000000}%
\pgfsetstrokecolor{currentstroke}%
\pgfsetdash{}{0pt}%
\pgfpathmoveto{\pgfqpoint{2.385647in}{1.393655in}}%
\pgfpathlineto{\pgfqpoint{2.272126in}{1.010635in}}%
\pgfusepath{stroke}%
\end{pgfscope}%
\begin{pgfscope}%
\pgfpathrectangle{\pgfqpoint{0.100000in}{0.220728in}}{\pgfqpoint{3.696000in}{3.696000in}}%
\pgfusepath{clip}%
\pgfsetrectcap%
\pgfsetroundjoin%
\pgfsetlinewidth{1.505625pt}%
\definecolor{currentstroke}{rgb}{1.000000,0.000000,0.000000}%
\pgfsetstrokecolor{currentstroke}%
\pgfsetdash{}{0pt}%
\pgfpathmoveto{\pgfqpoint{2.388194in}{1.387567in}}%
\pgfpathlineto{\pgfqpoint{2.272126in}{1.010635in}}%
\pgfusepath{stroke}%
\end{pgfscope}%
\begin{pgfscope}%
\pgfpathrectangle{\pgfqpoint{0.100000in}{0.220728in}}{\pgfqpoint{3.696000in}{3.696000in}}%
\pgfusepath{clip}%
\pgfsetrectcap%
\pgfsetroundjoin%
\pgfsetlinewidth{1.505625pt}%
\definecolor{currentstroke}{rgb}{1.000000,0.000000,0.000000}%
\pgfsetstrokecolor{currentstroke}%
\pgfsetdash{}{0pt}%
\pgfpathmoveto{\pgfqpoint{2.391498in}{1.381986in}}%
\pgfpathlineto{\pgfqpoint{2.272126in}{1.010635in}}%
\pgfusepath{stroke}%
\end{pgfscope}%
\begin{pgfscope}%
\pgfpathrectangle{\pgfqpoint{0.100000in}{0.220728in}}{\pgfqpoint{3.696000in}{3.696000in}}%
\pgfusepath{clip}%
\pgfsetrectcap%
\pgfsetroundjoin%
\pgfsetlinewidth{1.505625pt}%
\definecolor{currentstroke}{rgb}{1.000000,0.000000,0.000000}%
\pgfsetstrokecolor{currentstroke}%
\pgfsetdash{}{0pt}%
\pgfpathmoveto{\pgfqpoint{2.394064in}{1.376999in}}%
\pgfpathlineto{\pgfqpoint{2.272126in}{1.010635in}}%
\pgfusepath{stroke}%
\end{pgfscope}%
\begin{pgfscope}%
\pgfpathrectangle{\pgfqpoint{0.100000in}{0.220728in}}{\pgfqpoint{3.696000in}{3.696000in}}%
\pgfusepath{clip}%
\pgfsetrectcap%
\pgfsetroundjoin%
\pgfsetlinewidth{1.505625pt}%
\definecolor{currentstroke}{rgb}{1.000000,0.000000,0.000000}%
\pgfsetstrokecolor{currentstroke}%
\pgfsetdash{}{0pt}%
\pgfpathmoveto{\pgfqpoint{2.396921in}{1.370456in}}%
\pgfpathlineto{\pgfqpoint{2.272126in}{1.010635in}}%
\pgfusepath{stroke}%
\end{pgfscope}%
\begin{pgfscope}%
\pgfpathrectangle{\pgfqpoint{0.100000in}{0.220728in}}{\pgfqpoint{3.696000in}{3.696000in}}%
\pgfusepath{clip}%
\pgfsetrectcap%
\pgfsetroundjoin%
\pgfsetlinewidth{1.505625pt}%
\definecolor{currentstroke}{rgb}{1.000000,0.000000,0.000000}%
\pgfsetstrokecolor{currentstroke}%
\pgfsetdash{}{0pt}%
\pgfpathmoveto{\pgfqpoint{2.400941in}{1.366734in}}%
\pgfpathlineto{\pgfqpoint{2.272126in}{1.010635in}}%
\pgfusepath{stroke}%
\end{pgfscope}%
\begin{pgfscope}%
\pgfpathrectangle{\pgfqpoint{0.100000in}{0.220728in}}{\pgfqpoint{3.696000in}{3.696000in}}%
\pgfusepath{clip}%
\pgfsetrectcap%
\pgfsetroundjoin%
\pgfsetlinewidth{1.505625pt}%
\definecolor{currentstroke}{rgb}{1.000000,0.000000,0.000000}%
\pgfsetstrokecolor{currentstroke}%
\pgfsetdash{}{0pt}%
\pgfpathmoveto{\pgfqpoint{2.404185in}{1.361191in}}%
\pgfpathlineto{\pgfqpoint{2.272126in}{1.010635in}}%
\pgfusepath{stroke}%
\end{pgfscope}%
\begin{pgfscope}%
\pgfpathrectangle{\pgfqpoint{0.100000in}{0.220728in}}{\pgfqpoint{3.696000in}{3.696000in}}%
\pgfusepath{clip}%
\pgfsetrectcap%
\pgfsetroundjoin%
\pgfsetlinewidth{1.505625pt}%
\definecolor{currentstroke}{rgb}{1.000000,0.000000,0.000000}%
\pgfsetstrokecolor{currentstroke}%
\pgfsetdash{}{0pt}%
\pgfpathmoveto{\pgfqpoint{2.407531in}{1.351877in}}%
\pgfpathlineto{\pgfqpoint{2.272126in}{1.010635in}}%
\pgfusepath{stroke}%
\end{pgfscope}%
\begin{pgfscope}%
\pgfpathrectangle{\pgfqpoint{0.100000in}{0.220728in}}{\pgfqpoint{3.696000in}{3.696000in}}%
\pgfusepath{clip}%
\pgfsetrectcap%
\pgfsetroundjoin%
\pgfsetlinewidth{1.505625pt}%
\definecolor{currentstroke}{rgb}{1.000000,0.000000,0.000000}%
\pgfsetstrokecolor{currentstroke}%
\pgfsetdash{}{0pt}%
\pgfpathmoveto{\pgfqpoint{2.409055in}{1.347976in}}%
\pgfpathlineto{\pgfqpoint{2.272126in}{1.010635in}}%
\pgfusepath{stroke}%
\end{pgfscope}%
\begin{pgfscope}%
\pgfpathrectangle{\pgfqpoint{0.100000in}{0.220728in}}{\pgfqpoint{3.696000in}{3.696000in}}%
\pgfusepath{clip}%
\pgfsetrectcap%
\pgfsetroundjoin%
\pgfsetlinewidth{1.505625pt}%
\definecolor{currentstroke}{rgb}{1.000000,0.000000,0.000000}%
\pgfsetstrokecolor{currentstroke}%
\pgfsetdash{}{0pt}%
\pgfpathmoveto{\pgfqpoint{2.408486in}{1.340608in}}%
\pgfpathlineto{\pgfqpoint{2.272126in}{1.010635in}}%
\pgfusepath{stroke}%
\end{pgfscope}%
\begin{pgfscope}%
\pgfpathrectangle{\pgfqpoint{0.100000in}{0.220728in}}{\pgfqpoint{3.696000in}{3.696000in}}%
\pgfusepath{clip}%
\pgfsetrectcap%
\pgfsetroundjoin%
\pgfsetlinewidth{1.505625pt}%
\definecolor{currentstroke}{rgb}{1.000000,0.000000,0.000000}%
\pgfsetstrokecolor{currentstroke}%
\pgfsetdash{}{0pt}%
\pgfpathmoveto{\pgfqpoint{2.407468in}{1.339133in}}%
\pgfpathlineto{\pgfqpoint{2.272126in}{1.010635in}}%
\pgfusepath{stroke}%
\end{pgfscope}%
\begin{pgfscope}%
\pgfpathrectangle{\pgfqpoint{0.100000in}{0.220728in}}{\pgfqpoint{3.696000in}{3.696000in}}%
\pgfusepath{clip}%
\pgfsetrectcap%
\pgfsetroundjoin%
\pgfsetlinewidth{1.505625pt}%
\definecolor{currentstroke}{rgb}{1.000000,0.000000,0.000000}%
\pgfsetstrokecolor{currentstroke}%
\pgfsetdash{}{0pt}%
\pgfpathmoveto{\pgfqpoint{2.405766in}{1.337315in}}%
\pgfpathlineto{\pgfqpoint{2.272126in}{1.010635in}}%
\pgfusepath{stroke}%
\end{pgfscope}%
\begin{pgfscope}%
\pgfpathrectangle{\pgfqpoint{0.100000in}{0.220728in}}{\pgfqpoint{3.696000in}{3.696000in}}%
\pgfusepath{clip}%
\pgfsetrectcap%
\pgfsetroundjoin%
\pgfsetlinewidth{1.505625pt}%
\definecolor{currentstroke}{rgb}{1.000000,0.000000,0.000000}%
\pgfsetstrokecolor{currentstroke}%
\pgfsetdash{}{0pt}%
\pgfpathmoveto{\pgfqpoint{2.404577in}{1.337434in}}%
\pgfpathlineto{\pgfqpoint{2.272126in}{1.010635in}}%
\pgfusepath{stroke}%
\end{pgfscope}%
\begin{pgfscope}%
\pgfpathrectangle{\pgfqpoint{0.100000in}{0.220728in}}{\pgfqpoint{3.696000in}{3.696000in}}%
\pgfusepath{clip}%
\pgfsetrectcap%
\pgfsetroundjoin%
\pgfsetlinewidth{1.505625pt}%
\definecolor{currentstroke}{rgb}{1.000000,0.000000,0.000000}%
\pgfsetstrokecolor{currentstroke}%
\pgfsetdash{}{0pt}%
\pgfpathmoveto{\pgfqpoint{2.402369in}{1.336596in}}%
\pgfpathlineto{\pgfqpoint{2.272126in}{1.010635in}}%
\pgfusepath{stroke}%
\end{pgfscope}%
\begin{pgfscope}%
\pgfpathrectangle{\pgfqpoint{0.100000in}{0.220728in}}{\pgfqpoint{3.696000in}{3.696000in}}%
\pgfusepath{clip}%
\pgfsetrectcap%
\pgfsetroundjoin%
\pgfsetlinewidth{1.505625pt}%
\definecolor{currentstroke}{rgb}{1.000000,0.000000,0.000000}%
\pgfsetstrokecolor{currentstroke}%
\pgfsetdash{}{0pt}%
\pgfpathmoveto{\pgfqpoint{2.399048in}{1.335825in}}%
\pgfpathlineto{\pgfqpoint{2.272126in}{1.010635in}}%
\pgfusepath{stroke}%
\end{pgfscope}%
\begin{pgfscope}%
\pgfpathrectangle{\pgfqpoint{0.100000in}{0.220728in}}{\pgfqpoint{3.696000in}{3.696000in}}%
\pgfusepath{clip}%
\pgfsetrectcap%
\pgfsetroundjoin%
\pgfsetlinewidth{1.505625pt}%
\definecolor{currentstroke}{rgb}{1.000000,0.000000,0.000000}%
\pgfsetstrokecolor{currentstroke}%
\pgfsetdash{}{0pt}%
\pgfpathmoveto{\pgfqpoint{2.395083in}{1.339422in}}%
\pgfpathlineto{\pgfqpoint{2.272126in}{1.010635in}}%
\pgfusepath{stroke}%
\end{pgfscope}%
\begin{pgfscope}%
\pgfpathrectangle{\pgfqpoint{0.100000in}{0.220728in}}{\pgfqpoint{3.696000in}{3.696000in}}%
\pgfusepath{clip}%
\pgfsetrectcap%
\pgfsetroundjoin%
\pgfsetlinewidth{1.505625pt}%
\definecolor{currentstroke}{rgb}{1.000000,0.000000,0.000000}%
\pgfsetstrokecolor{currentstroke}%
\pgfsetdash{}{0pt}%
\pgfpathmoveto{\pgfqpoint{2.393174in}{1.340318in}}%
\pgfpathlineto{\pgfqpoint{2.272126in}{1.010635in}}%
\pgfusepath{stroke}%
\end{pgfscope}%
\begin{pgfscope}%
\pgfpathrectangle{\pgfqpoint{0.100000in}{0.220728in}}{\pgfqpoint{3.696000in}{3.696000in}}%
\pgfusepath{clip}%
\pgfsetrectcap%
\pgfsetroundjoin%
\pgfsetlinewidth{1.505625pt}%
\definecolor{currentstroke}{rgb}{1.000000,0.000000,0.000000}%
\pgfsetstrokecolor{currentstroke}%
\pgfsetdash{}{0pt}%
\pgfpathmoveto{\pgfqpoint{2.388983in}{1.343448in}}%
\pgfpathlineto{\pgfqpoint{2.272126in}{1.010635in}}%
\pgfusepath{stroke}%
\end{pgfscope}%
\begin{pgfscope}%
\pgfpathrectangle{\pgfqpoint{0.100000in}{0.220728in}}{\pgfqpoint{3.696000in}{3.696000in}}%
\pgfusepath{clip}%
\pgfsetrectcap%
\pgfsetroundjoin%
\pgfsetlinewidth{1.505625pt}%
\definecolor{currentstroke}{rgb}{1.000000,0.000000,0.000000}%
\pgfsetstrokecolor{currentstroke}%
\pgfsetdash{}{0pt}%
\pgfpathmoveto{\pgfqpoint{2.382776in}{1.344229in}}%
\pgfpathlineto{\pgfqpoint{2.272126in}{1.010635in}}%
\pgfusepath{stroke}%
\end{pgfscope}%
\begin{pgfscope}%
\pgfpathrectangle{\pgfqpoint{0.100000in}{0.220728in}}{\pgfqpoint{3.696000in}{3.696000in}}%
\pgfusepath{clip}%
\pgfsetrectcap%
\pgfsetroundjoin%
\pgfsetlinewidth{1.505625pt}%
\definecolor{currentstroke}{rgb}{1.000000,0.000000,0.000000}%
\pgfsetstrokecolor{currentstroke}%
\pgfsetdash{}{0pt}%
\pgfpathmoveto{\pgfqpoint{2.375710in}{1.353404in}}%
\pgfpathlineto{\pgfqpoint{2.272126in}{1.010635in}}%
\pgfusepath{stroke}%
\end{pgfscope}%
\begin{pgfscope}%
\pgfpathrectangle{\pgfqpoint{0.100000in}{0.220728in}}{\pgfqpoint{3.696000in}{3.696000in}}%
\pgfusepath{clip}%
\pgfsetrectcap%
\pgfsetroundjoin%
\pgfsetlinewidth{1.505625pt}%
\definecolor{currentstroke}{rgb}{1.000000,0.000000,0.000000}%
\pgfsetstrokecolor{currentstroke}%
\pgfsetdash{}{0pt}%
\pgfpathmoveto{\pgfqpoint{2.368341in}{1.358909in}}%
\pgfpathlineto{\pgfqpoint{2.272126in}{1.010635in}}%
\pgfusepath{stroke}%
\end{pgfscope}%
\begin{pgfscope}%
\pgfpathrectangle{\pgfqpoint{0.100000in}{0.220728in}}{\pgfqpoint{3.696000in}{3.696000in}}%
\pgfusepath{clip}%
\pgfsetrectcap%
\pgfsetroundjoin%
\pgfsetlinewidth{1.505625pt}%
\definecolor{currentstroke}{rgb}{1.000000,0.000000,0.000000}%
\pgfsetstrokecolor{currentstroke}%
\pgfsetdash{}{0pt}%
\pgfpathmoveto{\pgfqpoint{2.360146in}{1.366042in}}%
\pgfpathlineto{\pgfqpoint{2.272126in}{1.010635in}}%
\pgfusepath{stroke}%
\end{pgfscope}%
\begin{pgfscope}%
\pgfpathrectangle{\pgfqpoint{0.100000in}{0.220728in}}{\pgfqpoint{3.696000in}{3.696000in}}%
\pgfusepath{clip}%
\pgfsetrectcap%
\pgfsetroundjoin%
\pgfsetlinewidth{1.505625pt}%
\definecolor{currentstroke}{rgb}{1.000000,0.000000,0.000000}%
\pgfsetstrokecolor{currentstroke}%
\pgfsetdash{}{0pt}%
\pgfpathmoveto{\pgfqpoint{2.355736in}{1.369442in}}%
\pgfpathlineto{\pgfqpoint{2.272126in}{1.010635in}}%
\pgfusepath{stroke}%
\end{pgfscope}%
\begin{pgfscope}%
\pgfpathrectangle{\pgfqpoint{0.100000in}{0.220728in}}{\pgfqpoint{3.696000in}{3.696000in}}%
\pgfusepath{clip}%
\pgfsetrectcap%
\pgfsetroundjoin%
\pgfsetlinewidth{1.505625pt}%
\definecolor{currentstroke}{rgb}{1.000000,0.000000,0.000000}%
\pgfsetstrokecolor{currentstroke}%
\pgfsetdash{}{0pt}%
\pgfpathmoveto{\pgfqpoint{2.353301in}{1.371449in}}%
\pgfpathlineto{\pgfqpoint{2.272126in}{1.010635in}}%
\pgfusepath{stroke}%
\end{pgfscope}%
\begin{pgfscope}%
\pgfpathrectangle{\pgfqpoint{0.100000in}{0.220728in}}{\pgfqpoint{3.696000in}{3.696000in}}%
\pgfusepath{clip}%
\pgfsetrectcap%
\pgfsetroundjoin%
\pgfsetlinewidth{1.505625pt}%
\definecolor{currentstroke}{rgb}{1.000000,0.000000,0.000000}%
\pgfsetstrokecolor{currentstroke}%
\pgfsetdash{}{0pt}%
\pgfpathmoveto{\pgfqpoint{2.351981in}{1.372464in}}%
\pgfpathlineto{\pgfqpoint{2.272126in}{1.010635in}}%
\pgfusepath{stroke}%
\end{pgfscope}%
\begin{pgfscope}%
\pgfpathrectangle{\pgfqpoint{0.100000in}{0.220728in}}{\pgfqpoint{3.696000in}{3.696000in}}%
\pgfusepath{clip}%
\pgfsetrectcap%
\pgfsetroundjoin%
\pgfsetlinewidth{1.505625pt}%
\definecolor{currentstroke}{rgb}{1.000000,0.000000,0.000000}%
\pgfsetstrokecolor{currentstroke}%
\pgfsetdash{}{0pt}%
\pgfpathmoveto{\pgfqpoint{2.351243in}{1.372956in}}%
\pgfpathlineto{\pgfqpoint{2.272126in}{1.010635in}}%
\pgfusepath{stroke}%
\end{pgfscope}%
\begin{pgfscope}%
\pgfpathrectangle{\pgfqpoint{0.100000in}{0.220728in}}{\pgfqpoint{3.696000in}{3.696000in}}%
\pgfusepath{clip}%
\pgfsetrectcap%
\pgfsetroundjoin%
\pgfsetlinewidth{1.505625pt}%
\definecolor{currentstroke}{rgb}{1.000000,0.000000,0.000000}%
\pgfsetstrokecolor{currentstroke}%
\pgfsetdash{}{0pt}%
\pgfpathmoveto{\pgfqpoint{2.350822in}{1.373222in}}%
\pgfpathlineto{\pgfqpoint{2.272126in}{1.010635in}}%
\pgfusepath{stroke}%
\end{pgfscope}%
\begin{pgfscope}%
\pgfpathrectangle{\pgfqpoint{0.100000in}{0.220728in}}{\pgfqpoint{3.696000in}{3.696000in}}%
\pgfusepath{clip}%
\pgfsetrectcap%
\pgfsetroundjoin%
\pgfsetlinewidth{1.505625pt}%
\definecolor{currentstroke}{rgb}{1.000000,0.000000,0.000000}%
\pgfsetstrokecolor{currentstroke}%
\pgfsetdash{}{0pt}%
\pgfpathmoveto{\pgfqpoint{2.350586in}{1.373379in}}%
\pgfpathlineto{\pgfqpoint{2.272126in}{1.010635in}}%
\pgfusepath{stroke}%
\end{pgfscope}%
\begin{pgfscope}%
\pgfpathrectangle{\pgfqpoint{0.100000in}{0.220728in}}{\pgfqpoint{3.696000in}{3.696000in}}%
\pgfusepath{clip}%
\pgfsetrectcap%
\pgfsetroundjoin%
\pgfsetlinewidth{1.505625pt}%
\definecolor{currentstroke}{rgb}{1.000000,0.000000,0.000000}%
\pgfsetstrokecolor{currentstroke}%
\pgfsetdash{}{0pt}%
\pgfpathmoveto{\pgfqpoint{2.349447in}{1.373989in}}%
\pgfpathlineto{\pgfqpoint{2.272126in}{1.010635in}}%
\pgfusepath{stroke}%
\end{pgfscope}%
\begin{pgfscope}%
\pgfpathrectangle{\pgfqpoint{0.100000in}{0.220728in}}{\pgfqpoint{3.696000in}{3.696000in}}%
\pgfusepath{clip}%
\pgfsetrectcap%
\pgfsetroundjoin%
\pgfsetlinewidth{1.505625pt}%
\definecolor{currentstroke}{rgb}{1.000000,0.000000,0.000000}%
\pgfsetstrokecolor{currentstroke}%
\pgfsetdash{}{0pt}%
\pgfpathmoveto{\pgfqpoint{2.346199in}{1.375660in}}%
\pgfpathlineto{\pgfqpoint{2.272126in}{1.010635in}}%
\pgfusepath{stroke}%
\end{pgfscope}%
\begin{pgfscope}%
\pgfpathrectangle{\pgfqpoint{0.100000in}{0.220728in}}{\pgfqpoint{3.696000in}{3.696000in}}%
\pgfusepath{clip}%
\pgfsetrectcap%
\pgfsetroundjoin%
\pgfsetlinewidth{1.505625pt}%
\definecolor{currentstroke}{rgb}{1.000000,0.000000,0.000000}%
\pgfsetstrokecolor{currentstroke}%
\pgfsetdash{}{0pt}%
\pgfpathmoveto{\pgfqpoint{2.344405in}{1.376988in}}%
\pgfpathlineto{\pgfqpoint{2.272126in}{1.010635in}}%
\pgfusepath{stroke}%
\end{pgfscope}%
\begin{pgfscope}%
\pgfpathrectangle{\pgfqpoint{0.100000in}{0.220728in}}{\pgfqpoint{3.696000in}{3.696000in}}%
\pgfusepath{clip}%
\pgfsetrectcap%
\pgfsetroundjoin%
\pgfsetlinewidth{1.505625pt}%
\definecolor{currentstroke}{rgb}{1.000000,0.000000,0.000000}%
\pgfsetstrokecolor{currentstroke}%
\pgfsetdash{}{0pt}%
\pgfpathmoveto{\pgfqpoint{2.340560in}{1.381131in}}%
\pgfpathlineto{\pgfqpoint{2.272126in}{1.010635in}}%
\pgfusepath{stroke}%
\end{pgfscope}%
\begin{pgfscope}%
\pgfpathrectangle{\pgfqpoint{0.100000in}{0.220728in}}{\pgfqpoint{3.696000in}{3.696000in}}%
\pgfusepath{clip}%
\pgfsetrectcap%
\pgfsetroundjoin%
\pgfsetlinewidth{1.505625pt}%
\definecolor{currentstroke}{rgb}{1.000000,0.000000,0.000000}%
\pgfsetstrokecolor{currentstroke}%
\pgfsetdash{}{0pt}%
\pgfpathmoveto{\pgfqpoint{2.335668in}{1.382501in}}%
\pgfpathlineto{\pgfqpoint{2.272126in}{1.010635in}}%
\pgfusepath{stroke}%
\end{pgfscope}%
\begin{pgfscope}%
\pgfpathrectangle{\pgfqpoint{0.100000in}{0.220728in}}{\pgfqpoint{3.696000in}{3.696000in}}%
\pgfusepath{clip}%
\pgfsetrectcap%
\pgfsetroundjoin%
\pgfsetlinewidth{1.505625pt}%
\definecolor{currentstroke}{rgb}{1.000000,0.000000,0.000000}%
\pgfsetstrokecolor{currentstroke}%
\pgfsetdash{}{0pt}%
\pgfpathmoveto{\pgfqpoint{2.327316in}{1.387826in}}%
\pgfpathlineto{\pgfqpoint{2.272126in}{1.010635in}}%
\pgfusepath{stroke}%
\end{pgfscope}%
\begin{pgfscope}%
\pgfpathrectangle{\pgfqpoint{0.100000in}{0.220728in}}{\pgfqpoint{3.696000in}{3.696000in}}%
\pgfusepath{clip}%
\pgfsetrectcap%
\pgfsetroundjoin%
\pgfsetlinewidth{1.505625pt}%
\definecolor{currentstroke}{rgb}{1.000000,0.000000,0.000000}%
\pgfsetstrokecolor{currentstroke}%
\pgfsetdash{}{0pt}%
\pgfpathmoveto{\pgfqpoint{2.322884in}{1.390825in}}%
\pgfpathlineto{\pgfqpoint{2.272126in}{1.010635in}}%
\pgfusepath{stroke}%
\end{pgfscope}%
\begin{pgfscope}%
\pgfpathrectangle{\pgfqpoint{0.100000in}{0.220728in}}{\pgfqpoint{3.696000in}{3.696000in}}%
\pgfusepath{clip}%
\pgfsetrectcap%
\pgfsetroundjoin%
\pgfsetlinewidth{1.505625pt}%
\definecolor{currentstroke}{rgb}{1.000000,0.000000,0.000000}%
\pgfsetstrokecolor{currentstroke}%
\pgfsetdash{}{0pt}%
\pgfpathmoveto{\pgfqpoint{2.316989in}{1.394764in}}%
\pgfpathlineto{\pgfqpoint{2.272126in}{1.010635in}}%
\pgfusepath{stroke}%
\end{pgfscope}%
\begin{pgfscope}%
\pgfpathrectangle{\pgfqpoint{0.100000in}{0.220728in}}{\pgfqpoint{3.696000in}{3.696000in}}%
\pgfusepath{clip}%
\pgfsetrectcap%
\pgfsetroundjoin%
\pgfsetlinewidth{1.505625pt}%
\definecolor{currentstroke}{rgb}{1.000000,0.000000,0.000000}%
\pgfsetstrokecolor{currentstroke}%
\pgfsetdash{}{0pt}%
\pgfpathmoveto{\pgfqpoint{2.310330in}{1.398612in}}%
\pgfpathlineto{\pgfqpoint{2.272126in}{1.010635in}}%
\pgfusepath{stroke}%
\end{pgfscope}%
\begin{pgfscope}%
\pgfpathrectangle{\pgfqpoint{0.100000in}{0.220728in}}{\pgfqpoint{3.696000in}{3.696000in}}%
\pgfusepath{clip}%
\pgfsetrectcap%
\pgfsetroundjoin%
\pgfsetlinewidth{1.505625pt}%
\definecolor{currentstroke}{rgb}{1.000000,0.000000,0.000000}%
\pgfsetstrokecolor{currentstroke}%
\pgfsetdash{}{0pt}%
\pgfpathmoveto{\pgfqpoint{2.300540in}{1.404256in}}%
\pgfpathlineto{\pgfqpoint{2.272126in}{1.010635in}}%
\pgfusepath{stroke}%
\end{pgfscope}%
\begin{pgfscope}%
\pgfpathrectangle{\pgfqpoint{0.100000in}{0.220728in}}{\pgfqpoint{3.696000in}{3.696000in}}%
\pgfusepath{clip}%
\pgfsetrectcap%
\pgfsetroundjoin%
\pgfsetlinewidth{1.505625pt}%
\definecolor{currentstroke}{rgb}{1.000000,0.000000,0.000000}%
\pgfsetstrokecolor{currentstroke}%
\pgfsetdash{}{0pt}%
\pgfpathmoveto{\pgfqpoint{2.289982in}{1.416614in}}%
\pgfpathlineto{\pgfqpoint{2.272126in}{1.010635in}}%
\pgfusepath{stroke}%
\end{pgfscope}%
\begin{pgfscope}%
\pgfpathrectangle{\pgfqpoint{0.100000in}{0.220728in}}{\pgfqpoint{3.696000in}{3.696000in}}%
\pgfusepath{clip}%
\pgfsetrectcap%
\pgfsetroundjoin%
\pgfsetlinewidth{1.505625pt}%
\definecolor{currentstroke}{rgb}{1.000000,0.000000,0.000000}%
\pgfsetstrokecolor{currentstroke}%
\pgfsetdash{}{0pt}%
\pgfpathmoveto{\pgfqpoint{2.277415in}{1.421692in}}%
\pgfpathlineto{\pgfqpoint{2.272126in}{1.010635in}}%
\pgfusepath{stroke}%
\end{pgfscope}%
\begin{pgfscope}%
\pgfpathrectangle{\pgfqpoint{0.100000in}{0.220728in}}{\pgfqpoint{3.696000in}{3.696000in}}%
\pgfusepath{clip}%
\pgfsetrectcap%
\pgfsetroundjoin%
\pgfsetlinewidth{1.505625pt}%
\definecolor{currentstroke}{rgb}{1.000000,0.000000,0.000000}%
\pgfsetstrokecolor{currentstroke}%
\pgfsetdash{}{0pt}%
\pgfpathmoveto{\pgfqpoint{2.263921in}{1.426655in}}%
\pgfpathlineto{\pgfqpoint{2.272126in}{1.010635in}}%
\pgfusepath{stroke}%
\end{pgfscope}%
\begin{pgfscope}%
\pgfpathrectangle{\pgfqpoint{0.100000in}{0.220728in}}{\pgfqpoint{3.696000in}{3.696000in}}%
\pgfusepath{clip}%
\pgfsetrectcap%
\pgfsetroundjoin%
\pgfsetlinewidth{1.505625pt}%
\definecolor{currentstroke}{rgb}{1.000000,0.000000,0.000000}%
\pgfsetstrokecolor{currentstroke}%
\pgfsetdash{}{0pt}%
\pgfpathmoveto{\pgfqpoint{2.246719in}{1.441018in}}%
\pgfpathlineto{\pgfqpoint{2.272126in}{1.010635in}}%
\pgfusepath{stroke}%
\end{pgfscope}%
\begin{pgfscope}%
\pgfpathrectangle{\pgfqpoint{0.100000in}{0.220728in}}{\pgfqpoint{3.696000in}{3.696000in}}%
\pgfusepath{clip}%
\pgfsetrectcap%
\pgfsetroundjoin%
\pgfsetlinewidth{1.505625pt}%
\definecolor{currentstroke}{rgb}{1.000000,0.000000,0.000000}%
\pgfsetstrokecolor{currentstroke}%
\pgfsetdash{}{0pt}%
\pgfpathmoveto{\pgfqpoint{2.231249in}{1.452525in}}%
\pgfpathlineto{\pgfqpoint{2.272126in}{1.010635in}}%
\pgfusepath{stroke}%
\end{pgfscope}%
\begin{pgfscope}%
\pgfpathrectangle{\pgfqpoint{0.100000in}{0.220728in}}{\pgfqpoint{3.696000in}{3.696000in}}%
\pgfusepath{clip}%
\pgfsetrectcap%
\pgfsetroundjoin%
\pgfsetlinewidth{1.505625pt}%
\definecolor{currentstroke}{rgb}{1.000000,0.000000,0.000000}%
\pgfsetstrokecolor{currentstroke}%
\pgfsetdash{}{0pt}%
\pgfpathmoveto{\pgfqpoint{2.214061in}{1.468945in}}%
\pgfpathlineto{\pgfqpoint{2.272126in}{1.010635in}}%
\pgfusepath{stroke}%
\end{pgfscope}%
\begin{pgfscope}%
\pgfpathrectangle{\pgfqpoint{0.100000in}{0.220728in}}{\pgfqpoint{3.696000in}{3.696000in}}%
\pgfusepath{clip}%
\pgfsetrectcap%
\pgfsetroundjoin%
\pgfsetlinewidth{1.505625pt}%
\definecolor{currentstroke}{rgb}{1.000000,0.000000,0.000000}%
\pgfsetstrokecolor{currentstroke}%
\pgfsetdash{}{0pt}%
\pgfpathmoveto{\pgfqpoint{2.203375in}{1.472256in}}%
\pgfpathlineto{\pgfqpoint{2.272126in}{1.010635in}}%
\pgfusepath{stroke}%
\end{pgfscope}%
\begin{pgfscope}%
\pgfpathrectangle{\pgfqpoint{0.100000in}{0.220728in}}{\pgfqpoint{3.696000in}{3.696000in}}%
\pgfusepath{clip}%
\pgfsetrectcap%
\pgfsetroundjoin%
\pgfsetlinewidth{1.505625pt}%
\definecolor{currentstroke}{rgb}{1.000000,0.000000,0.000000}%
\pgfsetstrokecolor{currentstroke}%
\pgfsetdash{}{0pt}%
\pgfpathmoveto{\pgfqpoint{2.191706in}{1.479799in}}%
\pgfpathlineto{\pgfqpoint{2.272126in}{1.010635in}}%
\pgfusepath{stroke}%
\end{pgfscope}%
\begin{pgfscope}%
\pgfpathrectangle{\pgfqpoint{0.100000in}{0.220728in}}{\pgfqpoint{3.696000in}{3.696000in}}%
\pgfusepath{clip}%
\pgfsetrectcap%
\pgfsetroundjoin%
\pgfsetlinewidth{1.505625pt}%
\definecolor{currentstroke}{rgb}{1.000000,0.000000,0.000000}%
\pgfsetstrokecolor{currentstroke}%
\pgfsetdash{}{0pt}%
\pgfpathmoveto{\pgfqpoint{2.177430in}{1.486863in}}%
\pgfpathlineto{\pgfqpoint{2.272126in}{1.010635in}}%
\pgfusepath{stroke}%
\end{pgfscope}%
\begin{pgfscope}%
\pgfpathrectangle{\pgfqpoint{0.100000in}{0.220728in}}{\pgfqpoint{3.696000in}{3.696000in}}%
\pgfusepath{clip}%
\pgfsetrectcap%
\pgfsetroundjoin%
\pgfsetlinewidth{1.505625pt}%
\definecolor{currentstroke}{rgb}{1.000000,0.000000,0.000000}%
\pgfsetstrokecolor{currentstroke}%
\pgfsetdash{}{0pt}%
\pgfpathmoveto{\pgfqpoint{2.162068in}{1.497488in}}%
\pgfpathlineto{\pgfqpoint{2.257607in}{1.015198in}}%
\pgfusepath{stroke}%
\end{pgfscope}%
\begin{pgfscope}%
\pgfpathrectangle{\pgfqpoint{0.100000in}{0.220728in}}{\pgfqpoint{3.696000in}{3.696000in}}%
\pgfusepath{clip}%
\pgfsetrectcap%
\pgfsetroundjoin%
\pgfsetlinewidth{1.505625pt}%
\definecolor{currentstroke}{rgb}{1.000000,0.000000,0.000000}%
\pgfsetstrokecolor{currentstroke}%
\pgfsetdash{}{0pt}%
\pgfpathmoveto{\pgfqpoint{2.145904in}{1.511945in}}%
\pgfpathlineto{\pgfqpoint{2.243098in}{1.019758in}}%
\pgfusepath{stroke}%
\end{pgfscope}%
\begin{pgfscope}%
\pgfpathrectangle{\pgfqpoint{0.100000in}{0.220728in}}{\pgfqpoint{3.696000in}{3.696000in}}%
\pgfusepath{clip}%
\pgfsetrectcap%
\pgfsetroundjoin%
\pgfsetlinewidth{1.505625pt}%
\definecolor{currentstroke}{rgb}{1.000000,0.000000,0.000000}%
\pgfsetstrokecolor{currentstroke}%
\pgfsetdash{}{0pt}%
\pgfpathmoveto{\pgfqpoint{2.127992in}{1.524329in}}%
\pgfpathlineto{\pgfqpoint{2.228599in}{1.024314in}}%
\pgfusepath{stroke}%
\end{pgfscope}%
\begin{pgfscope}%
\pgfpathrectangle{\pgfqpoint{0.100000in}{0.220728in}}{\pgfqpoint{3.696000in}{3.696000in}}%
\pgfusepath{clip}%
\pgfsetrectcap%
\pgfsetroundjoin%
\pgfsetlinewidth{1.505625pt}%
\definecolor{currentstroke}{rgb}{1.000000,0.000000,0.000000}%
\pgfsetstrokecolor{currentstroke}%
\pgfsetdash{}{0pt}%
\pgfpathmoveto{\pgfqpoint{2.106788in}{1.545721in}}%
\pgfpathlineto{\pgfqpoint{2.199633in}{1.033416in}}%
\pgfusepath{stroke}%
\end{pgfscope}%
\begin{pgfscope}%
\pgfpathrectangle{\pgfqpoint{0.100000in}{0.220728in}}{\pgfqpoint{3.696000in}{3.696000in}}%
\pgfusepath{clip}%
\pgfsetrectcap%
\pgfsetroundjoin%
\pgfsetlinewidth{1.505625pt}%
\definecolor{currentstroke}{rgb}{1.000000,0.000000,0.000000}%
\pgfsetstrokecolor{currentstroke}%
\pgfsetdash{}{0pt}%
\pgfpathmoveto{\pgfqpoint{2.095085in}{1.552041in}}%
\pgfpathlineto{\pgfqpoint{2.199633in}{1.033416in}}%
\pgfusepath{stroke}%
\end{pgfscope}%
\begin{pgfscope}%
\pgfpathrectangle{\pgfqpoint{0.100000in}{0.220728in}}{\pgfqpoint{3.696000in}{3.696000in}}%
\pgfusepath{clip}%
\pgfsetrectcap%
\pgfsetroundjoin%
\pgfsetlinewidth{1.505625pt}%
\definecolor{currentstroke}{rgb}{1.000000,0.000000,0.000000}%
\pgfsetstrokecolor{currentstroke}%
\pgfsetdash{}{0pt}%
\pgfpathmoveto{\pgfqpoint{2.080911in}{1.564682in}}%
\pgfpathlineto{\pgfqpoint{2.185166in}{1.037963in}}%
\pgfusepath{stroke}%
\end{pgfscope}%
\begin{pgfscope}%
\pgfpathrectangle{\pgfqpoint{0.100000in}{0.220728in}}{\pgfqpoint{3.696000in}{3.696000in}}%
\pgfusepath{clip}%
\pgfsetrectcap%
\pgfsetroundjoin%
\pgfsetlinewidth{1.505625pt}%
\definecolor{currentstroke}{rgb}{1.000000,0.000000,0.000000}%
\pgfsetstrokecolor{currentstroke}%
\pgfsetdash{}{0pt}%
\pgfpathmoveto{\pgfqpoint{2.072460in}{1.567504in}}%
\pgfpathlineto{\pgfqpoint{2.170709in}{1.042506in}}%
\pgfusepath{stroke}%
\end{pgfscope}%
\begin{pgfscope}%
\pgfpathrectangle{\pgfqpoint{0.100000in}{0.220728in}}{\pgfqpoint{3.696000in}{3.696000in}}%
\pgfusepath{clip}%
\pgfsetrectcap%
\pgfsetroundjoin%
\pgfsetlinewidth{1.505625pt}%
\definecolor{currentstroke}{rgb}{1.000000,0.000000,0.000000}%
\pgfsetstrokecolor{currentstroke}%
\pgfsetdash{}{0pt}%
\pgfpathmoveto{\pgfqpoint{2.060408in}{1.574820in}}%
\pgfpathlineto{\pgfqpoint{2.156262in}{1.047046in}}%
\pgfusepath{stroke}%
\end{pgfscope}%
\begin{pgfscope}%
\pgfpathrectangle{\pgfqpoint{0.100000in}{0.220728in}}{\pgfqpoint{3.696000in}{3.696000in}}%
\pgfusepath{clip}%
\pgfsetrectcap%
\pgfsetroundjoin%
\pgfsetlinewidth{1.505625pt}%
\definecolor{currentstroke}{rgb}{1.000000,0.000000,0.000000}%
\pgfsetstrokecolor{currentstroke}%
\pgfsetdash{}{0pt}%
\pgfpathmoveto{\pgfqpoint{2.053984in}{1.580465in}}%
\pgfpathlineto{\pgfqpoint{2.156262in}{1.047046in}}%
\pgfusepath{stroke}%
\end{pgfscope}%
\begin{pgfscope}%
\pgfpathrectangle{\pgfqpoint{0.100000in}{0.220728in}}{\pgfqpoint{3.696000in}{3.696000in}}%
\pgfusepath{clip}%
\pgfsetrectcap%
\pgfsetroundjoin%
\pgfsetlinewidth{1.505625pt}%
\definecolor{currentstroke}{rgb}{1.000000,0.000000,0.000000}%
\pgfsetstrokecolor{currentstroke}%
\pgfsetdash{}{0pt}%
\pgfpathmoveto{\pgfqpoint{2.045275in}{1.585164in}}%
\pgfpathlineto{\pgfqpoint{2.141826in}{1.051582in}}%
\pgfusepath{stroke}%
\end{pgfscope}%
\begin{pgfscope}%
\pgfpathrectangle{\pgfqpoint{0.100000in}{0.220728in}}{\pgfqpoint{3.696000in}{3.696000in}}%
\pgfusepath{clip}%
\pgfsetrectcap%
\pgfsetroundjoin%
\pgfsetlinewidth{1.505625pt}%
\definecolor{currentstroke}{rgb}{1.000000,0.000000,0.000000}%
\pgfsetstrokecolor{currentstroke}%
\pgfsetdash{}{0pt}%
\pgfpathmoveto{\pgfqpoint{2.035031in}{1.592921in}}%
\pgfpathlineto{\pgfqpoint{2.127400in}{1.056116in}}%
\pgfusepath{stroke}%
\end{pgfscope}%
\begin{pgfscope}%
\pgfpathrectangle{\pgfqpoint{0.100000in}{0.220728in}}{\pgfqpoint{3.696000in}{3.696000in}}%
\pgfusepath{clip}%
\pgfsetrectcap%
\pgfsetroundjoin%
\pgfsetlinewidth{1.505625pt}%
\definecolor{currentstroke}{rgb}{1.000000,0.000000,0.000000}%
\pgfsetstrokecolor{currentstroke}%
\pgfsetdash{}{0pt}%
\pgfpathmoveto{\pgfqpoint{2.022758in}{1.601887in}}%
\pgfpathlineto{\pgfqpoint{2.112985in}{1.060646in}}%
\pgfusepath{stroke}%
\end{pgfscope}%
\begin{pgfscope}%
\pgfpathrectangle{\pgfqpoint{0.100000in}{0.220728in}}{\pgfqpoint{3.696000in}{3.696000in}}%
\pgfusepath{clip}%
\pgfsetrectcap%
\pgfsetroundjoin%
\pgfsetlinewidth{1.505625pt}%
\definecolor{currentstroke}{rgb}{1.000000,0.000000,0.000000}%
\pgfsetstrokecolor{currentstroke}%
\pgfsetdash{}{0pt}%
\pgfpathmoveto{\pgfqpoint{2.009034in}{1.611938in}}%
\pgfpathlineto{\pgfqpoint{2.112985in}{1.060646in}}%
\pgfusepath{stroke}%
\end{pgfscope}%
\begin{pgfscope}%
\pgfpathrectangle{\pgfqpoint{0.100000in}{0.220728in}}{\pgfqpoint{3.696000in}{3.696000in}}%
\pgfusepath{clip}%
\pgfsetrectcap%
\pgfsetroundjoin%
\pgfsetlinewidth{1.505625pt}%
\definecolor{currentstroke}{rgb}{1.000000,0.000000,0.000000}%
\pgfsetstrokecolor{currentstroke}%
\pgfsetdash{}{0pt}%
\pgfpathmoveto{\pgfqpoint{1.992573in}{1.623333in}}%
\pgfpathlineto{\pgfqpoint{2.084186in}{1.069696in}}%
\pgfusepath{stroke}%
\end{pgfscope}%
\begin{pgfscope}%
\pgfpathrectangle{\pgfqpoint{0.100000in}{0.220728in}}{\pgfqpoint{3.696000in}{3.696000in}}%
\pgfusepath{clip}%
\pgfsetrectcap%
\pgfsetroundjoin%
\pgfsetlinewidth{1.505625pt}%
\definecolor{currentstroke}{rgb}{1.000000,0.000000,0.000000}%
\pgfsetstrokecolor{currentstroke}%
\pgfsetdash{}{0pt}%
\pgfpathmoveto{\pgfqpoint{1.974749in}{1.632917in}}%
\pgfpathlineto{\pgfqpoint{2.069801in}{1.074216in}}%
\pgfusepath{stroke}%
\end{pgfscope}%
\begin{pgfscope}%
\pgfpathrectangle{\pgfqpoint{0.100000in}{0.220728in}}{\pgfqpoint{3.696000in}{3.696000in}}%
\pgfusepath{clip}%
\pgfsetrectcap%
\pgfsetroundjoin%
\pgfsetlinewidth{1.505625pt}%
\definecolor{currentstroke}{rgb}{1.000000,0.000000,0.000000}%
\pgfsetstrokecolor{currentstroke}%
\pgfsetdash{}{0pt}%
\pgfpathmoveto{\pgfqpoint{1.954393in}{1.644273in}}%
\pgfpathlineto{\pgfqpoint{2.055427in}{1.078733in}}%
\pgfusepath{stroke}%
\end{pgfscope}%
\begin{pgfscope}%
\pgfpathrectangle{\pgfqpoint{0.100000in}{0.220728in}}{\pgfqpoint{3.696000in}{3.696000in}}%
\pgfusepath{clip}%
\pgfsetrectcap%
\pgfsetroundjoin%
\pgfsetlinewidth{1.505625pt}%
\definecolor{currentstroke}{rgb}{1.000000,0.000000,0.000000}%
\pgfsetstrokecolor{currentstroke}%
\pgfsetdash{}{0pt}%
\pgfpathmoveto{\pgfqpoint{1.930932in}{1.670253in}}%
\pgfpathlineto{\pgfqpoint{2.026710in}{1.087757in}}%
\pgfusepath{stroke}%
\end{pgfscope}%
\begin{pgfscope}%
\pgfpathrectangle{\pgfqpoint{0.100000in}{0.220728in}}{\pgfqpoint{3.696000in}{3.696000in}}%
\pgfusepath{clip}%
\pgfsetrectcap%
\pgfsetroundjoin%
\pgfsetlinewidth{1.505625pt}%
\definecolor{currentstroke}{rgb}{1.000000,0.000000,0.000000}%
\pgfsetstrokecolor{currentstroke}%
\pgfsetdash{}{0pt}%
\pgfpathmoveto{\pgfqpoint{1.903096in}{1.684629in}}%
\pgfpathlineto{\pgfqpoint{1.998035in}{1.096769in}}%
\pgfusepath{stroke}%
\end{pgfscope}%
\begin{pgfscope}%
\pgfpathrectangle{\pgfqpoint{0.100000in}{0.220728in}}{\pgfqpoint{3.696000in}{3.696000in}}%
\pgfusepath{clip}%
\pgfsetrectcap%
\pgfsetroundjoin%
\pgfsetlinewidth{1.505625pt}%
\definecolor{currentstroke}{rgb}{1.000000,0.000000,0.000000}%
\pgfsetstrokecolor{currentstroke}%
\pgfsetdash{}{0pt}%
\pgfpathmoveto{\pgfqpoint{1.874279in}{1.699205in}}%
\pgfpathlineto{\pgfqpoint{1.969400in}{1.105767in}}%
\pgfusepath{stroke}%
\end{pgfscope}%
\begin{pgfscope}%
\pgfpathrectangle{\pgfqpoint{0.100000in}{0.220728in}}{\pgfqpoint{3.696000in}{3.696000in}}%
\pgfusepath{clip}%
\pgfsetrectcap%
\pgfsetroundjoin%
\pgfsetlinewidth{1.505625pt}%
\definecolor{currentstroke}{rgb}{1.000000,0.000000,0.000000}%
\pgfsetstrokecolor{currentstroke}%
\pgfsetdash{}{0pt}%
\pgfpathmoveto{\pgfqpoint{1.842136in}{1.723358in}}%
\pgfpathlineto{\pgfqpoint{1.940806in}{1.114753in}}%
\pgfusepath{stroke}%
\end{pgfscope}%
\begin{pgfscope}%
\pgfpathrectangle{\pgfqpoint{0.100000in}{0.220728in}}{\pgfqpoint{3.696000in}{3.696000in}}%
\pgfusepath{clip}%
\pgfsetrectcap%
\pgfsetroundjoin%
\pgfsetlinewidth{1.505625pt}%
\definecolor{currentstroke}{rgb}{1.000000,0.000000,0.000000}%
\pgfsetstrokecolor{currentstroke}%
\pgfsetdash{}{0pt}%
\pgfpathmoveto{\pgfqpoint{1.808439in}{1.754747in}}%
\pgfpathlineto{\pgfqpoint{1.912254in}{1.123725in}}%
\pgfusepath{stroke}%
\end{pgfscope}%
\begin{pgfscope}%
\pgfpathrectangle{\pgfqpoint{0.100000in}{0.220728in}}{\pgfqpoint{3.696000in}{3.696000in}}%
\pgfusepath{clip}%
\pgfsetrectcap%
\pgfsetroundjoin%
\pgfsetlinewidth{1.505625pt}%
\definecolor{currentstroke}{rgb}{1.000000,0.000000,0.000000}%
\pgfsetstrokecolor{currentstroke}%
\pgfsetdash{}{0pt}%
\pgfpathmoveto{\pgfqpoint{1.771125in}{1.773971in}}%
\pgfpathlineto{\pgfqpoint{1.869501in}{1.137160in}}%
\pgfusepath{stroke}%
\end{pgfscope}%
\begin{pgfscope}%
\pgfpathrectangle{\pgfqpoint{0.100000in}{0.220728in}}{\pgfqpoint{3.696000in}{3.696000in}}%
\pgfusepath{clip}%
\pgfsetrectcap%
\pgfsetroundjoin%
\pgfsetlinewidth{1.505625pt}%
\definecolor{currentstroke}{rgb}{1.000000,0.000000,0.000000}%
\pgfsetstrokecolor{currentstroke}%
\pgfsetdash{}{0pt}%
\pgfpathmoveto{\pgfqpoint{1.732638in}{1.797255in}}%
\pgfpathlineto{\pgfqpoint{1.826840in}{1.150567in}}%
\pgfusepath{stroke}%
\end{pgfscope}%
\begin{pgfscope}%
\pgfpathrectangle{\pgfqpoint{0.100000in}{0.220728in}}{\pgfqpoint{3.696000in}{3.696000in}}%
\pgfusepath{clip}%
\pgfsetrectcap%
\pgfsetroundjoin%
\pgfsetlinewidth{1.505625pt}%
\definecolor{currentstroke}{rgb}{1.000000,0.000000,0.000000}%
\pgfsetstrokecolor{currentstroke}%
\pgfsetdash{}{0pt}%
\pgfpathmoveto{\pgfqpoint{1.693756in}{1.817008in}}%
\pgfpathlineto{\pgfqpoint{1.798450in}{1.159488in}}%
\pgfusepath{stroke}%
\end{pgfscope}%
\begin{pgfscope}%
\pgfpathrectangle{\pgfqpoint{0.100000in}{0.220728in}}{\pgfqpoint{3.696000in}{3.696000in}}%
\pgfusepath{clip}%
\pgfsetrectcap%
\pgfsetroundjoin%
\pgfsetlinewidth{1.505625pt}%
\definecolor{currentstroke}{rgb}{1.000000,0.000000,0.000000}%
\pgfsetstrokecolor{currentstroke}%
\pgfsetdash{}{0pt}%
\pgfpathmoveto{\pgfqpoint{1.672804in}{1.835870in}}%
\pgfpathlineto{\pgfqpoint{1.770101in}{1.168397in}}%
\pgfusepath{stroke}%
\end{pgfscope}%
\begin{pgfscope}%
\pgfpathrectangle{\pgfqpoint{0.100000in}{0.220728in}}{\pgfqpoint{3.696000in}{3.696000in}}%
\pgfusepath{clip}%
\pgfsetrectcap%
\pgfsetroundjoin%
\pgfsetlinewidth{1.505625pt}%
\definecolor{currentstroke}{rgb}{1.000000,0.000000,0.000000}%
\pgfsetstrokecolor{currentstroke}%
\pgfsetdash{}{0pt}%
\pgfpathmoveto{\pgfqpoint{1.648667in}{1.859920in}}%
\pgfpathlineto{\pgfqpoint{1.741792in}{1.177293in}}%
\pgfusepath{stroke}%
\end{pgfscope}%
\begin{pgfscope}%
\pgfpathrectangle{\pgfqpoint{0.100000in}{0.220728in}}{\pgfqpoint{3.696000in}{3.696000in}}%
\pgfusepath{clip}%
\pgfsetrectcap%
\pgfsetroundjoin%
\pgfsetlinewidth{1.505625pt}%
\definecolor{currentstroke}{rgb}{1.000000,0.000000,0.000000}%
\pgfsetstrokecolor{currentstroke}%
\pgfsetdash{}{0pt}%
\pgfpathmoveto{\pgfqpoint{1.623167in}{1.874470in}}%
\pgfpathlineto{\pgfqpoint{1.713523in}{1.186176in}}%
\pgfusepath{stroke}%
\end{pgfscope}%
\begin{pgfscope}%
\pgfpathrectangle{\pgfqpoint{0.100000in}{0.220728in}}{\pgfqpoint{3.696000in}{3.696000in}}%
\pgfusepath{clip}%
\pgfsetrectcap%
\pgfsetroundjoin%
\pgfsetlinewidth{1.505625pt}%
\definecolor{currentstroke}{rgb}{1.000000,0.000000,0.000000}%
\pgfsetstrokecolor{currentstroke}%
\pgfsetdash{}{0pt}%
\pgfpathmoveto{\pgfqpoint{1.590742in}{1.908026in}}%
\pgfpathlineto{\pgfqpoint{1.685295in}{1.195047in}}%
\pgfusepath{stroke}%
\end{pgfscope}%
\begin{pgfscope}%
\pgfpathrectangle{\pgfqpoint{0.100000in}{0.220728in}}{\pgfqpoint{3.696000in}{3.696000in}}%
\pgfusepath{clip}%
\pgfsetrectcap%
\pgfsetroundjoin%
\pgfsetlinewidth{1.505625pt}%
\definecolor{currentstroke}{rgb}{1.000000,0.000000,0.000000}%
\pgfsetstrokecolor{currentstroke}%
\pgfsetdash{}{0pt}%
\pgfpathmoveto{\pgfqpoint{1.559772in}{1.924476in}}%
\pgfpathlineto{\pgfqpoint{1.657106in}{1.203905in}}%
\pgfusepath{stroke}%
\end{pgfscope}%
\begin{pgfscope}%
\pgfpathrectangle{\pgfqpoint{0.100000in}{0.220728in}}{\pgfqpoint{3.696000in}{3.696000in}}%
\pgfusepath{clip}%
\pgfsetrectcap%
\pgfsetroundjoin%
\pgfsetlinewidth{1.505625pt}%
\definecolor{currentstroke}{rgb}{1.000000,0.000000,0.000000}%
\pgfsetstrokecolor{currentstroke}%
\pgfsetdash{}{0pt}%
\pgfpathmoveto{\pgfqpoint{1.524892in}{1.945712in}}%
\pgfpathlineto{\pgfqpoint{1.628958in}{1.212751in}}%
\pgfusepath{stroke}%
\end{pgfscope}%
\begin{pgfscope}%
\pgfpathrectangle{\pgfqpoint{0.100000in}{0.220728in}}{\pgfqpoint{3.696000in}{3.696000in}}%
\pgfusepath{clip}%
\pgfsetrectcap%
\pgfsetroundjoin%
\pgfsetlinewidth{1.505625pt}%
\definecolor{currentstroke}{rgb}{1.000000,0.000000,0.000000}%
\pgfsetstrokecolor{currentstroke}%
\pgfsetdash{}{0pt}%
\pgfpathmoveto{\pgfqpoint{1.487633in}{1.968595in}}%
\pgfpathlineto{\pgfqpoint{1.586811in}{1.225996in}}%
\pgfusepath{stroke}%
\end{pgfscope}%
\begin{pgfscope}%
\pgfpathrectangle{\pgfqpoint{0.100000in}{0.220728in}}{\pgfqpoint{3.696000in}{3.696000in}}%
\pgfusepath{clip}%
\pgfsetrectcap%
\pgfsetroundjoin%
\pgfsetlinewidth{1.505625pt}%
\definecolor{currentstroke}{rgb}{1.000000,0.000000,0.000000}%
\pgfsetstrokecolor{currentstroke}%
\pgfsetdash{}{0pt}%
\pgfpathmoveto{\pgfqpoint{1.449382in}{2.001773in}}%
\pgfpathlineto{\pgfqpoint{1.544753in}{1.239212in}}%
\pgfusepath{stroke}%
\end{pgfscope}%
\begin{pgfscope}%
\pgfpathrectangle{\pgfqpoint{0.100000in}{0.220728in}}{\pgfqpoint{3.696000in}{3.696000in}}%
\pgfusepath{clip}%
\pgfsetrectcap%
\pgfsetroundjoin%
\pgfsetlinewidth{1.505625pt}%
\definecolor{currentstroke}{rgb}{1.000000,0.000000,0.000000}%
\pgfsetstrokecolor{currentstroke}%
\pgfsetdash{}{0pt}%
\pgfpathmoveto{\pgfqpoint{1.409481in}{2.021836in}}%
\pgfpathlineto{\pgfqpoint{1.502784in}{1.252401in}}%
\pgfusepath{stroke}%
\end{pgfscope}%
\begin{pgfscope}%
\pgfpathrectangle{\pgfqpoint{0.100000in}{0.220728in}}{\pgfqpoint{3.696000in}{3.696000in}}%
\pgfusepath{clip}%
\pgfsetrectcap%
\pgfsetroundjoin%
\pgfsetlinewidth{1.505625pt}%
\definecolor{currentstroke}{rgb}{1.000000,0.000000,0.000000}%
\pgfsetstrokecolor{currentstroke}%
\pgfsetdash{}{0pt}%
\pgfpathmoveto{\pgfqpoint{1.370390in}{2.055532in}}%
\pgfpathlineto{\pgfqpoint{1.460905in}{1.265562in}}%
\pgfusepath{stroke}%
\end{pgfscope}%
\begin{pgfscope}%
\pgfpathrectangle{\pgfqpoint{0.100000in}{0.220728in}}{\pgfqpoint{3.696000in}{3.696000in}}%
\pgfusepath{clip}%
\pgfsetrectcap%
\pgfsetroundjoin%
\pgfsetlinewidth{1.505625pt}%
\definecolor{currentstroke}{rgb}{1.000000,0.000000,0.000000}%
\pgfsetstrokecolor{currentstroke}%
\pgfsetdash{}{0pt}%
\pgfpathmoveto{\pgfqpoint{1.325194in}{2.089947in}}%
\pgfpathlineto{\pgfqpoint{1.419114in}{1.278694in}}%
\pgfusepath{stroke}%
\end{pgfscope}%
\begin{pgfscope}%
\pgfpathrectangle{\pgfqpoint{0.100000in}{0.220728in}}{\pgfqpoint{3.696000in}{3.696000in}}%
\pgfusepath{clip}%
\pgfsetrectcap%
\pgfsetroundjoin%
\pgfsetlinewidth{1.505625pt}%
\definecolor{currentstroke}{rgb}{1.000000,0.000000,0.000000}%
\pgfsetstrokecolor{currentstroke}%
\pgfsetdash{}{0pt}%
\pgfpathmoveto{\pgfqpoint{1.280557in}{2.121884in}}%
\pgfpathlineto{\pgfqpoint{1.377412in}{1.291799in}}%
\pgfusepath{stroke}%
\end{pgfscope}%
\begin{pgfscope}%
\pgfpathrectangle{\pgfqpoint{0.100000in}{0.220728in}}{\pgfqpoint{3.696000in}{3.696000in}}%
\pgfusepath{clip}%
\pgfsetrectcap%
\pgfsetroundjoin%
\pgfsetlinewidth{1.505625pt}%
\definecolor{currentstroke}{rgb}{1.000000,0.000000,0.000000}%
\pgfsetstrokecolor{currentstroke}%
\pgfsetdash{}{0pt}%
\pgfpathmoveto{\pgfqpoint{1.228095in}{2.160722in}}%
\pgfpathlineto{\pgfqpoint{1.321947in}{1.309229in}}%
\pgfusepath{stroke}%
\end{pgfscope}%
\begin{pgfscope}%
\pgfpathrectangle{\pgfqpoint{0.100000in}{0.220728in}}{\pgfqpoint{3.696000in}{3.696000in}}%
\pgfusepath{clip}%
\pgfsetrectcap%
\pgfsetroundjoin%
\pgfsetlinewidth{1.505625pt}%
\definecolor{currentstroke}{rgb}{1.000000,0.000000,0.000000}%
\pgfsetstrokecolor{currentstroke}%
\pgfsetdash{}{0pt}%
\pgfpathmoveto{\pgfqpoint{1.178449in}{2.190267in}}%
\pgfpathlineto{\pgfqpoint{1.280450in}{1.322269in}}%
\pgfusepath{stroke}%
\end{pgfscope}%
\begin{pgfscope}%
\pgfpathrectangle{\pgfqpoint{0.100000in}{0.220728in}}{\pgfqpoint{3.696000in}{3.696000in}}%
\pgfusepath{clip}%
\pgfsetrectcap%
\pgfsetroundjoin%
\pgfsetlinewidth{1.505625pt}%
\definecolor{currentstroke}{rgb}{1.000000,0.000000,0.000000}%
\pgfsetstrokecolor{currentstroke}%
\pgfsetdash{}{0pt}%
\pgfpathmoveto{\pgfqpoint{1.129135in}{2.231599in}}%
\pgfpathlineto{\pgfqpoint{1.225258in}{1.339614in}}%
\pgfusepath{stroke}%
\end{pgfscope}%
\begin{pgfscope}%
\pgfpathrectangle{\pgfqpoint{0.100000in}{0.220728in}}{\pgfqpoint{3.696000in}{3.696000in}}%
\pgfusepath{clip}%
\pgfsetrectcap%
\pgfsetroundjoin%
\pgfsetlinewidth{1.505625pt}%
\definecolor{currentstroke}{rgb}{1.000000,0.000000,0.000000}%
\pgfsetstrokecolor{currentstroke}%
\pgfsetdash{}{0pt}%
\pgfpathmoveto{\pgfqpoint{1.076500in}{2.275057in}}%
\pgfpathlineto{\pgfqpoint{1.170220in}{1.356909in}}%
\pgfusepath{stroke}%
\end{pgfscope}%
\begin{pgfscope}%
\pgfpathrectangle{\pgfqpoint{0.100000in}{0.220728in}}{\pgfqpoint{3.696000in}{3.696000in}}%
\pgfusepath{clip}%
\pgfsetrectcap%
\pgfsetroundjoin%
\pgfsetlinewidth{1.505625pt}%
\definecolor{currentstroke}{rgb}{1.000000,0.000000,0.000000}%
\pgfsetstrokecolor{currentstroke}%
\pgfsetdash{}{0pt}%
\pgfpathmoveto{\pgfqpoint{1.022447in}{2.321087in}}%
\pgfpathlineto{\pgfqpoint{1.115337in}{1.374156in}}%
\pgfusepath{stroke}%
\end{pgfscope}%
\begin{pgfscope}%
\pgfpathrectangle{\pgfqpoint{0.100000in}{0.220728in}}{\pgfqpoint{3.696000in}{3.696000in}}%
\pgfusepath{clip}%
\pgfsetrectcap%
\pgfsetroundjoin%
\pgfsetlinewidth{1.505625pt}%
\definecolor{currentstroke}{rgb}{1.000000,0.000000,0.000000}%
\pgfsetstrokecolor{currentstroke}%
\pgfsetdash{}{0pt}%
\pgfpathmoveto{\pgfqpoint{0.963315in}{2.376217in}}%
\pgfpathlineto{\pgfqpoint{1.060607in}{1.391355in}}%
\pgfusepath{stroke}%
\end{pgfscope}%
\begin{pgfscope}%
\pgfpathrectangle{\pgfqpoint{0.100000in}{0.220728in}}{\pgfqpoint{3.696000in}{3.696000in}}%
\pgfusepath{clip}%
\pgfsetrectcap%
\pgfsetroundjoin%
\pgfsetlinewidth{1.505625pt}%
\definecolor{currentstroke}{rgb}{1.000000,0.000000,0.000000}%
\pgfsetstrokecolor{currentstroke}%
\pgfsetdash{}{0pt}%
\pgfpathmoveto{\pgfqpoint{0.931121in}{2.397359in}}%
\pgfpathlineto{\pgfqpoint{1.019661in}{1.404222in}}%
\pgfusepath{stroke}%
\end{pgfscope}%
\begin{pgfscope}%
\pgfpathrectangle{\pgfqpoint{0.100000in}{0.220728in}}{\pgfqpoint{3.696000in}{3.696000in}}%
\pgfusepath{clip}%
\pgfsetrectcap%
\pgfsetroundjoin%
\pgfsetlinewidth{1.505625pt}%
\definecolor{currentstroke}{rgb}{1.000000,0.000000,0.000000}%
\pgfsetstrokecolor{currentstroke}%
\pgfsetdash{}{0pt}%
\pgfpathmoveto{\pgfqpoint{0.900341in}{2.440087in}}%
\pgfpathlineto{\pgfqpoint{0.992411in}{1.412786in}}%
\pgfusepath{stroke}%
\end{pgfscope}%
\begin{pgfscope}%
\pgfpathrectangle{\pgfqpoint{0.100000in}{0.220728in}}{\pgfqpoint{3.696000in}{3.696000in}}%
\pgfusepath{clip}%
\pgfsetrectcap%
\pgfsetroundjoin%
\pgfsetlinewidth{1.505625pt}%
\definecolor{currentstroke}{rgb}{1.000000,0.000000,0.000000}%
\pgfsetstrokecolor{currentstroke}%
\pgfsetdash{}{0pt}%
\pgfpathmoveto{\pgfqpoint{0.863686in}{2.457282in}}%
\pgfpathlineto{\pgfqpoint{0.951607in}{1.425608in}}%
\pgfusepath{stroke}%
\end{pgfscope}%
\begin{pgfscope}%
\pgfpathrectangle{\pgfqpoint{0.100000in}{0.220728in}}{\pgfqpoint{3.696000in}{3.696000in}}%
\pgfusepath{clip}%
\pgfsetrectcap%
\pgfsetroundjoin%
\pgfsetlinewidth{1.505625pt}%
\definecolor{currentstroke}{rgb}{1.000000,0.000000,0.000000}%
\pgfsetstrokecolor{currentstroke}%
\pgfsetdash{}{0pt}%
\pgfpathmoveto{\pgfqpoint{0.824082in}{2.483649in}}%
\pgfpathlineto{\pgfqpoint{0.910889in}{1.438404in}}%
\pgfusepath{stroke}%
\end{pgfscope}%
\begin{pgfscope}%
\pgfpathrectangle{\pgfqpoint{0.100000in}{0.220728in}}{\pgfqpoint{3.696000in}{3.696000in}}%
\pgfusepath{clip}%
\pgfsetrectcap%
\pgfsetroundjoin%
\pgfsetlinewidth{1.505625pt}%
\definecolor{currentstroke}{rgb}{1.000000,0.000000,0.000000}%
\pgfsetstrokecolor{currentstroke}%
\pgfsetdash{}{0pt}%
\pgfpathmoveto{\pgfqpoint{0.785957in}{2.515365in}}%
\pgfpathlineto{\pgfqpoint{0.883791in}{1.446920in}}%
\pgfusepath{stroke}%
\end{pgfscope}%
\begin{pgfscope}%
\pgfpathrectangle{\pgfqpoint{0.100000in}{0.220728in}}{\pgfqpoint{3.696000in}{3.696000in}}%
\pgfusepath{clip}%
\pgfsetrectcap%
\pgfsetroundjoin%
\pgfsetlinewidth{1.505625pt}%
\definecolor{currentstroke}{rgb}{1.000000,0.000000,0.000000}%
\pgfsetstrokecolor{currentstroke}%
\pgfsetdash{}{0pt}%
\pgfpathmoveto{\pgfqpoint{0.744444in}{2.560464in}}%
\pgfpathlineto{\pgfqpoint{0.883791in}{1.446920in}}%
\pgfusepath{stroke}%
\end{pgfscope}%
\begin{pgfscope}%
\pgfpathrectangle{\pgfqpoint{0.100000in}{0.220728in}}{\pgfqpoint{3.696000in}{3.696000in}}%
\pgfusepath{clip}%
\pgfsetrectcap%
\pgfsetroundjoin%
\pgfsetlinewidth{1.505625pt}%
\definecolor{currentstroke}{rgb}{1.000000,0.000000,0.000000}%
\pgfsetstrokecolor{currentstroke}%
\pgfsetdash{}{0pt}%
\pgfpathmoveto{\pgfqpoint{0.703063in}{2.602504in}}%
\pgfpathlineto{\pgfqpoint{0.883791in}{1.446920in}}%
\pgfusepath{stroke}%
\end{pgfscope}%
\begin{pgfscope}%
\pgfpathrectangle{\pgfqpoint{0.100000in}{0.220728in}}{\pgfqpoint{3.696000in}{3.696000in}}%
\pgfusepath{clip}%
\pgfsetrectcap%
\pgfsetroundjoin%
\pgfsetlinewidth{1.505625pt}%
\definecolor{currentstroke}{rgb}{1.000000,0.000000,0.000000}%
\pgfsetstrokecolor{currentstroke}%
\pgfsetdash{}{0pt}%
\pgfpathmoveto{\pgfqpoint{0.682250in}{2.609493in}}%
\pgfpathlineto{\pgfqpoint{0.883791in}{1.446920in}}%
\pgfusepath{stroke}%
\end{pgfscope}%
\begin{pgfscope}%
\pgfpathrectangle{\pgfqpoint{0.100000in}{0.220728in}}{\pgfqpoint{3.696000in}{3.696000in}}%
\pgfusepath{clip}%
\pgfsetrectcap%
\pgfsetroundjoin%
\pgfsetlinewidth{1.505625pt}%
\definecolor{currentstroke}{rgb}{1.000000,0.000000,0.000000}%
\pgfsetstrokecolor{currentstroke}%
\pgfsetdash{}{0pt}%
\pgfpathmoveto{\pgfqpoint{0.669862in}{2.622925in}}%
\pgfpathlineto{\pgfqpoint{0.883791in}{1.446920in}}%
\pgfusepath{stroke}%
\end{pgfscope}%
\begin{pgfscope}%
\pgfpathrectangle{\pgfqpoint{0.100000in}{0.220728in}}{\pgfqpoint{3.696000in}{3.696000in}}%
\pgfusepath{clip}%
\pgfsetrectcap%
\pgfsetroundjoin%
\pgfsetlinewidth{1.505625pt}%
\definecolor{currentstroke}{rgb}{1.000000,0.000000,0.000000}%
\pgfsetstrokecolor{currentstroke}%
\pgfsetdash{}{0pt}%
\pgfpathmoveto{\pgfqpoint{0.658315in}{2.628329in}}%
\pgfpathlineto{\pgfqpoint{0.883791in}{1.446920in}}%
\pgfusepath{stroke}%
\end{pgfscope}%
\begin{pgfscope}%
\pgfpathrectangle{\pgfqpoint{0.100000in}{0.220728in}}{\pgfqpoint{3.696000in}{3.696000in}}%
\pgfusepath{clip}%
\pgfsetrectcap%
\pgfsetroundjoin%
\pgfsetlinewidth{1.505625pt}%
\definecolor{currentstroke}{rgb}{1.000000,0.000000,0.000000}%
\pgfsetstrokecolor{currentstroke}%
\pgfsetdash{}{0pt}%
\pgfpathmoveto{\pgfqpoint{0.652050in}{2.635546in}}%
\pgfpathlineto{\pgfqpoint{0.883791in}{1.446920in}}%
\pgfusepath{stroke}%
\end{pgfscope}%
\begin{pgfscope}%
\pgfpathrectangle{\pgfqpoint{0.100000in}{0.220728in}}{\pgfqpoint{3.696000in}{3.696000in}}%
\pgfusepath{clip}%
\pgfsetrectcap%
\pgfsetroundjoin%
\pgfsetlinewidth{1.505625pt}%
\definecolor{currentstroke}{rgb}{1.000000,0.000000,0.000000}%
\pgfsetstrokecolor{currentstroke}%
\pgfsetdash{}{0pt}%
\pgfpathmoveto{\pgfqpoint{0.645732in}{2.637163in}}%
\pgfpathlineto{\pgfqpoint{0.883791in}{1.446920in}}%
\pgfusepath{stroke}%
\end{pgfscope}%
\begin{pgfscope}%
\pgfpathrectangle{\pgfqpoint{0.100000in}{0.220728in}}{\pgfqpoint{3.696000in}{3.696000in}}%
\pgfusepath{clip}%
\pgfsetrectcap%
\pgfsetroundjoin%
\pgfsetlinewidth{1.505625pt}%
\definecolor{currentstroke}{rgb}{1.000000,0.000000,0.000000}%
\pgfsetstrokecolor{currentstroke}%
\pgfsetdash{}{0pt}%
\pgfpathmoveto{\pgfqpoint{0.635681in}{2.655980in}}%
\pgfpathlineto{\pgfqpoint{0.883791in}{1.446920in}}%
\pgfusepath{stroke}%
\end{pgfscope}%
\begin{pgfscope}%
\pgfpathrectangle{\pgfqpoint{0.100000in}{0.220728in}}{\pgfqpoint{3.696000in}{3.696000in}}%
\pgfusepath{clip}%
\pgfsetrectcap%
\pgfsetroundjoin%
\pgfsetlinewidth{1.505625pt}%
\definecolor{currentstroke}{rgb}{1.000000,0.000000,0.000000}%
\pgfsetstrokecolor{currentstroke}%
\pgfsetdash{}{0pt}%
\pgfpathmoveto{\pgfqpoint{0.624273in}{2.661974in}}%
\pgfpathlineto{\pgfqpoint{0.883791in}{1.446920in}}%
\pgfusepath{stroke}%
\end{pgfscope}%
\begin{pgfscope}%
\pgfpathrectangle{\pgfqpoint{0.100000in}{0.220728in}}{\pgfqpoint{3.696000in}{3.696000in}}%
\pgfusepath{clip}%
\pgfsetrectcap%
\pgfsetroundjoin%
\pgfsetlinewidth{1.505625pt}%
\definecolor{currentstroke}{rgb}{1.000000,0.000000,0.000000}%
\pgfsetstrokecolor{currentstroke}%
\pgfsetdash{}{0pt}%
\pgfpathmoveto{\pgfqpoint{0.617087in}{2.678689in}}%
\pgfpathlineto{\pgfqpoint{0.883791in}{1.446920in}}%
\pgfusepath{stroke}%
\end{pgfscope}%
\begin{pgfscope}%
\pgfpathrectangle{\pgfqpoint{0.100000in}{0.220728in}}{\pgfqpoint{3.696000in}{3.696000in}}%
\pgfusepath{clip}%
\pgfsetrectcap%
\pgfsetroundjoin%
\pgfsetlinewidth{1.505625pt}%
\definecolor{currentstroke}{rgb}{1.000000,0.000000,0.000000}%
\pgfsetstrokecolor{currentstroke}%
\pgfsetdash{}{0pt}%
\pgfpathmoveto{\pgfqpoint{0.614009in}{2.677886in}}%
\pgfpathlineto{\pgfqpoint{0.883791in}{1.446920in}}%
\pgfusepath{stroke}%
\end{pgfscope}%
\begin{pgfscope}%
\pgfpathrectangle{\pgfqpoint{0.100000in}{0.220728in}}{\pgfqpoint{3.696000in}{3.696000in}}%
\pgfusepath{clip}%
\pgfsetrectcap%
\pgfsetroundjoin%
\pgfsetlinewidth{1.505625pt}%
\definecolor{currentstroke}{rgb}{1.000000,0.000000,0.000000}%
\pgfsetstrokecolor{currentstroke}%
\pgfsetdash{}{0pt}%
\pgfpathmoveto{\pgfqpoint{0.611889in}{2.680374in}}%
\pgfpathlineto{\pgfqpoint{0.883791in}{1.446920in}}%
\pgfusepath{stroke}%
\end{pgfscope}%
\begin{pgfscope}%
\pgfpathrectangle{\pgfqpoint{0.100000in}{0.220728in}}{\pgfqpoint{3.696000in}{3.696000in}}%
\pgfusepath{clip}%
\pgfsetrectcap%
\pgfsetroundjoin%
\pgfsetlinewidth{1.505625pt}%
\definecolor{currentstroke}{rgb}{1.000000,0.000000,0.000000}%
\pgfsetstrokecolor{currentstroke}%
\pgfsetdash{}{0pt}%
\pgfpathmoveto{\pgfqpoint{0.610810in}{2.679961in}}%
\pgfpathlineto{\pgfqpoint{0.883791in}{1.446920in}}%
\pgfusepath{stroke}%
\end{pgfscope}%
\begin{pgfscope}%
\pgfpathrectangle{\pgfqpoint{0.100000in}{0.220728in}}{\pgfqpoint{3.696000in}{3.696000in}}%
\pgfusepath{clip}%
\pgfsetrectcap%
\pgfsetroundjoin%
\pgfsetlinewidth{1.505625pt}%
\definecolor{currentstroke}{rgb}{1.000000,0.000000,0.000000}%
\pgfsetstrokecolor{currentstroke}%
\pgfsetdash{}{0pt}%
\pgfpathmoveto{\pgfqpoint{0.610257in}{2.679970in}}%
\pgfpathlineto{\pgfqpoint{0.883791in}{1.446920in}}%
\pgfusepath{stroke}%
\end{pgfscope}%
\begin{pgfscope}%
\pgfpathrectangle{\pgfqpoint{0.100000in}{0.220728in}}{\pgfqpoint{3.696000in}{3.696000in}}%
\pgfusepath{clip}%
\pgfsetrectcap%
\pgfsetroundjoin%
\pgfsetlinewidth{1.505625pt}%
\definecolor{currentstroke}{rgb}{1.000000,0.000000,0.000000}%
\pgfsetstrokecolor{currentstroke}%
\pgfsetdash{}{0pt}%
\pgfpathmoveto{\pgfqpoint{0.607750in}{2.678230in}}%
\pgfpathlineto{\pgfqpoint{0.883791in}{1.446920in}}%
\pgfusepath{stroke}%
\end{pgfscope}%
\begin{pgfscope}%
\pgfpathrectangle{\pgfqpoint{0.100000in}{0.220728in}}{\pgfqpoint{3.696000in}{3.696000in}}%
\pgfusepath{clip}%
\pgfsetrectcap%
\pgfsetroundjoin%
\pgfsetlinewidth{1.505625pt}%
\definecolor{currentstroke}{rgb}{1.000000,0.000000,0.000000}%
\pgfsetstrokecolor{currentstroke}%
\pgfsetdash{}{0pt}%
\pgfpathmoveto{\pgfqpoint{0.603932in}{2.675580in}}%
\pgfpathlineto{\pgfqpoint{0.883791in}{1.446920in}}%
\pgfusepath{stroke}%
\end{pgfscope}%
\begin{pgfscope}%
\pgfpathrectangle{\pgfqpoint{0.100000in}{0.220728in}}{\pgfqpoint{3.696000in}{3.696000in}}%
\pgfusepath{clip}%
\pgfsetbuttcap%
\pgfsetroundjoin%
\definecolor{currentfill}{rgb}{0.121569,0.466667,0.705882}%
\pgfsetfillcolor{currentfill}%
\pgfsetfillopacity{0.300000}%
\pgfsetlinewidth{1.003750pt}%
\definecolor{currentstroke}{rgb}{0.121569,0.466667,0.705882}%
\pgfsetstrokecolor{currentstroke}%
\pgfsetstrokeopacity{0.300000}%
\pgfsetdash{}{0pt}%
\pgfpathmoveto{\pgfqpoint{1.950563in}{2.031750in}}%
\pgfpathcurveto{\pgfqpoint{1.958799in}{2.031750in}}{\pgfqpoint{1.966699in}{2.035022in}}{\pgfqpoint{1.972523in}{2.040846in}}%
\pgfpathcurveto{\pgfqpoint{1.978347in}{2.046670in}}{\pgfqpoint{1.981619in}{2.054570in}}{\pgfqpoint{1.981619in}{2.062806in}}%
\pgfpathcurveto{\pgfqpoint{1.981619in}{2.071043in}}{\pgfqpoint{1.978347in}{2.078943in}}{\pgfqpoint{1.972523in}{2.084767in}}%
\pgfpathcurveto{\pgfqpoint{1.966699in}{2.090591in}}{\pgfqpoint{1.958799in}{2.093863in}}{\pgfqpoint{1.950563in}{2.093863in}}%
\pgfpathcurveto{\pgfqpoint{1.942326in}{2.093863in}}{\pgfqpoint{1.934426in}{2.090591in}}{\pgfqpoint{1.928603in}{2.084767in}}%
\pgfpathcurveto{\pgfqpoint{1.922779in}{2.078943in}}{\pgfqpoint{1.919506in}{2.071043in}}{\pgfqpoint{1.919506in}{2.062806in}}%
\pgfpathcurveto{\pgfqpoint{1.919506in}{2.054570in}}{\pgfqpoint{1.922779in}{2.046670in}}{\pgfqpoint{1.928603in}{2.040846in}}%
\pgfpathcurveto{\pgfqpoint{1.934426in}{2.035022in}}{\pgfqpoint{1.942326in}{2.031750in}}{\pgfqpoint{1.950563in}{2.031750in}}%
\pgfpathclose%
\pgfusepath{stroke,fill}%
\end{pgfscope}%
\begin{pgfscope}%
\pgfpathrectangle{\pgfqpoint{0.100000in}{0.220728in}}{\pgfqpoint{3.696000in}{3.696000in}}%
\pgfusepath{clip}%
\pgfsetbuttcap%
\pgfsetroundjoin%
\definecolor{currentfill}{rgb}{0.121569,0.466667,0.705882}%
\pgfsetfillcolor{currentfill}%
\pgfsetfillopacity{0.300003}%
\pgfsetlinewidth{1.003750pt}%
\definecolor{currentstroke}{rgb}{0.121569,0.466667,0.705882}%
\pgfsetstrokecolor{currentstroke}%
\pgfsetstrokeopacity{0.300003}%
\pgfsetdash{}{0pt}%
\pgfpathmoveto{\pgfqpoint{1.950916in}{2.031370in}}%
\pgfpathcurveto{\pgfqpoint{1.959152in}{2.031370in}}{\pgfqpoint{1.967052in}{2.034642in}}{\pgfqpoint{1.972876in}{2.040466in}}%
\pgfpathcurveto{\pgfqpoint{1.978700in}{2.046290in}}{\pgfqpoint{1.981972in}{2.054190in}}{\pgfqpoint{1.981972in}{2.062427in}}%
\pgfpathcurveto{\pgfqpoint{1.981972in}{2.070663in}}{\pgfqpoint{1.978700in}{2.078563in}}{\pgfqpoint{1.972876in}{2.084387in}}%
\pgfpathcurveto{\pgfqpoint{1.967052in}{2.090211in}}{\pgfqpoint{1.959152in}{2.093483in}}{\pgfqpoint{1.950916in}{2.093483in}}%
\pgfpathcurveto{\pgfqpoint{1.942679in}{2.093483in}}{\pgfqpoint{1.934779in}{2.090211in}}{\pgfqpoint{1.928955in}{2.084387in}}%
\pgfpathcurveto{\pgfqpoint{1.923132in}{2.078563in}}{\pgfqpoint{1.919859in}{2.070663in}}{\pgfqpoint{1.919859in}{2.062427in}}%
\pgfpathcurveto{\pgfqpoint{1.919859in}{2.054190in}}{\pgfqpoint{1.923132in}{2.046290in}}{\pgfqpoint{1.928955in}{2.040466in}}%
\pgfpathcurveto{\pgfqpoint{1.934779in}{2.034642in}}{\pgfqpoint{1.942679in}{2.031370in}}{\pgfqpoint{1.950916in}{2.031370in}}%
\pgfpathclose%
\pgfusepath{stroke,fill}%
\end{pgfscope}%
\begin{pgfscope}%
\pgfpathrectangle{\pgfqpoint{0.100000in}{0.220728in}}{\pgfqpoint{3.696000in}{3.696000in}}%
\pgfusepath{clip}%
\pgfsetbuttcap%
\pgfsetroundjoin%
\definecolor{currentfill}{rgb}{0.121569,0.466667,0.705882}%
\pgfsetfillcolor{currentfill}%
\pgfsetfillopacity{0.300012}%
\pgfsetlinewidth{1.003750pt}%
\definecolor{currentstroke}{rgb}{0.121569,0.466667,0.705882}%
\pgfsetstrokecolor{currentstroke}%
\pgfsetstrokeopacity{0.300012}%
\pgfsetdash{}{0pt}%
\pgfpathmoveto{\pgfqpoint{1.949818in}{2.031984in}}%
\pgfpathcurveto{\pgfqpoint{1.958055in}{2.031984in}}{\pgfqpoint{1.965955in}{2.035256in}}{\pgfqpoint{1.971779in}{2.041080in}}%
\pgfpathcurveto{\pgfqpoint{1.977603in}{2.046904in}}{\pgfqpoint{1.980875in}{2.054804in}}{\pgfqpoint{1.980875in}{2.063041in}}%
\pgfpathcurveto{\pgfqpoint{1.980875in}{2.071277in}}{\pgfqpoint{1.977603in}{2.079177in}}{\pgfqpoint{1.971779in}{2.085001in}}%
\pgfpathcurveto{\pgfqpoint{1.965955in}{2.090825in}}{\pgfqpoint{1.958055in}{2.094097in}}{\pgfqpoint{1.949818in}{2.094097in}}%
\pgfpathcurveto{\pgfqpoint{1.941582in}{2.094097in}}{\pgfqpoint{1.933682in}{2.090825in}}{\pgfqpoint{1.927858in}{2.085001in}}%
\pgfpathcurveto{\pgfqpoint{1.922034in}{2.079177in}}{\pgfqpoint{1.918762in}{2.071277in}}{\pgfqpoint{1.918762in}{2.063041in}}%
\pgfpathcurveto{\pgfqpoint{1.918762in}{2.054804in}}{\pgfqpoint{1.922034in}{2.046904in}}{\pgfqpoint{1.927858in}{2.041080in}}%
\pgfpathcurveto{\pgfqpoint{1.933682in}{2.035256in}}{\pgfqpoint{1.941582in}{2.031984in}}{\pgfqpoint{1.949818in}{2.031984in}}%
\pgfpathclose%
\pgfusepath{stroke,fill}%
\end{pgfscope}%
\begin{pgfscope}%
\pgfpathrectangle{\pgfqpoint{0.100000in}{0.220728in}}{\pgfqpoint{3.696000in}{3.696000in}}%
\pgfusepath{clip}%
\pgfsetbuttcap%
\pgfsetroundjoin%
\definecolor{currentfill}{rgb}{0.121569,0.466667,0.705882}%
\pgfsetfillcolor{currentfill}%
\pgfsetfillopacity{0.300033}%
\pgfsetlinewidth{1.003750pt}%
\definecolor{currentstroke}{rgb}{0.121569,0.466667,0.705882}%
\pgfsetstrokecolor{currentstroke}%
\pgfsetstrokeopacity{0.300033}%
\pgfsetdash{}{0pt}%
\pgfpathmoveto{\pgfqpoint{1.951094in}{2.031312in}}%
\pgfpathcurveto{\pgfqpoint{1.959330in}{2.031312in}}{\pgfqpoint{1.967230in}{2.034584in}}{\pgfqpoint{1.973054in}{2.040408in}}%
\pgfpathcurveto{\pgfqpoint{1.978878in}{2.046232in}}{\pgfqpoint{1.982150in}{2.054132in}}{\pgfqpoint{1.982150in}{2.062368in}}%
\pgfpathcurveto{\pgfqpoint{1.982150in}{2.070604in}}{\pgfqpoint{1.978878in}{2.078504in}}{\pgfqpoint{1.973054in}{2.084328in}}%
\pgfpathcurveto{\pgfqpoint{1.967230in}{2.090152in}}{\pgfqpoint{1.959330in}{2.093425in}}{\pgfqpoint{1.951094in}{2.093425in}}%
\pgfpathcurveto{\pgfqpoint{1.942858in}{2.093425in}}{\pgfqpoint{1.934958in}{2.090152in}}{\pgfqpoint{1.929134in}{2.084328in}}%
\pgfpathcurveto{\pgfqpoint{1.923310in}{2.078504in}}{\pgfqpoint{1.920037in}{2.070604in}}{\pgfqpoint{1.920037in}{2.062368in}}%
\pgfpathcurveto{\pgfqpoint{1.920037in}{2.054132in}}{\pgfqpoint{1.923310in}{2.046232in}}{\pgfqpoint{1.929134in}{2.040408in}}%
\pgfpathcurveto{\pgfqpoint{1.934958in}{2.034584in}}{\pgfqpoint{1.942858in}{2.031312in}}{\pgfqpoint{1.951094in}{2.031312in}}%
\pgfpathclose%
\pgfusepath{stroke,fill}%
\end{pgfscope}%
\begin{pgfscope}%
\pgfpathrectangle{\pgfqpoint{0.100000in}{0.220728in}}{\pgfqpoint{3.696000in}{3.696000in}}%
\pgfusepath{clip}%
\pgfsetbuttcap%
\pgfsetroundjoin%
\definecolor{currentfill}{rgb}{0.121569,0.466667,0.705882}%
\pgfsetfillcolor{currentfill}%
\pgfsetfillopacity{0.300037}%
\pgfsetlinewidth{1.003750pt}%
\definecolor{currentstroke}{rgb}{0.121569,0.466667,0.705882}%
\pgfsetstrokecolor{currentstroke}%
\pgfsetstrokeopacity{0.300037}%
\pgfsetdash{}{0pt}%
\pgfpathmoveto{\pgfqpoint{1.949612in}{2.032128in}}%
\pgfpathcurveto{\pgfqpoint{1.957849in}{2.032128in}}{\pgfqpoint{1.965749in}{2.035400in}}{\pgfqpoint{1.971573in}{2.041224in}}%
\pgfpathcurveto{\pgfqpoint{1.977397in}{2.047048in}}{\pgfqpoint{1.980669in}{2.054948in}}{\pgfqpoint{1.980669in}{2.063185in}}%
\pgfpathcurveto{\pgfqpoint{1.980669in}{2.071421in}}{\pgfqpoint{1.977397in}{2.079321in}}{\pgfqpoint{1.971573in}{2.085145in}}%
\pgfpathcurveto{\pgfqpoint{1.965749in}{2.090969in}}{\pgfqpoint{1.957849in}{2.094241in}}{\pgfqpoint{1.949612in}{2.094241in}}%
\pgfpathcurveto{\pgfqpoint{1.941376in}{2.094241in}}{\pgfqpoint{1.933476in}{2.090969in}}{\pgfqpoint{1.927652in}{2.085145in}}%
\pgfpathcurveto{\pgfqpoint{1.921828in}{2.079321in}}{\pgfqpoint{1.918556in}{2.071421in}}{\pgfqpoint{1.918556in}{2.063185in}}%
\pgfpathcurveto{\pgfqpoint{1.918556in}{2.054948in}}{\pgfqpoint{1.921828in}{2.047048in}}{\pgfqpoint{1.927652in}{2.041224in}}%
\pgfpathcurveto{\pgfqpoint{1.933476in}{2.035400in}}{\pgfqpoint{1.941376in}{2.032128in}}{\pgfqpoint{1.949612in}{2.032128in}}%
\pgfpathclose%
\pgfusepath{stroke,fill}%
\end{pgfscope}%
\begin{pgfscope}%
\pgfpathrectangle{\pgfqpoint{0.100000in}{0.220728in}}{\pgfqpoint{3.696000in}{3.696000in}}%
\pgfusepath{clip}%
\pgfsetbuttcap%
\pgfsetroundjoin%
\definecolor{currentfill}{rgb}{0.121569,0.466667,0.705882}%
\pgfsetfillcolor{currentfill}%
\pgfsetfillopacity{0.300062}%
\pgfsetlinewidth{1.003750pt}%
\definecolor{currentstroke}{rgb}{0.121569,0.466667,0.705882}%
\pgfsetstrokecolor{currentstroke}%
\pgfsetstrokeopacity{0.300062}%
\pgfsetdash{}{0pt}%
\pgfpathmoveto{\pgfqpoint{1.951168in}{2.031316in}}%
\pgfpathcurveto{\pgfqpoint{1.959404in}{2.031316in}}{\pgfqpoint{1.967304in}{2.034588in}}{\pgfqpoint{1.973128in}{2.040412in}}%
\pgfpathcurveto{\pgfqpoint{1.978952in}{2.046236in}}{\pgfqpoint{1.982224in}{2.054136in}}{\pgfqpoint{1.982224in}{2.062372in}}%
\pgfpathcurveto{\pgfqpoint{1.982224in}{2.070608in}}{\pgfqpoint{1.978952in}{2.078508in}}{\pgfqpoint{1.973128in}{2.084332in}}%
\pgfpathcurveto{\pgfqpoint{1.967304in}{2.090156in}}{\pgfqpoint{1.959404in}{2.093429in}}{\pgfqpoint{1.951168in}{2.093429in}}%
\pgfpathcurveto{\pgfqpoint{1.942932in}{2.093429in}}{\pgfqpoint{1.935032in}{2.090156in}}{\pgfqpoint{1.929208in}{2.084332in}}%
\pgfpathcurveto{\pgfqpoint{1.923384in}{2.078508in}}{\pgfqpoint{1.920111in}{2.070608in}}{\pgfqpoint{1.920111in}{2.062372in}}%
\pgfpathcurveto{\pgfqpoint{1.920111in}{2.054136in}}{\pgfqpoint{1.923384in}{2.046236in}}{\pgfqpoint{1.929208in}{2.040412in}}%
\pgfpathcurveto{\pgfqpoint{1.935032in}{2.034588in}}{\pgfqpoint{1.942932in}{2.031316in}}{\pgfqpoint{1.951168in}{2.031316in}}%
\pgfpathclose%
\pgfusepath{stroke,fill}%
\end{pgfscope}%
\begin{pgfscope}%
\pgfpathrectangle{\pgfqpoint{0.100000in}{0.220728in}}{\pgfqpoint{3.696000in}{3.696000in}}%
\pgfusepath{clip}%
\pgfsetbuttcap%
\pgfsetroundjoin%
\definecolor{currentfill}{rgb}{0.121569,0.466667,0.705882}%
\pgfsetfillcolor{currentfill}%
\pgfsetfillopacity{0.300071}%
\pgfsetlinewidth{1.003750pt}%
\definecolor{currentstroke}{rgb}{0.121569,0.466667,0.705882}%
\pgfsetstrokecolor{currentstroke}%
\pgfsetstrokeopacity{0.300071}%
\pgfsetdash{}{0pt}%
\pgfpathmoveto{\pgfqpoint{1.949252in}{2.032099in}}%
\pgfpathcurveto{\pgfqpoint{1.957488in}{2.032099in}}{\pgfqpoint{1.965388in}{2.035371in}}{\pgfqpoint{1.971212in}{2.041195in}}%
\pgfpathcurveto{\pgfqpoint{1.977036in}{2.047019in}}{\pgfqpoint{1.980308in}{2.054919in}}{\pgfqpoint{1.980308in}{2.063155in}}%
\pgfpathcurveto{\pgfqpoint{1.980308in}{2.071391in}}{\pgfqpoint{1.977036in}{2.079291in}}{\pgfqpoint{1.971212in}{2.085115in}}%
\pgfpathcurveto{\pgfqpoint{1.965388in}{2.090939in}}{\pgfqpoint{1.957488in}{2.094212in}}{\pgfqpoint{1.949252in}{2.094212in}}%
\pgfpathcurveto{\pgfqpoint{1.941015in}{2.094212in}}{\pgfqpoint{1.933115in}{2.090939in}}{\pgfqpoint{1.927291in}{2.085115in}}%
\pgfpathcurveto{\pgfqpoint{1.921468in}{2.079291in}}{\pgfqpoint{1.918195in}{2.071391in}}{\pgfqpoint{1.918195in}{2.063155in}}%
\pgfpathcurveto{\pgfqpoint{1.918195in}{2.054919in}}{\pgfqpoint{1.921468in}{2.047019in}}{\pgfqpoint{1.927291in}{2.041195in}}%
\pgfpathcurveto{\pgfqpoint{1.933115in}{2.035371in}}{\pgfqpoint{1.941015in}{2.032099in}}{\pgfqpoint{1.949252in}{2.032099in}}%
\pgfpathclose%
\pgfusepath{stroke,fill}%
\end{pgfscope}%
\begin{pgfscope}%
\pgfpathrectangle{\pgfqpoint{0.100000in}{0.220728in}}{\pgfqpoint{3.696000in}{3.696000in}}%
\pgfusepath{clip}%
\pgfsetbuttcap%
\pgfsetroundjoin%
\definecolor{currentfill}{rgb}{0.121569,0.466667,0.705882}%
\pgfsetfillcolor{currentfill}%
\pgfsetfillopacity{0.300077}%
\pgfsetlinewidth{1.003750pt}%
\definecolor{currentstroke}{rgb}{0.121569,0.466667,0.705882}%
\pgfsetstrokecolor{currentstroke}%
\pgfsetstrokeopacity{0.300077}%
\pgfsetdash{}{0pt}%
\pgfpathmoveto{\pgfqpoint{1.951195in}{2.031295in}}%
\pgfpathcurveto{\pgfqpoint{1.959431in}{2.031295in}}{\pgfqpoint{1.967331in}{2.034567in}}{\pgfqpoint{1.973155in}{2.040391in}}%
\pgfpathcurveto{\pgfqpoint{1.978979in}{2.046215in}}{\pgfqpoint{1.982251in}{2.054115in}}{\pgfqpoint{1.982251in}{2.062351in}}%
\pgfpathcurveto{\pgfqpoint{1.982251in}{2.070588in}}{\pgfqpoint{1.978979in}{2.078488in}}{\pgfqpoint{1.973155in}{2.084312in}}%
\pgfpathcurveto{\pgfqpoint{1.967331in}{2.090135in}}{\pgfqpoint{1.959431in}{2.093408in}}{\pgfqpoint{1.951195in}{2.093408in}}%
\pgfpathcurveto{\pgfqpoint{1.942958in}{2.093408in}}{\pgfqpoint{1.935058in}{2.090135in}}{\pgfqpoint{1.929234in}{2.084312in}}%
\pgfpathcurveto{\pgfqpoint{1.923410in}{2.078488in}}{\pgfqpoint{1.920138in}{2.070588in}}{\pgfqpoint{1.920138in}{2.062351in}}%
\pgfpathcurveto{\pgfqpoint{1.920138in}{2.054115in}}{\pgfqpoint{1.923410in}{2.046215in}}{\pgfqpoint{1.929234in}{2.040391in}}%
\pgfpathcurveto{\pgfqpoint{1.935058in}{2.034567in}}{\pgfqpoint{1.942958in}{2.031295in}}{\pgfqpoint{1.951195in}{2.031295in}}%
\pgfpathclose%
\pgfusepath{stroke,fill}%
\end{pgfscope}%
\begin{pgfscope}%
\pgfpathrectangle{\pgfqpoint{0.100000in}{0.220728in}}{\pgfqpoint{3.696000in}{3.696000in}}%
\pgfusepath{clip}%
\pgfsetbuttcap%
\pgfsetroundjoin%
\definecolor{currentfill}{rgb}{0.121569,0.466667,0.705882}%
\pgfsetfillcolor{currentfill}%
\pgfsetfillopacity{0.300087}%
\pgfsetlinewidth{1.003750pt}%
\definecolor{currentstroke}{rgb}{0.121569,0.466667,0.705882}%
\pgfsetstrokecolor{currentstroke}%
\pgfsetstrokeopacity{0.300087}%
\pgfsetdash{}{0pt}%
\pgfpathmoveto{\pgfqpoint{1.951206in}{2.031290in}}%
\pgfpathcurveto{\pgfqpoint{1.959442in}{2.031290in}}{\pgfqpoint{1.967342in}{2.034562in}}{\pgfqpoint{1.973166in}{2.040386in}}%
\pgfpathcurveto{\pgfqpoint{1.978990in}{2.046210in}}{\pgfqpoint{1.982262in}{2.054110in}}{\pgfqpoint{1.982262in}{2.062346in}}%
\pgfpathcurveto{\pgfqpoint{1.982262in}{2.070582in}}{\pgfqpoint{1.978990in}{2.078482in}}{\pgfqpoint{1.973166in}{2.084306in}}%
\pgfpathcurveto{\pgfqpoint{1.967342in}{2.090130in}}{\pgfqpoint{1.959442in}{2.093403in}}{\pgfqpoint{1.951206in}{2.093403in}}%
\pgfpathcurveto{\pgfqpoint{1.942970in}{2.093403in}}{\pgfqpoint{1.935070in}{2.090130in}}{\pgfqpoint{1.929246in}{2.084306in}}%
\pgfpathcurveto{\pgfqpoint{1.923422in}{2.078482in}}{\pgfqpoint{1.920149in}{2.070582in}}{\pgfqpoint{1.920149in}{2.062346in}}%
\pgfpathcurveto{\pgfqpoint{1.920149in}{2.054110in}}{\pgfqpoint{1.923422in}{2.046210in}}{\pgfqpoint{1.929246in}{2.040386in}}%
\pgfpathcurveto{\pgfqpoint{1.935070in}{2.034562in}}{\pgfqpoint{1.942970in}{2.031290in}}{\pgfqpoint{1.951206in}{2.031290in}}%
\pgfpathclose%
\pgfusepath{stroke,fill}%
\end{pgfscope}%
\begin{pgfscope}%
\pgfpathrectangle{\pgfqpoint{0.100000in}{0.220728in}}{\pgfqpoint{3.696000in}{3.696000in}}%
\pgfusepath{clip}%
\pgfsetbuttcap%
\pgfsetroundjoin%
\definecolor{currentfill}{rgb}{0.121569,0.466667,0.705882}%
\pgfsetfillcolor{currentfill}%
\pgfsetfillopacity{0.300233}%
\pgfsetlinewidth{1.003750pt}%
\definecolor{currentstroke}{rgb}{0.121569,0.466667,0.705882}%
\pgfsetstrokecolor{currentstroke}%
\pgfsetstrokeopacity{0.300233}%
\pgfsetdash{}{0pt}%
\pgfpathmoveto{\pgfqpoint{1.951269in}{2.031013in}}%
\pgfpathcurveto{\pgfqpoint{1.959505in}{2.031013in}}{\pgfqpoint{1.967405in}{2.034285in}}{\pgfqpoint{1.973229in}{2.040109in}}%
\pgfpathcurveto{\pgfqpoint{1.979053in}{2.045933in}}{\pgfqpoint{1.982326in}{2.053833in}}{\pgfqpoint{1.982326in}{2.062069in}}%
\pgfpathcurveto{\pgfqpoint{1.982326in}{2.070306in}}{\pgfqpoint{1.979053in}{2.078206in}}{\pgfqpoint{1.973229in}{2.084030in}}%
\pgfpathcurveto{\pgfqpoint{1.967405in}{2.089854in}}{\pgfqpoint{1.959505in}{2.093126in}}{\pgfqpoint{1.951269in}{2.093126in}}%
\pgfpathcurveto{\pgfqpoint{1.943033in}{2.093126in}}{\pgfqpoint{1.935133in}{2.089854in}}{\pgfqpoint{1.929309in}{2.084030in}}%
\pgfpathcurveto{\pgfqpoint{1.923485in}{2.078206in}}{\pgfqpoint{1.920213in}{2.070306in}}{\pgfqpoint{1.920213in}{2.062069in}}%
\pgfpathcurveto{\pgfqpoint{1.920213in}{2.053833in}}{\pgfqpoint{1.923485in}{2.045933in}}{\pgfqpoint{1.929309in}{2.040109in}}%
\pgfpathcurveto{\pgfqpoint{1.935133in}{2.034285in}}{\pgfqpoint{1.943033in}{2.031013in}}{\pgfqpoint{1.951269in}{2.031013in}}%
\pgfpathclose%
\pgfusepath{stroke,fill}%
\end{pgfscope}%
\begin{pgfscope}%
\pgfpathrectangle{\pgfqpoint{0.100000in}{0.220728in}}{\pgfqpoint{3.696000in}{3.696000in}}%
\pgfusepath{clip}%
\pgfsetbuttcap%
\pgfsetroundjoin%
\definecolor{currentfill}{rgb}{0.121569,0.466667,0.705882}%
\pgfsetfillcolor{currentfill}%
\pgfsetfillopacity{0.300275}%
\pgfsetlinewidth{1.003750pt}%
\definecolor{currentstroke}{rgb}{0.121569,0.466667,0.705882}%
\pgfsetstrokecolor{currentstroke}%
\pgfsetstrokeopacity{0.300275}%
\pgfsetdash{}{0pt}%
\pgfpathmoveto{\pgfqpoint{1.948642in}{2.032783in}}%
\pgfpathcurveto{\pgfqpoint{1.956879in}{2.032783in}}{\pgfqpoint{1.964779in}{2.036055in}}{\pgfqpoint{1.970603in}{2.041879in}}%
\pgfpathcurveto{\pgfqpoint{1.976426in}{2.047703in}}{\pgfqpoint{1.979699in}{2.055603in}}{\pgfqpoint{1.979699in}{2.063839in}}%
\pgfpathcurveto{\pgfqpoint{1.979699in}{2.072075in}}{\pgfqpoint{1.976426in}{2.079975in}}{\pgfqpoint{1.970603in}{2.085799in}}%
\pgfpathcurveto{\pgfqpoint{1.964779in}{2.091623in}}{\pgfqpoint{1.956879in}{2.094896in}}{\pgfqpoint{1.948642in}{2.094896in}}%
\pgfpathcurveto{\pgfqpoint{1.940406in}{2.094896in}}{\pgfqpoint{1.932506in}{2.091623in}}{\pgfqpoint{1.926682in}{2.085799in}}%
\pgfpathcurveto{\pgfqpoint{1.920858in}{2.079975in}}{\pgfqpoint{1.917586in}{2.072075in}}{\pgfqpoint{1.917586in}{2.063839in}}%
\pgfpathcurveto{\pgfqpoint{1.917586in}{2.055603in}}{\pgfqpoint{1.920858in}{2.047703in}}{\pgfqpoint{1.926682in}{2.041879in}}%
\pgfpathcurveto{\pgfqpoint{1.932506in}{2.036055in}}{\pgfqpoint{1.940406in}{2.032783in}}{\pgfqpoint{1.948642in}{2.032783in}}%
\pgfpathclose%
\pgfusepath{stroke,fill}%
\end{pgfscope}%
\begin{pgfscope}%
\pgfpathrectangle{\pgfqpoint{0.100000in}{0.220728in}}{\pgfqpoint{3.696000in}{3.696000in}}%
\pgfusepath{clip}%
\pgfsetbuttcap%
\pgfsetroundjoin%
\definecolor{currentfill}{rgb}{0.121569,0.466667,0.705882}%
\pgfsetfillcolor{currentfill}%
\pgfsetfillopacity{0.300289}%
\pgfsetlinewidth{1.003750pt}%
\definecolor{currentstroke}{rgb}{0.121569,0.466667,0.705882}%
\pgfsetstrokecolor{currentstroke}%
\pgfsetstrokeopacity{0.300289}%
\pgfsetdash{}{0pt}%
\pgfpathmoveto{\pgfqpoint{1.948583in}{2.032791in}}%
\pgfpathcurveto{\pgfqpoint{1.956819in}{2.032791in}}{\pgfqpoint{1.964719in}{2.036064in}}{\pgfqpoint{1.970543in}{2.041888in}}%
\pgfpathcurveto{\pgfqpoint{1.976367in}{2.047712in}}{\pgfqpoint{1.979639in}{2.055612in}}{\pgfqpoint{1.979639in}{2.063848in}}%
\pgfpathcurveto{\pgfqpoint{1.979639in}{2.072084in}}{\pgfqpoint{1.976367in}{2.079984in}}{\pgfqpoint{1.970543in}{2.085808in}}%
\pgfpathcurveto{\pgfqpoint{1.964719in}{2.091632in}}{\pgfqpoint{1.956819in}{2.094904in}}{\pgfqpoint{1.948583in}{2.094904in}}%
\pgfpathcurveto{\pgfqpoint{1.940346in}{2.094904in}}{\pgfqpoint{1.932446in}{2.091632in}}{\pgfqpoint{1.926622in}{2.085808in}}%
\pgfpathcurveto{\pgfqpoint{1.920799in}{2.079984in}}{\pgfqpoint{1.917526in}{2.072084in}}{\pgfqpoint{1.917526in}{2.063848in}}%
\pgfpathcurveto{\pgfqpoint{1.917526in}{2.055612in}}{\pgfqpoint{1.920799in}{2.047712in}}{\pgfqpoint{1.926622in}{2.041888in}}%
\pgfpathcurveto{\pgfqpoint{1.932446in}{2.036064in}}{\pgfqpoint{1.940346in}{2.032791in}}{\pgfqpoint{1.948583in}{2.032791in}}%
\pgfpathclose%
\pgfusepath{stroke,fill}%
\end{pgfscope}%
\begin{pgfscope}%
\pgfpathrectangle{\pgfqpoint{0.100000in}{0.220728in}}{\pgfqpoint{3.696000in}{3.696000in}}%
\pgfusepath{clip}%
\pgfsetbuttcap%
\pgfsetroundjoin%
\definecolor{currentfill}{rgb}{0.121569,0.466667,0.705882}%
\pgfsetfillcolor{currentfill}%
\pgfsetfillopacity{0.300314}%
\pgfsetlinewidth{1.003750pt}%
\definecolor{currentstroke}{rgb}{0.121569,0.466667,0.705882}%
\pgfsetstrokecolor{currentstroke}%
\pgfsetstrokeopacity{0.300314}%
\pgfsetdash{}{0pt}%
\pgfpathmoveto{\pgfqpoint{1.948492in}{2.032762in}}%
\pgfpathcurveto{\pgfqpoint{1.956728in}{2.032762in}}{\pgfqpoint{1.964628in}{2.036034in}}{\pgfqpoint{1.970452in}{2.041858in}}%
\pgfpathcurveto{\pgfqpoint{1.976276in}{2.047682in}}{\pgfqpoint{1.979548in}{2.055582in}}{\pgfqpoint{1.979548in}{2.063818in}}%
\pgfpathcurveto{\pgfqpoint{1.979548in}{2.072055in}}{\pgfqpoint{1.976276in}{2.079955in}}{\pgfqpoint{1.970452in}{2.085779in}}%
\pgfpathcurveto{\pgfqpoint{1.964628in}{2.091603in}}{\pgfqpoint{1.956728in}{2.094875in}}{\pgfqpoint{1.948492in}{2.094875in}}%
\pgfpathcurveto{\pgfqpoint{1.940255in}{2.094875in}}{\pgfqpoint{1.932355in}{2.091603in}}{\pgfqpoint{1.926531in}{2.085779in}}%
\pgfpathcurveto{\pgfqpoint{1.920707in}{2.079955in}}{\pgfqpoint{1.917435in}{2.072055in}}{\pgfqpoint{1.917435in}{2.063818in}}%
\pgfpathcurveto{\pgfqpoint{1.917435in}{2.055582in}}{\pgfqpoint{1.920707in}{2.047682in}}{\pgfqpoint{1.926531in}{2.041858in}}%
\pgfpathcurveto{\pgfqpoint{1.932355in}{2.036034in}}{\pgfqpoint{1.940255in}{2.032762in}}{\pgfqpoint{1.948492in}{2.032762in}}%
\pgfpathclose%
\pgfusepath{stroke,fill}%
\end{pgfscope}%
\begin{pgfscope}%
\pgfpathrectangle{\pgfqpoint{0.100000in}{0.220728in}}{\pgfqpoint{3.696000in}{3.696000in}}%
\pgfusepath{clip}%
\pgfsetbuttcap%
\pgfsetroundjoin%
\definecolor{currentfill}{rgb}{0.121569,0.466667,0.705882}%
\pgfsetfillcolor{currentfill}%
\pgfsetfillopacity{0.300357}%
\pgfsetlinewidth{1.003750pt}%
\definecolor{currentstroke}{rgb}{0.121569,0.466667,0.705882}%
\pgfsetstrokecolor{currentstroke}%
\pgfsetstrokeopacity{0.300357}%
\pgfsetdash{}{0pt}%
\pgfpathmoveto{\pgfqpoint{1.948313in}{2.032719in}}%
\pgfpathcurveto{\pgfqpoint{1.956549in}{2.032719in}}{\pgfqpoint{1.964449in}{2.035991in}}{\pgfqpoint{1.970273in}{2.041815in}}%
\pgfpathcurveto{\pgfqpoint{1.976097in}{2.047639in}}{\pgfqpoint{1.979370in}{2.055539in}}{\pgfqpoint{1.979370in}{2.063775in}}%
\pgfpathcurveto{\pgfqpoint{1.979370in}{2.072011in}}{\pgfqpoint{1.976097in}{2.079912in}}{\pgfqpoint{1.970273in}{2.085735in}}%
\pgfpathcurveto{\pgfqpoint{1.964449in}{2.091559in}}{\pgfqpoint{1.956549in}{2.094832in}}{\pgfqpoint{1.948313in}{2.094832in}}%
\pgfpathcurveto{\pgfqpoint{1.940077in}{2.094832in}}{\pgfqpoint{1.932177in}{2.091559in}}{\pgfqpoint{1.926353in}{2.085735in}}%
\pgfpathcurveto{\pgfqpoint{1.920529in}{2.079912in}}{\pgfqpoint{1.917257in}{2.072011in}}{\pgfqpoint{1.917257in}{2.063775in}}%
\pgfpathcurveto{\pgfqpoint{1.917257in}{2.055539in}}{\pgfqpoint{1.920529in}{2.047639in}}{\pgfqpoint{1.926353in}{2.041815in}}%
\pgfpathcurveto{\pgfqpoint{1.932177in}{2.035991in}}{\pgfqpoint{1.940077in}{2.032719in}}{\pgfqpoint{1.948313in}{2.032719in}}%
\pgfpathclose%
\pgfusepath{stroke,fill}%
\end{pgfscope}%
\begin{pgfscope}%
\pgfpathrectangle{\pgfqpoint{0.100000in}{0.220728in}}{\pgfqpoint{3.696000in}{3.696000in}}%
\pgfusepath{clip}%
\pgfsetbuttcap%
\pgfsetroundjoin%
\definecolor{currentfill}{rgb}{0.121569,0.466667,0.705882}%
\pgfsetfillcolor{currentfill}%
\pgfsetfillopacity{0.300434}%
\pgfsetlinewidth{1.003750pt}%
\definecolor{currentstroke}{rgb}{0.121569,0.466667,0.705882}%
\pgfsetstrokecolor{currentstroke}%
\pgfsetstrokeopacity{0.300434}%
\pgfsetdash{}{0pt}%
\pgfpathmoveto{\pgfqpoint{1.947990in}{2.032624in}}%
\pgfpathcurveto{\pgfqpoint{1.956226in}{2.032624in}}{\pgfqpoint{1.964126in}{2.035896in}}{\pgfqpoint{1.969950in}{2.041720in}}%
\pgfpathcurveto{\pgfqpoint{1.975774in}{2.047544in}}{\pgfqpoint{1.979047in}{2.055444in}}{\pgfqpoint{1.979047in}{2.063680in}}%
\pgfpathcurveto{\pgfqpoint{1.979047in}{2.071917in}}{\pgfqpoint{1.975774in}{2.079817in}}{\pgfqpoint{1.969950in}{2.085641in}}%
\pgfpathcurveto{\pgfqpoint{1.964126in}{2.091465in}}{\pgfqpoint{1.956226in}{2.094737in}}{\pgfqpoint{1.947990in}{2.094737in}}%
\pgfpathcurveto{\pgfqpoint{1.939754in}{2.094737in}}{\pgfqpoint{1.931854in}{2.091465in}}{\pgfqpoint{1.926030in}{2.085641in}}%
\pgfpathcurveto{\pgfqpoint{1.920206in}{2.079817in}}{\pgfqpoint{1.916934in}{2.071917in}}{\pgfqpoint{1.916934in}{2.063680in}}%
\pgfpathcurveto{\pgfqpoint{1.916934in}{2.055444in}}{\pgfqpoint{1.920206in}{2.047544in}}{\pgfqpoint{1.926030in}{2.041720in}}%
\pgfpathcurveto{\pgfqpoint{1.931854in}{2.035896in}}{\pgfqpoint{1.939754in}{2.032624in}}{\pgfqpoint{1.947990in}{2.032624in}}%
\pgfpathclose%
\pgfusepath{stroke,fill}%
\end{pgfscope}%
\begin{pgfscope}%
\pgfpathrectangle{\pgfqpoint{0.100000in}{0.220728in}}{\pgfqpoint{3.696000in}{3.696000in}}%
\pgfusepath{clip}%
\pgfsetbuttcap%
\pgfsetroundjoin%
\definecolor{currentfill}{rgb}{0.121569,0.466667,0.705882}%
\pgfsetfillcolor{currentfill}%
\pgfsetfillopacity{0.300575}%
\pgfsetlinewidth{1.003750pt}%
\definecolor{currentstroke}{rgb}{0.121569,0.466667,0.705882}%
\pgfsetstrokecolor{currentstroke}%
\pgfsetstrokeopacity{0.300575}%
\pgfsetdash{}{0pt}%
\pgfpathmoveto{\pgfqpoint{1.947408in}{2.032445in}}%
\pgfpathcurveto{\pgfqpoint{1.955644in}{2.032445in}}{\pgfqpoint{1.963544in}{2.035718in}}{\pgfqpoint{1.969368in}{2.041542in}}%
\pgfpathcurveto{\pgfqpoint{1.975192in}{2.047366in}}{\pgfqpoint{1.978464in}{2.055266in}}{\pgfqpoint{1.978464in}{2.063502in}}%
\pgfpathcurveto{\pgfqpoint{1.978464in}{2.071738in}}{\pgfqpoint{1.975192in}{2.079638in}}{\pgfqpoint{1.969368in}{2.085462in}}%
\pgfpathcurveto{\pgfqpoint{1.963544in}{2.091286in}}{\pgfqpoint{1.955644in}{2.094558in}}{\pgfqpoint{1.947408in}{2.094558in}}%
\pgfpathcurveto{\pgfqpoint{1.939171in}{2.094558in}}{\pgfqpoint{1.931271in}{2.091286in}}{\pgfqpoint{1.925447in}{2.085462in}}%
\pgfpathcurveto{\pgfqpoint{1.919623in}{2.079638in}}{\pgfqpoint{1.916351in}{2.071738in}}{\pgfqpoint{1.916351in}{2.063502in}}%
\pgfpathcurveto{\pgfqpoint{1.916351in}{2.055266in}}{\pgfqpoint{1.919623in}{2.047366in}}{\pgfqpoint{1.925447in}{2.041542in}}%
\pgfpathcurveto{\pgfqpoint{1.931271in}{2.035718in}}{\pgfqpoint{1.939171in}{2.032445in}}{\pgfqpoint{1.947408in}{2.032445in}}%
\pgfpathclose%
\pgfusepath{stroke,fill}%
\end{pgfscope}%
\begin{pgfscope}%
\pgfpathrectangle{\pgfqpoint{0.100000in}{0.220728in}}{\pgfqpoint{3.696000in}{3.696000in}}%
\pgfusepath{clip}%
\pgfsetbuttcap%
\pgfsetroundjoin%
\definecolor{currentfill}{rgb}{0.121569,0.466667,0.705882}%
\pgfsetfillcolor{currentfill}%
\pgfsetfillopacity{0.300691}%
\pgfsetlinewidth{1.003750pt}%
\definecolor{currentstroke}{rgb}{0.121569,0.466667,0.705882}%
\pgfsetstrokecolor{currentstroke}%
\pgfsetstrokeopacity{0.300691}%
\pgfsetdash{}{0pt}%
\pgfpathmoveto{\pgfqpoint{1.951334in}{2.031332in}}%
\pgfpathcurveto{\pgfqpoint{1.959570in}{2.031332in}}{\pgfqpoint{1.967470in}{2.034604in}}{\pgfqpoint{1.973294in}{2.040428in}}%
\pgfpathcurveto{\pgfqpoint{1.979118in}{2.046252in}}{\pgfqpoint{1.982390in}{2.054152in}}{\pgfqpoint{1.982390in}{2.062388in}}%
\pgfpathcurveto{\pgfqpoint{1.982390in}{2.070625in}}{\pgfqpoint{1.979118in}{2.078525in}}{\pgfqpoint{1.973294in}{2.084349in}}%
\pgfpathcurveto{\pgfqpoint{1.967470in}{2.090173in}}{\pgfqpoint{1.959570in}{2.093445in}}{\pgfqpoint{1.951334in}{2.093445in}}%
\pgfpathcurveto{\pgfqpoint{1.943097in}{2.093445in}}{\pgfqpoint{1.935197in}{2.090173in}}{\pgfqpoint{1.929373in}{2.084349in}}%
\pgfpathcurveto{\pgfqpoint{1.923549in}{2.078525in}}{\pgfqpoint{1.920277in}{2.070625in}}{\pgfqpoint{1.920277in}{2.062388in}}%
\pgfpathcurveto{\pgfqpoint{1.920277in}{2.054152in}}{\pgfqpoint{1.923549in}{2.046252in}}{\pgfqpoint{1.929373in}{2.040428in}}%
\pgfpathcurveto{\pgfqpoint{1.935197in}{2.034604in}}{\pgfqpoint{1.943097in}{2.031332in}}{\pgfqpoint{1.951334in}{2.031332in}}%
\pgfpathclose%
\pgfusepath{stroke,fill}%
\end{pgfscope}%
\begin{pgfscope}%
\pgfpathrectangle{\pgfqpoint{0.100000in}{0.220728in}}{\pgfqpoint{3.696000in}{3.696000in}}%
\pgfusepath{clip}%
\pgfsetbuttcap%
\pgfsetroundjoin%
\definecolor{currentfill}{rgb}{0.121569,0.466667,0.705882}%
\pgfsetfillcolor{currentfill}%
\pgfsetfillopacity{0.300826}%
\pgfsetlinewidth{1.003750pt}%
\definecolor{currentstroke}{rgb}{0.121569,0.466667,0.705882}%
\pgfsetstrokecolor{currentstroke}%
\pgfsetstrokeopacity{0.300826}%
\pgfsetdash{}{0pt}%
\pgfpathmoveto{\pgfqpoint{1.946313in}{2.032173in}}%
\pgfpathcurveto{\pgfqpoint{1.954550in}{2.032173in}}{\pgfqpoint{1.962450in}{2.035445in}}{\pgfqpoint{1.968274in}{2.041269in}}%
\pgfpathcurveto{\pgfqpoint{1.974098in}{2.047093in}}{\pgfqpoint{1.977370in}{2.054993in}}{\pgfqpoint{1.977370in}{2.063230in}}%
\pgfpathcurveto{\pgfqpoint{1.977370in}{2.071466in}}{\pgfqpoint{1.974098in}{2.079366in}}{\pgfqpoint{1.968274in}{2.085190in}}%
\pgfpathcurveto{\pgfqpoint{1.962450in}{2.091014in}}{\pgfqpoint{1.954550in}{2.094286in}}{\pgfqpoint{1.946313in}{2.094286in}}%
\pgfpathcurveto{\pgfqpoint{1.938077in}{2.094286in}}{\pgfqpoint{1.930177in}{2.091014in}}{\pgfqpoint{1.924353in}{2.085190in}}%
\pgfpathcurveto{\pgfqpoint{1.918529in}{2.079366in}}{\pgfqpoint{1.915257in}{2.071466in}}{\pgfqpoint{1.915257in}{2.063230in}}%
\pgfpathcurveto{\pgfqpoint{1.915257in}{2.054993in}}{\pgfqpoint{1.918529in}{2.047093in}}{\pgfqpoint{1.924353in}{2.041269in}}%
\pgfpathcurveto{\pgfqpoint{1.930177in}{2.035445in}}{\pgfqpoint{1.938077in}{2.032173in}}{\pgfqpoint{1.946313in}{2.032173in}}%
\pgfpathclose%
\pgfusepath{stroke,fill}%
\end{pgfscope}%
\begin{pgfscope}%
\pgfpathrectangle{\pgfqpoint{0.100000in}{0.220728in}}{\pgfqpoint{3.696000in}{3.696000in}}%
\pgfusepath{clip}%
\pgfsetbuttcap%
\pgfsetroundjoin%
\definecolor{currentfill}{rgb}{0.121569,0.466667,0.705882}%
\pgfsetfillcolor{currentfill}%
\pgfsetfillopacity{0.300919}%
\pgfsetlinewidth{1.003750pt}%
\definecolor{currentstroke}{rgb}{0.121569,0.466667,0.705882}%
\pgfsetstrokecolor{currentstroke}%
\pgfsetstrokeopacity{0.300919}%
\pgfsetdash{}{0pt}%
\pgfpathmoveto{\pgfqpoint{1.951334in}{2.031329in}}%
\pgfpathcurveto{\pgfqpoint{1.959570in}{2.031329in}}{\pgfqpoint{1.967470in}{2.034601in}}{\pgfqpoint{1.973294in}{2.040425in}}%
\pgfpathcurveto{\pgfqpoint{1.979118in}{2.046249in}}{\pgfqpoint{1.982390in}{2.054149in}}{\pgfqpoint{1.982390in}{2.062385in}}%
\pgfpathcurveto{\pgfqpoint{1.982390in}{2.070622in}}{\pgfqpoint{1.979118in}{2.078522in}}{\pgfqpoint{1.973294in}{2.084346in}}%
\pgfpathcurveto{\pgfqpoint{1.967470in}{2.090169in}}{\pgfqpoint{1.959570in}{2.093442in}}{\pgfqpoint{1.951334in}{2.093442in}}%
\pgfpathcurveto{\pgfqpoint{1.943097in}{2.093442in}}{\pgfqpoint{1.935197in}{2.090169in}}{\pgfqpoint{1.929373in}{2.084346in}}%
\pgfpathcurveto{\pgfqpoint{1.923549in}{2.078522in}}{\pgfqpoint{1.920277in}{2.070622in}}{\pgfqpoint{1.920277in}{2.062385in}}%
\pgfpathcurveto{\pgfqpoint{1.920277in}{2.054149in}}{\pgfqpoint{1.923549in}{2.046249in}}{\pgfqpoint{1.929373in}{2.040425in}}%
\pgfpathcurveto{\pgfqpoint{1.935197in}{2.034601in}}{\pgfqpoint{1.943097in}{2.031329in}}{\pgfqpoint{1.951334in}{2.031329in}}%
\pgfpathclose%
\pgfusepath{stroke,fill}%
\end{pgfscope}%
\begin{pgfscope}%
\pgfpathrectangle{\pgfqpoint{0.100000in}{0.220728in}}{\pgfqpoint{3.696000in}{3.696000in}}%
\pgfusepath{clip}%
\pgfsetbuttcap%
\pgfsetroundjoin%
\definecolor{currentfill}{rgb}{0.121569,0.466667,0.705882}%
\pgfsetfillcolor{currentfill}%
\pgfsetfillopacity{0.301024}%
\pgfsetlinewidth{1.003750pt}%
\definecolor{currentstroke}{rgb}{0.121569,0.466667,0.705882}%
\pgfsetstrokecolor{currentstroke}%
\pgfsetstrokeopacity{0.301024}%
\pgfsetdash{}{0pt}%
\pgfpathmoveto{\pgfqpoint{1.951315in}{2.031183in}}%
\pgfpathcurveto{\pgfqpoint{1.959551in}{2.031183in}}{\pgfqpoint{1.967451in}{2.034455in}}{\pgfqpoint{1.973275in}{2.040279in}}%
\pgfpathcurveto{\pgfqpoint{1.979099in}{2.046103in}}{\pgfqpoint{1.982371in}{2.054003in}}{\pgfqpoint{1.982371in}{2.062239in}}%
\pgfpathcurveto{\pgfqpoint{1.982371in}{2.070476in}}{\pgfqpoint{1.979099in}{2.078376in}}{\pgfqpoint{1.973275in}{2.084200in}}%
\pgfpathcurveto{\pgfqpoint{1.967451in}{2.090023in}}{\pgfqpoint{1.959551in}{2.093296in}}{\pgfqpoint{1.951315in}{2.093296in}}%
\pgfpathcurveto{\pgfqpoint{1.943078in}{2.093296in}}{\pgfqpoint{1.935178in}{2.090023in}}{\pgfqpoint{1.929354in}{2.084200in}}%
\pgfpathcurveto{\pgfqpoint{1.923530in}{2.078376in}}{\pgfqpoint{1.920258in}{2.070476in}}{\pgfqpoint{1.920258in}{2.062239in}}%
\pgfpathcurveto{\pgfqpoint{1.920258in}{2.054003in}}{\pgfqpoint{1.923530in}{2.046103in}}{\pgfqpoint{1.929354in}{2.040279in}}%
\pgfpathcurveto{\pgfqpoint{1.935178in}{2.034455in}}{\pgfqpoint{1.943078in}{2.031183in}}{\pgfqpoint{1.951315in}{2.031183in}}%
\pgfpathclose%
\pgfusepath{stroke,fill}%
\end{pgfscope}%
\begin{pgfscope}%
\pgfpathrectangle{\pgfqpoint{0.100000in}{0.220728in}}{\pgfqpoint{3.696000in}{3.696000in}}%
\pgfusepath{clip}%
\pgfsetbuttcap%
\pgfsetroundjoin%
\definecolor{currentfill}{rgb}{0.121569,0.466667,0.705882}%
\pgfsetfillcolor{currentfill}%
\pgfsetfillopacity{0.301334}%
\pgfsetlinewidth{1.003750pt}%
\definecolor{currentstroke}{rgb}{0.121569,0.466667,0.705882}%
\pgfsetstrokecolor{currentstroke}%
\pgfsetstrokeopacity{0.301334}%
\pgfsetdash{}{0pt}%
\pgfpathmoveto{\pgfqpoint{1.944486in}{2.031679in}}%
\pgfpathcurveto{\pgfqpoint{1.952722in}{2.031679in}}{\pgfqpoint{1.960622in}{2.034951in}}{\pgfqpoint{1.966446in}{2.040775in}}%
\pgfpathcurveto{\pgfqpoint{1.972270in}{2.046599in}}{\pgfqpoint{1.975542in}{2.054499in}}{\pgfqpoint{1.975542in}{2.062736in}}%
\pgfpathcurveto{\pgfqpoint{1.975542in}{2.070972in}}{\pgfqpoint{1.972270in}{2.078872in}}{\pgfqpoint{1.966446in}{2.084696in}}%
\pgfpathcurveto{\pgfqpoint{1.960622in}{2.090520in}}{\pgfqpoint{1.952722in}{2.093792in}}{\pgfqpoint{1.944486in}{2.093792in}}%
\pgfpathcurveto{\pgfqpoint{1.936249in}{2.093792in}}{\pgfqpoint{1.928349in}{2.090520in}}{\pgfqpoint{1.922525in}{2.084696in}}%
\pgfpathcurveto{\pgfqpoint{1.916701in}{2.078872in}}{\pgfqpoint{1.913429in}{2.070972in}}{\pgfqpoint{1.913429in}{2.062736in}}%
\pgfpathcurveto{\pgfqpoint{1.913429in}{2.054499in}}{\pgfqpoint{1.916701in}{2.046599in}}{\pgfqpoint{1.922525in}{2.040775in}}%
\pgfpathcurveto{\pgfqpoint{1.928349in}{2.034951in}}{\pgfqpoint{1.936249in}{2.031679in}}{\pgfqpoint{1.944486in}{2.031679in}}%
\pgfpathclose%
\pgfusepath{stroke,fill}%
\end{pgfscope}%
\begin{pgfscope}%
\pgfpathrectangle{\pgfqpoint{0.100000in}{0.220728in}}{\pgfqpoint{3.696000in}{3.696000in}}%
\pgfusepath{clip}%
\pgfsetbuttcap%
\pgfsetroundjoin%
\definecolor{currentfill}{rgb}{0.121569,0.466667,0.705882}%
\pgfsetfillcolor{currentfill}%
\pgfsetfillopacity{0.301363}%
\pgfsetlinewidth{1.003750pt}%
\definecolor{currentstroke}{rgb}{0.121569,0.466667,0.705882}%
\pgfsetstrokecolor{currentstroke}%
\pgfsetstrokeopacity{0.301363}%
\pgfsetdash{}{0pt}%
\pgfpathmoveto{\pgfqpoint{1.951257in}{2.031108in}}%
\pgfpathcurveto{\pgfqpoint{1.959493in}{2.031108in}}{\pgfqpoint{1.967393in}{2.034380in}}{\pgfqpoint{1.973217in}{2.040204in}}%
\pgfpathcurveto{\pgfqpoint{1.979041in}{2.046028in}}{\pgfqpoint{1.982313in}{2.053928in}}{\pgfqpoint{1.982313in}{2.062164in}}%
\pgfpathcurveto{\pgfqpoint{1.982313in}{2.070401in}}{\pgfqpoint{1.979041in}{2.078301in}}{\pgfqpoint{1.973217in}{2.084125in}}%
\pgfpathcurveto{\pgfqpoint{1.967393in}{2.089949in}}{\pgfqpoint{1.959493in}{2.093221in}}{\pgfqpoint{1.951257in}{2.093221in}}%
\pgfpathcurveto{\pgfqpoint{1.943020in}{2.093221in}}{\pgfqpoint{1.935120in}{2.089949in}}{\pgfqpoint{1.929296in}{2.084125in}}%
\pgfpathcurveto{\pgfqpoint{1.923472in}{2.078301in}}{\pgfqpoint{1.920200in}{2.070401in}}{\pgfqpoint{1.920200in}{2.062164in}}%
\pgfpathcurveto{\pgfqpoint{1.920200in}{2.053928in}}{\pgfqpoint{1.923472in}{2.046028in}}{\pgfqpoint{1.929296in}{2.040204in}}%
\pgfpathcurveto{\pgfqpoint{1.935120in}{2.034380in}}{\pgfqpoint{1.943020in}{2.031108in}}{\pgfqpoint{1.951257in}{2.031108in}}%
\pgfpathclose%
\pgfusepath{stroke,fill}%
\end{pgfscope}%
\begin{pgfscope}%
\pgfpathrectangle{\pgfqpoint{0.100000in}{0.220728in}}{\pgfqpoint{3.696000in}{3.696000in}}%
\pgfusepath{clip}%
\pgfsetbuttcap%
\pgfsetroundjoin%
\definecolor{currentfill}{rgb}{0.121569,0.466667,0.705882}%
\pgfsetfillcolor{currentfill}%
\pgfsetfillopacity{0.301539}%
\pgfsetlinewidth{1.003750pt}%
\definecolor{currentstroke}{rgb}{0.121569,0.466667,0.705882}%
\pgfsetstrokecolor{currentstroke}%
\pgfsetstrokeopacity{0.301539}%
\pgfsetdash{}{0pt}%
\pgfpathmoveto{\pgfqpoint{1.951178in}{2.030996in}}%
\pgfpathcurveto{\pgfqpoint{1.959415in}{2.030996in}}{\pgfqpoint{1.967315in}{2.034268in}}{\pgfqpoint{1.973139in}{2.040092in}}%
\pgfpathcurveto{\pgfqpoint{1.978963in}{2.045916in}}{\pgfqpoint{1.982235in}{2.053816in}}{\pgfqpoint{1.982235in}{2.062052in}}%
\pgfpathcurveto{\pgfqpoint{1.982235in}{2.070288in}}{\pgfqpoint{1.978963in}{2.078188in}}{\pgfqpoint{1.973139in}{2.084012in}}%
\pgfpathcurveto{\pgfqpoint{1.967315in}{2.089836in}}{\pgfqpoint{1.959415in}{2.093109in}}{\pgfqpoint{1.951178in}{2.093109in}}%
\pgfpathcurveto{\pgfqpoint{1.942942in}{2.093109in}}{\pgfqpoint{1.935042in}{2.089836in}}{\pgfqpoint{1.929218in}{2.084012in}}%
\pgfpathcurveto{\pgfqpoint{1.923394in}{2.078188in}}{\pgfqpoint{1.920122in}{2.070288in}}{\pgfqpoint{1.920122in}{2.062052in}}%
\pgfpathcurveto{\pgfqpoint{1.920122in}{2.053816in}}{\pgfqpoint{1.923394in}{2.045916in}}{\pgfqpoint{1.929218in}{2.040092in}}%
\pgfpathcurveto{\pgfqpoint{1.935042in}{2.034268in}}{\pgfqpoint{1.942942in}{2.030996in}}{\pgfqpoint{1.951178in}{2.030996in}}%
\pgfpathclose%
\pgfusepath{stroke,fill}%
\end{pgfscope}%
\begin{pgfscope}%
\pgfpathrectangle{\pgfqpoint{0.100000in}{0.220728in}}{\pgfqpoint{3.696000in}{3.696000in}}%
\pgfusepath{clip}%
\pgfsetbuttcap%
\pgfsetroundjoin%
\definecolor{currentfill}{rgb}{0.121569,0.466667,0.705882}%
\pgfsetfillcolor{currentfill}%
\pgfsetfillopacity{0.301726}%
\pgfsetlinewidth{1.003750pt}%
\definecolor{currentstroke}{rgb}{0.121569,0.466667,0.705882}%
\pgfsetstrokecolor{currentstroke}%
\pgfsetstrokeopacity{0.301726}%
\pgfsetdash{}{0pt}%
\pgfpathmoveto{\pgfqpoint{1.942972in}{2.031335in}}%
\pgfpathcurveto{\pgfqpoint{1.951208in}{2.031335in}}{\pgfqpoint{1.959108in}{2.034607in}}{\pgfqpoint{1.964932in}{2.040431in}}%
\pgfpathcurveto{\pgfqpoint{1.970756in}{2.046255in}}{\pgfqpoint{1.974028in}{2.054155in}}{\pgfqpoint{1.974028in}{2.062391in}}%
\pgfpathcurveto{\pgfqpoint{1.974028in}{2.070627in}}{\pgfqpoint{1.970756in}{2.078527in}}{\pgfqpoint{1.964932in}{2.084351in}}%
\pgfpathcurveto{\pgfqpoint{1.959108in}{2.090175in}}{\pgfqpoint{1.951208in}{2.093448in}}{\pgfqpoint{1.942972in}{2.093448in}}%
\pgfpathcurveto{\pgfqpoint{1.934736in}{2.093448in}}{\pgfqpoint{1.926836in}{2.090175in}}{\pgfqpoint{1.921012in}{2.084351in}}%
\pgfpathcurveto{\pgfqpoint{1.915188in}{2.078527in}}{\pgfqpoint{1.911915in}{2.070627in}}{\pgfqpoint{1.911915in}{2.062391in}}%
\pgfpathcurveto{\pgfqpoint{1.911915in}{2.054155in}}{\pgfqpoint{1.915188in}{2.046255in}}{\pgfqpoint{1.921012in}{2.040431in}}%
\pgfpathcurveto{\pgfqpoint{1.926836in}{2.034607in}}{\pgfqpoint{1.934736in}{2.031335in}}{\pgfqpoint{1.942972in}{2.031335in}}%
\pgfpathclose%
\pgfusepath{stroke,fill}%
\end{pgfscope}%
\begin{pgfscope}%
\pgfpathrectangle{\pgfqpoint{0.100000in}{0.220728in}}{\pgfqpoint{3.696000in}{3.696000in}}%
\pgfusepath{clip}%
\pgfsetbuttcap%
\pgfsetroundjoin%
\definecolor{currentfill}{rgb}{0.121569,0.466667,0.705882}%
\pgfsetfillcolor{currentfill}%
\pgfsetfillopacity{0.301953}%
\pgfsetlinewidth{1.003750pt}%
\definecolor{currentstroke}{rgb}{0.121569,0.466667,0.705882}%
\pgfsetstrokecolor{currentstroke}%
\pgfsetstrokeopacity{0.301953}%
\pgfsetdash{}{0pt}%
\pgfpathmoveto{\pgfqpoint{1.951026in}{2.031534in}}%
\pgfpathcurveto{\pgfqpoint{1.959262in}{2.031534in}}{\pgfqpoint{1.967162in}{2.034807in}}{\pgfqpoint{1.972986in}{2.040631in}}%
\pgfpathcurveto{\pgfqpoint{1.978810in}{2.046455in}}{\pgfqpoint{1.982082in}{2.054355in}}{\pgfqpoint{1.982082in}{2.062591in}}%
\pgfpathcurveto{\pgfqpoint{1.982082in}{2.070827in}}{\pgfqpoint{1.978810in}{2.078727in}}{\pgfqpoint{1.972986in}{2.084551in}}%
\pgfpathcurveto{\pgfqpoint{1.967162in}{2.090375in}}{\pgfqpoint{1.959262in}{2.093647in}}{\pgfqpoint{1.951026in}{2.093647in}}%
\pgfpathcurveto{\pgfqpoint{1.942790in}{2.093647in}}{\pgfqpoint{1.934890in}{2.090375in}}{\pgfqpoint{1.929066in}{2.084551in}}%
\pgfpathcurveto{\pgfqpoint{1.923242in}{2.078727in}}{\pgfqpoint{1.919969in}{2.070827in}}{\pgfqpoint{1.919969in}{2.062591in}}%
\pgfpathcurveto{\pgfqpoint{1.919969in}{2.054355in}}{\pgfqpoint{1.923242in}{2.046455in}}{\pgfqpoint{1.929066in}{2.040631in}}%
\pgfpathcurveto{\pgfqpoint{1.934890in}{2.034807in}}{\pgfqpoint{1.942790in}{2.031534in}}{\pgfqpoint{1.951026in}{2.031534in}}%
\pgfpathclose%
\pgfusepath{stroke,fill}%
\end{pgfscope}%
\begin{pgfscope}%
\pgfpathrectangle{\pgfqpoint{0.100000in}{0.220728in}}{\pgfqpoint{3.696000in}{3.696000in}}%
\pgfusepath{clip}%
\pgfsetbuttcap%
\pgfsetroundjoin%
\definecolor{currentfill}{rgb}{0.121569,0.466667,0.705882}%
\pgfsetfillcolor{currentfill}%
\pgfsetfillopacity{0.301984}%
\pgfsetlinewidth{1.003750pt}%
\definecolor{currentstroke}{rgb}{0.121569,0.466667,0.705882}%
\pgfsetstrokecolor{currentstroke}%
\pgfsetstrokeopacity{0.301984}%
\pgfsetdash{}{0pt}%
\pgfpathmoveto{\pgfqpoint{1.941953in}{2.031038in}}%
\pgfpathcurveto{\pgfqpoint{1.950189in}{2.031038in}}{\pgfqpoint{1.958089in}{2.034310in}}{\pgfqpoint{1.963913in}{2.040134in}}%
\pgfpathcurveto{\pgfqpoint{1.969737in}{2.045958in}}{\pgfqpoint{1.973009in}{2.053858in}}{\pgfqpoint{1.973009in}{2.062094in}}%
\pgfpathcurveto{\pgfqpoint{1.973009in}{2.070330in}}{\pgfqpoint{1.969737in}{2.078230in}}{\pgfqpoint{1.963913in}{2.084054in}}%
\pgfpathcurveto{\pgfqpoint{1.958089in}{2.089878in}}{\pgfqpoint{1.950189in}{2.093151in}}{\pgfqpoint{1.941953in}{2.093151in}}%
\pgfpathcurveto{\pgfqpoint{1.933716in}{2.093151in}}{\pgfqpoint{1.925816in}{2.089878in}}{\pgfqpoint{1.919992in}{2.084054in}}%
\pgfpathcurveto{\pgfqpoint{1.914169in}{2.078230in}}{\pgfqpoint{1.910896in}{2.070330in}}{\pgfqpoint{1.910896in}{2.062094in}}%
\pgfpathcurveto{\pgfqpoint{1.910896in}{2.053858in}}{\pgfqpoint{1.914169in}{2.045958in}}{\pgfqpoint{1.919992in}{2.040134in}}%
\pgfpathcurveto{\pgfqpoint{1.925816in}{2.034310in}}{\pgfqpoint{1.933716in}{2.031038in}}{\pgfqpoint{1.941953in}{2.031038in}}%
\pgfpathclose%
\pgfusepath{stroke,fill}%
\end{pgfscope}%
\begin{pgfscope}%
\pgfpathrectangle{\pgfqpoint{0.100000in}{0.220728in}}{\pgfqpoint{3.696000in}{3.696000in}}%
\pgfusepath{clip}%
\pgfsetbuttcap%
\pgfsetroundjoin%
\definecolor{currentfill}{rgb}{0.121569,0.466667,0.705882}%
\pgfsetfillcolor{currentfill}%
\pgfsetfillopacity{0.302399}%
\pgfsetlinewidth{1.003750pt}%
\definecolor{currentstroke}{rgb}{0.121569,0.466667,0.705882}%
\pgfsetstrokecolor{currentstroke}%
\pgfsetstrokeopacity{0.302399}%
\pgfsetdash{}{0pt}%
\pgfpathmoveto{\pgfqpoint{1.940183in}{2.029941in}}%
\pgfpathcurveto{\pgfqpoint{1.948419in}{2.029941in}}{\pgfqpoint{1.956319in}{2.033213in}}{\pgfqpoint{1.962143in}{2.039037in}}%
\pgfpathcurveto{\pgfqpoint{1.967967in}{2.044861in}}{\pgfqpoint{1.971239in}{2.052761in}}{\pgfqpoint{1.971239in}{2.060997in}}%
\pgfpathcurveto{\pgfqpoint{1.971239in}{2.069234in}}{\pgfqpoint{1.967967in}{2.077134in}}{\pgfqpoint{1.962143in}{2.082958in}}%
\pgfpathcurveto{\pgfqpoint{1.956319in}{2.088782in}}{\pgfqpoint{1.948419in}{2.092054in}}{\pgfqpoint{1.940183in}{2.092054in}}%
\pgfpathcurveto{\pgfqpoint{1.931946in}{2.092054in}}{\pgfqpoint{1.924046in}{2.088782in}}{\pgfqpoint{1.918222in}{2.082958in}}%
\pgfpathcurveto{\pgfqpoint{1.912398in}{2.077134in}}{\pgfqpoint{1.909126in}{2.069234in}}{\pgfqpoint{1.909126in}{2.060997in}}%
\pgfpathcurveto{\pgfqpoint{1.909126in}{2.052761in}}{\pgfqpoint{1.912398in}{2.044861in}}{\pgfqpoint{1.918222in}{2.039037in}}%
\pgfpathcurveto{\pgfqpoint{1.924046in}{2.033213in}}{\pgfqpoint{1.931946in}{2.029941in}}{\pgfqpoint{1.940183in}{2.029941in}}%
\pgfpathclose%
\pgfusepath{stroke,fill}%
\end{pgfscope}%
\begin{pgfscope}%
\pgfpathrectangle{\pgfqpoint{0.100000in}{0.220728in}}{\pgfqpoint{3.696000in}{3.696000in}}%
\pgfusepath{clip}%
\pgfsetbuttcap%
\pgfsetroundjoin%
\definecolor{currentfill}{rgb}{0.121569,0.466667,0.705882}%
\pgfsetfillcolor{currentfill}%
\pgfsetfillopacity{0.302518}%
\pgfsetlinewidth{1.003750pt}%
\definecolor{currentstroke}{rgb}{0.121569,0.466667,0.705882}%
\pgfsetstrokecolor{currentstroke}%
\pgfsetstrokeopacity{0.302518}%
\pgfsetdash{}{0pt}%
\pgfpathmoveto{\pgfqpoint{1.951405in}{2.028941in}}%
\pgfpathcurveto{\pgfqpoint{1.959641in}{2.028941in}}{\pgfqpoint{1.967541in}{2.032213in}}{\pgfqpoint{1.973365in}{2.038037in}}%
\pgfpathcurveto{\pgfqpoint{1.979189in}{2.043861in}}{\pgfqpoint{1.982461in}{2.051761in}}{\pgfqpoint{1.982461in}{2.059997in}}%
\pgfpathcurveto{\pgfqpoint{1.982461in}{2.068233in}}{\pgfqpoint{1.979189in}{2.076134in}}{\pgfqpoint{1.973365in}{2.081957in}}%
\pgfpathcurveto{\pgfqpoint{1.967541in}{2.087781in}}{\pgfqpoint{1.959641in}{2.091054in}}{\pgfqpoint{1.951405in}{2.091054in}}%
\pgfpathcurveto{\pgfqpoint{1.943168in}{2.091054in}}{\pgfqpoint{1.935268in}{2.087781in}}{\pgfqpoint{1.929444in}{2.081957in}}%
\pgfpathcurveto{\pgfqpoint{1.923620in}{2.076134in}}{\pgfqpoint{1.920348in}{2.068233in}}{\pgfqpoint{1.920348in}{2.059997in}}%
\pgfpathcurveto{\pgfqpoint{1.920348in}{2.051761in}}{\pgfqpoint{1.923620in}{2.043861in}}{\pgfqpoint{1.929444in}{2.038037in}}%
\pgfpathcurveto{\pgfqpoint{1.935268in}{2.032213in}}{\pgfqpoint{1.943168in}{2.028941in}}{\pgfqpoint{1.951405in}{2.028941in}}%
\pgfpathclose%
\pgfusepath{stroke,fill}%
\end{pgfscope}%
\begin{pgfscope}%
\pgfpathrectangle{\pgfqpoint{0.100000in}{0.220728in}}{\pgfqpoint{3.696000in}{3.696000in}}%
\pgfusepath{clip}%
\pgfsetbuttcap%
\pgfsetroundjoin%
\definecolor{currentfill}{rgb}{0.121569,0.466667,0.705882}%
\pgfsetfillcolor{currentfill}%
\pgfsetfillopacity{0.303493}%
\pgfsetlinewidth{1.003750pt}%
\definecolor{currentstroke}{rgb}{0.121569,0.466667,0.705882}%
\pgfsetstrokecolor{currentstroke}%
\pgfsetstrokeopacity{0.303493}%
\pgfsetdash{}{0pt}%
\pgfpathmoveto{\pgfqpoint{1.952205in}{2.026418in}}%
\pgfpathcurveto{\pgfqpoint{1.960442in}{2.026418in}}{\pgfqpoint{1.968342in}{2.029690in}}{\pgfqpoint{1.974166in}{2.035514in}}%
\pgfpathcurveto{\pgfqpoint{1.979990in}{2.041338in}}{\pgfqpoint{1.983262in}{2.049238in}}{\pgfqpoint{1.983262in}{2.057475in}}%
\pgfpathcurveto{\pgfqpoint{1.983262in}{2.065711in}}{\pgfqpoint{1.979990in}{2.073611in}}{\pgfqpoint{1.974166in}{2.079435in}}%
\pgfpathcurveto{\pgfqpoint{1.968342in}{2.085259in}}{\pgfqpoint{1.960442in}{2.088531in}}{\pgfqpoint{1.952205in}{2.088531in}}%
\pgfpathcurveto{\pgfqpoint{1.943969in}{2.088531in}}{\pgfqpoint{1.936069in}{2.085259in}}{\pgfqpoint{1.930245in}{2.079435in}}%
\pgfpathcurveto{\pgfqpoint{1.924421in}{2.073611in}}{\pgfqpoint{1.921149in}{2.065711in}}{\pgfqpoint{1.921149in}{2.057475in}}%
\pgfpathcurveto{\pgfqpoint{1.921149in}{2.049238in}}{\pgfqpoint{1.924421in}{2.041338in}}{\pgfqpoint{1.930245in}{2.035514in}}%
\pgfpathcurveto{\pgfqpoint{1.936069in}{2.029690in}}{\pgfqpoint{1.943969in}{2.026418in}}{\pgfqpoint{1.952205in}{2.026418in}}%
\pgfpathclose%
\pgfusepath{stroke,fill}%
\end{pgfscope}%
\begin{pgfscope}%
\pgfpathrectangle{\pgfqpoint{0.100000in}{0.220728in}}{\pgfqpoint{3.696000in}{3.696000in}}%
\pgfusepath{clip}%
\pgfsetbuttcap%
\pgfsetroundjoin%
\definecolor{currentfill}{rgb}{0.121569,0.466667,0.705882}%
\pgfsetfillcolor{currentfill}%
\pgfsetfillopacity{0.303502}%
\pgfsetlinewidth{1.003750pt}%
\definecolor{currentstroke}{rgb}{0.121569,0.466667,0.705882}%
\pgfsetstrokecolor{currentstroke}%
\pgfsetstrokeopacity{0.303502}%
\pgfsetdash{}{0pt}%
\pgfpathmoveto{\pgfqpoint{1.937012in}{2.030291in}}%
\pgfpathcurveto{\pgfqpoint{1.945248in}{2.030291in}}{\pgfqpoint{1.953149in}{2.033563in}}{\pgfqpoint{1.958972in}{2.039387in}}%
\pgfpathcurveto{\pgfqpoint{1.964796in}{2.045211in}}{\pgfqpoint{1.968069in}{2.053111in}}{\pgfqpoint{1.968069in}{2.061347in}}%
\pgfpathcurveto{\pgfqpoint{1.968069in}{2.069583in}}{\pgfqpoint{1.964796in}{2.077484in}}{\pgfqpoint{1.958972in}{2.083307in}}%
\pgfpathcurveto{\pgfqpoint{1.953149in}{2.089131in}}{\pgfqpoint{1.945248in}{2.092404in}}{\pgfqpoint{1.937012in}{2.092404in}}%
\pgfpathcurveto{\pgfqpoint{1.928776in}{2.092404in}}{\pgfqpoint{1.920876in}{2.089131in}}{\pgfqpoint{1.915052in}{2.083307in}}%
\pgfpathcurveto{\pgfqpoint{1.909228in}{2.077484in}}{\pgfqpoint{1.905956in}{2.069583in}}{\pgfqpoint{1.905956in}{2.061347in}}%
\pgfpathcurveto{\pgfqpoint{1.905956in}{2.053111in}}{\pgfqpoint{1.909228in}{2.045211in}}{\pgfqpoint{1.915052in}{2.039387in}}%
\pgfpathcurveto{\pgfqpoint{1.920876in}{2.033563in}}{\pgfqpoint{1.928776in}{2.030291in}}{\pgfqpoint{1.937012in}{2.030291in}}%
\pgfpathclose%
\pgfusepath{stroke,fill}%
\end{pgfscope}%
\begin{pgfscope}%
\pgfpathrectangle{\pgfqpoint{0.100000in}{0.220728in}}{\pgfqpoint{3.696000in}{3.696000in}}%
\pgfusepath{clip}%
\pgfsetbuttcap%
\pgfsetroundjoin%
\definecolor{currentfill}{rgb}{0.121569,0.466667,0.705882}%
\pgfsetfillcolor{currentfill}%
\pgfsetfillopacity{0.305289}%
\pgfsetlinewidth{1.003750pt}%
\definecolor{currentstroke}{rgb}{0.121569,0.466667,0.705882}%
\pgfsetstrokecolor{currentstroke}%
\pgfsetstrokeopacity{0.305289}%
\pgfsetdash{}{0pt}%
\pgfpathmoveto{\pgfqpoint{1.931388in}{2.029140in}}%
\pgfpathcurveto{\pgfqpoint{1.939624in}{2.029140in}}{\pgfqpoint{1.947524in}{2.032412in}}{\pgfqpoint{1.953348in}{2.038236in}}%
\pgfpathcurveto{\pgfqpoint{1.959172in}{2.044060in}}{\pgfqpoint{1.962444in}{2.051960in}}{\pgfqpoint{1.962444in}{2.060197in}}%
\pgfpathcurveto{\pgfqpoint{1.962444in}{2.068433in}}{\pgfqpoint{1.959172in}{2.076333in}}{\pgfqpoint{1.953348in}{2.082157in}}%
\pgfpathcurveto{\pgfqpoint{1.947524in}{2.087981in}}{\pgfqpoint{1.939624in}{2.091253in}}{\pgfqpoint{1.931388in}{2.091253in}}%
\pgfpathcurveto{\pgfqpoint{1.923152in}{2.091253in}}{\pgfqpoint{1.915252in}{2.087981in}}{\pgfqpoint{1.909428in}{2.082157in}}%
\pgfpathcurveto{\pgfqpoint{1.903604in}{2.076333in}}{\pgfqpoint{1.900331in}{2.068433in}}{\pgfqpoint{1.900331in}{2.060197in}}%
\pgfpathcurveto{\pgfqpoint{1.900331in}{2.051960in}}{\pgfqpoint{1.903604in}{2.044060in}}{\pgfqpoint{1.909428in}{2.038236in}}%
\pgfpathcurveto{\pgfqpoint{1.915252in}{2.032412in}}{\pgfqpoint{1.923152in}{2.029140in}}{\pgfqpoint{1.931388in}{2.029140in}}%
\pgfpathclose%
\pgfusepath{stroke,fill}%
\end{pgfscope}%
\begin{pgfscope}%
\pgfpathrectangle{\pgfqpoint{0.100000in}{0.220728in}}{\pgfqpoint{3.696000in}{3.696000in}}%
\pgfusepath{clip}%
\pgfsetbuttcap%
\pgfsetroundjoin%
\definecolor{currentfill}{rgb}{0.121569,0.466667,0.705882}%
\pgfsetfillcolor{currentfill}%
\pgfsetfillopacity{0.305323}%
\pgfsetlinewidth{1.003750pt}%
\definecolor{currentstroke}{rgb}{0.121569,0.466667,0.705882}%
\pgfsetstrokecolor{currentstroke}%
\pgfsetstrokeopacity{0.305323}%
\pgfsetdash{}{0pt}%
\pgfpathmoveto{\pgfqpoint{1.953631in}{2.028939in}}%
\pgfpathcurveto{\pgfqpoint{1.961867in}{2.028939in}}{\pgfqpoint{1.969767in}{2.032211in}}{\pgfqpoint{1.975591in}{2.038035in}}%
\pgfpathcurveto{\pgfqpoint{1.981415in}{2.043859in}}{\pgfqpoint{1.984687in}{2.051759in}}{\pgfqpoint{1.984687in}{2.059996in}}%
\pgfpathcurveto{\pgfqpoint{1.984687in}{2.068232in}}{\pgfqpoint{1.981415in}{2.076132in}}{\pgfqpoint{1.975591in}{2.081956in}}%
\pgfpathcurveto{\pgfqpoint{1.969767in}{2.087780in}}{\pgfqpoint{1.961867in}{2.091052in}}{\pgfqpoint{1.953631in}{2.091052in}}%
\pgfpathcurveto{\pgfqpoint{1.945394in}{2.091052in}}{\pgfqpoint{1.937494in}{2.087780in}}{\pgfqpoint{1.931670in}{2.081956in}}%
\pgfpathcurveto{\pgfqpoint{1.925847in}{2.076132in}}{\pgfqpoint{1.922574in}{2.068232in}}{\pgfqpoint{1.922574in}{2.059996in}}%
\pgfpathcurveto{\pgfqpoint{1.922574in}{2.051759in}}{\pgfqpoint{1.925847in}{2.043859in}}{\pgfqpoint{1.931670in}{2.038035in}}%
\pgfpathcurveto{\pgfqpoint{1.937494in}{2.032211in}}{\pgfqpoint{1.945394in}{2.028939in}}{\pgfqpoint{1.953631in}{2.028939in}}%
\pgfpathclose%
\pgfusepath{stroke,fill}%
\end{pgfscope}%
\begin{pgfscope}%
\pgfpathrectangle{\pgfqpoint{0.100000in}{0.220728in}}{\pgfqpoint{3.696000in}{3.696000in}}%
\pgfusepath{clip}%
\pgfsetbuttcap%
\pgfsetroundjoin%
\definecolor{currentfill}{rgb}{0.121569,0.466667,0.705882}%
\pgfsetfillcolor{currentfill}%
\pgfsetfillopacity{0.307011}%
\pgfsetlinewidth{1.003750pt}%
\definecolor{currentstroke}{rgb}{0.121569,0.466667,0.705882}%
\pgfsetstrokecolor{currentstroke}%
\pgfsetstrokeopacity{0.307011}%
\pgfsetdash{}{0pt}%
\pgfpathmoveto{\pgfqpoint{1.926896in}{2.028505in}}%
\pgfpathcurveto{\pgfqpoint{1.935132in}{2.028505in}}{\pgfqpoint{1.943032in}{2.031778in}}{\pgfqpoint{1.948856in}{2.037602in}}%
\pgfpathcurveto{\pgfqpoint{1.954680in}{2.043426in}}{\pgfqpoint{1.957952in}{2.051326in}}{\pgfqpoint{1.957952in}{2.059562in}}%
\pgfpathcurveto{\pgfqpoint{1.957952in}{2.067798in}}{\pgfqpoint{1.954680in}{2.075698in}}{\pgfqpoint{1.948856in}{2.081522in}}%
\pgfpathcurveto{\pgfqpoint{1.943032in}{2.087346in}}{\pgfqpoint{1.935132in}{2.090618in}}{\pgfqpoint{1.926896in}{2.090618in}}%
\pgfpathcurveto{\pgfqpoint{1.918659in}{2.090618in}}{\pgfqpoint{1.910759in}{2.087346in}}{\pgfqpoint{1.904935in}{2.081522in}}%
\pgfpathcurveto{\pgfqpoint{1.899111in}{2.075698in}}{\pgfqpoint{1.895839in}{2.067798in}}{\pgfqpoint{1.895839in}{2.059562in}}%
\pgfpathcurveto{\pgfqpoint{1.895839in}{2.051326in}}{\pgfqpoint{1.899111in}{2.043426in}}{\pgfqpoint{1.904935in}{2.037602in}}%
\pgfpathcurveto{\pgfqpoint{1.910759in}{2.031778in}}{\pgfqpoint{1.918659in}{2.028505in}}{\pgfqpoint{1.926896in}{2.028505in}}%
\pgfpathclose%
\pgfusepath{stroke,fill}%
\end{pgfscope}%
\begin{pgfscope}%
\pgfpathrectangle{\pgfqpoint{0.100000in}{0.220728in}}{\pgfqpoint{3.696000in}{3.696000in}}%
\pgfusepath{clip}%
\pgfsetbuttcap%
\pgfsetroundjoin%
\definecolor{currentfill}{rgb}{0.121569,0.466667,0.705882}%
\pgfsetfillcolor{currentfill}%
\pgfsetfillopacity{0.307582}%
\pgfsetlinewidth{1.003750pt}%
\definecolor{currentstroke}{rgb}{0.121569,0.466667,0.705882}%
\pgfsetstrokecolor{currentstroke}%
\pgfsetstrokeopacity{0.307582}%
\pgfsetdash{}{0pt}%
\pgfpathmoveto{\pgfqpoint{1.954148in}{2.025231in}}%
\pgfpathcurveto{\pgfqpoint{1.962384in}{2.025231in}}{\pgfqpoint{1.970284in}{2.028504in}}{\pgfqpoint{1.976108in}{2.034328in}}%
\pgfpathcurveto{\pgfqpoint{1.981932in}{2.040151in}}{\pgfqpoint{1.985205in}{2.048052in}}{\pgfqpoint{1.985205in}{2.056288in}}%
\pgfpathcurveto{\pgfqpoint{1.985205in}{2.064524in}}{\pgfqpoint{1.981932in}{2.072424in}}{\pgfqpoint{1.976108in}{2.078248in}}%
\pgfpathcurveto{\pgfqpoint{1.970284in}{2.084072in}}{\pgfqpoint{1.962384in}{2.087344in}}{\pgfqpoint{1.954148in}{2.087344in}}%
\pgfpathcurveto{\pgfqpoint{1.945912in}{2.087344in}}{\pgfqpoint{1.938012in}{2.084072in}}{\pgfqpoint{1.932188in}{2.078248in}}%
\pgfpathcurveto{\pgfqpoint{1.926364in}{2.072424in}}{\pgfqpoint{1.923092in}{2.064524in}}{\pgfqpoint{1.923092in}{2.056288in}}%
\pgfpathcurveto{\pgfqpoint{1.923092in}{2.048052in}}{\pgfqpoint{1.926364in}{2.040151in}}{\pgfqpoint{1.932188in}{2.034328in}}%
\pgfpathcurveto{\pgfqpoint{1.938012in}{2.028504in}}{\pgfqpoint{1.945912in}{2.025231in}}{\pgfqpoint{1.954148in}{2.025231in}}%
\pgfpathclose%
\pgfusepath{stroke,fill}%
\end{pgfscope}%
\begin{pgfscope}%
\pgfpathrectangle{\pgfqpoint{0.100000in}{0.220728in}}{\pgfqpoint{3.696000in}{3.696000in}}%
\pgfusepath{clip}%
\pgfsetbuttcap%
\pgfsetroundjoin%
\definecolor{currentfill}{rgb}{0.121569,0.466667,0.705882}%
\pgfsetfillcolor{currentfill}%
\pgfsetfillopacity{0.308306}%
\pgfsetlinewidth{1.003750pt}%
\definecolor{currentstroke}{rgb}{0.121569,0.466667,0.705882}%
\pgfsetstrokecolor{currentstroke}%
\pgfsetstrokeopacity{0.308306}%
\pgfsetdash{}{0pt}%
\pgfpathmoveto{\pgfqpoint{1.922368in}{2.026740in}}%
\pgfpathcurveto{\pgfqpoint{1.930604in}{2.026740in}}{\pgfqpoint{1.938504in}{2.030013in}}{\pgfqpoint{1.944328in}{2.035837in}}%
\pgfpathcurveto{\pgfqpoint{1.950152in}{2.041660in}}{\pgfqpoint{1.953424in}{2.049561in}}{\pgfqpoint{1.953424in}{2.057797in}}%
\pgfpathcurveto{\pgfqpoint{1.953424in}{2.066033in}}{\pgfqpoint{1.950152in}{2.073933in}}{\pgfqpoint{1.944328in}{2.079757in}}%
\pgfpathcurveto{\pgfqpoint{1.938504in}{2.085581in}}{\pgfqpoint{1.930604in}{2.088853in}}{\pgfqpoint{1.922368in}{2.088853in}}%
\pgfpathcurveto{\pgfqpoint{1.914131in}{2.088853in}}{\pgfqpoint{1.906231in}{2.085581in}}{\pgfqpoint{1.900407in}{2.079757in}}%
\pgfpathcurveto{\pgfqpoint{1.894584in}{2.073933in}}{\pgfqpoint{1.891311in}{2.066033in}}{\pgfqpoint{1.891311in}{2.057797in}}%
\pgfpathcurveto{\pgfqpoint{1.891311in}{2.049561in}}{\pgfqpoint{1.894584in}{2.041660in}}{\pgfqpoint{1.900407in}{2.035837in}}%
\pgfpathcurveto{\pgfqpoint{1.906231in}{2.030013in}}{\pgfqpoint{1.914131in}{2.026740in}}{\pgfqpoint{1.922368in}{2.026740in}}%
\pgfpathclose%
\pgfusepath{stroke,fill}%
\end{pgfscope}%
\begin{pgfscope}%
\pgfpathrectangle{\pgfqpoint{0.100000in}{0.220728in}}{\pgfqpoint{3.696000in}{3.696000in}}%
\pgfusepath{clip}%
\pgfsetbuttcap%
\pgfsetroundjoin%
\definecolor{currentfill}{rgb}{0.121569,0.466667,0.705882}%
\pgfsetfillcolor{currentfill}%
\pgfsetfillopacity{0.308879}%
\pgfsetlinewidth{1.003750pt}%
\definecolor{currentstroke}{rgb}{0.121569,0.466667,0.705882}%
\pgfsetstrokecolor{currentstroke}%
\pgfsetstrokeopacity{0.308879}%
\pgfsetdash{}{0pt}%
\pgfpathmoveto{\pgfqpoint{1.920009in}{2.025544in}}%
\pgfpathcurveto{\pgfqpoint{1.928246in}{2.025544in}}{\pgfqpoint{1.936146in}{2.028817in}}{\pgfqpoint{1.941970in}{2.034641in}}%
\pgfpathcurveto{\pgfqpoint{1.947794in}{2.040465in}}{\pgfqpoint{1.951066in}{2.048365in}}{\pgfqpoint{1.951066in}{2.056601in}}%
\pgfpathcurveto{\pgfqpoint{1.951066in}{2.064837in}}{\pgfqpoint{1.947794in}{2.072737in}}{\pgfqpoint{1.941970in}{2.078561in}}%
\pgfpathcurveto{\pgfqpoint{1.936146in}{2.084385in}}{\pgfqpoint{1.928246in}{2.087657in}}{\pgfqpoint{1.920009in}{2.087657in}}%
\pgfpathcurveto{\pgfqpoint{1.911773in}{2.087657in}}{\pgfqpoint{1.903873in}{2.084385in}}{\pgfqpoint{1.898049in}{2.078561in}}%
\pgfpathcurveto{\pgfqpoint{1.892225in}{2.072737in}}{\pgfqpoint{1.888953in}{2.064837in}}{\pgfqpoint{1.888953in}{2.056601in}}%
\pgfpathcurveto{\pgfqpoint{1.888953in}{2.048365in}}{\pgfqpoint{1.892225in}{2.040465in}}{\pgfqpoint{1.898049in}{2.034641in}}%
\pgfpathcurveto{\pgfqpoint{1.903873in}{2.028817in}}{\pgfqpoint{1.911773in}{2.025544in}}{\pgfqpoint{1.920009in}{2.025544in}}%
\pgfpathclose%
\pgfusepath{stroke,fill}%
\end{pgfscope}%
\begin{pgfscope}%
\pgfpathrectangle{\pgfqpoint{0.100000in}{0.220728in}}{\pgfqpoint{3.696000in}{3.696000in}}%
\pgfusepath{clip}%
\pgfsetbuttcap%
\pgfsetroundjoin%
\definecolor{currentfill}{rgb}{0.121569,0.466667,0.705882}%
\pgfsetfillcolor{currentfill}%
\pgfsetfillopacity{0.309921}%
\pgfsetlinewidth{1.003750pt}%
\definecolor{currentstroke}{rgb}{0.121569,0.466667,0.705882}%
\pgfsetstrokecolor{currentstroke}%
\pgfsetstrokeopacity{0.309921}%
\pgfsetdash{}{0pt}%
\pgfpathmoveto{\pgfqpoint{1.915862in}{2.023106in}}%
\pgfpathcurveto{\pgfqpoint{1.924098in}{2.023106in}}{\pgfqpoint{1.931998in}{2.026378in}}{\pgfqpoint{1.937822in}{2.032202in}}%
\pgfpathcurveto{\pgfqpoint{1.943646in}{2.038026in}}{\pgfqpoint{1.946919in}{2.045926in}}{\pgfqpoint{1.946919in}{2.054162in}}%
\pgfpathcurveto{\pgfqpoint{1.946919in}{2.062399in}}{\pgfqpoint{1.943646in}{2.070299in}}{\pgfqpoint{1.937822in}{2.076123in}}%
\pgfpathcurveto{\pgfqpoint{1.931998in}{2.081947in}}{\pgfqpoint{1.924098in}{2.085219in}}{\pgfqpoint{1.915862in}{2.085219in}}%
\pgfpathcurveto{\pgfqpoint{1.907626in}{2.085219in}}{\pgfqpoint{1.899726in}{2.081947in}}{\pgfqpoint{1.893902in}{2.076123in}}%
\pgfpathcurveto{\pgfqpoint{1.888078in}{2.070299in}}{\pgfqpoint{1.884806in}{2.062399in}}{\pgfqpoint{1.884806in}{2.054162in}}%
\pgfpathcurveto{\pgfqpoint{1.884806in}{2.045926in}}{\pgfqpoint{1.888078in}{2.038026in}}{\pgfqpoint{1.893902in}{2.032202in}}%
\pgfpathcurveto{\pgfqpoint{1.899726in}{2.026378in}}{\pgfqpoint{1.907626in}{2.023106in}}{\pgfqpoint{1.915862in}{2.023106in}}%
\pgfpathclose%
\pgfusepath{stroke,fill}%
\end{pgfscope}%
\begin{pgfscope}%
\pgfpathrectangle{\pgfqpoint{0.100000in}{0.220728in}}{\pgfqpoint{3.696000in}{3.696000in}}%
\pgfusepath{clip}%
\pgfsetbuttcap%
\pgfsetroundjoin%
\definecolor{currentfill}{rgb}{0.121569,0.466667,0.705882}%
\pgfsetfillcolor{currentfill}%
\pgfsetfillopacity{0.310025}%
\pgfsetlinewidth{1.003750pt}%
\definecolor{currentstroke}{rgb}{0.121569,0.466667,0.705882}%
\pgfsetstrokecolor{currentstroke}%
\pgfsetstrokeopacity{0.310025}%
\pgfsetdash{}{0pt}%
\pgfpathmoveto{\pgfqpoint{1.956741in}{2.021057in}}%
\pgfpathcurveto{\pgfqpoint{1.964977in}{2.021057in}}{\pgfqpoint{1.972877in}{2.024329in}}{\pgfqpoint{1.978701in}{2.030153in}}%
\pgfpathcurveto{\pgfqpoint{1.984525in}{2.035977in}}{\pgfqpoint{1.987797in}{2.043877in}}{\pgfqpoint{1.987797in}{2.052113in}}%
\pgfpathcurveto{\pgfqpoint{1.987797in}{2.060349in}}{\pgfqpoint{1.984525in}{2.068250in}}{\pgfqpoint{1.978701in}{2.074073in}}%
\pgfpathcurveto{\pgfqpoint{1.972877in}{2.079897in}}{\pgfqpoint{1.964977in}{2.083170in}}{\pgfqpoint{1.956741in}{2.083170in}}%
\pgfpathcurveto{\pgfqpoint{1.948504in}{2.083170in}}{\pgfqpoint{1.940604in}{2.079897in}}{\pgfqpoint{1.934780in}{2.074073in}}%
\pgfpathcurveto{\pgfqpoint{1.928957in}{2.068250in}}{\pgfqpoint{1.925684in}{2.060349in}}{\pgfqpoint{1.925684in}{2.052113in}}%
\pgfpathcurveto{\pgfqpoint{1.925684in}{2.043877in}}{\pgfqpoint{1.928957in}{2.035977in}}{\pgfqpoint{1.934780in}{2.030153in}}%
\pgfpathcurveto{\pgfqpoint{1.940604in}{2.024329in}}{\pgfqpoint{1.948504in}{2.021057in}}{\pgfqpoint{1.956741in}{2.021057in}}%
\pgfpathclose%
\pgfusepath{stroke,fill}%
\end{pgfscope}%
\begin{pgfscope}%
\pgfpathrectangle{\pgfqpoint{0.100000in}{0.220728in}}{\pgfqpoint{3.696000in}{3.696000in}}%
\pgfusepath{clip}%
\pgfsetbuttcap%
\pgfsetroundjoin%
\definecolor{currentfill}{rgb}{0.121569,0.466667,0.705882}%
\pgfsetfillcolor{currentfill}%
\pgfsetfillopacity{0.311814}%
\pgfsetlinewidth{1.003750pt}%
\definecolor{currentstroke}{rgb}{0.121569,0.466667,0.705882}%
\pgfsetstrokecolor{currentstroke}%
\pgfsetstrokeopacity{0.311814}%
\pgfsetdash{}{0pt}%
\pgfpathmoveto{\pgfqpoint{1.958161in}{2.021857in}}%
\pgfpathcurveto{\pgfqpoint{1.966398in}{2.021857in}}{\pgfqpoint{1.974298in}{2.025129in}}{\pgfqpoint{1.980122in}{2.030953in}}%
\pgfpathcurveto{\pgfqpoint{1.985945in}{2.036777in}}{\pgfqpoint{1.989218in}{2.044677in}}{\pgfqpoint{1.989218in}{2.052914in}}%
\pgfpathcurveto{\pgfqpoint{1.989218in}{2.061150in}}{\pgfqpoint{1.985945in}{2.069050in}}{\pgfqpoint{1.980122in}{2.074874in}}%
\pgfpathcurveto{\pgfqpoint{1.974298in}{2.080698in}}{\pgfqpoint{1.966398in}{2.083970in}}{\pgfqpoint{1.958161in}{2.083970in}}%
\pgfpathcurveto{\pgfqpoint{1.949925in}{2.083970in}}{\pgfqpoint{1.942025in}{2.080698in}}{\pgfqpoint{1.936201in}{2.074874in}}%
\pgfpathcurveto{\pgfqpoint{1.930377in}{2.069050in}}{\pgfqpoint{1.927105in}{2.061150in}}{\pgfqpoint{1.927105in}{2.052914in}}%
\pgfpathcurveto{\pgfqpoint{1.927105in}{2.044677in}}{\pgfqpoint{1.930377in}{2.036777in}}{\pgfqpoint{1.936201in}{2.030953in}}%
\pgfpathcurveto{\pgfqpoint{1.942025in}{2.025129in}}{\pgfqpoint{1.949925in}{2.021857in}}{\pgfqpoint{1.958161in}{2.021857in}}%
\pgfpathclose%
\pgfusepath{stroke,fill}%
\end{pgfscope}%
\begin{pgfscope}%
\pgfpathrectangle{\pgfqpoint{0.100000in}{0.220728in}}{\pgfqpoint{3.696000in}{3.696000in}}%
\pgfusepath{clip}%
\pgfsetbuttcap%
\pgfsetroundjoin%
\definecolor{currentfill}{rgb}{0.121569,0.466667,0.705882}%
\pgfsetfillcolor{currentfill}%
\pgfsetfillopacity{0.312492}%
\pgfsetlinewidth{1.003750pt}%
\definecolor{currentstroke}{rgb}{0.121569,0.466667,0.705882}%
\pgfsetstrokecolor{currentstroke}%
\pgfsetstrokeopacity{0.312492}%
\pgfsetdash{}{0pt}%
\pgfpathmoveto{\pgfqpoint{1.909042in}{2.022161in}}%
\pgfpathcurveto{\pgfqpoint{1.917279in}{2.022161in}}{\pgfqpoint{1.925179in}{2.025433in}}{\pgfqpoint{1.931003in}{2.031257in}}%
\pgfpathcurveto{\pgfqpoint{1.936827in}{2.037081in}}{\pgfqpoint{1.940099in}{2.044981in}}{\pgfqpoint{1.940099in}{2.053218in}}%
\pgfpathcurveto{\pgfqpoint{1.940099in}{2.061454in}}{\pgfqpoint{1.936827in}{2.069354in}}{\pgfqpoint{1.931003in}{2.075178in}}%
\pgfpathcurveto{\pgfqpoint{1.925179in}{2.081002in}}{\pgfqpoint{1.917279in}{2.084274in}}{\pgfqpoint{1.909042in}{2.084274in}}%
\pgfpathcurveto{\pgfqpoint{1.900806in}{2.084274in}}{\pgfqpoint{1.892906in}{2.081002in}}{\pgfqpoint{1.887082in}{2.075178in}}%
\pgfpathcurveto{\pgfqpoint{1.881258in}{2.069354in}}{\pgfqpoint{1.877986in}{2.061454in}}{\pgfqpoint{1.877986in}{2.053218in}}%
\pgfpathcurveto{\pgfqpoint{1.877986in}{2.044981in}}{\pgfqpoint{1.881258in}{2.037081in}}{\pgfqpoint{1.887082in}{2.031257in}}%
\pgfpathcurveto{\pgfqpoint{1.892906in}{2.025433in}}{\pgfqpoint{1.900806in}{2.022161in}}{\pgfqpoint{1.909042in}{2.022161in}}%
\pgfpathclose%
\pgfusepath{stroke,fill}%
\end{pgfscope}%
\begin{pgfscope}%
\pgfpathrectangle{\pgfqpoint{0.100000in}{0.220728in}}{\pgfqpoint{3.696000in}{3.696000in}}%
\pgfusepath{clip}%
\pgfsetbuttcap%
\pgfsetroundjoin%
\definecolor{currentfill}{rgb}{0.121569,0.466667,0.705882}%
\pgfsetfillcolor{currentfill}%
\pgfsetfillopacity{0.313530}%
\pgfsetlinewidth{1.003750pt}%
\definecolor{currentstroke}{rgb}{0.121569,0.466667,0.705882}%
\pgfsetstrokecolor{currentstroke}%
\pgfsetstrokeopacity{0.313530}%
\pgfsetdash{}{0pt}%
\pgfpathmoveto{\pgfqpoint{1.904493in}{2.019587in}}%
\pgfpathcurveto{\pgfqpoint{1.912729in}{2.019587in}}{\pgfqpoint{1.920630in}{2.022859in}}{\pgfqpoint{1.926453in}{2.028683in}}%
\pgfpathcurveto{\pgfqpoint{1.932277in}{2.034507in}}{\pgfqpoint{1.935550in}{2.042407in}}{\pgfqpoint{1.935550in}{2.050644in}}%
\pgfpathcurveto{\pgfqpoint{1.935550in}{2.058880in}}{\pgfqpoint{1.932277in}{2.066780in}}{\pgfqpoint{1.926453in}{2.072604in}}%
\pgfpathcurveto{\pgfqpoint{1.920630in}{2.078428in}}{\pgfqpoint{1.912729in}{2.081700in}}{\pgfqpoint{1.904493in}{2.081700in}}%
\pgfpathcurveto{\pgfqpoint{1.896257in}{2.081700in}}{\pgfqpoint{1.888357in}{2.078428in}}{\pgfqpoint{1.882533in}{2.072604in}}%
\pgfpathcurveto{\pgfqpoint{1.876709in}{2.066780in}}{\pgfqpoint{1.873437in}{2.058880in}}{\pgfqpoint{1.873437in}{2.050644in}}%
\pgfpathcurveto{\pgfqpoint{1.873437in}{2.042407in}}{\pgfqpoint{1.876709in}{2.034507in}}{\pgfqpoint{1.882533in}{2.028683in}}%
\pgfpathcurveto{\pgfqpoint{1.888357in}{2.022859in}}{\pgfqpoint{1.896257in}{2.019587in}}{\pgfqpoint{1.904493in}{2.019587in}}%
\pgfpathclose%
\pgfusepath{stroke,fill}%
\end{pgfscope}%
\begin{pgfscope}%
\pgfpathrectangle{\pgfqpoint{0.100000in}{0.220728in}}{\pgfqpoint{3.696000in}{3.696000in}}%
\pgfusepath{clip}%
\pgfsetbuttcap%
\pgfsetroundjoin%
\definecolor{currentfill}{rgb}{0.121569,0.466667,0.705882}%
\pgfsetfillcolor{currentfill}%
\pgfsetfillopacity{0.313800}%
\pgfsetlinewidth{1.003750pt}%
\definecolor{currentstroke}{rgb}{0.121569,0.466667,0.705882}%
\pgfsetstrokecolor{currentstroke}%
\pgfsetstrokeopacity{0.313800}%
\pgfsetdash{}{0pt}%
\pgfpathmoveto{\pgfqpoint{1.958574in}{2.018678in}}%
\pgfpathcurveto{\pgfqpoint{1.966810in}{2.018678in}}{\pgfqpoint{1.974710in}{2.021951in}}{\pgfqpoint{1.980534in}{2.027774in}}%
\pgfpathcurveto{\pgfqpoint{1.986358in}{2.033598in}}{\pgfqpoint{1.989630in}{2.041498in}}{\pgfqpoint{1.989630in}{2.049735in}}%
\pgfpathcurveto{\pgfqpoint{1.989630in}{2.057971in}}{\pgfqpoint{1.986358in}{2.065871in}}{\pgfqpoint{1.980534in}{2.071695in}}%
\pgfpathcurveto{\pgfqpoint{1.974710in}{2.077519in}}{\pgfqpoint{1.966810in}{2.080791in}}{\pgfqpoint{1.958574in}{2.080791in}}%
\pgfpathcurveto{\pgfqpoint{1.950338in}{2.080791in}}{\pgfqpoint{1.942437in}{2.077519in}}{\pgfqpoint{1.936614in}{2.071695in}}%
\pgfpathcurveto{\pgfqpoint{1.930790in}{2.065871in}}{\pgfqpoint{1.927517in}{2.057971in}}{\pgfqpoint{1.927517in}{2.049735in}}%
\pgfpathcurveto{\pgfqpoint{1.927517in}{2.041498in}}{\pgfqpoint{1.930790in}{2.033598in}}{\pgfqpoint{1.936614in}{2.027774in}}%
\pgfpathcurveto{\pgfqpoint{1.942437in}{2.021951in}}{\pgfqpoint{1.950338in}{2.018678in}}{\pgfqpoint{1.958574in}{2.018678in}}%
\pgfpathclose%
\pgfusepath{stroke,fill}%
\end{pgfscope}%
\begin{pgfscope}%
\pgfpathrectangle{\pgfqpoint{0.100000in}{0.220728in}}{\pgfqpoint{3.696000in}{3.696000in}}%
\pgfusepath{clip}%
\pgfsetbuttcap%
\pgfsetroundjoin%
\definecolor{currentfill}{rgb}{0.121569,0.466667,0.705882}%
\pgfsetfillcolor{currentfill}%
\pgfsetfillopacity{0.314431}%
\pgfsetlinewidth{1.003750pt}%
\definecolor{currentstroke}{rgb}{0.121569,0.466667,0.705882}%
\pgfsetstrokecolor{currentstroke}%
\pgfsetstrokeopacity{0.314431}%
\pgfsetdash{}{0pt}%
\pgfpathmoveto{\pgfqpoint{1.900587in}{2.016612in}}%
\pgfpathcurveto{\pgfqpoint{1.908823in}{2.016612in}}{\pgfqpoint{1.916723in}{2.019884in}}{\pgfqpoint{1.922547in}{2.025708in}}%
\pgfpathcurveto{\pgfqpoint{1.928371in}{2.031532in}}{\pgfqpoint{1.931643in}{2.039432in}}{\pgfqpoint{1.931643in}{2.047668in}}%
\pgfpathcurveto{\pgfqpoint{1.931643in}{2.055904in}}{\pgfqpoint{1.928371in}{2.063804in}}{\pgfqpoint{1.922547in}{2.069628in}}%
\pgfpathcurveto{\pgfqpoint{1.916723in}{2.075452in}}{\pgfqpoint{1.908823in}{2.078725in}}{\pgfqpoint{1.900587in}{2.078725in}}%
\pgfpathcurveto{\pgfqpoint{1.892351in}{2.078725in}}{\pgfqpoint{1.884451in}{2.075452in}}{\pgfqpoint{1.878627in}{2.069628in}}%
\pgfpathcurveto{\pgfqpoint{1.872803in}{2.063804in}}{\pgfqpoint{1.869530in}{2.055904in}}{\pgfqpoint{1.869530in}{2.047668in}}%
\pgfpathcurveto{\pgfqpoint{1.869530in}{2.039432in}}{\pgfqpoint{1.872803in}{2.031532in}}{\pgfqpoint{1.878627in}{2.025708in}}%
\pgfpathcurveto{\pgfqpoint{1.884451in}{2.019884in}}{\pgfqpoint{1.892351in}{2.016612in}}{\pgfqpoint{1.900587in}{2.016612in}}%
\pgfpathclose%
\pgfusepath{stroke,fill}%
\end{pgfscope}%
\begin{pgfscope}%
\pgfpathrectangle{\pgfqpoint{0.100000in}{0.220728in}}{\pgfqpoint{3.696000in}{3.696000in}}%
\pgfusepath{clip}%
\pgfsetbuttcap%
\pgfsetroundjoin%
\definecolor{currentfill}{rgb}{0.121569,0.466667,0.705882}%
\pgfsetfillcolor{currentfill}%
\pgfsetfillopacity{0.314979}%
\pgfsetlinewidth{1.003750pt}%
\definecolor{currentstroke}{rgb}{0.121569,0.466667,0.705882}%
\pgfsetstrokecolor{currentstroke}%
\pgfsetstrokeopacity{0.314979}%
\pgfsetdash{}{0pt}%
\pgfpathmoveto{\pgfqpoint{1.959619in}{2.017919in}}%
\pgfpathcurveto{\pgfqpoint{1.967855in}{2.017919in}}{\pgfqpoint{1.975755in}{2.021192in}}{\pgfqpoint{1.981579in}{2.027016in}}%
\pgfpathcurveto{\pgfqpoint{1.987403in}{2.032840in}}{\pgfqpoint{1.990675in}{2.040740in}}{\pgfqpoint{1.990675in}{2.048976in}}%
\pgfpathcurveto{\pgfqpoint{1.990675in}{2.057212in}}{\pgfqpoint{1.987403in}{2.065112in}}{\pgfqpoint{1.981579in}{2.070936in}}%
\pgfpathcurveto{\pgfqpoint{1.975755in}{2.076760in}}{\pgfqpoint{1.967855in}{2.080032in}}{\pgfqpoint{1.959619in}{2.080032in}}%
\pgfpathcurveto{\pgfqpoint{1.951383in}{2.080032in}}{\pgfqpoint{1.943483in}{2.076760in}}{\pgfqpoint{1.937659in}{2.070936in}}%
\pgfpathcurveto{\pgfqpoint{1.931835in}{2.065112in}}{\pgfqpoint{1.928562in}{2.057212in}}{\pgfqpoint{1.928562in}{2.048976in}}%
\pgfpathcurveto{\pgfqpoint{1.928562in}{2.040740in}}{\pgfqpoint{1.931835in}{2.032840in}}{\pgfqpoint{1.937659in}{2.027016in}}%
\pgfpathcurveto{\pgfqpoint{1.943483in}{2.021192in}}{\pgfqpoint{1.951383in}{2.017919in}}{\pgfqpoint{1.959619in}{2.017919in}}%
\pgfpathclose%
\pgfusepath{stroke,fill}%
\end{pgfscope}%
\begin{pgfscope}%
\pgfpathrectangle{\pgfqpoint{0.100000in}{0.220728in}}{\pgfqpoint{3.696000in}{3.696000in}}%
\pgfusepath{clip}%
\pgfsetbuttcap%
\pgfsetroundjoin%
\definecolor{currentfill}{rgb}{0.121569,0.466667,0.705882}%
\pgfsetfillcolor{currentfill}%
\pgfsetfillopacity{0.316500}%
\pgfsetlinewidth{1.003750pt}%
\definecolor{currentstroke}{rgb}{0.121569,0.466667,0.705882}%
\pgfsetstrokecolor{currentstroke}%
\pgfsetstrokeopacity{0.316500}%
\pgfsetdash{}{0pt}%
\pgfpathmoveto{\pgfqpoint{1.894540in}{2.012541in}}%
\pgfpathcurveto{\pgfqpoint{1.902776in}{2.012541in}}{\pgfqpoint{1.910676in}{2.015813in}}{\pgfqpoint{1.916500in}{2.021637in}}%
\pgfpathcurveto{\pgfqpoint{1.922324in}{2.027461in}}{\pgfqpoint{1.925597in}{2.035361in}}{\pgfqpoint{1.925597in}{2.043597in}}%
\pgfpathcurveto{\pgfqpoint{1.925597in}{2.051834in}}{\pgfqpoint{1.922324in}{2.059734in}}{\pgfqpoint{1.916500in}{2.065558in}}%
\pgfpathcurveto{\pgfqpoint{1.910676in}{2.071381in}}{\pgfqpoint{1.902776in}{2.074654in}}{\pgfqpoint{1.894540in}{2.074654in}}%
\pgfpathcurveto{\pgfqpoint{1.886304in}{2.074654in}}{\pgfqpoint{1.878404in}{2.071381in}}{\pgfqpoint{1.872580in}{2.065558in}}%
\pgfpathcurveto{\pgfqpoint{1.866756in}{2.059734in}}{\pgfqpoint{1.863484in}{2.051834in}}{\pgfqpoint{1.863484in}{2.043597in}}%
\pgfpathcurveto{\pgfqpoint{1.863484in}{2.035361in}}{\pgfqpoint{1.866756in}{2.027461in}}{\pgfqpoint{1.872580in}{2.021637in}}%
\pgfpathcurveto{\pgfqpoint{1.878404in}{2.015813in}}{\pgfqpoint{1.886304in}{2.012541in}}{\pgfqpoint{1.894540in}{2.012541in}}%
\pgfpathclose%
\pgfusepath{stroke,fill}%
\end{pgfscope}%
\begin{pgfscope}%
\pgfpathrectangle{\pgfqpoint{0.100000in}{0.220728in}}{\pgfqpoint{3.696000in}{3.696000in}}%
\pgfusepath{clip}%
\pgfsetbuttcap%
\pgfsetroundjoin%
\definecolor{currentfill}{rgb}{0.121569,0.466667,0.705882}%
\pgfsetfillcolor{currentfill}%
\pgfsetfillopacity{0.316882}%
\pgfsetlinewidth{1.003750pt}%
\definecolor{currentstroke}{rgb}{0.121569,0.466667,0.705882}%
\pgfsetstrokecolor{currentstroke}%
\pgfsetstrokeopacity{0.316882}%
\pgfsetdash{}{0pt}%
\pgfpathmoveto{\pgfqpoint{1.961435in}{2.019512in}}%
\pgfpathcurveto{\pgfqpoint{1.969671in}{2.019512in}}{\pgfqpoint{1.977571in}{2.022785in}}{\pgfqpoint{1.983395in}{2.028609in}}%
\pgfpathcurveto{\pgfqpoint{1.989219in}{2.034433in}}{\pgfqpoint{1.992491in}{2.042333in}}{\pgfqpoint{1.992491in}{2.050569in}}%
\pgfpathcurveto{\pgfqpoint{1.992491in}{2.058805in}}{\pgfqpoint{1.989219in}{2.066705in}}{\pgfqpoint{1.983395in}{2.072529in}}%
\pgfpathcurveto{\pgfqpoint{1.977571in}{2.078353in}}{\pgfqpoint{1.969671in}{2.081625in}}{\pgfqpoint{1.961435in}{2.081625in}}%
\pgfpathcurveto{\pgfqpoint{1.953199in}{2.081625in}}{\pgfqpoint{1.945299in}{2.078353in}}{\pgfqpoint{1.939475in}{2.072529in}}%
\pgfpathcurveto{\pgfqpoint{1.933651in}{2.066705in}}{\pgfqpoint{1.930378in}{2.058805in}}{\pgfqpoint{1.930378in}{2.050569in}}%
\pgfpathcurveto{\pgfqpoint{1.930378in}{2.042333in}}{\pgfqpoint{1.933651in}{2.034433in}}{\pgfqpoint{1.939475in}{2.028609in}}%
\pgfpathcurveto{\pgfqpoint{1.945299in}{2.022785in}}{\pgfqpoint{1.953199in}{2.019512in}}{\pgfqpoint{1.961435in}{2.019512in}}%
\pgfpathclose%
\pgfusepath{stroke,fill}%
\end{pgfscope}%
\begin{pgfscope}%
\pgfpathrectangle{\pgfqpoint{0.100000in}{0.220728in}}{\pgfqpoint{3.696000in}{3.696000in}}%
\pgfusepath{clip}%
\pgfsetbuttcap%
\pgfsetroundjoin%
\definecolor{currentfill}{rgb}{0.121569,0.466667,0.705882}%
\pgfsetfillcolor{currentfill}%
\pgfsetfillopacity{0.317565}%
\pgfsetlinewidth{1.003750pt}%
\definecolor{currentstroke}{rgb}{0.121569,0.466667,0.705882}%
\pgfsetstrokecolor{currentstroke}%
\pgfsetstrokeopacity{0.317565}%
\pgfsetdash{}{0pt}%
\pgfpathmoveto{\pgfqpoint{1.889775in}{2.008815in}}%
\pgfpathcurveto{\pgfqpoint{1.898011in}{2.008815in}}{\pgfqpoint{1.905911in}{2.012087in}}{\pgfqpoint{1.911735in}{2.017911in}}%
\pgfpathcurveto{\pgfqpoint{1.917559in}{2.023735in}}{\pgfqpoint{1.920832in}{2.031635in}}{\pgfqpoint{1.920832in}{2.039871in}}%
\pgfpathcurveto{\pgfqpoint{1.920832in}{2.048107in}}{\pgfqpoint{1.917559in}{2.056007in}}{\pgfqpoint{1.911735in}{2.061831in}}%
\pgfpathcurveto{\pgfqpoint{1.905911in}{2.067655in}}{\pgfqpoint{1.898011in}{2.070928in}}{\pgfqpoint{1.889775in}{2.070928in}}%
\pgfpathcurveto{\pgfqpoint{1.881539in}{2.070928in}}{\pgfqpoint{1.873639in}{2.067655in}}{\pgfqpoint{1.867815in}{2.061831in}}%
\pgfpathcurveto{\pgfqpoint{1.861991in}{2.056007in}}{\pgfqpoint{1.858719in}{2.048107in}}{\pgfqpoint{1.858719in}{2.039871in}}%
\pgfpathcurveto{\pgfqpoint{1.858719in}{2.031635in}}{\pgfqpoint{1.861991in}{2.023735in}}{\pgfqpoint{1.867815in}{2.017911in}}%
\pgfpathcurveto{\pgfqpoint{1.873639in}{2.012087in}}{\pgfqpoint{1.881539in}{2.008815in}}{\pgfqpoint{1.889775in}{2.008815in}}%
\pgfpathclose%
\pgfusepath{stroke,fill}%
\end{pgfscope}%
\begin{pgfscope}%
\pgfpathrectangle{\pgfqpoint{0.100000in}{0.220728in}}{\pgfqpoint{3.696000in}{3.696000in}}%
\pgfusepath{clip}%
\pgfsetbuttcap%
\pgfsetroundjoin%
\definecolor{currentfill}{rgb}{0.121569,0.466667,0.705882}%
\pgfsetfillcolor{currentfill}%
\pgfsetfillopacity{0.317750}%
\pgfsetlinewidth{1.003750pt}%
\definecolor{currentstroke}{rgb}{0.121569,0.466667,0.705882}%
\pgfsetstrokecolor{currentstroke}%
\pgfsetstrokeopacity{0.317750}%
\pgfsetdash{}{0pt}%
\pgfpathmoveto{\pgfqpoint{1.961618in}{2.018695in}}%
\pgfpathcurveto{\pgfqpoint{1.969855in}{2.018695in}}{\pgfqpoint{1.977755in}{2.021967in}}{\pgfqpoint{1.983579in}{2.027791in}}%
\pgfpathcurveto{\pgfqpoint{1.989403in}{2.033615in}}{\pgfqpoint{1.992675in}{2.041515in}}{\pgfqpoint{1.992675in}{2.049751in}}%
\pgfpathcurveto{\pgfqpoint{1.992675in}{2.057987in}}{\pgfqpoint{1.989403in}{2.065887in}}{\pgfqpoint{1.983579in}{2.071711in}}%
\pgfpathcurveto{\pgfqpoint{1.977755in}{2.077535in}}{\pgfqpoint{1.969855in}{2.080808in}}{\pgfqpoint{1.961618in}{2.080808in}}%
\pgfpathcurveto{\pgfqpoint{1.953382in}{2.080808in}}{\pgfqpoint{1.945482in}{2.077535in}}{\pgfqpoint{1.939658in}{2.071711in}}%
\pgfpathcurveto{\pgfqpoint{1.933834in}{2.065887in}}{\pgfqpoint{1.930562in}{2.057987in}}{\pgfqpoint{1.930562in}{2.049751in}}%
\pgfpathcurveto{\pgfqpoint{1.930562in}{2.041515in}}{\pgfqpoint{1.933834in}{2.033615in}}{\pgfqpoint{1.939658in}{2.027791in}}%
\pgfpathcurveto{\pgfqpoint{1.945482in}{2.021967in}}{\pgfqpoint{1.953382in}{2.018695in}}{\pgfqpoint{1.961618in}{2.018695in}}%
\pgfpathclose%
\pgfusepath{stroke,fill}%
\end{pgfscope}%
\begin{pgfscope}%
\pgfpathrectangle{\pgfqpoint{0.100000in}{0.220728in}}{\pgfqpoint{3.696000in}{3.696000in}}%
\pgfusepath{clip}%
\pgfsetbuttcap%
\pgfsetroundjoin%
\definecolor{currentfill}{rgb}{0.121569,0.466667,0.705882}%
\pgfsetfillcolor{currentfill}%
\pgfsetfillopacity{0.318847}%
\pgfsetlinewidth{1.003750pt}%
\definecolor{currentstroke}{rgb}{0.121569,0.466667,0.705882}%
\pgfsetstrokecolor{currentstroke}%
\pgfsetstrokeopacity{0.318847}%
\pgfsetdash{}{0pt}%
\pgfpathmoveto{\pgfqpoint{1.885657in}{2.006897in}}%
\pgfpathcurveto{\pgfqpoint{1.893894in}{2.006897in}}{\pgfqpoint{1.901794in}{2.010170in}}{\pgfqpoint{1.907618in}{2.015994in}}%
\pgfpathcurveto{\pgfqpoint{1.913442in}{2.021817in}}{\pgfqpoint{1.916714in}{2.029717in}}{\pgfqpoint{1.916714in}{2.037954in}}%
\pgfpathcurveto{\pgfqpoint{1.916714in}{2.046190in}}{\pgfqpoint{1.913442in}{2.054090in}}{\pgfqpoint{1.907618in}{2.059914in}}%
\pgfpathcurveto{\pgfqpoint{1.901794in}{2.065738in}}{\pgfqpoint{1.893894in}{2.069010in}}{\pgfqpoint{1.885657in}{2.069010in}}%
\pgfpathcurveto{\pgfqpoint{1.877421in}{2.069010in}}{\pgfqpoint{1.869521in}{2.065738in}}{\pgfqpoint{1.863697in}{2.059914in}}%
\pgfpathcurveto{\pgfqpoint{1.857873in}{2.054090in}}{\pgfqpoint{1.854601in}{2.046190in}}{\pgfqpoint{1.854601in}{2.037954in}}%
\pgfpathcurveto{\pgfqpoint{1.854601in}{2.029717in}}{\pgfqpoint{1.857873in}{2.021817in}}{\pgfqpoint{1.863697in}{2.015994in}}%
\pgfpathcurveto{\pgfqpoint{1.869521in}{2.010170in}}{\pgfqpoint{1.877421in}{2.006897in}}{\pgfqpoint{1.885657in}{2.006897in}}%
\pgfpathclose%
\pgfusepath{stroke,fill}%
\end{pgfscope}%
\begin{pgfscope}%
\pgfpathrectangle{\pgfqpoint{0.100000in}{0.220728in}}{\pgfqpoint{3.696000in}{3.696000in}}%
\pgfusepath{clip}%
\pgfsetbuttcap%
\pgfsetroundjoin%
\definecolor{currentfill}{rgb}{0.121569,0.466667,0.705882}%
\pgfsetfillcolor{currentfill}%
\pgfsetfillopacity{0.319293}%
\pgfsetlinewidth{1.003750pt}%
\definecolor{currentstroke}{rgb}{0.121569,0.466667,0.705882}%
\pgfsetstrokecolor{currentstroke}%
\pgfsetstrokeopacity{0.319293}%
\pgfsetdash{}{0pt}%
\pgfpathmoveto{\pgfqpoint{1.963072in}{2.017778in}}%
\pgfpathcurveto{\pgfqpoint{1.971308in}{2.017778in}}{\pgfqpoint{1.979208in}{2.021050in}}{\pgfqpoint{1.985032in}{2.026874in}}%
\pgfpathcurveto{\pgfqpoint{1.990856in}{2.032698in}}{\pgfqpoint{1.994128in}{2.040598in}}{\pgfqpoint{1.994128in}{2.048834in}}%
\pgfpathcurveto{\pgfqpoint{1.994128in}{2.057070in}}{\pgfqpoint{1.990856in}{2.064970in}}{\pgfqpoint{1.985032in}{2.070794in}}%
\pgfpathcurveto{\pgfqpoint{1.979208in}{2.076618in}}{\pgfqpoint{1.971308in}{2.079891in}}{\pgfqpoint{1.963072in}{2.079891in}}%
\pgfpathcurveto{\pgfqpoint{1.954836in}{2.079891in}}{\pgfqpoint{1.946936in}{2.076618in}}{\pgfqpoint{1.941112in}{2.070794in}}%
\pgfpathcurveto{\pgfqpoint{1.935288in}{2.064970in}}{\pgfqpoint{1.932015in}{2.057070in}}{\pgfqpoint{1.932015in}{2.048834in}}%
\pgfpathcurveto{\pgfqpoint{1.932015in}{2.040598in}}{\pgfqpoint{1.935288in}{2.032698in}}{\pgfqpoint{1.941112in}{2.026874in}}%
\pgfpathcurveto{\pgfqpoint{1.946936in}{2.021050in}}{\pgfqpoint{1.954836in}{2.017778in}}{\pgfqpoint{1.963072in}{2.017778in}}%
\pgfpathclose%
\pgfusepath{stroke,fill}%
\end{pgfscope}%
\begin{pgfscope}%
\pgfpathrectangle{\pgfqpoint{0.100000in}{0.220728in}}{\pgfqpoint{3.696000in}{3.696000in}}%
\pgfusepath{clip}%
\pgfsetbuttcap%
\pgfsetroundjoin%
\definecolor{currentfill}{rgb}{0.121569,0.466667,0.705882}%
\pgfsetfillcolor{currentfill}%
\pgfsetfillopacity{0.321364}%
\pgfsetlinewidth{1.003750pt}%
\definecolor{currentstroke}{rgb}{0.121569,0.466667,0.705882}%
\pgfsetstrokecolor{currentstroke}%
\pgfsetstrokeopacity{0.321364}%
\pgfsetdash{}{0pt}%
\pgfpathmoveto{\pgfqpoint{1.964962in}{2.017942in}}%
\pgfpathcurveto{\pgfqpoint{1.973198in}{2.017942in}}{\pgfqpoint{1.981098in}{2.021215in}}{\pgfqpoint{1.986922in}{2.027038in}}%
\pgfpathcurveto{\pgfqpoint{1.992746in}{2.032862in}}{\pgfqpoint{1.996018in}{2.040762in}}{\pgfqpoint{1.996018in}{2.048999in}}%
\pgfpathcurveto{\pgfqpoint{1.996018in}{2.057235in}}{\pgfqpoint{1.992746in}{2.065135in}}{\pgfqpoint{1.986922in}{2.070959in}}%
\pgfpathcurveto{\pgfqpoint{1.981098in}{2.076783in}}{\pgfqpoint{1.973198in}{2.080055in}}{\pgfqpoint{1.964962in}{2.080055in}}%
\pgfpathcurveto{\pgfqpoint{1.956725in}{2.080055in}}{\pgfqpoint{1.948825in}{2.076783in}}{\pgfqpoint{1.943001in}{2.070959in}}%
\pgfpathcurveto{\pgfqpoint{1.937177in}{2.065135in}}{\pgfqpoint{1.933905in}{2.057235in}}{\pgfqpoint{1.933905in}{2.048999in}}%
\pgfpathcurveto{\pgfqpoint{1.933905in}{2.040762in}}{\pgfqpoint{1.937177in}{2.032862in}}{\pgfqpoint{1.943001in}{2.027038in}}%
\pgfpathcurveto{\pgfqpoint{1.948825in}{2.021215in}}{\pgfqpoint{1.956725in}{2.017942in}}{\pgfqpoint{1.964962in}{2.017942in}}%
\pgfpathclose%
\pgfusepath{stroke,fill}%
\end{pgfscope}%
\begin{pgfscope}%
\pgfpathrectangle{\pgfqpoint{0.100000in}{0.220728in}}{\pgfqpoint{3.696000in}{3.696000in}}%
\pgfusepath{clip}%
\pgfsetbuttcap%
\pgfsetroundjoin%
\definecolor{currentfill}{rgb}{0.121569,0.466667,0.705882}%
\pgfsetfillcolor{currentfill}%
\pgfsetfillopacity{0.321416}%
\pgfsetlinewidth{1.003750pt}%
\definecolor{currentstroke}{rgb}{0.121569,0.466667,0.705882}%
\pgfsetstrokecolor{currentstroke}%
\pgfsetstrokeopacity{0.321416}%
\pgfsetdash{}{0pt}%
\pgfpathmoveto{\pgfqpoint{1.879478in}{2.003277in}}%
\pgfpathcurveto{\pgfqpoint{1.887714in}{2.003277in}}{\pgfqpoint{1.895614in}{2.006550in}}{\pgfqpoint{1.901438in}{2.012374in}}%
\pgfpathcurveto{\pgfqpoint{1.907262in}{2.018198in}}{\pgfqpoint{1.910534in}{2.026098in}}{\pgfqpoint{1.910534in}{2.034334in}}%
\pgfpathcurveto{\pgfqpoint{1.910534in}{2.042570in}}{\pgfqpoint{1.907262in}{2.050470in}}{\pgfqpoint{1.901438in}{2.056294in}}%
\pgfpathcurveto{\pgfqpoint{1.895614in}{2.062118in}}{\pgfqpoint{1.887714in}{2.065390in}}{\pgfqpoint{1.879478in}{2.065390in}}%
\pgfpathcurveto{\pgfqpoint{1.871242in}{2.065390in}}{\pgfqpoint{1.863341in}{2.062118in}}{\pgfqpoint{1.857518in}{2.056294in}}%
\pgfpathcurveto{\pgfqpoint{1.851694in}{2.050470in}}{\pgfqpoint{1.848421in}{2.042570in}}{\pgfqpoint{1.848421in}{2.034334in}}%
\pgfpathcurveto{\pgfqpoint{1.848421in}{2.026098in}}{\pgfqpoint{1.851694in}{2.018198in}}{\pgfqpoint{1.857518in}{2.012374in}}%
\pgfpathcurveto{\pgfqpoint{1.863341in}{2.006550in}}{\pgfqpoint{1.871242in}{2.003277in}}{\pgfqpoint{1.879478in}{2.003277in}}%
\pgfpathclose%
\pgfusepath{stroke,fill}%
\end{pgfscope}%
\begin{pgfscope}%
\pgfpathrectangle{\pgfqpoint{0.100000in}{0.220728in}}{\pgfqpoint{3.696000in}{3.696000in}}%
\pgfusepath{clip}%
\pgfsetbuttcap%
\pgfsetroundjoin%
\definecolor{currentfill}{rgb}{0.121569,0.466667,0.705882}%
\pgfsetfillcolor{currentfill}%
\pgfsetfillopacity{0.322566}%
\pgfsetlinewidth{1.003750pt}%
\definecolor{currentstroke}{rgb}{0.121569,0.466667,0.705882}%
\pgfsetstrokecolor{currentstroke}%
\pgfsetstrokeopacity{0.322566}%
\pgfsetdash{}{0pt}%
\pgfpathmoveto{\pgfqpoint{1.873954in}{1.999486in}}%
\pgfpathcurveto{\pgfqpoint{1.882191in}{1.999486in}}{\pgfqpoint{1.890091in}{2.002759in}}{\pgfqpoint{1.895915in}{2.008582in}}%
\pgfpathcurveto{\pgfqpoint{1.901739in}{2.014406in}}{\pgfqpoint{1.905011in}{2.022306in}}{\pgfqpoint{1.905011in}{2.030543in}}%
\pgfpathcurveto{\pgfqpoint{1.905011in}{2.038779in}}{\pgfqpoint{1.901739in}{2.046679in}}{\pgfqpoint{1.895915in}{2.052503in}}%
\pgfpathcurveto{\pgfqpoint{1.890091in}{2.058327in}}{\pgfqpoint{1.882191in}{2.061599in}}{\pgfqpoint{1.873954in}{2.061599in}}%
\pgfpathcurveto{\pgfqpoint{1.865718in}{2.061599in}}{\pgfqpoint{1.857818in}{2.058327in}}{\pgfqpoint{1.851994in}{2.052503in}}%
\pgfpathcurveto{\pgfqpoint{1.846170in}{2.046679in}}{\pgfqpoint{1.842898in}{2.038779in}}{\pgfqpoint{1.842898in}{2.030543in}}%
\pgfpathcurveto{\pgfqpoint{1.842898in}{2.022306in}}{\pgfqpoint{1.846170in}{2.014406in}}{\pgfqpoint{1.851994in}{2.008582in}}%
\pgfpathcurveto{\pgfqpoint{1.857818in}{2.002759in}}{\pgfqpoint{1.865718in}{1.999486in}}{\pgfqpoint{1.873954in}{1.999486in}}%
\pgfpathclose%
\pgfusepath{stroke,fill}%
\end{pgfscope}%
\begin{pgfscope}%
\pgfpathrectangle{\pgfqpoint{0.100000in}{0.220728in}}{\pgfqpoint{3.696000in}{3.696000in}}%
\pgfusepath{clip}%
\pgfsetbuttcap%
\pgfsetroundjoin%
\definecolor{currentfill}{rgb}{0.121569,0.466667,0.705882}%
\pgfsetfillcolor{currentfill}%
\pgfsetfillopacity{0.323499}%
\pgfsetlinewidth{1.003750pt}%
\definecolor{currentstroke}{rgb}{0.121569,0.466667,0.705882}%
\pgfsetstrokecolor{currentstroke}%
\pgfsetstrokeopacity{0.323499}%
\pgfsetdash{}{0pt}%
\pgfpathmoveto{\pgfqpoint{1.965670in}{2.014632in}}%
\pgfpathcurveto{\pgfqpoint{1.973906in}{2.014632in}}{\pgfqpoint{1.981807in}{2.017904in}}{\pgfqpoint{1.987630in}{2.023728in}}%
\pgfpathcurveto{\pgfqpoint{1.993454in}{2.029552in}}{\pgfqpoint{1.996727in}{2.037452in}}{\pgfqpoint{1.996727in}{2.045688in}}%
\pgfpathcurveto{\pgfqpoint{1.996727in}{2.053924in}}{\pgfqpoint{1.993454in}{2.061824in}}{\pgfqpoint{1.987630in}{2.067648in}}%
\pgfpathcurveto{\pgfqpoint{1.981807in}{2.073472in}}{\pgfqpoint{1.973906in}{2.076745in}}{\pgfqpoint{1.965670in}{2.076745in}}%
\pgfpathcurveto{\pgfqpoint{1.957434in}{2.076745in}}{\pgfqpoint{1.949534in}{2.073472in}}{\pgfqpoint{1.943710in}{2.067648in}}%
\pgfpathcurveto{\pgfqpoint{1.937886in}{2.061824in}}{\pgfqpoint{1.934614in}{2.053924in}}{\pgfqpoint{1.934614in}{2.045688in}}%
\pgfpathcurveto{\pgfqpoint{1.934614in}{2.037452in}}{\pgfqpoint{1.937886in}{2.029552in}}{\pgfqpoint{1.943710in}{2.023728in}}%
\pgfpathcurveto{\pgfqpoint{1.949534in}{2.017904in}}{\pgfqpoint{1.957434in}{2.014632in}}{\pgfqpoint{1.965670in}{2.014632in}}%
\pgfpathclose%
\pgfusepath{stroke,fill}%
\end{pgfscope}%
\begin{pgfscope}%
\pgfpathrectangle{\pgfqpoint{0.100000in}{0.220728in}}{\pgfqpoint{3.696000in}{3.696000in}}%
\pgfusepath{clip}%
\pgfsetbuttcap%
\pgfsetroundjoin%
\definecolor{currentfill}{rgb}{0.121569,0.466667,0.705882}%
\pgfsetfillcolor{currentfill}%
\pgfsetfillopacity{0.323746}%
\pgfsetlinewidth{1.003750pt}%
\definecolor{currentstroke}{rgb}{0.121569,0.466667,0.705882}%
\pgfsetstrokecolor{currentstroke}%
\pgfsetstrokeopacity{0.323746}%
\pgfsetdash{}{0pt}%
\pgfpathmoveto{\pgfqpoint{1.869277in}{1.996450in}}%
\pgfpathcurveto{\pgfqpoint{1.877513in}{1.996450in}}{\pgfqpoint{1.885413in}{1.999722in}}{\pgfqpoint{1.891237in}{2.005546in}}%
\pgfpathcurveto{\pgfqpoint{1.897061in}{2.011370in}}{\pgfqpoint{1.900333in}{2.019270in}}{\pgfqpoint{1.900333in}{2.027506in}}%
\pgfpathcurveto{\pgfqpoint{1.900333in}{2.035743in}}{\pgfqpoint{1.897061in}{2.043643in}}{\pgfqpoint{1.891237in}{2.049467in}}%
\pgfpathcurveto{\pgfqpoint{1.885413in}{2.055290in}}{\pgfqpoint{1.877513in}{2.058563in}}{\pgfqpoint{1.869277in}{2.058563in}}%
\pgfpathcurveto{\pgfqpoint{1.861040in}{2.058563in}}{\pgfqpoint{1.853140in}{2.055290in}}{\pgfqpoint{1.847316in}{2.049467in}}%
\pgfpathcurveto{\pgfqpoint{1.841492in}{2.043643in}}{\pgfqpoint{1.838220in}{2.035743in}}{\pgfqpoint{1.838220in}{2.027506in}}%
\pgfpathcurveto{\pgfqpoint{1.838220in}{2.019270in}}{\pgfqpoint{1.841492in}{2.011370in}}{\pgfqpoint{1.847316in}{2.005546in}}%
\pgfpathcurveto{\pgfqpoint{1.853140in}{1.999722in}}{\pgfqpoint{1.861040in}{1.996450in}}{\pgfqpoint{1.869277in}{1.996450in}}%
\pgfpathclose%
\pgfusepath{stroke,fill}%
\end{pgfscope}%
\begin{pgfscope}%
\pgfpathrectangle{\pgfqpoint{0.100000in}{0.220728in}}{\pgfqpoint{3.696000in}{3.696000in}}%
\pgfusepath{clip}%
\pgfsetbuttcap%
\pgfsetroundjoin%
\definecolor{currentfill}{rgb}{0.121569,0.466667,0.705882}%
\pgfsetfillcolor{currentfill}%
\pgfsetfillopacity{0.326144}%
\pgfsetlinewidth{1.003750pt}%
\definecolor{currentstroke}{rgb}{0.121569,0.466667,0.705882}%
\pgfsetstrokecolor{currentstroke}%
\pgfsetstrokeopacity{0.326144}%
\pgfsetdash{}{0pt}%
\pgfpathmoveto{\pgfqpoint{1.967987in}{2.014414in}}%
\pgfpathcurveto{\pgfqpoint{1.976224in}{2.014414in}}{\pgfqpoint{1.984124in}{2.017687in}}{\pgfqpoint{1.989948in}{2.023511in}}%
\pgfpathcurveto{\pgfqpoint{1.995772in}{2.029334in}}{\pgfqpoint{1.999044in}{2.037235in}}{\pgfqpoint{1.999044in}{2.045471in}}%
\pgfpathcurveto{\pgfqpoint{1.999044in}{2.053707in}}{\pgfqpoint{1.995772in}{2.061607in}}{\pgfqpoint{1.989948in}{2.067431in}}%
\pgfpathcurveto{\pgfqpoint{1.984124in}{2.073255in}}{\pgfqpoint{1.976224in}{2.076527in}}{\pgfqpoint{1.967987in}{2.076527in}}%
\pgfpathcurveto{\pgfqpoint{1.959751in}{2.076527in}}{\pgfqpoint{1.951851in}{2.073255in}}{\pgfqpoint{1.946027in}{2.067431in}}%
\pgfpathcurveto{\pgfqpoint{1.940203in}{2.061607in}}{\pgfqpoint{1.936931in}{2.053707in}}{\pgfqpoint{1.936931in}{2.045471in}}%
\pgfpathcurveto{\pgfqpoint{1.936931in}{2.037235in}}{\pgfqpoint{1.940203in}{2.029334in}}{\pgfqpoint{1.946027in}{2.023511in}}%
\pgfpathcurveto{\pgfqpoint{1.951851in}{2.017687in}}{\pgfqpoint{1.959751in}{2.014414in}}{\pgfqpoint{1.967987in}{2.014414in}}%
\pgfpathclose%
\pgfusepath{stroke,fill}%
\end{pgfscope}%
\begin{pgfscope}%
\pgfpathrectangle{\pgfqpoint{0.100000in}{0.220728in}}{\pgfqpoint{3.696000in}{3.696000in}}%
\pgfusepath{clip}%
\pgfsetbuttcap%
\pgfsetroundjoin%
\definecolor{currentfill}{rgb}{0.121569,0.466667,0.705882}%
\pgfsetfillcolor{currentfill}%
\pgfsetfillopacity{0.326811}%
\pgfsetlinewidth{1.003750pt}%
\definecolor{currentstroke}{rgb}{0.121569,0.466667,0.705882}%
\pgfsetstrokecolor{currentstroke}%
\pgfsetstrokeopacity{0.326811}%
\pgfsetdash{}{0pt}%
\pgfpathmoveto{\pgfqpoint{1.862630in}{1.994612in}}%
\pgfpathcurveto{\pgfqpoint{1.870866in}{1.994612in}}{\pgfqpoint{1.878766in}{1.997884in}}{\pgfqpoint{1.884590in}{2.003708in}}%
\pgfpathcurveto{\pgfqpoint{1.890414in}{2.009532in}}{\pgfqpoint{1.893686in}{2.017432in}}{\pgfqpoint{1.893686in}{2.025668in}}%
\pgfpathcurveto{\pgfqpoint{1.893686in}{2.033904in}}{\pgfqpoint{1.890414in}{2.041805in}}{\pgfqpoint{1.884590in}{2.047628in}}%
\pgfpathcurveto{\pgfqpoint{1.878766in}{2.053452in}}{\pgfqpoint{1.870866in}{2.056725in}}{\pgfqpoint{1.862630in}{2.056725in}}%
\pgfpathcurveto{\pgfqpoint{1.854394in}{2.056725in}}{\pgfqpoint{1.846494in}{2.053452in}}{\pgfqpoint{1.840670in}{2.047628in}}%
\pgfpathcurveto{\pgfqpoint{1.834846in}{2.041805in}}{\pgfqpoint{1.831573in}{2.033904in}}{\pgfqpoint{1.831573in}{2.025668in}}%
\pgfpathcurveto{\pgfqpoint{1.831573in}{2.017432in}}{\pgfqpoint{1.834846in}{2.009532in}}{\pgfqpoint{1.840670in}{2.003708in}}%
\pgfpathcurveto{\pgfqpoint{1.846494in}{1.997884in}}{\pgfqpoint{1.854394in}{1.994612in}}{\pgfqpoint{1.862630in}{1.994612in}}%
\pgfpathclose%
\pgfusepath{stroke,fill}%
\end{pgfscope}%
\begin{pgfscope}%
\pgfpathrectangle{\pgfqpoint{0.100000in}{0.220728in}}{\pgfqpoint{3.696000in}{3.696000in}}%
\pgfusepath{clip}%
\pgfsetbuttcap%
\pgfsetroundjoin%
\definecolor{currentfill}{rgb}{0.121569,0.466667,0.705882}%
\pgfsetfillcolor{currentfill}%
\pgfsetfillopacity{0.328393}%
\pgfsetlinewidth{1.003750pt}%
\definecolor{currentstroke}{rgb}{0.121569,0.466667,0.705882}%
\pgfsetstrokecolor{currentstroke}%
\pgfsetstrokeopacity{0.328393}%
\pgfsetdash{}{0pt}%
\pgfpathmoveto{\pgfqpoint{1.855854in}{1.994670in}}%
\pgfpathcurveto{\pgfqpoint{1.864090in}{1.994670in}}{\pgfqpoint{1.871991in}{1.997943in}}{\pgfqpoint{1.877814in}{2.003767in}}%
\pgfpathcurveto{\pgfqpoint{1.883638in}{2.009591in}}{\pgfqpoint{1.886911in}{2.017491in}}{\pgfqpoint{1.886911in}{2.025727in}}%
\pgfpathcurveto{\pgfqpoint{1.886911in}{2.033963in}}{\pgfqpoint{1.883638in}{2.041863in}}{\pgfqpoint{1.877814in}{2.047687in}}%
\pgfpathcurveto{\pgfqpoint{1.871991in}{2.053511in}}{\pgfqpoint{1.864090in}{2.056783in}}{\pgfqpoint{1.855854in}{2.056783in}}%
\pgfpathcurveto{\pgfqpoint{1.847618in}{2.056783in}}{\pgfqpoint{1.839718in}{2.053511in}}{\pgfqpoint{1.833894in}{2.047687in}}%
\pgfpathcurveto{\pgfqpoint{1.828070in}{2.041863in}}{\pgfqpoint{1.824798in}{2.033963in}}{\pgfqpoint{1.824798in}{2.025727in}}%
\pgfpathcurveto{\pgfqpoint{1.824798in}{2.017491in}}{\pgfqpoint{1.828070in}{2.009591in}}{\pgfqpoint{1.833894in}{2.003767in}}%
\pgfpathcurveto{\pgfqpoint{1.839718in}{1.997943in}}{\pgfqpoint{1.847618in}{1.994670in}}{\pgfqpoint{1.855854in}{1.994670in}}%
\pgfpathclose%
\pgfusepath{stroke,fill}%
\end{pgfscope}%
\begin{pgfscope}%
\pgfpathrectangle{\pgfqpoint{0.100000in}{0.220728in}}{\pgfqpoint{3.696000in}{3.696000in}}%
\pgfusepath{clip}%
\pgfsetbuttcap%
\pgfsetroundjoin%
\definecolor{currentfill}{rgb}{0.121569,0.466667,0.705882}%
\pgfsetfillcolor{currentfill}%
\pgfsetfillopacity{0.329185}%
\pgfsetlinewidth{1.003750pt}%
\definecolor{currentstroke}{rgb}{0.121569,0.466667,0.705882}%
\pgfsetstrokecolor{currentstroke}%
\pgfsetstrokeopacity{0.329185}%
\pgfsetdash{}{0pt}%
\pgfpathmoveto{\pgfqpoint{1.852696in}{1.992067in}}%
\pgfpathcurveto{\pgfqpoint{1.860932in}{1.992067in}}{\pgfqpoint{1.868832in}{1.995339in}}{\pgfqpoint{1.874656in}{2.001163in}}%
\pgfpathcurveto{\pgfqpoint{1.880480in}{2.006987in}}{\pgfqpoint{1.883753in}{2.014887in}}{\pgfqpoint{1.883753in}{2.023123in}}%
\pgfpathcurveto{\pgfqpoint{1.883753in}{2.031360in}}{\pgfqpoint{1.880480in}{2.039260in}}{\pgfqpoint{1.874656in}{2.045084in}}%
\pgfpathcurveto{\pgfqpoint{1.868832in}{2.050908in}}{\pgfqpoint{1.860932in}{2.054180in}}{\pgfqpoint{1.852696in}{2.054180in}}%
\pgfpathcurveto{\pgfqpoint{1.844460in}{2.054180in}}{\pgfqpoint{1.836560in}{2.050908in}}{\pgfqpoint{1.830736in}{2.045084in}}%
\pgfpathcurveto{\pgfqpoint{1.824912in}{2.039260in}}{\pgfqpoint{1.821640in}{2.031360in}}{\pgfqpoint{1.821640in}{2.023123in}}%
\pgfpathcurveto{\pgfqpoint{1.821640in}{2.014887in}}{\pgfqpoint{1.824912in}{2.006987in}}{\pgfqpoint{1.830736in}{2.001163in}}%
\pgfpathcurveto{\pgfqpoint{1.836560in}{1.995339in}}{\pgfqpoint{1.844460in}{1.992067in}}{\pgfqpoint{1.852696in}{1.992067in}}%
\pgfpathclose%
\pgfusepath{stroke,fill}%
\end{pgfscope}%
\begin{pgfscope}%
\pgfpathrectangle{\pgfqpoint{0.100000in}{0.220728in}}{\pgfqpoint{3.696000in}{3.696000in}}%
\pgfusepath{clip}%
\pgfsetbuttcap%
\pgfsetroundjoin%
\definecolor{currentfill}{rgb}{0.121569,0.466667,0.705882}%
\pgfsetfillcolor{currentfill}%
\pgfsetfillopacity{0.329238}%
\pgfsetlinewidth{1.003750pt}%
\definecolor{currentstroke}{rgb}{0.121569,0.466667,0.705882}%
\pgfsetstrokecolor{currentstroke}%
\pgfsetstrokeopacity{0.329238}%
\pgfsetdash{}{0pt}%
\pgfpathmoveto{\pgfqpoint{1.970741in}{2.015904in}}%
\pgfpathcurveto{\pgfqpoint{1.978977in}{2.015904in}}{\pgfqpoint{1.986877in}{2.019176in}}{\pgfqpoint{1.992701in}{2.025000in}}%
\pgfpathcurveto{\pgfqpoint{1.998525in}{2.030824in}}{\pgfqpoint{2.001797in}{2.038724in}}{\pgfqpoint{2.001797in}{2.046960in}}%
\pgfpathcurveto{\pgfqpoint{2.001797in}{2.055196in}}{\pgfqpoint{1.998525in}{2.063096in}}{\pgfqpoint{1.992701in}{2.068920in}}%
\pgfpathcurveto{\pgfqpoint{1.986877in}{2.074744in}}{\pgfqpoint{1.978977in}{2.078017in}}{\pgfqpoint{1.970741in}{2.078017in}}%
\pgfpathcurveto{\pgfqpoint{1.962505in}{2.078017in}}{\pgfqpoint{1.954605in}{2.074744in}}{\pgfqpoint{1.948781in}{2.068920in}}%
\pgfpathcurveto{\pgfqpoint{1.942957in}{2.063096in}}{\pgfqpoint{1.939684in}{2.055196in}}{\pgfqpoint{1.939684in}{2.046960in}}%
\pgfpathcurveto{\pgfqpoint{1.939684in}{2.038724in}}{\pgfqpoint{1.942957in}{2.030824in}}{\pgfqpoint{1.948781in}{2.025000in}}%
\pgfpathcurveto{\pgfqpoint{1.954605in}{2.019176in}}{\pgfqpoint{1.962505in}{2.015904in}}{\pgfqpoint{1.970741in}{2.015904in}}%
\pgfpathclose%
\pgfusepath{stroke,fill}%
\end{pgfscope}%
\begin{pgfscope}%
\pgfpathrectangle{\pgfqpoint{0.100000in}{0.220728in}}{\pgfqpoint{3.696000in}{3.696000in}}%
\pgfusepath{clip}%
\pgfsetbuttcap%
\pgfsetroundjoin%
\definecolor{currentfill}{rgb}{0.121569,0.466667,0.705882}%
\pgfsetfillcolor{currentfill}%
\pgfsetfillopacity{0.330738}%
\pgfsetlinewidth{1.003750pt}%
\definecolor{currentstroke}{rgb}{0.121569,0.466667,0.705882}%
\pgfsetstrokecolor{currentstroke}%
\pgfsetstrokeopacity{0.330738}%
\pgfsetdash{}{0pt}%
\pgfpathmoveto{\pgfqpoint{1.971625in}{2.014916in}}%
\pgfpathcurveto{\pgfqpoint{1.979861in}{2.014916in}}{\pgfqpoint{1.987761in}{2.018189in}}{\pgfqpoint{1.993585in}{2.024013in}}%
\pgfpathcurveto{\pgfqpoint{1.999409in}{2.029836in}}{\pgfqpoint{2.002681in}{2.037737in}}{\pgfqpoint{2.002681in}{2.045973in}}%
\pgfpathcurveto{\pgfqpoint{2.002681in}{2.054209in}}{\pgfqpoint{1.999409in}{2.062109in}}{\pgfqpoint{1.993585in}{2.067933in}}%
\pgfpathcurveto{\pgfqpoint{1.987761in}{2.073757in}}{\pgfqpoint{1.979861in}{2.077029in}}{\pgfqpoint{1.971625in}{2.077029in}}%
\pgfpathcurveto{\pgfqpoint{1.963388in}{2.077029in}}{\pgfqpoint{1.955488in}{2.073757in}}{\pgfqpoint{1.949664in}{2.067933in}}%
\pgfpathcurveto{\pgfqpoint{1.943840in}{2.062109in}}{\pgfqpoint{1.940568in}{2.054209in}}{\pgfqpoint{1.940568in}{2.045973in}}%
\pgfpathcurveto{\pgfqpoint{1.940568in}{2.037737in}}{\pgfqpoint{1.943840in}{2.029836in}}{\pgfqpoint{1.949664in}{2.024013in}}%
\pgfpathcurveto{\pgfqpoint{1.955488in}{2.018189in}}{\pgfqpoint{1.963388in}{2.014916in}}{\pgfqpoint{1.971625in}{2.014916in}}%
\pgfpathclose%
\pgfusepath{stroke,fill}%
\end{pgfscope}%
\begin{pgfscope}%
\pgfpathrectangle{\pgfqpoint{0.100000in}{0.220728in}}{\pgfqpoint{3.696000in}{3.696000in}}%
\pgfusepath{clip}%
\pgfsetbuttcap%
\pgfsetroundjoin%
\definecolor{currentfill}{rgb}{0.121569,0.466667,0.705882}%
\pgfsetfillcolor{currentfill}%
\pgfsetfillopacity{0.331172}%
\pgfsetlinewidth{1.003750pt}%
\definecolor{currentstroke}{rgb}{0.121569,0.466667,0.705882}%
\pgfsetstrokecolor{currentstroke}%
\pgfsetstrokeopacity{0.331172}%
\pgfsetdash{}{0pt}%
\pgfpathmoveto{\pgfqpoint{1.847604in}{1.990141in}}%
\pgfpathcurveto{\pgfqpoint{1.855840in}{1.990141in}}{\pgfqpoint{1.863740in}{1.993414in}}{\pgfqpoint{1.869564in}{1.999238in}}%
\pgfpathcurveto{\pgfqpoint{1.875388in}{2.005062in}}{\pgfqpoint{1.878660in}{2.012962in}}{\pgfqpoint{1.878660in}{2.021198in}}%
\pgfpathcurveto{\pgfqpoint{1.878660in}{2.029434in}}{\pgfqpoint{1.875388in}{2.037334in}}{\pgfqpoint{1.869564in}{2.043158in}}%
\pgfpathcurveto{\pgfqpoint{1.863740in}{2.048982in}}{\pgfqpoint{1.855840in}{2.052254in}}{\pgfqpoint{1.847604in}{2.052254in}}%
\pgfpathcurveto{\pgfqpoint{1.839367in}{2.052254in}}{\pgfqpoint{1.831467in}{2.048982in}}{\pgfqpoint{1.825643in}{2.043158in}}%
\pgfpathcurveto{\pgfqpoint{1.819819in}{2.037334in}}{\pgfqpoint{1.816547in}{2.029434in}}{\pgfqpoint{1.816547in}{2.021198in}}%
\pgfpathcurveto{\pgfqpoint{1.816547in}{2.012962in}}{\pgfqpoint{1.819819in}{2.005062in}}{\pgfqpoint{1.825643in}{1.999238in}}%
\pgfpathcurveto{\pgfqpoint{1.831467in}{1.993414in}}{\pgfqpoint{1.839367in}{1.990141in}}{\pgfqpoint{1.847604in}{1.990141in}}%
\pgfpathclose%
\pgfusepath{stroke,fill}%
\end{pgfscope}%
\begin{pgfscope}%
\pgfpathrectangle{\pgfqpoint{0.100000in}{0.220728in}}{\pgfqpoint{3.696000in}{3.696000in}}%
\pgfusepath{clip}%
\pgfsetbuttcap%
\pgfsetroundjoin%
\definecolor{currentfill}{rgb}{0.121569,0.466667,0.705882}%
\pgfsetfillcolor{currentfill}%
\pgfsetfillopacity{0.332147}%
\pgfsetlinewidth{1.003750pt}%
\definecolor{currentstroke}{rgb}{0.121569,0.466667,0.705882}%
\pgfsetstrokecolor{currentstroke}%
\pgfsetstrokeopacity{0.332147}%
\pgfsetdash{}{0pt}%
\pgfpathmoveto{\pgfqpoint{1.843704in}{1.989370in}}%
\pgfpathcurveto{\pgfqpoint{1.851940in}{1.989370in}}{\pgfqpoint{1.859840in}{1.992642in}}{\pgfqpoint{1.865664in}{1.998466in}}%
\pgfpathcurveto{\pgfqpoint{1.871488in}{2.004290in}}{\pgfqpoint{1.874760in}{2.012190in}}{\pgfqpoint{1.874760in}{2.020427in}}%
\pgfpathcurveto{\pgfqpoint{1.874760in}{2.028663in}}{\pgfqpoint{1.871488in}{2.036563in}}{\pgfqpoint{1.865664in}{2.042387in}}%
\pgfpathcurveto{\pgfqpoint{1.859840in}{2.048211in}}{\pgfqpoint{1.851940in}{2.051483in}}{\pgfqpoint{1.843704in}{2.051483in}}%
\pgfpathcurveto{\pgfqpoint{1.835467in}{2.051483in}}{\pgfqpoint{1.827567in}{2.048211in}}{\pgfqpoint{1.821743in}{2.042387in}}%
\pgfpathcurveto{\pgfqpoint{1.815920in}{2.036563in}}{\pgfqpoint{1.812647in}{2.028663in}}{\pgfqpoint{1.812647in}{2.020427in}}%
\pgfpathcurveto{\pgfqpoint{1.812647in}{2.012190in}}{\pgfqpoint{1.815920in}{2.004290in}}{\pgfqpoint{1.821743in}{1.998466in}}%
\pgfpathcurveto{\pgfqpoint{1.827567in}{1.992642in}}{\pgfqpoint{1.835467in}{1.989370in}}{\pgfqpoint{1.843704in}{1.989370in}}%
\pgfpathclose%
\pgfusepath{stroke,fill}%
\end{pgfscope}%
\begin{pgfscope}%
\pgfpathrectangle{\pgfqpoint{0.100000in}{0.220728in}}{\pgfqpoint{3.696000in}{3.696000in}}%
\pgfusepath{clip}%
\pgfsetbuttcap%
\pgfsetroundjoin%
\definecolor{currentfill}{rgb}{0.121569,0.466667,0.705882}%
\pgfsetfillcolor{currentfill}%
\pgfsetfillopacity{0.332633}%
\pgfsetlinewidth{1.003750pt}%
\definecolor{currentstroke}{rgb}{0.121569,0.466667,0.705882}%
\pgfsetstrokecolor{currentstroke}%
\pgfsetstrokeopacity{0.332633}%
\pgfsetdash{}{0pt}%
\pgfpathmoveto{\pgfqpoint{1.973457in}{2.014270in}}%
\pgfpathcurveto{\pgfqpoint{1.981693in}{2.014270in}}{\pgfqpoint{1.989593in}{2.017542in}}{\pgfqpoint{1.995417in}{2.023366in}}%
\pgfpathcurveto{\pgfqpoint{2.001241in}{2.029190in}}{\pgfqpoint{2.004513in}{2.037090in}}{\pgfqpoint{2.004513in}{2.045326in}}%
\pgfpathcurveto{\pgfqpoint{2.004513in}{2.053563in}}{\pgfqpoint{2.001241in}{2.061463in}}{\pgfqpoint{1.995417in}{2.067287in}}%
\pgfpathcurveto{\pgfqpoint{1.989593in}{2.073111in}}{\pgfqpoint{1.981693in}{2.076383in}}{\pgfqpoint{1.973457in}{2.076383in}}%
\pgfpathcurveto{\pgfqpoint{1.965220in}{2.076383in}}{\pgfqpoint{1.957320in}{2.073111in}}{\pgfqpoint{1.951496in}{2.067287in}}%
\pgfpathcurveto{\pgfqpoint{1.945672in}{2.061463in}}{\pgfqpoint{1.942400in}{2.053563in}}{\pgfqpoint{1.942400in}{2.045326in}}%
\pgfpathcurveto{\pgfqpoint{1.942400in}{2.037090in}}{\pgfqpoint{1.945672in}{2.029190in}}{\pgfqpoint{1.951496in}{2.023366in}}%
\pgfpathcurveto{\pgfqpoint{1.957320in}{2.017542in}}{\pgfqpoint{1.965220in}{2.014270in}}{\pgfqpoint{1.973457in}{2.014270in}}%
\pgfpathclose%
\pgfusepath{stroke,fill}%
\end{pgfscope}%
\begin{pgfscope}%
\pgfpathrectangle{\pgfqpoint{0.100000in}{0.220728in}}{\pgfqpoint{3.696000in}{3.696000in}}%
\pgfusepath{clip}%
\pgfsetbuttcap%
\pgfsetroundjoin%
\definecolor{currentfill}{rgb}{0.121569,0.466667,0.705882}%
\pgfsetfillcolor{currentfill}%
\pgfsetfillopacity{0.332899}%
\pgfsetlinewidth{1.003750pt}%
\definecolor{currentstroke}{rgb}{0.121569,0.466667,0.705882}%
\pgfsetstrokecolor{currentstroke}%
\pgfsetstrokeopacity{0.332899}%
\pgfsetdash{}{0pt}%
\pgfpathmoveto{\pgfqpoint{1.840946in}{1.987201in}}%
\pgfpathcurveto{\pgfqpoint{1.849182in}{1.987201in}}{\pgfqpoint{1.857082in}{1.990473in}}{\pgfqpoint{1.862906in}{1.996297in}}%
\pgfpathcurveto{\pgfqpoint{1.868730in}{2.002121in}}{\pgfqpoint{1.872002in}{2.010021in}}{\pgfqpoint{1.872002in}{2.018257in}}%
\pgfpathcurveto{\pgfqpoint{1.872002in}{2.026493in}}{\pgfqpoint{1.868730in}{2.034394in}}{\pgfqpoint{1.862906in}{2.040217in}}%
\pgfpathcurveto{\pgfqpoint{1.857082in}{2.046041in}}{\pgfqpoint{1.849182in}{2.049314in}}{\pgfqpoint{1.840946in}{2.049314in}}%
\pgfpathcurveto{\pgfqpoint{1.832710in}{2.049314in}}{\pgfqpoint{1.824810in}{2.046041in}}{\pgfqpoint{1.818986in}{2.040217in}}%
\pgfpathcurveto{\pgfqpoint{1.813162in}{2.034394in}}{\pgfqpoint{1.809889in}{2.026493in}}{\pgfqpoint{1.809889in}{2.018257in}}%
\pgfpathcurveto{\pgfqpoint{1.809889in}{2.010021in}}{\pgfqpoint{1.813162in}{2.002121in}}{\pgfqpoint{1.818986in}{1.996297in}}%
\pgfpathcurveto{\pgfqpoint{1.824810in}{1.990473in}}{\pgfqpoint{1.832710in}{1.987201in}}{\pgfqpoint{1.840946in}{1.987201in}}%
\pgfpathclose%
\pgfusepath{stroke,fill}%
\end{pgfscope}%
\begin{pgfscope}%
\pgfpathrectangle{\pgfqpoint{0.100000in}{0.220728in}}{\pgfqpoint{3.696000in}{3.696000in}}%
\pgfusepath{clip}%
\pgfsetbuttcap%
\pgfsetroundjoin%
\definecolor{currentfill}{rgb}{0.121569,0.466667,0.705882}%
\pgfsetfillcolor{currentfill}%
\pgfsetfillopacity{0.334909}%
\pgfsetlinewidth{1.003750pt}%
\definecolor{currentstroke}{rgb}{0.121569,0.466667,0.705882}%
\pgfsetstrokecolor{currentstroke}%
\pgfsetstrokeopacity{0.334909}%
\pgfsetdash{}{0pt}%
\pgfpathmoveto{\pgfqpoint{1.836431in}{1.986985in}}%
\pgfpathcurveto{\pgfqpoint{1.844667in}{1.986985in}}{\pgfqpoint{1.852567in}{1.990258in}}{\pgfqpoint{1.858391in}{1.996082in}}%
\pgfpathcurveto{\pgfqpoint{1.864215in}{2.001906in}}{\pgfqpoint{1.867487in}{2.009806in}}{\pgfqpoint{1.867487in}{2.018042in}}%
\pgfpathcurveto{\pgfqpoint{1.867487in}{2.026278in}}{\pgfqpoint{1.864215in}{2.034178in}}{\pgfqpoint{1.858391in}{2.040002in}}%
\pgfpathcurveto{\pgfqpoint{1.852567in}{2.045826in}}{\pgfqpoint{1.844667in}{2.049098in}}{\pgfqpoint{1.836431in}{2.049098in}}%
\pgfpathcurveto{\pgfqpoint{1.828194in}{2.049098in}}{\pgfqpoint{1.820294in}{2.045826in}}{\pgfqpoint{1.814470in}{2.040002in}}%
\pgfpathcurveto{\pgfqpoint{1.808646in}{2.034178in}}{\pgfqpoint{1.805374in}{2.026278in}}{\pgfqpoint{1.805374in}{2.018042in}}%
\pgfpathcurveto{\pgfqpoint{1.805374in}{2.009806in}}{\pgfqpoint{1.808646in}{2.001906in}}{\pgfqpoint{1.814470in}{1.996082in}}%
\pgfpathcurveto{\pgfqpoint{1.820294in}{1.990258in}}{\pgfqpoint{1.828194in}{1.986985in}}{\pgfqpoint{1.836431in}{1.986985in}}%
\pgfpathclose%
\pgfusepath{stroke,fill}%
\end{pgfscope}%
\begin{pgfscope}%
\pgfpathrectangle{\pgfqpoint{0.100000in}{0.220728in}}{\pgfqpoint{3.696000in}{3.696000in}}%
\pgfusepath{clip}%
\pgfsetbuttcap%
\pgfsetroundjoin%
\definecolor{currentfill}{rgb}{0.121569,0.466667,0.705882}%
\pgfsetfillcolor{currentfill}%
\pgfsetfillopacity{0.335299}%
\pgfsetlinewidth{1.003750pt}%
\definecolor{currentstroke}{rgb}{0.121569,0.466667,0.705882}%
\pgfsetstrokecolor{currentstroke}%
\pgfsetstrokeopacity{0.335299}%
\pgfsetdash{}{0pt}%
\pgfpathmoveto{\pgfqpoint{1.976093in}{2.015949in}}%
\pgfpathcurveto{\pgfqpoint{1.984330in}{2.015949in}}{\pgfqpoint{1.992230in}{2.019221in}}{\pgfqpoint{1.998054in}{2.025045in}}%
\pgfpathcurveto{\pgfqpoint{2.003878in}{2.030869in}}{\pgfqpoint{2.007150in}{2.038769in}}{\pgfqpoint{2.007150in}{2.047006in}}%
\pgfpathcurveto{\pgfqpoint{2.007150in}{2.055242in}}{\pgfqpoint{2.003878in}{2.063142in}}{\pgfqpoint{1.998054in}{2.068966in}}%
\pgfpathcurveto{\pgfqpoint{1.992230in}{2.074790in}}{\pgfqpoint{1.984330in}{2.078062in}}{\pgfqpoint{1.976093in}{2.078062in}}%
\pgfpathcurveto{\pgfqpoint{1.967857in}{2.078062in}}{\pgfqpoint{1.959957in}{2.074790in}}{\pgfqpoint{1.954133in}{2.068966in}}%
\pgfpathcurveto{\pgfqpoint{1.948309in}{2.063142in}}{\pgfqpoint{1.945037in}{2.055242in}}{\pgfqpoint{1.945037in}{2.047006in}}%
\pgfpathcurveto{\pgfqpoint{1.945037in}{2.038769in}}{\pgfqpoint{1.948309in}{2.030869in}}{\pgfqpoint{1.954133in}{2.025045in}}%
\pgfpathcurveto{\pgfqpoint{1.959957in}{2.019221in}}{\pgfqpoint{1.967857in}{2.015949in}}{\pgfqpoint{1.976093in}{2.015949in}}%
\pgfpathclose%
\pgfusepath{stroke,fill}%
\end{pgfscope}%
\begin{pgfscope}%
\pgfpathrectangle{\pgfqpoint{0.100000in}{0.220728in}}{\pgfqpoint{3.696000in}{3.696000in}}%
\pgfusepath{clip}%
\pgfsetbuttcap%
\pgfsetroundjoin%
\definecolor{currentfill}{rgb}{0.121569,0.466667,0.705882}%
\pgfsetfillcolor{currentfill}%
\pgfsetfillopacity{0.335752}%
\pgfsetlinewidth{1.003750pt}%
\definecolor{currentstroke}{rgb}{0.121569,0.466667,0.705882}%
\pgfsetstrokecolor{currentstroke}%
\pgfsetstrokeopacity{0.335752}%
\pgfsetdash{}{0pt}%
\pgfpathmoveto{\pgfqpoint{1.832781in}{1.984042in}}%
\pgfpathcurveto{\pgfqpoint{1.841017in}{1.984042in}}{\pgfqpoint{1.848917in}{1.987314in}}{\pgfqpoint{1.854741in}{1.993138in}}%
\pgfpathcurveto{\pgfqpoint{1.860565in}{1.998962in}}{\pgfqpoint{1.863837in}{2.006862in}}{\pgfqpoint{1.863837in}{2.015099in}}%
\pgfpathcurveto{\pgfqpoint{1.863837in}{2.023335in}}{\pgfqpoint{1.860565in}{2.031235in}}{\pgfqpoint{1.854741in}{2.037059in}}%
\pgfpathcurveto{\pgfqpoint{1.848917in}{2.042883in}}{\pgfqpoint{1.841017in}{2.046155in}}{\pgfqpoint{1.832781in}{2.046155in}}%
\pgfpathcurveto{\pgfqpoint{1.824544in}{2.046155in}}{\pgfqpoint{1.816644in}{2.042883in}}{\pgfqpoint{1.810820in}{2.037059in}}%
\pgfpathcurveto{\pgfqpoint{1.804996in}{2.031235in}}{\pgfqpoint{1.801724in}{2.023335in}}{\pgfqpoint{1.801724in}{2.015099in}}%
\pgfpathcurveto{\pgfqpoint{1.801724in}{2.006862in}}{\pgfqpoint{1.804996in}{1.998962in}}{\pgfqpoint{1.810820in}{1.993138in}}%
\pgfpathcurveto{\pgfqpoint{1.816644in}{1.987314in}}{\pgfqpoint{1.824544in}{1.984042in}}{\pgfqpoint{1.832781in}{1.984042in}}%
\pgfpathclose%
\pgfusepath{stroke,fill}%
\end{pgfscope}%
\begin{pgfscope}%
\pgfpathrectangle{\pgfqpoint{0.100000in}{0.220728in}}{\pgfqpoint{3.696000in}{3.696000in}}%
\pgfusepath{clip}%
\pgfsetbuttcap%
\pgfsetroundjoin%
\definecolor{currentfill}{rgb}{0.121569,0.466667,0.705882}%
\pgfsetfillcolor{currentfill}%
\pgfsetfillopacity{0.336583}%
\pgfsetlinewidth{1.003750pt}%
\definecolor{currentstroke}{rgb}{0.121569,0.466667,0.705882}%
\pgfsetstrokecolor{currentstroke}%
\pgfsetstrokeopacity{0.336583}%
\pgfsetdash{}{0pt}%
\pgfpathmoveto{\pgfqpoint{1.829734in}{1.982779in}}%
\pgfpathcurveto{\pgfqpoint{1.837971in}{1.982779in}}{\pgfqpoint{1.845871in}{1.986052in}}{\pgfqpoint{1.851695in}{1.991876in}}%
\pgfpathcurveto{\pgfqpoint{1.857519in}{1.997700in}}{\pgfqpoint{1.860791in}{2.005600in}}{\pgfqpoint{1.860791in}{2.013836in}}%
\pgfpathcurveto{\pgfqpoint{1.860791in}{2.022072in}}{\pgfqpoint{1.857519in}{2.029972in}}{\pgfqpoint{1.851695in}{2.035796in}}%
\pgfpathcurveto{\pgfqpoint{1.845871in}{2.041620in}}{\pgfqpoint{1.837971in}{2.044892in}}{\pgfqpoint{1.829734in}{2.044892in}}%
\pgfpathcurveto{\pgfqpoint{1.821498in}{2.044892in}}{\pgfqpoint{1.813598in}{2.041620in}}{\pgfqpoint{1.807774in}{2.035796in}}%
\pgfpathcurveto{\pgfqpoint{1.801950in}{2.029972in}}{\pgfqpoint{1.798678in}{2.022072in}}{\pgfqpoint{1.798678in}{2.013836in}}%
\pgfpathcurveto{\pgfqpoint{1.798678in}{2.005600in}}{\pgfqpoint{1.801950in}{1.997700in}}{\pgfqpoint{1.807774in}{1.991876in}}%
\pgfpathcurveto{\pgfqpoint{1.813598in}{1.986052in}}{\pgfqpoint{1.821498in}{1.982779in}}{\pgfqpoint{1.829734in}{1.982779in}}%
\pgfpathclose%
\pgfusepath{stroke,fill}%
\end{pgfscope}%
\begin{pgfscope}%
\pgfpathrectangle{\pgfqpoint{0.100000in}{0.220728in}}{\pgfqpoint{3.696000in}{3.696000in}}%
\pgfusepath{clip}%
\pgfsetbuttcap%
\pgfsetroundjoin%
\definecolor{currentfill}{rgb}{0.121569,0.466667,0.705882}%
\pgfsetfillcolor{currentfill}%
\pgfsetfillopacity{0.338336}%
\pgfsetlinewidth{1.003750pt}%
\definecolor{currentstroke}{rgb}{0.121569,0.466667,0.705882}%
\pgfsetstrokecolor{currentstroke}%
\pgfsetstrokeopacity{0.338336}%
\pgfsetdash{}{0pt}%
\pgfpathmoveto{\pgfqpoint{1.977742in}{2.013886in}}%
\pgfpathcurveto{\pgfqpoint{1.985978in}{2.013886in}}{\pgfqpoint{1.993878in}{2.017158in}}{\pgfqpoint{1.999702in}{2.022982in}}%
\pgfpathcurveto{\pgfqpoint{2.005526in}{2.028806in}}{\pgfqpoint{2.008798in}{2.036706in}}{\pgfqpoint{2.008798in}{2.044943in}}%
\pgfpathcurveto{\pgfqpoint{2.008798in}{2.053179in}}{\pgfqpoint{2.005526in}{2.061079in}}{\pgfqpoint{1.999702in}{2.066903in}}%
\pgfpathcurveto{\pgfqpoint{1.993878in}{2.072727in}}{\pgfqpoint{1.985978in}{2.075999in}}{\pgfqpoint{1.977742in}{2.075999in}}%
\pgfpathcurveto{\pgfqpoint{1.969506in}{2.075999in}}{\pgfqpoint{1.961606in}{2.072727in}}{\pgfqpoint{1.955782in}{2.066903in}}%
\pgfpathcurveto{\pgfqpoint{1.949958in}{2.061079in}}{\pgfqpoint{1.946685in}{2.053179in}}{\pgfqpoint{1.946685in}{2.044943in}}%
\pgfpathcurveto{\pgfqpoint{1.946685in}{2.036706in}}{\pgfqpoint{1.949958in}{2.028806in}}{\pgfqpoint{1.955782in}{2.022982in}}%
\pgfpathcurveto{\pgfqpoint{1.961606in}{2.017158in}}{\pgfqpoint{1.969506in}{2.013886in}}{\pgfqpoint{1.977742in}{2.013886in}}%
\pgfpathclose%
\pgfusepath{stroke,fill}%
\end{pgfscope}%
\begin{pgfscope}%
\pgfpathrectangle{\pgfqpoint{0.100000in}{0.220728in}}{\pgfqpoint{3.696000in}{3.696000in}}%
\pgfusepath{clip}%
\pgfsetbuttcap%
\pgfsetroundjoin%
\definecolor{currentfill}{rgb}{0.121569,0.466667,0.705882}%
\pgfsetfillcolor{currentfill}%
\pgfsetfillopacity{0.338656}%
\pgfsetlinewidth{1.003750pt}%
\definecolor{currentstroke}{rgb}{0.121569,0.466667,0.705882}%
\pgfsetstrokecolor{currentstroke}%
\pgfsetstrokeopacity{0.338656}%
\pgfsetdash{}{0pt}%
\pgfpathmoveto{\pgfqpoint{1.825639in}{1.982295in}}%
\pgfpathcurveto{\pgfqpoint{1.833876in}{1.982295in}}{\pgfqpoint{1.841776in}{1.985567in}}{\pgfqpoint{1.847600in}{1.991391in}}%
\pgfpathcurveto{\pgfqpoint{1.853423in}{1.997215in}}{\pgfqpoint{1.856696in}{2.005115in}}{\pgfqpoint{1.856696in}{2.013351in}}%
\pgfpathcurveto{\pgfqpoint{1.856696in}{2.021587in}}{\pgfqpoint{1.853423in}{2.029487in}}{\pgfqpoint{1.847600in}{2.035311in}}%
\pgfpathcurveto{\pgfqpoint{1.841776in}{2.041135in}}{\pgfqpoint{1.833876in}{2.044408in}}{\pgfqpoint{1.825639in}{2.044408in}}%
\pgfpathcurveto{\pgfqpoint{1.817403in}{2.044408in}}{\pgfqpoint{1.809503in}{2.041135in}}{\pgfqpoint{1.803679in}{2.035311in}}%
\pgfpathcurveto{\pgfqpoint{1.797855in}{2.029487in}}{\pgfqpoint{1.794583in}{2.021587in}}{\pgfqpoint{1.794583in}{2.013351in}}%
\pgfpathcurveto{\pgfqpoint{1.794583in}{2.005115in}}{\pgfqpoint{1.797855in}{1.997215in}}{\pgfqpoint{1.803679in}{1.991391in}}%
\pgfpathcurveto{\pgfqpoint{1.809503in}{1.985567in}}{\pgfqpoint{1.817403in}{1.982295in}}{\pgfqpoint{1.825639in}{1.982295in}}%
\pgfpathclose%
\pgfusepath{stroke,fill}%
\end{pgfscope}%
\begin{pgfscope}%
\pgfpathrectangle{\pgfqpoint{0.100000in}{0.220728in}}{\pgfqpoint{3.696000in}{3.696000in}}%
\pgfusepath{clip}%
\pgfsetbuttcap%
\pgfsetroundjoin%
\definecolor{currentfill}{rgb}{0.121569,0.466667,0.705882}%
\pgfsetfillcolor{currentfill}%
\pgfsetfillopacity{0.339524}%
\pgfsetlinewidth{1.003750pt}%
\definecolor{currentstroke}{rgb}{0.121569,0.466667,0.705882}%
\pgfsetstrokecolor{currentstroke}%
\pgfsetstrokeopacity{0.339524}%
\pgfsetdash{}{0pt}%
\pgfpathmoveto{\pgfqpoint{1.821664in}{1.979823in}}%
\pgfpathcurveto{\pgfqpoint{1.829900in}{1.979823in}}{\pgfqpoint{1.837800in}{1.983095in}}{\pgfqpoint{1.843624in}{1.988919in}}%
\pgfpathcurveto{\pgfqpoint{1.849448in}{1.994743in}}{\pgfqpoint{1.852720in}{2.002643in}}{\pgfqpoint{1.852720in}{2.010879in}}%
\pgfpathcurveto{\pgfqpoint{1.852720in}{2.019116in}}{\pgfqpoint{1.849448in}{2.027016in}}{\pgfqpoint{1.843624in}{2.032840in}}%
\pgfpathcurveto{\pgfqpoint{1.837800in}{2.038664in}}{\pgfqpoint{1.829900in}{2.041936in}}{\pgfqpoint{1.821664in}{2.041936in}}%
\pgfpathcurveto{\pgfqpoint{1.813428in}{2.041936in}}{\pgfqpoint{1.805528in}{2.038664in}}{\pgfqpoint{1.799704in}{2.032840in}}%
\pgfpathcurveto{\pgfqpoint{1.793880in}{2.027016in}}{\pgfqpoint{1.790607in}{2.019116in}}{\pgfqpoint{1.790607in}{2.010879in}}%
\pgfpathcurveto{\pgfqpoint{1.790607in}{2.002643in}}{\pgfqpoint{1.793880in}{1.994743in}}{\pgfqpoint{1.799704in}{1.988919in}}%
\pgfpathcurveto{\pgfqpoint{1.805528in}{1.983095in}}{\pgfqpoint{1.813428in}{1.979823in}}{\pgfqpoint{1.821664in}{1.979823in}}%
\pgfpathclose%
\pgfusepath{stroke,fill}%
\end{pgfscope}%
\begin{pgfscope}%
\pgfpathrectangle{\pgfqpoint{0.100000in}{0.220728in}}{\pgfqpoint{3.696000in}{3.696000in}}%
\pgfusepath{clip}%
\pgfsetbuttcap%
\pgfsetroundjoin%
\definecolor{currentfill}{rgb}{0.121569,0.466667,0.705882}%
\pgfsetfillcolor{currentfill}%
\pgfsetfillopacity{0.340246}%
\pgfsetlinewidth{1.003750pt}%
\definecolor{currentstroke}{rgb}{0.121569,0.466667,0.705882}%
\pgfsetstrokecolor{currentstroke}%
\pgfsetstrokeopacity{0.340246}%
\pgfsetdash{}{0pt}%
\pgfpathmoveto{\pgfqpoint{1.819047in}{1.977823in}}%
\pgfpathcurveto{\pgfqpoint{1.827283in}{1.977823in}}{\pgfqpoint{1.835183in}{1.981095in}}{\pgfqpoint{1.841007in}{1.986919in}}%
\pgfpathcurveto{\pgfqpoint{1.846831in}{1.992743in}}{\pgfqpoint{1.850103in}{2.000643in}}{\pgfqpoint{1.850103in}{2.008879in}}%
\pgfpathcurveto{\pgfqpoint{1.850103in}{2.017116in}}{\pgfqpoint{1.846831in}{2.025016in}}{\pgfqpoint{1.841007in}{2.030840in}}%
\pgfpathcurveto{\pgfqpoint{1.835183in}{2.036664in}}{\pgfqpoint{1.827283in}{2.039936in}}{\pgfqpoint{1.819047in}{2.039936in}}%
\pgfpathcurveto{\pgfqpoint{1.810810in}{2.039936in}}{\pgfqpoint{1.802910in}{2.036664in}}{\pgfqpoint{1.797086in}{2.030840in}}%
\pgfpathcurveto{\pgfqpoint{1.791262in}{2.025016in}}{\pgfqpoint{1.787990in}{2.017116in}}{\pgfqpoint{1.787990in}{2.008879in}}%
\pgfpathcurveto{\pgfqpoint{1.787990in}{2.000643in}}{\pgfqpoint{1.791262in}{1.992743in}}{\pgfqpoint{1.797086in}{1.986919in}}%
\pgfpathcurveto{\pgfqpoint{1.802910in}{1.981095in}}{\pgfqpoint{1.810810in}{1.977823in}}{\pgfqpoint{1.819047in}{1.977823in}}%
\pgfpathclose%
\pgfusepath{stroke,fill}%
\end{pgfscope}%
\begin{pgfscope}%
\pgfpathrectangle{\pgfqpoint{0.100000in}{0.220728in}}{\pgfqpoint{3.696000in}{3.696000in}}%
\pgfusepath{clip}%
\pgfsetbuttcap%
\pgfsetroundjoin%
\definecolor{currentfill}{rgb}{0.121569,0.466667,0.705882}%
\pgfsetfillcolor{currentfill}%
\pgfsetfillopacity{0.341165}%
\pgfsetlinewidth{1.003750pt}%
\definecolor{currentstroke}{rgb}{0.121569,0.466667,0.705882}%
\pgfsetstrokecolor{currentstroke}%
\pgfsetstrokeopacity{0.341165}%
\pgfsetdash{}{0pt}%
\pgfpathmoveto{\pgfqpoint{1.980262in}{2.008591in}}%
\pgfpathcurveto{\pgfqpoint{1.988498in}{2.008591in}}{\pgfqpoint{1.996398in}{2.011863in}}{\pgfqpoint{2.002222in}{2.017687in}}%
\pgfpathcurveto{\pgfqpoint{2.008046in}{2.023511in}}{\pgfqpoint{2.011318in}{2.031411in}}{\pgfqpoint{2.011318in}{2.039647in}}%
\pgfpathcurveto{\pgfqpoint{2.011318in}{2.047884in}}{\pgfqpoint{2.008046in}{2.055784in}}{\pgfqpoint{2.002222in}{2.061608in}}%
\pgfpathcurveto{\pgfqpoint{1.996398in}{2.067432in}}{\pgfqpoint{1.988498in}{2.070704in}}{\pgfqpoint{1.980262in}{2.070704in}}%
\pgfpathcurveto{\pgfqpoint{1.972026in}{2.070704in}}{\pgfqpoint{1.964125in}{2.067432in}}{\pgfqpoint{1.958302in}{2.061608in}}%
\pgfpathcurveto{\pgfqpoint{1.952478in}{2.055784in}}{\pgfqpoint{1.949205in}{2.047884in}}{\pgfqpoint{1.949205in}{2.039647in}}%
\pgfpathcurveto{\pgfqpoint{1.949205in}{2.031411in}}{\pgfqpoint{1.952478in}{2.023511in}}{\pgfqpoint{1.958302in}{2.017687in}}%
\pgfpathcurveto{\pgfqpoint{1.964125in}{2.011863in}}{\pgfqpoint{1.972026in}{2.008591in}}{\pgfqpoint{1.980262in}{2.008591in}}%
\pgfpathclose%
\pgfusepath{stroke,fill}%
\end{pgfscope}%
\begin{pgfscope}%
\pgfpathrectangle{\pgfqpoint{0.100000in}{0.220728in}}{\pgfqpoint{3.696000in}{3.696000in}}%
\pgfusepath{clip}%
\pgfsetbuttcap%
\pgfsetroundjoin%
\definecolor{currentfill}{rgb}{0.121569,0.466667,0.705882}%
\pgfsetfillcolor{currentfill}%
\pgfsetfillopacity{0.341968}%
\pgfsetlinewidth{1.003750pt}%
\definecolor{currentstroke}{rgb}{0.121569,0.466667,0.705882}%
\pgfsetstrokecolor{currentstroke}%
\pgfsetstrokeopacity{0.341968}%
\pgfsetdash{}{0pt}%
\pgfpathmoveto{\pgfqpoint{1.815151in}{1.975870in}}%
\pgfpathcurveto{\pgfqpoint{1.823387in}{1.975870in}}{\pgfqpoint{1.831287in}{1.979142in}}{\pgfqpoint{1.837111in}{1.984966in}}%
\pgfpathcurveto{\pgfqpoint{1.842935in}{1.990790in}}{\pgfqpoint{1.846207in}{1.998690in}}{\pgfqpoint{1.846207in}{2.006927in}}%
\pgfpathcurveto{\pgfqpoint{1.846207in}{2.015163in}}{\pgfqpoint{1.842935in}{2.023063in}}{\pgfqpoint{1.837111in}{2.028887in}}%
\pgfpathcurveto{\pgfqpoint{1.831287in}{2.034711in}}{\pgfqpoint{1.823387in}{2.037983in}}{\pgfqpoint{1.815151in}{2.037983in}}%
\pgfpathcurveto{\pgfqpoint{1.806915in}{2.037983in}}{\pgfqpoint{1.799015in}{2.034711in}}{\pgfqpoint{1.793191in}{2.028887in}}%
\pgfpathcurveto{\pgfqpoint{1.787367in}{2.023063in}}{\pgfqpoint{1.784094in}{2.015163in}}{\pgfqpoint{1.784094in}{2.006927in}}%
\pgfpathcurveto{\pgfqpoint{1.784094in}{1.998690in}}{\pgfqpoint{1.787367in}{1.990790in}}{\pgfqpoint{1.793191in}{1.984966in}}%
\pgfpathcurveto{\pgfqpoint{1.799015in}{1.979142in}}{\pgfqpoint{1.806915in}{1.975870in}}{\pgfqpoint{1.815151in}{1.975870in}}%
\pgfpathclose%
\pgfusepath{stroke,fill}%
\end{pgfscope}%
\begin{pgfscope}%
\pgfpathrectangle{\pgfqpoint{0.100000in}{0.220728in}}{\pgfqpoint{3.696000in}{3.696000in}}%
\pgfusepath{clip}%
\pgfsetbuttcap%
\pgfsetroundjoin%
\definecolor{currentfill}{rgb}{0.121569,0.466667,0.705882}%
\pgfsetfillcolor{currentfill}%
\pgfsetfillopacity{0.342666}%
\pgfsetlinewidth{1.003750pt}%
\definecolor{currentstroke}{rgb}{0.121569,0.466667,0.705882}%
\pgfsetstrokecolor{currentstroke}%
\pgfsetstrokeopacity{0.342666}%
\pgfsetdash{}{0pt}%
\pgfpathmoveto{\pgfqpoint{1.812166in}{1.973909in}}%
\pgfpathcurveto{\pgfqpoint{1.820402in}{1.973909in}}{\pgfqpoint{1.828302in}{1.977182in}}{\pgfqpoint{1.834126in}{1.983006in}}%
\pgfpathcurveto{\pgfqpoint{1.839950in}{1.988830in}}{\pgfqpoint{1.843223in}{1.996730in}}{\pgfqpoint{1.843223in}{2.004966in}}%
\pgfpathcurveto{\pgfqpoint{1.843223in}{2.013202in}}{\pgfqpoint{1.839950in}{2.021102in}}{\pgfqpoint{1.834126in}{2.026926in}}%
\pgfpathcurveto{\pgfqpoint{1.828302in}{2.032750in}}{\pgfqpoint{1.820402in}{2.036022in}}{\pgfqpoint{1.812166in}{2.036022in}}%
\pgfpathcurveto{\pgfqpoint{1.803930in}{2.036022in}}{\pgfqpoint{1.796030in}{2.032750in}}{\pgfqpoint{1.790206in}{2.026926in}}%
\pgfpathcurveto{\pgfqpoint{1.784382in}{2.021102in}}{\pgfqpoint{1.781110in}{2.013202in}}{\pgfqpoint{1.781110in}{2.004966in}}%
\pgfpathcurveto{\pgfqpoint{1.781110in}{1.996730in}}{\pgfqpoint{1.784382in}{1.988830in}}{\pgfqpoint{1.790206in}{1.983006in}}%
\pgfpathcurveto{\pgfqpoint{1.796030in}{1.977182in}}{\pgfqpoint{1.803930in}{1.973909in}}{\pgfqpoint{1.812166in}{1.973909in}}%
\pgfpathclose%
\pgfusepath{stroke,fill}%
\end{pgfscope}%
\begin{pgfscope}%
\pgfpathrectangle{\pgfqpoint{0.100000in}{0.220728in}}{\pgfqpoint{3.696000in}{3.696000in}}%
\pgfusepath{clip}%
\pgfsetbuttcap%
\pgfsetroundjoin%
\definecolor{currentfill}{rgb}{0.121569,0.466667,0.705882}%
\pgfsetfillcolor{currentfill}%
\pgfsetfillopacity{0.344276}%
\pgfsetlinewidth{1.003750pt}%
\definecolor{currentstroke}{rgb}{0.121569,0.466667,0.705882}%
\pgfsetstrokecolor{currentstroke}%
\pgfsetstrokeopacity{0.344276}%
\pgfsetdash{}{0pt}%
\pgfpathmoveto{\pgfqpoint{1.807253in}{1.971812in}}%
\pgfpathcurveto{\pgfqpoint{1.815489in}{1.971812in}}{\pgfqpoint{1.823389in}{1.975084in}}{\pgfqpoint{1.829213in}{1.980908in}}%
\pgfpathcurveto{\pgfqpoint{1.835037in}{1.986732in}}{\pgfqpoint{1.838310in}{1.994632in}}{\pgfqpoint{1.838310in}{2.002868in}}%
\pgfpathcurveto{\pgfqpoint{1.838310in}{2.011105in}}{\pgfqpoint{1.835037in}{2.019005in}}{\pgfqpoint{1.829213in}{2.024829in}}%
\pgfpathcurveto{\pgfqpoint{1.823389in}{2.030653in}}{\pgfqpoint{1.815489in}{2.033925in}}{\pgfqpoint{1.807253in}{2.033925in}}%
\pgfpathcurveto{\pgfqpoint{1.799017in}{2.033925in}}{\pgfqpoint{1.791117in}{2.030653in}}{\pgfqpoint{1.785293in}{2.024829in}}%
\pgfpathcurveto{\pgfqpoint{1.779469in}{2.019005in}}{\pgfqpoint{1.776197in}{2.011105in}}{\pgfqpoint{1.776197in}{2.002868in}}%
\pgfpathcurveto{\pgfqpoint{1.776197in}{1.994632in}}{\pgfqpoint{1.779469in}{1.986732in}}{\pgfqpoint{1.785293in}{1.980908in}}%
\pgfpathcurveto{\pgfqpoint{1.791117in}{1.975084in}}{\pgfqpoint{1.799017in}{1.971812in}}{\pgfqpoint{1.807253in}{1.971812in}}%
\pgfpathclose%
\pgfusepath{stroke,fill}%
\end{pgfscope}%
\begin{pgfscope}%
\pgfpathrectangle{\pgfqpoint{0.100000in}{0.220728in}}{\pgfqpoint{3.696000in}{3.696000in}}%
\pgfusepath{clip}%
\pgfsetbuttcap%
\pgfsetroundjoin%
\definecolor{currentfill}{rgb}{0.121569,0.466667,0.705882}%
\pgfsetfillcolor{currentfill}%
\pgfsetfillopacity{0.345437}%
\pgfsetlinewidth{1.003750pt}%
\definecolor{currentstroke}{rgb}{0.121569,0.466667,0.705882}%
\pgfsetstrokecolor{currentstroke}%
\pgfsetstrokeopacity{0.345437}%
\pgfsetdash{}{0pt}%
\pgfpathmoveto{\pgfqpoint{1.984160in}{2.010433in}}%
\pgfpathcurveto{\pgfqpoint{1.992396in}{2.010433in}}{\pgfqpoint{2.000296in}{2.013706in}}{\pgfqpoint{2.006120in}{2.019530in}}%
\pgfpathcurveto{\pgfqpoint{2.011944in}{2.025354in}}{\pgfqpoint{2.015217in}{2.033254in}}{\pgfqpoint{2.015217in}{2.041490in}}%
\pgfpathcurveto{\pgfqpoint{2.015217in}{2.049726in}}{\pgfqpoint{2.011944in}{2.057626in}}{\pgfqpoint{2.006120in}{2.063450in}}%
\pgfpathcurveto{\pgfqpoint{2.000296in}{2.069274in}}{\pgfqpoint{1.992396in}{2.072546in}}{\pgfqpoint{1.984160in}{2.072546in}}%
\pgfpathcurveto{\pgfqpoint{1.975924in}{2.072546in}}{\pgfqpoint{1.968024in}{2.069274in}}{\pgfqpoint{1.962200in}{2.063450in}}%
\pgfpathcurveto{\pgfqpoint{1.956376in}{2.057626in}}{\pgfqpoint{1.953104in}{2.049726in}}{\pgfqpoint{1.953104in}{2.041490in}}%
\pgfpathcurveto{\pgfqpoint{1.953104in}{2.033254in}}{\pgfqpoint{1.956376in}{2.025354in}}{\pgfqpoint{1.962200in}{2.019530in}}%
\pgfpathcurveto{\pgfqpoint{1.968024in}{2.013706in}}{\pgfqpoint{1.975924in}{2.010433in}}{\pgfqpoint{1.984160in}{2.010433in}}%
\pgfpathclose%
\pgfusepath{stroke,fill}%
\end{pgfscope}%
\begin{pgfscope}%
\pgfpathrectangle{\pgfqpoint{0.100000in}{0.220728in}}{\pgfqpoint{3.696000in}{3.696000in}}%
\pgfusepath{clip}%
\pgfsetbuttcap%
\pgfsetroundjoin%
\definecolor{currentfill}{rgb}{0.121569,0.466667,0.705882}%
\pgfsetfillcolor{currentfill}%
\pgfsetfillopacity{0.347462}%
\pgfsetlinewidth{1.003750pt}%
\definecolor{currentstroke}{rgb}{0.121569,0.466667,0.705882}%
\pgfsetstrokecolor{currentstroke}%
\pgfsetstrokeopacity{0.347462}%
\pgfsetdash{}{0pt}%
\pgfpathmoveto{\pgfqpoint{1.799501in}{1.968130in}}%
\pgfpathcurveto{\pgfqpoint{1.807737in}{1.968130in}}{\pgfqpoint{1.815637in}{1.971402in}}{\pgfqpoint{1.821461in}{1.977226in}}%
\pgfpathcurveto{\pgfqpoint{1.827285in}{1.983050in}}{\pgfqpoint{1.830558in}{1.990950in}}{\pgfqpoint{1.830558in}{1.999186in}}%
\pgfpathcurveto{\pgfqpoint{1.830558in}{2.007422in}}{\pgfqpoint{1.827285in}{2.015322in}}{\pgfqpoint{1.821461in}{2.021146in}}%
\pgfpathcurveto{\pgfqpoint{1.815637in}{2.026970in}}{\pgfqpoint{1.807737in}{2.030243in}}{\pgfqpoint{1.799501in}{2.030243in}}%
\pgfpathcurveto{\pgfqpoint{1.791265in}{2.030243in}}{\pgfqpoint{1.783365in}{2.026970in}}{\pgfqpoint{1.777541in}{2.021146in}}%
\pgfpathcurveto{\pgfqpoint{1.771717in}{2.015322in}}{\pgfqpoint{1.768445in}{2.007422in}}{\pgfqpoint{1.768445in}{1.999186in}}%
\pgfpathcurveto{\pgfqpoint{1.768445in}{1.990950in}}{\pgfqpoint{1.771717in}{1.983050in}}{\pgfqpoint{1.777541in}{1.977226in}}%
\pgfpathcurveto{\pgfqpoint{1.783365in}{1.971402in}}{\pgfqpoint{1.791265in}{1.968130in}}{\pgfqpoint{1.799501in}{1.968130in}}%
\pgfpathclose%
\pgfusepath{stroke,fill}%
\end{pgfscope}%
\begin{pgfscope}%
\pgfpathrectangle{\pgfqpoint{0.100000in}{0.220728in}}{\pgfqpoint{3.696000in}{3.696000in}}%
\pgfusepath{clip}%
\pgfsetbuttcap%
\pgfsetroundjoin%
\definecolor{currentfill}{rgb}{0.121569,0.466667,0.705882}%
\pgfsetfillcolor{currentfill}%
\pgfsetfillopacity{0.348706}%
\pgfsetlinewidth{1.003750pt}%
\definecolor{currentstroke}{rgb}{0.121569,0.466667,0.705882}%
\pgfsetstrokecolor{currentstroke}%
\pgfsetstrokeopacity{0.348706}%
\pgfsetdash{}{0pt}%
\pgfpathmoveto{\pgfqpoint{1.988033in}{1.998933in}}%
\pgfpathcurveto{\pgfqpoint{1.996269in}{1.998933in}}{\pgfqpoint{2.004169in}{2.002205in}}{\pgfqpoint{2.009993in}{2.008029in}}%
\pgfpathcurveto{\pgfqpoint{2.015817in}{2.013853in}}{\pgfqpoint{2.019089in}{2.021753in}}{\pgfqpoint{2.019089in}{2.029989in}}%
\pgfpathcurveto{\pgfqpoint{2.019089in}{2.038225in}}{\pgfqpoint{2.015817in}{2.046125in}}{\pgfqpoint{2.009993in}{2.051949in}}%
\pgfpathcurveto{\pgfqpoint{2.004169in}{2.057773in}}{\pgfqpoint{1.996269in}{2.061046in}}{\pgfqpoint{1.988033in}{2.061046in}}%
\pgfpathcurveto{\pgfqpoint{1.979797in}{2.061046in}}{\pgfqpoint{1.971897in}{2.057773in}}{\pgfqpoint{1.966073in}{2.051949in}}%
\pgfpathcurveto{\pgfqpoint{1.960249in}{2.046125in}}{\pgfqpoint{1.956976in}{2.038225in}}{\pgfqpoint{1.956976in}{2.029989in}}%
\pgfpathcurveto{\pgfqpoint{1.956976in}{2.021753in}}{\pgfqpoint{1.960249in}{2.013853in}}{\pgfqpoint{1.966073in}{2.008029in}}%
\pgfpathcurveto{\pgfqpoint{1.971897in}{2.002205in}}{\pgfqpoint{1.979797in}{1.998933in}}{\pgfqpoint{1.988033in}{1.998933in}}%
\pgfpathclose%
\pgfusepath{stroke,fill}%
\end{pgfscope}%
\begin{pgfscope}%
\pgfpathrectangle{\pgfqpoint{0.100000in}{0.220728in}}{\pgfqpoint{3.696000in}{3.696000in}}%
\pgfusepath{clip}%
\pgfsetbuttcap%
\pgfsetroundjoin%
\definecolor{currentfill}{rgb}{0.121569,0.466667,0.705882}%
\pgfsetfillcolor{currentfill}%
\pgfsetfillopacity{0.348957}%
\pgfsetlinewidth{1.003750pt}%
\definecolor{currentstroke}{rgb}{0.121569,0.466667,0.705882}%
\pgfsetstrokecolor{currentstroke}%
\pgfsetstrokeopacity{0.348957}%
\pgfsetdash{}{0pt}%
\pgfpathmoveto{\pgfqpoint{1.793173in}{1.960548in}}%
\pgfpathcurveto{\pgfqpoint{1.801410in}{1.960548in}}{\pgfqpoint{1.809310in}{1.963820in}}{\pgfqpoint{1.815134in}{1.969644in}}%
\pgfpathcurveto{\pgfqpoint{1.820958in}{1.975468in}}{\pgfqpoint{1.824230in}{1.983368in}}{\pgfqpoint{1.824230in}{1.991604in}}%
\pgfpathcurveto{\pgfqpoint{1.824230in}{1.999841in}}{\pgfqpoint{1.820958in}{2.007741in}}{\pgfqpoint{1.815134in}{2.013565in}}%
\pgfpathcurveto{\pgfqpoint{1.809310in}{2.019389in}}{\pgfqpoint{1.801410in}{2.022661in}}{\pgfqpoint{1.793173in}{2.022661in}}%
\pgfpathcurveto{\pgfqpoint{1.784937in}{2.022661in}}{\pgfqpoint{1.777037in}{2.019389in}}{\pgfqpoint{1.771213in}{2.013565in}}%
\pgfpathcurveto{\pgfqpoint{1.765389in}{2.007741in}}{\pgfqpoint{1.762117in}{1.999841in}}{\pgfqpoint{1.762117in}{1.991604in}}%
\pgfpathcurveto{\pgfqpoint{1.762117in}{1.983368in}}{\pgfqpoint{1.765389in}{1.975468in}}{\pgfqpoint{1.771213in}{1.969644in}}%
\pgfpathcurveto{\pgfqpoint{1.777037in}{1.963820in}}{\pgfqpoint{1.784937in}{1.960548in}}{\pgfqpoint{1.793173in}{1.960548in}}%
\pgfpathclose%
\pgfusepath{stroke,fill}%
\end{pgfscope}%
\begin{pgfscope}%
\pgfpathrectangle{\pgfqpoint{0.100000in}{0.220728in}}{\pgfqpoint{3.696000in}{3.696000in}}%
\pgfusepath{clip}%
\pgfsetbuttcap%
\pgfsetroundjoin%
\definecolor{currentfill}{rgb}{0.121569,0.466667,0.705882}%
\pgfsetfillcolor{currentfill}%
\pgfsetfillopacity{0.350005}%
\pgfsetlinewidth{1.003750pt}%
\definecolor{currentstroke}{rgb}{0.121569,0.466667,0.705882}%
\pgfsetstrokecolor{currentstroke}%
\pgfsetstrokeopacity{0.350005}%
\pgfsetdash{}{0pt}%
\pgfpathmoveto{\pgfqpoint{1.789883in}{1.958832in}}%
\pgfpathcurveto{\pgfqpoint{1.798119in}{1.958832in}}{\pgfqpoint{1.806019in}{1.962105in}}{\pgfqpoint{1.811843in}{1.967929in}}%
\pgfpathcurveto{\pgfqpoint{1.817667in}{1.973753in}}{\pgfqpoint{1.820939in}{1.981653in}}{\pgfqpoint{1.820939in}{1.989889in}}%
\pgfpathcurveto{\pgfqpoint{1.820939in}{1.998125in}}{\pgfqpoint{1.817667in}{2.006025in}}{\pgfqpoint{1.811843in}{2.011849in}}%
\pgfpathcurveto{\pgfqpoint{1.806019in}{2.017673in}}{\pgfqpoint{1.798119in}{2.020945in}}{\pgfqpoint{1.789883in}{2.020945in}}%
\pgfpathcurveto{\pgfqpoint{1.781646in}{2.020945in}}{\pgfqpoint{1.773746in}{2.017673in}}{\pgfqpoint{1.767922in}{2.011849in}}%
\pgfpathcurveto{\pgfqpoint{1.762098in}{2.006025in}}{\pgfqpoint{1.758826in}{1.998125in}}{\pgfqpoint{1.758826in}{1.989889in}}%
\pgfpathcurveto{\pgfqpoint{1.758826in}{1.981653in}}{\pgfqpoint{1.762098in}{1.973753in}}{\pgfqpoint{1.767922in}{1.967929in}}%
\pgfpathcurveto{\pgfqpoint{1.773746in}{1.962105in}}{\pgfqpoint{1.781646in}{1.958832in}}{\pgfqpoint{1.789883in}{1.958832in}}%
\pgfpathclose%
\pgfusepath{stroke,fill}%
\end{pgfscope}%
\begin{pgfscope}%
\pgfpathrectangle{\pgfqpoint{0.100000in}{0.220728in}}{\pgfqpoint{3.696000in}{3.696000in}}%
\pgfusepath{clip}%
\pgfsetbuttcap%
\pgfsetroundjoin%
\definecolor{currentfill}{rgb}{0.121569,0.466667,0.705882}%
\pgfsetfillcolor{currentfill}%
\pgfsetfillopacity{0.350514}%
\pgfsetlinewidth{1.003750pt}%
\definecolor{currentstroke}{rgb}{0.121569,0.466667,0.705882}%
\pgfsetstrokecolor{currentstroke}%
\pgfsetstrokeopacity{0.350514}%
\pgfsetdash{}{0pt}%
\pgfpathmoveto{\pgfqpoint{1.788171in}{1.957330in}}%
\pgfpathcurveto{\pgfqpoint{1.796408in}{1.957330in}}{\pgfqpoint{1.804308in}{1.960602in}}{\pgfqpoint{1.810132in}{1.966426in}}%
\pgfpathcurveto{\pgfqpoint{1.815956in}{1.972250in}}{\pgfqpoint{1.819228in}{1.980150in}}{\pgfqpoint{1.819228in}{1.988386in}}%
\pgfpathcurveto{\pgfqpoint{1.819228in}{1.996623in}}{\pgfqpoint{1.815956in}{2.004523in}}{\pgfqpoint{1.810132in}{2.010347in}}%
\pgfpathcurveto{\pgfqpoint{1.804308in}{2.016171in}}{\pgfqpoint{1.796408in}{2.019443in}}{\pgfqpoint{1.788171in}{2.019443in}}%
\pgfpathcurveto{\pgfqpoint{1.779935in}{2.019443in}}{\pgfqpoint{1.772035in}{2.016171in}}{\pgfqpoint{1.766211in}{2.010347in}}%
\pgfpathcurveto{\pgfqpoint{1.760387in}{2.004523in}}{\pgfqpoint{1.757115in}{1.996623in}}{\pgfqpoint{1.757115in}{1.988386in}}%
\pgfpathcurveto{\pgfqpoint{1.757115in}{1.980150in}}{\pgfqpoint{1.760387in}{1.972250in}}{\pgfqpoint{1.766211in}{1.966426in}}%
\pgfpathcurveto{\pgfqpoint{1.772035in}{1.960602in}}{\pgfqpoint{1.779935in}{1.957330in}}{\pgfqpoint{1.788171in}{1.957330in}}%
\pgfpathclose%
\pgfusepath{stroke,fill}%
\end{pgfscope}%
\begin{pgfscope}%
\pgfpathrectangle{\pgfqpoint{0.100000in}{0.220728in}}{\pgfqpoint{3.696000in}{3.696000in}}%
\pgfusepath{clip}%
\pgfsetbuttcap%
\pgfsetroundjoin%
\definecolor{currentfill}{rgb}{0.121569,0.466667,0.705882}%
\pgfsetfillcolor{currentfill}%
\pgfsetfillopacity{0.351000}%
\pgfsetlinewidth{1.003750pt}%
\definecolor{currentstroke}{rgb}{0.121569,0.466667,0.705882}%
\pgfsetstrokecolor{currentstroke}%
\pgfsetstrokeopacity{0.351000}%
\pgfsetdash{}{0pt}%
\pgfpathmoveto{\pgfqpoint{1.775567in}{1.943552in}}%
\pgfpathcurveto{\pgfqpoint{1.783803in}{1.943552in}}{\pgfqpoint{1.791703in}{1.946824in}}{\pgfqpoint{1.797527in}{1.952648in}}%
\pgfpathcurveto{\pgfqpoint{1.803351in}{1.958472in}}{\pgfqpoint{1.806623in}{1.966372in}}{\pgfqpoint{1.806623in}{1.974608in}}%
\pgfpathcurveto{\pgfqpoint{1.806623in}{1.982845in}}{\pgfqpoint{1.803351in}{1.990745in}}{\pgfqpoint{1.797527in}{1.996569in}}%
\pgfpathcurveto{\pgfqpoint{1.791703in}{2.002392in}}{\pgfqpoint{1.783803in}{2.005665in}}{\pgfqpoint{1.775567in}{2.005665in}}%
\pgfpathcurveto{\pgfqpoint{1.767331in}{2.005665in}}{\pgfqpoint{1.759431in}{2.002392in}}{\pgfqpoint{1.753607in}{1.996569in}}%
\pgfpathcurveto{\pgfqpoint{1.747783in}{1.990745in}}{\pgfqpoint{1.744510in}{1.982845in}}{\pgfqpoint{1.744510in}{1.974608in}}%
\pgfpathcurveto{\pgfqpoint{1.744510in}{1.966372in}}{\pgfqpoint{1.747783in}{1.958472in}}{\pgfqpoint{1.753607in}{1.952648in}}%
\pgfpathcurveto{\pgfqpoint{1.759431in}{1.946824in}}{\pgfqpoint{1.767331in}{1.943552in}}{\pgfqpoint{1.775567in}{1.943552in}}%
\pgfpathclose%
\pgfusepath{stroke,fill}%
\end{pgfscope}%
\begin{pgfscope}%
\pgfpathrectangle{\pgfqpoint{0.100000in}{0.220728in}}{\pgfqpoint{3.696000in}{3.696000in}}%
\pgfusepath{clip}%
\pgfsetbuttcap%
\pgfsetroundjoin%
\definecolor{currentfill}{rgb}{0.121569,0.466667,0.705882}%
\pgfsetfillcolor{currentfill}%
\pgfsetfillopacity{0.351062}%
\pgfsetlinewidth{1.003750pt}%
\definecolor{currentstroke}{rgb}{0.121569,0.466667,0.705882}%
\pgfsetstrokecolor{currentstroke}%
\pgfsetstrokeopacity{0.351062}%
\pgfsetdash{}{0pt}%
\pgfpathmoveto{\pgfqpoint{1.783806in}{1.954620in}}%
\pgfpathcurveto{\pgfqpoint{1.792042in}{1.954620in}}{\pgfqpoint{1.799942in}{1.957893in}}{\pgfqpoint{1.805766in}{1.963716in}}%
\pgfpathcurveto{\pgfqpoint{1.811590in}{1.969540in}}{\pgfqpoint{1.814862in}{1.977440in}}{\pgfqpoint{1.814862in}{1.985677in}}%
\pgfpathcurveto{\pgfqpoint{1.814862in}{1.993913in}}{\pgfqpoint{1.811590in}{2.001813in}}{\pgfqpoint{1.805766in}{2.007637in}}%
\pgfpathcurveto{\pgfqpoint{1.799942in}{2.013461in}}{\pgfqpoint{1.792042in}{2.016733in}}{\pgfqpoint{1.783806in}{2.016733in}}%
\pgfpathcurveto{\pgfqpoint{1.775570in}{2.016733in}}{\pgfqpoint{1.767669in}{2.013461in}}{\pgfqpoint{1.761846in}{2.007637in}}%
\pgfpathcurveto{\pgfqpoint{1.756022in}{2.001813in}}{\pgfqpoint{1.752749in}{1.993913in}}{\pgfqpoint{1.752749in}{1.985677in}}%
\pgfpathcurveto{\pgfqpoint{1.752749in}{1.977440in}}{\pgfqpoint{1.756022in}{1.969540in}}{\pgfqpoint{1.761846in}{1.963716in}}%
\pgfpathcurveto{\pgfqpoint{1.767669in}{1.957893in}}{\pgfqpoint{1.775570in}{1.954620in}}{\pgfqpoint{1.783806in}{1.954620in}}%
\pgfpathclose%
\pgfusepath{stroke,fill}%
\end{pgfscope}%
\begin{pgfscope}%
\pgfpathrectangle{\pgfqpoint{0.100000in}{0.220728in}}{\pgfqpoint{3.696000in}{3.696000in}}%
\pgfusepath{clip}%
\pgfsetbuttcap%
\pgfsetroundjoin%
\definecolor{currentfill}{rgb}{0.121569,0.466667,0.705882}%
\pgfsetfillcolor{currentfill}%
\pgfsetfillopacity{0.353073}%
\pgfsetlinewidth{1.003750pt}%
\definecolor{currentstroke}{rgb}{0.121569,0.466667,0.705882}%
\pgfsetstrokecolor{currentstroke}%
\pgfsetstrokeopacity{0.353073}%
\pgfsetdash{}{0pt}%
\pgfpathmoveto{\pgfqpoint{1.771818in}{1.943811in}}%
\pgfpathcurveto{\pgfqpoint{1.780054in}{1.943811in}}{\pgfqpoint{1.787954in}{1.947083in}}{\pgfqpoint{1.793778in}{1.952907in}}%
\pgfpathcurveto{\pgfqpoint{1.799602in}{1.958731in}}{\pgfqpoint{1.802875in}{1.966631in}}{\pgfqpoint{1.802875in}{1.974868in}}%
\pgfpathcurveto{\pgfqpoint{1.802875in}{1.983104in}}{\pgfqpoint{1.799602in}{1.991004in}}{\pgfqpoint{1.793778in}{1.996828in}}%
\pgfpathcurveto{\pgfqpoint{1.787954in}{2.002652in}}{\pgfqpoint{1.780054in}{2.005924in}}{\pgfqpoint{1.771818in}{2.005924in}}%
\pgfpathcurveto{\pgfqpoint{1.763582in}{2.005924in}}{\pgfqpoint{1.755682in}{2.002652in}}{\pgfqpoint{1.749858in}{1.996828in}}%
\pgfpathcurveto{\pgfqpoint{1.744034in}{1.991004in}}{\pgfqpoint{1.740762in}{1.983104in}}{\pgfqpoint{1.740762in}{1.974868in}}%
\pgfpathcurveto{\pgfqpoint{1.740762in}{1.966631in}}{\pgfqpoint{1.744034in}{1.958731in}}{\pgfqpoint{1.749858in}{1.952907in}}%
\pgfpathcurveto{\pgfqpoint{1.755682in}{1.947083in}}{\pgfqpoint{1.763582in}{1.943811in}}{\pgfqpoint{1.771818in}{1.943811in}}%
\pgfpathclose%
\pgfusepath{stroke,fill}%
\end{pgfscope}%
\begin{pgfscope}%
\pgfpathrectangle{\pgfqpoint{0.100000in}{0.220728in}}{\pgfqpoint{3.696000in}{3.696000in}}%
\pgfusepath{clip}%
\pgfsetbuttcap%
\pgfsetroundjoin%
\definecolor{currentfill}{rgb}{0.121569,0.466667,0.705882}%
\pgfsetfillcolor{currentfill}%
\pgfsetfillopacity{0.353390}%
\pgfsetlinewidth{1.003750pt}%
\definecolor{currentstroke}{rgb}{0.121569,0.466667,0.705882}%
\pgfsetstrokecolor{currentstroke}%
\pgfsetstrokeopacity{0.353390}%
\pgfsetdash{}{0pt}%
\pgfpathmoveto{\pgfqpoint{1.989544in}{1.994045in}}%
\pgfpathcurveto{\pgfqpoint{1.997780in}{1.994045in}}{\pgfqpoint{2.005680in}{1.997317in}}{\pgfqpoint{2.011504in}{2.003141in}}%
\pgfpathcurveto{\pgfqpoint{2.017328in}{2.008965in}}{\pgfqpoint{2.020600in}{2.016865in}}{\pgfqpoint{2.020600in}{2.025101in}}%
\pgfpathcurveto{\pgfqpoint{2.020600in}{2.033338in}}{\pgfqpoint{2.017328in}{2.041238in}}{\pgfqpoint{2.011504in}{2.047062in}}%
\pgfpathcurveto{\pgfqpoint{2.005680in}{2.052886in}}{\pgfqpoint{1.997780in}{2.056158in}}{\pgfqpoint{1.989544in}{2.056158in}}%
\pgfpathcurveto{\pgfqpoint{1.981308in}{2.056158in}}{\pgfqpoint{1.973408in}{2.052886in}}{\pgfqpoint{1.967584in}{2.047062in}}%
\pgfpathcurveto{\pgfqpoint{1.961760in}{2.041238in}}{\pgfqpoint{1.958487in}{2.033338in}}{\pgfqpoint{1.958487in}{2.025101in}}%
\pgfpathcurveto{\pgfqpoint{1.958487in}{2.016865in}}{\pgfqpoint{1.961760in}{2.008965in}}{\pgfqpoint{1.967584in}{2.003141in}}%
\pgfpathcurveto{\pgfqpoint{1.973408in}{1.997317in}}{\pgfqpoint{1.981308in}{1.994045in}}{\pgfqpoint{1.989544in}{1.994045in}}%
\pgfpathclose%
\pgfusepath{stroke,fill}%
\end{pgfscope}%
\begin{pgfscope}%
\pgfpathrectangle{\pgfqpoint{0.100000in}{0.220728in}}{\pgfqpoint{3.696000in}{3.696000in}}%
\pgfusepath{clip}%
\pgfsetbuttcap%
\pgfsetroundjoin%
\definecolor{currentfill}{rgb}{0.121569,0.466667,0.705882}%
\pgfsetfillcolor{currentfill}%
\pgfsetfillopacity{0.354044}%
\pgfsetlinewidth{1.003750pt}%
\definecolor{currentstroke}{rgb}{0.121569,0.466667,0.705882}%
\pgfsetstrokecolor{currentstroke}%
\pgfsetstrokeopacity{0.354044}%
\pgfsetdash{}{0pt}%
\pgfpathmoveto{\pgfqpoint{1.767351in}{1.940455in}}%
\pgfpathcurveto{\pgfqpoint{1.775587in}{1.940455in}}{\pgfqpoint{1.783487in}{1.943727in}}{\pgfqpoint{1.789311in}{1.949551in}}%
\pgfpathcurveto{\pgfqpoint{1.795135in}{1.955375in}}{\pgfqpoint{1.798407in}{1.963275in}}{\pgfqpoint{1.798407in}{1.971511in}}%
\pgfpathcurveto{\pgfqpoint{1.798407in}{1.979748in}}{\pgfqpoint{1.795135in}{1.987648in}}{\pgfqpoint{1.789311in}{1.993472in}}%
\pgfpathcurveto{\pgfqpoint{1.783487in}{1.999296in}}{\pgfqpoint{1.775587in}{2.002568in}}{\pgfqpoint{1.767351in}{2.002568in}}%
\pgfpathcurveto{\pgfqpoint{1.759115in}{2.002568in}}{\pgfqpoint{1.751215in}{1.999296in}}{\pgfqpoint{1.745391in}{1.993472in}}%
\pgfpathcurveto{\pgfqpoint{1.739567in}{1.987648in}}{\pgfqpoint{1.736294in}{1.979748in}}{\pgfqpoint{1.736294in}{1.971511in}}%
\pgfpathcurveto{\pgfqpoint{1.736294in}{1.963275in}}{\pgfqpoint{1.739567in}{1.955375in}}{\pgfqpoint{1.745391in}{1.949551in}}%
\pgfpathcurveto{\pgfqpoint{1.751215in}{1.943727in}}{\pgfqpoint{1.759115in}{1.940455in}}{\pgfqpoint{1.767351in}{1.940455in}}%
\pgfpathclose%
\pgfusepath{stroke,fill}%
\end{pgfscope}%
\begin{pgfscope}%
\pgfpathrectangle{\pgfqpoint{0.100000in}{0.220728in}}{\pgfqpoint{3.696000in}{3.696000in}}%
\pgfusepath{clip}%
\pgfsetbuttcap%
\pgfsetroundjoin%
\definecolor{currentfill}{rgb}{0.121569,0.466667,0.705882}%
\pgfsetfillcolor{currentfill}%
\pgfsetfillopacity{0.355615}%
\pgfsetlinewidth{1.003750pt}%
\definecolor{currentstroke}{rgb}{0.121569,0.466667,0.705882}%
\pgfsetstrokecolor{currentstroke}%
\pgfsetstrokeopacity{0.355615}%
\pgfsetdash{}{0pt}%
\pgfpathmoveto{\pgfqpoint{1.759566in}{1.932389in}}%
\pgfpathcurveto{\pgfqpoint{1.767802in}{1.932389in}}{\pgfqpoint{1.775702in}{1.935661in}}{\pgfqpoint{1.781526in}{1.941485in}}%
\pgfpathcurveto{\pgfqpoint{1.787350in}{1.947309in}}{\pgfqpoint{1.790622in}{1.955209in}}{\pgfqpoint{1.790622in}{1.963445in}}%
\pgfpathcurveto{\pgfqpoint{1.790622in}{1.971681in}}{\pgfqpoint{1.787350in}{1.979581in}}{\pgfqpoint{1.781526in}{1.985405in}}%
\pgfpathcurveto{\pgfqpoint{1.775702in}{1.991229in}}{\pgfqpoint{1.767802in}{1.994502in}}{\pgfqpoint{1.759566in}{1.994502in}}%
\pgfpathcurveto{\pgfqpoint{1.751329in}{1.994502in}}{\pgfqpoint{1.743429in}{1.991229in}}{\pgfqpoint{1.737606in}{1.985405in}}%
\pgfpathcurveto{\pgfqpoint{1.731782in}{1.979581in}}{\pgfqpoint{1.728509in}{1.971681in}}{\pgfqpoint{1.728509in}{1.963445in}}%
\pgfpathcurveto{\pgfqpoint{1.728509in}{1.955209in}}{\pgfqpoint{1.731782in}{1.947309in}}{\pgfqpoint{1.737606in}{1.941485in}}%
\pgfpathcurveto{\pgfqpoint{1.743429in}{1.935661in}}{\pgfqpoint{1.751329in}{1.932389in}}{\pgfqpoint{1.759566in}{1.932389in}}%
\pgfpathclose%
\pgfusepath{stroke,fill}%
\end{pgfscope}%
\begin{pgfscope}%
\pgfpathrectangle{\pgfqpoint{0.100000in}{0.220728in}}{\pgfqpoint{3.696000in}{3.696000in}}%
\pgfusepath{clip}%
\pgfsetbuttcap%
\pgfsetroundjoin%
\definecolor{currentfill}{rgb}{0.121569,0.466667,0.705882}%
\pgfsetfillcolor{currentfill}%
\pgfsetfillopacity{0.358258}%
\pgfsetlinewidth{1.003750pt}%
\definecolor{currentstroke}{rgb}{0.121569,0.466667,0.705882}%
\pgfsetstrokecolor{currentstroke}%
\pgfsetstrokeopacity{0.358258}%
\pgfsetdash{}{0pt}%
\pgfpathmoveto{\pgfqpoint{1.753328in}{1.930662in}}%
\pgfpathcurveto{\pgfqpoint{1.761565in}{1.930662in}}{\pgfqpoint{1.769465in}{1.933934in}}{\pgfqpoint{1.775289in}{1.939758in}}%
\pgfpathcurveto{\pgfqpoint{1.781112in}{1.945582in}}{\pgfqpoint{1.784385in}{1.953482in}}{\pgfqpoint{1.784385in}{1.961718in}}%
\pgfpathcurveto{\pgfqpoint{1.784385in}{1.969955in}}{\pgfqpoint{1.781112in}{1.977855in}}{\pgfqpoint{1.775289in}{1.983679in}}%
\pgfpathcurveto{\pgfqpoint{1.769465in}{1.989503in}}{\pgfqpoint{1.761565in}{1.992775in}}{\pgfqpoint{1.753328in}{1.992775in}}%
\pgfpathcurveto{\pgfqpoint{1.745092in}{1.992775in}}{\pgfqpoint{1.737192in}{1.989503in}}{\pgfqpoint{1.731368in}{1.983679in}}%
\pgfpathcurveto{\pgfqpoint{1.725544in}{1.977855in}}{\pgfqpoint{1.722272in}{1.969955in}}{\pgfqpoint{1.722272in}{1.961718in}}%
\pgfpathcurveto{\pgfqpoint{1.722272in}{1.953482in}}{\pgfqpoint{1.725544in}{1.945582in}}{\pgfqpoint{1.731368in}{1.939758in}}%
\pgfpathcurveto{\pgfqpoint{1.737192in}{1.933934in}}{\pgfqpoint{1.745092in}{1.930662in}}{\pgfqpoint{1.753328in}{1.930662in}}%
\pgfpathclose%
\pgfusepath{stroke,fill}%
\end{pgfscope}%
\begin{pgfscope}%
\pgfpathrectangle{\pgfqpoint{0.100000in}{0.220728in}}{\pgfqpoint{3.696000in}{3.696000in}}%
\pgfusepath{clip}%
\pgfsetbuttcap%
\pgfsetroundjoin%
\definecolor{currentfill}{rgb}{0.121569,0.466667,0.705882}%
\pgfsetfillcolor{currentfill}%
\pgfsetfillopacity{0.358625}%
\pgfsetlinewidth{1.003750pt}%
\definecolor{currentstroke}{rgb}{0.121569,0.466667,0.705882}%
\pgfsetstrokecolor{currentstroke}%
\pgfsetstrokeopacity{0.358625}%
\pgfsetdash{}{0pt}%
\pgfpathmoveto{\pgfqpoint{1.993204in}{1.990800in}}%
\pgfpathcurveto{\pgfqpoint{2.001440in}{1.990800in}}{\pgfqpoint{2.009340in}{1.994073in}}{\pgfqpoint{2.015164in}{1.999897in}}%
\pgfpathcurveto{\pgfqpoint{2.020988in}{2.005720in}}{\pgfqpoint{2.024261in}{2.013621in}}{\pgfqpoint{2.024261in}{2.021857in}}%
\pgfpathcurveto{\pgfqpoint{2.024261in}{2.030093in}}{\pgfqpoint{2.020988in}{2.037993in}}{\pgfqpoint{2.015164in}{2.043817in}}%
\pgfpathcurveto{\pgfqpoint{2.009340in}{2.049641in}}{\pgfqpoint{2.001440in}{2.052913in}}{\pgfqpoint{1.993204in}{2.052913in}}%
\pgfpathcurveto{\pgfqpoint{1.984968in}{2.052913in}}{\pgfqpoint{1.977068in}{2.049641in}}{\pgfqpoint{1.971244in}{2.043817in}}%
\pgfpathcurveto{\pgfqpoint{1.965420in}{2.037993in}}{\pgfqpoint{1.962148in}{2.030093in}}{\pgfqpoint{1.962148in}{2.021857in}}%
\pgfpathcurveto{\pgfqpoint{1.962148in}{2.013621in}}{\pgfqpoint{1.965420in}{2.005720in}}{\pgfqpoint{1.971244in}{1.999897in}}%
\pgfpathcurveto{\pgfqpoint{1.977068in}{1.994073in}}{\pgfqpoint{1.984968in}{1.990800in}}{\pgfqpoint{1.993204in}{1.990800in}}%
\pgfpathclose%
\pgfusepath{stroke,fill}%
\end{pgfscope}%
\begin{pgfscope}%
\pgfpathrectangle{\pgfqpoint{0.100000in}{0.220728in}}{\pgfqpoint{3.696000in}{3.696000in}}%
\pgfusepath{clip}%
\pgfsetbuttcap%
\pgfsetroundjoin%
\definecolor{currentfill}{rgb}{0.121569,0.466667,0.705882}%
\pgfsetfillcolor{currentfill}%
\pgfsetfillopacity{0.359564}%
\pgfsetlinewidth{1.003750pt}%
\definecolor{currentstroke}{rgb}{0.121569,0.466667,0.705882}%
\pgfsetstrokecolor{currentstroke}%
\pgfsetstrokeopacity{0.359564}%
\pgfsetdash{}{0pt}%
\pgfpathmoveto{\pgfqpoint{1.748243in}{1.927832in}}%
\pgfpathcurveto{\pgfqpoint{1.756480in}{1.927832in}}{\pgfqpoint{1.764380in}{1.931104in}}{\pgfqpoint{1.770204in}{1.936928in}}%
\pgfpathcurveto{\pgfqpoint{1.776028in}{1.942752in}}{\pgfqpoint{1.779300in}{1.950652in}}{\pgfqpoint{1.779300in}{1.958888in}}%
\pgfpathcurveto{\pgfqpoint{1.779300in}{1.967124in}}{\pgfqpoint{1.776028in}{1.975024in}}{\pgfqpoint{1.770204in}{1.980848in}}%
\pgfpathcurveto{\pgfqpoint{1.764380in}{1.986672in}}{\pgfqpoint{1.756480in}{1.989945in}}{\pgfqpoint{1.748243in}{1.989945in}}%
\pgfpathcurveto{\pgfqpoint{1.740007in}{1.989945in}}{\pgfqpoint{1.732107in}{1.986672in}}{\pgfqpoint{1.726283in}{1.980848in}}%
\pgfpathcurveto{\pgfqpoint{1.720459in}{1.975024in}}{\pgfqpoint{1.717187in}{1.967124in}}{\pgfqpoint{1.717187in}{1.958888in}}%
\pgfpathcurveto{\pgfqpoint{1.717187in}{1.950652in}}{\pgfqpoint{1.720459in}{1.942752in}}{\pgfqpoint{1.726283in}{1.936928in}}%
\pgfpathcurveto{\pgfqpoint{1.732107in}{1.931104in}}{\pgfqpoint{1.740007in}{1.927832in}}{\pgfqpoint{1.748243in}{1.927832in}}%
\pgfpathclose%
\pgfusepath{stroke,fill}%
\end{pgfscope}%
\begin{pgfscope}%
\pgfpathrectangle{\pgfqpoint{0.100000in}{0.220728in}}{\pgfqpoint{3.696000in}{3.696000in}}%
\pgfusepath{clip}%
\pgfsetbuttcap%
\pgfsetroundjoin%
\definecolor{currentfill}{rgb}{0.121569,0.466667,0.705882}%
\pgfsetfillcolor{currentfill}%
\pgfsetfillopacity{0.360928}%
\pgfsetlinewidth{1.003750pt}%
\definecolor{currentstroke}{rgb}{0.121569,0.466667,0.705882}%
\pgfsetstrokecolor{currentstroke}%
\pgfsetstrokeopacity{0.360928}%
\pgfsetdash{}{0pt}%
\pgfpathmoveto{\pgfqpoint{1.744053in}{1.925975in}}%
\pgfpathcurveto{\pgfqpoint{1.752289in}{1.925975in}}{\pgfqpoint{1.760189in}{1.929247in}}{\pgfqpoint{1.766013in}{1.935071in}}%
\pgfpathcurveto{\pgfqpoint{1.771837in}{1.940895in}}{\pgfqpoint{1.775109in}{1.948795in}}{\pgfqpoint{1.775109in}{1.957031in}}%
\pgfpathcurveto{\pgfqpoint{1.775109in}{1.965267in}}{\pgfqpoint{1.771837in}{1.973168in}}{\pgfqpoint{1.766013in}{1.978991in}}%
\pgfpathcurveto{\pgfqpoint{1.760189in}{1.984815in}}{\pgfqpoint{1.752289in}{1.988088in}}{\pgfqpoint{1.744053in}{1.988088in}}%
\pgfpathcurveto{\pgfqpoint{1.735816in}{1.988088in}}{\pgfqpoint{1.727916in}{1.984815in}}{\pgfqpoint{1.722092in}{1.978991in}}%
\pgfpathcurveto{\pgfqpoint{1.716268in}{1.973168in}}{\pgfqpoint{1.712996in}{1.965267in}}{\pgfqpoint{1.712996in}{1.957031in}}%
\pgfpathcurveto{\pgfqpoint{1.712996in}{1.948795in}}{\pgfqpoint{1.716268in}{1.940895in}}{\pgfqpoint{1.722092in}{1.935071in}}%
\pgfpathcurveto{\pgfqpoint{1.727916in}{1.929247in}}{\pgfqpoint{1.735816in}{1.925975in}}{\pgfqpoint{1.744053in}{1.925975in}}%
\pgfpathclose%
\pgfusepath{stroke,fill}%
\end{pgfscope}%
\begin{pgfscope}%
\pgfpathrectangle{\pgfqpoint{0.100000in}{0.220728in}}{\pgfqpoint{3.696000in}{3.696000in}}%
\pgfusepath{clip}%
\pgfsetbuttcap%
\pgfsetroundjoin%
\definecolor{currentfill}{rgb}{0.121569,0.466667,0.705882}%
\pgfsetfillcolor{currentfill}%
\pgfsetfillopacity{0.363559}%
\pgfsetlinewidth{1.003750pt}%
\definecolor{currentstroke}{rgb}{0.121569,0.466667,0.705882}%
\pgfsetstrokecolor{currentstroke}%
\pgfsetstrokeopacity{0.363559}%
\pgfsetdash{}{0pt}%
\pgfpathmoveto{\pgfqpoint{1.736954in}{1.922864in}}%
\pgfpathcurveto{\pgfqpoint{1.745190in}{1.922864in}}{\pgfqpoint{1.753090in}{1.926136in}}{\pgfqpoint{1.758914in}{1.931960in}}%
\pgfpathcurveto{\pgfqpoint{1.764738in}{1.937784in}}{\pgfqpoint{1.768010in}{1.945684in}}{\pgfqpoint{1.768010in}{1.953921in}}%
\pgfpathcurveto{\pgfqpoint{1.768010in}{1.962157in}}{\pgfqpoint{1.764738in}{1.970057in}}{\pgfqpoint{1.758914in}{1.975881in}}%
\pgfpathcurveto{\pgfqpoint{1.753090in}{1.981705in}}{\pgfqpoint{1.745190in}{1.984977in}}{\pgfqpoint{1.736954in}{1.984977in}}%
\pgfpathcurveto{\pgfqpoint{1.728717in}{1.984977in}}{\pgfqpoint{1.720817in}{1.981705in}}{\pgfqpoint{1.714993in}{1.975881in}}%
\pgfpathcurveto{\pgfqpoint{1.709169in}{1.970057in}}{\pgfqpoint{1.705897in}{1.962157in}}{\pgfqpoint{1.705897in}{1.953921in}}%
\pgfpathcurveto{\pgfqpoint{1.705897in}{1.945684in}}{\pgfqpoint{1.709169in}{1.937784in}}{\pgfqpoint{1.714993in}{1.931960in}}%
\pgfpathcurveto{\pgfqpoint{1.720817in}{1.926136in}}{\pgfqpoint{1.728717in}{1.922864in}}{\pgfqpoint{1.736954in}{1.922864in}}%
\pgfpathclose%
\pgfusepath{stroke,fill}%
\end{pgfscope}%
\begin{pgfscope}%
\pgfpathrectangle{\pgfqpoint{0.100000in}{0.220728in}}{\pgfqpoint{3.696000in}{3.696000in}}%
\pgfusepath{clip}%
\pgfsetbuttcap%
\pgfsetroundjoin%
\definecolor{currentfill}{rgb}{0.121569,0.466667,0.705882}%
\pgfsetfillcolor{currentfill}%
\pgfsetfillopacity{0.364230}%
\pgfsetlinewidth{1.003750pt}%
\definecolor{currentstroke}{rgb}{0.121569,0.466667,0.705882}%
\pgfsetstrokecolor{currentstroke}%
\pgfsetstrokeopacity{0.364230}%
\pgfsetdash{}{0pt}%
\pgfpathmoveto{\pgfqpoint{1.997161in}{1.986387in}}%
\pgfpathcurveto{\pgfqpoint{2.005397in}{1.986387in}}{\pgfqpoint{2.013297in}{1.989659in}}{\pgfqpoint{2.019121in}{1.995483in}}%
\pgfpathcurveto{\pgfqpoint{2.024945in}{2.001307in}}{\pgfqpoint{2.028217in}{2.009207in}}{\pgfqpoint{2.028217in}{2.017443in}}%
\pgfpathcurveto{\pgfqpoint{2.028217in}{2.025679in}}{\pgfqpoint{2.024945in}{2.033579in}}{\pgfqpoint{2.019121in}{2.039403in}}%
\pgfpathcurveto{\pgfqpoint{2.013297in}{2.045227in}}{\pgfqpoint{2.005397in}{2.048500in}}{\pgfqpoint{1.997161in}{2.048500in}}%
\pgfpathcurveto{\pgfqpoint{1.988924in}{2.048500in}}{\pgfqpoint{1.981024in}{2.045227in}}{\pgfqpoint{1.975200in}{2.039403in}}%
\pgfpathcurveto{\pgfqpoint{1.969377in}{2.033579in}}{\pgfqpoint{1.966104in}{2.025679in}}{\pgfqpoint{1.966104in}{2.017443in}}%
\pgfpathcurveto{\pgfqpoint{1.966104in}{2.009207in}}{\pgfqpoint{1.969377in}{2.001307in}}{\pgfqpoint{1.975200in}{1.995483in}}%
\pgfpathcurveto{\pgfqpoint{1.981024in}{1.989659in}}{\pgfqpoint{1.988924in}{1.986387in}}{\pgfqpoint{1.997161in}{1.986387in}}%
\pgfpathclose%
\pgfusepath{stroke,fill}%
\end{pgfscope}%
\begin{pgfscope}%
\pgfpathrectangle{\pgfqpoint{0.100000in}{0.220728in}}{\pgfqpoint{3.696000in}{3.696000in}}%
\pgfusepath{clip}%
\pgfsetbuttcap%
\pgfsetroundjoin%
\definecolor{currentfill}{rgb}{0.121569,0.466667,0.705882}%
\pgfsetfillcolor{currentfill}%
\pgfsetfillopacity{0.365231}%
\pgfsetlinewidth{1.003750pt}%
\definecolor{currentstroke}{rgb}{0.121569,0.466667,0.705882}%
\pgfsetstrokecolor{currentstroke}%
\pgfsetstrokeopacity{0.365231}%
\pgfsetdash{}{0pt}%
\pgfpathmoveto{\pgfqpoint{1.730406in}{1.921128in}}%
\pgfpathcurveto{\pgfqpoint{1.738642in}{1.921128in}}{\pgfqpoint{1.746542in}{1.924400in}}{\pgfqpoint{1.752366in}{1.930224in}}%
\pgfpathcurveto{\pgfqpoint{1.758190in}{1.936048in}}{\pgfqpoint{1.761462in}{1.943948in}}{\pgfqpoint{1.761462in}{1.952185in}}%
\pgfpathcurveto{\pgfqpoint{1.761462in}{1.960421in}}{\pgfqpoint{1.758190in}{1.968321in}}{\pgfqpoint{1.752366in}{1.974145in}}%
\pgfpathcurveto{\pgfqpoint{1.746542in}{1.979969in}}{\pgfqpoint{1.738642in}{1.983241in}}{\pgfqpoint{1.730406in}{1.983241in}}%
\pgfpathcurveto{\pgfqpoint{1.722170in}{1.983241in}}{\pgfqpoint{1.714269in}{1.979969in}}{\pgfqpoint{1.708446in}{1.974145in}}%
\pgfpathcurveto{\pgfqpoint{1.702622in}{1.968321in}}{\pgfqpoint{1.699349in}{1.960421in}}{\pgfqpoint{1.699349in}{1.952185in}}%
\pgfpathcurveto{\pgfqpoint{1.699349in}{1.943948in}}{\pgfqpoint{1.702622in}{1.936048in}}{\pgfqpoint{1.708446in}{1.930224in}}%
\pgfpathcurveto{\pgfqpoint{1.714269in}{1.924400in}}{\pgfqpoint{1.722170in}{1.921128in}}{\pgfqpoint{1.730406in}{1.921128in}}%
\pgfpathclose%
\pgfusepath{stroke,fill}%
\end{pgfscope}%
\begin{pgfscope}%
\pgfpathrectangle{\pgfqpoint{0.100000in}{0.220728in}}{\pgfqpoint{3.696000in}{3.696000in}}%
\pgfusepath{clip}%
\pgfsetbuttcap%
\pgfsetroundjoin%
\definecolor{currentfill}{rgb}{0.121569,0.466667,0.705882}%
\pgfsetfillcolor{currentfill}%
\pgfsetfillopacity{0.366875}%
\pgfsetlinewidth{1.003750pt}%
\definecolor{currentstroke}{rgb}{0.121569,0.466667,0.705882}%
\pgfsetstrokecolor{currentstroke}%
\pgfsetstrokeopacity{0.366875}%
\pgfsetdash{}{0pt}%
\pgfpathmoveto{\pgfqpoint{1.725939in}{1.918821in}}%
\pgfpathcurveto{\pgfqpoint{1.734176in}{1.918821in}}{\pgfqpoint{1.742076in}{1.922094in}}{\pgfqpoint{1.747900in}{1.927917in}}%
\pgfpathcurveto{\pgfqpoint{1.753724in}{1.933741in}}{\pgfqpoint{1.756996in}{1.941641in}}{\pgfqpoint{1.756996in}{1.949878in}}%
\pgfpathcurveto{\pgfqpoint{1.756996in}{1.958114in}}{\pgfqpoint{1.753724in}{1.966014in}}{\pgfqpoint{1.747900in}{1.971838in}}%
\pgfpathcurveto{\pgfqpoint{1.742076in}{1.977662in}}{\pgfqpoint{1.734176in}{1.980934in}}{\pgfqpoint{1.725939in}{1.980934in}}%
\pgfpathcurveto{\pgfqpoint{1.717703in}{1.980934in}}{\pgfqpoint{1.709803in}{1.977662in}}{\pgfqpoint{1.703979in}{1.971838in}}%
\pgfpathcurveto{\pgfqpoint{1.698155in}{1.966014in}}{\pgfqpoint{1.694883in}{1.958114in}}{\pgfqpoint{1.694883in}{1.949878in}}%
\pgfpathcurveto{\pgfqpoint{1.694883in}{1.941641in}}{\pgfqpoint{1.698155in}{1.933741in}}{\pgfqpoint{1.703979in}{1.927917in}}%
\pgfpathcurveto{\pgfqpoint{1.709803in}{1.922094in}}{\pgfqpoint{1.717703in}{1.918821in}}{\pgfqpoint{1.725939in}{1.918821in}}%
\pgfpathclose%
\pgfusepath{stroke,fill}%
\end{pgfscope}%
\begin{pgfscope}%
\pgfpathrectangle{\pgfqpoint{0.100000in}{0.220728in}}{\pgfqpoint{3.696000in}{3.696000in}}%
\pgfusepath{clip}%
\pgfsetbuttcap%
\pgfsetroundjoin%
\definecolor{currentfill}{rgb}{0.121569,0.466667,0.705882}%
\pgfsetfillcolor{currentfill}%
\pgfsetfillopacity{0.369939}%
\pgfsetlinewidth{1.003750pt}%
\definecolor{currentstroke}{rgb}{0.121569,0.466667,0.705882}%
\pgfsetstrokecolor{currentstroke}%
\pgfsetstrokeopacity{0.369939}%
\pgfsetdash{}{0pt}%
\pgfpathmoveto{\pgfqpoint{1.717553in}{1.915469in}}%
\pgfpathcurveto{\pgfqpoint{1.725789in}{1.915469in}}{\pgfqpoint{1.733689in}{1.918741in}}{\pgfqpoint{1.739513in}{1.924565in}}%
\pgfpathcurveto{\pgfqpoint{1.745337in}{1.930389in}}{\pgfqpoint{1.748609in}{1.938289in}}{\pgfqpoint{1.748609in}{1.946525in}}%
\pgfpathcurveto{\pgfqpoint{1.748609in}{1.954761in}}{\pgfqpoint{1.745337in}{1.962662in}}{\pgfqpoint{1.739513in}{1.968485in}}%
\pgfpathcurveto{\pgfqpoint{1.733689in}{1.974309in}}{\pgfqpoint{1.725789in}{1.977582in}}{\pgfqpoint{1.717553in}{1.977582in}}%
\pgfpathcurveto{\pgfqpoint{1.709316in}{1.977582in}}{\pgfqpoint{1.701416in}{1.974309in}}{\pgfqpoint{1.695592in}{1.968485in}}%
\pgfpathcurveto{\pgfqpoint{1.689769in}{1.962662in}}{\pgfqpoint{1.686496in}{1.954761in}}{\pgfqpoint{1.686496in}{1.946525in}}%
\pgfpathcurveto{\pgfqpoint{1.686496in}{1.938289in}}{\pgfqpoint{1.689769in}{1.930389in}}{\pgfqpoint{1.695592in}{1.924565in}}%
\pgfpathcurveto{\pgfqpoint{1.701416in}{1.918741in}}{\pgfqpoint{1.709316in}{1.915469in}}{\pgfqpoint{1.717553in}{1.915469in}}%
\pgfpathclose%
\pgfusepath{stroke,fill}%
\end{pgfscope}%
\begin{pgfscope}%
\pgfpathrectangle{\pgfqpoint{0.100000in}{0.220728in}}{\pgfqpoint{3.696000in}{3.696000in}}%
\pgfusepath{clip}%
\pgfsetbuttcap%
\pgfsetroundjoin%
\definecolor{currentfill}{rgb}{0.121569,0.466667,0.705882}%
\pgfsetfillcolor{currentfill}%
\pgfsetfillopacity{0.371230}%
\pgfsetlinewidth{1.003750pt}%
\definecolor{currentstroke}{rgb}{0.121569,0.466667,0.705882}%
\pgfsetstrokecolor{currentstroke}%
\pgfsetstrokeopacity{0.371230}%
\pgfsetdash{}{0pt}%
\pgfpathmoveto{\pgfqpoint{1.997183in}{1.987858in}}%
\pgfpathcurveto{\pgfqpoint{2.005419in}{1.987858in}}{\pgfqpoint{2.013319in}{1.991130in}}{\pgfqpoint{2.019143in}{1.996954in}}%
\pgfpathcurveto{\pgfqpoint{2.024967in}{2.002778in}}{\pgfqpoint{2.028239in}{2.010678in}}{\pgfqpoint{2.028239in}{2.018914in}}%
\pgfpathcurveto{\pgfqpoint{2.028239in}{2.027151in}}{\pgfqpoint{2.024967in}{2.035051in}}{\pgfqpoint{2.019143in}{2.040875in}}%
\pgfpathcurveto{\pgfqpoint{2.013319in}{2.046698in}}{\pgfqpoint{2.005419in}{2.049971in}}{\pgfqpoint{1.997183in}{2.049971in}}%
\pgfpathcurveto{\pgfqpoint{1.988947in}{2.049971in}}{\pgfqpoint{1.981047in}{2.046698in}}{\pgfqpoint{1.975223in}{2.040875in}}%
\pgfpathcurveto{\pgfqpoint{1.969399in}{2.035051in}}{\pgfqpoint{1.966126in}{2.027151in}}{\pgfqpoint{1.966126in}{2.018914in}}%
\pgfpathcurveto{\pgfqpoint{1.966126in}{2.010678in}}{\pgfqpoint{1.969399in}{2.002778in}}{\pgfqpoint{1.975223in}{1.996954in}}%
\pgfpathcurveto{\pgfqpoint{1.981047in}{1.991130in}}{\pgfqpoint{1.988947in}{1.987858in}}{\pgfqpoint{1.997183in}{1.987858in}}%
\pgfpathclose%
\pgfusepath{stroke,fill}%
\end{pgfscope}%
\begin{pgfscope}%
\pgfpathrectangle{\pgfqpoint{0.100000in}{0.220728in}}{\pgfqpoint{3.696000in}{3.696000in}}%
\pgfusepath{clip}%
\pgfsetbuttcap%
\pgfsetroundjoin%
\definecolor{currentfill}{rgb}{0.121569,0.466667,0.705882}%
\pgfsetfillcolor{currentfill}%
\pgfsetfillopacity{0.372168}%
\pgfsetlinewidth{1.003750pt}%
\definecolor{currentstroke}{rgb}{0.121569,0.466667,0.705882}%
\pgfsetstrokecolor{currentstroke}%
\pgfsetstrokeopacity{0.372168}%
\pgfsetdash{}{0pt}%
\pgfpathmoveto{\pgfqpoint{1.708694in}{1.909432in}}%
\pgfpathcurveto{\pgfqpoint{1.716930in}{1.909432in}}{\pgfqpoint{1.724830in}{1.912704in}}{\pgfqpoint{1.730654in}{1.918528in}}%
\pgfpathcurveto{\pgfqpoint{1.736478in}{1.924352in}}{\pgfqpoint{1.739750in}{1.932252in}}{\pgfqpoint{1.739750in}{1.940488in}}%
\pgfpathcurveto{\pgfqpoint{1.739750in}{1.948724in}}{\pgfqpoint{1.736478in}{1.956624in}}{\pgfqpoint{1.730654in}{1.962448in}}%
\pgfpathcurveto{\pgfqpoint{1.724830in}{1.968272in}}{\pgfqpoint{1.716930in}{1.971545in}}{\pgfqpoint{1.708694in}{1.971545in}}%
\pgfpathcurveto{\pgfqpoint{1.700458in}{1.971545in}}{\pgfqpoint{1.692557in}{1.968272in}}{\pgfqpoint{1.686734in}{1.962448in}}%
\pgfpathcurveto{\pgfqpoint{1.680910in}{1.956624in}}{\pgfqpoint{1.677637in}{1.948724in}}{\pgfqpoint{1.677637in}{1.940488in}}%
\pgfpathcurveto{\pgfqpoint{1.677637in}{1.932252in}}{\pgfqpoint{1.680910in}{1.924352in}}{\pgfqpoint{1.686734in}{1.918528in}}%
\pgfpathcurveto{\pgfqpoint{1.692557in}{1.912704in}}{\pgfqpoint{1.700458in}{1.909432in}}{\pgfqpoint{1.708694in}{1.909432in}}%
\pgfpathclose%
\pgfusepath{stroke,fill}%
\end{pgfscope}%
\begin{pgfscope}%
\pgfpathrectangle{\pgfqpoint{0.100000in}{0.220728in}}{\pgfqpoint{3.696000in}{3.696000in}}%
\pgfusepath{clip}%
\pgfsetbuttcap%
\pgfsetroundjoin%
\definecolor{currentfill}{rgb}{0.121569,0.466667,0.705882}%
\pgfsetfillcolor{currentfill}%
\pgfsetfillopacity{0.375488}%
\pgfsetlinewidth{1.003750pt}%
\definecolor{currentstroke}{rgb}{0.121569,0.466667,0.705882}%
\pgfsetstrokecolor{currentstroke}%
\pgfsetstrokeopacity{0.375488}%
\pgfsetdash{}{0pt}%
\pgfpathmoveto{\pgfqpoint{1.702842in}{1.911466in}}%
\pgfpathcurveto{\pgfqpoint{1.711078in}{1.911466in}}{\pgfqpoint{1.718979in}{1.914739in}}{\pgfqpoint{1.724802in}{1.920563in}}%
\pgfpathcurveto{\pgfqpoint{1.730626in}{1.926386in}}{\pgfqpoint{1.733899in}{1.934286in}}{\pgfqpoint{1.733899in}{1.942523in}}%
\pgfpathcurveto{\pgfqpoint{1.733899in}{1.950759in}}{\pgfqpoint{1.730626in}{1.958659in}}{\pgfqpoint{1.724802in}{1.964483in}}%
\pgfpathcurveto{\pgfqpoint{1.718979in}{1.970307in}}{\pgfqpoint{1.711078in}{1.973579in}}{\pgfqpoint{1.702842in}{1.973579in}}%
\pgfpathcurveto{\pgfqpoint{1.694606in}{1.973579in}}{\pgfqpoint{1.686706in}{1.970307in}}{\pgfqpoint{1.680882in}{1.964483in}}%
\pgfpathcurveto{\pgfqpoint{1.675058in}{1.958659in}}{\pgfqpoint{1.671786in}{1.950759in}}{\pgfqpoint{1.671786in}{1.942523in}}%
\pgfpathcurveto{\pgfqpoint{1.671786in}{1.934286in}}{\pgfqpoint{1.675058in}{1.926386in}}{\pgfqpoint{1.680882in}{1.920563in}}%
\pgfpathcurveto{\pgfqpoint{1.686706in}{1.914739in}}{\pgfqpoint{1.694606in}{1.911466in}}{\pgfqpoint{1.702842in}{1.911466in}}%
\pgfpathclose%
\pgfusepath{stroke,fill}%
\end{pgfscope}%
\begin{pgfscope}%
\pgfpathrectangle{\pgfqpoint{0.100000in}{0.220728in}}{\pgfqpoint{3.696000in}{3.696000in}}%
\pgfusepath{clip}%
\pgfsetbuttcap%
\pgfsetroundjoin%
\definecolor{currentfill}{rgb}{0.121569,0.466667,0.705882}%
\pgfsetfillcolor{currentfill}%
\pgfsetfillopacity{0.376725}%
\pgfsetlinewidth{1.003750pt}%
\definecolor{currentstroke}{rgb}{0.121569,0.466667,0.705882}%
\pgfsetstrokecolor{currentstroke}%
\pgfsetstrokeopacity{0.376725}%
\pgfsetdash{}{0pt}%
\pgfpathmoveto{\pgfqpoint{1.998458in}{1.977693in}}%
\pgfpathcurveto{\pgfqpoint{2.006694in}{1.977693in}}{\pgfqpoint{2.014594in}{1.980965in}}{\pgfqpoint{2.020418in}{1.986789in}}%
\pgfpathcurveto{\pgfqpoint{2.026242in}{1.992613in}}{\pgfqpoint{2.029514in}{2.000513in}}{\pgfqpoint{2.029514in}{2.008750in}}%
\pgfpathcurveto{\pgfqpoint{2.029514in}{2.016986in}}{\pgfqpoint{2.026242in}{2.024886in}}{\pgfqpoint{2.020418in}{2.030710in}}%
\pgfpathcurveto{\pgfqpoint{2.014594in}{2.036534in}}{\pgfqpoint{2.006694in}{2.039806in}}{\pgfqpoint{1.998458in}{2.039806in}}%
\pgfpathcurveto{\pgfqpoint{1.990221in}{2.039806in}}{\pgfqpoint{1.982321in}{2.036534in}}{\pgfqpoint{1.976497in}{2.030710in}}%
\pgfpathcurveto{\pgfqpoint{1.970674in}{2.024886in}}{\pgfqpoint{1.967401in}{2.016986in}}{\pgfqpoint{1.967401in}{2.008750in}}%
\pgfpathcurveto{\pgfqpoint{1.967401in}{2.000513in}}{\pgfqpoint{1.970674in}{1.992613in}}{\pgfqpoint{1.976497in}{1.986789in}}%
\pgfpathcurveto{\pgfqpoint{1.982321in}{1.980965in}}{\pgfqpoint{1.990221in}{1.977693in}}{\pgfqpoint{1.998458in}{1.977693in}}%
\pgfpathclose%
\pgfusepath{stroke,fill}%
\end{pgfscope}%
\begin{pgfscope}%
\pgfpathrectangle{\pgfqpoint{0.100000in}{0.220728in}}{\pgfqpoint{3.696000in}{3.696000in}}%
\pgfusepath{clip}%
\pgfsetbuttcap%
\pgfsetroundjoin%
\definecolor{currentfill}{rgb}{0.121569,0.466667,0.705882}%
\pgfsetfillcolor{currentfill}%
\pgfsetfillopacity{0.377108}%
\pgfsetlinewidth{1.003750pt}%
\definecolor{currentstroke}{rgb}{0.121569,0.466667,0.705882}%
\pgfsetstrokecolor{currentstroke}%
\pgfsetstrokeopacity{0.377108}%
\pgfsetdash{}{0pt}%
\pgfpathmoveto{\pgfqpoint{1.695785in}{1.904903in}}%
\pgfpathcurveto{\pgfqpoint{1.704022in}{1.904903in}}{\pgfqpoint{1.711922in}{1.908176in}}{\pgfqpoint{1.717746in}{1.914000in}}%
\pgfpathcurveto{\pgfqpoint{1.723570in}{1.919824in}}{\pgfqpoint{1.726842in}{1.927724in}}{\pgfqpoint{1.726842in}{1.935960in}}%
\pgfpathcurveto{\pgfqpoint{1.726842in}{1.944196in}}{\pgfqpoint{1.723570in}{1.952096in}}{\pgfqpoint{1.717746in}{1.957920in}}%
\pgfpathcurveto{\pgfqpoint{1.711922in}{1.963744in}}{\pgfqpoint{1.704022in}{1.967016in}}{\pgfqpoint{1.695785in}{1.967016in}}%
\pgfpathcurveto{\pgfqpoint{1.687549in}{1.967016in}}{\pgfqpoint{1.679649in}{1.963744in}}{\pgfqpoint{1.673825in}{1.957920in}}%
\pgfpathcurveto{\pgfqpoint{1.668001in}{1.952096in}}{\pgfqpoint{1.664729in}{1.944196in}}{\pgfqpoint{1.664729in}{1.935960in}}%
\pgfpathcurveto{\pgfqpoint{1.664729in}{1.927724in}}{\pgfqpoint{1.668001in}{1.919824in}}{\pgfqpoint{1.673825in}{1.914000in}}%
\pgfpathcurveto{\pgfqpoint{1.679649in}{1.908176in}}{\pgfqpoint{1.687549in}{1.904903in}}{\pgfqpoint{1.695785in}{1.904903in}}%
\pgfpathclose%
\pgfusepath{stroke,fill}%
\end{pgfscope}%
\begin{pgfscope}%
\pgfpathrectangle{\pgfqpoint{0.100000in}{0.220728in}}{\pgfqpoint{3.696000in}{3.696000in}}%
\pgfusepath{clip}%
\pgfsetbuttcap%
\pgfsetroundjoin%
\definecolor{currentfill}{rgb}{0.121569,0.466667,0.705882}%
\pgfsetfillcolor{currentfill}%
\pgfsetfillopacity{0.378796}%
\pgfsetlinewidth{1.003750pt}%
\definecolor{currentstroke}{rgb}{0.121569,0.466667,0.705882}%
\pgfsetstrokecolor{currentstroke}%
\pgfsetstrokeopacity{0.378796}%
\pgfsetdash{}{0pt}%
\pgfpathmoveto{\pgfqpoint{1.689212in}{1.901237in}}%
\pgfpathcurveto{\pgfqpoint{1.697448in}{1.901237in}}{\pgfqpoint{1.705348in}{1.904510in}}{\pgfqpoint{1.711172in}{1.910334in}}%
\pgfpathcurveto{\pgfqpoint{1.716996in}{1.916158in}}{\pgfqpoint{1.720268in}{1.924058in}}{\pgfqpoint{1.720268in}{1.932294in}}%
\pgfpathcurveto{\pgfqpoint{1.720268in}{1.940530in}}{\pgfqpoint{1.716996in}{1.948430in}}{\pgfqpoint{1.711172in}{1.954254in}}%
\pgfpathcurveto{\pgfqpoint{1.705348in}{1.960078in}}{\pgfqpoint{1.697448in}{1.963350in}}{\pgfqpoint{1.689212in}{1.963350in}}%
\pgfpathcurveto{\pgfqpoint{1.680976in}{1.963350in}}{\pgfqpoint{1.673075in}{1.960078in}}{\pgfqpoint{1.667252in}{1.954254in}}%
\pgfpathcurveto{\pgfqpoint{1.661428in}{1.948430in}}{\pgfqpoint{1.658155in}{1.940530in}}{\pgfqpoint{1.658155in}{1.932294in}}%
\pgfpathcurveto{\pgfqpoint{1.658155in}{1.924058in}}{\pgfqpoint{1.661428in}{1.916158in}}{\pgfqpoint{1.667252in}{1.910334in}}%
\pgfpathcurveto{\pgfqpoint{1.673075in}{1.904510in}}{\pgfqpoint{1.680976in}{1.901237in}}{\pgfqpoint{1.689212in}{1.901237in}}%
\pgfpathclose%
\pgfusepath{stroke,fill}%
\end{pgfscope}%
\begin{pgfscope}%
\pgfpathrectangle{\pgfqpoint{0.100000in}{0.220728in}}{\pgfqpoint{3.696000in}{3.696000in}}%
\pgfusepath{clip}%
\pgfsetbuttcap%
\pgfsetroundjoin%
\definecolor{currentfill}{rgb}{0.121569,0.466667,0.705882}%
\pgfsetfillcolor{currentfill}%
\pgfsetfillopacity{0.380995}%
\pgfsetlinewidth{1.003750pt}%
\definecolor{currentstroke}{rgb}{0.121569,0.466667,0.705882}%
\pgfsetstrokecolor{currentstroke}%
\pgfsetstrokeopacity{0.380995}%
\pgfsetdash{}{0pt}%
\pgfpathmoveto{\pgfqpoint{1.685058in}{1.900799in}}%
\pgfpathcurveto{\pgfqpoint{1.693294in}{1.900799in}}{\pgfqpoint{1.701194in}{1.904071in}}{\pgfqpoint{1.707018in}{1.909895in}}%
\pgfpathcurveto{\pgfqpoint{1.712842in}{1.915719in}}{\pgfqpoint{1.716114in}{1.923619in}}{\pgfqpoint{1.716114in}{1.931856in}}%
\pgfpathcurveto{\pgfqpoint{1.716114in}{1.940092in}}{\pgfqpoint{1.712842in}{1.947992in}}{\pgfqpoint{1.707018in}{1.953816in}}%
\pgfpathcurveto{\pgfqpoint{1.701194in}{1.959640in}}{\pgfqpoint{1.693294in}{1.962912in}}{\pgfqpoint{1.685058in}{1.962912in}}%
\pgfpathcurveto{\pgfqpoint{1.676821in}{1.962912in}}{\pgfqpoint{1.668921in}{1.959640in}}{\pgfqpoint{1.663097in}{1.953816in}}%
\pgfpathcurveto{\pgfqpoint{1.657273in}{1.947992in}}{\pgfqpoint{1.654001in}{1.940092in}}{\pgfqpoint{1.654001in}{1.931856in}}%
\pgfpathcurveto{\pgfqpoint{1.654001in}{1.923619in}}{\pgfqpoint{1.657273in}{1.915719in}}{\pgfqpoint{1.663097in}{1.909895in}}%
\pgfpathcurveto{\pgfqpoint{1.668921in}{1.904071in}}{\pgfqpoint{1.676821in}{1.900799in}}{\pgfqpoint{1.685058in}{1.900799in}}%
\pgfpathclose%
\pgfusepath{stroke,fill}%
\end{pgfscope}%
\begin{pgfscope}%
\pgfpathrectangle{\pgfqpoint{0.100000in}{0.220728in}}{\pgfqpoint{3.696000in}{3.696000in}}%
\pgfusepath{clip}%
\pgfsetbuttcap%
\pgfsetroundjoin%
\definecolor{currentfill}{rgb}{0.121569,0.466667,0.705882}%
\pgfsetfillcolor{currentfill}%
\pgfsetfillopacity{0.381659}%
\pgfsetlinewidth{1.003750pt}%
\definecolor{currentstroke}{rgb}{0.121569,0.466667,0.705882}%
\pgfsetstrokecolor{currentstroke}%
\pgfsetstrokeopacity{0.381659}%
\pgfsetdash{}{0pt}%
\pgfpathmoveto{\pgfqpoint{1.681783in}{1.897573in}}%
\pgfpathcurveto{\pgfqpoint{1.690020in}{1.897573in}}{\pgfqpoint{1.697920in}{1.900845in}}{\pgfqpoint{1.703744in}{1.906669in}}%
\pgfpathcurveto{\pgfqpoint{1.709567in}{1.912493in}}{\pgfqpoint{1.712840in}{1.920393in}}{\pgfqpoint{1.712840in}{1.928629in}}%
\pgfpathcurveto{\pgfqpoint{1.712840in}{1.936866in}}{\pgfqpoint{1.709567in}{1.944766in}}{\pgfqpoint{1.703744in}{1.950590in}}%
\pgfpathcurveto{\pgfqpoint{1.697920in}{1.956413in}}{\pgfqpoint{1.690020in}{1.959686in}}{\pgfqpoint{1.681783in}{1.959686in}}%
\pgfpathcurveto{\pgfqpoint{1.673547in}{1.959686in}}{\pgfqpoint{1.665647in}{1.956413in}}{\pgfqpoint{1.659823in}{1.950590in}}%
\pgfpathcurveto{\pgfqpoint{1.653999in}{1.944766in}}{\pgfqpoint{1.650727in}{1.936866in}}{\pgfqpoint{1.650727in}{1.928629in}}%
\pgfpathcurveto{\pgfqpoint{1.650727in}{1.920393in}}{\pgfqpoint{1.653999in}{1.912493in}}{\pgfqpoint{1.659823in}{1.906669in}}%
\pgfpathcurveto{\pgfqpoint{1.665647in}{1.900845in}}{\pgfqpoint{1.673547in}{1.897573in}}{\pgfqpoint{1.681783in}{1.897573in}}%
\pgfpathclose%
\pgfusepath{stroke,fill}%
\end{pgfscope}%
\begin{pgfscope}%
\pgfpathrectangle{\pgfqpoint{0.100000in}{0.220728in}}{\pgfqpoint{3.696000in}{3.696000in}}%
\pgfusepath{clip}%
\pgfsetbuttcap%
\pgfsetroundjoin%
\definecolor{currentfill}{rgb}{0.121569,0.466667,0.705882}%
\pgfsetfillcolor{currentfill}%
\pgfsetfillopacity{0.382313}%
\pgfsetlinewidth{1.003750pt}%
\definecolor{currentstroke}{rgb}{0.121569,0.466667,0.705882}%
\pgfsetstrokecolor{currentstroke}%
\pgfsetstrokeopacity{0.382313}%
\pgfsetdash{}{0pt}%
\pgfpathmoveto{\pgfqpoint{1.679244in}{1.895353in}}%
\pgfpathcurveto{\pgfqpoint{1.687480in}{1.895353in}}{\pgfqpoint{1.695380in}{1.898625in}}{\pgfqpoint{1.701204in}{1.904449in}}%
\pgfpathcurveto{\pgfqpoint{1.707028in}{1.910273in}}{\pgfqpoint{1.710300in}{1.918173in}}{\pgfqpoint{1.710300in}{1.926410in}}%
\pgfpathcurveto{\pgfqpoint{1.710300in}{1.934646in}}{\pgfqpoint{1.707028in}{1.942546in}}{\pgfqpoint{1.701204in}{1.948370in}}%
\pgfpathcurveto{\pgfqpoint{1.695380in}{1.954194in}}{\pgfqpoint{1.687480in}{1.957466in}}{\pgfqpoint{1.679244in}{1.957466in}}%
\pgfpathcurveto{\pgfqpoint{1.671007in}{1.957466in}}{\pgfqpoint{1.663107in}{1.954194in}}{\pgfqpoint{1.657283in}{1.948370in}}%
\pgfpathcurveto{\pgfqpoint{1.651460in}{1.942546in}}{\pgfqpoint{1.648187in}{1.934646in}}{\pgfqpoint{1.648187in}{1.926410in}}%
\pgfpathcurveto{\pgfqpoint{1.648187in}{1.918173in}}{\pgfqpoint{1.651460in}{1.910273in}}{\pgfqpoint{1.657283in}{1.904449in}}%
\pgfpathcurveto{\pgfqpoint{1.663107in}{1.898625in}}{\pgfqpoint{1.671007in}{1.895353in}}{\pgfqpoint{1.679244in}{1.895353in}}%
\pgfpathclose%
\pgfusepath{stroke,fill}%
\end{pgfscope}%
\begin{pgfscope}%
\pgfpathrectangle{\pgfqpoint{0.100000in}{0.220728in}}{\pgfqpoint{3.696000in}{3.696000in}}%
\pgfusepath{clip}%
\pgfsetbuttcap%
\pgfsetroundjoin%
\definecolor{currentfill}{rgb}{0.121569,0.466667,0.705882}%
\pgfsetfillcolor{currentfill}%
\pgfsetfillopacity{0.383481}%
\pgfsetlinewidth{1.003750pt}%
\definecolor{currentstroke}{rgb}{0.121569,0.466667,0.705882}%
\pgfsetstrokecolor{currentstroke}%
\pgfsetstrokeopacity{0.383481}%
\pgfsetdash{}{0pt}%
\pgfpathmoveto{\pgfqpoint{2.000258in}{1.969923in}}%
\pgfpathcurveto{\pgfqpoint{2.008494in}{1.969923in}}{\pgfqpoint{2.016394in}{1.973196in}}{\pgfqpoint{2.022218in}{1.979020in}}%
\pgfpathcurveto{\pgfqpoint{2.028042in}{1.984844in}}{\pgfqpoint{2.031314in}{1.992744in}}{\pgfqpoint{2.031314in}{2.000980in}}%
\pgfpathcurveto{\pgfqpoint{2.031314in}{2.009216in}}{\pgfqpoint{2.028042in}{2.017116in}}{\pgfqpoint{2.022218in}{2.022940in}}%
\pgfpathcurveto{\pgfqpoint{2.016394in}{2.028764in}}{\pgfqpoint{2.008494in}{2.032036in}}{\pgfqpoint{2.000258in}{2.032036in}}%
\pgfpathcurveto{\pgfqpoint{1.992022in}{2.032036in}}{\pgfqpoint{1.984122in}{2.028764in}}{\pgfqpoint{1.978298in}{2.022940in}}%
\pgfpathcurveto{\pgfqpoint{1.972474in}{2.017116in}}{\pgfqpoint{1.969201in}{2.009216in}}{\pgfqpoint{1.969201in}{2.000980in}}%
\pgfpathcurveto{\pgfqpoint{1.969201in}{1.992744in}}{\pgfqpoint{1.972474in}{1.984844in}}{\pgfqpoint{1.978298in}{1.979020in}}%
\pgfpathcurveto{\pgfqpoint{1.984122in}{1.973196in}}{\pgfqpoint{1.992022in}{1.969923in}}{\pgfqpoint{2.000258in}{1.969923in}}%
\pgfpathclose%
\pgfusepath{stroke,fill}%
\end{pgfscope}%
\begin{pgfscope}%
\pgfpathrectangle{\pgfqpoint{0.100000in}{0.220728in}}{\pgfqpoint{3.696000in}{3.696000in}}%
\pgfusepath{clip}%
\pgfsetbuttcap%
\pgfsetroundjoin%
\definecolor{currentfill}{rgb}{0.121569,0.466667,0.705882}%
\pgfsetfillcolor{currentfill}%
\pgfsetfillopacity{0.384153}%
\pgfsetlinewidth{1.003750pt}%
\definecolor{currentstroke}{rgb}{0.121569,0.466667,0.705882}%
\pgfsetstrokecolor{currentstroke}%
\pgfsetstrokeopacity{0.384153}%
\pgfsetdash{}{0pt}%
\pgfpathmoveto{\pgfqpoint{1.675596in}{1.894480in}}%
\pgfpathcurveto{\pgfqpoint{1.683832in}{1.894480in}}{\pgfqpoint{1.691732in}{1.897753in}}{\pgfqpoint{1.697556in}{1.903576in}}%
\pgfpathcurveto{\pgfqpoint{1.703380in}{1.909400in}}{\pgfqpoint{1.706652in}{1.917300in}}{\pgfqpoint{1.706652in}{1.925537in}}%
\pgfpathcurveto{\pgfqpoint{1.706652in}{1.933773in}}{\pgfqpoint{1.703380in}{1.941673in}}{\pgfqpoint{1.697556in}{1.947497in}}%
\pgfpathcurveto{\pgfqpoint{1.691732in}{1.953321in}}{\pgfqpoint{1.683832in}{1.956593in}}{\pgfqpoint{1.675596in}{1.956593in}}%
\pgfpathcurveto{\pgfqpoint{1.667359in}{1.956593in}}{\pgfqpoint{1.659459in}{1.953321in}}{\pgfqpoint{1.653635in}{1.947497in}}%
\pgfpathcurveto{\pgfqpoint{1.647811in}{1.941673in}}{\pgfqpoint{1.644539in}{1.933773in}}{\pgfqpoint{1.644539in}{1.925537in}}%
\pgfpathcurveto{\pgfqpoint{1.644539in}{1.917300in}}{\pgfqpoint{1.647811in}{1.909400in}}{\pgfqpoint{1.653635in}{1.903576in}}%
\pgfpathcurveto{\pgfqpoint{1.659459in}{1.897753in}}{\pgfqpoint{1.667359in}{1.894480in}}{\pgfqpoint{1.675596in}{1.894480in}}%
\pgfpathclose%
\pgfusepath{stroke,fill}%
\end{pgfscope}%
\begin{pgfscope}%
\pgfpathrectangle{\pgfqpoint{0.100000in}{0.220728in}}{\pgfqpoint{3.696000in}{3.696000in}}%
\pgfusepath{clip}%
\pgfsetbuttcap%
\pgfsetroundjoin%
\definecolor{currentfill}{rgb}{0.121569,0.466667,0.705882}%
\pgfsetfillcolor{currentfill}%
\pgfsetfillopacity{0.384588}%
\pgfsetlinewidth{1.003750pt}%
\definecolor{currentstroke}{rgb}{0.121569,0.466667,0.705882}%
\pgfsetstrokecolor{currentstroke}%
\pgfsetstrokeopacity{0.384588}%
\pgfsetdash{}{0pt}%
\pgfpathmoveto{\pgfqpoint{1.673323in}{1.892992in}}%
\pgfpathcurveto{\pgfqpoint{1.681559in}{1.892992in}}{\pgfqpoint{1.689459in}{1.896264in}}{\pgfqpoint{1.695283in}{1.902088in}}%
\pgfpathcurveto{\pgfqpoint{1.701107in}{1.907912in}}{\pgfqpoint{1.704379in}{1.915812in}}{\pgfqpoint{1.704379in}{1.924049in}}%
\pgfpathcurveto{\pgfqpoint{1.704379in}{1.932285in}}{\pgfqpoint{1.701107in}{1.940185in}}{\pgfqpoint{1.695283in}{1.946009in}}%
\pgfpathcurveto{\pgfqpoint{1.689459in}{1.951833in}}{\pgfqpoint{1.681559in}{1.955105in}}{\pgfqpoint{1.673323in}{1.955105in}}%
\pgfpathcurveto{\pgfqpoint{1.665086in}{1.955105in}}{\pgfqpoint{1.657186in}{1.951833in}}{\pgfqpoint{1.651362in}{1.946009in}}%
\pgfpathcurveto{\pgfqpoint{1.645538in}{1.940185in}}{\pgfqpoint{1.642266in}{1.932285in}}{\pgfqpoint{1.642266in}{1.924049in}}%
\pgfpathcurveto{\pgfqpoint{1.642266in}{1.915812in}}{\pgfqpoint{1.645538in}{1.907912in}}{\pgfqpoint{1.651362in}{1.902088in}}%
\pgfpathcurveto{\pgfqpoint{1.657186in}{1.896264in}}{\pgfqpoint{1.665086in}{1.892992in}}{\pgfqpoint{1.673323in}{1.892992in}}%
\pgfpathclose%
\pgfusepath{stroke,fill}%
\end{pgfscope}%
\begin{pgfscope}%
\pgfpathrectangle{\pgfqpoint{0.100000in}{0.220728in}}{\pgfqpoint{3.696000in}{3.696000in}}%
\pgfusepath{clip}%
\pgfsetbuttcap%
\pgfsetroundjoin%
\definecolor{currentfill}{rgb}{0.121569,0.466667,0.705882}%
\pgfsetfillcolor{currentfill}%
\pgfsetfillopacity{0.385460}%
\pgfsetlinewidth{1.003750pt}%
\definecolor{currentstroke}{rgb}{0.121569,0.466667,0.705882}%
\pgfsetstrokecolor{currentstroke}%
\pgfsetstrokeopacity{0.385460}%
\pgfsetdash{}{0pt}%
\pgfpathmoveto{\pgfqpoint{1.669587in}{1.890040in}}%
\pgfpathcurveto{\pgfqpoint{1.677824in}{1.890040in}}{\pgfqpoint{1.685724in}{1.893312in}}{\pgfqpoint{1.691548in}{1.899136in}}%
\pgfpathcurveto{\pgfqpoint{1.697372in}{1.904960in}}{\pgfqpoint{1.700644in}{1.912860in}}{\pgfqpoint{1.700644in}{1.921097in}}%
\pgfpathcurveto{\pgfqpoint{1.700644in}{1.929333in}}{\pgfqpoint{1.697372in}{1.937233in}}{\pgfqpoint{1.691548in}{1.943057in}}%
\pgfpathcurveto{\pgfqpoint{1.685724in}{1.948881in}}{\pgfqpoint{1.677824in}{1.952153in}}{\pgfqpoint{1.669587in}{1.952153in}}%
\pgfpathcurveto{\pgfqpoint{1.661351in}{1.952153in}}{\pgfqpoint{1.653451in}{1.948881in}}{\pgfqpoint{1.647627in}{1.943057in}}%
\pgfpathcurveto{\pgfqpoint{1.641803in}{1.937233in}}{\pgfqpoint{1.638531in}{1.929333in}}{\pgfqpoint{1.638531in}{1.921097in}}%
\pgfpathcurveto{\pgfqpoint{1.638531in}{1.912860in}}{\pgfqpoint{1.641803in}{1.904960in}}{\pgfqpoint{1.647627in}{1.899136in}}%
\pgfpathcurveto{\pgfqpoint{1.653451in}{1.893312in}}{\pgfqpoint{1.661351in}{1.890040in}}{\pgfqpoint{1.669587in}{1.890040in}}%
\pgfpathclose%
\pgfusepath{stroke,fill}%
\end{pgfscope}%
\begin{pgfscope}%
\pgfpathrectangle{\pgfqpoint{0.100000in}{0.220728in}}{\pgfqpoint{3.696000in}{3.696000in}}%
\pgfusepath{clip}%
\pgfsetbuttcap%
\pgfsetroundjoin%
\definecolor{currentfill}{rgb}{0.121569,0.466667,0.705882}%
\pgfsetfillcolor{currentfill}%
\pgfsetfillopacity{0.387474}%
\pgfsetlinewidth{1.003750pt}%
\definecolor{currentstroke}{rgb}{0.121569,0.466667,0.705882}%
\pgfsetstrokecolor{currentstroke}%
\pgfsetstrokeopacity{0.387474}%
\pgfsetdash{}{0pt}%
\pgfpathmoveto{\pgfqpoint{1.664162in}{1.885595in}}%
\pgfpathcurveto{\pgfqpoint{1.672398in}{1.885595in}}{\pgfqpoint{1.680298in}{1.888867in}}{\pgfqpoint{1.686122in}{1.894691in}}%
\pgfpathcurveto{\pgfqpoint{1.691946in}{1.900515in}}{\pgfqpoint{1.695219in}{1.908415in}}{\pgfqpoint{1.695219in}{1.916652in}}%
\pgfpathcurveto{\pgfqpoint{1.695219in}{1.924888in}}{\pgfqpoint{1.691946in}{1.932788in}}{\pgfqpoint{1.686122in}{1.938612in}}%
\pgfpathcurveto{\pgfqpoint{1.680298in}{1.944436in}}{\pgfqpoint{1.672398in}{1.947708in}}{\pgfqpoint{1.664162in}{1.947708in}}%
\pgfpathcurveto{\pgfqpoint{1.655926in}{1.947708in}}{\pgfqpoint{1.648026in}{1.944436in}}{\pgfqpoint{1.642202in}{1.938612in}}%
\pgfpathcurveto{\pgfqpoint{1.636378in}{1.932788in}}{\pgfqpoint{1.633106in}{1.924888in}}{\pgfqpoint{1.633106in}{1.916652in}}%
\pgfpathcurveto{\pgfqpoint{1.633106in}{1.908415in}}{\pgfqpoint{1.636378in}{1.900515in}}{\pgfqpoint{1.642202in}{1.894691in}}%
\pgfpathcurveto{\pgfqpoint{1.648026in}{1.888867in}}{\pgfqpoint{1.655926in}{1.885595in}}{\pgfqpoint{1.664162in}{1.885595in}}%
\pgfpathclose%
\pgfusepath{stroke,fill}%
\end{pgfscope}%
\begin{pgfscope}%
\pgfpathrectangle{\pgfqpoint{0.100000in}{0.220728in}}{\pgfqpoint{3.696000in}{3.696000in}}%
\pgfusepath{clip}%
\pgfsetbuttcap%
\pgfsetroundjoin%
\definecolor{currentfill}{rgb}{0.121569,0.466667,0.705882}%
\pgfsetfillcolor{currentfill}%
\pgfsetfillopacity{0.388537}%
\pgfsetlinewidth{1.003750pt}%
\definecolor{currentstroke}{rgb}{0.121569,0.466667,0.705882}%
\pgfsetstrokecolor{currentstroke}%
\pgfsetstrokeopacity{0.388537}%
\pgfsetdash{}{0pt}%
\pgfpathmoveto{\pgfqpoint{1.659197in}{1.882788in}}%
\pgfpathcurveto{\pgfqpoint{1.667433in}{1.882788in}}{\pgfqpoint{1.675333in}{1.886060in}}{\pgfqpoint{1.681157in}{1.891884in}}%
\pgfpathcurveto{\pgfqpoint{1.686981in}{1.897708in}}{\pgfqpoint{1.690253in}{1.905608in}}{\pgfqpoint{1.690253in}{1.913844in}}%
\pgfpathcurveto{\pgfqpoint{1.690253in}{1.922080in}}{\pgfqpoint{1.686981in}{1.929980in}}{\pgfqpoint{1.681157in}{1.935804in}}%
\pgfpathcurveto{\pgfqpoint{1.675333in}{1.941628in}}{\pgfqpoint{1.667433in}{1.944901in}}{\pgfqpoint{1.659197in}{1.944901in}}%
\pgfpathcurveto{\pgfqpoint{1.650961in}{1.944901in}}{\pgfqpoint{1.643061in}{1.941628in}}{\pgfqpoint{1.637237in}{1.935804in}}%
\pgfpathcurveto{\pgfqpoint{1.631413in}{1.929980in}}{\pgfqpoint{1.628140in}{1.922080in}}{\pgfqpoint{1.628140in}{1.913844in}}%
\pgfpathcurveto{\pgfqpoint{1.628140in}{1.905608in}}{\pgfqpoint{1.631413in}{1.897708in}}{\pgfqpoint{1.637237in}{1.891884in}}%
\pgfpathcurveto{\pgfqpoint{1.643061in}{1.886060in}}{\pgfqpoint{1.650961in}{1.882788in}}{\pgfqpoint{1.659197in}{1.882788in}}%
\pgfpathclose%
\pgfusepath{stroke,fill}%
\end{pgfscope}%
\begin{pgfscope}%
\pgfpathrectangle{\pgfqpoint{0.100000in}{0.220728in}}{\pgfqpoint{3.696000in}{3.696000in}}%
\pgfusepath{clip}%
\pgfsetbuttcap%
\pgfsetroundjoin%
\definecolor{currentfill}{rgb}{0.121569,0.466667,0.705882}%
\pgfsetfillcolor{currentfill}%
\pgfsetfillopacity{0.389859}%
\pgfsetlinewidth{1.003750pt}%
\definecolor{currentstroke}{rgb}{0.121569,0.466667,0.705882}%
\pgfsetstrokecolor{currentstroke}%
\pgfsetstrokeopacity{0.389859}%
\pgfsetdash{}{0pt}%
\pgfpathmoveto{\pgfqpoint{1.656350in}{1.882612in}}%
\pgfpathcurveto{\pgfqpoint{1.664586in}{1.882612in}}{\pgfqpoint{1.672486in}{1.885884in}}{\pgfqpoint{1.678310in}{1.891708in}}%
\pgfpathcurveto{\pgfqpoint{1.684134in}{1.897532in}}{\pgfqpoint{1.687406in}{1.905432in}}{\pgfqpoint{1.687406in}{1.913668in}}%
\pgfpathcurveto{\pgfqpoint{1.687406in}{1.921904in}}{\pgfqpoint{1.684134in}{1.929804in}}{\pgfqpoint{1.678310in}{1.935628in}}%
\pgfpathcurveto{\pgfqpoint{1.672486in}{1.941452in}}{\pgfqpoint{1.664586in}{1.944725in}}{\pgfqpoint{1.656350in}{1.944725in}}%
\pgfpathcurveto{\pgfqpoint{1.648113in}{1.944725in}}{\pgfqpoint{1.640213in}{1.941452in}}{\pgfqpoint{1.634389in}{1.935628in}}%
\pgfpathcurveto{\pgfqpoint{1.628566in}{1.929804in}}{\pgfqpoint{1.625293in}{1.921904in}}{\pgfqpoint{1.625293in}{1.913668in}}%
\pgfpathcurveto{\pgfqpoint{1.625293in}{1.905432in}}{\pgfqpoint{1.628566in}{1.897532in}}{\pgfqpoint{1.634389in}{1.891708in}}%
\pgfpathcurveto{\pgfqpoint{1.640213in}{1.885884in}}{\pgfqpoint{1.648113in}{1.882612in}}{\pgfqpoint{1.656350in}{1.882612in}}%
\pgfpathclose%
\pgfusepath{stroke,fill}%
\end{pgfscope}%
\begin{pgfscope}%
\pgfpathrectangle{\pgfqpoint{0.100000in}{0.220728in}}{\pgfqpoint{3.696000in}{3.696000in}}%
\pgfusepath{clip}%
\pgfsetbuttcap%
\pgfsetroundjoin%
\definecolor{currentfill}{rgb}{0.121569,0.466667,0.705882}%
\pgfsetfillcolor{currentfill}%
\pgfsetfillopacity{0.390637}%
\pgfsetlinewidth{1.003750pt}%
\definecolor{currentstroke}{rgb}{0.121569,0.466667,0.705882}%
\pgfsetstrokecolor{currentstroke}%
\pgfsetstrokeopacity{0.390637}%
\pgfsetdash{}{0pt}%
\pgfpathmoveto{\pgfqpoint{1.999286in}{1.961875in}}%
\pgfpathcurveto{\pgfqpoint{2.007522in}{1.961875in}}{\pgfqpoint{2.015422in}{1.965148in}}{\pgfqpoint{2.021246in}{1.970972in}}%
\pgfpathcurveto{\pgfqpoint{2.027070in}{1.976795in}}{\pgfqpoint{2.030342in}{1.984696in}}{\pgfqpoint{2.030342in}{1.992932in}}%
\pgfpathcurveto{\pgfqpoint{2.030342in}{2.001168in}}{\pgfqpoint{2.027070in}{2.009068in}}{\pgfqpoint{2.021246in}{2.014892in}}%
\pgfpathcurveto{\pgfqpoint{2.015422in}{2.020716in}}{\pgfqpoint{2.007522in}{2.023988in}}{\pgfqpoint{1.999286in}{2.023988in}}%
\pgfpathcurveto{\pgfqpoint{1.991050in}{2.023988in}}{\pgfqpoint{1.983150in}{2.020716in}}{\pgfqpoint{1.977326in}{2.014892in}}%
\pgfpathcurveto{\pgfqpoint{1.971502in}{2.009068in}}{\pgfqpoint{1.968229in}{2.001168in}}{\pgfqpoint{1.968229in}{1.992932in}}%
\pgfpathcurveto{\pgfqpoint{1.968229in}{1.984696in}}{\pgfqpoint{1.971502in}{1.976795in}}{\pgfqpoint{1.977326in}{1.970972in}}%
\pgfpathcurveto{\pgfqpoint{1.983150in}{1.965148in}}{\pgfqpoint{1.991050in}{1.961875in}}{\pgfqpoint{1.999286in}{1.961875in}}%
\pgfpathclose%
\pgfusepath{stroke,fill}%
\end{pgfscope}%
\begin{pgfscope}%
\pgfpathrectangle{\pgfqpoint{0.100000in}{0.220728in}}{\pgfqpoint{3.696000in}{3.696000in}}%
\pgfusepath{clip}%
\pgfsetbuttcap%
\pgfsetroundjoin%
\definecolor{currentfill}{rgb}{0.121569,0.466667,0.705882}%
\pgfsetfillcolor{currentfill}%
\pgfsetfillopacity{0.391968}%
\pgfsetlinewidth{1.003750pt}%
\definecolor{currentstroke}{rgb}{0.121569,0.466667,0.705882}%
\pgfsetstrokecolor{currentstroke}%
\pgfsetstrokeopacity{0.391968}%
\pgfsetdash{}{0pt}%
\pgfpathmoveto{\pgfqpoint{1.650340in}{1.881447in}}%
\pgfpathcurveto{\pgfqpoint{1.658576in}{1.881447in}}{\pgfqpoint{1.666476in}{1.884719in}}{\pgfqpoint{1.672300in}{1.890543in}}%
\pgfpathcurveto{\pgfqpoint{1.678124in}{1.896367in}}{\pgfqpoint{1.681396in}{1.904267in}}{\pgfqpoint{1.681396in}{1.912503in}}%
\pgfpathcurveto{\pgfqpoint{1.681396in}{1.920739in}}{\pgfqpoint{1.678124in}{1.928639in}}{\pgfqpoint{1.672300in}{1.934463in}}%
\pgfpathcurveto{\pgfqpoint{1.666476in}{1.940287in}}{\pgfqpoint{1.658576in}{1.943560in}}{\pgfqpoint{1.650340in}{1.943560in}}%
\pgfpathcurveto{\pgfqpoint{1.642104in}{1.943560in}}{\pgfqpoint{1.634204in}{1.940287in}}{\pgfqpoint{1.628380in}{1.934463in}}%
\pgfpathcurveto{\pgfqpoint{1.622556in}{1.928639in}}{\pgfqpoint{1.619283in}{1.920739in}}{\pgfqpoint{1.619283in}{1.912503in}}%
\pgfpathcurveto{\pgfqpoint{1.619283in}{1.904267in}}{\pgfqpoint{1.622556in}{1.896367in}}{\pgfqpoint{1.628380in}{1.890543in}}%
\pgfpathcurveto{\pgfqpoint{1.634204in}{1.884719in}}{\pgfqpoint{1.642104in}{1.881447in}}{\pgfqpoint{1.650340in}{1.881447in}}%
\pgfpathclose%
\pgfusepath{stroke,fill}%
\end{pgfscope}%
\begin{pgfscope}%
\pgfpathrectangle{\pgfqpoint{0.100000in}{0.220728in}}{\pgfqpoint{3.696000in}{3.696000in}}%
\pgfusepath{clip}%
\pgfsetbuttcap%
\pgfsetroundjoin%
\definecolor{currentfill}{rgb}{0.121569,0.466667,0.705882}%
\pgfsetfillcolor{currentfill}%
\pgfsetfillopacity{0.393095}%
\pgfsetlinewidth{1.003750pt}%
\definecolor{currentstroke}{rgb}{0.121569,0.466667,0.705882}%
\pgfsetstrokecolor{currentstroke}%
\pgfsetstrokeopacity{0.393095}%
\pgfsetdash{}{0pt}%
\pgfpathmoveto{\pgfqpoint{1.645501in}{1.877382in}}%
\pgfpathcurveto{\pgfqpoint{1.653738in}{1.877382in}}{\pgfqpoint{1.661638in}{1.880655in}}{\pgfqpoint{1.667462in}{1.886479in}}%
\pgfpathcurveto{\pgfqpoint{1.673286in}{1.892303in}}{\pgfqpoint{1.676558in}{1.900203in}}{\pgfqpoint{1.676558in}{1.908439in}}%
\pgfpathcurveto{\pgfqpoint{1.676558in}{1.916675in}}{\pgfqpoint{1.673286in}{1.924575in}}{\pgfqpoint{1.667462in}{1.930399in}}%
\pgfpathcurveto{\pgfqpoint{1.661638in}{1.936223in}}{\pgfqpoint{1.653738in}{1.939495in}}{\pgfqpoint{1.645501in}{1.939495in}}%
\pgfpathcurveto{\pgfqpoint{1.637265in}{1.939495in}}{\pgfqpoint{1.629365in}{1.936223in}}{\pgfqpoint{1.623541in}{1.930399in}}%
\pgfpathcurveto{\pgfqpoint{1.617717in}{1.924575in}}{\pgfqpoint{1.614445in}{1.916675in}}{\pgfqpoint{1.614445in}{1.908439in}}%
\pgfpathcurveto{\pgfqpoint{1.614445in}{1.900203in}}{\pgfqpoint{1.617717in}{1.892303in}}{\pgfqpoint{1.623541in}{1.886479in}}%
\pgfpathcurveto{\pgfqpoint{1.629365in}{1.880655in}}{\pgfqpoint{1.637265in}{1.877382in}}{\pgfqpoint{1.645501in}{1.877382in}}%
\pgfpathclose%
\pgfusepath{stroke,fill}%
\end{pgfscope}%
\begin{pgfscope}%
\pgfpathrectangle{\pgfqpoint{0.100000in}{0.220728in}}{\pgfqpoint{3.696000in}{3.696000in}}%
\pgfusepath{clip}%
\pgfsetbuttcap%
\pgfsetroundjoin%
\definecolor{currentfill}{rgb}{0.121569,0.466667,0.705882}%
\pgfsetfillcolor{currentfill}%
\pgfsetfillopacity{0.394457}%
\pgfsetlinewidth{1.003750pt}%
\definecolor{currentstroke}{rgb}{0.121569,0.466667,0.705882}%
\pgfsetstrokecolor{currentstroke}%
\pgfsetstrokeopacity{0.394457}%
\pgfsetdash{}{0pt}%
\pgfpathmoveto{\pgfqpoint{1.641416in}{1.876903in}}%
\pgfpathcurveto{\pgfqpoint{1.649652in}{1.876903in}}{\pgfqpoint{1.657552in}{1.880175in}}{\pgfqpoint{1.663376in}{1.885999in}}%
\pgfpathcurveto{\pgfqpoint{1.669200in}{1.891823in}}{\pgfqpoint{1.672472in}{1.899723in}}{\pgfqpoint{1.672472in}{1.907959in}}%
\pgfpathcurveto{\pgfqpoint{1.672472in}{1.916196in}}{\pgfqpoint{1.669200in}{1.924096in}}{\pgfqpoint{1.663376in}{1.929920in}}%
\pgfpathcurveto{\pgfqpoint{1.657552in}{1.935744in}}{\pgfqpoint{1.649652in}{1.939016in}}{\pgfqpoint{1.641416in}{1.939016in}}%
\pgfpathcurveto{\pgfqpoint{1.633179in}{1.939016in}}{\pgfqpoint{1.625279in}{1.935744in}}{\pgfqpoint{1.619455in}{1.929920in}}%
\pgfpathcurveto{\pgfqpoint{1.613631in}{1.924096in}}{\pgfqpoint{1.610359in}{1.916196in}}{\pgfqpoint{1.610359in}{1.907959in}}%
\pgfpathcurveto{\pgfqpoint{1.610359in}{1.899723in}}{\pgfqpoint{1.613631in}{1.891823in}}{\pgfqpoint{1.619455in}{1.885999in}}%
\pgfpathcurveto{\pgfqpoint{1.625279in}{1.880175in}}{\pgfqpoint{1.633179in}{1.876903in}}{\pgfqpoint{1.641416in}{1.876903in}}%
\pgfpathclose%
\pgfusepath{stroke,fill}%
\end{pgfscope}%
\begin{pgfscope}%
\pgfpathrectangle{\pgfqpoint{0.100000in}{0.220728in}}{\pgfqpoint{3.696000in}{3.696000in}}%
\pgfusepath{clip}%
\pgfsetbuttcap%
\pgfsetroundjoin%
\definecolor{currentfill}{rgb}{0.121569,0.466667,0.705882}%
\pgfsetfillcolor{currentfill}%
\pgfsetfillopacity{0.396791}%
\pgfsetlinewidth{1.003750pt}%
\definecolor{currentstroke}{rgb}{0.121569,0.466667,0.705882}%
\pgfsetstrokecolor{currentstroke}%
\pgfsetstrokeopacity{0.396791}%
\pgfsetdash{}{0pt}%
\pgfpathmoveto{\pgfqpoint{1.635309in}{1.873124in}}%
\pgfpathcurveto{\pgfqpoint{1.643545in}{1.873124in}}{\pgfqpoint{1.651445in}{1.876396in}}{\pgfqpoint{1.657269in}{1.882220in}}%
\pgfpathcurveto{\pgfqpoint{1.663093in}{1.888044in}}{\pgfqpoint{1.666365in}{1.895944in}}{\pgfqpoint{1.666365in}{1.904181in}}%
\pgfpathcurveto{\pgfqpoint{1.666365in}{1.912417in}}{\pgfqpoint{1.663093in}{1.920317in}}{\pgfqpoint{1.657269in}{1.926141in}}%
\pgfpathcurveto{\pgfqpoint{1.651445in}{1.931965in}}{\pgfqpoint{1.643545in}{1.935237in}}{\pgfqpoint{1.635309in}{1.935237in}}%
\pgfpathcurveto{\pgfqpoint{1.627073in}{1.935237in}}{\pgfqpoint{1.619173in}{1.931965in}}{\pgfqpoint{1.613349in}{1.926141in}}%
\pgfpathcurveto{\pgfqpoint{1.607525in}{1.920317in}}{\pgfqpoint{1.604252in}{1.912417in}}{\pgfqpoint{1.604252in}{1.904181in}}%
\pgfpathcurveto{\pgfqpoint{1.604252in}{1.895944in}}{\pgfqpoint{1.607525in}{1.888044in}}{\pgfqpoint{1.613349in}{1.882220in}}%
\pgfpathcurveto{\pgfqpoint{1.619173in}{1.876396in}}{\pgfqpoint{1.627073in}{1.873124in}}{\pgfqpoint{1.635309in}{1.873124in}}%
\pgfpathclose%
\pgfusepath{stroke,fill}%
\end{pgfscope}%
\begin{pgfscope}%
\pgfpathrectangle{\pgfqpoint{0.100000in}{0.220728in}}{\pgfqpoint{3.696000in}{3.696000in}}%
\pgfusepath{clip}%
\pgfsetbuttcap%
\pgfsetroundjoin%
\definecolor{currentfill}{rgb}{0.121569,0.466667,0.705882}%
\pgfsetfillcolor{currentfill}%
\pgfsetfillopacity{0.397646}%
\pgfsetlinewidth{1.003750pt}%
\definecolor{currentstroke}{rgb}{0.121569,0.466667,0.705882}%
\pgfsetstrokecolor{currentstroke}%
\pgfsetstrokeopacity{0.397646}%
\pgfsetdash{}{0pt}%
\pgfpathmoveto{\pgfqpoint{1.631726in}{1.869812in}}%
\pgfpathcurveto{\pgfqpoint{1.639963in}{1.869812in}}{\pgfqpoint{1.647863in}{1.873084in}}{\pgfqpoint{1.653687in}{1.878908in}}%
\pgfpathcurveto{\pgfqpoint{1.659511in}{1.884732in}}{\pgfqpoint{1.662783in}{1.892632in}}{\pgfqpoint{1.662783in}{1.900868in}}%
\pgfpathcurveto{\pgfqpoint{1.662783in}{1.909105in}}{\pgfqpoint{1.659511in}{1.917005in}}{\pgfqpoint{1.653687in}{1.922829in}}%
\pgfpathcurveto{\pgfqpoint{1.647863in}{1.928653in}}{\pgfqpoint{1.639963in}{1.931925in}}{\pgfqpoint{1.631726in}{1.931925in}}%
\pgfpathcurveto{\pgfqpoint{1.623490in}{1.931925in}}{\pgfqpoint{1.615590in}{1.928653in}}{\pgfqpoint{1.609766in}{1.922829in}}%
\pgfpathcurveto{\pgfqpoint{1.603942in}{1.917005in}}{\pgfqpoint{1.600670in}{1.909105in}}{\pgfqpoint{1.600670in}{1.900868in}}%
\pgfpathcurveto{\pgfqpoint{1.600670in}{1.892632in}}{\pgfqpoint{1.603942in}{1.884732in}}{\pgfqpoint{1.609766in}{1.878908in}}%
\pgfpathcurveto{\pgfqpoint{1.615590in}{1.873084in}}{\pgfqpoint{1.623490in}{1.869812in}}{\pgfqpoint{1.631726in}{1.869812in}}%
\pgfpathclose%
\pgfusepath{stroke,fill}%
\end{pgfscope}%
\begin{pgfscope}%
\pgfpathrectangle{\pgfqpoint{0.100000in}{0.220728in}}{\pgfqpoint{3.696000in}{3.696000in}}%
\pgfusepath{clip}%
\pgfsetbuttcap%
\pgfsetroundjoin%
\definecolor{currentfill}{rgb}{0.121569,0.466667,0.705882}%
\pgfsetfillcolor{currentfill}%
\pgfsetfillopacity{0.399017}%
\pgfsetlinewidth{1.003750pt}%
\definecolor{currentstroke}{rgb}{0.121569,0.466667,0.705882}%
\pgfsetstrokecolor{currentstroke}%
\pgfsetstrokeopacity{0.399017}%
\pgfsetdash{}{0pt}%
\pgfpathmoveto{\pgfqpoint{2.003842in}{1.958879in}}%
\pgfpathcurveto{\pgfqpoint{2.012078in}{1.958879in}}{\pgfqpoint{2.019978in}{1.962152in}}{\pgfqpoint{2.025802in}{1.967976in}}%
\pgfpathcurveto{\pgfqpoint{2.031626in}{1.973800in}}{\pgfqpoint{2.034898in}{1.981700in}}{\pgfqpoint{2.034898in}{1.989936in}}%
\pgfpathcurveto{\pgfqpoint{2.034898in}{1.998172in}}{\pgfqpoint{2.031626in}{2.006072in}}{\pgfqpoint{2.025802in}{2.011896in}}%
\pgfpathcurveto{\pgfqpoint{2.019978in}{2.017720in}}{\pgfqpoint{2.012078in}{2.020992in}}{\pgfqpoint{2.003842in}{2.020992in}}%
\pgfpathcurveto{\pgfqpoint{1.995606in}{2.020992in}}{\pgfqpoint{1.987706in}{2.017720in}}{\pgfqpoint{1.981882in}{2.011896in}}%
\pgfpathcurveto{\pgfqpoint{1.976058in}{2.006072in}}{\pgfqpoint{1.972785in}{1.998172in}}{\pgfqpoint{1.972785in}{1.989936in}}%
\pgfpathcurveto{\pgfqpoint{1.972785in}{1.981700in}}{\pgfqpoint{1.976058in}{1.973800in}}{\pgfqpoint{1.981882in}{1.967976in}}%
\pgfpathcurveto{\pgfqpoint{1.987706in}{1.962152in}}{\pgfqpoint{1.995606in}{1.958879in}}{\pgfqpoint{2.003842in}{1.958879in}}%
\pgfpathclose%
\pgfusepath{stroke,fill}%
\end{pgfscope}%
\begin{pgfscope}%
\pgfpathrectangle{\pgfqpoint{0.100000in}{0.220728in}}{\pgfqpoint{3.696000in}{3.696000in}}%
\pgfusepath{clip}%
\pgfsetbuttcap%
\pgfsetroundjoin%
\definecolor{currentfill}{rgb}{0.121569,0.466667,0.705882}%
\pgfsetfillcolor{currentfill}%
\pgfsetfillopacity{0.399400}%
\pgfsetlinewidth{1.003750pt}%
\definecolor{currentstroke}{rgb}{0.121569,0.466667,0.705882}%
\pgfsetstrokecolor{currentstroke}%
\pgfsetstrokeopacity{0.399400}%
\pgfsetdash{}{0pt}%
\pgfpathmoveto{\pgfqpoint{1.625033in}{1.865402in}}%
\pgfpathcurveto{\pgfqpoint{1.633269in}{1.865402in}}{\pgfqpoint{1.641169in}{1.868674in}}{\pgfqpoint{1.646993in}{1.874498in}}%
\pgfpathcurveto{\pgfqpoint{1.652817in}{1.880322in}}{\pgfqpoint{1.656090in}{1.888222in}}{\pgfqpoint{1.656090in}{1.896458in}}%
\pgfpathcurveto{\pgfqpoint{1.656090in}{1.904695in}}{\pgfqpoint{1.652817in}{1.912595in}}{\pgfqpoint{1.646993in}{1.918419in}}%
\pgfpathcurveto{\pgfqpoint{1.641169in}{1.924243in}}{\pgfqpoint{1.633269in}{1.927515in}}{\pgfqpoint{1.625033in}{1.927515in}}%
\pgfpathcurveto{\pgfqpoint{1.616797in}{1.927515in}}{\pgfqpoint{1.608897in}{1.924243in}}{\pgfqpoint{1.603073in}{1.918419in}}%
\pgfpathcurveto{\pgfqpoint{1.597249in}{1.912595in}}{\pgfqpoint{1.593977in}{1.904695in}}{\pgfqpoint{1.593977in}{1.896458in}}%
\pgfpathcurveto{\pgfqpoint{1.593977in}{1.888222in}}{\pgfqpoint{1.597249in}{1.880322in}}{\pgfqpoint{1.603073in}{1.874498in}}%
\pgfpathcurveto{\pgfqpoint{1.608897in}{1.868674in}}{\pgfqpoint{1.616797in}{1.865402in}}{\pgfqpoint{1.625033in}{1.865402in}}%
\pgfpathclose%
\pgfusepath{stroke,fill}%
\end{pgfscope}%
\begin{pgfscope}%
\pgfpathrectangle{\pgfqpoint{0.100000in}{0.220728in}}{\pgfqpoint{3.696000in}{3.696000in}}%
\pgfusepath{clip}%
\pgfsetbuttcap%
\pgfsetroundjoin%
\definecolor{currentfill}{rgb}{0.121569,0.466667,0.705882}%
\pgfsetfillcolor{currentfill}%
\pgfsetfillopacity{0.403428}%
\pgfsetlinewidth{1.003750pt}%
\definecolor{currentstroke}{rgb}{0.121569,0.466667,0.705882}%
\pgfsetstrokecolor{currentstroke}%
\pgfsetstrokeopacity{0.403428}%
\pgfsetdash{}{0pt}%
\pgfpathmoveto{\pgfqpoint{2.005560in}{1.955447in}}%
\pgfpathcurveto{\pgfqpoint{2.013796in}{1.955447in}}{\pgfqpoint{2.021696in}{1.958720in}}{\pgfqpoint{2.027520in}{1.964544in}}%
\pgfpathcurveto{\pgfqpoint{2.033344in}{1.970368in}}{\pgfqpoint{2.036617in}{1.978268in}}{\pgfqpoint{2.036617in}{1.986504in}}%
\pgfpathcurveto{\pgfqpoint{2.036617in}{1.994740in}}{\pgfqpoint{2.033344in}{2.002640in}}{\pgfqpoint{2.027520in}{2.008464in}}%
\pgfpathcurveto{\pgfqpoint{2.021696in}{2.014288in}}{\pgfqpoint{2.013796in}{2.017560in}}{\pgfqpoint{2.005560in}{2.017560in}}%
\pgfpathcurveto{\pgfqpoint{1.997324in}{2.017560in}}{\pgfqpoint{1.989424in}{2.014288in}}{\pgfqpoint{1.983600in}{2.008464in}}%
\pgfpathcurveto{\pgfqpoint{1.977776in}{2.002640in}}{\pgfqpoint{1.974504in}{1.994740in}}{\pgfqpoint{1.974504in}{1.986504in}}%
\pgfpathcurveto{\pgfqpoint{1.974504in}{1.978268in}}{\pgfqpoint{1.977776in}{1.970368in}}{\pgfqpoint{1.983600in}{1.964544in}}%
\pgfpathcurveto{\pgfqpoint{1.989424in}{1.958720in}}{\pgfqpoint{1.997324in}{1.955447in}}{\pgfqpoint{2.005560in}{1.955447in}}%
\pgfpathclose%
\pgfusepath{stroke,fill}%
\end{pgfscope}%
\begin{pgfscope}%
\pgfpathrectangle{\pgfqpoint{0.100000in}{0.220728in}}{\pgfqpoint{3.696000in}{3.696000in}}%
\pgfusepath{clip}%
\pgfsetbuttcap%
\pgfsetroundjoin%
\definecolor{currentfill}{rgb}{0.121569,0.466667,0.705882}%
\pgfsetfillcolor{currentfill}%
\pgfsetfillopacity{0.404405}%
\pgfsetlinewidth{1.003750pt}%
\definecolor{currentstroke}{rgb}{0.121569,0.466667,0.705882}%
\pgfsetstrokecolor{currentstroke}%
\pgfsetstrokeopacity{0.404405}%
\pgfsetdash{}{0pt}%
\pgfpathmoveto{\pgfqpoint{1.614589in}{1.866997in}}%
\pgfpathcurveto{\pgfqpoint{1.622825in}{1.866997in}}{\pgfqpoint{1.630725in}{1.870270in}}{\pgfqpoint{1.636549in}{1.876093in}}%
\pgfpathcurveto{\pgfqpoint{1.642373in}{1.881917in}}{\pgfqpoint{1.645645in}{1.889817in}}{\pgfqpoint{1.645645in}{1.898054in}}%
\pgfpathcurveto{\pgfqpoint{1.645645in}{1.906290in}}{\pgfqpoint{1.642373in}{1.914190in}}{\pgfqpoint{1.636549in}{1.920014in}}%
\pgfpathcurveto{\pgfqpoint{1.630725in}{1.925838in}}{\pgfqpoint{1.622825in}{1.929110in}}{\pgfqpoint{1.614589in}{1.929110in}}%
\pgfpathcurveto{\pgfqpoint{1.606352in}{1.929110in}}{\pgfqpoint{1.598452in}{1.925838in}}{\pgfqpoint{1.592628in}{1.920014in}}%
\pgfpathcurveto{\pgfqpoint{1.586804in}{1.914190in}}{\pgfqpoint{1.583532in}{1.906290in}}{\pgfqpoint{1.583532in}{1.898054in}}%
\pgfpathcurveto{\pgfqpoint{1.583532in}{1.889817in}}{\pgfqpoint{1.586804in}{1.881917in}}{\pgfqpoint{1.592628in}{1.876093in}}%
\pgfpathcurveto{\pgfqpoint{1.598452in}{1.870270in}}{\pgfqpoint{1.606352in}{1.866997in}}{\pgfqpoint{1.614589in}{1.866997in}}%
\pgfpathclose%
\pgfusepath{stroke,fill}%
\end{pgfscope}%
\begin{pgfscope}%
\pgfpathrectangle{\pgfqpoint{0.100000in}{0.220728in}}{\pgfqpoint{3.696000in}{3.696000in}}%
\pgfusepath{clip}%
\pgfsetbuttcap%
\pgfsetroundjoin%
\definecolor{currentfill}{rgb}{0.121569,0.466667,0.705882}%
\pgfsetfillcolor{currentfill}%
\pgfsetfillopacity{0.406612}%
\pgfsetlinewidth{1.003750pt}%
\definecolor{currentstroke}{rgb}{0.121569,0.466667,0.705882}%
\pgfsetstrokecolor{currentstroke}%
\pgfsetstrokeopacity{0.406612}%
\pgfsetdash{}{0pt}%
\pgfpathmoveto{\pgfqpoint{1.604664in}{1.859938in}}%
\pgfpathcurveto{\pgfqpoint{1.612900in}{1.859938in}}{\pgfqpoint{1.620801in}{1.863211in}}{\pgfqpoint{1.626624in}{1.869035in}}%
\pgfpathcurveto{\pgfqpoint{1.632448in}{1.874859in}}{\pgfqpoint{1.635721in}{1.882759in}}{\pgfqpoint{1.635721in}{1.890995in}}%
\pgfpathcurveto{\pgfqpoint{1.635721in}{1.899231in}}{\pgfqpoint{1.632448in}{1.907131in}}{\pgfqpoint{1.626624in}{1.912955in}}%
\pgfpathcurveto{\pgfqpoint{1.620801in}{1.918779in}}{\pgfqpoint{1.612900in}{1.922051in}}{\pgfqpoint{1.604664in}{1.922051in}}%
\pgfpathcurveto{\pgfqpoint{1.596428in}{1.922051in}}{\pgfqpoint{1.588528in}{1.918779in}}{\pgfqpoint{1.582704in}{1.912955in}}%
\pgfpathcurveto{\pgfqpoint{1.576880in}{1.907131in}}{\pgfqpoint{1.573608in}{1.899231in}}{\pgfqpoint{1.573608in}{1.890995in}}%
\pgfpathcurveto{\pgfqpoint{1.573608in}{1.882759in}}{\pgfqpoint{1.576880in}{1.874859in}}{\pgfqpoint{1.582704in}{1.869035in}}%
\pgfpathcurveto{\pgfqpoint{1.588528in}{1.863211in}}{\pgfqpoint{1.596428in}{1.859938in}}{\pgfqpoint{1.604664in}{1.859938in}}%
\pgfpathclose%
\pgfusepath{stroke,fill}%
\end{pgfscope}%
\begin{pgfscope}%
\pgfpathrectangle{\pgfqpoint{0.100000in}{0.220728in}}{\pgfqpoint{3.696000in}{3.696000in}}%
\pgfusepath{clip}%
\pgfsetbuttcap%
\pgfsetroundjoin%
\definecolor{currentfill}{rgb}{0.121569,0.466667,0.705882}%
\pgfsetfillcolor{currentfill}%
\pgfsetfillopacity{0.408016}%
\pgfsetlinewidth{1.003750pt}%
\definecolor{currentstroke}{rgb}{0.121569,0.466667,0.705882}%
\pgfsetstrokecolor{currentstroke}%
\pgfsetstrokeopacity{0.408016}%
\pgfsetdash{}{0pt}%
\pgfpathmoveto{\pgfqpoint{1.598023in}{1.855828in}}%
\pgfpathcurveto{\pgfqpoint{1.606259in}{1.855828in}}{\pgfqpoint{1.614159in}{1.859100in}}{\pgfqpoint{1.619983in}{1.864924in}}%
\pgfpathcurveto{\pgfqpoint{1.625807in}{1.870748in}}{\pgfqpoint{1.629079in}{1.878648in}}{\pgfqpoint{1.629079in}{1.886884in}}%
\pgfpathcurveto{\pgfqpoint{1.629079in}{1.895121in}}{\pgfqpoint{1.625807in}{1.903021in}}{\pgfqpoint{1.619983in}{1.908845in}}%
\pgfpathcurveto{\pgfqpoint{1.614159in}{1.914669in}}{\pgfqpoint{1.606259in}{1.917941in}}{\pgfqpoint{1.598023in}{1.917941in}}%
\pgfpathcurveto{\pgfqpoint{1.589786in}{1.917941in}}{\pgfqpoint{1.581886in}{1.914669in}}{\pgfqpoint{1.576062in}{1.908845in}}%
\pgfpathcurveto{\pgfqpoint{1.570238in}{1.903021in}}{\pgfqpoint{1.566966in}{1.895121in}}{\pgfqpoint{1.566966in}{1.886884in}}%
\pgfpathcurveto{\pgfqpoint{1.566966in}{1.878648in}}{\pgfqpoint{1.570238in}{1.870748in}}{\pgfqpoint{1.576062in}{1.864924in}}%
\pgfpathcurveto{\pgfqpoint{1.581886in}{1.859100in}}{\pgfqpoint{1.589786in}{1.855828in}}{\pgfqpoint{1.598023in}{1.855828in}}%
\pgfpathclose%
\pgfusepath{stroke,fill}%
\end{pgfscope}%
\begin{pgfscope}%
\pgfpathrectangle{\pgfqpoint{0.100000in}{0.220728in}}{\pgfqpoint{3.696000in}{3.696000in}}%
\pgfusepath{clip}%
\pgfsetbuttcap%
\pgfsetroundjoin%
\definecolor{currentfill}{rgb}{0.121569,0.466667,0.705882}%
\pgfsetfillcolor{currentfill}%
\pgfsetfillopacity{0.408143}%
\pgfsetlinewidth{1.003750pt}%
\definecolor{currentstroke}{rgb}{0.121569,0.466667,0.705882}%
\pgfsetstrokecolor{currentstroke}%
\pgfsetstrokeopacity{0.408143}%
\pgfsetdash{}{0pt}%
\pgfpathmoveto{\pgfqpoint{2.005466in}{1.950645in}}%
\pgfpathcurveto{\pgfqpoint{2.013702in}{1.950645in}}{\pgfqpoint{2.021602in}{1.953917in}}{\pgfqpoint{2.027426in}{1.959741in}}%
\pgfpathcurveto{\pgfqpoint{2.033250in}{1.965565in}}{\pgfqpoint{2.036523in}{1.973465in}}{\pgfqpoint{2.036523in}{1.981702in}}%
\pgfpathcurveto{\pgfqpoint{2.036523in}{1.989938in}}{\pgfqpoint{2.033250in}{1.997838in}}{\pgfqpoint{2.027426in}{2.003662in}}%
\pgfpathcurveto{\pgfqpoint{2.021602in}{2.009486in}}{\pgfqpoint{2.013702in}{2.012758in}}{\pgfqpoint{2.005466in}{2.012758in}}%
\pgfpathcurveto{\pgfqpoint{1.997230in}{2.012758in}}{\pgfqpoint{1.989330in}{2.009486in}}{\pgfqpoint{1.983506in}{2.003662in}}%
\pgfpathcurveto{\pgfqpoint{1.977682in}{1.997838in}}{\pgfqpoint{1.974410in}{1.989938in}}{\pgfqpoint{1.974410in}{1.981702in}}%
\pgfpathcurveto{\pgfqpoint{1.974410in}{1.973465in}}{\pgfqpoint{1.977682in}{1.965565in}}{\pgfqpoint{1.983506in}{1.959741in}}%
\pgfpathcurveto{\pgfqpoint{1.989330in}{1.953917in}}{\pgfqpoint{1.997230in}{1.950645in}}{\pgfqpoint{2.005466in}{1.950645in}}%
\pgfpathclose%
\pgfusepath{stroke,fill}%
\end{pgfscope}%
\begin{pgfscope}%
\pgfpathrectangle{\pgfqpoint{0.100000in}{0.220728in}}{\pgfqpoint{3.696000in}{3.696000in}}%
\pgfusepath{clip}%
\pgfsetbuttcap%
\pgfsetroundjoin%
\definecolor{currentfill}{rgb}{0.121569,0.466667,0.705882}%
\pgfsetfillcolor{currentfill}%
\pgfsetfillopacity{0.412603}%
\pgfsetlinewidth{1.003750pt}%
\definecolor{currentstroke}{rgb}{0.121569,0.466667,0.705882}%
\pgfsetstrokecolor{currentstroke}%
\pgfsetstrokeopacity{0.412603}%
\pgfsetdash{}{0pt}%
\pgfpathmoveto{\pgfqpoint{1.589172in}{1.856909in}}%
\pgfpathcurveto{\pgfqpoint{1.597409in}{1.856909in}}{\pgfqpoint{1.605309in}{1.860181in}}{\pgfqpoint{1.611133in}{1.866005in}}%
\pgfpathcurveto{\pgfqpoint{1.616956in}{1.871829in}}{\pgfqpoint{1.620229in}{1.879729in}}{\pgfqpoint{1.620229in}{1.887966in}}%
\pgfpathcurveto{\pgfqpoint{1.620229in}{1.896202in}}{\pgfqpoint{1.616956in}{1.904102in}}{\pgfqpoint{1.611133in}{1.909926in}}%
\pgfpathcurveto{\pgfqpoint{1.605309in}{1.915750in}}{\pgfqpoint{1.597409in}{1.919022in}}{\pgfqpoint{1.589172in}{1.919022in}}%
\pgfpathcurveto{\pgfqpoint{1.580936in}{1.919022in}}{\pgfqpoint{1.573036in}{1.915750in}}{\pgfqpoint{1.567212in}{1.909926in}}%
\pgfpathcurveto{\pgfqpoint{1.561388in}{1.904102in}}{\pgfqpoint{1.558116in}{1.896202in}}{\pgfqpoint{1.558116in}{1.887966in}}%
\pgfpathcurveto{\pgfqpoint{1.558116in}{1.879729in}}{\pgfqpoint{1.561388in}{1.871829in}}{\pgfqpoint{1.567212in}{1.866005in}}%
\pgfpathcurveto{\pgfqpoint{1.573036in}{1.860181in}}{\pgfqpoint{1.580936in}{1.856909in}}{\pgfqpoint{1.589172in}{1.856909in}}%
\pgfpathclose%
\pgfusepath{stroke,fill}%
\end{pgfscope}%
\begin{pgfscope}%
\pgfpathrectangle{\pgfqpoint{0.100000in}{0.220728in}}{\pgfqpoint{3.696000in}{3.696000in}}%
\pgfusepath{clip}%
\pgfsetbuttcap%
\pgfsetroundjoin%
\definecolor{currentfill}{rgb}{0.121569,0.466667,0.705882}%
\pgfsetfillcolor{currentfill}%
\pgfsetfillopacity{0.413679}%
\pgfsetlinewidth{1.003750pt}%
\definecolor{currentstroke}{rgb}{0.121569,0.466667,0.705882}%
\pgfsetstrokecolor{currentstroke}%
\pgfsetstrokeopacity{0.413679}%
\pgfsetdash{}{0pt}%
\pgfpathmoveto{\pgfqpoint{2.008475in}{1.947454in}}%
\pgfpathcurveto{\pgfqpoint{2.016711in}{1.947454in}}{\pgfqpoint{2.024612in}{1.950726in}}{\pgfqpoint{2.030435in}{1.956550in}}%
\pgfpathcurveto{\pgfqpoint{2.036259in}{1.962374in}}{\pgfqpoint{2.039532in}{1.970274in}}{\pgfqpoint{2.039532in}{1.978511in}}%
\pgfpathcurveto{\pgfqpoint{2.039532in}{1.986747in}}{\pgfqpoint{2.036259in}{1.994647in}}{\pgfqpoint{2.030435in}{2.000471in}}%
\pgfpathcurveto{\pgfqpoint{2.024612in}{2.006295in}}{\pgfqpoint{2.016711in}{2.009567in}}{\pgfqpoint{2.008475in}{2.009567in}}%
\pgfpathcurveto{\pgfqpoint{2.000239in}{2.009567in}}{\pgfqpoint{1.992339in}{2.006295in}}{\pgfqpoint{1.986515in}{2.000471in}}%
\pgfpathcurveto{\pgfqpoint{1.980691in}{1.994647in}}{\pgfqpoint{1.977419in}{1.986747in}}{\pgfqpoint{1.977419in}{1.978511in}}%
\pgfpathcurveto{\pgfqpoint{1.977419in}{1.970274in}}{\pgfqpoint{1.980691in}{1.962374in}}{\pgfqpoint{1.986515in}{1.956550in}}%
\pgfpathcurveto{\pgfqpoint{1.992339in}{1.950726in}}{\pgfqpoint{2.000239in}{1.947454in}}{\pgfqpoint{2.008475in}{1.947454in}}%
\pgfpathclose%
\pgfusepath{stroke,fill}%
\end{pgfscope}%
\begin{pgfscope}%
\pgfpathrectangle{\pgfqpoint{0.100000in}{0.220728in}}{\pgfqpoint{3.696000in}{3.696000in}}%
\pgfusepath{clip}%
\pgfsetbuttcap%
\pgfsetroundjoin%
\definecolor{currentfill}{rgb}{0.121569,0.466667,0.705882}%
\pgfsetfillcolor{currentfill}%
\pgfsetfillopacity{0.414631}%
\pgfsetlinewidth{1.003750pt}%
\definecolor{currentstroke}{rgb}{0.121569,0.466667,0.705882}%
\pgfsetstrokecolor{currentstroke}%
\pgfsetstrokeopacity{0.414631}%
\pgfsetdash{}{0pt}%
\pgfpathmoveto{\pgfqpoint{1.580538in}{1.852905in}}%
\pgfpathcurveto{\pgfqpoint{1.588774in}{1.852905in}}{\pgfqpoint{1.596674in}{1.856177in}}{\pgfqpoint{1.602498in}{1.862001in}}%
\pgfpathcurveto{\pgfqpoint{1.608322in}{1.867825in}}{\pgfqpoint{1.611594in}{1.875725in}}{\pgfqpoint{1.611594in}{1.883962in}}%
\pgfpathcurveto{\pgfqpoint{1.611594in}{1.892198in}}{\pgfqpoint{1.608322in}{1.900098in}}{\pgfqpoint{1.602498in}{1.905922in}}%
\pgfpathcurveto{\pgfqpoint{1.596674in}{1.911746in}}{\pgfqpoint{1.588774in}{1.915018in}}{\pgfqpoint{1.580538in}{1.915018in}}%
\pgfpathcurveto{\pgfqpoint{1.572301in}{1.915018in}}{\pgfqpoint{1.564401in}{1.911746in}}{\pgfqpoint{1.558578in}{1.905922in}}%
\pgfpathcurveto{\pgfqpoint{1.552754in}{1.900098in}}{\pgfqpoint{1.549481in}{1.892198in}}{\pgfqpoint{1.549481in}{1.883962in}}%
\pgfpathcurveto{\pgfqpoint{1.549481in}{1.875725in}}{\pgfqpoint{1.552754in}{1.867825in}}{\pgfqpoint{1.558578in}{1.862001in}}%
\pgfpathcurveto{\pgfqpoint{1.564401in}{1.856177in}}{\pgfqpoint{1.572301in}{1.852905in}}{\pgfqpoint{1.580538in}{1.852905in}}%
\pgfpathclose%
\pgfusepath{stroke,fill}%
\end{pgfscope}%
\begin{pgfscope}%
\pgfpathrectangle{\pgfqpoint{0.100000in}{0.220728in}}{\pgfqpoint{3.696000in}{3.696000in}}%
\pgfusepath{clip}%
\pgfsetbuttcap%
\pgfsetroundjoin%
\definecolor{currentfill}{rgb}{0.121569,0.466667,0.705882}%
\pgfsetfillcolor{currentfill}%
\pgfsetfillopacity{0.415564}%
\pgfsetlinewidth{1.003750pt}%
\definecolor{currentstroke}{rgb}{0.121569,0.466667,0.705882}%
\pgfsetstrokecolor{currentstroke}%
\pgfsetstrokeopacity{0.415564}%
\pgfsetdash{}{0pt}%
\pgfpathmoveto{\pgfqpoint{1.575400in}{1.847402in}}%
\pgfpathcurveto{\pgfqpoint{1.583636in}{1.847402in}}{\pgfqpoint{1.591536in}{1.850674in}}{\pgfqpoint{1.597360in}{1.856498in}}%
\pgfpathcurveto{\pgfqpoint{1.603184in}{1.862322in}}{\pgfqpoint{1.606456in}{1.870222in}}{\pgfqpoint{1.606456in}{1.878458in}}%
\pgfpathcurveto{\pgfqpoint{1.606456in}{1.886695in}}{\pgfqpoint{1.603184in}{1.894595in}}{\pgfqpoint{1.597360in}{1.900419in}}%
\pgfpathcurveto{\pgfqpoint{1.591536in}{1.906243in}}{\pgfqpoint{1.583636in}{1.909515in}}{\pgfqpoint{1.575400in}{1.909515in}}%
\pgfpathcurveto{\pgfqpoint{1.567164in}{1.909515in}}{\pgfqpoint{1.559263in}{1.906243in}}{\pgfqpoint{1.553440in}{1.900419in}}%
\pgfpathcurveto{\pgfqpoint{1.547616in}{1.894595in}}{\pgfqpoint{1.544343in}{1.886695in}}{\pgfqpoint{1.544343in}{1.878458in}}%
\pgfpathcurveto{\pgfqpoint{1.544343in}{1.870222in}}{\pgfqpoint{1.547616in}{1.862322in}}{\pgfqpoint{1.553440in}{1.856498in}}%
\pgfpathcurveto{\pgfqpoint{1.559263in}{1.850674in}}{\pgfqpoint{1.567164in}{1.847402in}}{\pgfqpoint{1.575400in}{1.847402in}}%
\pgfpathclose%
\pgfusepath{stroke,fill}%
\end{pgfscope}%
\begin{pgfscope}%
\pgfpathrectangle{\pgfqpoint{0.100000in}{0.220728in}}{\pgfqpoint{3.696000in}{3.696000in}}%
\pgfusepath{clip}%
\pgfsetbuttcap%
\pgfsetroundjoin%
\definecolor{currentfill}{rgb}{0.121569,0.466667,0.705882}%
\pgfsetfillcolor{currentfill}%
\pgfsetfillopacity{0.419382}%
\pgfsetlinewidth{1.003750pt}%
\definecolor{currentstroke}{rgb}{0.121569,0.466667,0.705882}%
\pgfsetstrokecolor{currentstroke}%
\pgfsetstrokeopacity{0.419382}%
\pgfsetdash{}{0pt}%
\pgfpathmoveto{\pgfqpoint{1.567635in}{1.849128in}}%
\pgfpathcurveto{\pgfqpoint{1.575871in}{1.849128in}}{\pgfqpoint{1.583771in}{1.852400in}}{\pgfqpoint{1.589595in}{1.858224in}}%
\pgfpathcurveto{\pgfqpoint{1.595419in}{1.864048in}}{\pgfqpoint{1.598691in}{1.871948in}}{\pgfqpoint{1.598691in}{1.880185in}}%
\pgfpathcurveto{\pgfqpoint{1.598691in}{1.888421in}}{\pgfqpoint{1.595419in}{1.896321in}}{\pgfqpoint{1.589595in}{1.902145in}}%
\pgfpathcurveto{\pgfqpoint{1.583771in}{1.907969in}}{\pgfqpoint{1.575871in}{1.911241in}}{\pgfqpoint{1.567635in}{1.911241in}}%
\pgfpathcurveto{\pgfqpoint{1.559398in}{1.911241in}}{\pgfqpoint{1.551498in}{1.907969in}}{\pgfqpoint{1.545674in}{1.902145in}}%
\pgfpathcurveto{\pgfqpoint{1.539851in}{1.896321in}}{\pgfqpoint{1.536578in}{1.888421in}}{\pgfqpoint{1.536578in}{1.880185in}}%
\pgfpathcurveto{\pgfqpoint{1.536578in}{1.871948in}}{\pgfqpoint{1.539851in}{1.864048in}}{\pgfqpoint{1.545674in}{1.858224in}}%
\pgfpathcurveto{\pgfqpoint{1.551498in}{1.852400in}}{\pgfqpoint{1.559398in}{1.849128in}}{\pgfqpoint{1.567635in}{1.849128in}}%
\pgfpathclose%
\pgfusepath{stroke,fill}%
\end{pgfscope}%
\begin{pgfscope}%
\pgfpathrectangle{\pgfqpoint{0.100000in}{0.220728in}}{\pgfqpoint{3.696000in}{3.696000in}}%
\pgfusepath{clip}%
\pgfsetbuttcap%
\pgfsetroundjoin%
\definecolor{currentfill}{rgb}{0.121569,0.466667,0.705882}%
\pgfsetfillcolor{currentfill}%
\pgfsetfillopacity{0.419431}%
\pgfsetlinewidth{1.003750pt}%
\definecolor{currentstroke}{rgb}{0.121569,0.466667,0.705882}%
\pgfsetstrokecolor{currentstroke}%
\pgfsetstrokeopacity{0.419431}%
\pgfsetdash{}{0pt}%
\pgfpathmoveto{\pgfqpoint{2.013129in}{1.944933in}}%
\pgfpathcurveto{\pgfqpoint{2.021365in}{1.944933in}}{\pgfqpoint{2.029265in}{1.948205in}}{\pgfqpoint{2.035089in}{1.954029in}}%
\pgfpathcurveto{\pgfqpoint{2.040913in}{1.959853in}}{\pgfqpoint{2.044185in}{1.967753in}}{\pgfqpoint{2.044185in}{1.975989in}}%
\pgfpathcurveto{\pgfqpoint{2.044185in}{1.984225in}}{\pgfqpoint{2.040913in}{1.992125in}}{\pgfqpoint{2.035089in}{1.997949in}}%
\pgfpathcurveto{\pgfqpoint{2.029265in}{2.003773in}}{\pgfqpoint{2.021365in}{2.007046in}}{\pgfqpoint{2.013129in}{2.007046in}}%
\pgfpathcurveto{\pgfqpoint{2.004892in}{2.007046in}}{\pgfqpoint{1.996992in}{2.003773in}}{\pgfqpoint{1.991168in}{1.997949in}}%
\pgfpathcurveto{\pgfqpoint{1.985344in}{1.992125in}}{\pgfqpoint{1.982072in}{1.984225in}}{\pgfqpoint{1.982072in}{1.975989in}}%
\pgfpathcurveto{\pgfqpoint{1.982072in}{1.967753in}}{\pgfqpoint{1.985344in}{1.959853in}}{\pgfqpoint{1.991168in}{1.954029in}}%
\pgfpathcurveto{\pgfqpoint{1.996992in}{1.948205in}}{\pgfqpoint{2.004892in}{1.944933in}}{\pgfqpoint{2.013129in}{1.944933in}}%
\pgfpathclose%
\pgfusepath{stroke,fill}%
\end{pgfscope}%
\begin{pgfscope}%
\pgfpathrectangle{\pgfqpoint{0.100000in}{0.220728in}}{\pgfqpoint{3.696000in}{3.696000in}}%
\pgfusepath{clip}%
\pgfsetbuttcap%
\pgfsetroundjoin%
\definecolor{currentfill}{rgb}{0.121569,0.466667,0.705882}%
\pgfsetfillcolor{currentfill}%
\pgfsetfillopacity{0.420660}%
\pgfsetlinewidth{1.003750pt}%
\definecolor{currentstroke}{rgb}{0.121569,0.466667,0.705882}%
\pgfsetstrokecolor{currentstroke}%
\pgfsetstrokeopacity{0.420660}%
\pgfsetdash{}{0pt}%
\pgfpathmoveto{\pgfqpoint{1.560704in}{1.845109in}}%
\pgfpathcurveto{\pgfqpoint{1.568941in}{1.845109in}}{\pgfqpoint{1.576841in}{1.848381in}}{\pgfqpoint{1.582665in}{1.854205in}}%
\pgfpathcurveto{\pgfqpoint{1.588488in}{1.860029in}}{\pgfqpoint{1.591761in}{1.867929in}}{\pgfqpoint{1.591761in}{1.876165in}}%
\pgfpathcurveto{\pgfqpoint{1.591761in}{1.884401in}}{\pgfqpoint{1.588488in}{1.892301in}}{\pgfqpoint{1.582665in}{1.898125in}}%
\pgfpathcurveto{\pgfqpoint{1.576841in}{1.903949in}}{\pgfqpoint{1.568941in}{1.907222in}}{\pgfqpoint{1.560704in}{1.907222in}}%
\pgfpathcurveto{\pgfqpoint{1.552468in}{1.907222in}}{\pgfqpoint{1.544568in}{1.903949in}}{\pgfqpoint{1.538744in}{1.898125in}}%
\pgfpathcurveto{\pgfqpoint{1.532920in}{1.892301in}}{\pgfqpoint{1.529648in}{1.884401in}}{\pgfqpoint{1.529648in}{1.876165in}}%
\pgfpathcurveto{\pgfqpoint{1.529648in}{1.867929in}}{\pgfqpoint{1.532920in}{1.860029in}}{\pgfqpoint{1.538744in}{1.854205in}}%
\pgfpathcurveto{\pgfqpoint{1.544568in}{1.848381in}}{\pgfqpoint{1.552468in}{1.845109in}}{\pgfqpoint{1.560704in}{1.845109in}}%
\pgfpathclose%
\pgfusepath{stroke,fill}%
\end{pgfscope}%
\begin{pgfscope}%
\pgfpathrectangle{\pgfqpoint{0.100000in}{0.220728in}}{\pgfqpoint{3.696000in}{3.696000in}}%
\pgfusepath{clip}%
\pgfsetbuttcap%
\pgfsetroundjoin%
\definecolor{currentfill}{rgb}{0.121569,0.466667,0.705882}%
\pgfsetfillcolor{currentfill}%
\pgfsetfillopacity{0.421128}%
\pgfsetlinewidth{1.003750pt}%
\definecolor{currentstroke}{rgb}{0.121569,0.466667,0.705882}%
\pgfsetstrokecolor{currentstroke}%
\pgfsetstrokeopacity{0.421128}%
\pgfsetdash{}{0pt}%
\pgfpathmoveto{\pgfqpoint{1.557232in}{1.841381in}}%
\pgfpathcurveto{\pgfqpoint{1.565468in}{1.841381in}}{\pgfqpoint{1.573368in}{1.844653in}}{\pgfqpoint{1.579192in}{1.850477in}}%
\pgfpathcurveto{\pgfqpoint{1.585016in}{1.856301in}}{\pgfqpoint{1.588288in}{1.864201in}}{\pgfqpoint{1.588288in}{1.872437in}}%
\pgfpathcurveto{\pgfqpoint{1.588288in}{1.880674in}}{\pgfqpoint{1.585016in}{1.888574in}}{\pgfqpoint{1.579192in}{1.894398in}}%
\pgfpathcurveto{\pgfqpoint{1.573368in}{1.900222in}}{\pgfqpoint{1.565468in}{1.903494in}}{\pgfqpoint{1.557232in}{1.903494in}}%
\pgfpathcurveto{\pgfqpoint{1.548996in}{1.903494in}}{\pgfqpoint{1.541096in}{1.900222in}}{\pgfqpoint{1.535272in}{1.894398in}}%
\pgfpathcurveto{\pgfqpoint{1.529448in}{1.888574in}}{\pgfqpoint{1.526175in}{1.880674in}}{\pgfqpoint{1.526175in}{1.872437in}}%
\pgfpathcurveto{\pgfqpoint{1.526175in}{1.864201in}}{\pgfqpoint{1.529448in}{1.856301in}}{\pgfqpoint{1.535272in}{1.850477in}}%
\pgfpathcurveto{\pgfqpoint{1.541096in}{1.844653in}}{\pgfqpoint{1.548996in}{1.841381in}}{\pgfqpoint{1.557232in}{1.841381in}}%
\pgfpathclose%
\pgfusepath{stroke,fill}%
\end{pgfscope}%
\begin{pgfscope}%
\pgfpathrectangle{\pgfqpoint{0.100000in}{0.220728in}}{\pgfqpoint{3.696000in}{3.696000in}}%
\pgfusepath{clip}%
\pgfsetbuttcap%
\pgfsetroundjoin%
\definecolor{currentfill}{rgb}{0.121569,0.466667,0.705882}%
\pgfsetfillcolor{currentfill}%
\pgfsetfillopacity{0.422392}%
\pgfsetlinewidth{1.003750pt}%
\definecolor{currentstroke}{rgb}{0.121569,0.466667,0.705882}%
\pgfsetstrokecolor{currentstroke}%
\pgfsetstrokeopacity{0.422392}%
\pgfsetdash{}{0pt}%
\pgfpathmoveto{\pgfqpoint{2.013597in}{1.941194in}}%
\pgfpathcurveto{\pgfqpoint{2.021833in}{1.941194in}}{\pgfqpoint{2.029733in}{1.944466in}}{\pgfqpoint{2.035557in}{1.950290in}}%
\pgfpathcurveto{\pgfqpoint{2.041381in}{1.956114in}}{\pgfqpoint{2.044653in}{1.964014in}}{\pgfqpoint{2.044653in}{1.972250in}}%
\pgfpathcurveto{\pgfqpoint{2.044653in}{1.980486in}}{\pgfqpoint{2.041381in}{1.988387in}}{\pgfqpoint{2.035557in}{1.994210in}}%
\pgfpathcurveto{\pgfqpoint{2.029733in}{2.000034in}}{\pgfqpoint{2.021833in}{2.003307in}}{\pgfqpoint{2.013597in}{2.003307in}}%
\pgfpathcurveto{\pgfqpoint{2.005360in}{2.003307in}}{\pgfqpoint{1.997460in}{2.000034in}}{\pgfqpoint{1.991636in}{1.994210in}}%
\pgfpathcurveto{\pgfqpoint{1.985812in}{1.988387in}}{\pgfqpoint{1.982540in}{1.980486in}}{\pgfqpoint{1.982540in}{1.972250in}}%
\pgfpathcurveto{\pgfqpoint{1.982540in}{1.964014in}}{\pgfqpoint{1.985812in}{1.956114in}}{\pgfqpoint{1.991636in}{1.950290in}}%
\pgfpathcurveto{\pgfqpoint{1.997460in}{1.944466in}}{\pgfqpoint{2.005360in}{1.941194in}}{\pgfqpoint{2.013597in}{1.941194in}}%
\pgfpathclose%
\pgfusepath{stroke,fill}%
\end{pgfscope}%
\begin{pgfscope}%
\pgfpathrectangle{\pgfqpoint{0.100000in}{0.220728in}}{\pgfqpoint{3.696000in}{3.696000in}}%
\pgfusepath{clip}%
\pgfsetbuttcap%
\pgfsetroundjoin%
\definecolor{currentfill}{rgb}{0.121569,0.466667,0.705882}%
\pgfsetfillcolor{currentfill}%
\pgfsetfillopacity{0.423387}%
\pgfsetlinewidth{1.003750pt}%
\definecolor{currentstroke}{rgb}{0.121569,0.466667,0.705882}%
\pgfsetstrokecolor{currentstroke}%
\pgfsetstrokeopacity{0.423387}%
\pgfsetdash{}{0pt}%
\pgfpathmoveto{\pgfqpoint{1.552667in}{1.841178in}}%
\pgfpathcurveto{\pgfqpoint{1.560903in}{1.841178in}}{\pgfqpoint{1.568803in}{1.844450in}}{\pgfqpoint{1.574627in}{1.850274in}}%
\pgfpathcurveto{\pgfqpoint{1.580451in}{1.856098in}}{\pgfqpoint{1.583723in}{1.863998in}}{\pgfqpoint{1.583723in}{1.872234in}}%
\pgfpathcurveto{\pgfqpoint{1.583723in}{1.880470in}}{\pgfqpoint{1.580451in}{1.888370in}}{\pgfqpoint{1.574627in}{1.894194in}}%
\pgfpathcurveto{\pgfqpoint{1.568803in}{1.900018in}}{\pgfqpoint{1.560903in}{1.903291in}}{\pgfqpoint{1.552667in}{1.903291in}}%
\pgfpathcurveto{\pgfqpoint{1.544430in}{1.903291in}}{\pgfqpoint{1.536530in}{1.900018in}}{\pgfqpoint{1.530706in}{1.894194in}}%
\pgfpathcurveto{\pgfqpoint{1.524882in}{1.888370in}}{\pgfqpoint{1.521610in}{1.880470in}}{\pgfqpoint{1.521610in}{1.872234in}}%
\pgfpathcurveto{\pgfqpoint{1.521610in}{1.863998in}}{\pgfqpoint{1.524882in}{1.856098in}}{\pgfqpoint{1.530706in}{1.850274in}}%
\pgfpathcurveto{\pgfqpoint{1.536530in}{1.844450in}}{\pgfqpoint{1.544430in}{1.841178in}}{\pgfqpoint{1.552667in}{1.841178in}}%
\pgfpathclose%
\pgfusepath{stroke,fill}%
\end{pgfscope}%
\begin{pgfscope}%
\pgfpathrectangle{\pgfqpoint{0.100000in}{0.220728in}}{\pgfqpoint{3.696000in}{3.696000in}}%
\pgfusepath{clip}%
\pgfsetbuttcap%
\pgfsetroundjoin%
\definecolor{currentfill}{rgb}{0.121569,0.466667,0.705882}%
\pgfsetfillcolor{currentfill}%
\pgfsetfillopacity{0.424020}%
\pgfsetlinewidth{1.003750pt}%
\definecolor{currentstroke}{rgb}{0.121569,0.466667,0.705882}%
\pgfsetstrokecolor{currentstroke}%
\pgfsetstrokeopacity{0.424020}%
\pgfsetdash{}{0pt}%
\pgfpathmoveto{\pgfqpoint{2.014848in}{1.939538in}}%
\pgfpathcurveto{\pgfqpoint{2.023084in}{1.939538in}}{\pgfqpoint{2.030984in}{1.942810in}}{\pgfqpoint{2.036808in}{1.948634in}}%
\pgfpathcurveto{\pgfqpoint{2.042632in}{1.954458in}}{\pgfqpoint{2.045904in}{1.962358in}}{\pgfqpoint{2.045904in}{1.970595in}}%
\pgfpathcurveto{\pgfqpoint{2.045904in}{1.978831in}}{\pgfqpoint{2.042632in}{1.986731in}}{\pgfqpoint{2.036808in}{1.992555in}}%
\pgfpathcurveto{\pgfqpoint{2.030984in}{1.998379in}}{\pgfqpoint{2.023084in}{2.001651in}}{\pgfqpoint{2.014848in}{2.001651in}}%
\pgfpathcurveto{\pgfqpoint{2.006612in}{2.001651in}}{\pgfqpoint{1.998712in}{1.998379in}}{\pgfqpoint{1.992888in}{1.992555in}}%
\pgfpathcurveto{\pgfqpoint{1.987064in}{1.986731in}}{\pgfqpoint{1.983791in}{1.978831in}}{\pgfqpoint{1.983791in}{1.970595in}}%
\pgfpathcurveto{\pgfqpoint{1.983791in}{1.962358in}}{\pgfqpoint{1.987064in}{1.954458in}}{\pgfqpoint{1.992888in}{1.948634in}}%
\pgfpathcurveto{\pgfqpoint{1.998712in}{1.942810in}}{\pgfqpoint{2.006612in}{1.939538in}}{\pgfqpoint{2.014848in}{1.939538in}}%
\pgfpathclose%
\pgfusepath{stroke,fill}%
\end{pgfscope}%
\begin{pgfscope}%
\pgfpathrectangle{\pgfqpoint{0.100000in}{0.220728in}}{\pgfqpoint{3.696000in}{3.696000in}}%
\pgfusepath{clip}%
\pgfsetbuttcap%
\pgfsetroundjoin%
\definecolor{currentfill}{rgb}{0.121569,0.466667,0.705882}%
\pgfsetfillcolor{currentfill}%
\pgfsetfillopacity{0.424105}%
\pgfsetlinewidth{1.003750pt}%
\definecolor{currentstroke}{rgb}{0.121569,0.466667,0.705882}%
\pgfsetstrokecolor{currentstroke}%
\pgfsetstrokeopacity{0.424105}%
\pgfsetdash{}{0pt}%
\pgfpathmoveto{\pgfqpoint{1.549667in}{1.839450in}}%
\pgfpathcurveto{\pgfqpoint{1.557903in}{1.839450in}}{\pgfqpoint{1.565803in}{1.842722in}}{\pgfqpoint{1.571627in}{1.848546in}}%
\pgfpathcurveto{\pgfqpoint{1.577451in}{1.854370in}}{\pgfqpoint{1.580723in}{1.862270in}}{\pgfqpoint{1.580723in}{1.870506in}}%
\pgfpathcurveto{\pgfqpoint{1.580723in}{1.878743in}}{\pgfqpoint{1.577451in}{1.886643in}}{\pgfqpoint{1.571627in}{1.892466in}}%
\pgfpathcurveto{\pgfqpoint{1.565803in}{1.898290in}}{\pgfqpoint{1.557903in}{1.901563in}}{\pgfqpoint{1.549667in}{1.901563in}}%
\pgfpathcurveto{\pgfqpoint{1.541430in}{1.901563in}}{\pgfqpoint{1.533530in}{1.898290in}}{\pgfqpoint{1.527706in}{1.892466in}}%
\pgfpathcurveto{\pgfqpoint{1.521882in}{1.886643in}}{\pgfqpoint{1.518610in}{1.878743in}}{\pgfqpoint{1.518610in}{1.870506in}}%
\pgfpathcurveto{\pgfqpoint{1.518610in}{1.862270in}}{\pgfqpoint{1.521882in}{1.854370in}}{\pgfqpoint{1.527706in}{1.848546in}}%
\pgfpathcurveto{\pgfqpoint{1.533530in}{1.842722in}}{\pgfqpoint{1.541430in}{1.839450in}}{\pgfqpoint{1.549667in}{1.839450in}}%
\pgfpathclose%
\pgfusepath{stroke,fill}%
\end{pgfscope}%
\begin{pgfscope}%
\pgfpathrectangle{\pgfqpoint{0.100000in}{0.220728in}}{\pgfqpoint{3.696000in}{3.696000in}}%
\pgfusepath{clip}%
\pgfsetbuttcap%
\pgfsetroundjoin%
\definecolor{currentfill}{rgb}{0.121569,0.466667,0.705882}%
\pgfsetfillcolor{currentfill}%
\pgfsetfillopacity{0.424910}%
\pgfsetlinewidth{1.003750pt}%
\definecolor{currentstroke}{rgb}{0.121569,0.466667,0.705882}%
\pgfsetstrokecolor{currentstroke}%
\pgfsetstrokeopacity{0.424910}%
\pgfsetdash{}{0pt}%
\pgfpathmoveto{\pgfqpoint{1.547186in}{1.838875in}}%
\pgfpathcurveto{\pgfqpoint{1.555422in}{1.838875in}}{\pgfqpoint{1.563322in}{1.842148in}}{\pgfqpoint{1.569146in}{1.847972in}}%
\pgfpathcurveto{\pgfqpoint{1.574970in}{1.853796in}}{\pgfqpoint{1.578242in}{1.861696in}}{\pgfqpoint{1.578242in}{1.869932in}}%
\pgfpathcurveto{\pgfqpoint{1.578242in}{1.878168in}}{\pgfqpoint{1.574970in}{1.886068in}}{\pgfqpoint{1.569146in}{1.891892in}}%
\pgfpathcurveto{\pgfqpoint{1.563322in}{1.897716in}}{\pgfqpoint{1.555422in}{1.900988in}}{\pgfqpoint{1.547186in}{1.900988in}}%
\pgfpathcurveto{\pgfqpoint{1.538950in}{1.900988in}}{\pgfqpoint{1.531049in}{1.897716in}}{\pgfqpoint{1.525226in}{1.891892in}}%
\pgfpathcurveto{\pgfqpoint{1.519402in}{1.886068in}}{\pgfqpoint{1.516129in}{1.878168in}}{\pgfqpoint{1.516129in}{1.869932in}}%
\pgfpathcurveto{\pgfqpoint{1.516129in}{1.861696in}}{\pgfqpoint{1.519402in}{1.853796in}}{\pgfqpoint{1.525226in}{1.847972in}}%
\pgfpathcurveto{\pgfqpoint{1.531049in}{1.842148in}}{\pgfqpoint{1.538950in}{1.838875in}}{\pgfqpoint{1.547186in}{1.838875in}}%
\pgfpathclose%
\pgfusepath{stroke,fill}%
\end{pgfscope}%
\begin{pgfscope}%
\pgfpathrectangle{\pgfqpoint{0.100000in}{0.220728in}}{\pgfqpoint{3.696000in}{3.696000in}}%
\pgfusepath{clip}%
\pgfsetbuttcap%
\pgfsetroundjoin%
\definecolor{currentfill}{rgb}{0.121569,0.466667,0.705882}%
\pgfsetfillcolor{currentfill}%
\pgfsetfillopacity{0.426102}%
\pgfsetlinewidth{1.003750pt}%
\definecolor{currentstroke}{rgb}{0.121569,0.466667,0.705882}%
\pgfsetstrokecolor{currentstroke}%
\pgfsetstrokeopacity{0.426102}%
\pgfsetdash{}{0pt}%
\pgfpathmoveto{\pgfqpoint{2.016367in}{1.938483in}}%
\pgfpathcurveto{\pgfqpoint{2.024604in}{1.938483in}}{\pgfqpoint{2.032504in}{1.941755in}}{\pgfqpoint{2.038328in}{1.947579in}}%
\pgfpathcurveto{\pgfqpoint{2.044152in}{1.953403in}}{\pgfqpoint{2.047424in}{1.961303in}}{\pgfqpoint{2.047424in}{1.969539in}}%
\pgfpathcurveto{\pgfqpoint{2.047424in}{1.977776in}}{\pgfqpoint{2.044152in}{1.985676in}}{\pgfqpoint{2.038328in}{1.991500in}}%
\pgfpathcurveto{\pgfqpoint{2.032504in}{1.997324in}}{\pgfqpoint{2.024604in}{2.000596in}}{\pgfqpoint{2.016367in}{2.000596in}}%
\pgfpathcurveto{\pgfqpoint{2.008131in}{2.000596in}}{\pgfqpoint{2.000231in}{1.997324in}}{\pgfqpoint{1.994407in}{1.991500in}}%
\pgfpathcurveto{\pgfqpoint{1.988583in}{1.985676in}}{\pgfqpoint{1.985311in}{1.977776in}}{\pgfqpoint{1.985311in}{1.969539in}}%
\pgfpathcurveto{\pgfqpoint{1.985311in}{1.961303in}}{\pgfqpoint{1.988583in}{1.953403in}}{\pgfqpoint{1.994407in}{1.947579in}}%
\pgfpathcurveto{\pgfqpoint{2.000231in}{1.941755in}}{\pgfqpoint{2.008131in}{1.938483in}}{\pgfqpoint{2.016367in}{1.938483in}}%
\pgfpathclose%
\pgfusepath{stroke,fill}%
\end{pgfscope}%
\begin{pgfscope}%
\pgfpathrectangle{\pgfqpoint{0.100000in}{0.220728in}}{\pgfqpoint{3.696000in}{3.696000in}}%
\pgfusepath{clip}%
\pgfsetbuttcap%
\pgfsetroundjoin%
\definecolor{currentfill}{rgb}{0.121569,0.466667,0.705882}%
\pgfsetfillcolor{currentfill}%
\pgfsetfillopacity{0.426497}%
\pgfsetlinewidth{1.003750pt}%
\definecolor{currentstroke}{rgb}{0.121569,0.466667,0.705882}%
\pgfsetstrokecolor{currentstroke}%
\pgfsetstrokeopacity{0.426497}%
\pgfsetdash{}{0pt}%
\pgfpathmoveto{\pgfqpoint{1.543274in}{1.837763in}}%
\pgfpathcurveto{\pgfqpoint{1.551510in}{1.837763in}}{\pgfqpoint{1.559410in}{1.841035in}}{\pgfqpoint{1.565234in}{1.846859in}}%
\pgfpathcurveto{\pgfqpoint{1.571058in}{1.852683in}}{\pgfqpoint{1.574330in}{1.860583in}}{\pgfqpoint{1.574330in}{1.868819in}}%
\pgfpathcurveto{\pgfqpoint{1.574330in}{1.877055in}}{\pgfqpoint{1.571058in}{1.884955in}}{\pgfqpoint{1.565234in}{1.890779in}}%
\pgfpathcurveto{\pgfqpoint{1.559410in}{1.896603in}}{\pgfqpoint{1.551510in}{1.899876in}}{\pgfqpoint{1.543274in}{1.899876in}}%
\pgfpathcurveto{\pgfqpoint{1.535037in}{1.899876in}}{\pgfqpoint{1.527137in}{1.896603in}}{\pgfqpoint{1.521313in}{1.890779in}}%
\pgfpathcurveto{\pgfqpoint{1.515489in}{1.884955in}}{\pgfqpoint{1.512217in}{1.877055in}}{\pgfqpoint{1.512217in}{1.868819in}}%
\pgfpathcurveto{\pgfqpoint{1.512217in}{1.860583in}}{\pgfqpoint{1.515489in}{1.852683in}}{\pgfqpoint{1.521313in}{1.846859in}}%
\pgfpathcurveto{\pgfqpoint{1.527137in}{1.841035in}}{\pgfqpoint{1.535037in}{1.837763in}}{\pgfqpoint{1.543274in}{1.837763in}}%
\pgfpathclose%
\pgfusepath{stroke,fill}%
\end{pgfscope}%
\begin{pgfscope}%
\pgfpathrectangle{\pgfqpoint{0.100000in}{0.220728in}}{\pgfqpoint{3.696000in}{3.696000in}}%
\pgfusepath{clip}%
\pgfsetbuttcap%
\pgfsetroundjoin%
\definecolor{currentfill}{rgb}{0.121569,0.466667,0.705882}%
\pgfsetfillcolor{currentfill}%
\pgfsetfillopacity{0.427283}%
\pgfsetlinewidth{1.003750pt}%
\definecolor{currentstroke}{rgb}{0.121569,0.466667,0.705882}%
\pgfsetstrokecolor{currentstroke}%
\pgfsetstrokeopacity{0.427283}%
\pgfsetdash{}{0pt}%
\pgfpathmoveto{\pgfqpoint{1.540130in}{1.836378in}}%
\pgfpathcurveto{\pgfqpoint{1.548367in}{1.836378in}}{\pgfqpoint{1.556267in}{1.839650in}}{\pgfqpoint{1.562091in}{1.845474in}}%
\pgfpathcurveto{\pgfqpoint{1.567915in}{1.851298in}}{\pgfqpoint{1.571187in}{1.859198in}}{\pgfqpoint{1.571187in}{1.867434in}}%
\pgfpathcurveto{\pgfqpoint{1.571187in}{1.875671in}}{\pgfqpoint{1.567915in}{1.883571in}}{\pgfqpoint{1.562091in}{1.889395in}}%
\pgfpathcurveto{\pgfqpoint{1.556267in}{1.895219in}}{\pgfqpoint{1.548367in}{1.898491in}}{\pgfqpoint{1.540130in}{1.898491in}}%
\pgfpathcurveto{\pgfqpoint{1.531894in}{1.898491in}}{\pgfqpoint{1.523994in}{1.895219in}}{\pgfqpoint{1.518170in}{1.889395in}}%
\pgfpathcurveto{\pgfqpoint{1.512346in}{1.883571in}}{\pgfqpoint{1.509074in}{1.875671in}}{\pgfqpoint{1.509074in}{1.867434in}}%
\pgfpathcurveto{\pgfqpoint{1.509074in}{1.859198in}}{\pgfqpoint{1.512346in}{1.851298in}}{\pgfqpoint{1.518170in}{1.845474in}}%
\pgfpathcurveto{\pgfqpoint{1.523994in}{1.839650in}}{\pgfqpoint{1.531894in}{1.836378in}}{\pgfqpoint{1.540130in}{1.836378in}}%
\pgfpathclose%
\pgfusepath{stroke,fill}%
\end{pgfscope}%
\begin{pgfscope}%
\pgfpathrectangle{\pgfqpoint{0.100000in}{0.220728in}}{\pgfqpoint{3.696000in}{3.696000in}}%
\pgfusepath{clip}%
\pgfsetbuttcap%
\pgfsetroundjoin%
\definecolor{currentfill}{rgb}{0.121569,0.466667,0.705882}%
\pgfsetfillcolor{currentfill}%
\pgfsetfillopacity{0.428124}%
\pgfsetlinewidth{1.003750pt}%
\definecolor{currentstroke}{rgb}{0.121569,0.466667,0.705882}%
\pgfsetstrokecolor{currentstroke}%
\pgfsetstrokeopacity{0.428124}%
\pgfsetdash{}{0pt}%
\pgfpathmoveto{\pgfqpoint{1.538025in}{1.836120in}}%
\pgfpathcurveto{\pgfqpoint{1.546261in}{1.836120in}}{\pgfqpoint{1.554161in}{1.839393in}}{\pgfqpoint{1.559985in}{1.845217in}}%
\pgfpathcurveto{\pgfqpoint{1.565809in}{1.851041in}}{\pgfqpoint{1.569082in}{1.858941in}}{\pgfqpoint{1.569082in}{1.867177in}}%
\pgfpathcurveto{\pgfqpoint{1.569082in}{1.875413in}}{\pgfqpoint{1.565809in}{1.883313in}}{\pgfqpoint{1.559985in}{1.889137in}}%
\pgfpathcurveto{\pgfqpoint{1.554161in}{1.894961in}}{\pgfqpoint{1.546261in}{1.898233in}}{\pgfqpoint{1.538025in}{1.898233in}}%
\pgfpathcurveto{\pgfqpoint{1.529789in}{1.898233in}}{\pgfqpoint{1.521889in}{1.894961in}}{\pgfqpoint{1.516065in}{1.889137in}}%
\pgfpathcurveto{\pgfqpoint{1.510241in}{1.883313in}}{\pgfqpoint{1.506969in}{1.875413in}}{\pgfqpoint{1.506969in}{1.867177in}}%
\pgfpathcurveto{\pgfqpoint{1.506969in}{1.858941in}}{\pgfqpoint{1.510241in}{1.851041in}}{\pgfqpoint{1.516065in}{1.845217in}}%
\pgfpathcurveto{\pgfqpoint{1.521889in}{1.839393in}}{\pgfqpoint{1.529789in}{1.836120in}}{\pgfqpoint{1.538025in}{1.836120in}}%
\pgfpathclose%
\pgfusepath{stroke,fill}%
\end{pgfscope}%
\begin{pgfscope}%
\pgfpathrectangle{\pgfqpoint{0.100000in}{0.220728in}}{\pgfqpoint{3.696000in}{3.696000in}}%
\pgfusepath{clip}%
\pgfsetbuttcap%
\pgfsetroundjoin%
\definecolor{currentfill}{rgb}{0.121569,0.466667,0.705882}%
\pgfsetfillcolor{currentfill}%
\pgfsetfillopacity{0.428606}%
\pgfsetlinewidth{1.003750pt}%
\definecolor{currentstroke}{rgb}{0.121569,0.466667,0.705882}%
\pgfsetstrokecolor{currentstroke}%
\pgfsetstrokeopacity{0.428606}%
\pgfsetdash{}{0pt}%
\pgfpathmoveto{\pgfqpoint{2.017032in}{1.933888in}}%
\pgfpathcurveto{\pgfqpoint{2.025269in}{1.933888in}}{\pgfqpoint{2.033169in}{1.937161in}}{\pgfqpoint{2.038993in}{1.942985in}}%
\pgfpathcurveto{\pgfqpoint{2.044817in}{1.948808in}}{\pgfqpoint{2.048089in}{1.956709in}}{\pgfqpoint{2.048089in}{1.964945in}}%
\pgfpathcurveto{\pgfqpoint{2.048089in}{1.973181in}}{\pgfqpoint{2.044817in}{1.981081in}}{\pgfqpoint{2.038993in}{1.986905in}}%
\pgfpathcurveto{\pgfqpoint{2.033169in}{1.992729in}}{\pgfqpoint{2.025269in}{1.996001in}}{\pgfqpoint{2.017032in}{1.996001in}}%
\pgfpathcurveto{\pgfqpoint{2.008796in}{1.996001in}}{\pgfqpoint{2.000896in}{1.992729in}}{\pgfqpoint{1.995072in}{1.986905in}}%
\pgfpathcurveto{\pgfqpoint{1.989248in}{1.981081in}}{\pgfqpoint{1.985976in}{1.973181in}}{\pgfqpoint{1.985976in}{1.964945in}}%
\pgfpathcurveto{\pgfqpoint{1.985976in}{1.956709in}}{\pgfqpoint{1.989248in}{1.948808in}}{\pgfqpoint{1.995072in}{1.942985in}}%
\pgfpathcurveto{\pgfqpoint{2.000896in}{1.937161in}}{\pgfqpoint{2.008796in}{1.933888in}}{\pgfqpoint{2.017032in}{1.933888in}}%
\pgfpathclose%
\pgfusepath{stroke,fill}%
\end{pgfscope}%
\begin{pgfscope}%
\pgfpathrectangle{\pgfqpoint{0.100000in}{0.220728in}}{\pgfqpoint{3.696000in}{3.696000in}}%
\pgfusepath{clip}%
\pgfsetbuttcap%
\pgfsetroundjoin%
\definecolor{currentfill}{rgb}{0.121569,0.466667,0.705882}%
\pgfsetfillcolor{currentfill}%
\pgfsetfillopacity{0.429520}%
\pgfsetlinewidth{1.003750pt}%
\definecolor{currentstroke}{rgb}{0.121569,0.466667,0.705882}%
\pgfsetstrokecolor{currentstroke}%
\pgfsetstrokeopacity{0.429520}%
\pgfsetdash{}{0pt}%
\pgfpathmoveto{\pgfqpoint{1.534550in}{1.834305in}}%
\pgfpathcurveto{\pgfqpoint{1.542787in}{1.834305in}}{\pgfqpoint{1.550687in}{1.837578in}}{\pgfqpoint{1.556511in}{1.843401in}}%
\pgfpathcurveto{\pgfqpoint{1.562335in}{1.849225in}}{\pgfqpoint{1.565607in}{1.857125in}}{\pgfqpoint{1.565607in}{1.865362in}}%
\pgfpathcurveto{\pgfqpoint{1.565607in}{1.873598in}}{\pgfqpoint{1.562335in}{1.881498in}}{\pgfqpoint{1.556511in}{1.887322in}}%
\pgfpathcurveto{\pgfqpoint{1.550687in}{1.893146in}}{\pgfqpoint{1.542787in}{1.896418in}}{\pgfqpoint{1.534550in}{1.896418in}}%
\pgfpathcurveto{\pgfqpoint{1.526314in}{1.896418in}}{\pgfqpoint{1.518414in}{1.893146in}}{\pgfqpoint{1.512590in}{1.887322in}}%
\pgfpathcurveto{\pgfqpoint{1.506766in}{1.881498in}}{\pgfqpoint{1.503494in}{1.873598in}}{\pgfqpoint{1.503494in}{1.865362in}}%
\pgfpathcurveto{\pgfqpoint{1.503494in}{1.857125in}}{\pgfqpoint{1.506766in}{1.849225in}}{\pgfqpoint{1.512590in}{1.843401in}}%
\pgfpathcurveto{\pgfqpoint{1.518414in}{1.837578in}}{\pgfqpoint{1.526314in}{1.834305in}}{\pgfqpoint{1.534550in}{1.834305in}}%
\pgfpathclose%
\pgfusepath{stroke,fill}%
\end{pgfscope}%
\begin{pgfscope}%
\pgfpathrectangle{\pgfqpoint{0.100000in}{0.220728in}}{\pgfqpoint{3.696000in}{3.696000in}}%
\pgfusepath{clip}%
\pgfsetbuttcap%
\pgfsetroundjoin%
\definecolor{currentfill}{rgb}{0.121569,0.466667,0.705882}%
\pgfsetfillcolor{currentfill}%
\pgfsetfillopacity{0.430036}%
\pgfsetlinewidth{1.003750pt}%
\definecolor{currentstroke}{rgb}{0.121569,0.466667,0.705882}%
\pgfsetstrokecolor{currentstroke}%
\pgfsetstrokeopacity{0.430036}%
\pgfsetdash{}{0pt}%
\pgfpathmoveto{\pgfqpoint{1.532297in}{1.832991in}}%
\pgfpathcurveto{\pgfqpoint{1.540533in}{1.832991in}}{\pgfqpoint{1.548434in}{1.836263in}}{\pgfqpoint{1.554257in}{1.842087in}}%
\pgfpathcurveto{\pgfqpoint{1.560081in}{1.847911in}}{\pgfqpoint{1.563354in}{1.855811in}}{\pgfqpoint{1.563354in}{1.864047in}}%
\pgfpathcurveto{\pgfqpoint{1.563354in}{1.872283in}}{\pgfqpoint{1.560081in}{1.880183in}}{\pgfqpoint{1.554257in}{1.886007in}}%
\pgfpathcurveto{\pgfqpoint{1.548434in}{1.891831in}}{\pgfqpoint{1.540533in}{1.895104in}}{\pgfqpoint{1.532297in}{1.895104in}}%
\pgfpathcurveto{\pgfqpoint{1.524061in}{1.895104in}}{\pgfqpoint{1.516161in}{1.891831in}}{\pgfqpoint{1.510337in}{1.886007in}}%
\pgfpathcurveto{\pgfqpoint{1.504513in}{1.880183in}}{\pgfqpoint{1.501241in}{1.872283in}}{\pgfqpoint{1.501241in}{1.864047in}}%
\pgfpathcurveto{\pgfqpoint{1.501241in}{1.855811in}}{\pgfqpoint{1.504513in}{1.847911in}}{\pgfqpoint{1.510337in}{1.842087in}}%
\pgfpathcurveto{\pgfqpoint{1.516161in}{1.836263in}}{\pgfqpoint{1.524061in}{1.832991in}}{\pgfqpoint{1.532297in}{1.832991in}}%
\pgfpathclose%
\pgfusepath{stroke,fill}%
\end{pgfscope}%
\begin{pgfscope}%
\pgfpathrectangle{\pgfqpoint{0.100000in}{0.220728in}}{\pgfqpoint{3.696000in}{3.696000in}}%
\pgfusepath{clip}%
\pgfsetbuttcap%
\pgfsetroundjoin%
\definecolor{currentfill}{rgb}{0.121569,0.466667,0.705882}%
\pgfsetfillcolor{currentfill}%
\pgfsetfillopacity{0.430588}%
\pgfsetlinewidth{1.003750pt}%
\definecolor{currentstroke}{rgb}{0.121569,0.466667,0.705882}%
\pgfsetstrokecolor{currentstroke}%
\pgfsetstrokeopacity{0.430588}%
\pgfsetdash{}{0pt}%
\pgfpathmoveto{\pgfqpoint{1.530796in}{1.832715in}}%
\pgfpathcurveto{\pgfqpoint{1.539032in}{1.832715in}}{\pgfqpoint{1.546933in}{1.835987in}}{\pgfqpoint{1.552756in}{1.841811in}}%
\pgfpathcurveto{\pgfqpoint{1.558580in}{1.847635in}}{\pgfqpoint{1.561853in}{1.855535in}}{\pgfqpoint{1.561853in}{1.863771in}}%
\pgfpathcurveto{\pgfqpoint{1.561853in}{1.872008in}}{\pgfqpoint{1.558580in}{1.879908in}}{\pgfqpoint{1.552756in}{1.885732in}}%
\pgfpathcurveto{\pgfqpoint{1.546933in}{1.891556in}}{\pgfqpoint{1.539032in}{1.894828in}}{\pgfqpoint{1.530796in}{1.894828in}}%
\pgfpathcurveto{\pgfqpoint{1.522560in}{1.894828in}}{\pgfqpoint{1.514660in}{1.891556in}}{\pgfqpoint{1.508836in}{1.885732in}}%
\pgfpathcurveto{\pgfqpoint{1.503012in}{1.879908in}}{\pgfqpoint{1.499740in}{1.872008in}}{\pgfqpoint{1.499740in}{1.863771in}}%
\pgfpathcurveto{\pgfqpoint{1.499740in}{1.855535in}}{\pgfqpoint{1.503012in}{1.847635in}}{\pgfqpoint{1.508836in}{1.841811in}}%
\pgfpathcurveto{\pgfqpoint{1.514660in}{1.835987in}}{\pgfqpoint{1.522560in}{1.832715in}}{\pgfqpoint{1.530796in}{1.832715in}}%
\pgfpathclose%
\pgfusepath{stroke,fill}%
\end{pgfscope}%
\begin{pgfscope}%
\pgfpathrectangle{\pgfqpoint{0.100000in}{0.220728in}}{\pgfqpoint{3.696000in}{3.696000in}}%
\pgfusepath{clip}%
\pgfsetbuttcap%
\pgfsetroundjoin%
\definecolor{currentfill}{rgb}{0.121569,0.466667,0.705882}%
\pgfsetfillcolor{currentfill}%
\pgfsetfillopacity{0.431486}%
\pgfsetlinewidth{1.003750pt}%
\definecolor{currentstroke}{rgb}{0.121569,0.466667,0.705882}%
\pgfsetstrokecolor{currentstroke}%
\pgfsetstrokeopacity{0.431486}%
\pgfsetdash{}{0pt}%
\pgfpathmoveto{\pgfqpoint{2.018394in}{1.930115in}}%
\pgfpathcurveto{\pgfqpoint{2.026630in}{1.930115in}}{\pgfqpoint{2.034530in}{1.933388in}}{\pgfqpoint{2.040354in}{1.939212in}}%
\pgfpathcurveto{\pgfqpoint{2.046178in}{1.945035in}}{\pgfqpoint{2.049450in}{1.952936in}}{\pgfqpoint{2.049450in}{1.961172in}}%
\pgfpathcurveto{\pgfqpoint{2.049450in}{1.969408in}}{\pgfqpoint{2.046178in}{1.977308in}}{\pgfqpoint{2.040354in}{1.983132in}}%
\pgfpathcurveto{\pgfqpoint{2.034530in}{1.988956in}}{\pgfqpoint{2.026630in}{1.992228in}}{\pgfqpoint{2.018394in}{1.992228in}}%
\pgfpathcurveto{\pgfqpoint{2.010157in}{1.992228in}}{\pgfqpoint{2.002257in}{1.988956in}}{\pgfqpoint{1.996433in}{1.983132in}}%
\pgfpathcurveto{\pgfqpoint{1.990609in}{1.977308in}}{\pgfqpoint{1.987337in}{1.969408in}}{\pgfqpoint{1.987337in}{1.961172in}}%
\pgfpathcurveto{\pgfqpoint{1.987337in}{1.952936in}}{\pgfqpoint{1.990609in}{1.945035in}}{\pgfqpoint{1.996433in}{1.939212in}}%
\pgfpathcurveto{\pgfqpoint{2.002257in}{1.933388in}}{\pgfqpoint{2.010157in}{1.930115in}}{\pgfqpoint{2.018394in}{1.930115in}}%
\pgfpathclose%
\pgfusepath{stroke,fill}%
\end{pgfscope}%
\begin{pgfscope}%
\pgfpathrectangle{\pgfqpoint{0.100000in}{0.220728in}}{\pgfqpoint{3.696000in}{3.696000in}}%
\pgfusepath{clip}%
\pgfsetbuttcap%
\pgfsetroundjoin%
\definecolor{currentfill}{rgb}{0.121569,0.466667,0.705882}%
\pgfsetfillcolor{currentfill}%
\pgfsetfillopacity{0.431692}%
\pgfsetlinewidth{1.003750pt}%
\definecolor{currentstroke}{rgb}{0.121569,0.466667,0.705882}%
\pgfsetstrokecolor{currentstroke}%
\pgfsetstrokeopacity{0.431692}%
\pgfsetdash{}{0pt}%
\pgfpathmoveto{\pgfqpoint{1.528426in}{1.832396in}}%
\pgfpathcurveto{\pgfqpoint{1.536662in}{1.832396in}}{\pgfqpoint{1.544562in}{1.835668in}}{\pgfqpoint{1.550386in}{1.841492in}}%
\pgfpathcurveto{\pgfqpoint{1.556210in}{1.847316in}}{\pgfqpoint{1.559482in}{1.855216in}}{\pgfqpoint{1.559482in}{1.863452in}}%
\pgfpathcurveto{\pgfqpoint{1.559482in}{1.871688in}}{\pgfqpoint{1.556210in}{1.879588in}}{\pgfqpoint{1.550386in}{1.885412in}}%
\pgfpathcurveto{\pgfqpoint{1.544562in}{1.891236in}}{\pgfqpoint{1.536662in}{1.894509in}}{\pgfqpoint{1.528426in}{1.894509in}}%
\pgfpathcurveto{\pgfqpoint{1.520189in}{1.894509in}}{\pgfqpoint{1.512289in}{1.891236in}}{\pgfqpoint{1.506465in}{1.885412in}}%
\pgfpathcurveto{\pgfqpoint{1.500641in}{1.879588in}}{\pgfqpoint{1.497369in}{1.871688in}}{\pgfqpoint{1.497369in}{1.863452in}}%
\pgfpathcurveto{\pgfqpoint{1.497369in}{1.855216in}}{\pgfqpoint{1.500641in}{1.847316in}}{\pgfqpoint{1.506465in}{1.841492in}}%
\pgfpathcurveto{\pgfqpoint{1.512289in}{1.835668in}}{\pgfqpoint{1.520189in}{1.832396in}}{\pgfqpoint{1.528426in}{1.832396in}}%
\pgfpathclose%
\pgfusepath{stroke,fill}%
\end{pgfscope}%
\begin{pgfscope}%
\pgfpathrectangle{\pgfqpoint{0.100000in}{0.220728in}}{\pgfqpoint{3.696000in}{3.696000in}}%
\pgfusepath{clip}%
\pgfsetbuttcap%
\pgfsetroundjoin%
\definecolor{currentfill}{rgb}{0.121569,0.466667,0.705882}%
\pgfsetfillcolor{currentfill}%
\pgfsetfillopacity{0.431955}%
\pgfsetlinewidth{1.003750pt}%
\definecolor{currentstroke}{rgb}{0.121569,0.466667,0.705882}%
\pgfsetstrokecolor{currentstroke}%
\pgfsetstrokeopacity{0.431955}%
\pgfsetdash{}{0pt}%
\pgfpathmoveto{\pgfqpoint{1.527285in}{1.831676in}}%
\pgfpathcurveto{\pgfqpoint{1.535521in}{1.831676in}}{\pgfqpoint{1.543421in}{1.834948in}}{\pgfqpoint{1.549245in}{1.840772in}}%
\pgfpathcurveto{\pgfqpoint{1.555069in}{1.846596in}}{\pgfqpoint{1.558341in}{1.854496in}}{\pgfqpoint{1.558341in}{1.862732in}}%
\pgfpathcurveto{\pgfqpoint{1.558341in}{1.870968in}}{\pgfqpoint{1.555069in}{1.878869in}}{\pgfqpoint{1.549245in}{1.884692in}}%
\pgfpathcurveto{\pgfqpoint{1.543421in}{1.890516in}}{\pgfqpoint{1.535521in}{1.893789in}}{\pgfqpoint{1.527285in}{1.893789in}}%
\pgfpathcurveto{\pgfqpoint{1.519049in}{1.893789in}}{\pgfqpoint{1.511148in}{1.890516in}}{\pgfqpoint{1.505325in}{1.884692in}}%
\pgfpathcurveto{\pgfqpoint{1.499501in}{1.878869in}}{\pgfqpoint{1.496228in}{1.870968in}}{\pgfqpoint{1.496228in}{1.862732in}}%
\pgfpathcurveto{\pgfqpoint{1.496228in}{1.854496in}}{\pgfqpoint{1.499501in}{1.846596in}}{\pgfqpoint{1.505325in}{1.840772in}}%
\pgfpathcurveto{\pgfqpoint{1.511148in}{1.834948in}}{\pgfqpoint{1.519049in}{1.831676in}}{\pgfqpoint{1.527285in}{1.831676in}}%
\pgfpathclose%
\pgfusepath{stroke,fill}%
\end{pgfscope}%
\begin{pgfscope}%
\pgfpathrectangle{\pgfqpoint{0.100000in}{0.220728in}}{\pgfqpoint{3.696000in}{3.696000in}}%
\pgfusepath{clip}%
\pgfsetbuttcap%
\pgfsetroundjoin%
\definecolor{currentfill}{rgb}{0.121569,0.466667,0.705882}%
\pgfsetfillcolor{currentfill}%
\pgfsetfillopacity{0.432592}%
\pgfsetlinewidth{1.003750pt}%
\definecolor{currentstroke}{rgb}{0.121569,0.466667,0.705882}%
\pgfsetstrokecolor{currentstroke}%
\pgfsetstrokeopacity{0.432592}%
\pgfsetdash{}{0pt}%
\pgfpathmoveto{\pgfqpoint{1.525417in}{1.831062in}}%
\pgfpathcurveto{\pgfqpoint{1.533653in}{1.831062in}}{\pgfqpoint{1.541553in}{1.834335in}}{\pgfqpoint{1.547377in}{1.840159in}}%
\pgfpathcurveto{\pgfqpoint{1.553201in}{1.845982in}}{\pgfqpoint{1.556473in}{1.853883in}}{\pgfqpoint{1.556473in}{1.862119in}}%
\pgfpathcurveto{\pgfqpoint{1.556473in}{1.870355in}}{\pgfqpoint{1.553201in}{1.878255in}}{\pgfqpoint{1.547377in}{1.884079in}}%
\pgfpathcurveto{\pgfqpoint{1.541553in}{1.889903in}}{\pgfqpoint{1.533653in}{1.893175in}}{\pgfqpoint{1.525417in}{1.893175in}}%
\pgfpathcurveto{\pgfqpoint{1.517180in}{1.893175in}}{\pgfqpoint{1.509280in}{1.889903in}}{\pgfqpoint{1.503456in}{1.884079in}}%
\pgfpathcurveto{\pgfqpoint{1.497632in}{1.878255in}}{\pgfqpoint{1.494360in}{1.870355in}}{\pgfqpoint{1.494360in}{1.862119in}}%
\pgfpathcurveto{\pgfqpoint{1.494360in}{1.853883in}}{\pgfqpoint{1.497632in}{1.845982in}}{\pgfqpoint{1.503456in}{1.840159in}}%
\pgfpathcurveto{\pgfqpoint{1.509280in}{1.834335in}}{\pgfqpoint{1.517180in}{1.831062in}}{\pgfqpoint{1.525417in}{1.831062in}}%
\pgfpathclose%
\pgfusepath{stroke,fill}%
\end{pgfscope}%
\begin{pgfscope}%
\pgfpathrectangle{\pgfqpoint{0.100000in}{0.220728in}}{\pgfqpoint{3.696000in}{3.696000in}}%
\pgfusepath{clip}%
\pgfsetbuttcap%
\pgfsetroundjoin%
\definecolor{currentfill}{rgb}{0.121569,0.466667,0.705882}%
\pgfsetfillcolor{currentfill}%
\pgfsetfillopacity{0.433970}%
\pgfsetlinewidth{1.003750pt}%
\definecolor{currentstroke}{rgb}{0.121569,0.466667,0.705882}%
\pgfsetstrokecolor{currentstroke}%
\pgfsetstrokeopacity{0.433970}%
\pgfsetdash{}{0pt}%
\pgfpathmoveto{\pgfqpoint{1.522436in}{1.830843in}}%
\pgfpathcurveto{\pgfqpoint{1.530672in}{1.830843in}}{\pgfqpoint{1.538572in}{1.834115in}}{\pgfqpoint{1.544396in}{1.839939in}}%
\pgfpathcurveto{\pgfqpoint{1.550220in}{1.845763in}}{\pgfqpoint{1.553492in}{1.853663in}}{\pgfqpoint{1.553492in}{1.861900in}}%
\pgfpathcurveto{\pgfqpoint{1.553492in}{1.870136in}}{\pgfqpoint{1.550220in}{1.878036in}}{\pgfqpoint{1.544396in}{1.883860in}}%
\pgfpathcurveto{\pgfqpoint{1.538572in}{1.889684in}}{\pgfqpoint{1.530672in}{1.892956in}}{\pgfqpoint{1.522436in}{1.892956in}}%
\pgfpathcurveto{\pgfqpoint{1.514199in}{1.892956in}}{\pgfqpoint{1.506299in}{1.889684in}}{\pgfqpoint{1.500475in}{1.883860in}}%
\pgfpathcurveto{\pgfqpoint{1.494652in}{1.878036in}}{\pgfqpoint{1.491379in}{1.870136in}}{\pgfqpoint{1.491379in}{1.861900in}}%
\pgfpathcurveto{\pgfqpoint{1.491379in}{1.853663in}}{\pgfqpoint{1.494652in}{1.845763in}}{\pgfqpoint{1.500475in}{1.839939in}}%
\pgfpathcurveto{\pgfqpoint{1.506299in}{1.834115in}}{\pgfqpoint{1.514199in}{1.830843in}}{\pgfqpoint{1.522436in}{1.830843in}}%
\pgfpathclose%
\pgfusepath{stroke,fill}%
\end{pgfscope}%
\begin{pgfscope}%
\pgfpathrectangle{\pgfqpoint{0.100000in}{0.220728in}}{\pgfqpoint{3.696000in}{3.696000in}}%
\pgfusepath{clip}%
\pgfsetbuttcap%
\pgfsetroundjoin%
\definecolor{currentfill}{rgb}{0.121569,0.466667,0.705882}%
\pgfsetfillcolor{currentfill}%
\pgfsetfillopacity{0.434273}%
\pgfsetlinewidth{1.003750pt}%
\definecolor{currentstroke}{rgb}{0.121569,0.466667,0.705882}%
\pgfsetstrokecolor{currentstroke}%
\pgfsetstrokeopacity{0.434273}%
\pgfsetdash{}{0pt}%
\pgfpathmoveto{\pgfqpoint{1.521007in}{1.830296in}}%
\pgfpathcurveto{\pgfqpoint{1.529243in}{1.830296in}}{\pgfqpoint{1.537143in}{1.833569in}}{\pgfqpoint{1.542967in}{1.839393in}}%
\pgfpathcurveto{\pgfqpoint{1.548791in}{1.845216in}}{\pgfqpoint{1.552063in}{1.853117in}}{\pgfqpoint{1.552063in}{1.861353in}}%
\pgfpathcurveto{\pgfqpoint{1.552063in}{1.869589in}}{\pgfqpoint{1.548791in}{1.877489in}}{\pgfqpoint{1.542967in}{1.883313in}}%
\pgfpathcurveto{\pgfqpoint{1.537143in}{1.889137in}}{\pgfqpoint{1.529243in}{1.892409in}}{\pgfqpoint{1.521007in}{1.892409in}}%
\pgfpathcurveto{\pgfqpoint{1.512771in}{1.892409in}}{\pgfqpoint{1.504871in}{1.889137in}}{\pgfqpoint{1.499047in}{1.883313in}}%
\pgfpathcurveto{\pgfqpoint{1.493223in}{1.877489in}}{\pgfqpoint{1.489950in}{1.869589in}}{\pgfqpoint{1.489950in}{1.861353in}}%
\pgfpathcurveto{\pgfqpoint{1.489950in}{1.853117in}}{\pgfqpoint{1.493223in}{1.845216in}}{\pgfqpoint{1.499047in}{1.839393in}}%
\pgfpathcurveto{\pgfqpoint{1.504871in}{1.833569in}}{\pgfqpoint{1.512771in}{1.830296in}}{\pgfqpoint{1.521007in}{1.830296in}}%
\pgfpathclose%
\pgfusepath{stroke,fill}%
\end{pgfscope}%
\begin{pgfscope}%
\pgfpathrectangle{\pgfqpoint{0.100000in}{0.220728in}}{\pgfqpoint{3.696000in}{3.696000in}}%
\pgfusepath{clip}%
\pgfsetbuttcap%
\pgfsetroundjoin%
\definecolor{currentfill}{rgb}{0.121569,0.466667,0.705882}%
\pgfsetfillcolor{currentfill}%
\pgfsetfillopacity{0.434672}%
\pgfsetlinewidth{1.003750pt}%
\definecolor{currentstroke}{rgb}{0.121569,0.466667,0.705882}%
\pgfsetstrokecolor{currentstroke}%
\pgfsetstrokeopacity{0.434672}%
\pgfsetdash{}{0pt}%
\pgfpathmoveto{\pgfqpoint{1.518726in}{1.827651in}}%
\pgfpathcurveto{\pgfqpoint{1.526962in}{1.827651in}}{\pgfqpoint{1.534862in}{1.830923in}}{\pgfqpoint{1.540686in}{1.836747in}}%
\pgfpathcurveto{\pgfqpoint{1.546510in}{1.842571in}}{\pgfqpoint{1.549782in}{1.850471in}}{\pgfqpoint{1.549782in}{1.858707in}}%
\pgfpathcurveto{\pgfqpoint{1.549782in}{1.866943in}}{\pgfqpoint{1.546510in}{1.874844in}}{\pgfqpoint{1.540686in}{1.880667in}}%
\pgfpathcurveto{\pgfqpoint{1.534862in}{1.886491in}}{\pgfqpoint{1.526962in}{1.889764in}}{\pgfqpoint{1.518726in}{1.889764in}}%
\pgfpathcurveto{\pgfqpoint{1.510489in}{1.889764in}}{\pgfqpoint{1.502589in}{1.886491in}}{\pgfqpoint{1.496765in}{1.880667in}}%
\pgfpathcurveto{\pgfqpoint{1.490942in}{1.874844in}}{\pgfqpoint{1.487669in}{1.866943in}}{\pgfqpoint{1.487669in}{1.858707in}}%
\pgfpathcurveto{\pgfqpoint{1.487669in}{1.850471in}}{\pgfqpoint{1.490942in}{1.842571in}}{\pgfqpoint{1.496765in}{1.836747in}}%
\pgfpathcurveto{\pgfqpoint{1.502589in}{1.830923in}}{\pgfqpoint{1.510489in}{1.827651in}}{\pgfqpoint{1.518726in}{1.827651in}}%
\pgfpathclose%
\pgfusepath{stroke,fill}%
\end{pgfscope}%
\begin{pgfscope}%
\pgfpathrectangle{\pgfqpoint{0.100000in}{0.220728in}}{\pgfqpoint{3.696000in}{3.696000in}}%
\pgfusepath{clip}%
\pgfsetbuttcap%
\pgfsetroundjoin%
\definecolor{currentfill}{rgb}{0.121569,0.466667,0.705882}%
\pgfsetfillcolor{currentfill}%
\pgfsetfillopacity{0.434755}%
\pgfsetlinewidth{1.003750pt}%
\definecolor{currentstroke}{rgb}{0.121569,0.466667,0.705882}%
\pgfsetstrokecolor{currentstroke}%
\pgfsetstrokeopacity{0.434755}%
\pgfsetdash{}{0pt}%
\pgfpathmoveto{\pgfqpoint{2.021019in}{1.926907in}}%
\pgfpathcurveto{\pgfqpoint{2.029255in}{1.926907in}}{\pgfqpoint{2.037155in}{1.930179in}}{\pgfqpoint{2.042979in}{1.936003in}}%
\pgfpathcurveto{\pgfqpoint{2.048803in}{1.941827in}}{\pgfqpoint{2.052075in}{1.949727in}}{\pgfqpoint{2.052075in}{1.957963in}}%
\pgfpathcurveto{\pgfqpoint{2.052075in}{1.966199in}}{\pgfqpoint{2.048803in}{1.974099in}}{\pgfqpoint{2.042979in}{1.979923in}}%
\pgfpathcurveto{\pgfqpoint{2.037155in}{1.985747in}}{\pgfqpoint{2.029255in}{1.989020in}}{\pgfqpoint{2.021019in}{1.989020in}}%
\pgfpathcurveto{\pgfqpoint{2.012782in}{1.989020in}}{\pgfqpoint{2.004882in}{1.985747in}}{\pgfqpoint{1.999058in}{1.979923in}}%
\pgfpathcurveto{\pgfqpoint{1.993235in}{1.974099in}}{\pgfqpoint{1.989962in}{1.966199in}}{\pgfqpoint{1.989962in}{1.957963in}}%
\pgfpathcurveto{\pgfqpoint{1.989962in}{1.949727in}}{\pgfqpoint{1.993235in}{1.941827in}}{\pgfqpoint{1.999058in}{1.936003in}}%
\pgfpathcurveto{\pgfqpoint{2.004882in}{1.930179in}}{\pgfqpoint{2.012782in}{1.926907in}}{\pgfqpoint{2.021019in}{1.926907in}}%
\pgfpathclose%
\pgfusepath{stroke,fill}%
\end{pgfscope}%
\begin{pgfscope}%
\pgfpathrectangle{\pgfqpoint{0.100000in}{0.220728in}}{\pgfqpoint{3.696000in}{3.696000in}}%
\pgfusepath{clip}%
\pgfsetbuttcap%
\pgfsetroundjoin%
\definecolor{currentfill}{rgb}{0.121569,0.466667,0.705882}%
\pgfsetfillcolor{currentfill}%
\pgfsetfillopacity{0.436174}%
\pgfsetlinewidth{1.003750pt}%
\definecolor{currentstroke}{rgb}{0.121569,0.466667,0.705882}%
\pgfsetstrokecolor{currentstroke}%
\pgfsetstrokeopacity{0.436174}%
\pgfsetdash{}{0pt}%
\pgfpathmoveto{\pgfqpoint{1.515105in}{1.827164in}}%
\pgfpathcurveto{\pgfqpoint{1.523342in}{1.827164in}}{\pgfqpoint{1.531242in}{1.830436in}}{\pgfqpoint{1.537066in}{1.836260in}}%
\pgfpathcurveto{\pgfqpoint{1.542890in}{1.842084in}}{\pgfqpoint{1.546162in}{1.849984in}}{\pgfqpoint{1.546162in}{1.858221in}}%
\pgfpathcurveto{\pgfqpoint{1.546162in}{1.866457in}}{\pgfqpoint{1.542890in}{1.874357in}}{\pgfqpoint{1.537066in}{1.880181in}}%
\pgfpathcurveto{\pgfqpoint{1.531242in}{1.886005in}}{\pgfqpoint{1.523342in}{1.889277in}}{\pgfqpoint{1.515105in}{1.889277in}}%
\pgfpathcurveto{\pgfqpoint{1.506869in}{1.889277in}}{\pgfqpoint{1.498969in}{1.886005in}}{\pgfqpoint{1.493145in}{1.880181in}}%
\pgfpathcurveto{\pgfqpoint{1.487321in}{1.874357in}}{\pgfqpoint{1.484049in}{1.866457in}}{\pgfqpoint{1.484049in}{1.858221in}}%
\pgfpathcurveto{\pgfqpoint{1.484049in}{1.849984in}}{\pgfqpoint{1.487321in}{1.842084in}}{\pgfqpoint{1.493145in}{1.836260in}}%
\pgfpathcurveto{\pgfqpoint{1.498969in}{1.830436in}}{\pgfqpoint{1.506869in}{1.827164in}}{\pgfqpoint{1.515105in}{1.827164in}}%
\pgfpathclose%
\pgfusepath{stroke,fill}%
\end{pgfscope}%
\begin{pgfscope}%
\pgfpathrectangle{\pgfqpoint{0.100000in}{0.220728in}}{\pgfqpoint{3.696000in}{3.696000in}}%
\pgfusepath{clip}%
\pgfsetbuttcap%
\pgfsetroundjoin%
\definecolor{currentfill}{rgb}{0.121569,0.466667,0.705882}%
\pgfsetfillcolor{currentfill}%
\pgfsetfillopacity{0.436592}%
\pgfsetlinewidth{1.003750pt}%
\definecolor{currentstroke}{rgb}{0.121569,0.466667,0.705882}%
\pgfsetstrokecolor{currentstroke}%
\pgfsetstrokeopacity{0.436592}%
\pgfsetdash{}{0pt}%
\pgfpathmoveto{\pgfqpoint{1.513151in}{1.825404in}}%
\pgfpathcurveto{\pgfqpoint{1.521388in}{1.825404in}}{\pgfqpoint{1.529288in}{1.828677in}}{\pgfqpoint{1.535112in}{1.834501in}}%
\pgfpathcurveto{\pgfqpoint{1.540936in}{1.840325in}}{\pgfqpoint{1.544208in}{1.848225in}}{\pgfqpoint{1.544208in}{1.856461in}}%
\pgfpathcurveto{\pgfqpoint{1.544208in}{1.864697in}}{\pgfqpoint{1.540936in}{1.872597in}}{\pgfqpoint{1.535112in}{1.878421in}}%
\pgfpathcurveto{\pgfqpoint{1.529288in}{1.884245in}}{\pgfqpoint{1.521388in}{1.887517in}}{\pgfqpoint{1.513151in}{1.887517in}}%
\pgfpathcurveto{\pgfqpoint{1.504915in}{1.887517in}}{\pgfqpoint{1.497015in}{1.884245in}}{\pgfqpoint{1.491191in}{1.878421in}}%
\pgfpathcurveto{\pgfqpoint{1.485367in}{1.872597in}}{\pgfqpoint{1.482095in}{1.864697in}}{\pgfqpoint{1.482095in}{1.856461in}}%
\pgfpathcurveto{\pgfqpoint{1.482095in}{1.848225in}}{\pgfqpoint{1.485367in}{1.840325in}}{\pgfqpoint{1.491191in}{1.834501in}}%
\pgfpathcurveto{\pgfqpoint{1.497015in}{1.828677in}}{\pgfqpoint{1.504915in}{1.825404in}}{\pgfqpoint{1.513151in}{1.825404in}}%
\pgfpathclose%
\pgfusepath{stroke,fill}%
\end{pgfscope}%
\begin{pgfscope}%
\pgfpathrectangle{\pgfqpoint{0.100000in}{0.220728in}}{\pgfqpoint{3.696000in}{3.696000in}}%
\pgfusepath{clip}%
\pgfsetbuttcap%
\pgfsetroundjoin%
\definecolor{currentfill}{rgb}{0.121569,0.466667,0.705882}%
\pgfsetfillcolor{currentfill}%
\pgfsetfillopacity{0.436878}%
\pgfsetlinewidth{1.003750pt}%
\definecolor{currentstroke}{rgb}{0.121569,0.466667,0.705882}%
\pgfsetstrokecolor{currentstroke}%
\pgfsetstrokeopacity{0.436878}%
\pgfsetdash{}{0pt}%
\pgfpathmoveto{\pgfqpoint{2.022096in}{1.927123in}}%
\pgfpathcurveto{\pgfqpoint{2.030332in}{1.927123in}}{\pgfqpoint{2.038232in}{1.930396in}}{\pgfqpoint{2.044056in}{1.936219in}}%
\pgfpathcurveto{\pgfqpoint{2.049880in}{1.942043in}}{\pgfqpoint{2.053152in}{1.949943in}}{\pgfqpoint{2.053152in}{1.958180in}}%
\pgfpathcurveto{\pgfqpoint{2.053152in}{1.966416in}}{\pgfqpoint{2.049880in}{1.974316in}}{\pgfqpoint{2.044056in}{1.980140in}}%
\pgfpathcurveto{\pgfqpoint{2.038232in}{1.985964in}}{\pgfqpoint{2.030332in}{1.989236in}}{\pgfqpoint{2.022096in}{1.989236in}}%
\pgfpathcurveto{\pgfqpoint{2.013859in}{1.989236in}}{\pgfqpoint{2.005959in}{1.985964in}}{\pgfqpoint{2.000135in}{1.980140in}}%
\pgfpathcurveto{\pgfqpoint{1.994312in}{1.974316in}}{\pgfqpoint{1.991039in}{1.966416in}}{\pgfqpoint{1.991039in}{1.958180in}}%
\pgfpathcurveto{\pgfqpoint{1.991039in}{1.949943in}}{\pgfqpoint{1.994312in}{1.942043in}}{\pgfqpoint{2.000135in}{1.936219in}}%
\pgfpathcurveto{\pgfqpoint{2.005959in}{1.930396in}}{\pgfqpoint{2.013859in}{1.927123in}}{\pgfqpoint{2.022096in}{1.927123in}}%
\pgfpathclose%
\pgfusepath{stroke,fill}%
\end{pgfscope}%
\begin{pgfscope}%
\pgfpathrectangle{\pgfqpoint{0.100000in}{0.220728in}}{\pgfqpoint{3.696000in}{3.696000in}}%
\pgfusepath{clip}%
\pgfsetbuttcap%
\pgfsetroundjoin%
\definecolor{currentfill}{rgb}{0.121569,0.466667,0.705882}%
\pgfsetfillcolor{currentfill}%
\pgfsetfillopacity{0.437702}%
\pgfsetlinewidth{1.003750pt}%
\definecolor{currentstroke}{rgb}{0.121569,0.466667,0.705882}%
\pgfsetstrokecolor{currentstroke}%
\pgfsetstrokeopacity{0.437702}%
\pgfsetdash{}{0pt}%
\pgfpathmoveto{\pgfqpoint{1.509741in}{1.824270in}}%
\pgfpathcurveto{\pgfqpoint{1.517978in}{1.824270in}}{\pgfqpoint{1.525878in}{1.827542in}}{\pgfqpoint{1.531702in}{1.833366in}}%
\pgfpathcurveto{\pgfqpoint{1.537526in}{1.839190in}}{\pgfqpoint{1.540798in}{1.847090in}}{\pgfqpoint{1.540798in}{1.855327in}}%
\pgfpathcurveto{\pgfqpoint{1.540798in}{1.863563in}}{\pgfqpoint{1.537526in}{1.871463in}}{\pgfqpoint{1.531702in}{1.877287in}}%
\pgfpathcurveto{\pgfqpoint{1.525878in}{1.883111in}}{\pgfqpoint{1.517978in}{1.886383in}}{\pgfqpoint{1.509741in}{1.886383in}}%
\pgfpathcurveto{\pgfqpoint{1.501505in}{1.886383in}}{\pgfqpoint{1.493605in}{1.883111in}}{\pgfqpoint{1.487781in}{1.877287in}}%
\pgfpathcurveto{\pgfqpoint{1.481957in}{1.871463in}}{\pgfqpoint{1.478685in}{1.863563in}}{\pgfqpoint{1.478685in}{1.855327in}}%
\pgfpathcurveto{\pgfqpoint{1.478685in}{1.847090in}}{\pgfqpoint{1.481957in}{1.839190in}}{\pgfqpoint{1.487781in}{1.833366in}}%
\pgfpathcurveto{\pgfqpoint{1.493605in}{1.827542in}}{\pgfqpoint{1.501505in}{1.824270in}}{\pgfqpoint{1.509741in}{1.824270in}}%
\pgfpathclose%
\pgfusepath{stroke,fill}%
\end{pgfscope}%
\begin{pgfscope}%
\pgfpathrectangle{\pgfqpoint{0.100000in}{0.220728in}}{\pgfqpoint{3.696000in}{3.696000in}}%
\pgfusepath{clip}%
\pgfsetbuttcap%
\pgfsetroundjoin%
\definecolor{currentfill}{rgb}{0.121569,0.466667,0.705882}%
\pgfsetfillcolor{currentfill}%
\pgfsetfillopacity{0.439182}%
\pgfsetlinewidth{1.003750pt}%
\definecolor{currentstroke}{rgb}{0.121569,0.466667,0.705882}%
\pgfsetstrokecolor{currentstroke}%
\pgfsetstrokeopacity{0.439182}%
\pgfsetdash{}{0pt}%
\pgfpathmoveto{\pgfqpoint{2.022231in}{1.925685in}}%
\pgfpathcurveto{\pgfqpoint{2.030467in}{1.925685in}}{\pgfqpoint{2.038367in}{1.928957in}}{\pgfqpoint{2.044191in}{1.934781in}}%
\pgfpathcurveto{\pgfqpoint{2.050015in}{1.940605in}}{\pgfqpoint{2.053287in}{1.948505in}}{\pgfqpoint{2.053287in}{1.956741in}}%
\pgfpathcurveto{\pgfqpoint{2.053287in}{1.964978in}}{\pgfqpoint{2.050015in}{1.972878in}}{\pgfqpoint{2.044191in}{1.978702in}}%
\pgfpathcurveto{\pgfqpoint{2.038367in}{1.984525in}}{\pgfqpoint{2.030467in}{1.987798in}}{\pgfqpoint{2.022231in}{1.987798in}}%
\pgfpathcurveto{\pgfqpoint{2.013994in}{1.987798in}}{\pgfqpoint{2.006094in}{1.984525in}}{\pgfqpoint{2.000270in}{1.978702in}}%
\pgfpathcurveto{\pgfqpoint{1.994446in}{1.972878in}}{\pgfqpoint{1.991174in}{1.964978in}}{\pgfqpoint{1.991174in}{1.956741in}}%
\pgfpathcurveto{\pgfqpoint{1.991174in}{1.948505in}}{\pgfqpoint{1.994446in}{1.940605in}}{\pgfqpoint{2.000270in}{1.934781in}}%
\pgfpathcurveto{\pgfqpoint{2.006094in}{1.928957in}}{\pgfqpoint{2.013994in}{1.925685in}}{\pgfqpoint{2.022231in}{1.925685in}}%
\pgfpathclose%
\pgfusepath{stroke,fill}%
\end{pgfscope}%
\begin{pgfscope}%
\pgfpathrectangle{\pgfqpoint{0.100000in}{0.220728in}}{\pgfqpoint{3.696000in}{3.696000in}}%
\pgfusepath{clip}%
\pgfsetbuttcap%
\pgfsetroundjoin%
\definecolor{currentfill}{rgb}{0.121569,0.466667,0.705882}%
\pgfsetfillcolor{currentfill}%
\pgfsetfillopacity{0.440163}%
\pgfsetlinewidth{1.003750pt}%
\definecolor{currentstroke}{rgb}{0.121569,0.466667,0.705882}%
\pgfsetstrokecolor{currentstroke}%
\pgfsetstrokeopacity{0.440163}%
\pgfsetdash{}{0pt}%
\pgfpathmoveto{\pgfqpoint{1.504335in}{1.824005in}}%
\pgfpathcurveto{\pgfqpoint{1.512571in}{1.824005in}}{\pgfqpoint{1.520471in}{1.827277in}}{\pgfqpoint{1.526295in}{1.833101in}}%
\pgfpathcurveto{\pgfqpoint{1.532119in}{1.838925in}}{\pgfqpoint{1.535391in}{1.846825in}}{\pgfqpoint{1.535391in}{1.855061in}}%
\pgfpathcurveto{\pgfqpoint{1.535391in}{1.863298in}}{\pgfqpoint{1.532119in}{1.871198in}}{\pgfqpoint{1.526295in}{1.877022in}}%
\pgfpathcurveto{\pgfqpoint{1.520471in}{1.882846in}}{\pgfqpoint{1.512571in}{1.886118in}}{\pgfqpoint{1.504335in}{1.886118in}}%
\pgfpathcurveto{\pgfqpoint{1.496099in}{1.886118in}}{\pgfqpoint{1.488199in}{1.882846in}}{\pgfqpoint{1.482375in}{1.877022in}}%
\pgfpathcurveto{\pgfqpoint{1.476551in}{1.871198in}}{\pgfqpoint{1.473278in}{1.863298in}}{\pgfqpoint{1.473278in}{1.855061in}}%
\pgfpathcurveto{\pgfqpoint{1.473278in}{1.846825in}}{\pgfqpoint{1.476551in}{1.838925in}}{\pgfqpoint{1.482375in}{1.833101in}}%
\pgfpathcurveto{\pgfqpoint{1.488199in}{1.827277in}}{\pgfqpoint{1.496099in}{1.824005in}}{\pgfqpoint{1.504335in}{1.824005in}}%
\pgfpathclose%
\pgfusepath{stroke,fill}%
\end{pgfscope}%
\begin{pgfscope}%
\pgfpathrectangle{\pgfqpoint{0.100000in}{0.220728in}}{\pgfqpoint{3.696000in}{3.696000in}}%
\pgfusepath{clip}%
\pgfsetbuttcap%
\pgfsetroundjoin%
\definecolor{currentfill}{rgb}{0.121569,0.466667,0.705882}%
\pgfsetfillcolor{currentfill}%
\pgfsetfillopacity{0.441321}%
\pgfsetlinewidth{1.003750pt}%
\definecolor{currentstroke}{rgb}{0.121569,0.466667,0.705882}%
\pgfsetstrokecolor{currentstroke}%
\pgfsetstrokeopacity{0.441321}%
\pgfsetdash{}{0pt}%
\pgfpathmoveto{\pgfqpoint{1.499083in}{1.820974in}}%
\pgfpathcurveto{\pgfqpoint{1.507319in}{1.820974in}}{\pgfqpoint{1.515219in}{1.824246in}}{\pgfqpoint{1.521043in}{1.830070in}}%
\pgfpathcurveto{\pgfqpoint{1.526867in}{1.835894in}}{\pgfqpoint{1.530139in}{1.843794in}}{\pgfqpoint{1.530139in}{1.852030in}}%
\pgfpathcurveto{\pgfqpoint{1.530139in}{1.860266in}}{\pgfqpoint{1.526867in}{1.868166in}}{\pgfqpoint{1.521043in}{1.873990in}}%
\pgfpathcurveto{\pgfqpoint{1.515219in}{1.879814in}}{\pgfqpoint{1.507319in}{1.883087in}}{\pgfqpoint{1.499083in}{1.883087in}}%
\pgfpathcurveto{\pgfqpoint{1.490846in}{1.883087in}}{\pgfqpoint{1.482946in}{1.879814in}}{\pgfqpoint{1.477122in}{1.873990in}}%
\pgfpathcurveto{\pgfqpoint{1.471299in}{1.868166in}}{\pgfqpoint{1.468026in}{1.860266in}}{\pgfqpoint{1.468026in}{1.852030in}}%
\pgfpathcurveto{\pgfqpoint{1.468026in}{1.843794in}}{\pgfqpoint{1.471299in}{1.835894in}}{\pgfqpoint{1.477122in}{1.830070in}}%
\pgfpathcurveto{\pgfqpoint{1.482946in}{1.824246in}}{\pgfqpoint{1.490846in}{1.820974in}}{\pgfqpoint{1.499083in}{1.820974in}}%
\pgfpathclose%
\pgfusepath{stroke,fill}%
\end{pgfscope}%
\begin{pgfscope}%
\pgfpathrectangle{\pgfqpoint{0.100000in}{0.220728in}}{\pgfqpoint{3.696000in}{3.696000in}}%
\pgfusepath{clip}%
\pgfsetbuttcap%
\pgfsetroundjoin%
\definecolor{currentfill}{rgb}{0.121569,0.466667,0.705882}%
\pgfsetfillcolor{currentfill}%
\pgfsetfillopacity{0.441800}%
\pgfsetlinewidth{1.003750pt}%
\definecolor{currentstroke}{rgb}{0.121569,0.466667,0.705882}%
\pgfsetstrokecolor{currentstroke}%
\pgfsetstrokeopacity{0.441800}%
\pgfsetdash{}{0pt}%
\pgfpathmoveto{\pgfqpoint{2.022693in}{1.922880in}}%
\pgfpathcurveto{\pgfqpoint{2.030929in}{1.922880in}}{\pgfqpoint{2.038829in}{1.926153in}}{\pgfqpoint{2.044653in}{1.931977in}}%
\pgfpathcurveto{\pgfqpoint{2.050477in}{1.937801in}}{\pgfqpoint{2.053749in}{1.945701in}}{\pgfqpoint{2.053749in}{1.953937in}}%
\pgfpathcurveto{\pgfqpoint{2.053749in}{1.962173in}}{\pgfqpoint{2.050477in}{1.970073in}}{\pgfqpoint{2.044653in}{1.975897in}}%
\pgfpathcurveto{\pgfqpoint{2.038829in}{1.981721in}}{\pgfqpoint{2.030929in}{1.984993in}}{\pgfqpoint{2.022693in}{1.984993in}}%
\pgfpathcurveto{\pgfqpoint{2.014457in}{1.984993in}}{\pgfqpoint{2.006557in}{1.981721in}}{\pgfqpoint{2.000733in}{1.975897in}}%
\pgfpathcurveto{\pgfqpoint{1.994909in}{1.970073in}}{\pgfqpoint{1.991636in}{1.962173in}}{\pgfqpoint{1.991636in}{1.953937in}}%
\pgfpathcurveto{\pgfqpoint{1.991636in}{1.945701in}}{\pgfqpoint{1.994909in}{1.937801in}}{\pgfqpoint{2.000733in}{1.931977in}}%
\pgfpathcurveto{\pgfqpoint{2.006557in}{1.926153in}}{\pgfqpoint{2.014457in}{1.922880in}}{\pgfqpoint{2.022693in}{1.922880in}}%
\pgfpathclose%
\pgfusepath{stroke,fill}%
\end{pgfscope}%
\begin{pgfscope}%
\pgfpathrectangle{\pgfqpoint{0.100000in}{0.220728in}}{\pgfqpoint{3.696000in}{3.696000in}}%
\pgfusepath{clip}%
\pgfsetbuttcap%
\pgfsetroundjoin%
\definecolor{currentfill}{rgb}{0.121569,0.466667,0.705882}%
\pgfsetfillcolor{currentfill}%
\pgfsetfillopacity{0.442433}%
\pgfsetlinewidth{1.003750pt}%
\definecolor{currentstroke}{rgb}{0.121569,0.466667,0.705882}%
\pgfsetstrokecolor{currentstroke}%
\pgfsetstrokeopacity{0.442433}%
\pgfsetdash{}{0pt}%
\pgfpathmoveto{\pgfqpoint{1.495338in}{1.818693in}}%
\pgfpathcurveto{\pgfqpoint{1.503574in}{1.818693in}}{\pgfqpoint{1.511474in}{1.821965in}}{\pgfqpoint{1.517298in}{1.827789in}}%
\pgfpathcurveto{\pgfqpoint{1.523122in}{1.833613in}}{\pgfqpoint{1.526395in}{1.841513in}}{\pgfqpoint{1.526395in}{1.849750in}}%
\pgfpathcurveto{\pgfqpoint{1.526395in}{1.857986in}}{\pgfqpoint{1.523122in}{1.865886in}}{\pgfqpoint{1.517298in}{1.871710in}}%
\pgfpathcurveto{\pgfqpoint{1.511474in}{1.877534in}}{\pgfqpoint{1.503574in}{1.880806in}}{\pgfqpoint{1.495338in}{1.880806in}}%
\pgfpathcurveto{\pgfqpoint{1.487102in}{1.880806in}}{\pgfqpoint{1.479202in}{1.877534in}}{\pgfqpoint{1.473378in}{1.871710in}}%
\pgfpathcurveto{\pgfqpoint{1.467554in}{1.865886in}}{\pgfqpoint{1.464282in}{1.857986in}}{\pgfqpoint{1.464282in}{1.849750in}}%
\pgfpathcurveto{\pgfqpoint{1.464282in}{1.841513in}}{\pgfqpoint{1.467554in}{1.833613in}}{\pgfqpoint{1.473378in}{1.827789in}}%
\pgfpathcurveto{\pgfqpoint{1.479202in}{1.821965in}}{\pgfqpoint{1.487102in}{1.818693in}}{\pgfqpoint{1.495338in}{1.818693in}}%
\pgfpathclose%
\pgfusepath{stroke,fill}%
\end{pgfscope}%
\begin{pgfscope}%
\pgfpathrectangle{\pgfqpoint{0.100000in}{0.220728in}}{\pgfqpoint{3.696000in}{3.696000in}}%
\pgfusepath{clip}%
\pgfsetbuttcap%
\pgfsetroundjoin%
\definecolor{currentfill}{rgb}{0.121569,0.466667,0.705882}%
\pgfsetfillcolor{currentfill}%
\pgfsetfillopacity{0.443355}%
\pgfsetlinewidth{1.003750pt}%
\definecolor{currentstroke}{rgb}{0.121569,0.466667,0.705882}%
\pgfsetstrokecolor{currentstroke}%
\pgfsetstrokeopacity{0.443355}%
\pgfsetdash{}{0pt}%
\pgfpathmoveto{\pgfqpoint{2.023742in}{1.922446in}}%
\pgfpathcurveto{\pgfqpoint{2.031979in}{1.922446in}}{\pgfqpoint{2.039879in}{1.925718in}}{\pgfqpoint{2.045703in}{1.931542in}}%
\pgfpathcurveto{\pgfqpoint{2.051526in}{1.937366in}}{\pgfqpoint{2.054799in}{1.945266in}}{\pgfqpoint{2.054799in}{1.953502in}}%
\pgfpathcurveto{\pgfqpoint{2.054799in}{1.961738in}}{\pgfqpoint{2.051526in}{1.969638in}}{\pgfqpoint{2.045703in}{1.975462in}}%
\pgfpathcurveto{\pgfqpoint{2.039879in}{1.981286in}}{\pgfqpoint{2.031979in}{1.984559in}}{\pgfqpoint{2.023742in}{1.984559in}}%
\pgfpathcurveto{\pgfqpoint{2.015506in}{1.984559in}}{\pgfqpoint{2.007606in}{1.981286in}}{\pgfqpoint{2.001782in}{1.975462in}}%
\pgfpathcurveto{\pgfqpoint{1.995958in}{1.969638in}}{\pgfqpoint{1.992686in}{1.961738in}}{\pgfqpoint{1.992686in}{1.953502in}}%
\pgfpathcurveto{\pgfqpoint{1.992686in}{1.945266in}}{\pgfqpoint{1.995958in}{1.937366in}}{\pgfqpoint{2.001782in}{1.931542in}}%
\pgfpathcurveto{\pgfqpoint{2.007606in}{1.925718in}}{\pgfqpoint{2.015506in}{1.922446in}}{\pgfqpoint{2.023742in}{1.922446in}}%
\pgfpathclose%
\pgfusepath{stroke,fill}%
\end{pgfscope}%
\begin{pgfscope}%
\pgfpathrectangle{\pgfqpoint{0.100000in}{0.220728in}}{\pgfqpoint{3.696000in}{3.696000in}}%
\pgfusepath{clip}%
\pgfsetbuttcap%
\pgfsetroundjoin%
\definecolor{currentfill}{rgb}{0.121569,0.466667,0.705882}%
\pgfsetfillcolor{currentfill}%
\pgfsetfillopacity{0.443630}%
\pgfsetlinewidth{1.003750pt}%
\definecolor{currentstroke}{rgb}{0.121569,0.466667,0.705882}%
\pgfsetstrokecolor{currentstroke}%
\pgfsetstrokeopacity{0.443630}%
\pgfsetdash{}{0pt}%
\pgfpathmoveto{\pgfqpoint{1.492317in}{1.817433in}}%
\pgfpathcurveto{\pgfqpoint{1.500553in}{1.817433in}}{\pgfqpoint{1.508453in}{1.820705in}}{\pgfqpoint{1.514277in}{1.826529in}}%
\pgfpathcurveto{\pgfqpoint{1.520101in}{1.832353in}}{\pgfqpoint{1.523373in}{1.840253in}}{\pgfqpoint{1.523373in}{1.848489in}}%
\pgfpathcurveto{\pgfqpoint{1.523373in}{1.856725in}}{\pgfqpoint{1.520101in}{1.864626in}}{\pgfqpoint{1.514277in}{1.870449in}}%
\pgfpathcurveto{\pgfqpoint{1.508453in}{1.876273in}}{\pgfqpoint{1.500553in}{1.879546in}}{\pgfqpoint{1.492317in}{1.879546in}}%
\pgfpathcurveto{\pgfqpoint{1.484080in}{1.879546in}}{\pgfqpoint{1.476180in}{1.876273in}}{\pgfqpoint{1.470356in}{1.870449in}}%
\pgfpathcurveto{\pgfqpoint{1.464532in}{1.864626in}}{\pgfqpoint{1.461260in}{1.856725in}}{\pgfqpoint{1.461260in}{1.848489in}}%
\pgfpathcurveto{\pgfqpoint{1.461260in}{1.840253in}}{\pgfqpoint{1.464532in}{1.832353in}}{\pgfqpoint{1.470356in}{1.826529in}}%
\pgfpathcurveto{\pgfqpoint{1.476180in}{1.820705in}}{\pgfqpoint{1.484080in}{1.817433in}}{\pgfqpoint{1.492317in}{1.817433in}}%
\pgfpathclose%
\pgfusepath{stroke,fill}%
\end{pgfscope}%
\begin{pgfscope}%
\pgfpathrectangle{\pgfqpoint{0.100000in}{0.220728in}}{\pgfqpoint{3.696000in}{3.696000in}}%
\pgfusepath{clip}%
\pgfsetbuttcap%
\pgfsetroundjoin%
\definecolor{currentfill}{rgb}{0.121569,0.466667,0.705882}%
\pgfsetfillcolor{currentfill}%
\pgfsetfillopacity{0.444238}%
\pgfsetlinewidth{1.003750pt}%
\definecolor{currentstroke}{rgb}{0.121569,0.466667,0.705882}%
\pgfsetstrokecolor{currentstroke}%
\pgfsetstrokeopacity{0.444238}%
\pgfsetdash{}{0pt}%
\pgfpathmoveto{\pgfqpoint{1.489927in}{1.815901in}}%
\pgfpathcurveto{\pgfqpoint{1.498163in}{1.815901in}}{\pgfqpoint{1.506063in}{1.819174in}}{\pgfqpoint{1.511887in}{1.824998in}}%
\pgfpathcurveto{\pgfqpoint{1.517711in}{1.830821in}}{\pgfqpoint{1.520983in}{1.838722in}}{\pgfqpoint{1.520983in}{1.846958in}}%
\pgfpathcurveto{\pgfqpoint{1.520983in}{1.855194in}}{\pgfqpoint{1.517711in}{1.863094in}}{\pgfqpoint{1.511887in}{1.868918in}}%
\pgfpathcurveto{\pgfqpoint{1.506063in}{1.874742in}}{\pgfqpoint{1.498163in}{1.878014in}}{\pgfqpoint{1.489927in}{1.878014in}}%
\pgfpathcurveto{\pgfqpoint{1.481691in}{1.878014in}}{\pgfqpoint{1.473791in}{1.874742in}}{\pgfqpoint{1.467967in}{1.868918in}}%
\pgfpathcurveto{\pgfqpoint{1.462143in}{1.863094in}}{\pgfqpoint{1.458870in}{1.855194in}}{\pgfqpoint{1.458870in}{1.846958in}}%
\pgfpathcurveto{\pgfqpoint{1.458870in}{1.838722in}}{\pgfqpoint{1.462143in}{1.830821in}}{\pgfqpoint{1.467967in}{1.824998in}}%
\pgfpathcurveto{\pgfqpoint{1.473791in}{1.819174in}}{\pgfqpoint{1.481691in}{1.815901in}}{\pgfqpoint{1.489927in}{1.815901in}}%
\pgfpathclose%
\pgfusepath{stroke,fill}%
\end{pgfscope}%
\begin{pgfscope}%
\pgfpathrectangle{\pgfqpoint{0.100000in}{0.220728in}}{\pgfqpoint{3.696000in}{3.696000in}}%
\pgfusepath{clip}%
\pgfsetbuttcap%
\pgfsetroundjoin%
\definecolor{currentfill}{rgb}{0.121569,0.466667,0.705882}%
\pgfsetfillcolor{currentfill}%
\pgfsetfillopacity{0.444842}%
\pgfsetlinewidth{1.003750pt}%
\definecolor{currentstroke}{rgb}{0.121569,0.466667,0.705882}%
\pgfsetstrokecolor{currentstroke}%
\pgfsetstrokeopacity{0.444842}%
\pgfsetdash{}{0pt}%
\pgfpathmoveto{\pgfqpoint{1.488511in}{1.815664in}}%
\pgfpathcurveto{\pgfqpoint{1.496747in}{1.815664in}}{\pgfqpoint{1.504647in}{1.818937in}}{\pgfqpoint{1.510471in}{1.824761in}}%
\pgfpathcurveto{\pgfqpoint{1.516295in}{1.830584in}}{\pgfqpoint{1.519568in}{1.838484in}}{\pgfqpoint{1.519568in}{1.846721in}}%
\pgfpathcurveto{\pgfqpoint{1.519568in}{1.854957in}}{\pgfqpoint{1.516295in}{1.862857in}}{\pgfqpoint{1.510471in}{1.868681in}}%
\pgfpathcurveto{\pgfqpoint{1.504647in}{1.874505in}}{\pgfqpoint{1.496747in}{1.877777in}}{\pgfqpoint{1.488511in}{1.877777in}}%
\pgfpathcurveto{\pgfqpoint{1.480275in}{1.877777in}}{\pgfqpoint{1.472375in}{1.874505in}}{\pgfqpoint{1.466551in}{1.868681in}}%
\pgfpathcurveto{\pgfqpoint{1.460727in}{1.862857in}}{\pgfqpoint{1.457455in}{1.854957in}}{\pgfqpoint{1.457455in}{1.846721in}}%
\pgfpathcurveto{\pgfqpoint{1.457455in}{1.838484in}}{\pgfqpoint{1.460727in}{1.830584in}}{\pgfqpoint{1.466551in}{1.824761in}}%
\pgfpathcurveto{\pgfqpoint{1.472375in}{1.818937in}}{\pgfqpoint{1.480275in}{1.815664in}}{\pgfqpoint{1.488511in}{1.815664in}}%
\pgfpathclose%
\pgfusepath{stroke,fill}%
\end{pgfscope}%
\begin{pgfscope}%
\pgfpathrectangle{\pgfqpoint{0.100000in}{0.220728in}}{\pgfqpoint{3.696000in}{3.696000in}}%
\pgfusepath{clip}%
\pgfsetbuttcap%
\pgfsetroundjoin%
\definecolor{currentfill}{rgb}{0.121569,0.466667,0.705882}%
\pgfsetfillcolor{currentfill}%
\pgfsetfillopacity{0.445433}%
\pgfsetlinewidth{1.003750pt}%
\definecolor{currentstroke}{rgb}{0.121569,0.466667,0.705882}%
\pgfsetstrokecolor{currentstroke}%
\pgfsetstrokeopacity{0.445433}%
\pgfsetdash{}{0pt}%
\pgfpathmoveto{\pgfqpoint{2.024469in}{1.920720in}}%
\pgfpathcurveto{\pgfqpoint{2.032705in}{1.920720in}}{\pgfqpoint{2.040605in}{1.923993in}}{\pgfqpoint{2.046429in}{1.929817in}}%
\pgfpathcurveto{\pgfqpoint{2.052253in}{1.935641in}}{\pgfqpoint{2.055526in}{1.943541in}}{\pgfqpoint{2.055526in}{1.951777in}}%
\pgfpathcurveto{\pgfqpoint{2.055526in}{1.960013in}}{\pgfqpoint{2.052253in}{1.967913in}}{\pgfqpoint{2.046429in}{1.973737in}}%
\pgfpathcurveto{\pgfqpoint{2.040605in}{1.979561in}}{\pgfqpoint{2.032705in}{1.982833in}}{\pgfqpoint{2.024469in}{1.982833in}}%
\pgfpathcurveto{\pgfqpoint{2.016233in}{1.982833in}}{\pgfqpoint{2.008333in}{1.979561in}}{\pgfqpoint{2.002509in}{1.973737in}}%
\pgfpathcurveto{\pgfqpoint{1.996685in}{1.967913in}}{\pgfqpoint{1.993413in}{1.960013in}}{\pgfqpoint{1.993413in}{1.951777in}}%
\pgfpathcurveto{\pgfqpoint{1.993413in}{1.943541in}}{\pgfqpoint{1.996685in}{1.935641in}}{\pgfqpoint{2.002509in}{1.929817in}}%
\pgfpathcurveto{\pgfqpoint{2.008333in}{1.923993in}}{\pgfqpoint{2.016233in}{1.920720in}}{\pgfqpoint{2.024469in}{1.920720in}}%
\pgfpathclose%
\pgfusepath{stroke,fill}%
\end{pgfscope}%
\begin{pgfscope}%
\pgfpathrectangle{\pgfqpoint{0.100000in}{0.220728in}}{\pgfqpoint{3.696000in}{3.696000in}}%
\pgfusepath{clip}%
\pgfsetbuttcap%
\pgfsetroundjoin%
\definecolor{currentfill}{rgb}{0.121569,0.466667,0.705882}%
\pgfsetfillcolor{currentfill}%
\pgfsetfillopacity{0.445768}%
\pgfsetlinewidth{1.003750pt}%
\definecolor{currentstroke}{rgb}{0.121569,0.466667,0.705882}%
\pgfsetstrokecolor{currentstroke}%
\pgfsetstrokeopacity{0.445768}%
\pgfsetdash{}{0pt}%
\pgfpathmoveto{\pgfqpoint{1.485682in}{1.814439in}}%
\pgfpathcurveto{\pgfqpoint{1.493918in}{1.814439in}}{\pgfqpoint{1.501818in}{1.817711in}}{\pgfqpoint{1.507642in}{1.823535in}}%
\pgfpathcurveto{\pgfqpoint{1.513466in}{1.829359in}}{\pgfqpoint{1.516738in}{1.837259in}}{\pgfqpoint{1.516738in}{1.845495in}}%
\pgfpathcurveto{\pgfqpoint{1.516738in}{1.853732in}}{\pgfqpoint{1.513466in}{1.861632in}}{\pgfqpoint{1.507642in}{1.867456in}}%
\pgfpathcurveto{\pgfqpoint{1.501818in}{1.873280in}}{\pgfqpoint{1.493918in}{1.876552in}}{\pgfqpoint{1.485682in}{1.876552in}}%
\pgfpathcurveto{\pgfqpoint{1.477446in}{1.876552in}}{\pgfqpoint{1.469545in}{1.873280in}}{\pgfqpoint{1.463722in}{1.867456in}}%
\pgfpathcurveto{\pgfqpoint{1.457898in}{1.861632in}}{\pgfqpoint{1.454625in}{1.853732in}}{\pgfqpoint{1.454625in}{1.845495in}}%
\pgfpathcurveto{\pgfqpoint{1.454625in}{1.837259in}}{\pgfqpoint{1.457898in}{1.829359in}}{\pgfqpoint{1.463722in}{1.823535in}}%
\pgfpathcurveto{\pgfqpoint{1.469545in}{1.817711in}}{\pgfqpoint{1.477446in}{1.814439in}}{\pgfqpoint{1.485682in}{1.814439in}}%
\pgfpathclose%
\pgfusepath{stroke,fill}%
\end{pgfscope}%
\begin{pgfscope}%
\pgfpathrectangle{\pgfqpoint{0.100000in}{0.220728in}}{\pgfqpoint{3.696000in}{3.696000in}}%
\pgfusepath{clip}%
\pgfsetbuttcap%
\pgfsetroundjoin%
\definecolor{currentfill}{rgb}{0.121569,0.466667,0.705882}%
\pgfsetfillcolor{currentfill}%
\pgfsetfillopacity{0.446578}%
\pgfsetlinewidth{1.003750pt}%
\definecolor{currentstroke}{rgb}{0.121569,0.466667,0.705882}%
\pgfsetstrokecolor{currentstroke}%
\pgfsetstrokeopacity{0.446578}%
\pgfsetdash{}{0pt}%
\pgfpathmoveto{\pgfqpoint{2.024819in}{1.919764in}}%
\pgfpathcurveto{\pgfqpoint{2.033056in}{1.919764in}}{\pgfqpoint{2.040956in}{1.923036in}}{\pgfqpoint{2.046780in}{1.928860in}}%
\pgfpathcurveto{\pgfqpoint{2.052604in}{1.934684in}}{\pgfqpoint{2.055876in}{1.942584in}}{\pgfqpoint{2.055876in}{1.950820in}}%
\pgfpathcurveto{\pgfqpoint{2.055876in}{1.959057in}}{\pgfqpoint{2.052604in}{1.966957in}}{\pgfqpoint{2.046780in}{1.972781in}}%
\pgfpathcurveto{\pgfqpoint{2.040956in}{1.978605in}}{\pgfqpoint{2.033056in}{1.981877in}}{\pgfqpoint{2.024819in}{1.981877in}}%
\pgfpathcurveto{\pgfqpoint{2.016583in}{1.981877in}}{\pgfqpoint{2.008683in}{1.978605in}}{\pgfqpoint{2.002859in}{1.972781in}}%
\pgfpathcurveto{\pgfqpoint{1.997035in}{1.966957in}}{\pgfqpoint{1.993763in}{1.959057in}}{\pgfqpoint{1.993763in}{1.950820in}}%
\pgfpathcurveto{\pgfqpoint{1.993763in}{1.942584in}}{\pgfqpoint{1.997035in}{1.934684in}}{\pgfqpoint{2.002859in}{1.928860in}}%
\pgfpathcurveto{\pgfqpoint{2.008683in}{1.923036in}}{\pgfqpoint{2.016583in}{1.919764in}}{\pgfqpoint{2.024819in}{1.919764in}}%
\pgfpathclose%
\pgfusepath{stroke,fill}%
\end{pgfscope}%
\begin{pgfscope}%
\pgfpathrectangle{\pgfqpoint{0.100000in}{0.220728in}}{\pgfqpoint{3.696000in}{3.696000in}}%
\pgfusepath{clip}%
\pgfsetbuttcap%
\pgfsetroundjoin%
\definecolor{currentfill}{rgb}{0.121569,0.466667,0.705882}%
\pgfsetfillcolor{currentfill}%
\pgfsetfillopacity{0.447093}%
\pgfsetlinewidth{1.003750pt}%
\definecolor{currentstroke}{rgb}{0.121569,0.466667,0.705882}%
\pgfsetstrokecolor{currentstroke}%
\pgfsetstrokeopacity{0.447093}%
\pgfsetdash{}{0pt}%
\pgfpathmoveto{\pgfqpoint{1.480823in}{1.809417in}}%
\pgfpathcurveto{\pgfqpoint{1.489059in}{1.809417in}}{\pgfqpoint{1.496959in}{1.812689in}}{\pgfqpoint{1.502783in}{1.818513in}}%
\pgfpathcurveto{\pgfqpoint{1.508607in}{1.824337in}}{\pgfqpoint{1.511880in}{1.832237in}}{\pgfqpoint{1.511880in}{1.840474in}}%
\pgfpathcurveto{\pgfqpoint{1.511880in}{1.848710in}}{\pgfqpoint{1.508607in}{1.856610in}}{\pgfqpoint{1.502783in}{1.862434in}}%
\pgfpathcurveto{\pgfqpoint{1.496959in}{1.868258in}}{\pgfqpoint{1.489059in}{1.871530in}}{\pgfqpoint{1.480823in}{1.871530in}}%
\pgfpathcurveto{\pgfqpoint{1.472587in}{1.871530in}}{\pgfqpoint{1.464687in}{1.868258in}}{\pgfqpoint{1.458863in}{1.862434in}}%
\pgfpathcurveto{\pgfqpoint{1.453039in}{1.856610in}}{\pgfqpoint{1.449767in}{1.848710in}}{\pgfqpoint{1.449767in}{1.840474in}}%
\pgfpathcurveto{\pgfqpoint{1.449767in}{1.832237in}}{\pgfqpoint{1.453039in}{1.824337in}}{\pgfqpoint{1.458863in}{1.818513in}}%
\pgfpathcurveto{\pgfqpoint{1.464687in}{1.812689in}}{\pgfqpoint{1.472587in}{1.809417in}}{\pgfqpoint{1.480823in}{1.809417in}}%
\pgfpathclose%
\pgfusepath{stroke,fill}%
\end{pgfscope}%
\begin{pgfscope}%
\pgfpathrectangle{\pgfqpoint{0.100000in}{0.220728in}}{\pgfqpoint{3.696000in}{3.696000in}}%
\pgfusepath{clip}%
\pgfsetbuttcap%
\pgfsetroundjoin%
\definecolor{currentfill}{rgb}{0.121569,0.466667,0.705882}%
\pgfsetfillcolor{currentfill}%
\pgfsetfillopacity{0.447198}%
\pgfsetlinewidth{1.003750pt}%
\definecolor{currentstroke}{rgb}{0.121569,0.466667,0.705882}%
\pgfsetstrokecolor{currentstroke}%
\pgfsetstrokeopacity{0.447198}%
\pgfsetdash{}{0pt}%
\pgfpathmoveto{\pgfqpoint{2.025201in}{1.919249in}}%
\pgfpathcurveto{\pgfqpoint{2.033438in}{1.919249in}}{\pgfqpoint{2.041338in}{1.922521in}}{\pgfqpoint{2.047162in}{1.928345in}}%
\pgfpathcurveto{\pgfqpoint{2.052986in}{1.934169in}}{\pgfqpoint{2.056258in}{1.942069in}}{\pgfqpoint{2.056258in}{1.950305in}}%
\pgfpathcurveto{\pgfqpoint{2.056258in}{1.958541in}}{\pgfqpoint{2.052986in}{1.966442in}}{\pgfqpoint{2.047162in}{1.972265in}}%
\pgfpathcurveto{\pgfqpoint{2.041338in}{1.978089in}}{\pgfqpoint{2.033438in}{1.981362in}}{\pgfqpoint{2.025201in}{1.981362in}}%
\pgfpathcurveto{\pgfqpoint{2.016965in}{1.981362in}}{\pgfqpoint{2.009065in}{1.978089in}}{\pgfqpoint{2.003241in}{1.972265in}}%
\pgfpathcurveto{\pgfqpoint{1.997417in}{1.966442in}}{\pgfqpoint{1.994145in}{1.958541in}}{\pgfqpoint{1.994145in}{1.950305in}}%
\pgfpathcurveto{\pgfqpoint{1.994145in}{1.942069in}}{\pgfqpoint{1.997417in}{1.934169in}}{\pgfqpoint{2.003241in}{1.928345in}}%
\pgfpathcurveto{\pgfqpoint{2.009065in}{1.922521in}}{\pgfqpoint{2.016965in}{1.919249in}}{\pgfqpoint{2.025201in}{1.919249in}}%
\pgfpathclose%
\pgfusepath{stroke,fill}%
\end{pgfscope}%
\begin{pgfscope}%
\pgfpathrectangle{\pgfqpoint{0.100000in}{0.220728in}}{\pgfqpoint{3.696000in}{3.696000in}}%
\pgfusepath{clip}%
\pgfsetbuttcap%
\pgfsetroundjoin%
\definecolor{currentfill}{rgb}{0.121569,0.466667,0.705882}%
\pgfsetfillcolor{currentfill}%
\pgfsetfillopacity{0.447609}%
\pgfsetlinewidth{1.003750pt}%
\definecolor{currentstroke}{rgb}{0.121569,0.466667,0.705882}%
\pgfsetstrokecolor{currentstroke}%
\pgfsetstrokeopacity{0.447609}%
\pgfsetdash{}{0pt}%
\pgfpathmoveto{\pgfqpoint{2.025367in}{1.919417in}}%
\pgfpathcurveto{\pgfqpoint{2.033603in}{1.919417in}}{\pgfqpoint{2.041503in}{1.922689in}}{\pgfqpoint{2.047327in}{1.928513in}}%
\pgfpathcurveto{\pgfqpoint{2.053151in}{1.934337in}}{\pgfqpoint{2.056423in}{1.942237in}}{\pgfqpoint{2.056423in}{1.950473in}}%
\pgfpathcurveto{\pgfqpoint{2.056423in}{1.958710in}}{\pgfqpoint{2.053151in}{1.966610in}}{\pgfqpoint{2.047327in}{1.972434in}}%
\pgfpathcurveto{\pgfqpoint{2.041503in}{1.978258in}}{\pgfqpoint{2.033603in}{1.981530in}}{\pgfqpoint{2.025367in}{1.981530in}}%
\pgfpathcurveto{\pgfqpoint{2.017131in}{1.981530in}}{\pgfqpoint{2.009230in}{1.978258in}}{\pgfqpoint{2.003407in}{1.972434in}}%
\pgfpathcurveto{\pgfqpoint{1.997583in}{1.966610in}}{\pgfqpoint{1.994310in}{1.958710in}}{\pgfqpoint{1.994310in}{1.950473in}}%
\pgfpathcurveto{\pgfqpoint{1.994310in}{1.942237in}}{\pgfqpoint{1.997583in}{1.934337in}}{\pgfqpoint{2.003407in}{1.928513in}}%
\pgfpathcurveto{\pgfqpoint{2.009230in}{1.922689in}}{\pgfqpoint{2.017131in}{1.919417in}}{\pgfqpoint{2.025367in}{1.919417in}}%
\pgfpathclose%
\pgfusepath{stroke,fill}%
\end{pgfscope}%
\begin{pgfscope}%
\pgfpathrectangle{\pgfqpoint{0.100000in}{0.220728in}}{\pgfqpoint{3.696000in}{3.696000in}}%
\pgfusepath{clip}%
\pgfsetbuttcap%
\pgfsetroundjoin%
\definecolor{currentfill}{rgb}{0.121569,0.466667,0.705882}%
\pgfsetfillcolor{currentfill}%
\pgfsetfillopacity{0.447791}%
\pgfsetlinewidth{1.003750pt}%
\definecolor{currentstroke}{rgb}{0.121569,0.466667,0.705882}%
\pgfsetstrokecolor{currentstroke}%
\pgfsetstrokeopacity{0.447791}%
\pgfsetdash{}{0pt}%
\pgfpathmoveto{\pgfqpoint{2.025424in}{1.919204in}}%
\pgfpathcurveto{\pgfqpoint{2.033660in}{1.919204in}}{\pgfqpoint{2.041560in}{1.922476in}}{\pgfqpoint{2.047384in}{1.928300in}}%
\pgfpathcurveto{\pgfqpoint{2.053208in}{1.934124in}}{\pgfqpoint{2.056481in}{1.942024in}}{\pgfqpoint{2.056481in}{1.950260in}}%
\pgfpathcurveto{\pgfqpoint{2.056481in}{1.958497in}}{\pgfqpoint{2.053208in}{1.966397in}}{\pgfqpoint{2.047384in}{1.972221in}}%
\pgfpathcurveto{\pgfqpoint{2.041560in}{1.978045in}}{\pgfqpoint{2.033660in}{1.981317in}}{\pgfqpoint{2.025424in}{1.981317in}}%
\pgfpathcurveto{\pgfqpoint{2.017188in}{1.981317in}}{\pgfqpoint{2.009288in}{1.978045in}}{\pgfqpoint{2.003464in}{1.972221in}}%
\pgfpathcurveto{\pgfqpoint{1.997640in}{1.966397in}}{\pgfqpoint{1.994368in}{1.958497in}}{\pgfqpoint{1.994368in}{1.950260in}}%
\pgfpathcurveto{\pgfqpoint{1.994368in}{1.942024in}}{\pgfqpoint{1.997640in}{1.934124in}}{\pgfqpoint{2.003464in}{1.928300in}}%
\pgfpathcurveto{\pgfqpoint{2.009288in}{1.922476in}}{\pgfqpoint{2.017188in}{1.919204in}}{\pgfqpoint{2.025424in}{1.919204in}}%
\pgfpathclose%
\pgfusepath{stroke,fill}%
\end{pgfscope}%
\begin{pgfscope}%
\pgfpathrectangle{\pgfqpoint{0.100000in}{0.220728in}}{\pgfqpoint{3.696000in}{3.696000in}}%
\pgfusepath{clip}%
\pgfsetbuttcap%
\pgfsetroundjoin%
\definecolor{currentfill}{rgb}{0.121569,0.466667,0.705882}%
\pgfsetfillcolor{currentfill}%
\pgfsetfillopacity{0.448318}%
\pgfsetlinewidth{1.003750pt}%
\definecolor{currentstroke}{rgb}{0.121569,0.466667,0.705882}%
\pgfsetstrokecolor{currentstroke}%
\pgfsetstrokeopacity{0.448318}%
\pgfsetdash{}{0pt}%
\pgfpathmoveto{\pgfqpoint{2.025830in}{1.918149in}}%
\pgfpathcurveto{\pgfqpoint{2.034066in}{1.918149in}}{\pgfqpoint{2.041966in}{1.921422in}}{\pgfqpoint{2.047790in}{1.927246in}}%
\pgfpathcurveto{\pgfqpoint{2.053614in}{1.933070in}}{\pgfqpoint{2.056886in}{1.940970in}}{\pgfqpoint{2.056886in}{1.949206in}}%
\pgfpathcurveto{\pgfqpoint{2.056886in}{1.957442in}}{\pgfqpoint{2.053614in}{1.965342in}}{\pgfqpoint{2.047790in}{1.971166in}}%
\pgfpathcurveto{\pgfqpoint{2.041966in}{1.976990in}}{\pgfqpoint{2.034066in}{1.980262in}}{\pgfqpoint{2.025830in}{1.980262in}}%
\pgfpathcurveto{\pgfqpoint{2.017594in}{1.980262in}}{\pgfqpoint{2.009694in}{1.976990in}}{\pgfqpoint{2.003870in}{1.971166in}}%
\pgfpathcurveto{\pgfqpoint{1.998046in}{1.965342in}}{\pgfqpoint{1.994773in}{1.957442in}}{\pgfqpoint{1.994773in}{1.949206in}}%
\pgfpathcurveto{\pgfqpoint{1.994773in}{1.940970in}}{\pgfqpoint{1.998046in}{1.933070in}}{\pgfqpoint{2.003870in}{1.927246in}}%
\pgfpathcurveto{\pgfqpoint{2.009694in}{1.921422in}}{\pgfqpoint{2.017594in}{1.918149in}}{\pgfqpoint{2.025830in}{1.918149in}}%
\pgfpathclose%
\pgfusepath{stroke,fill}%
\end{pgfscope}%
\begin{pgfscope}%
\pgfpathrectangle{\pgfqpoint{0.100000in}{0.220728in}}{\pgfqpoint{3.696000in}{3.696000in}}%
\pgfusepath{clip}%
\pgfsetbuttcap%
\pgfsetroundjoin%
\definecolor{currentfill}{rgb}{0.121569,0.466667,0.705882}%
\pgfsetfillcolor{currentfill}%
\pgfsetfillopacity{0.448687}%
\pgfsetlinewidth{1.003750pt}%
\definecolor{currentstroke}{rgb}{0.121569,0.466667,0.705882}%
\pgfsetstrokecolor{currentstroke}%
\pgfsetstrokeopacity{0.448687}%
\pgfsetdash{}{0pt}%
\pgfpathmoveto{\pgfqpoint{2.026151in}{1.918170in}}%
\pgfpathcurveto{\pgfqpoint{2.034387in}{1.918170in}}{\pgfqpoint{2.042287in}{1.921442in}}{\pgfqpoint{2.048111in}{1.927266in}}%
\pgfpathcurveto{\pgfqpoint{2.053935in}{1.933090in}}{\pgfqpoint{2.057208in}{1.940990in}}{\pgfqpoint{2.057208in}{1.949226in}}%
\pgfpathcurveto{\pgfqpoint{2.057208in}{1.957462in}}{\pgfqpoint{2.053935in}{1.965362in}}{\pgfqpoint{2.048111in}{1.971186in}}%
\pgfpathcurveto{\pgfqpoint{2.042287in}{1.977010in}}{\pgfqpoint{2.034387in}{1.980283in}}{\pgfqpoint{2.026151in}{1.980283in}}%
\pgfpathcurveto{\pgfqpoint{2.017915in}{1.980283in}}{\pgfqpoint{2.010015in}{1.977010in}}{\pgfqpoint{2.004191in}{1.971186in}}%
\pgfpathcurveto{\pgfqpoint{1.998367in}{1.965362in}}{\pgfqpoint{1.995095in}{1.957462in}}{\pgfqpoint{1.995095in}{1.949226in}}%
\pgfpathcurveto{\pgfqpoint{1.995095in}{1.940990in}}{\pgfqpoint{1.998367in}{1.933090in}}{\pgfqpoint{2.004191in}{1.927266in}}%
\pgfpathcurveto{\pgfqpoint{2.010015in}{1.921442in}}{\pgfqpoint{2.017915in}{1.918170in}}{\pgfqpoint{2.026151in}{1.918170in}}%
\pgfpathclose%
\pgfusepath{stroke,fill}%
\end{pgfscope}%
\begin{pgfscope}%
\pgfpathrectangle{\pgfqpoint{0.100000in}{0.220728in}}{\pgfqpoint{3.696000in}{3.696000in}}%
\pgfusepath{clip}%
\pgfsetbuttcap%
\pgfsetroundjoin%
\definecolor{currentfill}{rgb}{0.121569,0.466667,0.705882}%
\pgfsetfillcolor{currentfill}%
\pgfsetfillopacity{0.449097}%
\pgfsetlinewidth{1.003750pt}%
\definecolor{currentstroke}{rgb}{0.121569,0.466667,0.705882}%
\pgfsetstrokecolor{currentstroke}%
\pgfsetstrokeopacity{0.449097}%
\pgfsetdash{}{0pt}%
\pgfpathmoveto{\pgfqpoint{1.477105in}{1.810772in}}%
\pgfpathcurveto{\pgfqpoint{1.485341in}{1.810772in}}{\pgfqpoint{1.493241in}{1.814045in}}{\pgfqpoint{1.499065in}{1.819869in}}%
\pgfpathcurveto{\pgfqpoint{1.504889in}{1.825693in}}{\pgfqpoint{1.508161in}{1.833593in}}{\pgfqpoint{1.508161in}{1.841829in}}%
\pgfpathcurveto{\pgfqpoint{1.508161in}{1.850065in}}{\pgfqpoint{1.504889in}{1.857965in}}{\pgfqpoint{1.499065in}{1.863789in}}%
\pgfpathcurveto{\pgfqpoint{1.493241in}{1.869613in}}{\pgfqpoint{1.485341in}{1.872885in}}{\pgfqpoint{1.477105in}{1.872885in}}%
\pgfpathcurveto{\pgfqpoint{1.468868in}{1.872885in}}{\pgfqpoint{1.460968in}{1.869613in}}{\pgfqpoint{1.455144in}{1.863789in}}%
\pgfpathcurveto{\pgfqpoint{1.449320in}{1.857965in}}{\pgfqpoint{1.446048in}{1.850065in}}{\pgfqpoint{1.446048in}{1.841829in}}%
\pgfpathcurveto{\pgfqpoint{1.446048in}{1.833593in}}{\pgfqpoint{1.449320in}{1.825693in}}{\pgfqpoint{1.455144in}{1.819869in}}%
\pgfpathcurveto{\pgfqpoint{1.460968in}{1.814045in}}{\pgfqpoint{1.468868in}{1.810772in}}{\pgfqpoint{1.477105in}{1.810772in}}%
\pgfpathclose%
\pgfusepath{stroke,fill}%
\end{pgfscope}%
\begin{pgfscope}%
\pgfpathrectangle{\pgfqpoint{0.100000in}{0.220728in}}{\pgfqpoint{3.696000in}{3.696000in}}%
\pgfusepath{clip}%
\pgfsetbuttcap%
\pgfsetroundjoin%
\definecolor{currentfill}{rgb}{0.121569,0.466667,0.705882}%
\pgfsetfillcolor{currentfill}%
\pgfsetfillopacity{0.449692}%
\pgfsetlinewidth{1.003750pt}%
\definecolor{currentstroke}{rgb}{0.121569,0.466667,0.705882}%
\pgfsetstrokecolor{currentstroke}%
\pgfsetstrokeopacity{0.449692}%
\pgfsetdash{}{0pt}%
\pgfpathmoveto{\pgfqpoint{2.026606in}{1.917612in}}%
\pgfpathcurveto{\pgfqpoint{2.034843in}{1.917612in}}{\pgfqpoint{2.042743in}{1.920884in}}{\pgfqpoint{2.048567in}{1.926708in}}%
\pgfpathcurveto{\pgfqpoint{2.054391in}{1.932532in}}{\pgfqpoint{2.057663in}{1.940432in}}{\pgfqpoint{2.057663in}{1.948669in}}%
\pgfpathcurveto{\pgfqpoint{2.057663in}{1.956905in}}{\pgfqpoint{2.054391in}{1.964805in}}{\pgfqpoint{2.048567in}{1.970629in}}%
\pgfpathcurveto{\pgfqpoint{2.042743in}{1.976453in}}{\pgfqpoint{2.034843in}{1.979725in}}{\pgfqpoint{2.026606in}{1.979725in}}%
\pgfpathcurveto{\pgfqpoint{2.018370in}{1.979725in}}{\pgfqpoint{2.010470in}{1.976453in}}{\pgfqpoint{2.004646in}{1.970629in}}%
\pgfpathcurveto{\pgfqpoint{1.998822in}{1.964805in}}{\pgfqpoint{1.995550in}{1.956905in}}{\pgfqpoint{1.995550in}{1.948669in}}%
\pgfpathcurveto{\pgfqpoint{1.995550in}{1.940432in}}{\pgfqpoint{1.998822in}{1.932532in}}{\pgfqpoint{2.004646in}{1.926708in}}%
\pgfpathcurveto{\pgfqpoint{2.010470in}{1.920884in}}{\pgfqpoint{2.018370in}{1.917612in}}{\pgfqpoint{2.026606in}{1.917612in}}%
\pgfpathclose%
\pgfusepath{stroke,fill}%
\end{pgfscope}%
\begin{pgfscope}%
\pgfpathrectangle{\pgfqpoint{0.100000in}{0.220728in}}{\pgfqpoint{3.696000in}{3.696000in}}%
\pgfusepath{clip}%
\pgfsetbuttcap%
\pgfsetroundjoin%
\definecolor{currentfill}{rgb}{0.121569,0.466667,0.705882}%
\pgfsetfillcolor{currentfill}%
\pgfsetfillopacity{0.449862}%
\pgfsetlinewidth{1.003750pt}%
\definecolor{currentstroke}{rgb}{0.121569,0.466667,0.705882}%
\pgfsetstrokecolor{currentstroke}%
\pgfsetstrokeopacity{0.449862}%
\pgfsetdash{}{0pt}%
\pgfpathmoveto{\pgfqpoint{1.473660in}{1.806199in}}%
\pgfpathcurveto{\pgfqpoint{1.481896in}{1.806199in}}{\pgfqpoint{1.489796in}{1.809472in}}{\pgfqpoint{1.495620in}{1.815296in}}%
\pgfpathcurveto{\pgfqpoint{1.501444in}{1.821119in}}{\pgfqpoint{1.504717in}{1.829020in}}{\pgfqpoint{1.504717in}{1.837256in}}%
\pgfpathcurveto{\pgfqpoint{1.504717in}{1.845492in}}{\pgfqpoint{1.501444in}{1.853392in}}{\pgfqpoint{1.495620in}{1.859216in}}%
\pgfpathcurveto{\pgfqpoint{1.489796in}{1.865040in}}{\pgfqpoint{1.481896in}{1.868312in}}{\pgfqpoint{1.473660in}{1.868312in}}%
\pgfpathcurveto{\pgfqpoint{1.465424in}{1.868312in}}{\pgfqpoint{1.457524in}{1.865040in}}{\pgfqpoint{1.451700in}{1.859216in}}%
\pgfpathcurveto{\pgfqpoint{1.445876in}{1.853392in}}{\pgfqpoint{1.442604in}{1.845492in}}{\pgfqpoint{1.442604in}{1.837256in}}%
\pgfpathcurveto{\pgfqpoint{1.442604in}{1.829020in}}{\pgfqpoint{1.445876in}{1.821119in}}{\pgfqpoint{1.451700in}{1.815296in}}%
\pgfpathcurveto{\pgfqpoint{1.457524in}{1.809472in}}{\pgfqpoint{1.465424in}{1.806199in}}{\pgfqpoint{1.473660in}{1.806199in}}%
\pgfpathclose%
\pgfusepath{stroke,fill}%
\end{pgfscope}%
\begin{pgfscope}%
\pgfpathrectangle{\pgfqpoint{0.100000in}{0.220728in}}{\pgfqpoint{3.696000in}{3.696000in}}%
\pgfusepath{clip}%
\pgfsetbuttcap%
\pgfsetroundjoin%
\definecolor{currentfill}{rgb}{0.121569,0.466667,0.705882}%
\pgfsetfillcolor{currentfill}%
\pgfsetfillopacity{0.450866}%
\pgfsetlinewidth{1.003750pt}%
\definecolor{currentstroke}{rgb}{0.121569,0.466667,0.705882}%
\pgfsetstrokecolor{currentstroke}%
\pgfsetstrokeopacity{0.450866}%
\pgfsetdash{}{0pt}%
\pgfpathmoveto{\pgfqpoint{2.027046in}{1.916703in}}%
\pgfpathcurveto{\pgfqpoint{2.035282in}{1.916703in}}{\pgfqpoint{2.043182in}{1.919975in}}{\pgfqpoint{2.049006in}{1.925799in}}%
\pgfpathcurveto{\pgfqpoint{2.054830in}{1.931623in}}{\pgfqpoint{2.058103in}{1.939523in}}{\pgfqpoint{2.058103in}{1.947760in}}%
\pgfpathcurveto{\pgfqpoint{2.058103in}{1.955996in}}{\pgfqpoint{2.054830in}{1.963896in}}{\pgfqpoint{2.049006in}{1.969720in}}%
\pgfpathcurveto{\pgfqpoint{2.043182in}{1.975544in}}{\pgfqpoint{2.035282in}{1.978816in}}{\pgfqpoint{2.027046in}{1.978816in}}%
\pgfpathcurveto{\pgfqpoint{2.018810in}{1.978816in}}{\pgfqpoint{2.010910in}{1.975544in}}{\pgfqpoint{2.005086in}{1.969720in}}%
\pgfpathcurveto{\pgfqpoint{1.999262in}{1.963896in}}{\pgfqpoint{1.995990in}{1.955996in}}{\pgfqpoint{1.995990in}{1.947760in}}%
\pgfpathcurveto{\pgfqpoint{1.995990in}{1.939523in}}{\pgfqpoint{1.999262in}{1.931623in}}{\pgfqpoint{2.005086in}{1.925799in}}%
\pgfpathcurveto{\pgfqpoint{2.010910in}{1.919975in}}{\pgfqpoint{2.018810in}{1.916703in}}{\pgfqpoint{2.027046in}{1.916703in}}%
\pgfpathclose%
\pgfusepath{stroke,fill}%
\end{pgfscope}%
\begin{pgfscope}%
\pgfpathrectangle{\pgfqpoint{0.100000in}{0.220728in}}{\pgfqpoint{3.696000in}{3.696000in}}%
\pgfusepath{clip}%
\pgfsetbuttcap%
\pgfsetroundjoin%
\definecolor{currentfill}{rgb}{0.121569,0.466667,0.705882}%
\pgfsetfillcolor{currentfill}%
\pgfsetfillopacity{0.452276}%
\pgfsetlinewidth{1.003750pt}%
\definecolor{currentstroke}{rgb}{0.121569,0.466667,0.705882}%
\pgfsetstrokecolor{currentstroke}%
\pgfsetstrokeopacity{0.452276}%
\pgfsetdash{}{0pt}%
\pgfpathmoveto{\pgfqpoint{2.027649in}{1.915178in}}%
\pgfpathcurveto{\pgfqpoint{2.035885in}{1.915178in}}{\pgfqpoint{2.043785in}{1.918450in}}{\pgfqpoint{2.049609in}{1.924274in}}%
\pgfpathcurveto{\pgfqpoint{2.055433in}{1.930098in}}{\pgfqpoint{2.058705in}{1.937998in}}{\pgfqpoint{2.058705in}{1.946234in}}%
\pgfpathcurveto{\pgfqpoint{2.058705in}{1.954470in}}{\pgfqpoint{2.055433in}{1.962370in}}{\pgfqpoint{2.049609in}{1.968194in}}%
\pgfpathcurveto{\pgfqpoint{2.043785in}{1.974018in}}{\pgfqpoint{2.035885in}{1.977291in}}{\pgfqpoint{2.027649in}{1.977291in}}%
\pgfpathcurveto{\pgfqpoint{2.019413in}{1.977291in}}{\pgfqpoint{2.011512in}{1.974018in}}{\pgfqpoint{2.005689in}{1.968194in}}%
\pgfpathcurveto{\pgfqpoint{1.999865in}{1.962370in}}{\pgfqpoint{1.996592in}{1.954470in}}{\pgfqpoint{1.996592in}{1.946234in}}%
\pgfpathcurveto{\pgfqpoint{1.996592in}{1.937998in}}{\pgfqpoint{1.999865in}{1.930098in}}{\pgfqpoint{2.005689in}{1.924274in}}%
\pgfpathcurveto{\pgfqpoint{2.011512in}{1.918450in}}{\pgfqpoint{2.019413in}{1.915178in}}{\pgfqpoint{2.027649in}{1.915178in}}%
\pgfpathclose%
\pgfusepath{stroke,fill}%
\end{pgfscope}%
\begin{pgfscope}%
\pgfpathrectangle{\pgfqpoint{0.100000in}{0.220728in}}{\pgfqpoint{3.696000in}{3.696000in}}%
\pgfusepath{clip}%
\pgfsetbuttcap%
\pgfsetroundjoin%
\definecolor{currentfill}{rgb}{0.121569,0.466667,0.705882}%
\pgfsetfillcolor{currentfill}%
\pgfsetfillopacity{0.453162}%
\pgfsetlinewidth{1.003750pt}%
\definecolor{currentstroke}{rgb}{0.121569,0.466667,0.705882}%
\pgfsetstrokecolor{currentstroke}%
\pgfsetstrokeopacity{0.453162}%
\pgfsetdash{}{0pt}%
\pgfpathmoveto{\pgfqpoint{1.466766in}{1.811409in}}%
\pgfpathcurveto{\pgfqpoint{1.475003in}{1.811409in}}{\pgfqpoint{1.482903in}{1.814681in}}{\pgfqpoint{1.488727in}{1.820505in}}%
\pgfpathcurveto{\pgfqpoint{1.494550in}{1.826329in}}{\pgfqpoint{1.497823in}{1.834229in}}{\pgfqpoint{1.497823in}{1.842466in}}%
\pgfpathcurveto{\pgfqpoint{1.497823in}{1.850702in}}{\pgfqpoint{1.494550in}{1.858602in}}{\pgfqpoint{1.488727in}{1.864426in}}%
\pgfpathcurveto{\pgfqpoint{1.482903in}{1.870250in}}{\pgfqpoint{1.475003in}{1.873522in}}{\pgfqpoint{1.466766in}{1.873522in}}%
\pgfpathcurveto{\pgfqpoint{1.458530in}{1.873522in}}{\pgfqpoint{1.450630in}{1.870250in}}{\pgfqpoint{1.444806in}{1.864426in}}%
\pgfpathcurveto{\pgfqpoint{1.438982in}{1.858602in}}{\pgfqpoint{1.435710in}{1.850702in}}{\pgfqpoint{1.435710in}{1.842466in}}%
\pgfpathcurveto{\pgfqpoint{1.435710in}{1.834229in}}{\pgfqpoint{1.438982in}{1.826329in}}{\pgfqpoint{1.444806in}{1.820505in}}%
\pgfpathcurveto{\pgfqpoint{1.450630in}{1.814681in}}{\pgfqpoint{1.458530in}{1.811409in}}{\pgfqpoint{1.466766in}{1.811409in}}%
\pgfpathclose%
\pgfusepath{stroke,fill}%
\end{pgfscope}%
\begin{pgfscope}%
\pgfpathrectangle{\pgfqpoint{0.100000in}{0.220728in}}{\pgfqpoint{3.696000in}{3.696000in}}%
\pgfusepath{clip}%
\pgfsetbuttcap%
\pgfsetroundjoin%
\definecolor{currentfill}{rgb}{0.121569,0.466667,0.705882}%
\pgfsetfillcolor{currentfill}%
\pgfsetfillopacity{0.454169}%
\pgfsetlinewidth{1.003750pt}%
\definecolor{currentstroke}{rgb}{0.121569,0.466667,0.705882}%
\pgfsetstrokecolor{currentstroke}%
\pgfsetstrokeopacity{0.454169}%
\pgfsetdash{}{0pt}%
\pgfpathmoveto{\pgfqpoint{2.029074in}{1.914808in}}%
\pgfpathcurveto{\pgfqpoint{2.037311in}{1.914808in}}{\pgfqpoint{2.045211in}{1.918080in}}{\pgfqpoint{2.051035in}{1.923904in}}%
\pgfpathcurveto{\pgfqpoint{2.056859in}{1.929728in}}{\pgfqpoint{2.060131in}{1.937628in}}{\pgfqpoint{2.060131in}{1.945864in}}%
\pgfpathcurveto{\pgfqpoint{2.060131in}{1.954101in}}{\pgfqpoint{2.056859in}{1.962001in}}{\pgfqpoint{2.051035in}{1.967825in}}%
\pgfpathcurveto{\pgfqpoint{2.045211in}{1.973649in}}{\pgfqpoint{2.037311in}{1.976921in}}{\pgfqpoint{2.029074in}{1.976921in}}%
\pgfpathcurveto{\pgfqpoint{2.020838in}{1.976921in}}{\pgfqpoint{2.012938in}{1.973649in}}{\pgfqpoint{2.007114in}{1.967825in}}%
\pgfpathcurveto{\pgfqpoint{2.001290in}{1.962001in}}{\pgfqpoint{1.998018in}{1.954101in}}{\pgfqpoint{1.998018in}{1.945864in}}%
\pgfpathcurveto{\pgfqpoint{1.998018in}{1.937628in}}{\pgfqpoint{2.001290in}{1.929728in}}{\pgfqpoint{2.007114in}{1.923904in}}%
\pgfpathcurveto{\pgfqpoint{2.012938in}{1.918080in}}{\pgfqpoint{2.020838in}{1.914808in}}{\pgfqpoint{2.029074in}{1.914808in}}%
\pgfpathclose%
\pgfusepath{stroke,fill}%
\end{pgfscope}%
\begin{pgfscope}%
\pgfpathrectangle{\pgfqpoint{0.100000in}{0.220728in}}{\pgfqpoint{3.696000in}{3.696000in}}%
\pgfusepath{clip}%
\pgfsetbuttcap%
\pgfsetroundjoin%
\definecolor{currentfill}{rgb}{0.121569,0.466667,0.705882}%
\pgfsetfillcolor{currentfill}%
\pgfsetfillopacity{0.455092}%
\pgfsetlinewidth{1.003750pt}%
\definecolor{currentstroke}{rgb}{0.121569,0.466667,0.705882}%
\pgfsetstrokecolor{currentstroke}%
\pgfsetstrokeopacity{0.455092}%
\pgfsetdash{}{0pt}%
\pgfpathmoveto{\pgfqpoint{1.462021in}{1.810946in}}%
\pgfpathcurveto{\pgfqpoint{1.470257in}{1.810946in}}{\pgfqpoint{1.478157in}{1.814219in}}{\pgfqpoint{1.483981in}{1.820042in}}%
\pgfpathcurveto{\pgfqpoint{1.489805in}{1.825866in}}{\pgfqpoint{1.493078in}{1.833766in}}{\pgfqpoint{1.493078in}{1.842003in}}%
\pgfpathcurveto{\pgfqpoint{1.493078in}{1.850239in}}{\pgfqpoint{1.489805in}{1.858139in}}{\pgfqpoint{1.483981in}{1.863963in}}%
\pgfpathcurveto{\pgfqpoint{1.478157in}{1.869787in}}{\pgfqpoint{1.470257in}{1.873059in}}{\pgfqpoint{1.462021in}{1.873059in}}%
\pgfpathcurveto{\pgfqpoint{1.453785in}{1.873059in}}{\pgfqpoint{1.445885in}{1.869787in}}{\pgfqpoint{1.440061in}{1.863963in}}%
\pgfpathcurveto{\pgfqpoint{1.434237in}{1.858139in}}{\pgfqpoint{1.430965in}{1.850239in}}{\pgfqpoint{1.430965in}{1.842003in}}%
\pgfpathcurveto{\pgfqpoint{1.430965in}{1.833766in}}{\pgfqpoint{1.434237in}{1.825866in}}{\pgfqpoint{1.440061in}{1.820042in}}%
\pgfpathcurveto{\pgfqpoint{1.445885in}{1.814219in}}{\pgfqpoint{1.453785in}{1.810946in}}{\pgfqpoint{1.462021in}{1.810946in}}%
\pgfpathclose%
\pgfusepath{stroke,fill}%
\end{pgfscope}%
\begin{pgfscope}%
\pgfpathrectangle{\pgfqpoint{0.100000in}{0.220728in}}{\pgfqpoint{3.696000in}{3.696000in}}%
\pgfusepath{clip}%
\pgfsetbuttcap%
\pgfsetroundjoin%
\definecolor{currentfill}{rgb}{0.121569,0.466667,0.705882}%
\pgfsetfillcolor{currentfill}%
\pgfsetfillopacity{0.456165}%
\pgfsetlinewidth{1.003750pt}%
\definecolor{currentstroke}{rgb}{0.121569,0.466667,0.705882}%
\pgfsetstrokecolor{currentstroke}%
\pgfsetstrokeopacity{0.456165}%
\pgfsetdash{}{0pt}%
\pgfpathmoveto{\pgfqpoint{2.029495in}{1.912341in}}%
\pgfpathcurveto{\pgfqpoint{2.037731in}{1.912341in}}{\pgfqpoint{2.045631in}{1.915613in}}{\pgfqpoint{2.051455in}{1.921437in}}%
\pgfpathcurveto{\pgfqpoint{2.057279in}{1.927261in}}{\pgfqpoint{2.060551in}{1.935161in}}{\pgfqpoint{2.060551in}{1.943397in}}%
\pgfpathcurveto{\pgfqpoint{2.060551in}{1.951634in}}{\pgfqpoint{2.057279in}{1.959534in}}{\pgfqpoint{2.051455in}{1.965358in}}%
\pgfpathcurveto{\pgfqpoint{2.045631in}{1.971182in}}{\pgfqpoint{2.037731in}{1.974454in}}{\pgfqpoint{2.029495in}{1.974454in}}%
\pgfpathcurveto{\pgfqpoint{2.021258in}{1.974454in}}{\pgfqpoint{2.013358in}{1.971182in}}{\pgfqpoint{2.007534in}{1.965358in}}%
\pgfpathcurveto{\pgfqpoint{2.001710in}{1.959534in}}{\pgfqpoint{1.998438in}{1.951634in}}{\pgfqpoint{1.998438in}{1.943397in}}%
\pgfpathcurveto{\pgfqpoint{1.998438in}{1.935161in}}{\pgfqpoint{2.001710in}{1.927261in}}{\pgfqpoint{2.007534in}{1.921437in}}%
\pgfpathcurveto{\pgfqpoint{2.013358in}{1.915613in}}{\pgfqpoint{2.021258in}{1.912341in}}{\pgfqpoint{2.029495in}{1.912341in}}%
\pgfpathclose%
\pgfusepath{stroke,fill}%
\end{pgfscope}%
\begin{pgfscope}%
\pgfpathrectangle{\pgfqpoint{0.100000in}{0.220728in}}{\pgfqpoint{3.696000in}{3.696000in}}%
\pgfusepath{clip}%
\pgfsetbuttcap%
\pgfsetroundjoin%
\definecolor{currentfill}{rgb}{0.121569,0.466667,0.705882}%
\pgfsetfillcolor{currentfill}%
\pgfsetfillopacity{0.456534}%
\pgfsetlinewidth{1.003750pt}%
\definecolor{currentstroke}{rgb}{0.121569,0.466667,0.705882}%
\pgfsetstrokecolor{currentstroke}%
\pgfsetstrokeopacity{0.456534}%
\pgfsetdash{}{0pt}%
\pgfpathmoveto{\pgfqpoint{1.458261in}{1.807372in}}%
\pgfpathcurveto{\pgfqpoint{1.466497in}{1.807372in}}{\pgfqpoint{1.474397in}{1.810644in}}{\pgfqpoint{1.480221in}{1.816468in}}%
\pgfpathcurveto{\pgfqpoint{1.486045in}{1.822292in}}{\pgfqpoint{1.489318in}{1.830192in}}{\pgfqpoint{1.489318in}{1.838428in}}%
\pgfpathcurveto{\pgfqpoint{1.489318in}{1.846664in}}{\pgfqpoint{1.486045in}{1.854564in}}{\pgfqpoint{1.480221in}{1.860388in}}%
\pgfpathcurveto{\pgfqpoint{1.474397in}{1.866212in}}{\pgfqpoint{1.466497in}{1.869485in}}{\pgfqpoint{1.458261in}{1.869485in}}%
\pgfpathcurveto{\pgfqpoint{1.450025in}{1.869485in}}{\pgfqpoint{1.442125in}{1.866212in}}{\pgfqpoint{1.436301in}{1.860388in}}%
\pgfpathcurveto{\pgfqpoint{1.430477in}{1.854564in}}{\pgfqpoint{1.427205in}{1.846664in}}{\pgfqpoint{1.427205in}{1.838428in}}%
\pgfpathcurveto{\pgfqpoint{1.427205in}{1.830192in}}{\pgfqpoint{1.430477in}{1.822292in}}{\pgfqpoint{1.436301in}{1.816468in}}%
\pgfpathcurveto{\pgfqpoint{1.442125in}{1.810644in}}{\pgfqpoint{1.450025in}{1.807372in}}{\pgfqpoint{1.458261in}{1.807372in}}%
\pgfpathclose%
\pgfusepath{stroke,fill}%
\end{pgfscope}%
\begin{pgfscope}%
\pgfpathrectangle{\pgfqpoint{0.100000in}{0.220728in}}{\pgfqpoint{3.696000in}{3.696000in}}%
\pgfusepath{clip}%
\pgfsetbuttcap%
\pgfsetroundjoin%
\definecolor{currentfill}{rgb}{0.121569,0.466667,0.705882}%
\pgfsetfillcolor{currentfill}%
\pgfsetfillopacity{0.459270}%
\pgfsetlinewidth{1.003750pt}%
\definecolor{currentstroke}{rgb}{0.121569,0.466667,0.705882}%
\pgfsetstrokecolor{currentstroke}%
\pgfsetstrokeopacity{0.459270}%
\pgfsetdash{}{0pt}%
\pgfpathmoveto{\pgfqpoint{2.030248in}{1.913690in}}%
\pgfpathcurveto{\pgfqpoint{2.038485in}{1.913690in}}{\pgfqpoint{2.046385in}{1.916962in}}{\pgfqpoint{2.052209in}{1.922786in}}%
\pgfpathcurveto{\pgfqpoint{2.058033in}{1.928610in}}{\pgfqpoint{2.061305in}{1.936510in}}{\pgfqpoint{2.061305in}{1.944746in}}%
\pgfpathcurveto{\pgfqpoint{2.061305in}{1.952982in}}{\pgfqpoint{2.058033in}{1.960882in}}{\pgfqpoint{2.052209in}{1.966706in}}%
\pgfpathcurveto{\pgfqpoint{2.046385in}{1.972530in}}{\pgfqpoint{2.038485in}{1.975803in}}{\pgfqpoint{2.030248in}{1.975803in}}%
\pgfpathcurveto{\pgfqpoint{2.022012in}{1.975803in}}{\pgfqpoint{2.014112in}{1.972530in}}{\pgfqpoint{2.008288in}{1.966706in}}%
\pgfpathcurveto{\pgfqpoint{2.002464in}{1.960882in}}{\pgfqpoint{1.999192in}{1.952982in}}{\pgfqpoint{1.999192in}{1.944746in}}%
\pgfpathcurveto{\pgfqpoint{1.999192in}{1.936510in}}{\pgfqpoint{2.002464in}{1.928610in}}{\pgfqpoint{2.008288in}{1.922786in}}%
\pgfpathcurveto{\pgfqpoint{2.014112in}{1.916962in}}{\pgfqpoint{2.022012in}{1.913690in}}{\pgfqpoint{2.030248in}{1.913690in}}%
\pgfpathclose%
\pgfusepath{stroke,fill}%
\end{pgfscope}%
\begin{pgfscope}%
\pgfpathrectangle{\pgfqpoint{0.100000in}{0.220728in}}{\pgfqpoint{3.696000in}{3.696000in}}%
\pgfusepath{clip}%
\pgfsetbuttcap%
\pgfsetroundjoin%
\definecolor{currentfill}{rgb}{0.121569,0.466667,0.705882}%
\pgfsetfillcolor{currentfill}%
\pgfsetfillopacity{0.460673}%
\pgfsetlinewidth{1.003750pt}%
\definecolor{currentstroke}{rgb}{0.121569,0.466667,0.705882}%
\pgfsetstrokecolor{currentstroke}%
\pgfsetstrokeopacity{0.460673}%
\pgfsetdash{}{0pt}%
\pgfpathmoveto{\pgfqpoint{2.031504in}{1.912783in}}%
\pgfpathcurveto{\pgfqpoint{2.039740in}{1.912783in}}{\pgfqpoint{2.047640in}{1.916056in}}{\pgfqpoint{2.053464in}{1.921880in}}%
\pgfpathcurveto{\pgfqpoint{2.059288in}{1.927703in}}{\pgfqpoint{2.062561in}{1.935604in}}{\pgfqpoint{2.062561in}{1.943840in}}%
\pgfpathcurveto{\pgfqpoint{2.062561in}{1.952076in}}{\pgfqpoint{2.059288in}{1.959976in}}{\pgfqpoint{2.053464in}{1.965800in}}%
\pgfpathcurveto{\pgfqpoint{2.047640in}{1.971624in}}{\pgfqpoint{2.039740in}{1.974896in}}{\pgfqpoint{2.031504in}{1.974896in}}%
\pgfpathcurveto{\pgfqpoint{2.023268in}{1.974896in}}{\pgfqpoint{2.015368in}{1.971624in}}{\pgfqpoint{2.009544in}{1.965800in}}%
\pgfpathcurveto{\pgfqpoint{2.003720in}{1.959976in}}{\pgfqpoint{2.000448in}{1.952076in}}{\pgfqpoint{2.000448in}{1.943840in}}%
\pgfpathcurveto{\pgfqpoint{2.000448in}{1.935604in}}{\pgfqpoint{2.003720in}{1.927703in}}{\pgfqpoint{2.009544in}{1.921880in}}%
\pgfpathcurveto{\pgfqpoint{2.015368in}{1.916056in}}{\pgfqpoint{2.023268in}{1.912783in}}{\pgfqpoint{2.031504in}{1.912783in}}%
\pgfpathclose%
\pgfusepath{stroke,fill}%
\end{pgfscope}%
\begin{pgfscope}%
\pgfpathrectangle{\pgfqpoint{0.100000in}{0.220728in}}{\pgfqpoint{3.696000in}{3.696000in}}%
\pgfusepath{clip}%
\pgfsetbuttcap%
\pgfsetroundjoin%
\definecolor{currentfill}{rgb}{0.121569,0.466667,0.705882}%
\pgfsetfillcolor{currentfill}%
\pgfsetfillopacity{0.462663}%
\pgfsetlinewidth{1.003750pt}%
\definecolor{currentstroke}{rgb}{0.121569,0.466667,0.705882}%
\pgfsetstrokecolor{currentstroke}%
\pgfsetstrokeopacity{0.462663}%
\pgfsetdash{}{0pt}%
\pgfpathmoveto{\pgfqpoint{1.451956in}{1.823601in}}%
\pgfpathcurveto{\pgfqpoint{1.460192in}{1.823601in}}{\pgfqpoint{1.468092in}{1.826873in}}{\pgfqpoint{1.473916in}{1.832697in}}%
\pgfpathcurveto{\pgfqpoint{1.479740in}{1.838521in}}{\pgfqpoint{1.483012in}{1.846421in}}{\pgfqpoint{1.483012in}{1.854657in}}%
\pgfpathcurveto{\pgfqpoint{1.483012in}{1.862894in}}{\pgfqpoint{1.479740in}{1.870794in}}{\pgfqpoint{1.473916in}{1.876618in}}%
\pgfpathcurveto{\pgfqpoint{1.468092in}{1.882442in}}{\pgfqpoint{1.460192in}{1.885714in}}{\pgfqpoint{1.451956in}{1.885714in}}%
\pgfpathcurveto{\pgfqpoint{1.443720in}{1.885714in}}{\pgfqpoint{1.435820in}{1.882442in}}{\pgfqpoint{1.429996in}{1.876618in}}%
\pgfpathcurveto{\pgfqpoint{1.424172in}{1.870794in}}{\pgfqpoint{1.420899in}{1.862894in}}{\pgfqpoint{1.420899in}{1.854657in}}%
\pgfpathcurveto{\pgfqpoint{1.420899in}{1.846421in}}{\pgfqpoint{1.424172in}{1.838521in}}{\pgfqpoint{1.429996in}{1.832697in}}%
\pgfpathcurveto{\pgfqpoint{1.435820in}{1.826873in}}{\pgfqpoint{1.443720in}{1.823601in}}{\pgfqpoint{1.451956in}{1.823601in}}%
\pgfpathclose%
\pgfusepath{stroke,fill}%
\end{pgfscope}%
\begin{pgfscope}%
\pgfpathrectangle{\pgfqpoint{0.100000in}{0.220728in}}{\pgfqpoint{3.696000in}{3.696000in}}%
\pgfusepath{clip}%
\pgfsetbuttcap%
\pgfsetroundjoin%
\definecolor{currentfill}{rgb}{0.121569,0.466667,0.705882}%
\pgfsetfillcolor{currentfill}%
\pgfsetfillopacity{0.462735}%
\pgfsetlinewidth{1.003750pt}%
\definecolor{currentstroke}{rgb}{0.121569,0.466667,0.705882}%
\pgfsetstrokecolor{currentstroke}%
\pgfsetstrokeopacity{0.462735}%
\pgfsetdash{}{0pt}%
\pgfpathmoveto{\pgfqpoint{2.032441in}{1.912209in}}%
\pgfpathcurveto{\pgfqpoint{2.040677in}{1.912209in}}{\pgfqpoint{2.048577in}{1.915481in}}{\pgfqpoint{2.054401in}{1.921305in}}%
\pgfpathcurveto{\pgfqpoint{2.060225in}{1.927129in}}{\pgfqpoint{2.063497in}{1.935029in}}{\pgfqpoint{2.063497in}{1.943265in}}%
\pgfpathcurveto{\pgfqpoint{2.063497in}{1.951501in}}{\pgfqpoint{2.060225in}{1.959402in}}{\pgfqpoint{2.054401in}{1.965225in}}%
\pgfpathcurveto{\pgfqpoint{2.048577in}{1.971049in}}{\pgfqpoint{2.040677in}{1.974322in}}{\pgfqpoint{2.032441in}{1.974322in}}%
\pgfpathcurveto{\pgfqpoint{2.024204in}{1.974322in}}{\pgfqpoint{2.016304in}{1.971049in}}{\pgfqpoint{2.010480in}{1.965225in}}%
\pgfpathcurveto{\pgfqpoint{2.004656in}{1.959402in}}{\pgfqpoint{2.001384in}{1.951501in}}{\pgfqpoint{2.001384in}{1.943265in}}%
\pgfpathcurveto{\pgfqpoint{2.001384in}{1.935029in}}{\pgfqpoint{2.004656in}{1.927129in}}{\pgfqpoint{2.010480in}{1.921305in}}%
\pgfpathcurveto{\pgfqpoint{2.016304in}{1.915481in}}{\pgfqpoint{2.024204in}{1.912209in}}{\pgfqpoint{2.032441in}{1.912209in}}%
\pgfpathclose%
\pgfusepath{stroke,fill}%
\end{pgfscope}%
\begin{pgfscope}%
\pgfpathrectangle{\pgfqpoint{0.100000in}{0.220728in}}{\pgfqpoint{3.696000in}{3.696000in}}%
\pgfusepath{clip}%
\pgfsetbuttcap%
\pgfsetroundjoin%
\definecolor{currentfill}{rgb}{0.121569,0.466667,0.705882}%
\pgfsetfillcolor{currentfill}%
\pgfsetfillopacity{0.463765}%
\pgfsetlinewidth{1.003750pt}%
\definecolor{currentstroke}{rgb}{0.121569,0.466667,0.705882}%
\pgfsetstrokecolor{currentstroke}%
\pgfsetstrokeopacity{0.463765}%
\pgfsetdash{}{0pt}%
\pgfpathmoveto{\pgfqpoint{2.032780in}{1.911120in}}%
\pgfpathcurveto{\pgfqpoint{2.041016in}{1.911120in}}{\pgfqpoint{2.048916in}{1.914392in}}{\pgfqpoint{2.054740in}{1.920216in}}%
\pgfpathcurveto{\pgfqpoint{2.060564in}{1.926040in}}{\pgfqpoint{2.063836in}{1.933940in}}{\pgfqpoint{2.063836in}{1.942176in}}%
\pgfpathcurveto{\pgfqpoint{2.063836in}{1.950413in}}{\pgfqpoint{2.060564in}{1.958313in}}{\pgfqpoint{2.054740in}{1.964137in}}%
\pgfpathcurveto{\pgfqpoint{2.048916in}{1.969961in}}{\pgfqpoint{2.041016in}{1.973233in}}{\pgfqpoint{2.032780in}{1.973233in}}%
\pgfpathcurveto{\pgfqpoint{2.024544in}{1.973233in}}{\pgfqpoint{2.016644in}{1.969961in}}{\pgfqpoint{2.010820in}{1.964137in}}%
\pgfpathcurveto{\pgfqpoint{2.004996in}{1.958313in}}{\pgfqpoint{2.001723in}{1.950413in}}{\pgfqpoint{2.001723in}{1.942176in}}%
\pgfpathcurveto{\pgfqpoint{2.001723in}{1.933940in}}{\pgfqpoint{2.004996in}{1.926040in}}{\pgfqpoint{2.010820in}{1.920216in}}%
\pgfpathcurveto{\pgfqpoint{2.016644in}{1.914392in}}{\pgfqpoint{2.024544in}{1.911120in}}{\pgfqpoint{2.032780in}{1.911120in}}%
\pgfpathclose%
\pgfusepath{stroke,fill}%
\end{pgfscope}%
\begin{pgfscope}%
\pgfpathrectangle{\pgfqpoint{0.100000in}{0.220728in}}{\pgfqpoint{3.696000in}{3.696000in}}%
\pgfusepath{clip}%
\pgfsetbuttcap%
\pgfsetroundjoin%
\definecolor{currentfill}{rgb}{0.121569,0.466667,0.705882}%
\pgfsetfillcolor{currentfill}%
\pgfsetfillopacity{0.465001}%
\pgfsetlinewidth{1.003750pt}%
\definecolor{currentstroke}{rgb}{0.121569,0.466667,0.705882}%
\pgfsetstrokecolor{currentstroke}%
\pgfsetstrokeopacity{0.465001}%
\pgfsetdash{}{0pt}%
\pgfpathmoveto{\pgfqpoint{2.033392in}{1.910472in}}%
\pgfpathcurveto{\pgfqpoint{2.041628in}{1.910472in}}{\pgfqpoint{2.049528in}{1.913745in}}{\pgfqpoint{2.055352in}{1.919569in}}%
\pgfpathcurveto{\pgfqpoint{2.061176in}{1.925393in}}{\pgfqpoint{2.064448in}{1.933293in}}{\pgfqpoint{2.064448in}{1.941529in}}%
\pgfpathcurveto{\pgfqpoint{2.064448in}{1.949765in}}{\pgfqpoint{2.061176in}{1.957665in}}{\pgfqpoint{2.055352in}{1.963489in}}%
\pgfpathcurveto{\pgfqpoint{2.049528in}{1.969313in}}{\pgfqpoint{2.041628in}{1.972585in}}{\pgfqpoint{2.033392in}{1.972585in}}%
\pgfpathcurveto{\pgfqpoint{2.025155in}{1.972585in}}{\pgfqpoint{2.017255in}{1.969313in}}{\pgfqpoint{2.011432in}{1.963489in}}%
\pgfpathcurveto{\pgfqpoint{2.005608in}{1.957665in}}{\pgfqpoint{2.002335in}{1.949765in}}{\pgfqpoint{2.002335in}{1.941529in}}%
\pgfpathcurveto{\pgfqpoint{2.002335in}{1.933293in}}{\pgfqpoint{2.005608in}{1.925393in}}{\pgfqpoint{2.011432in}{1.919569in}}%
\pgfpathcurveto{\pgfqpoint{2.017255in}{1.913745in}}{\pgfqpoint{2.025155in}{1.910472in}}{\pgfqpoint{2.033392in}{1.910472in}}%
\pgfpathclose%
\pgfusepath{stroke,fill}%
\end{pgfscope}%
\begin{pgfscope}%
\pgfpathrectangle{\pgfqpoint{0.100000in}{0.220728in}}{\pgfqpoint{3.696000in}{3.696000in}}%
\pgfusepath{clip}%
\pgfsetbuttcap%
\pgfsetroundjoin%
\definecolor{currentfill}{rgb}{0.121569,0.466667,0.705882}%
\pgfsetfillcolor{currentfill}%
\pgfsetfillopacity{0.465532}%
\pgfsetlinewidth{1.003750pt}%
\definecolor{currentstroke}{rgb}{0.121569,0.466667,0.705882}%
\pgfsetstrokecolor{currentstroke}%
\pgfsetstrokeopacity{0.465532}%
\pgfsetdash{}{0pt}%
\pgfpathmoveto{\pgfqpoint{1.445456in}{1.821200in}}%
\pgfpathcurveto{\pgfqpoint{1.453693in}{1.821200in}}{\pgfqpoint{1.461593in}{1.824472in}}{\pgfqpoint{1.467417in}{1.830296in}}%
\pgfpathcurveto{\pgfqpoint{1.473241in}{1.836120in}}{\pgfqpoint{1.476513in}{1.844020in}}{\pgfqpoint{1.476513in}{1.852257in}}%
\pgfpathcurveto{\pgfqpoint{1.476513in}{1.860493in}}{\pgfqpoint{1.473241in}{1.868393in}}{\pgfqpoint{1.467417in}{1.874217in}}%
\pgfpathcurveto{\pgfqpoint{1.461593in}{1.880041in}}{\pgfqpoint{1.453693in}{1.883313in}}{\pgfqpoint{1.445456in}{1.883313in}}%
\pgfpathcurveto{\pgfqpoint{1.437220in}{1.883313in}}{\pgfqpoint{1.429320in}{1.880041in}}{\pgfqpoint{1.423496in}{1.874217in}}%
\pgfpathcurveto{\pgfqpoint{1.417672in}{1.868393in}}{\pgfqpoint{1.414400in}{1.860493in}}{\pgfqpoint{1.414400in}{1.852257in}}%
\pgfpathcurveto{\pgfqpoint{1.414400in}{1.844020in}}{\pgfqpoint{1.417672in}{1.836120in}}{\pgfqpoint{1.423496in}{1.830296in}}%
\pgfpathcurveto{\pgfqpoint{1.429320in}{1.824472in}}{\pgfqpoint{1.437220in}{1.821200in}}{\pgfqpoint{1.445456in}{1.821200in}}%
\pgfpathclose%
\pgfusepath{stroke,fill}%
\end{pgfscope}%
\begin{pgfscope}%
\pgfpathrectangle{\pgfqpoint{0.100000in}{0.220728in}}{\pgfqpoint{3.696000in}{3.696000in}}%
\pgfusepath{clip}%
\pgfsetbuttcap%
\pgfsetroundjoin%
\definecolor{currentfill}{rgb}{0.121569,0.466667,0.705882}%
\pgfsetfillcolor{currentfill}%
\pgfsetfillopacity{0.465754}%
\pgfsetlinewidth{1.003750pt}%
\definecolor{currentstroke}{rgb}{0.121569,0.466667,0.705882}%
\pgfsetstrokecolor{currentstroke}%
\pgfsetstrokeopacity{0.465754}%
\pgfsetdash{}{0pt}%
\pgfpathmoveto{\pgfqpoint{2.033933in}{1.910724in}}%
\pgfpathcurveto{\pgfqpoint{2.042170in}{1.910724in}}{\pgfqpoint{2.050070in}{1.913996in}}{\pgfqpoint{2.055894in}{1.919820in}}%
\pgfpathcurveto{\pgfqpoint{2.061717in}{1.925644in}}{\pgfqpoint{2.064990in}{1.933544in}}{\pgfqpoint{2.064990in}{1.941781in}}%
\pgfpathcurveto{\pgfqpoint{2.064990in}{1.950017in}}{\pgfqpoint{2.061717in}{1.957917in}}{\pgfqpoint{2.055894in}{1.963741in}}%
\pgfpathcurveto{\pgfqpoint{2.050070in}{1.969565in}}{\pgfqpoint{2.042170in}{1.972837in}}{\pgfqpoint{2.033933in}{1.972837in}}%
\pgfpathcurveto{\pgfqpoint{2.025697in}{1.972837in}}{\pgfqpoint{2.017797in}{1.969565in}}{\pgfqpoint{2.011973in}{1.963741in}}%
\pgfpathcurveto{\pgfqpoint{2.006149in}{1.957917in}}{\pgfqpoint{2.002877in}{1.950017in}}{\pgfqpoint{2.002877in}{1.941781in}}%
\pgfpathcurveto{\pgfqpoint{2.002877in}{1.933544in}}{\pgfqpoint{2.006149in}{1.925644in}}{\pgfqpoint{2.011973in}{1.919820in}}%
\pgfpathcurveto{\pgfqpoint{2.017797in}{1.913996in}}{\pgfqpoint{2.025697in}{1.910724in}}{\pgfqpoint{2.033933in}{1.910724in}}%
\pgfpathclose%
\pgfusepath{stroke,fill}%
\end{pgfscope}%
\begin{pgfscope}%
\pgfpathrectangle{\pgfqpoint{0.100000in}{0.220728in}}{\pgfqpoint{3.696000in}{3.696000in}}%
\pgfusepath{clip}%
\pgfsetbuttcap%
\pgfsetroundjoin%
\definecolor{currentfill}{rgb}{0.121569,0.466667,0.705882}%
\pgfsetfillcolor{currentfill}%
\pgfsetfillopacity{0.466630}%
\pgfsetlinewidth{1.003750pt}%
\definecolor{currentstroke}{rgb}{0.121569,0.466667,0.705882}%
\pgfsetstrokecolor{currentstroke}%
\pgfsetstrokeopacity{0.466630}%
\pgfsetdash{}{0pt}%
\pgfpathmoveto{\pgfqpoint{2.033824in}{1.910098in}}%
\pgfpathcurveto{\pgfqpoint{2.042061in}{1.910098in}}{\pgfqpoint{2.049961in}{1.913370in}}{\pgfqpoint{2.055785in}{1.919194in}}%
\pgfpathcurveto{\pgfqpoint{2.061609in}{1.925018in}}{\pgfqpoint{2.064881in}{1.932918in}}{\pgfqpoint{2.064881in}{1.941155in}}%
\pgfpathcurveto{\pgfqpoint{2.064881in}{1.949391in}}{\pgfqpoint{2.061609in}{1.957291in}}{\pgfqpoint{2.055785in}{1.963115in}}%
\pgfpathcurveto{\pgfqpoint{2.049961in}{1.968939in}}{\pgfqpoint{2.042061in}{1.972211in}}{\pgfqpoint{2.033824in}{1.972211in}}%
\pgfpathcurveto{\pgfqpoint{2.025588in}{1.972211in}}{\pgfqpoint{2.017688in}{1.968939in}}{\pgfqpoint{2.011864in}{1.963115in}}%
\pgfpathcurveto{\pgfqpoint{2.006040in}{1.957291in}}{\pgfqpoint{2.002768in}{1.949391in}}{\pgfqpoint{2.002768in}{1.941155in}}%
\pgfpathcurveto{\pgfqpoint{2.002768in}{1.932918in}}{\pgfqpoint{2.006040in}{1.925018in}}{\pgfqpoint{2.011864in}{1.919194in}}%
\pgfpathcurveto{\pgfqpoint{2.017688in}{1.913370in}}{\pgfqpoint{2.025588in}{1.910098in}}{\pgfqpoint{2.033824in}{1.910098in}}%
\pgfpathclose%
\pgfusepath{stroke,fill}%
\end{pgfscope}%
\begin{pgfscope}%
\pgfpathrectangle{\pgfqpoint{0.100000in}{0.220728in}}{\pgfqpoint{3.696000in}{3.696000in}}%
\pgfusepath{clip}%
\pgfsetbuttcap%
\pgfsetroundjoin%
\definecolor{currentfill}{rgb}{0.121569,0.466667,0.705882}%
\pgfsetfillcolor{currentfill}%
\pgfsetfillopacity{0.467041}%
\pgfsetlinewidth{1.003750pt}%
\definecolor{currentstroke}{rgb}{0.121569,0.466667,0.705882}%
\pgfsetstrokecolor{currentstroke}%
\pgfsetstrokeopacity{0.467041}%
\pgfsetdash{}{0pt}%
\pgfpathmoveto{\pgfqpoint{1.438737in}{1.816114in}}%
\pgfpathcurveto{\pgfqpoint{1.446973in}{1.816114in}}{\pgfqpoint{1.454873in}{1.819386in}}{\pgfqpoint{1.460697in}{1.825210in}}%
\pgfpathcurveto{\pgfqpoint{1.466521in}{1.831034in}}{\pgfqpoint{1.469794in}{1.838934in}}{\pgfqpoint{1.469794in}{1.847170in}}%
\pgfpathcurveto{\pgfqpoint{1.469794in}{1.855407in}}{\pgfqpoint{1.466521in}{1.863307in}}{\pgfqpoint{1.460697in}{1.869131in}}%
\pgfpathcurveto{\pgfqpoint{1.454873in}{1.874955in}}{\pgfqpoint{1.446973in}{1.878227in}}{\pgfqpoint{1.438737in}{1.878227in}}%
\pgfpathcurveto{\pgfqpoint{1.430501in}{1.878227in}}{\pgfqpoint{1.422601in}{1.874955in}}{\pgfqpoint{1.416777in}{1.869131in}}%
\pgfpathcurveto{\pgfqpoint{1.410953in}{1.863307in}}{\pgfqpoint{1.407681in}{1.855407in}}{\pgfqpoint{1.407681in}{1.847170in}}%
\pgfpathcurveto{\pgfqpoint{1.407681in}{1.838934in}}{\pgfqpoint{1.410953in}{1.831034in}}{\pgfqpoint{1.416777in}{1.825210in}}%
\pgfpathcurveto{\pgfqpoint{1.422601in}{1.819386in}}{\pgfqpoint{1.430501in}{1.816114in}}{\pgfqpoint{1.438737in}{1.816114in}}%
\pgfpathclose%
\pgfusepath{stroke,fill}%
\end{pgfscope}%
\begin{pgfscope}%
\pgfpathrectangle{\pgfqpoint{0.100000in}{0.220728in}}{\pgfqpoint{3.696000in}{3.696000in}}%
\pgfusepath{clip}%
\pgfsetbuttcap%
\pgfsetroundjoin%
\definecolor{currentfill}{rgb}{0.121569,0.466667,0.705882}%
\pgfsetfillcolor{currentfill}%
\pgfsetfillopacity{0.467913}%
\pgfsetlinewidth{1.003750pt}%
\definecolor{currentstroke}{rgb}{0.121569,0.466667,0.705882}%
\pgfsetstrokecolor{currentstroke}%
\pgfsetstrokeopacity{0.467913}%
\pgfsetdash{}{0pt}%
\pgfpathmoveto{\pgfqpoint{2.034928in}{1.907655in}}%
\pgfpathcurveto{\pgfqpoint{2.043165in}{1.907655in}}{\pgfqpoint{2.051065in}{1.910928in}}{\pgfqpoint{2.056889in}{1.916752in}}%
\pgfpathcurveto{\pgfqpoint{2.062713in}{1.922576in}}{\pgfqpoint{2.065985in}{1.930476in}}{\pgfqpoint{2.065985in}{1.938712in}}%
\pgfpathcurveto{\pgfqpoint{2.065985in}{1.946948in}}{\pgfqpoint{2.062713in}{1.954848in}}{\pgfqpoint{2.056889in}{1.960672in}}%
\pgfpathcurveto{\pgfqpoint{2.051065in}{1.966496in}}{\pgfqpoint{2.043165in}{1.969768in}}{\pgfqpoint{2.034928in}{1.969768in}}%
\pgfpathcurveto{\pgfqpoint{2.026692in}{1.969768in}}{\pgfqpoint{2.018792in}{1.966496in}}{\pgfqpoint{2.012968in}{1.960672in}}%
\pgfpathcurveto{\pgfqpoint{2.007144in}{1.954848in}}{\pgfqpoint{2.003872in}{1.946948in}}{\pgfqpoint{2.003872in}{1.938712in}}%
\pgfpathcurveto{\pgfqpoint{2.003872in}{1.930476in}}{\pgfqpoint{2.007144in}{1.922576in}}{\pgfqpoint{2.012968in}{1.916752in}}%
\pgfpathcurveto{\pgfqpoint{2.018792in}{1.910928in}}{\pgfqpoint{2.026692in}{1.907655in}}{\pgfqpoint{2.034928in}{1.907655in}}%
\pgfpathclose%
\pgfusepath{stroke,fill}%
\end{pgfscope}%
\begin{pgfscope}%
\pgfpathrectangle{\pgfqpoint{0.100000in}{0.220728in}}{\pgfqpoint{3.696000in}{3.696000in}}%
\pgfusepath{clip}%
\pgfsetbuttcap%
\pgfsetroundjoin%
\definecolor{currentfill}{rgb}{0.121569,0.466667,0.705882}%
\pgfsetfillcolor{currentfill}%
\pgfsetfillopacity{0.468740}%
\pgfsetlinewidth{1.003750pt}%
\definecolor{currentstroke}{rgb}{0.121569,0.466667,0.705882}%
\pgfsetstrokecolor{currentstroke}%
\pgfsetstrokeopacity{0.468740}%
\pgfsetdash{}{0pt}%
\pgfpathmoveto{\pgfqpoint{1.432886in}{1.812287in}}%
\pgfpathcurveto{\pgfqpoint{1.441122in}{1.812287in}}{\pgfqpoint{1.449022in}{1.815560in}}{\pgfqpoint{1.454846in}{1.821384in}}%
\pgfpathcurveto{\pgfqpoint{1.460670in}{1.827207in}}{\pgfqpoint{1.463942in}{1.835107in}}{\pgfqpoint{1.463942in}{1.843344in}}%
\pgfpathcurveto{\pgfqpoint{1.463942in}{1.851580in}}{\pgfqpoint{1.460670in}{1.859480in}}{\pgfqpoint{1.454846in}{1.865304in}}%
\pgfpathcurveto{\pgfqpoint{1.449022in}{1.871128in}}{\pgfqpoint{1.441122in}{1.874400in}}{\pgfqpoint{1.432886in}{1.874400in}}%
\pgfpathcurveto{\pgfqpoint{1.424649in}{1.874400in}}{\pgfqpoint{1.416749in}{1.871128in}}{\pgfqpoint{1.410925in}{1.865304in}}%
\pgfpathcurveto{\pgfqpoint{1.405101in}{1.859480in}}{\pgfqpoint{1.401829in}{1.851580in}}{\pgfqpoint{1.401829in}{1.843344in}}%
\pgfpathcurveto{\pgfqpoint{1.401829in}{1.835107in}}{\pgfqpoint{1.405101in}{1.827207in}}{\pgfqpoint{1.410925in}{1.821384in}}%
\pgfpathcurveto{\pgfqpoint{1.416749in}{1.815560in}}{\pgfqpoint{1.424649in}{1.812287in}}{\pgfqpoint{1.432886in}{1.812287in}}%
\pgfpathclose%
\pgfusepath{stroke,fill}%
\end{pgfscope}%
\begin{pgfscope}%
\pgfpathrectangle{\pgfqpoint{0.100000in}{0.220728in}}{\pgfqpoint{3.696000in}{3.696000in}}%
\pgfusepath{clip}%
\pgfsetbuttcap%
\pgfsetroundjoin%
\definecolor{currentfill}{rgb}{0.121569,0.466667,0.705882}%
\pgfsetfillcolor{currentfill}%
\pgfsetfillopacity{0.468822}%
\pgfsetlinewidth{1.003750pt}%
\definecolor{currentstroke}{rgb}{0.121569,0.466667,0.705882}%
\pgfsetstrokecolor{currentstroke}%
\pgfsetstrokeopacity{0.468822}%
\pgfsetdash{}{0pt}%
\pgfpathmoveto{\pgfqpoint{2.035517in}{1.907670in}}%
\pgfpathcurveto{\pgfqpoint{2.043753in}{1.907670in}}{\pgfqpoint{2.051653in}{1.910943in}}{\pgfqpoint{2.057477in}{1.916767in}}%
\pgfpathcurveto{\pgfqpoint{2.063301in}{1.922590in}}{\pgfqpoint{2.066573in}{1.930490in}}{\pgfqpoint{2.066573in}{1.938727in}}%
\pgfpathcurveto{\pgfqpoint{2.066573in}{1.946963in}}{\pgfqpoint{2.063301in}{1.954863in}}{\pgfqpoint{2.057477in}{1.960687in}}%
\pgfpathcurveto{\pgfqpoint{2.051653in}{1.966511in}}{\pgfqpoint{2.043753in}{1.969783in}}{\pgfqpoint{2.035517in}{1.969783in}}%
\pgfpathcurveto{\pgfqpoint{2.027280in}{1.969783in}}{\pgfqpoint{2.019380in}{1.966511in}}{\pgfqpoint{2.013556in}{1.960687in}}%
\pgfpathcurveto{\pgfqpoint{2.007732in}{1.954863in}}{\pgfqpoint{2.004460in}{1.946963in}}{\pgfqpoint{2.004460in}{1.938727in}}%
\pgfpathcurveto{\pgfqpoint{2.004460in}{1.930490in}}{\pgfqpoint{2.007732in}{1.922590in}}{\pgfqpoint{2.013556in}{1.916767in}}%
\pgfpathcurveto{\pgfqpoint{2.019380in}{1.910943in}}{\pgfqpoint{2.027280in}{1.907670in}}{\pgfqpoint{2.035517in}{1.907670in}}%
\pgfpathclose%
\pgfusepath{stroke,fill}%
\end{pgfscope}%
\begin{pgfscope}%
\pgfpathrectangle{\pgfqpoint{0.100000in}{0.220728in}}{\pgfqpoint{3.696000in}{3.696000in}}%
\pgfusepath{clip}%
\pgfsetbuttcap%
\pgfsetroundjoin%
\definecolor{currentfill}{rgb}{0.121569,0.466667,0.705882}%
\pgfsetfillcolor{currentfill}%
\pgfsetfillopacity{0.470135}%
\pgfsetlinewidth{1.003750pt}%
\definecolor{currentstroke}{rgb}{0.121569,0.466667,0.705882}%
\pgfsetstrokecolor{currentstroke}%
\pgfsetstrokeopacity{0.470135}%
\pgfsetdash{}{0pt}%
\pgfpathmoveto{\pgfqpoint{2.035876in}{1.905987in}}%
\pgfpathcurveto{\pgfqpoint{2.044112in}{1.905987in}}{\pgfqpoint{2.052013in}{1.909259in}}{\pgfqpoint{2.057836in}{1.915083in}}%
\pgfpathcurveto{\pgfqpoint{2.063660in}{1.920907in}}{\pgfqpoint{2.066933in}{1.928807in}}{\pgfqpoint{2.066933in}{1.937043in}}%
\pgfpathcurveto{\pgfqpoint{2.066933in}{1.945280in}}{\pgfqpoint{2.063660in}{1.953180in}}{\pgfqpoint{2.057836in}{1.959004in}}%
\pgfpathcurveto{\pgfqpoint{2.052013in}{1.964828in}}{\pgfqpoint{2.044112in}{1.968100in}}{\pgfqpoint{2.035876in}{1.968100in}}%
\pgfpathcurveto{\pgfqpoint{2.027640in}{1.968100in}}{\pgfqpoint{2.019740in}{1.964828in}}{\pgfqpoint{2.013916in}{1.959004in}}%
\pgfpathcurveto{\pgfqpoint{2.008092in}{1.953180in}}{\pgfqpoint{2.004820in}{1.945280in}}{\pgfqpoint{2.004820in}{1.937043in}}%
\pgfpathcurveto{\pgfqpoint{2.004820in}{1.928807in}}{\pgfqpoint{2.008092in}{1.920907in}}{\pgfqpoint{2.013916in}{1.915083in}}%
\pgfpathcurveto{\pgfqpoint{2.019740in}{1.909259in}}{\pgfqpoint{2.027640in}{1.905987in}}{\pgfqpoint{2.035876in}{1.905987in}}%
\pgfpathclose%
\pgfusepath{stroke,fill}%
\end{pgfscope}%
\begin{pgfscope}%
\pgfpathrectangle{\pgfqpoint{0.100000in}{0.220728in}}{\pgfqpoint{3.696000in}{3.696000in}}%
\pgfusepath{clip}%
\pgfsetbuttcap%
\pgfsetroundjoin%
\definecolor{currentfill}{rgb}{0.121569,0.466667,0.705882}%
\pgfsetfillcolor{currentfill}%
\pgfsetfillopacity{0.470627}%
\pgfsetlinewidth{1.003750pt}%
\definecolor{currentstroke}{rgb}{0.121569,0.466667,0.705882}%
\pgfsetstrokecolor{currentstroke}%
\pgfsetstrokeopacity{0.470627}%
\pgfsetdash{}{0pt}%
\pgfpathmoveto{\pgfqpoint{1.428003in}{1.810198in}}%
\pgfpathcurveto{\pgfqpoint{1.436240in}{1.810198in}}{\pgfqpoint{1.444140in}{1.813470in}}{\pgfqpoint{1.449964in}{1.819294in}}%
\pgfpathcurveto{\pgfqpoint{1.455788in}{1.825118in}}{\pgfqpoint{1.459060in}{1.833018in}}{\pgfqpoint{1.459060in}{1.841254in}}%
\pgfpathcurveto{\pgfqpoint{1.459060in}{1.849491in}}{\pgfqpoint{1.455788in}{1.857391in}}{\pgfqpoint{1.449964in}{1.863215in}}%
\pgfpathcurveto{\pgfqpoint{1.444140in}{1.869039in}}{\pgfqpoint{1.436240in}{1.872311in}}{\pgfqpoint{1.428003in}{1.872311in}}%
\pgfpathcurveto{\pgfqpoint{1.419767in}{1.872311in}}{\pgfqpoint{1.411867in}{1.869039in}}{\pgfqpoint{1.406043in}{1.863215in}}%
\pgfpathcurveto{\pgfqpoint{1.400219in}{1.857391in}}{\pgfqpoint{1.396947in}{1.849491in}}{\pgfqpoint{1.396947in}{1.841254in}}%
\pgfpathcurveto{\pgfqpoint{1.396947in}{1.833018in}}{\pgfqpoint{1.400219in}{1.825118in}}{\pgfqpoint{1.406043in}{1.819294in}}%
\pgfpathcurveto{\pgfqpoint{1.411867in}{1.813470in}}{\pgfqpoint{1.419767in}{1.810198in}}{\pgfqpoint{1.428003in}{1.810198in}}%
\pgfpathclose%
\pgfusepath{stroke,fill}%
\end{pgfscope}%
\begin{pgfscope}%
\pgfpathrectangle{\pgfqpoint{0.100000in}{0.220728in}}{\pgfqpoint{3.696000in}{3.696000in}}%
\pgfusepath{clip}%
\pgfsetbuttcap%
\pgfsetroundjoin%
\definecolor{currentfill}{rgb}{0.121569,0.466667,0.705882}%
\pgfsetfillcolor{currentfill}%
\pgfsetfillopacity{0.471569}%
\pgfsetlinewidth{1.003750pt}%
\definecolor{currentstroke}{rgb}{0.121569,0.466667,0.705882}%
\pgfsetstrokecolor{currentstroke}%
\pgfsetstrokeopacity{0.471569}%
\pgfsetdash{}{0pt}%
\pgfpathmoveto{\pgfqpoint{1.423673in}{1.805841in}}%
\pgfpathcurveto{\pgfqpoint{1.431910in}{1.805841in}}{\pgfqpoint{1.439810in}{1.809114in}}{\pgfqpoint{1.445633in}{1.814938in}}%
\pgfpathcurveto{\pgfqpoint{1.451457in}{1.820762in}}{\pgfqpoint{1.454730in}{1.828662in}}{\pgfqpoint{1.454730in}{1.836898in}}%
\pgfpathcurveto{\pgfqpoint{1.454730in}{1.845134in}}{\pgfqpoint{1.451457in}{1.853034in}}{\pgfqpoint{1.445633in}{1.858858in}}%
\pgfpathcurveto{\pgfqpoint{1.439810in}{1.864682in}}{\pgfqpoint{1.431910in}{1.867954in}}{\pgfqpoint{1.423673in}{1.867954in}}%
\pgfpathcurveto{\pgfqpoint{1.415437in}{1.867954in}}{\pgfqpoint{1.407537in}{1.864682in}}{\pgfqpoint{1.401713in}{1.858858in}}%
\pgfpathcurveto{\pgfqpoint{1.395889in}{1.853034in}}{\pgfqpoint{1.392617in}{1.845134in}}{\pgfqpoint{1.392617in}{1.836898in}}%
\pgfpathcurveto{\pgfqpoint{1.392617in}{1.828662in}}{\pgfqpoint{1.395889in}{1.820762in}}{\pgfqpoint{1.401713in}{1.814938in}}%
\pgfpathcurveto{\pgfqpoint{1.407537in}{1.809114in}}{\pgfqpoint{1.415437in}{1.805841in}}{\pgfqpoint{1.423673in}{1.805841in}}%
\pgfpathclose%
\pgfusepath{stroke,fill}%
\end{pgfscope}%
\begin{pgfscope}%
\pgfpathrectangle{\pgfqpoint{0.100000in}{0.220728in}}{\pgfqpoint{3.696000in}{3.696000in}}%
\pgfusepath{clip}%
\pgfsetbuttcap%
\pgfsetroundjoin%
\definecolor{currentfill}{rgb}{0.121569,0.466667,0.705882}%
\pgfsetfillcolor{currentfill}%
\pgfsetfillopacity{0.472153}%
\pgfsetlinewidth{1.003750pt}%
\definecolor{currentstroke}{rgb}{0.121569,0.466667,0.705882}%
\pgfsetstrokecolor{currentstroke}%
\pgfsetstrokeopacity{0.472153}%
\pgfsetdash{}{0pt}%
\pgfpathmoveto{\pgfqpoint{2.036493in}{1.905591in}}%
\pgfpathcurveto{\pgfqpoint{2.044729in}{1.905591in}}{\pgfqpoint{2.052629in}{1.908863in}}{\pgfqpoint{2.058453in}{1.914687in}}%
\pgfpathcurveto{\pgfqpoint{2.064277in}{1.920511in}}{\pgfqpoint{2.067549in}{1.928411in}}{\pgfqpoint{2.067549in}{1.936647in}}%
\pgfpathcurveto{\pgfqpoint{2.067549in}{1.944884in}}{\pgfqpoint{2.064277in}{1.952784in}}{\pgfqpoint{2.058453in}{1.958608in}}%
\pgfpathcurveto{\pgfqpoint{2.052629in}{1.964432in}}{\pgfqpoint{2.044729in}{1.967704in}}{\pgfqpoint{2.036493in}{1.967704in}}%
\pgfpathcurveto{\pgfqpoint{2.028257in}{1.967704in}}{\pgfqpoint{2.020357in}{1.964432in}}{\pgfqpoint{2.014533in}{1.958608in}}%
\pgfpathcurveto{\pgfqpoint{2.008709in}{1.952784in}}{\pgfqpoint{2.005436in}{1.944884in}}{\pgfqpoint{2.005436in}{1.936647in}}%
\pgfpathcurveto{\pgfqpoint{2.005436in}{1.928411in}}{\pgfqpoint{2.008709in}{1.920511in}}{\pgfqpoint{2.014533in}{1.914687in}}%
\pgfpathcurveto{\pgfqpoint{2.020357in}{1.908863in}}{\pgfqpoint{2.028257in}{1.905591in}}{\pgfqpoint{2.036493in}{1.905591in}}%
\pgfpathclose%
\pgfusepath{stroke,fill}%
\end{pgfscope}%
\begin{pgfscope}%
\pgfpathrectangle{\pgfqpoint{0.100000in}{0.220728in}}{\pgfqpoint{3.696000in}{3.696000in}}%
\pgfusepath{clip}%
\pgfsetbuttcap%
\pgfsetroundjoin%
\definecolor{currentfill}{rgb}{0.121569,0.466667,0.705882}%
\pgfsetfillcolor{currentfill}%
\pgfsetfillopacity{0.473914}%
\pgfsetlinewidth{1.003750pt}%
\definecolor{currentstroke}{rgb}{0.121569,0.466667,0.705882}%
\pgfsetstrokecolor{currentstroke}%
\pgfsetstrokeopacity{0.473914}%
\pgfsetdash{}{0pt}%
\pgfpathmoveto{\pgfqpoint{1.416246in}{1.801348in}}%
\pgfpathcurveto{\pgfqpoint{1.424483in}{1.801348in}}{\pgfqpoint{1.432383in}{1.804620in}}{\pgfqpoint{1.438206in}{1.810444in}}%
\pgfpathcurveto{\pgfqpoint{1.444030in}{1.816268in}}{\pgfqpoint{1.447303in}{1.824168in}}{\pgfqpoint{1.447303in}{1.832405in}}%
\pgfpathcurveto{\pgfqpoint{1.447303in}{1.840641in}}{\pgfqpoint{1.444030in}{1.848541in}}{\pgfqpoint{1.438206in}{1.854365in}}%
\pgfpathcurveto{\pgfqpoint{1.432383in}{1.860189in}}{\pgfqpoint{1.424483in}{1.863461in}}{\pgfqpoint{1.416246in}{1.863461in}}%
\pgfpathcurveto{\pgfqpoint{1.408010in}{1.863461in}}{\pgfqpoint{1.400110in}{1.860189in}}{\pgfqpoint{1.394286in}{1.854365in}}%
\pgfpathcurveto{\pgfqpoint{1.388462in}{1.848541in}}{\pgfqpoint{1.385190in}{1.840641in}}{\pgfqpoint{1.385190in}{1.832405in}}%
\pgfpathcurveto{\pgfqpoint{1.385190in}{1.824168in}}{\pgfqpoint{1.388462in}{1.816268in}}{\pgfqpoint{1.394286in}{1.810444in}}%
\pgfpathcurveto{\pgfqpoint{1.400110in}{1.804620in}}{\pgfqpoint{1.408010in}{1.801348in}}{\pgfqpoint{1.416246in}{1.801348in}}%
\pgfpathclose%
\pgfusepath{stroke,fill}%
\end{pgfscope}%
\begin{pgfscope}%
\pgfpathrectangle{\pgfqpoint{0.100000in}{0.220728in}}{\pgfqpoint{3.696000in}{3.696000in}}%
\pgfusepath{clip}%
\pgfsetbuttcap%
\pgfsetroundjoin%
\definecolor{currentfill}{rgb}{0.121569,0.466667,0.705882}%
\pgfsetfillcolor{currentfill}%
\pgfsetfillopacity{0.474297}%
\pgfsetlinewidth{1.003750pt}%
\definecolor{currentstroke}{rgb}{0.121569,0.466667,0.705882}%
\pgfsetstrokecolor{currentstroke}%
\pgfsetstrokeopacity{0.474297}%
\pgfsetdash{}{0pt}%
\pgfpathmoveto{\pgfqpoint{2.037402in}{1.903257in}}%
\pgfpathcurveto{\pgfqpoint{2.045639in}{1.903257in}}{\pgfqpoint{2.053539in}{1.906529in}}{\pgfqpoint{2.059363in}{1.912353in}}%
\pgfpathcurveto{\pgfqpoint{2.065187in}{1.918177in}}{\pgfqpoint{2.068459in}{1.926077in}}{\pgfqpoint{2.068459in}{1.934313in}}%
\pgfpathcurveto{\pgfqpoint{2.068459in}{1.942549in}}{\pgfqpoint{2.065187in}{1.950449in}}{\pgfqpoint{2.059363in}{1.956273in}}%
\pgfpathcurveto{\pgfqpoint{2.053539in}{1.962097in}}{\pgfqpoint{2.045639in}{1.965370in}}{\pgfqpoint{2.037402in}{1.965370in}}%
\pgfpathcurveto{\pgfqpoint{2.029166in}{1.965370in}}{\pgfqpoint{2.021266in}{1.962097in}}{\pgfqpoint{2.015442in}{1.956273in}}%
\pgfpathcurveto{\pgfqpoint{2.009618in}{1.950449in}}{\pgfqpoint{2.006346in}{1.942549in}}{\pgfqpoint{2.006346in}{1.934313in}}%
\pgfpathcurveto{\pgfqpoint{2.006346in}{1.926077in}}{\pgfqpoint{2.009618in}{1.918177in}}{\pgfqpoint{2.015442in}{1.912353in}}%
\pgfpathcurveto{\pgfqpoint{2.021266in}{1.906529in}}{\pgfqpoint{2.029166in}{1.903257in}}{\pgfqpoint{2.037402in}{1.903257in}}%
\pgfpathclose%
\pgfusepath{stroke,fill}%
\end{pgfscope}%
\begin{pgfscope}%
\pgfpathrectangle{\pgfqpoint{0.100000in}{0.220728in}}{\pgfqpoint{3.696000in}{3.696000in}}%
\pgfusepath{clip}%
\pgfsetbuttcap%
\pgfsetroundjoin%
\definecolor{currentfill}{rgb}{0.121569,0.466667,0.705882}%
\pgfsetfillcolor{currentfill}%
\pgfsetfillopacity{0.476500}%
\pgfsetlinewidth{1.003750pt}%
\definecolor{currentstroke}{rgb}{0.121569,0.466667,0.705882}%
\pgfsetstrokecolor{currentstroke}%
\pgfsetstrokeopacity{0.476500}%
\pgfsetdash{}{0pt}%
\pgfpathmoveto{\pgfqpoint{1.410496in}{1.799903in}}%
\pgfpathcurveto{\pgfqpoint{1.418733in}{1.799903in}}{\pgfqpoint{1.426633in}{1.803175in}}{\pgfqpoint{1.432457in}{1.808999in}}%
\pgfpathcurveto{\pgfqpoint{1.438281in}{1.814823in}}{\pgfqpoint{1.441553in}{1.822723in}}{\pgfqpoint{1.441553in}{1.830959in}}%
\pgfpathcurveto{\pgfqpoint{1.441553in}{1.839195in}}{\pgfqpoint{1.438281in}{1.847095in}}{\pgfqpoint{1.432457in}{1.852919in}}%
\pgfpathcurveto{\pgfqpoint{1.426633in}{1.858743in}}{\pgfqpoint{1.418733in}{1.862016in}}{\pgfqpoint{1.410496in}{1.862016in}}%
\pgfpathcurveto{\pgfqpoint{1.402260in}{1.862016in}}{\pgfqpoint{1.394360in}{1.858743in}}{\pgfqpoint{1.388536in}{1.852919in}}%
\pgfpathcurveto{\pgfqpoint{1.382712in}{1.847095in}}{\pgfqpoint{1.379440in}{1.839195in}}{\pgfqpoint{1.379440in}{1.830959in}}%
\pgfpathcurveto{\pgfqpoint{1.379440in}{1.822723in}}{\pgfqpoint{1.382712in}{1.814823in}}{\pgfqpoint{1.388536in}{1.808999in}}%
\pgfpathcurveto{\pgfqpoint{1.394360in}{1.803175in}}{\pgfqpoint{1.402260in}{1.799903in}}{\pgfqpoint{1.410496in}{1.799903in}}%
\pgfpathclose%
\pgfusepath{stroke,fill}%
\end{pgfscope}%
\begin{pgfscope}%
\pgfpathrectangle{\pgfqpoint{0.100000in}{0.220728in}}{\pgfqpoint{3.696000in}{3.696000in}}%
\pgfusepath{clip}%
\pgfsetbuttcap%
\pgfsetroundjoin%
\definecolor{currentfill}{rgb}{0.121569,0.466667,0.705882}%
\pgfsetfillcolor{currentfill}%
\pgfsetfillopacity{0.476826}%
\pgfsetlinewidth{1.003750pt}%
\definecolor{currentstroke}{rgb}{0.121569,0.466667,0.705882}%
\pgfsetstrokecolor{currentstroke}%
\pgfsetstrokeopacity{0.476826}%
\pgfsetdash{}{0pt}%
\pgfpathmoveto{\pgfqpoint{2.038903in}{1.901676in}}%
\pgfpathcurveto{\pgfqpoint{2.047139in}{1.901676in}}{\pgfqpoint{2.055039in}{1.904948in}}{\pgfqpoint{2.060863in}{1.910772in}}%
\pgfpathcurveto{\pgfqpoint{2.066687in}{1.916596in}}{\pgfqpoint{2.069960in}{1.924496in}}{\pgfqpoint{2.069960in}{1.932732in}}%
\pgfpathcurveto{\pgfqpoint{2.069960in}{1.940968in}}{\pgfqpoint{2.066687in}{1.948868in}}{\pgfqpoint{2.060863in}{1.954692in}}%
\pgfpathcurveto{\pgfqpoint{2.055039in}{1.960516in}}{\pgfqpoint{2.047139in}{1.963789in}}{\pgfqpoint{2.038903in}{1.963789in}}%
\pgfpathcurveto{\pgfqpoint{2.030667in}{1.963789in}}{\pgfqpoint{2.022767in}{1.960516in}}{\pgfqpoint{2.016943in}{1.954692in}}%
\pgfpathcurveto{\pgfqpoint{2.011119in}{1.948868in}}{\pgfqpoint{2.007847in}{1.940968in}}{\pgfqpoint{2.007847in}{1.932732in}}%
\pgfpathcurveto{\pgfqpoint{2.007847in}{1.924496in}}{\pgfqpoint{2.011119in}{1.916596in}}{\pgfqpoint{2.016943in}{1.910772in}}%
\pgfpathcurveto{\pgfqpoint{2.022767in}{1.904948in}}{\pgfqpoint{2.030667in}{1.901676in}}{\pgfqpoint{2.038903in}{1.901676in}}%
\pgfpathclose%
\pgfusepath{stroke,fill}%
\end{pgfscope}%
\begin{pgfscope}%
\pgfpathrectangle{\pgfqpoint{0.100000in}{0.220728in}}{\pgfqpoint{3.696000in}{3.696000in}}%
\pgfusepath{clip}%
\pgfsetbuttcap%
\pgfsetroundjoin%
\definecolor{currentfill}{rgb}{0.121569,0.466667,0.705882}%
\pgfsetfillcolor{currentfill}%
\pgfsetfillopacity{0.477753}%
\pgfsetlinewidth{1.003750pt}%
\definecolor{currentstroke}{rgb}{0.121569,0.466667,0.705882}%
\pgfsetstrokecolor{currentstroke}%
\pgfsetstrokeopacity{0.477753}%
\pgfsetdash{}{0pt}%
\pgfpathmoveto{\pgfqpoint{1.405181in}{1.796062in}}%
\pgfpathcurveto{\pgfqpoint{1.413418in}{1.796062in}}{\pgfqpoint{1.421318in}{1.799334in}}{\pgfqpoint{1.427142in}{1.805158in}}%
\pgfpathcurveto{\pgfqpoint{1.432966in}{1.810982in}}{\pgfqpoint{1.436238in}{1.818882in}}{\pgfqpoint{1.436238in}{1.827118in}}%
\pgfpathcurveto{\pgfqpoint{1.436238in}{1.835355in}}{\pgfqpoint{1.432966in}{1.843255in}}{\pgfqpoint{1.427142in}{1.849079in}}%
\pgfpathcurveto{\pgfqpoint{1.421318in}{1.854902in}}{\pgfqpoint{1.413418in}{1.858175in}}{\pgfqpoint{1.405181in}{1.858175in}}%
\pgfpathcurveto{\pgfqpoint{1.396945in}{1.858175in}}{\pgfqpoint{1.389045in}{1.854902in}}{\pgfqpoint{1.383221in}{1.849079in}}%
\pgfpathcurveto{\pgfqpoint{1.377397in}{1.843255in}}{\pgfqpoint{1.374125in}{1.835355in}}{\pgfqpoint{1.374125in}{1.827118in}}%
\pgfpathcurveto{\pgfqpoint{1.374125in}{1.818882in}}{\pgfqpoint{1.377397in}{1.810982in}}{\pgfqpoint{1.383221in}{1.805158in}}%
\pgfpathcurveto{\pgfqpoint{1.389045in}{1.799334in}}{\pgfqpoint{1.396945in}{1.796062in}}{\pgfqpoint{1.405181in}{1.796062in}}%
\pgfpathclose%
\pgfusepath{stroke,fill}%
\end{pgfscope}%
\begin{pgfscope}%
\pgfpathrectangle{\pgfqpoint{0.100000in}{0.220728in}}{\pgfqpoint{3.696000in}{3.696000in}}%
\pgfusepath{clip}%
\pgfsetbuttcap%
\pgfsetroundjoin%
\definecolor{currentfill}{rgb}{0.121569,0.466667,0.705882}%
\pgfsetfillcolor{currentfill}%
\pgfsetfillopacity{0.479136}%
\pgfsetlinewidth{1.003750pt}%
\definecolor{currentstroke}{rgb}{0.121569,0.466667,0.705882}%
\pgfsetstrokecolor{currentstroke}%
\pgfsetstrokeopacity{0.479136}%
\pgfsetdash{}{0pt}%
\pgfpathmoveto{\pgfqpoint{1.400928in}{1.792799in}}%
\pgfpathcurveto{\pgfqpoint{1.409164in}{1.792799in}}{\pgfqpoint{1.417064in}{1.796072in}}{\pgfqpoint{1.422888in}{1.801896in}}%
\pgfpathcurveto{\pgfqpoint{1.428712in}{1.807720in}}{\pgfqpoint{1.431984in}{1.815620in}}{\pgfqpoint{1.431984in}{1.823856in}}%
\pgfpathcurveto{\pgfqpoint{1.431984in}{1.832092in}}{\pgfqpoint{1.428712in}{1.839992in}}{\pgfqpoint{1.422888in}{1.845816in}}%
\pgfpathcurveto{\pgfqpoint{1.417064in}{1.851640in}}{\pgfqpoint{1.409164in}{1.854912in}}{\pgfqpoint{1.400928in}{1.854912in}}%
\pgfpathcurveto{\pgfqpoint{1.392692in}{1.854912in}}{\pgfqpoint{1.384791in}{1.851640in}}{\pgfqpoint{1.378968in}{1.845816in}}%
\pgfpathcurveto{\pgfqpoint{1.373144in}{1.839992in}}{\pgfqpoint{1.369871in}{1.832092in}}{\pgfqpoint{1.369871in}{1.823856in}}%
\pgfpathcurveto{\pgfqpoint{1.369871in}{1.815620in}}{\pgfqpoint{1.373144in}{1.807720in}}{\pgfqpoint{1.378968in}{1.801896in}}%
\pgfpathcurveto{\pgfqpoint{1.384791in}{1.796072in}}{\pgfqpoint{1.392692in}{1.792799in}}{\pgfqpoint{1.400928in}{1.792799in}}%
\pgfpathclose%
\pgfusepath{stroke,fill}%
\end{pgfscope}%
\begin{pgfscope}%
\pgfpathrectangle{\pgfqpoint{0.100000in}{0.220728in}}{\pgfqpoint{3.696000in}{3.696000in}}%
\pgfusepath{clip}%
\pgfsetbuttcap%
\pgfsetroundjoin%
\definecolor{currentfill}{rgb}{0.121569,0.466667,0.705882}%
\pgfsetfillcolor{currentfill}%
\pgfsetfillopacity{0.479269}%
\pgfsetlinewidth{1.003750pt}%
\definecolor{currentstroke}{rgb}{0.121569,0.466667,0.705882}%
\pgfsetstrokecolor{currentstroke}%
\pgfsetstrokeopacity{0.479269}%
\pgfsetdash{}{0pt}%
\pgfpathmoveto{\pgfqpoint{2.040740in}{1.896931in}}%
\pgfpathcurveto{\pgfqpoint{2.048977in}{1.896931in}}{\pgfqpoint{2.056877in}{1.900203in}}{\pgfqpoint{2.062701in}{1.906027in}}%
\pgfpathcurveto{\pgfqpoint{2.068525in}{1.911851in}}{\pgfqpoint{2.071797in}{1.919751in}}{\pgfqpoint{2.071797in}{1.927987in}}%
\pgfpathcurveto{\pgfqpoint{2.071797in}{1.936223in}}{\pgfqpoint{2.068525in}{1.944124in}}{\pgfqpoint{2.062701in}{1.949947in}}%
\pgfpathcurveto{\pgfqpoint{2.056877in}{1.955771in}}{\pgfqpoint{2.048977in}{1.959044in}}{\pgfqpoint{2.040740in}{1.959044in}}%
\pgfpathcurveto{\pgfqpoint{2.032504in}{1.959044in}}{\pgfqpoint{2.024604in}{1.955771in}}{\pgfqpoint{2.018780in}{1.949947in}}%
\pgfpathcurveto{\pgfqpoint{2.012956in}{1.944124in}}{\pgfqpoint{2.009684in}{1.936223in}}{\pgfqpoint{2.009684in}{1.927987in}}%
\pgfpathcurveto{\pgfqpoint{2.009684in}{1.919751in}}{\pgfqpoint{2.012956in}{1.911851in}}{\pgfqpoint{2.018780in}{1.906027in}}%
\pgfpathcurveto{\pgfqpoint{2.024604in}{1.900203in}}{\pgfqpoint{2.032504in}{1.896931in}}{\pgfqpoint{2.040740in}{1.896931in}}%
\pgfpathclose%
\pgfusepath{stroke,fill}%
\end{pgfscope}%
\begin{pgfscope}%
\pgfpathrectangle{\pgfqpoint{0.100000in}{0.220728in}}{\pgfqpoint{3.696000in}{3.696000in}}%
\pgfusepath{clip}%
\pgfsetbuttcap%
\pgfsetroundjoin%
\definecolor{currentfill}{rgb}{0.121569,0.466667,0.705882}%
\pgfsetfillcolor{currentfill}%
\pgfsetfillopacity{0.481008}%
\pgfsetlinewidth{1.003750pt}%
\definecolor{currentstroke}{rgb}{0.121569,0.466667,0.705882}%
\pgfsetstrokecolor{currentstroke}%
\pgfsetstrokeopacity{0.481008}%
\pgfsetdash{}{0pt}%
\pgfpathmoveto{\pgfqpoint{2.041325in}{1.896771in}}%
\pgfpathcurveto{\pgfqpoint{2.049561in}{1.896771in}}{\pgfqpoint{2.057461in}{1.900043in}}{\pgfqpoint{2.063285in}{1.905867in}}%
\pgfpathcurveto{\pgfqpoint{2.069109in}{1.911691in}}{\pgfqpoint{2.072381in}{1.919591in}}{\pgfqpoint{2.072381in}{1.927827in}}%
\pgfpathcurveto{\pgfqpoint{2.072381in}{1.936064in}}{\pgfqpoint{2.069109in}{1.943964in}}{\pgfqpoint{2.063285in}{1.949788in}}%
\pgfpathcurveto{\pgfqpoint{2.057461in}{1.955612in}}{\pgfqpoint{2.049561in}{1.958884in}}{\pgfqpoint{2.041325in}{1.958884in}}%
\pgfpathcurveto{\pgfqpoint{2.033088in}{1.958884in}}{\pgfqpoint{2.025188in}{1.955612in}}{\pgfqpoint{2.019365in}{1.949788in}}%
\pgfpathcurveto{\pgfqpoint{2.013541in}{1.943964in}}{\pgfqpoint{2.010268in}{1.936064in}}{\pgfqpoint{2.010268in}{1.927827in}}%
\pgfpathcurveto{\pgfqpoint{2.010268in}{1.919591in}}{\pgfqpoint{2.013541in}{1.911691in}}{\pgfqpoint{2.019365in}{1.905867in}}%
\pgfpathcurveto{\pgfqpoint{2.025188in}{1.900043in}}{\pgfqpoint{2.033088in}{1.896771in}}{\pgfqpoint{2.041325in}{1.896771in}}%
\pgfpathclose%
\pgfusepath{stroke,fill}%
\end{pgfscope}%
\begin{pgfscope}%
\pgfpathrectangle{\pgfqpoint{0.100000in}{0.220728in}}{\pgfqpoint{3.696000in}{3.696000in}}%
\pgfusepath{clip}%
\pgfsetbuttcap%
\pgfsetroundjoin%
\definecolor{currentfill}{rgb}{0.121569,0.466667,0.705882}%
\pgfsetfillcolor{currentfill}%
\pgfsetfillopacity{0.481895}%
\pgfsetlinewidth{1.003750pt}%
\definecolor{currentstroke}{rgb}{0.121569,0.466667,0.705882}%
\pgfsetstrokecolor{currentstroke}%
\pgfsetstrokeopacity{0.481895}%
\pgfsetdash{}{0pt}%
\pgfpathmoveto{\pgfqpoint{1.392960in}{1.788776in}}%
\pgfpathcurveto{\pgfqpoint{1.401197in}{1.788776in}}{\pgfqpoint{1.409097in}{1.792048in}}{\pgfqpoint{1.414921in}{1.797872in}}%
\pgfpathcurveto{\pgfqpoint{1.420745in}{1.803696in}}{\pgfqpoint{1.424017in}{1.811596in}}{\pgfqpoint{1.424017in}{1.819832in}}%
\pgfpathcurveto{\pgfqpoint{1.424017in}{1.828068in}}{\pgfqpoint{1.420745in}{1.835969in}}{\pgfqpoint{1.414921in}{1.841792in}}%
\pgfpathcurveto{\pgfqpoint{1.409097in}{1.847616in}}{\pgfqpoint{1.401197in}{1.850889in}}{\pgfqpoint{1.392960in}{1.850889in}}%
\pgfpathcurveto{\pgfqpoint{1.384724in}{1.850889in}}{\pgfqpoint{1.376824in}{1.847616in}}{\pgfqpoint{1.371000in}{1.841792in}}%
\pgfpathcurveto{\pgfqpoint{1.365176in}{1.835969in}}{\pgfqpoint{1.361904in}{1.828068in}}{\pgfqpoint{1.361904in}{1.819832in}}%
\pgfpathcurveto{\pgfqpoint{1.361904in}{1.811596in}}{\pgfqpoint{1.365176in}{1.803696in}}{\pgfqpoint{1.371000in}{1.797872in}}%
\pgfpathcurveto{\pgfqpoint{1.376824in}{1.792048in}}{\pgfqpoint{1.384724in}{1.788776in}}{\pgfqpoint{1.392960in}{1.788776in}}%
\pgfpathclose%
\pgfusepath{stroke,fill}%
\end{pgfscope}%
\begin{pgfscope}%
\pgfpathrectangle{\pgfqpoint{0.100000in}{0.220728in}}{\pgfqpoint{3.696000in}{3.696000in}}%
\pgfusepath{clip}%
\pgfsetbuttcap%
\pgfsetroundjoin%
\definecolor{currentfill}{rgb}{0.121569,0.466667,0.705882}%
\pgfsetfillcolor{currentfill}%
\pgfsetfillopacity{0.482828}%
\pgfsetlinewidth{1.003750pt}%
\definecolor{currentstroke}{rgb}{0.121569,0.466667,0.705882}%
\pgfsetstrokecolor{currentstroke}%
\pgfsetstrokeopacity{0.482828}%
\pgfsetdash{}{0pt}%
\pgfpathmoveto{\pgfqpoint{2.042365in}{1.894475in}}%
\pgfpathcurveto{\pgfqpoint{2.050602in}{1.894475in}}{\pgfqpoint{2.058502in}{1.897747in}}{\pgfqpoint{2.064326in}{1.903571in}}%
\pgfpathcurveto{\pgfqpoint{2.070149in}{1.909395in}}{\pgfqpoint{2.073422in}{1.917295in}}{\pgfqpoint{2.073422in}{1.925532in}}%
\pgfpathcurveto{\pgfqpoint{2.073422in}{1.933768in}}{\pgfqpoint{2.070149in}{1.941668in}}{\pgfqpoint{2.064326in}{1.947492in}}%
\pgfpathcurveto{\pgfqpoint{2.058502in}{1.953316in}}{\pgfqpoint{2.050602in}{1.956588in}}{\pgfqpoint{2.042365in}{1.956588in}}%
\pgfpathcurveto{\pgfqpoint{2.034129in}{1.956588in}}{\pgfqpoint{2.026229in}{1.953316in}}{\pgfqpoint{2.020405in}{1.947492in}}%
\pgfpathcurveto{\pgfqpoint{2.014581in}{1.941668in}}{\pgfqpoint{2.011309in}{1.933768in}}{\pgfqpoint{2.011309in}{1.925532in}}%
\pgfpathcurveto{\pgfqpoint{2.011309in}{1.917295in}}{\pgfqpoint{2.014581in}{1.909395in}}{\pgfqpoint{2.020405in}{1.903571in}}%
\pgfpathcurveto{\pgfqpoint{2.026229in}{1.897747in}}{\pgfqpoint{2.034129in}{1.894475in}}{\pgfqpoint{2.042365in}{1.894475in}}%
\pgfpathclose%
\pgfusepath{stroke,fill}%
\end{pgfscope}%
\begin{pgfscope}%
\pgfpathrectangle{\pgfqpoint{0.100000in}{0.220728in}}{\pgfqpoint{3.696000in}{3.696000in}}%
\pgfusepath{clip}%
\pgfsetbuttcap%
\pgfsetroundjoin%
\definecolor{currentfill}{rgb}{0.121569,0.466667,0.705882}%
\pgfsetfillcolor{currentfill}%
\pgfsetfillopacity{0.483154}%
\pgfsetlinewidth{1.003750pt}%
\definecolor{currentstroke}{rgb}{0.121569,0.466667,0.705882}%
\pgfsetstrokecolor{currentstroke}%
\pgfsetstrokeopacity{0.483154}%
\pgfsetdash{}{0pt}%
\pgfpathmoveto{\pgfqpoint{1.387627in}{1.781803in}}%
\pgfpathcurveto{\pgfqpoint{1.395863in}{1.781803in}}{\pgfqpoint{1.403763in}{1.785075in}}{\pgfqpoint{1.409587in}{1.790899in}}%
\pgfpathcurveto{\pgfqpoint{1.415411in}{1.796723in}}{\pgfqpoint{1.418683in}{1.804623in}}{\pgfqpoint{1.418683in}{1.812859in}}%
\pgfpathcurveto{\pgfqpoint{1.418683in}{1.821095in}}{\pgfqpoint{1.415411in}{1.828995in}}{\pgfqpoint{1.409587in}{1.834819in}}%
\pgfpathcurveto{\pgfqpoint{1.403763in}{1.840643in}}{\pgfqpoint{1.395863in}{1.843916in}}{\pgfqpoint{1.387627in}{1.843916in}}%
\pgfpathcurveto{\pgfqpoint{1.379391in}{1.843916in}}{\pgfqpoint{1.371491in}{1.840643in}}{\pgfqpoint{1.365667in}{1.834819in}}%
\pgfpathcurveto{\pgfqpoint{1.359843in}{1.828995in}}{\pgfqpoint{1.356570in}{1.821095in}}{\pgfqpoint{1.356570in}{1.812859in}}%
\pgfpathcurveto{\pgfqpoint{1.356570in}{1.804623in}}{\pgfqpoint{1.359843in}{1.796723in}}{\pgfqpoint{1.365667in}{1.790899in}}%
\pgfpathcurveto{\pgfqpoint{1.371491in}{1.785075in}}{\pgfqpoint{1.379391in}{1.781803in}}{\pgfqpoint{1.387627in}{1.781803in}}%
\pgfpathclose%
\pgfusepath{stroke,fill}%
\end{pgfscope}%
\begin{pgfscope}%
\pgfpathrectangle{\pgfqpoint{0.100000in}{0.220728in}}{\pgfqpoint{3.696000in}{3.696000in}}%
\pgfusepath{clip}%
\pgfsetbuttcap%
\pgfsetroundjoin%
\definecolor{currentfill}{rgb}{0.121569,0.466667,0.705882}%
\pgfsetfillcolor{currentfill}%
\pgfsetfillopacity{0.484246}%
\pgfsetlinewidth{1.003750pt}%
\definecolor{currentstroke}{rgb}{0.121569,0.466667,0.705882}%
\pgfsetstrokecolor{currentstroke}%
\pgfsetstrokeopacity{0.484246}%
\pgfsetdash{}{0pt}%
\pgfpathmoveto{\pgfqpoint{1.383144in}{1.779088in}}%
\pgfpathcurveto{\pgfqpoint{1.391380in}{1.779088in}}{\pgfqpoint{1.399280in}{1.782360in}}{\pgfqpoint{1.405104in}{1.788184in}}%
\pgfpathcurveto{\pgfqpoint{1.410928in}{1.794008in}}{\pgfqpoint{1.414200in}{1.801908in}}{\pgfqpoint{1.414200in}{1.810144in}}%
\pgfpathcurveto{\pgfqpoint{1.414200in}{1.818380in}}{\pgfqpoint{1.410928in}{1.826280in}}{\pgfqpoint{1.405104in}{1.832104in}}%
\pgfpathcurveto{\pgfqpoint{1.399280in}{1.837928in}}{\pgfqpoint{1.391380in}{1.841201in}}{\pgfqpoint{1.383144in}{1.841201in}}%
\pgfpathcurveto{\pgfqpoint{1.374907in}{1.841201in}}{\pgfqpoint{1.367007in}{1.837928in}}{\pgfqpoint{1.361183in}{1.832104in}}%
\pgfpathcurveto{\pgfqpoint{1.355359in}{1.826280in}}{\pgfqpoint{1.352087in}{1.818380in}}{\pgfqpoint{1.352087in}{1.810144in}}%
\pgfpathcurveto{\pgfqpoint{1.352087in}{1.801908in}}{\pgfqpoint{1.355359in}{1.794008in}}{\pgfqpoint{1.361183in}{1.788184in}}%
\pgfpathcurveto{\pgfqpoint{1.367007in}{1.782360in}}{\pgfqpoint{1.374907in}{1.779088in}}{\pgfqpoint{1.383144in}{1.779088in}}%
\pgfpathclose%
\pgfusepath{stroke,fill}%
\end{pgfscope}%
\begin{pgfscope}%
\pgfpathrectangle{\pgfqpoint{0.100000in}{0.220728in}}{\pgfqpoint{3.696000in}{3.696000in}}%
\pgfusepath{clip}%
\pgfsetbuttcap%
\pgfsetroundjoin%
\definecolor{currentfill}{rgb}{0.121569,0.466667,0.705882}%
\pgfsetfillcolor{currentfill}%
\pgfsetfillopacity{0.485692}%
\pgfsetlinewidth{1.003750pt}%
\definecolor{currentstroke}{rgb}{0.121569,0.466667,0.705882}%
\pgfsetstrokecolor{currentstroke}%
\pgfsetstrokeopacity{0.485692}%
\pgfsetdash{}{0pt}%
\pgfpathmoveto{\pgfqpoint{2.044323in}{1.893894in}}%
\pgfpathcurveto{\pgfqpoint{2.052560in}{1.893894in}}{\pgfqpoint{2.060460in}{1.897167in}}{\pgfqpoint{2.066284in}{1.902991in}}%
\pgfpathcurveto{\pgfqpoint{2.072108in}{1.908815in}}{\pgfqpoint{2.075380in}{1.916715in}}{\pgfqpoint{2.075380in}{1.924951in}}%
\pgfpathcurveto{\pgfqpoint{2.075380in}{1.933187in}}{\pgfqpoint{2.072108in}{1.941087in}}{\pgfqpoint{2.066284in}{1.946911in}}%
\pgfpathcurveto{\pgfqpoint{2.060460in}{1.952735in}}{\pgfqpoint{2.052560in}{1.956007in}}{\pgfqpoint{2.044323in}{1.956007in}}%
\pgfpathcurveto{\pgfqpoint{2.036087in}{1.956007in}}{\pgfqpoint{2.028187in}{1.952735in}}{\pgfqpoint{2.022363in}{1.946911in}}%
\pgfpathcurveto{\pgfqpoint{2.016539in}{1.941087in}}{\pgfqpoint{2.013267in}{1.933187in}}{\pgfqpoint{2.013267in}{1.924951in}}%
\pgfpathcurveto{\pgfqpoint{2.013267in}{1.916715in}}{\pgfqpoint{2.016539in}{1.908815in}}{\pgfqpoint{2.022363in}{1.902991in}}%
\pgfpathcurveto{\pgfqpoint{2.028187in}{1.897167in}}{\pgfqpoint{2.036087in}{1.893894in}}{\pgfqpoint{2.044323in}{1.893894in}}%
\pgfpathclose%
\pgfusepath{stroke,fill}%
\end{pgfscope}%
\begin{pgfscope}%
\pgfpathrectangle{\pgfqpoint{0.100000in}{0.220728in}}{\pgfqpoint{3.696000in}{3.696000in}}%
\pgfusepath{clip}%
\pgfsetbuttcap%
\pgfsetroundjoin%
\definecolor{currentfill}{rgb}{0.121569,0.466667,0.705882}%
\pgfsetfillcolor{currentfill}%
\pgfsetfillopacity{0.486797}%
\pgfsetlinewidth{1.003750pt}%
\definecolor{currentstroke}{rgb}{0.121569,0.466667,0.705882}%
\pgfsetstrokecolor{currentstroke}%
\pgfsetstrokeopacity{0.486797}%
\pgfsetdash{}{0pt}%
\pgfpathmoveto{\pgfqpoint{1.376489in}{1.775518in}}%
\pgfpathcurveto{\pgfqpoint{1.384725in}{1.775518in}}{\pgfqpoint{1.392625in}{1.778791in}}{\pgfqpoint{1.398449in}{1.784615in}}%
\pgfpathcurveto{\pgfqpoint{1.404273in}{1.790439in}}{\pgfqpoint{1.407545in}{1.798339in}}{\pgfqpoint{1.407545in}{1.806575in}}%
\pgfpathcurveto{\pgfqpoint{1.407545in}{1.814811in}}{\pgfqpoint{1.404273in}{1.822711in}}{\pgfqpoint{1.398449in}{1.828535in}}%
\pgfpathcurveto{\pgfqpoint{1.392625in}{1.834359in}}{\pgfqpoint{1.384725in}{1.837631in}}{\pgfqpoint{1.376489in}{1.837631in}}%
\pgfpathcurveto{\pgfqpoint{1.368252in}{1.837631in}}{\pgfqpoint{1.360352in}{1.834359in}}{\pgfqpoint{1.354528in}{1.828535in}}%
\pgfpathcurveto{\pgfqpoint{1.348705in}{1.822711in}}{\pgfqpoint{1.345432in}{1.814811in}}{\pgfqpoint{1.345432in}{1.806575in}}%
\pgfpathcurveto{\pgfqpoint{1.345432in}{1.798339in}}{\pgfqpoint{1.348705in}{1.790439in}}{\pgfqpoint{1.354528in}{1.784615in}}%
\pgfpathcurveto{\pgfqpoint{1.360352in}{1.778791in}}{\pgfqpoint{1.368252in}{1.775518in}}{\pgfqpoint{1.376489in}{1.775518in}}%
\pgfpathclose%
\pgfusepath{stroke,fill}%
\end{pgfscope}%
\begin{pgfscope}%
\pgfpathrectangle{\pgfqpoint{0.100000in}{0.220728in}}{\pgfqpoint{3.696000in}{3.696000in}}%
\pgfusepath{clip}%
\pgfsetbuttcap%
\pgfsetroundjoin%
\definecolor{currentfill}{rgb}{0.121569,0.466667,0.705882}%
\pgfsetfillcolor{currentfill}%
\pgfsetfillopacity{0.488233}%
\pgfsetlinewidth{1.003750pt}%
\definecolor{currentstroke}{rgb}{0.121569,0.466667,0.705882}%
\pgfsetstrokecolor{currentstroke}%
\pgfsetstrokeopacity{0.488233}%
\pgfsetdash{}{0pt}%
\pgfpathmoveto{\pgfqpoint{1.371037in}{1.771899in}}%
\pgfpathcurveto{\pgfqpoint{1.379274in}{1.771899in}}{\pgfqpoint{1.387174in}{1.775171in}}{\pgfqpoint{1.392998in}{1.780995in}}%
\pgfpathcurveto{\pgfqpoint{1.398822in}{1.786819in}}{\pgfqpoint{1.402094in}{1.794719in}}{\pgfqpoint{1.402094in}{1.802955in}}%
\pgfpathcurveto{\pgfqpoint{1.402094in}{1.811191in}}{\pgfqpoint{1.398822in}{1.819091in}}{\pgfqpoint{1.392998in}{1.824915in}}%
\pgfpathcurveto{\pgfqpoint{1.387174in}{1.830739in}}{\pgfqpoint{1.379274in}{1.834012in}}{\pgfqpoint{1.371037in}{1.834012in}}%
\pgfpathcurveto{\pgfqpoint{1.362801in}{1.834012in}}{\pgfqpoint{1.354901in}{1.830739in}}{\pgfqpoint{1.349077in}{1.824915in}}%
\pgfpathcurveto{\pgfqpoint{1.343253in}{1.819091in}}{\pgfqpoint{1.339981in}{1.811191in}}{\pgfqpoint{1.339981in}{1.802955in}}%
\pgfpathcurveto{\pgfqpoint{1.339981in}{1.794719in}}{\pgfqpoint{1.343253in}{1.786819in}}{\pgfqpoint{1.349077in}{1.780995in}}%
\pgfpathcurveto{\pgfqpoint{1.354901in}{1.775171in}}{\pgfqpoint{1.362801in}{1.771899in}}{\pgfqpoint{1.371037in}{1.771899in}}%
\pgfpathclose%
\pgfusepath{stroke,fill}%
\end{pgfscope}%
\begin{pgfscope}%
\pgfpathrectangle{\pgfqpoint{0.100000in}{0.220728in}}{\pgfqpoint{3.696000in}{3.696000in}}%
\pgfusepath{clip}%
\pgfsetbuttcap%
\pgfsetroundjoin%
\definecolor{currentfill}{rgb}{0.121569,0.466667,0.705882}%
\pgfsetfillcolor{currentfill}%
\pgfsetfillopacity{0.489123}%
\pgfsetlinewidth{1.003750pt}%
\definecolor{currentstroke}{rgb}{0.121569,0.466667,0.705882}%
\pgfsetstrokecolor{currentstroke}%
\pgfsetstrokeopacity{0.489123}%
\pgfsetdash{}{0pt}%
\pgfpathmoveto{\pgfqpoint{2.044834in}{1.894449in}}%
\pgfpathcurveto{\pgfqpoint{2.053070in}{1.894449in}}{\pgfqpoint{2.060970in}{1.897722in}}{\pgfqpoint{2.066794in}{1.903545in}}%
\pgfpathcurveto{\pgfqpoint{2.072618in}{1.909369in}}{\pgfqpoint{2.075890in}{1.917269in}}{\pgfqpoint{2.075890in}{1.925506in}}%
\pgfpathcurveto{\pgfqpoint{2.075890in}{1.933742in}}{\pgfqpoint{2.072618in}{1.941642in}}{\pgfqpoint{2.066794in}{1.947466in}}%
\pgfpathcurveto{\pgfqpoint{2.060970in}{1.953290in}}{\pgfqpoint{2.053070in}{1.956562in}}{\pgfqpoint{2.044834in}{1.956562in}}%
\pgfpathcurveto{\pgfqpoint{2.036598in}{1.956562in}}{\pgfqpoint{2.028697in}{1.953290in}}{\pgfqpoint{2.022874in}{1.947466in}}%
\pgfpathcurveto{\pgfqpoint{2.017050in}{1.941642in}}{\pgfqpoint{2.013777in}{1.933742in}}{\pgfqpoint{2.013777in}{1.925506in}}%
\pgfpathcurveto{\pgfqpoint{2.013777in}{1.917269in}}{\pgfqpoint{2.017050in}{1.909369in}}{\pgfqpoint{2.022874in}{1.903545in}}%
\pgfpathcurveto{\pgfqpoint{2.028697in}{1.897722in}}{\pgfqpoint{2.036598in}{1.894449in}}{\pgfqpoint{2.044834in}{1.894449in}}%
\pgfpathclose%
\pgfusepath{stroke,fill}%
\end{pgfscope}%
\begin{pgfscope}%
\pgfpathrectangle{\pgfqpoint{0.100000in}{0.220728in}}{\pgfqpoint{3.696000in}{3.696000in}}%
\pgfusepath{clip}%
\pgfsetbuttcap%
\pgfsetroundjoin%
\definecolor{currentfill}{rgb}{0.121569,0.466667,0.705882}%
\pgfsetfillcolor{currentfill}%
\pgfsetfillopacity{0.489201}%
\pgfsetlinewidth{1.003750pt}%
\definecolor{currentstroke}{rgb}{0.121569,0.466667,0.705882}%
\pgfsetstrokecolor{currentstroke}%
\pgfsetstrokeopacity{0.489201}%
\pgfsetdash{}{0pt}%
\pgfpathmoveto{\pgfqpoint{1.367126in}{1.768209in}}%
\pgfpathcurveto{\pgfqpoint{1.375362in}{1.768209in}}{\pgfqpoint{1.383263in}{1.771481in}}{\pgfqpoint{1.389086in}{1.777305in}}%
\pgfpathcurveto{\pgfqpoint{1.394910in}{1.783129in}}{\pgfqpoint{1.398183in}{1.791029in}}{\pgfqpoint{1.398183in}{1.799265in}}%
\pgfpathcurveto{\pgfqpoint{1.398183in}{1.807502in}}{\pgfqpoint{1.394910in}{1.815402in}}{\pgfqpoint{1.389086in}{1.821226in}}%
\pgfpathcurveto{\pgfqpoint{1.383263in}{1.827050in}}{\pgfqpoint{1.375362in}{1.830322in}}{\pgfqpoint{1.367126in}{1.830322in}}%
\pgfpathcurveto{\pgfqpoint{1.358890in}{1.830322in}}{\pgfqpoint{1.350990in}{1.827050in}}{\pgfqpoint{1.345166in}{1.821226in}}%
\pgfpathcurveto{\pgfqpoint{1.339342in}{1.815402in}}{\pgfqpoint{1.336070in}{1.807502in}}{\pgfqpoint{1.336070in}{1.799265in}}%
\pgfpathcurveto{\pgfqpoint{1.336070in}{1.791029in}}{\pgfqpoint{1.339342in}{1.783129in}}{\pgfqpoint{1.345166in}{1.777305in}}%
\pgfpathcurveto{\pgfqpoint{1.350990in}{1.771481in}}{\pgfqpoint{1.358890in}{1.768209in}}{\pgfqpoint{1.367126in}{1.768209in}}%
\pgfpathclose%
\pgfusepath{stroke,fill}%
\end{pgfscope}%
\begin{pgfscope}%
\pgfpathrectangle{\pgfqpoint{0.100000in}{0.220728in}}{\pgfqpoint{3.696000in}{3.696000in}}%
\pgfusepath{clip}%
\pgfsetbuttcap%
\pgfsetroundjoin%
\definecolor{currentfill}{rgb}{0.121569,0.466667,0.705882}%
\pgfsetfillcolor{currentfill}%
\pgfsetfillopacity{0.490319}%
\pgfsetlinewidth{1.003750pt}%
\definecolor{currentstroke}{rgb}{0.121569,0.466667,0.705882}%
\pgfsetstrokecolor{currentstroke}%
\pgfsetstrokeopacity{0.490319}%
\pgfsetdash{}{0pt}%
\pgfpathmoveto{\pgfqpoint{1.364436in}{1.767097in}}%
\pgfpathcurveto{\pgfqpoint{1.372673in}{1.767097in}}{\pgfqpoint{1.380573in}{1.770369in}}{\pgfqpoint{1.386397in}{1.776193in}}%
\pgfpathcurveto{\pgfqpoint{1.392221in}{1.782017in}}{\pgfqpoint{1.395493in}{1.789917in}}{\pgfqpoint{1.395493in}{1.798153in}}%
\pgfpathcurveto{\pgfqpoint{1.395493in}{1.806389in}}{\pgfqpoint{1.392221in}{1.814289in}}{\pgfqpoint{1.386397in}{1.820113in}}%
\pgfpathcurveto{\pgfqpoint{1.380573in}{1.825937in}}{\pgfqpoint{1.372673in}{1.829210in}}{\pgfqpoint{1.364436in}{1.829210in}}%
\pgfpathcurveto{\pgfqpoint{1.356200in}{1.829210in}}{\pgfqpoint{1.348300in}{1.825937in}}{\pgfqpoint{1.342476in}{1.820113in}}%
\pgfpathcurveto{\pgfqpoint{1.336652in}{1.814289in}}{\pgfqpoint{1.333380in}{1.806389in}}{\pgfqpoint{1.333380in}{1.798153in}}%
\pgfpathcurveto{\pgfqpoint{1.333380in}{1.789917in}}{\pgfqpoint{1.336652in}{1.782017in}}{\pgfqpoint{1.342476in}{1.776193in}}%
\pgfpathcurveto{\pgfqpoint{1.348300in}{1.770369in}}{\pgfqpoint{1.356200in}{1.767097in}}{\pgfqpoint{1.364436in}{1.767097in}}%
\pgfpathclose%
\pgfusepath{stroke,fill}%
\end{pgfscope}%
\begin{pgfscope}%
\pgfpathrectangle{\pgfqpoint{0.100000in}{0.220728in}}{\pgfqpoint{3.696000in}{3.696000in}}%
\pgfusepath{clip}%
\pgfsetbuttcap%
\pgfsetroundjoin%
\definecolor{currentfill}{rgb}{0.121569,0.466667,0.705882}%
\pgfsetfillcolor{currentfill}%
\pgfsetfillopacity{0.490951}%
\pgfsetlinewidth{1.003750pt}%
\definecolor{currentstroke}{rgb}{0.121569,0.466667,0.705882}%
\pgfsetstrokecolor{currentstroke}%
\pgfsetstrokeopacity{0.490951}%
\pgfsetdash{}{0pt}%
\pgfpathmoveto{\pgfqpoint{1.361612in}{1.764818in}}%
\pgfpathcurveto{\pgfqpoint{1.369848in}{1.764818in}}{\pgfqpoint{1.377748in}{1.768091in}}{\pgfqpoint{1.383572in}{1.773914in}}%
\pgfpathcurveto{\pgfqpoint{1.389396in}{1.779738in}}{\pgfqpoint{1.392668in}{1.787638in}}{\pgfqpoint{1.392668in}{1.795875in}}%
\pgfpathcurveto{\pgfqpoint{1.392668in}{1.804111in}}{\pgfqpoint{1.389396in}{1.812011in}}{\pgfqpoint{1.383572in}{1.817835in}}%
\pgfpathcurveto{\pgfqpoint{1.377748in}{1.823659in}}{\pgfqpoint{1.369848in}{1.826931in}}{\pgfqpoint{1.361612in}{1.826931in}}%
\pgfpathcurveto{\pgfqpoint{1.353376in}{1.826931in}}{\pgfqpoint{1.345476in}{1.823659in}}{\pgfqpoint{1.339652in}{1.817835in}}%
\pgfpathcurveto{\pgfqpoint{1.333828in}{1.812011in}}{\pgfqpoint{1.330555in}{1.804111in}}{\pgfqpoint{1.330555in}{1.795875in}}%
\pgfpathcurveto{\pgfqpoint{1.330555in}{1.787638in}}{\pgfqpoint{1.333828in}{1.779738in}}{\pgfqpoint{1.339652in}{1.773914in}}%
\pgfpathcurveto{\pgfqpoint{1.345476in}{1.768091in}}{\pgfqpoint{1.353376in}{1.764818in}}{\pgfqpoint{1.361612in}{1.764818in}}%
\pgfpathclose%
\pgfusepath{stroke,fill}%
\end{pgfscope}%
\begin{pgfscope}%
\pgfpathrectangle{\pgfqpoint{0.100000in}{0.220728in}}{\pgfqpoint{3.696000in}{3.696000in}}%
\pgfusepath{clip}%
\pgfsetbuttcap%
\pgfsetroundjoin%
\definecolor{currentfill}{rgb}{0.121569,0.466667,0.705882}%
\pgfsetfillcolor{currentfill}%
\pgfsetfillopacity{0.492255}%
\pgfsetlinewidth{1.003750pt}%
\definecolor{currentstroke}{rgb}{0.121569,0.466667,0.705882}%
\pgfsetstrokecolor{currentstroke}%
\pgfsetstrokeopacity{0.492255}%
\pgfsetdash{}{0pt}%
\pgfpathmoveto{\pgfqpoint{1.356464in}{1.761695in}}%
\pgfpathcurveto{\pgfqpoint{1.364701in}{1.761695in}}{\pgfqpoint{1.372601in}{1.764967in}}{\pgfqpoint{1.378425in}{1.770791in}}%
\pgfpathcurveto{\pgfqpoint{1.384249in}{1.776615in}}{\pgfqpoint{1.387521in}{1.784515in}}{\pgfqpoint{1.387521in}{1.792751in}}%
\pgfpathcurveto{\pgfqpoint{1.387521in}{1.800988in}}{\pgfqpoint{1.384249in}{1.808888in}}{\pgfqpoint{1.378425in}{1.814712in}}%
\pgfpathcurveto{\pgfqpoint{1.372601in}{1.820536in}}{\pgfqpoint{1.364701in}{1.823808in}}{\pgfqpoint{1.356464in}{1.823808in}}%
\pgfpathcurveto{\pgfqpoint{1.348228in}{1.823808in}}{\pgfqpoint{1.340328in}{1.820536in}}{\pgfqpoint{1.334504in}{1.814712in}}%
\pgfpathcurveto{\pgfqpoint{1.328680in}{1.808888in}}{\pgfqpoint{1.325408in}{1.800988in}}{\pgfqpoint{1.325408in}{1.792751in}}%
\pgfpathcurveto{\pgfqpoint{1.325408in}{1.784515in}}{\pgfqpoint{1.328680in}{1.776615in}}{\pgfqpoint{1.334504in}{1.770791in}}%
\pgfpathcurveto{\pgfqpoint{1.340328in}{1.764967in}}{\pgfqpoint{1.348228in}{1.761695in}}{\pgfqpoint{1.356464in}{1.761695in}}%
\pgfpathclose%
\pgfusepath{stroke,fill}%
\end{pgfscope}%
\begin{pgfscope}%
\pgfpathrectangle{\pgfqpoint{0.100000in}{0.220728in}}{\pgfqpoint{3.696000in}{3.696000in}}%
\pgfusepath{clip}%
\pgfsetbuttcap%
\pgfsetroundjoin%
\definecolor{currentfill}{rgb}{0.121569,0.466667,0.705882}%
\pgfsetfillcolor{currentfill}%
\pgfsetfillopacity{0.492328}%
\pgfsetlinewidth{1.003750pt}%
\definecolor{currentstroke}{rgb}{0.121569,0.466667,0.705882}%
\pgfsetstrokecolor{currentstroke}%
\pgfsetstrokeopacity{0.492328}%
\pgfsetdash{}{0pt}%
\pgfpathmoveto{\pgfqpoint{2.047003in}{1.889663in}}%
\pgfpathcurveto{\pgfqpoint{2.055239in}{1.889663in}}{\pgfqpoint{2.063139in}{1.892935in}}{\pgfqpoint{2.068963in}{1.898759in}}%
\pgfpathcurveto{\pgfqpoint{2.074787in}{1.904583in}}{\pgfqpoint{2.078059in}{1.912483in}}{\pgfqpoint{2.078059in}{1.920720in}}%
\pgfpathcurveto{\pgfqpoint{2.078059in}{1.928956in}}{\pgfqpoint{2.074787in}{1.936856in}}{\pgfqpoint{2.068963in}{1.942680in}}%
\pgfpathcurveto{\pgfqpoint{2.063139in}{1.948504in}}{\pgfqpoint{2.055239in}{1.951776in}}{\pgfqpoint{2.047003in}{1.951776in}}%
\pgfpathcurveto{\pgfqpoint{2.038766in}{1.951776in}}{\pgfqpoint{2.030866in}{1.948504in}}{\pgfqpoint{2.025042in}{1.942680in}}%
\pgfpathcurveto{\pgfqpoint{2.019218in}{1.936856in}}{\pgfqpoint{2.015946in}{1.928956in}}{\pgfqpoint{2.015946in}{1.920720in}}%
\pgfpathcurveto{\pgfqpoint{2.015946in}{1.912483in}}{\pgfqpoint{2.019218in}{1.904583in}}{\pgfqpoint{2.025042in}{1.898759in}}%
\pgfpathcurveto{\pgfqpoint{2.030866in}{1.892935in}}{\pgfqpoint{2.038766in}{1.889663in}}{\pgfqpoint{2.047003in}{1.889663in}}%
\pgfpathclose%
\pgfusepath{stroke,fill}%
\end{pgfscope}%
\begin{pgfscope}%
\pgfpathrectangle{\pgfqpoint{0.100000in}{0.220728in}}{\pgfqpoint{3.696000in}{3.696000in}}%
\pgfusepath{clip}%
\pgfsetbuttcap%
\pgfsetroundjoin%
\definecolor{currentfill}{rgb}{0.121569,0.466667,0.705882}%
\pgfsetfillcolor{currentfill}%
\pgfsetfillopacity{0.495464}%
\pgfsetlinewidth{1.003750pt}%
\definecolor{currentstroke}{rgb}{0.121569,0.466667,0.705882}%
\pgfsetstrokecolor{currentstroke}%
\pgfsetstrokeopacity{0.495464}%
\pgfsetdash{}{0pt}%
\pgfpathmoveto{\pgfqpoint{1.348715in}{1.759015in}}%
\pgfpathcurveto{\pgfqpoint{1.356951in}{1.759015in}}{\pgfqpoint{1.364851in}{1.762287in}}{\pgfqpoint{1.370675in}{1.768111in}}%
\pgfpathcurveto{\pgfqpoint{1.376499in}{1.773935in}}{\pgfqpoint{1.379771in}{1.781835in}}{\pgfqpoint{1.379771in}{1.790071in}}%
\pgfpathcurveto{\pgfqpoint{1.379771in}{1.798308in}}{\pgfqpoint{1.376499in}{1.806208in}}{\pgfqpoint{1.370675in}{1.812032in}}%
\pgfpathcurveto{\pgfqpoint{1.364851in}{1.817856in}}{\pgfqpoint{1.356951in}{1.821128in}}{\pgfqpoint{1.348715in}{1.821128in}}%
\pgfpathcurveto{\pgfqpoint{1.340478in}{1.821128in}}{\pgfqpoint{1.332578in}{1.817856in}}{\pgfqpoint{1.326754in}{1.812032in}}%
\pgfpathcurveto{\pgfqpoint{1.320930in}{1.806208in}}{\pgfqpoint{1.317658in}{1.798308in}}{\pgfqpoint{1.317658in}{1.790071in}}%
\pgfpathcurveto{\pgfqpoint{1.317658in}{1.781835in}}{\pgfqpoint{1.320930in}{1.773935in}}{\pgfqpoint{1.326754in}{1.768111in}}%
\pgfpathcurveto{\pgfqpoint{1.332578in}{1.762287in}}{\pgfqpoint{1.340478in}{1.759015in}}{\pgfqpoint{1.348715in}{1.759015in}}%
\pgfpathclose%
\pgfusepath{stroke,fill}%
\end{pgfscope}%
\begin{pgfscope}%
\pgfpathrectangle{\pgfqpoint{0.100000in}{0.220728in}}{\pgfqpoint{3.696000in}{3.696000in}}%
\pgfusepath{clip}%
\pgfsetbuttcap%
\pgfsetroundjoin%
\definecolor{currentfill}{rgb}{0.121569,0.466667,0.705882}%
\pgfsetfillcolor{currentfill}%
\pgfsetfillopacity{0.496141}%
\pgfsetlinewidth{1.003750pt}%
\definecolor{currentstroke}{rgb}{0.121569,0.466667,0.705882}%
\pgfsetstrokecolor{currentstroke}%
\pgfsetstrokeopacity{0.496141}%
\pgfsetdash{}{0pt}%
\pgfpathmoveto{\pgfqpoint{2.049123in}{1.885688in}}%
\pgfpathcurveto{\pgfqpoint{2.057360in}{1.885688in}}{\pgfqpoint{2.065260in}{1.888960in}}{\pgfqpoint{2.071084in}{1.894784in}}%
\pgfpathcurveto{\pgfqpoint{2.076908in}{1.900608in}}{\pgfqpoint{2.080180in}{1.908508in}}{\pgfqpoint{2.080180in}{1.916744in}}%
\pgfpathcurveto{\pgfqpoint{2.080180in}{1.924981in}}{\pgfqpoint{2.076908in}{1.932881in}}{\pgfqpoint{2.071084in}{1.938705in}}%
\pgfpathcurveto{\pgfqpoint{2.065260in}{1.944529in}}{\pgfqpoint{2.057360in}{1.947801in}}{\pgfqpoint{2.049123in}{1.947801in}}%
\pgfpathcurveto{\pgfqpoint{2.040887in}{1.947801in}}{\pgfqpoint{2.032987in}{1.944529in}}{\pgfqpoint{2.027163in}{1.938705in}}%
\pgfpathcurveto{\pgfqpoint{2.021339in}{1.932881in}}{\pgfqpoint{2.018067in}{1.924981in}}{\pgfqpoint{2.018067in}{1.916744in}}%
\pgfpathcurveto{\pgfqpoint{2.018067in}{1.908508in}}{\pgfqpoint{2.021339in}{1.900608in}}{\pgfqpoint{2.027163in}{1.894784in}}%
\pgfpathcurveto{\pgfqpoint{2.032987in}{1.888960in}}{\pgfqpoint{2.040887in}{1.885688in}}{\pgfqpoint{2.049123in}{1.885688in}}%
\pgfpathclose%
\pgfusepath{stroke,fill}%
\end{pgfscope}%
\begin{pgfscope}%
\pgfpathrectangle{\pgfqpoint{0.100000in}{0.220728in}}{\pgfqpoint{3.696000in}{3.696000in}}%
\pgfusepath{clip}%
\pgfsetbuttcap%
\pgfsetroundjoin%
\definecolor{currentfill}{rgb}{0.121569,0.466667,0.705882}%
\pgfsetfillcolor{currentfill}%
\pgfsetfillopacity{0.497070}%
\pgfsetlinewidth{1.003750pt}%
\definecolor{currentstroke}{rgb}{0.121569,0.466667,0.705882}%
\pgfsetstrokecolor{currentstroke}%
\pgfsetstrokeopacity{0.497070}%
\pgfsetdash{}{0pt}%
\pgfpathmoveto{\pgfqpoint{1.342213in}{1.756816in}}%
\pgfpathcurveto{\pgfqpoint{1.350450in}{1.756816in}}{\pgfqpoint{1.358350in}{1.760088in}}{\pgfqpoint{1.364174in}{1.765912in}}%
\pgfpathcurveto{\pgfqpoint{1.369997in}{1.771736in}}{\pgfqpoint{1.373270in}{1.779636in}}{\pgfqpoint{1.373270in}{1.787872in}}%
\pgfpathcurveto{\pgfqpoint{1.373270in}{1.796109in}}{\pgfqpoint{1.369997in}{1.804009in}}{\pgfqpoint{1.364174in}{1.809833in}}%
\pgfpathcurveto{\pgfqpoint{1.358350in}{1.815657in}}{\pgfqpoint{1.350450in}{1.818929in}}{\pgfqpoint{1.342213in}{1.818929in}}%
\pgfpathcurveto{\pgfqpoint{1.333977in}{1.818929in}}{\pgfqpoint{1.326077in}{1.815657in}}{\pgfqpoint{1.320253in}{1.809833in}}%
\pgfpathcurveto{\pgfqpoint{1.314429in}{1.804009in}}{\pgfqpoint{1.311157in}{1.796109in}}{\pgfqpoint{1.311157in}{1.787872in}}%
\pgfpathcurveto{\pgfqpoint{1.311157in}{1.779636in}}{\pgfqpoint{1.314429in}{1.771736in}}{\pgfqpoint{1.320253in}{1.765912in}}%
\pgfpathcurveto{\pgfqpoint{1.326077in}{1.760088in}}{\pgfqpoint{1.333977in}{1.756816in}}{\pgfqpoint{1.342213in}{1.756816in}}%
\pgfpathclose%
\pgfusepath{stroke,fill}%
\end{pgfscope}%
\begin{pgfscope}%
\pgfpathrectangle{\pgfqpoint{0.100000in}{0.220728in}}{\pgfqpoint{3.696000in}{3.696000in}}%
\pgfusepath{clip}%
\pgfsetbuttcap%
\pgfsetroundjoin%
\definecolor{currentfill}{rgb}{0.121569,0.466667,0.705882}%
\pgfsetfillcolor{currentfill}%
\pgfsetfillopacity{0.498111}%
\pgfsetlinewidth{1.003750pt}%
\definecolor{currentstroke}{rgb}{0.121569,0.466667,0.705882}%
\pgfsetstrokecolor{currentstroke}%
\pgfsetstrokeopacity{0.498111}%
\pgfsetdash{}{0pt}%
\pgfpathmoveto{\pgfqpoint{1.337973in}{1.752602in}}%
\pgfpathcurveto{\pgfqpoint{1.346210in}{1.752602in}}{\pgfqpoint{1.354110in}{1.755875in}}{\pgfqpoint{1.359934in}{1.761699in}}%
\pgfpathcurveto{\pgfqpoint{1.365758in}{1.767523in}}{\pgfqpoint{1.369030in}{1.775423in}}{\pgfqpoint{1.369030in}{1.783659in}}%
\pgfpathcurveto{\pgfqpoint{1.369030in}{1.791895in}}{\pgfqpoint{1.365758in}{1.799795in}}{\pgfqpoint{1.359934in}{1.805619in}}%
\pgfpathcurveto{\pgfqpoint{1.354110in}{1.811443in}}{\pgfqpoint{1.346210in}{1.814715in}}{\pgfqpoint{1.337973in}{1.814715in}}%
\pgfpathcurveto{\pgfqpoint{1.329737in}{1.814715in}}{\pgfqpoint{1.321837in}{1.811443in}}{\pgfqpoint{1.316013in}{1.805619in}}%
\pgfpathcurveto{\pgfqpoint{1.310189in}{1.799795in}}{\pgfqpoint{1.306917in}{1.791895in}}{\pgfqpoint{1.306917in}{1.783659in}}%
\pgfpathcurveto{\pgfqpoint{1.306917in}{1.775423in}}{\pgfqpoint{1.310189in}{1.767523in}}{\pgfqpoint{1.316013in}{1.761699in}}%
\pgfpathcurveto{\pgfqpoint{1.321837in}{1.755875in}}{\pgfqpoint{1.329737in}{1.752602in}}{\pgfqpoint{1.337973in}{1.752602in}}%
\pgfpathclose%
\pgfusepath{stroke,fill}%
\end{pgfscope}%
\begin{pgfscope}%
\pgfpathrectangle{\pgfqpoint{0.100000in}{0.220728in}}{\pgfqpoint{3.696000in}{3.696000in}}%
\pgfusepath{clip}%
\pgfsetbuttcap%
\pgfsetroundjoin%
\definecolor{currentfill}{rgb}{0.121569,0.466667,0.705882}%
\pgfsetfillcolor{currentfill}%
\pgfsetfillopacity{0.500391}%
\pgfsetlinewidth{1.003750pt}%
\definecolor{currentstroke}{rgb}{0.121569,0.466667,0.705882}%
\pgfsetstrokecolor{currentstroke}%
\pgfsetstrokeopacity{0.500391}%
\pgfsetdash{}{0pt}%
\pgfpathmoveto{\pgfqpoint{2.049857in}{1.880448in}}%
\pgfpathcurveto{\pgfqpoint{2.058094in}{1.880448in}}{\pgfqpoint{2.065994in}{1.883721in}}{\pgfqpoint{2.071818in}{1.889545in}}%
\pgfpathcurveto{\pgfqpoint{2.077642in}{1.895369in}}{\pgfqpoint{2.080914in}{1.903269in}}{\pgfqpoint{2.080914in}{1.911505in}}%
\pgfpathcurveto{\pgfqpoint{2.080914in}{1.919741in}}{\pgfqpoint{2.077642in}{1.927641in}}{\pgfqpoint{2.071818in}{1.933465in}}%
\pgfpathcurveto{\pgfqpoint{2.065994in}{1.939289in}}{\pgfqpoint{2.058094in}{1.942561in}}{\pgfqpoint{2.049857in}{1.942561in}}%
\pgfpathcurveto{\pgfqpoint{2.041621in}{1.942561in}}{\pgfqpoint{2.033721in}{1.939289in}}{\pgfqpoint{2.027897in}{1.933465in}}%
\pgfpathcurveto{\pgfqpoint{2.022073in}{1.927641in}}{\pgfqpoint{2.018801in}{1.919741in}}{\pgfqpoint{2.018801in}{1.911505in}}%
\pgfpathcurveto{\pgfqpoint{2.018801in}{1.903269in}}{\pgfqpoint{2.022073in}{1.895369in}}{\pgfqpoint{2.027897in}{1.889545in}}%
\pgfpathcurveto{\pgfqpoint{2.033721in}{1.883721in}}{\pgfqpoint{2.041621in}{1.880448in}}{\pgfqpoint{2.049857in}{1.880448in}}%
\pgfpathclose%
\pgfusepath{stroke,fill}%
\end{pgfscope}%
\begin{pgfscope}%
\pgfpathrectangle{\pgfqpoint{0.100000in}{0.220728in}}{\pgfqpoint{3.696000in}{3.696000in}}%
\pgfusepath{clip}%
\pgfsetbuttcap%
\pgfsetroundjoin%
\definecolor{currentfill}{rgb}{0.121569,0.466667,0.705882}%
\pgfsetfillcolor{currentfill}%
\pgfsetfillopacity{0.501188}%
\pgfsetlinewidth{1.003750pt}%
\definecolor{currentstroke}{rgb}{0.121569,0.466667,0.705882}%
\pgfsetstrokecolor{currentstroke}%
\pgfsetstrokeopacity{0.501188}%
\pgfsetdash{}{0pt}%
\pgfpathmoveto{\pgfqpoint{1.331351in}{1.751183in}}%
\pgfpathcurveto{\pgfqpoint{1.339587in}{1.751183in}}{\pgfqpoint{1.347487in}{1.754455in}}{\pgfqpoint{1.353311in}{1.760279in}}%
\pgfpathcurveto{\pgfqpoint{1.359135in}{1.766103in}}{\pgfqpoint{1.362407in}{1.774003in}}{\pgfqpoint{1.362407in}{1.782240in}}%
\pgfpathcurveto{\pgfqpoint{1.362407in}{1.790476in}}{\pgfqpoint{1.359135in}{1.798376in}}{\pgfqpoint{1.353311in}{1.804200in}}%
\pgfpathcurveto{\pgfqpoint{1.347487in}{1.810024in}}{\pgfqpoint{1.339587in}{1.813296in}}{\pgfqpoint{1.331351in}{1.813296in}}%
\pgfpathcurveto{\pgfqpoint{1.323115in}{1.813296in}}{\pgfqpoint{1.315214in}{1.810024in}}{\pgfqpoint{1.309391in}{1.804200in}}%
\pgfpathcurveto{\pgfqpoint{1.303567in}{1.798376in}}{\pgfqpoint{1.300294in}{1.790476in}}{\pgfqpoint{1.300294in}{1.782240in}}%
\pgfpathcurveto{\pgfqpoint{1.300294in}{1.774003in}}{\pgfqpoint{1.303567in}{1.766103in}}{\pgfqpoint{1.309391in}{1.760279in}}%
\pgfpathcurveto{\pgfqpoint{1.315214in}{1.754455in}}{\pgfqpoint{1.323115in}{1.751183in}}{\pgfqpoint{1.331351in}{1.751183in}}%
\pgfpathclose%
\pgfusepath{stroke,fill}%
\end{pgfscope}%
\begin{pgfscope}%
\pgfpathrectangle{\pgfqpoint{0.100000in}{0.220728in}}{\pgfqpoint{3.696000in}{3.696000in}}%
\pgfusepath{clip}%
\pgfsetbuttcap%
\pgfsetroundjoin%
\definecolor{currentfill}{rgb}{0.121569,0.466667,0.705882}%
\pgfsetfillcolor{currentfill}%
\pgfsetfillopacity{0.502594}%
\pgfsetlinewidth{1.003750pt}%
\definecolor{currentstroke}{rgb}{0.121569,0.466667,0.705882}%
\pgfsetstrokecolor{currentstroke}%
\pgfsetstrokeopacity{0.502594}%
\pgfsetdash{}{0pt}%
\pgfpathmoveto{\pgfqpoint{1.325218in}{1.747994in}}%
\pgfpathcurveto{\pgfqpoint{1.333455in}{1.747994in}}{\pgfqpoint{1.341355in}{1.751266in}}{\pgfqpoint{1.347179in}{1.757090in}}%
\pgfpathcurveto{\pgfqpoint{1.353003in}{1.762914in}}{\pgfqpoint{1.356275in}{1.770814in}}{\pgfqpoint{1.356275in}{1.779050in}}%
\pgfpathcurveto{\pgfqpoint{1.356275in}{1.787286in}}{\pgfqpoint{1.353003in}{1.795187in}}{\pgfqpoint{1.347179in}{1.801010in}}%
\pgfpathcurveto{\pgfqpoint{1.341355in}{1.806834in}}{\pgfqpoint{1.333455in}{1.810107in}}{\pgfqpoint{1.325218in}{1.810107in}}%
\pgfpathcurveto{\pgfqpoint{1.316982in}{1.810107in}}{\pgfqpoint{1.309082in}{1.806834in}}{\pgfqpoint{1.303258in}{1.801010in}}%
\pgfpathcurveto{\pgfqpoint{1.297434in}{1.795187in}}{\pgfqpoint{1.294162in}{1.787286in}}{\pgfqpoint{1.294162in}{1.779050in}}%
\pgfpathcurveto{\pgfqpoint{1.294162in}{1.770814in}}{\pgfqpoint{1.297434in}{1.762914in}}{\pgfqpoint{1.303258in}{1.757090in}}%
\pgfpathcurveto{\pgfqpoint{1.309082in}{1.751266in}}{\pgfqpoint{1.316982in}{1.747994in}}{\pgfqpoint{1.325218in}{1.747994in}}%
\pgfpathclose%
\pgfusepath{stroke,fill}%
\end{pgfscope}%
\begin{pgfscope}%
\pgfpathrectangle{\pgfqpoint{0.100000in}{0.220728in}}{\pgfqpoint{3.696000in}{3.696000in}}%
\pgfusepath{clip}%
\pgfsetbuttcap%
\pgfsetroundjoin%
\definecolor{currentfill}{rgb}{0.121569,0.466667,0.705882}%
\pgfsetfillcolor{currentfill}%
\pgfsetfillopacity{0.503388}%
\pgfsetlinewidth{1.003750pt}%
\definecolor{currentstroke}{rgb}{0.121569,0.466667,0.705882}%
\pgfsetstrokecolor{currentstroke}%
\pgfsetstrokeopacity{0.503388}%
\pgfsetdash{}{0pt}%
\pgfpathmoveto{\pgfqpoint{1.322089in}{1.745366in}}%
\pgfpathcurveto{\pgfqpoint{1.330326in}{1.745366in}}{\pgfqpoint{1.338226in}{1.748638in}}{\pgfqpoint{1.344050in}{1.754462in}}%
\pgfpathcurveto{\pgfqpoint{1.349874in}{1.760286in}}{\pgfqpoint{1.353146in}{1.768186in}}{\pgfqpoint{1.353146in}{1.776422in}}%
\pgfpathcurveto{\pgfqpoint{1.353146in}{1.784659in}}{\pgfqpoint{1.349874in}{1.792559in}}{\pgfqpoint{1.344050in}{1.798383in}}%
\pgfpathcurveto{\pgfqpoint{1.338226in}{1.804206in}}{\pgfqpoint{1.330326in}{1.807479in}}{\pgfqpoint{1.322089in}{1.807479in}}%
\pgfpathcurveto{\pgfqpoint{1.313853in}{1.807479in}}{\pgfqpoint{1.305953in}{1.804206in}}{\pgfqpoint{1.300129in}{1.798383in}}%
\pgfpathcurveto{\pgfqpoint{1.294305in}{1.792559in}}{\pgfqpoint{1.291033in}{1.784659in}}{\pgfqpoint{1.291033in}{1.776422in}}%
\pgfpathcurveto{\pgfqpoint{1.291033in}{1.768186in}}{\pgfqpoint{1.294305in}{1.760286in}}{\pgfqpoint{1.300129in}{1.754462in}}%
\pgfpathcurveto{\pgfqpoint{1.305953in}{1.748638in}}{\pgfqpoint{1.313853in}{1.745366in}}{\pgfqpoint{1.322089in}{1.745366in}}%
\pgfpathclose%
\pgfusepath{stroke,fill}%
\end{pgfscope}%
\begin{pgfscope}%
\pgfpathrectangle{\pgfqpoint{0.100000in}{0.220728in}}{\pgfqpoint{3.696000in}{3.696000in}}%
\pgfusepath{clip}%
\pgfsetbuttcap%
\pgfsetroundjoin%
\definecolor{currentfill}{rgb}{0.121569,0.466667,0.705882}%
\pgfsetfillcolor{currentfill}%
\pgfsetfillopacity{0.505373}%
\pgfsetlinewidth{1.003750pt}%
\definecolor{currentstroke}{rgb}{0.121569,0.466667,0.705882}%
\pgfsetstrokecolor{currentstroke}%
\pgfsetstrokeopacity{0.505373}%
\pgfsetdash{}{0pt}%
\pgfpathmoveto{\pgfqpoint{2.052617in}{1.874323in}}%
\pgfpathcurveto{\pgfqpoint{2.060853in}{1.874323in}}{\pgfqpoint{2.068753in}{1.877595in}}{\pgfqpoint{2.074577in}{1.883419in}}%
\pgfpathcurveto{\pgfqpoint{2.080401in}{1.889243in}}{\pgfqpoint{2.083673in}{1.897143in}}{\pgfqpoint{2.083673in}{1.905379in}}%
\pgfpathcurveto{\pgfqpoint{2.083673in}{1.913616in}}{\pgfqpoint{2.080401in}{1.921516in}}{\pgfqpoint{2.074577in}{1.927340in}}%
\pgfpathcurveto{\pgfqpoint{2.068753in}{1.933164in}}{\pgfqpoint{2.060853in}{1.936436in}}{\pgfqpoint{2.052617in}{1.936436in}}%
\pgfpathcurveto{\pgfqpoint{2.044380in}{1.936436in}}{\pgfqpoint{2.036480in}{1.933164in}}{\pgfqpoint{2.030656in}{1.927340in}}%
\pgfpathcurveto{\pgfqpoint{2.024832in}{1.921516in}}{\pgfqpoint{2.021560in}{1.913616in}}{\pgfqpoint{2.021560in}{1.905379in}}%
\pgfpathcurveto{\pgfqpoint{2.021560in}{1.897143in}}{\pgfqpoint{2.024832in}{1.889243in}}{\pgfqpoint{2.030656in}{1.883419in}}%
\pgfpathcurveto{\pgfqpoint{2.036480in}{1.877595in}}{\pgfqpoint{2.044380in}{1.874323in}}{\pgfqpoint{2.052617in}{1.874323in}}%
\pgfpathclose%
\pgfusepath{stroke,fill}%
\end{pgfscope}%
\begin{pgfscope}%
\pgfpathrectangle{\pgfqpoint{0.100000in}{0.220728in}}{\pgfqpoint{3.696000in}{3.696000in}}%
\pgfusepath{clip}%
\pgfsetbuttcap%
\pgfsetroundjoin%
\definecolor{currentfill}{rgb}{0.121569,0.466667,0.705882}%
\pgfsetfillcolor{currentfill}%
\pgfsetfillopacity{0.505394}%
\pgfsetlinewidth{1.003750pt}%
\definecolor{currentstroke}{rgb}{0.121569,0.466667,0.705882}%
\pgfsetstrokecolor{currentstroke}%
\pgfsetstrokeopacity{0.505394}%
\pgfsetdash{}{0pt}%
\pgfpathmoveto{\pgfqpoint{1.316653in}{1.743826in}}%
\pgfpathcurveto{\pgfqpoint{1.324889in}{1.743826in}}{\pgfqpoint{1.332789in}{1.747098in}}{\pgfqpoint{1.338613in}{1.752922in}}%
\pgfpathcurveto{\pgfqpoint{1.344437in}{1.758746in}}{\pgfqpoint{1.347709in}{1.766646in}}{\pgfqpoint{1.347709in}{1.774882in}}%
\pgfpathcurveto{\pgfqpoint{1.347709in}{1.783118in}}{\pgfqpoint{1.344437in}{1.791018in}}{\pgfqpoint{1.338613in}{1.796842in}}%
\pgfpathcurveto{\pgfqpoint{1.332789in}{1.802666in}}{\pgfqpoint{1.324889in}{1.805939in}}{\pgfqpoint{1.316653in}{1.805939in}}%
\pgfpathcurveto{\pgfqpoint{1.308417in}{1.805939in}}{\pgfqpoint{1.300517in}{1.802666in}}{\pgfqpoint{1.294693in}{1.796842in}}%
\pgfpathcurveto{\pgfqpoint{1.288869in}{1.791018in}}{\pgfqpoint{1.285596in}{1.783118in}}{\pgfqpoint{1.285596in}{1.774882in}}%
\pgfpathcurveto{\pgfqpoint{1.285596in}{1.766646in}}{\pgfqpoint{1.288869in}{1.758746in}}{\pgfqpoint{1.294693in}{1.752922in}}%
\pgfpathcurveto{\pgfqpoint{1.300517in}{1.747098in}}{\pgfqpoint{1.308417in}{1.743826in}}{\pgfqpoint{1.316653in}{1.743826in}}%
\pgfpathclose%
\pgfusepath{stroke,fill}%
\end{pgfscope}%
\begin{pgfscope}%
\pgfpathrectangle{\pgfqpoint{0.100000in}{0.220728in}}{\pgfqpoint{3.696000in}{3.696000in}}%
\pgfusepath{clip}%
\pgfsetbuttcap%
\pgfsetroundjoin%
\definecolor{currentfill}{rgb}{0.121569,0.466667,0.705882}%
\pgfsetfillcolor{currentfill}%
\pgfsetfillopacity{0.506388}%
\pgfsetlinewidth{1.003750pt}%
\definecolor{currentstroke}{rgb}{0.121569,0.466667,0.705882}%
\pgfsetstrokecolor{currentstroke}%
\pgfsetstrokeopacity{0.506388}%
\pgfsetdash{}{0pt}%
\pgfpathmoveto{\pgfqpoint{1.312772in}{1.741032in}}%
\pgfpathcurveto{\pgfqpoint{1.321009in}{1.741032in}}{\pgfqpoint{1.328909in}{1.744304in}}{\pgfqpoint{1.334733in}{1.750128in}}%
\pgfpathcurveto{\pgfqpoint{1.340557in}{1.755952in}}{\pgfqpoint{1.343829in}{1.763852in}}{\pgfqpoint{1.343829in}{1.772089in}}%
\pgfpathcurveto{\pgfqpoint{1.343829in}{1.780325in}}{\pgfqpoint{1.340557in}{1.788225in}}{\pgfqpoint{1.334733in}{1.794049in}}%
\pgfpathcurveto{\pgfqpoint{1.328909in}{1.799873in}}{\pgfqpoint{1.321009in}{1.803145in}}{\pgfqpoint{1.312772in}{1.803145in}}%
\pgfpathcurveto{\pgfqpoint{1.304536in}{1.803145in}}{\pgfqpoint{1.296636in}{1.799873in}}{\pgfqpoint{1.290812in}{1.794049in}}%
\pgfpathcurveto{\pgfqpoint{1.284988in}{1.788225in}}{\pgfqpoint{1.281716in}{1.780325in}}{\pgfqpoint{1.281716in}{1.772089in}}%
\pgfpathcurveto{\pgfqpoint{1.281716in}{1.763852in}}{\pgfqpoint{1.284988in}{1.755952in}}{\pgfqpoint{1.290812in}{1.750128in}}%
\pgfpathcurveto{\pgfqpoint{1.296636in}{1.744304in}}{\pgfqpoint{1.304536in}{1.741032in}}{\pgfqpoint{1.312772in}{1.741032in}}%
\pgfpathclose%
\pgfusepath{stroke,fill}%
\end{pgfscope}%
\begin{pgfscope}%
\pgfpathrectangle{\pgfqpoint{0.100000in}{0.220728in}}{\pgfqpoint{3.696000in}{3.696000in}}%
\pgfusepath{clip}%
\pgfsetbuttcap%
\pgfsetroundjoin%
\definecolor{currentfill}{rgb}{0.121569,0.466667,0.705882}%
\pgfsetfillcolor{currentfill}%
\pgfsetfillopacity{0.508360}%
\pgfsetlinewidth{1.003750pt}%
\definecolor{currentstroke}{rgb}{0.121569,0.466667,0.705882}%
\pgfsetstrokecolor{currentstroke}%
\pgfsetstrokeopacity{0.508360}%
\pgfsetdash{}{0pt}%
\pgfpathmoveto{\pgfqpoint{2.053904in}{1.872461in}}%
\pgfpathcurveto{\pgfqpoint{2.062140in}{1.872461in}}{\pgfqpoint{2.070040in}{1.875733in}}{\pgfqpoint{2.075864in}{1.881557in}}%
\pgfpathcurveto{\pgfqpoint{2.081688in}{1.887381in}}{\pgfqpoint{2.084960in}{1.895281in}}{\pgfqpoint{2.084960in}{1.903517in}}%
\pgfpathcurveto{\pgfqpoint{2.084960in}{1.911754in}}{\pgfqpoint{2.081688in}{1.919654in}}{\pgfqpoint{2.075864in}{1.925477in}}%
\pgfpathcurveto{\pgfqpoint{2.070040in}{1.931301in}}{\pgfqpoint{2.062140in}{1.934574in}}{\pgfqpoint{2.053904in}{1.934574in}}%
\pgfpathcurveto{\pgfqpoint{2.045667in}{1.934574in}}{\pgfqpoint{2.037767in}{1.931301in}}{\pgfqpoint{2.031943in}{1.925477in}}%
\pgfpathcurveto{\pgfqpoint{2.026119in}{1.919654in}}{\pgfqpoint{2.022847in}{1.911754in}}{\pgfqpoint{2.022847in}{1.903517in}}%
\pgfpathcurveto{\pgfqpoint{2.022847in}{1.895281in}}{\pgfqpoint{2.026119in}{1.887381in}}{\pgfqpoint{2.031943in}{1.881557in}}%
\pgfpathcurveto{\pgfqpoint{2.037767in}{1.875733in}}{\pgfqpoint{2.045667in}{1.872461in}}{\pgfqpoint{2.053904in}{1.872461in}}%
\pgfpathclose%
\pgfusepath{stroke,fill}%
\end{pgfscope}%
\begin{pgfscope}%
\pgfpathrectangle{\pgfqpoint{0.100000in}{0.220728in}}{\pgfqpoint{3.696000in}{3.696000in}}%
\pgfusepath{clip}%
\pgfsetbuttcap%
\pgfsetroundjoin%
\definecolor{currentfill}{rgb}{0.121569,0.466667,0.705882}%
\pgfsetfillcolor{currentfill}%
\pgfsetfillopacity{0.508749}%
\pgfsetlinewidth{1.003750pt}%
\definecolor{currentstroke}{rgb}{0.121569,0.466667,0.705882}%
\pgfsetstrokecolor{currentstroke}%
\pgfsetstrokeopacity{0.508749}%
\pgfsetdash{}{0pt}%
\pgfpathmoveto{\pgfqpoint{1.306094in}{1.738911in}}%
\pgfpathcurveto{\pgfqpoint{1.314330in}{1.738911in}}{\pgfqpoint{1.322230in}{1.742183in}}{\pgfqpoint{1.328054in}{1.748007in}}%
\pgfpathcurveto{\pgfqpoint{1.333878in}{1.753831in}}{\pgfqpoint{1.337150in}{1.761731in}}{\pgfqpoint{1.337150in}{1.769967in}}%
\pgfpathcurveto{\pgfqpoint{1.337150in}{1.778204in}}{\pgfqpoint{1.333878in}{1.786104in}}{\pgfqpoint{1.328054in}{1.791928in}}%
\pgfpathcurveto{\pgfqpoint{1.322230in}{1.797752in}}{\pgfqpoint{1.314330in}{1.801024in}}{\pgfqpoint{1.306094in}{1.801024in}}%
\pgfpathcurveto{\pgfqpoint{1.297858in}{1.801024in}}{\pgfqpoint{1.289958in}{1.797752in}}{\pgfqpoint{1.284134in}{1.791928in}}%
\pgfpathcurveto{\pgfqpoint{1.278310in}{1.786104in}}{\pgfqpoint{1.275037in}{1.778204in}}{\pgfqpoint{1.275037in}{1.769967in}}%
\pgfpathcurveto{\pgfqpoint{1.275037in}{1.761731in}}{\pgfqpoint{1.278310in}{1.753831in}}{\pgfqpoint{1.284134in}{1.748007in}}%
\pgfpathcurveto{\pgfqpoint{1.289958in}{1.742183in}}{\pgfqpoint{1.297858in}{1.738911in}}{\pgfqpoint{1.306094in}{1.738911in}}%
\pgfpathclose%
\pgfusepath{stroke,fill}%
\end{pgfscope}%
\begin{pgfscope}%
\pgfpathrectangle{\pgfqpoint{0.100000in}{0.220728in}}{\pgfqpoint{3.696000in}{3.696000in}}%
\pgfusepath{clip}%
\pgfsetbuttcap%
\pgfsetroundjoin%
\definecolor{currentfill}{rgb}{0.121569,0.466667,0.705882}%
\pgfsetfillcolor{currentfill}%
\pgfsetfillopacity{0.512072}%
\pgfsetlinewidth{1.003750pt}%
\definecolor{currentstroke}{rgb}{0.121569,0.466667,0.705882}%
\pgfsetstrokecolor{currentstroke}%
\pgfsetstrokeopacity{0.512072}%
\pgfsetdash{}{0pt}%
\pgfpathmoveto{\pgfqpoint{2.056221in}{1.871353in}}%
\pgfpathcurveto{\pgfqpoint{2.064457in}{1.871353in}}{\pgfqpoint{2.072357in}{1.874625in}}{\pgfqpoint{2.078181in}{1.880449in}}%
\pgfpathcurveto{\pgfqpoint{2.084005in}{1.886273in}}{\pgfqpoint{2.087278in}{1.894173in}}{\pgfqpoint{2.087278in}{1.902409in}}%
\pgfpathcurveto{\pgfqpoint{2.087278in}{1.910645in}}{\pgfqpoint{2.084005in}{1.918545in}}{\pgfqpoint{2.078181in}{1.924369in}}%
\pgfpathcurveto{\pgfqpoint{2.072357in}{1.930193in}}{\pgfqpoint{2.064457in}{1.933466in}}{\pgfqpoint{2.056221in}{1.933466in}}%
\pgfpathcurveto{\pgfqpoint{2.047985in}{1.933466in}}{\pgfqpoint{2.040085in}{1.930193in}}{\pgfqpoint{2.034261in}{1.924369in}}%
\pgfpathcurveto{\pgfqpoint{2.028437in}{1.918545in}}{\pgfqpoint{2.025165in}{1.910645in}}{\pgfqpoint{2.025165in}{1.902409in}}%
\pgfpathcurveto{\pgfqpoint{2.025165in}{1.894173in}}{\pgfqpoint{2.028437in}{1.886273in}}{\pgfqpoint{2.034261in}{1.880449in}}%
\pgfpathcurveto{\pgfqpoint{2.040085in}{1.874625in}}{\pgfqpoint{2.047985in}{1.871353in}}{\pgfqpoint{2.056221in}{1.871353in}}%
\pgfpathclose%
\pgfusepath{stroke,fill}%
\end{pgfscope}%
\begin{pgfscope}%
\pgfpathrectangle{\pgfqpoint{0.100000in}{0.220728in}}{\pgfqpoint{3.696000in}{3.696000in}}%
\pgfusepath{clip}%
\pgfsetbuttcap%
\pgfsetroundjoin%
\definecolor{currentfill}{rgb}{0.121569,0.466667,0.705882}%
\pgfsetfillcolor{currentfill}%
\pgfsetfillopacity{0.513023}%
\pgfsetlinewidth{1.003750pt}%
\definecolor{currentstroke}{rgb}{0.121569,0.466667,0.705882}%
\pgfsetstrokecolor{currentstroke}%
\pgfsetstrokeopacity{0.513023}%
\pgfsetdash{}{0pt}%
\pgfpathmoveto{\pgfqpoint{1.295085in}{1.733336in}}%
\pgfpathcurveto{\pgfqpoint{1.303321in}{1.733336in}}{\pgfqpoint{1.311222in}{1.736609in}}{\pgfqpoint{1.317045in}{1.742433in}}%
\pgfpathcurveto{\pgfqpoint{1.322869in}{1.748256in}}{\pgfqpoint{1.326142in}{1.756157in}}{\pgfqpoint{1.326142in}{1.764393in}}%
\pgfpathcurveto{\pgfqpoint{1.326142in}{1.772629in}}{\pgfqpoint{1.322869in}{1.780529in}}{\pgfqpoint{1.317045in}{1.786353in}}%
\pgfpathcurveto{\pgfqpoint{1.311222in}{1.792177in}}{\pgfqpoint{1.303321in}{1.795449in}}{\pgfqpoint{1.295085in}{1.795449in}}%
\pgfpathcurveto{\pgfqpoint{1.286849in}{1.795449in}}{\pgfqpoint{1.278949in}{1.792177in}}{\pgfqpoint{1.273125in}{1.786353in}}%
\pgfpathcurveto{\pgfqpoint{1.267301in}{1.780529in}}{\pgfqpoint{1.264029in}{1.772629in}}{\pgfqpoint{1.264029in}{1.764393in}}%
\pgfpathcurveto{\pgfqpoint{1.264029in}{1.756157in}}{\pgfqpoint{1.267301in}{1.748256in}}{\pgfqpoint{1.273125in}{1.742433in}}%
\pgfpathcurveto{\pgfqpoint{1.278949in}{1.736609in}}{\pgfqpoint{1.286849in}{1.733336in}}{\pgfqpoint{1.295085in}{1.733336in}}%
\pgfpathclose%
\pgfusepath{stroke,fill}%
\end{pgfscope}%
\begin{pgfscope}%
\pgfpathrectangle{\pgfqpoint{0.100000in}{0.220728in}}{\pgfqpoint{3.696000in}{3.696000in}}%
\pgfusepath{clip}%
\pgfsetbuttcap%
\pgfsetroundjoin%
\definecolor{currentfill}{rgb}{0.121569,0.466667,0.705882}%
\pgfsetfillcolor{currentfill}%
\pgfsetfillopacity{0.515477}%
\pgfsetlinewidth{1.003750pt}%
\definecolor{currentstroke}{rgb}{0.121569,0.466667,0.705882}%
\pgfsetstrokecolor{currentstroke}%
\pgfsetstrokeopacity{0.515477}%
\pgfsetdash{}{0pt}%
\pgfpathmoveto{\pgfqpoint{1.283576in}{1.725487in}}%
\pgfpathcurveto{\pgfqpoint{1.291812in}{1.725487in}}{\pgfqpoint{1.299712in}{1.728759in}}{\pgfqpoint{1.305536in}{1.734583in}}%
\pgfpathcurveto{\pgfqpoint{1.311360in}{1.740407in}}{\pgfqpoint{1.314632in}{1.748307in}}{\pgfqpoint{1.314632in}{1.756544in}}%
\pgfpathcurveto{\pgfqpoint{1.314632in}{1.764780in}}{\pgfqpoint{1.311360in}{1.772680in}}{\pgfqpoint{1.305536in}{1.778504in}}%
\pgfpathcurveto{\pgfqpoint{1.299712in}{1.784328in}}{\pgfqpoint{1.291812in}{1.787600in}}{\pgfqpoint{1.283576in}{1.787600in}}%
\pgfpathcurveto{\pgfqpoint{1.275340in}{1.787600in}}{\pgfqpoint{1.267440in}{1.784328in}}{\pgfqpoint{1.261616in}{1.778504in}}%
\pgfpathcurveto{\pgfqpoint{1.255792in}{1.772680in}}{\pgfqpoint{1.252519in}{1.764780in}}{\pgfqpoint{1.252519in}{1.756544in}}%
\pgfpathcurveto{\pgfqpoint{1.252519in}{1.748307in}}{\pgfqpoint{1.255792in}{1.740407in}}{\pgfqpoint{1.261616in}{1.734583in}}%
\pgfpathcurveto{\pgfqpoint{1.267440in}{1.728759in}}{\pgfqpoint{1.275340in}{1.725487in}}{\pgfqpoint{1.283576in}{1.725487in}}%
\pgfpathclose%
\pgfusepath{stroke,fill}%
\end{pgfscope}%
\begin{pgfscope}%
\pgfpathrectangle{\pgfqpoint{0.100000in}{0.220728in}}{\pgfqpoint{3.696000in}{3.696000in}}%
\pgfusepath{clip}%
\pgfsetbuttcap%
\pgfsetroundjoin%
\definecolor{currentfill}{rgb}{0.121569,0.466667,0.705882}%
\pgfsetfillcolor{currentfill}%
\pgfsetfillopacity{0.515967}%
\pgfsetlinewidth{1.003750pt}%
\definecolor{currentstroke}{rgb}{0.121569,0.466667,0.705882}%
\pgfsetstrokecolor{currentstroke}%
\pgfsetstrokeopacity{0.515967}%
\pgfsetdash{}{0pt}%
\pgfpathmoveto{\pgfqpoint{2.058053in}{1.868926in}}%
\pgfpathcurveto{\pgfqpoint{2.066289in}{1.868926in}}{\pgfqpoint{2.074189in}{1.872198in}}{\pgfqpoint{2.080013in}{1.878022in}}%
\pgfpathcurveto{\pgfqpoint{2.085837in}{1.883846in}}{\pgfqpoint{2.089109in}{1.891746in}}{\pgfqpoint{2.089109in}{1.899983in}}%
\pgfpathcurveto{\pgfqpoint{2.089109in}{1.908219in}}{\pgfqpoint{2.085837in}{1.916119in}}{\pgfqpoint{2.080013in}{1.921943in}}%
\pgfpathcurveto{\pgfqpoint{2.074189in}{1.927767in}}{\pgfqpoint{2.066289in}{1.931039in}}{\pgfqpoint{2.058053in}{1.931039in}}%
\pgfpathcurveto{\pgfqpoint{2.049817in}{1.931039in}}{\pgfqpoint{2.041917in}{1.927767in}}{\pgfqpoint{2.036093in}{1.921943in}}%
\pgfpathcurveto{\pgfqpoint{2.030269in}{1.916119in}}{\pgfqpoint{2.026996in}{1.908219in}}{\pgfqpoint{2.026996in}{1.899983in}}%
\pgfpathcurveto{\pgfqpoint{2.026996in}{1.891746in}}{\pgfqpoint{2.030269in}{1.883846in}}{\pgfqpoint{2.036093in}{1.878022in}}%
\pgfpathcurveto{\pgfqpoint{2.041917in}{1.872198in}}{\pgfqpoint{2.049817in}{1.868926in}}{\pgfqpoint{2.058053in}{1.868926in}}%
\pgfpathclose%
\pgfusepath{stroke,fill}%
\end{pgfscope}%
\begin{pgfscope}%
\pgfpathrectangle{\pgfqpoint{0.100000in}{0.220728in}}{\pgfqpoint{3.696000in}{3.696000in}}%
\pgfusepath{clip}%
\pgfsetbuttcap%
\pgfsetroundjoin%
\definecolor{currentfill}{rgb}{0.121569,0.466667,0.705882}%
\pgfsetfillcolor{currentfill}%
\pgfsetfillopacity{0.517959}%
\pgfsetlinewidth{1.003750pt}%
\definecolor{currentstroke}{rgb}{0.121569,0.466667,0.705882}%
\pgfsetstrokecolor{currentstroke}%
\pgfsetstrokeopacity{0.517959}%
\pgfsetdash{}{0pt}%
\pgfpathmoveto{\pgfqpoint{1.274407in}{1.721551in}}%
\pgfpathcurveto{\pgfqpoint{1.282644in}{1.721551in}}{\pgfqpoint{1.290544in}{1.724823in}}{\pgfqpoint{1.296367in}{1.730647in}}%
\pgfpathcurveto{\pgfqpoint{1.302191in}{1.736471in}}{\pgfqpoint{1.305464in}{1.744371in}}{\pgfqpoint{1.305464in}{1.752607in}}%
\pgfpathcurveto{\pgfqpoint{1.305464in}{1.760844in}}{\pgfqpoint{1.302191in}{1.768744in}}{\pgfqpoint{1.296367in}{1.774568in}}%
\pgfpathcurveto{\pgfqpoint{1.290544in}{1.780392in}}{\pgfqpoint{1.282644in}{1.783664in}}{\pgfqpoint{1.274407in}{1.783664in}}%
\pgfpathcurveto{\pgfqpoint{1.266171in}{1.783664in}}{\pgfqpoint{1.258271in}{1.780392in}}{\pgfqpoint{1.252447in}{1.774568in}}%
\pgfpathcurveto{\pgfqpoint{1.246623in}{1.768744in}}{\pgfqpoint{1.243351in}{1.760844in}}{\pgfqpoint{1.243351in}{1.752607in}}%
\pgfpathcurveto{\pgfqpoint{1.243351in}{1.744371in}}{\pgfqpoint{1.246623in}{1.736471in}}{\pgfqpoint{1.252447in}{1.730647in}}%
\pgfpathcurveto{\pgfqpoint{1.258271in}{1.724823in}}{\pgfqpoint{1.266171in}{1.721551in}}{\pgfqpoint{1.274407in}{1.721551in}}%
\pgfpathclose%
\pgfusepath{stroke,fill}%
\end{pgfscope}%
\begin{pgfscope}%
\pgfpathrectangle{\pgfqpoint{0.100000in}{0.220728in}}{\pgfqpoint{3.696000in}{3.696000in}}%
\pgfusepath{clip}%
\pgfsetbuttcap%
\pgfsetroundjoin%
\definecolor{currentfill}{rgb}{0.121569,0.466667,0.705882}%
\pgfsetfillcolor{currentfill}%
\pgfsetfillopacity{0.520606}%
\pgfsetlinewidth{1.003750pt}%
\definecolor{currentstroke}{rgb}{0.121569,0.466667,0.705882}%
\pgfsetstrokecolor{currentstroke}%
\pgfsetstrokeopacity{0.520606}%
\pgfsetdash{}{0pt}%
\pgfpathmoveto{\pgfqpoint{2.059874in}{1.867396in}}%
\pgfpathcurveto{\pgfqpoint{2.068110in}{1.867396in}}{\pgfqpoint{2.076010in}{1.870669in}}{\pgfqpoint{2.081834in}{1.876492in}}%
\pgfpathcurveto{\pgfqpoint{2.087658in}{1.882316in}}{\pgfqpoint{2.090930in}{1.890216in}}{\pgfqpoint{2.090930in}{1.898453in}}%
\pgfpathcurveto{\pgfqpoint{2.090930in}{1.906689in}}{\pgfqpoint{2.087658in}{1.914589in}}{\pgfqpoint{2.081834in}{1.920413in}}%
\pgfpathcurveto{\pgfqpoint{2.076010in}{1.926237in}}{\pgfqpoint{2.068110in}{1.929509in}}{\pgfqpoint{2.059874in}{1.929509in}}%
\pgfpathcurveto{\pgfqpoint{2.051638in}{1.929509in}}{\pgfqpoint{2.043737in}{1.926237in}}{\pgfqpoint{2.037914in}{1.920413in}}%
\pgfpathcurveto{\pgfqpoint{2.032090in}{1.914589in}}{\pgfqpoint{2.028817in}{1.906689in}}{\pgfqpoint{2.028817in}{1.898453in}}%
\pgfpathcurveto{\pgfqpoint{2.028817in}{1.890216in}}{\pgfqpoint{2.032090in}{1.882316in}}{\pgfqpoint{2.037914in}{1.876492in}}%
\pgfpathcurveto{\pgfqpoint{2.043737in}{1.870669in}}{\pgfqpoint{2.051638in}{1.867396in}}{\pgfqpoint{2.059874in}{1.867396in}}%
\pgfpathclose%
\pgfusepath{stroke,fill}%
\end{pgfscope}%
\begin{pgfscope}%
\pgfpathrectangle{\pgfqpoint{0.100000in}{0.220728in}}{\pgfqpoint{3.696000in}{3.696000in}}%
\pgfusepath{clip}%
\pgfsetbuttcap%
\pgfsetroundjoin%
\definecolor{currentfill}{rgb}{0.121569,0.466667,0.705882}%
\pgfsetfillcolor{currentfill}%
\pgfsetfillopacity{0.523704}%
\pgfsetlinewidth{1.003750pt}%
\definecolor{currentstroke}{rgb}{0.121569,0.466667,0.705882}%
\pgfsetstrokecolor{currentstroke}%
\pgfsetstrokeopacity{0.523704}%
\pgfsetdash{}{0pt}%
\pgfpathmoveto{\pgfqpoint{1.261178in}{1.717099in}}%
\pgfpathcurveto{\pgfqpoint{1.269414in}{1.717099in}}{\pgfqpoint{1.277314in}{1.720371in}}{\pgfqpoint{1.283138in}{1.726195in}}%
\pgfpathcurveto{\pgfqpoint{1.288962in}{1.732019in}}{\pgfqpoint{1.292235in}{1.739919in}}{\pgfqpoint{1.292235in}{1.748156in}}%
\pgfpathcurveto{\pgfqpoint{1.292235in}{1.756392in}}{\pgfqpoint{1.288962in}{1.764292in}}{\pgfqpoint{1.283138in}{1.770116in}}%
\pgfpathcurveto{\pgfqpoint{1.277314in}{1.775940in}}{\pgfqpoint{1.269414in}{1.779212in}}{\pgfqpoint{1.261178in}{1.779212in}}%
\pgfpathcurveto{\pgfqpoint{1.252942in}{1.779212in}}{\pgfqpoint{1.245042in}{1.775940in}}{\pgfqpoint{1.239218in}{1.770116in}}%
\pgfpathcurveto{\pgfqpoint{1.233394in}{1.764292in}}{\pgfqpoint{1.230122in}{1.756392in}}{\pgfqpoint{1.230122in}{1.748156in}}%
\pgfpathcurveto{\pgfqpoint{1.230122in}{1.739919in}}{\pgfqpoint{1.233394in}{1.732019in}}{\pgfqpoint{1.239218in}{1.726195in}}%
\pgfpathcurveto{\pgfqpoint{1.245042in}{1.720371in}}{\pgfqpoint{1.252942in}{1.717099in}}{\pgfqpoint{1.261178in}{1.717099in}}%
\pgfpathclose%
\pgfusepath{stroke,fill}%
\end{pgfscope}%
\begin{pgfscope}%
\pgfpathrectangle{\pgfqpoint{0.100000in}{0.220728in}}{\pgfqpoint{3.696000in}{3.696000in}}%
\pgfusepath{clip}%
\pgfsetbuttcap%
\pgfsetroundjoin%
\definecolor{currentfill}{rgb}{0.121569,0.466667,0.705882}%
\pgfsetfillcolor{currentfill}%
\pgfsetfillopacity{0.525670}%
\pgfsetlinewidth{1.003750pt}%
\definecolor{currentstroke}{rgb}{0.121569,0.466667,0.705882}%
\pgfsetstrokecolor{currentstroke}%
\pgfsetstrokeopacity{0.525670}%
\pgfsetdash{}{0pt}%
\pgfpathmoveto{\pgfqpoint{2.064514in}{1.867542in}}%
\pgfpathcurveto{\pgfqpoint{2.072750in}{1.867542in}}{\pgfqpoint{2.080650in}{1.870814in}}{\pgfqpoint{2.086474in}{1.876638in}}%
\pgfpathcurveto{\pgfqpoint{2.092298in}{1.882462in}}{\pgfqpoint{2.095570in}{1.890362in}}{\pgfqpoint{2.095570in}{1.898598in}}%
\pgfpathcurveto{\pgfqpoint{2.095570in}{1.906835in}}{\pgfqpoint{2.092298in}{1.914735in}}{\pgfqpoint{2.086474in}{1.920559in}}%
\pgfpathcurveto{\pgfqpoint{2.080650in}{1.926383in}}{\pgfqpoint{2.072750in}{1.929655in}}{\pgfqpoint{2.064514in}{1.929655in}}%
\pgfpathcurveto{\pgfqpoint{2.056278in}{1.929655in}}{\pgfqpoint{2.048377in}{1.926383in}}{\pgfqpoint{2.042554in}{1.920559in}}%
\pgfpathcurveto{\pgfqpoint{2.036730in}{1.914735in}}{\pgfqpoint{2.033457in}{1.906835in}}{\pgfqpoint{2.033457in}{1.898598in}}%
\pgfpathcurveto{\pgfqpoint{2.033457in}{1.890362in}}{\pgfqpoint{2.036730in}{1.882462in}}{\pgfqpoint{2.042554in}{1.876638in}}%
\pgfpathcurveto{\pgfqpoint{2.048377in}{1.870814in}}{\pgfqpoint{2.056278in}{1.867542in}}{\pgfqpoint{2.064514in}{1.867542in}}%
\pgfpathclose%
\pgfusepath{stroke,fill}%
\end{pgfscope}%
\begin{pgfscope}%
\pgfpathrectangle{\pgfqpoint{0.100000in}{0.220728in}}{\pgfqpoint{3.696000in}{3.696000in}}%
\pgfusepath{clip}%
\pgfsetbuttcap%
\pgfsetroundjoin%
\definecolor{currentfill}{rgb}{0.121569,0.466667,0.705882}%
\pgfsetfillcolor{currentfill}%
\pgfsetfillopacity{0.526959}%
\pgfsetlinewidth{1.003750pt}%
\definecolor{currentstroke}{rgb}{0.121569,0.466667,0.705882}%
\pgfsetstrokecolor{currentstroke}%
\pgfsetstrokeopacity{0.526959}%
\pgfsetdash{}{0pt}%
\pgfpathmoveto{\pgfqpoint{1.247559in}{1.710419in}}%
\pgfpathcurveto{\pgfqpoint{1.255795in}{1.710419in}}{\pgfqpoint{1.263695in}{1.713691in}}{\pgfqpoint{1.269519in}{1.719515in}}%
\pgfpathcurveto{\pgfqpoint{1.275343in}{1.725339in}}{\pgfqpoint{1.278615in}{1.733239in}}{\pgfqpoint{1.278615in}{1.741475in}}%
\pgfpathcurveto{\pgfqpoint{1.278615in}{1.749712in}}{\pgfqpoint{1.275343in}{1.757612in}}{\pgfqpoint{1.269519in}{1.763436in}}%
\pgfpathcurveto{\pgfqpoint{1.263695in}{1.769260in}}{\pgfqpoint{1.255795in}{1.772532in}}{\pgfqpoint{1.247559in}{1.772532in}}%
\pgfpathcurveto{\pgfqpoint{1.239322in}{1.772532in}}{\pgfqpoint{1.231422in}{1.769260in}}{\pgfqpoint{1.225598in}{1.763436in}}%
\pgfpathcurveto{\pgfqpoint{1.219774in}{1.757612in}}{\pgfqpoint{1.216502in}{1.749712in}}{\pgfqpoint{1.216502in}{1.741475in}}%
\pgfpathcurveto{\pgfqpoint{1.216502in}{1.733239in}}{\pgfqpoint{1.219774in}{1.725339in}}{\pgfqpoint{1.225598in}{1.719515in}}%
\pgfpathcurveto{\pgfqpoint{1.231422in}{1.713691in}}{\pgfqpoint{1.239322in}{1.710419in}}{\pgfqpoint{1.247559in}{1.710419in}}%
\pgfpathclose%
\pgfusepath{stroke,fill}%
\end{pgfscope}%
\begin{pgfscope}%
\pgfpathrectangle{\pgfqpoint{0.100000in}{0.220728in}}{\pgfqpoint{3.696000in}{3.696000in}}%
\pgfusepath{clip}%
\pgfsetbuttcap%
\pgfsetroundjoin%
\definecolor{currentfill}{rgb}{0.121569,0.466667,0.705882}%
\pgfsetfillcolor{currentfill}%
\pgfsetfillopacity{0.530050}%
\pgfsetlinewidth{1.003750pt}%
\definecolor{currentstroke}{rgb}{0.121569,0.466667,0.705882}%
\pgfsetstrokecolor{currentstroke}%
\pgfsetstrokeopacity{0.530050}%
\pgfsetdash{}{0pt}%
\pgfpathmoveto{\pgfqpoint{1.237089in}{1.704448in}}%
\pgfpathcurveto{\pgfqpoint{1.245325in}{1.704448in}}{\pgfqpoint{1.253225in}{1.707720in}}{\pgfqpoint{1.259049in}{1.713544in}}%
\pgfpathcurveto{\pgfqpoint{1.264873in}{1.719368in}}{\pgfqpoint{1.268146in}{1.727268in}}{\pgfqpoint{1.268146in}{1.735504in}}%
\pgfpathcurveto{\pgfqpoint{1.268146in}{1.743741in}}{\pgfqpoint{1.264873in}{1.751641in}}{\pgfqpoint{1.259049in}{1.757465in}}%
\pgfpathcurveto{\pgfqpoint{1.253225in}{1.763289in}}{\pgfqpoint{1.245325in}{1.766561in}}{\pgfqpoint{1.237089in}{1.766561in}}%
\pgfpathcurveto{\pgfqpoint{1.228853in}{1.766561in}}{\pgfqpoint{1.220953in}{1.763289in}}{\pgfqpoint{1.215129in}{1.757465in}}%
\pgfpathcurveto{\pgfqpoint{1.209305in}{1.751641in}}{\pgfqpoint{1.206033in}{1.743741in}}{\pgfqpoint{1.206033in}{1.735504in}}%
\pgfpathcurveto{\pgfqpoint{1.206033in}{1.727268in}}{\pgfqpoint{1.209305in}{1.719368in}}{\pgfqpoint{1.215129in}{1.713544in}}%
\pgfpathcurveto{\pgfqpoint{1.220953in}{1.707720in}}{\pgfqpoint{1.228853in}{1.704448in}}{\pgfqpoint{1.237089in}{1.704448in}}%
\pgfpathclose%
\pgfusepath{stroke,fill}%
\end{pgfscope}%
\begin{pgfscope}%
\pgfpathrectangle{\pgfqpoint{0.100000in}{0.220728in}}{\pgfqpoint{3.696000in}{3.696000in}}%
\pgfusepath{clip}%
\pgfsetbuttcap%
\pgfsetroundjoin%
\definecolor{currentfill}{rgb}{0.121569,0.466667,0.705882}%
\pgfsetfillcolor{currentfill}%
\pgfsetfillopacity{0.530598}%
\pgfsetlinewidth{1.003750pt}%
\definecolor{currentstroke}{rgb}{0.121569,0.466667,0.705882}%
\pgfsetstrokecolor{currentstroke}%
\pgfsetstrokeopacity{0.530598}%
\pgfsetdash{}{0pt}%
\pgfpathmoveto{\pgfqpoint{2.065242in}{1.862047in}}%
\pgfpathcurveto{\pgfqpoint{2.073478in}{1.862047in}}{\pgfqpoint{2.081378in}{1.865319in}}{\pgfqpoint{2.087202in}{1.871143in}}%
\pgfpathcurveto{\pgfqpoint{2.093026in}{1.876967in}}{\pgfqpoint{2.096298in}{1.884867in}}{\pgfqpoint{2.096298in}{1.893104in}}%
\pgfpathcurveto{\pgfqpoint{2.096298in}{1.901340in}}{\pgfqpoint{2.093026in}{1.909240in}}{\pgfqpoint{2.087202in}{1.915064in}}%
\pgfpathcurveto{\pgfqpoint{2.081378in}{1.920888in}}{\pgfqpoint{2.073478in}{1.924160in}}{\pgfqpoint{2.065242in}{1.924160in}}%
\pgfpathcurveto{\pgfqpoint{2.057005in}{1.924160in}}{\pgfqpoint{2.049105in}{1.920888in}}{\pgfqpoint{2.043281in}{1.915064in}}%
\pgfpathcurveto{\pgfqpoint{2.037458in}{1.909240in}}{\pgfqpoint{2.034185in}{1.901340in}}{\pgfqpoint{2.034185in}{1.893104in}}%
\pgfpathcurveto{\pgfqpoint{2.034185in}{1.884867in}}{\pgfqpoint{2.037458in}{1.876967in}}{\pgfqpoint{2.043281in}{1.871143in}}%
\pgfpathcurveto{\pgfqpoint{2.049105in}{1.865319in}}{\pgfqpoint{2.057005in}{1.862047in}}{\pgfqpoint{2.065242in}{1.862047in}}%
\pgfpathclose%
\pgfusepath{stroke,fill}%
\end{pgfscope}%
\begin{pgfscope}%
\pgfpathrectangle{\pgfqpoint{0.100000in}{0.220728in}}{\pgfqpoint{3.696000in}{3.696000in}}%
\pgfusepath{clip}%
\pgfsetbuttcap%
\pgfsetroundjoin%
\definecolor{currentfill}{rgb}{0.121569,0.466667,0.705882}%
\pgfsetfillcolor{currentfill}%
\pgfsetfillopacity{0.533797}%
\pgfsetlinewidth{1.003750pt}%
\definecolor{currentstroke}{rgb}{0.121569,0.466667,0.705882}%
\pgfsetstrokecolor{currentstroke}%
\pgfsetstrokeopacity{0.533797}%
\pgfsetdash{}{0pt}%
\pgfpathmoveto{\pgfqpoint{1.228427in}{1.701483in}}%
\pgfpathcurveto{\pgfqpoint{1.236663in}{1.701483in}}{\pgfqpoint{1.244563in}{1.704755in}}{\pgfqpoint{1.250387in}{1.710579in}}%
\pgfpathcurveto{\pgfqpoint{1.256211in}{1.716403in}}{\pgfqpoint{1.259483in}{1.724303in}}{\pgfqpoint{1.259483in}{1.732539in}}%
\pgfpathcurveto{\pgfqpoint{1.259483in}{1.740776in}}{\pgfqpoint{1.256211in}{1.748676in}}{\pgfqpoint{1.250387in}{1.754500in}}%
\pgfpathcurveto{\pgfqpoint{1.244563in}{1.760324in}}{\pgfqpoint{1.236663in}{1.763596in}}{\pgfqpoint{1.228427in}{1.763596in}}%
\pgfpathcurveto{\pgfqpoint{1.220191in}{1.763596in}}{\pgfqpoint{1.212290in}{1.760324in}}{\pgfqpoint{1.206467in}{1.754500in}}%
\pgfpathcurveto{\pgfqpoint{1.200643in}{1.748676in}}{\pgfqpoint{1.197370in}{1.740776in}}{\pgfqpoint{1.197370in}{1.732539in}}%
\pgfpathcurveto{\pgfqpoint{1.197370in}{1.724303in}}{\pgfqpoint{1.200643in}{1.716403in}}{\pgfqpoint{1.206467in}{1.710579in}}%
\pgfpathcurveto{\pgfqpoint{1.212290in}{1.704755in}}{\pgfqpoint{1.220191in}{1.701483in}}{\pgfqpoint{1.228427in}{1.701483in}}%
\pgfpathclose%
\pgfusepath{stroke,fill}%
\end{pgfscope}%
\begin{pgfscope}%
\pgfpathrectangle{\pgfqpoint{0.100000in}{0.220728in}}{\pgfqpoint{3.696000in}{3.696000in}}%
\pgfusepath{clip}%
\pgfsetbuttcap%
\pgfsetroundjoin%
\definecolor{currentfill}{rgb}{0.121569,0.466667,0.705882}%
\pgfsetfillcolor{currentfill}%
\pgfsetfillopacity{0.535401}%
\pgfsetlinewidth{1.003750pt}%
\definecolor{currentstroke}{rgb}{0.121569,0.466667,0.705882}%
\pgfsetstrokecolor{currentstroke}%
\pgfsetstrokeopacity{0.535401}%
\pgfsetdash{}{0pt}%
\pgfpathmoveto{\pgfqpoint{1.220841in}{1.695231in}}%
\pgfpathcurveto{\pgfqpoint{1.229078in}{1.695231in}}{\pgfqpoint{1.236978in}{1.698503in}}{\pgfqpoint{1.242802in}{1.704327in}}%
\pgfpathcurveto{\pgfqpoint{1.248625in}{1.710151in}}{\pgfqpoint{1.251898in}{1.718051in}}{\pgfqpoint{1.251898in}{1.726288in}}%
\pgfpathcurveto{\pgfqpoint{1.251898in}{1.734524in}}{\pgfqpoint{1.248625in}{1.742424in}}{\pgfqpoint{1.242802in}{1.748248in}}%
\pgfpathcurveto{\pgfqpoint{1.236978in}{1.754072in}}{\pgfqpoint{1.229078in}{1.757344in}}{\pgfqpoint{1.220841in}{1.757344in}}%
\pgfpathcurveto{\pgfqpoint{1.212605in}{1.757344in}}{\pgfqpoint{1.204705in}{1.754072in}}{\pgfqpoint{1.198881in}{1.748248in}}%
\pgfpathcurveto{\pgfqpoint{1.193057in}{1.742424in}}{\pgfqpoint{1.189785in}{1.734524in}}{\pgfqpoint{1.189785in}{1.726288in}}%
\pgfpathcurveto{\pgfqpoint{1.189785in}{1.718051in}}{\pgfqpoint{1.193057in}{1.710151in}}{\pgfqpoint{1.198881in}{1.704327in}}%
\pgfpathcurveto{\pgfqpoint{1.204705in}{1.698503in}}{\pgfqpoint{1.212605in}{1.695231in}}{\pgfqpoint{1.220841in}{1.695231in}}%
\pgfpathclose%
\pgfusepath{stroke,fill}%
\end{pgfscope}%
\begin{pgfscope}%
\pgfpathrectangle{\pgfqpoint{0.100000in}{0.220728in}}{\pgfqpoint{3.696000in}{3.696000in}}%
\pgfusepath{clip}%
\pgfsetbuttcap%
\pgfsetroundjoin%
\definecolor{currentfill}{rgb}{0.121569,0.466667,0.705882}%
\pgfsetfillcolor{currentfill}%
\pgfsetfillopacity{0.535713}%
\pgfsetlinewidth{1.003750pt}%
\definecolor{currentstroke}{rgb}{0.121569,0.466667,0.705882}%
\pgfsetstrokecolor{currentstroke}%
\pgfsetstrokeopacity{0.535713}%
\pgfsetdash{}{0pt}%
\pgfpathmoveto{\pgfqpoint{2.067658in}{1.854709in}}%
\pgfpathcurveto{\pgfqpoint{2.075894in}{1.854709in}}{\pgfqpoint{2.083794in}{1.857981in}}{\pgfqpoint{2.089618in}{1.863805in}}%
\pgfpathcurveto{\pgfqpoint{2.095442in}{1.869629in}}{\pgfqpoint{2.098714in}{1.877529in}}{\pgfqpoint{2.098714in}{1.885766in}}%
\pgfpathcurveto{\pgfqpoint{2.098714in}{1.894002in}}{\pgfqpoint{2.095442in}{1.901902in}}{\pgfqpoint{2.089618in}{1.907726in}}%
\pgfpathcurveto{\pgfqpoint{2.083794in}{1.913550in}}{\pgfqpoint{2.075894in}{1.916822in}}{\pgfqpoint{2.067658in}{1.916822in}}%
\pgfpathcurveto{\pgfqpoint{2.059421in}{1.916822in}}{\pgfqpoint{2.051521in}{1.913550in}}{\pgfqpoint{2.045697in}{1.907726in}}%
\pgfpathcurveto{\pgfqpoint{2.039873in}{1.901902in}}{\pgfqpoint{2.036601in}{1.894002in}}{\pgfqpoint{2.036601in}{1.885766in}}%
\pgfpathcurveto{\pgfqpoint{2.036601in}{1.877529in}}{\pgfqpoint{2.039873in}{1.869629in}}{\pgfqpoint{2.045697in}{1.863805in}}%
\pgfpathcurveto{\pgfqpoint{2.051521in}{1.857981in}}{\pgfqpoint{2.059421in}{1.854709in}}{\pgfqpoint{2.067658in}{1.854709in}}%
\pgfpathclose%
\pgfusepath{stroke,fill}%
\end{pgfscope}%
\begin{pgfscope}%
\pgfpathrectangle{\pgfqpoint{0.100000in}{0.220728in}}{\pgfqpoint{3.696000in}{3.696000in}}%
\pgfusepath{clip}%
\pgfsetbuttcap%
\pgfsetroundjoin%
\definecolor{currentfill}{rgb}{0.121569,0.466667,0.705882}%
\pgfsetfillcolor{currentfill}%
\pgfsetfillopacity{0.536617}%
\pgfsetlinewidth{1.003750pt}%
\definecolor{currentstroke}{rgb}{0.121569,0.466667,0.705882}%
\pgfsetstrokecolor{currentstroke}%
\pgfsetstrokeopacity{0.536617}%
\pgfsetdash{}{0pt}%
\pgfpathmoveto{\pgfqpoint{1.216164in}{1.691580in}}%
\pgfpathcurveto{\pgfqpoint{1.224401in}{1.691580in}}{\pgfqpoint{1.232301in}{1.694852in}}{\pgfqpoint{1.238125in}{1.700676in}}%
\pgfpathcurveto{\pgfqpoint{1.243948in}{1.706500in}}{\pgfqpoint{1.247221in}{1.714400in}}{\pgfqpoint{1.247221in}{1.722636in}}%
\pgfpathcurveto{\pgfqpoint{1.247221in}{1.730873in}}{\pgfqpoint{1.243948in}{1.738773in}}{\pgfqpoint{1.238125in}{1.744597in}}%
\pgfpathcurveto{\pgfqpoint{1.232301in}{1.750421in}}{\pgfqpoint{1.224401in}{1.753693in}}{\pgfqpoint{1.216164in}{1.753693in}}%
\pgfpathcurveto{\pgfqpoint{1.207928in}{1.753693in}}{\pgfqpoint{1.200028in}{1.750421in}}{\pgfqpoint{1.194204in}{1.744597in}}%
\pgfpathcurveto{\pgfqpoint{1.188380in}{1.738773in}}{\pgfqpoint{1.185108in}{1.730873in}}{\pgfqpoint{1.185108in}{1.722636in}}%
\pgfpathcurveto{\pgfqpoint{1.185108in}{1.714400in}}{\pgfqpoint{1.188380in}{1.706500in}}{\pgfqpoint{1.194204in}{1.700676in}}%
\pgfpathcurveto{\pgfqpoint{1.200028in}{1.694852in}}{\pgfqpoint{1.207928in}{1.691580in}}{\pgfqpoint{1.216164in}{1.691580in}}%
\pgfpathclose%
\pgfusepath{stroke,fill}%
\end{pgfscope}%
\begin{pgfscope}%
\pgfpathrectangle{\pgfqpoint{0.100000in}{0.220728in}}{\pgfqpoint{3.696000in}{3.696000in}}%
\pgfusepath{clip}%
\pgfsetbuttcap%
\pgfsetroundjoin%
\definecolor{currentfill}{rgb}{0.121569,0.466667,0.705882}%
\pgfsetfillcolor{currentfill}%
\pgfsetfillopacity{0.537908}%
\pgfsetlinewidth{1.003750pt}%
\definecolor{currentstroke}{rgb}{0.121569,0.466667,0.705882}%
\pgfsetstrokecolor{currentstroke}%
\pgfsetstrokeopacity{0.537908}%
\pgfsetdash{}{0pt}%
\pgfpathmoveto{\pgfqpoint{1.212305in}{1.689199in}}%
\pgfpathcurveto{\pgfqpoint{1.220541in}{1.689199in}}{\pgfqpoint{1.228441in}{1.692471in}}{\pgfqpoint{1.234265in}{1.698295in}}%
\pgfpathcurveto{\pgfqpoint{1.240089in}{1.704119in}}{\pgfqpoint{1.243361in}{1.712019in}}{\pgfqpoint{1.243361in}{1.720256in}}%
\pgfpathcurveto{\pgfqpoint{1.243361in}{1.728492in}}{\pgfqpoint{1.240089in}{1.736392in}}{\pgfqpoint{1.234265in}{1.742216in}}%
\pgfpathcurveto{\pgfqpoint{1.228441in}{1.748040in}}{\pgfqpoint{1.220541in}{1.751312in}}{\pgfqpoint{1.212305in}{1.751312in}}%
\pgfpathcurveto{\pgfqpoint{1.204068in}{1.751312in}}{\pgfqpoint{1.196168in}{1.748040in}}{\pgfqpoint{1.190344in}{1.742216in}}%
\pgfpathcurveto{\pgfqpoint{1.184520in}{1.736392in}}{\pgfqpoint{1.181248in}{1.728492in}}{\pgfqpoint{1.181248in}{1.720256in}}%
\pgfpathcurveto{\pgfqpoint{1.181248in}{1.712019in}}{\pgfqpoint{1.184520in}{1.704119in}}{\pgfqpoint{1.190344in}{1.698295in}}%
\pgfpathcurveto{\pgfqpoint{1.196168in}{1.692471in}}{\pgfqpoint{1.204068in}{1.689199in}}{\pgfqpoint{1.212305in}{1.689199in}}%
\pgfpathclose%
\pgfusepath{stroke,fill}%
\end{pgfscope}%
\begin{pgfscope}%
\pgfpathrectangle{\pgfqpoint{0.100000in}{0.220728in}}{\pgfqpoint{3.696000in}{3.696000in}}%
\pgfusepath{clip}%
\pgfsetbuttcap%
\pgfsetroundjoin%
\definecolor{currentfill}{rgb}{0.121569,0.466667,0.705882}%
\pgfsetfillcolor{currentfill}%
\pgfsetfillopacity{0.538167}%
\pgfsetlinewidth{1.003750pt}%
\definecolor{currentstroke}{rgb}{0.121569,0.466667,0.705882}%
\pgfsetstrokecolor{currentstroke}%
\pgfsetstrokeopacity{0.538167}%
\pgfsetdash{}{0pt}%
\pgfpathmoveto{\pgfqpoint{1.209360in}{1.685669in}}%
\pgfpathcurveto{\pgfqpoint{1.217596in}{1.685669in}}{\pgfqpoint{1.225496in}{1.688941in}}{\pgfqpoint{1.231320in}{1.694765in}}%
\pgfpathcurveto{\pgfqpoint{1.237144in}{1.700589in}}{\pgfqpoint{1.240416in}{1.708489in}}{\pgfqpoint{1.240416in}{1.716725in}}%
\pgfpathcurveto{\pgfqpoint{1.240416in}{1.724962in}}{\pgfqpoint{1.237144in}{1.732862in}}{\pgfqpoint{1.231320in}{1.738686in}}%
\pgfpathcurveto{\pgfqpoint{1.225496in}{1.744509in}}{\pgfqpoint{1.217596in}{1.747782in}}{\pgfqpoint{1.209360in}{1.747782in}}%
\pgfpathcurveto{\pgfqpoint{1.201124in}{1.747782in}}{\pgfqpoint{1.193224in}{1.744509in}}{\pgfqpoint{1.187400in}{1.738686in}}%
\pgfpathcurveto{\pgfqpoint{1.181576in}{1.732862in}}{\pgfqpoint{1.178303in}{1.724962in}}{\pgfqpoint{1.178303in}{1.716725in}}%
\pgfpathcurveto{\pgfqpoint{1.178303in}{1.708489in}}{\pgfqpoint{1.181576in}{1.700589in}}{\pgfqpoint{1.187400in}{1.694765in}}%
\pgfpathcurveto{\pgfqpoint{1.193224in}{1.688941in}}{\pgfqpoint{1.201124in}{1.685669in}}{\pgfqpoint{1.209360in}{1.685669in}}%
\pgfpathclose%
\pgfusepath{stroke,fill}%
\end{pgfscope}%
\begin{pgfscope}%
\pgfpathrectangle{\pgfqpoint{0.100000in}{0.220728in}}{\pgfqpoint{3.696000in}{3.696000in}}%
\pgfusepath{clip}%
\pgfsetbuttcap%
\pgfsetroundjoin%
\definecolor{currentfill}{rgb}{0.121569,0.466667,0.705882}%
\pgfsetfillcolor{currentfill}%
\pgfsetfillopacity{0.538202}%
\pgfsetlinewidth{1.003750pt}%
\definecolor{currentstroke}{rgb}{0.121569,0.466667,0.705882}%
\pgfsetstrokecolor{currentstroke}%
\pgfsetstrokeopacity{0.538202}%
\pgfsetdash{}{0pt}%
\pgfpathmoveto{\pgfqpoint{1.203488in}{1.678101in}}%
\pgfpathcurveto{\pgfqpoint{1.211724in}{1.678101in}}{\pgfqpoint{1.219624in}{1.681374in}}{\pgfqpoint{1.225448in}{1.687198in}}%
\pgfpathcurveto{\pgfqpoint{1.231272in}{1.693022in}}{\pgfqpoint{1.234544in}{1.700922in}}{\pgfqpoint{1.234544in}{1.709158in}}%
\pgfpathcurveto{\pgfqpoint{1.234544in}{1.717394in}}{\pgfqpoint{1.231272in}{1.725294in}}{\pgfqpoint{1.225448in}{1.731118in}}%
\pgfpathcurveto{\pgfqpoint{1.219624in}{1.736942in}}{\pgfqpoint{1.211724in}{1.740214in}}{\pgfqpoint{1.203488in}{1.740214in}}%
\pgfpathcurveto{\pgfqpoint{1.195251in}{1.740214in}}{\pgfqpoint{1.187351in}{1.736942in}}{\pgfqpoint{1.181527in}{1.731118in}}%
\pgfpathcurveto{\pgfqpoint{1.175703in}{1.725294in}}{\pgfqpoint{1.172431in}{1.717394in}}{\pgfqpoint{1.172431in}{1.709158in}}%
\pgfpathcurveto{\pgfqpoint{1.172431in}{1.700922in}}{\pgfqpoint{1.175703in}{1.693022in}}{\pgfqpoint{1.181527in}{1.687198in}}%
\pgfpathcurveto{\pgfqpoint{1.187351in}{1.681374in}}{\pgfqpoint{1.195251in}{1.678101in}}{\pgfqpoint{1.203488in}{1.678101in}}%
\pgfpathclose%
\pgfusepath{stroke,fill}%
\end{pgfscope}%
\begin{pgfscope}%
\pgfpathrectangle{\pgfqpoint{0.100000in}{0.220728in}}{\pgfqpoint{3.696000in}{3.696000in}}%
\pgfusepath{clip}%
\pgfsetbuttcap%
\pgfsetroundjoin%
\definecolor{currentfill}{rgb}{0.121569,0.466667,0.705882}%
\pgfsetfillcolor{currentfill}%
\pgfsetfillopacity{0.538958}%
\pgfsetlinewidth{1.003750pt}%
\definecolor{currentstroke}{rgb}{0.121569,0.466667,0.705882}%
\pgfsetstrokecolor{currentstroke}%
\pgfsetstrokeopacity{0.538958}%
\pgfsetdash{}{0pt}%
\pgfpathmoveto{\pgfqpoint{2.070187in}{1.854136in}}%
\pgfpathcurveto{\pgfqpoint{2.078423in}{1.854136in}}{\pgfqpoint{2.086323in}{1.857408in}}{\pgfqpoint{2.092147in}{1.863232in}}%
\pgfpathcurveto{\pgfqpoint{2.097971in}{1.869056in}}{\pgfqpoint{2.101244in}{1.876956in}}{\pgfqpoint{2.101244in}{1.885193in}}%
\pgfpathcurveto{\pgfqpoint{2.101244in}{1.893429in}}{\pgfqpoint{2.097971in}{1.901329in}}{\pgfqpoint{2.092147in}{1.907153in}}%
\pgfpathcurveto{\pgfqpoint{2.086323in}{1.912977in}}{\pgfqpoint{2.078423in}{1.916249in}}{\pgfqpoint{2.070187in}{1.916249in}}%
\pgfpathcurveto{\pgfqpoint{2.061951in}{1.916249in}}{\pgfqpoint{2.054051in}{1.912977in}}{\pgfqpoint{2.048227in}{1.907153in}}%
\pgfpathcurveto{\pgfqpoint{2.042403in}{1.901329in}}{\pgfqpoint{2.039131in}{1.893429in}}{\pgfqpoint{2.039131in}{1.885193in}}%
\pgfpathcurveto{\pgfqpoint{2.039131in}{1.876956in}}{\pgfqpoint{2.042403in}{1.869056in}}{\pgfqpoint{2.048227in}{1.863232in}}%
\pgfpathcurveto{\pgfqpoint{2.054051in}{1.857408in}}{\pgfqpoint{2.061951in}{1.854136in}}{\pgfqpoint{2.070187in}{1.854136in}}%
\pgfpathclose%
\pgfusepath{stroke,fill}%
\end{pgfscope}%
\begin{pgfscope}%
\pgfpathrectangle{\pgfqpoint{0.100000in}{0.220728in}}{\pgfqpoint{3.696000in}{3.696000in}}%
\pgfusepath{clip}%
\pgfsetbuttcap%
\pgfsetroundjoin%
\definecolor{currentfill}{rgb}{0.121569,0.466667,0.705882}%
\pgfsetfillcolor{currentfill}%
\pgfsetfillopacity{0.539297}%
\pgfsetlinewidth{1.003750pt}%
\definecolor{currentstroke}{rgb}{0.121569,0.466667,0.705882}%
\pgfsetstrokecolor{currentstroke}%
\pgfsetstrokeopacity{0.539297}%
\pgfsetdash{}{0pt}%
\pgfpathmoveto{\pgfqpoint{1.201039in}{1.677077in}}%
\pgfpathcurveto{\pgfqpoint{1.209275in}{1.677077in}}{\pgfqpoint{1.217175in}{1.680349in}}{\pgfqpoint{1.222999in}{1.686173in}}%
\pgfpathcurveto{\pgfqpoint{1.228823in}{1.691997in}}{\pgfqpoint{1.232096in}{1.699897in}}{\pgfqpoint{1.232096in}{1.708134in}}%
\pgfpathcurveto{\pgfqpoint{1.232096in}{1.716370in}}{\pgfqpoint{1.228823in}{1.724270in}}{\pgfqpoint{1.222999in}{1.730094in}}%
\pgfpathcurveto{\pgfqpoint{1.217175in}{1.735918in}}{\pgfqpoint{1.209275in}{1.739190in}}{\pgfqpoint{1.201039in}{1.739190in}}%
\pgfpathcurveto{\pgfqpoint{1.192803in}{1.739190in}}{\pgfqpoint{1.184903in}{1.735918in}}{\pgfqpoint{1.179079in}{1.730094in}}%
\pgfpathcurveto{\pgfqpoint{1.173255in}{1.724270in}}{\pgfqpoint{1.169983in}{1.716370in}}{\pgfqpoint{1.169983in}{1.708134in}}%
\pgfpathcurveto{\pgfqpoint{1.169983in}{1.699897in}}{\pgfqpoint{1.173255in}{1.691997in}}{\pgfqpoint{1.179079in}{1.686173in}}%
\pgfpathcurveto{\pgfqpoint{1.184903in}{1.680349in}}{\pgfqpoint{1.192803in}{1.677077in}}{\pgfqpoint{1.201039in}{1.677077in}}%
\pgfpathclose%
\pgfusepath{stroke,fill}%
\end{pgfscope}%
\begin{pgfscope}%
\pgfpathrectangle{\pgfqpoint{0.100000in}{0.220728in}}{\pgfqpoint{3.696000in}{3.696000in}}%
\pgfusepath{clip}%
\pgfsetbuttcap%
\pgfsetroundjoin%
\definecolor{currentfill}{rgb}{0.121569,0.466667,0.705882}%
\pgfsetfillcolor{currentfill}%
\pgfsetfillopacity{0.539587}%
\pgfsetlinewidth{1.003750pt}%
\definecolor{currentstroke}{rgb}{0.121569,0.466667,0.705882}%
\pgfsetstrokecolor{currentstroke}%
\pgfsetstrokeopacity{0.539587}%
\pgfsetdash{}{0pt}%
\pgfpathmoveto{\pgfqpoint{1.198879in}{1.675208in}}%
\pgfpathcurveto{\pgfqpoint{1.207115in}{1.675208in}}{\pgfqpoint{1.215015in}{1.678480in}}{\pgfqpoint{1.220839in}{1.684304in}}%
\pgfpathcurveto{\pgfqpoint{1.226663in}{1.690128in}}{\pgfqpoint{1.229935in}{1.698028in}}{\pgfqpoint{1.229935in}{1.706265in}}%
\pgfpathcurveto{\pgfqpoint{1.229935in}{1.714501in}}{\pgfqpoint{1.226663in}{1.722401in}}{\pgfqpoint{1.220839in}{1.728225in}}%
\pgfpathcurveto{\pgfqpoint{1.215015in}{1.734049in}}{\pgfqpoint{1.207115in}{1.737321in}}{\pgfqpoint{1.198879in}{1.737321in}}%
\pgfpathcurveto{\pgfqpoint{1.190642in}{1.737321in}}{\pgfqpoint{1.182742in}{1.734049in}}{\pgfqpoint{1.176918in}{1.728225in}}%
\pgfpathcurveto{\pgfqpoint{1.171094in}{1.722401in}}{\pgfqpoint{1.167822in}{1.714501in}}{\pgfqpoint{1.167822in}{1.706265in}}%
\pgfpathcurveto{\pgfqpoint{1.167822in}{1.698028in}}{\pgfqpoint{1.171094in}{1.690128in}}{\pgfqpoint{1.176918in}{1.684304in}}%
\pgfpathcurveto{\pgfqpoint{1.182742in}{1.678480in}}{\pgfqpoint{1.190642in}{1.675208in}}{\pgfqpoint{1.198879in}{1.675208in}}%
\pgfpathclose%
\pgfusepath{stroke,fill}%
\end{pgfscope}%
\begin{pgfscope}%
\pgfpathrectangle{\pgfqpoint{0.100000in}{0.220728in}}{\pgfqpoint{3.696000in}{3.696000in}}%
\pgfusepath{clip}%
\pgfsetbuttcap%
\pgfsetroundjoin%
\definecolor{currentfill}{rgb}{0.121569,0.466667,0.705882}%
\pgfsetfillcolor{currentfill}%
\pgfsetfillopacity{0.539864}%
\pgfsetlinewidth{1.003750pt}%
\definecolor{currentstroke}{rgb}{0.121569,0.466667,0.705882}%
\pgfsetstrokecolor{currentstroke}%
\pgfsetstrokeopacity{0.539864}%
\pgfsetdash{}{0pt}%
\pgfpathmoveto{\pgfqpoint{1.195053in}{1.670005in}}%
\pgfpathcurveto{\pgfqpoint{1.203290in}{1.670005in}}{\pgfqpoint{1.211190in}{1.673277in}}{\pgfqpoint{1.217014in}{1.679101in}}%
\pgfpathcurveto{\pgfqpoint{1.222838in}{1.684925in}}{\pgfqpoint{1.226110in}{1.692825in}}{\pgfqpoint{1.226110in}{1.701061in}}%
\pgfpathcurveto{\pgfqpoint{1.226110in}{1.709297in}}{\pgfqpoint{1.222838in}{1.717197in}}{\pgfqpoint{1.217014in}{1.723021in}}%
\pgfpathcurveto{\pgfqpoint{1.211190in}{1.728845in}}{\pgfqpoint{1.203290in}{1.732118in}}{\pgfqpoint{1.195053in}{1.732118in}}%
\pgfpathcurveto{\pgfqpoint{1.186817in}{1.732118in}}{\pgfqpoint{1.178917in}{1.728845in}}{\pgfqpoint{1.173093in}{1.723021in}}%
\pgfpathcurveto{\pgfqpoint{1.167269in}{1.717197in}}{\pgfqpoint{1.163997in}{1.709297in}}{\pgfqpoint{1.163997in}{1.701061in}}%
\pgfpathcurveto{\pgfqpoint{1.163997in}{1.692825in}}{\pgfqpoint{1.167269in}{1.684925in}}{\pgfqpoint{1.173093in}{1.679101in}}%
\pgfpathcurveto{\pgfqpoint{1.178917in}{1.673277in}}{\pgfqpoint{1.186817in}{1.670005in}}{\pgfqpoint{1.195053in}{1.670005in}}%
\pgfpathclose%
\pgfusepath{stroke,fill}%
\end{pgfscope}%
\begin{pgfscope}%
\pgfpathrectangle{\pgfqpoint{0.100000in}{0.220728in}}{\pgfqpoint{3.696000in}{3.696000in}}%
\pgfusepath{clip}%
\pgfsetbuttcap%
\pgfsetroundjoin%
\definecolor{currentfill}{rgb}{0.121569,0.466667,0.705882}%
\pgfsetfillcolor{currentfill}%
\pgfsetfillopacity{0.540494}%
\pgfsetlinewidth{1.003750pt}%
\definecolor{currentstroke}{rgb}{0.121569,0.466667,0.705882}%
\pgfsetstrokecolor{currentstroke}%
\pgfsetstrokeopacity{0.540494}%
\pgfsetdash{}{0pt}%
\pgfpathmoveto{\pgfqpoint{1.193685in}{1.669930in}}%
\pgfpathcurveto{\pgfqpoint{1.201921in}{1.669930in}}{\pgfqpoint{1.209822in}{1.673203in}}{\pgfqpoint{1.215645in}{1.679026in}}%
\pgfpathcurveto{\pgfqpoint{1.221469in}{1.684850in}}{\pgfqpoint{1.224742in}{1.692750in}}{\pgfqpoint{1.224742in}{1.700987in}}%
\pgfpathcurveto{\pgfqpoint{1.224742in}{1.709223in}}{\pgfqpoint{1.221469in}{1.717123in}}{\pgfqpoint{1.215645in}{1.722947in}}%
\pgfpathcurveto{\pgfqpoint{1.209822in}{1.728771in}}{\pgfqpoint{1.201921in}{1.732043in}}{\pgfqpoint{1.193685in}{1.732043in}}%
\pgfpathcurveto{\pgfqpoint{1.185449in}{1.732043in}}{\pgfqpoint{1.177549in}{1.728771in}}{\pgfqpoint{1.171725in}{1.722947in}}%
\pgfpathcurveto{\pgfqpoint{1.165901in}{1.717123in}}{\pgfqpoint{1.162629in}{1.709223in}}{\pgfqpoint{1.162629in}{1.700987in}}%
\pgfpathcurveto{\pgfqpoint{1.162629in}{1.692750in}}{\pgfqpoint{1.165901in}{1.684850in}}{\pgfqpoint{1.171725in}{1.679026in}}%
\pgfpathcurveto{\pgfqpoint{1.177549in}{1.673203in}}{\pgfqpoint{1.185449in}{1.669930in}}{\pgfqpoint{1.193685in}{1.669930in}}%
\pgfpathclose%
\pgfusepath{stroke,fill}%
\end{pgfscope}%
\begin{pgfscope}%
\pgfpathrectangle{\pgfqpoint{0.100000in}{0.220728in}}{\pgfqpoint{3.696000in}{3.696000in}}%
\pgfusepath{clip}%
\pgfsetbuttcap%
\pgfsetroundjoin%
\definecolor{currentfill}{rgb}{0.121569,0.466667,0.705882}%
\pgfsetfillcolor{currentfill}%
\pgfsetfillopacity{0.541176}%
\pgfsetlinewidth{1.003750pt}%
\definecolor{currentstroke}{rgb}{0.121569,0.466667,0.705882}%
\pgfsetstrokecolor{currentstroke}%
\pgfsetstrokeopacity{0.541176}%
\pgfsetdash{}{0pt}%
\pgfpathmoveto{\pgfqpoint{1.190662in}{1.667645in}}%
\pgfpathcurveto{\pgfqpoint{1.198898in}{1.667645in}}{\pgfqpoint{1.206798in}{1.670917in}}{\pgfqpoint{1.212622in}{1.676741in}}%
\pgfpathcurveto{\pgfqpoint{1.218446in}{1.682565in}}{\pgfqpoint{1.221718in}{1.690465in}}{\pgfqpoint{1.221718in}{1.698701in}}%
\pgfpathcurveto{\pgfqpoint{1.221718in}{1.706937in}}{\pgfqpoint{1.218446in}{1.714837in}}{\pgfqpoint{1.212622in}{1.720661in}}%
\pgfpathcurveto{\pgfqpoint{1.206798in}{1.726485in}}{\pgfqpoint{1.198898in}{1.729758in}}{\pgfqpoint{1.190662in}{1.729758in}}%
\pgfpathcurveto{\pgfqpoint{1.182426in}{1.729758in}}{\pgfqpoint{1.174525in}{1.726485in}}{\pgfqpoint{1.168702in}{1.720661in}}%
\pgfpathcurveto{\pgfqpoint{1.162878in}{1.714837in}}{\pgfqpoint{1.159605in}{1.706937in}}{\pgfqpoint{1.159605in}{1.698701in}}%
\pgfpathcurveto{\pgfqpoint{1.159605in}{1.690465in}}{\pgfqpoint{1.162878in}{1.682565in}}{\pgfqpoint{1.168702in}{1.676741in}}%
\pgfpathcurveto{\pgfqpoint{1.174525in}{1.670917in}}{\pgfqpoint{1.182426in}{1.667645in}}{\pgfqpoint{1.190662in}{1.667645in}}%
\pgfpathclose%
\pgfusepath{stroke,fill}%
\end{pgfscope}%
\begin{pgfscope}%
\pgfpathrectangle{\pgfqpoint{0.100000in}{0.220728in}}{\pgfqpoint{3.696000in}{3.696000in}}%
\pgfusepath{clip}%
\pgfsetbuttcap%
\pgfsetroundjoin%
\definecolor{currentfill}{rgb}{0.121569,0.466667,0.705882}%
\pgfsetfillcolor{currentfill}%
\pgfsetfillopacity{0.542254}%
\pgfsetlinewidth{1.003750pt}%
\definecolor{currentstroke}{rgb}{0.121569,0.466667,0.705882}%
\pgfsetstrokecolor{currentstroke}%
\pgfsetstrokeopacity{0.542254}%
\pgfsetdash{}{0pt}%
\pgfpathmoveto{\pgfqpoint{2.072852in}{1.852039in}}%
\pgfpathcurveto{\pgfqpoint{2.081089in}{1.852039in}}{\pgfqpoint{2.088989in}{1.855311in}}{\pgfqpoint{2.094813in}{1.861135in}}%
\pgfpathcurveto{\pgfqpoint{2.100637in}{1.866959in}}{\pgfqpoint{2.103909in}{1.874859in}}{\pgfqpoint{2.103909in}{1.883095in}}%
\pgfpathcurveto{\pgfqpoint{2.103909in}{1.891331in}}{\pgfqpoint{2.100637in}{1.899231in}}{\pgfqpoint{2.094813in}{1.905055in}}%
\pgfpathcurveto{\pgfqpoint{2.088989in}{1.910879in}}{\pgfqpoint{2.081089in}{1.914152in}}{\pgfqpoint{2.072852in}{1.914152in}}%
\pgfpathcurveto{\pgfqpoint{2.064616in}{1.914152in}}{\pgfqpoint{2.056716in}{1.910879in}}{\pgfqpoint{2.050892in}{1.905055in}}%
\pgfpathcurveto{\pgfqpoint{2.045068in}{1.899231in}}{\pgfqpoint{2.041796in}{1.891331in}}{\pgfqpoint{2.041796in}{1.883095in}}%
\pgfpathcurveto{\pgfqpoint{2.041796in}{1.874859in}}{\pgfqpoint{2.045068in}{1.866959in}}{\pgfqpoint{2.050892in}{1.861135in}}%
\pgfpathcurveto{\pgfqpoint{2.056716in}{1.855311in}}{\pgfqpoint{2.064616in}{1.852039in}}{\pgfqpoint{2.072852in}{1.852039in}}%
\pgfpathclose%
\pgfusepath{stroke,fill}%
\end{pgfscope}%
\begin{pgfscope}%
\pgfpathrectangle{\pgfqpoint{0.100000in}{0.220728in}}{\pgfqpoint{3.696000in}{3.696000in}}%
\pgfusepath{clip}%
\pgfsetbuttcap%
\pgfsetroundjoin%
\definecolor{currentfill}{rgb}{0.121569,0.466667,0.705882}%
\pgfsetfillcolor{currentfill}%
\pgfsetfillopacity{0.542554}%
\pgfsetlinewidth{1.003750pt}%
\definecolor{currentstroke}{rgb}{0.121569,0.466667,0.705882}%
\pgfsetstrokecolor{currentstroke}%
\pgfsetstrokeopacity{0.542554}%
\pgfsetdash{}{0pt}%
\pgfpathmoveto{\pgfqpoint{1.185873in}{1.663220in}}%
\pgfpathcurveto{\pgfqpoint{1.194109in}{1.663220in}}{\pgfqpoint{1.202009in}{1.666492in}}{\pgfqpoint{1.207833in}{1.672316in}}%
\pgfpathcurveto{\pgfqpoint{1.213657in}{1.678140in}}{\pgfqpoint{1.216929in}{1.686040in}}{\pgfqpoint{1.216929in}{1.694276in}}%
\pgfpathcurveto{\pgfqpoint{1.216929in}{1.702513in}}{\pgfqpoint{1.213657in}{1.710413in}}{\pgfqpoint{1.207833in}{1.716237in}}%
\pgfpathcurveto{\pgfqpoint{1.202009in}{1.722061in}}{\pgfqpoint{1.194109in}{1.725333in}}{\pgfqpoint{1.185873in}{1.725333in}}%
\pgfpathcurveto{\pgfqpoint{1.177637in}{1.725333in}}{\pgfqpoint{1.169737in}{1.722061in}}{\pgfqpoint{1.163913in}{1.716237in}}%
\pgfpathcurveto{\pgfqpoint{1.158089in}{1.710413in}}{\pgfqpoint{1.154816in}{1.702513in}}{\pgfqpoint{1.154816in}{1.694276in}}%
\pgfpathcurveto{\pgfqpoint{1.154816in}{1.686040in}}{\pgfqpoint{1.158089in}{1.678140in}}{\pgfqpoint{1.163913in}{1.672316in}}%
\pgfpathcurveto{\pgfqpoint{1.169737in}{1.666492in}}{\pgfqpoint{1.177637in}{1.663220in}}{\pgfqpoint{1.185873in}{1.663220in}}%
\pgfpathclose%
\pgfusepath{stroke,fill}%
\end{pgfscope}%
\begin{pgfscope}%
\pgfpathrectangle{\pgfqpoint{0.100000in}{0.220728in}}{\pgfqpoint{3.696000in}{3.696000in}}%
\pgfusepath{clip}%
\pgfsetbuttcap%
\pgfsetroundjoin%
\definecolor{currentfill}{rgb}{0.121569,0.466667,0.705882}%
\pgfsetfillcolor{currentfill}%
\pgfsetfillopacity{0.544032}%
\pgfsetlinewidth{1.003750pt}%
\definecolor{currentstroke}{rgb}{0.121569,0.466667,0.705882}%
\pgfsetstrokecolor{currentstroke}%
\pgfsetstrokeopacity{0.544032}%
\pgfsetdash{}{0pt}%
\pgfpathmoveto{\pgfqpoint{1.183241in}{1.663715in}}%
\pgfpathcurveto{\pgfqpoint{1.191478in}{1.663715in}}{\pgfqpoint{1.199378in}{1.666988in}}{\pgfqpoint{1.205202in}{1.672812in}}%
\pgfpathcurveto{\pgfqpoint{1.211025in}{1.678636in}}{\pgfqpoint{1.214298in}{1.686536in}}{\pgfqpoint{1.214298in}{1.694772in}}%
\pgfpathcurveto{\pgfqpoint{1.214298in}{1.703008in}}{\pgfqpoint{1.211025in}{1.710908in}}{\pgfqpoint{1.205202in}{1.716732in}}%
\pgfpathcurveto{\pgfqpoint{1.199378in}{1.722556in}}{\pgfqpoint{1.191478in}{1.725828in}}{\pgfqpoint{1.183241in}{1.725828in}}%
\pgfpathcurveto{\pgfqpoint{1.175005in}{1.725828in}}{\pgfqpoint{1.167105in}{1.722556in}}{\pgfqpoint{1.161281in}{1.716732in}}%
\pgfpathcurveto{\pgfqpoint{1.155457in}{1.710908in}}{\pgfqpoint{1.152185in}{1.703008in}}{\pgfqpoint{1.152185in}{1.694772in}}%
\pgfpathcurveto{\pgfqpoint{1.152185in}{1.686536in}}{\pgfqpoint{1.155457in}{1.678636in}}{\pgfqpoint{1.161281in}{1.672812in}}%
\pgfpathcurveto{\pgfqpoint{1.167105in}{1.666988in}}{\pgfqpoint{1.175005in}{1.663715in}}{\pgfqpoint{1.183241in}{1.663715in}}%
\pgfpathclose%
\pgfusepath{stroke,fill}%
\end{pgfscope}%
\begin{pgfscope}%
\pgfpathrectangle{\pgfqpoint{0.100000in}{0.220728in}}{\pgfqpoint{3.696000in}{3.696000in}}%
\pgfusepath{clip}%
\pgfsetbuttcap%
\pgfsetroundjoin%
\definecolor{currentfill}{rgb}{0.121569,0.466667,0.705882}%
\pgfsetfillcolor{currentfill}%
\pgfsetfillopacity{0.545165}%
\pgfsetlinewidth{1.003750pt}%
\definecolor{currentstroke}{rgb}{0.121569,0.466667,0.705882}%
\pgfsetstrokecolor{currentstroke}%
\pgfsetstrokeopacity{0.545165}%
\pgfsetdash{}{0pt}%
\pgfpathmoveto{\pgfqpoint{1.175789in}{1.659400in}}%
\pgfpathcurveto{\pgfqpoint{1.184026in}{1.659400in}}{\pgfqpoint{1.191926in}{1.662672in}}{\pgfqpoint{1.197750in}{1.668496in}}%
\pgfpathcurveto{\pgfqpoint{1.203573in}{1.674320in}}{\pgfqpoint{1.206846in}{1.682220in}}{\pgfqpoint{1.206846in}{1.690456in}}%
\pgfpathcurveto{\pgfqpoint{1.206846in}{1.698693in}}{\pgfqpoint{1.203573in}{1.706593in}}{\pgfqpoint{1.197750in}{1.712417in}}%
\pgfpathcurveto{\pgfqpoint{1.191926in}{1.718241in}}{\pgfqpoint{1.184026in}{1.721513in}}{\pgfqpoint{1.175789in}{1.721513in}}%
\pgfpathcurveto{\pgfqpoint{1.167553in}{1.721513in}}{\pgfqpoint{1.159653in}{1.718241in}}{\pgfqpoint{1.153829in}{1.712417in}}%
\pgfpathcurveto{\pgfqpoint{1.148005in}{1.706593in}}{\pgfqpoint{1.144733in}{1.698693in}}{\pgfqpoint{1.144733in}{1.690456in}}%
\pgfpathcurveto{\pgfqpoint{1.144733in}{1.682220in}}{\pgfqpoint{1.148005in}{1.674320in}}{\pgfqpoint{1.153829in}{1.668496in}}%
\pgfpathcurveto{\pgfqpoint{1.159653in}{1.662672in}}{\pgfqpoint{1.167553in}{1.659400in}}{\pgfqpoint{1.175789in}{1.659400in}}%
\pgfpathclose%
\pgfusepath{stroke,fill}%
\end{pgfscope}%
\begin{pgfscope}%
\pgfpathrectangle{\pgfqpoint{0.100000in}{0.220728in}}{\pgfqpoint{3.696000in}{3.696000in}}%
\pgfusepath{clip}%
\pgfsetbuttcap%
\pgfsetroundjoin%
\definecolor{currentfill}{rgb}{0.121569,0.466667,0.705882}%
\pgfsetfillcolor{currentfill}%
\pgfsetfillopacity{0.545771}%
\pgfsetlinewidth{1.003750pt}%
\definecolor{currentstroke}{rgb}{0.121569,0.466667,0.705882}%
\pgfsetstrokecolor{currentstroke}%
\pgfsetstrokeopacity{0.545771}%
\pgfsetdash{}{0pt}%
\pgfpathmoveto{\pgfqpoint{2.073935in}{1.847823in}}%
\pgfpathcurveto{\pgfqpoint{2.082171in}{1.847823in}}{\pgfqpoint{2.090071in}{1.851096in}}{\pgfqpoint{2.095895in}{1.856920in}}%
\pgfpathcurveto{\pgfqpoint{2.101719in}{1.862743in}}{\pgfqpoint{2.104991in}{1.870644in}}{\pgfqpoint{2.104991in}{1.878880in}}%
\pgfpathcurveto{\pgfqpoint{2.104991in}{1.887116in}}{\pgfqpoint{2.101719in}{1.895016in}}{\pgfqpoint{2.095895in}{1.900840in}}%
\pgfpathcurveto{\pgfqpoint{2.090071in}{1.906664in}}{\pgfqpoint{2.082171in}{1.909936in}}{\pgfqpoint{2.073935in}{1.909936in}}%
\pgfpathcurveto{\pgfqpoint{2.065698in}{1.909936in}}{\pgfqpoint{2.057798in}{1.906664in}}{\pgfqpoint{2.051974in}{1.900840in}}%
\pgfpathcurveto{\pgfqpoint{2.046150in}{1.895016in}}{\pgfqpoint{2.042878in}{1.887116in}}{\pgfqpoint{2.042878in}{1.878880in}}%
\pgfpathcurveto{\pgfqpoint{2.042878in}{1.870644in}}{\pgfqpoint{2.046150in}{1.862743in}}{\pgfqpoint{2.051974in}{1.856920in}}%
\pgfpathcurveto{\pgfqpoint{2.057798in}{1.851096in}}{\pgfqpoint{2.065698in}{1.847823in}}{\pgfqpoint{2.073935in}{1.847823in}}%
\pgfpathclose%
\pgfusepath{stroke,fill}%
\end{pgfscope}%
\begin{pgfscope}%
\pgfpathrectangle{\pgfqpoint{0.100000in}{0.220728in}}{\pgfqpoint{3.696000in}{3.696000in}}%
\pgfusepath{clip}%
\pgfsetbuttcap%
\pgfsetroundjoin%
\definecolor{currentfill}{rgb}{0.121569,0.466667,0.705882}%
\pgfsetfillcolor{currentfill}%
\pgfsetfillopacity{0.545948}%
\pgfsetlinewidth{1.003750pt}%
\definecolor{currentstroke}{rgb}{0.121569,0.466667,0.705882}%
\pgfsetstrokecolor{currentstroke}%
\pgfsetstrokeopacity{0.545948}%
\pgfsetdash{}{0pt}%
\pgfpathmoveto{\pgfqpoint{1.163162in}{1.641227in}}%
\pgfpathcurveto{\pgfqpoint{1.171399in}{1.641227in}}{\pgfqpoint{1.179299in}{1.644499in}}{\pgfqpoint{1.185123in}{1.650323in}}%
\pgfpathcurveto{\pgfqpoint{1.190947in}{1.656147in}}{\pgfqpoint{1.194219in}{1.664047in}}{\pgfqpoint{1.194219in}{1.672284in}}%
\pgfpathcurveto{\pgfqpoint{1.194219in}{1.680520in}}{\pgfqpoint{1.190947in}{1.688420in}}{\pgfqpoint{1.185123in}{1.694244in}}%
\pgfpathcurveto{\pgfqpoint{1.179299in}{1.700068in}}{\pgfqpoint{1.171399in}{1.703340in}}{\pgfqpoint{1.163162in}{1.703340in}}%
\pgfpathcurveto{\pgfqpoint{1.154926in}{1.703340in}}{\pgfqpoint{1.147026in}{1.700068in}}{\pgfqpoint{1.141202in}{1.694244in}}%
\pgfpathcurveto{\pgfqpoint{1.135378in}{1.688420in}}{\pgfqpoint{1.132106in}{1.680520in}}{\pgfqpoint{1.132106in}{1.672284in}}%
\pgfpathcurveto{\pgfqpoint{1.132106in}{1.664047in}}{\pgfqpoint{1.135378in}{1.656147in}}{\pgfqpoint{1.141202in}{1.650323in}}%
\pgfpathcurveto{\pgfqpoint{1.147026in}{1.644499in}}{\pgfqpoint{1.154926in}{1.641227in}}{\pgfqpoint{1.163162in}{1.641227in}}%
\pgfpathclose%
\pgfusepath{stroke,fill}%
\end{pgfscope}%
\begin{pgfscope}%
\pgfpathrectangle{\pgfqpoint{0.100000in}{0.220728in}}{\pgfqpoint{3.696000in}{3.696000in}}%
\pgfusepath{clip}%
\pgfsetbuttcap%
\pgfsetroundjoin%
\definecolor{currentfill}{rgb}{0.121569,0.466667,0.705882}%
\pgfsetfillcolor{currentfill}%
\pgfsetfillopacity{0.549399}%
\pgfsetlinewidth{1.003750pt}%
\definecolor{currentstroke}{rgb}{0.121569,0.466667,0.705882}%
\pgfsetstrokecolor{currentstroke}%
\pgfsetstrokeopacity{0.549399}%
\pgfsetdash{}{0pt}%
\pgfpathmoveto{\pgfqpoint{2.076355in}{1.840541in}}%
\pgfpathcurveto{\pgfqpoint{2.084591in}{1.840541in}}{\pgfqpoint{2.092491in}{1.843814in}}{\pgfqpoint{2.098315in}{1.849638in}}%
\pgfpathcurveto{\pgfqpoint{2.104139in}{1.855462in}}{\pgfqpoint{2.107411in}{1.863362in}}{\pgfqpoint{2.107411in}{1.871598in}}%
\pgfpathcurveto{\pgfqpoint{2.107411in}{1.879834in}}{\pgfqpoint{2.104139in}{1.887734in}}{\pgfqpoint{2.098315in}{1.893558in}}%
\pgfpathcurveto{\pgfqpoint{2.092491in}{1.899382in}}{\pgfqpoint{2.084591in}{1.902654in}}{\pgfqpoint{2.076355in}{1.902654in}}%
\pgfpathcurveto{\pgfqpoint{2.068118in}{1.902654in}}{\pgfqpoint{2.060218in}{1.899382in}}{\pgfqpoint{2.054394in}{1.893558in}}%
\pgfpathcurveto{\pgfqpoint{2.048570in}{1.887734in}}{\pgfqpoint{2.045298in}{1.879834in}}{\pgfqpoint{2.045298in}{1.871598in}}%
\pgfpathcurveto{\pgfqpoint{2.045298in}{1.863362in}}{\pgfqpoint{2.048570in}{1.855462in}}{\pgfqpoint{2.054394in}{1.849638in}}%
\pgfpathcurveto{\pgfqpoint{2.060218in}{1.843814in}}{\pgfqpoint{2.068118in}{1.840541in}}{\pgfqpoint{2.076355in}{1.840541in}}%
\pgfpathclose%
\pgfusepath{stroke,fill}%
\end{pgfscope}%
\begin{pgfscope}%
\pgfpathrectangle{\pgfqpoint{0.100000in}{0.220728in}}{\pgfqpoint{3.696000in}{3.696000in}}%
\pgfusepath{clip}%
\pgfsetbuttcap%
\pgfsetroundjoin%
\definecolor{currentfill}{rgb}{0.121569,0.466667,0.705882}%
\pgfsetfillcolor{currentfill}%
\pgfsetfillopacity{0.550051}%
\pgfsetlinewidth{1.003750pt}%
\definecolor{currentstroke}{rgb}{0.121569,0.466667,0.705882}%
\pgfsetstrokecolor{currentstroke}%
\pgfsetstrokeopacity{0.550051}%
\pgfsetdash{}{0pt}%
\pgfpathmoveto{\pgfqpoint{1.154185in}{1.639259in}}%
\pgfpathcurveto{\pgfqpoint{1.162421in}{1.639259in}}{\pgfqpoint{1.170321in}{1.642531in}}{\pgfqpoint{1.176145in}{1.648355in}}%
\pgfpathcurveto{\pgfqpoint{1.181969in}{1.654179in}}{\pgfqpoint{1.185241in}{1.662079in}}{\pgfqpoint{1.185241in}{1.670316in}}%
\pgfpathcurveto{\pgfqpoint{1.185241in}{1.678552in}}{\pgfqpoint{1.181969in}{1.686452in}}{\pgfqpoint{1.176145in}{1.692276in}}%
\pgfpathcurveto{\pgfqpoint{1.170321in}{1.698100in}}{\pgfqpoint{1.162421in}{1.701372in}}{\pgfqpoint{1.154185in}{1.701372in}}%
\pgfpathcurveto{\pgfqpoint{1.145948in}{1.701372in}}{\pgfqpoint{1.138048in}{1.698100in}}{\pgfqpoint{1.132224in}{1.692276in}}%
\pgfpathcurveto{\pgfqpoint{1.126400in}{1.686452in}}{\pgfqpoint{1.123128in}{1.678552in}}{\pgfqpoint{1.123128in}{1.670316in}}%
\pgfpathcurveto{\pgfqpoint{1.123128in}{1.662079in}}{\pgfqpoint{1.126400in}{1.654179in}}{\pgfqpoint{1.132224in}{1.648355in}}%
\pgfpathcurveto{\pgfqpoint{1.138048in}{1.642531in}}{\pgfqpoint{1.145948in}{1.639259in}}{\pgfqpoint{1.154185in}{1.639259in}}%
\pgfpathclose%
\pgfusepath{stroke,fill}%
\end{pgfscope}%
\begin{pgfscope}%
\pgfpathrectangle{\pgfqpoint{0.100000in}{0.220728in}}{\pgfqpoint{3.696000in}{3.696000in}}%
\pgfusepath{clip}%
\pgfsetbuttcap%
\pgfsetroundjoin%
\definecolor{currentfill}{rgb}{0.121569,0.466667,0.705882}%
\pgfsetfillcolor{currentfill}%
\pgfsetfillopacity{0.552338}%
\pgfsetlinewidth{1.003750pt}%
\definecolor{currentstroke}{rgb}{0.121569,0.466667,0.705882}%
\pgfsetstrokecolor{currentstroke}%
\pgfsetstrokeopacity{0.552338}%
\pgfsetdash{}{0pt}%
\pgfpathmoveto{\pgfqpoint{1.145164in}{1.636326in}}%
\pgfpathcurveto{\pgfqpoint{1.153400in}{1.636326in}}{\pgfqpoint{1.161300in}{1.639598in}}{\pgfqpoint{1.167124in}{1.645422in}}%
\pgfpathcurveto{\pgfqpoint{1.172948in}{1.651246in}}{\pgfqpoint{1.176220in}{1.659146in}}{\pgfqpoint{1.176220in}{1.667382in}}%
\pgfpathcurveto{\pgfqpoint{1.176220in}{1.675619in}}{\pgfqpoint{1.172948in}{1.683519in}}{\pgfqpoint{1.167124in}{1.689343in}}%
\pgfpathcurveto{\pgfqpoint{1.161300in}{1.695167in}}{\pgfqpoint{1.153400in}{1.698439in}}{\pgfqpoint{1.145164in}{1.698439in}}%
\pgfpathcurveto{\pgfqpoint{1.136928in}{1.698439in}}{\pgfqpoint{1.129028in}{1.695167in}}{\pgfqpoint{1.123204in}{1.689343in}}%
\pgfpathcurveto{\pgfqpoint{1.117380in}{1.683519in}}{\pgfqpoint{1.114107in}{1.675619in}}{\pgfqpoint{1.114107in}{1.667382in}}%
\pgfpathcurveto{\pgfqpoint{1.114107in}{1.659146in}}{\pgfqpoint{1.117380in}{1.651246in}}{\pgfqpoint{1.123204in}{1.645422in}}%
\pgfpathcurveto{\pgfqpoint{1.129028in}{1.639598in}}{\pgfqpoint{1.136928in}{1.636326in}}{\pgfqpoint{1.145164in}{1.636326in}}%
\pgfpathclose%
\pgfusepath{stroke,fill}%
\end{pgfscope}%
\begin{pgfscope}%
\pgfpathrectangle{\pgfqpoint{0.100000in}{0.220728in}}{\pgfqpoint{3.696000in}{3.696000in}}%
\pgfusepath{clip}%
\pgfsetbuttcap%
\pgfsetroundjoin%
\definecolor{currentfill}{rgb}{0.121569,0.466667,0.705882}%
\pgfsetfillcolor{currentfill}%
\pgfsetfillopacity{0.553716}%
\pgfsetlinewidth{1.003750pt}%
\definecolor{currentstroke}{rgb}{0.121569,0.466667,0.705882}%
\pgfsetstrokecolor{currentstroke}%
\pgfsetstrokeopacity{0.553716}%
\pgfsetdash{}{0pt}%
\pgfpathmoveto{\pgfqpoint{1.140665in}{1.632125in}}%
\pgfpathcurveto{\pgfqpoint{1.148901in}{1.632125in}}{\pgfqpoint{1.156801in}{1.635397in}}{\pgfqpoint{1.162625in}{1.641221in}}%
\pgfpathcurveto{\pgfqpoint{1.168449in}{1.647045in}}{\pgfqpoint{1.171721in}{1.654945in}}{\pgfqpoint{1.171721in}{1.663181in}}%
\pgfpathcurveto{\pgfqpoint{1.171721in}{1.671418in}}{\pgfqpoint{1.168449in}{1.679318in}}{\pgfqpoint{1.162625in}{1.685142in}}%
\pgfpathcurveto{\pgfqpoint{1.156801in}{1.690966in}}{\pgfqpoint{1.148901in}{1.694238in}}{\pgfqpoint{1.140665in}{1.694238in}}%
\pgfpathcurveto{\pgfqpoint{1.132429in}{1.694238in}}{\pgfqpoint{1.124528in}{1.690966in}}{\pgfqpoint{1.118705in}{1.685142in}}%
\pgfpathcurveto{\pgfqpoint{1.112881in}{1.679318in}}{\pgfqpoint{1.109608in}{1.671418in}}{\pgfqpoint{1.109608in}{1.663181in}}%
\pgfpathcurveto{\pgfqpoint{1.109608in}{1.654945in}}{\pgfqpoint{1.112881in}{1.647045in}}{\pgfqpoint{1.118705in}{1.641221in}}%
\pgfpathcurveto{\pgfqpoint{1.124528in}{1.635397in}}{\pgfqpoint{1.132429in}{1.632125in}}{\pgfqpoint{1.140665in}{1.632125in}}%
\pgfpathclose%
\pgfusepath{stroke,fill}%
\end{pgfscope}%
\begin{pgfscope}%
\pgfpathrectangle{\pgfqpoint{0.100000in}{0.220728in}}{\pgfqpoint{3.696000in}{3.696000in}}%
\pgfusepath{clip}%
\pgfsetbuttcap%
\pgfsetroundjoin%
\definecolor{currentfill}{rgb}{0.121569,0.466667,0.705882}%
\pgfsetfillcolor{currentfill}%
\pgfsetfillopacity{0.554084}%
\pgfsetlinewidth{1.003750pt}%
\definecolor{currentstroke}{rgb}{0.121569,0.466667,0.705882}%
\pgfsetstrokecolor{currentstroke}%
\pgfsetstrokeopacity{0.554084}%
\pgfsetdash{}{0pt}%
\pgfpathmoveto{\pgfqpoint{2.079652in}{1.838735in}}%
\pgfpathcurveto{\pgfqpoint{2.087888in}{1.838735in}}{\pgfqpoint{2.095788in}{1.842008in}}{\pgfqpoint{2.101612in}{1.847831in}}%
\pgfpathcurveto{\pgfqpoint{2.107436in}{1.853655in}}{\pgfqpoint{2.110708in}{1.861555in}}{\pgfqpoint{2.110708in}{1.869792in}}%
\pgfpathcurveto{\pgfqpoint{2.110708in}{1.878028in}}{\pgfqpoint{2.107436in}{1.885928in}}{\pgfqpoint{2.101612in}{1.891752in}}%
\pgfpathcurveto{\pgfqpoint{2.095788in}{1.897576in}}{\pgfqpoint{2.087888in}{1.900848in}}{\pgfqpoint{2.079652in}{1.900848in}}%
\pgfpathcurveto{\pgfqpoint{2.071415in}{1.900848in}}{\pgfqpoint{2.063515in}{1.897576in}}{\pgfqpoint{2.057691in}{1.891752in}}%
\pgfpathcurveto{\pgfqpoint{2.051867in}{1.885928in}}{\pgfqpoint{2.048595in}{1.878028in}}{\pgfqpoint{2.048595in}{1.869792in}}%
\pgfpathcurveto{\pgfqpoint{2.048595in}{1.861555in}}{\pgfqpoint{2.051867in}{1.853655in}}{\pgfqpoint{2.057691in}{1.847831in}}%
\pgfpathcurveto{\pgfqpoint{2.063515in}{1.842008in}}{\pgfqpoint{2.071415in}{1.838735in}}{\pgfqpoint{2.079652in}{1.838735in}}%
\pgfpathclose%
\pgfusepath{stroke,fill}%
\end{pgfscope}%
\begin{pgfscope}%
\pgfpathrectangle{\pgfqpoint{0.100000in}{0.220728in}}{\pgfqpoint{3.696000in}{3.696000in}}%
\pgfusepath{clip}%
\pgfsetbuttcap%
\pgfsetroundjoin%
\definecolor{currentfill}{rgb}{0.121569,0.466667,0.705882}%
\pgfsetfillcolor{currentfill}%
\pgfsetfillopacity{0.556394}%
\pgfsetlinewidth{1.003750pt}%
\definecolor{currentstroke}{rgb}{0.121569,0.466667,0.705882}%
\pgfsetstrokecolor{currentstroke}%
\pgfsetstrokeopacity{0.556394}%
\pgfsetdash{}{0pt}%
\pgfpathmoveto{\pgfqpoint{1.132686in}{1.625303in}}%
\pgfpathcurveto{\pgfqpoint{1.140922in}{1.625303in}}{\pgfqpoint{1.148822in}{1.628575in}}{\pgfqpoint{1.154646in}{1.634399in}}%
\pgfpathcurveto{\pgfqpoint{1.160470in}{1.640223in}}{\pgfqpoint{1.163743in}{1.648123in}}{\pgfqpoint{1.163743in}{1.656360in}}%
\pgfpathcurveto{\pgfqpoint{1.163743in}{1.664596in}}{\pgfqpoint{1.160470in}{1.672496in}}{\pgfqpoint{1.154646in}{1.678320in}}%
\pgfpathcurveto{\pgfqpoint{1.148822in}{1.684144in}}{\pgfqpoint{1.140922in}{1.687416in}}{\pgfqpoint{1.132686in}{1.687416in}}%
\pgfpathcurveto{\pgfqpoint{1.124450in}{1.687416in}}{\pgfqpoint{1.116550in}{1.684144in}}{\pgfqpoint{1.110726in}{1.678320in}}%
\pgfpathcurveto{\pgfqpoint{1.104902in}{1.672496in}}{\pgfqpoint{1.101630in}{1.664596in}}{\pgfqpoint{1.101630in}{1.656360in}}%
\pgfpathcurveto{\pgfqpoint{1.101630in}{1.648123in}}{\pgfqpoint{1.104902in}{1.640223in}}{\pgfqpoint{1.110726in}{1.634399in}}%
\pgfpathcurveto{\pgfqpoint{1.116550in}{1.628575in}}{\pgfqpoint{1.124450in}{1.625303in}}{\pgfqpoint{1.132686in}{1.625303in}}%
\pgfpathclose%
\pgfusepath{stroke,fill}%
\end{pgfscope}%
\begin{pgfscope}%
\pgfpathrectangle{\pgfqpoint{0.100000in}{0.220728in}}{\pgfqpoint{3.696000in}{3.696000in}}%
\pgfusepath{clip}%
\pgfsetbuttcap%
\pgfsetroundjoin%
\definecolor{currentfill}{rgb}{0.121569,0.466667,0.705882}%
\pgfsetfillcolor{currentfill}%
\pgfsetfillopacity{0.558664}%
\pgfsetlinewidth{1.003750pt}%
\definecolor{currentstroke}{rgb}{0.121569,0.466667,0.705882}%
\pgfsetstrokecolor{currentstroke}%
\pgfsetstrokeopacity{0.558664}%
\pgfsetdash{}{0pt}%
\pgfpathmoveto{\pgfqpoint{1.124330in}{1.620915in}}%
\pgfpathcurveto{\pgfqpoint{1.132566in}{1.620915in}}{\pgfqpoint{1.140466in}{1.624187in}}{\pgfqpoint{1.146290in}{1.630011in}}%
\pgfpathcurveto{\pgfqpoint{1.152114in}{1.635835in}}{\pgfqpoint{1.155387in}{1.643735in}}{\pgfqpoint{1.155387in}{1.651971in}}%
\pgfpathcurveto{\pgfqpoint{1.155387in}{1.660207in}}{\pgfqpoint{1.152114in}{1.668107in}}{\pgfqpoint{1.146290in}{1.673931in}}%
\pgfpathcurveto{\pgfqpoint{1.140466in}{1.679755in}}{\pgfqpoint{1.132566in}{1.683028in}}{\pgfqpoint{1.124330in}{1.683028in}}%
\pgfpathcurveto{\pgfqpoint{1.116094in}{1.683028in}}{\pgfqpoint{1.108194in}{1.679755in}}{\pgfqpoint{1.102370in}{1.673931in}}%
\pgfpathcurveto{\pgfqpoint{1.096546in}{1.668107in}}{\pgfqpoint{1.093274in}{1.660207in}}{\pgfqpoint{1.093274in}{1.651971in}}%
\pgfpathcurveto{\pgfqpoint{1.093274in}{1.643735in}}{\pgfqpoint{1.096546in}{1.635835in}}{\pgfqpoint{1.102370in}{1.630011in}}%
\pgfpathcurveto{\pgfqpoint{1.108194in}{1.624187in}}{\pgfqpoint{1.116094in}{1.620915in}}{\pgfqpoint{1.124330in}{1.620915in}}%
\pgfpathclose%
\pgfusepath{stroke,fill}%
\end{pgfscope}%
\begin{pgfscope}%
\pgfpathrectangle{\pgfqpoint{0.100000in}{0.220728in}}{\pgfqpoint{3.696000in}{3.696000in}}%
\pgfusepath{clip}%
\pgfsetbuttcap%
\pgfsetroundjoin%
\definecolor{currentfill}{rgb}{0.121569,0.466667,0.705882}%
\pgfsetfillcolor{currentfill}%
\pgfsetfillopacity{0.558853}%
\pgfsetlinewidth{1.003750pt}%
\definecolor{currentstroke}{rgb}{0.121569,0.466667,0.705882}%
\pgfsetstrokecolor{currentstroke}%
\pgfsetstrokeopacity{0.558853}%
\pgfsetdash{}{0pt}%
\pgfpathmoveto{\pgfqpoint{2.083238in}{1.835326in}}%
\pgfpathcurveto{\pgfqpoint{2.091474in}{1.835326in}}{\pgfqpoint{2.099374in}{1.838598in}}{\pgfqpoint{2.105198in}{1.844422in}}%
\pgfpathcurveto{\pgfqpoint{2.111022in}{1.850246in}}{\pgfqpoint{2.114294in}{1.858146in}}{\pgfqpoint{2.114294in}{1.866382in}}%
\pgfpathcurveto{\pgfqpoint{2.114294in}{1.874619in}}{\pgfqpoint{2.111022in}{1.882519in}}{\pgfqpoint{2.105198in}{1.888343in}}%
\pgfpathcurveto{\pgfqpoint{2.099374in}{1.894166in}}{\pgfqpoint{2.091474in}{1.897439in}}{\pgfqpoint{2.083238in}{1.897439in}}%
\pgfpathcurveto{\pgfqpoint{2.075002in}{1.897439in}}{\pgfqpoint{2.067102in}{1.894166in}}{\pgfqpoint{2.061278in}{1.888343in}}%
\pgfpathcurveto{\pgfqpoint{2.055454in}{1.882519in}}{\pgfqpoint{2.052181in}{1.874619in}}{\pgfqpoint{2.052181in}{1.866382in}}%
\pgfpathcurveto{\pgfqpoint{2.052181in}{1.858146in}}{\pgfqpoint{2.055454in}{1.850246in}}{\pgfqpoint{2.061278in}{1.844422in}}%
\pgfpathcurveto{\pgfqpoint{2.067102in}{1.838598in}}{\pgfqpoint{2.075002in}{1.835326in}}{\pgfqpoint{2.083238in}{1.835326in}}%
\pgfpathclose%
\pgfusepath{stroke,fill}%
\end{pgfscope}%
\begin{pgfscope}%
\pgfpathrectangle{\pgfqpoint{0.100000in}{0.220728in}}{\pgfqpoint{3.696000in}{3.696000in}}%
\pgfusepath{clip}%
\pgfsetbuttcap%
\pgfsetroundjoin%
\definecolor{currentfill}{rgb}{0.121569,0.466667,0.705882}%
\pgfsetfillcolor{currentfill}%
\pgfsetfillopacity{0.561683}%
\pgfsetlinewidth{1.003750pt}%
\definecolor{currentstroke}{rgb}{0.121569,0.466667,0.705882}%
\pgfsetstrokecolor{currentstroke}%
\pgfsetstrokeopacity{0.561683}%
\pgfsetdash{}{0pt}%
\pgfpathmoveto{\pgfqpoint{1.120203in}{1.623062in}}%
\pgfpathcurveto{\pgfqpoint{1.128439in}{1.623062in}}{\pgfqpoint{1.136339in}{1.626334in}}{\pgfqpoint{1.142163in}{1.632158in}}%
\pgfpathcurveto{\pgfqpoint{1.147987in}{1.637982in}}{\pgfqpoint{1.151260in}{1.645882in}}{\pgfqpoint{1.151260in}{1.654118in}}%
\pgfpathcurveto{\pgfqpoint{1.151260in}{1.662355in}}{\pgfqpoint{1.147987in}{1.670255in}}{\pgfqpoint{1.142163in}{1.676079in}}%
\pgfpathcurveto{\pgfqpoint{1.136339in}{1.681903in}}{\pgfqpoint{1.128439in}{1.685175in}}{\pgfqpoint{1.120203in}{1.685175in}}%
\pgfpathcurveto{\pgfqpoint{1.111967in}{1.685175in}}{\pgfqpoint{1.104067in}{1.681903in}}{\pgfqpoint{1.098243in}{1.676079in}}%
\pgfpathcurveto{\pgfqpoint{1.092419in}{1.670255in}}{\pgfqpoint{1.089147in}{1.662355in}}{\pgfqpoint{1.089147in}{1.654118in}}%
\pgfpathcurveto{\pgfqpoint{1.089147in}{1.645882in}}{\pgfqpoint{1.092419in}{1.637982in}}{\pgfqpoint{1.098243in}{1.632158in}}%
\pgfpathcurveto{\pgfqpoint{1.104067in}{1.626334in}}{\pgfqpoint{1.111967in}{1.623062in}}{\pgfqpoint{1.120203in}{1.623062in}}%
\pgfpathclose%
\pgfusepath{stroke,fill}%
\end{pgfscope}%
\begin{pgfscope}%
\pgfpathrectangle{\pgfqpoint{0.100000in}{0.220728in}}{\pgfqpoint{3.696000in}{3.696000in}}%
\pgfusepath{clip}%
\pgfsetbuttcap%
\pgfsetroundjoin%
\definecolor{currentfill}{rgb}{0.121569,0.466667,0.705882}%
\pgfsetfillcolor{currentfill}%
\pgfsetfillopacity{0.562942}%
\pgfsetlinewidth{1.003750pt}%
\definecolor{currentstroke}{rgb}{0.121569,0.466667,0.705882}%
\pgfsetstrokecolor{currentstroke}%
\pgfsetstrokeopacity{0.562942}%
\pgfsetdash{}{0pt}%
\pgfpathmoveto{\pgfqpoint{1.113944in}{1.618168in}}%
\pgfpathcurveto{\pgfqpoint{1.122180in}{1.618168in}}{\pgfqpoint{1.130080in}{1.621440in}}{\pgfqpoint{1.135904in}{1.627264in}}%
\pgfpathcurveto{\pgfqpoint{1.141728in}{1.633088in}}{\pgfqpoint{1.145001in}{1.640988in}}{\pgfqpoint{1.145001in}{1.649225in}}%
\pgfpathcurveto{\pgfqpoint{1.145001in}{1.657461in}}{\pgfqpoint{1.141728in}{1.665361in}}{\pgfqpoint{1.135904in}{1.671185in}}%
\pgfpathcurveto{\pgfqpoint{1.130080in}{1.677009in}}{\pgfqpoint{1.122180in}{1.680281in}}{\pgfqpoint{1.113944in}{1.680281in}}%
\pgfpathcurveto{\pgfqpoint{1.105708in}{1.680281in}}{\pgfqpoint{1.097808in}{1.677009in}}{\pgfqpoint{1.091984in}{1.671185in}}%
\pgfpathcurveto{\pgfqpoint{1.086160in}{1.665361in}}{\pgfqpoint{1.082888in}{1.657461in}}{\pgfqpoint{1.082888in}{1.649225in}}%
\pgfpathcurveto{\pgfqpoint{1.082888in}{1.640988in}}{\pgfqpoint{1.086160in}{1.633088in}}{\pgfqpoint{1.091984in}{1.627264in}}%
\pgfpathcurveto{\pgfqpoint{1.097808in}{1.621440in}}{\pgfqpoint{1.105708in}{1.618168in}}{\pgfqpoint{1.113944in}{1.618168in}}%
\pgfpathclose%
\pgfusepath{stroke,fill}%
\end{pgfscope}%
\begin{pgfscope}%
\pgfpathrectangle{\pgfqpoint{0.100000in}{0.220728in}}{\pgfqpoint{3.696000in}{3.696000in}}%
\pgfusepath{clip}%
\pgfsetbuttcap%
\pgfsetroundjoin%
\definecolor{currentfill}{rgb}{0.121569,0.466667,0.705882}%
\pgfsetfillcolor{currentfill}%
\pgfsetfillopacity{0.564165}%
\pgfsetlinewidth{1.003750pt}%
\definecolor{currentstroke}{rgb}{0.121569,0.466667,0.705882}%
\pgfsetstrokecolor{currentstroke}%
\pgfsetstrokeopacity{0.564165}%
\pgfsetdash{}{0pt}%
\pgfpathmoveto{\pgfqpoint{2.084304in}{1.831964in}}%
\pgfpathcurveto{\pgfqpoint{2.092541in}{1.831964in}}{\pgfqpoint{2.100441in}{1.835236in}}{\pgfqpoint{2.106265in}{1.841060in}}%
\pgfpathcurveto{\pgfqpoint{2.112089in}{1.846884in}}{\pgfqpoint{2.115361in}{1.854784in}}{\pgfqpoint{2.115361in}{1.863021in}}%
\pgfpathcurveto{\pgfqpoint{2.115361in}{1.871257in}}{\pgfqpoint{2.112089in}{1.879157in}}{\pgfqpoint{2.106265in}{1.884981in}}%
\pgfpathcurveto{\pgfqpoint{2.100441in}{1.890805in}}{\pgfqpoint{2.092541in}{1.894077in}}{\pgfqpoint{2.084304in}{1.894077in}}%
\pgfpathcurveto{\pgfqpoint{2.076068in}{1.894077in}}{\pgfqpoint{2.068168in}{1.890805in}}{\pgfqpoint{2.062344in}{1.884981in}}%
\pgfpathcurveto{\pgfqpoint{2.056520in}{1.879157in}}{\pgfqpoint{2.053248in}{1.871257in}}{\pgfqpoint{2.053248in}{1.863021in}}%
\pgfpathcurveto{\pgfqpoint{2.053248in}{1.854784in}}{\pgfqpoint{2.056520in}{1.846884in}}{\pgfqpoint{2.062344in}{1.841060in}}%
\pgfpathcurveto{\pgfqpoint{2.068168in}{1.835236in}}{\pgfqpoint{2.076068in}{1.831964in}}{\pgfqpoint{2.084304in}{1.831964in}}%
\pgfpathclose%
\pgfusepath{stroke,fill}%
\end{pgfscope}%
\begin{pgfscope}%
\pgfpathrectangle{\pgfqpoint{0.100000in}{0.220728in}}{\pgfqpoint{3.696000in}{3.696000in}}%
\pgfusepath{clip}%
\pgfsetbuttcap%
\pgfsetroundjoin%
\definecolor{currentfill}{rgb}{0.121569,0.466667,0.705882}%
\pgfsetfillcolor{currentfill}%
\pgfsetfillopacity{0.565604}%
\pgfsetlinewidth{1.003750pt}%
\definecolor{currentstroke}{rgb}{0.121569,0.466667,0.705882}%
\pgfsetstrokecolor{currentstroke}%
\pgfsetstrokeopacity{0.565604}%
\pgfsetdash{}{0pt}%
\pgfpathmoveto{\pgfqpoint{1.104023in}{1.609049in}}%
\pgfpathcurveto{\pgfqpoint{1.112259in}{1.609049in}}{\pgfqpoint{1.120159in}{1.612322in}}{\pgfqpoint{1.125983in}{1.618146in}}%
\pgfpathcurveto{\pgfqpoint{1.131807in}{1.623970in}}{\pgfqpoint{1.135079in}{1.631870in}}{\pgfqpoint{1.135079in}{1.640106in}}%
\pgfpathcurveto{\pgfqpoint{1.135079in}{1.648342in}}{\pgfqpoint{1.131807in}{1.656242in}}{\pgfqpoint{1.125983in}{1.662066in}}%
\pgfpathcurveto{\pgfqpoint{1.120159in}{1.667890in}}{\pgfqpoint{1.112259in}{1.671162in}}{\pgfqpoint{1.104023in}{1.671162in}}%
\pgfpathcurveto{\pgfqpoint{1.095787in}{1.671162in}}{\pgfqpoint{1.087887in}{1.667890in}}{\pgfqpoint{1.082063in}{1.662066in}}%
\pgfpathcurveto{\pgfqpoint{1.076239in}{1.656242in}}{\pgfqpoint{1.072966in}{1.648342in}}{\pgfqpoint{1.072966in}{1.640106in}}%
\pgfpathcurveto{\pgfqpoint{1.072966in}{1.631870in}}{\pgfqpoint{1.076239in}{1.623970in}}{\pgfqpoint{1.082063in}{1.618146in}}%
\pgfpathcurveto{\pgfqpoint{1.087887in}{1.612322in}}{\pgfqpoint{1.095787in}{1.609049in}}{\pgfqpoint{1.104023in}{1.609049in}}%
\pgfpathclose%
\pgfusepath{stroke,fill}%
\end{pgfscope}%
\begin{pgfscope}%
\pgfpathrectangle{\pgfqpoint{0.100000in}{0.220728in}}{\pgfqpoint{3.696000in}{3.696000in}}%
\pgfusepath{clip}%
\pgfsetbuttcap%
\pgfsetroundjoin%
\definecolor{currentfill}{rgb}{0.121569,0.466667,0.705882}%
\pgfsetfillcolor{currentfill}%
\pgfsetfillopacity{0.569427}%
\pgfsetlinewidth{1.003750pt}%
\definecolor{currentstroke}{rgb}{0.121569,0.466667,0.705882}%
\pgfsetstrokecolor{currentstroke}%
\pgfsetstrokeopacity{0.569427}%
\pgfsetdash{}{0pt}%
\pgfpathmoveto{\pgfqpoint{1.096668in}{1.608965in}}%
\pgfpathcurveto{\pgfqpoint{1.104904in}{1.608965in}}{\pgfqpoint{1.112804in}{1.612237in}}{\pgfqpoint{1.118628in}{1.618061in}}%
\pgfpathcurveto{\pgfqpoint{1.124452in}{1.623885in}}{\pgfqpoint{1.127725in}{1.631785in}}{\pgfqpoint{1.127725in}{1.640021in}}%
\pgfpathcurveto{\pgfqpoint{1.127725in}{1.648258in}}{\pgfqpoint{1.124452in}{1.656158in}}{\pgfqpoint{1.118628in}{1.661982in}}%
\pgfpathcurveto{\pgfqpoint{1.112804in}{1.667806in}}{\pgfqpoint{1.104904in}{1.671078in}}{\pgfqpoint{1.096668in}{1.671078in}}%
\pgfpathcurveto{\pgfqpoint{1.088432in}{1.671078in}}{\pgfqpoint{1.080532in}{1.667806in}}{\pgfqpoint{1.074708in}{1.661982in}}%
\pgfpathcurveto{\pgfqpoint{1.068884in}{1.656158in}}{\pgfqpoint{1.065612in}{1.648258in}}{\pgfqpoint{1.065612in}{1.640021in}}%
\pgfpathcurveto{\pgfqpoint{1.065612in}{1.631785in}}{\pgfqpoint{1.068884in}{1.623885in}}{\pgfqpoint{1.074708in}{1.618061in}}%
\pgfpathcurveto{\pgfqpoint{1.080532in}{1.612237in}}{\pgfqpoint{1.088432in}{1.608965in}}{\pgfqpoint{1.096668in}{1.608965in}}%
\pgfpathclose%
\pgfusepath{stroke,fill}%
\end{pgfscope}%
\begin{pgfscope}%
\pgfpathrectangle{\pgfqpoint{0.100000in}{0.220728in}}{\pgfqpoint{3.696000in}{3.696000in}}%
\pgfusepath{clip}%
\pgfsetbuttcap%
\pgfsetroundjoin%
\definecolor{currentfill}{rgb}{0.121569,0.466667,0.705882}%
\pgfsetfillcolor{currentfill}%
\pgfsetfillopacity{0.569998}%
\pgfsetlinewidth{1.003750pt}%
\definecolor{currentstroke}{rgb}{0.121569,0.466667,0.705882}%
\pgfsetstrokecolor{currentstroke}%
\pgfsetstrokeopacity{0.569998}%
\pgfsetdash{}{0pt}%
\pgfpathmoveto{\pgfqpoint{2.088749in}{1.827221in}}%
\pgfpathcurveto{\pgfqpoint{2.096985in}{1.827221in}}{\pgfqpoint{2.104885in}{1.830494in}}{\pgfqpoint{2.110709in}{1.836317in}}%
\pgfpathcurveto{\pgfqpoint{2.116533in}{1.842141in}}{\pgfqpoint{2.119806in}{1.850041in}}{\pgfqpoint{2.119806in}{1.858278in}}%
\pgfpathcurveto{\pgfqpoint{2.119806in}{1.866514in}}{\pgfqpoint{2.116533in}{1.874414in}}{\pgfqpoint{2.110709in}{1.880238in}}%
\pgfpathcurveto{\pgfqpoint{2.104885in}{1.886062in}}{\pgfqpoint{2.096985in}{1.889334in}}{\pgfqpoint{2.088749in}{1.889334in}}%
\pgfpathcurveto{\pgfqpoint{2.080513in}{1.889334in}}{\pgfqpoint{2.072613in}{1.886062in}}{\pgfqpoint{2.066789in}{1.880238in}}%
\pgfpathcurveto{\pgfqpoint{2.060965in}{1.874414in}}{\pgfqpoint{2.057693in}{1.866514in}}{\pgfqpoint{2.057693in}{1.858278in}}%
\pgfpathcurveto{\pgfqpoint{2.057693in}{1.850041in}}{\pgfqpoint{2.060965in}{1.842141in}}{\pgfqpoint{2.066789in}{1.836317in}}%
\pgfpathcurveto{\pgfqpoint{2.072613in}{1.830494in}}{\pgfqpoint{2.080513in}{1.827221in}}{\pgfqpoint{2.088749in}{1.827221in}}%
\pgfpathclose%
\pgfusepath{stroke,fill}%
\end{pgfscope}%
\begin{pgfscope}%
\pgfpathrectangle{\pgfqpoint{0.100000in}{0.220728in}}{\pgfqpoint{3.696000in}{3.696000in}}%
\pgfusepath{clip}%
\pgfsetbuttcap%
\pgfsetroundjoin%
\definecolor{currentfill}{rgb}{0.121569,0.466667,0.705882}%
\pgfsetfillcolor{currentfill}%
\pgfsetfillopacity{0.570652}%
\pgfsetlinewidth{1.003750pt}%
\definecolor{currentstroke}{rgb}{0.121569,0.466667,0.705882}%
\pgfsetstrokecolor{currentstroke}%
\pgfsetstrokeopacity{0.570652}%
\pgfsetdash{}{0pt}%
\pgfpathmoveto{\pgfqpoint{1.087622in}{1.602512in}}%
\pgfpathcurveto{\pgfqpoint{1.095858in}{1.602512in}}{\pgfqpoint{1.103758in}{1.605784in}}{\pgfqpoint{1.109582in}{1.611608in}}%
\pgfpathcurveto{\pgfqpoint{1.115406in}{1.617432in}}{\pgfqpoint{1.118678in}{1.625332in}}{\pgfqpoint{1.118678in}{1.633569in}}%
\pgfpathcurveto{\pgfqpoint{1.118678in}{1.641805in}}{\pgfqpoint{1.115406in}{1.649705in}}{\pgfqpoint{1.109582in}{1.655529in}}%
\pgfpathcurveto{\pgfqpoint{1.103758in}{1.661353in}}{\pgfqpoint{1.095858in}{1.664625in}}{\pgfqpoint{1.087622in}{1.664625in}}%
\pgfpathcurveto{\pgfqpoint{1.079385in}{1.664625in}}{\pgfqpoint{1.071485in}{1.661353in}}{\pgfqpoint{1.065661in}{1.655529in}}%
\pgfpathcurveto{\pgfqpoint{1.059837in}{1.649705in}}{\pgfqpoint{1.056565in}{1.641805in}}{\pgfqpoint{1.056565in}{1.633569in}}%
\pgfpathcurveto{\pgfqpoint{1.056565in}{1.625332in}}{\pgfqpoint{1.059837in}{1.617432in}}{\pgfqpoint{1.065661in}{1.611608in}}%
\pgfpathcurveto{\pgfqpoint{1.071485in}{1.605784in}}{\pgfqpoint{1.079385in}{1.602512in}}{\pgfqpoint{1.087622in}{1.602512in}}%
\pgfpathclose%
\pgfusepath{stroke,fill}%
\end{pgfscope}%
\begin{pgfscope}%
\pgfpathrectangle{\pgfqpoint{0.100000in}{0.220728in}}{\pgfqpoint{3.696000in}{3.696000in}}%
\pgfusepath{clip}%
\pgfsetbuttcap%
\pgfsetroundjoin%
\definecolor{currentfill}{rgb}{0.121569,0.466667,0.705882}%
\pgfsetfillcolor{currentfill}%
\pgfsetfillopacity{0.570978}%
\pgfsetlinewidth{1.003750pt}%
\definecolor{currentstroke}{rgb}{0.121569,0.466667,0.705882}%
\pgfsetstrokecolor{currentstroke}%
\pgfsetstrokeopacity{0.570978}%
\pgfsetdash{}{0pt}%
\pgfpathmoveto{\pgfqpoint{1.079664in}{1.589669in}}%
\pgfpathcurveto{\pgfqpoint{1.087901in}{1.589669in}}{\pgfqpoint{1.095801in}{1.592942in}}{\pgfqpoint{1.101625in}{1.598766in}}%
\pgfpathcurveto{\pgfqpoint{1.107449in}{1.604590in}}{\pgfqpoint{1.110721in}{1.612490in}}{\pgfqpoint{1.110721in}{1.620726in}}%
\pgfpathcurveto{\pgfqpoint{1.110721in}{1.628962in}}{\pgfqpoint{1.107449in}{1.636862in}}{\pgfqpoint{1.101625in}{1.642686in}}%
\pgfpathcurveto{\pgfqpoint{1.095801in}{1.648510in}}{\pgfqpoint{1.087901in}{1.651782in}}{\pgfqpoint{1.079664in}{1.651782in}}%
\pgfpathcurveto{\pgfqpoint{1.071428in}{1.651782in}}{\pgfqpoint{1.063528in}{1.648510in}}{\pgfqpoint{1.057704in}{1.642686in}}%
\pgfpathcurveto{\pgfqpoint{1.051880in}{1.636862in}}{\pgfqpoint{1.048608in}{1.628962in}}{\pgfqpoint{1.048608in}{1.620726in}}%
\pgfpathcurveto{\pgfqpoint{1.048608in}{1.612490in}}{\pgfqpoint{1.051880in}{1.604590in}}{\pgfqpoint{1.057704in}{1.598766in}}%
\pgfpathcurveto{\pgfqpoint{1.063528in}{1.592942in}}{\pgfqpoint{1.071428in}{1.589669in}}{\pgfqpoint{1.079664in}{1.589669in}}%
\pgfpathclose%
\pgfusepath{stroke,fill}%
\end{pgfscope}%
\begin{pgfscope}%
\pgfpathrectangle{\pgfqpoint{0.100000in}{0.220728in}}{\pgfqpoint{3.696000in}{3.696000in}}%
\pgfusepath{clip}%
\pgfsetbuttcap%
\pgfsetroundjoin%
\definecolor{currentfill}{rgb}{0.121569,0.466667,0.705882}%
\pgfsetfillcolor{currentfill}%
\pgfsetfillopacity{0.573191}%
\pgfsetlinewidth{1.003750pt}%
\definecolor{currentstroke}{rgb}{0.121569,0.466667,0.705882}%
\pgfsetstrokecolor{currentstroke}%
\pgfsetstrokeopacity{0.573191}%
\pgfsetdash{}{0pt}%
\pgfpathmoveto{\pgfqpoint{2.090641in}{1.824173in}}%
\pgfpathcurveto{\pgfqpoint{2.098877in}{1.824173in}}{\pgfqpoint{2.106777in}{1.827445in}}{\pgfqpoint{2.112601in}{1.833269in}}%
\pgfpathcurveto{\pgfqpoint{2.118425in}{1.839093in}}{\pgfqpoint{2.121697in}{1.846993in}}{\pgfqpoint{2.121697in}{1.855230in}}%
\pgfpathcurveto{\pgfqpoint{2.121697in}{1.863466in}}{\pgfqpoint{2.118425in}{1.871366in}}{\pgfqpoint{2.112601in}{1.877190in}}%
\pgfpathcurveto{\pgfqpoint{2.106777in}{1.883014in}}{\pgfqpoint{2.098877in}{1.886286in}}{\pgfqpoint{2.090641in}{1.886286in}}%
\pgfpathcurveto{\pgfqpoint{2.082405in}{1.886286in}}{\pgfqpoint{2.074505in}{1.883014in}}{\pgfqpoint{2.068681in}{1.877190in}}%
\pgfpathcurveto{\pgfqpoint{2.062857in}{1.871366in}}{\pgfqpoint{2.059584in}{1.863466in}}{\pgfqpoint{2.059584in}{1.855230in}}%
\pgfpathcurveto{\pgfqpoint{2.059584in}{1.846993in}}{\pgfqpoint{2.062857in}{1.839093in}}{\pgfqpoint{2.068681in}{1.833269in}}%
\pgfpathcurveto{\pgfqpoint{2.074505in}{1.827445in}}{\pgfqpoint{2.082405in}{1.824173in}}{\pgfqpoint{2.090641in}{1.824173in}}%
\pgfpathclose%
\pgfusepath{stroke,fill}%
\end{pgfscope}%
\begin{pgfscope}%
\pgfpathrectangle{\pgfqpoint{0.100000in}{0.220728in}}{\pgfqpoint{3.696000in}{3.696000in}}%
\pgfusepath{clip}%
\pgfsetbuttcap%
\pgfsetroundjoin%
\definecolor{currentfill}{rgb}{0.121569,0.466667,0.705882}%
\pgfsetfillcolor{currentfill}%
\pgfsetfillopacity{0.573281}%
\pgfsetlinewidth{1.003750pt}%
\definecolor{currentstroke}{rgb}{0.121569,0.466667,0.705882}%
\pgfsetstrokecolor{currentstroke}%
\pgfsetstrokeopacity{0.573281}%
\pgfsetdash{}{0pt}%
\pgfpathmoveto{\pgfqpoint{1.074572in}{1.587481in}}%
\pgfpathcurveto{\pgfqpoint{1.082808in}{1.587481in}}{\pgfqpoint{1.090708in}{1.590754in}}{\pgfqpoint{1.096532in}{1.596578in}}%
\pgfpathcurveto{\pgfqpoint{1.102356in}{1.602402in}}{\pgfqpoint{1.105628in}{1.610302in}}{\pgfqpoint{1.105628in}{1.618538in}}%
\pgfpathcurveto{\pgfqpoint{1.105628in}{1.626774in}}{\pgfqpoint{1.102356in}{1.634674in}}{\pgfqpoint{1.096532in}{1.640498in}}%
\pgfpathcurveto{\pgfqpoint{1.090708in}{1.646322in}}{\pgfqpoint{1.082808in}{1.649594in}}{\pgfqpoint{1.074572in}{1.649594in}}%
\pgfpathcurveto{\pgfqpoint{1.066335in}{1.649594in}}{\pgfqpoint{1.058435in}{1.646322in}}{\pgfqpoint{1.052611in}{1.640498in}}%
\pgfpathcurveto{\pgfqpoint{1.046787in}{1.634674in}}{\pgfqpoint{1.043515in}{1.626774in}}{\pgfqpoint{1.043515in}{1.618538in}}%
\pgfpathcurveto{\pgfqpoint{1.043515in}{1.610302in}}{\pgfqpoint{1.046787in}{1.602402in}}{\pgfqpoint{1.052611in}{1.596578in}}%
\pgfpathcurveto{\pgfqpoint{1.058435in}{1.590754in}}{\pgfqpoint{1.066335in}{1.587481in}}{\pgfqpoint{1.074572in}{1.587481in}}%
\pgfpathclose%
\pgfusepath{stroke,fill}%
\end{pgfscope}%
\begin{pgfscope}%
\pgfpathrectangle{\pgfqpoint{0.100000in}{0.220728in}}{\pgfqpoint{3.696000in}{3.696000in}}%
\pgfusepath{clip}%
\pgfsetbuttcap%
\pgfsetroundjoin%
\definecolor{currentfill}{rgb}{0.121569,0.466667,0.705882}%
\pgfsetfillcolor{currentfill}%
\pgfsetfillopacity{0.574192}%
\pgfsetlinewidth{1.003750pt}%
\definecolor{currentstroke}{rgb}{0.121569,0.466667,0.705882}%
\pgfsetstrokecolor{currentstroke}%
\pgfsetstrokeopacity{0.574192}%
\pgfsetdash{}{0pt}%
\pgfpathmoveto{\pgfqpoint{1.070147in}{1.583062in}}%
\pgfpathcurveto{\pgfqpoint{1.078383in}{1.583062in}}{\pgfqpoint{1.086283in}{1.586335in}}{\pgfqpoint{1.092107in}{1.592158in}}%
\pgfpathcurveto{\pgfqpoint{1.097931in}{1.597982in}}{\pgfqpoint{1.101203in}{1.605882in}}{\pgfqpoint{1.101203in}{1.614119in}}%
\pgfpathcurveto{\pgfqpoint{1.101203in}{1.622355in}}{\pgfqpoint{1.097931in}{1.630255in}}{\pgfqpoint{1.092107in}{1.636079in}}%
\pgfpathcurveto{\pgfqpoint{1.086283in}{1.641903in}}{\pgfqpoint{1.078383in}{1.645175in}}{\pgfqpoint{1.070147in}{1.645175in}}%
\pgfpathcurveto{\pgfqpoint{1.061911in}{1.645175in}}{\pgfqpoint{1.054010in}{1.641903in}}{\pgfqpoint{1.048187in}{1.636079in}}%
\pgfpathcurveto{\pgfqpoint{1.042363in}{1.630255in}}{\pgfqpoint{1.039090in}{1.622355in}}{\pgfqpoint{1.039090in}{1.614119in}}%
\pgfpathcurveto{\pgfqpoint{1.039090in}{1.605882in}}{\pgfqpoint{1.042363in}{1.597982in}}{\pgfqpoint{1.048187in}{1.592158in}}%
\pgfpathcurveto{\pgfqpoint{1.054010in}{1.586335in}}{\pgfqpoint{1.061911in}{1.583062in}}{\pgfqpoint{1.070147in}{1.583062in}}%
\pgfpathclose%
\pgfusepath{stroke,fill}%
\end{pgfscope}%
\begin{pgfscope}%
\pgfpathrectangle{\pgfqpoint{0.100000in}{0.220728in}}{\pgfqpoint{3.696000in}{3.696000in}}%
\pgfusepath{clip}%
\pgfsetbuttcap%
\pgfsetroundjoin%
\definecolor{currentfill}{rgb}{0.121569,0.466667,0.705882}%
\pgfsetfillcolor{currentfill}%
\pgfsetfillopacity{0.576011}%
\pgfsetlinewidth{1.003750pt}%
\definecolor{currentstroke}{rgb}{0.121569,0.466667,0.705882}%
\pgfsetstrokecolor{currentstroke}%
\pgfsetstrokeopacity{0.576011}%
\pgfsetdash{}{0pt}%
\pgfpathmoveto{\pgfqpoint{1.062834in}{1.574867in}}%
\pgfpathcurveto{\pgfqpoint{1.071071in}{1.574867in}}{\pgfqpoint{1.078971in}{1.578139in}}{\pgfqpoint{1.084794in}{1.583963in}}%
\pgfpathcurveto{\pgfqpoint{1.090618in}{1.589787in}}{\pgfqpoint{1.093891in}{1.597687in}}{\pgfqpoint{1.093891in}{1.605923in}}%
\pgfpathcurveto{\pgfqpoint{1.093891in}{1.614159in}}{\pgfqpoint{1.090618in}{1.622060in}}{\pgfqpoint{1.084794in}{1.627883in}}%
\pgfpathcurveto{\pgfqpoint{1.078971in}{1.633707in}}{\pgfqpoint{1.071071in}{1.636980in}}{\pgfqpoint{1.062834in}{1.636980in}}%
\pgfpathcurveto{\pgfqpoint{1.054598in}{1.636980in}}{\pgfqpoint{1.046698in}{1.633707in}}{\pgfqpoint{1.040874in}{1.627883in}}%
\pgfpathcurveto{\pgfqpoint{1.035050in}{1.622060in}}{\pgfqpoint{1.031778in}{1.614159in}}{\pgfqpoint{1.031778in}{1.605923in}}%
\pgfpathcurveto{\pgfqpoint{1.031778in}{1.597687in}}{\pgfqpoint{1.035050in}{1.589787in}}{\pgfqpoint{1.040874in}{1.583963in}}%
\pgfpathcurveto{\pgfqpoint{1.046698in}{1.578139in}}{\pgfqpoint{1.054598in}{1.574867in}}{\pgfqpoint{1.062834in}{1.574867in}}%
\pgfpathclose%
\pgfusepath{stroke,fill}%
\end{pgfscope}%
\begin{pgfscope}%
\pgfpathrectangle{\pgfqpoint{0.100000in}{0.220728in}}{\pgfqpoint{3.696000in}{3.696000in}}%
\pgfusepath{clip}%
\pgfsetbuttcap%
\pgfsetroundjoin%
\definecolor{currentfill}{rgb}{0.121569,0.466667,0.705882}%
\pgfsetfillcolor{currentfill}%
\pgfsetfillopacity{0.577196}%
\pgfsetlinewidth{1.003750pt}%
\definecolor{currentstroke}{rgb}{0.121569,0.466667,0.705882}%
\pgfsetstrokecolor{currentstroke}%
\pgfsetstrokeopacity{0.577196}%
\pgfsetdash{}{0pt}%
\pgfpathmoveto{\pgfqpoint{2.093075in}{1.821934in}}%
\pgfpathcurveto{\pgfqpoint{2.101312in}{1.821934in}}{\pgfqpoint{2.109212in}{1.825206in}}{\pgfqpoint{2.115036in}{1.831030in}}%
\pgfpathcurveto{\pgfqpoint{2.120859in}{1.836854in}}{\pgfqpoint{2.124132in}{1.844754in}}{\pgfqpoint{2.124132in}{1.852990in}}%
\pgfpathcurveto{\pgfqpoint{2.124132in}{1.861227in}}{\pgfqpoint{2.120859in}{1.869127in}}{\pgfqpoint{2.115036in}{1.874950in}}%
\pgfpathcurveto{\pgfqpoint{2.109212in}{1.880774in}}{\pgfqpoint{2.101312in}{1.884047in}}{\pgfqpoint{2.093075in}{1.884047in}}%
\pgfpathcurveto{\pgfqpoint{2.084839in}{1.884047in}}{\pgfqpoint{2.076939in}{1.880774in}}{\pgfqpoint{2.071115in}{1.874950in}}%
\pgfpathcurveto{\pgfqpoint{2.065291in}{1.869127in}}{\pgfqpoint{2.062019in}{1.861227in}}{\pgfqpoint{2.062019in}{1.852990in}}%
\pgfpathcurveto{\pgfqpoint{2.062019in}{1.844754in}}{\pgfqpoint{2.065291in}{1.836854in}}{\pgfqpoint{2.071115in}{1.831030in}}%
\pgfpathcurveto{\pgfqpoint{2.076939in}{1.825206in}}{\pgfqpoint{2.084839in}{1.821934in}}{\pgfqpoint{2.093075in}{1.821934in}}%
\pgfpathclose%
\pgfusepath{stroke,fill}%
\end{pgfscope}%
\begin{pgfscope}%
\pgfpathrectangle{\pgfqpoint{0.100000in}{0.220728in}}{\pgfqpoint{3.696000in}{3.696000in}}%
\pgfusepath{clip}%
\pgfsetbuttcap%
\pgfsetroundjoin%
\definecolor{currentfill}{rgb}{0.121569,0.466667,0.705882}%
\pgfsetfillcolor{currentfill}%
\pgfsetfillopacity{0.579310}%
\pgfsetlinewidth{1.003750pt}%
\definecolor{currentstroke}{rgb}{0.121569,0.466667,0.705882}%
\pgfsetstrokecolor{currentstroke}%
\pgfsetstrokeopacity{0.579310}%
\pgfsetdash{}{0pt}%
\pgfpathmoveto{\pgfqpoint{2.094220in}{1.820000in}}%
\pgfpathcurveto{\pgfqpoint{2.102457in}{1.820000in}}{\pgfqpoint{2.110357in}{1.823272in}}{\pgfqpoint{2.116181in}{1.829096in}}%
\pgfpathcurveto{\pgfqpoint{2.122005in}{1.834920in}}{\pgfqpoint{2.125277in}{1.842820in}}{\pgfqpoint{2.125277in}{1.851056in}}%
\pgfpathcurveto{\pgfqpoint{2.125277in}{1.859293in}}{\pgfqpoint{2.122005in}{1.867193in}}{\pgfqpoint{2.116181in}{1.873017in}}%
\pgfpathcurveto{\pgfqpoint{2.110357in}{1.878840in}}{\pgfqpoint{2.102457in}{1.882113in}}{\pgfqpoint{2.094220in}{1.882113in}}%
\pgfpathcurveto{\pgfqpoint{2.085984in}{1.882113in}}{\pgfqpoint{2.078084in}{1.878840in}}{\pgfqpoint{2.072260in}{1.873017in}}%
\pgfpathcurveto{\pgfqpoint{2.066436in}{1.867193in}}{\pgfqpoint{2.063164in}{1.859293in}}{\pgfqpoint{2.063164in}{1.851056in}}%
\pgfpathcurveto{\pgfqpoint{2.063164in}{1.842820in}}{\pgfqpoint{2.066436in}{1.834920in}}{\pgfqpoint{2.072260in}{1.829096in}}%
\pgfpathcurveto{\pgfqpoint{2.078084in}{1.823272in}}{\pgfqpoint{2.085984in}{1.820000in}}{\pgfqpoint{2.094220in}{1.820000in}}%
\pgfpathclose%
\pgfusepath{stroke,fill}%
\end{pgfscope}%
\begin{pgfscope}%
\pgfpathrectangle{\pgfqpoint{0.100000in}{0.220728in}}{\pgfqpoint{3.696000in}{3.696000in}}%
\pgfusepath{clip}%
\pgfsetbuttcap%
\pgfsetroundjoin%
\definecolor{currentfill}{rgb}{0.121569,0.466667,0.705882}%
\pgfsetfillcolor{currentfill}%
\pgfsetfillopacity{0.580420}%
\pgfsetlinewidth{1.003750pt}%
\definecolor{currentstroke}{rgb}{0.121569,0.466667,0.705882}%
\pgfsetstrokecolor{currentstroke}%
\pgfsetstrokeopacity{0.580420}%
\pgfsetdash{}{0pt}%
\pgfpathmoveto{\pgfqpoint{2.094828in}{1.818581in}}%
\pgfpathcurveto{\pgfqpoint{2.103064in}{1.818581in}}{\pgfqpoint{2.110964in}{1.821853in}}{\pgfqpoint{2.116788in}{1.827677in}}%
\pgfpathcurveto{\pgfqpoint{2.122612in}{1.833501in}}{\pgfqpoint{2.125884in}{1.841401in}}{\pgfqpoint{2.125884in}{1.849637in}}%
\pgfpathcurveto{\pgfqpoint{2.125884in}{1.857874in}}{\pgfqpoint{2.122612in}{1.865774in}}{\pgfqpoint{2.116788in}{1.871598in}}%
\pgfpathcurveto{\pgfqpoint{2.110964in}{1.877422in}}{\pgfqpoint{2.103064in}{1.880694in}}{\pgfqpoint{2.094828in}{1.880694in}}%
\pgfpathcurveto{\pgfqpoint{2.086591in}{1.880694in}}{\pgfqpoint{2.078691in}{1.877422in}}{\pgfqpoint{2.072867in}{1.871598in}}%
\pgfpathcurveto{\pgfqpoint{2.067043in}{1.865774in}}{\pgfqpoint{2.063771in}{1.857874in}}{\pgfqpoint{2.063771in}{1.849637in}}%
\pgfpathcurveto{\pgfqpoint{2.063771in}{1.841401in}}{\pgfqpoint{2.067043in}{1.833501in}}{\pgfqpoint{2.072867in}{1.827677in}}%
\pgfpathcurveto{\pgfqpoint{2.078691in}{1.821853in}}{\pgfqpoint{2.086591in}{1.818581in}}{\pgfqpoint{2.094828in}{1.818581in}}%
\pgfpathclose%
\pgfusepath{stroke,fill}%
\end{pgfscope}%
\begin{pgfscope}%
\pgfpathrectangle{\pgfqpoint{0.100000in}{0.220728in}}{\pgfqpoint{3.696000in}{3.696000in}}%
\pgfusepath{clip}%
\pgfsetbuttcap%
\pgfsetroundjoin%
\definecolor{currentfill}{rgb}{0.121569,0.466667,0.705882}%
\pgfsetfillcolor{currentfill}%
\pgfsetfillopacity{0.581183}%
\pgfsetlinewidth{1.003750pt}%
\definecolor{currentstroke}{rgb}{0.121569,0.466667,0.705882}%
\pgfsetstrokecolor{currentstroke}%
\pgfsetstrokeopacity{0.581183}%
\pgfsetdash{}{0pt}%
\pgfpathmoveto{\pgfqpoint{1.052356in}{1.568437in}}%
\pgfpathcurveto{\pgfqpoint{1.060592in}{1.568437in}}{\pgfqpoint{1.068492in}{1.571709in}}{\pgfqpoint{1.074316in}{1.577533in}}%
\pgfpathcurveto{\pgfqpoint{1.080140in}{1.583357in}}{\pgfqpoint{1.083412in}{1.591257in}}{\pgfqpoint{1.083412in}{1.599493in}}%
\pgfpathcurveto{\pgfqpoint{1.083412in}{1.607730in}}{\pgfqpoint{1.080140in}{1.615630in}}{\pgfqpoint{1.074316in}{1.621454in}}%
\pgfpathcurveto{\pgfqpoint{1.068492in}{1.627277in}}{\pgfqpoint{1.060592in}{1.630550in}}{\pgfqpoint{1.052356in}{1.630550in}}%
\pgfpathcurveto{\pgfqpoint{1.044120in}{1.630550in}}{\pgfqpoint{1.036219in}{1.627277in}}{\pgfqpoint{1.030396in}{1.621454in}}%
\pgfpathcurveto{\pgfqpoint{1.024572in}{1.615630in}}{\pgfqpoint{1.021299in}{1.607730in}}{\pgfqpoint{1.021299in}{1.599493in}}%
\pgfpathcurveto{\pgfqpoint{1.021299in}{1.591257in}}{\pgfqpoint{1.024572in}{1.583357in}}{\pgfqpoint{1.030396in}{1.577533in}}%
\pgfpathcurveto{\pgfqpoint{1.036219in}{1.571709in}}{\pgfqpoint{1.044120in}{1.568437in}}{\pgfqpoint{1.052356in}{1.568437in}}%
\pgfpathclose%
\pgfusepath{stroke,fill}%
\end{pgfscope}%
\begin{pgfscope}%
\pgfpathrectangle{\pgfqpoint{0.100000in}{0.220728in}}{\pgfqpoint{3.696000in}{3.696000in}}%
\pgfusepath{clip}%
\pgfsetbuttcap%
\pgfsetroundjoin%
\definecolor{currentfill}{rgb}{0.121569,0.466667,0.705882}%
\pgfsetfillcolor{currentfill}%
\pgfsetfillopacity{0.581717}%
\pgfsetlinewidth{1.003750pt}%
\definecolor{currentstroke}{rgb}{0.121569,0.466667,0.705882}%
\pgfsetstrokecolor{currentstroke}%
\pgfsetstrokeopacity{0.581717}%
\pgfsetdash{}{0pt}%
\pgfpathmoveto{\pgfqpoint{2.096143in}{1.817232in}}%
\pgfpathcurveto{\pgfqpoint{2.104379in}{1.817232in}}{\pgfqpoint{2.112279in}{1.820504in}}{\pgfqpoint{2.118103in}{1.826328in}}%
\pgfpathcurveto{\pgfqpoint{2.123927in}{1.832152in}}{\pgfqpoint{2.127199in}{1.840052in}}{\pgfqpoint{2.127199in}{1.848288in}}%
\pgfpathcurveto{\pgfqpoint{2.127199in}{1.856525in}}{\pgfqpoint{2.123927in}{1.864425in}}{\pgfqpoint{2.118103in}{1.870249in}}%
\pgfpathcurveto{\pgfqpoint{2.112279in}{1.876073in}}{\pgfqpoint{2.104379in}{1.879345in}}{\pgfqpoint{2.096143in}{1.879345in}}%
\pgfpathcurveto{\pgfqpoint{2.087906in}{1.879345in}}{\pgfqpoint{2.080006in}{1.876073in}}{\pgfqpoint{2.074182in}{1.870249in}}%
\pgfpathcurveto{\pgfqpoint{2.068358in}{1.864425in}}{\pgfqpoint{2.065086in}{1.856525in}}{\pgfqpoint{2.065086in}{1.848288in}}%
\pgfpathcurveto{\pgfqpoint{2.065086in}{1.840052in}}{\pgfqpoint{2.068358in}{1.832152in}}{\pgfqpoint{2.074182in}{1.826328in}}%
\pgfpathcurveto{\pgfqpoint{2.080006in}{1.820504in}}{\pgfqpoint{2.087906in}{1.817232in}}{\pgfqpoint{2.096143in}{1.817232in}}%
\pgfpathclose%
\pgfusepath{stroke,fill}%
\end{pgfscope}%
\begin{pgfscope}%
\pgfpathrectangle{\pgfqpoint{0.100000in}{0.220728in}}{\pgfqpoint{3.696000in}{3.696000in}}%
\pgfusepath{clip}%
\pgfsetbuttcap%
\pgfsetroundjoin%
\definecolor{currentfill}{rgb}{0.121569,0.466667,0.705882}%
\pgfsetfillcolor{currentfill}%
\pgfsetfillopacity{0.583595}%
\pgfsetlinewidth{1.003750pt}%
\definecolor{currentstroke}{rgb}{0.121569,0.466667,0.705882}%
\pgfsetstrokecolor{currentstroke}%
\pgfsetstrokeopacity{0.583595}%
\pgfsetdash{}{0pt}%
\pgfpathmoveto{\pgfqpoint{2.096511in}{1.815483in}}%
\pgfpathcurveto{\pgfqpoint{2.104747in}{1.815483in}}{\pgfqpoint{2.112647in}{1.818755in}}{\pgfqpoint{2.118471in}{1.824579in}}%
\pgfpathcurveto{\pgfqpoint{2.124295in}{1.830403in}}{\pgfqpoint{2.127567in}{1.838303in}}{\pgfqpoint{2.127567in}{1.846539in}}%
\pgfpathcurveto{\pgfqpoint{2.127567in}{1.854776in}}{\pgfqpoint{2.124295in}{1.862676in}}{\pgfqpoint{2.118471in}{1.868500in}}%
\pgfpathcurveto{\pgfqpoint{2.112647in}{1.874323in}}{\pgfqpoint{2.104747in}{1.877596in}}{\pgfqpoint{2.096511in}{1.877596in}}%
\pgfpathcurveto{\pgfqpoint{2.088274in}{1.877596in}}{\pgfqpoint{2.080374in}{1.874323in}}{\pgfqpoint{2.074550in}{1.868500in}}%
\pgfpathcurveto{\pgfqpoint{2.068726in}{1.862676in}}{\pgfqpoint{2.065454in}{1.854776in}}{\pgfqpoint{2.065454in}{1.846539in}}%
\pgfpathcurveto{\pgfqpoint{2.065454in}{1.838303in}}{\pgfqpoint{2.068726in}{1.830403in}}{\pgfqpoint{2.074550in}{1.824579in}}%
\pgfpathcurveto{\pgfqpoint{2.080374in}{1.818755in}}{\pgfqpoint{2.088274in}{1.815483in}}{\pgfqpoint{2.096511in}{1.815483in}}%
\pgfpathclose%
\pgfusepath{stroke,fill}%
\end{pgfscope}%
\begin{pgfscope}%
\pgfpathrectangle{\pgfqpoint{0.100000in}{0.220728in}}{\pgfqpoint{3.696000in}{3.696000in}}%
\pgfusepath{clip}%
\pgfsetbuttcap%
\pgfsetroundjoin%
\definecolor{currentfill}{rgb}{0.121569,0.466667,0.705882}%
\pgfsetfillcolor{currentfill}%
\pgfsetfillopacity{0.583614}%
\pgfsetlinewidth{1.003750pt}%
\definecolor{currentstroke}{rgb}{0.121569,0.466667,0.705882}%
\pgfsetstrokecolor{currentstroke}%
\pgfsetstrokeopacity{0.583614}%
\pgfsetdash{}{0pt}%
\pgfpathmoveto{\pgfqpoint{1.041499in}{1.554711in}}%
\pgfpathcurveto{\pgfqpoint{1.049735in}{1.554711in}}{\pgfqpoint{1.057635in}{1.557983in}}{\pgfqpoint{1.063459in}{1.563807in}}%
\pgfpathcurveto{\pgfqpoint{1.069283in}{1.569631in}}{\pgfqpoint{1.072556in}{1.577531in}}{\pgfqpoint{1.072556in}{1.585767in}}%
\pgfpathcurveto{\pgfqpoint{1.072556in}{1.594004in}}{\pgfqpoint{1.069283in}{1.601904in}}{\pgfqpoint{1.063459in}{1.607728in}}%
\pgfpathcurveto{\pgfqpoint{1.057635in}{1.613552in}}{\pgfqpoint{1.049735in}{1.616824in}}{\pgfqpoint{1.041499in}{1.616824in}}%
\pgfpathcurveto{\pgfqpoint{1.033263in}{1.616824in}}{\pgfqpoint{1.025363in}{1.613552in}}{\pgfqpoint{1.019539in}{1.607728in}}%
\pgfpathcurveto{\pgfqpoint{1.013715in}{1.601904in}}{\pgfqpoint{1.010443in}{1.594004in}}{\pgfqpoint{1.010443in}{1.585767in}}%
\pgfpathcurveto{\pgfqpoint{1.010443in}{1.577531in}}{\pgfqpoint{1.013715in}{1.569631in}}{\pgfqpoint{1.019539in}{1.563807in}}%
\pgfpathcurveto{\pgfqpoint{1.025363in}{1.557983in}}{\pgfqpoint{1.033263in}{1.554711in}}{\pgfqpoint{1.041499in}{1.554711in}}%
\pgfpathclose%
\pgfusepath{stroke,fill}%
\end{pgfscope}%
\begin{pgfscope}%
\pgfpathrectangle{\pgfqpoint{0.100000in}{0.220728in}}{\pgfqpoint{3.696000in}{3.696000in}}%
\pgfusepath{clip}%
\pgfsetbuttcap%
\pgfsetroundjoin%
\definecolor{currentfill}{rgb}{0.121569,0.466667,0.705882}%
\pgfsetfillcolor{currentfill}%
\pgfsetfillopacity{0.586137}%
\pgfsetlinewidth{1.003750pt}%
\definecolor{currentstroke}{rgb}{0.121569,0.466667,0.705882}%
\pgfsetstrokecolor{currentstroke}%
\pgfsetstrokeopacity{0.586137}%
\pgfsetdash{}{0pt}%
\pgfpathmoveto{\pgfqpoint{2.098445in}{1.813016in}}%
\pgfpathcurveto{\pgfqpoint{2.106681in}{1.813016in}}{\pgfqpoint{2.114581in}{1.816289in}}{\pgfqpoint{2.120405in}{1.822113in}}%
\pgfpathcurveto{\pgfqpoint{2.126229in}{1.827937in}}{\pgfqpoint{2.129502in}{1.835837in}}{\pgfqpoint{2.129502in}{1.844073in}}%
\pgfpathcurveto{\pgfqpoint{2.129502in}{1.852309in}}{\pgfqpoint{2.126229in}{1.860209in}}{\pgfqpoint{2.120405in}{1.866033in}}%
\pgfpathcurveto{\pgfqpoint{2.114581in}{1.871857in}}{\pgfqpoint{2.106681in}{1.875129in}}{\pgfqpoint{2.098445in}{1.875129in}}%
\pgfpathcurveto{\pgfqpoint{2.090209in}{1.875129in}}{\pgfqpoint{2.082309in}{1.871857in}}{\pgfqpoint{2.076485in}{1.866033in}}%
\pgfpathcurveto{\pgfqpoint{2.070661in}{1.860209in}}{\pgfqpoint{2.067389in}{1.852309in}}{\pgfqpoint{2.067389in}{1.844073in}}%
\pgfpathcurveto{\pgfqpoint{2.067389in}{1.835837in}}{\pgfqpoint{2.070661in}{1.827937in}}{\pgfqpoint{2.076485in}{1.822113in}}%
\pgfpathcurveto{\pgfqpoint{2.082309in}{1.816289in}}{\pgfqpoint{2.090209in}{1.813016in}}{\pgfqpoint{2.098445in}{1.813016in}}%
\pgfpathclose%
\pgfusepath{stroke,fill}%
\end{pgfscope}%
\begin{pgfscope}%
\pgfpathrectangle{\pgfqpoint{0.100000in}{0.220728in}}{\pgfqpoint{3.696000in}{3.696000in}}%
\pgfusepath{clip}%
\pgfsetbuttcap%
\pgfsetroundjoin%
\definecolor{currentfill}{rgb}{0.121569,0.466667,0.705882}%
\pgfsetfillcolor{currentfill}%
\pgfsetfillopacity{0.586210}%
\pgfsetlinewidth{1.003750pt}%
\definecolor{currentstroke}{rgb}{0.121569,0.466667,0.705882}%
\pgfsetstrokecolor{currentstroke}%
\pgfsetstrokeopacity{0.586210}%
\pgfsetdash{}{0pt}%
\pgfpathmoveto{\pgfqpoint{1.030863in}{1.545608in}}%
\pgfpathcurveto{\pgfqpoint{1.039100in}{1.545608in}}{\pgfqpoint{1.047000in}{1.548880in}}{\pgfqpoint{1.052824in}{1.554704in}}%
\pgfpathcurveto{\pgfqpoint{1.058648in}{1.560528in}}{\pgfqpoint{1.061920in}{1.568428in}}{\pgfqpoint{1.061920in}{1.576665in}}%
\pgfpathcurveto{\pgfqpoint{1.061920in}{1.584901in}}{\pgfqpoint{1.058648in}{1.592801in}}{\pgfqpoint{1.052824in}{1.598625in}}%
\pgfpathcurveto{\pgfqpoint{1.047000in}{1.604449in}}{\pgfqpoint{1.039100in}{1.607721in}}{\pgfqpoint{1.030863in}{1.607721in}}%
\pgfpathcurveto{\pgfqpoint{1.022627in}{1.607721in}}{\pgfqpoint{1.014727in}{1.604449in}}{\pgfqpoint{1.008903in}{1.598625in}}%
\pgfpathcurveto{\pgfqpoint{1.003079in}{1.592801in}}{\pgfqpoint{0.999807in}{1.584901in}}{\pgfqpoint{0.999807in}{1.576665in}}%
\pgfpathcurveto{\pgfqpoint{0.999807in}{1.568428in}}{\pgfqpoint{1.003079in}{1.560528in}}{\pgfqpoint{1.008903in}{1.554704in}}%
\pgfpathcurveto{\pgfqpoint{1.014727in}{1.548880in}}{\pgfqpoint{1.022627in}{1.545608in}}{\pgfqpoint{1.030863in}{1.545608in}}%
\pgfpathclose%
\pgfusepath{stroke,fill}%
\end{pgfscope}%
\begin{pgfscope}%
\pgfpathrectangle{\pgfqpoint{0.100000in}{0.220728in}}{\pgfqpoint{3.696000in}{3.696000in}}%
\pgfusepath{clip}%
\pgfsetbuttcap%
\pgfsetroundjoin%
\definecolor{currentfill}{rgb}{0.121569,0.466667,0.705882}%
\pgfsetfillcolor{currentfill}%
\pgfsetfillopacity{0.589278}%
\pgfsetlinewidth{1.003750pt}%
\definecolor{currentstroke}{rgb}{0.121569,0.466667,0.705882}%
\pgfsetstrokecolor{currentstroke}%
\pgfsetstrokeopacity{0.589278}%
\pgfsetdash{}{0pt}%
\pgfpathmoveto{\pgfqpoint{2.100254in}{1.811101in}}%
\pgfpathcurveto{\pgfqpoint{2.108491in}{1.811101in}}{\pgfqpoint{2.116391in}{1.814373in}}{\pgfqpoint{2.122215in}{1.820197in}}%
\pgfpathcurveto{\pgfqpoint{2.128038in}{1.826021in}}{\pgfqpoint{2.131311in}{1.833921in}}{\pgfqpoint{2.131311in}{1.842157in}}%
\pgfpathcurveto{\pgfqpoint{2.131311in}{1.850394in}}{\pgfqpoint{2.128038in}{1.858294in}}{\pgfqpoint{2.122215in}{1.864118in}}%
\pgfpathcurveto{\pgfqpoint{2.116391in}{1.869942in}}{\pgfqpoint{2.108491in}{1.873214in}}{\pgfqpoint{2.100254in}{1.873214in}}%
\pgfpathcurveto{\pgfqpoint{2.092018in}{1.873214in}}{\pgfqpoint{2.084118in}{1.869942in}}{\pgfqpoint{2.078294in}{1.864118in}}%
\pgfpathcurveto{\pgfqpoint{2.072470in}{1.858294in}}{\pgfqpoint{2.069198in}{1.850394in}}{\pgfqpoint{2.069198in}{1.842157in}}%
\pgfpathcurveto{\pgfqpoint{2.069198in}{1.833921in}}{\pgfqpoint{2.072470in}{1.826021in}}{\pgfqpoint{2.078294in}{1.820197in}}%
\pgfpathcurveto{\pgfqpoint{2.084118in}{1.814373in}}{\pgfqpoint{2.092018in}{1.811101in}}{\pgfqpoint{2.100254in}{1.811101in}}%
\pgfpathclose%
\pgfusepath{stroke,fill}%
\end{pgfscope}%
\begin{pgfscope}%
\pgfpathrectangle{\pgfqpoint{0.100000in}{0.220728in}}{\pgfqpoint{3.696000in}{3.696000in}}%
\pgfusepath{clip}%
\pgfsetbuttcap%
\pgfsetroundjoin%
\definecolor{currentfill}{rgb}{0.121569,0.466667,0.705882}%
\pgfsetfillcolor{currentfill}%
\pgfsetfillopacity{0.589828}%
\pgfsetlinewidth{1.003750pt}%
\definecolor{currentstroke}{rgb}{0.121569,0.466667,0.705882}%
\pgfsetstrokecolor{currentstroke}%
\pgfsetstrokeopacity{0.589828}%
\pgfsetdash{}{0pt}%
\pgfpathmoveto{\pgfqpoint{1.022715in}{1.541059in}}%
\pgfpathcurveto{\pgfqpoint{1.030951in}{1.541059in}}{\pgfqpoint{1.038851in}{1.544331in}}{\pgfqpoint{1.044675in}{1.550155in}}%
\pgfpathcurveto{\pgfqpoint{1.050499in}{1.555979in}}{\pgfqpoint{1.053771in}{1.563879in}}{\pgfqpoint{1.053771in}{1.572116in}}%
\pgfpathcurveto{\pgfqpoint{1.053771in}{1.580352in}}{\pgfqpoint{1.050499in}{1.588252in}}{\pgfqpoint{1.044675in}{1.594076in}}%
\pgfpathcurveto{\pgfqpoint{1.038851in}{1.599900in}}{\pgfqpoint{1.030951in}{1.603172in}}{\pgfqpoint{1.022715in}{1.603172in}}%
\pgfpathcurveto{\pgfqpoint{1.014478in}{1.603172in}}{\pgfqpoint{1.006578in}{1.599900in}}{\pgfqpoint{1.000754in}{1.594076in}}%
\pgfpathcurveto{\pgfqpoint{0.994930in}{1.588252in}}{\pgfqpoint{0.991658in}{1.580352in}}{\pgfqpoint{0.991658in}{1.572116in}}%
\pgfpathcurveto{\pgfqpoint{0.991658in}{1.563879in}}{\pgfqpoint{0.994930in}{1.555979in}}{\pgfqpoint{1.000754in}{1.550155in}}%
\pgfpathcurveto{\pgfqpoint{1.006578in}{1.544331in}}{\pgfqpoint{1.014478in}{1.541059in}}{\pgfqpoint{1.022715in}{1.541059in}}%
\pgfpathclose%
\pgfusepath{stroke,fill}%
\end{pgfscope}%
\begin{pgfscope}%
\pgfpathrectangle{\pgfqpoint{0.100000in}{0.220728in}}{\pgfqpoint{3.696000in}{3.696000in}}%
\pgfusepath{clip}%
\pgfsetbuttcap%
\pgfsetroundjoin%
\definecolor{currentfill}{rgb}{0.121569,0.466667,0.705882}%
\pgfsetfillcolor{currentfill}%
\pgfsetfillopacity{0.591613}%
\pgfsetlinewidth{1.003750pt}%
\definecolor{currentstroke}{rgb}{0.121569,0.466667,0.705882}%
\pgfsetstrokecolor{currentstroke}%
\pgfsetstrokeopacity{0.591613}%
\pgfsetdash{}{0pt}%
\pgfpathmoveto{\pgfqpoint{1.014355in}{1.533907in}}%
\pgfpathcurveto{\pgfqpoint{1.022591in}{1.533907in}}{\pgfqpoint{1.030491in}{1.537179in}}{\pgfqpoint{1.036315in}{1.543003in}}%
\pgfpathcurveto{\pgfqpoint{1.042139in}{1.548827in}}{\pgfqpoint{1.045412in}{1.556727in}}{\pgfqpoint{1.045412in}{1.564963in}}%
\pgfpathcurveto{\pgfqpoint{1.045412in}{1.573199in}}{\pgfqpoint{1.042139in}{1.581100in}}{\pgfqpoint{1.036315in}{1.586923in}}%
\pgfpathcurveto{\pgfqpoint{1.030491in}{1.592747in}}{\pgfqpoint{1.022591in}{1.596020in}}{\pgfqpoint{1.014355in}{1.596020in}}%
\pgfpathcurveto{\pgfqpoint{1.006119in}{1.596020in}}{\pgfqpoint{0.998219in}{1.592747in}}{\pgfqpoint{0.992395in}{1.586923in}}%
\pgfpathcurveto{\pgfqpoint{0.986571in}{1.581100in}}{\pgfqpoint{0.983299in}{1.573199in}}{\pgfqpoint{0.983299in}{1.564963in}}%
\pgfpathcurveto{\pgfqpoint{0.983299in}{1.556727in}}{\pgfqpoint{0.986571in}{1.548827in}}{\pgfqpoint{0.992395in}{1.543003in}}%
\pgfpathcurveto{\pgfqpoint{0.998219in}{1.537179in}}{\pgfqpoint{1.006119in}{1.533907in}}{\pgfqpoint{1.014355in}{1.533907in}}%
\pgfpathclose%
\pgfusepath{stroke,fill}%
\end{pgfscope}%
\begin{pgfscope}%
\pgfpathrectangle{\pgfqpoint{0.100000in}{0.220728in}}{\pgfqpoint{3.696000in}{3.696000in}}%
\pgfusepath{clip}%
\pgfsetbuttcap%
\pgfsetroundjoin%
\definecolor{currentfill}{rgb}{0.121569,0.466667,0.705882}%
\pgfsetfillcolor{currentfill}%
\pgfsetfillopacity{0.592673}%
\pgfsetlinewidth{1.003750pt}%
\definecolor{currentstroke}{rgb}{0.121569,0.466667,0.705882}%
\pgfsetstrokecolor{currentstroke}%
\pgfsetstrokeopacity{0.592673}%
\pgfsetdash{}{0pt}%
\pgfpathmoveto{\pgfqpoint{2.102170in}{1.806951in}}%
\pgfpathcurveto{\pgfqpoint{2.110407in}{1.806951in}}{\pgfqpoint{2.118307in}{1.810223in}}{\pgfqpoint{2.124131in}{1.816047in}}%
\pgfpathcurveto{\pgfqpoint{2.129955in}{1.821871in}}{\pgfqpoint{2.133227in}{1.829771in}}{\pgfqpoint{2.133227in}{1.838008in}}%
\pgfpathcurveto{\pgfqpoint{2.133227in}{1.846244in}}{\pgfqpoint{2.129955in}{1.854144in}}{\pgfqpoint{2.124131in}{1.859968in}}%
\pgfpathcurveto{\pgfqpoint{2.118307in}{1.865792in}}{\pgfqpoint{2.110407in}{1.869064in}}{\pgfqpoint{2.102170in}{1.869064in}}%
\pgfpathcurveto{\pgfqpoint{2.093934in}{1.869064in}}{\pgfqpoint{2.086034in}{1.865792in}}{\pgfqpoint{2.080210in}{1.859968in}}%
\pgfpathcurveto{\pgfqpoint{2.074386in}{1.854144in}}{\pgfqpoint{2.071114in}{1.846244in}}{\pgfqpoint{2.071114in}{1.838008in}}%
\pgfpathcurveto{\pgfqpoint{2.071114in}{1.829771in}}{\pgfqpoint{2.074386in}{1.821871in}}{\pgfqpoint{2.080210in}{1.816047in}}%
\pgfpathcurveto{\pgfqpoint{2.086034in}{1.810223in}}{\pgfqpoint{2.093934in}{1.806951in}}{\pgfqpoint{2.102170in}{1.806951in}}%
\pgfpathclose%
\pgfusepath{stroke,fill}%
\end{pgfscope}%
\begin{pgfscope}%
\pgfpathrectangle{\pgfqpoint{0.100000in}{0.220728in}}{\pgfqpoint{3.696000in}{3.696000in}}%
\pgfusepath{clip}%
\pgfsetbuttcap%
\pgfsetroundjoin%
\definecolor{currentfill}{rgb}{0.121569,0.466667,0.705882}%
\pgfsetfillcolor{currentfill}%
\pgfsetfillopacity{0.593853}%
\pgfsetlinewidth{1.003750pt}%
\definecolor{currentstroke}{rgb}{0.121569,0.466667,0.705882}%
\pgfsetstrokecolor{currentstroke}%
\pgfsetstrokeopacity{0.593853}%
\pgfsetdash{}{0pt}%
\pgfpathmoveto{\pgfqpoint{1.007838in}{1.529383in}}%
\pgfpathcurveto{\pgfqpoint{1.016074in}{1.529383in}}{\pgfqpoint{1.023974in}{1.532655in}}{\pgfqpoint{1.029798in}{1.538479in}}%
\pgfpathcurveto{\pgfqpoint{1.035622in}{1.544303in}}{\pgfqpoint{1.038894in}{1.552203in}}{\pgfqpoint{1.038894in}{1.560439in}}%
\pgfpathcurveto{\pgfqpoint{1.038894in}{1.568676in}}{\pgfqpoint{1.035622in}{1.576576in}}{\pgfqpoint{1.029798in}{1.582400in}}%
\pgfpathcurveto{\pgfqpoint{1.023974in}{1.588224in}}{\pgfqpoint{1.016074in}{1.591496in}}{\pgfqpoint{1.007838in}{1.591496in}}%
\pgfpathcurveto{\pgfqpoint{0.999602in}{1.591496in}}{\pgfqpoint{0.991702in}{1.588224in}}{\pgfqpoint{0.985878in}{1.582400in}}%
\pgfpathcurveto{\pgfqpoint{0.980054in}{1.576576in}}{\pgfqpoint{0.976781in}{1.568676in}}{\pgfqpoint{0.976781in}{1.560439in}}%
\pgfpathcurveto{\pgfqpoint{0.976781in}{1.552203in}}{\pgfqpoint{0.980054in}{1.544303in}}{\pgfqpoint{0.985878in}{1.538479in}}%
\pgfpathcurveto{\pgfqpoint{0.991702in}{1.532655in}}{\pgfqpoint{0.999602in}{1.529383in}}{\pgfqpoint{1.007838in}{1.529383in}}%
\pgfpathclose%
\pgfusepath{stroke,fill}%
\end{pgfscope}%
\begin{pgfscope}%
\pgfpathrectangle{\pgfqpoint{0.100000in}{0.220728in}}{\pgfqpoint{3.696000in}{3.696000in}}%
\pgfusepath{clip}%
\pgfsetbuttcap%
\pgfsetroundjoin%
\definecolor{currentfill}{rgb}{0.121569,0.466667,0.705882}%
\pgfsetfillcolor{currentfill}%
\pgfsetfillopacity{0.596214}%
\pgfsetlinewidth{1.003750pt}%
\definecolor{currentstroke}{rgb}{0.121569,0.466667,0.705882}%
\pgfsetstrokecolor{currentstroke}%
\pgfsetstrokeopacity{0.596214}%
\pgfsetdash{}{0pt}%
\pgfpathmoveto{\pgfqpoint{2.104591in}{1.802335in}}%
\pgfpathcurveto{\pgfqpoint{2.112827in}{1.802335in}}{\pgfqpoint{2.120727in}{1.805607in}}{\pgfqpoint{2.126551in}{1.811431in}}%
\pgfpathcurveto{\pgfqpoint{2.132375in}{1.817255in}}{\pgfqpoint{2.135647in}{1.825155in}}{\pgfqpoint{2.135647in}{1.833392in}}%
\pgfpathcurveto{\pgfqpoint{2.135647in}{1.841628in}}{\pgfqpoint{2.132375in}{1.849528in}}{\pgfqpoint{2.126551in}{1.855352in}}%
\pgfpathcurveto{\pgfqpoint{2.120727in}{1.861176in}}{\pgfqpoint{2.112827in}{1.864448in}}{\pgfqpoint{2.104591in}{1.864448in}}%
\pgfpathcurveto{\pgfqpoint{2.096354in}{1.864448in}}{\pgfqpoint{2.088454in}{1.861176in}}{\pgfqpoint{2.082631in}{1.855352in}}%
\pgfpathcurveto{\pgfqpoint{2.076807in}{1.849528in}}{\pgfqpoint{2.073534in}{1.841628in}}{\pgfqpoint{2.073534in}{1.833392in}}%
\pgfpathcurveto{\pgfqpoint{2.073534in}{1.825155in}}{\pgfqpoint{2.076807in}{1.817255in}}{\pgfqpoint{2.082631in}{1.811431in}}%
\pgfpathcurveto{\pgfqpoint{2.088454in}{1.805607in}}{\pgfqpoint{2.096354in}{1.802335in}}{\pgfqpoint{2.104591in}{1.802335in}}%
\pgfpathclose%
\pgfusepath{stroke,fill}%
\end{pgfscope}%
\begin{pgfscope}%
\pgfpathrectangle{\pgfqpoint{0.100000in}{0.220728in}}{\pgfqpoint{3.696000in}{3.696000in}}%
\pgfusepath{clip}%
\pgfsetbuttcap%
\pgfsetroundjoin%
\definecolor{currentfill}{rgb}{0.121569,0.466667,0.705882}%
\pgfsetfillcolor{currentfill}%
\pgfsetfillopacity{0.597946}%
\pgfsetlinewidth{1.003750pt}%
\definecolor{currentstroke}{rgb}{0.121569,0.466667,0.705882}%
\pgfsetstrokecolor{currentstroke}%
\pgfsetstrokeopacity{0.597946}%
\pgfsetdash{}{0pt}%
\pgfpathmoveto{\pgfqpoint{0.995565in}{1.521796in}}%
\pgfpathcurveto{\pgfqpoint{1.003801in}{1.521796in}}{\pgfqpoint{1.011701in}{1.525068in}}{\pgfqpoint{1.017525in}{1.530892in}}%
\pgfpathcurveto{\pgfqpoint{1.023349in}{1.536716in}}{\pgfqpoint{1.026622in}{1.544616in}}{\pgfqpoint{1.026622in}{1.552852in}}%
\pgfpathcurveto{\pgfqpoint{1.026622in}{1.561088in}}{\pgfqpoint{1.023349in}{1.568989in}}{\pgfqpoint{1.017525in}{1.574812in}}%
\pgfpathcurveto{\pgfqpoint{1.011701in}{1.580636in}}{\pgfqpoint{1.003801in}{1.583909in}}{\pgfqpoint{0.995565in}{1.583909in}}%
\pgfpathcurveto{\pgfqpoint{0.987329in}{1.583909in}}{\pgfqpoint{0.979429in}{1.580636in}}{\pgfqpoint{0.973605in}{1.574812in}}%
\pgfpathcurveto{\pgfqpoint{0.967781in}{1.568989in}}{\pgfqpoint{0.964509in}{1.561088in}}{\pgfqpoint{0.964509in}{1.552852in}}%
\pgfpathcurveto{\pgfqpoint{0.964509in}{1.544616in}}{\pgfqpoint{0.967781in}{1.536716in}}{\pgfqpoint{0.973605in}{1.530892in}}%
\pgfpathcurveto{\pgfqpoint{0.979429in}{1.525068in}}{\pgfqpoint{0.987329in}{1.521796in}}{\pgfqpoint{0.995565in}{1.521796in}}%
\pgfpathclose%
\pgfusepath{stroke,fill}%
\end{pgfscope}%
\begin{pgfscope}%
\pgfpathrectangle{\pgfqpoint{0.100000in}{0.220728in}}{\pgfqpoint{3.696000in}{3.696000in}}%
\pgfusepath{clip}%
\pgfsetbuttcap%
\pgfsetroundjoin%
\definecolor{currentfill}{rgb}{0.121569,0.466667,0.705882}%
\pgfsetfillcolor{currentfill}%
\pgfsetfillopacity{0.600019}%
\pgfsetlinewidth{1.003750pt}%
\definecolor{currentstroke}{rgb}{0.121569,0.466667,0.705882}%
\pgfsetstrokecolor{currentstroke}%
\pgfsetstrokeopacity{0.600019}%
\pgfsetdash{}{0pt}%
\pgfpathmoveto{\pgfqpoint{2.107115in}{1.796085in}}%
\pgfpathcurveto{\pgfqpoint{2.115351in}{1.796085in}}{\pgfqpoint{2.123251in}{1.799357in}}{\pgfqpoint{2.129075in}{1.805181in}}%
\pgfpathcurveto{\pgfqpoint{2.134899in}{1.811005in}}{\pgfqpoint{2.138171in}{1.818905in}}{\pgfqpoint{2.138171in}{1.827141in}}%
\pgfpathcurveto{\pgfqpoint{2.138171in}{1.835377in}}{\pgfqpoint{2.134899in}{1.843277in}}{\pgfqpoint{2.129075in}{1.849101in}}%
\pgfpathcurveto{\pgfqpoint{2.123251in}{1.854925in}}{\pgfqpoint{2.115351in}{1.858198in}}{\pgfqpoint{2.107115in}{1.858198in}}%
\pgfpathcurveto{\pgfqpoint{2.098878in}{1.858198in}}{\pgfqpoint{2.090978in}{1.854925in}}{\pgfqpoint{2.085154in}{1.849101in}}%
\pgfpathcurveto{\pgfqpoint{2.079330in}{1.843277in}}{\pgfqpoint{2.076058in}{1.835377in}}{\pgfqpoint{2.076058in}{1.827141in}}%
\pgfpathcurveto{\pgfqpoint{2.076058in}{1.818905in}}{\pgfqpoint{2.079330in}{1.811005in}}{\pgfqpoint{2.085154in}{1.805181in}}%
\pgfpathcurveto{\pgfqpoint{2.090978in}{1.799357in}}{\pgfqpoint{2.098878in}{1.796085in}}{\pgfqpoint{2.107115in}{1.796085in}}%
\pgfpathclose%
\pgfusepath{stroke,fill}%
\end{pgfscope}%
\begin{pgfscope}%
\pgfpathrectangle{\pgfqpoint{0.100000in}{0.220728in}}{\pgfqpoint{3.696000in}{3.696000in}}%
\pgfusepath{clip}%
\pgfsetbuttcap%
\pgfsetroundjoin%
\definecolor{currentfill}{rgb}{0.121569,0.466667,0.705882}%
\pgfsetfillcolor{currentfill}%
\pgfsetfillopacity{0.601477}%
\pgfsetlinewidth{1.003750pt}%
\definecolor{currentstroke}{rgb}{0.121569,0.466667,0.705882}%
\pgfsetstrokecolor{currentstroke}%
\pgfsetstrokeopacity{0.601477}%
\pgfsetdash{}{0pt}%
\pgfpathmoveto{\pgfqpoint{0.982120in}{1.514438in}}%
\pgfpathcurveto{\pgfqpoint{0.990356in}{1.514438in}}{\pgfqpoint{0.998256in}{1.517710in}}{\pgfqpoint{1.004080in}{1.523534in}}%
\pgfpathcurveto{\pgfqpoint{1.009904in}{1.529358in}}{\pgfqpoint{1.013176in}{1.537258in}}{\pgfqpoint{1.013176in}{1.545494in}}%
\pgfpathcurveto{\pgfqpoint{1.013176in}{1.553731in}}{\pgfqpoint{1.009904in}{1.561631in}}{\pgfqpoint{1.004080in}{1.567455in}}%
\pgfpathcurveto{\pgfqpoint{0.998256in}{1.573279in}}{\pgfqpoint{0.990356in}{1.576551in}}{\pgfqpoint{0.982120in}{1.576551in}}%
\pgfpathcurveto{\pgfqpoint{0.973883in}{1.576551in}}{\pgfqpoint{0.965983in}{1.573279in}}{\pgfqpoint{0.960159in}{1.567455in}}%
\pgfpathcurveto{\pgfqpoint{0.954336in}{1.561631in}}{\pgfqpoint{0.951063in}{1.553731in}}{\pgfqpoint{0.951063in}{1.545494in}}%
\pgfpathcurveto{\pgfqpoint{0.951063in}{1.537258in}}{\pgfqpoint{0.954336in}{1.529358in}}{\pgfqpoint{0.960159in}{1.523534in}}%
\pgfpathcurveto{\pgfqpoint{0.965983in}{1.517710in}}{\pgfqpoint{0.973883in}{1.514438in}}{\pgfqpoint{0.982120in}{1.514438in}}%
\pgfpathclose%
\pgfusepath{stroke,fill}%
\end{pgfscope}%
\begin{pgfscope}%
\pgfpathrectangle{\pgfqpoint{0.100000in}{0.220728in}}{\pgfqpoint{3.696000in}{3.696000in}}%
\pgfusepath{clip}%
\pgfsetbuttcap%
\pgfsetroundjoin%
\definecolor{currentfill}{rgb}{0.121569,0.466667,0.705882}%
\pgfsetfillcolor{currentfill}%
\pgfsetfillopacity{0.604976}%
\pgfsetlinewidth{1.003750pt}%
\definecolor{currentstroke}{rgb}{0.121569,0.466667,0.705882}%
\pgfsetstrokecolor{currentstroke}%
\pgfsetstrokeopacity{0.604976}%
\pgfsetdash{}{0pt}%
\pgfpathmoveto{\pgfqpoint{2.112626in}{1.795459in}}%
\pgfpathcurveto{\pgfqpoint{2.120863in}{1.795459in}}{\pgfqpoint{2.128763in}{1.798732in}}{\pgfqpoint{2.134587in}{1.804556in}}%
\pgfpathcurveto{\pgfqpoint{2.140411in}{1.810380in}}{\pgfqpoint{2.143683in}{1.818280in}}{\pgfqpoint{2.143683in}{1.826516in}}%
\pgfpathcurveto{\pgfqpoint{2.143683in}{1.834752in}}{\pgfqpoint{2.140411in}{1.842652in}}{\pgfqpoint{2.134587in}{1.848476in}}%
\pgfpathcurveto{\pgfqpoint{2.128763in}{1.854300in}}{\pgfqpoint{2.120863in}{1.857572in}}{\pgfqpoint{2.112626in}{1.857572in}}%
\pgfpathcurveto{\pgfqpoint{2.104390in}{1.857572in}}{\pgfqpoint{2.096490in}{1.854300in}}{\pgfqpoint{2.090666in}{1.848476in}}%
\pgfpathcurveto{\pgfqpoint{2.084842in}{1.842652in}}{\pgfqpoint{2.081570in}{1.834752in}}{\pgfqpoint{2.081570in}{1.826516in}}%
\pgfpathcurveto{\pgfqpoint{2.081570in}{1.818280in}}{\pgfqpoint{2.084842in}{1.810380in}}{\pgfqpoint{2.090666in}{1.804556in}}%
\pgfpathcurveto{\pgfqpoint{2.096490in}{1.798732in}}{\pgfqpoint{2.104390in}{1.795459in}}{\pgfqpoint{2.112626in}{1.795459in}}%
\pgfpathclose%
\pgfusepath{stroke,fill}%
\end{pgfscope}%
\begin{pgfscope}%
\pgfpathrectangle{\pgfqpoint{0.100000in}{0.220728in}}{\pgfqpoint{3.696000in}{3.696000in}}%
\pgfusepath{clip}%
\pgfsetbuttcap%
\pgfsetroundjoin%
\definecolor{currentfill}{rgb}{0.121569,0.466667,0.705882}%
\pgfsetfillcolor{currentfill}%
\pgfsetfillopacity{0.605483}%
\pgfsetlinewidth{1.003750pt}%
\definecolor{currentstroke}{rgb}{0.121569,0.466667,0.705882}%
\pgfsetstrokecolor{currentstroke}%
\pgfsetstrokeopacity{0.605483}%
\pgfsetdash{}{0pt}%
\pgfpathmoveto{\pgfqpoint{0.974112in}{1.511435in}}%
\pgfpathcurveto{\pgfqpoint{0.982349in}{1.511435in}}{\pgfqpoint{0.990249in}{1.514707in}}{\pgfqpoint{0.996073in}{1.520531in}}%
\pgfpathcurveto{\pgfqpoint{1.001897in}{1.526355in}}{\pgfqpoint{1.005169in}{1.534255in}}{\pgfqpoint{1.005169in}{1.542491in}}%
\pgfpathcurveto{\pgfqpoint{1.005169in}{1.550727in}}{\pgfqpoint{1.001897in}{1.558628in}}{\pgfqpoint{0.996073in}{1.564451in}}%
\pgfpathcurveto{\pgfqpoint{0.990249in}{1.570275in}}{\pgfqpoint{0.982349in}{1.573548in}}{\pgfqpoint{0.974112in}{1.573548in}}%
\pgfpathcurveto{\pgfqpoint{0.965876in}{1.573548in}}{\pgfqpoint{0.957976in}{1.570275in}}{\pgfqpoint{0.952152in}{1.564451in}}%
\pgfpathcurveto{\pgfqpoint{0.946328in}{1.558628in}}{\pgfqpoint{0.943056in}{1.550727in}}{\pgfqpoint{0.943056in}{1.542491in}}%
\pgfpathcurveto{\pgfqpoint{0.943056in}{1.534255in}}{\pgfqpoint{0.946328in}{1.526355in}}{\pgfqpoint{0.952152in}{1.520531in}}%
\pgfpathcurveto{\pgfqpoint{0.957976in}{1.514707in}}{\pgfqpoint{0.965876in}{1.511435in}}{\pgfqpoint{0.974112in}{1.511435in}}%
\pgfpathclose%
\pgfusepath{stroke,fill}%
\end{pgfscope}%
\begin{pgfscope}%
\pgfpathrectangle{\pgfqpoint{0.100000in}{0.220728in}}{\pgfqpoint{3.696000in}{3.696000in}}%
\pgfusepath{clip}%
\pgfsetbuttcap%
\pgfsetroundjoin%
\definecolor{currentfill}{rgb}{0.121569,0.466667,0.705882}%
\pgfsetfillcolor{currentfill}%
\pgfsetfillopacity{0.607575}%
\pgfsetlinewidth{1.003750pt}%
\definecolor{currentstroke}{rgb}{0.121569,0.466667,0.705882}%
\pgfsetstrokecolor{currentstroke}%
\pgfsetstrokeopacity{0.607575}%
\pgfsetdash{}{0pt}%
\pgfpathmoveto{\pgfqpoint{0.965392in}{1.498830in}}%
\pgfpathcurveto{\pgfqpoint{0.973629in}{1.498830in}}{\pgfqpoint{0.981529in}{1.502103in}}{\pgfqpoint{0.987353in}{1.507927in}}%
\pgfpathcurveto{\pgfqpoint{0.993177in}{1.513751in}}{\pgfqpoint{0.996449in}{1.521651in}}{\pgfqpoint{0.996449in}{1.529887in}}%
\pgfpathcurveto{\pgfqpoint{0.996449in}{1.538123in}}{\pgfqpoint{0.993177in}{1.546023in}}{\pgfqpoint{0.987353in}{1.551847in}}%
\pgfpathcurveto{\pgfqpoint{0.981529in}{1.557671in}}{\pgfqpoint{0.973629in}{1.560943in}}{\pgfqpoint{0.965392in}{1.560943in}}%
\pgfpathcurveto{\pgfqpoint{0.957156in}{1.560943in}}{\pgfqpoint{0.949256in}{1.557671in}}{\pgfqpoint{0.943432in}{1.551847in}}%
\pgfpathcurveto{\pgfqpoint{0.937608in}{1.546023in}}{\pgfqpoint{0.934336in}{1.538123in}}{\pgfqpoint{0.934336in}{1.529887in}}%
\pgfpathcurveto{\pgfqpoint{0.934336in}{1.521651in}}{\pgfqpoint{0.937608in}{1.513751in}}{\pgfqpoint{0.943432in}{1.507927in}}%
\pgfpathcurveto{\pgfqpoint{0.949256in}{1.502103in}}{\pgfqpoint{0.957156in}{1.498830in}}{\pgfqpoint{0.965392in}{1.498830in}}%
\pgfpathclose%
\pgfusepath{stroke,fill}%
\end{pgfscope}%
\begin{pgfscope}%
\pgfpathrectangle{\pgfqpoint{0.100000in}{0.220728in}}{\pgfqpoint{3.696000in}{3.696000in}}%
\pgfusepath{clip}%
\pgfsetbuttcap%
\pgfsetroundjoin%
\definecolor{currentfill}{rgb}{0.121569,0.466667,0.705882}%
\pgfsetfillcolor{currentfill}%
\pgfsetfillopacity{0.607582}%
\pgfsetlinewidth{1.003750pt}%
\definecolor{currentstroke}{rgb}{0.121569,0.466667,0.705882}%
\pgfsetstrokecolor{currentstroke}%
\pgfsetstrokeopacity{0.607582}%
\pgfsetdash{}{0pt}%
\pgfpathmoveto{\pgfqpoint{2.113001in}{1.792866in}}%
\pgfpathcurveto{\pgfqpoint{2.121237in}{1.792866in}}{\pgfqpoint{2.129138in}{1.796138in}}{\pgfqpoint{2.134961in}{1.801962in}}%
\pgfpathcurveto{\pgfqpoint{2.140785in}{1.807786in}}{\pgfqpoint{2.144058in}{1.815686in}}{\pgfqpoint{2.144058in}{1.823922in}}%
\pgfpathcurveto{\pgfqpoint{2.144058in}{1.832158in}}{\pgfqpoint{2.140785in}{1.840059in}}{\pgfqpoint{2.134961in}{1.845882in}}%
\pgfpathcurveto{\pgfqpoint{2.129138in}{1.851706in}}{\pgfqpoint{2.121237in}{1.854979in}}{\pgfqpoint{2.113001in}{1.854979in}}%
\pgfpathcurveto{\pgfqpoint{2.104765in}{1.854979in}}{\pgfqpoint{2.096865in}{1.851706in}}{\pgfqpoint{2.091041in}{1.845882in}}%
\pgfpathcurveto{\pgfqpoint{2.085217in}{1.840059in}}{\pgfqpoint{2.081945in}{1.832158in}}{\pgfqpoint{2.081945in}{1.823922in}}%
\pgfpathcurveto{\pgfqpoint{2.081945in}{1.815686in}}{\pgfqpoint{2.085217in}{1.807786in}}{\pgfqpoint{2.091041in}{1.801962in}}%
\pgfpathcurveto{\pgfqpoint{2.096865in}{1.796138in}}{\pgfqpoint{2.104765in}{1.792866in}}{\pgfqpoint{2.113001in}{1.792866in}}%
\pgfpathclose%
\pgfusepath{stroke,fill}%
\end{pgfscope}%
\begin{pgfscope}%
\pgfpathrectangle{\pgfqpoint{0.100000in}{0.220728in}}{\pgfqpoint{3.696000in}{3.696000in}}%
\pgfusepath{clip}%
\pgfsetbuttcap%
\pgfsetroundjoin%
\definecolor{currentfill}{rgb}{0.121569,0.466667,0.705882}%
\pgfsetfillcolor{currentfill}%
\pgfsetfillopacity{0.610265}%
\pgfsetlinewidth{1.003750pt}%
\definecolor{currentstroke}{rgb}{0.121569,0.466667,0.705882}%
\pgfsetstrokecolor{currentstroke}%
\pgfsetstrokeopacity{0.610265}%
\pgfsetdash{}{0pt}%
\pgfpathmoveto{\pgfqpoint{0.956692in}{1.492377in}}%
\pgfpathcurveto{\pgfqpoint{0.964929in}{1.492377in}}{\pgfqpoint{0.972829in}{1.495650in}}{\pgfqpoint{0.978653in}{1.501473in}}%
\pgfpathcurveto{\pgfqpoint{0.984477in}{1.507297in}}{\pgfqpoint{0.987749in}{1.515197in}}{\pgfqpoint{0.987749in}{1.523434in}}%
\pgfpathcurveto{\pgfqpoint{0.987749in}{1.531670in}}{\pgfqpoint{0.984477in}{1.539570in}}{\pgfqpoint{0.978653in}{1.545394in}}%
\pgfpathcurveto{\pgfqpoint{0.972829in}{1.551218in}}{\pgfqpoint{0.964929in}{1.554490in}}{\pgfqpoint{0.956692in}{1.554490in}}%
\pgfpathcurveto{\pgfqpoint{0.948456in}{1.554490in}}{\pgfqpoint{0.940556in}{1.551218in}}{\pgfqpoint{0.934732in}{1.545394in}}%
\pgfpathcurveto{\pgfqpoint{0.928908in}{1.539570in}}{\pgfqpoint{0.925636in}{1.531670in}}{\pgfqpoint{0.925636in}{1.523434in}}%
\pgfpathcurveto{\pgfqpoint{0.925636in}{1.515197in}}{\pgfqpoint{0.928908in}{1.507297in}}{\pgfqpoint{0.934732in}{1.501473in}}%
\pgfpathcurveto{\pgfqpoint{0.940556in}{1.495650in}}{\pgfqpoint{0.948456in}{1.492377in}}{\pgfqpoint{0.956692in}{1.492377in}}%
\pgfpathclose%
\pgfusepath{stroke,fill}%
\end{pgfscope}%
\begin{pgfscope}%
\pgfpathrectangle{\pgfqpoint{0.100000in}{0.220728in}}{\pgfqpoint{3.696000in}{3.696000in}}%
\pgfusepath{clip}%
\pgfsetbuttcap%
\pgfsetroundjoin%
\definecolor{currentfill}{rgb}{0.121569,0.466667,0.705882}%
\pgfsetfillcolor{currentfill}%
\pgfsetfillopacity{0.610284}%
\pgfsetlinewidth{1.003750pt}%
\definecolor{currentstroke}{rgb}{0.121569,0.466667,0.705882}%
\pgfsetstrokecolor{currentstroke}%
\pgfsetstrokeopacity{0.610284}%
\pgfsetdash{}{0pt}%
\pgfpathmoveto{\pgfqpoint{2.115098in}{1.789310in}}%
\pgfpathcurveto{\pgfqpoint{2.123334in}{1.789310in}}{\pgfqpoint{2.131234in}{1.792582in}}{\pgfqpoint{2.137058in}{1.798406in}}%
\pgfpathcurveto{\pgfqpoint{2.142882in}{1.804230in}}{\pgfqpoint{2.146155in}{1.812130in}}{\pgfqpoint{2.146155in}{1.820366in}}%
\pgfpathcurveto{\pgfqpoint{2.146155in}{1.828602in}}{\pgfqpoint{2.142882in}{1.836502in}}{\pgfqpoint{2.137058in}{1.842326in}}%
\pgfpathcurveto{\pgfqpoint{2.131234in}{1.848150in}}{\pgfqpoint{2.123334in}{1.851423in}}{\pgfqpoint{2.115098in}{1.851423in}}%
\pgfpathcurveto{\pgfqpoint{2.106862in}{1.851423in}}{\pgfqpoint{2.098962in}{1.848150in}}{\pgfqpoint{2.093138in}{1.842326in}}%
\pgfpathcurveto{\pgfqpoint{2.087314in}{1.836502in}}{\pgfqpoint{2.084042in}{1.828602in}}{\pgfqpoint{2.084042in}{1.820366in}}%
\pgfpathcurveto{\pgfqpoint{2.084042in}{1.812130in}}{\pgfqpoint{2.087314in}{1.804230in}}{\pgfqpoint{2.093138in}{1.798406in}}%
\pgfpathcurveto{\pgfqpoint{2.098962in}{1.792582in}}{\pgfqpoint{2.106862in}{1.789310in}}{\pgfqpoint{2.115098in}{1.789310in}}%
\pgfpathclose%
\pgfusepath{stroke,fill}%
\end{pgfscope}%
\begin{pgfscope}%
\pgfpathrectangle{\pgfqpoint{0.100000in}{0.220728in}}{\pgfqpoint{3.696000in}{3.696000in}}%
\pgfusepath{clip}%
\pgfsetbuttcap%
\pgfsetroundjoin%
\definecolor{currentfill}{rgb}{0.121569,0.466667,0.705882}%
\pgfsetfillcolor{currentfill}%
\pgfsetfillopacity{0.613265}%
\pgfsetlinewidth{1.003750pt}%
\definecolor{currentstroke}{rgb}{0.121569,0.466667,0.705882}%
\pgfsetstrokecolor{currentstroke}%
\pgfsetstrokeopacity{0.613265}%
\pgfsetdash{}{0pt}%
\pgfpathmoveto{\pgfqpoint{2.117839in}{1.785730in}}%
\pgfpathcurveto{\pgfqpoint{2.126075in}{1.785730in}}{\pgfqpoint{2.133975in}{1.789002in}}{\pgfqpoint{2.139799in}{1.794826in}}%
\pgfpathcurveto{\pgfqpoint{2.145623in}{1.800650in}}{\pgfqpoint{2.148895in}{1.808550in}}{\pgfqpoint{2.148895in}{1.816787in}}%
\pgfpathcurveto{\pgfqpoint{2.148895in}{1.825023in}}{\pgfqpoint{2.145623in}{1.832923in}}{\pgfqpoint{2.139799in}{1.838747in}}%
\pgfpathcurveto{\pgfqpoint{2.133975in}{1.844571in}}{\pgfqpoint{2.126075in}{1.847843in}}{\pgfqpoint{2.117839in}{1.847843in}}%
\pgfpathcurveto{\pgfqpoint{2.109602in}{1.847843in}}{\pgfqpoint{2.101702in}{1.844571in}}{\pgfqpoint{2.095878in}{1.838747in}}%
\pgfpathcurveto{\pgfqpoint{2.090054in}{1.832923in}}{\pgfqpoint{2.086782in}{1.825023in}}{\pgfqpoint{2.086782in}{1.816787in}}%
\pgfpathcurveto{\pgfqpoint{2.086782in}{1.808550in}}{\pgfqpoint{2.090054in}{1.800650in}}{\pgfqpoint{2.095878in}{1.794826in}}%
\pgfpathcurveto{\pgfqpoint{2.101702in}{1.789002in}}{\pgfqpoint{2.109602in}{1.785730in}}{\pgfqpoint{2.117839in}{1.785730in}}%
\pgfpathclose%
\pgfusepath{stroke,fill}%
\end{pgfscope}%
\begin{pgfscope}%
\pgfpathrectangle{\pgfqpoint{0.100000in}{0.220728in}}{\pgfqpoint{3.696000in}{3.696000in}}%
\pgfusepath{clip}%
\pgfsetbuttcap%
\pgfsetroundjoin%
\definecolor{currentfill}{rgb}{0.121569,0.466667,0.705882}%
\pgfsetfillcolor{currentfill}%
\pgfsetfillopacity{0.613459}%
\pgfsetlinewidth{1.003750pt}%
\definecolor{currentstroke}{rgb}{0.121569,0.466667,0.705882}%
\pgfsetstrokecolor{currentstroke}%
\pgfsetstrokeopacity{0.613459}%
\pgfsetdash{}{0pt}%
\pgfpathmoveto{\pgfqpoint{0.951508in}{1.489229in}}%
\pgfpathcurveto{\pgfqpoint{0.959744in}{1.489229in}}{\pgfqpoint{0.967645in}{1.492501in}}{\pgfqpoint{0.973468in}{1.498325in}}%
\pgfpathcurveto{\pgfqpoint{0.979292in}{1.504149in}}{\pgfqpoint{0.982565in}{1.512049in}}{\pgfqpoint{0.982565in}{1.520285in}}%
\pgfpathcurveto{\pgfqpoint{0.982565in}{1.528522in}}{\pgfqpoint{0.979292in}{1.536422in}}{\pgfqpoint{0.973468in}{1.542246in}}%
\pgfpathcurveto{\pgfqpoint{0.967645in}{1.548070in}}{\pgfqpoint{0.959744in}{1.551342in}}{\pgfqpoint{0.951508in}{1.551342in}}%
\pgfpathcurveto{\pgfqpoint{0.943272in}{1.551342in}}{\pgfqpoint{0.935372in}{1.548070in}}{\pgfqpoint{0.929548in}{1.542246in}}%
\pgfpathcurveto{\pgfqpoint{0.923724in}{1.536422in}}{\pgfqpoint{0.920452in}{1.528522in}}{\pgfqpoint{0.920452in}{1.520285in}}%
\pgfpathcurveto{\pgfqpoint{0.920452in}{1.512049in}}{\pgfqpoint{0.923724in}{1.504149in}}{\pgfqpoint{0.929548in}{1.498325in}}%
\pgfpathcurveto{\pgfqpoint{0.935372in}{1.492501in}}{\pgfqpoint{0.943272in}{1.489229in}}{\pgfqpoint{0.951508in}{1.489229in}}%
\pgfpathclose%
\pgfusepath{stroke,fill}%
\end{pgfscope}%
\begin{pgfscope}%
\pgfpathrectangle{\pgfqpoint{0.100000in}{0.220728in}}{\pgfqpoint{3.696000in}{3.696000in}}%
\pgfusepath{clip}%
\pgfsetbuttcap%
\pgfsetroundjoin%
\definecolor{currentfill}{rgb}{0.121569,0.466667,0.705882}%
\pgfsetfillcolor{currentfill}%
\pgfsetfillopacity{0.615466}%
\pgfsetlinewidth{1.003750pt}%
\definecolor{currentstroke}{rgb}{0.121569,0.466667,0.705882}%
\pgfsetstrokecolor{currentstroke}%
\pgfsetstrokeopacity{0.615466}%
\pgfsetdash{}{0pt}%
\pgfpathmoveto{\pgfqpoint{0.945090in}{1.488347in}}%
\pgfpathcurveto{\pgfqpoint{0.953326in}{1.488347in}}{\pgfqpoint{0.961226in}{1.491619in}}{\pgfqpoint{0.967050in}{1.497443in}}%
\pgfpathcurveto{\pgfqpoint{0.972874in}{1.503267in}}{\pgfqpoint{0.976147in}{1.511167in}}{\pgfqpoint{0.976147in}{1.519403in}}%
\pgfpathcurveto{\pgfqpoint{0.976147in}{1.527640in}}{\pgfqpoint{0.972874in}{1.535540in}}{\pgfqpoint{0.967050in}{1.541364in}}%
\pgfpathcurveto{\pgfqpoint{0.961226in}{1.547188in}}{\pgfqpoint{0.953326in}{1.550460in}}{\pgfqpoint{0.945090in}{1.550460in}}%
\pgfpathcurveto{\pgfqpoint{0.936854in}{1.550460in}}{\pgfqpoint{0.928954in}{1.547188in}}{\pgfqpoint{0.923130in}{1.541364in}}%
\pgfpathcurveto{\pgfqpoint{0.917306in}{1.535540in}}{\pgfqpoint{0.914034in}{1.527640in}}{\pgfqpoint{0.914034in}{1.519403in}}%
\pgfpathcurveto{\pgfqpoint{0.914034in}{1.511167in}}{\pgfqpoint{0.917306in}{1.503267in}}{\pgfqpoint{0.923130in}{1.497443in}}%
\pgfpathcurveto{\pgfqpoint{0.928954in}{1.491619in}}{\pgfqpoint{0.936854in}{1.488347in}}{\pgfqpoint{0.945090in}{1.488347in}}%
\pgfpathclose%
\pgfusepath{stroke,fill}%
\end{pgfscope}%
\begin{pgfscope}%
\pgfpathrectangle{\pgfqpoint{0.100000in}{0.220728in}}{\pgfqpoint{3.696000in}{3.696000in}}%
\pgfusepath{clip}%
\pgfsetbuttcap%
\pgfsetroundjoin%
\definecolor{currentfill}{rgb}{0.121569,0.466667,0.705882}%
\pgfsetfillcolor{currentfill}%
\pgfsetfillopacity{0.616906}%
\pgfsetlinewidth{1.003750pt}%
\definecolor{currentstroke}{rgb}{0.121569,0.466667,0.705882}%
\pgfsetstrokecolor{currentstroke}%
\pgfsetstrokeopacity{0.616906}%
\pgfsetdash{}{0pt}%
\pgfpathmoveto{\pgfqpoint{2.119775in}{1.782849in}}%
\pgfpathcurveto{\pgfqpoint{2.128012in}{1.782849in}}{\pgfqpoint{2.135912in}{1.786122in}}{\pgfqpoint{2.141736in}{1.791946in}}%
\pgfpathcurveto{\pgfqpoint{2.147560in}{1.797770in}}{\pgfqpoint{2.150832in}{1.805670in}}{\pgfqpoint{2.150832in}{1.813906in}}%
\pgfpathcurveto{\pgfqpoint{2.150832in}{1.822142in}}{\pgfqpoint{2.147560in}{1.830042in}}{\pgfqpoint{2.141736in}{1.835866in}}%
\pgfpathcurveto{\pgfqpoint{2.135912in}{1.841690in}}{\pgfqpoint{2.128012in}{1.844962in}}{\pgfqpoint{2.119775in}{1.844962in}}%
\pgfpathcurveto{\pgfqpoint{2.111539in}{1.844962in}}{\pgfqpoint{2.103639in}{1.841690in}}{\pgfqpoint{2.097815in}{1.835866in}}%
\pgfpathcurveto{\pgfqpoint{2.091991in}{1.830042in}}{\pgfqpoint{2.088719in}{1.822142in}}{\pgfqpoint{2.088719in}{1.813906in}}%
\pgfpathcurveto{\pgfqpoint{2.088719in}{1.805670in}}{\pgfqpoint{2.091991in}{1.797770in}}{\pgfqpoint{2.097815in}{1.791946in}}%
\pgfpathcurveto{\pgfqpoint{2.103639in}{1.786122in}}{\pgfqpoint{2.111539in}{1.782849in}}{\pgfqpoint{2.119775in}{1.782849in}}%
\pgfpathclose%
\pgfusepath{stroke,fill}%
\end{pgfscope}%
\begin{pgfscope}%
\pgfpathrectangle{\pgfqpoint{0.100000in}{0.220728in}}{\pgfqpoint{3.696000in}{3.696000in}}%
\pgfusepath{clip}%
\pgfsetbuttcap%
\pgfsetroundjoin%
\definecolor{currentfill}{rgb}{0.121569,0.466667,0.705882}%
\pgfsetfillcolor{currentfill}%
\pgfsetfillopacity{0.617418}%
\pgfsetlinewidth{1.003750pt}%
\definecolor{currentstroke}{rgb}{0.121569,0.466667,0.705882}%
\pgfsetstrokecolor{currentstroke}%
\pgfsetstrokeopacity{0.617418}%
\pgfsetdash{}{0pt}%
\pgfpathmoveto{\pgfqpoint{0.942860in}{1.488574in}}%
\pgfpathcurveto{\pgfqpoint{0.951097in}{1.488574in}}{\pgfqpoint{0.958997in}{1.491846in}}{\pgfqpoint{0.964821in}{1.497670in}}%
\pgfpathcurveto{\pgfqpoint{0.970644in}{1.503494in}}{\pgfqpoint{0.973917in}{1.511394in}}{\pgfqpoint{0.973917in}{1.519631in}}%
\pgfpathcurveto{\pgfqpoint{0.973917in}{1.527867in}}{\pgfqpoint{0.970644in}{1.535767in}}{\pgfqpoint{0.964821in}{1.541591in}}%
\pgfpathcurveto{\pgfqpoint{0.958997in}{1.547415in}}{\pgfqpoint{0.951097in}{1.550687in}}{\pgfqpoint{0.942860in}{1.550687in}}%
\pgfpathcurveto{\pgfqpoint{0.934624in}{1.550687in}}{\pgfqpoint{0.926724in}{1.547415in}}{\pgfqpoint{0.920900in}{1.541591in}}%
\pgfpathcurveto{\pgfqpoint{0.915076in}{1.535767in}}{\pgfqpoint{0.911804in}{1.527867in}}{\pgfqpoint{0.911804in}{1.519631in}}%
\pgfpathcurveto{\pgfqpoint{0.911804in}{1.511394in}}{\pgfqpoint{0.915076in}{1.503494in}}{\pgfqpoint{0.920900in}{1.497670in}}%
\pgfpathcurveto{\pgfqpoint{0.926724in}{1.491846in}}{\pgfqpoint{0.934624in}{1.488574in}}{\pgfqpoint{0.942860in}{1.488574in}}%
\pgfpathclose%
\pgfusepath{stroke,fill}%
\end{pgfscope}%
\begin{pgfscope}%
\pgfpathrectangle{\pgfqpoint{0.100000in}{0.220728in}}{\pgfqpoint{3.696000in}{3.696000in}}%
\pgfusepath{clip}%
\pgfsetbuttcap%
\pgfsetroundjoin%
\definecolor{currentfill}{rgb}{0.121569,0.466667,0.705882}%
\pgfsetfillcolor{currentfill}%
\pgfsetfillopacity{0.618133}%
\pgfsetlinewidth{1.003750pt}%
\definecolor{currentstroke}{rgb}{0.121569,0.466667,0.705882}%
\pgfsetstrokecolor{currentstroke}%
\pgfsetstrokeopacity{0.618133}%
\pgfsetdash{}{0pt}%
\pgfpathmoveto{\pgfqpoint{0.939828in}{1.485476in}}%
\pgfpathcurveto{\pgfqpoint{0.948064in}{1.485476in}}{\pgfqpoint{0.955964in}{1.488748in}}{\pgfqpoint{0.961788in}{1.494572in}}%
\pgfpathcurveto{\pgfqpoint{0.967612in}{1.500396in}}{\pgfqpoint{0.970884in}{1.508296in}}{\pgfqpoint{0.970884in}{1.516533in}}%
\pgfpathcurveto{\pgfqpoint{0.970884in}{1.524769in}}{\pgfqpoint{0.967612in}{1.532669in}}{\pgfqpoint{0.961788in}{1.538493in}}%
\pgfpathcurveto{\pgfqpoint{0.955964in}{1.544317in}}{\pgfqpoint{0.948064in}{1.547589in}}{\pgfqpoint{0.939828in}{1.547589in}}%
\pgfpathcurveto{\pgfqpoint{0.931591in}{1.547589in}}{\pgfqpoint{0.923691in}{1.544317in}}{\pgfqpoint{0.917867in}{1.538493in}}%
\pgfpathcurveto{\pgfqpoint{0.912043in}{1.532669in}}{\pgfqpoint{0.908771in}{1.524769in}}{\pgfqpoint{0.908771in}{1.516533in}}%
\pgfpathcurveto{\pgfqpoint{0.908771in}{1.508296in}}{\pgfqpoint{0.912043in}{1.500396in}}{\pgfqpoint{0.917867in}{1.494572in}}%
\pgfpathcurveto{\pgfqpoint{0.923691in}{1.488748in}}{\pgfqpoint{0.931591in}{1.485476in}}{\pgfqpoint{0.939828in}{1.485476in}}%
\pgfpathclose%
\pgfusepath{stroke,fill}%
\end{pgfscope}%
\begin{pgfscope}%
\pgfpathrectangle{\pgfqpoint{0.100000in}{0.220728in}}{\pgfqpoint{3.696000in}{3.696000in}}%
\pgfusepath{clip}%
\pgfsetbuttcap%
\pgfsetroundjoin%
\definecolor{currentfill}{rgb}{0.121569,0.466667,0.705882}%
\pgfsetfillcolor{currentfill}%
\pgfsetfillopacity{0.620064}%
\pgfsetlinewidth{1.003750pt}%
\definecolor{currentstroke}{rgb}{0.121569,0.466667,0.705882}%
\pgfsetstrokecolor{currentstroke}%
\pgfsetstrokeopacity{0.620064}%
\pgfsetdash{}{0pt}%
\pgfpathmoveto{\pgfqpoint{0.935004in}{1.482809in}}%
\pgfpathcurveto{\pgfqpoint{0.943240in}{1.482809in}}{\pgfqpoint{0.951140in}{1.486081in}}{\pgfqpoint{0.956964in}{1.491905in}}%
\pgfpathcurveto{\pgfqpoint{0.962788in}{1.497729in}}{\pgfqpoint{0.966061in}{1.505629in}}{\pgfqpoint{0.966061in}{1.513865in}}%
\pgfpathcurveto{\pgfqpoint{0.966061in}{1.522101in}}{\pgfqpoint{0.962788in}{1.530001in}}{\pgfqpoint{0.956964in}{1.535825in}}%
\pgfpathcurveto{\pgfqpoint{0.951140in}{1.541649in}}{\pgfqpoint{0.943240in}{1.544922in}}{\pgfqpoint{0.935004in}{1.544922in}}%
\pgfpathcurveto{\pgfqpoint{0.926768in}{1.544922in}}{\pgfqpoint{0.918868in}{1.541649in}}{\pgfqpoint{0.913044in}{1.535825in}}%
\pgfpathcurveto{\pgfqpoint{0.907220in}{1.530001in}}{\pgfqpoint{0.903948in}{1.522101in}}{\pgfqpoint{0.903948in}{1.513865in}}%
\pgfpathcurveto{\pgfqpoint{0.903948in}{1.505629in}}{\pgfqpoint{0.907220in}{1.497729in}}{\pgfqpoint{0.913044in}{1.491905in}}%
\pgfpathcurveto{\pgfqpoint{0.918868in}{1.486081in}}{\pgfqpoint{0.926768in}{1.482809in}}{\pgfqpoint{0.935004in}{1.482809in}}%
\pgfpathclose%
\pgfusepath{stroke,fill}%
\end{pgfscope}%
\begin{pgfscope}%
\pgfpathrectangle{\pgfqpoint{0.100000in}{0.220728in}}{\pgfqpoint{3.696000in}{3.696000in}}%
\pgfusepath{clip}%
\pgfsetbuttcap%
\pgfsetroundjoin%
\definecolor{currentfill}{rgb}{0.121569,0.466667,0.705882}%
\pgfsetfillcolor{currentfill}%
\pgfsetfillopacity{0.620949}%
\pgfsetlinewidth{1.003750pt}%
\definecolor{currentstroke}{rgb}{0.121569,0.466667,0.705882}%
\pgfsetstrokecolor{currentstroke}%
\pgfsetstrokeopacity{0.620949}%
\pgfsetdash{}{0pt}%
\pgfpathmoveto{\pgfqpoint{2.120884in}{1.779644in}}%
\pgfpathcurveto{\pgfqpoint{2.129120in}{1.779644in}}{\pgfqpoint{2.137020in}{1.782916in}}{\pgfqpoint{2.142844in}{1.788740in}}%
\pgfpathcurveto{\pgfqpoint{2.148668in}{1.794564in}}{\pgfqpoint{2.151940in}{1.802464in}}{\pgfqpoint{2.151940in}{1.810700in}}%
\pgfpathcurveto{\pgfqpoint{2.151940in}{1.818936in}}{\pgfqpoint{2.148668in}{1.826836in}}{\pgfqpoint{2.142844in}{1.832660in}}%
\pgfpathcurveto{\pgfqpoint{2.137020in}{1.838484in}}{\pgfqpoint{2.129120in}{1.841757in}}{\pgfqpoint{2.120884in}{1.841757in}}%
\pgfpathcurveto{\pgfqpoint{2.112648in}{1.841757in}}{\pgfqpoint{2.104748in}{1.838484in}}{\pgfqpoint{2.098924in}{1.832660in}}%
\pgfpathcurveto{\pgfqpoint{2.093100in}{1.826836in}}{\pgfqpoint{2.089827in}{1.818936in}}{\pgfqpoint{2.089827in}{1.810700in}}%
\pgfpathcurveto{\pgfqpoint{2.089827in}{1.802464in}}{\pgfqpoint{2.093100in}{1.794564in}}{\pgfqpoint{2.098924in}{1.788740in}}%
\pgfpathcurveto{\pgfqpoint{2.104748in}{1.782916in}}{\pgfqpoint{2.112648in}{1.779644in}}{\pgfqpoint{2.120884in}{1.779644in}}%
\pgfpathclose%
\pgfusepath{stroke,fill}%
\end{pgfscope}%
\begin{pgfscope}%
\pgfpathrectangle{\pgfqpoint{0.100000in}{0.220728in}}{\pgfqpoint{3.696000in}{3.696000in}}%
\pgfusepath{clip}%
\pgfsetbuttcap%
\pgfsetroundjoin%
\definecolor{currentfill}{rgb}{0.121569,0.466667,0.705882}%
\pgfsetfillcolor{currentfill}%
\pgfsetfillopacity{0.622949}%
\pgfsetlinewidth{1.003750pt}%
\definecolor{currentstroke}{rgb}{0.121569,0.466667,0.705882}%
\pgfsetstrokecolor{currentstroke}%
\pgfsetstrokeopacity{0.622949}%
\pgfsetdash{}{0pt}%
\pgfpathmoveto{\pgfqpoint{0.927578in}{1.472603in}}%
\pgfpathcurveto{\pgfqpoint{0.935814in}{1.472603in}}{\pgfqpoint{0.943714in}{1.475876in}}{\pgfqpoint{0.949538in}{1.481700in}}%
\pgfpathcurveto{\pgfqpoint{0.955362in}{1.487524in}}{\pgfqpoint{0.958635in}{1.495424in}}{\pgfqpoint{0.958635in}{1.503660in}}%
\pgfpathcurveto{\pgfqpoint{0.958635in}{1.511896in}}{\pgfqpoint{0.955362in}{1.519796in}}{\pgfqpoint{0.949538in}{1.525620in}}%
\pgfpathcurveto{\pgfqpoint{0.943714in}{1.531444in}}{\pgfqpoint{0.935814in}{1.534716in}}{\pgfqpoint{0.927578in}{1.534716in}}%
\pgfpathcurveto{\pgfqpoint{0.919342in}{1.534716in}}{\pgfqpoint{0.911442in}{1.531444in}}{\pgfqpoint{0.905618in}{1.525620in}}%
\pgfpathcurveto{\pgfqpoint{0.899794in}{1.519796in}}{\pgfqpoint{0.896522in}{1.511896in}}{\pgfqpoint{0.896522in}{1.503660in}}%
\pgfpathcurveto{\pgfqpoint{0.896522in}{1.495424in}}{\pgfqpoint{0.899794in}{1.487524in}}{\pgfqpoint{0.905618in}{1.481700in}}%
\pgfpathcurveto{\pgfqpoint{0.911442in}{1.475876in}}{\pgfqpoint{0.919342in}{1.472603in}}{\pgfqpoint{0.927578in}{1.472603in}}%
\pgfpathclose%
\pgfusepath{stroke,fill}%
\end{pgfscope}%
\begin{pgfscope}%
\pgfpathrectangle{\pgfqpoint{0.100000in}{0.220728in}}{\pgfqpoint{3.696000in}{3.696000in}}%
\pgfusepath{clip}%
\pgfsetbuttcap%
\pgfsetroundjoin%
\definecolor{currentfill}{rgb}{0.121569,0.466667,0.705882}%
\pgfsetfillcolor{currentfill}%
\pgfsetfillopacity{0.625268}%
\pgfsetlinewidth{1.003750pt}%
\definecolor{currentstroke}{rgb}{0.121569,0.466667,0.705882}%
\pgfsetstrokecolor{currentstroke}%
\pgfsetstrokeopacity{0.625268}%
\pgfsetdash{}{0pt}%
\pgfpathmoveto{\pgfqpoint{2.124490in}{1.778041in}}%
\pgfpathcurveto{\pgfqpoint{2.132727in}{1.778041in}}{\pgfqpoint{2.140627in}{1.781314in}}{\pgfqpoint{2.146451in}{1.787138in}}%
\pgfpathcurveto{\pgfqpoint{2.152275in}{1.792962in}}{\pgfqpoint{2.155547in}{1.800862in}}{\pgfqpoint{2.155547in}{1.809098in}}%
\pgfpathcurveto{\pgfqpoint{2.155547in}{1.817334in}}{\pgfqpoint{2.152275in}{1.825234in}}{\pgfqpoint{2.146451in}{1.831058in}}%
\pgfpathcurveto{\pgfqpoint{2.140627in}{1.836882in}}{\pgfqpoint{2.132727in}{1.840154in}}{\pgfqpoint{2.124490in}{1.840154in}}%
\pgfpathcurveto{\pgfqpoint{2.116254in}{1.840154in}}{\pgfqpoint{2.108354in}{1.836882in}}{\pgfqpoint{2.102530in}{1.831058in}}%
\pgfpathcurveto{\pgfqpoint{2.096706in}{1.825234in}}{\pgfqpoint{2.093434in}{1.817334in}}{\pgfqpoint{2.093434in}{1.809098in}}%
\pgfpathcurveto{\pgfqpoint{2.093434in}{1.800862in}}{\pgfqpoint{2.096706in}{1.792962in}}{\pgfqpoint{2.102530in}{1.787138in}}%
\pgfpathcurveto{\pgfqpoint{2.108354in}{1.781314in}}{\pgfqpoint{2.116254in}{1.778041in}}{\pgfqpoint{2.124490in}{1.778041in}}%
\pgfpathclose%
\pgfusepath{stroke,fill}%
\end{pgfscope}%
\begin{pgfscope}%
\pgfpathrectangle{\pgfqpoint{0.100000in}{0.220728in}}{\pgfqpoint{3.696000in}{3.696000in}}%
\pgfusepath{clip}%
\pgfsetbuttcap%
\pgfsetroundjoin%
\definecolor{currentfill}{rgb}{0.121569,0.466667,0.705882}%
\pgfsetfillcolor{currentfill}%
\pgfsetfillopacity{0.625909}%
\pgfsetlinewidth{1.003750pt}%
\definecolor{currentstroke}{rgb}{0.121569,0.466667,0.705882}%
\pgfsetstrokecolor{currentstroke}%
\pgfsetstrokeopacity{0.625909}%
\pgfsetdash{}{0pt}%
\pgfpathmoveto{\pgfqpoint{0.917662in}{1.470900in}}%
\pgfpathcurveto{\pgfqpoint{0.925898in}{1.470900in}}{\pgfqpoint{0.933798in}{1.474173in}}{\pgfqpoint{0.939622in}{1.479996in}}%
\pgfpathcurveto{\pgfqpoint{0.945446in}{1.485820in}}{\pgfqpoint{0.948718in}{1.493720in}}{\pgfqpoint{0.948718in}{1.501957in}}%
\pgfpathcurveto{\pgfqpoint{0.948718in}{1.510193in}}{\pgfqpoint{0.945446in}{1.518093in}}{\pgfqpoint{0.939622in}{1.523917in}}%
\pgfpathcurveto{\pgfqpoint{0.933798in}{1.529741in}}{\pgfqpoint{0.925898in}{1.533013in}}{\pgfqpoint{0.917662in}{1.533013in}}%
\pgfpathcurveto{\pgfqpoint{0.909425in}{1.533013in}}{\pgfqpoint{0.901525in}{1.529741in}}{\pgfqpoint{0.895701in}{1.523917in}}%
\pgfpathcurveto{\pgfqpoint{0.889877in}{1.518093in}}{\pgfqpoint{0.886605in}{1.510193in}}{\pgfqpoint{0.886605in}{1.501957in}}%
\pgfpathcurveto{\pgfqpoint{0.886605in}{1.493720in}}{\pgfqpoint{0.889877in}{1.485820in}}{\pgfqpoint{0.895701in}{1.479996in}}%
\pgfpathcurveto{\pgfqpoint{0.901525in}{1.474173in}}{\pgfqpoint{0.909425in}{1.470900in}}{\pgfqpoint{0.917662in}{1.470900in}}%
\pgfpathclose%
\pgfusepath{stroke,fill}%
\end{pgfscope}%
\begin{pgfscope}%
\pgfpathrectangle{\pgfqpoint{0.100000in}{0.220728in}}{\pgfqpoint{3.696000in}{3.696000in}}%
\pgfusepath{clip}%
\pgfsetbuttcap%
\pgfsetroundjoin%
\definecolor{currentfill}{rgb}{0.121569,0.466667,0.705882}%
\pgfsetfillcolor{currentfill}%
\pgfsetfillopacity{0.629325}%
\pgfsetlinewidth{1.003750pt}%
\definecolor{currentstroke}{rgb}{0.121569,0.466667,0.705882}%
\pgfsetstrokecolor{currentstroke}%
\pgfsetstrokeopacity{0.629325}%
\pgfsetdash{}{0pt}%
\pgfpathmoveto{\pgfqpoint{0.912270in}{1.468476in}}%
\pgfpathcurveto{\pgfqpoint{0.920507in}{1.468476in}}{\pgfqpoint{0.928407in}{1.471749in}}{\pgfqpoint{0.934231in}{1.477573in}}%
\pgfpathcurveto{\pgfqpoint{0.940055in}{1.483397in}}{\pgfqpoint{0.943327in}{1.491297in}}{\pgfqpoint{0.943327in}{1.499533in}}%
\pgfpathcurveto{\pgfqpoint{0.943327in}{1.507769in}}{\pgfqpoint{0.940055in}{1.515669in}}{\pgfqpoint{0.934231in}{1.521493in}}%
\pgfpathcurveto{\pgfqpoint{0.928407in}{1.527317in}}{\pgfqpoint{0.920507in}{1.530589in}}{\pgfqpoint{0.912270in}{1.530589in}}%
\pgfpathcurveto{\pgfqpoint{0.904034in}{1.530589in}}{\pgfqpoint{0.896134in}{1.527317in}}{\pgfqpoint{0.890310in}{1.521493in}}%
\pgfpathcurveto{\pgfqpoint{0.884486in}{1.515669in}}{\pgfqpoint{0.881214in}{1.507769in}}{\pgfqpoint{0.881214in}{1.499533in}}%
\pgfpathcurveto{\pgfqpoint{0.881214in}{1.491297in}}{\pgfqpoint{0.884486in}{1.483397in}}{\pgfqpoint{0.890310in}{1.477573in}}%
\pgfpathcurveto{\pgfqpoint{0.896134in}{1.471749in}}{\pgfqpoint{0.904034in}{1.468476in}}{\pgfqpoint{0.912270in}{1.468476in}}%
\pgfpathclose%
\pgfusepath{stroke,fill}%
\end{pgfscope}%
\begin{pgfscope}%
\pgfpathrectangle{\pgfqpoint{0.100000in}{0.220728in}}{\pgfqpoint{3.696000in}{3.696000in}}%
\pgfusepath{clip}%
\pgfsetbuttcap%
\pgfsetroundjoin%
\definecolor{currentfill}{rgb}{0.121569,0.466667,0.705882}%
\pgfsetfillcolor{currentfill}%
\pgfsetfillopacity{0.629707}%
\pgfsetlinewidth{1.003750pt}%
\definecolor{currentstroke}{rgb}{0.121569,0.466667,0.705882}%
\pgfsetstrokecolor{currentstroke}%
\pgfsetstrokeopacity{0.629707}%
\pgfsetdash{}{0pt}%
\pgfpathmoveto{\pgfqpoint{2.127846in}{1.774997in}}%
\pgfpathcurveto{\pgfqpoint{2.136082in}{1.774997in}}{\pgfqpoint{2.143982in}{1.778269in}}{\pgfqpoint{2.149806in}{1.784093in}}%
\pgfpathcurveto{\pgfqpoint{2.155630in}{1.789917in}}{\pgfqpoint{2.158903in}{1.797817in}}{\pgfqpoint{2.158903in}{1.806053in}}%
\pgfpathcurveto{\pgfqpoint{2.158903in}{1.814290in}}{\pgfqpoint{2.155630in}{1.822190in}}{\pgfqpoint{2.149806in}{1.828014in}}%
\pgfpathcurveto{\pgfqpoint{2.143982in}{1.833838in}}{\pgfqpoint{2.136082in}{1.837110in}}{\pgfqpoint{2.127846in}{1.837110in}}%
\pgfpathcurveto{\pgfqpoint{2.119610in}{1.837110in}}{\pgfqpoint{2.111710in}{1.833838in}}{\pgfqpoint{2.105886in}{1.828014in}}%
\pgfpathcurveto{\pgfqpoint{2.100062in}{1.822190in}}{\pgfqpoint{2.096790in}{1.814290in}}{\pgfqpoint{2.096790in}{1.806053in}}%
\pgfpathcurveto{\pgfqpoint{2.096790in}{1.797817in}}{\pgfqpoint{2.100062in}{1.789917in}}{\pgfqpoint{2.105886in}{1.784093in}}%
\pgfpathcurveto{\pgfqpoint{2.111710in}{1.778269in}}{\pgfqpoint{2.119610in}{1.774997in}}{\pgfqpoint{2.127846in}{1.774997in}}%
\pgfpathclose%
\pgfusepath{stroke,fill}%
\end{pgfscope}%
\begin{pgfscope}%
\pgfpathrectangle{\pgfqpoint{0.100000in}{0.220728in}}{\pgfqpoint{3.696000in}{3.696000in}}%
\pgfusepath{clip}%
\pgfsetbuttcap%
\pgfsetroundjoin%
\definecolor{currentfill}{rgb}{0.121569,0.466667,0.705882}%
\pgfsetfillcolor{currentfill}%
\pgfsetfillopacity{0.631256}%
\pgfsetlinewidth{1.003750pt}%
\definecolor{currentstroke}{rgb}{0.121569,0.466667,0.705882}%
\pgfsetstrokecolor{currentstroke}%
\pgfsetstrokeopacity{0.631256}%
\pgfsetdash{}{0pt}%
\pgfpathmoveto{\pgfqpoint{0.906328in}{1.467016in}}%
\pgfpathcurveto{\pgfqpoint{0.914565in}{1.467016in}}{\pgfqpoint{0.922465in}{1.470289in}}{\pgfqpoint{0.928289in}{1.476112in}}%
\pgfpathcurveto{\pgfqpoint{0.934113in}{1.481936in}}{\pgfqpoint{0.937385in}{1.489836in}}{\pgfqpoint{0.937385in}{1.498073in}}%
\pgfpathcurveto{\pgfqpoint{0.937385in}{1.506309in}}{\pgfqpoint{0.934113in}{1.514209in}}{\pgfqpoint{0.928289in}{1.520033in}}%
\pgfpathcurveto{\pgfqpoint{0.922465in}{1.525857in}}{\pgfqpoint{0.914565in}{1.529129in}}{\pgfqpoint{0.906328in}{1.529129in}}%
\pgfpathcurveto{\pgfqpoint{0.898092in}{1.529129in}}{\pgfqpoint{0.890192in}{1.525857in}}{\pgfqpoint{0.884368in}{1.520033in}}%
\pgfpathcurveto{\pgfqpoint{0.878544in}{1.514209in}}{\pgfqpoint{0.875272in}{1.506309in}}{\pgfqpoint{0.875272in}{1.498073in}}%
\pgfpathcurveto{\pgfqpoint{0.875272in}{1.489836in}}{\pgfqpoint{0.878544in}{1.481936in}}{\pgfqpoint{0.884368in}{1.476112in}}%
\pgfpathcurveto{\pgfqpoint{0.890192in}{1.470289in}}{\pgfqpoint{0.898092in}{1.467016in}}{\pgfqpoint{0.906328in}{1.467016in}}%
\pgfpathclose%
\pgfusepath{stroke,fill}%
\end{pgfscope}%
\begin{pgfscope}%
\pgfpathrectangle{\pgfqpoint{0.100000in}{0.220728in}}{\pgfqpoint{3.696000in}{3.696000in}}%
\pgfusepath{clip}%
\pgfsetbuttcap%
\pgfsetroundjoin%
\definecolor{currentfill}{rgb}{0.121569,0.466667,0.705882}%
\pgfsetfillcolor{currentfill}%
\pgfsetfillopacity{0.631975}%
\pgfsetlinewidth{1.003750pt}%
\definecolor{currentstroke}{rgb}{0.121569,0.466667,0.705882}%
\pgfsetstrokecolor{currentstroke}%
\pgfsetstrokeopacity{0.631975}%
\pgfsetdash{}{0pt}%
\pgfpathmoveto{\pgfqpoint{2.128955in}{1.771795in}}%
\pgfpathcurveto{\pgfqpoint{2.137192in}{1.771795in}}{\pgfqpoint{2.145092in}{1.775067in}}{\pgfqpoint{2.150916in}{1.780891in}}%
\pgfpathcurveto{\pgfqpoint{2.156740in}{1.786715in}}{\pgfqpoint{2.160012in}{1.794615in}}{\pgfqpoint{2.160012in}{1.802851in}}%
\pgfpathcurveto{\pgfqpoint{2.160012in}{1.811087in}}{\pgfqpoint{2.156740in}{1.818987in}}{\pgfqpoint{2.150916in}{1.824811in}}%
\pgfpathcurveto{\pgfqpoint{2.145092in}{1.830635in}}{\pgfqpoint{2.137192in}{1.833908in}}{\pgfqpoint{2.128955in}{1.833908in}}%
\pgfpathcurveto{\pgfqpoint{2.120719in}{1.833908in}}{\pgfqpoint{2.112819in}{1.830635in}}{\pgfqpoint{2.106995in}{1.824811in}}%
\pgfpathcurveto{\pgfqpoint{2.101171in}{1.818987in}}{\pgfqpoint{2.097899in}{1.811087in}}{\pgfqpoint{2.097899in}{1.802851in}}%
\pgfpathcurveto{\pgfqpoint{2.097899in}{1.794615in}}{\pgfqpoint{2.101171in}{1.786715in}}{\pgfqpoint{2.106995in}{1.780891in}}%
\pgfpathcurveto{\pgfqpoint{2.112819in}{1.775067in}}{\pgfqpoint{2.120719in}{1.771795in}}{\pgfqpoint{2.128955in}{1.771795in}}%
\pgfpathclose%
\pgfusepath{stroke,fill}%
\end{pgfscope}%
\begin{pgfscope}%
\pgfpathrectangle{\pgfqpoint{0.100000in}{0.220728in}}{\pgfqpoint{3.696000in}{3.696000in}}%
\pgfusepath{clip}%
\pgfsetbuttcap%
\pgfsetroundjoin%
\definecolor{currentfill}{rgb}{0.121569,0.466667,0.705882}%
\pgfsetfillcolor{currentfill}%
\pgfsetfillopacity{0.633182}%
\pgfsetlinewidth{1.003750pt}%
\definecolor{currentstroke}{rgb}{0.121569,0.466667,0.705882}%
\pgfsetstrokecolor{currentstroke}%
\pgfsetstrokeopacity{0.633182}%
\pgfsetdash{}{0pt}%
\pgfpathmoveto{\pgfqpoint{2.129900in}{1.769924in}}%
\pgfpathcurveto{\pgfqpoint{2.138136in}{1.769924in}}{\pgfqpoint{2.146036in}{1.773197in}}{\pgfqpoint{2.151860in}{1.779021in}}%
\pgfpathcurveto{\pgfqpoint{2.157684in}{1.784845in}}{\pgfqpoint{2.160956in}{1.792745in}}{\pgfqpoint{2.160956in}{1.800981in}}%
\pgfpathcurveto{\pgfqpoint{2.160956in}{1.809217in}}{\pgfqpoint{2.157684in}{1.817117in}}{\pgfqpoint{2.151860in}{1.822941in}}%
\pgfpathcurveto{\pgfqpoint{2.146036in}{1.828765in}}{\pgfqpoint{2.138136in}{1.832037in}}{\pgfqpoint{2.129900in}{1.832037in}}%
\pgfpathcurveto{\pgfqpoint{2.121663in}{1.832037in}}{\pgfqpoint{2.113763in}{1.828765in}}{\pgfqpoint{2.107939in}{1.822941in}}%
\pgfpathcurveto{\pgfqpoint{2.102115in}{1.817117in}}{\pgfqpoint{2.098843in}{1.809217in}}{\pgfqpoint{2.098843in}{1.800981in}}%
\pgfpathcurveto{\pgfqpoint{2.098843in}{1.792745in}}{\pgfqpoint{2.102115in}{1.784845in}}{\pgfqpoint{2.107939in}{1.779021in}}%
\pgfpathcurveto{\pgfqpoint{2.113763in}{1.773197in}}{\pgfqpoint{2.121663in}{1.769924in}}{\pgfqpoint{2.129900in}{1.769924in}}%
\pgfpathclose%
\pgfusepath{stroke,fill}%
\end{pgfscope}%
\begin{pgfscope}%
\pgfpathrectangle{\pgfqpoint{0.100000in}{0.220728in}}{\pgfqpoint{3.696000in}{3.696000in}}%
\pgfusepath{clip}%
\pgfsetbuttcap%
\pgfsetroundjoin%
\definecolor{currentfill}{rgb}{0.121569,0.466667,0.705882}%
\pgfsetfillcolor{currentfill}%
\pgfsetfillopacity{0.633815}%
\pgfsetlinewidth{1.003750pt}%
\definecolor{currentstroke}{rgb}{0.121569,0.466667,0.705882}%
\pgfsetstrokecolor{currentstroke}%
\pgfsetstrokeopacity{0.633815}%
\pgfsetdash{}{0pt}%
\pgfpathmoveto{\pgfqpoint{0.903552in}{1.467866in}}%
\pgfpathcurveto{\pgfqpoint{0.911788in}{1.467866in}}{\pgfqpoint{0.919688in}{1.471139in}}{\pgfqpoint{0.925512in}{1.476963in}}%
\pgfpathcurveto{\pgfqpoint{0.931336in}{1.482787in}}{\pgfqpoint{0.934608in}{1.490687in}}{\pgfqpoint{0.934608in}{1.498923in}}%
\pgfpathcurveto{\pgfqpoint{0.934608in}{1.507159in}}{\pgfqpoint{0.931336in}{1.515059in}}{\pgfqpoint{0.925512in}{1.520883in}}%
\pgfpathcurveto{\pgfqpoint{0.919688in}{1.526707in}}{\pgfqpoint{0.911788in}{1.529979in}}{\pgfqpoint{0.903552in}{1.529979in}}%
\pgfpathcurveto{\pgfqpoint{0.895315in}{1.529979in}}{\pgfqpoint{0.887415in}{1.526707in}}{\pgfqpoint{0.881591in}{1.520883in}}%
\pgfpathcurveto{\pgfqpoint{0.875767in}{1.515059in}}{\pgfqpoint{0.872495in}{1.507159in}}{\pgfqpoint{0.872495in}{1.498923in}}%
\pgfpathcurveto{\pgfqpoint{0.872495in}{1.490687in}}{\pgfqpoint{0.875767in}{1.482787in}}{\pgfqpoint{0.881591in}{1.476963in}}%
\pgfpathcurveto{\pgfqpoint{0.887415in}{1.471139in}}{\pgfqpoint{0.895315in}{1.467866in}}{\pgfqpoint{0.903552in}{1.467866in}}%
\pgfpathclose%
\pgfusepath{stroke,fill}%
\end{pgfscope}%
\begin{pgfscope}%
\pgfpathrectangle{\pgfqpoint{0.100000in}{0.220728in}}{\pgfqpoint{3.696000in}{3.696000in}}%
\pgfusepath{clip}%
\pgfsetbuttcap%
\pgfsetroundjoin%
\definecolor{currentfill}{rgb}{0.121569,0.466667,0.705882}%
\pgfsetfillcolor{currentfill}%
\pgfsetfillopacity{0.634476}%
\pgfsetlinewidth{1.003750pt}%
\definecolor{currentstroke}{rgb}{0.121569,0.466667,0.705882}%
\pgfsetstrokecolor{currentstroke}%
\pgfsetstrokeopacity{0.634476}%
\pgfsetdash{}{0pt}%
\pgfpathmoveto{\pgfqpoint{0.900798in}{1.465380in}}%
\pgfpathcurveto{\pgfqpoint{0.909035in}{1.465380in}}{\pgfqpoint{0.916935in}{1.468653in}}{\pgfqpoint{0.922759in}{1.474477in}}%
\pgfpathcurveto{\pgfqpoint{0.928583in}{1.480300in}}{\pgfqpoint{0.931855in}{1.488200in}}{\pgfqpoint{0.931855in}{1.496437in}}%
\pgfpathcurveto{\pgfqpoint{0.931855in}{1.504673in}}{\pgfqpoint{0.928583in}{1.512573in}}{\pgfqpoint{0.922759in}{1.518397in}}%
\pgfpathcurveto{\pgfqpoint{0.916935in}{1.524221in}}{\pgfqpoint{0.909035in}{1.527493in}}{\pgfqpoint{0.900798in}{1.527493in}}%
\pgfpathcurveto{\pgfqpoint{0.892562in}{1.527493in}}{\pgfqpoint{0.884662in}{1.524221in}}{\pgfqpoint{0.878838in}{1.518397in}}%
\pgfpathcurveto{\pgfqpoint{0.873014in}{1.512573in}}{\pgfqpoint{0.869742in}{1.504673in}}{\pgfqpoint{0.869742in}{1.496437in}}%
\pgfpathcurveto{\pgfqpoint{0.869742in}{1.488200in}}{\pgfqpoint{0.873014in}{1.480300in}}{\pgfqpoint{0.878838in}{1.474477in}}%
\pgfpathcurveto{\pgfqpoint{0.884662in}{1.468653in}}{\pgfqpoint{0.892562in}{1.465380in}}{\pgfqpoint{0.900798in}{1.465380in}}%
\pgfpathclose%
\pgfusepath{stroke,fill}%
\end{pgfscope}%
\begin{pgfscope}%
\pgfpathrectangle{\pgfqpoint{0.100000in}{0.220728in}}{\pgfqpoint{3.696000in}{3.696000in}}%
\pgfusepath{clip}%
\pgfsetbuttcap%
\pgfsetroundjoin%
\definecolor{currentfill}{rgb}{0.121569,0.466667,0.705882}%
\pgfsetfillcolor{currentfill}%
\pgfsetfillopacity{0.634718}%
\pgfsetlinewidth{1.003750pt}%
\definecolor{currentstroke}{rgb}{0.121569,0.466667,0.705882}%
\pgfsetstrokecolor{currentstroke}%
\pgfsetstrokeopacity{0.634718}%
\pgfsetdash{}{0pt}%
\pgfpathmoveto{\pgfqpoint{2.130922in}{1.768579in}}%
\pgfpathcurveto{\pgfqpoint{2.139158in}{1.768579in}}{\pgfqpoint{2.147058in}{1.771851in}}{\pgfqpoint{2.152882in}{1.777675in}}%
\pgfpathcurveto{\pgfqpoint{2.158706in}{1.783499in}}{\pgfqpoint{2.161978in}{1.791399in}}{\pgfqpoint{2.161978in}{1.799635in}}%
\pgfpathcurveto{\pgfqpoint{2.161978in}{1.807871in}}{\pgfqpoint{2.158706in}{1.815771in}}{\pgfqpoint{2.152882in}{1.821595in}}%
\pgfpathcurveto{\pgfqpoint{2.147058in}{1.827419in}}{\pgfqpoint{2.139158in}{1.830692in}}{\pgfqpoint{2.130922in}{1.830692in}}%
\pgfpathcurveto{\pgfqpoint{2.122686in}{1.830692in}}{\pgfqpoint{2.114785in}{1.827419in}}{\pgfqpoint{2.108962in}{1.821595in}}%
\pgfpathcurveto{\pgfqpoint{2.103138in}{1.815771in}}{\pgfqpoint{2.099865in}{1.807871in}}{\pgfqpoint{2.099865in}{1.799635in}}%
\pgfpathcurveto{\pgfqpoint{2.099865in}{1.791399in}}{\pgfqpoint{2.103138in}{1.783499in}}{\pgfqpoint{2.108962in}{1.777675in}}%
\pgfpathcurveto{\pgfqpoint{2.114785in}{1.771851in}}{\pgfqpoint{2.122686in}{1.768579in}}{\pgfqpoint{2.130922in}{1.768579in}}%
\pgfpathclose%
\pgfusepath{stroke,fill}%
\end{pgfscope}%
\begin{pgfscope}%
\pgfpathrectangle{\pgfqpoint{0.100000in}{0.220728in}}{\pgfqpoint{3.696000in}{3.696000in}}%
\pgfusepath{clip}%
\pgfsetbuttcap%
\pgfsetroundjoin%
\definecolor{currentfill}{rgb}{0.121569,0.466667,0.705882}%
\pgfsetfillcolor{currentfill}%
\pgfsetfillopacity{0.636151}%
\pgfsetlinewidth{1.003750pt}%
\definecolor{currentstroke}{rgb}{0.121569,0.466667,0.705882}%
\pgfsetstrokecolor{currentstroke}%
\pgfsetstrokeopacity{0.636151}%
\pgfsetdash{}{0pt}%
\pgfpathmoveto{\pgfqpoint{0.896622in}{1.462640in}}%
\pgfpathcurveto{\pgfqpoint{0.904858in}{1.462640in}}{\pgfqpoint{0.912758in}{1.465913in}}{\pgfqpoint{0.918582in}{1.471736in}}%
\pgfpathcurveto{\pgfqpoint{0.924406in}{1.477560in}}{\pgfqpoint{0.927678in}{1.485460in}}{\pgfqpoint{0.927678in}{1.493697in}}%
\pgfpathcurveto{\pgfqpoint{0.927678in}{1.501933in}}{\pgfqpoint{0.924406in}{1.509833in}}{\pgfqpoint{0.918582in}{1.515657in}}%
\pgfpathcurveto{\pgfqpoint{0.912758in}{1.521481in}}{\pgfqpoint{0.904858in}{1.524753in}}{\pgfqpoint{0.896622in}{1.524753in}}%
\pgfpathcurveto{\pgfqpoint{0.888385in}{1.524753in}}{\pgfqpoint{0.880485in}{1.521481in}}{\pgfqpoint{0.874661in}{1.515657in}}%
\pgfpathcurveto{\pgfqpoint{0.868838in}{1.509833in}}{\pgfqpoint{0.865565in}{1.501933in}}{\pgfqpoint{0.865565in}{1.493697in}}%
\pgfpathcurveto{\pgfqpoint{0.865565in}{1.485460in}}{\pgfqpoint{0.868838in}{1.477560in}}{\pgfqpoint{0.874661in}{1.471736in}}%
\pgfpathcurveto{\pgfqpoint{0.880485in}{1.465913in}}{\pgfqpoint{0.888385in}{1.462640in}}{\pgfqpoint{0.896622in}{1.462640in}}%
\pgfpathclose%
\pgfusepath{stroke,fill}%
\end{pgfscope}%
\begin{pgfscope}%
\pgfpathrectangle{\pgfqpoint{0.100000in}{0.220728in}}{\pgfqpoint{3.696000in}{3.696000in}}%
\pgfusepath{clip}%
\pgfsetbuttcap%
\pgfsetroundjoin%
\definecolor{currentfill}{rgb}{0.121569,0.466667,0.705882}%
\pgfsetfillcolor{currentfill}%
\pgfsetfillopacity{0.636672}%
\pgfsetlinewidth{1.003750pt}%
\definecolor{currentstroke}{rgb}{0.121569,0.466667,0.705882}%
\pgfsetstrokecolor{currentstroke}%
\pgfsetstrokeopacity{0.636672}%
\pgfsetdash{}{0pt}%
\pgfpathmoveto{\pgfqpoint{2.132402in}{1.767752in}}%
\pgfpathcurveto{\pgfqpoint{2.140638in}{1.767752in}}{\pgfqpoint{2.148538in}{1.771024in}}{\pgfqpoint{2.154362in}{1.776848in}}%
\pgfpathcurveto{\pgfqpoint{2.160186in}{1.782672in}}{\pgfqpoint{2.163458in}{1.790572in}}{\pgfqpoint{2.163458in}{1.798808in}}%
\pgfpathcurveto{\pgfqpoint{2.163458in}{1.807045in}}{\pgfqpoint{2.160186in}{1.814945in}}{\pgfqpoint{2.154362in}{1.820769in}}%
\pgfpathcurveto{\pgfqpoint{2.148538in}{1.826593in}}{\pgfqpoint{2.140638in}{1.829865in}}{\pgfqpoint{2.132402in}{1.829865in}}%
\pgfpathcurveto{\pgfqpoint{2.124165in}{1.829865in}}{\pgfqpoint{2.116265in}{1.826593in}}{\pgfqpoint{2.110441in}{1.820769in}}%
\pgfpathcurveto{\pgfqpoint{2.104617in}{1.814945in}}{\pgfqpoint{2.101345in}{1.807045in}}{\pgfqpoint{2.101345in}{1.798808in}}%
\pgfpathcurveto{\pgfqpoint{2.101345in}{1.790572in}}{\pgfqpoint{2.104617in}{1.782672in}}{\pgfqpoint{2.110441in}{1.776848in}}%
\pgfpathcurveto{\pgfqpoint{2.116265in}{1.771024in}}{\pgfqpoint{2.124165in}{1.767752in}}{\pgfqpoint{2.132402in}{1.767752in}}%
\pgfpathclose%
\pgfusepath{stroke,fill}%
\end{pgfscope}%
\begin{pgfscope}%
\pgfpathrectangle{\pgfqpoint{0.100000in}{0.220728in}}{\pgfqpoint{3.696000in}{3.696000in}}%
\pgfusepath{clip}%
\pgfsetbuttcap%
\pgfsetroundjoin%
\definecolor{currentfill}{rgb}{0.121569,0.466667,0.705882}%
\pgfsetfillcolor{currentfill}%
\pgfsetfillopacity{0.637620}%
\pgfsetlinewidth{1.003750pt}%
\definecolor{currentstroke}{rgb}{0.121569,0.466667,0.705882}%
\pgfsetstrokecolor{currentstroke}%
\pgfsetstrokeopacity{0.637620}%
\pgfsetdash{}{0pt}%
\pgfpathmoveto{\pgfqpoint{2.133065in}{1.766381in}}%
\pgfpathcurveto{\pgfqpoint{2.141301in}{1.766381in}}{\pgfqpoint{2.149201in}{1.769654in}}{\pgfqpoint{2.155025in}{1.775478in}}%
\pgfpathcurveto{\pgfqpoint{2.160849in}{1.781302in}}{\pgfqpoint{2.164121in}{1.789202in}}{\pgfqpoint{2.164121in}{1.797438in}}%
\pgfpathcurveto{\pgfqpoint{2.164121in}{1.805674in}}{\pgfqpoint{2.160849in}{1.813574in}}{\pgfqpoint{2.155025in}{1.819398in}}%
\pgfpathcurveto{\pgfqpoint{2.149201in}{1.825222in}}{\pgfqpoint{2.141301in}{1.828494in}}{\pgfqpoint{2.133065in}{1.828494in}}%
\pgfpathcurveto{\pgfqpoint{2.124828in}{1.828494in}}{\pgfqpoint{2.116928in}{1.825222in}}{\pgfqpoint{2.111104in}{1.819398in}}%
\pgfpathcurveto{\pgfqpoint{2.105280in}{1.813574in}}{\pgfqpoint{2.102008in}{1.805674in}}{\pgfqpoint{2.102008in}{1.797438in}}%
\pgfpathcurveto{\pgfqpoint{2.102008in}{1.789202in}}{\pgfqpoint{2.105280in}{1.781302in}}{\pgfqpoint{2.111104in}{1.775478in}}%
\pgfpathcurveto{\pgfqpoint{2.116928in}{1.769654in}}{\pgfqpoint{2.124828in}{1.766381in}}{\pgfqpoint{2.133065in}{1.766381in}}%
\pgfpathclose%
\pgfusepath{stroke,fill}%
\end{pgfscope}%
\begin{pgfscope}%
\pgfpathrectangle{\pgfqpoint{0.100000in}{0.220728in}}{\pgfqpoint{3.696000in}{3.696000in}}%
\pgfusepath{clip}%
\pgfsetbuttcap%
\pgfsetroundjoin%
\definecolor{currentfill}{rgb}{0.121569,0.466667,0.705882}%
\pgfsetfillcolor{currentfill}%
\pgfsetfillopacity{0.638988}%
\pgfsetlinewidth{1.003750pt}%
\definecolor{currentstroke}{rgb}{0.121569,0.466667,0.705882}%
\pgfsetstrokecolor{currentstroke}%
\pgfsetstrokeopacity{0.638988}%
\pgfsetdash{}{0pt}%
\pgfpathmoveto{\pgfqpoint{0.890019in}{1.455326in}}%
\pgfpathcurveto{\pgfqpoint{0.898255in}{1.455326in}}{\pgfqpoint{0.906155in}{1.458599in}}{\pgfqpoint{0.911979in}{1.464423in}}%
\pgfpathcurveto{\pgfqpoint{0.917803in}{1.470247in}}{\pgfqpoint{0.921075in}{1.478147in}}{\pgfqpoint{0.921075in}{1.486383in}}%
\pgfpathcurveto{\pgfqpoint{0.921075in}{1.494619in}}{\pgfqpoint{0.917803in}{1.502519in}}{\pgfqpoint{0.911979in}{1.508343in}}%
\pgfpathcurveto{\pgfqpoint{0.906155in}{1.514167in}}{\pgfqpoint{0.898255in}{1.517439in}}{\pgfqpoint{0.890019in}{1.517439in}}%
\pgfpathcurveto{\pgfqpoint{0.881783in}{1.517439in}}{\pgfqpoint{0.873883in}{1.514167in}}{\pgfqpoint{0.868059in}{1.508343in}}%
\pgfpathcurveto{\pgfqpoint{0.862235in}{1.502519in}}{\pgfqpoint{0.858962in}{1.494619in}}{\pgfqpoint{0.858962in}{1.486383in}}%
\pgfpathcurveto{\pgfqpoint{0.858962in}{1.478147in}}{\pgfqpoint{0.862235in}{1.470247in}}{\pgfqpoint{0.868059in}{1.464423in}}%
\pgfpathcurveto{\pgfqpoint{0.873883in}{1.458599in}}{\pgfqpoint{0.881783in}{1.455326in}}{\pgfqpoint{0.890019in}{1.455326in}}%
\pgfpathclose%
\pgfusepath{stroke,fill}%
\end{pgfscope}%
\begin{pgfscope}%
\pgfpathrectangle{\pgfqpoint{0.100000in}{0.220728in}}{\pgfqpoint{3.696000in}{3.696000in}}%
\pgfusepath{clip}%
\pgfsetbuttcap%
\pgfsetroundjoin%
\definecolor{currentfill}{rgb}{0.121569,0.466667,0.705882}%
\pgfsetfillcolor{currentfill}%
\pgfsetfillopacity{0.639074}%
\pgfsetlinewidth{1.003750pt}%
\definecolor{currentstroke}{rgb}{0.121569,0.466667,0.705882}%
\pgfsetstrokecolor{currentstroke}%
\pgfsetstrokeopacity{0.639074}%
\pgfsetdash{}{0pt}%
\pgfpathmoveto{\pgfqpoint{2.133907in}{1.764422in}}%
\pgfpathcurveto{\pgfqpoint{2.142143in}{1.764422in}}{\pgfqpoint{2.150043in}{1.767695in}}{\pgfqpoint{2.155867in}{1.773519in}}%
\pgfpathcurveto{\pgfqpoint{2.161691in}{1.779342in}}{\pgfqpoint{2.164964in}{1.787243in}}{\pgfqpoint{2.164964in}{1.795479in}}%
\pgfpathcurveto{\pgfqpoint{2.164964in}{1.803715in}}{\pgfqpoint{2.161691in}{1.811615in}}{\pgfqpoint{2.155867in}{1.817439in}}%
\pgfpathcurveto{\pgfqpoint{2.150043in}{1.823263in}}{\pgfqpoint{2.142143in}{1.826535in}}{\pgfqpoint{2.133907in}{1.826535in}}%
\pgfpathcurveto{\pgfqpoint{2.125671in}{1.826535in}}{\pgfqpoint{2.117771in}{1.823263in}}{\pgfqpoint{2.111947in}{1.817439in}}%
\pgfpathcurveto{\pgfqpoint{2.106123in}{1.811615in}}{\pgfqpoint{2.102851in}{1.803715in}}{\pgfqpoint{2.102851in}{1.795479in}}%
\pgfpathcurveto{\pgfqpoint{2.102851in}{1.787243in}}{\pgfqpoint{2.106123in}{1.779342in}}{\pgfqpoint{2.111947in}{1.773519in}}%
\pgfpathcurveto{\pgfqpoint{2.117771in}{1.767695in}}{\pgfqpoint{2.125671in}{1.764422in}}{\pgfqpoint{2.133907in}{1.764422in}}%
\pgfpathclose%
\pgfusepath{stroke,fill}%
\end{pgfscope}%
\begin{pgfscope}%
\pgfpathrectangle{\pgfqpoint{0.100000in}{0.220728in}}{\pgfqpoint{3.696000in}{3.696000in}}%
\pgfusepath{clip}%
\pgfsetbuttcap%
\pgfsetroundjoin%
\definecolor{currentfill}{rgb}{0.121569,0.466667,0.705882}%
\pgfsetfillcolor{currentfill}%
\pgfsetfillopacity{0.640975}%
\pgfsetlinewidth{1.003750pt}%
\definecolor{currentstroke}{rgb}{0.121569,0.466667,0.705882}%
\pgfsetstrokecolor{currentstroke}%
\pgfsetstrokeopacity{0.640975}%
\pgfsetdash{}{0pt}%
\pgfpathmoveto{\pgfqpoint{2.135227in}{1.763006in}}%
\pgfpathcurveto{\pgfqpoint{2.143463in}{1.763006in}}{\pgfqpoint{2.151363in}{1.766278in}}{\pgfqpoint{2.157187in}{1.772102in}}%
\pgfpathcurveto{\pgfqpoint{2.163011in}{1.777926in}}{\pgfqpoint{2.166283in}{1.785826in}}{\pgfqpoint{2.166283in}{1.794062in}}%
\pgfpathcurveto{\pgfqpoint{2.166283in}{1.802299in}}{\pgfqpoint{2.163011in}{1.810199in}}{\pgfqpoint{2.157187in}{1.816023in}}%
\pgfpathcurveto{\pgfqpoint{2.151363in}{1.821847in}}{\pgfqpoint{2.143463in}{1.825119in}}{\pgfqpoint{2.135227in}{1.825119in}}%
\pgfpathcurveto{\pgfqpoint{2.126990in}{1.825119in}}{\pgfqpoint{2.119090in}{1.821847in}}{\pgfqpoint{2.113266in}{1.816023in}}%
\pgfpathcurveto{\pgfqpoint{2.107442in}{1.810199in}}{\pgfqpoint{2.104170in}{1.802299in}}{\pgfqpoint{2.104170in}{1.794062in}}%
\pgfpathcurveto{\pgfqpoint{2.104170in}{1.785826in}}{\pgfqpoint{2.107442in}{1.777926in}}{\pgfqpoint{2.113266in}{1.772102in}}%
\pgfpathcurveto{\pgfqpoint{2.119090in}{1.766278in}}{\pgfqpoint{2.126990in}{1.763006in}}{\pgfqpoint{2.135227in}{1.763006in}}%
\pgfpathclose%
\pgfusepath{stroke,fill}%
\end{pgfscope}%
\begin{pgfscope}%
\pgfpathrectangle{\pgfqpoint{0.100000in}{0.220728in}}{\pgfqpoint{3.696000in}{3.696000in}}%
\pgfusepath{clip}%
\pgfsetbuttcap%
\pgfsetroundjoin%
\definecolor{currentfill}{rgb}{0.121569,0.466667,0.705882}%
\pgfsetfillcolor{currentfill}%
\pgfsetfillopacity{0.641499}%
\pgfsetlinewidth{1.003750pt}%
\definecolor{currentstroke}{rgb}{0.121569,0.466667,0.705882}%
\pgfsetstrokecolor{currentstroke}%
\pgfsetstrokeopacity{0.641499}%
\pgfsetdash{}{0pt}%
\pgfpathmoveto{\pgfqpoint{0.881627in}{1.454813in}}%
\pgfpathcurveto{\pgfqpoint{0.889863in}{1.454813in}}{\pgfqpoint{0.897763in}{1.458085in}}{\pgfqpoint{0.903587in}{1.463909in}}%
\pgfpathcurveto{\pgfqpoint{0.909411in}{1.469733in}}{\pgfqpoint{0.912683in}{1.477633in}}{\pgfqpoint{0.912683in}{1.485870in}}%
\pgfpathcurveto{\pgfqpoint{0.912683in}{1.494106in}}{\pgfqpoint{0.909411in}{1.502006in}}{\pgfqpoint{0.903587in}{1.507830in}}%
\pgfpathcurveto{\pgfqpoint{0.897763in}{1.513654in}}{\pgfqpoint{0.889863in}{1.516926in}}{\pgfqpoint{0.881627in}{1.516926in}}%
\pgfpathcurveto{\pgfqpoint{0.873390in}{1.516926in}}{\pgfqpoint{0.865490in}{1.513654in}}{\pgfqpoint{0.859666in}{1.507830in}}%
\pgfpathcurveto{\pgfqpoint{0.853842in}{1.502006in}}{\pgfqpoint{0.850570in}{1.494106in}}{\pgfqpoint{0.850570in}{1.485870in}}%
\pgfpathcurveto{\pgfqpoint{0.850570in}{1.477633in}}{\pgfqpoint{0.853842in}{1.469733in}}{\pgfqpoint{0.859666in}{1.463909in}}%
\pgfpathcurveto{\pgfqpoint{0.865490in}{1.458085in}}{\pgfqpoint{0.873390in}{1.454813in}}{\pgfqpoint{0.881627in}{1.454813in}}%
\pgfpathclose%
\pgfusepath{stroke,fill}%
\end{pgfscope}%
\begin{pgfscope}%
\pgfpathrectangle{\pgfqpoint{0.100000in}{0.220728in}}{\pgfqpoint{3.696000in}{3.696000in}}%
\pgfusepath{clip}%
\pgfsetbuttcap%
\pgfsetroundjoin%
\definecolor{currentfill}{rgb}{0.121569,0.466667,0.705882}%
\pgfsetfillcolor{currentfill}%
\pgfsetfillopacity{0.643765}%
\pgfsetlinewidth{1.003750pt}%
\definecolor{currentstroke}{rgb}{0.121569,0.466667,0.705882}%
\pgfsetstrokecolor{currentstroke}%
\pgfsetstrokeopacity{0.643765}%
\pgfsetdash{}{0pt}%
\pgfpathmoveto{\pgfqpoint{2.137615in}{1.763191in}}%
\pgfpathcurveto{\pgfqpoint{2.145851in}{1.763191in}}{\pgfqpoint{2.153751in}{1.766463in}}{\pgfqpoint{2.159575in}{1.772287in}}%
\pgfpathcurveto{\pgfqpoint{2.165399in}{1.778111in}}{\pgfqpoint{2.168671in}{1.786011in}}{\pgfqpoint{2.168671in}{1.794248in}}%
\pgfpathcurveto{\pgfqpoint{2.168671in}{1.802484in}}{\pgfqpoint{2.165399in}{1.810384in}}{\pgfqpoint{2.159575in}{1.816208in}}%
\pgfpathcurveto{\pgfqpoint{2.153751in}{1.822032in}}{\pgfqpoint{2.145851in}{1.825304in}}{\pgfqpoint{2.137615in}{1.825304in}}%
\pgfpathcurveto{\pgfqpoint{2.129378in}{1.825304in}}{\pgfqpoint{2.121478in}{1.822032in}}{\pgfqpoint{2.115654in}{1.816208in}}%
\pgfpathcurveto{\pgfqpoint{2.109831in}{1.810384in}}{\pgfqpoint{2.106558in}{1.802484in}}{\pgfqpoint{2.106558in}{1.794248in}}%
\pgfpathcurveto{\pgfqpoint{2.106558in}{1.786011in}}{\pgfqpoint{2.109831in}{1.778111in}}{\pgfqpoint{2.115654in}{1.772287in}}%
\pgfpathcurveto{\pgfqpoint{2.121478in}{1.766463in}}{\pgfqpoint{2.129378in}{1.763191in}}{\pgfqpoint{2.137615in}{1.763191in}}%
\pgfpathclose%
\pgfusepath{stroke,fill}%
\end{pgfscope}%
\begin{pgfscope}%
\pgfpathrectangle{\pgfqpoint{0.100000in}{0.220728in}}{\pgfqpoint{3.696000in}{3.696000in}}%
\pgfusepath{clip}%
\pgfsetbuttcap%
\pgfsetroundjoin%
\definecolor{currentfill}{rgb}{0.121569,0.466667,0.705882}%
\pgfsetfillcolor{currentfill}%
\pgfsetfillopacity{0.643826}%
\pgfsetlinewidth{1.003750pt}%
\definecolor{currentstroke}{rgb}{0.121569,0.466667,0.705882}%
\pgfsetstrokecolor{currentstroke}%
\pgfsetstrokeopacity{0.643826}%
\pgfsetdash{}{0pt}%
\pgfpathmoveto{\pgfqpoint{0.878188in}{1.453838in}}%
\pgfpathcurveto{\pgfqpoint{0.886424in}{1.453838in}}{\pgfqpoint{0.894324in}{1.457110in}}{\pgfqpoint{0.900148in}{1.462934in}}%
\pgfpathcurveto{\pgfqpoint{0.905972in}{1.468758in}}{\pgfqpoint{0.909244in}{1.476658in}}{\pgfqpoint{0.909244in}{1.484894in}}%
\pgfpathcurveto{\pgfqpoint{0.909244in}{1.493130in}}{\pgfqpoint{0.905972in}{1.501030in}}{\pgfqpoint{0.900148in}{1.506854in}}%
\pgfpathcurveto{\pgfqpoint{0.894324in}{1.512678in}}{\pgfqpoint{0.886424in}{1.515951in}}{\pgfqpoint{0.878188in}{1.515951in}}%
\pgfpathcurveto{\pgfqpoint{0.869951in}{1.515951in}}{\pgfqpoint{0.862051in}{1.512678in}}{\pgfqpoint{0.856227in}{1.506854in}}%
\pgfpathcurveto{\pgfqpoint{0.850404in}{1.501030in}}{\pgfqpoint{0.847131in}{1.493130in}}{\pgfqpoint{0.847131in}{1.484894in}}%
\pgfpathcurveto{\pgfqpoint{0.847131in}{1.476658in}}{\pgfqpoint{0.850404in}{1.468758in}}{\pgfqpoint{0.856227in}{1.462934in}}%
\pgfpathcurveto{\pgfqpoint{0.862051in}{1.457110in}}{\pgfqpoint{0.869951in}{1.453838in}}{\pgfqpoint{0.878188in}{1.453838in}}%
\pgfpathclose%
\pgfusepath{stroke,fill}%
\end{pgfscope}%
\begin{pgfscope}%
\pgfpathrectangle{\pgfqpoint{0.100000in}{0.220728in}}{\pgfqpoint{3.696000in}{3.696000in}}%
\pgfusepath{clip}%
\pgfsetbuttcap%
\pgfsetroundjoin%
\definecolor{currentfill}{rgb}{0.121569,0.466667,0.705882}%
\pgfsetfillcolor{currentfill}%
\pgfsetfillopacity{0.644912}%
\pgfsetlinewidth{1.003750pt}%
\definecolor{currentstroke}{rgb}{0.121569,0.466667,0.705882}%
\pgfsetstrokecolor{currentstroke}%
\pgfsetstrokeopacity{0.644912}%
\pgfsetdash{}{0pt}%
\pgfpathmoveto{\pgfqpoint{0.874753in}{1.453772in}}%
\pgfpathcurveto{\pgfqpoint{0.882989in}{1.453772in}}{\pgfqpoint{0.890889in}{1.457044in}}{\pgfqpoint{0.896713in}{1.462868in}}%
\pgfpathcurveto{\pgfqpoint{0.902537in}{1.468692in}}{\pgfqpoint{0.905810in}{1.476592in}}{\pgfqpoint{0.905810in}{1.484828in}}%
\pgfpathcurveto{\pgfqpoint{0.905810in}{1.493064in}}{\pgfqpoint{0.902537in}{1.500964in}}{\pgfqpoint{0.896713in}{1.506788in}}%
\pgfpathcurveto{\pgfqpoint{0.890889in}{1.512612in}}{\pgfqpoint{0.882989in}{1.515885in}}{\pgfqpoint{0.874753in}{1.515885in}}%
\pgfpathcurveto{\pgfqpoint{0.866517in}{1.515885in}}{\pgfqpoint{0.858617in}{1.512612in}}{\pgfqpoint{0.852793in}{1.506788in}}%
\pgfpathcurveto{\pgfqpoint{0.846969in}{1.500964in}}{\pgfqpoint{0.843697in}{1.493064in}}{\pgfqpoint{0.843697in}{1.484828in}}%
\pgfpathcurveto{\pgfqpoint{0.843697in}{1.476592in}}{\pgfqpoint{0.846969in}{1.468692in}}{\pgfqpoint{0.852793in}{1.462868in}}%
\pgfpathcurveto{\pgfqpoint{0.858617in}{1.457044in}}{\pgfqpoint{0.866517in}{1.453772in}}{\pgfqpoint{0.874753in}{1.453772in}}%
\pgfpathclose%
\pgfusepath{stroke,fill}%
\end{pgfscope}%
\begin{pgfscope}%
\pgfpathrectangle{\pgfqpoint{0.100000in}{0.220728in}}{\pgfqpoint{3.696000in}{3.696000in}}%
\pgfusepath{clip}%
\pgfsetbuttcap%
\pgfsetroundjoin%
\definecolor{currentfill}{rgb}{0.121569,0.466667,0.705882}%
\pgfsetfillcolor{currentfill}%
\pgfsetfillopacity{0.645676}%
\pgfsetlinewidth{1.003750pt}%
\definecolor{currentstroke}{rgb}{0.121569,0.466667,0.705882}%
\pgfsetstrokecolor{currentstroke}%
\pgfsetstrokeopacity{0.645676}%
\pgfsetdash{}{0pt}%
\pgfpathmoveto{\pgfqpoint{0.873671in}{1.454310in}}%
\pgfpathcurveto{\pgfqpoint{0.881907in}{1.454310in}}{\pgfqpoint{0.889808in}{1.457582in}}{\pgfqpoint{0.895631in}{1.463406in}}%
\pgfpathcurveto{\pgfqpoint{0.901455in}{1.469230in}}{\pgfqpoint{0.904728in}{1.477130in}}{\pgfqpoint{0.904728in}{1.485366in}}%
\pgfpathcurveto{\pgfqpoint{0.904728in}{1.493603in}}{\pgfqpoint{0.901455in}{1.501503in}}{\pgfqpoint{0.895631in}{1.507327in}}%
\pgfpathcurveto{\pgfqpoint{0.889808in}{1.513151in}}{\pgfqpoint{0.881907in}{1.516423in}}{\pgfqpoint{0.873671in}{1.516423in}}%
\pgfpathcurveto{\pgfqpoint{0.865435in}{1.516423in}}{\pgfqpoint{0.857535in}{1.513151in}}{\pgfqpoint{0.851711in}{1.507327in}}%
\pgfpathcurveto{\pgfqpoint{0.845887in}{1.501503in}}{\pgfqpoint{0.842615in}{1.493603in}}{\pgfqpoint{0.842615in}{1.485366in}}%
\pgfpathcurveto{\pgfqpoint{0.842615in}{1.477130in}}{\pgfqpoint{0.845887in}{1.469230in}}{\pgfqpoint{0.851711in}{1.463406in}}%
\pgfpathcurveto{\pgfqpoint{0.857535in}{1.457582in}}{\pgfqpoint{0.865435in}{1.454310in}}{\pgfqpoint{0.873671in}{1.454310in}}%
\pgfpathclose%
\pgfusepath{stroke,fill}%
\end{pgfscope}%
\begin{pgfscope}%
\pgfpathrectangle{\pgfqpoint{0.100000in}{0.220728in}}{\pgfqpoint{3.696000in}{3.696000in}}%
\pgfusepath{clip}%
\pgfsetbuttcap%
\pgfsetroundjoin%
\definecolor{currentfill}{rgb}{0.121569,0.466667,0.705882}%
\pgfsetfillcolor{currentfill}%
\pgfsetfillopacity{0.645676}%
\pgfsetlinewidth{1.003750pt}%
\definecolor{currentstroke}{rgb}{0.121569,0.466667,0.705882}%
\pgfsetstrokecolor{currentstroke}%
\pgfsetstrokeopacity{0.645676}%
\pgfsetdash{}{0pt}%
\pgfpathmoveto{\pgfqpoint{0.873671in}{1.454310in}}%
\pgfpathcurveto{\pgfqpoint{0.881907in}{1.454310in}}{\pgfqpoint{0.889807in}{1.457582in}}{\pgfqpoint{0.895631in}{1.463406in}}%
\pgfpathcurveto{\pgfqpoint{0.901455in}{1.469230in}}{\pgfqpoint{0.904728in}{1.477130in}}{\pgfqpoint{0.904728in}{1.485366in}}%
\pgfpathcurveto{\pgfqpoint{0.904728in}{1.493603in}}{\pgfqpoint{0.901455in}{1.501503in}}{\pgfqpoint{0.895631in}{1.507327in}}%
\pgfpathcurveto{\pgfqpoint{0.889807in}{1.513151in}}{\pgfqpoint{0.881907in}{1.516423in}}{\pgfqpoint{0.873671in}{1.516423in}}%
\pgfpathcurveto{\pgfqpoint{0.865435in}{1.516423in}}{\pgfqpoint{0.857535in}{1.513151in}}{\pgfqpoint{0.851711in}{1.507327in}}%
\pgfpathcurveto{\pgfqpoint{0.845887in}{1.501503in}}{\pgfqpoint{0.842615in}{1.493603in}}{\pgfqpoint{0.842615in}{1.485366in}}%
\pgfpathcurveto{\pgfqpoint{0.842615in}{1.477130in}}{\pgfqpoint{0.845887in}{1.469230in}}{\pgfqpoint{0.851711in}{1.463406in}}%
\pgfpathcurveto{\pgfqpoint{0.857535in}{1.457582in}}{\pgfqpoint{0.865435in}{1.454310in}}{\pgfqpoint{0.873671in}{1.454310in}}%
\pgfpathclose%
\pgfusepath{stroke,fill}%
\end{pgfscope}%
\begin{pgfscope}%
\pgfpathrectangle{\pgfqpoint{0.100000in}{0.220728in}}{\pgfqpoint{3.696000in}{3.696000in}}%
\pgfusepath{clip}%
\pgfsetbuttcap%
\pgfsetroundjoin%
\definecolor{currentfill}{rgb}{0.121569,0.466667,0.705882}%
\pgfsetfillcolor{currentfill}%
\pgfsetfillopacity{0.645676}%
\pgfsetlinewidth{1.003750pt}%
\definecolor{currentstroke}{rgb}{0.121569,0.466667,0.705882}%
\pgfsetstrokecolor{currentstroke}%
\pgfsetstrokeopacity{0.645676}%
\pgfsetdash{}{0pt}%
\pgfpathmoveto{\pgfqpoint{0.873671in}{1.454310in}}%
\pgfpathcurveto{\pgfqpoint{0.881907in}{1.454310in}}{\pgfqpoint{0.889807in}{1.457582in}}{\pgfqpoint{0.895631in}{1.463406in}}%
\pgfpathcurveto{\pgfqpoint{0.901455in}{1.469230in}}{\pgfqpoint{0.904728in}{1.477130in}}{\pgfqpoint{0.904728in}{1.485366in}}%
\pgfpathcurveto{\pgfqpoint{0.904728in}{1.493603in}}{\pgfqpoint{0.901455in}{1.501503in}}{\pgfqpoint{0.895631in}{1.507326in}}%
\pgfpathcurveto{\pgfqpoint{0.889807in}{1.513150in}}{\pgfqpoint{0.881907in}{1.516423in}}{\pgfqpoint{0.873671in}{1.516423in}}%
\pgfpathcurveto{\pgfqpoint{0.865435in}{1.516423in}}{\pgfqpoint{0.857535in}{1.513150in}}{\pgfqpoint{0.851711in}{1.507326in}}%
\pgfpathcurveto{\pgfqpoint{0.845887in}{1.501503in}}{\pgfqpoint{0.842615in}{1.493603in}}{\pgfqpoint{0.842615in}{1.485366in}}%
\pgfpathcurveto{\pgfqpoint{0.842615in}{1.477130in}}{\pgfqpoint{0.845887in}{1.469230in}}{\pgfqpoint{0.851711in}{1.463406in}}%
\pgfpathcurveto{\pgfqpoint{0.857535in}{1.457582in}}{\pgfqpoint{0.865435in}{1.454310in}}{\pgfqpoint{0.873671in}{1.454310in}}%
\pgfpathclose%
\pgfusepath{stroke,fill}%
\end{pgfscope}%
\begin{pgfscope}%
\pgfpathrectangle{\pgfqpoint{0.100000in}{0.220728in}}{\pgfqpoint{3.696000in}{3.696000in}}%
\pgfusepath{clip}%
\pgfsetbuttcap%
\pgfsetroundjoin%
\definecolor{currentfill}{rgb}{0.121569,0.466667,0.705882}%
\pgfsetfillcolor{currentfill}%
\pgfsetfillopacity{0.645676}%
\pgfsetlinewidth{1.003750pt}%
\definecolor{currentstroke}{rgb}{0.121569,0.466667,0.705882}%
\pgfsetstrokecolor{currentstroke}%
\pgfsetstrokeopacity{0.645676}%
\pgfsetdash{}{0pt}%
\pgfpathmoveto{\pgfqpoint{0.873671in}{1.454310in}}%
\pgfpathcurveto{\pgfqpoint{0.881907in}{1.454310in}}{\pgfqpoint{0.889807in}{1.457582in}}{\pgfqpoint{0.895631in}{1.463406in}}%
\pgfpathcurveto{\pgfqpoint{0.901455in}{1.469230in}}{\pgfqpoint{0.904727in}{1.477130in}}{\pgfqpoint{0.904727in}{1.485366in}}%
\pgfpathcurveto{\pgfqpoint{0.904727in}{1.493602in}}{\pgfqpoint{0.901455in}{1.501502in}}{\pgfqpoint{0.895631in}{1.507326in}}%
\pgfpathcurveto{\pgfqpoint{0.889807in}{1.513150in}}{\pgfqpoint{0.881907in}{1.516423in}}{\pgfqpoint{0.873671in}{1.516423in}}%
\pgfpathcurveto{\pgfqpoint{0.865435in}{1.516423in}}{\pgfqpoint{0.857535in}{1.513150in}}{\pgfqpoint{0.851711in}{1.507326in}}%
\pgfpathcurveto{\pgfqpoint{0.845887in}{1.501502in}}{\pgfqpoint{0.842614in}{1.493602in}}{\pgfqpoint{0.842614in}{1.485366in}}%
\pgfpathcurveto{\pgfqpoint{0.842614in}{1.477130in}}{\pgfqpoint{0.845887in}{1.469230in}}{\pgfqpoint{0.851711in}{1.463406in}}%
\pgfpathcurveto{\pgfqpoint{0.857535in}{1.457582in}}{\pgfqpoint{0.865435in}{1.454310in}}{\pgfqpoint{0.873671in}{1.454310in}}%
\pgfpathclose%
\pgfusepath{stroke,fill}%
\end{pgfscope}%
\begin{pgfscope}%
\pgfpathrectangle{\pgfqpoint{0.100000in}{0.220728in}}{\pgfqpoint{3.696000in}{3.696000in}}%
\pgfusepath{clip}%
\pgfsetbuttcap%
\pgfsetroundjoin%
\definecolor{currentfill}{rgb}{0.121569,0.466667,0.705882}%
\pgfsetfillcolor{currentfill}%
\pgfsetfillopacity{0.645677}%
\pgfsetlinewidth{1.003750pt}%
\definecolor{currentstroke}{rgb}{0.121569,0.466667,0.705882}%
\pgfsetstrokecolor{currentstroke}%
\pgfsetstrokeopacity{0.645677}%
\pgfsetdash{}{0pt}%
\pgfpathmoveto{\pgfqpoint{0.873671in}{1.454309in}}%
\pgfpathcurveto{\pgfqpoint{0.881907in}{1.454309in}}{\pgfqpoint{0.889807in}{1.457582in}}{\pgfqpoint{0.895631in}{1.463406in}}%
\pgfpathcurveto{\pgfqpoint{0.901455in}{1.469230in}}{\pgfqpoint{0.904727in}{1.477130in}}{\pgfqpoint{0.904727in}{1.485366in}}%
\pgfpathcurveto{\pgfqpoint{0.904727in}{1.493602in}}{\pgfqpoint{0.901455in}{1.501502in}}{\pgfqpoint{0.895631in}{1.507326in}}%
\pgfpathcurveto{\pgfqpoint{0.889807in}{1.513150in}}{\pgfqpoint{0.881907in}{1.516422in}}{\pgfqpoint{0.873671in}{1.516422in}}%
\pgfpathcurveto{\pgfqpoint{0.865434in}{1.516422in}}{\pgfqpoint{0.857534in}{1.513150in}}{\pgfqpoint{0.851710in}{1.507326in}}%
\pgfpathcurveto{\pgfqpoint{0.845886in}{1.501502in}}{\pgfqpoint{0.842614in}{1.493602in}}{\pgfqpoint{0.842614in}{1.485366in}}%
\pgfpathcurveto{\pgfqpoint{0.842614in}{1.477130in}}{\pgfqpoint{0.845886in}{1.469230in}}{\pgfqpoint{0.851710in}{1.463406in}}%
\pgfpathcurveto{\pgfqpoint{0.857534in}{1.457582in}}{\pgfqpoint{0.865434in}{1.454309in}}{\pgfqpoint{0.873671in}{1.454309in}}%
\pgfpathclose%
\pgfusepath{stroke,fill}%
\end{pgfscope}%
\begin{pgfscope}%
\pgfpathrectangle{\pgfqpoint{0.100000in}{0.220728in}}{\pgfqpoint{3.696000in}{3.696000in}}%
\pgfusepath{clip}%
\pgfsetbuttcap%
\pgfsetroundjoin%
\definecolor{currentfill}{rgb}{0.121569,0.466667,0.705882}%
\pgfsetfillcolor{currentfill}%
\pgfsetfillopacity{0.645677}%
\pgfsetlinewidth{1.003750pt}%
\definecolor{currentstroke}{rgb}{0.121569,0.466667,0.705882}%
\pgfsetstrokecolor{currentstroke}%
\pgfsetstrokeopacity{0.645677}%
\pgfsetdash{}{0pt}%
\pgfpathmoveto{\pgfqpoint{0.873670in}{1.454309in}}%
\pgfpathcurveto{\pgfqpoint{0.881906in}{1.454309in}}{\pgfqpoint{0.889806in}{1.457581in}}{\pgfqpoint{0.895630in}{1.463405in}}%
\pgfpathcurveto{\pgfqpoint{0.901454in}{1.469229in}}{\pgfqpoint{0.904726in}{1.477129in}}{\pgfqpoint{0.904726in}{1.485366in}}%
\pgfpathcurveto{\pgfqpoint{0.904726in}{1.493602in}}{\pgfqpoint{0.901454in}{1.501502in}}{\pgfqpoint{0.895630in}{1.507326in}}%
\pgfpathcurveto{\pgfqpoint{0.889806in}{1.513150in}}{\pgfqpoint{0.881906in}{1.516422in}}{\pgfqpoint{0.873670in}{1.516422in}}%
\pgfpathcurveto{\pgfqpoint{0.865434in}{1.516422in}}{\pgfqpoint{0.857534in}{1.513150in}}{\pgfqpoint{0.851710in}{1.507326in}}%
\pgfpathcurveto{\pgfqpoint{0.845886in}{1.501502in}}{\pgfqpoint{0.842613in}{1.493602in}}{\pgfqpoint{0.842613in}{1.485366in}}%
\pgfpathcurveto{\pgfqpoint{0.842613in}{1.477129in}}{\pgfqpoint{0.845886in}{1.469229in}}{\pgfqpoint{0.851710in}{1.463405in}}%
\pgfpathcurveto{\pgfqpoint{0.857534in}{1.457581in}}{\pgfqpoint{0.865434in}{1.454309in}}{\pgfqpoint{0.873670in}{1.454309in}}%
\pgfpathclose%
\pgfusepath{stroke,fill}%
\end{pgfscope}%
\begin{pgfscope}%
\pgfpathrectangle{\pgfqpoint{0.100000in}{0.220728in}}{\pgfqpoint{3.696000in}{3.696000in}}%
\pgfusepath{clip}%
\pgfsetbuttcap%
\pgfsetroundjoin%
\definecolor{currentfill}{rgb}{0.121569,0.466667,0.705882}%
\pgfsetfillcolor{currentfill}%
\pgfsetfillopacity{0.645677}%
\pgfsetlinewidth{1.003750pt}%
\definecolor{currentstroke}{rgb}{0.121569,0.466667,0.705882}%
\pgfsetstrokecolor{currentstroke}%
\pgfsetstrokeopacity{0.645677}%
\pgfsetdash{}{0pt}%
\pgfpathmoveto{\pgfqpoint{0.873669in}{1.454309in}}%
\pgfpathcurveto{\pgfqpoint{0.881905in}{1.454309in}}{\pgfqpoint{0.889805in}{1.457581in}}{\pgfqpoint{0.895629in}{1.463405in}}%
\pgfpathcurveto{\pgfqpoint{0.901453in}{1.469229in}}{\pgfqpoint{0.904725in}{1.477129in}}{\pgfqpoint{0.904725in}{1.485365in}}%
\pgfpathcurveto{\pgfqpoint{0.904725in}{1.493601in}}{\pgfqpoint{0.901453in}{1.501501in}}{\pgfqpoint{0.895629in}{1.507325in}}%
\pgfpathcurveto{\pgfqpoint{0.889805in}{1.513149in}}{\pgfqpoint{0.881905in}{1.516422in}}{\pgfqpoint{0.873669in}{1.516422in}}%
\pgfpathcurveto{\pgfqpoint{0.865433in}{1.516422in}}{\pgfqpoint{0.857533in}{1.513149in}}{\pgfqpoint{0.851709in}{1.507325in}}%
\pgfpathcurveto{\pgfqpoint{0.845885in}{1.501501in}}{\pgfqpoint{0.842612in}{1.493601in}}{\pgfqpoint{0.842612in}{1.485365in}}%
\pgfpathcurveto{\pgfqpoint{0.842612in}{1.477129in}}{\pgfqpoint{0.845885in}{1.469229in}}{\pgfqpoint{0.851709in}{1.463405in}}%
\pgfpathcurveto{\pgfqpoint{0.857533in}{1.457581in}}{\pgfqpoint{0.865433in}{1.454309in}}{\pgfqpoint{0.873669in}{1.454309in}}%
\pgfpathclose%
\pgfusepath{stroke,fill}%
\end{pgfscope}%
\begin{pgfscope}%
\pgfpathrectangle{\pgfqpoint{0.100000in}{0.220728in}}{\pgfqpoint{3.696000in}{3.696000in}}%
\pgfusepath{clip}%
\pgfsetbuttcap%
\pgfsetroundjoin%
\definecolor{currentfill}{rgb}{0.121569,0.466667,0.705882}%
\pgfsetfillcolor{currentfill}%
\pgfsetfillopacity{0.645678}%
\pgfsetlinewidth{1.003750pt}%
\definecolor{currentstroke}{rgb}{0.121569,0.466667,0.705882}%
\pgfsetstrokecolor{currentstroke}%
\pgfsetstrokeopacity{0.645678}%
\pgfsetdash{}{0pt}%
\pgfpathmoveto{\pgfqpoint{0.873667in}{1.454307in}}%
\pgfpathcurveto{\pgfqpoint{0.881903in}{1.454307in}}{\pgfqpoint{0.889803in}{1.457580in}}{\pgfqpoint{0.895627in}{1.463404in}}%
\pgfpathcurveto{\pgfqpoint{0.901451in}{1.469228in}}{\pgfqpoint{0.904724in}{1.477128in}}{\pgfqpoint{0.904724in}{1.485364in}}%
\pgfpathcurveto{\pgfqpoint{0.904724in}{1.493600in}}{\pgfqpoint{0.901451in}{1.501500in}}{\pgfqpoint{0.895627in}{1.507324in}}%
\pgfpathcurveto{\pgfqpoint{0.889803in}{1.513148in}}{\pgfqpoint{0.881903in}{1.516420in}}{\pgfqpoint{0.873667in}{1.516420in}}%
\pgfpathcurveto{\pgfqpoint{0.865431in}{1.516420in}}{\pgfqpoint{0.857531in}{1.513148in}}{\pgfqpoint{0.851707in}{1.507324in}}%
\pgfpathcurveto{\pgfqpoint{0.845883in}{1.501500in}}{\pgfqpoint{0.842611in}{1.493600in}}{\pgfqpoint{0.842611in}{1.485364in}}%
\pgfpathcurveto{\pgfqpoint{0.842611in}{1.477128in}}{\pgfqpoint{0.845883in}{1.469228in}}{\pgfqpoint{0.851707in}{1.463404in}}%
\pgfpathcurveto{\pgfqpoint{0.857531in}{1.457580in}}{\pgfqpoint{0.865431in}{1.454307in}}{\pgfqpoint{0.873667in}{1.454307in}}%
\pgfpathclose%
\pgfusepath{stroke,fill}%
\end{pgfscope}%
\begin{pgfscope}%
\pgfpathrectangle{\pgfqpoint{0.100000in}{0.220728in}}{\pgfqpoint{3.696000in}{3.696000in}}%
\pgfusepath{clip}%
\pgfsetbuttcap%
\pgfsetroundjoin%
\definecolor{currentfill}{rgb}{0.121569,0.466667,0.705882}%
\pgfsetfillcolor{currentfill}%
\pgfsetfillopacity{0.645679}%
\pgfsetlinewidth{1.003750pt}%
\definecolor{currentstroke}{rgb}{0.121569,0.466667,0.705882}%
\pgfsetstrokecolor{currentstroke}%
\pgfsetstrokeopacity{0.645679}%
\pgfsetdash{}{0pt}%
\pgfpathmoveto{\pgfqpoint{0.873663in}{1.454305in}}%
\pgfpathcurveto{\pgfqpoint{0.881900in}{1.454305in}}{\pgfqpoint{0.889800in}{1.457577in}}{\pgfqpoint{0.895624in}{1.463401in}}%
\pgfpathcurveto{\pgfqpoint{0.901448in}{1.469225in}}{\pgfqpoint{0.904720in}{1.477125in}}{\pgfqpoint{0.904720in}{1.485362in}}%
\pgfpathcurveto{\pgfqpoint{0.904720in}{1.493598in}}{\pgfqpoint{0.901448in}{1.501498in}}{\pgfqpoint{0.895624in}{1.507322in}}%
\pgfpathcurveto{\pgfqpoint{0.889800in}{1.513146in}}{\pgfqpoint{0.881900in}{1.516418in}}{\pgfqpoint{0.873663in}{1.516418in}}%
\pgfpathcurveto{\pgfqpoint{0.865427in}{1.516418in}}{\pgfqpoint{0.857527in}{1.513146in}}{\pgfqpoint{0.851703in}{1.507322in}}%
\pgfpathcurveto{\pgfqpoint{0.845879in}{1.501498in}}{\pgfqpoint{0.842607in}{1.493598in}}{\pgfqpoint{0.842607in}{1.485362in}}%
\pgfpathcurveto{\pgfqpoint{0.842607in}{1.477125in}}{\pgfqpoint{0.845879in}{1.469225in}}{\pgfqpoint{0.851703in}{1.463401in}}%
\pgfpathcurveto{\pgfqpoint{0.857527in}{1.457577in}}{\pgfqpoint{0.865427in}{1.454305in}}{\pgfqpoint{0.873663in}{1.454305in}}%
\pgfpathclose%
\pgfusepath{stroke,fill}%
\end{pgfscope}%
\begin{pgfscope}%
\pgfpathrectangle{\pgfqpoint{0.100000in}{0.220728in}}{\pgfqpoint{3.696000in}{3.696000in}}%
\pgfusepath{clip}%
\pgfsetbuttcap%
\pgfsetroundjoin%
\definecolor{currentfill}{rgb}{0.121569,0.466667,0.705882}%
\pgfsetfillcolor{currentfill}%
\pgfsetfillopacity{0.645681}%
\pgfsetlinewidth{1.003750pt}%
\definecolor{currentstroke}{rgb}{0.121569,0.466667,0.705882}%
\pgfsetstrokecolor{currentstroke}%
\pgfsetstrokeopacity{0.645681}%
\pgfsetdash{}{0pt}%
\pgfpathmoveto{\pgfqpoint{0.873657in}{1.454300in}}%
\pgfpathcurveto{\pgfqpoint{0.881893in}{1.454300in}}{\pgfqpoint{0.889794in}{1.457572in}}{\pgfqpoint{0.895617in}{1.463396in}}%
\pgfpathcurveto{\pgfqpoint{0.901441in}{1.469220in}}{\pgfqpoint{0.904714in}{1.477120in}}{\pgfqpoint{0.904714in}{1.485356in}}%
\pgfpathcurveto{\pgfqpoint{0.904714in}{1.493592in}}{\pgfqpoint{0.901441in}{1.501492in}}{\pgfqpoint{0.895617in}{1.507316in}}%
\pgfpathcurveto{\pgfqpoint{0.889794in}{1.513140in}}{\pgfqpoint{0.881893in}{1.516413in}}{\pgfqpoint{0.873657in}{1.516413in}}%
\pgfpathcurveto{\pgfqpoint{0.865421in}{1.516413in}}{\pgfqpoint{0.857521in}{1.513140in}}{\pgfqpoint{0.851697in}{1.507316in}}%
\pgfpathcurveto{\pgfqpoint{0.845873in}{1.501492in}}{\pgfqpoint{0.842601in}{1.493592in}}{\pgfqpoint{0.842601in}{1.485356in}}%
\pgfpathcurveto{\pgfqpoint{0.842601in}{1.477120in}}{\pgfqpoint{0.845873in}{1.469220in}}{\pgfqpoint{0.851697in}{1.463396in}}%
\pgfpathcurveto{\pgfqpoint{0.857521in}{1.457572in}}{\pgfqpoint{0.865421in}{1.454300in}}{\pgfqpoint{0.873657in}{1.454300in}}%
\pgfpathclose%
\pgfusepath{stroke,fill}%
\end{pgfscope}%
\begin{pgfscope}%
\pgfpathrectangle{\pgfqpoint{0.100000in}{0.220728in}}{\pgfqpoint{3.696000in}{3.696000in}}%
\pgfusepath{clip}%
\pgfsetbuttcap%
\pgfsetroundjoin%
\definecolor{currentfill}{rgb}{0.121569,0.466667,0.705882}%
\pgfsetfillcolor{currentfill}%
\pgfsetfillopacity{0.645685}%
\pgfsetlinewidth{1.003750pt}%
\definecolor{currentstroke}{rgb}{0.121569,0.466667,0.705882}%
\pgfsetstrokecolor{currentstroke}%
\pgfsetstrokeopacity{0.645685}%
\pgfsetdash{}{0pt}%
\pgfpathmoveto{\pgfqpoint{0.873645in}{1.454294in}}%
\pgfpathcurveto{\pgfqpoint{0.881881in}{1.454294in}}{\pgfqpoint{0.889782in}{1.457567in}}{\pgfqpoint{0.895605in}{1.463391in}}%
\pgfpathcurveto{\pgfqpoint{0.901429in}{1.469215in}}{\pgfqpoint{0.904702in}{1.477115in}}{\pgfqpoint{0.904702in}{1.485351in}}%
\pgfpathcurveto{\pgfqpoint{0.904702in}{1.493587in}}{\pgfqpoint{0.901429in}{1.501487in}}{\pgfqpoint{0.895605in}{1.507311in}}%
\pgfpathcurveto{\pgfqpoint{0.889782in}{1.513135in}}{\pgfqpoint{0.881881in}{1.516407in}}{\pgfqpoint{0.873645in}{1.516407in}}%
\pgfpathcurveto{\pgfqpoint{0.865409in}{1.516407in}}{\pgfqpoint{0.857509in}{1.513135in}}{\pgfqpoint{0.851685in}{1.507311in}}%
\pgfpathcurveto{\pgfqpoint{0.845861in}{1.501487in}}{\pgfqpoint{0.842589in}{1.493587in}}{\pgfqpoint{0.842589in}{1.485351in}}%
\pgfpathcurveto{\pgfqpoint{0.842589in}{1.477115in}}{\pgfqpoint{0.845861in}{1.469215in}}{\pgfqpoint{0.851685in}{1.463391in}}%
\pgfpathcurveto{\pgfqpoint{0.857509in}{1.457567in}}{\pgfqpoint{0.865409in}{1.454294in}}{\pgfqpoint{0.873645in}{1.454294in}}%
\pgfpathclose%
\pgfusepath{stroke,fill}%
\end{pgfscope}%
\begin{pgfscope}%
\pgfpathrectangle{\pgfqpoint{0.100000in}{0.220728in}}{\pgfqpoint{3.696000in}{3.696000in}}%
\pgfusepath{clip}%
\pgfsetbuttcap%
\pgfsetroundjoin%
\definecolor{currentfill}{rgb}{0.121569,0.466667,0.705882}%
\pgfsetfillcolor{currentfill}%
\pgfsetfillopacity{0.645691}%
\pgfsetlinewidth{1.003750pt}%
\definecolor{currentstroke}{rgb}{0.121569,0.466667,0.705882}%
\pgfsetstrokecolor{currentstroke}%
\pgfsetstrokeopacity{0.645691}%
\pgfsetdash{}{0pt}%
\pgfpathmoveto{\pgfqpoint{0.873623in}{1.454279in}}%
\pgfpathcurveto{\pgfqpoint{0.881859in}{1.454279in}}{\pgfqpoint{0.889759in}{1.457551in}}{\pgfqpoint{0.895583in}{1.463375in}}%
\pgfpathcurveto{\pgfqpoint{0.901407in}{1.469199in}}{\pgfqpoint{0.904679in}{1.477099in}}{\pgfqpoint{0.904679in}{1.485336in}}%
\pgfpathcurveto{\pgfqpoint{0.904679in}{1.493572in}}{\pgfqpoint{0.901407in}{1.501472in}}{\pgfqpoint{0.895583in}{1.507296in}}%
\pgfpathcurveto{\pgfqpoint{0.889759in}{1.513120in}}{\pgfqpoint{0.881859in}{1.516392in}}{\pgfqpoint{0.873623in}{1.516392in}}%
\pgfpathcurveto{\pgfqpoint{0.865387in}{1.516392in}}{\pgfqpoint{0.857486in}{1.513120in}}{\pgfqpoint{0.851663in}{1.507296in}}%
\pgfpathcurveto{\pgfqpoint{0.845839in}{1.501472in}}{\pgfqpoint{0.842566in}{1.493572in}}{\pgfqpoint{0.842566in}{1.485336in}}%
\pgfpathcurveto{\pgfqpoint{0.842566in}{1.477099in}}{\pgfqpoint{0.845839in}{1.469199in}}{\pgfqpoint{0.851663in}{1.463375in}}%
\pgfpathcurveto{\pgfqpoint{0.857486in}{1.457551in}}{\pgfqpoint{0.865387in}{1.454279in}}{\pgfqpoint{0.873623in}{1.454279in}}%
\pgfpathclose%
\pgfusepath{stroke,fill}%
\end{pgfscope}%
\begin{pgfscope}%
\pgfpathrectangle{\pgfqpoint{0.100000in}{0.220728in}}{\pgfqpoint{3.696000in}{3.696000in}}%
\pgfusepath{clip}%
\pgfsetbuttcap%
\pgfsetroundjoin%
\definecolor{currentfill}{rgb}{0.121569,0.466667,0.705882}%
\pgfsetfillcolor{currentfill}%
\pgfsetfillopacity{0.645705}%
\pgfsetlinewidth{1.003750pt}%
\definecolor{currentstroke}{rgb}{0.121569,0.466667,0.705882}%
\pgfsetstrokecolor{currentstroke}%
\pgfsetstrokeopacity{0.645705}%
\pgfsetdash{}{0pt}%
\pgfpathmoveto{\pgfqpoint{0.873588in}{1.454258in}}%
\pgfpathcurveto{\pgfqpoint{0.881824in}{1.454258in}}{\pgfqpoint{0.889724in}{1.457530in}}{\pgfqpoint{0.895548in}{1.463354in}}%
\pgfpathcurveto{\pgfqpoint{0.901372in}{1.469178in}}{\pgfqpoint{0.904645in}{1.477078in}}{\pgfqpoint{0.904645in}{1.485314in}}%
\pgfpathcurveto{\pgfqpoint{0.904645in}{1.493551in}}{\pgfqpoint{0.901372in}{1.501451in}}{\pgfqpoint{0.895548in}{1.507275in}}%
\pgfpathcurveto{\pgfqpoint{0.889724in}{1.513099in}}{\pgfqpoint{0.881824in}{1.516371in}}{\pgfqpoint{0.873588in}{1.516371in}}%
\pgfpathcurveto{\pgfqpoint{0.865352in}{1.516371in}}{\pgfqpoint{0.857452in}{1.513099in}}{\pgfqpoint{0.851628in}{1.507275in}}%
\pgfpathcurveto{\pgfqpoint{0.845804in}{1.501451in}}{\pgfqpoint{0.842532in}{1.493551in}}{\pgfqpoint{0.842532in}{1.485314in}}%
\pgfpathcurveto{\pgfqpoint{0.842532in}{1.477078in}}{\pgfqpoint{0.845804in}{1.469178in}}{\pgfqpoint{0.851628in}{1.463354in}}%
\pgfpathcurveto{\pgfqpoint{0.857452in}{1.457530in}}{\pgfqpoint{0.865352in}{1.454258in}}{\pgfqpoint{0.873588in}{1.454258in}}%
\pgfpathclose%
\pgfusepath{stroke,fill}%
\end{pgfscope}%
\begin{pgfscope}%
\pgfpathrectangle{\pgfqpoint{0.100000in}{0.220728in}}{\pgfqpoint{3.696000in}{3.696000in}}%
\pgfusepath{clip}%
\pgfsetbuttcap%
\pgfsetroundjoin%
\definecolor{currentfill}{rgb}{0.121569,0.466667,0.705882}%
\pgfsetfillcolor{currentfill}%
\pgfsetfillopacity{0.645730}%
\pgfsetlinewidth{1.003750pt}%
\definecolor{currentstroke}{rgb}{0.121569,0.466667,0.705882}%
\pgfsetstrokecolor{currentstroke}%
\pgfsetstrokeopacity{0.645730}%
\pgfsetdash{}{0pt}%
\pgfpathmoveto{\pgfqpoint{0.873504in}{1.454253in}}%
\pgfpathcurveto{\pgfqpoint{0.881741in}{1.454253in}}{\pgfqpoint{0.889641in}{1.457525in}}{\pgfqpoint{0.895465in}{1.463349in}}%
\pgfpathcurveto{\pgfqpoint{0.901288in}{1.469173in}}{\pgfqpoint{0.904561in}{1.477073in}}{\pgfqpoint{0.904561in}{1.485310in}}%
\pgfpathcurveto{\pgfqpoint{0.904561in}{1.493546in}}{\pgfqpoint{0.901288in}{1.501446in}}{\pgfqpoint{0.895465in}{1.507270in}}%
\pgfpathcurveto{\pgfqpoint{0.889641in}{1.513094in}}{\pgfqpoint{0.881741in}{1.516366in}}{\pgfqpoint{0.873504in}{1.516366in}}%
\pgfpathcurveto{\pgfqpoint{0.865268in}{1.516366in}}{\pgfqpoint{0.857368in}{1.513094in}}{\pgfqpoint{0.851544in}{1.507270in}}%
\pgfpathcurveto{\pgfqpoint{0.845720in}{1.501446in}}{\pgfqpoint{0.842448in}{1.493546in}}{\pgfqpoint{0.842448in}{1.485310in}}%
\pgfpathcurveto{\pgfqpoint{0.842448in}{1.477073in}}{\pgfqpoint{0.845720in}{1.469173in}}{\pgfqpoint{0.851544in}{1.463349in}}%
\pgfpathcurveto{\pgfqpoint{0.857368in}{1.457525in}}{\pgfqpoint{0.865268in}{1.454253in}}{\pgfqpoint{0.873504in}{1.454253in}}%
\pgfpathclose%
\pgfusepath{stroke,fill}%
\end{pgfscope}%
\begin{pgfscope}%
\pgfpathrectangle{\pgfqpoint{0.100000in}{0.220728in}}{\pgfqpoint{3.696000in}{3.696000in}}%
\pgfusepath{clip}%
\pgfsetbuttcap%
\pgfsetroundjoin%
\definecolor{currentfill}{rgb}{0.121569,0.466667,0.705882}%
\pgfsetfillcolor{currentfill}%
\pgfsetfillopacity{0.645764}%
\pgfsetlinewidth{1.003750pt}%
\definecolor{currentstroke}{rgb}{0.121569,0.466667,0.705882}%
\pgfsetstrokecolor{currentstroke}%
\pgfsetstrokeopacity{0.645764}%
\pgfsetdash{}{0pt}%
\pgfpathmoveto{\pgfqpoint{0.873380in}{1.454129in}}%
\pgfpathcurveto{\pgfqpoint{0.881616in}{1.454129in}}{\pgfqpoint{0.889516in}{1.457401in}}{\pgfqpoint{0.895340in}{1.463225in}}%
\pgfpathcurveto{\pgfqpoint{0.901164in}{1.469049in}}{\pgfqpoint{0.904437in}{1.476949in}}{\pgfqpoint{0.904437in}{1.485185in}}%
\pgfpathcurveto{\pgfqpoint{0.904437in}{1.493421in}}{\pgfqpoint{0.901164in}{1.501321in}}{\pgfqpoint{0.895340in}{1.507145in}}%
\pgfpathcurveto{\pgfqpoint{0.889516in}{1.512969in}}{\pgfqpoint{0.881616in}{1.516242in}}{\pgfqpoint{0.873380in}{1.516242in}}%
\pgfpathcurveto{\pgfqpoint{0.865144in}{1.516242in}}{\pgfqpoint{0.857244in}{1.512969in}}{\pgfqpoint{0.851420in}{1.507145in}}%
\pgfpathcurveto{\pgfqpoint{0.845596in}{1.501321in}}{\pgfqpoint{0.842324in}{1.493421in}}{\pgfqpoint{0.842324in}{1.485185in}}%
\pgfpathcurveto{\pgfqpoint{0.842324in}{1.476949in}}{\pgfqpoint{0.845596in}{1.469049in}}{\pgfqpoint{0.851420in}{1.463225in}}%
\pgfpathcurveto{\pgfqpoint{0.857244in}{1.457401in}}{\pgfqpoint{0.865144in}{1.454129in}}{\pgfqpoint{0.873380in}{1.454129in}}%
\pgfpathclose%
\pgfusepath{stroke,fill}%
\end{pgfscope}%
\begin{pgfscope}%
\pgfpathrectangle{\pgfqpoint{0.100000in}{0.220728in}}{\pgfqpoint{3.696000in}{3.696000in}}%
\pgfusepath{clip}%
\pgfsetbuttcap%
\pgfsetroundjoin%
\definecolor{currentfill}{rgb}{0.121569,0.466667,0.705882}%
\pgfsetfillcolor{currentfill}%
\pgfsetfillopacity{0.645842}%
\pgfsetlinewidth{1.003750pt}%
\definecolor{currentstroke}{rgb}{0.121569,0.466667,0.705882}%
\pgfsetstrokecolor{currentstroke}%
\pgfsetstrokeopacity{0.645842}%
\pgfsetdash{}{0pt}%
\pgfpathmoveto{\pgfqpoint{0.873152in}{1.453999in}}%
\pgfpathcurveto{\pgfqpoint{0.881388in}{1.453999in}}{\pgfqpoint{0.889288in}{1.457272in}}{\pgfqpoint{0.895112in}{1.463095in}}%
\pgfpathcurveto{\pgfqpoint{0.900936in}{1.468919in}}{\pgfqpoint{0.904208in}{1.476819in}}{\pgfqpoint{0.904208in}{1.485056in}}%
\pgfpathcurveto{\pgfqpoint{0.904208in}{1.493292in}}{\pgfqpoint{0.900936in}{1.501192in}}{\pgfqpoint{0.895112in}{1.507016in}}%
\pgfpathcurveto{\pgfqpoint{0.889288in}{1.512840in}}{\pgfqpoint{0.881388in}{1.516112in}}{\pgfqpoint{0.873152in}{1.516112in}}%
\pgfpathcurveto{\pgfqpoint{0.864916in}{1.516112in}}{\pgfqpoint{0.857016in}{1.512840in}}{\pgfqpoint{0.851192in}{1.507016in}}%
\pgfpathcurveto{\pgfqpoint{0.845368in}{1.501192in}}{\pgfqpoint{0.842095in}{1.493292in}}{\pgfqpoint{0.842095in}{1.485056in}}%
\pgfpathcurveto{\pgfqpoint{0.842095in}{1.476819in}}{\pgfqpoint{0.845368in}{1.468919in}}{\pgfqpoint{0.851192in}{1.463095in}}%
\pgfpathcurveto{\pgfqpoint{0.857016in}{1.457272in}}{\pgfqpoint{0.864916in}{1.453999in}}{\pgfqpoint{0.873152in}{1.453999in}}%
\pgfpathclose%
\pgfusepath{stroke,fill}%
\end{pgfscope}%
\begin{pgfscope}%
\pgfpathrectangle{\pgfqpoint{0.100000in}{0.220728in}}{\pgfqpoint{3.696000in}{3.696000in}}%
\pgfusepath{clip}%
\pgfsetbuttcap%
\pgfsetroundjoin%
\definecolor{currentfill}{rgb}{0.121569,0.466667,0.705882}%
\pgfsetfillcolor{currentfill}%
\pgfsetfillopacity{0.645967}%
\pgfsetlinewidth{1.003750pt}%
\definecolor{currentstroke}{rgb}{0.121569,0.466667,0.705882}%
\pgfsetstrokecolor{currentstroke}%
\pgfsetstrokeopacity{0.645967}%
\pgfsetdash{}{0pt}%
\pgfpathmoveto{\pgfqpoint{0.872717in}{1.453689in}}%
\pgfpathcurveto{\pgfqpoint{0.880953in}{1.453689in}}{\pgfqpoint{0.888853in}{1.456961in}}{\pgfqpoint{0.894677in}{1.462785in}}%
\pgfpathcurveto{\pgfqpoint{0.900501in}{1.468609in}}{\pgfqpoint{0.903773in}{1.476509in}}{\pgfqpoint{0.903773in}{1.484745in}}%
\pgfpathcurveto{\pgfqpoint{0.903773in}{1.492981in}}{\pgfqpoint{0.900501in}{1.500881in}}{\pgfqpoint{0.894677in}{1.506705in}}%
\pgfpathcurveto{\pgfqpoint{0.888853in}{1.512529in}}{\pgfqpoint{0.880953in}{1.515802in}}{\pgfqpoint{0.872717in}{1.515802in}}%
\pgfpathcurveto{\pgfqpoint{0.864480in}{1.515802in}}{\pgfqpoint{0.856580in}{1.512529in}}{\pgfqpoint{0.850756in}{1.506705in}}%
\pgfpathcurveto{\pgfqpoint{0.844932in}{1.500881in}}{\pgfqpoint{0.841660in}{1.492981in}}{\pgfqpoint{0.841660in}{1.484745in}}%
\pgfpathcurveto{\pgfqpoint{0.841660in}{1.476509in}}{\pgfqpoint{0.844932in}{1.468609in}}{\pgfqpoint{0.850756in}{1.462785in}}%
\pgfpathcurveto{\pgfqpoint{0.856580in}{1.456961in}}{\pgfqpoint{0.864480in}{1.453689in}}{\pgfqpoint{0.872717in}{1.453689in}}%
\pgfpathclose%
\pgfusepath{stroke,fill}%
\end{pgfscope}%
\begin{pgfscope}%
\pgfpathrectangle{\pgfqpoint{0.100000in}{0.220728in}}{\pgfqpoint{3.696000in}{3.696000in}}%
\pgfusepath{clip}%
\pgfsetbuttcap%
\pgfsetroundjoin%
\definecolor{currentfill}{rgb}{0.121569,0.466667,0.705882}%
\pgfsetfillcolor{currentfill}%
\pgfsetfillopacity{0.646227}%
\pgfsetlinewidth{1.003750pt}%
\definecolor{currentstroke}{rgb}{0.121569,0.466667,0.705882}%
\pgfsetstrokecolor{currentstroke}%
\pgfsetstrokeopacity{0.646227}%
\pgfsetdash{}{0pt}%
\pgfpathmoveto{\pgfqpoint{0.871974in}{1.453262in}}%
\pgfpathcurveto{\pgfqpoint{0.880211in}{1.453262in}}{\pgfqpoint{0.888111in}{1.456535in}}{\pgfqpoint{0.893935in}{1.462359in}}%
\pgfpathcurveto{\pgfqpoint{0.899759in}{1.468183in}}{\pgfqpoint{0.903031in}{1.476083in}}{\pgfqpoint{0.903031in}{1.484319in}}%
\pgfpathcurveto{\pgfqpoint{0.903031in}{1.492555in}}{\pgfqpoint{0.899759in}{1.500455in}}{\pgfqpoint{0.893935in}{1.506279in}}%
\pgfpathcurveto{\pgfqpoint{0.888111in}{1.512103in}}{\pgfqpoint{0.880211in}{1.515375in}}{\pgfqpoint{0.871974in}{1.515375in}}%
\pgfpathcurveto{\pgfqpoint{0.863738in}{1.515375in}}{\pgfqpoint{0.855838in}{1.512103in}}{\pgfqpoint{0.850014in}{1.506279in}}%
\pgfpathcurveto{\pgfqpoint{0.844190in}{1.500455in}}{\pgfqpoint{0.840918in}{1.492555in}}{\pgfqpoint{0.840918in}{1.484319in}}%
\pgfpathcurveto{\pgfqpoint{0.840918in}{1.476083in}}{\pgfqpoint{0.844190in}{1.468183in}}{\pgfqpoint{0.850014in}{1.462359in}}%
\pgfpathcurveto{\pgfqpoint{0.855838in}{1.456535in}}{\pgfqpoint{0.863738in}{1.453262in}}{\pgfqpoint{0.871974in}{1.453262in}}%
\pgfpathclose%
\pgfusepath{stroke,fill}%
\end{pgfscope}%
\begin{pgfscope}%
\pgfpathrectangle{\pgfqpoint{0.100000in}{0.220728in}}{\pgfqpoint{3.696000in}{3.696000in}}%
\pgfusepath{clip}%
\pgfsetbuttcap%
\pgfsetroundjoin%
\definecolor{currentfill}{rgb}{0.121569,0.466667,0.705882}%
\pgfsetfillcolor{currentfill}%
\pgfsetfillopacity{0.646406}%
\pgfsetlinewidth{1.003750pt}%
\definecolor{currentstroke}{rgb}{0.121569,0.466667,0.705882}%
\pgfsetstrokecolor{currentstroke}%
\pgfsetstrokeopacity{0.646406}%
\pgfsetdash{}{0pt}%
\pgfpathmoveto{\pgfqpoint{2.138653in}{1.760050in}}%
\pgfpathcurveto{\pgfqpoint{2.146889in}{1.760050in}}{\pgfqpoint{2.154789in}{1.763323in}}{\pgfqpoint{2.160613in}{1.769147in}}%
\pgfpathcurveto{\pgfqpoint{2.166437in}{1.774971in}}{\pgfqpoint{2.169709in}{1.782871in}}{\pgfqpoint{2.169709in}{1.791107in}}%
\pgfpathcurveto{\pgfqpoint{2.169709in}{1.799343in}}{\pgfqpoint{2.166437in}{1.807243in}}{\pgfqpoint{2.160613in}{1.813067in}}%
\pgfpathcurveto{\pgfqpoint{2.154789in}{1.818891in}}{\pgfqpoint{2.146889in}{1.822163in}}{\pgfqpoint{2.138653in}{1.822163in}}%
\pgfpathcurveto{\pgfqpoint{2.130416in}{1.822163in}}{\pgfqpoint{2.122516in}{1.818891in}}{\pgfqpoint{2.116692in}{1.813067in}}%
\pgfpathcurveto{\pgfqpoint{2.110868in}{1.807243in}}{\pgfqpoint{2.107596in}{1.799343in}}{\pgfqpoint{2.107596in}{1.791107in}}%
\pgfpathcurveto{\pgfqpoint{2.107596in}{1.782871in}}{\pgfqpoint{2.110868in}{1.774971in}}{\pgfqpoint{2.116692in}{1.769147in}}%
\pgfpathcurveto{\pgfqpoint{2.122516in}{1.763323in}}{\pgfqpoint{2.130416in}{1.760050in}}{\pgfqpoint{2.138653in}{1.760050in}}%
\pgfpathclose%
\pgfusepath{stroke,fill}%
\end{pgfscope}%
\begin{pgfscope}%
\pgfpathrectangle{\pgfqpoint{0.100000in}{0.220728in}}{\pgfqpoint{3.696000in}{3.696000in}}%
\pgfusepath{clip}%
\pgfsetbuttcap%
\pgfsetroundjoin%
\definecolor{currentfill}{rgb}{0.121569,0.466667,0.705882}%
\pgfsetfillcolor{currentfill}%
\pgfsetfillopacity{0.646646}%
\pgfsetlinewidth{1.003750pt}%
\definecolor{currentstroke}{rgb}{0.121569,0.466667,0.705882}%
\pgfsetstrokecolor{currentstroke}%
\pgfsetstrokeopacity{0.646646}%
\pgfsetdash{}{0pt}%
\pgfpathmoveto{\pgfqpoint{0.870640in}{1.452133in}}%
\pgfpathcurveto{\pgfqpoint{0.878876in}{1.452133in}}{\pgfqpoint{0.886776in}{1.455405in}}{\pgfqpoint{0.892600in}{1.461229in}}%
\pgfpathcurveto{\pgfqpoint{0.898424in}{1.467053in}}{\pgfqpoint{0.901696in}{1.474953in}}{\pgfqpoint{0.901696in}{1.483189in}}%
\pgfpathcurveto{\pgfqpoint{0.901696in}{1.491426in}}{\pgfqpoint{0.898424in}{1.499326in}}{\pgfqpoint{0.892600in}{1.505150in}}%
\pgfpathcurveto{\pgfqpoint{0.886776in}{1.510974in}}{\pgfqpoint{0.878876in}{1.514246in}}{\pgfqpoint{0.870640in}{1.514246in}}%
\pgfpathcurveto{\pgfqpoint{0.862403in}{1.514246in}}{\pgfqpoint{0.854503in}{1.510974in}}{\pgfqpoint{0.848680in}{1.505150in}}%
\pgfpathcurveto{\pgfqpoint{0.842856in}{1.499326in}}{\pgfqpoint{0.839583in}{1.491426in}}{\pgfqpoint{0.839583in}{1.483189in}}%
\pgfpathcurveto{\pgfqpoint{0.839583in}{1.474953in}}{\pgfqpoint{0.842856in}{1.467053in}}{\pgfqpoint{0.848680in}{1.461229in}}%
\pgfpathcurveto{\pgfqpoint{0.854503in}{1.455405in}}{\pgfqpoint{0.862403in}{1.452133in}}{\pgfqpoint{0.870640in}{1.452133in}}%
\pgfpathclose%
\pgfusepath{stroke,fill}%
\end{pgfscope}%
\begin{pgfscope}%
\pgfpathrectangle{\pgfqpoint{0.100000in}{0.220728in}}{\pgfqpoint{3.696000in}{3.696000in}}%
\pgfusepath{clip}%
\pgfsetbuttcap%
\pgfsetroundjoin%
\definecolor{currentfill}{rgb}{0.121569,0.466667,0.705882}%
\pgfsetfillcolor{currentfill}%
\pgfsetfillopacity{0.646928}%
\pgfsetlinewidth{1.003750pt}%
\definecolor{currentstroke}{rgb}{0.121569,0.466667,0.705882}%
\pgfsetstrokecolor{currentstroke}%
\pgfsetstrokeopacity{0.646928}%
\pgfsetdash{}{0pt}%
\pgfpathmoveto{\pgfqpoint{0.883216in}{1.415445in}}%
\pgfpathcurveto{\pgfqpoint{0.891452in}{1.415445in}}{\pgfqpoint{0.899352in}{1.418717in}}{\pgfqpoint{0.905176in}{1.424541in}}%
\pgfpathcurveto{\pgfqpoint{0.911000in}{1.430365in}}{\pgfqpoint{0.914272in}{1.438265in}}{\pgfqpoint{0.914272in}{1.446502in}}%
\pgfpathcurveto{\pgfqpoint{0.914272in}{1.454738in}}{\pgfqpoint{0.911000in}{1.462638in}}{\pgfqpoint{0.905176in}{1.468462in}}%
\pgfpathcurveto{\pgfqpoint{0.899352in}{1.474286in}}{\pgfqpoint{0.891452in}{1.477558in}}{\pgfqpoint{0.883216in}{1.477558in}}%
\pgfpathcurveto{\pgfqpoint{0.874980in}{1.477558in}}{\pgfqpoint{0.867080in}{1.474286in}}{\pgfqpoint{0.861256in}{1.468462in}}%
\pgfpathcurveto{\pgfqpoint{0.855432in}{1.462638in}}{\pgfqpoint{0.852159in}{1.454738in}}{\pgfqpoint{0.852159in}{1.446502in}}%
\pgfpathcurveto{\pgfqpoint{0.852159in}{1.438265in}}{\pgfqpoint{0.855432in}{1.430365in}}{\pgfqpoint{0.861256in}{1.424541in}}%
\pgfpathcurveto{\pgfqpoint{0.867080in}{1.418717in}}{\pgfqpoint{0.874980in}{1.415445in}}{\pgfqpoint{0.883216in}{1.415445in}}%
\pgfpathclose%
\pgfusepath{stroke,fill}%
\end{pgfscope}%
\begin{pgfscope}%
\pgfpathrectangle{\pgfqpoint{0.100000in}{0.220728in}}{\pgfqpoint{3.696000in}{3.696000in}}%
\pgfusepath{clip}%
\pgfsetbuttcap%
\pgfsetroundjoin%
\definecolor{currentfill}{rgb}{0.121569,0.466667,0.705882}%
\pgfsetfillcolor{currentfill}%
\pgfsetfillopacity{0.647072}%
\pgfsetlinewidth{1.003750pt}%
\definecolor{currentstroke}{rgb}{0.121569,0.466667,0.705882}%
\pgfsetstrokecolor{currentstroke}%
\pgfsetstrokeopacity{0.647072}%
\pgfsetdash{}{0pt}%
\pgfpathmoveto{\pgfqpoint{0.883030in}{1.415640in}}%
\pgfpathcurveto{\pgfqpoint{0.891266in}{1.415640in}}{\pgfqpoint{0.899166in}{1.418912in}}{\pgfqpoint{0.904990in}{1.424736in}}%
\pgfpathcurveto{\pgfqpoint{0.910814in}{1.430560in}}{\pgfqpoint{0.914086in}{1.438460in}}{\pgfqpoint{0.914086in}{1.446696in}}%
\pgfpathcurveto{\pgfqpoint{0.914086in}{1.454932in}}{\pgfqpoint{0.910814in}{1.462832in}}{\pgfqpoint{0.904990in}{1.468656in}}%
\pgfpathcurveto{\pgfqpoint{0.899166in}{1.474480in}}{\pgfqpoint{0.891266in}{1.477753in}}{\pgfqpoint{0.883030in}{1.477753in}}%
\pgfpathcurveto{\pgfqpoint{0.874794in}{1.477753in}}{\pgfqpoint{0.866894in}{1.474480in}}{\pgfqpoint{0.861070in}{1.468656in}}%
\pgfpathcurveto{\pgfqpoint{0.855246in}{1.462832in}}{\pgfqpoint{0.851973in}{1.454932in}}{\pgfqpoint{0.851973in}{1.446696in}}%
\pgfpathcurveto{\pgfqpoint{0.851973in}{1.438460in}}{\pgfqpoint{0.855246in}{1.430560in}}{\pgfqpoint{0.861070in}{1.424736in}}%
\pgfpathcurveto{\pgfqpoint{0.866894in}{1.418912in}}{\pgfqpoint{0.874794in}{1.415640in}}{\pgfqpoint{0.883030in}{1.415640in}}%
\pgfpathclose%
\pgfusepath{stroke,fill}%
\end{pgfscope}%
\begin{pgfscope}%
\pgfpathrectangle{\pgfqpoint{0.100000in}{0.220728in}}{\pgfqpoint{3.696000in}{3.696000in}}%
\pgfusepath{clip}%
\pgfsetbuttcap%
\pgfsetroundjoin%
\definecolor{currentfill}{rgb}{0.121569,0.466667,0.705882}%
\pgfsetfillcolor{currentfill}%
\pgfsetfillopacity{0.647400}%
\pgfsetlinewidth{1.003750pt}%
\definecolor{currentstroke}{rgb}{0.121569,0.466667,0.705882}%
\pgfsetstrokecolor{currentstroke}%
\pgfsetstrokeopacity{0.647400}%
\pgfsetdash{}{0pt}%
\pgfpathmoveto{\pgfqpoint{0.882371in}{1.415847in}}%
\pgfpathcurveto{\pgfqpoint{0.890608in}{1.415847in}}{\pgfqpoint{0.898508in}{1.419120in}}{\pgfqpoint{0.904332in}{1.424944in}}%
\pgfpathcurveto{\pgfqpoint{0.910156in}{1.430768in}}{\pgfqpoint{0.913428in}{1.438668in}}{\pgfqpoint{0.913428in}{1.446904in}}%
\pgfpathcurveto{\pgfqpoint{0.913428in}{1.455140in}}{\pgfqpoint{0.910156in}{1.463040in}}{\pgfqpoint{0.904332in}{1.468864in}}%
\pgfpathcurveto{\pgfqpoint{0.898508in}{1.474688in}}{\pgfqpoint{0.890608in}{1.477960in}}{\pgfqpoint{0.882371in}{1.477960in}}%
\pgfpathcurveto{\pgfqpoint{0.874135in}{1.477960in}}{\pgfqpoint{0.866235in}{1.474688in}}{\pgfqpoint{0.860411in}{1.468864in}}%
\pgfpathcurveto{\pgfqpoint{0.854587in}{1.463040in}}{\pgfqpoint{0.851315in}{1.455140in}}{\pgfqpoint{0.851315in}{1.446904in}}%
\pgfpathcurveto{\pgfqpoint{0.851315in}{1.438668in}}{\pgfqpoint{0.854587in}{1.430768in}}{\pgfqpoint{0.860411in}{1.424944in}}%
\pgfpathcurveto{\pgfqpoint{0.866235in}{1.419120in}}{\pgfqpoint{0.874135in}{1.415847in}}{\pgfqpoint{0.882371in}{1.415847in}}%
\pgfpathclose%
\pgfusepath{stroke,fill}%
\end{pgfscope}%
\begin{pgfscope}%
\pgfpathrectangle{\pgfqpoint{0.100000in}{0.220728in}}{\pgfqpoint{3.696000in}{3.696000in}}%
\pgfusepath{clip}%
\pgfsetbuttcap%
\pgfsetroundjoin%
\definecolor{currentfill}{rgb}{0.121569,0.466667,0.705882}%
\pgfsetfillcolor{currentfill}%
\pgfsetfillopacity{0.647419}%
\pgfsetlinewidth{1.003750pt}%
\definecolor{currentstroke}{rgb}{0.121569,0.466667,0.705882}%
\pgfsetstrokecolor{currentstroke}%
\pgfsetstrokeopacity{0.647419}%
\pgfsetdash{}{0pt}%
\pgfpathmoveto{\pgfqpoint{0.868006in}{1.450439in}}%
\pgfpathcurveto{\pgfqpoint{0.876242in}{1.450439in}}{\pgfqpoint{0.884142in}{1.453712in}}{\pgfqpoint{0.889966in}{1.459535in}}%
\pgfpathcurveto{\pgfqpoint{0.895790in}{1.465359in}}{\pgfqpoint{0.899062in}{1.473259in}}{\pgfqpoint{0.899062in}{1.481496in}}%
\pgfpathcurveto{\pgfqpoint{0.899062in}{1.489732in}}{\pgfqpoint{0.895790in}{1.497632in}}{\pgfqpoint{0.889966in}{1.503456in}}%
\pgfpathcurveto{\pgfqpoint{0.884142in}{1.509280in}}{\pgfqpoint{0.876242in}{1.512552in}}{\pgfqpoint{0.868006in}{1.512552in}}%
\pgfpathcurveto{\pgfqpoint{0.859769in}{1.512552in}}{\pgfqpoint{0.851869in}{1.509280in}}{\pgfqpoint{0.846045in}{1.503456in}}%
\pgfpathcurveto{\pgfqpoint{0.840221in}{1.497632in}}{\pgfqpoint{0.836949in}{1.489732in}}{\pgfqpoint{0.836949in}{1.481496in}}%
\pgfpathcurveto{\pgfqpoint{0.836949in}{1.473259in}}{\pgfqpoint{0.840221in}{1.465359in}}{\pgfqpoint{0.846045in}{1.459535in}}%
\pgfpathcurveto{\pgfqpoint{0.851869in}{1.453712in}}{\pgfqpoint{0.859769in}{1.450439in}}{\pgfqpoint{0.868006in}{1.450439in}}%
\pgfpathclose%
\pgfusepath{stroke,fill}%
\end{pgfscope}%
\begin{pgfscope}%
\pgfpathrectangle{\pgfqpoint{0.100000in}{0.220728in}}{\pgfqpoint{3.696000in}{3.696000in}}%
\pgfusepath{clip}%
\pgfsetbuttcap%
\pgfsetroundjoin%
\definecolor{currentfill}{rgb}{0.121569,0.466667,0.705882}%
\pgfsetfillcolor{currentfill}%
\pgfsetfillopacity{0.648138}%
\pgfsetlinewidth{1.003750pt}%
\definecolor{currentstroke}{rgb}{0.121569,0.466667,0.705882}%
\pgfsetstrokecolor{currentstroke}%
\pgfsetstrokeopacity{0.648138}%
\pgfsetdash{}{0pt}%
\pgfpathmoveto{\pgfqpoint{0.881305in}{1.416284in}}%
\pgfpathcurveto{\pgfqpoint{0.889541in}{1.416284in}}{\pgfqpoint{0.897441in}{1.419557in}}{\pgfqpoint{0.903265in}{1.425381in}}%
\pgfpathcurveto{\pgfqpoint{0.909089in}{1.431205in}}{\pgfqpoint{0.912361in}{1.439105in}}{\pgfqpoint{0.912361in}{1.447341in}}%
\pgfpathcurveto{\pgfqpoint{0.912361in}{1.455577in}}{\pgfqpoint{0.909089in}{1.463477in}}{\pgfqpoint{0.903265in}{1.469301in}}%
\pgfpathcurveto{\pgfqpoint{0.897441in}{1.475125in}}{\pgfqpoint{0.889541in}{1.478397in}}{\pgfqpoint{0.881305in}{1.478397in}}%
\pgfpathcurveto{\pgfqpoint{0.873068in}{1.478397in}}{\pgfqpoint{0.865168in}{1.475125in}}{\pgfqpoint{0.859344in}{1.469301in}}%
\pgfpathcurveto{\pgfqpoint{0.853520in}{1.463477in}}{\pgfqpoint{0.850248in}{1.455577in}}{\pgfqpoint{0.850248in}{1.447341in}}%
\pgfpathcurveto{\pgfqpoint{0.850248in}{1.439105in}}{\pgfqpoint{0.853520in}{1.431205in}}{\pgfqpoint{0.859344in}{1.425381in}}%
\pgfpathcurveto{\pgfqpoint{0.865168in}{1.419557in}}{\pgfqpoint{0.873068in}{1.416284in}}{\pgfqpoint{0.881305in}{1.416284in}}%
\pgfpathclose%
\pgfusepath{stroke,fill}%
\end{pgfscope}%
\begin{pgfscope}%
\pgfpathrectangle{\pgfqpoint{0.100000in}{0.220728in}}{\pgfqpoint{3.696000in}{3.696000in}}%
\pgfusepath{clip}%
\pgfsetbuttcap%
\pgfsetroundjoin%
\definecolor{currentfill}{rgb}{0.121569,0.466667,0.705882}%
\pgfsetfillcolor{currentfill}%
\pgfsetfillopacity{0.649134}%
\pgfsetlinewidth{1.003750pt}%
\definecolor{currentstroke}{rgb}{0.121569,0.466667,0.705882}%
\pgfsetstrokecolor{currentstroke}%
\pgfsetstrokeopacity{0.649134}%
\pgfsetdash{}{0pt}%
\pgfpathmoveto{\pgfqpoint{0.863184in}{1.449290in}}%
\pgfpathcurveto{\pgfqpoint{0.871421in}{1.449290in}}{\pgfqpoint{0.879321in}{1.452563in}}{\pgfqpoint{0.885145in}{1.458387in}}%
\pgfpathcurveto{\pgfqpoint{0.890969in}{1.464211in}}{\pgfqpoint{0.894241in}{1.472111in}}{\pgfqpoint{0.894241in}{1.480347in}}%
\pgfpathcurveto{\pgfqpoint{0.894241in}{1.488583in}}{\pgfqpoint{0.890969in}{1.496483in}}{\pgfqpoint{0.885145in}{1.502307in}}%
\pgfpathcurveto{\pgfqpoint{0.879321in}{1.508131in}}{\pgfqpoint{0.871421in}{1.511403in}}{\pgfqpoint{0.863184in}{1.511403in}}%
\pgfpathcurveto{\pgfqpoint{0.854948in}{1.511403in}}{\pgfqpoint{0.847048in}{1.508131in}}{\pgfqpoint{0.841224in}{1.502307in}}%
\pgfpathcurveto{\pgfqpoint{0.835400in}{1.496483in}}{\pgfqpoint{0.832128in}{1.488583in}}{\pgfqpoint{0.832128in}{1.480347in}}%
\pgfpathcurveto{\pgfqpoint{0.832128in}{1.472111in}}{\pgfqpoint{0.835400in}{1.464211in}}{\pgfqpoint{0.841224in}{1.458387in}}%
\pgfpathcurveto{\pgfqpoint{0.847048in}{1.452563in}}{\pgfqpoint{0.854948in}{1.449290in}}{\pgfqpoint{0.863184in}{1.449290in}}%
\pgfpathclose%
\pgfusepath{stroke,fill}%
\end{pgfscope}%
\begin{pgfscope}%
\pgfpathrectangle{\pgfqpoint{0.100000in}{0.220728in}}{\pgfqpoint{3.696000in}{3.696000in}}%
\pgfusepath{clip}%
\pgfsetbuttcap%
\pgfsetroundjoin%
\definecolor{currentfill}{rgb}{0.121569,0.466667,0.705882}%
\pgfsetfillcolor{currentfill}%
\pgfsetfillopacity{0.649209}%
\pgfsetlinewidth{1.003750pt}%
\definecolor{currentstroke}{rgb}{0.121569,0.466667,0.705882}%
\pgfsetstrokecolor{currentstroke}%
\pgfsetstrokeopacity{0.649209}%
\pgfsetdash{}{0pt}%
\pgfpathmoveto{\pgfqpoint{2.141081in}{1.754194in}}%
\pgfpathcurveto{\pgfqpoint{2.149317in}{1.754194in}}{\pgfqpoint{2.157217in}{1.757466in}}{\pgfqpoint{2.163041in}{1.763290in}}%
\pgfpathcurveto{\pgfqpoint{2.168865in}{1.769114in}}{\pgfqpoint{2.172137in}{1.777014in}}{\pgfqpoint{2.172137in}{1.785250in}}%
\pgfpathcurveto{\pgfqpoint{2.172137in}{1.793486in}}{\pgfqpoint{2.168865in}{1.801386in}}{\pgfqpoint{2.163041in}{1.807210in}}%
\pgfpathcurveto{\pgfqpoint{2.157217in}{1.813034in}}{\pgfqpoint{2.149317in}{1.816307in}}{\pgfqpoint{2.141081in}{1.816307in}}%
\pgfpathcurveto{\pgfqpoint{2.132844in}{1.816307in}}{\pgfqpoint{2.124944in}{1.813034in}}{\pgfqpoint{2.119120in}{1.807210in}}%
\pgfpathcurveto{\pgfqpoint{2.113296in}{1.801386in}}{\pgfqpoint{2.110024in}{1.793486in}}{\pgfqpoint{2.110024in}{1.785250in}}%
\pgfpathcurveto{\pgfqpoint{2.110024in}{1.777014in}}{\pgfqpoint{2.113296in}{1.769114in}}{\pgfqpoint{2.119120in}{1.763290in}}%
\pgfpathcurveto{\pgfqpoint{2.124944in}{1.757466in}}{\pgfqpoint{2.132844in}{1.754194in}}{\pgfqpoint{2.141081in}{1.754194in}}%
\pgfpathclose%
\pgfusepath{stroke,fill}%
\end{pgfscope}%
\begin{pgfscope}%
\pgfpathrectangle{\pgfqpoint{0.100000in}{0.220728in}}{\pgfqpoint{3.696000in}{3.696000in}}%
\pgfusepath{clip}%
\pgfsetbuttcap%
\pgfsetroundjoin%
\definecolor{currentfill}{rgb}{0.121569,0.466667,0.705882}%
\pgfsetfillcolor{currentfill}%
\pgfsetfillopacity{0.649249}%
\pgfsetlinewidth{1.003750pt}%
\definecolor{currentstroke}{rgb}{0.121569,0.466667,0.705882}%
\pgfsetstrokecolor{currentstroke}%
\pgfsetstrokeopacity{0.649249}%
\pgfsetdash{}{0pt}%
\pgfpathmoveto{\pgfqpoint{0.879009in}{1.417099in}}%
\pgfpathcurveto{\pgfqpoint{0.887245in}{1.417099in}}{\pgfqpoint{0.895145in}{1.420372in}}{\pgfqpoint{0.900969in}{1.426195in}}%
\pgfpathcurveto{\pgfqpoint{0.906793in}{1.432019in}}{\pgfqpoint{0.910065in}{1.439919in}}{\pgfqpoint{0.910065in}{1.448156in}}%
\pgfpathcurveto{\pgfqpoint{0.910065in}{1.456392in}}{\pgfqpoint{0.906793in}{1.464292in}}{\pgfqpoint{0.900969in}{1.470116in}}%
\pgfpathcurveto{\pgfqpoint{0.895145in}{1.475940in}}{\pgfqpoint{0.887245in}{1.479212in}}{\pgfqpoint{0.879009in}{1.479212in}}%
\pgfpathcurveto{\pgfqpoint{0.870773in}{1.479212in}}{\pgfqpoint{0.862873in}{1.475940in}}{\pgfqpoint{0.857049in}{1.470116in}}%
\pgfpathcurveto{\pgfqpoint{0.851225in}{1.464292in}}{\pgfqpoint{0.847952in}{1.456392in}}{\pgfqpoint{0.847952in}{1.448156in}}%
\pgfpathcurveto{\pgfqpoint{0.847952in}{1.439919in}}{\pgfqpoint{0.851225in}{1.432019in}}{\pgfqpoint{0.857049in}{1.426195in}}%
\pgfpathcurveto{\pgfqpoint{0.862873in}{1.420372in}}{\pgfqpoint{0.870773in}{1.417099in}}{\pgfqpoint{0.879009in}{1.417099in}}%
\pgfpathclose%
\pgfusepath{stroke,fill}%
\end{pgfscope}%
\begin{pgfscope}%
\pgfpathrectangle{\pgfqpoint{0.100000in}{0.220728in}}{\pgfqpoint{3.696000in}{3.696000in}}%
\pgfusepath{clip}%
\pgfsetbuttcap%
\pgfsetroundjoin%
\definecolor{currentfill}{rgb}{0.121569,0.466667,0.705882}%
\pgfsetfillcolor{currentfill}%
\pgfsetfillopacity{0.650125}%
\pgfsetlinewidth{1.003750pt}%
\definecolor{currentstroke}{rgb}{0.121569,0.466667,0.705882}%
\pgfsetstrokecolor{currentstroke}%
\pgfsetstrokeopacity{0.650125}%
\pgfsetdash{}{0pt}%
\pgfpathmoveto{\pgfqpoint{0.859232in}{1.445088in}}%
\pgfpathcurveto{\pgfqpoint{0.867469in}{1.445088in}}{\pgfqpoint{0.875369in}{1.448360in}}{\pgfqpoint{0.881193in}{1.454184in}}%
\pgfpathcurveto{\pgfqpoint{0.887017in}{1.460008in}}{\pgfqpoint{0.890289in}{1.467908in}}{\pgfqpoint{0.890289in}{1.476144in}}%
\pgfpathcurveto{\pgfqpoint{0.890289in}{1.484381in}}{\pgfqpoint{0.887017in}{1.492281in}}{\pgfqpoint{0.881193in}{1.498105in}}%
\pgfpathcurveto{\pgfqpoint{0.875369in}{1.503929in}}{\pgfqpoint{0.867469in}{1.507201in}}{\pgfqpoint{0.859232in}{1.507201in}}%
\pgfpathcurveto{\pgfqpoint{0.850996in}{1.507201in}}{\pgfqpoint{0.843096in}{1.503929in}}{\pgfqpoint{0.837272in}{1.498105in}}%
\pgfpathcurveto{\pgfqpoint{0.831448in}{1.492281in}}{\pgfqpoint{0.828176in}{1.484381in}}{\pgfqpoint{0.828176in}{1.476144in}}%
\pgfpathcurveto{\pgfqpoint{0.828176in}{1.467908in}}{\pgfqpoint{0.831448in}{1.460008in}}{\pgfqpoint{0.837272in}{1.454184in}}%
\pgfpathcurveto{\pgfqpoint{0.843096in}{1.448360in}}{\pgfqpoint{0.850996in}{1.445088in}}{\pgfqpoint{0.859232in}{1.445088in}}%
\pgfpathclose%
\pgfusepath{stroke,fill}%
\end{pgfscope}%
\begin{pgfscope}%
\pgfpathrectangle{\pgfqpoint{0.100000in}{0.220728in}}{\pgfqpoint{3.696000in}{3.696000in}}%
\pgfusepath{clip}%
\pgfsetbuttcap%
\pgfsetroundjoin%
\definecolor{currentfill}{rgb}{0.121569,0.466667,0.705882}%
\pgfsetfillcolor{currentfill}%
\pgfsetfillopacity{0.650857}%
\pgfsetlinewidth{1.003750pt}%
\definecolor{currentstroke}{rgb}{0.121569,0.466667,0.705882}%
\pgfsetstrokecolor{currentstroke}%
\pgfsetstrokeopacity{0.650857}%
\pgfsetdash{}{0pt}%
\pgfpathmoveto{\pgfqpoint{0.876546in}{1.418139in}}%
\pgfpathcurveto{\pgfqpoint{0.884783in}{1.418139in}}{\pgfqpoint{0.892683in}{1.421411in}}{\pgfqpoint{0.898507in}{1.427235in}}%
\pgfpathcurveto{\pgfqpoint{0.904331in}{1.433059in}}{\pgfqpoint{0.907603in}{1.440959in}}{\pgfqpoint{0.907603in}{1.449196in}}%
\pgfpathcurveto{\pgfqpoint{0.907603in}{1.457432in}}{\pgfqpoint{0.904331in}{1.465332in}}{\pgfqpoint{0.898507in}{1.471156in}}%
\pgfpathcurveto{\pgfqpoint{0.892683in}{1.476980in}}{\pgfqpoint{0.884783in}{1.480252in}}{\pgfqpoint{0.876546in}{1.480252in}}%
\pgfpathcurveto{\pgfqpoint{0.868310in}{1.480252in}}{\pgfqpoint{0.860410in}{1.476980in}}{\pgfqpoint{0.854586in}{1.471156in}}%
\pgfpathcurveto{\pgfqpoint{0.848762in}{1.465332in}}{\pgfqpoint{0.845490in}{1.457432in}}{\pgfqpoint{0.845490in}{1.449196in}}%
\pgfpathcurveto{\pgfqpoint{0.845490in}{1.440959in}}{\pgfqpoint{0.848762in}{1.433059in}}{\pgfqpoint{0.854586in}{1.427235in}}%
\pgfpathcurveto{\pgfqpoint{0.860410in}{1.421411in}}{\pgfqpoint{0.868310in}{1.418139in}}{\pgfqpoint{0.876546in}{1.418139in}}%
\pgfpathclose%
\pgfusepath{stroke,fill}%
\end{pgfscope}%
\begin{pgfscope}%
\pgfpathrectangle{\pgfqpoint{0.100000in}{0.220728in}}{\pgfqpoint{3.696000in}{3.696000in}}%
\pgfusepath{clip}%
\pgfsetbuttcap%
\pgfsetroundjoin%
\definecolor{currentfill}{rgb}{0.121569,0.466667,0.705882}%
\pgfsetfillcolor{currentfill}%
\pgfsetfillopacity{0.651230}%
\pgfsetlinewidth{1.003750pt}%
\definecolor{currentstroke}{rgb}{0.121569,0.466667,0.705882}%
\pgfsetstrokecolor{currentstroke}%
\pgfsetstrokeopacity{0.651230}%
\pgfsetdash{}{0pt}%
\pgfpathmoveto{\pgfqpoint{0.855668in}{1.443742in}}%
\pgfpathcurveto{\pgfqpoint{0.863904in}{1.443742in}}{\pgfqpoint{0.871804in}{1.447014in}}{\pgfqpoint{0.877628in}{1.452838in}}%
\pgfpathcurveto{\pgfqpoint{0.883452in}{1.458662in}}{\pgfqpoint{0.886724in}{1.466562in}}{\pgfqpoint{0.886724in}{1.474798in}}%
\pgfpathcurveto{\pgfqpoint{0.886724in}{1.483035in}}{\pgfqpoint{0.883452in}{1.490935in}}{\pgfqpoint{0.877628in}{1.496759in}}%
\pgfpathcurveto{\pgfqpoint{0.871804in}{1.502583in}}{\pgfqpoint{0.863904in}{1.505855in}}{\pgfqpoint{0.855668in}{1.505855in}}%
\pgfpathcurveto{\pgfqpoint{0.847431in}{1.505855in}}{\pgfqpoint{0.839531in}{1.502583in}}{\pgfqpoint{0.833708in}{1.496759in}}%
\pgfpathcurveto{\pgfqpoint{0.827884in}{1.490935in}}{\pgfqpoint{0.824611in}{1.483035in}}{\pgfqpoint{0.824611in}{1.474798in}}%
\pgfpathcurveto{\pgfqpoint{0.824611in}{1.466562in}}{\pgfqpoint{0.827884in}{1.458662in}}{\pgfqpoint{0.833708in}{1.452838in}}%
\pgfpathcurveto{\pgfqpoint{0.839531in}{1.447014in}}{\pgfqpoint{0.847431in}{1.443742in}}{\pgfqpoint{0.855668in}{1.443742in}}%
\pgfpathclose%
\pgfusepath{stroke,fill}%
\end{pgfscope}%
\begin{pgfscope}%
\pgfpathrectangle{\pgfqpoint{0.100000in}{0.220728in}}{\pgfqpoint{3.696000in}{3.696000in}}%
\pgfusepath{clip}%
\pgfsetbuttcap%
\pgfsetroundjoin%
\definecolor{currentfill}{rgb}{0.121569,0.466667,0.705882}%
\pgfsetfillcolor{currentfill}%
\pgfsetfillopacity{0.652170}%
\pgfsetlinewidth{1.003750pt}%
\definecolor{currentstroke}{rgb}{0.121569,0.466667,0.705882}%
\pgfsetstrokecolor{currentstroke}%
\pgfsetstrokeopacity{0.652170}%
\pgfsetdash{}{0pt}%
\pgfpathmoveto{\pgfqpoint{0.852889in}{1.442461in}}%
\pgfpathcurveto{\pgfqpoint{0.861125in}{1.442461in}}{\pgfqpoint{0.869025in}{1.445733in}}{\pgfqpoint{0.874849in}{1.451557in}}%
\pgfpathcurveto{\pgfqpoint{0.880673in}{1.457381in}}{\pgfqpoint{0.883945in}{1.465281in}}{\pgfqpoint{0.883945in}{1.473517in}}%
\pgfpathcurveto{\pgfqpoint{0.883945in}{1.481754in}}{\pgfqpoint{0.880673in}{1.489654in}}{\pgfqpoint{0.874849in}{1.495478in}}%
\pgfpathcurveto{\pgfqpoint{0.869025in}{1.501302in}}{\pgfqpoint{0.861125in}{1.504574in}}{\pgfqpoint{0.852889in}{1.504574in}}%
\pgfpathcurveto{\pgfqpoint{0.844652in}{1.504574in}}{\pgfqpoint{0.836752in}{1.501302in}}{\pgfqpoint{0.830929in}{1.495478in}}%
\pgfpathcurveto{\pgfqpoint{0.825105in}{1.489654in}}{\pgfqpoint{0.821832in}{1.481754in}}{\pgfqpoint{0.821832in}{1.473517in}}%
\pgfpathcurveto{\pgfqpoint{0.821832in}{1.465281in}}{\pgfqpoint{0.825105in}{1.457381in}}{\pgfqpoint{0.830929in}{1.451557in}}%
\pgfpathcurveto{\pgfqpoint{0.836752in}{1.445733in}}{\pgfqpoint{0.844652in}{1.442461in}}{\pgfqpoint{0.852889in}{1.442461in}}%
\pgfpathclose%
\pgfusepath{stroke,fill}%
\end{pgfscope}%
\begin{pgfscope}%
\pgfpathrectangle{\pgfqpoint{0.100000in}{0.220728in}}{\pgfqpoint{3.696000in}{3.696000in}}%
\pgfusepath{clip}%
\pgfsetbuttcap%
\pgfsetroundjoin%
\definecolor{currentfill}{rgb}{0.121569,0.466667,0.705882}%
\pgfsetfillcolor{currentfill}%
\pgfsetfillopacity{0.652723}%
\pgfsetlinewidth{1.003750pt}%
\definecolor{currentstroke}{rgb}{0.121569,0.466667,0.705882}%
\pgfsetstrokecolor{currentstroke}%
\pgfsetstrokeopacity{0.652723}%
\pgfsetdash{}{0pt}%
\pgfpathmoveto{\pgfqpoint{0.872935in}{1.419572in}}%
\pgfpathcurveto{\pgfqpoint{0.881172in}{1.419572in}}{\pgfqpoint{0.889072in}{1.422844in}}{\pgfqpoint{0.894896in}{1.428668in}}%
\pgfpathcurveto{\pgfqpoint{0.900720in}{1.434492in}}{\pgfqpoint{0.903992in}{1.442392in}}{\pgfqpoint{0.903992in}{1.450629in}}%
\pgfpathcurveto{\pgfqpoint{0.903992in}{1.458865in}}{\pgfqpoint{0.900720in}{1.466765in}}{\pgfqpoint{0.894896in}{1.472589in}}%
\pgfpathcurveto{\pgfqpoint{0.889072in}{1.478413in}}{\pgfqpoint{0.881172in}{1.481685in}}{\pgfqpoint{0.872935in}{1.481685in}}%
\pgfpathcurveto{\pgfqpoint{0.864699in}{1.481685in}}{\pgfqpoint{0.856799in}{1.478413in}}{\pgfqpoint{0.850975in}{1.472589in}}%
\pgfpathcurveto{\pgfqpoint{0.845151in}{1.466765in}}{\pgfqpoint{0.841879in}{1.458865in}}{\pgfqpoint{0.841879in}{1.450629in}}%
\pgfpathcurveto{\pgfqpoint{0.841879in}{1.442392in}}{\pgfqpoint{0.845151in}{1.434492in}}{\pgfqpoint{0.850975in}{1.428668in}}%
\pgfpathcurveto{\pgfqpoint{0.856799in}{1.422844in}}{\pgfqpoint{0.864699in}{1.419572in}}{\pgfqpoint{0.872935in}{1.419572in}}%
\pgfpathclose%
\pgfusepath{stroke,fill}%
\end{pgfscope}%
\begin{pgfscope}%
\pgfpathrectangle{\pgfqpoint{0.100000in}{0.220728in}}{\pgfqpoint{3.696000in}{3.696000in}}%
\pgfusepath{clip}%
\pgfsetbuttcap%
\pgfsetroundjoin%
\definecolor{currentfill}{rgb}{0.121569,0.466667,0.705882}%
\pgfsetfillcolor{currentfill}%
\pgfsetfillopacity{0.652826}%
\pgfsetlinewidth{1.003750pt}%
\definecolor{currentstroke}{rgb}{0.121569,0.466667,0.705882}%
\pgfsetstrokecolor{currentstroke}%
\pgfsetstrokeopacity{0.652826}%
\pgfsetdash{}{0pt}%
\pgfpathmoveto{\pgfqpoint{0.851099in}{1.442112in}}%
\pgfpathcurveto{\pgfqpoint{0.859335in}{1.442112in}}{\pgfqpoint{0.867236in}{1.445384in}}{\pgfqpoint{0.873059in}{1.451208in}}%
\pgfpathcurveto{\pgfqpoint{0.878883in}{1.457032in}}{\pgfqpoint{0.882156in}{1.464932in}}{\pgfqpoint{0.882156in}{1.473168in}}%
\pgfpathcurveto{\pgfqpoint{0.882156in}{1.481404in}}{\pgfqpoint{0.878883in}{1.489305in}}{\pgfqpoint{0.873059in}{1.495128in}}%
\pgfpathcurveto{\pgfqpoint{0.867236in}{1.500952in}}{\pgfqpoint{0.859335in}{1.504225in}}{\pgfqpoint{0.851099in}{1.504225in}}%
\pgfpathcurveto{\pgfqpoint{0.842863in}{1.504225in}}{\pgfqpoint{0.834963in}{1.500952in}}{\pgfqpoint{0.829139in}{1.495128in}}%
\pgfpathcurveto{\pgfqpoint{0.823315in}{1.489305in}}{\pgfqpoint{0.820043in}{1.481404in}}{\pgfqpoint{0.820043in}{1.473168in}}%
\pgfpathcurveto{\pgfqpoint{0.820043in}{1.464932in}}{\pgfqpoint{0.823315in}{1.457032in}}{\pgfqpoint{0.829139in}{1.451208in}}%
\pgfpathcurveto{\pgfqpoint{0.834963in}{1.445384in}}{\pgfqpoint{0.842863in}{1.442112in}}{\pgfqpoint{0.851099in}{1.442112in}}%
\pgfpathclose%
\pgfusepath{stroke,fill}%
\end{pgfscope}%
\begin{pgfscope}%
\pgfpathrectangle{\pgfqpoint{0.100000in}{0.220728in}}{\pgfqpoint{3.696000in}{3.696000in}}%
\pgfusepath{clip}%
\pgfsetbuttcap%
\pgfsetroundjoin%
\definecolor{currentfill}{rgb}{0.121569,0.466667,0.705882}%
\pgfsetfillcolor{currentfill}%
\pgfsetfillopacity{0.652868}%
\pgfsetlinewidth{1.003750pt}%
\definecolor{currentstroke}{rgb}{0.121569,0.466667,0.705882}%
\pgfsetstrokecolor{currentstroke}%
\pgfsetstrokeopacity{0.652868}%
\pgfsetdash{}{0pt}%
\pgfpathmoveto{\pgfqpoint{2.143094in}{1.752022in}}%
\pgfpathcurveto{\pgfqpoint{2.151330in}{1.752022in}}{\pgfqpoint{2.159230in}{1.755294in}}{\pgfqpoint{2.165054in}{1.761118in}}%
\pgfpathcurveto{\pgfqpoint{2.170878in}{1.766942in}}{\pgfqpoint{2.174150in}{1.774842in}}{\pgfqpoint{2.174150in}{1.783079in}}%
\pgfpathcurveto{\pgfqpoint{2.174150in}{1.791315in}}{\pgfqpoint{2.170878in}{1.799215in}}{\pgfqpoint{2.165054in}{1.805039in}}%
\pgfpathcurveto{\pgfqpoint{2.159230in}{1.810863in}}{\pgfqpoint{2.151330in}{1.814135in}}{\pgfqpoint{2.143094in}{1.814135in}}%
\pgfpathcurveto{\pgfqpoint{2.134858in}{1.814135in}}{\pgfqpoint{2.126958in}{1.810863in}}{\pgfqpoint{2.121134in}{1.805039in}}%
\pgfpathcurveto{\pgfqpoint{2.115310in}{1.799215in}}{\pgfqpoint{2.112037in}{1.791315in}}{\pgfqpoint{2.112037in}{1.783079in}}%
\pgfpathcurveto{\pgfqpoint{2.112037in}{1.774842in}}{\pgfqpoint{2.115310in}{1.766942in}}{\pgfqpoint{2.121134in}{1.761118in}}%
\pgfpathcurveto{\pgfqpoint{2.126958in}{1.755294in}}{\pgfqpoint{2.134858in}{1.752022in}}{\pgfqpoint{2.143094in}{1.752022in}}%
\pgfpathclose%
\pgfusepath{stroke,fill}%
\end{pgfscope}%
\begin{pgfscope}%
\pgfpathrectangle{\pgfqpoint{0.100000in}{0.220728in}}{\pgfqpoint{3.696000in}{3.696000in}}%
\pgfusepath{clip}%
\pgfsetbuttcap%
\pgfsetroundjoin%
\definecolor{currentfill}{rgb}{0.121569,0.466667,0.705882}%
\pgfsetfillcolor{currentfill}%
\pgfsetfillopacity{0.653751}%
\pgfsetlinewidth{1.003750pt}%
\definecolor{currentstroke}{rgb}{0.121569,0.466667,0.705882}%
\pgfsetstrokecolor{currentstroke}%
\pgfsetstrokeopacity{0.653751}%
\pgfsetdash{}{0pt}%
\pgfpathmoveto{\pgfqpoint{0.849354in}{1.438333in}}%
\pgfpathcurveto{\pgfqpoint{0.857590in}{1.438333in}}{\pgfqpoint{0.865490in}{1.441605in}}{\pgfqpoint{0.871314in}{1.447429in}}%
\pgfpathcurveto{\pgfqpoint{0.877138in}{1.453253in}}{\pgfqpoint{0.880410in}{1.461153in}}{\pgfqpoint{0.880410in}{1.469389in}}%
\pgfpathcurveto{\pgfqpoint{0.880410in}{1.477625in}}{\pgfqpoint{0.877138in}{1.485525in}}{\pgfqpoint{0.871314in}{1.491349in}}%
\pgfpathcurveto{\pgfqpoint{0.865490in}{1.497173in}}{\pgfqpoint{0.857590in}{1.500446in}}{\pgfqpoint{0.849354in}{1.500446in}}%
\pgfpathcurveto{\pgfqpoint{0.841117in}{1.500446in}}{\pgfqpoint{0.833217in}{1.497173in}}{\pgfqpoint{0.827393in}{1.491349in}}%
\pgfpathcurveto{\pgfqpoint{0.821569in}{1.485525in}}{\pgfqpoint{0.818297in}{1.477625in}}{\pgfqpoint{0.818297in}{1.469389in}}%
\pgfpathcurveto{\pgfqpoint{0.818297in}{1.461153in}}{\pgfqpoint{0.821569in}{1.453253in}}{\pgfqpoint{0.827393in}{1.447429in}}%
\pgfpathcurveto{\pgfqpoint{0.833217in}{1.441605in}}{\pgfqpoint{0.841117in}{1.438333in}}{\pgfqpoint{0.849354in}{1.438333in}}%
\pgfpathclose%
\pgfusepath{stroke,fill}%
\end{pgfscope}%
\begin{pgfscope}%
\pgfpathrectangle{\pgfqpoint{0.100000in}{0.220728in}}{\pgfqpoint{3.696000in}{3.696000in}}%
\pgfusepath{clip}%
\pgfsetbuttcap%
\pgfsetroundjoin%
\definecolor{currentfill}{rgb}{0.121569,0.466667,0.705882}%
\pgfsetfillcolor{currentfill}%
\pgfsetfillopacity{0.654773}%
\pgfsetlinewidth{1.003750pt}%
\definecolor{currentstroke}{rgb}{0.121569,0.466667,0.705882}%
\pgfsetstrokecolor{currentstroke}%
\pgfsetstrokeopacity{0.654773}%
\pgfsetdash{}{0pt}%
\pgfpathmoveto{\pgfqpoint{0.869345in}{1.420637in}}%
\pgfpathcurveto{\pgfqpoint{0.877581in}{1.420637in}}{\pgfqpoint{0.885481in}{1.423909in}}{\pgfqpoint{0.891305in}{1.429733in}}%
\pgfpathcurveto{\pgfqpoint{0.897129in}{1.435557in}}{\pgfqpoint{0.900401in}{1.443457in}}{\pgfqpoint{0.900401in}{1.451694in}}%
\pgfpathcurveto{\pgfqpoint{0.900401in}{1.459930in}}{\pgfqpoint{0.897129in}{1.467830in}}{\pgfqpoint{0.891305in}{1.473654in}}%
\pgfpathcurveto{\pgfqpoint{0.885481in}{1.479478in}}{\pgfqpoint{0.877581in}{1.482750in}}{\pgfqpoint{0.869345in}{1.482750in}}%
\pgfpathcurveto{\pgfqpoint{0.861109in}{1.482750in}}{\pgfqpoint{0.853209in}{1.479478in}}{\pgfqpoint{0.847385in}{1.473654in}}%
\pgfpathcurveto{\pgfqpoint{0.841561in}{1.467830in}}{\pgfqpoint{0.838288in}{1.459930in}}{\pgfqpoint{0.838288in}{1.451694in}}%
\pgfpathcurveto{\pgfqpoint{0.838288in}{1.443457in}}{\pgfqpoint{0.841561in}{1.435557in}}{\pgfqpoint{0.847385in}{1.429733in}}%
\pgfpathcurveto{\pgfqpoint{0.853209in}{1.423909in}}{\pgfqpoint{0.861109in}{1.420637in}}{\pgfqpoint{0.869345in}{1.420637in}}%
\pgfpathclose%
\pgfusepath{stroke,fill}%
\end{pgfscope}%
\begin{pgfscope}%
\pgfpathrectangle{\pgfqpoint{0.100000in}{0.220728in}}{\pgfqpoint{3.696000in}{3.696000in}}%
\pgfusepath{clip}%
\pgfsetbuttcap%
\pgfsetroundjoin%
\definecolor{currentfill}{rgb}{0.121569,0.466667,0.705882}%
\pgfsetfillcolor{currentfill}%
\pgfsetfillopacity{0.655513}%
\pgfsetlinewidth{1.003750pt}%
\definecolor{currentstroke}{rgb}{0.121569,0.466667,0.705882}%
\pgfsetstrokecolor{currentstroke}%
\pgfsetstrokeopacity{0.655513}%
\pgfsetdash{}{0pt}%
\pgfpathmoveto{\pgfqpoint{0.848760in}{1.431432in}}%
\pgfpathcurveto{\pgfqpoint{0.856997in}{1.431432in}}{\pgfqpoint{0.864897in}{1.434705in}}{\pgfqpoint{0.870721in}{1.440529in}}%
\pgfpathcurveto{\pgfqpoint{0.876545in}{1.446352in}}{\pgfqpoint{0.879817in}{1.454252in}}{\pgfqpoint{0.879817in}{1.462489in}}%
\pgfpathcurveto{\pgfqpoint{0.879817in}{1.470725in}}{\pgfqpoint{0.876545in}{1.478625in}}{\pgfqpoint{0.870721in}{1.484449in}}%
\pgfpathcurveto{\pgfqpoint{0.864897in}{1.490273in}}{\pgfqpoint{0.856997in}{1.493545in}}{\pgfqpoint{0.848760in}{1.493545in}}%
\pgfpathcurveto{\pgfqpoint{0.840524in}{1.493545in}}{\pgfqpoint{0.832624in}{1.490273in}}{\pgfqpoint{0.826800in}{1.484449in}}%
\pgfpathcurveto{\pgfqpoint{0.820976in}{1.478625in}}{\pgfqpoint{0.817704in}{1.470725in}}{\pgfqpoint{0.817704in}{1.462489in}}%
\pgfpathcurveto{\pgfqpoint{0.817704in}{1.454252in}}{\pgfqpoint{0.820976in}{1.446352in}}{\pgfqpoint{0.826800in}{1.440529in}}%
\pgfpathcurveto{\pgfqpoint{0.832624in}{1.434705in}}{\pgfqpoint{0.840524in}{1.431432in}}{\pgfqpoint{0.848760in}{1.431432in}}%
\pgfpathclose%
\pgfusepath{stroke,fill}%
\end{pgfscope}%
\begin{pgfscope}%
\pgfpathrectangle{\pgfqpoint{0.100000in}{0.220728in}}{\pgfqpoint{3.696000in}{3.696000in}}%
\pgfusepath{clip}%
\pgfsetbuttcap%
\pgfsetroundjoin%
\definecolor{currentfill}{rgb}{0.121569,0.466667,0.705882}%
\pgfsetfillcolor{currentfill}%
\pgfsetfillopacity{0.655917}%
\pgfsetlinewidth{1.003750pt}%
\definecolor{currentstroke}{rgb}{0.121569,0.466667,0.705882}%
\pgfsetstrokecolor{currentstroke}%
\pgfsetstrokeopacity{0.655917}%
\pgfsetdash{}{0pt}%
\pgfpathmoveto{\pgfqpoint{0.867054in}{1.421698in}}%
\pgfpathcurveto{\pgfqpoint{0.875290in}{1.421698in}}{\pgfqpoint{0.883190in}{1.424971in}}{\pgfqpoint{0.889014in}{1.430795in}}%
\pgfpathcurveto{\pgfqpoint{0.894838in}{1.436618in}}{\pgfqpoint{0.898110in}{1.444519in}}{\pgfqpoint{0.898110in}{1.452755in}}%
\pgfpathcurveto{\pgfqpoint{0.898110in}{1.460991in}}{\pgfqpoint{0.894838in}{1.468891in}}{\pgfqpoint{0.889014in}{1.474715in}}%
\pgfpathcurveto{\pgfqpoint{0.883190in}{1.480539in}}{\pgfqpoint{0.875290in}{1.483811in}}{\pgfqpoint{0.867054in}{1.483811in}}%
\pgfpathcurveto{\pgfqpoint{0.858817in}{1.483811in}}{\pgfqpoint{0.850917in}{1.480539in}}{\pgfqpoint{0.845093in}{1.474715in}}%
\pgfpathcurveto{\pgfqpoint{0.839270in}{1.468891in}}{\pgfqpoint{0.835997in}{1.460991in}}{\pgfqpoint{0.835997in}{1.452755in}}%
\pgfpathcurveto{\pgfqpoint{0.835997in}{1.444519in}}{\pgfqpoint{0.839270in}{1.436618in}}{\pgfqpoint{0.845093in}{1.430795in}}%
\pgfpathcurveto{\pgfqpoint{0.850917in}{1.424971in}}{\pgfqpoint{0.858817in}{1.421698in}}{\pgfqpoint{0.867054in}{1.421698in}}%
\pgfpathclose%
\pgfusepath{stroke,fill}%
\end{pgfscope}%
\begin{pgfscope}%
\pgfpathrectangle{\pgfqpoint{0.100000in}{0.220728in}}{\pgfqpoint{3.696000in}{3.696000in}}%
\pgfusepath{clip}%
\pgfsetbuttcap%
\pgfsetroundjoin%
\definecolor{currentfill}{rgb}{0.121569,0.466667,0.705882}%
\pgfsetfillcolor{currentfill}%
\pgfsetfillopacity{0.656559}%
\pgfsetlinewidth{1.003750pt}%
\definecolor{currentstroke}{rgb}{0.121569,0.466667,0.705882}%
\pgfsetstrokecolor{currentstroke}%
\pgfsetstrokeopacity{0.656559}%
\pgfsetdash{}{0pt}%
\pgfpathmoveto{\pgfqpoint{0.866012in}{1.422107in}}%
\pgfpathcurveto{\pgfqpoint{0.874248in}{1.422107in}}{\pgfqpoint{0.882148in}{1.425379in}}{\pgfqpoint{0.887972in}{1.431203in}}%
\pgfpathcurveto{\pgfqpoint{0.893796in}{1.437027in}}{\pgfqpoint{0.897069in}{1.444927in}}{\pgfqpoint{0.897069in}{1.453163in}}%
\pgfpathcurveto{\pgfqpoint{0.897069in}{1.461400in}}{\pgfqpoint{0.893796in}{1.469300in}}{\pgfqpoint{0.887972in}{1.475124in}}%
\pgfpathcurveto{\pgfqpoint{0.882148in}{1.480948in}}{\pgfqpoint{0.874248in}{1.484220in}}{\pgfqpoint{0.866012in}{1.484220in}}%
\pgfpathcurveto{\pgfqpoint{0.857776in}{1.484220in}}{\pgfqpoint{0.849876in}{1.480948in}}{\pgfqpoint{0.844052in}{1.475124in}}%
\pgfpathcurveto{\pgfqpoint{0.838228in}{1.469300in}}{\pgfqpoint{0.834956in}{1.461400in}}{\pgfqpoint{0.834956in}{1.453163in}}%
\pgfpathcurveto{\pgfqpoint{0.834956in}{1.444927in}}{\pgfqpoint{0.838228in}{1.437027in}}{\pgfqpoint{0.844052in}{1.431203in}}%
\pgfpathcurveto{\pgfqpoint{0.849876in}{1.425379in}}{\pgfqpoint{0.857776in}{1.422107in}}{\pgfqpoint{0.866012in}{1.422107in}}%
\pgfpathclose%
\pgfusepath{stroke,fill}%
\end{pgfscope}%
\begin{pgfscope}%
\pgfpathrectangle{\pgfqpoint{0.100000in}{0.220728in}}{\pgfqpoint{3.696000in}{3.696000in}}%
\pgfusepath{clip}%
\pgfsetbuttcap%
\pgfsetroundjoin%
\definecolor{currentfill}{rgb}{0.121569,0.466667,0.705882}%
\pgfsetfillcolor{currentfill}%
\pgfsetfillopacity{0.656695}%
\pgfsetlinewidth{1.003750pt}%
\definecolor{currentstroke}{rgb}{0.121569,0.466667,0.705882}%
\pgfsetstrokecolor{currentstroke}%
\pgfsetstrokeopacity{0.656695}%
\pgfsetdash{}{0pt}%
\pgfpathmoveto{\pgfqpoint{0.851847in}{1.424705in}}%
\pgfpathcurveto{\pgfqpoint{0.860083in}{1.424705in}}{\pgfqpoint{0.867983in}{1.427977in}}{\pgfqpoint{0.873807in}{1.433801in}}%
\pgfpathcurveto{\pgfqpoint{0.879631in}{1.439625in}}{\pgfqpoint{0.882903in}{1.447525in}}{\pgfqpoint{0.882903in}{1.455761in}}%
\pgfpathcurveto{\pgfqpoint{0.882903in}{1.463998in}}{\pgfqpoint{0.879631in}{1.471898in}}{\pgfqpoint{0.873807in}{1.477722in}}%
\pgfpathcurveto{\pgfqpoint{0.867983in}{1.483546in}}{\pgfqpoint{0.860083in}{1.486818in}}{\pgfqpoint{0.851847in}{1.486818in}}%
\pgfpathcurveto{\pgfqpoint{0.843610in}{1.486818in}}{\pgfqpoint{0.835710in}{1.483546in}}{\pgfqpoint{0.829886in}{1.477722in}}%
\pgfpathcurveto{\pgfqpoint{0.824063in}{1.471898in}}{\pgfqpoint{0.820790in}{1.463998in}}{\pgfqpoint{0.820790in}{1.455761in}}%
\pgfpathcurveto{\pgfqpoint{0.820790in}{1.447525in}}{\pgfqpoint{0.824063in}{1.439625in}}{\pgfqpoint{0.829886in}{1.433801in}}%
\pgfpathcurveto{\pgfqpoint{0.835710in}{1.427977in}}{\pgfqpoint{0.843610in}{1.424705in}}{\pgfqpoint{0.851847in}{1.424705in}}%
\pgfpathclose%
\pgfusepath{stroke,fill}%
\end{pgfscope}%
\begin{pgfscope}%
\pgfpathrectangle{\pgfqpoint{0.100000in}{0.220728in}}{\pgfqpoint{3.696000in}{3.696000in}}%
\pgfusepath{clip}%
\pgfsetbuttcap%
\pgfsetroundjoin%
\definecolor{currentfill}{rgb}{0.121569,0.466667,0.705882}%
\pgfsetfillcolor{currentfill}%
\pgfsetfillopacity{0.656913}%
\pgfsetlinewidth{1.003750pt}%
\definecolor{currentstroke}{rgb}{0.121569,0.466667,0.705882}%
\pgfsetstrokecolor{currentstroke}%
\pgfsetstrokeopacity{0.656913}%
\pgfsetdash{}{0pt}%
\pgfpathmoveto{\pgfqpoint{0.865352in}{1.422428in}}%
\pgfpathcurveto{\pgfqpoint{0.873589in}{1.422428in}}{\pgfqpoint{0.881489in}{1.425701in}}{\pgfqpoint{0.887313in}{1.431525in}}%
\pgfpathcurveto{\pgfqpoint{0.893136in}{1.437349in}}{\pgfqpoint{0.896409in}{1.445249in}}{\pgfqpoint{0.896409in}{1.453485in}}%
\pgfpathcurveto{\pgfqpoint{0.896409in}{1.461721in}}{\pgfqpoint{0.893136in}{1.469621in}}{\pgfqpoint{0.887313in}{1.475445in}}%
\pgfpathcurveto{\pgfqpoint{0.881489in}{1.481269in}}{\pgfqpoint{0.873589in}{1.484541in}}{\pgfqpoint{0.865352in}{1.484541in}}%
\pgfpathcurveto{\pgfqpoint{0.857116in}{1.484541in}}{\pgfqpoint{0.849216in}{1.481269in}}{\pgfqpoint{0.843392in}{1.475445in}}%
\pgfpathcurveto{\pgfqpoint{0.837568in}{1.469621in}}{\pgfqpoint{0.834296in}{1.461721in}}{\pgfqpoint{0.834296in}{1.453485in}}%
\pgfpathcurveto{\pgfqpoint{0.834296in}{1.445249in}}{\pgfqpoint{0.837568in}{1.437349in}}{\pgfqpoint{0.843392in}{1.431525in}}%
\pgfpathcurveto{\pgfqpoint{0.849216in}{1.425701in}}{\pgfqpoint{0.857116in}{1.422428in}}{\pgfqpoint{0.865352in}{1.422428in}}%
\pgfpathclose%
\pgfusepath{stroke,fill}%
\end{pgfscope}%
\begin{pgfscope}%
\pgfpathrectangle{\pgfqpoint{0.100000in}{0.220728in}}{\pgfqpoint{3.696000in}{3.696000in}}%
\pgfusepath{clip}%
\pgfsetbuttcap%
\pgfsetroundjoin%
\definecolor{currentfill}{rgb}{0.121569,0.466667,0.705882}%
\pgfsetfillcolor{currentfill}%
\pgfsetfillopacity{0.657099}%
\pgfsetlinewidth{1.003750pt}%
\definecolor{currentstroke}{rgb}{0.121569,0.466667,0.705882}%
\pgfsetstrokecolor{currentstroke}%
\pgfsetstrokeopacity{0.657099}%
\pgfsetdash{}{0pt}%
\pgfpathmoveto{\pgfqpoint{0.865009in}{1.422535in}}%
\pgfpathcurveto{\pgfqpoint{0.873245in}{1.422535in}}{\pgfqpoint{0.881145in}{1.425808in}}{\pgfqpoint{0.886969in}{1.431632in}}%
\pgfpathcurveto{\pgfqpoint{0.892793in}{1.437456in}}{\pgfqpoint{0.896065in}{1.445356in}}{\pgfqpoint{0.896065in}{1.453592in}}%
\pgfpathcurveto{\pgfqpoint{0.896065in}{1.461828in}}{\pgfqpoint{0.892793in}{1.469728in}}{\pgfqpoint{0.886969in}{1.475552in}}%
\pgfpathcurveto{\pgfqpoint{0.881145in}{1.481376in}}{\pgfqpoint{0.873245in}{1.484648in}}{\pgfqpoint{0.865009in}{1.484648in}}%
\pgfpathcurveto{\pgfqpoint{0.856773in}{1.484648in}}{\pgfqpoint{0.848873in}{1.481376in}}{\pgfqpoint{0.843049in}{1.475552in}}%
\pgfpathcurveto{\pgfqpoint{0.837225in}{1.469728in}}{\pgfqpoint{0.833952in}{1.461828in}}{\pgfqpoint{0.833952in}{1.453592in}}%
\pgfpathcurveto{\pgfqpoint{0.833952in}{1.445356in}}{\pgfqpoint{0.837225in}{1.437456in}}{\pgfqpoint{0.843049in}{1.431632in}}%
\pgfpathcurveto{\pgfqpoint{0.848873in}{1.425808in}}{\pgfqpoint{0.856773in}{1.422535in}}{\pgfqpoint{0.865009in}{1.422535in}}%
\pgfpathclose%
\pgfusepath{stroke,fill}%
\end{pgfscope}%
\begin{pgfscope}%
\pgfpathrectangle{\pgfqpoint{0.100000in}{0.220728in}}{\pgfqpoint{3.696000in}{3.696000in}}%
\pgfusepath{clip}%
\pgfsetbuttcap%
\pgfsetroundjoin%
\definecolor{currentfill}{rgb}{0.121569,0.466667,0.705882}%
\pgfsetfillcolor{currentfill}%
\pgfsetfillopacity{0.657209}%
\pgfsetlinewidth{1.003750pt}%
\definecolor{currentstroke}{rgb}{0.121569,0.466667,0.705882}%
\pgfsetstrokecolor{currentstroke}%
\pgfsetstrokeopacity{0.657209}%
\pgfsetdash{}{0pt}%
\pgfpathmoveto{\pgfqpoint{0.864820in}{1.422639in}}%
\pgfpathcurveto{\pgfqpoint{0.873056in}{1.422639in}}{\pgfqpoint{0.880956in}{1.425911in}}{\pgfqpoint{0.886780in}{1.431735in}}%
\pgfpathcurveto{\pgfqpoint{0.892604in}{1.437559in}}{\pgfqpoint{0.895876in}{1.445459in}}{\pgfqpoint{0.895876in}{1.453695in}}%
\pgfpathcurveto{\pgfqpoint{0.895876in}{1.461932in}}{\pgfqpoint{0.892604in}{1.469832in}}{\pgfqpoint{0.886780in}{1.475656in}}%
\pgfpathcurveto{\pgfqpoint{0.880956in}{1.481480in}}{\pgfqpoint{0.873056in}{1.484752in}}{\pgfqpoint{0.864820in}{1.484752in}}%
\pgfpathcurveto{\pgfqpoint{0.856583in}{1.484752in}}{\pgfqpoint{0.848683in}{1.481480in}}{\pgfqpoint{0.842859in}{1.475656in}}%
\pgfpathcurveto{\pgfqpoint{0.837035in}{1.469832in}}{\pgfqpoint{0.833763in}{1.461932in}}{\pgfqpoint{0.833763in}{1.453695in}}%
\pgfpathcurveto{\pgfqpoint{0.833763in}{1.445459in}}{\pgfqpoint{0.837035in}{1.437559in}}{\pgfqpoint{0.842859in}{1.431735in}}%
\pgfpathcurveto{\pgfqpoint{0.848683in}{1.425911in}}{\pgfqpoint{0.856583in}{1.422639in}}{\pgfqpoint{0.864820in}{1.422639in}}%
\pgfpathclose%
\pgfusepath{stroke,fill}%
\end{pgfscope}%
\begin{pgfscope}%
\pgfpathrectangle{\pgfqpoint{0.100000in}{0.220728in}}{\pgfqpoint{3.696000in}{3.696000in}}%
\pgfusepath{clip}%
\pgfsetbuttcap%
\pgfsetroundjoin%
\definecolor{currentfill}{rgb}{0.121569,0.466667,0.705882}%
\pgfsetfillcolor{currentfill}%
\pgfsetfillopacity{0.657267}%
\pgfsetlinewidth{1.003750pt}%
\definecolor{currentstroke}{rgb}{0.121569,0.466667,0.705882}%
\pgfsetstrokecolor{currentstroke}%
\pgfsetstrokeopacity{0.657267}%
\pgfsetdash{}{0pt}%
\pgfpathmoveto{\pgfqpoint{0.864719in}{1.422675in}}%
\pgfpathcurveto{\pgfqpoint{0.872955in}{1.422675in}}{\pgfqpoint{0.880855in}{1.425948in}}{\pgfqpoint{0.886679in}{1.431772in}}%
\pgfpathcurveto{\pgfqpoint{0.892503in}{1.437596in}}{\pgfqpoint{0.895775in}{1.445496in}}{\pgfqpoint{0.895775in}{1.453732in}}%
\pgfpathcurveto{\pgfqpoint{0.895775in}{1.461968in}}{\pgfqpoint{0.892503in}{1.469868in}}{\pgfqpoint{0.886679in}{1.475692in}}%
\pgfpathcurveto{\pgfqpoint{0.880855in}{1.481516in}}{\pgfqpoint{0.872955in}{1.484788in}}{\pgfqpoint{0.864719in}{1.484788in}}%
\pgfpathcurveto{\pgfqpoint{0.856483in}{1.484788in}}{\pgfqpoint{0.848583in}{1.481516in}}{\pgfqpoint{0.842759in}{1.475692in}}%
\pgfpathcurveto{\pgfqpoint{0.836935in}{1.469868in}}{\pgfqpoint{0.833662in}{1.461968in}}{\pgfqpoint{0.833662in}{1.453732in}}%
\pgfpathcurveto{\pgfqpoint{0.833662in}{1.445496in}}{\pgfqpoint{0.836935in}{1.437596in}}{\pgfqpoint{0.842759in}{1.431772in}}%
\pgfpathcurveto{\pgfqpoint{0.848583in}{1.425948in}}{\pgfqpoint{0.856483in}{1.422675in}}{\pgfqpoint{0.864719in}{1.422675in}}%
\pgfpathclose%
\pgfusepath{stroke,fill}%
\end{pgfscope}%
\begin{pgfscope}%
\pgfpathrectangle{\pgfqpoint{0.100000in}{0.220728in}}{\pgfqpoint{3.696000in}{3.696000in}}%
\pgfusepath{clip}%
\pgfsetbuttcap%
\pgfsetroundjoin%
\definecolor{currentfill}{rgb}{0.121569,0.466667,0.705882}%
\pgfsetfillcolor{currentfill}%
\pgfsetfillopacity{0.657300}%
\pgfsetlinewidth{1.003750pt}%
\definecolor{currentstroke}{rgb}{0.121569,0.466667,0.705882}%
\pgfsetstrokecolor{currentstroke}%
\pgfsetstrokeopacity{0.657300}%
\pgfsetdash{}{0pt}%
\pgfpathmoveto{\pgfqpoint{0.864659in}{1.422706in}}%
\pgfpathcurveto{\pgfqpoint{0.872895in}{1.422706in}}{\pgfqpoint{0.880795in}{1.425979in}}{\pgfqpoint{0.886619in}{1.431802in}}%
\pgfpathcurveto{\pgfqpoint{0.892443in}{1.437626in}}{\pgfqpoint{0.895715in}{1.445526in}}{\pgfqpoint{0.895715in}{1.453763in}}%
\pgfpathcurveto{\pgfqpoint{0.895715in}{1.461999in}}{\pgfqpoint{0.892443in}{1.469899in}}{\pgfqpoint{0.886619in}{1.475723in}}%
\pgfpathcurveto{\pgfqpoint{0.880795in}{1.481547in}}{\pgfqpoint{0.872895in}{1.484819in}}{\pgfqpoint{0.864659in}{1.484819in}}%
\pgfpathcurveto{\pgfqpoint{0.856422in}{1.484819in}}{\pgfqpoint{0.848522in}{1.481547in}}{\pgfqpoint{0.842698in}{1.475723in}}%
\pgfpathcurveto{\pgfqpoint{0.836874in}{1.469899in}}{\pgfqpoint{0.833602in}{1.461999in}}{\pgfqpoint{0.833602in}{1.453763in}}%
\pgfpathcurveto{\pgfqpoint{0.833602in}{1.445526in}}{\pgfqpoint{0.836874in}{1.437626in}}{\pgfqpoint{0.842698in}{1.431802in}}%
\pgfpathcurveto{\pgfqpoint{0.848522in}{1.425979in}}{\pgfqpoint{0.856422in}{1.422706in}}{\pgfqpoint{0.864659in}{1.422706in}}%
\pgfpathclose%
\pgfusepath{stroke,fill}%
\end{pgfscope}%
\begin{pgfscope}%
\pgfpathrectangle{\pgfqpoint{0.100000in}{0.220728in}}{\pgfqpoint{3.696000in}{3.696000in}}%
\pgfusepath{clip}%
\pgfsetbuttcap%
\pgfsetroundjoin%
\definecolor{currentfill}{rgb}{0.121569,0.466667,0.705882}%
\pgfsetfillcolor{currentfill}%
\pgfsetfillopacity{0.657318}%
\pgfsetlinewidth{1.003750pt}%
\definecolor{currentstroke}{rgb}{0.121569,0.466667,0.705882}%
\pgfsetstrokecolor{currentstroke}%
\pgfsetstrokeopacity{0.657318}%
\pgfsetdash{}{0pt}%
\pgfpathmoveto{\pgfqpoint{0.864628in}{1.422722in}}%
\pgfpathcurveto{\pgfqpoint{0.872865in}{1.422722in}}{\pgfqpoint{0.880765in}{1.425995in}}{\pgfqpoint{0.886589in}{1.431819in}}%
\pgfpathcurveto{\pgfqpoint{0.892412in}{1.437642in}}{\pgfqpoint{0.895685in}{1.445542in}}{\pgfqpoint{0.895685in}{1.453779in}}%
\pgfpathcurveto{\pgfqpoint{0.895685in}{1.462015in}}{\pgfqpoint{0.892412in}{1.469915in}}{\pgfqpoint{0.886589in}{1.475739in}}%
\pgfpathcurveto{\pgfqpoint{0.880765in}{1.481563in}}{\pgfqpoint{0.872865in}{1.484835in}}{\pgfqpoint{0.864628in}{1.484835in}}%
\pgfpathcurveto{\pgfqpoint{0.856392in}{1.484835in}}{\pgfqpoint{0.848492in}{1.481563in}}{\pgfqpoint{0.842668in}{1.475739in}}%
\pgfpathcurveto{\pgfqpoint{0.836844in}{1.469915in}}{\pgfqpoint{0.833572in}{1.462015in}}{\pgfqpoint{0.833572in}{1.453779in}}%
\pgfpathcurveto{\pgfqpoint{0.833572in}{1.445542in}}{\pgfqpoint{0.836844in}{1.437642in}}{\pgfqpoint{0.842668in}{1.431819in}}%
\pgfpathcurveto{\pgfqpoint{0.848492in}{1.425995in}}{\pgfqpoint{0.856392in}{1.422722in}}{\pgfqpoint{0.864628in}{1.422722in}}%
\pgfpathclose%
\pgfusepath{stroke,fill}%
\end{pgfscope}%
\begin{pgfscope}%
\pgfpathrectangle{\pgfqpoint{0.100000in}{0.220728in}}{\pgfqpoint{3.696000in}{3.696000in}}%
\pgfusepath{clip}%
\pgfsetbuttcap%
\pgfsetroundjoin%
\definecolor{currentfill}{rgb}{0.121569,0.466667,0.705882}%
\pgfsetfillcolor{currentfill}%
\pgfsetfillopacity{0.657328}%
\pgfsetlinewidth{1.003750pt}%
\definecolor{currentstroke}{rgb}{0.121569,0.466667,0.705882}%
\pgfsetstrokecolor{currentstroke}%
\pgfsetstrokeopacity{0.657328}%
\pgfsetdash{}{0pt}%
\pgfpathmoveto{\pgfqpoint{0.864610in}{1.422733in}}%
\pgfpathcurveto{\pgfqpoint{0.872846in}{1.422733in}}{\pgfqpoint{0.880746in}{1.426006in}}{\pgfqpoint{0.886570in}{1.431830in}}%
\pgfpathcurveto{\pgfqpoint{0.892394in}{1.437654in}}{\pgfqpoint{0.895666in}{1.445554in}}{\pgfqpoint{0.895666in}{1.453790in}}%
\pgfpathcurveto{\pgfqpoint{0.895666in}{1.462026in}}{\pgfqpoint{0.892394in}{1.469926in}}{\pgfqpoint{0.886570in}{1.475750in}}%
\pgfpathcurveto{\pgfqpoint{0.880746in}{1.481574in}}{\pgfqpoint{0.872846in}{1.484846in}}{\pgfqpoint{0.864610in}{1.484846in}}%
\pgfpathcurveto{\pgfqpoint{0.856374in}{1.484846in}}{\pgfqpoint{0.848474in}{1.481574in}}{\pgfqpoint{0.842650in}{1.475750in}}%
\pgfpathcurveto{\pgfqpoint{0.836826in}{1.469926in}}{\pgfqpoint{0.833553in}{1.462026in}}{\pgfqpoint{0.833553in}{1.453790in}}%
\pgfpathcurveto{\pgfqpoint{0.833553in}{1.445554in}}{\pgfqpoint{0.836826in}{1.437654in}}{\pgfqpoint{0.842650in}{1.431830in}}%
\pgfpathcurveto{\pgfqpoint{0.848474in}{1.426006in}}{\pgfqpoint{0.856374in}{1.422733in}}{\pgfqpoint{0.864610in}{1.422733in}}%
\pgfpathclose%
\pgfusepath{stroke,fill}%
\end{pgfscope}%
\begin{pgfscope}%
\pgfpathrectangle{\pgfqpoint{0.100000in}{0.220728in}}{\pgfqpoint{3.696000in}{3.696000in}}%
\pgfusepath{clip}%
\pgfsetbuttcap%
\pgfsetroundjoin%
\definecolor{currentfill}{rgb}{0.121569,0.466667,0.705882}%
\pgfsetfillcolor{currentfill}%
\pgfsetfillopacity{0.657333}%
\pgfsetlinewidth{1.003750pt}%
\definecolor{currentstroke}{rgb}{0.121569,0.466667,0.705882}%
\pgfsetstrokecolor{currentstroke}%
\pgfsetstrokeopacity{0.657333}%
\pgfsetdash{}{0pt}%
\pgfpathmoveto{\pgfqpoint{0.864600in}{1.422737in}}%
\pgfpathcurveto{\pgfqpoint{0.872837in}{1.422737in}}{\pgfqpoint{0.880737in}{1.426010in}}{\pgfqpoint{0.886561in}{1.431834in}}%
\pgfpathcurveto{\pgfqpoint{0.892385in}{1.437658in}}{\pgfqpoint{0.895657in}{1.445558in}}{\pgfqpoint{0.895657in}{1.453794in}}%
\pgfpathcurveto{\pgfqpoint{0.895657in}{1.462030in}}{\pgfqpoint{0.892385in}{1.469930in}}{\pgfqpoint{0.886561in}{1.475754in}}%
\pgfpathcurveto{\pgfqpoint{0.880737in}{1.481578in}}{\pgfqpoint{0.872837in}{1.484850in}}{\pgfqpoint{0.864600in}{1.484850in}}%
\pgfpathcurveto{\pgfqpoint{0.856364in}{1.484850in}}{\pgfqpoint{0.848464in}{1.481578in}}{\pgfqpoint{0.842640in}{1.475754in}}%
\pgfpathcurveto{\pgfqpoint{0.836816in}{1.469930in}}{\pgfqpoint{0.833544in}{1.462030in}}{\pgfqpoint{0.833544in}{1.453794in}}%
\pgfpathcurveto{\pgfqpoint{0.833544in}{1.445558in}}{\pgfqpoint{0.836816in}{1.437658in}}{\pgfqpoint{0.842640in}{1.431834in}}%
\pgfpathcurveto{\pgfqpoint{0.848464in}{1.426010in}}{\pgfqpoint{0.856364in}{1.422737in}}{\pgfqpoint{0.864600in}{1.422737in}}%
\pgfpathclose%
\pgfusepath{stroke,fill}%
\end{pgfscope}%
\begin{pgfscope}%
\pgfpathrectangle{\pgfqpoint{0.100000in}{0.220728in}}{\pgfqpoint{3.696000in}{3.696000in}}%
\pgfusepath{clip}%
\pgfsetbuttcap%
\pgfsetroundjoin%
\definecolor{currentfill}{rgb}{0.121569,0.466667,0.705882}%
\pgfsetfillcolor{currentfill}%
\pgfsetfillopacity{0.657336}%
\pgfsetlinewidth{1.003750pt}%
\definecolor{currentstroke}{rgb}{0.121569,0.466667,0.705882}%
\pgfsetstrokecolor{currentstroke}%
\pgfsetstrokeopacity{0.657336}%
\pgfsetdash{}{0pt}%
\pgfpathmoveto{\pgfqpoint{0.864595in}{1.422741in}}%
\pgfpathcurveto{\pgfqpoint{0.872831in}{1.422741in}}{\pgfqpoint{0.880731in}{1.426013in}}{\pgfqpoint{0.886555in}{1.431837in}}%
\pgfpathcurveto{\pgfqpoint{0.892379in}{1.437661in}}{\pgfqpoint{0.895651in}{1.445561in}}{\pgfqpoint{0.895651in}{1.453797in}}%
\pgfpathcurveto{\pgfqpoint{0.895651in}{1.462033in}}{\pgfqpoint{0.892379in}{1.469933in}}{\pgfqpoint{0.886555in}{1.475757in}}%
\pgfpathcurveto{\pgfqpoint{0.880731in}{1.481581in}}{\pgfqpoint{0.872831in}{1.484854in}}{\pgfqpoint{0.864595in}{1.484854in}}%
\pgfpathcurveto{\pgfqpoint{0.856358in}{1.484854in}}{\pgfqpoint{0.848458in}{1.481581in}}{\pgfqpoint{0.842634in}{1.475757in}}%
\pgfpathcurveto{\pgfqpoint{0.836810in}{1.469933in}}{\pgfqpoint{0.833538in}{1.462033in}}{\pgfqpoint{0.833538in}{1.453797in}}%
\pgfpathcurveto{\pgfqpoint{0.833538in}{1.445561in}}{\pgfqpoint{0.836810in}{1.437661in}}{\pgfqpoint{0.842634in}{1.431837in}}%
\pgfpathcurveto{\pgfqpoint{0.848458in}{1.426013in}}{\pgfqpoint{0.856358in}{1.422741in}}{\pgfqpoint{0.864595in}{1.422741in}}%
\pgfpathclose%
\pgfusepath{stroke,fill}%
\end{pgfscope}%
\begin{pgfscope}%
\pgfpathrectangle{\pgfqpoint{0.100000in}{0.220728in}}{\pgfqpoint{3.696000in}{3.696000in}}%
\pgfusepath{clip}%
\pgfsetbuttcap%
\pgfsetroundjoin%
\definecolor{currentfill}{rgb}{0.121569,0.466667,0.705882}%
\pgfsetfillcolor{currentfill}%
\pgfsetfillopacity{0.657338}%
\pgfsetlinewidth{1.003750pt}%
\definecolor{currentstroke}{rgb}{0.121569,0.466667,0.705882}%
\pgfsetstrokecolor{currentstroke}%
\pgfsetstrokeopacity{0.657338}%
\pgfsetdash{}{0pt}%
\pgfpathmoveto{\pgfqpoint{0.864592in}{1.422742in}}%
\pgfpathcurveto{\pgfqpoint{0.872828in}{1.422742in}}{\pgfqpoint{0.880728in}{1.426014in}}{\pgfqpoint{0.886552in}{1.431838in}}%
\pgfpathcurveto{\pgfqpoint{0.892376in}{1.437662in}}{\pgfqpoint{0.895648in}{1.445562in}}{\pgfqpoint{0.895648in}{1.453798in}}%
\pgfpathcurveto{\pgfqpoint{0.895648in}{1.462034in}}{\pgfqpoint{0.892376in}{1.469935in}}{\pgfqpoint{0.886552in}{1.475758in}}%
\pgfpathcurveto{\pgfqpoint{0.880728in}{1.481582in}}{\pgfqpoint{0.872828in}{1.484855in}}{\pgfqpoint{0.864592in}{1.484855in}}%
\pgfpathcurveto{\pgfqpoint{0.856355in}{1.484855in}}{\pgfqpoint{0.848455in}{1.481582in}}{\pgfqpoint{0.842632in}{1.475758in}}%
\pgfpathcurveto{\pgfqpoint{0.836808in}{1.469935in}}{\pgfqpoint{0.833535in}{1.462034in}}{\pgfqpoint{0.833535in}{1.453798in}}%
\pgfpathcurveto{\pgfqpoint{0.833535in}{1.445562in}}{\pgfqpoint{0.836808in}{1.437662in}}{\pgfqpoint{0.842632in}{1.431838in}}%
\pgfpathcurveto{\pgfqpoint{0.848455in}{1.426014in}}{\pgfqpoint{0.856355in}{1.422742in}}{\pgfqpoint{0.864592in}{1.422742in}}%
\pgfpathclose%
\pgfusepath{stroke,fill}%
\end{pgfscope}%
\begin{pgfscope}%
\pgfpathrectangle{\pgfqpoint{0.100000in}{0.220728in}}{\pgfqpoint{3.696000in}{3.696000in}}%
\pgfusepath{clip}%
\pgfsetbuttcap%
\pgfsetroundjoin%
\definecolor{currentfill}{rgb}{0.121569,0.466667,0.705882}%
\pgfsetfillcolor{currentfill}%
\pgfsetfillopacity{0.657339}%
\pgfsetlinewidth{1.003750pt}%
\definecolor{currentstroke}{rgb}{0.121569,0.466667,0.705882}%
\pgfsetstrokecolor{currentstroke}%
\pgfsetstrokeopacity{0.657339}%
\pgfsetdash{}{0pt}%
\pgfpathmoveto{\pgfqpoint{0.864590in}{1.422742in}}%
\pgfpathcurveto{\pgfqpoint{0.872826in}{1.422742in}}{\pgfqpoint{0.880727in}{1.426015in}}{\pgfqpoint{0.886550in}{1.431839in}}%
\pgfpathcurveto{\pgfqpoint{0.892374in}{1.437662in}}{\pgfqpoint{0.895647in}{1.445563in}}{\pgfqpoint{0.895647in}{1.453799in}}%
\pgfpathcurveto{\pgfqpoint{0.895647in}{1.462035in}}{\pgfqpoint{0.892374in}{1.469935in}}{\pgfqpoint{0.886550in}{1.475759in}}%
\pgfpathcurveto{\pgfqpoint{0.880727in}{1.481583in}}{\pgfqpoint{0.872826in}{1.484855in}}{\pgfqpoint{0.864590in}{1.484855in}}%
\pgfpathcurveto{\pgfqpoint{0.856354in}{1.484855in}}{\pgfqpoint{0.848454in}{1.481583in}}{\pgfqpoint{0.842630in}{1.475759in}}%
\pgfpathcurveto{\pgfqpoint{0.836806in}{1.469935in}}{\pgfqpoint{0.833534in}{1.462035in}}{\pgfqpoint{0.833534in}{1.453799in}}%
\pgfpathcurveto{\pgfqpoint{0.833534in}{1.445563in}}{\pgfqpoint{0.836806in}{1.437662in}}{\pgfqpoint{0.842630in}{1.431839in}}%
\pgfpathcurveto{\pgfqpoint{0.848454in}{1.426015in}}{\pgfqpoint{0.856354in}{1.422742in}}{\pgfqpoint{0.864590in}{1.422742in}}%
\pgfpathclose%
\pgfusepath{stroke,fill}%
\end{pgfscope}%
\begin{pgfscope}%
\pgfpathrectangle{\pgfqpoint{0.100000in}{0.220728in}}{\pgfqpoint{3.696000in}{3.696000in}}%
\pgfusepath{clip}%
\pgfsetbuttcap%
\pgfsetroundjoin%
\definecolor{currentfill}{rgb}{0.121569,0.466667,0.705882}%
\pgfsetfillcolor{currentfill}%
\pgfsetfillopacity{0.657339}%
\pgfsetlinewidth{1.003750pt}%
\definecolor{currentstroke}{rgb}{0.121569,0.466667,0.705882}%
\pgfsetstrokecolor{currentstroke}%
\pgfsetstrokeopacity{0.657339}%
\pgfsetdash{}{0pt}%
\pgfpathmoveto{\pgfqpoint{0.864589in}{1.422743in}}%
\pgfpathcurveto{\pgfqpoint{0.872826in}{1.422743in}}{\pgfqpoint{0.880726in}{1.426015in}}{\pgfqpoint{0.886550in}{1.431839in}}%
\pgfpathcurveto{\pgfqpoint{0.892373in}{1.437663in}}{\pgfqpoint{0.895646in}{1.445563in}}{\pgfqpoint{0.895646in}{1.453799in}}%
\pgfpathcurveto{\pgfqpoint{0.895646in}{1.462035in}}{\pgfqpoint{0.892373in}{1.469935in}}{\pgfqpoint{0.886550in}{1.475759in}}%
\pgfpathcurveto{\pgfqpoint{0.880726in}{1.481583in}}{\pgfqpoint{0.872826in}{1.484856in}}{\pgfqpoint{0.864589in}{1.484856in}}%
\pgfpathcurveto{\pgfqpoint{0.856353in}{1.484856in}}{\pgfqpoint{0.848453in}{1.481583in}}{\pgfqpoint{0.842629in}{1.475759in}}%
\pgfpathcurveto{\pgfqpoint{0.836805in}{1.469935in}}{\pgfqpoint{0.833533in}{1.462035in}}{\pgfqpoint{0.833533in}{1.453799in}}%
\pgfpathcurveto{\pgfqpoint{0.833533in}{1.445563in}}{\pgfqpoint{0.836805in}{1.437663in}}{\pgfqpoint{0.842629in}{1.431839in}}%
\pgfpathcurveto{\pgfqpoint{0.848453in}{1.426015in}}{\pgfqpoint{0.856353in}{1.422743in}}{\pgfqpoint{0.864589in}{1.422743in}}%
\pgfpathclose%
\pgfusepath{stroke,fill}%
\end{pgfscope}%
\begin{pgfscope}%
\pgfpathrectangle{\pgfqpoint{0.100000in}{0.220728in}}{\pgfqpoint{3.696000in}{3.696000in}}%
\pgfusepath{clip}%
\pgfsetbuttcap%
\pgfsetroundjoin%
\definecolor{currentfill}{rgb}{0.121569,0.466667,0.705882}%
\pgfsetfillcolor{currentfill}%
\pgfsetfillopacity{0.657340}%
\pgfsetlinewidth{1.003750pt}%
\definecolor{currentstroke}{rgb}{0.121569,0.466667,0.705882}%
\pgfsetstrokecolor{currentstroke}%
\pgfsetstrokeopacity{0.657340}%
\pgfsetdash{}{0pt}%
\pgfpathmoveto{\pgfqpoint{0.864589in}{1.422743in}}%
\pgfpathcurveto{\pgfqpoint{0.872825in}{1.422743in}}{\pgfqpoint{0.880725in}{1.426015in}}{\pgfqpoint{0.886549in}{1.431839in}}%
\pgfpathcurveto{\pgfqpoint{0.892373in}{1.437663in}}{\pgfqpoint{0.895645in}{1.445563in}}{\pgfqpoint{0.895645in}{1.453799in}}%
\pgfpathcurveto{\pgfqpoint{0.895645in}{1.462035in}}{\pgfqpoint{0.892373in}{1.469935in}}{\pgfqpoint{0.886549in}{1.475759in}}%
\pgfpathcurveto{\pgfqpoint{0.880725in}{1.481583in}}{\pgfqpoint{0.872825in}{1.484856in}}{\pgfqpoint{0.864589in}{1.484856in}}%
\pgfpathcurveto{\pgfqpoint{0.856353in}{1.484856in}}{\pgfqpoint{0.848452in}{1.481583in}}{\pgfqpoint{0.842629in}{1.475759in}}%
\pgfpathcurveto{\pgfqpoint{0.836805in}{1.469935in}}{\pgfqpoint{0.833532in}{1.462035in}}{\pgfqpoint{0.833532in}{1.453799in}}%
\pgfpathcurveto{\pgfqpoint{0.833532in}{1.445563in}}{\pgfqpoint{0.836805in}{1.437663in}}{\pgfqpoint{0.842629in}{1.431839in}}%
\pgfpathcurveto{\pgfqpoint{0.848452in}{1.426015in}}{\pgfqpoint{0.856353in}{1.422743in}}{\pgfqpoint{0.864589in}{1.422743in}}%
\pgfpathclose%
\pgfusepath{stroke,fill}%
\end{pgfscope}%
\begin{pgfscope}%
\pgfpathrectangle{\pgfqpoint{0.100000in}{0.220728in}}{\pgfqpoint{3.696000in}{3.696000in}}%
\pgfusepath{clip}%
\pgfsetbuttcap%
\pgfsetroundjoin%
\definecolor{currentfill}{rgb}{0.121569,0.466667,0.705882}%
\pgfsetfillcolor{currentfill}%
\pgfsetfillopacity{0.657340}%
\pgfsetlinewidth{1.003750pt}%
\definecolor{currentstroke}{rgb}{0.121569,0.466667,0.705882}%
\pgfsetstrokecolor{currentstroke}%
\pgfsetstrokeopacity{0.657340}%
\pgfsetdash{}{0pt}%
\pgfpathmoveto{\pgfqpoint{0.864589in}{1.422743in}}%
\pgfpathcurveto{\pgfqpoint{0.872825in}{1.422743in}}{\pgfqpoint{0.880725in}{1.426015in}}{\pgfqpoint{0.886549in}{1.431839in}}%
\pgfpathcurveto{\pgfqpoint{0.892373in}{1.437663in}}{\pgfqpoint{0.895645in}{1.445563in}}{\pgfqpoint{0.895645in}{1.453799in}}%
\pgfpathcurveto{\pgfqpoint{0.895645in}{1.462035in}}{\pgfqpoint{0.892373in}{1.469935in}}{\pgfqpoint{0.886549in}{1.475759in}}%
\pgfpathcurveto{\pgfqpoint{0.880725in}{1.481583in}}{\pgfqpoint{0.872825in}{1.484856in}}{\pgfqpoint{0.864589in}{1.484856in}}%
\pgfpathcurveto{\pgfqpoint{0.856352in}{1.484856in}}{\pgfqpoint{0.848452in}{1.481583in}}{\pgfqpoint{0.842628in}{1.475759in}}%
\pgfpathcurveto{\pgfqpoint{0.836804in}{1.469935in}}{\pgfqpoint{0.833532in}{1.462035in}}{\pgfqpoint{0.833532in}{1.453799in}}%
\pgfpathcurveto{\pgfqpoint{0.833532in}{1.445563in}}{\pgfqpoint{0.836804in}{1.437663in}}{\pgfqpoint{0.842628in}{1.431839in}}%
\pgfpathcurveto{\pgfqpoint{0.848452in}{1.426015in}}{\pgfqpoint{0.856352in}{1.422743in}}{\pgfqpoint{0.864589in}{1.422743in}}%
\pgfpathclose%
\pgfusepath{stroke,fill}%
\end{pgfscope}%
\begin{pgfscope}%
\pgfpathrectangle{\pgfqpoint{0.100000in}{0.220728in}}{\pgfqpoint{3.696000in}{3.696000in}}%
\pgfusepath{clip}%
\pgfsetbuttcap%
\pgfsetroundjoin%
\definecolor{currentfill}{rgb}{0.121569,0.466667,0.705882}%
\pgfsetfillcolor{currentfill}%
\pgfsetfillopacity{0.657340}%
\pgfsetlinewidth{1.003750pt}%
\definecolor{currentstroke}{rgb}{0.121569,0.466667,0.705882}%
\pgfsetstrokecolor{currentstroke}%
\pgfsetstrokeopacity{0.657340}%
\pgfsetdash{}{0pt}%
\pgfpathmoveto{\pgfqpoint{0.864588in}{1.422743in}}%
\pgfpathcurveto{\pgfqpoint{0.872825in}{1.422743in}}{\pgfqpoint{0.880725in}{1.426015in}}{\pgfqpoint{0.886549in}{1.431839in}}%
\pgfpathcurveto{\pgfqpoint{0.892373in}{1.437663in}}{\pgfqpoint{0.895645in}{1.445563in}}{\pgfqpoint{0.895645in}{1.453799in}}%
\pgfpathcurveto{\pgfqpoint{0.895645in}{1.462035in}}{\pgfqpoint{0.892373in}{1.469935in}}{\pgfqpoint{0.886549in}{1.475759in}}%
\pgfpathcurveto{\pgfqpoint{0.880725in}{1.481583in}}{\pgfqpoint{0.872825in}{1.484856in}}{\pgfqpoint{0.864588in}{1.484856in}}%
\pgfpathcurveto{\pgfqpoint{0.856352in}{1.484856in}}{\pgfqpoint{0.848452in}{1.481583in}}{\pgfqpoint{0.842628in}{1.475759in}}%
\pgfpathcurveto{\pgfqpoint{0.836804in}{1.469935in}}{\pgfqpoint{0.833532in}{1.462035in}}{\pgfqpoint{0.833532in}{1.453799in}}%
\pgfpathcurveto{\pgfqpoint{0.833532in}{1.445563in}}{\pgfqpoint{0.836804in}{1.437663in}}{\pgfqpoint{0.842628in}{1.431839in}}%
\pgfpathcurveto{\pgfqpoint{0.848452in}{1.426015in}}{\pgfqpoint{0.856352in}{1.422743in}}{\pgfqpoint{0.864588in}{1.422743in}}%
\pgfpathclose%
\pgfusepath{stroke,fill}%
\end{pgfscope}%
\begin{pgfscope}%
\pgfpathrectangle{\pgfqpoint{0.100000in}{0.220728in}}{\pgfqpoint{3.696000in}{3.696000in}}%
\pgfusepath{clip}%
\pgfsetbuttcap%
\pgfsetroundjoin%
\definecolor{currentfill}{rgb}{0.121569,0.466667,0.705882}%
\pgfsetfillcolor{currentfill}%
\pgfsetfillopacity{0.657340}%
\pgfsetlinewidth{1.003750pt}%
\definecolor{currentstroke}{rgb}{0.121569,0.466667,0.705882}%
\pgfsetstrokecolor{currentstroke}%
\pgfsetstrokeopacity{0.657340}%
\pgfsetdash{}{0pt}%
\pgfpathmoveto{\pgfqpoint{0.864588in}{1.422743in}}%
\pgfpathcurveto{\pgfqpoint{0.872825in}{1.422743in}}{\pgfqpoint{0.880725in}{1.426015in}}{\pgfqpoint{0.886549in}{1.431839in}}%
\pgfpathcurveto{\pgfqpoint{0.892373in}{1.437663in}}{\pgfqpoint{0.895645in}{1.445563in}}{\pgfqpoint{0.895645in}{1.453799in}}%
\pgfpathcurveto{\pgfqpoint{0.895645in}{1.462035in}}{\pgfqpoint{0.892373in}{1.469935in}}{\pgfqpoint{0.886549in}{1.475759in}}%
\pgfpathcurveto{\pgfqpoint{0.880725in}{1.481583in}}{\pgfqpoint{0.872825in}{1.484856in}}{\pgfqpoint{0.864588in}{1.484856in}}%
\pgfpathcurveto{\pgfqpoint{0.856352in}{1.484856in}}{\pgfqpoint{0.848452in}{1.481583in}}{\pgfqpoint{0.842628in}{1.475759in}}%
\pgfpathcurveto{\pgfqpoint{0.836804in}{1.469935in}}{\pgfqpoint{0.833532in}{1.462035in}}{\pgfqpoint{0.833532in}{1.453799in}}%
\pgfpathcurveto{\pgfqpoint{0.833532in}{1.445563in}}{\pgfqpoint{0.836804in}{1.437663in}}{\pgfqpoint{0.842628in}{1.431839in}}%
\pgfpathcurveto{\pgfqpoint{0.848452in}{1.426015in}}{\pgfqpoint{0.856352in}{1.422743in}}{\pgfqpoint{0.864588in}{1.422743in}}%
\pgfpathclose%
\pgfusepath{stroke,fill}%
\end{pgfscope}%
\begin{pgfscope}%
\pgfpathrectangle{\pgfqpoint{0.100000in}{0.220728in}}{\pgfqpoint{3.696000in}{3.696000in}}%
\pgfusepath{clip}%
\pgfsetbuttcap%
\pgfsetroundjoin%
\definecolor{currentfill}{rgb}{0.121569,0.466667,0.705882}%
\pgfsetfillcolor{currentfill}%
\pgfsetfillopacity{0.657340}%
\pgfsetlinewidth{1.003750pt}%
\definecolor{currentstroke}{rgb}{0.121569,0.466667,0.705882}%
\pgfsetstrokecolor{currentstroke}%
\pgfsetstrokeopacity{0.657340}%
\pgfsetdash{}{0pt}%
\pgfpathmoveto{\pgfqpoint{0.864588in}{1.422743in}}%
\pgfpathcurveto{\pgfqpoint{0.872825in}{1.422743in}}{\pgfqpoint{0.880725in}{1.426015in}}{\pgfqpoint{0.886549in}{1.431839in}}%
\pgfpathcurveto{\pgfqpoint{0.892372in}{1.437663in}}{\pgfqpoint{0.895645in}{1.445563in}}{\pgfqpoint{0.895645in}{1.453799in}}%
\pgfpathcurveto{\pgfqpoint{0.895645in}{1.462035in}}{\pgfqpoint{0.892372in}{1.469935in}}{\pgfqpoint{0.886549in}{1.475759in}}%
\pgfpathcurveto{\pgfqpoint{0.880725in}{1.481583in}}{\pgfqpoint{0.872825in}{1.484856in}}{\pgfqpoint{0.864588in}{1.484856in}}%
\pgfpathcurveto{\pgfqpoint{0.856352in}{1.484856in}}{\pgfqpoint{0.848452in}{1.481583in}}{\pgfqpoint{0.842628in}{1.475759in}}%
\pgfpathcurveto{\pgfqpoint{0.836804in}{1.469935in}}{\pgfqpoint{0.833532in}{1.462035in}}{\pgfqpoint{0.833532in}{1.453799in}}%
\pgfpathcurveto{\pgfqpoint{0.833532in}{1.445563in}}{\pgfqpoint{0.836804in}{1.437663in}}{\pgfqpoint{0.842628in}{1.431839in}}%
\pgfpathcurveto{\pgfqpoint{0.848452in}{1.426015in}}{\pgfqpoint{0.856352in}{1.422743in}}{\pgfqpoint{0.864588in}{1.422743in}}%
\pgfpathclose%
\pgfusepath{stroke,fill}%
\end{pgfscope}%
\begin{pgfscope}%
\pgfpathrectangle{\pgfqpoint{0.100000in}{0.220728in}}{\pgfqpoint{3.696000in}{3.696000in}}%
\pgfusepath{clip}%
\pgfsetbuttcap%
\pgfsetroundjoin%
\definecolor{currentfill}{rgb}{0.121569,0.466667,0.705882}%
\pgfsetfillcolor{currentfill}%
\pgfsetfillopacity{0.657340}%
\pgfsetlinewidth{1.003750pt}%
\definecolor{currentstroke}{rgb}{0.121569,0.466667,0.705882}%
\pgfsetstrokecolor{currentstroke}%
\pgfsetstrokeopacity{0.657340}%
\pgfsetdash{}{0pt}%
\pgfpathmoveto{\pgfqpoint{0.864588in}{1.422743in}}%
\pgfpathcurveto{\pgfqpoint{0.872825in}{1.422743in}}{\pgfqpoint{0.880725in}{1.426015in}}{\pgfqpoint{0.886549in}{1.431839in}}%
\pgfpathcurveto{\pgfqpoint{0.892372in}{1.437663in}}{\pgfqpoint{0.895645in}{1.445563in}}{\pgfqpoint{0.895645in}{1.453799in}}%
\pgfpathcurveto{\pgfqpoint{0.895645in}{1.462035in}}{\pgfqpoint{0.892372in}{1.469935in}}{\pgfqpoint{0.886549in}{1.475759in}}%
\pgfpathcurveto{\pgfqpoint{0.880725in}{1.481583in}}{\pgfqpoint{0.872825in}{1.484856in}}{\pgfqpoint{0.864588in}{1.484856in}}%
\pgfpathcurveto{\pgfqpoint{0.856352in}{1.484856in}}{\pgfqpoint{0.848452in}{1.481583in}}{\pgfqpoint{0.842628in}{1.475759in}}%
\pgfpathcurveto{\pgfqpoint{0.836804in}{1.469935in}}{\pgfqpoint{0.833532in}{1.462035in}}{\pgfqpoint{0.833532in}{1.453799in}}%
\pgfpathcurveto{\pgfqpoint{0.833532in}{1.445563in}}{\pgfqpoint{0.836804in}{1.437663in}}{\pgfqpoint{0.842628in}{1.431839in}}%
\pgfpathcurveto{\pgfqpoint{0.848452in}{1.426015in}}{\pgfqpoint{0.856352in}{1.422743in}}{\pgfqpoint{0.864588in}{1.422743in}}%
\pgfpathclose%
\pgfusepath{stroke,fill}%
\end{pgfscope}%
\begin{pgfscope}%
\pgfpathrectangle{\pgfqpoint{0.100000in}{0.220728in}}{\pgfqpoint{3.696000in}{3.696000in}}%
\pgfusepath{clip}%
\pgfsetbuttcap%
\pgfsetroundjoin%
\definecolor{currentfill}{rgb}{0.121569,0.466667,0.705882}%
\pgfsetfillcolor{currentfill}%
\pgfsetfillopacity{0.657340}%
\pgfsetlinewidth{1.003750pt}%
\definecolor{currentstroke}{rgb}{0.121569,0.466667,0.705882}%
\pgfsetstrokecolor{currentstroke}%
\pgfsetstrokeopacity{0.657340}%
\pgfsetdash{}{0pt}%
\pgfpathmoveto{\pgfqpoint{0.864588in}{1.422743in}}%
\pgfpathcurveto{\pgfqpoint{0.872825in}{1.422743in}}{\pgfqpoint{0.880725in}{1.426015in}}{\pgfqpoint{0.886549in}{1.431839in}}%
\pgfpathcurveto{\pgfqpoint{0.892372in}{1.437663in}}{\pgfqpoint{0.895645in}{1.445563in}}{\pgfqpoint{0.895645in}{1.453799in}}%
\pgfpathcurveto{\pgfqpoint{0.895645in}{1.462035in}}{\pgfqpoint{0.892372in}{1.469935in}}{\pgfqpoint{0.886549in}{1.475759in}}%
\pgfpathcurveto{\pgfqpoint{0.880725in}{1.481583in}}{\pgfqpoint{0.872825in}{1.484856in}}{\pgfqpoint{0.864588in}{1.484856in}}%
\pgfpathcurveto{\pgfqpoint{0.856352in}{1.484856in}}{\pgfqpoint{0.848452in}{1.481583in}}{\pgfqpoint{0.842628in}{1.475759in}}%
\pgfpathcurveto{\pgfqpoint{0.836804in}{1.469935in}}{\pgfqpoint{0.833532in}{1.462035in}}{\pgfqpoint{0.833532in}{1.453799in}}%
\pgfpathcurveto{\pgfqpoint{0.833532in}{1.445563in}}{\pgfqpoint{0.836804in}{1.437663in}}{\pgfqpoint{0.842628in}{1.431839in}}%
\pgfpathcurveto{\pgfqpoint{0.848452in}{1.426015in}}{\pgfqpoint{0.856352in}{1.422743in}}{\pgfqpoint{0.864588in}{1.422743in}}%
\pgfpathclose%
\pgfusepath{stroke,fill}%
\end{pgfscope}%
\begin{pgfscope}%
\pgfpathrectangle{\pgfqpoint{0.100000in}{0.220728in}}{\pgfqpoint{3.696000in}{3.696000in}}%
\pgfusepath{clip}%
\pgfsetbuttcap%
\pgfsetroundjoin%
\definecolor{currentfill}{rgb}{0.121569,0.466667,0.705882}%
\pgfsetfillcolor{currentfill}%
\pgfsetfillopacity{0.657340}%
\pgfsetlinewidth{1.003750pt}%
\definecolor{currentstroke}{rgb}{0.121569,0.466667,0.705882}%
\pgfsetstrokecolor{currentstroke}%
\pgfsetstrokeopacity{0.657340}%
\pgfsetdash{}{0pt}%
\pgfpathmoveto{\pgfqpoint{0.864588in}{1.422743in}}%
\pgfpathcurveto{\pgfqpoint{0.872825in}{1.422743in}}{\pgfqpoint{0.880725in}{1.426015in}}{\pgfqpoint{0.886548in}{1.431839in}}%
\pgfpathcurveto{\pgfqpoint{0.892372in}{1.437663in}}{\pgfqpoint{0.895645in}{1.445563in}}{\pgfqpoint{0.895645in}{1.453799in}}%
\pgfpathcurveto{\pgfqpoint{0.895645in}{1.462035in}}{\pgfqpoint{0.892372in}{1.469935in}}{\pgfqpoint{0.886548in}{1.475759in}}%
\pgfpathcurveto{\pgfqpoint{0.880725in}{1.481583in}}{\pgfqpoint{0.872825in}{1.484856in}}{\pgfqpoint{0.864588in}{1.484856in}}%
\pgfpathcurveto{\pgfqpoint{0.856352in}{1.484856in}}{\pgfqpoint{0.848452in}{1.481583in}}{\pgfqpoint{0.842628in}{1.475759in}}%
\pgfpathcurveto{\pgfqpoint{0.836804in}{1.469935in}}{\pgfqpoint{0.833532in}{1.462035in}}{\pgfqpoint{0.833532in}{1.453799in}}%
\pgfpathcurveto{\pgfqpoint{0.833532in}{1.445563in}}{\pgfqpoint{0.836804in}{1.437663in}}{\pgfqpoint{0.842628in}{1.431839in}}%
\pgfpathcurveto{\pgfqpoint{0.848452in}{1.426015in}}{\pgfqpoint{0.856352in}{1.422743in}}{\pgfqpoint{0.864588in}{1.422743in}}%
\pgfpathclose%
\pgfusepath{stroke,fill}%
\end{pgfscope}%
\begin{pgfscope}%
\pgfpathrectangle{\pgfqpoint{0.100000in}{0.220728in}}{\pgfqpoint{3.696000in}{3.696000in}}%
\pgfusepath{clip}%
\pgfsetbuttcap%
\pgfsetroundjoin%
\definecolor{currentfill}{rgb}{0.121569,0.466667,0.705882}%
\pgfsetfillcolor{currentfill}%
\pgfsetfillopacity{0.657340}%
\pgfsetlinewidth{1.003750pt}%
\definecolor{currentstroke}{rgb}{0.121569,0.466667,0.705882}%
\pgfsetstrokecolor{currentstroke}%
\pgfsetstrokeopacity{0.657340}%
\pgfsetdash{}{0pt}%
\pgfpathmoveto{\pgfqpoint{0.864588in}{1.422743in}}%
\pgfpathcurveto{\pgfqpoint{0.872825in}{1.422743in}}{\pgfqpoint{0.880725in}{1.426015in}}{\pgfqpoint{0.886548in}{1.431839in}}%
\pgfpathcurveto{\pgfqpoint{0.892372in}{1.437663in}}{\pgfqpoint{0.895645in}{1.445563in}}{\pgfqpoint{0.895645in}{1.453799in}}%
\pgfpathcurveto{\pgfqpoint{0.895645in}{1.462035in}}{\pgfqpoint{0.892372in}{1.469935in}}{\pgfqpoint{0.886548in}{1.475759in}}%
\pgfpathcurveto{\pgfqpoint{0.880725in}{1.481583in}}{\pgfqpoint{0.872825in}{1.484856in}}{\pgfqpoint{0.864588in}{1.484856in}}%
\pgfpathcurveto{\pgfqpoint{0.856352in}{1.484856in}}{\pgfqpoint{0.848452in}{1.481583in}}{\pgfqpoint{0.842628in}{1.475759in}}%
\pgfpathcurveto{\pgfqpoint{0.836804in}{1.469935in}}{\pgfqpoint{0.833532in}{1.462035in}}{\pgfqpoint{0.833532in}{1.453799in}}%
\pgfpathcurveto{\pgfqpoint{0.833532in}{1.445563in}}{\pgfqpoint{0.836804in}{1.437663in}}{\pgfqpoint{0.842628in}{1.431839in}}%
\pgfpathcurveto{\pgfqpoint{0.848452in}{1.426015in}}{\pgfqpoint{0.856352in}{1.422743in}}{\pgfqpoint{0.864588in}{1.422743in}}%
\pgfpathclose%
\pgfusepath{stroke,fill}%
\end{pgfscope}%
\begin{pgfscope}%
\pgfpathrectangle{\pgfqpoint{0.100000in}{0.220728in}}{\pgfqpoint{3.696000in}{3.696000in}}%
\pgfusepath{clip}%
\pgfsetbuttcap%
\pgfsetroundjoin%
\definecolor{currentfill}{rgb}{0.121569,0.466667,0.705882}%
\pgfsetfillcolor{currentfill}%
\pgfsetfillopacity{0.657340}%
\pgfsetlinewidth{1.003750pt}%
\definecolor{currentstroke}{rgb}{0.121569,0.466667,0.705882}%
\pgfsetstrokecolor{currentstroke}%
\pgfsetstrokeopacity{0.657340}%
\pgfsetdash{}{0pt}%
\pgfpathmoveto{\pgfqpoint{0.864588in}{1.422743in}}%
\pgfpathcurveto{\pgfqpoint{0.872825in}{1.422743in}}{\pgfqpoint{0.880725in}{1.426015in}}{\pgfqpoint{0.886548in}{1.431839in}}%
\pgfpathcurveto{\pgfqpoint{0.892372in}{1.437663in}}{\pgfqpoint{0.895645in}{1.445563in}}{\pgfqpoint{0.895645in}{1.453799in}}%
\pgfpathcurveto{\pgfqpoint{0.895645in}{1.462035in}}{\pgfqpoint{0.892372in}{1.469935in}}{\pgfqpoint{0.886548in}{1.475759in}}%
\pgfpathcurveto{\pgfqpoint{0.880725in}{1.481583in}}{\pgfqpoint{0.872825in}{1.484856in}}{\pgfqpoint{0.864588in}{1.484856in}}%
\pgfpathcurveto{\pgfqpoint{0.856352in}{1.484856in}}{\pgfqpoint{0.848452in}{1.481583in}}{\pgfqpoint{0.842628in}{1.475759in}}%
\pgfpathcurveto{\pgfqpoint{0.836804in}{1.469935in}}{\pgfqpoint{0.833532in}{1.462035in}}{\pgfqpoint{0.833532in}{1.453799in}}%
\pgfpathcurveto{\pgfqpoint{0.833532in}{1.445563in}}{\pgfqpoint{0.836804in}{1.437663in}}{\pgfqpoint{0.842628in}{1.431839in}}%
\pgfpathcurveto{\pgfqpoint{0.848452in}{1.426015in}}{\pgfqpoint{0.856352in}{1.422743in}}{\pgfqpoint{0.864588in}{1.422743in}}%
\pgfpathclose%
\pgfusepath{stroke,fill}%
\end{pgfscope}%
\begin{pgfscope}%
\pgfpathrectangle{\pgfqpoint{0.100000in}{0.220728in}}{\pgfqpoint{3.696000in}{3.696000in}}%
\pgfusepath{clip}%
\pgfsetbuttcap%
\pgfsetroundjoin%
\definecolor{currentfill}{rgb}{0.121569,0.466667,0.705882}%
\pgfsetfillcolor{currentfill}%
\pgfsetfillopacity{0.657340}%
\pgfsetlinewidth{1.003750pt}%
\definecolor{currentstroke}{rgb}{0.121569,0.466667,0.705882}%
\pgfsetstrokecolor{currentstroke}%
\pgfsetstrokeopacity{0.657340}%
\pgfsetdash{}{0pt}%
\pgfpathmoveto{\pgfqpoint{0.864588in}{1.422743in}}%
\pgfpathcurveto{\pgfqpoint{0.872825in}{1.422743in}}{\pgfqpoint{0.880725in}{1.426015in}}{\pgfqpoint{0.886548in}{1.431839in}}%
\pgfpathcurveto{\pgfqpoint{0.892372in}{1.437663in}}{\pgfqpoint{0.895645in}{1.445563in}}{\pgfqpoint{0.895645in}{1.453799in}}%
\pgfpathcurveto{\pgfqpoint{0.895645in}{1.462035in}}{\pgfqpoint{0.892372in}{1.469935in}}{\pgfqpoint{0.886548in}{1.475759in}}%
\pgfpathcurveto{\pgfqpoint{0.880725in}{1.481583in}}{\pgfqpoint{0.872825in}{1.484856in}}{\pgfqpoint{0.864588in}{1.484856in}}%
\pgfpathcurveto{\pgfqpoint{0.856352in}{1.484856in}}{\pgfqpoint{0.848452in}{1.481583in}}{\pgfqpoint{0.842628in}{1.475759in}}%
\pgfpathcurveto{\pgfqpoint{0.836804in}{1.469935in}}{\pgfqpoint{0.833532in}{1.462035in}}{\pgfqpoint{0.833532in}{1.453799in}}%
\pgfpathcurveto{\pgfqpoint{0.833532in}{1.445563in}}{\pgfqpoint{0.836804in}{1.437663in}}{\pgfqpoint{0.842628in}{1.431839in}}%
\pgfpathcurveto{\pgfqpoint{0.848452in}{1.426015in}}{\pgfqpoint{0.856352in}{1.422743in}}{\pgfqpoint{0.864588in}{1.422743in}}%
\pgfpathclose%
\pgfusepath{stroke,fill}%
\end{pgfscope}%
\begin{pgfscope}%
\pgfpathrectangle{\pgfqpoint{0.100000in}{0.220728in}}{\pgfqpoint{3.696000in}{3.696000in}}%
\pgfusepath{clip}%
\pgfsetbuttcap%
\pgfsetroundjoin%
\definecolor{currentfill}{rgb}{0.121569,0.466667,0.705882}%
\pgfsetfillcolor{currentfill}%
\pgfsetfillopacity{0.657340}%
\pgfsetlinewidth{1.003750pt}%
\definecolor{currentstroke}{rgb}{0.121569,0.466667,0.705882}%
\pgfsetstrokecolor{currentstroke}%
\pgfsetstrokeopacity{0.657340}%
\pgfsetdash{}{0pt}%
\pgfpathmoveto{\pgfqpoint{0.864588in}{1.422743in}}%
\pgfpathcurveto{\pgfqpoint{0.872825in}{1.422743in}}{\pgfqpoint{0.880725in}{1.426015in}}{\pgfqpoint{0.886548in}{1.431839in}}%
\pgfpathcurveto{\pgfqpoint{0.892372in}{1.437663in}}{\pgfqpoint{0.895645in}{1.445563in}}{\pgfqpoint{0.895645in}{1.453799in}}%
\pgfpathcurveto{\pgfqpoint{0.895645in}{1.462035in}}{\pgfqpoint{0.892372in}{1.469935in}}{\pgfqpoint{0.886548in}{1.475759in}}%
\pgfpathcurveto{\pgfqpoint{0.880725in}{1.481583in}}{\pgfqpoint{0.872825in}{1.484856in}}{\pgfqpoint{0.864588in}{1.484856in}}%
\pgfpathcurveto{\pgfqpoint{0.856352in}{1.484856in}}{\pgfqpoint{0.848452in}{1.481583in}}{\pgfqpoint{0.842628in}{1.475759in}}%
\pgfpathcurveto{\pgfqpoint{0.836804in}{1.469935in}}{\pgfqpoint{0.833532in}{1.462035in}}{\pgfqpoint{0.833532in}{1.453799in}}%
\pgfpathcurveto{\pgfqpoint{0.833532in}{1.445563in}}{\pgfqpoint{0.836804in}{1.437663in}}{\pgfqpoint{0.842628in}{1.431839in}}%
\pgfpathcurveto{\pgfqpoint{0.848452in}{1.426015in}}{\pgfqpoint{0.856352in}{1.422743in}}{\pgfqpoint{0.864588in}{1.422743in}}%
\pgfpathclose%
\pgfusepath{stroke,fill}%
\end{pgfscope}%
\begin{pgfscope}%
\pgfpathrectangle{\pgfqpoint{0.100000in}{0.220728in}}{\pgfqpoint{3.696000in}{3.696000in}}%
\pgfusepath{clip}%
\pgfsetbuttcap%
\pgfsetroundjoin%
\definecolor{currentfill}{rgb}{0.121569,0.466667,0.705882}%
\pgfsetfillcolor{currentfill}%
\pgfsetfillopacity{0.657340}%
\pgfsetlinewidth{1.003750pt}%
\definecolor{currentstroke}{rgb}{0.121569,0.466667,0.705882}%
\pgfsetstrokecolor{currentstroke}%
\pgfsetstrokeopacity{0.657340}%
\pgfsetdash{}{0pt}%
\pgfpathmoveto{\pgfqpoint{0.864588in}{1.422743in}}%
\pgfpathcurveto{\pgfqpoint{0.872825in}{1.422743in}}{\pgfqpoint{0.880725in}{1.426015in}}{\pgfqpoint{0.886548in}{1.431839in}}%
\pgfpathcurveto{\pgfqpoint{0.892372in}{1.437663in}}{\pgfqpoint{0.895645in}{1.445563in}}{\pgfqpoint{0.895645in}{1.453799in}}%
\pgfpathcurveto{\pgfqpoint{0.895645in}{1.462035in}}{\pgfqpoint{0.892372in}{1.469935in}}{\pgfqpoint{0.886548in}{1.475759in}}%
\pgfpathcurveto{\pgfqpoint{0.880725in}{1.481583in}}{\pgfqpoint{0.872825in}{1.484856in}}{\pgfqpoint{0.864588in}{1.484856in}}%
\pgfpathcurveto{\pgfqpoint{0.856352in}{1.484856in}}{\pgfqpoint{0.848452in}{1.481583in}}{\pgfqpoint{0.842628in}{1.475759in}}%
\pgfpathcurveto{\pgfqpoint{0.836804in}{1.469935in}}{\pgfqpoint{0.833532in}{1.462035in}}{\pgfqpoint{0.833532in}{1.453799in}}%
\pgfpathcurveto{\pgfqpoint{0.833532in}{1.445563in}}{\pgfqpoint{0.836804in}{1.437663in}}{\pgfqpoint{0.842628in}{1.431839in}}%
\pgfpathcurveto{\pgfqpoint{0.848452in}{1.426015in}}{\pgfqpoint{0.856352in}{1.422743in}}{\pgfqpoint{0.864588in}{1.422743in}}%
\pgfpathclose%
\pgfusepath{stroke,fill}%
\end{pgfscope}%
\begin{pgfscope}%
\pgfpathrectangle{\pgfqpoint{0.100000in}{0.220728in}}{\pgfqpoint{3.696000in}{3.696000in}}%
\pgfusepath{clip}%
\pgfsetbuttcap%
\pgfsetroundjoin%
\definecolor{currentfill}{rgb}{0.121569,0.466667,0.705882}%
\pgfsetfillcolor{currentfill}%
\pgfsetfillopacity{0.657340}%
\pgfsetlinewidth{1.003750pt}%
\definecolor{currentstroke}{rgb}{0.121569,0.466667,0.705882}%
\pgfsetstrokecolor{currentstroke}%
\pgfsetstrokeopacity{0.657340}%
\pgfsetdash{}{0pt}%
\pgfpathmoveto{\pgfqpoint{0.864588in}{1.422743in}}%
\pgfpathcurveto{\pgfqpoint{0.872825in}{1.422743in}}{\pgfqpoint{0.880725in}{1.426015in}}{\pgfqpoint{0.886548in}{1.431839in}}%
\pgfpathcurveto{\pgfqpoint{0.892372in}{1.437663in}}{\pgfqpoint{0.895645in}{1.445563in}}{\pgfqpoint{0.895645in}{1.453799in}}%
\pgfpathcurveto{\pgfqpoint{0.895645in}{1.462035in}}{\pgfqpoint{0.892372in}{1.469935in}}{\pgfqpoint{0.886548in}{1.475759in}}%
\pgfpathcurveto{\pgfqpoint{0.880725in}{1.481583in}}{\pgfqpoint{0.872825in}{1.484856in}}{\pgfqpoint{0.864588in}{1.484856in}}%
\pgfpathcurveto{\pgfqpoint{0.856352in}{1.484856in}}{\pgfqpoint{0.848452in}{1.481583in}}{\pgfqpoint{0.842628in}{1.475759in}}%
\pgfpathcurveto{\pgfqpoint{0.836804in}{1.469935in}}{\pgfqpoint{0.833532in}{1.462035in}}{\pgfqpoint{0.833532in}{1.453799in}}%
\pgfpathcurveto{\pgfqpoint{0.833532in}{1.445563in}}{\pgfqpoint{0.836804in}{1.437663in}}{\pgfqpoint{0.842628in}{1.431839in}}%
\pgfpathcurveto{\pgfqpoint{0.848452in}{1.426015in}}{\pgfqpoint{0.856352in}{1.422743in}}{\pgfqpoint{0.864588in}{1.422743in}}%
\pgfpathclose%
\pgfusepath{stroke,fill}%
\end{pgfscope}%
\begin{pgfscope}%
\pgfpathrectangle{\pgfqpoint{0.100000in}{0.220728in}}{\pgfqpoint{3.696000in}{3.696000in}}%
\pgfusepath{clip}%
\pgfsetbuttcap%
\pgfsetroundjoin%
\definecolor{currentfill}{rgb}{0.121569,0.466667,0.705882}%
\pgfsetfillcolor{currentfill}%
\pgfsetfillopacity{0.657340}%
\pgfsetlinewidth{1.003750pt}%
\definecolor{currentstroke}{rgb}{0.121569,0.466667,0.705882}%
\pgfsetstrokecolor{currentstroke}%
\pgfsetstrokeopacity{0.657340}%
\pgfsetdash{}{0pt}%
\pgfpathmoveto{\pgfqpoint{0.864588in}{1.422743in}}%
\pgfpathcurveto{\pgfqpoint{0.872825in}{1.422743in}}{\pgfqpoint{0.880725in}{1.426015in}}{\pgfqpoint{0.886548in}{1.431839in}}%
\pgfpathcurveto{\pgfqpoint{0.892372in}{1.437663in}}{\pgfqpoint{0.895645in}{1.445563in}}{\pgfqpoint{0.895645in}{1.453799in}}%
\pgfpathcurveto{\pgfqpoint{0.895645in}{1.462035in}}{\pgfqpoint{0.892372in}{1.469935in}}{\pgfqpoint{0.886548in}{1.475759in}}%
\pgfpathcurveto{\pgfqpoint{0.880725in}{1.481583in}}{\pgfqpoint{0.872825in}{1.484856in}}{\pgfqpoint{0.864588in}{1.484856in}}%
\pgfpathcurveto{\pgfqpoint{0.856352in}{1.484856in}}{\pgfqpoint{0.848452in}{1.481583in}}{\pgfqpoint{0.842628in}{1.475759in}}%
\pgfpathcurveto{\pgfqpoint{0.836804in}{1.469935in}}{\pgfqpoint{0.833532in}{1.462035in}}{\pgfqpoint{0.833532in}{1.453799in}}%
\pgfpathcurveto{\pgfqpoint{0.833532in}{1.445563in}}{\pgfqpoint{0.836804in}{1.437663in}}{\pgfqpoint{0.842628in}{1.431839in}}%
\pgfpathcurveto{\pgfqpoint{0.848452in}{1.426015in}}{\pgfqpoint{0.856352in}{1.422743in}}{\pgfqpoint{0.864588in}{1.422743in}}%
\pgfpathclose%
\pgfusepath{stroke,fill}%
\end{pgfscope}%
\begin{pgfscope}%
\pgfpathrectangle{\pgfqpoint{0.100000in}{0.220728in}}{\pgfqpoint{3.696000in}{3.696000in}}%
\pgfusepath{clip}%
\pgfsetbuttcap%
\pgfsetroundjoin%
\definecolor{currentfill}{rgb}{0.121569,0.466667,0.705882}%
\pgfsetfillcolor{currentfill}%
\pgfsetfillopacity{0.657340}%
\pgfsetlinewidth{1.003750pt}%
\definecolor{currentstroke}{rgb}{0.121569,0.466667,0.705882}%
\pgfsetstrokecolor{currentstroke}%
\pgfsetstrokeopacity{0.657340}%
\pgfsetdash{}{0pt}%
\pgfpathmoveto{\pgfqpoint{0.864588in}{1.422743in}}%
\pgfpathcurveto{\pgfqpoint{0.872825in}{1.422743in}}{\pgfqpoint{0.880725in}{1.426015in}}{\pgfqpoint{0.886548in}{1.431839in}}%
\pgfpathcurveto{\pgfqpoint{0.892372in}{1.437663in}}{\pgfqpoint{0.895645in}{1.445563in}}{\pgfqpoint{0.895645in}{1.453799in}}%
\pgfpathcurveto{\pgfqpoint{0.895645in}{1.462035in}}{\pgfqpoint{0.892372in}{1.469935in}}{\pgfqpoint{0.886548in}{1.475759in}}%
\pgfpathcurveto{\pgfqpoint{0.880725in}{1.481583in}}{\pgfqpoint{0.872825in}{1.484856in}}{\pgfqpoint{0.864588in}{1.484856in}}%
\pgfpathcurveto{\pgfqpoint{0.856352in}{1.484856in}}{\pgfqpoint{0.848452in}{1.481583in}}{\pgfqpoint{0.842628in}{1.475759in}}%
\pgfpathcurveto{\pgfqpoint{0.836804in}{1.469935in}}{\pgfqpoint{0.833532in}{1.462035in}}{\pgfqpoint{0.833532in}{1.453799in}}%
\pgfpathcurveto{\pgfqpoint{0.833532in}{1.445563in}}{\pgfqpoint{0.836804in}{1.437663in}}{\pgfqpoint{0.842628in}{1.431839in}}%
\pgfpathcurveto{\pgfqpoint{0.848452in}{1.426015in}}{\pgfqpoint{0.856352in}{1.422743in}}{\pgfqpoint{0.864588in}{1.422743in}}%
\pgfpathclose%
\pgfusepath{stroke,fill}%
\end{pgfscope}%
\begin{pgfscope}%
\pgfpathrectangle{\pgfqpoint{0.100000in}{0.220728in}}{\pgfqpoint{3.696000in}{3.696000in}}%
\pgfusepath{clip}%
\pgfsetbuttcap%
\pgfsetroundjoin%
\definecolor{currentfill}{rgb}{0.121569,0.466667,0.705882}%
\pgfsetfillcolor{currentfill}%
\pgfsetfillopacity{0.657340}%
\pgfsetlinewidth{1.003750pt}%
\definecolor{currentstroke}{rgb}{0.121569,0.466667,0.705882}%
\pgfsetstrokecolor{currentstroke}%
\pgfsetstrokeopacity{0.657340}%
\pgfsetdash{}{0pt}%
\pgfpathmoveto{\pgfqpoint{0.864588in}{1.422743in}}%
\pgfpathcurveto{\pgfqpoint{0.872825in}{1.422743in}}{\pgfqpoint{0.880725in}{1.426015in}}{\pgfqpoint{0.886548in}{1.431839in}}%
\pgfpathcurveto{\pgfqpoint{0.892372in}{1.437663in}}{\pgfqpoint{0.895645in}{1.445563in}}{\pgfqpoint{0.895645in}{1.453799in}}%
\pgfpathcurveto{\pgfqpoint{0.895645in}{1.462035in}}{\pgfqpoint{0.892372in}{1.469935in}}{\pgfqpoint{0.886548in}{1.475759in}}%
\pgfpathcurveto{\pgfqpoint{0.880725in}{1.481583in}}{\pgfqpoint{0.872825in}{1.484856in}}{\pgfqpoint{0.864588in}{1.484856in}}%
\pgfpathcurveto{\pgfqpoint{0.856352in}{1.484856in}}{\pgfqpoint{0.848452in}{1.481583in}}{\pgfqpoint{0.842628in}{1.475759in}}%
\pgfpathcurveto{\pgfqpoint{0.836804in}{1.469935in}}{\pgfqpoint{0.833532in}{1.462035in}}{\pgfqpoint{0.833532in}{1.453799in}}%
\pgfpathcurveto{\pgfqpoint{0.833532in}{1.445563in}}{\pgfqpoint{0.836804in}{1.437663in}}{\pgfqpoint{0.842628in}{1.431839in}}%
\pgfpathcurveto{\pgfqpoint{0.848452in}{1.426015in}}{\pgfqpoint{0.856352in}{1.422743in}}{\pgfqpoint{0.864588in}{1.422743in}}%
\pgfpathclose%
\pgfusepath{stroke,fill}%
\end{pgfscope}%
\begin{pgfscope}%
\pgfpathrectangle{\pgfqpoint{0.100000in}{0.220728in}}{\pgfqpoint{3.696000in}{3.696000in}}%
\pgfusepath{clip}%
\pgfsetbuttcap%
\pgfsetroundjoin%
\definecolor{currentfill}{rgb}{0.121569,0.466667,0.705882}%
\pgfsetfillcolor{currentfill}%
\pgfsetfillopacity{0.657340}%
\pgfsetlinewidth{1.003750pt}%
\definecolor{currentstroke}{rgb}{0.121569,0.466667,0.705882}%
\pgfsetstrokecolor{currentstroke}%
\pgfsetstrokeopacity{0.657340}%
\pgfsetdash{}{0pt}%
\pgfpathmoveto{\pgfqpoint{0.864588in}{1.422743in}}%
\pgfpathcurveto{\pgfqpoint{0.872825in}{1.422743in}}{\pgfqpoint{0.880725in}{1.426015in}}{\pgfqpoint{0.886548in}{1.431839in}}%
\pgfpathcurveto{\pgfqpoint{0.892372in}{1.437663in}}{\pgfqpoint{0.895645in}{1.445563in}}{\pgfqpoint{0.895645in}{1.453799in}}%
\pgfpathcurveto{\pgfqpoint{0.895645in}{1.462035in}}{\pgfqpoint{0.892372in}{1.469935in}}{\pgfqpoint{0.886548in}{1.475759in}}%
\pgfpathcurveto{\pgfqpoint{0.880725in}{1.481583in}}{\pgfqpoint{0.872825in}{1.484856in}}{\pgfqpoint{0.864588in}{1.484856in}}%
\pgfpathcurveto{\pgfqpoint{0.856352in}{1.484856in}}{\pgfqpoint{0.848452in}{1.481583in}}{\pgfqpoint{0.842628in}{1.475759in}}%
\pgfpathcurveto{\pgfqpoint{0.836804in}{1.469935in}}{\pgfqpoint{0.833532in}{1.462035in}}{\pgfqpoint{0.833532in}{1.453799in}}%
\pgfpathcurveto{\pgfqpoint{0.833532in}{1.445563in}}{\pgfqpoint{0.836804in}{1.437663in}}{\pgfqpoint{0.842628in}{1.431839in}}%
\pgfpathcurveto{\pgfqpoint{0.848452in}{1.426015in}}{\pgfqpoint{0.856352in}{1.422743in}}{\pgfqpoint{0.864588in}{1.422743in}}%
\pgfpathclose%
\pgfusepath{stroke,fill}%
\end{pgfscope}%
\begin{pgfscope}%
\pgfpathrectangle{\pgfqpoint{0.100000in}{0.220728in}}{\pgfqpoint{3.696000in}{3.696000in}}%
\pgfusepath{clip}%
\pgfsetbuttcap%
\pgfsetroundjoin%
\definecolor{currentfill}{rgb}{0.121569,0.466667,0.705882}%
\pgfsetfillcolor{currentfill}%
\pgfsetfillopacity{0.657340}%
\pgfsetlinewidth{1.003750pt}%
\definecolor{currentstroke}{rgb}{0.121569,0.466667,0.705882}%
\pgfsetstrokecolor{currentstroke}%
\pgfsetstrokeopacity{0.657340}%
\pgfsetdash{}{0pt}%
\pgfpathmoveto{\pgfqpoint{0.864588in}{1.422743in}}%
\pgfpathcurveto{\pgfqpoint{0.872825in}{1.422743in}}{\pgfqpoint{0.880725in}{1.426015in}}{\pgfqpoint{0.886548in}{1.431839in}}%
\pgfpathcurveto{\pgfqpoint{0.892372in}{1.437663in}}{\pgfqpoint{0.895645in}{1.445563in}}{\pgfqpoint{0.895645in}{1.453799in}}%
\pgfpathcurveto{\pgfqpoint{0.895645in}{1.462035in}}{\pgfqpoint{0.892372in}{1.469935in}}{\pgfqpoint{0.886548in}{1.475759in}}%
\pgfpathcurveto{\pgfqpoint{0.880725in}{1.481583in}}{\pgfqpoint{0.872825in}{1.484856in}}{\pgfqpoint{0.864588in}{1.484856in}}%
\pgfpathcurveto{\pgfqpoint{0.856352in}{1.484856in}}{\pgfqpoint{0.848452in}{1.481583in}}{\pgfqpoint{0.842628in}{1.475759in}}%
\pgfpathcurveto{\pgfqpoint{0.836804in}{1.469935in}}{\pgfqpoint{0.833532in}{1.462035in}}{\pgfqpoint{0.833532in}{1.453799in}}%
\pgfpathcurveto{\pgfqpoint{0.833532in}{1.445563in}}{\pgfqpoint{0.836804in}{1.437663in}}{\pgfqpoint{0.842628in}{1.431839in}}%
\pgfpathcurveto{\pgfqpoint{0.848452in}{1.426015in}}{\pgfqpoint{0.856352in}{1.422743in}}{\pgfqpoint{0.864588in}{1.422743in}}%
\pgfpathclose%
\pgfusepath{stroke,fill}%
\end{pgfscope}%
\begin{pgfscope}%
\pgfpathrectangle{\pgfqpoint{0.100000in}{0.220728in}}{\pgfqpoint{3.696000in}{3.696000in}}%
\pgfusepath{clip}%
\pgfsetbuttcap%
\pgfsetroundjoin%
\definecolor{currentfill}{rgb}{0.121569,0.466667,0.705882}%
\pgfsetfillcolor{currentfill}%
\pgfsetfillopacity{0.657340}%
\pgfsetlinewidth{1.003750pt}%
\definecolor{currentstroke}{rgb}{0.121569,0.466667,0.705882}%
\pgfsetstrokecolor{currentstroke}%
\pgfsetstrokeopacity{0.657340}%
\pgfsetdash{}{0pt}%
\pgfpathmoveto{\pgfqpoint{0.864588in}{1.422743in}}%
\pgfpathcurveto{\pgfqpoint{0.872825in}{1.422743in}}{\pgfqpoint{0.880725in}{1.426015in}}{\pgfqpoint{0.886548in}{1.431839in}}%
\pgfpathcurveto{\pgfqpoint{0.892372in}{1.437663in}}{\pgfqpoint{0.895645in}{1.445563in}}{\pgfqpoint{0.895645in}{1.453799in}}%
\pgfpathcurveto{\pgfqpoint{0.895645in}{1.462035in}}{\pgfqpoint{0.892372in}{1.469935in}}{\pgfqpoint{0.886548in}{1.475759in}}%
\pgfpathcurveto{\pgfqpoint{0.880725in}{1.481583in}}{\pgfqpoint{0.872825in}{1.484856in}}{\pgfqpoint{0.864588in}{1.484856in}}%
\pgfpathcurveto{\pgfqpoint{0.856352in}{1.484856in}}{\pgfqpoint{0.848452in}{1.481583in}}{\pgfqpoint{0.842628in}{1.475759in}}%
\pgfpathcurveto{\pgfqpoint{0.836804in}{1.469935in}}{\pgfqpoint{0.833532in}{1.462035in}}{\pgfqpoint{0.833532in}{1.453799in}}%
\pgfpathcurveto{\pgfqpoint{0.833532in}{1.445563in}}{\pgfqpoint{0.836804in}{1.437663in}}{\pgfqpoint{0.842628in}{1.431839in}}%
\pgfpathcurveto{\pgfqpoint{0.848452in}{1.426015in}}{\pgfqpoint{0.856352in}{1.422743in}}{\pgfqpoint{0.864588in}{1.422743in}}%
\pgfpathclose%
\pgfusepath{stroke,fill}%
\end{pgfscope}%
\begin{pgfscope}%
\pgfpathrectangle{\pgfqpoint{0.100000in}{0.220728in}}{\pgfqpoint{3.696000in}{3.696000in}}%
\pgfusepath{clip}%
\pgfsetbuttcap%
\pgfsetroundjoin%
\definecolor{currentfill}{rgb}{0.121569,0.466667,0.705882}%
\pgfsetfillcolor{currentfill}%
\pgfsetfillopacity{0.657340}%
\pgfsetlinewidth{1.003750pt}%
\definecolor{currentstroke}{rgb}{0.121569,0.466667,0.705882}%
\pgfsetstrokecolor{currentstroke}%
\pgfsetstrokeopacity{0.657340}%
\pgfsetdash{}{0pt}%
\pgfpathmoveto{\pgfqpoint{0.864588in}{1.422743in}}%
\pgfpathcurveto{\pgfqpoint{0.872825in}{1.422743in}}{\pgfqpoint{0.880725in}{1.426015in}}{\pgfqpoint{0.886548in}{1.431839in}}%
\pgfpathcurveto{\pgfqpoint{0.892372in}{1.437663in}}{\pgfqpoint{0.895645in}{1.445563in}}{\pgfqpoint{0.895645in}{1.453799in}}%
\pgfpathcurveto{\pgfqpoint{0.895645in}{1.462035in}}{\pgfqpoint{0.892372in}{1.469935in}}{\pgfqpoint{0.886548in}{1.475759in}}%
\pgfpathcurveto{\pgfqpoint{0.880725in}{1.481583in}}{\pgfqpoint{0.872825in}{1.484856in}}{\pgfqpoint{0.864588in}{1.484856in}}%
\pgfpathcurveto{\pgfqpoint{0.856352in}{1.484856in}}{\pgfqpoint{0.848452in}{1.481583in}}{\pgfqpoint{0.842628in}{1.475759in}}%
\pgfpathcurveto{\pgfqpoint{0.836804in}{1.469935in}}{\pgfqpoint{0.833532in}{1.462035in}}{\pgfqpoint{0.833532in}{1.453799in}}%
\pgfpathcurveto{\pgfqpoint{0.833532in}{1.445563in}}{\pgfqpoint{0.836804in}{1.437663in}}{\pgfqpoint{0.842628in}{1.431839in}}%
\pgfpathcurveto{\pgfqpoint{0.848452in}{1.426015in}}{\pgfqpoint{0.856352in}{1.422743in}}{\pgfqpoint{0.864588in}{1.422743in}}%
\pgfpathclose%
\pgfusepath{stroke,fill}%
\end{pgfscope}%
\begin{pgfscope}%
\pgfpathrectangle{\pgfqpoint{0.100000in}{0.220728in}}{\pgfqpoint{3.696000in}{3.696000in}}%
\pgfusepath{clip}%
\pgfsetbuttcap%
\pgfsetroundjoin%
\definecolor{currentfill}{rgb}{0.121569,0.466667,0.705882}%
\pgfsetfillcolor{currentfill}%
\pgfsetfillopacity{0.657340}%
\pgfsetlinewidth{1.003750pt}%
\definecolor{currentstroke}{rgb}{0.121569,0.466667,0.705882}%
\pgfsetstrokecolor{currentstroke}%
\pgfsetstrokeopacity{0.657340}%
\pgfsetdash{}{0pt}%
\pgfpathmoveto{\pgfqpoint{0.864588in}{1.422743in}}%
\pgfpathcurveto{\pgfqpoint{0.872825in}{1.422743in}}{\pgfqpoint{0.880725in}{1.426015in}}{\pgfqpoint{0.886548in}{1.431839in}}%
\pgfpathcurveto{\pgfqpoint{0.892372in}{1.437663in}}{\pgfqpoint{0.895645in}{1.445563in}}{\pgfqpoint{0.895645in}{1.453799in}}%
\pgfpathcurveto{\pgfqpoint{0.895645in}{1.462035in}}{\pgfqpoint{0.892372in}{1.469935in}}{\pgfqpoint{0.886548in}{1.475759in}}%
\pgfpathcurveto{\pgfqpoint{0.880725in}{1.481583in}}{\pgfqpoint{0.872825in}{1.484856in}}{\pgfqpoint{0.864588in}{1.484856in}}%
\pgfpathcurveto{\pgfqpoint{0.856352in}{1.484856in}}{\pgfqpoint{0.848452in}{1.481583in}}{\pgfqpoint{0.842628in}{1.475759in}}%
\pgfpathcurveto{\pgfqpoint{0.836804in}{1.469935in}}{\pgfqpoint{0.833532in}{1.462035in}}{\pgfqpoint{0.833532in}{1.453799in}}%
\pgfpathcurveto{\pgfqpoint{0.833532in}{1.445563in}}{\pgfqpoint{0.836804in}{1.437663in}}{\pgfqpoint{0.842628in}{1.431839in}}%
\pgfpathcurveto{\pgfqpoint{0.848452in}{1.426015in}}{\pgfqpoint{0.856352in}{1.422743in}}{\pgfqpoint{0.864588in}{1.422743in}}%
\pgfpathclose%
\pgfusepath{stroke,fill}%
\end{pgfscope}%
\begin{pgfscope}%
\pgfpathrectangle{\pgfqpoint{0.100000in}{0.220728in}}{\pgfqpoint{3.696000in}{3.696000in}}%
\pgfusepath{clip}%
\pgfsetbuttcap%
\pgfsetroundjoin%
\definecolor{currentfill}{rgb}{0.121569,0.466667,0.705882}%
\pgfsetfillcolor{currentfill}%
\pgfsetfillopacity{0.657340}%
\pgfsetlinewidth{1.003750pt}%
\definecolor{currentstroke}{rgb}{0.121569,0.466667,0.705882}%
\pgfsetstrokecolor{currentstroke}%
\pgfsetstrokeopacity{0.657340}%
\pgfsetdash{}{0pt}%
\pgfpathmoveto{\pgfqpoint{0.864588in}{1.422743in}}%
\pgfpathcurveto{\pgfqpoint{0.872825in}{1.422743in}}{\pgfqpoint{0.880725in}{1.426015in}}{\pgfqpoint{0.886548in}{1.431839in}}%
\pgfpathcurveto{\pgfqpoint{0.892372in}{1.437663in}}{\pgfqpoint{0.895645in}{1.445563in}}{\pgfqpoint{0.895645in}{1.453799in}}%
\pgfpathcurveto{\pgfqpoint{0.895645in}{1.462035in}}{\pgfqpoint{0.892372in}{1.469935in}}{\pgfqpoint{0.886548in}{1.475759in}}%
\pgfpathcurveto{\pgfqpoint{0.880725in}{1.481583in}}{\pgfqpoint{0.872825in}{1.484856in}}{\pgfqpoint{0.864588in}{1.484856in}}%
\pgfpathcurveto{\pgfqpoint{0.856352in}{1.484856in}}{\pgfqpoint{0.848452in}{1.481583in}}{\pgfqpoint{0.842628in}{1.475759in}}%
\pgfpathcurveto{\pgfqpoint{0.836804in}{1.469935in}}{\pgfqpoint{0.833532in}{1.462035in}}{\pgfqpoint{0.833532in}{1.453799in}}%
\pgfpathcurveto{\pgfqpoint{0.833532in}{1.445563in}}{\pgfqpoint{0.836804in}{1.437663in}}{\pgfqpoint{0.842628in}{1.431839in}}%
\pgfpathcurveto{\pgfqpoint{0.848452in}{1.426015in}}{\pgfqpoint{0.856352in}{1.422743in}}{\pgfqpoint{0.864588in}{1.422743in}}%
\pgfpathclose%
\pgfusepath{stroke,fill}%
\end{pgfscope}%
\begin{pgfscope}%
\pgfpathrectangle{\pgfqpoint{0.100000in}{0.220728in}}{\pgfqpoint{3.696000in}{3.696000in}}%
\pgfusepath{clip}%
\pgfsetbuttcap%
\pgfsetroundjoin%
\definecolor{currentfill}{rgb}{0.121569,0.466667,0.705882}%
\pgfsetfillcolor{currentfill}%
\pgfsetfillopacity{0.657340}%
\pgfsetlinewidth{1.003750pt}%
\definecolor{currentstroke}{rgb}{0.121569,0.466667,0.705882}%
\pgfsetstrokecolor{currentstroke}%
\pgfsetstrokeopacity{0.657340}%
\pgfsetdash{}{0pt}%
\pgfpathmoveto{\pgfqpoint{0.864588in}{1.422743in}}%
\pgfpathcurveto{\pgfqpoint{0.872825in}{1.422743in}}{\pgfqpoint{0.880725in}{1.426015in}}{\pgfqpoint{0.886548in}{1.431839in}}%
\pgfpathcurveto{\pgfqpoint{0.892372in}{1.437663in}}{\pgfqpoint{0.895645in}{1.445563in}}{\pgfqpoint{0.895645in}{1.453799in}}%
\pgfpathcurveto{\pgfqpoint{0.895645in}{1.462035in}}{\pgfqpoint{0.892372in}{1.469935in}}{\pgfqpoint{0.886548in}{1.475759in}}%
\pgfpathcurveto{\pgfqpoint{0.880725in}{1.481583in}}{\pgfqpoint{0.872825in}{1.484856in}}{\pgfqpoint{0.864588in}{1.484856in}}%
\pgfpathcurveto{\pgfqpoint{0.856352in}{1.484856in}}{\pgfqpoint{0.848452in}{1.481583in}}{\pgfqpoint{0.842628in}{1.475759in}}%
\pgfpathcurveto{\pgfqpoint{0.836804in}{1.469935in}}{\pgfqpoint{0.833532in}{1.462035in}}{\pgfqpoint{0.833532in}{1.453799in}}%
\pgfpathcurveto{\pgfqpoint{0.833532in}{1.445563in}}{\pgfqpoint{0.836804in}{1.437663in}}{\pgfqpoint{0.842628in}{1.431839in}}%
\pgfpathcurveto{\pgfqpoint{0.848452in}{1.426015in}}{\pgfqpoint{0.856352in}{1.422743in}}{\pgfqpoint{0.864588in}{1.422743in}}%
\pgfpathclose%
\pgfusepath{stroke,fill}%
\end{pgfscope}%
\begin{pgfscope}%
\pgfpathrectangle{\pgfqpoint{0.100000in}{0.220728in}}{\pgfqpoint{3.696000in}{3.696000in}}%
\pgfusepath{clip}%
\pgfsetbuttcap%
\pgfsetroundjoin%
\definecolor{currentfill}{rgb}{0.121569,0.466667,0.705882}%
\pgfsetfillcolor{currentfill}%
\pgfsetfillopacity{0.657340}%
\pgfsetlinewidth{1.003750pt}%
\definecolor{currentstroke}{rgb}{0.121569,0.466667,0.705882}%
\pgfsetstrokecolor{currentstroke}%
\pgfsetstrokeopacity{0.657340}%
\pgfsetdash{}{0pt}%
\pgfpathmoveto{\pgfqpoint{0.864588in}{1.422743in}}%
\pgfpathcurveto{\pgfqpoint{0.872825in}{1.422743in}}{\pgfqpoint{0.880725in}{1.426015in}}{\pgfqpoint{0.886548in}{1.431839in}}%
\pgfpathcurveto{\pgfqpoint{0.892372in}{1.437663in}}{\pgfqpoint{0.895645in}{1.445563in}}{\pgfqpoint{0.895645in}{1.453799in}}%
\pgfpathcurveto{\pgfqpoint{0.895645in}{1.462035in}}{\pgfqpoint{0.892372in}{1.469935in}}{\pgfqpoint{0.886548in}{1.475759in}}%
\pgfpathcurveto{\pgfqpoint{0.880725in}{1.481583in}}{\pgfqpoint{0.872825in}{1.484856in}}{\pgfqpoint{0.864588in}{1.484856in}}%
\pgfpathcurveto{\pgfqpoint{0.856352in}{1.484856in}}{\pgfqpoint{0.848452in}{1.481583in}}{\pgfqpoint{0.842628in}{1.475759in}}%
\pgfpathcurveto{\pgfqpoint{0.836804in}{1.469935in}}{\pgfqpoint{0.833532in}{1.462035in}}{\pgfqpoint{0.833532in}{1.453799in}}%
\pgfpathcurveto{\pgfqpoint{0.833532in}{1.445563in}}{\pgfqpoint{0.836804in}{1.437663in}}{\pgfqpoint{0.842628in}{1.431839in}}%
\pgfpathcurveto{\pgfqpoint{0.848452in}{1.426015in}}{\pgfqpoint{0.856352in}{1.422743in}}{\pgfqpoint{0.864588in}{1.422743in}}%
\pgfpathclose%
\pgfusepath{stroke,fill}%
\end{pgfscope}%
\begin{pgfscope}%
\pgfpathrectangle{\pgfqpoint{0.100000in}{0.220728in}}{\pgfqpoint{3.696000in}{3.696000in}}%
\pgfusepath{clip}%
\pgfsetbuttcap%
\pgfsetroundjoin%
\definecolor{currentfill}{rgb}{0.121569,0.466667,0.705882}%
\pgfsetfillcolor{currentfill}%
\pgfsetfillopacity{0.657340}%
\pgfsetlinewidth{1.003750pt}%
\definecolor{currentstroke}{rgb}{0.121569,0.466667,0.705882}%
\pgfsetstrokecolor{currentstroke}%
\pgfsetstrokeopacity{0.657340}%
\pgfsetdash{}{0pt}%
\pgfpathmoveto{\pgfqpoint{0.864588in}{1.422743in}}%
\pgfpathcurveto{\pgfqpoint{0.872825in}{1.422743in}}{\pgfqpoint{0.880725in}{1.426015in}}{\pgfqpoint{0.886548in}{1.431839in}}%
\pgfpathcurveto{\pgfqpoint{0.892372in}{1.437663in}}{\pgfqpoint{0.895645in}{1.445563in}}{\pgfqpoint{0.895645in}{1.453799in}}%
\pgfpathcurveto{\pgfqpoint{0.895645in}{1.462035in}}{\pgfqpoint{0.892372in}{1.469935in}}{\pgfqpoint{0.886548in}{1.475759in}}%
\pgfpathcurveto{\pgfqpoint{0.880725in}{1.481583in}}{\pgfqpoint{0.872825in}{1.484856in}}{\pgfqpoint{0.864588in}{1.484856in}}%
\pgfpathcurveto{\pgfqpoint{0.856352in}{1.484856in}}{\pgfqpoint{0.848452in}{1.481583in}}{\pgfqpoint{0.842628in}{1.475759in}}%
\pgfpathcurveto{\pgfqpoint{0.836804in}{1.469935in}}{\pgfqpoint{0.833532in}{1.462035in}}{\pgfqpoint{0.833532in}{1.453799in}}%
\pgfpathcurveto{\pgfqpoint{0.833532in}{1.445563in}}{\pgfqpoint{0.836804in}{1.437663in}}{\pgfqpoint{0.842628in}{1.431839in}}%
\pgfpathcurveto{\pgfqpoint{0.848452in}{1.426015in}}{\pgfqpoint{0.856352in}{1.422743in}}{\pgfqpoint{0.864588in}{1.422743in}}%
\pgfpathclose%
\pgfusepath{stroke,fill}%
\end{pgfscope}%
\begin{pgfscope}%
\pgfpathrectangle{\pgfqpoint{0.100000in}{0.220728in}}{\pgfqpoint{3.696000in}{3.696000in}}%
\pgfusepath{clip}%
\pgfsetbuttcap%
\pgfsetroundjoin%
\definecolor{currentfill}{rgb}{0.121569,0.466667,0.705882}%
\pgfsetfillcolor{currentfill}%
\pgfsetfillopacity{0.657340}%
\pgfsetlinewidth{1.003750pt}%
\definecolor{currentstroke}{rgb}{0.121569,0.466667,0.705882}%
\pgfsetstrokecolor{currentstroke}%
\pgfsetstrokeopacity{0.657340}%
\pgfsetdash{}{0pt}%
\pgfpathmoveto{\pgfqpoint{0.864588in}{1.422743in}}%
\pgfpathcurveto{\pgfqpoint{0.872825in}{1.422743in}}{\pgfqpoint{0.880725in}{1.426015in}}{\pgfqpoint{0.886548in}{1.431839in}}%
\pgfpathcurveto{\pgfqpoint{0.892372in}{1.437663in}}{\pgfqpoint{0.895645in}{1.445563in}}{\pgfqpoint{0.895645in}{1.453799in}}%
\pgfpathcurveto{\pgfqpoint{0.895645in}{1.462035in}}{\pgfqpoint{0.892372in}{1.469935in}}{\pgfqpoint{0.886548in}{1.475759in}}%
\pgfpathcurveto{\pgfqpoint{0.880725in}{1.481583in}}{\pgfqpoint{0.872825in}{1.484856in}}{\pgfqpoint{0.864588in}{1.484856in}}%
\pgfpathcurveto{\pgfqpoint{0.856352in}{1.484856in}}{\pgfqpoint{0.848452in}{1.481583in}}{\pgfqpoint{0.842628in}{1.475759in}}%
\pgfpathcurveto{\pgfqpoint{0.836804in}{1.469935in}}{\pgfqpoint{0.833532in}{1.462035in}}{\pgfqpoint{0.833532in}{1.453799in}}%
\pgfpathcurveto{\pgfqpoint{0.833532in}{1.445563in}}{\pgfqpoint{0.836804in}{1.437663in}}{\pgfqpoint{0.842628in}{1.431839in}}%
\pgfpathcurveto{\pgfqpoint{0.848452in}{1.426015in}}{\pgfqpoint{0.856352in}{1.422743in}}{\pgfqpoint{0.864588in}{1.422743in}}%
\pgfpathclose%
\pgfusepath{stroke,fill}%
\end{pgfscope}%
\begin{pgfscope}%
\pgfpathrectangle{\pgfqpoint{0.100000in}{0.220728in}}{\pgfqpoint{3.696000in}{3.696000in}}%
\pgfusepath{clip}%
\pgfsetbuttcap%
\pgfsetroundjoin%
\definecolor{currentfill}{rgb}{0.121569,0.466667,0.705882}%
\pgfsetfillcolor{currentfill}%
\pgfsetfillopacity{0.657340}%
\pgfsetlinewidth{1.003750pt}%
\definecolor{currentstroke}{rgb}{0.121569,0.466667,0.705882}%
\pgfsetstrokecolor{currentstroke}%
\pgfsetstrokeopacity{0.657340}%
\pgfsetdash{}{0pt}%
\pgfpathmoveto{\pgfqpoint{0.864588in}{1.422743in}}%
\pgfpathcurveto{\pgfqpoint{0.872825in}{1.422743in}}{\pgfqpoint{0.880725in}{1.426015in}}{\pgfqpoint{0.886548in}{1.431839in}}%
\pgfpathcurveto{\pgfqpoint{0.892372in}{1.437663in}}{\pgfqpoint{0.895645in}{1.445563in}}{\pgfqpoint{0.895645in}{1.453799in}}%
\pgfpathcurveto{\pgfqpoint{0.895645in}{1.462035in}}{\pgfqpoint{0.892372in}{1.469935in}}{\pgfqpoint{0.886548in}{1.475759in}}%
\pgfpathcurveto{\pgfqpoint{0.880725in}{1.481583in}}{\pgfqpoint{0.872825in}{1.484856in}}{\pgfqpoint{0.864588in}{1.484856in}}%
\pgfpathcurveto{\pgfqpoint{0.856352in}{1.484856in}}{\pgfqpoint{0.848452in}{1.481583in}}{\pgfqpoint{0.842628in}{1.475759in}}%
\pgfpathcurveto{\pgfqpoint{0.836804in}{1.469935in}}{\pgfqpoint{0.833532in}{1.462035in}}{\pgfqpoint{0.833532in}{1.453799in}}%
\pgfpathcurveto{\pgfqpoint{0.833532in}{1.445563in}}{\pgfqpoint{0.836804in}{1.437663in}}{\pgfqpoint{0.842628in}{1.431839in}}%
\pgfpathcurveto{\pgfqpoint{0.848452in}{1.426015in}}{\pgfqpoint{0.856352in}{1.422743in}}{\pgfqpoint{0.864588in}{1.422743in}}%
\pgfpathclose%
\pgfusepath{stroke,fill}%
\end{pgfscope}%
\begin{pgfscope}%
\pgfpathrectangle{\pgfqpoint{0.100000in}{0.220728in}}{\pgfqpoint{3.696000in}{3.696000in}}%
\pgfusepath{clip}%
\pgfsetbuttcap%
\pgfsetroundjoin%
\definecolor{currentfill}{rgb}{0.121569,0.466667,0.705882}%
\pgfsetfillcolor{currentfill}%
\pgfsetfillopacity{0.657340}%
\pgfsetlinewidth{1.003750pt}%
\definecolor{currentstroke}{rgb}{0.121569,0.466667,0.705882}%
\pgfsetstrokecolor{currentstroke}%
\pgfsetstrokeopacity{0.657340}%
\pgfsetdash{}{0pt}%
\pgfpathmoveto{\pgfqpoint{0.864588in}{1.422743in}}%
\pgfpathcurveto{\pgfqpoint{0.872825in}{1.422743in}}{\pgfqpoint{0.880725in}{1.426015in}}{\pgfqpoint{0.886548in}{1.431839in}}%
\pgfpathcurveto{\pgfqpoint{0.892372in}{1.437663in}}{\pgfqpoint{0.895645in}{1.445563in}}{\pgfqpoint{0.895645in}{1.453799in}}%
\pgfpathcurveto{\pgfqpoint{0.895645in}{1.462035in}}{\pgfqpoint{0.892372in}{1.469935in}}{\pgfqpoint{0.886548in}{1.475759in}}%
\pgfpathcurveto{\pgfqpoint{0.880725in}{1.481583in}}{\pgfqpoint{0.872825in}{1.484856in}}{\pgfqpoint{0.864588in}{1.484856in}}%
\pgfpathcurveto{\pgfqpoint{0.856352in}{1.484856in}}{\pgfqpoint{0.848452in}{1.481583in}}{\pgfqpoint{0.842628in}{1.475759in}}%
\pgfpathcurveto{\pgfqpoint{0.836804in}{1.469935in}}{\pgfqpoint{0.833532in}{1.462035in}}{\pgfqpoint{0.833532in}{1.453799in}}%
\pgfpathcurveto{\pgfqpoint{0.833532in}{1.445563in}}{\pgfqpoint{0.836804in}{1.437663in}}{\pgfqpoint{0.842628in}{1.431839in}}%
\pgfpathcurveto{\pgfqpoint{0.848452in}{1.426015in}}{\pgfqpoint{0.856352in}{1.422743in}}{\pgfqpoint{0.864588in}{1.422743in}}%
\pgfpathclose%
\pgfusepath{stroke,fill}%
\end{pgfscope}%
\begin{pgfscope}%
\pgfpathrectangle{\pgfqpoint{0.100000in}{0.220728in}}{\pgfqpoint{3.696000in}{3.696000in}}%
\pgfusepath{clip}%
\pgfsetbuttcap%
\pgfsetroundjoin%
\definecolor{currentfill}{rgb}{0.121569,0.466667,0.705882}%
\pgfsetfillcolor{currentfill}%
\pgfsetfillopacity{0.657340}%
\pgfsetlinewidth{1.003750pt}%
\definecolor{currentstroke}{rgb}{0.121569,0.466667,0.705882}%
\pgfsetstrokecolor{currentstroke}%
\pgfsetstrokeopacity{0.657340}%
\pgfsetdash{}{0pt}%
\pgfpathmoveto{\pgfqpoint{0.864588in}{1.422743in}}%
\pgfpathcurveto{\pgfqpoint{0.872825in}{1.422743in}}{\pgfqpoint{0.880725in}{1.426015in}}{\pgfqpoint{0.886548in}{1.431839in}}%
\pgfpathcurveto{\pgfqpoint{0.892372in}{1.437663in}}{\pgfqpoint{0.895645in}{1.445563in}}{\pgfqpoint{0.895645in}{1.453799in}}%
\pgfpathcurveto{\pgfqpoint{0.895645in}{1.462035in}}{\pgfqpoint{0.892372in}{1.469935in}}{\pgfqpoint{0.886548in}{1.475759in}}%
\pgfpathcurveto{\pgfqpoint{0.880725in}{1.481583in}}{\pgfqpoint{0.872825in}{1.484856in}}{\pgfqpoint{0.864588in}{1.484856in}}%
\pgfpathcurveto{\pgfqpoint{0.856352in}{1.484856in}}{\pgfqpoint{0.848452in}{1.481583in}}{\pgfqpoint{0.842628in}{1.475759in}}%
\pgfpathcurveto{\pgfqpoint{0.836804in}{1.469935in}}{\pgfqpoint{0.833532in}{1.462035in}}{\pgfqpoint{0.833532in}{1.453799in}}%
\pgfpathcurveto{\pgfqpoint{0.833532in}{1.445563in}}{\pgfqpoint{0.836804in}{1.437663in}}{\pgfqpoint{0.842628in}{1.431839in}}%
\pgfpathcurveto{\pgfqpoint{0.848452in}{1.426015in}}{\pgfqpoint{0.856352in}{1.422743in}}{\pgfqpoint{0.864588in}{1.422743in}}%
\pgfpathclose%
\pgfusepath{stroke,fill}%
\end{pgfscope}%
\begin{pgfscope}%
\pgfpathrectangle{\pgfqpoint{0.100000in}{0.220728in}}{\pgfqpoint{3.696000in}{3.696000in}}%
\pgfusepath{clip}%
\pgfsetbuttcap%
\pgfsetroundjoin%
\definecolor{currentfill}{rgb}{0.121569,0.466667,0.705882}%
\pgfsetfillcolor{currentfill}%
\pgfsetfillopacity{0.657340}%
\pgfsetlinewidth{1.003750pt}%
\definecolor{currentstroke}{rgb}{0.121569,0.466667,0.705882}%
\pgfsetstrokecolor{currentstroke}%
\pgfsetstrokeopacity{0.657340}%
\pgfsetdash{}{0pt}%
\pgfpathmoveto{\pgfqpoint{0.864588in}{1.422743in}}%
\pgfpathcurveto{\pgfqpoint{0.872825in}{1.422743in}}{\pgfqpoint{0.880725in}{1.426015in}}{\pgfqpoint{0.886548in}{1.431839in}}%
\pgfpathcurveto{\pgfqpoint{0.892372in}{1.437663in}}{\pgfqpoint{0.895645in}{1.445563in}}{\pgfqpoint{0.895645in}{1.453799in}}%
\pgfpathcurveto{\pgfqpoint{0.895645in}{1.462035in}}{\pgfqpoint{0.892372in}{1.469935in}}{\pgfqpoint{0.886548in}{1.475759in}}%
\pgfpathcurveto{\pgfqpoint{0.880725in}{1.481583in}}{\pgfqpoint{0.872825in}{1.484856in}}{\pgfqpoint{0.864588in}{1.484856in}}%
\pgfpathcurveto{\pgfqpoint{0.856352in}{1.484856in}}{\pgfqpoint{0.848452in}{1.481583in}}{\pgfqpoint{0.842628in}{1.475759in}}%
\pgfpathcurveto{\pgfqpoint{0.836804in}{1.469935in}}{\pgfqpoint{0.833532in}{1.462035in}}{\pgfqpoint{0.833532in}{1.453799in}}%
\pgfpathcurveto{\pgfqpoint{0.833532in}{1.445563in}}{\pgfqpoint{0.836804in}{1.437663in}}{\pgfqpoint{0.842628in}{1.431839in}}%
\pgfpathcurveto{\pgfqpoint{0.848452in}{1.426015in}}{\pgfqpoint{0.856352in}{1.422743in}}{\pgfqpoint{0.864588in}{1.422743in}}%
\pgfpathclose%
\pgfusepath{stroke,fill}%
\end{pgfscope}%
\begin{pgfscope}%
\pgfpathrectangle{\pgfqpoint{0.100000in}{0.220728in}}{\pgfqpoint{3.696000in}{3.696000in}}%
\pgfusepath{clip}%
\pgfsetbuttcap%
\pgfsetroundjoin%
\definecolor{currentfill}{rgb}{0.121569,0.466667,0.705882}%
\pgfsetfillcolor{currentfill}%
\pgfsetfillopacity{0.657340}%
\pgfsetlinewidth{1.003750pt}%
\definecolor{currentstroke}{rgb}{0.121569,0.466667,0.705882}%
\pgfsetstrokecolor{currentstroke}%
\pgfsetstrokeopacity{0.657340}%
\pgfsetdash{}{0pt}%
\pgfpathmoveto{\pgfqpoint{0.864588in}{1.422743in}}%
\pgfpathcurveto{\pgfqpoint{0.872825in}{1.422743in}}{\pgfqpoint{0.880725in}{1.426015in}}{\pgfqpoint{0.886548in}{1.431839in}}%
\pgfpathcurveto{\pgfqpoint{0.892372in}{1.437663in}}{\pgfqpoint{0.895645in}{1.445563in}}{\pgfqpoint{0.895645in}{1.453799in}}%
\pgfpathcurveto{\pgfqpoint{0.895645in}{1.462035in}}{\pgfqpoint{0.892372in}{1.469935in}}{\pgfqpoint{0.886548in}{1.475759in}}%
\pgfpathcurveto{\pgfqpoint{0.880725in}{1.481583in}}{\pgfqpoint{0.872825in}{1.484856in}}{\pgfqpoint{0.864588in}{1.484856in}}%
\pgfpathcurveto{\pgfqpoint{0.856352in}{1.484856in}}{\pgfqpoint{0.848452in}{1.481583in}}{\pgfqpoint{0.842628in}{1.475759in}}%
\pgfpathcurveto{\pgfqpoint{0.836804in}{1.469935in}}{\pgfqpoint{0.833532in}{1.462035in}}{\pgfqpoint{0.833532in}{1.453799in}}%
\pgfpathcurveto{\pgfqpoint{0.833532in}{1.445563in}}{\pgfqpoint{0.836804in}{1.437663in}}{\pgfqpoint{0.842628in}{1.431839in}}%
\pgfpathcurveto{\pgfqpoint{0.848452in}{1.426015in}}{\pgfqpoint{0.856352in}{1.422743in}}{\pgfqpoint{0.864588in}{1.422743in}}%
\pgfpathclose%
\pgfusepath{stroke,fill}%
\end{pgfscope}%
\begin{pgfscope}%
\pgfpathrectangle{\pgfqpoint{0.100000in}{0.220728in}}{\pgfqpoint{3.696000in}{3.696000in}}%
\pgfusepath{clip}%
\pgfsetbuttcap%
\pgfsetroundjoin%
\definecolor{currentfill}{rgb}{0.121569,0.466667,0.705882}%
\pgfsetfillcolor{currentfill}%
\pgfsetfillopacity{0.657340}%
\pgfsetlinewidth{1.003750pt}%
\definecolor{currentstroke}{rgb}{0.121569,0.466667,0.705882}%
\pgfsetstrokecolor{currentstroke}%
\pgfsetstrokeopacity{0.657340}%
\pgfsetdash{}{0pt}%
\pgfpathmoveto{\pgfqpoint{0.864588in}{1.422743in}}%
\pgfpathcurveto{\pgfqpoint{0.872825in}{1.422743in}}{\pgfqpoint{0.880725in}{1.426015in}}{\pgfqpoint{0.886548in}{1.431839in}}%
\pgfpathcurveto{\pgfqpoint{0.892372in}{1.437663in}}{\pgfqpoint{0.895645in}{1.445563in}}{\pgfqpoint{0.895645in}{1.453799in}}%
\pgfpathcurveto{\pgfqpoint{0.895645in}{1.462035in}}{\pgfqpoint{0.892372in}{1.469935in}}{\pgfqpoint{0.886548in}{1.475759in}}%
\pgfpathcurveto{\pgfqpoint{0.880725in}{1.481583in}}{\pgfqpoint{0.872825in}{1.484856in}}{\pgfqpoint{0.864588in}{1.484856in}}%
\pgfpathcurveto{\pgfqpoint{0.856352in}{1.484856in}}{\pgfqpoint{0.848452in}{1.481583in}}{\pgfqpoint{0.842628in}{1.475759in}}%
\pgfpathcurveto{\pgfqpoint{0.836804in}{1.469935in}}{\pgfqpoint{0.833532in}{1.462035in}}{\pgfqpoint{0.833532in}{1.453799in}}%
\pgfpathcurveto{\pgfqpoint{0.833532in}{1.445563in}}{\pgfqpoint{0.836804in}{1.437663in}}{\pgfqpoint{0.842628in}{1.431839in}}%
\pgfpathcurveto{\pgfqpoint{0.848452in}{1.426015in}}{\pgfqpoint{0.856352in}{1.422743in}}{\pgfqpoint{0.864588in}{1.422743in}}%
\pgfpathclose%
\pgfusepath{stroke,fill}%
\end{pgfscope}%
\begin{pgfscope}%
\pgfpathrectangle{\pgfqpoint{0.100000in}{0.220728in}}{\pgfqpoint{3.696000in}{3.696000in}}%
\pgfusepath{clip}%
\pgfsetbuttcap%
\pgfsetroundjoin%
\definecolor{currentfill}{rgb}{0.121569,0.466667,0.705882}%
\pgfsetfillcolor{currentfill}%
\pgfsetfillopacity{0.657340}%
\pgfsetlinewidth{1.003750pt}%
\definecolor{currentstroke}{rgb}{0.121569,0.466667,0.705882}%
\pgfsetstrokecolor{currentstroke}%
\pgfsetstrokeopacity{0.657340}%
\pgfsetdash{}{0pt}%
\pgfpathmoveto{\pgfqpoint{0.864588in}{1.422743in}}%
\pgfpathcurveto{\pgfqpoint{0.872825in}{1.422743in}}{\pgfqpoint{0.880725in}{1.426015in}}{\pgfqpoint{0.886548in}{1.431839in}}%
\pgfpathcurveto{\pgfqpoint{0.892372in}{1.437663in}}{\pgfqpoint{0.895645in}{1.445563in}}{\pgfqpoint{0.895645in}{1.453799in}}%
\pgfpathcurveto{\pgfqpoint{0.895645in}{1.462035in}}{\pgfqpoint{0.892372in}{1.469935in}}{\pgfqpoint{0.886548in}{1.475759in}}%
\pgfpathcurveto{\pgfqpoint{0.880725in}{1.481583in}}{\pgfqpoint{0.872825in}{1.484856in}}{\pgfqpoint{0.864588in}{1.484856in}}%
\pgfpathcurveto{\pgfqpoint{0.856352in}{1.484856in}}{\pgfqpoint{0.848452in}{1.481583in}}{\pgfqpoint{0.842628in}{1.475759in}}%
\pgfpathcurveto{\pgfqpoint{0.836804in}{1.469935in}}{\pgfqpoint{0.833532in}{1.462035in}}{\pgfqpoint{0.833532in}{1.453799in}}%
\pgfpathcurveto{\pgfqpoint{0.833532in}{1.445563in}}{\pgfqpoint{0.836804in}{1.437663in}}{\pgfqpoint{0.842628in}{1.431839in}}%
\pgfpathcurveto{\pgfqpoint{0.848452in}{1.426015in}}{\pgfqpoint{0.856352in}{1.422743in}}{\pgfqpoint{0.864588in}{1.422743in}}%
\pgfpathclose%
\pgfusepath{stroke,fill}%
\end{pgfscope}%
\begin{pgfscope}%
\pgfpathrectangle{\pgfqpoint{0.100000in}{0.220728in}}{\pgfqpoint{3.696000in}{3.696000in}}%
\pgfusepath{clip}%
\pgfsetbuttcap%
\pgfsetroundjoin%
\definecolor{currentfill}{rgb}{0.121569,0.466667,0.705882}%
\pgfsetfillcolor{currentfill}%
\pgfsetfillopacity{0.657340}%
\pgfsetlinewidth{1.003750pt}%
\definecolor{currentstroke}{rgb}{0.121569,0.466667,0.705882}%
\pgfsetstrokecolor{currentstroke}%
\pgfsetstrokeopacity{0.657340}%
\pgfsetdash{}{0pt}%
\pgfpathmoveto{\pgfqpoint{0.864588in}{1.422743in}}%
\pgfpathcurveto{\pgfqpoint{0.872825in}{1.422743in}}{\pgfqpoint{0.880725in}{1.426015in}}{\pgfqpoint{0.886548in}{1.431839in}}%
\pgfpathcurveto{\pgfqpoint{0.892372in}{1.437663in}}{\pgfqpoint{0.895645in}{1.445563in}}{\pgfqpoint{0.895645in}{1.453799in}}%
\pgfpathcurveto{\pgfqpoint{0.895645in}{1.462035in}}{\pgfqpoint{0.892372in}{1.469935in}}{\pgfqpoint{0.886548in}{1.475759in}}%
\pgfpathcurveto{\pgfqpoint{0.880725in}{1.481583in}}{\pgfqpoint{0.872825in}{1.484856in}}{\pgfqpoint{0.864588in}{1.484856in}}%
\pgfpathcurveto{\pgfqpoint{0.856352in}{1.484856in}}{\pgfqpoint{0.848452in}{1.481583in}}{\pgfqpoint{0.842628in}{1.475759in}}%
\pgfpathcurveto{\pgfqpoint{0.836804in}{1.469935in}}{\pgfqpoint{0.833532in}{1.462035in}}{\pgfqpoint{0.833532in}{1.453799in}}%
\pgfpathcurveto{\pgfqpoint{0.833532in}{1.445563in}}{\pgfqpoint{0.836804in}{1.437663in}}{\pgfqpoint{0.842628in}{1.431839in}}%
\pgfpathcurveto{\pgfqpoint{0.848452in}{1.426015in}}{\pgfqpoint{0.856352in}{1.422743in}}{\pgfqpoint{0.864588in}{1.422743in}}%
\pgfpathclose%
\pgfusepath{stroke,fill}%
\end{pgfscope}%
\begin{pgfscope}%
\pgfpathrectangle{\pgfqpoint{0.100000in}{0.220728in}}{\pgfqpoint{3.696000in}{3.696000in}}%
\pgfusepath{clip}%
\pgfsetbuttcap%
\pgfsetroundjoin%
\definecolor{currentfill}{rgb}{0.121569,0.466667,0.705882}%
\pgfsetfillcolor{currentfill}%
\pgfsetfillopacity{0.657340}%
\pgfsetlinewidth{1.003750pt}%
\definecolor{currentstroke}{rgb}{0.121569,0.466667,0.705882}%
\pgfsetstrokecolor{currentstroke}%
\pgfsetstrokeopacity{0.657340}%
\pgfsetdash{}{0pt}%
\pgfpathmoveto{\pgfqpoint{0.864588in}{1.422743in}}%
\pgfpathcurveto{\pgfqpoint{0.872825in}{1.422743in}}{\pgfqpoint{0.880725in}{1.426015in}}{\pgfqpoint{0.886548in}{1.431839in}}%
\pgfpathcurveto{\pgfqpoint{0.892372in}{1.437663in}}{\pgfqpoint{0.895645in}{1.445563in}}{\pgfqpoint{0.895645in}{1.453799in}}%
\pgfpathcurveto{\pgfqpoint{0.895645in}{1.462035in}}{\pgfqpoint{0.892372in}{1.469935in}}{\pgfqpoint{0.886548in}{1.475759in}}%
\pgfpathcurveto{\pgfqpoint{0.880725in}{1.481583in}}{\pgfqpoint{0.872825in}{1.484856in}}{\pgfqpoint{0.864588in}{1.484856in}}%
\pgfpathcurveto{\pgfqpoint{0.856352in}{1.484856in}}{\pgfqpoint{0.848452in}{1.481583in}}{\pgfqpoint{0.842628in}{1.475759in}}%
\pgfpathcurveto{\pgfqpoint{0.836804in}{1.469935in}}{\pgfqpoint{0.833532in}{1.462035in}}{\pgfqpoint{0.833532in}{1.453799in}}%
\pgfpathcurveto{\pgfqpoint{0.833532in}{1.445563in}}{\pgfqpoint{0.836804in}{1.437663in}}{\pgfqpoint{0.842628in}{1.431839in}}%
\pgfpathcurveto{\pgfqpoint{0.848452in}{1.426015in}}{\pgfqpoint{0.856352in}{1.422743in}}{\pgfqpoint{0.864588in}{1.422743in}}%
\pgfpathclose%
\pgfusepath{stroke,fill}%
\end{pgfscope}%
\begin{pgfscope}%
\pgfpathrectangle{\pgfqpoint{0.100000in}{0.220728in}}{\pgfqpoint{3.696000in}{3.696000in}}%
\pgfusepath{clip}%
\pgfsetbuttcap%
\pgfsetroundjoin%
\definecolor{currentfill}{rgb}{0.121569,0.466667,0.705882}%
\pgfsetfillcolor{currentfill}%
\pgfsetfillopacity{0.657340}%
\pgfsetlinewidth{1.003750pt}%
\definecolor{currentstroke}{rgb}{0.121569,0.466667,0.705882}%
\pgfsetstrokecolor{currentstroke}%
\pgfsetstrokeopacity{0.657340}%
\pgfsetdash{}{0pt}%
\pgfpathmoveto{\pgfqpoint{0.864588in}{1.422743in}}%
\pgfpathcurveto{\pgfqpoint{0.872825in}{1.422743in}}{\pgfqpoint{0.880725in}{1.426015in}}{\pgfqpoint{0.886548in}{1.431839in}}%
\pgfpathcurveto{\pgfqpoint{0.892372in}{1.437663in}}{\pgfqpoint{0.895645in}{1.445563in}}{\pgfqpoint{0.895645in}{1.453799in}}%
\pgfpathcurveto{\pgfqpoint{0.895645in}{1.462035in}}{\pgfqpoint{0.892372in}{1.469935in}}{\pgfqpoint{0.886548in}{1.475759in}}%
\pgfpathcurveto{\pgfqpoint{0.880725in}{1.481583in}}{\pgfqpoint{0.872825in}{1.484856in}}{\pgfqpoint{0.864588in}{1.484856in}}%
\pgfpathcurveto{\pgfqpoint{0.856352in}{1.484856in}}{\pgfqpoint{0.848452in}{1.481583in}}{\pgfqpoint{0.842628in}{1.475759in}}%
\pgfpathcurveto{\pgfqpoint{0.836804in}{1.469935in}}{\pgfqpoint{0.833532in}{1.462035in}}{\pgfqpoint{0.833532in}{1.453799in}}%
\pgfpathcurveto{\pgfqpoint{0.833532in}{1.445563in}}{\pgfqpoint{0.836804in}{1.437663in}}{\pgfqpoint{0.842628in}{1.431839in}}%
\pgfpathcurveto{\pgfqpoint{0.848452in}{1.426015in}}{\pgfqpoint{0.856352in}{1.422743in}}{\pgfqpoint{0.864588in}{1.422743in}}%
\pgfpathclose%
\pgfusepath{stroke,fill}%
\end{pgfscope}%
\begin{pgfscope}%
\pgfpathrectangle{\pgfqpoint{0.100000in}{0.220728in}}{\pgfqpoint{3.696000in}{3.696000in}}%
\pgfusepath{clip}%
\pgfsetbuttcap%
\pgfsetroundjoin%
\definecolor{currentfill}{rgb}{0.121569,0.466667,0.705882}%
\pgfsetfillcolor{currentfill}%
\pgfsetfillopacity{0.657340}%
\pgfsetlinewidth{1.003750pt}%
\definecolor{currentstroke}{rgb}{0.121569,0.466667,0.705882}%
\pgfsetstrokecolor{currentstroke}%
\pgfsetstrokeopacity{0.657340}%
\pgfsetdash{}{0pt}%
\pgfpathmoveto{\pgfqpoint{0.864588in}{1.422743in}}%
\pgfpathcurveto{\pgfqpoint{0.872825in}{1.422743in}}{\pgfqpoint{0.880725in}{1.426015in}}{\pgfqpoint{0.886548in}{1.431839in}}%
\pgfpathcurveto{\pgfqpoint{0.892372in}{1.437663in}}{\pgfqpoint{0.895645in}{1.445563in}}{\pgfqpoint{0.895645in}{1.453799in}}%
\pgfpathcurveto{\pgfqpoint{0.895645in}{1.462035in}}{\pgfqpoint{0.892372in}{1.469935in}}{\pgfqpoint{0.886548in}{1.475759in}}%
\pgfpathcurveto{\pgfqpoint{0.880725in}{1.481583in}}{\pgfqpoint{0.872825in}{1.484856in}}{\pgfqpoint{0.864588in}{1.484856in}}%
\pgfpathcurveto{\pgfqpoint{0.856352in}{1.484856in}}{\pgfqpoint{0.848452in}{1.481583in}}{\pgfqpoint{0.842628in}{1.475759in}}%
\pgfpathcurveto{\pgfqpoint{0.836804in}{1.469935in}}{\pgfqpoint{0.833532in}{1.462035in}}{\pgfqpoint{0.833532in}{1.453799in}}%
\pgfpathcurveto{\pgfqpoint{0.833532in}{1.445563in}}{\pgfqpoint{0.836804in}{1.437663in}}{\pgfqpoint{0.842628in}{1.431839in}}%
\pgfpathcurveto{\pgfqpoint{0.848452in}{1.426015in}}{\pgfqpoint{0.856352in}{1.422743in}}{\pgfqpoint{0.864588in}{1.422743in}}%
\pgfpathclose%
\pgfusepath{stroke,fill}%
\end{pgfscope}%
\begin{pgfscope}%
\pgfpathrectangle{\pgfqpoint{0.100000in}{0.220728in}}{\pgfqpoint{3.696000in}{3.696000in}}%
\pgfusepath{clip}%
\pgfsetbuttcap%
\pgfsetroundjoin%
\definecolor{currentfill}{rgb}{0.121569,0.466667,0.705882}%
\pgfsetfillcolor{currentfill}%
\pgfsetfillopacity{0.657340}%
\pgfsetlinewidth{1.003750pt}%
\definecolor{currentstroke}{rgb}{0.121569,0.466667,0.705882}%
\pgfsetstrokecolor{currentstroke}%
\pgfsetstrokeopacity{0.657340}%
\pgfsetdash{}{0pt}%
\pgfpathmoveto{\pgfqpoint{0.864588in}{1.422743in}}%
\pgfpathcurveto{\pgfqpoint{0.872825in}{1.422743in}}{\pgfqpoint{0.880725in}{1.426015in}}{\pgfqpoint{0.886548in}{1.431839in}}%
\pgfpathcurveto{\pgfqpoint{0.892372in}{1.437663in}}{\pgfqpoint{0.895645in}{1.445563in}}{\pgfqpoint{0.895645in}{1.453799in}}%
\pgfpathcurveto{\pgfqpoint{0.895645in}{1.462035in}}{\pgfqpoint{0.892372in}{1.469935in}}{\pgfqpoint{0.886548in}{1.475759in}}%
\pgfpathcurveto{\pgfqpoint{0.880725in}{1.481583in}}{\pgfqpoint{0.872825in}{1.484856in}}{\pgfqpoint{0.864588in}{1.484856in}}%
\pgfpathcurveto{\pgfqpoint{0.856352in}{1.484856in}}{\pgfqpoint{0.848452in}{1.481583in}}{\pgfqpoint{0.842628in}{1.475759in}}%
\pgfpathcurveto{\pgfqpoint{0.836804in}{1.469935in}}{\pgfqpoint{0.833532in}{1.462035in}}{\pgfqpoint{0.833532in}{1.453799in}}%
\pgfpathcurveto{\pgfqpoint{0.833532in}{1.445563in}}{\pgfqpoint{0.836804in}{1.437663in}}{\pgfqpoint{0.842628in}{1.431839in}}%
\pgfpathcurveto{\pgfqpoint{0.848452in}{1.426015in}}{\pgfqpoint{0.856352in}{1.422743in}}{\pgfqpoint{0.864588in}{1.422743in}}%
\pgfpathclose%
\pgfusepath{stroke,fill}%
\end{pgfscope}%
\begin{pgfscope}%
\pgfpathrectangle{\pgfqpoint{0.100000in}{0.220728in}}{\pgfqpoint{3.696000in}{3.696000in}}%
\pgfusepath{clip}%
\pgfsetbuttcap%
\pgfsetroundjoin%
\definecolor{currentfill}{rgb}{0.121569,0.466667,0.705882}%
\pgfsetfillcolor{currentfill}%
\pgfsetfillopacity{0.657340}%
\pgfsetlinewidth{1.003750pt}%
\definecolor{currentstroke}{rgb}{0.121569,0.466667,0.705882}%
\pgfsetstrokecolor{currentstroke}%
\pgfsetstrokeopacity{0.657340}%
\pgfsetdash{}{0pt}%
\pgfpathmoveto{\pgfqpoint{0.864588in}{1.422743in}}%
\pgfpathcurveto{\pgfqpoint{0.872825in}{1.422743in}}{\pgfqpoint{0.880725in}{1.426015in}}{\pgfqpoint{0.886548in}{1.431839in}}%
\pgfpathcurveto{\pgfqpoint{0.892372in}{1.437663in}}{\pgfqpoint{0.895645in}{1.445563in}}{\pgfqpoint{0.895645in}{1.453799in}}%
\pgfpathcurveto{\pgfqpoint{0.895645in}{1.462035in}}{\pgfqpoint{0.892372in}{1.469935in}}{\pgfqpoint{0.886548in}{1.475759in}}%
\pgfpathcurveto{\pgfqpoint{0.880725in}{1.481583in}}{\pgfqpoint{0.872825in}{1.484856in}}{\pgfqpoint{0.864588in}{1.484856in}}%
\pgfpathcurveto{\pgfqpoint{0.856352in}{1.484856in}}{\pgfqpoint{0.848452in}{1.481583in}}{\pgfqpoint{0.842628in}{1.475759in}}%
\pgfpathcurveto{\pgfqpoint{0.836804in}{1.469935in}}{\pgfqpoint{0.833532in}{1.462035in}}{\pgfqpoint{0.833532in}{1.453799in}}%
\pgfpathcurveto{\pgfqpoint{0.833532in}{1.445563in}}{\pgfqpoint{0.836804in}{1.437663in}}{\pgfqpoint{0.842628in}{1.431839in}}%
\pgfpathcurveto{\pgfqpoint{0.848452in}{1.426015in}}{\pgfqpoint{0.856352in}{1.422743in}}{\pgfqpoint{0.864588in}{1.422743in}}%
\pgfpathclose%
\pgfusepath{stroke,fill}%
\end{pgfscope}%
\begin{pgfscope}%
\pgfpathrectangle{\pgfqpoint{0.100000in}{0.220728in}}{\pgfqpoint{3.696000in}{3.696000in}}%
\pgfusepath{clip}%
\pgfsetbuttcap%
\pgfsetroundjoin%
\definecolor{currentfill}{rgb}{0.121569,0.466667,0.705882}%
\pgfsetfillcolor{currentfill}%
\pgfsetfillopacity{0.657340}%
\pgfsetlinewidth{1.003750pt}%
\definecolor{currentstroke}{rgb}{0.121569,0.466667,0.705882}%
\pgfsetstrokecolor{currentstroke}%
\pgfsetstrokeopacity{0.657340}%
\pgfsetdash{}{0pt}%
\pgfpathmoveto{\pgfqpoint{0.864588in}{1.422743in}}%
\pgfpathcurveto{\pgfqpoint{0.872825in}{1.422743in}}{\pgfqpoint{0.880725in}{1.426015in}}{\pgfqpoint{0.886548in}{1.431839in}}%
\pgfpathcurveto{\pgfqpoint{0.892372in}{1.437663in}}{\pgfqpoint{0.895645in}{1.445563in}}{\pgfqpoint{0.895645in}{1.453799in}}%
\pgfpathcurveto{\pgfqpoint{0.895645in}{1.462035in}}{\pgfqpoint{0.892372in}{1.469935in}}{\pgfqpoint{0.886548in}{1.475759in}}%
\pgfpathcurveto{\pgfqpoint{0.880725in}{1.481583in}}{\pgfqpoint{0.872825in}{1.484856in}}{\pgfqpoint{0.864588in}{1.484856in}}%
\pgfpathcurveto{\pgfqpoint{0.856352in}{1.484856in}}{\pgfqpoint{0.848452in}{1.481583in}}{\pgfqpoint{0.842628in}{1.475759in}}%
\pgfpathcurveto{\pgfqpoint{0.836804in}{1.469935in}}{\pgfqpoint{0.833532in}{1.462035in}}{\pgfqpoint{0.833532in}{1.453799in}}%
\pgfpathcurveto{\pgfqpoint{0.833532in}{1.445563in}}{\pgfqpoint{0.836804in}{1.437663in}}{\pgfqpoint{0.842628in}{1.431839in}}%
\pgfpathcurveto{\pgfqpoint{0.848452in}{1.426015in}}{\pgfqpoint{0.856352in}{1.422743in}}{\pgfqpoint{0.864588in}{1.422743in}}%
\pgfpathclose%
\pgfusepath{stroke,fill}%
\end{pgfscope}%
\begin{pgfscope}%
\pgfpathrectangle{\pgfqpoint{0.100000in}{0.220728in}}{\pgfqpoint{3.696000in}{3.696000in}}%
\pgfusepath{clip}%
\pgfsetbuttcap%
\pgfsetroundjoin%
\definecolor{currentfill}{rgb}{0.121569,0.466667,0.705882}%
\pgfsetfillcolor{currentfill}%
\pgfsetfillopacity{0.657340}%
\pgfsetlinewidth{1.003750pt}%
\definecolor{currentstroke}{rgb}{0.121569,0.466667,0.705882}%
\pgfsetstrokecolor{currentstroke}%
\pgfsetstrokeopacity{0.657340}%
\pgfsetdash{}{0pt}%
\pgfpathmoveto{\pgfqpoint{0.864588in}{1.422743in}}%
\pgfpathcurveto{\pgfqpoint{0.872825in}{1.422743in}}{\pgfqpoint{0.880725in}{1.426015in}}{\pgfqpoint{0.886548in}{1.431839in}}%
\pgfpathcurveto{\pgfqpoint{0.892372in}{1.437663in}}{\pgfqpoint{0.895645in}{1.445563in}}{\pgfqpoint{0.895645in}{1.453799in}}%
\pgfpathcurveto{\pgfqpoint{0.895645in}{1.462035in}}{\pgfqpoint{0.892372in}{1.469935in}}{\pgfqpoint{0.886548in}{1.475759in}}%
\pgfpathcurveto{\pgfqpoint{0.880725in}{1.481583in}}{\pgfqpoint{0.872825in}{1.484856in}}{\pgfqpoint{0.864588in}{1.484856in}}%
\pgfpathcurveto{\pgfqpoint{0.856352in}{1.484856in}}{\pgfqpoint{0.848452in}{1.481583in}}{\pgfqpoint{0.842628in}{1.475759in}}%
\pgfpathcurveto{\pgfqpoint{0.836804in}{1.469935in}}{\pgfqpoint{0.833532in}{1.462035in}}{\pgfqpoint{0.833532in}{1.453799in}}%
\pgfpathcurveto{\pgfqpoint{0.833532in}{1.445563in}}{\pgfqpoint{0.836804in}{1.437663in}}{\pgfqpoint{0.842628in}{1.431839in}}%
\pgfpathcurveto{\pgfqpoint{0.848452in}{1.426015in}}{\pgfqpoint{0.856352in}{1.422743in}}{\pgfqpoint{0.864588in}{1.422743in}}%
\pgfpathclose%
\pgfusepath{stroke,fill}%
\end{pgfscope}%
\begin{pgfscope}%
\pgfpathrectangle{\pgfqpoint{0.100000in}{0.220728in}}{\pgfqpoint{3.696000in}{3.696000in}}%
\pgfusepath{clip}%
\pgfsetbuttcap%
\pgfsetroundjoin%
\definecolor{currentfill}{rgb}{0.121569,0.466667,0.705882}%
\pgfsetfillcolor{currentfill}%
\pgfsetfillopacity{0.657340}%
\pgfsetlinewidth{1.003750pt}%
\definecolor{currentstroke}{rgb}{0.121569,0.466667,0.705882}%
\pgfsetstrokecolor{currentstroke}%
\pgfsetstrokeopacity{0.657340}%
\pgfsetdash{}{0pt}%
\pgfpathmoveto{\pgfqpoint{0.864588in}{1.422743in}}%
\pgfpathcurveto{\pgfqpoint{0.872825in}{1.422743in}}{\pgfqpoint{0.880725in}{1.426015in}}{\pgfqpoint{0.886548in}{1.431839in}}%
\pgfpathcurveto{\pgfqpoint{0.892372in}{1.437663in}}{\pgfqpoint{0.895645in}{1.445563in}}{\pgfqpoint{0.895645in}{1.453799in}}%
\pgfpathcurveto{\pgfqpoint{0.895645in}{1.462035in}}{\pgfqpoint{0.892372in}{1.469935in}}{\pgfqpoint{0.886548in}{1.475759in}}%
\pgfpathcurveto{\pgfqpoint{0.880725in}{1.481583in}}{\pgfqpoint{0.872825in}{1.484856in}}{\pgfqpoint{0.864588in}{1.484856in}}%
\pgfpathcurveto{\pgfqpoint{0.856352in}{1.484856in}}{\pgfqpoint{0.848452in}{1.481583in}}{\pgfqpoint{0.842628in}{1.475759in}}%
\pgfpathcurveto{\pgfqpoint{0.836804in}{1.469935in}}{\pgfqpoint{0.833532in}{1.462035in}}{\pgfqpoint{0.833532in}{1.453799in}}%
\pgfpathcurveto{\pgfqpoint{0.833532in}{1.445563in}}{\pgfqpoint{0.836804in}{1.437663in}}{\pgfqpoint{0.842628in}{1.431839in}}%
\pgfpathcurveto{\pgfqpoint{0.848452in}{1.426015in}}{\pgfqpoint{0.856352in}{1.422743in}}{\pgfqpoint{0.864588in}{1.422743in}}%
\pgfpathclose%
\pgfusepath{stroke,fill}%
\end{pgfscope}%
\begin{pgfscope}%
\pgfpathrectangle{\pgfqpoint{0.100000in}{0.220728in}}{\pgfqpoint{3.696000in}{3.696000in}}%
\pgfusepath{clip}%
\pgfsetbuttcap%
\pgfsetroundjoin%
\definecolor{currentfill}{rgb}{0.121569,0.466667,0.705882}%
\pgfsetfillcolor{currentfill}%
\pgfsetfillopacity{0.657340}%
\pgfsetlinewidth{1.003750pt}%
\definecolor{currentstroke}{rgb}{0.121569,0.466667,0.705882}%
\pgfsetstrokecolor{currentstroke}%
\pgfsetstrokeopacity{0.657340}%
\pgfsetdash{}{0pt}%
\pgfpathmoveto{\pgfqpoint{0.864588in}{1.422743in}}%
\pgfpathcurveto{\pgfqpoint{0.872825in}{1.422743in}}{\pgfqpoint{0.880725in}{1.426015in}}{\pgfqpoint{0.886548in}{1.431839in}}%
\pgfpathcurveto{\pgfqpoint{0.892372in}{1.437663in}}{\pgfqpoint{0.895645in}{1.445563in}}{\pgfqpoint{0.895645in}{1.453799in}}%
\pgfpathcurveto{\pgfqpoint{0.895645in}{1.462035in}}{\pgfqpoint{0.892372in}{1.469935in}}{\pgfqpoint{0.886548in}{1.475759in}}%
\pgfpathcurveto{\pgfqpoint{0.880725in}{1.481583in}}{\pgfqpoint{0.872825in}{1.484856in}}{\pgfqpoint{0.864588in}{1.484856in}}%
\pgfpathcurveto{\pgfqpoint{0.856352in}{1.484856in}}{\pgfqpoint{0.848452in}{1.481583in}}{\pgfqpoint{0.842628in}{1.475759in}}%
\pgfpathcurveto{\pgfqpoint{0.836804in}{1.469935in}}{\pgfqpoint{0.833532in}{1.462035in}}{\pgfqpoint{0.833532in}{1.453799in}}%
\pgfpathcurveto{\pgfqpoint{0.833532in}{1.445563in}}{\pgfqpoint{0.836804in}{1.437663in}}{\pgfqpoint{0.842628in}{1.431839in}}%
\pgfpathcurveto{\pgfqpoint{0.848452in}{1.426015in}}{\pgfqpoint{0.856352in}{1.422743in}}{\pgfqpoint{0.864588in}{1.422743in}}%
\pgfpathclose%
\pgfusepath{stroke,fill}%
\end{pgfscope}%
\begin{pgfscope}%
\pgfpathrectangle{\pgfqpoint{0.100000in}{0.220728in}}{\pgfqpoint{3.696000in}{3.696000in}}%
\pgfusepath{clip}%
\pgfsetbuttcap%
\pgfsetroundjoin%
\definecolor{currentfill}{rgb}{0.121569,0.466667,0.705882}%
\pgfsetfillcolor{currentfill}%
\pgfsetfillopacity{0.657340}%
\pgfsetlinewidth{1.003750pt}%
\definecolor{currentstroke}{rgb}{0.121569,0.466667,0.705882}%
\pgfsetstrokecolor{currentstroke}%
\pgfsetstrokeopacity{0.657340}%
\pgfsetdash{}{0pt}%
\pgfpathmoveto{\pgfqpoint{0.864588in}{1.422743in}}%
\pgfpathcurveto{\pgfqpoint{0.872825in}{1.422743in}}{\pgfqpoint{0.880725in}{1.426015in}}{\pgfqpoint{0.886548in}{1.431839in}}%
\pgfpathcurveto{\pgfqpoint{0.892372in}{1.437663in}}{\pgfqpoint{0.895645in}{1.445563in}}{\pgfqpoint{0.895645in}{1.453799in}}%
\pgfpathcurveto{\pgfqpoint{0.895645in}{1.462035in}}{\pgfqpoint{0.892372in}{1.469935in}}{\pgfqpoint{0.886548in}{1.475759in}}%
\pgfpathcurveto{\pgfqpoint{0.880725in}{1.481583in}}{\pgfqpoint{0.872825in}{1.484856in}}{\pgfqpoint{0.864588in}{1.484856in}}%
\pgfpathcurveto{\pgfqpoint{0.856352in}{1.484856in}}{\pgfqpoint{0.848452in}{1.481583in}}{\pgfqpoint{0.842628in}{1.475759in}}%
\pgfpathcurveto{\pgfqpoint{0.836804in}{1.469935in}}{\pgfqpoint{0.833532in}{1.462035in}}{\pgfqpoint{0.833532in}{1.453799in}}%
\pgfpathcurveto{\pgfqpoint{0.833532in}{1.445563in}}{\pgfqpoint{0.836804in}{1.437663in}}{\pgfqpoint{0.842628in}{1.431839in}}%
\pgfpathcurveto{\pgfqpoint{0.848452in}{1.426015in}}{\pgfqpoint{0.856352in}{1.422743in}}{\pgfqpoint{0.864588in}{1.422743in}}%
\pgfpathclose%
\pgfusepath{stroke,fill}%
\end{pgfscope}%
\begin{pgfscope}%
\pgfpathrectangle{\pgfqpoint{0.100000in}{0.220728in}}{\pgfqpoint{3.696000in}{3.696000in}}%
\pgfusepath{clip}%
\pgfsetbuttcap%
\pgfsetroundjoin%
\definecolor{currentfill}{rgb}{0.121569,0.466667,0.705882}%
\pgfsetfillcolor{currentfill}%
\pgfsetfillopacity{0.657340}%
\pgfsetlinewidth{1.003750pt}%
\definecolor{currentstroke}{rgb}{0.121569,0.466667,0.705882}%
\pgfsetstrokecolor{currentstroke}%
\pgfsetstrokeopacity{0.657340}%
\pgfsetdash{}{0pt}%
\pgfpathmoveto{\pgfqpoint{0.864588in}{1.422743in}}%
\pgfpathcurveto{\pgfqpoint{0.872825in}{1.422743in}}{\pgfqpoint{0.880725in}{1.426015in}}{\pgfqpoint{0.886548in}{1.431839in}}%
\pgfpathcurveto{\pgfqpoint{0.892372in}{1.437663in}}{\pgfqpoint{0.895645in}{1.445563in}}{\pgfqpoint{0.895645in}{1.453799in}}%
\pgfpathcurveto{\pgfqpoint{0.895645in}{1.462035in}}{\pgfqpoint{0.892372in}{1.469935in}}{\pgfqpoint{0.886548in}{1.475759in}}%
\pgfpathcurveto{\pgfqpoint{0.880725in}{1.481583in}}{\pgfqpoint{0.872825in}{1.484856in}}{\pgfqpoint{0.864588in}{1.484856in}}%
\pgfpathcurveto{\pgfqpoint{0.856352in}{1.484856in}}{\pgfqpoint{0.848452in}{1.481583in}}{\pgfqpoint{0.842628in}{1.475759in}}%
\pgfpathcurveto{\pgfqpoint{0.836804in}{1.469935in}}{\pgfqpoint{0.833532in}{1.462035in}}{\pgfqpoint{0.833532in}{1.453799in}}%
\pgfpathcurveto{\pgfqpoint{0.833532in}{1.445563in}}{\pgfqpoint{0.836804in}{1.437663in}}{\pgfqpoint{0.842628in}{1.431839in}}%
\pgfpathcurveto{\pgfqpoint{0.848452in}{1.426015in}}{\pgfqpoint{0.856352in}{1.422743in}}{\pgfqpoint{0.864588in}{1.422743in}}%
\pgfpathclose%
\pgfusepath{stroke,fill}%
\end{pgfscope}%
\begin{pgfscope}%
\pgfpathrectangle{\pgfqpoint{0.100000in}{0.220728in}}{\pgfqpoint{3.696000in}{3.696000in}}%
\pgfusepath{clip}%
\pgfsetbuttcap%
\pgfsetroundjoin%
\definecolor{currentfill}{rgb}{0.121569,0.466667,0.705882}%
\pgfsetfillcolor{currentfill}%
\pgfsetfillopacity{0.657340}%
\pgfsetlinewidth{1.003750pt}%
\definecolor{currentstroke}{rgb}{0.121569,0.466667,0.705882}%
\pgfsetstrokecolor{currentstroke}%
\pgfsetstrokeopacity{0.657340}%
\pgfsetdash{}{0pt}%
\pgfpathmoveto{\pgfqpoint{0.864588in}{1.422743in}}%
\pgfpathcurveto{\pgfqpoint{0.872825in}{1.422743in}}{\pgfqpoint{0.880725in}{1.426015in}}{\pgfqpoint{0.886548in}{1.431839in}}%
\pgfpathcurveto{\pgfqpoint{0.892372in}{1.437663in}}{\pgfqpoint{0.895645in}{1.445563in}}{\pgfqpoint{0.895645in}{1.453799in}}%
\pgfpathcurveto{\pgfqpoint{0.895645in}{1.462035in}}{\pgfqpoint{0.892372in}{1.469935in}}{\pgfqpoint{0.886548in}{1.475759in}}%
\pgfpathcurveto{\pgfqpoint{0.880725in}{1.481583in}}{\pgfqpoint{0.872825in}{1.484856in}}{\pgfqpoint{0.864588in}{1.484856in}}%
\pgfpathcurveto{\pgfqpoint{0.856352in}{1.484856in}}{\pgfqpoint{0.848452in}{1.481583in}}{\pgfqpoint{0.842628in}{1.475759in}}%
\pgfpathcurveto{\pgfqpoint{0.836804in}{1.469935in}}{\pgfqpoint{0.833532in}{1.462035in}}{\pgfqpoint{0.833532in}{1.453799in}}%
\pgfpathcurveto{\pgfqpoint{0.833532in}{1.445563in}}{\pgfqpoint{0.836804in}{1.437663in}}{\pgfqpoint{0.842628in}{1.431839in}}%
\pgfpathcurveto{\pgfqpoint{0.848452in}{1.426015in}}{\pgfqpoint{0.856352in}{1.422743in}}{\pgfqpoint{0.864588in}{1.422743in}}%
\pgfpathclose%
\pgfusepath{stroke,fill}%
\end{pgfscope}%
\begin{pgfscope}%
\pgfpathrectangle{\pgfqpoint{0.100000in}{0.220728in}}{\pgfqpoint{3.696000in}{3.696000in}}%
\pgfusepath{clip}%
\pgfsetbuttcap%
\pgfsetroundjoin%
\definecolor{currentfill}{rgb}{0.121569,0.466667,0.705882}%
\pgfsetfillcolor{currentfill}%
\pgfsetfillopacity{0.657340}%
\pgfsetlinewidth{1.003750pt}%
\definecolor{currentstroke}{rgb}{0.121569,0.466667,0.705882}%
\pgfsetstrokecolor{currentstroke}%
\pgfsetstrokeopacity{0.657340}%
\pgfsetdash{}{0pt}%
\pgfpathmoveto{\pgfqpoint{0.864588in}{1.422743in}}%
\pgfpathcurveto{\pgfqpoint{0.872825in}{1.422743in}}{\pgfqpoint{0.880725in}{1.426015in}}{\pgfqpoint{0.886548in}{1.431839in}}%
\pgfpathcurveto{\pgfqpoint{0.892372in}{1.437663in}}{\pgfqpoint{0.895645in}{1.445563in}}{\pgfqpoint{0.895645in}{1.453799in}}%
\pgfpathcurveto{\pgfqpoint{0.895645in}{1.462035in}}{\pgfqpoint{0.892372in}{1.469935in}}{\pgfqpoint{0.886548in}{1.475759in}}%
\pgfpathcurveto{\pgfqpoint{0.880725in}{1.481583in}}{\pgfqpoint{0.872825in}{1.484856in}}{\pgfqpoint{0.864588in}{1.484856in}}%
\pgfpathcurveto{\pgfqpoint{0.856352in}{1.484856in}}{\pgfqpoint{0.848452in}{1.481583in}}{\pgfqpoint{0.842628in}{1.475759in}}%
\pgfpathcurveto{\pgfqpoint{0.836804in}{1.469935in}}{\pgfqpoint{0.833532in}{1.462035in}}{\pgfqpoint{0.833532in}{1.453799in}}%
\pgfpathcurveto{\pgfqpoint{0.833532in}{1.445563in}}{\pgfqpoint{0.836804in}{1.437663in}}{\pgfqpoint{0.842628in}{1.431839in}}%
\pgfpathcurveto{\pgfqpoint{0.848452in}{1.426015in}}{\pgfqpoint{0.856352in}{1.422743in}}{\pgfqpoint{0.864588in}{1.422743in}}%
\pgfpathclose%
\pgfusepath{stroke,fill}%
\end{pgfscope}%
\begin{pgfscope}%
\pgfpathrectangle{\pgfqpoint{0.100000in}{0.220728in}}{\pgfqpoint{3.696000in}{3.696000in}}%
\pgfusepath{clip}%
\pgfsetbuttcap%
\pgfsetroundjoin%
\definecolor{currentfill}{rgb}{0.121569,0.466667,0.705882}%
\pgfsetfillcolor{currentfill}%
\pgfsetfillopacity{0.657340}%
\pgfsetlinewidth{1.003750pt}%
\definecolor{currentstroke}{rgb}{0.121569,0.466667,0.705882}%
\pgfsetstrokecolor{currentstroke}%
\pgfsetstrokeopacity{0.657340}%
\pgfsetdash{}{0pt}%
\pgfpathmoveto{\pgfqpoint{0.864588in}{1.422743in}}%
\pgfpathcurveto{\pgfqpoint{0.872825in}{1.422743in}}{\pgfqpoint{0.880725in}{1.426015in}}{\pgfqpoint{0.886548in}{1.431839in}}%
\pgfpathcurveto{\pgfqpoint{0.892372in}{1.437663in}}{\pgfqpoint{0.895645in}{1.445563in}}{\pgfqpoint{0.895645in}{1.453799in}}%
\pgfpathcurveto{\pgfqpoint{0.895645in}{1.462035in}}{\pgfqpoint{0.892372in}{1.469935in}}{\pgfqpoint{0.886548in}{1.475759in}}%
\pgfpathcurveto{\pgfqpoint{0.880725in}{1.481583in}}{\pgfqpoint{0.872825in}{1.484856in}}{\pgfqpoint{0.864588in}{1.484856in}}%
\pgfpathcurveto{\pgfqpoint{0.856352in}{1.484856in}}{\pgfqpoint{0.848452in}{1.481583in}}{\pgfqpoint{0.842628in}{1.475759in}}%
\pgfpathcurveto{\pgfqpoint{0.836804in}{1.469935in}}{\pgfqpoint{0.833532in}{1.462035in}}{\pgfqpoint{0.833532in}{1.453799in}}%
\pgfpathcurveto{\pgfqpoint{0.833532in}{1.445563in}}{\pgfqpoint{0.836804in}{1.437663in}}{\pgfqpoint{0.842628in}{1.431839in}}%
\pgfpathcurveto{\pgfqpoint{0.848452in}{1.426015in}}{\pgfqpoint{0.856352in}{1.422743in}}{\pgfqpoint{0.864588in}{1.422743in}}%
\pgfpathclose%
\pgfusepath{stroke,fill}%
\end{pgfscope}%
\begin{pgfscope}%
\pgfpathrectangle{\pgfqpoint{0.100000in}{0.220728in}}{\pgfqpoint{3.696000in}{3.696000in}}%
\pgfusepath{clip}%
\pgfsetbuttcap%
\pgfsetroundjoin%
\definecolor{currentfill}{rgb}{0.121569,0.466667,0.705882}%
\pgfsetfillcolor{currentfill}%
\pgfsetfillopacity{0.657340}%
\pgfsetlinewidth{1.003750pt}%
\definecolor{currentstroke}{rgb}{0.121569,0.466667,0.705882}%
\pgfsetstrokecolor{currentstroke}%
\pgfsetstrokeopacity{0.657340}%
\pgfsetdash{}{0pt}%
\pgfpathmoveto{\pgfqpoint{0.864588in}{1.422743in}}%
\pgfpathcurveto{\pgfqpoint{0.872825in}{1.422743in}}{\pgfqpoint{0.880725in}{1.426015in}}{\pgfqpoint{0.886548in}{1.431839in}}%
\pgfpathcurveto{\pgfqpoint{0.892372in}{1.437663in}}{\pgfqpoint{0.895645in}{1.445563in}}{\pgfqpoint{0.895645in}{1.453799in}}%
\pgfpathcurveto{\pgfqpoint{0.895645in}{1.462035in}}{\pgfqpoint{0.892372in}{1.469935in}}{\pgfqpoint{0.886548in}{1.475759in}}%
\pgfpathcurveto{\pgfqpoint{0.880725in}{1.481583in}}{\pgfqpoint{0.872825in}{1.484856in}}{\pgfqpoint{0.864588in}{1.484856in}}%
\pgfpathcurveto{\pgfqpoint{0.856352in}{1.484856in}}{\pgfqpoint{0.848452in}{1.481583in}}{\pgfqpoint{0.842628in}{1.475759in}}%
\pgfpathcurveto{\pgfqpoint{0.836804in}{1.469935in}}{\pgfqpoint{0.833532in}{1.462035in}}{\pgfqpoint{0.833532in}{1.453799in}}%
\pgfpathcurveto{\pgfqpoint{0.833532in}{1.445563in}}{\pgfqpoint{0.836804in}{1.437663in}}{\pgfqpoint{0.842628in}{1.431839in}}%
\pgfpathcurveto{\pgfqpoint{0.848452in}{1.426015in}}{\pgfqpoint{0.856352in}{1.422743in}}{\pgfqpoint{0.864588in}{1.422743in}}%
\pgfpathclose%
\pgfusepath{stroke,fill}%
\end{pgfscope}%
\begin{pgfscope}%
\pgfpathrectangle{\pgfqpoint{0.100000in}{0.220728in}}{\pgfqpoint{3.696000in}{3.696000in}}%
\pgfusepath{clip}%
\pgfsetbuttcap%
\pgfsetroundjoin%
\definecolor{currentfill}{rgb}{0.121569,0.466667,0.705882}%
\pgfsetfillcolor{currentfill}%
\pgfsetfillopacity{0.657340}%
\pgfsetlinewidth{1.003750pt}%
\definecolor{currentstroke}{rgb}{0.121569,0.466667,0.705882}%
\pgfsetstrokecolor{currentstroke}%
\pgfsetstrokeopacity{0.657340}%
\pgfsetdash{}{0pt}%
\pgfpathmoveto{\pgfqpoint{0.864588in}{1.422743in}}%
\pgfpathcurveto{\pgfqpoint{0.872825in}{1.422743in}}{\pgfqpoint{0.880725in}{1.426015in}}{\pgfqpoint{0.886548in}{1.431839in}}%
\pgfpathcurveto{\pgfqpoint{0.892372in}{1.437663in}}{\pgfqpoint{0.895645in}{1.445563in}}{\pgfqpoint{0.895645in}{1.453799in}}%
\pgfpathcurveto{\pgfqpoint{0.895645in}{1.462035in}}{\pgfqpoint{0.892372in}{1.469935in}}{\pgfqpoint{0.886548in}{1.475759in}}%
\pgfpathcurveto{\pgfqpoint{0.880725in}{1.481583in}}{\pgfqpoint{0.872825in}{1.484856in}}{\pgfqpoint{0.864588in}{1.484856in}}%
\pgfpathcurveto{\pgfqpoint{0.856352in}{1.484856in}}{\pgfqpoint{0.848452in}{1.481583in}}{\pgfqpoint{0.842628in}{1.475759in}}%
\pgfpathcurveto{\pgfqpoint{0.836804in}{1.469935in}}{\pgfqpoint{0.833532in}{1.462035in}}{\pgfqpoint{0.833532in}{1.453799in}}%
\pgfpathcurveto{\pgfqpoint{0.833532in}{1.445563in}}{\pgfqpoint{0.836804in}{1.437663in}}{\pgfqpoint{0.842628in}{1.431839in}}%
\pgfpathcurveto{\pgfqpoint{0.848452in}{1.426015in}}{\pgfqpoint{0.856352in}{1.422743in}}{\pgfqpoint{0.864588in}{1.422743in}}%
\pgfpathclose%
\pgfusepath{stroke,fill}%
\end{pgfscope}%
\begin{pgfscope}%
\pgfpathrectangle{\pgfqpoint{0.100000in}{0.220728in}}{\pgfqpoint{3.696000in}{3.696000in}}%
\pgfusepath{clip}%
\pgfsetbuttcap%
\pgfsetroundjoin%
\definecolor{currentfill}{rgb}{0.121569,0.466667,0.705882}%
\pgfsetfillcolor{currentfill}%
\pgfsetfillopacity{0.657340}%
\pgfsetlinewidth{1.003750pt}%
\definecolor{currentstroke}{rgb}{0.121569,0.466667,0.705882}%
\pgfsetstrokecolor{currentstroke}%
\pgfsetstrokeopacity{0.657340}%
\pgfsetdash{}{0pt}%
\pgfpathmoveto{\pgfqpoint{0.864588in}{1.422743in}}%
\pgfpathcurveto{\pgfqpoint{0.872825in}{1.422743in}}{\pgfqpoint{0.880725in}{1.426015in}}{\pgfqpoint{0.886548in}{1.431839in}}%
\pgfpathcurveto{\pgfqpoint{0.892372in}{1.437663in}}{\pgfqpoint{0.895645in}{1.445563in}}{\pgfqpoint{0.895645in}{1.453799in}}%
\pgfpathcurveto{\pgfqpoint{0.895645in}{1.462035in}}{\pgfqpoint{0.892372in}{1.469935in}}{\pgfqpoint{0.886548in}{1.475759in}}%
\pgfpathcurveto{\pgfqpoint{0.880725in}{1.481583in}}{\pgfqpoint{0.872825in}{1.484856in}}{\pgfqpoint{0.864588in}{1.484856in}}%
\pgfpathcurveto{\pgfqpoint{0.856352in}{1.484856in}}{\pgfqpoint{0.848452in}{1.481583in}}{\pgfqpoint{0.842628in}{1.475759in}}%
\pgfpathcurveto{\pgfqpoint{0.836804in}{1.469935in}}{\pgfqpoint{0.833532in}{1.462035in}}{\pgfqpoint{0.833532in}{1.453799in}}%
\pgfpathcurveto{\pgfqpoint{0.833532in}{1.445563in}}{\pgfqpoint{0.836804in}{1.437663in}}{\pgfqpoint{0.842628in}{1.431839in}}%
\pgfpathcurveto{\pgfqpoint{0.848452in}{1.426015in}}{\pgfqpoint{0.856352in}{1.422743in}}{\pgfqpoint{0.864588in}{1.422743in}}%
\pgfpathclose%
\pgfusepath{stroke,fill}%
\end{pgfscope}%
\begin{pgfscope}%
\pgfpathrectangle{\pgfqpoint{0.100000in}{0.220728in}}{\pgfqpoint{3.696000in}{3.696000in}}%
\pgfusepath{clip}%
\pgfsetbuttcap%
\pgfsetroundjoin%
\definecolor{currentfill}{rgb}{0.121569,0.466667,0.705882}%
\pgfsetfillcolor{currentfill}%
\pgfsetfillopacity{0.657340}%
\pgfsetlinewidth{1.003750pt}%
\definecolor{currentstroke}{rgb}{0.121569,0.466667,0.705882}%
\pgfsetstrokecolor{currentstroke}%
\pgfsetstrokeopacity{0.657340}%
\pgfsetdash{}{0pt}%
\pgfpathmoveto{\pgfqpoint{0.864588in}{1.422743in}}%
\pgfpathcurveto{\pgfqpoint{0.872825in}{1.422743in}}{\pgfqpoint{0.880725in}{1.426015in}}{\pgfqpoint{0.886548in}{1.431839in}}%
\pgfpathcurveto{\pgfqpoint{0.892372in}{1.437663in}}{\pgfqpoint{0.895645in}{1.445563in}}{\pgfqpoint{0.895645in}{1.453799in}}%
\pgfpathcurveto{\pgfqpoint{0.895645in}{1.462035in}}{\pgfqpoint{0.892372in}{1.469935in}}{\pgfqpoint{0.886548in}{1.475759in}}%
\pgfpathcurveto{\pgfqpoint{0.880725in}{1.481583in}}{\pgfqpoint{0.872825in}{1.484856in}}{\pgfqpoint{0.864588in}{1.484856in}}%
\pgfpathcurveto{\pgfqpoint{0.856352in}{1.484856in}}{\pgfqpoint{0.848452in}{1.481583in}}{\pgfqpoint{0.842628in}{1.475759in}}%
\pgfpathcurveto{\pgfqpoint{0.836804in}{1.469935in}}{\pgfqpoint{0.833532in}{1.462035in}}{\pgfqpoint{0.833532in}{1.453799in}}%
\pgfpathcurveto{\pgfqpoint{0.833532in}{1.445563in}}{\pgfqpoint{0.836804in}{1.437663in}}{\pgfqpoint{0.842628in}{1.431839in}}%
\pgfpathcurveto{\pgfqpoint{0.848452in}{1.426015in}}{\pgfqpoint{0.856352in}{1.422743in}}{\pgfqpoint{0.864588in}{1.422743in}}%
\pgfpathclose%
\pgfusepath{stroke,fill}%
\end{pgfscope}%
\begin{pgfscope}%
\pgfpathrectangle{\pgfqpoint{0.100000in}{0.220728in}}{\pgfqpoint{3.696000in}{3.696000in}}%
\pgfusepath{clip}%
\pgfsetbuttcap%
\pgfsetroundjoin%
\definecolor{currentfill}{rgb}{0.121569,0.466667,0.705882}%
\pgfsetfillcolor{currentfill}%
\pgfsetfillopacity{0.657340}%
\pgfsetlinewidth{1.003750pt}%
\definecolor{currentstroke}{rgb}{0.121569,0.466667,0.705882}%
\pgfsetstrokecolor{currentstroke}%
\pgfsetstrokeopacity{0.657340}%
\pgfsetdash{}{0pt}%
\pgfpathmoveto{\pgfqpoint{0.864588in}{1.422743in}}%
\pgfpathcurveto{\pgfqpoint{0.872825in}{1.422743in}}{\pgfqpoint{0.880725in}{1.426015in}}{\pgfqpoint{0.886548in}{1.431839in}}%
\pgfpathcurveto{\pgfqpoint{0.892372in}{1.437663in}}{\pgfqpoint{0.895645in}{1.445563in}}{\pgfqpoint{0.895645in}{1.453799in}}%
\pgfpathcurveto{\pgfqpoint{0.895645in}{1.462035in}}{\pgfqpoint{0.892372in}{1.469935in}}{\pgfqpoint{0.886548in}{1.475759in}}%
\pgfpathcurveto{\pgfqpoint{0.880725in}{1.481583in}}{\pgfqpoint{0.872825in}{1.484856in}}{\pgfqpoint{0.864588in}{1.484856in}}%
\pgfpathcurveto{\pgfqpoint{0.856352in}{1.484856in}}{\pgfqpoint{0.848452in}{1.481583in}}{\pgfqpoint{0.842628in}{1.475759in}}%
\pgfpathcurveto{\pgfqpoint{0.836804in}{1.469935in}}{\pgfqpoint{0.833532in}{1.462035in}}{\pgfqpoint{0.833532in}{1.453799in}}%
\pgfpathcurveto{\pgfqpoint{0.833532in}{1.445563in}}{\pgfqpoint{0.836804in}{1.437663in}}{\pgfqpoint{0.842628in}{1.431839in}}%
\pgfpathcurveto{\pgfqpoint{0.848452in}{1.426015in}}{\pgfqpoint{0.856352in}{1.422743in}}{\pgfqpoint{0.864588in}{1.422743in}}%
\pgfpathclose%
\pgfusepath{stroke,fill}%
\end{pgfscope}%
\begin{pgfscope}%
\pgfpathrectangle{\pgfqpoint{0.100000in}{0.220728in}}{\pgfqpoint{3.696000in}{3.696000in}}%
\pgfusepath{clip}%
\pgfsetbuttcap%
\pgfsetroundjoin%
\definecolor{currentfill}{rgb}{0.121569,0.466667,0.705882}%
\pgfsetfillcolor{currentfill}%
\pgfsetfillopacity{0.657340}%
\pgfsetlinewidth{1.003750pt}%
\definecolor{currentstroke}{rgb}{0.121569,0.466667,0.705882}%
\pgfsetstrokecolor{currentstroke}%
\pgfsetstrokeopacity{0.657340}%
\pgfsetdash{}{0pt}%
\pgfpathmoveto{\pgfqpoint{0.864588in}{1.422743in}}%
\pgfpathcurveto{\pgfqpoint{0.872825in}{1.422743in}}{\pgfqpoint{0.880725in}{1.426015in}}{\pgfqpoint{0.886548in}{1.431839in}}%
\pgfpathcurveto{\pgfqpoint{0.892372in}{1.437663in}}{\pgfqpoint{0.895645in}{1.445563in}}{\pgfqpoint{0.895645in}{1.453799in}}%
\pgfpathcurveto{\pgfqpoint{0.895645in}{1.462035in}}{\pgfqpoint{0.892372in}{1.469935in}}{\pgfqpoint{0.886548in}{1.475759in}}%
\pgfpathcurveto{\pgfqpoint{0.880725in}{1.481583in}}{\pgfqpoint{0.872825in}{1.484856in}}{\pgfqpoint{0.864588in}{1.484856in}}%
\pgfpathcurveto{\pgfqpoint{0.856352in}{1.484856in}}{\pgfqpoint{0.848452in}{1.481583in}}{\pgfqpoint{0.842628in}{1.475759in}}%
\pgfpathcurveto{\pgfqpoint{0.836804in}{1.469935in}}{\pgfqpoint{0.833532in}{1.462035in}}{\pgfqpoint{0.833532in}{1.453799in}}%
\pgfpathcurveto{\pgfqpoint{0.833532in}{1.445563in}}{\pgfqpoint{0.836804in}{1.437663in}}{\pgfqpoint{0.842628in}{1.431839in}}%
\pgfpathcurveto{\pgfqpoint{0.848452in}{1.426015in}}{\pgfqpoint{0.856352in}{1.422743in}}{\pgfqpoint{0.864588in}{1.422743in}}%
\pgfpathclose%
\pgfusepath{stroke,fill}%
\end{pgfscope}%
\begin{pgfscope}%
\pgfpathrectangle{\pgfqpoint{0.100000in}{0.220728in}}{\pgfqpoint{3.696000in}{3.696000in}}%
\pgfusepath{clip}%
\pgfsetbuttcap%
\pgfsetroundjoin%
\definecolor{currentfill}{rgb}{0.121569,0.466667,0.705882}%
\pgfsetfillcolor{currentfill}%
\pgfsetfillopacity{0.657340}%
\pgfsetlinewidth{1.003750pt}%
\definecolor{currentstroke}{rgb}{0.121569,0.466667,0.705882}%
\pgfsetstrokecolor{currentstroke}%
\pgfsetstrokeopacity{0.657340}%
\pgfsetdash{}{0pt}%
\pgfpathmoveto{\pgfqpoint{0.864588in}{1.422743in}}%
\pgfpathcurveto{\pgfqpoint{0.872825in}{1.422743in}}{\pgfqpoint{0.880725in}{1.426015in}}{\pgfqpoint{0.886548in}{1.431839in}}%
\pgfpathcurveto{\pgfqpoint{0.892372in}{1.437663in}}{\pgfqpoint{0.895645in}{1.445563in}}{\pgfqpoint{0.895645in}{1.453799in}}%
\pgfpathcurveto{\pgfqpoint{0.895645in}{1.462035in}}{\pgfqpoint{0.892372in}{1.469935in}}{\pgfqpoint{0.886548in}{1.475759in}}%
\pgfpathcurveto{\pgfqpoint{0.880725in}{1.481583in}}{\pgfqpoint{0.872825in}{1.484856in}}{\pgfqpoint{0.864588in}{1.484856in}}%
\pgfpathcurveto{\pgfqpoint{0.856352in}{1.484856in}}{\pgfqpoint{0.848452in}{1.481583in}}{\pgfqpoint{0.842628in}{1.475759in}}%
\pgfpathcurveto{\pgfqpoint{0.836804in}{1.469935in}}{\pgfqpoint{0.833532in}{1.462035in}}{\pgfqpoint{0.833532in}{1.453799in}}%
\pgfpathcurveto{\pgfqpoint{0.833532in}{1.445563in}}{\pgfqpoint{0.836804in}{1.437663in}}{\pgfqpoint{0.842628in}{1.431839in}}%
\pgfpathcurveto{\pgfqpoint{0.848452in}{1.426015in}}{\pgfqpoint{0.856352in}{1.422743in}}{\pgfqpoint{0.864588in}{1.422743in}}%
\pgfpathclose%
\pgfusepath{stroke,fill}%
\end{pgfscope}%
\begin{pgfscope}%
\pgfpathrectangle{\pgfqpoint{0.100000in}{0.220728in}}{\pgfqpoint{3.696000in}{3.696000in}}%
\pgfusepath{clip}%
\pgfsetbuttcap%
\pgfsetroundjoin%
\definecolor{currentfill}{rgb}{0.121569,0.466667,0.705882}%
\pgfsetfillcolor{currentfill}%
\pgfsetfillopacity{0.657340}%
\pgfsetlinewidth{1.003750pt}%
\definecolor{currentstroke}{rgb}{0.121569,0.466667,0.705882}%
\pgfsetstrokecolor{currentstroke}%
\pgfsetstrokeopacity{0.657340}%
\pgfsetdash{}{0pt}%
\pgfpathmoveto{\pgfqpoint{0.864588in}{1.422743in}}%
\pgfpathcurveto{\pgfqpoint{0.872825in}{1.422743in}}{\pgfqpoint{0.880725in}{1.426015in}}{\pgfqpoint{0.886548in}{1.431839in}}%
\pgfpathcurveto{\pgfqpoint{0.892372in}{1.437663in}}{\pgfqpoint{0.895645in}{1.445563in}}{\pgfqpoint{0.895645in}{1.453799in}}%
\pgfpathcurveto{\pgfqpoint{0.895645in}{1.462035in}}{\pgfqpoint{0.892372in}{1.469935in}}{\pgfqpoint{0.886548in}{1.475759in}}%
\pgfpathcurveto{\pgfqpoint{0.880725in}{1.481583in}}{\pgfqpoint{0.872825in}{1.484856in}}{\pgfqpoint{0.864588in}{1.484856in}}%
\pgfpathcurveto{\pgfqpoint{0.856352in}{1.484856in}}{\pgfqpoint{0.848452in}{1.481583in}}{\pgfqpoint{0.842628in}{1.475759in}}%
\pgfpathcurveto{\pgfqpoint{0.836804in}{1.469935in}}{\pgfqpoint{0.833532in}{1.462035in}}{\pgfqpoint{0.833532in}{1.453799in}}%
\pgfpathcurveto{\pgfqpoint{0.833532in}{1.445563in}}{\pgfqpoint{0.836804in}{1.437663in}}{\pgfqpoint{0.842628in}{1.431839in}}%
\pgfpathcurveto{\pgfqpoint{0.848452in}{1.426015in}}{\pgfqpoint{0.856352in}{1.422743in}}{\pgfqpoint{0.864588in}{1.422743in}}%
\pgfpathclose%
\pgfusepath{stroke,fill}%
\end{pgfscope}%
\begin{pgfscope}%
\pgfpathrectangle{\pgfqpoint{0.100000in}{0.220728in}}{\pgfqpoint{3.696000in}{3.696000in}}%
\pgfusepath{clip}%
\pgfsetbuttcap%
\pgfsetroundjoin%
\definecolor{currentfill}{rgb}{0.121569,0.466667,0.705882}%
\pgfsetfillcolor{currentfill}%
\pgfsetfillopacity{0.657340}%
\pgfsetlinewidth{1.003750pt}%
\definecolor{currentstroke}{rgb}{0.121569,0.466667,0.705882}%
\pgfsetstrokecolor{currentstroke}%
\pgfsetstrokeopacity{0.657340}%
\pgfsetdash{}{0pt}%
\pgfpathmoveto{\pgfqpoint{0.864588in}{1.422743in}}%
\pgfpathcurveto{\pgfqpoint{0.872825in}{1.422743in}}{\pgfqpoint{0.880725in}{1.426015in}}{\pgfqpoint{0.886548in}{1.431839in}}%
\pgfpathcurveto{\pgfqpoint{0.892372in}{1.437663in}}{\pgfqpoint{0.895645in}{1.445563in}}{\pgfqpoint{0.895645in}{1.453799in}}%
\pgfpathcurveto{\pgfqpoint{0.895645in}{1.462035in}}{\pgfqpoint{0.892372in}{1.469935in}}{\pgfqpoint{0.886548in}{1.475759in}}%
\pgfpathcurveto{\pgfqpoint{0.880725in}{1.481583in}}{\pgfqpoint{0.872825in}{1.484856in}}{\pgfqpoint{0.864588in}{1.484856in}}%
\pgfpathcurveto{\pgfqpoint{0.856352in}{1.484856in}}{\pgfqpoint{0.848452in}{1.481583in}}{\pgfqpoint{0.842628in}{1.475759in}}%
\pgfpathcurveto{\pgfqpoint{0.836804in}{1.469935in}}{\pgfqpoint{0.833532in}{1.462035in}}{\pgfqpoint{0.833532in}{1.453799in}}%
\pgfpathcurveto{\pgfqpoint{0.833532in}{1.445563in}}{\pgfqpoint{0.836804in}{1.437663in}}{\pgfqpoint{0.842628in}{1.431839in}}%
\pgfpathcurveto{\pgfqpoint{0.848452in}{1.426015in}}{\pgfqpoint{0.856352in}{1.422743in}}{\pgfqpoint{0.864588in}{1.422743in}}%
\pgfpathclose%
\pgfusepath{stroke,fill}%
\end{pgfscope}%
\begin{pgfscope}%
\pgfpathrectangle{\pgfqpoint{0.100000in}{0.220728in}}{\pgfqpoint{3.696000in}{3.696000in}}%
\pgfusepath{clip}%
\pgfsetbuttcap%
\pgfsetroundjoin%
\definecolor{currentfill}{rgb}{0.121569,0.466667,0.705882}%
\pgfsetfillcolor{currentfill}%
\pgfsetfillopacity{0.657340}%
\pgfsetlinewidth{1.003750pt}%
\definecolor{currentstroke}{rgb}{0.121569,0.466667,0.705882}%
\pgfsetstrokecolor{currentstroke}%
\pgfsetstrokeopacity{0.657340}%
\pgfsetdash{}{0pt}%
\pgfpathmoveto{\pgfqpoint{0.864588in}{1.422743in}}%
\pgfpathcurveto{\pgfqpoint{0.872825in}{1.422743in}}{\pgfqpoint{0.880725in}{1.426015in}}{\pgfqpoint{0.886548in}{1.431839in}}%
\pgfpathcurveto{\pgfqpoint{0.892372in}{1.437663in}}{\pgfqpoint{0.895645in}{1.445563in}}{\pgfqpoint{0.895645in}{1.453799in}}%
\pgfpathcurveto{\pgfqpoint{0.895645in}{1.462035in}}{\pgfqpoint{0.892372in}{1.469935in}}{\pgfqpoint{0.886548in}{1.475759in}}%
\pgfpathcurveto{\pgfqpoint{0.880725in}{1.481583in}}{\pgfqpoint{0.872825in}{1.484856in}}{\pgfqpoint{0.864588in}{1.484856in}}%
\pgfpathcurveto{\pgfqpoint{0.856352in}{1.484856in}}{\pgfqpoint{0.848452in}{1.481583in}}{\pgfqpoint{0.842628in}{1.475759in}}%
\pgfpathcurveto{\pgfqpoint{0.836804in}{1.469935in}}{\pgfqpoint{0.833532in}{1.462035in}}{\pgfqpoint{0.833532in}{1.453799in}}%
\pgfpathcurveto{\pgfqpoint{0.833532in}{1.445563in}}{\pgfqpoint{0.836804in}{1.437663in}}{\pgfqpoint{0.842628in}{1.431839in}}%
\pgfpathcurveto{\pgfqpoint{0.848452in}{1.426015in}}{\pgfqpoint{0.856352in}{1.422743in}}{\pgfqpoint{0.864588in}{1.422743in}}%
\pgfpathclose%
\pgfusepath{stroke,fill}%
\end{pgfscope}%
\begin{pgfscope}%
\pgfpathrectangle{\pgfqpoint{0.100000in}{0.220728in}}{\pgfqpoint{3.696000in}{3.696000in}}%
\pgfusepath{clip}%
\pgfsetbuttcap%
\pgfsetroundjoin%
\definecolor{currentfill}{rgb}{0.121569,0.466667,0.705882}%
\pgfsetfillcolor{currentfill}%
\pgfsetfillopacity{0.657340}%
\pgfsetlinewidth{1.003750pt}%
\definecolor{currentstroke}{rgb}{0.121569,0.466667,0.705882}%
\pgfsetstrokecolor{currentstroke}%
\pgfsetstrokeopacity{0.657340}%
\pgfsetdash{}{0pt}%
\pgfpathmoveto{\pgfqpoint{0.864588in}{1.422743in}}%
\pgfpathcurveto{\pgfqpoint{0.872825in}{1.422743in}}{\pgfqpoint{0.880725in}{1.426015in}}{\pgfqpoint{0.886548in}{1.431839in}}%
\pgfpathcurveto{\pgfqpoint{0.892372in}{1.437663in}}{\pgfqpoint{0.895645in}{1.445563in}}{\pgfqpoint{0.895645in}{1.453799in}}%
\pgfpathcurveto{\pgfqpoint{0.895645in}{1.462035in}}{\pgfqpoint{0.892372in}{1.469935in}}{\pgfqpoint{0.886548in}{1.475759in}}%
\pgfpathcurveto{\pgfqpoint{0.880725in}{1.481583in}}{\pgfqpoint{0.872825in}{1.484856in}}{\pgfqpoint{0.864588in}{1.484856in}}%
\pgfpathcurveto{\pgfqpoint{0.856352in}{1.484856in}}{\pgfqpoint{0.848452in}{1.481583in}}{\pgfqpoint{0.842628in}{1.475759in}}%
\pgfpathcurveto{\pgfqpoint{0.836804in}{1.469935in}}{\pgfqpoint{0.833532in}{1.462035in}}{\pgfqpoint{0.833532in}{1.453799in}}%
\pgfpathcurveto{\pgfqpoint{0.833532in}{1.445563in}}{\pgfqpoint{0.836804in}{1.437663in}}{\pgfqpoint{0.842628in}{1.431839in}}%
\pgfpathcurveto{\pgfqpoint{0.848452in}{1.426015in}}{\pgfqpoint{0.856352in}{1.422743in}}{\pgfqpoint{0.864588in}{1.422743in}}%
\pgfpathclose%
\pgfusepath{stroke,fill}%
\end{pgfscope}%
\begin{pgfscope}%
\pgfpathrectangle{\pgfqpoint{0.100000in}{0.220728in}}{\pgfqpoint{3.696000in}{3.696000in}}%
\pgfusepath{clip}%
\pgfsetbuttcap%
\pgfsetroundjoin%
\definecolor{currentfill}{rgb}{0.121569,0.466667,0.705882}%
\pgfsetfillcolor{currentfill}%
\pgfsetfillopacity{0.657340}%
\pgfsetlinewidth{1.003750pt}%
\definecolor{currentstroke}{rgb}{0.121569,0.466667,0.705882}%
\pgfsetstrokecolor{currentstroke}%
\pgfsetstrokeopacity{0.657340}%
\pgfsetdash{}{0pt}%
\pgfpathmoveto{\pgfqpoint{0.864588in}{1.422743in}}%
\pgfpathcurveto{\pgfqpoint{0.872825in}{1.422743in}}{\pgfqpoint{0.880725in}{1.426015in}}{\pgfqpoint{0.886548in}{1.431839in}}%
\pgfpathcurveto{\pgfqpoint{0.892372in}{1.437663in}}{\pgfqpoint{0.895645in}{1.445563in}}{\pgfqpoint{0.895645in}{1.453799in}}%
\pgfpathcurveto{\pgfqpoint{0.895645in}{1.462035in}}{\pgfqpoint{0.892372in}{1.469935in}}{\pgfqpoint{0.886548in}{1.475759in}}%
\pgfpathcurveto{\pgfqpoint{0.880725in}{1.481583in}}{\pgfqpoint{0.872825in}{1.484856in}}{\pgfqpoint{0.864588in}{1.484856in}}%
\pgfpathcurveto{\pgfqpoint{0.856352in}{1.484856in}}{\pgfqpoint{0.848452in}{1.481583in}}{\pgfqpoint{0.842628in}{1.475759in}}%
\pgfpathcurveto{\pgfqpoint{0.836804in}{1.469935in}}{\pgfqpoint{0.833532in}{1.462035in}}{\pgfqpoint{0.833532in}{1.453799in}}%
\pgfpathcurveto{\pgfqpoint{0.833532in}{1.445563in}}{\pgfqpoint{0.836804in}{1.437663in}}{\pgfqpoint{0.842628in}{1.431839in}}%
\pgfpathcurveto{\pgfqpoint{0.848452in}{1.426015in}}{\pgfqpoint{0.856352in}{1.422743in}}{\pgfqpoint{0.864588in}{1.422743in}}%
\pgfpathclose%
\pgfusepath{stroke,fill}%
\end{pgfscope}%
\begin{pgfscope}%
\pgfpathrectangle{\pgfqpoint{0.100000in}{0.220728in}}{\pgfqpoint{3.696000in}{3.696000in}}%
\pgfusepath{clip}%
\pgfsetbuttcap%
\pgfsetroundjoin%
\definecolor{currentfill}{rgb}{0.121569,0.466667,0.705882}%
\pgfsetfillcolor{currentfill}%
\pgfsetfillopacity{0.657340}%
\pgfsetlinewidth{1.003750pt}%
\definecolor{currentstroke}{rgb}{0.121569,0.466667,0.705882}%
\pgfsetstrokecolor{currentstroke}%
\pgfsetstrokeopacity{0.657340}%
\pgfsetdash{}{0pt}%
\pgfpathmoveto{\pgfqpoint{0.864588in}{1.422743in}}%
\pgfpathcurveto{\pgfqpoint{0.872825in}{1.422743in}}{\pgfqpoint{0.880725in}{1.426015in}}{\pgfqpoint{0.886548in}{1.431839in}}%
\pgfpathcurveto{\pgfqpoint{0.892372in}{1.437663in}}{\pgfqpoint{0.895645in}{1.445563in}}{\pgfqpoint{0.895645in}{1.453799in}}%
\pgfpathcurveto{\pgfqpoint{0.895645in}{1.462035in}}{\pgfqpoint{0.892372in}{1.469935in}}{\pgfqpoint{0.886548in}{1.475759in}}%
\pgfpathcurveto{\pgfqpoint{0.880725in}{1.481583in}}{\pgfqpoint{0.872825in}{1.484856in}}{\pgfqpoint{0.864588in}{1.484856in}}%
\pgfpathcurveto{\pgfqpoint{0.856352in}{1.484856in}}{\pgfqpoint{0.848452in}{1.481583in}}{\pgfqpoint{0.842628in}{1.475759in}}%
\pgfpathcurveto{\pgfqpoint{0.836804in}{1.469935in}}{\pgfqpoint{0.833532in}{1.462035in}}{\pgfqpoint{0.833532in}{1.453799in}}%
\pgfpathcurveto{\pgfqpoint{0.833532in}{1.445563in}}{\pgfqpoint{0.836804in}{1.437663in}}{\pgfqpoint{0.842628in}{1.431839in}}%
\pgfpathcurveto{\pgfqpoint{0.848452in}{1.426015in}}{\pgfqpoint{0.856352in}{1.422743in}}{\pgfqpoint{0.864588in}{1.422743in}}%
\pgfpathclose%
\pgfusepath{stroke,fill}%
\end{pgfscope}%
\begin{pgfscope}%
\pgfpathrectangle{\pgfqpoint{0.100000in}{0.220728in}}{\pgfqpoint{3.696000in}{3.696000in}}%
\pgfusepath{clip}%
\pgfsetbuttcap%
\pgfsetroundjoin%
\definecolor{currentfill}{rgb}{0.121569,0.466667,0.705882}%
\pgfsetfillcolor{currentfill}%
\pgfsetfillopacity{0.657340}%
\pgfsetlinewidth{1.003750pt}%
\definecolor{currentstroke}{rgb}{0.121569,0.466667,0.705882}%
\pgfsetstrokecolor{currentstroke}%
\pgfsetstrokeopacity{0.657340}%
\pgfsetdash{}{0pt}%
\pgfpathmoveto{\pgfqpoint{0.864588in}{1.422743in}}%
\pgfpathcurveto{\pgfqpoint{0.872825in}{1.422743in}}{\pgfqpoint{0.880725in}{1.426015in}}{\pgfqpoint{0.886548in}{1.431839in}}%
\pgfpathcurveto{\pgfqpoint{0.892372in}{1.437663in}}{\pgfqpoint{0.895645in}{1.445563in}}{\pgfqpoint{0.895645in}{1.453799in}}%
\pgfpathcurveto{\pgfqpoint{0.895645in}{1.462035in}}{\pgfqpoint{0.892372in}{1.469935in}}{\pgfqpoint{0.886548in}{1.475759in}}%
\pgfpathcurveto{\pgfqpoint{0.880725in}{1.481583in}}{\pgfqpoint{0.872825in}{1.484856in}}{\pgfqpoint{0.864588in}{1.484856in}}%
\pgfpathcurveto{\pgfqpoint{0.856352in}{1.484856in}}{\pgfqpoint{0.848452in}{1.481583in}}{\pgfqpoint{0.842628in}{1.475759in}}%
\pgfpathcurveto{\pgfqpoint{0.836804in}{1.469935in}}{\pgfqpoint{0.833532in}{1.462035in}}{\pgfqpoint{0.833532in}{1.453799in}}%
\pgfpathcurveto{\pgfqpoint{0.833532in}{1.445563in}}{\pgfqpoint{0.836804in}{1.437663in}}{\pgfqpoint{0.842628in}{1.431839in}}%
\pgfpathcurveto{\pgfqpoint{0.848452in}{1.426015in}}{\pgfqpoint{0.856352in}{1.422743in}}{\pgfqpoint{0.864588in}{1.422743in}}%
\pgfpathclose%
\pgfusepath{stroke,fill}%
\end{pgfscope}%
\begin{pgfscope}%
\pgfpathrectangle{\pgfqpoint{0.100000in}{0.220728in}}{\pgfqpoint{3.696000in}{3.696000in}}%
\pgfusepath{clip}%
\pgfsetbuttcap%
\pgfsetroundjoin%
\definecolor{currentfill}{rgb}{0.121569,0.466667,0.705882}%
\pgfsetfillcolor{currentfill}%
\pgfsetfillopacity{0.657340}%
\pgfsetlinewidth{1.003750pt}%
\definecolor{currentstroke}{rgb}{0.121569,0.466667,0.705882}%
\pgfsetstrokecolor{currentstroke}%
\pgfsetstrokeopacity{0.657340}%
\pgfsetdash{}{0pt}%
\pgfpathmoveto{\pgfqpoint{0.864588in}{1.422743in}}%
\pgfpathcurveto{\pgfqpoint{0.872825in}{1.422743in}}{\pgfqpoint{0.880725in}{1.426015in}}{\pgfqpoint{0.886548in}{1.431839in}}%
\pgfpathcurveto{\pgfqpoint{0.892372in}{1.437663in}}{\pgfqpoint{0.895645in}{1.445563in}}{\pgfqpoint{0.895645in}{1.453799in}}%
\pgfpathcurveto{\pgfqpoint{0.895645in}{1.462035in}}{\pgfqpoint{0.892372in}{1.469935in}}{\pgfqpoint{0.886548in}{1.475759in}}%
\pgfpathcurveto{\pgfqpoint{0.880725in}{1.481583in}}{\pgfqpoint{0.872825in}{1.484856in}}{\pgfqpoint{0.864588in}{1.484856in}}%
\pgfpathcurveto{\pgfqpoint{0.856352in}{1.484856in}}{\pgfqpoint{0.848452in}{1.481583in}}{\pgfqpoint{0.842628in}{1.475759in}}%
\pgfpathcurveto{\pgfqpoint{0.836804in}{1.469935in}}{\pgfqpoint{0.833532in}{1.462035in}}{\pgfqpoint{0.833532in}{1.453799in}}%
\pgfpathcurveto{\pgfqpoint{0.833532in}{1.445563in}}{\pgfqpoint{0.836804in}{1.437663in}}{\pgfqpoint{0.842628in}{1.431839in}}%
\pgfpathcurveto{\pgfqpoint{0.848452in}{1.426015in}}{\pgfqpoint{0.856352in}{1.422743in}}{\pgfqpoint{0.864588in}{1.422743in}}%
\pgfpathclose%
\pgfusepath{stroke,fill}%
\end{pgfscope}%
\begin{pgfscope}%
\pgfpathrectangle{\pgfqpoint{0.100000in}{0.220728in}}{\pgfqpoint{3.696000in}{3.696000in}}%
\pgfusepath{clip}%
\pgfsetbuttcap%
\pgfsetroundjoin%
\definecolor{currentfill}{rgb}{0.121569,0.466667,0.705882}%
\pgfsetfillcolor{currentfill}%
\pgfsetfillopacity{0.657340}%
\pgfsetlinewidth{1.003750pt}%
\definecolor{currentstroke}{rgb}{0.121569,0.466667,0.705882}%
\pgfsetstrokecolor{currentstroke}%
\pgfsetstrokeopacity{0.657340}%
\pgfsetdash{}{0pt}%
\pgfpathmoveto{\pgfqpoint{0.864588in}{1.422743in}}%
\pgfpathcurveto{\pgfqpoint{0.872825in}{1.422743in}}{\pgfqpoint{0.880725in}{1.426015in}}{\pgfqpoint{0.886548in}{1.431839in}}%
\pgfpathcurveto{\pgfqpoint{0.892372in}{1.437663in}}{\pgfqpoint{0.895645in}{1.445563in}}{\pgfqpoint{0.895645in}{1.453799in}}%
\pgfpathcurveto{\pgfqpoint{0.895645in}{1.462035in}}{\pgfqpoint{0.892372in}{1.469935in}}{\pgfqpoint{0.886548in}{1.475759in}}%
\pgfpathcurveto{\pgfqpoint{0.880725in}{1.481583in}}{\pgfqpoint{0.872825in}{1.484856in}}{\pgfqpoint{0.864588in}{1.484856in}}%
\pgfpathcurveto{\pgfqpoint{0.856352in}{1.484856in}}{\pgfqpoint{0.848452in}{1.481583in}}{\pgfqpoint{0.842628in}{1.475759in}}%
\pgfpathcurveto{\pgfqpoint{0.836804in}{1.469935in}}{\pgfqpoint{0.833532in}{1.462035in}}{\pgfqpoint{0.833532in}{1.453799in}}%
\pgfpathcurveto{\pgfqpoint{0.833532in}{1.445563in}}{\pgfqpoint{0.836804in}{1.437663in}}{\pgfqpoint{0.842628in}{1.431839in}}%
\pgfpathcurveto{\pgfqpoint{0.848452in}{1.426015in}}{\pgfqpoint{0.856352in}{1.422743in}}{\pgfqpoint{0.864588in}{1.422743in}}%
\pgfpathclose%
\pgfusepath{stroke,fill}%
\end{pgfscope}%
\begin{pgfscope}%
\pgfpathrectangle{\pgfqpoint{0.100000in}{0.220728in}}{\pgfqpoint{3.696000in}{3.696000in}}%
\pgfusepath{clip}%
\pgfsetbuttcap%
\pgfsetroundjoin%
\definecolor{currentfill}{rgb}{0.121569,0.466667,0.705882}%
\pgfsetfillcolor{currentfill}%
\pgfsetfillopacity{0.657340}%
\pgfsetlinewidth{1.003750pt}%
\definecolor{currentstroke}{rgb}{0.121569,0.466667,0.705882}%
\pgfsetstrokecolor{currentstroke}%
\pgfsetstrokeopacity{0.657340}%
\pgfsetdash{}{0pt}%
\pgfpathmoveto{\pgfqpoint{0.864588in}{1.422743in}}%
\pgfpathcurveto{\pgfqpoint{0.872825in}{1.422743in}}{\pgfqpoint{0.880725in}{1.426015in}}{\pgfqpoint{0.886548in}{1.431839in}}%
\pgfpathcurveto{\pgfqpoint{0.892372in}{1.437663in}}{\pgfqpoint{0.895645in}{1.445563in}}{\pgfqpoint{0.895645in}{1.453799in}}%
\pgfpathcurveto{\pgfqpoint{0.895645in}{1.462035in}}{\pgfqpoint{0.892372in}{1.469935in}}{\pgfqpoint{0.886548in}{1.475759in}}%
\pgfpathcurveto{\pgfqpoint{0.880725in}{1.481583in}}{\pgfqpoint{0.872825in}{1.484856in}}{\pgfqpoint{0.864588in}{1.484856in}}%
\pgfpathcurveto{\pgfqpoint{0.856352in}{1.484856in}}{\pgfqpoint{0.848452in}{1.481583in}}{\pgfqpoint{0.842628in}{1.475759in}}%
\pgfpathcurveto{\pgfqpoint{0.836804in}{1.469935in}}{\pgfqpoint{0.833532in}{1.462035in}}{\pgfqpoint{0.833532in}{1.453799in}}%
\pgfpathcurveto{\pgfqpoint{0.833532in}{1.445563in}}{\pgfqpoint{0.836804in}{1.437663in}}{\pgfqpoint{0.842628in}{1.431839in}}%
\pgfpathcurveto{\pgfqpoint{0.848452in}{1.426015in}}{\pgfqpoint{0.856352in}{1.422743in}}{\pgfqpoint{0.864588in}{1.422743in}}%
\pgfpathclose%
\pgfusepath{stroke,fill}%
\end{pgfscope}%
\begin{pgfscope}%
\pgfpathrectangle{\pgfqpoint{0.100000in}{0.220728in}}{\pgfqpoint{3.696000in}{3.696000in}}%
\pgfusepath{clip}%
\pgfsetbuttcap%
\pgfsetroundjoin%
\definecolor{currentfill}{rgb}{0.121569,0.466667,0.705882}%
\pgfsetfillcolor{currentfill}%
\pgfsetfillopacity{0.657340}%
\pgfsetlinewidth{1.003750pt}%
\definecolor{currentstroke}{rgb}{0.121569,0.466667,0.705882}%
\pgfsetstrokecolor{currentstroke}%
\pgfsetstrokeopacity{0.657340}%
\pgfsetdash{}{0pt}%
\pgfpathmoveto{\pgfqpoint{0.864588in}{1.422743in}}%
\pgfpathcurveto{\pgfqpoint{0.872825in}{1.422743in}}{\pgfqpoint{0.880725in}{1.426015in}}{\pgfqpoint{0.886548in}{1.431839in}}%
\pgfpathcurveto{\pgfqpoint{0.892372in}{1.437663in}}{\pgfqpoint{0.895645in}{1.445563in}}{\pgfqpoint{0.895645in}{1.453799in}}%
\pgfpathcurveto{\pgfqpoint{0.895645in}{1.462035in}}{\pgfqpoint{0.892372in}{1.469935in}}{\pgfqpoint{0.886548in}{1.475759in}}%
\pgfpathcurveto{\pgfqpoint{0.880725in}{1.481583in}}{\pgfqpoint{0.872825in}{1.484856in}}{\pgfqpoint{0.864588in}{1.484856in}}%
\pgfpathcurveto{\pgfqpoint{0.856352in}{1.484856in}}{\pgfqpoint{0.848452in}{1.481583in}}{\pgfqpoint{0.842628in}{1.475759in}}%
\pgfpathcurveto{\pgfqpoint{0.836804in}{1.469935in}}{\pgfqpoint{0.833532in}{1.462035in}}{\pgfqpoint{0.833532in}{1.453799in}}%
\pgfpathcurveto{\pgfqpoint{0.833532in}{1.445563in}}{\pgfqpoint{0.836804in}{1.437663in}}{\pgfqpoint{0.842628in}{1.431839in}}%
\pgfpathcurveto{\pgfqpoint{0.848452in}{1.426015in}}{\pgfqpoint{0.856352in}{1.422743in}}{\pgfqpoint{0.864588in}{1.422743in}}%
\pgfpathclose%
\pgfusepath{stroke,fill}%
\end{pgfscope}%
\begin{pgfscope}%
\pgfpathrectangle{\pgfqpoint{0.100000in}{0.220728in}}{\pgfqpoint{3.696000in}{3.696000in}}%
\pgfusepath{clip}%
\pgfsetbuttcap%
\pgfsetroundjoin%
\definecolor{currentfill}{rgb}{0.121569,0.466667,0.705882}%
\pgfsetfillcolor{currentfill}%
\pgfsetfillopacity{0.657340}%
\pgfsetlinewidth{1.003750pt}%
\definecolor{currentstroke}{rgb}{0.121569,0.466667,0.705882}%
\pgfsetstrokecolor{currentstroke}%
\pgfsetstrokeopacity{0.657340}%
\pgfsetdash{}{0pt}%
\pgfpathmoveto{\pgfqpoint{0.864588in}{1.422743in}}%
\pgfpathcurveto{\pgfqpoint{0.872825in}{1.422743in}}{\pgfqpoint{0.880725in}{1.426015in}}{\pgfqpoint{0.886548in}{1.431839in}}%
\pgfpathcurveto{\pgfqpoint{0.892372in}{1.437663in}}{\pgfqpoint{0.895645in}{1.445563in}}{\pgfqpoint{0.895645in}{1.453799in}}%
\pgfpathcurveto{\pgfqpoint{0.895645in}{1.462035in}}{\pgfqpoint{0.892372in}{1.469935in}}{\pgfqpoint{0.886548in}{1.475759in}}%
\pgfpathcurveto{\pgfqpoint{0.880725in}{1.481583in}}{\pgfqpoint{0.872825in}{1.484856in}}{\pgfqpoint{0.864588in}{1.484856in}}%
\pgfpathcurveto{\pgfqpoint{0.856352in}{1.484856in}}{\pgfqpoint{0.848452in}{1.481583in}}{\pgfqpoint{0.842628in}{1.475759in}}%
\pgfpathcurveto{\pgfqpoint{0.836804in}{1.469935in}}{\pgfqpoint{0.833532in}{1.462035in}}{\pgfqpoint{0.833532in}{1.453799in}}%
\pgfpathcurveto{\pgfqpoint{0.833532in}{1.445563in}}{\pgfqpoint{0.836804in}{1.437663in}}{\pgfqpoint{0.842628in}{1.431839in}}%
\pgfpathcurveto{\pgfqpoint{0.848452in}{1.426015in}}{\pgfqpoint{0.856352in}{1.422743in}}{\pgfqpoint{0.864588in}{1.422743in}}%
\pgfpathclose%
\pgfusepath{stroke,fill}%
\end{pgfscope}%
\begin{pgfscope}%
\pgfpathrectangle{\pgfqpoint{0.100000in}{0.220728in}}{\pgfqpoint{3.696000in}{3.696000in}}%
\pgfusepath{clip}%
\pgfsetbuttcap%
\pgfsetroundjoin%
\definecolor{currentfill}{rgb}{0.121569,0.466667,0.705882}%
\pgfsetfillcolor{currentfill}%
\pgfsetfillopacity{0.657340}%
\pgfsetlinewidth{1.003750pt}%
\definecolor{currentstroke}{rgb}{0.121569,0.466667,0.705882}%
\pgfsetstrokecolor{currentstroke}%
\pgfsetstrokeopacity{0.657340}%
\pgfsetdash{}{0pt}%
\pgfpathmoveto{\pgfqpoint{0.864588in}{1.422743in}}%
\pgfpathcurveto{\pgfqpoint{0.872825in}{1.422743in}}{\pgfqpoint{0.880725in}{1.426015in}}{\pgfqpoint{0.886548in}{1.431839in}}%
\pgfpathcurveto{\pgfqpoint{0.892372in}{1.437663in}}{\pgfqpoint{0.895645in}{1.445563in}}{\pgfqpoint{0.895645in}{1.453799in}}%
\pgfpathcurveto{\pgfqpoint{0.895645in}{1.462035in}}{\pgfqpoint{0.892372in}{1.469935in}}{\pgfqpoint{0.886548in}{1.475759in}}%
\pgfpathcurveto{\pgfqpoint{0.880725in}{1.481583in}}{\pgfqpoint{0.872825in}{1.484856in}}{\pgfqpoint{0.864588in}{1.484856in}}%
\pgfpathcurveto{\pgfqpoint{0.856352in}{1.484856in}}{\pgfqpoint{0.848452in}{1.481583in}}{\pgfqpoint{0.842628in}{1.475759in}}%
\pgfpathcurveto{\pgfqpoint{0.836804in}{1.469935in}}{\pgfqpoint{0.833532in}{1.462035in}}{\pgfqpoint{0.833532in}{1.453799in}}%
\pgfpathcurveto{\pgfqpoint{0.833532in}{1.445563in}}{\pgfqpoint{0.836804in}{1.437663in}}{\pgfqpoint{0.842628in}{1.431839in}}%
\pgfpathcurveto{\pgfqpoint{0.848452in}{1.426015in}}{\pgfqpoint{0.856352in}{1.422743in}}{\pgfqpoint{0.864588in}{1.422743in}}%
\pgfpathclose%
\pgfusepath{stroke,fill}%
\end{pgfscope}%
\begin{pgfscope}%
\pgfpathrectangle{\pgfqpoint{0.100000in}{0.220728in}}{\pgfqpoint{3.696000in}{3.696000in}}%
\pgfusepath{clip}%
\pgfsetbuttcap%
\pgfsetroundjoin%
\definecolor{currentfill}{rgb}{0.121569,0.466667,0.705882}%
\pgfsetfillcolor{currentfill}%
\pgfsetfillopacity{0.657340}%
\pgfsetlinewidth{1.003750pt}%
\definecolor{currentstroke}{rgb}{0.121569,0.466667,0.705882}%
\pgfsetstrokecolor{currentstroke}%
\pgfsetstrokeopacity{0.657340}%
\pgfsetdash{}{0pt}%
\pgfpathmoveto{\pgfqpoint{0.864588in}{1.422743in}}%
\pgfpathcurveto{\pgfqpoint{0.872825in}{1.422743in}}{\pgfqpoint{0.880725in}{1.426015in}}{\pgfqpoint{0.886548in}{1.431839in}}%
\pgfpathcurveto{\pgfqpoint{0.892372in}{1.437663in}}{\pgfqpoint{0.895645in}{1.445563in}}{\pgfqpoint{0.895645in}{1.453799in}}%
\pgfpathcurveto{\pgfqpoint{0.895645in}{1.462035in}}{\pgfqpoint{0.892372in}{1.469935in}}{\pgfqpoint{0.886548in}{1.475759in}}%
\pgfpathcurveto{\pgfqpoint{0.880725in}{1.481583in}}{\pgfqpoint{0.872825in}{1.484856in}}{\pgfqpoint{0.864588in}{1.484856in}}%
\pgfpathcurveto{\pgfqpoint{0.856352in}{1.484856in}}{\pgfqpoint{0.848452in}{1.481583in}}{\pgfqpoint{0.842628in}{1.475759in}}%
\pgfpathcurveto{\pgfqpoint{0.836804in}{1.469935in}}{\pgfqpoint{0.833532in}{1.462035in}}{\pgfqpoint{0.833532in}{1.453799in}}%
\pgfpathcurveto{\pgfqpoint{0.833532in}{1.445563in}}{\pgfqpoint{0.836804in}{1.437663in}}{\pgfqpoint{0.842628in}{1.431839in}}%
\pgfpathcurveto{\pgfqpoint{0.848452in}{1.426015in}}{\pgfqpoint{0.856352in}{1.422743in}}{\pgfqpoint{0.864588in}{1.422743in}}%
\pgfpathclose%
\pgfusepath{stroke,fill}%
\end{pgfscope}%
\begin{pgfscope}%
\pgfpathrectangle{\pgfqpoint{0.100000in}{0.220728in}}{\pgfqpoint{3.696000in}{3.696000in}}%
\pgfusepath{clip}%
\pgfsetbuttcap%
\pgfsetroundjoin%
\definecolor{currentfill}{rgb}{0.121569,0.466667,0.705882}%
\pgfsetfillcolor{currentfill}%
\pgfsetfillopacity{0.657340}%
\pgfsetlinewidth{1.003750pt}%
\definecolor{currentstroke}{rgb}{0.121569,0.466667,0.705882}%
\pgfsetstrokecolor{currentstroke}%
\pgfsetstrokeopacity{0.657340}%
\pgfsetdash{}{0pt}%
\pgfpathmoveto{\pgfqpoint{0.864588in}{1.422743in}}%
\pgfpathcurveto{\pgfqpoint{0.872825in}{1.422743in}}{\pgfqpoint{0.880725in}{1.426015in}}{\pgfqpoint{0.886548in}{1.431839in}}%
\pgfpathcurveto{\pgfqpoint{0.892372in}{1.437663in}}{\pgfqpoint{0.895645in}{1.445563in}}{\pgfqpoint{0.895645in}{1.453799in}}%
\pgfpathcurveto{\pgfqpoint{0.895645in}{1.462035in}}{\pgfqpoint{0.892372in}{1.469935in}}{\pgfqpoint{0.886548in}{1.475759in}}%
\pgfpathcurveto{\pgfqpoint{0.880725in}{1.481583in}}{\pgfqpoint{0.872825in}{1.484856in}}{\pgfqpoint{0.864588in}{1.484856in}}%
\pgfpathcurveto{\pgfqpoint{0.856352in}{1.484856in}}{\pgfqpoint{0.848452in}{1.481583in}}{\pgfqpoint{0.842628in}{1.475759in}}%
\pgfpathcurveto{\pgfqpoint{0.836804in}{1.469935in}}{\pgfqpoint{0.833532in}{1.462035in}}{\pgfqpoint{0.833532in}{1.453799in}}%
\pgfpathcurveto{\pgfqpoint{0.833532in}{1.445563in}}{\pgfqpoint{0.836804in}{1.437663in}}{\pgfqpoint{0.842628in}{1.431839in}}%
\pgfpathcurveto{\pgfqpoint{0.848452in}{1.426015in}}{\pgfqpoint{0.856352in}{1.422743in}}{\pgfqpoint{0.864588in}{1.422743in}}%
\pgfpathclose%
\pgfusepath{stroke,fill}%
\end{pgfscope}%
\begin{pgfscope}%
\pgfpathrectangle{\pgfqpoint{0.100000in}{0.220728in}}{\pgfqpoint{3.696000in}{3.696000in}}%
\pgfusepath{clip}%
\pgfsetbuttcap%
\pgfsetroundjoin%
\definecolor{currentfill}{rgb}{0.121569,0.466667,0.705882}%
\pgfsetfillcolor{currentfill}%
\pgfsetfillopacity{0.657340}%
\pgfsetlinewidth{1.003750pt}%
\definecolor{currentstroke}{rgb}{0.121569,0.466667,0.705882}%
\pgfsetstrokecolor{currentstroke}%
\pgfsetstrokeopacity{0.657340}%
\pgfsetdash{}{0pt}%
\pgfpathmoveto{\pgfqpoint{0.864588in}{1.422743in}}%
\pgfpathcurveto{\pgfqpoint{0.872825in}{1.422743in}}{\pgfqpoint{0.880725in}{1.426015in}}{\pgfqpoint{0.886548in}{1.431839in}}%
\pgfpathcurveto{\pgfqpoint{0.892372in}{1.437663in}}{\pgfqpoint{0.895645in}{1.445563in}}{\pgfqpoint{0.895645in}{1.453799in}}%
\pgfpathcurveto{\pgfqpoint{0.895645in}{1.462035in}}{\pgfqpoint{0.892372in}{1.469935in}}{\pgfqpoint{0.886548in}{1.475759in}}%
\pgfpathcurveto{\pgfqpoint{0.880725in}{1.481583in}}{\pgfqpoint{0.872825in}{1.484856in}}{\pgfqpoint{0.864588in}{1.484856in}}%
\pgfpathcurveto{\pgfqpoint{0.856352in}{1.484856in}}{\pgfqpoint{0.848452in}{1.481583in}}{\pgfqpoint{0.842628in}{1.475759in}}%
\pgfpathcurveto{\pgfqpoint{0.836804in}{1.469935in}}{\pgfqpoint{0.833532in}{1.462035in}}{\pgfqpoint{0.833532in}{1.453799in}}%
\pgfpathcurveto{\pgfqpoint{0.833532in}{1.445563in}}{\pgfqpoint{0.836804in}{1.437663in}}{\pgfqpoint{0.842628in}{1.431839in}}%
\pgfpathcurveto{\pgfqpoint{0.848452in}{1.426015in}}{\pgfqpoint{0.856352in}{1.422743in}}{\pgfqpoint{0.864588in}{1.422743in}}%
\pgfpathclose%
\pgfusepath{stroke,fill}%
\end{pgfscope}%
\begin{pgfscope}%
\pgfpathrectangle{\pgfqpoint{0.100000in}{0.220728in}}{\pgfqpoint{3.696000in}{3.696000in}}%
\pgfusepath{clip}%
\pgfsetbuttcap%
\pgfsetroundjoin%
\definecolor{currentfill}{rgb}{0.121569,0.466667,0.705882}%
\pgfsetfillcolor{currentfill}%
\pgfsetfillopacity{0.657340}%
\pgfsetlinewidth{1.003750pt}%
\definecolor{currentstroke}{rgb}{0.121569,0.466667,0.705882}%
\pgfsetstrokecolor{currentstroke}%
\pgfsetstrokeopacity{0.657340}%
\pgfsetdash{}{0pt}%
\pgfpathmoveto{\pgfqpoint{0.864588in}{1.422743in}}%
\pgfpathcurveto{\pgfqpoint{0.872825in}{1.422743in}}{\pgfqpoint{0.880725in}{1.426015in}}{\pgfqpoint{0.886548in}{1.431839in}}%
\pgfpathcurveto{\pgfqpoint{0.892372in}{1.437663in}}{\pgfqpoint{0.895645in}{1.445563in}}{\pgfqpoint{0.895645in}{1.453799in}}%
\pgfpathcurveto{\pgfqpoint{0.895645in}{1.462035in}}{\pgfqpoint{0.892372in}{1.469935in}}{\pgfqpoint{0.886548in}{1.475759in}}%
\pgfpathcurveto{\pgfqpoint{0.880725in}{1.481583in}}{\pgfqpoint{0.872825in}{1.484856in}}{\pgfqpoint{0.864588in}{1.484856in}}%
\pgfpathcurveto{\pgfqpoint{0.856352in}{1.484856in}}{\pgfqpoint{0.848452in}{1.481583in}}{\pgfqpoint{0.842628in}{1.475759in}}%
\pgfpathcurveto{\pgfqpoint{0.836804in}{1.469935in}}{\pgfqpoint{0.833532in}{1.462035in}}{\pgfqpoint{0.833532in}{1.453799in}}%
\pgfpathcurveto{\pgfqpoint{0.833532in}{1.445563in}}{\pgfqpoint{0.836804in}{1.437663in}}{\pgfqpoint{0.842628in}{1.431839in}}%
\pgfpathcurveto{\pgfqpoint{0.848452in}{1.426015in}}{\pgfqpoint{0.856352in}{1.422743in}}{\pgfqpoint{0.864588in}{1.422743in}}%
\pgfpathclose%
\pgfusepath{stroke,fill}%
\end{pgfscope}%
\begin{pgfscope}%
\pgfpathrectangle{\pgfqpoint{0.100000in}{0.220728in}}{\pgfqpoint{3.696000in}{3.696000in}}%
\pgfusepath{clip}%
\pgfsetbuttcap%
\pgfsetroundjoin%
\definecolor{currentfill}{rgb}{0.121569,0.466667,0.705882}%
\pgfsetfillcolor{currentfill}%
\pgfsetfillopacity{0.657340}%
\pgfsetlinewidth{1.003750pt}%
\definecolor{currentstroke}{rgb}{0.121569,0.466667,0.705882}%
\pgfsetstrokecolor{currentstroke}%
\pgfsetstrokeopacity{0.657340}%
\pgfsetdash{}{0pt}%
\pgfpathmoveto{\pgfqpoint{0.864588in}{1.422743in}}%
\pgfpathcurveto{\pgfqpoint{0.872825in}{1.422743in}}{\pgfqpoint{0.880725in}{1.426015in}}{\pgfqpoint{0.886548in}{1.431839in}}%
\pgfpathcurveto{\pgfqpoint{0.892372in}{1.437663in}}{\pgfqpoint{0.895645in}{1.445563in}}{\pgfqpoint{0.895645in}{1.453799in}}%
\pgfpathcurveto{\pgfqpoint{0.895645in}{1.462035in}}{\pgfqpoint{0.892372in}{1.469935in}}{\pgfqpoint{0.886548in}{1.475759in}}%
\pgfpathcurveto{\pgfqpoint{0.880725in}{1.481583in}}{\pgfqpoint{0.872825in}{1.484856in}}{\pgfqpoint{0.864588in}{1.484856in}}%
\pgfpathcurveto{\pgfqpoint{0.856352in}{1.484856in}}{\pgfqpoint{0.848452in}{1.481583in}}{\pgfqpoint{0.842628in}{1.475759in}}%
\pgfpathcurveto{\pgfqpoint{0.836804in}{1.469935in}}{\pgfqpoint{0.833532in}{1.462035in}}{\pgfqpoint{0.833532in}{1.453799in}}%
\pgfpathcurveto{\pgfqpoint{0.833532in}{1.445563in}}{\pgfqpoint{0.836804in}{1.437663in}}{\pgfqpoint{0.842628in}{1.431839in}}%
\pgfpathcurveto{\pgfqpoint{0.848452in}{1.426015in}}{\pgfqpoint{0.856352in}{1.422743in}}{\pgfqpoint{0.864588in}{1.422743in}}%
\pgfpathclose%
\pgfusepath{stroke,fill}%
\end{pgfscope}%
\begin{pgfscope}%
\pgfpathrectangle{\pgfqpoint{0.100000in}{0.220728in}}{\pgfqpoint{3.696000in}{3.696000in}}%
\pgfusepath{clip}%
\pgfsetbuttcap%
\pgfsetroundjoin%
\definecolor{currentfill}{rgb}{0.121569,0.466667,0.705882}%
\pgfsetfillcolor{currentfill}%
\pgfsetfillopacity{0.657340}%
\pgfsetlinewidth{1.003750pt}%
\definecolor{currentstroke}{rgb}{0.121569,0.466667,0.705882}%
\pgfsetstrokecolor{currentstroke}%
\pgfsetstrokeopacity{0.657340}%
\pgfsetdash{}{0pt}%
\pgfpathmoveto{\pgfqpoint{0.864588in}{1.422743in}}%
\pgfpathcurveto{\pgfqpoint{0.872825in}{1.422743in}}{\pgfqpoint{0.880725in}{1.426015in}}{\pgfqpoint{0.886548in}{1.431839in}}%
\pgfpathcurveto{\pgfqpoint{0.892372in}{1.437663in}}{\pgfqpoint{0.895645in}{1.445563in}}{\pgfqpoint{0.895645in}{1.453799in}}%
\pgfpathcurveto{\pgfqpoint{0.895645in}{1.462035in}}{\pgfqpoint{0.892372in}{1.469935in}}{\pgfqpoint{0.886548in}{1.475759in}}%
\pgfpathcurveto{\pgfqpoint{0.880725in}{1.481583in}}{\pgfqpoint{0.872825in}{1.484856in}}{\pgfqpoint{0.864588in}{1.484856in}}%
\pgfpathcurveto{\pgfqpoint{0.856352in}{1.484856in}}{\pgfqpoint{0.848452in}{1.481583in}}{\pgfqpoint{0.842628in}{1.475759in}}%
\pgfpathcurveto{\pgfqpoint{0.836804in}{1.469935in}}{\pgfqpoint{0.833532in}{1.462035in}}{\pgfqpoint{0.833532in}{1.453799in}}%
\pgfpathcurveto{\pgfqpoint{0.833532in}{1.445563in}}{\pgfqpoint{0.836804in}{1.437663in}}{\pgfqpoint{0.842628in}{1.431839in}}%
\pgfpathcurveto{\pgfqpoint{0.848452in}{1.426015in}}{\pgfqpoint{0.856352in}{1.422743in}}{\pgfqpoint{0.864588in}{1.422743in}}%
\pgfpathclose%
\pgfusepath{stroke,fill}%
\end{pgfscope}%
\begin{pgfscope}%
\pgfpathrectangle{\pgfqpoint{0.100000in}{0.220728in}}{\pgfqpoint{3.696000in}{3.696000in}}%
\pgfusepath{clip}%
\pgfsetbuttcap%
\pgfsetroundjoin%
\definecolor{currentfill}{rgb}{0.121569,0.466667,0.705882}%
\pgfsetfillcolor{currentfill}%
\pgfsetfillopacity{0.657340}%
\pgfsetlinewidth{1.003750pt}%
\definecolor{currentstroke}{rgb}{0.121569,0.466667,0.705882}%
\pgfsetstrokecolor{currentstroke}%
\pgfsetstrokeopacity{0.657340}%
\pgfsetdash{}{0pt}%
\pgfpathmoveto{\pgfqpoint{0.864588in}{1.422743in}}%
\pgfpathcurveto{\pgfqpoint{0.872825in}{1.422743in}}{\pgfqpoint{0.880725in}{1.426015in}}{\pgfqpoint{0.886548in}{1.431839in}}%
\pgfpathcurveto{\pgfqpoint{0.892372in}{1.437663in}}{\pgfqpoint{0.895645in}{1.445563in}}{\pgfqpoint{0.895645in}{1.453799in}}%
\pgfpathcurveto{\pgfqpoint{0.895645in}{1.462035in}}{\pgfqpoint{0.892372in}{1.469935in}}{\pgfqpoint{0.886548in}{1.475759in}}%
\pgfpathcurveto{\pgfqpoint{0.880725in}{1.481583in}}{\pgfqpoint{0.872825in}{1.484856in}}{\pgfqpoint{0.864588in}{1.484856in}}%
\pgfpathcurveto{\pgfqpoint{0.856352in}{1.484856in}}{\pgfqpoint{0.848452in}{1.481583in}}{\pgfqpoint{0.842628in}{1.475759in}}%
\pgfpathcurveto{\pgfqpoint{0.836804in}{1.469935in}}{\pgfqpoint{0.833532in}{1.462035in}}{\pgfqpoint{0.833532in}{1.453799in}}%
\pgfpathcurveto{\pgfqpoint{0.833532in}{1.445563in}}{\pgfqpoint{0.836804in}{1.437663in}}{\pgfqpoint{0.842628in}{1.431839in}}%
\pgfpathcurveto{\pgfqpoint{0.848452in}{1.426015in}}{\pgfqpoint{0.856352in}{1.422743in}}{\pgfqpoint{0.864588in}{1.422743in}}%
\pgfpathclose%
\pgfusepath{stroke,fill}%
\end{pgfscope}%
\begin{pgfscope}%
\pgfpathrectangle{\pgfqpoint{0.100000in}{0.220728in}}{\pgfqpoint{3.696000in}{3.696000in}}%
\pgfusepath{clip}%
\pgfsetbuttcap%
\pgfsetroundjoin%
\definecolor{currentfill}{rgb}{0.121569,0.466667,0.705882}%
\pgfsetfillcolor{currentfill}%
\pgfsetfillopacity{0.657452}%
\pgfsetlinewidth{1.003750pt}%
\definecolor{currentstroke}{rgb}{0.121569,0.466667,0.705882}%
\pgfsetstrokecolor{currentstroke}%
\pgfsetstrokeopacity{0.657452}%
\pgfsetdash{}{0pt}%
\pgfpathmoveto{\pgfqpoint{0.864035in}{1.422563in}}%
\pgfpathcurveto{\pgfqpoint{0.872271in}{1.422563in}}{\pgfqpoint{0.880171in}{1.425836in}}{\pgfqpoint{0.885995in}{1.431660in}}%
\pgfpathcurveto{\pgfqpoint{0.891819in}{1.437484in}}{\pgfqpoint{0.895091in}{1.445384in}}{\pgfqpoint{0.895091in}{1.453620in}}%
\pgfpathcurveto{\pgfqpoint{0.895091in}{1.461856in}}{\pgfqpoint{0.891819in}{1.469756in}}{\pgfqpoint{0.885995in}{1.475580in}}%
\pgfpathcurveto{\pgfqpoint{0.880171in}{1.481404in}}{\pgfqpoint{0.872271in}{1.484676in}}{\pgfqpoint{0.864035in}{1.484676in}}%
\pgfpathcurveto{\pgfqpoint{0.855798in}{1.484676in}}{\pgfqpoint{0.847898in}{1.481404in}}{\pgfqpoint{0.842074in}{1.475580in}}%
\pgfpathcurveto{\pgfqpoint{0.836250in}{1.469756in}}{\pgfqpoint{0.832978in}{1.461856in}}{\pgfqpoint{0.832978in}{1.453620in}}%
\pgfpathcurveto{\pgfqpoint{0.832978in}{1.445384in}}{\pgfqpoint{0.836250in}{1.437484in}}{\pgfqpoint{0.842074in}{1.431660in}}%
\pgfpathcurveto{\pgfqpoint{0.847898in}{1.425836in}}{\pgfqpoint{0.855798in}{1.422563in}}{\pgfqpoint{0.864035in}{1.422563in}}%
\pgfpathclose%
\pgfusepath{stroke,fill}%
\end{pgfscope}%
\begin{pgfscope}%
\pgfpathrectangle{\pgfqpoint{0.100000in}{0.220728in}}{\pgfqpoint{3.696000in}{3.696000in}}%
\pgfusepath{clip}%
\pgfsetbuttcap%
\pgfsetroundjoin%
\definecolor{currentfill}{rgb}{0.121569,0.466667,0.705882}%
\pgfsetfillcolor{currentfill}%
\pgfsetfillopacity{0.657530}%
\pgfsetlinewidth{1.003750pt}%
\definecolor{currentstroke}{rgb}{0.121569,0.466667,0.705882}%
\pgfsetstrokecolor{currentstroke}%
\pgfsetstrokeopacity{0.657530}%
\pgfsetdash{}{0pt}%
\pgfpathmoveto{\pgfqpoint{2.145438in}{1.750527in}}%
\pgfpathcurveto{\pgfqpoint{2.153674in}{1.750527in}}{\pgfqpoint{2.161574in}{1.753799in}}{\pgfqpoint{2.167398in}{1.759623in}}%
\pgfpathcurveto{\pgfqpoint{2.173222in}{1.765447in}}{\pgfqpoint{2.176495in}{1.773347in}}{\pgfqpoint{2.176495in}{1.781583in}}%
\pgfpathcurveto{\pgfqpoint{2.176495in}{1.789819in}}{\pgfqpoint{2.173222in}{1.797719in}}{\pgfqpoint{2.167398in}{1.803543in}}%
\pgfpathcurveto{\pgfqpoint{2.161574in}{1.809367in}}{\pgfqpoint{2.153674in}{1.812640in}}{\pgfqpoint{2.145438in}{1.812640in}}%
\pgfpathcurveto{\pgfqpoint{2.137202in}{1.812640in}}{\pgfqpoint{2.129302in}{1.809367in}}{\pgfqpoint{2.123478in}{1.803543in}}%
\pgfpathcurveto{\pgfqpoint{2.117654in}{1.797719in}}{\pgfqpoint{2.114382in}{1.789819in}}{\pgfqpoint{2.114382in}{1.781583in}}%
\pgfpathcurveto{\pgfqpoint{2.114382in}{1.773347in}}{\pgfqpoint{2.117654in}{1.765447in}}{\pgfqpoint{2.123478in}{1.759623in}}%
\pgfpathcurveto{\pgfqpoint{2.129302in}{1.753799in}}{\pgfqpoint{2.137202in}{1.750527in}}{\pgfqpoint{2.145438in}{1.750527in}}%
\pgfpathclose%
\pgfusepath{stroke,fill}%
\end{pgfscope}%
\begin{pgfscope}%
\pgfpathrectangle{\pgfqpoint{0.100000in}{0.220728in}}{\pgfqpoint{3.696000in}{3.696000in}}%
\pgfusepath{clip}%
\pgfsetbuttcap%
\pgfsetroundjoin%
\definecolor{currentfill}{rgb}{0.121569,0.466667,0.705882}%
\pgfsetfillcolor{currentfill}%
\pgfsetfillopacity{0.657567}%
\pgfsetlinewidth{1.003750pt}%
\definecolor{currentstroke}{rgb}{0.121569,0.466667,0.705882}%
\pgfsetstrokecolor{currentstroke}%
\pgfsetstrokeopacity{0.657567}%
\pgfsetdash{}{0pt}%
\pgfpathmoveto{\pgfqpoint{0.855887in}{1.422799in}}%
\pgfpathcurveto{\pgfqpoint{0.864123in}{1.422799in}}{\pgfqpoint{0.872023in}{1.426071in}}{\pgfqpoint{0.877847in}{1.431895in}}%
\pgfpathcurveto{\pgfqpoint{0.883671in}{1.437719in}}{\pgfqpoint{0.886943in}{1.445619in}}{\pgfqpoint{0.886943in}{1.453856in}}%
\pgfpathcurveto{\pgfqpoint{0.886943in}{1.462092in}}{\pgfqpoint{0.883671in}{1.469992in}}{\pgfqpoint{0.877847in}{1.475816in}}%
\pgfpathcurveto{\pgfqpoint{0.872023in}{1.481640in}}{\pgfqpoint{0.864123in}{1.484912in}}{\pgfqpoint{0.855887in}{1.484912in}}%
\pgfpathcurveto{\pgfqpoint{0.847651in}{1.484912in}}{\pgfqpoint{0.839750in}{1.481640in}}{\pgfqpoint{0.833927in}{1.475816in}}%
\pgfpathcurveto{\pgfqpoint{0.828103in}{1.469992in}}{\pgfqpoint{0.824830in}{1.462092in}}{\pgfqpoint{0.824830in}{1.453856in}}%
\pgfpathcurveto{\pgfqpoint{0.824830in}{1.445619in}}{\pgfqpoint{0.828103in}{1.437719in}}{\pgfqpoint{0.833927in}{1.431895in}}%
\pgfpathcurveto{\pgfqpoint{0.839750in}{1.426071in}}{\pgfqpoint{0.847651in}{1.422799in}}{\pgfqpoint{0.855887in}{1.422799in}}%
\pgfpathclose%
\pgfusepath{stroke,fill}%
\end{pgfscope}%
\begin{pgfscope}%
\pgfpathrectangle{\pgfqpoint{0.100000in}{0.220728in}}{\pgfqpoint{3.696000in}{3.696000in}}%
\pgfusepath{clip}%
\pgfsetbuttcap%
\pgfsetroundjoin%
\definecolor{currentfill}{rgb}{0.121569,0.466667,0.705882}%
\pgfsetfillcolor{currentfill}%
\pgfsetfillopacity{0.657636}%
\pgfsetlinewidth{1.003750pt}%
\definecolor{currentstroke}{rgb}{0.121569,0.466667,0.705882}%
\pgfsetstrokecolor{currentstroke}%
\pgfsetstrokeopacity{0.657636}%
\pgfsetdash{}{0pt}%
\pgfpathmoveto{\pgfqpoint{0.862639in}{1.422320in}}%
\pgfpathcurveto{\pgfqpoint{0.870875in}{1.422320in}}{\pgfqpoint{0.878775in}{1.425592in}}{\pgfqpoint{0.884599in}{1.431416in}}%
\pgfpathcurveto{\pgfqpoint{0.890423in}{1.437240in}}{\pgfqpoint{0.893695in}{1.445140in}}{\pgfqpoint{0.893695in}{1.453377in}}%
\pgfpathcurveto{\pgfqpoint{0.893695in}{1.461613in}}{\pgfqpoint{0.890423in}{1.469513in}}{\pgfqpoint{0.884599in}{1.475337in}}%
\pgfpathcurveto{\pgfqpoint{0.878775in}{1.481161in}}{\pgfqpoint{0.870875in}{1.484433in}}{\pgfqpoint{0.862639in}{1.484433in}}%
\pgfpathcurveto{\pgfqpoint{0.854402in}{1.484433in}}{\pgfqpoint{0.846502in}{1.481161in}}{\pgfqpoint{0.840678in}{1.475337in}}%
\pgfpathcurveto{\pgfqpoint{0.834855in}{1.469513in}}{\pgfqpoint{0.831582in}{1.461613in}}{\pgfqpoint{0.831582in}{1.453377in}}%
\pgfpathcurveto{\pgfqpoint{0.831582in}{1.445140in}}{\pgfqpoint{0.834855in}{1.437240in}}{\pgfqpoint{0.840678in}{1.431416in}}%
\pgfpathcurveto{\pgfqpoint{0.846502in}{1.425592in}}{\pgfqpoint{0.854402in}{1.422320in}}{\pgfqpoint{0.862639in}{1.422320in}}%
\pgfpathclose%
\pgfusepath{stroke,fill}%
\end{pgfscope}%
\begin{pgfscope}%
\pgfpathrectangle{\pgfqpoint{0.100000in}{0.220728in}}{\pgfqpoint{3.696000in}{3.696000in}}%
\pgfusepath{clip}%
\pgfsetbuttcap%
\pgfsetroundjoin%
\definecolor{currentfill}{rgb}{0.121569,0.466667,0.705882}%
\pgfsetfillcolor{currentfill}%
\pgfsetfillopacity{0.657796}%
\pgfsetlinewidth{1.003750pt}%
\definecolor{currentstroke}{rgb}{0.121569,0.466667,0.705882}%
\pgfsetstrokecolor{currentstroke}%
\pgfsetstrokeopacity{0.657796}%
\pgfsetdash{}{0pt}%
\pgfpathmoveto{\pgfqpoint{0.860032in}{1.422223in}}%
\pgfpathcurveto{\pgfqpoint{0.868268in}{1.422223in}}{\pgfqpoint{0.876168in}{1.425496in}}{\pgfqpoint{0.881992in}{1.431320in}}%
\pgfpathcurveto{\pgfqpoint{0.887816in}{1.437143in}}{\pgfqpoint{0.891089in}{1.445043in}}{\pgfqpoint{0.891089in}{1.453280in}}%
\pgfpathcurveto{\pgfqpoint{0.891089in}{1.461516in}}{\pgfqpoint{0.887816in}{1.469416in}}{\pgfqpoint{0.881992in}{1.475240in}}%
\pgfpathcurveto{\pgfqpoint{0.876168in}{1.481064in}}{\pgfqpoint{0.868268in}{1.484336in}}{\pgfqpoint{0.860032in}{1.484336in}}%
\pgfpathcurveto{\pgfqpoint{0.851796in}{1.484336in}}{\pgfqpoint{0.843896in}{1.481064in}}{\pgfqpoint{0.838072in}{1.475240in}}%
\pgfpathcurveto{\pgfqpoint{0.832248in}{1.469416in}}{\pgfqpoint{0.828976in}{1.461516in}}{\pgfqpoint{0.828976in}{1.453280in}}%
\pgfpathcurveto{\pgfqpoint{0.828976in}{1.445043in}}{\pgfqpoint{0.832248in}{1.437143in}}{\pgfqpoint{0.838072in}{1.431320in}}%
\pgfpathcurveto{\pgfqpoint{0.843896in}{1.425496in}}{\pgfqpoint{0.851796in}{1.422223in}}{\pgfqpoint{0.860032in}{1.422223in}}%
\pgfpathclose%
\pgfusepath{stroke,fill}%
\end{pgfscope}%
\begin{pgfscope}%
\pgfpathrectangle{\pgfqpoint{0.100000in}{0.220728in}}{\pgfqpoint{3.696000in}{3.696000in}}%
\pgfusepath{clip}%
\pgfsetbuttcap%
\pgfsetroundjoin%
\definecolor{currentfill}{rgb}{0.121569,0.466667,0.705882}%
\pgfsetfillcolor{currentfill}%
\pgfsetfillopacity{0.662081}%
\pgfsetlinewidth{1.003750pt}%
\definecolor{currentstroke}{rgb}{0.121569,0.466667,0.705882}%
\pgfsetstrokecolor{currentstroke}%
\pgfsetstrokeopacity{0.662081}%
\pgfsetdash{}{0pt}%
\pgfpathmoveto{\pgfqpoint{2.147696in}{1.745851in}}%
\pgfpathcurveto{\pgfqpoint{2.155933in}{1.745851in}}{\pgfqpoint{2.163833in}{1.749124in}}{\pgfqpoint{2.169657in}{1.754948in}}%
\pgfpathcurveto{\pgfqpoint{2.175481in}{1.760772in}}{\pgfqpoint{2.178753in}{1.768672in}}{\pgfqpoint{2.178753in}{1.776908in}}%
\pgfpathcurveto{\pgfqpoint{2.178753in}{1.785144in}}{\pgfqpoint{2.175481in}{1.793044in}}{\pgfqpoint{2.169657in}{1.798868in}}%
\pgfpathcurveto{\pgfqpoint{2.163833in}{1.804692in}}{\pgfqpoint{2.155933in}{1.807964in}}{\pgfqpoint{2.147696in}{1.807964in}}%
\pgfpathcurveto{\pgfqpoint{2.139460in}{1.807964in}}{\pgfqpoint{2.131560in}{1.804692in}}{\pgfqpoint{2.125736in}{1.798868in}}%
\pgfpathcurveto{\pgfqpoint{2.119912in}{1.793044in}}{\pgfqpoint{2.116640in}{1.785144in}}{\pgfqpoint{2.116640in}{1.776908in}}%
\pgfpathcurveto{\pgfqpoint{2.116640in}{1.768672in}}{\pgfqpoint{2.119912in}{1.760772in}}{\pgfqpoint{2.125736in}{1.754948in}}%
\pgfpathcurveto{\pgfqpoint{2.131560in}{1.749124in}}{\pgfqpoint{2.139460in}{1.745851in}}{\pgfqpoint{2.147696in}{1.745851in}}%
\pgfpathclose%
\pgfusepath{stroke,fill}%
\end{pgfscope}%
\begin{pgfscope}%
\pgfpathrectangle{\pgfqpoint{0.100000in}{0.220728in}}{\pgfqpoint{3.696000in}{3.696000in}}%
\pgfusepath{clip}%
\pgfsetbuttcap%
\pgfsetroundjoin%
\definecolor{currentfill}{rgb}{0.121569,0.466667,0.705882}%
\pgfsetfillcolor{currentfill}%
\pgfsetfillopacity{0.666939}%
\pgfsetlinewidth{1.003750pt}%
\definecolor{currentstroke}{rgb}{0.121569,0.466667,0.705882}%
\pgfsetstrokecolor{currentstroke}%
\pgfsetstrokeopacity{0.666939}%
\pgfsetdash{}{0pt}%
\pgfpathmoveto{\pgfqpoint{2.151038in}{1.738760in}}%
\pgfpathcurveto{\pgfqpoint{2.159274in}{1.738760in}}{\pgfqpoint{2.167174in}{1.742033in}}{\pgfqpoint{2.172998in}{1.747856in}}%
\pgfpathcurveto{\pgfqpoint{2.178822in}{1.753680in}}{\pgfqpoint{2.182095in}{1.761580in}}{\pgfqpoint{2.182095in}{1.769817in}}%
\pgfpathcurveto{\pgfqpoint{2.182095in}{1.778053in}}{\pgfqpoint{2.178822in}{1.785953in}}{\pgfqpoint{2.172998in}{1.791777in}}%
\pgfpathcurveto{\pgfqpoint{2.167174in}{1.797601in}}{\pgfqpoint{2.159274in}{1.800873in}}{\pgfqpoint{2.151038in}{1.800873in}}%
\pgfpathcurveto{\pgfqpoint{2.142802in}{1.800873in}}{\pgfqpoint{2.134902in}{1.797601in}}{\pgfqpoint{2.129078in}{1.791777in}}%
\pgfpathcurveto{\pgfqpoint{2.123254in}{1.785953in}}{\pgfqpoint{2.119982in}{1.778053in}}{\pgfqpoint{2.119982in}{1.769817in}}%
\pgfpathcurveto{\pgfqpoint{2.119982in}{1.761580in}}{\pgfqpoint{2.123254in}{1.753680in}}{\pgfqpoint{2.129078in}{1.747856in}}%
\pgfpathcurveto{\pgfqpoint{2.134902in}{1.742033in}}{\pgfqpoint{2.142802in}{1.738760in}}{\pgfqpoint{2.151038in}{1.738760in}}%
\pgfpathclose%
\pgfusepath{stroke,fill}%
\end{pgfscope}%
\begin{pgfscope}%
\pgfpathrectangle{\pgfqpoint{0.100000in}{0.220728in}}{\pgfqpoint{3.696000in}{3.696000in}}%
\pgfusepath{clip}%
\pgfsetbuttcap%
\pgfsetroundjoin%
\definecolor{currentfill}{rgb}{0.121569,0.466667,0.705882}%
\pgfsetfillcolor{currentfill}%
\pgfsetfillopacity{0.672707}%
\pgfsetlinewidth{1.003750pt}%
\definecolor{currentstroke}{rgb}{0.121569,0.466667,0.705882}%
\pgfsetstrokecolor{currentstroke}%
\pgfsetstrokeopacity{0.672707}%
\pgfsetdash{}{0pt}%
\pgfpathmoveto{\pgfqpoint{2.157478in}{1.736543in}}%
\pgfpathcurveto{\pgfqpoint{2.165715in}{1.736543in}}{\pgfqpoint{2.173615in}{1.739816in}}{\pgfqpoint{2.179439in}{1.745640in}}%
\pgfpathcurveto{\pgfqpoint{2.185262in}{1.751464in}}{\pgfqpoint{2.188535in}{1.759364in}}{\pgfqpoint{2.188535in}{1.767600in}}%
\pgfpathcurveto{\pgfqpoint{2.188535in}{1.775836in}}{\pgfqpoint{2.185262in}{1.783736in}}{\pgfqpoint{2.179439in}{1.789560in}}%
\pgfpathcurveto{\pgfqpoint{2.173615in}{1.795384in}}{\pgfqpoint{2.165715in}{1.798656in}}{\pgfqpoint{2.157478in}{1.798656in}}%
\pgfpathcurveto{\pgfqpoint{2.149242in}{1.798656in}}{\pgfqpoint{2.141342in}{1.795384in}}{\pgfqpoint{2.135518in}{1.789560in}}%
\pgfpathcurveto{\pgfqpoint{2.129694in}{1.783736in}}{\pgfqpoint{2.126422in}{1.775836in}}{\pgfqpoint{2.126422in}{1.767600in}}%
\pgfpathcurveto{\pgfqpoint{2.126422in}{1.759364in}}{\pgfqpoint{2.129694in}{1.751464in}}{\pgfqpoint{2.135518in}{1.745640in}}%
\pgfpathcurveto{\pgfqpoint{2.141342in}{1.739816in}}{\pgfqpoint{2.149242in}{1.736543in}}{\pgfqpoint{2.157478in}{1.736543in}}%
\pgfpathclose%
\pgfusepath{stroke,fill}%
\end{pgfscope}%
\begin{pgfscope}%
\pgfpathrectangle{\pgfqpoint{0.100000in}{0.220728in}}{\pgfqpoint{3.696000in}{3.696000in}}%
\pgfusepath{clip}%
\pgfsetbuttcap%
\pgfsetroundjoin%
\definecolor{currentfill}{rgb}{0.121569,0.466667,0.705882}%
\pgfsetfillcolor{currentfill}%
\pgfsetfillopacity{0.678967}%
\pgfsetlinewidth{1.003750pt}%
\definecolor{currentstroke}{rgb}{0.121569,0.466667,0.705882}%
\pgfsetstrokecolor{currentstroke}%
\pgfsetstrokeopacity{0.678967}%
\pgfsetdash{}{0pt}%
\pgfpathmoveto{\pgfqpoint{2.160594in}{1.729454in}}%
\pgfpathcurveto{\pgfqpoint{2.168831in}{1.729454in}}{\pgfqpoint{2.176731in}{1.732726in}}{\pgfqpoint{2.182555in}{1.738550in}}%
\pgfpathcurveto{\pgfqpoint{2.188378in}{1.744374in}}{\pgfqpoint{2.191651in}{1.752274in}}{\pgfqpoint{2.191651in}{1.760511in}}%
\pgfpathcurveto{\pgfqpoint{2.191651in}{1.768747in}}{\pgfqpoint{2.188378in}{1.776647in}}{\pgfqpoint{2.182555in}{1.782471in}}%
\pgfpathcurveto{\pgfqpoint{2.176731in}{1.788295in}}{\pgfqpoint{2.168831in}{1.791567in}}{\pgfqpoint{2.160594in}{1.791567in}}%
\pgfpathcurveto{\pgfqpoint{2.152358in}{1.791567in}}{\pgfqpoint{2.144458in}{1.788295in}}{\pgfqpoint{2.138634in}{1.782471in}}%
\pgfpathcurveto{\pgfqpoint{2.132810in}{1.776647in}}{\pgfqpoint{2.129538in}{1.768747in}}{\pgfqpoint{2.129538in}{1.760511in}}%
\pgfpathcurveto{\pgfqpoint{2.129538in}{1.752274in}}{\pgfqpoint{2.132810in}{1.744374in}}{\pgfqpoint{2.138634in}{1.738550in}}%
\pgfpathcurveto{\pgfqpoint{2.144458in}{1.732726in}}{\pgfqpoint{2.152358in}{1.729454in}}{\pgfqpoint{2.160594in}{1.729454in}}%
\pgfpathclose%
\pgfusepath{stroke,fill}%
\end{pgfscope}%
\begin{pgfscope}%
\pgfpathrectangle{\pgfqpoint{0.100000in}{0.220728in}}{\pgfqpoint{3.696000in}{3.696000in}}%
\pgfusepath{clip}%
\pgfsetbuttcap%
\pgfsetroundjoin%
\definecolor{currentfill}{rgb}{0.121569,0.466667,0.705882}%
\pgfsetfillcolor{currentfill}%
\pgfsetfillopacity{0.684971}%
\pgfsetlinewidth{1.003750pt}%
\definecolor{currentstroke}{rgb}{0.121569,0.466667,0.705882}%
\pgfsetstrokecolor{currentstroke}%
\pgfsetstrokeopacity{0.684971}%
\pgfsetdash{}{0pt}%
\pgfpathmoveto{\pgfqpoint{2.163139in}{1.717480in}}%
\pgfpathcurveto{\pgfqpoint{2.171375in}{1.717480in}}{\pgfqpoint{2.179275in}{1.720752in}}{\pgfqpoint{2.185099in}{1.726576in}}%
\pgfpathcurveto{\pgfqpoint{2.190923in}{1.732400in}}{\pgfqpoint{2.194195in}{1.740300in}}{\pgfqpoint{2.194195in}{1.748536in}}%
\pgfpathcurveto{\pgfqpoint{2.194195in}{1.756772in}}{\pgfqpoint{2.190923in}{1.764672in}}{\pgfqpoint{2.185099in}{1.770496in}}%
\pgfpathcurveto{\pgfqpoint{2.179275in}{1.776320in}}{\pgfqpoint{2.171375in}{1.779593in}}{\pgfqpoint{2.163139in}{1.779593in}}%
\pgfpathcurveto{\pgfqpoint{2.154903in}{1.779593in}}{\pgfqpoint{2.147002in}{1.776320in}}{\pgfqpoint{2.141179in}{1.770496in}}%
\pgfpathcurveto{\pgfqpoint{2.135355in}{1.764672in}}{\pgfqpoint{2.132082in}{1.756772in}}{\pgfqpoint{2.132082in}{1.748536in}}%
\pgfpathcurveto{\pgfqpoint{2.132082in}{1.740300in}}{\pgfqpoint{2.135355in}{1.732400in}}{\pgfqpoint{2.141179in}{1.726576in}}%
\pgfpathcurveto{\pgfqpoint{2.147002in}{1.720752in}}{\pgfqpoint{2.154903in}{1.717480in}}{\pgfqpoint{2.163139in}{1.717480in}}%
\pgfpathclose%
\pgfusepath{stroke,fill}%
\end{pgfscope}%
\begin{pgfscope}%
\pgfpathrectangle{\pgfqpoint{0.100000in}{0.220728in}}{\pgfqpoint{3.696000in}{3.696000in}}%
\pgfusepath{clip}%
\pgfsetbuttcap%
\pgfsetroundjoin%
\definecolor{currentfill}{rgb}{0.121569,0.466667,0.705882}%
\pgfsetfillcolor{currentfill}%
\pgfsetfillopacity{0.691385}%
\pgfsetlinewidth{1.003750pt}%
\definecolor{currentstroke}{rgb}{0.121569,0.466667,0.705882}%
\pgfsetstrokecolor{currentstroke}%
\pgfsetstrokeopacity{0.691385}%
\pgfsetdash{}{0pt}%
\pgfpathmoveto{\pgfqpoint{2.167626in}{1.705057in}}%
\pgfpathcurveto{\pgfqpoint{2.175863in}{1.705057in}}{\pgfqpoint{2.183763in}{1.708330in}}{\pgfqpoint{2.189587in}{1.714154in}}%
\pgfpathcurveto{\pgfqpoint{2.195411in}{1.719978in}}{\pgfqpoint{2.198683in}{1.727878in}}{\pgfqpoint{2.198683in}{1.736114in}}%
\pgfpathcurveto{\pgfqpoint{2.198683in}{1.744350in}}{\pgfqpoint{2.195411in}{1.752250in}}{\pgfqpoint{2.189587in}{1.758074in}}%
\pgfpathcurveto{\pgfqpoint{2.183763in}{1.763898in}}{\pgfqpoint{2.175863in}{1.767170in}}{\pgfqpoint{2.167626in}{1.767170in}}%
\pgfpathcurveto{\pgfqpoint{2.159390in}{1.767170in}}{\pgfqpoint{2.151490in}{1.763898in}}{\pgfqpoint{2.145666in}{1.758074in}}%
\pgfpathcurveto{\pgfqpoint{2.139842in}{1.752250in}}{\pgfqpoint{2.136570in}{1.744350in}}{\pgfqpoint{2.136570in}{1.736114in}}%
\pgfpathcurveto{\pgfqpoint{2.136570in}{1.727878in}}{\pgfqpoint{2.139842in}{1.719978in}}{\pgfqpoint{2.145666in}{1.714154in}}%
\pgfpathcurveto{\pgfqpoint{2.151490in}{1.708330in}}{\pgfqpoint{2.159390in}{1.705057in}}{\pgfqpoint{2.167626in}{1.705057in}}%
\pgfpathclose%
\pgfusepath{stroke,fill}%
\end{pgfscope}%
\begin{pgfscope}%
\pgfpathrectangle{\pgfqpoint{0.100000in}{0.220728in}}{\pgfqpoint{3.696000in}{3.696000in}}%
\pgfusepath{clip}%
\pgfsetbuttcap%
\pgfsetroundjoin%
\definecolor{currentfill}{rgb}{0.121569,0.466667,0.705882}%
\pgfsetfillcolor{currentfill}%
\pgfsetfillopacity{0.699703}%
\pgfsetlinewidth{1.003750pt}%
\definecolor{currentstroke}{rgb}{0.121569,0.466667,0.705882}%
\pgfsetstrokecolor{currentstroke}%
\pgfsetstrokeopacity{0.699703}%
\pgfsetdash{}{0pt}%
\pgfpathmoveto{\pgfqpoint{2.175747in}{1.705862in}}%
\pgfpathcurveto{\pgfqpoint{2.183984in}{1.705862in}}{\pgfqpoint{2.191884in}{1.709134in}}{\pgfqpoint{2.197708in}{1.714958in}}%
\pgfpathcurveto{\pgfqpoint{2.203532in}{1.720782in}}{\pgfqpoint{2.206804in}{1.728682in}}{\pgfqpoint{2.206804in}{1.736918in}}%
\pgfpathcurveto{\pgfqpoint{2.206804in}{1.745155in}}{\pgfqpoint{2.203532in}{1.753055in}}{\pgfqpoint{2.197708in}{1.758879in}}%
\pgfpathcurveto{\pgfqpoint{2.191884in}{1.764702in}}{\pgfqpoint{2.183984in}{1.767975in}}{\pgfqpoint{2.175747in}{1.767975in}}%
\pgfpathcurveto{\pgfqpoint{2.167511in}{1.767975in}}{\pgfqpoint{2.159611in}{1.764702in}}{\pgfqpoint{2.153787in}{1.758879in}}%
\pgfpathcurveto{\pgfqpoint{2.147963in}{1.753055in}}{\pgfqpoint{2.144691in}{1.745155in}}{\pgfqpoint{2.144691in}{1.736918in}}%
\pgfpathcurveto{\pgfqpoint{2.144691in}{1.728682in}}{\pgfqpoint{2.147963in}{1.720782in}}{\pgfqpoint{2.153787in}{1.714958in}}%
\pgfpathcurveto{\pgfqpoint{2.159611in}{1.709134in}}{\pgfqpoint{2.167511in}{1.705862in}}{\pgfqpoint{2.175747in}{1.705862in}}%
\pgfpathclose%
\pgfusepath{stroke,fill}%
\end{pgfscope}%
\begin{pgfscope}%
\pgfpathrectangle{\pgfqpoint{0.100000in}{0.220728in}}{\pgfqpoint{3.696000in}{3.696000in}}%
\pgfusepath{clip}%
\pgfsetbuttcap%
\pgfsetroundjoin%
\definecolor{currentfill}{rgb}{0.121569,0.466667,0.705882}%
\pgfsetfillcolor{currentfill}%
\pgfsetfillopacity{0.703700}%
\pgfsetlinewidth{1.003750pt}%
\definecolor{currentstroke}{rgb}{0.121569,0.466667,0.705882}%
\pgfsetstrokecolor{currentstroke}%
\pgfsetstrokeopacity{0.703700}%
\pgfsetdash{}{0pt}%
\pgfpathmoveto{\pgfqpoint{2.178160in}{1.701218in}}%
\pgfpathcurveto{\pgfqpoint{2.186396in}{1.701218in}}{\pgfqpoint{2.194296in}{1.704491in}}{\pgfqpoint{2.200120in}{1.710314in}}%
\pgfpathcurveto{\pgfqpoint{2.205944in}{1.716138in}}{\pgfqpoint{2.209217in}{1.724038in}}{\pgfqpoint{2.209217in}{1.732275in}}%
\pgfpathcurveto{\pgfqpoint{2.209217in}{1.740511in}}{\pgfqpoint{2.205944in}{1.748411in}}{\pgfqpoint{2.200120in}{1.754235in}}%
\pgfpathcurveto{\pgfqpoint{2.194296in}{1.760059in}}{\pgfqpoint{2.186396in}{1.763331in}}{\pgfqpoint{2.178160in}{1.763331in}}%
\pgfpathcurveto{\pgfqpoint{2.169924in}{1.763331in}}{\pgfqpoint{2.162024in}{1.760059in}}{\pgfqpoint{2.156200in}{1.754235in}}%
\pgfpathcurveto{\pgfqpoint{2.150376in}{1.748411in}}{\pgfqpoint{2.147104in}{1.740511in}}{\pgfqpoint{2.147104in}{1.732275in}}%
\pgfpathcurveto{\pgfqpoint{2.147104in}{1.724038in}}{\pgfqpoint{2.150376in}{1.716138in}}{\pgfqpoint{2.156200in}{1.710314in}}%
\pgfpathcurveto{\pgfqpoint{2.162024in}{1.704491in}}{\pgfqpoint{2.169924in}{1.701218in}}{\pgfqpoint{2.178160in}{1.701218in}}%
\pgfpathclose%
\pgfusepath{stroke,fill}%
\end{pgfscope}%
\begin{pgfscope}%
\pgfpathrectangle{\pgfqpoint{0.100000in}{0.220728in}}{\pgfqpoint{3.696000in}{3.696000in}}%
\pgfusepath{clip}%
\pgfsetbuttcap%
\pgfsetroundjoin%
\definecolor{currentfill}{rgb}{0.121569,0.466667,0.705882}%
\pgfsetfillcolor{currentfill}%
\pgfsetfillopacity{0.707091}%
\pgfsetlinewidth{1.003750pt}%
\definecolor{currentstroke}{rgb}{0.121569,0.466667,0.705882}%
\pgfsetstrokecolor{currentstroke}%
\pgfsetstrokeopacity{0.707091}%
\pgfsetdash{}{0pt}%
\pgfpathmoveto{\pgfqpoint{2.181012in}{1.691768in}}%
\pgfpathcurveto{\pgfqpoint{2.189248in}{1.691768in}}{\pgfqpoint{2.197148in}{1.695041in}}{\pgfqpoint{2.202972in}{1.700864in}}%
\pgfpathcurveto{\pgfqpoint{2.208796in}{1.706688in}}{\pgfqpoint{2.212068in}{1.714588in}}{\pgfqpoint{2.212068in}{1.722825in}}%
\pgfpathcurveto{\pgfqpoint{2.212068in}{1.731061in}}{\pgfqpoint{2.208796in}{1.738961in}}{\pgfqpoint{2.202972in}{1.744785in}}%
\pgfpathcurveto{\pgfqpoint{2.197148in}{1.750609in}}{\pgfqpoint{2.189248in}{1.753881in}}{\pgfqpoint{2.181012in}{1.753881in}}%
\pgfpathcurveto{\pgfqpoint{2.172776in}{1.753881in}}{\pgfqpoint{2.164875in}{1.750609in}}{\pgfqpoint{2.159052in}{1.744785in}}%
\pgfpathcurveto{\pgfqpoint{2.153228in}{1.738961in}}{\pgfqpoint{2.149955in}{1.731061in}}{\pgfqpoint{2.149955in}{1.722825in}}%
\pgfpathcurveto{\pgfqpoint{2.149955in}{1.714588in}}{\pgfqpoint{2.153228in}{1.706688in}}{\pgfqpoint{2.159052in}{1.700864in}}%
\pgfpathcurveto{\pgfqpoint{2.164875in}{1.695041in}}{\pgfqpoint{2.172776in}{1.691768in}}{\pgfqpoint{2.181012in}{1.691768in}}%
\pgfpathclose%
\pgfusepath{stroke,fill}%
\end{pgfscope}%
\begin{pgfscope}%
\pgfpathrectangle{\pgfqpoint{0.100000in}{0.220728in}}{\pgfqpoint{3.696000in}{3.696000in}}%
\pgfusepath{clip}%
\pgfsetbuttcap%
\pgfsetroundjoin%
\definecolor{currentfill}{rgb}{0.121569,0.466667,0.705882}%
\pgfsetfillcolor{currentfill}%
\pgfsetfillopacity{0.709203}%
\pgfsetlinewidth{1.003750pt}%
\definecolor{currentstroke}{rgb}{0.121569,0.466667,0.705882}%
\pgfsetstrokecolor{currentstroke}%
\pgfsetstrokeopacity{0.709203}%
\pgfsetdash{}{0pt}%
\pgfpathmoveto{\pgfqpoint{2.183147in}{1.688479in}}%
\pgfpathcurveto{\pgfqpoint{2.191384in}{1.688479in}}{\pgfqpoint{2.199284in}{1.691751in}}{\pgfqpoint{2.205108in}{1.697575in}}%
\pgfpathcurveto{\pgfqpoint{2.210931in}{1.703399in}}{\pgfqpoint{2.214204in}{1.711299in}}{\pgfqpoint{2.214204in}{1.719536in}}%
\pgfpathcurveto{\pgfqpoint{2.214204in}{1.727772in}}{\pgfqpoint{2.210931in}{1.735672in}}{\pgfqpoint{2.205108in}{1.741496in}}%
\pgfpathcurveto{\pgfqpoint{2.199284in}{1.747320in}}{\pgfqpoint{2.191384in}{1.750592in}}{\pgfqpoint{2.183147in}{1.750592in}}%
\pgfpathcurveto{\pgfqpoint{2.174911in}{1.750592in}}{\pgfqpoint{2.167011in}{1.747320in}}{\pgfqpoint{2.161187in}{1.741496in}}%
\pgfpathcurveto{\pgfqpoint{2.155363in}{1.735672in}}{\pgfqpoint{2.152091in}{1.727772in}}{\pgfqpoint{2.152091in}{1.719536in}}%
\pgfpathcurveto{\pgfqpoint{2.152091in}{1.711299in}}{\pgfqpoint{2.155363in}{1.703399in}}{\pgfqpoint{2.161187in}{1.697575in}}%
\pgfpathcurveto{\pgfqpoint{2.167011in}{1.691751in}}{\pgfqpoint{2.174911in}{1.688479in}}{\pgfqpoint{2.183147in}{1.688479in}}%
\pgfpathclose%
\pgfusepath{stroke,fill}%
\end{pgfscope}%
\begin{pgfscope}%
\pgfpathrectangle{\pgfqpoint{0.100000in}{0.220728in}}{\pgfqpoint{3.696000in}{3.696000in}}%
\pgfusepath{clip}%
\pgfsetbuttcap%
\pgfsetroundjoin%
\definecolor{currentfill}{rgb}{0.121569,0.466667,0.705882}%
\pgfsetfillcolor{currentfill}%
\pgfsetfillopacity{0.710587}%
\pgfsetlinewidth{1.003750pt}%
\definecolor{currentstroke}{rgb}{0.121569,0.466667,0.705882}%
\pgfsetstrokecolor{currentstroke}%
\pgfsetstrokeopacity{0.710587}%
\pgfsetdash{}{0pt}%
\pgfpathmoveto{\pgfqpoint{2.184324in}{1.688083in}}%
\pgfpathcurveto{\pgfqpoint{2.192560in}{1.688083in}}{\pgfqpoint{2.200460in}{1.691355in}}{\pgfqpoint{2.206284in}{1.697179in}}%
\pgfpathcurveto{\pgfqpoint{2.212108in}{1.703003in}}{\pgfqpoint{2.215380in}{1.710903in}}{\pgfqpoint{2.215380in}{1.719139in}}%
\pgfpathcurveto{\pgfqpoint{2.215380in}{1.727376in}}{\pgfqpoint{2.212108in}{1.735276in}}{\pgfqpoint{2.206284in}{1.741100in}}%
\pgfpathcurveto{\pgfqpoint{2.200460in}{1.746924in}}{\pgfqpoint{2.192560in}{1.750196in}}{\pgfqpoint{2.184324in}{1.750196in}}%
\pgfpathcurveto{\pgfqpoint{2.176088in}{1.750196in}}{\pgfqpoint{2.168188in}{1.746924in}}{\pgfqpoint{2.162364in}{1.741100in}}%
\pgfpathcurveto{\pgfqpoint{2.156540in}{1.735276in}}{\pgfqpoint{2.153267in}{1.727376in}}{\pgfqpoint{2.153267in}{1.719139in}}%
\pgfpathcurveto{\pgfqpoint{2.153267in}{1.710903in}}{\pgfqpoint{2.156540in}{1.703003in}}{\pgfqpoint{2.162364in}{1.697179in}}%
\pgfpathcurveto{\pgfqpoint{2.168188in}{1.691355in}}{\pgfqpoint{2.176088in}{1.688083in}}{\pgfqpoint{2.184324in}{1.688083in}}%
\pgfpathclose%
\pgfusepath{stroke,fill}%
\end{pgfscope}%
\begin{pgfscope}%
\pgfpathrectangle{\pgfqpoint{0.100000in}{0.220728in}}{\pgfqpoint{3.696000in}{3.696000in}}%
\pgfusepath{clip}%
\pgfsetbuttcap%
\pgfsetroundjoin%
\definecolor{currentfill}{rgb}{0.121569,0.466667,0.705882}%
\pgfsetfillcolor{currentfill}%
\pgfsetfillopacity{0.711208}%
\pgfsetlinewidth{1.003750pt}%
\definecolor{currentstroke}{rgb}{0.121569,0.466667,0.705882}%
\pgfsetstrokecolor{currentstroke}%
\pgfsetstrokeopacity{0.711208}%
\pgfsetdash{}{0pt}%
\pgfpathmoveto{\pgfqpoint{2.184590in}{1.686773in}}%
\pgfpathcurveto{\pgfqpoint{2.192827in}{1.686773in}}{\pgfqpoint{2.200727in}{1.690045in}}{\pgfqpoint{2.206551in}{1.695869in}}%
\pgfpathcurveto{\pgfqpoint{2.212375in}{1.701693in}}{\pgfqpoint{2.215647in}{1.709593in}}{\pgfqpoint{2.215647in}{1.717829in}}%
\pgfpathcurveto{\pgfqpoint{2.215647in}{1.726066in}}{\pgfqpoint{2.212375in}{1.733966in}}{\pgfqpoint{2.206551in}{1.739789in}}%
\pgfpathcurveto{\pgfqpoint{2.200727in}{1.745613in}}{\pgfqpoint{2.192827in}{1.748886in}}{\pgfqpoint{2.184590in}{1.748886in}}%
\pgfpathcurveto{\pgfqpoint{2.176354in}{1.748886in}}{\pgfqpoint{2.168454in}{1.745613in}}{\pgfqpoint{2.162630in}{1.739789in}}%
\pgfpathcurveto{\pgfqpoint{2.156806in}{1.733966in}}{\pgfqpoint{2.153534in}{1.726066in}}{\pgfqpoint{2.153534in}{1.717829in}}%
\pgfpathcurveto{\pgfqpoint{2.153534in}{1.709593in}}{\pgfqpoint{2.156806in}{1.701693in}}{\pgfqpoint{2.162630in}{1.695869in}}%
\pgfpathcurveto{\pgfqpoint{2.168454in}{1.690045in}}{\pgfqpoint{2.176354in}{1.686773in}}{\pgfqpoint{2.184590in}{1.686773in}}%
\pgfpathclose%
\pgfusepath{stroke,fill}%
\end{pgfscope}%
\begin{pgfscope}%
\pgfpathrectangle{\pgfqpoint{0.100000in}{0.220728in}}{\pgfqpoint{3.696000in}{3.696000in}}%
\pgfusepath{clip}%
\pgfsetbuttcap%
\pgfsetroundjoin%
\definecolor{currentfill}{rgb}{0.121569,0.466667,0.705882}%
\pgfsetfillcolor{currentfill}%
\pgfsetfillopacity{0.712032}%
\pgfsetlinewidth{1.003750pt}%
\definecolor{currentstroke}{rgb}{0.121569,0.466667,0.705882}%
\pgfsetstrokecolor{currentstroke}%
\pgfsetstrokeopacity{0.712032}%
\pgfsetdash{}{0pt}%
\pgfpathmoveto{\pgfqpoint{2.185240in}{1.684728in}}%
\pgfpathcurveto{\pgfqpoint{2.193476in}{1.684728in}}{\pgfqpoint{2.201376in}{1.688001in}}{\pgfqpoint{2.207200in}{1.693824in}}%
\pgfpathcurveto{\pgfqpoint{2.213024in}{1.699648in}}{\pgfqpoint{2.216296in}{1.707548in}}{\pgfqpoint{2.216296in}{1.715785in}}%
\pgfpathcurveto{\pgfqpoint{2.216296in}{1.724021in}}{\pgfqpoint{2.213024in}{1.731921in}}{\pgfqpoint{2.207200in}{1.737745in}}%
\pgfpathcurveto{\pgfqpoint{2.201376in}{1.743569in}}{\pgfqpoint{2.193476in}{1.746841in}}{\pgfqpoint{2.185240in}{1.746841in}}%
\pgfpathcurveto{\pgfqpoint{2.177004in}{1.746841in}}{\pgfqpoint{2.169104in}{1.743569in}}{\pgfqpoint{2.163280in}{1.737745in}}%
\pgfpathcurveto{\pgfqpoint{2.157456in}{1.731921in}}{\pgfqpoint{2.154183in}{1.724021in}}{\pgfqpoint{2.154183in}{1.715785in}}%
\pgfpathcurveto{\pgfqpoint{2.154183in}{1.707548in}}{\pgfqpoint{2.157456in}{1.699648in}}{\pgfqpoint{2.163280in}{1.693824in}}%
\pgfpathcurveto{\pgfqpoint{2.169104in}{1.688001in}}{\pgfqpoint{2.177004in}{1.684728in}}{\pgfqpoint{2.185240in}{1.684728in}}%
\pgfpathclose%
\pgfusepath{stroke,fill}%
\end{pgfscope}%
\begin{pgfscope}%
\pgfpathrectangle{\pgfqpoint{0.100000in}{0.220728in}}{\pgfqpoint{3.696000in}{3.696000in}}%
\pgfusepath{clip}%
\pgfsetbuttcap%
\pgfsetroundjoin%
\definecolor{currentfill}{rgb}{0.121569,0.466667,0.705882}%
\pgfsetfillcolor{currentfill}%
\pgfsetfillopacity{0.712537}%
\pgfsetlinewidth{1.003750pt}%
\definecolor{currentstroke}{rgb}{0.121569,0.466667,0.705882}%
\pgfsetstrokecolor{currentstroke}%
\pgfsetstrokeopacity{0.712537}%
\pgfsetdash{}{0pt}%
\pgfpathmoveto{\pgfqpoint{2.185626in}{1.683951in}}%
\pgfpathcurveto{\pgfqpoint{2.193862in}{1.683951in}}{\pgfqpoint{2.201762in}{1.687224in}}{\pgfqpoint{2.207586in}{1.693048in}}%
\pgfpathcurveto{\pgfqpoint{2.213410in}{1.698872in}}{\pgfqpoint{2.216683in}{1.706772in}}{\pgfqpoint{2.216683in}{1.715008in}}%
\pgfpathcurveto{\pgfqpoint{2.216683in}{1.723244in}}{\pgfqpoint{2.213410in}{1.731144in}}{\pgfqpoint{2.207586in}{1.736968in}}%
\pgfpathcurveto{\pgfqpoint{2.201762in}{1.742792in}}{\pgfqpoint{2.193862in}{1.746064in}}{\pgfqpoint{2.185626in}{1.746064in}}%
\pgfpathcurveto{\pgfqpoint{2.177390in}{1.746064in}}{\pgfqpoint{2.169490in}{1.742792in}}{\pgfqpoint{2.163666in}{1.736968in}}%
\pgfpathcurveto{\pgfqpoint{2.157842in}{1.731144in}}{\pgfqpoint{2.154570in}{1.723244in}}{\pgfqpoint{2.154570in}{1.715008in}}%
\pgfpathcurveto{\pgfqpoint{2.154570in}{1.706772in}}{\pgfqpoint{2.157842in}{1.698872in}}{\pgfqpoint{2.163666in}{1.693048in}}%
\pgfpathcurveto{\pgfqpoint{2.169490in}{1.687224in}}{\pgfqpoint{2.177390in}{1.683951in}}{\pgfqpoint{2.185626in}{1.683951in}}%
\pgfpathclose%
\pgfusepath{stroke,fill}%
\end{pgfscope}%
\begin{pgfscope}%
\pgfpathrectangle{\pgfqpoint{0.100000in}{0.220728in}}{\pgfqpoint{3.696000in}{3.696000in}}%
\pgfusepath{clip}%
\pgfsetbuttcap%
\pgfsetroundjoin%
\definecolor{currentfill}{rgb}{0.121569,0.466667,0.705882}%
\pgfsetfillcolor{currentfill}%
\pgfsetfillopacity{0.713405}%
\pgfsetlinewidth{1.003750pt}%
\definecolor{currentstroke}{rgb}{0.121569,0.466667,0.705882}%
\pgfsetstrokecolor{currentstroke}%
\pgfsetstrokeopacity{0.713405}%
\pgfsetdash{}{0pt}%
\pgfpathmoveto{\pgfqpoint{2.186279in}{1.684026in}}%
\pgfpathcurveto{\pgfqpoint{2.194515in}{1.684026in}}{\pgfqpoint{2.202416in}{1.687298in}}{\pgfqpoint{2.208239in}{1.693122in}}%
\pgfpathcurveto{\pgfqpoint{2.214063in}{1.698946in}}{\pgfqpoint{2.217336in}{1.706846in}}{\pgfqpoint{2.217336in}{1.715082in}}%
\pgfpathcurveto{\pgfqpoint{2.217336in}{1.723319in}}{\pgfqpoint{2.214063in}{1.731219in}}{\pgfqpoint{2.208239in}{1.737042in}}%
\pgfpathcurveto{\pgfqpoint{2.202416in}{1.742866in}}{\pgfqpoint{2.194515in}{1.746139in}}{\pgfqpoint{2.186279in}{1.746139in}}%
\pgfpathcurveto{\pgfqpoint{2.178043in}{1.746139in}}{\pgfqpoint{2.170143in}{1.742866in}}{\pgfqpoint{2.164319in}{1.737042in}}%
\pgfpathcurveto{\pgfqpoint{2.158495in}{1.731219in}}{\pgfqpoint{2.155223in}{1.723319in}}{\pgfqpoint{2.155223in}{1.715082in}}%
\pgfpathcurveto{\pgfqpoint{2.155223in}{1.706846in}}{\pgfqpoint{2.158495in}{1.698946in}}{\pgfqpoint{2.164319in}{1.693122in}}%
\pgfpathcurveto{\pgfqpoint{2.170143in}{1.687298in}}{\pgfqpoint{2.178043in}{1.684026in}}{\pgfqpoint{2.186279in}{1.684026in}}%
\pgfpathclose%
\pgfusepath{stroke,fill}%
\end{pgfscope}%
\begin{pgfscope}%
\pgfpathrectangle{\pgfqpoint{0.100000in}{0.220728in}}{\pgfqpoint{3.696000in}{3.696000in}}%
\pgfusepath{clip}%
\pgfsetbuttcap%
\pgfsetroundjoin%
\definecolor{currentfill}{rgb}{0.121569,0.466667,0.705882}%
\pgfsetfillcolor{currentfill}%
\pgfsetfillopacity{0.713745}%
\pgfsetlinewidth{1.003750pt}%
\definecolor{currentstroke}{rgb}{0.121569,0.466667,0.705882}%
\pgfsetstrokecolor{currentstroke}%
\pgfsetstrokeopacity{0.713745}%
\pgfsetdash{}{0pt}%
\pgfpathmoveto{\pgfqpoint{2.186446in}{1.683092in}}%
\pgfpathcurveto{\pgfqpoint{2.194683in}{1.683092in}}{\pgfqpoint{2.202583in}{1.686364in}}{\pgfqpoint{2.208407in}{1.692188in}}%
\pgfpathcurveto{\pgfqpoint{2.214231in}{1.698012in}}{\pgfqpoint{2.217503in}{1.705912in}}{\pgfqpoint{2.217503in}{1.714148in}}%
\pgfpathcurveto{\pgfqpoint{2.217503in}{1.722385in}}{\pgfqpoint{2.214231in}{1.730285in}}{\pgfqpoint{2.208407in}{1.736109in}}%
\pgfpathcurveto{\pgfqpoint{2.202583in}{1.741933in}}{\pgfqpoint{2.194683in}{1.745205in}}{\pgfqpoint{2.186446in}{1.745205in}}%
\pgfpathcurveto{\pgfqpoint{2.178210in}{1.745205in}}{\pgfqpoint{2.170310in}{1.741933in}}{\pgfqpoint{2.164486in}{1.736109in}}%
\pgfpathcurveto{\pgfqpoint{2.158662in}{1.730285in}}{\pgfqpoint{2.155390in}{1.722385in}}{\pgfqpoint{2.155390in}{1.714148in}}%
\pgfpathcurveto{\pgfqpoint{2.155390in}{1.705912in}}{\pgfqpoint{2.158662in}{1.698012in}}{\pgfqpoint{2.164486in}{1.692188in}}%
\pgfpathcurveto{\pgfqpoint{2.170310in}{1.686364in}}{\pgfqpoint{2.178210in}{1.683092in}}{\pgfqpoint{2.186446in}{1.683092in}}%
\pgfpathclose%
\pgfusepath{stroke,fill}%
\end{pgfscope}%
\begin{pgfscope}%
\pgfpathrectangle{\pgfqpoint{0.100000in}{0.220728in}}{\pgfqpoint{3.696000in}{3.696000in}}%
\pgfusepath{clip}%
\pgfsetbuttcap%
\pgfsetroundjoin%
\definecolor{currentfill}{rgb}{0.121569,0.466667,0.705882}%
\pgfsetfillcolor{currentfill}%
\pgfsetfillopacity{0.714393}%
\pgfsetlinewidth{1.003750pt}%
\definecolor{currentstroke}{rgb}{0.121569,0.466667,0.705882}%
\pgfsetstrokecolor{currentstroke}%
\pgfsetstrokeopacity{0.714393}%
\pgfsetdash{}{0pt}%
\pgfpathmoveto{\pgfqpoint{2.187341in}{1.681261in}}%
\pgfpathcurveto{\pgfqpoint{2.195578in}{1.681261in}}{\pgfqpoint{2.203478in}{1.684533in}}{\pgfqpoint{2.209302in}{1.690357in}}%
\pgfpathcurveto{\pgfqpoint{2.215126in}{1.696181in}}{\pgfqpoint{2.218398in}{1.704081in}}{\pgfqpoint{2.218398in}{1.712318in}}%
\pgfpathcurveto{\pgfqpoint{2.218398in}{1.720554in}}{\pgfqpoint{2.215126in}{1.728454in}}{\pgfqpoint{2.209302in}{1.734278in}}%
\pgfpathcurveto{\pgfqpoint{2.203478in}{1.740102in}}{\pgfqpoint{2.195578in}{1.743374in}}{\pgfqpoint{2.187341in}{1.743374in}}%
\pgfpathcurveto{\pgfqpoint{2.179105in}{1.743374in}}{\pgfqpoint{2.171205in}{1.740102in}}{\pgfqpoint{2.165381in}{1.734278in}}%
\pgfpathcurveto{\pgfqpoint{2.159557in}{1.728454in}}{\pgfqpoint{2.156285in}{1.720554in}}{\pgfqpoint{2.156285in}{1.712318in}}%
\pgfpathcurveto{\pgfqpoint{2.156285in}{1.704081in}}{\pgfqpoint{2.159557in}{1.696181in}}{\pgfqpoint{2.165381in}{1.690357in}}%
\pgfpathcurveto{\pgfqpoint{2.171205in}{1.684533in}}{\pgfqpoint{2.179105in}{1.681261in}}{\pgfqpoint{2.187341in}{1.681261in}}%
\pgfpathclose%
\pgfusepath{stroke,fill}%
\end{pgfscope}%
\begin{pgfscope}%
\pgfpathrectangle{\pgfqpoint{0.100000in}{0.220728in}}{\pgfqpoint{3.696000in}{3.696000in}}%
\pgfusepath{clip}%
\pgfsetbuttcap%
\pgfsetroundjoin%
\definecolor{currentfill}{rgb}{0.121569,0.466667,0.705882}%
\pgfsetfillcolor{currentfill}%
\pgfsetfillopacity{0.714861}%
\pgfsetlinewidth{1.003750pt}%
\definecolor{currentstroke}{rgb}{0.121569,0.466667,0.705882}%
\pgfsetstrokecolor{currentstroke}%
\pgfsetstrokeopacity{0.714861}%
\pgfsetdash{}{0pt}%
\pgfpathmoveto{\pgfqpoint{2.187629in}{1.680832in}}%
\pgfpathcurveto{\pgfqpoint{2.195865in}{1.680832in}}{\pgfqpoint{2.203765in}{1.684105in}}{\pgfqpoint{2.209589in}{1.689929in}}%
\pgfpathcurveto{\pgfqpoint{2.215413in}{1.695753in}}{\pgfqpoint{2.218685in}{1.703653in}}{\pgfqpoint{2.218685in}{1.711889in}}%
\pgfpathcurveto{\pgfqpoint{2.218685in}{1.720125in}}{\pgfqpoint{2.215413in}{1.728025in}}{\pgfqpoint{2.209589in}{1.733849in}}%
\pgfpathcurveto{\pgfqpoint{2.203765in}{1.739673in}}{\pgfqpoint{2.195865in}{1.742945in}}{\pgfqpoint{2.187629in}{1.742945in}}%
\pgfpathcurveto{\pgfqpoint{2.179392in}{1.742945in}}{\pgfqpoint{2.171492in}{1.739673in}}{\pgfqpoint{2.165668in}{1.733849in}}%
\pgfpathcurveto{\pgfqpoint{2.159844in}{1.728025in}}{\pgfqpoint{2.156572in}{1.720125in}}{\pgfqpoint{2.156572in}{1.711889in}}%
\pgfpathcurveto{\pgfqpoint{2.156572in}{1.703653in}}{\pgfqpoint{2.159844in}{1.695753in}}{\pgfqpoint{2.165668in}{1.689929in}}%
\pgfpathcurveto{\pgfqpoint{2.171492in}{1.684105in}}{\pgfqpoint{2.179392in}{1.680832in}}{\pgfqpoint{2.187629in}{1.680832in}}%
\pgfpathclose%
\pgfusepath{stroke,fill}%
\end{pgfscope}%
\begin{pgfscope}%
\pgfpathrectangle{\pgfqpoint{0.100000in}{0.220728in}}{\pgfqpoint{3.696000in}{3.696000in}}%
\pgfusepath{clip}%
\pgfsetbuttcap%
\pgfsetroundjoin%
\definecolor{currentfill}{rgb}{0.121569,0.466667,0.705882}%
\pgfsetfillcolor{currentfill}%
\pgfsetfillopacity{0.716031}%
\pgfsetlinewidth{1.003750pt}%
\definecolor{currentstroke}{rgb}{0.121569,0.466667,0.705882}%
\pgfsetstrokecolor{currentstroke}%
\pgfsetstrokeopacity{0.716031}%
\pgfsetdash{}{0pt}%
\pgfpathmoveto{\pgfqpoint{2.188565in}{1.681114in}}%
\pgfpathcurveto{\pgfqpoint{2.196801in}{1.681114in}}{\pgfqpoint{2.204701in}{1.684387in}}{\pgfqpoint{2.210525in}{1.690211in}}%
\pgfpathcurveto{\pgfqpoint{2.216349in}{1.696034in}}{\pgfqpoint{2.219622in}{1.703935in}}{\pgfqpoint{2.219622in}{1.712171in}}%
\pgfpathcurveto{\pgfqpoint{2.219622in}{1.720407in}}{\pgfqpoint{2.216349in}{1.728307in}}{\pgfqpoint{2.210525in}{1.734131in}}%
\pgfpathcurveto{\pgfqpoint{2.204701in}{1.739955in}}{\pgfqpoint{2.196801in}{1.743227in}}{\pgfqpoint{2.188565in}{1.743227in}}%
\pgfpathcurveto{\pgfqpoint{2.180329in}{1.743227in}}{\pgfqpoint{2.172429in}{1.739955in}}{\pgfqpoint{2.166605in}{1.734131in}}%
\pgfpathcurveto{\pgfqpoint{2.160781in}{1.728307in}}{\pgfqpoint{2.157509in}{1.720407in}}{\pgfqpoint{2.157509in}{1.712171in}}%
\pgfpathcurveto{\pgfqpoint{2.157509in}{1.703935in}}{\pgfqpoint{2.160781in}{1.696034in}}{\pgfqpoint{2.166605in}{1.690211in}}%
\pgfpathcurveto{\pgfqpoint{2.172429in}{1.684387in}}{\pgfqpoint{2.180329in}{1.681114in}}{\pgfqpoint{2.188565in}{1.681114in}}%
\pgfpathclose%
\pgfusepath{stroke,fill}%
\end{pgfscope}%
\begin{pgfscope}%
\pgfpathrectangle{\pgfqpoint{0.100000in}{0.220728in}}{\pgfqpoint{3.696000in}{3.696000in}}%
\pgfusepath{clip}%
\pgfsetbuttcap%
\pgfsetroundjoin%
\definecolor{currentfill}{rgb}{0.121569,0.466667,0.705882}%
\pgfsetfillcolor{currentfill}%
\pgfsetfillopacity{0.717080}%
\pgfsetlinewidth{1.003750pt}%
\definecolor{currentstroke}{rgb}{0.121569,0.466667,0.705882}%
\pgfsetstrokecolor{currentstroke}%
\pgfsetstrokeopacity{0.717080}%
\pgfsetdash{}{0pt}%
\pgfpathmoveto{\pgfqpoint{2.189278in}{1.679155in}}%
\pgfpathcurveto{\pgfqpoint{2.197514in}{1.679155in}}{\pgfqpoint{2.205414in}{1.682427in}}{\pgfqpoint{2.211238in}{1.688251in}}%
\pgfpathcurveto{\pgfqpoint{2.217062in}{1.694075in}}{\pgfqpoint{2.220334in}{1.701975in}}{\pgfqpoint{2.220334in}{1.710212in}}%
\pgfpathcurveto{\pgfqpoint{2.220334in}{1.718448in}}{\pgfqpoint{2.217062in}{1.726348in}}{\pgfqpoint{2.211238in}{1.732172in}}%
\pgfpathcurveto{\pgfqpoint{2.205414in}{1.737996in}}{\pgfqpoint{2.197514in}{1.741268in}}{\pgfqpoint{2.189278in}{1.741268in}}%
\pgfpathcurveto{\pgfqpoint{2.181041in}{1.741268in}}{\pgfqpoint{2.173141in}{1.737996in}}{\pgfqpoint{2.167317in}{1.732172in}}%
\pgfpathcurveto{\pgfqpoint{2.161493in}{1.726348in}}{\pgfqpoint{2.158221in}{1.718448in}}{\pgfqpoint{2.158221in}{1.710212in}}%
\pgfpathcurveto{\pgfqpoint{2.158221in}{1.701975in}}{\pgfqpoint{2.161493in}{1.694075in}}{\pgfqpoint{2.167317in}{1.688251in}}%
\pgfpathcurveto{\pgfqpoint{2.173141in}{1.682427in}}{\pgfqpoint{2.181041in}{1.679155in}}{\pgfqpoint{2.189278in}{1.679155in}}%
\pgfpathclose%
\pgfusepath{stroke,fill}%
\end{pgfscope}%
\begin{pgfscope}%
\pgfpathrectangle{\pgfqpoint{0.100000in}{0.220728in}}{\pgfqpoint{3.696000in}{3.696000in}}%
\pgfusepath{clip}%
\pgfsetbuttcap%
\pgfsetroundjoin%
\definecolor{currentfill}{rgb}{0.121569,0.466667,0.705882}%
\pgfsetfillcolor{currentfill}%
\pgfsetfillopacity{0.718945}%
\pgfsetlinewidth{1.003750pt}%
\definecolor{currentstroke}{rgb}{0.121569,0.466667,0.705882}%
\pgfsetstrokecolor{currentstroke}%
\pgfsetstrokeopacity{0.718945}%
\pgfsetdash{}{0pt}%
\pgfpathmoveto{\pgfqpoint{2.190480in}{1.675897in}}%
\pgfpathcurveto{\pgfqpoint{2.198716in}{1.675897in}}{\pgfqpoint{2.206616in}{1.679169in}}{\pgfqpoint{2.212440in}{1.684993in}}%
\pgfpathcurveto{\pgfqpoint{2.218264in}{1.690817in}}{\pgfqpoint{2.221537in}{1.698717in}}{\pgfqpoint{2.221537in}{1.706953in}}%
\pgfpathcurveto{\pgfqpoint{2.221537in}{1.715189in}}{\pgfqpoint{2.218264in}{1.723089in}}{\pgfqpoint{2.212440in}{1.728913in}}%
\pgfpathcurveto{\pgfqpoint{2.206616in}{1.734737in}}{\pgfqpoint{2.198716in}{1.738010in}}{\pgfqpoint{2.190480in}{1.738010in}}%
\pgfpathcurveto{\pgfqpoint{2.182244in}{1.738010in}}{\pgfqpoint{2.174344in}{1.734737in}}{\pgfqpoint{2.168520in}{1.728913in}}%
\pgfpathcurveto{\pgfqpoint{2.162696in}{1.723089in}}{\pgfqpoint{2.159424in}{1.715189in}}{\pgfqpoint{2.159424in}{1.706953in}}%
\pgfpathcurveto{\pgfqpoint{2.159424in}{1.698717in}}{\pgfqpoint{2.162696in}{1.690817in}}{\pgfqpoint{2.168520in}{1.684993in}}%
\pgfpathcurveto{\pgfqpoint{2.174344in}{1.679169in}}{\pgfqpoint{2.182244in}{1.675897in}}{\pgfqpoint{2.190480in}{1.675897in}}%
\pgfpathclose%
\pgfusepath{stroke,fill}%
\end{pgfscope}%
\begin{pgfscope}%
\pgfpathrectangle{\pgfqpoint{0.100000in}{0.220728in}}{\pgfqpoint{3.696000in}{3.696000in}}%
\pgfusepath{clip}%
\pgfsetbuttcap%
\pgfsetroundjoin%
\definecolor{currentfill}{rgb}{0.121569,0.466667,0.705882}%
\pgfsetfillcolor{currentfill}%
\pgfsetfillopacity{0.720278}%
\pgfsetlinewidth{1.003750pt}%
\definecolor{currentstroke}{rgb}{0.121569,0.466667,0.705882}%
\pgfsetstrokecolor{currentstroke}%
\pgfsetstrokeopacity{0.720278}%
\pgfsetdash{}{0pt}%
\pgfpathmoveto{\pgfqpoint{2.191485in}{1.676248in}}%
\pgfpathcurveto{\pgfqpoint{2.199721in}{1.676248in}}{\pgfqpoint{2.207621in}{1.679520in}}{\pgfqpoint{2.213445in}{1.685344in}}%
\pgfpathcurveto{\pgfqpoint{2.219269in}{1.691168in}}{\pgfqpoint{2.222541in}{1.699068in}}{\pgfqpoint{2.222541in}{1.707304in}}%
\pgfpathcurveto{\pgfqpoint{2.222541in}{1.715541in}}{\pgfqpoint{2.219269in}{1.723441in}}{\pgfqpoint{2.213445in}{1.729265in}}%
\pgfpathcurveto{\pgfqpoint{2.207621in}{1.735089in}}{\pgfqpoint{2.199721in}{1.738361in}}{\pgfqpoint{2.191485in}{1.738361in}}%
\pgfpathcurveto{\pgfqpoint{2.183249in}{1.738361in}}{\pgfqpoint{2.175349in}{1.735089in}}{\pgfqpoint{2.169525in}{1.729265in}}%
\pgfpathcurveto{\pgfqpoint{2.163701in}{1.723441in}}{\pgfqpoint{2.160428in}{1.715541in}}{\pgfqpoint{2.160428in}{1.707304in}}%
\pgfpathcurveto{\pgfqpoint{2.160428in}{1.699068in}}{\pgfqpoint{2.163701in}{1.691168in}}{\pgfqpoint{2.169525in}{1.685344in}}%
\pgfpathcurveto{\pgfqpoint{2.175349in}{1.679520in}}{\pgfqpoint{2.183249in}{1.676248in}}{\pgfqpoint{2.191485in}{1.676248in}}%
\pgfpathclose%
\pgfusepath{stroke,fill}%
\end{pgfscope}%
\begin{pgfscope}%
\pgfpathrectangle{\pgfqpoint{0.100000in}{0.220728in}}{\pgfqpoint{3.696000in}{3.696000in}}%
\pgfusepath{clip}%
\pgfsetbuttcap%
\pgfsetroundjoin%
\definecolor{currentfill}{rgb}{0.121569,0.466667,0.705882}%
\pgfsetfillcolor{currentfill}%
\pgfsetfillopacity{0.722271}%
\pgfsetlinewidth{1.003750pt}%
\definecolor{currentstroke}{rgb}{0.121569,0.466667,0.705882}%
\pgfsetstrokecolor{currentstroke}%
\pgfsetstrokeopacity{0.722271}%
\pgfsetdash{}{0pt}%
\pgfpathmoveto{\pgfqpoint{2.192849in}{1.674532in}}%
\pgfpathcurveto{\pgfqpoint{2.201085in}{1.674532in}}{\pgfqpoint{2.208985in}{1.677804in}}{\pgfqpoint{2.214809in}{1.683628in}}%
\pgfpathcurveto{\pgfqpoint{2.220633in}{1.689452in}}{\pgfqpoint{2.223905in}{1.697352in}}{\pgfqpoint{2.223905in}{1.705588in}}%
\pgfpathcurveto{\pgfqpoint{2.223905in}{1.713824in}}{\pgfqpoint{2.220633in}{1.721724in}}{\pgfqpoint{2.214809in}{1.727548in}}%
\pgfpathcurveto{\pgfqpoint{2.208985in}{1.733372in}}{\pgfqpoint{2.201085in}{1.736645in}}{\pgfqpoint{2.192849in}{1.736645in}}%
\pgfpathcurveto{\pgfqpoint{2.184612in}{1.736645in}}{\pgfqpoint{2.176712in}{1.733372in}}{\pgfqpoint{2.170889in}{1.727548in}}%
\pgfpathcurveto{\pgfqpoint{2.165065in}{1.721724in}}{\pgfqpoint{2.161792in}{1.713824in}}{\pgfqpoint{2.161792in}{1.705588in}}%
\pgfpathcurveto{\pgfqpoint{2.161792in}{1.697352in}}{\pgfqpoint{2.165065in}{1.689452in}}{\pgfqpoint{2.170889in}{1.683628in}}%
\pgfpathcurveto{\pgfqpoint{2.176712in}{1.677804in}}{\pgfqpoint{2.184612in}{1.674532in}}{\pgfqpoint{2.192849in}{1.674532in}}%
\pgfpathclose%
\pgfusepath{stroke,fill}%
\end{pgfscope}%
\begin{pgfscope}%
\pgfpathrectangle{\pgfqpoint{0.100000in}{0.220728in}}{\pgfqpoint{3.696000in}{3.696000in}}%
\pgfusepath{clip}%
\pgfsetbuttcap%
\pgfsetroundjoin%
\definecolor{currentfill}{rgb}{0.121569,0.466667,0.705882}%
\pgfsetfillcolor{currentfill}%
\pgfsetfillopacity{0.724234}%
\pgfsetlinewidth{1.003750pt}%
\definecolor{currentstroke}{rgb}{0.121569,0.466667,0.705882}%
\pgfsetstrokecolor{currentstroke}%
\pgfsetstrokeopacity{0.724234}%
\pgfsetdash{}{0pt}%
\pgfpathmoveto{\pgfqpoint{2.194345in}{1.669628in}}%
\pgfpathcurveto{\pgfqpoint{2.202582in}{1.669628in}}{\pgfqpoint{2.210482in}{1.672900in}}{\pgfqpoint{2.216306in}{1.678724in}}%
\pgfpathcurveto{\pgfqpoint{2.222129in}{1.684548in}}{\pgfqpoint{2.225402in}{1.692448in}}{\pgfqpoint{2.225402in}{1.700684in}}%
\pgfpathcurveto{\pgfqpoint{2.225402in}{1.708921in}}{\pgfqpoint{2.222129in}{1.716821in}}{\pgfqpoint{2.216306in}{1.722645in}}%
\pgfpathcurveto{\pgfqpoint{2.210482in}{1.728469in}}{\pgfqpoint{2.202582in}{1.731741in}}{\pgfqpoint{2.194345in}{1.731741in}}%
\pgfpathcurveto{\pgfqpoint{2.186109in}{1.731741in}}{\pgfqpoint{2.178209in}{1.728469in}}{\pgfqpoint{2.172385in}{1.722645in}}%
\pgfpathcurveto{\pgfqpoint{2.166561in}{1.716821in}}{\pgfqpoint{2.163289in}{1.708921in}}{\pgfqpoint{2.163289in}{1.700684in}}%
\pgfpathcurveto{\pgfqpoint{2.163289in}{1.692448in}}{\pgfqpoint{2.166561in}{1.684548in}}{\pgfqpoint{2.172385in}{1.678724in}}%
\pgfpathcurveto{\pgfqpoint{2.178209in}{1.672900in}}{\pgfqpoint{2.186109in}{1.669628in}}{\pgfqpoint{2.194345in}{1.669628in}}%
\pgfpathclose%
\pgfusepath{stroke,fill}%
\end{pgfscope}%
\begin{pgfscope}%
\pgfpathrectangle{\pgfqpoint{0.100000in}{0.220728in}}{\pgfqpoint{3.696000in}{3.696000in}}%
\pgfusepath{clip}%
\pgfsetbuttcap%
\pgfsetroundjoin%
\definecolor{currentfill}{rgb}{0.121569,0.466667,0.705882}%
\pgfsetfillcolor{currentfill}%
\pgfsetfillopacity{0.726763}%
\pgfsetlinewidth{1.003750pt}%
\definecolor{currentstroke}{rgb}{0.121569,0.466667,0.705882}%
\pgfsetstrokecolor{currentstroke}%
\pgfsetstrokeopacity{0.726763}%
\pgfsetdash{}{0pt}%
\pgfpathmoveto{\pgfqpoint{2.196074in}{1.664433in}}%
\pgfpathcurveto{\pgfqpoint{2.204311in}{1.664433in}}{\pgfqpoint{2.212211in}{1.667705in}}{\pgfqpoint{2.218035in}{1.673529in}}%
\pgfpathcurveto{\pgfqpoint{2.223859in}{1.679353in}}{\pgfqpoint{2.227131in}{1.687253in}}{\pgfqpoint{2.227131in}{1.695489in}}%
\pgfpathcurveto{\pgfqpoint{2.227131in}{1.703726in}}{\pgfqpoint{2.223859in}{1.711626in}}{\pgfqpoint{2.218035in}{1.717449in}}%
\pgfpathcurveto{\pgfqpoint{2.212211in}{1.723273in}}{\pgfqpoint{2.204311in}{1.726546in}}{\pgfqpoint{2.196074in}{1.726546in}}%
\pgfpathcurveto{\pgfqpoint{2.187838in}{1.726546in}}{\pgfqpoint{2.179938in}{1.723273in}}{\pgfqpoint{2.174114in}{1.717449in}}%
\pgfpathcurveto{\pgfqpoint{2.168290in}{1.711626in}}{\pgfqpoint{2.165018in}{1.703726in}}{\pgfqpoint{2.165018in}{1.695489in}}%
\pgfpathcurveto{\pgfqpoint{2.165018in}{1.687253in}}{\pgfqpoint{2.168290in}{1.679353in}}{\pgfqpoint{2.174114in}{1.673529in}}%
\pgfpathcurveto{\pgfqpoint{2.179938in}{1.667705in}}{\pgfqpoint{2.187838in}{1.664433in}}{\pgfqpoint{2.196074in}{1.664433in}}%
\pgfpathclose%
\pgfusepath{stroke,fill}%
\end{pgfscope}%
\begin{pgfscope}%
\pgfpathrectangle{\pgfqpoint{0.100000in}{0.220728in}}{\pgfqpoint{3.696000in}{3.696000in}}%
\pgfusepath{clip}%
\pgfsetbuttcap%
\pgfsetroundjoin%
\definecolor{currentfill}{rgb}{0.121569,0.466667,0.705882}%
\pgfsetfillcolor{currentfill}%
\pgfsetfillopacity{0.730509}%
\pgfsetlinewidth{1.003750pt}%
\definecolor{currentstroke}{rgb}{0.121569,0.466667,0.705882}%
\pgfsetstrokecolor{currentstroke}%
\pgfsetstrokeopacity{0.730509}%
\pgfsetdash{}{0pt}%
\pgfpathmoveto{\pgfqpoint{2.199476in}{1.666885in}}%
\pgfpathcurveto{\pgfqpoint{2.207713in}{1.666885in}}{\pgfqpoint{2.215613in}{1.670158in}}{\pgfqpoint{2.221437in}{1.675982in}}%
\pgfpathcurveto{\pgfqpoint{2.227261in}{1.681805in}}{\pgfqpoint{2.230533in}{1.689706in}}{\pgfqpoint{2.230533in}{1.697942in}}%
\pgfpathcurveto{\pgfqpoint{2.230533in}{1.706178in}}{\pgfqpoint{2.227261in}{1.714078in}}{\pgfqpoint{2.221437in}{1.719902in}}%
\pgfpathcurveto{\pgfqpoint{2.215613in}{1.725726in}}{\pgfqpoint{2.207713in}{1.728998in}}{\pgfqpoint{2.199476in}{1.728998in}}%
\pgfpathcurveto{\pgfqpoint{2.191240in}{1.728998in}}{\pgfqpoint{2.183340in}{1.725726in}}{\pgfqpoint{2.177516in}{1.719902in}}%
\pgfpathcurveto{\pgfqpoint{2.171692in}{1.714078in}}{\pgfqpoint{2.168420in}{1.706178in}}{\pgfqpoint{2.168420in}{1.697942in}}%
\pgfpathcurveto{\pgfqpoint{2.168420in}{1.689706in}}{\pgfqpoint{2.171692in}{1.681805in}}{\pgfqpoint{2.177516in}{1.675982in}}%
\pgfpathcurveto{\pgfqpoint{2.183340in}{1.670158in}}{\pgfqpoint{2.191240in}{1.666885in}}{\pgfqpoint{2.199476in}{1.666885in}}%
\pgfpathclose%
\pgfusepath{stroke,fill}%
\end{pgfscope}%
\begin{pgfscope}%
\pgfpathrectangle{\pgfqpoint{0.100000in}{0.220728in}}{\pgfqpoint{3.696000in}{3.696000in}}%
\pgfusepath{clip}%
\pgfsetbuttcap%
\pgfsetroundjoin%
\definecolor{currentfill}{rgb}{0.121569,0.466667,0.705882}%
\pgfsetfillcolor{currentfill}%
\pgfsetfillopacity{0.732109}%
\pgfsetlinewidth{1.003750pt}%
\definecolor{currentstroke}{rgb}{0.121569,0.466667,0.705882}%
\pgfsetstrokecolor{currentstroke}%
\pgfsetstrokeopacity{0.732109}%
\pgfsetdash{}{0pt}%
\pgfpathmoveto{\pgfqpoint{2.199916in}{1.664512in}}%
\pgfpathcurveto{\pgfqpoint{2.208153in}{1.664512in}}{\pgfqpoint{2.216053in}{1.667784in}}{\pgfqpoint{2.221877in}{1.673608in}}%
\pgfpathcurveto{\pgfqpoint{2.227701in}{1.679432in}}{\pgfqpoint{2.230973in}{1.687332in}}{\pgfqpoint{2.230973in}{1.695569in}}%
\pgfpathcurveto{\pgfqpoint{2.230973in}{1.703805in}}{\pgfqpoint{2.227701in}{1.711705in}}{\pgfqpoint{2.221877in}{1.717529in}}%
\pgfpathcurveto{\pgfqpoint{2.216053in}{1.723353in}}{\pgfqpoint{2.208153in}{1.726625in}}{\pgfqpoint{2.199916in}{1.726625in}}%
\pgfpathcurveto{\pgfqpoint{2.191680in}{1.726625in}}{\pgfqpoint{2.183780in}{1.723353in}}{\pgfqpoint{2.177956in}{1.717529in}}%
\pgfpathcurveto{\pgfqpoint{2.172132in}{1.711705in}}{\pgfqpoint{2.168860in}{1.703805in}}{\pgfqpoint{2.168860in}{1.695569in}}%
\pgfpathcurveto{\pgfqpoint{2.168860in}{1.687332in}}{\pgfqpoint{2.172132in}{1.679432in}}{\pgfqpoint{2.177956in}{1.673608in}}%
\pgfpathcurveto{\pgfqpoint{2.183780in}{1.667784in}}{\pgfqpoint{2.191680in}{1.664512in}}{\pgfqpoint{2.199916in}{1.664512in}}%
\pgfpathclose%
\pgfusepath{stroke,fill}%
\end{pgfscope}%
\begin{pgfscope}%
\pgfpathrectangle{\pgfqpoint{0.100000in}{0.220728in}}{\pgfqpoint{3.696000in}{3.696000in}}%
\pgfusepath{clip}%
\pgfsetbuttcap%
\pgfsetroundjoin%
\definecolor{currentfill}{rgb}{0.121569,0.466667,0.705882}%
\pgfsetfillcolor{currentfill}%
\pgfsetfillopacity{0.733694}%
\pgfsetlinewidth{1.003750pt}%
\definecolor{currentstroke}{rgb}{0.121569,0.466667,0.705882}%
\pgfsetstrokecolor{currentstroke}%
\pgfsetstrokeopacity{0.733694}%
\pgfsetdash{}{0pt}%
\pgfpathmoveto{\pgfqpoint{2.201375in}{1.660082in}}%
\pgfpathcurveto{\pgfqpoint{2.209611in}{1.660082in}}{\pgfqpoint{2.217511in}{1.663354in}}{\pgfqpoint{2.223335in}{1.669178in}}%
\pgfpathcurveto{\pgfqpoint{2.229159in}{1.675002in}}{\pgfqpoint{2.232431in}{1.682902in}}{\pgfqpoint{2.232431in}{1.691138in}}%
\pgfpathcurveto{\pgfqpoint{2.232431in}{1.699374in}}{\pgfqpoint{2.229159in}{1.707274in}}{\pgfqpoint{2.223335in}{1.713098in}}%
\pgfpathcurveto{\pgfqpoint{2.217511in}{1.718922in}}{\pgfqpoint{2.209611in}{1.722195in}}{\pgfqpoint{2.201375in}{1.722195in}}%
\pgfpathcurveto{\pgfqpoint{2.193138in}{1.722195in}}{\pgfqpoint{2.185238in}{1.718922in}}{\pgfqpoint{2.179414in}{1.713098in}}%
\pgfpathcurveto{\pgfqpoint{2.173590in}{1.707274in}}{\pgfqpoint{2.170318in}{1.699374in}}{\pgfqpoint{2.170318in}{1.691138in}}%
\pgfpathcurveto{\pgfqpoint{2.170318in}{1.682902in}}{\pgfqpoint{2.173590in}{1.675002in}}{\pgfqpoint{2.179414in}{1.669178in}}%
\pgfpathcurveto{\pgfqpoint{2.185238in}{1.663354in}}{\pgfqpoint{2.193138in}{1.660082in}}{\pgfqpoint{2.201375in}{1.660082in}}%
\pgfpathclose%
\pgfusepath{stroke,fill}%
\end{pgfscope}%
\begin{pgfscope}%
\pgfpathrectangle{\pgfqpoint{0.100000in}{0.220728in}}{\pgfqpoint{3.696000in}{3.696000in}}%
\pgfusepath{clip}%
\pgfsetbuttcap%
\pgfsetroundjoin%
\definecolor{currentfill}{rgb}{0.121569,0.466667,0.705882}%
\pgfsetfillcolor{currentfill}%
\pgfsetfillopacity{0.734779}%
\pgfsetlinewidth{1.003750pt}%
\definecolor{currentstroke}{rgb}{0.121569,0.466667,0.705882}%
\pgfsetstrokecolor{currentstroke}%
\pgfsetstrokeopacity{0.734779}%
\pgfsetdash{}{0pt}%
\pgfpathmoveto{\pgfqpoint{2.202028in}{1.658908in}}%
\pgfpathcurveto{\pgfqpoint{2.210264in}{1.658908in}}{\pgfqpoint{2.218164in}{1.662180in}}{\pgfqpoint{2.223988in}{1.668004in}}%
\pgfpathcurveto{\pgfqpoint{2.229812in}{1.673828in}}{\pgfqpoint{2.233084in}{1.681728in}}{\pgfqpoint{2.233084in}{1.689964in}}%
\pgfpathcurveto{\pgfqpoint{2.233084in}{1.698200in}}{\pgfqpoint{2.229812in}{1.706100in}}{\pgfqpoint{2.223988in}{1.711924in}}%
\pgfpathcurveto{\pgfqpoint{2.218164in}{1.717748in}}{\pgfqpoint{2.210264in}{1.721021in}}{\pgfqpoint{2.202028in}{1.721021in}}%
\pgfpathcurveto{\pgfqpoint{2.193791in}{1.721021in}}{\pgfqpoint{2.185891in}{1.717748in}}{\pgfqpoint{2.180067in}{1.711924in}}%
\pgfpathcurveto{\pgfqpoint{2.174244in}{1.706100in}}{\pgfqpoint{2.170971in}{1.698200in}}{\pgfqpoint{2.170971in}{1.689964in}}%
\pgfpathcurveto{\pgfqpoint{2.170971in}{1.681728in}}{\pgfqpoint{2.174244in}{1.673828in}}{\pgfqpoint{2.180067in}{1.668004in}}%
\pgfpathcurveto{\pgfqpoint{2.185891in}{1.662180in}}{\pgfqpoint{2.193791in}{1.658908in}}{\pgfqpoint{2.202028in}{1.658908in}}%
\pgfpathclose%
\pgfusepath{stroke,fill}%
\end{pgfscope}%
\begin{pgfscope}%
\pgfpathrectangle{\pgfqpoint{0.100000in}{0.220728in}}{\pgfqpoint{3.696000in}{3.696000in}}%
\pgfusepath{clip}%
\pgfsetbuttcap%
\pgfsetroundjoin%
\definecolor{currentfill}{rgb}{0.121569,0.466667,0.705882}%
\pgfsetfillcolor{currentfill}%
\pgfsetfillopacity{0.736498}%
\pgfsetlinewidth{1.003750pt}%
\definecolor{currentstroke}{rgb}{0.121569,0.466667,0.705882}%
\pgfsetstrokecolor{currentstroke}%
\pgfsetstrokeopacity{0.736498}%
\pgfsetdash{}{0pt}%
\pgfpathmoveto{\pgfqpoint{2.203514in}{1.659838in}}%
\pgfpathcurveto{\pgfqpoint{2.211750in}{1.659838in}}{\pgfqpoint{2.219650in}{1.663110in}}{\pgfqpoint{2.225474in}{1.668934in}}%
\pgfpathcurveto{\pgfqpoint{2.231298in}{1.674758in}}{\pgfqpoint{2.234571in}{1.682658in}}{\pgfqpoint{2.234571in}{1.690894in}}%
\pgfpathcurveto{\pgfqpoint{2.234571in}{1.699131in}}{\pgfqpoint{2.231298in}{1.707031in}}{\pgfqpoint{2.225474in}{1.712855in}}%
\pgfpathcurveto{\pgfqpoint{2.219650in}{1.718678in}}{\pgfqpoint{2.211750in}{1.721951in}}{\pgfqpoint{2.203514in}{1.721951in}}%
\pgfpathcurveto{\pgfqpoint{2.195278in}{1.721951in}}{\pgfqpoint{2.187378in}{1.718678in}}{\pgfqpoint{2.181554in}{1.712855in}}%
\pgfpathcurveto{\pgfqpoint{2.175730in}{1.707031in}}{\pgfqpoint{2.172458in}{1.699131in}}{\pgfqpoint{2.172458in}{1.690894in}}%
\pgfpathcurveto{\pgfqpoint{2.172458in}{1.682658in}}{\pgfqpoint{2.175730in}{1.674758in}}{\pgfqpoint{2.181554in}{1.668934in}}%
\pgfpathcurveto{\pgfqpoint{2.187378in}{1.663110in}}{\pgfqpoint{2.195278in}{1.659838in}}{\pgfqpoint{2.203514in}{1.659838in}}%
\pgfpathclose%
\pgfusepath{stroke,fill}%
\end{pgfscope}%
\begin{pgfscope}%
\pgfpathrectangle{\pgfqpoint{0.100000in}{0.220728in}}{\pgfqpoint{3.696000in}{3.696000in}}%
\pgfusepath{clip}%
\pgfsetbuttcap%
\pgfsetroundjoin%
\definecolor{currentfill}{rgb}{0.121569,0.466667,0.705882}%
\pgfsetfillcolor{currentfill}%
\pgfsetfillopacity{0.737189}%
\pgfsetlinewidth{1.003750pt}%
\definecolor{currentstroke}{rgb}{0.121569,0.466667,0.705882}%
\pgfsetstrokecolor{currentstroke}%
\pgfsetstrokeopacity{0.737189}%
\pgfsetdash{}{0pt}%
\pgfpathmoveto{\pgfqpoint{2.203769in}{1.658423in}}%
\pgfpathcurveto{\pgfqpoint{2.212005in}{1.658423in}}{\pgfqpoint{2.219905in}{1.661695in}}{\pgfqpoint{2.225729in}{1.667519in}}%
\pgfpathcurveto{\pgfqpoint{2.231553in}{1.673343in}}{\pgfqpoint{2.234825in}{1.681243in}}{\pgfqpoint{2.234825in}{1.689480in}}%
\pgfpathcurveto{\pgfqpoint{2.234825in}{1.697716in}}{\pgfqpoint{2.231553in}{1.705616in}}{\pgfqpoint{2.225729in}{1.711440in}}%
\pgfpathcurveto{\pgfqpoint{2.219905in}{1.717264in}}{\pgfqpoint{2.212005in}{1.720536in}}{\pgfqpoint{2.203769in}{1.720536in}}%
\pgfpathcurveto{\pgfqpoint{2.195532in}{1.720536in}}{\pgfqpoint{2.187632in}{1.717264in}}{\pgfqpoint{2.181808in}{1.711440in}}%
\pgfpathcurveto{\pgfqpoint{2.175984in}{1.705616in}}{\pgfqpoint{2.172712in}{1.697716in}}{\pgfqpoint{2.172712in}{1.689480in}}%
\pgfpathcurveto{\pgfqpoint{2.172712in}{1.681243in}}{\pgfqpoint{2.175984in}{1.673343in}}{\pgfqpoint{2.181808in}{1.667519in}}%
\pgfpathcurveto{\pgfqpoint{2.187632in}{1.661695in}}{\pgfqpoint{2.195532in}{1.658423in}}{\pgfqpoint{2.203769in}{1.658423in}}%
\pgfpathclose%
\pgfusepath{stroke,fill}%
\end{pgfscope}%
\begin{pgfscope}%
\pgfpathrectangle{\pgfqpoint{0.100000in}{0.220728in}}{\pgfqpoint{3.696000in}{3.696000in}}%
\pgfusepath{clip}%
\pgfsetbuttcap%
\pgfsetroundjoin%
\definecolor{currentfill}{rgb}{0.121569,0.466667,0.705882}%
\pgfsetfillcolor{currentfill}%
\pgfsetfillopacity{0.738179}%
\pgfsetlinewidth{1.003750pt}%
\definecolor{currentstroke}{rgb}{0.121569,0.466667,0.705882}%
\pgfsetstrokecolor{currentstroke}%
\pgfsetstrokeopacity{0.738179}%
\pgfsetdash{}{0pt}%
\pgfpathmoveto{\pgfqpoint{2.205052in}{1.655837in}}%
\pgfpathcurveto{\pgfqpoint{2.213288in}{1.655837in}}{\pgfqpoint{2.221188in}{1.659110in}}{\pgfqpoint{2.227012in}{1.664934in}}%
\pgfpathcurveto{\pgfqpoint{2.232836in}{1.670758in}}{\pgfqpoint{2.236108in}{1.678658in}}{\pgfqpoint{2.236108in}{1.686894in}}%
\pgfpathcurveto{\pgfqpoint{2.236108in}{1.695130in}}{\pgfqpoint{2.232836in}{1.703030in}}{\pgfqpoint{2.227012in}{1.708854in}}%
\pgfpathcurveto{\pgfqpoint{2.221188in}{1.714678in}}{\pgfqpoint{2.213288in}{1.717950in}}{\pgfqpoint{2.205052in}{1.717950in}}%
\pgfpathcurveto{\pgfqpoint{2.196815in}{1.717950in}}{\pgfqpoint{2.188915in}{1.714678in}}{\pgfqpoint{2.183091in}{1.708854in}}%
\pgfpathcurveto{\pgfqpoint{2.177267in}{1.703030in}}{\pgfqpoint{2.173995in}{1.695130in}}{\pgfqpoint{2.173995in}{1.686894in}}%
\pgfpathcurveto{\pgfqpoint{2.173995in}{1.678658in}}{\pgfqpoint{2.177267in}{1.670758in}}{\pgfqpoint{2.183091in}{1.664934in}}%
\pgfpathcurveto{\pgfqpoint{2.188915in}{1.659110in}}{\pgfqpoint{2.196815in}{1.655837in}}{\pgfqpoint{2.205052in}{1.655837in}}%
\pgfpathclose%
\pgfusepath{stroke,fill}%
\end{pgfscope}%
\begin{pgfscope}%
\pgfpathrectangle{\pgfqpoint{0.100000in}{0.220728in}}{\pgfqpoint{3.696000in}{3.696000in}}%
\pgfusepath{clip}%
\pgfsetbuttcap%
\pgfsetroundjoin%
\definecolor{currentfill}{rgb}{0.121569,0.466667,0.705882}%
\pgfsetfillcolor{currentfill}%
\pgfsetfillopacity{0.739291}%
\pgfsetlinewidth{1.003750pt}%
\definecolor{currentstroke}{rgb}{0.121569,0.466667,0.705882}%
\pgfsetstrokecolor{currentstroke}%
\pgfsetstrokeopacity{0.739291}%
\pgfsetdash{}{0pt}%
\pgfpathmoveto{\pgfqpoint{2.206081in}{1.652270in}}%
\pgfpathcurveto{\pgfqpoint{2.214317in}{1.652270in}}{\pgfqpoint{2.222217in}{1.655542in}}{\pgfqpoint{2.228041in}{1.661366in}}%
\pgfpathcurveto{\pgfqpoint{2.233865in}{1.667190in}}{\pgfqpoint{2.237138in}{1.675090in}}{\pgfqpoint{2.237138in}{1.683327in}}%
\pgfpathcurveto{\pgfqpoint{2.237138in}{1.691563in}}{\pgfqpoint{2.233865in}{1.699463in}}{\pgfqpoint{2.228041in}{1.705287in}}%
\pgfpathcurveto{\pgfqpoint{2.222217in}{1.711111in}}{\pgfqpoint{2.214317in}{1.714383in}}{\pgfqpoint{2.206081in}{1.714383in}}%
\pgfpathcurveto{\pgfqpoint{2.197845in}{1.714383in}}{\pgfqpoint{2.189945in}{1.711111in}}{\pgfqpoint{2.184121in}{1.705287in}}%
\pgfpathcurveto{\pgfqpoint{2.178297in}{1.699463in}}{\pgfqpoint{2.175025in}{1.691563in}}{\pgfqpoint{2.175025in}{1.683327in}}%
\pgfpathcurveto{\pgfqpoint{2.175025in}{1.675090in}}{\pgfqpoint{2.178297in}{1.667190in}}{\pgfqpoint{2.184121in}{1.661366in}}%
\pgfpathcurveto{\pgfqpoint{2.189945in}{1.655542in}}{\pgfqpoint{2.197845in}{1.652270in}}{\pgfqpoint{2.206081in}{1.652270in}}%
\pgfpathclose%
\pgfusepath{stroke,fill}%
\end{pgfscope}%
\begin{pgfscope}%
\pgfpathrectangle{\pgfqpoint{0.100000in}{0.220728in}}{\pgfqpoint{3.696000in}{3.696000in}}%
\pgfusepath{clip}%
\pgfsetbuttcap%
\pgfsetroundjoin%
\definecolor{currentfill}{rgb}{0.121569,0.466667,0.705882}%
\pgfsetfillcolor{currentfill}%
\pgfsetfillopacity{0.741647}%
\pgfsetlinewidth{1.003750pt}%
\definecolor{currentstroke}{rgb}{0.121569,0.466667,0.705882}%
\pgfsetstrokecolor{currentstroke}%
\pgfsetstrokeopacity{0.741647}%
\pgfsetdash{}{0pt}%
\pgfpathmoveto{\pgfqpoint{2.208383in}{1.653902in}}%
\pgfpathcurveto{\pgfqpoint{2.216619in}{1.653902in}}{\pgfqpoint{2.224519in}{1.657174in}}{\pgfqpoint{2.230343in}{1.662998in}}%
\pgfpathcurveto{\pgfqpoint{2.236167in}{1.668822in}}{\pgfqpoint{2.239439in}{1.676722in}}{\pgfqpoint{2.239439in}{1.684958in}}%
\pgfpathcurveto{\pgfqpoint{2.239439in}{1.693195in}}{\pgfqpoint{2.236167in}{1.701095in}}{\pgfqpoint{2.230343in}{1.706919in}}%
\pgfpathcurveto{\pgfqpoint{2.224519in}{1.712743in}}{\pgfqpoint{2.216619in}{1.716015in}}{\pgfqpoint{2.208383in}{1.716015in}}%
\pgfpathcurveto{\pgfqpoint{2.200146in}{1.716015in}}{\pgfqpoint{2.192246in}{1.712743in}}{\pgfqpoint{2.186422in}{1.706919in}}%
\pgfpathcurveto{\pgfqpoint{2.180598in}{1.701095in}}{\pgfqpoint{2.177326in}{1.693195in}}{\pgfqpoint{2.177326in}{1.684958in}}%
\pgfpathcurveto{\pgfqpoint{2.177326in}{1.676722in}}{\pgfqpoint{2.180598in}{1.668822in}}{\pgfqpoint{2.186422in}{1.662998in}}%
\pgfpathcurveto{\pgfqpoint{2.192246in}{1.657174in}}{\pgfqpoint{2.200146in}{1.653902in}}{\pgfqpoint{2.208383in}{1.653902in}}%
\pgfpathclose%
\pgfusepath{stroke,fill}%
\end{pgfscope}%
\begin{pgfscope}%
\pgfpathrectangle{\pgfqpoint{0.100000in}{0.220728in}}{\pgfqpoint{3.696000in}{3.696000in}}%
\pgfusepath{clip}%
\pgfsetbuttcap%
\pgfsetroundjoin%
\definecolor{currentfill}{rgb}{0.121569,0.466667,0.705882}%
\pgfsetfillcolor{currentfill}%
\pgfsetfillopacity{0.742639}%
\pgfsetlinewidth{1.003750pt}%
\definecolor{currentstroke}{rgb}{0.121569,0.466667,0.705882}%
\pgfsetstrokecolor{currentstroke}%
\pgfsetstrokeopacity{0.742639}%
\pgfsetdash{}{0pt}%
\pgfpathmoveto{\pgfqpoint{2.209400in}{1.652671in}}%
\pgfpathcurveto{\pgfqpoint{2.217637in}{1.652671in}}{\pgfqpoint{2.225537in}{1.655943in}}{\pgfqpoint{2.231361in}{1.661767in}}%
\pgfpathcurveto{\pgfqpoint{2.237185in}{1.667591in}}{\pgfqpoint{2.240457in}{1.675491in}}{\pgfqpoint{2.240457in}{1.683727in}}%
\pgfpathcurveto{\pgfqpoint{2.240457in}{1.691964in}}{\pgfqpoint{2.237185in}{1.699864in}}{\pgfqpoint{2.231361in}{1.705687in}}%
\pgfpathcurveto{\pgfqpoint{2.225537in}{1.711511in}}{\pgfqpoint{2.217637in}{1.714784in}}{\pgfqpoint{2.209400in}{1.714784in}}%
\pgfpathcurveto{\pgfqpoint{2.201164in}{1.714784in}}{\pgfqpoint{2.193264in}{1.711511in}}{\pgfqpoint{2.187440in}{1.705687in}}%
\pgfpathcurveto{\pgfqpoint{2.181616in}{1.699864in}}{\pgfqpoint{2.178344in}{1.691964in}}{\pgfqpoint{2.178344in}{1.683727in}}%
\pgfpathcurveto{\pgfqpoint{2.178344in}{1.675491in}}{\pgfqpoint{2.181616in}{1.667591in}}{\pgfqpoint{2.187440in}{1.661767in}}%
\pgfpathcurveto{\pgfqpoint{2.193264in}{1.655943in}}{\pgfqpoint{2.201164in}{1.652671in}}{\pgfqpoint{2.209400in}{1.652671in}}%
\pgfpathclose%
\pgfusepath{stroke,fill}%
\end{pgfscope}%
\begin{pgfscope}%
\pgfpathrectangle{\pgfqpoint{0.100000in}{0.220728in}}{\pgfqpoint{3.696000in}{3.696000in}}%
\pgfusepath{clip}%
\pgfsetbuttcap%
\pgfsetroundjoin%
\definecolor{currentfill}{rgb}{0.121569,0.466667,0.705882}%
\pgfsetfillcolor{currentfill}%
\pgfsetfillopacity{0.744040}%
\pgfsetlinewidth{1.003750pt}%
\definecolor{currentstroke}{rgb}{0.121569,0.466667,0.705882}%
\pgfsetstrokecolor{currentstroke}%
\pgfsetstrokeopacity{0.744040}%
\pgfsetdash{}{0pt}%
\pgfpathmoveto{\pgfqpoint{2.209915in}{1.651690in}}%
\pgfpathcurveto{\pgfqpoint{2.218151in}{1.651690in}}{\pgfqpoint{2.226051in}{1.654962in}}{\pgfqpoint{2.231875in}{1.660786in}}%
\pgfpathcurveto{\pgfqpoint{2.237699in}{1.666610in}}{\pgfqpoint{2.240971in}{1.674510in}}{\pgfqpoint{2.240971in}{1.682747in}}%
\pgfpathcurveto{\pgfqpoint{2.240971in}{1.690983in}}{\pgfqpoint{2.237699in}{1.698883in}}{\pgfqpoint{2.231875in}{1.704707in}}%
\pgfpathcurveto{\pgfqpoint{2.226051in}{1.710531in}}{\pgfqpoint{2.218151in}{1.713803in}}{\pgfqpoint{2.209915in}{1.713803in}}%
\pgfpathcurveto{\pgfqpoint{2.201678in}{1.713803in}}{\pgfqpoint{2.193778in}{1.710531in}}{\pgfqpoint{2.187954in}{1.704707in}}%
\pgfpathcurveto{\pgfqpoint{2.182130in}{1.698883in}}{\pgfqpoint{2.178858in}{1.690983in}}{\pgfqpoint{2.178858in}{1.682747in}}%
\pgfpathcurveto{\pgfqpoint{2.178858in}{1.674510in}}{\pgfqpoint{2.182130in}{1.666610in}}{\pgfqpoint{2.187954in}{1.660786in}}%
\pgfpathcurveto{\pgfqpoint{2.193778in}{1.654962in}}{\pgfqpoint{2.201678in}{1.651690in}}{\pgfqpoint{2.209915in}{1.651690in}}%
\pgfpathclose%
\pgfusepath{stroke,fill}%
\end{pgfscope}%
\begin{pgfscope}%
\pgfpathrectangle{\pgfqpoint{0.100000in}{0.220728in}}{\pgfqpoint{3.696000in}{3.696000in}}%
\pgfusepath{clip}%
\pgfsetbuttcap%
\pgfsetroundjoin%
\definecolor{currentfill}{rgb}{0.121569,0.466667,0.705882}%
\pgfsetfillcolor{currentfill}%
\pgfsetfillopacity{0.744616}%
\pgfsetlinewidth{1.003750pt}%
\definecolor{currentstroke}{rgb}{0.121569,0.466667,0.705882}%
\pgfsetstrokecolor{currentstroke}%
\pgfsetstrokeopacity{0.744616}%
\pgfsetdash{}{0pt}%
\pgfpathmoveto{\pgfqpoint{2.210584in}{1.650123in}}%
\pgfpathcurveto{\pgfqpoint{2.218821in}{1.650123in}}{\pgfqpoint{2.226721in}{1.653395in}}{\pgfqpoint{2.232545in}{1.659219in}}%
\pgfpathcurveto{\pgfqpoint{2.238368in}{1.665043in}}{\pgfqpoint{2.241641in}{1.672943in}}{\pgfqpoint{2.241641in}{1.681179in}}%
\pgfpathcurveto{\pgfqpoint{2.241641in}{1.689415in}}{\pgfqpoint{2.238368in}{1.697315in}}{\pgfqpoint{2.232545in}{1.703139in}}%
\pgfpathcurveto{\pgfqpoint{2.226721in}{1.708963in}}{\pgfqpoint{2.218821in}{1.712236in}}{\pgfqpoint{2.210584in}{1.712236in}}%
\pgfpathcurveto{\pgfqpoint{2.202348in}{1.712236in}}{\pgfqpoint{2.194448in}{1.708963in}}{\pgfqpoint{2.188624in}{1.703139in}}%
\pgfpathcurveto{\pgfqpoint{2.182800in}{1.697315in}}{\pgfqpoint{2.179528in}{1.689415in}}{\pgfqpoint{2.179528in}{1.681179in}}%
\pgfpathcurveto{\pgfqpoint{2.179528in}{1.672943in}}{\pgfqpoint{2.182800in}{1.665043in}}{\pgfqpoint{2.188624in}{1.659219in}}%
\pgfpathcurveto{\pgfqpoint{2.194448in}{1.653395in}}{\pgfqpoint{2.202348in}{1.650123in}}{\pgfqpoint{2.210584in}{1.650123in}}%
\pgfpathclose%
\pgfusepath{stroke,fill}%
\end{pgfscope}%
\begin{pgfscope}%
\pgfpathrectangle{\pgfqpoint{0.100000in}{0.220728in}}{\pgfqpoint{3.696000in}{3.696000in}}%
\pgfusepath{clip}%
\pgfsetbuttcap%
\pgfsetroundjoin%
\definecolor{currentfill}{rgb}{0.121569,0.466667,0.705882}%
\pgfsetfillcolor{currentfill}%
\pgfsetfillopacity{0.745641}%
\pgfsetlinewidth{1.003750pt}%
\definecolor{currentstroke}{rgb}{0.121569,0.466667,0.705882}%
\pgfsetstrokecolor{currentstroke}%
\pgfsetstrokeopacity{0.745641}%
\pgfsetdash{}{0pt}%
\pgfpathmoveto{\pgfqpoint{2.211616in}{1.650527in}}%
\pgfpathcurveto{\pgfqpoint{2.219852in}{1.650527in}}{\pgfqpoint{2.227752in}{1.653799in}}{\pgfqpoint{2.233576in}{1.659623in}}%
\pgfpathcurveto{\pgfqpoint{2.239400in}{1.665447in}}{\pgfqpoint{2.242672in}{1.673347in}}{\pgfqpoint{2.242672in}{1.681584in}}%
\pgfpathcurveto{\pgfqpoint{2.242672in}{1.689820in}}{\pgfqpoint{2.239400in}{1.697720in}}{\pgfqpoint{2.233576in}{1.703544in}}%
\pgfpathcurveto{\pgfqpoint{2.227752in}{1.709368in}}{\pgfqpoint{2.219852in}{1.712640in}}{\pgfqpoint{2.211616in}{1.712640in}}%
\pgfpathcurveto{\pgfqpoint{2.203380in}{1.712640in}}{\pgfqpoint{2.195480in}{1.709368in}}{\pgfqpoint{2.189656in}{1.703544in}}%
\pgfpathcurveto{\pgfqpoint{2.183832in}{1.697720in}}{\pgfqpoint{2.180559in}{1.689820in}}{\pgfqpoint{2.180559in}{1.681584in}}%
\pgfpathcurveto{\pgfqpoint{2.180559in}{1.673347in}}{\pgfqpoint{2.183832in}{1.665447in}}{\pgfqpoint{2.189656in}{1.659623in}}%
\pgfpathcurveto{\pgfqpoint{2.195480in}{1.653799in}}{\pgfqpoint{2.203380in}{1.650527in}}{\pgfqpoint{2.211616in}{1.650527in}}%
\pgfpathclose%
\pgfusepath{stroke,fill}%
\end{pgfscope}%
\begin{pgfscope}%
\pgfpathrectangle{\pgfqpoint{0.100000in}{0.220728in}}{\pgfqpoint{3.696000in}{3.696000in}}%
\pgfusepath{clip}%
\pgfsetbuttcap%
\pgfsetroundjoin%
\definecolor{currentfill}{rgb}{0.121569,0.466667,0.705882}%
\pgfsetfillcolor{currentfill}%
\pgfsetfillopacity{0.746021}%
\pgfsetlinewidth{1.003750pt}%
\definecolor{currentstroke}{rgb}{0.121569,0.466667,0.705882}%
\pgfsetstrokecolor{currentstroke}%
\pgfsetstrokeopacity{0.746021}%
\pgfsetdash{}{0pt}%
\pgfpathmoveto{\pgfqpoint{2.211869in}{1.649386in}}%
\pgfpathcurveto{\pgfqpoint{2.220105in}{1.649386in}}{\pgfqpoint{2.228005in}{1.652658in}}{\pgfqpoint{2.233829in}{1.658482in}}%
\pgfpathcurveto{\pgfqpoint{2.239653in}{1.664306in}}{\pgfqpoint{2.242925in}{1.672206in}}{\pgfqpoint{2.242925in}{1.680442in}}%
\pgfpathcurveto{\pgfqpoint{2.242925in}{1.688679in}}{\pgfqpoint{2.239653in}{1.696579in}}{\pgfqpoint{2.233829in}{1.702403in}}%
\pgfpathcurveto{\pgfqpoint{2.228005in}{1.708227in}}{\pgfqpoint{2.220105in}{1.711499in}}{\pgfqpoint{2.211869in}{1.711499in}}%
\pgfpathcurveto{\pgfqpoint{2.203632in}{1.711499in}}{\pgfqpoint{2.195732in}{1.708227in}}{\pgfqpoint{2.189909in}{1.702403in}}%
\pgfpathcurveto{\pgfqpoint{2.184085in}{1.696579in}}{\pgfqpoint{2.180812in}{1.688679in}}{\pgfqpoint{2.180812in}{1.680442in}}%
\pgfpathcurveto{\pgfqpoint{2.180812in}{1.672206in}}{\pgfqpoint{2.184085in}{1.664306in}}{\pgfqpoint{2.189909in}{1.658482in}}%
\pgfpathcurveto{\pgfqpoint{2.195732in}{1.652658in}}{\pgfqpoint{2.203632in}{1.649386in}}{\pgfqpoint{2.211869in}{1.649386in}}%
\pgfpathclose%
\pgfusepath{stroke,fill}%
\end{pgfscope}%
\begin{pgfscope}%
\pgfpathrectangle{\pgfqpoint{0.100000in}{0.220728in}}{\pgfqpoint{3.696000in}{3.696000in}}%
\pgfusepath{clip}%
\pgfsetbuttcap%
\pgfsetroundjoin%
\definecolor{currentfill}{rgb}{0.121569,0.466667,0.705882}%
\pgfsetfillcolor{currentfill}%
\pgfsetfillopacity{0.746759}%
\pgfsetlinewidth{1.003750pt}%
\definecolor{currentstroke}{rgb}{0.121569,0.466667,0.705882}%
\pgfsetstrokecolor{currentstroke}%
\pgfsetstrokeopacity{0.746759}%
\pgfsetdash{}{0pt}%
\pgfpathmoveto{\pgfqpoint{2.212511in}{1.648696in}}%
\pgfpathcurveto{\pgfqpoint{2.220747in}{1.648696in}}{\pgfqpoint{2.228647in}{1.651968in}}{\pgfqpoint{2.234471in}{1.657792in}}%
\pgfpathcurveto{\pgfqpoint{2.240295in}{1.663616in}}{\pgfqpoint{2.243568in}{1.671516in}}{\pgfqpoint{2.243568in}{1.679753in}}%
\pgfpathcurveto{\pgfqpoint{2.243568in}{1.687989in}}{\pgfqpoint{2.240295in}{1.695889in}}{\pgfqpoint{2.234471in}{1.701713in}}%
\pgfpathcurveto{\pgfqpoint{2.228647in}{1.707537in}}{\pgfqpoint{2.220747in}{1.710809in}}{\pgfqpoint{2.212511in}{1.710809in}}%
\pgfpathcurveto{\pgfqpoint{2.204275in}{1.710809in}}{\pgfqpoint{2.196375in}{1.707537in}}{\pgfqpoint{2.190551in}{1.701713in}}%
\pgfpathcurveto{\pgfqpoint{2.184727in}{1.695889in}}{\pgfqpoint{2.181455in}{1.687989in}}{\pgfqpoint{2.181455in}{1.679753in}}%
\pgfpathcurveto{\pgfqpoint{2.181455in}{1.671516in}}{\pgfqpoint{2.184727in}{1.663616in}}{\pgfqpoint{2.190551in}{1.657792in}}%
\pgfpathcurveto{\pgfqpoint{2.196375in}{1.651968in}}{\pgfqpoint{2.204275in}{1.648696in}}{\pgfqpoint{2.212511in}{1.648696in}}%
\pgfpathclose%
\pgfusepath{stroke,fill}%
\end{pgfscope}%
\begin{pgfscope}%
\pgfpathrectangle{\pgfqpoint{0.100000in}{0.220728in}}{\pgfqpoint{3.696000in}{3.696000in}}%
\pgfusepath{clip}%
\pgfsetbuttcap%
\pgfsetroundjoin%
\definecolor{currentfill}{rgb}{0.121569,0.466667,0.705882}%
\pgfsetfillcolor{currentfill}%
\pgfsetfillopacity{0.747214}%
\pgfsetlinewidth{1.003750pt}%
\definecolor{currentstroke}{rgb}{0.121569,0.466667,0.705882}%
\pgfsetstrokecolor{currentstroke}%
\pgfsetstrokeopacity{0.747214}%
\pgfsetdash{}{0pt}%
\pgfpathmoveto{\pgfqpoint{2.212753in}{1.648560in}}%
\pgfpathcurveto{\pgfqpoint{2.220989in}{1.648560in}}{\pgfqpoint{2.228889in}{1.651833in}}{\pgfqpoint{2.234713in}{1.657657in}}%
\pgfpathcurveto{\pgfqpoint{2.240537in}{1.663481in}}{\pgfqpoint{2.243810in}{1.671381in}}{\pgfqpoint{2.243810in}{1.679617in}}%
\pgfpathcurveto{\pgfqpoint{2.243810in}{1.687853in}}{\pgfqpoint{2.240537in}{1.695753in}}{\pgfqpoint{2.234713in}{1.701577in}}%
\pgfpathcurveto{\pgfqpoint{2.228889in}{1.707401in}}{\pgfqpoint{2.220989in}{1.710673in}}{\pgfqpoint{2.212753in}{1.710673in}}%
\pgfpathcurveto{\pgfqpoint{2.204517in}{1.710673in}}{\pgfqpoint{2.196617in}{1.707401in}}{\pgfqpoint{2.190793in}{1.701577in}}%
\pgfpathcurveto{\pgfqpoint{2.184969in}{1.695753in}}{\pgfqpoint{2.181697in}{1.687853in}}{\pgfqpoint{2.181697in}{1.679617in}}%
\pgfpathcurveto{\pgfqpoint{2.181697in}{1.671381in}}{\pgfqpoint{2.184969in}{1.663481in}}{\pgfqpoint{2.190793in}{1.657657in}}%
\pgfpathcurveto{\pgfqpoint{2.196617in}{1.651833in}}{\pgfqpoint{2.204517in}{1.648560in}}{\pgfqpoint{2.212753in}{1.648560in}}%
\pgfpathclose%
\pgfusepath{stroke,fill}%
\end{pgfscope}%
\begin{pgfscope}%
\pgfpathrectangle{\pgfqpoint{0.100000in}{0.220728in}}{\pgfqpoint{3.696000in}{3.696000in}}%
\pgfusepath{clip}%
\pgfsetbuttcap%
\pgfsetroundjoin%
\definecolor{currentfill}{rgb}{0.121569,0.466667,0.705882}%
\pgfsetfillcolor{currentfill}%
\pgfsetfillopacity{0.747973}%
\pgfsetlinewidth{1.003750pt}%
\definecolor{currentstroke}{rgb}{0.121569,0.466667,0.705882}%
\pgfsetstrokecolor{currentstroke}%
\pgfsetstrokeopacity{0.747973}%
\pgfsetdash{}{0pt}%
\pgfpathmoveto{\pgfqpoint{2.213606in}{1.648501in}}%
\pgfpathcurveto{\pgfqpoint{2.221842in}{1.648501in}}{\pgfqpoint{2.229742in}{1.651773in}}{\pgfqpoint{2.235566in}{1.657597in}}%
\pgfpathcurveto{\pgfqpoint{2.241390in}{1.663421in}}{\pgfqpoint{2.244663in}{1.671321in}}{\pgfqpoint{2.244663in}{1.679557in}}%
\pgfpathcurveto{\pgfqpoint{2.244663in}{1.687794in}}{\pgfqpoint{2.241390in}{1.695694in}}{\pgfqpoint{2.235566in}{1.701517in}}%
\pgfpathcurveto{\pgfqpoint{2.229742in}{1.707341in}}{\pgfqpoint{2.221842in}{1.710614in}}{\pgfqpoint{2.213606in}{1.710614in}}%
\pgfpathcurveto{\pgfqpoint{2.205370in}{1.710614in}}{\pgfqpoint{2.197470in}{1.707341in}}{\pgfqpoint{2.191646in}{1.701517in}}%
\pgfpathcurveto{\pgfqpoint{2.185822in}{1.695694in}}{\pgfqpoint{2.182550in}{1.687794in}}{\pgfqpoint{2.182550in}{1.679557in}}%
\pgfpathcurveto{\pgfqpoint{2.182550in}{1.671321in}}{\pgfqpoint{2.185822in}{1.663421in}}{\pgfqpoint{2.191646in}{1.657597in}}%
\pgfpathcurveto{\pgfqpoint{2.197470in}{1.651773in}}{\pgfqpoint{2.205370in}{1.648501in}}{\pgfqpoint{2.213606in}{1.648501in}}%
\pgfpathclose%
\pgfusepath{stroke,fill}%
\end{pgfscope}%
\begin{pgfscope}%
\pgfpathrectangle{\pgfqpoint{0.100000in}{0.220728in}}{\pgfqpoint{3.696000in}{3.696000in}}%
\pgfusepath{clip}%
\pgfsetbuttcap%
\pgfsetroundjoin%
\definecolor{currentfill}{rgb}{0.121569,0.466667,0.705882}%
\pgfsetfillcolor{currentfill}%
\pgfsetfillopacity{0.748869}%
\pgfsetlinewidth{1.003750pt}%
\definecolor{currentstroke}{rgb}{0.121569,0.466667,0.705882}%
\pgfsetstrokecolor{currentstroke}%
\pgfsetstrokeopacity{0.748869}%
\pgfsetdash{}{0pt}%
\pgfpathmoveto{\pgfqpoint{2.213985in}{1.647548in}}%
\pgfpathcurveto{\pgfqpoint{2.222222in}{1.647548in}}{\pgfqpoint{2.230122in}{1.650820in}}{\pgfqpoint{2.235946in}{1.656644in}}%
\pgfpathcurveto{\pgfqpoint{2.241770in}{1.662468in}}{\pgfqpoint{2.245042in}{1.670368in}}{\pgfqpoint{2.245042in}{1.678604in}}%
\pgfpathcurveto{\pgfqpoint{2.245042in}{1.686840in}}{\pgfqpoint{2.241770in}{1.694740in}}{\pgfqpoint{2.235946in}{1.700564in}}%
\pgfpathcurveto{\pgfqpoint{2.230122in}{1.706388in}}{\pgfqpoint{2.222222in}{1.709661in}}{\pgfqpoint{2.213985in}{1.709661in}}%
\pgfpathcurveto{\pgfqpoint{2.205749in}{1.709661in}}{\pgfqpoint{2.197849in}{1.706388in}}{\pgfqpoint{2.192025in}{1.700564in}}%
\pgfpathcurveto{\pgfqpoint{2.186201in}{1.694740in}}{\pgfqpoint{2.182929in}{1.686840in}}{\pgfqpoint{2.182929in}{1.678604in}}%
\pgfpathcurveto{\pgfqpoint{2.182929in}{1.670368in}}{\pgfqpoint{2.186201in}{1.662468in}}{\pgfqpoint{2.192025in}{1.656644in}}%
\pgfpathcurveto{\pgfqpoint{2.197849in}{1.650820in}}{\pgfqpoint{2.205749in}{1.647548in}}{\pgfqpoint{2.213985in}{1.647548in}}%
\pgfpathclose%
\pgfusepath{stroke,fill}%
\end{pgfscope}%
\begin{pgfscope}%
\pgfpathrectangle{\pgfqpoint{0.100000in}{0.220728in}}{\pgfqpoint{3.696000in}{3.696000in}}%
\pgfusepath{clip}%
\pgfsetbuttcap%
\pgfsetroundjoin%
\definecolor{currentfill}{rgb}{0.121569,0.466667,0.705882}%
\pgfsetfillcolor{currentfill}%
\pgfsetfillopacity{0.750056}%
\pgfsetlinewidth{1.003750pt}%
\definecolor{currentstroke}{rgb}{0.121569,0.466667,0.705882}%
\pgfsetstrokecolor{currentstroke}%
\pgfsetstrokeopacity{0.750056}%
\pgfsetdash{}{0pt}%
\pgfpathmoveto{\pgfqpoint{2.215392in}{1.643330in}}%
\pgfpathcurveto{\pgfqpoint{2.223628in}{1.643330in}}{\pgfqpoint{2.231528in}{1.646602in}}{\pgfqpoint{2.237352in}{1.652426in}}%
\pgfpathcurveto{\pgfqpoint{2.243176in}{1.658250in}}{\pgfqpoint{2.246448in}{1.666150in}}{\pgfqpoint{2.246448in}{1.674386in}}%
\pgfpathcurveto{\pgfqpoint{2.246448in}{1.682623in}}{\pgfqpoint{2.243176in}{1.690523in}}{\pgfqpoint{2.237352in}{1.696347in}}%
\pgfpathcurveto{\pgfqpoint{2.231528in}{1.702170in}}{\pgfqpoint{2.223628in}{1.705443in}}{\pgfqpoint{2.215392in}{1.705443in}}%
\pgfpathcurveto{\pgfqpoint{2.207156in}{1.705443in}}{\pgfqpoint{2.199256in}{1.702170in}}{\pgfqpoint{2.193432in}{1.696347in}}%
\pgfpathcurveto{\pgfqpoint{2.187608in}{1.690523in}}{\pgfqpoint{2.184335in}{1.682623in}}{\pgfqpoint{2.184335in}{1.674386in}}%
\pgfpathcurveto{\pgfqpoint{2.184335in}{1.666150in}}{\pgfqpoint{2.187608in}{1.658250in}}{\pgfqpoint{2.193432in}{1.652426in}}%
\pgfpathcurveto{\pgfqpoint{2.199256in}{1.646602in}}{\pgfqpoint{2.207156in}{1.643330in}}{\pgfqpoint{2.215392in}{1.643330in}}%
\pgfpathclose%
\pgfusepath{stroke,fill}%
\end{pgfscope}%
\begin{pgfscope}%
\pgfpathrectangle{\pgfqpoint{0.100000in}{0.220728in}}{\pgfqpoint{3.696000in}{3.696000in}}%
\pgfusepath{clip}%
\pgfsetbuttcap%
\pgfsetroundjoin%
\definecolor{currentfill}{rgb}{0.121569,0.466667,0.705882}%
\pgfsetfillcolor{currentfill}%
\pgfsetfillopacity{0.752204}%
\pgfsetlinewidth{1.003750pt}%
\definecolor{currentstroke}{rgb}{0.121569,0.466667,0.705882}%
\pgfsetstrokecolor{currentstroke}%
\pgfsetstrokeopacity{0.752204}%
\pgfsetdash{}{0pt}%
\pgfpathmoveto{\pgfqpoint{2.217742in}{1.643648in}}%
\pgfpathcurveto{\pgfqpoint{2.225979in}{1.643648in}}{\pgfqpoint{2.233879in}{1.646920in}}{\pgfqpoint{2.239703in}{1.652744in}}%
\pgfpathcurveto{\pgfqpoint{2.245526in}{1.658568in}}{\pgfqpoint{2.248799in}{1.666468in}}{\pgfqpoint{2.248799in}{1.674705in}}%
\pgfpathcurveto{\pgfqpoint{2.248799in}{1.682941in}}{\pgfqpoint{2.245526in}{1.690841in}}{\pgfqpoint{2.239703in}{1.696665in}}%
\pgfpathcurveto{\pgfqpoint{2.233879in}{1.702489in}}{\pgfqpoint{2.225979in}{1.705761in}}{\pgfqpoint{2.217742in}{1.705761in}}%
\pgfpathcurveto{\pgfqpoint{2.209506in}{1.705761in}}{\pgfqpoint{2.201606in}{1.702489in}}{\pgfqpoint{2.195782in}{1.696665in}}%
\pgfpathcurveto{\pgfqpoint{2.189958in}{1.690841in}}{\pgfqpoint{2.186686in}{1.682941in}}{\pgfqpoint{2.186686in}{1.674705in}}%
\pgfpathcurveto{\pgfqpoint{2.186686in}{1.666468in}}{\pgfqpoint{2.189958in}{1.658568in}}{\pgfqpoint{2.195782in}{1.652744in}}%
\pgfpathcurveto{\pgfqpoint{2.201606in}{1.646920in}}{\pgfqpoint{2.209506in}{1.643648in}}{\pgfqpoint{2.217742in}{1.643648in}}%
\pgfpathclose%
\pgfusepath{stroke,fill}%
\end{pgfscope}%
\begin{pgfscope}%
\pgfpathrectangle{\pgfqpoint{0.100000in}{0.220728in}}{\pgfqpoint{3.696000in}{3.696000in}}%
\pgfusepath{clip}%
\pgfsetbuttcap%
\pgfsetroundjoin%
\definecolor{currentfill}{rgb}{0.121569,0.466667,0.705882}%
\pgfsetfillcolor{currentfill}%
\pgfsetfillopacity{0.754740}%
\pgfsetlinewidth{1.003750pt}%
\definecolor{currentstroke}{rgb}{0.121569,0.466667,0.705882}%
\pgfsetstrokecolor{currentstroke}%
\pgfsetstrokeopacity{0.754740}%
\pgfsetdash{}{0pt}%
\pgfpathmoveto{\pgfqpoint{2.219308in}{1.641577in}}%
\pgfpathcurveto{\pgfqpoint{2.227544in}{1.641577in}}{\pgfqpoint{2.235444in}{1.644849in}}{\pgfqpoint{2.241268in}{1.650673in}}%
\pgfpathcurveto{\pgfqpoint{2.247092in}{1.656497in}}{\pgfqpoint{2.250364in}{1.664397in}}{\pgfqpoint{2.250364in}{1.672633in}}%
\pgfpathcurveto{\pgfqpoint{2.250364in}{1.680870in}}{\pgfqpoint{2.247092in}{1.688770in}}{\pgfqpoint{2.241268in}{1.694594in}}%
\pgfpathcurveto{\pgfqpoint{2.235444in}{1.700418in}}{\pgfqpoint{2.227544in}{1.703690in}}{\pgfqpoint{2.219308in}{1.703690in}}%
\pgfpathcurveto{\pgfqpoint{2.211071in}{1.703690in}}{\pgfqpoint{2.203171in}{1.700418in}}{\pgfqpoint{2.197347in}{1.694594in}}%
\pgfpathcurveto{\pgfqpoint{2.191524in}{1.688770in}}{\pgfqpoint{2.188251in}{1.680870in}}{\pgfqpoint{2.188251in}{1.672633in}}%
\pgfpathcurveto{\pgfqpoint{2.188251in}{1.664397in}}{\pgfqpoint{2.191524in}{1.656497in}}{\pgfqpoint{2.197347in}{1.650673in}}%
\pgfpathcurveto{\pgfqpoint{2.203171in}{1.644849in}}{\pgfqpoint{2.211071in}{1.641577in}}{\pgfqpoint{2.219308in}{1.641577in}}%
\pgfpathclose%
\pgfusepath{stroke,fill}%
\end{pgfscope}%
\begin{pgfscope}%
\pgfpathrectangle{\pgfqpoint{0.100000in}{0.220728in}}{\pgfqpoint{3.696000in}{3.696000in}}%
\pgfusepath{clip}%
\pgfsetbuttcap%
\pgfsetroundjoin%
\definecolor{currentfill}{rgb}{0.121569,0.466667,0.705882}%
\pgfsetfillcolor{currentfill}%
\pgfsetfillopacity{0.757746}%
\pgfsetlinewidth{1.003750pt}%
\definecolor{currentstroke}{rgb}{0.121569,0.466667,0.705882}%
\pgfsetstrokecolor{currentstroke}%
\pgfsetstrokeopacity{0.757746}%
\pgfsetdash{}{0pt}%
\pgfpathmoveto{\pgfqpoint{2.220832in}{1.638987in}}%
\pgfpathcurveto{\pgfqpoint{2.229068in}{1.638987in}}{\pgfqpoint{2.236968in}{1.642259in}}{\pgfqpoint{2.242792in}{1.648083in}}%
\pgfpathcurveto{\pgfqpoint{2.248616in}{1.653907in}}{\pgfqpoint{2.251888in}{1.661807in}}{\pgfqpoint{2.251888in}{1.670044in}}%
\pgfpathcurveto{\pgfqpoint{2.251888in}{1.678280in}}{\pgfqpoint{2.248616in}{1.686180in}}{\pgfqpoint{2.242792in}{1.692004in}}%
\pgfpathcurveto{\pgfqpoint{2.236968in}{1.697828in}}{\pgfqpoint{2.229068in}{1.701100in}}{\pgfqpoint{2.220832in}{1.701100in}}%
\pgfpathcurveto{\pgfqpoint{2.212595in}{1.701100in}}{\pgfqpoint{2.204695in}{1.697828in}}{\pgfqpoint{2.198871in}{1.692004in}}%
\pgfpathcurveto{\pgfqpoint{2.193048in}{1.686180in}}{\pgfqpoint{2.189775in}{1.678280in}}{\pgfqpoint{2.189775in}{1.670044in}}%
\pgfpathcurveto{\pgfqpoint{2.189775in}{1.661807in}}{\pgfqpoint{2.193048in}{1.653907in}}{\pgfqpoint{2.198871in}{1.648083in}}%
\pgfpathcurveto{\pgfqpoint{2.204695in}{1.642259in}}{\pgfqpoint{2.212595in}{1.638987in}}{\pgfqpoint{2.220832in}{1.638987in}}%
\pgfpathclose%
\pgfusepath{stroke,fill}%
\end{pgfscope}%
\begin{pgfscope}%
\pgfpathrectangle{\pgfqpoint{0.100000in}{0.220728in}}{\pgfqpoint{3.696000in}{3.696000in}}%
\pgfusepath{clip}%
\pgfsetbuttcap%
\pgfsetroundjoin%
\definecolor{currentfill}{rgb}{0.121569,0.466667,0.705882}%
\pgfsetfillcolor{currentfill}%
\pgfsetfillopacity{0.760660}%
\pgfsetlinewidth{1.003750pt}%
\definecolor{currentstroke}{rgb}{0.121569,0.466667,0.705882}%
\pgfsetstrokecolor{currentstroke}%
\pgfsetstrokeopacity{0.760660}%
\pgfsetdash{}{0pt}%
\pgfpathmoveto{\pgfqpoint{2.223806in}{1.633430in}}%
\pgfpathcurveto{\pgfqpoint{2.232043in}{1.633430in}}{\pgfqpoint{2.239943in}{1.636702in}}{\pgfqpoint{2.245767in}{1.642526in}}%
\pgfpathcurveto{\pgfqpoint{2.251591in}{1.648350in}}{\pgfqpoint{2.254863in}{1.656250in}}{\pgfqpoint{2.254863in}{1.664486in}}%
\pgfpathcurveto{\pgfqpoint{2.254863in}{1.672722in}}{\pgfqpoint{2.251591in}{1.680623in}}{\pgfqpoint{2.245767in}{1.686446in}}%
\pgfpathcurveto{\pgfqpoint{2.239943in}{1.692270in}}{\pgfqpoint{2.232043in}{1.695543in}}{\pgfqpoint{2.223806in}{1.695543in}}%
\pgfpathcurveto{\pgfqpoint{2.215570in}{1.695543in}}{\pgfqpoint{2.207670in}{1.692270in}}{\pgfqpoint{2.201846in}{1.686446in}}%
\pgfpathcurveto{\pgfqpoint{2.196022in}{1.680623in}}{\pgfqpoint{2.192750in}{1.672722in}}{\pgfqpoint{2.192750in}{1.664486in}}%
\pgfpathcurveto{\pgfqpoint{2.192750in}{1.656250in}}{\pgfqpoint{2.196022in}{1.648350in}}{\pgfqpoint{2.201846in}{1.642526in}}%
\pgfpathcurveto{\pgfqpoint{2.207670in}{1.636702in}}{\pgfqpoint{2.215570in}{1.633430in}}{\pgfqpoint{2.223806in}{1.633430in}}%
\pgfpathclose%
\pgfusepath{stroke,fill}%
\end{pgfscope}%
\begin{pgfscope}%
\pgfpathrectangle{\pgfqpoint{0.100000in}{0.220728in}}{\pgfqpoint{3.696000in}{3.696000in}}%
\pgfusepath{clip}%
\pgfsetbuttcap%
\pgfsetroundjoin%
\definecolor{currentfill}{rgb}{0.121569,0.466667,0.705882}%
\pgfsetfillcolor{currentfill}%
\pgfsetfillopacity{0.764679}%
\pgfsetlinewidth{1.003750pt}%
\definecolor{currentstroke}{rgb}{0.121569,0.466667,0.705882}%
\pgfsetstrokecolor{currentstroke}%
\pgfsetstrokeopacity{0.764679}%
\pgfsetdash{}{0pt}%
\pgfpathmoveto{\pgfqpoint{2.227786in}{1.634221in}}%
\pgfpathcurveto{\pgfqpoint{2.236022in}{1.634221in}}{\pgfqpoint{2.243922in}{1.637494in}}{\pgfqpoint{2.249746in}{1.643318in}}%
\pgfpathcurveto{\pgfqpoint{2.255570in}{1.649141in}}{\pgfqpoint{2.258843in}{1.657042in}}{\pgfqpoint{2.258843in}{1.665278in}}%
\pgfpathcurveto{\pgfqpoint{2.258843in}{1.673514in}}{\pgfqpoint{2.255570in}{1.681414in}}{\pgfqpoint{2.249746in}{1.687238in}}%
\pgfpathcurveto{\pgfqpoint{2.243922in}{1.693062in}}{\pgfqpoint{2.236022in}{1.696334in}}{\pgfqpoint{2.227786in}{1.696334in}}%
\pgfpathcurveto{\pgfqpoint{2.219550in}{1.696334in}}{\pgfqpoint{2.211650in}{1.693062in}}{\pgfqpoint{2.205826in}{1.687238in}}%
\pgfpathcurveto{\pgfqpoint{2.200002in}{1.681414in}}{\pgfqpoint{2.196730in}{1.673514in}}{\pgfqpoint{2.196730in}{1.665278in}}%
\pgfpathcurveto{\pgfqpoint{2.196730in}{1.657042in}}{\pgfqpoint{2.200002in}{1.649141in}}{\pgfqpoint{2.205826in}{1.643318in}}%
\pgfpathcurveto{\pgfqpoint{2.211650in}{1.637494in}}{\pgfqpoint{2.219550in}{1.634221in}}{\pgfqpoint{2.227786in}{1.634221in}}%
\pgfpathclose%
\pgfusepath{stroke,fill}%
\end{pgfscope}%
\begin{pgfscope}%
\pgfpathrectangle{\pgfqpoint{0.100000in}{0.220728in}}{\pgfqpoint{3.696000in}{3.696000in}}%
\pgfusepath{clip}%
\pgfsetbuttcap%
\pgfsetroundjoin%
\definecolor{currentfill}{rgb}{0.121569,0.466667,0.705882}%
\pgfsetfillcolor{currentfill}%
\pgfsetfillopacity{0.768229}%
\pgfsetlinewidth{1.003750pt}%
\definecolor{currentstroke}{rgb}{0.121569,0.466667,0.705882}%
\pgfsetstrokecolor{currentstroke}%
\pgfsetstrokeopacity{0.768229}%
\pgfsetdash{}{0pt}%
\pgfpathmoveto{\pgfqpoint{2.230498in}{1.629128in}}%
\pgfpathcurveto{\pgfqpoint{2.238734in}{1.629128in}}{\pgfqpoint{2.246634in}{1.632400in}}{\pgfqpoint{2.252458in}{1.638224in}}%
\pgfpathcurveto{\pgfqpoint{2.258282in}{1.644048in}}{\pgfqpoint{2.261554in}{1.651948in}}{\pgfqpoint{2.261554in}{1.660184in}}%
\pgfpathcurveto{\pgfqpoint{2.261554in}{1.668421in}}{\pgfqpoint{2.258282in}{1.676321in}}{\pgfqpoint{2.252458in}{1.682145in}}%
\pgfpathcurveto{\pgfqpoint{2.246634in}{1.687969in}}{\pgfqpoint{2.238734in}{1.691241in}}{\pgfqpoint{2.230498in}{1.691241in}}%
\pgfpathcurveto{\pgfqpoint{2.222261in}{1.691241in}}{\pgfqpoint{2.214361in}{1.687969in}}{\pgfqpoint{2.208537in}{1.682145in}}%
\pgfpathcurveto{\pgfqpoint{2.202713in}{1.676321in}}{\pgfqpoint{2.199441in}{1.668421in}}{\pgfqpoint{2.199441in}{1.660184in}}%
\pgfpathcurveto{\pgfqpoint{2.199441in}{1.651948in}}{\pgfqpoint{2.202713in}{1.644048in}}{\pgfqpoint{2.208537in}{1.638224in}}%
\pgfpathcurveto{\pgfqpoint{2.214361in}{1.632400in}}{\pgfqpoint{2.222261in}{1.629128in}}{\pgfqpoint{2.230498in}{1.629128in}}%
\pgfpathclose%
\pgfusepath{stroke,fill}%
\end{pgfscope}%
\begin{pgfscope}%
\pgfpathrectangle{\pgfqpoint{0.100000in}{0.220728in}}{\pgfqpoint{3.696000in}{3.696000in}}%
\pgfusepath{clip}%
\pgfsetbuttcap%
\pgfsetroundjoin%
\definecolor{currentfill}{rgb}{0.121569,0.466667,0.705882}%
\pgfsetfillcolor{currentfill}%
\pgfsetfillopacity{0.769960}%
\pgfsetlinewidth{1.003750pt}%
\definecolor{currentstroke}{rgb}{0.121569,0.466667,0.705882}%
\pgfsetstrokecolor{currentstroke}%
\pgfsetstrokeopacity{0.769960}%
\pgfsetdash{}{0pt}%
\pgfpathmoveto{\pgfqpoint{2.232320in}{1.625126in}}%
\pgfpathcurveto{\pgfqpoint{2.240557in}{1.625126in}}{\pgfqpoint{2.248457in}{1.628399in}}{\pgfqpoint{2.254281in}{1.634223in}}%
\pgfpathcurveto{\pgfqpoint{2.260104in}{1.640047in}}{\pgfqpoint{2.263377in}{1.647947in}}{\pgfqpoint{2.263377in}{1.656183in}}%
\pgfpathcurveto{\pgfqpoint{2.263377in}{1.664419in}}{\pgfqpoint{2.260104in}{1.672319in}}{\pgfqpoint{2.254281in}{1.678143in}}%
\pgfpathcurveto{\pgfqpoint{2.248457in}{1.683967in}}{\pgfqpoint{2.240557in}{1.687239in}}{\pgfqpoint{2.232320in}{1.687239in}}%
\pgfpathcurveto{\pgfqpoint{2.224084in}{1.687239in}}{\pgfqpoint{2.216184in}{1.683967in}}{\pgfqpoint{2.210360in}{1.678143in}}%
\pgfpathcurveto{\pgfqpoint{2.204536in}{1.672319in}}{\pgfqpoint{2.201264in}{1.664419in}}{\pgfqpoint{2.201264in}{1.656183in}}%
\pgfpathcurveto{\pgfqpoint{2.201264in}{1.647947in}}{\pgfqpoint{2.204536in}{1.640047in}}{\pgfqpoint{2.210360in}{1.634223in}}%
\pgfpathcurveto{\pgfqpoint{2.216184in}{1.628399in}}{\pgfqpoint{2.224084in}{1.625126in}}{\pgfqpoint{2.232320in}{1.625126in}}%
\pgfpathclose%
\pgfusepath{stroke,fill}%
\end{pgfscope}%
\begin{pgfscope}%
\pgfpathrectangle{\pgfqpoint{0.100000in}{0.220728in}}{\pgfqpoint{3.696000in}{3.696000in}}%
\pgfusepath{clip}%
\pgfsetbuttcap%
\pgfsetroundjoin%
\definecolor{currentfill}{rgb}{0.121569,0.466667,0.705882}%
\pgfsetfillcolor{currentfill}%
\pgfsetfillopacity{0.772193}%
\pgfsetlinewidth{1.003750pt}%
\definecolor{currentstroke}{rgb}{0.121569,0.466667,0.705882}%
\pgfsetstrokecolor{currentstroke}%
\pgfsetstrokeopacity{0.772193}%
\pgfsetdash{}{0pt}%
\pgfpathmoveto{\pgfqpoint{2.234178in}{1.622192in}}%
\pgfpathcurveto{\pgfqpoint{2.242414in}{1.622192in}}{\pgfqpoint{2.250314in}{1.625465in}}{\pgfqpoint{2.256138in}{1.631289in}}%
\pgfpathcurveto{\pgfqpoint{2.261962in}{1.637112in}}{\pgfqpoint{2.265234in}{1.645012in}}{\pgfqpoint{2.265234in}{1.653249in}}%
\pgfpathcurveto{\pgfqpoint{2.265234in}{1.661485in}}{\pgfqpoint{2.261962in}{1.669385in}}{\pgfqpoint{2.256138in}{1.675209in}}%
\pgfpathcurveto{\pgfqpoint{2.250314in}{1.681033in}}{\pgfqpoint{2.242414in}{1.684305in}}{\pgfqpoint{2.234178in}{1.684305in}}%
\pgfpathcurveto{\pgfqpoint{2.225941in}{1.684305in}}{\pgfqpoint{2.218041in}{1.681033in}}{\pgfqpoint{2.212217in}{1.675209in}}%
\pgfpathcurveto{\pgfqpoint{2.206393in}{1.669385in}}{\pgfqpoint{2.203121in}{1.661485in}}{\pgfqpoint{2.203121in}{1.653249in}}%
\pgfpathcurveto{\pgfqpoint{2.203121in}{1.645012in}}{\pgfqpoint{2.206393in}{1.637112in}}{\pgfqpoint{2.212217in}{1.631289in}}%
\pgfpathcurveto{\pgfqpoint{2.218041in}{1.625465in}}{\pgfqpoint{2.225941in}{1.622192in}}{\pgfqpoint{2.234178in}{1.622192in}}%
\pgfpathclose%
\pgfusepath{stroke,fill}%
\end{pgfscope}%
\begin{pgfscope}%
\pgfpathrectangle{\pgfqpoint{0.100000in}{0.220728in}}{\pgfqpoint{3.696000in}{3.696000in}}%
\pgfusepath{clip}%
\pgfsetbuttcap%
\pgfsetroundjoin%
\definecolor{currentfill}{rgb}{0.121569,0.466667,0.705882}%
\pgfsetfillcolor{currentfill}%
\pgfsetfillopacity{0.773627}%
\pgfsetlinewidth{1.003750pt}%
\definecolor{currentstroke}{rgb}{0.121569,0.466667,0.705882}%
\pgfsetstrokecolor{currentstroke}%
\pgfsetstrokeopacity{0.773627}%
\pgfsetdash{}{0pt}%
\pgfpathmoveto{\pgfqpoint{2.235518in}{1.622086in}}%
\pgfpathcurveto{\pgfqpoint{2.243754in}{1.622086in}}{\pgfqpoint{2.251654in}{1.625359in}}{\pgfqpoint{2.257478in}{1.631182in}}%
\pgfpathcurveto{\pgfqpoint{2.263302in}{1.637006in}}{\pgfqpoint{2.266575in}{1.644906in}}{\pgfqpoint{2.266575in}{1.653143in}}%
\pgfpathcurveto{\pgfqpoint{2.266575in}{1.661379in}}{\pgfqpoint{2.263302in}{1.669279in}}{\pgfqpoint{2.257478in}{1.675103in}}%
\pgfpathcurveto{\pgfqpoint{2.251654in}{1.680927in}}{\pgfqpoint{2.243754in}{1.684199in}}{\pgfqpoint{2.235518in}{1.684199in}}%
\pgfpathcurveto{\pgfqpoint{2.227282in}{1.684199in}}{\pgfqpoint{2.219382in}{1.680927in}}{\pgfqpoint{2.213558in}{1.675103in}}%
\pgfpathcurveto{\pgfqpoint{2.207734in}{1.669279in}}{\pgfqpoint{2.204462in}{1.661379in}}{\pgfqpoint{2.204462in}{1.653143in}}%
\pgfpathcurveto{\pgfqpoint{2.204462in}{1.644906in}}{\pgfqpoint{2.207734in}{1.637006in}}{\pgfqpoint{2.213558in}{1.631182in}}%
\pgfpathcurveto{\pgfqpoint{2.219382in}{1.625359in}}{\pgfqpoint{2.227282in}{1.622086in}}{\pgfqpoint{2.235518in}{1.622086in}}%
\pgfpathclose%
\pgfusepath{stroke,fill}%
\end{pgfscope}%
\begin{pgfscope}%
\pgfpathrectangle{\pgfqpoint{0.100000in}{0.220728in}}{\pgfqpoint{3.696000in}{3.696000in}}%
\pgfusepath{clip}%
\pgfsetbuttcap%
\pgfsetroundjoin%
\definecolor{currentfill}{rgb}{0.121569,0.466667,0.705882}%
\pgfsetfillcolor{currentfill}%
\pgfsetfillopacity{0.774228}%
\pgfsetlinewidth{1.003750pt}%
\definecolor{currentstroke}{rgb}{0.121569,0.466667,0.705882}%
\pgfsetstrokecolor{currentstroke}%
\pgfsetstrokeopacity{0.774228}%
\pgfsetdash{}{0pt}%
\pgfpathmoveto{\pgfqpoint{2.235853in}{1.620607in}}%
\pgfpathcurveto{\pgfqpoint{2.244089in}{1.620607in}}{\pgfqpoint{2.251989in}{1.623879in}}{\pgfqpoint{2.257813in}{1.629703in}}%
\pgfpathcurveto{\pgfqpoint{2.263637in}{1.635527in}}{\pgfqpoint{2.266909in}{1.643427in}}{\pgfqpoint{2.266909in}{1.651663in}}%
\pgfpathcurveto{\pgfqpoint{2.266909in}{1.659899in}}{\pgfqpoint{2.263637in}{1.667799in}}{\pgfqpoint{2.257813in}{1.673623in}}%
\pgfpathcurveto{\pgfqpoint{2.251989in}{1.679447in}}{\pgfqpoint{2.244089in}{1.682720in}}{\pgfqpoint{2.235853in}{1.682720in}}%
\pgfpathcurveto{\pgfqpoint{2.227616in}{1.682720in}}{\pgfqpoint{2.219716in}{1.679447in}}{\pgfqpoint{2.213892in}{1.673623in}}%
\pgfpathcurveto{\pgfqpoint{2.208068in}{1.667799in}}{\pgfqpoint{2.204796in}{1.659899in}}{\pgfqpoint{2.204796in}{1.651663in}}%
\pgfpathcurveto{\pgfqpoint{2.204796in}{1.643427in}}{\pgfqpoint{2.208068in}{1.635527in}}{\pgfqpoint{2.213892in}{1.629703in}}%
\pgfpathcurveto{\pgfqpoint{2.219716in}{1.623879in}}{\pgfqpoint{2.227616in}{1.620607in}}{\pgfqpoint{2.235853in}{1.620607in}}%
\pgfpathclose%
\pgfusepath{stroke,fill}%
\end{pgfscope}%
\begin{pgfscope}%
\pgfpathrectangle{\pgfqpoint{0.100000in}{0.220728in}}{\pgfqpoint{3.696000in}{3.696000in}}%
\pgfusepath{clip}%
\pgfsetbuttcap%
\pgfsetroundjoin%
\definecolor{currentfill}{rgb}{0.121569,0.466667,0.705882}%
\pgfsetfillcolor{currentfill}%
\pgfsetfillopacity{0.775115}%
\pgfsetlinewidth{1.003750pt}%
\definecolor{currentstroke}{rgb}{0.121569,0.466667,0.705882}%
\pgfsetstrokecolor{currentstroke}%
\pgfsetstrokeopacity{0.775115}%
\pgfsetdash{}{0pt}%
\pgfpathmoveto{\pgfqpoint{2.237022in}{1.618155in}}%
\pgfpathcurveto{\pgfqpoint{2.245258in}{1.618155in}}{\pgfqpoint{2.253158in}{1.621427in}}{\pgfqpoint{2.258982in}{1.627251in}}%
\pgfpathcurveto{\pgfqpoint{2.264806in}{1.633075in}}{\pgfqpoint{2.268078in}{1.640975in}}{\pgfqpoint{2.268078in}{1.649211in}}%
\pgfpathcurveto{\pgfqpoint{2.268078in}{1.657447in}}{\pgfqpoint{2.264806in}{1.665347in}}{\pgfqpoint{2.258982in}{1.671171in}}%
\pgfpathcurveto{\pgfqpoint{2.253158in}{1.676995in}}{\pgfqpoint{2.245258in}{1.680268in}}{\pgfqpoint{2.237022in}{1.680268in}}%
\pgfpathcurveto{\pgfqpoint{2.228786in}{1.680268in}}{\pgfqpoint{2.220885in}{1.676995in}}{\pgfqpoint{2.215062in}{1.671171in}}%
\pgfpathcurveto{\pgfqpoint{2.209238in}{1.665347in}}{\pgfqpoint{2.205965in}{1.657447in}}{\pgfqpoint{2.205965in}{1.649211in}}%
\pgfpathcurveto{\pgfqpoint{2.205965in}{1.640975in}}{\pgfqpoint{2.209238in}{1.633075in}}{\pgfqpoint{2.215062in}{1.627251in}}%
\pgfpathcurveto{\pgfqpoint{2.220885in}{1.621427in}}{\pgfqpoint{2.228786in}{1.618155in}}{\pgfqpoint{2.237022in}{1.618155in}}%
\pgfpathclose%
\pgfusepath{stroke,fill}%
\end{pgfscope}%
\begin{pgfscope}%
\pgfpathrectangle{\pgfqpoint{0.100000in}{0.220728in}}{\pgfqpoint{3.696000in}{3.696000in}}%
\pgfusepath{clip}%
\pgfsetbuttcap%
\pgfsetroundjoin%
\definecolor{currentfill}{rgb}{0.121569,0.466667,0.705882}%
\pgfsetfillcolor{currentfill}%
\pgfsetfillopacity{0.775697}%
\pgfsetlinewidth{1.003750pt}%
\definecolor{currentstroke}{rgb}{0.121569,0.466667,0.705882}%
\pgfsetstrokecolor{currentstroke}%
\pgfsetstrokeopacity{0.775697}%
\pgfsetdash{}{0pt}%
\pgfpathmoveto{\pgfqpoint{2.237494in}{1.617283in}}%
\pgfpathcurveto{\pgfqpoint{2.245730in}{1.617283in}}{\pgfqpoint{2.253630in}{1.620556in}}{\pgfqpoint{2.259454in}{1.626380in}}%
\pgfpathcurveto{\pgfqpoint{2.265278in}{1.632203in}}{\pgfqpoint{2.268550in}{1.640103in}}{\pgfqpoint{2.268550in}{1.648340in}}%
\pgfpathcurveto{\pgfqpoint{2.268550in}{1.656576in}}{\pgfqpoint{2.265278in}{1.664476in}}{\pgfqpoint{2.259454in}{1.670300in}}%
\pgfpathcurveto{\pgfqpoint{2.253630in}{1.676124in}}{\pgfqpoint{2.245730in}{1.679396in}}{\pgfqpoint{2.237494in}{1.679396in}}%
\pgfpathcurveto{\pgfqpoint{2.229258in}{1.679396in}}{\pgfqpoint{2.221357in}{1.676124in}}{\pgfqpoint{2.215534in}{1.670300in}}%
\pgfpathcurveto{\pgfqpoint{2.209710in}{1.664476in}}{\pgfqpoint{2.206437in}{1.656576in}}{\pgfqpoint{2.206437in}{1.648340in}}%
\pgfpathcurveto{\pgfqpoint{2.206437in}{1.640103in}}{\pgfqpoint{2.209710in}{1.632203in}}{\pgfqpoint{2.215534in}{1.626380in}}%
\pgfpathcurveto{\pgfqpoint{2.221357in}{1.620556in}}{\pgfqpoint{2.229258in}{1.617283in}}{\pgfqpoint{2.237494in}{1.617283in}}%
\pgfpathclose%
\pgfusepath{stroke,fill}%
\end{pgfscope}%
\begin{pgfscope}%
\pgfpathrectangle{\pgfqpoint{0.100000in}{0.220728in}}{\pgfqpoint{3.696000in}{3.696000in}}%
\pgfusepath{clip}%
\pgfsetbuttcap%
\pgfsetroundjoin%
\definecolor{currentfill}{rgb}{0.121569,0.466667,0.705882}%
\pgfsetfillcolor{currentfill}%
\pgfsetfillopacity{0.776582}%
\pgfsetlinewidth{1.003750pt}%
\definecolor{currentstroke}{rgb}{0.121569,0.466667,0.705882}%
\pgfsetstrokecolor{currentstroke}%
\pgfsetstrokeopacity{0.776582}%
\pgfsetdash{}{0pt}%
\pgfpathmoveto{\pgfqpoint{2.238420in}{1.617124in}}%
\pgfpathcurveto{\pgfqpoint{2.246657in}{1.617124in}}{\pgfqpoint{2.254557in}{1.620397in}}{\pgfqpoint{2.260381in}{1.626221in}}%
\pgfpathcurveto{\pgfqpoint{2.266205in}{1.632045in}}{\pgfqpoint{2.269477in}{1.639945in}}{\pgfqpoint{2.269477in}{1.648181in}}%
\pgfpathcurveto{\pgfqpoint{2.269477in}{1.656417in}}{\pgfqpoint{2.266205in}{1.664317in}}{\pgfqpoint{2.260381in}{1.670141in}}%
\pgfpathcurveto{\pgfqpoint{2.254557in}{1.675965in}}{\pgfqpoint{2.246657in}{1.679237in}}{\pgfqpoint{2.238420in}{1.679237in}}%
\pgfpathcurveto{\pgfqpoint{2.230184in}{1.679237in}}{\pgfqpoint{2.222284in}{1.675965in}}{\pgfqpoint{2.216460in}{1.670141in}}%
\pgfpathcurveto{\pgfqpoint{2.210636in}{1.664317in}}{\pgfqpoint{2.207364in}{1.656417in}}{\pgfqpoint{2.207364in}{1.648181in}}%
\pgfpathcurveto{\pgfqpoint{2.207364in}{1.639945in}}{\pgfqpoint{2.210636in}{1.632045in}}{\pgfqpoint{2.216460in}{1.626221in}}%
\pgfpathcurveto{\pgfqpoint{2.222284in}{1.620397in}}{\pgfqpoint{2.230184in}{1.617124in}}{\pgfqpoint{2.238420in}{1.617124in}}%
\pgfpathclose%
\pgfusepath{stroke,fill}%
\end{pgfscope}%
\begin{pgfscope}%
\pgfpathrectangle{\pgfqpoint{0.100000in}{0.220728in}}{\pgfqpoint{3.696000in}{3.696000in}}%
\pgfusepath{clip}%
\pgfsetbuttcap%
\pgfsetroundjoin%
\definecolor{currentfill}{rgb}{0.121569,0.466667,0.705882}%
\pgfsetfillcolor{currentfill}%
\pgfsetfillopacity{0.776989}%
\pgfsetlinewidth{1.003750pt}%
\definecolor{currentstroke}{rgb}{0.121569,0.466667,0.705882}%
\pgfsetstrokecolor{currentstroke}%
\pgfsetstrokeopacity{0.776989}%
\pgfsetdash{}{0pt}%
\pgfpathmoveto{\pgfqpoint{2.238742in}{1.616401in}}%
\pgfpathcurveto{\pgfqpoint{2.246979in}{1.616401in}}{\pgfqpoint{2.254879in}{1.619673in}}{\pgfqpoint{2.260703in}{1.625497in}}%
\pgfpathcurveto{\pgfqpoint{2.266526in}{1.631321in}}{\pgfqpoint{2.269799in}{1.639221in}}{\pgfqpoint{2.269799in}{1.647457in}}%
\pgfpathcurveto{\pgfqpoint{2.269799in}{1.655693in}}{\pgfqpoint{2.266526in}{1.663593in}}{\pgfqpoint{2.260703in}{1.669417in}}%
\pgfpathcurveto{\pgfqpoint{2.254879in}{1.675241in}}{\pgfqpoint{2.246979in}{1.678514in}}{\pgfqpoint{2.238742in}{1.678514in}}%
\pgfpathcurveto{\pgfqpoint{2.230506in}{1.678514in}}{\pgfqpoint{2.222606in}{1.675241in}}{\pgfqpoint{2.216782in}{1.669417in}}%
\pgfpathcurveto{\pgfqpoint{2.210958in}{1.663593in}}{\pgfqpoint{2.207686in}{1.655693in}}{\pgfqpoint{2.207686in}{1.647457in}}%
\pgfpathcurveto{\pgfqpoint{2.207686in}{1.639221in}}{\pgfqpoint{2.210958in}{1.631321in}}{\pgfqpoint{2.216782in}{1.625497in}}%
\pgfpathcurveto{\pgfqpoint{2.222606in}{1.619673in}}{\pgfqpoint{2.230506in}{1.616401in}}{\pgfqpoint{2.238742in}{1.616401in}}%
\pgfpathclose%
\pgfusepath{stroke,fill}%
\end{pgfscope}%
\begin{pgfscope}%
\pgfpathrectangle{\pgfqpoint{0.100000in}{0.220728in}}{\pgfqpoint{3.696000in}{3.696000in}}%
\pgfusepath{clip}%
\pgfsetbuttcap%
\pgfsetroundjoin%
\definecolor{currentfill}{rgb}{0.121569,0.466667,0.705882}%
\pgfsetfillcolor{currentfill}%
\pgfsetfillopacity{0.777868}%
\pgfsetlinewidth{1.003750pt}%
\definecolor{currentstroke}{rgb}{0.121569,0.466667,0.705882}%
\pgfsetstrokecolor{currentstroke}%
\pgfsetstrokeopacity{0.777868}%
\pgfsetdash{}{0pt}%
\pgfpathmoveto{\pgfqpoint{2.239690in}{1.613511in}}%
\pgfpathcurveto{\pgfqpoint{2.247926in}{1.613511in}}{\pgfqpoint{2.255826in}{1.616783in}}{\pgfqpoint{2.261650in}{1.622607in}}%
\pgfpathcurveto{\pgfqpoint{2.267474in}{1.628431in}}{\pgfqpoint{2.270746in}{1.636331in}}{\pgfqpoint{2.270746in}{1.644567in}}%
\pgfpathcurveto{\pgfqpoint{2.270746in}{1.652804in}}{\pgfqpoint{2.267474in}{1.660704in}}{\pgfqpoint{2.261650in}{1.666528in}}%
\pgfpathcurveto{\pgfqpoint{2.255826in}{1.672352in}}{\pgfqpoint{2.247926in}{1.675624in}}{\pgfqpoint{2.239690in}{1.675624in}}%
\pgfpathcurveto{\pgfqpoint{2.231453in}{1.675624in}}{\pgfqpoint{2.223553in}{1.672352in}}{\pgfqpoint{2.217729in}{1.666528in}}%
\pgfpathcurveto{\pgfqpoint{2.211905in}{1.660704in}}{\pgfqpoint{2.208633in}{1.652804in}}{\pgfqpoint{2.208633in}{1.644567in}}%
\pgfpathcurveto{\pgfqpoint{2.208633in}{1.636331in}}{\pgfqpoint{2.211905in}{1.628431in}}{\pgfqpoint{2.217729in}{1.622607in}}%
\pgfpathcurveto{\pgfqpoint{2.223553in}{1.616783in}}{\pgfqpoint{2.231453in}{1.613511in}}{\pgfqpoint{2.239690in}{1.613511in}}%
\pgfpathclose%
\pgfusepath{stroke,fill}%
\end{pgfscope}%
\begin{pgfscope}%
\pgfpathrectangle{\pgfqpoint{0.100000in}{0.220728in}}{\pgfqpoint{3.696000in}{3.696000in}}%
\pgfusepath{clip}%
\pgfsetbuttcap%
\pgfsetroundjoin%
\definecolor{currentfill}{rgb}{0.121569,0.466667,0.705882}%
\pgfsetfillcolor{currentfill}%
\pgfsetfillopacity{0.779172}%
\pgfsetlinewidth{1.003750pt}%
\definecolor{currentstroke}{rgb}{0.121569,0.466667,0.705882}%
\pgfsetstrokecolor{currentstroke}%
\pgfsetstrokeopacity{0.779172}%
\pgfsetdash{}{0pt}%
\pgfpathmoveto{\pgfqpoint{2.240744in}{1.611976in}}%
\pgfpathcurveto{\pgfqpoint{2.248980in}{1.611976in}}{\pgfqpoint{2.256880in}{1.615248in}}{\pgfqpoint{2.262704in}{1.621072in}}%
\pgfpathcurveto{\pgfqpoint{2.268528in}{1.626896in}}{\pgfqpoint{2.271801in}{1.634796in}}{\pgfqpoint{2.271801in}{1.643033in}}%
\pgfpathcurveto{\pgfqpoint{2.271801in}{1.651269in}}{\pgfqpoint{2.268528in}{1.659169in}}{\pgfqpoint{2.262704in}{1.664993in}}%
\pgfpathcurveto{\pgfqpoint{2.256880in}{1.670817in}}{\pgfqpoint{2.248980in}{1.674089in}}{\pgfqpoint{2.240744in}{1.674089in}}%
\pgfpathcurveto{\pgfqpoint{2.232508in}{1.674089in}}{\pgfqpoint{2.224608in}{1.670817in}}{\pgfqpoint{2.218784in}{1.664993in}}%
\pgfpathcurveto{\pgfqpoint{2.212960in}{1.659169in}}{\pgfqpoint{2.209688in}{1.651269in}}{\pgfqpoint{2.209688in}{1.643033in}}%
\pgfpathcurveto{\pgfqpoint{2.209688in}{1.634796in}}{\pgfqpoint{2.212960in}{1.626896in}}{\pgfqpoint{2.218784in}{1.621072in}}%
\pgfpathcurveto{\pgfqpoint{2.224608in}{1.615248in}}{\pgfqpoint{2.232508in}{1.611976in}}{\pgfqpoint{2.240744in}{1.611976in}}%
\pgfpathclose%
\pgfusepath{stroke,fill}%
\end{pgfscope}%
\begin{pgfscope}%
\pgfpathrectangle{\pgfqpoint{0.100000in}{0.220728in}}{\pgfqpoint{3.696000in}{3.696000in}}%
\pgfusepath{clip}%
\pgfsetbuttcap%
\pgfsetroundjoin%
\definecolor{currentfill}{rgb}{0.121569,0.466667,0.705882}%
\pgfsetfillcolor{currentfill}%
\pgfsetfillopacity{0.781119}%
\pgfsetlinewidth{1.003750pt}%
\definecolor{currentstroke}{rgb}{0.121569,0.466667,0.705882}%
\pgfsetstrokecolor{currentstroke}%
\pgfsetstrokeopacity{0.781119}%
\pgfsetdash{}{0pt}%
\pgfpathmoveto{\pgfqpoint{2.243162in}{1.610739in}}%
\pgfpathcurveto{\pgfqpoint{2.251399in}{1.610739in}}{\pgfqpoint{2.259299in}{1.614011in}}{\pgfqpoint{2.265123in}{1.619835in}}%
\pgfpathcurveto{\pgfqpoint{2.270947in}{1.625659in}}{\pgfqpoint{2.274219in}{1.633559in}}{\pgfqpoint{2.274219in}{1.641796in}}%
\pgfpathcurveto{\pgfqpoint{2.274219in}{1.650032in}}{\pgfqpoint{2.270947in}{1.657932in}}{\pgfqpoint{2.265123in}{1.663756in}}%
\pgfpathcurveto{\pgfqpoint{2.259299in}{1.669580in}}{\pgfqpoint{2.251399in}{1.672852in}}{\pgfqpoint{2.243162in}{1.672852in}}%
\pgfpathcurveto{\pgfqpoint{2.234926in}{1.672852in}}{\pgfqpoint{2.227026in}{1.669580in}}{\pgfqpoint{2.221202in}{1.663756in}}%
\pgfpathcurveto{\pgfqpoint{2.215378in}{1.657932in}}{\pgfqpoint{2.212106in}{1.650032in}}{\pgfqpoint{2.212106in}{1.641796in}}%
\pgfpathcurveto{\pgfqpoint{2.212106in}{1.633559in}}{\pgfqpoint{2.215378in}{1.625659in}}{\pgfqpoint{2.221202in}{1.619835in}}%
\pgfpathcurveto{\pgfqpoint{2.227026in}{1.614011in}}{\pgfqpoint{2.234926in}{1.610739in}}{\pgfqpoint{2.243162in}{1.610739in}}%
\pgfpathclose%
\pgfusepath{stroke,fill}%
\end{pgfscope}%
\begin{pgfscope}%
\pgfpathrectangle{\pgfqpoint{0.100000in}{0.220728in}}{\pgfqpoint{3.696000in}{3.696000in}}%
\pgfusepath{clip}%
\pgfsetbuttcap%
\pgfsetroundjoin%
\definecolor{currentfill}{rgb}{0.121569,0.466667,0.705882}%
\pgfsetfillcolor{currentfill}%
\pgfsetfillopacity{0.782136}%
\pgfsetlinewidth{1.003750pt}%
\definecolor{currentstroke}{rgb}{0.121569,0.466667,0.705882}%
\pgfsetstrokecolor{currentstroke}%
\pgfsetstrokeopacity{0.782136}%
\pgfsetdash{}{0pt}%
\pgfpathmoveto{\pgfqpoint{2.243441in}{1.609092in}}%
\pgfpathcurveto{\pgfqpoint{2.251678in}{1.609092in}}{\pgfqpoint{2.259578in}{1.612364in}}{\pgfqpoint{2.265402in}{1.618188in}}%
\pgfpathcurveto{\pgfqpoint{2.271226in}{1.624012in}}{\pgfqpoint{2.274498in}{1.631912in}}{\pgfqpoint{2.274498in}{1.640148in}}%
\pgfpathcurveto{\pgfqpoint{2.274498in}{1.648385in}}{\pgfqpoint{2.271226in}{1.656285in}}{\pgfqpoint{2.265402in}{1.662109in}}%
\pgfpathcurveto{\pgfqpoint{2.259578in}{1.667932in}}{\pgfqpoint{2.251678in}{1.671205in}}{\pgfqpoint{2.243441in}{1.671205in}}%
\pgfpathcurveto{\pgfqpoint{2.235205in}{1.671205in}}{\pgfqpoint{2.227305in}{1.667932in}}{\pgfqpoint{2.221481in}{1.662109in}}%
\pgfpathcurveto{\pgfqpoint{2.215657in}{1.656285in}}{\pgfqpoint{2.212385in}{1.648385in}}{\pgfqpoint{2.212385in}{1.640148in}}%
\pgfpathcurveto{\pgfqpoint{2.212385in}{1.631912in}}{\pgfqpoint{2.215657in}{1.624012in}}{\pgfqpoint{2.221481in}{1.618188in}}%
\pgfpathcurveto{\pgfqpoint{2.227305in}{1.612364in}}{\pgfqpoint{2.235205in}{1.609092in}}{\pgfqpoint{2.243441in}{1.609092in}}%
\pgfpathclose%
\pgfusepath{stroke,fill}%
\end{pgfscope}%
\begin{pgfscope}%
\pgfpathrectangle{\pgfqpoint{0.100000in}{0.220728in}}{\pgfqpoint{3.696000in}{3.696000in}}%
\pgfusepath{clip}%
\pgfsetbuttcap%
\pgfsetroundjoin%
\definecolor{currentfill}{rgb}{0.121569,0.466667,0.705882}%
\pgfsetfillcolor{currentfill}%
\pgfsetfillopacity{0.783312}%
\pgfsetlinewidth{1.003750pt}%
\definecolor{currentstroke}{rgb}{0.121569,0.466667,0.705882}%
\pgfsetstrokecolor{currentstroke}%
\pgfsetstrokeopacity{0.783312}%
\pgfsetdash{}{0pt}%
\pgfpathmoveto{\pgfqpoint{2.244831in}{1.603682in}}%
\pgfpathcurveto{\pgfqpoint{2.253067in}{1.603682in}}{\pgfqpoint{2.260967in}{1.606954in}}{\pgfqpoint{2.266791in}{1.612778in}}%
\pgfpathcurveto{\pgfqpoint{2.272615in}{1.618602in}}{\pgfqpoint{2.275887in}{1.626502in}}{\pgfqpoint{2.275887in}{1.634739in}}%
\pgfpathcurveto{\pgfqpoint{2.275887in}{1.642975in}}{\pgfqpoint{2.272615in}{1.650875in}}{\pgfqpoint{2.266791in}{1.656699in}}%
\pgfpathcurveto{\pgfqpoint{2.260967in}{1.662523in}}{\pgfqpoint{2.253067in}{1.665795in}}{\pgfqpoint{2.244831in}{1.665795in}}%
\pgfpathcurveto{\pgfqpoint{2.236594in}{1.665795in}}{\pgfqpoint{2.228694in}{1.662523in}}{\pgfqpoint{2.222871in}{1.656699in}}%
\pgfpathcurveto{\pgfqpoint{2.217047in}{1.650875in}}{\pgfqpoint{2.213774in}{1.642975in}}{\pgfqpoint{2.213774in}{1.634739in}}%
\pgfpathcurveto{\pgfqpoint{2.213774in}{1.626502in}}{\pgfqpoint{2.217047in}{1.618602in}}{\pgfqpoint{2.222871in}{1.612778in}}%
\pgfpathcurveto{\pgfqpoint{2.228694in}{1.606954in}}{\pgfqpoint{2.236594in}{1.603682in}}{\pgfqpoint{2.244831in}{1.603682in}}%
\pgfpathclose%
\pgfusepath{stroke,fill}%
\end{pgfscope}%
\begin{pgfscope}%
\pgfpathrectangle{\pgfqpoint{0.100000in}{0.220728in}}{\pgfqpoint{3.696000in}{3.696000in}}%
\pgfusepath{clip}%
\pgfsetbuttcap%
\pgfsetroundjoin%
\definecolor{currentfill}{rgb}{0.121569,0.466667,0.705882}%
\pgfsetfillcolor{currentfill}%
\pgfsetfillopacity{0.784307}%
\pgfsetlinewidth{1.003750pt}%
\definecolor{currentstroke}{rgb}{0.121569,0.466667,0.705882}%
\pgfsetstrokecolor{currentstroke}%
\pgfsetstrokeopacity{0.784307}%
\pgfsetdash{}{0pt}%
\pgfpathmoveto{\pgfqpoint{2.245780in}{1.603007in}}%
\pgfpathcurveto{\pgfqpoint{2.254017in}{1.603007in}}{\pgfqpoint{2.261917in}{1.606279in}}{\pgfqpoint{2.267740in}{1.612103in}}%
\pgfpathcurveto{\pgfqpoint{2.273564in}{1.617927in}}{\pgfqpoint{2.276837in}{1.625827in}}{\pgfqpoint{2.276837in}{1.634063in}}%
\pgfpathcurveto{\pgfqpoint{2.276837in}{1.642300in}}{\pgfqpoint{2.273564in}{1.650200in}}{\pgfqpoint{2.267740in}{1.656024in}}%
\pgfpathcurveto{\pgfqpoint{2.261917in}{1.661847in}}{\pgfqpoint{2.254017in}{1.665120in}}{\pgfqpoint{2.245780in}{1.665120in}}%
\pgfpathcurveto{\pgfqpoint{2.237544in}{1.665120in}}{\pgfqpoint{2.229644in}{1.661847in}}{\pgfqpoint{2.223820in}{1.656024in}}%
\pgfpathcurveto{\pgfqpoint{2.217996in}{1.650200in}}{\pgfqpoint{2.214724in}{1.642300in}}{\pgfqpoint{2.214724in}{1.634063in}}%
\pgfpathcurveto{\pgfqpoint{2.214724in}{1.625827in}}{\pgfqpoint{2.217996in}{1.617927in}}{\pgfqpoint{2.223820in}{1.612103in}}%
\pgfpathcurveto{\pgfqpoint{2.229644in}{1.606279in}}{\pgfqpoint{2.237544in}{1.603007in}}{\pgfqpoint{2.245780in}{1.603007in}}%
\pgfpathclose%
\pgfusepath{stroke,fill}%
\end{pgfscope}%
\begin{pgfscope}%
\pgfpathrectangle{\pgfqpoint{0.100000in}{0.220728in}}{\pgfqpoint{3.696000in}{3.696000in}}%
\pgfusepath{clip}%
\pgfsetbuttcap%
\pgfsetroundjoin%
\definecolor{currentfill}{rgb}{0.121569,0.466667,0.705882}%
\pgfsetfillcolor{currentfill}%
\pgfsetfillopacity{0.785557}%
\pgfsetlinewidth{1.003750pt}%
\definecolor{currentstroke}{rgb}{0.121569,0.466667,0.705882}%
\pgfsetstrokecolor{currentstroke}%
\pgfsetstrokeopacity{0.785557}%
\pgfsetdash{}{0pt}%
\pgfpathmoveto{\pgfqpoint{2.246790in}{1.602219in}}%
\pgfpathcurveto{\pgfqpoint{2.255026in}{1.602219in}}{\pgfqpoint{2.262926in}{1.605491in}}{\pgfqpoint{2.268750in}{1.611315in}}%
\pgfpathcurveto{\pgfqpoint{2.274574in}{1.617139in}}{\pgfqpoint{2.277846in}{1.625039in}}{\pgfqpoint{2.277846in}{1.633275in}}%
\pgfpathcurveto{\pgfqpoint{2.277846in}{1.641512in}}{\pgfqpoint{2.274574in}{1.649412in}}{\pgfqpoint{2.268750in}{1.655236in}}%
\pgfpathcurveto{\pgfqpoint{2.262926in}{1.661060in}}{\pgfqpoint{2.255026in}{1.664332in}}{\pgfqpoint{2.246790in}{1.664332in}}%
\pgfpathcurveto{\pgfqpoint{2.238553in}{1.664332in}}{\pgfqpoint{2.230653in}{1.661060in}}{\pgfqpoint{2.224829in}{1.655236in}}%
\pgfpathcurveto{\pgfqpoint{2.219005in}{1.649412in}}{\pgfqpoint{2.215733in}{1.641512in}}{\pgfqpoint{2.215733in}{1.633275in}}%
\pgfpathcurveto{\pgfqpoint{2.215733in}{1.625039in}}{\pgfqpoint{2.219005in}{1.617139in}}{\pgfqpoint{2.224829in}{1.611315in}}%
\pgfpathcurveto{\pgfqpoint{2.230653in}{1.605491in}}{\pgfqpoint{2.238553in}{1.602219in}}{\pgfqpoint{2.246790in}{1.602219in}}%
\pgfpathclose%
\pgfusepath{stroke,fill}%
\end{pgfscope}%
\begin{pgfscope}%
\pgfpathrectangle{\pgfqpoint{0.100000in}{0.220728in}}{\pgfqpoint{3.696000in}{3.696000in}}%
\pgfusepath{clip}%
\pgfsetbuttcap%
\pgfsetroundjoin%
\definecolor{currentfill}{rgb}{0.121569,0.466667,0.705882}%
\pgfsetfillcolor{currentfill}%
\pgfsetfillopacity{0.786220}%
\pgfsetlinewidth{1.003750pt}%
\definecolor{currentstroke}{rgb}{0.121569,0.466667,0.705882}%
\pgfsetstrokecolor{currentstroke}%
\pgfsetstrokeopacity{0.786220}%
\pgfsetdash{}{0pt}%
\pgfpathmoveto{\pgfqpoint{2.247088in}{1.601494in}}%
\pgfpathcurveto{\pgfqpoint{2.255324in}{1.601494in}}{\pgfqpoint{2.263224in}{1.604767in}}{\pgfqpoint{2.269048in}{1.610591in}}%
\pgfpathcurveto{\pgfqpoint{2.274872in}{1.616415in}}{\pgfqpoint{2.278144in}{1.624315in}}{\pgfqpoint{2.278144in}{1.632551in}}%
\pgfpathcurveto{\pgfqpoint{2.278144in}{1.640787in}}{\pgfqpoint{2.274872in}{1.648687in}}{\pgfqpoint{2.269048in}{1.654511in}}%
\pgfpathcurveto{\pgfqpoint{2.263224in}{1.660335in}}{\pgfqpoint{2.255324in}{1.663607in}}{\pgfqpoint{2.247088in}{1.663607in}}%
\pgfpathcurveto{\pgfqpoint{2.238851in}{1.663607in}}{\pgfqpoint{2.230951in}{1.660335in}}{\pgfqpoint{2.225127in}{1.654511in}}%
\pgfpathcurveto{\pgfqpoint{2.219304in}{1.648687in}}{\pgfqpoint{2.216031in}{1.640787in}}{\pgfqpoint{2.216031in}{1.632551in}}%
\pgfpathcurveto{\pgfqpoint{2.216031in}{1.624315in}}{\pgfqpoint{2.219304in}{1.616415in}}{\pgfqpoint{2.225127in}{1.610591in}}%
\pgfpathcurveto{\pgfqpoint{2.230951in}{1.604767in}}{\pgfqpoint{2.238851in}{1.601494in}}{\pgfqpoint{2.247088in}{1.601494in}}%
\pgfpathclose%
\pgfusepath{stroke,fill}%
\end{pgfscope}%
\begin{pgfscope}%
\pgfpathrectangle{\pgfqpoint{0.100000in}{0.220728in}}{\pgfqpoint{3.696000in}{3.696000in}}%
\pgfusepath{clip}%
\pgfsetbuttcap%
\pgfsetroundjoin%
\definecolor{currentfill}{rgb}{0.121569,0.466667,0.705882}%
\pgfsetfillcolor{currentfill}%
\pgfsetfillopacity{0.787132}%
\pgfsetlinewidth{1.003750pt}%
\definecolor{currentstroke}{rgb}{0.121569,0.466667,0.705882}%
\pgfsetstrokecolor{currentstroke}%
\pgfsetstrokeopacity{0.787132}%
\pgfsetdash{}{0pt}%
\pgfpathmoveto{\pgfqpoint{2.248118in}{1.598598in}}%
\pgfpathcurveto{\pgfqpoint{2.256355in}{1.598598in}}{\pgfqpoint{2.264255in}{1.601870in}}{\pgfqpoint{2.270079in}{1.607694in}}%
\pgfpathcurveto{\pgfqpoint{2.275903in}{1.613518in}}{\pgfqpoint{2.279175in}{1.621418in}}{\pgfqpoint{2.279175in}{1.629655in}}%
\pgfpathcurveto{\pgfqpoint{2.279175in}{1.637891in}}{\pgfqpoint{2.275903in}{1.645791in}}{\pgfqpoint{2.270079in}{1.651615in}}%
\pgfpathcurveto{\pgfqpoint{2.264255in}{1.657439in}}{\pgfqpoint{2.256355in}{1.660711in}}{\pgfqpoint{2.248118in}{1.660711in}}%
\pgfpathcurveto{\pgfqpoint{2.239882in}{1.660711in}}{\pgfqpoint{2.231982in}{1.657439in}}{\pgfqpoint{2.226158in}{1.651615in}}%
\pgfpathcurveto{\pgfqpoint{2.220334in}{1.645791in}}{\pgfqpoint{2.217062in}{1.637891in}}{\pgfqpoint{2.217062in}{1.629655in}}%
\pgfpathcurveto{\pgfqpoint{2.217062in}{1.621418in}}{\pgfqpoint{2.220334in}{1.613518in}}{\pgfqpoint{2.226158in}{1.607694in}}%
\pgfpathcurveto{\pgfqpoint{2.231982in}{1.601870in}}{\pgfqpoint{2.239882in}{1.598598in}}{\pgfqpoint{2.248118in}{1.598598in}}%
\pgfpathclose%
\pgfusepath{stroke,fill}%
\end{pgfscope}%
\begin{pgfscope}%
\pgfpathrectangle{\pgfqpoint{0.100000in}{0.220728in}}{\pgfqpoint{3.696000in}{3.696000in}}%
\pgfusepath{clip}%
\pgfsetbuttcap%
\pgfsetroundjoin%
\definecolor{currentfill}{rgb}{0.121569,0.466667,0.705882}%
\pgfsetfillcolor{currentfill}%
\pgfsetfillopacity{0.787766}%
\pgfsetlinewidth{1.003750pt}%
\definecolor{currentstroke}{rgb}{0.121569,0.466667,0.705882}%
\pgfsetstrokecolor{currentstroke}%
\pgfsetstrokeopacity{0.787766}%
\pgfsetdash{}{0pt}%
\pgfpathmoveto{\pgfqpoint{2.248782in}{1.597902in}}%
\pgfpathcurveto{\pgfqpoint{2.257018in}{1.597902in}}{\pgfqpoint{2.264918in}{1.601174in}}{\pgfqpoint{2.270742in}{1.606998in}}%
\pgfpathcurveto{\pgfqpoint{2.276566in}{1.612822in}}{\pgfqpoint{2.279838in}{1.620722in}}{\pgfqpoint{2.279838in}{1.628958in}}%
\pgfpathcurveto{\pgfqpoint{2.279838in}{1.637194in}}{\pgfqpoint{2.276566in}{1.645094in}}{\pgfqpoint{2.270742in}{1.650918in}}%
\pgfpathcurveto{\pgfqpoint{2.264918in}{1.656742in}}{\pgfqpoint{2.257018in}{1.660015in}}{\pgfqpoint{2.248782in}{1.660015in}}%
\pgfpathcurveto{\pgfqpoint{2.240545in}{1.660015in}}{\pgfqpoint{2.232645in}{1.656742in}}{\pgfqpoint{2.226821in}{1.650918in}}%
\pgfpathcurveto{\pgfqpoint{2.220998in}{1.645094in}}{\pgfqpoint{2.217725in}{1.637194in}}{\pgfqpoint{2.217725in}{1.628958in}}%
\pgfpathcurveto{\pgfqpoint{2.217725in}{1.620722in}}{\pgfqpoint{2.220998in}{1.612822in}}{\pgfqpoint{2.226821in}{1.606998in}}%
\pgfpathcurveto{\pgfqpoint{2.232645in}{1.601174in}}{\pgfqpoint{2.240545in}{1.597902in}}{\pgfqpoint{2.248782in}{1.597902in}}%
\pgfpathclose%
\pgfusepath{stroke,fill}%
\end{pgfscope}%
\begin{pgfscope}%
\pgfpathrectangle{\pgfqpoint{0.100000in}{0.220728in}}{\pgfqpoint{3.696000in}{3.696000in}}%
\pgfusepath{clip}%
\pgfsetbuttcap%
\pgfsetroundjoin%
\definecolor{currentfill}{rgb}{0.121569,0.466667,0.705882}%
\pgfsetfillcolor{currentfill}%
\pgfsetfillopacity{0.788699}%
\pgfsetlinewidth{1.003750pt}%
\definecolor{currentstroke}{rgb}{0.121569,0.466667,0.705882}%
\pgfsetstrokecolor{currentstroke}%
\pgfsetstrokeopacity{0.788699}%
\pgfsetdash{}{0pt}%
\pgfpathmoveto{\pgfqpoint{2.249587in}{1.597787in}}%
\pgfpathcurveto{\pgfqpoint{2.257823in}{1.597787in}}{\pgfqpoint{2.265723in}{1.601059in}}{\pgfqpoint{2.271547in}{1.606883in}}%
\pgfpathcurveto{\pgfqpoint{2.277371in}{1.612707in}}{\pgfqpoint{2.280644in}{1.620607in}}{\pgfqpoint{2.280644in}{1.628843in}}%
\pgfpathcurveto{\pgfqpoint{2.280644in}{1.637080in}}{\pgfqpoint{2.277371in}{1.644980in}}{\pgfqpoint{2.271547in}{1.650804in}}%
\pgfpathcurveto{\pgfqpoint{2.265723in}{1.656628in}}{\pgfqpoint{2.257823in}{1.659900in}}{\pgfqpoint{2.249587in}{1.659900in}}%
\pgfpathcurveto{\pgfqpoint{2.241351in}{1.659900in}}{\pgfqpoint{2.233451in}{1.656628in}}{\pgfqpoint{2.227627in}{1.650804in}}%
\pgfpathcurveto{\pgfqpoint{2.221803in}{1.644980in}}{\pgfqpoint{2.218531in}{1.637080in}}{\pgfqpoint{2.218531in}{1.628843in}}%
\pgfpathcurveto{\pgfqpoint{2.218531in}{1.620607in}}{\pgfqpoint{2.221803in}{1.612707in}}{\pgfqpoint{2.227627in}{1.606883in}}%
\pgfpathcurveto{\pgfqpoint{2.233451in}{1.601059in}}{\pgfqpoint{2.241351in}{1.597787in}}{\pgfqpoint{2.249587in}{1.597787in}}%
\pgfpathclose%
\pgfusepath{stroke,fill}%
\end{pgfscope}%
\begin{pgfscope}%
\pgfpathrectangle{\pgfqpoint{0.100000in}{0.220728in}}{\pgfqpoint{3.696000in}{3.696000in}}%
\pgfusepath{clip}%
\pgfsetbuttcap%
\pgfsetroundjoin%
\definecolor{currentfill}{rgb}{0.121569,0.466667,0.705882}%
\pgfsetfillcolor{currentfill}%
\pgfsetfillopacity{0.789143}%
\pgfsetlinewidth{1.003750pt}%
\definecolor{currentstroke}{rgb}{0.121569,0.466667,0.705882}%
\pgfsetstrokecolor{currentstroke}%
\pgfsetstrokeopacity{0.789143}%
\pgfsetdash{}{0pt}%
\pgfpathmoveto{\pgfqpoint{2.249830in}{1.597176in}}%
\pgfpathcurveto{\pgfqpoint{2.258067in}{1.597176in}}{\pgfqpoint{2.265967in}{1.600449in}}{\pgfqpoint{2.271791in}{1.606273in}}%
\pgfpathcurveto{\pgfqpoint{2.277615in}{1.612097in}}{\pgfqpoint{2.280887in}{1.619997in}}{\pgfqpoint{2.280887in}{1.628233in}}%
\pgfpathcurveto{\pgfqpoint{2.280887in}{1.636469in}}{\pgfqpoint{2.277615in}{1.644369in}}{\pgfqpoint{2.271791in}{1.650193in}}%
\pgfpathcurveto{\pgfqpoint{2.265967in}{1.656017in}}{\pgfqpoint{2.258067in}{1.659289in}}{\pgfqpoint{2.249830in}{1.659289in}}%
\pgfpathcurveto{\pgfqpoint{2.241594in}{1.659289in}}{\pgfqpoint{2.233694in}{1.656017in}}{\pgfqpoint{2.227870in}{1.650193in}}%
\pgfpathcurveto{\pgfqpoint{2.222046in}{1.644369in}}{\pgfqpoint{2.218774in}{1.636469in}}{\pgfqpoint{2.218774in}{1.628233in}}%
\pgfpathcurveto{\pgfqpoint{2.218774in}{1.619997in}}{\pgfqpoint{2.222046in}{1.612097in}}{\pgfqpoint{2.227870in}{1.606273in}}%
\pgfpathcurveto{\pgfqpoint{2.233694in}{1.600449in}}{\pgfqpoint{2.241594in}{1.597176in}}{\pgfqpoint{2.249830in}{1.597176in}}%
\pgfpathclose%
\pgfusepath{stroke,fill}%
\end{pgfscope}%
\begin{pgfscope}%
\pgfpathrectangle{\pgfqpoint{0.100000in}{0.220728in}}{\pgfqpoint{3.696000in}{3.696000in}}%
\pgfusepath{clip}%
\pgfsetbuttcap%
\pgfsetroundjoin%
\definecolor{currentfill}{rgb}{0.121569,0.466667,0.705882}%
\pgfsetfillcolor{currentfill}%
\pgfsetfillopacity{0.789873}%
\pgfsetlinewidth{1.003750pt}%
\definecolor{currentstroke}{rgb}{0.121569,0.466667,0.705882}%
\pgfsetstrokecolor{currentstroke}%
\pgfsetstrokeopacity{0.789873}%
\pgfsetdash{}{0pt}%
\pgfpathmoveto{\pgfqpoint{2.250458in}{1.596152in}}%
\pgfpathcurveto{\pgfqpoint{2.258694in}{1.596152in}}{\pgfqpoint{2.266594in}{1.599424in}}{\pgfqpoint{2.272418in}{1.605248in}}%
\pgfpathcurveto{\pgfqpoint{2.278242in}{1.611072in}}{\pgfqpoint{2.281515in}{1.618972in}}{\pgfqpoint{2.281515in}{1.627208in}}%
\pgfpathcurveto{\pgfqpoint{2.281515in}{1.635445in}}{\pgfqpoint{2.278242in}{1.643345in}}{\pgfqpoint{2.272418in}{1.649169in}}%
\pgfpathcurveto{\pgfqpoint{2.266594in}{1.654993in}}{\pgfqpoint{2.258694in}{1.658265in}}{\pgfqpoint{2.250458in}{1.658265in}}%
\pgfpathcurveto{\pgfqpoint{2.242222in}{1.658265in}}{\pgfqpoint{2.234322in}{1.654993in}}{\pgfqpoint{2.228498in}{1.649169in}}%
\pgfpathcurveto{\pgfqpoint{2.222674in}{1.643345in}}{\pgfqpoint{2.219402in}{1.635445in}}{\pgfqpoint{2.219402in}{1.627208in}}%
\pgfpathcurveto{\pgfqpoint{2.219402in}{1.618972in}}{\pgfqpoint{2.222674in}{1.611072in}}{\pgfqpoint{2.228498in}{1.605248in}}%
\pgfpathcurveto{\pgfqpoint{2.234322in}{1.599424in}}{\pgfqpoint{2.242222in}{1.596152in}}{\pgfqpoint{2.250458in}{1.596152in}}%
\pgfpathclose%
\pgfusepath{stroke,fill}%
\end{pgfscope}%
\begin{pgfscope}%
\pgfpathrectangle{\pgfqpoint{0.100000in}{0.220728in}}{\pgfqpoint{3.696000in}{3.696000in}}%
\pgfusepath{clip}%
\pgfsetbuttcap%
\pgfsetroundjoin%
\definecolor{currentfill}{rgb}{0.121569,0.466667,0.705882}%
\pgfsetfillcolor{currentfill}%
\pgfsetfillopacity{0.790464}%
\pgfsetlinewidth{1.003750pt}%
\definecolor{currentstroke}{rgb}{0.121569,0.466667,0.705882}%
\pgfsetstrokecolor{currentstroke}%
\pgfsetstrokeopacity{0.790464}%
\pgfsetdash{}{0pt}%
\pgfpathmoveto{\pgfqpoint{2.251459in}{1.593318in}}%
\pgfpathcurveto{\pgfqpoint{2.259695in}{1.593318in}}{\pgfqpoint{2.267595in}{1.596591in}}{\pgfqpoint{2.273419in}{1.602415in}}%
\pgfpathcurveto{\pgfqpoint{2.279243in}{1.608238in}}{\pgfqpoint{2.282515in}{1.616138in}}{\pgfqpoint{2.282515in}{1.624375in}}%
\pgfpathcurveto{\pgfqpoint{2.282515in}{1.632611in}}{\pgfqpoint{2.279243in}{1.640511in}}{\pgfqpoint{2.273419in}{1.646335in}}%
\pgfpathcurveto{\pgfqpoint{2.267595in}{1.652159in}}{\pgfqpoint{2.259695in}{1.655431in}}{\pgfqpoint{2.251459in}{1.655431in}}%
\pgfpathcurveto{\pgfqpoint{2.243222in}{1.655431in}}{\pgfqpoint{2.235322in}{1.652159in}}{\pgfqpoint{2.229498in}{1.646335in}}%
\pgfpathcurveto{\pgfqpoint{2.223675in}{1.640511in}}{\pgfqpoint{2.220402in}{1.632611in}}{\pgfqpoint{2.220402in}{1.624375in}}%
\pgfpathcurveto{\pgfqpoint{2.220402in}{1.616138in}}{\pgfqpoint{2.223675in}{1.608238in}}{\pgfqpoint{2.229498in}{1.602415in}}%
\pgfpathcurveto{\pgfqpoint{2.235322in}{1.596591in}}{\pgfqpoint{2.243222in}{1.593318in}}{\pgfqpoint{2.251459in}{1.593318in}}%
\pgfpathclose%
\pgfusepath{stroke,fill}%
\end{pgfscope}%
\begin{pgfscope}%
\pgfpathrectangle{\pgfqpoint{0.100000in}{0.220728in}}{\pgfqpoint{3.696000in}{3.696000in}}%
\pgfusepath{clip}%
\pgfsetbuttcap%
\pgfsetroundjoin%
\definecolor{currentfill}{rgb}{0.121569,0.466667,0.705882}%
\pgfsetfillcolor{currentfill}%
\pgfsetfillopacity{0.790992}%
\pgfsetlinewidth{1.003750pt}%
\definecolor{currentstroke}{rgb}{0.121569,0.466667,0.705882}%
\pgfsetstrokecolor{currentstroke}%
\pgfsetstrokeopacity{0.790992}%
\pgfsetdash{}{0pt}%
\pgfpathmoveto{\pgfqpoint{2.252061in}{1.593069in}}%
\pgfpathcurveto{\pgfqpoint{2.260298in}{1.593069in}}{\pgfqpoint{2.268198in}{1.596341in}}{\pgfqpoint{2.274022in}{1.602165in}}%
\pgfpathcurveto{\pgfqpoint{2.279846in}{1.607989in}}{\pgfqpoint{2.283118in}{1.615889in}}{\pgfqpoint{2.283118in}{1.624126in}}%
\pgfpathcurveto{\pgfqpoint{2.283118in}{1.632362in}}{\pgfqpoint{2.279846in}{1.640262in}}{\pgfqpoint{2.274022in}{1.646086in}}%
\pgfpathcurveto{\pgfqpoint{2.268198in}{1.651910in}}{\pgfqpoint{2.260298in}{1.655182in}}{\pgfqpoint{2.252061in}{1.655182in}}%
\pgfpathcurveto{\pgfqpoint{2.243825in}{1.655182in}}{\pgfqpoint{2.235925in}{1.651910in}}{\pgfqpoint{2.230101in}{1.646086in}}%
\pgfpathcurveto{\pgfqpoint{2.224277in}{1.640262in}}{\pgfqpoint{2.221005in}{1.632362in}}{\pgfqpoint{2.221005in}{1.624126in}}%
\pgfpathcurveto{\pgfqpoint{2.221005in}{1.615889in}}{\pgfqpoint{2.224277in}{1.607989in}}{\pgfqpoint{2.230101in}{1.602165in}}%
\pgfpathcurveto{\pgfqpoint{2.235925in}{1.596341in}}{\pgfqpoint{2.243825in}{1.593069in}}{\pgfqpoint{2.252061in}{1.593069in}}%
\pgfpathclose%
\pgfusepath{stroke,fill}%
\end{pgfscope}%
\begin{pgfscope}%
\pgfpathrectangle{\pgfqpoint{0.100000in}{0.220728in}}{\pgfqpoint{3.696000in}{3.696000in}}%
\pgfusepath{clip}%
\pgfsetbuttcap%
\pgfsetroundjoin%
\definecolor{currentfill}{rgb}{0.121569,0.466667,0.705882}%
\pgfsetfillcolor{currentfill}%
\pgfsetfillopacity{0.791790}%
\pgfsetlinewidth{1.003750pt}%
\definecolor{currentstroke}{rgb}{0.121569,0.466667,0.705882}%
\pgfsetstrokecolor{currentstroke}%
\pgfsetstrokeopacity{0.791790}%
\pgfsetdash{}{0pt}%
\pgfpathmoveto{\pgfqpoint{2.252438in}{1.592658in}}%
\pgfpathcurveto{\pgfqpoint{2.260674in}{1.592658in}}{\pgfqpoint{2.268575in}{1.595930in}}{\pgfqpoint{2.274398in}{1.601754in}}%
\pgfpathcurveto{\pgfqpoint{2.280222in}{1.607578in}}{\pgfqpoint{2.283495in}{1.615478in}}{\pgfqpoint{2.283495in}{1.623714in}}%
\pgfpathcurveto{\pgfqpoint{2.283495in}{1.631951in}}{\pgfqpoint{2.280222in}{1.639851in}}{\pgfqpoint{2.274398in}{1.645675in}}%
\pgfpathcurveto{\pgfqpoint{2.268575in}{1.651498in}}{\pgfqpoint{2.260674in}{1.654771in}}{\pgfqpoint{2.252438in}{1.654771in}}%
\pgfpathcurveto{\pgfqpoint{2.244202in}{1.654771in}}{\pgfqpoint{2.236302in}{1.651498in}}{\pgfqpoint{2.230478in}{1.645675in}}%
\pgfpathcurveto{\pgfqpoint{2.224654in}{1.639851in}}{\pgfqpoint{2.221382in}{1.631951in}}{\pgfqpoint{2.221382in}{1.623714in}}%
\pgfpathcurveto{\pgfqpoint{2.221382in}{1.615478in}}{\pgfqpoint{2.224654in}{1.607578in}}{\pgfqpoint{2.230478in}{1.601754in}}%
\pgfpathcurveto{\pgfqpoint{2.236302in}{1.595930in}}{\pgfqpoint{2.244202in}{1.592658in}}{\pgfqpoint{2.252438in}{1.592658in}}%
\pgfpathclose%
\pgfusepath{stroke,fill}%
\end{pgfscope}%
\begin{pgfscope}%
\pgfpathrectangle{\pgfqpoint{0.100000in}{0.220728in}}{\pgfqpoint{3.696000in}{3.696000in}}%
\pgfusepath{clip}%
\pgfsetbuttcap%
\pgfsetroundjoin%
\definecolor{currentfill}{rgb}{0.121569,0.466667,0.705882}%
\pgfsetfillcolor{currentfill}%
\pgfsetfillopacity{0.792933}%
\pgfsetlinewidth{1.003750pt}%
\definecolor{currentstroke}{rgb}{0.121569,0.466667,0.705882}%
\pgfsetstrokecolor{currentstroke}%
\pgfsetstrokeopacity{0.792933}%
\pgfsetdash{}{0pt}%
\pgfpathmoveto{\pgfqpoint{2.253016in}{1.591100in}}%
\pgfpathcurveto{\pgfqpoint{2.261252in}{1.591100in}}{\pgfqpoint{2.269152in}{1.594372in}}{\pgfqpoint{2.274976in}{1.600196in}}%
\pgfpathcurveto{\pgfqpoint{2.280800in}{1.606020in}}{\pgfqpoint{2.284072in}{1.613920in}}{\pgfqpoint{2.284072in}{1.622156in}}%
\pgfpathcurveto{\pgfqpoint{2.284072in}{1.630392in}}{\pgfqpoint{2.280800in}{1.638292in}}{\pgfqpoint{2.274976in}{1.644116in}}%
\pgfpathcurveto{\pgfqpoint{2.269152in}{1.649940in}}{\pgfqpoint{2.261252in}{1.653213in}}{\pgfqpoint{2.253016in}{1.653213in}}%
\pgfpathcurveto{\pgfqpoint{2.244780in}{1.653213in}}{\pgfqpoint{2.236879in}{1.649940in}}{\pgfqpoint{2.231056in}{1.644116in}}%
\pgfpathcurveto{\pgfqpoint{2.225232in}{1.638292in}}{\pgfqpoint{2.221959in}{1.630392in}}{\pgfqpoint{2.221959in}{1.622156in}}%
\pgfpathcurveto{\pgfqpoint{2.221959in}{1.613920in}}{\pgfqpoint{2.225232in}{1.606020in}}{\pgfqpoint{2.231056in}{1.600196in}}%
\pgfpathcurveto{\pgfqpoint{2.236879in}{1.594372in}}{\pgfqpoint{2.244780in}{1.591100in}}{\pgfqpoint{2.253016in}{1.591100in}}%
\pgfpathclose%
\pgfusepath{stroke,fill}%
\end{pgfscope}%
\begin{pgfscope}%
\pgfpathrectangle{\pgfqpoint{0.100000in}{0.220728in}}{\pgfqpoint{3.696000in}{3.696000in}}%
\pgfusepath{clip}%
\pgfsetbuttcap%
\pgfsetroundjoin%
\definecolor{currentfill}{rgb}{0.121569,0.466667,0.705882}%
\pgfsetfillcolor{currentfill}%
\pgfsetfillopacity{0.794606}%
\pgfsetlinewidth{1.003750pt}%
\definecolor{currentstroke}{rgb}{0.121569,0.466667,0.705882}%
\pgfsetstrokecolor{currentstroke}%
\pgfsetstrokeopacity{0.794606}%
\pgfsetdash{}{0pt}%
\pgfpathmoveto{\pgfqpoint{2.254201in}{1.587217in}}%
\pgfpathcurveto{\pgfqpoint{2.262437in}{1.587217in}}{\pgfqpoint{2.270337in}{1.590490in}}{\pgfqpoint{2.276161in}{1.596313in}}%
\pgfpathcurveto{\pgfqpoint{2.281985in}{1.602137in}}{\pgfqpoint{2.285257in}{1.610037in}}{\pgfqpoint{2.285257in}{1.618274in}}%
\pgfpathcurveto{\pgfqpoint{2.285257in}{1.626510in}}{\pgfqpoint{2.281985in}{1.634410in}}{\pgfqpoint{2.276161in}{1.640234in}}%
\pgfpathcurveto{\pgfqpoint{2.270337in}{1.646058in}}{\pgfqpoint{2.262437in}{1.649330in}}{\pgfqpoint{2.254201in}{1.649330in}}%
\pgfpathcurveto{\pgfqpoint{2.245965in}{1.649330in}}{\pgfqpoint{2.238065in}{1.646058in}}{\pgfqpoint{2.232241in}{1.640234in}}%
\pgfpathcurveto{\pgfqpoint{2.226417in}{1.634410in}}{\pgfqpoint{2.223144in}{1.626510in}}{\pgfqpoint{2.223144in}{1.618274in}}%
\pgfpathcurveto{\pgfqpoint{2.223144in}{1.610037in}}{\pgfqpoint{2.226417in}{1.602137in}}{\pgfqpoint{2.232241in}{1.596313in}}%
\pgfpathcurveto{\pgfqpoint{2.238065in}{1.590490in}}{\pgfqpoint{2.245965in}{1.587217in}}{\pgfqpoint{2.254201in}{1.587217in}}%
\pgfpathclose%
\pgfusepath{stroke,fill}%
\end{pgfscope}%
\begin{pgfscope}%
\pgfpathrectangle{\pgfqpoint{0.100000in}{0.220728in}}{\pgfqpoint{3.696000in}{3.696000in}}%
\pgfusepath{clip}%
\pgfsetbuttcap%
\pgfsetroundjoin%
\definecolor{currentfill}{rgb}{0.121569,0.466667,0.705882}%
\pgfsetfillcolor{currentfill}%
\pgfsetfillopacity{0.795800}%
\pgfsetlinewidth{1.003750pt}%
\definecolor{currentstroke}{rgb}{0.121569,0.466667,0.705882}%
\pgfsetstrokecolor{currentstroke}%
\pgfsetstrokeopacity{0.795800}%
\pgfsetdash{}{0pt}%
\pgfpathmoveto{\pgfqpoint{2.255289in}{1.587059in}}%
\pgfpathcurveto{\pgfqpoint{2.263525in}{1.587059in}}{\pgfqpoint{2.271425in}{1.590331in}}{\pgfqpoint{2.277249in}{1.596155in}}%
\pgfpathcurveto{\pgfqpoint{2.283073in}{1.601979in}}{\pgfqpoint{2.286345in}{1.609879in}}{\pgfqpoint{2.286345in}{1.618115in}}%
\pgfpathcurveto{\pgfqpoint{2.286345in}{1.626352in}}{\pgfqpoint{2.283073in}{1.634252in}}{\pgfqpoint{2.277249in}{1.640076in}}%
\pgfpathcurveto{\pgfqpoint{2.271425in}{1.645900in}}{\pgfqpoint{2.263525in}{1.649172in}}{\pgfqpoint{2.255289in}{1.649172in}}%
\pgfpathcurveto{\pgfqpoint{2.247052in}{1.649172in}}{\pgfqpoint{2.239152in}{1.645900in}}{\pgfqpoint{2.233328in}{1.640076in}}%
\pgfpathcurveto{\pgfqpoint{2.227504in}{1.634252in}}{\pgfqpoint{2.224232in}{1.626352in}}{\pgfqpoint{2.224232in}{1.618115in}}%
\pgfpathcurveto{\pgfqpoint{2.224232in}{1.609879in}}{\pgfqpoint{2.227504in}{1.601979in}}{\pgfqpoint{2.233328in}{1.596155in}}%
\pgfpathcurveto{\pgfqpoint{2.239152in}{1.590331in}}{\pgfqpoint{2.247052in}{1.587059in}}{\pgfqpoint{2.255289in}{1.587059in}}%
\pgfpathclose%
\pgfusepath{stroke,fill}%
\end{pgfscope}%
\begin{pgfscope}%
\pgfpathrectangle{\pgfqpoint{0.100000in}{0.220728in}}{\pgfqpoint{3.696000in}{3.696000in}}%
\pgfusepath{clip}%
\pgfsetbuttcap%
\pgfsetroundjoin%
\definecolor{currentfill}{rgb}{0.121569,0.466667,0.705882}%
\pgfsetfillcolor{currentfill}%
\pgfsetfillopacity{0.797302}%
\pgfsetlinewidth{1.003750pt}%
\definecolor{currentstroke}{rgb}{0.121569,0.466667,0.705882}%
\pgfsetstrokecolor{currentstroke}%
\pgfsetstrokeopacity{0.797302}%
\pgfsetdash{}{0pt}%
\pgfpathmoveto{\pgfqpoint{2.256377in}{1.586221in}}%
\pgfpathcurveto{\pgfqpoint{2.264613in}{1.586221in}}{\pgfqpoint{2.272513in}{1.589493in}}{\pgfqpoint{2.278337in}{1.595317in}}%
\pgfpathcurveto{\pgfqpoint{2.284161in}{1.601141in}}{\pgfqpoint{2.287433in}{1.609041in}}{\pgfqpoint{2.287433in}{1.617277in}}%
\pgfpathcurveto{\pgfqpoint{2.287433in}{1.625514in}}{\pgfqpoint{2.284161in}{1.633414in}}{\pgfqpoint{2.278337in}{1.639238in}}%
\pgfpathcurveto{\pgfqpoint{2.272513in}{1.645062in}}{\pgfqpoint{2.264613in}{1.648334in}}{\pgfqpoint{2.256377in}{1.648334in}}%
\pgfpathcurveto{\pgfqpoint{2.248141in}{1.648334in}}{\pgfqpoint{2.240241in}{1.645062in}}{\pgfqpoint{2.234417in}{1.639238in}}%
\pgfpathcurveto{\pgfqpoint{2.228593in}{1.633414in}}{\pgfqpoint{2.225320in}{1.625514in}}{\pgfqpoint{2.225320in}{1.617277in}}%
\pgfpathcurveto{\pgfqpoint{2.225320in}{1.609041in}}{\pgfqpoint{2.228593in}{1.601141in}}{\pgfqpoint{2.234417in}{1.595317in}}%
\pgfpathcurveto{\pgfqpoint{2.240241in}{1.589493in}}{\pgfqpoint{2.248141in}{1.586221in}}{\pgfqpoint{2.256377in}{1.586221in}}%
\pgfpathclose%
\pgfusepath{stroke,fill}%
\end{pgfscope}%
\begin{pgfscope}%
\pgfpathrectangle{\pgfqpoint{0.100000in}{0.220728in}}{\pgfqpoint{3.696000in}{3.696000in}}%
\pgfusepath{clip}%
\pgfsetbuttcap%
\pgfsetroundjoin%
\definecolor{currentfill}{rgb}{0.121569,0.466667,0.705882}%
\pgfsetfillcolor{currentfill}%
\pgfsetfillopacity{0.798740}%
\pgfsetlinewidth{1.003750pt}%
\definecolor{currentstroke}{rgb}{0.121569,0.466667,0.705882}%
\pgfsetstrokecolor{currentstroke}%
\pgfsetstrokeopacity{0.798740}%
\pgfsetdash{}{0pt}%
\pgfpathmoveto{\pgfqpoint{2.257240in}{1.583016in}}%
\pgfpathcurveto{\pgfqpoint{2.265476in}{1.583016in}}{\pgfqpoint{2.273376in}{1.586288in}}{\pgfqpoint{2.279200in}{1.592112in}}%
\pgfpathcurveto{\pgfqpoint{2.285024in}{1.597936in}}{\pgfqpoint{2.288296in}{1.605836in}}{\pgfqpoint{2.288296in}{1.614072in}}%
\pgfpathcurveto{\pgfqpoint{2.288296in}{1.622308in}}{\pgfqpoint{2.285024in}{1.630208in}}{\pgfqpoint{2.279200in}{1.636032in}}%
\pgfpathcurveto{\pgfqpoint{2.273376in}{1.641856in}}{\pgfqpoint{2.265476in}{1.645129in}}{\pgfqpoint{2.257240in}{1.645129in}}%
\pgfpathcurveto{\pgfqpoint{2.249003in}{1.645129in}}{\pgfqpoint{2.241103in}{1.641856in}}{\pgfqpoint{2.235279in}{1.636032in}}%
\pgfpathcurveto{\pgfqpoint{2.229455in}{1.630208in}}{\pgfqpoint{2.226183in}{1.622308in}}{\pgfqpoint{2.226183in}{1.614072in}}%
\pgfpathcurveto{\pgfqpoint{2.226183in}{1.605836in}}{\pgfqpoint{2.229455in}{1.597936in}}{\pgfqpoint{2.235279in}{1.592112in}}%
\pgfpathcurveto{\pgfqpoint{2.241103in}{1.586288in}}{\pgfqpoint{2.249003in}{1.583016in}}{\pgfqpoint{2.257240in}{1.583016in}}%
\pgfpathclose%
\pgfusepath{stroke,fill}%
\end{pgfscope}%
\begin{pgfscope}%
\pgfpathrectangle{\pgfqpoint{0.100000in}{0.220728in}}{\pgfqpoint{3.696000in}{3.696000in}}%
\pgfusepath{clip}%
\pgfsetbuttcap%
\pgfsetroundjoin%
\definecolor{currentfill}{rgb}{0.121569,0.466667,0.705882}%
\pgfsetfillcolor{currentfill}%
\pgfsetfillopacity{0.800235}%
\pgfsetlinewidth{1.003750pt}%
\definecolor{currentstroke}{rgb}{0.121569,0.466667,0.705882}%
\pgfsetstrokecolor{currentstroke}%
\pgfsetstrokeopacity{0.800235}%
\pgfsetdash{}{0pt}%
\pgfpathmoveto{\pgfqpoint{2.259598in}{1.576037in}}%
\pgfpathcurveto{\pgfqpoint{2.267835in}{1.576037in}}{\pgfqpoint{2.275735in}{1.579310in}}{\pgfqpoint{2.281559in}{1.585134in}}%
\pgfpathcurveto{\pgfqpoint{2.287383in}{1.590958in}}{\pgfqpoint{2.290655in}{1.598858in}}{\pgfqpoint{2.290655in}{1.607094in}}%
\pgfpathcurveto{\pgfqpoint{2.290655in}{1.615330in}}{\pgfqpoint{2.287383in}{1.623230in}}{\pgfqpoint{2.281559in}{1.629054in}}%
\pgfpathcurveto{\pgfqpoint{2.275735in}{1.634878in}}{\pgfqpoint{2.267835in}{1.638150in}}{\pgfqpoint{2.259598in}{1.638150in}}%
\pgfpathcurveto{\pgfqpoint{2.251362in}{1.638150in}}{\pgfqpoint{2.243462in}{1.634878in}}{\pgfqpoint{2.237638in}{1.629054in}}%
\pgfpathcurveto{\pgfqpoint{2.231814in}{1.623230in}}{\pgfqpoint{2.228542in}{1.615330in}}{\pgfqpoint{2.228542in}{1.607094in}}%
\pgfpathcurveto{\pgfqpoint{2.228542in}{1.598858in}}{\pgfqpoint{2.231814in}{1.590958in}}{\pgfqpoint{2.237638in}{1.585134in}}%
\pgfpathcurveto{\pgfqpoint{2.243462in}{1.579310in}}{\pgfqpoint{2.251362in}{1.576037in}}{\pgfqpoint{2.259598in}{1.576037in}}%
\pgfpathclose%
\pgfusepath{stroke,fill}%
\end{pgfscope}%
\begin{pgfscope}%
\pgfpathrectangle{\pgfqpoint{0.100000in}{0.220728in}}{\pgfqpoint{3.696000in}{3.696000in}}%
\pgfusepath{clip}%
\pgfsetbuttcap%
\pgfsetroundjoin%
\definecolor{currentfill}{rgb}{0.121569,0.466667,0.705882}%
\pgfsetfillcolor{currentfill}%
\pgfsetfillopacity{0.802436}%
\pgfsetlinewidth{1.003750pt}%
\definecolor{currentstroke}{rgb}{0.121569,0.466667,0.705882}%
\pgfsetstrokecolor{currentstroke}%
\pgfsetstrokeopacity{0.802436}%
\pgfsetdash{}{0pt}%
\pgfpathmoveto{\pgfqpoint{2.261588in}{1.570805in}}%
\pgfpathcurveto{\pgfqpoint{2.269825in}{1.570805in}}{\pgfqpoint{2.277725in}{1.574078in}}{\pgfqpoint{2.283549in}{1.579902in}}%
\pgfpathcurveto{\pgfqpoint{2.289372in}{1.585726in}}{\pgfqpoint{2.292645in}{1.593626in}}{\pgfqpoint{2.292645in}{1.601862in}}%
\pgfpathcurveto{\pgfqpoint{2.292645in}{1.610098in}}{\pgfqpoint{2.289372in}{1.617998in}}{\pgfqpoint{2.283549in}{1.623822in}}%
\pgfpathcurveto{\pgfqpoint{2.277725in}{1.629646in}}{\pgfqpoint{2.269825in}{1.632918in}}{\pgfqpoint{2.261588in}{1.632918in}}%
\pgfpathcurveto{\pgfqpoint{2.253352in}{1.632918in}}{\pgfqpoint{2.245452in}{1.629646in}}{\pgfqpoint{2.239628in}{1.623822in}}%
\pgfpathcurveto{\pgfqpoint{2.233804in}{1.617998in}}{\pgfqpoint{2.230532in}{1.610098in}}{\pgfqpoint{2.230532in}{1.601862in}}%
\pgfpathcurveto{\pgfqpoint{2.230532in}{1.593626in}}{\pgfqpoint{2.233804in}{1.585726in}}{\pgfqpoint{2.239628in}{1.579902in}}%
\pgfpathcurveto{\pgfqpoint{2.245452in}{1.574078in}}{\pgfqpoint{2.253352in}{1.570805in}}{\pgfqpoint{2.261588in}{1.570805in}}%
\pgfpathclose%
\pgfusepath{stroke,fill}%
\end{pgfscope}%
\begin{pgfscope}%
\pgfpathrectangle{\pgfqpoint{0.100000in}{0.220728in}}{\pgfqpoint{3.696000in}{3.696000in}}%
\pgfusepath{clip}%
\pgfsetbuttcap%
\pgfsetroundjoin%
\definecolor{currentfill}{rgb}{0.121569,0.466667,0.705882}%
\pgfsetfillcolor{currentfill}%
\pgfsetfillopacity{0.805785}%
\pgfsetlinewidth{1.003750pt}%
\definecolor{currentstroke}{rgb}{0.121569,0.466667,0.705882}%
\pgfsetstrokecolor{currentstroke}%
\pgfsetstrokeopacity{0.805785}%
\pgfsetdash{}{0pt}%
\pgfpathmoveto{\pgfqpoint{2.265258in}{1.572465in}}%
\pgfpathcurveto{\pgfqpoint{2.273494in}{1.572465in}}{\pgfqpoint{2.281394in}{1.575738in}}{\pgfqpoint{2.287218in}{1.581561in}}%
\pgfpathcurveto{\pgfqpoint{2.293042in}{1.587385in}}{\pgfqpoint{2.296314in}{1.595285in}}{\pgfqpoint{2.296314in}{1.603522in}}%
\pgfpathcurveto{\pgfqpoint{2.296314in}{1.611758in}}{\pgfqpoint{2.293042in}{1.619658in}}{\pgfqpoint{2.287218in}{1.625482in}}%
\pgfpathcurveto{\pgfqpoint{2.281394in}{1.631306in}}{\pgfqpoint{2.273494in}{1.634578in}}{\pgfqpoint{2.265258in}{1.634578in}}%
\pgfpathcurveto{\pgfqpoint{2.257022in}{1.634578in}}{\pgfqpoint{2.249122in}{1.631306in}}{\pgfqpoint{2.243298in}{1.625482in}}%
\pgfpathcurveto{\pgfqpoint{2.237474in}{1.619658in}}{\pgfqpoint{2.234201in}{1.611758in}}{\pgfqpoint{2.234201in}{1.603522in}}%
\pgfpathcurveto{\pgfqpoint{2.234201in}{1.595285in}}{\pgfqpoint{2.237474in}{1.587385in}}{\pgfqpoint{2.243298in}{1.581561in}}%
\pgfpathcurveto{\pgfqpoint{2.249122in}{1.575738in}}{\pgfqpoint{2.257022in}{1.572465in}}{\pgfqpoint{2.265258in}{1.572465in}}%
\pgfpathclose%
\pgfusepath{stroke,fill}%
\end{pgfscope}%
\begin{pgfscope}%
\pgfpathrectangle{\pgfqpoint{0.100000in}{0.220728in}}{\pgfqpoint{3.696000in}{3.696000in}}%
\pgfusepath{clip}%
\pgfsetbuttcap%
\pgfsetroundjoin%
\definecolor{currentfill}{rgb}{0.121569,0.466667,0.705882}%
\pgfsetfillcolor{currentfill}%
\pgfsetfillopacity{0.807180}%
\pgfsetlinewidth{1.003750pt}%
\definecolor{currentstroke}{rgb}{0.121569,0.466667,0.705882}%
\pgfsetstrokecolor{currentstroke}%
\pgfsetstrokeopacity{0.807180}%
\pgfsetdash{}{0pt}%
\pgfpathmoveto{\pgfqpoint{2.265793in}{1.569650in}}%
\pgfpathcurveto{\pgfqpoint{2.274030in}{1.569650in}}{\pgfqpoint{2.281930in}{1.572923in}}{\pgfqpoint{2.287753in}{1.578746in}}%
\pgfpathcurveto{\pgfqpoint{2.293577in}{1.584570in}}{\pgfqpoint{2.296850in}{1.592470in}}{\pgfqpoint{2.296850in}{1.600707in}}%
\pgfpathcurveto{\pgfqpoint{2.296850in}{1.608943in}}{\pgfqpoint{2.293577in}{1.616843in}}{\pgfqpoint{2.287753in}{1.622667in}}%
\pgfpathcurveto{\pgfqpoint{2.281930in}{1.628491in}}{\pgfqpoint{2.274030in}{1.631763in}}{\pgfqpoint{2.265793in}{1.631763in}}%
\pgfpathcurveto{\pgfqpoint{2.257557in}{1.631763in}}{\pgfqpoint{2.249657in}{1.628491in}}{\pgfqpoint{2.243833in}{1.622667in}}%
\pgfpathcurveto{\pgfqpoint{2.238009in}{1.616843in}}{\pgfqpoint{2.234737in}{1.608943in}}{\pgfqpoint{2.234737in}{1.600707in}}%
\pgfpathcurveto{\pgfqpoint{2.234737in}{1.592470in}}{\pgfqpoint{2.238009in}{1.584570in}}{\pgfqpoint{2.243833in}{1.578746in}}%
\pgfpathcurveto{\pgfqpoint{2.249657in}{1.572923in}}{\pgfqpoint{2.257557in}{1.569650in}}{\pgfqpoint{2.265793in}{1.569650in}}%
\pgfpathclose%
\pgfusepath{stroke,fill}%
\end{pgfscope}%
\begin{pgfscope}%
\pgfpathrectangle{\pgfqpoint{0.100000in}{0.220728in}}{\pgfqpoint{3.696000in}{3.696000in}}%
\pgfusepath{clip}%
\pgfsetbuttcap%
\pgfsetroundjoin%
\definecolor{currentfill}{rgb}{0.121569,0.466667,0.705882}%
\pgfsetfillcolor{currentfill}%
\pgfsetfillopacity{0.808917}%
\pgfsetlinewidth{1.003750pt}%
\definecolor{currentstroke}{rgb}{0.121569,0.466667,0.705882}%
\pgfsetstrokecolor{currentstroke}%
\pgfsetstrokeopacity{0.808917}%
\pgfsetdash{}{0pt}%
\pgfpathmoveto{\pgfqpoint{2.268032in}{1.565417in}}%
\pgfpathcurveto{\pgfqpoint{2.276268in}{1.565417in}}{\pgfqpoint{2.284168in}{1.568689in}}{\pgfqpoint{2.289992in}{1.574513in}}%
\pgfpathcurveto{\pgfqpoint{2.295816in}{1.580337in}}{\pgfqpoint{2.299088in}{1.588237in}}{\pgfqpoint{2.299088in}{1.596473in}}%
\pgfpathcurveto{\pgfqpoint{2.299088in}{1.604709in}}{\pgfqpoint{2.295816in}{1.612609in}}{\pgfqpoint{2.289992in}{1.618433in}}%
\pgfpathcurveto{\pgfqpoint{2.284168in}{1.624257in}}{\pgfqpoint{2.276268in}{1.627530in}}{\pgfqpoint{2.268032in}{1.627530in}}%
\pgfpathcurveto{\pgfqpoint{2.259795in}{1.627530in}}{\pgfqpoint{2.251895in}{1.624257in}}{\pgfqpoint{2.246072in}{1.618433in}}%
\pgfpathcurveto{\pgfqpoint{2.240248in}{1.612609in}}{\pgfqpoint{2.236975in}{1.604709in}}{\pgfqpoint{2.236975in}{1.596473in}}%
\pgfpathcurveto{\pgfqpoint{2.236975in}{1.588237in}}{\pgfqpoint{2.240248in}{1.580337in}}{\pgfqpoint{2.246072in}{1.574513in}}%
\pgfpathcurveto{\pgfqpoint{2.251895in}{1.568689in}}{\pgfqpoint{2.259795in}{1.565417in}}{\pgfqpoint{2.268032in}{1.565417in}}%
\pgfpathclose%
\pgfusepath{stroke,fill}%
\end{pgfscope}%
\begin{pgfscope}%
\pgfpathrectangle{\pgfqpoint{0.100000in}{0.220728in}}{\pgfqpoint{3.696000in}{3.696000in}}%
\pgfusepath{clip}%
\pgfsetbuttcap%
\pgfsetroundjoin%
\definecolor{currentfill}{rgb}{0.121569,0.466667,0.705882}%
\pgfsetfillcolor{currentfill}%
\pgfsetfillopacity{0.810847}%
\pgfsetlinewidth{1.003750pt}%
\definecolor{currentstroke}{rgb}{0.121569,0.466667,0.705882}%
\pgfsetstrokecolor{currentstroke}%
\pgfsetstrokeopacity{0.810847}%
\pgfsetdash{}{0pt}%
\pgfpathmoveto{\pgfqpoint{2.269340in}{1.560297in}}%
\pgfpathcurveto{\pgfqpoint{2.277577in}{1.560297in}}{\pgfqpoint{2.285477in}{1.563569in}}{\pgfqpoint{2.291301in}{1.569393in}}%
\pgfpathcurveto{\pgfqpoint{2.297125in}{1.575217in}}{\pgfqpoint{2.300397in}{1.583117in}}{\pgfqpoint{2.300397in}{1.591353in}}%
\pgfpathcurveto{\pgfqpoint{2.300397in}{1.599590in}}{\pgfqpoint{2.297125in}{1.607490in}}{\pgfqpoint{2.291301in}{1.613314in}}%
\pgfpathcurveto{\pgfqpoint{2.285477in}{1.619138in}}{\pgfqpoint{2.277577in}{1.622410in}}{\pgfqpoint{2.269340in}{1.622410in}}%
\pgfpathcurveto{\pgfqpoint{2.261104in}{1.622410in}}{\pgfqpoint{2.253204in}{1.619138in}}{\pgfqpoint{2.247380in}{1.613314in}}%
\pgfpathcurveto{\pgfqpoint{2.241556in}{1.607490in}}{\pgfqpoint{2.238284in}{1.599590in}}{\pgfqpoint{2.238284in}{1.591353in}}%
\pgfpathcurveto{\pgfqpoint{2.238284in}{1.583117in}}{\pgfqpoint{2.241556in}{1.575217in}}{\pgfqpoint{2.247380in}{1.569393in}}%
\pgfpathcurveto{\pgfqpoint{2.253204in}{1.563569in}}{\pgfqpoint{2.261104in}{1.560297in}}{\pgfqpoint{2.269340in}{1.560297in}}%
\pgfpathclose%
\pgfusepath{stroke,fill}%
\end{pgfscope}%
\begin{pgfscope}%
\pgfpathrectangle{\pgfqpoint{0.100000in}{0.220728in}}{\pgfqpoint{3.696000in}{3.696000in}}%
\pgfusepath{clip}%
\pgfsetbuttcap%
\pgfsetroundjoin%
\definecolor{currentfill}{rgb}{0.121569,0.466667,0.705882}%
\pgfsetfillcolor{currentfill}%
\pgfsetfillopacity{0.812310}%
\pgfsetlinewidth{1.003750pt}%
\definecolor{currentstroke}{rgb}{0.121569,0.466667,0.705882}%
\pgfsetstrokecolor{currentstroke}%
\pgfsetstrokeopacity{0.812310}%
\pgfsetdash{}{0pt}%
\pgfpathmoveto{\pgfqpoint{2.270736in}{1.560404in}}%
\pgfpathcurveto{\pgfqpoint{2.278972in}{1.560404in}}{\pgfqpoint{2.286872in}{1.563676in}}{\pgfqpoint{2.292696in}{1.569500in}}%
\pgfpathcurveto{\pgfqpoint{2.298520in}{1.575324in}}{\pgfqpoint{2.301792in}{1.583224in}}{\pgfqpoint{2.301792in}{1.591460in}}%
\pgfpathcurveto{\pgfqpoint{2.301792in}{1.599697in}}{\pgfqpoint{2.298520in}{1.607597in}}{\pgfqpoint{2.292696in}{1.613421in}}%
\pgfpathcurveto{\pgfqpoint{2.286872in}{1.619245in}}{\pgfqpoint{2.278972in}{1.622517in}}{\pgfqpoint{2.270736in}{1.622517in}}%
\pgfpathcurveto{\pgfqpoint{2.262499in}{1.622517in}}{\pgfqpoint{2.254599in}{1.619245in}}{\pgfqpoint{2.248775in}{1.613421in}}%
\pgfpathcurveto{\pgfqpoint{2.242951in}{1.607597in}}{\pgfqpoint{2.239679in}{1.599697in}}{\pgfqpoint{2.239679in}{1.591460in}}%
\pgfpathcurveto{\pgfqpoint{2.239679in}{1.583224in}}{\pgfqpoint{2.242951in}{1.575324in}}{\pgfqpoint{2.248775in}{1.569500in}}%
\pgfpathcurveto{\pgfqpoint{2.254599in}{1.563676in}}{\pgfqpoint{2.262499in}{1.560404in}}{\pgfqpoint{2.270736in}{1.560404in}}%
\pgfpathclose%
\pgfusepath{stroke,fill}%
\end{pgfscope}%
\begin{pgfscope}%
\pgfpathrectangle{\pgfqpoint{0.100000in}{0.220728in}}{\pgfqpoint{3.696000in}{3.696000in}}%
\pgfusepath{clip}%
\pgfsetbuttcap%
\pgfsetroundjoin%
\definecolor{currentfill}{rgb}{0.121569,0.466667,0.705882}%
\pgfsetfillcolor{currentfill}%
\pgfsetfillopacity{0.813067}%
\pgfsetlinewidth{1.003750pt}%
\definecolor{currentstroke}{rgb}{0.121569,0.466667,0.705882}%
\pgfsetstrokecolor{currentstroke}%
\pgfsetstrokeopacity{0.813067}%
\pgfsetdash{}{0pt}%
\pgfpathmoveto{\pgfqpoint{2.271106in}{1.559918in}}%
\pgfpathcurveto{\pgfqpoint{2.279343in}{1.559918in}}{\pgfqpoint{2.287243in}{1.563191in}}{\pgfqpoint{2.293067in}{1.569015in}}%
\pgfpathcurveto{\pgfqpoint{2.298890in}{1.574839in}}{\pgfqpoint{2.302163in}{1.582739in}}{\pgfqpoint{2.302163in}{1.590975in}}%
\pgfpathcurveto{\pgfqpoint{2.302163in}{1.599211in}}{\pgfqpoint{2.298890in}{1.607111in}}{\pgfqpoint{2.293067in}{1.612935in}}%
\pgfpathcurveto{\pgfqpoint{2.287243in}{1.618759in}}{\pgfqpoint{2.279343in}{1.622031in}}{\pgfqpoint{2.271106in}{1.622031in}}%
\pgfpathcurveto{\pgfqpoint{2.262870in}{1.622031in}}{\pgfqpoint{2.254970in}{1.618759in}}{\pgfqpoint{2.249146in}{1.612935in}}%
\pgfpathcurveto{\pgfqpoint{2.243322in}{1.607111in}}{\pgfqpoint{2.240050in}{1.599211in}}{\pgfqpoint{2.240050in}{1.590975in}}%
\pgfpathcurveto{\pgfqpoint{2.240050in}{1.582739in}}{\pgfqpoint{2.243322in}{1.574839in}}{\pgfqpoint{2.249146in}{1.569015in}}%
\pgfpathcurveto{\pgfqpoint{2.254970in}{1.563191in}}{\pgfqpoint{2.262870in}{1.559918in}}{\pgfqpoint{2.271106in}{1.559918in}}%
\pgfpathclose%
\pgfusepath{stroke,fill}%
\end{pgfscope}%
\begin{pgfscope}%
\pgfpathrectangle{\pgfqpoint{0.100000in}{0.220728in}}{\pgfqpoint{3.696000in}{3.696000in}}%
\pgfusepath{clip}%
\pgfsetbuttcap%
\pgfsetroundjoin%
\definecolor{currentfill}{rgb}{0.121569,0.466667,0.705882}%
\pgfsetfillcolor{currentfill}%
\pgfsetfillopacity{0.814058}%
\pgfsetlinewidth{1.003750pt}%
\definecolor{currentstroke}{rgb}{0.121569,0.466667,0.705882}%
\pgfsetstrokecolor{currentstroke}%
\pgfsetstrokeopacity{0.814058}%
\pgfsetdash{}{0pt}%
\pgfpathmoveto{\pgfqpoint{2.271930in}{1.558289in}}%
\pgfpathcurveto{\pgfqpoint{2.280167in}{1.558289in}}{\pgfqpoint{2.288067in}{1.561561in}}{\pgfqpoint{2.293891in}{1.567385in}}%
\pgfpathcurveto{\pgfqpoint{2.299715in}{1.573209in}}{\pgfqpoint{2.302987in}{1.581109in}}{\pgfqpoint{2.302987in}{1.589345in}}%
\pgfpathcurveto{\pgfqpoint{2.302987in}{1.597582in}}{\pgfqpoint{2.299715in}{1.605482in}}{\pgfqpoint{2.293891in}{1.611306in}}%
\pgfpathcurveto{\pgfqpoint{2.288067in}{1.617129in}}{\pgfqpoint{2.280167in}{1.620402in}}{\pgfqpoint{2.271930in}{1.620402in}}%
\pgfpathcurveto{\pgfqpoint{2.263694in}{1.620402in}}{\pgfqpoint{2.255794in}{1.617129in}}{\pgfqpoint{2.249970in}{1.611306in}}%
\pgfpathcurveto{\pgfqpoint{2.244146in}{1.605482in}}{\pgfqpoint{2.240874in}{1.597582in}}{\pgfqpoint{2.240874in}{1.589345in}}%
\pgfpathcurveto{\pgfqpoint{2.240874in}{1.581109in}}{\pgfqpoint{2.244146in}{1.573209in}}{\pgfqpoint{2.249970in}{1.567385in}}%
\pgfpathcurveto{\pgfqpoint{2.255794in}{1.561561in}}{\pgfqpoint{2.263694in}{1.558289in}}{\pgfqpoint{2.271930in}{1.558289in}}%
\pgfpathclose%
\pgfusepath{stroke,fill}%
\end{pgfscope}%
\begin{pgfscope}%
\pgfpathrectangle{\pgfqpoint{0.100000in}{0.220728in}}{\pgfqpoint{3.696000in}{3.696000in}}%
\pgfusepath{clip}%
\pgfsetbuttcap%
\pgfsetroundjoin%
\definecolor{currentfill}{rgb}{0.121569,0.466667,0.705882}%
\pgfsetfillcolor{currentfill}%
\pgfsetfillopacity{0.815050}%
\pgfsetlinewidth{1.003750pt}%
\definecolor{currentstroke}{rgb}{0.121569,0.466667,0.705882}%
\pgfsetstrokecolor{currentstroke}%
\pgfsetstrokeopacity{0.815050}%
\pgfsetdash{}{0pt}%
\pgfpathmoveto{\pgfqpoint{2.272762in}{1.555370in}}%
\pgfpathcurveto{\pgfqpoint{2.280998in}{1.555370in}}{\pgfqpoint{2.288899in}{1.558642in}}{\pgfqpoint{2.294722in}{1.564466in}}%
\pgfpathcurveto{\pgfqpoint{2.300546in}{1.570290in}}{\pgfqpoint{2.303819in}{1.578190in}}{\pgfqpoint{2.303819in}{1.586426in}}%
\pgfpathcurveto{\pgfqpoint{2.303819in}{1.594663in}}{\pgfqpoint{2.300546in}{1.602563in}}{\pgfqpoint{2.294722in}{1.608387in}}%
\pgfpathcurveto{\pgfqpoint{2.288899in}{1.614211in}}{\pgfqpoint{2.280998in}{1.617483in}}{\pgfqpoint{2.272762in}{1.617483in}}%
\pgfpathcurveto{\pgfqpoint{2.264526in}{1.617483in}}{\pgfqpoint{2.256626in}{1.614211in}}{\pgfqpoint{2.250802in}{1.608387in}}%
\pgfpathcurveto{\pgfqpoint{2.244978in}{1.602563in}}{\pgfqpoint{2.241706in}{1.594663in}}{\pgfqpoint{2.241706in}{1.586426in}}%
\pgfpathcurveto{\pgfqpoint{2.241706in}{1.578190in}}{\pgfqpoint{2.244978in}{1.570290in}}{\pgfqpoint{2.250802in}{1.564466in}}%
\pgfpathcurveto{\pgfqpoint{2.256626in}{1.558642in}}{\pgfqpoint{2.264526in}{1.555370in}}{\pgfqpoint{2.272762in}{1.555370in}}%
\pgfpathclose%
\pgfusepath{stroke,fill}%
\end{pgfscope}%
\begin{pgfscope}%
\pgfpathrectangle{\pgfqpoint{0.100000in}{0.220728in}}{\pgfqpoint{3.696000in}{3.696000in}}%
\pgfusepath{clip}%
\pgfsetbuttcap%
\pgfsetroundjoin%
\definecolor{currentfill}{rgb}{0.121569,0.466667,0.705882}%
\pgfsetfillcolor{currentfill}%
\pgfsetfillopacity{0.816513}%
\pgfsetlinewidth{1.003750pt}%
\definecolor{currentstroke}{rgb}{0.121569,0.466667,0.705882}%
\pgfsetstrokecolor{currentstroke}%
\pgfsetstrokeopacity{0.816513}%
\pgfsetdash{}{0pt}%
\pgfpathmoveto{\pgfqpoint{2.274403in}{1.554262in}}%
\pgfpathcurveto{\pgfqpoint{2.282639in}{1.554262in}}{\pgfqpoint{2.290539in}{1.557535in}}{\pgfqpoint{2.296363in}{1.563359in}}%
\pgfpathcurveto{\pgfqpoint{2.302187in}{1.569183in}}{\pgfqpoint{2.305459in}{1.577083in}}{\pgfqpoint{2.305459in}{1.585319in}}%
\pgfpathcurveto{\pgfqpoint{2.305459in}{1.593555in}}{\pgfqpoint{2.302187in}{1.601455in}}{\pgfqpoint{2.296363in}{1.607279in}}%
\pgfpathcurveto{\pgfqpoint{2.290539in}{1.613103in}}{\pgfqpoint{2.282639in}{1.616375in}}{\pgfqpoint{2.274403in}{1.616375in}}%
\pgfpathcurveto{\pgfqpoint{2.266167in}{1.616375in}}{\pgfqpoint{2.258267in}{1.613103in}}{\pgfqpoint{2.252443in}{1.607279in}}%
\pgfpathcurveto{\pgfqpoint{2.246619in}{1.601455in}}{\pgfqpoint{2.243346in}{1.593555in}}{\pgfqpoint{2.243346in}{1.585319in}}%
\pgfpathcurveto{\pgfqpoint{2.243346in}{1.577083in}}{\pgfqpoint{2.246619in}{1.569183in}}{\pgfqpoint{2.252443in}{1.563359in}}%
\pgfpathcurveto{\pgfqpoint{2.258267in}{1.557535in}}{\pgfqpoint{2.266167in}{1.554262in}}{\pgfqpoint{2.274403in}{1.554262in}}%
\pgfpathclose%
\pgfusepath{stroke,fill}%
\end{pgfscope}%
\begin{pgfscope}%
\pgfpathrectangle{\pgfqpoint{0.100000in}{0.220728in}}{\pgfqpoint{3.696000in}{3.696000in}}%
\pgfusepath{clip}%
\pgfsetbuttcap%
\pgfsetroundjoin%
\definecolor{currentfill}{rgb}{0.121569,0.466667,0.705882}%
\pgfsetfillcolor{currentfill}%
\pgfsetfillopacity{0.817365}%
\pgfsetlinewidth{1.003750pt}%
\definecolor{currentstroke}{rgb}{0.121569,0.466667,0.705882}%
\pgfsetstrokecolor{currentstroke}%
\pgfsetstrokeopacity{0.817365}%
\pgfsetdash{}{0pt}%
\pgfpathmoveto{\pgfqpoint{2.274976in}{1.553714in}}%
\pgfpathcurveto{\pgfqpoint{2.283212in}{1.553714in}}{\pgfqpoint{2.291112in}{1.556986in}}{\pgfqpoint{2.296936in}{1.562810in}}%
\pgfpathcurveto{\pgfqpoint{2.302760in}{1.568634in}}{\pgfqpoint{2.306032in}{1.576534in}}{\pgfqpoint{2.306032in}{1.584771in}}%
\pgfpathcurveto{\pgfqpoint{2.306032in}{1.593007in}}{\pgfqpoint{2.302760in}{1.600907in}}{\pgfqpoint{2.296936in}{1.606731in}}%
\pgfpathcurveto{\pgfqpoint{2.291112in}{1.612555in}}{\pgfqpoint{2.283212in}{1.615827in}}{\pgfqpoint{2.274976in}{1.615827in}}%
\pgfpathcurveto{\pgfqpoint{2.266740in}{1.615827in}}{\pgfqpoint{2.258839in}{1.612555in}}{\pgfqpoint{2.253016in}{1.606731in}}%
\pgfpathcurveto{\pgfqpoint{2.247192in}{1.600907in}}{\pgfqpoint{2.243919in}{1.593007in}}{\pgfqpoint{2.243919in}{1.584771in}}%
\pgfpathcurveto{\pgfqpoint{2.243919in}{1.576534in}}{\pgfqpoint{2.247192in}{1.568634in}}{\pgfqpoint{2.253016in}{1.562810in}}%
\pgfpathcurveto{\pgfqpoint{2.258839in}{1.556986in}}{\pgfqpoint{2.266740in}{1.553714in}}{\pgfqpoint{2.274976in}{1.553714in}}%
\pgfpathclose%
\pgfusepath{stroke,fill}%
\end{pgfscope}%
\begin{pgfscope}%
\pgfpathrectangle{\pgfqpoint{0.100000in}{0.220728in}}{\pgfqpoint{3.696000in}{3.696000in}}%
\pgfusepath{clip}%
\pgfsetbuttcap%
\pgfsetroundjoin%
\definecolor{currentfill}{rgb}{0.121569,0.466667,0.705882}%
\pgfsetfillcolor{currentfill}%
\pgfsetfillopacity{0.818432}%
\pgfsetlinewidth{1.003750pt}%
\definecolor{currentstroke}{rgb}{0.121569,0.466667,0.705882}%
\pgfsetstrokecolor{currentstroke}%
\pgfsetstrokeopacity{0.818432}%
\pgfsetdash{}{0pt}%
\pgfpathmoveto{\pgfqpoint{2.275368in}{1.552139in}}%
\pgfpathcurveto{\pgfqpoint{2.283604in}{1.552139in}}{\pgfqpoint{2.291504in}{1.555411in}}{\pgfqpoint{2.297328in}{1.561235in}}%
\pgfpathcurveto{\pgfqpoint{2.303152in}{1.567059in}}{\pgfqpoint{2.306425in}{1.574959in}}{\pgfqpoint{2.306425in}{1.583196in}}%
\pgfpathcurveto{\pgfqpoint{2.306425in}{1.591432in}}{\pgfqpoint{2.303152in}{1.599332in}}{\pgfqpoint{2.297328in}{1.605156in}}%
\pgfpathcurveto{\pgfqpoint{2.291504in}{1.610980in}}{\pgfqpoint{2.283604in}{1.614252in}}{\pgfqpoint{2.275368in}{1.614252in}}%
\pgfpathcurveto{\pgfqpoint{2.267132in}{1.614252in}}{\pgfqpoint{2.259232in}{1.610980in}}{\pgfqpoint{2.253408in}{1.605156in}}%
\pgfpathcurveto{\pgfqpoint{2.247584in}{1.599332in}}{\pgfqpoint{2.244312in}{1.591432in}}{\pgfqpoint{2.244312in}{1.583196in}}%
\pgfpathcurveto{\pgfqpoint{2.244312in}{1.574959in}}{\pgfqpoint{2.247584in}{1.567059in}}{\pgfqpoint{2.253408in}{1.561235in}}%
\pgfpathcurveto{\pgfqpoint{2.259232in}{1.555411in}}{\pgfqpoint{2.267132in}{1.552139in}}{\pgfqpoint{2.275368in}{1.552139in}}%
\pgfpathclose%
\pgfusepath{stroke,fill}%
\end{pgfscope}%
\begin{pgfscope}%
\pgfpathrectangle{\pgfqpoint{0.100000in}{0.220728in}}{\pgfqpoint{3.696000in}{3.696000in}}%
\pgfusepath{clip}%
\pgfsetbuttcap%
\pgfsetroundjoin%
\definecolor{currentfill}{rgb}{0.121569,0.466667,0.705882}%
\pgfsetfillcolor{currentfill}%
\pgfsetfillopacity{0.819694}%
\pgfsetlinewidth{1.003750pt}%
\definecolor{currentstroke}{rgb}{0.121569,0.466667,0.705882}%
\pgfsetstrokecolor{currentstroke}%
\pgfsetstrokeopacity{0.819694}%
\pgfsetdash{}{0pt}%
\pgfpathmoveto{\pgfqpoint{2.276454in}{1.546476in}}%
\pgfpathcurveto{\pgfqpoint{2.284691in}{1.546476in}}{\pgfqpoint{2.292591in}{1.549748in}}{\pgfqpoint{2.298415in}{1.555572in}}%
\pgfpathcurveto{\pgfqpoint{2.304239in}{1.561396in}}{\pgfqpoint{2.307511in}{1.569296in}}{\pgfqpoint{2.307511in}{1.577532in}}%
\pgfpathcurveto{\pgfqpoint{2.307511in}{1.585768in}}{\pgfqpoint{2.304239in}{1.593669in}}{\pgfqpoint{2.298415in}{1.599492in}}%
\pgfpathcurveto{\pgfqpoint{2.292591in}{1.605316in}}{\pgfqpoint{2.284691in}{1.608589in}}{\pgfqpoint{2.276454in}{1.608589in}}%
\pgfpathcurveto{\pgfqpoint{2.268218in}{1.608589in}}{\pgfqpoint{2.260318in}{1.605316in}}{\pgfqpoint{2.254494in}{1.599492in}}%
\pgfpathcurveto{\pgfqpoint{2.248670in}{1.593669in}}{\pgfqpoint{2.245398in}{1.585768in}}{\pgfqpoint{2.245398in}{1.577532in}}%
\pgfpathcurveto{\pgfqpoint{2.245398in}{1.569296in}}{\pgfqpoint{2.248670in}{1.561396in}}{\pgfqpoint{2.254494in}{1.555572in}}%
\pgfpathcurveto{\pgfqpoint{2.260318in}{1.549748in}}{\pgfqpoint{2.268218in}{1.546476in}}{\pgfqpoint{2.276454in}{1.546476in}}%
\pgfpathclose%
\pgfusepath{stroke,fill}%
\end{pgfscope}%
\begin{pgfscope}%
\pgfpathrectangle{\pgfqpoint{0.100000in}{0.220728in}}{\pgfqpoint{3.696000in}{3.696000in}}%
\pgfusepath{clip}%
\pgfsetbuttcap%
\pgfsetroundjoin%
\definecolor{currentfill}{rgb}{0.121569,0.466667,0.705882}%
\pgfsetfillcolor{currentfill}%
\pgfsetfillopacity{0.820705}%
\pgfsetlinewidth{1.003750pt}%
\definecolor{currentstroke}{rgb}{0.121569,0.466667,0.705882}%
\pgfsetstrokecolor{currentstroke}%
\pgfsetstrokeopacity{0.820705}%
\pgfsetdash{}{0pt}%
\pgfpathmoveto{\pgfqpoint{2.277374in}{1.545507in}}%
\pgfpathcurveto{\pgfqpoint{2.285610in}{1.545507in}}{\pgfqpoint{2.293510in}{1.548780in}}{\pgfqpoint{2.299334in}{1.554604in}}%
\pgfpathcurveto{\pgfqpoint{2.305158in}{1.560427in}}{\pgfqpoint{2.308431in}{1.568328in}}{\pgfqpoint{2.308431in}{1.576564in}}%
\pgfpathcurveto{\pgfqpoint{2.308431in}{1.584800in}}{\pgfqpoint{2.305158in}{1.592700in}}{\pgfqpoint{2.299334in}{1.598524in}}%
\pgfpathcurveto{\pgfqpoint{2.293510in}{1.604348in}}{\pgfqpoint{2.285610in}{1.607620in}}{\pgfqpoint{2.277374in}{1.607620in}}%
\pgfpathcurveto{\pgfqpoint{2.269138in}{1.607620in}}{\pgfqpoint{2.261238in}{1.604348in}}{\pgfqpoint{2.255414in}{1.598524in}}%
\pgfpathcurveto{\pgfqpoint{2.249590in}{1.592700in}}{\pgfqpoint{2.246318in}{1.584800in}}{\pgfqpoint{2.246318in}{1.576564in}}%
\pgfpathcurveto{\pgfqpoint{2.246318in}{1.568328in}}{\pgfqpoint{2.249590in}{1.560427in}}{\pgfqpoint{2.255414in}{1.554604in}}%
\pgfpathcurveto{\pgfqpoint{2.261238in}{1.548780in}}{\pgfqpoint{2.269138in}{1.545507in}}{\pgfqpoint{2.277374in}{1.545507in}}%
\pgfpathclose%
\pgfusepath{stroke,fill}%
\end{pgfscope}%
\begin{pgfscope}%
\pgfpathrectangle{\pgfqpoint{0.100000in}{0.220728in}}{\pgfqpoint{3.696000in}{3.696000in}}%
\pgfusepath{clip}%
\pgfsetbuttcap%
\pgfsetroundjoin%
\definecolor{currentfill}{rgb}{0.121569,0.466667,0.705882}%
\pgfsetfillcolor{currentfill}%
\pgfsetfillopacity{0.822141}%
\pgfsetlinewidth{1.003750pt}%
\definecolor{currentstroke}{rgb}{0.121569,0.466667,0.705882}%
\pgfsetstrokecolor{currentstroke}%
\pgfsetstrokeopacity{0.822141}%
\pgfsetdash{}{0pt}%
\pgfpathmoveto{\pgfqpoint{2.278694in}{1.545506in}}%
\pgfpathcurveto{\pgfqpoint{2.286931in}{1.545506in}}{\pgfqpoint{2.294831in}{1.548778in}}{\pgfqpoint{2.300655in}{1.554602in}}%
\pgfpathcurveto{\pgfqpoint{2.306479in}{1.560426in}}{\pgfqpoint{2.309751in}{1.568326in}}{\pgfqpoint{2.309751in}{1.576562in}}%
\pgfpathcurveto{\pgfqpoint{2.309751in}{1.584799in}}{\pgfqpoint{2.306479in}{1.592699in}}{\pgfqpoint{2.300655in}{1.598523in}}%
\pgfpathcurveto{\pgfqpoint{2.294831in}{1.604347in}}{\pgfqpoint{2.286931in}{1.607619in}}{\pgfqpoint{2.278694in}{1.607619in}}%
\pgfpathcurveto{\pgfqpoint{2.270458in}{1.607619in}}{\pgfqpoint{2.262558in}{1.604347in}}{\pgfqpoint{2.256734in}{1.598523in}}%
\pgfpathcurveto{\pgfqpoint{2.250910in}{1.592699in}}{\pgfqpoint{2.247638in}{1.584799in}}{\pgfqpoint{2.247638in}{1.576562in}}%
\pgfpathcurveto{\pgfqpoint{2.247638in}{1.568326in}}{\pgfqpoint{2.250910in}{1.560426in}}{\pgfqpoint{2.256734in}{1.554602in}}%
\pgfpathcurveto{\pgfqpoint{2.262558in}{1.548778in}}{\pgfqpoint{2.270458in}{1.545506in}}{\pgfqpoint{2.278694in}{1.545506in}}%
\pgfpathclose%
\pgfusepath{stroke,fill}%
\end{pgfscope}%
\begin{pgfscope}%
\pgfpathrectangle{\pgfqpoint{0.100000in}{0.220728in}}{\pgfqpoint{3.696000in}{3.696000in}}%
\pgfusepath{clip}%
\pgfsetbuttcap%
\pgfsetroundjoin%
\definecolor{currentfill}{rgb}{0.121569,0.466667,0.705882}%
\pgfsetfillcolor{currentfill}%
\pgfsetfillopacity{0.822775}%
\pgfsetlinewidth{1.003750pt}%
\definecolor{currentstroke}{rgb}{0.121569,0.466667,0.705882}%
\pgfsetstrokecolor{currentstroke}%
\pgfsetstrokeopacity{0.822775}%
\pgfsetdash{}{0pt}%
\pgfpathmoveto{\pgfqpoint{2.279011in}{1.544296in}}%
\pgfpathcurveto{\pgfqpoint{2.287247in}{1.544296in}}{\pgfqpoint{2.295147in}{1.547568in}}{\pgfqpoint{2.300971in}{1.553392in}}%
\pgfpathcurveto{\pgfqpoint{2.306795in}{1.559216in}}{\pgfqpoint{2.310067in}{1.567116in}}{\pgfqpoint{2.310067in}{1.575352in}}%
\pgfpathcurveto{\pgfqpoint{2.310067in}{1.583589in}}{\pgfqpoint{2.306795in}{1.591489in}}{\pgfqpoint{2.300971in}{1.597313in}}%
\pgfpathcurveto{\pgfqpoint{2.295147in}{1.603137in}}{\pgfqpoint{2.287247in}{1.606409in}}{\pgfqpoint{2.279011in}{1.606409in}}%
\pgfpathcurveto{\pgfqpoint{2.270775in}{1.606409in}}{\pgfqpoint{2.262875in}{1.603137in}}{\pgfqpoint{2.257051in}{1.597313in}}%
\pgfpathcurveto{\pgfqpoint{2.251227in}{1.591489in}}{\pgfqpoint{2.247954in}{1.583589in}}{\pgfqpoint{2.247954in}{1.575352in}}%
\pgfpathcurveto{\pgfqpoint{2.247954in}{1.567116in}}{\pgfqpoint{2.251227in}{1.559216in}}{\pgfqpoint{2.257051in}{1.553392in}}%
\pgfpathcurveto{\pgfqpoint{2.262875in}{1.547568in}}{\pgfqpoint{2.270775in}{1.544296in}}{\pgfqpoint{2.279011in}{1.544296in}}%
\pgfpathclose%
\pgfusepath{stroke,fill}%
\end{pgfscope}%
\begin{pgfscope}%
\pgfpathrectangle{\pgfqpoint{0.100000in}{0.220728in}}{\pgfqpoint{3.696000in}{3.696000in}}%
\pgfusepath{clip}%
\pgfsetbuttcap%
\pgfsetroundjoin%
\definecolor{currentfill}{rgb}{0.121569,0.466667,0.705882}%
\pgfsetfillcolor{currentfill}%
\pgfsetfillopacity{0.822844}%
\pgfsetlinewidth{1.003750pt}%
\definecolor{currentstroke}{rgb}{0.121569,0.466667,0.705882}%
\pgfsetstrokecolor{currentstroke}%
\pgfsetstrokeopacity{0.822844}%
\pgfsetdash{}{0pt}%
\pgfpathmoveto{\pgfqpoint{0.645732in}{2.606106in}}%
\pgfpathcurveto{\pgfqpoint{0.653968in}{2.606106in}}{\pgfqpoint{0.661868in}{2.609379in}}{\pgfqpoint{0.667692in}{2.615203in}}%
\pgfpathcurveto{\pgfqpoint{0.673516in}{2.621027in}}{\pgfqpoint{0.676788in}{2.628927in}}{\pgfqpoint{0.676788in}{2.637163in}}%
\pgfpathcurveto{\pgfqpoint{0.676788in}{2.645399in}}{\pgfqpoint{0.673516in}{2.653299in}}{\pgfqpoint{0.667692in}{2.659123in}}%
\pgfpathcurveto{\pgfqpoint{0.661868in}{2.664947in}}{\pgfqpoint{0.653968in}{2.668219in}}{\pgfqpoint{0.645732in}{2.668219in}}%
\pgfpathcurveto{\pgfqpoint{0.637495in}{2.668219in}}{\pgfqpoint{0.629595in}{2.664947in}}{\pgfqpoint{0.623771in}{2.659123in}}%
\pgfpathcurveto{\pgfqpoint{0.617947in}{2.653299in}}{\pgfqpoint{0.614675in}{2.645399in}}{\pgfqpoint{0.614675in}{2.637163in}}%
\pgfpathcurveto{\pgfqpoint{0.614675in}{2.628927in}}{\pgfqpoint{0.617947in}{2.621027in}}{\pgfqpoint{0.623771in}{2.615203in}}%
\pgfpathcurveto{\pgfqpoint{0.629595in}{2.609379in}}{\pgfqpoint{0.637495in}{2.606106in}}{\pgfqpoint{0.645732in}{2.606106in}}%
\pgfpathclose%
\pgfusepath{stroke,fill}%
\end{pgfscope}%
\begin{pgfscope}%
\pgfpathrectangle{\pgfqpoint{0.100000in}{0.220728in}}{\pgfqpoint{3.696000in}{3.696000in}}%
\pgfusepath{clip}%
\pgfsetbuttcap%
\pgfsetroundjoin%
\definecolor{currentfill}{rgb}{0.121569,0.466667,0.705882}%
\pgfsetfillcolor{currentfill}%
\pgfsetfillopacity{0.823951}%
\pgfsetlinewidth{1.003750pt}%
\definecolor{currentstroke}{rgb}{0.121569,0.466667,0.705882}%
\pgfsetstrokecolor{currentstroke}%
\pgfsetstrokeopacity{0.823951}%
\pgfsetdash{}{0pt}%
\pgfpathmoveto{\pgfqpoint{2.279444in}{1.543697in}}%
\pgfpathcurveto{\pgfqpoint{2.287681in}{1.543697in}}{\pgfqpoint{2.295581in}{1.546969in}}{\pgfqpoint{2.301405in}{1.552793in}}%
\pgfpathcurveto{\pgfqpoint{2.307229in}{1.558617in}}{\pgfqpoint{2.310501in}{1.566517in}}{\pgfqpoint{2.310501in}{1.574753in}}%
\pgfpathcurveto{\pgfqpoint{2.310501in}{1.582989in}}{\pgfqpoint{2.307229in}{1.590890in}}{\pgfqpoint{2.301405in}{1.596713in}}%
\pgfpathcurveto{\pgfqpoint{2.295581in}{1.602537in}}{\pgfqpoint{2.287681in}{1.605810in}}{\pgfqpoint{2.279444in}{1.605810in}}%
\pgfpathcurveto{\pgfqpoint{2.271208in}{1.605810in}}{\pgfqpoint{2.263308in}{1.602537in}}{\pgfqpoint{2.257484in}{1.596713in}}%
\pgfpathcurveto{\pgfqpoint{2.251660in}{1.590890in}}{\pgfqpoint{2.248388in}{1.582989in}}{\pgfqpoint{2.248388in}{1.574753in}}%
\pgfpathcurveto{\pgfqpoint{2.248388in}{1.566517in}}{\pgfqpoint{2.251660in}{1.558617in}}{\pgfqpoint{2.257484in}{1.552793in}}%
\pgfpathcurveto{\pgfqpoint{2.263308in}{1.546969in}}{\pgfqpoint{2.271208in}{1.543697in}}{\pgfqpoint{2.279444in}{1.543697in}}%
\pgfpathclose%
\pgfusepath{stroke,fill}%
\end{pgfscope}%
\begin{pgfscope}%
\pgfpathrectangle{\pgfqpoint{0.100000in}{0.220728in}}{\pgfqpoint{3.696000in}{3.696000in}}%
\pgfusepath{clip}%
\pgfsetbuttcap%
\pgfsetroundjoin%
\definecolor{currentfill}{rgb}{0.121569,0.466667,0.705882}%
\pgfsetfillcolor{currentfill}%
\pgfsetfillopacity{0.824140}%
\pgfsetlinewidth{1.003750pt}%
\definecolor{currentstroke}{rgb}{0.121569,0.466667,0.705882}%
\pgfsetstrokecolor{currentstroke}%
\pgfsetstrokeopacity{0.824140}%
\pgfsetdash{}{0pt}%
\pgfpathmoveto{\pgfqpoint{0.624273in}{2.630918in}}%
\pgfpathcurveto{\pgfqpoint{0.632509in}{2.630918in}}{\pgfqpoint{0.640409in}{2.634190in}}{\pgfqpoint{0.646233in}{2.640014in}}%
\pgfpathcurveto{\pgfqpoint{0.652057in}{2.645838in}}{\pgfqpoint{0.655329in}{2.653738in}}{\pgfqpoint{0.655329in}{2.661974in}}%
\pgfpathcurveto{\pgfqpoint{0.655329in}{2.670211in}}{\pgfqpoint{0.652057in}{2.678111in}}{\pgfqpoint{0.646233in}{2.683935in}}%
\pgfpathcurveto{\pgfqpoint{0.640409in}{2.689758in}}{\pgfqpoint{0.632509in}{2.693031in}}{\pgfqpoint{0.624273in}{2.693031in}}%
\pgfpathcurveto{\pgfqpoint{0.616036in}{2.693031in}}{\pgfqpoint{0.608136in}{2.689758in}}{\pgfqpoint{0.602312in}{2.683935in}}%
\pgfpathcurveto{\pgfqpoint{0.596489in}{2.678111in}}{\pgfqpoint{0.593216in}{2.670211in}}{\pgfqpoint{0.593216in}{2.661974in}}%
\pgfpathcurveto{\pgfqpoint{0.593216in}{2.653738in}}{\pgfqpoint{0.596489in}{2.645838in}}{\pgfqpoint{0.602312in}{2.640014in}}%
\pgfpathcurveto{\pgfqpoint{0.608136in}{2.634190in}}{\pgfqpoint{0.616036in}{2.630918in}}{\pgfqpoint{0.624273in}{2.630918in}}%
\pgfpathclose%
\pgfusepath{stroke,fill}%
\end{pgfscope}%
\begin{pgfscope}%
\pgfpathrectangle{\pgfqpoint{0.100000in}{0.220728in}}{\pgfqpoint{3.696000in}{3.696000in}}%
\pgfusepath{clip}%
\pgfsetbuttcap%
\pgfsetroundjoin%
\definecolor{currentfill}{rgb}{0.121569,0.466667,0.705882}%
\pgfsetfillcolor{currentfill}%
\pgfsetfillopacity{0.824147}%
\pgfsetlinewidth{1.003750pt}%
\definecolor{currentstroke}{rgb}{0.121569,0.466667,0.705882}%
\pgfsetstrokecolor{currentstroke}%
\pgfsetstrokeopacity{0.824147}%
\pgfsetdash{}{0pt}%
\pgfpathmoveto{\pgfqpoint{0.635681in}{2.624924in}}%
\pgfpathcurveto{\pgfqpoint{0.643917in}{2.624924in}}{\pgfqpoint{0.651817in}{2.628196in}}{\pgfqpoint{0.657641in}{2.634020in}}%
\pgfpathcurveto{\pgfqpoint{0.663465in}{2.639844in}}{\pgfqpoint{0.666737in}{2.647744in}}{\pgfqpoint{0.666737in}{2.655980in}}%
\pgfpathcurveto{\pgfqpoint{0.666737in}{2.664217in}}{\pgfqpoint{0.663465in}{2.672117in}}{\pgfqpoint{0.657641in}{2.677941in}}%
\pgfpathcurveto{\pgfqpoint{0.651817in}{2.683765in}}{\pgfqpoint{0.643917in}{2.687037in}}{\pgfqpoint{0.635681in}{2.687037in}}%
\pgfpathcurveto{\pgfqpoint{0.627444in}{2.687037in}}{\pgfqpoint{0.619544in}{2.683765in}}{\pgfqpoint{0.613720in}{2.677941in}}%
\pgfpathcurveto{\pgfqpoint{0.607897in}{2.672117in}}{\pgfqpoint{0.604624in}{2.664217in}}{\pgfqpoint{0.604624in}{2.655980in}}%
\pgfpathcurveto{\pgfqpoint{0.604624in}{2.647744in}}{\pgfqpoint{0.607897in}{2.639844in}}{\pgfqpoint{0.613720in}{2.634020in}}%
\pgfpathcurveto{\pgfqpoint{0.619544in}{2.628196in}}{\pgfqpoint{0.627444in}{2.624924in}}{\pgfqpoint{0.635681in}{2.624924in}}%
\pgfpathclose%
\pgfusepath{stroke,fill}%
\end{pgfscope}%
\begin{pgfscope}%
\pgfpathrectangle{\pgfqpoint{0.100000in}{0.220728in}}{\pgfqpoint{3.696000in}{3.696000in}}%
\pgfusepath{clip}%
\pgfsetbuttcap%
\pgfsetroundjoin%
\definecolor{currentfill}{rgb}{0.121569,0.466667,0.705882}%
\pgfsetfillcolor{currentfill}%
\pgfsetfillopacity{0.824452}%
\pgfsetlinewidth{1.003750pt}%
\definecolor{currentstroke}{rgb}{0.121569,0.466667,0.705882}%
\pgfsetstrokecolor{currentstroke}%
\pgfsetstrokeopacity{0.824452}%
\pgfsetdash{}{0pt}%
\pgfpathmoveto{\pgfqpoint{2.280062in}{1.542680in}}%
\pgfpathcurveto{\pgfqpoint{2.288298in}{1.542680in}}{\pgfqpoint{2.296199in}{1.545952in}}{\pgfqpoint{2.302022in}{1.551776in}}%
\pgfpathcurveto{\pgfqpoint{2.307846in}{1.557600in}}{\pgfqpoint{2.311119in}{1.565500in}}{\pgfqpoint{2.311119in}{1.573737in}}%
\pgfpathcurveto{\pgfqpoint{2.311119in}{1.581973in}}{\pgfqpoint{2.307846in}{1.589873in}}{\pgfqpoint{2.302022in}{1.595697in}}%
\pgfpathcurveto{\pgfqpoint{2.296199in}{1.601521in}}{\pgfqpoint{2.288298in}{1.604793in}}{\pgfqpoint{2.280062in}{1.604793in}}%
\pgfpathcurveto{\pgfqpoint{2.271826in}{1.604793in}}{\pgfqpoint{2.263926in}{1.601521in}}{\pgfqpoint{2.258102in}{1.595697in}}%
\pgfpathcurveto{\pgfqpoint{2.252278in}{1.589873in}}{\pgfqpoint{2.249006in}{1.581973in}}{\pgfqpoint{2.249006in}{1.573737in}}%
\pgfpathcurveto{\pgfqpoint{2.249006in}{1.565500in}}{\pgfqpoint{2.252278in}{1.557600in}}{\pgfqpoint{2.258102in}{1.551776in}}%
\pgfpathcurveto{\pgfqpoint{2.263926in}{1.545952in}}{\pgfqpoint{2.271826in}{1.542680in}}{\pgfqpoint{2.280062in}{1.542680in}}%
\pgfpathclose%
\pgfusepath{stroke,fill}%
\end{pgfscope}%
\begin{pgfscope}%
\pgfpathrectangle{\pgfqpoint{0.100000in}{0.220728in}}{\pgfqpoint{3.696000in}{3.696000in}}%
\pgfusepath{clip}%
\pgfsetbuttcap%
\pgfsetroundjoin%
\definecolor{currentfill}{rgb}{0.121569,0.466667,0.705882}%
\pgfsetfillcolor{currentfill}%
\pgfsetfillopacity{0.824916}%
\pgfsetlinewidth{1.003750pt}%
\definecolor{currentstroke}{rgb}{0.121569,0.466667,0.705882}%
\pgfsetstrokecolor{currentstroke}%
\pgfsetstrokeopacity{0.824916}%
\pgfsetdash{}{0pt}%
\pgfpathmoveto{\pgfqpoint{0.652050in}{2.604489in}}%
\pgfpathcurveto{\pgfqpoint{0.660286in}{2.604489in}}{\pgfqpoint{0.668186in}{2.607761in}}{\pgfqpoint{0.674010in}{2.613585in}}%
\pgfpathcurveto{\pgfqpoint{0.679834in}{2.619409in}}{\pgfqpoint{0.683106in}{2.627309in}}{\pgfqpoint{0.683106in}{2.635546in}}%
\pgfpathcurveto{\pgfqpoint{0.683106in}{2.643782in}}{\pgfqpoint{0.679834in}{2.651682in}}{\pgfqpoint{0.674010in}{2.657506in}}%
\pgfpathcurveto{\pgfqpoint{0.668186in}{2.663330in}}{\pgfqpoint{0.660286in}{2.666602in}}{\pgfqpoint{0.652050in}{2.666602in}}%
\pgfpathcurveto{\pgfqpoint{0.643814in}{2.666602in}}{\pgfqpoint{0.635914in}{2.663330in}}{\pgfqpoint{0.630090in}{2.657506in}}%
\pgfpathcurveto{\pgfqpoint{0.624266in}{2.651682in}}{\pgfqpoint{0.620993in}{2.643782in}}{\pgfqpoint{0.620993in}{2.635546in}}%
\pgfpathcurveto{\pgfqpoint{0.620993in}{2.627309in}}{\pgfqpoint{0.624266in}{2.619409in}}{\pgfqpoint{0.630090in}{2.613585in}}%
\pgfpathcurveto{\pgfqpoint{0.635914in}{2.607761in}}{\pgfqpoint{0.643814in}{2.604489in}}{\pgfqpoint{0.652050in}{2.604489in}}%
\pgfpathclose%
\pgfusepath{stroke,fill}%
\end{pgfscope}%
\begin{pgfscope}%
\pgfpathrectangle{\pgfqpoint{0.100000in}{0.220728in}}{\pgfqpoint{3.696000in}{3.696000in}}%
\pgfusepath{clip}%
\pgfsetbuttcap%
\pgfsetroundjoin%
\definecolor{currentfill}{rgb}{0.121569,0.466667,0.705882}%
\pgfsetfillcolor{currentfill}%
\pgfsetfillopacity{0.825443}%
\pgfsetlinewidth{1.003750pt}%
\definecolor{currentstroke}{rgb}{0.121569,0.466667,0.705882}%
\pgfsetstrokecolor{currentstroke}%
\pgfsetstrokeopacity{0.825443}%
\pgfsetdash{}{0pt}%
\pgfpathmoveto{\pgfqpoint{2.281044in}{1.543013in}}%
\pgfpathcurveto{\pgfqpoint{2.289280in}{1.543013in}}{\pgfqpoint{2.297180in}{1.546286in}}{\pgfqpoint{2.303004in}{1.552109in}}%
\pgfpathcurveto{\pgfqpoint{2.308828in}{1.557933in}}{\pgfqpoint{2.312100in}{1.565833in}}{\pgfqpoint{2.312100in}{1.574070in}}%
\pgfpathcurveto{\pgfqpoint{2.312100in}{1.582306in}}{\pgfqpoint{2.308828in}{1.590206in}}{\pgfqpoint{2.303004in}{1.596030in}}%
\pgfpathcurveto{\pgfqpoint{2.297180in}{1.601854in}}{\pgfqpoint{2.289280in}{1.605126in}}{\pgfqpoint{2.281044in}{1.605126in}}%
\pgfpathcurveto{\pgfqpoint{2.272807in}{1.605126in}}{\pgfqpoint{2.264907in}{1.601854in}}{\pgfqpoint{2.259083in}{1.596030in}}%
\pgfpathcurveto{\pgfqpoint{2.253259in}{1.590206in}}{\pgfqpoint{2.249987in}{1.582306in}}{\pgfqpoint{2.249987in}{1.574070in}}%
\pgfpathcurveto{\pgfqpoint{2.249987in}{1.565833in}}{\pgfqpoint{2.253259in}{1.557933in}}{\pgfqpoint{2.259083in}{1.552109in}}%
\pgfpathcurveto{\pgfqpoint{2.264907in}{1.546286in}}{\pgfqpoint{2.272807in}{1.543013in}}{\pgfqpoint{2.281044in}{1.543013in}}%
\pgfpathclose%
\pgfusepath{stroke,fill}%
\end{pgfscope}%
\begin{pgfscope}%
\pgfpathrectangle{\pgfqpoint{0.100000in}{0.220728in}}{\pgfqpoint{3.696000in}{3.696000in}}%
\pgfusepath{clip}%
\pgfsetbuttcap%
\pgfsetroundjoin%
\definecolor{currentfill}{rgb}{0.121569,0.466667,0.705882}%
\pgfsetfillcolor{currentfill}%
\pgfsetfillopacity{0.825734}%
\pgfsetlinewidth{1.003750pt}%
\definecolor{currentstroke}{rgb}{0.121569,0.466667,0.705882}%
\pgfsetstrokecolor{currentstroke}%
\pgfsetstrokeopacity{0.825734}%
\pgfsetdash{}{0pt}%
\pgfpathmoveto{\pgfqpoint{0.658315in}{2.597272in}}%
\pgfpathcurveto{\pgfqpoint{0.666551in}{2.597272in}}{\pgfqpoint{0.674451in}{2.600544in}}{\pgfqpoint{0.680275in}{2.606368in}}%
\pgfpathcurveto{\pgfqpoint{0.686099in}{2.612192in}}{\pgfqpoint{0.689371in}{2.620092in}}{\pgfqpoint{0.689371in}{2.628329in}}%
\pgfpathcurveto{\pgfqpoint{0.689371in}{2.636565in}}{\pgfqpoint{0.686099in}{2.644465in}}{\pgfqpoint{0.680275in}{2.650289in}}%
\pgfpathcurveto{\pgfqpoint{0.674451in}{2.656113in}}{\pgfqpoint{0.666551in}{2.659385in}}{\pgfqpoint{0.658315in}{2.659385in}}%
\pgfpathcurveto{\pgfqpoint{0.650078in}{2.659385in}}{\pgfqpoint{0.642178in}{2.656113in}}{\pgfqpoint{0.636354in}{2.650289in}}%
\pgfpathcurveto{\pgfqpoint{0.630530in}{2.644465in}}{\pgfqpoint{0.627258in}{2.636565in}}{\pgfqpoint{0.627258in}{2.628329in}}%
\pgfpathcurveto{\pgfqpoint{0.627258in}{2.620092in}}{\pgfqpoint{0.630530in}{2.612192in}}{\pgfqpoint{0.636354in}{2.606368in}}%
\pgfpathcurveto{\pgfqpoint{0.642178in}{2.600544in}}{\pgfqpoint{0.650078in}{2.597272in}}{\pgfqpoint{0.658315in}{2.597272in}}%
\pgfpathclose%
\pgfusepath{stroke,fill}%
\end{pgfscope}%
\begin{pgfscope}%
\pgfpathrectangle{\pgfqpoint{0.100000in}{0.220728in}}{\pgfqpoint{3.696000in}{3.696000in}}%
\pgfusepath{clip}%
\pgfsetbuttcap%
\pgfsetroundjoin%
\definecolor{currentfill}{rgb}{0.121569,0.466667,0.705882}%
\pgfsetfillcolor{currentfill}%
\pgfsetfillopacity{0.825922}%
\pgfsetlinewidth{1.003750pt}%
\definecolor{currentstroke}{rgb}{0.121569,0.466667,0.705882}%
\pgfsetstrokecolor{currentstroke}%
\pgfsetstrokeopacity{0.825922}%
\pgfsetdash{}{0pt}%
\pgfpathmoveto{\pgfqpoint{2.281249in}{1.542585in}}%
\pgfpathcurveto{\pgfqpoint{2.289486in}{1.542585in}}{\pgfqpoint{2.297386in}{1.545857in}}{\pgfqpoint{2.303210in}{1.551681in}}%
\pgfpathcurveto{\pgfqpoint{2.309034in}{1.557505in}}{\pgfqpoint{2.312306in}{1.565405in}}{\pgfqpoint{2.312306in}{1.573641in}}%
\pgfpathcurveto{\pgfqpoint{2.312306in}{1.581877in}}{\pgfqpoint{2.309034in}{1.589778in}}{\pgfqpoint{2.303210in}{1.595601in}}%
\pgfpathcurveto{\pgfqpoint{2.297386in}{1.601425in}}{\pgfqpoint{2.289486in}{1.604698in}}{\pgfqpoint{2.281249in}{1.604698in}}%
\pgfpathcurveto{\pgfqpoint{2.273013in}{1.604698in}}{\pgfqpoint{2.265113in}{1.601425in}}{\pgfqpoint{2.259289in}{1.595601in}}%
\pgfpathcurveto{\pgfqpoint{2.253465in}{1.589778in}}{\pgfqpoint{2.250193in}{1.581877in}}{\pgfqpoint{2.250193in}{1.573641in}}%
\pgfpathcurveto{\pgfqpoint{2.250193in}{1.565405in}}{\pgfqpoint{2.253465in}{1.557505in}}{\pgfqpoint{2.259289in}{1.551681in}}%
\pgfpathcurveto{\pgfqpoint{2.265113in}{1.545857in}}{\pgfqpoint{2.273013in}{1.542585in}}{\pgfqpoint{2.281249in}{1.542585in}}%
\pgfpathclose%
\pgfusepath{stroke,fill}%
\end{pgfscope}%
\begin{pgfscope}%
\pgfpathrectangle{\pgfqpoint{0.100000in}{0.220728in}}{\pgfqpoint{3.696000in}{3.696000in}}%
\pgfusepath{clip}%
\pgfsetbuttcap%
\pgfsetroundjoin%
\definecolor{currentfill}{rgb}{0.121569,0.466667,0.705882}%
\pgfsetfillcolor{currentfill}%
\pgfsetfillopacity{0.826037}%
\pgfsetlinewidth{1.003750pt}%
\definecolor{currentstroke}{rgb}{0.121569,0.466667,0.705882}%
\pgfsetstrokecolor{currentstroke}%
\pgfsetstrokeopacity{0.826037}%
\pgfsetdash{}{0pt}%
\pgfpathmoveto{\pgfqpoint{0.614009in}{2.646829in}}%
\pgfpathcurveto{\pgfqpoint{0.622245in}{2.646829in}}{\pgfqpoint{0.630145in}{2.650102in}}{\pgfqpoint{0.635969in}{2.655926in}}%
\pgfpathcurveto{\pgfqpoint{0.641793in}{2.661749in}}{\pgfqpoint{0.645065in}{2.669650in}}{\pgfqpoint{0.645065in}{2.677886in}}%
\pgfpathcurveto{\pgfqpoint{0.645065in}{2.686122in}}{\pgfqpoint{0.641793in}{2.694022in}}{\pgfqpoint{0.635969in}{2.699846in}}%
\pgfpathcurveto{\pgfqpoint{0.630145in}{2.705670in}}{\pgfqpoint{0.622245in}{2.708942in}}{\pgfqpoint{0.614009in}{2.708942in}}%
\pgfpathcurveto{\pgfqpoint{0.605772in}{2.708942in}}{\pgfqpoint{0.597872in}{2.705670in}}{\pgfqpoint{0.592048in}{2.699846in}}%
\pgfpathcurveto{\pgfqpoint{0.586224in}{2.694022in}}{\pgfqpoint{0.582952in}{2.686122in}}{\pgfqpoint{0.582952in}{2.677886in}}%
\pgfpathcurveto{\pgfqpoint{0.582952in}{2.669650in}}{\pgfqpoint{0.586224in}{2.661749in}}{\pgfqpoint{0.592048in}{2.655926in}}%
\pgfpathcurveto{\pgfqpoint{0.597872in}{2.650102in}}{\pgfqpoint{0.605772in}{2.646829in}}{\pgfqpoint{0.614009in}{2.646829in}}%
\pgfpathclose%
\pgfusepath{stroke,fill}%
\end{pgfscope}%
\begin{pgfscope}%
\pgfpathrectangle{\pgfqpoint{0.100000in}{0.220728in}}{\pgfqpoint{3.696000in}{3.696000in}}%
\pgfusepath{clip}%
\pgfsetbuttcap%
\pgfsetroundjoin%
\definecolor{currentfill}{rgb}{0.121569,0.466667,0.705882}%
\pgfsetfillcolor{currentfill}%
\pgfsetfillopacity{0.826463}%
\pgfsetlinewidth{1.003750pt}%
\definecolor{currentstroke}{rgb}{0.121569,0.466667,0.705882}%
\pgfsetstrokecolor{currentstroke}%
\pgfsetstrokeopacity{0.826463}%
\pgfsetdash{}{0pt}%
\pgfpathmoveto{\pgfqpoint{0.610810in}{2.648905in}}%
\pgfpathcurveto{\pgfqpoint{0.619047in}{2.648905in}}{\pgfqpoint{0.626947in}{2.652177in}}{\pgfqpoint{0.632771in}{2.658001in}}%
\pgfpathcurveto{\pgfqpoint{0.638595in}{2.663825in}}{\pgfqpoint{0.641867in}{2.671725in}}{\pgfqpoint{0.641867in}{2.679961in}}%
\pgfpathcurveto{\pgfqpoint{0.641867in}{2.688197in}}{\pgfqpoint{0.638595in}{2.696097in}}{\pgfqpoint{0.632771in}{2.701921in}}%
\pgfpathcurveto{\pgfqpoint{0.626947in}{2.707745in}}{\pgfqpoint{0.619047in}{2.711018in}}{\pgfqpoint{0.610810in}{2.711018in}}%
\pgfpathcurveto{\pgfqpoint{0.602574in}{2.711018in}}{\pgfqpoint{0.594674in}{2.707745in}}{\pgfqpoint{0.588850in}{2.701921in}}%
\pgfpathcurveto{\pgfqpoint{0.583026in}{2.696097in}}{\pgfqpoint{0.579754in}{2.688197in}}{\pgfqpoint{0.579754in}{2.679961in}}%
\pgfpathcurveto{\pgfqpoint{0.579754in}{2.671725in}}{\pgfqpoint{0.583026in}{2.663825in}}{\pgfqpoint{0.588850in}{2.658001in}}%
\pgfpathcurveto{\pgfqpoint{0.594674in}{2.652177in}}{\pgfqpoint{0.602574in}{2.648905in}}{\pgfqpoint{0.610810in}{2.648905in}}%
\pgfpathclose%
\pgfusepath{stroke,fill}%
\end{pgfscope}%
\begin{pgfscope}%
\pgfpathrectangle{\pgfqpoint{0.100000in}{0.220728in}}{\pgfqpoint{3.696000in}{3.696000in}}%
\pgfusepath{clip}%
\pgfsetbuttcap%
\pgfsetroundjoin%
\definecolor{currentfill}{rgb}{0.121569,0.466667,0.705882}%
\pgfsetfillcolor{currentfill}%
\pgfsetfillopacity{0.826533}%
\pgfsetlinewidth{1.003750pt}%
\definecolor{currentstroke}{rgb}{0.121569,0.466667,0.705882}%
\pgfsetstrokecolor{currentstroke}%
\pgfsetstrokeopacity{0.826533}%
\pgfsetdash{}{0pt}%
\pgfpathmoveto{\pgfqpoint{0.611889in}{2.649318in}}%
\pgfpathcurveto{\pgfqpoint{0.620125in}{2.649318in}}{\pgfqpoint{0.628025in}{2.652590in}}{\pgfqpoint{0.633849in}{2.658414in}}%
\pgfpathcurveto{\pgfqpoint{0.639673in}{2.664238in}}{\pgfqpoint{0.642945in}{2.672138in}}{\pgfqpoint{0.642945in}{2.680374in}}%
\pgfpathcurveto{\pgfqpoint{0.642945in}{2.688610in}}{\pgfqpoint{0.639673in}{2.696510in}}{\pgfqpoint{0.633849in}{2.702334in}}%
\pgfpathcurveto{\pgfqpoint{0.628025in}{2.708158in}}{\pgfqpoint{0.620125in}{2.711431in}}{\pgfqpoint{0.611889in}{2.711431in}}%
\pgfpathcurveto{\pgfqpoint{0.603652in}{2.711431in}}{\pgfqpoint{0.595752in}{2.708158in}}{\pgfqpoint{0.589928in}{2.702334in}}%
\pgfpathcurveto{\pgfqpoint{0.584105in}{2.696510in}}{\pgfqpoint{0.580832in}{2.688610in}}{\pgfqpoint{0.580832in}{2.680374in}}%
\pgfpathcurveto{\pgfqpoint{0.580832in}{2.672138in}}{\pgfqpoint{0.584105in}{2.664238in}}{\pgfqpoint{0.589928in}{2.658414in}}%
\pgfpathcurveto{\pgfqpoint{0.595752in}{2.652590in}}{\pgfqpoint{0.603652in}{2.649318in}}{\pgfqpoint{0.611889in}{2.649318in}}%
\pgfpathclose%
\pgfusepath{stroke,fill}%
\end{pgfscope}%
\begin{pgfscope}%
\pgfpathrectangle{\pgfqpoint{0.100000in}{0.220728in}}{\pgfqpoint{3.696000in}{3.696000in}}%
\pgfusepath{clip}%
\pgfsetbuttcap%
\pgfsetroundjoin%
\definecolor{currentfill}{rgb}{0.121569,0.466667,0.705882}%
\pgfsetfillcolor{currentfill}%
\pgfsetfillopacity{0.826568}%
\pgfsetlinewidth{1.003750pt}%
\definecolor{currentstroke}{rgb}{0.121569,0.466667,0.705882}%
\pgfsetstrokecolor{currentstroke}%
\pgfsetstrokeopacity{0.826568}%
\pgfsetdash{}{0pt}%
\pgfpathmoveto{\pgfqpoint{2.281684in}{1.541035in}}%
\pgfpathcurveto{\pgfqpoint{2.289921in}{1.541035in}}{\pgfqpoint{2.297821in}{1.544308in}}{\pgfqpoint{2.303645in}{1.550132in}}%
\pgfpathcurveto{\pgfqpoint{2.309469in}{1.555955in}}{\pgfqpoint{2.312741in}{1.563855in}}{\pgfqpoint{2.312741in}{1.572092in}}%
\pgfpathcurveto{\pgfqpoint{2.312741in}{1.580328in}}{\pgfqpoint{2.309469in}{1.588228in}}{\pgfqpoint{2.303645in}{1.594052in}}%
\pgfpathcurveto{\pgfqpoint{2.297821in}{1.599876in}}{\pgfqpoint{2.289921in}{1.603148in}}{\pgfqpoint{2.281684in}{1.603148in}}%
\pgfpathcurveto{\pgfqpoint{2.273448in}{1.603148in}}{\pgfqpoint{2.265548in}{1.599876in}}{\pgfqpoint{2.259724in}{1.594052in}}%
\pgfpathcurveto{\pgfqpoint{2.253900in}{1.588228in}}{\pgfqpoint{2.250628in}{1.580328in}}{\pgfqpoint{2.250628in}{1.572092in}}%
\pgfpathcurveto{\pgfqpoint{2.250628in}{1.563855in}}{\pgfqpoint{2.253900in}{1.555955in}}{\pgfqpoint{2.259724in}{1.550132in}}%
\pgfpathcurveto{\pgfqpoint{2.265548in}{1.544308in}}{\pgfqpoint{2.273448in}{1.541035in}}{\pgfqpoint{2.281684in}{1.541035in}}%
\pgfpathclose%
\pgfusepath{stroke,fill}%
\end{pgfscope}%
\begin{pgfscope}%
\pgfpathrectangle{\pgfqpoint{0.100000in}{0.220728in}}{\pgfqpoint{3.696000in}{3.696000in}}%
\pgfusepath{clip}%
\pgfsetbuttcap%
\pgfsetroundjoin%
\definecolor{currentfill}{rgb}{0.121569,0.466667,0.705882}%
\pgfsetfillcolor{currentfill}%
\pgfsetfillopacity{0.826576}%
\pgfsetlinewidth{1.003750pt}%
\definecolor{currentstroke}{rgb}{0.121569,0.466667,0.705882}%
\pgfsetstrokecolor{currentstroke}%
\pgfsetstrokeopacity{0.826576}%
\pgfsetdash{}{0pt}%
\pgfpathmoveto{\pgfqpoint{0.610257in}{2.648913in}}%
\pgfpathcurveto{\pgfqpoint{0.618494in}{2.648913in}}{\pgfqpoint{0.626394in}{2.652186in}}{\pgfqpoint{0.632218in}{2.658010in}}%
\pgfpathcurveto{\pgfqpoint{0.638042in}{2.663833in}}{\pgfqpoint{0.641314in}{2.671734in}}{\pgfqpoint{0.641314in}{2.679970in}}%
\pgfpathcurveto{\pgfqpoint{0.641314in}{2.688206in}}{\pgfqpoint{0.638042in}{2.696106in}}{\pgfqpoint{0.632218in}{2.701930in}}%
\pgfpathcurveto{\pgfqpoint{0.626394in}{2.707754in}}{\pgfqpoint{0.618494in}{2.711026in}}{\pgfqpoint{0.610257in}{2.711026in}}%
\pgfpathcurveto{\pgfqpoint{0.602021in}{2.711026in}}{\pgfqpoint{0.594121in}{2.707754in}}{\pgfqpoint{0.588297in}{2.701930in}}%
\pgfpathcurveto{\pgfqpoint{0.582473in}{2.696106in}}{\pgfqpoint{0.579201in}{2.688206in}}{\pgfqpoint{0.579201in}{2.679970in}}%
\pgfpathcurveto{\pgfqpoint{0.579201in}{2.671734in}}{\pgfqpoint{0.582473in}{2.663833in}}{\pgfqpoint{0.588297in}{2.658010in}}%
\pgfpathcurveto{\pgfqpoint{0.594121in}{2.652186in}}{\pgfqpoint{0.602021in}{2.648913in}}{\pgfqpoint{0.610257in}{2.648913in}}%
\pgfpathclose%
\pgfusepath{stroke,fill}%
\end{pgfscope}%
\begin{pgfscope}%
\pgfpathrectangle{\pgfqpoint{0.100000in}{0.220728in}}{\pgfqpoint{3.696000in}{3.696000in}}%
\pgfusepath{clip}%
\pgfsetbuttcap%
\pgfsetroundjoin%
\definecolor{currentfill}{rgb}{0.121569,0.466667,0.705882}%
\pgfsetfillcolor{currentfill}%
\pgfsetfillopacity{0.826624}%
\pgfsetlinewidth{1.003750pt}%
\definecolor{currentstroke}{rgb}{0.121569,0.466667,0.705882}%
\pgfsetstrokecolor{currentstroke}%
\pgfsetstrokeopacity{0.826624}%
\pgfsetdash{}{0pt}%
\pgfpathmoveto{\pgfqpoint{0.617087in}{2.647632in}}%
\pgfpathcurveto{\pgfqpoint{0.625323in}{2.647632in}}{\pgfqpoint{0.633223in}{2.650905in}}{\pgfqpoint{0.639047in}{2.656729in}}%
\pgfpathcurveto{\pgfqpoint{0.644871in}{2.662553in}}{\pgfqpoint{0.648143in}{2.670453in}}{\pgfqpoint{0.648143in}{2.678689in}}%
\pgfpathcurveto{\pgfqpoint{0.648143in}{2.686925in}}{\pgfqpoint{0.644871in}{2.694825in}}{\pgfqpoint{0.639047in}{2.700649in}}%
\pgfpathcurveto{\pgfqpoint{0.633223in}{2.706473in}}{\pgfqpoint{0.625323in}{2.709745in}}{\pgfqpoint{0.617087in}{2.709745in}}%
\pgfpathcurveto{\pgfqpoint{0.608850in}{2.709745in}}{\pgfqpoint{0.600950in}{2.706473in}}{\pgfqpoint{0.595126in}{2.700649in}}%
\pgfpathcurveto{\pgfqpoint{0.589302in}{2.694825in}}{\pgfqpoint{0.586030in}{2.686925in}}{\pgfqpoint{0.586030in}{2.678689in}}%
\pgfpathcurveto{\pgfqpoint{0.586030in}{2.670453in}}{\pgfqpoint{0.589302in}{2.662553in}}{\pgfqpoint{0.595126in}{2.656729in}}%
\pgfpathcurveto{\pgfqpoint{0.600950in}{2.650905in}}{\pgfqpoint{0.608850in}{2.647632in}}{\pgfqpoint{0.617087in}{2.647632in}}%
\pgfpathclose%
\pgfusepath{stroke,fill}%
\end{pgfscope}%
\begin{pgfscope}%
\pgfpathrectangle{\pgfqpoint{0.100000in}{0.220728in}}{\pgfqpoint{3.696000in}{3.696000in}}%
\pgfusepath{clip}%
\pgfsetbuttcap%
\pgfsetroundjoin%
\definecolor{currentfill}{rgb}{0.121569,0.466667,0.705882}%
\pgfsetfillcolor{currentfill}%
\pgfsetfillopacity{0.826897}%
\pgfsetlinewidth{1.003750pt}%
\definecolor{currentstroke}{rgb}{0.121569,0.466667,0.705882}%
\pgfsetstrokecolor{currentstroke}%
\pgfsetstrokeopacity{0.826897}%
\pgfsetdash{}{0pt}%
\pgfpathmoveto{\pgfqpoint{0.607750in}{2.647174in}}%
\pgfpathcurveto{\pgfqpoint{0.615987in}{2.647174in}}{\pgfqpoint{0.623887in}{2.650446in}}{\pgfqpoint{0.629711in}{2.656270in}}%
\pgfpathcurveto{\pgfqpoint{0.635535in}{2.662094in}}{\pgfqpoint{0.638807in}{2.669994in}}{\pgfqpoint{0.638807in}{2.678230in}}%
\pgfpathcurveto{\pgfqpoint{0.638807in}{2.686466in}}{\pgfqpoint{0.635535in}{2.694366in}}{\pgfqpoint{0.629711in}{2.700190in}}%
\pgfpathcurveto{\pgfqpoint{0.623887in}{2.706014in}}{\pgfqpoint{0.615987in}{2.709287in}}{\pgfqpoint{0.607750in}{2.709287in}}%
\pgfpathcurveto{\pgfqpoint{0.599514in}{2.709287in}}{\pgfqpoint{0.591614in}{2.706014in}}{\pgfqpoint{0.585790in}{2.700190in}}%
\pgfpathcurveto{\pgfqpoint{0.579966in}{2.694366in}}{\pgfqpoint{0.576694in}{2.686466in}}{\pgfqpoint{0.576694in}{2.678230in}}%
\pgfpathcurveto{\pgfqpoint{0.576694in}{2.669994in}}{\pgfqpoint{0.579966in}{2.662094in}}{\pgfqpoint{0.585790in}{2.656270in}}%
\pgfpathcurveto{\pgfqpoint{0.591614in}{2.650446in}}{\pgfqpoint{0.599514in}{2.647174in}}{\pgfqpoint{0.607750in}{2.647174in}}%
\pgfpathclose%
\pgfusepath{stroke,fill}%
\end{pgfscope}%
\begin{pgfscope}%
\pgfpathrectangle{\pgfqpoint{0.100000in}{0.220728in}}{\pgfqpoint{3.696000in}{3.696000in}}%
\pgfusepath{clip}%
\pgfsetbuttcap%
\pgfsetroundjoin%
\definecolor{currentfill}{rgb}{0.121569,0.466667,0.705882}%
\pgfsetfillcolor{currentfill}%
\pgfsetfillopacity{0.827162}%
\pgfsetlinewidth{1.003750pt}%
\definecolor{currentstroke}{rgb}{0.121569,0.466667,0.705882}%
\pgfsetstrokecolor{currentstroke}%
\pgfsetstrokeopacity{0.827162}%
\pgfsetdash{}{0pt}%
\pgfpathmoveto{\pgfqpoint{2.282685in}{1.538292in}}%
\pgfpathcurveto{\pgfqpoint{2.290921in}{1.538292in}}{\pgfqpoint{2.298821in}{1.541564in}}{\pgfqpoint{2.304645in}{1.547388in}}%
\pgfpathcurveto{\pgfqpoint{2.310469in}{1.553212in}}{\pgfqpoint{2.313742in}{1.561112in}}{\pgfqpoint{2.313742in}{1.569348in}}%
\pgfpathcurveto{\pgfqpoint{2.313742in}{1.577584in}}{\pgfqpoint{2.310469in}{1.585484in}}{\pgfqpoint{2.304645in}{1.591308in}}%
\pgfpathcurveto{\pgfqpoint{2.298821in}{1.597132in}}{\pgfqpoint{2.290921in}{1.600405in}}{\pgfqpoint{2.282685in}{1.600405in}}%
\pgfpathcurveto{\pgfqpoint{2.274449in}{1.600405in}}{\pgfqpoint{2.266549in}{1.597132in}}{\pgfqpoint{2.260725in}{1.591308in}}%
\pgfpathcurveto{\pgfqpoint{2.254901in}{1.585484in}}{\pgfqpoint{2.251629in}{1.577584in}}{\pgfqpoint{2.251629in}{1.569348in}}%
\pgfpathcurveto{\pgfqpoint{2.251629in}{1.561112in}}{\pgfqpoint{2.254901in}{1.553212in}}{\pgfqpoint{2.260725in}{1.547388in}}%
\pgfpathcurveto{\pgfqpoint{2.266549in}{1.541564in}}{\pgfqpoint{2.274449in}{1.538292in}}{\pgfqpoint{2.282685in}{1.538292in}}%
\pgfpathclose%
\pgfusepath{stroke,fill}%
\end{pgfscope}%
\begin{pgfscope}%
\pgfpathrectangle{\pgfqpoint{0.100000in}{0.220728in}}{\pgfqpoint{3.696000in}{3.696000in}}%
\pgfusepath{clip}%
\pgfsetbuttcap%
\pgfsetroundjoin%
\definecolor{currentfill}{rgb}{0.121569,0.466667,0.705882}%
\pgfsetfillcolor{currentfill}%
\pgfsetfillopacity{0.827386}%
\pgfsetlinewidth{1.003750pt}%
\definecolor{currentstroke}{rgb}{0.121569,0.466667,0.705882}%
\pgfsetstrokecolor{currentstroke}%
\pgfsetstrokeopacity{0.827386}%
\pgfsetdash{}{0pt}%
\pgfpathmoveto{\pgfqpoint{0.603932in}{2.644523in}}%
\pgfpathcurveto{\pgfqpoint{0.612168in}{2.644523in}}{\pgfqpoint{0.620068in}{2.647796in}}{\pgfqpoint{0.625892in}{2.653620in}}%
\pgfpathcurveto{\pgfqpoint{0.631716in}{2.659444in}}{\pgfqpoint{0.634988in}{2.667344in}}{\pgfqpoint{0.634988in}{2.675580in}}%
\pgfpathcurveto{\pgfqpoint{0.634988in}{2.683816in}}{\pgfqpoint{0.631716in}{2.691716in}}{\pgfqpoint{0.625892in}{2.697540in}}%
\pgfpathcurveto{\pgfqpoint{0.620068in}{2.703364in}}{\pgfqpoint{0.612168in}{2.706636in}}{\pgfqpoint{0.603932in}{2.706636in}}%
\pgfpathcurveto{\pgfqpoint{0.595695in}{2.706636in}}{\pgfqpoint{0.587795in}{2.703364in}}{\pgfqpoint{0.581971in}{2.697540in}}%
\pgfpathcurveto{\pgfqpoint{0.576147in}{2.691716in}}{\pgfqpoint{0.572875in}{2.683816in}}{\pgfqpoint{0.572875in}{2.675580in}}%
\pgfpathcurveto{\pgfqpoint{0.572875in}{2.667344in}}{\pgfqpoint{0.576147in}{2.659444in}}{\pgfqpoint{0.581971in}{2.653620in}}%
\pgfpathcurveto{\pgfqpoint{0.587795in}{2.647796in}}{\pgfqpoint{0.595695in}{2.644523in}}{\pgfqpoint{0.603932in}{2.644523in}}%
\pgfpathclose%
\pgfusepath{stroke,fill}%
\end{pgfscope}%
\begin{pgfscope}%
\pgfpathrectangle{\pgfqpoint{0.100000in}{0.220728in}}{\pgfqpoint{3.696000in}{3.696000in}}%
\pgfusepath{clip}%
\pgfsetbuttcap%
\pgfsetroundjoin%
\definecolor{currentfill}{rgb}{0.121569,0.466667,0.705882}%
\pgfsetfillcolor{currentfill}%
\pgfsetfillopacity{0.828276}%
\pgfsetlinewidth{1.003750pt}%
\definecolor{currentstroke}{rgb}{0.121569,0.466667,0.705882}%
\pgfsetstrokecolor{currentstroke}%
\pgfsetstrokeopacity{0.828276}%
\pgfsetdash{}{0pt}%
\pgfpathmoveto{\pgfqpoint{0.669862in}{2.591868in}}%
\pgfpathcurveto{\pgfqpoint{0.678098in}{2.591868in}}{\pgfqpoint{0.685998in}{2.595141in}}{\pgfqpoint{0.691822in}{2.600965in}}%
\pgfpathcurveto{\pgfqpoint{0.697646in}{2.606789in}}{\pgfqpoint{0.700919in}{2.614689in}}{\pgfqpoint{0.700919in}{2.622925in}}%
\pgfpathcurveto{\pgfqpoint{0.700919in}{2.631161in}}{\pgfqpoint{0.697646in}{2.639061in}}{\pgfqpoint{0.691822in}{2.644885in}}%
\pgfpathcurveto{\pgfqpoint{0.685998in}{2.650709in}}{\pgfqpoint{0.678098in}{2.653981in}}{\pgfqpoint{0.669862in}{2.653981in}}%
\pgfpathcurveto{\pgfqpoint{0.661626in}{2.653981in}}{\pgfqpoint{0.653726in}{2.650709in}}{\pgfqpoint{0.647902in}{2.644885in}}%
\pgfpathcurveto{\pgfqpoint{0.642078in}{2.639061in}}{\pgfqpoint{0.638806in}{2.631161in}}{\pgfqpoint{0.638806in}{2.622925in}}%
\pgfpathcurveto{\pgfqpoint{0.638806in}{2.614689in}}{\pgfqpoint{0.642078in}{2.606789in}}{\pgfqpoint{0.647902in}{2.600965in}}%
\pgfpathcurveto{\pgfqpoint{0.653726in}{2.595141in}}{\pgfqpoint{0.661626in}{2.591868in}}{\pgfqpoint{0.669862in}{2.591868in}}%
\pgfpathclose%
\pgfusepath{stroke,fill}%
\end{pgfscope}%
\begin{pgfscope}%
\pgfpathrectangle{\pgfqpoint{0.100000in}{0.220728in}}{\pgfqpoint{3.696000in}{3.696000in}}%
\pgfusepath{clip}%
\pgfsetbuttcap%
\pgfsetroundjoin%
\definecolor{currentfill}{rgb}{0.121569,0.466667,0.705882}%
\pgfsetfillcolor{currentfill}%
\pgfsetfillopacity{0.828617}%
\pgfsetlinewidth{1.003750pt}%
\definecolor{currentstroke}{rgb}{0.121569,0.466667,0.705882}%
\pgfsetstrokecolor{currentstroke}%
\pgfsetstrokeopacity{0.828617}%
\pgfsetdash{}{0pt}%
\pgfpathmoveto{\pgfqpoint{2.283736in}{1.536887in}}%
\pgfpathcurveto{\pgfqpoint{2.291973in}{1.536887in}}{\pgfqpoint{2.299873in}{1.540159in}}{\pgfqpoint{2.305697in}{1.545983in}}%
\pgfpathcurveto{\pgfqpoint{2.311521in}{1.551807in}}{\pgfqpoint{2.314793in}{1.559707in}}{\pgfqpoint{2.314793in}{1.567944in}}%
\pgfpathcurveto{\pgfqpoint{2.314793in}{1.576180in}}{\pgfqpoint{2.311521in}{1.584080in}}{\pgfqpoint{2.305697in}{1.589904in}}%
\pgfpathcurveto{\pgfqpoint{2.299873in}{1.595728in}}{\pgfqpoint{2.291973in}{1.599000in}}{\pgfqpoint{2.283736in}{1.599000in}}%
\pgfpathcurveto{\pgfqpoint{2.275500in}{1.599000in}}{\pgfqpoint{2.267600in}{1.595728in}}{\pgfqpoint{2.261776in}{1.589904in}}%
\pgfpathcurveto{\pgfqpoint{2.255952in}{1.584080in}}{\pgfqpoint{2.252680in}{1.576180in}}{\pgfqpoint{2.252680in}{1.567944in}}%
\pgfpathcurveto{\pgfqpoint{2.252680in}{1.559707in}}{\pgfqpoint{2.255952in}{1.551807in}}{\pgfqpoint{2.261776in}{1.545983in}}%
\pgfpathcurveto{\pgfqpoint{2.267600in}{1.540159in}}{\pgfqpoint{2.275500in}{1.536887in}}{\pgfqpoint{2.283736in}{1.536887in}}%
\pgfpathclose%
\pgfusepath{stroke,fill}%
\end{pgfscope}%
\begin{pgfscope}%
\pgfpathrectangle{\pgfqpoint{0.100000in}{0.220728in}}{\pgfqpoint{3.696000in}{3.696000in}}%
\pgfusepath{clip}%
\pgfsetbuttcap%
\pgfsetroundjoin%
\definecolor{currentfill}{rgb}{0.121569,0.466667,0.705882}%
\pgfsetfillcolor{currentfill}%
\pgfsetfillopacity{0.829106}%
\pgfsetlinewidth{1.003750pt}%
\definecolor{currentstroke}{rgb}{0.121569,0.466667,0.705882}%
\pgfsetstrokecolor{currentstroke}%
\pgfsetstrokeopacity{0.829106}%
\pgfsetdash{}{0pt}%
\pgfpathmoveto{\pgfqpoint{0.682250in}{2.578436in}}%
\pgfpathcurveto{\pgfqpoint{0.690487in}{2.578436in}}{\pgfqpoint{0.698387in}{2.581709in}}{\pgfqpoint{0.704211in}{2.587533in}}%
\pgfpathcurveto{\pgfqpoint{0.710035in}{2.593356in}}{\pgfqpoint{0.713307in}{2.601256in}}{\pgfqpoint{0.713307in}{2.609493in}}%
\pgfpathcurveto{\pgfqpoint{0.713307in}{2.617729in}}{\pgfqpoint{0.710035in}{2.625629in}}{\pgfqpoint{0.704211in}{2.631453in}}%
\pgfpathcurveto{\pgfqpoint{0.698387in}{2.637277in}}{\pgfqpoint{0.690487in}{2.640549in}}{\pgfqpoint{0.682250in}{2.640549in}}%
\pgfpathcurveto{\pgfqpoint{0.674014in}{2.640549in}}{\pgfqpoint{0.666114in}{2.637277in}}{\pgfqpoint{0.660290in}{2.631453in}}%
\pgfpathcurveto{\pgfqpoint{0.654466in}{2.625629in}}{\pgfqpoint{0.651194in}{2.617729in}}{\pgfqpoint{0.651194in}{2.609493in}}%
\pgfpathcurveto{\pgfqpoint{0.651194in}{2.601256in}}{\pgfqpoint{0.654466in}{2.593356in}}{\pgfqpoint{0.660290in}{2.587533in}}%
\pgfpathcurveto{\pgfqpoint{0.666114in}{2.581709in}}{\pgfqpoint{0.674014in}{2.578436in}}{\pgfqpoint{0.682250in}{2.578436in}}%
\pgfpathclose%
\pgfusepath{stroke,fill}%
\end{pgfscope}%
\begin{pgfscope}%
\pgfpathrectangle{\pgfqpoint{0.100000in}{0.220728in}}{\pgfqpoint{3.696000in}{3.696000in}}%
\pgfusepath{clip}%
\pgfsetbuttcap%
\pgfsetroundjoin%
\definecolor{currentfill}{rgb}{0.121569,0.466667,0.705882}%
\pgfsetfillcolor{currentfill}%
\pgfsetfillopacity{0.830864}%
\pgfsetlinewidth{1.003750pt}%
\definecolor{currentstroke}{rgb}{0.121569,0.466667,0.705882}%
\pgfsetstrokecolor{currentstroke}%
\pgfsetstrokeopacity{0.830864}%
\pgfsetdash{}{0pt}%
\pgfpathmoveto{\pgfqpoint{2.284900in}{1.535036in}}%
\pgfpathcurveto{\pgfqpoint{2.293136in}{1.535036in}}{\pgfqpoint{2.301036in}{1.538308in}}{\pgfqpoint{2.306860in}{1.544132in}}%
\pgfpathcurveto{\pgfqpoint{2.312684in}{1.549956in}}{\pgfqpoint{2.315956in}{1.557856in}}{\pgfqpoint{2.315956in}{1.566092in}}%
\pgfpathcurveto{\pgfqpoint{2.315956in}{1.574329in}}{\pgfqpoint{2.312684in}{1.582229in}}{\pgfqpoint{2.306860in}{1.588053in}}%
\pgfpathcurveto{\pgfqpoint{2.301036in}{1.593877in}}{\pgfqpoint{2.293136in}{1.597149in}}{\pgfqpoint{2.284900in}{1.597149in}}%
\pgfpathcurveto{\pgfqpoint{2.276663in}{1.597149in}}{\pgfqpoint{2.268763in}{1.593877in}}{\pgfqpoint{2.262939in}{1.588053in}}%
\pgfpathcurveto{\pgfqpoint{2.257115in}{1.582229in}}{\pgfqpoint{2.253843in}{1.574329in}}{\pgfqpoint{2.253843in}{1.566092in}}%
\pgfpathcurveto{\pgfqpoint{2.253843in}{1.557856in}}{\pgfqpoint{2.257115in}{1.549956in}}{\pgfqpoint{2.262939in}{1.544132in}}%
\pgfpathcurveto{\pgfqpoint{2.268763in}{1.538308in}}{\pgfqpoint{2.276663in}{1.535036in}}{\pgfqpoint{2.284900in}{1.535036in}}%
\pgfpathclose%
\pgfusepath{stroke,fill}%
\end{pgfscope}%
\begin{pgfscope}%
\pgfpathrectangle{\pgfqpoint{0.100000in}{0.220728in}}{\pgfqpoint{3.696000in}{3.696000in}}%
\pgfusepath{clip}%
\pgfsetbuttcap%
\pgfsetroundjoin%
\definecolor{currentfill}{rgb}{0.121569,0.466667,0.705882}%
\pgfsetfillcolor{currentfill}%
\pgfsetfillopacity{0.833497}%
\pgfsetlinewidth{1.003750pt}%
\definecolor{currentstroke}{rgb}{0.121569,0.466667,0.705882}%
\pgfsetstrokecolor{currentstroke}%
\pgfsetstrokeopacity{0.833497}%
\pgfsetdash{}{0pt}%
\pgfpathmoveto{\pgfqpoint{2.287026in}{1.532426in}}%
\pgfpathcurveto{\pgfqpoint{2.295262in}{1.532426in}}{\pgfqpoint{2.303162in}{1.535698in}}{\pgfqpoint{2.308986in}{1.541522in}}%
\pgfpathcurveto{\pgfqpoint{2.314810in}{1.547346in}}{\pgfqpoint{2.318082in}{1.555246in}}{\pgfqpoint{2.318082in}{1.563482in}}%
\pgfpathcurveto{\pgfqpoint{2.318082in}{1.571719in}}{\pgfqpoint{2.314810in}{1.579619in}}{\pgfqpoint{2.308986in}{1.585443in}}%
\pgfpathcurveto{\pgfqpoint{2.303162in}{1.591267in}}{\pgfqpoint{2.295262in}{1.594539in}}{\pgfqpoint{2.287026in}{1.594539in}}%
\pgfpathcurveto{\pgfqpoint{2.278789in}{1.594539in}}{\pgfqpoint{2.270889in}{1.591267in}}{\pgfqpoint{2.265065in}{1.585443in}}%
\pgfpathcurveto{\pgfqpoint{2.259241in}{1.579619in}}{\pgfqpoint{2.255969in}{1.571719in}}{\pgfqpoint{2.255969in}{1.563482in}}%
\pgfpathcurveto{\pgfqpoint{2.255969in}{1.555246in}}{\pgfqpoint{2.259241in}{1.547346in}}{\pgfqpoint{2.265065in}{1.541522in}}%
\pgfpathcurveto{\pgfqpoint{2.270889in}{1.535698in}}{\pgfqpoint{2.278789in}{1.532426in}}{\pgfqpoint{2.287026in}{1.532426in}}%
\pgfpathclose%
\pgfusepath{stroke,fill}%
\end{pgfscope}%
\begin{pgfscope}%
\pgfpathrectangle{\pgfqpoint{0.100000in}{0.220728in}}{\pgfqpoint{3.696000in}{3.696000in}}%
\pgfusepath{clip}%
\pgfsetbuttcap%
\pgfsetroundjoin%
\definecolor{currentfill}{rgb}{0.121569,0.466667,0.705882}%
\pgfsetfillcolor{currentfill}%
\pgfsetfillopacity{0.833741}%
\pgfsetlinewidth{1.003750pt}%
\definecolor{currentstroke}{rgb}{0.121569,0.466667,0.705882}%
\pgfsetstrokecolor{currentstroke}%
\pgfsetstrokeopacity{0.833741}%
\pgfsetdash{}{0pt}%
\pgfpathmoveto{\pgfqpoint{0.703063in}{2.571447in}}%
\pgfpathcurveto{\pgfqpoint{0.711299in}{2.571447in}}{\pgfqpoint{0.719199in}{2.574720in}}{\pgfqpoint{0.725023in}{2.580544in}}%
\pgfpathcurveto{\pgfqpoint{0.730847in}{2.586367in}}{\pgfqpoint{0.734120in}{2.594268in}}{\pgfqpoint{0.734120in}{2.602504in}}%
\pgfpathcurveto{\pgfqpoint{0.734120in}{2.610740in}}{\pgfqpoint{0.730847in}{2.618640in}}{\pgfqpoint{0.725023in}{2.624464in}}%
\pgfpathcurveto{\pgfqpoint{0.719199in}{2.630288in}}{\pgfqpoint{0.711299in}{2.633560in}}{\pgfqpoint{0.703063in}{2.633560in}}%
\pgfpathcurveto{\pgfqpoint{0.694827in}{2.633560in}}{\pgfqpoint{0.686927in}{2.630288in}}{\pgfqpoint{0.681103in}{2.624464in}}%
\pgfpathcurveto{\pgfqpoint{0.675279in}{2.618640in}}{\pgfqpoint{0.672007in}{2.610740in}}{\pgfqpoint{0.672007in}{2.602504in}}%
\pgfpathcurveto{\pgfqpoint{0.672007in}{2.594268in}}{\pgfqpoint{0.675279in}{2.586367in}}{\pgfqpoint{0.681103in}{2.580544in}}%
\pgfpathcurveto{\pgfqpoint{0.686927in}{2.574720in}}{\pgfqpoint{0.694827in}{2.571447in}}{\pgfqpoint{0.703063in}{2.571447in}}%
\pgfpathclose%
\pgfusepath{stroke,fill}%
\end{pgfscope}%
\begin{pgfscope}%
\pgfpathrectangle{\pgfqpoint{0.100000in}{0.220728in}}{\pgfqpoint{3.696000in}{3.696000in}}%
\pgfusepath{clip}%
\pgfsetbuttcap%
\pgfsetroundjoin%
\definecolor{currentfill}{rgb}{0.121569,0.466667,0.705882}%
\pgfsetfillcolor{currentfill}%
\pgfsetfillopacity{0.834912}%
\pgfsetlinewidth{1.003750pt}%
\definecolor{currentstroke}{rgb}{0.121569,0.466667,0.705882}%
\pgfsetstrokecolor{currentstroke}%
\pgfsetstrokeopacity{0.834912}%
\pgfsetdash{}{0pt}%
\pgfpathmoveto{\pgfqpoint{2.287872in}{1.530602in}}%
\pgfpathcurveto{\pgfqpoint{2.296108in}{1.530602in}}{\pgfqpoint{2.304008in}{1.533875in}}{\pgfqpoint{2.309832in}{1.539699in}}%
\pgfpathcurveto{\pgfqpoint{2.315656in}{1.545522in}}{\pgfqpoint{2.318928in}{1.553423in}}{\pgfqpoint{2.318928in}{1.561659in}}%
\pgfpathcurveto{\pgfqpoint{2.318928in}{1.569895in}}{\pgfqpoint{2.315656in}{1.577795in}}{\pgfqpoint{2.309832in}{1.583619in}}%
\pgfpathcurveto{\pgfqpoint{2.304008in}{1.589443in}}{\pgfqpoint{2.296108in}{1.592715in}}{\pgfqpoint{2.287872in}{1.592715in}}%
\pgfpathcurveto{\pgfqpoint{2.279635in}{1.592715in}}{\pgfqpoint{2.271735in}{1.589443in}}{\pgfqpoint{2.265911in}{1.583619in}}%
\pgfpathcurveto{\pgfqpoint{2.260088in}{1.577795in}}{\pgfqpoint{2.256815in}{1.569895in}}{\pgfqpoint{2.256815in}{1.561659in}}%
\pgfpathcurveto{\pgfqpoint{2.256815in}{1.553423in}}{\pgfqpoint{2.260088in}{1.545522in}}{\pgfqpoint{2.265911in}{1.539699in}}%
\pgfpathcurveto{\pgfqpoint{2.271735in}{1.533875in}}{\pgfqpoint{2.279635in}{1.530602in}}{\pgfqpoint{2.287872in}{1.530602in}}%
\pgfpathclose%
\pgfusepath{stroke,fill}%
\end{pgfscope}%
\begin{pgfscope}%
\pgfpathrectangle{\pgfqpoint{0.100000in}{0.220728in}}{\pgfqpoint{3.696000in}{3.696000in}}%
\pgfusepath{clip}%
\pgfsetbuttcap%
\pgfsetroundjoin%
\definecolor{currentfill}{rgb}{0.121569,0.466667,0.705882}%
\pgfsetfillcolor{currentfill}%
\pgfsetfillopacity{0.836434}%
\pgfsetlinewidth{1.003750pt}%
\definecolor{currentstroke}{rgb}{0.121569,0.466667,0.705882}%
\pgfsetstrokecolor{currentstroke}%
\pgfsetstrokeopacity{0.836434}%
\pgfsetdash{}{0pt}%
\pgfpathmoveto{\pgfqpoint{2.288891in}{1.526822in}}%
\pgfpathcurveto{\pgfqpoint{2.297127in}{1.526822in}}{\pgfqpoint{2.305027in}{1.530094in}}{\pgfqpoint{2.310851in}{1.535918in}}%
\pgfpathcurveto{\pgfqpoint{2.316675in}{1.541742in}}{\pgfqpoint{2.319947in}{1.549642in}}{\pgfqpoint{2.319947in}{1.557879in}}%
\pgfpathcurveto{\pgfqpoint{2.319947in}{1.566115in}}{\pgfqpoint{2.316675in}{1.574015in}}{\pgfqpoint{2.310851in}{1.579839in}}%
\pgfpathcurveto{\pgfqpoint{2.305027in}{1.585663in}}{\pgfqpoint{2.297127in}{1.588935in}}{\pgfqpoint{2.288891in}{1.588935in}}%
\pgfpathcurveto{\pgfqpoint{2.280655in}{1.588935in}}{\pgfqpoint{2.272755in}{1.585663in}}{\pgfqpoint{2.266931in}{1.579839in}}%
\pgfpathcurveto{\pgfqpoint{2.261107in}{1.574015in}}{\pgfqpoint{2.257834in}{1.566115in}}{\pgfqpoint{2.257834in}{1.557879in}}%
\pgfpathcurveto{\pgfqpoint{2.257834in}{1.549642in}}{\pgfqpoint{2.261107in}{1.541742in}}{\pgfqpoint{2.266931in}{1.535918in}}%
\pgfpathcurveto{\pgfqpoint{2.272755in}{1.530094in}}{\pgfqpoint{2.280655in}{1.526822in}}{\pgfqpoint{2.288891in}{1.526822in}}%
\pgfpathclose%
\pgfusepath{stroke,fill}%
\end{pgfscope}%
\begin{pgfscope}%
\pgfpathrectangle{\pgfqpoint{0.100000in}{0.220728in}}{\pgfqpoint{3.696000in}{3.696000in}}%
\pgfusepath{clip}%
\pgfsetbuttcap%
\pgfsetroundjoin%
\definecolor{currentfill}{rgb}{0.121569,0.466667,0.705882}%
\pgfsetfillcolor{currentfill}%
\pgfsetfillopacity{0.836693}%
\pgfsetlinewidth{1.003750pt}%
\definecolor{currentstroke}{rgb}{0.121569,0.466667,0.705882}%
\pgfsetstrokecolor{currentstroke}%
\pgfsetstrokeopacity{0.836693}%
\pgfsetdash{}{0pt}%
\pgfpathmoveto{\pgfqpoint{0.744444in}{2.529408in}}%
\pgfpathcurveto{\pgfqpoint{0.752681in}{2.529408in}}{\pgfqpoint{0.760581in}{2.532680in}}{\pgfqpoint{0.766404in}{2.538504in}}%
\pgfpathcurveto{\pgfqpoint{0.772228in}{2.544328in}}{\pgfqpoint{0.775501in}{2.552228in}}{\pgfqpoint{0.775501in}{2.560464in}}%
\pgfpathcurveto{\pgfqpoint{0.775501in}{2.568701in}}{\pgfqpoint{0.772228in}{2.576601in}}{\pgfqpoint{0.766404in}{2.582425in}}%
\pgfpathcurveto{\pgfqpoint{0.760581in}{2.588249in}}{\pgfqpoint{0.752681in}{2.591521in}}{\pgfqpoint{0.744444in}{2.591521in}}%
\pgfpathcurveto{\pgfqpoint{0.736208in}{2.591521in}}{\pgfqpoint{0.728308in}{2.588249in}}{\pgfqpoint{0.722484in}{2.582425in}}%
\pgfpathcurveto{\pgfqpoint{0.716660in}{2.576601in}}{\pgfqpoint{0.713388in}{2.568701in}}{\pgfqpoint{0.713388in}{2.560464in}}%
\pgfpathcurveto{\pgfqpoint{0.713388in}{2.552228in}}{\pgfqpoint{0.716660in}{2.544328in}}{\pgfqpoint{0.722484in}{2.538504in}}%
\pgfpathcurveto{\pgfqpoint{0.728308in}{2.532680in}}{\pgfqpoint{0.736208in}{2.529408in}}{\pgfqpoint{0.744444in}{2.529408in}}%
\pgfpathclose%
\pgfusepath{stroke,fill}%
\end{pgfscope}%
\begin{pgfscope}%
\pgfpathrectangle{\pgfqpoint{0.100000in}{0.220728in}}{\pgfqpoint{3.696000in}{3.696000in}}%
\pgfusepath{clip}%
\pgfsetbuttcap%
\pgfsetroundjoin%
\definecolor{currentfill}{rgb}{0.121569,0.466667,0.705882}%
\pgfsetfillcolor{currentfill}%
\pgfsetfillopacity{0.837571}%
\pgfsetlinewidth{1.003750pt}%
\definecolor{currentstroke}{rgb}{0.121569,0.466667,0.705882}%
\pgfsetstrokecolor{currentstroke}%
\pgfsetstrokeopacity{0.837571}%
\pgfsetdash{}{0pt}%
\pgfpathmoveto{\pgfqpoint{2.289574in}{1.526663in}}%
\pgfpathcurveto{\pgfqpoint{2.297810in}{1.526663in}}{\pgfqpoint{2.305710in}{1.529935in}}{\pgfqpoint{2.311534in}{1.535759in}}%
\pgfpathcurveto{\pgfqpoint{2.317358in}{1.541583in}}{\pgfqpoint{2.320631in}{1.549483in}}{\pgfqpoint{2.320631in}{1.557719in}}%
\pgfpathcurveto{\pgfqpoint{2.320631in}{1.565956in}}{\pgfqpoint{2.317358in}{1.573856in}}{\pgfqpoint{2.311534in}{1.579680in}}%
\pgfpathcurveto{\pgfqpoint{2.305710in}{1.585503in}}{\pgfqpoint{2.297810in}{1.588776in}}{\pgfqpoint{2.289574in}{1.588776in}}%
\pgfpathcurveto{\pgfqpoint{2.281338in}{1.588776in}}{\pgfqpoint{2.273438in}{1.585503in}}{\pgfqpoint{2.267614in}{1.579680in}}%
\pgfpathcurveto{\pgfqpoint{2.261790in}{1.573856in}}{\pgfqpoint{2.258518in}{1.565956in}}{\pgfqpoint{2.258518in}{1.557719in}}%
\pgfpathcurveto{\pgfqpoint{2.258518in}{1.549483in}}{\pgfqpoint{2.261790in}{1.541583in}}{\pgfqpoint{2.267614in}{1.535759in}}%
\pgfpathcurveto{\pgfqpoint{2.273438in}{1.529935in}}{\pgfqpoint{2.281338in}{1.526663in}}{\pgfqpoint{2.289574in}{1.526663in}}%
\pgfpathclose%
\pgfusepath{stroke,fill}%
\end{pgfscope}%
\begin{pgfscope}%
\pgfpathrectangle{\pgfqpoint{0.100000in}{0.220728in}}{\pgfqpoint{3.696000in}{3.696000in}}%
\pgfusepath{clip}%
\pgfsetbuttcap%
\pgfsetroundjoin%
\definecolor{currentfill}{rgb}{0.121569,0.466667,0.705882}%
\pgfsetfillcolor{currentfill}%
\pgfsetfillopacity{0.838813}%
\pgfsetlinewidth{1.003750pt}%
\definecolor{currentstroke}{rgb}{0.121569,0.466667,0.705882}%
\pgfsetstrokecolor{currentstroke}%
\pgfsetstrokeopacity{0.838813}%
\pgfsetdash{}{0pt}%
\pgfpathmoveto{\pgfqpoint{2.290350in}{1.525552in}}%
\pgfpathcurveto{\pgfqpoint{2.298586in}{1.525552in}}{\pgfqpoint{2.306486in}{1.528824in}}{\pgfqpoint{2.312310in}{1.534648in}}%
\pgfpathcurveto{\pgfqpoint{2.318134in}{1.540472in}}{\pgfqpoint{2.321406in}{1.548372in}}{\pgfqpoint{2.321406in}{1.556608in}}%
\pgfpathcurveto{\pgfqpoint{2.321406in}{1.564844in}}{\pgfqpoint{2.318134in}{1.572744in}}{\pgfqpoint{2.312310in}{1.578568in}}%
\pgfpathcurveto{\pgfqpoint{2.306486in}{1.584392in}}{\pgfqpoint{2.298586in}{1.587665in}}{\pgfqpoint{2.290350in}{1.587665in}}%
\pgfpathcurveto{\pgfqpoint{2.282113in}{1.587665in}}{\pgfqpoint{2.274213in}{1.584392in}}{\pgfqpoint{2.268389in}{1.578568in}}%
\pgfpathcurveto{\pgfqpoint{2.262565in}{1.572744in}}{\pgfqpoint{2.259293in}{1.564844in}}{\pgfqpoint{2.259293in}{1.556608in}}%
\pgfpathcurveto{\pgfqpoint{2.259293in}{1.548372in}}{\pgfqpoint{2.262565in}{1.540472in}}{\pgfqpoint{2.268389in}{1.534648in}}%
\pgfpathcurveto{\pgfqpoint{2.274213in}{1.528824in}}{\pgfqpoint{2.282113in}{1.525552in}}{\pgfqpoint{2.290350in}{1.525552in}}%
\pgfpathclose%
\pgfusepath{stroke,fill}%
\end{pgfscope}%
\begin{pgfscope}%
\pgfpathrectangle{\pgfqpoint{0.100000in}{0.220728in}}{\pgfqpoint{3.696000in}{3.696000in}}%
\pgfusepath{clip}%
\pgfsetbuttcap%
\pgfsetroundjoin%
\definecolor{currentfill}{rgb}{0.121569,0.466667,0.705882}%
\pgfsetfillcolor{currentfill}%
\pgfsetfillopacity{0.838889}%
\pgfsetlinewidth{1.003750pt}%
\definecolor{currentstroke}{rgb}{0.121569,0.466667,0.705882}%
\pgfsetstrokecolor{currentstroke}%
\pgfsetstrokeopacity{0.838889}%
\pgfsetdash{}{0pt}%
\pgfpathmoveto{\pgfqpoint{0.785957in}{2.484308in}}%
\pgfpathcurveto{\pgfqpoint{0.794193in}{2.484308in}}{\pgfqpoint{0.802093in}{2.487581in}}{\pgfqpoint{0.807917in}{2.493405in}}%
\pgfpathcurveto{\pgfqpoint{0.813741in}{2.499229in}}{\pgfqpoint{0.817013in}{2.507129in}}{\pgfqpoint{0.817013in}{2.515365in}}%
\pgfpathcurveto{\pgfqpoint{0.817013in}{2.523601in}}{\pgfqpoint{0.813741in}{2.531501in}}{\pgfqpoint{0.807917in}{2.537325in}}%
\pgfpathcurveto{\pgfqpoint{0.802093in}{2.543149in}}{\pgfqpoint{0.794193in}{2.546421in}}{\pgfqpoint{0.785957in}{2.546421in}}%
\pgfpathcurveto{\pgfqpoint{0.777720in}{2.546421in}}{\pgfqpoint{0.769820in}{2.543149in}}{\pgfqpoint{0.763996in}{2.537325in}}%
\pgfpathcurveto{\pgfqpoint{0.758172in}{2.531501in}}{\pgfqpoint{0.754900in}{2.523601in}}{\pgfqpoint{0.754900in}{2.515365in}}%
\pgfpathcurveto{\pgfqpoint{0.754900in}{2.507129in}}{\pgfqpoint{0.758172in}{2.499229in}}{\pgfqpoint{0.763996in}{2.493405in}}%
\pgfpathcurveto{\pgfqpoint{0.769820in}{2.487581in}}{\pgfqpoint{0.777720in}{2.484308in}}{\pgfqpoint{0.785957in}{2.484308in}}%
\pgfpathclose%
\pgfusepath{stroke,fill}%
\end{pgfscope}%
\begin{pgfscope}%
\pgfpathrectangle{\pgfqpoint{0.100000in}{0.220728in}}{\pgfqpoint{3.696000in}{3.696000in}}%
\pgfusepath{clip}%
\pgfsetbuttcap%
\pgfsetroundjoin%
\definecolor{currentfill}{rgb}{0.121569,0.466667,0.705882}%
\pgfsetfillcolor{currentfill}%
\pgfsetfillopacity{0.839402}%
\pgfsetlinewidth{1.003750pt}%
\definecolor{currentstroke}{rgb}{0.121569,0.466667,0.705882}%
\pgfsetstrokecolor{currentstroke}%
\pgfsetstrokeopacity{0.839402}%
\pgfsetdash{}{0pt}%
\pgfpathmoveto{\pgfqpoint{2.290615in}{1.524287in}}%
\pgfpathcurveto{\pgfqpoint{2.298851in}{1.524287in}}{\pgfqpoint{2.306751in}{1.527559in}}{\pgfqpoint{2.312575in}{1.533383in}}%
\pgfpathcurveto{\pgfqpoint{2.318399in}{1.539207in}}{\pgfqpoint{2.321671in}{1.547107in}}{\pgfqpoint{2.321671in}{1.555343in}}%
\pgfpathcurveto{\pgfqpoint{2.321671in}{1.563579in}}{\pgfqpoint{2.318399in}{1.571479in}}{\pgfqpoint{2.312575in}{1.577303in}}%
\pgfpathcurveto{\pgfqpoint{2.306751in}{1.583127in}}{\pgfqpoint{2.298851in}{1.586399in}}{\pgfqpoint{2.290615in}{1.586399in}}%
\pgfpathcurveto{\pgfqpoint{2.282379in}{1.586399in}}{\pgfqpoint{2.274479in}{1.583127in}}{\pgfqpoint{2.268655in}{1.577303in}}%
\pgfpathcurveto{\pgfqpoint{2.262831in}{1.571479in}}{\pgfqpoint{2.259558in}{1.563579in}}{\pgfqpoint{2.259558in}{1.555343in}}%
\pgfpathcurveto{\pgfqpoint{2.259558in}{1.547107in}}{\pgfqpoint{2.262831in}{1.539207in}}{\pgfqpoint{2.268655in}{1.533383in}}%
\pgfpathcurveto{\pgfqpoint{2.274479in}{1.527559in}}{\pgfqpoint{2.282379in}{1.524287in}}{\pgfqpoint{2.290615in}{1.524287in}}%
\pgfpathclose%
\pgfusepath{stroke,fill}%
\end{pgfscope}%
\begin{pgfscope}%
\pgfpathrectangle{\pgfqpoint{0.100000in}{0.220728in}}{\pgfqpoint{3.696000in}{3.696000in}}%
\pgfusepath{clip}%
\pgfsetbuttcap%
\pgfsetroundjoin%
\definecolor{currentfill}{rgb}{0.121569,0.466667,0.705882}%
\pgfsetfillcolor{currentfill}%
\pgfsetfillopacity{0.840074}%
\pgfsetlinewidth{1.003750pt}%
\definecolor{currentstroke}{rgb}{0.121569,0.466667,0.705882}%
\pgfsetstrokecolor{currentstroke}%
\pgfsetstrokeopacity{0.840074}%
\pgfsetdash{}{0pt}%
\pgfpathmoveto{\pgfqpoint{2.291312in}{1.521620in}}%
\pgfpathcurveto{\pgfqpoint{2.299548in}{1.521620in}}{\pgfqpoint{2.307448in}{1.524892in}}{\pgfqpoint{2.313272in}{1.530716in}}%
\pgfpathcurveto{\pgfqpoint{2.319096in}{1.536540in}}{\pgfqpoint{2.322368in}{1.544440in}}{\pgfqpoint{2.322368in}{1.552676in}}%
\pgfpathcurveto{\pgfqpoint{2.322368in}{1.560913in}}{\pgfqpoint{2.319096in}{1.568813in}}{\pgfqpoint{2.313272in}{1.574637in}}%
\pgfpathcurveto{\pgfqpoint{2.307448in}{1.580461in}}{\pgfqpoint{2.299548in}{1.583733in}}{\pgfqpoint{2.291312in}{1.583733in}}%
\pgfpathcurveto{\pgfqpoint{2.283076in}{1.583733in}}{\pgfqpoint{2.275175in}{1.580461in}}{\pgfqpoint{2.269352in}{1.574637in}}%
\pgfpathcurveto{\pgfqpoint{2.263528in}{1.568813in}}{\pgfqpoint{2.260255in}{1.560913in}}{\pgfqpoint{2.260255in}{1.552676in}}%
\pgfpathcurveto{\pgfqpoint{2.260255in}{1.544440in}}{\pgfqpoint{2.263528in}{1.536540in}}{\pgfqpoint{2.269352in}{1.530716in}}%
\pgfpathcurveto{\pgfqpoint{2.275175in}{1.524892in}}{\pgfqpoint{2.283076in}{1.521620in}}{\pgfqpoint{2.291312in}{1.521620in}}%
\pgfpathclose%
\pgfusepath{stroke,fill}%
\end{pgfscope}%
\begin{pgfscope}%
\pgfpathrectangle{\pgfqpoint{0.100000in}{0.220728in}}{\pgfqpoint{3.696000in}{3.696000in}}%
\pgfusepath{clip}%
\pgfsetbuttcap%
\pgfsetroundjoin%
\definecolor{currentfill}{rgb}{0.121569,0.466667,0.705882}%
\pgfsetfillcolor{currentfill}%
\pgfsetfillopacity{0.841581}%
\pgfsetlinewidth{1.003750pt}%
\definecolor{currentstroke}{rgb}{0.121569,0.466667,0.705882}%
\pgfsetstrokecolor{currentstroke}%
\pgfsetstrokeopacity{0.841581}%
\pgfsetdash{}{0pt}%
\pgfpathmoveto{\pgfqpoint{2.292506in}{1.521621in}}%
\pgfpathcurveto{\pgfqpoint{2.300743in}{1.521621in}}{\pgfqpoint{2.308643in}{1.524894in}}{\pgfqpoint{2.314467in}{1.530717in}}%
\pgfpathcurveto{\pgfqpoint{2.320290in}{1.536541in}}{\pgfqpoint{2.323563in}{1.544441in}}{\pgfqpoint{2.323563in}{1.552678in}}%
\pgfpathcurveto{\pgfqpoint{2.323563in}{1.560914in}}{\pgfqpoint{2.320290in}{1.568814in}}{\pgfqpoint{2.314467in}{1.574638in}}%
\pgfpathcurveto{\pgfqpoint{2.308643in}{1.580462in}}{\pgfqpoint{2.300743in}{1.583734in}}{\pgfqpoint{2.292506in}{1.583734in}}%
\pgfpathcurveto{\pgfqpoint{2.284270in}{1.583734in}}{\pgfqpoint{2.276370in}{1.580462in}}{\pgfqpoint{2.270546in}{1.574638in}}%
\pgfpathcurveto{\pgfqpoint{2.264722in}{1.568814in}}{\pgfqpoint{2.261450in}{1.560914in}}{\pgfqpoint{2.261450in}{1.552678in}}%
\pgfpathcurveto{\pgfqpoint{2.261450in}{1.544441in}}{\pgfqpoint{2.264722in}{1.536541in}}{\pgfqpoint{2.270546in}{1.530717in}}%
\pgfpathcurveto{\pgfqpoint{2.276370in}{1.524894in}}{\pgfqpoint{2.284270in}{1.521621in}}{\pgfqpoint{2.292506in}{1.521621in}}%
\pgfpathclose%
\pgfusepath{stroke,fill}%
\end{pgfscope}%
\begin{pgfscope}%
\pgfpathrectangle{\pgfqpoint{0.100000in}{0.220728in}}{\pgfqpoint{3.696000in}{3.696000in}}%
\pgfusepath{clip}%
\pgfsetbuttcap%
\pgfsetroundjoin%
\definecolor{currentfill}{rgb}{0.121569,0.466667,0.705882}%
\pgfsetfillcolor{currentfill}%
\pgfsetfillopacity{0.843849}%
\pgfsetlinewidth{1.003750pt}%
\definecolor{currentstroke}{rgb}{0.121569,0.466667,0.705882}%
\pgfsetstrokecolor{currentstroke}%
\pgfsetstrokeopacity{0.843849}%
\pgfsetdash{}{0pt}%
\pgfpathmoveto{\pgfqpoint{0.824082in}{2.452593in}}%
\pgfpathcurveto{\pgfqpoint{0.832318in}{2.452593in}}{\pgfqpoint{0.840218in}{2.455865in}}{\pgfqpoint{0.846042in}{2.461689in}}%
\pgfpathcurveto{\pgfqpoint{0.851866in}{2.467513in}}{\pgfqpoint{0.855138in}{2.475413in}}{\pgfqpoint{0.855138in}{2.483649in}}%
\pgfpathcurveto{\pgfqpoint{0.855138in}{2.491885in}}{\pgfqpoint{0.851866in}{2.499785in}}{\pgfqpoint{0.846042in}{2.505609in}}%
\pgfpathcurveto{\pgfqpoint{0.840218in}{2.511433in}}{\pgfqpoint{0.832318in}{2.514706in}}{\pgfqpoint{0.824082in}{2.514706in}}%
\pgfpathcurveto{\pgfqpoint{0.815846in}{2.514706in}}{\pgfqpoint{0.807946in}{2.511433in}}{\pgfqpoint{0.802122in}{2.505609in}}%
\pgfpathcurveto{\pgfqpoint{0.796298in}{2.499785in}}{\pgfqpoint{0.793025in}{2.491885in}}{\pgfqpoint{0.793025in}{2.483649in}}%
\pgfpathcurveto{\pgfqpoint{0.793025in}{2.475413in}}{\pgfqpoint{0.796298in}{2.467513in}}{\pgfqpoint{0.802122in}{2.461689in}}%
\pgfpathcurveto{\pgfqpoint{0.807946in}{2.455865in}}{\pgfqpoint{0.815846in}{2.452593in}}{\pgfqpoint{0.824082in}{2.452593in}}%
\pgfpathclose%
\pgfusepath{stroke,fill}%
\end{pgfscope}%
\begin{pgfscope}%
\pgfpathrectangle{\pgfqpoint{0.100000in}{0.220728in}}{\pgfqpoint{3.696000in}{3.696000in}}%
\pgfusepath{clip}%
\pgfsetbuttcap%
\pgfsetroundjoin%
\definecolor{currentfill}{rgb}{0.121569,0.466667,0.705882}%
\pgfsetfillcolor{currentfill}%
\pgfsetfillopacity{0.843932}%
\pgfsetlinewidth{1.003750pt}%
\definecolor{currentstroke}{rgb}{0.121569,0.466667,0.705882}%
\pgfsetstrokecolor{currentstroke}%
\pgfsetstrokeopacity{0.843932}%
\pgfsetdash{}{0pt}%
\pgfpathmoveto{\pgfqpoint{2.294080in}{1.520865in}}%
\pgfpathcurveto{\pgfqpoint{2.302316in}{1.520865in}}{\pgfqpoint{2.310216in}{1.524137in}}{\pgfqpoint{2.316040in}{1.529961in}}%
\pgfpathcurveto{\pgfqpoint{2.321864in}{1.535785in}}{\pgfqpoint{2.325137in}{1.543685in}}{\pgfqpoint{2.325137in}{1.551921in}}%
\pgfpathcurveto{\pgfqpoint{2.325137in}{1.560157in}}{\pgfqpoint{2.321864in}{1.568057in}}{\pgfqpoint{2.316040in}{1.573881in}}%
\pgfpathcurveto{\pgfqpoint{2.310216in}{1.579705in}}{\pgfqpoint{2.302316in}{1.582978in}}{\pgfqpoint{2.294080in}{1.582978in}}%
\pgfpathcurveto{\pgfqpoint{2.285844in}{1.582978in}}{\pgfqpoint{2.277944in}{1.579705in}}{\pgfqpoint{2.272120in}{1.573881in}}%
\pgfpathcurveto{\pgfqpoint{2.266296in}{1.568057in}}{\pgfqpoint{2.263024in}{1.560157in}}{\pgfqpoint{2.263024in}{1.551921in}}%
\pgfpathcurveto{\pgfqpoint{2.263024in}{1.543685in}}{\pgfqpoint{2.266296in}{1.535785in}}{\pgfqpoint{2.272120in}{1.529961in}}%
\pgfpathcurveto{\pgfqpoint{2.277944in}{1.524137in}}{\pgfqpoint{2.285844in}{1.520865in}}{\pgfqpoint{2.294080in}{1.520865in}}%
\pgfpathclose%
\pgfusepath{stroke,fill}%
\end{pgfscope}%
\begin{pgfscope}%
\pgfpathrectangle{\pgfqpoint{0.100000in}{0.220728in}}{\pgfqpoint{3.696000in}{3.696000in}}%
\pgfusepath{clip}%
\pgfsetbuttcap%
\pgfsetroundjoin%
\definecolor{currentfill}{rgb}{0.121569,0.466667,0.705882}%
\pgfsetfillcolor{currentfill}%
\pgfsetfillopacity{0.846256}%
\pgfsetlinewidth{1.003750pt}%
\definecolor{currentstroke}{rgb}{0.121569,0.466667,0.705882}%
\pgfsetstrokecolor{currentstroke}%
\pgfsetstrokeopacity{0.846256}%
\pgfsetdash{}{0pt}%
\pgfpathmoveto{\pgfqpoint{2.294766in}{1.517588in}}%
\pgfpathcurveto{\pgfqpoint{2.303002in}{1.517588in}}{\pgfqpoint{2.310902in}{1.520860in}}{\pgfqpoint{2.316726in}{1.526684in}}%
\pgfpathcurveto{\pgfqpoint{2.322550in}{1.532508in}}{\pgfqpoint{2.325823in}{1.540408in}}{\pgfqpoint{2.325823in}{1.548644in}}%
\pgfpathcurveto{\pgfqpoint{2.325823in}{1.556880in}}{\pgfqpoint{2.322550in}{1.564781in}}{\pgfqpoint{2.316726in}{1.570604in}}%
\pgfpathcurveto{\pgfqpoint{2.310902in}{1.576428in}}{\pgfqpoint{2.303002in}{1.579701in}}{\pgfqpoint{2.294766in}{1.579701in}}%
\pgfpathcurveto{\pgfqpoint{2.286530in}{1.579701in}}{\pgfqpoint{2.278630in}{1.576428in}}{\pgfqpoint{2.272806in}{1.570604in}}%
\pgfpathcurveto{\pgfqpoint{2.266982in}{1.564781in}}{\pgfqpoint{2.263710in}{1.556880in}}{\pgfqpoint{2.263710in}{1.548644in}}%
\pgfpathcurveto{\pgfqpoint{2.263710in}{1.540408in}}{\pgfqpoint{2.266982in}{1.532508in}}{\pgfqpoint{2.272806in}{1.526684in}}%
\pgfpathcurveto{\pgfqpoint{2.278630in}{1.520860in}}{\pgfqpoint{2.286530in}{1.517588in}}{\pgfqpoint{2.294766in}{1.517588in}}%
\pgfpathclose%
\pgfusepath{stroke,fill}%
\end{pgfscope}%
\begin{pgfscope}%
\pgfpathrectangle{\pgfqpoint{0.100000in}{0.220728in}}{\pgfqpoint{3.696000in}{3.696000in}}%
\pgfusepath{clip}%
\pgfsetbuttcap%
\pgfsetroundjoin%
\definecolor{currentfill}{rgb}{0.121569,0.466667,0.705882}%
\pgfsetfillcolor{currentfill}%
\pgfsetfillopacity{0.847633}%
\pgfsetlinewidth{1.003750pt}%
\definecolor{currentstroke}{rgb}{0.121569,0.466667,0.705882}%
\pgfsetstrokecolor{currentstroke}%
\pgfsetstrokeopacity{0.847633}%
\pgfsetdash{}{0pt}%
\pgfpathmoveto{\pgfqpoint{0.863686in}{2.426226in}}%
\pgfpathcurveto{\pgfqpoint{0.871922in}{2.426226in}}{\pgfqpoint{0.879822in}{2.429498in}}{\pgfqpoint{0.885646in}{2.435322in}}%
\pgfpathcurveto{\pgfqpoint{0.891470in}{2.441146in}}{\pgfqpoint{0.894742in}{2.449046in}}{\pgfqpoint{0.894742in}{2.457282in}}%
\pgfpathcurveto{\pgfqpoint{0.894742in}{2.465518in}}{\pgfqpoint{0.891470in}{2.473418in}}{\pgfqpoint{0.885646in}{2.479242in}}%
\pgfpathcurveto{\pgfqpoint{0.879822in}{2.485066in}}{\pgfqpoint{0.871922in}{2.488339in}}{\pgfqpoint{0.863686in}{2.488339in}}%
\pgfpathcurveto{\pgfqpoint{0.855450in}{2.488339in}}{\pgfqpoint{0.847550in}{2.485066in}}{\pgfqpoint{0.841726in}{2.479242in}}%
\pgfpathcurveto{\pgfqpoint{0.835902in}{2.473418in}}{\pgfqpoint{0.832629in}{2.465518in}}{\pgfqpoint{0.832629in}{2.457282in}}%
\pgfpathcurveto{\pgfqpoint{0.832629in}{2.449046in}}{\pgfqpoint{0.835902in}{2.441146in}}{\pgfqpoint{0.841726in}{2.435322in}}%
\pgfpathcurveto{\pgfqpoint{0.847550in}{2.429498in}}{\pgfqpoint{0.855450in}{2.426226in}}{\pgfqpoint{0.863686in}{2.426226in}}%
\pgfpathclose%
\pgfusepath{stroke,fill}%
\end{pgfscope}%
\begin{pgfscope}%
\pgfpathrectangle{\pgfqpoint{0.100000in}{0.220728in}}{\pgfqpoint{3.696000in}{3.696000in}}%
\pgfusepath{clip}%
\pgfsetbuttcap%
\pgfsetroundjoin%
\definecolor{currentfill}{rgb}{0.121569,0.466667,0.705882}%
\pgfsetfillcolor{currentfill}%
\pgfsetfillopacity{0.848654}%
\pgfsetlinewidth{1.003750pt}%
\definecolor{currentstroke}{rgb}{0.121569,0.466667,0.705882}%
\pgfsetstrokecolor{currentstroke}%
\pgfsetstrokeopacity{0.848654}%
\pgfsetdash{}{0pt}%
\pgfpathmoveto{\pgfqpoint{2.296969in}{1.514286in}}%
\pgfpathcurveto{\pgfqpoint{2.305205in}{1.514286in}}{\pgfqpoint{2.313105in}{1.517558in}}{\pgfqpoint{2.318929in}{1.523382in}}%
\pgfpathcurveto{\pgfqpoint{2.324753in}{1.529206in}}{\pgfqpoint{2.328025in}{1.537106in}}{\pgfqpoint{2.328025in}{1.545342in}}%
\pgfpathcurveto{\pgfqpoint{2.328025in}{1.553578in}}{\pgfqpoint{2.324753in}{1.561479in}}{\pgfqpoint{2.318929in}{1.567302in}}%
\pgfpathcurveto{\pgfqpoint{2.313105in}{1.573126in}}{\pgfqpoint{2.305205in}{1.576399in}}{\pgfqpoint{2.296969in}{1.576399in}}%
\pgfpathcurveto{\pgfqpoint{2.288732in}{1.576399in}}{\pgfqpoint{2.280832in}{1.573126in}}{\pgfqpoint{2.275009in}{1.567302in}}%
\pgfpathcurveto{\pgfqpoint{2.269185in}{1.561479in}}{\pgfqpoint{2.265912in}{1.553578in}}{\pgfqpoint{2.265912in}{1.545342in}}%
\pgfpathcurveto{\pgfqpoint{2.265912in}{1.537106in}}{\pgfqpoint{2.269185in}{1.529206in}}{\pgfqpoint{2.275009in}{1.523382in}}%
\pgfpathcurveto{\pgfqpoint{2.280832in}{1.517558in}}{\pgfqpoint{2.288732in}{1.514286in}}{\pgfqpoint{2.296969in}{1.514286in}}%
\pgfpathclose%
\pgfusepath{stroke,fill}%
\end{pgfscope}%
\begin{pgfscope}%
\pgfpathrectangle{\pgfqpoint{0.100000in}{0.220728in}}{\pgfqpoint{3.696000in}{3.696000in}}%
\pgfusepath{clip}%
\pgfsetbuttcap%
\pgfsetroundjoin%
\definecolor{currentfill}{rgb}{0.121569,0.466667,0.705882}%
\pgfsetfillcolor{currentfill}%
\pgfsetfillopacity{0.851207}%
\pgfsetlinewidth{1.003750pt}%
\definecolor{currentstroke}{rgb}{0.121569,0.466667,0.705882}%
\pgfsetstrokecolor{currentstroke}%
\pgfsetstrokeopacity{0.851207}%
\pgfsetdash{}{0pt}%
\pgfpathmoveto{\pgfqpoint{2.298970in}{1.509885in}}%
\pgfpathcurveto{\pgfqpoint{2.307207in}{1.509885in}}{\pgfqpoint{2.315107in}{1.513158in}}{\pgfqpoint{2.320931in}{1.518981in}}%
\pgfpathcurveto{\pgfqpoint{2.326755in}{1.524805in}}{\pgfqpoint{2.330027in}{1.532705in}}{\pgfqpoint{2.330027in}{1.540942in}}%
\pgfpathcurveto{\pgfqpoint{2.330027in}{1.549178in}}{\pgfqpoint{2.326755in}{1.557078in}}{\pgfqpoint{2.320931in}{1.562902in}}%
\pgfpathcurveto{\pgfqpoint{2.315107in}{1.568726in}}{\pgfqpoint{2.307207in}{1.571998in}}{\pgfqpoint{2.298970in}{1.571998in}}%
\pgfpathcurveto{\pgfqpoint{2.290734in}{1.571998in}}{\pgfqpoint{2.282834in}{1.568726in}}{\pgfqpoint{2.277010in}{1.562902in}}%
\pgfpathcurveto{\pgfqpoint{2.271186in}{1.557078in}}{\pgfqpoint{2.267914in}{1.549178in}}{\pgfqpoint{2.267914in}{1.540942in}}%
\pgfpathcurveto{\pgfqpoint{2.267914in}{1.532705in}}{\pgfqpoint{2.271186in}{1.524805in}}{\pgfqpoint{2.277010in}{1.518981in}}%
\pgfpathcurveto{\pgfqpoint{2.282834in}{1.513158in}}{\pgfqpoint{2.290734in}{1.509885in}}{\pgfqpoint{2.298970in}{1.509885in}}%
\pgfpathclose%
\pgfusepath{stroke,fill}%
\end{pgfscope}%
\begin{pgfscope}%
\pgfpathrectangle{\pgfqpoint{0.100000in}{0.220728in}}{\pgfqpoint{3.696000in}{3.696000in}}%
\pgfusepath{clip}%
\pgfsetbuttcap%
\pgfsetroundjoin%
\definecolor{currentfill}{rgb}{0.121569,0.466667,0.705882}%
\pgfsetfillcolor{currentfill}%
\pgfsetfillopacity{0.852093}%
\pgfsetlinewidth{1.003750pt}%
\definecolor{currentstroke}{rgb}{0.121569,0.466667,0.705882}%
\pgfsetstrokecolor{currentstroke}%
\pgfsetstrokeopacity{0.852093}%
\pgfsetdash{}{0pt}%
\pgfpathmoveto{\pgfqpoint{0.900341in}{2.409030in}}%
\pgfpathcurveto{\pgfqpoint{0.908577in}{2.409030in}}{\pgfqpoint{0.916477in}{2.412303in}}{\pgfqpoint{0.922301in}{2.418127in}}%
\pgfpathcurveto{\pgfqpoint{0.928125in}{2.423951in}}{\pgfqpoint{0.931397in}{2.431851in}}{\pgfqpoint{0.931397in}{2.440087in}}%
\pgfpathcurveto{\pgfqpoint{0.931397in}{2.448323in}}{\pgfqpoint{0.928125in}{2.456223in}}{\pgfqpoint{0.922301in}{2.462047in}}%
\pgfpathcurveto{\pgfqpoint{0.916477in}{2.467871in}}{\pgfqpoint{0.908577in}{2.471143in}}{\pgfqpoint{0.900341in}{2.471143in}}%
\pgfpathcurveto{\pgfqpoint{0.892105in}{2.471143in}}{\pgfqpoint{0.884204in}{2.467871in}}{\pgfqpoint{0.878381in}{2.462047in}}%
\pgfpathcurveto{\pgfqpoint{0.872557in}{2.456223in}}{\pgfqpoint{0.869284in}{2.448323in}}{\pgfqpoint{0.869284in}{2.440087in}}%
\pgfpathcurveto{\pgfqpoint{0.869284in}{2.431851in}}{\pgfqpoint{0.872557in}{2.423951in}}{\pgfqpoint{0.878381in}{2.418127in}}%
\pgfpathcurveto{\pgfqpoint{0.884204in}{2.412303in}}{\pgfqpoint{0.892105in}{2.409030in}}{\pgfqpoint{0.900341in}{2.409030in}}%
\pgfpathclose%
\pgfusepath{stroke,fill}%
\end{pgfscope}%
\begin{pgfscope}%
\pgfpathrectangle{\pgfqpoint{0.100000in}{0.220728in}}{\pgfqpoint{3.696000in}{3.696000in}}%
\pgfusepath{clip}%
\pgfsetbuttcap%
\pgfsetroundjoin%
\definecolor{currentfill}{rgb}{0.121569,0.466667,0.705882}%
\pgfsetfillcolor{currentfill}%
\pgfsetfillopacity{0.854743}%
\pgfsetlinewidth{1.003750pt}%
\definecolor{currentstroke}{rgb}{0.121569,0.466667,0.705882}%
\pgfsetstrokecolor{currentstroke}%
\pgfsetstrokeopacity{0.854743}%
\pgfsetdash{}{0pt}%
\pgfpathmoveto{\pgfqpoint{0.931121in}{2.366303in}}%
\pgfpathcurveto{\pgfqpoint{0.939357in}{2.366303in}}{\pgfqpoint{0.947257in}{2.369575in}}{\pgfqpoint{0.953081in}{2.375399in}}%
\pgfpathcurveto{\pgfqpoint{0.958905in}{2.381223in}}{\pgfqpoint{0.962177in}{2.389123in}}{\pgfqpoint{0.962177in}{2.397359in}}%
\pgfpathcurveto{\pgfqpoint{0.962177in}{2.405595in}}{\pgfqpoint{0.958905in}{2.413495in}}{\pgfqpoint{0.953081in}{2.419319in}}%
\pgfpathcurveto{\pgfqpoint{0.947257in}{2.425143in}}{\pgfqpoint{0.939357in}{2.428416in}}{\pgfqpoint{0.931121in}{2.428416in}}%
\pgfpathcurveto{\pgfqpoint{0.922884in}{2.428416in}}{\pgfqpoint{0.914984in}{2.425143in}}{\pgfqpoint{0.909160in}{2.419319in}}%
\pgfpathcurveto{\pgfqpoint{0.903336in}{2.413495in}}{\pgfqpoint{0.900064in}{2.405595in}}{\pgfqpoint{0.900064in}{2.397359in}}%
\pgfpathcurveto{\pgfqpoint{0.900064in}{2.389123in}}{\pgfqpoint{0.903336in}{2.381223in}}{\pgfqpoint{0.909160in}{2.375399in}}%
\pgfpathcurveto{\pgfqpoint{0.914984in}{2.369575in}}{\pgfqpoint{0.922884in}{2.366303in}}{\pgfqpoint{0.931121in}{2.366303in}}%
\pgfpathclose%
\pgfusepath{stroke,fill}%
\end{pgfscope}%
\begin{pgfscope}%
\pgfpathrectangle{\pgfqpoint{0.100000in}{0.220728in}}{\pgfqpoint{3.696000in}{3.696000in}}%
\pgfusepath{clip}%
\pgfsetbuttcap%
\pgfsetroundjoin%
\definecolor{currentfill}{rgb}{0.121569,0.466667,0.705882}%
\pgfsetfillcolor{currentfill}%
\pgfsetfillopacity{0.854974}%
\pgfsetlinewidth{1.003750pt}%
\definecolor{currentstroke}{rgb}{0.121569,0.466667,0.705882}%
\pgfsetstrokecolor{currentstroke}%
\pgfsetstrokeopacity{0.854974}%
\pgfsetdash{}{0pt}%
\pgfpathmoveto{\pgfqpoint{2.302895in}{1.512683in}}%
\pgfpathcurveto{\pgfqpoint{2.311131in}{1.512683in}}{\pgfqpoint{2.319031in}{1.515956in}}{\pgfqpoint{2.324855in}{1.521780in}}%
\pgfpathcurveto{\pgfqpoint{2.330679in}{1.527603in}}{\pgfqpoint{2.333951in}{1.535504in}}{\pgfqpoint{2.333951in}{1.543740in}}%
\pgfpathcurveto{\pgfqpoint{2.333951in}{1.551976in}}{\pgfqpoint{2.330679in}{1.559876in}}{\pgfqpoint{2.324855in}{1.565700in}}%
\pgfpathcurveto{\pgfqpoint{2.319031in}{1.571524in}}{\pgfqpoint{2.311131in}{1.574796in}}{\pgfqpoint{2.302895in}{1.574796in}}%
\pgfpathcurveto{\pgfqpoint{2.294658in}{1.574796in}}{\pgfqpoint{2.286758in}{1.571524in}}{\pgfqpoint{2.280934in}{1.565700in}}%
\pgfpathcurveto{\pgfqpoint{2.275110in}{1.559876in}}{\pgfqpoint{2.271838in}{1.551976in}}{\pgfqpoint{2.271838in}{1.543740in}}%
\pgfpathcurveto{\pgfqpoint{2.271838in}{1.535504in}}{\pgfqpoint{2.275110in}{1.527603in}}{\pgfqpoint{2.280934in}{1.521780in}}%
\pgfpathcurveto{\pgfqpoint{2.286758in}{1.515956in}}{\pgfqpoint{2.294658in}{1.512683in}}{\pgfqpoint{2.302895in}{1.512683in}}%
\pgfpathclose%
\pgfusepath{stroke,fill}%
\end{pgfscope}%
\begin{pgfscope}%
\pgfpathrectangle{\pgfqpoint{0.100000in}{0.220728in}}{\pgfqpoint{3.696000in}{3.696000in}}%
\pgfusepath{clip}%
\pgfsetbuttcap%
\pgfsetroundjoin%
\definecolor{currentfill}{rgb}{0.121569,0.466667,0.705882}%
\pgfsetfillcolor{currentfill}%
\pgfsetfillopacity{0.858065}%
\pgfsetlinewidth{1.003750pt}%
\definecolor{currentstroke}{rgb}{0.121569,0.466667,0.705882}%
\pgfsetstrokecolor{currentstroke}%
\pgfsetstrokeopacity{0.858065}%
\pgfsetdash{}{0pt}%
\pgfpathmoveto{\pgfqpoint{0.963315in}{2.345161in}}%
\pgfpathcurveto{\pgfqpoint{0.971551in}{2.345161in}}{\pgfqpoint{0.979451in}{2.348433in}}{\pgfqpoint{0.985275in}{2.354257in}}%
\pgfpathcurveto{\pgfqpoint{0.991099in}{2.360081in}}{\pgfqpoint{0.994371in}{2.367981in}}{\pgfqpoint{0.994371in}{2.376217in}}%
\pgfpathcurveto{\pgfqpoint{0.994371in}{2.384454in}}{\pgfqpoint{0.991099in}{2.392354in}}{\pgfqpoint{0.985275in}{2.398178in}}%
\pgfpathcurveto{\pgfqpoint{0.979451in}{2.404002in}}{\pgfqpoint{0.971551in}{2.407274in}}{\pgfqpoint{0.963315in}{2.407274in}}%
\pgfpathcurveto{\pgfqpoint{0.955078in}{2.407274in}}{\pgfqpoint{0.947178in}{2.404002in}}{\pgfqpoint{0.941354in}{2.398178in}}%
\pgfpathcurveto{\pgfqpoint{0.935531in}{2.392354in}}{\pgfqpoint{0.932258in}{2.384454in}}{\pgfqpoint{0.932258in}{2.376217in}}%
\pgfpathcurveto{\pgfqpoint{0.932258in}{2.367981in}}{\pgfqpoint{0.935531in}{2.360081in}}{\pgfqpoint{0.941354in}{2.354257in}}%
\pgfpathcurveto{\pgfqpoint{0.947178in}{2.348433in}}{\pgfqpoint{0.955078in}{2.345161in}}{\pgfqpoint{0.963315in}{2.345161in}}%
\pgfpathclose%
\pgfusepath{stroke,fill}%
\end{pgfscope}%
\begin{pgfscope}%
\pgfpathrectangle{\pgfqpoint{0.100000in}{0.220728in}}{\pgfqpoint{3.696000in}{3.696000in}}%
\pgfusepath{clip}%
\pgfsetbuttcap%
\pgfsetroundjoin%
\definecolor{currentfill}{rgb}{0.121569,0.466667,0.705882}%
\pgfsetfillcolor{currentfill}%
\pgfsetfillopacity{0.858390}%
\pgfsetlinewidth{1.003750pt}%
\definecolor{currentstroke}{rgb}{0.121569,0.466667,0.705882}%
\pgfsetstrokecolor{currentstroke}%
\pgfsetstrokeopacity{0.858390}%
\pgfsetdash{}{0pt}%
\pgfpathmoveto{\pgfqpoint{2.304367in}{1.508671in}}%
\pgfpathcurveto{\pgfqpoint{2.312604in}{1.508671in}}{\pgfqpoint{2.320504in}{1.511943in}}{\pgfqpoint{2.326327in}{1.517767in}}%
\pgfpathcurveto{\pgfqpoint{2.332151in}{1.523591in}}{\pgfqpoint{2.335424in}{1.531491in}}{\pgfqpoint{2.335424in}{1.539727in}}%
\pgfpathcurveto{\pgfqpoint{2.335424in}{1.547963in}}{\pgfqpoint{2.332151in}{1.555864in}}{\pgfqpoint{2.326327in}{1.561687in}}%
\pgfpathcurveto{\pgfqpoint{2.320504in}{1.567511in}}{\pgfqpoint{2.312604in}{1.570784in}}{\pgfqpoint{2.304367in}{1.570784in}}%
\pgfpathcurveto{\pgfqpoint{2.296131in}{1.570784in}}{\pgfqpoint{2.288231in}{1.567511in}}{\pgfqpoint{2.282407in}{1.561687in}}%
\pgfpathcurveto{\pgfqpoint{2.276583in}{1.555864in}}{\pgfqpoint{2.273311in}{1.547963in}}{\pgfqpoint{2.273311in}{1.539727in}}%
\pgfpathcurveto{\pgfqpoint{2.273311in}{1.531491in}}{\pgfqpoint{2.276583in}{1.523591in}}{\pgfqpoint{2.282407in}{1.517767in}}%
\pgfpathcurveto{\pgfqpoint{2.288231in}{1.511943in}}{\pgfqpoint{2.296131in}{1.508671in}}{\pgfqpoint{2.304367in}{1.508671in}}%
\pgfpathclose%
\pgfusepath{stroke,fill}%
\end{pgfscope}%
\begin{pgfscope}%
\pgfpathrectangle{\pgfqpoint{0.100000in}{0.220728in}}{\pgfqpoint{3.696000in}{3.696000in}}%
\pgfusepath{clip}%
\pgfsetbuttcap%
\pgfsetroundjoin%
\definecolor{currentfill}{rgb}{0.121569,0.466667,0.705882}%
\pgfsetfillcolor{currentfill}%
\pgfsetfillopacity{0.861477}%
\pgfsetlinewidth{1.003750pt}%
\definecolor{currentstroke}{rgb}{0.121569,0.466667,0.705882}%
\pgfsetstrokecolor{currentstroke}%
\pgfsetstrokeopacity{0.861477}%
\pgfsetdash{}{0pt}%
\pgfpathmoveto{\pgfqpoint{1.022447in}{2.290030in}}%
\pgfpathcurveto{\pgfqpoint{1.030683in}{2.290030in}}{\pgfqpoint{1.038583in}{2.293303in}}{\pgfqpoint{1.044407in}{2.299126in}}%
\pgfpathcurveto{\pgfqpoint{1.050231in}{2.304950in}}{\pgfqpoint{1.053503in}{2.312850in}}{\pgfqpoint{1.053503in}{2.321087in}}%
\pgfpathcurveto{\pgfqpoint{1.053503in}{2.329323in}}{\pgfqpoint{1.050231in}{2.337223in}}{\pgfqpoint{1.044407in}{2.343047in}}%
\pgfpathcurveto{\pgfqpoint{1.038583in}{2.348871in}}{\pgfqpoint{1.030683in}{2.352143in}}{\pgfqpoint{1.022447in}{2.352143in}}%
\pgfpathcurveto{\pgfqpoint{1.014210in}{2.352143in}}{\pgfqpoint{1.006310in}{2.348871in}}{\pgfqpoint{1.000486in}{2.343047in}}%
\pgfpathcurveto{\pgfqpoint{0.994662in}{2.337223in}}{\pgfqpoint{0.991390in}{2.329323in}}{\pgfqpoint{0.991390in}{2.321087in}}%
\pgfpathcurveto{\pgfqpoint{0.991390in}{2.312850in}}{\pgfqpoint{0.994662in}{2.304950in}}{\pgfqpoint{1.000486in}{2.299126in}}%
\pgfpathcurveto{\pgfqpoint{1.006310in}{2.293303in}}{\pgfqpoint{1.014210in}{2.290030in}}{\pgfqpoint{1.022447in}{2.290030in}}%
\pgfpathclose%
\pgfusepath{stroke,fill}%
\end{pgfscope}%
\begin{pgfscope}%
\pgfpathrectangle{\pgfqpoint{0.100000in}{0.220728in}}{\pgfqpoint{3.696000in}{3.696000in}}%
\pgfusepath{clip}%
\pgfsetbuttcap%
\pgfsetroundjoin%
\definecolor{currentfill}{rgb}{0.121569,0.466667,0.705882}%
\pgfsetfillcolor{currentfill}%
\pgfsetfillopacity{0.862067}%
\pgfsetlinewidth{1.003750pt}%
\definecolor{currentstroke}{rgb}{0.121569,0.466667,0.705882}%
\pgfsetstrokecolor{currentstroke}%
\pgfsetstrokeopacity{0.862067}%
\pgfsetdash{}{0pt}%
\pgfpathmoveto{\pgfqpoint{2.307500in}{1.504750in}}%
\pgfpathcurveto{\pgfqpoint{2.315737in}{1.504750in}}{\pgfqpoint{2.323637in}{1.508022in}}{\pgfqpoint{2.329461in}{1.513846in}}%
\pgfpathcurveto{\pgfqpoint{2.335284in}{1.519670in}}{\pgfqpoint{2.338557in}{1.527570in}}{\pgfqpoint{2.338557in}{1.535807in}}%
\pgfpathcurveto{\pgfqpoint{2.338557in}{1.544043in}}{\pgfqpoint{2.335284in}{1.551943in}}{\pgfqpoint{2.329461in}{1.557767in}}%
\pgfpathcurveto{\pgfqpoint{2.323637in}{1.563591in}}{\pgfqpoint{2.315737in}{1.566863in}}{\pgfqpoint{2.307500in}{1.566863in}}%
\pgfpathcurveto{\pgfqpoint{2.299264in}{1.566863in}}{\pgfqpoint{2.291364in}{1.563591in}}{\pgfqpoint{2.285540in}{1.557767in}}%
\pgfpathcurveto{\pgfqpoint{2.279716in}{1.551943in}}{\pgfqpoint{2.276444in}{1.544043in}}{\pgfqpoint{2.276444in}{1.535807in}}%
\pgfpathcurveto{\pgfqpoint{2.276444in}{1.527570in}}{\pgfqpoint{2.279716in}{1.519670in}}{\pgfqpoint{2.285540in}{1.513846in}}%
\pgfpathcurveto{\pgfqpoint{2.291364in}{1.508022in}}{\pgfqpoint{2.299264in}{1.504750in}}{\pgfqpoint{2.307500in}{1.504750in}}%
\pgfpathclose%
\pgfusepath{stroke,fill}%
\end{pgfscope}%
\begin{pgfscope}%
\pgfpathrectangle{\pgfqpoint{0.100000in}{0.220728in}}{\pgfqpoint{3.696000in}{3.696000in}}%
\pgfusepath{clip}%
\pgfsetbuttcap%
\pgfsetroundjoin%
\definecolor{currentfill}{rgb}{0.121569,0.466667,0.705882}%
\pgfsetfillcolor{currentfill}%
\pgfsetfillopacity{0.864669}%
\pgfsetlinewidth{1.003750pt}%
\definecolor{currentstroke}{rgb}{0.121569,0.466667,0.705882}%
\pgfsetstrokecolor{currentstroke}%
\pgfsetstrokeopacity{0.864669}%
\pgfsetdash{}{0pt}%
\pgfpathmoveto{\pgfqpoint{2.310315in}{1.491917in}}%
\pgfpathcurveto{\pgfqpoint{2.318551in}{1.491917in}}{\pgfqpoint{2.326451in}{1.495190in}}{\pgfqpoint{2.332275in}{1.501014in}}%
\pgfpathcurveto{\pgfqpoint{2.338099in}{1.506838in}}{\pgfqpoint{2.341371in}{1.514738in}}{\pgfqpoint{2.341371in}{1.522974in}}%
\pgfpathcurveto{\pgfqpoint{2.341371in}{1.531210in}}{\pgfqpoint{2.338099in}{1.539110in}}{\pgfqpoint{2.332275in}{1.544934in}}%
\pgfpathcurveto{\pgfqpoint{2.326451in}{1.550758in}}{\pgfqpoint{2.318551in}{1.554030in}}{\pgfqpoint{2.310315in}{1.554030in}}%
\pgfpathcurveto{\pgfqpoint{2.302078in}{1.554030in}}{\pgfqpoint{2.294178in}{1.550758in}}{\pgfqpoint{2.288354in}{1.544934in}}%
\pgfpathcurveto{\pgfqpoint{2.282530in}{1.539110in}}{\pgfqpoint{2.279258in}{1.531210in}}{\pgfqpoint{2.279258in}{1.522974in}}%
\pgfpathcurveto{\pgfqpoint{2.279258in}{1.514738in}}{\pgfqpoint{2.282530in}{1.506838in}}{\pgfqpoint{2.288354in}{1.501014in}}%
\pgfpathcurveto{\pgfqpoint{2.294178in}{1.495190in}}{\pgfqpoint{2.302078in}{1.491917in}}{\pgfqpoint{2.310315in}{1.491917in}}%
\pgfpathclose%
\pgfusepath{stroke,fill}%
\end{pgfscope}%
\begin{pgfscope}%
\pgfpathrectangle{\pgfqpoint{0.100000in}{0.220728in}}{\pgfqpoint{3.696000in}{3.696000in}}%
\pgfusepath{clip}%
\pgfsetbuttcap%
\pgfsetroundjoin%
\definecolor{currentfill}{rgb}{0.121569,0.466667,0.705882}%
\pgfsetfillcolor{currentfill}%
\pgfsetfillopacity{0.866186}%
\pgfsetlinewidth{1.003750pt}%
\definecolor{currentstroke}{rgb}{0.121569,0.466667,0.705882}%
\pgfsetstrokecolor{currentstroke}%
\pgfsetstrokeopacity{0.866186}%
\pgfsetdash{}{0pt}%
\pgfpathmoveto{\pgfqpoint{1.076500in}{2.244001in}}%
\pgfpathcurveto{\pgfqpoint{1.084737in}{2.244001in}}{\pgfqpoint{1.092637in}{2.247273in}}{\pgfqpoint{1.098461in}{2.253097in}}%
\pgfpathcurveto{\pgfqpoint{1.104285in}{2.258921in}}{\pgfqpoint{1.107557in}{2.266821in}}{\pgfqpoint{1.107557in}{2.275057in}}%
\pgfpathcurveto{\pgfqpoint{1.107557in}{2.283294in}}{\pgfqpoint{1.104285in}{2.291194in}}{\pgfqpoint{1.098461in}{2.297018in}}%
\pgfpathcurveto{\pgfqpoint{1.092637in}{2.302842in}}{\pgfqpoint{1.084737in}{2.306114in}}{\pgfqpoint{1.076500in}{2.306114in}}%
\pgfpathcurveto{\pgfqpoint{1.068264in}{2.306114in}}{\pgfqpoint{1.060364in}{2.302842in}}{\pgfqpoint{1.054540in}{2.297018in}}%
\pgfpathcurveto{\pgfqpoint{1.048716in}{2.291194in}}{\pgfqpoint{1.045444in}{2.283294in}}{\pgfqpoint{1.045444in}{2.275057in}}%
\pgfpathcurveto{\pgfqpoint{1.045444in}{2.266821in}}{\pgfqpoint{1.048716in}{2.258921in}}{\pgfqpoint{1.054540in}{2.253097in}}%
\pgfpathcurveto{\pgfqpoint{1.060364in}{2.247273in}}{\pgfqpoint{1.068264in}{2.244001in}}{\pgfqpoint{1.076500in}{2.244001in}}%
\pgfpathclose%
\pgfusepath{stroke,fill}%
\end{pgfscope}%
\begin{pgfscope}%
\pgfpathrectangle{\pgfqpoint{0.100000in}{0.220728in}}{\pgfqpoint{3.696000in}{3.696000in}}%
\pgfusepath{clip}%
\pgfsetbuttcap%
\pgfsetroundjoin%
\definecolor{currentfill}{rgb}{0.121569,0.466667,0.705882}%
\pgfsetfillcolor{currentfill}%
\pgfsetfillopacity{0.867005}%
\pgfsetlinewidth{1.003750pt}%
\definecolor{currentstroke}{rgb}{0.121569,0.466667,0.705882}%
\pgfsetstrokecolor{currentstroke}%
\pgfsetstrokeopacity{0.867005}%
\pgfsetdash{}{0pt}%
\pgfpathmoveto{\pgfqpoint{2.312279in}{1.490660in}}%
\pgfpathcurveto{\pgfqpoint{2.320515in}{1.490660in}}{\pgfqpoint{2.328415in}{1.493932in}}{\pgfqpoint{2.334239in}{1.499756in}}%
\pgfpathcurveto{\pgfqpoint{2.340063in}{1.505580in}}{\pgfqpoint{2.343335in}{1.513480in}}{\pgfqpoint{2.343335in}{1.521717in}}%
\pgfpathcurveto{\pgfqpoint{2.343335in}{1.529953in}}{\pgfqpoint{2.340063in}{1.537853in}}{\pgfqpoint{2.334239in}{1.543677in}}%
\pgfpathcurveto{\pgfqpoint{2.328415in}{1.549501in}}{\pgfqpoint{2.320515in}{1.552773in}}{\pgfqpoint{2.312279in}{1.552773in}}%
\pgfpathcurveto{\pgfqpoint{2.304043in}{1.552773in}}{\pgfqpoint{2.296142in}{1.549501in}}{\pgfqpoint{2.290319in}{1.543677in}}%
\pgfpathcurveto{\pgfqpoint{2.284495in}{1.537853in}}{\pgfqpoint{2.281222in}{1.529953in}}{\pgfqpoint{2.281222in}{1.521717in}}%
\pgfpathcurveto{\pgfqpoint{2.281222in}{1.513480in}}{\pgfqpoint{2.284495in}{1.505580in}}{\pgfqpoint{2.290319in}{1.499756in}}%
\pgfpathcurveto{\pgfqpoint{2.296142in}{1.493932in}}{\pgfqpoint{2.304043in}{1.490660in}}{\pgfqpoint{2.312279in}{1.490660in}}%
\pgfpathclose%
\pgfusepath{stroke,fill}%
\end{pgfscope}%
\begin{pgfscope}%
\pgfpathrectangle{\pgfqpoint{0.100000in}{0.220728in}}{\pgfqpoint{3.696000in}{3.696000in}}%
\pgfusepath{clip}%
\pgfsetbuttcap%
\pgfsetroundjoin%
\definecolor{currentfill}{rgb}{0.121569,0.466667,0.705882}%
\pgfsetfillcolor{currentfill}%
\pgfsetfillopacity{0.869702}%
\pgfsetlinewidth{1.003750pt}%
\definecolor{currentstroke}{rgb}{0.121569,0.466667,0.705882}%
\pgfsetstrokecolor{currentstroke}%
\pgfsetstrokeopacity{0.869702}%
\pgfsetdash{}{0pt}%
\pgfpathmoveto{\pgfqpoint{2.314949in}{1.489935in}}%
\pgfpathcurveto{\pgfqpoint{2.323186in}{1.489935in}}{\pgfqpoint{2.331086in}{1.493207in}}{\pgfqpoint{2.336910in}{1.499031in}}%
\pgfpathcurveto{\pgfqpoint{2.342733in}{1.504855in}}{\pgfqpoint{2.346006in}{1.512755in}}{\pgfqpoint{2.346006in}{1.520992in}}%
\pgfpathcurveto{\pgfqpoint{2.346006in}{1.529228in}}{\pgfqpoint{2.342733in}{1.537128in}}{\pgfqpoint{2.336910in}{1.542952in}}%
\pgfpathcurveto{\pgfqpoint{2.331086in}{1.548776in}}{\pgfqpoint{2.323186in}{1.552048in}}{\pgfqpoint{2.314949in}{1.552048in}}%
\pgfpathcurveto{\pgfqpoint{2.306713in}{1.552048in}}{\pgfqpoint{2.298813in}{1.548776in}}{\pgfqpoint{2.292989in}{1.542952in}}%
\pgfpathcurveto{\pgfqpoint{2.287165in}{1.537128in}}{\pgfqpoint{2.283893in}{1.529228in}}{\pgfqpoint{2.283893in}{1.520992in}}%
\pgfpathcurveto{\pgfqpoint{2.283893in}{1.512755in}}{\pgfqpoint{2.287165in}{1.504855in}}{\pgfqpoint{2.292989in}{1.499031in}}%
\pgfpathcurveto{\pgfqpoint{2.298813in}{1.493207in}}{\pgfqpoint{2.306713in}{1.489935in}}{\pgfqpoint{2.314949in}{1.489935in}}%
\pgfpathclose%
\pgfusepath{stroke,fill}%
\end{pgfscope}%
\begin{pgfscope}%
\pgfpathrectangle{\pgfqpoint{0.100000in}{0.220728in}}{\pgfqpoint{3.696000in}{3.696000in}}%
\pgfusepath{clip}%
\pgfsetbuttcap%
\pgfsetroundjoin%
\definecolor{currentfill}{rgb}{0.121569,0.466667,0.705882}%
\pgfsetfillcolor{currentfill}%
\pgfsetfillopacity{0.871632}%
\pgfsetlinewidth{1.003750pt}%
\definecolor{currentstroke}{rgb}{0.121569,0.466667,0.705882}%
\pgfsetstrokecolor{currentstroke}%
\pgfsetstrokeopacity{0.871632}%
\pgfsetdash{}{0pt}%
\pgfpathmoveto{\pgfqpoint{1.129135in}{2.200543in}}%
\pgfpathcurveto{\pgfqpoint{1.137371in}{2.200543in}}{\pgfqpoint{1.145271in}{2.203815in}}{\pgfqpoint{1.151095in}{2.209639in}}%
\pgfpathcurveto{\pgfqpoint{1.156919in}{2.215463in}}{\pgfqpoint{1.160191in}{2.223363in}}{\pgfqpoint{1.160191in}{2.231599in}}%
\pgfpathcurveto{\pgfqpoint{1.160191in}{2.239836in}}{\pgfqpoint{1.156919in}{2.247736in}}{\pgfqpoint{1.151095in}{2.253560in}}%
\pgfpathcurveto{\pgfqpoint{1.145271in}{2.259383in}}{\pgfqpoint{1.137371in}{2.262656in}}{\pgfqpoint{1.129135in}{2.262656in}}%
\pgfpathcurveto{\pgfqpoint{1.120898in}{2.262656in}}{\pgfqpoint{1.112998in}{2.259383in}}{\pgfqpoint{1.107174in}{2.253560in}}%
\pgfpathcurveto{\pgfqpoint{1.101350in}{2.247736in}}{\pgfqpoint{1.098078in}{2.239836in}}{\pgfqpoint{1.098078in}{2.231599in}}%
\pgfpathcurveto{\pgfqpoint{1.098078in}{2.223363in}}{\pgfqpoint{1.101350in}{2.215463in}}{\pgfqpoint{1.107174in}{2.209639in}}%
\pgfpathcurveto{\pgfqpoint{1.112998in}{2.203815in}}{\pgfqpoint{1.120898in}{2.200543in}}{\pgfqpoint{1.129135in}{2.200543in}}%
\pgfpathclose%
\pgfusepath{stroke,fill}%
\end{pgfscope}%
\begin{pgfscope}%
\pgfpathrectangle{\pgfqpoint{0.100000in}{0.220728in}}{\pgfqpoint{3.696000in}{3.696000in}}%
\pgfusepath{clip}%
\pgfsetbuttcap%
\pgfsetroundjoin%
\definecolor{currentfill}{rgb}{0.121569,0.466667,0.705882}%
\pgfsetfillcolor{currentfill}%
\pgfsetfillopacity{0.872210}%
\pgfsetlinewidth{1.003750pt}%
\definecolor{currentstroke}{rgb}{0.121569,0.466667,0.705882}%
\pgfsetstrokecolor{currentstroke}%
\pgfsetstrokeopacity{0.872210}%
\pgfsetdash{}{0pt}%
\pgfpathmoveto{\pgfqpoint{2.316292in}{1.486062in}}%
\pgfpathcurveto{\pgfqpoint{2.324528in}{1.486062in}}{\pgfqpoint{2.332428in}{1.489335in}}{\pgfqpoint{2.338252in}{1.495158in}}%
\pgfpathcurveto{\pgfqpoint{2.344076in}{1.500982in}}{\pgfqpoint{2.347348in}{1.508882in}}{\pgfqpoint{2.347348in}{1.517119in}}%
\pgfpathcurveto{\pgfqpoint{2.347348in}{1.525355in}}{\pgfqpoint{2.344076in}{1.533255in}}{\pgfqpoint{2.338252in}{1.539079in}}%
\pgfpathcurveto{\pgfqpoint{2.332428in}{1.544903in}}{\pgfqpoint{2.324528in}{1.548175in}}{\pgfqpoint{2.316292in}{1.548175in}}%
\pgfpathcurveto{\pgfqpoint{2.308055in}{1.548175in}}{\pgfqpoint{2.300155in}{1.544903in}}{\pgfqpoint{2.294331in}{1.539079in}}%
\pgfpathcurveto{\pgfqpoint{2.288507in}{1.533255in}}{\pgfqpoint{2.285235in}{1.525355in}}{\pgfqpoint{2.285235in}{1.517119in}}%
\pgfpathcurveto{\pgfqpoint{2.285235in}{1.508882in}}{\pgfqpoint{2.288507in}{1.500982in}}{\pgfqpoint{2.294331in}{1.495158in}}%
\pgfpathcurveto{\pgfqpoint{2.300155in}{1.489335in}}{\pgfqpoint{2.308055in}{1.486062in}}{\pgfqpoint{2.316292in}{1.486062in}}%
\pgfpathclose%
\pgfusepath{stroke,fill}%
\end{pgfscope}%
\begin{pgfscope}%
\pgfpathrectangle{\pgfqpoint{0.100000in}{0.220728in}}{\pgfqpoint{3.696000in}{3.696000in}}%
\pgfusepath{clip}%
\pgfsetbuttcap%
\pgfsetroundjoin%
\definecolor{currentfill}{rgb}{0.121569,0.466667,0.705882}%
\pgfsetfillcolor{currentfill}%
\pgfsetfillopacity{0.873717}%
\pgfsetlinewidth{1.003750pt}%
\definecolor{currentstroke}{rgb}{0.121569,0.466667,0.705882}%
\pgfsetstrokecolor{currentstroke}%
\pgfsetstrokeopacity{0.873717}%
\pgfsetdash{}{0pt}%
\pgfpathmoveto{\pgfqpoint{2.317302in}{1.484839in}}%
\pgfpathcurveto{\pgfqpoint{2.325539in}{1.484839in}}{\pgfqpoint{2.333439in}{1.488111in}}{\pgfqpoint{2.339263in}{1.493935in}}%
\pgfpathcurveto{\pgfqpoint{2.345087in}{1.499759in}}{\pgfqpoint{2.348359in}{1.507659in}}{\pgfqpoint{2.348359in}{1.515896in}}%
\pgfpathcurveto{\pgfqpoint{2.348359in}{1.524132in}}{\pgfqpoint{2.345087in}{1.532032in}}{\pgfqpoint{2.339263in}{1.537856in}}%
\pgfpathcurveto{\pgfqpoint{2.333439in}{1.543680in}}{\pgfqpoint{2.325539in}{1.546952in}}{\pgfqpoint{2.317302in}{1.546952in}}%
\pgfpathcurveto{\pgfqpoint{2.309066in}{1.546952in}}{\pgfqpoint{2.301166in}{1.543680in}}{\pgfqpoint{2.295342in}{1.537856in}}%
\pgfpathcurveto{\pgfqpoint{2.289518in}{1.532032in}}{\pgfqpoint{2.286246in}{1.524132in}}{\pgfqpoint{2.286246in}{1.515896in}}%
\pgfpathcurveto{\pgfqpoint{2.286246in}{1.507659in}}{\pgfqpoint{2.289518in}{1.499759in}}{\pgfqpoint{2.295342in}{1.493935in}}%
\pgfpathcurveto{\pgfqpoint{2.301166in}{1.488111in}}{\pgfqpoint{2.309066in}{1.484839in}}{\pgfqpoint{2.317302in}{1.484839in}}%
\pgfpathclose%
\pgfusepath{stroke,fill}%
\end{pgfscope}%
\begin{pgfscope}%
\pgfpathrectangle{\pgfqpoint{0.100000in}{0.220728in}}{\pgfqpoint{3.696000in}{3.696000in}}%
\pgfusepath{clip}%
\pgfsetbuttcap%
\pgfsetroundjoin%
\definecolor{currentfill}{rgb}{0.121569,0.466667,0.705882}%
\pgfsetfillcolor{currentfill}%
\pgfsetfillopacity{0.874278}%
\pgfsetlinewidth{1.003750pt}%
\definecolor{currentstroke}{rgb}{0.121569,0.466667,0.705882}%
\pgfsetstrokecolor{currentstroke}%
\pgfsetstrokeopacity{0.874278}%
\pgfsetdash{}{0pt}%
\pgfpathmoveto{\pgfqpoint{2.317976in}{1.482591in}}%
\pgfpathcurveto{\pgfqpoint{2.326212in}{1.482591in}}{\pgfqpoint{2.334112in}{1.485864in}}{\pgfqpoint{2.339936in}{1.491688in}}%
\pgfpathcurveto{\pgfqpoint{2.345760in}{1.497512in}}{\pgfqpoint{2.349032in}{1.505412in}}{\pgfqpoint{2.349032in}{1.513648in}}%
\pgfpathcurveto{\pgfqpoint{2.349032in}{1.521884in}}{\pgfqpoint{2.345760in}{1.529784in}}{\pgfqpoint{2.339936in}{1.535608in}}%
\pgfpathcurveto{\pgfqpoint{2.334112in}{1.541432in}}{\pgfqpoint{2.326212in}{1.544704in}}{\pgfqpoint{2.317976in}{1.544704in}}%
\pgfpathcurveto{\pgfqpoint{2.309739in}{1.544704in}}{\pgfqpoint{2.301839in}{1.541432in}}{\pgfqpoint{2.296015in}{1.535608in}}%
\pgfpathcurveto{\pgfqpoint{2.290191in}{1.529784in}}{\pgfqpoint{2.286919in}{1.521884in}}{\pgfqpoint{2.286919in}{1.513648in}}%
\pgfpathcurveto{\pgfqpoint{2.286919in}{1.505412in}}{\pgfqpoint{2.290191in}{1.497512in}}{\pgfqpoint{2.296015in}{1.491688in}}%
\pgfpathcurveto{\pgfqpoint{2.301839in}{1.485864in}}{\pgfqpoint{2.309739in}{1.482591in}}{\pgfqpoint{2.317976in}{1.482591in}}%
\pgfpathclose%
\pgfusepath{stroke,fill}%
\end{pgfscope}%
\begin{pgfscope}%
\pgfpathrectangle{\pgfqpoint{0.100000in}{0.220728in}}{\pgfqpoint{3.696000in}{3.696000in}}%
\pgfusepath{clip}%
\pgfsetbuttcap%
\pgfsetroundjoin%
\definecolor{currentfill}{rgb}{0.121569,0.466667,0.705882}%
\pgfsetfillcolor{currentfill}%
\pgfsetfillopacity{0.874749}%
\pgfsetlinewidth{1.003750pt}%
\definecolor{currentstroke}{rgb}{0.121569,0.466667,0.705882}%
\pgfsetstrokecolor{currentstroke}%
\pgfsetstrokeopacity{0.874749}%
\pgfsetdash{}{0pt}%
\pgfpathmoveto{\pgfqpoint{2.318333in}{1.482345in}}%
\pgfpathcurveto{\pgfqpoint{2.326569in}{1.482345in}}{\pgfqpoint{2.334469in}{1.485617in}}{\pgfqpoint{2.340293in}{1.491441in}}%
\pgfpathcurveto{\pgfqpoint{2.346117in}{1.497265in}}{\pgfqpoint{2.349390in}{1.505165in}}{\pgfqpoint{2.349390in}{1.513402in}}%
\pgfpathcurveto{\pgfqpoint{2.349390in}{1.521638in}}{\pgfqpoint{2.346117in}{1.529538in}}{\pgfqpoint{2.340293in}{1.535362in}}%
\pgfpathcurveto{\pgfqpoint{2.334469in}{1.541186in}}{\pgfqpoint{2.326569in}{1.544458in}}{\pgfqpoint{2.318333in}{1.544458in}}%
\pgfpathcurveto{\pgfqpoint{2.310097in}{1.544458in}}{\pgfqpoint{2.302197in}{1.541186in}}{\pgfqpoint{2.296373in}{1.535362in}}%
\pgfpathcurveto{\pgfqpoint{2.290549in}{1.529538in}}{\pgfqpoint{2.287277in}{1.521638in}}{\pgfqpoint{2.287277in}{1.513402in}}%
\pgfpathcurveto{\pgfqpoint{2.287277in}{1.505165in}}{\pgfqpoint{2.290549in}{1.497265in}}{\pgfqpoint{2.296373in}{1.491441in}}%
\pgfpathcurveto{\pgfqpoint{2.302197in}{1.485617in}}{\pgfqpoint{2.310097in}{1.482345in}}{\pgfqpoint{2.318333in}{1.482345in}}%
\pgfpathclose%
\pgfusepath{stroke,fill}%
\end{pgfscope}%
\begin{pgfscope}%
\pgfpathrectangle{\pgfqpoint{0.100000in}{0.220728in}}{\pgfqpoint{3.696000in}{3.696000in}}%
\pgfusepath{clip}%
\pgfsetbuttcap%
\pgfsetroundjoin%
\definecolor{currentfill}{rgb}{0.121569,0.466667,0.705882}%
\pgfsetfillcolor{currentfill}%
\pgfsetfillopacity{0.875785}%
\pgfsetlinewidth{1.003750pt}%
\definecolor{currentstroke}{rgb}{0.121569,0.466667,0.705882}%
\pgfsetstrokecolor{currentstroke}%
\pgfsetstrokeopacity{0.875785}%
\pgfsetdash{}{0pt}%
\pgfpathmoveto{\pgfqpoint{2.319073in}{1.481941in}}%
\pgfpathcurveto{\pgfqpoint{2.327310in}{1.481941in}}{\pgfqpoint{2.335210in}{1.485213in}}{\pgfqpoint{2.341034in}{1.491037in}}%
\pgfpathcurveto{\pgfqpoint{2.346857in}{1.496861in}}{\pgfqpoint{2.350130in}{1.504761in}}{\pgfqpoint{2.350130in}{1.512997in}}%
\pgfpathcurveto{\pgfqpoint{2.350130in}{1.521233in}}{\pgfqpoint{2.346857in}{1.529133in}}{\pgfqpoint{2.341034in}{1.534957in}}%
\pgfpathcurveto{\pgfqpoint{2.335210in}{1.540781in}}{\pgfqpoint{2.327310in}{1.544054in}}{\pgfqpoint{2.319073in}{1.544054in}}%
\pgfpathcurveto{\pgfqpoint{2.310837in}{1.544054in}}{\pgfqpoint{2.302937in}{1.540781in}}{\pgfqpoint{2.297113in}{1.534957in}}%
\pgfpathcurveto{\pgfqpoint{2.291289in}{1.529133in}}{\pgfqpoint{2.288017in}{1.521233in}}{\pgfqpoint{2.288017in}{1.512997in}}%
\pgfpathcurveto{\pgfqpoint{2.288017in}{1.504761in}}{\pgfqpoint{2.291289in}{1.496861in}}{\pgfqpoint{2.297113in}{1.491037in}}%
\pgfpathcurveto{\pgfqpoint{2.302937in}{1.485213in}}{\pgfqpoint{2.310837in}{1.481941in}}{\pgfqpoint{2.319073in}{1.481941in}}%
\pgfpathclose%
\pgfusepath{stroke,fill}%
\end{pgfscope}%
\begin{pgfscope}%
\pgfpathrectangle{\pgfqpoint{0.100000in}{0.220728in}}{\pgfqpoint{3.696000in}{3.696000in}}%
\pgfusepath{clip}%
\pgfsetbuttcap%
\pgfsetroundjoin%
\definecolor{currentfill}{rgb}{0.121569,0.466667,0.705882}%
\pgfsetfillcolor{currentfill}%
\pgfsetfillopacity{0.877117}%
\pgfsetlinewidth{1.003750pt}%
\definecolor{currentstroke}{rgb}{0.121569,0.466667,0.705882}%
\pgfsetstrokecolor{currentstroke}%
\pgfsetstrokeopacity{0.877117}%
\pgfsetdash{}{0pt}%
\pgfpathmoveto{\pgfqpoint{2.319857in}{1.479347in}}%
\pgfpathcurveto{\pgfqpoint{2.328093in}{1.479347in}}{\pgfqpoint{2.335993in}{1.482620in}}{\pgfqpoint{2.341817in}{1.488444in}}%
\pgfpathcurveto{\pgfqpoint{2.347641in}{1.494268in}}{\pgfqpoint{2.350914in}{1.502168in}}{\pgfqpoint{2.350914in}{1.510404in}}%
\pgfpathcurveto{\pgfqpoint{2.350914in}{1.518640in}}{\pgfqpoint{2.347641in}{1.526540in}}{\pgfqpoint{2.341817in}{1.532364in}}%
\pgfpathcurveto{\pgfqpoint{2.335993in}{1.538188in}}{\pgfqpoint{2.328093in}{1.541460in}}{\pgfqpoint{2.319857in}{1.541460in}}%
\pgfpathcurveto{\pgfqpoint{2.311621in}{1.541460in}}{\pgfqpoint{2.303721in}{1.538188in}}{\pgfqpoint{2.297897in}{1.532364in}}%
\pgfpathcurveto{\pgfqpoint{2.292073in}{1.526540in}}{\pgfqpoint{2.288801in}{1.518640in}}{\pgfqpoint{2.288801in}{1.510404in}}%
\pgfpathcurveto{\pgfqpoint{2.288801in}{1.502168in}}{\pgfqpoint{2.292073in}{1.494268in}}{\pgfqpoint{2.297897in}{1.488444in}}%
\pgfpathcurveto{\pgfqpoint{2.303721in}{1.482620in}}{\pgfqpoint{2.311621in}{1.479347in}}{\pgfqpoint{2.319857in}{1.479347in}}%
\pgfpathclose%
\pgfusepath{stroke,fill}%
\end{pgfscope}%
\begin{pgfscope}%
\pgfpathrectangle{\pgfqpoint{0.100000in}{0.220728in}}{\pgfqpoint{3.696000in}{3.696000in}}%
\pgfusepath{clip}%
\pgfsetbuttcap%
\pgfsetroundjoin%
\definecolor{currentfill}{rgb}{0.121569,0.466667,0.705882}%
\pgfsetfillcolor{currentfill}%
\pgfsetfillopacity{0.877365}%
\pgfsetlinewidth{1.003750pt}%
\definecolor{currentstroke}{rgb}{0.121569,0.466667,0.705882}%
\pgfsetstrokecolor{currentstroke}%
\pgfsetstrokeopacity{0.877365}%
\pgfsetdash{}{0pt}%
\pgfpathmoveto{\pgfqpoint{1.178449in}{2.159211in}}%
\pgfpathcurveto{\pgfqpoint{1.186686in}{2.159211in}}{\pgfqpoint{1.194586in}{2.162483in}}{\pgfqpoint{1.200410in}{2.168307in}}%
\pgfpathcurveto{\pgfqpoint{1.206234in}{2.174131in}}{\pgfqpoint{1.209506in}{2.182031in}}{\pgfqpoint{1.209506in}{2.190267in}}%
\pgfpathcurveto{\pgfqpoint{1.209506in}{2.198504in}}{\pgfqpoint{1.206234in}{2.206404in}}{\pgfqpoint{1.200410in}{2.212228in}}%
\pgfpathcurveto{\pgfqpoint{1.194586in}{2.218052in}}{\pgfqpoint{1.186686in}{2.221324in}}{\pgfqpoint{1.178449in}{2.221324in}}%
\pgfpathcurveto{\pgfqpoint{1.170213in}{2.221324in}}{\pgfqpoint{1.162313in}{2.218052in}}{\pgfqpoint{1.156489in}{2.212228in}}%
\pgfpathcurveto{\pgfqpoint{1.150665in}{2.206404in}}{\pgfqpoint{1.147393in}{2.198504in}}{\pgfqpoint{1.147393in}{2.190267in}}%
\pgfpathcurveto{\pgfqpoint{1.147393in}{2.182031in}}{\pgfqpoint{1.150665in}{2.174131in}}{\pgfqpoint{1.156489in}{2.168307in}}%
\pgfpathcurveto{\pgfqpoint{1.162313in}{2.162483in}}{\pgfqpoint{1.170213in}{2.159211in}}{\pgfqpoint{1.178449in}{2.159211in}}%
\pgfpathclose%
\pgfusepath{stroke,fill}%
\end{pgfscope}%
\begin{pgfscope}%
\pgfpathrectangle{\pgfqpoint{0.100000in}{0.220728in}}{\pgfqpoint{3.696000in}{3.696000in}}%
\pgfusepath{clip}%
\pgfsetbuttcap%
\pgfsetroundjoin%
\definecolor{currentfill}{rgb}{0.121569,0.466667,0.705882}%
\pgfsetfillcolor{currentfill}%
\pgfsetfillopacity{0.879190}%
\pgfsetlinewidth{1.003750pt}%
\definecolor{currentstroke}{rgb}{0.121569,0.466667,0.705882}%
\pgfsetstrokecolor{currentstroke}%
\pgfsetstrokeopacity{0.879190}%
\pgfsetdash{}{0pt}%
\pgfpathmoveto{\pgfqpoint{2.321087in}{1.475764in}}%
\pgfpathcurveto{\pgfqpoint{2.329323in}{1.475764in}}{\pgfqpoint{2.337223in}{1.479036in}}{\pgfqpoint{2.343047in}{1.484860in}}%
\pgfpathcurveto{\pgfqpoint{2.348871in}{1.490684in}}{\pgfqpoint{2.352144in}{1.498584in}}{\pgfqpoint{2.352144in}{1.506820in}}%
\pgfpathcurveto{\pgfqpoint{2.352144in}{1.515057in}}{\pgfqpoint{2.348871in}{1.522957in}}{\pgfqpoint{2.343047in}{1.528780in}}%
\pgfpathcurveto{\pgfqpoint{2.337223in}{1.534604in}}{\pgfqpoint{2.329323in}{1.537877in}}{\pgfqpoint{2.321087in}{1.537877in}}%
\pgfpathcurveto{\pgfqpoint{2.312851in}{1.537877in}}{\pgfqpoint{2.304951in}{1.534604in}}{\pgfqpoint{2.299127in}{1.528780in}}%
\pgfpathcurveto{\pgfqpoint{2.293303in}{1.522957in}}{\pgfqpoint{2.290031in}{1.515057in}}{\pgfqpoint{2.290031in}{1.506820in}}%
\pgfpathcurveto{\pgfqpoint{2.290031in}{1.498584in}}{\pgfqpoint{2.293303in}{1.490684in}}{\pgfqpoint{2.299127in}{1.484860in}}%
\pgfpathcurveto{\pgfqpoint{2.304951in}{1.479036in}}{\pgfqpoint{2.312851in}{1.475764in}}{\pgfqpoint{2.321087in}{1.475764in}}%
\pgfpathclose%
\pgfusepath{stroke,fill}%
\end{pgfscope}%
\begin{pgfscope}%
\pgfpathrectangle{\pgfqpoint{0.100000in}{0.220728in}}{\pgfqpoint{3.696000in}{3.696000in}}%
\pgfusepath{clip}%
\pgfsetbuttcap%
\pgfsetroundjoin%
\definecolor{currentfill}{rgb}{0.121569,0.466667,0.705882}%
\pgfsetfillcolor{currentfill}%
\pgfsetfillopacity{0.880408}%
\pgfsetlinewidth{1.003750pt}%
\definecolor{currentstroke}{rgb}{0.121569,0.466667,0.705882}%
\pgfsetstrokecolor{currentstroke}%
\pgfsetstrokeopacity{0.880408}%
\pgfsetdash{}{0pt}%
\pgfpathmoveto{\pgfqpoint{2.321834in}{1.474296in}}%
\pgfpathcurveto{\pgfqpoint{2.330070in}{1.474296in}}{\pgfqpoint{2.337970in}{1.477569in}}{\pgfqpoint{2.343794in}{1.483393in}}%
\pgfpathcurveto{\pgfqpoint{2.349618in}{1.489217in}}{\pgfqpoint{2.352891in}{1.497117in}}{\pgfqpoint{2.352891in}{1.505353in}}%
\pgfpathcurveto{\pgfqpoint{2.352891in}{1.513589in}}{\pgfqpoint{2.349618in}{1.521489in}}{\pgfqpoint{2.343794in}{1.527313in}}%
\pgfpathcurveto{\pgfqpoint{2.337970in}{1.533137in}}{\pgfqpoint{2.330070in}{1.536409in}}{\pgfqpoint{2.321834in}{1.536409in}}%
\pgfpathcurveto{\pgfqpoint{2.313598in}{1.536409in}}{\pgfqpoint{2.305698in}{1.533137in}}{\pgfqpoint{2.299874in}{1.527313in}}%
\pgfpathcurveto{\pgfqpoint{2.294050in}{1.521489in}}{\pgfqpoint{2.290778in}{1.513589in}}{\pgfqpoint{2.290778in}{1.505353in}}%
\pgfpathcurveto{\pgfqpoint{2.290778in}{1.497117in}}{\pgfqpoint{2.294050in}{1.489217in}}{\pgfqpoint{2.299874in}{1.483393in}}%
\pgfpathcurveto{\pgfqpoint{2.305698in}{1.477569in}}{\pgfqpoint{2.313598in}{1.474296in}}{\pgfqpoint{2.321834in}{1.474296in}}%
\pgfpathclose%
\pgfusepath{stroke,fill}%
\end{pgfscope}%
\begin{pgfscope}%
\pgfpathrectangle{\pgfqpoint{0.100000in}{0.220728in}}{\pgfqpoint{3.696000in}{3.696000in}}%
\pgfusepath{clip}%
\pgfsetbuttcap%
\pgfsetroundjoin%
\definecolor{currentfill}{rgb}{0.121569,0.466667,0.705882}%
\pgfsetfillcolor{currentfill}%
\pgfsetfillopacity{0.881081}%
\pgfsetlinewidth{1.003750pt}%
\definecolor{currentstroke}{rgb}{0.121569,0.466667,0.705882}%
\pgfsetstrokecolor{currentstroke}%
\pgfsetstrokeopacity{0.881081}%
\pgfsetdash{}{0pt}%
\pgfpathmoveto{\pgfqpoint{2.322369in}{1.473578in}}%
\pgfpathcurveto{\pgfqpoint{2.330605in}{1.473578in}}{\pgfqpoint{2.338505in}{1.476851in}}{\pgfqpoint{2.344329in}{1.482674in}}%
\pgfpathcurveto{\pgfqpoint{2.350153in}{1.488498in}}{\pgfqpoint{2.353425in}{1.496398in}}{\pgfqpoint{2.353425in}{1.504635in}}%
\pgfpathcurveto{\pgfqpoint{2.353425in}{1.512871in}}{\pgfqpoint{2.350153in}{1.520771in}}{\pgfqpoint{2.344329in}{1.526595in}}%
\pgfpathcurveto{\pgfqpoint{2.338505in}{1.532419in}}{\pgfqpoint{2.330605in}{1.535691in}}{\pgfqpoint{2.322369in}{1.535691in}}%
\pgfpathcurveto{\pgfqpoint{2.314133in}{1.535691in}}{\pgfqpoint{2.306232in}{1.532419in}}{\pgfqpoint{2.300409in}{1.526595in}}%
\pgfpathcurveto{\pgfqpoint{2.294585in}{1.520771in}}{\pgfqpoint{2.291312in}{1.512871in}}{\pgfqpoint{2.291312in}{1.504635in}}%
\pgfpathcurveto{\pgfqpoint{2.291312in}{1.496398in}}{\pgfqpoint{2.294585in}{1.488498in}}{\pgfqpoint{2.300409in}{1.482674in}}%
\pgfpathcurveto{\pgfqpoint{2.306232in}{1.476851in}}{\pgfqpoint{2.314133in}{1.473578in}}{\pgfqpoint{2.322369in}{1.473578in}}%
\pgfpathclose%
\pgfusepath{stroke,fill}%
\end{pgfscope}%
\begin{pgfscope}%
\pgfpathrectangle{\pgfqpoint{0.100000in}{0.220728in}}{\pgfqpoint{3.696000in}{3.696000in}}%
\pgfusepath{clip}%
\pgfsetbuttcap%
\pgfsetroundjoin%
\definecolor{currentfill}{rgb}{0.121569,0.466667,0.705882}%
\pgfsetfillcolor{currentfill}%
\pgfsetfillopacity{0.881439}%
\pgfsetlinewidth{1.003750pt}%
\definecolor{currentstroke}{rgb}{0.121569,0.466667,0.705882}%
\pgfsetstrokecolor{currentstroke}%
\pgfsetstrokeopacity{0.881439}%
\pgfsetdash{}{0pt}%
\pgfpathmoveto{\pgfqpoint{2.322606in}{1.473073in}}%
\pgfpathcurveto{\pgfqpoint{2.330842in}{1.473073in}}{\pgfqpoint{2.338742in}{1.476345in}}{\pgfqpoint{2.344566in}{1.482169in}}%
\pgfpathcurveto{\pgfqpoint{2.350390in}{1.487993in}}{\pgfqpoint{2.353662in}{1.495893in}}{\pgfqpoint{2.353662in}{1.504130in}}%
\pgfpathcurveto{\pgfqpoint{2.353662in}{1.512366in}}{\pgfqpoint{2.350390in}{1.520266in}}{\pgfqpoint{2.344566in}{1.526090in}}%
\pgfpathcurveto{\pgfqpoint{2.338742in}{1.531914in}}{\pgfqpoint{2.330842in}{1.535186in}}{\pgfqpoint{2.322606in}{1.535186in}}%
\pgfpathcurveto{\pgfqpoint{2.314369in}{1.535186in}}{\pgfqpoint{2.306469in}{1.531914in}}{\pgfqpoint{2.300645in}{1.526090in}}%
\pgfpathcurveto{\pgfqpoint{2.294822in}{1.520266in}}{\pgfqpoint{2.291549in}{1.512366in}}{\pgfqpoint{2.291549in}{1.504130in}}%
\pgfpathcurveto{\pgfqpoint{2.291549in}{1.495893in}}{\pgfqpoint{2.294822in}{1.487993in}}{\pgfqpoint{2.300645in}{1.482169in}}%
\pgfpathcurveto{\pgfqpoint{2.306469in}{1.476345in}}{\pgfqpoint{2.314369in}{1.473073in}}{\pgfqpoint{2.322606in}{1.473073in}}%
\pgfpathclose%
\pgfusepath{stroke,fill}%
\end{pgfscope}%
\begin{pgfscope}%
\pgfpathrectangle{\pgfqpoint{0.100000in}{0.220728in}}{\pgfqpoint{3.696000in}{3.696000in}}%
\pgfusepath{clip}%
\pgfsetbuttcap%
\pgfsetroundjoin%
\definecolor{currentfill}{rgb}{0.121569,0.466667,0.705882}%
\pgfsetfillcolor{currentfill}%
\pgfsetfillopacity{0.882153}%
\pgfsetlinewidth{1.003750pt}%
\definecolor{currentstroke}{rgb}{0.121569,0.466667,0.705882}%
\pgfsetstrokecolor{currentstroke}%
\pgfsetstrokeopacity{0.882153}%
\pgfsetdash{}{0pt}%
\pgfpathmoveto{\pgfqpoint{2.323297in}{1.472470in}}%
\pgfpathcurveto{\pgfqpoint{2.331533in}{1.472470in}}{\pgfqpoint{2.339434in}{1.475742in}}{\pgfqpoint{2.345257in}{1.481566in}}%
\pgfpathcurveto{\pgfqpoint{2.351081in}{1.487390in}}{\pgfqpoint{2.354354in}{1.495290in}}{\pgfqpoint{2.354354in}{1.503526in}}%
\pgfpathcurveto{\pgfqpoint{2.354354in}{1.511763in}}{\pgfqpoint{2.351081in}{1.519663in}}{\pgfqpoint{2.345257in}{1.525487in}}%
\pgfpathcurveto{\pgfqpoint{2.339434in}{1.531311in}}{\pgfqpoint{2.331533in}{1.534583in}}{\pgfqpoint{2.323297in}{1.534583in}}%
\pgfpathcurveto{\pgfqpoint{2.315061in}{1.534583in}}{\pgfqpoint{2.307161in}{1.531311in}}{\pgfqpoint{2.301337in}{1.525487in}}%
\pgfpathcurveto{\pgfqpoint{2.295513in}{1.519663in}}{\pgfqpoint{2.292241in}{1.511763in}}{\pgfqpoint{2.292241in}{1.503526in}}%
\pgfpathcurveto{\pgfqpoint{2.292241in}{1.495290in}}{\pgfqpoint{2.295513in}{1.487390in}}{\pgfqpoint{2.301337in}{1.481566in}}%
\pgfpathcurveto{\pgfqpoint{2.307161in}{1.475742in}}{\pgfqpoint{2.315061in}{1.472470in}}{\pgfqpoint{2.323297in}{1.472470in}}%
\pgfpathclose%
\pgfusepath{stroke,fill}%
\end{pgfscope}%
\begin{pgfscope}%
\pgfpathrectangle{\pgfqpoint{0.100000in}{0.220728in}}{\pgfqpoint{3.696000in}{3.696000in}}%
\pgfusepath{clip}%
\pgfsetbuttcap%
\pgfsetroundjoin%
\definecolor{currentfill}{rgb}{0.121569,0.466667,0.705882}%
\pgfsetfillcolor{currentfill}%
\pgfsetfillopacity{0.882468}%
\pgfsetlinewidth{1.003750pt}%
\definecolor{currentstroke}{rgb}{0.121569,0.466667,0.705882}%
\pgfsetstrokecolor{currentstroke}%
\pgfsetstrokeopacity{0.882468}%
\pgfsetdash{}{0pt}%
\pgfpathmoveto{\pgfqpoint{2.323487in}{1.471553in}}%
\pgfpathcurveto{\pgfqpoint{2.331723in}{1.471553in}}{\pgfqpoint{2.339623in}{1.474825in}}{\pgfqpoint{2.345447in}{1.480649in}}%
\pgfpathcurveto{\pgfqpoint{2.351271in}{1.486473in}}{\pgfqpoint{2.354543in}{1.494373in}}{\pgfqpoint{2.354543in}{1.502609in}}%
\pgfpathcurveto{\pgfqpoint{2.354543in}{1.510846in}}{\pgfqpoint{2.351271in}{1.518746in}}{\pgfqpoint{2.345447in}{1.524570in}}%
\pgfpathcurveto{\pgfqpoint{2.339623in}{1.530393in}}{\pgfqpoint{2.331723in}{1.533666in}}{\pgfqpoint{2.323487in}{1.533666in}}%
\pgfpathcurveto{\pgfqpoint{2.315250in}{1.533666in}}{\pgfqpoint{2.307350in}{1.530393in}}{\pgfqpoint{2.301526in}{1.524570in}}%
\pgfpathcurveto{\pgfqpoint{2.295703in}{1.518746in}}{\pgfqpoint{2.292430in}{1.510846in}}{\pgfqpoint{2.292430in}{1.502609in}}%
\pgfpathcurveto{\pgfqpoint{2.292430in}{1.494373in}}{\pgfqpoint{2.295703in}{1.486473in}}{\pgfqpoint{2.301526in}{1.480649in}}%
\pgfpathcurveto{\pgfqpoint{2.307350in}{1.474825in}}{\pgfqpoint{2.315250in}{1.471553in}}{\pgfqpoint{2.323487in}{1.471553in}}%
\pgfpathclose%
\pgfusepath{stroke,fill}%
\end{pgfscope}%
\begin{pgfscope}%
\pgfpathrectangle{\pgfqpoint{0.100000in}{0.220728in}}{\pgfqpoint{3.696000in}{3.696000in}}%
\pgfusepath{clip}%
\pgfsetbuttcap%
\pgfsetroundjoin%
\definecolor{currentfill}{rgb}{0.121569,0.466667,0.705882}%
\pgfsetfillcolor{currentfill}%
\pgfsetfillopacity{0.883185}%
\pgfsetlinewidth{1.003750pt}%
\definecolor{currentstroke}{rgb}{0.121569,0.466667,0.705882}%
\pgfsetstrokecolor{currentstroke}%
\pgfsetstrokeopacity{0.883185}%
\pgfsetdash{}{0pt}%
\pgfpathmoveto{\pgfqpoint{2.324079in}{1.471466in}}%
\pgfpathcurveto{\pgfqpoint{2.332316in}{1.471466in}}{\pgfqpoint{2.340216in}{1.474738in}}{\pgfqpoint{2.346040in}{1.480562in}}%
\pgfpathcurveto{\pgfqpoint{2.351864in}{1.486386in}}{\pgfqpoint{2.355136in}{1.494286in}}{\pgfqpoint{2.355136in}{1.502523in}}%
\pgfpathcurveto{\pgfqpoint{2.355136in}{1.510759in}}{\pgfqpoint{2.351864in}{1.518659in}}{\pgfqpoint{2.346040in}{1.524483in}}%
\pgfpathcurveto{\pgfqpoint{2.340216in}{1.530307in}}{\pgfqpoint{2.332316in}{1.533579in}}{\pgfqpoint{2.324079in}{1.533579in}}%
\pgfpathcurveto{\pgfqpoint{2.315843in}{1.533579in}}{\pgfqpoint{2.307943in}{1.530307in}}{\pgfqpoint{2.302119in}{1.524483in}}%
\pgfpathcurveto{\pgfqpoint{2.296295in}{1.518659in}}{\pgfqpoint{2.293023in}{1.510759in}}{\pgfqpoint{2.293023in}{1.502523in}}%
\pgfpathcurveto{\pgfqpoint{2.293023in}{1.494286in}}{\pgfqpoint{2.296295in}{1.486386in}}{\pgfqpoint{2.302119in}{1.480562in}}%
\pgfpathcurveto{\pgfqpoint{2.307943in}{1.474738in}}{\pgfqpoint{2.315843in}{1.471466in}}{\pgfqpoint{2.324079in}{1.471466in}}%
\pgfpathclose%
\pgfusepath{stroke,fill}%
\end{pgfscope}%
\begin{pgfscope}%
\pgfpathrectangle{\pgfqpoint{0.100000in}{0.220728in}}{\pgfqpoint{3.696000in}{3.696000in}}%
\pgfusepath{clip}%
\pgfsetbuttcap%
\pgfsetroundjoin%
\definecolor{currentfill}{rgb}{0.121569,0.466667,0.705882}%
\pgfsetfillcolor{currentfill}%
\pgfsetfillopacity{0.883551}%
\pgfsetlinewidth{1.003750pt}%
\definecolor{currentstroke}{rgb}{0.121569,0.466667,0.705882}%
\pgfsetstrokecolor{currentstroke}%
\pgfsetstrokeopacity{0.883551}%
\pgfsetdash{}{0pt}%
\pgfpathmoveto{\pgfqpoint{2.324347in}{1.471202in}}%
\pgfpathcurveto{\pgfqpoint{2.332584in}{1.471202in}}{\pgfqpoint{2.340484in}{1.474474in}}{\pgfqpoint{2.346307in}{1.480298in}}%
\pgfpathcurveto{\pgfqpoint{2.352131in}{1.486122in}}{\pgfqpoint{2.355404in}{1.494022in}}{\pgfqpoint{2.355404in}{1.502258in}}%
\pgfpathcurveto{\pgfqpoint{2.355404in}{1.510495in}}{\pgfqpoint{2.352131in}{1.518395in}}{\pgfqpoint{2.346307in}{1.524219in}}%
\pgfpathcurveto{\pgfqpoint{2.340484in}{1.530042in}}{\pgfqpoint{2.332584in}{1.533315in}}{\pgfqpoint{2.324347in}{1.533315in}}%
\pgfpathcurveto{\pgfqpoint{2.316111in}{1.533315in}}{\pgfqpoint{2.308211in}{1.530042in}}{\pgfqpoint{2.302387in}{1.524219in}}%
\pgfpathcurveto{\pgfqpoint{2.296563in}{1.518395in}}{\pgfqpoint{2.293291in}{1.510495in}}{\pgfqpoint{2.293291in}{1.502258in}}%
\pgfpathcurveto{\pgfqpoint{2.293291in}{1.494022in}}{\pgfqpoint{2.296563in}{1.486122in}}{\pgfqpoint{2.302387in}{1.480298in}}%
\pgfpathcurveto{\pgfqpoint{2.308211in}{1.474474in}}{\pgfqpoint{2.316111in}{1.471202in}}{\pgfqpoint{2.324347in}{1.471202in}}%
\pgfpathclose%
\pgfusepath{stroke,fill}%
\end{pgfscope}%
\begin{pgfscope}%
\pgfpathrectangle{\pgfqpoint{0.100000in}{0.220728in}}{\pgfqpoint{3.696000in}{3.696000in}}%
\pgfusepath{clip}%
\pgfsetbuttcap%
\pgfsetroundjoin%
\definecolor{currentfill}{rgb}{0.121569,0.466667,0.705882}%
\pgfsetfillcolor{currentfill}%
\pgfsetfillopacity{0.883717}%
\pgfsetlinewidth{1.003750pt}%
\definecolor{currentstroke}{rgb}{0.121569,0.466667,0.705882}%
\pgfsetstrokecolor{currentstroke}%
\pgfsetstrokeopacity{0.883717}%
\pgfsetdash{}{0pt}%
\pgfpathmoveto{\pgfqpoint{2.324409in}{1.470807in}}%
\pgfpathcurveto{\pgfqpoint{2.332645in}{1.470807in}}{\pgfqpoint{2.340545in}{1.474079in}}{\pgfqpoint{2.346369in}{1.479903in}}%
\pgfpathcurveto{\pgfqpoint{2.352193in}{1.485727in}}{\pgfqpoint{2.355466in}{1.493627in}}{\pgfqpoint{2.355466in}{1.501863in}}%
\pgfpathcurveto{\pgfqpoint{2.355466in}{1.510100in}}{\pgfqpoint{2.352193in}{1.518000in}}{\pgfqpoint{2.346369in}{1.523824in}}%
\pgfpathcurveto{\pgfqpoint{2.340545in}{1.529647in}}{\pgfqpoint{2.332645in}{1.532920in}}{\pgfqpoint{2.324409in}{1.532920in}}%
\pgfpathcurveto{\pgfqpoint{2.316173in}{1.532920in}}{\pgfqpoint{2.308273in}{1.529647in}}{\pgfqpoint{2.302449in}{1.523824in}}%
\pgfpathcurveto{\pgfqpoint{2.296625in}{1.518000in}}{\pgfqpoint{2.293353in}{1.510100in}}{\pgfqpoint{2.293353in}{1.501863in}}%
\pgfpathcurveto{\pgfqpoint{2.293353in}{1.493627in}}{\pgfqpoint{2.296625in}{1.485727in}}{\pgfqpoint{2.302449in}{1.479903in}}%
\pgfpathcurveto{\pgfqpoint{2.308273in}{1.474079in}}{\pgfqpoint{2.316173in}{1.470807in}}{\pgfqpoint{2.324409in}{1.470807in}}%
\pgfpathclose%
\pgfusepath{stroke,fill}%
\end{pgfscope}%
\begin{pgfscope}%
\pgfpathrectangle{\pgfqpoint{0.100000in}{0.220728in}}{\pgfqpoint{3.696000in}{3.696000in}}%
\pgfusepath{clip}%
\pgfsetbuttcap%
\pgfsetroundjoin%
\definecolor{currentfill}{rgb}{0.121569,0.466667,0.705882}%
\pgfsetfillcolor{currentfill}%
\pgfsetfillopacity{0.883866}%
\pgfsetlinewidth{1.003750pt}%
\definecolor{currentstroke}{rgb}{0.121569,0.466667,0.705882}%
\pgfsetstrokecolor{currentstroke}%
\pgfsetstrokeopacity{0.883866}%
\pgfsetdash{}{0pt}%
\pgfpathmoveto{\pgfqpoint{1.228095in}{2.129665in}}%
\pgfpathcurveto{\pgfqpoint{1.236331in}{2.129665in}}{\pgfqpoint{1.244231in}{2.132938in}}{\pgfqpoint{1.250055in}{2.138762in}}%
\pgfpathcurveto{\pgfqpoint{1.255879in}{2.144586in}}{\pgfqpoint{1.259151in}{2.152486in}}{\pgfqpoint{1.259151in}{2.160722in}}%
\pgfpathcurveto{\pgfqpoint{1.259151in}{2.168958in}}{\pgfqpoint{1.255879in}{2.176858in}}{\pgfqpoint{1.250055in}{2.182682in}}%
\pgfpathcurveto{\pgfqpoint{1.244231in}{2.188506in}}{\pgfqpoint{1.236331in}{2.191778in}}{\pgfqpoint{1.228095in}{2.191778in}}%
\pgfpathcurveto{\pgfqpoint{1.219858in}{2.191778in}}{\pgfqpoint{1.211958in}{2.188506in}}{\pgfqpoint{1.206134in}{2.182682in}}%
\pgfpathcurveto{\pgfqpoint{1.200310in}{2.176858in}}{\pgfqpoint{1.197038in}{2.168958in}}{\pgfqpoint{1.197038in}{2.160722in}}%
\pgfpathcurveto{\pgfqpoint{1.197038in}{2.152486in}}{\pgfqpoint{1.200310in}{2.144586in}}{\pgfqpoint{1.206134in}{2.138762in}}%
\pgfpathcurveto{\pgfqpoint{1.211958in}{2.132938in}}{\pgfqpoint{1.219858in}{2.129665in}}{\pgfqpoint{1.228095in}{2.129665in}}%
\pgfpathclose%
\pgfusepath{stroke,fill}%
\end{pgfscope}%
\begin{pgfscope}%
\pgfpathrectangle{\pgfqpoint{0.100000in}{0.220728in}}{\pgfqpoint{3.696000in}{3.696000in}}%
\pgfusepath{clip}%
\pgfsetbuttcap%
\pgfsetroundjoin%
\definecolor{currentfill}{rgb}{0.121569,0.466667,0.705882}%
\pgfsetfillcolor{currentfill}%
\pgfsetfillopacity{0.884243}%
\pgfsetlinewidth{1.003750pt}%
\definecolor{currentstroke}{rgb}{0.121569,0.466667,0.705882}%
\pgfsetstrokecolor{currentstroke}%
\pgfsetstrokeopacity{0.884243}%
\pgfsetdash{}{0pt}%
\pgfpathmoveto{\pgfqpoint{2.324998in}{1.468863in}}%
\pgfpathcurveto{\pgfqpoint{2.333234in}{1.468863in}}{\pgfqpoint{2.341135in}{1.472135in}}{\pgfqpoint{2.346958in}{1.477959in}}%
\pgfpathcurveto{\pgfqpoint{2.352782in}{1.483783in}}{\pgfqpoint{2.356055in}{1.491683in}}{\pgfqpoint{2.356055in}{1.499919in}}%
\pgfpathcurveto{\pgfqpoint{2.356055in}{1.508155in}}{\pgfqpoint{2.352782in}{1.516056in}}{\pgfqpoint{2.346958in}{1.521879in}}%
\pgfpathcurveto{\pgfqpoint{2.341135in}{1.527703in}}{\pgfqpoint{2.333234in}{1.530976in}}{\pgfqpoint{2.324998in}{1.530976in}}%
\pgfpathcurveto{\pgfqpoint{2.316762in}{1.530976in}}{\pgfqpoint{2.308862in}{1.527703in}}{\pgfqpoint{2.303038in}{1.521879in}}%
\pgfpathcurveto{\pgfqpoint{2.297214in}{1.516056in}}{\pgfqpoint{2.293942in}{1.508155in}}{\pgfqpoint{2.293942in}{1.499919in}}%
\pgfpathcurveto{\pgfqpoint{2.293942in}{1.491683in}}{\pgfqpoint{2.297214in}{1.483783in}}{\pgfqpoint{2.303038in}{1.477959in}}%
\pgfpathcurveto{\pgfqpoint{2.308862in}{1.472135in}}{\pgfqpoint{2.316762in}{1.468863in}}{\pgfqpoint{2.324998in}{1.468863in}}%
\pgfpathclose%
\pgfusepath{stroke,fill}%
\end{pgfscope}%
\begin{pgfscope}%
\pgfpathrectangle{\pgfqpoint{0.100000in}{0.220728in}}{\pgfqpoint{3.696000in}{3.696000in}}%
\pgfusepath{clip}%
\pgfsetbuttcap%
\pgfsetroundjoin%
\definecolor{currentfill}{rgb}{0.121569,0.466667,0.705882}%
\pgfsetfillcolor{currentfill}%
\pgfsetfillopacity{0.885236}%
\pgfsetlinewidth{1.003750pt}%
\definecolor{currentstroke}{rgb}{0.121569,0.466667,0.705882}%
\pgfsetstrokecolor{currentstroke}%
\pgfsetstrokeopacity{0.885236}%
\pgfsetdash{}{0pt}%
\pgfpathmoveto{\pgfqpoint{2.325615in}{1.467980in}}%
\pgfpathcurveto{\pgfqpoint{2.333851in}{1.467980in}}{\pgfqpoint{2.341751in}{1.471252in}}{\pgfqpoint{2.347575in}{1.477076in}}%
\pgfpathcurveto{\pgfqpoint{2.353399in}{1.482900in}}{\pgfqpoint{2.356671in}{1.490800in}}{\pgfqpoint{2.356671in}{1.499037in}}%
\pgfpathcurveto{\pgfqpoint{2.356671in}{1.507273in}}{\pgfqpoint{2.353399in}{1.515173in}}{\pgfqpoint{2.347575in}{1.520997in}}%
\pgfpathcurveto{\pgfqpoint{2.341751in}{1.526821in}}{\pgfqpoint{2.333851in}{1.530093in}}{\pgfqpoint{2.325615in}{1.530093in}}%
\pgfpathcurveto{\pgfqpoint{2.317378in}{1.530093in}}{\pgfqpoint{2.309478in}{1.526821in}}{\pgfqpoint{2.303654in}{1.520997in}}%
\pgfpathcurveto{\pgfqpoint{2.297830in}{1.515173in}}{\pgfqpoint{2.294558in}{1.507273in}}{\pgfqpoint{2.294558in}{1.499037in}}%
\pgfpathcurveto{\pgfqpoint{2.294558in}{1.490800in}}{\pgfqpoint{2.297830in}{1.482900in}}{\pgfqpoint{2.303654in}{1.477076in}}%
\pgfpathcurveto{\pgfqpoint{2.309478in}{1.471252in}}{\pgfqpoint{2.317378in}{1.467980in}}{\pgfqpoint{2.325615in}{1.467980in}}%
\pgfpathclose%
\pgfusepath{stroke,fill}%
\end{pgfscope}%
\begin{pgfscope}%
\pgfpathrectangle{\pgfqpoint{0.100000in}{0.220728in}}{\pgfqpoint{3.696000in}{3.696000in}}%
\pgfusepath{clip}%
\pgfsetbuttcap%
\pgfsetroundjoin%
\definecolor{currentfill}{rgb}{0.121569,0.466667,0.705882}%
\pgfsetfillcolor{currentfill}%
\pgfsetfillopacity{0.886274}%
\pgfsetlinewidth{1.003750pt}%
\definecolor{currentstroke}{rgb}{0.121569,0.466667,0.705882}%
\pgfsetstrokecolor{currentstroke}%
\pgfsetstrokeopacity{0.886274}%
\pgfsetdash{}{0pt}%
\pgfpathmoveto{\pgfqpoint{1.280557in}{2.090827in}}%
\pgfpathcurveto{\pgfqpoint{1.288793in}{2.090827in}}{\pgfqpoint{1.296693in}{2.094099in}}{\pgfqpoint{1.302517in}{2.099923in}}%
\pgfpathcurveto{\pgfqpoint{1.308341in}{2.105747in}}{\pgfqpoint{1.311613in}{2.113647in}}{\pgfqpoint{1.311613in}{2.121884in}}%
\pgfpathcurveto{\pgfqpoint{1.311613in}{2.130120in}}{\pgfqpoint{1.308341in}{2.138020in}}{\pgfqpoint{1.302517in}{2.143844in}}%
\pgfpathcurveto{\pgfqpoint{1.296693in}{2.149668in}}{\pgfqpoint{1.288793in}{2.152940in}}{\pgfqpoint{1.280557in}{2.152940in}}%
\pgfpathcurveto{\pgfqpoint{1.272321in}{2.152940in}}{\pgfqpoint{1.264421in}{2.149668in}}{\pgfqpoint{1.258597in}{2.143844in}}%
\pgfpathcurveto{\pgfqpoint{1.252773in}{2.138020in}}{\pgfqpoint{1.249500in}{2.130120in}}{\pgfqpoint{1.249500in}{2.121884in}}%
\pgfpathcurveto{\pgfqpoint{1.249500in}{2.113647in}}{\pgfqpoint{1.252773in}{2.105747in}}{\pgfqpoint{1.258597in}{2.099923in}}%
\pgfpathcurveto{\pgfqpoint{1.264421in}{2.094099in}}{\pgfqpoint{1.272321in}{2.090827in}}{\pgfqpoint{1.280557in}{2.090827in}}%
\pgfpathclose%
\pgfusepath{stroke,fill}%
\end{pgfscope}%
\begin{pgfscope}%
\pgfpathrectangle{\pgfqpoint{0.100000in}{0.220728in}}{\pgfqpoint{3.696000in}{3.696000in}}%
\pgfusepath{clip}%
\pgfsetbuttcap%
\pgfsetroundjoin%
\definecolor{currentfill}{rgb}{0.121569,0.466667,0.705882}%
\pgfsetfillcolor{currentfill}%
\pgfsetfillopacity{0.886770}%
\pgfsetlinewidth{1.003750pt}%
\definecolor{currentstroke}{rgb}{0.121569,0.466667,0.705882}%
\pgfsetstrokecolor{currentstroke}%
\pgfsetstrokeopacity{0.886770}%
\pgfsetdash{}{0pt}%
\pgfpathmoveto{\pgfqpoint{2.326783in}{1.466882in}}%
\pgfpathcurveto{\pgfqpoint{2.335019in}{1.466882in}}{\pgfqpoint{2.342919in}{1.470154in}}{\pgfqpoint{2.348743in}{1.475978in}}%
\pgfpathcurveto{\pgfqpoint{2.354567in}{1.481802in}}{\pgfqpoint{2.357839in}{1.489702in}}{\pgfqpoint{2.357839in}{1.497939in}}%
\pgfpathcurveto{\pgfqpoint{2.357839in}{1.506175in}}{\pgfqpoint{2.354567in}{1.514075in}}{\pgfqpoint{2.348743in}{1.519899in}}%
\pgfpathcurveto{\pgfqpoint{2.342919in}{1.525723in}}{\pgfqpoint{2.335019in}{1.528995in}}{\pgfqpoint{2.326783in}{1.528995in}}%
\pgfpathcurveto{\pgfqpoint{2.318546in}{1.528995in}}{\pgfqpoint{2.310646in}{1.525723in}}{\pgfqpoint{2.304822in}{1.519899in}}%
\pgfpathcurveto{\pgfqpoint{2.298998in}{1.514075in}}{\pgfqpoint{2.295726in}{1.506175in}}{\pgfqpoint{2.295726in}{1.497939in}}%
\pgfpathcurveto{\pgfqpoint{2.295726in}{1.489702in}}{\pgfqpoint{2.298998in}{1.481802in}}{\pgfqpoint{2.304822in}{1.475978in}}%
\pgfpathcurveto{\pgfqpoint{2.310646in}{1.470154in}}{\pgfqpoint{2.318546in}{1.466882in}}{\pgfqpoint{2.326783in}{1.466882in}}%
\pgfpathclose%
\pgfusepath{stroke,fill}%
\end{pgfscope}%
\begin{pgfscope}%
\pgfpathrectangle{\pgfqpoint{0.100000in}{0.220728in}}{\pgfqpoint{3.696000in}{3.696000in}}%
\pgfusepath{clip}%
\pgfsetbuttcap%
\pgfsetroundjoin%
\definecolor{currentfill}{rgb}{0.121569,0.466667,0.705882}%
\pgfsetfillcolor{currentfill}%
\pgfsetfillopacity{0.888638}%
\pgfsetlinewidth{1.003750pt}%
\definecolor{currentstroke}{rgb}{0.121569,0.466667,0.705882}%
\pgfsetstrokecolor{currentstroke}%
\pgfsetstrokeopacity{0.888638}%
\pgfsetdash{}{0pt}%
\pgfpathmoveto{\pgfqpoint{2.327377in}{1.465725in}}%
\pgfpathcurveto{\pgfqpoint{2.335613in}{1.465725in}}{\pgfqpoint{2.343513in}{1.468998in}}{\pgfqpoint{2.349337in}{1.474822in}}%
\pgfpathcurveto{\pgfqpoint{2.355161in}{1.480646in}}{\pgfqpoint{2.358433in}{1.488546in}}{\pgfqpoint{2.358433in}{1.496782in}}%
\pgfpathcurveto{\pgfqpoint{2.358433in}{1.505018in}}{\pgfqpoint{2.355161in}{1.512918in}}{\pgfqpoint{2.349337in}{1.518742in}}%
\pgfpathcurveto{\pgfqpoint{2.343513in}{1.524566in}}{\pgfqpoint{2.335613in}{1.527838in}}{\pgfqpoint{2.327377in}{1.527838in}}%
\pgfpathcurveto{\pgfqpoint{2.319141in}{1.527838in}}{\pgfqpoint{2.311241in}{1.524566in}}{\pgfqpoint{2.305417in}{1.518742in}}%
\pgfpathcurveto{\pgfqpoint{2.299593in}{1.512918in}}{\pgfqpoint{2.296320in}{1.505018in}}{\pgfqpoint{2.296320in}{1.496782in}}%
\pgfpathcurveto{\pgfqpoint{2.296320in}{1.488546in}}{\pgfqpoint{2.299593in}{1.480646in}}{\pgfqpoint{2.305417in}{1.474822in}}%
\pgfpathcurveto{\pgfqpoint{2.311241in}{1.468998in}}{\pgfqpoint{2.319141in}{1.465725in}}{\pgfqpoint{2.327377in}{1.465725in}}%
\pgfpathclose%
\pgfusepath{stroke,fill}%
\end{pgfscope}%
\begin{pgfscope}%
\pgfpathrectangle{\pgfqpoint{0.100000in}{0.220728in}}{\pgfqpoint{3.696000in}{3.696000in}}%
\pgfusepath{clip}%
\pgfsetbuttcap%
\pgfsetroundjoin%
\definecolor{currentfill}{rgb}{0.121569,0.466667,0.705882}%
\pgfsetfillcolor{currentfill}%
\pgfsetfillopacity{0.889566}%
\pgfsetlinewidth{1.003750pt}%
\definecolor{currentstroke}{rgb}{0.121569,0.466667,0.705882}%
\pgfsetstrokecolor{currentstroke}%
\pgfsetstrokeopacity{0.889566}%
\pgfsetdash{}{0pt}%
\pgfpathmoveto{\pgfqpoint{2.328035in}{1.464624in}}%
\pgfpathcurveto{\pgfqpoint{2.336272in}{1.464624in}}{\pgfqpoint{2.344172in}{1.467897in}}{\pgfqpoint{2.349996in}{1.473720in}}%
\pgfpathcurveto{\pgfqpoint{2.355820in}{1.479544in}}{\pgfqpoint{2.359092in}{1.487444in}}{\pgfqpoint{2.359092in}{1.495681in}}%
\pgfpathcurveto{\pgfqpoint{2.359092in}{1.503917in}}{\pgfqpoint{2.355820in}{1.511817in}}{\pgfqpoint{2.349996in}{1.517641in}}%
\pgfpathcurveto{\pgfqpoint{2.344172in}{1.523465in}}{\pgfqpoint{2.336272in}{1.526737in}}{\pgfqpoint{2.328035in}{1.526737in}}%
\pgfpathcurveto{\pgfqpoint{2.319799in}{1.526737in}}{\pgfqpoint{2.311899in}{1.523465in}}{\pgfqpoint{2.306075in}{1.517641in}}%
\pgfpathcurveto{\pgfqpoint{2.300251in}{1.511817in}}{\pgfqpoint{2.296979in}{1.503917in}}{\pgfqpoint{2.296979in}{1.495681in}}%
\pgfpathcurveto{\pgfqpoint{2.296979in}{1.487444in}}{\pgfqpoint{2.300251in}{1.479544in}}{\pgfqpoint{2.306075in}{1.473720in}}%
\pgfpathcurveto{\pgfqpoint{2.311899in}{1.467897in}}{\pgfqpoint{2.319799in}{1.464624in}}{\pgfqpoint{2.328035in}{1.464624in}}%
\pgfpathclose%
\pgfusepath{stroke,fill}%
\end{pgfscope}%
\begin{pgfscope}%
\pgfpathrectangle{\pgfqpoint{0.100000in}{0.220728in}}{\pgfqpoint{3.696000in}{3.696000in}}%
\pgfusepath{clip}%
\pgfsetbuttcap%
\pgfsetroundjoin%
\definecolor{currentfill}{rgb}{0.121569,0.466667,0.705882}%
\pgfsetfillcolor{currentfill}%
\pgfsetfillopacity{0.890567}%
\pgfsetlinewidth{1.003750pt}%
\definecolor{currentstroke}{rgb}{0.121569,0.466667,0.705882}%
\pgfsetstrokecolor{currentstroke}%
\pgfsetstrokeopacity{0.890567}%
\pgfsetdash{}{0pt}%
\pgfpathmoveto{\pgfqpoint{2.329131in}{1.462478in}}%
\pgfpathcurveto{\pgfqpoint{2.337368in}{1.462478in}}{\pgfqpoint{2.345268in}{1.465750in}}{\pgfqpoint{2.351092in}{1.471574in}}%
\pgfpathcurveto{\pgfqpoint{2.356916in}{1.477398in}}{\pgfqpoint{2.360188in}{1.485298in}}{\pgfqpoint{2.360188in}{1.493534in}}%
\pgfpathcurveto{\pgfqpoint{2.360188in}{1.501771in}}{\pgfqpoint{2.356916in}{1.509671in}}{\pgfqpoint{2.351092in}{1.515495in}}%
\pgfpathcurveto{\pgfqpoint{2.345268in}{1.521319in}}{\pgfqpoint{2.337368in}{1.524591in}}{\pgfqpoint{2.329131in}{1.524591in}}%
\pgfpathcurveto{\pgfqpoint{2.320895in}{1.524591in}}{\pgfqpoint{2.312995in}{1.521319in}}{\pgfqpoint{2.307171in}{1.515495in}}%
\pgfpathcurveto{\pgfqpoint{2.301347in}{1.509671in}}{\pgfqpoint{2.298075in}{1.501771in}}{\pgfqpoint{2.298075in}{1.493534in}}%
\pgfpathcurveto{\pgfqpoint{2.298075in}{1.485298in}}{\pgfqpoint{2.301347in}{1.477398in}}{\pgfqpoint{2.307171in}{1.471574in}}%
\pgfpathcurveto{\pgfqpoint{2.312995in}{1.465750in}}{\pgfqpoint{2.320895in}{1.462478in}}{\pgfqpoint{2.329131in}{1.462478in}}%
\pgfpathclose%
\pgfusepath{stroke,fill}%
\end{pgfscope}%
\begin{pgfscope}%
\pgfpathrectangle{\pgfqpoint{0.100000in}{0.220728in}}{\pgfqpoint{3.696000in}{3.696000in}}%
\pgfusepath{clip}%
\pgfsetbuttcap%
\pgfsetroundjoin%
\definecolor{currentfill}{rgb}{0.121569,0.466667,0.705882}%
\pgfsetfillcolor{currentfill}%
\pgfsetfillopacity{0.891319}%
\pgfsetlinewidth{1.003750pt}%
\definecolor{currentstroke}{rgb}{0.121569,0.466667,0.705882}%
\pgfsetstrokecolor{currentstroke}%
\pgfsetstrokeopacity{0.891319}%
\pgfsetdash{}{0pt}%
\pgfpathmoveto{\pgfqpoint{2.329724in}{1.462527in}}%
\pgfpathcurveto{\pgfqpoint{2.337960in}{1.462527in}}{\pgfqpoint{2.345860in}{1.465799in}}{\pgfqpoint{2.351684in}{1.471623in}}%
\pgfpathcurveto{\pgfqpoint{2.357508in}{1.477447in}}{\pgfqpoint{2.360781in}{1.485347in}}{\pgfqpoint{2.360781in}{1.493583in}}%
\pgfpathcurveto{\pgfqpoint{2.360781in}{1.501819in}}{\pgfqpoint{2.357508in}{1.509719in}}{\pgfqpoint{2.351684in}{1.515543in}}%
\pgfpathcurveto{\pgfqpoint{2.345860in}{1.521367in}}{\pgfqpoint{2.337960in}{1.524640in}}{\pgfqpoint{2.329724in}{1.524640in}}%
\pgfpathcurveto{\pgfqpoint{2.321488in}{1.524640in}}{\pgfqpoint{2.313588in}{1.521367in}}{\pgfqpoint{2.307764in}{1.515543in}}%
\pgfpathcurveto{\pgfqpoint{2.301940in}{1.509719in}}{\pgfqpoint{2.298668in}{1.501819in}}{\pgfqpoint{2.298668in}{1.493583in}}%
\pgfpathcurveto{\pgfqpoint{2.298668in}{1.485347in}}{\pgfqpoint{2.301940in}{1.477447in}}{\pgfqpoint{2.307764in}{1.471623in}}%
\pgfpathcurveto{\pgfqpoint{2.313588in}{1.465799in}}{\pgfqpoint{2.321488in}{1.462527in}}{\pgfqpoint{2.329724in}{1.462527in}}%
\pgfpathclose%
\pgfusepath{stroke,fill}%
\end{pgfscope}%
\begin{pgfscope}%
\pgfpathrectangle{\pgfqpoint{0.100000in}{0.220728in}}{\pgfqpoint{3.696000in}{3.696000in}}%
\pgfusepath{clip}%
\pgfsetbuttcap%
\pgfsetroundjoin%
\definecolor{currentfill}{rgb}{0.121569,0.466667,0.705882}%
\pgfsetfillcolor{currentfill}%
\pgfsetfillopacity{0.891823}%
\pgfsetlinewidth{1.003750pt}%
\definecolor{currentstroke}{rgb}{0.121569,0.466667,0.705882}%
\pgfsetstrokecolor{currentstroke}%
\pgfsetstrokeopacity{0.891823}%
\pgfsetdash{}{0pt}%
\pgfpathmoveto{\pgfqpoint{1.325194in}{2.058890in}}%
\pgfpathcurveto{\pgfqpoint{1.333430in}{2.058890in}}{\pgfqpoint{1.341330in}{2.062163in}}{\pgfqpoint{1.347154in}{2.067987in}}%
\pgfpathcurveto{\pgfqpoint{1.352978in}{2.073810in}}{\pgfqpoint{1.356250in}{2.081710in}}{\pgfqpoint{1.356250in}{2.089947in}}%
\pgfpathcurveto{\pgfqpoint{1.356250in}{2.098183in}}{\pgfqpoint{1.352978in}{2.106083in}}{\pgfqpoint{1.347154in}{2.111907in}}%
\pgfpathcurveto{\pgfqpoint{1.341330in}{2.117731in}}{\pgfqpoint{1.333430in}{2.121003in}}{\pgfqpoint{1.325194in}{2.121003in}}%
\pgfpathcurveto{\pgfqpoint{1.316957in}{2.121003in}}{\pgfqpoint{1.309057in}{2.117731in}}{\pgfqpoint{1.303233in}{2.111907in}}%
\pgfpathcurveto{\pgfqpoint{1.297409in}{2.106083in}}{\pgfqpoint{1.294137in}{2.098183in}}{\pgfqpoint{1.294137in}{2.089947in}}%
\pgfpathcurveto{\pgfqpoint{1.294137in}{2.081710in}}{\pgfqpoint{1.297409in}{2.073810in}}{\pgfqpoint{1.303233in}{2.067987in}}%
\pgfpathcurveto{\pgfqpoint{1.309057in}{2.062163in}}{\pgfqpoint{1.316957in}{2.058890in}}{\pgfqpoint{1.325194in}{2.058890in}}%
\pgfpathclose%
\pgfusepath{stroke,fill}%
\end{pgfscope}%
\begin{pgfscope}%
\pgfpathrectangle{\pgfqpoint{0.100000in}{0.220728in}}{\pgfqpoint{3.696000in}{3.696000in}}%
\pgfusepath{clip}%
\pgfsetbuttcap%
\pgfsetroundjoin%
\definecolor{currentfill}{rgb}{0.121569,0.466667,0.705882}%
\pgfsetfillcolor{currentfill}%
\pgfsetfillopacity{0.892399}%
\pgfsetlinewidth{1.003750pt}%
\definecolor{currentstroke}{rgb}{0.121569,0.466667,0.705882}%
\pgfsetstrokecolor{currentstroke}%
\pgfsetstrokeopacity{0.892399}%
\pgfsetdash{}{0pt}%
\pgfpathmoveto{\pgfqpoint{2.330262in}{1.462333in}}%
\pgfpathcurveto{\pgfqpoint{2.338499in}{1.462333in}}{\pgfqpoint{2.346399in}{1.465605in}}{\pgfqpoint{2.352222in}{1.471429in}}%
\pgfpathcurveto{\pgfqpoint{2.358046in}{1.477253in}}{\pgfqpoint{2.361319in}{1.485153in}}{\pgfqpoint{2.361319in}{1.493389in}}%
\pgfpathcurveto{\pgfqpoint{2.361319in}{1.501626in}}{\pgfqpoint{2.358046in}{1.509526in}}{\pgfqpoint{2.352222in}{1.515350in}}%
\pgfpathcurveto{\pgfqpoint{2.346399in}{1.521174in}}{\pgfqpoint{2.338499in}{1.524446in}}{\pgfqpoint{2.330262in}{1.524446in}}%
\pgfpathcurveto{\pgfqpoint{2.322026in}{1.524446in}}{\pgfqpoint{2.314126in}{1.521174in}}{\pgfqpoint{2.308302in}{1.515350in}}%
\pgfpathcurveto{\pgfqpoint{2.302478in}{1.509526in}}{\pgfqpoint{2.299206in}{1.501626in}}{\pgfqpoint{2.299206in}{1.493389in}}%
\pgfpathcurveto{\pgfqpoint{2.299206in}{1.485153in}}{\pgfqpoint{2.302478in}{1.477253in}}{\pgfqpoint{2.308302in}{1.471429in}}%
\pgfpathcurveto{\pgfqpoint{2.314126in}{1.465605in}}{\pgfqpoint{2.322026in}{1.462333in}}{\pgfqpoint{2.330262in}{1.462333in}}%
\pgfpathclose%
\pgfusepath{stroke,fill}%
\end{pgfscope}%
\begin{pgfscope}%
\pgfpathrectangle{\pgfqpoint{0.100000in}{0.220728in}}{\pgfqpoint{3.696000in}{3.696000in}}%
\pgfusepath{clip}%
\pgfsetbuttcap%
\pgfsetroundjoin%
\definecolor{currentfill}{rgb}{0.121569,0.466667,0.705882}%
\pgfsetfillcolor{currentfill}%
\pgfsetfillopacity{0.892824}%
\pgfsetlinewidth{1.003750pt}%
\definecolor{currentstroke}{rgb}{0.121569,0.466667,0.705882}%
\pgfsetstrokecolor{currentstroke}%
\pgfsetstrokeopacity{0.892824}%
\pgfsetdash{}{0pt}%
\pgfpathmoveto{\pgfqpoint{2.330660in}{1.461255in}}%
\pgfpathcurveto{\pgfqpoint{2.338897in}{1.461255in}}{\pgfqpoint{2.346797in}{1.464527in}}{\pgfqpoint{2.352621in}{1.470351in}}%
\pgfpathcurveto{\pgfqpoint{2.358445in}{1.476175in}}{\pgfqpoint{2.361717in}{1.484075in}}{\pgfqpoint{2.361717in}{1.492311in}}%
\pgfpathcurveto{\pgfqpoint{2.361717in}{1.500547in}}{\pgfqpoint{2.358445in}{1.508448in}}{\pgfqpoint{2.352621in}{1.514271in}}%
\pgfpathcurveto{\pgfqpoint{2.346797in}{1.520095in}}{\pgfqpoint{2.338897in}{1.523368in}}{\pgfqpoint{2.330660in}{1.523368in}}%
\pgfpathcurveto{\pgfqpoint{2.322424in}{1.523368in}}{\pgfqpoint{2.314524in}{1.520095in}}{\pgfqpoint{2.308700in}{1.514271in}}%
\pgfpathcurveto{\pgfqpoint{2.302876in}{1.508448in}}{\pgfqpoint{2.299604in}{1.500547in}}{\pgfqpoint{2.299604in}{1.492311in}}%
\pgfpathcurveto{\pgfqpoint{2.299604in}{1.484075in}}{\pgfqpoint{2.302876in}{1.476175in}}{\pgfqpoint{2.308700in}{1.470351in}}%
\pgfpathcurveto{\pgfqpoint{2.314524in}{1.464527in}}{\pgfqpoint{2.322424in}{1.461255in}}{\pgfqpoint{2.330660in}{1.461255in}}%
\pgfpathclose%
\pgfusepath{stroke,fill}%
\end{pgfscope}%
\begin{pgfscope}%
\pgfpathrectangle{\pgfqpoint{0.100000in}{0.220728in}}{\pgfqpoint{3.696000in}{3.696000in}}%
\pgfusepath{clip}%
\pgfsetbuttcap%
\pgfsetroundjoin%
\definecolor{currentfill}{rgb}{0.121569,0.466667,0.705882}%
\pgfsetfillcolor{currentfill}%
\pgfsetfillopacity{0.893286}%
\pgfsetlinewidth{1.003750pt}%
\definecolor{currentstroke}{rgb}{0.121569,0.466667,0.705882}%
\pgfsetstrokecolor{currentstroke}%
\pgfsetstrokeopacity{0.893286}%
\pgfsetdash{}{0pt}%
\pgfpathmoveto{\pgfqpoint{2.331350in}{1.459273in}}%
\pgfpathcurveto{\pgfqpoint{2.339586in}{1.459273in}}{\pgfqpoint{2.347486in}{1.462545in}}{\pgfqpoint{2.353310in}{1.468369in}}%
\pgfpathcurveto{\pgfqpoint{2.359134in}{1.474193in}}{\pgfqpoint{2.362406in}{1.482093in}}{\pgfqpoint{2.362406in}{1.490330in}}%
\pgfpathcurveto{\pgfqpoint{2.362406in}{1.498566in}}{\pgfqpoint{2.359134in}{1.506466in}}{\pgfqpoint{2.353310in}{1.512290in}}%
\pgfpathcurveto{\pgfqpoint{2.347486in}{1.518114in}}{\pgfqpoint{2.339586in}{1.521386in}}{\pgfqpoint{2.331350in}{1.521386in}}%
\pgfpathcurveto{\pgfqpoint{2.323114in}{1.521386in}}{\pgfqpoint{2.315214in}{1.518114in}}{\pgfqpoint{2.309390in}{1.512290in}}%
\pgfpathcurveto{\pgfqpoint{2.303566in}{1.506466in}}{\pgfqpoint{2.300293in}{1.498566in}}{\pgfqpoint{2.300293in}{1.490330in}}%
\pgfpathcurveto{\pgfqpoint{2.300293in}{1.482093in}}{\pgfqpoint{2.303566in}{1.474193in}}{\pgfqpoint{2.309390in}{1.468369in}}%
\pgfpathcurveto{\pgfqpoint{2.315214in}{1.462545in}}{\pgfqpoint{2.323114in}{1.459273in}}{\pgfqpoint{2.331350in}{1.459273in}}%
\pgfpathclose%
\pgfusepath{stroke,fill}%
\end{pgfscope}%
\begin{pgfscope}%
\pgfpathrectangle{\pgfqpoint{0.100000in}{0.220728in}}{\pgfqpoint{3.696000in}{3.696000in}}%
\pgfusepath{clip}%
\pgfsetbuttcap%
\pgfsetroundjoin%
\definecolor{currentfill}{rgb}{0.121569,0.466667,0.705882}%
\pgfsetfillcolor{currentfill}%
\pgfsetfillopacity{0.894560}%
\pgfsetlinewidth{1.003750pt}%
\definecolor{currentstroke}{rgb}{0.121569,0.466667,0.705882}%
\pgfsetstrokecolor{currentstroke}%
\pgfsetstrokeopacity{0.894560}%
\pgfsetdash{}{0pt}%
\pgfpathmoveto{\pgfqpoint{2.332179in}{1.458754in}}%
\pgfpathcurveto{\pgfqpoint{2.340415in}{1.458754in}}{\pgfqpoint{2.348315in}{1.462026in}}{\pgfqpoint{2.354139in}{1.467850in}}%
\pgfpathcurveto{\pgfqpoint{2.359963in}{1.473674in}}{\pgfqpoint{2.363235in}{1.481574in}}{\pgfqpoint{2.363235in}{1.489810in}}%
\pgfpathcurveto{\pgfqpoint{2.363235in}{1.498046in}}{\pgfqpoint{2.359963in}{1.505946in}}{\pgfqpoint{2.354139in}{1.511770in}}%
\pgfpathcurveto{\pgfqpoint{2.348315in}{1.517594in}}{\pgfqpoint{2.340415in}{1.520867in}}{\pgfqpoint{2.332179in}{1.520867in}}%
\pgfpathcurveto{\pgfqpoint{2.323942in}{1.520867in}}{\pgfqpoint{2.316042in}{1.517594in}}{\pgfqpoint{2.310218in}{1.511770in}}%
\pgfpathcurveto{\pgfqpoint{2.304394in}{1.505946in}}{\pgfqpoint{2.301122in}{1.498046in}}{\pgfqpoint{2.301122in}{1.489810in}}%
\pgfpathcurveto{\pgfqpoint{2.301122in}{1.481574in}}{\pgfqpoint{2.304394in}{1.473674in}}{\pgfqpoint{2.310218in}{1.467850in}}%
\pgfpathcurveto{\pgfqpoint{2.316042in}{1.462026in}}{\pgfqpoint{2.323942in}{1.458754in}}{\pgfqpoint{2.332179in}{1.458754in}}%
\pgfpathclose%
\pgfusepath{stroke,fill}%
\end{pgfscope}%
\begin{pgfscope}%
\pgfpathrectangle{\pgfqpoint{0.100000in}{0.220728in}}{\pgfqpoint{3.696000in}{3.696000in}}%
\pgfusepath{clip}%
\pgfsetbuttcap%
\pgfsetroundjoin%
\definecolor{currentfill}{rgb}{0.121569,0.466667,0.705882}%
\pgfsetfillcolor{currentfill}%
\pgfsetfillopacity{0.895705}%
\pgfsetlinewidth{1.003750pt}%
\definecolor{currentstroke}{rgb}{0.121569,0.466667,0.705882}%
\pgfsetstrokecolor{currentstroke}%
\pgfsetstrokeopacity{0.895705}%
\pgfsetdash{}{0pt}%
\pgfpathmoveto{\pgfqpoint{1.370390in}{2.024475in}}%
\pgfpathcurveto{\pgfqpoint{1.378626in}{2.024475in}}{\pgfqpoint{1.386526in}{2.027748in}}{\pgfqpoint{1.392350in}{2.033572in}}%
\pgfpathcurveto{\pgfqpoint{1.398174in}{2.039396in}}{\pgfqpoint{1.401446in}{2.047296in}}{\pgfqpoint{1.401446in}{2.055532in}}%
\pgfpathcurveto{\pgfqpoint{1.401446in}{2.063768in}}{\pgfqpoint{1.398174in}{2.071668in}}{\pgfqpoint{1.392350in}{2.077492in}}%
\pgfpathcurveto{\pgfqpoint{1.386526in}{2.083316in}}{\pgfqpoint{1.378626in}{2.086588in}}{\pgfqpoint{1.370390in}{2.086588in}}%
\pgfpathcurveto{\pgfqpoint{1.362154in}{2.086588in}}{\pgfqpoint{1.354254in}{2.083316in}}{\pgfqpoint{1.348430in}{2.077492in}}%
\pgfpathcurveto{\pgfqpoint{1.342606in}{2.071668in}}{\pgfqpoint{1.339333in}{2.063768in}}{\pgfqpoint{1.339333in}{2.055532in}}%
\pgfpathcurveto{\pgfqpoint{1.339333in}{2.047296in}}{\pgfqpoint{1.342606in}{2.039396in}}{\pgfqpoint{1.348430in}{2.033572in}}%
\pgfpathcurveto{\pgfqpoint{1.354254in}{2.027748in}}{\pgfqpoint{1.362154in}{2.024475in}}{\pgfqpoint{1.370390in}{2.024475in}}%
\pgfpathclose%
\pgfusepath{stroke,fill}%
\end{pgfscope}%
\begin{pgfscope}%
\pgfpathrectangle{\pgfqpoint{0.100000in}{0.220728in}}{\pgfqpoint{3.696000in}{3.696000in}}%
\pgfusepath{clip}%
\pgfsetbuttcap%
\pgfsetroundjoin%
\definecolor{currentfill}{rgb}{0.121569,0.466667,0.705882}%
\pgfsetfillcolor{currentfill}%
\pgfsetfillopacity{0.896648}%
\pgfsetlinewidth{1.003750pt}%
\definecolor{currentstroke}{rgb}{0.121569,0.466667,0.705882}%
\pgfsetstrokecolor{currentstroke}%
\pgfsetstrokeopacity{0.896648}%
\pgfsetdash{}{0pt}%
\pgfpathmoveto{\pgfqpoint{2.334592in}{1.458087in}}%
\pgfpathcurveto{\pgfqpoint{2.342829in}{1.458087in}}{\pgfqpoint{2.350729in}{1.461359in}}{\pgfqpoint{2.356553in}{1.467183in}}%
\pgfpathcurveto{\pgfqpoint{2.362376in}{1.473007in}}{\pgfqpoint{2.365649in}{1.480907in}}{\pgfqpoint{2.365649in}{1.489143in}}%
\pgfpathcurveto{\pgfqpoint{2.365649in}{1.497379in}}{\pgfqpoint{2.362376in}{1.505279in}}{\pgfqpoint{2.356553in}{1.511103in}}%
\pgfpathcurveto{\pgfqpoint{2.350729in}{1.516927in}}{\pgfqpoint{2.342829in}{1.520200in}}{\pgfqpoint{2.334592in}{1.520200in}}%
\pgfpathcurveto{\pgfqpoint{2.326356in}{1.520200in}}{\pgfqpoint{2.318456in}{1.516927in}}{\pgfqpoint{2.312632in}{1.511103in}}%
\pgfpathcurveto{\pgfqpoint{2.306808in}{1.505279in}}{\pgfqpoint{2.303536in}{1.497379in}}{\pgfqpoint{2.303536in}{1.489143in}}%
\pgfpathcurveto{\pgfqpoint{2.303536in}{1.480907in}}{\pgfqpoint{2.306808in}{1.473007in}}{\pgfqpoint{2.312632in}{1.467183in}}%
\pgfpathcurveto{\pgfqpoint{2.318456in}{1.461359in}}{\pgfqpoint{2.326356in}{1.458087in}}{\pgfqpoint{2.334592in}{1.458087in}}%
\pgfpathclose%
\pgfusepath{stroke,fill}%
\end{pgfscope}%
\begin{pgfscope}%
\pgfpathrectangle{\pgfqpoint{0.100000in}{0.220728in}}{\pgfqpoint{3.696000in}{3.696000in}}%
\pgfusepath{clip}%
\pgfsetbuttcap%
\pgfsetroundjoin%
\definecolor{currentfill}{rgb}{0.121569,0.466667,0.705882}%
\pgfsetfillcolor{currentfill}%
\pgfsetfillopacity{0.898659}%
\pgfsetlinewidth{1.003750pt}%
\definecolor{currentstroke}{rgb}{0.121569,0.466667,0.705882}%
\pgfsetstrokecolor{currentstroke}%
\pgfsetstrokeopacity{0.898659}%
\pgfsetdash{}{0pt}%
\pgfpathmoveto{\pgfqpoint{2.335717in}{1.454943in}}%
\pgfpathcurveto{\pgfqpoint{2.343954in}{1.454943in}}{\pgfqpoint{2.351854in}{1.458215in}}{\pgfqpoint{2.357678in}{1.464039in}}%
\pgfpathcurveto{\pgfqpoint{2.363502in}{1.469863in}}{\pgfqpoint{2.366774in}{1.477763in}}{\pgfqpoint{2.366774in}{1.486000in}}%
\pgfpathcurveto{\pgfqpoint{2.366774in}{1.494236in}}{\pgfqpoint{2.363502in}{1.502136in}}{\pgfqpoint{2.357678in}{1.507960in}}%
\pgfpathcurveto{\pgfqpoint{2.351854in}{1.513784in}}{\pgfqpoint{2.343954in}{1.517056in}}{\pgfqpoint{2.335717in}{1.517056in}}%
\pgfpathcurveto{\pgfqpoint{2.327481in}{1.517056in}}{\pgfqpoint{2.319581in}{1.513784in}}{\pgfqpoint{2.313757in}{1.507960in}}%
\pgfpathcurveto{\pgfqpoint{2.307933in}{1.502136in}}{\pgfqpoint{2.304661in}{1.494236in}}{\pgfqpoint{2.304661in}{1.486000in}}%
\pgfpathcurveto{\pgfqpoint{2.304661in}{1.477763in}}{\pgfqpoint{2.307933in}{1.469863in}}{\pgfqpoint{2.313757in}{1.464039in}}%
\pgfpathcurveto{\pgfqpoint{2.319581in}{1.458215in}}{\pgfqpoint{2.327481in}{1.454943in}}{\pgfqpoint{2.335717in}{1.454943in}}%
\pgfpathclose%
\pgfusepath{stroke,fill}%
\end{pgfscope}%
\begin{pgfscope}%
\pgfpathrectangle{\pgfqpoint{0.100000in}{0.220728in}}{\pgfqpoint{3.696000in}{3.696000in}}%
\pgfusepath{clip}%
\pgfsetbuttcap%
\pgfsetroundjoin%
\definecolor{currentfill}{rgb}{0.121569,0.466667,0.705882}%
\pgfsetfillcolor{currentfill}%
\pgfsetfillopacity{0.900178}%
\pgfsetlinewidth{1.003750pt}%
\definecolor{currentstroke}{rgb}{0.121569,0.466667,0.705882}%
\pgfsetstrokecolor{currentstroke}%
\pgfsetstrokeopacity{0.900178}%
\pgfsetdash{}{0pt}%
\pgfpathmoveto{\pgfqpoint{1.409481in}{1.990780in}}%
\pgfpathcurveto{\pgfqpoint{1.417717in}{1.990780in}}{\pgfqpoint{1.425617in}{1.994052in}}{\pgfqpoint{1.431441in}{1.999876in}}%
\pgfpathcurveto{\pgfqpoint{1.437265in}{2.005700in}}{\pgfqpoint{1.440537in}{2.013600in}}{\pgfqpoint{1.440537in}{2.021836in}}%
\pgfpathcurveto{\pgfqpoint{1.440537in}{2.030072in}}{\pgfqpoint{1.437265in}{2.037973in}}{\pgfqpoint{1.431441in}{2.043796in}}%
\pgfpathcurveto{\pgfqpoint{1.425617in}{2.049620in}}{\pgfqpoint{1.417717in}{2.052893in}}{\pgfqpoint{1.409481in}{2.052893in}}%
\pgfpathcurveto{\pgfqpoint{1.401245in}{2.052893in}}{\pgfqpoint{1.393345in}{2.049620in}}{\pgfqpoint{1.387521in}{2.043796in}}%
\pgfpathcurveto{\pgfqpoint{1.381697in}{2.037973in}}{\pgfqpoint{1.378424in}{2.030072in}}{\pgfqpoint{1.378424in}{2.021836in}}%
\pgfpathcurveto{\pgfqpoint{1.378424in}{2.013600in}}{\pgfqpoint{1.381697in}{2.005700in}}{\pgfqpoint{1.387521in}{1.999876in}}%
\pgfpathcurveto{\pgfqpoint{1.393345in}{1.994052in}}{\pgfqpoint{1.401245in}{1.990780in}}{\pgfqpoint{1.409481in}{1.990780in}}%
\pgfpathclose%
\pgfusepath{stroke,fill}%
\end{pgfscope}%
\begin{pgfscope}%
\pgfpathrectangle{\pgfqpoint{0.100000in}{0.220728in}}{\pgfqpoint{3.696000in}{3.696000in}}%
\pgfusepath{clip}%
\pgfsetbuttcap%
\pgfsetroundjoin%
\definecolor{currentfill}{rgb}{0.121569,0.466667,0.705882}%
\pgfsetfillcolor{currentfill}%
\pgfsetfillopacity{0.900764}%
\pgfsetlinewidth{1.003750pt}%
\definecolor{currentstroke}{rgb}{0.121569,0.466667,0.705882}%
\pgfsetstrokecolor{currentstroke}%
\pgfsetstrokeopacity{0.900764}%
\pgfsetdash{}{0pt}%
\pgfpathmoveto{\pgfqpoint{2.337674in}{1.450323in}}%
\pgfpathcurveto{\pgfqpoint{2.345911in}{1.450323in}}{\pgfqpoint{2.353811in}{1.453595in}}{\pgfqpoint{2.359635in}{1.459419in}}%
\pgfpathcurveto{\pgfqpoint{2.365458in}{1.465243in}}{\pgfqpoint{2.368731in}{1.473143in}}{\pgfqpoint{2.368731in}{1.481379in}}%
\pgfpathcurveto{\pgfqpoint{2.368731in}{1.489615in}}{\pgfqpoint{2.365458in}{1.497515in}}{\pgfqpoint{2.359635in}{1.503339in}}%
\pgfpathcurveto{\pgfqpoint{2.353811in}{1.509163in}}{\pgfqpoint{2.345911in}{1.512436in}}{\pgfqpoint{2.337674in}{1.512436in}}%
\pgfpathcurveto{\pgfqpoint{2.329438in}{1.512436in}}{\pgfqpoint{2.321538in}{1.509163in}}{\pgfqpoint{2.315714in}{1.503339in}}%
\pgfpathcurveto{\pgfqpoint{2.309890in}{1.497515in}}{\pgfqpoint{2.306618in}{1.489615in}}{\pgfqpoint{2.306618in}{1.481379in}}%
\pgfpathcurveto{\pgfqpoint{2.306618in}{1.473143in}}{\pgfqpoint{2.309890in}{1.465243in}}{\pgfqpoint{2.315714in}{1.459419in}}%
\pgfpathcurveto{\pgfqpoint{2.321538in}{1.453595in}}{\pgfqpoint{2.329438in}{1.450323in}}{\pgfqpoint{2.337674in}{1.450323in}}%
\pgfpathclose%
\pgfusepath{stroke,fill}%
\end{pgfscope}%
\begin{pgfscope}%
\pgfpathrectangle{\pgfqpoint{0.100000in}{0.220728in}}{\pgfqpoint{3.696000in}{3.696000in}}%
\pgfusepath{clip}%
\pgfsetbuttcap%
\pgfsetroundjoin%
\definecolor{currentfill}{rgb}{0.121569,0.466667,0.705882}%
\pgfsetfillcolor{currentfill}%
\pgfsetfillopacity{0.903326}%
\pgfsetlinewidth{1.003750pt}%
\definecolor{currentstroke}{rgb}{0.121569,0.466667,0.705882}%
\pgfsetstrokecolor{currentstroke}%
\pgfsetstrokeopacity{0.903326}%
\pgfsetdash{}{0pt}%
\pgfpathmoveto{\pgfqpoint{2.339564in}{1.446683in}}%
\pgfpathcurveto{\pgfqpoint{2.347801in}{1.446683in}}{\pgfqpoint{2.355701in}{1.449955in}}{\pgfqpoint{2.361525in}{1.455779in}}%
\pgfpathcurveto{\pgfqpoint{2.367349in}{1.461603in}}{\pgfqpoint{2.370621in}{1.469503in}}{\pgfqpoint{2.370621in}{1.477739in}}%
\pgfpathcurveto{\pgfqpoint{2.370621in}{1.485975in}}{\pgfqpoint{2.367349in}{1.493875in}}{\pgfqpoint{2.361525in}{1.499699in}}%
\pgfpathcurveto{\pgfqpoint{2.355701in}{1.505523in}}{\pgfqpoint{2.347801in}{1.508796in}}{\pgfqpoint{2.339564in}{1.508796in}}%
\pgfpathcurveto{\pgfqpoint{2.331328in}{1.508796in}}{\pgfqpoint{2.323428in}{1.505523in}}{\pgfqpoint{2.317604in}{1.499699in}}%
\pgfpathcurveto{\pgfqpoint{2.311780in}{1.493875in}}{\pgfqpoint{2.308508in}{1.485975in}}{\pgfqpoint{2.308508in}{1.477739in}}%
\pgfpathcurveto{\pgfqpoint{2.308508in}{1.469503in}}{\pgfqpoint{2.311780in}{1.461603in}}{\pgfqpoint{2.317604in}{1.455779in}}%
\pgfpathcurveto{\pgfqpoint{2.323428in}{1.449955in}}{\pgfqpoint{2.331328in}{1.446683in}}{\pgfqpoint{2.339564in}{1.446683in}}%
\pgfpathclose%
\pgfusepath{stroke,fill}%
\end{pgfscope}%
\begin{pgfscope}%
\pgfpathrectangle{\pgfqpoint{0.100000in}{0.220728in}}{\pgfqpoint{3.696000in}{3.696000in}}%
\pgfusepath{clip}%
\pgfsetbuttcap%
\pgfsetroundjoin%
\definecolor{currentfill}{rgb}{0.121569,0.466667,0.705882}%
\pgfsetfillcolor{currentfill}%
\pgfsetfillopacity{0.904927}%
\pgfsetlinewidth{1.003750pt}%
\definecolor{currentstroke}{rgb}{0.121569,0.466667,0.705882}%
\pgfsetstrokecolor{currentstroke}%
\pgfsetstrokeopacity{0.904927}%
\pgfsetdash{}{0pt}%
\pgfpathmoveto{\pgfqpoint{2.341170in}{1.446223in}}%
\pgfpathcurveto{\pgfqpoint{2.349407in}{1.446223in}}{\pgfqpoint{2.357307in}{1.449495in}}{\pgfqpoint{2.363131in}{1.455319in}}%
\pgfpathcurveto{\pgfqpoint{2.368955in}{1.461143in}}{\pgfqpoint{2.372227in}{1.469043in}}{\pgfqpoint{2.372227in}{1.477279in}}%
\pgfpathcurveto{\pgfqpoint{2.372227in}{1.485516in}}{\pgfqpoint{2.368955in}{1.493416in}}{\pgfqpoint{2.363131in}{1.499240in}}%
\pgfpathcurveto{\pgfqpoint{2.357307in}{1.505063in}}{\pgfqpoint{2.349407in}{1.508336in}}{\pgfqpoint{2.341170in}{1.508336in}}%
\pgfpathcurveto{\pgfqpoint{2.332934in}{1.508336in}}{\pgfqpoint{2.325034in}{1.505063in}}{\pgfqpoint{2.319210in}{1.499240in}}%
\pgfpathcurveto{\pgfqpoint{2.313386in}{1.493416in}}{\pgfqpoint{2.310114in}{1.485516in}}{\pgfqpoint{2.310114in}{1.477279in}}%
\pgfpathcurveto{\pgfqpoint{2.310114in}{1.469043in}}{\pgfqpoint{2.313386in}{1.461143in}}{\pgfqpoint{2.319210in}{1.455319in}}%
\pgfpathcurveto{\pgfqpoint{2.325034in}{1.449495in}}{\pgfqpoint{2.332934in}{1.446223in}}{\pgfqpoint{2.341170in}{1.446223in}}%
\pgfpathclose%
\pgfusepath{stroke,fill}%
\end{pgfscope}%
\begin{pgfscope}%
\pgfpathrectangle{\pgfqpoint{0.100000in}{0.220728in}}{\pgfqpoint{3.696000in}{3.696000in}}%
\pgfusepath{clip}%
\pgfsetbuttcap%
\pgfsetroundjoin%
\definecolor{currentfill}{rgb}{0.121569,0.466667,0.705882}%
\pgfsetfillcolor{currentfill}%
\pgfsetfillopacity{0.906076}%
\pgfsetlinewidth{1.003750pt}%
\definecolor{currentstroke}{rgb}{0.121569,0.466667,0.705882}%
\pgfsetstrokecolor{currentstroke}%
\pgfsetstrokeopacity{0.906076}%
\pgfsetdash{}{0pt}%
\pgfpathmoveto{\pgfqpoint{1.449382in}{1.970716in}}%
\pgfpathcurveto{\pgfqpoint{1.457618in}{1.970716in}}{\pgfqpoint{1.465518in}{1.973988in}}{\pgfqpoint{1.471342in}{1.979812in}}%
\pgfpathcurveto{\pgfqpoint{1.477166in}{1.985636in}}{\pgfqpoint{1.480439in}{1.993536in}}{\pgfqpoint{1.480439in}{2.001773in}}%
\pgfpathcurveto{\pgfqpoint{1.480439in}{2.010009in}}{\pgfqpoint{1.477166in}{2.017909in}}{\pgfqpoint{1.471342in}{2.023733in}}%
\pgfpathcurveto{\pgfqpoint{1.465518in}{2.029557in}}{\pgfqpoint{1.457618in}{2.032829in}}{\pgfqpoint{1.449382in}{2.032829in}}%
\pgfpathcurveto{\pgfqpoint{1.441146in}{2.032829in}}{\pgfqpoint{1.433246in}{2.029557in}}{\pgfqpoint{1.427422in}{2.023733in}}%
\pgfpathcurveto{\pgfqpoint{1.421598in}{2.017909in}}{\pgfqpoint{1.418326in}{2.010009in}}{\pgfqpoint{1.418326in}{2.001773in}}%
\pgfpathcurveto{\pgfqpoint{1.418326in}{1.993536in}}{\pgfqpoint{1.421598in}{1.985636in}}{\pgfqpoint{1.427422in}{1.979812in}}%
\pgfpathcurveto{\pgfqpoint{1.433246in}{1.973988in}}{\pgfqpoint{1.441146in}{1.970716in}}{\pgfqpoint{1.449382in}{1.970716in}}%
\pgfpathclose%
\pgfusepath{stroke,fill}%
\end{pgfscope}%
\begin{pgfscope}%
\pgfpathrectangle{\pgfqpoint{0.100000in}{0.220728in}}{\pgfqpoint{3.696000in}{3.696000in}}%
\pgfusepath{clip}%
\pgfsetbuttcap%
\pgfsetroundjoin%
\definecolor{currentfill}{rgb}{0.121569,0.466667,0.705882}%
\pgfsetfillcolor{currentfill}%
\pgfsetfillopacity{0.906609}%
\pgfsetlinewidth{1.003750pt}%
\definecolor{currentstroke}{rgb}{0.121569,0.466667,0.705882}%
\pgfsetstrokecolor{currentstroke}%
\pgfsetstrokeopacity{0.906609}%
\pgfsetdash{}{0pt}%
\pgfpathmoveto{\pgfqpoint{2.342353in}{1.444046in}}%
\pgfpathcurveto{\pgfqpoint{2.350589in}{1.444046in}}{\pgfqpoint{2.358489in}{1.447318in}}{\pgfqpoint{2.364313in}{1.453142in}}%
\pgfpathcurveto{\pgfqpoint{2.370137in}{1.458966in}}{\pgfqpoint{2.373409in}{1.466866in}}{\pgfqpoint{2.373409in}{1.475102in}}%
\pgfpathcurveto{\pgfqpoint{2.373409in}{1.483338in}}{\pgfqpoint{2.370137in}{1.491238in}}{\pgfqpoint{2.364313in}{1.497062in}}%
\pgfpathcurveto{\pgfqpoint{2.358489in}{1.502886in}}{\pgfqpoint{2.350589in}{1.506159in}}{\pgfqpoint{2.342353in}{1.506159in}}%
\pgfpathcurveto{\pgfqpoint{2.334117in}{1.506159in}}{\pgfqpoint{2.326217in}{1.502886in}}{\pgfqpoint{2.320393in}{1.497062in}}%
\pgfpathcurveto{\pgfqpoint{2.314569in}{1.491238in}}{\pgfqpoint{2.311296in}{1.483338in}}{\pgfqpoint{2.311296in}{1.475102in}}%
\pgfpathcurveto{\pgfqpoint{2.311296in}{1.466866in}}{\pgfqpoint{2.314569in}{1.458966in}}{\pgfqpoint{2.320393in}{1.453142in}}%
\pgfpathcurveto{\pgfqpoint{2.326217in}{1.447318in}}{\pgfqpoint{2.334117in}{1.444046in}}{\pgfqpoint{2.342353in}{1.444046in}}%
\pgfpathclose%
\pgfusepath{stroke,fill}%
\end{pgfscope}%
\begin{pgfscope}%
\pgfpathrectangle{\pgfqpoint{0.100000in}{0.220728in}}{\pgfqpoint{3.696000in}{3.696000in}}%
\pgfusepath{clip}%
\pgfsetbuttcap%
\pgfsetroundjoin%
\definecolor{currentfill}{rgb}{0.121569,0.466667,0.705882}%
\pgfsetfillcolor{currentfill}%
\pgfsetfillopacity{0.908232}%
\pgfsetlinewidth{1.003750pt}%
\definecolor{currentstroke}{rgb}{0.121569,0.466667,0.705882}%
\pgfsetstrokecolor{currentstroke}%
\pgfsetstrokeopacity{0.908232}%
\pgfsetdash{}{0pt}%
\pgfpathmoveto{\pgfqpoint{1.487633in}{1.937538in}}%
\pgfpathcurveto{\pgfqpoint{1.495869in}{1.937538in}}{\pgfqpoint{1.503769in}{1.940810in}}{\pgfqpoint{1.509593in}{1.946634in}}%
\pgfpathcurveto{\pgfqpoint{1.515417in}{1.952458in}}{\pgfqpoint{1.518689in}{1.960358in}}{\pgfqpoint{1.518689in}{1.968595in}}%
\pgfpathcurveto{\pgfqpoint{1.518689in}{1.976831in}}{\pgfqpoint{1.515417in}{1.984731in}}{\pgfqpoint{1.509593in}{1.990555in}}%
\pgfpathcurveto{\pgfqpoint{1.503769in}{1.996379in}}{\pgfqpoint{1.495869in}{1.999651in}}{\pgfqpoint{1.487633in}{1.999651in}}%
\pgfpathcurveto{\pgfqpoint{1.479396in}{1.999651in}}{\pgfqpoint{1.471496in}{1.996379in}}{\pgfqpoint{1.465672in}{1.990555in}}%
\pgfpathcurveto{\pgfqpoint{1.459848in}{1.984731in}}{\pgfqpoint{1.456576in}{1.976831in}}{\pgfqpoint{1.456576in}{1.968595in}}%
\pgfpathcurveto{\pgfqpoint{1.456576in}{1.960358in}}{\pgfqpoint{1.459848in}{1.952458in}}{\pgfqpoint{1.465672in}{1.946634in}}%
\pgfpathcurveto{\pgfqpoint{1.471496in}{1.940810in}}{\pgfqpoint{1.479396in}{1.937538in}}{\pgfqpoint{1.487633in}{1.937538in}}%
\pgfpathclose%
\pgfusepath{stroke,fill}%
\end{pgfscope}%
\begin{pgfscope}%
\pgfpathrectangle{\pgfqpoint{0.100000in}{0.220728in}}{\pgfqpoint{3.696000in}{3.696000in}}%
\pgfusepath{clip}%
\pgfsetbuttcap%
\pgfsetroundjoin%
\definecolor{currentfill}{rgb}{0.121569,0.466667,0.705882}%
\pgfsetfillcolor{currentfill}%
\pgfsetfillopacity{0.908379}%
\pgfsetlinewidth{1.003750pt}%
\definecolor{currentstroke}{rgb}{0.121569,0.466667,0.705882}%
\pgfsetstrokecolor{currentstroke}%
\pgfsetstrokeopacity{0.908379}%
\pgfsetdash{}{0pt}%
\pgfpathmoveto{\pgfqpoint{2.343943in}{1.440704in}}%
\pgfpathcurveto{\pgfqpoint{2.352179in}{1.440704in}}{\pgfqpoint{2.360079in}{1.443977in}}{\pgfqpoint{2.365903in}{1.449801in}}%
\pgfpathcurveto{\pgfqpoint{2.371727in}{1.455625in}}{\pgfqpoint{2.374999in}{1.463525in}}{\pgfqpoint{2.374999in}{1.471761in}}%
\pgfpathcurveto{\pgfqpoint{2.374999in}{1.479997in}}{\pgfqpoint{2.371727in}{1.487897in}}{\pgfqpoint{2.365903in}{1.493721in}}%
\pgfpathcurveto{\pgfqpoint{2.360079in}{1.499545in}}{\pgfqpoint{2.352179in}{1.502817in}}{\pgfqpoint{2.343943in}{1.502817in}}%
\pgfpathcurveto{\pgfqpoint{2.335707in}{1.502817in}}{\pgfqpoint{2.327807in}{1.499545in}}{\pgfqpoint{2.321983in}{1.493721in}}%
\pgfpathcurveto{\pgfqpoint{2.316159in}{1.487897in}}{\pgfqpoint{2.312886in}{1.479997in}}{\pgfqpoint{2.312886in}{1.471761in}}%
\pgfpathcurveto{\pgfqpoint{2.312886in}{1.463525in}}{\pgfqpoint{2.316159in}{1.455625in}}{\pgfqpoint{2.321983in}{1.449801in}}%
\pgfpathcurveto{\pgfqpoint{2.327807in}{1.443977in}}{\pgfqpoint{2.335707in}{1.440704in}}{\pgfqpoint{2.343943in}{1.440704in}}%
\pgfpathclose%
\pgfusepath{stroke,fill}%
\end{pgfscope}%
\begin{pgfscope}%
\pgfpathrectangle{\pgfqpoint{0.100000in}{0.220728in}}{\pgfqpoint{3.696000in}{3.696000in}}%
\pgfusepath{clip}%
\pgfsetbuttcap%
\pgfsetroundjoin%
\definecolor{currentfill}{rgb}{0.121569,0.466667,0.705882}%
\pgfsetfillcolor{currentfill}%
\pgfsetfillopacity{0.910015}%
\pgfsetlinewidth{1.003750pt}%
\definecolor{currentstroke}{rgb}{0.121569,0.466667,0.705882}%
\pgfsetstrokecolor{currentstroke}%
\pgfsetstrokeopacity{0.910015}%
\pgfsetdash{}{0pt}%
\pgfpathmoveto{\pgfqpoint{2.345791in}{1.434726in}}%
\pgfpathcurveto{\pgfqpoint{2.354027in}{1.434726in}}{\pgfqpoint{2.361927in}{1.437999in}}{\pgfqpoint{2.367751in}{1.443823in}}%
\pgfpathcurveto{\pgfqpoint{2.373575in}{1.449647in}}{\pgfqpoint{2.376847in}{1.457547in}}{\pgfqpoint{2.376847in}{1.465783in}}%
\pgfpathcurveto{\pgfqpoint{2.376847in}{1.474019in}}{\pgfqpoint{2.373575in}{1.481919in}}{\pgfqpoint{2.367751in}{1.487743in}}%
\pgfpathcurveto{\pgfqpoint{2.361927in}{1.493567in}}{\pgfqpoint{2.354027in}{1.496839in}}{\pgfqpoint{2.345791in}{1.496839in}}%
\pgfpathcurveto{\pgfqpoint{2.337555in}{1.496839in}}{\pgfqpoint{2.329654in}{1.493567in}}{\pgfqpoint{2.323831in}{1.487743in}}%
\pgfpathcurveto{\pgfqpoint{2.318007in}{1.481919in}}{\pgfqpoint{2.314734in}{1.474019in}}{\pgfqpoint{2.314734in}{1.465783in}}%
\pgfpathcurveto{\pgfqpoint{2.314734in}{1.457547in}}{\pgfqpoint{2.318007in}{1.449647in}}{\pgfqpoint{2.323831in}{1.443823in}}%
\pgfpathcurveto{\pgfqpoint{2.329654in}{1.437999in}}{\pgfqpoint{2.337555in}{1.434726in}}{\pgfqpoint{2.345791in}{1.434726in}}%
\pgfpathclose%
\pgfusepath{stroke,fill}%
\end{pgfscope}%
\begin{pgfscope}%
\pgfpathrectangle{\pgfqpoint{0.100000in}{0.220728in}}{\pgfqpoint{3.696000in}{3.696000in}}%
\pgfusepath{clip}%
\pgfsetbuttcap%
\pgfsetroundjoin%
\definecolor{currentfill}{rgb}{0.121569,0.466667,0.705882}%
\pgfsetfillcolor{currentfill}%
\pgfsetfillopacity{0.911842}%
\pgfsetlinewidth{1.003750pt}%
\definecolor{currentstroke}{rgb}{0.121569,0.466667,0.705882}%
\pgfsetstrokecolor{currentstroke}%
\pgfsetstrokeopacity{0.911842}%
\pgfsetdash{}{0pt}%
\pgfpathmoveto{\pgfqpoint{1.524892in}{1.914656in}}%
\pgfpathcurveto{\pgfqpoint{1.533128in}{1.914656in}}{\pgfqpoint{1.541028in}{1.917928in}}{\pgfqpoint{1.546852in}{1.923752in}}%
\pgfpathcurveto{\pgfqpoint{1.552676in}{1.929576in}}{\pgfqpoint{1.555948in}{1.937476in}}{\pgfqpoint{1.555948in}{1.945712in}}%
\pgfpathcurveto{\pgfqpoint{1.555948in}{1.953949in}}{\pgfqpoint{1.552676in}{1.961849in}}{\pgfqpoint{1.546852in}{1.967673in}}%
\pgfpathcurveto{\pgfqpoint{1.541028in}{1.973497in}}{\pgfqpoint{1.533128in}{1.976769in}}{\pgfqpoint{1.524892in}{1.976769in}}%
\pgfpathcurveto{\pgfqpoint{1.516655in}{1.976769in}}{\pgfqpoint{1.508755in}{1.973497in}}{\pgfqpoint{1.502931in}{1.967673in}}%
\pgfpathcurveto{\pgfqpoint{1.497107in}{1.961849in}}{\pgfqpoint{1.493835in}{1.953949in}}{\pgfqpoint{1.493835in}{1.945712in}}%
\pgfpathcurveto{\pgfqpoint{1.493835in}{1.937476in}}{\pgfqpoint{1.497107in}{1.929576in}}{\pgfqpoint{1.502931in}{1.923752in}}%
\pgfpathcurveto{\pgfqpoint{1.508755in}{1.917928in}}{\pgfqpoint{1.516655in}{1.914656in}}{\pgfqpoint{1.524892in}{1.914656in}}%
\pgfpathclose%
\pgfusepath{stroke,fill}%
\end{pgfscope}%
\begin{pgfscope}%
\pgfpathrectangle{\pgfqpoint{0.100000in}{0.220728in}}{\pgfqpoint{3.696000in}{3.696000in}}%
\pgfusepath{clip}%
\pgfsetbuttcap%
\pgfsetroundjoin%
\definecolor{currentfill}{rgb}{0.121569,0.466667,0.705882}%
\pgfsetfillcolor{currentfill}%
\pgfsetfillopacity{0.913077}%
\pgfsetlinewidth{1.003750pt}%
\definecolor{currentstroke}{rgb}{0.121569,0.466667,0.705882}%
\pgfsetstrokecolor{currentstroke}%
\pgfsetstrokeopacity{0.913077}%
\pgfsetdash{}{0pt}%
\pgfpathmoveto{\pgfqpoint{2.347844in}{1.432642in}}%
\pgfpathcurveto{\pgfqpoint{2.356080in}{1.432642in}}{\pgfqpoint{2.363980in}{1.435914in}}{\pgfqpoint{2.369804in}{1.441738in}}%
\pgfpathcurveto{\pgfqpoint{2.375628in}{1.447562in}}{\pgfqpoint{2.378901in}{1.455462in}}{\pgfqpoint{2.378901in}{1.463699in}}%
\pgfpathcurveto{\pgfqpoint{2.378901in}{1.471935in}}{\pgfqpoint{2.375628in}{1.479835in}}{\pgfqpoint{2.369804in}{1.485659in}}%
\pgfpathcurveto{\pgfqpoint{2.363980in}{1.491483in}}{\pgfqpoint{2.356080in}{1.494755in}}{\pgfqpoint{2.347844in}{1.494755in}}%
\pgfpathcurveto{\pgfqpoint{2.339608in}{1.494755in}}{\pgfqpoint{2.331708in}{1.491483in}}{\pgfqpoint{2.325884in}{1.485659in}}%
\pgfpathcurveto{\pgfqpoint{2.320060in}{1.479835in}}{\pgfqpoint{2.316788in}{1.471935in}}{\pgfqpoint{2.316788in}{1.463699in}}%
\pgfpathcurveto{\pgfqpoint{2.316788in}{1.455462in}}{\pgfqpoint{2.320060in}{1.447562in}}{\pgfqpoint{2.325884in}{1.441738in}}%
\pgfpathcurveto{\pgfqpoint{2.331708in}{1.435914in}}{\pgfqpoint{2.339608in}{1.432642in}}{\pgfqpoint{2.347844in}{1.432642in}}%
\pgfpathclose%
\pgfusepath{stroke,fill}%
\end{pgfscope}%
\begin{pgfscope}%
\pgfpathrectangle{\pgfqpoint{0.100000in}{0.220728in}}{\pgfqpoint{3.696000in}{3.696000in}}%
\pgfusepath{clip}%
\pgfsetbuttcap%
\pgfsetroundjoin%
\definecolor{currentfill}{rgb}{0.121569,0.466667,0.705882}%
\pgfsetfillcolor{currentfill}%
\pgfsetfillopacity{0.915397}%
\pgfsetlinewidth{1.003750pt}%
\definecolor{currentstroke}{rgb}{0.121569,0.466667,0.705882}%
\pgfsetstrokecolor{currentstroke}%
\pgfsetstrokeopacity{0.915397}%
\pgfsetdash{}{0pt}%
\pgfpathmoveto{\pgfqpoint{1.559772in}{1.893420in}}%
\pgfpathcurveto{\pgfqpoint{1.568009in}{1.893420in}}{\pgfqpoint{1.575909in}{1.896692in}}{\pgfqpoint{1.581733in}{1.902516in}}%
\pgfpathcurveto{\pgfqpoint{1.587557in}{1.908340in}}{\pgfqpoint{1.590829in}{1.916240in}}{\pgfqpoint{1.590829in}{1.924476in}}%
\pgfpathcurveto{\pgfqpoint{1.590829in}{1.932713in}}{\pgfqpoint{1.587557in}{1.940613in}}{\pgfqpoint{1.581733in}{1.946437in}}%
\pgfpathcurveto{\pgfqpoint{1.575909in}{1.952261in}}{\pgfqpoint{1.568009in}{1.955533in}}{\pgfqpoint{1.559772in}{1.955533in}}%
\pgfpathcurveto{\pgfqpoint{1.551536in}{1.955533in}}{\pgfqpoint{1.543636in}{1.952261in}}{\pgfqpoint{1.537812in}{1.946437in}}%
\pgfpathcurveto{\pgfqpoint{1.531988in}{1.940613in}}{\pgfqpoint{1.528716in}{1.932713in}}{\pgfqpoint{1.528716in}{1.924476in}}%
\pgfpathcurveto{\pgfqpoint{1.528716in}{1.916240in}}{\pgfqpoint{1.531988in}{1.908340in}}{\pgfqpoint{1.537812in}{1.902516in}}%
\pgfpathcurveto{\pgfqpoint{1.543636in}{1.896692in}}{\pgfqpoint{1.551536in}{1.893420in}}{\pgfqpoint{1.559772in}{1.893420in}}%
\pgfpathclose%
\pgfusepath{stroke,fill}%
\end{pgfscope}%
\begin{pgfscope}%
\pgfpathrectangle{\pgfqpoint{0.100000in}{0.220728in}}{\pgfqpoint{3.696000in}{3.696000in}}%
\pgfusepath{clip}%
\pgfsetbuttcap%
\pgfsetroundjoin%
\definecolor{currentfill}{rgb}{0.121569,0.466667,0.705882}%
\pgfsetfillcolor{currentfill}%
\pgfsetfillopacity{0.916743}%
\pgfsetlinewidth{1.003750pt}%
\definecolor{currentstroke}{rgb}{0.121569,0.466667,0.705882}%
\pgfsetstrokecolor{currentstroke}%
\pgfsetstrokeopacity{0.916743}%
\pgfsetdash{}{0pt}%
\pgfpathmoveto{\pgfqpoint{2.351536in}{1.430143in}}%
\pgfpathcurveto{\pgfqpoint{2.359772in}{1.430143in}}{\pgfqpoint{2.367672in}{1.433416in}}{\pgfqpoint{2.373496in}{1.439240in}}%
\pgfpathcurveto{\pgfqpoint{2.379320in}{1.445064in}}{\pgfqpoint{2.382592in}{1.452964in}}{\pgfqpoint{2.382592in}{1.461200in}}%
\pgfpathcurveto{\pgfqpoint{2.382592in}{1.469436in}}{\pgfqpoint{2.379320in}{1.477336in}}{\pgfqpoint{2.373496in}{1.483160in}}%
\pgfpathcurveto{\pgfqpoint{2.367672in}{1.488984in}}{\pgfqpoint{2.359772in}{1.492256in}}{\pgfqpoint{2.351536in}{1.492256in}}%
\pgfpathcurveto{\pgfqpoint{2.343300in}{1.492256in}}{\pgfqpoint{2.335399in}{1.488984in}}{\pgfqpoint{2.329576in}{1.483160in}}%
\pgfpathcurveto{\pgfqpoint{2.323752in}{1.477336in}}{\pgfqpoint{2.320479in}{1.469436in}}{\pgfqpoint{2.320479in}{1.461200in}}%
\pgfpathcurveto{\pgfqpoint{2.320479in}{1.452964in}}{\pgfqpoint{2.323752in}{1.445064in}}{\pgfqpoint{2.329576in}{1.439240in}}%
\pgfpathcurveto{\pgfqpoint{2.335399in}{1.433416in}}{\pgfqpoint{2.343300in}{1.430143in}}{\pgfqpoint{2.351536in}{1.430143in}}%
\pgfpathclose%
\pgfusepath{stroke,fill}%
\end{pgfscope}%
\begin{pgfscope}%
\pgfpathrectangle{\pgfqpoint{0.100000in}{0.220728in}}{\pgfqpoint{3.696000in}{3.696000in}}%
\pgfusepath{clip}%
\pgfsetbuttcap%
\pgfsetroundjoin%
\definecolor{currentfill}{rgb}{0.121569,0.466667,0.705882}%
\pgfsetfillcolor{currentfill}%
\pgfsetfillopacity{0.919707}%
\pgfsetlinewidth{1.003750pt}%
\definecolor{currentstroke}{rgb}{0.121569,0.466667,0.705882}%
\pgfsetstrokecolor{currentstroke}%
\pgfsetstrokeopacity{0.919707}%
\pgfsetdash{}{0pt}%
\pgfpathmoveto{\pgfqpoint{1.623167in}{1.843414in}}%
\pgfpathcurveto{\pgfqpoint{1.631403in}{1.843414in}}{\pgfqpoint{1.639303in}{1.846686in}}{\pgfqpoint{1.645127in}{1.852510in}}%
\pgfpathcurveto{\pgfqpoint{1.650951in}{1.858334in}}{\pgfqpoint{1.654223in}{1.866234in}}{\pgfqpoint{1.654223in}{1.874470in}}%
\pgfpathcurveto{\pgfqpoint{1.654223in}{1.882706in}}{\pgfqpoint{1.650951in}{1.890606in}}{\pgfqpoint{1.645127in}{1.896430in}}%
\pgfpathcurveto{\pgfqpoint{1.639303in}{1.902254in}}{\pgfqpoint{1.631403in}{1.905527in}}{\pgfqpoint{1.623167in}{1.905527in}}%
\pgfpathcurveto{\pgfqpoint{1.614930in}{1.905527in}}{\pgfqpoint{1.607030in}{1.902254in}}{\pgfqpoint{1.601207in}{1.896430in}}%
\pgfpathcurveto{\pgfqpoint{1.595383in}{1.890606in}}{\pgfqpoint{1.592110in}{1.882706in}}{\pgfqpoint{1.592110in}{1.874470in}}%
\pgfpathcurveto{\pgfqpoint{1.592110in}{1.866234in}}{\pgfqpoint{1.595383in}{1.858334in}}{\pgfqpoint{1.601207in}{1.852510in}}%
\pgfpathcurveto{\pgfqpoint{1.607030in}{1.846686in}}{\pgfqpoint{1.614930in}{1.843414in}}{\pgfqpoint{1.623167in}{1.843414in}}%
\pgfpathclose%
\pgfusepath{stroke,fill}%
\end{pgfscope}%
\begin{pgfscope}%
\pgfpathrectangle{\pgfqpoint{0.100000in}{0.220728in}}{\pgfqpoint{3.696000in}{3.696000in}}%
\pgfusepath{clip}%
\pgfsetbuttcap%
\pgfsetroundjoin%
\definecolor{currentfill}{rgb}{0.121569,0.466667,0.705882}%
\pgfsetfillcolor{currentfill}%
\pgfsetfillopacity{0.919933}%
\pgfsetlinewidth{1.003750pt}%
\definecolor{currentstroke}{rgb}{0.121569,0.466667,0.705882}%
\pgfsetstrokecolor{currentstroke}%
\pgfsetstrokeopacity{0.919933}%
\pgfsetdash{}{0pt}%
\pgfpathmoveto{\pgfqpoint{1.590742in}{1.876970in}}%
\pgfpathcurveto{\pgfqpoint{1.598978in}{1.876970in}}{\pgfqpoint{1.606878in}{1.880242in}}{\pgfqpoint{1.612702in}{1.886066in}}%
\pgfpathcurveto{\pgfqpoint{1.618526in}{1.891890in}}{\pgfqpoint{1.621798in}{1.899790in}}{\pgfqpoint{1.621798in}{1.908026in}}%
\pgfpathcurveto{\pgfqpoint{1.621798in}{1.916262in}}{\pgfqpoint{1.618526in}{1.924162in}}{\pgfqpoint{1.612702in}{1.929986in}}%
\pgfpathcurveto{\pgfqpoint{1.606878in}{1.935810in}}{\pgfqpoint{1.598978in}{1.939083in}}{\pgfqpoint{1.590742in}{1.939083in}}%
\pgfpathcurveto{\pgfqpoint{1.582505in}{1.939083in}}{\pgfqpoint{1.574605in}{1.935810in}}{\pgfqpoint{1.568781in}{1.929986in}}%
\pgfpathcurveto{\pgfqpoint{1.562957in}{1.924162in}}{\pgfqpoint{1.559685in}{1.916262in}}{\pgfqpoint{1.559685in}{1.908026in}}%
\pgfpathcurveto{\pgfqpoint{1.559685in}{1.899790in}}{\pgfqpoint{1.562957in}{1.891890in}}{\pgfqpoint{1.568781in}{1.886066in}}%
\pgfpathcurveto{\pgfqpoint{1.574605in}{1.880242in}}{\pgfqpoint{1.582505in}{1.876970in}}{\pgfqpoint{1.590742in}{1.876970in}}%
\pgfpathclose%
\pgfusepath{stroke,fill}%
\end{pgfscope}%
\begin{pgfscope}%
\pgfpathrectangle{\pgfqpoint{0.100000in}{0.220728in}}{\pgfqpoint{3.696000in}{3.696000in}}%
\pgfusepath{clip}%
\pgfsetbuttcap%
\pgfsetroundjoin%
\definecolor{currentfill}{rgb}{0.121569,0.466667,0.705882}%
\pgfsetfillcolor{currentfill}%
\pgfsetfillopacity{0.921009}%
\pgfsetlinewidth{1.003750pt}%
\definecolor{currentstroke}{rgb}{0.121569,0.466667,0.705882}%
\pgfsetstrokecolor{currentstroke}%
\pgfsetstrokeopacity{0.921009}%
\pgfsetdash{}{0pt}%
\pgfpathmoveto{\pgfqpoint{2.354394in}{1.427385in}}%
\pgfpathcurveto{\pgfqpoint{2.362631in}{1.427385in}}{\pgfqpoint{2.370531in}{1.430657in}}{\pgfqpoint{2.376354in}{1.436481in}}%
\pgfpathcurveto{\pgfqpoint{2.382178in}{1.442305in}}{\pgfqpoint{2.385451in}{1.450205in}}{\pgfqpoint{2.385451in}{1.458441in}}%
\pgfpathcurveto{\pgfqpoint{2.385451in}{1.466678in}}{\pgfqpoint{2.382178in}{1.474578in}}{\pgfqpoint{2.376354in}{1.480402in}}%
\pgfpathcurveto{\pgfqpoint{2.370531in}{1.486225in}}{\pgfqpoint{2.362631in}{1.489498in}}{\pgfqpoint{2.354394in}{1.489498in}}%
\pgfpathcurveto{\pgfqpoint{2.346158in}{1.489498in}}{\pgfqpoint{2.338258in}{1.486225in}}{\pgfqpoint{2.332434in}{1.480402in}}%
\pgfpathcurveto{\pgfqpoint{2.326610in}{1.474578in}}{\pgfqpoint{2.323338in}{1.466678in}}{\pgfqpoint{2.323338in}{1.458441in}}%
\pgfpathcurveto{\pgfqpoint{2.323338in}{1.450205in}}{\pgfqpoint{2.326610in}{1.442305in}}{\pgfqpoint{2.332434in}{1.436481in}}%
\pgfpathcurveto{\pgfqpoint{2.338258in}{1.430657in}}{\pgfqpoint{2.346158in}{1.427385in}}{\pgfqpoint{2.354394in}{1.427385in}}%
\pgfpathclose%
\pgfusepath{stroke,fill}%
\end{pgfscope}%
\begin{pgfscope}%
\pgfpathrectangle{\pgfqpoint{0.100000in}{0.220728in}}{\pgfqpoint{3.696000in}{3.696000in}}%
\pgfusepath{clip}%
\pgfsetbuttcap%
\pgfsetroundjoin%
\definecolor{currentfill}{rgb}{0.121569,0.466667,0.705882}%
\pgfsetfillcolor{currentfill}%
\pgfsetfillopacity{0.923434}%
\pgfsetlinewidth{1.003750pt}%
\definecolor{currentstroke}{rgb}{0.121569,0.466667,0.705882}%
\pgfsetstrokecolor{currentstroke}%
\pgfsetstrokeopacity{0.923434}%
\pgfsetdash{}{0pt}%
\pgfpathmoveto{\pgfqpoint{1.648667in}{1.828864in}}%
\pgfpathcurveto{\pgfqpoint{1.656903in}{1.828864in}}{\pgfqpoint{1.664803in}{1.832136in}}{\pgfqpoint{1.670627in}{1.837960in}}%
\pgfpathcurveto{\pgfqpoint{1.676451in}{1.843784in}}{\pgfqpoint{1.679723in}{1.851684in}}{\pgfqpoint{1.679723in}{1.859920in}}%
\pgfpathcurveto{\pgfqpoint{1.679723in}{1.868157in}}{\pgfqpoint{1.676451in}{1.876057in}}{\pgfqpoint{1.670627in}{1.881881in}}%
\pgfpathcurveto{\pgfqpoint{1.664803in}{1.887705in}}{\pgfqpoint{1.656903in}{1.890977in}}{\pgfqpoint{1.648667in}{1.890977in}}%
\pgfpathcurveto{\pgfqpoint{1.640430in}{1.890977in}}{\pgfqpoint{1.632530in}{1.887705in}}{\pgfqpoint{1.626706in}{1.881881in}}%
\pgfpathcurveto{\pgfqpoint{1.620882in}{1.876057in}}{\pgfqpoint{1.617610in}{1.868157in}}{\pgfqpoint{1.617610in}{1.859920in}}%
\pgfpathcurveto{\pgfqpoint{1.617610in}{1.851684in}}{\pgfqpoint{1.620882in}{1.843784in}}{\pgfqpoint{1.626706in}{1.837960in}}%
\pgfpathcurveto{\pgfqpoint{1.632530in}{1.832136in}}{\pgfqpoint{1.640430in}{1.828864in}}{\pgfqpoint{1.648667in}{1.828864in}}%
\pgfpathclose%
\pgfusepath{stroke,fill}%
\end{pgfscope}%
\begin{pgfscope}%
\pgfpathrectangle{\pgfqpoint{0.100000in}{0.220728in}}{\pgfqpoint{3.696000in}{3.696000in}}%
\pgfusepath{clip}%
\pgfsetbuttcap%
\pgfsetroundjoin%
\definecolor{currentfill}{rgb}{0.121569,0.466667,0.705882}%
\pgfsetfillcolor{currentfill}%
\pgfsetfillopacity{0.925079}%
\pgfsetlinewidth{1.003750pt}%
\definecolor{currentstroke}{rgb}{0.121569,0.466667,0.705882}%
\pgfsetstrokecolor{currentstroke}%
\pgfsetstrokeopacity{0.925079}%
\pgfsetdash{}{0pt}%
\pgfpathmoveto{\pgfqpoint{2.357660in}{1.421224in}}%
\pgfpathcurveto{\pgfqpoint{2.365896in}{1.421224in}}{\pgfqpoint{2.373796in}{1.424496in}}{\pgfqpoint{2.379620in}{1.430320in}}%
\pgfpathcurveto{\pgfqpoint{2.385444in}{1.436144in}}{\pgfqpoint{2.388716in}{1.444044in}}{\pgfqpoint{2.388716in}{1.452280in}}%
\pgfpathcurveto{\pgfqpoint{2.388716in}{1.460516in}}{\pgfqpoint{2.385444in}{1.468416in}}{\pgfqpoint{2.379620in}{1.474240in}}%
\pgfpathcurveto{\pgfqpoint{2.373796in}{1.480064in}}{\pgfqpoint{2.365896in}{1.483337in}}{\pgfqpoint{2.357660in}{1.483337in}}%
\pgfpathcurveto{\pgfqpoint{2.349423in}{1.483337in}}{\pgfqpoint{2.341523in}{1.480064in}}{\pgfqpoint{2.335699in}{1.474240in}}%
\pgfpathcurveto{\pgfqpoint{2.329875in}{1.468416in}}{\pgfqpoint{2.326603in}{1.460516in}}{\pgfqpoint{2.326603in}{1.452280in}}%
\pgfpathcurveto{\pgfqpoint{2.326603in}{1.444044in}}{\pgfqpoint{2.329875in}{1.436144in}}{\pgfqpoint{2.335699in}{1.430320in}}%
\pgfpathcurveto{\pgfqpoint{2.341523in}{1.424496in}}{\pgfqpoint{2.349423in}{1.421224in}}{\pgfqpoint{2.357660in}{1.421224in}}%
\pgfpathclose%
\pgfusepath{stroke,fill}%
\end{pgfscope}%
\begin{pgfscope}%
\pgfpathrectangle{\pgfqpoint{0.100000in}{0.220728in}}{\pgfqpoint{3.696000in}{3.696000in}}%
\pgfusepath{clip}%
\pgfsetbuttcap%
\pgfsetroundjoin%
\definecolor{currentfill}{rgb}{0.121569,0.466667,0.705882}%
\pgfsetfillcolor{currentfill}%
\pgfsetfillopacity{0.925631}%
\pgfsetlinewidth{1.003750pt}%
\definecolor{currentstroke}{rgb}{0.121569,0.466667,0.705882}%
\pgfsetstrokecolor{currentstroke}%
\pgfsetstrokeopacity{0.925631}%
\pgfsetdash{}{0pt}%
\pgfpathmoveto{\pgfqpoint{1.672804in}{1.804814in}}%
\pgfpathcurveto{\pgfqpoint{1.681040in}{1.804814in}}{\pgfqpoint{1.688940in}{1.808086in}}{\pgfqpoint{1.694764in}{1.813910in}}%
\pgfpathcurveto{\pgfqpoint{1.700588in}{1.819734in}}{\pgfqpoint{1.703860in}{1.827634in}}{\pgfqpoint{1.703860in}{1.835870in}}%
\pgfpathcurveto{\pgfqpoint{1.703860in}{1.844106in}}{\pgfqpoint{1.700588in}{1.852007in}}{\pgfqpoint{1.694764in}{1.857830in}}%
\pgfpathcurveto{\pgfqpoint{1.688940in}{1.863654in}}{\pgfqpoint{1.681040in}{1.866927in}}{\pgfqpoint{1.672804in}{1.866927in}}%
\pgfpathcurveto{\pgfqpoint{1.664568in}{1.866927in}}{\pgfqpoint{1.656668in}{1.863654in}}{\pgfqpoint{1.650844in}{1.857830in}}%
\pgfpathcurveto{\pgfqpoint{1.645020in}{1.852007in}}{\pgfqpoint{1.641747in}{1.844106in}}{\pgfqpoint{1.641747in}{1.835870in}}%
\pgfpathcurveto{\pgfqpoint{1.641747in}{1.827634in}}{\pgfqpoint{1.645020in}{1.819734in}}{\pgfqpoint{1.650844in}{1.813910in}}%
\pgfpathcurveto{\pgfqpoint{1.656668in}{1.808086in}}{\pgfqpoint{1.664568in}{1.804814in}}{\pgfqpoint{1.672804in}{1.804814in}}%
\pgfpathclose%
\pgfusepath{stroke,fill}%
\end{pgfscope}%
\begin{pgfscope}%
\pgfpathrectangle{\pgfqpoint{0.100000in}{0.220728in}}{\pgfqpoint{3.696000in}{3.696000in}}%
\pgfusepath{clip}%
\pgfsetbuttcap%
\pgfsetroundjoin%
\definecolor{currentfill}{rgb}{0.121569,0.466667,0.705882}%
\pgfsetfillcolor{currentfill}%
\pgfsetfillopacity{0.927874}%
\pgfsetlinewidth{1.003750pt}%
\definecolor{currentstroke}{rgb}{0.121569,0.466667,0.705882}%
\pgfsetstrokecolor{currentstroke}%
\pgfsetstrokeopacity{0.927874}%
\pgfsetdash{}{0pt}%
\pgfpathmoveto{\pgfqpoint{1.693756in}{1.785952in}}%
\pgfpathcurveto{\pgfqpoint{1.701992in}{1.785952in}}{\pgfqpoint{1.709892in}{1.789224in}}{\pgfqpoint{1.715716in}{1.795048in}}%
\pgfpathcurveto{\pgfqpoint{1.721540in}{1.800872in}}{\pgfqpoint{1.724812in}{1.808772in}}{\pgfqpoint{1.724812in}{1.817008in}}%
\pgfpathcurveto{\pgfqpoint{1.724812in}{1.825244in}}{\pgfqpoint{1.721540in}{1.833144in}}{\pgfqpoint{1.715716in}{1.838968in}}%
\pgfpathcurveto{\pgfqpoint{1.709892in}{1.844792in}}{\pgfqpoint{1.701992in}{1.848065in}}{\pgfqpoint{1.693756in}{1.848065in}}%
\pgfpathcurveto{\pgfqpoint{1.685519in}{1.848065in}}{\pgfqpoint{1.677619in}{1.844792in}}{\pgfqpoint{1.671795in}{1.838968in}}%
\pgfpathcurveto{\pgfqpoint{1.665971in}{1.833144in}}{\pgfqpoint{1.662699in}{1.825244in}}{\pgfqpoint{1.662699in}{1.817008in}}%
\pgfpathcurveto{\pgfqpoint{1.662699in}{1.808772in}}{\pgfqpoint{1.665971in}{1.800872in}}{\pgfqpoint{1.671795in}{1.795048in}}%
\pgfpathcurveto{\pgfqpoint{1.677619in}{1.789224in}}{\pgfqpoint{1.685519in}{1.785952in}}{\pgfqpoint{1.693756in}{1.785952in}}%
\pgfpathclose%
\pgfusepath{stroke,fill}%
\end{pgfscope}%
\begin{pgfscope}%
\pgfpathrectangle{\pgfqpoint{0.100000in}{0.220728in}}{\pgfqpoint{3.696000in}{3.696000in}}%
\pgfusepath{clip}%
\pgfsetbuttcap%
\pgfsetroundjoin%
\definecolor{currentfill}{rgb}{0.121569,0.466667,0.705882}%
\pgfsetfillcolor{currentfill}%
\pgfsetfillopacity{0.928633}%
\pgfsetlinewidth{1.003750pt}%
\definecolor{currentstroke}{rgb}{0.121569,0.466667,0.705882}%
\pgfsetstrokecolor{currentstroke}%
\pgfsetstrokeopacity{0.928633}%
\pgfsetdash{}{0pt}%
\pgfpathmoveto{\pgfqpoint{2.359937in}{1.409993in}}%
\pgfpathcurveto{\pgfqpoint{2.368173in}{1.409993in}}{\pgfqpoint{2.376073in}{1.413265in}}{\pgfqpoint{2.381897in}{1.419089in}}%
\pgfpathcurveto{\pgfqpoint{2.387721in}{1.424913in}}{\pgfqpoint{2.390993in}{1.432813in}}{\pgfqpoint{2.390993in}{1.441049in}}%
\pgfpathcurveto{\pgfqpoint{2.390993in}{1.449285in}}{\pgfqpoint{2.387721in}{1.457185in}}{\pgfqpoint{2.381897in}{1.463009in}}%
\pgfpathcurveto{\pgfqpoint{2.376073in}{1.468833in}}{\pgfqpoint{2.368173in}{1.472106in}}{\pgfqpoint{2.359937in}{1.472106in}}%
\pgfpathcurveto{\pgfqpoint{2.351700in}{1.472106in}}{\pgfqpoint{2.343800in}{1.468833in}}{\pgfqpoint{2.337976in}{1.463009in}}%
\pgfpathcurveto{\pgfqpoint{2.332152in}{1.457185in}}{\pgfqpoint{2.328880in}{1.449285in}}{\pgfqpoint{2.328880in}{1.441049in}}%
\pgfpathcurveto{\pgfqpoint{2.328880in}{1.432813in}}{\pgfqpoint{2.332152in}{1.424913in}}{\pgfqpoint{2.337976in}{1.419089in}}%
\pgfpathcurveto{\pgfqpoint{2.343800in}{1.413265in}}{\pgfqpoint{2.351700in}{1.409993in}}{\pgfqpoint{2.359937in}{1.409993in}}%
\pgfpathclose%
\pgfusepath{stroke,fill}%
\end{pgfscope}%
\begin{pgfscope}%
\pgfpathrectangle{\pgfqpoint{0.100000in}{0.220728in}}{\pgfqpoint{3.696000in}{3.696000in}}%
\pgfusepath{clip}%
\pgfsetbuttcap%
\pgfsetroundjoin%
\definecolor{currentfill}{rgb}{0.121569,0.466667,0.705882}%
\pgfsetfillcolor{currentfill}%
\pgfsetfillopacity{0.933392}%
\pgfsetlinewidth{1.003750pt}%
\definecolor{currentstroke}{rgb}{0.121569,0.466667,0.705882}%
\pgfsetstrokecolor{currentstroke}%
\pgfsetstrokeopacity{0.933392}%
\pgfsetdash{}{0pt}%
\pgfpathmoveto{\pgfqpoint{2.364362in}{1.404592in}}%
\pgfpathcurveto{\pgfqpoint{2.372598in}{1.404592in}}{\pgfqpoint{2.380498in}{1.407864in}}{\pgfqpoint{2.386322in}{1.413688in}}%
\pgfpathcurveto{\pgfqpoint{2.392146in}{1.419512in}}{\pgfqpoint{2.395418in}{1.427412in}}{\pgfqpoint{2.395418in}{1.435648in}}%
\pgfpathcurveto{\pgfqpoint{2.395418in}{1.443885in}}{\pgfqpoint{2.392146in}{1.451785in}}{\pgfqpoint{2.386322in}{1.457609in}}%
\pgfpathcurveto{\pgfqpoint{2.380498in}{1.463433in}}{\pgfqpoint{2.372598in}{1.466705in}}{\pgfqpoint{2.364362in}{1.466705in}}%
\pgfpathcurveto{\pgfqpoint{2.356125in}{1.466705in}}{\pgfqpoint{2.348225in}{1.463433in}}{\pgfqpoint{2.342401in}{1.457609in}}%
\pgfpathcurveto{\pgfqpoint{2.336577in}{1.451785in}}{\pgfqpoint{2.333305in}{1.443885in}}{\pgfqpoint{2.333305in}{1.435648in}}%
\pgfpathcurveto{\pgfqpoint{2.333305in}{1.427412in}}{\pgfqpoint{2.336577in}{1.419512in}}{\pgfqpoint{2.342401in}{1.413688in}}%
\pgfpathcurveto{\pgfqpoint{2.348225in}{1.407864in}}{\pgfqpoint{2.356125in}{1.404592in}}{\pgfqpoint{2.364362in}{1.404592in}}%
\pgfpathclose%
\pgfusepath{stroke,fill}%
\end{pgfscope}%
\begin{pgfscope}%
\pgfpathrectangle{\pgfqpoint{0.100000in}{0.220728in}}{\pgfqpoint{3.696000in}{3.696000in}}%
\pgfusepath{clip}%
\pgfsetbuttcap%
\pgfsetroundjoin%
\definecolor{currentfill}{rgb}{0.121569,0.466667,0.705882}%
\pgfsetfillcolor{currentfill}%
\pgfsetfillopacity{0.933946}%
\pgfsetlinewidth{1.003750pt}%
\definecolor{currentstroke}{rgb}{0.121569,0.466667,0.705882}%
\pgfsetstrokecolor{currentstroke}%
\pgfsetstrokeopacity{0.933946}%
\pgfsetdash{}{0pt}%
\pgfpathmoveto{\pgfqpoint{1.732638in}{1.766198in}}%
\pgfpathcurveto{\pgfqpoint{1.740875in}{1.766198in}}{\pgfqpoint{1.748775in}{1.769470in}}{\pgfqpoint{1.754599in}{1.775294in}}%
\pgfpathcurveto{\pgfqpoint{1.760422in}{1.781118in}}{\pgfqpoint{1.763695in}{1.789018in}}{\pgfqpoint{1.763695in}{1.797255in}}%
\pgfpathcurveto{\pgfqpoint{1.763695in}{1.805491in}}{\pgfqpoint{1.760422in}{1.813391in}}{\pgfqpoint{1.754599in}{1.819215in}}%
\pgfpathcurveto{\pgfqpoint{1.748775in}{1.825039in}}{\pgfqpoint{1.740875in}{1.828311in}}{\pgfqpoint{1.732638in}{1.828311in}}%
\pgfpathcurveto{\pgfqpoint{1.724402in}{1.828311in}}{\pgfqpoint{1.716502in}{1.825039in}}{\pgfqpoint{1.710678in}{1.819215in}}%
\pgfpathcurveto{\pgfqpoint{1.704854in}{1.813391in}}{\pgfqpoint{1.701582in}{1.805491in}}{\pgfqpoint{1.701582in}{1.797255in}}%
\pgfpathcurveto{\pgfqpoint{1.701582in}{1.789018in}}{\pgfqpoint{1.704854in}{1.781118in}}{\pgfqpoint{1.710678in}{1.775294in}}%
\pgfpathcurveto{\pgfqpoint{1.716502in}{1.769470in}}{\pgfqpoint{1.724402in}{1.766198in}}{\pgfqpoint{1.732638in}{1.766198in}}%
\pgfpathclose%
\pgfusepath{stroke,fill}%
\end{pgfscope}%
\begin{pgfscope}%
\pgfpathrectangle{\pgfqpoint{0.100000in}{0.220728in}}{\pgfqpoint{3.696000in}{3.696000in}}%
\pgfusepath{clip}%
\pgfsetbuttcap%
\pgfsetroundjoin%
\definecolor{currentfill}{rgb}{0.121569,0.466667,0.705882}%
\pgfsetfillcolor{currentfill}%
\pgfsetfillopacity{0.937502}%
\pgfsetlinewidth{1.003750pt}%
\definecolor{currentstroke}{rgb}{0.121569,0.466667,0.705882}%
\pgfsetstrokecolor{currentstroke}%
\pgfsetstrokeopacity{0.937502}%
\pgfsetdash{}{0pt}%
\pgfpathmoveto{\pgfqpoint{1.771125in}{1.742915in}}%
\pgfpathcurveto{\pgfqpoint{1.779361in}{1.742915in}}{\pgfqpoint{1.787261in}{1.746187in}}{\pgfqpoint{1.793085in}{1.752011in}}%
\pgfpathcurveto{\pgfqpoint{1.798909in}{1.757835in}}{\pgfqpoint{1.802181in}{1.765735in}}{\pgfqpoint{1.802181in}{1.773971in}}%
\pgfpathcurveto{\pgfqpoint{1.802181in}{1.782208in}}{\pgfqpoint{1.798909in}{1.790108in}}{\pgfqpoint{1.793085in}{1.795932in}}%
\pgfpathcurveto{\pgfqpoint{1.787261in}{1.801756in}}{\pgfqpoint{1.779361in}{1.805028in}}{\pgfqpoint{1.771125in}{1.805028in}}%
\pgfpathcurveto{\pgfqpoint{1.762889in}{1.805028in}}{\pgfqpoint{1.754989in}{1.801756in}}{\pgfqpoint{1.749165in}{1.795932in}}%
\pgfpathcurveto{\pgfqpoint{1.743341in}{1.790108in}}{\pgfqpoint{1.740068in}{1.782208in}}{\pgfqpoint{1.740068in}{1.773971in}}%
\pgfpathcurveto{\pgfqpoint{1.740068in}{1.765735in}}{\pgfqpoint{1.743341in}{1.757835in}}{\pgfqpoint{1.749165in}{1.752011in}}%
\pgfpathcurveto{\pgfqpoint{1.754989in}{1.746187in}}{\pgfqpoint{1.762889in}{1.742915in}}{\pgfqpoint{1.771125in}{1.742915in}}%
\pgfpathclose%
\pgfusepath{stroke,fill}%
\end{pgfscope}%
\begin{pgfscope}%
\pgfpathrectangle{\pgfqpoint{0.100000in}{0.220728in}}{\pgfqpoint{3.696000in}{3.696000in}}%
\pgfusepath{clip}%
\pgfsetbuttcap%
\pgfsetroundjoin%
\definecolor{currentfill}{rgb}{0.121569,0.466667,0.705882}%
\pgfsetfillcolor{currentfill}%
\pgfsetfillopacity{0.938570}%
\pgfsetlinewidth{1.003750pt}%
\definecolor{currentstroke}{rgb}{0.121569,0.466667,0.705882}%
\pgfsetstrokecolor{currentstroke}%
\pgfsetstrokeopacity{0.938570}%
\pgfsetdash{}{0pt}%
\pgfpathmoveto{\pgfqpoint{2.368911in}{1.399325in}}%
\pgfpathcurveto{\pgfqpoint{2.377148in}{1.399325in}}{\pgfqpoint{2.385048in}{1.402597in}}{\pgfqpoint{2.390872in}{1.408421in}}%
\pgfpathcurveto{\pgfqpoint{2.396696in}{1.414245in}}{\pgfqpoint{2.399968in}{1.422145in}}{\pgfqpoint{2.399968in}{1.430381in}}%
\pgfpathcurveto{\pgfqpoint{2.399968in}{1.438617in}}{\pgfqpoint{2.396696in}{1.446517in}}{\pgfqpoint{2.390872in}{1.452341in}}%
\pgfpathcurveto{\pgfqpoint{2.385048in}{1.458165in}}{\pgfqpoint{2.377148in}{1.461438in}}{\pgfqpoint{2.368911in}{1.461438in}}%
\pgfpathcurveto{\pgfqpoint{2.360675in}{1.461438in}}{\pgfqpoint{2.352775in}{1.458165in}}{\pgfqpoint{2.346951in}{1.452341in}}%
\pgfpathcurveto{\pgfqpoint{2.341127in}{1.446517in}}{\pgfqpoint{2.337855in}{1.438617in}}{\pgfqpoint{2.337855in}{1.430381in}}%
\pgfpathcurveto{\pgfqpoint{2.337855in}{1.422145in}}{\pgfqpoint{2.341127in}{1.414245in}}{\pgfqpoint{2.346951in}{1.408421in}}%
\pgfpathcurveto{\pgfqpoint{2.352775in}{1.402597in}}{\pgfqpoint{2.360675in}{1.399325in}}{\pgfqpoint{2.368911in}{1.399325in}}%
\pgfpathclose%
\pgfusepath{stroke,fill}%
\end{pgfscope}%
\begin{pgfscope}%
\pgfpathrectangle{\pgfqpoint{0.100000in}{0.220728in}}{\pgfqpoint{3.696000in}{3.696000in}}%
\pgfusepath{clip}%
\pgfsetbuttcap%
\pgfsetroundjoin%
\definecolor{currentfill}{rgb}{0.121569,0.466667,0.705882}%
\pgfsetfillcolor{currentfill}%
\pgfsetfillopacity{0.941524}%
\pgfsetlinewidth{1.003750pt}%
\definecolor{currentstroke}{rgb}{0.121569,0.466667,0.705882}%
\pgfsetstrokecolor{currentstroke}%
\pgfsetstrokeopacity{0.941524}%
\pgfsetdash{}{0pt}%
\pgfpathmoveto{\pgfqpoint{1.808439in}{1.723691in}}%
\pgfpathcurveto{\pgfqpoint{1.816675in}{1.723691in}}{\pgfqpoint{1.824575in}{1.726963in}}{\pgfqpoint{1.830399in}{1.732787in}}%
\pgfpathcurveto{\pgfqpoint{1.836223in}{1.738611in}}{\pgfqpoint{1.839495in}{1.746511in}}{\pgfqpoint{1.839495in}{1.754747in}}%
\pgfpathcurveto{\pgfqpoint{1.839495in}{1.762983in}}{\pgfqpoint{1.836223in}{1.770883in}}{\pgfqpoint{1.830399in}{1.776707in}}%
\pgfpathcurveto{\pgfqpoint{1.824575in}{1.782531in}}{\pgfqpoint{1.816675in}{1.785804in}}{\pgfqpoint{1.808439in}{1.785804in}}%
\pgfpathcurveto{\pgfqpoint{1.800203in}{1.785804in}}{\pgfqpoint{1.792303in}{1.782531in}}{\pgfqpoint{1.786479in}{1.776707in}}%
\pgfpathcurveto{\pgfqpoint{1.780655in}{1.770883in}}{\pgfqpoint{1.777382in}{1.762983in}}{\pgfqpoint{1.777382in}{1.754747in}}%
\pgfpathcurveto{\pgfqpoint{1.777382in}{1.746511in}}{\pgfqpoint{1.780655in}{1.738611in}}{\pgfqpoint{1.786479in}{1.732787in}}%
\pgfpathcurveto{\pgfqpoint{1.792303in}{1.726963in}}{\pgfqpoint{1.800203in}{1.723691in}}{\pgfqpoint{1.808439in}{1.723691in}}%
\pgfpathclose%
\pgfusepath{stroke,fill}%
\end{pgfscope}%
\begin{pgfscope}%
\pgfpathrectangle{\pgfqpoint{0.100000in}{0.220728in}}{\pgfqpoint{3.696000in}{3.696000in}}%
\pgfusepath{clip}%
\pgfsetbuttcap%
\pgfsetroundjoin%
\definecolor{currentfill}{rgb}{0.121569,0.466667,0.705882}%
\pgfsetfillcolor{currentfill}%
\pgfsetfillopacity{0.943614}%
\pgfsetlinewidth{1.003750pt}%
\definecolor{currentstroke}{rgb}{0.121569,0.466667,0.705882}%
\pgfsetstrokecolor{currentstroke}%
\pgfsetstrokeopacity{0.943614}%
\pgfsetdash{}{0pt}%
\pgfpathmoveto{\pgfqpoint{1.842136in}{1.692301in}}%
\pgfpathcurveto{\pgfqpoint{1.850372in}{1.692301in}}{\pgfqpoint{1.858272in}{1.695574in}}{\pgfqpoint{1.864096in}{1.701398in}}%
\pgfpathcurveto{\pgfqpoint{1.869920in}{1.707222in}}{\pgfqpoint{1.873192in}{1.715122in}}{\pgfqpoint{1.873192in}{1.723358in}}%
\pgfpathcurveto{\pgfqpoint{1.873192in}{1.731594in}}{\pgfqpoint{1.869920in}{1.739494in}}{\pgfqpoint{1.864096in}{1.745318in}}%
\pgfpathcurveto{\pgfqpoint{1.858272in}{1.751142in}}{\pgfqpoint{1.850372in}{1.754414in}}{\pgfqpoint{1.842136in}{1.754414in}}%
\pgfpathcurveto{\pgfqpoint{1.833899in}{1.754414in}}{\pgfqpoint{1.825999in}{1.751142in}}{\pgfqpoint{1.820175in}{1.745318in}}%
\pgfpathcurveto{\pgfqpoint{1.814351in}{1.739494in}}{\pgfqpoint{1.811079in}{1.731594in}}{\pgfqpoint{1.811079in}{1.723358in}}%
\pgfpathcurveto{\pgfqpoint{1.811079in}{1.715122in}}{\pgfqpoint{1.814351in}{1.707222in}}{\pgfqpoint{1.820175in}{1.701398in}}%
\pgfpathcurveto{\pgfqpoint{1.825999in}{1.695574in}}{\pgfqpoint{1.833899in}{1.692301in}}{\pgfqpoint{1.842136in}{1.692301in}}%
\pgfpathclose%
\pgfusepath{stroke,fill}%
\end{pgfscope}%
\begin{pgfscope}%
\pgfpathrectangle{\pgfqpoint{0.100000in}{0.220728in}}{\pgfqpoint{3.696000in}{3.696000in}}%
\pgfusepath{clip}%
\pgfsetbuttcap%
\pgfsetroundjoin%
\definecolor{currentfill}{rgb}{0.121569,0.466667,0.705882}%
\pgfsetfillcolor{currentfill}%
\pgfsetfillopacity{0.944658}%
\pgfsetlinewidth{1.003750pt}%
\definecolor{currentstroke}{rgb}{0.121569,0.466667,0.705882}%
\pgfsetstrokecolor{currentstroke}%
\pgfsetstrokeopacity{0.944658}%
\pgfsetdash{}{0pt}%
\pgfpathmoveto{\pgfqpoint{2.373013in}{1.397728in}}%
\pgfpathcurveto{\pgfqpoint{2.381249in}{1.397728in}}{\pgfqpoint{2.389149in}{1.401000in}}{\pgfqpoint{2.394973in}{1.406824in}}%
\pgfpathcurveto{\pgfqpoint{2.400797in}{1.412648in}}{\pgfqpoint{2.404069in}{1.420548in}}{\pgfqpoint{2.404069in}{1.428785in}}%
\pgfpathcurveto{\pgfqpoint{2.404069in}{1.437021in}}{\pgfqpoint{2.400797in}{1.444921in}}{\pgfqpoint{2.394973in}{1.450745in}}%
\pgfpathcurveto{\pgfqpoint{2.389149in}{1.456569in}}{\pgfqpoint{2.381249in}{1.459841in}}{\pgfqpoint{2.373013in}{1.459841in}}%
\pgfpathcurveto{\pgfqpoint{2.364776in}{1.459841in}}{\pgfqpoint{2.356876in}{1.456569in}}{\pgfqpoint{2.351052in}{1.450745in}}%
\pgfpathcurveto{\pgfqpoint{2.345228in}{1.444921in}}{\pgfqpoint{2.341956in}{1.437021in}}{\pgfqpoint{2.341956in}{1.428785in}}%
\pgfpathcurveto{\pgfqpoint{2.341956in}{1.420548in}}{\pgfqpoint{2.345228in}{1.412648in}}{\pgfqpoint{2.351052in}{1.406824in}}%
\pgfpathcurveto{\pgfqpoint{2.356876in}{1.401000in}}{\pgfqpoint{2.364776in}{1.397728in}}{\pgfqpoint{2.373013in}{1.397728in}}%
\pgfpathclose%
\pgfusepath{stroke,fill}%
\end{pgfscope}%
\begin{pgfscope}%
\pgfpathrectangle{\pgfqpoint{0.100000in}{0.220728in}}{\pgfqpoint{3.696000in}{3.696000in}}%
\pgfusepath{clip}%
\pgfsetbuttcap%
\pgfsetroundjoin%
\definecolor{currentfill}{rgb}{0.121569,0.466667,0.705882}%
\pgfsetfillcolor{currentfill}%
\pgfsetfillopacity{0.946181}%
\pgfsetlinewidth{1.003750pt}%
\definecolor{currentstroke}{rgb}{0.121569,0.466667,0.705882}%
\pgfsetstrokecolor{currentstroke}%
\pgfsetstrokeopacity{0.946181}%
\pgfsetdash{}{0pt}%
\pgfpathmoveto{\pgfqpoint{1.874279in}{1.668149in}}%
\pgfpathcurveto{\pgfqpoint{1.882515in}{1.668149in}}{\pgfqpoint{1.890415in}{1.671421in}}{\pgfqpoint{1.896239in}{1.677245in}}%
\pgfpathcurveto{\pgfqpoint{1.902063in}{1.683069in}}{\pgfqpoint{1.905335in}{1.690969in}}{\pgfqpoint{1.905335in}{1.699205in}}%
\pgfpathcurveto{\pgfqpoint{1.905335in}{1.707442in}}{\pgfqpoint{1.902063in}{1.715342in}}{\pgfqpoint{1.896239in}{1.721166in}}%
\pgfpathcurveto{\pgfqpoint{1.890415in}{1.726990in}}{\pgfqpoint{1.882515in}{1.730262in}}{\pgfqpoint{1.874279in}{1.730262in}}%
\pgfpathcurveto{\pgfqpoint{1.866042in}{1.730262in}}{\pgfqpoint{1.858142in}{1.726990in}}{\pgfqpoint{1.852318in}{1.721166in}}%
\pgfpathcurveto{\pgfqpoint{1.846494in}{1.715342in}}{\pgfqpoint{1.843222in}{1.707442in}}{\pgfqpoint{1.843222in}{1.699205in}}%
\pgfpathcurveto{\pgfqpoint{1.843222in}{1.690969in}}{\pgfqpoint{1.846494in}{1.683069in}}{\pgfqpoint{1.852318in}{1.677245in}}%
\pgfpathcurveto{\pgfqpoint{1.858142in}{1.671421in}}{\pgfqpoint{1.866042in}{1.668149in}}{\pgfqpoint{1.874279in}{1.668149in}}%
\pgfpathclose%
\pgfusepath{stroke,fill}%
\end{pgfscope}%
\begin{pgfscope}%
\pgfpathrectangle{\pgfqpoint{0.100000in}{0.220728in}}{\pgfqpoint{3.696000in}{3.696000in}}%
\pgfusepath{clip}%
\pgfsetbuttcap%
\pgfsetroundjoin%
\definecolor{currentfill}{rgb}{0.121569,0.466667,0.705882}%
\pgfsetfillcolor{currentfill}%
\pgfsetfillopacity{0.949818}%
\pgfsetlinewidth{1.003750pt}%
\definecolor{currentstroke}{rgb}{0.121569,0.466667,0.705882}%
\pgfsetstrokecolor{currentstroke}%
\pgfsetstrokeopacity{0.949818}%
\pgfsetdash{}{0pt}%
\pgfpathmoveto{\pgfqpoint{2.376083in}{1.386936in}}%
\pgfpathcurveto{\pgfqpoint{2.384319in}{1.386936in}}{\pgfqpoint{2.392219in}{1.390208in}}{\pgfqpoint{2.398043in}{1.396032in}}%
\pgfpathcurveto{\pgfqpoint{2.403867in}{1.401856in}}{\pgfqpoint{2.407139in}{1.409756in}}{\pgfqpoint{2.407139in}{1.417992in}}%
\pgfpathcurveto{\pgfqpoint{2.407139in}{1.426228in}}{\pgfqpoint{2.403867in}{1.434128in}}{\pgfqpoint{2.398043in}{1.439952in}}%
\pgfpathcurveto{\pgfqpoint{2.392219in}{1.445776in}}{\pgfqpoint{2.384319in}{1.449049in}}{\pgfqpoint{2.376083in}{1.449049in}}%
\pgfpathcurveto{\pgfqpoint{2.367846in}{1.449049in}}{\pgfqpoint{2.359946in}{1.445776in}}{\pgfqpoint{2.354122in}{1.439952in}}%
\pgfpathcurveto{\pgfqpoint{2.348298in}{1.434128in}}{\pgfqpoint{2.345026in}{1.426228in}}{\pgfqpoint{2.345026in}{1.417992in}}%
\pgfpathcurveto{\pgfqpoint{2.345026in}{1.409756in}}{\pgfqpoint{2.348298in}{1.401856in}}{\pgfqpoint{2.354122in}{1.396032in}}%
\pgfpathcurveto{\pgfqpoint{2.359946in}{1.390208in}}{\pgfqpoint{2.367846in}{1.386936in}}{\pgfqpoint{2.376083in}{1.386936in}}%
\pgfpathclose%
\pgfusepath{stroke,fill}%
\end{pgfscope}%
\begin{pgfscope}%
\pgfpathrectangle{\pgfqpoint{0.100000in}{0.220728in}}{\pgfqpoint{3.696000in}{3.696000in}}%
\pgfusepath{clip}%
\pgfsetbuttcap%
\pgfsetroundjoin%
\definecolor{currentfill}{rgb}{0.121569,0.466667,0.705882}%
\pgfsetfillcolor{currentfill}%
\pgfsetfillopacity{0.950228}%
\pgfsetlinewidth{1.003750pt}%
\definecolor{currentstroke}{rgb}{0.121569,0.466667,0.705882}%
\pgfsetstrokecolor{currentstroke}%
\pgfsetstrokeopacity{0.950228}%
\pgfsetdash{}{0pt}%
\pgfpathmoveto{\pgfqpoint{1.903096in}{1.653573in}}%
\pgfpathcurveto{\pgfqpoint{1.911332in}{1.653573in}}{\pgfqpoint{1.919232in}{1.656845in}}{\pgfqpoint{1.925056in}{1.662669in}}%
\pgfpathcurveto{\pgfqpoint{1.930880in}{1.668493in}}{\pgfqpoint{1.934153in}{1.676393in}}{\pgfqpoint{1.934153in}{1.684629in}}%
\pgfpathcurveto{\pgfqpoint{1.934153in}{1.692866in}}{\pgfqpoint{1.930880in}{1.700766in}}{\pgfqpoint{1.925056in}{1.706590in}}%
\pgfpathcurveto{\pgfqpoint{1.919232in}{1.712414in}}{\pgfqpoint{1.911332in}{1.715686in}}{\pgfqpoint{1.903096in}{1.715686in}}%
\pgfpathcurveto{\pgfqpoint{1.894860in}{1.715686in}}{\pgfqpoint{1.886960in}{1.712414in}}{\pgfqpoint{1.881136in}{1.706590in}}%
\pgfpathcurveto{\pgfqpoint{1.875312in}{1.700766in}}{\pgfqpoint{1.872040in}{1.692866in}}{\pgfqpoint{1.872040in}{1.684629in}}%
\pgfpathcurveto{\pgfqpoint{1.872040in}{1.676393in}}{\pgfqpoint{1.875312in}{1.668493in}}{\pgfqpoint{1.881136in}{1.662669in}}%
\pgfpathcurveto{\pgfqpoint{1.886960in}{1.656845in}}{\pgfqpoint{1.894860in}{1.653573in}}{\pgfqpoint{1.903096in}{1.653573in}}%
\pgfpathclose%
\pgfusepath{stroke,fill}%
\end{pgfscope}%
\begin{pgfscope}%
\pgfpathrectangle{\pgfqpoint{0.100000in}{0.220728in}}{\pgfqpoint{3.696000in}{3.696000in}}%
\pgfusepath{clip}%
\pgfsetbuttcap%
\pgfsetroundjoin%
\definecolor{currentfill}{rgb}{0.121569,0.466667,0.705882}%
\pgfsetfillcolor{currentfill}%
\pgfsetfillopacity{0.953008}%
\pgfsetlinewidth{1.003750pt}%
\definecolor{currentstroke}{rgb}{0.121569,0.466667,0.705882}%
\pgfsetstrokecolor{currentstroke}%
\pgfsetstrokeopacity{0.953008}%
\pgfsetdash{}{0pt}%
\pgfpathmoveto{\pgfqpoint{1.930932in}{1.639197in}}%
\pgfpathcurveto{\pgfqpoint{1.939169in}{1.639197in}}{\pgfqpoint{1.947069in}{1.642469in}}{\pgfqpoint{1.952892in}{1.648293in}}%
\pgfpathcurveto{\pgfqpoint{1.958716in}{1.654117in}}{\pgfqpoint{1.961989in}{1.662017in}}{\pgfqpoint{1.961989in}{1.670253in}}%
\pgfpathcurveto{\pgfqpoint{1.961989in}{1.678490in}}{\pgfqpoint{1.958716in}{1.686390in}}{\pgfqpoint{1.952892in}{1.692214in}}%
\pgfpathcurveto{\pgfqpoint{1.947069in}{1.698038in}}{\pgfqpoint{1.939169in}{1.701310in}}{\pgfqpoint{1.930932in}{1.701310in}}%
\pgfpathcurveto{\pgfqpoint{1.922696in}{1.701310in}}{\pgfqpoint{1.914796in}{1.698038in}}{\pgfqpoint{1.908972in}{1.692214in}}%
\pgfpathcurveto{\pgfqpoint{1.903148in}{1.686390in}}{\pgfqpoint{1.899876in}{1.678490in}}{\pgfqpoint{1.899876in}{1.670253in}}%
\pgfpathcurveto{\pgfqpoint{1.899876in}{1.662017in}}{\pgfqpoint{1.903148in}{1.654117in}}{\pgfqpoint{1.908972in}{1.648293in}}%
\pgfpathcurveto{\pgfqpoint{1.914796in}{1.642469in}}{\pgfqpoint{1.922696in}{1.639197in}}{\pgfqpoint{1.930932in}{1.639197in}}%
\pgfpathclose%
\pgfusepath{stroke,fill}%
\end{pgfscope}%
\begin{pgfscope}%
\pgfpathrectangle{\pgfqpoint{0.100000in}{0.220728in}}{\pgfqpoint{3.696000in}{3.696000in}}%
\pgfusepath{clip}%
\pgfsetbuttcap%
\pgfsetroundjoin%
\definecolor{currentfill}{rgb}{0.121569,0.466667,0.705882}%
\pgfsetfillcolor{currentfill}%
\pgfsetfillopacity{0.954108}%
\pgfsetlinewidth{1.003750pt}%
\definecolor{currentstroke}{rgb}{0.121569,0.466667,0.705882}%
\pgfsetstrokecolor{currentstroke}%
\pgfsetstrokeopacity{0.954108}%
\pgfsetdash{}{0pt}%
\pgfpathmoveto{\pgfqpoint{1.954393in}{1.613217in}}%
\pgfpathcurveto{\pgfqpoint{1.962629in}{1.613217in}}{\pgfqpoint{1.970529in}{1.616489in}}{\pgfqpoint{1.976353in}{1.622313in}}%
\pgfpathcurveto{\pgfqpoint{1.982177in}{1.628137in}}{\pgfqpoint{1.985449in}{1.636037in}}{\pgfqpoint{1.985449in}{1.644273in}}%
\pgfpathcurveto{\pgfqpoint{1.985449in}{1.652509in}}{\pgfqpoint{1.982177in}{1.660409in}}{\pgfqpoint{1.976353in}{1.666233in}}%
\pgfpathcurveto{\pgfqpoint{1.970529in}{1.672057in}}{\pgfqpoint{1.962629in}{1.675330in}}{\pgfqpoint{1.954393in}{1.675330in}}%
\pgfpathcurveto{\pgfqpoint{1.946156in}{1.675330in}}{\pgfqpoint{1.938256in}{1.672057in}}{\pgfqpoint{1.932432in}{1.666233in}}%
\pgfpathcurveto{\pgfqpoint{1.926609in}{1.660409in}}{\pgfqpoint{1.923336in}{1.652509in}}{\pgfqpoint{1.923336in}{1.644273in}}%
\pgfpathcurveto{\pgfqpoint{1.923336in}{1.636037in}}{\pgfqpoint{1.926609in}{1.628137in}}{\pgfqpoint{1.932432in}{1.622313in}}%
\pgfpathcurveto{\pgfqpoint{1.938256in}{1.616489in}}{\pgfqpoint{1.946156in}{1.613217in}}{\pgfqpoint{1.954393in}{1.613217in}}%
\pgfpathclose%
\pgfusepath{stroke,fill}%
\end{pgfscope}%
\begin{pgfscope}%
\pgfpathrectangle{\pgfqpoint{0.100000in}{0.220728in}}{\pgfqpoint{3.696000in}{3.696000in}}%
\pgfusepath{clip}%
\pgfsetbuttcap%
\pgfsetroundjoin%
\definecolor{currentfill}{rgb}{0.121569,0.466667,0.705882}%
\pgfsetfillcolor{currentfill}%
\pgfsetfillopacity{0.955449}%
\pgfsetlinewidth{1.003750pt}%
\definecolor{currentstroke}{rgb}{0.121569,0.466667,0.705882}%
\pgfsetstrokecolor{currentstroke}%
\pgfsetstrokeopacity{0.955449}%
\pgfsetdash{}{0pt}%
\pgfpathmoveto{\pgfqpoint{2.380541in}{1.377117in}}%
\pgfpathcurveto{\pgfqpoint{2.388777in}{1.377117in}}{\pgfqpoint{2.396677in}{1.380389in}}{\pgfqpoint{2.402501in}{1.386213in}}%
\pgfpathcurveto{\pgfqpoint{2.408325in}{1.392037in}}{\pgfqpoint{2.411597in}{1.399937in}}{\pgfqpoint{2.411597in}{1.408174in}}%
\pgfpathcurveto{\pgfqpoint{2.411597in}{1.416410in}}{\pgfqpoint{2.408325in}{1.424310in}}{\pgfqpoint{2.402501in}{1.430134in}}%
\pgfpathcurveto{\pgfqpoint{2.396677in}{1.435958in}}{\pgfqpoint{2.388777in}{1.439230in}}{\pgfqpoint{2.380541in}{1.439230in}}%
\pgfpathcurveto{\pgfqpoint{2.372304in}{1.439230in}}{\pgfqpoint{2.364404in}{1.435958in}}{\pgfqpoint{2.358580in}{1.430134in}}%
\pgfpathcurveto{\pgfqpoint{2.352756in}{1.424310in}}{\pgfqpoint{2.349484in}{1.416410in}}{\pgfqpoint{2.349484in}{1.408174in}}%
\pgfpathcurveto{\pgfqpoint{2.349484in}{1.399937in}}{\pgfqpoint{2.352756in}{1.392037in}}{\pgfqpoint{2.358580in}{1.386213in}}%
\pgfpathcurveto{\pgfqpoint{2.364404in}{1.380389in}}{\pgfqpoint{2.372304in}{1.377117in}}{\pgfqpoint{2.380541in}{1.377117in}}%
\pgfpathclose%
\pgfusepath{stroke,fill}%
\end{pgfscope}%
\begin{pgfscope}%
\pgfpathrectangle{\pgfqpoint{0.100000in}{0.220728in}}{\pgfqpoint{3.696000in}{3.696000in}}%
\pgfusepath{clip}%
\pgfsetbuttcap%
\pgfsetroundjoin%
\definecolor{currentfill}{rgb}{0.121569,0.466667,0.705882}%
\pgfsetfillcolor{currentfill}%
\pgfsetfillopacity{0.955762}%
\pgfsetlinewidth{1.003750pt}%
\definecolor{currentstroke}{rgb}{0.121569,0.466667,0.705882}%
\pgfsetstrokecolor{currentstroke}%
\pgfsetstrokeopacity{0.955762}%
\pgfsetdash{}{0pt}%
\pgfpathmoveto{\pgfqpoint{1.974749in}{1.601860in}}%
\pgfpathcurveto{\pgfqpoint{1.982986in}{1.601860in}}{\pgfqpoint{1.990886in}{1.605133in}}{\pgfqpoint{1.996710in}{1.610957in}}%
\pgfpathcurveto{\pgfqpoint{2.002533in}{1.616781in}}{\pgfqpoint{2.005806in}{1.624681in}}{\pgfqpoint{2.005806in}{1.632917in}}%
\pgfpathcurveto{\pgfqpoint{2.005806in}{1.641153in}}{\pgfqpoint{2.002533in}{1.649053in}}{\pgfqpoint{1.996710in}{1.654877in}}%
\pgfpathcurveto{\pgfqpoint{1.990886in}{1.660701in}}{\pgfqpoint{1.982986in}{1.663973in}}{\pgfqpoint{1.974749in}{1.663973in}}%
\pgfpathcurveto{\pgfqpoint{1.966513in}{1.663973in}}{\pgfqpoint{1.958613in}{1.660701in}}{\pgfqpoint{1.952789in}{1.654877in}}%
\pgfpathcurveto{\pgfqpoint{1.946965in}{1.649053in}}{\pgfqpoint{1.943693in}{1.641153in}}{\pgfqpoint{1.943693in}{1.632917in}}%
\pgfpathcurveto{\pgfqpoint{1.943693in}{1.624681in}}{\pgfqpoint{1.946965in}{1.616781in}}{\pgfqpoint{1.952789in}{1.610957in}}%
\pgfpathcurveto{\pgfqpoint{1.958613in}{1.605133in}}{\pgfqpoint{1.966513in}{1.601860in}}{\pgfqpoint{1.974749in}{1.601860in}}%
\pgfpathclose%
\pgfusepath{stroke,fill}%
\end{pgfscope}%
\begin{pgfscope}%
\pgfpathrectangle{\pgfqpoint{0.100000in}{0.220728in}}{\pgfqpoint{3.696000in}{3.696000in}}%
\pgfusepath{clip}%
\pgfsetbuttcap%
\pgfsetroundjoin%
\definecolor{currentfill}{rgb}{0.121569,0.466667,0.705882}%
\pgfsetfillcolor{currentfill}%
\pgfsetfillopacity{0.957715}%
\pgfsetlinewidth{1.003750pt}%
\definecolor{currentstroke}{rgb}{0.121569,0.466667,0.705882}%
\pgfsetstrokecolor{currentstroke}%
\pgfsetstrokeopacity{0.957715}%
\pgfsetdash{}{0pt}%
\pgfpathmoveto{\pgfqpoint{1.992573in}{1.592276in}}%
\pgfpathcurveto{\pgfqpoint{2.000810in}{1.592276in}}{\pgfqpoint{2.008710in}{1.595548in}}{\pgfqpoint{2.014534in}{1.601372in}}%
\pgfpathcurveto{\pgfqpoint{2.020358in}{1.607196in}}{\pgfqpoint{2.023630in}{1.615096in}}{\pgfqpoint{2.023630in}{1.623333in}}%
\pgfpathcurveto{\pgfqpoint{2.023630in}{1.631569in}}{\pgfqpoint{2.020358in}{1.639469in}}{\pgfqpoint{2.014534in}{1.645293in}}%
\pgfpathcurveto{\pgfqpoint{2.008710in}{1.651117in}}{\pgfqpoint{2.000810in}{1.654389in}}{\pgfqpoint{1.992573in}{1.654389in}}%
\pgfpathcurveto{\pgfqpoint{1.984337in}{1.654389in}}{\pgfqpoint{1.976437in}{1.651117in}}{\pgfqpoint{1.970613in}{1.645293in}}%
\pgfpathcurveto{\pgfqpoint{1.964789in}{1.639469in}}{\pgfqpoint{1.961517in}{1.631569in}}{\pgfqpoint{1.961517in}{1.623333in}}%
\pgfpathcurveto{\pgfqpoint{1.961517in}{1.615096in}}{\pgfqpoint{1.964789in}{1.607196in}}{\pgfqpoint{1.970613in}{1.601372in}}%
\pgfpathcurveto{\pgfqpoint{1.976437in}{1.595548in}}{\pgfqpoint{1.984337in}{1.592276in}}{\pgfqpoint{1.992573in}{1.592276in}}%
\pgfpathclose%
\pgfusepath{stroke,fill}%
\end{pgfscope}%
\begin{pgfscope}%
\pgfpathrectangle{\pgfqpoint{0.100000in}{0.220728in}}{\pgfqpoint{3.696000in}{3.696000in}}%
\pgfusepath{clip}%
\pgfsetbuttcap%
\pgfsetroundjoin%
\definecolor{currentfill}{rgb}{0.121569,0.466667,0.705882}%
\pgfsetfillcolor{currentfill}%
\pgfsetfillopacity{0.958750}%
\pgfsetlinewidth{1.003750pt}%
\definecolor{currentstroke}{rgb}{0.121569,0.466667,0.705882}%
\pgfsetstrokecolor{currentstroke}%
\pgfsetstrokeopacity{0.958750}%
\pgfsetdash{}{0pt}%
\pgfpathmoveto{\pgfqpoint{2.009034in}{1.580882in}}%
\pgfpathcurveto{\pgfqpoint{2.017270in}{1.580882in}}{\pgfqpoint{2.025170in}{1.584154in}}{\pgfqpoint{2.030994in}{1.589978in}}%
\pgfpathcurveto{\pgfqpoint{2.036818in}{1.595802in}}{\pgfqpoint{2.040091in}{1.603702in}}{\pgfqpoint{2.040091in}{1.611938in}}%
\pgfpathcurveto{\pgfqpoint{2.040091in}{1.620175in}}{\pgfqpoint{2.036818in}{1.628075in}}{\pgfqpoint{2.030994in}{1.633899in}}%
\pgfpathcurveto{\pgfqpoint{2.025170in}{1.639723in}}{\pgfqpoint{2.017270in}{1.642995in}}{\pgfqpoint{2.009034in}{1.642995in}}%
\pgfpathcurveto{\pgfqpoint{2.000798in}{1.642995in}}{\pgfqpoint{1.992898in}{1.639723in}}{\pgfqpoint{1.987074in}{1.633899in}}%
\pgfpathcurveto{\pgfqpoint{1.981250in}{1.628075in}}{\pgfqpoint{1.977978in}{1.620175in}}{\pgfqpoint{1.977978in}{1.611938in}}%
\pgfpathcurveto{\pgfqpoint{1.977978in}{1.603702in}}{\pgfqpoint{1.981250in}{1.595802in}}{\pgfqpoint{1.987074in}{1.589978in}}%
\pgfpathcurveto{\pgfqpoint{1.992898in}{1.584154in}}{\pgfqpoint{2.000798in}{1.580882in}}{\pgfqpoint{2.009034in}{1.580882in}}%
\pgfpathclose%
\pgfusepath{stroke,fill}%
\end{pgfscope}%
\begin{pgfscope}%
\pgfpathrectangle{\pgfqpoint{0.100000in}{0.220728in}}{\pgfqpoint{3.696000in}{3.696000in}}%
\pgfusepath{clip}%
\pgfsetbuttcap%
\pgfsetroundjoin%
\definecolor{currentfill}{rgb}{0.121569,0.466667,0.705882}%
\pgfsetfillcolor{currentfill}%
\pgfsetfillopacity{0.959863}%
\pgfsetlinewidth{1.003750pt}%
\definecolor{currentstroke}{rgb}{0.121569,0.466667,0.705882}%
\pgfsetstrokecolor{currentstroke}%
\pgfsetstrokeopacity{0.959863}%
\pgfsetdash{}{0pt}%
\pgfpathmoveto{\pgfqpoint{2.022758in}{1.570830in}}%
\pgfpathcurveto{\pgfqpoint{2.030995in}{1.570830in}}{\pgfqpoint{2.038895in}{1.574103in}}{\pgfqpoint{2.044719in}{1.579926in}}%
\pgfpathcurveto{\pgfqpoint{2.050543in}{1.585750in}}{\pgfqpoint{2.053815in}{1.593650in}}{\pgfqpoint{2.053815in}{1.601887in}}%
\pgfpathcurveto{\pgfqpoint{2.053815in}{1.610123in}}{\pgfqpoint{2.050543in}{1.618023in}}{\pgfqpoint{2.044719in}{1.623847in}}%
\pgfpathcurveto{\pgfqpoint{2.038895in}{1.629671in}}{\pgfqpoint{2.030995in}{1.632943in}}{\pgfqpoint{2.022758in}{1.632943in}}%
\pgfpathcurveto{\pgfqpoint{2.014522in}{1.632943in}}{\pgfqpoint{2.006622in}{1.629671in}}{\pgfqpoint{2.000798in}{1.623847in}}%
\pgfpathcurveto{\pgfqpoint{1.994974in}{1.618023in}}{\pgfqpoint{1.991702in}{1.610123in}}{\pgfqpoint{1.991702in}{1.601887in}}%
\pgfpathcurveto{\pgfqpoint{1.991702in}{1.593650in}}{\pgfqpoint{1.994974in}{1.585750in}}{\pgfqpoint{2.000798in}{1.579926in}}%
\pgfpathcurveto{\pgfqpoint{2.006622in}{1.574103in}}{\pgfqpoint{2.014522in}{1.570830in}}{\pgfqpoint{2.022758in}{1.570830in}}%
\pgfpathclose%
\pgfusepath{stroke,fill}%
\end{pgfscope}%
\begin{pgfscope}%
\pgfpathrectangle{\pgfqpoint{0.100000in}{0.220728in}}{\pgfqpoint{3.696000in}{3.696000in}}%
\pgfusepath{clip}%
\pgfsetbuttcap%
\pgfsetroundjoin%
\definecolor{currentfill}{rgb}{0.121569,0.466667,0.705882}%
\pgfsetfillcolor{currentfill}%
\pgfsetfillopacity{0.960602}%
\pgfsetlinewidth{1.003750pt}%
\definecolor{currentstroke}{rgb}{0.121569,0.466667,0.705882}%
\pgfsetstrokecolor{currentstroke}%
\pgfsetstrokeopacity{0.960602}%
\pgfsetdash{}{0pt}%
\pgfpathmoveto{\pgfqpoint{2.385647in}{1.362599in}}%
\pgfpathcurveto{\pgfqpoint{2.393883in}{1.362599in}}{\pgfqpoint{2.401783in}{1.365871in}}{\pgfqpoint{2.407607in}{1.371695in}}%
\pgfpathcurveto{\pgfqpoint{2.413431in}{1.377519in}}{\pgfqpoint{2.416704in}{1.385419in}}{\pgfqpoint{2.416704in}{1.393655in}}%
\pgfpathcurveto{\pgfqpoint{2.416704in}{1.401892in}}{\pgfqpoint{2.413431in}{1.409792in}}{\pgfqpoint{2.407607in}{1.415616in}}%
\pgfpathcurveto{\pgfqpoint{2.401783in}{1.421439in}}{\pgfqpoint{2.393883in}{1.424712in}}{\pgfqpoint{2.385647in}{1.424712in}}%
\pgfpathcurveto{\pgfqpoint{2.377411in}{1.424712in}}{\pgfqpoint{2.369511in}{1.421439in}}{\pgfqpoint{2.363687in}{1.415616in}}%
\pgfpathcurveto{\pgfqpoint{2.357863in}{1.409792in}}{\pgfqpoint{2.354591in}{1.401892in}}{\pgfqpoint{2.354591in}{1.393655in}}%
\pgfpathcurveto{\pgfqpoint{2.354591in}{1.385419in}}{\pgfqpoint{2.357863in}{1.377519in}}{\pgfqpoint{2.363687in}{1.371695in}}%
\pgfpathcurveto{\pgfqpoint{2.369511in}{1.365871in}}{\pgfqpoint{2.377411in}{1.362599in}}{\pgfqpoint{2.385647in}{1.362599in}}%
\pgfpathclose%
\pgfusepath{stroke,fill}%
\end{pgfscope}%
\begin{pgfscope}%
\pgfpathrectangle{\pgfqpoint{0.100000in}{0.220728in}}{\pgfqpoint{3.696000in}{3.696000in}}%
\pgfusepath{clip}%
\pgfsetbuttcap%
\pgfsetroundjoin%
\definecolor{currentfill}{rgb}{0.121569,0.466667,0.705882}%
\pgfsetfillcolor{currentfill}%
\pgfsetfillopacity{0.961343}%
\pgfsetlinewidth{1.003750pt}%
\definecolor{currentstroke}{rgb}{0.121569,0.466667,0.705882}%
\pgfsetstrokecolor{currentstroke}%
\pgfsetstrokeopacity{0.961343}%
\pgfsetdash{}{0pt}%
\pgfpathmoveto{\pgfqpoint{2.035031in}{1.561865in}}%
\pgfpathcurveto{\pgfqpoint{2.043267in}{1.561865in}}{\pgfqpoint{2.051167in}{1.565137in}}{\pgfqpoint{2.056991in}{1.570961in}}%
\pgfpathcurveto{\pgfqpoint{2.062815in}{1.576785in}}{\pgfqpoint{2.066087in}{1.584685in}}{\pgfqpoint{2.066087in}{1.592921in}}%
\pgfpathcurveto{\pgfqpoint{2.066087in}{1.601158in}}{\pgfqpoint{2.062815in}{1.609058in}}{\pgfqpoint{2.056991in}{1.614882in}}%
\pgfpathcurveto{\pgfqpoint{2.051167in}{1.620706in}}{\pgfqpoint{2.043267in}{1.623978in}}{\pgfqpoint{2.035031in}{1.623978in}}%
\pgfpathcurveto{\pgfqpoint{2.026794in}{1.623978in}}{\pgfqpoint{2.018894in}{1.620706in}}{\pgfqpoint{2.013070in}{1.614882in}}%
\pgfpathcurveto{\pgfqpoint{2.007247in}{1.609058in}}{\pgfqpoint{2.003974in}{1.601158in}}{\pgfqpoint{2.003974in}{1.592921in}}%
\pgfpathcurveto{\pgfqpoint{2.003974in}{1.584685in}}{\pgfqpoint{2.007247in}{1.576785in}}{\pgfqpoint{2.013070in}{1.570961in}}%
\pgfpathcurveto{\pgfqpoint{2.018894in}{1.565137in}}{\pgfqpoint{2.026794in}{1.561865in}}{\pgfqpoint{2.035031in}{1.561865in}}%
\pgfpathclose%
\pgfusepath{stroke,fill}%
\end{pgfscope}%
\begin{pgfscope}%
\pgfpathrectangle{\pgfqpoint{0.100000in}{0.220728in}}{\pgfqpoint{3.696000in}{3.696000in}}%
\pgfusepath{clip}%
\pgfsetbuttcap%
\pgfsetroundjoin%
\definecolor{currentfill}{rgb}{0.121569,0.466667,0.705882}%
\pgfsetfillcolor{currentfill}%
\pgfsetfillopacity{0.962019}%
\pgfsetlinewidth{1.003750pt}%
\definecolor{currentstroke}{rgb}{0.121569,0.466667,0.705882}%
\pgfsetstrokecolor{currentstroke}%
\pgfsetstrokeopacity{0.962019}%
\pgfsetdash{}{0pt}%
\pgfpathmoveto{\pgfqpoint{2.045275in}{1.554107in}}%
\pgfpathcurveto{\pgfqpoint{2.053511in}{1.554107in}}{\pgfqpoint{2.061411in}{1.557380in}}{\pgfqpoint{2.067235in}{1.563203in}}%
\pgfpathcurveto{\pgfqpoint{2.073059in}{1.569027in}}{\pgfqpoint{2.076331in}{1.576927in}}{\pgfqpoint{2.076331in}{1.585164in}}%
\pgfpathcurveto{\pgfqpoint{2.076331in}{1.593400in}}{\pgfqpoint{2.073059in}{1.601300in}}{\pgfqpoint{2.067235in}{1.607124in}}%
\pgfpathcurveto{\pgfqpoint{2.061411in}{1.612948in}}{\pgfqpoint{2.053511in}{1.616220in}}{\pgfqpoint{2.045275in}{1.616220in}}%
\pgfpathcurveto{\pgfqpoint{2.037038in}{1.616220in}}{\pgfqpoint{2.029138in}{1.612948in}}{\pgfqpoint{2.023314in}{1.607124in}}%
\pgfpathcurveto{\pgfqpoint{2.017490in}{1.601300in}}{\pgfqpoint{2.014218in}{1.593400in}}{\pgfqpoint{2.014218in}{1.585164in}}%
\pgfpathcurveto{\pgfqpoint{2.014218in}{1.576927in}}{\pgfqpoint{2.017490in}{1.569027in}}{\pgfqpoint{2.023314in}{1.563203in}}%
\pgfpathcurveto{\pgfqpoint{2.029138in}{1.557380in}}{\pgfqpoint{2.037038in}{1.554107in}}{\pgfqpoint{2.045275in}{1.554107in}}%
\pgfpathclose%
\pgfusepath{stroke,fill}%
\end{pgfscope}%
\begin{pgfscope}%
\pgfpathrectangle{\pgfqpoint{0.100000in}{0.220728in}}{\pgfqpoint{3.696000in}{3.696000in}}%
\pgfusepath{clip}%
\pgfsetbuttcap%
\pgfsetroundjoin%
\definecolor{currentfill}{rgb}{0.121569,0.466667,0.705882}%
\pgfsetfillcolor{currentfill}%
\pgfsetfillopacity{0.963072}%
\pgfsetlinewidth{1.003750pt}%
\definecolor{currentstroke}{rgb}{0.121569,0.466667,0.705882}%
\pgfsetstrokecolor{currentstroke}%
\pgfsetstrokeopacity{0.963072}%
\pgfsetdash{}{0pt}%
\pgfpathmoveto{\pgfqpoint{2.053984in}{1.549409in}}%
\pgfpathcurveto{\pgfqpoint{2.062220in}{1.549409in}}{\pgfqpoint{2.070121in}{1.552681in}}{\pgfqpoint{2.075944in}{1.558505in}}%
\pgfpathcurveto{\pgfqpoint{2.081768in}{1.564329in}}{\pgfqpoint{2.085041in}{1.572229in}}{\pgfqpoint{2.085041in}{1.580465in}}%
\pgfpathcurveto{\pgfqpoint{2.085041in}{1.588702in}}{\pgfqpoint{2.081768in}{1.596602in}}{\pgfqpoint{2.075944in}{1.602426in}}%
\pgfpathcurveto{\pgfqpoint{2.070121in}{1.608249in}}{\pgfqpoint{2.062220in}{1.611522in}}{\pgfqpoint{2.053984in}{1.611522in}}%
\pgfpathcurveto{\pgfqpoint{2.045748in}{1.611522in}}{\pgfqpoint{2.037848in}{1.608249in}}{\pgfqpoint{2.032024in}{1.602426in}}%
\pgfpathcurveto{\pgfqpoint{2.026200in}{1.596602in}}{\pgfqpoint{2.022928in}{1.588702in}}{\pgfqpoint{2.022928in}{1.580465in}}%
\pgfpathcurveto{\pgfqpoint{2.022928in}{1.572229in}}{\pgfqpoint{2.026200in}{1.564329in}}{\pgfqpoint{2.032024in}{1.558505in}}%
\pgfpathcurveto{\pgfqpoint{2.037848in}{1.552681in}}{\pgfqpoint{2.045748in}{1.549409in}}{\pgfqpoint{2.053984in}{1.549409in}}%
\pgfpathclose%
\pgfusepath{stroke,fill}%
\end{pgfscope}%
\begin{pgfscope}%
\pgfpathrectangle{\pgfqpoint{0.100000in}{0.220728in}}{\pgfqpoint{3.696000in}{3.696000in}}%
\pgfusepath{clip}%
\pgfsetbuttcap%
\pgfsetroundjoin%
\definecolor{currentfill}{rgb}{0.121569,0.466667,0.705882}%
\pgfsetfillcolor{currentfill}%
\pgfsetfillopacity{0.963535}%
\pgfsetlinewidth{1.003750pt}%
\definecolor{currentstroke}{rgb}{0.121569,0.466667,0.705882}%
\pgfsetstrokecolor{currentstroke}%
\pgfsetstrokeopacity{0.963535}%
\pgfsetdash{}{0pt}%
\pgfpathmoveto{\pgfqpoint{2.060408in}{1.543763in}}%
\pgfpathcurveto{\pgfqpoint{2.068644in}{1.543763in}}{\pgfqpoint{2.076545in}{1.547036in}}{\pgfqpoint{2.082368in}{1.552859in}}%
\pgfpathcurveto{\pgfqpoint{2.088192in}{1.558683in}}{\pgfqpoint{2.091465in}{1.566583in}}{\pgfqpoint{2.091465in}{1.574820in}}%
\pgfpathcurveto{\pgfqpoint{2.091465in}{1.583056in}}{\pgfqpoint{2.088192in}{1.590956in}}{\pgfqpoint{2.082368in}{1.596780in}}%
\pgfpathcurveto{\pgfqpoint{2.076545in}{1.602604in}}{\pgfqpoint{2.068644in}{1.605876in}}{\pgfqpoint{2.060408in}{1.605876in}}%
\pgfpathcurveto{\pgfqpoint{2.052172in}{1.605876in}}{\pgfqpoint{2.044272in}{1.602604in}}{\pgfqpoint{2.038448in}{1.596780in}}%
\pgfpathcurveto{\pgfqpoint{2.032624in}{1.590956in}}{\pgfqpoint{2.029352in}{1.583056in}}{\pgfqpoint{2.029352in}{1.574820in}}%
\pgfpathcurveto{\pgfqpoint{2.029352in}{1.566583in}}{\pgfqpoint{2.032624in}{1.558683in}}{\pgfqpoint{2.038448in}{1.552859in}}%
\pgfpathcurveto{\pgfqpoint{2.044272in}{1.547036in}}{\pgfqpoint{2.052172in}{1.543763in}}{\pgfqpoint{2.060408in}{1.543763in}}%
\pgfpathclose%
\pgfusepath{stroke,fill}%
\end{pgfscope}%
\begin{pgfscope}%
\pgfpathrectangle{\pgfqpoint{0.100000in}{0.220728in}}{\pgfqpoint{3.696000in}{3.696000in}}%
\pgfusepath{clip}%
\pgfsetbuttcap%
\pgfsetroundjoin%
\definecolor{currentfill}{rgb}{0.121569,0.466667,0.705882}%
\pgfsetfillcolor{currentfill}%
\pgfsetfillopacity{0.963784}%
\pgfsetlinewidth{1.003750pt}%
\definecolor{currentstroke}{rgb}{0.121569,0.466667,0.705882}%
\pgfsetstrokecolor{currentstroke}%
\pgfsetstrokeopacity{0.963784}%
\pgfsetdash{}{0pt}%
\pgfpathmoveto{\pgfqpoint{2.388194in}{1.356510in}}%
\pgfpathcurveto{\pgfqpoint{2.396430in}{1.356510in}}{\pgfqpoint{2.404330in}{1.359782in}}{\pgfqpoint{2.410154in}{1.365606in}}%
\pgfpathcurveto{\pgfqpoint{2.415978in}{1.371430in}}{\pgfqpoint{2.419250in}{1.379330in}}{\pgfqpoint{2.419250in}{1.387567in}}%
\pgfpathcurveto{\pgfqpoint{2.419250in}{1.395803in}}{\pgfqpoint{2.415978in}{1.403703in}}{\pgfqpoint{2.410154in}{1.409527in}}%
\pgfpathcurveto{\pgfqpoint{2.404330in}{1.415351in}}{\pgfqpoint{2.396430in}{1.418623in}}{\pgfqpoint{2.388194in}{1.418623in}}%
\pgfpathcurveto{\pgfqpoint{2.379957in}{1.418623in}}{\pgfqpoint{2.372057in}{1.415351in}}{\pgfqpoint{2.366233in}{1.409527in}}%
\pgfpathcurveto{\pgfqpoint{2.360410in}{1.403703in}}{\pgfqpoint{2.357137in}{1.395803in}}{\pgfqpoint{2.357137in}{1.387567in}}%
\pgfpathcurveto{\pgfqpoint{2.357137in}{1.379330in}}{\pgfqpoint{2.360410in}{1.371430in}}{\pgfqpoint{2.366233in}{1.365606in}}%
\pgfpathcurveto{\pgfqpoint{2.372057in}{1.359782in}}{\pgfqpoint{2.379957in}{1.356510in}}{\pgfqpoint{2.388194in}{1.356510in}}%
\pgfpathclose%
\pgfusepath{stroke,fill}%
\end{pgfscope}%
\begin{pgfscope}%
\pgfpathrectangle{\pgfqpoint{0.100000in}{0.220728in}}{\pgfqpoint{3.696000in}{3.696000in}}%
\pgfusepath{clip}%
\pgfsetbuttcap%
\pgfsetroundjoin%
\definecolor{currentfill}{rgb}{0.121569,0.466667,0.705882}%
\pgfsetfillcolor{currentfill}%
\pgfsetfillopacity{0.964666}%
\pgfsetlinewidth{1.003750pt}%
\definecolor{currentstroke}{rgb}{0.121569,0.466667,0.705882}%
\pgfsetstrokecolor{currentstroke}%
\pgfsetstrokeopacity{0.964666}%
\pgfsetdash{}{0pt}%
\pgfpathmoveto{\pgfqpoint{2.072460in}{1.536448in}}%
\pgfpathcurveto{\pgfqpoint{2.080697in}{1.536448in}}{\pgfqpoint{2.088597in}{1.539720in}}{\pgfqpoint{2.094420in}{1.545544in}}%
\pgfpathcurveto{\pgfqpoint{2.100244in}{1.551368in}}{\pgfqpoint{2.103517in}{1.559268in}}{\pgfqpoint{2.103517in}{1.567504in}}%
\pgfpathcurveto{\pgfqpoint{2.103517in}{1.575740in}}{\pgfqpoint{2.100244in}{1.583641in}}{\pgfqpoint{2.094420in}{1.589464in}}%
\pgfpathcurveto{\pgfqpoint{2.088597in}{1.595288in}}{\pgfqpoint{2.080697in}{1.598561in}}{\pgfqpoint{2.072460in}{1.598561in}}%
\pgfpathcurveto{\pgfqpoint{2.064224in}{1.598561in}}{\pgfqpoint{2.056324in}{1.595288in}}{\pgfqpoint{2.050500in}{1.589464in}}%
\pgfpathcurveto{\pgfqpoint{2.044676in}{1.583641in}}{\pgfqpoint{2.041404in}{1.575740in}}{\pgfqpoint{2.041404in}{1.567504in}}%
\pgfpathcurveto{\pgfqpoint{2.041404in}{1.559268in}}{\pgfqpoint{2.044676in}{1.551368in}}{\pgfqpoint{2.050500in}{1.545544in}}%
\pgfpathcurveto{\pgfqpoint{2.056324in}{1.539720in}}{\pgfqpoint{2.064224in}{1.536448in}}{\pgfqpoint{2.072460in}{1.536448in}}%
\pgfpathclose%
\pgfusepath{stroke,fill}%
\end{pgfscope}%
\begin{pgfscope}%
\pgfpathrectangle{\pgfqpoint{0.100000in}{0.220728in}}{\pgfqpoint{3.696000in}{3.696000in}}%
\pgfusepath{clip}%
\pgfsetbuttcap%
\pgfsetroundjoin%
\definecolor{currentfill}{rgb}{0.121569,0.466667,0.705882}%
\pgfsetfillcolor{currentfill}%
\pgfsetfillopacity{0.965659}%
\pgfsetlinewidth{1.003750pt}%
\definecolor{currentstroke}{rgb}{0.121569,0.466667,0.705882}%
\pgfsetstrokecolor{currentstroke}%
\pgfsetstrokeopacity{0.965659}%
\pgfsetdash{}{0pt}%
\pgfpathmoveto{\pgfqpoint{2.080911in}{1.533626in}}%
\pgfpathcurveto{\pgfqpoint{2.089147in}{1.533626in}}{\pgfqpoint{2.097047in}{1.536898in}}{\pgfqpoint{2.102871in}{1.542722in}}%
\pgfpathcurveto{\pgfqpoint{2.108695in}{1.548546in}}{\pgfqpoint{2.111967in}{1.556446in}}{\pgfqpoint{2.111967in}{1.564682in}}%
\pgfpathcurveto{\pgfqpoint{2.111967in}{1.572918in}}{\pgfqpoint{2.108695in}{1.580818in}}{\pgfqpoint{2.102871in}{1.586642in}}%
\pgfpathcurveto{\pgfqpoint{2.097047in}{1.592466in}}{\pgfqpoint{2.089147in}{1.595739in}}{\pgfqpoint{2.080911in}{1.595739in}}%
\pgfpathcurveto{\pgfqpoint{2.072674in}{1.595739in}}{\pgfqpoint{2.064774in}{1.592466in}}{\pgfqpoint{2.058950in}{1.586642in}}%
\pgfpathcurveto{\pgfqpoint{2.053127in}{1.580818in}}{\pgfqpoint{2.049854in}{1.572918in}}{\pgfqpoint{2.049854in}{1.564682in}}%
\pgfpathcurveto{\pgfqpoint{2.049854in}{1.556446in}}{\pgfqpoint{2.053127in}{1.548546in}}{\pgfqpoint{2.058950in}{1.542722in}}%
\pgfpathcurveto{\pgfqpoint{2.064774in}{1.536898in}}{\pgfqpoint{2.072674in}{1.533626in}}{\pgfqpoint{2.080911in}{1.533626in}}%
\pgfpathclose%
\pgfusepath{stroke,fill}%
\end{pgfscope}%
\begin{pgfscope}%
\pgfpathrectangle{\pgfqpoint{0.100000in}{0.220728in}}{\pgfqpoint{3.696000in}{3.696000in}}%
\pgfusepath{clip}%
\pgfsetbuttcap%
\pgfsetroundjoin%
\definecolor{currentfill}{rgb}{0.121569,0.466667,0.705882}%
\pgfsetfillcolor{currentfill}%
\pgfsetfillopacity{0.966874}%
\pgfsetlinewidth{1.003750pt}%
\definecolor{currentstroke}{rgb}{0.121569,0.466667,0.705882}%
\pgfsetstrokecolor{currentstroke}%
\pgfsetstrokeopacity{0.966874}%
\pgfsetdash{}{0pt}%
\pgfpathmoveto{\pgfqpoint{2.095085in}{1.520985in}}%
\pgfpathcurveto{\pgfqpoint{2.103321in}{1.520985in}}{\pgfqpoint{2.111221in}{1.524257in}}{\pgfqpoint{2.117045in}{1.530081in}}%
\pgfpathcurveto{\pgfqpoint{2.122869in}{1.535905in}}{\pgfqpoint{2.126142in}{1.543805in}}{\pgfqpoint{2.126142in}{1.552041in}}%
\pgfpathcurveto{\pgfqpoint{2.126142in}{1.560277in}}{\pgfqpoint{2.122869in}{1.568177in}}{\pgfqpoint{2.117045in}{1.574001in}}%
\pgfpathcurveto{\pgfqpoint{2.111221in}{1.579825in}}{\pgfqpoint{2.103321in}{1.583098in}}{\pgfqpoint{2.095085in}{1.583098in}}%
\pgfpathcurveto{\pgfqpoint{2.086849in}{1.583098in}}{\pgfqpoint{2.078949in}{1.579825in}}{\pgfqpoint{2.073125in}{1.574001in}}%
\pgfpathcurveto{\pgfqpoint{2.067301in}{1.568177in}}{\pgfqpoint{2.064029in}{1.560277in}}{\pgfqpoint{2.064029in}{1.552041in}}%
\pgfpathcurveto{\pgfqpoint{2.064029in}{1.543805in}}{\pgfqpoint{2.067301in}{1.535905in}}{\pgfqpoint{2.073125in}{1.530081in}}%
\pgfpathcurveto{\pgfqpoint{2.078949in}{1.524257in}}{\pgfqpoint{2.086849in}{1.520985in}}{\pgfqpoint{2.095085in}{1.520985in}}%
\pgfpathclose%
\pgfusepath{stroke,fill}%
\end{pgfscope}%
\begin{pgfscope}%
\pgfpathrectangle{\pgfqpoint{0.100000in}{0.220728in}}{\pgfqpoint{3.696000in}{3.696000in}}%
\pgfusepath{clip}%
\pgfsetbuttcap%
\pgfsetroundjoin%
\definecolor{currentfill}{rgb}{0.121569,0.466667,0.705882}%
\pgfsetfillcolor{currentfill}%
\pgfsetfillopacity{0.967235}%
\pgfsetlinewidth{1.003750pt}%
\definecolor{currentstroke}{rgb}{0.121569,0.466667,0.705882}%
\pgfsetstrokecolor{currentstroke}%
\pgfsetstrokeopacity{0.967235}%
\pgfsetdash{}{0pt}%
\pgfpathmoveto{\pgfqpoint{2.391498in}{1.350930in}}%
\pgfpathcurveto{\pgfqpoint{2.399735in}{1.350930in}}{\pgfqpoint{2.407635in}{1.354202in}}{\pgfqpoint{2.413459in}{1.360026in}}%
\pgfpathcurveto{\pgfqpoint{2.419283in}{1.365850in}}{\pgfqpoint{2.422555in}{1.373750in}}{\pgfqpoint{2.422555in}{1.381986in}}%
\pgfpathcurveto{\pgfqpoint{2.422555in}{1.390222in}}{\pgfqpoint{2.419283in}{1.398123in}}{\pgfqpoint{2.413459in}{1.403946in}}%
\pgfpathcurveto{\pgfqpoint{2.407635in}{1.409770in}}{\pgfqpoint{2.399735in}{1.413043in}}{\pgfqpoint{2.391498in}{1.413043in}}%
\pgfpathcurveto{\pgfqpoint{2.383262in}{1.413043in}}{\pgfqpoint{2.375362in}{1.409770in}}{\pgfqpoint{2.369538in}{1.403946in}}%
\pgfpathcurveto{\pgfqpoint{2.363714in}{1.398123in}}{\pgfqpoint{2.360442in}{1.390222in}}{\pgfqpoint{2.360442in}{1.381986in}}%
\pgfpathcurveto{\pgfqpoint{2.360442in}{1.373750in}}{\pgfqpoint{2.363714in}{1.365850in}}{\pgfqpoint{2.369538in}{1.360026in}}%
\pgfpathcurveto{\pgfqpoint{2.375362in}{1.354202in}}{\pgfqpoint{2.383262in}{1.350930in}}{\pgfqpoint{2.391498in}{1.350930in}}%
\pgfpathclose%
\pgfusepath{stroke,fill}%
\end{pgfscope}%
\begin{pgfscope}%
\pgfpathrectangle{\pgfqpoint{0.100000in}{0.220728in}}{\pgfqpoint{3.696000in}{3.696000in}}%
\pgfusepath{clip}%
\pgfsetbuttcap%
\pgfsetroundjoin%
\definecolor{currentfill}{rgb}{0.121569,0.466667,0.705882}%
\pgfsetfillcolor{currentfill}%
\pgfsetfillopacity{0.968408}%
\pgfsetlinewidth{1.003750pt}%
\definecolor{currentstroke}{rgb}{0.121569,0.466667,0.705882}%
\pgfsetstrokecolor{currentstroke}%
\pgfsetstrokeopacity{0.968408}%
\pgfsetdash{}{0pt}%
\pgfpathmoveto{\pgfqpoint{2.106788in}{1.514664in}}%
\pgfpathcurveto{\pgfqpoint{2.115024in}{1.514664in}}{\pgfqpoint{2.122924in}{1.517936in}}{\pgfqpoint{2.128748in}{1.523760in}}%
\pgfpathcurveto{\pgfqpoint{2.134572in}{1.529584in}}{\pgfqpoint{2.137845in}{1.537484in}}{\pgfqpoint{2.137845in}{1.545721in}}%
\pgfpathcurveto{\pgfqpoint{2.137845in}{1.553957in}}{\pgfqpoint{2.134572in}{1.561857in}}{\pgfqpoint{2.128748in}{1.567681in}}%
\pgfpathcurveto{\pgfqpoint{2.122924in}{1.573505in}}{\pgfqpoint{2.115024in}{1.576777in}}{\pgfqpoint{2.106788in}{1.576777in}}%
\pgfpathcurveto{\pgfqpoint{2.098552in}{1.576777in}}{\pgfqpoint{2.090652in}{1.573505in}}{\pgfqpoint{2.084828in}{1.567681in}}%
\pgfpathcurveto{\pgfqpoint{2.079004in}{1.561857in}}{\pgfqpoint{2.075732in}{1.553957in}}{\pgfqpoint{2.075732in}{1.545721in}}%
\pgfpathcurveto{\pgfqpoint{2.075732in}{1.537484in}}{\pgfqpoint{2.079004in}{1.529584in}}{\pgfqpoint{2.084828in}{1.523760in}}%
\pgfpathcurveto{\pgfqpoint{2.090652in}{1.517936in}}{\pgfqpoint{2.098552in}{1.514664in}}{\pgfqpoint{2.106788in}{1.514664in}}%
\pgfpathclose%
\pgfusepath{stroke,fill}%
\end{pgfscope}%
\begin{pgfscope}%
\pgfpathrectangle{\pgfqpoint{0.100000in}{0.220728in}}{\pgfqpoint{3.696000in}{3.696000in}}%
\pgfusepath{clip}%
\pgfsetbuttcap%
\pgfsetroundjoin%
\definecolor{currentfill}{rgb}{0.121569,0.466667,0.705882}%
\pgfsetfillcolor{currentfill}%
\pgfsetfillopacity{0.969598}%
\pgfsetlinewidth{1.003750pt}%
\definecolor{currentstroke}{rgb}{0.121569,0.466667,0.705882}%
\pgfsetstrokecolor{currentstroke}%
\pgfsetstrokeopacity{0.969598}%
\pgfsetdash{}{0pt}%
\pgfpathmoveto{\pgfqpoint{2.127992in}{1.493273in}}%
\pgfpathcurveto{\pgfqpoint{2.136228in}{1.493273in}}{\pgfqpoint{2.144128in}{1.496545in}}{\pgfqpoint{2.149952in}{1.502369in}}%
\pgfpathcurveto{\pgfqpoint{2.155776in}{1.508193in}}{\pgfqpoint{2.159048in}{1.516093in}}{\pgfqpoint{2.159048in}{1.524329in}}%
\pgfpathcurveto{\pgfqpoint{2.159048in}{1.532566in}}{\pgfqpoint{2.155776in}{1.540466in}}{\pgfqpoint{2.149952in}{1.546290in}}%
\pgfpathcurveto{\pgfqpoint{2.144128in}{1.552114in}}{\pgfqpoint{2.136228in}{1.555386in}}{\pgfqpoint{2.127992in}{1.555386in}}%
\pgfpathcurveto{\pgfqpoint{2.119755in}{1.555386in}}{\pgfqpoint{2.111855in}{1.552114in}}{\pgfqpoint{2.106031in}{1.546290in}}%
\pgfpathcurveto{\pgfqpoint{2.100207in}{1.540466in}}{\pgfqpoint{2.096935in}{1.532566in}}{\pgfqpoint{2.096935in}{1.524329in}}%
\pgfpathcurveto{\pgfqpoint{2.096935in}{1.516093in}}{\pgfqpoint{2.100207in}{1.508193in}}{\pgfqpoint{2.106031in}{1.502369in}}%
\pgfpathcurveto{\pgfqpoint{2.111855in}{1.496545in}}{\pgfqpoint{2.119755in}{1.493273in}}{\pgfqpoint{2.127992in}{1.493273in}}%
\pgfpathclose%
\pgfusepath{stroke,fill}%
\end{pgfscope}%
\begin{pgfscope}%
\pgfpathrectangle{\pgfqpoint{0.100000in}{0.220728in}}{\pgfqpoint{3.696000in}{3.696000in}}%
\pgfusepath{clip}%
\pgfsetbuttcap%
\pgfsetroundjoin%
\definecolor{currentfill}{rgb}{0.121569,0.466667,0.705882}%
\pgfsetfillcolor{currentfill}%
\pgfsetfillopacity{0.971384}%
\pgfsetlinewidth{1.003750pt}%
\definecolor{currentstroke}{rgb}{0.121569,0.466667,0.705882}%
\pgfsetstrokecolor{currentstroke}%
\pgfsetstrokeopacity{0.971384}%
\pgfsetdash{}{0pt}%
\pgfpathmoveto{\pgfqpoint{2.145904in}{1.480889in}}%
\pgfpathcurveto{\pgfqpoint{2.154141in}{1.480889in}}{\pgfqpoint{2.162041in}{1.484161in}}{\pgfqpoint{2.167865in}{1.489985in}}%
\pgfpathcurveto{\pgfqpoint{2.173689in}{1.495809in}}{\pgfqpoint{2.176961in}{1.503709in}}{\pgfqpoint{2.176961in}{1.511945in}}%
\pgfpathcurveto{\pgfqpoint{2.176961in}{1.520182in}}{\pgfqpoint{2.173689in}{1.528082in}}{\pgfqpoint{2.167865in}{1.533906in}}%
\pgfpathcurveto{\pgfqpoint{2.162041in}{1.539730in}}{\pgfqpoint{2.154141in}{1.543002in}}{\pgfqpoint{2.145904in}{1.543002in}}%
\pgfpathcurveto{\pgfqpoint{2.137668in}{1.543002in}}{\pgfqpoint{2.129768in}{1.539730in}}{\pgfqpoint{2.123944in}{1.533906in}}%
\pgfpathcurveto{\pgfqpoint{2.118120in}{1.528082in}}{\pgfqpoint{2.114848in}{1.520182in}}{\pgfqpoint{2.114848in}{1.511945in}}%
\pgfpathcurveto{\pgfqpoint{2.114848in}{1.503709in}}{\pgfqpoint{2.118120in}{1.495809in}}{\pgfqpoint{2.123944in}{1.489985in}}%
\pgfpathcurveto{\pgfqpoint{2.129768in}{1.484161in}}{\pgfqpoint{2.137668in}{1.480889in}}{\pgfqpoint{2.145904in}{1.480889in}}%
\pgfpathclose%
\pgfusepath{stroke,fill}%
\end{pgfscope}%
\begin{pgfscope}%
\pgfpathrectangle{\pgfqpoint{0.100000in}{0.220728in}}{\pgfqpoint{3.696000in}{3.696000in}}%
\pgfusepath{clip}%
\pgfsetbuttcap%
\pgfsetroundjoin%
\definecolor{currentfill}{rgb}{0.121569,0.466667,0.705882}%
\pgfsetfillcolor{currentfill}%
\pgfsetfillopacity{0.971521}%
\pgfsetlinewidth{1.003750pt}%
\definecolor{currentstroke}{rgb}{0.121569,0.466667,0.705882}%
\pgfsetstrokecolor{currentstroke}%
\pgfsetstrokeopacity{0.971521}%
\pgfsetdash{}{0pt}%
\pgfpathmoveto{\pgfqpoint{2.394064in}{1.345943in}}%
\pgfpathcurveto{\pgfqpoint{2.402300in}{1.345943in}}{\pgfqpoint{2.410200in}{1.349215in}}{\pgfqpoint{2.416024in}{1.355039in}}%
\pgfpathcurveto{\pgfqpoint{2.421848in}{1.360863in}}{\pgfqpoint{2.425120in}{1.368763in}}{\pgfqpoint{2.425120in}{1.376999in}}%
\pgfpathcurveto{\pgfqpoint{2.425120in}{1.385235in}}{\pgfqpoint{2.421848in}{1.393135in}}{\pgfqpoint{2.416024in}{1.398959in}}%
\pgfpathcurveto{\pgfqpoint{2.410200in}{1.404783in}}{\pgfqpoint{2.402300in}{1.408056in}}{\pgfqpoint{2.394064in}{1.408056in}}%
\pgfpathcurveto{\pgfqpoint{2.385827in}{1.408056in}}{\pgfqpoint{2.377927in}{1.404783in}}{\pgfqpoint{2.372103in}{1.398959in}}%
\pgfpathcurveto{\pgfqpoint{2.366279in}{1.393135in}}{\pgfqpoint{2.363007in}{1.385235in}}{\pgfqpoint{2.363007in}{1.376999in}}%
\pgfpathcurveto{\pgfqpoint{2.363007in}{1.368763in}}{\pgfqpoint{2.366279in}{1.360863in}}{\pgfqpoint{2.372103in}{1.355039in}}%
\pgfpathcurveto{\pgfqpoint{2.377927in}{1.349215in}}{\pgfqpoint{2.385827in}{1.345943in}}{\pgfqpoint{2.394064in}{1.345943in}}%
\pgfpathclose%
\pgfusepath{stroke,fill}%
\end{pgfscope}%
\begin{pgfscope}%
\pgfpathrectangle{\pgfqpoint{0.100000in}{0.220728in}}{\pgfqpoint{3.696000in}{3.696000in}}%
\pgfusepath{clip}%
\pgfsetbuttcap%
\pgfsetroundjoin%
\definecolor{currentfill}{rgb}{0.121569,0.466667,0.705882}%
\pgfsetfillcolor{currentfill}%
\pgfsetfillopacity{0.973347}%
\pgfsetlinewidth{1.003750pt}%
\definecolor{currentstroke}{rgb}{0.121569,0.466667,0.705882}%
\pgfsetstrokecolor{currentstroke}%
\pgfsetstrokeopacity{0.973347}%
\pgfsetdash{}{0pt}%
\pgfpathmoveto{\pgfqpoint{2.162068in}{1.466432in}}%
\pgfpathcurveto{\pgfqpoint{2.170304in}{1.466432in}}{\pgfqpoint{2.178204in}{1.469704in}}{\pgfqpoint{2.184028in}{1.475528in}}%
\pgfpathcurveto{\pgfqpoint{2.189852in}{1.481352in}}{\pgfqpoint{2.193124in}{1.489252in}}{\pgfqpoint{2.193124in}{1.497488in}}%
\pgfpathcurveto{\pgfqpoint{2.193124in}{1.505725in}}{\pgfqpoint{2.189852in}{1.513625in}}{\pgfqpoint{2.184028in}{1.519449in}}%
\pgfpathcurveto{\pgfqpoint{2.178204in}{1.525273in}}{\pgfqpoint{2.170304in}{1.528545in}}{\pgfqpoint{2.162068in}{1.528545in}}%
\pgfpathcurveto{\pgfqpoint{2.153831in}{1.528545in}}{\pgfqpoint{2.145931in}{1.525273in}}{\pgfqpoint{2.140107in}{1.519449in}}%
\pgfpathcurveto{\pgfqpoint{2.134283in}{1.513625in}}{\pgfqpoint{2.131011in}{1.505725in}}{\pgfqpoint{2.131011in}{1.497488in}}%
\pgfpathcurveto{\pgfqpoint{2.131011in}{1.489252in}}{\pgfqpoint{2.134283in}{1.481352in}}{\pgfqpoint{2.140107in}{1.475528in}}%
\pgfpathcurveto{\pgfqpoint{2.145931in}{1.469704in}}{\pgfqpoint{2.153831in}{1.466432in}}{\pgfqpoint{2.162068in}{1.466432in}}%
\pgfpathclose%
\pgfusepath{stroke,fill}%
\end{pgfscope}%
\begin{pgfscope}%
\pgfpathrectangle{\pgfqpoint{0.100000in}{0.220728in}}{\pgfqpoint{3.696000in}{3.696000in}}%
\pgfusepath{clip}%
\pgfsetbuttcap%
\pgfsetroundjoin%
\definecolor{currentfill}{rgb}{0.121569,0.466667,0.705882}%
\pgfsetfillcolor{currentfill}%
\pgfsetfillopacity{0.974915}%
\pgfsetlinewidth{1.003750pt}%
\definecolor{currentstroke}{rgb}{0.121569,0.466667,0.705882}%
\pgfsetstrokecolor{currentstroke}%
\pgfsetstrokeopacity{0.974915}%
\pgfsetdash{}{0pt}%
\pgfpathmoveto{\pgfqpoint{2.177430in}{1.455806in}}%
\pgfpathcurveto{\pgfqpoint{2.185667in}{1.455806in}}{\pgfqpoint{2.193567in}{1.459078in}}{\pgfqpoint{2.199391in}{1.464902in}}%
\pgfpathcurveto{\pgfqpoint{2.205214in}{1.470726in}}{\pgfqpoint{2.208487in}{1.478626in}}{\pgfqpoint{2.208487in}{1.486863in}}%
\pgfpathcurveto{\pgfqpoint{2.208487in}{1.495099in}}{\pgfqpoint{2.205214in}{1.502999in}}{\pgfqpoint{2.199391in}{1.508823in}}%
\pgfpathcurveto{\pgfqpoint{2.193567in}{1.514647in}}{\pgfqpoint{2.185667in}{1.517919in}}{\pgfqpoint{2.177430in}{1.517919in}}%
\pgfpathcurveto{\pgfqpoint{2.169194in}{1.517919in}}{\pgfqpoint{2.161294in}{1.514647in}}{\pgfqpoint{2.155470in}{1.508823in}}%
\pgfpathcurveto{\pgfqpoint{2.149646in}{1.502999in}}{\pgfqpoint{2.146374in}{1.495099in}}{\pgfqpoint{2.146374in}{1.486863in}}%
\pgfpathcurveto{\pgfqpoint{2.146374in}{1.478626in}}{\pgfqpoint{2.149646in}{1.470726in}}{\pgfqpoint{2.155470in}{1.464902in}}%
\pgfpathcurveto{\pgfqpoint{2.161294in}{1.459078in}}{\pgfqpoint{2.169194in}{1.455806in}}{\pgfqpoint{2.177430in}{1.455806in}}%
\pgfpathclose%
\pgfusepath{stroke,fill}%
\end{pgfscope}%
\begin{pgfscope}%
\pgfpathrectangle{\pgfqpoint{0.100000in}{0.220728in}}{\pgfqpoint{3.696000in}{3.696000in}}%
\pgfusepath{clip}%
\pgfsetbuttcap%
\pgfsetroundjoin%
\definecolor{currentfill}{rgb}{0.121569,0.466667,0.705882}%
\pgfsetfillcolor{currentfill}%
\pgfsetfillopacity{0.975890}%
\pgfsetlinewidth{1.003750pt}%
\definecolor{currentstroke}{rgb}{0.121569,0.466667,0.705882}%
\pgfsetstrokecolor{currentstroke}%
\pgfsetstrokeopacity{0.975890}%
\pgfsetdash{}{0pt}%
\pgfpathmoveto{\pgfqpoint{2.396921in}{1.339399in}}%
\pgfpathcurveto{\pgfqpoint{2.405157in}{1.339399in}}{\pgfqpoint{2.413057in}{1.342672in}}{\pgfqpoint{2.418881in}{1.348496in}}%
\pgfpathcurveto{\pgfqpoint{2.424705in}{1.354320in}}{\pgfqpoint{2.427977in}{1.362220in}}{\pgfqpoint{2.427977in}{1.370456in}}%
\pgfpathcurveto{\pgfqpoint{2.427977in}{1.378692in}}{\pgfqpoint{2.424705in}{1.386592in}}{\pgfqpoint{2.418881in}{1.392416in}}%
\pgfpathcurveto{\pgfqpoint{2.413057in}{1.398240in}}{\pgfqpoint{2.405157in}{1.401512in}}{\pgfqpoint{2.396921in}{1.401512in}}%
\pgfpathcurveto{\pgfqpoint{2.388685in}{1.401512in}}{\pgfqpoint{2.380785in}{1.398240in}}{\pgfqpoint{2.374961in}{1.392416in}}%
\pgfpathcurveto{\pgfqpoint{2.369137in}{1.386592in}}{\pgfqpoint{2.365864in}{1.378692in}}{\pgfqpoint{2.365864in}{1.370456in}}%
\pgfpathcurveto{\pgfqpoint{2.365864in}{1.362220in}}{\pgfqpoint{2.369137in}{1.354320in}}{\pgfqpoint{2.374961in}{1.348496in}}%
\pgfpathcurveto{\pgfqpoint{2.380785in}{1.342672in}}{\pgfqpoint{2.388685in}{1.339399in}}{\pgfqpoint{2.396921in}{1.339399in}}%
\pgfpathclose%
\pgfusepath{stroke,fill}%
\end{pgfscope}%
\begin{pgfscope}%
\pgfpathrectangle{\pgfqpoint{0.100000in}{0.220728in}}{\pgfqpoint{3.696000in}{3.696000in}}%
\pgfusepath{clip}%
\pgfsetbuttcap%
\pgfsetroundjoin%
\definecolor{currentfill}{rgb}{0.121569,0.466667,0.705882}%
\pgfsetfillcolor{currentfill}%
\pgfsetfillopacity{0.977245}%
\pgfsetlinewidth{1.003750pt}%
\definecolor{currentstroke}{rgb}{0.121569,0.466667,0.705882}%
\pgfsetstrokecolor{currentstroke}%
\pgfsetstrokeopacity{0.977245}%
\pgfsetdash{}{0pt}%
\pgfpathmoveto{\pgfqpoint{2.191706in}{1.448742in}}%
\pgfpathcurveto{\pgfqpoint{2.199942in}{1.448742in}}{\pgfqpoint{2.207843in}{1.452014in}}{\pgfqpoint{2.213666in}{1.457838in}}%
\pgfpathcurveto{\pgfqpoint{2.219490in}{1.463662in}}{\pgfqpoint{2.222763in}{1.471562in}}{\pgfqpoint{2.222763in}{1.479799in}}%
\pgfpathcurveto{\pgfqpoint{2.222763in}{1.488035in}}{\pgfqpoint{2.219490in}{1.495935in}}{\pgfqpoint{2.213666in}{1.501759in}}%
\pgfpathcurveto{\pgfqpoint{2.207843in}{1.507583in}}{\pgfqpoint{2.199942in}{1.510855in}}{\pgfqpoint{2.191706in}{1.510855in}}%
\pgfpathcurveto{\pgfqpoint{2.183470in}{1.510855in}}{\pgfqpoint{2.175570in}{1.507583in}}{\pgfqpoint{2.169746in}{1.501759in}}%
\pgfpathcurveto{\pgfqpoint{2.163922in}{1.495935in}}{\pgfqpoint{2.160650in}{1.488035in}}{\pgfqpoint{2.160650in}{1.479799in}}%
\pgfpathcurveto{\pgfqpoint{2.160650in}{1.471562in}}{\pgfqpoint{2.163922in}{1.463662in}}{\pgfqpoint{2.169746in}{1.457838in}}%
\pgfpathcurveto{\pgfqpoint{2.175570in}{1.452014in}}{\pgfqpoint{2.183470in}{1.448742in}}{\pgfqpoint{2.191706in}{1.448742in}}%
\pgfpathclose%
\pgfusepath{stroke,fill}%
\end{pgfscope}%
\begin{pgfscope}%
\pgfpathrectangle{\pgfqpoint{0.100000in}{0.220728in}}{\pgfqpoint{3.696000in}{3.696000in}}%
\pgfusepath{clip}%
\pgfsetbuttcap%
\pgfsetroundjoin%
\definecolor{currentfill}{rgb}{0.121569,0.466667,0.705882}%
\pgfsetfillcolor{currentfill}%
\pgfsetfillopacity{0.978786}%
\pgfsetlinewidth{1.003750pt}%
\definecolor{currentstroke}{rgb}{0.121569,0.466667,0.705882}%
\pgfsetstrokecolor{currentstroke}%
\pgfsetstrokeopacity{0.978786}%
\pgfsetdash{}{0pt}%
\pgfpathmoveto{\pgfqpoint{2.203375in}{1.441200in}}%
\pgfpathcurveto{\pgfqpoint{2.211611in}{1.441200in}}{\pgfqpoint{2.219511in}{1.444472in}}{\pgfqpoint{2.225335in}{1.450296in}}%
\pgfpathcurveto{\pgfqpoint{2.231159in}{1.456120in}}{\pgfqpoint{2.234432in}{1.464020in}}{\pgfqpoint{2.234432in}{1.472256in}}%
\pgfpathcurveto{\pgfqpoint{2.234432in}{1.480492in}}{\pgfqpoint{2.231159in}{1.488392in}}{\pgfqpoint{2.225335in}{1.494216in}}%
\pgfpathcurveto{\pgfqpoint{2.219511in}{1.500040in}}{\pgfqpoint{2.211611in}{1.503313in}}{\pgfqpoint{2.203375in}{1.503313in}}%
\pgfpathcurveto{\pgfqpoint{2.195139in}{1.503313in}}{\pgfqpoint{2.187239in}{1.500040in}}{\pgfqpoint{2.181415in}{1.494216in}}%
\pgfpathcurveto{\pgfqpoint{2.175591in}{1.488392in}}{\pgfqpoint{2.172319in}{1.480492in}}{\pgfqpoint{2.172319in}{1.472256in}}%
\pgfpathcurveto{\pgfqpoint{2.172319in}{1.464020in}}{\pgfqpoint{2.175591in}{1.456120in}}{\pgfqpoint{2.181415in}{1.450296in}}%
\pgfpathcurveto{\pgfqpoint{2.187239in}{1.444472in}}{\pgfqpoint{2.195139in}{1.441200in}}{\pgfqpoint{2.203375in}{1.441200in}}%
\pgfpathclose%
\pgfusepath{stroke,fill}%
\end{pgfscope}%
\begin{pgfscope}%
\pgfpathrectangle{\pgfqpoint{0.100000in}{0.220728in}}{\pgfqpoint{3.696000in}{3.696000in}}%
\pgfusepath{clip}%
\pgfsetbuttcap%
\pgfsetroundjoin%
\definecolor{currentfill}{rgb}{0.121569,0.466667,0.705882}%
\pgfsetfillcolor{currentfill}%
\pgfsetfillopacity{0.980167}%
\pgfsetlinewidth{1.003750pt}%
\definecolor{currentstroke}{rgb}{0.121569,0.466667,0.705882}%
\pgfsetstrokecolor{currentstroke}%
\pgfsetstrokeopacity{0.980167}%
\pgfsetdash{}{0pt}%
\pgfpathmoveto{\pgfqpoint{2.214061in}{1.437889in}}%
\pgfpathcurveto{\pgfqpoint{2.222298in}{1.437889in}}{\pgfqpoint{2.230198in}{1.441161in}}{\pgfqpoint{2.236022in}{1.446985in}}%
\pgfpathcurveto{\pgfqpoint{2.241846in}{1.452809in}}{\pgfqpoint{2.245118in}{1.460709in}}{\pgfqpoint{2.245118in}{1.468945in}}%
\pgfpathcurveto{\pgfqpoint{2.245118in}{1.477182in}}{\pgfqpoint{2.241846in}{1.485082in}}{\pgfqpoint{2.236022in}{1.490906in}}%
\pgfpathcurveto{\pgfqpoint{2.230198in}{1.496729in}}{\pgfqpoint{2.222298in}{1.500002in}}{\pgfqpoint{2.214061in}{1.500002in}}%
\pgfpathcurveto{\pgfqpoint{2.205825in}{1.500002in}}{\pgfqpoint{2.197925in}{1.496729in}}{\pgfqpoint{2.192101in}{1.490906in}}%
\pgfpathcurveto{\pgfqpoint{2.186277in}{1.485082in}}{\pgfqpoint{2.183005in}{1.477182in}}{\pgfqpoint{2.183005in}{1.468945in}}%
\pgfpathcurveto{\pgfqpoint{2.183005in}{1.460709in}}{\pgfqpoint{2.186277in}{1.452809in}}{\pgfqpoint{2.192101in}{1.446985in}}%
\pgfpathcurveto{\pgfqpoint{2.197925in}{1.441161in}}{\pgfqpoint{2.205825in}{1.437889in}}{\pgfqpoint{2.214061in}{1.437889in}}%
\pgfpathclose%
\pgfusepath{stroke,fill}%
\end{pgfscope}%
\begin{pgfscope}%
\pgfpathrectangle{\pgfqpoint{0.100000in}{0.220728in}}{\pgfqpoint{3.696000in}{3.696000in}}%
\pgfusepath{clip}%
\pgfsetbuttcap%
\pgfsetroundjoin%
\definecolor{currentfill}{rgb}{0.121569,0.466667,0.705882}%
\pgfsetfillcolor{currentfill}%
\pgfsetfillopacity{0.980893}%
\pgfsetlinewidth{1.003750pt}%
\definecolor{currentstroke}{rgb}{0.121569,0.466667,0.705882}%
\pgfsetstrokecolor{currentstroke}%
\pgfsetstrokeopacity{0.980893}%
\pgfsetdash{}{0pt}%
\pgfpathmoveto{\pgfqpoint{2.400941in}{1.335677in}}%
\pgfpathcurveto{\pgfqpoint{2.409177in}{1.335677in}}{\pgfqpoint{2.417077in}{1.338950in}}{\pgfqpoint{2.422901in}{1.344774in}}%
\pgfpathcurveto{\pgfqpoint{2.428725in}{1.350597in}}{\pgfqpoint{2.431998in}{1.358498in}}{\pgfqpoint{2.431998in}{1.366734in}}%
\pgfpathcurveto{\pgfqpoint{2.431998in}{1.374970in}}{\pgfqpoint{2.428725in}{1.382870in}}{\pgfqpoint{2.422901in}{1.388694in}}%
\pgfpathcurveto{\pgfqpoint{2.417077in}{1.394518in}}{\pgfqpoint{2.409177in}{1.397790in}}{\pgfqpoint{2.400941in}{1.397790in}}%
\pgfpathcurveto{\pgfqpoint{2.392705in}{1.397790in}}{\pgfqpoint{2.384805in}{1.394518in}}{\pgfqpoint{2.378981in}{1.388694in}}%
\pgfpathcurveto{\pgfqpoint{2.373157in}{1.382870in}}{\pgfqpoint{2.369885in}{1.374970in}}{\pgfqpoint{2.369885in}{1.366734in}}%
\pgfpathcurveto{\pgfqpoint{2.369885in}{1.358498in}}{\pgfqpoint{2.373157in}{1.350597in}}{\pgfqpoint{2.378981in}{1.344774in}}%
\pgfpathcurveto{\pgfqpoint{2.384805in}{1.338950in}}{\pgfqpoint{2.392705in}{1.335677in}}{\pgfqpoint{2.400941in}{1.335677in}}%
\pgfpathclose%
\pgfusepath{stroke,fill}%
\end{pgfscope}%
\begin{pgfscope}%
\pgfpathrectangle{\pgfqpoint{0.100000in}{0.220728in}}{\pgfqpoint{3.696000in}{3.696000in}}%
\pgfusepath{clip}%
\pgfsetbuttcap%
\pgfsetroundjoin%
\definecolor{currentfill}{rgb}{0.121569,0.466667,0.705882}%
\pgfsetfillcolor{currentfill}%
\pgfsetfillopacity{0.982081}%
\pgfsetlinewidth{1.003750pt}%
\definecolor{currentstroke}{rgb}{0.121569,0.466667,0.705882}%
\pgfsetstrokecolor{currentstroke}%
\pgfsetstrokeopacity{0.982081}%
\pgfsetdash{}{0pt}%
\pgfpathmoveto{\pgfqpoint{2.231249in}{1.421468in}}%
\pgfpathcurveto{\pgfqpoint{2.239485in}{1.421468in}}{\pgfqpoint{2.247385in}{1.424740in}}{\pgfqpoint{2.253209in}{1.430564in}}%
\pgfpathcurveto{\pgfqpoint{2.259033in}{1.436388in}}{\pgfqpoint{2.262305in}{1.444288in}}{\pgfqpoint{2.262305in}{1.452525in}}%
\pgfpathcurveto{\pgfqpoint{2.262305in}{1.460761in}}{\pgfqpoint{2.259033in}{1.468661in}}{\pgfqpoint{2.253209in}{1.474485in}}%
\pgfpathcurveto{\pgfqpoint{2.247385in}{1.480309in}}{\pgfqpoint{2.239485in}{1.483581in}}{\pgfqpoint{2.231249in}{1.483581in}}%
\pgfpathcurveto{\pgfqpoint{2.223013in}{1.483581in}}{\pgfqpoint{2.215113in}{1.480309in}}{\pgfqpoint{2.209289in}{1.474485in}}%
\pgfpathcurveto{\pgfqpoint{2.203465in}{1.468661in}}{\pgfqpoint{2.200192in}{1.460761in}}{\pgfqpoint{2.200192in}{1.452525in}}%
\pgfpathcurveto{\pgfqpoint{2.200192in}{1.444288in}}{\pgfqpoint{2.203465in}{1.436388in}}{\pgfqpoint{2.209289in}{1.430564in}}%
\pgfpathcurveto{\pgfqpoint{2.215113in}{1.424740in}}{\pgfqpoint{2.223013in}{1.421468in}}{\pgfqpoint{2.231249in}{1.421468in}}%
\pgfpathclose%
\pgfusepath{stroke,fill}%
\end{pgfscope}%
\begin{pgfscope}%
\pgfpathrectangle{\pgfqpoint{0.100000in}{0.220728in}}{\pgfqpoint{3.696000in}{3.696000in}}%
\pgfusepath{clip}%
\pgfsetbuttcap%
\pgfsetroundjoin%
\definecolor{currentfill}{rgb}{0.121569,0.466667,0.705882}%
\pgfsetfillcolor{currentfill}%
\pgfsetfillopacity{0.984701}%
\pgfsetlinewidth{1.003750pt}%
\definecolor{currentstroke}{rgb}{0.121569,0.466667,0.705882}%
\pgfsetstrokecolor{currentstroke}%
\pgfsetstrokeopacity{0.984701}%
\pgfsetdash{}{0pt}%
\pgfpathmoveto{\pgfqpoint{2.246719in}{1.409962in}}%
\pgfpathcurveto{\pgfqpoint{2.254955in}{1.409962in}}{\pgfqpoint{2.262855in}{1.413234in}}{\pgfqpoint{2.268679in}{1.419058in}}%
\pgfpathcurveto{\pgfqpoint{2.274503in}{1.424882in}}{\pgfqpoint{2.277776in}{1.432782in}}{\pgfqpoint{2.277776in}{1.441018in}}%
\pgfpathcurveto{\pgfqpoint{2.277776in}{1.449254in}}{\pgfqpoint{2.274503in}{1.457154in}}{\pgfqpoint{2.268679in}{1.462978in}}%
\pgfpathcurveto{\pgfqpoint{2.262855in}{1.468802in}}{\pgfqpoint{2.254955in}{1.472075in}}{\pgfqpoint{2.246719in}{1.472075in}}%
\pgfpathcurveto{\pgfqpoint{2.238483in}{1.472075in}}{\pgfqpoint{2.230583in}{1.468802in}}{\pgfqpoint{2.224759in}{1.462978in}}%
\pgfpathcurveto{\pgfqpoint{2.218935in}{1.457154in}}{\pgfqpoint{2.215663in}{1.449254in}}{\pgfqpoint{2.215663in}{1.441018in}}%
\pgfpathcurveto{\pgfqpoint{2.215663in}{1.432782in}}{\pgfqpoint{2.218935in}{1.424882in}}{\pgfqpoint{2.224759in}{1.419058in}}%
\pgfpathcurveto{\pgfqpoint{2.230583in}{1.413234in}}{\pgfqpoint{2.238483in}{1.409962in}}{\pgfqpoint{2.246719in}{1.409962in}}%
\pgfpathclose%
\pgfusepath{stroke,fill}%
\end{pgfscope}%
\begin{pgfscope}%
\pgfpathrectangle{\pgfqpoint{0.100000in}{0.220728in}}{\pgfqpoint{3.696000in}{3.696000in}}%
\pgfusepath{clip}%
\pgfsetbuttcap%
\pgfsetroundjoin%
\definecolor{currentfill}{rgb}{0.121569,0.466667,0.705882}%
\pgfsetfillcolor{currentfill}%
\pgfsetfillopacity{0.985182}%
\pgfsetlinewidth{1.003750pt}%
\definecolor{currentstroke}{rgb}{0.121569,0.466667,0.705882}%
\pgfsetstrokecolor{currentstroke}%
\pgfsetstrokeopacity{0.985182}%
\pgfsetdash{}{0pt}%
\pgfpathmoveto{\pgfqpoint{2.263921in}{1.395598in}}%
\pgfpathcurveto{\pgfqpoint{2.272157in}{1.395598in}}{\pgfqpoint{2.280057in}{1.398871in}}{\pgfqpoint{2.285881in}{1.404695in}}%
\pgfpathcurveto{\pgfqpoint{2.291705in}{1.410519in}}{\pgfqpoint{2.294978in}{1.418419in}}{\pgfqpoint{2.294978in}{1.426655in}}%
\pgfpathcurveto{\pgfqpoint{2.294978in}{1.434891in}}{\pgfqpoint{2.291705in}{1.442791in}}{\pgfqpoint{2.285881in}{1.448615in}}%
\pgfpathcurveto{\pgfqpoint{2.280057in}{1.454439in}}{\pgfqpoint{2.272157in}{1.457711in}}{\pgfqpoint{2.263921in}{1.457711in}}%
\pgfpathcurveto{\pgfqpoint{2.255685in}{1.457711in}}{\pgfqpoint{2.247785in}{1.454439in}}{\pgfqpoint{2.241961in}{1.448615in}}%
\pgfpathcurveto{\pgfqpoint{2.236137in}{1.442791in}}{\pgfqpoint{2.232865in}{1.434891in}}{\pgfqpoint{2.232865in}{1.426655in}}%
\pgfpathcurveto{\pgfqpoint{2.232865in}{1.418419in}}{\pgfqpoint{2.236137in}{1.410519in}}{\pgfqpoint{2.241961in}{1.404695in}}%
\pgfpathcurveto{\pgfqpoint{2.247785in}{1.398871in}}{\pgfqpoint{2.255685in}{1.395598in}}{\pgfqpoint{2.263921in}{1.395598in}}%
\pgfpathclose%
\pgfusepath{stroke,fill}%
\end{pgfscope}%
\begin{pgfscope}%
\pgfpathrectangle{\pgfqpoint{0.100000in}{0.220728in}}{\pgfqpoint{3.696000in}{3.696000in}}%
\pgfusepath{clip}%
\pgfsetbuttcap%
\pgfsetroundjoin%
\definecolor{currentfill}{rgb}{0.121569,0.466667,0.705882}%
\pgfsetfillcolor{currentfill}%
\pgfsetfillopacity{0.986030}%
\pgfsetlinewidth{1.003750pt}%
\definecolor{currentstroke}{rgb}{0.121569,0.466667,0.705882}%
\pgfsetstrokecolor{currentstroke}%
\pgfsetstrokeopacity{0.986030}%
\pgfsetdash{}{0pt}%
\pgfpathmoveto{\pgfqpoint{2.404185in}{1.330135in}}%
\pgfpathcurveto{\pgfqpoint{2.412421in}{1.330135in}}{\pgfqpoint{2.420321in}{1.333407in}}{\pgfqpoint{2.426145in}{1.339231in}}%
\pgfpathcurveto{\pgfqpoint{2.431969in}{1.345055in}}{\pgfqpoint{2.435242in}{1.352955in}}{\pgfqpoint{2.435242in}{1.361191in}}%
\pgfpathcurveto{\pgfqpoint{2.435242in}{1.369428in}}{\pgfqpoint{2.431969in}{1.377328in}}{\pgfqpoint{2.426145in}{1.383152in}}%
\pgfpathcurveto{\pgfqpoint{2.420321in}{1.388975in}}{\pgfqpoint{2.412421in}{1.392248in}}{\pgfqpoint{2.404185in}{1.392248in}}%
\pgfpathcurveto{\pgfqpoint{2.395949in}{1.392248in}}{\pgfqpoint{2.388049in}{1.388975in}}{\pgfqpoint{2.382225in}{1.383152in}}%
\pgfpathcurveto{\pgfqpoint{2.376401in}{1.377328in}}{\pgfqpoint{2.373129in}{1.369428in}}{\pgfqpoint{2.373129in}{1.361191in}}%
\pgfpathcurveto{\pgfqpoint{2.373129in}{1.352955in}}{\pgfqpoint{2.376401in}{1.345055in}}{\pgfqpoint{2.382225in}{1.339231in}}%
\pgfpathcurveto{\pgfqpoint{2.388049in}{1.333407in}}{\pgfqpoint{2.395949in}{1.330135in}}{\pgfqpoint{2.404185in}{1.330135in}}%
\pgfpathclose%
\pgfusepath{stroke,fill}%
\end{pgfscope}%
\begin{pgfscope}%
\pgfpathrectangle{\pgfqpoint{0.100000in}{0.220728in}}{\pgfqpoint{3.696000in}{3.696000in}}%
\pgfusepath{clip}%
\pgfsetbuttcap%
\pgfsetroundjoin%
\definecolor{currentfill}{rgb}{0.121569,0.466667,0.705882}%
\pgfsetfillcolor{currentfill}%
\pgfsetfillopacity{0.987416}%
\pgfsetlinewidth{1.003750pt}%
\definecolor{currentstroke}{rgb}{0.121569,0.466667,0.705882}%
\pgfsetstrokecolor{currentstroke}%
\pgfsetstrokeopacity{0.987416}%
\pgfsetdash{}{0pt}%
\pgfpathmoveto{\pgfqpoint{2.277415in}{1.390636in}}%
\pgfpathcurveto{\pgfqpoint{2.285652in}{1.390636in}}{\pgfqpoint{2.293552in}{1.393908in}}{\pgfqpoint{2.299376in}{1.399732in}}%
\pgfpathcurveto{\pgfqpoint{2.305200in}{1.405556in}}{\pgfqpoint{2.308472in}{1.413456in}}{\pgfqpoint{2.308472in}{1.421692in}}%
\pgfpathcurveto{\pgfqpoint{2.308472in}{1.429929in}}{\pgfqpoint{2.305200in}{1.437829in}}{\pgfqpoint{2.299376in}{1.443653in}}%
\pgfpathcurveto{\pgfqpoint{2.293552in}{1.449477in}}{\pgfqpoint{2.285652in}{1.452749in}}{\pgfqpoint{2.277415in}{1.452749in}}%
\pgfpathcurveto{\pgfqpoint{2.269179in}{1.452749in}}{\pgfqpoint{2.261279in}{1.449477in}}{\pgfqpoint{2.255455in}{1.443653in}}%
\pgfpathcurveto{\pgfqpoint{2.249631in}{1.437829in}}{\pgfqpoint{2.246359in}{1.429929in}}{\pgfqpoint{2.246359in}{1.421692in}}%
\pgfpathcurveto{\pgfqpoint{2.246359in}{1.413456in}}{\pgfqpoint{2.249631in}{1.405556in}}{\pgfqpoint{2.255455in}{1.399732in}}%
\pgfpathcurveto{\pgfqpoint{2.261279in}{1.393908in}}{\pgfqpoint{2.269179in}{1.390636in}}{\pgfqpoint{2.277415in}{1.390636in}}%
\pgfpathclose%
\pgfusepath{stroke,fill}%
\end{pgfscope}%
\begin{pgfscope}%
\pgfpathrectangle{\pgfqpoint{0.100000in}{0.220728in}}{\pgfqpoint{3.696000in}{3.696000in}}%
\pgfusepath{clip}%
\pgfsetbuttcap%
\pgfsetroundjoin%
\definecolor{currentfill}{rgb}{0.121569,0.466667,0.705882}%
\pgfsetfillcolor{currentfill}%
\pgfsetfillopacity{0.989174}%
\pgfsetlinewidth{1.003750pt}%
\definecolor{currentstroke}{rgb}{0.121569,0.466667,0.705882}%
\pgfsetstrokecolor{currentstroke}%
\pgfsetstrokeopacity{0.989174}%
\pgfsetdash{}{0pt}%
\pgfpathmoveto{\pgfqpoint{2.289982in}{1.385558in}}%
\pgfpathcurveto{\pgfqpoint{2.298218in}{1.385558in}}{\pgfqpoint{2.306118in}{1.388830in}}{\pgfqpoint{2.311942in}{1.394654in}}%
\pgfpathcurveto{\pgfqpoint{2.317766in}{1.400478in}}{\pgfqpoint{2.321038in}{1.408378in}}{\pgfqpoint{2.321038in}{1.416614in}}%
\pgfpathcurveto{\pgfqpoint{2.321038in}{1.424851in}}{\pgfqpoint{2.317766in}{1.432751in}}{\pgfqpoint{2.311942in}{1.438575in}}%
\pgfpathcurveto{\pgfqpoint{2.306118in}{1.444398in}}{\pgfqpoint{2.298218in}{1.447671in}}{\pgfqpoint{2.289982in}{1.447671in}}%
\pgfpathcurveto{\pgfqpoint{2.281746in}{1.447671in}}{\pgfqpoint{2.273846in}{1.444398in}}{\pgfqpoint{2.268022in}{1.438575in}}%
\pgfpathcurveto{\pgfqpoint{2.262198in}{1.432751in}}{\pgfqpoint{2.258925in}{1.424851in}}{\pgfqpoint{2.258925in}{1.416614in}}%
\pgfpathcurveto{\pgfqpoint{2.258925in}{1.408378in}}{\pgfqpoint{2.262198in}{1.400478in}}{\pgfqpoint{2.268022in}{1.394654in}}%
\pgfpathcurveto{\pgfqpoint{2.273846in}{1.388830in}}{\pgfqpoint{2.281746in}{1.385558in}}{\pgfqpoint{2.289982in}{1.385558in}}%
\pgfpathclose%
\pgfusepath{stroke,fill}%
\end{pgfscope}%
\begin{pgfscope}%
\pgfpathrectangle{\pgfqpoint{0.100000in}{0.220728in}}{\pgfqpoint{3.696000in}{3.696000in}}%
\pgfusepath{clip}%
\pgfsetbuttcap%
\pgfsetroundjoin%
\definecolor{currentfill}{rgb}{0.121569,0.466667,0.705882}%
\pgfsetfillcolor{currentfill}%
\pgfsetfillopacity{0.989933}%
\pgfsetlinewidth{1.003750pt}%
\definecolor{currentstroke}{rgb}{0.121569,0.466667,0.705882}%
\pgfsetstrokecolor{currentstroke}%
\pgfsetstrokeopacity{0.989933}%
\pgfsetdash{}{0pt}%
\pgfpathmoveto{\pgfqpoint{2.300540in}{1.373199in}}%
\pgfpathcurveto{\pgfqpoint{2.308776in}{1.373199in}}{\pgfqpoint{2.316676in}{1.376472in}}{\pgfqpoint{2.322500in}{1.382296in}}%
\pgfpathcurveto{\pgfqpoint{2.328324in}{1.388120in}}{\pgfqpoint{2.331596in}{1.396020in}}{\pgfqpoint{2.331596in}{1.404256in}}%
\pgfpathcurveto{\pgfqpoint{2.331596in}{1.412492in}}{\pgfqpoint{2.328324in}{1.420392in}}{\pgfqpoint{2.322500in}{1.426216in}}%
\pgfpathcurveto{\pgfqpoint{2.316676in}{1.432040in}}{\pgfqpoint{2.308776in}{1.435312in}}{\pgfqpoint{2.300540in}{1.435312in}}%
\pgfpathcurveto{\pgfqpoint{2.292303in}{1.435312in}}{\pgfqpoint{2.284403in}{1.432040in}}{\pgfqpoint{2.278579in}{1.426216in}}%
\pgfpathcurveto{\pgfqpoint{2.272755in}{1.420392in}}{\pgfqpoint{2.269483in}{1.412492in}}{\pgfqpoint{2.269483in}{1.404256in}}%
\pgfpathcurveto{\pgfqpoint{2.269483in}{1.396020in}}{\pgfqpoint{2.272755in}{1.388120in}}{\pgfqpoint{2.278579in}{1.382296in}}%
\pgfpathcurveto{\pgfqpoint{2.284403in}{1.376472in}}{\pgfqpoint{2.292303in}{1.373199in}}{\pgfqpoint{2.300540in}{1.373199in}}%
\pgfpathclose%
\pgfusepath{stroke,fill}%
\end{pgfscope}%
\begin{pgfscope}%
\pgfpathrectangle{\pgfqpoint{0.100000in}{0.220728in}}{\pgfqpoint{3.696000in}{3.696000in}}%
\pgfusepath{clip}%
\pgfsetbuttcap%
\pgfsetroundjoin%
\definecolor{currentfill}{rgb}{0.121569,0.466667,0.705882}%
\pgfsetfillcolor{currentfill}%
\pgfsetfillopacity{0.990875}%
\pgfsetlinewidth{1.003750pt}%
\definecolor{currentstroke}{rgb}{0.121569,0.466667,0.705882}%
\pgfsetstrokecolor{currentstroke}%
\pgfsetstrokeopacity{0.990875}%
\pgfsetdash{}{0pt}%
\pgfpathmoveto{\pgfqpoint{2.407531in}{1.320821in}}%
\pgfpathcurveto{\pgfqpoint{2.415767in}{1.320821in}}{\pgfqpoint{2.423667in}{1.324093in}}{\pgfqpoint{2.429491in}{1.329917in}}%
\pgfpathcurveto{\pgfqpoint{2.435315in}{1.335741in}}{\pgfqpoint{2.438587in}{1.343641in}}{\pgfqpoint{2.438587in}{1.351877in}}%
\pgfpathcurveto{\pgfqpoint{2.438587in}{1.360113in}}{\pgfqpoint{2.435315in}{1.368013in}}{\pgfqpoint{2.429491in}{1.373837in}}%
\pgfpathcurveto{\pgfqpoint{2.423667in}{1.379661in}}{\pgfqpoint{2.415767in}{1.382934in}}{\pgfqpoint{2.407531in}{1.382934in}}%
\pgfpathcurveto{\pgfqpoint{2.399294in}{1.382934in}}{\pgfqpoint{2.391394in}{1.379661in}}{\pgfqpoint{2.385570in}{1.373837in}}%
\pgfpathcurveto{\pgfqpoint{2.379746in}{1.368013in}}{\pgfqpoint{2.376474in}{1.360113in}}{\pgfqpoint{2.376474in}{1.351877in}}%
\pgfpathcurveto{\pgfqpoint{2.376474in}{1.343641in}}{\pgfqpoint{2.379746in}{1.335741in}}{\pgfqpoint{2.385570in}{1.329917in}}%
\pgfpathcurveto{\pgfqpoint{2.391394in}{1.324093in}}{\pgfqpoint{2.399294in}{1.320821in}}{\pgfqpoint{2.407531in}{1.320821in}}%
\pgfpathclose%
\pgfusepath{stroke,fill}%
\end{pgfscope}%
\begin{pgfscope}%
\pgfpathrectangle{\pgfqpoint{0.100000in}{0.220728in}}{\pgfqpoint{3.696000in}{3.696000in}}%
\pgfusepath{clip}%
\pgfsetbuttcap%
\pgfsetroundjoin%
\definecolor{currentfill}{rgb}{0.121569,0.466667,0.705882}%
\pgfsetfillcolor{currentfill}%
\pgfsetfillopacity{0.991099}%
\pgfsetlinewidth{1.003750pt}%
\definecolor{currentstroke}{rgb}{0.121569,0.466667,0.705882}%
\pgfsetstrokecolor{currentstroke}%
\pgfsetstrokeopacity{0.991099}%
\pgfsetdash{}{0pt}%
\pgfpathmoveto{\pgfqpoint{2.310330in}{1.367556in}}%
\pgfpathcurveto{\pgfqpoint{2.318567in}{1.367556in}}{\pgfqpoint{2.326467in}{1.370828in}}{\pgfqpoint{2.332290in}{1.376652in}}%
\pgfpathcurveto{\pgfqpoint{2.338114in}{1.382476in}}{\pgfqpoint{2.341387in}{1.390376in}}{\pgfqpoint{2.341387in}{1.398612in}}%
\pgfpathcurveto{\pgfqpoint{2.341387in}{1.406848in}}{\pgfqpoint{2.338114in}{1.414748in}}{\pgfqpoint{2.332290in}{1.420572in}}%
\pgfpathcurveto{\pgfqpoint{2.326467in}{1.426396in}}{\pgfqpoint{2.318567in}{1.429669in}}{\pgfqpoint{2.310330in}{1.429669in}}%
\pgfpathcurveto{\pgfqpoint{2.302094in}{1.429669in}}{\pgfqpoint{2.294194in}{1.426396in}}{\pgfqpoint{2.288370in}{1.420572in}}%
\pgfpathcurveto{\pgfqpoint{2.282546in}{1.414748in}}{\pgfqpoint{2.279274in}{1.406848in}}{\pgfqpoint{2.279274in}{1.398612in}}%
\pgfpathcurveto{\pgfqpoint{2.279274in}{1.390376in}}{\pgfqpoint{2.282546in}{1.382476in}}{\pgfqpoint{2.288370in}{1.376652in}}%
\pgfpathcurveto{\pgfqpoint{2.294194in}{1.370828in}}{\pgfqpoint{2.302094in}{1.367556in}}{\pgfqpoint{2.310330in}{1.367556in}}%
\pgfpathclose%
\pgfusepath{stroke,fill}%
\end{pgfscope}%
\begin{pgfscope}%
\pgfpathrectangle{\pgfqpoint{0.100000in}{0.220728in}}{\pgfqpoint{3.696000in}{3.696000in}}%
\pgfusepath{clip}%
\pgfsetbuttcap%
\pgfsetroundjoin%
\definecolor{currentfill}{rgb}{0.121569,0.466667,0.705882}%
\pgfsetfillcolor{currentfill}%
\pgfsetfillopacity{0.991910}%
\pgfsetlinewidth{1.003750pt}%
\definecolor{currentstroke}{rgb}{0.121569,0.466667,0.705882}%
\pgfsetstrokecolor{currentstroke}%
\pgfsetstrokeopacity{0.991910}%
\pgfsetdash{}{0pt}%
\pgfpathmoveto{\pgfqpoint{2.316989in}{1.363707in}}%
\pgfpathcurveto{\pgfqpoint{2.325225in}{1.363707in}}{\pgfqpoint{2.333125in}{1.366980in}}{\pgfqpoint{2.338949in}{1.372804in}}%
\pgfpathcurveto{\pgfqpoint{2.344773in}{1.378628in}}{\pgfqpoint{2.348046in}{1.386528in}}{\pgfqpoint{2.348046in}{1.394764in}}%
\pgfpathcurveto{\pgfqpoint{2.348046in}{1.403000in}}{\pgfqpoint{2.344773in}{1.410900in}}{\pgfqpoint{2.338949in}{1.416724in}}%
\pgfpathcurveto{\pgfqpoint{2.333125in}{1.422548in}}{\pgfqpoint{2.325225in}{1.425820in}}{\pgfqpoint{2.316989in}{1.425820in}}%
\pgfpathcurveto{\pgfqpoint{2.308753in}{1.425820in}}{\pgfqpoint{2.300853in}{1.422548in}}{\pgfqpoint{2.295029in}{1.416724in}}%
\pgfpathcurveto{\pgfqpoint{2.289205in}{1.410900in}}{\pgfqpoint{2.285933in}{1.403000in}}{\pgfqpoint{2.285933in}{1.394764in}}%
\pgfpathcurveto{\pgfqpoint{2.285933in}{1.386528in}}{\pgfqpoint{2.289205in}{1.378628in}}{\pgfqpoint{2.295029in}{1.372804in}}%
\pgfpathcurveto{\pgfqpoint{2.300853in}{1.366980in}}{\pgfqpoint{2.308753in}{1.363707in}}{\pgfqpoint{2.316989in}{1.363707in}}%
\pgfpathclose%
\pgfusepath{stroke,fill}%
\end{pgfscope}%
\begin{pgfscope}%
\pgfpathrectangle{\pgfqpoint{0.100000in}{0.220728in}}{\pgfqpoint{3.696000in}{3.696000in}}%
\pgfusepath{clip}%
\pgfsetbuttcap%
\pgfsetroundjoin%
\definecolor{currentfill}{rgb}{0.121569,0.466667,0.705882}%
\pgfsetfillcolor{currentfill}%
\pgfsetfillopacity{0.992568}%
\pgfsetlinewidth{1.003750pt}%
\definecolor{currentstroke}{rgb}{0.121569,0.466667,0.705882}%
\pgfsetstrokecolor{currentstroke}%
\pgfsetstrokeopacity{0.992568}%
\pgfsetdash{}{0pt}%
\pgfpathmoveto{\pgfqpoint{2.322884in}{1.359768in}}%
\pgfpathcurveto{\pgfqpoint{2.331121in}{1.359768in}}{\pgfqpoint{2.339021in}{1.363041in}}{\pgfqpoint{2.344845in}{1.368865in}}%
\pgfpathcurveto{\pgfqpoint{2.350669in}{1.374689in}}{\pgfqpoint{2.353941in}{1.382589in}}{\pgfqpoint{2.353941in}{1.390825in}}%
\pgfpathcurveto{\pgfqpoint{2.353941in}{1.399061in}}{\pgfqpoint{2.350669in}{1.406961in}}{\pgfqpoint{2.344845in}{1.412785in}}%
\pgfpathcurveto{\pgfqpoint{2.339021in}{1.418609in}}{\pgfqpoint{2.331121in}{1.421881in}}{\pgfqpoint{2.322884in}{1.421881in}}%
\pgfpathcurveto{\pgfqpoint{2.314648in}{1.421881in}}{\pgfqpoint{2.306748in}{1.418609in}}{\pgfqpoint{2.300924in}{1.412785in}}%
\pgfpathcurveto{\pgfqpoint{2.295100in}{1.406961in}}{\pgfqpoint{2.291828in}{1.399061in}}{\pgfqpoint{2.291828in}{1.390825in}}%
\pgfpathcurveto{\pgfqpoint{2.291828in}{1.382589in}}{\pgfqpoint{2.295100in}{1.374689in}}{\pgfqpoint{2.300924in}{1.368865in}}%
\pgfpathcurveto{\pgfqpoint{2.306748in}{1.363041in}}{\pgfqpoint{2.314648in}{1.359768in}}{\pgfqpoint{2.322884in}{1.359768in}}%
\pgfpathclose%
\pgfusepath{stroke,fill}%
\end{pgfscope}%
\begin{pgfscope}%
\pgfpathrectangle{\pgfqpoint{0.100000in}{0.220728in}}{\pgfqpoint{3.696000in}{3.696000in}}%
\pgfusepath{clip}%
\pgfsetbuttcap%
\pgfsetroundjoin%
\definecolor{currentfill}{rgb}{0.121569,0.466667,0.705882}%
\pgfsetfillcolor{currentfill}%
\pgfsetfillopacity{0.993013}%
\pgfsetlinewidth{1.003750pt}%
\definecolor{currentstroke}{rgb}{0.121569,0.466667,0.705882}%
\pgfsetstrokecolor{currentstroke}%
\pgfsetstrokeopacity{0.993013}%
\pgfsetdash{}{0pt}%
\pgfpathmoveto{\pgfqpoint{2.327316in}{1.356770in}}%
\pgfpathcurveto{\pgfqpoint{2.335553in}{1.356770in}}{\pgfqpoint{2.343453in}{1.360042in}}{\pgfqpoint{2.349277in}{1.365866in}}%
\pgfpathcurveto{\pgfqpoint{2.355100in}{1.371690in}}{\pgfqpoint{2.358373in}{1.379590in}}{\pgfqpoint{2.358373in}{1.387826in}}%
\pgfpathcurveto{\pgfqpoint{2.358373in}{1.396063in}}{\pgfqpoint{2.355100in}{1.403963in}}{\pgfqpoint{2.349277in}{1.409787in}}%
\pgfpathcurveto{\pgfqpoint{2.343453in}{1.415610in}}{\pgfqpoint{2.335553in}{1.418883in}}{\pgfqpoint{2.327316in}{1.418883in}}%
\pgfpathcurveto{\pgfqpoint{2.319080in}{1.418883in}}{\pgfqpoint{2.311180in}{1.415610in}}{\pgfqpoint{2.305356in}{1.409787in}}%
\pgfpathcurveto{\pgfqpoint{2.299532in}{1.403963in}}{\pgfqpoint{2.296260in}{1.396063in}}{\pgfqpoint{2.296260in}{1.387826in}}%
\pgfpathcurveto{\pgfqpoint{2.296260in}{1.379590in}}{\pgfqpoint{2.299532in}{1.371690in}}{\pgfqpoint{2.305356in}{1.365866in}}%
\pgfpathcurveto{\pgfqpoint{2.311180in}{1.360042in}}{\pgfqpoint{2.319080in}{1.356770in}}{\pgfqpoint{2.327316in}{1.356770in}}%
\pgfpathclose%
\pgfusepath{stroke,fill}%
\end{pgfscope}%
\begin{pgfscope}%
\pgfpathrectangle{\pgfqpoint{0.100000in}{0.220728in}}{\pgfqpoint{3.696000in}{3.696000in}}%
\pgfusepath{clip}%
\pgfsetbuttcap%
\pgfsetroundjoin%
\definecolor{currentfill}{rgb}{0.121569,0.466667,0.705882}%
\pgfsetfillcolor{currentfill}%
\pgfsetfillopacity{0.993675}%
\pgfsetlinewidth{1.003750pt}%
\definecolor{currentstroke}{rgb}{0.121569,0.466667,0.705882}%
\pgfsetstrokecolor{currentstroke}%
\pgfsetstrokeopacity{0.993675}%
\pgfsetdash{}{0pt}%
\pgfpathmoveto{\pgfqpoint{2.335668in}{1.351444in}}%
\pgfpathcurveto{\pgfqpoint{2.343905in}{1.351444in}}{\pgfqpoint{2.351805in}{1.354716in}}{\pgfqpoint{2.357628in}{1.360540in}}%
\pgfpathcurveto{\pgfqpoint{2.363452in}{1.366364in}}{\pgfqpoint{2.366725in}{1.374264in}}{\pgfqpoint{2.366725in}{1.382501in}}%
\pgfpathcurveto{\pgfqpoint{2.366725in}{1.390737in}}{\pgfqpoint{2.363452in}{1.398637in}}{\pgfqpoint{2.357628in}{1.404461in}}%
\pgfpathcurveto{\pgfqpoint{2.351805in}{1.410285in}}{\pgfqpoint{2.343905in}{1.413557in}}{\pgfqpoint{2.335668in}{1.413557in}}%
\pgfpathcurveto{\pgfqpoint{2.327432in}{1.413557in}}{\pgfqpoint{2.319532in}{1.410285in}}{\pgfqpoint{2.313708in}{1.404461in}}%
\pgfpathcurveto{\pgfqpoint{2.307884in}{1.398637in}}{\pgfqpoint{2.304612in}{1.390737in}}{\pgfqpoint{2.304612in}{1.382501in}}%
\pgfpathcurveto{\pgfqpoint{2.304612in}{1.374264in}}{\pgfqpoint{2.307884in}{1.366364in}}{\pgfqpoint{2.313708in}{1.360540in}}%
\pgfpathcurveto{\pgfqpoint{2.319532in}{1.354716in}}{\pgfqpoint{2.327432in}{1.351444in}}{\pgfqpoint{2.335668in}{1.351444in}}%
\pgfpathclose%
\pgfusepath{stroke,fill}%
\end{pgfscope}%
\begin{pgfscope}%
\pgfpathrectangle{\pgfqpoint{0.100000in}{0.220728in}}{\pgfqpoint{3.696000in}{3.696000in}}%
\pgfusepath{clip}%
\pgfsetbuttcap%
\pgfsetroundjoin%
\definecolor{currentfill}{rgb}{0.121569,0.466667,0.705882}%
\pgfsetfillcolor{currentfill}%
\pgfsetfillopacity{0.993774}%
\pgfsetlinewidth{1.003750pt}%
\definecolor{currentstroke}{rgb}{0.121569,0.466667,0.705882}%
\pgfsetstrokecolor{currentstroke}%
\pgfsetstrokeopacity{0.993774}%
\pgfsetdash{}{0pt}%
\pgfpathmoveto{\pgfqpoint{2.409055in}{1.316919in}}%
\pgfpathcurveto{\pgfqpoint{2.417291in}{1.316919in}}{\pgfqpoint{2.425191in}{1.320191in}}{\pgfqpoint{2.431015in}{1.326015in}}%
\pgfpathcurveto{\pgfqpoint{2.436839in}{1.331839in}}{\pgfqpoint{2.440111in}{1.339739in}}{\pgfqpoint{2.440111in}{1.347976in}}%
\pgfpathcurveto{\pgfqpoint{2.440111in}{1.356212in}}{\pgfqpoint{2.436839in}{1.364112in}}{\pgfqpoint{2.431015in}{1.369936in}}%
\pgfpathcurveto{\pgfqpoint{2.425191in}{1.375760in}}{\pgfqpoint{2.417291in}{1.379032in}}{\pgfqpoint{2.409055in}{1.379032in}}%
\pgfpathcurveto{\pgfqpoint{2.400819in}{1.379032in}}{\pgfqpoint{2.392919in}{1.375760in}}{\pgfqpoint{2.387095in}{1.369936in}}%
\pgfpathcurveto{\pgfqpoint{2.381271in}{1.364112in}}{\pgfqpoint{2.377998in}{1.356212in}}{\pgfqpoint{2.377998in}{1.347976in}}%
\pgfpathcurveto{\pgfqpoint{2.377998in}{1.339739in}}{\pgfqpoint{2.381271in}{1.331839in}}{\pgfqpoint{2.387095in}{1.326015in}}%
\pgfpathcurveto{\pgfqpoint{2.392919in}{1.320191in}}{\pgfqpoint{2.400819in}{1.316919in}}{\pgfqpoint{2.409055in}{1.316919in}}%
\pgfpathclose%
\pgfusepath{stroke,fill}%
\end{pgfscope}%
\begin{pgfscope}%
\pgfpathrectangle{\pgfqpoint{0.100000in}{0.220728in}}{\pgfqpoint{3.696000in}{3.696000in}}%
\pgfusepath{clip}%
\pgfsetbuttcap%
\pgfsetroundjoin%
\definecolor{currentfill}{rgb}{0.121569,0.466667,0.705882}%
\pgfsetfillcolor{currentfill}%
\pgfsetfillopacity{0.994332}%
\pgfsetlinewidth{1.003750pt}%
\definecolor{currentstroke}{rgb}{0.121569,0.466667,0.705882}%
\pgfsetstrokecolor{currentstroke}%
\pgfsetstrokeopacity{0.994332}%
\pgfsetdash{}{0pt}%
\pgfpathmoveto{\pgfqpoint{2.340560in}{1.350074in}}%
\pgfpathcurveto{\pgfqpoint{2.348796in}{1.350074in}}{\pgfqpoint{2.356696in}{1.353347in}}{\pgfqpoint{2.362520in}{1.359171in}}%
\pgfpathcurveto{\pgfqpoint{2.368344in}{1.364995in}}{\pgfqpoint{2.371616in}{1.372895in}}{\pgfqpoint{2.371616in}{1.381131in}}%
\pgfpathcurveto{\pgfqpoint{2.371616in}{1.389367in}}{\pgfqpoint{2.368344in}{1.397267in}}{\pgfqpoint{2.362520in}{1.403091in}}%
\pgfpathcurveto{\pgfqpoint{2.356696in}{1.408915in}}{\pgfqpoint{2.348796in}{1.412187in}}{\pgfqpoint{2.340560in}{1.412187in}}%
\pgfpathcurveto{\pgfqpoint{2.332324in}{1.412187in}}{\pgfqpoint{2.324424in}{1.408915in}}{\pgfqpoint{2.318600in}{1.403091in}}%
\pgfpathcurveto{\pgfqpoint{2.312776in}{1.397267in}}{\pgfqpoint{2.309503in}{1.389367in}}{\pgfqpoint{2.309503in}{1.381131in}}%
\pgfpathcurveto{\pgfqpoint{2.309503in}{1.372895in}}{\pgfqpoint{2.312776in}{1.364995in}}{\pgfqpoint{2.318600in}{1.359171in}}%
\pgfpathcurveto{\pgfqpoint{2.324424in}{1.353347in}}{\pgfqpoint{2.332324in}{1.350074in}}{\pgfqpoint{2.340560in}{1.350074in}}%
\pgfpathclose%
\pgfusepath{stroke,fill}%
\end{pgfscope}%
\begin{pgfscope}%
\pgfpathrectangle{\pgfqpoint{0.100000in}{0.220728in}}{\pgfqpoint{3.696000in}{3.696000in}}%
\pgfusepath{clip}%
\pgfsetbuttcap%
\pgfsetroundjoin%
\definecolor{currentfill}{rgb}{0.121569,0.466667,0.705882}%
\pgfsetfillcolor{currentfill}%
\pgfsetfillopacity{0.994624}%
\pgfsetlinewidth{1.003750pt}%
\definecolor{currentstroke}{rgb}{0.121569,0.466667,0.705882}%
\pgfsetstrokecolor{currentstroke}%
\pgfsetstrokeopacity{0.994624}%
\pgfsetdash{}{0pt}%
\pgfpathmoveto{\pgfqpoint{2.344405in}{1.345931in}}%
\pgfpathcurveto{\pgfqpoint{2.352642in}{1.345931in}}{\pgfqpoint{2.360542in}{1.349203in}}{\pgfqpoint{2.366366in}{1.355027in}}%
\pgfpathcurveto{\pgfqpoint{2.372190in}{1.360851in}}{\pgfqpoint{2.375462in}{1.368751in}}{\pgfqpoint{2.375462in}{1.376988in}}%
\pgfpathcurveto{\pgfqpoint{2.375462in}{1.385224in}}{\pgfqpoint{2.372190in}{1.393124in}}{\pgfqpoint{2.366366in}{1.398948in}}%
\pgfpathcurveto{\pgfqpoint{2.360542in}{1.404772in}}{\pgfqpoint{2.352642in}{1.408044in}}{\pgfqpoint{2.344405in}{1.408044in}}%
\pgfpathcurveto{\pgfqpoint{2.336169in}{1.408044in}}{\pgfqpoint{2.328269in}{1.404772in}}{\pgfqpoint{2.322445in}{1.398948in}}%
\pgfpathcurveto{\pgfqpoint{2.316621in}{1.393124in}}{\pgfqpoint{2.313349in}{1.385224in}}{\pgfqpoint{2.313349in}{1.376988in}}%
\pgfpathcurveto{\pgfqpoint{2.313349in}{1.368751in}}{\pgfqpoint{2.316621in}{1.360851in}}{\pgfqpoint{2.322445in}{1.355027in}}%
\pgfpathcurveto{\pgfqpoint{2.328269in}{1.349203in}}{\pgfqpoint{2.336169in}{1.345931in}}{\pgfqpoint{2.344405in}{1.345931in}}%
\pgfpathclose%
\pgfusepath{stroke,fill}%
\end{pgfscope}%
\begin{pgfscope}%
\pgfpathrectangle{\pgfqpoint{0.100000in}{0.220728in}}{\pgfqpoint{3.696000in}{3.696000in}}%
\pgfusepath{clip}%
\pgfsetbuttcap%
\pgfsetroundjoin%
\definecolor{currentfill}{rgb}{0.121569,0.466667,0.705882}%
\pgfsetfillcolor{currentfill}%
\pgfsetfillopacity{0.994741}%
\pgfsetlinewidth{1.003750pt}%
\definecolor{currentstroke}{rgb}{0.121569,0.466667,0.705882}%
\pgfsetstrokecolor{currentstroke}%
\pgfsetstrokeopacity{0.994741}%
\pgfsetdash{}{0pt}%
\pgfpathmoveto{\pgfqpoint{2.346199in}{1.344603in}}%
\pgfpathcurveto{\pgfqpoint{2.354435in}{1.344603in}}{\pgfqpoint{2.362335in}{1.347876in}}{\pgfqpoint{2.368159in}{1.353700in}}%
\pgfpathcurveto{\pgfqpoint{2.373983in}{1.359524in}}{\pgfqpoint{2.377255in}{1.367424in}}{\pgfqpoint{2.377255in}{1.375660in}}%
\pgfpathcurveto{\pgfqpoint{2.377255in}{1.383896in}}{\pgfqpoint{2.373983in}{1.391796in}}{\pgfqpoint{2.368159in}{1.397620in}}%
\pgfpathcurveto{\pgfqpoint{2.362335in}{1.403444in}}{\pgfqpoint{2.354435in}{1.406716in}}{\pgfqpoint{2.346199in}{1.406716in}}%
\pgfpathcurveto{\pgfqpoint{2.337963in}{1.406716in}}{\pgfqpoint{2.330063in}{1.403444in}}{\pgfqpoint{2.324239in}{1.397620in}}%
\pgfpathcurveto{\pgfqpoint{2.318415in}{1.391796in}}{\pgfqpoint{2.315142in}{1.383896in}}{\pgfqpoint{2.315142in}{1.375660in}}%
\pgfpathcurveto{\pgfqpoint{2.315142in}{1.367424in}}{\pgfqpoint{2.318415in}{1.359524in}}{\pgfqpoint{2.324239in}{1.353700in}}%
\pgfpathcurveto{\pgfqpoint{2.330063in}{1.347876in}}{\pgfqpoint{2.337963in}{1.344603in}}{\pgfqpoint{2.346199in}{1.344603in}}%
\pgfpathclose%
\pgfusepath{stroke,fill}%
\end{pgfscope}%
\begin{pgfscope}%
\pgfpathrectangle{\pgfqpoint{0.100000in}{0.220728in}}{\pgfqpoint{3.696000in}{3.696000in}}%
\pgfusepath{clip}%
\pgfsetbuttcap%
\pgfsetroundjoin%
\definecolor{currentfill}{rgb}{0.121569,0.466667,0.705882}%
\pgfsetfillcolor{currentfill}%
\pgfsetfillopacity{0.995088}%
\pgfsetlinewidth{1.003750pt}%
\definecolor{currentstroke}{rgb}{0.121569,0.466667,0.705882}%
\pgfsetstrokecolor{currentstroke}%
\pgfsetstrokeopacity{0.995088}%
\pgfsetdash{}{0pt}%
\pgfpathmoveto{\pgfqpoint{2.349447in}{1.342933in}}%
\pgfpathcurveto{\pgfqpoint{2.357683in}{1.342933in}}{\pgfqpoint{2.365584in}{1.346205in}}{\pgfqpoint{2.371407in}{1.352029in}}%
\pgfpathcurveto{\pgfqpoint{2.377231in}{1.357853in}}{\pgfqpoint{2.380504in}{1.365753in}}{\pgfqpoint{2.380504in}{1.373989in}}%
\pgfpathcurveto{\pgfqpoint{2.380504in}{1.382226in}}{\pgfqpoint{2.377231in}{1.390126in}}{\pgfqpoint{2.371407in}{1.395950in}}%
\pgfpathcurveto{\pgfqpoint{2.365584in}{1.401774in}}{\pgfqpoint{2.357683in}{1.405046in}}{\pgfqpoint{2.349447in}{1.405046in}}%
\pgfpathcurveto{\pgfqpoint{2.341211in}{1.405046in}}{\pgfqpoint{2.333311in}{1.401774in}}{\pgfqpoint{2.327487in}{1.395950in}}%
\pgfpathcurveto{\pgfqpoint{2.321663in}{1.390126in}}{\pgfqpoint{2.318391in}{1.382226in}}{\pgfqpoint{2.318391in}{1.373989in}}%
\pgfpathcurveto{\pgfqpoint{2.318391in}{1.365753in}}{\pgfqpoint{2.321663in}{1.357853in}}{\pgfqpoint{2.327487in}{1.352029in}}%
\pgfpathcurveto{\pgfqpoint{2.333311in}{1.346205in}}{\pgfqpoint{2.341211in}{1.342933in}}{\pgfqpoint{2.349447in}{1.342933in}}%
\pgfpathclose%
\pgfusepath{stroke,fill}%
\end{pgfscope}%
\begin{pgfscope}%
\pgfpathrectangle{\pgfqpoint{0.100000in}{0.220728in}}{\pgfqpoint{3.696000in}{3.696000in}}%
\pgfusepath{clip}%
\pgfsetbuttcap%
\pgfsetroundjoin%
\definecolor{currentfill}{rgb}{0.121569,0.466667,0.705882}%
\pgfsetfillcolor{currentfill}%
\pgfsetfillopacity{0.995214}%
\pgfsetlinewidth{1.003750pt}%
\definecolor{currentstroke}{rgb}{0.121569,0.466667,0.705882}%
\pgfsetstrokecolor{currentstroke}%
\pgfsetstrokeopacity{0.995214}%
\pgfsetdash{}{0pt}%
\pgfpathmoveto{\pgfqpoint{2.350586in}{1.342322in}}%
\pgfpathcurveto{\pgfqpoint{2.358823in}{1.342322in}}{\pgfqpoint{2.366723in}{1.345595in}}{\pgfqpoint{2.372547in}{1.351419in}}%
\pgfpathcurveto{\pgfqpoint{2.378371in}{1.357243in}}{\pgfqpoint{2.381643in}{1.365143in}}{\pgfqpoint{2.381643in}{1.373379in}}%
\pgfpathcurveto{\pgfqpoint{2.381643in}{1.381615in}}{\pgfqpoint{2.378371in}{1.389515in}}{\pgfqpoint{2.372547in}{1.395339in}}%
\pgfpathcurveto{\pgfqpoint{2.366723in}{1.401163in}}{\pgfqpoint{2.358823in}{1.404435in}}{\pgfqpoint{2.350586in}{1.404435in}}%
\pgfpathcurveto{\pgfqpoint{2.342350in}{1.404435in}}{\pgfqpoint{2.334450in}{1.401163in}}{\pgfqpoint{2.328626in}{1.395339in}}%
\pgfpathcurveto{\pgfqpoint{2.322802in}{1.389515in}}{\pgfqpoint{2.319530in}{1.381615in}}{\pgfqpoint{2.319530in}{1.373379in}}%
\pgfpathcurveto{\pgfqpoint{2.319530in}{1.365143in}}{\pgfqpoint{2.322802in}{1.357243in}}{\pgfqpoint{2.328626in}{1.351419in}}%
\pgfpathcurveto{\pgfqpoint{2.334450in}{1.345595in}}{\pgfqpoint{2.342350in}{1.342322in}}{\pgfqpoint{2.350586in}{1.342322in}}%
\pgfpathclose%
\pgfusepath{stroke,fill}%
\end{pgfscope}%
\begin{pgfscope}%
\pgfpathrectangle{\pgfqpoint{0.100000in}{0.220728in}}{\pgfqpoint{3.696000in}{3.696000in}}%
\pgfusepath{clip}%
\pgfsetbuttcap%
\pgfsetroundjoin%
\definecolor{currentfill}{rgb}{0.121569,0.466667,0.705882}%
\pgfsetfillcolor{currentfill}%
\pgfsetfillopacity{0.995236}%
\pgfsetlinewidth{1.003750pt}%
\definecolor{currentstroke}{rgb}{0.121569,0.466667,0.705882}%
\pgfsetstrokecolor{currentstroke}%
\pgfsetstrokeopacity{0.995236}%
\pgfsetdash{}{0pt}%
\pgfpathmoveto{\pgfqpoint{2.350822in}{1.342165in}}%
\pgfpathcurveto{\pgfqpoint{2.359058in}{1.342165in}}{\pgfqpoint{2.366958in}{1.345438in}}{\pgfqpoint{2.372782in}{1.351262in}}%
\pgfpathcurveto{\pgfqpoint{2.378606in}{1.357085in}}{\pgfqpoint{2.381878in}{1.364986in}}{\pgfqpoint{2.381878in}{1.373222in}}%
\pgfpathcurveto{\pgfqpoint{2.381878in}{1.381458in}}{\pgfqpoint{2.378606in}{1.389358in}}{\pgfqpoint{2.372782in}{1.395182in}}%
\pgfpathcurveto{\pgfqpoint{2.366958in}{1.401006in}}{\pgfqpoint{2.359058in}{1.404278in}}{\pgfqpoint{2.350822in}{1.404278in}}%
\pgfpathcurveto{\pgfqpoint{2.342586in}{1.404278in}}{\pgfqpoint{2.334685in}{1.401006in}}{\pgfqpoint{2.328862in}{1.395182in}}%
\pgfpathcurveto{\pgfqpoint{2.323038in}{1.389358in}}{\pgfqpoint{2.319765in}{1.381458in}}{\pgfqpoint{2.319765in}{1.373222in}}%
\pgfpathcurveto{\pgfqpoint{2.319765in}{1.364986in}}{\pgfqpoint{2.323038in}{1.357085in}}{\pgfqpoint{2.328862in}{1.351262in}}%
\pgfpathcurveto{\pgfqpoint{2.334685in}{1.345438in}}{\pgfqpoint{2.342586in}{1.342165in}}{\pgfqpoint{2.350822in}{1.342165in}}%
\pgfpathclose%
\pgfusepath{stroke,fill}%
\end{pgfscope}%
\begin{pgfscope}%
\pgfpathrectangle{\pgfqpoint{0.100000in}{0.220728in}}{\pgfqpoint{3.696000in}{3.696000in}}%
\pgfusepath{clip}%
\pgfsetbuttcap%
\pgfsetroundjoin%
\definecolor{currentfill}{rgb}{0.121569,0.466667,0.705882}%
\pgfsetfillcolor{currentfill}%
\pgfsetfillopacity{0.995284}%
\pgfsetlinewidth{1.003750pt}%
\definecolor{currentstroke}{rgb}{0.121569,0.466667,0.705882}%
\pgfsetstrokecolor{currentstroke}%
\pgfsetstrokeopacity{0.995284}%
\pgfsetdash{}{0pt}%
\pgfpathmoveto{\pgfqpoint{2.351243in}{1.341900in}}%
\pgfpathcurveto{\pgfqpoint{2.359479in}{1.341900in}}{\pgfqpoint{2.367379in}{1.345172in}}{\pgfqpoint{2.373203in}{1.350996in}}%
\pgfpathcurveto{\pgfqpoint{2.379027in}{1.356820in}}{\pgfqpoint{2.382300in}{1.364720in}}{\pgfqpoint{2.382300in}{1.372956in}}%
\pgfpathcurveto{\pgfqpoint{2.382300in}{1.381192in}}{\pgfqpoint{2.379027in}{1.389092in}}{\pgfqpoint{2.373203in}{1.394916in}}%
\pgfpathcurveto{\pgfqpoint{2.367379in}{1.400740in}}{\pgfqpoint{2.359479in}{1.404013in}}{\pgfqpoint{2.351243in}{1.404013in}}%
\pgfpathcurveto{\pgfqpoint{2.343007in}{1.404013in}}{\pgfqpoint{2.335107in}{1.400740in}}{\pgfqpoint{2.329283in}{1.394916in}}%
\pgfpathcurveto{\pgfqpoint{2.323459in}{1.389092in}}{\pgfqpoint{2.320187in}{1.381192in}}{\pgfqpoint{2.320187in}{1.372956in}}%
\pgfpathcurveto{\pgfqpoint{2.320187in}{1.364720in}}{\pgfqpoint{2.323459in}{1.356820in}}{\pgfqpoint{2.329283in}{1.350996in}}%
\pgfpathcurveto{\pgfqpoint{2.335107in}{1.345172in}}{\pgfqpoint{2.343007in}{1.341900in}}{\pgfqpoint{2.351243in}{1.341900in}}%
\pgfpathclose%
\pgfusepath{stroke,fill}%
\end{pgfscope}%
\begin{pgfscope}%
\pgfpathrectangle{\pgfqpoint{0.100000in}{0.220728in}}{\pgfqpoint{3.696000in}{3.696000in}}%
\pgfusepath{clip}%
\pgfsetbuttcap%
\pgfsetroundjoin%
\definecolor{currentfill}{rgb}{0.121569,0.466667,0.705882}%
\pgfsetfillcolor{currentfill}%
\pgfsetfillopacity{0.995384}%
\pgfsetlinewidth{1.003750pt}%
\definecolor{currentstroke}{rgb}{0.121569,0.466667,0.705882}%
\pgfsetstrokecolor{currentstroke}%
\pgfsetstrokeopacity{0.995384}%
\pgfsetdash{}{0pt}%
\pgfpathmoveto{\pgfqpoint{2.351981in}{1.341408in}}%
\pgfpathcurveto{\pgfqpoint{2.360218in}{1.341408in}}{\pgfqpoint{2.368118in}{1.344680in}}{\pgfqpoint{2.373942in}{1.350504in}}%
\pgfpathcurveto{\pgfqpoint{2.379766in}{1.356328in}}{\pgfqpoint{2.383038in}{1.364228in}}{\pgfqpoint{2.383038in}{1.372464in}}%
\pgfpathcurveto{\pgfqpoint{2.383038in}{1.380701in}}{\pgfqpoint{2.379766in}{1.388601in}}{\pgfqpoint{2.373942in}{1.394425in}}%
\pgfpathcurveto{\pgfqpoint{2.368118in}{1.400249in}}{\pgfqpoint{2.360218in}{1.403521in}}{\pgfqpoint{2.351981in}{1.403521in}}%
\pgfpathcurveto{\pgfqpoint{2.343745in}{1.403521in}}{\pgfqpoint{2.335845in}{1.400249in}}{\pgfqpoint{2.330021in}{1.394425in}}%
\pgfpathcurveto{\pgfqpoint{2.324197in}{1.388601in}}{\pgfqpoint{2.320925in}{1.380701in}}{\pgfqpoint{2.320925in}{1.372464in}}%
\pgfpathcurveto{\pgfqpoint{2.320925in}{1.364228in}}{\pgfqpoint{2.324197in}{1.356328in}}{\pgfqpoint{2.330021in}{1.350504in}}%
\pgfpathcurveto{\pgfqpoint{2.335845in}{1.344680in}}{\pgfqpoint{2.343745in}{1.341408in}}{\pgfqpoint{2.351981in}{1.341408in}}%
\pgfpathclose%
\pgfusepath{stroke,fill}%
\end{pgfscope}%
\begin{pgfscope}%
\pgfpathrectangle{\pgfqpoint{0.100000in}{0.220728in}}{\pgfqpoint{3.696000in}{3.696000in}}%
\pgfusepath{clip}%
\pgfsetbuttcap%
\pgfsetroundjoin%
\definecolor{currentfill}{rgb}{0.121569,0.466667,0.705882}%
\pgfsetfillcolor{currentfill}%
\pgfsetfillopacity{0.995556}%
\pgfsetlinewidth{1.003750pt}%
\definecolor{currentstroke}{rgb}{0.121569,0.466667,0.705882}%
\pgfsetstrokecolor{currentstroke}%
\pgfsetstrokeopacity{0.995556}%
\pgfsetdash{}{0pt}%
\pgfpathmoveto{\pgfqpoint{2.353301in}{1.340393in}}%
\pgfpathcurveto{\pgfqpoint{2.361537in}{1.340393in}}{\pgfqpoint{2.369437in}{1.343665in}}{\pgfqpoint{2.375261in}{1.349489in}}%
\pgfpathcurveto{\pgfqpoint{2.381085in}{1.355313in}}{\pgfqpoint{2.384357in}{1.363213in}}{\pgfqpoint{2.384357in}{1.371449in}}%
\pgfpathcurveto{\pgfqpoint{2.384357in}{1.379686in}}{\pgfqpoint{2.381085in}{1.387586in}}{\pgfqpoint{2.375261in}{1.393410in}}%
\pgfpathcurveto{\pgfqpoint{2.369437in}{1.399234in}}{\pgfqpoint{2.361537in}{1.402506in}}{\pgfqpoint{2.353301in}{1.402506in}}%
\pgfpathcurveto{\pgfqpoint{2.345064in}{1.402506in}}{\pgfqpoint{2.337164in}{1.399234in}}{\pgfqpoint{2.331340in}{1.393410in}}%
\pgfpathcurveto{\pgfqpoint{2.325516in}{1.387586in}}{\pgfqpoint{2.322244in}{1.379686in}}{\pgfqpoint{2.322244in}{1.371449in}}%
\pgfpathcurveto{\pgfqpoint{2.322244in}{1.363213in}}{\pgfqpoint{2.325516in}{1.355313in}}{\pgfqpoint{2.331340in}{1.349489in}}%
\pgfpathcurveto{\pgfqpoint{2.337164in}{1.343665in}}{\pgfqpoint{2.345064in}{1.340393in}}{\pgfqpoint{2.353301in}{1.340393in}}%
\pgfpathclose%
\pgfusepath{stroke,fill}%
\end{pgfscope}%
\begin{pgfscope}%
\pgfpathrectangle{\pgfqpoint{0.100000in}{0.220728in}}{\pgfqpoint{3.696000in}{3.696000in}}%
\pgfusepath{clip}%
\pgfsetbuttcap%
\pgfsetroundjoin%
\definecolor{currentfill}{rgb}{0.121569,0.466667,0.705882}%
\pgfsetfillcolor{currentfill}%
\pgfsetfillopacity{0.995823}%
\pgfsetlinewidth{1.003750pt}%
\definecolor{currentstroke}{rgb}{0.121569,0.466667,0.705882}%
\pgfsetstrokecolor{currentstroke}%
\pgfsetstrokeopacity{0.995823}%
\pgfsetdash{}{0pt}%
\pgfpathmoveto{\pgfqpoint{2.355736in}{1.338386in}}%
\pgfpathcurveto{\pgfqpoint{2.363972in}{1.338386in}}{\pgfqpoint{2.371872in}{1.341658in}}{\pgfqpoint{2.377696in}{1.347482in}}%
\pgfpathcurveto{\pgfqpoint{2.383520in}{1.353306in}}{\pgfqpoint{2.386792in}{1.361206in}}{\pgfqpoint{2.386792in}{1.369442in}}%
\pgfpathcurveto{\pgfqpoint{2.386792in}{1.377679in}}{\pgfqpoint{2.383520in}{1.385579in}}{\pgfqpoint{2.377696in}{1.391403in}}%
\pgfpathcurveto{\pgfqpoint{2.371872in}{1.397227in}}{\pgfqpoint{2.363972in}{1.400499in}}{\pgfqpoint{2.355736in}{1.400499in}}%
\pgfpathcurveto{\pgfqpoint{2.347499in}{1.400499in}}{\pgfqpoint{2.339599in}{1.397227in}}{\pgfqpoint{2.333775in}{1.391403in}}%
\pgfpathcurveto{\pgfqpoint{2.327951in}{1.385579in}}{\pgfqpoint{2.324679in}{1.377679in}}{\pgfqpoint{2.324679in}{1.369442in}}%
\pgfpathcurveto{\pgfqpoint{2.324679in}{1.361206in}}{\pgfqpoint{2.327951in}{1.353306in}}{\pgfqpoint{2.333775in}{1.347482in}}%
\pgfpathcurveto{\pgfqpoint{2.339599in}{1.341658in}}{\pgfqpoint{2.347499in}{1.338386in}}{\pgfqpoint{2.355736in}{1.338386in}}%
\pgfpathclose%
\pgfusepath{stroke,fill}%
\end{pgfscope}%
\begin{pgfscope}%
\pgfpathrectangle{\pgfqpoint{0.100000in}{0.220728in}}{\pgfqpoint{3.696000in}{3.696000in}}%
\pgfusepath{clip}%
\pgfsetbuttcap%
\pgfsetroundjoin%
\definecolor{currentfill}{rgb}{0.121569,0.466667,0.705882}%
\pgfsetfillcolor{currentfill}%
\pgfsetfillopacity{0.996360}%
\pgfsetlinewidth{1.003750pt}%
\definecolor{currentstroke}{rgb}{0.121569,0.466667,0.705882}%
\pgfsetstrokecolor{currentstroke}%
\pgfsetstrokeopacity{0.996360}%
\pgfsetdash{}{0pt}%
\pgfpathmoveto{\pgfqpoint{2.360146in}{1.334985in}}%
\pgfpathcurveto{\pgfqpoint{2.368382in}{1.334985in}}{\pgfqpoint{2.376282in}{1.338258in}}{\pgfqpoint{2.382106in}{1.344082in}}%
\pgfpathcurveto{\pgfqpoint{2.387930in}{1.349905in}}{\pgfqpoint{2.391202in}{1.357805in}}{\pgfqpoint{2.391202in}{1.366042in}}%
\pgfpathcurveto{\pgfqpoint{2.391202in}{1.374278in}}{\pgfqpoint{2.387930in}{1.382178in}}{\pgfqpoint{2.382106in}{1.388002in}}%
\pgfpathcurveto{\pgfqpoint{2.376282in}{1.393826in}}{\pgfqpoint{2.368382in}{1.397098in}}{\pgfqpoint{2.360146in}{1.397098in}}%
\pgfpathcurveto{\pgfqpoint{2.351909in}{1.397098in}}{\pgfqpoint{2.344009in}{1.393826in}}{\pgfqpoint{2.338185in}{1.388002in}}%
\pgfpathcurveto{\pgfqpoint{2.332361in}{1.382178in}}{\pgfqpoint{2.329089in}{1.374278in}}{\pgfqpoint{2.329089in}{1.366042in}}%
\pgfpathcurveto{\pgfqpoint{2.329089in}{1.357805in}}{\pgfqpoint{2.332361in}{1.349905in}}{\pgfqpoint{2.338185in}{1.344082in}}%
\pgfpathcurveto{\pgfqpoint{2.344009in}{1.338258in}}{\pgfqpoint{2.351909in}{1.334985in}}{\pgfqpoint{2.360146in}{1.334985in}}%
\pgfpathclose%
\pgfusepath{stroke,fill}%
\end{pgfscope}%
\begin{pgfscope}%
\pgfpathrectangle{\pgfqpoint{0.100000in}{0.220728in}}{\pgfqpoint{3.696000in}{3.696000in}}%
\pgfusepath{clip}%
\pgfsetbuttcap%
\pgfsetroundjoin%
\definecolor{currentfill}{rgb}{0.121569,0.466667,0.705882}%
\pgfsetfillcolor{currentfill}%
\pgfsetfillopacity{0.996402}%
\pgfsetlinewidth{1.003750pt}%
\definecolor{currentstroke}{rgb}{0.121569,0.466667,0.705882}%
\pgfsetstrokecolor{currentstroke}%
\pgfsetstrokeopacity{0.996402}%
\pgfsetdash{}{0pt}%
\pgfpathmoveto{\pgfqpoint{2.408486in}{1.309552in}}%
\pgfpathcurveto{\pgfqpoint{2.416722in}{1.309552in}}{\pgfqpoint{2.424622in}{1.312824in}}{\pgfqpoint{2.430446in}{1.318648in}}%
\pgfpathcurveto{\pgfqpoint{2.436270in}{1.324472in}}{\pgfqpoint{2.439542in}{1.332372in}}{\pgfqpoint{2.439542in}{1.340608in}}%
\pgfpathcurveto{\pgfqpoint{2.439542in}{1.348844in}}{\pgfqpoint{2.436270in}{1.356744in}}{\pgfqpoint{2.430446in}{1.362568in}}%
\pgfpathcurveto{\pgfqpoint{2.424622in}{1.368392in}}{\pgfqpoint{2.416722in}{1.371665in}}{\pgfqpoint{2.408486in}{1.371665in}}%
\pgfpathcurveto{\pgfqpoint{2.400250in}{1.371665in}}{\pgfqpoint{2.392349in}{1.368392in}}{\pgfqpoint{2.386526in}{1.362568in}}%
\pgfpathcurveto{\pgfqpoint{2.380702in}{1.356744in}}{\pgfqpoint{2.377429in}{1.348844in}}{\pgfqpoint{2.377429in}{1.340608in}}%
\pgfpathcurveto{\pgfqpoint{2.377429in}{1.332372in}}{\pgfqpoint{2.380702in}{1.324472in}}{\pgfqpoint{2.386526in}{1.318648in}}%
\pgfpathcurveto{\pgfqpoint{2.392349in}{1.312824in}}{\pgfqpoint{2.400250in}{1.309552in}}{\pgfqpoint{2.408486in}{1.309552in}}%
\pgfpathclose%
\pgfusepath{stroke,fill}%
\end{pgfscope}%
\begin{pgfscope}%
\pgfpathrectangle{\pgfqpoint{0.100000in}{0.220728in}}{\pgfqpoint{3.696000in}{3.696000in}}%
\pgfusepath{clip}%
\pgfsetbuttcap%
\pgfsetroundjoin%
\definecolor{currentfill}{rgb}{0.121569,0.466667,0.705882}%
\pgfsetfillcolor{currentfill}%
\pgfsetfillopacity{0.997089}%
\pgfsetlinewidth{1.003750pt}%
\definecolor{currentstroke}{rgb}{0.121569,0.466667,0.705882}%
\pgfsetstrokecolor{currentstroke}%
\pgfsetstrokeopacity{0.997089}%
\pgfsetdash{}{0pt}%
\pgfpathmoveto{\pgfqpoint{2.368341in}{1.327852in}}%
\pgfpathcurveto{\pgfqpoint{2.376577in}{1.327852in}}{\pgfqpoint{2.384477in}{1.331125in}}{\pgfqpoint{2.390301in}{1.336949in}}%
\pgfpathcurveto{\pgfqpoint{2.396125in}{1.342773in}}{\pgfqpoint{2.399397in}{1.350673in}}{\pgfqpoint{2.399397in}{1.358909in}}%
\pgfpathcurveto{\pgfqpoint{2.399397in}{1.367145in}}{\pgfqpoint{2.396125in}{1.375045in}}{\pgfqpoint{2.390301in}{1.380869in}}%
\pgfpathcurveto{\pgfqpoint{2.384477in}{1.386693in}}{\pgfqpoint{2.376577in}{1.389965in}}{\pgfqpoint{2.368341in}{1.389965in}}%
\pgfpathcurveto{\pgfqpoint{2.360105in}{1.389965in}}{\pgfqpoint{2.352204in}{1.386693in}}{\pgfqpoint{2.346381in}{1.380869in}}%
\pgfpathcurveto{\pgfqpoint{2.340557in}{1.375045in}}{\pgfqpoint{2.337284in}{1.367145in}}{\pgfqpoint{2.337284in}{1.358909in}}%
\pgfpathcurveto{\pgfqpoint{2.337284in}{1.350673in}}{\pgfqpoint{2.340557in}{1.342773in}}{\pgfqpoint{2.346381in}{1.336949in}}%
\pgfpathcurveto{\pgfqpoint{2.352204in}{1.331125in}}{\pgfqpoint{2.360105in}{1.327852in}}{\pgfqpoint{2.368341in}{1.327852in}}%
\pgfpathclose%
\pgfusepath{stroke,fill}%
\end{pgfscope}%
\begin{pgfscope}%
\pgfpathrectangle{\pgfqpoint{0.100000in}{0.220728in}}{\pgfqpoint{3.696000in}{3.696000in}}%
\pgfusepath{clip}%
\pgfsetbuttcap%
\pgfsetroundjoin%
\definecolor{currentfill}{rgb}{0.121569,0.466667,0.705882}%
\pgfsetfillcolor{currentfill}%
\pgfsetfillopacity{0.998141}%
\pgfsetlinewidth{1.003750pt}%
\definecolor{currentstroke}{rgb}{0.121569,0.466667,0.705882}%
\pgfsetstrokecolor{currentstroke}%
\pgfsetstrokeopacity{0.998141}%
\pgfsetdash{}{0pt}%
\pgfpathmoveto{\pgfqpoint{2.375710in}{1.322347in}}%
\pgfpathcurveto{\pgfqpoint{2.383946in}{1.322347in}}{\pgfqpoint{2.391846in}{1.325620in}}{\pgfqpoint{2.397670in}{1.331444in}}%
\pgfpathcurveto{\pgfqpoint{2.403494in}{1.337268in}}{\pgfqpoint{2.406767in}{1.345168in}}{\pgfqpoint{2.406767in}{1.353404in}}%
\pgfpathcurveto{\pgfqpoint{2.406767in}{1.361640in}}{\pgfqpoint{2.403494in}{1.369540in}}{\pgfqpoint{2.397670in}{1.375364in}}%
\pgfpathcurveto{\pgfqpoint{2.391846in}{1.381188in}}{\pgfqpoint{2.383946in}{1.384460in}}{\pgfqpoint{2.375710in}{1.384460in}}%
\pgfpathcurveto{\pgfqpoint{2.367474in}{1.384460in}}{\pgfqpoint{2.359574in}{1.381188in}}{\pgfqpoint{2.353750in}{1.375364in}}%
\pgfpathcurveto{\pgfqpoint{2.347926in}{1.369540in}}{\pgfqpoint{2.344654in}{1.361640in}}{\pgfqpoint{2.344654in}{1.353404in}}%
\pgfpathcurveto{\pgfqpoint{2.344654in}{1.345168in}}{\pgfqpoint{2.347926in}{1.337268in}}{\pgfqpoint{2.353750in}{1.331444in}}%
\pgfpathcurveto{\pgfqpoint{2.359574in}{1.325620in}}{\pgfqpoint{2.367474in}{1.322347in}}{\pgfqpoint{2.375710in}{1.322347in}}%
\pgfpathclose%
\pgfusepath{stroke,fill}%
\end{pgfscope}%
\begin{pgfscope}%
\pgfpathrectangle{\pgfqpoint{0.100000in}{0.220728in}}{\pgfqpoint{3.696000in}{3.696000in}}%
\pgfusepath{clip}%
\pgfsetbuttcap%
\pgfsetroundjoin%
\definecolor{currentfill}{rgb}{0.121569,0.466667,0.705882}%
\pgfsetfillcolor{currentfill}%
\pgfsetfillopacity{0.998213}%
\pgfsetlinewidth{1.003750pt}%
\definecolor{currentstroke}{rgb}{0.121569,0.466667,0.705882}%
\pgfsetstrokecolor{currentstroke}%
\pgfsetstrokeopacity{0.998213}%
\pgfsetdash{}{0pt}%
\pgfpathmoveto{\pgfqpoint{2.382776in}{1.313172in}}%
\pgfpathcurveto{\pgfqpoint{2.391012in}{1.313172in}}{\pgfqpoint{2.398912in}{1.316444in}}{\pgfqpoint{2.404736in}{1.322268in}}%
\pgfpathcurveto{\pgfqpoint{2.410560in}{1.328092in}}{\pgfqpoint{2.413832in}{1.335992in}}{\pgfqpoint{2.413832in}{1.344229in}}%
\pgfpathcurveto{\pgfqpoint{2.413832in}{1.352465in}}{\pgfqpoint{2.410560in}{1.360365in}}{\pgfqpoint{2.404736in}{1.366189in}}%
\pgfpathcurveto{\pgfqpoint{2.398912in}{1.372013in}}{\pgfqpoint{2.391012in}{1.375285in}}{\pgfqpoint{2.382776in}{1.375285in}}%
\pgfpathcurveto{\pgfqpoint{2.374539in}{1.375285in}}{\pgfqpoint{2.366639in}{1.372013in}}{\pgfqpoint{2.360816in}{1.366189in}}%
\pgfpathcurveto{\pgfqpoint{2.354992in}{1.360365in}}{\pgfqpoint{2.351719in}{1.352465in}}{\pgfqpoint{2.351719in}{1.344229in}}%
\pgfpathcurveto{\pgfqpoint{2.351719in}{1.335992in}}{\pgfqpoint{2.354992in}{1.328092in}}{\pgfqpoint{2.360816in}{1.322268in}}%
\pgfpathcurveto{\pgfqpoint{2.366639in}{1.316444in}}{\pgfqpoint{2.374539in}{1.313172in}}{\pgfqpoint{2.382776in}{1.313172in}}%
\pgfpathclose%
\pgfusepath{stroke,fill}%
\end{pgfscope}%
\begin{pgfscope}%
\pgfpathrectangle{\pgfqpoint{0.100000in}{0.220728in}}{\pgfqpoint{3.696000in}{3.696000in}}%
\pgfusepath{clip}%
\pgfsetbuttcap%
\pgfsetroundjoin%
\definecolor{currentfill}{rgb}{0.121569,0.466667,0.705882}%
\pgfsetfillcolor{currentfill}%
\pgfsetfillopacity{0.998265}%
\pgfsetlinewidth{1.003750pt}%
\definecolor{currentstroke}{rgb}{0.121569,0.466667,0.705882}%
\pgfsetstrokecolor{currentstroke}%
\pgfsetstrokeopacity{0.998265}%
\pgfsetdash{}{0pt}%
\pgfpathmoveto{\pgfqpoint{2.407468in}{1.308076in}}%
\pgfpathcurveto{\pgfqpoint{2.415704in}{1.308076in}}{\pgfqpoint{2.423604in}{1.311348in}}{\pgfqpoint{2.429428in}{1.317172in}}%
\pgfpathcurveto{\pgfqpoint{2.435252in}{1.322996in}}{\pgfqpoint{2.438525in}{1.330896in}}{\pgfqpoint{2.438525in}{1.339133in}}%
\pgfpathcurveto{\pgfqpoint{2.438525in}{1.347369in}}{\pgfqpoint{2.435252in}{1.355269in}}{\pgfqpoint{2.429428in}{1.361093in}}%
\pgfpathcurveto{\pgfqpoint{2.423604in}{1.366917in}}{\pgfqpoint{2.415704in}{1.370189in}}{\pgfqpoint{2.407468in}{1.370189in}}%
\pgfpathcurveto{\pgfqpoint{2.399232in}{1.370189in}}{\pgfqpoint{2.391332in}{1.366917in}}{\pgfqpoint{2.385508in}{1.361093in}}%
\pgfpathcurveto{\pgfqpoint{2.379684in}{1.355269in}}{\pgfqpoint{2.376412in}{1.347369in}}{\pgfqpoint{2.376412in}{1.339133in}}%
\pgfpathcurveto{\pgfqpoint{2.376412in}{1.330896in}}{\pgfqpoint{2.379684in}{1.322996in}}{\pgfqpoint{2.385508in}{1.317172in}}%
\pgfpathcurveto{\pgfqpoint{2.391332in}{1.311348in}}{\pgfqpoint{2.399232in}{1.308076in}}{\pgfqpoint{2.407468in}{1.308076in}}%
\pgfpathclose%
\pgfusepath{stroke,fill}%
\end{pgfscope}%
\begin{pgfscope}%
\pgfpathrectangle{\pgfqpoint{0.100000in}{0.220728in}}{\pgfqpoint{3.696000in}{3.696000in}}%
\pgfusepath{clip}%
\pgfsetbuttcap%
\pgfsetroundjoin%
\definecolor{currentfill}{rgb}{0.121569,0.466667,0.705882}%
\pgfsetfillcolor{currentfill}%
\pgfsetfillopacity{0.999002}%
\pgfsetlinewidth{1.003750pt}%
\definecolor{currentstroke}{rgb}{0.121569,0.466667,0.705882}%
\pgfsetstrokecolor{currentstroke}%
\pgfsetstrokeopacity{0.999002}%
\pgfsetdash{}{0pt}%
\pgfpathmoveto{\pgfqpoint{2.405766in}{1.306258in}}%
\pgfpathcurveto{\pgfqpoint{2.414003in}{1.306258in}}{\pgfqpoint{2.421903in}{1.309531in}}{\pgfqpoint{2.427727in}{1.315355in}}%
\pgfpathcurveto{\pgfqpoint{2.433550in}{1.321178in}}{\pgfqpoint{2.436823in}{1.329079in}}{\pgfqpoint{2.436823in}{1.337315in}}%
\pgfpathcurveto{\pgfqpoint{2.436823in}{1.345551in}}{\pgfqpoint{2.433550in}{1.353451in}}{\pgfqpoint{2.427727in}{1.359275in}}%
\pgfpathcurveto{\pgfqpoint{2.421903in}{1.365099in}}{\pgfqpoint{2.414003in}{1.368371in}}{\pgfqpoint{2.405766in}{1.368371in}}%
\pgfpathcurveto{\pgfqpoint{2.397530in}{1.368371in}}{\pgfqpoint{2.389630in}{1.365099in}}{\pgfqpoint{2.383806in}{1.359275in}}%
\pgfpathcurveto{\pgfqpoint{2.377982in}{1.353451in}}{\pgfqpoint{2.374710in}{1.345551in}}{\pgfqpoint{2.374710in}{1.337315in}}%
\pgfpathcurveto{\pgfqpoint{2.374710in}{1.329079in}}{\pgfqpoint{2.377982in}{1.321178in}}{\pgfqpoint{2.383806in}{1.315355in}}%
\pgfpathcurveto{\pgfqpoint{2.389630in}{1.309531in}}{\pgfqpoint{2.397530in}{1.306258in}}{\pgfqpoint{2.405766in}{1.306258in}}%
\pgfpathclose%
\pgfusepath{stroke,fill}%
\end{pgfscope}%
\begin{pgfscope}%
\pgfpathrectangle{\pgfqpoint{0.100000in}{0.220728in}}{\pgfqpoint{3.696000in}{3.696000in}}%
\pgfusepath{clip}%
\pgfsetbuttcap%
\pgfsetroundjoin%
\definecolor{currentfill}{rgb}{0.121569,0.466667,0.705882}%
\pgfsetfillcolor{currentfill}%
\pgfsetfillopacity{0.999486}%
\pgfsetlinewidth{1.003750pt}%
\definecolor{currentstroke}{rgb}{0.121569,0.466667,0.705882}%
\pgfsetstrokecolor{currentstroke}%
\pgfsetstrokeopacity{0.999486}%
\pgfsetdash{}{0pt}%
\pgfpathmoveto{\pgfqpoint{2.388983in}{1.312392in}}%
\pgfpathcurveto{\pgfqpoint{2.397219in}{1.312392in}}{\pgfqpoint{2.405119in}{1.315664in}}{\pgfqpoint{2.410943in}{1.321488in}}%
\pgfpathcurveto{\pgfqpoint{2.416767in}{1.327312in}}{\pgfqpoint{2.420039in}{1.335212in}}{\pgfqpoint{2.420039in}{1.343448in}}%
\pgfpathcurveto{\pgfqpoint{2.420039in}{1.351684in}}{\pgfqpoint{2.416767in}{1.359584in}}{\pgfqpoint{2.410943in}{1.365408in}}%
\pgfpathcurveto{\pgfqpoint{2.405119in}{1.371232in}}{\pgfqpoint{2.397219in}{1.374505in}}{\pgfqpoint{2.388983in}{1.374505in}}%
\pgfpathcurveto{\pgfqpoint{2.380746in}{1.374505in}}{\pgfqpoint{2.372846in}{1.371232in}}{\pgfqpoint{2.367022in}{1.365408in}}%
\pgfpathcurveto{\pgfqpoint{2.361198in}{1.359584in}}{\pgfqpoint{2.357926in}{1.351684in}}{\pgfqpoint{2.357926in}{1.343448in}}%
\pgfpathcurveto{\pgfqpoint{2.357926in}{1.335212in}}{\pgfqpoint{2.361198in}{1.327312in}}{\pgfqpoint{2.367022in}{1.321488in}}%
\pgfpathcurveto{\pgfqpoint{2.372846in}{1.315664in}}{\pgfqpoint{2.380746in}{1.312392in}}{\pgfqpoint{2.388983in}{1.312392in}}%
\pgfpathclose%
\pgfusepath{stroke,fill}%
\end{pgfscope}%
\begin{pgfscope}%
\pgfpathrectangle{\pgfqpoint{0.100000in}{0.220728in}}{\pgfqpoint{3.696000in}{3.696000in}}%
\pgfusepath{clip}%
\pgfsetbuttcap%
\pgfsetroundjoin%
\definecolor{currentfill}{rgb}{0.121569,0.466667,0.705882}%
\pgfsetfillcolor{currentfill}%
\pgfsetfillopacity{0.999533}%
\pgfsetlinewidth{1.003750pt}%
\definecolor{currentstroke}{rgb}{0.121569,0.466667,0.705882}%
\pgfsetstrokecolor{currentstroke}%
\pgfsetstrokeopacity{0.999533}%
\pgfsetdash{}{0pt}%
\pgfpathmoveto{\pgfqpoint{2.404577in}{1.306378in}}%
\pgfpathcurveto{\pgfqpoint{2.412814in}{1.306378in}}{\pgfqpoint{2.420714in}{1.309650in}}{\pgfqpoint{2.426538in}{1.315474in}}%
\pgfpathcurveto{\pgfqpoint{2.432362in}{1.321298in}}{\pgfqpoint{2.435634in}{1.329198in}}{\pgfqpoint{2.435634in}{1.337434in}}%
\pgfpathcurveto{\pgfqpoint{2.435634in}{1.345670in}}{\pgfqpoint{2.432362in}{1.353570in}}{\pgfqpoint{2.426538in}{1.359394in}}%
\pgfpathcurveto{\pgfqpoint{2.420714in}{1.365218in}}{\pgfqpoint{2.412814in}{1.368491in}}{\pgfqpoint{2.404577in}{1.368491in}}%
\pgfpathcurveto{\pgfqpoint{2.396341in}{1.368491in}}{\pgfqpoint{2.388441in}{1.365218in}}{\pgfqpoint{2.382617in}{1.359394in}}%
\pgfpathcurveto{\pgfqpoint{2.376793in}{1.353570in}}{\pgfqpoint{2.373521in}{1.345670in}}{\pgfqpoint{2.373521in}{1.337434in}}%
\pgfpathcurveto{\pgfqpoint{2.373521in}{1.329198in}}{\pgfqpoint{2.376793in}{1.321298in}}{\pgfqpoint{2.382617in}{1.315474in}}%
\pgfpathcurveto{\pgfqpoint{2.388441in}{1.309650in}}{\pgfqpoint{2.396341in}{1.306378in}}{\pgfqpoint{2.404577in}{1.306378in}}%
\pgfpathclose%
\pgfusepath{stroke,fill}%
\end{pgfscope}%
\begin{pgfscope}%
\pgfpathrectangle{\pgfqpoint{0.100000in}{0.220728in}}{\pgfqpoint{3.696000in}{3.696000in}}%
\pgfusepath{clip}%
\pgfsetbuttcap%
\pgfsetroundjoin%
\definecolor{currentfill}{rgb}{0.121569,0.466667,0.705882}%
\pgfsetfillcolor{currentfill}%
\pgfsetfillopacity{0.999636}%
\pgfsetlinewidth{1.003750pt}%
\definecolor{currentstroke}{rgb}{0.121569,0.466667,0.705882}%
\pgfsetstrokecolor{currentstroke}%
\pgfsetstrokeopacity{0.999636}%
\pgfsetdash{}{0pt}%
\pgfpathmoveto{\pgfqpoint{2.399048in}{1.304769in}}%
\pgfpathcurveto{\pgfqpoint{2.407284in}{1.304769in}}{\pgfqpoint{2.415184in}{1.308041in}}{\pgfqpoint{2.421008in}{1.313865in}}%
\pgfpathcurveto{\pgfqpoint{2.426832in}{1.319689in}}{\pgfqpoint{2.430104in}{1.327589in}}{\pgfqpoint{2.430104in}{1.335825in}}%
\pgfpathcurveto{\pgfqpoint{2.430104in}{1.344062in}}{\pgfqpoint{2.426832in}{1.351962in}}{\pgfqpoint{2.421008in}{1.357786in}}%
\pgfpathcurveto{\pgfqpoint{2.415184in}{1.363610in}}{\pgfqpoint{2.407284in}{1.366882in}}{\pgfqpoint{2.399048in}{1.366882in}}%
\pgfpathcurveto{\pgfqpoint{2.390812in}{1.366882in}}{\pgfqpoint{2.382912in}{1.363610in}}{\pgfqpoint{2.377088in}{1.357786in}}%
\pgfpathcurveto{\pgfqpoint{2.371264in}{1.351962in}}{\pgfqpoint{2.367991in}{1.344062in}}{\pgfqpoint{2.367991in}{1.335825in}}%
\pgfpathcurveto{\pgfqpoint{2.367991in}{1.327589in}}{\pgfqpoint{2.371264in}{1.319689in}}{\pgfqpoint{2.377088in}{1.313865in}}%
\pgfpathcurveto{\pgfqpoint{2.382912in}{1.308041in}}{\pgfqpoint{2.390812in}{1.304769in}}{\pgfqpoint{2.399048in}{1.304769in}}%
\pgfpathclose%
\pgfusepath{stroke,fill}%
\end{pgfscope}%
\begin{pgfscope}%
\pgfpathrectangle{\pgfqpoint{0.100000in}{0.220728in}}{\pgfqpoint{3.696000in}{3.696000in}}%
\pgfusepath{clip}%
\pgfsetbuttcap%
\pgfsetroundjoin%
\definecolor{currentfill}{rgb}{0.121569,0.466667,0.705882}%
\pgfsetfillcolor{currentfill}%
\pgfsetfillopacity{0.999771}%
\pgfsetlinewidth{1.003750pt}%
\definecolor{currentstroke}{rgb}{0.121569,0.466667,0.705882}%
\pgfsetstrokecolor{currentstroke}%
\pgfsetstrokeopacity{0.999771}%
\pgfsetdash{}{0pt}%
\pgfpathmoveto{\pgfqpoint{2.393174in}{1.309261in}}%
\pgfpathcurveto{\pgfqpoint{2.401410in}{1.309261in}}{\pgfqpoint{2.409311in}{1.312534in}}{\pgfqpoint{2.415134in}{1.318358in}}%
\pgfpathcurveto{\pgfqpoint{2.420958in}{1.324182in}}{\pgfqpoint{2.424231in}{1.332082in}}{\pgfqpoint{2.424231in}{1.340318in}}%
\pgfpathcurveto{\pgfqpoint{2.424231in}{1.348554in}}{\pgfqpoint{2.420958in}{1.356454in}}{\pgfqpoint{2.415134in}{1.362278in}}%
\pgfpathcurveto{\pgfqpoint{2.409311in}{1.368102in}}{\pgfqpoint{2.401410in}{1.371374in}}{\pgfqpoint{2.393174in}{1.371374in}}%
\pgfpathcurveto{\pgfqpoint{2.384938in}{1.371374in}}{\pgfqpoint{2.377038in}{1.368102in}}{\pgfqpoint{2.371214in}{1.362278in}}%
\pgfpathcurveto{\pgfqpoint{2.365390in}{1.356454in}}{\pgfqpoint{2.362118in}{1.348554in}}{\pgfqpoint{2.362118in}{1.340318in}}%
\pgfpathcurveto{\pgfqpoint{2.362118in}{1.332082in}}{\pgfqpoint{2.365390in}{1.324182in}}{\pgfqpoint{2.371214in}{1.318358in}}%
\pgfpathcurveto{\pgfqpoint{2.377038in}{1.312534in}}{\pgfqpoint{2.384938in}{1.309261in}}{\pgfqpoint{2.393174in}{1.309261in}}%
\pgfpathclose%
\pgfusepath{stroke,fill}%
\end{pgfscope}%
\begin{pgfscope}%
\pgfpathrectangle{\pgfqpoint{0.100000in}{0.220728in}}{\pgfqpoint{3.696000in}{3.696000in}}%
\pgfusepath{clip}%
\pgfsetbuttcap%
\pgfsetroundjoin%
\definecolor{currentfill}{rgb}{0.121569,0.466667,0.705882}%
\pgfsetfillcolor{currentfill}%
\pgfsetfillopacity{0.999925}%
\pgfsetlinewidth{1.003750pt}%
\definecolor{currentstroke}{rgb}{0.121569,0.466667,0.705882}%
\pgfsetstrokecolor{currentstroke}%
\pgfsetstrokeopacity{0.999925}%
\pgfsetdash{}{0pt}%
\pgfpathmoveto{\pgfqpoint{2.402369in}{1.305540in}}%
\pgfpathcurveto{\pgfqpoint{2.410605in}{1.305540in}}{\pgfqpoint{2.418505in}{1.308812in}}{\pgfqpoint{2.424329in}{1.314636in}}%
\pgfpathcurveto{\pgfqpoint{2.430153in}{1.320460in}}{\pgfqpoint{2.433425in}{1.328360in}}{\pgfqpoint{2.433425in}{1.336596in}}%
\pgfpathcurveto{\pgfqpoint{2.433425in}{1.344833in}}{\pgfqpoint{2.430153in}{1.352733in}}{\pgfqpoint{2.424329in}{1.358557in}}%
\pgfpathcurveto{\pgfqpoint{2.418505in}{1.364381in}}{\pgfqpoint{2.410605in}{1.367653in}}{\pgfqpoint{2.402369in}{1.367653in}}%
\pgfpathcurveto{\pgfqpoint{2.394133in}{1.367653in}}{\pgfqpoint{2.386233in}{1.364381in}}{\pgfqpoint{2.380409in}{1.358557in}}%
\pgfpathcurveto{\pgfqpoint{2.374585in}{1.352733in}}{\pgfqpoint{2.371312in}{1.344833in}}{\pgfqpoint{2.371312in}{1.336596in}}%
\pgfpathcurveto{\pgfqpoint{2.371312in}{1.328360in}}{\pgfqpoint{2.374585in}{1.320460in}}{\pgfqpoint{2.380409in}{1.314636in}}%
\pgfpathcurveto{\pgfqpoint{2.386233in}{1.308812in}}{\pgfqpoint{2.394133in}{1.305540in}}{\pgfqpoint{2.402369in}{1.305540in}}%
\pgfpathclose%
\pgfusepath{stroke,fill}%
\end{pgfscope}%
\begin{pgfscope}%
\pgfpathrectangle{\pgfqpoint{0.100000in}{0.220728in}}{\pgfqpoint{3.696000in}{3.696000in}}%
\pgfusepath{clip}%
\pgfsetbuttcap%
\pgfsetroundjoin%
\definecolor{currentfill}{rgb}{0.121569,0.466667,0.705882}%
\pgfsetfillcolor{currentfill}%
\pgfsetlinewidth{1.003750pt}%
\definecolor{currentstroke}{rgb}{0.121569,0.466667,0.705882}%
\pgfsetstrokecolor{currentstroke}%
\pgfsetdash{}{0pt}%
\pgfpathmoveto{\pgfqpoint{2.395083in}{1.308366in}}%
\pgfpathcurveto{\pgfqpoint{2.403320in}{1.308366in}}{\pgfqpoint{2.411220in}{1.311638in}}{\pgfqpoint{2.417044in}{1.317462in}}%
\pgfpathcurveto{\pgfqpoint{2.422868in}{1.323286in}}{\pgfqpoint{2.426140in}{1.331186in}}{\pgfqpoint{2.426140in}{1.339422in}}%
\pgfpathcurveto{\pgfqpoint{2.426140in}{1.347658in}}{\pgfqpoint{2.422868in}{1.355558in}}{\pgfqpoint{2.417044in}{1.361382in}}%
\pgfpathcurveto{\pgfqpoint{2.411220in}{1.367206in}}{\pgfqpoint{2.403320in}{1.370479in}}{\pgfqpoint{2.395083in}{1.370479in}}%
\pgfpathcurveto{\pgfqpoint{2.386847in}{1.370479in}}{\pgfqpoint{2.378947in}{1.367206in}}{\pgfqpoint{2.373123in}{1.361382in}}%
\pgfpathcurveto{\pgfqpoint{2.367299in}{1.355558in}}{\pgfqpoint{2.364027in}{1.347658in}}{\pgfqpoint{2.364027in}{1.339422in}}%
\pgfpathcurveto{\pgfqpoint{2.364027in}{1.331186in}}{\pgfqpoint{2.367299in}{1.323286in}}{\pgfqpoint{2.373123in}{1.317462in}}%
\pgfpathcurveto{\pgfqpoint{2.378947in}{1.311638in}}{\pgfqpoint{2.386847in}{1.308366in}}{\pgfqpoint{2.395083in}{1.308366in}}%
\pgfpathclose%
\pgfusepath{stroke,fill}%
\end{pgfscope}%
\begin{pgfscope}%
\definecolor{textcolor}{rgb}{0.000000,0.000000,0.000000}%
\pgfsetstrokecolor{textcolor}%
\pgfsetfillcolor{textcolor}%
\pgftext[x=1.948000in,y=4.000061in,,base]{\color{textcolor}\sffamily\fontsize{12.000000}{14.400000}\selectfont Mahony}%
\end{pgfscope}%
\begin{pgfscope}%
\pgfsetbuttcap%
\pgfsetmiterjoin%
\definecolor{currentfill}{rgb}{1.000000,1.000000,1.000000}%
\pgfsetfillcolor{currentfill}%
\pgfsetfillopacity{0.800000}%
\pgfsetlinewidth{1.003750pt}%
\definecolor{currentstroke}{rgb}{0.800000,0.800000,0.800000}%
\pgfsetstrokecolor{currentstroke}%
\pgfsetstrokeopacity{0.800000}%
\pgfsetdash{}{0pt}%
\pgfpathmoveto{\pgfqpoint{1.958421in}{3.397902in}}%
\pgfpathlineto{\pgfqpoint{3.698778in}{3.397902in}}%
\pgfpathquadraticcurveto{\pgfqpoint{3.726556in}{3.397902in}}{\pgfqpoint{3.726556in}{3.425680in}}%
\pgfpathlineto{\pgfqpoint{3.726556in}{3.819506in}}%
\pgfpathquadraticcurveto{\pgfqpoint{3.726556in}{3.847284in}}{\pgfqpoint{3.698778in}{3.847284in}}%
\pgfpathlineto{\pgfqpoint{1.958421in}{3.847284in}}%
\pgfpathquadraticcurveto{\pgfqpoint{1.930644in}{3.847284in}}{\pgfqpoint{1.930644in}{3.819506in}}%
\pgfpathlineto{\pgfqpoint{1.930644in}{3.425680in}}%
\pgfpathquadraticcurveto{\pgfqpoint{1.930644in}{3.397902in}}{\pgfqpoint{1.958421in}{3.397902in}}%
\pgfpathclose%
\pgfusepath{stroke,fill}%
\end{pgfscope}%
\begin{pgfscope}%
\pgfsetrectcap%
\pgfsetroundjoin%
\pgfsetlinewidth{1.505625pt}%
\definecolor{currentstroke}{rgb}{0.121569,0.466667,0.705882}%
\pgfsetstrokecolor{currentstroke}%
\pgfsetdash{}{0pt}%
\pgfpathmoveto{\pgfqpoint{1.986199in}{3.734816in}}%
\pgfpathlineto{\pgfqpoint{2.263977in}{3.734816in}}%
\pgfusepath{stroke}%
\end{pgfscope}%
\begin{pgfscope}%
\definecolor{textcolor}{rgb}{0.000000,0.000000,0.000000}%
\pgfsetstrokecolor{textcolor}%
\pgfsetfillcolor{textcolor}%
\pgftext[x=2.375088in,y=3.686205in,left,base]{\color{textcolor}\sffamily\fontsize{10.000000}{12.000000}\selectfont Ground truth}%
\end{pgfscope}%
\begin{pgfscope}%
\pgfsetbuttcap%
\pgfsetroundjoin%
\definecolor{currentfill}{rgb}{0.121569,0.466667,0.705882}%
\pgfsetfillcolor{currentfill}%
\pgfsetlinewidth{1.003750pt}%
\definecolor{currentstroke}{rgb}{0.121569,0.466667,0.705882}%
\pgfsetstrokecolor{currentstroke}%
\pgfsetdash{}{0pt}%
\pgfsys@defobject{currentmarker}{\pgfqpoint{-0.031056in}{-0.031056in}}{\pgfqpoint{0.031056in}{0.031056in}}{%
\pgfpathmoveto{\pgfqpoint{0.000000in}{-0.031056in}}%
\pgfpathcurveto{\pgfqpoint{0.008236in}{-0.031056in}}{\pgfqpoint{0.016136in}{-0.027784in}}{\pgfqpoint{0.021960in}{-0.021960in}}%
\pgfpathcurveto{\pgfqpoint{0.027784in}{-0.016136in}}{\pgfqpoint{0.031056in}{-0.008236in}}{\pgfqpoint{0.031056in}{0.000000in}}%
\pgfpathcurveto{\pgfqpoint{0.031056in}{0.008236in}}{\pgfqpoint{0.027784in}{0.016136in}}{\pgfqpoint{0.021960in}{0.021960in}}%
\pgfpathcurveto{\pgfqpoint{0.016136in}{0.027784in}}{\pgfqpoint{0.008236in}{0.031056in}}{\pgfqpoint{0.000000in}{0.031056in}}%
\pgfpathcurveto{\pgfqpoint{-0.008236in}{0.031056in}}{\pgfqpoint{-0.016136in}{0.027784in}}{\pgfqpoint{-0.021960in}{0.021960in}}%
\pgfpathcurveto{\pgfqpoint{-0.027784in}{0.016136in}}{\pgfqpoint{-0.031056in}{0.008236in}}{\pgfqpoint{-0.031056in}{0.000000in}}%
\pgfpathcurveto{\pgfqpoint{-0.031056in}{-0.008236in}}{\pgfqpoint{-0.027784in}{-0.016136in}}{\pgfqpoint{-0.021960in}{-0.021960in}}%
\pgfpathcurveto{\pgfqpoint{-0.016136in}{-0.027784in}}{\pgfqpoint{-0.008236in}{-0.031056in}}{\pgfqpoint{0.000000in}{-0.031056in}}%
\pgfpathclose%
\pgfusepath{stroke,fill}%
}%
\begin{pgfscope}%
\pgfsys@transformshift{2.125088in}{3.518806in}%
\pgfsys@useobject{currentmarker}{}%
\end{pgfscope}%
\end{pgfscope}%
\begin{pgfscope}%
\definecolor{textcolor}{rgb}{0.000000,0.000000,0.000000}%
\pgfsetstrokecolor{textcolor}%
\pgfsetfillcolor{textcolor}%
\pgftext[x=2.375088in,y=3.482348in,left,base]{\color{textcolor}\sffamily\fontsize{10.000000}{12.000000}\selectfont Estimated position}%
\end{pgfscope}%
\end{pgfpicture}%
\makeatother%
\endgroup%
}
%         \caption{Mahony's 3D position estimation had the lowest displacement error for the 16-meter side triangle experiment.}
%         \label{fig:triangle16_2D}
%     \end{subfigure}
%     \begin{subfigure}{0.49\textwidth}
%         \centering
%         \resizebox{1\linewidth}{!}{%% Creator: Matplotlib, PGF backend
%%
%% To include the figure in your LaTeX document, write
%%   \input{<filename>.pgf}
%%
%% Make sure the required packages are loaded in your preamble
%%   \usepackage{pgf}
%%
%% and, on pdftex
%%   \usepackage[utf8]{inputenc}\DeclareUnicodeCharacter{2212}{-}
%%
%% or, on luatex and xetex
%%   \usepackage{unicode-math}
%%
%% Figures using additional raster images can only be included by \input if
%% they are in the same directory as the main LaTeX file. For loading figures
%% from other directories you can use the `import` package
%%   \usepackage{import}
%%
%% and then include the figures with
%%   \import{<path to file>}{<filename>.pgf}
%%
%% Matplotlib used the following preamble
%%   \usepackage{fontspec}
%%
\begingroup%
\makeatletter%
\begin{pgfpicture}%
\pgfpathrectangle{\pgfpointorigin}{\pgfqpoint{4.342355in}{4.207622in}}%
\pgfusepath{use as bounding box, clip}%
\begin{pgfscope}%
\pgfsetbuttcap%
\pgfsetmiterjoin%
\definecolor{currentfill}{rgb}{1.000000,1.000000,1.000000}%
\pgfsetfillcolor{currentfill}%
\pgfsetlinewidth{0.000000pt}%
\definecolor{currentstroke}{rgb}{1.000000,1.000000,1.000000}%
\pgfsetstrokecolor{currentstroke}%
\pgfsetdash{}{0pt}%
\pgfpathmoveto{\pgfqpoint{0.000000in}{-0.000000in}}%
\pgfpathlineto{\pgfqpoint{4.342355in}{-0.000000in}}%
\pgfpathlineto{\pgfqpoint{4.342355in}{4.207622in}}%
\pgfpathlineto{\pgfqpoint{0.000000in}{4.207622in}}%
\pgfpathclose%
\pgfusepath{fill}%
\end{pgfscope}%
\begin{pgfscope}%
\pgfsetbuttcap%
\pgfsetmiterjoin%
\definecolor{currentfill}{rgb}{1.000000,1.000000,1.000000}%
\pgfsetfillcolor{currentfill}%
\pgfsetlinewidth{0.000000pt}%
\definecolor{currentstroke}{rgb}{0.000000,0.000000,0.000000}%
\pgfsetstrokecolor{currentstroke}%
\pgfsetstrokeopacity{0.000000}%
\pgfsetdash{}{0pt}%
\pgfpathmoveto{\pgfqpoint{0.100000in}{0.212622in}}%
\pgfpathlineto{\pgfqpoint{3.796000in}{0.212622in}}%
\pgfpathlineto{\pgfqpoint{3.796000in}{3.908622in}}%
\pgfpathlineto{\pgfqpoint{0.100000in}{3.908622in}}%
\pgfpathclose%
\pgfusepath{fill}%
\end{pgfscope}%
\begin{pgfscope}%
\pgfsetbuttcap%
\pgfsetmiterjoin%
\definecolor{currentfill}{rgb}{0.950000,0.950000,0.950000}%
\pgfsetfillcolor{currentfill}%
\pgfsetfillopacity{0.500000}%
\pgfsetlinewidth{1.003750pt}%
\definecolor{currentstroke}{rgb}{0.950000,0.950000,0.950000}%
\pgfsetstrokecolor{currentstroke}%
\pgfsetstrokeopacity{0.500000}%
\pgfsetdash{}{0pt}%
\pgfpathmoveto{\pgfqpoint{0.379073in}{1.123938in}}%
\pgfpathlineto{\pgfqpoint{1.599613in}{2.147018in}}%
\pgfpathlineto{\pgfqpoint{1.582647in}{3.622484in}}%
\pgfpathlineto{\pgfqpoint{0.303698in}{2.689165in}}%
\pgfusepath{stroke,fill}%
\end{pgfscope}%
\begin{pgfscope}%
\pgfsetbuttcap%
\pgfsetmiterjoin%
\definecolor{currentfill}{rgb}{0.900000,0.900000,0.900000}%
\pgfsetfillcolor{currentfill}%
\pgfsetfillopacity{0.500000}%
\pgfsetlinewidth{1.003750pt}%
\definecolor{currentstroke}{rgb}{0.900000,0.900000,0.900000}%
\pgfsetstrokecolor{currentstroke}%
\pgfsetstrokeopacity{0.500000}%
\pgfsetdash{}{0pt}%
\pgfpathmoveto{\pgfqpoint{1.599613in}{2.147018in}}%
\pgfpathlineto{\pgfqpoint{3.558144in}{1.577751in}}%
\pgfpathlineto{\pgfqpoint{3.628038in}{3.104037in}}%
\pgfpathlineto{\pgfqpoint{1.582647in}{3.622484in}}%
\pgfusepath{stroke,fill}%
\end{pgfscope}%
\begin{pgfscope}%
\pgfsetbuttcap%
\pgfsetmiterjoin%
\definecolor{currentfill}{rgb}{0.925000,0.925000,0.925000}%
\pgfsetfillcolor{currentfill}%
\pgfsetfillopacity{0.500000}%
\pgfsetlinewidth{1.003750pt}%
\definecolor{currentstroke}{rgb}{0.925000,0.925000,0.925000}%
\pgfsetstrokecolor{currentstroke}%
\pgfsetstrokeopacity{0.500000}%
\pgfsetdash{}{0pt}%
\pgfpathmoveto{\pgfqpoint{0.379073in}{1.123938in}}%
\pgfpathlineto{\pgfqpoint{2.455212in}{0.445871in}}%
\pgfpathlineto{\pgfqpoint{3.558144in}{1.577751in}}%
\pgfpathlineto{\pgfqpoint{1.599613in}{2.147018in}}%
\pgfusepath{stroke,fill}%
\end{pgfscope}%
\begin{pgfscope}%
\pgfsetrectcap%
\pgfsetroundjoin%
\pgfsetlinewidth{0.803000pt}%
\definecolor{currentstroke}{rgb}{0.000000,0.000000,0.000000}%
\pgfsetstrokecolor{currentstroke}%
\pgfsetdash{}{0pt}%
\pgfpathmoveto{\pgfqpoint{0.379073in}{1.123938in}}%
\pgfpathlineto{\pgfqpoint{2.455212in}{0.445871in}}%
\pgfusepath{stroke}%
\end{pgfscope}%
\begin{pgfscope}%
\definecolor{textcolor}{rgb}{0.000000,0.000000,0.000000}%
\pgfsetstrokecolor{textcolor}%
\pgfsetfillcolor{textcolor}%
\pgftext[x=0.730374in, y=0.408886in, left, base,rotate=341.912962]{\color{textcolor}\rmfamily\fontsize{10.000000}{12.000000}\selectfont Position X [\(\displaystyle m\)]}%
\end{pgfscope}%
\begin{pgfscope}%
\pgfsetbuttcap%
\pgfsetroundjoin%
\pgfsetlinewidth{0.803000pt}%
\definecolor{currentstroke}{rgb}{0.690196,0.690196,0.690196}%
\pgfsetstrokecolor{currentstroke}%
\pgfsetdash{}{0pt}%
\pgfpathmoveto{\pgfqpoint{0.713978in}{1.014558in}}%
\pgfpathlineto{\pgfqpoint{1.916718in}{2.054849in}}%
\pgfpathlineto{\pgfqpoint{1.913229in}{3.538691in}}%
\pgfusepath{stroke}%
\end{pgfscope}%
\begin{pgfscope}%
\pgfsetbuttcap%
\pgfsetroundjoin%
\pgfsetlinewidth{0.803000pt}%
\definecolor{currentstroke}{rgb}{0.690196,0.690196,0.690196}%
\pgfsetstrokecolor{currentstroke}%
\pgfsetdash{}{0pt}%
\pgfpathmoveto{\pgfqpoint{1.138855in}{0.875793in}}%
\pgfpathlineto{\pgfqpoint{2.318363in}{1.938106in}}%
\pgfpathlineto{\pgfqpoint{2.332270in}{3.432476in}}%
\pgfusepath{stroke}%
\end{pgfscope}%
\begin{pgfscope}%
\pgfsetbuttcap%
\pgfsetroundjoin%
\pgfsetlinewidth{0.803000pt}%
\definecolor{currentstroke}{rgb}{0.690196,0.690196,0.690196}%
\pgfsetstrokecolor{currentstroke}%
\pgfsetdash{}{0pt}%
\pgfpathmoveto{\pgfqpoint{1.573172in}{0.733945in}}%
\pgfpathlineto{\pgfqpoint{2.728182in}{1.818988in}}%
\pgfpathlineto{\pgfqpoint{2.760212in}{3.324005in}}%
\pgfusepath{stroke}%
\end{pgfscope}%
\begin{pgfscope}%
\pgfsetbuttcap%
\pgfsetroundjoin%
\pgfsetlinewidth{0.803000pt}%
\definecolor{currentstroke}{rgb}{0.690196,0.690196,0.690196}%
\pgfsetstrokecolor{currentstroke}%
\pgfsetdash{}{0pt}%
\pgfpathmoveto{\pgfqpoint{2.017247in}{0.588910in}}%
\pgfpathlineto{\pgfqpoint{3.146426in}{1.697421in}}%
\pgfpathlineto{\pgfqpoint{3.197342in}{3.213205in}}%
\pgfusepath{stroke}%
\end{pgfscope}%
\begin{pgfscope}%
\pgfsetrectcap%
\pgfsetroundjoin%
\pgfsetlinewidth{0.803000pt}%
\definecolor{currentstroke}{rgb}{0.000000,0.000000,0.000000}%
\pgfsetstrokecolor{currentstroke}%
\pgfsetdash{}{0pt}%
\pgfpathmoveto{\pgfqpoint{0.724456in}{1.023621in}}%
\pgfpathlineto{\pgfqpoint{0.692977in}{0.996393in}}%
\pgfusepath{stroke}%
\end{pgfscope}%
\begin{pgfscope}%
\definecolor{textcolor}{rgb}{0.000000,0.000000,0.000000}%
\pgfsetstrokecolor{textcolor}%
\pgfsetfillcolor{textcolor}%
\pgftext[x=0.609619in,y=0.794907in,,top]{\color{textcolor}\rmfamily\fontsize{10.000000}{12.000000}\selectfont \(\displaystyle {0}\)}%
\end{pgfscope}%
\begin{pgfscope}%
\pgfsetrectcap%
\pgfsetroundjoin%
\pgfsetlinewidth{0.803000pt}%
\definecolor{currentstroke}{rgb}{0.000000,0.000000,0.000000}%
\pgfsetstrokecolor{currentstroke}%
\pgfsetdash{}{0pt}%
\pgfpathmoveto{\pgfqpoint{1.149140in}{0.885056in}}%
\pgfpathlineto{\pgfqpoint{1.118241in}{0.857227in}}%
\pgfusepath{stroke}%
\end{pgfscope}%
\begin{pgfscope}%
\definecolor{textcolor}{rgb}{0.000000,0.000000,0.000000}%
\pgfsetstrokecolor{textcolor}%
\pgfsetfillcolor{textcolor}%
\pgftext[x=1.034951in,y=0.653195in,,top]{\color{textcolor}\rmfamily\fontsize{10.000000}{12.000000}\selectfont \(\displaystyle {5}\)}%
\end{pgfscope}%
\begin{pgfscope}%
\pgfsetrectcap%
\pgfsetroundjoin%
\pgfsetlinewidth{0.803000pt}%
\definecolor{currentstroke}{rgb}{0.000000,0.000000,0.000000}%
\pgfsetstrokecolor{currentstroke}%
\pgfsetdash{}{0pt}%
\pgfpathmoveto{\pgfqpoint{1.583253in}{0.743415in}}%
\pgfpathlineto{\pgfqpoint{1.552967in}{0.714964in}}%
\pgfusepath{stroke}%
\end{pgfscope}%
\begin{pgfscope}%
\definecolor{textcolor}{rgb}{0.000000,0.000000,0.000000}%
\pgfsetstrokecolor{textcolor}%
\pgfsetfillcolor{textcolor}%
\pgftext[x=1.469769in,y=0.508323in,,top]{\color{textcolor}\rmfamily\fontsize{10.000000}{12.000000}\selectfont \(\displaystyle {10}\)}%
\end{pgfscope}%
\begin{pgfscope}%
\pgfsetrectcap%
\pgfsetroundjoin%
\pgfsetlinewidth{0.803000pt}%
\definecolor{currentstroke}{rgb}{0.000000,0.000000,0.000000}%
\pgfsetstrokecolor{currentstroke}%
\pgfsetdash{}{0pt}%
\pgfpathmoveto{\pgfqpoint{2.027111in}{0.598594in}}%
\pgfpathlineto{\pgfqpoint{1.997474in}{0.569500in}}%
\pgfusepath{stroke}%
\end{pgfscope}%
\begin{pgfscope}%
\definecolor{textcolor}{rgb}{0.000000,0.000000,0.000000}%
\pgfsetstrokecolor{textcolor}%
\pgfsetfillcolor{textcolor}%
\pgftext[x=1.914393in,y=0.360184in,,top]{\color{textcolor}\rmfamily\fontsize{10.000000}{12.000000}\selectfont \(\displaystyle {15}\)}%
\end{pgfscope}%
\begin{pgfscope}%
\pgfsetrectcap%
\pgfsetroundjoin%
\pgfsetlinewidth{0.803000pt}%
\definecolor{currentstroke}{rgb}{0.000000,0.000000,0.000000}%
\pgfsetstrokecolor{currentstroke}%
\pgfsetdash{}{0pt}%
\pgfpathmoveto{\pgfqpoint{3.558144in}{1.577751in}}%
\pgfpathlineto{\pgfqpoint{2.455212in}{0.445871in}}%
\pgfusepath{stroke}%
\end{pgfscope}%
\begin{pgfscope}%
\definecolor{textcolor}{rgb}{0.000000,0.000000,0.000000}%
\pgfsetstrokecolor{textcolor}%
\pgfsetfillcolor{textcolor}%
\pgftext[x=3.120747in, y=0.305657in, left, base,rotate=45.742112]{\color{textcolor}\rmfamily\fontsize{10.000000}{12.000000}\selectfont Position Y [\(\displaystyle m\)]}%
\end{pgfscope}%
\begin{pgfscope}%
\pgfsetbuttcap%
\pgfsetroundjoin%
\pgfsetlinewidth{0.803000pt}%
\definecolor{currentstroke}{rgb}{0.690196,0.690196,0.690196}%
\pgfsetstrokecolor{currentstroke}%
\pgfsetdash{}{0pt}%
\pgfpathmoveto{\pgfqpoint{0.520509in}{2.847384in}}%
\pgfpathlineto{\pgfqpoint{0.585332in}{1.296827in}}%
\pgfpathlineto{\pgfqpoint{2.642281in}{0.637849in}}%
\pgfusepath{stroke}%
\end{pgfscope}%
\begin{pgfscope}%
\pgfsetbuttcap%
\pgfsetroundjoin%
\pgfsetlinewidth{0.803000pt}%
\definecolor{currentstroke}{rgb}{0.690196,0.690196,0.690196}%
\pgfsetstrokecolor{currentstroke}%
\pgfsetdash{}{0pt}%
\pgfpathmoveto{\pgfqpoint{0.825985in}{3.070307in}}%
\pgfpathlineto{\pgfqpoint{0.876389in}{1.540797in}}%
\pgfpathlineto{\pgfqpoint{2.905784in}{0.908268in}}%
\pgfusepath{stroke}%
\end{pgfscope}%
\begin{pgfscope}%
\pgfsetbuttcap%
\pgfsetroundjoin%
\pgfsetlinewidth{0.803000pt}%
\definecolor{currentstroke}{rgb}{0.690196,0.690196,0.690196}%
\pgfsetstrokecolor{currentstroke}%
\pgfsetdash{}{0pt}%
\pgfpathmoveto{\pgfqpoint{1.119103in}{3.284211in}}%
\pgfpathlineto{\pgfqpoint{1.156168in}{1.775314in}}%
\pgfpathlineto{\pgfqpoint{3.158554in}{1.167673in}}%
\pgfusepath{stroke}%
\end{pgfscope}%
\begin{pgfscope}%
\pgfsetbuttcap%
\pgfsetroundjoin%
\pgfsetlinewidth{0.803000pt}%
\definecolor{currentstroke}{rgb}{0.690196,0.690196,0.690196}%
\pgfsetstrokecolor{currentstroke}%
\pgfsetdash{}{0pt}%
\pgfpathmoveto{\pgfqpoint{1.400598in}{3.489633in}}%
\pgfpathlineto{\pgfqpoint{1.425312in}{2.000915in}}%
\pgfpathlineto{\pgfqpoint{3.401233in}{1.416721in}}%
\pgfusepath{stroke}%
\end{pgfscope}%
\begin{pgfscope}%
\pgfsetrectcap%
\pgfsetroundjoin%
\pgfsetlinewidth{0.803000pt}%
\definecolor{currentstroke}{rgb}{0.000000,0.000000,0.000000}%
\pgfsetstrokecolor{currentstroke}%
\pgfsetdash{}{0pt}%
\pgfpathmoveto{\pgfqpoint{2.624954in}{0.643400in}}%
\pgfpathlineto{\pgfqpoint{2.676978in}{0.626733in}}%
\pgfusepath{stroke}%
\end{pgfscope}%
\begin{pgfscope}%
\definecolor{textcolor}{rgb}{0.000000,0.000000,0.000000}%
\pgfsetstrokecolor{textcolor}%
\pgfsetfillcolor{textcolor}%
\pgftext[x=2.819693in,y=0.453098in,,top]{\color{textcolor}\rmfamily\fontsize{10.000000}{12.000000}\selectfont \(\displaystyle {0}\)}%
\end{pgfscope}%
\begin{pgfscope}%
\pgfsetrectcap%
\pgfsetroundjoin%
\pgfsetlinewidth{0.803000pt}%
\definecolor{currentstroke}{rgb}{0.000000,0.000000,0.000000}%
\pgfsetstrokecolor{currentstroke}%
\pgfsetdash{}{0pt}%
\pgfpathmoveto{\pgfqpoint{2.888708in}{0.913591in}}%
\pgfpathlineto{\pgfqpoint{2.939980in}{0.897610in}}%
\pgfusepath{stroke}%
\end{pgfscope}%
\begin{pgfscope}%
\definecolor{textcolor}{rgb}{0.000000,0.000000,0.000000}%
\pgfsetstrokecolor{textcolor}%
\pgfsetfillcolor{textcolor}%
\pgftext[x=3.079659in,y=0.727517in,,top]{\color{textcolor}\rmfamily\fontsize{10.000000}{12.000000}\selectfont \(\displaystyle {5}\)}%
\end{pgfscope}%
\begin{pgfscope}%
\pgfsetrectcap%
\pgfsetroundjoin%
\pgfsetlinewidth{0.803000pt}%
\definecolor{currentstroke}{rgb}{0.000000,0.000000,0.000000}%
\pgfsetstrokecolor{currentstroke}%
\pgfsetdash{}{0pt}%
\pgfpathmoveto{\pgfqpoint{3.141722in}{1.172780in}}%
\pgfpathlineto{\pgfqpoint{3.192260in}{1.157444in}}%
\pgfusepath{stroke}%
\end{pgfscope}%
\begin{pgfscope}%
\definecolor{textcolor}{rgb}{0.000000,0.000000,0.000000}%
\pgfsetstrokecolor{textcolor}%
\pgfsetfillcolor{textcolor}%
\pgftext[x=3.329030in,y=0.990753in,,top]{\color{textcolor}\rmfamily\fontsize{10.000000}{12.000000}\selectfont \(\displaystyle {10}\)}%
\end{pgfscope}%
\begin{pgfscope}%
\pgfsetrectcap%
\pgfsetroundjoin%
\pgfsetlinewidth{0.803000pt}%
\definecolor{currentstroke}{rgb}{0.000000,0.000000,0.000000}%
\pgfsetstrokecolor{currentstroke}%
\pgfsetdash{}{0pt}%
\pgfpathmoveto{\pgfqpoint{3.384640in}{1.421627in}}%
\pgfpathlineto{\pgfqpoint{3.434461in}{1.406897in}}%
\pgfusepath{stroke}%
\end{pgfscope}%
\begin{pgfscope}%
\definecolor{textcolor}{rgb}{0.000000,0.000000,0.000000}%
\pgfsetstrokecolor{textcolor}%
\pgfsetfillcolor{textcolor}%
\pgftext[x=3.568440in,y=1.243474in,,top]{\color{textcolor}\rmfamily\fontsize{10.000000}{12.000000}\selectfont \(\displaystyle {15}\)}%
\end{pgfscope}%
\begin{pgfscope}%
\pgfsetrectcap%
\pgfsetroundjoin%
\pgfsetlinewidth{0.803000pt}%
\definecolor{currentstroke}{rgb}{0.000000,0.000000,0.000000}%
\pgfsetstrokecolor{currentstroke}%
\pgfsetdash{}{0pt}%
\pgfpathmoveto{\pgfqpoint{3.558144in}{1.577751in}}%
\pgfpathlineto{\pgfqpoint{3.628038in}{3.104037in}}%
\pgfusepath{stroke}%
\end{pgfscope}%
\begin{pgfscope}%
\definecolor{textcolor}{rgb}{0.000000,0.000000,0.000000}%
\pgfsetstrokecolor{textcolor}%
\pgfsetfillcolor{textcolor}%
\pgftext[x=4.167903in, y=1.963517in, left, base,rotate=87.378092]{\color{textcolor}\rmfamily\fontsize{10.000000}{12.000000}\selectfont Position Z [\(\displaystyle m\)]}%
\end{pgfscope}%
\begin{pgfscope}%
\pgfsetbuttcap%
\pgfsetroundjoin%
\pgfsetlinewidth{0.803000pt}%
\definecolor{currentstroke}{rgb}{0.690196,0.690196,0.690196}%
\pgfsetstrokecolor{currentstroke}%
\pgfsetdash{}{0pt}%
\pgfpathmoveto{\pgfqpoint{3.562420in}{1.671128in}}%
\pgfpathlineto{\pgfqpoint{1.598573in}{2.237465in}}%
\pgfpathlineto{\pgfqpoint{0.374469in}{1.219546in}}%
\pgfusepath{stroke}%
\end{pgfscope}%
\begin{pgfscope}%
\pgfsetbuttcap%
\pgfsetroundjoin%
\pgfsetlinewidth{0.803000pt}%
\definecolor{currentstroke}{rgb}{0.690196,0.690196,0.690196}%
\pgfsetstrokecolor{currentstroke}%
\pgfsetdash{}{0pt}%
\pgfpathmoveto{\pgfqpoint{3.575840in}{1.964176in}}%
\pgfpathlineto{\pgfqpoint{1.595311in}{2.521166in}}%
\pgfpathlineto{\pgfqpoint{0.360014in}{1.519723in}}%
\pgfusepath{stroke}%
\end{pgfscope}%
\begin{pgfscope}%
\pgfsetbuttcap%
\pgfsetroundjoin%
\pgfsetlinewidth{0.803000pt}%
\definecolor{currentstroke}{rgb}{0.690196,0.690196,0.690196}%
\pgfsetstrokecolor{currentstroke}%
\pgfsetdash{}{0pt}%
\pgfpathmoveto{\pgfqpoint{3.589492in}{2.262295in}}%
\pgfpathlineto{\pgfqpoint{1.591995in}{2.809540in}}%
\pgfpathlineto{\pgfqpoint{0.345299in}{1.825295in}}%
\pgfusepath{stroke}%
\end{pgfscope}%
\begin{pgfscope}%
\pgfsetbuttcap%
\pgfsetroundjoin%
\pgfsetlinewidth{0.803000pt}%
\definecolor{currentstroke}{rgb}{0.690196,0.690196,0.690196}%
\pgfsetstrokecolor{currentstroke}%
\pgfsetdash{}{0pt}%
\pgfpathmoveto{\pgfqpoint{3.603382in}{2.565618in}}%
\pgfpathlineto{\pgfqpoint{1.588624in}{3.102704in}}%
\pgfpathlineto{\pgfqpoint{0.330317in}{2.136406in}}%
\pgfusepath{stroke}%
\end{pgfscope}%
\begin{pgfscope}%
\pgfsetbuttcap%
\pgfsetroundjoin%
\pgfsetlinewidth{0.803000pt}%
\definecolor{currentstroke}{rgb}{0.690196,0.690196,0.690196}%
\pgfsetstrokecolor{currentstroke}%
\pgfsetdash{}{0pt}%
\pgfpathmoveto{\pgfqpoint{3.617516in}{2.874282in}}%
\pgfpathlineto{\pgfqpoint{1.585196in}{3.400777in}}%
\pgfpathlineto{\pgfqpoint{0.315060in}{2.453210in}}%
\pgfusepath{stroke}%
\end{pgfscope}%
\begin{pgfscope}%
\pgfsetrectcap%
\pgfsetroundjoin%
\pgfsetlinewidth{0.803000pt}%
\definecolor{currentstroke}{rgb}{0.000000,0.000000,0.000000}%
\pgfsetstrokecolor{currentstroke}%
\pgfsetdash{}{0pt}%
\pgfpathmoveto{\pgfqpoint{3.545936in}{1.675882in}}%
\pgfpathlineto{\pgfqpoint{3.595428in}{1.661609in}}%
\pgfusepath{stroke}%
\end{pgfscope}%
\begin{pgfscope}%
\definecolor{textcolor}{rgb}{0.000000,0.000000,0.000000}%
\pgfsetstrokecolor{textcolor}%
\pgfsetfillcolor{textcolor}%
\pgftext[x=3.816553in,y=1.707127in,,top]{\color{textcolor}\rmfamily\fontsize{10.000000}{12.000000}\selectfont \(\displaystyle {0}\)}%
\end{pgfscope}%
\begin{pgfscope}%
\pgfsetrectcap%
\pgfsetroundjoin%
\pgfsetlinewidth{0.803000pt}%
\definecolor{currentstroke}{rgb}{0.000000,0.000000,0.000000}%
\pgfsetstrokecolor{currentstroke}%
\pgfsetdash{}{0pt}%
\pgfpathmoveto{\pgfqpoint{3.559209in}{1.968853in}}%
\pgfpathlineto{\pgfqpoint{3.609142in}{1.954811in}}%
\pgfusepath{stroke}%
\end{pgfscope}%
\begin{pgfscope}%
\definecolor{textcolor}{rgb}{0.000000,0.000000,0.000000}%
\pgfsetstrokecolor{textcolor}%
\pgfsetfillcolor{textcolor}%
\pgftext[x=3.832106in,y=1.999594in,,top]{\color{textcolor}\rmfamily\fontsize{10.000000}{12.000000}\selectfont \(\displaystyle {1}\)}%
\end{pgfscope}%
\begin{pgfscope}%
\pgfsetrectcap%
\pgfsetroundjoin%
\pgfsetlinewidth{0.803000pt}%
\definecolor{currentstroke}{rgb}{0.000000,0.000000,0.000000}%
\pgfsetstrokecolor{currentstroke}%
\pgfsetdash{}{0pt}%
\pgfpathmoveto{\pgfqpoint{3.572711in}{2.266892in}}%
\pgfpathlineto{\pgfqpoint{3.623093in}{2.253090in}}%
\pgfusepath{stroke}%
\end{pgfscope}%
\begin{pgfscope}%
\definecolor{textcolor}{rgb}{0.000000,0.000000,0.000000}%
\pgfsetstrokecolor{textcolor}%
\pgfsetfillcolor{textcolor}%
\pgftext[x=3.847926in,y=2.297107in,,top]{\color{textcolor}\rmfamily\fontsize{10.000000}{12.000000}\selectfont \(\displaystyle {2}\)}%
\end{pgfscope}%
\begin{pgfscope}%
\pgfsetrectcap%
\pgfsetroundjoin%
\pgfsetlinewidth{0.803000pt}%
\definecolor{currentstroke}{rgb}{0.000000,0.000000,0.000000}%
\pgfsetstrokecolor{currentstroke}%
\pgfsetdash{}{0pt}%
\pgfpathmoveto{\pgfqpoint{3.586449in}{2.570132in}}%
\pgfpathlineto{\pgfqpoint{3.637288in}{2.556579in}}%
\pgfusepath{stroke}%
\end{pgfscope}%
\begin{pgfscope}%
\definecolor{textcolor}{rgb}{0.000000,0.000000,0.000000}%
\pgfsetstrokecolor{textcolor}%
\pgfsetfillcolor{textcolor}%
\pgftext[x=3.864022in,y=2.599797in,,top]{\color{textcolor}\rmfamily\fontsize{10.000000}{12.000000}\selectfont \(\displaystyle {3}\)}%
\end{pgfscope}%
\begin{pgfscope}%
\pgfsetrectcap%
\pgfsetroundjoin%
\pgfsetlinewidth{0.803000pt}%
\definecolor{currentstroke}{rgb}{0.000000,0.000000,0.000000}%
\pgfsetstrokecolor{currentstroke}%
\pgfsetdash{}{0pt}%
\pgfpathmoveto{\pgfqpoint{3.600429in}{2.878708in}}%
\pgfpathlineto{\pgfqpoint{3.651733in}{2.865418in}}%
\pgfusepath{stroke}%
\end{pgfscope}%
\begin{pgfscope}%
\definecolor{textcolor}{rgb}{0.000000,0.000000,0.000000}%
\pgfsetstrokecolor{textcolor}%
\pgfsetfillcolor{textcolor}%
\pgftext[x=3.880400in,y=2.907800in,,top]{\color{textcolor}\rmfamily\fontsize{10.000000}{12.000000}\selectfont \(\displaystyle {4}\)}%
\end{pgfscope}%
\begin{pgfscope}%
\pgfpathrectangle{\pgfqpoint{0.100000in}{0.212622in}}{\pgfqpoint{3.696000in}{3.696000in}}%
\pgfusepath{clip}%
\pgfsetrectcap%
\pgfsetroundjoin%
\pgfsetlinewidth{1.505625pt}%
\definecolor{currentstroke}{rgb}{0.121569,0.466667,0.705882}%
\pgfsetstrokecolor{currentstroke}%
\pgfsetdash{}{0pt}%
\pgfpathmoveto{\pgfqpoint{0.914292in}{1.285747in}}%
\pgfpathlineto{\pgfqpoint{1.796340in}{2.042663in}}%
\pgfpathlineto{\pgfqpoint{2.298639in}{0.845607in}}%
\pgfpathlineto{\pgfqpoint{0.914292in}{1.285747in}}%
\pgfusepath{stroke}%
\end{pgfscope}%
\begin{pgfscope}%
\pgfpathrectangle{\pgfqpoint{0.100000in}{0.212622in}}{\pgfqpoint{3.696000in}{3.696000in}}%
\pgfusepath{clip}%
\pgfsetrectcap%
\pgfsetroundjoin%
\pgfsetlinewidth{1.505625pt}%
\definecolor{currentstroke}{rgb}{1.000000,0.000000,0.000000}%
\pgfsetstrokecolor{currentstroke}%
\pgfsetdash{}{0pt}%
\pgfpathmoveto{\pgfqpoint{0.913757in}{1.285322in}}%
\pgfpathlineto{\pgfqpoint{0.914292in}{1.285747in}}%
\pgfusepath{stroke}%
\end{pgfscope}%
\begin{pgfscope}%
\pgfpathrectangle{\pgfqpoint{0.100000in}{0.212622in}}{\pgfqpoint{3.696000in}{3.696000in}}%
\pgfusepath{clip}%
\pgfsetrectcap%
\pgfsetroundjoin%
\pgfsetlinewidth{1.505625pt}%
\definecolor{currentstroke}{rgb}{1.000000,0.000000,0.000000}%
\pgfsetstrokecolor{currentstroke}%
\pgfsetdash{}{0pt}%
\pgfpathmoveto{\pgfqpoint{1.093551in}{1.527512in}}%
\pgfpathlineto{\pgfqpoint{0.914292in}{1.285747in}}%
\pgfusepath{stroke}%
\end{pgfscope}%
\begin{pgfscope}%
\pgfpathrectangle{\pgfqpoint{0.100000in}{0.212622in}}{\pgfqpoint{3.696000in}{3.696000in}}%
\pgfusepath{clip}%
\pgfsetrectcap%
\pgfsetroundjoin%
\pgfsetlinewidth{1.505625pt}%
\definecolor{currentstroke}{rgb}{1.000000,0.000000,0.000000}%
\pgfsetstrokecolor{currentstroke}%
\pgfsetdash{}{0pt}%
\pgfpathmoveto{\pgfqpoint{1.943659in}{2.109762in}}%
\pgfpathlineto{\pgfqpoint{1.796340in}{2.042663in}}%
\pgfusepath{stroke}%
\end{pgfscope}%
\begin{pgfscope}%
\pgfpathrectangle{\pgfqpoint{0.100000in}{0.212622in}}{\pgfqpoint{3.696000in}{3.696000in}}%
\pgfusepath{clip}%
\pgfsetrectcap%
\pgfsetroundjoin%
\pgfsetlinewidth{1.505625pt}%
\definecolor{currentstroke}{rgb}{1.000000,0.000000,0.000000}%
\pgfsetstrokecolor{currentstroke}%
\pgfsetdash{}{0pt}%
\pgfpathmoveto{\pgfqpoint{2.404987in}{1.351665in}}%
\pgfpathlineto{\pgfqpoint{2.298639in}{0.845607in}}%
\pgfusepath{stroke}%
\end{pgfscope}%
\begin{pgfscope}%
\pgfpathrectangle{\pgfqpoint{0.100000in}{0.212622in}}{\pgfqpoint{3.696000in}{3.696000in}}%
\pgfusepath{clip}%
\pgfsetrectcap%
\pgfsetroundjoin%
\pgfsetlinewidth{1.505625pt}%
\definecolor{currentstroke}{rgb}{1.000000,0.000000,0.000000}%
\pgfsetstrokecolor{currentstroke}%
\pgfsetdash{}{0pt}%
\pgfpathmoveto{\pgfqpoint{0.604784in}{2.669036in}}%
\pgfpathlineto{\pgfqpoint{0.914292in}{1.285747in}}%
\pgfusepath{stroke}%
\end{pgfscope}%
\begin{pgfscope}%
\pgfpathrectangle{\pgfqpoint{0.100000in}{0.212622in}}{\pgfqpoint{3.696000in}{3.696000in}}%
\pgfusepath{clip}%
\pgfsetbuttcap%
\pgfsetroundjoin%
\definecolor{currentfill}{rgb}{1.000000,0.498039,0.054902}%
\pgfsetfillcolor{currentfill}%
\pgfsetfillopacity{0.300000}%
\pgfsetlinewidth{1.003750pt}%
\definecolor{currentstroke}{rgb}{1.000000,0.498039,0.054902}%
\pgfsetstrokecolor{currentstroke}%
\pgfsetstrokeopacity{0.300000}%
\pgfsetdash{}{0pt}%
\pgfpathmoveto{\pgfqpoint{1.943659in}{2.078705in}}%
\pgfpathcurveto{\pgfqpoint{1.951895in}{2.078705in}}{\pgfqpoint{1.959795in}{2.081978in}}{\pgfqpoint{1.965619in}{2.087802in}}%
\pgfpathcurveto{\pgfqpoint{1.971443in}{2.093625in}}{\pgfqpoint{1.974715in}{2.101525in}}{\pgfqpoint{1.974715in}{2.109762in}}%
\pgfpathcurveto{\pgfqpoint{1.974715in}{2.117998in}}{\pgfqpoint{1.971443in}{2.125898in}}{\pgfqpoint{1.965619in}{2.131722in}}%
\pgfpathcurveto{\pgfqpoint{1.959795in}{2.137546in}}{\pgfqpoint{1.951895in}{2.140818in}}{\pgfqpoint{1.943659in}{2.140818in}}%
\pgfpathcurveto{\pgfqpoint{1.935422in}{2.140818in}}{\pgfqpoint{1.927522in}{2.137546in}}{\pgfqpoint{1.921698in}{2.131722in}}%
\pgfpathcurveto{\pgfqpoint{1.915874in}{2.125898in}}{\pgfqpoint{1.912602in}{2.117998in}}{\pgfqpoint{1.912602in}{2.109762in}}%
\pgfpathcurveto{\pgfqpoint{1.912602in}{2.101525in}}{\pgfqpoint{1.915874in}{2.093625in}}{\pgfqpoint{1.921698in}{2.087802in}}%
\pgfpathcurveto{\pgfqpoint{1.927522in}{2.081978in}}{\pgfqpoint{1.935422in}{2.078705in}}{\pgfqpoint{1.943659in}{2.078705in}}%
\pgfpathclose%
\pgfusepath{stroke,fill}%
\end{pgfscope}%
\begin{pgfscope}%
\pgfpathrectangle{\pgfqpoint{0.100000in}{0.212622in}}{\pgfqpoint{3.696000in}{3.696000in}}%
\pgfusepath{clip}%
\pgfsetbuttcap%
\pgfsetroundjoin%
\definecolor{currentfill}{rgb}{1.000000,0.498039,0.054902}%
\pgfsetfillcolor{currentfill}%
\pgfsetfillopacity{0.558330}%
\pgfsetlinewidth{1.003750pt}%
\definecolor{currentstroke}{rgb}{1.000000,0.498039,0.054902}%
\pgfsetstrokecolor{currentstroke}%
\pgfsetstrokeopacity{0.558330}%
\pgfsetdash{}{0pt}%
\pgfpathmoveto{\pgfqpoint{1.093551in}{1.496456in}}%
\pgfpathcurveto{\pgfqpoint{1.101787in}{1.496456in}}{\pgfqpoint{1.109687in}{1.499728in}}{\pgfqpoint{1.115511in}{1.505552in}}%
\pgfpathcurveto{\pgfqpoint{1.121335in}{1.511376in}}{\pgfqpoint{1.124607in}{1.519276in}}{\pgfqpoint{1.124607in}{1.527512in}}%
\pgfpathcurveto{\pgfqpoint{1.124607in}{1.535749in}}{\pgfqpoint{1.121335in}{1.543649in}}{\pgfqpoint{1.115511in}{1.549473in}}%
\pgfpathcurveto{\pgfqpoint{1.109687in}{1.555296in}}{\pgfqpoint{1.101787in}{1.558569in}}{\pgfqpoint{1.093551in}{1.558569in}}%
\pgfpathcurveto{\pgfqpoint{1.085315in}{1.558569in}}{\pgfqpoint{1.077414in}{1.555296in}}{\pgfqpoint{1.071591in}{1.549473in}}%
\pgfpathcurveto{\pgfqpoint{1.065767in}{1.543649in}}{\pgfqpoint{1.062494in}{1.535749in}}{\pgfqpoint{1.062494in}{1.527512in}}%
\pgfpathcurveto{\pgfqpoint{1.062494in}{1.519276in}}{\pgfqpoint{1.065767in}{1.511376in}}{\pgfqpoint{1.071591in}{1.505552in}}%
\pgfpathcurveto{\pgfqpoint{1.077414in}{1.499728in}}{\pgfqpoint{1.085315in}{1.496456in}}{\pgfqpoint{1.093551in}{1.496456in}}%
\pgfpathclose%
\pgfusepath{stroke,fill}%
\end{pgfscope}%
\begin{pgfscope}%
\pgfpathrectangle{\pgfqpoint{0.100000in}{0.212622in}}{\pgfqpoint{3.696000in}{3.696000in}}%
\pgfusepath{clip}%
\pgfsetbuttcap%
\pgfsetroundjoin%
\definecolor{currentfill}{rgb}{1.000000,0.498039,0.054902}%
\pgfsetfillcolor{currentfill}%
\pgfsetfillopacity{0.622372}%
\pgfsetlinewidth{1.003750pt}%
\definecolor{currentstroke}{rgb}{1.000000,0.498039,0.054902}%
\pgfsetstrokecolor{currentstroke}%
\pgfsetstrokeopacity{0.622372}%
\pgfsetdash{}{0pt}%
\pgfpathmoveto{\pgfqpoint{0.913757in}{1.254265in}}%
\pgfpathcurveto{\pgfqpoint{0.921993in}{1.254265in}}{\pgfqpoint{0.929893in}{1.257538in}}{\pgfqpoint{0.935717in}{1.263362in}}%
\pgfpathcurveto{\pgfqpoint{0.941541in}{1.269186in}}{\pgfqpoint{0.944813in}{1.277086in}}{\pgfqpoint{0.944813in}{1.285322in}}%
\pgfpathcurveto{\pgfqpoint{0.944813in}{1.293558in}}{\pgfqpoint{0.941541in}{1.301458in}}{\pgfqpoint{0.935717in}{1.307282in}}%
\pgfpathcurveto{\pgfqpoint{0.929893in}{1.313106in}}{\pgfqpoint{0.921993in}{1.316378in}}{\pgfqpoint{0.913757in}{1.316378in}}%
\pgfpathcurveto{\pgfqpoint{0.905520in}{1.316378in}}{\pgfqpoint{0.897620in}{1.313106in}}{\pgfqpoint{0.891796in}{1.307282in}}%
\pgfpathcurveto{\pgfqpoint{0.885972in}{1.301458in}}{\pgfqpoint{0.882700in}{1.293558in}}{\pgfqpoint{0.882700in}{1.285322in}}%
\pgfpathcurveto{\pgfqpoint{0.882700in}{1.277086in}}{\pgfqpoint{0.885972in}{1.269186in}}{\pgfqpoint{0.891796in}{1.263362in}}%
\pgfpathcurveto{\pgfqpoint{0.897620in}{1.257538in}}{\pgfqpoint{0.905520in}{1.254265in}}{\pgfqpoint{0.913757in}{1.254265in}}%
\pgfpathclose%
\pgfusepath{stroke,fill}%
\end{pgfscope}%
\begin{pgfscope}%
\pgfpathrectangle{\pgfqpoint{0.100000in}{0.212622in}}{\pgfqpoint{3.696000in}{3.696000in}}%
\pgfusepath{clip}%
\pgfsetbuttcap%
\pgfsetroundjoin%
\definecolor{currentfill}{rgb}{1.000000,0.498039,0.054902}%
\pgfsetfillcolor{currentfill}%
\pgfsetfillopacity{0.823184}%
\pgfsetlinewidth{1.003750pt}%
\definecolor{currentstroke}{rgb}{1.000000,0.498039,0.054902}%
\pgfsetstrokecolor{currentstroke}%
\pgfsetstrokeopacity{0.823184}%
\pgfsetdash{}{0pt}%
\pgfpathmoveto{\pgfqpoint{0.604784in}{2.637979in}}%
\pgfpathcurveto{\pgfqpoint{0.613021in}{2.637979in}}{\pgfqpoint{0.620921in}{2.641251in}}{\pgfqpoint{0.626745in}{2.647075in}}%
\pgfpathcurveto{\pgfqpoint{0.632569in}{2.652899in}}{\pgfqpoint{0.635841in}{2.660799in}}{\pgfqpoint{0.635841in}{2.669036in}}%
\pgfpathcurveto{\pgfqpoint{0.635841in}{2.677272in}}{\pgfqpoint{0.632569in}{2.685172in}}{\pgfqpoint{0.626745in}{2.690996in}}%
\pgfpathcurveto{\pgfqpoint{0.620921in}{2.696820in}}{\pgfqpoint{0.613021in}{2.700092in}}{\pgfqpoint{0.604784in}{2.700092in}}%
\pgfpathcurveto{\pgfqpoint{0.596548in}{2.700092in}}{\pgfqpoint{0.588648in}{2.696820in}}{\pgfqpoint{0.582824in}{2.690996in}}%
\pgfpathcurveto{\pgfqpoint{0.577000in}{2.685172in}}{\pgfqpoint{0.573728in}{2.677272in}}{\pgfqpoint{0.573728in}{2.669036in}}%
\pgfpathcurveto{\pgfqpoint{0.573728in}{2.660799in}}{\pgfqpoint{0.577000in}{2.652899in}}{\pgfqpoint{0.582824in}{2.647075in}}%
\pgfpathcurveto{\pgfqpoint{0.588648in}{2.641251in}}{\pgfqpoint{0.596548in}{2.637979in}}{\pgfqpoint{0.604784in}{2.637979in}}%
\pgfpathclose%
\pgfusepath{stroke,fill}%
\end{pgfscope}%
\begin{pgfscope}%
\pgfpathrectangle{\pgfqpoint{0.100000in}{0.212622in}}{\pgfqpoint{3.696000in}{3.696000in}}%
\pgfusepath{clip}%
\pgfsetbuttcap%
\pgfsetroundjoin%
\definecolor{currentfill}{rgb}{1.000000,0.498039,0.054902}%
\pgfsetfillcolor{currentfill}%
\pgfsetlinewidth{1.003750pt}%
\definecolor{currentstroke}{rgb}{1.000000,0.498039,0.054902}%
\pgfsetstrokecolor{currentstroke}%
\pgfsetdash{}{0pt}%
\pgfpathmoveto{\pgfqpoint{2.404987in}{1.320609in}}%
\pgfpathcurveto{\pgfqpoint{2.413223in}{1.320609in}}{\pgfqpoint{2.421123in}{1.323881in}}{\pgfqpoint{2.426947in}{1.329705in}}%
\pgfpathcurveto{\pgfqpoint{2.432771in}{1.335529in}}{\pgfqpoint{2.436043in}{1.343429in}}{\pgfqpoint{2.436043in}{1.351665in}}%
\pgfpathcurveto{\pgfqpoint{2.436043in}{1.359901in}}{\pgfqpoint{2.432771in}{1.367802in}}{\pgfqpoint{2.426947in}{1.373625in}}%
\pgfpathcurveto{\pgfqpoint{2.421123in}{1.379449in}}{\pgfqpoint{2.413223in}{1.382722in}}{\pgfqpoint{2.404987in}{1.382722in}}%
\pgfpathcurveto{\pgfqpoint{2.396751in}{1.382722in}}{\pgfqpoint{2.388851in}{1.379449in}}{\pgfqpoint{2.383027in}{1.373625in}}%
\pgfpathcurveto{\pgfqpoint{2.377203in}{1.367802in}}{\pgfqpoint{2.373930in}{1.359901in}}{\pgfqpoint{2.373930in}{1.351665in}}%
\pgfpathcurveto{\pgfqpoint{2.373930in}{1.343429in}}{\pgfqpoint{2.377203in}{1.335529in}}{\pgfqpoint{2.383027in}{1.329705in}}%
\pgfpathcurveto{\pgfqpoint{2.388851in}{1.323881in}}{\pgfqpoint{2.396751in}{1.320609in}}{\pgfqpoint{2.404987in}{1.320609in}}%
\pgfpathclose%
\pgfusepath{stroke,fill}%
\end{pgfscope}%
\begin{pgfscope}%
\definecolor{textcolor}{rgb}{0.000000,0.000000,0.000000}%
\pgfsetstrokecolor{textcolor}%
\pgfsetfillcolor{textcolor}%
\pgftext[x=1.948000in,y=3.991956in,,base]{\color{textcolor}\rmfamily\fontsize{12.000000}{14.400000}\selectfont Tilt}%
\end{pgfscope}%
\begin{pgfscope}%
\pgfpathrectangle{\pgfqpoint{0.100000in}{0.212622in}}{\pgfqpoint{3.696000in}{3.696000in}}%
\pgfusepath{clip}%
\pgfsetbuttcap%
\pgfsetroundjoin%
\definecolor{currentfill}{rgb}{0.121569,0.466667,0.705882}%
\pgfsetfillcolor{currentfill}%
\pgfsetfillopacity{0.300000}%
\pgfsetlinewidth{1.003750pt}%
\definecolor{currentstroke}{rgb}{0.121569,0.466667,0.705882}%
\pgfsetstrokecolor{currentstroke}%
\pgfsetstrokeopacity{0.300000}%
\pgfsetdash{}{0pt}%
\pgfpathmoveto{\pgfqpoint{1.943052in}{2.078993in}}%
\pgfpathcurveto{\pgfqpoint{1.951288in}{2.078993in}}{\pgfqpoint{1.959188in}{2.082266in}}{\pgfqpoint{1.965012in}{2.088090in}}%
\pgfpathcurveto{\pgfqpoint{1.970836in}{2.093914in}}{\pgfqpoint{1.974108in}{2.101814in}}{\pgfqpoint{1.974108in}{2.110050in}}%
\pgfpathcurveto{\pgfqpoint{1.974108in}{2.118286in}}{\pgfqpoint{1.970836in}{2.126186in}}{\pgfqpoint{1.965012in}{2.132010in}}%
\pgfpathcurveto{\pgfqpoint{1.959188in}{2.137834in}}{\pgfqpoint{1.951288in}{2.141106in}}{\pgfqpoint{1.943052in}{2.141106in}}%
\pgfpathcurveto{\pgfqpoint{1.934816in}{2.141106in}}{\pgfqpoint{1.926916in}{2.137834in}}{\pgfqpoint{1.921092in}{2.132010in}}%
\pgfpathcurveto{\pgfqpoint{1.915268in}{2.126186in}}{\pgfqpoint{1.911995in}{2.118286in}}{\pgfqpoint{1.911995in}{2.110050in}}%
\pgfpathcurveto{\pgfqpoint{1.911995in}{2.101814in}}{\pgfqpoint{1.915268in}{2.093914in}}{\pgfqpoint{1.921092in}{2.088090in}}%
\pgfpathcurveto{\pgfqpoint{1.926916in}{2.082266in}}{\pgfqpoint{1.934816in}{2.078993in}}{\pgfqpoint{1.943052in}{2.078993in}}%
\pgfpathclose%
\pgfusepath{stroke,fill}%
\end{pgfscope}%
\begin{pgfscope}%
\pgfpathrectangle{\pgfqpoint{0.100000in}{0.212622in}}{\pgfqpoint{3.696000in}{3.696000in}}%
\pgfusepath{clip}%
\pgfsetbuttcap%
\pgfsetroundjoin%
\definecolor{currentfill}{rgb}{0.121569,0.466667,0.705882}%
\pgfsetfillcolor{currentfill}%
\pgfsetfillopacity{0.300002}%
\pgfsetlinewidth{1.003750pt}%
\definecolor{currentstroke}{rgb}{0.121569,0.466667,0.705882}%
\pgfsetstrokecolor{currentstroke}%
\pgfsetstrokeopacity{0.300002}%
\pgfsetdash{}{0pt}%
\pgfpathmoveto{\pgfqpoint{1.943455in}{2.078852in}}%
\pgfpathcurveto{\pgfqpoint{1.951691in}{2.078852in}}{\pgfqpoint{1.959591in}{2.082125in}}{\pgfqpoint{1.965415in}{2.087949in}}%
\pgfpathcurveto{\pgfqpoint{1.971239in}{2.093772in}}{\pgfqpoint{1.974511in}{2.101673in}}{\pgfqpoint{1.974511in}{2.109909in}}%
\pgfpathcurveto{\pgfqpoint{1.974511in}{2.118145in}}{\pgfqpoint{1.971239in}{2.126045in}}{\pgfqpoint{1.965415in}{2.131869in}}%
\pgfpathcurveto{\pgfqpoint{1.959591in}{2.137693in}}{\pgfqpoint{1.951691in}{2.140965in}}{\pgfqpoint{1.943455in}{2.140965in}}%
\pgfpathcurveto{\pgfqpoint{1.935218in}{2.140965in}}{\pgfqpoint{1.927318in}{2.137693in}}{\pgfqpoint{1.921494in}{2.131869in}}%
\pgfpathcurveto{\pgfqpoint{1.915670in}{2.126045in}}{\pgfqpoint{1.912398in}{2.118145in}}{\pgfqpoint{1.912398in}{2.109909in}}%
\pgfpathcurveto{\pgfqpoint{1.912398in}{2.101673in}}{\pgfqpoint{1.915670in}{2.093772in}}{\pgfqpoint{1.921494in}{2.087949in}}%
\pgfpathcurveto{\pgfqpoint{1.927318in}{2.082125in}}{\pgfqpoint{1.935218in}{2.078852in}}{\pgfqpoint{1.943455in}{2.078852in}}%
\pgfpathclose%
\pgfusepath{stroke,fill}%
\end{pgfscope}%
\begin{pgfscope}%
\pgfpathrectangle{\pgfqpoint{0.100000in}{0.212622in}}{\pgfqpoint{3.696000in}{3.696000in}}%
\pgfusepath{clip}%
\pgfsetbuttcap%
\pgfsetroundjoin%
\definecolor{currentfill}{rgb}{0.121569,0.466667,0.705882}%
\pgfsetfillcolor{currentfill}%
\pgfsetfillopacity{0.300007}%
\pgfsetlinewidth{1.003750pt}%
\definecolor{currentstroke}{rgb}{0.121569,0.466667,0.705882}%
\pgfsetstrokecolor{currentstroke}%
\pgfsetstrokeopacity{0.300007}%
\pgfsetdash{}{0pt}%
\pgfpathmoveto{\pgfqpoint{1.943659in}{2.078705in}}%
\pgfpathcurveto{\pgfqpoint{1.951895in}{2.078705in}}{\pgfqpoint{1.959795in}{2.081978in}}{\pgfqpoint{1.965619in}{2.087802in}}%
\pgfpathcurveto{\pgfqpoint{1.971443in}{2.093625in}}{\pgfqpoint{1.974715in}{2.101525in}}{\pgfqpoint{1.974715in}{2.109762in}}%
\pgfpathcurveto{\pgfqpoint{1.974715in}{2.117998in}}{\pgfqpoint{1.971443in}{2.125898in}}{\pgfqpoint{1.965619in}{2.131722in}}%
\pgfpathcurveto{\pgfqpoint{1.959795in}{2.137546in}}{\pgfqpoint{1.951895in}{2.140818in}}{\pgfqpoint{1.943659in}{2.140818in}}%
\pgfpathcurveto{\pgfqpoint{1.935422in}{2.140818in}}{\pgfqpoint{1.927522in}{2.137546in}}{\pgfqpoint{1.921698in}{2.131722in}}%
\pgfpathcurveto{\pgfqpoint{1.915874in}{2.125898in}}{\pgfqpoint{1.912602in}{2.117998in}}{\pgfqpoint{1.912602in}{2.109762in}}%
\pgfpathcurveto{\pgfqpoint{1.912602in}{2.101525in}}{\pgfqpoint{1.915874in}{2.093625in}}{\pgfqpoint{1.921698in}{2.087802in}}%
\pgfpathcurveto{\pgfqpoint{1.927522in}{2.081978in}}{\pgfqpoint{1.935422in}{2.078705in}}{\pgfqpoint{1.943659in}{2.078705in}}%
\pgfpathclose%
\pgfusepath{stroke,fill}%
\end{pgfscope}%
\begin{pgfscope}%
\pgfpathrectangle{\pgfqpoint{0.100000in}{0.212622in}}{\pgfqpoint{3.696000in}{3.696000in}}%
\pgfusepath{clip}%
\pgfsetbuttcap%
\pgfsetroundjoin%
\definecolor{currentfill}{rgb}{0.121569,0.466667,0.705882}%
\pgfsetfillcolor{currentfill}%
\pgfsetfillopacity{0.300023}%
\pgfsetlinewidth{1.003750pt}%
\definecolor{currentstroke}{rgb}{0.121569,0.466667,0.705882}%
\pgfsetstrokecolor{currentstroke}%
\pgfsetstrokeopacity{0.300023}%
\pgfsetdash{}{0pt}%
\pgfpathmoveto{\pgfqpoint{1.943754in}{2.078664in}}%
\pgfpathcurveto{\pgfqpoint{1.951990in}{2.078664in}}{\pgfqpoint{1.959890in}{2.081937in}}{\pgfqpoint{1.965714in}{2.087761in}}%
\pgfpathcurveto{\pgfqpoint{1.971538in}{2.093584in}}{\pgfqpoint{1.974810in}{2.101485in}}{\pgfqpoint{1.974810in}{2.109721in}}%
\pgfpathcurveto{\pgfqpoint{1.974810in}{2.117957in}}{\pgfqpoint{1.971538in}{2.125857in}}{\pgfqpoint{1.965714in}{2.131681in}}%
\pgfpathcurveto{\pgfqpoint{1.959890in}{2.137505in}}{\pgfqpoint{1.951990in}{2.140777in}}{\pgfqpoint{1.943754in}{2.140777in}}%
\pgfpathcurveto{\pgfqpoint{1.935517in}{2.140777in}}{\pgfqpoint{1.927617in}{2.137505in}}{\pgfqpoint{1.921793in}{2.131681in}}%
\pgfpathcurveto{\pgfqpoint{1.915969in}{2.125857in}}{\pgfqpoint{1.912697in}{2.117957in}}{\pgfqpoint{1.912697in}{2.109721in}}%
\pgfpathcurveto{\pgfqpoint{1.912697in}{2.101485in}}{\pgfqpoint{1.915969in}{2.093584in}}{\pgfqpoint{1.921793in}{2.087761in}}%
\pgfpathcurveto{\pgfqpoint{1.927617in}{2.081937in}}{\pgfqpoint{1.935517in}{2.078664in}}{\pgfqpoint{1.943754in}{2.078664in}}%
\pgfpathclose%
\pgfusepath{stroke,fill}%
\end{pgfscope}%
\begin{pgfscope}%
\pgfpathrectangle{\pgfqpoint{0.100000in}{0.212622in}}{\pgfqpoint{3.696000in}{3.696000in}}%
\pgfusepath{clip}%
\pgfsetbuttcap%
\pgfsetroundjoin%
\definecolor{currentfill}{rgb}{0.121569,0.466667,0.705882}%
\pgfsetfillcolor{currentfill}%
\pgfsetfillopacity{0.300039}%
\pgfsetlinewidth{1.003750pt}%
\definecolor{currentstroke}{rgb}{0.121569,0.466667,0.705882}%
\pgfsetstrokecolor{currentstroke}%
\pgfsetstrokeopacity{0.300039}%
\pgfsetdash{}{0pt}%
\pgfpathmoveto{\pgfqpoint{1.943796in}{2.078664in}}%
\pgfpathcurveto{\pgfqpoint{1.952033in}{2.078664in}}{\pgfqpoint{1.959933in}{2.081937in}}{\pgfqpoint{1.965757in}{2.087761in}}%
\pgfpathcurveto{\pgfqpoint{1.971581in}{2.093584in}}{\pgfqpoint{1.974853in}{2.101485in}}{\pgfqpoint{1.974853in}{2.109721in}}%
\pgfpathcurveto{\pgfqpoint{1.974853in}{2.117957in}}{\pgfqpoint{1.971581in}{2.125857in}}{\pgfqpoint{1.965757in}{2.131681in}}%
\pgfpathcurveto{\pgfqpoint{1.959933in}{2.137505in}}{\pgfqpoint{1.952033in}{2.140777in}}{\pgfqpoint{1.943796in}{2.140777in}}%
\pgfpathcurveto{\pgfqpoint{1.935560in}{2.140777in}}{\pgfqpoint{1.927660in}{2.137505in}}{\pgfqpoint{1.921836in}{2.131681in}}%
\pgfpathcurveto{\pgfqpoint{1.916012in}{2.125857in}}{\pgfqpoint{1.912740in}{2.117957in}}{\pgfqpoint{1.912740in}{2.109721in}}%
\pgfpathcurveto{\pgfqpoint{1.912740in}{2.101485in}}{\pgfqpoint{1.916012in}{2.093584in}}{\pgfqpoint{1.921836in}{2.087761in}}%
\pgfpathcurveto{\pgfqpoint{1.927660in}{2.081937in}}{\pgfqpoint{1.935560in}{2.078664in}}{\pgfqpoint{1.943796in}{2.078664in}}%
\pgfpathclose%
\pgfusepath{stroke,fill}%
\end{pgfscope}%
\begin{pgfscope}%
\pgfpathrectangle{\pgfqpoint{0.100000in}{0.212622in}}{\pgfqpoint{3.696000in}{3.696000in}}%
\pgfusepath{clip}%
\pgfsetbuttcap%
\pgfsetroundjoin%
\definecolor{currentfill}{rgb}{0.121569,0.466667,0.705882}%
\pgfsetfillcolor{currentfill}%
\pgfsetfillopacity{0.300047}%
\pgfsetlinewidth{1.003750pt}%
\definecolor{currentstroke}{rgb}{0.121569,0.466667,0.705882}%
\pgfsetstrokecolor{currentstroke}%
\pgfsetstrokeopacity{0.300047}%
\pgfsetdash{}{0pt}%
\pgfpathmoveto{\pgfqpoint{1.943813in}{2.078655in}}%
\pgfpathcurveto{\pgfqpoint{1.952049in}{2.078655in}}{\pgfqpoint{1.959949in}{2.081927in}}{\pgfqpoint{1.965773in}{2.087751in}}%
\pgfpathcurveto{\pgfqpoint{1.971597in}{2.093575in}}{\pgfqpoint{1.974869in}{2.101475in}}{\pgfqpoint{1.974869in}{2.109711in}}%
\pgfpathcurveto{\pgfqpoint{1.974869in}{2.117948in}}{\pgfqpoint{1.971597in}{2.125848in}}{\pgfqpoint{1.965773in}{2.131672in}}%
\pgfpathcurveto{\pgfqpoint{1.959949in}{2.137496in}}{\pgfqpoint{1.952049in}{2.140768in}}{\pgfqpoint{1.943813in}{2.140768in}}%
\pgfpathcurveto{\pgfqpoint{1.935576in}{2.140768in}}{\pgfqpoint{1.927676in}{2.137496in}}{\pgfqpoint{1.921852in}{2.131672in}}%
\pgfpathcurveto{\pgfqpoint{1.916028in}{2.125848in}}{\pgfqpoint{1.912756in}{2.117948in}}{\pgfqpoint{1.912756in}{2.109711in}}%
\pgfpathcurveto{\pgfqpoint{1.912756in}{2.101475in}}{\pgfqpoint{1.916028in}{2.093575in}}{\pgfqpoint{1.921852in}{2.087751in}}%
\pgfpathcurveto{\pgfqpoint{1.927676in}{2.081927in}}{\pgfqpoint{1.935576in}{2.078655in}}{\pgfqpoint{1.943813in}{2.078655in}}%
\pgfpathclose%
\pgfusepath{stroke,fill}%
\end{pgfscope}%
\begin{pgfscope}%
\pgfpathrectangle{\pgfqpoint{0.100000in}{0.212622in}}{\pgfqpoint{3.696000in}{3.696000in}}%
\pgfusepath{clip}%
\pgfsetbuttcap%
\pgfsetroundjoin%
\definecolor{currentfill}{rgb}{0.121569,0.466667,0.705882}%
\pgfsetfillcolor{currentfill}%
\pgfsetfillopacity{0.300073}%
\pgfsetlinewidth{1.003750pt}%
\definecolor{currentstroke}{rgb}{0.121569,0.466667,0.705882}%
\pgfsetstrokecolor{currentstroke}%
\pgfsetstrokeopacity{0.300073}%
\pgfsetdash{}{0pt}%
\pgfpathmoveto{\pgfqpoint{1.942299in}{2.079206in}}%
\pgfpathcurveto{\pgfqpoint{1.950536in}{2.079206in}}{\pgfqpoint{1.958436in}{2.082479in}}{\pgfqpoint{1.964260in}{2.088303in}}%
\pgfpathcurveto{\pgfqpoint{1.970083in}{2.094126in}}{\pgfqpoint{1.973356in}{2.102026in}}{\pgfqpoint{1.973356in}{2.110263in}}%
\pgfpathcurveto{\pgfqpoint{1.973356in}{2.118499in}}{\pgfqpoint{1.970083in}{2.126399in}}{\pgfqpoint{1.964260in}{2.132223in}}%
\pgfpathcurveto{\pgfqpoint{1.958436in}{2.138047in}}{\pgfqpoint{1.950536in}{2.141319in}}{\pgfqpoint{1.942299in}{2.141319in}}%
\pgfpathcurveto{\pgfqpoint{1.934063in}{2.141319in}}{\pgfqpoint{1.926163in}{2.138047in}}{\pgfqpoint{1.920339in}{2.132223in}}%
\pgfpathcurveto{\pgfqpoint{1.914515in}{2.126399in}}{\pgfqpoint{1.911243in}{2.118499in}}{\pgfqpoint{1.911243in}{2.110263in}}%
\pgfpathcurveto{\pgfqpoint{1.911243in}{2.102026in}}{\pgfqpoint{1.914515in}{2.094126in}}{\pgfqpoint{1.920339in}{2.088303in}}%
\pgfpathcurveto{\pgfqpoint{1.926163in}{2.082479in}}{\pgfqpoint{1.934063in}{2.079206in}}{\pgfqpoint{1.942299in}{2.079206in}}%
\pgfpathclose%
\pgfusepath{stroke,fill}%
\end{pgfscope}%
\begin{pgfscope}%
\pgfpathrectangle{\pgfqpoint{0.100000in}{0.212622in}}{\pgfqpoint{3.696000in}{3.696000in}}%
\pgfusepath{clip}%
\pgfsetbuttcap%
\pgfsetroundjoin%
\definecolor{currentfill}{rgb}{0.121569,0.466667,0.705882}%
\pgfsetfillcolor{currentfill}%
\pgfsetfillopacity{0.300101}%
\pgfsetlinewidth{1.003750pt}%
\definecolor{currentstroke}{rgb}{0.121569,0.466667,0.705882}%
\pgfsetstrokecolor{currentstroke}%
\pgfsetstrokeopacity{0.300101}%
\pgfsetdash{}{0pt}%
\pgfpathmoveto{\pgfqpoint{1.942102in}{2.079242in}}%
\pgfpathcurveto{\pgfqpoint{1.950338in}{2.079242in}}{\pgfqpoint{1.958238in}{2.082515in}}{\pgfqpoint{1.964062in}{2.088339in}}%
\pgfpathcurveto{\pgfqpoint{1.969886in}{2.094163in}}{\pgfqpoint{1.973158in}{2.102063in}}{\pgfqpoint{1.973158in}{2.110299in}}%
\pgfpathcurveto{\pgfqpoint{1.973158in}{2.118535in}}{\pgfqpoint{1.969886in}{2.126435in}}{\pgfqpoint{1.964062in}{2.132259in}}%
\pgfpathcurveto{\pgfqpoint{1.958238in}{2.138083in}}{\pgfqpoint{1.950338in}{2.141355in}}{\pgfqpoint{1.942102in}{2.141355in}}%
\pgfpathcurveto{\pgfqpoint{1.933866in}{2.141355in}}{\pgfqpoint{1.925966in}{2.138083in}}{\pgfqpoint{1.920142in}{2.132259in}}%
\pgfpathcurveto{\pgfqpoint{1.914318in}{2.126435in}}{\pgfqpoint{1.911045in}{2.118535in}}{\pgfqpoint{1.911045in}{2.110299in}}%
\pgfpathcurveto{\pgfqpoint{1.911045in}{2.102063in}}{\pgfqpoint{1.914318in}{2.094163in}}{\pgfqpoint{1.920142in}{2.088339in}}%
\pgfpathcurveto{\pgfqpoint{1.925966in}{2.082515in}}{\pgfqpoint{1.933866in}{2.079242in}}{\pgfqpoint{1.942102in}{2.079242in}}%
\pgfpathclose%
\pgfusepath{stroke,fill}%
\end{pgfscope}%
\begin{pgfscope}%
\pgfpathrectangle{\pgfqpoint{0.100000in}{0.212622in}}{\pgfqpoint{3.696000in}{3.696000in}}%
\pgfusepath{clip}%
\pgfsetbuttcap%
\pgfsetroundjoin%
\definecolor{currentfill}{rgb}{0.121569,0.466667,0.705882}%
\pgfsetfillcolor{currentfill}%
\pgfsetfillopacity{0.300204}%
\pgfsetlinewidth{1.003750pt}%
\definecolor{currentstroke}{rgb}{0.121569,0.466667,0.705882}%
\pgfsetstrokecolor{currentstroke}%
\pgfsetstrokeopacity{0.300204}%
\pgfsetdash{}{0pt}%
\pgfpathmoveto{\pgfqpoint{1.941767in}{2.079547in}}%
\pgfpathcurveto{\pgfqpoint{1.950003in}{2.079547in}}{\pgfqpoint{1.957903in}{2.082819in}}{\pgfqpoint{1.963727in}{2.088643in}}%
\pgfpathcurveto{\pgfqpoint{1.969551in}{2.094467in}}{\pgfqpoint{1.972824in}{2.102367in}}{\pgfqpoint{1.972824in}{2.110603in}}%
\pgfpathcurveto{\pgfqpoint{1.972824in}{2.118839in}}{\pgfqpoint{1.969551in}{2.126739in}}{\pgfqpoint{1.963727in}{2.132563in}}%
\pgfpathcurveto{\pgfqpoint{1.957903in}{2.138387in}}{\pgfqpoint{1.950003in}{2.141660in}}{\pgfqpoint{1.941767in}{2.141660in}}%
\pgfpathcurveto{\pgfqpoint{1.933531in}{2.141660in}}{\pgfqpoint{1.925631in}{2.138387in}}{\pgfqpoint{1.919807in}{2.132563in}}%
\pgfpathcurveto{\pgfqpoint{1.913983in}{2.126739in}}{\pgfqpoint{1.910711in}{2.118839in}}{\pgfqpoint{1.910711in}{2.110603in}}%
\pgfpathcurveto{\pgfqpoint{1.910711in}{2.102367in}}{\pgfqpoint{1.913983in}{2.094467in}}{\pgfqpoint{1.919807in}{2.088643in}}%
\pgfpathcurveto{\pgfqpoint{1.925631in}{2.082819in}}{\pgfqpoint{1.933531in}{2.079547in}}{\pgfqpoint{1.941767in}{2.079547in}}%
\pgfpathclose%
\pgfusepath{stroke,fill}%
\end{pgfscope}%
\begin{pgfscope}%
\pgfpathrectangle{\pgfqpoint{0.100000in}{0.212622in}}{\pgfqpoint{3.696000in}{3.696000in}}%
\pgfusepath{clip}%
\pgfsetbuttcap%
\pgfsetroundjoin%
\definecolor{currentfill}{rgb}{0.121569,0.466667,0.705882}%
\pgfsetfillcolor{currentfill}%
\pgfsetfillopacity{0.300208}%
\pgfsetlinewidth{1.003750pt}%
\definecolor{currentstroke}{rgb}{0.121569,0.466667,0.705882}%
\pgfsetstrokecolor{currentstroke}%
\pgfsetstrokeopacity{0.300208}%
\pgfsetdash{}{0pt}%
\pgfpathmoveto{\pgfqpoint{1.944005in}{2.078557in}}%
\pgfpathcurveto{\pgfqpoint{1.952242in}{2.078557in}}{\pgfqpoint{1.960142in}{2.081829in}}{\pgfqpoint{1.965966in}{2.087653in}}%
\pgfpathcurveto{\pgfqpoint{1.971790in}{2.093477in}}{\pgfqpoint{1.975062in}{2.101377in}}{\pgfqpoint{1.975062in}{2.109613in}}%
\pgfpathcurveto{\pgfqpoint{1.975062in}{2.117850in}}{\pgfqpoint{1.971790in}{2.125750in}}{\pgfqpoint{1.965966in}{2.131574in}}%
\pgfpathcurveto{\pgfqpoint{1.960142in}{2.137397in}}{\pgfqpoint{1.952242in}{2.140670in}}{\pgfqpoint{1.944005in}{2.140670in}}%
\pgfpathcurveto{\pgfqpoint{1.935769in}{2.140670in}}{\pgfqpoint{1.927869in}{2.137397in}}{\pgfqpoint{1.922045in}{2.131574in}}%
\pgfpathcurveto{\pgfqpoint{1.916221in}{2.125750in}}{\pgfqpoint{1.912949in}{2.117850in}}{\pgfqpoint{1.912949in}{2.109613in}}%
\pgfpathcurveto{\pgfqpoint{1.912949in}{2.101377in}}{\pgfqpoint{1.916221in}{2.093477in}}{\pgfqpoint{1.922045in}{2.087653in}}%
\pgfpathcurveto{\pgfqpoint{1.927869in}{2.081829in}}{\pgfqpoint{1.935769in}{2.078557in}}{\pgfqpoint{1.944005in}{2.078557in}}%
\pgfpathclose%
\pgfusepath{stroke,fill}%
\end{pgfscope}%
\begin{pgfscope}%
\pgfpathrectangle{\pgfqpoint{0.100000in}{0.212622in}}{\pgfqpoint{3.696000in}{3.696000in}}%
\pgfusepath{clip}%
\pgfsetbuttcap%
\pgfsetroundjoin%
\definecolor{currentfill}{rgb}{0.121569,0.466667,0.705882}%
\pgfsetfillcolor{currentfill}%
\pgfsetfillopacity{0.300336}%
\pgfsetlinewidth{1.003750pt}%
\definecolor{currentstroke}{rgb}{0.121569,0.466667,0.705882}%
\pgfsetstrokecolor{currentstroke}%
\pgfsetstrokeopacity{0.300336}%
\pgfsetdash{}{0pt}%
\pgfpathmoveto{\pgfqpoint{1.941210in}{2.079556in}}%
\pgfpathcurveto{\pgfqpoint{1.949446in}{2.079556in}}{\pgfqpoint{1.957346in}{2.082828in}}{\pgfqpoint{1.963170in}{2.088652in}}%
\pgfpathcurveto{\pgfqpoint{1.968994in}{2.094476in}}{\pgfqpoint{1.972267in}{2.102376in}}{\pgfqpoint{1.972267in}{2.110613in}}%
\pgfpathcurveto{\pgfqpoint{1.972267in}{2.118849in}}{\pgfqpoint{1.968994in}{2.126749in}}{\pgfqpoint{1.963170in}{2.132573in}}%
\pgfpathcurveto{\pgfqpoint{1.957346in}{2.138397in}}{\pgfqpoint{1.949446in}{2.141669in}}{\pgfqpoint{1.941210in}{2.141669in}}%
\pgfpathcurveto{\pgfqpoint{1.932974in}{2.141669in}}{\pgfqpoint{1.925074in}{2.138397in}}{\pgfqpoint{1.919250in}{2.132573in}}%
\pgfpathcurveto{\pgfqpoint{1.913426in}{2.126749in}}{\pgfqpoint{1.910154in}{2.118849in}}{\pgfqpoint{1.910154in}{2.110613in}}%
\pgfpathcurveto{\pgfqpoint{1.910154in}{2.102376in}}{\pgfqpoint{1.913426in}{2.094476in}}{\pgfqpoint{1.919250in}{2.088652in}}%
\pgfpathcurveto{\pgfqpoint{1.925074in}{2.082828in}}{\pgfqpoint{1.932974in}{2.079556in}}{\pgfqpoint{1.941210in}{2.079556in}}%
\pgfpathclose%
\pgfusepath{stroke,fill}%
\end{pgfscope}%
\begin{pgfscope}%
\pgfpathrectangle{\pgfqpoint{0.100000in}{0.212622in}}{\pgfqpoint{3.696000in}{3.696000in}}%
\pgfusepath{clip}%
\pgfsetbuttcap%
\pgfsetroundjoin%
\definecolor{currentfill}{rgb}{0.121569,0.466667,0.705882}%
\pgfsetfillcolor{currentfill}%
\pgfsetfillopacity{0.300350}%
\pgfsetlinewidth{1.003750pt}%
\definecolor{currentstroke}{rgb}{0.121569,0.466667,0.705882}%
\pgfsetstrokecolor{currentstroke}%
\pgfsetstrokeopacity{0.300350}%
\pgfsetdash{}{0pt}%
\pgfpathmoveto{\pgfqpoint{1.941159in}{2.079540in}}%
\pgfpathcurveto{\pgfqpoint{1.949395in}{2.079540in}}{\pgfqpoint{1.957295in}{2.082813in}}{\pgfqpoint{1.963119in}{2.088637in}}%
\pgfpathcurveto{\pgfqpoint{1.968943in}{2.094460in}}{\pgfqpoint{1.972215in}{2.102361in}}{\pgfqpoint{1.972215in}{2.110597in}}%
\pgfpathcurveto{\pgfqpoint{1.972215in}{2.118833in}}{\pgfqpoint{1.968943in}{2.126733in}}{\pgfqpoint{1.963119in}{2.132557in}}%
\pgfpathcurveto{\pgfqpoint{1.957295in}{2.138381in}}{\pgfqpoint{1.949395in}{2.141653in}}{\pgfqpoint{1.941159in}{2.141653in}}%
\pgfpathcurveto{\pgfqpoint{1.932923in}{2.141653in}}{\pgfqpoint{1.925023in}{2.138381in}}{\pgfqpoint{1.919199in}{2.132557in}}%
\pgfpathcurveto{\pgfqpoint{1.913375in}{2.126733in}}{\pgfqpoint{1.910102in}{2.118833in}}{\pgfqpoint{1.910102in}{2.110597in}}%
\pgfpathcurveto{\pgfqpoint{1.910102in}{2.102361in}}{\pgfqpoint{1.913375in}{2.094460in}}{\pgfqpoint{1.919199in}{2.088637in}}%
\pgfpathcurveto{\pgfqpoint{1.925023in}{2.082813in}}{\pgfqpoint{1.932923in}{2.079540in}}{\pgfqpoint{1.941159in}{2.079540in}}%
\pgfpathclose%
\pgfusepath{stroke,fill}%
\end{pgfscope}%
\begin{pgfscope}%
\pgfpathrectangle{\pgfqpoint{0.100000in}{0.212622in}}{\pgfqpoint{3.696000in}{3.696000in}}%
\pgfusepath{clip}%
\pgfsetbuttcap%
\pgfsetroundjoin%
\definecolor{currentfill}{rgb}{0.121569,0.466667,0.705882}%
\pgfsetfillcolor{currentfill}%
\pgfsetfillopacity{0.300373}%
\pgfsetlinewidth{1.003750pt}%
\definecolor{currentstroke}{rgb}{0.121569,0.466667,0.705882}%
\pgfsetstrokecolor{currentstroke}%
\pgfsetstrokeopacity{0.300373}%
\pgfsetdash{}{0pt}%
\pgfpathmoveto{\pgfqpoint{1.941061in}{2.079515in}}%
\pgfpathcurveto{\pgfqpoint{1.949297in}{2.079515in}}{\pgfqpoint{1.957197in}{2.082788in}}{\pgfqpoint{1.963021in}{2.088612in}}%
\pgfpathcurveto{\pgfqpoint{1.968845in}{2.094436in}}{\pgfqpoint{1.972117in}{2.102336in}}{\pgfqpoint{1.972117in}{2.110572in}}%
\pgfpathcurveto{\pgfqpoint{1.972117in}{2.118808in}}{\pgfqpoint{1.968845in}{2.126708in}}{\pgfqpoint{1.963021in}{2.132532in}}%
\pgfpathcurveto{\pgfqpoint{1.957197in}{2.138356in}}{\pgfqpoint{1.949297in}{2.141628in}}{\pgfqpoint{1.941061in}{2.141628in}}%
\pgfpathcurveto{\pgfqpoint{1.932825in}{2.141628in}}{\pgfqpoint{1.924924in}{2.138356in}}{\pgfqpoint{1.919101in}{2.132532in}}%
\pgfpathcurveto{\pgfqpoint{1.913277in}{2.126708in}}{\pgfqpoint{1.910004in}{2.118808in}}{\pgfqpoint{1.910004in}{2.110572in}}%
\pgfpathcurveto{\pgfqpoint{1.910004in}{2.102336in}}{\pgfqpoint{1.913277in}{2.094436in}}{\pgfqpoint{1.919101in}{2.088612in}}%
\pgfpathcurveto{\pgfqpoint{1.924924in}{2.082788in}}{\pgfqpoint{1.932825in}{2.079515in}}{\pgfqpoint{1.941061in}{2.079515in}}%
\pgfpathclose%
\pgfusepath{stroke,fill}%
\end{pgfscope}%
\begin{pgfscope}%
\pgfpathrectangle{\pgfqpoint{0.100000in}{0.212622in}}{\pgfqpoint{3.696000in}{3.696000in}}%
\pgfusepath{clip}%
\pgfsetbuttcap%
\pgfsetroundjoin%
\definecolor{currentfill}{rgb}{0.121569,0.466667,0.705882}%
\pgfsetfillcolor{currentfill}%
\pgfsetfillopacity{0.300416}%
\pgfsetlinewidth{1.003750pt}%
\definecolor{currentstroke}{rgb}{0.121569,0.466667,0.705882}%
\pgfsetstrokecolor{currentstroke}%
\pgfsetstrokeopacity{0.300416}%
\pgfsetdash{}{0pt}%
\pgfpathmoveto{\pgfqpoint{1.940883in}{2.079462in}}%
\pgfpathcurveto{\pgfqpoint{1.949120in}{2.079462in}}{\pgfqpoint{1.957020in}{2.082735in}}{\pgfqpoint{1.962844in}{2.088559in}}%
\pgfpathcurveto{\pgfqpoint{1.968667in}{2.094383in}}{\pgfqpoint{1.971940in}{2.102283in}}{\pgfqpoint{1.971940in}{2.110519in}}%
\pgfpathcurveto{\pgfqpoint{1.971940in}{2.118755in}}{\pgfqpoint{1.968667in}{2.126655in}}{\pgfqpoint{1.962844in}{2.132479in}}%
\pgfpathcurveto{\pgfqpoint{1.957020in}{2.138303in}}{\pgfqpoint{1.949120in}{2.141575in}}{\pgfqpoint{1.940883in}{2.141575in}}%
\pgfpathcurveto{\pgfqpoint{1.932647in}{2.141575in}}{\pgfqpoint{1.924747in}{2.138303in}}{\pgfqpoint{1.918923in}{2.132479in}}%
\pgfpathcurveto{\pgfqpoint{1.913099in}{2.126655in}}{\pgfqpoint{1.909827in}{2.118755in}}{\pgfqpoint{1.909827in}{2.110519in}}%
\pgfpathcurveto{\pgfqpoint{1.909827in}{2.102283in}}{\pgfqpoint{1.913099in}{2.094383in}}{\pgfqpoint{1.918923in}{2.088559in}}%
\pgfpathcurveto{\pgfqpoint{1.924747in}{2.082735in}}{\pgfqpoint{1.932647in}{2.079462in}}{\pgfqpoint{1.940883in}{2.079462in}}%
\pgfpathclose%
\pgfusepath{stroke,fill}%
\end{pgfscope}%
\begin{pgfscope}%
\pgfpathrectangle{\pgfqpoint{0.100000in}{0.212622in}}{\pgfqpoint{3.696000in}{3.696000in}}%
\pgfusepath{clip}%
\pgfsetbuttcap%
\pgfsetroundjoin%
\definecolor{currentfill}{rgb}{0.121569,0.466667,0.705882}%
\pgfsetfillcolor{currentfill}%
\pgfsetfillopacity{0.300493}%
\pgfsetlinewidth{1.003750pt}%
\definecolor{currentstroke}{rgb}{0.121569,0.466667,0.705882}%
\pgfsetstrokecolor{currentstroke}%
\pgfsetstrokeopacity{0.300493}%
\pgfsetdash{}{0pt}%
\pgfpathmoveto{\pgfqpoint{1.940560in}{2.079369in}}%
\pgfpathcurveto{\pgfqpoint{1.948796in}{2.079369in}}{\pgfqpoint{1.956696in}{2.082641in}}{\pgfqpoint{1.962520in}{2.088465in}}%
\pgfpathcurveto{\pgfqpoint{1.968344in}{2.094289in}}{\pgfqpoint{1.971616in}{2.102189in}}{\pgfqpoint{1.971616in}{2.110426in}}%
\pgfpathcurveto{\pgfqpoint{1.971616in}{2.118662in}}{\pgfqpoint{1.968344in}{2.126562in}}{\pgfqpoint{1.962520in}{2.132386in}}%
\pgfpathcurveto{\pgfqpoint{1.956696in}{2.138210in}}{\pgfqpoint{1.948796in}{2.141482in}}{\pgfqpoint{1.940560in}{2.141482in}}%
\pgfpathcurveto{\pgfqpoint{1.932324in}{2.141482in}}{\pgfqpoint{1.924424in}{2.138210in}}{\pgfqpoint{1.918600in}{2.132386in}}%
\pgfpathcurveto{\pgfqpoint{1.912776in}{2.126562in}}{\pgfqpoint{1.909503in}{2.118662in}}{\pgfqpoint{1.909503in}{2.110426in}}%
\pgfpathcurveto{\pgfqpoint{1.909503in}{2.102189in}}{\pgfqpoint{1.912776in}{2.094289in}}{\pgfqpoint{1.918600in}{2.088465in}}%
\pgfpathcurveto{\pgfqpoint{1.924424in}{2.082641in}}{\pgfqpoint{1.932324in}{2.079369in}}{\pgfqpoint{1.940560in}{2.079369in}}%
\pgfpathclose%
\pgfusepath{stroke,fill}%
\end{pgfscope}%
\begin{pgfscope}%
\pgfpathrectangle{\pgfqpoint{0.100000in}{0.212622in}}{\pgfqpoint{3.696000in}{3.696000in}}%
\pgfusepath{clip}%
\pgfsetbuttcap%
\pgfsetroundjoin%
\definecolor{currentfill}{rgb}{0.121569,0.466667,0.705882}%
\pgfsetfillcolor{currentfill}%
\pgfsetfillopacity{0.300537}%
\pgfsetlinewidth{1.003750pt}%
\definecolor{currentstroke}{rgb}{0.121569,0.466667,0.705882}%
\pgfsetstrokecolor{currentstroke}%
\pgfsetstrokeopacity{0.300537}%
\pgfsetdash{}{0pt}%
\pgfpathmoveto{\pgfqpoint{1.944199in}{2.078003in}}%
\pgfpathcurveto{\pgfqpoint{1.952435in}{2.078003in}}{\pgfqpoint{1.960335in}{2.081275in}}{\pgfqpoint{1.966159in}{2.087099in}}%
\pgfpathcurveto{\pgfqpoint{1.971983in}{2.092923in}}{\pgfqpoint{1.975256in}{2.100823in}}{\pgfqpoint{1.975256in}{2.109059in}}%
\pgfpathcurveto{\pgfqpoint{1.975256in}{2.117295in}}{\pgfqpoint{1.971983in}{2.125195in}}{\pgfqpoint{1.966159in}{2.131019in}}%
\pgfpathcurveto{\pgfqpoint{1.960335in}{2.136843in}}{\pgfqpoint{1.952435in}{2.140116in}}{\pgfqpoint{1.944199in}{2.140116in}}%
\pgfpathcurveto{\pgfqpoint{1.935963in}{2.140116in}}{\pgfqpoint{1.928063in}{2.136843in}}{\pgfqpoint{1.922239in}{2.131019in}}%
\pgfpathcurveto{\pgfqpoint{1.916415in}{2.125195in}}{\pgfqpoint{1.913143in}{2.117295in}}{\pgfqpoint{1.913143in}{2.109059in}}%
\pgfpathcurveto{\pgfqpoint{1.913143in}{2.100823in}}{\pgfqpoint{1.916415in}{2.092923in}}{\pgfqpoint{1.922239in}{2.087099in}}%
\pgfpathcurveto{\pgfqpoint{1.928063in}{2.081275in}}{\pgfqpoint{1.935963in}{2.078003in}}{\pgfqpoint{1.944199in}{2.078003in}}%
\pgfpathclose%
\pgfusepath{stroke,fill}%
\end{pgfscope}%
\begin{pgfscope}%
\pgfpathrectangle{\pgfqpoint{0.100000in}{0.212622in}}{\pgfqpoint{3.696000in}{3.696000in}}%
\pgfusepath{clip}%
\pgfsetbuttcap%
\pgfsetroundjoin%
\definecolor{currentfill}{rgb}{0.121569,0.466667,0.705882}%
\pgfsetfillcolor{currentfill}%
\pgfsetfillopacity{0.300635}%
\pgfsetlinewidth{1.003750pt}%
\definecolor{currentstroke}{rgb}{0.121569,0.466667,0.705882}%
\pgfsetstrokecolor{currentstroke}%
\pgfsetstrokeopacity{0.300635}%
\pgfsetdash{}{0pt}%
\pgfpathmoveto{\pgfqpoint{1.939968in}{2.079214in}}%
\pgfpathcurveto{\pgfqpoint{1.948204in}{2.079214in}}{\pgfqpoint{1.956104in}{2.082487in}}{\pgfqpoint{1.961928in}{2.088311in}}%
\pgfpathcurveto{\pgfqpoint{1.967752in}{2.094134in}}{\pgfqpoint{1.971024in}{2.102035in}}{\pgfqpoint{1.971024in}{2.110271in}}%
\pgfpathcurveto{\pgfqpoint{1.971024in}{2.118507in}}{\pgfqpoint{1.967752in}{2.126407in}}{\pgfqpoint{1.961928in}{2.132231in}}%
\pgfpathcurveto{\pgfqpoint{1.956104in}{2.138055in}}{\pgfqpoint{1.948204in}{2.141327in}}{\pgfqpoint{1.939968in}{2.141327in}}%
\pgfpathcurveto{\pgfqpoint{1.931732in}{2.141327in}}{\pgfqpoint{1.923831in}{2.138055in}}{\pgfqpoint{1.918008in}{2.132231in}}%
\pgfpathcurveto{\pgfqpoint{1.912184in}{2.126407in}}{\pgfqpoint{1.908911in}{2.118507in}}{\pgfqpoint{1.908911in}{2.110271in}}%
\pgfpathcurveto{\pgfqpoint{1.908911in}{2.102035in}}{\pgfqpoint{1.912184in}{2.094134in}}{\pgfqpoint{1.918008in}{2.088311in}}%
\pgfpathcurveto{\pgfqpoint{1.923831in}{2.082487in}}{\pgfqpoint{1.931732in}{2.079214in}}{\pgfqpoint{1.939968in}{2.079214in}}%
\pgfpathclose%
\pgfusepath{stroke,fill}%
\end{pgfscope}%
\begin{pgfscope}%
\pgfpathrectangle{\pgfqpoint{0.100000in}{0.212622in}}{\pgfqpoint{3.696000in}{3.696000in}}%
\pgfusepath{clip}%
\pgfsetbuttcap%
\pgfsetroundjoin%
\definecolor{currentfill}{rgb}{0.121569,0.466667,0.705882}%
\pgfsetfillcolor{currentfill}%
\pgfsetfillopacity{0.300770}%
\pgfsetlinewidth{1.003750pt}%
\definecolor{currentstroke}{rgb}{0.121569,0.466667,0.705882}%
\pgfsetstrokecolor{currentstroke}%
\pgfsetstrokeopacity{0.300770}%
\pgfsetdash{}{0pt}%
\pgfpathmoveto{\pgfqpoint{1.944244in}{2.078045in}}%
\pgfpathcurveto{\pgfqpoint{1.952480in}{2.078045in}}{\pgfqpoint{1.960380in}{2.081317in}}{\pgfqpoint{1.966204in}{2.087141in}}%
\pgfpathcurveto{\pgfqpoint{1.972028in}{2.092965in}}{\pgfqpoint{1.975301in}{2.100865in}}{\pgfqpoint{1.975301in}{2.109101in}}%
\pgfpathcurveto{\pgfqpoint{1.975301in}{2.117338in}}{\pgfqpoint{1.972028in}{2.125238in}}{\pgfqpoint{1.966204in}{2.131062in}}%
\pgfpathcurveto{\pgfqpoint{1.960380in}{2.136886in}}{\pgfqpoint{1.952480in}{2.140158in}}{\pgfqpoint{1.944244in}{2.140158in}}%
\pgfpathcurveto{\pgfqpoint{1.936008in}{2.140158in}}{\pgfqpoint{1.928108in}{2.136886in}}{\pgfqpoint{1.922284in}{2.131062in}}%
\pgfpathcurveto{\pgfqpoint{1.916460in}{2.125238in}}{\pgfqpoint{1.913188in}{2.117338in}}{\pgfqpoint{1.913188in}{2.109101in}}%
\pgfpathcurveto{\pgfqpoint{1.913188in}{2.100865in}}{\pgfqpoint{1.916460in}{2.092965in}}{\pgfqpoint{1.922284in}{2.087141in}}%
\pgfpathcurveto{\pgfqpoint{1.928108in}{2.081317in}}{\pgfqpoint{1.936008in}{2.078045in}}{\pgfqpoint{1.944244in}{2.078045in}}%
\pgfpathclose%
\pgfusepath{stroke,fill}%
\end{pgfscope}%
\begin{pgfscope}%
\pgfpathrectangle{\pgfqpoint{0.100000in}{0.212622in}}{\pgfqpoint{3.696000in}{3.696000in}}%
\pgfusepath{clip}%
\pgfsetbuttcap%
\pgfsetroundjoin%
\definecolor{currentfill}{rgb}{0.121569,0.466667,0.705882}%
\pgfsetfillcolor{currentfill}%
\pgfsetfillopacity{0.300898}%
\pgfsetlinewidth{1.003750pt}%
\definecolor{currentstroke}{rgb}{0.121569,0.466667,0.705882}%
\pgfsetstrokecolor{currentstroke}%
\pgfsetstrokeopacity{0.300898}%
\pgfsetdash{}{0pt}%
\pgfpathmoveto{\pgfqpoint{1.944242in}{2.078054in}}%
\pgfpathcurveto{\pgfqpoint{1.952479in}{2.078054in}}{\pgfqpoint{1.960379in}{2.081326in}}{\pgfqpoint{1.966203in}{2.087150in}}%
\pgfpathcurveto{\pgfqpoint{1.972027in}{2.092974in}}{\pgfqpoint{1.975299in}{2.100874in}}{\pgfqpoint{1.975299in}{2.109110in}}%
\pgfpathcurveto{\pgfqpoint{1.975299in}{2.117347in}}{\pgfqpoint{1.972027in}{2.125247in}}{\pgfqpoint{1.966203in}{2.131071in}}%
\pgfpathcurveto{\pgfqpoint{1.960379in}{2.136894in}}{\pgfqpoint{1.952479in}{2.140167in}}{\pgfqpoint{1.944242in}{2.140167in}}%
\pgfpathcurveto{\pgfqpoint{1.936006in}{2.140167in}}{\pgfqpoint{1.928106in}{2.136894in}}{\pgfqpoint{1.922282in}{2.131071in}}%
\pgfpathcurveto{\pgfqpoint{1.916458in}{2.125247in}}{\pgfqpoint{1.913186in}{2.117347in}}{\pgfqpoint{1.913186in}{2.109110in}}%
\pgfpathcurveto{\pgfqpoint{1.913186in}{2.100874in}}{\pgfqpoint{1.916458in}{2.092974in}}{\pgfqpoint{1.922282in}{2.087150in}}%
\pgfpathcurveto{\pgfqpoint{1.928106in}{2.081326in}}{\pgfqpoint{1.936006in}{2.078054in}}{\pgfqpoint{1.944242in}{2.078054in}}%
\pgfpathclose%
\pgfusepath{stroke,fill}%
\end{pgfscope}%
\begin{pgfscope}%
\pgfpathrectangle{\pgfqpoint{0.100000in}{0.212622in}}{\pgfqpoint{3.696000in}{3.696000in}}%
\pgfusepath{clip}%
\pgfsetbuttcap%
\pgfsetroundjoin%
\definecolor{currentfill}{rgb}{0.121569,0.466667,0.705882}%
\pgfsetfillcolor{currentfill}%
\pgfsetfillopacity{0.300913}%
\pgfsetlinewidth{1.003750pt}%
\definecolor{currentstroke}{rgb}{0.121569,0.466667,0.705882}%
\pgfsetstrokecolor{currentstroke}%
\pgfsetstrokeopacity{0.300913}%
\pgfsetdash{}{0pt}%
\pgfpathmoveto{\pgfqpoint{1.938952in}{2.078945in}}%
\pgfpathcurveto{\pgfqpoint{1.947188in}{2.078945in}}{\pgfqpoint{1.955088in}{2.082217in}}{\pgfqpoint{1.960912in}{2.088041in}}%
\pgfpathcurveto{\pgfqpoint{1.966736in}{2.093865in}}{\pgfqpoint{1.970008in}{2.101765in}}{\pgfqpoint{1.970008in}{2.110001in}}%
\pgfpathcurveto{\pgfqpoint{1.970008in}{2.118238in}}{\pgfqpoint{1.966736in}{2.126138in}}{\pgfqpoint{1.960912in}{2.131962in}}%
\pgfpathcurveto{\pgfqpoint{1.955088in}{2.137786in}}{\pgfqpoint{1.947188in}{2.141058in}}{\pgfqpoint{1.938952in}{2.141058in}}%
\pgfpathcurveto{\pgfqpoint{1.930715in}{2.141058in}}{\pgfqpoint{1.922815in}{2.137786in}}{\pgfqpoint{1.916991in}{2.131962in}}%
\pgfpathcurveto{\pgfqpoint{1.911167in}{2.126138in}}{\pgfqpoint{1.907895in}{2.118238in}}{\pgfqpoint{1.907895in}{2.110001in}}%
\pgfpathcurveto{\pgfqpoint{1.907895in}{2.101765in}}{\pgfqpoint{1.911167in}{2.093865in}}{\pgfqpoint{1.916991in}{2.088041in}}%
\pgfpathcurveto{\pgfqpoint{1.922815in}{2.082217in}}{\pgfqpoint{1.930715in}{2.078945in}}{\pgfqpoint{1.938952in}{2.078945in}}%
\pgfpathclose%
\pgfusepath{stroke,fill}%
\end{pgfscope}%
\begin{pgfscope}%
\pgfpathrectangle{\pgfqpoint{0.100000in}{0.212622in}}{\pgfqpoint{3.696000in}{3.696000in}}%
\pgfusepath{clip}%
\pgfsetbuttcap%
\pgfsetroundjoin%
\definecolor{currentfill}{rgb}{0.121569,0.466667,0.705882}%
\pgfsetfillcolor{currentfill}%
\pgfsetfillopacity{0.301203}%
\pgfsetlinewidth{1.003750pt}%
\definecolor{currentstroke}{rgb}{0.121569,0.466667,0.705882}%
\pgfsetstrokecolor{currentstroke}%
\pgfsetstrokeopacity{0.301203}%
\pgfsetdash{}{0pt}%
\pgfpathmoveto{\pgfqpoint{1.944203in}{2.077733in}}%
\pgfpathcurveto{\pgfqpoint{1.952439in}{2.077733in}}{\pgfqpoint{1.960339in}{2.081005in}}{\pgfqpoint{1.966163in}{2.086829in}}%
\pgfpathcurveto{\pgfqpoint{1.971987in}{2.092653in}}{\pgfqpoint{1.975260in}{2.100553in}}{\pgfqpoint{1.975260in}{2.108790in}}%
\pgfpathcurveto{\pgfqpoint{1.975260in}{2.117026in}}{\pgfqpoint{1.971987in}{2.124926in}}{\pgfqpoint{1.966163in}{2.130750in}}%
\pgfpathcurveto{\pgfqpoint{1.960339in}{2.136574in}}{\pgfqpoint{1.952439in}{2.139846in}}{\pgfqpoint{1.944203in}{2.139846in}}%
\pgfpathcurveto{\pgfqpoint{1.935967in}{2.139846in}}{\pgfqpoint{1.928067in}{2.136574in}}{\pgfqpoint{1.922243in}{2.130750in}}%
\pgfpathcurveto{\pgfqpoint{1.916419in}{2.124926in}}{\pgfqpoint{1.913147in}{2.117026in}}{\pgfqpoint{1.913147in}{2.108790in}}%
\pgfpathcurveto{\pgfqpoint{1.913147in}{2.100553in}}{\pgfqpoint{1.916419in}{2.092653in}}{\pgfqpoint{1.922243in}{2.086829in}}%
\pgfpathcurveto{\pgfqpoint{1.928067in}{2.081005in}}{\pgfqpoint{1.935967in}{2.077733in}}{\pgfqpoint{1.944203in}{2.077733in}}%
\pgfpathclose%
\pgfusepath{stroke,fill}%
\end{pgfscope}%
\begin{pgfscope}%
\pgfpathrectangle{\pgfqpoint{0.100000in}{0.212622in}}{\pgfqpoint{3.696000in}{3.696000in}}%
\pgfusepath{clip}%
\pgfsetbuttcap%
\pgfsetroundjoin%
\definecolor{currentfill}{rgb}{0.121569,0.466667,0.705882}%
\pgfsetfillcolor{currentfill}%
\pgfsetfillopacity{0.301383}%
\pgfsetlinewidth{1.003750pt}%
\definecolor{currentstroke}{rgb}{0.121569,0.466667,0.705882}%
\pgfsetstrokecolor{currentstroke}%
\pgfsetstrokeopacity{0.301383}%
\pgfsetdash{}{0pt}%
\pgfpathmoveto{\pgfqpoint{1.944173in}{2.077642in}}%
\pgfpathcurveto{\pgfqpoint{1.952409in}{2.077642in}}{\pgfqpoint{1.960309in}{2.080914in}}{\pgfqpoint{1.966133in}{2.086738in}}%
\pgfpathcurveto{\pgfqpoint{1.971957in}{2.092562in}}{\pgfqpoint{1.975229in}{2.100462in}}{\pgfqpoint{1.975229in}{2.108698in}}%
\pgfpathcurveto{\pgfqpoint{1.975229in}{2.116934in}}{\pgfqpoint{1.971957in}{2.124834in}}{\pgfqpoint{1.966133in}{2.130658in}}%
\pgfpathcurveto{\pgfqpoint{1.960309in}{2.136482in}}{\pgfqpoint{1.952409in}{2.139755in}}{\pgfqpoint{1.944173in}{2.139755in}}%
\pgfpathcurveto{\pgfqpoint{1.935936in}{2.139755in}}{\pgfqpoint{1.928036in}{2.136482in}}{\pgfqpoint{1.922212in}{2.130658in}}%
\pgfpathcurveto{\pgfqpoint{1.916389in}{2.124834in}}{\pgfqpoint{1.913116in}{2.116934in}}{\pgfqpoint{1.913116in}{2.108698in}}%
\pgfpathcurveto{\pgfqpoint{1.913116in}{2.100462in}}{\pgfqpoint{1.916389in}{2.092562in}}{\pgfqpoint{1.922212in}{2.086738in}}%
\pgfpathcurveto{\pgfqpoint{1.928036in}{2.080914in}}{\pgfqpoint{1.935936in}{2.077642in}}{\pgfqpoint{1.944173in}{2.077642in}}%
\pgfpathclose%
\pgfusepath{stroke,fill}%
\end{pgfscope}%
\begin{pgfscope}%
\pgfpathrectangle{\pgfqpoint{0.100000in}{0.212622in}}{\pgfqpoint{3.696000in}{3.696000in}}%
\pgfusepath{clip}%
\pgfsetbuttcap%
\pgfsetroundjoin%
\definecolor{currentfill}{rgb}{0.121569,0.466667,0.705882}%
\pgfsetfillcolor{currentfill}%
\pgfsetfillopacity{0.301404}%
\pgfsetlinewidth{1.003750pt}%
\definecolor{currentstroke}{rgb}{0.121569,0.466667,0.705882}%
\pgfsetstrokecolor{currentstroke}%
\pgfsetstrokeopacity{0.301404}%
\pgfsetdash{}{0pt}%
\pgfpathmoveto{\pgfqpoint{1.937042in}{2.078489in}}%
\pgfpathcurveto{\pgfqpoint{1.945278in}{2.078489in}}{\pgfqpoint{1.953178in}{2.081761in}}{\pgfqpoint{1.959002in}{2.087585in}}%
\pgfpathcurveto{\pgfqpoint{1.964826in}{2.093409in}}{\pgfqpoint{1.968098in}{2.101309in}}{\pgfqpoint{1.968098in}{2.109546in}}%
\pgfpathcurveto{\pgfqpoint{1.968098in}{2.117782in}}{\pgfqpoint{1.964826in}{2.125682in}}{\pgfqpoint{1.959002in}{2.131506in}}%
\pgfpathcurveto{\pgfqpoint{1.953178in}{2.137330in}}{\pgfqpoint{1.945278in}{2.140602in}}{\pgfqpoint{1.937042in}{2.140602in}}%
\pgfpathcurveto{\pgfqpoint{1.928806in}{2.140602in}}{\pgfqpoint{1.920906in}{2.137330in}}{\pgfqpoint{1.915082in}{2.131506in}}%
\pgfpathcurveto{\pgfqpoint{1.909258in}{2.125682in}}{\pgfqpoint{1.905985in}{2.117782in}}{\pgfqpoint{1.905985in}{2.109546in}}%
\pgfpathcurveto{\pgfqpoint{1.905985in}{2.101309in}}{\pgfqpoint{1.909258in}{2.093409in}}{\pgfqpoint{1.915082in}{2.087585in}}%
\pgfpathcurveto{\pgfqpoint{1.920906in}{2.081761in}}{\pgfqpoint{1.928806in}{2.078489in}}{\pgfqpoint{1.937042in}{2.078489in}}%
\pgfpathclose%
\pgfusepath{stroke,fill}%
\end{pgfscope}%
\begin{pgfscope}%
\pgfpathrectangle{\pgfqpoint{0.100000in}{0.212622in}}{\pgfqpoint{3.696000in}{3.696000in}}%
\pgfusepath{clip}%
\pgfsetbuttcap%
\pgfsetroundjoin%
\definecolor{currentfill}{rgb}{0.121569,0.466667,0.705882}%
\pgfsetfillcolor{currentfill}%
\pgfsetfillopacity{0.301685}%
\pgfsetlinewidth{1.003750pt}%
\definecolor{currentstroke}{rgb}{0.121569,0.466667,0.705882}%
\pgfsetstrokecolor{currentstroke}%
\pgfsetstrokeopacity{0.301685}%
\pgfsetdash{}{0pt}%
\pgfpathmoveto{\pgfqpoint{1.943991in}{2.077399in}}%
\pgfpathcurveto{\pgfqpoint{1.952227in}{2.077399in}}{\pgfqpoint{1.960127in}{2.080671in}}{\pgfqpoint{1.965951in}{2.086495in}}%
\pgfpathcurveto{\pgfqpoint{1.971775in}{2.092319in}}{\pgfqpoint{1.975047in}{2.100219in}}{\pgfqpoint{1.975047in}{2.108456in}}%
\pgfpathcurveto{\pgfqpoint{1.975047in}{2.116692in}}{\pgfqpoint{1.971775in}{2.124592in}}{\pgfqpoint{1.965951in}{2.130416in}}%
\pgfpathcurveto{\pgfqpoint{1.960127in}{2.136240in}}{\pgfqpoint{1.952227in}{2.139512in}}{\pgfqpoint{1.943991in}{2.139512in}}%
\pgfpathcurveto{\pgfqpoint{1.935755in}{2.139512in}}{\pgfqpoint{1.927855in}{2.136240in}}{\pgfqpoint{1.922031in}{2.130416in}}%
\pgfpathcurveto{\pgfqpoint{1.916207in}{2.124592in}}{\pgfqpoint{1.912934in}{2.116692in}}{\pgfqpoint{1.912934in}{2.108456in}}%
\pgfpathcurveto{\pgfqpoint{1.912934in}{2.100219in}}{\pgfqpoint{1.916207in}{2.092319in}}{\pgfqpoint{1.922031in}{2.086495in}}%
\pgfpathcurveto{\pgfqpoint{1.927855in}{2.080671in}}{\pgfqpoint{1.935755in}{2.077399in}}{\pgfqpoint{1.943991in}{2.077399in}}%
\pgfpathclose%
\pgfusepath{stroke,fill}%
\end{pgfscope}%
\begin{pgfscope}%
\pgfpathrectangle{\pgfqpoint{0.100000in}{0.212622in}}{\pgfqpoint{3.696000in}{3.696000in}}%
\pgfusepath{clip}%
\pgfsetbuttcap%
\pgfsetroundjoin%
\definecolor{currentfill}{rgb}{0.121569,0.466667,0.705882}%
\pgfsetfillcolor{currentfill}%
\pgfsetfillopacity{0.301776}%
\pgfsetlinewidth{1.003750pt}%
\definecolor{currentstroke}{rgb}{0.121569,0.466667,0.705882}%
\pgfsetstrokecolor{currentstroke}%
\pgfsetstrokeopacity{0.301776}%
\pgfsetdash{}{0pt}%
\pgfpathmoveto{\pgfqpoint{1.935550in}{2.077945in}}%
\pgfpathcurveto{\pgfqpoint{1.943787in}{2.077945in}}{\pgfqpoint{1.951687in}{2.081217in}}{\pgfqpoint{1.957511in}{2.087041in}}%
\pgfpathcurveto{\pgfqpoint{1.963335in}{2.092865in}}{\pgfqpoint{1.966607in}{2.100765in}}{\pgfqpoint{1.966607in}{2.109001in}}%
\pgfpathcurveto{\pgfqpoint{1.966607in}{2.117237in}}{\pgfqpoint{1.963335in}{2.125137in}}{\pgfqpoint{1.957511in}{2.130961in}}%
\pgfpathcurveto{\pgfqpoint{1.951687in}{2.136785in}}{\pgfqpoint{1.943787in}{2.140058in}}{\pgfqpoint{1.935550in}{2.140058in}}%
\pgfpathcurveto{\pgfqpoint{1.927314in}{2.140058in}}{\pgfqpoint{1.919414in}{2.136785in}}{\pgfqpoint{1.913590in}{2.130961in}}%
\pgfpathcurveto{\pgfqpoint{1.907766in}{2.125137in}}{\pgfqpoint{1.904494in}{2.117237in}}{\pgfqpoint{1.904494in}{2.109001in}}%
\pgfpathcurveto{\pgfqpoint{1.904494in}{2.100765in}}{\pgfqpoint{1.907766in}{2.092865in}}{\pgfqpoint{1.913590in}{2.087041in}}%
\pgfpathcurveto{\pgfqpoint{1.919414in}{2.081217in}}{\pgfqpoint{1.927314in}{2.077945in}}{\pgfqpoint{1.935550in}{2.077945in}}%
\pgfpathclose%
\pgfusepath{stroke,fill}%
\end{pgfscope}%
\begin{pgfscope}%
\pgfpathrectangle{\pgfqpoint{0.100000in}{0.212622in}}{\pgfqpoint{3.696000in}{3.696000in}}%
\pgfusepath{clip}%
\pgfsetbuttcap%
\pgfsetroundjoin%
\definecolor{currentfill}{rgb}{0.121569,0.466667,0.705882}%
\pgfsetfillcolor{currentfill}%
\pgfsetfillopacity{0.302025}%
\pgfsetlinewidth{1.003750pt}%
\definecolor{currentstroke}{rgb}{0.121569,0.466667,0.705882}%
\pgfsetstrokecolor{currentstroke}%
\pgfsetstrokeopacity{0.302025}%
\pgfsetdash{}{0pt}%
\pgfpathmoveto{\pgfqpoint{1.934567in}{2.077497in}}%
\pgfpathcurveto{\pgfqpoint{1.942803in}{2.077497in}}{\pgfqpoint{1.950703in}{2.080770in}}{\pgfqpoint{1.956527in}{2.086594in}}%
\pgfpathcurveto{\pgfqpoint{1.962351in}{2.092417in}}{\pgfqpoint{1.965623in}{2.100318in}}{\pgfqpoint{1.965623in}{2.108554in}}%
\pgfpathcurveto{\pgfqpoint{1.965623in}{2.116790in}}{\pgfqpoint{1.962351in}{2.124690in}}{\pgfqpoint{1.956527in}{2.130514in}}%
\pgfpathcurveto{\pgfqpoint{1.950703in}{2.136338in}}{\pgfqpoint{1.942803in}{2.139610in}}{\pgfqpoint{1.934567in}{2.139610in}}%
\pgfpathcurveto{\pgfqpoint{1.926330in}{2.139610in}}{\pgfqpoint{1.918430in}{2.136338in}}{\pgfqpoint{1.912606in}{2.130514in}}%
\pgfpathcurveto{\pgfqpoint{1.906783in}{2.124690in}}{\pgfqpoint{1.903510in}{2.116790in}}{\pgfqpoint{1.903510in}{2.108554in}}%
\pgfpathcurveto{\pgfqpoint{1.903510in}{2.100318in}}{\pgfqpoint{1.906783in}{2.092417in}}{\pgfqpoint{1.912606in}{2.086594in}}%
\pgfpathcurveto{\pgfqpoint{1.918430in}{2.080770in}}{\pgfqpoint{1.926330in}{2.077497in}}{\pgfqpoint{1.934567in}{2.077497in}}%
\pgfpathclose%
\pgfusepath{stroke,fill}%
\end{pgfscope}%
\begin{pgfscope}%
\pgfpathrectangle{\pgfqpoint{0.100000in}{0.212622in}}{\pgfqpoint{3.696000in}{3.696000in}}%
\pgfusepath{clip}%
\pgfsetbuttcap%
\pgfsetroundjoin%
\definecolor{currentfill}{rgb}{0.121569,0.466667,0.705882}%
\pgfsetfillcolor{currentfill}%
\pgfsetfillopacity{0.302620}%
\pgfsetlinewidth{1.003750pt}%
\definecolor{currentstroke}{rgb}{0.121569,0.466667,0.705882}%
\pgfsetstrokecolor{currentstroke}%
\pgfsetstrokeopacity{0.302620}%
\pgfsetdash{}{0pt}%
\pgfpathmoveto{\pgfqpoint{1.932838in}{2.077564in}}%
\pgfpathcurveto{\pgfqpoint{1.941075in}{2.077564in}}{\pgfqpoint{1.948975in}{2.080837in}}{\pgfqpoint{1.954798in}{2.086660in}}%
\pgfpathcurveto{\pgfqpoint{1.960622in}{2.092484in}}{\pgfqpoint{1.963895in}{2.100384in}}{\pgfqpoint{1.963895in}{2.108621in}}%
\pgfpathcurveto{\pgfqpoint{1.963895in}{2.116857in}}{\pgfqpoint{1.960622in}{2.124757in}}{\pgfqpoint{1.954798in}{2.130581in}}%
\pgfpathcurveto{\pgfqpoint{1.948975in}{2.136405in}}{\pgfqpoint{1.941075in}{2.139677in}}{\pgfqpoint{1.932838in}{2.139677in}}%
\pgfpathcurveto{\pgfqpoint{1.924602in}{2.139677in}}{\pgfqpoint{1.916702in}{2.136405in}}{\pgfqpoint{1.910878in}{2.130581in}}%
\pgfpathcurveto{\pgfqpoint{1.905054in}{2.124757in}}{\pgfqpoint{1.901782in}{2.116857in}}{\pgfqpoint{1.901782in}{2.108621in}}%
\pgfpathcurveto{\pgfqpoint{1.901782in}{2.100384in}}{\pgfqpoint{1.905054in}{2.092484in}}{\pgfqpoint{1.910878in}{2.086660in}}%
\pgfpathcurveto{\pgfqpoint{1.916702in}{2.080837in}}{\pgfqpoint{1.924602in}{2.077564in}}{\pgfqpoint{1.932838in}{2.077564in}}%
\pgfpathclose%
\pgfusepath{stroke,fill}%
\end{pgfscope}%
\begin{pgfscope}%
\pgfpathrectangle{\pgfqpoint{0.100000in}{0.212622in}}{\pgfqpoint{3.696000in}{3.696000in}}%
\pgfusepath{clip}%
\pgfsetbuttcap%
\pgfsetroundjoin%
\definecolor{currentfill}{rgb}{0.121569,0.466667,0.705882}%
\pgfsetfillcolor{currentfill}%
\pgfsetfillopacity{0.302837}%
\pgfsetlinewidth{1.003750pt}%
\definecolor{currentstroke}{rgb}{0.121569,0.466667,0.705882}%
\pgfsetstrokecolor{currentstroke}%
\pgfsetstrokeopacity{0.302837}%
\pgfsetdash{}{0pt}%
\pgfpathmoveto{\pgfqpoint{1.943914in}{2.078748in}}%
\pgfpathcurveto{\pgfqpoint{1.952151in}{2.078748in}}{\pgfqpoint{1.960051in}{2.082020in}}{\pgfqpoint{1.965875in}{2.087844in}}%
\pgfpathcurveto{\pgfqpoint{1.971699in}{2.093668in}}{\pgfqpoint{1.974971in}{2.101568in}}{\pgfqpoint{1.974971in}{2.109804in}}%
\pgfpathcurveto{\pgfqpoint{1.974971in}{2.118041in}}{\pgfqpoint{1.971699in}{2.125941in}}{\pgfqpoint{1.965875in}{2.131765in}}%
\pgfpathcurveto{\pgfqpoint{1.960051in}{2.137589in}}{\pgfqpoint{1.952151in}{2.140861in}}{\pgfqpoint{1.943914in}{2.140861in}}%
\pgfpathcurveto{\pgfqpoint{1.935678in}{2.140861in}}{\pgfqpoint{1.927778in}{2.137589in}}{\pgfqpoint{1.921954in}{2.131765in}}%
\pgfpathcurveto{\pgfqpoint{1.916130in}{2.125941in}}{\pgfqpoint{1.912858in}{2.118041in}}{\pgfqpoint{1.912858in}{2.109804in}}%
\pgfpathcurveto{\pgfqpoint{1.912858in}{2.101568in}}{\pgfqpoint{1.916130in}{2.093668in}}{\pgfqpoint{1.921954in}{2.087844in}}%
\pgfpathcurveto{\pgfqpoint{1.927778in}{2.082020in}}{\pgfqpoint{1.935678in}{2.078748in}}{\pgfqpoint{1.943914in}{2.078748in}}%
\pgfpathclose%
\pgfusepath{stroke,fill}%
\end{pgfscope}%
\begin{pgfscope}%
\pgfpathrectangle{\pgfqpoint{0.100000in}{0.212622in}}{\pgfqpoint{3.696000in}{3.696000in}}%
\pgfusepath{clip}%
\pgfsetbuttcap%
\pgfsetroundjoin%
\definecolor{currentfill}{rgb}{0.121569,0.466667,0.705882}%
\pgfsetfillcolor{currentfill}%
\pgfsetfillopacity{0.303603}%
\pgfsetlinewidth{1.003750pt}%
\definecolor{currentstroke}{rgb}{0.121569,0.466667,0.705882}%
\pgfsetstrokecolor{currentstroke}%
\pgfsetstrokeopacity{0.303603}%
\pgfsetdash{}{0pt}%
\pgfpathmoveto{\pgfqpoint{1.929783in}{2.076841in}}%
\pgfpathcurveto{\pgfqpoint{1.938020in}{2.076841in}}{\pgfqpoint{1.945920in}{2.080113in}}{\pgfqpoint{1.951744in}{2.085937in}}%
\pgfpathcurveto{\pgfqpoint{1.957568in}{2.091761in}}{\pgfqpoint{1.960840in}{2.099661in}}{\pgfqpoint{1.960840in}{2.107897in}}%
\pgfpathcurveto{\pgfqpoint{1.960840in}{2.116133in}}{\pgfqpoint{1.957568in}{2.124033in}}{\pgfqpoint{1.951744in}{2.129857in}}%
\pgfpathcurveto{\pgfqpoint{1.945920in}{2.135681in}}{\pgfqpoint{1.938020in}{2.138954in}}{\pgfqpoint{1.929783in}{2.138954in}}%
\pgfpathcurveto{\pgfqpoint{1.921547in}{2.138954in}}{\pgfqpoint{1.913647in}{2.135681in}}{\pgfqpoint{1.907823in}{2.129857in}}%
\pgfpathcurveto{\pgfqpoint{1.901999in}{2.124033in}}{\pgfqpoint{1.898727in}{2.116133in}}{\pgfqpoint{1.898727in}{2.107897in}}%
\pgfpathcurveto{\pgfqpoint{1.898727in}{2.099661in}}{\pgfqpoint{1.901999in}{2.091761in}}{\pgfqpoint{1.907823in}{2.085937in}}%
\pgfpathcurveto{\pgfqpoint{1.913647in}{2.080113in}}{\pgfqpoint{1.921547in}{2.076841in}}{\pgfqpoint{1.929783in}{2.076841in}}%
\pgfpathclose%
\pgfusepath{stroke,fill}%
\end{pgfscope}%
\begin{pgfscope}%
\pgfpathrectangle{\pgfqpoint{0.100000in}{0.212622in}}{\pgfqpoint{3.696000in}{3.696000in}}%
\pgfusepath{clip}%
\pgfsetbuttcap%
\pgfsetroundjoin%
\definecolor{currentfill}{rgb}{0.121569,0.466667,0.705882}%
\pgfsetfillcolor{currentfill}%
\pgfsetfillopacity{0.303879}%
\pgfsetlinewidth{1.003750pt}%
\definecolor{currentstroke}{rgb}{0.121569,0.466667,0.705882}%
\pgfsetstrokecolor{currentstroke}%
\pgfsetstrokeopacity{0.303879}%
\pgfsetdash{}{0pt}%
\pgfpathmoveto{\pgfqpoint{1.944522in}{2.076566in}}%
\pgfpathcurveto{\pgfqpoint{1.952758in}{2.076566in}}{\pgfqpoint{1.960658in}{2.079839in}}{\pgfqpoint{1.966482in}{2.085663in}}%
\pgfpathcurveto{\pgfqpoint{1.972306in}{2.091487in}}{\pgfqpoint{1.975578in}{2.099387in}}{\pgfqpoint{1.975578in}{2.107623in}}%
\pgfpathcurveto{\pgfqpoint{1.975578in}{2.115859in}}{\pgfqpoint{1.972306in}{2.123759in}}{\pgfqpoint{1.966482in}{2.129583in}}%
\pgfpathcurveto{\pgfqpoint{1.960658in}{2.135407in}}{\pgfqpoint{1.952758in}{2.138679in}}{\pgfqpoint{1.944522in}{2.138679in}}%
\pgfpathcurveto{\pgfqpoint{1.936286in}{2.138679in}}{\pgfqpoint{1.928386in}{2.135407in}}{\pgfqpoint{1.922562in}{2.129583in}}%
\pgfpathcurveto{\pgfqpoint{1.916738in}{2.123759in}}{\pgfqpoint{1.913465in}{2.115859in}}{\pgfqpoint{1.913465in}{2.107623in}}%
\pgfpathcurveto{\pgfqpoint{1.913465in}{2.099387in}}{\pgfqpoint{1.916738in}{2.091487in}}{\pgfqpoint{1.922562in}{2.085663in}}%
\pgfpathcurveto{\pgfqpoint{1.928386in}{2.079839in}}{\pgfqpoint{1.936286in}{2.076566in}}{\pgfqpoint{1.944522in}{2.076566in}}%
\pgfpathclose%
\pgfusepath{stroke,fill}%
\end{pgfscope}%
\begin{pgfscope}%
\pgfpathrectangle{\pgfqpoint{0.100000in}{0.212622in}}{\pgfqpoint{3.696000in}{3.696000in}}%
\pgfusepath{clip}%
\pgfsetbuttcap%
\pgfsetroundjoin%
\definecolor{currentfill}{rgb}{0.121569,0.466667,0.705882}%
\pgfsetfillcolor{currentfill}%
\pgfsetfillopacity{0.304937}%
\pgfsetlinewidth{1.003750pt}%
\definecolor{currentstroke}{rgb}{0.121569,0.466667,0.705882}%
\pgfsetstrokecolor{currentstroke}%
\pgfsetstrokeopacity{0.304937}%
\pgfsetdash{}{0pt}%
\pgfpathmoveto{\pgfqpoint{1.945533in}{2.073362in}}%
\pgfpathcurveto{\pgfqpoint{1.953770in}{2.073362in}}{\pgfqpoint{1.961670in}{2.076635in}}{\pgfqpoint{1.967494in}{2.082459in}}%
\pgfpathcurveto{\pgfqpoint{1.973317in}{2.088283in}}{\pgfqpoint{1.976590in}{2.096183in}}{\pgfqpoint{1.976590in}{2.104419in}}%
\pgfpathcurveto{\pgfqpoint{1.976590in}{2.112655in}}{\pgfqpoint{1.973317in}{2.120555in}}{\pgfqpoint{1.967494in}{2.126379in}}%
\pgfpathcurveto{\pgfqpoint{1.961670in}{2.132203in}}{\pgfqpoint{1.953770in}{2.135475in}}{\pgfqpoint{1.945533in}{2.135475in}}%
\pgfpathcurveto{\pgfqpoint{1.937297in}{2.135475in}}{\pgfqpoint{1.929397in}{2.132203in}}{\pgfqpoint{1.923573in}{2.126379in}}%
\pgfpathcurveto{\pgfqpoint{1.917749in}{2.120555in}}{\pgfqpoint{1.914477in}{2.112655in}}{\pgfqpoint{1.914477in}{2.104419in}}%
\pgfpathcurveto{\pgfqpoint{1.914477in}{2.096183in}}{\pgfqpoint{1.917749in}{2.088283in}}{\pgfqpoint{1.923573in}{2.082459in}}%
\pgfpathcurveto{\pgfqpoint{1.929397in}{2.076635in}}{\pgfqpoint{1.937297in}{2.073362in}}{\pgfqpoint{1.945533in}{2.073362in}}%
\pgfpathclose%
\pgfusepath{stroke,fill}%
\end{pgfscope}%
\begin{pgfscope}%
\pgfpathrectangle{\pgfqpoint{0.100000in}{0.212622in}}{\pgfqpoint{3.696000in}{3.696000in}}%
\pgfusepath{clip}%
\pgfsetbuttcap%
\pgfsetroundjoin%
\definecolor{currentfill}{rgb}{0.121569,0.466667,0.705882}%
\pgfsetfillcolor{currentfill}%
\pgfsetfillopacity{0.305399}%
\pgfsetlinewidth{1.003750pt}%
\definecolor{currentstroke}{rgb}{0.121569,0.466667,0.705882}%
\pgfsetstrokecolor{currentstroke}%
\pgfsetstrokeopacity{0.305399}%
\pgfsetdash{}{0pt}%
\pgfpathmoveto{\pgfqpoint{1.924217in}{2.075585in}}%
\pgfpathcurveto{\pgfqpoint{1.932453in}{2.075585in}}{\pgfqpoint{1.940353in}{2.078857in}}{\pgfqpoint{1.946177in}{2.084681in}}%
\pgfpathcurveto{\pgfqpoint{1.952001in}{2.090505in}}{\pgfqpoint{1.955273in}{2.098405in}}{\pgfqpoint{1.955273in}{2.106641in}}%
\pgfpathcurveto{\pgfqpoint{1.955273in}{2.114878in}}{\pgfqpoint{1.952001in}{2.122778in}}{\pgfqpoint{1.946177in}{2.128602in}}%
\pgfpathcurveto{\pgfqpoint{1.940353in}{2.134425in}}{\pgfqpoint{1.932453in}{2.137698in}}{\pgfqpoint{1.924217in}{2.137698in}}%
\pgfpathcurveto{\pgfqpoint{1.915981in}{2.137698in}}{\pgfqpoint{1.908081in}{2.134425in}}{\pgfqpoint{1.902257in}{2.128602in}}%
\pgfpathcurveto{\pgfqpoint{1.896433in}{2.122778in}}{\pgfqpoint{1.893160in}{2.114878in}}{\pgfqpoint{1.893160in}{2.106641in}}%
\pgfpathcurveto{\pgfqpoint{1.893160in}{2.098405in}}{\pgfqpoint{1.896433in}{2.090505in}}{\pgfqpoint{1.902257in}{2.084681in}}%
\pgfpathcurveto{\pgfqpoint{1.908081in}{2.078857in}}{\pgfqpoint{1.915981in}{2.075585in}}{\pgfqpoint{1.924217in}{2.075585in}}%
\pgfpathclose%
\pgfusepath{stroke,fill}%
\end{pgfscope}%
\begin{pgfscope}%
\pgfpathrectangle{\pgfqpoint{0.100000in}{0.212622in}}{\pgfqpoint{3.696000in}{3.696000in}}%
\pgfusepath{clip}%
\pgfsetbuttcap%
\pgfsetroundjoin%
\definecolor{currentfill}{rgb}{0.121569,0.466667,0.705882}%
\pgfsetfillcolor{currentfill}%
\pgfsetfillopacity{0.306740}%
\pgfsetlinewidth{1.003750pt}%
\definecolor{currentstroke}{rgb}{0.121569,0.466667,0.705882}%
\pgfsetstrokecolor{currentstroke}%
\pgfsetstrokeopacity{0.306740}%
\pgfsetdash{}{0pt}%
\pgfpathmoveto{\pgfqpoint{1.918960in}{2.073568in}}%
\pgfpathcurveto{\pgfqpoint{1.927196in}{2.073568in}}{\pgfqpoint{1.935096in}{2.076840in}}{\pgfqpoint{1.940920in}{2.082664in}}%
\pgfpathcurveto{\pgfqpoint{1.946744in}{2.088488in}}{\pgfqpoint{1.950016in}{2.096388in}}{\pgfqpoint{1.950016in}{2.104624in}}%
\pgfpathcurveto{\pgfqpoint{1.950016in}{2.112860in}}{\pgfqpoint{1.946744in}{2.120761in}}{\pgfqpoint{1.940920in}{2.126584in}}%
\pgfpathcurveto{\pgfqpoint{1.935096in}{2.132408in}}{\pgfqpoint{1.927196in}{2.135681in}}{\pgfqpoint{1.918960in}{2.135681in}}%
\pgfpathcurveto{\pgfqpoint{1.910724in}{2.135681in}}{\pgfqpoint{1.902823in}{2.132408in}}{\pgfqpoint{1.897000in}{2.126584in}}%
\pgfpathcurveto{\pgfqpoint{1.891176in}{2.120761in}}{\pgfqpoint{1.887903in}{2.112860in}}{\pgfqpoint{1.887903in}{2.104624in}}%
\pgfpathcurveto{\pgfqpoint{1.887903in}{2.096388in}}{\pgfqpoint{1.891176in}{2.088488in}}{\pgfqpoint{1.897000in}{2.082664in}}%
\pgfpathcurveto{\pgfqpoint{1.902823in}{2.076840in}}{\pgfqpoint{1.910724in}{2.073568in}}{\pgfqpoint{1.918960in}{2.073568in}}%
\pgfpathclose%
\pgfusepath{stroke,fill}%
\end{pgfscope}%
\begin{pgfscope}%
\pgfpathrectangle{\pgfqpoint{0.100000in}{0.212622in}}{\pgfqpoint{3.696000in}{3.696000in}}%
\pgfusepath{clip}%
\pgfsetbuttcap%
\pgfsetroundjoin%
\definecolor{currentfill}{rgb}{0.121569,0.466667,0.705882}%
\pgfsetfillcolor{currentfill}%
\pgfsetfillopacity{0.308011}%
\pgfsetlinewidth{1.003750pt}%
\definecolor{currentstroke}{rgb}{0.121569,0.466667,0.705882}%
\pgfsetstrokecolor{currentstroke}%
\pgfsetstrokeopacity{0.308011}%
\pgfsetdash{}{0pt}%
\pgfpathmoveto{\pgfqpoint{1.915024in}{2.070669in}}%
\pgfpathcurveto{\pgfqpoint{1.923260in}{2.070669in}}{\pgfqpoint{1.931160in}{2.073941in}}{\pgfqpoint{1.936984in}{2.079765in}}%
\pgfpathcurveto{\pgfqpoint{1.942808in}{2.085589in}}{\pgfqpoint{1.946080in}{2.093489in}}{\pgfqpoint{1.946080in}{2.101726in}}%
\pgfpathcurveto{\pgfqpoint{1.946080in}{2.109962in}}{\pgfqpoint{1.942808in}{2.117862in}}{\pgfqpoint{1.936984in}{2.123686in}}%
\pgfpathcurveto{\pgfqpoint{1.931160in}{2.129510in}}{\pgfqpoint{1.923260in}{2.132782in}}{\pgfqpoint{1.915024in}{2.132782in}}%
\pgfpathcurveto{\pgfqpoint{1.906787in}{2.132782in}}{\pgfqpoint{1.898887in}{2.129510in}}{\pgfqpoint{1.893063in}{2.123686in}}%
\pgfpathcurveto{\pgfqpoint{1.887239in}{2.117862in}}{\pgfqpoint{1.883967in}{2.109962in}}{\pgfqpoint{1.883967in}{2.101726in}}%
\pgfpathcurveto{\pgfqpoint{1.883967in}{2.093489in}}{\pgfqpoint{1.887239in}{2.085589in}}{\pgfqpoint{1.893063in}{2.079765in}}%
\pgfpathcurveto{\pgfqpoint{1.898887in}{2.073941in}}{\pgfqpoint{1.906787in}{2.070669in}}{\pgfqpoint{1.915024in}{2.070669in}}%
\pgfpathclose%
\pgfusepath{stroke,fill}%
\end{pgfscope}%
\begin{pgfscope}%
\pgfpathrectangle{\pgfqpoint{0.100000in}{0.212622in}}{\pgfqpoint{3.696000in}{3.696000in}}%
\pgfusepath{clip}%
\pgfsetbuttcap%
\pgfsetroundjoin%
\definecolor{currentfill}{rgb}{0.121569,0.466667,0.705882}%
\pgfsetfillcolor{currentfill}%
\pgfsetfillopacity{0.308061}%
\pgfsetlinewidth{1.003750pt}%
\definecolor{currentstroke}{rgb}{0.121569,0.466667,0.705882}%
\pgfsetstrokecolor{currentstroke}%
\pgfsetstrokeopacity{0.308061}%
\pgfsetdash{}{0pt}%
\pgfpathmoveto{\pgfqpoint{1.946626in}{2.075810in}}%
\pgfpathcurveto{\pgfqpoint{1.954862in}{2.075810in}}{\pgfqpoint{1.962763in}{2.079082in}}{\pgfqpoint{1.968586in}{2.084906in}}%
\pgfpathcurveto{\pgfqpoint{1.974410in}{2.090730in}}{\pgfqpoint{1.977683in}{2.098630in}}{\pgfqpoint{1.977683in}{2.106867in}}%
\pgfpathcurveto{\pgfqpoint{1.977683in}{2.115103in}}{\pgfqpoint{1.974410in}{2.123003in}}{\pgfqpoint{1.968586in}{2.128827in}}%
\pgfpathcurveto{\pgfqpoint{1.962763in}{2.134651in}}{\pgfqpoint{1.954862in}{2.137923in}}{\pgfqpoint{1.946626in}{2.137923in}}%
\pgfpathcurveto{\pgfqpoint{1.938390in}{2.137923in}}{\pgfqpoint{1.930490in}{2.134651in}}{\pgfqpoint{1.924666in}{2.128827in}}%
\pgfpathcurveto{\pgfqpoint{1.918842in}{2.123003in}}{\pgfqpoint{1.915570in}{2.115103in}}{\pgfqpoint{1.915570in}{2.106867in}}%
\pgfpathcurveto{\pgfqpoint{1.915570in}{2.098630in}}{\pgfqpoint{1.918842in}{2.090730in}}{\pgfqpoint{1.924666in}{2.084906in}}%
\pgfpathcurveto{\pgfqpoint{1.930490in}{2.079082in}}{\pgfqpoint{1.938390in}{2.075810in}}{\pgfqpoint{1.946626in}{2.075810in}}%
\pgfpathclose%
\pgfusepath{stroke,fill}%
\end{pgfscope}%
\begin{pgfscope}%
\pgfpathrectangle{\pgfqpoint{0.100000in}{0.212622in}}{\pgfqpoint{3.696000in}{3.696000in}}%
\pgfusepath{clip}%
\pgfsetbuttcap%
\pgfsetroundjoin%
\definecolor{currentfill}{rgb}{0.121569,0.466667,0.705882}%
\pgfsetfillcolor{currentfill}%
\pgfsetfillopacity{0.308644}%
\pgfsetlinewidth{1.003750pt}%
\definecolor{currentstroke}{rgb}{0.121569,0.466667,0.705882}%
\pgfsetstrokecolor{currentstroke}%
\pgfsetstrokeopacity{0.308644}%
\pgfsetdash{}{0pt}%
\pgfpathmoveto{\pgfqpoint{1.912764in}{2.069684in}}%
\pgfpathcurveto{\pgfqpoint{1.921001in}{2.069684in}}{\pgfqpoint{1.928901in}{2.072956in}}{\pgfqpoint{1.934725in}{2.078780in}}%
\pgfpathcurveto{\pgfqpoint{1.940549in}{2.084604in}}{\pgfqpoint{1.943821in}{2.092504in}}{\pgfqpoint{1.943821in}{2.100740in}}%
\pgfpathcurveto{\pgfqpoint{1.943821in}{2.108976in}}{\pgfqpoint{1.940549in}{2.116876in}}{\pgfqpoint{1.934725in}{2.122700in}}%
\pgfpathcurveto{\pgfqpoint{1.928901in}{2.128524in}}{\pgfqpoint{1.921001in}{2.131797in}}{\pgfqpoint{1.912764in}{2.131797in}}%
\pgfpathcurveto{\pgfqpoint{1.904528in}{2.131797in}}{\pgfqpoint{1.896628in}{2.128524in}}{\pgfqpoint{1.890804in}{2.122700in}}%
\pgfpathcurveto{\pgfqpoint{1.884980in}{2.116876in}}{\pgfqpoint{1.881708in}{2.108976in}}{\pgfqpoint{1.881708in}{2.100740in}}%
\pgfpathcurveto{\pgfqpoint{1.881708in}{2.092504in}}{\pgfqpoint{1.884980in}{2.084604in}}{\pgfqpoint{1.890804in}{2.078780in}}%
\pgfpathcurveto{\pgfqpoint{1.896628in}{2.072956in}}{\pgfqpoint{1.904528in}{2.069684in}}{\pgfqpoint{1.912764in}{2.069684in}}%
\pgfpathclose%
\pgfusepath{stroke,fill}%
\end{pgfscope}%
\begin{pgfscope}%
\pgfpathrectangle{\pgfqpoint{0.100000in}{0.212622in}}{\pgfqpoint{3.696000in}{3.696000in}}%
\pgfusepath{clip}%
\pgfsetbuttcap%
\pgfsetroundjoin%
\definecolor{currentfill}{rgb}{0.121569,0.466667,0.705882}%
\pgfsetfillcolor{currentfill}%
\pgfsetfillopacity{0.309902}%
\pgfsetlinewidth{1.003750pt}%
\definecolor{currentstroke}{rgb}{0.121569,0.466667,0.705882}%
\pgfsetstrokecolor{currentstroke}%
\pgfsetstrokeopacity{0.309902}%
\pgfsetdash{}{0pt}%
\pgfpathmoveto{\pgfqpoint{1.908806in}{2.068375in}}%
\pgfpathcurveto{\pgfqpoint{1.917042in}{2.068375in}}{\pgfqpoint{1.924942in}{2.071648in}}{\pgfqpoint{1.930766in}{2.077472in}}%
\pgfpathcurveto{\pgfqpoint{1.936590in}{2.083295in}}{\pgfqpoint{1.939862in}{2.091195in}}{\pgfqpoint{1.939862in}{2.099432in}}%
\pgfpathcurveto{\pgfqpoint{1.939862in}{2.107668in}}{\pgfqpoint{1.936590in}{2.115568in}}{\pgfqpoint{1.930766in}{2.121392in}}%
\pgfpathcurveto{\pgfqpoint{1.924942in}{2.127216in}}{\pgfqpoint{1.917042in}{2.130488in}}{\pgfqpoint{1.908806in}{2.130488in}}%
\pgfpathcurveto{\pgfqpoint{1.900570in}{2.130488in}}{\pgfqpoint{1.892670in}{2.127216in}}{\pgfqpoint{1.886846in}{2.121392in}}%
\pgfpathcurveto{\pgfqpoint{1.881022in}{2.115568in}}{\pgfqpoint{1.877749in}{2.107668in}}{\pgfqpoint{1.877749in}{2.099432in}}%
\pgfpathcurveto{\pgfqpoint{1.877749in}{2.091195in}}{\pgfqpoint{1.881022in}{2.083295in}}{\pgfqpoint{1.886846in}{2.077472in}}%
\pgfpathcurveto{\pgfqpoint{1.892670in}{2.071648in}}{\pgfqpoint{1.900570in}{2.068375in}}{\pgfqpoint{1.908806in}{2.068375in}}%
\pgfpathclose%
\pgfusepath{stroke,fill}%
\end{pgfscope}%
\begin{pgfscope}%
\pgfpathrectangle{\pgfqpoint{0.100000in}{0.212622in}}{\pgfqpoint{3.696000in}{3.696000in}}%
\pgfusepath{clip}%
\pgfsetbuttcap%
\pgfsetroundjoin%
\definecolor{currentfill}{rgb}{0.121569,0.466667,0.705882}%
\pgfsetfillcolor{currentfill}%
\pgfsetfillopacity{0.310742}%
\pgfsetlinewidth{1.003750pt}%
\definecolor{currentstroke}{rgb}{0.121569,0.466667,0.705882}%
\pgfsetstrokecolor{currentstroke}%
\pgfsetstrokeopacity{0.310742}%
\pgfsetdash{}{0pt}%
\pgfpathmoveto{\pgfqpoint{1.948522in}{2.072769in}}%
\pgfpathcurveto{\pgfqpoint{1.956759in}{2.072769in}}{\pgfqpoint{1.964659in}{2.076041in}}{\pgfqpoint{1.970483in}{2.081865in}}%
\pgfpathcurveto{\pgfqpoint{1.976307in}{2.087689in}}{\pgfqpoint{1.979579in}{2.095589in}}{\pgfqpoint{1.979579in}{2.103825in}}%
\pgfpathcurveto{\pgfqpoint{1.979579in}{2.112061in}}{\pgfqpoint{1.976307in}{2.119962in}}{\pgfqpoint{1.970483in}{2.125785in}}%
\pgfpathcurveto{\pgfqpoint{1.964659in}{2.131609in}}{\pgfqpoint{1.956759in}{2.134882in}}{\pgfqpoint{1.948522in}{2.134882in}}%
\pgfpathcurveto{\pgfqpoint{1.940286in}{2.134882in}}{\pgfqpoint{1.932386in}{2.131609in}}{\pgfqpoint{1.926562in}{2.125785in}}%
\pgfpathcurveto{\pgfqpoint{1.920738in}{2.119962in}}{\pgfqpoint{1.917466in}{2.112061in}}{\pgfqpoint{1.917466in}{2.103825in}}%
\pgfpathcurveto{\pgfqpoint{1.917466in}{2.095589in}}{\pgfqpoint{1.920738in}{2.087689in}}{\pgfqpoint{1.926562in}{2.081865in}}%
\pgfpathcurveto{\pgfqpoint{1.932386in}{2.076041in}}{\pgfqpoint{1.940286in}{2.072769in}}{\pgfqpoint{1.948522in}{2.072769in}}%
\pgfpathclose%
\pgfusepath{stroke,fill}%
\end{pgfscope}%
\begin{pgfscope}%
\pgfpathrectangle{\pgfqpoint{0.100000in}{0.212622in}}{\pgfqpoint{3.696000in}{3.696000in}}%
\pgfusepath{clip}%
\pgfsetbuttcap%
\pgfsetroundjoin%
\definecolor{currentfill}{rgb}{0.121569,0.466667,0.705882}%
\pgfsetfillcolor{currentfill}%
\pgfsetfillopacity{0.311804}%
\pgfsetlinewidth{1.003750pt}%
\definecolor{currentstroke}{rgb}{0.121569,0.466667,0.705882}%
\pgfsetstrokecolor{currentstroke}%
\pgfsetstrokeopacity{0.311804}%
\pgfsetdash{}{0pt}%
\pgfpathmoveto{\pgfqpoint{1.901879in}{2.062877in}}%
\pgfpathcurveto{\pgfqpoint{1.910115in}{2.062877in}}{\pgfqpoint{1.918015in}{2.066149in}}{\pgfqpoint{1.923839in}{2.071973in}}%
\pgfpathcurveto{\pgfqpoint{1.929663in}{2.077797in}}{\pgfqpoint{1.932935in}{2.085697in}}{\pgfqpoint{1.932935in}{2.093934in}}%
\pgfpathcurveto{\pgfqpoint{1.932935in}{2.102170in}}{\pgfqpoint{1.929663in}{2.110070in}}{\pgfqpoint{1.923839in}{2.115894in}}%
\pgfpathcurveto{\pgfqpoint{1.918015in}{2.121718in}}{\pgfqpoint{1.910115in}{2.124990in}}{\pgfqpoint{1.901879in}{2.124990in}}%
\pgfpathcurveto{\pgfqpoint{1.893643in}{2.124990in}}{\pgfqpoint{1.885742in}{2.121718in}}{\pgfqpoint{1.879919in}{2.115894in}}%
\pgfpathcurveto{\pgfqpoint{1.874095in}{2.110070in}}{\pgfqpoint{1.870822in}{2.102170in}}{\pgfqpoint{1.870822in}{2.093934in}}%
\pgfpathcurveto{\pgfqpoint{1.870822in}{2.085697in}}{\pgfqpoint{1.874095in}{2.077797in}}{\pgfqpoint{1.879919in}{2.071973in}}%
\pgfpathcurveto{\pgfqpoint{1.885742in}{2.066149in}}{\pgfqpoint{1.893643in}{2.062877in}}{\pgfqpoint{1.901879in}{2.062877in}}%
\pgfpathclose%
\pgfusepath{stroke,fill}%
\end{pgfscope}%
\begin{pgfscope}%
\pgfpathrectangle{\pgfqpoint{0.100000in}{0.212622in}}{\pgfqpoint{3.696000in}{3.696000in}}%
\pgfusepath{clip}%
\pgfsetbuttcap%
\pgfsetroundjoin%
\definecolor{currentfill}{rgb}{0.121569,0.466667,0.705882}%
\pgfsetfillcolor{currentfill}%
\pgfsetfillopacity{0.312153}%
\pgfsetlinewidth{1.003750pt}%
\definecolor{currentstroke}{rgb}{0.121569,0.466667,0.705882}%
\pgfsetstrokecolor{currentstroke}%
\pgfsetstrokeopacity{0.312153}%
\pgfsetdash{}{0pt}%
\pgfpathmoveto{\pgfqpoint{1.949673in}{2.070710in}}%
\pgfpathcurveto{\pgfqpoint{1.957909in}{2.070710in}}{\pgfqpoint{1.965809in}{2.073983in}}{\pgfqpoint{1.971633in}{2.079807in}}%
\pgfpathcurveto{\pgfqpoint{1.977457in}{2.085631in}}{\pgfqpoint{1.980730in}{2.093531in}}{\pgfqpoint{1.980730in}{2.101767in}}%
\pgfpathcurveto{\pgfqpoint{1.980730in}{2.110003in}}{\pgfqpoint{1.977457in}{2.117903in}}{\pgfqpoint{1.971633in}{2.123727in}}%
\pgfpathcurveto{\pgfqpoint{1.965809in}{2.129551in}}{\pgfqpoint{1.957909in}{2.132823in}}{\pgfqpoint{1.949673in}{2.132823in}}%
\pgfpathcurveto{\pgfqpoint{1.941437in}{2.132823in}}{\pgfqpoint{1.933537in}{2.129551in}}{\pgfqpoint{1.927713in}{2.123727in}}%
\pgfpathcurveto{\pgfqpoint{1.921889in}{2.117903in}}{\pgfqpoint{1.918617in}{2.110003in}}{\pgfqpoint{1.918617in}{2.101767in}}%
\pgfpathcurveto{\pgfqpoint{1.918617in}{2.093531in}}{\pgfqpoint{1.921889in}{2.085631in}}{\pgfqpoint{1.927713in}{2.079807in}}%
\pgfpathcurveto{\pgfqpoint{1.933537in}{2.073983in}}{\pgfqpoint{1.941437in}{2.070710in}}{\pgfqpoint{1.949673in}{2.070710in}}%
\pgfpathclose%
\pgfusepath{stroke,fill}%
\end{pgfscope}%
\begin{pgfscope}%
\pgfpathrectangle{\pgfqpoint{0.100000in}{0.212622in}}{\pgfqpoint{3.696000in}{3.696000in}}%
\pgfusepath{clip}%
\pgfsetbuttcap%
\pgfsetroundjoin%
\definecolor{currentfill}{rgb}{0.121569,0.466667,0.705882}%
\pgfsetfillcolor{currentfill}%
\pgfsetfillopacity{0.312923}%
\pgfsetlinewidth{1.003750pt}%
\definecolor{currentstroke}{rgb}{0.121569,0.466667,0.705882}%
\pgfsetstrokecolor{currentstroke}%
\pgfsetstrokeopacity{0.312923}%
\pgfsetdash{}{0pt}%
\pgfpathmoveto{\pgfqpoint{1.897682in}{2.060179in}}%
\pgfpathcurveto{\pgfqpoint{1.905919in}{2.060179in}}{\pgfqpoint{1.913819in}{2.063452in}}{\pgfqpoint{1.919643in}{2.069276in}}%
\pgfpathcurveto{\pgfqpoint{1.925467in}{2.075100in}}{\pgfqpoint{1.928739in}{2.083000in}}{\pgfqpoint{1.928739in}{2.091236in}}%
\pgfpathcurveto{\pgfqpoint{1.928739in}{2.099472in}}{\pgfqpoint{1.925467in}{2.107372in}}{\pgfqpoint{1.919643in}{2.113196in}}%
\pgfpathcurveto{\pgfqpoint{1.913819in}{2.119020in}}{\pgfqpoint{1.905919in}{2.122292in}}{\pgfqpoint{1.897682in}{2.122292in}}%
\pgfpathcurveto{\pgfqpoint{1.889446in}{2.122292in}}{\pgfqpoint{1.881546in}{2.119020in}}{\pgfqpoint{1.875722in}{2.113196in}}%
\pgfpathcurveto{\pgfqpoint{1.869898in}{2.107372in}}{\pgfqpoint{1.866626in}{2.099472in}}{\pgfqpoint{1.866626in}{2.091236in}}%
\pgfpathcurveto{\pgfqpoint{1.866626in}{2.083000in}}{\pgfqpoint{1.869898in}{2.075100in}}{\pgfqpoint{1.875722in}{2.069276in}}%
\pgfpathcurveto{\pgfqpoint{1.881546in}{2.063452in}}{\pgfqpoint{1.889446in}{2.060179in}}{\pgfqpoint{1.897682in}{2.060179in}}%
\pgfpathclose%
\pgfusepath{stroke,fill}%
\end{pgfscope}%
\begin{pgfscope}%
\pgfpathrectangle{\pgfqpoint{0.100000in}{0.212622in}}{\pgfqpoint{3.696000in}{3.696000in}}%
\pgfusepath{clip}%
\pgfsetbuttcap%
\pgfsetroundjoin%
\definecolor{currentfill}{rgb}{0.121569,0.466667,0.705882}%
\pgfsetfillcolor{currentfill}%
\pgfsetfillopacity{0.313897}%
\pgfsetlinewidth{1.003750pt}%
\definecolor{currentstroke}{rgb}{0.121569,0.466667,0.705882}%
\pgfsetstrokecolor{currentstroke}%
\pgfsetstrokeopacity{0.313897}%
\pgfsetdash{}{0pt}%
\pgfpathmoveto{\pgfqpoint{1.893844in}{2.057564in}}%
\pgfpathcurveto{\pgfqpoint{1.902080in}{2.057564in}}{\pgfqpoint{1.909980in}{2.060837in}}{\pgfqpoint{1.915804in}{2.066661in}}%
\pgfpathcurveto{\pgfqpoint{1.921628in}{2.072485in}}{\pgfqpoint{1.924901in}{2.080385in}}{\pgfqpoint{1.924901in}{2.088621in}}%
\pgfpathcurveto{\pgfqpoint{1.924901in}{2.096857in}}{\pgfqpoint{1.921628in}{2.104757in}}{\pgfqpoint{1.915804in}{2.110581in}}%
\pgfpathcurveto{\pgfqpoint{1.909980in}{2.116405in}}{\pgfqpoint{1.902080in}{2.119677in}}{\pgfqpoint{1.893844in}{2.119677in}}%
\pgfpathcurveto{\pgfqpoint{1.885608in}{2.119677in}}{\pgfqpoint{1.877708in}{2.116405in}}{\pgfqpoint{1.871884in}{2.110581in}}%
\pgfpathcurveto{\pgfqpoint{1.866060in}{2.104757in}}{\pgfqpoint{1.862788in}{2.096857in}}{\pgfqpoint{1.862788in}{2.088621in}}%
\pgfpathcurveto{\pgfqpoint{1.862788in}{2.080385in}}{\pgfqpoint{1.866060in}{2.072485in}}{\pgfqpoint{1.871884in}{2.066661in}}%
\pgfpathcurveto{\pgfqpoint{1.877708in}{2.060837in}}{\pgfqpoint{1.885608in}{2.057564in}}{\pgfqpoint{1.893844in}{2.057564in}}%
\pgfpathclose%
\pgfusepath{stroke,fill}%
\end{pgfscope}%
\begin{pgfscope}%
\pgfpathrectangle{\pgfqpoint{0.100000in}{0.212622in}}{\pgfqpoint{3.696000in}{3.696000in}}%
\pgfusepath{clip}%
\pgfsetbuttcap%
\pgfsetroundjoin%
\definecolor{currentfill}{rgb}{0.121569,0.466667,0.705882}%
\pgfsetfillcolor{currentfill}%
\pgfsetfillopacity{0.314663}%
\pgfsetlinewidth{1.003750pt}%
\definecolor{currentstroke}{rgb}{0.121569,0.466667,0.705882}%
\pgfsetstrokecolor{currentstroke}%
\pgfsetstrokeopacity{0.314663}%
\pgfsetdash{}{0pt}%
\pgfpathmoveto{\pgfqpoint{1.950510in}{2.071246in}}%
\pgfpathcurveto{\pgfqpoint{1.958746in}{2.071246in}}{\pgfqpoint{1.966646in}{2.074518in}}{\pgfqpoint{1.972470in}{2.080342in}}%
\pgfpathcurveto{\pgfqpoint{1.978294in}{2.086166in}}{\pgfqpoint{1.981566in}{2.094066in}}{\pgfqpoint{1.981566in}{2.102302in}}%
\pgfpathcurveto{\pgfqpoint{1.981566in}{2.110539in}}{\pgfqpoint{1.978294in}{2.118439in}}{\pgfqpoint{1.972470in}{2.124263in}}%
\pgfpathcurveto{\pgfqpoint{1.966646in}{2.130087in}}{\pgfqpoint{1.958746in}{2.133359in}}{\pgfqpoint{1.950510in}{2.133359in}}%
\pgfpathcurveto{\pgfqpoint{1.942273in}{2.133359in}}{\pgfqpoint{1.934373in}{2.130087in}}{\pgfqpoint{1.928549in}{2.124263in}}%
\pgfpathcurveto{\pgfqpoint{1.922725in}{2.118439in}}{\pgfqpoint{1.919453in}{2.110539in}}{\pgfqpoint{1.919453in}{2.102302in}}%
\pgfpathcurveto{\pgfqpoint{1.919453in}{2.094066in}}{\pgfqpoint{1.922725in}{2.086166in}}{\pgfqpoint{1.928549in}{2.080342in}}%
\pgfpathcurveto{\pgfqpoint{1.934373in}{2.074518in}}{\pgfqpoint{1.942273in}{2.071246in}}{\pgfqpoint{1.950510in}{2.071246in}}%
\pgfpathclose%
\pgfusepath{stroke,fill}%
\end{pgfscope}%
\begin{pgfscope}%
\pgfpathrectangle{\pgfqpoint{0.100000in}{0.212622in}}{\pgfqpoint{3.696000in}{3.696000in}}%
\pgfusepath{clip}%
\pgfsetbuttcap%
\pgfsetroundjoin%
\definecolor{currentfill}{rgb}{0.121569,0.466667,0.705882}%
\pgfsetfillcolor{currentfill}%
\pgfsetfillopacity{0.315607}%
\pgfsetlinewidth{1.003750pt}%
\definecolor{currentstroke}{rgb}{0.121569,0.466667,0.705882}%
\pgfsetstrokecolor{currentstroke}%
\pgfsetstrokeopacity{0.315607}%
\pgfsetdash{}{0pt}%
\pgfpathmoveto{\pgfqpoint{1.887687in}{2.051090in}}%
\pgfpathcurveto{\pgfqpoint{1.895923in}{2.051090in}}{\pgfqpoint{1.903823in}{2.054362in}}{\pgfqpoint{1.909647in}{2.060186in}}%
\pgfpathcurveto{\pgfqpoint{1.915471in}{2.066010in}}{\pgfqpoint{1.918743in}{2.073910in}}{\pgfqpoint{1.918743in}{2.082146in}}%
\pgfpathcurveto{\pgfqpoint{1.918743in}{2.090383in}}{\pgfqpoint{1.915471in}{2.098283in}}{\pgfqpoint{1.909647in}{2.104107in}}%
\pgfpathcurveto{\pgfqpoint{1.903823in}{2.109931in}}{\pgfqpoint{1.895923in}{2.113203in}}{\pgfqpoint{1.887687in}{2.113203in}}%
\pgfpathcurveto{\pgfqpoint{1.879450in}{2.113203in}}{\pgfqpoint{1.871550in}{2.109931in}}{\pgfqpoint{1.865726in}{2.104107in}}%
\pgfpathcurveto{\pgfqpoint{1.859902in}{2.098283in}}{\pgfqpoint{1.856630in}{2.090383in}}{\pgfqpoint{1.856630in}{2.082146in}}%
\pgfpathcurveto{\pgfqpoint{1.856630in}{2.073910in}}{\pgfqpoint{1.859902in}{2.066010in}}{\pgfqpoint{1.865726in}{2.060186in}}%
\pgfpathcurveto{\pgfqpoint{1.871550in}{2.054362in}}{\pgfqpoint{1.879450in}{2.051090in}}{\pgfqpoint{1.887687in}{2.051090in}}%
\pgfpathclose%
\pgfusepath{stroke,fill}%
\end{pgfscope}%
\begin{pgfscope}%
\pgfpathrectangle{\pgfqpoint{0.100000in}{0.212622in}}{\pgfqpoint{3.696000in}{3.696000in}}%
\pgfusepath{clip}%
\pgfsetbuttcap%
\pgfsetroundjoin%
\definecolor{currentfill}{rgb}{0.121569,0.466667,0.705882}%
\pgfsetfillcolor{currentfill}%
\pgfsetfillopacity{0.315740}%
\pgfsetlinewidth{1.003750pt}%
\definecolor{currentstroke}{rgb}{0.121569,0.466667,0.705882}%
\pgfsetstrokecolor{currentstroke}%
\pgfsetstrokeopacity{0.315740}%
\pgfsetdash{}{0pt}%
\pgfpathmoveto{\pgfqpoint{1.951538in}{2.069721in}}%
\pgfpathcurveto{\pgfqpoint{1.959774in}{2.069721in}}{\pgfqpoint{1.967674in}{2.072993in}}{\pgfqpoint{1.973498in}{2.078817in}}%
\pgfpathcurveto{\pgfqpoint{1.979322in}{2.084641in}}{\pgfqpoint{1.982594in}{2.092541in}}{\pgfqpoint{1.982594in}{2.100777in}}%
\pgfpathcurveto{\pgfqpoint{1.982594in}{2.109014in}}{\pgfqpoint{1.979322in}{2.116914in}}{\pgfqpoint{1.973498in}{2.122738in}}%
\pgfpathcurveto{\pgfqpoint{1.967674in}{2.128562in}}{\pgfqpoint{1.959774in}{2.131834in}}{\pgfqpoint{1.951538in}{2.131834in}}%
\pgfpathcurveto{\pgfqpoint{1.943302in}{2.131834in}}{\pgfqpoint{1.935402in}{2.128562in}}{\pgfqpoint{1.929578in}{2.122738in}}%
\pgfpathcurveto{\pgfqpoint{1.923754in}{2.116914in}}{\pgfqpoint{1.920481in}{2.109014in}}{\pgfqpoint{1.920481in}{2.100777in}}%
\pgfpathcurveto{\pgfqpoint{1.920481in}{2.092541in}}{\pgfqpoint{1.923754in}{2.084641in}}{\pgfqpoint{1.929578in}{2.078817in}}%
\pgfpathcurveto{\pgfqpoint{1.935402in}{2.072993in}}{\pgfqpoint{1.943302in}{2.069721in}}{\pgfqpoint{1.951538in}{2.069721in}}%
\pgfpathclose%
\pgfusepath{stroke,fill}%
\end{pgfscope}%
\begin{pgfscope}%
\pgfpathrectangle{\pgfqpoint{0.100000in}{0.212622in}}{\pgfqpoint{3.696000in}{3.696000in}}%
\pgfusepath{clip}%
\pgfsetbuttcap%
\pgfsetroundjoin%
\definecolor{currentfill}{rgb}{0.121569,0.466667,0.705882}%
\pgfsetfillcolor{currentfill}%
\pgfsetfillopacity{0.317076}%
\pgfsetlinewidth{1.003750pt}%
\definecolor{currentstroke}{rgb}{0.121569,0.466667,0.705882}%
\pgfsetstrokecolor{currentstroke}%
\pgfsetstrokeopacity{0.317076}%
\pgfsetdash{}{0pt}%
\pgfpathmoveto{\pgfqpoint{1.883298in}{2.049492in}}%
\pgfpathcurveto{\pgfqpoint{1.891534in}{2.049492in}}{\pgfqpoint{1.899434in}{2.052764in}}{\pgfqpoint{1.905258in}{2.058588in}}%
\pgfpathcurveto{\pgfqpoint{1.911082in}{2.064412in}}{\pgfqpoint{1.914354in}{2.072312in}}{\pgfqpoint{1.914354in}{2.080548in}}%
\pgfpathcurveto{\pgfqpoint{1.914354in}{2.088785in}}{\pgfqpoint{1.911082in}{2.096685in}}{\pgfqpoint{1.905258in}{2.102509in}}%
\pgfpathcurveto{\pgfqpoint{1.899434in}{2.108332in}}{\pgfqpoint{1.891534in}{2.111605in}}{\pgfqpoint{1.883298in}{2.111605in}}%
\pgfpathcurveto{\pgfqpoint{1.875061in}{2.111605in}}{\pgfqpoint{1.867161in}{2.108332in}}{\pgfqpoint{1.861337in}{2.102509in}}%
\pgfpathcurveto{\pgfqpoint{1.855513in}{2.096685in}}{\pgfqpoint{1.852241in}{2.088785in}}{\pgfqpoint{1.852241in}{2.080548in}}%
\pgfpathcurveto{\pgfqpoint{1.852241in}{2.072312in}}{\pgfqpoint{1.855513in}{2.064412in}}{\pgfqpoint{1.861337in}{2.058588in}}%
\pgfpathcurveto{\pgfqpoint{1.867161in}{2.052764in}}{\pgfqpoint{1.875061in}{2.049492in}}{\pgfqpoint{1.883298in}{2.049492in}}%
\pgfpathclose%
\pgfusepath{stroke,fill}%
\end{pgfscope}%
\begin{pgfscope}%
\pgfpathrectangle{\pgfqpoint{0.100000in}{0.212622in}}{\pgfqpoint{3.696000in}{3.696000in}}%
\pgfusepath{clip}%
\pgfsetbuttcap%
\pgfsetroundjoin%
\definecolor{currentfill}{rgb}{0.121569,0.466667,0.705882}%
\pgfsetfillcolor{currentfill}%
\pgfsetfillopacity{0.317285}%
\pgfsetlinewidth{1.003750pt}%
\definecolor{currentstroke}{rgb}{0.121569,0.466667,0.705882}%
\pgfsetstrokecolor{currentstroke}%
\pgfsetstrokeopacity{0.317285}%
\pgfsetdash{}{0pt}%
\pgfpathmoveto{\pgfqpoint{1.952707in}{2.068305in}}%
\pgfpathcurveto{\pgfqpoint{1.960944in}{2.068305in}}{\pgfqpoint{1.968844in}{2.071577in}}{\pgfqpoint{1.974668in}{2.077401in}}%
\pgfpathcurveto{\pgfqpoint{1.980492in}{2.083225in}}{\pgfqpoint{1.983764in}{2.091125in}}{\pgfqpoint{1.983764in}{2.099361in}}%
\pgfpathcurveto{\pgfqpoint{1.983764in}{2.107598in}}{\pgfqpoint{1.980492in}{2.115498in}}{\pgfqpoint{1.974668in}{2.121322in}}%
\pgfpathcurveto{\pgfqpoint{1.968844in}{2.127146in}}{\pgfqpoint{1.960944in}{2.130418in}}{\pgfqpoint{1.952707in}{2.130418in}}%
\pgfpathcurveto{\pgfqpoint{1.944471in}{2.130418in}}{\pgfqpoint{1.936571in}{2.127146in}}{\pgfqpoint{1.930747in}{2.121322in}}%
\pgfpathcurveto{\pgfqpoint{1.924923in}{2.115498in}}{\pgfqpoint{1.921651in}{2.107598in}}{\pgfqpoint{1.921651in}{2.099361in}}%
\pgfpathcurveto{\pgfqpoint{1.921651in}{2.091125in}}{\pgfqpoint{1.924923in}{2.083225in}}{\pgfqpoint{1.930747in}{2.077401in}}%
\pgfpathcurveto{\pgfqpoint{1.936571in}{2.071577in}}{\pgfqpoint{1.944471in}{2.068305in}}{\pgfqpoint{1.952707in}{2.068305in}}%
\pgfpathclose%
\pgfusepath{stroke,fill}%
\end{pgfscope}%
\begin{pgfscope}%
\pgfpathrectangle{\pgfqpoint{0.100000in}{0.212622in}}{\pgfqpoint{3.696000in}{3.696000in}}%
\pgfusepath{clip}%
\pgfsetbuttcap%
\pgfsetroundjoin%
\definecolor{currentfill}{rgb}{0.121569,0.466667,0.705882}%
\pgfsetfillcolor{currentfill}%
\pgfsetfillopacity{0.318268}%
\pgfsetlinewidth{1.003750pt}%
\definecolor{currentstroke}{rgb}{0.121569,0.466667,0.705882}%
\pgfsetstrokecolor{currentstroke}%
\pgfsetstrokeopacity{0.318268}%
\pgfsetdash{}{0pt}%
\pgfpathmoveto{\pgfqpoint{1.879378in}{2.046613in}}%
\pgfpathcurveto{\pgfqpoint{1.887614in}{2.046613in}}{\pgfqpoint{1.895515in}{2.049886in}}{\pgfqpoint{1.901338in}{2.055709in}}%
\pgfpathcurveto{\pgfqpoint{1.907162in}{2.061533in}}{\pgfqpoint{1.910435in}{2.069433in}}{\pgfqpoint{1.910435in}{2.077670in}}%
\pgfpathcurveto{\pgfqpoint{1.910435in}{2.085906in}}{\pgfqpoint{1.907162in}{2.093806in}}{\pgfqpoint{1.901338in}{2.099630in}}%
\pgfpathcurveto{\pgfqpoint{1.895515in}{2.105454in}}{\pgfqpoint{1.887614in}{2.108726in}}{\pgfqpoint{1.879378in}{2.108726in}}%
\pgfpathcurveto{\pgfqpoint{1.871142in}{2.108726in}}{\pgfqpoint{1.863242in}{2.105454in}}{\pgfqpoint{1.857418in}{2.099630in}}%
\pgfpathcurveto{\pgfqpoint{1.851594in}{2.093806in}}{\pgfqpoint{1.848322in}{2.085906in}}{\pgfqpoint{1.848322in}{2.077670in}}%
\pgfpathcurveto{\pgfqpoint{1.848322in}{2.069433in}}{\pgfqpoint{1.851594in}{2.061533in}}{\pgfqpoint{1.857418in}{2.055709in}}%
\pgfpathcurveto{\pgfqpoint{1.863242in}{2.049886in}}{\pgfqpoint{1.871142in}{2.046613in}}{\pgfqpoint{1.879378in}{2.046613in}}%
\pgfpathclose%
\pgfusepath{stroke,fill}%
\end{pgfscope}%
\begin{pgfscope}%
\pgfpathrectangle{\pgfqpoint{0.100000in}{0.212622in}}{\pgfqpoint{3.696000in}{3.696000in}}%
\pgfusepath{clip}%
\pgfsetbuttcap%
\pgfsetroundjoin%
\definecolor{currentfill}{rgb}{0.121569,0.466667,0.705882}%
\pgfsetfillcolor{currentfill}%
\pgfsetfillopacity{0.318307}%
\pgfsetlinewidth{1.003750pt}%
\definecolor{currentstroke}{rgb}{0.121569,0.466667,0.705882}%
\pgfsetstrokecolor{currentstroke}%
\pgfsetstrokeopacity{0.318307}%
\pgfsetdash{}{0pt}%
\pgfpathmoveto{\pgfqpoint{1.953263in}{2.068676in}}%
\pgfpathcurveto{\pgfqpoint{1.961499in}{2.068676in}}{\pgfqpoint{1.969399in}{2.071948in}}{\pgfqpoint{1.975223in}{2.077772in}}%
\pgfpathcurveto{\pgfqpoint{1.981047in}{2.083596in}}{\pgfqpoint{1.984320in}{2.091496in}}{\pgfqpoint{1.984320in}{2.099733in}}%
\pgfpathcurveto{\pgfqpoint{1.984320in}{2.107969in}}{\pgfqpoint{1.981047in}{2.115869in}}{\pgfqpoint{1.975223in}{2.121693in}}%
\pgfpathcurveto{\pgfqpoint{1.969399in}{2.127517in}}{\pgfqpoint{1.961499in}{2.130789in}}{\pgfqpoint{1.953263in}{2.130789in}}%
\pgfpathcurveto{\pgfqpoint{1.945027in}{2.130789in}}{\pgfqpoint{1.937127in}{2.127517in}}{\pgfqpoint{1.931303in}{2.121693in}}%
\pgfpathcurveto{\pgfqpoint{1.925479in}{2.115869in}}{\pgfqpoint{1.922207in}{2.107969in}}{\pgfqpoint{1.922207in}{2.099733in}}%
\pgfpathcurveto{\pgfqpoint{1.922207in}{2.091496in}}{\pgfqpoint{1.925479in}{2.083596in}}{\pgfqpoint{1.931303in}{2.077772in}}%
\pgfpathcurveto{\pgfqpoint{1.937127in}{2.071948in}}{\pgfqpoint{1.945027in}{2.068676in}}{\pgfqpoint{1.953263in}{2.068676in}}%
\pgfpathclose%
\pgfusepath{stroke,fill}%
\end{pgfscope}%
\begin{pgfscope}%
\pgfpathrectangle{\pgfqpoint{0.100000in}{0.212622in}}{\pgfqpoint{3.696000in}{3.696000in}}%
\pgfusepath{clip}%
\pgfsetbuttcap%
\pgfsetroundjoin%
\definecolor{currentfill}{rgb}{0.121569,0.466667,0.705882}%
\pgfsetfillcolor{currentfill}%
\pgfsetfillopacity{0.319766}%
\pgfsetlinewidth{1.003750pt}%
\definecolor{currentstroke}{rgb}{0.121569,0.466667,0.705882}%
\pgfsetstrokecolor{currentstroke}%
\pgfsetstrokeopacity{0.319766}%
\pgfsetdash{}{0pt}%
\pgfpathmoveto{\pgfqpoint{1.954444in}{2.066949in}}%
\pgfpathcurveto{\pgfqpoint{1.962680in}{2.066949in}}{\pgfqpoint{1.970580in}{2.070222in}}{\pgfqpoint{1.976404in}{2.076045in}}%
\pgfpathcurveto{\pgfqpoint{1.982228in}{2.081869in}}{\pgfqpoint{1.985501in}{2.089769in}}{\pgfqpoint{1.985501in}{2.098006in}}%
\pgfpathcurveto{\pgfqpoint{1.985501in}{2.106242in}}{\pgfqpoint{1.982228in}{2.114142in}}{\pgfqpoint{1.976404in}{2.119966in}}%
\pgfpathcurveto{\pgfqpoint{1.970580in}{2.125790in}}{\pgfqpoint{1.962680in}{2.129062in}}{\pgfqpoint{1.954444in}{2.129062in}}%
\pgfpathcurveto{\pgfqpoint{1.946208in}{2.129062in}}{\pgfqpoint{1.938308in}{2.125790in}}{\pgfqpoint{1.932484in}{2.119966in}}%
\pgfpathcurveto{\pgfqpoint{1.926660in}{2.114142in}}{\pgfqpoint{1.923388in}{2.106242in}}{\pgfqpoint{1.923388in}{2.098006in}}%
\pgfpathcurveto{\pgfqpoint{1.923388in}{2.089769in}}{\pgfqpoint{1.926660in}{2.081869in}}{\pgfqpoint{1.932484in}{2.076045in}}%
\pgfpathcurveto{\pgfqpoint{1.938308in}{2.070222in}}{\pgfqpoint{1.946208in}{2.066949in}}{\pgfqpoint{1.954444in}{2.066949in}}%
\pgfpathclose%
\pgfusepath{stroke,fill}%
\end{pgfscope}%
\begin{pgfscope}%
\pgfpathrectangle{\pgfqpoint{0.100000in}{0.212622in}}{\pgfqpoint{3.696000in}{3.696000in}}%
\pgfusepath{clip}%
\pgfsetbuttcap%
\pgfsetroundjoin%
\definecolor{currentfill}{rgb}{0.121569,0.466667,0.705882}%
\pgfsetfillcolor{currentfill}%
\pgfsetfillopacity{0.320258}%
\pgfsetlinewidth{1.003750pt}%
\definecolor{currentstroke}{rgb}{0.121569,0.466667,0.705882}%
\pgfsetstrokecolor{currentstroke}%
\pgfsetstrokeopacity{0.320258}%
\pgfsetdash{}{0pt}%
\pgfpathmoveto{\pgfqpoint{1.872719in}{2.039500in}}%
\pgfpathcurveto{\pgfqpoint{1.880956in}{2.039500in}}{\pgfqpoint{1.888856in}{2.042772in}}{\pgfqpoint{1.894680in}{2.048596in}}%
\pgfpathcurveto{\pgfqpoint{1.900503in}{2.054420in}}{\pgfqpoint{1.903776in}{2.062320in}}{\pgfqpoint{1.903776in}{2.070556in}}%
\pgfpathcurveto{\pgfqpoint{1.903776in}{2.078793in}}{\pgfqpoint{1.900503in}{2.086693in}}{\pgfqpoint{1.894680in}{2.092517in}}%
\pgfpathcurveto{\pgfqpoint{1.888856in}{2.098341in}}{\pgfqpoint{1.880956in}{2.101613in}}{\pgfqpoint{1.872719in}{2.101613in}}%
\pgfpathcurveto{\pgfqpoint{1.864483in}{2.101613in}}{\pgfqpoint{1.856583in}{2.098341in}}{\pgfqpoint{1.850759in}{2.092517in}}%
\pgfpathcurveto{\pgfqpoint{1.844935in}{2.086693in}}{\pgfqpoint{1.841663in}{2.078793in}}{\pgfqpoint{1.841663in}{2.070556in}}%
\pgfpathcurveto{\pgfqpoint{1.841663in}{2.062320in}}{\pgfqpoint{1.844935in}{2.054420in}}{\pgfqpoint{1.850759in}{2.048596in}}%
\pgfpathcurveto{\pgfqpoint{1.856583in}{2.042772in}}{\pgfqpoint{1.864483in}{2.039500in}}{\pgfqpoint{1.872719in}{2.039500in}}%
\pgfpathclose%
\pgfusepath{stroke,fill}%
\end{pgfscope}%
\begin{pgfscope}%
\pgfpathrectangle{\pgfqpoint{0.100000in}{0.212622in}}{\pgfqpoint{3.696000in}{3.696000in}}%
\pgfusepath{clip}%
\pgfsetbuttcap%
\pgfsetroundjoin%
\definecolor{currentfill}{rgb}{0.121569,0.466667,0.705882}%
\pgfsetfillcolor{currentfill}%
\pgfsetfillopacity{0.321663}%
\pgfsetlinewidth{1.003750pt}%
\definecolor{currentstroke}{rgb}{0.121569,0.466667,0.705882}%
\pgfsetstrokecolor{currentstroke}%
\pgfsetstrokeopacity{0.321663}%
\pgfsetdash{}{0pt}%
\pgfpathmoveto{\pgfqpoint{1.955978in}{2.065593in}}%
\pgfpathcurveto{\pgfqpoint{1.964214in}{2.065593in}}{\pgfqpoint{1.972114in}{2.068866in}}{\pgfqpoint{1.977938in}{2.074690in}}%
\pgfpathcurveto{\pgfqpoint{1.983762in}{2.080514in}}{\pgfqpoint{1.987034in}{2.088414in}}{\pgfqpoint{1.987034in}{2.096650in}}%
\pgfpathcurveto{\pgfqpoint{1.987034in}{2.104886in}}{\pgfqpoint{1.983762in}{2.112786in}}{\pgfqpoint{1.977938in}{2.118610in}}%
\pgfpathcurveto{\pgfqpoint{1.972114in}{2.124434in}}{\pgfqpoint{1.964214in}{2.127706in}}{\pgfqpoint{1.955978in}{2.127706in}}%
\pgfpathcurveto{\pgfqpoint{1.947742in}{2.127706in}}{\pgfqpoint{1.939842in}{2.124434in}}{\pgfqpoint{1.934018in}{2.118610in}}%
\pgfpathcurveto{\pgfqpoint{1.928194in}{2.112786in}}{\pgfqpoint{1.924921in}{2.104886in}}{\pgfqpoint{1.924921in}{2.096650in}}%
\pgfpathcurveto{\pgfqpoint{1.924921in}{2.088414in}}{\pgfqpoint{1.928194in}{2.080514in}}{\pgfqpoint{1.934018in}{2.074690in}}%
\pgfpathcurveto{\pgfqpoint{1.939842in}{2.068866in}}{\pgfqpoint{1.947742in}{2.065593in}}{\pgfqpoint{1.955978in}{2.065593in}}%
\pgfpathclose%
\pgfusepath{stroke,fill}%
\end{pgfscope}%
\begin{pgfscope}%
\pgfpathrectangle{\pgfqpoint{0.100000in}{0.212622in}}{\pgfqpoint{3.696000in}{3.696000in}}%
\pgfusepath{clip}%
\pgfsetbuttcap%
\pgfsetroundjoin%
\definecolor{currentfill}{rgb}{0.121569,0.466667,0.705882}%
\pgfsetfillcolor{currentfill}%
\pgfsetfillopacity{0.321725}%
\pgfsetlinewidth{1.003750pt}%
\definecolor{currentstroke}{rgb}{0.121569,0.466667,0.705882}%
\pgfsetstrokecolor{currentstroke}%
\pgfsetstrokeopacity{0.321725}%
\pgfsetdash{}{0pt}%
\pgfpathmoveto{\pgfqpoint{1.867524in}{2.037232in}}%
\pgfpathcurveto{\pgfqpoint{1.875760in}{2.037232in}}{\pgfqpoint{1.883660in}{2.040504in}}{\pgfqpoint{1.889484in}{2.046328in}}%
\pgfpathcurveto{\pgfqpoint{1.895308in}{2.052152in}}{\pgfqpoint{1.898581in}{2.060052in}}{\pgfqpoint{1.898581in}{2.068289in}}%
\pgfpathcurveto{\pgfqpoint{1.898581in}{2.076525in}}{\pgfqpoint{1.895308in}{2.084425in}}{\pgfqpoint{1.889484in}{2.090249in}}%
\pgfpathcurveto{\pgfqpoint{1.883660in}{2.096073in}}{\pgfqpoint{1.875760in}{2.099345in}}{\pgfqpoint{1.867524in}{2.099345in}}%
\pgfpathcurveto{\pgfqpoint{1.859288in}{2.099345in}}{\pgfqpoint{1.851388in}{2.096073in}}{\pgfqpoint{1.845564in}{2.090249in}}%
\pgfpathcurveto{\pgfqpoint{1.839740in}{2.084425in}}{\pgfqpoint{1.836468in}{2.076525in}}{\pgfqpoint{1.836468in}{2.068289in}}%
\pgfpathcurveto{\pgfqpoint{1.836468in}{2.060052in}}{\pgfqpoint{1.839740in}{2.052152in}}{\pgfqpoint{1.845564in}{2.046328in}}%
\pgfpathcurveto{\pgfqpoint{1.851388in}{2.040504in}}{\pgfqpoint{1.859288in}{2.037232in}}{\pgfqpoint{1.867524in}{2.037232in}}%
\pgfpathclose%
\pgfusepath{stroke,fill}%
\end{pgfscope}%
\begin{pgfscope}%
\pgfpathrectangle{\pgfqpoint{0.100000in}{0.212622in}}{\pgfqpoint{3.696000in}{3.696000in}}%
\pgfusepath{clip}%
\pgfsetbuttcap%
\pgfsetroundjoin%
\definecolor{currentfill}{rgb}{0.121569,0.466667,0.705882}%
\pgfsetfillcolor{currentfill}%
\pgfsetfillopacity{0.323109}%
\pgfsetlinewidth{1.003750pt}%
\definecolor{currentstroke}{rgb}{0.121569,0.466667,0.705882}%
\pgfsetstrokecolor{currentstroke}%
\pgfsetstrokeopacity{0.323109}%
\pgfsetdash{}{0pt}%
\pgfpathmoveto{\pgfqpoint{1.863244in}{2.034919in}}%
\pgfpathcurveto{\pgfqpoint{1.871480in}{2.034919in}}{\pgfqpoint{1.879380in}{2.038191in}}{\pgfqpoint{1.885204in}{2.044015in}}%
\pgfpathcurveto{\pgfqpoint{1.891028in}{2.049839in}}{\pgfqpoint{1.894300in}{2.057739in}}{\pgfqpoint{1.894300in}{2.065975in}}%
\pgfpathcurveto{\pgfqpoint{1.894300in}{2.074211in}}{\pgfqpoint{1.891028in}{2.082112in}}{\pgfqpoint{1.885204in}{2.087935in}}%
\pgfpathcurveto{\pgfqpoint{1.879380in}{2.093759in}}{\pgfqpoint{1.871480in}{2.097032in}}{\pgfqpoint{1.863244in}{2.097032in}}%
\pgfpathcurveto{\pgfqpoint{1.855008in}{2.097032in}}{\pgfqpoint{1.847108in}{2.093759in}}{\pgfqpoint{1.841284in}{2.087935in}}%
\pgfpathcurveto{\pgfqpoint{1.835460in}{2.082112in}}{\pgfqpoint{1.832187in}{2.074211in}}{\pgfqpoint{1.832187in}{2.065975in}}%
\pgfpathcurveto{\pgfqpoint{1.832187in}{2.057739in}}{\pgfqpoint{1.835460in}{2.049839in}}{\pgfqpoint{1.841284in}{2.044015in}}%
\pgfpathcurveto{\pgfqpoint{1.847108in}{2.038191in}}{\pgfqpoint{1.855008in}{2.034919in}}{\pgfqpoint{1.863244in}{2.034919in}}%
\pgfpathclose%
\pgfusepath{stroke,fill}%
\end{pgfscope}%
\begin{pgfscope}%
\pgfpathrectangle{\pgfqpoint{0.100000in}{0.212622in}}{\pgfqpoint{3.696000in}{3.696000in}}%
\pgfusepath{clip}%
\pgfsetbuttcap%
\pgfsetroundjoin%
\definecolor{currentfill}{rgb}{0.121569,0.466667,0.705882}%
\pgfsetfillcolor{currentfill}%
\pgfsetfillopacity{0.324186}%
\pgfsetlinewidth{1.003750pt}%
\definecolor{currentstroke}{rgb}{0.121569,0.466667,0.705882}%
\pgfsetstrokecolor{currentstroke}%
\pgfsetstrokeopacity{0.324186}%
\pgfsetdash{}{0pt}%
\pgfpathmoveto{\pgfqpoint{1.957545in}{2.065281in}}%
\pgfpathcurveto{\pgfqpoint{1.965781in}{2.065281in}}{\pgfqpoint{1.973681in}{2.068553in}}{\pgfqpoint{1.979505in}{2.074377in}}%
\pgfpathcurveto{\pgfqpoint{1.985329in}{2.080201in}}{\pgfqpoint{1.988601in}{2.088101in}}{\pgfqpoint{1.988601in}{2.096337in}}%
\pgfpathcurveto{\pgfqpoint{1.988601in}{2.104574in}}{\pgfqpoint{1.985329in}{2.112474in}}{\pgfqpoint{1.979505in}{2.118298in}}%
\pgfpathcurveto{\pgfqpoint{1.973681in}{2.124122in}}{\pgfqpoint{1.965781in}{2.127394in}}{\pgfqpoint{1.957545in}{2.127394in}}%
\pgfpathcurveto{\pgfqpoint{1.949308in}{2.127394in}}{\pgfqpoint{1.941408in}{2.124122in}}{\pgfqpoint{1.935585in}{2.118298in}}%
\pgfpathcurveto{\pgfqpoint{1.929761in}{2.112474in}}{\pgfqpoint{1.926488in}{2.104574in}}{\pgfqpoint{1.926488in}{2.096337in}}%
\pgfpathcurveto{\pgfqpoint{1.926488in}{2.088101in}}{\pgfqpoint{1.929761in}{2.080201in}}{\pgfqpoint{1.935585in}{2.074377in}}%
\pgfpathcurveto{\pgfqpoint{1.941408in}{2.068553in}}{\pgfqpoint{1.949308in}{2.065281in}}{\pgfqpoint{1.957545in}{2.065281in}}%
\pgfpathclose%
\pgfusepath{stroke,fill}%
\end{pgfscope}%
\begin{pgfscope}%
\pgfpathrectangle{\pgfqpoint{0.100000in}{0.212622in}}{\pgfqpoint{3.696000in}{3.696000in}}%
\pgfusepath{clip}%
\pgfsetbuttcap%
\pgfsetroundjoin%
\definecolor{currentfill}{rgb}{0.121569,0.466667,0.705882}%
\pgfsetfillcolor{currentfill}%
\pgfsetfillopacity{0.325410}%
\pgfsetlinewidth{1.003750pt}%
\definecolor{currentstroke}{rgb}{0.121569,0.466667,0.705882}%
\pgfsetstrokecolor{currentstroke}%
\pgfsetstrokeopacity{0.325410}%
\pgfsetdash{}{0pt}%
\pgfpathmoveto{\pgfqpoint{1.855212in}{2.029577in}}%
\pgfpathcurveto{\pgfqpoint{1.863448in}{2.029577in}}{\pgfqpoint{1.871348in}{2.032849in}}{\pgfqpoint{1.877172in}{2.038673in}}%
\pgfpathcurveto{\pgfqpoint{1.882996in}{2.044497in}}{\pgfqpoint{1.886269in}{2.052397in}}{\pgfqpoint{1.886269in}{2.060633in}}%
\pgfpathcurveto{\pgfqpoint{1.886269in}{2.068870in}}{\pgfqpoint{1.882996in}{2.076770in}}{\pgfqpoint{1.877172in}{2.082594in}}%
\pgfpathcurveto{\pgfqpoint{1.871348in}{2.088418in}}{\pgfqpoint{1.863448in}{2.091690in}}{\pgfqpoint{1.855212in}{2.091690in}}%
\pgfpathcurveto{\pgfqpoint{1.846976in}{2.091690in}}{\pgfqpoint{1.839076in}{2.088418in}}{\pgfqpoint{1.833252in}{2.082594in}}%
\pgfpathcurveto{\pgfqpoint{1.827428in}{2.076770in}}{\pgfqpoint{1.824156in}{2.068870in}}{\pgfqpoint{1.824156in}{2.060633in}}%
\pgfpathcurveto{\pgfqpoint{1.824156in}{2.052397in}}{\pgfqpoint{1.827428in}{2.044497in}}{\pgfqpoint{1.833252in}{2.038673in}}%
\pgfpathcurveto{\pgfqpoint{1.839076in}{2.032849in}}{\pgfqpoint{1.846976in}{2.029577in}}{\pgfqpoint{1.855212in}{2.029577in}}%
\pgfpathclose%
\pgfusepath{stroke,fill}%
\end{pgfscope}%
\begin{pgfscope}%
\pgfpathrectangle{\pgfqpoint{0.100000in}{0.212622in}}{\pgfqpoint{3.696000in}{3.696000in}}%
\pgfusepath{clip}%
\pgfsetbuttcap%
\pgfsetroundjoin%
\definecolor{currentfill}{rgb}{0.121569,0.466667,0.705882}%
\pgfsetfillcolor{currentfill}%
\pgfsetfillopacity{0.326434}%
\pgfsetlinewidth{1.003750pt}%
\definecolor{currentstroke}{rgb}{0.121569,0.466667,0.705882}%
\pgfsetstrokecolor{currentstroke}%
\pgfsetstrokeopacity{0.326434}%
\pgfsetdash{}{0pt}%
\pgfpathmoveto{\pgfqpoint{1.959560in}{2.062018in}}%
\pgfpathcurveto{\pgfqpoint{1.967796in}{2.062018in}}{\pgfqpoint{1.975696in}{2.065290in}}{\pgfqpoint{1.981520in}{2.071114in}}%
\pgfpathcurveto{\pgfqpoint{1.987344in}{2.076938in}}{\pgfqpoint{1.990617in}{2.084838in}}{\pgfqpoint{1.990617in}{2.093074in}}%
\pgfpathcurveto{\pgfqpoint{1.990617in}{2.101310in}}{\pgfqpoint{1.987344in}{2.109210in}}{\pgfqpoint{1.981520in}{2.115034in}}%
\pgfpathcurveto{\pgfqpoint{1.975696in}{2.120858in}}{\pgfqpoint{1.967796in}{2.124131in}}{\pgfqpoint{1.959560in}{2.124131in}}%
\pgfpathcurveto{\pgfqpoint{1.951324in}{2.124131in}}{\pgfqpoint{1.943424in}{2.120858in}}{\pgfqpoint{1.937600in}{2.115034in}}%
\pgfpathcurveto{\pgfqpoint{1.931776in}{2.109210in}}{\pgfqpoint{1.928504in}{2.101310in}}{\pgfqpoint{1.928504in}{2.093074in}}%
\pgfpathcurveto{\pgfqpoint{1.928504in}{2.084838in}}{\pgfqpoint{1.931776in}{2.076938in}}{\pgfqpoint{1.937600in}{2.071114in}}%
\pgfpathcurveto{\pgfqpoint{1.943424in}{2.065290in}}{\pgfqpoint{1.951324in}{2.062018in}}{\pgfqpoint{1.959560in}{2.062018in}}%
\pgfpathclose%
\pgfusepath{stroke,fill}%
\end{pgfscope}%
\begin{pgfscope}%
\pgfpathrectangle{\pgfqpoint{0.100000in}{0.212622in}}{\pgfqpoint{3.696000in}{3.696000in}}%
\pgfusepath{clip}%
\pgfsetbuttcap%
\pgfsetroundjoin%
\definecolor{currentfill}{rgb}{0.121569,0.466667,0.705882}%
\pgfsetfillcolor{currentfill}%
\pgfsetfillopacity{0.326885}%
\pgfsetlinewidth{1.003750pt}%
\definecolor{currentstroke}{rgb}{0.121569,0.466667,0.705882}%
\pgfsetstrokecolor{currentstroke}%
\pgfsetstrokeopacity{0.326885}%
\pgfsetdash{}{0pt}%
\pgfpathmoveto{\pgfqpoint{1.849686in}{2.026120in}}%
\pgfpathcurveto{\pgfqpoint{1.857923in}{2.026120in}}{\pgfqpoint{1.865823in}{2.029392in}}{\pgfqpoint{1.871647in}{2.035216in}}%
\pgfpathcurveto{\pgfqpoint{1.877471in}{2.041040in}}{\pgfqpoint{1.880743in}{2.048940in}}{\pgfqpoint{1.880743in}{2.057177in}}%
\pgfpathcurveto{\pgfqpoint{1.880743in}{2.065413in}}{\pgfqpoint{1.877471in}{2.073313in}}{\pgfqpoint{1.871647in}{2.079137in}}%
\pgfpathcurveto{\pgfqpoint{1.865823in}{2.084961in}}{\pgfqpoint{1.857923in}{2.088233in}}{\pgfqpoint{1.849686in}{2.088233in}}%
\pgfpathcurveto{\pgfqpoint{1.841450in}{2.088233in}}{\pgfqpoint{1.833550in}{2.084961in}}{\pgfqpoint{1.827726in}{2.079137in}}%
\pgfpathcurveto{\pgfqpoint{1.821902in}{2.073313in}}{\pgfqpoint{1.818630in}{2.065413in}}{\pgfqpoint{1.818630in}{2.057177in}}%
\pgfpathcurveto{\pgfqpoint{1.818630in}{2.048940in}}{\pgfqpoint{1.821902in}{2.041040in}}{\pgfqpoint{1.827726in}{2.035216in}}%
\pgfpathcurveto{\pgfqpoint{1.833550in}{2.029392in}}{\pgfqpoint{1.841450in}{2.026120in}}{\pgfqpoint{1.849686in}{2.026120in}}%
\pgfpathclose%
\pgfusepath{stroke,fill}%
\end{pgfscope}%
\begin{pgfscope}%
\pgfpathrectangle{\pgfqpoint{0.100000in}{0.212622in}}{\pgfqpoint{3.696000in}{3.696000in}}%
\pgfusepath{clip}%
\pgfsetbuttcap%
\pgfsetroundjoin%
\definecolor{currentfill}{rgb}{0.121569,0.466667,0.705882}%
\pgfsetfillcolor{currentfill}%
\pgfsetfillopacity{0.327832}%
\pgfsetlinewidth{1.003750pt}%
\definecolor{currentstroke}{rgb}{0.121569,0.466667,0.705882}%
\pgfsetstrokecolor{currentstroke}%
\pgfsetstrokeopacity{0.327832}%
\pgfsetdash{}{0pt}%
\pgfpathmoveto{\pgfqpoint{1.846513in}{2.024583in}}%
\pgfpathcurveto{\pgfqpoint{1.854749in}{2.024583in}}{\pgfqpoint{1.862650in}{2.027855in}}{\pgfqpoint{1.868473in}{2.033679in}}%
\pgfpathcurveto{\pgfqpoint{1.874297in}{2.039503in}}{\pgfqpoint{1.877570in}{2.047403in}}{\pgfqpoint{1.877570in}{2.055640in}}%
\pgfpathcurveto{\pgfqpoint{1.877570in}{2.063876in}}{\pgfqpoint{1.874297in}{2.071776in}}{\pgfqpoint{1.868473in}{2.077600in}}%
\pgfpathcurveto{\pgfqpoint{1.862650in}{2.083424in}}{\pgfqpoint{1.854749in}{2.086696in}}{\pgfqpoint{1.846513in}{2.086696in}}%
\pgfpathcurveto{\pgfqpoint{1.838277in}{2.086696in}}{\pgfqpoint{1.830377in}{2.083424in}}{\pgfqpoint{1.824553in}{2.077600in}}%
\pgfpathcurveto{\pgfqpoint{1.818729in}{2.071776in}}{\pgfqpoint{1.815457in}{2.063876in}}{\pgfqpoint{1.815457in}{2.055640in}}%
\pgfpathcurveto{\pgfqpoint{1.815457in}{2.047403in}}{\pgfqpoint{1.818729in}{2.039503in}}{\pgfqpoint{1.824553in}{2.033679in}}%
\pgfpathcurveto{\pgfqpoint{1.830377in}{2.027855in}}{\pgfqpoint{1.838277in}{2.024583in}}{\pgfqpoint{1.846513in}{2.024583in}}%
\pgfpathclose%
\pgfusepath{stroke,fill}%
\end{pgfscope}%
\begin{pgfscope}%
\pgfpathrectangle{\pgfqpoint{0.100000in}{0.212622in}}{\pgfqpoint{3.696000in}{3.696000in}}%
\pgfusepath{clip}%
\pgfsetbuttcap%
\pgfsetroundjoin%
\definecolor{currentfill}{rgb}{0.121569,0.466667,0.705882}%
\pgfsetfillcolor{currentfill}%
\pgfsetfillopacity{0.329249}%
\pgfsetlinewidth{1.003750pt}%
\definecolor{currentstroke}{rgb}{0.121569,0.466667,0.705882}%
\pgfsetstrokecolor{currentstroke}%
\pgfsetstrokeopacity{0.329249}%
\pgfsetdash{}{0pt}%
\pgfpathmoveto{\pgfqpoint{1.961767in}{2.061086in}}%
\pgfpathcurveto{\pgfqpoint{1.970004in}{2.061086in}}{\pgfqpoint{1.977904in}{2.064359in}}{\pgfqpoint{1.983728in}{2.070183in}}%
\pgfpathcurveto{\pgfqpoint{1.989552in}{2.076006in}}{\pgfqpoint{1.992824in}{2.083907in}}{\pgfqpoint{1.992824in}{2.092143in}}%
\pgfpathcurveto{\pgfqpoint{1.992824in}{2.100379in}}{\pgfqpoint{1.989552in}{2.108279in}}{\pgfqpoint{1.983728in}{2.114103in}}%
\pgfpathcurveto{\pgfqpoint{1.977904in}{2.119927in}}{\pgfqpoint{1.970004in}{2.123199in}}{\pgfqpoint{1.961767in}{2.123199in}}%
\pgfpathcurveto{\pgfqpoint{1.953531in}{2.123199in}}{\pgfqpoint{1.945631in}{2.119927in}}{\pgfqpoint{1.939807in}{2.114103in}}%
\pgfpathcurveto{\pgfqpoint{1.933983in}{2.108279in}}{\pgfqpoint{1.930711in}{2.100379in}}{\pgfqpoint{1.930711in}{2.092143in}}%
\pgfpathcurveto{\pgfqpoint{1.930711in}{2.083907in}}{\pgfqpoint{1.933983in}{2.076006in}}{\pgfqpoint{1.939807in}{2.070183in}}%
\pgfpathcurveto{\pgfqpoint{1.945631in}{2.064359in}}{\pgfqpoint{1.953531in}{2.061086in}}{\pgfqpoint{1.961767in}{2.061086in}}%
\pgfpathclose%
\pgfusepath{stroke,fill}%
\end{pgfscope}%
\begin{pgfscope}%
\pgfpathrectangle{\pgfqpoint{0.100000in}{0.212622in}}{\pgfqpoint{3.696000in}{3.696000in}}%
\pgfusepath{clip}%
\pgfsetbuttcap%
\pgfsetroundjoin%
\definecolor{currentfill}{rgb}{0.121569,0.466667,0.705882}%
\pgfsetfillcolor{currentfill}%
\pgfsetfillopacity{0.329476}%
\pgfsetlinewidth{1.003750pt}%
\definecolor{currentstroke}{rgb}{0.121569,0.466667,0.705882}%
\pgfsetstrokecolor{currentstroke}%
\pgfsetstrokeopacity{0.329476}%
\pgfsetdash{}{0pt}%
\pgfpathmoveto{\pgfqpoint{1.841109in}{2.020676in}}%
\pgfpathcurveto{\pgfqpoint{1.849346in}{2.020676in}}{\pgfqpoint{1.857246in}{2.023948in}}{\pgfqpoint{1.863070in}{2.029772in}}%
\pgfpathcurveto{\pgfqpoint{1.868894in}{2.035596in}}{\pgfqpoint{1.872166in}{2.043496in}}{\pgfqpoint{1.872166in}{2.051732in}}%
\pgfpathcurveto{\pgfqpoint{1.872166in}{2.059969in}}{\pgfqpoint{1.868894in}{2.067869in}}{\pgfqpoint{1.863070in}{2.073693in}}%
\pgfpathcurveto{\pgfqpoint{1.857246in}{2.079517in}}{\pgfqpoint{1.849346in}{2.082789in}}{\pgfqpoint{1.841109in}{2.082789in}}%
\pgfpathcurveto{\pgfqpoint{1.832873in}{2.082789in}}{\pgfqpoint{1.824973in}{2.079517in}}{\pgfqpoint{1.819149in}{2.073693in}}%
\pgfpathcurveto{\pgfqpoint{1.813325in}{2.067869in}}{\pgfqpoint{1.810053in}{2.059969in}}{\pgfqpoint{1.810053in}{2.051732in}}%
\pgfpathcurveto{\pgfqpoint{1.810053in}{2.043496in}}{\pgfqpoint{1.813325in}{2.035596in}}{\pgfqpoint{1.819149in}{2.029772in}}%
\pgfpathcurveto{\pgfqpoint{1.824973in}{2.023948in}}{\pgfqpoint{1.832873in}{2.020676in}}{\pgfqpoint{1.841109in}{2.020676in}}%
\pgfpathclose%
\pgfusepath{stroke,fill}%
\end{pgfscope}%
\begin{pgfscope}%
\pgfpathrectangle{\pgfqpoint{0.100000in}{0.212622in}}{\pgfqpoint{3.696000in}{3.696000in}}%
\pgfusepath{clip}%
\pgfsetbuttcap%
\pgfsetroundjoin%
\definecolor{currentfill}{rgb}{0.121569,0.466667,0.705882}%
\pgfsetfillcolor{currentfill}%
\pgfsetfillopacity{0.330463}%
\pgfsetlinewidth{1.003750pt}%
\definecolor{currentstroke}{rgb}{0.121569,0.466667,0.705882}%
\pgfsetstrokecolor{currentstroke}%
\pgfsetstrokeopacity{0.330463}%
\pgfsetdash{}{0pt}%
\pgfpathmoveto{\pgfqpoint{1.837794in}{2.018838in}}%
\pgfpathcurveto{\pgfqpoint{1.846030in}{2.018838in}}{\pgfqpoint{1.853931in}{2.022110in}}{\pgfqpoint{1.859754in}{2.027934in}}%
\pgfpathcurveto{\pgfqpoint{1.865578in}{2.033758in}}{\pgfqpoint{1.868851in}{2.041658in}}{\pgfqpoint{1.868851in}{2.049894in}}%
\pgfpathcurveto{\pgfqpoint{1.868851in}{2.058130in}}{\pgfqpoint{1.865578in}{2.066030in}}{\pgfqpoint{1.859754in}{2.071854in}}%
\pgfpathcurveto{\pgfqpoint{1.853931in}{2.077678in}}{\pgfqpoint{1.846030in}{2.080951in}}{\pgfqpoint{1.837794in}{2.080951in}}%
\pgfpathcurveto{\pgfqpoint{1.829558in}{2.080951in}}{\pgfqpoint{1.821658in}{2.077678in}}{\pgfqpoint{1.815834in}{2.071854in}}%
\pgfpathcurveto{\pgfqpoint{1.810010in}{2.066030in}}{\pgfqpoint{1.806738in}{2.058130in}}{\pgfqpoint{1.806738in}{2.049894in}}%
\pgfpathcurveto{\pgfqpoint{1.806738in}{2.041658in}}{\pgfqpoint{1.810010in}{2.033758in}}{\pgfqpoint{1.815834in}{2.027934in}}%
\pgfpathcurveto{\pgfqpoint{1.821658in}{2.022110in}}{\pgfqpoint{1.829558in}{2.018838in}}{\pgfqpoint{1.837794in}{2.018838in}}%
\pgfpathclose%
\pgfusepath{stroke,fill}%
\end{pgfscope}%
\begin{pgfscope}%
\pgfpathrectangle{\pgfqpoint{0.100000in}{0.212622in}}{\pgfqpoint{3.696000in}{3.696000in}}%
\pgfusepath{clip}%
\pgfsetbuttcap%
\pgfsetroundjoin%
\definecolor{currentfill}{rgb}{0.121569,0.466667,0.705882}%
\pgfsetfillcolor{currentfill}%
\pgfsetfillopacity{0.330944}%
\pgfsetlinewidth{1.003750pt}%
\definecolor{currentstroke}{rgb}{0.121569,0.466667,0.705882}%
\pgfsetstrokecolor{currentstroke}%
\pgfsetstrokeopacity{0.330944}%
\pgfsetdash{}{0pt}%
\pgfpathmoveto{\pgfqpoint{1.963043in}{2.061626in}}%
\pgfpathcurveto{\pgfqpoint{1.971279in}{2.061626in}}{\pgfqpoint{1.979179in}{2.064898in}}{\pgfqpoint{1.985003in}{2.070722in}}%
\pgfpathcurveto{\pgfqpoint{1.990827in}{2.076546in}}{\pgfqpoint{1.994100in}{2.084446in}}{\pgfqpoint{1.994100in}{2.092682in}}%
\pgfpathcurveto{\pgfqpoint{1.994100in}{2.100918in}}{\pgfqpoint{1.990827in}{2.108819in}}{\pgfqpoint{1.985003in}{2.114642in}}%
\pgfpathcurveto{\pgfqpoint{1.979179in}{2.120466in}}{\pgfqpoint{1.971279in}{2.123739in}}{\pgfqpoint{1.963043in}{2.123739in}}%
\pgfpathcurveto{\pgfqpoint{1.954807in}{2.123739in}}{\pgfqpoint{1.946907in}{2.120466in}}{\pgfqpoint{1.941083in}{2.114642in}}%
\pgfpathcurveto{\pgfqpoint{1.935259in}{2.108819in}}{\pgfqpoint{1.931987in}{2.100918in}}{\pgfqpoint{1.931987in}{2.092682in}}%
\pgfpathcurveto{\pgfqpoint{1.931987in}{2.084446in}}{\pgfqpoint{1.935259in}{2.076546in}}{\pgfqpoint{1.941083in}{2.070722in}}%
\pgfpathcurveto{\pgfqpoint{1.946907in}{2.064898in}}{\pgfqpoint{1.954807in}{2.061626in}}{\pgfqpoint{1.963043in}{2.061626in}}%
\pgfpathclose%
\pgfusepath{stroke,fill}%
\end{pgfscope}%
\begin{pgfscope}%
\pgfpathrectangle{\pgfqpoint{0.100000in}{0.212622in}}{\pgfqpoint{3.696000in}{3.696000in}}%
\pgfusepath{clip}%
\pgfsetbuttcap%
\pgfsetroundjoin%
\definecolor{currentfill}{rgb}{0.121569,0.466667,0.705882}%
\pgfsetfillcolor{currentfill}%
\pgfsetfillopacity{0.331419}%
\pgfsetlinewidth{1.003750pt}%
\definecolor{currentstroke}{rgb}{0.121569,0.466667,0.705882}%
\pgfsetstrokecolor{currentstroke}%
\pgfsetstrokeopacity{0.331419}%
\pgfsetdash{}{0pt}%
\pgfpathmoveto{\pgfqpoint{1.835078in}{2.017988in}}%
\pgfpathcurveto{\pgfqpoint{1.843314in}{2.017988in}}{\pgfqpoint{1.851214in}{2.021260in}}{\pgfqpoint{1.857038in}{2.027084in}}%
\pgfpathcurveto{\pgfqpoint{1.862862in}{2.032908in}}{\pgfqpoint{1.866134in}{2.040808in}}{\pgfqpoint{1.866134in}{2.049045in}}%
\pgfpathcurveto{\pgfqpoint{1.866134in}{2.057281in}}{\pgfqpoint{1.862862in}{2.065181in}}{\pgfqpoint{1.857038in}{2.071005in}}%
\pgfpathcurveto{\pgfqpoint{1.851214in}{2.076829in}}{\pgfqpoint{1.843314in}{2.080101in}}{\pgfqpoint{1.835078in}{2.080101in}}%
\pgfpathcurveto{\pgfqpoint{1.826842in}{2.080101in}}{\pgfqpoint{1.818942in}{2.076829in}}{\pgfqpoint{1.813118in}{2.071005in}}%
\pgfpathcurveto{\pgfqpoint{1.807294in}{2.065181in}}{\pgfqpoint{1.804021in}{2.057281in}}{\pgfqpoint{1.804021in}{2.049045in}}%
\pgfpathcurveto{\pgfqpoint{1.804021in}{2.040808in}}{\pgfqpoint{1.807294in}{2.032908in}}{\pgfqpoint{1.813118in}{2.027084in}}%
\pgfpathcurveto{\pgfqpoint{1.818942in}{2.021260in}}{\pgfqpoint{1.826842in}{2.017988in}}{\pgfqpoint{1.835078in}{2.017988in}}%
\pgfpathclose%
\pgfusepath{stroke,fill}%
\end{pgfscope}%
\begin{pgfscope}%
\pgfpathrectangle{\pgfqpoint{0.100000in}{0.212622in}}{\pgfqpoint{3.696000in}{3.696000in}}%
\pgfusepath{clip}%
\pgfsetbuttcap%
\pgfsetroundjoin%
\definecolor{currentfill}{rgb}{0.121569,0.466667,0.705882}%
\pgfsetfillcolor{currentfill}%
\pgfsetfillopacity{0.332800}%
\pgfsetlinewidth{1.003750pt}%
\definecolor{currentstroke}{rgb}{0.121569,0.466667,0.705882}%
\pgfsetstrokecolor{currentstroke}%
\pgfsetstrokeopacity{0.332800}%
\pgfsetdash{}{0pt}%
\pgfpathmoveto{\pgfqpoint{1.964748in}{2.060557in}}%
\pgfpathcurveto{\pgfqpoint{1.972985in}{2.060557in}}{\pgfqpoint{1.980885in}{2.063829in}}{\pgfqpoint{1.986709in}{2.069653in}}%
\pgfpathcurveto{\pgfqpoint{1.992533in}{2.075477in}}{\pgfqpoint{1.995805in}{2.083377in}}{\pgfqpoint{1.995805in}{2.091613in}}%
\pgfpathcurveto{\pgfqpoint{1.995805in}{2.099850in}}{\pgfqpoint{1.992533in}{2.107750in}}{\pgfqpoint{1.986709in}{2.113574in}}%
\pgfpathcurveto{\pgfqpoint{1.980885in}{2.119397in}}{\pgfqpoint{1.972985in}{2.122670in}}{\pgfqpoint{1.964748in}{2.122670in}}%
\pgfpathcurveto{\pgfqpoint{1.956512in}{2.122670in}}{\pgfqpoint{1.948612in}{2.119397in}}{\pgfqpoint{1.942788in}{2.113574in}}%
\pgfpathcurveto{\pgfqpoint{1.936964in}{2.107750in}}{\pgfqpoint{1.933692in}{2.099850in}}{\pgfqpoint{1.933692in}{2.091613in}}%
\pgfpathcurveto{\pgfqpoint{1.933692in}{2.083377in}}{\pgfqpoint{1.936964in}{2.075477in}}{\pgfqpoint{1.942788in}{2.069653in}}%
\pgfpathcurveto{\pgfqpoint{1.948612in}{2.063829in}}{\pgfqpoint{1.956512in}{2.060557in}}{\pgfqpoint{1.964748in}{2.060557in}}%
\pgfpathclose%
\pgfusepath{stroke,fill}%
\end{pgfscope}%
\begin{pgfscope}%
\pgfpathrectangle{\pgfqpoint{0.100000in}{0.212622in}}{\pgfqpoint{3.696000in}{3.696000in}}%
\pgfusepath{clip}%
\pgfsetbuttcap%
\pgfsetroundjoin%
\definecolor{currentfill}{rgb}{0.121569,0.466667,0.705882}%
\pgfsetfillcolor{currentfill}%
\pgfsetfillopacity{0.332828}%
\pgfsetlinewidth{1.003750pt}%
\definecolor{currentstroke}{rgb}{0.121569,0.466667,0.705882}%
\pgfsetstrokecolor{currentstroke}%
\pgfsetstrokeopacity{0.332828}%
\pgfsetdash{}{0pt}%
\pgfpathmoveto{\pgfqpoint{1.830297in}{2.013941in}}%
\pgfpathcurveto{\pgfqpoint{1.838534in}{2.013941in}}{\pgfqpoint{1.846434in}{2.017213in}}{\pgfqpoint{1.852258in}{2.023037in}}%
\pgfpathcurveto{\pgfqpoint{1.858081in}{2.028861in}}{\pgfqpoint{1.861354in}{2.036761in}}{\pgfqpoint{1.861354in}{2.044997in}}%
\pgfpathcurveto{\pgfqpoint{1.861354in}{2.053234in}}{\pgfqpoint{1.858081in}{2.061134in}}{\pgfqpoint{1.852258in}{2.066958in}}%
\pgfpathcurveto{\pgfqpoint{1.846434in}{2.072782in}}{\pgfqpoint{1.838534in}{2.076054in}}{\pgfqpoint{1.830297in}{2.076054in}}%
\pgfpathcurveto{\pgfqpoint{1.822061in}{2.076054in}}{\pgfqpoint{1.814161in}{2.072782in}}{\pgfqpoint{1.808337in}{2.066958in}}%
\pgfpathcurveto{\pgfqpoint{1.802513in}{2.061134in}}{\pgfqpoint{1.799241in}{2.053234in}}{\pgfqpoint{1.799241in}{2.044997in}}%
\pgfpathcurveto{\pgfqpoint{1.799241in}{2.036761in}}{\pgfqpoint{1.802513in}{2.028861in}}{\pgfqpoint{1.808337in}{2.023037in}}%
\pgfpathcurveto{\pgfqpoint{1.814161in}{2.017213in}}{\pgfqpoint{1.822061in}{2.013941in}}{\pgfqpoint{1.830297in}{2.013941in}}%
\pgfpathclose%
\pgfusepath{stroke,fill}%
\end{pgfscope}%
\begin{pgfscope}%
\pgfpathrectangle{\pgfqpoint{0.100000in}{0.212622in}}{\pgfqpoint{3.696000in}{3.696000in}}%
\pgfusepath{clip}%
\pgfsetbuttcap%
\pgfsetroundjoin%
\definecolor{currentfill}{rgb}{0.121569,0.466667,0.705882}%
\pgfsetfillcolor{currentfill}%
\pgfsetfillopacity{0.333954}%
\pgfsetlinewidth{1.003750pt}%
\definecolor{currentstroke}{rgb}{0.121569,0.466667,0.705882}%
\pgfsetstrokecolor{currentstroke}%
\pgfsetstrokeopacity{0.333954}%
\pgfsetdash{}{0pt}%
\pgfpathmoveto{\pgfqpoint{1.826652in}{2.012904in}}%
\pgfpathcurveto{\pgfqpoint{1.834888in}{2.012904in}}{\pgfqpoint{1.842788in}{2.016176in}}{\pgfqpoint{1.848612in}{2.022000in}}%
\pgfpathcurveto{\pgfqpoint{1.854436in}{2.027824in}}{\pgfqpoint{1.857708in}{2.035724in}}{\pgfqpoint{1.857708in}{2.043960in}}%
\pgfpathcurveto{\pgfqpoint{1.857708in}{2.052197in}}{\pgfqpoint{1.854436in}{2.060097in}}{\pgfqpoint{1.848612in}{2.065920in}}%
\pgfpathcurveto{\pgfqpoint{1.842788in}{2.071744in}}{\pgfqpoint{1.834888in}{2.075017in}}{\pgfqpoint{1.826652in}{2.075017in}}%
\pgfpathcurveto{\pgfqpoint{1.818415in}{2.075017in}}{\pgfqpoint{1.810515in}{2.071744in}}{\pgfqpoint{1.804691in}{2.065920in}}%
\pgfpathcurveto{\pgfqpoint{1.798867in}{2.060097in}}{\pgfqpoint{1.795595in}{2.052197in}}{\pgfqpoint{1.795595in}{2.043960in}}%
\pgfpathcurveto{\pgfqpoint{1.795595in}{2.035724in}}{\pgfqpoint{1.798867in}{2.027824in}}{\pgfqpoint{1.804691in}{2.022000in}}%
\pgfpathcurveto{\pgfqpoint{1.810515in}{2.016176in}}{\pgfqpoint{1.818415in}{2.012904in}}{\pgfqpoint{1.826652in}{2.012904in}}%
\pgfpathclose%
\pgfusepath{stroke,fill}%
\end{pgfscope}%
\begin{pgfscope}%
\pgfpathrectangle{\pgfqpoint{0.100000in}{0.212622in}}{\pgfqpoint{3.696000in}{3.696000in}}%
\pgfusepath{clip}%
\pgfsetbuttcap%
\pgfsetroundjoin%
\definecolor{currentfill}{rgb}{0.121569,0.466667,0.705882}%
\pgfsetfillcolor{currentfill}%
\pgfsetfillopacity{0.334911}%
\pgfsetlinewidth{1.003750pt}%
\definecolor{currentstroke}{rgb}{0.121569,0.466667,0.705882}%
\pgfsetstrokecolor{currentstroke}%
\pgfsetstrokeopacity{0.334911}%
\pgfsetdash{}{0pt}%
\pgfpathmoveto{\pgfqpoint{1.824062in}{2.011765in}}%
\pgfpathcurveto{\pgfqpoint{1.832298in}{2.011765in}}{\pgfqpoint{1.840198in}{2.015037in}}{\pgfqpoint{1.846022in}{2.020861in}}%
\pgfpathcurveto{\pgfqpoint{1.851846in}{2.026685in}}{\pgfqpoint{1.855118in}{2.034585in}}{\pgfqpoint{1.855118in}{2.042822in}}%
\pgfpathcurveto{\pgfqpoint{1.855118in}{2.051058in}}{\pgfqpoint{1.851846in}{2.058958in}}{\pgfqpoint{1.846022in}{2.064782in}}%
\pgfpathcurveto{\pgfqpoint{1.840198in}{2.070606in}}{\pgfqpoint{1.832298in}{2.073878in}}{\pgfqpoint{1.824062in}{2.073878in}}%
\pgfpathcurveto{\pgfqpoint{1.815825in}{2.073878in}}{\pgfqpoint{1.807925in}{2.070606in}}{\pgfqpoint{1.802101in}{2.064782in}}%
\pgfpathcurveto{\pgfqpoint{1.796277in}{2.058958in}}{\pgfqpoint{1.793005in}{2.051058in}}{\pgfqpoint{1.793005in}{2.042822in}}%
\pgfpathcurveto{\pgfqpoint{1.793005in}{2.034585in}}{\pgfqpoint{1.796277in}{2.026685in}}{\pgfqpoint{1.802101in}{2.020861in}}%
\pgfpathcurveto{\pgfqpoint{1.807925in}{2.015037in}}{\pgfqpoint{1.815825in}{2.011765in}}{\pgfqpoint{1.824062in}{2.011765in}}%
\pgfpathclose%
\pgfusepath{stroke,fill}%
\end{pgfscope}%
\begin{pgfscope}%
\pgfpathrectangle{\pgfqpoint{0.100000in}{0.212622in}}{\pgfqpoint{3.696000in}{3.696000in}}%
\pgfusepath{clip}%
\pgfsetbuttcap%
\pgfsetroundjoin%
\definecolor{currentfill}{rgb}{0.121569,0.466667,0.705882}%
\pgfsetfillcolor{currentfill}%
\pgfsetfillopacity{0.335153}%
\pgfsetlinewidth{1.003750pt}%
\definecolor{currentstroke}{rgb}{0.121569,0.466667,0.705882}%
\pgfsetstrokecolor{currentstroke}%
\pgfsetstrokeopacity{0.335153}%
\pgfsetdash{}{0pt}%
\pgfpathmoveto{\pgfqpoint{1.966845in}{2.059560in}}%
\pgfpathcurveto{\pgfqpoint{1.975081in}{2.059560in}}{\pgfqpoint{1.982981in}{2.062833in}}{\pgfqpoint{1.988805in}{2.068657in}}%
\pgfpathcurveto{\pgfqpoint{1.994629in}{2.074481in}}{\pgfqpoint{1.997901in}{2.082381in}}{\pgfqpoint{1.997901in}{2.090617in}}%
\pgfpathcurveto{\pgfqpoint{1.997901in}{2.098853in}}{\pgfqpoint{1.994629in}{2.106753in}}{\pgfqpoint{1.988805in}{2.112577in}}%
\pgfpathcurveto{\pgfqpoint{1.982981in}{2.118401in}}{\pgfqpoint{1.975081in}{2.121673in}}{\pgfqpoint{1.966845in}{2.121673in}}%
\pgfpathcurveto{\pgfqpoint{1.958608in}{2.121673in}}{\pgfqpoint{1.950708in}{2.118401in}}{\pgfqpoint{1.944884in}{2.112577in}}%
\pgfpathcurveto{\pgfqpoint{1.939060in}{2.106753in}}{\pgfqpoint{1.935788in}{2.098853in}}{\pgfqpoint{1.935788in}{2.090617in}}%
\pgfpathcurveto{\pgfqpoint{1.935788in}{2.082381in}}{\pgfqpoint{1.939060in}{2.074481in}}{\pgfqpoint{1.944884in}{2.068657in}}%
\pgfpathcurveto{\pgfqpoint{1.950708in}{2.062833in}}{\pgfqpoint{1.958608in}{2.059560in}}{\pgfqpoint{1.966845in}{2.059560in}}%
\pgfpathclose%
\pgfusepath{stroke,fill}%
\end{pgfscope}%
\begin{pgfscope}%
\pgfpathrectangle{\pgfqpoint{0.100000in}{0.212622in}}{\pgfqpoint{3.696000in}{3.696000in}}%
\pgfusepath{clip}%
\pgfsetbuttcap%
\pgfsetroundjoin%
\definecolor{currentfill}{rgb}{0.121569,0.466667,0.705882}%
\pgfsetfillcolor{currentfill}%
\pgfsetfillopacity{0.336188}%
\pgfsetlinewidth{1.003750pt}%
\definecolor{currentstroke}{rgb}{0.121569,0.466667,0.705882}%
\pgfsetstrokecolor{currentstroke}%
\pgfsetstrokeopacity{0.336188}%
\pgfsetdash{}{0pt}%
\pgfpathmoveto{\pgfqpoint{1.819117in}{2.006827in}}%
\pgfpathcurveto{\pgfqpoint{1.827354in}{2.006827in}}{\pgfqpoint{1.835254in}{2.010100in}}{\pgfqpoint{1.841078in}{2.015924in}}%
\pgfpathcurveto{\pgfqpoint{1.846902in}{2.021747in}}{\pgfqpoint{1.850174in}{2.029647in}}{\pgfqpoint{1.850174in}{2.037884in}}%
\pgfpathcurveto{\pgfqpoint{1.850174in}{2.046120in}}{\pgfqpoint{1.846902in}{2.054020in}}{\pgfqpoint{1.841078in}{2.059844in}}%
\pgfpathcurveto{\pgfqpoint{1.835254in}{2.065668in}}{\pgfqpoint{1.827354in}{2.068940in}}{\pgfqpoint{1.819117in}{2.068940in}}%
\pgfpathcurveto{\pgfqpoint{1.810881in}{2.068940in}}{\pgfqpoint{1.802981in}{2.065668in}}{\pgfqpoint{1.797157in}{2.059844in}}%
\pgfpathcurveto{\pgfqpoint{1.791333in}{2.054020in}}{\pgfqpoint{1.788061in}{2.046120in}}{\pgfqpoint{1.788061in}{2.037884in}}%
\pgfpathcurveto{\pgfqpoint{1.788061in}{2.029647in}}{\pgfqpoint{1.791333in}{2.021747in}}{\pgfqpoint{1.797157in}{2.015924in}}%
\pgfpathcurveto{\pgfqpoint{1.802981in}{2.010100in}}{\pgfqpoint{1.810881in}{2.006827in}}{\pgfqpoint{1.819117in}{2.006827in}}%
\pgfpathclose%
\pgfusepath{stroke,fill}%
\end{pgfscope}%
\begin{pgfscope}%
\pgfpathrectangle{\pgfqpoint{0.100000in}{0.212622in}}{\pgfqpoint{3.696000in}{3.696000in}}%
\pgfusepath{clip}%
\pgfsetbuttcap%
\pgfsetroundjoin%
\definecolor{currentfill}{rgb}{0.121569,0.466667,0.705882}%
\pgfsetfillcolor{currentfill}%
\pgfsetfillopacity{0.337259}%
\pgfsetlinewidth{1.003750pt}%
\definecolor{currentstroke}{rgb}{0.121569,0.466667,0.705882}%
\pgfsetstrokecolor{currentstroke}%
\pgfsetstrokeopacity{0.337259}%
\pgfsetdash{}{0pt}%
\pgfpathmoveto{\pgfqpoint{1.815596in}{2.004913in}}%
\pgfpathcurveto{\pgfqpoint{1.823833in}{2.004913in}}{\pgfqpoint{1.831733in}{2.008185in}}{\pgfqpoint{1.837556in}{2.014009in}}%
\pgfpathcurveto{\pgfqpoint{1.843380in}{2.019833in}}{\pgfqpoint{1.846653in}{2.027733in}}{\pgfqpoint{1.846653in}{2.035969in}}%
\pgfpathcurveto{\pgfqpoint{1.846653in}{2.044205in}}{\pgfqpoint{1.843380in}{2.052105in}}{\pgfqpoint{1.837556in}{2.057929in}}%
\pgfpathcurveto{\pgfqpoint{1.831733in}{2.063753in}}{\pgfqpoint{1.823833in}{2.067026in}}{\pgfqpoint{1.815596in}{2.067026in}}%
\pgfpathcurveto{\pgfqpoint{1.807360in}{2.067026in}}{\pgfqpoint{1.799460in}{2.063753in}}{\pgfqpoint{1.793636in}{2.057929in}}%
\pgfpathcurveto{\pgfqpoint{1.787812in}{2.052105in}}{\pgfqpoint{1.784540in}{2.044205in}}{\pgfqpoint{1.784540in}{2.035969in}}%
\pgfpathcurveto{\pgfqpoint{1.784540in}{2.027733in}}{\pgfqpoint{1.787812in}{2.019833in}}{\pgfqpoint{1.793636in}{2.014009in}}%
\pgfpathcurveto{\pgfqpoint{1.799460in}{2.008185in}}{\pgfqpoint{1.807360in}{2.004913in}}{\pgfqpoint{1.815596in}{2.004913in}}%
\pgfpathclose%
\pgfusepath{stroke,fill}%
\end{pgfscope}%
\begin{pgfscope}%
\pgfpathrectangle{\pgfqpoint{0.100000in}{0.212622in}}{\pgfqpoint{3.696000in}{3.696000in}}%
\pgfusepath{clip}%
\pgfsetbuttcap%
\pgfsetroundjoin%
\definecolor{currentfill}{rgb}{0.121569,0.466667,0.705882}%
\pgfsetfillcolor{currentfill}%
\pgfsetfillopacity{0.338073}%
\pgfsetlinewidth{1.003750pt}%
\definecolor{currentstroke}{rgb}{0.121569,0.466667,0.705882}%
\pgfsetstrokecolor{currentstroke}%
\pgfsetstrokeopacity{0.338073}%
\pgfsetdash{}{0pt}%
\pgfpathmoveto{\pgfqpoint{1.813156in}{2.003267in}}%
\pgfpathcurveto{\pgfqpoint{1.821392in}{2.003267in}}{\pgfqpoint{1.829292in}{2.006539in}}{\pgfqpoint{1.835116in}{2.012363in}}%
\pgfpathcurveto{\pgfqpoint{1.840940in}{2.018187in}}{\pgfqpoint{1.844212in}{2.026087in}}{\pgfqpoint{1.844212in}{2.034324in}}%
\pgfpathcurveto{\pgfqpoint{1.844212in}{2.042560in}}{\pgfqpoint{1.840940in}{2.050460in}}{\pgfqpoint{1.835116in}{2.056284in}}%
\pgfpathcurveto{\pgfqpoint{1.829292in}{2.062108in}}{\pgfqpoint{1.821392in}{2.065380in}}{\pgfqpoint{1.813156in}{2.065380in}}%
\pgfpathcurveto{\pgfqpoint{1.804919in}{2.065380in}}{\pgfqpoint{1.797019in}{2.062108in}}{\pgfqpoint{1.791196in}{2.056284in}}%
\pgfpathcurveto{\pgfqpoint{1.785372in}{2.050460in}}{\pgfqpoint{1.782099in}{2.042560in}}{\pgfqpoint{1.782099in}{2.034324in}}%
\pgfpathcurveto{\pgfqpoint{1.782099in}{2.026087in}}{\pgfqpoint{1.785372in}{2.018187in}}{\pgfqpoint{1.791196in}{2.012363in}}%
\pgfpathcurveto{\pgfqpoint{1.797019in}{2.006539in}}{\pgfqpoint{1.804919in}{2.003267in}}{\pgfqpoint{1.813156in}{2.003267in}}%
\pgfpathclose%
\pgfusepath{stroke,fill}%
\end{pgfscope}%
\begin{pgfscope}%
\pgfpathrectangle{\pgfqpoint{0.100000in}{0.212622in}}{\pgfqpoint{3.696000in}{3.696000in}}%
\pgfusepath{clip}%
\pgfsetbuttcap%
\pgfsetroundjoin%
\definecolor{currentfill}{rgb}{0.121569,0.466667,0.705882}%
\pgfsetfillcolor{currentfill}%
\pgfsetfillopacity{0.338608}%
\pgfsetlinewidth{1.003750pt}%
\definecolor{currentstroke}{rgb}{0.121569,0.466667,0.705882}%
\pgfsetstrokecolor{currentstroke}%
\pgfsetstrokeopacity{0.338608}%
\pgfsetdash{}{0pt}%
\pgfpathmoveto{\pgfqpoint{1.969176in}{2.060662in}}%
\pgfpathcurveto{\pgfqpoint{1.977412in}{2.060662in}}{\pgfqpoint{1.985312in}{2.063935in}}{\pgfqpoint{1.991136in}{2.069759in}}%
\pgfpathcurveto{\pgfqpoint{1.996960in}{2.075583in}}{\pgfqpoint{2.000233in}{2.083483in}}{\pgfqpoint{2.000233in}{2.091719in}}%
\pgfpathcurveto{\pgfqpoint{2.000233in}{2.099955in}}{\pgfqpoint{1.996960in}{2.107855in}}{\pgfqpoint{1.991136in}{2.113679in}}%
\pgfpathcurveto{\pgfqpoint{1.985312in}{2.119503in}}{\pgfqpoint{1.977412in}{2.122775in}}{\pgfqpoint{1.969176in}{2.122775in}}%
\pgfpathcurveto{\pgfqpoint{1.960940in}{2.122775in}}{\pgfqpoint{1.953040in}{2.119503in}}{\pgfqpoint{1.947216in}{2.113679in}}%
\pgfpathcurveto{\pgfqpoint{1.941392in}{2.107855in}}{\pgfqpoint{1.938120in}{2.099955in}}{\pgfqpoint{1.938120in}{2.091719in}}%
\pgfpathcurveto{\pgfqpoint{1.938120in}{2.083483in}}{\pgfqpoint{1.941392in}{2.075583in}}{\pgfqpoint{1.947216in}{2.069759in}}%
\pgfpathcurveto{\pgfqpoint{1.953040in}{2.063935in}}{\pgfqpoint{1.960940in}{2.060662in}}{\pgfqpoint{1.969176in}{2.060662in}}%
\pgfpathclose%
\pgfusepath{stroke,fill}%
\end{pgfscope}%
\begin{pgfscope}%
\pgfpathrectangle{\pgfqpoint{0.100000in}{0.212622in}}{\pgfqpoint{3.696000in}{3.696000in}}%
\pgfusepath{clip}%
\pgfsetbuttcap%
\pgfsetroundjoin%
\definecolor{currentfill}{rgb}{0.121569,0.466667,0.705882}%
\pgfsetfillcolor{currentfill}%
\pgfsetfillopacity{0.339417}%
\pgfsetlinewidth{1.003750pt}%
\definecolor{currentstroke}{rgb}{0.121569,0.466667,0.705882}%
\pgfsetstrokecolor{currentstroke}%
\pgfsetstrokeopacity{0.339417}%
\pgfsetdash{}{0pt}%
\pgfpathmoveto{\pgfqpoint{1.809188in}{1.998757in}}%
\pgfpathcurveto{\pgfqpoint{1.817425in}{1.998757in}}{\pgfqpoint{1.825325in}{2.002030in}}{\pgfqpoint{1.831149in}{2.007854in}}%
\pgfpathcurveto{\pgfqpoint{1.836972in}{2.013677in}}{\pgfqpoint{1.840245in}{2.021578in}}{\pgfqpoint{1.840245in}{2.029814in}}%
\pgfpathcurveto{\pgfqpoint{1.840245in}{2.038050in}}{\pgfqpoint{1.836972in}{2.045950in}}{\pgfqpoint{1.831149in}{2.051774in}}%
\pgfpathcurveto{\pgfqpoint{1.825325in}{2.057598in}}{\pgfqpoint{1.817425in}{2.060870in}}{\pgfqpoint{1.809188in}{2.060870in}}%
\pgfpathcurveto{\pgfqpoint{1.800952in}{2.060870in}}{\pgfqpoint{1.793052in}{2.057598in}}{\pgfqpoint{1.787228in}{2.051774in}}%
\pgfpathcurveto{\pgfqpoint{1.781404in}{2.045950in}}{\pgfqpoint{1.778132in}{2.038050in}}{\pgfqpoint{1.778132in}{2.029814in}}%
\pgfpathcurveto{\pgfqpoint{1.778132in}{2.021578in}}{\pgfqpoint{1.781404in}{2.013677in}}{\pgfqpoint{1.787228in}{2.007854in}}%
\pgfpathcurveto{\pgfqpoint{1.793052in}{2.002030in}}{\pgfqpoint{1.800952in}{1.998757in}}{\pgfqpoint{1.809188in}{1.998757in}}%
\pgfpathclose%
\pgfusepath{stroke,fill}%
\end{pgfscope}%
\begin{pgfscope}%
\pgfpathrectangle{\pgfqpoint{0.100000in}{0.212622in}}{\pgfqpoint{3.696000in}{3.696000in}}%
\pgfusepath{clip}%
\pgfsetbuttcap%
\pgfsetroundjoin%
\definecolor{currentfill}{rgb}{0.121569,0.466667,0.705882}%
\pgfsetfillcolor{currentfill}%
\pgfsetfillopacity{0.340315}%
\pgfsetlinewidth{1.003750pt}%
\definecolor{currentstroke}{rgb}{0.121569,0.466667,0.705882}%
\pgfsetstrokecolor{currentstroke}%
\pgfsetstrokeopacity{0.340315}%
\pgfsetdash{}{0pt}%
\pgfpathmoveto{\pgfqpoint{1.806341in}{1.997900in}}%
\pgfpathcurveto{\pgfqpoint{1.814577in}{1.997900in}}{\pgfqpoint{1.822477in}{2.001173in}}{\pgfqpoint{1.828301in}{2.006996in}}%
\pgfpathcurveto{\pgfqpoint{1.834125in}{2.012820in}}{\pgfqpoint{1.837397in}{2.020720in}}{\pgfqpoint{1.837397in}{2.028957in}}%
\pgfpathcurveto{\pgfqpoint{1.837397in}{2.037193in}}{\pgfqpoint{1.834125in}{2.045093in}}{\pgfqpoint{1.828301in}{2.050917in}}%
\pgfpathcurveto{\pgfqpoint{1.822477in}{2.056741in}}{\pgfqpoint{1.814577in}{2.060013in}}{\pgfqpoint{1.806341in}{2.060013in}}%
\pgfpathcurveto{\pgfqpoint{1.798104in}{2.060013in}}{\pgfqpoint{1.790204in}{2.056741in}}{\pgfqpoint{1.784380in}{2.050917in}}%
\pgfpathcurveto{\pgfqpoint{1.778556in}{2.045093in}}{\pgfqpoint{1.775284in}{2.037193in}}{\pgfqpoint{1.775284in}{2.028957in}}%
\pgfpathcurveto{\pgfqpoint{1.775284in}{2.020720in}}{\pgfqpoint{1.778556in}{2.012820in}}{\pgfqpoint{1.784380in}{2.006996in}}%
\pgfpathcurveto{\pgfqpoint{1.790204in}{2.001173in}}{\pgfqpoint{1.798104in}{1.997900in}}{\pgfqpoint{1.806341in}{1.997900in}}%
\pgfpathclose%
\pgfusepath{stroke,fill}%
\end{pgfscope}%
\begin{pgfscope}%
\pgfpathrectangle{\pgfqpoint{0.100000in}{0.212622in}}{\pgfqpoint{3.696000in}{3.696000in}}%
\pgfusepath{clip}%
\pgfsetbuttcap%
\pgfsetroundjoin%
\definecolor{currentfill}{rgb}{0.121569,0.466667,0.705882}%
\pgfsetfillcolor{currentfill}%
\pgfsetfillopacity{0.341842}%
\pgfsetlinewidth{1.003750pt}%
\definecolor{currentstroke}{rgb}{0.121569,0.466667,0.705882}%
\pgfsetstrokecolor{currentstroke}%
\pgfsetstrokeopacity{0.341842}%
\pgfsetdash{}{0pt}%
\pgfpathmoveto{\pgfqpoint{1.801575in}{1.994972in}}%
\pgfpathcurveto{\pgfqpoint{1.809812in}{1.994972in}}{\pgfqpoint{1.817712in}{1.998244in}}{\pgfqpoint{1.823536in}{2.004068in}}%
\pgfpathcurveto{\pgfqpoint{1.829359in}{2.009892in}}{\pgfqpoint{1.832632in}{2.017792in}}{\pgfqpoint{1.832632in}{2.026028in}}%
\pgfpathcurveto{\pgfqpoint{1.832632in}{2.034264in}}{\pgfqpoint{1.829359in}{2.042164in}}{\pgfqpoint{1.823536in}{2.047988in}}%
\pgfpathcurveto{\pgfqpoint{1.817712in}{2.053812in}}{\pgfqpoint{1.809812in}{2.057085in}}{\pgfqpoint{1.801575in}{2.057085in}}%
\pgfpathcurveto{\pgfqpoint{1.793339in}{2.057085in}}{\pgfqpoint{1.785439in}{2.053812in}}{\pgfqpoint{1.779615in}{2.047988in}}%
\pgfpathcurveto{\pgfqpoint{1.773791in}{2.042164in}}{\pgfqpoint{1.770519in}{2.034264in}}{\pgfqpoint{1.770519in}{2.026028in}}%
\pgfpathcurveto{\pgfqpoint{1.770519in}{2.017792in}}{\pgfqpoint{1.773791in}{2.009892in}}{\pgfqpoint{1.779615in}{2.004068in}}%
\pgfpathcurveto{\pgfqpoint{1.785439in}{1.998244in}}{\pgfqpoint{1.793339in}{1.994972in}}{\pgfqpoint{1.801575in}{1.994972in}}%
\pgfpathclose%
\pgfusepath{stroke,fill}%
\end{pgfscope}%
\begin{pgfscope}%
\pgfpathrectangle{\pgfqpoint{0.100000in}{0.212622in}}{\pgfqpoint{3.696000in}{3.696000in}}%
\pgfusepath{clip}%
\pgfsetbuttcap%
\pgfsetroundjoin%
\definecolor{currentfill}{rgb}{0.121569,0.466667,0.705882}%
\pgfsetfillcolor{currentfill}%
\pgfsetfillopacity{0.341935}%
\pgfsetlinewidth{1.003750pt}%
\definecolor{currentstroke}{rgb}{0.121569,0.466667,0.705882}%
\pgfsetstrokecolor{currentstroke}%
\pgfsetstrokeopacity{0.341935}%
\pgfsetdash{}{0pt}%
\pgfpathmoveto{\pgfqpoint{1.972271in}{2.059086in}}%
\pgfpathcurveto{\pgfqpoint{1.980507in}{2.059086in}}{\pgfqpoint{1.988407in}{2.062359in}}{\pgfqpoint{1.994231in}{2.068183in}}%
\pgfpathcurveto{\pgfqpoint{2.000055in}{2.074007in}}{\pgfqpoint{2.003327in}{2.081907in}}{\pgfqpoint{2.003327in}{2.090143in}}%
\pgfpathcurveto{\pgfqpoint{2.003327in}{2.098379in}}{\pgfqpoint{2.000055in}{2.106279in}}{\pgfqpoint{1.994231in}{2.112103in}}%
\pgfpathcurveto{\pgfqpoint{1.988407in}{2.117927in}}{\pgfqpoint{1.980507in}{2.121199in}}{\pgfqpoint{1.972271in}{2.121199in}}%
\pgfpathcurveto{\pgfqpoint{1.964035in}{2.121199in}}{\pgfqpoint{1.956135in}{2.117927in}}{\pgfqpoint{1.950311in}{2.112103in}}%
\pgfpathcurveto{\pgfqpoint{1.944487in}{2.106279in}}{\pgfqpoint{1.941214in}{2.098379in}}{\pgfqpoint{1.941214in}{2.090143in}}%
\pgfpathcurveto{\pgfqpoint{1.941214in}{2.081907in}}{\pgfqpoint{1.944487in}{2.074007in}}{\pgfqpoint{1.950311in}{2.068183in}}%
\pgfpathcurveto{\pgfqpoint{1.956135in}{2.062359in}}{\pgfqpoint{1.964035in}{2.059086in}}{\pgfqpoint{1.972271in}{2.059086in}}%
\pgfpathclose%
\pgfusepath{stroke,fill}%
\end{pgfscope}%
\begin{pgfscope}%
\pgfpathrectangle{\pgfqpoint{0.100000in}{0.212622in}}{\pgfqpoint{3.696000in}{3.696000in}}%
\pgfusepath{clip}%
\pgfsetbuttcap%
\pgfsetroundjoin%
\definecolor{currentfill}{rgb}{0.121569,0.466667,0.705882}%
\pgfsetfillcolor{currentfill}%
\pgfsetfillopacity{0.344152}%
\pgfsetlinewidth{1.003750pt}%
\definecolor{currentstroke}{rgb}{0.121569,0.466667,0.705882}%
\pgfsetstrokecolor{currentstroke}%
\pgfsetstrokeopacity{0.344152}%
\pgfsetdash{}{0pt}%
\pgfpathmoveto{\pgfqpoint{1.793970in}{1.985044in}}%
\pgfpathcurveto{\pgfqpoint{1.802207in}{1.985044in}}{\pgfqpoint{1.810107in}{1.988317in}}{\pgfqpoint{1.815931in}{1.994140in}}%
\pgfpathcurveto{\pgfqpoint{1.821755in}{1.999964in}}{\pgfqpoint{1.825027in}{2.007864in}}{\pgfqpoint{1.825027in}{2.016101in}}%
\pgfpathcurveto{\pgfqpoint{1.825027in}{2.024337in}}{\pgfqpoint{1.821755in}{2.032237in}}{\pgfqpoint{1.815931in}{2.038061in}}%
\pgfpathcurveto{\pgfqpoint{1.810107in}{2.043885in}}{\pgfqpoint{1.802207in}{2.047157in}}{\pgfqpoint{1.793970in}{2.047157in}}%
\pgfpathcurveto{\pgfqpoint{1.785734in}{2.047157in}}{\pgfqpoint{1.777834in}{2.043885in}}{\pgfqpoint{1.772010in}{2.038061in}}%
\pgfpathcurveto{\pgfqpoint{1.766186in}{2.032237in}}{\pgfqpoint{1.762914in}{2.024337in}}{\pgfqpoint{1.762914in}{2.016101in}}%
\pgfpathcurveto{\pgfqpoint{1.762914in}{2.007864in}}{\pgfqpoint{1.766186in}{1.999964in}}{\pgfqpoint{1.772010in}{1.994140in}}%
\pgfpathcurveto{\pgfqpoint{1.777834in}{1.988317in}}{\pgfqpoint{1.785734in}{1.985044in}}{\pgfqpoint{1.793970in}{1.985044in}}%
\pgfpathclose%
\pgfusepath{stroke,fill}%
\end{pgfscope}%
\begin{pgfscope}%
\pgfpathrectangle{\pgfqpoint{0.100000in}{0.212622in}}{\pgfqpoint{3.696000in}{3.696000in}}%
\pgfusepath{clip}%
\pgfsetbuttcap%
\pgfsetroundjoin%
\definecolor{currentfill}{rgb}{0.121569,0.466667,0.705882}%
\pgfsetfillcolor{currentfill}%
\pgfsetfillopacity{0.345302}%
\pgfsetlinewidth{1.003750pt}%
\definecolor{currentstroke}{rgb}{0.121569,0.466667,0.705882}%
\pgfsetstrokecolor{currentstroke}%
\pgfsetstrokeopacity{0.345302}%
\pgfsetdash{}{0pt}%
\pgfpathmoveto{\pgfqpoint{1.975096in}{2.053802in}}%
\pgfpathcurveto{\pgfqpoint{1.983332in}{2.053802in}}{\pgfqpoint{1.991232in}{2.057074in}}{\pgfqpoint{1.997056in}{2.062898in}}%
\pgfpathcurveto{\pgfqpoint{2.002880in}{2.068722in}}{\pgfqpoint{2.006152in}{2.076622in}}{\pgfqpoint{2.006152in}{2.084858in}}%
\pgfpathcurveto{\pgfqpoint{2.006152in}{2.093094in}}{\pgfqpoint{2.002880in}{2.100994in}}{\pgfqpoint{1.997056in}{2.106818in}}%
\pgfpathcurveto{\pgfqpoint{1.991232in}{2.112642in}}{\pgfqpoint{1.983332in}{2.115915in}}{\pgfqpoint{1.975096in}{2.115915in}}%
\pgfpathcurveto{\pgfqpoint{1.966859in}{2.115915in}}{\pgfqpoint{1.958959in}{2.112642in}}{\pgfqpoint{1.953135in}{2.106818in}}%
\pgfpathcurveto{\pgfqpoint{1.947311in}{2.100994in}}{\pgfqpoint{1.944039in}{2.093094in}}{\pgfqpoint{1.944039in}{2.084858in}}%
\pgfpathcurveto{\pgfqpoint{1.944039in}{2.076622in}}{\pgfqpoint{1.947311in}{2.068722in}}{\pgfqpoint{1.953135in}{2.062898in}}%
\pgfpathcurveto{\pgfqpoint{1.958959in}{2.057074in}}{\pgfqpoint{1.966859in}{2.053802in}}{\pgfqpoint{1.975096in}{2.053802in}}%
\pgfpathclose%
\pgfusepath{stroke,fill}%
\end{pgfscope}%
\begin{pgfscope}%
\pgfpathrectangle{\pgfqpoint{0.100000in}{0.212622in}}{\pgfqpoint{3.696000in}{3.696000in}}%
\pgfusepath{clip}%
\pgfsetbuttcap%
\pgfsetroundjoin%
\definecolor{currentfill}{rgb}{0.121569,0.466667,0.705882}%
\pgfsetfillcolor{currentfill}%
\pgfsetfillopacity{0.346140}%
\pgfsetlinewidth{1.003750pt}%
\definecolor{currentstroke}{rgb}{0.121569,0.466667,0.705882}%
\pgfsetstrokecolor{currentstroke}%
\pgfsetstrokeopacity{0.346140}%
\pgfsetdash{}{0pt}%
\pgfpathmoveto{\pgfqpoint{1.787051in}{1.981723in}}%
\pgfpathcurveto{\pgfqpoint{1.795287in}{1.981723in}}{\pgfqpoint{1.803187in}{1.984996in}}{\pgfqpoint{1.809011in}{1.990820in}}%
\pgfpathcurveto{\pgfqpoint{1.814835in}{1.996644in}}{\pgfqpoint{1.818108in}{2.004544in}}{\pgfqpoint{1.818108in}{2.012780in}}%
\pgfpathcurveto{\pgfqpoint{1.818108in}{2.021016in}}{\pgfqpoint{1.814835in}{2.028916in}}{\pgfqpoint{1.809011in}{2.034740in}}%
\pgfpathcurveto{\pgfqpoint{1.803187in}{2.040564in}}{\pgfqpoint{1.795287in}{2.043836in}}{\pgfqpoint{1.787051in}{2.043836in}}%
\pgfpathcurveto{\pgfqpoint{1.778815in}{2.043836in}}{\pgfqpoint{1.770915in}{2.040564in}}{\pgfqpoint{1.765091in}{2.034740in}}%
\pgfpathcurveto{\pgfqpoint{1.759267in}{2.028916in}}{\pgfqpoint{1.755995in}{2.021016in}}{\pgfqpoint{1.755995in}{2.012780in}}%
\pgfpathcurveto{\pgfqpoint{1.755995in}{2.004544in}}{\pgfqpoint{1.759267in}{1.996644in}}{\pgfqpoint{1.765091in}{1.990820in}}%
\pgfpathcurveto{\pgfqpoint{1.770915in}{1.984996in}}{\pgfqpoint{1.778815in}{1.981723in}}{\pgfqpoint{1.787051in}{1.981723in}}%
\pgfpathclose%
\pgfusepath{stroke,fill}%
\end{pgfscope}%
\begin{pgfscope}%
\pgfpathrectangle{\pgfqpoint{0.100000in}{0.212622in}}{\pgfqpoint{3.696000in}{3.696000in}}%
\pgfusepath{clip}%
\pgfsetbuttcap%
\pgfsetroundjoin%
\definecolor{currentfill}{rgb}{0.121569,0.466667,0.705882}%
\pgfsetfillcolor{currentfill}%
\pgfsetfillopacity{0.347106}%
\pgfsetlinewidth{1.003750pt}%
\definecolor{currentstroke}{rgb}{0.121569,0.466667,0.705882}%
\pgfsetstrokecolor{currentstroke}%
\pgfsetstrokeopacity{0.347106}%
\pgfsetdash{}{0pt}%
\pgfpathmoveto{\pgfqpoint{1.783760in}{1.979427in}}%
\pgfpathcurveto{\pgfqpoint{1.791997in}{1.979427in}}{\pgfqpoint{1.799897in}{1.982699in}}{\pgfqpoint{1.805721in}{1.988523in}}%
\pgfpathcurveto{\pgfqpoint{1.811545in}{1.994347in}}{\pgfqpoint{1.814817in}{2.002247in}}{\pgfqpoint{1.814817in}{2.010483in}}%
\pgfpathcurveto{\pgfqpoint{1.814817in}{2.018719in}}{\pgfqpoint{1.811545in}{2.026619in}}{\pgfqpoint{1.805721in}{2.032443in}}%
\pgfpathcurveto{\pgfqpoint{1.799897in}{2.038267in}}{\pgfqpoint{1.791997in}{2.041540in}}{\pgfqpoint{1.783760in}{2.041540in}}%
\pgfpathcurveto{\pgfqpoint{1.775524in}{2.041540in}}{\pgfqpoint{1.767624in}{2.038267in}}{\pgfqpoint{1.761800in}{2.032443in}}%
\pgfpathcurveto{\pgfqpoint{1.755976in}{2.026619in}}{\pgfqpoint{1.752704in}{2.018719in}}{\pgfqpoint{1.752704in}{2.010483in}}%
\pgfpathcurveto{\pgfqpoint{1.752704in}{2.002247in}}{\pgfqpoint{1.755976in}{1.994347in}}{\pgfqpoint{1.761800in}{1.988523in}}%
\pgfpathcurveto{\pgfqpoint{1.767624in}{1.982699in}}{\pgfqpoint{1.775524in}{1.979427in}}{\pgfqpoint{1.783760in}{1.979427in}}%
\pgfpathclose%
\pgfusepath{stroke,fill}%
\end{pgfscope}%
\begin{pgfscope}%
\pgfpathrectangle{\pgfqpoint{0.100000in}{0.212622in}}{\pgfqpoint{3.696000in}{3.696000in}}%
\pgfusepath{clip}%
\pgfsetbuttcap%
\pgfsetroundjoin%
\definecolor{currentfill}{rgb}{0.121569,0.466667,0.705882}%
\pgfsetfillcolor{currentfill}%
\pgfsetfillopacity{0.347591}%
\pgfsetlinewidth{1.003750pt}%
\definecolor{currentstroke}{rgb}{0.121569,0.466667,0.705882}%
\pgfsetstrokecolor{currentstroke}%
\pgfsetstrokeopacity{0.347591}%
\pgfsetdash{}{0pt}%
\pgfpathmoveto{\pgfqpoint{1.782110in}{1.977659in}}%
\pgfpathcurveto{\pgfqpoint{1.790346in}{1.977659in}}{\pgfqpoint{1.798246in}{1.980932in}}{\pgfqpoint{1.804070in}{1.986756in}}%
\pgfpathcurveto{\pgfqpoint{1.809894in}{1.992580in}}{\pgfqpoint{1.813166in}{2.000480in}}{\pgfqpoint{1.813166in}{2.008716in}}%
\pgfpathcurveto{\pgfqpoint{1.813166in}{2.016952in}}{\pgfqpoint{1.809894in}{2.024852in}}{\pgfqpoint{1.804070in}{2.030676in}}%
\pgfpathcurveto{\pgfqpoint{1.798246in}{2.036500in}}{\pgfqpoint{1.790346in}{2.039772in}}{\pgfqpoint{1.782110in}{2.039772in}}%
\pgfpathcurveto{\pgfqpoint{1.773873in}{2.039772in}}{\pgfqpoint{1.765973in}{2.036500in}}{\pgfqpoint{1.760149in}{2.030676in}}%
\pgfpathcurveto{\pgfqpoint{1.754326in}{2.024852in}}{\pgfqpoint{1.751053in}{2.016952in}}{\pgfqpoint{1.751053in}{2.008716in}}%
\pgfpathcurveto{\pgfqpoint{1.751053in}{2.000480in}}{\pgfqpoint{1.754326in}{1.992580in}}{\pgfqpoint{1.760149in}{1.986756in}}%
\pgfpathcurveto{\pgfqpoint{1.765973in}{1.980932in}}{\pgfqpoint{1.773873in}{1.977659in}}{\pgfqpoint{1.782110in}{1.977659in}}%
\pgfpathclose%
\pgfusepath{stroke,fill}%
\end{pgfscope}%
\begin{pgfscope}%
\pgfpathrectangle{\pgfqpoint{0.100000in}{0.212622in}}{\pgfqpoint{3.696000in}{3.696000in}}%
\pgfusepath{clip}%
\pgfsetbuttcap%
\pgfsetroundjoin%
\definecolor{currentfill}{rgb}{0.121569,0.466667,0.705882}%
\pgfsetfillcolor{currentfill}%
\pgfsetfillopacity{0.348296}%
\pgfsetlinewidth{1.003750pt}%
\definecolor{currentstroke}{rgb}{0.121569,0.466667,0.705882}%
\pgfsetstrokecolor{currentstroke}%
\pgfsetstrokeopacity{0.348296}%
\pgfsetdash{}{0pt}%
\pgfpathmoveto{\pgfqpoint{1.778335in}{1.974520in}}%
\pgfpathcurveto{\pgfqpoint{1.786571in}{1.974520in}}{\pgfqpoint{1.794471in}{1.977793in}}{\pgfqpoint{1.800295in}{1.983617in}}%
\pgfpathcurveto{\pgfqpoint{1.806119in}{1.989441in}}{\pgfqpoint{1.809391in}{1.997341in}}{\pgfqpoint{1.809391in}{2.005577in}}%
\pgfpathcurveto{\pgfqpoint{1.809391in}{2.013813in}}{\pgfqpoint{1.806119in}{2.021713in}}{\pgfqpoint{1.800295in}{2.027537in}}%
\pgfpathcurveto{\pgfqpoint{1.794471in}{2.033361in}}{\pgfqpoint{1.786571in}{2.036633in}}{\pgfqpoint{1.778335in}{2.036633in}}%
\pgfpathcurveto{\pgfqpoint{1.770098in}{2.036633in}}{\pgfqpoint{1.762198in}{2.033361in}}{\pgfqpoint{1.756374in}{2.027537in}}%
\pgfpathcurveto{\pgfqpoint{1.750550in}{2.021713in}}{\pgfqpoint{1.747278in}{2.013813in}}{\pgfqpoint{1.747278in}{2.005577in}}%
\pgfpathcurveto{\pgfqpoint{1.747278in}{1.997341in}}{\pgfqpoint{1.750550in}{1.989441in}}{\pgfqpoint{1.756374in}{1.983617in}}%
\pgfpathcurveto{\pgfqpoint{1.762198in}{1.977793in}}{\pgfqpoint{1.770098in}{1.974520in}}{\pgfqpoint{1.778335in}{1.974520in}}%
\pgfpathclose%
\pgfusepath{stroke,fill}%
\end{pgfscope}%
\begin{pgfscope}%
\pgfpathrectangle{\pgfqpoint{0.100000in}{0.212622in}}{\pgfqpoint{3.696000in}{3.696000in}}%
\pgfusepath{clip}%
\pgfsetbuttcap%
\pgfsetroundjoin%
\definecolor{currentfill}{rgb}{0.121569,0.466667,0.705882}%
\pgfsetfillcolor{currentfill}%
\pgfsetfillopacity{0.350254}%
\pgfsetlinewidth{1.003750pt}%
\definecolor{currentstroke}{rgb}{0.121569,0.466667,0.705882}%
\pgfsetstrokecolor{currentstroke}%
\pgfsetstrokeopacity{0.350254}%
\pgfsetdash{}{0pt}%
\pgfpathmoveto{\pgfqpoint{1.978125in}{2.053295in}}%
\pgfpathcurveto{\pgfqpoint{1.986361in}{2.053295in}}{\pgfqpoint{1.994261in}{2.056568in}}{\pgfqpoint{2.000085in}{2.062392in}}%
\pgfpathcurveto{\pgfqpoint{2.005909in}{2.068216in}}{\pgfqpoint{2.009181in}{2.076116in}}{\pgfqpoint{2.009181in}{2.084352in}}%
\pgfpathcurveto{\pgfqpoint{2.009181in}{2.092588in}}{\pgfqpoint{2.005909in}{2.100488in}}{\pgfqpoint{2.000085in}{2.106312in}}%
\pgfpathcurveto{\pgfqpoint{1.994261in}{2.112136in}}{\pgfqpoint{1.986361in}{2.115408in}}{\pgfqpoint{1.978125in}{2.115408in}}%
\pgfpathcurveto{\pgfqpoint{1.969889in}{2.115408in}}{\pgfqpoint{1.961988in}{2.112136in}}{\pgfqpoint{1.956165in}{2.106312in}}%
\pgfpathcurveto{\pgfqpoint{1.950341in}{2.100488in}}{\pgfqpoint{1.947068in}{2.092588in}}{\pgfqpoint{1.947068in}{2.084352in}}%
\pgfpathcurveto{\pgfqpoint{1.947068in}{2.076116in}}{\pgfqpoint{1.950341in}{2.068216in}}{\pgfqpoint{1.956165in}{2.062392in}}%
\pgfpathcurveto{\pgfqpoint{1.961988in}{2.056568in}}{\pgfqpoint{1.969889in}{2.053295in}}{\pgfqpoint{1.978125in}{2.053295in}}%
\pgfpathclose%
\pgfusepath{stroke,fill}%
\end{pgfscope}%
\begin{pgfscope}%
\pgfpathrectangle{\pgfqpoint{0.100000in}{0.212622in}}{\pgfqpoint{3.696000in}{3.696000in}}%
\pgfusepath{clip}%
\pgfsetbuttcap%
\pgfsetroundjoin%
\definecolor{currentfill}{rgb}{0.121569,0.466667,0.705882}%
\pgfsetfillcolor{currentfill}%
\pgfsetfillopacity{0.350310}%
\pgfsetlinewidth{1.003750pt}%
\definecolor{currentstroke}{rgb}{0.121569,0.466667,0.705882}%
\pgfsetstrokecolor{currentstroke}%
\pgfsetstrokeopacity{0.350310}%
\pgfsetdash{}{0pt}%
\pgfpathmoveto{\pgfqpoint{1.772542in}{1.971938in}}%
\pgfpathcurveto{\pgfqpoint{1.780779in}{1.971938in}}{\pgfqpoint{1.788679in}{1.975210in}}{\pgfqpoint{1.794503in}{1.981034in}}%
\pgfpathcurveto{\pgfqpoint{1.800327in}{1.986858in}}{\pgfqpoint{1.803599in}{1.994758in}}{\pgfqpoint{1.803599in}{2.002994in}}%
\pgfpathcurveto{\pgfqpoint{1.803599in}{2.011231in}}{\pgfqpoint{1.800327in}{2.019131in}}{\pgfqpoint{1.794503in}{2.024955in}}%
\pgfpathcurveto{\pgfqpoint{1.788679in}{2.030779in}}{\pgfqpoint{1.780779in}{2.034051in}}{\pgfqpoint{1.772542in}{2.034051in}}%
\pgfpathcurveto{\pgfqpoint{1.764306in}{2.034051in}}{\pgfqpoint{1.756406in}{2.030779in}}{\pgfqpoint{1.750582in}{2.024955in}}%
\pgfpathcurveto{\pgfqpoint{1.744758in}{2.019131in}}{\pgfqpoint{1.741486in}{2.011231in}}{\pgfqpoint{1.741486in}{2.002994in}}%
\pgfpathcurveto{\pgfqpoint{1.741486in}{1.994758in}}{\pgfqpoint{1.744758in}{1.986858in}}{\pgfqpoint{1.750582in}{1.981034in}}%
\pgfpathcurveto{\pgfqpoint{1.756406in}{1.975210in}}{\pgfqpoint{1.764306in}{1.971938in}}{\pgfqpoint{1.772542in}{1.971938in}}%
\pgfpathclose%
\pgfusepath{stroke,fill}%
\end{pgfscope}%
\begin{pgfscope}%
\pgfpathrectangle{\pgfqpoint{0.100000in}{0.212622in}}{\pgfqpoint{3.696000in}{3.696000in}}%
\pgfusepath{clip}%
\pgfsetbuttcap%
\pgfsetroundjoin%
\definecolor{currentfill}{rgb}{0.121569,0.466667,0.705882}%
\pgfsetfillcolor{currentfill}%
\pgfsetfillopacity{0.351625}%
\pgfsetlinewidth{1.003750pt}%
\definecolor{currentstroke}{rgb}{0.121569,0.466667,0.705882}%
\pgfsetstrokecolor{currentstroke}%
\pgfsetstrokeopacity{0.351625}%
\pgfsetdash{}{0pt}%
\pgfpathmoveto{\pgfqpoint{1.768624in}{1.967136in}}%
\pgfpathcurveto{\pgfqpoint{1.776861in}{1.967136in}}{\pgfqpoint{1.784761in}{1.970408in}}{\pgfqpoint{1.790585in}{1.976232in}}%
\pgfpathcurveto{\pgfqpoint{1.796409in}{1.982056in}}{\pgfqpoint{1.799681in}{1.989956in}}{\pgfqpoint{1.799681in}{1.998192in}}%
\pgfpathcurveto{\pgfqpoint{1.799681in}{2.006428in}}{\pgfqpoint{1.796409in}{2.014329in}}{\pgfqpoint{1.790585in}{2.020152in}}%
\pgfpathcurveto{\pgfqpoint{1.784761in}{2.025976in}}{\pgfqpoint{1.776861in}{2.029249in}}{\pgfqpoint{1.768624in}{2.029249in}}%
\pgfpathcurveto{\pgfqpoint{1.760388in}{2.029249in}}{\pgfqpoint{1.752488in}{2.025976in}}{\pgfqpoint{1.746664in}{2.020152in}}%
\pgfpathcurveto{\pgfqpoint{1.740840in}{2.014329in}}{\pgfqpoint{1.737568in}{2.006428in}}{\pgfqpoint{1.737568in}{1.998192in}}%
\pgfpathcurveto{\pgfqpoint{1.737568in}{1.989956in}}{\pgfqpoint{1.740840in}{1.982056in}}{\pgfqpoint{1.746664in}{1.976232in}}%
\pgfpathcurveto{\pgfqpoint{1.752488in}{1.970408in}}{\pgfqpoint{1.760388in}{1.967136in}}{\pgfqpoint{1.768624in}{1.967136in}}%
\pgfpathclose%
\pgfusepath{stroke,fill}%
\end{pgfscope}%
\begin{pgfscope}%
\pgfpathrectangle{\pgfqpoint{0.100000in}{0.212622in}}{\pgfqpoint{3.696000in}{3.696000in}}%
\pgfusepath{clip}%
\pgfsetbuttcap%
\pgfsetroundjoin%
\definecolor{currentfill}{rgb}{0.121569,0.466667,0.705882}%
\pgfsetfillcolor{currentfill}%
\pgfsetfillopacity{0.352635}%
\pgfsetlinewidth{1.003750pt}%
\definecolor{currentstroke}{rgb}{0.121569,0.466667,0.705882}%
\pgfsetstrokecolor{currentstroke}%
\pgfsetstrokeopacity{0.352635}%
\pgfsetdash{}{0pt}%
\pgfpathmoveto{\pgfqpoint{1.764462in}{1.963481in}}%
\pgfpathcurveto{\pgfqpoint{1.772698in}{1.963481in}}{\pgfqpoint{1.780598in}{1.966753in}}{\pgfqpoint{1.786422in}{1.972577in}}%
\pgfpathcurveto{\pgfqpoint{1.792246in}{1.978401in}}{\pgfqpoint{1.795518in}{1.986301in}}{\pgfqpoint{1.795518in}{1.994537in}}%
\pgfpathcurveto{\pgfqpoint{1.795518in}{2.002774in}}{\pgfqpoint{1.792246in}{2.010674in}}{\pgfqpoint{1.786422in}{2.016498in}}%
\pgfpathcurveto{\pgfqpoint{1.780598in}{2.022321in}}{\pgfqpoint{1.772698in}{2.025594in}}{\pgfqpoint{1.764462in}{2.025594in}}%
\pgfpathcurveto{\pgfqpoint{1.756225in}{2.025594in}}{\pgfqpoint{1.748325in}{2.022321in}}{\pgfqpoint{1.742501in}{2.016498in}}%
\pgfpathcurveto{\pgfqpoint{1.736677in}{2.010674in}}{\pgfqpoint{1.733405in}{2.002774in}}{\pgfqpoint{1.733405in}{1.994537in}}%
\pgfpathcurveto{\pgfqpoint{1.733405in}{1.986301in}}{\pgfqpoint{1.736677in}{1.978401in}}{\pgfqpoint{1.742501in}{1.972577in}}%
\pgfpathcurveto{\pgfqpoint{1.748325in}{1.966753in}}{\pgfqpoint{1.756225in}{1.963481in}}{\pgfqpoint{1.764462in}{1.963481in}}%
\pgfpathclose%
\pgfusepath{stroke,fill}%
\end{pgfscope}%
\begin{pgfscope}%
\pgfpathrectangle{\pgfqpoint{0.100000in}{0.212622in}}{\pgfqpoint{3.696000in}{3.696000in}}%
\pgfusepath{clip}%
\pgfsetbuttcap%
\pgfsetroundjoin%
\definecolor{currentfill}{rgb}{0.121569,0.466667,0.705882}%
\pgfsetfillcolor{currentfill}%
\pgfsetfillopacity{0.354059}%
\pgfsetlinewidth{1.003750pt}%
\definecolor{currentstroke}{rgb}{0.121569,0.466667,0.705882}%
\pgfsetstrokecolor{currentstroke}%
\pgfsetstrokeopacity{0.354059}%
\pgfsetdash{}{0pt}%
\pgfpathmoveto{\pgfqpoint{1.983485in}{2.044384in}}%
\pgfpathcurveto{\pgfqpoint{1.991721in}{2.044384in}}{\pgfqpoint{1.999621in}{2.047657in}}{\pgfqpoint{2.005445in}{2.053480in}}%
\pgfpathcurveto{\pgfqpoint{2.011269in}{2.059304in}}{\pgfqpoint{2.014541in}{2.067204in}}{\pgfqpoint{2.014541in}{2.075441in}}%
\pgfpathcurveto{\pgfqpoint{2.014541in}{2.083677in}}{\pgfqpoint{2.011269in}{2.091577in}}{\pgfqpoint{2.005445in}{2.097401in}}%
\pgfpathcurveto{\pgfqpoint{1.999621in}{2.103225in}}{\pgfqpoint{1.991721in}{2.106497in}}{\pgfqpoint{1.983485in}{2.106497in}}%
\pgfpathcurveto{\pgfqpoint{1.975249in}{2.106497in}}{\pgfqpoint{1.967348in}{2.103225in}}{\pgfqpoint{1.961525in}{2.097401in}}%
\pgfpathcurveto{\pgfqpoint{1.955701in}{2.091577in}}{\pgfqpoint{1.952428in}{2.083677in}}{\pgfqpoint{1.952428in}{2.075441in}}%
\pgfpathcurveto{\pgfqpoint{1.952428in}{2.067204in}}{\pgfqpoint{1.955701in}{2.059304in}}{\pgfqpoint{1.961525in}{2.053480in}}%
\pgfpathcurveto{\pgfqpoint{1.967348in}{2.047657in}}{\pgfqpoint{1.975249in}{2.044384in}}{\pgfqpoint{1.983485in}{2.044384in}}%
\pgfpathclose%
\pgfusepath{stroke,fill}%
\end{pgfscope}%
\begin{pgfscope}%
\pgfpathrectangle{\pgfqpoint{0.100000in}{0.212622in}}{\pgfqpoint{3.696000in}{3.696000in}}%
\pgfusepath{clip}%
\pgfsetbuttcap%
\pgfsetroundjoin%
\definecolor{currentfill}{rgb}{0.121569,0.466667,0.705882}%
\pgfsetfillcolor{currentfill}%
\pgfsetfillopacity{0.354995}%
\pgfsetlinewidth{1.003750pt}%
\definecolor{currentstroke}{rgb}{0.121569,0.466667,0.705882}%
\pgfsetstrokecolor{currentstroke}%
\pgfsetstrokeopacity{0.354995}%
\pgfsetdash{}{0pt}%
\pgfpathmoveto{\pgfqpoint{1.757273in}{1.959780in}}%
\pgfpathcurveto{\pgfqpoint{1.765509in}{1.959780in}}{\pgfqpoint{1.773409in}{1.963053in}}{\pgfqpoint{1.779233in}{1.968877in}}%
\pgfpathcurveto{\pgfqpoint{1.785057in}{1.974701in}}{\pgfqpoint{1.788330in}{1.982601in}}{\pgfqpoint{1.788330in}{1.990837in}}%
\pgfpathcurveto{\pgfqpoint{1.788330in}{1.999073in}}{\pgfqpoint{1.785057in}{2.006973in}}{\pgfqpoint{1.779233in}{2.012797in}}%
\pgfpathcurveto{\pgfqpoint{1.773409in}{2.018621in}}{\pgfqpoint{1.765509in}{2.021893in}}{\pgfqpoint{1.757273in}{2.021893in}}%
\pgfpathcurveto{\pgfqpoint{1.749037in}{2.021893in}}{\pgfqpoint{1.741137in}{2.018621in}}{\pgfqpoint{1.735313in}{2.012797in}}%
\pgfpathcurveto{\pgfqpoint{1.729489in}{2.006973in}}{\pgfqpoint{1.726217in}{1.999073in}}{\pgfqpoint{1.726217in}{1.990837in}}%
\pgfpathcurveto{\pgfqpoint{1.726217in}{1.982601in}}{\pgfqpoint{1.729489in}{1.974701in}}{\pgfqpoint{1.735313in}{1.968877in}}%
\pgfpathcurveto{\pgfqpoint{1.741137in}{1.963053in}}{\pgfqpoint{1.749037in}{1.959780in}}{\pgfqpoint{1.757273in}{1.959780in}}%
\pgfpathclose%
\pgfusepath{stroke,fill}%
\end{pgfscope}%
\begin{pgfscope}%
\pgfpathrectangle{\pgfqpoint{0.100000in}{0.212622in}}{\pgfqpoint{3.696000in}{3.696000in}}%
\pgfusepath{clip}%
\pgfsetbuttcap%
\pgfsetroundjoin%
\definecolor{currentfill}{rgb}{0.121569,0.466667,0.705882}%
\pgfsetfillcolor{currentfill}%
\pgfsetfillopacity{0.357076}%
\pgfsetlinewidth{1.003750pt}%
\definecolor{currentstroke}{rgb}{0.121569,0.466667,0.705882}%
\pgfsetstrokecolor{currentstroke}%
\pgfsetstrokeopacity{0.357076}%
\pgfsetdash{}{0pt}%
\pgfpathmoveto{\pgfqpoint{1.750968in}{1.954226in}}%
\pgfpathcurveto{\pgfqpoint{1.759205in}{1.954226in}}{\pgfqpoint{1.767105in}{1.957499in}}{\pgfqpoint{1.772929in}{1.963323in}}%
\pgfpathcurveto{\pgfqpoint{1.778752in}{1.969147in}}{\pgfqpoint{1.782025in}{1.977047in}}{\pgfqpoint{1.782025in}{1.985283in}}%
\pgfpathcurveto{\pgfqpoint{1.782025in}{1.993519in}}{\pgfqpoint{1.778752in}{2.001419in}}{\pgfqpoint{1.772929in}{2.007243in}}%
\pgfpathcurveto{\pgfqpoint{1.767105in}{2.013067in}}{\pgfqpoint{1.759205in}{2.016339in}}{\pgfqpoint{1.750968in}{2.016339in}}%
\pgfpathcurveto{\pgfqpoint{1.742732in}{2.016339in}}{\pgfqpoint{1.734832in}{2.013067in}}{\pgfqpoint{1.729008in}{2.007243in}}%
\pgfpathcurveto{\pgfqpoint{1.723184in}{2.001419in}}{\pgfqpoint{1.719912in}{1.993519in}}{\pgfqpoint{1.719912in}{1.985283in}}%
\pgfpathcurveto{\pgfqpoint{1.719912in}{1.977047in}}{\pgfqpoint{1.723184in}{1.969147in}}{\pgfqpoint{1.729008in}{1.963323in}}%
\pgfpathcurveto{\pgfqpoint{1.734832in}{1.957499in}}{\pgfqpoint{1.742732in}{1.954226in}}{\pgfqpoint{1.750968in}{1.954226in}}%
\pgfpathclose%
\pgfusepath{stroke,fill}%
\end{pgfscope}%
\begin{pgfscope}%
\pgfpathrectangle{\pgfqpoint{0.100000in}{0.212622in}}{\pgfqpoint{3.696000in}{3.696000in}}%
\pgfusepath{clip}%
\pgfsetbuttcap%
\pgfsetroundjoin%
\definecolor{currentfill}{rgb}{0.121569,0.466667,0.705882}%
\pgfsetfillcolor{currentfill}%
\pgfsetfillopacity{0.358619}%
\pgfsetlinewidth{1.003750pt}%
\definecolor{currentstroke}{rgb}{0.121569,0.466667,0.705882}%
\pgfsetstrokecolor{currentstroke}%
\pgfsetstrokeopacity{0.358619}%
\pgfsetdash{}{0pt}%
\pgfpathmoveto{\pgfqpoint{1.746234in}{1.952370in}}%
\pgfpathcurveto{\pgfqpoint{1.754471in}{1.952370in}}{\pgfqpoint{1.762371in}{1.955642in}}{\pgfqpoint{1.768195in}{1.961466in}}%
\pgfpathcurveto{\pgfqpoint{1.774019in}{1.967290in}}{\pgfqpoint{1.777291in}{1.975190in}}{\pgfqpoint{1.777291in}{1.983427in}}%
\pgfpathcurveto{\pgfqpoint{1.777291in}{1.991663in}}{\pgfqpoint{1.774019in}{1.999563in}}{\pgfqpoint{1.768195in}{2.005387in}}%
\pgfpathcurveto{\pgfqpoint{1.762371in}{2.011211in}}{\pgfqpoint{1.754471in}{2.014483in}}{\pgfqpoint{1.746234in}{2.014483in}}%
\pgfpathcurveto{\pgfqpoint{1.737998in}{2.014483in}}{\pgfqpoint{1.730098in}{2.011211in}}{\pgfqpoint{1.724274in}{2.005387in}}%
\pgfpathcurveto{\pgfqpoint{1.718450in}{1.999563in}}{\pgfqpoint{1.715178in}{1.991663in}}{\pgfqpoint{1.715178in}{1.983427in}}%
\pgfpathcurveto{\pgfqpoint{1.715178in}{1.975190in}}{\pgfqpoint{1.718450in}{1.967290in}}{\pgfqpoint{1.724274in}{1.961466in}}%
\pgfpathcurveto{\pgfqpoint{1.730098in}{1.955642in}}{\pgfqpoint{1.737998in}{1.952370in}}{\pgfqpoint{1.746234in}{1.952370in}}%
\pgfpathclose%
\pgfusepath{stroke,fill}%
\end{pgfscope}%
\begin{pgfscope}%
\pgfpathrectangle{\pgfqpoint{0.100000in}{0.212622in}}{\pgfqpoint{3.696000in}{3.696000in}}%
\pgfusepath{clip}%
\pgfsetbuttcap%
\pgfsetroundjoin%
\definecolor{currentfill}{rgb}{0.121569,0.466667,0.705882}%
\pgfsetfillcolor{currentfill}%
\pgfsetfillopacity{0.358775}%
\pgfsetlinewidth{1.003750pt}%
\definecolor{currentstroke}{rgb}{0.121569,0.466667,0.705882}%
\pgfsetstrokecolor{currentstroke}%
\pgfsetstrokeopacity{0.358775}%
\pgfsetdash{}{0pt}%
\pgfpathmoveto{\pgfqpoint{1.986542in}{2.037070in}}%
\pgfpathcurveto{\pgfqpoint{1.994778in}{2.037070in}}{\pgfqpoint{2.002678in}{2.040342in}}{\pgfqpoint{2.008502in}{2.046166in}}%
\pgfpathcurveto{\pgfqpoint{2.014326in}{2.051990in}}{\pgfqpoint{2.017598in}{2.059890in}}{\pgfqpoint{2.017598in}{2.068126in}}%
\pgfpathcurveto{\pgfqpoint{2.017598in}{2.076363in}}{\pgfqpoint{2.014326in}{2.084263in}}{\pgfqpoint{2.008502in}{2.090086in}}%
\pgfpathcurveto{\pgfqpoint{2.002678in}{2.095910in}}{\pgfqpoint{1.994778in}{2.099183in}}{\pgfqpoint{1.986542in}{2.099183in}}%
\pgfpathcurveto{\pgfqpoint{1.978306in}{2.099183in}}{\pgfqpoint{1.970406in}{2.095910in}}{\pgfqpoint{1.964582in}{2.090086in}}%
\pgfpathcurveto{\pgfqpoint{1.958758in}{2.084263in}}{\pgfqpoint{1.955485in}{2.076363in}}{\pgfqpoint{1.955485in}{2.068126in}}%
\pgfpathcurveto{\pgfqpoint{1.955485in}{2.059890in}}{\pgfqpoint{1.958758in}{2.051990in}}{\pgfqpoint{1.964582in}{2.046166in}}%
\pgfpathcurveto{\pgfqpoint{1.970406in}{2.040342in}}{\pgfqpoint{1.978306in}{2.037070in}}{\pgfqpoint{1.986542in}{2.037070in}}%
\pgfpathclose%
\pgfusepath{stroke,fill}%
\end{pgfscope}%
\begin{pgfscope}%
\pgfpathrectangle{\pgfqpoint{0.100000in}{0.212622in}}{\pgfqpoint{3.696000in}{3.696000in}}%
\pgfusepath{clip}%
\pgfsetbuttcap%
\pgfsetroundjoin%
\definecolor{currentfill}{rgb}{0.121569,0.466667,0.705882}%
\pgfsetfillcolor{currentfill}%
\pgfsetfillopacity{0.359913}%
\pgfsetlinewidth{1.003750pt}%
\definecolor{currentstroke}{rgb}{0.121569,0.466667,0.705882}%
\pgfsetstrokecolor{currentstroke}%
\pgfsetstrokeopacity{0.359913}%
\pgfsetdash{}{0pt}%
\pgfpathmoveto{\pgfqpoint{1.741943in}{1.950155in}}%
\pgfpathcurveto{\pgfqpoint{1.750180in}{1.950155in}}{\pgfqpoint{1.758080in}{1.953427in}}{\pgfqpoint{1.763904in}{1.959251in}}%
\pgfpathcurveto{\pgfqpoint{1.769728in}{1.965075in}}{\pgfqpoint{1.773000in}{1.972975in}}{\pgfqpoint{1.773000in}{1.981211in}}%
\pgfpathcurveto{\pgfqpoint{1.773000in}{1.989448in}}{\pgfqpoint{1.769728in}{1.997348in}}{\pgfqpoint{1.763904in}{2.003172in}}%
\pgfpathcurveto{\pgfqpoint{1.758080in}{2.008995in}}{\pgfqpoint{1.750180in}{2.012268in}}{\pgfqpoint{1.741943in}{2.012268in}}%
\pgfpathcurveto{\pgfqpoint{1.733707in}{2.012268in}}{\pgfqpoint{1.725807in}{2.008995in}}{\pgfqpoint{1.719983in}{2.003172in}}%
\pgfpathcurveto{\pgfqpoint{1.714159in}{1.997348in}}{\pgfqpoint{1.710887in}{1.989448in}}{\pgfqpoint{1.710887in}{1.981211in}}%
\pgfpathcurveto{\pgfqpoint{1.710887in}{1.972975in}}{\pgfqpoint{1.714159in}{1.965075in}}{\pgfqpoint{1.719983in}{1.959251in}}%
\pgfpathcurveto{\pgfqpoint{1.725807in}{1.953427in}}{\pgfqpoint{1.733707in}{1.950155in}}{\pgfqpoint{1.741943in}{1.950155in}}%
\pgfpathclose%
\pgfusepath{stroke,fill}%
\end{pgfscope}%
\begin{pgfscope}%
\pgfpathrectangle{\pgfqpoint{0.100000in}{0.212622in}}{\pgfqpoint{3.696000in}{3.696000in}}%
\pgfusepath{clip}%
\pgfsetbuttcap%
\pgfsetroundjoin%
\definecolor{currentfill}{rgb}{0.121569,0.466667,0.705882}%
\pgfsetfillcolor{currentfill}%
\pgfsetfillopacity{0.362016}%
\pgfsetlinewidth{1.003750pt}%
\definecolor{currentstroke}{rgb}{0.121569,0.466667,0.705882}%
\pgfsetstrokecolor{currentstroke}%
\pgfsetstrokeopacity{0.362016}%
\pgfsetdash{}{0pt}%
\pgfpathmoveto{\pgfqpoint{1.734151in}{1.944409in}}%
\pgfpathcurveto{\pgfqpoint{1.742387in}{1.944409in}}{\pgfqpoint{1.750287in}{1.947681in}}{\pgfqpoint{1.756111in}{1.953505in}}%
\pgfpathcurveto{\pgfqpoint{1.761935in}{1.959329in}}{\pgfqpoint{1.765207in}{1.967229in}}{\pgfqpoint{1.765207in}{1.975465in}}%
\pgfpathcurveto{\pgfqpoint{1.765207in}{1.983701in}}{\pgfqpoint{1.761935in}{1.991602in}}{\pgfqpoint{1.756111in}{1.997425in}}%
\pgfpathcurveto{\pgfqpoint{1.750287in}{2.003249in}}{\pgfqpoint{1.742387in}{2.006522in}}{\pgfqpoint{1.734151in}{2.006522in}}%
\pgfpathcurveto{\pgfqpoint{1.725914in}{2.006522in}}{\pgfqpoint{1.718014in}{2.003249in}}{\pgfqpoint{1.712190in}{1.997425in}}%
\pgfpathcurveto{\pgfqpoint{1.706366in}{1.991602in}}{\pgfqpoint{1.703094in}{1.983701in}}{\pgfqpoint{1.703094in}{1.975465in}}%
\pgfpathcurveto{\pgfqpoint{1.703094in}{1.967229in}}{\pgfqpoint{1.706366in}{1.959329in}}{\pgfqpoint{1.712190in}{1.953505in}}%
\pgfpathcurveto{\pgfqpoint{1.718014in}{1.947681in}}{\pgfqpoint{1.725914in}{1.944409in}}{\pgfqpoint{1.734151in}{1.944409in}}%
\pgfpathclose%
\pgfusepath{stroke,fill}%
\end{pgfscope}%
\begin{pgfscope}%
\pgfpathrectangle{\pgfqpoint{0.100000in}{0.212622in}}{\pgfqpoint{3.696000in}{3.696000in}}%
\pgfusepath{clip}%
\pgfsetbuttcap%
\pgfsetroundjoin%
\definecolor{currentfill}{rgb}{0.121569,0.466667,0.705882}%
\pgfsetfillcolor{currentfill}%
\pgfsetfillopacity{0.363856}%
\pgfsetlinewidth{1.003750pt}%
\definecolor{currentstroke}{rgb}{0.121569,0.466667,0.705882}%
\pgfsetstrokecolor{currentstroke}%
\pgfsetstrokeopacity{0.363856}%
\pgfsetdash{}{0pt}%
\pgfpathmoveto{\pgfqpoint{1.728678in}{1.941846in}}%
\pgfpathcurveto{\pgfqpoint{1.736915in}{1.941846in}}{\pgfqpoint{1.744815in}{1.945118in}}{\pgfqpoint{1.750639in}{1.950942in}}%
\pgfpathcurveto{\pgfqpoint{1.756463in}{1.956766in}}{\pgfqpoint{1.759735in}{1.964666in}}{\pgfqpoint{1.759735in}{1.972902in}}%
\pgfpathcurveto{\pgfqpoint{1.759735in}{1.981138in}}{\pgfqpoint{1.756463in}{1.989038in}}{\pgfqpoint{1.750639in}{1.994862in}}%
\pgfpathcurveto{\pgfqpoint{1.744815in}{2.000686in}}{\pgfqpoint{1.736915in}{2.003959in}}{\pgfqpoint{1.728678in}{2.003959in}}%
\pgfpathcurveto{\pgfqpoint{1.720442in}{2.003959in}}{\pgfqpoint{1.712542in}{2.000686in}}{\pgfqpoint{1.706718in}{1.994862in}}%
\pgfpathcurveto{\pgfqpoint{1.700894in}{1.989038in}}{\pgfqpoint{1.697622in}{1.981138in}}{\pgfqpoint{1.697622in}{1.972902in}}%
\pgfpathcurveto{\pgfqpoint{1.697622in}{1.964666in}}{\pgfqpoint{1.700894in}{1.956766in}}{\pgfqpoint{1.706718in}{1.950942in}}%
\pgfpathcurveto{\pgfqpoint{1.712542in}{1.945118in}}{\pgfqpoint{1.720442in}{1.941846in}}{\pgfqpoint{1.728678in}{1.941846in}}%
\pgfpathclose%
\pgfusepath{stroke,fill}%
\end{pgfscope}%
\begin{pgfscope}%
\pgfpathrectangle{\pgfqpoint{0.100000in}{0.212622in}}{\pgfqpoint{3.696000in}{3.696000in}}%
\pgfusepath{clip}%
\pgfsetbuttcap%
\pgfsetroundjoin%
\definecolor{currentfill}{rgb}{0.121569,0.466667,0.705882}%
\pgfsetfillcolor{currentfill}%
\pgfsetfillopacity{0.364384}%
\pgfsetlinewidth{1.003750pt}%
\definecolor{currentstroke}{rgb}{0.121569,0.466667,0.705882}%
\pgfsetstrokecolor{currentstroke}%
\pgfsetstrokeopacity{0.364384}%
\pgfsetdash{}{0pt}%
\pgfpathmoveto{\pgfqpoint{1.988793in}{2.031694in}}%
\pgfpathcurveto{\pgfqpoint{1.997029in}{2.031694in}}{\pgfqpoint{2.004929in}{2.034967in}}{\pgfqpoint{2.010753in}{2.040791in}}%
\pgfpathcurveto{\pgfqpoint{2.016577in}{2.046615in}}{\pgfqpoint{2.019850in}{2.054515in}}{\pgfqpoint{2.019850in}{2.062751in}}%
\pgfpathcurveto{\pgfqpoint{2.019850in}{2.070987in}}{\pgfqpoint{2.016577in}{2.078887in}}{\pgfqpoint{2.010753in}{2.084711in}}%
\pgfpathcurveto{\pgfqpoint{2.004929in}{2.090535in}}{\pgfqpoint{1.997029in}{2.093807in}}{\pgfqpoint{1.988793in}{2.093807in}}%
\pgfpathcurveto{\pgfqpoint{1.980557in}{2.093807in}}{\pgfqpoint{1.972657in}{2.090535in}}{\pgfqpoint{1.966833in}{2.084711in}}%
\pgfpathcurveto{\pgfqpoint{1.961009in}{2.078887in}}{\pgfqpoint{1.957737in}{2.070987in}}{\pgfqpoint{1.957737in}{2.062751in}}%
\pgfpathcurveto{\pgfqpoint{1.957737in}{2.054515in}}{\pgfqpoint{1.961009in}{2.046615in}}{\pgfqpoint{1.966833in}{2.040791in}}%
\pgfpathcurveto{\pgfqpoint{1.972657in}{2.034967in}}{\pgfqpoint{1.980557in}{2.031694in}}{\pgfqpoint{1.988793in}{2.031694in}}%
\pgfpathclose%
\pgfusepath{stroke,fill}%
\end{pgfscope}%
\begin{pgfscope}%
\pgfpathrectangle{\pgfqpoint{0.100000in}{0.212622in}}{\pgfqpoint{3.696000in}{3.696000in}}%
\pgfusepath{clip}%
\pgfsetbuttcap%
\pgfsetroundjoin%
\definecolor{currentfill}{rgb}{0.121569,0.466667,0.705882}%
\pgfsetfillcolor{currentfill}%
\pgfsetfillopacity{0.365440}%
\pgfsetlinewidth{1.003750pt}%
\definecolor{currentstroke}{rgb}{0.121569,0.466667,0.705882}%
\pgfsetstrokecolor{currentstroke}%
\pgfsetstrokeopacity{0.365440}%
\pgfsetdash{}{0pt}%
\pgfpathmoveto{\pgfqpoint{1.723833in}{1.939624in}}%
\pgfpathcurveto{\pgfqpoint{1.732069in}{1.939624in}}{\pgfqpoint{1.739969in}{1.942896in}}{\pgfqpoint{1.745793in}{1.948720in}}%
\pgfpathcurveto{\pgfqpoint{1.751617in}{1.954544in}}{\pgfqpoint{1.754889in}{1.962444in}}{\pgfqpoint{1.754889in}{1.970681in}}%
\pgfpathcurveto{\pgfqpoint{1.754889in}{1.978917in}}{\pgfqpoint{1.751617in}{1.986817in}}{\pgfqpoint{1.745793in}{1.992641in}}%
\pgfpathcurveto{\pgfqpoint{1.739969in}{1.998465in}}{\pgfqpoint{1.732069in}{2.001737in}}{\pgfqpoint{1.723833in}{2.001737in}}%
\pgfpathcurveto{\pgfqpoint{1.715596in}{2.001737in}}{\pgfqpoint{1.707696in}{1.998465in}}{\pgfqpoint{1.701872in}{1.992641in}}%
\pgfpathcurveto{\pgfqpoint{1.696049in}{1.986817in}}{\pgfqpoint{1.692776in}{1.978917in}}{\pgfqpoint{1.692776in}{1.970681in}}%
\pgfpathcurveto{\pgfqpoint{1.692776in}{1.962444in}}{\pgfqpoint{1.696049in}{1.954544in}}{\pgfqpoint{1.701872in}{1.948720in}}%
\pgfpathcurveto{\pgfqpoint{1.707696in}{1.942896in}}{\pgfqpoint{1.715596in}{1.939624in}}{\pgfqpoint{1.723833in}{1.939624in}}%
\pgfpathclose%
\pgfusepath{stroke,fill}%
\end{pgfscope}%
\begin{pgfscope}%
\pgfpathrectangle{\pgfqpoint{0.100000in}{0.212622in}}{\pgfqpoint{3.696000in}{3.696000in}}%
\pgfusepath{clip}%
\pgfsetbuttcap%
\pgfsetroundjoin%
\definecolor{currentfill}{rgb}{0.121569,0.466667,0.705882}%
\pgfsetfillcolor{currentfill}%
\pgfsetfillopacity{0.367702}%
\pgfsetlinewidth{1.003750pt}%
\definecolor{currentstroke}{rgb}{0.121569,0.466667,0.705882}%
\pgfsetstrokecolor{currentstroke}%
\pgfsetstrokeopacity{0.367702}%
\pgfsetdash{}{0pt}%
\pgfpathmoveto{\pgfqpoint{1.714850in}{1.931640in}}%
\pgfpathcurveto{\pgfqpoint{1.723086in}{1.931640in}}{\pgfqpoint{1.730986in}{1.934913in}}{\pgfqpoint{1.736810in}{1.940736in}}%
\pgfpathcurveto{\pgfqpoint{1.742634in}{1.946560in}}{\pgfqpoint{1.745906in}{1.954460in}}{\pgfqpoint{1.745906in}{1.962697in}}%
\pgfpathcurveto{\pgfqpoint{1.745906in}{1.970933in}}{\pgfqpoint{1.742634in}{1.978833in}}{\pgfqpoint{1.736810in}{1.984657in}}%
\pgfpathcurveto{\pgfqpoint{1.730986in}{1.990481in}}{\pgfqpoint{1.723086in}{1.993753in}}{\pgfqpoint{1.714850in}{1.993753in}}%
\pgfpathcurveto{\pgfqpoint{1.706613in}{1.993753in}}{\pgfqpoint{1.698713in}{1.990481in}}{\pgfqpoint{1.692889in}{1.984657in}}%
\pgfpathcurveto{\pgfqpoint{1.687065in}{1.978833in}}{\pgfqpoint{1.683793in}{1.970933in}}{\pgfqpoint{1.683793in}{1.962697in}}%
\pgfpathcurveto{\pgfqpoint{1.683793in}{1.954460in}}{\pgfqpoint{1.687065in}{1.946560in}}{\pgfqpoint{1.692889in}{1.940736in}}%
\pgfpathcurveto{\pgfqpoint{1.698713in}{1.934913in}}{\pgfqpoint{1.706613in}{1.931640in}}{\pgfqpoint{1.714850in}{1.931640in}}%
\pgfpathclose%
\pgfusepath{stroke,fill}%
\end{pgfscope}%
\begin{pgfscope}%
\pgfpathrectangle{\pgfqpoint{0.100000in}{0.212622in}}{\pgfqpoint{3.696000in}{3.696000in}}%
\pgfusepath{clip}%
\pgfsetbuttcap%
\pgfsetroundjoin%
\definecolor{currentfill}{rgb}{0.121569,0.466667,0.705882}%
\pgfsetfillcolor{currentfill}%
\pgfsetfillopacity{0.370279}%
\pgfsetlinewidth{1.003750pt}%
\definecolor{currentstroke}{rgb}{0.121569,0.466667,0.705882}%
\pgfsetstrokecolor{currentstroke}%
\pgfsetstrokeopacity{0.370279}%
\pgfsetdash{}{0pt}%
\pgfpathmoveto{\pgfqpoint{1.993279in}{2.027046in}}%
\pgfpathcurveto{\pgfqpoint{2.001515in}{2.027046in}}{\pgfqpoint{2.009416in}{2.030319in}}{\pgfqpoint{2.015239in}{2.036143in}}%
\pgfpathcurveto{\pgfqpoint{2.021063in}{2.041966in}}{\pgfqpoint{2.024336in}{2.049866in}}{\pgfqpoint{2.024336in}{2.058103in}}%
\pgfpathcurveto{\pgfqpoint{2.024336in}{2.066339in}}{\pgfqpoint{2.021063in}{2.074239in}}{\pgfqpoint{2.015239in}{2.080063in}}%
\pgfpathcurveto{\pgfqpoint{2.009416in}{2.085887in}}{\pgfqpoint{2.001515in}{2.089159in}}{\pgfqpoint{1.993279in}{2.089159in}}%
\pgfpathcurveto{\pgfqpoint{1.985043in}{2.089159in}}{\pgfqpoint{1.977143in}{2.085887in}}{\pgfqpoint{1.971319in}{2.080063in}}%
\pgfpathcurveto{\pgfqpoint{1.965495in}{2.074239in}}{\pgfqpoint{1.962223in}{2.066339in}}{\pgfqpoint{1.962223in}{2.058103in}}%
\pgfpathcurveto{\pgfqpoint{1.962223in}{2.049866in}}{\pgfqpoint{1.965495in}{2.041966in}}{\pgfqpoint{1.971319in}{2.036143in}}%
\pgfpathcurveto{\pgfqpoint{1.977143in}{2.030319in}}{\pgfqpoint{1.985043in}{2.027046in}}{\pgfqpoint{1.993279in}{2.027046in}}%
\pgfpathclose%
\pgfusepath{stroke,fill}%
\end{pgfscope}%
\begin{pgfscope}%
\pgfpathrectangle{\pgfqpoint{0.100000in}{0.212622in}}{\pgfqpoint{3.696000in}{3.696000in}}%
\pgfusepath{clip}%
\pgfsetbuttcap%
\pgfsetroundjoin%
\definecolor{currentfill}{rgb}{0.121569,0.466667,0.705882}%
\pgfsetfillcolor{currentfill}%
\pgfsetfillopacity{0.370839}%
\pgfsetlinewidth{1.003750pt}%
\definecolor{currentstroke}{rgb}{0.121569,0.466667,0.705882}%
\pgfsetstrokecolor{currentstroke}%
\pgfsetstrokeopacity{0.370839}%
\pgfsetdash{}{0pt}%
\pgfpathmoveto{\pgfqpoint{1.706593in}{1.930700in}}%
\pgfpathcurveto{\pgfqpoint{1.714830in}{1.930700in}}{\pgfqpoint{1.722730in}{1.933972in}}{\pgfqpoint{1.728554in}{1.939796in}}%
\pgfpathcurveto{\pgfqpoint{1.734378in}{1.945620in}}{\pgfqpoint{1.737650in}{1.953520in}}{\pgfqpoint{1.737650in}{1.961757in}}%
\pgfpathcurveto{\pgfqpoint{1.737650in}{1.969993in}}{\pgfqpoint{1.734378in}{1.977893in}}{\pgfqpoint{1.728554in}{1.983717in}}%
\pgfpathcurveto{\pgfqpoint{1.722730in}{1.989541in}}{\pgfqpoint{1.714830in}{1.992813in}}{\pgfqpoint{1.706593in}{1.992813in}}%
\pgfpathcurveto{\pgfqpoint{1.698357in}{1.992813in}}{\pgfqpoint{1.690457in}{1.989541in}}{\pgfqpoint{1.684633in}{1.983717in}}%
\pgfpathcurveto{\pgfqpoint{1.678809in}{1.977893in}}{\pgfqpoint{1.675537in}{1.969993in}}{\pgfqpoint{1.675537in}{1.961757in}}%
\pgfpathcurveto{\pgfqpoint{1.675537in}{1.953520in}}{\pgfqpoint{1.678809in}{1.945620in}}{\pgfqpoint{1.684633in}{1.939796in}}%
\pgfpathcurveto{\pgfqpoint{1.690457in}{1.933972in}}{\pgfqpoint{1.698357in}{1.930700in}}{\pgfqpoint{1.706593in}{1.930700in}}%
\pgfpathclose%
\pgfusepath{stroke,fill}%
\end{pgfscope}%
\begin{pgfscope}%
\pgfpathrectangle{\pgfqpoint{0.100000in}{0.212622in}}{\pgfqpoint{3.696000in}{3.696000in}}%
\pgfusepath{clip}%
\pgfsetbuttcap%
\pgfsetroundjoin%
\definecolor{currentfill}{rgb}{0.121569,0.466667,0.705882}%
\pgfsetfillcolor{currentfill}%
\pgfsetfillopacity{0.372786}%
\pgfsetlinewidth{1.003750pt}%
\definecolor{currentstroke}{rgb}{0.121569,0.466667,0.705882}%
\pgfsetstrokecolor{currentstroke}%
\pgfsetstrokeopacity{0.372786}%
\pgfsetdash{}{0pt}%
\pgfpathmoveto{\pgfqpoint{1.699975in}{1.924270in}}%
\pgfpathcurveto{\pgfqpoint{1.708211in}{1.924270in}}{\pgfqpoint{1.716111in}{1.927542in}}{\pgfqpoint{1.721935in}{1.933366in}}%
\pgfpathcurveto{\pgfqpoint{1.727759in}{1.939190in}}{\pgfqpoint{1.731031in}{1.947090in}}{\pgfqpoint{1.731031in}{1.955326in}}%
\pgfpathcurveto{\pgfqpoint{1.731031in}{1.963563in}}{\pgfqpoint{1.727759in}{1.971463in}}{\pgfqpoint{1.721935in}{1.977287in}}%
\pgfpathcurveto{\pgfqpoint{1.716111in}{1.983111in}}{\pgfqpoint{1.708211in}{1.986383in}}{\pgfqpoint{1.699975in}{1.986383in}}%
\pgfpathcurveto{\pgfqpoint{1.691739in}{1.986383in}}{\pgfqpoint{1.683839in}{1.983111in}}{\pgfqpoint{1.678015in}{1.977287in}}%
\pgfpathcurveto{\pgfqpoint{1.672191in}{1.971463in}}{\pgfqpoint{1.668918in}{1.963563in}}{\pgfqpoint{1.668918in}{1.955326in}}%
\pgfpathcurveto{\pgfqpoint{1.668918in}{1.947090in}}{\pgfqpoint{1.672191in}{1.939190in}}{\pgfqpoint{1.678015in}{1.933366in}}%
\pgfpathcurveto{\pgfqpoint{1.683839in}{1.927542in}}{\pgfqpoint{1.691739in}{1.924270in}}{\pgfqpoint{1.699975in}{1.924270in}}%
\pgfpathclose%
\pgfusepath{stroke,fill}%
\end{pgfscope}%
\begin{pgfscope}%
\pgfpathrectangle{\pgfqpoint{0.100000in}{0.212622in}}{\pgfqpoint{3.696000in}{3.696000in}}%
\pgfusepath{clip}%
\pgfsetbuttcap%
\pgfsetroundjoin%
\definecolor{currentfill}{rgb}{0.121569,0.466667,0.705882}%
\pgfsetfillcolor{currentfill}%
\pgfsetfillopacity{0.374793}%
\pgfsetlinewidth{1.003750pt}%
\definecolor{currentstroke}{rgb}{0.121569,0.466667,0.705882}%
\pgfsetstrokecolor{currentstroke}%
\pgfsetstrokeopacity{0.374793}%
\pgfsetdash{}{0pt}%
\pgfpathmoveto{\pgfqpoint{1.693105in}{1.919975in}}%
\pgfpathcurveto{\pgfqpoint{1.701342in}{1.919975in}}{\pgfqpoint{1.709242in}{1.923247in}}{\pgfqpoint{1.715066in}{1.929071in}}%
\pgfpathcurveto{\pgfqpoint{1.720890in}{1.934895in}}{\pgfqpoint{1.724162in}{1.942795in}}{\pgfqpoint{1.724162in}{1.951031in}}%
\pgfpathcurveto{\pgfqpoint{1.724162in}{1.959267in}}{\pgfqpoint{1.720890in}{1.967167in}}{\pgfqpoint{1.715066in}{1.972991in}}%
\pgfpathcurveto{\pgfqpoint{1.709242in}{1.978815in}}{\pgfqpoint{1.701342in}{1.982087in}}{\pgfqpoint{1.693105in}{1.982087in}}%
\pgfpathcurveto{\pgfqpoint{1.684869in}{1.982087in}}{\pgfqpoint{1.676969in}{1.978815in}}{\pgfqpoint{1.671145in}{1.972991in}}%
\pgfpathcurveto{\pgfqpoint{1.665321in}{1.967167in}}{\pgfqpoint{1.662049in}{1.959267in}}{\pgfqpoint{1.662049in}{1.951031in}}%
\pgfpathcurveto{\pgfqpoint{1.662049in}{1.942795in}}{\pgfqpoint{1.665321in}{1.934895in}}{\pgfqpoint{1.671145in}{1.929071in}}%
\pgfpathcurveto{\pgfqpoint{1.676969in}{1.923247in}}{\pgfqpoint{1.684869in}{1.919975in}}{\pgfqpoint{1.693105in}{1.919975in}}%
\pgfpathclose%
\pgfusepath{stroke,fill}%
\end{pgfscope}%
\begin{pgfscope}%
\pgfpathrectangle{\pgfqpoint{0.100000in}{0.212622in}}{\pgfqpoint{3.696000in}{3.696000in}}%
\pgfusepath{clip}%
\pgfsetbuttcap%
\pgfsetroundjoin%
\definecolor{currentfill}{rgb}{0.121569,0.466667,0.705882}%
\pgfsetfillcolor{currentfill}%
\pgfsetfillopacity{0.376719}%
\pgfsetlinewidth{1.003750pt}%
\definecolor{currentstroke}{rgb}{0.121569,0.466667,0.705882}%
\pgfsetstrokecolor{currentstroke}%
\pgfsetstrokeopacity{0.376719}%
\pgfsetdash{}{0pt}%
\pgfpathmoveto{\pgfqpoint{1.687395in}{1.916497in}}%
\pgfpathcurveto{\pgfqpoint{1.695632in}{1.916497in}}{\pgfqpoint{1.703532in}{1.919769in}}{\pgfqpoint{1.709356in}{1.925593in}}%
\pgfpathcurveto{\pgfqpoint{1.715180in}{1.931417in}}{\pgfqpoint{1.718452in}{1.939317in}}{\pgfqpoint{1.718452in}{1.947553in}}%
\pgfpathcurveto{\pgfqpoint{1.718452in}{1.955790in}}{\pgfqpoint{1.715180in}{1.963690in}}{\pgfqpoint{1.709356in}{1.969514in}}%
\pgfpathcurveto{\pgfqpoint{1.703532in}{1.975338in}}{\pgfqpoint{1.695632in}{1.978610in}}{\pgfqpoint{1.687395in}{1.978610in}}%
\pgfpathcurveto{\pgfqpoint{1.679159in}{1.978610in}}{\pgfqpoint{1.671259in}{1.975338in}}{\pgfqpoint{1.665435in}{1.969514in}}%
\pgfpathcurveto{\pgfqpoint{1.659611in}{1.963690in}}{\pgfqpoint{1.656339in}{1.955790in}}{\pgfqpoint{1.656339in}{1.947553in}}%
\pgfpathcurveto{\pgfqpoint{1.656339in}{1.939317in}}{\pgfqpoint{1.659611in}{1.931417in}}{\pgfqpoint{1.665435in}{1.925593in}}%
\pgfpathcurveto{\pgfqpoint{1.671259in}{1.919769in}}{\pgfqpoint{1.679159in}{1.916497in}}{\pgfqpoint{1.687395in}{1.916497in}}%
\pgfpathclose%
\pgfusepath{stroke,fill}%
\end{pgfscope}%
\begin{pgfscope}%
\pgfpathrectangle{\pgfqpoint{0.100000in}{0.212622in}}{\pgfqpoint{3.696000in}{3.696000in}}%
\pgfusepath{clip}%
\pgfsetbuttcap%
\pgfsetroundjoin%
\definecolor{currentfill}{rgb}{0.121569,0.466667,0.705882}%
\pgfsetfillcolor{currentfill}%
\pgfsetfillopacity{0.376981}%
\pgfsetlinewidth{1.003750pt}%
\definecolor{currentstroke}{rgb}{0.121569,0.466667,0.705882}%
\pgfsetstrokecolor{currentstroke}%
\pgfsetstrokeopacity{0.376981}%
\pgfsetdash{}{0pt}%
\pgfpathmoveto{\pgfqpoint{1.995215in}{2.025193in}}%
\pgfpathcurveto{\pgfqpoint{2.003452in}{2.025193in}}{\pgfqpoint{2.011352in}{2.028465in}}{\pgfqpoint{2.017176in}{2.034289in}}%
\pgfpathcurveto{\pgfqpoint{2.022999in}{2.040113in}}{\pgfqpoint{2.026272in}{2.048013in}}{\pgfqpoint{2.026272in}{2.056249in}}%
\pgfpathcurveto{\pgfqpoint{2.026272in}{2.064486in}}{\pgfqpoint{2.022999in}{2.072386in}}{\pgfqpoint{2.017176in}{2.078210in}}%
\pgfpathcurveto{\pgfqpoint{2.011352in}{2.084034in}}{\pgfqpoint{2.003452in}{2.087306in}}{\pgfqpoint{1.995215in}{2.087306in}}%
\pgfpathcurveto{\pgfqpoint{1.986979in}{2.087306in}}{\pgfqpoint{1.979079in}{2.084034in}}{\pgfqpoint{1.973255in}{2.078210in}}%
\pgfpathcurveto{\pgfqpoint{1.967431in}{2.072386in}}{\pgfqpoint{1.964159in}{2.064486in}}{\pgfqpoint{1.964159in}{2.056249in}}%
\pgfpathcurveto{\pgfqpoint{1.964159in}{2.048013in}}{\pgfqpoint{1.967431in}{2.040113in}}{\pgfqpoint{1.973255in}{2.034289in}}%
\pgfpathcurveto{\pgfqpoint{1.979079in}{2.028465in}}{\pgfqpoint{1.986979in}{2.025193in}}{\pgfqpoint{1.995215in}{2.025193in}}%
\pgfpathclose%
\pgfusepath{stroke,fill}%
\end{pgfscope}%
\begin{pgfscope}%
\pgfpathrectangle{\pgfqpoint{0.100000in}{0.212622in}}{\pgfqpoint{3.696000in}{3.696000in}}%
\pgfusepath{clip}%
\pgfsetbuttcap%
\pgfsetroundjoin%
\definecolor{currentfill}{rgb}{0.121569,0.466667,0.705882}%
\pgfsetfillcolor{currentfill}%
\pgfsetfillopacity{0.377983}%
\pgfsetlinewidth{1.003750pt}%
\definecolor{currentstroke}{rgb}{0.121569,0.466667,0.705882}%
\pgfsetstrokecolor{currentstroke}%
\pgfsetstrokeopacity{0.377983}%
\pgfsetdash{}{0pt}%
\pgfpathmoveto{\pgfqpoint{1.682798in}{1.910164in}}%
\pgfpathcurveto{\pgfqpoint{1.691034in}{1.910164in}}{\pgfqpoint{1.698934in}{1.913436in}}{\pgfqpoint{1.704758in}{1.919260in}}%
\pgfpathcurveto{\pgfqpoint{1.710582in}{1.925084in}}{\pgfqpoint{1.713854in}{1.932984in}}{\pgfqpoint{1.713854in}{1.941220in}}%
\pgfpathcurveto{\pgfqpoint{1.713854in}{1.949456in}}{\pgfqpoint{1.710582in}{1.957356in}}{\pgfqpoint{1.704758in}{1.963180in}}%
\pgfpathcurveto{\pgfqpoint{1.698934in}{1.969004in}}{\pgfqpoint{1.691034in}{1.972277in}}{\pgfqpoint{1.682798in}{1.972277in}}%
\pgfpathcurveto{\pgfqpoint{1.674562in}{1.972277in}}{\pgfqpoint{1.666662in}{1.969004in}}{\pgfqpoint{1.660838in}{1.963180in}}%
\pgfpathcurveto{\pgfqpoint{1.655014in}{1.957356in}}{\pgfqpoint{1.651741in}{1.949456in}}{\pgfqpoint{1.651741in}{1.941220in}}%
\pgfpathcurveto{\pgfqpoint{1.651741in}{1.932984in}}{\pgfqpoint{1.655014in}{1.925084in}}{\pgfqpoint{1.660838in}{1.919260in}}%
\pgfpathcurveto{\pgfqpoint{1.666662in}{1.913436in}}{\pgfqpoint{1.674562in}{1.910164in}}{\pgfqpoint{1.682798in}{1.910164in}}%
\pgfpathclose%
\pgfusepath{stroke,fill}%
\end{pgfscope}%
\begin{pgfscope}%
\pgfpathrectangle{\pgfqpoint{0.100000in}{0.212622in}}{\pgfqpoint{3.696000in}{3.696000in}}%
\pgfusepath{clip}%
\pgfsetbuttcap%
\pgfsetroundjoin%
\definecolor{currentfill}{rgb}{0.121569,0.466667,0.705882}%
\pgfsetfillcolor{currentfill}%
\pgfsetfillopacity{0.378875}%
\pgfsetlinewidth{1.003750pt}%
\definecolor{currentstroke}{rgb}{0.121569,0.466667,0.705882}%
\pgfsetstrokecolor{currentstroke}%
\pgfsetstrokeopacity{0.378875}%
\pgfsetdash{}{0pt}%
\pgfpathmoveto{\pgfqpoint{1.679857in}{1.907925in}}%
\pgfpathcurveto{\pgfqpoint{1.688093in}{1.907925in}}{\pgfqpoint{1.695993in}{1.911197in}}{\pgfqpoint{1.701817in}{1.917021in}}%
\pgfpathcurveto{\pgfqpoint{1.707641in}{1.922845in}}{\pgfqpoint{1.710913in}{1.930745in}}{\pgfqpoint{1.710913in}{1.938981in}}%
\pgfpathcurveto{\pgfqpoint{1.710913in}{1.947217in}}{\pgfqpoint{1.707641in}{1.955117in}}{\pgfqpoint{1.701817in}{1.960941in}}%
\pgfpathcurveto{\pgfqpoint{1.695993in}{1.966765in}}{\pgfqpoint{1.688093in}{1.970038in}}{\pgfqpoint{1.679857in}{1.970038in}}%
\pgfpathcurveto{\pgfqpoint{1.671620in}{1.970038in}}{\pgfqpoint{1.663720in}{1.966765in}}{\pgfqpoint{1.657896in}{1.960941in}}%
\pgfpathcurveto{\pgfqpoint{1.652073in}{1.955117in}}{\pgfqpoint{1.648800in}{1.947217in}}{\pgfqpoint{1.648800in}{1.938981in}}%
\pgfpathcurveto{\pgfqpoint{1.648800in}{1.930745in}}{\pgfqpoint{1.652073in}{1.922845in}}{\pgfqpoint{1.657896in}{1.917021in}}%
\pgfpathcurveto{\pgfqpoint{1.663720in}{1.911197in}}{\pgfqpoint{1.671620in}{1.907925in}}{\pgfqpoint{1.679857in}{1.907925in}}%
\pgfpathclose%
\pgfusepath{stroke,fill}%
\end{pgfscope}%
\begin{pgfscope}%
\pgfpathrectangle{\pgfqpoint{0.100000in}{0.212622in}}{\pgfqpoint{3.696000in}{3.696000in}}%
\pgfusepath{clip}%
\pgfsetbuttcap%
\pgfsetroundjoin%
\definecolor{currentfill}{rgb}{0.121569,0.466667,0.705882}%
\pgfsetfillcolor{currentfill}%
\pgfsetfillopacity{0.379619}%
\pgfsetlinewidth{1.003750pt}%
\definecolor{currentstroke}{rgb}{0.121569,0.466667,0.705882}%
\pgfsetstrokecolor{currentstroke}%
\pgfsetstrokeopacity{0.379619}%
\pgfsetdash{}{0pt}%
\pgfpathmoveto{\pgfqpoint{1.677337in}{1.906266in}}%
\pgfpathcurveto{\pgfqpoint{1.685573in}{1.906266in}}{\pgfqpoint{1.693473in}{1.909538in}}{\pgfqpoint{1.699297in}{1.915362in}}%
\pgfpathcurveto{\pgfqpoint{1.705121in}{1.921186in}}{\pgfqpoint{1.708393in}{1.929086in}}{\pgfqpoint{1.708393in}{1.937322in}}%
\pgfpathcurveto{\pgfqpoint{1.708393in}{1.945558in}}{\pgfqpoint{1.705121in}{1.953458in}}{\pgfqpoint{1.699297in}{1.959282in}}%
\pgfpathcurveto{\pgfqpoint{1.693473in}{1.965106in}}{\pgfqpoint{1.685573in}{1.968379in}}{\pgfqpoint{1.677337in}{1.968379in}}%
\pgfpathcurveto{\pgfqpoint{1.669101in}{1.968379in}}{\pgfqpoint{1.661201in}{1.965106in}}{\pgfqpoint{1.655377in}{1.959282in}}%
\pgfpathcurveto{\pgfqpoint{1.649553in}{1.953458in}}{\pgfqpoint{1.646280in}{1.945558in}}{\pgfqpoint{1.646280in}{1.937322in}}%
\pgfpathcurveto{\pgfqpoint{1.646280in}{1.929086in}}{\pgfqpoint{1.649553in}{1.921186in}}{\pgfqpoint{1.655377in}{1.915362in}}%
\pgfpathcurveto{\pgfqpoint{1.661201in}{1.909538in}}{\pgfqpoint{1.669101in}{1.906266in}}{\pgfqpoint{1.677337in}{1.906266in}}%
\pgfpathclose%
\pgfusepath{stroke,fill}%
\end{pgfscope}%
\begin{pgfscope}%
\pgfpathrectangle{\pgfqpoint{0.100000in}{0.212622in}}{\pgfqpoint{3.696000in}{3.696000in}}%
\pgfusepath{clip}%
\pgfsetbuttcap%
\pgfsetroundjoin%
\definecolor{currentfill}{rgb}{0.121569,0.466667,0.705882}%
\pgfsetfillcolor{currentfill}%
\pgfsetfillopacity{0.380790}%
\pgfsetlinewidth{1.003750pt}%
\definecolor{currentstroke}{rgb}{0.121569,0.466667,0.705882}%
\pgfsetstrokecolor{currentstroke}%
\pgfsetstrokeopacity{0.380790}%
\pgfsetdash{}{0pt}%
\pgfpathmoveto{\pgfqpoint{1.673425in}{1.901120in}}%
\pgfpathcurveto{\pgfqpoint{1.681661in}{1.901120in}}{\pgfqpoint{1.689561in}{1.904392in}}{\pgfqpoint{1.695385in}{1.910216in}}%
\pgfpathcurveto{\pgfqpoint{1.701209in}{1.916040in}}{\pgfqpoint{1.704482in}{1.923940in}}{\pgfqpoint{1.704482in}{1.932176in}}%
\pgfpathcurveto{\pgfqpoint{1.704482in}{1.940412in}}{\pgfqpoint{1.701209in}{1.948313in}}{\pgfqpoint{1.695385in}{1.954136in}}%
\pgfpathcurveto{\pgfqpoint{1.689561in}{1.959960in}}{\pgfqpoint{1.681661in}{1.963233in}}{\pgfqpoint{1.673425in}{1.963233in}}%
\pgfpathcurveto{\pgfqpoint{1.665189in}{1.963233in}}{\pgfqpoint{1.657289in}{1.959960in}}{\pgfqpoint{1.651465in}{1.954136in}}%
\pgfpathcurveto{\pgfqpoint{1.645641in}{1.948313in}}{\pgfqpoint{1.642369in}{1.940412in}}{\pgfqpoint{1.642369in}{1.932176in}}%
\pgfpathcurveto{\pgfqpoint{1.642369in}{1.923940in}}{\pgfqpoint{1.645641in}{1.916040in}}{\pgfqpoint{1.651465in}{1.910216in}}%
\pgfpathcurveto{\pgfqpoint{1.657289in}{1.904392in}}{\pgfqpoint{1.665189in}{1.901120in}}{\pgfqpoint{1.673425in}{1.901120in}}%
\pgfpathclose%
\pgfusepath{stroke,fill}%
\end{pgfscope}%
\begin{pgfscope}%
\pgfpathrectangle{\pgfqpoint{0.100000in}{0.212622in}}{\pgfqpoint{3.696000in}{3.696000in}}%
\pgfusepath{clip}%
\pgfsetbuttcap%
\pgfsetroundjoin%
\definecolor{currentfill}{rgb}{0.121569,0.466667,0.705882}%
\pgfsetfillcolor{currentfill}%
\pgfsetfillopacity{0.381319}%
\pgfsetlinewidth{1.003750pt}%
\definecolor{currentstroke}{rgb}{0.121569,0.466667,0.705882}%
\pgfsetstrokecolor{currentstroke}%
\pgfsetstrokeopacity{0.381319}%
\pgfsetdash{}{0pt}%
\pgfpathmoveto{\pgfqpoint{1.671356in}{1.899844in}}%
\pgfpathcurveto{\pgfqpoint{1.679592in}{1.899844in}}{\pgfqpoint{1.687492in}{1.903116in}}{\pgfqpoint{1.693316in}{1.908940in}}%
\pgfpathcurveto{\pgfqpoint{1.699140in}{1.914764in}}{\pgfqpoint{1.702413in}{1.922664in}}{\pgfqpoint{1.702413in}{1.930900in}}%
\pgfpathcurveto{\pgfqpoint{1.702413in}{1.939137in}}{\pgfqpoint{1.699140in}{1.947037in}}{\pgfqpoint{1.693316in}{1.952861in}}%
\pgfpathcurveto{\pgfqpoint{1.687492in}{1.958684in}}{\pgfqpoint{1.679592in}{1.961957in}}{\pgfqpoint{1.671356in}{1.961957in}}%
\pgfpathcurveto{\pgfqpoint{1.663120in}{1.961957in}}{\pgfqpoint{1.655220in}{1.958684in}}{\pgfqpoint{1.649396in}{1.952861in}}%
\pgfpathcurveto{\pgfqpoint{1.643572in}{1.947037in}}{\pgfqpoint{1.640300in}{1.939137in}}{\pgfqpoint{1.640300in}{1.930900in}}%
\pgfpathcurveto{\pgfqpoint{1.640300in}{1.922664in}}{\pgfqpoint{1.643572in}{1.914764in}}{\pgfqpoint{1.649396in}{1.908940in}}%
\pgfpathcurveto{\pgfqpoint{1.655220in}{1.903116in}}{\pgfqpoint{1.663120in}{1.899844in}}{\pgfqpoint{1.671356in}{1.899844in}}%
\pgfpathclose%
\pgfusepath{stroke,fill}%
\end{pgfscope}%
\begin{pgfscope}%
\pgfpathrectangle{\pgfqpoint{0.100000in}{0.212622in}}{\pgfqpoint{3.696000in}{3.696000in}}%
\pgfusepath{clip}%
\pgfsetbuttcap%
\pgfsetroundjoin%
\definecolor{currentfill}{rgb}{0.121569,0.466667,0.705882}%
\pgfsetfillcolor{currentfill}%
\pgfsetfillopacity{0.382259}%
\pgfsetlinewidth{1.003750pt}%
\definecolor{currentstroke}{rgb}{0.121569,0.466667,0.705882}%
\pgfsetstrokecolor{currentstroke}%
\pgfsetstrokeopacity{0.382259}%
\pgfsetdash{}{0pt}%
\pgfpathmoveto{\pgfqpoint{1.667875in}{1.896915in}}%
\pgfpathcurveto{\pgfqpoint{1.676111in}{1.896915in}}{\pgfqpoint{1.684011in}{1.900188in}}{\pgfqpoint{1.689835in}{1.906012in}}%
\pgfpathcurveto{\pgfqpoint{1.695659in}{1.911835in}}{\pgfqpoint{1.698932in}{1.919735in}}{\pgfqpoint{1.698932in}{1.927972in}}%
\pgfpathcurveto{\pgfqpoint{1.698932in}{1.936208in}}{\pgfqpoint{1.695659in}{1.944108in}}{\pgfqpoint{1.689835in}{1.949932in}}%
\pgfpathcurveto{\pgfqpoint{1.684011in}{1.955756in}}{\pgfqpoint{1.676111in}{1.959028in}}{\pgfqpoint{1.667875in}{1.959028in}}%
\pgfpathcurveto{\pgfqpoint{1.659639in}{1.959028in}}{\pgfqpoint{1.651739in}{1.955756in}}{\pgfqpoint{1.645915in}{1.949932in}}%
\pgfpathcurveto{\pgfqpoint{1.640091in}{1.944108in}}{\pgfqpoint{1.636819in}{1.936208in}}{\pgfqpoint{1.636819in}{1.927972in}}%
\pgfpathcurveto{\pgfqpoint{1.636819in}{1.919735in}}{\pgfqpoint{1.640091in}{1.911835in}}{\pgfqpoint{1.645915in}{1.906012in}}%
\pgfpathcurveto{\pgfqpoint{1.651739in}{1.900188in}}{\pgfqpoint{1.659639in}{1.896915in}}{\pgfqpoint{1.667875in}{1.896915in}}%
\pgfpathclose%
\pgfusepath{stroke,fill}%
\end{pgfscope}%
\begin{pgfscope}%
\pgfpathrectangle{\pgfqpoint{0.100000in}{0.212622in}}{\pgfqpoint{3.696000in}{3.696000in}}%
\pgfusepath{clip}%
\pgfsetbuttcap%
\pgfsetroundjoin%
\definecolor{currentfill}{rgb}{0.121569,0.466667,0.705882}%
\pgfsetfillcolor{currentfill}%
\pgfsetfillopacity{0.383429}%
\pgfsetlinewidth{1.003750pt}%
\definecolor{currentstroke}{rgb}{0.121569,0.466667,0.705882}%
\pgfsetstrokecolor{currentstroke}%
\pgfsetstrokeopacity{0.383429}%
\pgfsetdash{}{0pt}%
\pgfpathmoveto{\pgfqpoint{1.995974in}{2.014816in}}%
\pgfpathcurveto{\pgfqpoint{2.004211in}{2.014816in}}{\pgfqpoint{2.012111in}{2.018088in}}{\pgfqpoint{2.017935in}{2.023912in}}%
\pgfpathcurveto{\pgfqpoint{2.023759in}{2.029736in}}{\pgfqpoint{2.027031in}{2.037636in}}{\pgfqpoint{2.027031in}{2.045872in}}%
\pgfpathcurveto{\pgfqpoint{2.027031in}{2.054108in}}{\pgfqpoint{2.023759in}{2.062008in}}{\pgfqpoint{2.017935in}{2.067832in}}%
\pgfpathcurveto{\pgfqpoint{2.012111in}{2.073656in}}{\pgfqpoint{2.004211in}{2.076929in}}{\pgfqpoint{1.995974in}{2.076929in}}%
\pgfpathcurveto{\pgfqpoint{1.987738in}{2.076929in}}{\pgfqpoint{1.979838in}{2.073656in}}{\pgfqpoint{1.974014in}{2.067832in}}%
\pgfpathcurveto{\pgfqpoint{1.968190in}{2.062008in}}{\pgfqpoint{1.964918in}{2.054108in}}{\pgfqpoint{1.964918in}{2.045872in}}%
\pgfpathcurveto{\pgfqpoint{1.964918in}{2.037636in}}{\pgfqpoint{1.968190in}{2.029736in}}{\pgfqpoint{1.974014in}{2.023912in}}%
\pgfpathcurveto{\pgfqpoint{1.979838in}{2.018088in}}{\pgfqpoint{1.987738in}{2.014816in}}{\pgfqpoint{1.995974in}{2.014816in}}%
\pgfpathclose%
\pgfusepath{stroke,fill}%
\end{pgfscope}%
\begin{pgfscope}%
\pgfpathrectangle{\pgfqpoint{0.100000in}{0.212622in}}{\pgfqpoint{3.696000in}{3.696000in}}%
\pgfusepath{clip}%
\pgfsetbuttcap%
\pgfsetroundjoin%
\definecolor{currentfill}{rgb}{0.121569,0.466667,0.705882}%
\pgfsetfillcolor{currentfill}%
\pgfsetfillopacity{0.383666}%
\pgfsetlinewidth{1.003750pt}%
\definecolor{currentstroke}{rgb}{0.121569,0.466667,0.705882}%
\pgfsetstrokecolor{currentstroke}%
\pgfsetstrokeopacity{0.383666}%
\pgfsetdash{}{0pt}%
\pgfpathmoveto{\pgfqpoint{1.662488in}{1.888282in}}%
\pgfpathcurveto{\pgfqpoint{1.670724in}{1.888282in}}{\pgfqpoint{1.678624in}{1.891554in}}{\pgfqpoint{1.684448in}{1.897378in}}%
\pgfpathcurveto{\pgfqpoint{1.690272in}{1.903202in}}{\pgfqpoint{1.693544in}{1.911102in}}{\pgfqpoint{1.693544in}{1.919338in}}%
\pgfpathcurveto{\pgfqpoint{1.693544in}{1.927574in}}{\pgfqpoint{1.690272in}{1.935475in}}{\pgfqpoint{1.684448in}{1.941298in}}%
\pgfpathcurveto{\pgfqpoint{1.678624in}{1.947122in}}{\pgfqpoint{1.670724in}{1.950395in}}{\pgfqpoint{1.662488in}{1.950395in}}%
\pgfpathcurveto{\pgfqpoint{1.654252in}{1.950395in}}{\pgfqpoint{1.646352in}{1.947122in}}{\pgfqpoint{1.640528in}{1.941298in}}%
\pgfpathcurveto{\pgfqpoint{1.634704in}{1.935475in}}{\pgfqpoint{1.631431in}{1.927574in}}{\pgfqpoint{1.631431in}{1.919338in}}%
\pgfpathcurveto{\pgfqpoint{1.631431in}{1.911102in}}{\pgfqpoint{1.634704in}{1.903202in}}{\pgfqpoint{1.640528in}{1.897378in}}%
\pgfpathcurveto{\pgfqpoint{1.646352in}{1.891554in}}{\pgfqpoint{1.654252in}{1.888282in}}{\pgfqpoint{1.662488in}{1.888282in}}%
\pgfpathclose%
\pgfusepath{stroke,fill}%
\end{pgfscope}%
\begin{pgfscope}%
\pgfpathrectangle{\pgfqpoint{0.100000in}{0.212622in}}{\pgfqpoint{3.696000in}{3.696000in}}%
\pgfusepath{clip}%
\pgfsetbuttcap%
\pgfsetroundjoin%
\definecolor{currentfill}{rgb}{0.121569,0.466667,0.705882}%
\pgfsetfillcolor{currentfill}%
\pgfsetfillopacity{0.385219}%
\pgfsetlinewidth{1.003750pt}%
\definecolor{currentstroke}{rgb}{0.121569,0.466667,0.705882}%
\pgfsetstrokecolor{currentstroke}%
\pgfsetstrokeopacity{0.385219}%
\pgfsetdash{}{0pt}%
\pgfpathmoveto{\pgfqpoint{1.658288in}{1.887369in}}%
\pgfpathcurveto{\pgfqpoint{1.666524in}{1.887369in}}{\pgfqpoint{1.674424in}{1.890641in}}{\pgfqpoint{1.680248in}{1.896465in}}%
\pgfpathcurveto{\pgfqpoint{1.686072in}{1.902289in}}{\pgfqpoint{1.689344in}{1.910189in}}{\pgfqpoint{1.689344in}{1.918426in}}%
\pgfpathcurveto{\pgfqpoint{1.689344in}{1.926662in}}{\pgfqpoint{1.686072in}{1.934562in}}{\pgfqpoint{1.680248in}{1.940386in}}%
\pgfpathcurveto{\pgfqpoint{1.674424in}{1.946210in}}{\pgfqpoint{1.666524in}{1.949482in}}{\pgfqpoint{1.658288in}{1.949482in}}%
\pgfpathcurveto{\pgfqpoint{1.650052in}{1.949482in}}{\pgfqpoint{1.642151in}{1.946210in}}{\pgfqpoint{1.636328in}{1.940386in}}%
\pgfpathcurveto{\pgfqpoint{1.630504in}{1.934562in}}{\pgfqpoint{1.627231in}{1.926662in}}{\pgfqpoint{1.627231in}{1.918426in}}%
\pgfpathcurveto{\pgfqpoint{1.627231in}{1.910189in}}{\pgfqpoint{1.630504in}{1.902289in}}{\pgfqpoint{1.636328in}{1.896465in}}%
\pgfpathcurveto{\pgfqpoint{1.642151in}{1.890641in}}{\pgfqpoint{1.650052in}{1.887369in}}{\pgfqpoint{1.658288in}{1.887369in}}%
\pgfpathclose%
\pgfusepath{stroke,fill}%
\end{pgfscope}%
\begin{pgfscope}%
\pgfpathrectangle{\pgfqpoint{0.100000in}{0.212622in}}{\pgfqpoint{3.696000in}{3.696000in}}%
\pgfusepath{clip}%
\pgfsetbuttcap%
\pgfsetroundjoin%
\definecolor{currentfill}{rgb}{0.121569,0.466667,0.705882}%
\pgfsetfillcolor{currentfill}%
\pgfsetfillopacity{0.386312}%
\pgfsetlinewidth{1.003750pt}%
\definecolor{currentstroke}{rgb}{0.121569,0.466667,0.705882}%
\pgfsetstrokecolor{currentstroke}%
\pgfsetstrokeopacity{0.386312}%
\pgfsetdash{}{0pt}%
\pgfpathmoveto{\pgfqpoint{1.654959in}{1.886295in}}%
\pgfpathcurveto{\pgfqpoint{1.663196in}{1.886295in}}{\pgfqpoint{1.671096in}{1.889568in}}{\pgfqpoint{1.676920in}{1.895392in}}%
\pgfpathcurveto{\pgfqpoint{1.682743in}{1.901216in}}{\pgfqpoint{1.686016in}{1.909116in}}{\pgfqpoint{1.686016in}{1.917352in}}%
\pgfpathcurveto{\pgfqpoint{1.686016in}{1.925588in}}{\pgfqpoint{1.682743in}{1.933488in}}{\pgfqpoint{1.676920in}{1.939312in}}%
\pgfpathcurveto{\pgfqpoint{1.671096in}{1.945136in}}{\pgfqpoint{1.663196in}{1.948408in}}{\pgfqpoint{1.654959in}{1.948408in}}%
\pgfpathcurveto{\pgfqpoint{1.646723in}{1.948408in}}{\pgfqpoint{1.638823in}{1.945136in}}{\pgfqpoint{1.632999in}{1.939312in}}%
\pgfpathcurveto{\pgfqpoint{1.627175in}{1.933488in}}{\pgfqpoint{1.623903in}{1.925588in}}{\pgfqpoint{1.623903in}{1.917352in}}%
\pgfpathcurveto{\pgfqpoint{1.623903in}{1.909116in}}{\pgfqpoint{1.627175in}{1.901216in}}{\pgfqpoint{1.632999in}{1.895392in}}%
\pgfpathcurveto{\pgfqpoint{1.638823in}{1.889568in}}{\pgfqpoint{1.646723in}{1.886295in}}{\pgfqpoint{1.654959in}{1.886295in}}%
\pgfpathclose%
\pgfusepath{stroke,fill}%
\end{pgfscope}%
\begin{pgfscope}%
\pgfpathrectangle{\pgfqpoint{0.100000in}{0.212622in}}{\pgfqpoint{3.696000in}{3.696000in}}%
\pgfusepath{clip}%
\pgfsetbuttcap%
\pgfsetroundjoin%
\definecolor{currentfill}{rgb}{0.121569,0.466667,0.705882}%
\pgfsetfillcolor{currentfill}%
\pgfsetfillopacity{0.387966}%
\pgfsetlinewidth{1.003750pt}%
\definecolor{currentstroke}{rgb}{0.121569,0.466667,0.705882}%
\pgfsetstrokecolor{currentstroke}%
\pgfsetstrokeopacity{0.387966}%
\pgfsetdash{}{0pt}%
\pgfpathmoveto{\pgfqpoint{1.649248in}{1.881574in}}%
\pgfpathcurveto{\pgfqpoint{1.657485in}{1.881574in}}{\pgfqpoint{1.665385in}{1.884847in}}{\pgfqpoint{1.671209in}{1.890671in}}%
\pgfpathcurveto{\pgfqpoint{1.677033in}{1.896495in}}{\pgfqpoint{1.680305in}{1.904395in}}{\pgfqpoint{1.680305in}{1.912631in}}%
\pgfpathcurveto{\pgfqpoint{1.680305in}{1.920867in}}{\pgfqpoint{1.677033in}{1.928767in}}{\pgfqpoint{1.671209in}{1.934591in}}%
\pgfpathcurveto{\pgfqpoint{1.665385in}{1.940415in}}{\pgfqpoint{1.657485in}{1.943687in}}{\pgfqpoint{1.649248in}{1.943687in}}%
\pgfpathcurveto{\pgfqpoint{1.641012in}{1.943687in}}{\pgfqpoint{1.633112in}{1.940415in}}{\pgfqpoint{1.627288in}{1.934591in}}%
\pgfpathcurveto{\pgfqpoint{1.621464in}{1.928767in}}{\pgfqpoint{1.618192in}{1.920867in}}{\pgfqpoint{1.618192in}{1.912631in}}%
\pgfpathcurveto{\pgfqpoint{1.618192in}{1.904395in}}{\pgfqpoint{1.621464in}{1.896495in}}{\pgfqpoint{1.627288in}{1.890671in}}%
\pgfpathcurveto{\pgfqpoint{1.633112in}{1.884847in}}{\pgfqpoint{1.641012in}{1.881574in}}{\pgfqpoint{1.649248in}{1.881574in}}%
\pgfpathclose%
\pgfusepath{stroke,fill}%
\end{pgfscope}%
\begin{pgfscope}%
\pgfpathrectangle{\pgfqpoint{0.100000in}{0.212622in}}{\pgfqpoint{3.696000in}{3.696000in}}%
\pgfusepath{clip}%
\pgfsetbuttcap%
\pgfsetroundjoin%
\definecolor{currentfill}{rgb}{0.121569,0.466667,0.705882}%
\pgfsetfillcolor{currentfill}%
\pgfsetfillopacity{0.389677}%
\pgfsetlinewidth{1.003750pt}%
\definecolor{currentstroke}{rgb}{0.121569,0.466667,0.705882}%
\pgfsetstrokecolor{currentstroke}%
\pgfsetstrokeopacity{0.389677}%
\pgfsetdash{}{0pt}%
\pgfpathmoveto{\pgfqpoint{1.644460in}{1.881315in}}%
\pgfpathcurveto{\pgfqpoint{1.652696in}{1.881315in}}{\pgfqpoint{1.660596in}{1.884588in}}{\pgfqpoint{1.666420in}{1.890412in}}%
\pgfpathcurveto{\pgfqpoint{1.672244in}{1.896235in}}{\pgfqpoint{1.675516in}{1.904136in}}{\pgfqpoint{1.675516in}{1.912372in}}%
\pgfpathcurveto{\pgfqpoint{1.675516in}{1.920608in}}{\pgfqpoint{1.672244in}{1.928508in}}{\pgfqpoint{1.666420in}{1.934332in}}%
\pgfpathcurveto{\pgfqpoint{1.660596in}{1.940156in}}{\pgfqpoint{1.652696in}{1.943428in}}{\pgfqpoint{1.644460in}{1.943428in}}%
\pgfpathcurveto{\pgfqpoint{1.636224in}{1.943428in}}{\pgfqpoint{1.628324in}{1.940156in}}{\pgfqpoint{1.622500in}{1.934332in}}%
\pgfpathcurveto{\pgfqpoint{1.616676in}{1.928508in}}{\pgfqpoint{1.613403in}{1.920608in}}{\pgfqpoint{1.613403in}{1.912372in}}%
\pgfpathcurveto{\pgfqpoint{1.613403in}{1.904136in}}{\pgfqpoint{1.616676in}{1.896235in}}{\pgfqpoint{1.622500in}{1.890412in}}%
\pgfpathcurveto{\pgfqpoint{1.628324in}{1.884588in}}{\pgfqpoint{1.636224in}{1.881315in}}{\pgfqpoint{1.644460in}{1.881315in}}%
\pgfpathclose%
\pgfusepath{stroke,fill}%
\end{pgfscope}%
\begin{pgfscope}%
\pgfpathrectangle{\pgfqpoint{0.100000in}{0.212622in}}{\pgfqpoint{3.696000in}{3.696000in}}%
\pgfusepath{clip}%
\pgfsetbuttcap%
\pgfsetroundjoin%
\definecolor{currentfill}{rgb}{0.121569,0.466667,0.705882}%
\pgfsetfillcolor{currentfill}%
\pgfsetfillopacity{0.390359}%
\pgfsetlinewidth{1.003750pt}%
\definecolor{currentstroke}{rgb}{0.121569,0.466667,0.705882}%
\pgfsetstrokecolor{currentstroke}%
\pgfsetstrokeopacity{0.390359}%
\pgfsetdash{}{0pt}%
\pgfpathmoveto{\pgfqpoint{1.997699in}{2.005528in}}%
\pgfpathcurveto{\pgfqpoint{2.005936in}{2.005528in}}{\pgfqpoint{2.013836in}{2.008800in}}{\pgfqpoint{2.019660in}{2.014624in}}%
\pgfpathcurveto{\pgfqpoint{2.025483in}{2.020448in}}{\pgfqpoint{2.028756in}{2.028348in}}{\pgfqpoint{2.028756in}{2.036584in}}%
\pgfpathcurveto{\pgfqpoint{2.028756in}{2.044821in}}{\pgfqpoint{2.025483in}{2.052721in}}{\pgfqpoint{2.019660in}{2.058545in}}%
\pgfpathcurveto{\pgfqpoint{2.013836in}{2.064369in}}{\pgfqpoint{2.005936in}{2.067641in}}{\pgfqpoint{1.997699in}{2.067641in}}%
\pgfpathcurveto{\pgfqpoint{1.989463in}{2.067641in}}{\pgfqpoint{1.981563in}{2.064369in}}{\pgfqpoint{1.975739in}{2.058545in}}%
\pgfpathcurveto{\pgfqpoint{1.969915in}{2.052721in}}{\pgfqpoint{1.966643in}{2.044821in}}{\pgfqpoint{1.966643in}{2.036584in}}%
\pgfpathcurveto{\pgfqpoint{1.966643in}{2.028348in}}{\pgfqpoint{1.969915in}{2.020448in}}{\pgfqpoint{1.975739in}{2.014624in}}%
\pgfpathcurveto{\pgfqpoint{1.981563in}{2.008800in}}{\pgfqpoint{1.989463in}{2.005528in}}{\pgfqpoint{1.997699in}{2.005528in}}%
\pgfpathclose%
\pgfusepath{stroke,fill}%
\end{pgfscope}%
\begin{pgfscope}%
\pgfpathrectangle{\pgfqpoint{0.100000in}{0.212622in}}{\pgfqpoint{3.696000in}{3.696000in}}%
\pgfusepath{clip}%
\pgfsetbuttcap%
\pgfsetroundjoin%
\definecolor{currentfill}{rgb}{0.121569,0.466667,0.705882}%
\pgfsetfillcolor{currentfill}%
\pgfsetfillopacity{0.390817}%
\pgfsetlinewidth{1.003750pt}%
\definecolor{currentstroke}{rgb}{0.121569,0.466667,0.705882}%
\pgfsetstrokecolor{currentstroke}%
\pgfsetstrokeopacity{0.390817}%
\pgfsetdash{}{0pt}%
\pgfpathmoveto{\pgfqpoint{1.640760in}{1.878705in}}%
\pgfpathcurveto{\pgfqpoint{1.648996in}{1.878705in}}{\pgfqpoint{1.656896in}{1.881977in}}{\pgfqpoint{1.662720in}{1.887801in}}%
\pgfpathcurveto{\pgfqpoint{1.668544in}{1.893625in}}{\pgfqpoint{1.671816in}{1.901525in}}{\pgfqpoint{1.671816in}{1.909762in}}%
\pgfpathcurveto{\pgfqpoint{1.671816in}{1.917998in}}{\pgfqpoint{1.668544in}{1.925898in}}{\pgfqpoint{1.662720in}{1.931722in}}%
\pgfpathcurveto{\pgfqpoint{1.656896in}{1.937546in}}{\pgfqpoint{1.648996in}{1.940818in}}{\pgfqpoint{1.640760in}{1.940818in}}%
\pgfpathcurveto{\pgfqpoint{1.632524in}{1.940818in}}{\pgfqpoint{1.624624in}{1.937546in}}{\pgfqpoint{1.618800in}{1.931722in}}%
\pgfpathcurveto{\pgfqpoint{1.612976in}{1.925898in}}{\pgfqpoint{1.609703in}{1.917998in}}{\pgfqpoint{1.609703in}{1.909762in}}%
\pgfpathcurveto{\pgfqpoint{1.609703in}{1.901525in}}{\pgfqpoint{1.612976in}{1.893625in}}{\pgfqpoint{1.618800in}{1.887801in}}%
\pgfpathcurveto{\pgfqpoint{1.624624in}{1.881977in}}{\pgfqpoint{1.632524in}{1.878705in}}{\pgfqpoint{1.640760in}{1.878705in}}%
\pgfpathclose%
\pgfusepath{stroke,fill}%
\end{pgfscope}%
\begin{pgfscope}%
\pgfpathrectangle{\pgfqpoint{0.100000in}{0.212622in}}{\pgfqpoint{3.696000in}{3.696000in}}%
\pgfusepath{clip}%
\pgfsetbuttcap%
\pgfsetroundjoin%
\definecolor{currentfill}{rgb}{0.121569,0.466667,0.705882}%
\pgfsetfillcolor{currentfill}%
\pgfsetfillopacity{0.392759}%
\pgfsetlinewidth{1.003750pt}%
\definecolor{currentstroke}{rgb}{0.121569,0.466667,0.705882}%
\pgfsetstrokecolor{currentstroke}%
\pgfsetstrokeopacity{0.392759}%
\pgfsetdash{}{0pt}%
\pgfpathmoveto{\pgfqpoint{1.634612in}{1.872289in}}%
\pgfpathcurveto{\pgfqpoint{1.642849in}{1.872289in}}{\pgfqpoint{1.650749in}{1.875561in}}{\pgfqpoint{1.656573in}{1.881385in}}%
\pgfpathcurveto{\pgfqpoint{1.662396in}{1.887209in}}{\pgfqpoint{1.665669in}{1.895109in}}{\pgfqpoint{1.665669in}{1.903345in}}%
\pgfpathcurveto{\pgfqpoint{1.665669in}{1.911582in}}{\pgfqpoint{1.662396in}{1.919482in}}{\pgfqpoint{1.656573in}{1.925306in}}%
\pgfpathcurveto{\pgfqpoint{1.650749in}{1.931130in}}{\pgfqpoint{1.642849in}{1.934402in}}{\pgfqpoint{1.634612in}{1.934402in}}%
\pgfpathcurveto{\pgfqpoint{1.626376in}{1.934402in}}{\pgfqpoint{1.618476in}{1.931130in}}{\pgfqpoint{1.612652in}{1.925306in}}%
\pgfpathcurveto{\pgfqpoint{1.606828in}{1.919482in}}{\pgfqpoint{1.603556in}{1.911582in}}{\pgfqpoint{1.603556in}{1.903345in}}%
\pgfpathcurveto{\pgfqpoint{1.603556in}{1.895109in}}{\pgfqpoint{1.606828in}{1.887209in}}{\pgfqpoint{1.612652in}{1.881385in}}%
\pgfpathcurveto{\pgfqpoint{1.618476in}{1.875561in}}{\pgfqpoint{1.626376in}{1.872289in}}{\pgfqpoint{1.634612in}{1.872289in}}%
\pgfpathclose%
\pgfusepath{stroke,fill}%
\end{pgfscope}%
\begin{pgfscope}%
\pgfpathrectangle{\pgfqpoint{0.100000in}{0.212622in}}{\pgfqpoint{3.696000in}{3.696000in}}%
\pgfusepath{clip}%
\pgfsetbuttcap%
\pgfsetroundjoin%
\definecolor{currentfill}{rgb}{0.121569,0.466667,0.705882}%
\pgfsetfillcolor{currentfill}%
\pgfsetfillopacity{0.393843}%
\pgfsetlinewidth{1.003750pt}%
\definecolor{currentstroke}{rgb}{0.121569,0.466667,0.705882}%
\pgfsetstrokecolor{currentstroke}%
\pgfsetstrokeopacity{0.393843}%
\pgfsetdash{}{0pt}%
\pgfpathmoveto{\pgfqpoint{1.631046in}{1.870463in}}%
\pgfpathcurveto{\pgfqpoint{1.639283in}{1.870463in}}{\pgfqpoint{1.647183in}{1.873736in}}{\pgfqpoint{1.653007in}{1.879560in}}%
\pgfpathcurveto{\pgfqpoint{1.658830in}{1.885384in}}{\pgfqpoint{1.662103in}{1.893284in}}{\pgfqpoint{1.662103in}{1.901520in}}%
\pgfpathcurveto{\pgfqpoint{1.662103in}{1.909756in}}{\pgfqpoint{1.658830in}{1.917656in}}{\pgfqpoint{1.653007in}{1.923480in}}%
\pgfpathcurveto{\pgfqpoint{1.647183in}{1.929304in}}{\pgfqpoint{1.639283in}{1.932576in}}{\pgfqpoint{1.631046in}{1.932576in}}%
\pgfpathcurveto{\pgfqpoint{1.622810in}{1.932576in}}{\pgfqpoint{1.614910in}{1.929304in}}{\pgfqpoint{1.609086in}{1.923480in}}%
\pgfpathcurveto{\pgfqpoint{1.603262in}{1.917656in}}{\pgfqpoint{1.599990in}{1.909756in}}{\pgfqpoint{1.599990in}{1.901520in}}%
\pgfpathcurveto{\pgfqpoint{1.599990in}{1.893284in}}{\pgfqpoint{1.603262in}{1.885384in}}{\pgfqpoint{1.609086in}{1.879560in}}%
\pgfpathcurveto{\pgfqpoint{1.614910in}{1.873736in}}{\pgfqpoint{1.622810in}{1.870463in}}{\pgfqpoint{1.631046in}{1.870463in}}%
\pgfpathclose%
\pgfusepath{stroke,fill}%
\end{pgfscope}%
\begin{pgfscope}%
\pgfpathrectangle{\pgfqpoint{0.100000in}{0.212622in}}{\pgfqpoint{3.696000in}{3.696000in}}%
\pgfusepath{clip}%
\pgfsetbuttcap%
\pgfsetroundjoin%
\definecolor{currentfill}{rgb}{0.121569,0.466667,0.705882}%
\pgfsetfillcolor{currentfill}%
\pgfsetfillopacity{0.396042}%
\pgfsetlinewidth{1.003750pt}%
\definecolor{currentstroke}{rgb}{0.121569,0.466667,0.705882}%
\pgfsetstrokecolor{currentstroke}%
\pgfsetstrokeopacity{0.396042}%
\pgfsetdash{}{0pt}%
\pgfpathmoveto{\pgfqpoint{1.624244in}{1.869177in}}%
\pgfpathcurveto{\pgfqpoint{1.632481in}{1.869177in}}{\pgfqpoint{1.640381in}{1.872449in}}{\pgfqpoint{1.646205in}{1.878273in}}%
\pgfpathcurveto{\pgfqpoint{1.652029in}{1.884097in}}{\pgfqpoint{1.655301in}{1.891997in}}{\pgfqpoint{1.655301in}{1.900233in}}%
\pgfpathcurveto{\pgfqpoint{1.655301in}{1.908469in}}{\pgfqpoint{1.652029in}{1.916369in}}{\pgfqpoint{1.646205in}{1.922193in}}%
\pgfpathcurveto{\pgfqpoint{1.640381in}{1.928017in}}{\pgfqpoint{1.632481in}{1.931290in}}{\pgfqpoint{1.624244in}{1.931290in}}%
\pgfpathcurveto{\pgfqpoint{1.616008in}{1.931290in}}{\pgfqpoint{1.608108in}{1.928017in}}{\pgfqpoint{1.602284in}{1.922193in}}%
\pgfpathcurveto{\pgfqpoint{1.596460in}{1.916369in}}{\pgfqpoint{1.593188in}{1.908469in}}{\pgfqpoint{1.593188in}{1.900233in}}%
\pgfpathcurveto{\pgfqpoint{1.593188in}{1.891997in}}{\pgfqpoint{1.596460in}{1.884097in}}{\pgfqpoint{1.602284in}{1.878273in}}%
\pgfpathcurveto{\pgfqpoint{1.608108in}{1.872449in}}{\pgfqpoint{1.616008in}{1.869177in}}{\pgfqpoint{1.624244in}{1.869177in}}%
\pgfpathclose%
\pgfusepath{stroke,fill}%
\end{pgfscope}%
\begin{pgfscope}%
\pgfpathrectangle{\pgfqpoint{0.100000in}{0.212622in}}{\pgfqpoint{3.696000in}{3.696000in}}%
\pgfusepath{clip}%
\pgfsetbuttcap%
\pgfsetroundjoin%
\definecolor{currentfill}{rgb}{0.121569,0.466667,0.705882}%
\pgfsetfillcolor{currentfill}%
\pgfsetfillopacity{0.397892}%
\pgfsetlinewidth{1.003750pt}%
\definecolor{currentstroke}{rgb}{0.121569,0.466667,0.705882}%
\pgfsetstrokecolor{currentstroke}%
\pgfsetstrokeopacity{0.397892}%
\pgfsetdash{}{0pt}%
\pgfpathmoveto{\pgfqpoint{1.999269in}{1.995350in}}%
\pgfpathcurveto{\pgfqpoint{2.007506in}{1.995350in}}{\pgfqpoint{2.015406in}{1.998622in}}{\pgfqpoint{2.021230in}{2.004446in}}%
\pgfpathcurveto{\pgfqpoint{2.027054in}{2.010270in}}{\pgfqpoint{2.030326in}{2.018170in}}{\pgfqpoint{2.030326in}{2.026407in}}%
\pgfpathcurveto{\pgfqpoint{2.030326in}{2.034643in}}{\pgfqpoint{2.027054in}{2.042543in}}{\pgfqpoint{2.021230in}{2.048367in}}%
\pgfpathcurveto{\pgfqpoint{2.015406in}{2.054191in}}{\pgfqpoint{2.007506in}{2.057463in}}{\pgfqpoint{1.999269in}{2.057463in}}%
\pgfpathcurveto{\pgfqpoint{1.991033in}{2.057463in}}{\pgfqpoint{1.983133in}{2.054191in}}{\pgfqpoint{1.977309in}{2.048367in}}%
\pgfpathcurveto{\pgfqpoint{1.971485in}{2.042543in}}{\pgfqpoint{1.968213in}{2.034643in}}{\pgfqpoint{1.968213in}{2.026407in}}%
\pgfpathcurveto{\pgfqpoint{1.968213in}{2.018170in}}{\pgfqpoint{1.971485in}{2.010270in}}{\pgfqpoint{1.977309in}{2.004446in}}%
\pgfpathcurveto{\pgfqpoint{1.983133in}{1.998622in}}{\pgfqpoint{1.991033in}{1.995350in}}{\pgfqpoint{1.999269in}{1.995350in}}%
\pgfpathclose%
\pgfusepath{stroke,fill}%
\end{pgfscope}%
\begin{pgfscope}%
\pgfpathrectangle{\pgfqpoint{0.100000in}{0.212622in}}{\pgfqpoint{3.696000in}{3.696000in}}%
\pgfusepath{clip}%
\pgfsetbuttcap%
\pgfsetroundjoin%
\definecolor{currentfill}{rgb}{0.121569,0.466667,0.705882}%
\pgfsetfillcolor{currentfill}%
\pgfsetfillopacity{0.399467}%
\pgfsetlinewidth{1.003750pt}%
\definecolor{currentstroke}{rgb}{0.121569,0.466667,0.705882}%
\pgfsetstrokecolor{currentstroke}%
\pgfsetstrokeopacity{0.399467}%
\pgfsetdash{}{0pt}%
\pgfpathmoveto{\pgfqpoint{1.613500in}{1.860457in}}%
\pgfpathcurveto{\pgfqpoint{1.621736in}{1.860457in}}{\pgfqpoint{1.629636in}{1.863729in}}{\pgfqpoint{1.635460in}{1.869553in}}%
\pgfpathcurveto{\pgfqpoint{1.641284in}{1.875377in}}{\pgfqpoint{1.644556in}{1.883277in}}{\pgfqpoint{1.644556in}{1.891513in}}%
\pgfpathcurveto{\pgfqpoint{1.644556in}{1.899749in}}{\pgfqpoint{1.641284in}{1.907649in}}{\pgfqpoint{1.635460in}{1.913473in}}%
\pgfpathcurveto{\pgfqpoint{1.629636in}{1.919297in}}{\pgfqpoint{1.621736in}{1.922570in}}{\pgfqpoint{1.613500in}{1.922570in}}%
\pgfpathcurveto{\pgfqpoint{1.605263in}{1.922570in}}{\pgfqpoint{1.597363in}{1.919297in}}{\pgfqpoint{1.591539in}{1.913473in}}%
\pgfpathcurveto{\pgfqpoint{1.585715in}{1.907649in}}{\pgfqpoint{1.582443in}{1.899749in}}{\pgfqpoint{1.582443in}{1.891513in}}%
\pgfpathcurveto{\pgfqpoint{1.582443in}{1.883277in}}{\pgfqpoint{1.585715in}{1.875377in}}{\pgfqpoint{1.591539in}{1.869553in}}%
\pgfpathcurveto{\pgfqpoint{1.597363in}{1.863729in}}{\pgfqpoint{1.605263in}{1.860457in}}{\pgfqpoint{1.613500in}{1.860457in}}%
\pgfpathclose%
\pgfusepath{stroke,fill}%
\end{pgfscope}%
\begin{pgfscope}%
\pgfpathrectangle{\pgfqpoint{0.100000in}{0.212622in}}{\pgfqpoint{3.696000in}{3.696000in}}%
\pgfusepath{clip}%
\pgfsetbuttcap%
\pgfsetroundjoin%
\definecolor{currentfill}{rgb}{0.121569,0.466667,0.705882}%
\pgfsetfillcolor{currentfill}%
\pgfsetfillopacity{0.402140}%
\pgfsetlinewidth{1.003750pt}%
\definecolor{currentstroke}{rgb}{0.121569,0.466667,0.705882}%
\pgfsetstrokecolor{currentstroke}%
\pgfsetstrokeopacity{0.402140}%
\pgfsetdash{}{0pt}%
\pgfpathmoveto{\pgfqpoint{1.603648in}{1.856326in}}%
\pgfpathcurveto{\pgfqpoint{1.611884in}{1.856326in}}{\pgfqpoint{1.619784in}{1.859599in}}{\pgfqpoint{1.625608in}{1.865422in}}%
\pgfpathcurveto{\pgfqpoint{1.631432in}{1.871246in}}{\pgfqpoint{1.634705in}{1.879146in}}{\pgfqpoint{1.634705in}{1.887383in}}%
\pgfpathcurveto{\pgfqpoint{1.634705in}{1.895619in}}{\pgfqpoint{1.631432in}{1.903519in}}{\pgfqpoint{1.625608in}{1.909343in}}%
\pgfpathcurveto{\pgfqpoint{1.619784in}{1.915167in}}{\pgfqpoint{1.611884in}{1.918439in}}{\pgfqpoint{1.603648in}{1.918439in}}%
\pgfpathcurveto{\pgfqpoint{1.595412in}{1.918439in}}{\pgfqpoint{1.587512in}{1.915167in}}{\pgfqpoint{1.581688in}{1.909343in}}%
\pgfpathcurveto{\pgfqpoint{1.575864in}{1.903519in}}{\pgfqpoint{1.572592in}{1.895619in}}{\pgfqpoint{1.572592in}{1.887383in}}%
\pgfpathcurveto{\pgfqpoint{1.572592in}{1.879146in}}{\pgfqpoint{1.575864in}{1.871246in}}{\pgfqpoint{1.581688in}{1.865422in}}%
\pgfpathcurveto{\pgfqpoint{1.587512in}{1.859599in}}{\pgfqpoint{1.595412in}{1.856326in}}{\pgfqpoint{1.603648in}{1.856326in}}%
\pgfpathclose%
\pgfusepath{stroke,fill}%
\end{pgfscope}%
\begin{pgfscope}%
\pgfpathrectangle{\pgfqpoint{0.100000in}{0.212622in}}{\pgfqpoint{3.696000in}{3.696000in}}%
\pgfusepath{clip}%
\pgfsetbuttcap%
\pgfsetroundjoin%
\definecolor{currentfill}{rgb}{0.121569,0.466667,0.705882}%
\pgfsetfillcolor{currentfill}%
\pgfsetfillopacity{0.402312}%
\pgfsetlinewidth{1.003750pt}%
\definecolor{currentstroke}{rgb}{0.121569,0.466667,0.705882}%
\pgfsetstrokecolor{currentstroke}%
\pgfsetstrokeopacity{0.402312}%
\pgfsetdash{}{0pt}%
\pgfpathmoveto{\pgfqpoint{2.001390in}{1.992012in}}%
\pgfpathcurveto{\pgfqpoint{2.009626in}{1.992012in}}{\pgfqpoint{2.017526in}{1.995284in}}{\pgfqpoint{2.023350in}{2.001108in}}%
\pgfpathcurveto{\pgfqpoint{2.029174in}{2.006932in}}{\pgfqpoint{2.032446in}{2.014832in}}{\pgfqpoint{2.032446in}{2.023068in}}%
\pgfpathcurveto{\pgfqpoint{2.032446in}{2.031305in}}{\pgfqpoint{2.029174in}{2.039205in}}{\pgfqpoint{2.023350in}{2.045029in}}%
\pgfpathcurveto{\pgfqpoint{2.017526in}{2.050853in}}{\pgfqpoint{2.009626in}{2.054125in}}{\pgfqpoint{2.001390in}{2.054125in}}%
\pgfpathcurveto{\pgfqpoint{1.993154in}{2.054125in}}{\pgfqpoint{1.985254in}{2.050853in}}{\pgfqpoint{1.979430in}{2.045029in}}%
\pgfpathcurveto{\pgfqpoint{1.973606in}{2.039205in}}{\pgfqpoint{1.970333in}{2.031305in}}{\pgfqpoint{1.970333in}{2.023068in}}%
\pgfpathcurveto{\pgfqpoint{1.970333in}{2.014832in}}{\pgfqpoint{1.973606in}{2.006932in}}{\pgfqpoint{1.979430in}{2.001108in}}%
\pgfpathcurveto{\pgfqpoint{1.985254in}{1.995284in}}{\pgfqpoint{1.993154in}{1.992012in}}{\pgfqpoint{2.001390in}{1.992012in}}%
\pgfpathclose%
\pgfusepath{stroke,fill}%
\end{pgfscope}%
\begin{pgfscope}%
\pgfpathrectangle{\pgfqpoint{0.100000in}{0.212622in}}{\pgfqpoint{3.696000in}{3.696000in}}%
\pgfusepath{clip}%
\pgfsetbuttcap%
\pgfsetroundjoin%
\definecolor{currentfill}{rgb}{0.121569,0.466667,0.705882}%
\pgfsetfillcolor{currentfill}%
\pgfsetfillopacity{0.404126}%
\pgfsetlinewidth{1.003750pt}%
\definecolor{currentstroke}{rgb}{0.121569,0.466667,0.705882}%
\pgfsetstrokecolor{currentstroke}%
\pgfsetstrokeopacity{0.404126}%
\pgfsetdash{}{0pt}%
\pgfpathmoveto{\pgfqpoint{1.597756in}{1.854687in}}%
\pgfpathcurveto{\pgfqpoint{1.605992in}{1.854687in}}{\pgfqpoint{1.613892in}{1.857959in}}{\pgfqpoint{1.619716in}{1.863783in}}%
\pgfpathcurveto{\pgfqpoint{1.625540in}{1.869607in}}{\pgfqpoint{1.628812in}{1.877507in}}{\pgfqpoint{1.628812in}{1.885743in}}%
\pgfpathcurveto{\pgfqpoint{1.628812in}{1.893979in}}{\pgfqpoint{1.625540in}{1.901879in}}{\pgfqpoint{1.619716in}{1.907703in}}%
\pgfpathcurveto{\pgfqpoint{1.613892in}{1.913527in}}{\pgfqpoint{1.605992in}{1.916800in}}{\pgfqpoint{1.597756in}{1.916800in}}%
\pgfpathcurveto{\pgfqpoint{1.589519in}{1.916800in}}{\pgfqpoint{1.581619in}{1.913527in}}{\pgfqpoint{1.575795in}{1.907703in}}%
\pgfpathcurveto{\pgfqpoint{1.569971in}{1.901879in}}{\pgfqpoint{1.566699in}{1.893979in}}{\pgfqpoint{1.566699in}{1.885743in}}%
\pgfpathcurveto{\pgfqpoint{1.566699in}{1.877507in}}{\pgfqpoint{1.569971in}{1.869607in}}{\pgfqpoint{1.575795in}{1.863783in}}%
\pgfpathcurveto{\pgfqpoint{1.581619in}{1.857959in}}{\pgfqpoint{1.589519in}{1.854687in}}{\pgfqpoint{1.597756in}{1.854687in}}%
\pgfpathclose%
\pgfusepath{stroke,fill}%
\end{pgfscope}%
\begin{pgfscope}%
\pgfpathrectangle{\pgfqpoint{0.100000in}{0.212622in}}{\pgfqpoint{3.696000in}{3.696000in}}%
\pgfusepath{clip}%
\pgfsetbuttcap%
\pgfsetroundjoin%
\definecolor{currentfill}{rgb}{0.121569,0.466667,0.705882}%
\pgfsetfillcolor{currentfill}%
\pgfsetfillopacity{0.407200}%
\pgfsetlinewidth{1.003750pt}%
\definecolor{currentstroke}{rgb}{0.121569,0.466667,0.705882}%
\pgfsetstrokecolor{currentstroke}%
\pgfsetstrokeopacity{0.407200}%
\pgfsetdash{}{0pt}%
\pgfpathmoveto{\pgfqpoint{1.587941in}{1.846749in}}%
\pgfpathcurveto{\pgfqpoint{1.596178in}{1.846749in}}{\pgfqpoint{1.604078in}{1.850021in}}{\pgfqpoint{1.609902in}{1.855845in}}%
\pgfpathcurveto{\pgfqpoint{1.615726in}{1.861669in}}{\pgfqpoint{1.618998in}{1.869569in}}{\pgfqpoint{1.618998in}{1.877805in}}%
\pgfpathcurveto{\pgfqpoint{1.618998in}{1.886041in}}{\pgfqpoint{1.615726in}{1.893942in}}{\pgfqpoint{1.609902in}{1.899765in}}%
\pgfpathcurveto{\pgfqpoint{1.604078in}{1.905589in}}{\pgfqpoint{1.596178in}{1.908862in}}{\pgfqpoint{1.587941in}{1.908862in}}%
\pgfpathcurveto{\pgfqpoint{1.579705in}{1.908862in}}{\pgfqpoint{1.571805in}{1.905589in}}{\pgfqpoint{1.565981in}{1.899765in}}%
\pgfpathcurveto{\pgfqpoint{1.560157in}{1.893942in}}{\pgfqpoint{1.556885in}{1.886041in}}{\pgfqpoint{1.556885in}{1.877805in}}%
\pgfpathcurveto{\pgfqpoint{1.556885in}{1.869569in}}{\pgfqpoint{1.560157in}{1.861669in}}{\pgfqpoint{1.565981in}{1.855845in}}%
\pgfpathcurveto{\pgfqpoint{1.571805in}{1.850021in}}{\pgfqpoint{1.579705in}{1.846749in}}{\pgfqpoint{1.587941in}{1.846749in}}%
\pgfpathclose%
\pgfusepath{stroke,fill}%
\end{pgfscope}%
\begin{pgfscope}%
\pgfpathrectangle{\pgfqpoint{0.100000in}{0.212622in}}{\pgfqpoint{3.696000in}{3.696000in}}%
\pgfusepath{clip}%
\pgfsetbuttcap%
\pgfsetroundjoin%
\definecolor{currentfill}{rgb}{0.121569,0.466667,0.705882}%
\pgfsetfillcolor{currentfill}%
\pgfsetfillopacity{0.407250}%
\pgfsetlinewidth{1.003750pt}%
\definecolor{currentstroke}{rgb}{0.121569,0.466667,0.705882}%
\pgfsetstrokecolor{currentstroke}%
\pgfsetstrokeopacity{0.407250}%
\pgfsetdash{}{0pt}%
\pgfpathmoveto{\pgfqpoint{2.002784in}{1.988953in}}%
\pgfpathcurveto{\pgfqpoint{2.011020in}{1.988953in}}{\pgfqpoint{2.018920in}{1.992225in}}{\pgfqpoint{2.024744in}{1.998049in}}%
\pgfpathcurveto{\pgfqpoint{2.030568in}{2.003873in}}{\pgfqpoint{2.033841in}{2.011773in}}{\pgfqpoint{2.033841in}{2.020009in}}%
\pgfpathcurveto{\pgfqpoint{2.033841in}{2.028246in}}{\pgfqpoint{2.030568in}{2.036146in}}{\pgfqpoint{2.024744in}{2.041970in}}%
\pgfpathcurveto{\pgfqpoint{2.018920in}{2.047793in}}{\pgfqpoint{2.011020in}{2.051066in}}{\pgfqpoint{2.002784in}{2.051066in}}%
\pgfpathcurveto{\pgfqpoint{1.994548in}{2.051066in}}{\pgfqpoint{1.986648in}{2.047793in}}{\pgfqpoint{1.980824in}{2.041970in}}%
\pgfpathcurveto{\pgfqpoint{1.975000in}{2.036146in}}{\pgfqpoint{1.971728in}{2.028246in}}{\pgfqpoint{1.971728in}{2.020009in}}%
\pgfpathcurveto{\pgfqpoint{1.971728in}{2.011773in}}{\pgfqpoint{1.975000in}{2.003873in}}{\pgfqpoint{1.980824in}{1.998049in}}%
\pgfpathcurveto{\pgfqpoint{1.986648in}{1.992225in}}{\pgfqpoint{1.994548in}{1.988953in}}{\pgfqpoint{2.002784in}{1.988953in}}%
\pgfpathclose%
\pgfusepath{stroke,fill}%
\end{pgfscope}%
\begin{pgfscope}%
\pgfpathrectangle{\pgfqpoint{0.100000in}{0.212622in}}{\pgfqpoint{3.696000in}{3.696000in}}%
\pgfusepath{clip}%
\pgfsetbuttcap%
\pgfsetroundjoin%
\definecolor{currentfill}{rgb}{0.121569,0.466667,0.705882}%
\pgfsetfillcolor{currentfill}%
\pgfsetfillopacity{0.409106}%
\pgfsetlinewidth{1.003750pt}%
\definecolor{currentstroke}{rgb}{0.121569,0.466667,0.705882}%
\pgfsetstrokecolor{currentstroke}%
\pgfsetstrokeopacity{0.409106}%
\pgfsetdash{}{0pt}%
\pgfpathmoveto{\pgfqpoint{1.580092in}{1.840426in}}%
\pgfpathcurveto{\pgfqpoint{1.588329in}{1.840426in}}{\pgfqpoint{1.596229in}{1.843699in}}{\pgfqpoint{1.602053in}{1.849523in}}%
\pgfpathcurveto{\pgfqpoint{1.607877in}{1.855346in}}{\pgfqpoint{1.611149in}{1.863247in}}{\pgfqpoint{1.611149in}{1.871483in}}%
\pgfpathcurveto{\pgfqpoint{1.611149in}{1.879719in}}{\pgfqpoint{1.607877in}{1.887619in}}{\pgfqpoint{1.602053in}{1.893443in}}%
\pgfpathcurveto{\pgfqpoint{1.596229in}{1.899267in}}{\pgfqpoint{1.588329in}{1.902539in}}{\pgfqpoint{1.580092in}{1.902539in}}%
\pgfpathcurveto{\pgfqpoint{1.571856in}{1.902539in}}{\pgfqpoint{1.563956in}{1.899267in}}{\pgfqpoint{1.558132in}{1.893443in}}%
\pgfpathcurveto{\pgfqpoint{1.552308in}{1.887619in}}{\pgfqpoint{1.549036in}{1.879719in}}{\pgfqpoint{1.549036in}{1.871483in}}%
\pgfpathcurveto{\pgfqpoint{1.549036in}{1.863247in}}{\pgfqpoint{1.552308in}{1.855346in}}{\pgfqpoint{1.558132in}{1.849523in}}%
\pgfpathcurveto{\pgfqpoint{1.563956in}{1.843699in}}{\pgfqpoint{1.571856in}{1.840426in}}{\pgfqpoint{1.580092in}{1.840426in}}%
\pgfpathclose%
\pgfusepath{stroke,fill}%
\end{pgfscope}%
\begin{pgfscope}%
\pgfpathrectangle{\pgfqpoint{0.100000in}{0.212622in}}{\pgfqpoint{3.696000in}{3.696000in}}%
\pgfusepath{clip}%
\pgfsetbuttcap%
\pgfsetroundjoin%
\definecolor{currentfill}{rgb}{0.121569,0.466667,0.705882}%
\pgfsetfillcolor{currentfill}%
\pgfsetfillopacity{0.410721}%
\pgfsetlinewidth{1.003750pt}%
\definecolor{currentstroke}{rgb}{0.121569,0.466667,0.705882}%
\pgfsetstrokecolor{currentstroke}%
\pgfsetstrokeopacity{0.410721}%
\pgfsetdash{}{0pt}%
\pgfpathmoveto{\pgfqpoint{1.575002in}{1.839343in}}%
\pgfpathcurveto{\pgfqpoint{1.583239in}{1.839343in}}{\pgfqpoint{1.591139in}{1.842616in}}{\pgfqpoint{1.596963in}{1.848440in}}%
\pgfpathcurveto{\pgfqpoint{1.602787in}{1.854264in}}{\pgfqpoint{1.606059in}{1.862164in}}{\pgfqpoint{1.606059in}{1.870400in}}%
\pgfpathcurveto{\pgfqpoint{1.606059in}{1.878636in}}{\pgfqpoint{1.602787in}{1.886536in}}{\pgfqpoint{1.596963in}{1.892360in}}%
\pgfpathcurveto{\pgfqpoint{1.591139in}{1.898184in}}{\pgfqpoint{1.583239in}{1.901456in}}{\pgfqpoint{1.575002in}{1.901456in}}%
\pgfpathcurveto{\pgfqpoint{1.566766in}{1.901456in}}{\pgfqpoint{1.558866in}{1.898184in}}{\pgfqpoint{1.553042in}{1.892360in}}%
\pgfpathcurveto{\pgfqpoint{1.547218in}{1.886536in}}{\pgfqpoint{1.543946in}{1.878636in}}{\pgfqpoint{1.543946in}{1.870400in}}%
\pgfpathcurveto{\pgfqpoint{1.543946in}{1.862164in}}{\pgfqpoint{1.547218in}{1.854264in}}{\pgfqpoint{1.553042in}{1.848440in}}%
\pgfpathcurveto{\pgfqpoint{1.558866in}{1.842616in}}{\pgfqpoint{1.566766in}{1.839343in}}{\pgfqpoint{1.575002in}{1.839343in}}%
\pgfpathclose%
\pgfusepath{stroke,fill}%
\end{pgfscope}%
\begin{pgfscope}%
\pgfpathrectangle{\pgfqpoint{0.100000in}{0.212622in}}{\pgfqpoint{3.696000in}{3.696000in}}%
\pgfusepath{clip}%
\pgfsetbuttcap%
\pgfsetroundjoin%
\definecolor{currentfill}{rgb}{0.121569,0.466667,0.705882}%
\pgfsetfillcolor{currentfill}%
\pgfsetfillopacity{0.412470}%
\pgfsetlinewidth{1.003750pt}%
\definecolor{currentstroke}{rgb}{0.121569,0.466667,0.705882}%
\pgfsetstrokecolor{currentstroke}%
\pgfsetstrokeopacity{0.412470}%
\pgfsetdash{}{0pt}%
\pgfpathmoveto{\pgfqpoint{2.004685in}{1.982974in}}%
\pgfpathcurveto{\pgfqpoint{2.012922in}{1.982974in}}{\pgfqpoint{2.020822in}{1.986247in}}{\pgfqpoint{2.026646in}{1.992071in}}%
\pgfpathcurveto{\pgfqpoint{2.032470in}{1.997894in}}{\pgfqpoint{2.035742in}{2.005795in}}{\pgfqpoint{2.035742in}{2.014031in}}%
\pgfpathcurveto{\pgfqpoint{2.035742in}{2.022267in}}{\pgfqpoint{2.032470in}{2.030167in}}{\pgfqpoint{2.026646in}{2.035991in}}%
\pgfpathcurveto{\pgfqpoint{2.020822in}{2.041815in}}{\pgfqpoint{2.012922in}{2.045087in}}{\pgfqpoint{2.004685in}{2.045087in}}%
\pgfpathcurveto{\pgfqpoint{1.996449in}{2.045087in}}{\pgfqpoint{1.988549in}{2.041815in}}{\pgfqpoint{1.982725in}{2.035991in}}%
\pgfpathcurveto{\pgfqpoint{1.976901in}{2.030167in}}{\pgfqpoint{1.973629in}{2.022267in}}{\pgfqpoint{1.973629in}{2.014031in}}%
\pgfpathcurveto{\pgfqpoint{1.973629in}{2.005795in}}{\pgfqpoint{1.976901in}{1.997894in}}{\pgfqpoint{1.982725in}{1.992071in}}%
\pgfpathcurveto{\pgfqpoint{1.988549in}{1.986247in}}{\pgfqpoint{1.996449in}{1.982974in}}{\pgfqpoint{2.004685in}{1.982974in}}%
\pgfpathclose%
\pgfusepath{stroke,fill}%
\end{pgfscope}%
\begin{pgfscope}%
\pgfpathrectangle{\pgfqpoint{0.100000in}{0.212622in}}{\pgfqpoint{3.696000in}{3.696000in}}%
\pgfusepath{clip}%
\pgfsetbuttcap%
\pgfsetroundjoin%
\definecolor{currentfill}{rgb}{0.121569,0.466667,0.705882}%
\pgfsetfillcolor{currentfill}%
\pgfsetfillopacity{0.413010}%
\pgfsetlinewidth{1.003750pt}%
\definecolor{currentstroke}{rgb}{0.121569,0.466667,0.705882}%
\pgfsetstrokecolor{currentstroke}%
\pgfsetstrokeopacity{0.413010}%
\pgfsetdash{}{0pt}%
\pgfpathmoveto{\pgfqpoint{1.566692in}{1.831550in}}%
\pgfpathcurveto{\pgfqpoint{1.574928in}{1.831550in}}{\pgfqpoint{1.582828in}{1.834822in}}{\pgfqpoint{1.588652in}{1.840646in}}%
\pgfpathcurveto{\pgfqpoint{1.594476in}{1.846470in}}{\pgfqpoint{1.597748in}{1.854370in}}{\pgfqpoint{1.597748in}{1.862606in}}%
\pgfpathcurveto{\pgfqpoint{1.597748in}{1.870842in}}{\pgfqpoint{1.594476in}{1.878742in}}{\pgfqpoint{1.588652in}{1.884566in}}%
\pgfpathcurveto{\pgfqpoint{1.582828in}{1.890390in}}{\pgfqpoint{1.574928in}{1.893663in}}{\pgfqpoint{1.566692in}{1.893663in}}%
\pgfpathcurveto{\pgfqpoint{1.558456in}{1.893663in}}{\pgfqpoint{1.550556in}{1.890390in}}{\pgfqpoint{1.544732in}{1.884566in}}%
\pgfpathcurveto{\pgfqpoint{1.538908in}{1.878742in}}{\pgfqpoint{1.535635in}{1.870842in}}{\pgfqpoint{1.535635in}{1.862606in}}%
\pgfpathcurveto{\pgfqpoint{1.535635in}{1.854370in}}{\pgfqpoint{1.538908in}{1.846470in}}{\pgfqpoint{1.544732in}{1.840646in}}%
\pgfpathcurveto{\pgfqpoint{1.550556in}{1.834822in}}{\pgfqpoint{1.558456in}{1.831550in}}{\pgfqpoint{1.566692in}{1.831550in}}%
\pgfpathclose%
\pgfusepath{stroke,fill}%
\end{pgfscope}%
\begin{pgfscope}%
\pgfpathrectangle{\pgfqpoint{0.100000in}{0.212622in}}{\pgfqpoint{3.696000in}{3.696000in}}%
\pgfusepath{clip}%
\pgfsetbuttcap%
\pgfsetroundjoin%
\definecolor{currentfill}{rgb}{0.121569,0.466667,0.705882}%
\pgfsetfillcolor{currentfill}%
\pgfsetfillopacity{0.414363}%
\pgfsetlinewidth{1.003750pt}%
\definecolor{currentstroke}{rgb}{0.121569,0.466667,0.705882}%
\pgfsetstrokecolor{currentstroke}%
\pgfsetstrokeopacity{0.414363}%
\pgfsetdash{}{0pt}%
\pgfpathmoveto{\pgfqpoint{1.560410in}{1.826632in}}%
\pgfpathcurveto{\pgfqpoint{1.568646in}{1.826632in}}{\pgfqpoint{1.576546in}{1.829904in}}{\pgfqpoint{1.582370in}{1.835728in}}%
\pgfpathcurveto{\pgfqpoint{1.588194in}{1.841552in}}{\pgfqpoint{1.591466in}{1.849452in}}{\pgfqpoint{1.591466in}{1.857688in}}%
\pgfpathcurveto{\pgfqpoint{1.591466in}{1.865925in}}{\pgfqpoint{1.588194in}{1.873825in}}{\pgfqpoint{1.582370in}{1.879648in}}%
\pgfpathcurveto{\pgfqpoint{1.576546in}{1.885472in}}{\pgfqpoint{1.568646in}{1.888745in}}{\pgfqpoint{1.560410in}{1.888745in}}%
\pgfpathcurveto{\pgfqpoint{1.552174in}{1.888745in}}{\pgfqpoint{1.544273in}{1.885472in}}{\pgfqpoint{1.538450in}{1.879648in}}%
\pgfpathcurveto{\pgfqpoint{1.532626in}{1.873825in}}{\pgfqpoint{1.529353in}{1.865925in}}{\pgfqpoint{1.529353in}{1.857688in}}%
\pgfpathcurveto{\pgfqpoint{1.529353in}{1.849452in}}{\pgfqpoint{1.532626in}{1.841552in}}{\pgfqpoint{1.538450in}{1.835728in}}%
\pgfpathcurveto{\pgfqpoint{1.544273in}{1.829904in}}{\pgfqpoint{1.552174in}{1.826632in}}{\pgfqpoint{1.560410in}{1.826632in}}%
\pgfpathclose%
\pgfusepath{stroke,fill}%
\end{pgfscope}%
\begin{pgfscope}%
\pgfpathrectangle{\pgfqpoint{0.100000in}{0.212622in}}{\pgfqpoint{3.696000in}{3.696000in}}%
\pgfusepath{clip}%
\pgfsetbuttcap%
\pgfsetroundjoin%
\definecolor{currentfill}{rgb}{0.121569,0.466667,0.705882}%
\pgfsetfillcolor{currentfill}%
\pgfsetfillopacity{0.415368}%
\pgfsetlinewidth{1.003750pt}%
\definecolor{currentstroke}{rgb}{0.121569,0.466667,0.705882}%
\pgfsetstrokecolor{currentstroke}%
\pgfsetstrokeopacity{0.415368}%
\pgfsetdash{}{0pt}%
\pgfpathmoveto{\pgfqpoint{1.557484in}{1.825463in}}%
\pgfpathcurveto{\pgfqpoint{1.565720in}{1.825463in}}{\pgfqpoint{1.573620in}{1.828735in}}{\pgfqpoint{1.579444in}{1.834559in}}%
\pgfpathcurveto{\pgfqpoint{1.585268in}{1.840383in}}{\pgfqpoint{1.588540in}{1.848283in}}{\pgfqpoint{1.588540in}{1.856520in}}%
\pgfpathcurveto{\pgfqpoint{1.588540in}{1.864756in}}{\pgfqpoint{1.585268in}{1.872656in}}{\pgfqpoint{1.579444in}{1.878480in}}%
\pgfpathcurveto{\pgfqpoint{1.573620in}{1.884304in}}{\pgfqpoint{1.565720in}{1.887576in}}{\pgfqpoint{1.557484in}{1.887576in}}%
\pgfpathcurveto{\pgfqpoint{1.549248in}{1.887576in}}{\pgfqpoint{1.541348in}{1.884304in}}{\pgfqpoint{1.535524in}{1.878480in}}%
\pgfpathcurveto{\pgfqpoint{1.529700in}{1.872656in}}{\pgfqpoint{1.526427in}{1.864756in}}{\pgfqpoint{1.526427in}{1.856520in}}%
\pgfpathcurveto{\pgfqpoint{1.526427in}{1.848283in}}{\pgfqpoint{1.529700in}{1.840383in}}{\pgfqpoint{1.535524in}{1.834559in}}%
\pgfpathcurveto{\pgfqpoint{1.541348in}{1.828735in}}{\pgfqpoint{1.549248in}{1.825463in}}{\pgfqpoint{1.557484in}{1.825463in}}%
\pgfpathclose%
\pgfusepath{stroke,fill}%
\end{pgfscope}%
\begin{pgfscope}%
\pgfpathrectangle{\pgfqpoint{0.100000in}{0.212622in}}{\pgfqpoint{3.696000in}{3.696000in}}%
\pgfusepath{clip}%
\pgfsetbuttcap%
\pgfsetroundjoin%
\definecolor{currentfill}{rgb}{0.121569,0.466667,0.705882}%
\pgfsetfillcolor{currentfill}%
\pgfsetfillopacity{0.416953}%
\pgfsetlinewidth{1.003750pt}%
\definecolor{currentstroke}{rgb}{0.121569,0.466667,0.705882}%
\pgfsetstrokecolor{currentstroke}%
\pgfsetstrokeopacity{0.416953}%
\pgfsetdash{}{0pt}%
\pgfpathmoveto{\pgfqpoint{1.552557in}{1.821162in}}%
\pgfpathcurveto{\pgfqpoint{1.560794in}{1.821162in}}{\pgfqpoint{1.568694in}{1.824434in}}{\pgfqpoint{1.574518in}{1.830258in}}%
\pgfpathcurveto{\pgfqpoint{1.580341in}{1.836082in}}{\pgfqpoint{1.583614in}{1.843982in}}{\pgfqpoint{1.583614in}{1.852219in}}%
\pgfpathcurveto{\pgfqpoint{1.583614in}{1.860455in}}{\pgfqpoint{1.580341in}{1.868355in}}{\pgfqpoint{1.574518in}{1.874179in}}%
\pgfpathcurveto{\pgfqpoint{1.568694in}{1.880003in}}{\pgfqpoint{1.560794in}{1.883275in}}{\pgfqpoint{1.552557in}{1.883275in}}%
\pgfpathcurveto{\pgfqpoint{1.544321in}{1.883275in}}{\pgfqpoint{1.536421in}{1.880003in}}{\pgfqpoint{1.530597in}{1.874179in}}%
\pgfpathcurveto{\pgfqpoint{1.524773in}{1.868355in}}{\pgfqpoint{1.521501in}{1.860455in}}{\pgfqpoint{1.521501in}{1.852219in}}%
\pgfpathcurveto{\pgfqpoint{1.521501in}{1.843982in}}{\pgfqpoint{1.524773in}{1.836082in}}{\pgfqpoint{1.530597in}{1.830258in}}%
\pgfpathcurveto{\pgfqpoint{1.536421in}{1.824434in}}{\pgfqpoint{1.544321in}{1.821162in}}{\pgfqpoint{1.552557in}{1.821162in}}%
\pgfpathclose%
\pgfusepath{stroke,fill}%
\end{pgfscope}%
\begin{pgfscope}%
\pgfpathrectangle{\pgfqpoint{0.100000in}{0.212622in}}{\pgfqpoint{3.696000in}{3.696000in}}%
\pgfusepath{clip}%
\pgfsetbuttcap%
\pgfsetroundjoin%
\definecolor{currentfill}{rgb}{0.121569,0.466667,0.705882}%
\pgfsetfillcolor{currentfill}%
\pgfsetfillopacity{0.417901}%
\pgfsetlinewidth{1.003750pt}%
\definecolor{currentstroke}{rgb}{0.121569,0.466667,0.705882}%
\pgfsetstrokecolor{currentstroke}%
\pgfsetstrokeopacity{0.417901}%
\pgfsetdash{}{0pt}%
\pgfpathmoveto{\pgfqpoint{1.549660in}{1.820752in}}%
\pgfpathcurveto{\pgfqpoint{1.557896in}{1.820752in}}{\pgfqpoint{1.565796in}{1.824025in}}{\pgfqpoint{1.571620in}{1.829849in}}%
\pgfpathcurveto{\pgfqpoint{1.577444in}{1.835672in}}{\pgfqpoint{1.580716in}{1.843573in}}{\pgfqpoint{1.580716in}{1.851809in}}%
\pgfpathcurveto{\pgfqpoint{1.580716in}{1.860045in}}{\pgfqpoint{1.577444in}{1.867945in}}{\pgfqpoint{1.571620in}{1.873769in}}%
\pgfpathcurveto{\pgfqpoint{1.565796in}{1.879593in}}{\pgfqpoint{1.557896in}{1.882865in}}{\pgfqpoint{1.549660in}{1.882865in}}%
\pgfpathcurveto{\pgfqpoint{1.541424in}{1.882865in}}{\pgfqpoint{1.533523in}{1.879593in}}{\pgfqpoint{1.527700in}{1.873769in}}%
\pgfpathcurveto{\pgfqpoint{1.521876in}{1.867945in}}{\pgfqpoint{1.518603in}{1.860045in}}{\pgfqpoint{1.518603in}{1.851809in}}%
\pgfpathcurveto{\pgfqpoint{1.518603in}{1.843573in}}{\pgfqpoint{1.521876in}{1.835672in}}{\pgfqpoint{1.527700in}{1.829849in}}%
\pgfpathcurveto{\pgfqpoint{1.533523in}{1.824025in}}{\pgfqpoint{1.541424in}{1.820752in}}{\pgfqpoint{1.549660in}{1.820752in}}%
\pgfpathclose%
\pgfusepath{stroke,fill}%
\end{pgfscope}%
\begin{pgfscope}%
\pgfpathrectangle{\pgfqpoint{0.100000in}{0.212622in}}{\pgfqpoint{3.696000in}{3.696000in}}%
\pgfusepath{clip}%
\pgfsetbuttcap%
\pgfsetroundjoin%
\definecolor{currentfill}{rgb}{0.121569,0.466667,0.705882}%
\pgfsetfillcolor{currentfill}%
\pgfsetfillopacity{0.418199}%
\pgfsetlinewidth{1.003750pt}%
\definecolor{currentstroke}{rgb}{0.121569,0.466667,0.705882}%
\pgfsetstrokecolor{currentstroke}%
\pgfsetstrokeopacity{0.418199}%
\pgfsetdash{}{0pt}%
\pgfpathmoveto{\pgfqpoint{2.007266in}{1.979018in}}%
\pgfpathcurveto{\pgfqpoint{2.015502in}{1.979018in}}{\pgfqpoint{2.023402in}{1.982291in}}{\pgfqpoint{2.029226in}{1.988115in}}%
\pgfpathcurveto{\pgfqpoint{2.035050in}{1.993938in}}{\pgfqpoint{2.038323in}{2.001839in}}{\pgfqpoint{2.038323in}{2.010075in}}%
\pgfpathcurveto{\pgfqpoint{2.038323in}{2.018311in}}{\pgfqpoint{2.035050in}{2.026211in}}{\pgfqpoint{2.029226in}{2.032035in}}%
\pgfpathcurveto{\pgfqpoint{2.023402in}{2.037859in}}{\pgfqpoint{2.015502in}{2.041131in}}{\pgfqpoint{2.007266in}{2.041131in}}%
\pgfpathcurveto{\pgfqpoint{1.999030in}{2.041131in}}{\pgfqpoint{1.991130in}{2.037859in}}{\pgfqpoint{1.985306in}{2.032035in}}%
\pgfpathcurveto{\pgfqpoint{1.979482in}{2.026211in}}{\pgfqpoint{1.976210in}{2.018311in}}{\pgfqpoint{1.976210in}{2.010075in}}%
\pgfpathcurveto{\pgfqpoint{1.976210in}{2.001839in}}{\pgfqpoint{1.979482in}{1.993938in}}{\pgfqpoint{1.985306in}{1.988115in}}%
\pgfpathcurveto{\pgfqpoint{1.991130in}{1.982291in}}{\pgfqpoint{1.999030in}{1.979018in}}{\pgfqpoint{2.007266in}{1.979018in}}%
\pgfpathclose%
\pgfusepath{stroke,fill}%
\end{pgfscope}%
\begin{pgfscope}%
\pgfpathrectangle{\pgfqpoint{0.100000in}{0.212622in}}{\pgfqpoint{3.696000in}{3.696000in}}%
\pgfusepath{clip}%
\pgfsetbuttcap%
\pgfsetroundjoin%
\definecolor{currentfill}{rgb}{0.121569,0.466667,0.705882}%
\pgfsetfillcolor{currentfill}%
\pgfsetfillopacity{0.418680}%
\pgfsetlinewidth{1.003750pt}%
\definecolor{currentstroke}{rgb}{0.121569,0.466667,0.705882}%
\pgfsetstrokecolor{currentstroke}%
\pgfsetstrokeopacity{0.418680}%
\pgfsetdash{}{0pt}%
\pgfpathmoveto{\pgfqpoint{1.547281in}{1.819828in}}%
\pgfpathcurveto{\pgfqpoint{1.555517in}{1.819828in}}{\pgfqpoint{1.563418in}{1.823100in}}{\pgfqpoint{1.569241in}{1.828924in}}%
\pgfpathcurveto{\pgfqpoint{1.575065in}{1.834748in}}{\pgfqpoint{1.578338in}{1.842648in}}{\pgfqpoint{1.578338in}{1.850885in}}%
\pgfpathcurveto{\pgfqpoint{1.578338in}{1.859121in}}{\pgfqpoint{1.575065in}{1.867021in}}{\pgfqpoint{1.569241in}{1.872845in}}%
\pgfpathcurveto{\pgfqpoint{1.563418in}{1.878669in}}{\pgfqpoint{1.555517in}{1.881941in}}{\pgfqpoint{1.547281in}{1.881941in}}%
\pgfpathcurveto{\pgfqpoint{1.539045in}{1.881941in}}{\pgfqpoint{1.531145in}{1.878669in}}{\pgfqpoint{1.525321in}{1.872845in}}%
\pgfpathcurveto{\pgfqpoint{1.519497in}{1.867021in}}{\pgfqpoint{1.516225in}{1.859121in}}{\pgfqpoint{1.516225in}{1.850885in}}%
\pgfpathcurveto{\pgfqpoint{1.516225in}{1.842648in}}{\pgfqpoint{1.519497in}{1.834748in}}{\pgfqpoint{1.525321in}{1.828924in}}%
\pgfpathcurveto{\pgfqpoint{1.531145in}{1.823100in}}{\pgfqpoint{1.539045in}{1.819828in}}{\pgfqpoint{1.547281in}{1.819828in}}%
\pgfpathclose%
\pgfusepath{stroke,fill}%
\end{pgfscope}%
\begin{pgfscope}%
\pgfpathrectangle{\pgfqpoint{0.100000in}{0.212622in}}{\pgfqpoint{3.696000in}{3.696000in}}%
\pgfusepath{clip}%
\pgfsetbuttcap%
\pgfsetroundjoin%
\definecolor{currentfill}{rgb}{0.121569,0.466667,0.705882}%
\pgfsetfillcolor{currentfill}%
\pgfsetfillopacity{0.419924}%
\pgfsetlinewidth{1.003750pt}%
\definecolor{currentstroke}{rgb}{0.121569,0.466667,0.705882}%
\pgfsetstrokecolor{currentstroke}%
\pgfsetstrokeopacity{0.419924}%
\pgfsetdash{}{0pt}%
\pgfpathmoveto{\pgfqpoint{1.543235in}{1.816593in}}%
\pgfpathcurveto{\pgfqpoint{1.551471in}{1.816593in}}{\pgfqpoint{1.559371in}{1.819865in}}{\pgfqpoint{1.565195in}{1.825689in}}%
\pgfpathcurveto{\pgfqpoint{1.571019in}{1.831513in}}{\pgfqpoint{1.574291in}{1.839413in}}{\pgfqpoint{1.574291in}{1.847650in}}%
\pgfpathcurveto{\pgfqpoint{1.574291in}{1.855886in}}{\pgfqpoint{1.571019in}{1.863786in}}{\pgfqpoint{1.565195in}{1.869610in}}%
\pgfpathcurveto{\pgfqpoint{1.559371in}{1.875434in}}{\pgfqpoint{1.551471in}{1.878706in}}{\pgfqpoint{1.543235in}{1.878706in}}%
\pgfpathcurveto{\pgfqpoint{1.534998in}{1.878706in}}{\pgfqpoint{1.527098in}{1.875434in}}{\pgfqpoint{1.521274in}{1.869610in}}%
\pgfpathcurveto{\pgfqpoint{1.515450in}{1.863786in}}{\pgfqpoint{1.512178in}{1.855886in}}{\pgfqpoint{1.512178in}{1.847650in}}%
\pgfpathcurveto{\pgfqpoint{1.512178in}{1.839413in}}{\pgfqpoint{1.515450in}{1.831513in}}{\pgfqpoint{1.521274in}{1.825689in}}%
\pgfpathcurveto{\pgfqpoint{1.527098in}{1.819865in}}{\pgfqpoint{1.534998in}{1.816593in}}{\pgfqpoint{1.543235in}{1.816593in}}%
\pgfpathclose%
\pgfusepath{stroke,fill}%
\end{pgfscope}%
\begin{pgfscope}%
\pgfpathrectangle{\pgfqpoint{0.100000in}{0.212622in}}{\pgfqpoint{3.696000in}{3.696000in}}%
\pgfusepath{clip}%
\pgfsetbuttcap%
\pgfsetroundjoin%
\definecolor{currentfill}{rgb}{0.121569,0.466667,0.705882}%
\pgfsetfillcolor{currentfill}%
\pgfsetfillopacity{0.420975}%
\pgfsetlinewidth{1.003750pt}%
\definecolor{currentstroke}{rgb}{0.121569,0.466667,0.705882}%
\pgfsetstrokecolor{currentstroke}%
\pgfsetstrokeopacity{0.420975}%
\pgfsetdash{}{0pt}%
\pgfpathmoveto{\pgfqpoint{1.540504in}{1.816243in}}%
\pgfpathcurveto{\pgfqpoint{1.548740in}{1.816243in}}{\pgfqpoint{1.556640in}{1.819515in}}{\pgfqpoint{1.562464in}{1.825339in}}%
\pgfpathcurveto{\pgfqpoint{1.568288in}{1.831163in}}{\pgfqpoint{1.571560in}{1.839063in}}{\pgfqpoint{1.571560in}{1.847299in}}%
\pgfpathcurveto{\pgfqpoint{1.571560in}{1.855535in}}{\pgfqpoint{1.568288in}{1.863436in}}{\pgfqpoint{1.562464in}{1.869259in}}%
\pgfpathcurveto{\pgfqpoint{1.556640in}{1.875083in}}{\pgfqpoint{1.548740in}{1.878356in}}{\pgfqpoint{1.540504in}{1.878356in}}%
\pgfpathcurveto{\pgfqpoint{1.532268in}{1.878356in}}{\pgfqpoint{1.524367in}{1.875083in}}{\pgfqpoint{1.518544in}{1.869259in}}%
\pgfpathcurveto{\pgfqpoint{1.512720in}{1.863436in}}{\pgfqpoint{1.509447in}{1.855535in}}{\pgfqpoint{1.509447in}{1.847299in}}%
\pgfpathcurveto{\pgfqpoint{1.509447in}{1.839063in}}{\pgfqpoint{1.512720in}{1.831163in}}{\pgfqpoint{1.518544in}{1.825339in}}%
\pgfpathcurveto{\pgfqpoint{1.524367in}{1.819515in}}{\pgfqpoint{1.532268in}{1.816243in}}{\pgfqpoint{1.540504in}{1.816243in}}%
\pgfpathclose%
\pgfusepath{stroke,fill}%
\end{pgfscope}%
\begin{pgfscope}%
\pgfpathrectangle{\pgfqpoint{0.100000in}{0.212622in}}{\pgfqpoint{3.696000in}{3.696000in}}%
\pgfusepath{clip}%
\pgfsetbuttcap%
\pgfsetroundjoin%
\definecolor{currentfill}{rgb}{0.121569,0.466667,0.705882}%
\pgfsetfillcolor{currentfill}%
\pgfsetfillopacity{0.421352}%
\pgfsetlinewidth{1.003750pt}%
\definecolor{currentstroke}{rgb}{0.121569,0.466667,0.705882}%
\pgfsetstrokecolor{currentstroke}%
\pgfsetstrokeopacity{0.421352}%
\pgfsetdash{}{0pt}%
\pgfpathmoveto{\pgfqpoint{2.009175in}{1.977041in}}%
\pgfpathcurveto{\pgfqpoint{2.017411in}{1.977041in}}{\pgfqpoint{2.025311in}{1.980313in}}{\pgfqpoint{2.031135in}{1.986137in}}%
\pgfpathcurveto{\pgfqpoint{2.036959in}{1.991961in}}{\pgfqpoint{2.040231in}{1.999861in}}{\pgfqpoint{2.040231in}{2.008098in}}%
\pgfpathcurveto{\pgfqpoint{2.040231in}{2.016334in}}{\pgfqpoint{2.036959in}{2.024234in}}{\pgfqpoint{2.031135in}{2.030058in}}%
\pgfpathcurveto{\pgfqpoint{2.025311in}{2.035882in}}{\pgfqpoint{2.017411in}{2.039154in}}{\pgfqpoint{2.009175in}{2.039154in}}%
\pgfpathcurveto{\pgfqpoint{2.000939in}{2.039154in}}{\pgfqpoint{1.993039in}{2.035882in}}{\pgfqpoint{1.987215in}{2.030058in}}%
\pgfpathcurveto{\pgfqpoint{1.981391in}{2.024234in}}{\pgfqpoint{1.978118in}{2.016334in}}{\pgfqpoint{1.978118in}{2.008098in}}%
\pgfpathcurveto{\pgfqpoint{1.978118in}{1.999861in}}{\pgfqpoint{1.981391in}{1.991961in}}{\pgfqpoint{1.987215in}{1.986137in}}%
\pgfpathcurveto{\pgfqpoint{1.993039in}{1.980313in}}{\pgfqpoint{2.000939in}{1.977041in}}{\pgfqpoint{2.009175in}{1.977041in}}%
\pgfpathclose%
\pgfusepath{stroke,fill}%
\end{pgfscope}%
\begin{pgfscope}%
\pgfpathrectangle{\pgfqpoint{0.100000in}{0.212622in}}{\pgfqpoint{3.696000in}{3.696000in}}%
\pgfusepath{clip}%
\pgfsetbuttcap%
\pgfsetroundjoin%
\definecolor{currentfill}{rgb}{0.121569,0.466667,0.705882}%
\pgfsetfillcolor{currentfill}%
\pgfsetfillopacity{0.421649}%
\pgfsetlinewidth{1.003750pt}%
\definecolor{currentstroke}{rgb}{0.121569,0.466667,0.705882}%
\pgfsetstrokecolor{currentstroke}%
\pgfsetstrokeopacity{0.421649}%
\pgfsetdash{}{0pt}%
\pgfpathmoveto{\pgfqpoint{1.538363in}{1.814916in}}%
\pgfpathcurveto{\pgfqpoint{1.546599in}{1.814916in}}{\pgfqpoint{1.554499in}{1.818188in}}{\pgfqpoint{1.560323in}{1.824012in}}%
\pgfpathcurveto{\pgfqpoint{1.566147in}{1.829836in}}{\pgfqpoint{1.569419in}{1.837736in}}{\pgfqpoint{1.569419in}{1.845972in}}%
\pgfpathcurveto{\pgfqpoint{1.569419in}{1.854209in}}{\pgfqpoint{1.566147in}{1.862109in}}{\pgfqpoint{1.560323in}{1.867933in}}%
\pgfpathcurveto{\pgfqpoint{1.554499in}{1.873757in}}{\pgfqpoint{1.546599in}{1.877029in}}{\pgfqpoint{1.538363in}{1.877029in}}%
\pgfpathcurveto{\pgfqpoint{1.530126in}{1.877029in}}{\pgfqpoint{1.522226in}{1.873757in}}{\pgfqpoint{1.516402in}{1.867933in}}%
\pgfpathcurveto{\pgfqpoint{1.510578in}{1.862109in}}{\pgfqpoint{1.507306in}{1.854209in}}{\pgfqpoint{1.507306in}{1.845972in}}%
\pgfpathcurveto{\pgfqpoint{1.507306in}{1.837736in}}{\pgfqpoint{1.510578in}{1.829836in}}{\pgfqpoint{1.516402in}{1.824012in}}%
\pgfpathcurveto{\pgfqpoint{1.522226in}{1.818188in}}{\pgfqpoint{1.530126in}{1.814916in}}{\pgfqpoint{1.538363in}{1.814916in}}%
\pgfpathclose%
\pgfusepath{stroke,fill}%
\end{pgfscope}%
\begin{pgfscope}%
\pgfpathrectangle{\pgfqpoint{0.100000in}{0.212622in}}{\pgfqpoint{3.696000in}{3.696000in}}%
\pgfusepath{clip}%
\pgfsetbuttcap%
\pgfsetroundjoin%
\definecolor{currentfill}{rgb}{0.121569,0.466667,0.705882}%
\pgfsetfillcolor{currentfill}%
\pgfsetfillopacity{0.422743}%
\pgfsetlinewidth{1.003750pt}%
\definecolor{currentstroke}{rgb}{0.121569,0.466667,0.705882}%
\pgfsetstrokecolor{currentstroke}%
\pgfsetstrokeopacity{0.422743}%
\pgfsetdash{}{0pt}%
\pgfpathmoveto{\pgfqpoint{1.534883in}{1.811072in}}%
\pgfpathcurveto{\pgfqpoint{1.543119in}{1.811072in}}{\pgfqpoint{1.551019in}{1.814345in}}{\pgfqpoint{1.556843in}{1.820169in}}%
\pgfpathcurveto{\pgfqpoint{1.562667in}{1.825993in}}{\pgfqpoint{1.565939in}{1.833893in}}{\pgfqpoint{1.565939in}{1.842129in}}%
\pgfpathcurveto{\pgfqpoint{1.565939in}{1.850365in}}{\pgfqpoint{1.562667in}{1.858265in}}{\pgfqpoint{1.556843in}{1.864089in}}%
\pgfpathcurveto{\pgfqpoint{1.551019in}{1.869913in}}{\pgfqpoint{1.543119in}{1.873185in}}{\pgfqpoint{1.534883in}{1.873185in}}%
\pgfpathcurveto{\pgfqpoint{1.526647in}{1.873185in}}{\pgfqpoint{1.518747in}{1.869913in}}{\pgfqpoint{1.512923in}{1.864089in}}%
\pgfpathcurveto{\pgfqpoint{1.507099in}{1.858265in}}{\pgfqpoint{1.503826in}{1.850365in}}{\pgfqpoint{1.503826in}{1.842129in}}%
\pgfpathcurveto{\pgfqpoint{1.503826in}{1.833893in}}{\pgfqpoint{1.507099in}{1.825993in}}{\pgfqpoint{1.512923in}{1.820169in}}%
\pgfpathcurveto{\pgfqpoint{1.518747in}{1.814345in}}{\pgfqpoint{1.526647in}{1.811072in}}{\pgfqpoint{1.534883in}{1.811072in}}%
\pgfpathclose%
\pgfusepath{stroke,fill}%
\end{pgfscope}%
\begin{pgfscope}%
\pgfpathrectangle{\pgfqpoint{0.100000in}{0.212622in}}{\pgfqpoint{3.696000in}{3.696000in}}%
\pgfusepath{clip}%
\pgfsetbuttcap%
\pgfsetroundjoin%
\definecolor{currentfill}{rgb}{0.121569,0.466667,0.705882}%
\pgfsetfillcolor{currentfill}%
\pgfsetfillopacity{0.422958}%
\pgfsetlinewidth{1.003750pt}%
\definecolor{currentstroke}{rgb}{0.121569,0.466667,0.705882}%
\pgfsetstrokecolor{currentstroke}%
\pgfsetstrokeopacity{0.422958}%
\pgfsetdash{}{0pt}%
\pgfpathmoveto{\pgfqpoint{2.009929in}{1.974938in}}%
\pgfpathcurveto{\pgfqpoint{2.018165in}{1.974938in}}{\pgfqpoint{2.026065in}{1.978210in}}{\pgfqpoint{2.031889in}{1.984034in}}%
\pgfpathcurveto{\pgfqpoint{2.037713in}{1.989858in}}{\pgfqpoint{2.040985in}{1.997758in}}{\pgfqpoint{2.040985in}{2.005994in}}%
\pgfpathcurveto{\pgfqpoint{2.040985in}{2.014231in}}{\pgfqpoint{2.037713in}{2.022131in}}{\pgfqpoint{2.031889in}{2.027955in}}%
\pgfpathcurveto{\pgfqpoint{2.026065in}{2.033779in}}{\pgfqpoint{2.018165in}{2.037051in}}{\pgfqpoint{2.009929in}{2.037051in}}%
\pgfpathcurveto{\pgfqpoint{2.001692in}{2.037051in}}{\pgfqpoint{1.993792in}{2.033779in}}{\pgfqpoint{1.987968in}{2.027955in}}%
\pgfpathcurveto{\pgfqpoint{1.982144in}{2.022131in}}{\pgfqpoint{1.978872in}{2.014231in}}{\pgfqpoint{1.978872in}{2.005994in}}%
\pgfpathcurveto{\pgfqpoint{1.978872in}{1.997758in}}{\pgfqpoint{1.982144in}{1.989858in}}{\pgfqpoint{1.987968in}{1.984034in}}%
\pgfpathcurveto{\pgfqpoint{1.993792in}{1.978210in}}{\pgfqpoint{2.001692in}{1.974938in}}{\pgfqpoint{2.009929in}{1.974938in}}%
\pgfpathclose%
\pgfusepath{stroke,fill}%
\end{pgfscope}%
\begin{pgfscope}%
\pgfpathrectangle{\pgfqpoint{0.100000in}{0.212622in}}{\pgfqpoint{3.696000in}{3.696000in}}%
\pgfusepath{clip}%
\pgfsetbuttcap%
\pgfsetroundjoin%
\definecolor{currentfill}{rgb}{0.121569,0.466667,0.705882}%
\pgfsetfillcolor{currentfill}%
\pgfsetfillopacity{0.423504}%
\pgfsetlinewidth{1.003750pt}%
\definecolor{currentstroke}{rgb}{0.121569,0.466667,0.705882}%
\pgfsetstrokecolor{currentstroke}%
\pgfsetstrokeopacity{0.423504}%
\pgfsetdash{}{0pt}%
\pgfpathmoveto{\pgfqpoint{1.532893in}{1.810915in}}%
\pgfpathcurveto{\pgfqpoint{1.541129in}{1.810915in}}{\pgfqpoint{1.549029in}{1.814187in}}{\pgfqpoint{1.554853in}{1.820011in}}%
\pgfpathcurveto{\pgfqpoint{1.560677in}{1.825835in}}{\pgfqpoint{1.563949in}{1.833735in}}{\pgfqpoint{1.563949in}{1.841972in}}%
\pgfpathcurveto{\pgfqpoint{1.563949in}{1.850208in}}{\pgfqpoint{1.560677in}{1.858108in}}{\pgfqpoint{1.554853in}{1.863932in}}%
\pgfpathcurveto{\pgfqpoint{1.549029in}{1.869756in}}{\pgfqpoint{1.541129in}{1.873028in}}{\pgfqpoint{1.532893in}{1.873028in}}%
\pgfpathcurveto{\pgfqpoint{1.524657in}{1.873028in}}{\pgfqpoint{1.516757in}{1.869756in}}{\pgfqpoint{1.510933in}{1.863932in}}%
\pgfpathcurveto{\pgfqpoint{1.505109in}{1.858108in}}{\pgfqpoint{1.501836in}{1.850208in}}{\pgfqpoint{1.501836in}{1.841972in}}%
\pgfpathcurveto{\pgfqpoint{1.501836in}{1.833735in}}{\pgfqpoint{1.505109in}{1.825835in}}{\pgfqpoint{1.510933in}{1.820011in}}%
\pgfpathcurveto{\pgfqpoint{1.516757in}{1.814187in}}{\pgfqpoint{1.524657in}{1.810915in}}{\pgfqpoint{1.532893in}{1.810915in}}%
\pgfpathclose%
\pgfusepath{stroke,fill}%
\end{pgfscope}%
\begin{pgfscope}%
\pgfpathrectangle{\pgfqpoint{0.100000in}{0.212622in}}{\pgfqpoint{3.696000in}{3.696000in}}%
\pgfusepath{clip}%
\pgfsetbuttcap%
\pgfsetroundjoin%
\definecolor{currentfill}{rgb}{0.121569,0.466667,0.705882}%
\pgfsetfillcolor{currentfill}%
\pgfsetfillopacity{0.424035}%
\pgfsetlinewidth{1.003750pt}%
\definecolor{currentstroke}{rgb}{0.121569,0.466667,0.705882}%
\pgfsetstrokecolor{currentstroke}%
\pgfsetstrokeopacity{0.424035}%
\pgfsetdash{}{0pt}%
\pgfpathmoveto{\pgfqpoint{1.531454in}{1.810403in}}%
\pgfpathcurveto{\pgfqpoint{1.539690in}{1.810403in}}{\pgfqpoint{1.547590in}{1.813675in}}{\pgfqpoint{1.553414in}{1.819499in}}%
\pgfpathcurveto{\pgfqpoint{1.559238in}{1.825323in}}{\pgfqpoint{1.562510in}{1.833223in}}{\pgfqpoint{1.562510in}{1.841459in}}%
\pgfpathcurveto{\pgfqpoint{1.562510in}{1.849696in}}{\pgfqpoint{1.559238in}{1.857596in}}{\pgfqpoint{1.553414in}{1.863420in}}%
\pgfpathcurveto{\pgfqpoint{1.547590in}{1.869243in}}{\pgfqpoint{1.539690in}{1.872516in}}{\pgfqpoint{1.531454in}{1.872516in}}%
\pgfpathcurveto{\pgfqpoint{1.523217in}{1.872516in}}{\pgfqpoint{1.515317in}{1.869243in}}{\pgfqpoint{1.509493in}{1.863420in}}%
\pgfpathcurveto{\pgfqpoint{1.503669in}{1.857596in}}{\pgfqpoint{1.500397in}{1.849696in}}{\pgfqpoint{1.500397in}{1.841459in}}%
\pgfpathcurveto{\pgfqpoint{1.500397in}{1.833223in}}{\pgfqpoint{1.503669in}{1.825323in}}{\pgfqpoint{1.509493in}{1.819499in}}%
\pgfpathcurveto{\pgfqpoint{1.515317in}{1.813675in}}{\pgfqpoint{1.523217in}{1.810403in}}{\pgfqpoint{1.531454in}{1.810403in}}%
\pgfpathclose%
\pgfusepath{stroke,fill}%
\end{pgfscope}%
\begin{pgfscope}%
\pgfpathrectangle{\pgfqpoint{0.100000in}{0.212622in}}{\pgfqpoint{3.696000in}{3.696000in}}%
\pgfusepath{clip}%
\pgfsetbuttcap%
\pgfsetroundjoin%
\definecolor{currentfill}{rgb}{0.121569,0.466667,0.705882}%
\pgfsetfillcolor{currentfill}%
\pgfsetfillopacity{0.424770}%
\pgfsetlinewidth{1.003750pt}%
\definecolor{currentstroke}{rgb}{0.121569,0.466667,0.705882}%
\pgfsetstrokecolor{currentstroke}%
\pgfsetstrokeopacity{0.424770}%
\pgfsetdash{}{0pt}%
\pgfpathmoveto{\pgfqpoint{1.528703in}{1.808131in}}%
\pgfpathcurveto{\pgfqpoint{1.536939in}{1.808131in}}{\pgfqpoint{1.544839in}{1.811403in}}{\pgfqpoint{1.550663in}{1.817227in}}%
\pgfpathcurveto{\pgfqpoint{1.556487in}{1.823051in}}{\pgfqpoint{1.559759in}{1.830951in}}{\pgfqpoint{1.559759in}{1.839187in}}%
\pgfpathcurveto{\pgfqpoint{1.559759in}{1.847423in}}{\pgfqpoint{1.556487in}{1.855323in}}{\pgfqpoint{1.550663in}{1.861147in}}%
\pgfpathcurveto{\pgfqpoint{1.544839in}{1.866971in}}{\pgfqpoint{1.536939in}{1.870244in}}{\pgfqpoint{1.528703in}{1.870244in}}%
\pgfpathcurveto{\pgfqpoint{1.520466in}{1.870244in}}{\pgfqpoint{1.512566in}{1.866971in}}{\pgfqpoint{1.506742in}{1.861147in}}%
\pgfpathcurveto{\pgfqpoint{1.500918in}{1.855323in}}{\pgfqpoint{1.497646in}{1.847423in}}{\pgfqpoint{1.497646in}{1.839187in}}%
\pgfpathcurveto{\pgfqpoint{1.497646in}{1.830951in}}{\pgfqpoint{1.500918in}{1.823051in}}{\pgfqpoint{1.506742in}{1.817227in}}%
\pgfpathcurveto{\pgfqpoint{1.512566in}{1.811403in}}{\pgfqpoint{1.520466in}{1.808131in}}{\pgfqpoint{1.528703in}{1.808131in}}%
\pgfpathclose%
\pgfusepath{stroke,fill}%
\end{pgfscope}%
\begin{pgfscope}%
\pgfpathrectangle{\pgfqpoint{0.100000in}{0.212622in}}{\pgfqpoint{3.696000in}{3.696000in}}%
\pgfusepath{clip}%
\pgfsetbuttcap%
\pgfsetroundjoin%
\definecolor{currentfill}{rgb}{0.121569,0.466667,0.705882}%
\pgfsetfillcolor{currentfill}%
\pgfsetfillopacity{0.424903}%
\pgfsetlinewidth{1.003750pt}%
\definecolor{currentstroke}{rgb}{0.121569,0.466667,0.705882}%
\pgfsetstrokecolor{currentstroke}%
\pgfsetstrokeopacity{0.424903}%
\pgfsetdash{}{0pt}%
\pgfpathmoveto{\pgfqpoint{2.011301in}{1.972812in}}%
\pgfpathcurveto{\pgfqpoint{2.019538in}{1.972812in}}{\pgfqpoint{2.027438in}{1.976085in}}{\pgfqpoint{2.033262in}{1.981909in}}%
\pgfpathcurveto{\pgfqpoint{2.039086in}{1.987732in}}{\pgfqpoint{2.042358in}{1.995633in}}{\pgfqpoint{2.042358in}{2.003869in}}%
\pgfpathcurveto{\pgfqpoint{2.042358in}{2.012105in}}{\pgfqpoint{2.039086in}{2.020005in}}{\pgfqpoint{2.033262in}{2.025829in}}%
\pgfpathcurveto{\pgfqpoint{2.027438in}{2.031653in}}{\pgfqpoint{2.019538in}{2.034925in}}{\pgfqpoint{2.011301in}{2.034925in}}%
\pgfpathcurveto{\pgfqpoint{2.003065in}{2.034925in}}{\pgfqpoint{1.995165in}{2.031653in}}{\pgfqpoint{1.989341in}{2.025829in}}%
\pgfpathcurveto{\pgfqpoint{1.983517in}{2.020005in}}{\pgfqpoint{1.980245in}{2.012105in}}{\pgfqpoint{1.980245in}{2.003869in}}%
\pgfpathcurveto{\pgfqpoint{1.980245in}{1.995633in}}{\pgfqpoint{1.983517in}{1.987732in}}{\pgfqpoint{1.989341in}{1.981909in}}%
\pgfpathcurveto{\pgfqpoint{1.995165in}{1.976085in}}{\pgfqpoint{2.003065in}{1.972812in}}{\pgfqpoint{2.011301in}{1.972812in}}%
\pgfpathclose%
\pgfusepath{stroke,fill}%
\end{pgfscope}%
\begin{pgfscope}%
\pgfpathrectangle{\pgfqpoint{0.100000in}{0.212622in}}{\pgfqpoint{3.696000in}{3.696000in}}%
\pgfusepath{clip}%
\pgfsetbuttcap%
\pgfsetroundjoin%
\definecolor{currentfill}{rgb}{0.121569,0.466667,0.705882}%
\pgfsetfillcolor{currentfill}%
\pgfsetfillopacity{0.425139}%
\pgfsetlinewidth{1.003750pt}%
\definecolor{currentstroke}{rgb}{0.121569,0.466667,0.705882}%
\pgfsetstrokecolor{currentstroke}%
\pgfsetstrokeopacity{0.425139}%
\pgfsetdash{}{0pt}%
\pgfpathmoveto{\pgfqpoint{1.527693in}{1.807889in}}%
\pgfpathcurveto{\pgfqpoint{1.535929in}{1.807889in}}{\pgfqpoint{1.543829in}{1.811162in}}{\pgfqpoint{1.549653in}{1.816986in}}%
\pgfpathcurveto{\pgfqpoint{1.555477in}{1.822809in}}{\pgfqpoint{1.558749in}{1.830710in}}{\pgfqpoint{1.558749in}{1.838946in}}%
\pgfpathcurveto{\pgfqpoint{1.558749in}{1.847182in}}{\pgfqpoint{1.555477in}{1.855082in}}{\pgfqpoint{1.549653in}{1.860906in}}%
\pgfpathcurveto{\pgfqpoint{1.543829in}{1.866730in}}{\pgfqpoint{1.535929in}{1.870002in}}{\pgfqpoint{1.527693in}{1.870002in}}%
\pgfpathcurveto{\pgfqpoint{1.519456in}{1.870002in}}{\pgfqpoint{1.511556in}{1.866730in}}{\pgfqpoint{1.505732in}{1.860906in}}%
\pgfpathcurveto{\pgfqpoint{1.499909in}{1.855082in}}{\pgfqpoint{1.496636in}{1.847182in}}{\pgfqpoint{1.496636in}{1.838946in}}%
\pgfpathcurveto{\pgfqpoint{1.496636in}{1.830710in}}{\pgfqpoint{1.499909in}{1.822809in}}{\pgfqpoint{1.505732in}{1.816986in}}%
\pgfpathcurveto{\pgfqpoint{1.511556in}{1.811162in}}{\pgfqpoint{1.519456in}{1.807889in}}{\pgfqpoint{1.527693in}{1.807889in}}%
\pgfpathclose%
\pgfusepath{stroke,fill}%
\end{pgfscope}%
\begin{pgfscope}%
\pgfpathrectangle{\pgfqpoint{0.100000in}{0.212622in}}{\pgfqpoint{3.696000in}{3.696000in}}%
\pgfusepath{clip}%
\pgfsetbuttcap%
\pgfsetroundjoin%
\definecolor{currentfill}{rgb}{0.121569,0.466667,0.705882}%
\pgfsetfillcolor{currentfill}%
\pgfsetfillopacity{0.425769}%
\pgfsetlinewidth{1.003750pt}%
\definecolor{currentstroke}{rgb}{0.121569,0.466667,0.705882}%
\pgfsetstrokecolor{currentstroke}%
\pgfsetstrokeopacity{0.425769}%
\pgfsetdash{}{0pt}%
\pgfpathmoveto{\pgfqpoint{1.525844in}{1.807187in}}%
\pgfpathcurveto{\pgfqpoint{1.534081in}{1.807187in}}{\pgfqpoint{1.541981in}{1.810459in}}{\pgfqpoint{1.547805in}{1.816283in}}%
\pgfpathcurveto{\pgfqpoint{1.553629in}{1.822107in}}{\pgfqpoint{1.556901in}{1.830007in}}{\pgfqpoint{1.556901in}{1.838243in}}%
\pgfpathcurveto{\pgfqpoint{1.556901in}{1.846480in}}{\pgfqpoint{1.553629in}{1.854380in}}{\pgfqpoint{1.547805in}{1.860204in}}%
\pgfpathcurveto{\pgfqpoint{1.541981in}{1.866027in}}{\pgfqpoint{1.534081in}{1.869300in}}{\pgfqpoint{1.525844in}{1.869300in}}%
\pgfpathcurveto{\pgfqpoint{1.517608in}{1.869300in}}{\pgfqpoint{1.509708in}{1.866027in}}{\pgfqpoint{1.503884in}{1.860204in}}%
\pgfpathcurveto{\pgfqpoint{1.498060in}{1.854380in}}{\pgfqpoint{1.494788in}{1.846480in}}{\pgfqpoint{1.494788in}{1.838243in}}%
\pgfpathcurveto{\pgfqpoint{1.494788in}{1.830007in}}{\pgfqpoint{1.498060in}{1.822107in}}{\pgfqpoint{1.503884in}{1.816283in}}%
\pgfpathcurveto{\pgfqpoint{1.509708in}{1.810459in}}{\pgfqpoint{1.517608in}{1.807187in}}{\pgfqpoint{1.525844in}{1.807187in}}%
\pgfpathclose%
\pgfusepath{stroke,fill}%
\end{pgfscope}%
\begin{pgfscope}%
\pgfpathrectangle{\pgfqpoint{0.100000in}{0.212622in}}{\pgfqpoint{3.696000in}{3.696000in}}%
\pgfusepath{clip}%
\pgfsetbuttcap%
\pgfsetroundjoin%
\definecolor{currentfill}{rgb}{0.121569,0.466667,0.705882}%
\pgfsetfillcolor{currentfill}%
\pgfsetfillopacity{0.426680}%
\pgfsetlinewidth{1.003750pt}%
\definecolor{currentstroke}{rgb}{0.121569,0.466667,0.705882}%
\pgfsetstrokecolor{currentstroke}%
\pgfsetstrokeopacity{0.426680}%
\pgfsetdash{}{0pt}%
\pgfpathmoveto{\pgfqpoint{1.522570in}{1.804237in}}%
\pgfpathcurveto{\pgfqpoint{1.530806in}{1.804237in}}{\pgfqpoint{1.538706in}{1.807509in}}{\pgfqpoint{1.544530in}{1.813333in}}%
\pgfpathcurveto{\pgfqpoint{1.550354in}{1.819157in}}{\pgfqpoint{1.553627in}{1.827057in}}{\pgfqpoint{1.553627in}{1.835293in}}%
\pgfpathcurveto{\pgfqpoint{1.553627in}{1.843530in}}{\pgfqpoint{1.550354in}{1.851430in}}{\pgfqpoint{1.544530in}{1.857254in}}%
\pgfpathcurveto{\pgfqpoint{1.538706in}{1.863078in}}{\pgfqpoint{1.530806in}{1.866350in}}{\pgfqpoint{1.522570in}{1.866350in}}%
\pgfpathcurveto{\pgfqpoint{1.514334in}{1.866350in}}{\pgfqpoint{1.506434in}{1.863078in}}{\pgfqpoint{1.500610in}{1.857254in}}%
\pgfpathcurveto{\pgfqpoint{1.494786in}{1.851430in}}{\pgfqpoint{1.491514in}{1.843530in}}{\pgfqpoint{1.491514in}{1.835293in}}%
\pgfpathcurveto{\pgfqpoint{1.491514in}{1.827057in}}{\pgfqpoint{1.494786in}{1.819157in}}{\pgfqpoint{1.500610in}{1.813333in}}%
\pgfpathcurveto{\pgfqpoint{1.506434in}{1.807509in}}{\pgfqpoint{1.514334in}{1.804237in}}{\pgfqpoint{1.522570in}{1.804237in}}%
\pgfpathclose%
\pgfusepath{stroke,fill}%
\end{pgfscope}%
\begin{pgfscope}%
\pgfpathrectangle{\pgfqpoint{0.100000in}{0.212622in}}{\pgfqpoint{3.696000in}{3.696000in}}%
\pgfusepath{clip}%
\pgfsetbuttcap%
\pgfsetroundjoin%
\definecolor{currentfill}{rgb}{0.121569,0.466667,0.705882}%
\pgfsetfillcolor{currentfill}%
\pgfsetfillopacity{0.426956}%
\pgfsetlinewidth{1.003750pt}%
\definecolor{currentstroke}{rgb}{0.121569,0.466667,0.705882}%
\pgfsetstrokecolor{currentstroke}%
\pgfsetstrokeopacity{0.426956}%
\pgfsetdash{}{0pt}%
\pgfpathmoveto{\pgfqpoint{1.521318in}{1.803149in}}%
\pgfpathcurveto{\pgfqpoint{1.529554in}{1.803149in}}{\pgfqpoint{1.537454in}{1.806421in}}{\pgfqpoint{1.543278in}{1.812245in}}%
\pgfpathcurveto{\pgfqpoint{1.549102in}{1.818069in}}{\pgfqpoint{1.552374in}{1.825969in}}{\pgfqpoint{1.552374in}{1.834206in}}%
\pgfpathcurveto{\pgfqpoint{1.552374in}{1.842442in}}{\pgfqpoint{1.549102in}{1.850342in}}{\pgfqpoint{1.543278in}{1.856166in}}%
\pgfpathcurveto{\pgfqpoint{1.537454in}{1.861990in}}{\pgfqpoint{1.529554in}{1.865262in}}{\pgfqpoint{1.521318in}{1.865262in}}%
\pgfpathcurveto{\pgfqpoint{1.513081in}{1.865262in}}{\pgfqpoint{1.505181in}{1.861990in}}{\pgfqpoint{1.499357in}{1.856166in}}%
\pgfpathcurveto{\pgfqpoint{1.493533in}{1.850342in}}{\pgfqpoint{1.490261in}{1.842442in}}{\pgfqpoint{1.490261in}{1.834206in}}%
\pgfpathcurveto{\pgfqpoint{1.490261in}{1.825969in}}{\pgfqpoint{1.493533in}{1.818069in}}{\pgfqpoint{1.499357in}{1.812245in}}%
\pgfpathcurveto{\pgfqpoint{1.505181in}{1.806421in}}{\pgfqpoint{1.513081in}{1.803149in}}{\pgfqpoint{1.521318in}{1.803149in}}%
\pgfpathclose%
\pgfusepath{stroke,fill}%
\end{pgfscope}%
\begin{pgfscope}%
\pgfpathrectangle{\pgfqpoint{0.100000in}{0.212622in}}{\pgfqpoint{3.696000in}{3.696000in}}%
\pgfusepath{clip}%
\pgfsetbuttcap%
\pgfsetroundjoin%
\definecolor{currentfill}{rgb}{0.121569,0.466667,0.705882}%
\pgfsetfillcolor{currentfill}%
\pgfsetfillopacity{0.427649}%
\pgfsetlinewidth{1.003750pt}%
\definecolor{currentstroke}{rgb}{0.121569,0.466667,0.705882}%
\pgfsetstrokecolor{currentstroke}%
\pgfsetstrokeopacity{0.427649}%
\pgfsetdash{}{0pt}%
\pgfpathmoveto{\pgfqpoint{1.519095in}{1.802336in}}%
\pgfpathcurveto{\pgfqpoint{1.527331in}{1.802336in}}{\pgfqpoint{1.535231in}{1.805608in}}{\pgfqpoint{1.541055in}{1.811432in}}%
\pgfpathcurveto{\pgfqpoint{1.546879in}{1.817256in}}{\pgfqpoint{1.550151in}{1.825156in}}{\pgfqpoint{1.550151in}{1.833393in}}%
\pgfpathcurveto{\pgfqpoint{1.550151in}{1.841629in}}{\pgfqpoint{1.546879in}{1.849529in}}{\pgfqpoint{1.541055in}{1.855353in}}%
\pgfpathcurveto{\pgfqpoint{1.535231in}{1.861177in}}{\pgfqpoint{1.527331in}{1.864449in}}{\pgfqpoint{1.519095in}{1.864449in}}%
\pgfpathcurveto{\pgfqpoint{1.510858in}{1.864449in}}{\pgfqpoint{1.502958in}{1.861177in}}{\pgfqpoint{1.497134in}{1.855353in}}%
\pgfpathcurveto{\pgfqpoint{1.491310in}{1.849529in}}{\pgfqpoint{1.488038in}{1.841629in}}{\pgfqpoint{1.488038in}{1.833393in}}%
\pgfpathcurveto{\pgfqpoint{1.488038in}{1.825156in}}{\pgfqpoint{1.491310in}{1.817256in}}{\pgfqpoint{1.497134in}{1.811432in}}%
\pgfpathcurveto{\pgfqpoint{1.502958in}{1.805608in}}{\pgfqpoint{1.510858in}{1.802336in}}{\pgfqpoint{1.519095in}{1.802336in}}%
\pgfpathclose%
\pgfusepath{stroke,fill}%
\end{pgfscope}%
\begin{pgfscope}%
\pgfpathrectangle{\pgfqpoint{0.100000in}{0.212622in}}{\pgfqpoint{3.696000in}{3.696000in}}%
\pgfusepath{clip}%
\pgfsetbuttcap%
\pgfsetroundjoin%
\definecolor{currentfill}{rgb}{0.121569,0.466667,0.705882}%
\pgfsetfillcolor{currentfill}%
\pgfsetfillopacity{0.427781}%
\pgfsetlinewidth{1.003750pt}%
\definecolor{currentstroke}{rgb}{0.121569,0.466667,0.705882}%
\pgfsetstrokecolor{currentstroke}%
\pgfsetstrokeopacity{0.427781}%
\pgfsetdash{}{0pt}%
\pgfpathmoveto{\pgfqpoint{2.013103in}{1.971139in}}%
\pgfpathcurveto{\pgfqpoint{2.021339in}{1.971139in}}{\pgfqpoint{2.029239in}{1.974411in}}{\pgfqpoint{2.035063in}{1.980235in}}%
\pgfpathcurveto{\pgfqpoint{2.040887in}{1.986059in}}{\pgfqpoint{2.044159in}{1.993959in}}{\pgfqpoint{2.044159in}{2.002196in}}%
\pgfpathcurveto{\pgfqpoint{2.044159in}{2.010432in}}{\pgfqpoint{2.040887in}{2.018332in}}{\pgfqpoint{2.035063in}{2.024156in}}%
\pgfpathcurveto{\pgfqpoint{2.029239in}{2.029980in}}{\pgfqpoint{2.021339in}{2.033252in}}{\pgfqpoint{2.013103in}{2.033252in}}%
\pgfpathcurveto{\pgfqpoint{2.004866in}{2.033252in}}{\pgfqpoint{1.996966in}{2.029980in}}{\pgfqpoint{1.991142in}{2.024156in}}%
\pgfpathcurveto{\pgfqpoint{1.985318in}{2.018332in}}{\pgfqpoint{1.982046in}{2.010432in}}{\pgfqpoint{1.982046in}{2.002196in}}%
\pgfpathcurveto{\pgfqpoint{1.982046in}{1.993959in}}{\pgfqpoint{1.985318in}{1.986059in}}{\pgfqpoint{1.991142in}{1.980235in}}%
\pgfpathcurveto{\pgfqpoint{1.996966in}{1.974411in}}{\pgfqpoint{2.004866in}{1.971139in}}{\pgfqpoint{2.013103in}{1.971139in}}%
\pgfpathclose%
\pgfusepath{stroke,fill}%
\end{pgfscope}%
\begin{pgfscope}%
\pgfpathrectangle{\pgfqpoint{0.100000in}{0.212622in}}{\pgfqpoint{3.696000in}{3.696000in}}%
\pgfusepath{clip}%
\pgfsetbuttcap%
\pgfsetroundjoin%
\definecolor{currentfill}{rgb}{0.121569,0.466667,0.705882}%
\pgfsetfillcolor{currentfill}%
\pgfsetfillopacity{0.428654}%
\pgfsetlinewidth{1.003750pt}%
\definecolor{currentstroke}{rgb}{0.121569,0.466667,0.705882}%
\pgfsetstrokecolor{currentstroke}%
\pgfsetstrokeopacity{0.428654}%
\pgfsetdash{}{0pt}%
\pgfpathmoveto{\pgfqpoint{1.515775in}{1.798153in}}%
\pgfpathcurveto{\pgfqpoint{1.524011in}{1.798153in}}{\pgfqpoint{1.531911in}{1.801425in}}{\pgfqpoint{1.537735in}{1.807249in}}%
\pgfpathcurveto{\pgfqpoint{1.543559in}{1.813073in}}{\pgfqpoint{1.546831in}{1.820973in}}{\pgfqpoint{1.546831in}{1.829210in}}%
\pgfpathcurveto{\pgfqpoint{1.546831in}{1.837446in}}{\pgfqpoint{1.543559in}{1.845346in}}{\pgfqpoint{1.537735in}{1.851170in}}%
\pgfpathcurveto{\pgfqpoint{1.531911in}{1.856994in}}{\pgfqpoint{1.524011in}{1.860266in}}{\pgfqpoint{1.515775in}{1.860266in}}%
\pgfpathcurveto{\pgfqpoint{1.507539in}{1.860266in}}{\pgfqpoint{1.499639in}{1.856994in}}{\pgfqpoint{1.493815in}{1.851170in}}%
\pgfpathcurveto{\pgfqpoint{1.487991in}{1.845346in}}{\pgfqpoint{1.484718in}{1.837446in}}{\pgfqpoint{1.484718in}{1.829210in}}%
\pgfpathcurveto{\pgfqpoint{1.484718in}{1.820973in}}{\pgfqpoint{1.487991in}{1.813073in}}{\pgfqpoint{1.493815in}{1.807249in}}%
\pgfpathcurveto{\pgfqpoint{1.499639in}{1.801425in}}{\pgfqpoint{1.507539in}{1.798153in}}{\pgfqpoint{1.515775in}{1.798153in}}%
\pgfpathclose%
\pgfusepath{stroke,fill}%
\end{pgfscope}%
\begin{pgfscope}%
\pgfpathrectangle{\pgfqpoint{0.100000in}{0.212622in}}{\pgfqpoint{3.696000in}{3.696000in}}%
\pgfusepath{clip}%
\pgfsetbuttcap%
\pgfsetroundjoin%
\definecolor{currentfill}{rgb}{0.121569,0.466667,0.705882}%
\pgfsetfillcolor{currentfill}%
\pgfsetfillopacity{0.429309}%
\pgfsetlinewidth{1.003750pt}%
\definecolor{currentstroke}{rgb}{0.121569,0.466667,0.705882}%
\pgfsetstrokecolor{currentstroke}%
\pgfsetstrokeopacity{0.429309}%
\pgfsetdash{}{0pt}%
\pgfpathmoveto{\pgfqpoint{1.513937in}{1.797758in}}%
\pgfpathcurveto{\pgfqpoint{1.522173in}{1.797758in}}{\pgfqpoint{1.530073in}{1.801030in}}{\pgfqpoint{1.535897in}{1.806854in}}%
\pgfpathcurveto{\pgfqpoint{1.541721in}{1.812678in}}{\pgfqpoint{1.544993in}{1.820578in}}{\pgfqpoint{1.544993in}{1.828815in}}%
\pgfpathcurveto{\pgfqpoint{1.544993in}{1.837051in}}{\pgfqpoint{1.541721in}{1.844951in}}{\pgfqpoint{1.535897in}{1.850775in}}%
\pgfpathcurveto{\pgfqpoint{1.530073in}{1.856599in}}{\pgfqpoint{1.522173in}{1.859871in}}{\pgfqpoint{1.513937in}{1.859871in}}%
\pgfpathcurveto{\pgfqpoint{1.505700in}{1.859871in}}{\pgfqpoint{1.497800in}{1.856599in}}{\pgfqpoint{1.491976in}{1.850775in}}%
\pgfpathcurveto{\pgfqpoint{1.486153in}{1.844951in}}{\pgfqpoint{1.482880in}{1.837051in}}{\pgfqpoint{1.482880in}{1.828815in}}%
\pgfpathcurveto{\pgfqpoint{1.482880in}{1.820578in}}{\pgfqpoint{1.486153in}{1.812678in}}{\pgfqpoint{1.491976in}{1.806854in}}%
\pgfpathcurveto{\pgfqpoint{1.497800in}{1.801030in}}{\pgfqpoint{1.505700in}{1.797758in}}{\pgfqpoint{1.513937in}{1.797758in}}%
\pgfpathclose%
\pgfusepath{stroke,fill}%
\end{pgfscope}%
\begin{pgfscope}%
\pgfpathrectangle{\pgfqpoint{0.100000in}{0.212622in}}{\pgfqpoint{3.696000in}{3.696000in}}%
\pgfusepath{clip}%
\pgfsetbuttcap%
\pgfsetroundjoin%
\definecolor{currentfill}{rgb}{0.121569,0.466667,0.705882}%
\pgfsetfillcolor{currentfill}%
\pgfsetfillopacity{0.430437}%
\pgfsetlinewidth{1.003750pt}%
\definecolor{currentstroke}{rgb}{0.121569,0.466667,0.705882}%
\pgfsetstrokecolor{currentstroke}%
\pgfsetstrokeopacity{0.430437}%
\pgfsetdash{}{0pt}%
\pgfpathmoveto{\pgfqpoint{1.510565in}{1.796666in}}%
\pgfpathcurveto{\pgfqpoint{1.518802in}{1.796666in}}{\pgfqpoint{1.526702in}{1.799938in}}{\pgfqpoint{1.532525in}{1.805762in}}%
\pgfpathcurveto{\pgfqpoint{1.538349in}{1.811586in}}{\pgfqpoint{1.541622in}{1.819486in}}{\pgfqpoint{1.541622in}{1.827722in}}%
\pgfpathcurveto{\pgfqpoint{1.541622in}{1.835959in}}{\pgfqpoint{1.538349in}{1.843859in}}{\pgfqpoint{1.532525in}{1.849683in}}%
\pgfpathcurveto{\pgfqpoint{1.526702in}{1.855507in}}{\pgfqpoint{1.518802in}{1.858779in}}{\pgfqpoint{1.510565in}{1.858779in}}%
\pgfpathcurveto{\pgfqpoint{1.502329in}{1.858779in}}{\pgfqpoint{1.494429in}{1.855507in}}{\pgfqpoint{1.488605in}{1.849683in}}%
\pgfpathcurveto{\pgfqpoint{1.482781in}{1.843859in}}{\pgfqpoint{1.479509in}{1.835959in}}{\pgfqpoint{1.479509in}{1.827722in}}%
\pgfpathcurveto{\pgfqpoint{1.479509in}{1.819486in}}{\pgfqpoint{1.482781in}{1.811586in}}{\pgfqpoint{1.488605in}{1.805762in}}%
\pgfpathcurveto{\pgfqpoint{1.494429in}{1.799938in}}{\pgfqpoint{1.502329in}{1.796666in}}{\pgfqpoint{1.510565in}{1.796666in}}%
\pgfpathclose%
\pgfusepath{stroke,fill}%
\end{pgfscope}%
\begin{pgfscope}%
\pgfpathrectangle{\pgfqpoint{0.100000in}{0.212622in}}{\pgfqpoint{3.696000in}{3.696000in}}%
\pgfusepath{clip}%
\pgfsetbuttcap%
\pgfsetroundjoin%
\definecolor{currentfill}{rgb}{0.121569,0.466667,0.705882}%
\pgfsetfillcolor{currentfill}%
\pgfsetfillopacity{0.430639}%
\pgfsetlinewidth{1.003750pt}%
\definecolor{currentstroke}{rgb}{0.121569,0.466667,0.705882}%
\pgfsetstrokecolor{currentstroke}%
\pgfsetstrokeopacity{0.430639}%
\pgfsetdash{}{0pt}%
\pgfpathmoveto{\pgfqpoint{2.014618in}{1.967202in}}%
\pgfpathcurveto{\pgfqpoint{2.022854in}{1.967202in}}{\pgfqpoint{2.030754in}{1.970474in}}{\pgfqpoint{2.036578in}{1.976298in}}%
\pgfpathcurveto{\pgfqpoint{2.042402in}{1.982122in}}{\pgfqpoint{2.045674in}{1.990022in}}{\pgfqpoint{2.045674in}{1.998258in}}%
\pgfpathcurveto{\pgfqpoint{2.045674in}{2.006494in}}{\pgfqpoint{2.042402in}{2.014394in}}{\pgfqpoint{2.036578in}{2.020218in}}%
\pgfpathcurveto{\pgfqpoint{2.030754in}{2.026042in}}{\pgfqpoint{2.022854in}{2.029315in}}{\pgfqpoint{2.014618in}{2.029315in}}%
\pgfpathcurveto{\pgfqpoint{2.006381in}{2.029315in}}{\pgfqpoint{1.998481in}{2.026042in}}{\pgfqpoint{1.992657in}{2.020218in}}%
\pgfpathcurveto{\pgfqpoint{1.986834in}{2.014394in}}{\pgfqpoint{1.983561in}{2.006494in}}{\pgfqpoint{1.983561in}{1.998258in}}%
\pgfpathcurveto{\pgfqpoint{1.983561in}{1.990022in}}{\pgfqpoint{1.986834in}{1.982122in}}{\pgfqpoint{1.992657in}{1.976298in}}%
\pgfpathcurveto{\pgfqpoint{1.998481in}{1.970474in}}{\pgfqpoint{2.006381in}{1.967202in}}{\pgfqpoint{2.014618in}{1.967202in}}%
\pgfpathclose%
\pgfusepath{stroke,fill}%
\end{pgfscope}%
\begin{pgfscope}%
\pgfpathrectangle{\pgfqpoint{0.100000in}{0.212622in}}{\pgfqpoint{3.696000in}{3.696000in}}%
\pgfusepath{clip}%
\pgfsetbuttcap%
\pgfsetroundjoin%
\definecolor{currentfill}{rgb}{0.121569,0.466667,0.705882}%
\pgfsetfillcolor{currentfill}%
\pgfsetfillopacity{0.432050}%
\pgfsetlinewidth{1.003750pt}%
\definecolor{currentstroke}{rgb}{0.121569,0.466667,0.705882}%
\pgfsetstrokecolor{currentstroke}%
\pgfsetstrokeopacity{0.432050}%
\pgfsetdash{}{0pt}%
\pgfpathmoveto{\pgfqpoint{1.505343in}{1.790524in}}%
\pgfpathcurveto{\pgfqpoint{1.513579in}{1.790524in}}{\pgfqpoint{1.521479in}{1.793797in}}{\pgfqpoint{1.527303in}{1.799621in}}%
\pgfpathcurveto{\pgfqpoint{1.533127in}{1.805445in}}{\pgfqpoint{1.536400in}{1.813345in}}{\pgfqpoint{1.536400in}{1.821581in}}%
\pgfpathcurveto{\pgfqpoint{1.536400in}{1.829817in}}{\pgfqpoint{1.533127in}{1.837717in}}{\pgfqpoint{1.527303in}{1.843541in}}%
\pgfpathcurveto{\pgfqpoint{1.521479in}{1.849365in}}{\pgfqpoint{1.513579in}{1.852637in}}{\pgfqpoint{1.505343in}{1.852637in}}%
\pgfpathcurveto{\pgfqpoint{1.497107in}{1.852637in}}{\pgfqpoint{1.489207in}{1.849365in}}{\pgfqpoint{1.483383in}{1.843541in}}%
\pgfpathcurveto{\pgfqpoint{1.477559in}{1.837717in}}{\pgfqpoint{1.474287in}{1.829817in}}{\pgfqpoint{1.474287in}{1.821581in}}%
\pgfpathcurveto{\pgfqpoint{1.474287in}{1.813345in}}{\pgfqpoint{1.477559in}{1.805445in}}{\pgfqpoint{1.483383in}{1.799621in}}%
\pgfpathcurveto{\pgfqpoint{1.489207in}{1.793797in}}{\pgfqpoint{1.497107in}{1.790524in}}{\pgfqpoint{1.505343in}{1.790524in}}%
\pgfpathclose%
\pgfusepath{stroke,fill}%
\end{pgfscope}%
\begin{pgfscope}%
\pgfpathrectangle{\pgfqpoint{0.100000in}{0.212622in}}{\pgfqpoint{3.696000in}{3.696000in}}%
\pgfusepath{clip}%
\pgfsetbuttcap%
\pgfsetroundjoin%
\definecolor{currentfill}{rgb}{0.121569,0.466667,0.705882}%
\pgfsetfillcolor{currentfill}%
\pgfsetfillopacity{0.433490}%
\pgfsetlinewidth{1.003750pt}%
\definecolor{currentstroke}{rgb}{0.121569,0.466667,0.705882}%
\pgfsetstrokecolor{currentstroke}%
\pgfsetstrokeopacity{0.433490}%
\pgfsetdash{}{0pt}%
\pgfpathmoveto{\pgfqpoint{1.500538in}{1.788501in}}%
\pgfpathcurveto{\pgfqpoint{1.508774in}{1.788501in}}{\pgfqpoint{1.516674in}{1.791773in}}{\pgfqpoint{1.522498in}{1.797597in}}%
\pgfpathcurveto{\pgfqpoint{1.528322in}{1.803421in}}{\pgfqpoint{1.531594in}{1.811321in}}{\pgfqpoint{1.531594in}{1.819557in}}%
\pgfpathcurveto{\pgfqpoint{1.531594in}{1.827793in}}{\pgfqpoint{1.528322in}{1.835693in}}{\pgfqpoint{1.522498in}{1.841517in}}%
\pgfpathcurveto{\pgfqpoint{1.516674in}{1.847341in}}{\pgfqpoint{1.508774in}{1.850614in}}{\pgfqpoint{1.500538in}{1.850614in}}%
\pgfpathcurveto{\pgfqpoint{1.492301in}{1.850614in}}{\pgfqpoint{1.484401in}{1.847341in}}{\pgfqpoint{1.478577in}{1.841517in}}%
\pgfpathcurveto{\pgfqpoint{1.472754in}{1.835693in}}{\pgfqpoint{1.469481in}{1.827793in}}{\pgfqpoint{1.469481in}{1.819557in}}%
\pgfpathcurveto{\pgfqpoint{1.469481in}{1.811321in}}{\pgfqpoint{1.472754in}{1.803421in}}{\pgfqpoint{1.478577in}{1.797597in}}%
\pgfpathcurveto{\pgfqpoint{1.484401in}{1.791773in}}{\pgfqpoint{1.492301in}{1.788501in}}{\pgfqpoint{1.500538in}{1.788501in}}%
\pgfpathclose%
\pgfusepath{stroke,fill}%
\end{pgfscope}%
\begin{pgfscope}%
\pgfpathrectangle{\pgfqpoint{0.100000in}{0.212622in}}{\pgfqpoint{3.696000in}{3.696000in}}%
\pgfusepath{clip}%
\pgfsetbuttcap%
\pgfsetroundjoin%
\definecolor{currentfill}{rgb}{0.121569,0.466667,0.705882}%
\pgfsetfillcolor{currentfill}%
\pgfsetfillopacity{0.433787}%
\pgfsetlinewidth{1.003750pt}%
\definecolor{currentstroke}{rgb}{0.121569,0.466667,0.705882}%
\pgfsetstrokecolor{currentstroke}%
\pgfsetstrokeopacity{0.433787}%
\pgfsetdash{}{0pt}%
\pgfpathmoveto{\pgfqpoint{2.016637in}{1.962729in}}%
\pgfpathcurveto{\pgfqpoint{2.024873in}{1.962729in}}{\pgfqpoint{2.032773in}{1.966001in}}{\pgfqpoint{2.038597in}{1.971825in}}%
\pgfpathcurveto{\pgfqpoint{2.044421in}{1.977649in}}{\pgfqpoint{2.047693in}{1.985549in}}{\pgfqpoint{2.047693in}{1.993786in}}%
\pgfpathcurveto{\pgfqpoint{2.047693in}{2.002022in}}{\pgfqpoint{2.044421in}{2.009922in}}{\pgfqpoint{2.038597in}{2.015746in}}%
\pgfpathcurveto{\pgfqpoint{2.032773in}{2.021570in}}{\pgfqpoint{2.024873in}{2.024842in}}{\pgfqpoint{2.016637in}{2.024842in}}%
\pgfpathcurveto{\pgfqpoint{2.008400in}{2.024842in}}{\pgfqpoint{2.000500in}{2.021570in}}{\pgfqpoint{1.994676in}{2.015746in}}%
\pgfpathcurveto{\pgfqpoint{1.988852in}{2.009922in}}{\pgfqpoint{1.985580in}{2.002022in}}{\pgfqpoint{1.985580in}{1.993786in}}%
\pgfpathcurveto{\pgfqpoint{1.985580in}{1.985549in}}{\pgfqpoint{1.988852in}{1.977649in}}{\pgfqpoint{1.994676in}{1.971825in}}%
\pgfpathcurveto{\pgfqpoint{2.000500in}{1.966001in}}{\pgfqpoint{2.008400in}{1.962729in}}{\pgfqpoint{2.016637in}{1.962729in}}%
\pgfpathclose%
\pgfusepath{stroke,fill}%
\end{pgfscope}%
\begin{pgfscope}%
\pgfpathrectangle{\pgfqpoint{0.100000in}{0.212622in}}{\pgfqpoint{3.696000in}{3.696000in}}%
\pgfusepath{clip}%
\pgfsetbuttcap%
\pgfsetroundjoin%
\definecolor{currentfill}{rgb}{0.121569,0.466667,0.705882}%
\pgfsetfillcolor{currentfill}%
\pgfsetfillopacity{0.434719}%
\pgfsetlinewidth{1.003750pt}%
\definecolor{currentstroke}{rgb}{0.121569,0.466667,0.705882}%
\pgfsetstrokecolor{currentstroke}%
\pgfsetstrokeopacity{0.434719}%
\pgfsetdash{}{0pt}%
\pgfpathmoveto{\pgfqpoint{1.496965in}{1.786724in}}%
\pgfpathcurveto{\pgfqpoint{1.505201in}{1.786724in}}{\pgfqpoint{1.513101in}{1.789997in}}{\pgfqpoint{1.518925in}{1.795821in}}%
\pgfpathcurveto{\pgfqpoint{1.524749in}{1.801645in}}{\pgfqpoint{1.528022in}{1.809545in}}{\pgfqpoint{1.528022in}{1.817781in}}%
\pgfpathcurveto{\pgfqpoint{1.528022in}{1.826017in}}{\pgfqpoint{1.524749in}{1.833917in}}{\pgfqpoint{1.518925in}{1.839741in}}%
\pgfpathcurveto{\pgfqpoint{1.513101in}{1.845565in}}{\pgfqpoint{1.505201in}{1.848837in}}{\pgfqpoint{1.496965in}{1.848837in}}%
\pgfpathcurveto{\pgfqpoint{1.488729in}{1.848837in}}{\pgfqpoint{1.480829in}{1.845565in}}{\pgfqpoint{1.475005in}{1.839741in}}%
\pgfpathcurveto{\pgfqpoint{1.469181in}{1.833917in}}{\pgfqpoint{1.465909in}{1.826017in}}{\pgfqpoint{1.465909in}{1.817781in}}%
\pgfpathcurveto{\pgfqpoint{1.465909in}{1.809545in}}{\pgfqpoint{1.469181in}{1.801645in}}{\pgfqpoint{1.475005in}{1.795821in}}%
\pgfpathcurveto{\pgfqpoint{1.480829in}{1.789997in}}{\pgfqpoint{1.488729in}{1.786724in}}{\pgfqpoint{1.496965in}{1.786724in}}%
\pgfpathclose%
\pgfusepath{stroke,fill}%
\end{pgfscope}%
\begin{pgfscope}%
\pgfpathrectangle{\pgfqpoint{0.100000in}{0.212622in}}{\pgfqpoint{3.696000in}{3.696000in}}%
\pgfusepath{clip}%
\pgfsetbuttcap%
\pgfsetroundjoin%
\definecolor{currentfill}{rgb}{0.121569,0.466667,0.705882}%
\pgfsetfillcolor{currentfill}%
\pgfsetfillopacity{0.435565}%
\pgfsetlinewidth{1.003750pt}%
\definecolor{currentstroke}{rgb}{0.121569,0.466667,0.705882}%
\pgfsetstrokecolor{currentstroke}%
\pgfsetstrokeopacity{0.435565}%
\pgfsetdash{}{0pt}%
\pgfpathmoveto{\pgfqpoint{2.017818in}{1.960612in}}%
\pgfpathcurveto{\pgfqpoint{2.026054in}{1.960612in}}{\pgfqpoint{2.033954in}{1.963885in}}{\pgfqpoint{2.039778in}{1.969709in}}%
\pgfpathcurveto{\pgfqpoint{2.045602in}{1.975532in}}{\pgfqpoint{2.048875in}{1.983433in}}{\pgfqpoint{2.048875in}{1.991669in}}%
\pgfpathcurveto{\pgfqpoint{2.048875in}{1.999905in}}{\pgfqpoint{2.045602in}{2.007805in}}{\pgfqpoint{2.039778in}{2.013629in}}%
\pgfpathcurveto{\pgfqpoint{2.033954in}{2.019453in}}{\pgfqpoint{2.026054in}{2.022725in}}{\pgfqpoint{2.017818in}{2.022725in}}%
\pgfpathcurveto{\pgfqpoint{2.009582in}{2.022725in}}{\pgfqpoint{2.001682in}{2.019453in}}{\pgfqpoint{1.995858in}{2.013629in}}%
\pgfpathcurveto{\pgfqpoint{1.990034in}{2.007805in}}{\pgfqpoint{1.986762in}{1.999905in}}{\pgfqpoint{1.986762in}{1.991669in}}%
\pgfpathcurveto{\pgfqpoint{1.986762in}{1.983433in}}{\pgfqpoint{1.990034in}{1.975532in}}{\pgfqpoint{1.995858in}{1.969709in}}%
\pgfpathcurveto{\pgfqpoint{2.001682in}{1.963885in}}{\pgfqpoint{2.009582in}{1.960612in}}{\pgfqpoint{2.017818in}{1.960612in}}%
\pgfpathclose%
\pgfusepath{stroke,fill}%
\end{pgfscope}%
\begin{pgfscope}%
\pgfpathrectangle{\pgfqpoint{0.100000in}{0.212622in}}{\pgfqpoint{3.696000in}{3.696000in}}%
\pgfusepath{clip}%
\pgfsetbuttcap%
\pgfsetroundjoin%
\definecolor{currentfill}{rgb}{0.121569,0.466667,0.705882}%
\pgfsetfillcolor{currentfill}%
\pgfsetfillopacity{0.435641}%
\pgfsetlinewidth{1.003750pt}%
\definecolor{currentstroke}{rgb}{0.121569,0.466667,0.705882}%
\pgfsetstrokecolor{currentstroke}%
\pgfsetstrokeopacity{0.435641}%
\pgfsetdash{}{0pt}%
\pgfpathmoveto{\pgfqpoint{1.493830in}{1.783780in}}%
\pgfpathcurveto{\pgfqpoint{1.502066in}{1.783780in}}{\pgfqpoint{1.509966in}{1.787052in}}{\pgfqpoint{1.515790in}{1.792876in}}%
\pgfpathcurveto{\pgfqpoint{1.521614in}{1.798700in}}{\pgfqpoint{1.524887in}{1.806600in}}{\pgfqpoint{1.524887in}{1.814836in}}%
\pgfpathcurveto{\pgfqpoint{1.524887in}{1.823073in}}{\pgfqpoint{1.521614in}{1.830973in}}{\pgfqpoint{1.515790in}{1.836797in}}%
\pgfpathcurveto{\pgfqpoint{1.509966in}{1.842621in}}{\pgfqpoint{1.502066in}{1.845893in}}{\pgfqpoint{1.493830in}{1.845893in}}%
\pgfpathcurveto{\pgfqpoint{1.485594in}{1.845893in}}{\pgfqpoint{1.477694in}{1.842621in}}{\pgfqpoint{1.471870in}{1.836797in}}%
\pgfpathcurveto{\pgfqpoint{1.466046in}{1.830973in}}{\pgfqpoint{1.462774in}{1.823073in}}{\pgfqpoint{1.462774in}{1.814836in}}%
\pgfpathcurveto{\pgfqpoint{1.462774in}{1.806600in}}{\pgfqpoint{1.466046in}{1.798700in}}{\pgfqpoint{1.471870in}{1.792876in}}%
\pgfpathcurveto{\pgfqpoint{1.477694in}{1.787052in}}{\pgfqpoint{1.485594in}{1.783780in}}{\pgfqpoint{1.493830in}{1.783780in}}%
\pgfpathclose%
\pgfusepath{stroke,fill}%
\end{pgfscope}%
\begin{pgfscope}%
\pgfpathrectangle{\pgfqpoint{0.100000in}{0.212622in}}{\pgfqpoint{3.696000in}{3.696000in}}%
\pgfusepath{clip}%
\pgfsetbuttcap%
\pgfsetroundjoin%
\definecolor{currentfill}{rgb}{0.121569,0.466667,0.705882}%
\pgfsetfillcolor{currentfill}%
\pgfsetfillopacity{0.436447}%
\pgfsetlinewidth{1.003750pt}%
\definecolor{currentstroke}{rgb}{0.121569,0.466667,0.705882}%
\pgfsetstrokecolor{currentstroke}%
\pgfsetstrokeopacity{0.436447}%
\pgfsetdash{}{0pt}%
\pgfpathmoveto{\pgfqpoint{1.491698in}{1.783145in}}%
\pgfpathcurveto{\pgfqpoint{1.499934in}{1.783145in}}{\pgfqpoint{1.507834in}{1.786417in}}{\pgfqpoint{1.513658in}{1.792241in}}%
\pgfpathcurveto{\pgfqpoint{1.519482in}{1.798065in}}{\pgfqpoint{1.522754in}{1.805965in}}{\pgfqpoint{1.522754in}{1.814201in}}%
\pgfpathcurveto{\pgfqpoint{1.522754in}{1.822438in}}{\pgfqpoint{1.519482in}{1.830338in}}{\pgfqpoint{1.513658in}{1.836162in}}%
\pgfpathcurveto{\pgfqpoint{1.507834in}{1.841986in}}{\pgfqpoint{1.499934in}{1.845258in}}{\pgfqpoint{1.491698in}{1.845258in}}%
\pgfpathcurveto{\pgfqpoint{1.483461in}{1.845258in}}{\pgfqpoint{1.475561in}{1.841986in}}{\pgfqpoint{1.469737in}{1.836162in}}%
\pgfpathcurveto{\pgfqpoint{1.463914in}{1.830338in}}{\pgfqpoint{1.460641in}{1.822438in}}{\pgfqpoint{1.460641in}{1.814201in}}%
\pgfpathcurveto{\pgfqpoint{1.460641in}{1.805965in}}{\pgfqpoint{1.463914in}{1.798065in}}{\pgfqpoint{1.469737in}{1.792241in}}%
\pgfpathcurveto{\pgfqpoint{1.475561in}{1.786417in}}{\pgfqpoint{1.483461in}{1.783145in}}{\pgfqpoint{1.491698in}{1.783145in}}%
\pgfpathclose%
\pgfusepath{stroke,fill}%
\end{pgfscope}%
\begin{pgfscope}%
\pgfpathrectangle{\pgfqpoint{0.100000in}{0.212622in}}{\pgfqpoint{3.696000in}{3.696000in}}%
\pgfusepath{clip}%
\pgfsetbuttcap%
\pgfsetroundjoin%
\definecolor{currentfill}{rgb}{0.121569,0.466667,0.705882}%
\pgfsetfillcolor{currentfill}%
\pgfsetfillopacity{0.436934}%
\pgfsetlinewidth{1.003750pt}%
\definecolor{currentstroke}{rgb}{0.121569,0.466667,0.705882}%
\pgfsetstrokecolor{currentstroke}%
\pgfsetstrokeopacity{0.436934}%
\pgfsetdash{}{0pt}%
\pgfpathmoveto{\pgfqpoint{1.490174in}{1.782270in}}%
\pgfpathcurveto{\pgfqpoint{1.498411in}{1.782270in}}{\pgfqpoint{1.506311in}{1.785542in}}{\pgfqpoint{1.512134in}{1.791366in}}%
\pgfpathcurveto{\pgfqpoint{1.517958in}{1.797190in}}{\pgfqpoint{1.521231in}{1.805090in}}{\pgfqpoint{1.521231in}{1.813326in}}%
\pgfpathcurveto{\pgfqpoint{1.521231in}{1.821562in}}{\pgfqpoint{1.517958in}{1.829462in}}{\pgfqpoint{1.512134in}{1.835286in}}%
\pgfpathcurveto{\pgfqpoint{1.506311in}{1.841110in}}{\pgfqpoint{1.498411in}{1.844383in}}{\pgfqpoint{1.490174in}{1.844383in}}%
\pgfpathcurveto{\pgfqpoint{1.481938in}{1.844383in}}{\pgfqpoint{1.474038in}{1.841110in}}{\pgfqpoint{1.468214in}{1.835286in}}%
\pgfpathcurveto{\pgfqpoint{1.462390in}{1.829462in}}{\pgfqpoint{1.459118in}{1.821562in}}{\pgfqpoint{1.459118in}{1.813326in}}%
\pgfpathcurveto{\pgfqpoint{1.459118in}{1.805090in}}{\pgfqpoint{1.462390in}{1.797190in}}{\pgfqpoint{1.468214in}{1.791366in}}%
\pgfpathcurveto{\pgfqpoint{1.474038in}{1.785542in}}{\pgfqpoint{1.481938in}{1.782270in}}{\pgfqpoint{1.490174in}{1.782270in}}%
\pgfpathclose%
\pgfusepath{stroke,fill}%
\end{pgfscope}%
\begin{pgfscope}%
\pgfpathrectangle{\pgfqpoint{0.100000in}{0.212622in}}{\pgfqpoint{3.696000in}{3.696000in}}%
\pgfusepath{clip}%
\pgfsetbuttcap%
\pgfsetroundjoin%
\definecolor{currentfill}{rgb}{0.121569,0.466667,0.705882}%
\pgfsetfillcolor{currentfill}%
\pgfsetfillopacity{0.437718}%
\pgfsetlinewidth{1.003750pt}%
\definecolor{currentstroke}{rgb}{0.121569,0.466667,0.705882}%
\pgfsetstrokecolor{currentstroke}%
\pgfsetstrokeopacity{0.437718}%
\pgfsetdash{}{0pt}%
\pgfpathmoveto{\pgfqpoint{1.487592in}{1.779756in}}%
\pgfpathcurveto{\pgfqpoint{1.495829in}{1.779756in}}{\pgfqpoint{1.503729in}{1.783028in}}{\pgfqpoint{1.509553in}{1.788852in}}%
\pgfpathcurveto{\pgfqpoint{1.515377in}{1.794676in}}{\pgfqpoint{1.518649in}{1.802576in}}{\pgfqpoint{1.518649in}{1.810812in}}%
\pgfpathcurveto{\pgfqpoint{1.518649in}{1.819048in}}{\pgfqpoint{1.515377in}{1.826949in}}{\pgfqpoint{1.509553in}{1.832772in}}%
\pgfpathcurveto{\pgfqpoint{1.503729in}{1.838596in}}{\pgfqpoint{1.495829in}{1.841869in}}{\pgfqpoint{1.487592in}{1.841869in}}%
\pgfpathcurveto{\pgfqpoint{1.479356in}{1.841869in}}{\pgfqpoint{1.471456in}{1.838596in}}{\pgfqpoint{1.465632in}{1.832772in}}%
\pgfpathcurveto{\pgfqpoint{1.459808in}{1.826949in}}{\pgfqpoint{1.456536in}{1.819048in}}{\pgfqpoint{1.456536in}{1.810812in}}%
\pgfpathcurveto{\pgfqpoint{1.456536in}{1.802576in}}{\pgfqpoint{1.459808in}{1.794676in}}{\pgfqpoint{1.465632in}{1.788852in}}%
\pgfpathcurveto{\pgfqpoint{1.471456in}{1.783028in}}{\pgfqpoint{1.479356in}{1.779756in}}{\pgfqpoint{1.487592in}{1.779756in}}%
\pgfpathclose%
\pgfusepath{stroke,fill}%
\end{pgfscope}%
\begin{pgfscope}%
\pgfpathrectangle{\pgfqpoint{0.100000in}{0.212622in}}{\pgfqpoint{3.696000in}{3.696000in}}%
\pgfusepath{clip}%
\pgfsetbuttcap%
\pgfsetroundjoin%
\definecolor{currentfill}{rgb}{0.121569,0.466667,0.705882}%
\pgfsetfillcolor{currentfill}%
\pgfsetfillopacity{0.438061}%
\pgfsetlinewidth{1.003750pt}%
\definecolor{currentstroke}{rgb}{0.121569,0.466667,0.705882}%
\pgfsetstrokecolor{currentstroke}%
\pgfsetstrokeopacity{0.438061}%
\pgfsetdash{}{0pt}%
\pgfpathmoveto{\pgfqpoint{2.018621in}{1.960592in}}%
\pgfpathcurveto{\pgfqpoint{2.026857in}{1.960592in}}{\pgfqpoint{2.034757in}{1.963864in}}{\pgfqpoint{2.040581in}{1.969688in}}%
\pgfpathcurveto{\pgfqpoint{2.046405in}{1.975512in}}{\pgfqpoint{2.049678in}{1.983412in}}{\pgfqpoint{2.049678in}{1.991648in}}%
\pgfpathcurveto{\pgfqpoint{2.049678in}{1.999885in}}{\pgfqpoint{2.046405in}{2.007785in}}{\pgfqpoint{2.040581in}{2.013609in}}%
\pgfpathcurveto{\pgfqpoint{2.034757in}{2.019433in}}{\pgfqpoint{2.026857in}{2.022705in}}{\pgfqpoint{2.018621in}{2.022705in}}%
\pgfpathcurveto{\pgfqpoint{2.010385in}{2.022705in}}{\pgfqpoint{2.002485in}{2.019433in}}{\pgfqpoint{1.996661in}{2.013609in}}%
\pgfpathcurveto{\pgfqpoint{1.990837in}{2.007785in}}{\pgfqpoint{1.987565in}{1.999885in}}{\pgfqpoint{1.987565in}{1.991648in}}%
\pgfpathcurveto{\pgfqpoint{1.987565in}{1.983412in}}{\pgfqpoint{1.990837in}{1.975512in}}{\pgfqpoint{1.996661in}{1.969688in}}%
\pgfpathcurveto{\pgfqpoint{2.002485in}{1.963864in}}{\pgfqpoint{2.010385in}{1.960592in}}{\pgfqpoint{2.018621in}{1.960592in}}%
\pgfpathclose%
\pgfusepath{stroke,fill}%
\end{pgfscope}%
\begin{pgfscope}%
\pgfpathrectangle{\pgfqpoint{0.100000in}{0.212622in}}{\pgfqpoint{3.696000in}{3.696000in}}%
\pgfusepath{clip}%
\pgfsetbuttcap%
\pgfsetroundjoin%
\definecolor{currentfill}{rgb}{0.121569,0.466667,0.705882}%
\pgfsetfillcolor{currentfill}%
\pgfsetfillopacity{0.439787}%
\pgfsetlinewidth{1.003750pt}%
\definecolor{currentstroke}{rgb}{0.121569,0.466667,0.705882}%
\pgfsetstrokecolor{currentstroke}%
\pgfsetstrokeopacity{0.439787}%
\pgfsetdash{}{0pt}%
\pgfpathmoveto{\pgfqpoint{1.482328in}{1.780217in}}%
\pgfpathcurveto{\pgfqpoint{1.490565in}{1.780217in}}{\pgfqpoint{1.498465in}{1.783489in}}{\pgfqpoint{1.504289in}{1.789313in}}%
\pgfpathcurveto{\pgfqpoint{1.510113in}{1.795137in}}{\pgfqpoint{1.513385in}{1.803037in}}{\pgfqpoint{1.513385in}{1.811273in}}%
\pgfpathcurveto{\pgfqpoint{1.513385in}{1.819509in}}{\pgfqpoint{1.510113in}{1.827409in}}{\pgfqpoint{1.504289in}{1.833233in}}%
\pgfpathcurveto{\pgfqpoint{1.498465in}{1.839057in}}{\pgfqpoint{1.490565in}{1.842330in}}{\pgfqpoint{1.482328in}{1.842330in}}%
\pgfpathcurveto{\pgfqpoint{1.474092in}{1.842330in}}{\pgfqpoint{1.466192in}{1.839057in}}{\pgfqpoint{1.460368in}{1.833233in}}%
\pgfpathcurveto{\pgfqpoint{1.454544in}{1.827409in}}{\pgfqpoint{1.451272in}{1.819509in}}{\pgfqpoint{1.451272in}{1.811273in}}%
\pgfpathcurveto{\pgfqpoint{1.451272in}{1.803037in}}{\pgfqpoint{1.454544in}{1.795137in}}{\pgfqpoint{1.460368in}{1.789313in}}%
\pgfpathcurveto{\pgfqpoint{1.466192in}{1.783489in}}{\pgfqpoint{1.474092in}{1.780217in}}{\pgfqpoint{1.482328in}{1.780217in}}%
\pgfpathclose%
\pgfusepath{stroke,fill}%
\end{pgfscope}%
\begin{pgfscope}%
\pgfpathrectangle{\pgfqpoint{0.100000in}{0.212622in}}{\pgfqpoint{3.696000in}{3.696000in}}%
\pgfusepath{clip}%
\pgfsetbuttcap%
\pgfsetroundjoin%
\definecolor{currentfill}{rgb}{0.121569,0.466667,0.705882}%
\pgfsetfillcolor{currentfill}%
\pgfsetfillopacity{0.440774}%
\pgfsetlinewidth{1.003750pt}%
\definecolor{currentstroke}{rgb}{0.121569,0.466667,0.705882}%
\pgfsetstrokecolor{currentstroke}%
\pgfsetstrokeopacity{0.440774}%
\pgfsetdash{}{0pt}%
\pgfpathmoveto{\pgfqpoint{2.019409in}{1.958447in}}%
\pgfpathcurveto{\pgfqpoint{2.027645in}{1.958447in}}{\pgfqpoint{2.035546in}{1.961719in}}{\pgfqpoint{2.041369in}{1.967543in}}%
\pgfpathcurveto{\pgfqpoint{2.047193in}{1.973367in}}{\pgfqpoint{2.050466in}{1.981267in}}{\pgfqpoint{2.050466in}{1.989503in}}%
\pgfpathcurveto{\pgfqpoint{2.050466in}{1.997739in}}{\pgfqpoint{2.047193in}{2.005639in}}{\pgfqpoint{2.041369in}{2.011463in}}%
\pgfpathcurveto{\pgfqpoint{2.035546in}{2.017287in}}{\pgfqpoint{2.027645in}{2.020560in}}{\pgfqpoint{2.019409in}{2.020560in}}%
\pgfpathcurveto{\pgfqpoint{2.011173in}{2.020560in}}{\pgfqpoint{2.003273in}{2.017287in}}{\pgfqpoint{1.997449in}{2.011463in}}%
\pgfpathcurveto{\pgfqpoint{1.991625in}{2.005639in}}{\pgfqpoint{1.988353in}{1.997739in}}{\pgfqpoint{1.988353in}{1.989503in}}%
\pgfpathcurveto{\pgfqpoint{1.988353in}{1.981267in}}{\pgfqpoint{1.991625in}{1.973367in}}{\pgfqpoint{1.997449in}{1.967543in}}%
\pgfpathcurveto{\pgfqpoint{2.003273in}{1.961719in}}{\pgfqpoint{2.011173in}{1.958447in}}{\pgfqpoint{2.019409in}{1.958447in}}%
\pgfpathclose%
\pgfusepath{stroke,fill}%
\end{pgfscope}%
\begin{pgfscope}%
\pgfpathrectangle{\pgfqpoint{0.100000in}{0.212622in}}{\pgfqpoint{3.696000in}{3.696000in}}%
\pgfusepath{clip}%
\pgfsetbuttcap%
\pgfsetroundjoin%
\definecolor{currentfill}{rgb}{0.121569,0.466667,0.705882}%
\pgfsetfillcolor{currentfill}%
\pgfsetfillopacity{0.440930}%
\pgfsetlinewidth{1.003750pt}%
\definecolor{currentstroke}{rgb}{0.121569,0.466667,0.705882}%
\pgfsetstrokecolor{currentstroke}%
\pgfsetstrokeopacity{0.440930}%
\pgfsetdash{}{0pt}%
\pgfpathmoveto{\pgfqpoint{1.478152in}{1.776462in}}%
\pgfpathcurveto{\pgfqpoint{1.486388in}{1.776462in}}{\pgfqpoint{1.494288in}{1.779734in}}{\pgfqpoint{1.500112in}{1.785558in}}%
\pgfpathcurveto{\pgfqpoint{1.505936in}{1.791382in}}{\pgfqpoint{1.509209in}{1.799282in}}{\pgfqpoint{1.509209in}{1.807518in}}%
\pgfpathcurveto{\pgfqpoint{1.509209in}{1.815755in}}{\pgfqpoint{1.505936in}{1.823655in}}{\pgfqpoint{1.500112in}{1.829479in}}%
\pgfpathcurveto{\pgfqpoint{1.494288in}{1.835303in}}{\pgfqpoint{1.486388in}{1.838575in}}{\pgfqpoint{1.478152in}{1.838575in}}%
\pgfpathcurveto{\pgfqpoint{1.469916in}{1.838575in}}{\pgfqpoint{1.462016in}{1.835303in}}{\pgfqpoint{1.456192in}{1.829479in}}%
\pgfpathcurveto{\pgfqpoint{1.450368in}{1.823655in}}{\pgfqpoint{1.447096in}{1.815755in}}{\pgfqpoint{1.447096in}{1.807518in}}%
\pgfpathcurveto{\pgfqpoint{1.447096in}{1.799282in}}{\pgfqpoint{1.450368in}{1.791382in}}{\pgfqpoint{1.456192in}{1.785558in}}%
\pgfpathcurveto{\pgfqpoint{1.462016in}{1.779734in}}{\pgfqpoint{1.469916in}{1.776462in}}{\pgfqpoint{1.478152in}{1.776462in}}%
\pgfpathclose%
\pgfusepath{stroke,fill}%
\end{pgfscope}%
\begin{pgfscope}%
\pgfpathrectangle{\pgfqpoint{0.100000in}{0.212622in}}{\pgfqpoint{3.696000in}{3.696000in}}%
\pgfusepath{clip}%
\pgfsetbuttcap%
\pgfsetroundjoin%
\definecolor{currentfill}{rgb}{0.121569,0.466667,0.705882}%
\pgfsetfillcolor{currentfill}%
\pgfsetfillopacity{0.442231}%
\pgfsetlinewidth{1.003750pt}%
\definecolor{currentstroke}{rgb}{0.121569,0.466667,0.705882}%
\pgfsetstrokecolor{currentstroke}%
\pgfsetstrokeopacity{0.442231}%
\pgfsetdash{}{0pt}%
\pgfpathmoveto{\pgfqpoint{2.019682in}{1.956975in}}%
\pgfpathcurveto{\pgfqpoint{2.027918in}{1.956975in}}{\pgfqpoint{2.035818in}{1.960247in}}{\pgfqpoint{2.041642in}{1.966071in}}%
\pgfpathcurveto{\pgfqpoint{2.047466in}{1.971895in}}{\pgfqpoint{2.050738in}{1.979795in}}{\pgfqpoint{2.050738in}{1.988031in}}%
\pgfpathcurveto{\pgfqpoint{2.050738in}{1.996268in}}{\pgfqpoint{2.047466in}{2.004168in}}{\pgfqpoint{2.041642in}{2.009992in}}%
\pgfpathcurveto{\pgfqpoint{2.035818in}{2.015816in}}{\pgfqpoint{2.027918in}{2.019088in}}{\pgfqpoint{2.019682in}{2.019088in}}%
\pgfpathcurveto{\pgfqpoint{2.011446in}{2.019088in}}{\pgfqpoint{2.003546in}{2.015816in}}{\pgfqpoint{1.997722in}{2.009992in}}%
\pgfpathcurveto{\pgfqpoint{1.991898in}{2.004168in}}{\pgfqpoint{1.988625in}{1.996268in}}{\pgfqpoint{1.988625in}{1.988031in}}%
\pgfpathcurveto{\pgfqpoint{1.988625in}{1.979795in}}{\pgfqpoint{1.991898in}{1.971895in}}{\pgfqpoint{1.997722in}{1.966071in}}%
\pgfpathcurveto{\pgfqpoint{2.003546in}{1.960247in}}{\pgfqpoint{2.011446in}{1.956975in}}{\pgfqpoint{2.019682in}{1.956975in}}%
\pgfpathclose%
\pgfusepath{stroke,fill}%
\end{pgfscope}%
\begin{pgfscope}%
\pgfpathrectangle{\pgfqpoint{0.100000in}{0.212622in}}{\pgfqpoint{3.696000in}{3.696000in}}%
\pgfusepath{clip}%
\pgfsetbuttcap%
\pgfsetroundjoin%
\definecolor{currentfill}{rgb}{0.121569,0.466667,0.705882}%
\pgfsetfillcolor{currentfill}%
\pgfsetfillopacity{0.442602}%
\pgfsetlinewidth{1.003750pt}%
\definecolor{currentstroke}{rgb}{0.121569,0.466667,0.705882}%
\pgfsetstrokecolor{currentstroke}%
\pgfsetstrokeopacity{0.442602}%
\pgfsetdash{}{0pt}%
\pgfpathmoveto{\pgfqpoint{1.474387in}{1.778324in}}%
\pgfpathcurveto{\pgfqpoint{1.482624in}{1.778324in}}{\pgfqpoint{1.490524in}{1.781597in}}{\pgfqpoint{1.496348in}{1.787421in}}%
\pgfpathcurveto{\pgfqpoint{1.502172in}{1.793245in}}{\pgfqpoint{1.505444in}{1.801145in}}{\pgfqpoint{1.505444in}{1.809381in}}%
\pgfpathcurveto{\pgfqpoint{1.505444in}{1.817617in}}{\pgfqpoint{1.502172in}{1.825517in}}{\pgfqpoint{1.496348in}{1.831341in}}%
\pgfpathcurveto{\pgfqpoint{1.490524in}{1.837165in}}{\pgfqpoint{1.482624in}{1.840437in}}{\pgfqpoint{1.474387in}{1.840437in}}%
\pgfpathcurveto{\pgfqpoint{1.466151in}{1.840437in}}{\pgfqpoint{1.458251in}{1.837165in}}{\pgfqpoint{1.452427in}{1.831341in}}%
\pgfpathcurveto{\pgfqpoint{1.446603in}{1.825517in}}{\pgfqpoint{1.443331in}{1.817617in}}{\pgfqpoint{1.443331in}{1.809381in}}%
\pgfpathcurveto{\pgfqpoint{1.443331in}{1.801145in}}{\pgfqpoint{1.446603in}{1.793245in}}{\pgfqpoint{1.452427in}{1.787421in}}%
\pgfpathcurveto{\pgfqpoint{1.458251in}{1.781597in}}{\pgfqpoint{1.466151in}{1.778324in}}{\pgfqpoint{1.474387in}{1.778324in}}%
\pgfpathclose%
\pgfusepath{stroke,fill}%
\end{pgfscope}%
\begin{pgfscope}%
\pgfpathrectangle{\pgfqpoint{0.100000in}{0.212622in}}{\pgfqpoint{3.696000in}{3.696000in}}%
\pgfusepath{clip}%
\pgfsetbuttcap%
\pgfsetroundjoin%
\definecolor{currentfill}{rgb}{0.121569,0.466667,0.705882}%
\pgfsetfillcolor{currentfill}%
\pgfsetfillopacity{0.444332}%
\pgfsetlinewidth{1.003750pt}%
\definecolor{currentstroke}{rgb}{0.121569,0.466667,0.705882}%
\pgfsetstrokecolor{currentstroke}%
\pgfsetstrokeopacity{0.444332}%
\pgfsetdash{}{0pt}%
\pgfpathmoveto{\pgfqpoint{2.020614in}{1.955425in}}%
\pgfpathcurveto{\pgfqpoint{2.028850in}{1.955425in}}{\pgfqpoint{2.036750in}{1.958698in}}{\pgfqpoint{2.042574in}{1.964521in}}%
\pgfpathcurveto{\pgfqpoint{2.048398in}{1.970345in}}{\pgfqpoint{2.051670in}{1.978245in}}{\pgfqpoint{2.051670in}{1.986482in}}%
\pgfpathcurveto{\pgfqpoint{2.051670in}{1.994718in}}{\pgfqpoint{2.048398in}{2.002618in}}{\pgfqpoint{2.042574in}{2.008442in}}%
\pgfpathcurveto{\pgfqpoint{2.036750in}{2.014266in}}{\pgfqpoint{2.028850in}{2.017538in}}{\pgfqpoint{2.020614in}{2.017538in}}%
\pgfpathcurveto{\pgfqpoint{2.012378in}{2.017538in}}{\pgfqpoint{2.004478in}{2.014266in}}{\pgfqpoint{1.998654in}{2.008442in}}%
\pgfpathcurveto{\pgfqpoint{1.992830in}{2.002618in}}{\pgfqpoint{1.989557in}{1.994718in}}{\pgfqpoint{1.989557in}{1.986482in}}%
\pgfpathcurveto{\pgfqpoint{1.989557in}{1.978245in}}{\pgfqpoint{1.992830in}{1.970345in}}{\pgfqpoint{1.998654in}{1.964521in}}%
\pgfpathcurveto{\pgfqpoint{2.004478in}{1.958698in}}{\pgfqpoint{2.012378in}{1.955425in}}{\pgfqpoint{2.020614in}{1.955425in}}%
\pgfpathclose%
\pgfusepath{stroke,fill}%
\end{pgfscope}%
\begin{pgfscope}%
\pgfpathrectangle{\pgfqpoint{0.100000in}{0.212622in}}{\pgfqpoint{3.696000in}{3.696000in}}%
\pgfusepath{clip}%
\pgfsetbuttcap%
\pgfsetroundjoin%
\definecolor{currentfill}{rgb}{0.121569,0.466667,0.705882}%
\pgfsetfillcolor{currentfill}%
\pgfsetfillopacity{0.445147}%
\pgfsetlinewidth{1.003750pt}%
\definecolor{currentstroke}{rgb}{0.121569,0.466667,0.705882}%
\pgfsetstrokecolor{currentstroke}%
\pgfsetstrokeopacity{0.445147}%
\pgfsetdash{}{0pt}%
\pgfpathmoveto{\pgfqpoint{1.467767in}{1.778098in}}%
\pgfpathcurveto{\pgfqpoint{1.476003in}{1.778098in}}{\pgfqpoint{1.483903in}{1.781371in}}{\pgfqpoint{1.489727in}{1.787195in}}%
\pgfpathcurveto{\pgfqpoint{1.495551in}{1.793018in}}{\pgfqpoint{1.498823in}{1.800918in}}{\pgfqpoint{1.498823in}{1.809155in}}%
\pgfpathcurveto{\pgfqpoint{1.498823in}{1.817391in}}{\pgfqpoint{1.495551in}{1.825291in}}{\pgfqpoint{1.489727in}{1.831115in}}%
\pgfpathcurveto{\pgfqpoint{1.483903in}{1.836939in}}{\pgfqpoint{1.476003in}{1.840211in}}{\pgfqpoint{1.467767in}{1.840211in}}%
\pgfpathcurveto{\pgfqpoint{1.459530in}{1.840211in}}{\pgfqpoint{1.451630in}{1.836939in}}{\pgfqpoint{1.445806in}{1.831115in}}%
\pgfpathcurveto{\pgfqpoint{1.439982in}{1.825291in}}{\pgfqpoint{1.436710in}{1.817391in}}{\pgfqpoint{1.436710in}{1.809155in}}%
\pgfpathcurveto{\pgfqpoint{1.436710in}{1.800918in}}{\pgfqpoint{1.439982in}{1.793018in}}{\pgfqpoint{1.445806in}{1.787195in}}%
\pgfpathcurveto{\pgfqpoint{1.451630in}{1.781371in}}{\pgfqpoint{1.459530in}{1.778098in}}{\pgfqpoint{1.467767in}{1.778098in}}%
\pgfpathclose%
\pgfusepath{stroke,fill}%
\end{pgfscope}%
\begin{pgfscope}%
\pgfpathrectangle{\pgfqpoint{0.100000in}{0.212622in}}{\pgfqpoint{3.696000in}{3.696000in}}%
\pgfusepath{clip}%
\pgfsetbuttcap%
\pgfsetroundjoin%
\definecolor{currentfill}{rgb}{0.121569,0.466667,0.705882}%
\pgfsetfillcolor{currentfill}%
\pgfsetfillopacity{0.445500}%
\pgfsetlinewidth{1.003750pt}%
\definecolor{currentstroke}{rgb}{0.121569,0.466667,0.705882}%
\pgfsetstrokecolor{currentstroke}%
\pgfsetstrokeopacity{0.445500}%
\pgfsetdash{}{0pt}%
\pgfpathmoveto{\pgfqpoint{2.021164in}{1.954664in}}%
\pgfpathcurveto{\pgfqpoint{2.029401in}{1.954664in}}{\pgfqpoint{2.037301in}{1.957936in}}{\pgfqpoint{2.043125in}{1.963760in}}%
\pgfpathcurveto{\pgfqpoint{2.048949in}{1.969584in}}{\pgfqpoint{2.052221in}{1.977484in}}{\pgfqpoint{2.052221in}{1.985720in}}%
\pgfpathcurveto{\pgfqpoint{2.052221in}{1.993957in}}{\pgfqpoint{2.048949in}{2.001857in}}{\pgfqpoint{2.043125in}{2.007681in}}%
\pgfpathcurveto{\pgfqpoint{2.037301in}{2.013505in}}{\pgfqpoint{2.029401in}{2.016777in}}{\pgfqpoint{2.021164in}{2.016777in}}%
\pgfpathcurveto{\pgfqpoint{2.012928in}{2.016777in}}{\pgfqpoint{2.005028in}{2.013505in}}{\pgfqpoint{1.999204in}{2.007681in}}%
\pgfpathcurveto{\pgfqpoint{1.993380in}{2.001857in}}{\pgfqpoint{1.990108in}{1.993957in}}{\pgfqpoint{1.990108in}{1.985720in}}%
\pgfpathcurveto{\pgfqpoint{1.990108in}{1.977484in}}{\pgfqpoint{1.993380in}{1.969584in}}{\pgfqpoint{1.999204in}{1.963760in}}%
\pgfpathcurveto{\pgfqpoint{2.005028in}{1.957936in}}{\pgfqpoint{2.012928in}{1.954664in}}{\pgfqpoint{2.021164in}{1.954664in}}%
\pgfpathclose%
\pgfusepath{stroke,fill}%
\end{pgfscope}%
\begin{pgfscope}%
\pgfpathrectangle{\pgfqpoint{0.100000in}{0.212622in}}{\pgfqpoint{3.696000in}{3.696000in}}%
\pgfusepath{clip}%
\pgfsetbuttcap%
\pgfsetroundjoin%
\definecolor{currentfill}{rgb}{0.121569,0.466667,0.705882}%
\pgfsetfillcolor{currentfill}%
\pgfsetfillopacity{0.446124}%
\pgfsetlinewidth{1.003750pt}%
\definecolor{currentstroke}{rgb}{0.121569,0.466667,0.705882}%
\pgfsetstrokecolor{currentstroke}%
\pgfsetstrokeopacity{0.446124}%
\pgfsetdash{}{0pt}%
\pgfpathmoveto{\pgfqpoint{2.021466in}{1.954122in}}%
\pgfpathcurveto{\pgfqpoint{2.029702in}{1.954122in}}{\pgfqpoint{2.037602in}{1.957395in}}{\pgfqpoint{2.043426in}{1.963219in}}%
\pgfpathcurveto{\pgfqpoint{2.049250in}{1.969042in}}{\pgfqpoint{2.052522in}{1.976943in}}{\pgfqpoint{2.052522in}{1.985179in}}%
\pgfpathcurveto{\pgfqpoint{2.052522in}{1.993415in}}{\pgfqpoint{2.049250in}{2.001315in}}{\pgfqpoint{2.043426in}{2.007139in}}%
\pgfpathcurveto{\pgfqpoint{2.037602in}{2.012963in}}{\pgfqpoint{2.029702in}{2.016235in}}{\pgfqpoint{2.021466in}{2.016235in}}%
\pgfpathcurveto{\pgfqpoint{2.013229in}{2.016235in}}{\pgfqpoint{2.005329in}{2.012963in}}{\pgfqpoint{1.999505in}{2.007139in}}%
\pgfpathcurveto{\pgfqpoint{1.993681in}{2.001315in}}{\pgfqpoint{1.990409in}{1.993415in}}{\pgfqpoint{1.990409in}{1.985179in}}%
\pgfpathcurveto{\pgfqpoint{1.990409in}{1.976943in}}{\pgfqpoint{1.993681in}{1.969042in}}{\pgfqpoint{1.999505in}{1.963219in}}%
\pgfpathcurveto{\pgfqpoint{2.005329in}{1.957395in}}{\pgfqpoint{2.013229in}{1.954122in}}{\pgfqpoint{2.021466in}{1.954122in}}%
\pgfpathclose%
\pgfusepath{stroke,fill}%
\end{pgfscope}%
\begin{pgfscope}%
\pgfpathrectangle{\pgfqpoint{0.100000in}{0.212622in}}{\pgfqpoint{3.696000in}{3.696000in}}%
\pgfusepath{clip}%
\pgfsetbuttcap%
\pgfsetroundjoin%
\definecolor{currentfill}{rgb}{0.121569,0.466667,0.705882}%
\pgfsetfillcolor{currentfill}%
\pgfsetfillopacity{0.446466}%
\pgfsetlinewidth{1.003750pt}%
\definecolor{currentstroke}{rgb}{0.121569,0.466667,0.705882}%
\pgfsetstrokecolor{currentstroke}%
\pgfsetstrokeopacity{0.446466}%
\pgfsetdash{}{0pt}%
\pgfpathmoveto{\pgfqpoint{2.021652in}{1.953822in}}%
\pgfpathcurveto{\pgfqpoint{2.029888in}{1.953822in}}{\pgfqpoint{2.037788in}{1.957094in}}{\pgfqpoint{2.043612in}{1.962918in}}%
\pgfpathcurveto{\pgfqpoint{2.049436in}{1.968742in}}{\pgfqpoint{2.052708in}{1.976642in}}{\pgfqpoint{2.052708in}{1.984878in}}%
\pgfpathcurveto{\pgfqpoint{2.052708in}{1.993115in}}{\pgfqpoint{2.049436in}{2.001015in}}{\pgfqpoint{2.043612in}{2.006839in}}%
\pgfpathcurveto{\pgfqpoint{2.037788in}{2.012663in}}{\pgfqpoint{2.029888in}{2.015935in}}{\pgfqpoint{2.021652in}{2.015935in}}%
\pgfpathcurveto{\pgfqpoint{2.013415in}{2.015935in}}{\pgfqpoint{2.005515in}{2.012663in}}{\pgfqpoint{1.999691in}{2.006839in}}%
\pgfpathcurveto{\pgfqpoint{1.993867in}{2.001015in}}{\pgfqpoint{1.990595in}{1.993115in}}{\pgfqpoint{1.990595in}{1.984878in}}%
\pgfpathcurveto{\pgfqpoint{1.990595in}{1.976642in}}{\pgfqpoint{1.993867in}{1.968742in}}{\pgfqpoint{1.999691in}{1.962918in}}%
\pgfpathcurveto{\pgfqpoint{2.005515in}{1.957094in}}{\pgfqpoint{2.013415in}{1.953822in}}{\pgfqpoint{2.021652in}{1.953822in}}%
\pgfpathclose%
\pgfusepath{stroke,fill}%
\end{pgfscope}%
\begin{pgfscope}%
\pgfpathrectangle{\pgfqpoint{0.100000in}{0.212622in}}{\pgfqpoint{3.696000in}{3.696000in}}%
\pgfusepath{clip}%
\pgfsetbuttcap%
\pgfsetroundjoin%
\definecolor{currentfill}{rgb}{0.121569,0.466667,0.705882}%
\pgfsetfillcolor{currentfill}%
\pgfsetfillopacity{0.446546}%
\pgfsetlinewidth{1.003750pt}%
\definecolor{currentstroke}{rgb}{0.121569,0.466667,0.705882}%
\pgfsetstrokecolor{currentstroke}%
\pgfsetstrokeopacity{0.446546}%
\pgfsetdash{}{0pt}%
\pgfpathmoveto{\pgfqpoint{1.463012in}{1.774130in}}%
\pgfpathcurveto{\pgfqpoint{1.471248in}{1.774130in}}{\pgfqpoint{1.479148in}{1.777402in}}{\pgfqpoint{1.484972in}{1.783226in}}%
\pgfpathcurveto{\pgfqpoint{1.490796in}{1.789050in}}{\pgfqpoint{1.494068in}{1.796950in}}{\pgfqpoint{1.494068in}{1.805186in}}%
\pgfpathcurveto{\pgfqpoint{1.494068in}{1.813422in}}{\pgfqpoint{1.490796in}{1.821323in}}{\pgfqpoint{1.484972in}{1.827146in}}%
\pgfpathcurveto{\pgfqpoint{1.479148in}{1.832970in}}{\pgfqpoint{1.471248in}{1.836243in}}{\pgfqpoint{1.463012in}{1.836243in}}%
\pgfpathcurveto{\pgfqpoint{1.454775in}{1.836243in}}{\pgfqpoint{1.446875in}{1.832970in}}{\pgfqpoint{1.441051in}{1.827146in}}%
\pgfpathcurveto{\pgfqpoint{1.435228in}{1.821323in}}{\pgfqpoint{1.431955in}{1.813422in}}{\pgfqpoint{1.431955in}{1.805186in}}%
\pgfpathcurveto{\pgfqpoint{1.431955in}{1.796950in}}{\pgfqpoint{1.435228in}{1.789050in}}{\pgfqpoint{1.441051in}{1.783226in}}%
\pgfpathcurveto{\pgfqpoint{1.446875in}{1.777402in}}{\pgfqpoint{1.454775in}{1.774130in}}{\pgfqpoint{1.463012in}{1.774130in}}%
\pgfpathclose%
\pgfusepath{stroke,fill}%
\end{pgfscope}%
\begin{pgfscope}%
\pgfpathrectangle{\pgfqpoint{0.100000in}{0.212622in}}{\pgfqpoint{3.696000in}{3.696000in}}%
\pgfusepath{clip}%
\pgfsetbuttcap%
\pgfsetroundjoin%
\definecolor{currentfill}{rgb}{0.121569,0.466667,0.705882}%
\pgfsetfillcolor{currentfill}%
\pgfsetfillopacity{0.446690}%
\pgfsetlinewidth{1.003750pt}%
\definecolor{currentstroke}{rgb}{0.121569,0.466667,0.705882}%
\pgfsetstrokecolor{currentstroke}%
\pgfsetstrokeopacity{0.446690}%
\pgfsetdash{}{0pt}%
\pgfpathmoveto{\pgfqpoint{2.021692in}{1.953882in}}%
\pgfpathcurveto{\pgfqpoint{2.029929in}{1.953882in}}{\pgfqpoint{2.037829in}{1.957155in}}{\pgfqpoint{2.043652in}{1.962979in}}%
\pgfpathcurveto{\pgfqpoint{2.049476in}{1.968802in}}{\pgfqpoint{2.052749in}{1.976703in}}{\pgfqpoint{2.052749in}{1.984939in}}%
\pgfpathcurveto{\pgfqpoint{2.052749in}{1.993175in}}{\pgfqpoint{2.049476in}{2.001075in}}{\pgfqpoint{2.043652in}{2.006899in}}%
\pgfpathcurveto{\pgfqpoint{2.037829in}{2.012723in}}{\pgfqpoint{2.029929in}{2.015995in}}{\pgfqpoint{2.021692in}{2.015995in}}%
\pgfpathcurveto{\pgfqpoint{2.013456in}{2.015995in}}{\pgfqpoint{2.005556in}{2.012723in}}{\pgfqpoint{1.999732in}{2.006899in}}%
\pgfpathcurveto{\pgfqpoint{1.993908in}{2.001075in}}{\pgfqpoint{1.990636in}{1.993175in}}{\pgfqpoint{1.990636in}{1.984939in}}%
\pgfpathcurveto{\pgfqpoint{1.990636in}{1.976703in}}{\pgfqpoint{1.993908in}{1.968802in}}{\pgfqpoint{1.999732in}{1.962979in}}%
\pgfpathcurveto{\pgfqpoint{2.005556in}{1.957155in}}{\pgfqpoint{2.013456in}{1.953882in}}{\pgfqpoint{2.021692in}{1.953882in}}%
\pgfpathclose%
\pgfusepath{stroke,fill}%
\end{pgfscope}%
\begin{pgfscope}%
\pgfpathrectangle{\pgfqpoint{0.100000in}{0.212622in}}{\pgfqpoint{3.696000in}{3.696000in}}%
\pgfusepath{clip}%
\pgfsetbuttcap%
\pgfsetroundjoin%
\definecolor{currentfill}{rgb}{0.121569,0.466667,0.705882}%
\pgfsetfillcolor{currentfill}%
\pgfsetfillopacity{0.447295}%
\pgfsetlinewidth{1.003750pt}%
\definecolor{currentstroke}{rgb}{0.121569,0.466667,0.705882}%
\pgfsetstrokecolor{currentstroke}%
\pgfsetstrokeopacity{0.447295}%
\pgfsetdash{}{0pt}%
\pgfpathmoveto{\pgfqpoint{2.022112in}{1.953348in}}%
\pgfpathcurveto{\pgfqpoint{2.030348in}{1.953348in}}{\pgfqpoint{2.038248in}{1.956620in}}{\pgfqpoint{2.044072in}{1.962444in}}%
\pgfpathcurveto{\pgfqpoint{2.049896in}{1.968268in}}{\pgfqpoint{2.053168in}{1.976168in}}{\pgfqpoint{2.053168in}{1.984405in}}%
\pgfpathcurveto{\pgfqpoint{2.053168in}{1.992641in}}{\pgfqpoint{2.049896in}{2.000541in}}{\pgfqpoint{2.044072in}{2.006365in}}%
\pgfpathcurveto{\pgfqpoint{2.038248in}{2.012189in}}{\pgfqpoint{2.030348in}{2.015461in}}{\pgfqpoint{2.022112in}{2.015461in}}%
\pgfpathcurveto{\pgfqpoint{2.013875in}{2.015461in}}{\pgfqpoint{2.005975in}{2.012189in}}{\pgfqpoint{2.000151in}{2.006365in}}%
\pgfpathcurveto{\pgfqpoint{1.994328in}{2.000541in}}{\pgfqpoint{1.991055in}{1.992641in}}{\pgfqpoint{1.991055in}{1.984405in}}%
\pgfpathcurveto{\pgfqpoint{1.991055in}{1.976168in}}{\pgfqpoint{1.994328in}{1.968268in}}{\pgfqpoint{2.000151in}{1.962444in}}%
\pgfpathcurveto{\pgfqpoint{2.005975in}{1.956620in}}{\pgfqpoint{2.013875in}{1.953348in}}{\pgfqpoint{2.022112in}{1.953348in}}%
\pgfpathclose%
\pgfusepath{stroke,fill}%
\end{pgfscope}%
\begin{pgfscope}%
\pgfpathrectangle{\pgfqpoint{0.100000in}{0.212622in}}{\pgfqpoint{3.696000in}{3.696000in}}%
\pgfusepath{clip}%
\pgfsetbuttcap%
\pgfsetroundjoin%
\definecolor{currentfill}{rgb}{0.121569,0.466667,0.705882}%
\pgfsetfillcolor{currentfill}%
\pgfsetfillopacity{0.447594}%
\pgfsetlinewidth{1.003750pt}%
\definecolor{currentstroke}{rgb}{0.121569,0.466667,0.705882}%
\pgfsetstrokecolor{currentstroke}%
\pgfsetstrokeopacity{0.447594}%
\pgfsetdash{}{0pt}%
\pgfpathmoveto{\pgfqpoint{2.022319in}{1.952813in}}%
\pgfpathcurveto{\pgfqpoint{2.030556in}{1.952813in}}{\pgfqpoint{2.038456in}{1.956085in}}{\pgfqpoint{2.044280in}{1.961909in}}%
\pgfpathcurveto{\pgfqpoint{2.050104in}{1.967733in}}{\pgfqpoint{2.053376in}{1.975633in}}{\pgfqpoint{2.053376in}{1.983869in}}%
\pgfpathcurveto{\pgfqpoint{2.053376in}{1.992106in}}{\pgfqpoint{2.050104in}{2.000006in}}{\pgfqpoint{2.044280in}{2.005830in}}%
\pgfpathcurveto{\pgfqpoint{2.038456in}{2.011654in}}{\pgfqpoint{2.030556in}{2.014926in}}{\pgfqpoint{2.022319in}{2.014926in}}%
\pgfpathcurveto{\pgfqpoint{2.014083in}{2.014926in}}{\pgfqpoint{2.006183in}{2.011654in}}{\pgfqpoint{2.000359in}{2.005830in}}%
\pgfpathcurveto{\pgfqpoint{1.994535in}{2.000006in}}{\pgfqpoint{1.991263in}{1.992106in}}{\pgfqpoint{1.991263in}{1.983869in}}%
\pgfpathcurveto{\pgfqpoint{1.991263in}{1.975633in}}{\pgfqpoint{1.994535in}{1.967733in}}{\pgfqpoint{2.000359in}{1.961909in}}%
\pgfpathcurveto{\pgfqpoint{2.006183in}{1.956085in}}{\pgfqpoint{2.014083in}{1.952813in}}{\pgfqpoint{2.022319in}{1.952813in}}%
\pgfpathclose%
\pgfusepath{stroke,fill}%
\end{pgfscope}%
\begin{pgfscope}%
\pgfpathrectangle{\pgfqpoint{0.100000in}{0.212622in}}{\pgfqpoint{3.696000in}{3.696000in}}%
\pgfusepath{clip}%
\pgfsetbuttcap%
\pgfsetroundjoin%
\definecolor{currentfill}{rgb}{0.121569,0.466667,0.705882}%
\pgfsetfillcolor{currentfill}%
\pgfsetfillopacity{0.448604}%
\pgfsetlinewidth{1.003750pt}%
\definecolor{currentstroke}{rgb}{0.121569,0.466667,0.705882}%
\pgfsetstrokecolor{currentstroke}%
\pgfsetstrokeopacity{0.448604}%
\pgfsetdash{}{0pt}%
\pgfpathmoveto{\pgfqpoint{2.022901in}{1.952320in}}%
\pgfpathcurveto{\pgfqpoint{2.031138in}{1.952320in}}{\pgfqpoint{2.039038in}{1.955593in}}{\pgfqpoint{2.044862in}{1.961416in}}%
\pgfpathcurveto{\pgfqpoint{2.050686in}{1.967240in}}{\pgfqpoint{2.053958in}{1.975140in}}{\pgfqpoint{2.053958in}{1.983377in}}%
\pgfpathcurveto{\pgfqpoint{2.053958in}{1.991613in}}{\pgfqpoint{2.050686in}{1.999513in}}{\pgfqpoint{2.044862in}{2.005337in}}%
\pgfpathcurveto{\pgfqpoint{2.039038in}{2.011161in}}{\pgfqpoint{2.031138in}{2.014433in}}{\pgfqpoint{2.022901in}{2.014433in}}%
\pgfpathcurveto{\pgfqpoint{2.014665in}{2.014433in}}{\pgfqpoint{2.006765in}{2.011161in}}{\pgfqpoint{2.000941in}{2.005337in}}%
\pgfpathcurveto{\pgfqpoint{1.995117in}{1.999513in}}{\pgfqpoint{1.991845in}{1.991613in}}{\pgfqpoint{1.991845in}{1.983377in}}%
\pgfpathcurveto{\pgfqpoint{1.991845in}{1.975140in}}{\pgfqpoint{1.995117in}{1.967240in}}{\pgfqpoint{2.000941in}{1.961416in}}%
\pgfpathcurveto{\pgfqpoint{2.006765in}{1.955593in}}{\pgfqpoint{2.014665in}{1.952320in}}{\pgfqpoint{2.022901in}{1.952320in}}%
\pgfpathclose%
\pgfusepath{stroke,fill}%
\end{pgfscope}%
\begin{pgfscope}%
\pgfpathrectangle{\pgfqpoint{0.100000in}{0.212622in}}{\pgfqpoint{3.696000in}{3.696000in}}%
\pgfusepath{clip}%
\pgfsetbuttcap%
\pgfsetroundjoin%
\definecolor{currentfill}{rgb}{0.121569,0.466667,0.705882}%
\pgfsetfillcolor{currentfill}%
\pgfsetfillopacity{0.449842}%
\pgfsetlinewidth{1.003750pt}%
\definecolor{currentstroke}{rgb}{0.121569,0.466667,0.705882}%
\pgfsetstrokecolor{currentstroke}%
\pgfsetstrokeopacity{0.449842}%
\pgfsetdash{}{0pt}%
\pgfpathmoveto{\pgfqpoint{2.023575in}{1.951909in}}%
\pgfpathcurveto{\pgfqpoint{2.031812in}{1.951909in}}{\pgfqpoint{2.039712in}{1.955181in}}{\pgfqpoint{2.045536in}{1.961005in}}%
\pgfpathcurveto{\pgfqpoint{2.051360in}{1.966829in}}{\pgfqpoint{2.054632in}{1.974729in}}{\pgfqpoint{2.054632in}{1.982965in}}%
\pgfpathcurveto{\pgfqpoint{2.054632in}{1.991202in}}{\pgfqpoint{2.051360in}{1.999102in}}{\pgfqpoint{2.045536in}{2.004926in}}%
\pgfpathcurveto{\pgfqpoint{2.039712in}{2.010749in}}{\pgfqpoint{2.031812in}{2.014022in}}{\pgfqpoint{2.023575in}{2.014022in}}%
\pgfpathcurveto{\pgfqpoint{2.015339in}{2.014022in}}{\pgfqpoint{2.007439in}{2.010749in}}{\pgfqpoint{2.001615in}{2.004926in}}%
\pgfpathcurveto{\pgfqpoint{1.995791in}{1.999102in}}{\pgfqpoint{1.992519in}{1.991202in}}{\pgfqpoint{1.992519in}{1.982965in}}%
\pgfpathcurveto{\pgfqpoint{1.992519in}{1.974729in}}{\pgfqpoint{1.995791in}{1.966829in}}{\pgfqpoint{2.001615in}{1.961005in}}%
\pgfpathcurveto{\pgfqpoint{2.007439in}{1.955181in}}{\pgfqpoint{2.015339in}{1.951909in}}{\pgfqpoint{2.023575in}{1.951909in}}%
\pgfpathclose%
\pgfusepath{stroke,fill}%
\end{pgfscope}%
\begin{pgfscope}%
\pgfpathrectangle{\pgfqpoint{0.100000in}{0.212622in}}{\pgfqpoint{3.696000in}{3.696000in}}%
\pgfusepath{clip}%
\pgfsetbuttcap%
\pgfsetroundjoin%
\definecolor{currentfill}{rgb}{0.121569,0.466667,0.705882}%
\pgfsetfillcolor{currentfill}%
\pgfsetfillopacity{0.449938}%
\pgfsetlinewidth{1.003750pt}%
\definecolor{currentstroke}{rgb}{0.121569,0.466667,0.705882}%
\pgfsetstrokecolor{currentstroke}%
\pgfsetstrokeopacity{0.449938}%
\pgfsetdash{}{0pt}%
\pgfpathmoveto{\pgfqpoint{1.460747in}{1.782388in}}%
\pgfpathcurveto{\pgfqpoint{1.468984in}{1.782388in}}{\pgfqpoint{1.476884in}{1.785660in}}{\pgfqpoint{1.482708in}{1.791484in}}%
\pgfpathcurveto{\pgfqpoint{1.488532in}{1.797308in}}{\pgfqpoint{1.491804in}{1.805208in}}{\pgfqpoint{1.491804in}{1.813444in}}%
\pgfpathcurveto{\pgfqpoint{1.491804in}{1.821680in}}{\pgfqpoint{1.488532in}{1.829580in}}{\pgfqpoint{1.482708in}{1.835404in}}%
\pgfpathcurveto{\pgfqpoint{1.476884in}{1.841228in}}{\pgfqpoint{1.468984in}{1.844501in}}{\pgfqpoint{1.460747in}{1.844501in}}%
\pgfpathcurveto{\pgfqpoint{1.452511in}{1.844501in}}{\pgfqpoint{1.444611in}{1.841228in}}{\pgfqpoint{1.438787in}{1.835404in}}%
\pgfpathcurveto{\pgfqpoint{1.432963in}{1.829580in}}{\pgfqpoint{1.429691in}{1.821680in}}{\pgfqpoint{1.429691in}{1.813444in}}%
\pgfpathcurveto{\pgfqpoint{1.429691in}{1.805208in}}{\pgfqpoint{1.432963in}{1.797308in}}{\pgfqpoint{1.438787in}{1.791484in}}%
\pgfpathcurveto{\pgfqpoint{1.444611in}{1.785660in}}{\pgfqpoint{1.452511in}{1.782388in}}{\pgfqpoint{1.460747in}{1.782388in}}%
\pgfpathclose%
\pgfusepath{stroke,fill}%
\end{pgfscope}%
\begin{pgfscope}%
\pgfpathrectangle{\pgfqpoint{0.100000in}{0.212622in}}{\pgfqpoint{3.696000in}{3.696000in}}%
\pgfusepath{clip}%
\pgfsetbuttcap%
\pgfsetroundjoin%
\definecolor{currentfill}{rgb}{0.121569,0.466667,0.705882}%
\pgfsetfillcolor{currentfill}%
\pgfsetfillopacity{0.451292}%
\pgfsetlinewidth{1.003750pt}%
\definecolor{currentstroke}{rgb}{0.121569,0.466667,0.705882}%
\pgfsetstrokecolor{currentstroke}%
\pgfsetstrokeopacity{0.451292}%
\pgfsetdash{}{0pt}%
\pgfpathmoveto{\pgfqpoint{2.024574in}{1.950798in}}%
\pgfpathcurveto{\pgfqpoint{2.032810in}{1.950798in}}{\pgfqpoint{2.040710in}{1.954070in}}{\pgfqpoint{2.046534in}{1.959894in}}%
\pgfpathcurveto{\pgfqpoint{2.052358in}{1.965718in}}{\pgfqpoint{2.055630in}{1.973618in}}{\pgfqpoint{2.055630in}{1.981855in}}%
\pgfpathcurveto{\pgfqpoint{2.055630in}{1.990091in}}{\pgfqpoint{2.052358in}{1.997991in}}{\pgfqpoint{2.046534in}{2.003815in}}%
\pgfpathcurveto{\pgfqpoint{2.040710in}{2.009639in}}{\pgfqpoint{2.032810in}{2.012911in}}{\pgfqpoint{2.024574in}{2.012911in}}%
\pgfpathcurveto{\pgfqpoint{2.016338in}{2.012911in}}{\pgfqpoint{2.008438in}{2.009639in}}{\pgfqpoint{2.002614in}{2.003815in}}%
\pgfpathcurveto{\pgfqpoint{1.996790in}{1.997991in}}{\pgfqpoint{1.993517in}{1.990091in}}{\pgfqpoint{1.993517in}{1.981855in}}%
\pgfpathcurveto{\pgfqpoint{1.993517in}{1.973618in}}{\pgfqpoint{1.996790in}{1.965718in}}{\pgfqpoint{2.002614in}{1.959894in}}%
\pgfpathcurveto{\pgfqpoint{2.008438in}{1.954070in}}{\pgfqpoint{2.016338in}{1.950798in}}{\pgfqpoint{2.024574in}{1.950798in}}%
\pgfpathclose%
\pgfusepath{stroke,fill}%
\end{pgfscope}%
\begin{pgfscope}%
\pgfpathrectangle{\pgfqpoint{0.100000in}{0.212622in}}{\pgfqpoint{3.696000in}{3.696000in}}%
\pgfusepath{clip}%
\pgfsetbuttcap%
\pgfsetroundjoin%
\definecolor{currentfill}{rgb}{0.121569,0.466667,0.705882}%
\pgfsetfillcolor{currentfill}%
\pgfsetfillopacity{0.452478}%
\pgfsetlinewidth{1.003750pt}%
\definecolor{currentstroke}{rgb}{0.121569,0.466667,0.705882}%
\pgfsetstrokecolor{currentstroke}%
\pgfsetstrokeopacity{0.452478}%
\pgfsetdash{}{0pt}%
\pgfpathmoveto{\pgfqpoint{1.452431in}{1.777179in}}%
\pgfpathcurveto{\pgfqpoint{1.460667in}{1.777179in}}{\pgfqpoint{1.468567in}{1.780451in}}{\pgfqpoint{1.474391in}{1.786275in}}%
\pgfpathcurveto{\pgfqpoint{1.480215in}{1.792099in}}{\pgfqpoint{1.483487in}{1.799999in}}{\pgfqpoint{1.483487in}{1.808236in}}%
\pgfpathcurveto{\pgfqpoint{1.483487in}{1.816472in}}{\pgfqpoint{1.480215in}{1.824372in}}{\pgfqpoint{1.474391in}{1.830196in}}%
\pgfpathcurveto{\pgfqpoint{1.468567in}{1.836020in}}{\pgfqpoint{1.460667in}{1.839292in}}{\pgfqpoint{1.452431in}{1.839292in}}%
\pgfpathcurveto{\pgfqpoint{1.444194in}{1.839292in}}{\pgfqpoint{1.436294in}{1.836020in}}{\pgfqpoint{1.430471in}{1.830196in}}%
\pgfpathcurveto{\pgfqpoint{1.424647in}{1.824372in}}{\pgfqpoint{1.421374in}{1.816472in}}{\pgfqpoint{1.421374in}{1.808236in}}%
\pgfpathcurveto{\pgfqpoint{1.421374in}{1.799999in}}{\pgfqpoint{1.424647in}{1.792099in}}{\pgfqpoint{1.430471in}{1.786275in}}%
\pgfpathcurveto{\pgfqpoint{1.436294in}{1.780451in}}{\pgfqpoint{1.444194in}{1.777179in}}{\pgfqpoint{1.452431in}{1.777179in}}%
\pgfpathclose%
\pgfusepath{stroke,fill}%
\end{pgfscope}%
\begin{pgfscope}%
\pgfpathrectangle{\pgfqpoint{0.100000in}{0.212622in}}{\pgfqpoint{3.696000in}{3.696000in}}%
\pgfusepath{clip}%
\pgfsetbuttcap%
\pgfsetroundjoin%
\definecolor{currentfill}{rgb}{0.121569,0.466667,0.705882}%
\pgfsetfillcolor{currentfill}%
\pgfsetfillopacity{0.453035}%
\pgfsetlinewidth{1.003750pt}%
\definecolor{currentstroke}{rgb}{0.121569,0.466667,0.705882}%
\pgfsetstrokecolor{currentstroke}%
\pgfsetstrokeopacity{0.453035}%
\pgfsetdash{}{0pt}%
\pgfpathmoveto{\pgfqpoint{2.025300in}{1.949006in}}%
\pgfpathcurveto{\pgfqpoint{2.033536in}{1.949006in}}{\pgfqpoint{2.041436in}{1.952278in}}{\pgfqpoint{2.047260in}{1.958102in}}%
\pgfpathcurveto{\pgfqpoint{2.053084in}{1.963926in}}{\pgfqpoint{2.056357in}{1.971826in}}{\pgfqpoint{2.056357in}{1.980062in}}%
\pgfpathcurveto{\pgfqpoint{2.056357in}{1.988298in}}{\pgfqpoint{2.053084in}{1.996199in}}{\pgfqpoint{2.047260in}{2.002022in}}%
\pgfpathcurveto{\pgfqpoint{2.041436in}{2.007846in}}{\pgfqpoint{2.033536in}{2.011119in}}{\pgfqpoint{2.025300in}{2.011119in}}%
\pgfpathcurveto{\pgfqpoint{2.017064in}{2.011119in}}{\pgfqpoint{2.009164in}{2.007846in}}{\pgfqpoint{2.003340in}{2.002022in}}%
\pgfpathcurveto{\pgfqpoint{1.997516in}{1.996199in}}{\pgfqpoint{1.994244in}{1.988298in}}{\pgfqpoint{1.994244in}{1.980062in}}%
\pgfpathcurveto{\pgfqpoint{1.994244in}{1.971826in}}{\pgfqpoint{1.997516in}{1.963926in}}{\pgfqpoint{2.003340in}{1.958102in}}%
\pgfpathcurveto{\pgfqpoint{2.009164in}{1.952278in}}{\pgfqpoint{2.017064in}{1.949006in}}{\pgfqpoint{2.025300in}{1.949006in}}%
\pgfpathclose%
\pgfusepath{stroke,fill}%
\end{pgfscope}%
\begin{pgfscope}%
\pgfpathrectangle{\pgfqpoint{0.100000in}{0.212622in}}{\pgfqpoint{3.696000in}{3.696000in}}%
\pgfusepath{clip}%
\pgfsetbuttcap%
\pgfsetroundjoin%
\definecolor{currentfill}{rgb}{0.121569,0.466667,0.705882}%
\pgfsetfillcolor{currentfill}%
\pgfsetfillopacity{0.454461}%
\pgfsetlinewidth{1.003750pt}%
\definecolor{currentstroke}{rgb}{0.121569,0.466667,0.705882}%
\pgfsetstrokecolor{currentstroke}%
\pgfsetstrokeopacity{0.454461}%
\pgfsetdash{}{0pt}%
\pgfpathmoveto{\pgfqpoint{1.445416in}{1.769570in}}%
\pgfpathcurveto{\pgfqpoint{1.453652in}{1.769570in}}{\pgfqpoint{1.461552in}{1.772843in}}{\pgfqpoint{1.467376in}{1.778667in}}%
\pgfpathcurveto{\pgfqpoint{1.473200in}{1.784491in}}{\pgfqpoint{1.476473in}{1.792391in}}{\pgfqpoint{1.476473in}{1.800627in}}%
\pgfpathcurveto{\pgfqpoint{1.476473in}{1.808863in}}{\pgfqpoint{1.473200in}{1.816763in}}{\pgfqpoint{1.467376in}{1.822587in}}%
\pgfpathcurveto{\pgfqpoint{1.461552in}{1.828411in}}{\pgfqpoint{1.453652in}{1.831683in}}{\pgfqpoint{1.445416in}{1.831683in}}%
\pgfpathcurveto{\pgfqpoint{1.437180in}{1.831683in}}{\pgfqpoint{1.429280in}{1.828411in}}{\pgfqpoint{1.423456in}{1.822587in}}%
\pgfpathcurveto{\pgfqpoint{1.417632in}{1.816763in}}{\pgfqpoint{1.414360in}{1.808863in}}{\pgfqpoint{1.414360in}{1.800627in}}%
\pgfpathcurveto{\pgfqpoint{1.414360in}{1.792391in}}{\pgfqpoint{1.417632in}{1.784491in}}{\pgfqpoint{1.423456in}{1.778667in}}%
\pgfpathcurveto{\pgfqpoint{1.429280in}{1.772843in}}{\pgfqpoint{1.437180in}{1.769570in}}{\pgfqpoint{1.445416in}{1.769570in}}%
\pgfpathclose%
\pgfusepath{stroke,fill}%
\end{pgfscope}%
\begin{pgfscope}%
\pgfpathrectangle{\pgfqpoint{0.100000in}{0.212622in}}{\pgfqpoint{3.696000in}{3.696000in}}%
\pgfusepath{clip}%
\pgfsetbuttcap%
\pgfsetroundjoin%
\definecolor{currentfill}{rgb}{0.121569,0.466667,0.705882}%
\pgfsetfillcolor{currentfill}%
\pgfsetfillopacity{0.455173}%
\pgfsetlinewidth{1.003750pt}%
\definecolor{currentstroke}{rgb}{0.121569,0.466667,0.705882}%
\pgfsetstrokecolor{currentstroke}%
\pgfsetstrokeopacity{0.455173}%
\pgfsetdash{}{0pt}%
\pgfpathmoveto{\pgfqpoint{2.026315in}{1.947648in}}%
\pgfpathcurveto{\pgfqpoint{2.034551in}{1.947648in}}{\pgfqpoint{2.042451in}{1.950921in}}{\pgfqpoint{2.048275in}{1.956745in}}%
\pgfpathcurveto{\pgfqpoint{2.054099in}{1.962569in}}{\pgfqpoint{2.057371in}{1.970469in}}{\pgfqpoint{2.057371in}{1.978705in}}%
\pgfpathcurveto{\pgfqpoint{2.057371in}{1.986941in}}{\pgfqpoint{2.054099in}{1.994841in}}{\pgfqpoint{2.048275in}{2.000665in}}%
\pgfpathcurveto{\pgfqpoint{2.042451in}{2.006489in}}{\pgfqpoint{2.034551in}{2.009761in}}{\pgfqpoint{2.026315in}{2.009761in}}%
\pgfpathcurveto{\pgfqpoint{2.018079in}{2.009761in}}{\pgfqpoint{2.010179in}{2.006489in}}{\pgfqpoint{2.004355in}{2.000665in}}%
\pgfpathcurveto{\pgfqpoint{1.998531in}{1.994841in}}{\pgfqpoint{1.995258in}{1.986941in}}{\pgfqpoint{1.995258in}{1.978705in}}%
\pgfpathcurveto{\pgfqpoint{1.995258in}{1.970469in}}{\pgfqpoint{1.998531in}{1.962569in}}{\pgfqpoint{2.004355in}{1.956745in}}%
\pgfpathcurveto{\pgfqpoint{2.010179in}{1.950921in}}{\pgfqpoint{2.018079in}{1.947648in}}{\pgfqpoint{2.026315in}{1.947648in}}%
\pgfpathclose%
\pgfusepath{stroke,fill}%
\end{pgfscope}%
\begin{pgfscope}%
\pgfpathrectangle{\pgfqpoint{0.100000in}{0.212622in}}{\pgfqpoint{3.696000in}{3.696000in}}%
\pgfusepath{clip}%
\pgfsetbuttcap%
\pgfsetroundjoin%
\definecolor{currentfill}{rgb}{0.121569,0.466667,0.705882}%
\pgfsetfillcolor{currentfill}%
\pgfsetfillopacity{0.456354}%
\pgfsetlinewidth{1.003750pt}%
\definecolor{currentstroke}{rgb}{0.121569,0.466667,0.705882}%
\pgfsetstrokecolor{currentstroke}%
\pgfsetstrokeopacity{0.456354}%
\pgfsetdash{}{0pt}%
\pgfpathmoveto{\pgfqpoint{1.439319in}{1.765957in}}%
\pgfpathcurveto{\pgfqpoint{1.447555in}{1.765957in}}{\pgfqpoint{1.455455in}{1.769229in}}{\pgfqpoint{1.461279in}{1.775053in}}%
\pgfpathcurveto{\pgfqpoint{1.467103in}{1.780877in}}{\pgfqpoint{1.470375in}{1.788777in}}{\pgfqpoint{1.470375in}{1.797014in}}%
\pgfpathcurveto{\pgfqpoint{1.470375in}{1.805250in}}{\pgfqpoint{1.467103in}{1.813150in}}{\pgfqpoint{1.461279in}{1.818974in}}%
\pgfpathcurveto{\pgfqpoint{1.455455in}{1.824798in}}{\pgfqpoint{1.447555in}{1.828070in}}{\pgfqpoint{1.439319in}{1.828070in}}%
\pgfpathcurveto{\pgfqpoint{1.431082in}{1.828070in}}{\pgfqpoint{1.423182in}{1.824798in}}{\pgfqpoint{1.417358in}{1.818974in}}%
\pgfpathcurveto{\pgfqpoint{1.411534in}{1.813150in}}{\pgfqpoint{1.408262in}{1.805250in}}{\pgfqpoint{1.408262in}{1.797014in}}%
\pgfpathcurveto{\pgfqpoint{1.408262in}{1.788777in}}{\pgfqpoint{1.411534in}{1.780877in}}{\pgfqpoint{1.417358in}{1.775053in}}%
\pgfpathcurveto{\pgfqpoint{1.423182in}{1.769229in}}{\pgfqpoint{1.431082in}{1.765957in}}{\pgfqpoint{1.439319in}{1.765957in}}%
\pgfpathclose%
\pgfusepath{stroke,fill}%
\end{pgfscope}%
\begin{pgfscope}%
\pgfpathrectangle{\pgfqpoint{0.100000in}{0.212622in}}{\pgfqpoint{3.696000in}{3.696000in}}%
\pgfusepath{clip}%
\pgfsetbuttcap%
\pgfsetroundjoin%
\definecolor{currentfill}{rgb}{0.121569,0.466667,0.705882}%
\pgfsetfillcolor{currentfill}%
\pgfsetfillopacity{0.457677}%
\pgfsetlinewidth{1.003750pt}%
\definecolor{currentstroke}{rgb}{0.121569,0.466667,0.705882}%
\pgfsetstrokecolor{currentstroke}%
\pgfsetstrokeopacity{0.457677}%
\pgfsetdash{}{0pt}%
\pgfpathmoveto{\pgfqpoint{2.027710in}{1.945118in}}%
\pgfpathcurveto{\pgfqpoint{2.035946in}{1.945118in}}{\pgfqpoint{2.043846in}{1.948391in}}{\pgfqpoint{2.049670in}{1.954214in}}%
\pgfpathcurveto{\pgfqpoint{2.055494in}{1.960038in}}{\pgfqpoint{2.058767in}{1.967938in}}{\pgfqpoint{2.058767in}{1.976175in}}%
\pgfpathcurveto{\pgfqpoint{2.058767in}{1.984411in}}{\pgfqpoint{2.055494in}{1.992311in}}{\pgfqpoint{2.049670in}{1.998135in}}%
\pgfpathcurveto{\pgfqpoint{2.043846in}{2.003959in}}{\pgfqpoint{2.035946in}{2.007231in}}{\pgfqpoint{2.027710in}{2.007231in}}%
\pgfpathcurveto{\pgfqpoint{2.019474in}{2.007231in}}{\pgfqpoint{2.011574in}{2.003959in}}{\pgfqpoint{2.005750in}{1.998135in}}%
\pgfpathcurveto{\pgfqpoint{1.999926in}{1.992311in}}{\pgfqpoint{1.996654in}{1.984411in}}{\pgfqpoint{1.996654in}{1.976175in}}%
\pgfpathcurveto{\pgfqpoint{1.996654in}{1.967938in}}{\pgfqpoint{1.999926in}{1.960038in}}{\pgfqpoint{2.005750in}{1.954214in}}%
\pgfpathcurveto{\pgfqpoint{2.011574in}{1.948391in}}{\pgfqpoint{2.019474in}{1.945118in}}{\pgfqpoint{2.027710in}{1.945118in}}%
\pgfpathclose%
\pgfusepath{stroke,fill}%
\end{pgfscope}%
\begin{pgfscope}%
\pgfpathrectangle{\pgfqpoint{0.100000in}{0.212622in}}{\pgfqpoint{3.696000in}{3.696000in}}%
\pgfusepath{clip}%
\pgfsetbuttcap%
\pgfsetroundjoin%
\definecolor{currentfill}{rgb}{0.121569,0.466667,0.705882}%
\pgfsetfillcolor{currentfill}%
\pgfsetfillopacity{0.458088}%
\pgfsetlinewidth{1.003750pt}%
\definecolor{currentstroke}{rgb}{0.121569,0.466667,0.705882}%
\pgfsetstrokecolor{currentstroke}%
\pgfsetstrokeopacity{0.458088}%
\pgfsetdash{}{0pt}%
\pgfpathmoveto{\pgfqpoint{1.433408in}{1.762443in}}%
\pgfpathcurveto{\pgfqpoint{1.441644in}{1.762443in}}{\pgfqpoint{1.449544in}{1.765715in}}{\pgfqpoint{1.455368in}{1.771539in}}%
\pgfpathcurveto{\pgfqpoint{1.461192in}{1.777363in}}{\pgfqpoint{1.464465in}{1.785263in}}{\pgfqpoint{1.464465in}{1.793500in}}%
\pgfpathcurveto{\pgfqpoint{1.464465in}{1.801736in}}{\pgfqpoint{1.461192in}{1.809636in}}{\pgfqpoint{1.455368in}{1.815460in}}%
\pgfpathcurveto{\pgfqpoint{1.449544in}{1.821284in}}{\pgfqpoint{1.441644in}{1.824556in}}{\pgfqpoint{1.433408in}{1.824556in}}%
\pgfpathcurveto{\pgfqpoint{1.425172in}{1.824556in}}{\pgfqpoint{1.417272in}{1.821284in}}{\pgfqpoint{1.411448in}{1.815460in}}%
\pgfpathcurveto{\pgfqpoint{1.405624in}{1.809636in}}{\pgfqpoint{1.402352in}{1.801736in}}{\pgfqpoint{1.402352in}{1.793500in}}%
\pgfpathcurveto{\pgfqpoint{1.402352in}{1.785263in}}{\pgfqpoint{1.405624in}{1.777363in}}{\pgfqpoint{1.411448in}{1.771539in}}%
\pgfpathcurveto{\pgfqpoint{1.417272in}{1.765715in}}{\pgfqpoint{1.425172in}{1.762443in}}{\pgfqpoint{1.433408in}{1.762443in}}%
\pgfpathclose%
\pgfusepath{stroke,fill}%
\end{pgfscope}%
\begin{pgfscope}%
\pgfpathrectangle{\pgfqpoint{0.100000in}{0.212622in}}{\pgfqpoint{3.696000in}{3.696000in}}%
\pgfusepath{clip}%
\pgfsetbuttcap%
\pgfsetroundjoin%
\definecolor{currentfill}{rgb}{0.121569,0.466667,0.705882}%
\pgfsetfillcolor{currentfill}%
\pgfsetfillopacity{0.459312}%
\pgfsetlinewidth{1.003750pt}%
\definecolor{currentstroke}{rgb}{0.121569,0.466667,0.705882}%
\pgfsetstrokecolor{currentstroke}%
\pgfsetstrokeopacity{0.459312}%
\pgfsetdash{}{0pt}%
\pgfpathmoveto{\pgfqpoint{2.028141in}{1.945321in}}%
\pgfpathcurveto{\pgfqpoint{2.036377in}{1.945321in}}{\pgfqpoint{2.044277in}{1.948593in}}{\pgfqpoint{2.050101in}{1.954417in}}%
\pgfpathcurveto{\pgfqpoint{2.055925in}{1.960241in}}{\pgfqpoint{2.059197in}{1.968141in}}{\pgfqpoint{2.059197in}{1.976377in}}%
\pgfpathcurveto{\pgfqpoint{2.059197in}{1.984613in}}{\pgfqpoint{2.055925in}{1.992513in}}{\pgfqpoint{2.050101in}{1.998337in}}%
\pgfpathcurveto{\pgfqpoint{2.044277in}{2.004161in}}{\pgfqpoint{2.036377in}{2.007434in}}{\pgfqpoint{2.028141in}{2.007434in}}%
\pgfpathcurveto{\pgfqpoint{2.019904in}{2.007434in}}{\pgfqpoint{2.012004in}{2.004161in}}{\pgfqpoint{2.006180in}{1.998337in}}%
\pgfpathcurveto{\pgfqpoint{2.000356in}{1.992513in}}{\pgfqpoint{1.997084in}{1.984613in}}{\pgfqpoint{1.997084in}{1.976377in}}%
\pgfpathcurveto{\pgfqpoint{1.997084in}{1.968141in}}{\pgfqpoint{2.000356in}{1.960241in}}{\pgfqpoint{2.006180in}{1.954417in}}%
\pgfpathcurveto{\pgfqpoint{2.012004in}{1.948593in}}{\pgfqpoint{2.019904in}{1.945321in}}{\pgfqpoint{2.028141in}{1.945321in}}%
\pgfpathclose%
\pgfusepath{stroke,fill}%
\end{pgfscope}%
\begin{pgfscope}%
\pgfpathrectangle{\pgfqpoint{0.100000in}{0.212622in}}{\pgfqpoint{3.696000in}{3.696000in}}%
\pgfusepath{clip}%
\pgfsetbuttcap%
\pgfsetroundjoin%
\definecolor{currentfill}{rgb}{0.121569,0.466667,0.705882}%
\pgfsetfillcolor{currentfill}%
\pgfsetfillopacity{0.459418}%
\pgfsetlinewidth{1.003750pt}%
\definecolor{currentstroke}{rgb}{0.121569,0.466667,0.705882}%
\pgfsetstrokecolor{currentstroke}%
\pgfsetstrokeopacity{0.459418}%
\pgfsetdash{}{0pt}%
\pgfpathmoveto{\pgfqpoint{1.428518in}{1.756709in}}%
\pgfpathcurveto{\pgfqpoint{1.436754in}{1.756709in}}{\pgfqpoint{1.444654in}{1.759982in}}{\pgfqpoint{1.450478in}{1.765806in}}%
\pgfpathcurveto{\pgfqpoint{1.456302in}{1.771629in}}{\pgfqpoint{1.459574in}{1.779530in}}{\pgfqpoint{1.459574in}{1.787766in}}%
\pgfpathcurveto{\pgfqpoint{1.459574in}{1.796002in}}{\pgfqpoint{1.456302in}{1.803902in}}{\pgfqpoint{1.450478in}{1.809726in}}%
\pgfpathcurveto{\pgfqpoint{1.444654in}{1.815550in}}{\pgfqpoint{1.436754in}{1.818822in}}{\pgfqpoint{1.428518in}{1.818822in}}%
\pgfpathcurveto{\pgfqpoint{1.420281in}{1.818822in}}{\pgfqpoint{1.412381in}{1.815550in}}{\pgfqpoint{1.406557in}{1.809726in}}%
\pgfpathcurveto{\pgfqpoint{1.400734in}{1.803902in}}{\pgfqpoint{1.397461in}{1.796002in}}{\pgfqpoint{1.397461in}{1.787766in}}%
\pgfpathcurveto{\pgfqpoint{1.397461in}{1.779530in}}{\pgfqpoint{1.400734in}{1.771629in}}{\pgfqpoint{1.406557in}{1.765806in}}%
\pgfpathcurveto{\pgfqpoint{1.412381in}{1.759982in}}{\pgfqpoint{1.420281in}{1.756709in}}{\pgfqpoint{1.428518in}{1.756709in}}%
\pgfpathclose%
\pgfusepath{stroke,fill}%
\end{pgfscope}%
\begin{pgfscope}%
\pgfpathrectangle{\pgfqpoint{0.100000in}{0.212622in}}{\pgfqpoint{3.696000in}{3.696000in}}%
\pgfusepath{clip}%
\pgfsetbuttcap%
\pgfsetroundjoin%
\definecolor{currentfill}{rgb}{0.121569,0.466667,0.705882}%
\pgfsetfillcolor{currentfill}%
\pgfsetfillopacity{0.460763}%
\pgfsetlinewidth{1.003750pt}%
\definecolor{currentstroke}{rgb}{0.121569,0.466667,0.705882}%
\pgfsetstrokecolor{currentstroke}%
\pgfsetstrokeopacity{0.460763}%
\pgfsetdash{}{0pt}%
\pgfpathmoveto{\pgfqpoint{1.424592in}{1.754374in}}%
\pgfpathcurveto{\pgfqpoint{1.432828in}{1.754374in}}{\pgfqpoint{1.440728in}{1.757647in}}{\pgfqpoint{1.446552in}{1.763471in}}%
\pgfpathcurveto{\pgfqpoint{1.452376in}{1.769295in}}{\pgfqpoint{1.455648in}{1.777195in}}{\pgfqpoint{1.455648in}{1.785431in}}%
\pgfpathcurveto{\pgfqpoint{1.455648in}{1.793667in}}{\pgfqpoint{1.452376in}{1.801567in}}{\pgfqpoint{1.446552in}{1.807391in}}%
\pgfpathcurveto{\pgfqpoint{1.440728in}{1.813215in}}{\pgfqpoint{1.432828in}{1.816487in}}{\pgfqpoint{1.424592in}{1.816487in}}%
\pgfpathcurveto{\pgfqpoint{1.416355in}{1.816487in}}{\pgfqpoint{1.408455in}{1.813215in}}{\pgfqpoint{1.402631in}{1.807391in}}%
\pgfpathcurveto{\pgfqpoint{1.396807in}{1.801567in}}{\pgfqpoint{1.393535in}{1.793667in}}{\pgfqpoint{1.393535in}{1.785431in}}%
\pgfpathcurveto{\pgfqpoint{1.393535in}{1.777195in}}{\pgfqpoint{1.396807in}{1.769295in}}{\pgfqpoint{1.402631in}{1.763471in}}%
\pgfpathcurveto{\pgfqpoint{1.408455in}{1.757647in}}{\pgfqpoint{1.416355in}{1.754374in}}{\pgfqpoint{1.424592in}{1.754374in}}%
\pgfpathclose%
\pgfusepath{stroke,fill}%
\end{pgfscope}%
\begin{pgfscope}%
\pgfpathrectangle{\pgfqpoint{0.100000in}{0.212622in}}{\pgfqpoint{3.696000in}{3.696000in}}%
\pgfusepath{clip}%
\pgfsetbuttcap%
\pgfsetroundjoin%
\definecolor{currentfill}{rgb}{0.121569,0.466667,0.705882}%
\pgfsetfillcolor{currentfill}%
\pgfsetfillopacity{0.461211}%
\pgfsetlinewidth{1.003750pt}%
\definecolor{currentstroke}{rgb}{0.121569,0.466667,0.705882}%
\pgfsetstrokecolor{currentstroke}%
\pgfsetstrokeopacity{0.461211}%
\pgfsetdash{}{0pt}%
\pgfpathmoveto{\pgfqpoint{2.029395in}{1.943746in}}%
\pgfpathcurveto{\pgfqpoint{2.037632in}{1.943746in}}{\pgfqpoint{2.045532in}{1.947018in}}{\pgfqpoint{2.051356in}{1.952842in}}%
\pgfpathcurveto{\pgfqpoint{2.057179in}{1.958666in}}{\pgfqpoint{2.060452in}{1.966566in}}{\pgfqpoint{2.060452in}{1.974802in}}%
\pgfpathcurveto{\pgfqpoint{2.060452in}{1.983038in}}{\pgfqpoint{2.057179in}{1.990939in}}{\pgfqpoint{2.051356in}{1.996762in}}%
\pgfpathcurveto{\pgfqpoint{2.045532in}{2.002586in}}{\pgfqpoint{2.037632in}{2.005859in}}{\pgfqpoint{2.029395in}{2.005859in}}%
\pgfpathcurveto{\pgfqpoint{2.021159in}{2.005859in}}{\pgfqpoint{2.013259in}{2.002586in}}{\pgfqpoint{2.007435in}{1.996762in}}%
\pgfpathcurveto{\pgfqpoint{2.001611in}{1.990939in}}{\pgfqpoint{1.998339in}{1.983038in}}{\pgfqpoint{1.998339in}{1.974802in}}%
\pgfpathcurveto{\pgfqpoint{1.998339in}{1.966566in}}{\pgfqpoint{2.001611in}{1.958666in}}{\pgfqpoint{2.007435in}{1.952842in}}%
\pgfpathcurveto{\pgfqpoint{2.013259in}{1.947018in}}{\pgfqpoint{2.021159in}{1.943746in}}{\pgfqpoint{2.029395in}{1.943746in}}%
\pgfpathclose%
\pgfusepath{stroke,fill}%
\end{pgfscope}%
\begin{pgfscope}%
\pgfpathrectangle{\pgfqpoint{0.100000in}{0.212622in}}{\pgfqpoint{3.696000in}{3.696000in}}%
\pgfusepath{clip}%
\pgfsetbuttcap%
\pgfsetroundjoin%
\definecolor{currentfill}{rgb}{0.121569,0.466667,0.705882}%
\pgfsetfillcolor{currentfill}%
\pgfsetfillopacity{0.462343}%
\pgfsetlinewidth{1.003750pt}%
\definecolor{currentstroke}{rgb}{0.121569,0.466667,0.705882}%
\pgfsetstrokecolor{currentstroke}%
\pgfsetstrokeopacity{0.462343}%
\pgfsetdash{}{0pt}%
\pgfpathmoveto{\pgfqpoint{2.030043in}{1.943437in}}%
\pgfpathcurveto{\pgfqpoint{2.038279in}{1.943437in}}{\pgfqpoint{2.046179in}{1.946709in}}{\pgfqpoint{2.052003in}{1.952533in}}%
\pgfpathcurveto{\pgfqpoint{2.057827in}{1.958357in}}{\pgfqpoint{2.061099in}{1.966257in}}{\pgfqpoint{2.061099in}{1.974493in}}%
\pgfpathcurveto{\pgfqpoint{2.061099in}{1.982730in}}{\pgfqpoint{2.057827in}{1.990630in}}{\pgfqpoint{2.052003in}{1.996454in}}%
\pgfpathcurveto{\pgfqpoint{2.046179in}{2.002278in}}{\pgfqpoint{2.038279in}{2.005550in}}{\pgfqpoint{2.030043in}{2.005550in}}%
\pgfpathcurveto{\pgfqpoint{2.021807in}{2.005550in}}{\pgfqpoint{2.013907in}{2.002278in}}{\pgfqpoint{2.008083in}{1.996454in}}%
\pgfpathcurveto{\pgfqpoint{2.002259in}{1.990630in}}{\pgfqpoint{1.998986in}{1.982730in}}{\pgfqpoint{1.998986in}{1.974493in}}%
\pgfpathcurveto{\pgfqpoint{1.998986in}{1.966257in}}{\pgfqpoint{2.002259in}{1.958357in}}{\pgfqpoint{2.008083in}{1.952533in}}%
\pgfpathcurveto{\pgfqpoint{2.013907in}{1.946709in}}{\pgfqpoint{2.021807in}{1.943437in}}{\pgfqpoint{2.030043in}{1.943437in}}%
\pgfpathclose%
\pgfusepath{stroke,fill}%
\end{pgfscope}%
\begin{pgfscope}%
\pgfpathrectangle{\pgfqpoint{0.100000in}{0.212622in}}{\pgfqpoint{3.696000in}{3.696000in}}%
\pgfusepath{clip}%
\pgfsetbuttcap%
\pgfsetroundjoin%
\definecolor{currentfill}{rgb}{0.121569,0.466667,0.705882}%
\pgfsetfillcolor{currentfill}%
\pgfsetfillopacity{0.463151}%
\pgfsetlinewidth{1.003750pt}%
\definecolor{currentstroke}{rgb}{0.121569,0.466667,0.705882}%
\pgfsetstrokecolor{currentstroke}%
\pgfsetstrokeopacity{0.463151}%
\pgfsetdash{}{0pt}%
\pgfpathmoveto{\pgfqpoint{1.416891in}{1.750571in}}%
\pgfpathcurveto{\pgfqpoint{1.425127in}{1.750571in}}{\pgfqpoint{1.433027in}{1.753844in}}{\pgfqpoint{1.438851in}{1.759668in}}%
\pgfpathcurveto{\pgfqpoint{1.444675in}{1.765492in}}{\pgfqpoint{1.447948in}{1.773392in}}{\pgfqpoint{1.447948in}{1.781628in}}%
\pgfpathcurveto{\pgfqpoint{1.447948in}{1.789864in}}{\pgfqpoint{1.444675in}{1.797764in}}{\pgfqpoint{1.438851in}{1.803588in}}%
\pgfpathcurveto{\pgfqpoint{1.433027in}{1.809412in}}{\pgfqpoint{1.425127in}{1.812684in}}{\pgfqpoint{1.416891in}{1.812684in}}%
\pgfpathcurveto{\pgfqpoint{1.408655in}{1.812684in}}{\pgfqpoint{1.400755in}{1.809412in}}{\pgfqpoint{1.394931in}{1.803588in}}%
\pgfpathcurveto{\pgfqpoint{1.389107in}{1.797764in}}{\pgfqpoint{1.385835in}{1.789864in}}{\pgfqpoint{1.385835in}{1.781628in}}%
\pgfpathcurveto{\pgfqpoint{1.385835in}{1.773392in}}{\pgfqpoint{1.389107in}{1.765492in}}{\pgfqpoint{1.394931in}{1.759668in}}%
\pgfpathcurveto{\pgfqpoint{1.400755in}{1.753844in}}{\pgfqpoint{1.408655in}{1.750571in}}{\pgfqpoint{1.416891in}{1.750571in}}%
\pgfpathclose%
\pgfusepath{stroke,fill}%
\end{pgfscope}%
\begin{pgfscope}%
\pgfpathrectangle{\pgfqpoint{0.100000in}{0.212622in}}{\pgfqpoint{3.696000in}{3.696000in}}%
\pgfusepath{clip}%
\pgfsetbuttcap%
\pgfsetroundjoin%
\definecolor{currentfill}{rgb}{0.121569,0.466667,0.705882}%
\pgfsetfillcolor{currentfill}%
\pgfsetfillopacity{0.463510}%
\pgfsetlinewidth{1.003750pt}%
\definecolor{currentstroke}{rgb}{0.121569,0.466667,0.705882}%
\pgfsetstrokecolor{currentstroke}%
\pgfsetstrokeopacity{0.463510}%
\pgfsetdash{}{0pt}%
\pgfpathmoveto{\pgfqpoint{2.030731in}{1.942327in}}%
\pgfpathcurveto{\pgfqpoint{2.038967in}{1.942327in}}{\pgfqpoint{2.046867in}{1.945600in}}{\pgfqpoint{2.052691in}{1.951424in}}%
\pgfpathcurveto{\pgfqpoint{2.058515in}{1.957247in}}{\pgfqpoint{2.061788in}{1.965147in}}{\pgfqpoint{2.061788in}{1.973384in}}%
\pgfpathcurveto{\pgfqpoint{2.061788in}{1.981620in}}{\pgfqpoint{2.058515in}{1.989520in}}{\pgfqpoint{2.052691in}{1.995344in}}%
\pgfpathcurveto{\pgfqpoint{2.046867in}{2.001168in}}{\pgfqpoint{2.038967in}{2.004440in}}{\pgfqpoint{2.030731in}{2.004440in}}%
\pgfpathcurveto{\pgfqpoint{2.022495in}{2.004440in}}{\pgfqpoint{2.014595in}{2.001168in}}{\pgfqpoint{2.008771in}{1.995344in}}%
\pgfpathcurveto{\pgfqpoint{2.002947in}{1.989520in}}{\pgfqpoint{1.999675in}{1.981620in}}{\pgfqpoint{1.999675in}{1.973384in}}%
\pgfpathcurveto{\pgfqpoint{1.999675in}{1.965147in}}{\pgfqpoint{2.002947in}{1.957247in}}{\pgfqpoint{2.008771in}{1.951424in}}%
\pgfpathcurveto{\pgfqpoint{2.014595in}{1.945600in}}{\pgfqpoint{2.022495in}{1.942327in}}{\pgfqpoint{2.030731in}{1.942327in}}%
\pgfpathclose%
\pgfusepath{stroke,fill}%
\end{pgfscope}%
\begin{pgfscope}%
\pgfpathrectangle{\pgfqpoint{0.100000in}{0.212622in}}{\pgfqpoint{3.696000in}{3.696000in}}%
\pgfusepath{clip}%
\pgfsetbuttcap%
\pgfsetroundjoin%
\definecolor{currentfill}{rgb}{0.121569,0.466667,0.705882}%
\pgfsetfillcolor{currentfill}%
\pgfsetfillopacity{0.464179}%
\pgfsetlinewidth{1.003750pt}%
\definecolor{currentstroke}{rgb}{0.121569,0.466667,0.705882}%
\pgfsetstrokecolor{currentstroke}%
\pgfsetstrokeopacity{0.464179}%
\pgfsetdash{}{0pt}%
\pgfpathmoveto{\pgfqpoint{2.031077in}{1.941886in}}%
\pgfpathcurveto{\pgfqpoint{2.039313in}{1.941886in}}{\pgfqpoint{2.047213in}{1.945159in}}{\pgfqpoint{2.053037in}{1.950983in}}%
\pgfpathcurveto{\pgfqpoint{2.058861in}{1.956806in}}{\pgfqpoint{2.062134in}{1.964707in}}{\pgfqpoint{2.062134in}{1.972943in}}%
\pgfpathcurveto{\pgfqpoint{2.062134in}{1.981179in}}{\pgfqpoint{2.058861in}{1.989079in}}{\pgfqpoint{2.053037in}{1.994903in}}%
\pgfpathcurveto{\pgfqpoint{2.047213in}{2.000727in}}{\pgfqpoint{2.039313in}{2.003999in}}{\pgfqpoint{2.031077in}{2.003999in}}%
\pgfpathcurveto{\pgfqpoint{2.022841in}{2.003999in}}{\pgfqpoint{2.014941in}{2.000727in}}{\pgfqpoint{2.009117in}{1.994903in}}%
\pgfpathcurveto{\pgfqpoint{2.003293in}{1.989079in}}{\pgfqpoint{2.000021in}{1.981179in}}{\pgfqpoint{2.000021in}{1.972943in}}%
\pgfpathcurveto{\pgfqpoint{2.000021in}{1.964707in}}{\pgfqpoint{2.003293in}{1.956806in}}{\pgfqpoint{2.009117in}{1.950983in}}%
\pgfpathcurveto{\pgfqpoint{2.014941in}{1.945159in}}{\pgfqpoint{2.022841in}{1.941886in}}{\pgfqpoint{2.031077in}{1.941886in}}%
\pgfpathclose%
\pgfusepath{stroke,fill}%
\end{pgfscope}%
\begin{pgfscope}%
\pgfpathrectangle{\pgfqpoint{0.100000in}{0.212622in}}{\pgfqpoint{3.696000in}{3.696000in}}%
\pgfusepath{clip}%
\pgfsetbuttcap%
\pgfsetroundjoin%
\definecolor{currentfill}{rgb}{0.121569,0.466667,0.705882}%
\pgfsetfillcolor{currentfill}%
\pgfsetfillopacity{0.464833}%
\pgfsetlinewidth{1.003750pt}%
\definecolor{currentstroke}{rgb}{0.121569,0.466667,0.705882}%
\pgfsetstrokecolor{currentstroke}%
\pgfsetstrokeopacity{0.464833}%
\pgfsetdash{}{0pt}%
\pgfpathmoveto{\pgfqpoint{1.410927in}{1.743449in}}%
\pgfpathcurveto{\pgfqpoint{1.419163in}{1.743449in}}{\pgfqpoint{1.427063in}{1.746722in}}{\pgfqpoint{1.432887in}{1.752546in}}%
\pgfpathcurveto{\pgfqpoint{1.438711in}{1.758369in}}{\pgfqpoint{1.441983in}{1.766269in}}{\pgfqpoint{1.441983in}{1.774506in}}%
\pgfpathcurveto{\pgfqpoint{1.441983in}{1.782742in}}{\pgfqpoint{1.438711in}{1.790642in}}{\pgfqpoint{1.432887in}{1.796466in}}%
\pgfpathcurveto{\pgfqpoint{1.427063in}{1.802290in}}{\pgfqpoint{1.419163in}{1.805562in}}{\pgfqpoint{1.410927in}{1.805562in}}%
\pgfpathcurveto{\pgfqpoint{1.402691in}{1.805562in}}{\pgfqpoint{1.394791in}{1.802290in}}{\pgfqpoint{1.388967in}{1.796466in}}%
\pgfpathcurveto{\pgfqpoint{1.383143in}{1.790642in}}{\pgfqpoint{1.379870in}{1.782742in}}{\pgfqpoint{1.379870in}{1.774506in}}%
\pgfpathcurveto{\pgfqpoint{1.379870in}{1.766269in}}{\pgfqpoint{1.383143in}{1.758369in}}{\pgfqpoint{1.388967in}{1.752546in}}%
\pgfpathcurveto{\pgfqpoint{1.394791in}{1.746722in}}{\pgfqpoint{1.402691in}{1.743449in}}{\pgfqpoint{1.410927in}{1.743449in}}%
\pgfpathclose%
\pgfusepath{stroke,fill}%
\end{pgfscope}%
\begin{pgfscope}%
\pgfpathrectangle{\pgfqpoint{0.100000in}{0.212622in}}{\pgfqpoint{3.696000in}{3.696000in}}%
\pgfusepath{clip}%
\pgfsetbuttcap%
\pgfsetroundjoin%
\definecolor{currentfill}{rgb}{0.121569,0.466667,0.705882}%
\pgfsetfillcolor{currentfill}%
\pgfsetfillopacity{0.465124}%
\pgfsetlinewidth{1.003750pt}%
\definecolor{currentstroke}{rgb}{0.121569,0.466667,0.705882}%
\pgfsetstrokecolor{currentstroke}%
\pgfsetstrokeopacity{0.465124}%
\pgfsetdash{}{0pt}%
\pgfpathmoveto{\pgfqpoint{2.031486in}{1.941831in}}%
\pgfpathcurveto{\pgfqpoint{2.039722in}{1.941831in}}{\pgfqpoint{2.047622in}{1.945104in}}{\pgfqpoint{2.053446in}{1.950928in}}%
\pgfpathcurveto{\pgfqpoint{2.059270in}{1.956751in}}{\pgfqpoint{2.062542in}{1.964652in}}{\pgfqpoint{2.062542in}{1.972888in}}%
\pgfpathcurveto{\pgfqpoint{2.062542in}{1.981124in}}{\pgfqpoint{2.059270in}{1.989024in}}{\pgfqpoint{2.053446in}{1.994848in}}%
\pgfpathcurveto{\pgfqpoint{2.047622in}{2.000672in}}{\pgfqpoint{2.039722in}{2.003944in}}{\pgfqpoint{2.031486in}{2.003944in}}%
\pgfpathcurveto{\pgfqpoint{2.023250in}{2.003944in}}{\pgfqpoint{2.015349in}{2.000672in}}{\pgfqpoint{2.009526in}{1.994848in}}%
\pgfpathcurveto{\pgfqpoint{2.003702in}{1.989024in}}{\pgfqpoint{2.000429in}{1.981124in}}{\pgfqpoint{2.000429in}{1.972888in}}%
\pgfpathcurveto{\pgfqpoint{2.000429in}{1.964652in}}{\pgfqpoint{2.003702in}{1.956751in}}{\pgfqpoint{2.009526in}{1.950928in}}%
\pgfpathcurveto{\pgfqpoint{2.015349in}{1.945104in}}{\pgfqpoint{2.023250in}{1.941831in}}{\pgfqpoint{2.031486in}{1.941831in}}%
\pgfpathclose%
\pgfusepath{stroke,fill}%
\end{pgfscope}%
\begin{pgfscope}%
\pgfpathrectangle{\pgfqpoint{0.100000in}{0.212622in}}{\pgfqpoint{3.696000in}{3.696000in}}%
\pgfusepath{clip}%
\pgfsetbuttcap%
\pgfsetroundjoin%
\definecolor{currentfill}{rgb}{0.121569,0.466667,0.705882}%
\pgfsetfillcolor{currentfill}%
\pgfsetfillopacity{0.466326}%
\pgfsetlinewidth{1.003750pt}%
\definecolor{currentstroke}{rgb}{0.121569,0.466667,0.705882}%
\pgfsetstrokecolor{currentstroke}%
\pgfsetstrokeopacity{0.466326}%
\pgfsetdash{}{0pt}%
\pgfpathmoveto{\pgfqpoint{1.406302in}{1.740086in}}%
\pgfpathcurveto{\pgfqpoint{1.414538in}{1.740086in}}{\pgfqpoint{1.422439in}{1.743358in}}{\pgfqpoint{1.428262in}{1.749182in}}%
\pgfpathcurveto{\pgfqpoint{1.434086in}{1.755006in}}{\pgfqpoint{1.437359in}{1.762906in}}{\pgfqpoint{1.437359in}{1.771142in}}%
\pgfpathcurveto{\pgfqpoint{1.437359in}{1.779378in}}{\pgfqpoint{1.434086in}{1.787278in}}{\pgfqpoint{1.428262in}{1.793102in}}%
\pgfpathcurveto{\pgfqpoint{1.422439in}{1.798926in}}{\pgfqpoint{1.414538in}{1.802199in}}{\pgfqpoint{1.406302in}{1.802199in}}%
\pgfpathcurveto{\pgfqpoint{1.398066in}{1.802199in}}{\pgfqpoint{1.390166in}{1.798926in}}{\pgfqpoint{1.384342in}{1.793102in}}%
\pgfpathcurveto{\pgfqpoint{1.378518in}{1.787278in}}{\pgfqpoint{1.375246in}{1.779378in}}{\pgfqpoint{1.375246in}{1.771142in}}%
\pgfpathcurveto{\pgfqpoint{1.375246in}{1.762906in}}{\pgfqpoint{1.378518in}{1.755006in}}{\pgfqpoint{1.384342in}{1.749182in}}%
\pgfpathcurveto{\pgfqpoint{1.390166in}{1.743358in}}{\pgfqpoint{1.398066in}{1.740086in}}{\pgfqpoint{1.406302in}{1.740086in}}%
\pgfpathclose%
\pgfusepath{stroke,fill}%
\end{pgfscope}%
\begin{pgfscope}%
\pgfpathrectangle{\pgfqpoint{0.100000in}{0.212622in}}{\pgfqpoint{3.696000in}{3.696000in}}%
\pgfusepath{clip}%
\pgfsetbuttcap%
\pgfsetroundjoin%
\definecolor{currentfill}{rgb}{0.121569,0.466667,0.705882}%
\pgfsetfillcolor{currentfill}%
\pgfsetfillopacity{0.466628}%
\pgfsetlinewidth{1.003750pt}%
\definecolor{currentstroke}{rgb}{0.121569,0.466667,0.705882}%
\pgfsetstrokecolor{currentstroke}%
\pgfsetstrokeopacity{0.466628}%
\pgfsetdash{}{0pt}%
\pgfpathmoveto{\pgfqpoint{2.032112in}{1.940604in}}%
\pgfpathcurveto{\pgfqpoint{2.040348in}{1.940604in}}{\pgfqpoint{2.048248in}{1.943876in}}{\pgfqpoint{2.054072in}{1.949700in}}%
\pgfpathcurveto{\pgfqpoint{2.059896in}{1.955524in}}{\pgfqpoint{2.063168in}{1.963424in}}{\pgfqpoint{2.063168in}{1.971660in}}%
\pgfpathcurveto{\pgfqpoint{2.063168in}{1.979897in}}{\pgfqpoint{2.059896in}{1.987797in}}{\pgfqpoint{2.054072in}{1.993621in}}%
\pgfpathcurveto{\pgfqpoint{2.048248in}{1.999445in}}{\pgfqpoint{2.040348in}{2.002717in}}{\pgfqpoint{2.032112in}{2.002717in}}%
\pgfpathcurveto{\pgfqpoint{2.023876in}{2.002717in}}{\pgfqpoint{2.015976in}{1.999445in}}{\pgfqpoint{2.010152in}{1.993621in}}%
\pgfpathcurveto{\pgfqpoint{2.004328in}{1.987797in}}{\pgfqpoint{2.001055in}{1.979897in}}{\pgfqpoint{2.001055in}{1.971660in}}%
\pgfpathcurveto{\pgfqpoint{2.001055in}{1.963424in}}{\pgfqpoint{2.004328in}{1.955524in}}{\pgfqpoint{2.010152in}{1.949700in}}%
\pgfpathcurveto{\pgfqpoint{2.015976in}{1.943876in}}{\pgfqpoint{2.023876in}{1.940604in}}{\pgfqpoint{2.032112in}{1.940604in}}%
\pgfpathclose%
\pgfusepath{stroke,fill}%
\end{pgfscope}%
\begin{pgfscope}%
\pgfpathrectangle{\pgfqpoint{0.100000in}{0.212622in}}{\pgfqpoint{3.696000in}{3.696000in}}%
\pgfusepath{clip}%
\pgfsetbuttcap%
\pgfsetroundjoin%
\definecolor{currentfill}{rgb}{0.121569,0.466667,0.705882}%
\pgfsetfillcolor{currentfill}%
\pgfsetfillopacity{0.467402}%
\pgfsetlinewidth{1.003750pt}%
\definecolor{currentstroke}{rgb}{0.121569,0.466667,0.705882}%
\pgfsetstrokecolor{currentstroke}%
\pgfsetstrokeopacity{0.467402}%
\pgfsetdash{}{0pt}%
\pgfpathmoveto{\pgfqpoint{2.032537in}{1.939596in}}%
\pgfpathcurveto{\pgfqpoint{2.040773in}{1.939596in}}{\pgfqpoint{2.048674in}{1.942869in}}{\pgfqpoint{2.054497in}{1.948693in}}%
\pgfpathcurveto{\pgfqpoint{2.060321in}{1.954516in}}{\pgfqpoint{2.063594in}{1.962417in}}{\pgfqpoint{2.063594in}{1.970653in}}%
\pgfpathcurveto{\pgfqpoint{2.063594in}{1.978889in}}{\pgfqpoint{2.060321in}{1.986789in}}{\pgfqpoint{2.054497in}{1.992613in}}%
\pgfpathcurveto{\pgfqpoint{2.048674in}{1.998437in}}{\pgfqpoint{2.040773in}{2.001709in}}{\pgfqpoint{2.032537in}{2.001709in}}%
\pgfpathcurveto{\pgfqpoint{2.024301in}{2.001709in}}{\pgfqpoint{2.016401in}{1.998437in}}{\pgfqpoint{2.010577in}{1.992613in}}%
\pgfpathcurveto{\pgfqpoint{2.004753in}{1.986789in}}{\pgfqpoint{2.001481in}{1.978889in}}{\pgfqpoint{2.001481in}{1.970653in}}%
\pgfpathcurveto{\pgfqpoint{2.001481in}{1.962417in}}{\pgfqpoint{2.004753in}{1.954516in}}{\pgfqpoint{2.010577in}{1.948693in}}%
\pgfpathcurveto{\pgfqpoint{2.016401in}{1.942869in}}{\pgfqpoint{2.024301in}{1.939596in}}{\pgfqpoint{2.032537in}{1.939596in}}%
\pgfpathclose%
\pgfusepath{stroke,fill}%
\end{pgfscope}%
\begin{pgfscope}%
\pgfpathrectangle{\pgfqpoint{0.100000in}{0.212622in}}{\pgfqpoint{3.696000in}{3.696000in}}%
\pgfusepath{clip}%
\pgfsetbuttcap%
\pgfsetroundjoin%
\definecolor{currentfill}{rgb}{0.121569,0.466667,0.705882}%
\pgfsetfillcolor{currentfill}%
\pgfsetfillopacity{0.467690}%
\pgfsetlinewidth{1.003750pt}%
\definecolor{currentstroke}{rgb}{0.121569,0.466667,0.705882}%
\pgfsetstrokecolor{currentstroke}%
\pgfsetstrokeopacity{0.467690}%
\pgfsetdash{}{0pt}%
\pgfpathmoveto{\pgfqpoint{1.401649in}{1.737289in}}%
\pgfpathcurveto{\pgfqpoint{1.409886in}{1.737289in}}{\pgfqpoint{1.417786in}{1.740562in}}{\pgfqpoint{1.423610in}{1.746385in}}%
\pgfpathcurveto{\pgfqpoint{1.429434in}{1.752209in}}{\pgfqpoint{1.432706in}{1.760109in}}{\pgfqpoint{1.432706in}{1.768346in}}%
\pgfpathcurveto{\pgfqpoint{1.432706in}{1.776582in}}{\pgfqpoint{1.429434in}{1.784482in}}{\pgfqpoint{1.423610in}{1.790306in}}%
\pgfpathcurveto{\pgfqpoint{1.417786in}{1.796130in}}{\pgfqpoint{1.409886in}{1.799402in}}{\pgfqpoint{1.401649in}{1.799402in}}%
\pgfpathcurveto{\pgfqpoint{1.393413in}{1.799402in}}{\pgfqpoint{1.385513in}{1.796130in}}{\pgfqpoint{1.379689in}{1.790306in}}%
\pgfpathcurveto{\pgfqpoint{1.373865in}{1.784482in}}{\pgfqpoint{1.370593in}{1.776582in}}{\pgfqpoint{1.370593in}{1.768346in}}%
\pgfpathcurveto{\pgfqpoint{1.370593in}{1.760109in}}{\pgfqpoint{1.373865in}{1.752209in}}{\pgfqpoint{1.379689in}{1.746385in}}%
\pgfpathcurveto{\pgfqpoint{1.385513in}{1.740562in}}{\pgfqpoint{1.393413in}{1.737289in}}{\pgfqpoint{1.401649in}{1.737289in}}%
\pgfpathclose%
\pgfusepath{stroke,fill}%
\end{pgfscope}%
\begin{pgfscope}%
\pgfpathrectangle{\pgfqpoint{0.100000in}{0.212622in}}{\pgfqpoint{3.696000in}{3.696000in}}%
\pgfusepath{clip}%
\pgfsetbuttcap%
\pgfsetroundjoin%
\definecolor{currentfill}{rgb}{0.121569,0.466667,0.705882}%
\pgfsetfillcolor{currentfill}%
\pgfsetfillopacity{0.468899}%
\pgfsetlinewidth{1.003750pt}%
\definecolor{currentstroke}{rgb}{0.121569,0.466667,0.705882}%
\pgfsetstrokecolor{currentstroke}%
\pgfsetstrokeopacity{0.468899}%
\pgfsetdash{}{0pt}%
\pgfpathmoveto{\pgfqpoint{2.033336in}{1.939288in}}%
\pgfpathcurveto{\pgfqpoint{2.041572in}{1.939288in}}{\pgfqpoint{2.049472in}{1.942560in}}{\pgfqpoint{2.055296in}{1.948384in}}%
\pgfpathcurveto{\pgfqpoint{2.061120in}{1.954208in}}{\pgfqpoint{2.064393in}{1.962108in}}{\pgfqpoint{2.064393in}{1.970344in}}%
\pgfpathcurveto{\pgfqpoint{2.064393in}{1.978581in}}{\pgfqpoint{2.061120in}{1.986481in}}{\pgfqpoint{2.055296in}{1.992304in}}%
\pgfpathcurveto{\pgfqpoint{2.049472in}{1.998128in}}{\pgfqpoint{2.041572in}{2.001401in}}{\pgfqpoint{2.033336in}{2.001401in}}%
\pgfpathcurveto{\pgfqpoint{2.025100in}{2.001401in}}{\pgfqpoint{2.017200in}{1.998128in}}{\pgfqpoint{2.011376in}{1.992304in}}%
\pgfpathcurveto{\pgfqpoint{2.005552in}{1.986481in}}{\pgfqpoint{2.002280in}{1.978581in}}{\pgfqpoint{2.002280in}{1.970344in}}%
\pgfpathcurveto{\pgfqpoint{2.002280in}{1.962108in}}{\pgfqpoint{2.005552in}{1.954208in}}{\pgfqpoint{2.011376in}{1.948384in}}%
\pgfpathcurveto{\pgfqpoint{2.017200in}{1.942560in}}{\pgfqpoint{2.025100in}{1.939288in}}{\pgfqpoint{2.033336in}{1.939288in}}%
\pgfpathclose%
\pgfusepath{stroke,fill}%
\end{pgfscope}%
\begin{pgfscope}%
\pgfpathrectangle{\pgfqpoint{0.100000in}{0.212622in}}{\pgfqpoint{3.696000in}{3.696000in}}%
\pgfusepath{clip}%
\pgfsetbuttcap%
\pgfsetroundjoin%
\definecolor{currentfill}{rgb}{0.121569,0.466667,0.705882}%
\pgfsetfillcolor{currentfill}%
\pgfsetfillopacity{0.469985}%
\pgfsetlinewidth{1.003750pt}%
\definecolor{currentstroke}{rgb}{0.121569,0.466667,0.705882}%
\pgfsetstrokecolor{currentstroke}%
\pgfsetstrokeopacity{0.469985}%
\pgfsetdash{}{0pt}%
\pgfpathmoveto{\pgfqpoint{1.394375in}{1.729325in}}%
\pgfpathcurveto{\pgfqpoint{1.402612in}{1.729325in}}{\pgfqpoint{1.410512in}{1.732597in}}{\pgfqpoint{1.416336in}{1.738421in}}%
\pgfpathcurveto{\pgfqpoint{1.422160in}{1.744245in}}{\pgfqpoint{1.425432in}{1.752145in}}{\pgfqpoint{1.425432in}{1.760381in}}%
\pgfpathcurveto{\pgfqpoint{1.425432in}{1.768617in}}{\pgfqpoint{1.422160in}{1.776517in}}{\pgfqpoint{1.416336in}{1.782341in}}%
\pgfpathcurveto{\pgfqpoint{1.410512in}{1.788165in}}{\pgfqpoint{1.402612in}{1.791438in}}{\pgfqpoint{1.394375in}{1.791438in}}%
\pgfpathcurveto{\pgfqpoint{1.386139in}{1.791438in}}{\pgfqpoint{1.378239in}{1.788165in}}{\pgfqpoint{1.372415in}{1.782341in}}%
\pgfpathcurveto{\pgfqpoint{1.366591in}{1.776517in}}{\pgfqpoint{1.363319in}{1.768617in}}{\pgfqpoint{1.363319in}{1.760381in}}%
\pgfpathcurveto{\pgfqpoint{1.363319in}{1.752145in}}{\pgfqpoint{1.366591in}{1.744245in}}{\pgfqpoint{1.372415in}{1.738421in}}%
\pgfpathcurveto{\pgfqpoint{1.378239in}{1.732597in}}{\pgfqpoint{1.386139in}{1.729325in}}{\pgfqpoint{1.394375in}{1.729325in}}%
\pgfpathclose%
\pgfusepath{stroke,fill}%
\end{pgfscope}%
\begin{pgfscope}%
\pgfpathrectangle{\pgfqpoint{0.100000in}{0.212622in}}{\pgfqpoint{3.696000in}{3.696000in}}%
\pgfusepath{clip}%
\pgfsetbuttcap%
\pgfsetroundjoin%
\definecolor{currentfill}{rgb}{0.121569,0.466667,0.705882}%
\pgfsetfillcolor{currentfill}%
\pgfsetfillopacity{0.470744}%
\pgfsetlinewidth{1.003750pt}%
\definecolor{currentstroke}{rgb}{0.121569,0.466667,0.705882}%
\pgfsetstrokecolor{currentstroke}%
\pgfsetstrokeopacity{0.470744}%
\pgfsetdash{}{0pt}%
\pgfpathmoveto{\pgfqpoint{2.034106in}{1.937739in}}%
\pgfpathcurveto{\pgfqpoint{2.042342in}{1.937739in}}{\pgfqpoint{2.050242in}{1.941012in}}{\pgfqpoint{2.056066in}{1.946835in}}%
\pgfpathcurveto{\pgfqpoint{2.061890in}{1.952659in}}{\pgfqpoint{2.065162in}{1.960559in}}{\pgfqpoint{2.065162in}{1.968796in}}%
\pgfpathcurveto{\pgfqpoint{2.065162in}{1.977032in}}{\pgfqpoint{2.061890in}{1.984932in}}{\pgfqpoint{2.056066in}{1.990756in}}%
\pgfpathcurveto{\pgfqpoint{2.050242in}{1.996580in}}{\pgfqpoint{2.042342in}{1.999852in}}{\pgfqpoint{2.034106in}{1.999852in}}%
\pgfpathcurveto{\pgfqpoint{2.025870in}{1.999852in}}{\pgfqpoint{2.017970in}{1.996580in}}{\pgfqpoint{2.012146in}{1.990756in}}%
\pgfpathcurveto{\pgfqpoint{2.006322in}{1.984932in}}{\pgfqpoint{2.003049in}{1.977032in}}{\pgfqpoint{2.003049in}{1.968796in}}%
\pgfpathcurveto{\pgfqpoint{2.003049in}{1.960559in}}{\pgfqpoint{2.006322in}{1.952659in}}{\pgfqpoint{2.012146in}{1.946835in}}%
\pgfpathcurveto{\pgfqpoint{2.017970in}{1.941012in}}{\pgfqpoint{2.025870in}{1.937739in}}{\pgfqpoint{2.034106in}{1.937739in}}%
\pgfpathclose%
\pgfusepath{stroke,fill}%
\end{pgfscope}%
\begin{pgfscope}%
\pgfpathrectangle{\pgfqpoint{0.100000in}{0.212622in}}{\pgfqpoint{3.696000in}{3.696000in}}%
\pgfusepath{clip}%
\pgfsetbuttcap%
\pgfsetroundjoin%
\definecolor{currentfill}{rgb}{0.121569,0.466667,0.705882}%
\pgfsetfillcolor{currentfill}%
\pgfsetfillopacity{0.471691}%
\pgfsetlinewidth{1.003750pt}%
\definecolor{currentstroke}{rgb}{0.121569,0.466667,0.705882}%
\pgfsetstrokecolor{currentstroke}%
\pgfsetstrokeopacity{0.471691}%
\pgfsetdash{}{0pt}%
\pgfpathmoveto{\pgfqpoint{1.388344in}{1.726349in}}%
\pgfpathcurveto{\pgfqpoint{1.396580in}{1.726349in}}{\pgfqpoint{1.404480in}{1.729621in}}{\pgfqpoint{1.410304in}{1.735445in}}%
\pgfpathcurveto{\pgfqpoint{1.416128in}{1.741269in}}{\pgfqpoint{1.419401in}{1.749169in}}{\pgfqpoint{1.419401in}{1.757405in}}%
\pgfpathcurveto{\pgfqpoint{1.419401in}{1.765641in}}{\pgfqpoint{1.416128in}{1.773541in}}{\pgfqpoint{1.410304in}{1.779365in}}%
\pgfpathcurveto{\pgfqpoint{1.404480in}{1.785189in}}{\pgfqpoint{1.396580in}{1.788462in}}{\pgfqpoint{1.388344in}{1.788462in}}%
\pgfpathcurveto{\pgfqpoint{1.380108in}{1.788462in}}{\pgfqpoint{1.372208in}{1.785189in}}{\pgfqpoint{1.366384in}{1.779365in}}%
\pgfpathcurveto{\pgfqpoint{1.360560in}{1.773541in}}{\pgfqpoint{1.357288in}{1.765641in}}{\pgfqpoint{1.357288in}{1.757405in}}%
\pgfpathcurveto{\pgfqpoint{1.357288in}{1.749169in}}{\pgfqpoint{1.360560in}{1.741269in}}{\pgfqpoint{1.366384in}{1.735445in}}%
\pgfpathcurveto{\pgfqpoint{1.372208in}{1.729621in}}{\pgfqpoint{1.380108in}{1.726349in}}{\pgfqpoint{1.388344in}{1.726349in}}%
\pgfpathclose%
\pgfusepath{stroke,fill}%
\end{pgfscope}%
\begin{pgfscope}%
\pgfpathrectangle{\pgfqpoint{0.100000in}{0.212622in}}{\pgfqpoint{3.696000in}{3.696000in}}%
\pgfusepath{clip}%
\pgfsetbuttcap%
\pgfsetroundjoin%
\definecolor{currentfill}{rgb}{0.121569,0.466667,0.705882}%
\pgfsetfillcolor{currentfill}%
\pgfsetfillopacity{0.472913}%
\pgfsetlinewidth{1.003750pt}%
\definecolor{currentstroke}{rgb}{0.121569,0.466667,0.705882}%
\pgfsetstrokecolor{currentstroke}%
\pgfsetstrokeopacity{0.472913}%
\pgfsetdash{}{0pt}%
\pgfpathmoveto{\pgfqpoint{1.384185in}{1.723918in}}%
\pgfpathcurveto{\pgfqpoint{1.392421in}{1.723918in}}{\pgfqpoint{1.400321in}{1.727190in}}{\pgfqpoint{1.406145in}{1.733014in}}%
\pgfpathcurveto{\pgfqpoint{1.411969in}{1.738838in}}{\pgfqpoint{1.415241in}{1.746738in}}{\pgfqpoint{1.415241in}{1.754974in}}%
\pgfpathcurveto{\pgfqpoint{1.415241in}{1.763210in}}{\pgfqpoint{1.411969in}{1.771110in}}{\pgfqpoint{1.406145in}{1.776934in}}%
\pgfpathcurveto{\pgfqpoint{1.400321in}{1.782758in}}{\pgfqpoint{1.392421in}{1.786031in}}{\pgfqpoint{1.384185in}{1.786031in}}%
\pgfpathcurveto{\pgfqpoint{1.375948in}{1.786031in}}{\pgfqpoint{1.368048in}{1.782758in}}{\pgfqpoint{1.362224in}{1.776934in}}%
\pgfpathcurveto{\pgfqpoint{1.356400in}{1.771110in}}{\pgfqpoint{1.353128in}{1.763210in}}{\pgfqpoint{1.353128in}{1.754974in}}%
\pgfpathcurveto{\pgfqpoint{1.353128in}{1.746738in}}{\pgfqpoint{1.356400in}{1.738838in}}{\pgfqpoint{1.362224in}{1.733014in}}%
\pgfpathcurveto{\pgfqpoint{1.368048in}{1.727190in}}{\pgfqpoint{1.375948in}{1.723918in}}{\pgfqpoint{1.384185in}{1.723918in}}%
\pgfpathclose%
\pgfusepath{stroke,fill}%
\end{pgfscope}%
\begin{pgfscope}%
\pgfpathrectangle{\pgfqpoint{0.100000in}{0.212622in}}{\pgfqpoint{3.696000in}{3.696000in}}%
\pgfusepath{clip}%
\pgfsetbuttcap%
\pgfsetroundjoin%
\definecolor{currentfill}{rgb}{0.121569,0.466667,0.705882}%
\pgfsetfillcolor{currentfill}%
\pgfsetfillopacity{0.473104}%
\pgfsetlinewidth{1.003750pt}%
\definecolor{currentstroke}{rgb}{0.121569,0.466667,0.705882}%
\pgfsetstrokecolor{currentstroke}%
\pgfsetstrokeopacity{0.473104}%
\pgfsetdash{}{0pt}%
\pgfpathmoveto{\pgfqpoint{2.035128in}{1.936838in}}%
\pgfpathcurveto{\pgfqpoint{2.043365in}{1.936838in}}{\pgfqpoint{2.051265in}{1.940110in}}{\pgfqpoint{2.057089in}{1.945934in}}%
\pgfpathcurveto{\pgfqpoint{2.062913in}{1.951758in}}{\pgfqpoint{2.066185in}{1.959658in}}{\pgfqpoint{2.066185in}{1.967895in}}%
\pgfpathcurveto{\pgfqpoint{2.066185in}{1.976131in}}{\pgfqpoint{2.062913in}{1.984031in}}{\pgfqpoint{2.057089in}{1.989855in}}%
\pgfpathcurveto{\pgfqpoint{2.051265in}{1.995679in}}{\pgfqpoint{2.043365in}{1.998951in}}{\pgfqpoint{2.035128in}{1.998951in}}%
\pgfpathcurveto{\pgfqpoint{2.026892in}{1.998951in}}{\pgfqpoint{2.018992in}{1.995679in}}{\pgfqpoint{2.013168in}{1.989855in}}%
\pgfpathcurveto{\pgfqpoint{2.007344in}{1.984031in}}{\pgfqpoint{2.004072in}{1.976131in}}{\pgfqpoint{2.004072in}{1.967895in}}%
\pgfpathcurveto{\pgfqpoint{2.004072in}{1.959658in}}{\pgfqpoint{2.007344in}{1.951758in}}{\pgfqpoint{2.013168in}{1.945934in}}%
\pgfpathcurveto{\pgfqpoint{2.018992in}{1.940110in}}{\pgfqpoint{2.026892in}{1.936838in}}{\pgfqpoint{2.035128in}{1.936838in}}%
\pgfpathclose%
\pgfusepath{stroke,fill}%
\end{pgfscope}%
\begin{pgfscope}%
\pgfpathrectangle{\pgfqpoint{0.100000in}{0.212622in}}{\pgfqpoint{3.696000in}{3.696000in}}%
\pgfusepath{clip}%
\pgfsetbuttcap%
\pgfsetroundjoin%
\definecolor{currentfill}{rgb}{0.121569,0.466667,0.705882}%
\pgfsetfillcolor{currentfill}%
\pgfsetfillopacity{0.475035}%
\pgfsetlinewidth{1.003750pt}%
\definecolor{currentstroke}{rgb}{0.121569,0.466667,0.705882}%
\pgfsetstrokecolor{currentstroke}%
\pgfsetstrokeopacity{0.475035}%
\pgfsetdash{}{0pt}%
\pgfpathmoveto{\pgfqpoint{1.377304in}{1.717839in}}%
\pgfpathcurveto{\pgfqpoint{1.385540in}{1.717839in}}{\pgfqpoint{1.393440in}{1.721112in}}{\pgfqpoint{1.399264in}{1.726936in}}%
\pgfpathcurveto{\pgfqpoint{1.405088in}{1.732760in}}{\pgfqpoint{1.408361in}{1.740660in}}{\pgfqpoint{1.408361in}{1.748896in}}%
\pgfpathcurveto{\pgfqpoint{1.408361in}{1.757132in}}{\pgfqpoint{1.405088in}{1.765032in}}{\pgfqpoint{1.399264in}{1.770856in}}%
\pgfpathcurveto{\pgfqpoint{1.393440in}{1.776680in}}{\pgfqpoint{1.385540in}{1.779952in}}{\pgfqpoint{1.377304in}{1.779952in}}%
\pgfpathcurveto{\pgfqpoint{1.369068in}{1.779952in}}{\pgfqpoint{1.361168in}{1.776680in}}{\pgfqpoint{1.355344in}{1.770856in}}%
\pgfpathcurveto{\pgfqpoint{1.349520in}{1.765032in}}{\pgfqpoint{1.346248in}{1.757132in}}{\pgfqpoint{1.346248in}{1.748896in}}%
\pgfpathcurveto{\pgfqpoint{1.346248in}{1.740660in}}{\pgfqpoint{1.349520in}{1.732760in}}{\pgfqpoint{1.355344in}{1.726936in}}%
\pgfpathcurveto{\pgfqpoint{1.361168in}{1.721112in}}{\pgfqpoint{1.369068in}{1.717839in}}{\pgfqpoint{1.377304in}{1.717839in}}%
\pgfpathclose%
\pgfusepath{stroke,fill}%
\end{pgfscope}%
\begin{pgfscope}%
\pgfpathrectangle{\pgfqpoint{0.100000in}{0.212622in}}{\pgfqpoint{3.696000in}{3.696000in}}%
\pgfusepath{clip}%
\pgfsetbuttcap%
\pgfsetroundjoin%
\definecolor{currentfill}{rgb}{0.121569,0.466667,0.705882}%
\pgfsetfillcolor{currentfill}%
\pgfsetfillopacity{0.475589}%
\pgfsetlinewidth{1.003750pt}%
\definecolor{currentstroke}{rgb}{0.121569,0.466667,0.705882}%
\pgfsetstrokecolor{currentstroke}%
\pgfsetstrokeopacity{0.475589}%
\pgfsetdash{}{0pt}%
\pgfpathmoveto{\pgfqpoint{2.036130in}{1.934662in}}%
\pgfpathcurveto{\pgfqpoint{2.044366in}{1.934662in}}{\pgfqpoint{2.052266in}{1.937934in}}{\pgfqpoint{2.058090in}{1.943758in}}%
\pgfpathcurveto{\pgfqpoint{2.063914in}{1.949582in}}{\pgfqpoint{2.067186in}{1.957482in}}{\pgfqpoint{2.067186in}{1.965718in}}%
\pgfpathcurveto{\pgfqpoint{2.067186in}{1.973954in}}{\pgfqpoint{2.063914in}{1.981854in}}{\pgfqpoint{2.058090in}{1.987678in}}%
\pgfpathcurveto{\pgfqpoint{2.052266in}{1.993502in}}{\pgfqpoint{2.044366in}{1.996775in}}{\pgfqpoint{2.036130in}{1.996775in}}%
\pgfpathcurveto{\pgfqpoint{2.027894in}{1.996775in}}{\pgfqpoint{2.019994in}{1.993502in}}{\pgfqpoint{2.014170in}{1.987678in}}%
\pgfpathcurveto{\pgfqpoint{2.008346in}{1.981854in}}{\pgfqpoint{2.005073in}{1.973954in}}{\pgfqpoint{2.005073in}{1.965718in}}%
\pgfpathcurveto{\pgfqpoint{2.005073in}{1.957482in}}{\pgfqpoint{2.008346in}{1.949582in}}{\pgfqpoint{2.014170in}{1.943758in}}%
\pgfpathcurveto{\pgfqpoint{2.019994in}{1.937934in}}{\pgfqpoint{2.027894in}{1.934662in}}{\pgfqpoint{2.036130in}{1.934662in}}%
\pgfpathclose%
\pgfusepath{stroke,fill}%
\end{pgfscope}%
\begin{pgfscope}%
\pgfpathrectangle{\pgfqpoint{0.100000in}{0.212622in}}{\pgfqpoint{3.696000in}{3.696000in}}%
\pgfusepath{clip}%
\pgfsetbuttcap%
\pgfsetroundjoin%
\definecolor{currentfill}{rgb}{0.121569,0.466667,0.705882}%
\pgfsetfillcolor{currentfill}%
\pgfsetfillopacity{0.476462}%
\pgfsetlinewidth{1.003750pt}%
\definecolor{currentstroke}{rgb}{0.121569,0.466667,0.705882}%
\pgfsetstrokecolor{currentstroke}%
\pgfsetstrokeopacity{0.476462}%
\pgfsetdash{}{0pt}%
\pgfpathmoveto{\pgfqpoint{1.372239in}{1.713554in}}%
\pgfpathcurveto{\pgfqpoint{1.380476in}{1.713554in}}{\pgfqpoint{1.388376in}{1.716826in}}{\pgfqpoint{1.394200in}{1.722650in}}%
\pgfpathcurveto{\pgfqpoint{1.400024in}{1.728474in}}{\pgfqpoint{1.403296in}{1.736374in}}{\pgfqpoint{1.403296in}{1.744610in}}%
\pgfpathcurveto{\pgfqpoint{1.403296in}{1.752847in}}{\pgfqpoint{1.400024in}{1.760747in}}{\pgfqpoint{1.394200in}{1.766571in}}%
\pgfpathcurveto{\pgfqpoint{1.388376in}{1.772394in}}{\pgfqpoint{1.380476in}{1.775667in}}{\pgfqpoint{1.372239in}{1.775667in}}%
\pgfpathcurveto{\pgfqpoint{1.364003in}{1.775667in}}{\pgfqpoint{1.356103in}{1.772394in}}{\pgfqpoint{1.350279in}{1.766571in}}%
\pgfpathcurveto{\pgfqpoint{1.344455in}{1.760747in}}{\pgfqpoint{1.341183in}{1.752847in}}{\pgfqpoint{1.341183in}{1.744610in}}%
\pgfpathcurveto{\pgfqpoint{1.341183in}{1.736374in}}{\pgfqpoint{1.344455in}{1.728474in}}{\pgfqpoint{1.350279in}{1.722650in}}%
\pgfpathcurveto{\pgfqpoint{1.356103in}{1.716826in}}{\pgfqpoint{1.364003in}{1.713554in}}{\pgfqpoint{1.372239in}{1.713554in}}%
\pgfpathclose%
\pgfusepath{stroke,fill}%
\end{pgfscope}%
\begin{pgfscope}%
\pgfpathrectangle{\pgfqpoint{0.100000in}{0.212622in}}{\pgfqpoint{3.696000in}{3.696000in}}%
\pgfusepath{clip}%
\pgfsetbuttcap%
\pgfsetroundjoin%
\definecolor{currentfill}{rgb}{0.121569,0.466667,0.705882}%
\pgfsetfillcolor{currentfill}%
\pgfsetfillopacity{0.477662}%
\pgfsetlinewidth{1.003750pt}%
\definecolor{currentstroke}{rgb}{0.121569,0.466667,0.705882}%
\pgfsetstrokecolor{currentstroke}%
\pgfsetstrokeopacity{0.477662}%
\pgfsetdash{}{0pt}%
\pgfpathmoveto{\pgfqpoint{1.368422in}{1.711226in}}%
\pgfpathcurveto{\pgfqpoint{1.376659in}{1.711226in}}{\pgfqpoint{1.384559in}{1.714498in}}{\pgfqpoint{1.390383in}{1.720322in}}%
\pgfpathcurveto{\pgfqpoint{1.396207in}{1.726146in}}{\pgfqpoint{1.399479in}{1.734046in}}{\pgfqpoint{1.399479in}{1.742282in}}%
\pgfpathcurveto{\pgfqpoint{1.399479in}{1.750518in}}{\pgfqpoint{1.396207in}{1.758418in}}{\pgfqpoint{1.390383in}{1.764242in}}%
\pgfpathcurveto{\pgfqpoint{1.384559in}{1.770066in}}{\pgfqpoint{1.376659in}{1.773339in}}{\pgfqpoint{1.368422in}{1.773339in}}%
\pgfpathcurveto{\pgfqpoint{1.360186in}{1.773339in}}{\pgfqpoint{1.352286in}{1.770066in}}{\pgfqpoint{1.346462in}{1.764242in}}%
\pgfpathcurveto{\pgfqpoint{1.340638in}{1.758418in}}{\pgfqpoint{1.337366in}{1.750518in}}{\pgfqpoint{1.337366in}{1.742282in}}%
\pgfpathcurveto{\pgfqpoint{1.337366in}{1.734046in}}{\pgfqpoint{1.340638in}{1.726146in}}{\pgfqpoint{1.346462in}{1.720322in}}%
\pgfpathcurveto{\pgfqpoint{1.352286in}{1.714498in}}{\pgfqpoint{1.360186in}{1.711226in}}{\pgfqpoint{1.368422in}{1.711226in}}%
\pgfpathclose%
\pgfusepath{stroke,fill}%
\end{pgfscope}%
\begin{pgfscope}%
\pgfpathrectangle{\pgfqpoint{0.100000in}{0.212622in}}{\pgfqpoint{3.696000in}{3.696000in}}%
\pgfusepath{clip}%
\pgfsetbuttcap%
\pgfsetroundjoin%
\definecolor{currentfill}{rgb}{0.121569,0.466667,0.705882}%
\pgfsetfillcolor{currentfill}%
\pgfsetfillopacity{0.478446}%
\pgfsetlinewidth{1.003750pt}%
\definecolor{currentstroke}{rgb}{0.121569,0.466667,0.705882}%
\pgfsetstrokecolor{currentstroke}%
\pgfsetstrokeopacity{0.478446}%
\pgfsetdash{}{0pt}%
\pgfpathmoveto{\pgfqpoint{2.037059in}{1.932213in}}%
\pgfpathcurveto{\pgfqpoint{2.045296in}{1.932213in}}{\pgfqpoint{2.053196in}{1.935485in}}{\pgfqpoint{2.059020in}{1.941309in}}%
\pgfpathcurveto{\pgfqpoint{2.064843in}{1.947133in}}{\pgfqpoint{2.068116in}{1.955033in}}{\pgfqpoint{2.068116in}{1.963269in}}%
\pgfpathcurveto{\pgfqpoint{2.068116in}{1.971506in}}{\pgfqpoint{2.064843in}{1.979406in}}{\pgfqpoint{2.059020in}{1.985230in}}%
\pgfpathcurveto{\pgfqpoint{2.053196in}{1.991053in}}{\pgfqpoint{2.045296in}{1.994326in}}{\pgfqpoint{2.037059in}{1.994326in}}%
\pgfpathcurveto{\pgfqpoint{2.028823in}{1.994326in}}{\pgfqpoint{2.020923in}{1.991053in}}{\pgfqpoint{2.015099in}{1.985230in}}%
\pgfpathcurveto{\pgfqpoint{2.009275in}{1.979406in}}{\pgfqpoint{2.006003in}{1.971506in}}{\pgfqpoint{2.006003in}{1.963269in}}%
\pgfpathcurveto{\pgfqpoint{2.006003in}{1.955033in}}{\pgfqpoint{2.009275in}{1.947133in}}{\pgfqpoint{2.015099in}{1.941309in}}%
\pgfpathcurveto{\pgfqpoint{2.020923in}{1.935485in}}{\pgfqpoint{2.028823in}{1.932213in}}{\pgfqpoint{2.037059in}{1.932213in}}%
\pgfpathclose%
\pgfusepath{stroke,fill}%
\end{pgfscope}%
\begin{pgfscope}%
\pgfpathrectangle{\pgfqpoint{0.100000in}{0.212622in}}{\pgfqpoint{3.696000in}{3.696000in}}%
\pgfusepath{clip}%
\pgfsetbuttcap%
\pgfsetroundjoin%
\definecolor{currentfill}{rgb}{0.121569,0.466667,0.705882}%
\pgfsetfillcolor{currentfill}%
\pgfsetfillopacity{0.478505}%
\pgfsetlinewidth{1.003750pt}%
\definecolor{currentstroke}{rgb}{0.121569,0.466667,0.705882}%
\pgfsetstrokecolor{currentstroke}%
\pgfsetstrokeopacity{0.478505}%
\pgfsetdash{}{0pt}%
\pgfpathmoveto{\pgfqpoint{1.365606in}{1.708483in}}%
\pgfpathcurveto{\pgfqpoint{1.373842in}{1.708483in}}{\pgfqpoint{1.381742in}{1.711755in}}{\pgfqpoint{1.387566in}{1.717579in}}%
\pgfpathcurveto{\pgfqpoint{1.393390in}{1.723403in}}{\pgfqpoint{1.396662in}{1.731303in}}{\pgfqpoint{1.396662in}{1.739539in}}%
\pgfpathcurveto{\pgfqpoint{1.396662in}{1.747775in}}{\pgfqpoint{1.393390in}{1.755675in}}{\pgfqpoint{1.387566in}{1.761499in}}%
\pgfpathcurveto{\pgfqpoint{1.381742in}{1.767323in}}{\pgfqpoint{1.373842in}{1.770596in}}{\pgfqpoint{1.365606in}{1.770596in}}%
\pgfpathcurveto{\pgfqpoint{1.357369in}{1.770596in}}{\pgfqpoint{1.349469in}{1.767323in}}{\pgfqpoint{1.343645in}{1.761499in}}%
\pgfpathcurveto{\pgfqpoint{1.337821in}{1.755675in}}{\pgfqpoint{1.334549in}{1.747775in}}{\pgfqpoint{1.334549in}{1.739539in}}%
\pgfpathcurveto{\pgfqpoint{1.334549in}{1.731303in}}{\pgfqpoint{1.337821in}{1.723403in}}{\pgfqpoint{1.343645in}{1.717579in}}%
\pgfpathcurveto{\pgfqpoint{1.349469in}{1.711755in}}{\pgfqpoint{1.357369in}{1.708483in}}{\pgfqpoint{1.365606in}{1.708483in}}%
\pgfpathclose%
\pgfusepath{stroke,fill}%
\end{pgfscope}%
\begin{pgfscope}%
\pgfpathrectangle{\pgfqpoint{0.100000in}{0.212622in}}{\pgfqpoint{3.696000in}{3.696000in}}%
\pgfusepath{clip}%
\pgfsetbuttcap%
\pgfsetroundjoin%
\definecolor{currentfill}{rgb}{0.121569,0.466667,0.705882}%
\pgfsetfillcolor{currentfill}%
\pgfsetfillopacity{0.479315}%
\pgfsetlinewidth{1.003750pt}%
\definecolor{currentstroke}{rgb}{0.121569,0.466667,0.705882}%
\pgfsetstrokecolor{currentstroke}%
\pgfsetstrokeopacity{0.479315}%
\pgfsetdash{}{0pt}%
\pgfpathmoveto{\pgfqpoint{1.362901in}{1.707154in}}%
\pgfpathcurveto{\pgfqpoint{1.371137in}{1.707154in}}{\pgfqpoint{1.379037in}{1.710426in}}{\pgfqpoint{1.384861in}{1.716250in}}%
\pgfpathcurveto{\pgfqpoint{1.390685in}{1.722074in}}{\pgfqpoint{1.393957in}{1.729974in}}{\pgfqpoint{1.393957in}{1.738211in}}%
\pgfpathcurveto{\pgfqpoint{1.393957in}{1.746447in}}{\pgfqpoint{1.390685in}{1.754347in}}{\pgfqpoint{1.384861in}{1.760171in}}%
\pgfpathcurveto{\pgfqpoint{1.379037in}{1.765995in}}{\pgfqpoint{1.371137in}{1.769267in}}{\pgfqpoint{1.362901in}{1.769267in}}%
\pgfpathcurveto{\pgfqpoint{1.354664in}{1.769267in}}{\pgfqpoint{1.346764in}{1.765995in}}{\pgfqpoint{1.340940in}{1.760171in}}%
\pgfpathcurveto{\pgfqpoint{1.335116in}{1.754347in}}{\pgfqpoint{1.331844in}{1.746447in}}{\pgfqpoint{1.331844in}{1.738211in}}%
\pgfpathcurveto{\pgfqpoint{1.331844in}{1.729974in}}{\pgfqpoint{1.335116in}{1.722074in}}{\pgfqpoint{1.340940in}{1.716250in}}%
\pgfpathcurveto{\pgfqpoint{1.346764in}{1.710426in}}{\pgfqpoint{1.354664in}{1.707154in}}{\pgfqpoint{1.362901in}{1.707154in}}%
\pgfpathclose%
\pgfusepath{stroke,fill}%
\end{pgfscope}%
\begin{pgfscope}%
\pgfpathrectangle{\pgfqpoint{0.100000in}{0.212622in}}{\pgfqpoint{3.696000in}{3.696000in}}%
\pgfusepath{clip}%
\pgfsetbuttcap%
\pgfsetroundjoin%
\definecolor{currentfill}{rgb}{0.121569,0.466667,0.705882}%
\pgfsetfillcolor{currentfill}%
\pgfsetfillopacity{0.479813}%
\pgfsetlinewidth{1.003750pt}%
\definecolor{currentstroke}{rgb}{0.121569,0.466667,0.705882}%
\pgfsetstrokecolor{currentstroke}%
\pgfsetstrokeopacity{0.479813}%
\pgfsetdash{}{0pt}%
\pgfpathmoveto{\pgfqpoint{2.038490in}{1.929963in}}%
\pgfpathcurveto{\pgfqpoint{2.046727in}{1.929963in}}{\pgfqpoint{2.054627in}{1.933236in}}{\pgfqpoint{2.060451in}{1.939059in}}%
\pgfpathcurveto{\pgfqpoint{2.066275in}{1.944883in}}{\pgfqpoint{2.069547in}{1.952783in}}{\pgfqpoint{2.069547in}{1.961020in}}%
\pgfpathcurveto{\pgfqpoint{2.069547in}{1.969256in}}{\pgfqpoint{2.066275in}{1.977156in}}{\pgfqpoint{2.060451in}{1.982980in}}%
\pgfpathcurveto{\pgfqpoint{2.054627in}{1.988804in}}{\pgfqpoint{2.046727in}{1.992076in}}{\pgfqpoint{2.038490in}{1.992076in}}%
\pgfpathcurveto{\pgfqpoint{2.030254in}{1.992076in}}{\pgfqpoint{2.022354in}{1.988804in}}{\pgfqpoint{2.016530in}{1.982980in}}%
\pgfpathcurveto{\pgfqpoint{2.010706in}{1.977156in}}{\pgfqpoint{2.007434in}{1.969256in}}{\pgfqpoint{2.007434in}{1.961020in}}%
\pgfpathcurveto{\pgfqpoint{2.007434in}{1.952783in}}{\pgfqpoint{2.010706in}{1.944883in}}{\pgfqpoint{2.016530in}{1.939059in}}%
\pgfpathcurveto{\pgfqpoint{2.022354in}{1.933236in}}{\pgfqpoint{2.030254in}{1.929963in}}{\pgfqpoint{2.038490in}{1.929963in}}%
\pgfpathclose%
\pgfusepath{stroke,fill}%
\end{pgfscope}%
\begin{pgfscope}%
\pgfpathrectangle{\pgfqpoint{0.100000in}{0.212622in}}{\pgfqpoint{3.696000in}{3.696000in}}%
\pgfusepath{clip}%
\pgfsetbuttcap%
\pgfsetroundjoin%
\definecolor{currentfill}{rgb}{0.121569,0.466667,0.705882}%
\pgfsetfillcolor{currentfill}%
\pgfsetfillopacity{0.480787}%
\pgfsetlinewidth{1.003750pt}%
\definecolor{currentstroke}{rgb}{0.121569,0.466667,0.705882}%
\pgfsetstrokecolor{currentstroke}%
\pgfsetstrokeopacity{0.480787}%
\pgfsetdash{}{0pt}%
\pgfpathmoveto{\pgfqpoint{1.358011in}{1.704680in}}%
\pgfpathcurveto{\pgfqpoint{1.366247in}{1.704680in}}{\pgfqpoint{1.374147in}{1.707952in}}{\pgfqpoint{1.379971in}{1.713776in}}%
\pgfpathcurveto{\pgfqpoint{1.385795in}{1.719600in}}{\pgfqpoint{1.389068in}{1.727500in}}{\pgfqpoint{1.389068in}{1.735736in}}%
\pgfpathcurveto{\pgfqpoint{1.389068in}{1.743973in}}{\pgfqpoint{1.385795in}{1.751873in}}{\pgfqpoint{1.379971in}{1.757697in}}%
\pgfpathcurveto{\pgfqpoint{1.374147in}{1.763521in}}{\pgfqpoint{1.366247in}{1.766793in}}{\pgfqpoint{1.358011in}{1.766793in}}%
\pgfpathcurveto{\pgfqpoint{1.349775in}{1.766793in}}{\pgfqpoint{1.341875in}{1.763521in}}{\pgfqpoint{1.336051in}{1.757697in}}%
\pgfpathcurveto{\pgfqpoint{1.330227in}{1.751873in}}{\pgfqpoint{1.326955in}{1.743973in}}{\pgfqpoint{1.326955in}{1.735736in}}%
\pgfpathcurveto{\pgfqpoint{1.326955in}{1.727500in}}{\pgfqpoint{1.330227in}{1.719600in}}{\pgfqpoint{1.336051in}{1.713776in}}%
\pgfpathcurveto{\pgfqpoint{1.341875in}{1.707952in}}{\pgfqpoint{1.349775in}{1.704680in}}{\pgfqpoint{1.358011in}{1.704680in}}%
\pgfpathclose%
\pgfusepath{stroke,fill}%
\end{pgfscope}%
\begin{pgfscope}%
\pgfpathrectangle{\pgfqpoint{0.100000in}{0.212622in}}{\pgfqpoint{3.696000in}{3.696000in}}%
\pgfusepath{clip}%
\pgfsetbuttcap%
\pgfsetroundjoin%
\definecolor{currentfill}{rgb}{0.121569,0.466667,0.705882}%
\pgfsetfillcolor{currentfill}%
\pgfsetfillopacity{0.481835}%
\pgfsetlinewidth{1.003750pt}%
\definecolor{currentstroke}{rgb}{0.121569,0.466667,0.705882}%
\pgfsetstrokecolor{currentstroke}%
\pgfsetstrokeopacity{0.481835}%
\pgfsetdash{}{0pt}%
\pgfpathmoveto{\pgfqpoint{2.039437in}{1.928920in}}%
\pgfpathcurveto{\pgfqpoint{2.047674in}{1.928920in}}{\pgfqpoint{2.055574in}{1.932193in}}{\pgfqpoint{2.061398in}{1.938017in}}%
\pgfpathcurveto{\pgfqpoint{2.067222in}{1.943840in}}{\pgfqpoint{2.070494in}{1.951741in}}{\pgfqpoint{2.070494in}{1.959977in}}%
\pgfpathcurveto{\pgfqpoint{2.070494in}{1.968213in}}{\pgfqpoint{2.067222in}{1.976113in}}{\pgfqpoint{2.061398in}{1.981937in}}%
\pgfpathcurveto{\pgfqpoint{2.055574in}{1.987761in}}{\pgfqpoint{2.047674in}{1.991033in}}{\pgfqpoint{2.039437in}{1.991033in}}%
\pgfpathcurveto{\pgfqpoint{2.031201in}{1.991033in}}{\pgfqpoint{2.023301in}{1.987761in}}{\pgfqpoint{2.017477in}{1.981937in}}%
\pgfpathcurveto{\pgfqpoint{2.011653in}{1.976113in}}{\pgfqpoint{2.008381in}{1.968213in}}{\pgfqpoint{2.008381in}{1.959977in}}%
\pgfpathcurveto{\pgfqpoint{2.008381in}{1.951741in}}{\pgfqpoint{2.011653in}{1.943840in}}{\pgfqpoint{2.017477in}{1.938017in}}%
\pgfpathcurveto{\pgfqpoint{2.023301in}{1.932193in}}{\pgfqpoint{2.031201in}{1.928920in}}{\pgfqpoint{2.039437in}{1.928920in}}%
\pgfpathclose%
\pgfusepath{stroke,fill}%
\end{pgfscope}%
\begin{pgfscope}%
\pgfpathrectangle{\pgfqpoint{0.100000in}{0.212622in}}{\pgfqpoint{3.696000in}{3.696000in}}%
\pgfusepath{clip}%
\pgfsetbuttcap%
\pgfsetroundjoin%
\definecolor{currentfill}{rgb}{0.121569,0.466667,0.705882}%
\pgfsetfillcolor{currentfill}%
\pgfsetfillopacity{0.483303}%
\pgfsetlinewidth{1.003750pt}%
\definecolor{currentstroke}{rgb}{0.121569,0.466667,0.705882}%
\pgfsetstrokecolor{currentstroke}%
\pgfsetstrokeopacity{0.483303}%
\pgfsetdash{}{0pt}%
\pgfpathmoveto{\pgfqpoint{1.350150in}{1.697623in}}%
\pgfpathcurveto{\pgfqpoint{1.358387in}{1.697623in}}{\pgfqpoint{1.366287in}{1.700895in}}{\pgfqpoint{1.372111in}{1.706719in}}%
\pgfpathcurveto{\pgfqpoint{1.377935in}{1.712543in}}{\pgfqpoint{1.381207in}{1.720443in}}{\pgfqpoint{1.381207in}{1.728679in}}%
\pgfpathcurveto{\pgfqpoint{1.381207in}{1.736916in}}{\pgfqpoint{1.377935in}{1.744816in}}{\pgfqpoint{1.372111in}{1.750640in}}%
\pgfpathcurveto{\pgfqpoint{1.366287in}{1.756464in}}{\pgfqpoint{1.358387in}{1.759736in}}{\pgfqpoint{1.350150in}{1.759736in}}%
\pgfpathcurveto{\pgfqpoint{1.341914in}{1.759736in}}{\pgfqpoint{1.334014in}{1.756464in}}{\pgfqpoint{1.328190in}{1.750640in}}%
\pgfpathcurveto{\pgfqpoint{1.322366in}{1.744816in}}{\pgfqpoint{1.319094in}{1.736916in}}{\pgfqpoint{1.319094in}{1.728679in}}%
\pgfpathcurveto{\pgfqpoint{1.319094in}{1.720443in}}{\pgfqpoint{1.322366in}{1.712543in}}{\pgfqpoint{1.328190in}{1.706719in}}%
\pgfpathcurveto{\pgfqpoint{1.334014in}{1.700895in}}{\pgfqpoint{1.341914in}{1.697623in}}{\pgfqpoint{1.350150in}{1.697623in}}%
\pgfpathclose%
\pgfusepath{stroke,fill}%
\end{pgfscope}%
\begin{pgfscope}%
\pgfpathrectangle{\pgfqpoint{0.100000in}{0.212622in}}{\pgfqpoint{3.696000in}{3.696000in}}%
\pgfusepath{clip}%
\pgfsetbuttcap%
\pgfsetroundjoin%
\definecolor{currentfill}{rgb}{0.121569,0.466667,0.705882}%
\pgfsetfillcolor{currentfill}%
\pgfsetfillopacity{0.484439}%
\pgfsetlinewidth{1.003750pt}%
\definecolor{currentstroke}{rgb}{0.121569,0.466667,0.705882}%
\pgfsetstrokecolor{currentstroke}%
\pgfsetstrokeopacity{0.484439}%
\pgfsetdash{}{0pt}%
\pgfpathmoveto{\pgfqpoint{2.040730in}{1.926177in}}%
\pgfpathcurveto{\pgfqpoint{2.048966in}{1.926177in}}{\pgfqpoint{2.056866in}{1.929449in}}{\pgfqpoint{2.062690in}{1.935273in}}%
\pgfpathcurveto{\pgfqpoint{2.068514in}{1.941097in}}{\pgfqpoint{2.071787in}{1.948997in}}{\pgfqpoint{2.071787in}{1.957233in}}%
\pgfpathcurveto{\pgfqpoint{2.071787in}{1.965469in}}{\pgfqpoint{2.068514in}{1.973369in}}{\pgfqpoint{2.062690in}{1.979193in}}%
\pgfpathcurveto{\pgfqpoint{2.056866in}{1.985017in}}{\pgfqpoint{2.048966in}{1.988290in}}{\pgfqpoint{2.040730in}{1.988290in}}%
\pgfpathcurveto{\pgfqpoint{2.032494in}{1.988290in}}{\pgfqpoint{2.024594in}{1.985017in}}{\pgfqpoint{2.018770in}{1.979193in}}%
\pgfpathcurveto{\pgfqpoint{2.012946in}{1.973369in}}{\pgfqpoint{2.009674in}{1.965469in}}{\pgfqpoint{2.009674in}{1.957233in}}%
\pgfpathcurveto{\pgfqpoint{2.009674in}{1.948997in}}{\pgfqpoint{2.012946in}{1.941097in}}{\pgfqpoint{2.018770in}{1.935273in}}%
\pgfpathcurveto{\pgfqpoint{2.024594in}{1.929449in}}{\pgfqpoint{2.032494in}{1.926177in}}{\pgfqpoint{2.040730in}{1.926177in}}%
\pgfpathclose%
\pgfusepath{stroke,fill}%
\end{pgfscope}%
\begin{pgfscope}%
\pgfpathrectangle{\pgfqpoint{0.100000in}{0.212622in}}{\pgfqpoint{3.696000in}{3.696000in}}%
\pgfusepath{clip}%
\pgfsetbuttcap%
\pgfsetroundjoin%
\definecolor{currentfill}{rgb}{0.121569,0.466667,0.705882}%
\pgfsetfillcolor{currentfill}%
\pgfsetfillopacity{0.484905}%
\pgfsetlinewidth{1.003750pt}%
\definecolor{currentstroke}{rgb}{0.121569,0.466667,0.705882}%
\pgfsetstrokecolor{currentstroke}%
\pgfsetstrokeopacity{0.484905}%
\pgfsetdash{}{0pt}%
\pgfpathmoveto{\pgfqpoint{1.344845in}{1.693317in}}%
\pgfpathcurveto{\pgfqpoint{1.353081in}{1.693317in}}{\pgfqpoint{1.360981in}{1.696590in}}{\pgfqpoint{1.366805in}{1.702414in}}%
\pgfpathcurveto{\pgfqpoint{1.372629in}{1.708237in}}{\pgfqpoint{1.375901in}{1.716138in}}{\pgfqpoint{1.375901in}{1.724374in}}%
\pgfpathcurveto{\pgfqpoint{1.375901in}{1.732610in}}{\pgfqpoint{1.372629in}{1.740510in}}{\pgfqpoint{1.366805in}{1.746334in}}%
\pgfpathcurveto{\pgfqpoint{1.360981in}{1.752158in}}{\pgfqpoint{1.353081in}{1.755430in}}{\pgfqpoint{1.344845in}{1.755430in}}%
\pgfpathcurveto{\pgfqpoint{1.336608in}{1.755430in}}{\pgfqpoint{1.328708in}{1.752158in}}{\pgfqpoint{1.322884in}{1.746334in}}%
\pgfpathcurveto{\pgfqpoint{1.317061in}{1.740510in}}{\pgfqpoint{1.313788in}{1.732610in}}{\pgfqpoint{1.313788in}{1.724374in}}%
\pgfpathcurveto{\pgfqpoint{1.313788in}{1.716138in}}{\pgfqpoint{1.317061in}{1.708237in}}{\pgfqpoint{1.322884in}{1.702414in}}%
\pgfpathcurveto{\pgfqpoint{1.328708in}{1.696590in}}{\pgfqpoint{1.336608in}{1.693317in}}{\pgfqpoint{1.344845in}{1.693317in}}%
\pgfpathclose%
\pgfusepath{stroke,fill}%
\end{pgfscope}%
\begin{pgfscope}%
\pgfpathrectangle{\pgfqpoint{0.100000in}{0.212622in}}{\pgfqpoint{3.696000in}{3.696000in}}%
\pgfusepath{clip}%
\pgfsetbuttcap%
\pgfsetroundjoin%
\definecolor{currentfill}{rgb}{0.121569,0.466667,0.705882}%
\pgfsetfillcolor{currentfill}%
\pgfsetfillopacity{0.486244}%
\pgfsetlinewidth{1.003750pt}%
\definecolor{currentstroke}{rgb}{0.121569,0.466667,0.705882}%
\pgfsetstrokecolor{currentstroke}%
\pgfsetstrokeopacity{0.486244}%
\pgfsetdash{}{0pt}%
\pgfpathmoveto{\pgfqpoint{1.340577in}{1.691077in}}%
\pgfpathcurveto{\pgfqpoint{1.348813in}{1.691077in}}{\pgfqpoint{1.356713in}{1.694350in}}{\pgfqpoint{1.362537in}{1.700174in}}%
\pgfpathcurveto{\pgfqpoint{1.368361in}{1.705997in}}{\pgfqpoint{1.371633in}{1.713898in}}{\pgfqpoint{1.371633in}{1.722134in}}%
\pgfpathcurveto{\pgfqpoint{1.371633in}{1.730370in}}{\pgfqpoint{1.368361in}{1.738270in}}{\pgfqpoint{1.362537in}{1.744094in}}%
\pgfpathcurveto{\pgfqpoint{1.356713in}{1.749918in}}{\pgfqpoint{1.348813in}{1.753190in}}{\pgfqpoint{1.340577in}{1.753190in}}%
\pgfpathcurveto{\pgfqpoint{1.332341in}{1.753190in}}{\pgfqpoint{1.324441in}{1.749918in}}{\pgfqpoint{1.318617in}{1.744094in}}%
\pgfpathcurveto{\pgfqpoint{1.312793in}{1.738270in}}{\pgfqpoint{1.309520in}{1.730370in}}{\pgfqpoint{1.309520in}{1.722134in}}%
\pgfpathcurveto{\pgfqpoint{1.309520in}{1.713898in}}{\pgfqpoint{1.312793in}{1.705997in}}{\pgfqpoint{1.318617in}{1.700174in}}%
\pgfpathcurveto{\pgfqpoint{1.324441in}{1.694350in}}{\pgfqpoint{1.332341in}{1.691077in}}{\pgfqpoint{1.340577in}{1.691077in}}%
\pgfpathclose%
\pgfusepath{stroke,fill}%
\end{pgfscope}%
\begin{pgfscope}%
\pgfpathrectangle{\pgfqpoint{0.100000in}{0.212622in}}{\pgfqpoint{3.696000in}{3.696000in}}%
\pgfusepath{clip}%
\pgfsetbuttcap%
\pgfsetroundjoin%
\definecolor{currentfill}{rgb}{0.121569,0.466667,0.705882}%
\pgfsetfillcolor{currentfill}%
\pgfsetfillopacity{0.487490}%
\pgfsetlinewidth{1.003750pt}%
\definecolor{currentstroke}{rgb}{0.121569,0.466667,0.705882}%
\pgfsetstrokecolor{currentstroke}%
\pgfsetstrokeopacity{0.487490}%
\pgfsetdash{}{0pt}%
\pgfpathmoveto{\pgfqpoint{2.042335in}{1.924496in}}%
\pgfpathcurveto{\pgfqpoint{2.050571in}{1.924496in}}{\pgfqpoint{2.058471in}{1.927768in}}{\pgfqpoint{2.064295in}{1.933592in}}%
\pgfpathcurveto{\pgfqpoint{2.070119in}{1.939416in}}{\pgfqpoint{2.073391in}{1.947316in}}{\pgfqpoint{2.073391in}{1.955552in}}%
\pgfpathcurveto{\pgfqpoint{2.073391in}{1.963789in}}{\pgfqpoint{2.070119in}{1.971689in}}{\pgfqpoint{2.064295in}{1.977513in}}%
\pgfpathcurveto{\pgfqpoint{2.058471in}{1.983337in}}{\pgfqpoint{2.050571in}{1.986609in}}{\pgfqpoint{2.042335in}{1.986609in}}%
\pgfpathcurveto{\pgfqpoint{2.034099in}{1.986609in}}{\pgfqpoint{2.026199in}{1.983337in}}{\pgfqpoint{2.020375in}{1.977513in}}%
\pgfpathcurveto{\pgfqpoint{2.014551in}{1.971689in}}{\pgfqpoint{2.011278in}{1.963789in}}{\pgfqpoint{2.011278in}{1.955552in}}%
\pgfpathcurveto{\pgfqpoint{2.011278in}{1.947316in}}{\pgfqpoint{2.014551in}{1.939416in}}{\pgfqpoint{2.020375in}{1.933592in}}%
\pgfpathcurveto{\pgfqpoint{2.026199in}{1.927768in}}{\pgfqpoint{2.034099in}{1.924496in}}{\pgfqpoint{2.042335in}{1.924496in}}%
\pgfpathclose%
\pgfusepath{stroke,fill}%
\end{pgfscope}%
\begin{pgfscope}%
\pgfpathrectangle{\pgfqpoint{0.100000in}{0.212622in}}{\pgfqpoint{3.696000in}{3.696000in}}%
\pgfusepath{clip}%
\pgfsetbuttcap%
\pgfsetroundjoin%
\definecolor{currentfill}{rgb}{0.121569,0.466667,0.705882}%
\pgfsetfillcolor{currentfill}%
\pgfsetfillopacity{0.488378}%
\pgfsetlinewidth{1.003750pt}%
\definecolor{currentstroke}{rgb}{0.121569,0.466667,0.705882}%
\pgfsetstrokecolor{currentstroke}%
\pgfsetstrokeopacity{0.488378}%
\pgfsetdash{}{0pt}%
\pgfpathmoveto{\pgfqpoint{1.333566in}{1.683994in}}%
\pgfpathcurveto{\pgfqpoint{1.341802in}{1.683994in}}{\pgfqpoint{1.349702in}{1.687266in}}{\pgfqpoint{1.355526in}{1.693090in}}%
\pgfpathcurveto{\pgfqpoint{1.361350in}{1.698914in}}{\pgfqpoint{1.364622in}{1.706814in}}{\pgfqpoint{1.364622in}{1.715050in}}%
\pgfpathcurveto{\pgfqpoint{1.364622in}{1.723286in}}{\pgfqpoint{1.361350in}{1.731186in}}{\pgfqpoint{1.355526in}{1.737010in}}%
\pgfpathcurveto{\pgfqpoint{1.349702in}{1.742834in}}{\pgfqpoint{1.341802in}{1.746107in}}{\pgfqpoint{1.333566in}{1.746107in}}%
\pgfpathcurveto{\pgfqpoint{1.325329in}{1.746107in}}{\pgfqpoint{1.317429in}{1.742834in}}{\pgfqpoint{1.311605in}{1.737010in}}%
\pgfpathcurveto{\pgfqpoint{1.305781in}{1.731186in}}{\pgfqpoint{1.302509in}{1.723286in}}{\pgfqpoint{1.302509in}{1.715050in}}%
\pgfpathcurveto{\pgfqpoint{1.302509in}{1.706814in}}{\pgfqpoint{1.305781in}{1.698914in}}{\pgfqpoint{1.311605in}{1.693090in}}%
\pgfpathcurveto{\pgfqpoint{1.317429in}{1.687266in}}{\pgfqpoint{1.325329in}{1.683994in}}{\pgfqpoint{1.333566in}{1.683994in}}%
\pgfpathclose%
\pgfusepath{stroke,fill}%
\end{pgfscope}%
\begin{pgfscope}%
\pgfpathrectangle{\pgfqpoint{0.100000in}{0.212622in}}{\pgfqpoint{3.696000in}{3.696000in}}%
\pgfusepath{clip}%
\pgfsetbuttcap%
\pgfsetroundjoin%
\definecolor{currentfill}{rgb}{0.121569,0.466667,0.705882}%
\pgfsetfillcolor{currentfill}%
\pgfsetfillopacity{0.489896}%
\pgfsetlinewidth{1.003750pt}%
\definecolor{currentstroke}{rgb}{0.121569,0.466667,0.705882}%
\pgfsetstrokecolor{currentstroke}%
\pgfsetstrokeopacity{0.489896}%
\pgfsetdash{}{0pt}%
\pgfpathmoveto{\pgfqpoint{1.328178in}{1.680187in}}%
\pgfpathcurveto{\pgfqpoint{1.336414in}{1.680187in}}{\pgfqpoint{1.344314in}{1.683460in}}{\pgfqpoint{1.350138in}{1.689284in}}%
\pgfpathcurveto{\pgfqpoint{1.355962in}{1.695107in}}{\pgfqpoint{1.359234in}{1.703007in}}{\pgfqpoint{1.359234in}{1.711244in}}%
\pgfpathcurveto{\pgfqpoint{1.359234in}{1.719480in}}{\pgfqpoint{1.355962in}{1.727380in}}{\pgfqpoint{1.350138in}{1.733204in}}%
\pgfpathcurveto{\pgfqpoint{1.344314in}{1.739028in}}{\pgfqpoint{1.336414in}{1.742300in}}{\pgfqpoint{1.328178in}{1.742300in}}%
\pgfpathcurveto{\pgfqpoint{1.319942in}{1.742300in}}{\pgfqpoint{1.312042in}{1.739028in}}{\pgfqpoint{1.306218in}{1.733204in}}%
\pgfpathcurveto{\pgfqpoint{1.300394in}{1.727380in}}{\pgfqpoint{1.297121in}{1.719480in}}{\pgfqpoint{1.297121in}{1.711244in}}%
\pgfpathcurveto{\pgfqpoint{1.297121in}{1.703007in}}{\pgfqpoint{1.300394in}{1.695107in}}{\pgfqpoint{1.306218in}{1.689284in}}%
\pgfpathcurveto{\pgfqpoint{1.312042in}{1.683460in}}{\pgfqpoint{1.319942in}{1.680187in}}{\pgfqpoint{1.328178in}{1.680187in}}%
\pgfpathclose%
\pgfusepath{stroke,fill}%
\end{pgfscope}%
\begin{pgfscope}%
\pgfpathrectangle{\pgfqpoint{0.100000in}{0.212622in}}{\pgfqpoint{3.696000in}{3.696000in}}%
\pgfusepath{clip}%
\pgfsetbuttcap%
\pgfsetroundjoin%
\definecolor{currentfill}{rgb}{0.121569,0.466667,0.705882}%
\pgfsetfillcolor{currentfill}%
\pgfsetfillopacity{0.490882}%
\pgfsetlinewidth{1.003750pt}%
\definecolor{currentstroke}{rgb}{0.121569,0.466667,0.705882}%
\pgfsetstrokecolor{currentstroke}%
\pgfsetstrokeopacity{0.490882}%
\pgfsetdash{}{0pt}%
\pgfpathmoveto{\pgfqpoint{1.324939in}{1.678984in}}%
\pgfpathcurveto{\pgfqpoint{1.333175in}{1.678984in}}{\pgfqpoint{1.341075in}{1.682256in}}{\pgfqpoint{1.346899in}{1.688080in}}%
\pgfpathcurveto{\pgfqpoint{1.352723in}{1.693904in}}{\pgfqpoint{1.355996in}{1.701804in}}{\pgfqpoint{1.355996in}{1.710040in}}%
\pgfpathcurveto{\pgfqpoint{1.355996in}{1.718277in}}{\pgfqpoint{1.352723in}{1.726177in}}{\pgfqpoint{1.346899in}{1.732001in}}%
\pgfpathcurveto{\pgfqpoint{1.341075in}{1.737824in}}{\pgfqpoint{1.333175in}{1.741097in}}{\pgfqpoint{1.324939in}{1.741097in}}%
\pgfpathcurveto{\pgfqpoint{1.316703in}{1.741097in}}{\pgfqpoint{1.308803in}{1.737824in}}{\pgfqpoint{1.302979in}{1.732001in}}%
\pgfpathcurveto{\pgfqpoint{1.297155in}{1.726177in}}{\pgfqpoint{1.293883in}{1.718277in}}{\pgfqpoint{1.293883in}{1.710040in}}%
\pgfpathcurveto{\pgfqpoint{1.293883in}{1.701804in}}{\pgfqpoint{1.297155in}{1.693904in}}{\pgfqpoint{1.302979in}{1.688080in}}%
\pgfpathcurveto{\pgfqpoint{1.308803in}{1.682256in}}{\pgfqpoint{1.316703in}{1.678984in}}{\pgfqpoint{1.324939in}{1.678984in}}%
\pgfpathclose%
\pgfusepath{stroke,fill}%
\end{pgfscope}%
\begin{pgfscope}%
\pgfpathrectangle{\pgfqpoint{0.100000in}{0.212622in}}{\pgfqpoint{3.696000in}{3.696000in}}%
\pgfusepath{clip}%
\pgfsetbuttcap%
\pgfsetroundjoin%
\definecolor{currentfill}{rgb}{0.121569,0.466667,0.705882}%
\pgfsetfillcolor{currentfill}%
\pgfsetfillopacity{0.491380}%
\pgfsetlinewidth{1.003750pt}%
\definecolor{currentstroke}{rgb}{0.121569,0.466667,0.705882}%
\pgfsetstrokecolor{currentstroke}%
\pgfsetstrokeopacity{0.491380}%
\pgfsetdash{}{0pt}%
\pgfpathmoveto{\pgfqpoint{2.043923in}{1.923926in}}%
\pgfpathcurveto{\pgfqpoint{2.052160in}{1.923926in}}{\pgfqpoint{2.060060in}{1.927198in}}{\pgfqpoint{2.065884in}{1.933022in}}%
\pgfpathcurveto{\pgfqpoint{2.071708in}{1.938846in}}{\pgfqpoint{2.074980in}{1.946746in}}{\pgfqpoint{2.074980in}{1.954982in}}%
\pgfpathcurveto{\pgfqpoint{2.074980in}{1.963219in}}{\pgfqpoint{2.071708in}{1.971119in}}{\pgfqpoint{2.065884in}{1.976943in}}%
\pgfpathcurveto{\pgfqpoint{2.060060in}{1.982766in}}{\pgfqpoint{2.052160in}{1.986039in}}{\pgfqpoint{2.043923in}{1.986039in}}%
\pgfpathcurveto{\pgfqpoint{2.035687in}{1.986039in}}{\pgfqpoint{2.027787in}{1.982766in}}{\pgfqpoint{2.021963in}{1.976943in}}%
\pgfpathcurveto{\pgfqpoint{2.016139in}{1.971119in}}{\pgfqpoint{2.012867in}{1.963219in}}{\pgfqpoint{2.012867in}{1.954982in}}%
\pgfpathcurveto{\pgfqpoint{2.012867in}{1.946746in}}{\pgfqpoint{2.016139in}{1.938846in}}{\pgfqpoint{2.021963in}{1.933022in}}%
\pgfpathcurveto{\pgfqpoint{2.027787in}{1.927198in}}{\pgfqpoint{2.035687in}{1.923926in}}{\pgfqpoint{2.043923in}{1.923926in}}%
\pgfpathclose%
\pgfusepath{stroke,fill}%
\end{pgfscope}%
\begin{pgfscope}%
\pgfpathrectangle{\pgfqpoint{0.100000in}{0.212622in}}{\pgfqpoint{3.696000in}{3.696000in}}%
\pgfusepath{clip}%
\pgfsetbuttcap%
\pgfsetroundjoin%
\definecolor{currentfill}{rgb}{0.121569,0.466667,0.705882}%
\pgfsetfillcolor{currentfill}%
\pgfsetfillopacity{0.492454}%
\pgfsetlinewidth{1.003750pt}%
\definecolor{currentstroke}{rgb}{0.121569,0.466667,0.705882}%
\pgfsetstrokecolor{currentstroke}%
\pgfsetstrokeopacity{0.492454}%
\pgfsetdash{}{0pt}%
\pgfpathmoveto{\pgfqpoint{1.319609in}{1.674489in}}%
\pgfpathcurveto{\pgfqpoint{1.327845in}{1.674489in}}{\pgfqpoint{1.335745in}{1.677762in}}{\pgfqpoint{1.341569in}{1.683586in}}%
\pgfpathcurveto{\pgfqpoint{1.347393in}{1.689410in}}{\pgfqpoint{1.350666in}{1.697310in}}{\pgfqpoint{1.350666in}{1.705546in}}%
\pgfpathcurveto{\pgfqpoint{1.350666in}{1.713782in}}{\pgfqpoint{1.347393in}{1.721682in}}{\pgfqpoint{1.341569in}{1.727506in}}%
\pgfpathcurveto{\pgfqpoint{1.335745in}{1.733330in}}{\pgfqpoint{1.327845in}{1.736602in}}{\pgfqpoint{1.319609in}{1.736602in}}%
\pgfpathcurveto{\pgfqpoint{1.311373in}{1.736602in}}{\pgfqpoint{1.303473in}{1.733330in}}{\pgfqpoint{1.297649in}{1.727506in}}%
\pgfpathcurveto{\pgfqpoint{1.291825in}{1.721682in}}{\pgfqpoint{1.288553in}{1.713782in}}{\pgfqpoint{1.288553in}{1.705546in}}%
\pgfpathcurveto{\pgfqpoint{1.288553in}{1.697310in}}{\pgfqpoint{1.291825in}{1.689410in}}{\pgfqpoint{1.297649in}{1.683586in}}%
\pgfpathcurveto{\pgfqpoint{1.303473in}{1.677762in}}{\pgfqpoint{1.311373in}{1.674489in}}{\pgfqpoint{1.319609in}{1.674489in}}%
\pgfpathclose%
\pgfusepath{stroke,fill}%
\end{pgfscope}%
\begin{pgfscope}%
\pgfpathrectangle{\pgfqpoint{0.100000in}{0.212622in}}{\pgfqpoint{3.696000in}{3.696000in}}%
\pgfusepath{clip}%
\pgfsetbuttcap%
\pgfsetroundjoin%
\definecolor{currentfill}{rgb}{0.121569,0.466667,0.705882}%
\pgfsetfillcolor{currentfill}%
\pgfsetfillopacity{0.493784}%
\pgfsetlinewidth{1.003750pt}%
\definecolor{currentstroke}{rgb}{0.121569,0.466667,0.705882}%
\pgfsetstrokecolor{currentstroke}%
\pgfsetstrokeopacity{0.493784}%
\pgfsetdash{}{0pt}%
\pgfpathmoveto{\pgfqpoint{1.315993in}{1.673457in}}%
\pgfpathcurveto{\pgfqpoint{1.324229in}{1.673457in}}{\pgfqpoint{1.332129in}{1.676730in}}{\pgfqpoint{1.337953in}{1.682553in}}%
\pgfpathcurveto{\pgfqpoint{1.343777in}{1.688377in}}{\pgfqpoint{1.347049in}{1.696277in}}{\pgfqpoint{1.347049in}{1.704514in}}%
\pgfpathcurveto{\pgfqpoint{1.347049in}{1.712750in}}{\pgfqpoint{1.343777in}{1.720650in}}{\pgfqpoint{1.337953in}{1.726474in}}%
\pgfpathcurveto{\pgfqpoint{1.332129in}{1.732298in}}{\pgfqpoint{1.324229in}{1.735570in}}{\pgfqpoint{1.315993in}{1.735570in}}%
\pgfpathcurveto{\pgfqpoint{1.307756in}{1.735570in}}{\pgfqpoint{1.299856in}{1.732298in}}{\pgfqpoint{1.294032in}{1.726474in}}%
\pgfpathcurveto{\pgfqpoint{1.288208in}{1.720650in}}{\pgfqpoint{1.284936in}{1.712750in}}{\pgfqpoint{1.284936in}{1.704514in}}%
\pgfpathcurveto{\pgfqpoint{1.284936in}{1.696277in}}{\pgfqpoint{1.288208in}{1.688377in}}{\pgfqpoint{1.294032in}{1.682553in}}%
\pgfpathcurveto{\pgfqpoint{1.299856in}{1.676730in}}{\pgfqpoint{1.307756in}{1.673457in}}{\pgfqpoint{1.315993in}{1.673457in}}%
\pgfpathclose%
\pgfusepath{stroke,fill}%
\end{pgfscope}%
\begin{pgfscope}%
\pgfpathrectangle{\pgfqpoint{0.100000in}{0.212622in}}{\pgfqpoint{3.696000in}{3.696000in}}%
\pgfusepath{clip}%
\pgfsetbuttcap%
\pgfsetroundjoin%
\definecolor{currentfill}{rgb}{0.121569,0.466667,0.705882}%
\pgfsetfillcolor{currentfill}%
\pgfsetfillopacity{0.495118}%
\pgfsetlinewidth{1.003750pt}%
\definecolor{currentstroke}{rgb}{0.121569,0.466667,0.705882}%
\pgfsetstrokecolor{currentstroke}%
\pgfsetstrokeopacity{0.495118}%
\pgfsetdash{}{0pt}%
\pgfpathmoveto{\pgfqpoint{2.046215in}{1.919414in}}%
\pgfpathcurveto{\pgfqpoint{2.054451in}{1.919414in}}{\pgfqpoint{2.062351in}{1.922686in}}{\pgfqpoint{2.068175in}{1.928510in}}%
\pgfpathcurveto{\pgfqpoint{2.073999in}{1.934334in}}{\pgfqpoint{2.077271in}{1.942234in}}{\pgfqpoint{2.077271in}{1.950471in}}%
\pgfpathcurveto{\pgfqpoint{2.077271in}{1.958707in}}{\pgfqpoint{2.073999in}{1.966607in}}{\pgfqpoint{2.068175in}{1.972431in}}%
\pgfpathcurveto{\pgfqpoint{2.062351in}{1.978255in}}{\pgfqpoint{2.054451in}{1.981527in}}{\pgfqpoint{2.046215in}{1.981527in}}%
\pgfpathcurveto{\pgfqpoint{2.037979in}{1.981527in}}{\pgfqpoint{2.030079in}{1.978255in}}{\pgfqpoint{2.024255in}{1.972431in}}%
\pgfpathcurveto{\pgfqpoint{2.018431in}{1.966607in}}{\pgfqpoint{2.015158in}{1.958707in}}{\pgfqpoint{2.015158in}{1.950471in}}%
\pgfpathcurveto{\pgfqpoint{2.015158in}{1.942234in}}{\pgfqpoint{2.018431in}{1.934334in}}{\pgfqpoint{2.024255in}{1.928510in}}%
\pgfpathcurveto{\pgfqpoint{2.030079in}{1.922686in}}{\pgfqpoint{2.037979in}{1.919414in}}{\pgfqpoint{2.046215in}{1.919414in}}%
\pgfpathclose%
\pgfusepath{stroke,fill}%
\end{pgfscope}%
\begin{pgfscope}%
\pgfpathrectangle{\pgfqpoint{0.100000in}{0.212622in}}{\pgfqpoint{3.696000in}{3.696000in}}%
\pgfusepath{clip}%
\pgfsetbuttcap%
\pgfsetroundjoin%
\definecolor{currentfill}{rgb}{0.121569,0.466667,0.705882}%
\pgfsetfillcolor{currentfill}%
\pgfsetfillopacity{0.495825}%
\pgfsetlinewidth{1.003750pt}%
\definecolor{currentstroke}{rgb}{0.121569,0.466667,0.705882}%
\pgfsetstrokecolor{currentstroke}%
\pgfsetstrokeopacity{0.495825}%
\pgfsetdash{}{0pt}%
\pgfpathmoveto{\pgfqpoint{1.309355in}{1.669214in}}%
\pgfpathcurveto{\pgfqpoint{1.317592in}{1.669214in}}{\pgfqpoint{1.325492in}{1.672486in}}{\pgfqpoint{1.331316in}{1.678310in}}%
\pgfpathcurveto{\pgfqpoint{1.337140in}{1.684134in}}{\pgfqpoint{1.340412in}{1.692034in}}{\pgfqpoint{1.340412in}{1.700270in}}%
\pgfpathcurveto{\pgfqpoint{1.340412in}{1.708506in}}{\pgfqpoint{1.337140in}{1.716406in}}{\pgfqpoint{1.331316in}{1.722230in}}%
\pgfpathcurveto{\pgfqpoint{1.325492in}{1.728054in}}{\pgfqpoint{1.317592in}{1.731327in}}{\pgfqpoint{1.309355in}{1.731327in}}%
\pgfpathcurveto{\pgfqpoint{1.301119in}{1.731327in}}{\pgfqpoint{1.293219in}{1.728054in}}{\pgfqpoint{1.287395in}{1.722230in}}%
\pgfpathcurveto{\pgfqpoint{1.281571in}{1.716406in}}{\pgfqpoint{1.278299in}{1.708506in}}{\pgfqpoint{1.278299in}{1.700270in}}%
\pgfpathcurveto{\pgfqpoint{1.278299in}{1.692034in}}{\pgfqpoint{1.281571in}{1.684134in}}{\pgfqpoint{1.287395in}{1.678310in}}%
\pgfpathcurveto{\pgfqpoint{1.293219in}{1.672486in}}{\pgfqpoint{1.301119in}{1.669214in}}{\pgfqpoint{1.309355in}{1.669214in}}%
\pgfpathclose%
\pgfusepath{stroke,fill}%
\end{pgfscope}%
\begin{pgfscope}%
\pgfpathrectangle{\pgfqpoint{0.100000in}{0.212622in}}{\pgfqpoint{3.696000in}{3.696000in}}%
\pgfusepath{clip}%
\pgfsetbuttcap%
\pgfsetroundjoin%
\definecolor{currentfill}{rgb}{0.121569,0.466667,0.705882}%
\pgfsetfillcolor{currentfill}%
\pgfsetfillopacity{0.499123}%
\pgfsetlinewidth{1.003750pt}%
\definecolor{currentstroke}{rgb}{0.121569,0.466667,0.705882}%
\pgfsetstrokecolor{currentstroke}%
\pgfsetstrokeopacity{0.499123}%
\pgfsetdash{}{0pt}%
\pgfpathmoveto{\pgfqpoint{1.297271in}{1.658818in}}%
\pgfpathcurveto{\pgfqpoint{1.305508in}{1.658818in}}{\pgfqpoint{1.313408in}{1.662091in}}{\pgfqpoint{1.319232in}{1.667915in}}%
\pgfpathcurveto{\pgfqpoint{1.325056in}{1.673738in}}{\pgfqpoint{1.328328in}{1.681639in}}{\pgfqpoint{1.328328in}{1.689875in}}%
\pgfpathcurveto{\pgfqpoint{1.328328in}{1.698111in}}{\pgfqpoint{1.325056in}{1.706011in}}{\pgfqpoint{1.319232in}{1.711835in}}%
\pgfpathcurveto{\pgfqpoint{1.313408in}{1.717659in}}{\pgfqpoint{1.305508in}{1.720931in}}{\pgfqpoint{1.297271in}{1.720931in}}%
\pgfpathcurveto{\pgfqpoint{1.289035in}{1.720931in}}{\pgfqpoint{1.281135in}{1.717659in}}{\pgfqpoint{1.275311in}{1.711835in}}%
\pgfpathcurveto{\pgfqpoint{1.269487in}{1.706011in}}{\pgfqpoint{1.266215in}{1.698111in}}{\pgfqpoint{1.266215in}{1.689875in}}%
\pgfpathcurveto{\pgfqpoint{1.266215in}{1.681639in}}{\pgfqpoint{1.269487in}{1.673738in}}{\pgfqpoint{1.275311in}{1.667915in}}%
\pgfpathcurveto{\pgfqpoint{1.281135in}{1.662091in}}{\pgfqpoint{1.289035in}{1.658818in}}{\pgfqpoint{1.297271in}{1.658818in}}%
\pgfpathclose%
\pgfusepath{stroke,fill}%
\end{pgfscope}%
\begin{pgfscope}%
\pgfpathrectangle{\pgfqpoint{0.100000in}{0.212622in}}{\pgfqpoint{3.696000in}{3.696000in}}%
\pgfusepath{clip}%
\pgfsetbuttcap%
\pgfsetroundjoin%
\definecolor{currentfill}{rgb}{0.121569,0.466667,0.705882}%
\pgfsetfillcolor{currentfill}%
\pgfsetfillopacity{0.499444}%
\pgfsetlinewidth{1.003750pt}%
\definecolor{currentstroke}{rgb}{0.121569,0.466667,0.705882}%
\pgfsetstrokecolor{currentstroke}%
\pgfsetstrokeopacity{0.499444}%
\pgfsetdash{}{0pt}%
\pgfpathmoveto{\pgfqpoint{2.047627in}{1.914768in}}%
\pgfpathcurveto{\pgfqpoint{2.055863in}{1.914768in}}{\pgfqpoint{2.063763in}{1.918040in}}{\pgfqpoint{2.069587in}{1.923864in}}%
\pgfpathcurveto{\pgfqpoint{2.075411in}{1.929688in}}{\pgfqpoint{2.078683in}{1.937588in}}{\pgfqpoint{2.078683in}{1.945824in}}%
\pgfpathcurveto{\pgfqpoint{2.078683in}{1.954061in}}{\pgfqpoint{2.075411in}{1.961961in}}{\pgfqpoint{2.069587in}{1.967785in}}%
\pgfpathcurveto{\pgfqpoint{2.063763in}{1.973609in}}{\pgfqpoint{2.055863in}{1.976881in}}{\pgfqpoint{2.047627in}{1.976881in}}%
\pgfpathcurveto{\pgfqpoint{2.039391in}{1.976881in}}{\pgfqpoint{2.031491in}{1.973609in}}{\pgfqpoint{2.025667in}{1.967785in}}%
\pgfpathcurveto{\pgfqpoint{2.019843in}{1.961961in}}{\pgfqpoint{2.016570in}{1.954061in}}{\pgfqpoint{2.016570in}{1.945824in}}%
\pgfpathcurveto{\pgfqpoint{2.016570in}{1.937588in}}{\pgfqpoint{2.019843in}{1.929688in}}{\pgfqpoint{2.025667in}{1.923864in}}%
\pgfpathcurveto{\pgfqpoint{2.031491in}{1.918040in}}{\pgfqpoint{2.039391in}{1.914768in}}{\pgfqpoint{2.047627in}{1.914768in}}%
\pgfpathclose%
\pgfusepath{stroke,fill}%
\end{pgfscope}%
\begin{pgfscope}%
\pgfpathrectangle{\pgfqpoint{0.100000in}{0.212622in}}{\pgfqpoint{3.696000in}{3.696000in}}%
\pgfusepath{clip}%
\pgfsetbuttcap%
\pgfsetroundjoin%
\definecolor{currentfill}{rgb}{0.121569,0.466667,0.705882}%
\pgfsetfillcolor{currentfill}%
\pgfsetfillopacity{0.502667}%
\pgfsetlinewidth{1.003750pt}%
\definecolor{currentstroke}{rgb}{0.121569,0.466667,0.705882}%
\pgfsetstrokecolor{currentstroke}%
\pgfsetstrokeopacity{0.502667}%
\pgfsetdash{}{0pt}%
\pgfpathmoveto{\pgfqpoint{1.286970in}{1.655826in}}%
\pgfpathcurveto{\pgfqpoint{1.295206in}{1.655826in}}{\pgfqpoint{1.303106in}{1.659099in}}{\pgfqpoint{1.308930in}{1.664922in}}%
\pgfpathcurveto{\pgfqpoint{1.314754in}{1.670746in}}{\pgfqpoint{1.318027in}{1.678646in}}{\pgfqpoint{1.318027in}{1.686883in}}%
\pgfpathcurveto{\pgfqpoint{1.318027in}{1.695119in}}{\pgfqpoint{1.314754in}{1.703019in}}{\pgfqpoint{1.308930in}{1.708843in}}%
\pgfpathcurveto{\pgfqpoint{1.303106in}{1.714667in}}{\pgfqpoint{1.295206in}{1.717939in}}{\pgfqpoint{1.286970in}{1.717939in}}%
\pgfpathcurveto{\pgfqpoint{1.278734in}{1.717939in}}{\pgfqpoint{1.270834in}{1.714667in}}{\pgfqpoint{1.265010in}{1.708843in}}%
\pgfpathcurveto{\pgfqpoint{1.259186in}{1.703019in}}{\pgfqpoint{1.255914in}{1.695119in}}{\pgfqpoint{1.255914in}{1.686883in}}%
\pgfpathcurveto{\pgfqpoint{1.255914in}{1.678646in}}{\pgfqpoint{1.259186in}{1.670746in}}{\pgfqpoint{1.265010in}{1.664922in}}%
\pgfpathcurveto{\pgfqpoint{1.270834in}{1.659099in}}{\pgfqpoint{1.278734in}{1.655826in}}{\pgfqpoint{1.286970in}{1.655826in}}%
\pgfpathclose%
\pgfusepath{stroke,fill}%
\end{pgfscope}%
\begin{pgfscope}%
\pgfpathrectangle{\pgfqpoint{0.100000in}{0.212622in}}{\pgfqpoint{3.696000in}{3.696000in}}%
\pgfusepath{clip}%
\pgfsetbuttcap%
\pgfsetroundjoin%
\definecolor{currentfill}{rgb}{0.121569,0.466667,0.705882}%
\pgfsetfillcolor{currentfill}%
\pgfsetfillopacity{0.504396}%
\pgfsetlinewidth{1.003750pt}%
\definecolor{currentstroke}{rgb}{0.121569,0.466667,0.705882}%
\pgfsetstrokecolor{currentstroke}%
\pgfsetstrokeopacity{0.504396}%
\pgfsetdash{}{0pt}%
\pgfpathmoveto{\pgfqpoint{2.051172in}{1.908683in}}%
\pgfpathcurveto{\pgfqpoint{2.059408in}{1.908683in}}{\pgfqpoint{2.067308in}{1.911956in}}{\pgfqpoint{2.073132in}{1.917780in}}%
\pgfpathcurveto{\pgfqpoint{2.078956in}{1.923604in}}{\pgfqpoint{2.082228in}{1.931504in}}{\pgfqpoint{2.082228in}{1.939740in}}%
\pgfpathcurveto{\pgfqpoint{2.082228in}{1.947976in}}{\pgfqpoint{2.078956in}{1.955876in}}{\pgfqpoint{2.073132in}{1.961700in}}%
\pgfpathcurveto{\pgfqpoint{2.067308in}{1.967524in}}{\pgfqpoint{2.059408in}{1.970796in}}{\pgfqpoint{2.051172in}{1.970796in}}%
\pgfpathcurveto{\pgfqpoint{2.042936in}{1.970796in}}{\pgfqpoint{2.035036in}{1.967524in}}{\pgfqpoint{2.029212in}{1.961700in}}%
\pgfpathcurveto{\pgfqpoint{2.023388in}{1.955876in}}{\pgfqpoint{2.020115in}{1.947976in}}{\pgfqpoint{2.020115in}{1.939740in}}%
\pgfpathcurveto{\pgfqpoint{2.020115in}{1.931504in}}{\pgfqpoint{2.023388in}{1.923604in}}{\pgfqpoint{2.029212in}{1.917780in}}%
\pgfpathcurveto{\pgfqpoint{2.035036in}{1.911956in}}{\pgfqpoint{2.042936in}{1.908683in}}{\pgfqpoint{2.051172in}{1.908683in}}%
\pgfpathclose%
\pgfusepath{stroke,fill}%
\end{pgfscope}%
\begin{pgfscope}%
\pgfpathrectangle{\pgfqpoint{0.100000in}{0.212622in}}{\pgfqpoint{3.696000in}{3.696000in}}%
\pgfusepath{clip}%
\pgfsetbuttcap%
\pgfsetroundjoin%
\definecolor{currentfill}{rgb}{0.121569,0.466667,0.705882}%
\pgfsetfillcolor{currentfill}%
\pgfsetfillopacity{0.505389}%
\pgfsetlinewidth{1.003750pt}%
\definecolor{currentstroke}{rgb}{0.121569,0.466667,0.705882}%
\pgfsetstrokecolor{currentstroke}%
\pgfsetstrokeopacity{0.505389}%
\pgfsetdash{}{0pt}%
\pgfpathmoveto{\pgfqpoint{1.278905in}{1.651623in}}%
\pgfpathcurveto{\pgfqpoint{1.287142in}{1.651623in}}{\pgfqpoint{1.295042in}{1.654895in}}{\pgfqpoint{1.300866in}{1.660719in}}%
\pgfpathcurveto{\pgfqpoint{1.306689in}{1.666543in}}{\pgfqpoint{1.309962in}{1.674443in}}{\pgfqpoint{1.309962in}{1.682679in}}%
\pgfpathcurveto{\pgfqpoint{1.309962in}{1.690915in}}{\pgfqpoint{1.306689in}{1.698815in}}{\pgfqpoint{1.300866in}{1.704639in}}%
\pgfpathcurveto{\pgfqpoint{1.295042in}{1.710463in}}{\pgfqpoint{1.287142in}{1.713736in}}{\pgfqpoint{1.278905in}{1.713736in}}%
\pgfpathcurveto{\pgfqpoint{1.270669in}{1.713736in}}{\pgfqpoint{1.262769in}{1.710463in}}{\pgfqpoint{1.256945in}{1.704639in}}%
\pgfpathcurveto{\pgfqpoint{1.251121in}{1.698815in}}{\pgfqpoint{1.247849in}{1.690915in}}{\pgfqpoint{1.247849in}{1.682679in}}%
\pgfpathcurveto{\pgfqpoint{1.247849in}{1.674443in}}{\pgfqpoint{1.251121in}{1.666543in}}{\pgfqpoint{1.256945in}{1.660719in}}%
\pgfpathcurveto{\pgfqpoint{1.262769in}{1.654895in}}{\pgfqpoint{1.270669in}{1.651623in}}{\pgfqpoint{1.278905in}{1.651623in}}%
\pgfpathclose%
\pgfusepath{stroke,fill}%
\end{pgfscope}%
\begin{pgfscope}%
\pgfpathrectangle{\pgfqpoint{0.100000in}{0.212622in}}{\pgfqpoint{3.696000in}{3.696000in}}%
\pgfusepath{clip}%
\pgfsetbuttcap%
\pgfsetroundjoin%
\definecolor{currentfill}{rgb}{0.121569,0.466667,0.705882}%
\pgfsetfillcolor{currentfill}%
\pgfsetfillopacity{0.507155}%
\pgfsetlinewidth{1.003750pt}%
\definecolor{currentstroke}{rgb}{0.121569,0.466667,0.705882}%
\pgfsetstrokecolor{currentstroke}%
\pgfsetstrokeopacity{0.507155}%
\pgfsetdash{}{0pt}%
\pgfpathmoveto{\pgfqpoint{2.052542in}{1.905261in}}%
\pgfpathcurveto{\pgfqpoint{2.060779in}{1.905261in}}{\pgfqpoint{2.068679in}{1.908533in}}{\pgfqpoint{2.074503in}{1.914357in}}%
\pgfpathcurveto{\pgfqpoint{2.080326in}{1.920181in}}{\pgfqpoint{2.083599in}{1.928081in}}{\pgfqpoint{2.083599in}{1.936317in}}%
\pgfpathcurveto{\pgfqpoint{2.083599in}{1.944553in}}{\pgfqpoint{2.080326in}{1.952453in}}{\pgfqpoint{2.074503in}{1.958277in}}%
\pgfpathcurveto{\pgfqpoint{2.068679in}{1.964101in}}{\pgfqpoint{2.060779in}{1.967374in}}{\pgfqpoint{2.052542in}{1.967374in}}%
\pgfpathcurveto{\pgfqpoint{2.044306in}{1.967374in}}{\pgfqpoint{2.036406in}{1.964101in}}{\pgfqpoint{2.030582in}{1.958277in}}%
\pgfpathcurveto{\pgfqpoint{2.024758in}{1.952453in}}{\pgfqpoint{2.021486in}{1.944553in}}{\pgfqpoint{2.021486in}{1.936317in}}%
\pgfpathcurveto{\pgfqpoint{2.021486in}{1.928081in}}{\pgfqpoint{2.024758in}{1.920181in}}{\pgfqpoint{2.030582in}{1.914357in}}%
\pgfpathcurveto{\pgfqpoint{2.036406in}{1.908533in}}{\pgfqpoint{2.044306in}{1.905261in}}{\pgfqpoint{2.052542in}{1.905261in}}%
\pgfpathclose%
\pgfusepath{stroke,fill}%
\end{pgfscope}%
\begin{pgfscope}%
\pgfpathrectangle{\pgfqpoint{0.100000in}{0.212622in}}{\pgfqpoint{3.696000in}{3.696000in}}%
\pgfusepath{clip}%
\pgfsetbuttcap%
\pgfsetroundjoin%
\definecolor{currentfill}{rgb}{0.121569,0.466667,0.705882}%
\pgfsetfillcolor{currentfill}%
\pgfsetfillopacity{0.509669}%
\pgfsetlinewidth{1.003750pt}%
\definecolor{currentstroke}{rgb}{0.121569,0.466667,0.705882}%
\pgfsetstrokecolor{currentstroke}%
\pgfsetstrokeopacity{0.509669}%
\pgfsetdash{}{0pt}%
\pgfpathmoveto{\pgfqpoint{1.264592in}{1.639115in}}%
\pgfpathcurveto{\pgfqpoint{1.272828in}{1.639115in}}{\pgfqpoint{1.280728in}{1.642388in}}{\pgfqpoint{1.286552in}{1.648212in}}%
\pgfpathcurveto{\pgfqpoint{1.292376in}{1.654036in}}{\pgfqpoint{1.295648in}{1.661936in}}{\pgfqpoint{1.295648in}{1.670172in}}%
\pgfpathcurveto{\pgfqpoint{1.295648in}{1.678408in}}{\pgfqpoint{1.292376in}{1.686308in}}{\pgfqpoint{1.286552in}{1.692132in}}%
\pgfpathcurveto{\pgfqpoint{1.280728in}{1.697956in}}{\pgfqpoint{1.272828in}{1.701228in}}{\pgfqpoint{1.264592in}{1.701228in}}%
\pgfpathcurveto{\pgfqpoint{1.256355in}{1.701228in}}{\pgfqpoint{1.248455in}{1.697956in}}{\pgfqpoint{1.242631in}{1.692132in}}%
\pgfpathcurveto{\pgfqpoint{1.236808in}{1.686308in}}{\pgfqpoint{1.233535in}{1.678408in}}{\pgfqpoint{1.233535in}{1.670172in}}%
\pgfpathcurveto{\pgfqpoint{1.233535in}{1.661936in}}{\pgfqpoint{1.236808in}{1.654036in}}{\pgfqpoint{1.242631in}{1.648212in}}%
\pgfpathcurveto{\pgfqpoint{1.248455in}{1.642388in}}{\pgfqpoint{1.256355in}{1.639115in}}{\pgfqpoint{1.264592in}{1.639115in}}%
\pgfpathclose%
\pgfusepath{stroke,fill}%
\end{pgfscope}%
\begin{pgfscope}%
\pgfpathrectangle{\pgfqpoint{0.100000in}{0.212622in}}{\pgfqpoint{3.696000in}{3.696000in}}%
\pgfusepath{clip}%
\pgfsetbuttcap%
\pgfsetroundjoin%
\definecolor{currentfill}{rgb}{0.121569,0.466667,0.705882}%
\pgfsetfillcolor{currentfill}%
\pgfsetfillopacity{0.510784}%
\pgfsetlinewidth{1.003750pt}%
\definecolor{currentstroke}{rgb}{0.121569,0.466667,0.705882}%
\pgfsetstrokecolor{currentstroke}%
\pgfsetstrokeopacity{0.510784}%
\pgfsetdash{}{0pt}%
\pgfpathmoveto{\pgfqpoint{2.053565in}{1.902928in}}%
\pgfpathcurveto{\pgfqpoint{2.061801in}{1.902928in}}{\pgfqpoint{2.069701in}{1.906200in}}{\pgfqpoint{2.075525in}{1.912024in}}%
\pgfpathcurveto{\pgfqpoint{2.081349in}{1.917848in}}{\pgfqpoint{2.084622in}{1.925748in}}{\pgfqpoint{2.084622in}{1.933985in}}%
\pgfpathcurveto{\pgfqpoint{2.084622in}{1.942221in}}{\pgfqpoint{2.081349in}{1.950121in}}{\pgfqpoint{2.075525in}{1.955945in}}%
\pgfpathcurveto{\pgfqpoint{2.069701in}{1.961769in}}{\pgfqpoint{2.061801in}{1.965041in}}{\pgfqpoint{2.053565in}{1.965041in}}%
\pgfpathcurveto{\pgfqpoint{2.045329in}{1.965041in}}{\pgfqpoint{2.037429in}{1.961769in}}{\pgfqpoint{2.031605in}{1.955945in}}%
\pgfpathcurveto{\pgfqpoint{2.025781in}{1.950121in}}{\pgfqpoint{2.022509in}{1.942221in}}{\pgfqpoint{2.022509in}{1.933985in}}%
\pgfpathcurveto{\pgfqpoint{2.022509in}{1.925748in}}{\pgfqpoint{2.025781in}{1.917848in}}{\pgfqpoint{2.031605in}{1.912024in}}%
\pgfpathcurveto{\pgfqpoint{2.037429in}{1.906200in}}{\pgfqpoint{2.045329in}{1.902928in}}{\pgfqpoint{2.053565in}{1.902928in}}%
\pgfpathclose%
\pgfusepath{stroke,fill}%
\end{pgfscope}%
\begin{pgfscope}%
\pgfpathrectangle{\pgfqpoint{0.100000in}{0.212622in}}{\pgfqpoint{3.696000in}{3.696000in}}%
\pgfusepath{clip}%
\pgfsetbuttcap%
\pgfsetroundjoin%
\definecolor{currentfill}{rgb}{0.121569,0.466667,0.705882}%
\pgfsetfillcolor{currentfill}%
\pgfsetfillopacity{0.513724}%
\pgfsetlinewidth{1.003750pt}%
\definecolor{currentstroke}{rgb}{0.121569,0.466667,0.705882}%
\pgfsetstrokecolor{currentstroke}%
\pgfsetstrokeopacity{0.513724}%
\pgfsetdash{}{0pt}%
\pgfpathmoveto{\pgfqpoint{1.253220in}{1.633783in}}%
\pgfpathcurveto{\pgfqpoint{1.261456in}{1.633783in}}{\pgfqpoint{1.269356in}{1.637056in}}{\pgfqpoint{1.275180in}{1.642879in}}%
\pgfpathcurveto{\pgfqpoint{1.281004in}{1.648703in}}{\pgfqpoint{1.284276in}{1.656603in}}{\pgfqpoint{1.284276in}{1.664840in}}%
\pgfpathcurveto{\pgfqpoint{1.284276in}{1.673076in}}{\pgfqpoint{1.281004in}{1.680976in}}{\pgfqpoint{1.275180in}{1.686800in}}%
\pgfpathcurveto{\pgfqpoint{1.269356in}{1.692624in}}{\pgfqpoint{1.261456in}{1.695896in}}{\pgfqpoint{1.253220in}{1.695896in}}%
\pgfpathcurveto{\pgfqpoint{1.244984in}{1.695896in}}{\pgfqpoint{1.237084in}{1.692624in}}{\pgfqpoint{1.231260in}{1.686800in}}%
\pgfpathcurveto{\pgfqpoint{1.225436in}{1.680976in}}{\pgfqpoint{1.222163in}{1.673076in}}{\pgfqpoint{1.222163in}{1.664840in}}%
\pgfpathcurveto{\pgfqpoint{1.222163in}{1.656603in}}{\pgfqpoint{1.225436in}{1.648703in}}{\pgfqpoint{1.231260in}{1.642879in}}%
\pgfpathcurveto{\pgfqpoint{1.237084in}{1.637056in}}{\pgfqpoint{1.244984in}{1.633783in}}{\pgfqpoint{1.253220in}{1.633783in}}%
\pgfpathclose%
\pgfusepath{stroke,fill}%
\end{pgfscope}%
\begin{pgfscope}%
\pgfpathrectangle{\pgfqpoint{0.100000in}{0.212622in}}{\pgfqpoint{3.696000in}{3.696000in}}%
\pgfusepath{clip}%
\pgfsetbuttcap%
\pgfsetroundjoin%
\definecolor{currentfill}{rgb}{0.121569,0.466667,0.705882}%
\pgfsetfillcolor{currentfill}%
\pgfsetfillopacity{0.514728}%
\pgfsetlinewidth{1.003750pt}%
\definecolor{currentstroke}{rgb}{0.121569,0.466667,0.705882}%
\pgfsetstrokecolor{currentstroke}%
\pgfsetstrokeopacity{0.514728}%
\pgfsetdash{}{0pt}%
\pgfpathmoveto{\pgfqpoint{2.056001in}{1.901011in}}%
\pgfpathcurveto{\pgfqpoint{2.064238in}{1.901011in}}{\pgfqpoint{2.072138in}{1.904283in}}{\pgfqpoint{2.077962in}{1.910107in}}%
\pgfpathcurveto{\pgfqpoint{2.083786in}{1.915931in}}{\pgfqpoint{2.087058in}{1.923831in}}{\pgfqpoint{2.087058in}{1.932067in}}%
\pgfpathcurveto{\pgfqpoint{2.087058in}{1.940303in}}{\pgfqpoint{2.083786in}{1.948203in}}{\pgfqpoint{2.077962in}{1.954027in}}%
\pgfpathcurveto{\pgfqpoint{2.072138in}{1.959851in}}{\pgfqpoint{2.064238in}{1.963124in}}{\pgfqpoint{2.056001in}{1.963124in}}%
\pgfpathcurveto{\pgfqpoint{2.047765in}{1.963124in}}{\pgfqpoint{2.039865in}{1.959851in}}{\pgfqpoint{2.034041in}{1.954027in}}%
\pgfpathcurveto{\pgfqpoint{2.028217in}{1.948203in}}{\pgfqpoint{2.024945in}{1.940303in}}{\pgfqpoint{2.024945in}{1.932067in}}%
\pgfpathcurveto{\pgfqpoint{2.024945in}{1.923831in}}{\pgfqpoint{2.028217in}{1.915931in}}{\pgfqpoint{2.034041in}{1.910107in}}%
\pgfpathcurveto{\pgfqpoint{2.039865in}{1.904283in}}{\pgfqpoint{2.047765in}{1.901011in}}{\pgfqpoint{2.056001in}{1.901011in}}%
\pgfpathclose%
\pgfusepath{stroke,fill}%
\end{pgfscope}%
\begin{pgfscope}%
\pgfpathrectangle{\pgfqpoint{0.100000in}{0.212622in}}{\pgfqpoint{3.696000in}{3.696000in}}%
\pgfusepath{clip}%
\pgfsetbuttcap%
\pgfsetroundjoin%
\definecolor{currentfill}{rgb}{0.121569,0.466667,0.705882}%
\pgfsetfillcolor{currentfill}%
\pgfsetfillopacity{0.516874}%
\pgfsetlinewidth{1.003750pt}%
\definecolor{currentstroke}{rgb}{0.121569,0.466667,0.705882}%
\pgfsetstrokecolor{currentstroke}%
\pgfsetstrokeopacity{0.516874}%
\pgfsetdash{}{0pt}%
\pgfpathmoveto{\pgfqpoint{1.242987in}{1.627837in}}%
\pgfpathcurveto{\pgfqpoint{1.251223in}{1.627837in}}{\pgfqpoint{1.259123in}{1.631109in}}{\pgfqpoint{1.264947in}{1.636933in}}%
\pgfpathcurveto{\pgfqpoint{1.270771in}{1.642757in}}{\pgfqpoint{1.274043in}{1.650657in}}{\pgfqpoint{1.274043in}{1.658894in}}%
\pgfpathcurveto{\pgfqpoint{1.274043in}{1.667130in}}{\pgfqpoint{1.270771in}{1.675030in}}{\pgfqpoint{1.264947in}{1.680854in}}%
\pgfpathcurveto{\pgfqpoint{1.259123in}{1.686678in}}{\pgfqpoint{1.251223in}{1.689950in}}{\pgfqpoint{1.242987in}{1.689950in}}%
\pgfpathcurveto{\pgfqpoint{1.234750in}{1.689950in}}{\pgfqpoint{1.226850in}{1.686678in}}{\pgfqpoint{1.221026in}{1.680854in}}%
\pgfpathcurveto{\pgfqpoint{1.215203in}{1.675030in}}{\pgfqpoint{1.211930in}{1.667130in}}{\pgfqpoint{1.211930in}{1.658894in}}%
\pgfpathcurveto{\pgfqpoint{1.211930in}{1.650657in}}{\pgfqpoint{1.215203in}{1.642757in}}{\pgfqpoint{1.221026in}{1.636933in}}%
\pgfpathcurveto{\pgfqpoint{1.226850in}{1.631109in}}{\pgfqpoint{1.234750in}{1.627837in}}{\pgfqpoint{1.242987in}{1.627837in}}%
\pgfpathclose%
\pgfusepath{stroke,fill}%
\end{pgfscope}%
\begin{pgfscope}%
\pgfpathrectangle{\pgfqpoint{0.100000in}{0.212622in}}{\pgfqpoint{3.696000in}{3.696000in}}%
\pgfusepath{clip}%
\pgfsetbuttcap%
\pgfsetroundjoin%
\definecolor{currentfill}{rgb}{0.121569,0.466667,0.705882}%
\pgfsetfillcolor{currentfill}%
\pgfsetfillopacity{0.519118}%
\pgfsetlinewidth{1.003750pt}%
\definecolor{currentstroke}{rgb}{0.121569,0.466667,0.705882}%
\pgfsetstrokecolor{currentstroke}%
\pgfsetstrokeopacity{0.519118}%
\pgfsetdash{}{0pt}%
\pgfpathmoveto{\pgfqpoint{2.058893in}{1.898197in}}%
\pgfpathcurveto{\pgfqpoint{2.067129in}{1.898197in}}{\pgfqpoint{2.075029in}{1.901469in}}{\pgfqpoint{2.080853in}{1.907293in}}%
\pgfpathcurveto{\pgfqpoint{2.086677in}{1.913117in}}{\pgfqpoint{2.089949in}{1.921017in}}{\pgfqpoint{2.089949in}{1.929253in}}%
\pgfpathcurveto{\pgfqpoint{2.089949in}{1.937489in}}{\pgfqpoint{2.086677in}{1.945389in}}{\pgfqpoint{2.080853in}{1.951213in}}%
\pgfpathcurveto{\pgfqpoint{2.075029in}{1.957037in}}{\pgfqpoint{2.067129in}{1.960310in}}{\pgfqpoint{2.058893in}{1.960310in}}%
\pgfpathcurveto{\pgfqpoint{2.050657in}{1.960310in}}{\pgfqpoint{2.042757in}{1.957037in}}{\pgfqpoint{2.036933in}{1.951213in}}%
\pgfpathcurveto{\pgfqpoint{2.031109in}{1.945389in}}{\pgfqpoint{2.027836in}{1.937489in}}{\pgfqpoint{2.027836in}{1.929253in}}%
\pgfpathcurveto{\pgfqpoint{2.027836in}{1.921017in}}{\pgfqpoint{2.031109in}{1.913117in}}{\pgfqpoint{2.036933in}{1.907293in}}%
\pgfpathcurveto{\pgfqpoint{2.042757in}{1.901469in}}{\pgfqpoint{2.050657in}{1.898197in}}{\pgfqpoint{2.058893in}{1.898197in}}%
\pgfpathclose%
\pgfusepath{stroke,fill}%
\end{pgfscope}%
\begin{pgfscope}%
\pgfpathrectangle{\pgfqpoint{0.100000in}{0.212622in}}{\pgfqpoint{3.696000in}{3.696000in}}%
\pgfusepath{clip}%
\pgfsetbuttcap%
\pgfsetroundjoin%
\definecolor{currentfill}{rgb}{0.121569,0.466667,0.705882}%
\pgfsetfillcolor{currentfill}%
\pgfsetfillopacity{0.519348}%
\pgfsetlinewidth{1.003750pt}%
\definecolor{currentstroke}{rgb}{0.121569,0.466667,0.705882}%
\pgfsetstrokecolor{currentstroke}%
\pgfsetstrokeopacity{0.519348}%
\pgfsetdash{}{0pt}%
\pgfpathmoveto{\pgfqpoint{1.233724in}{1.617475in}}%
\pgfpathcurveto{\pgfqpoint{1.241960in}{1.617475in}}{\pgfqpoint{1.249860in}{1.620747in}}{\pgfqpoint{1.255684in}{1.626571in}}%
\pgfpathcurveto{\pgfqpoint{1.261508in}{1.632395in}}{\pgfqpoint{1.264780in}{1.640295in}}{\pgfqpoint{1.264780in}{1.648531in}}%
\pgfpathcurveto{\pgfqpoint{1.264780in}{1.656767in}}{\pgfqpoint{1.261508in}{1.664667in}}{\pgfqpoint{1.255684in}{1.670491in}}%
\pgfpathcurveto{\pgfqpoint{1.249860in}{1.676315in}}{\pgfqpoint{1.241960in}{1.679588in}}{\pgfqpoint{1.233724in}{1.679588in}}%
\pgfpathcurveto{\pgfqpoint{1.225487in}{1.679588in}}{\pgfqpoint{1.217587in}{1.676315in}}{\pgfqpoint{1.211763in}{1.670491in}}%
\pgfpathcurveto{\pgfqpoint{1.205940in}{1.664667in}}{\pgfqpoint{1.202667in}{1.656767in}}{\pgfqpoint{1.202667in}{1.648531in}}%
\pgfpathcurveto{\pgfqpoint{1.202667in}{1.640295in}}{\pgfqpoint{1.205940in}{1.632395in}}{\pgfqpoint{1.211763in}{1.626571in}}%
\pgfpathcurveto{\pgfqpoint{1.217587in}{1.620747in}}{\pgfqpoint{1.225487in}{1.617475in}}{\pgfqpoint{1.233724in}{1.617475in}}%
\pgfpathclose%
\pgfusepath{stroke,fill}%
\end{pgfscope}%
\begin{pgfscope}%
\pgfpathrectangle{\pgfqpoint{0.100000in}{0.212622in}}{\pgfqpoint{3.696000in}{3.696000in}}%
\pgfusepath{clip}%
\pgfsetbuttcap%
\pgfsetroundjoin%
\definecolor{currentfill}{rgb}{0.121569,0.466667,0.705882}%
\pgfsetfillcolor{currentfill}%
\pgfsetfillopacity{0.521277}%
\pgfsetlinewidth{1.003750pt}%
\definecolor{currentstroke}{rgb}{0.121569,0.466667,0.705882}%
\pgfsetstrokecolor{currentstroke}%
\pgfsetstrokeopacity{0.521277}%
\pgfsetdash{}{0pt}%
\pgfpathmoveto{\pgfqpoint{1.226606in}{1.612491in}}%
\pgfpathcurveto{\pgfqpoint{1.234842in}{1.612491in}}{\pgfqpoint{1.242742in}{1.615763in}}{\pgfqpoint{1.248566in}{1.621587in}}%
\pgfpathcurveto{\pgfqpoint{1.254390in}{1.627411in}}{\pgfqpoint{1.257662in}{1.635311in}}{\pgfqpoint{1.257662in}{1.643547in}}%
\pgfpathcurveto{\pgfqpoint{1.257662in}{1.651784in}}{\pgfqpoint{1.254390in}{1.659684in}}{\pgfqpoint{1.248566in}{1.665508in}}%
\pgfpathcurveto{\pgfqpoint{1.242742in}{1.671332in}}{\pgfqpoint{1.234842in}{1.674604in}}{\pgfqpoint{1.226606in}{1.674604in}}%
\pgfpathcurveto{\pgfqpoint{1.218370in}{1.674604in}}{\pgfqpoint{1.210470in}{1.671332in}}{\pgfqpoint{1.204646in}{1.665508in}}%
\pgfpathcurveto{\pgfqpoint{1.198822in}{1.659684in}}{\pgfqpoint{1.195549in}{1.651784in}}{\pgfqpoint{1.195549in}{1.643547in}}%
\pgfpathcurveto{\pgfqpoint{1.195549in}{1.635311in}}{\pgfqpoint{1.198822in}{1.627411in}}{\pgfqpoint{1.204646in}{1.621587in}}%
\pgfpathcurveto{\pgfqpoint{1.210470in}{1.615763in}}{\pgfqpoint{1.218370in}{1.612491in}}{\pgfqpoint{1.226606in}{1.612491in}}%
\pgfpathclose%
\pgfusepath{stroke,fill}%
\end{pgfscope}%
\begin{pgfscope}%
\pgfpathrectangle{\pgfqpoint{0.100000in}{0.212622in}}{\pgfqpoint{3.696000in}{3.696000in}}%
\pgfusepath{clip}%
\pgfsetbuttcap%
\pgfsetroundjoin%
\definecolor{currentfill}{rgb}{0.121569,0.466667,0.705882}%
\pgfsetfillcolor{currentfill}%
\pgfsetfillopacity{0.522616}%
\pgfsetlinewidth{1.003750pt}%
\definecolor{currentstroke}{rgb}{0.121569,0.466667,0.705882}%
\pgfsetstrokecolor{currentstroke}%
\pgfsetstrokeopacity{0.522616}%
\pgfsetdash{}{0pt}%
\pgfpathmoveto{\pgfqpoint{1.222195in}{1.609228in}}%
\pgfpathcurveto{\pgfqpoint{1.230432in}{1.609228in}}{\pgfqpoint{1.238332in}{1.612501in}}{\pgfqpoint{1.244156in}{1.618325in}}%
\pgfpathcurveto{\pgfqpoint{1.249980in}{1.624148in}}{\pgfqpoint{1.253252in}{1.632048in}}{\pgfqpoint{1.253252in}{1.640285in}}%
\pgfpathcurveto{\pgfqpoint{1.253252in}{1.648521in}}{\pgfqpoint{1.249980in}{1.656421in}}{\pgfqpoint{1.244156in}{1.662245in}}%
\pgfpathcurveto{\pgfqpoint{1.238332in}{1.668069in}}{\pgfqpoint{1.230432in}{1.671341in}}{\pgfqpoint{1.222195in}{1.671341in}}%
\pgfpathcurveto{\pgfqpoint{1.213959in}{1.671341in}}{\pgfqpoint{1.206059in}{1.668069in}}{\pgfqpoint{1.200235in}{1.662245in}}%
\pgfpathcurveto{\pgfqpoint{1.194411in}{1.656421in}}{\pgfqpoint{1.191139in}{1.648521in}}{\pgfqpoint{1.191139in}{1.640285in}}%
\pgfpathcurveto{\pgfqpoint{1.191139in}{1.632048in}}{\pgfqpoint{1.194411in}{1.624148in}}{\pgfqpoint{1.200235in}{1.618325in}}%
\pgfpathcurveto{\pgfqpoint{1.206059in}{1.612501in}}{\pgfqpoint{1.213959in}{1.609228in}}{\pgfqpoint{1.222195in}{1.609228in}}%
\pgfpathclose%
\pgfusepath{stroke,fill}%
\end{pgfscope}%
\begin{pgfscope}%
\pgfpathrectangle{\pgfqpoint{0.100000in}{0.212622in}}{\pgfqpoint{3.696000in}{3.696000in}}%
\pgfusepath{clip}%
\pgfsetbuttcap%
\pgfsetroundjoin%
\definecolor{currentfill}{rgb}{0.121569,0.466667,0.705882}%
\pgfsetfillcolor{currentfill}%
\pgfsetfillopacity{0.523593}%
\pgfsetlinewidth{1.003750pt}%
\definecolor{currentstroke}{rgb}{0.121569,0.466667,0.705882}%
\pgfsetstrokecolor{currentstroke}%
\pgfsetstrokeopacity{0.523593}%
\pgfsetdash{}{0pt}%
\pgfpathmoveto{\pgfqpoint{1.218666in}{1.604420in}}%
\pgfpathcurveto{\pgfqpoint{1.226903in}{1.604420in}}{\pgfqpoint{1.234803in}{1.607693in}}{\pgfqpoint{1.240627in}{1.613517in}}%
\pgfpathcurveto{\pgfqpoint{1.246450in}{1.619341in}}{\pgfqpoint{1.249723in}{1.627241in}}{\pgfqpoint{1.249723in}{1.635477in}}%
\pgfpathcurveto{\pgfqpoint{1.249723in}{1.643713in}}{\pgfqpoint{1.246450in}{1.651613in}}{\pgfqpoint{1.240627in}{1.657437in}}%
\pgfpathcurveto{\pgfqpoint{1.234803in}{1.663261in}}{\pgfqpoint{1.226903in}{1.666533in}}{\pgfqpoint{1.218666in}{1.666533in}}%
\pgfpathcurveto{\pgfqpoint{1.210430in}{1.666533in}}{\pgfqpoint{1.202530in}{1.663261in}}{\pgfqpoint{1.196706in}{1.657437in}}%
\pgfpathcurveto{\pgfqpoint{1.190882in}{1.651613in}}{\pgfqpoint{1.187610in}{1.643713in}}{\pgfqpoint{1.187610in}{1.635477in}}%
\pgfpathcurveto{\pgfqpoint{1.187610in}{1.627241in}}{\pgfqpoint{1.190882in}{1.619341in}}{\pgfqpoint{1.196706in}{1.613517in}}%
\pgfpathcurveto{\pgfqpoint{1.202530in}{1.607693in}}{\pgfqpoint{1.210430in}{1.604420in}}{\pgfqpoint{1.218666in}{1.604420in}}%
\pgfpathclose%
\pgfusepath{stroke,fill}%
\end{pgfscope}%
\begin{pgfscope}%
\pgfpathrectangle{\pgfqpoint{0.100000in}{0.212622in}}{\pgfqpoint{3.696000in}{3.696000in}}%
\pgfusepath{clip}%
\pgfsetbuttcap%
\pgfsetroundjoin%
\definecolor{currentfill}{rgb}{0.121569,0.466667,0.705882}%
\pgfsetfillcolor{currentfill}%
\pgfsetfillopacity{0.524102}%
\pgfsetlinewidth{1.003750pt}%
\definecolor{currentstroke}{rgb}{0.121569,0.466667,0.705882}%
\pgfsetstrokecolor{currentstroke}%
\pgfsetstrokeopacity{0.524102}%
\pgfsetdash{}{0pt}%
\pgfpathmoveto{\pgfqpoint{2.061068in}{1.896231in}}%
\pgfpathcurveto{\pgfqpoint{2.069304in}{1.896231in}}{\pgfqpoint{2.077205in}{1.899503in}}{\pgfqpoint{2.083028in}{1.905327in}}%
\pgfpathcurveto{\pgfqpoint{2.088852in}{1.911151in}}{\pgfqpoint{2.092125in}{1.919051in}}{\pgfqpoint{2.092125in}{1.927288in}}%
\pgfpathcurveto{\pgfqpoint{2.092125in}{1.935524in}}{\pgfqpoint{2.088852in}{1.943424in}}{\pgfqpoint{2.083028in}{1.949248in}}%
\pgfpathcurveto{\pgfqpoint{2.077205in}{1.955072in}}{\pgfqpoint{2.069304in}{1.958344in}}{\pgfqpoint{2.061068in}{1.958344in}}%
\pgfpathcurveto{\pgfqpoint{2.052832in}{1.958344in}}{\pgfqpoint{2.044932in}{1.955072in}}{\pgfqpoint{2.039108in}{1.949248in}}%
\pgfpathcurveto{\pgfqpoint{2.033284in}{1.943424in}}{\pgfqpoint{2.030012in}{1.935524in}}{\pgfqpoint{2.030012in}{1.927288in}}%
\pgfpathcurveto{\pgfqpoint{2.030012in}{1.919051in}}{\pgfqpoint{2.033284in}{1.911151in}}{\pgfqpoint{2.039108in}{1.905327in}}%
\pgfpathcurveto{\pgfqpoint{2.044932in}{1.899503in}}{\pgfqpoint{2.052832in}{1.896231in}}{\pgfqpoint{2.061068in}{1.896231in}}%
\pgfpathclose%
\pgfusepath{stroke,fill}%
\end{pgfscope}%
\begin{pgfscope}%
\pgfpathrectangle{\pgfqpoint{0.100000in}{0.212622in}}{\pgfqpoint{3.696000in}{3.696000in}}%
\pgfusepath{clip}%
\pgfsetbuttcap%
\pgfsetroundjoin%
\definecolor{currentfill}{rgb}{0.121569,0.466667,0.705882}%
\pgfsetfillcolor{currentfill}%
\pgfsetfillopacity{0.524202}%
\pgfsetlinewidth{1.003750pt}%
\definecolor{currentstroke}{rgb}{0.121569,0.466667,0.705882}%
\pgfsetstrokecolor{currentstroke}%
\pgfsetstrokeopacity{0.524202}%
\pgfsetdash{}{0pt}%
\pgfpathmoveto{\pgfqpoint{1.215919in}{1.602663in}}%
\pgfpathcurveto{\pgfqpoint{1.224155in}{1.602663in}}{\pgfqpoint{1.232055in}{1.605935in}}{\pgfqpoint{1.237879in}{1.611759in}}%
\pgfpathcurveto{\pgfqpoint{1.243703in}{1.617583in}}{\pgfqpoint{1.246975in}{1.625483in}}{\pgfqpoint{1.246975in}{1.633719in}}%
\pgfpathcurveto{\pgfqpoint{1.246975in}{1.641956in}}{\pgfqpoint{1.243703in}{1.649856in}}{\pgfqpoint{1.237879in}{1.655680in}}%
\pgfpathcurveto{\pgfqpoint{1.232055in}{1.661503in}}{\pgfqpoint{1.224155in}{1.664776in}}{\pgfqpoint{1.215919in}{1.664776in}}%
\pgfpathcurveto{\pgfqpoint{1.207683in}{1.664776in}}{\pgfqpoint{1.199783in}{1.661503in}}{\pgfqpoint{1.193959in}{1.655680in}}%
\pgfpathcurveto{\pgfqpoint{1.188135in}{1.649856in}}{\pgfqpoint{1.184862in}{1.641956in}}{\pgfqpoint{1.184862in}{1.633719in}}%
\pgfpathcurveto{\pgfqpoint{1.184862in}{1.625483in}}{\pgfqpoint{1.188135in}{1.617583in}}{\pgfqpoint{1.193959in}{1.611759in}}%
\pgfpathcurveto{\pgfqpoint{1.199783in}{1.605935in}}{\pgfqpoint{1.207683in}{1.602663in}}{\pgfqpoint{1.215919in}{1.602663in}}%
\pgfpathclose%
\pgfusepath{stroke,fill}%
\end{pgfscope}%
\begin{pgfscope}%
\pgfpathrectangle{\pgfqpoint{0.100000in}{0.212622in}}{\pgfqpoint{3.696000in}{3.696000in}}%
\pgfusepath{clip}%
\pgfsetbuttcap%
\pgfsetroundjoin%
\definecolor{currentfill}{rgb}{0.121569,0.466667,0.705882}%
\pgfsetfillcolor{currentfill}%
\pgfsetfillopacity{0.525530}%
\pgfsetlinewidth{1.003750pt}%
\definecolor{currentstroke}{rgb}{0.121569,0.466667,0.705882}%
\pgfsetstrokecolor{currentstroke}%
\pgfsetstrokeopacity{0.525530}%
\pgfsetdash{}{0pt}%
\pgfpathmoveto{\pgfqpoint{1.211591in}{1.599695in}}%
\pgfpathcurveto{\pgfqpoint{1.219828in}{1.599695in}}{\pgfqpoint{1.227728in}{1.602967in}}{\pgfqpoint{1.233552in}{1.608791in}}%
\pgfpathcurveto{\pgfqpoint{1.239376in}{1.614615in}}{\pgfqpoint{1.242648in}{1.622515in}}{\pgfqpoint{1.242648in}{1.630752in}}%
\pgfpathcurveto{\pgfqpoint{1.242648in}{1.638988in}}{\pgfqpoint{1.239376in}{1.646888in}}{\pgfqpoint{1.233552in}{1.652712in}}%
\pgfpathcurveto{\pgfqpoint{1.227728in}{1.658536in}}{\pgfqpoint{1.219828in}{1.661808in}}{\pgfqpoint{1.211591in}{1.661808in}}%
\pgfpathcurveto{\pgfqpoint{1.203355in}{1.661808in}}{\pgfqpoint{1.195455in}{1.658536in}}{\pgfqpoint{1.189631in}{1.652712in}}%
\pgfpathcurveto{\pgfqpoint{1.183807in}{1.646888in}}{\pgfqpoint{1.180535in}{1.638988in}}{\pgfqpoint{1.180535in}{1.630752in}}%
\pgfpathcurveto{\pgfqpoint{1.180535in}{1.622515in}}{\pgfqpoint{1.183807in}{1.614615in}}{\pgfqpoint{1.189631in}{1.608791in}}%
\pgfpathcurveto{\pgfqpoint{1.195455in}{1.602967in}}{\pgfqpoint{1.203355in}{1.599695in}}{\pgfqpoint{1.211591in}{1.599695in}}%
\pgfpathclose%
\pgfusepath{stroke,fill}%
\end{pgfscope}%
\begin{pgfscope}%
\pgfpathrectangle{\pgfqpoint{0.100000in}{0.212622in}}{\pgfqpoint{3.696000in}{3.696000in}}%
\pgfusepath{clip}%
\pgfsetbuttcap%
\pgfsetroundjoin%
\definecolor{currentfill}{rgb}{0.121569,0.466667,0.705882}%
\pgfsetfillcolor{currentfill}%
\pgfsetfillopacity{0.526183}%
\pgfsetlinewidth{1.003750pt}%
\definecolor{currentstroke}{rgb}{0.121569,0.466667,0.705882}%
\pgfsetstrokecolor{currentstroke}%
\pgfsetstrokeopacity{0.526183}%
\pgfsetdash{}{0pt}%
\pgfpathmoveto{\pgfqpoint{1.208962in}{1.596050in}}%
\pgfpathcurveto{\pgfqpoint{1.217199in}{1.596050in}}{\pgfqpoint{1.225099in}{1.599322in}}{\pgfqpoint{1.230923in}{1.605146in}}%
\pgfpathcurveto{\pgfqpoint{1.236747in}{1.610970in}}{\pgfqpoint{1.240019in}{1.618870in}}{\pgfqpoint{1.240019in}{1.627106in}}%
\pgfpathcurveto{\pgfqpoint{1.240019in}{1.635343in}}{\pgfqpoint{1.236747in}{1.643243in}}{\pgfqpoint{1.230923in}{1.649067in}}%
\pgfpathcurveto{\pgfqpoint{1.225099in}{1.654890in}}{\pgfqpoint{1.217199in}{1.658163in}}{\pgfqpoint{1.208962in}{1.658163in}}%
\pgfpathcurveto{\pgfqpoint{1.200726in}{1.658163in}}{\pgfqpoint{1.192826in}{1.654890in}}{\pgfqpoint{1.187002in}{1.649067in}}%
\pgfpathcurveto{\pgfqpoint{1.181178in}{1.643243in}}{\pgfqpoint{1.177906in}{1.635343in}}{\pgfqpoint{1.177906in}{1.627106in}}%
\pgfpathcurveto{\pgfqpoint{1.177906in}{1.618870in}}{\pgfqpoint{1.181178in}{1.610970in}}{\pgfqpoint{1.187002in}{1.605146in}}%
\pgfpathcurveto{\pgfqpoint{1.192826in}{1.599322in}}{\pgfqpoint{1.200726in}{1.596050in}}{\pgfqpoint{1.208962in}{1.596050in}}%
\pgfpathclose%
\pgfusepath{stroke,fill}%
\end{pgfscope}%
\begin{pgfscope}%
\pgfpathrectangle{\pgfqpoint{0.100000in}{0.212622in}}{\pgfqpoint{3.696000in}{3.696000in}}%
\pgfusepath{clip}%
\pgfsetbuttcap%
\pgfsetroundjoin%
\definecolor{currentfill}{rgb}{0.121569,0.466667,0.705882}%
\pgfsetfillcolor{currentfill}%
\pgfsetfillopacity{0.526495}%
\pgfsetlinewidth{1.003750pt}%
\definecolor{currentstroke}{rgb}{0.121569,0.466667,0.705882}%
\pgfsetstrokecolor{currentstroke}%
\pgfsetstrokeopacity{0.526495}%
\pgfsetdash{}{0pt}%
\pgfpathmoveto{\pgfqpoint{1.206980in}{1.593933in}}%
\pgfpathcurveto{\pgfqpoint{1.215216in}{1.593933in}}{\pgfqpoint{1.223116in}{1.597206in}}{\pgfqpoint{1.228940in}{1.603030in}}%
\pgfpathcurveto{\pgfqpoint{1.234764in}{1.608854in}}{\pgfqpoint{1.238036in}{1.616754in}}{\pgfqpoint{1.238036in}{1.624990in}}%
\pgfpathcurveto{\pgfqpoint{1.238036in}{1.633226in}}{\pgfqpoint{1.234764in}{1.641126in}}{\pgfqpoint{1.228940in}{1.646950in}}%
\pgfpathcurveto{\pgfqpoint{1.223116in}{1.652774in}}{\pgfqpoint{1.215216in}{1.656046in}}{\pgfqpoint{1.206980in}{1.656046in}}%
\pgfpathcurveto{\pgfqpoint{1.198743in}{1.656046in}}{\pgfqpoint{1.190843in}{1.652774in}}{\pgfqpoint{1.185019in}{1.646950in}}%
\pgfpathcurveto{\pgfqpoint{1.179195in}{1.641126in}}{\pgfqpoint{1.175923in}{1.633226in}}{\pgfqpoint{1.175923in}{1.624990in}}%
\pgfpathcurveto{\pgfqpoint{1.175923in}{1.616754in}}{\pgfqpoint{1.179195in}{1.608854in}}{\pgfqpoint{1.185019in}{1.603030in}}%
\pgfpathcurveto{\pgfqpoint{1.190843in}{1.597206in}}{\pgfqpoint{1.198743in}{1.593933in}}{\pgfqpoint{1.206980in}{1.593933in}}%
\pgfpathclose%
\pgfusepath{stroke,fill}%
\end{pgfscope}%
\begin{pgfscope}%
\pgfpathrectangle{\pgfqpoint{0.100000in}{0.212622in}}{\pgfqpoint{3.696000in}{3.696000in}}%
\pgfusepath{clip}%
\pgfsetbuttcap%
\pgfsetroundjoin%
\definecolor{currentfill}{rgb}{0.121569,0.466667,0.705882}%
\pgfsetfillcolor{currentfill}%
\pgfsetfillopacity{0.527584}%
\pgfsetlinewidth{1.003750pt}%
\definecolor{currentstroke}{rgb}{0.121569,0.466667,0.705882}%
\pgfsetstrokecolor{currentstroke}%
\pgfsetstrokeopacity{0.527584}%
\pgfsetdash{}{0pt}%
\pgfpathmoveto{\pgfqpoint{1.203791in}{1.592651in}}%
\pgfpathcurveto{\pgfqpoint{1.212027in}{1.592651in}}{\pgfqpoint{1.219927in}{1.595923in}}{\pgfqpoint{1.225751in}{1.601747in}}%
\pgfpathcurveto{\pgfqpoint{1.231575in}{1.607571in}}{\pgfqpoint{1.234848in}{1.615471in}}{\pgfqpoint{1.234848in}{1.623708in}}%
\pgfpathcurveto{\pgfqpoint{1.234848in}{1.631944in}}{\pgfqpoint{1.231575in}{1.639844in}}{\pgfqpoint{1.225751in}{1.645668in}}%
\pgfpathcurveto{\pgfqpoint{1.219927in}{1.651492in}}{\pgfqpoint{1.212027in}{1.654764in}}{\pgfqpoint{1.203791in}{1.654764in}}%
\pgfpathcurveto{\pgfqpoint{1.195555in}{1.654764in}}{\pgfqpoint{1.187655in}{1.651492in}}{\pgfqpoint{1.181831in}{1.645668in}}%
\pgfpathcurveto{\pgfqpoint{1.176007in}{1.639844in}}{\pgfqpoint{1.172735in}{1.631944in}}{\pgfqpoint{1.172735in}{1.623708in}}%
\pgfpathcurveto{\pgfqpoint{1.172735in}{1.615471in}}{\pgfqpoint{1.176007in}{1.607571in}}{\pgfqpoint{1.181831in}{1.601747in}}%
\pgfpathcurveto{\pgfqpoint{1.187655in}{1.595923in}}{\pgfqpoint{1.195555in}{1.592651in}}{\pgfqpoint{1.203791in}{1.592651in}}%
\pgfpathclose%
\pgfusepath{stroke,fill}%
\end{pgfscope}%
\begin{pgfscope}%
\pgfpathrectangle{\pgfqpoint{0.100000in}{0.212622in}}{\pgfqpoint{3.696000in}{3.696000in}}%
\pgfusepath{clip}%
\pgfsetbuttcap%
\pgfsetroundjoin%
\definecolor{currentfill}{rgb}{0.121569,0.466667,0.705882}%
\pgfsetfillcolor{currentfill}%
\pgfsetfillopacity{0.527962}%
\pgfsetlinewidth{1.003750pt}%
\definecolor{currentstroke}{rgb}{0.121569,0.466667,0.705882}%
\pgfsetstrokecolor{currentstroke}%
\pgfsetstrokeopacity{0.527962}%
\pgfsetdash{}{0pt}%
\pgfpathmoveto{\pgfqpoint{1.202367in}{1.591037in}}%
\pgfpathcurveto{\pgfqpoint{1.210604in}{1.591037in}}{\pgfqpoint{1.218504in}{1.594310in}}{\pgfqpoint{1.224328in}{1.600134in}}%
\pgfpathcurveto{\pgfqpoint{1.230151in}{1.605958in}}{\pgfqpoint{1.233424in}{1.613858in}}{\pgfqpoint{1.233424in}{1.622094in}}%
\pgfpathcurveto{\pgfqpoint{1.233424in}{1.630330in}}{\pgfqpoint{1.230151in}{1.638230in}}{\pgfqpoint{1.224328in}{1.644054in}}%
\pgfpathcurveto{\pgfqpoint{1.218504in}{1.649878in}}{\pgfqpoint{1.210604in}{1.653150in}}{\pgfqpoint{1.202367in}{1.653150in}}%
\pgfpathcurveto{\pgfqpoint{1.194131in}{1.653150in}}{\pgfqpoint{1.186231in}{1.649878in}}{\pgfqpoint{1.180407in}{1.644054in}}%
\pgfpathcurveto{\pgfqpoint{1.174583in}{1.638230in}}{\pgfqpoint{1.171311in}{1.630330in}}{\pgfqpoint{1.171311in}{1.622094in}}%
\pgfpathcurveto{\pgfqpoint{1.171311in}{1.613858in}}{\pgfqpoint{1.174583in}{1.605958in}}{\pgfqpoint{1.180407in}{1.600134in}}%
\pgfpathcurveto{\pgfqpoint{1.186231in}{1.594310in}}{\pgfqpoint{1.194131in}{1.591037in}}{\pgfqpoint{1.202367in}{1.591037in}}%
\pgfpathclose%
\pgfusepath{stroke,fill}%
\end{pgfscope}%
\begin{pgfscope}%
\pgfpathrectangle{\pgfqpoint{0.100000in}{0.212622in}}{\pgfqpoint{3.696000in}{3.696000in}}%
\pgfusepath{clip}%
\pgfsetbuttcap%
\pgfsetroundjoin%
\definecolor{currentfill}{rgb}{0.121569,0.466667,0.705882}%
\pgfsetfillcolor{currentfill}%
\pgfsetfillopacity{0.528810}%
\pgfsetlinewidth{1.003750pt}%
\definecolor{currentstroke}{rgb}{0.121569,0.466667,0.705882}%
\pgfsetstrokecolor{currentstroke}%
\pgfsetstrokeopacity{0.528810}%
\pgfsetdash{}{0pt}%
\pgfpathmoveto{\pgfqpoint{1.199782in}{1.589115in}}%
\pgfpathcurveto{\pgfqpoint{1.208018in}{1.589115in}}{\pgfqpoint{1.215918in}{1.592388in}}{\pgfqpoint{1.221742in}{1.598212in}}%
\pgfpathcurveto{\pgfqpoint{1.227566in}{1.604036in}}{\pgfqpoint{1.230839in}{1.611936in}}{\pgfqpoint{1.230839in}{1.620172in}}%
\pgfpathcurveto{\pgfqpoint{1.230839in}{1.628408in}}{\pgfqpoint{1.227566in}{1.636308in}}{\pgfqpoint{1.221742in}{1.642132in}}%
\pgfpathcurveto{\pgfqpoint{1.215918in}{1.647956in}}{\pgfqpoint{1.208018in}{1.651228in}}{\pgfqpoint{1.199782in}{1.651228in}}%
\pgfpathcurveto{\pgfqpoint{1.191546in}{1.651228in}}{\pgfqpoint{1.183646in}{1.647956in}}{\pgfqpoint{1.177822in}{1.642132in}}%
\pgfpathcurveto{\pgfqpoint{1.171998in}{1.636308in}}{\pgfqpoint{1.168726in}{1.628408in}}{\pgfqpoint{1.168726in}{1.620172in}}%
\pgfpathcurveto{\pgfqpoint{1.168726in}{1.611936in}}{\pgfqpoint{1.171998in}{1.604036in}}{\pgfqpoint{1.177822in}{1.598212in}}%
\pgfpathcurveto{\pgfqpoint{1.183646in}{1.592388in}}{\pgfqpoint{1.191546in}{1.589115in}}{\pgfqpoint{1.199782in}{1.589115in}}%
\pgfpathclose%
\pgfusepath{stroke,fill}%
\end{pgfscope}%
\begin{pgfscope}%
\pgfpathrectangle{\pgfqpoint{0.100000in}{0.212622in}}{\pgfqpoint{3.696000in}{3.696000in}}%
\pgfusepath{clip}%
\pgfsetbuttcap%
\pgfsetroundjoin%
\definecolor{currentfill}{rgb}{0.121569,0.466667,0.705882}%
\pgfsetfillcolor{currentfill}%
\pgfsetfillopacity{0.529364}%
\pgfsetlinewidth{1.003750pt}%
\definecolor{currentstroke}{rgb}{0.121569,0.466667,0.705882}%
\pgfsetstrokecolor{currentstroke}%
\pgfsetstrokeopacity{0.529364}%
\pgfsetdash{}{0pt}%
\pgfpathmoveto{\pgfqpoint{2.064066in}{1.893655in}}%
\pgfpathcurveto{\pgfqpoint{2.072302in}{1.893655in}}{\pgfqpoint{2.080202in}{1.896928in}}{\pgfqpoint{2.086026in}{1.902752in}}%
\pgfpathcurveto{\pgfqpoint{2.091850in}{1.908575in}}{\pgfqpoint{2.095122in}{1.916475in}}{\pgfqpoint{2.095122in}{1.924712in}}%
\pgfpathcurveto{\pgfqpoint{2.095122in}{1.932948in}}{\pgfqpoint{2.091850in}{1.940848in}}{\pgfqpoint{2.086026in}{1.946672in}}%
\pgfpathcurveto{\pgfqpoint{2.080202in}{1.952496in}}{\pgfqpoint{2.072302in}{1.955768in}}{\pgfqpoint{2.064066in}{1.955768in}}%
\pgfpathcurveto{\pgfqpoint{2.055830in}{1.955768in}}{\pgfqpoint{2.047929in}{1.952496in}}{\pgfqpoint{2.042106in}{1.946672in}}%
\pgfpathcurveto{\pgfqpoint{2.036282in}{1.940848in}}{\pgfqpoint{2.033009in}{1.932948in}}{\pgfqpoint{2.033009in}{1.924712in}}%
\pgfpathcurveto{\pgfqpoint{2.033009in}{1.916475in}}{\pgfqpoint{2.036282in}{1.908575in}}{\pgfqpoint{2.042106in}{1.902752in}}%
\pgfpathcurveto{\pgfqpoint{2.047929in}{1.896928in}}{\pgfqpoint{2.055830in}{1.893655in}}{\pgfqpoint{2.064066in}{1.893655in}}%
\pgfpathclose%
\pgfusepath{stroke,fill}%
\end{pgfscope}%
\begin{pgfscope}%
\pgfpathrectangle{\pgfqpoint{0.100000in}{0.212622in}}{\pgfqpoint{3.696000in}{3.696000in}}%
\pgfusepath{clip}%
\pgfsetbuttcap%
\pgfsetroundjoin%
\definecolor{currentfill}{rgb}{0.121569,0.466667,0.705882}%
\pgfsetfillcolor{currentfill}%
\pgfsetfillopacity{0.530583}%
\pgfsetlinewidth{1.003750pt}%
\definecolor{currentstroke}{rgb}{0.121569,0.466667,0.705882}%
\pgfsetstrokecolor{currentstroke}%
\pgfsetstrokeopacity{0.530583}%
\pgfsetdash{}{0pt}%
\pgfpathmoveto{\pgfqpoint{1.194901in}{1.587337in}}%
\pgfpathcurveto{\pgfqpoint{1.203137in}{1.587337in}}{\pgfqpoint{1.211037in}{1.590609in}}{\pgfqpoint{1.216861in}{1.596433in}}%
\pgfpathcurveto{\pgfqpoint{1.222685in}{1.602257in}}{\pgfqpoint{1.225957in}{1.610157in}}{\pgfqpoint{1.225957in}{1.618393in}}%
\pgfpathcurveto{\pgfqpoint{1.225957in}{1.626630in}}{\pgfqpoint{1.222685in}{1.634530in}}{\pgfqpoint{1.216861in}{1.640354in}}%
\pgfpathcurveto{\pgfqpoint{1.211037in}{1.646178in}}{\pgfqpoint{1.203137in}{1.649450in}}{\pgfqpoint{1.194901in}{1.649450in}}%
\pgfpathcurveto{\pgfqpoint{1.186664in}{1.649450in}}{\pgfqpoint{1.178764in}{1.646178in}}{\pgfqpoint{1.172940in}{1.640354in}}%
\pgfpathcurveto{\pgfqpoint{1.167116in}{1.634530in}}{\pgfqpoint{1.163844in}{1.626630in}}{\pgfqpoint{1.163844in}{1.618393in}}%
\pgfpathcurveto{\pgfqpoint{1.163844in}{1.610157in}}{\pgfqpoint{1.167116in}{1.602257in}}{\pgfqpoint{1.172940in}{1.596433in}}%
\pgfpathcurveto{\pgfqpoint{1.178764in}{1.590609in}}{\pgfqpoint{1.186664in}{1.587337in}}{\pgfqpoint{1.194901in}{1.587337in}}%
\pgfpathclose%
\pgfusepath{stroke,fill}%
\end{pgfscope}%
\begin{pgfscope}%
\pgfpathrectangle{\pgfqpoint{0.100000in}{0.212622in}}{\pgfqpoint{3.696000in}{3.696000in}}%
\pgfusepath{clip}%
\pgfsetbuttcap%
\pgfsetroundjoin%
\definecolor{currentfill}{rgb}{0.121569,0.466667,0.705882}%
\pgfsetfillcolor{currentfill}%
\pgfsetfillopacity{0.531353}%
\pgfsetlinewidth{1.003750pt}%
\definecolor{currentstroke}{rgb}{0.121569,0.466667,0.705882}%
\pgfsetstrokecolor{currentstroke}%
\pgfsetstrokeopacity{0.531353}%
\pgfsetdash{}{0pt}%
\pgfpathmoveto{\pgfqpoint{1.192051in}{1.583534in}}%
\pgfpathcurveto{\pgfqpoint{1.200288in}{1.583534in}}{\pgfqpoint{1.208188in}{1.586806in}}{\pgfqpoint{1.214012in}{1.592630in}}%
\pgfpathcurveto{\pgfqpoint{1.219836in}{1.598454in}}{\pgfqpoint{1.223108in}{1.606354in}}{\pgfqpoint{1.223108in}{1.614590in}}%
\pgfpathcurveto{\pgfqpoint{1.223108in}{1.622827in}}{\pgfqpoint{1.219836in}{1.630727in}}{\pgfqpoint{1.214012in}{1.636551in}}%
\pgfpathcurveto{\pgfqpoint{1.208188in}{1.642375in}}{\pgfqpoint{1.200288in}{1.645647in}}{\pgfqpoint{1.192051in}{1.645647in}}%
\pgfpathcurveto{\pgfqpoint{1.183815in}{1.645647in}}{\pgfqpoint{1.175915in}{1.642375in}}{\pgfqpoint{1.170091in}{1.636551in}}%
\pgfpathcurveto{\pgfqpoint{1.164267in}{1.630727in}}{\pgfqpoint{1.160995in}{1.622827in}}{\pgfqpoint{1.160995in}{1.614590in}}%
\pgfpathcurveto{\pgfqpoint{1.160995in}{1.606354in}}{\pgfqpoint{1.164267in}{1.598454in}}{\pgfqpoint{1.170091in}{1.592630in}}%
\pgfpathcurveto{\pgfqpoint{1.175915in}{1.586806in}}{\pgfqpoint{1.183815in}{1.583534in}}{\pgfqpoint{1.192051in}{1.583534in}}%
\pgfpathclose%
\pgfusepath{stroke,fill}%
\end{pgfscope}%
\begin{pgfscope}%
\pgfpathrectangle{\pgfqpoint{0.100000in}{0.212622in}}{\pgfqpoint{3.696000in}{3.696000in}}%
\pgfusepath{clip}%
\pgfsetbuttcap%
\pgfsetroundjoin%
\definecolor{currentfill}{rgb}{0.121569,0.466667,0.705882}%
\pgfsetfillcolor{currentfill}%
\pgfsetfillopacity{0.532289}%
\pgfsetlinewidth{1.003750pt}%
\definecolor{currentstroke}{rgb}{0.121569,0.466667,0.705882}%
\pgfsetstrokecolor{currentstroke}%
\pgfsetstrokeopacity{0.532289}%
\pgfsetdash{}{0pt}%
\pgfpathmoveto{\pgfqpoint{1.185226in}{1.576466in}}%
\pgfpathcurveto{\pgfqpoint{1.193463in}{1.576466in}}{\pgfqpoint{1.201363in}{1.579739in}}{\pgfqpoint{1.207187in}{1.585562in}}%
\pgfpathcurveto{\pgfqpoint{1.213011in}{1.591386in}}{\pgfqpoint{1.216283in}{1.599286in}}{\pgfqpoint{1.216283in}{1.607523in}}%
\pgfpathcurveto{\pgfqpoint{1.216283in}{1.615759in}}{\pgfqpoint{1.213011in}{1.623659in}}{\pgfqpoint{1.207187in}{1.629483in}}%
\pgfpathcurveto{\pgfqpoint{1.201363in}{1.635307in}}{\pgfqpoint{1.193463in}{1.638579in}}{\pgfqpoint{1.185226in}{1.638579in}}%
\pgfpathcurveto{\pgfqpoint{1.176990in}{1.638579in}}{\pgfqpoint{1.169090in}{1.635307in}}{\pgfqpoint{1.163266in}{1.629483in}}%
\pgfpathcurveto{\pgfqpoint{1.157442in}{1.623659in}}{\pgfqpoint{1.154170in}{1.615759in}}{\pgfqpoint{1.154170in}{1.607523in}}%
\pgfpathcurveto{\pgfqpoint{1.154170in}{1.599286in}}{\pgfqpoint{1.157442in}{1.591386in}}{\pgfqpoint{1.163266in}{1.585562in}}%
\pgfpathcurveto{\pgfqpoint{1.169090in}{1.579739in}}{\pgfqpoint{1.176990in}{1.576466in}}{\pgfqpoint{1.185226in}{1.576466in}}%
\pgfpathclose%
\pgfusepath{stroke,fill}%
\end{pgfscope}%
\begin{pgfscope}%
\pgfpathrectangle{\pgfqpoint{0.100000in}{0.212622in}}{\pgfqpoint{3.696000in}{3.696000in}}%
\pgfusepath{clip}%
\pgfsetbuttcap%
\pgfsetroundjoin%
\definecolor{currentfill}{rgb}{0.121569,0.466667,0.705882}%
\pgfsetfillcolor{currentfill}%
\pgfsetfillopacity{0.534692}%
\pgfsetlinewidth{1.003750pt}%
\definecolor{currentstroke}{rgb}{0.121569,0.466667,0.705882}%
\pgfsetstrokecolor{currentstroke}%
\pgfsetstrokeopacity{0.534692}%
\pgfsetdash{}{0pt}%
\pgfpathmoveto{\pgfqpoint{2.067230in}{1.887914in}}%
\pgfpathcurveto{\pgfqpoint{2.075466in}{1.887914in}}{\pgfqpoint{2.083366in}{1.891186in}}{\pgfqpoint{2.089190in}{1.897010in}}%
\pgfpathcurveto{\pgfqpoint{2.095014in}{1.902834in}}{\pgfqpoint{2.098287in}{1.910734in}}{\pgfqpoint{2.098287in}{1.918970in}}%
\pgfpathcurveto{\pgfqpoint{2.098287in}{1.927206in}}{\pgfqpoint{2.095014in}{1.935106in}}{\pgfqpoint{2.089190in}{1.940930in}}%
\pgfpathcurveto{\pgfqpoint{2.083366in}{1.946754in}}{\pgfqpoint{2.075466in}{1.950027in}}{\pgfqpoint{2.067230in}{1.950027in}}%
\pgfpathcurveto{\pgfqpoint{2.058994in}{1.950027in}}{\pgfqpoint{2.051094in}{1.946754in}}{\pgfqpoint{2.045270in}{1.940930in}}%
\pgfpathcurveto{\pgfqpoint{2.039446in}{1.935106in}}{\pgfqpoint{2.036174in}{1.927206in}}{\pgfqpoint{2.036174in}{1.918970in}}%
\pgfpathcurveto{\pgfqpoint{2.036174in}{1.910734in}}{\pgfqpoint{2.039446in}{1.902834in}}{\pgfqpoint{2.045270in}{1.897010in}}%
\pgfpathcurveto{\pgfqpoint{2.051094in}{1.891186in}}{\pgfqpoint{2.058994in}{1.887914in}}{\pgfqpoint{2.067230in}{1.887914in}}%
\pgfpathclose%
\pgfusepath{stroke,fill}%
\end{pgfscope}%
\begin{pgfscope}%
\pgfpathrectangle{\pgfqpoint{0.100000in}{0.212622in}}{\pgfqpoint{3.696000in}{3.696000in}}%
\pgfusepath{clip}%
\pgfsetbuttcap%
\pgfsetroundjoin%
\definecolor{currentfill}{rgb}{0.121569,0.466667,0.705882}%
\pgfsetfillcolor{currentfill}%
\pgfsetfillopacity{0.535611}%
\pgfsetlinewidth{1.003750pt}%
\definecolor{currentstroke}{rgb}{0.121569,0.466667,0.705882}%
\pgfsetstrokecolor{currentstroke}%
\pgfsetstrokeopacity{0.535611}%
\pgfsetdash{}{0pt}%
\pgfpathmoveto{\pgfqpoint{1.174496in}{1.570612in}}%
\pgfpathcurveto{\pgfqpoint{1.182732in}{1.570612in}}{\pgfqpoint{1.190632in}{1.573884in}}{\pgfqpoint{1.196456in}{1.579708in}}%
\pgfpathcurveto{\pgfqpoint{1.202280in}{1.585532in}}{\pgfqpoint{1.205553in}{1.593432in}}{\pgfqpoint{1.205553in}{1.601668in}}%
\pgfpathcurveto{\pgfqpoint{1.205553in}{1.609904in}}{\pgfqpoint{1.202280in}{1.617805in}}{\pgfqpoint{1.196456in}{1.623628in}}%
\pgfpathcurveto{\pgfqpoint{1.190632in}{1.629452in}}{\pgfqpoint{1.182732in}{1.632725in}}{\pgfqpoint{1.174496in}{1.632725in}}%
\pgfpathcurveto{\pgfqpoint{1.166260in}{1.632725in}}{\pgfqpoint{1.158360in}{1.629452in}}{\pgfqpoint{1.152536in}{1.623628in}}%
\pgfpathcurveto{\pgfqpoint{1.146712in}{1.617805in}}{\pgfqpoint{1.143440in}{1.609904in}}{\pgfqpoint{1.143440in}{1.601668in}}%
\pgfpathcurveto{\pgfqpoint{1.143440in}{1.593432in}}{\pgfqpoint{1.146712in}{1.585532in}}{\pgfqpoint{1.152536in}{1.579708in}}%
\pgfpathcurveto{\pgfqpoint{1.158360in}{1.573884in}}{\pgfqpoint{1.166260in}{1.570612in}}{\pgfqpoint{1.174496in}{1.570612in}}%
\pgfpathclose%
\pgfusepath{stroke,fill}%
\end{pgfscope}%
\begin{pgfscope}%
\pgfpathrectangle{\pgfqpoint{0.100000in}{0.212622in}}{\pgfqpoint{3.696000in}{3.696000in}}%
\pgfusepath{clip}%
\pgfsetbuttcap%
\pgfsetroundjoin%
\definecolor{currentfill}{rgb}{0.121569,0.466667,0.705882}%
\pgfsetfillcolor{currentfill}%
\pgfsetfillopacity{0.537556}%
\pgfsetlinewidth{1.003750pt}%
\definecolor{currentstroke}{rgb}{0.121569,0.466667,0.705882}%
\pgfsetstrokecolor{currentstroke}%
\pgfsetstrokeopacity{0.537556}%
\pgfsetdash{}{0pt}%
\pgfpathmoveto{\pgfqpoint{2.068423in}{1.884046in}}%
\pgfpathcurveto{\pgfqpoint{2.076660in}{1.884046in}}{\pgfqpoint{2.084560in}{1.887318in}}{\pgfqpoint{2.090384in}{1.893142in}}%
\pgfpathcurveto{\pgfqpoint{2.096208in}{1.898966in}}{\pgfqpoint{2.099480in}{1.906866in}}{\pgfqpoint{2.099480in}{1.915103in}}%
\pgfpathcurveto{\pgfqpoint{2.099480in}{1.923339in}}{\pgfqpoint{2.096208in}{1.931239in}}{\pgfqpoint{2.090384in}{1.937063in}}%
\pgfpathcurveto{\pgfqpoint{2.084560in}{1.942887in}}{\pgfqpoint{2.076660in}{1.946159in}}{\pgfqpoint{2.068423in}{1.946159in}}%
\pgfpathcurveto{\pgfqpoint{2.060187in}{1.946159in}}{\pgfqpoint{2.052287in}{1.942887in}}{\pgfqpoint{2.046463in}{1.937063in}}%
\pgfpathcurveto{\pgfqpoint{2.040639in}{1.931239in}}{\pgfqpoint{2.037367in}{1.923339in}}{\pgfqpoint{2.037367in}{1.915103in}}%
\pgfpathcurveto{\pgfqpoint{2.037367in}{1.906866in}}{\pgfqpoint{2.040639in}{1.898966in}}{\pgfqpoint{2.046463in}{1.893142in}}%
\pgfpathcurveto{\pgfqpoint{2.052287in}{1.887318in}}{\pgfqpoint{2.060187in}{1.884046in}}{\pgfqpoint{2.068423in}{1.884046in}}%
\pgfpathclose%
\pgfusepath{stroke,fill}%
\end{pgfscope}%
\begin{pgfscope}%
\pgfpathrectangle{\pgfqpoint{0.100000in}{0.212622in}}{\pgfqpoint{3.696000in}{3.696000in}}%
\pgfusepath{clip}%
\pgfsetbuttcap%
\pgfsetroundjoin%
\definecolor{currentfill}{rgb}{0.121569,0.466667,0.705882}%
\pgfsetfillcolor{currentfill}%
\pgfsetfillopacity{0.538545}%
\pgfsetlinewidth{1.003750pt}%
\definecolor{currentstroke}{rgb}{0.121569,0.466667,0.705882}%
\pgfsetstrokecolor{currentstroke}%
\pgfsetstrokeopacity{0.538545}%
\pgfsetdash{}{0pt}%
\pgfpathmoveto{\pgfqpoint{1.165088in}{1.561685in}}%
\pgfpathcurveto{\pgfqpoint{1.173325in}{1.561685in}}{\pgfqpoint{1.181225in}{1.564957in}}{\pgfqpoint{1.187049in}{1.570781in}}%
\pgfpathcurveto{\pgfqpoint{1.192873in}{1.576605in}}{\pgfqpoint{1.196145in}{1.584505in}}{\pgfqpoint{1.196145in}{1.592742in}}%
\pgfpathcurveto{\pgfqpoint{1.196145in}{1.600978in}}{\pgfqpoint{1.192873in}{1.608878in}}{\pgfqpoint{1.187049in}{1.614702in}}%
\pgfpathcurveto{\pgfqpoint{1.181225in}{1.620526in}}{\pgfqpoint{1.173325in}{1.623798in}}{\pgfqpoint{1.165088in}{1.623798in}}%
\pgfpathcurveto{\pgfqpoint{1.156852in}{1.623798in}}{\pgfqpoint{1.148952in}{1.620526in}}{\pgfqpoint{1.143128in}{1.614702in}}%
\pgfpathcurveto{\pgfqpoint{1.137304in}{1.608878in}}{\pgfqpoint{1.134032in}{1.600978in}}{\pgfqpoint{1.134032in}{1.592742in}}%
\pgfpathcurveto{\pgfqpoint{1.134032in}{1.584505in}}{\pgfqpoint{1.137304in}{1.576605in}}{\pgfqpoint{1.143128in}{1.570781in}}%
\pgfpathcurveto{\pgfqpoint{1.148952in}{1.564957in}}{\pgfqpoint{1.156852in}{1.561685in}}{\pgfqpoint{1.165088in}{1.561685in}}%
\pgfpathclose%
\pgfusepath{stroke,fill}%
\end{pgfscope}%
\begin{pgfscope}%
\pgfpathrectangle{\pgfqpoint{0.100000in}{0.212622in}}{\pgfqpoint{3.696000in}{3.696000in}}%
\pgfusepath{clip}%
\pgfsetbuttcap%
\pgfsetroundjoin%
\definecolor{currentfill}{rgb}{0.121569,0.466667,0.705882}%
\pgfsetfillcolor{currentfill}%
\pgfsetfillopacity{0.540770}%
\pgfsetlinewidth{1.003750pt}%
\definecolor{currentstroke}{rgb}{0.121569,0.466667,0.705882}%
\pgfsetstrokecolor{currentstroke}%
\pgfsetstrokeopacity{0.540770}%
\pgfsetdash{}{0pt}%
\pgfpathmoveto{\pgfqpoint{1.157900in}{1.555288in}}%
\pgfpathcurveto{\pgfqpoint{1.166136in}{1.555288in}}{\pgfqpoint{1.174036in}{1.558560in}}{\pgfqpoint{1.179860in}{1.564384in}}%
\pgfpathcurveto{\pgfqpoint{1.185684in}{1.570208in}}{\pgfqpoint{1.188956in}{1.578108in}}{\pgfqpoint{1.188956in}{1.586344in}}%
\pgfpathcurveto{\pgfqpoint{1.188956in}{1.594581in}}{\pgfqpoint{1.185684in}{1.602481in}}{\pgfqpoint{1.179860in}{1.608305in}}%
\pgfpathcurveto{\pgfqpoint{1.174036in}{1.614128in}}{\pgfqpoint{1.166136in}{1.617401in}}{\pgfqpoint{1.157900in}{1.617401in}}%
\pgfpathcurveto{\pgfqpoint{1.149664in}{1.617401in}}{\pgfqpoint{1.141764in}{1.614128in}}{\pgfqpoint{1.135940in}{1.608305in}}%
\pgfpathcurveto{\pgfqpoint{1.130116in}{1.602481in}}{\pgfqpoint{1.126843in}{1.594581in}}{\pgfqpoint{1.126843in}{1.586344in}}%
\pgfpathcurveto{\pgfqpoint{1.126843in}{1.578108in}}{\pgfqpoint{1.130116in}{1.570208in}}{\pgfqpoint{1.135940in}{1.564384in}}%
\pgfpathcurveto{\pgfqpoint{1.141764in}{1.558560in}}{\pgfqpoint{1.149664in}{1.555288in}}{\pgfqpoint{1.157900in}{1.555288in}}%
\pgfpathclose%
\pgfusepath{stroke,fill}%
\end{pgfscope}%
\begin{pgfscope}%
\pgfpathrectangle{\pgfqpoint{0.100000in}{0.212622in}}{\pgfqpoint{3.696000in}{3.696000in}}%
\pgfusepath{clip}%
\pgfsetbuttcap%
\pgfsetroundjoin%
\definecolor{currentfill}{rgb}{0.121569,0.466667,0.705882}%
\pgfsetfillcolor{currentfill}%
\pgfsetfillopacity{0.540919}%
\pgfsetlinewidth{1.003750pt}%
\definecolor{currentstroke}{rgb}{0.121569,0.466667,0.705882}%
\pgfsetstrokecolor{currentstroke}%
\pgfsetstrokeopacity{0.540919}%
\pgfsetdash{}{0pt}%
\pgfpathmoveto{\pgfqpoint{2.070474in}{1.881940in}}%
\pgfpathcurveto{\pgfqpoint{2.078711in}{1.881940in}}{\pgfqpoint{2.086611in}{1.885212in}}{\pgfqpoint{2.092435in}{1.891036in}}%
\pgfpathcurveto{\pgfqpoint{2.098258in}{1.896860in}}{\pgfqpoint{2.101531in}{1.904760in}}{\pgfqpoint{2.101531in}{1.912996in}}%
\pgfpathcurveto{\pgfqpoint{2.101531in}{1.921232in}}{\pgfqpoint{2.098258in}{1.929132in}}{\pgfqpoint{2.092435in}{1.934956in}}%
\pgfpathcurveto{\pgfqpoint{2.086611in}{1.940780in}}{\pgfqpoint{2.078711in}{1.944053in}}{\pgfqpoint{2.070474in}{1.944053in}}%
\pgfpathcurveto{\pgfqpoint{2.062238in}{1.944053in}}{\pgfqpoint{2.054338in}{1.940780in}}{\pgfqpoint{2.048514in}{1.934956in}}%
\pgfpathcurveto{\pgfqpoint{2.042690in}{1.929132in}}{\pgfqpoint{2.039418in}{1.921232in}}{\pgfqpoint{2.039418in}{1.912996in}}%
\pgfpathcurveto{\pgfqpoint{2.039418in}{1.904760in}}{\pgfqpoint{2.042690in}{1.896860in}}{\pgfqpoint{2.048514in}{1.891036in}}%
\pgfpathcurveto{\pgfqpoint{2.054338in}{1.885212in}}{\pgfqpoint{2.062238in}{1.881940in}}{\pgfqpoint{2.070474in}{1.881940in}}%
\pgfpathclose%
\pgfusepath{stroke,fill}%
\end{pgfscope}%
\begin{pgfscope}%
\pgfpathrectangle{\pgfqpoint{0.100000in}{0.212622in}}{\pgfqpoint{3.696000in}{3.696000in}}%
\pgfusepath{clip}%
\pgfsetbuttcap%
\pgfsetroundjoin%
\definecolor{currentfill}{rgb}{0.121569,0.466667,0.705882}%
\pgfsetfillcolor{currentfill}%
\pgfsetfillopacity{0.542139}%
\pgfsetlinewidth{1.003750pt}%
\definecolor{currentstroke}{rgb}{0.121569,0.466667,0.705882}%
\pgfsetstrokecolor{currentstroke}%
\pgfsetstrokeopacity{0.542139}%
\pgfsetdash{}{0pt}%
\pgfpathmoveto{\pgfqpoint{1.153082in}{1.551475in}}%
\pgfpathcurveto{\pgfqpoint{1.161319in}{1.551475in}}{\pgfqpoint{1.169219in}{1.554748in}}{\pgfqpoint{1.175043in}{1.560572in}}%
\pgfpathcurveto{\pgfqpoint{1.180867in}{1.566396in}}{\pgfqpoint{1.184139in}{1.574296in}}{\pgfqpoint{1.184139in}{1.582532in}}%
\pgfpathcurveto{\pgfqpoint{1.184139in}{1.590768in}}{\pgfqpoint{1.180867in}{1.598668in}}{\pgfqpoint{1.175043in}{1.604492in}}%
\pgfpathcurveto{\pgfqpoint{1.169219in}{1.610316in}}{\pgfqpoint{1.161319in}{1.613588in}}{\pgfqpoint{1.153082in}{1.613588in}}%
\pgfpathcurveto{\pgfqpoint{1.144846in}{1.613588in}}{\pgfqpoint{1.136946in}{1.610316in}}{\pgfqpoint{1.131122in}{1.604492in}}%
\pgfpathcurveto{\pgfqpoint{1.125298in}{1.598668in}}{\pgfqpoint{1.122026in}{1.590768in}}{\pgfqpoint{1.122026in}{1.582532in}}%
\pgfpathcurveto{\pgfqpoint{1.122026in}{1.574296in}}{\pgfqpoint{1.125298in}{1.566396in}}{\pgfqpoint{1.131122in}{1.560572in}}%
\pgfpathcurveto{\pgfqpoint{1.136946in}{1.554748in}}{\pgfqpoint{1.144846in}{1.551475in}}{\pgfqpoint{1.153082in}{1.551475in}}%
\pgfpathclose%
\pgfusepath{stroke,fill}%
\end{pgfscope}%
\begin{pgfscope}%
\pgfpathrectangle{\pgfqpoint{0.100000in}{0.212622in}}{\pgfqpoint{3.696000in}{3.696000in}}%
\pgfusepath{clip}%
\pgfsetbuttcap%
\pgfsetroundjoin%
\definecolor{currentfill}{rgb}{0.121569,0.466667,0.705882}%
\pgfsetfillcolor{currentfill}%
\pgfsetfillopacity{0.544333}%
\pgfsetlinewidth{1.003750pt}%
\definecolor{currentstroke}{rgb}{0.121569,0.466667,0.705882}%
\pgfsetstrokecolor{currentstroke}%
\pgfsetstrokeopacity{0.544333}%
\pgfsetdash{}{0pt}%
\pgfpathmoveto{\pgfqpoint{1.144518in}{1.542382in}}%
\pgfpathcurveto{\pgfqpoint{1.152754in}{1.542382in}}{\pgfqpoint{1.160654in}{1.545654in}}{\pgfqpoint{1.166478in}{1.551478in}}%
\pgfpathcurveto{\pgfqpoint{1.172302in}{1.557302in}}{\pgfqpoint{1.175574in}{1.565202in}}{\pgfqpoint{1.175574in}{1.573438in}}%
\pgfpathcurveto{\pgfqpoint{1.175574in}{1.581675in}}{\pgfqpoint{1.172302in}{1.589575in}}{\pgfqpoint{1.166478in}{1.595399in}}%
\pgfpathcurveto{\pgfqpoint{1.160654in}{1.601222in}}{\pgfqpoint{1.152754in}{1.604495in}}{\pgfqpoint{1.144518in}{1.604495in}}%
\pgfpathcurveto{\pgfqpoint{1.136281in}{1.604495in}}{\pgfqpoint{1.128381in}{1.601222in}}{\pgfqpoint{1.122557in}{1.595399in}}%
\pgfpathcurveto{\pgfqpoint{1.116733in}{1.589575in}}{\pgfqpoint{1.113461in}{1.581675in}}{\pgfqpoint{1.113461in}{1.573438in}}%
\pgfpathcurveto{\pgfqpoint{1.113461in}{1.565202in}}{\pgfqpoint{1.116733in}{1.557302in}}{\pgfqpoint{1.122557in}{1.551478in}}%
\pgfpathcurveto{\pgfqpoint{1.128381in}{1.545654in}}{\pgfqpoint{1.136281in}{1.542382in}}{\pgfqpoint{1.144518in}{1.542382in}}%
\pgfpathclose%
\pgfusepath{stroke,fill}%
\end{pgfscope}%
\begin{pgfscope}%
\pgfpathrectangle{\pgfqpoint{0.100000in}{0.212622in}}{\pgfqpoint{3.696000in}{3.696000in}}%
\pgfusepath{clip}%
\pgfsetbuttcap%
\pgfsetroundjoin%
\definecolor{currentfill}{rgb}{0.121569,0.466667,0.705882}%
\pgfsetfillcolor{currentfill}%
\pgfsetfillopacity{0.544664}%
\pgfsetlinewidth{1.003750pt}%
\definecolor{currentstroke}{rgb}{0.121569,0.466667,0.705882}%
\pgfsetstrokecolor{currentstroke}%
\pgfsetstrokeopacity{0.544664}%
\pgfsetdash{}{0pt}%
\pgfpathmoveto{\pgfqpoint{2.073072in}{1.879811in}}%
\pgfpathcurveto{\pgfqpoint{2.081309in}{1.879811in}}{\pgfqpoint{2.089209in}{1.883083in}}{\pgfqpoint{2.095033in}{1.888907in}}%
\pgfpathcurveto{\pgfqpoint{2.100857in}{1.894731in}}{\pgfqpoint{2.104129in}{1.902631in}}{\pgfqpoint{2.104129in}{1.910867in}}%
\pgfpathcurveto{\pgfqpoint{2.104129in}{1.919104in}}{\pgfqpoint{2.100857in}{1.927004in}}{\pgfqpoint{2.095033in}{1.932828in}}%
\pgfpathcurveto{\pgfqpoint{2.089209in}{1.938652in}}{\pgfqpoint{2.081309in}{1.941924in}}{\pgfqpoint{2.073072in}{1.941924in}}%
\pgfpathcurveto{\pgfqpoint{2.064836in}{1.941924in}}{\pgfqpoint{2.056936in}{1.938652in}}{\pgfqpoint{2.051112in}{1.932828in}}%
\pgfpathcurveto{\pgfqpoint{2.045288in}{1.927004in}}{\pgfqpoint{2.042016in}{1.919104in}}{\pgfqpoint{2.042016in}{1.910867in}}%
\pgfpathcurveto{\pgfqpoint{2.042016in}{1.902631in}}{\pgfqpoint{2.045288in}{1.894731in}}{\pgfqpoint{2.051112in}{1.888907in}}%
\pgfpathcurveto{\pgfqpoint{2.056936in}{1.883083in}}{\pgfqpoint{2.064836in}{1.879811in}}{\pgfqpoint{2.073072in}{1.879811in}}%
\pgfpathclose%
\pgfusepath{stroke,fill}%
\end{pgfscope}%
\begin{pgfscope}%
\pgfpathrectangle{\pgfqpoint{0.100000in}{0.212622in}}{\pgfqpoint{3.696000in}{3.696000in}}%
\pgfusepath{clip}%
\pgfsetbuttcap%
\pgfsetroundjoin%
\definecolor{currentfill}{rgb}{0.121569,0.466667,0.705882}%
\pgfsetfillcolor{currentfill}%
\pgfsetfillopacity{0.547126}%
\pgfsetlinewidth{1.003750pt}%
\definecolor{currentstroke}{rgb}{0.121569,0.466667,0.705882}%
\pgfsetstrokecolor{currentstroke}%
\pgfsetstrokeopacity{0.547126}%
\pgfsetdash{}{0pt}%
\pgfpathmoveto{\pgfqpoint{1.137302in}{1.539555in}}%
\pgfpathcurveto{\pgfqpoint{1.145538in}{1.539555in}}{\pgfqpoint{1.153438in}{1.542827in}}{\pgfqpoint{1.159262in}{1.548651in}}%
\pgfpathcurveto{\pgfqpoint{1.165086in}{1.554475in}}{\pgfqpoint{1.168359in}{1.562375in}}{\pgfqpoint{1.168359in}{1.570611in}}%
\pgfpathcurveto{\pgfqpoint{1.168359in}{1.578847in}}{\pgfqpoint{1.165086in}{1.586747in}}{\pgfqpoint{1.159262in}{1.592571in}}%
\pgfpathcurveto{\pgfqpoint{1.153438in}{1.598395in}}{\pgfqpoint{1.145538in}{1.601668in}}{\pgfqpoint{1.137302in}{1.601668in}}%
\pgfpathcurveto{\pgfqpoint{1.129066in}{1.601668in}}{\pgfqpoint{1.121166in}{1.598395in}}{\pgfqpoint{1.115342in}{1.592571in}}%
\pgfpathcurveto{\pgfqpoint{1.109518in}{1.586747in}}{\pgfqpoint{1.106246in}{1.578847in}}{\pgfqpoint{1.106246in}{1.570611in}}%
\pgfpathcurveto{\pgfqpoint{1.106246in}{1.562375in}}{\pgfqpoint{1.109518in}{1.554475in}}{\pgfqpoint{1.115342in}{1.548651in}}%
\pgfpathcurveto{\pgfqpoint{1.121166in}{1.542827in}}{\pgfqpoint{1.129066in}{1.539555in}}{\pgfqpoint{1.137302in}{1.539555in}}%
\pgfpathclose%
\pgfusepath{stroke,fill}%
\end{pgfscope}%
\begin{pgfscope}%
\pgfpathrectangle{\pgfqpoint{0.100000in}{0.212622in}}{\pgfqpoint{3.696000in}{3.696000in}}%
\pgfusepath{clip}%
\pgfsetbuttcap%
\pgfsetroundjoin%
\definecolor{currentfill}{rgb}{0.121569,0.466667,0.705882}%
\pgfsetfillcolor{currentfill}%
\pgfsetfillopacity{0.548536}%
\pgfsetlinewidth{1.003750pt}%
\definecolor{currentstroke}{rgb}{0.121569,0.466667,0.705882}%
\pgfsetstrokecolor{currentstroke}%
\pgfsetstrokeopacity{0.548536}%
\pgfsetdash{}{0pt}%
\pgfpathmoveto{\pgfqpoint{1.131735in}{1.532937in}}%
\pgfpathcurveto{\pgfqpoint{1.139971in}{1.532937in}}{\pgfqpoint{1.147871in}{1.536209in}}{\pgfqpoint{1.153695in}{1.542033in}}%
\pgfpathcurveto{\pgfqpoint{1.159519in}{1.547857in}}{\pgfqpoint{1.162792in}{1.555757in}}{\pgfqpoint{1.162792in}{1.563993in}}%
\pgfpathcurveto{\pgfqpoint{1.162792in}{1.572230in}}{\pgfqpoint{1.159519in}{1.580130in}}{\pgfqpoint{1.153695in}{1.585954in}}%
\pgfpathcurveto{\pgfqpoint{1.147871in}{1.591778in}}{\pgfqpoint{1.139971in}{1.595050in}}{\pgfqpoint{1.131735in}{1.595050in}}%
\pgfpathcurveto{\pgfqpoint{1.123499in}{1.595050in}}{\pgfqpoint{1.115599in}{1.591778in}}{\pgfqpoint{1.109775in}{1.585954in}}%
\pgfpathcurveto{\pgfqpoint{1.103951in}{1.580130in}}{\pgfqpoint{1.100679in}{1.572230in}}{\pgfqpoint{1.100679in}{1.563993in}}%
\pgfpathcurveto{\pgfqpoint{1.100679in}{1.555757in}}{\pgfqpoint{1.103951in}{1.547857in}}{\pgfqpoint{1.109775in}{1.542033in}}%
\pgfpathcurveto{\pgfqpoint{1.115599in}{1.536209in}}{\pgfqpoint{1.123499in}{1.532937in}}{\pgfqpoint{1.131735in}{1.532937in}}%
\pgfpathclose%
\pgfusepath{stroke,fill}%
\end{pgfscope}%
\begin{pgfscope}%
\pgfpathrectangle{\pgfqpoint{0.100000in}{0.212622in}}{\pgfqpoint{3.696000in}{3.696000in}}%
\pgfusepath{clip}%
\pgfsetbuttcap%
\pgfsetroundjoin%
\definecolor{currentfill}{rgb}{0.121569,0.466667,0.705882}%
\pgfsetfillcolor{currentfill}%
\pgfsetfillopacity{0.548664}%
\pgfsetlinewidth{1.003750pt}%
\definecolor{currentstroke}{rgb}{0.121569,0.466667,0.705882}%
\pgfsetstrokecolor{currentstroke}%
\pgfsetstrokeopacity{0.548664}%
\pgfsetdash{}{0pt}%
\pgfpathmoveto{\pgfqpoint{2.075562in}{1.874915in}}%
\pgfpathcurveto{\pgfqpoint{2.083798in}{1.874915in}}{\pgfqpoint{2.091698in}{1.878187in}}{\pgfqpoint{2.097522in}{1.884011in}}%
\pgfpathcurveto{\pgfqpoint{2.103346in}{1.889835in}}{\pgfqpoint{2.106619in}{1.897735in}}{\pgfqpoint{2.106619in}{1.905972in}}%
\pgfpathcurveto{\pgfqpoint{2.106619in}{1.914208in}}{\pgfqpoint{2.103346in}{1.922108in}}{\pgfqpoint{2.097522in}{1.927932in}}%
\pgfpathcurveto{\pgfqpoint{2.091698in}{1.933756in}}{\pgfqpoint{2.083798in}{1.937028in}}{\pgfqpoint{2.075562in}{1.937028in}}%
\pgfpathcurveto{\pgfqpoint{2.067326in}{1.937028in}}{\pgfqpoint{2.059426in}{1.933756in}}{\pgfqpoint{2.053602in}{1.927932in}}%
\pgfpathcurveto{\pgfqpoint{2.047778in}{1.922108in}}{\pgfqpoint{2.044506in}{1.914208in}}{\pgfqpoint{2.044506in}{1.905972in}}%
\pgfpathcurveto{\pgfqpoint{2.044506in}{1.897735in}}{\pgfqpoint{2.047778in}{1.889835in}}{\pgfqpoint{2.053602in}{1.884011in}}%
\pgfpathcurveto{\pgfqpoint{2.059426in}{1.878187in}}{\pgfqpoint{2.067326in}{1.874915in}}{\pgfqpoint{2.075562in}{1.874915in}}%
\pgfpathclose%
\pgfusepath{stroke,fill}%
\end{pgfscope}%
\begin{pgfscope}%
\pgfpathrectangle{\pgfqpoint{0.100000in}{0.212622in}}{\pgfqpoint{3.696000in}{3.696000in}}%
\pgfusepath{clip}%
\pgfsetbuttcap%
\pgfsetroundjoin%
\definecolor{currentfill}{rgb}{0.121569,0.466667,0.705882}%
\pgfsetfillcolor{currentfill}%
\pgfsetfillopacity{0.550135}%
\pgfsetlinewidth{1.003750pt}%
\definecolor{currentstroke}{rgb}{0.121569,0.466667,0.705882}%
\pgfsetstrokecolor{currentstroke}%
\pgfsetstrokeopacity{0.550135}%
\pgfsetdash{}{0pt}%
\pgfpathmoveto{\pgfqpoint{1.126484in}{1.528507in}}%
\pgfpathcurveto{\pgfqpoint{1.134721in}{1.528507in}}{\pgfqpoint{1.142621in}{1.531779in}}{\pgfqpoint{1.148445in}{1.537603in}}%
\pgfpathcurveto{\pgfqpoint{1.154268in}{1.543427in}}{\pgfqpoint{1.157541in}{1.551327in}}{\pgfqpoint{1.157541in}{1.559564in}}%
\pgfpathcurveto{\pgfqpoint{1.157541in}{1.567800in}}{\pgfqpoint{1.154268in}{1.575700in}}{\pgfqpoint{1.148445in}{1.581524in}}%
\pgfpathcurveto{\pgfqpoint{1.142621in}{1.587348in}}{\pgfqpoint{1.134721in}{1.590620in}}{\pgfqpoint{1.126484in}{1.590620in}}%
\pgfpathcurveto{\pgfqpoint{1.118248in}{1.590620in}}{\pgfqpoint{1.110348in}{1.587348in}}{\pgfqpoint{1.104524in}{1.581524in}}%
\pgfpathcurveto{\pgfqpoint{1.098700in}{1.575700in}}{\pgfqpoint{1.095428in}{1.567800in}}{\pgfqpoint{1.095428in}{1.559564in}}%
\pgfpathcurveto{\pgfqpoint{1.095428in}{1.551327in}}{\pgfqpoint{1.098700in}{1.543427in}}{\pgfqpoint{1.104524in}{1.537603in}}%
\pgfpathcurveto{\pgfqpoint{1.110348in}{1.531779in}}{\pgfqpoint{1.118248in}{1.528507in}}{\pgfqpoint{1.126484in}{1.528507in}}%
\pgfpathclose%
\pgfusepath{stroke,fill}%
\end{pgfscope}%
\begin{pgfscope}%
\pgfpathrectangle{\pgfqpoint{0.100000in}{0.212622in}}{\pgfqpoint{3.696000in}{3.696000in}}%
\pgfusepath{clip}%
\pgfsetbuttcap%
\pgfsetroundjoin%
\definecolor{currentfill}{rgb}{0.121569,0.466667,0.705882}%
\pgfsetfillcolor{currentfill}%
\pgfsetfillopacity{0.552649}%
\pgfsetlinewidth{1.003750pt}%
\definecolor{currentstroke}{rgb}{0.121569,0.466667,0.705882}%
\pgfsetstrokecolor{currentstroke}%
\pgfsetstrokeopacity{0.552649}%
\pgfsetdash{}{0pt}%
\pgfpathmoveto{\pgfqpoint{2.077942in}{1.867875in}}%
\pgfpathcurveto{\pgfqpoint{2.086178in}{1.867875in}}{\pgfqpoint{2.094079in}{1.871147in}}{\pgfqpoint{2.099902in}{1.876971in}}%
\pgfpathcurveto{\pgfqpoint{2.105726in}{1.882795in}}{\pgfqpoint{2.108999in}{1.890695in}}{\pgfqpoint{2.108999in}{1.898931in}}%
\pgfpathcurveto{\pgfqpoint{2.108999in}{1.907168in}}{\pgfqpoint{2.105726in}{1.915068in}}{\pgfqpoint{2.099902in}{1.920892in}}%
\pgfpathcurveto{\pgfqpoint{2.094079in}{1.926715in}}{\pgfqpoint{2.086178in}{1.929988in}}{\pgfqpoint{2.077942in}{1.929988in}}%
\pgfpathcurveto{\pgfqpoint{2.069706in}{1.929988in}}{\pgfqpoint{2.061806in}{1.926715in}}{\pgfqpoint{2.055982in}{1.920892in}}%
\pgfpathcurveto{\pgfqpoint{2.050158in}{1.915068in}}{\pgfqpoint{2.046886in}{1.907168in}}{\pgfqpoint{2.046886in}{1.898931in}}%
\pgfpathcurveto{\pgfqpoint{2.046886in}{1.890695in}}{\pgfqpoint{2.050158in}{1.882795in}}{\pgfqpoint{2.055982in}{1.876971in}}%
\pgfpathcurveto{\pgfqpoint{2.061806in}{1.871147in}}{\pgfqpoint{2.069706in}{1.867875in}}{\pgfqpoint{2.077942in}{1.867875in}}%
\pgfpathclose%
\pgfusepath{stroke,fill}%
\end{pgfscope}%
\begin{pgfscope}%
\pgfpathrectangle{\pgfqpoint{0.100000in}{0.212622in}}{\pgfqpoint{3.696000in}{3.696000in}}%
\pgfusepath{clip}%
\pgfsetbuttcap%
\pgfsetroundjoin%
\definecolor{currentfill}{rgb}{0.121569,0.466667,0.705882}%
\pgfsetfillcolor{currentfill}%
\pgfsetfillopacity{0.553064}%
\pgfsetlinewidth{1.003750pt}%
\definecolor{currentstroke}{rgb}{0.121569,0.466667,0.705882}%
\pgfsetstrokecolor{currentstroke}%
\pgfsetstrokeopacity{0.553064}%
\pgfsetdash{}{0pt}%
\pgfpathmoveto{\pgfqpoint{1.115897in}{1.522141in}}%
\pgfpathcurveto{\pgfqpoint{1.124133in}{1.522141in}}{\pgfqpoint{1.132033in}{1.525414in}}{\pgfqpoint{1.137857in}{1.531238in}}%
\pgfpathcurveto{\pgfqpoint{1.143681in}{1.537061in}}{\pgfqpoint{1.146953in}{1.544962in}}{\pgfqpoint{1.146953in}{1.553198in}}%
\pgfpathcurveto{\pgfqpoint{1.146953in}{1.561434in}}{\pgfqpoint{1.143681in}{1.569334in}}{\pgfqpoint{1.137857in}{1.575158in}}%
\pgfpathcurveto{\pgfqpoint{1.132033in}{1.580982in}}{\pgfqpoint{1.124133in}{1.584254in}}{\pgfqpoint{1.115897in}{1.584254in}}%
\pgfpathcurveto{\pgfqpoint{1.107660in}{1.584254in}}{\pgfqpoint{1.099760in}{1.580982in}}{\pgfqpoint{1.093936in}{1.575158in}}%
\pgfpathcurveto{\pgfqpoint{1.088112in}{1.569334in}}{\pgfqpoint{1.084840in}{1.561434in}}{\pgfqpoint{1.084840in}{1.553198in}}%
\pgfpathcurveto{\pgfqpoint{1.084840in}{1.544962in}}{\pgfqpoint{1.088112in}{1.537061in}}{\pgfqpoint{1.093936in}{1.531238in}}%
\pgfpathcurveto{\pgfqpoint{1.099760in}{1.525414in}}{\pgfqpoint{1.107660in}{1.522141in}}{\pgfqpoint{1.115897in}{1.522141in}}%
\pgfpathclose%
\pgfusepath{stroke,fill}%
\end{pgfscope}%
\begin{pgfscope}%
\pgfpathrectangle{\pgfqpoint{0.100000in}{0.212622in}}{\pgfqpoint{3.696000in}{3.696000in}}%
\pgfusepath{clip}%
\pgfsetbuttcap%
\pgfsetroundjoin%
\definecolor{currentfill}{rgb}{0.121569,0.466667,0.705882}%
\pgfsetfillcolor{currentfill}%
\pgfsetfillopacity{0.555149}%
\pgfsetlinewidth{1.003750pt}%
\definecolor{currentstroke}{rgb}{0.121569,0.466667,0.705882}%
\pgfsetstrokecolor{currentstroke}%
\pgfsetstrokeopacity{0.555149}%
\pgfsetdash{}{0pt}%
\pgfpathmoveto{\pgfqpoint{1.108754in}{1.510773in}}%
\pgfpathcurveto{\pgfqpoint{1.116990in}{1.510773in}}{\pgfqpoint{1.124890in}{1.514045in}}{\pgfqpoint{1.130714in}{1.519869in}}%
\pgfpathcurveto{\pgfqpoint{1.136538in}{1.525693in}}{\pgfqpoint{1.139810in}{1.533593in}}{\pgfqpoint{1.139810in}{1.541829in}}%
\pgfpathcurveto{\pgfqpoint{1.139810in}{1.550065in}}{\pgfqpoint{1.136538in}{1.557965in}}{\pgfqpoint{1.130714in}{1.563789in}}%
\pgfpathcurveto{\pgfqpoint{1.124890in}{1.569613in}}{\pgfqpoint{1.116990in}{1.572886in}}{\pgfqpoint{1.108754in}{1.572886in}}%
\pgfpathcurveto{\pgfqpoint{1.100517in}{1.572886in}}{\pgfqpoint{1.092617in}{1.569613in}}{\pgfqpoint{1.086793in}{1.563789in}}%
\pgfpathcurveto{\pgfqpoint{1.080969in}{1.557965in}}{\pgfqpoint{1.077697in}{1.550065in}}{\pgfqpoint{1.077697in}{1.541829in}}%
\pgfpathcurveto{\pgfqpoint{1.077697in}{1.533593in}}{\pgfqpoint{1.080969in}{1.525693in}}{\pgfqpoint{1.086793in}{1.519869in}}%
\pgfpathcurveto{\pgfqpoint{1.092617in}{1.514045in}}{\pgfqpoint{1.100517in}{1.510773in}}{\pgfqpoint{1.108754in}{1.510773in}}%
\pgfpathclose%
\pgfusepath{stroke,fill}%
\end{pgfscope}%
\begin{pgfscope}%
\pgfpathrectangle{\pgfqpoint{0.100000in}{0.212622in}}{\pgfqpoint{3.696000in}{3.696000in}}%
\pgfusepath{clip}%
\pgfsetbuttcap%
\pgfsetroundjoin%
\definecolor{currentfill}{rgb}{0.121569,0.466667,0.705882}%
\pgfsetfillcolor{currentfill}%
\pgfsetfillopacity{0.556139}%
\pgfsetlinewidth{1.003750pt}%
\definecolor{currentstroke}{rgb}{0.121569,0.466667,0.705882}%
\pgfsetstrokecolor{currentstroke}%
\pgfsetstrokeopacity{0.556139}%
\pgfsetdash{}{0pt}%
\pgfpathmoveto{\pgfqpoint{1.100439in}{1.501123in}}%
\pgfpathcurveto{\pgfqpoint{1.108675in}{1.501123in}}{\pgfqpoint{1.116575in}{1.504395in}}{\pgfqpoint{1.122399in}{1.510219in}}%
\pgfpathcurveto{\pgfqpoint{1.128223in}{1.516043in}}{\pgfqpoint{1.131495in}{1.523943in}}{\pgfqpoint{1.131495in}{1.532180in}}%
\pgfpathcurveto{\pgfqpoint{1.131495in}{1.540416in}}{\pgfqpoint{1.128223in}{1.548316in}}{\pgfqpoint{1.122399in}{1.554140in}}%
\pgfpathcurveto{\pgfqpoint{1.116575in}{1.559964in}}{\pgfqpoint{1.108675in}{1.563236in}}{\pgfqpoint{1.100439in}{1.563236in}}%
\pgfpathcurveto{\pgfqpoint{1.092203in}{1.563236in}}{\pgfqpoint{1.084303in}{1.559964in}}{\pgfqpoint{1.078479in}{1.554140in}}%
\pgfpathcurveto{\pgfqpoint{1.072655in}{1.548316in}}{\pgfqpoint{1.069382in}{1.540416in}}{\pgfqpoint{1.069382in}{1.532180in}}%
\pgfpathcurveto{\pgfqpoint{1.069382in}{1.523943in}}{\pgfqpoint{1.072655in}{1.516043in}}{\pgfqpoint{1.078479in}{1.510219in}}%
\pgfpathcurveto{\pgfqpoint{1.084303in}{1.504395in}}{\pgfqpoint{1.092203in}{1.501123in}}{\pgfqpoint{1.100439in}{1.501123in}}%
\pgfpathclose%
\pgfusepath{stroke,fill}%
\end{pgfscope}%
\begin{pgfscope}%
\pgfpathrectangle{\pgfqpoint{0.100000in}{0.212622in}}{\pgfqpoint{3.696000in}{3.696000in}}%
\pgfusepath{clip}%
\pgfsetbuttcap%
\pgfsetroundjoin%
\definecolor{currentfill}{rgb}{0.121569,0.466667,0.705882}%
\pgfsetfillcolor{currentfill}%
\pgfsetfillopacity{0.557455}%
\pgfsetlinewidth{1.003750pt}%
\definecolor{currentstroke}{rgb}{0.121569,0.466667,0.705882}%
\pgfsetstrokecolor{currentstroke}%
\pgfsetstrokeopacity{0.557455}%
\pgfsetdash{}{0pt}%
\pgfpathmoveto{\pgfqpoint{2.081079in}{1.864326in}}%
\pgfpathcurveto{\pgfqpoint{2.089315in}{1.864326in}}{\pgfqpoint{2.097215in}{1.867598in}}{\pgfqpoint{2.103039in}{1.873422in}}%
\pgfpathcurveto{\pgfqpoint{2.108863in}{1.879246in}}{\pgfqpoint{2.112135in}{1.887146in}}{\pgfqpoint{2.112135in}{1.895382in}}%
\pgfpathcurveto{\pgfqpoint{2.112135in}{1.903619in}}{\pgfqpoint{2.108863in}{1.911519in}}{\pgfqpoint{2.103039in}{1.917343in}}%
\pgfpathcurveto{\pgfqpoint{2.097215in}{1.923167in}}{\pgfqpoint{2.089315in}{1.926439in}}{\pgfqpoint{2.081079in}{1.926439in}}%
\pgfpathcurveto{\pgfqpoint{2.072842in}{1.926439in}}{\pgfqpoint{2.064942in}{1.923167in}}{\pgfqpoint{2.059118in}{1.917343in}}%
\pgfpathcurveto{\pgfqpoint{2.053294in}{1.911519in}}{\pgfqpoint{2.050022in}{1.903619in}}{\pgfqpoint{2.050022in}{1.895382in}}%
\pgfpathcurveto{\pgfqpoint{2.050022in}{1.887146in}}{\pgfqpoint{2.053294in}{1.879246in}}{\pgfqpoint{2.059118in}{1.873422in}}%
\pgfpathcurveto{\pgfqpoint{2.064942in}{1.867598in}}{\pgfqpoint{2.072842in}{1.864326in}}{\pgfqpoint{2.081079in}{1.864326in}}%
\pgfpathclose%
\pgfusepath{stroke,fill}%
\end{pgfscope}%
\begin{pgfscope}%
\pgfpathrectangle{\pgfqpoint{0.100000in}{0.212622in}}{\pgfqpoint{3.696000in}{3.696000in}}%
\pgfusepath{clip}%
\pgfsetbuttcap%
\pgfsetroundjoin%
\definecolor{currentfill}{rgb}{0.121569,0.466667,0.705882}%
\pgfsetfillcolor{currentfill}%
\pgfsetfillopacity{0.558108}%
\pgfsetlinewidth{1.003750pt}%
\definecolor{currentstroke}{rgb}{0.121569,0.466667,0.705882}%
\pgfsetstrokecolor{currentstroke}%
\pgfsetstrokeopacity{0.558108}%
\pgfsetdash{}{0pt}%
\pgfpathmoveto{\pgfqpoint{1.093551in}{1.496456in}}%
\pgfpathcurveto{\pgfqpoint{1.101787in}{1.496456in}}{\pgfqpoint{1.109687in}{1.499728in}}{\pgfqpoint{1.115511in}{1.505552in}}%
\pgfpathcurveto{\pgfqpoint{1.121335in}{1.511376in}}{\pgfqpoint{1.124607in}{1.519276in}}{\pgfqpoint{1.124607in}{1.527512in}}%
\pgfpathcurveto{\pgfqpoint{1.124607in}{1.535749in}}{\pgfqpoint{1.121335in}{1.543649in}}{\pgfqpoint{1.115511in}{1.549473in}}%
\pgfpathcurveto{\pgfqpoint{1.109687in}{1.555296in}}{\pgfqpoint{1.101787in}{1.558569in}}{\pgfqpoint{1.093551in}{1.558569in}}%
\pgfpathcurveto{\pgfqpoint{1.085315in}{1.558569in}}{\pgfqpoint{1.077414in}{1.555296in}}{\pgfqpoint{1.071591in}{1.549473in}}%
\pgfpathcurveto{\pgfqpoint{1.065767in}{1.543649in}}{\pgfqpoint{1.062494in}{1.535749in}}{\pgfqpoint{1.062494in}{1.527512in}}%
\pgfpathcurveto{\pgfqpoint{1.062494in}{1.519276in}}{\pgfqpoint{1.065767in}{1.511376in}}{\pgfqpoint{1.071591in}{1.505552in}}%
\pgfpathcurveto{\pgfqpoint{1.077414in}{1.499728in}}{\pgfqpoint{1.085315in}{1.496456in}}{\pgfqpoint{1.093551in}{1.496456in}}%
\pgfpathclose%
\pgfusepath{stroke,fill}%
\end{pgfscope}%
\begin{pgfscope}%
\pgfpathrectangle{\pgfqpoint{0.100000in}{0.212622in}}{\pgfqpoint{3.696000in}{3.696000in}}%
\pgfusepath{clip}%
\pgfsetbuttcap%
\pgfsetroundjoin%
\definecolor{currentfill}{rgb}{0.121569,0.466667,0.705882}%
\pgfsetfillcolor{currentfill}%
\pgfsetfillopacity{0.559588}%
\pgfsetlinewidth{1.003750pt}%
\definecolor{currentstroke}{rgb}{0.121569,0.466667,0.705882}%
\pgfsetstrokecolor{currentstroke}%
\pgfsetstrokeopacity{0.559588}%
\pgfsetdash{}{0pt}%
\pgfpathmoveto{\pgfqpoint{1.089065in}{1.488428in}}%
\pgfpathcurveto{\pgfqpoint{1.097301in}{1.488428in}}{\pgfqpoint{1.105202in}{1.491701in}}{\pgfqpoint{1.111025in}{1.497525in}}%
\pgfpathcurveto{\pgfqpoint{1.116849in}{1.503349in}}{\pgfqpoint{1.120122in}{1.511249in}}{\pgfqpoint{1.120122in}{1.519485in}}%
\pgfpathcurveto{\pgfqpoint{1.120122in}{1.527721in}}{\pgfqpoint{1.116849in}{1.535621in}}{\pgfqpoint{1.111025in}{1.541445in}}%
\pgfpathcurveto{\pgfqpoint{1.105202in}{1.547269in}}{\pgfqpoint{1.097301in}{1.550541in}}{\pgfqpoint{1.089065in}{1.550541in}}%
\pgfpathcurveto{\pgfqpoint{1.080829in}{1.550541in}}{\pgfqpoint{1.072929in}{1.547269in}}{\pgfqpoint{1.067105in}{1.541445in}}%
\pgfpathcurveto{\pgfqpoint{1.061281in}{1.535621in}}{\pgfqpoint{1.058009in}{1.527721in}}{\pgfqpoint{1.058009in}{1.519485in}}%
\pgfpathcurveto{\pgfqpoint{1.058009in}{1.511249in}}{\pgfqpoint{1.061281in}{1.503349in}}{\pgfqpoint{1.067105in}{1.497525in}}%
\pgfpathcurveto{\pgfqpoint{1.072929in}{1.491701in}}{\pgfqpoint{1.080829in}{1.488428in}}{\pgfqpoint{1.089065in}{1.488428in}}%
\pgfpathclose%
\pgfusepath{stroke,fill}%
\end{pgfscope}%
\begin{pgfscope}%
\pgfpathrectangle{\pgfqpoint{0.100000in}{0.212622in}}{\pgfqpoint{3.696000in}{3.696000in}}%
\pgfusepath{clip}%
\pgfsetbuttcap%
\pgfsetroundjoin%
\definecolor{currentfill}{rgb}{0.121569,0.466667,0.705882}%
\pgfsetfillcolor{currentfill}%
\pgfsetfillopacity{0.560766}%
\pgfsetlinewidth{1.003750pt}%
\definecolor{currentstroke}{rgb}{0.121569,0.466667,0.705882}%
\pgfsetstrokecolor{currentstroke}%
\pgfsetstrokeopacity{0.560766}%
\pgfsetdash{}{0pt}%
\pgfpathmoveto{\pgfqpoint{1.085137in}{1.484923in}}%
\pgfpathcurveto{\pgfqpoint{1.093373in}{1.484923in}}{\pgfqpoint{1.101273in}{1.488195in}}{\pgfqpoint{1.107097in}{1.494019in}}%
\pgfpathcurveto{\pgfqpoint{1.112921in}{1.499843in}}{\pgfqpoint{1.116193in}{1.507743in}}{\pgfqpoint{1.116193in}{1.515979in}}%
\pgfpathcurveto{\pgfqpoint{1.116193in}{1.524216in}}{\pgfqpoint{1.112921in}{1.532116in}}{\pgfqpoint{1.107097in}{1.537940in}}%
\pgfpathcurveto{\pgfqpoint{1.101273in}{1.543763in}}{\pgfqpoint{1.093373in}{1.547036in}}{\pgfqpoint{1.085137in}{1.547036in}}%
\pgfpathcurveto{\pgfqpoint{1.076900in}{1.547036in}}{\pgfqpoint{1.069000in}{1.543763in}}{\pgfqpoint{1.063176in}{1.537940in}}%
\pgfpathcurveto{\pgfqpoint{1.057352in}{1.532116in}}{\pgfqpoint{1.054080in}{1.524216in}}{\pgfqpoint{1.054080in}{1.515979in}}%
\pgfpathcurveto{\pgfqpoint{1.054080in}{1.507743in}}{\pgfqpoint{1.057352in}{1.499843in}}{\pgfqpoint{1.063176in}{1.494019in}}%
\pgfpathcurveto{\pgfqpoint{1.069000in}{1.488195in}}{\pgfqpoint{1.076900in}{1.484923in}}{\pgfqpoint{1.085137in}{1.484923in}}%
\pgfpathclose%
\pgfusepath{stroke,fill}%
\end{pgfscope}%
\begin{pgfscope}%
\pgfpathrectangle{\pgfqpoint{0.100000in}{0.212622in}}{\pgfqpoint{3.696000in}{3.696000in}}%
\pgfusepath{clip}%
\pgfsetbuttcap%
\pgfsetroundjoin%
\definecolor{currentfill}{rgb}{0.121569,0.466667,0.705882}%
\pgfsetfillcolor{currentfill}%
\pgfsetfillopacity{0.562649}%
\pgfsetlinewidth{1.003750pt}%
\definecolor{currentstroke}{rgb}{0.121569,0.466667,0.705882}%
\pgfsetstrokecolor{currentstroke}%
\pgfsetstrokeopacity{0.562649}%
\pgfsetdash{}{0pt}%
\pgfpathmoveto{\pgfqpoint{2.083961in}{1.860722in}}%
\pgfpathcurveto{\pgfqpoint{2.092197in}{1.860722in}}{\pgfqpoint{2.100097in}{1.863995in}}{\pgfqpoint{2.105921in}{1.869819in}}%
\pgfpathcurveto{\pgfqpoint{2.111745in}{1.875643in}}{\pgfqpoint{2.115017in}{1.883543in}}{\pgfqpoint{2.115017in}{1.891779in}}%
\pgfpathcurveto{\pgfqpoint{2.115017in}{1.900015in}}{\pgfqpoint{2.111745in}{1.907915in}}{\pgfqpoint{2.105921in}{1.913739in}}%
\pgfpathcurveto{\pgfqpoint{2.100097in}{1.919563in}}{\pgfqpoint{2.092197in}{1.922835in}}{\pgfqpoint{2.083961in}{1.922835in}}%
\pgfpathcurveto{\pgfqpoint{2.075725in}{1.922835in}}{\pgfqpoint{2.067825in}{1.919563in}}{\pgfqpoint{2.062001in}{1.913739in}}%
\pgfpathcurveto{\pgfqpoint{2.056177in}{1.907915in}}{\pgfqpoint{2.052904in}{1.900015in}}{\pgfqpoint{2.052904in}{1.891779in}}%
\pgfpathcurveto{\pgfqpoint{2.052904in}{1.883543in}}{\pgfqpoint{2.056177in}{1.875643in}}{\pgfqpoint{2.062001in}{1.869819in}}%
\pgfpathcurveto{\pgfqpoint{2.067825in}{1.863995in}}{\pgfqpoint{2.075725in}{1.860722in}}{\pgfqpoint{2.083961in}{1.860722in}}%
\pgfpathclose%
\pgfusepath{stroke,fill}%
\end{pgfscope}%
\begin{pgfscope}%
\pgfpathrectangle{\pgfqpoint{0.100000in}{0.212622in}}{\pgfqpoint{3.696000in}{3.696000in}}%
\pgfusepath{clip}%
\pgfsetbuttcap%
\pgfsetroundjoin%
\definecolor{currentfill}{rgb}{0.121569,0.466667,0.705882}%
\pgfsetfillcolor{currentfill}%
\pgfsetfillopacity{0.563109}%
\pgfsetlinewidth{1.003750pt}%
\definecolor{currentstroke}{rgb}{0.121569,0.466667,0.705882}%
\pgfsetstrokecolor{currentstroke}%
\pgfsetstrokeopacity{0.563109}%
\pgfsetdash{}{0pt}%
\pgfpathmoveto{\pgfqpoint{1.078363in}{1.479314in}}%
\pgfpathcurveto{\pgfqpoint{1.086599in}{1.479314in}}{\pgfqpoint{1.094499in}{1.482586in}}{\pgfqpoint{1.100323in}{1.488410in}}%
\pgfpathcurveto{\pgfqpoint{1.106147in}{1.494234in}}{\pgfqpoint{1.109419in}{1.502134in}}{\pgfqpoint{1.109419in}{1.510371in}}%
\pgfpathcurveto{\pgfqpoint{1.109419in}{1.518607in}}{\pgfqpoint{1.106147in}{1.526507in}}{\pgfqpoint{1.100323in}{1.532331in}}%
\pgfpathcurveto{\pgfqpoint{1.094499in}{1.538155in}}{\pgfqpoint{1.086599in}{1.541427in}}{\pgfqpoint{1.078363in}{1.541427in}}%
\pgfpathcurveto{\pgfqpoint{1.070126in}{1.541427in}}{\pgfqpoint{1.062226in}{1.538155in}}{\pgfqpoint{1.056402in}{1.532331in}}%
\pgfpathcurveto{\pgfqpoint{1.050578in}{1.526507in}}{\pgfqpoint{1.047306in}{1.518607in}}{\pgfqpoint{1.047306in}{1.510371in}}%
\pgfpathcurveto{\pgfqpoint{1.047306in}{1.502134in}}{\pgfqpoint{1.050578in}{1.494234in}}{\pgfqpoint{1.056402in}{1.488410in}}%
\pgfpathcurveto{\pgfqpoint{1.062226in}{1.482586in}}{\pgfqpoint{1.070126in}{1.479314in}}{\pgfqpoint{1.078363in}{1.479314in}}%
\pgfpathclose%
\pgfusepath{stroke,fill}%
\end{pgfscope}%
\begin{pgfscope}%
\pgfpathrectangle{\pgfqpoint{0.100000in}{0.212622in}}{\pgfqpoint{3.696000in}{3.696000in}}%
\pgfusepath{clip}%
\pgfsetbuttcap%
\pgfsetroundjoin%
\definecolor{currentfill}{rgb}{0.121569,0.466667,0.705882}%
\pgfsetfillcolor{currentfill}%
\pgfsetfillopacity{0.566754}%
\pgfsetlinewidth{1.003750pt}%
\definecolor{currentstroke}{rgb}{0.121569,0.466667,0.705882}%
\pgfsetstrokecolor{currentstroke}%
\pgfsetstrokeopacity{0.566754}%
\pgfsetdash{}{0pt}%
\pgfpathmoveto{\pgfqpoint{1.067612in}{1.463442in}}%
\pgfpathcurveto{\pgfqpoint{1.075849in}{1.463442in}}{\pgfqpoint{1.083749in}{1.466714in}}{\pgfqpoint{1.089573in}{1.472538in}}%
\pgfpathcurveto{\pgfqpoint{1.095397in}{1.478362in}}{\pgfqpoint{1.098669in}{1.486262in}}{\pgfqpoint{1.098669in}{1.494498in}}%
\pgfpathcurveto{\pgfqpoint{1.098669in}{1.502735in}}{\pgfqpoint{1.095397in}{1.510635in}}{\pgfqpoint{1.089573in}{1.516459in}}%
\pgfpathcurveto{\pgfqpoint{1.083749in}{1.522283in}}{\pgfqpoint{1.075849in}{1.525555in}}{\pgfqpoint{1.067612in}{1.525555in}}%
\pgfpathcurveto{\pgfqpoint{1.059376in}{1.525555in}}{\pgfqpoint{1.051476in}{1.522283in}}{\pgfqpoint{1.045652in}{1.516459in}}%
\pgfpathcurveto{\pgfqpoint{1.039828in}{1.510635in}}{\pgfqpoint{1.036556in}{1.502735in}}{\pgfqpoint{1.036556in}{1.494498in}}%
\pgfpathcurveto{\pgfqpoint{1.036556in}{1.486262in}}{\pgfqpoint{1.039828in}{1.478362in}}{\pgfqpoint{1.045652in}{1.472538in}}%
\pgfpathcurveto{\pgfqpoint{1.051476in}{1.466714in}}{\pgfqpoint{1.059376in}{1.463442in}}{\pgfqpoint{1.067612in}{1.463442in}}%
\pgfpathclose%
\pgfusepath{stroke,fill}%
\end{pgfscope}%
\begin{pgfscope}%
\pgfpathrectangle{\pgfqpoint{0.100000in}{0.212622in}}{\pgfqpoint{3.696000in}{3.696000in}}%
\pgfusepath{clip}%
\pgfsetbuttcap%
\pgfsetroundjoin%
\definecolor{currentfill}{rgb}{0.121569,0.466667,0.705882}%
\pgfsetfillcolor{currentfill}%
\pgfsetfillopacity{0.568627}%
\pgfsetlinewidth{1.003750pt}%
\definecolor{currentstroke}{rgb}{0.121569,0.466667,0.705882}%
\pgfsetstrokecolor{currentstroke}%
\pgfsetstrokeopacity{0.568627}%
\pgfsetdash{}{0pt}%
\pgfpathmoveto{\pgfqpoint{2.087834in}{1.856452in}}%
\pgfpathcurveto{\pgfqpoint{2.096070in}{1.856452in}}{\pgfqpoint{2.103970in}{1.859724in}}{\pgfqpoint{2.109794in}{1.865548in}}%
\pgfpathcurveto{\pgfqpoint{2.115618in}{1.871372in}}{\pgfqpoint{2.118891in}{1.879272in}}{\pgfqpoint{2.118891in}{1.887508in}}%
\pgfpathcurveto{\pgfqpoint{2.118891in}{1.895744in}}{\pgfqpoint{2.115618in}{1.903645in}}{\pgfqpoint{2.109794in}{1.909468in}}%
\pgfpathcurveto{\pgfqpoint{2.103970in}{1.915292in}}{\pgfqpoint{2.096070in}{1.918565in}}{\pgfqpoint{2.087834in}{1.918565in}}%
\pgfpathcurveto{\pgfqpoint{2.079598in}{1.918565in}}{\pgfqpoint{2.071698in}{1.915292in}}{\pgfqpoint{2.065874in}{1.909468in}}%
\pgfpathcurveto{\pgfqpoint{2.060050in}{1.903645in}}{\pgfqpoint{2.056778in}{1.895744in}}{\pgfqpoint{2.056778in}{1.887508in}}%
\pgfpathcurveto{\pgfqpoint{2.056778in}{1.879272in}}{\pgfqpoint{2.060050in}{1.871372in}}{\pgfqpoint{2.065874in}{1.865548in}}%
\pgfpathcurveto{\pgfqpoint{2.071698in}{1.859724in}}{\pgfqpoint{2.079598in}{1.856452in}}{\pgfqpoint{2.087834in}{1.856452in}}%
\pgfpathclose%
\pgfusepath{stroke,fill}%
\end{pgfscope}%
\begin{pgfscope}%
\pgfpathrectangle{\pgfqpoint{0.100000in}{0.212622in}}{\pgfqpoint{3.696000in}{3.696000in}}%
\pgfusepath{clip}%
\pgfsetbuttcap%
\pgfsetroundjoin%
\definecolor{currentfill}{rgb}{0.121569,0.466667,0.705882}%
\pgfsetfillcolor{currentfill}%
\pgfsetfillopacity{0.569857}%
\pgfsetlinewidth{1.003750pt}%
\definecolor{currentstroke}{rgb}{0.121569,0.466667,0.705882}%
\pgfsetstrokecolor{currentstroke}%
\pgfsetstrokeopacity{0.569857}%
\pgfsetdash{}{0pt}%
\pgfpathmoveto{\pgfqpoint{1.056102in}{1.454956in}}%
\pgfpathcurveto{\pgfqpoint{1.064338in}{1.454956in}}{\pgfqpoint{1.072239in}{1.458228in}}{\pgfqpoint{1.078062in}{1.464052in}}%
\pgfpathcurveto{\pgfqpoint{1.083886in}{1.469876in}}{\pgfqpoint{1.087159in}{1.477776in}}{\pgfqpoint{1.087159in}{1.486012in}}%
\pgfpathcurveto{\pgfqpoint{1.087159in}{1.494249in}}{\pgfqpoint{1.083886in}{1.502149in}}{\pgfqpoint{1.078062in}{1.507973in}}%
\pgfpathcurveto{\pgfqpoint{1.072239in}{1.513797in}}{\pgfqpoint{1.064338in}{1.517069in}}{\pgfqpoint{1.056102in}{1.517069in}}%
\pgfpathcurveto{\pgfqpoint{1.047866in}{1.517069in}}{\pgfqpoint{1.039966in}{1.513797in}}{\pgfqpoint{1.034142in}{1.507973in}}%
\pgfpathcurveto{\pgfqpoint{1.028318in}{1.502149in}}{\pgfqpoint{1.025046in}{1.494249in}}{\pgfqpoint{1.025046in}{1.486012in}}%
\pgfpathcurveto{\pgfqpoint{1.025046in}{1.477776in}}{\pgfqpoint{1.028318in}{1.469876in}}{\pgfqpoint{1.034142in}{1.464052in}}%
\pgfpathcurveto{\pgfqpoint{1.039966in}{1.458228in}}{\pgfqpoint{1.047866in}{1.454956in}}{\pgfqpoint{1.056102in}{1.454956in}}%
\pgfpathclose%
\pgfusepath{stroke,fill}%
\end{pgfscope}%
\begin{pgfscope}%
\pgfpathrectangle{\pgfqpoint{0.100000in}{0.212622in}}{\pgfqpoint{3.696000in}{3.696000in}}%
\pgfusepath{clip}%
\pgfsetbuttcap%
\pgfsetroundjoin%
\definecolor{currentfill}{rgb}{0.121569,0.466667,0.705882}%
\pgfsetfillcolor{currentfill}%
\pgfsetfillopacity{0.571872}%
\pgfsetlinewidth{1.003750pt}%
\definecolor{currentstroke}{rgb}{0.121569,0.466667,0.705882}%
\pgfsetstrokecolor{currentstroke}%
\pgfsetstrokeopacity{0.571872}%
\pgfsetdash{}{0pt}%
\pgfpathmoveto{\pgfqpoint{2.089962in}{1.853774in}}%
\pgfpathcurveto{\pgfqpoint{2.098198in}{1.853774in}}{\pgfqpoint{2.106098in}{1.857046in}}{\pgfqpoint{2.111922in}{1.862870in}}%
\pgfpathcurveto{\pgfqpoint{2.117746in}{1.868694in}}{\pgfqpoint{2.121018in}{1.876594in}}{\pgfqpoint{2.121018in}{1.884830in}}%
\pgfpathcurveto{\pgfqpoint{2.121018in}{1.893067in}}{\pgfqpoint{2.117746in}{1.900967in}}{\pgfqpoint{2.111922in}{1.906791in}}%
\pgfpathcurveto{\pgfqpoint{2.106098in}{1.912615in}}{\pgfqpoint{2.098198in}{1.915887in}}{\pgfqpoint{2.089962in}{1.915887in}}%
\pgfpathcurveto{\pgfqpoint{2.081726in}{1.915887in}}{\pgfqpoint{2.073826in}{1.912615in}}{\pgfqpoint{2.068002in}{1.906791in}}%
\pgfpathcurveto{\pgfqpoint{2.062178in}{1.900967in}}{\pgfqpoint{2.058905in}{1.893067in}}{\pgfqpoint{2.058905in}{1.884830in}}%
\pgfpathcurveto{\pgfqpoint{2.058905in}{1.876594in}}{\pgfqpoint{2.062178in}{1.868694in}}{\pgfqpoint{2.068002in}{1.862870in}}%
\pgfpathcurveto{\pgfqpoint{2.073826in}{1.857046in}}{\pgfqpoint{2.081726in}{1.853774in}}{\pgfqpoint{2.089962in}{1.853774in}}%
\pgfpathclose%
\pgfusepath{stroke,fill}%
\end{pgfscope}%
\begin{pgfscope}%
\pgfpathrectangle{\pgfqpoint{0.100000in}{0.212622in}}{\pgfqpoint{3.696000in}{3.696000in}}%
\pgfusepath{clip}%
\pgfsetbuttcap%
\pgfsetroundjoin%
\definecolor{currentfill}{rgb}{0.121569,0.466667,0.705882}%
\pgfsetfillcolor{currentfill}%
\pgfsetfillopacity{0.572744}%
\pgfsetlinewidth{1.003750pt}%
\definecolor{currentstroke}{rgb}{0.121569,0.466667,0.705882}%
\pgfsetstrokecolor{currentstroke}%
\pgfsetstrokeopacity{0.572744}%
\pgfsetdash{}{0pt}%
\pgfpathmoveto{\pgfqpoint{1.045913in}{1.446992in}}%
\pgfpathcurveto{\pgfqpoint{1.054149in}{1.446992in}}{\pgfqpoint{1.062049in}{1.450264in}}{\pgfqpoint{1.067873in}{1.456088in}}%
\pgfpathcurveto{\pgfqpoint{1.073697in}{1.461912in}}{\pgfqpoint{1.076969in}{1.469812in}}{\pgfqpoint{1.076969in}{1.478049in}}%
\pgfpathcurveto{\pgfqpoint{1.076969in}{1.486285in}}{\pgfqpoint{1.073697in}{1.494185in}}{\pgfqpoint{1.067873in}{1.500009in}}%
\pgfpathcurveto{\pgfqpoint{1.062049in}{1.505833in}}{\pgfqpoint{1.054149in}{1.509105in}}{\pgfqpoint{1.045913in}{1.509105in}}%
\pgfpathcurveto{\pgfqpoint{1.037676in}{1.509105in}}{\pgfqpoint{1.029776in}{1.505833in}}{\pgfqpoint{1.023952in}{1.500009in}}%
\pgfpathcurveto{\pgfqpoint{1.018128in}{1.494185in}}{\pgfqpoint{1.014856in}{1.486285in}}{\pgfqpoint{1.014856in}{1.478049in}}%
\pgfpathcurveto{\pgfqpoint{1.014856in}{1.469812in}}{\pgfqpoint{1.018128in}{1.461912in}}{\pgfqpoint{1.023952in}{1.456088in}}%
\pgfpathcurveto{\pgfqpoint{1.029776in}{1.450264in}}{\pgfqpoint{1.037676in}{1.446992in}}{\pgfqpoint{1.045913in}{1.446992in}}%
\pgfpathclose%
\pgfusepath{stroke,fill}%
\end{pgfscope}%
\begin{pgfscope}%
\pgfpathrectangle{\pgfqpoint{0.100000in}{0.212622in}}{\pgfqpoint{3.696000in}{3.696000in}}%
\pgfusepath{clip}%
\pgfsetbuttcap%
\pgfsetroundjoin%
\definecolor{currentfill}{rgb}{0.121569,0.466667,0.705882}%
\pgfsetfillcolor{currentfill}%
\pgfsetfillopacity{0.575160}%
\pgfsetlinewidth{1.003750pt}%
\definecolor{currentstroke}{rgb}{0.121569,0.466667,0.705882}%
\pgfsetstrokecolor{currentstroke}%
\pgfsetstrokeopacity{0.575160}%
\pgfsetdash{}{0pt}%
\pgfpathmoveto{\pgfqpoint{1.038305in}{1.434304in}}%
\pgfpathcurveto{\pgfqpoint{1.046541in}{1.434304in}}{\pgfqpoint{1.054441in}{1.437576in}}{\pgfqpoint{1.060265in}{1.443400in}}%
\pgfpathcurveto{\pgfqpoint{1.066089in}{1.449224in}}{\pgfqpoint{1.069361in}{1.457124in}}{\pgfqpoint{1.069361in}{1.465360in}}%
\pgfpathcurveto{\pgfqpoint{1.069361in}{1.473596in}}{\pgfqpoint{1.066089in}{1.481496in}}{\pgfqpoint{1.060265in}{1.487320in}}%
\pgfpathcurveto{\pgfqpoint{1.054441in}{1.493144in}}{\pgfqpoint{1.046541in}{1.496417in}}{\pgfqpoint{1.038305in}{1.496417in}}%
\pgfpathcurveto{\pgfqpoint{1.030068in}{1.496417in}}{\pgfqpoint{1.022168in}{1.493144in}}{\pgfqpoint{1.016344in}{1.487320in}}%
\pgfpathcurveto{\pgfqpoint{1.010520in}{1.481496in}}{\pgfqpoint{1.007248in}{1.473596in}}{\pgfqpoint{1.007248in}{1.465360in}}%
\pgfpathcurveto{\pgfqpoint{1.007248in}{1.457124in}}{\pgfqpoint{1.010520in}{1.449224in}}{\pgfqpoint{1.016344in}{1.443400in}}%
\pgfpathcurveto{\pgfqpoint{1.022168in}{1.437576in}}{\pgfqpoint{1.030068in}{1.434304in}}{\pgfqpoint{1.038305in}{1.434304in}}%
\pgfpathclose%
\pgfusepath{stroke,fill}%
\end{pgfscope}%
\begin{pgfscope}%
\pgfpathrectangle{\pgfqpoint{0.100000in}{0.212622in}}{\pgfqpoint{3.696000in}{3.696000in}}%
\pgfusepath{clip}%
\pgfsetbuttcap%
\pgfsetroundjoin%
\definecolor{currentfill}{rgb}{0.121569,0.466667,0.705882}%
\pgfsetfillcolor{currentfill}%
\pgfsetfillopacity{0.575777}%
\pgfsetlinewidth{1.003750pt}%
\definecolor{currentstroke}{rgb}{0.121569,0.466667,0.705882}%
\pgfsetstrokecolor{currentstroke}%
\pgfsetstrokeopacity{0.575777}%
\pgfsetdash{}{0pt}%
\pgfpathmoveto{\pgfqpoint{2.092038in}{1.850586in}}%
\pgfpathcurveto{\pgfqpoint{2.100275in}{1.850586in}}{\pgfqpoint{2.108175in}{1.853859in}}{\pgfqpoint{2.113999in}{1.859683in}}%
\pgfpathcurveto{\pgfqpoint{2.119823in}{1.865507in}}{\pgfqpoint{2.123095in}{1.873407in}}{\pgfqpoint{2.123095in}{1.881643in}}%
\pgfpathcurveto{\pgfqpoint{2.123095in}{1.889879in}}{\pgfqpoint{2.119823in}{1.897779in}}{\pgfqpoint{2.113999in}{1.903603in}}%
\pgfpathcurveto{\pgfqpoint{2.108175in}{1.909427in}}{\pgfqpoint{2.100275in}{1.912699in}}{\pgfqpoint{2.092038in}{1.912699in}}%
\pgfpathcurveto{\pgfqpoint{2.083802in}{1.912699in}}{\pgfqpoint{2.075902in}{1.909427in}}{\pgfqpoint{2.070078in}{1.903603in}}%
\pgfpathcurveto{\pgfqpoint{2.064254in}{1.897779in}}{\pgfqpoint{2.060982in}{1.889879in}}{\pgfqpoint{2.060982in}{1.881643in}}%
\pgfpathcurveto{\pgfqpoint{2.060982in}{1.873407in}}{\pgfqpoint{2.064254in}{1.865507in}}{\pgfqpoint{2.070078in}{1.859683in}}%
\pgfpathcurveto{\pgfqpoint{2.075902in}{1.853859in}}{\pgfqpoint{2.083802in}{1.850586in}}{\pgfqpoint{2.092038in}{1.850586in}}%
\pgfpathclose%
\pgfusepath{stroke,fill}%
\end{pgfscope}%
\begin{pgfscope}%
\pgfpathrectangle{\pgfqpoint{0.100000in}{0.212622in}}{\pgfqpoint{3.696000in}{3.696000in}}%
\pgfusepath{clip}%
\pgfsetbuttcap%
\pgfsetroundjoin%
\definecolor{currentfill}{rgb}{0.121569,0.466667,0.705882}%
\pgfsetfillcolor{currentfill}%
\pgfsetfillopacity{0.577546}%
\pgfsetlinewidth{1.003750pt}%
\definecolor{currentstroke}{rgb}{0.121569,0.466667,0.705882}%
\pgfsetstrokecolor{currentstroke}%
\pgfsetstrokeopacity{0.577546}%
\pgfsetdash{}{0pt}%
\pgfpathmoveto{\pgfqpoint{1.030911in}{1.429349in}}%
\pgfpathcurveto{\pgfqpoint{1.039147in}{1.429349in}}{\pgfqpoint{1.047047in}{1.432622in}}{\pgfqpoint{1.052871in}{1.438445in}}%
\pgfpathcurveto{\pgfqpoint{1.058695in}{1.444269in}}{\pgfqpoint{1.061967in}{1.452169in}}{\pgfqpoint{1.061967in}{1.460406in}}%
\pgfpathcurveto{\pgfqpoint{1.061967in}{1.468642in}}{\pgfqpoint{1.058695in}{1.476542in}}{\pgfqpoint{1.052871in}{1.482366in}}%
\pgfpathcurveto{\pgfqpoint{1.047047in}{1.488190in}}{\pgfqpoint{1.039147in}{1.491462in}}{\pgfqpoint{1.030911in}{1.491462in}}%
\pgfpathcurveto{\pgfqpoint{1.022674in}{1.491462in}}{\pgfqpoint{1.014774in}{1.488190in}}{\pgfqpoint{1.008950in}{1.482366in}}%
\pgfpathcurveto{\pgfqpoint{1.003127in}{1.476542in}}{\pgfqpoint{0.999854in}{1.468642in}}{\pgfqpoint{0.999854in}{1.460406in}}%
\pgfpathcurveto{\pgfqpoint{0.999854in}{1.452169in}}{\pgfqpoint{1.003127in}{1.444269in}}{\pgfqpoint{1.008950in}{1.438445in}}%
\pgfpathcurveto{\pgfqpoint{1.014774in}{1.432622in}}{\pgfqpoint{1.022674in}{1.429349in}}{\pgfqpoint{1.030911in}{1.429349in}}%
\pgfpathclose%
\pgfusepath{stroke,fill}%
\end{pgfscope}%
\begin{pgfscope}%
\pgfpathrectangle{\pgfqpoint{0.100000in}{0.212622in}}{\pgfqpoint{3.696000in}{3.696000in}}%
\pgfusepath{clip}%
\pgfsetbuttcap%
\pgfsetroundjoin%
\definecolor{currentfill}{rgb}{0.121569,0.466667,0.705882}%
\pgfsetfillcolor{currentfill}%
\pgfsetfillopacity{0.577945}%
\pgfsetlinewidth{1.003750pt}%
\definecolor{currentstroke}{rgb}{0.121569,0.466667,0.705882}%
\pgfsetstrokecolor{currentstroke}%
\pgfsetstrokeopacity{0.577945}%
\pgfsetdash{}{0pt}%
\pgfpathmoveto{\pgfqpoint{2.093486in}{1.849104in}}%
\pgfpathcurveto{\pgfqpoint{2.101722in}{1.849104in}}{\pgfqpoint{2.109623in}{1.852376in}}{\pgfqpoint{2.115446in}{1.858200in}}%
\pgfpathcurveto{\pgfqpoint{2.121270in}{1.864024in}}{\pgfqpoint{2.124543in}{1.871924in}}{\pgfqpoint{2.124543in}{1.880160in}}%
\pgfpathcurveto{\pgfqpoint{2.124543in}{1.888397in}}{\pgfqpoint{2.121270in}{1.896297in}}{\pgfqpoint{2.115446in}{1.902121in}}%
\pgfpathcurveto{\pgfqpoint{2.109623in}{1.907945in}}{\pgfqpoint{2.101722in}{1.911217in}}{\pgfqpoint{2.093486in}{1.911217in}}%
\pgfpathcurveto{\pgfqpoint{2.085250in}{1.911217in}}{\pgfqpoint{2.077350in}{1.907945in}}{\pgfqpoint{2.071526in}{1.902121in}}%
\pgfpathcurveto{\pgfqpoint{2.065702in}{1.896297in}}{\pgfqpoint{2.062430in}{1.888397in}}{\pgfqpoint{2.062430in}{1.880160in}}%
\pgfpathcurveto{\pgfqpoint{2.062430in}{1.871924in}}{\pgfqpoint{2.065702in}{1.864024in}}{\pgfqpoint{2.071526in}{1.858200in}}%
\pgfpathcurveto{\pgfqpoint{2.077350in}{1.852376in}}{\pgfqpoint{2.085250in}{1.849104in}}{\pgfqpoint{2.093486in}{1.849104in}}%
\pgfpathclose%
\pgfusepath{stroke,fill}%
\end{pgfscope}%
\begin{pgfscope}%
\pgfpathrectangle{\pgfqpoint{0.100000in}{0.212622in}}{\pgfqpoint{3.696000in}{3.696000in}}%
\pgfusepath{clip}%
\pgfsetbuttcap%
\pgfsetroundjoin%
\definecolor{currentfill}{rgb}{0.121569,0.466667,0.705882}%
\pgfsetfillcolor{currentfill}%
\pgfsetfillopacity{0.579096}%
\pgfsetlinewidth{1.003750pt}%
\definecolor{currentstroke}{rgb}{0.121569,0.466667,0.705882}%
\pgfsetstrokecolor{currentstroke}%
\pgfsetstrokeopacity{0.579096}%
\pgfsetdash{}{0pt}%
\pgfpathmoveto{\pgfqpoint{2.094417in}{1.848093in}}%
\pgfpathcurveto{\pgfqpoint{2.102653in}{1.848093in}}{\pgfqpoint{2.110553in}{1.851365in}}{\pgfqpoint{2.116377in}{1.857189in}}%
\pgfpathcurveto{\pgfqpoint{2.122201in}{1.863013in}}{\pgfqpoint{2.125474in}{1.870913in}}{\pgfqpoint{2.125474in}{1.879149in}}%
\pgfpathcurveto{\pgfqpoint{2.125474in}{1.887385in}}{\pgfqpoint{2.122201in}{1.895286in}}{\pgfqpoint{2.116377in}{1.901109in}}%
\pgfpathcurveto{\pgfqpoint{2.110553in}{1.906933in}}{\pgfqpoint{2.102653in}{1.910206in}}{\pgfqpoint{2.094417in}{1.910206in}}%
\pgfpathcurveto{\pgfqpoint{2.086181in}{1.910206in}}{\pgfqpoint{2.078281in}{1.906933in}}{\pgfqpoint{2.072457in}{1.901109in}}%
\pgfpathcurveto{\pgfqpoint{2.066633in}{1.895286in}}{\pgfqpoint{2.063361in}{1.887385in}}{\pgfqpoint{2.063361in}{1.879149in}}%
\pgfpathcurveto{\pgfqpoint{2.063361in}{1.870913in}}{\pgfqpoint{2.066633in}{1.863013in}}{\pgfqpoint{2.072457in}{1.857189in}}%
\pgfpathcurveto{\pgfqpoint{2.078281in}{1.851365in}}{\pgfqpoint{2.086181in}{1.848093in}}{\pgfqpoint{2.094417in}{1.848093in}}%
\pgfpathclose%
\pgfusepath{stroke,fill}%
\end{pgfscope}%
\begin{pgfscope}%
\pgfpathrectangle{\pgfqpoint{0.100000in}{0.212622in}}{\pgfqpoint{3.696000in}{3.696000in}}%
\pgfusepath{clip}%
\pgfsetbuttcap%
\pgfsetroundjoin%
\definecolor{currentfill}{rgb}{0.121569,0.466667,0.705882}%
\pgfsetfillcolor{currentfill}%
\pgfsetfillopacity{0.579652}%
\pgfsetlinewidth{1.003750pt}%
\definecolor{currentstroke}{rgb}{0.121569,0.466667,0.705882}%
\pgfsetstrokecolor{currentstroke}%
\pgfsetstrokeopacity{0.579652}%
\pgfsetdash{}{0pt}%
\pgfpathmoveto{\pgfqpoint{1.023691in}{1.425027in}}%
\pgfpathcurveto{\pgfqpoint{1.031927in}{1.425027in}}{\pgfqpoint{1.039827in}{1.428299in}}{\pgfqpoint{1.045651in}{1.434123in}}%
\pgfpathcurveto{\pgfqpoint{1.051475in}{1.439947in}}{\pgfqpoint{1.054748in}{1.447847in}}{\pgfqpoint{1.054748in}{1.456083in}}%
\pgfpathcurveto{\pgfqpoint{1.054748in}{1.464319in}}{\pgfqpoint{1.051475in}{1.472219in}}{\pgfqpoint{1.045651in}{1.478043in}}%
\pgfpathcurveto{\pgfqpoint{1.039827in}{1.483867in}}{\pgfqpoint{1.031927in}{1.487140in}}{\pgfqpoint{1.023691in}{1.487140in}}%
\pgfpathcurveto{\pgfqpoint{1.015455in}{1.487140in}}{\pgfqpoint{1.007555in}{1.483867in}}{\pgfqpoint{1.001731in}{1.478043in}}%
\pgfpathcurveto{\pgfqpoint{0.995907in}{1.472219in}}{\pgfqpoint{0.992635in}{1.464319in}}{\pgfqpoint{0.992635in}{1.456083in}}%
\pgfpathcurveto{\pgfqpoint{0.992635in}{1.447847in}}{\pgfqpoint{0.995907in}{1.439947in}}{\pgfqpoint{1.001731in}{1.434123in}}%
\pgfpathcurveto{\pgfqpoint{1.007555in}{1.428299in}}{\pgfqpoint{1.015455in}{1.425027in}}{\pgfqpoint{1.023691in}{1.425027in}}%
\pgfpathclose%
\pgfusepath{stroke,fill}%
\end{pgfscope}%
\begin{pgfscope}%
\pgfpathrectangle{\pgfqpoint{0.100000in}{0.212622in}}{\pgfqpoint{3.696000in}{3.696000in}}%
\pgfusepath{clip}%
\pgfsetbuttcap%
\pgfsetroundjoin%
\definecolor{currentfill}{rgb}{0.121569,0.466667,0.705882}%
\pgfsetfillcolor{currentfill}%
\pgfsetfillopacity{0.580425}%
\pgfsetlinewidth{1.003750pt}%
\definecolor{currentstroke}{rgb}{0.121569,0.466667,0.705882}%
\pgfsetstrokecolor{currentstroke}%
\pgfsetstrokeopacity{0.580425}%
\pgfsetdash{}{0pt}%
\pgfpathmoveto{\pgfqpoint{2.095112in}{1.846555in}}%
\pgfpathcurveto{\pgfqpoint{2.103349in}{1.846555in}}{\pgfqpoint{2.111249in}{1.849828in}}{\pgfqpoint{2.117073in}{1.855652in}}%
\pgfpathcurveto{\pgfqpoint{2.122897in}{1.861476in}}{\pgfqpoint{2.126169in}{1.869376in}}{\pgfqpoint{2.126169in}{1.877612in}}%
\pgfpathcurveto{\pgfqpoint{2.126169in}{1.885848in}}{\pgfqpoint{2.122897in}{1.893748in}}{\pgfqpoint{2.117073in}{1.899572in}}%
\pgfpathcurveto{\pgfqpoint{2.111249in}{1.905396in}}{\pgfqpoint{2.103349in}{1.908668in}}{\pgfqpoint{2.095112in}{1.908668in}}%
\pgfpathcurveto{\pgfqpoint{2.086876in}{1.908668in}}{\pgfqpoint{2.078976in}{1.905396in}}{\pgfqpoint{2.073152in}{1.899572in}}%
\pgfpathcurveto{\pgfqpoint{2.067328in}{1.893748in}}{\pgfqpoint{2.064056in}{1.885848in}}{\pgfqpoint{2.064056in}{1.877612in}}%
\pgfpathcurveto{\pgfqpoint{2.064056in}{1.869376in}}{\pgfqpoint{2.067328in}{1.861476in}}{\pgfqpoint{2.073152in}{1.855652in}}%
\pgfpathcurveto{\pgfqpoint{2.078976in}{1.849828in}}{\pgfqpoint{2.086876in}{1.846555in}}{\pgfqpoint{2.095112in}{1.846555in}}%
\pgfpathclose%
\pgfusepath{stroke,fill}%
\end{pgfscope}%
\begin{pgfscope}%
\pgfpathrectangle{\pgfqpoint{0.100000in}{0.212622in}}{\pgfqpoint{3.696000in}{3.696000in}}%
\pgfusepath{clip}%
\pgfsetbuttcap%
\pgfsetroundjoin%
\definecolor{currentfill}{rgb}{0.121569,0.466667,0.705882}%
\pgfsetfillcolor{currentfill}%
\pgfsetfillopacity{0.582137}%
\pgfsetlinewidth{1.003750pt}%
\definecolor{currentstroke}{rgb}{0.121569,0.466667,0.705882}%
\pgfsetstrokecolor{currentstroke}%
\pgfsetstrokeopacity{0.582137}%
\pgfsetdash{}{0pt}%
\pgfpathmoveto{\pgfqpoint{2.096322in}{1.844008in}}%
\pgfpathcurveto{\pgfqpoint{2.104558in}{1.844008in}}{\pgfqpoint{2.112458in}{1.847281in}}{\pgfqpoint{2.118282in}{1.853105in}}%
\pgfpathcurveto{\pgfqpoint{2.124106in}{1.858928in}}{\pgfqpoint{2.127378in}{1.866828in}}{\pgfqpoint{2.127378in}{1.875065in}}%
\pgfpathcurveto{\pgfqpoint{2.127378in}{1.883301in}}{\pgfqpoint{2.124106in}{1.891201in}}{\pgfqpoint{2.118282in}{1.897025in}}%
\pgfpathcurveto{\pgfqpoint{2.112458in}{1.902849in}}{\pgfqpoint{2.104558in}{1.906121in}}{\pgfqpoint{2.096322in}{1.906121in}}%
\pgfpathcurveto{\pgfqpoint{2.088086in}{1.906121in}}{\pgfqpoint{2.080186in}{1.902849in}}{\pgfqpoint{2.074362in}{1.897025in}}%
\pgfpathcurveto{\pgfqpoint{2.068538in}{1.891201in}}{\pgfqpoint{2.065265in}{1.883301in}}{\pgfqpoint{2.065265in}{1.875065in}}%
\pgfpathcurveto{\pgfqpoint{2.065265in}{1.866828in}}{\pgfqpoint{2.068538in}{1.858928in}}{\pgfqpoint{2.074362in}{1.853105in}}%
\pgfpathcurveto{\pgfqpoint{2.080186in}{1.847281in}}{\pgfqpoint{2.088086in}{1.844008in}}{\pgfqpoint{2.096322in}{1.844008in}}%
\pgfpathclose%
\pgfusepath{stroke,fill}%
\end{pgfscope}%
\begin{pgfscope}%
\pgfpathrectangle{\pgfqpoint{0.100000in}{0.212622in}}{\pgfqpoint{3.696000in}{3.696000in}}%
\pgfusepath{clip}%
\pgfsetbuttcap%
\pgfsetroundjoin%
\definecolor{currentfill}{rgb}{0.121569,0.466667,0.705882}%
\pgfsetfillcolor{currentfill}%
\pgfsetfillopacity{0.583125}%
\pgfsetlinewidth{1.003750pt}%
\definecolor{currentstroke}{rgb}{0.121569,0.466667,0.705882}%
\pgfsetstrokecolor{currentstroke}%
\pgfsetstrokeopacity{0.583125}%
\pgfsetdash{}{0pt}%
\pgfpathmoveto{\pgfqpoint{1.012738in}{1.411917in}}%
\pgfpathcurveto{\pgfqpoint{1.020974in}{1.411917in}}{\pgfqpoint{1.028874in}{1.415189in}}{\pgfqpoint{1.034698in}{1.421013in}}%
\pgfpathcurveto{\pgfqpoint{1.040522in}{1.426837in}}{\pgfqpoint{1.043795in}{1.434737in}}{\pgfqpoint{1.043795in}{1.442973in}}%
\pgfpathcurveto{\pgfqpoint{1.043795in}{1.451209in}}{\pgfqpoint{1.040522in}{1.459110in}}{\pgfqpoint{1.034698in}{1.464933in}}%
\pgfpathcurveto{\pgfqpoint{1.028874in}{1.470757in}}{\pgfqpoint{1.020974in}{1.474030in}}{\pgfqpoint{1.012738in}{1.474030in}}%
\pgfpathcurveto{\pgfqpoint{1.004502in}{1.474030in}}{\pgfqpoint{0.996602in}{1.470757in}}{\pgfqpoint{0.990778in}{1.464933in}}%
\pgfpathcurveto{\pgfqpoint{0.984954in}{1.459110in}}{\pgfqpoint{0.981682in}{1.451209in}}{\pgfqpoint{0.981682in}{1.442973in}}%
\pgfpathcurveto{\pgfqpoint{0.981682in}{1.434737in}}{\pgfqpoint{0.984954in}{1.426837in}}{\pgfqpoint{0.990778in}{1.421013in}}%
\pgfpathcurveto{\pgfqpoint{0.996602in}{1.415189in}}{\pgfqpoint{1.004502in}{1.411917in}}{\pgfqpoint{1.012738in}{1.411917in}}%
\pgfpathclose%
\pgfusepath{stroke,fill}%
\end{pgfscope}%
\begin{pgfscope}%
\pgfpathrectangle{\pgfqpoint{0.100000in}{0.212622in}}{\pgfqpoint{3.696000in}{3.696000in}}%
\pgfusepath{clip}%
\pgfsetbuttcap%
\pgfsetroundjoin%
\definecolor{currentfill}{rgb}{0.121569,0.466667,0.705882}%
\pgfsetfillcolor{currentfill}%
\pgfsetfillopacity{0.584635}%
\pgfsetlinewidth{1.003750pt}%
\definecolor{currentstroke}{rgb}{0.121569,0.466667,0.705882}%
\pgfsetstrokecolor{currentstroke}%
\pgfsetstrokeopacity{0.584635}%
\pgfsetdash{}{0pt}%
\pgfpathmoveto{\pgfqpoint{2.098161in}{1.841124in}}%
\pgfpathcurveto{\pgfqpoint{2.106397in}{1.841124in}}{\pgfqpoint{2.114297in}{1.844396in}}{\pgfqpoint{2.120121in}{1.850220in}}%
\pgfpathcurveto{\pgfqpoint{2.125945in}{1.856044in}}{\pgfqpoint{2.129217in}{1.863944in}}{\pgfqpoint{2.129217in}{1.872180in}}%
\pgfpathcurveto{\pgfqpoint{2.129217in}{1.880417in}}{\pgfqpoint{2.125945in}{1.888317in}}{\pgfqpoint{2.120121in}{1.894140in}}%
\pgfpathcurveto{\pgfqpoint{2.114297in}{1.899964in}}{\pgfqpoint{2.106397in}{1.903237in}}{\pgfqpoint{2.098161in}{1.903237in}}%
\pgfpathcurveto{\pgfqpoint{2.089924in}{1.903237in}}{\pgfqpoint{2.082024in}{1.899964in}}{\pgfqpoint{2.076200in}{1.894140in}}%
\pgfpathcurveto{\pgfqpoint{2.070376in}{1.888317in}}{\pgfqpoint{2.067104in}{1.880417in}}{\pgfqpoint{2.067104in}{1.872180in}}%
\pgfpathcurveto{\pgfqpoint{2.067104in}{1.863944in}}{\pgfqpoint{2.070376in}{1.856044in}}{\pgfqpoint{2.076200in}{1.850220in}}%
\pgfpathcurveto{\pgfqpoint{2.082024in}{1.844396in}}{\pgfqpoint{2.089924in}{1.841124in}}{\pgfqpoint{2.098161in}{1.841124in}}%
\pgfpathclose%
\pgfusepath{stroke,fill}%
\end{pgfscope}%
\begin{pgfscope}%
\pgfpathrectangle{\pgfqpoint{0.100000in}{0.212622in}}{\pgfqpoint{3.696000in}{3.696000in}}%
\pgfusepath{clip}%
\pgfsetbuttcap%
\pgfsetroundjoin%
\definecolor{currentfill}{rgb}{0.121569,0.466667,0.705882}%
\pgfsetfillcolor{currentfill}%
\pgfsetfillopacity{0.587092}%
\pgfsetlinewidth{1.003750pt}%
\definecolor{currentstroke}{rgb}{0.121569,0.466667,0.705882}%
\pgfsetstrokecolor{currentstroke}%
\pgfsetstrokeopacity{0.587092}%
\pgfsetdash{}{0pt}%
\pgfpathmoveto{\pgfqpoint{1.000284in}{1.405634in}}%
\pgfpathcurveto{\pgfqpoint{1.008520in}{1.405634in}}{\pgfqpoint{1.016420in}{1.408906in}}{\pgfqpoint{1.022244in}{1.414730in}}%
\pgfpathcurveto{\pgfqpoint{1.028068in}{1.420554in}}{\pgfqpoint{1.031340in}{1.428454in}}{\pgfqpoint{1.031340in}{1.436690in}}%
\pgfpathcurveto{\pgfqpoint{1.031340in}{1.444926in}}{\pgfqpoint{1.028068in}{1.452827in}}{\pgfqpoint{1.022244in}{1.458650in}}%
\pgfpathcurveto{\pgfqpoint{1.016420in}{1.464474in}}{\pgfqpoint{1.008520in}{1.467747in}}{\pgfqpoint{1.000284in}{1.467747in}}%
\pgfpathcurveto{\pgfqpoint{0.992048in}{1.467747in}}{\pgfqpoint{0.984147in}{1.464474in}}{\pgfqpoint{0.978324in}{1.458650in}}%
\pgfpathcurveto{\pgfqpoint{0.972500in}{1.452827in}}{\pgfqpoint{0.969227in}{1.444926in}}{\pgfqpoint{0.969227in}{1.436690in}}%
\pgfpathcurveto{\pgfqpoint{0.969227in}{1.428454in}}{\pgfqpoint{0.972500in}{1.420554in}}{\pgfqpoint{0.978324in}{1.414730in}}%
\pgfpathcurveto{\pgfqpoint{0.984147in}{1.408906in}}{\pgfqpoint{0.992048in}{1.405634in}}{\pgfqpoint{1.000284in}{1.405634in}}%
\pgfpathclose%
\pgfusepath{stroke,fill}%
\end{pgfscope}%
\begin{pgfscope}%
\pgfpathrectangle{\pgfqpoint{0.100000in}{0.212622in}}{\pgfqpoint{3.696000in}{3.696000in}}%
\pgfusepath{clip}%
\pgfsetbuttcap%
\pgfsetroundjoin%
\definecolor{currentfill}{rgb}{0.121569,0.466667,0.705882}%
\pgfsetfillcolor{currentfill}%
\pgfsetfillopacity{0.587635}%
\pgfsetlinewidth{1.003750pt}%
\definecolor{currentstroke}{rgb}{0.121569,0.466667,0.705882}%
\pgfsetstrokecolor{currentstroke}%
\pgfsetstrokeopacity{0.587635}%
\pgfsetdash{}{0pt}%
\pgfpathmoveto{\pgfqpoint{2.100161in}{1.838306in}}%
\pgfpathcurveto{\pgfqpoint{2.108397in}{1.838306in}}{\pgfqpoint{2.116297in}{1.841578in}}{\pgfqpoint{2.122121in}{1.847402in}}%
\pgfpathcurveto{\pgfqpoint{2.127945in}{1.853226in}}{\pgfqpoint{2.131218in}{1.861126in}}{\pgfqpoint{2.131218in}{1.869362in}}%
\pgfpathcurveto{\pgfqpoint{2.131218in}{1.877599in}}{\pgfqpoint{2.127945in}{1.885499in}}{\pgfqpoint{2.122121in}{1.891323in}}%
\pgfpathcurveto{\pgfqpoint{2.116297in}{1.897147in}}{\pgfqpoint{2.108397in}{1.900419in}}{\pgfqpoint{2.100161in}{1.900419in}}%
\pgfpathcurveto{\pgfqpoint{2.091925in}{1.900419in}}{\pgfqpoint{2.084025in}{1.897147in}}{\pgfqpoint{2.078201in}{1.891323in}}%
\pgfpathcurveto{\pgfqpoint{2.072377in}{1.885499in}}{\pgfqpoint{2.069105in}{1.877599in}}{\pgfqpoint{2.069105in}{1.869362in}}%
\pgfpathcurveto{\pgfqpoint{2.069105in}{1.861126in}}{\pgfqpoint{2.072377in}{1.853226in}}{\pgfqpoint{2.078201in}{1.847402in}}%
\pgfpathcurveto{\pgfqpoint{2.084025in}{1.841578in}}{\pgfqpoint{2.091925in}{1.838306in}}{\pgfqpoint{2.100161in}{1.838306in}}%
\pgfpathclose%
\pgfusepath{stroke,fill}%
\end{pgfscope}%
\begin{pgfscope}%
\pgfpathrectangle{\pgfqpoint{0.100000in}{0.212622in}}{\pgfqpoint{3.696000in}{3.696000in}}%
\pgfusepath{clip}%
\pgfsetbuttcap%
\pgfsetroundjoin%
\definecolor{currentfill}{rgb}{0.121569,0.466667,0.705882}%
\pgfsetfillcolor{currentfill}%
\pgfsetfillopacity{0.589657}%
\pgfsetlinewidth{1.003750pt}%
\definecolor{currentstroke}{rgb}{0.121569,0.466667,0.705882}%
\pgfsetstrokecolor{currentstroke}%
\pgfsetstrokeopacity{0.589657}%
\pgfsetdash{}{0pt}%
\pgfpathmoveto{\pgfqpoint{0.992393in}{1.393462in}}%
\pgfpathcurveto{\pgfqpoint{1.000629in}{1.393462in}}{\pgfqpoint{1.008529in}{1.396734in}}{\pgfqpoint{1.014353in}{1.402558in}}%
\pgfpathcurveto{\pgfqpoint{1.020177in}{1.408382in}}{\pgfqpoint{1.023450in}{1.416282in}}{\pgfqpoint{1.023450in}{1.424519in}}%
\pgfpathcurveto{\pgfqpoint{1.023450in}{1.432755in}}{\pgfqpoint{1.020177in}{1.440655in}}{\pgfqpoint{1.014353in}{1.446479in}}%
\pgfpathcurveto{\pgfqpoint{1.008529in}{1.452303in}}{\pgfqpoint{1.000629in}{1.455575in}}{\pgfqpoint{0.992393in}{1.455575in}}%
\pgfpathcurveto{\pgfqpoint{0.984157in}{1.455575in}}{\pgfqpoint{0.976257in}{1.452303in}}{\pgfqpoint{0.970433in}{1.446479in}}%
\pgfpathcurveto{\pgfqpoint{0.964609in}{1.440655in}}{\pgfqpoint{0.961337in}{1.432755in}}{\pgfqpoint{0.961337in}{1.424519in}}%
\pgfpathcurveto{\pgfqpoint{0.961337in}{1.416282in}}{\pgfqpoint{0.964609in}{1.408382in}}{\pgfqpoint{0.970433in}{1.402558in}}%
\pgfpathcurveto{\pgfqpoint{0.976257in}{1.396734in}}{\pgfqpoint{0.984157in}{1.393462in}}{\pgfqpoint{0.992393in}{1.393462in}}%
\pgfpathclose%
\pgfusepath{stroke,fill}%
\end{pgfscope}%
\begin{pgfscope}%
\pgfpathrectangle{\pgfqpoint{0.100000in}{0.212622in}}{\pgfqpoint{3.696000in}{3.696000in}}%
\pgfusepath{clip}%
\pgfsetbuttcap%
\pgfsetroundjoin%
\definecolor{currentfill}{rgb}{0.121569,0.466667,0.705882}%
\pgfsetfillcolor{currentfill}%
\pgfsetfillopacity{0.591369}%
\pgfsetlinewidth{1.003750pt}%
\definecolor{currentstroke}{rgb}{0.121569,0.466667,0.705882}%
\pgfsetstrokecolor{currentstroke}%
\pgfsetstrokeopacity{0.591369}%
\pgfsetdash{}{0pt}%
\pgfpathmoveto{\pgfqpoint{2.101837in}{1.836170in}}%
\pgfpathcurveto{\pgfqpoint{2.110074in}{1.836170in}}{\pgfqpoint{2.117974in}{1.839442in}}{\pgfqpoint{2.123798in}{1.845266in}}%
\pgfpathcurveto{\pgfqpoint{2.129622in}{1.851090in}}{\pgfqpoint{2.132894in}{1.858990in}}{\pgfqpoint{2.132894in}{1.867226in}}%
\pgfpathcurveto{\pgfqpoint{2.132894in}{1.875463in}}{\pgfqpoint{2.129622in}{1.883363in}}{\pgfqpoint{2.123798in}{1.889187in}}%
\pgfpathcurveto{\pgfqpoint{2.117974in}{1.895011in}}{\pgfqpoint{2.110074in}{1.898283in}}{\pgfqpoint{2.101837in}{1.898283in}}%
\pgfpathcurveto{\pgfqpoint{2.093601in}{1.898283in}}{\pgfqpoint{2.085701in}{1.895011in}}{\pgfqpoint{2.079877in}{1.889187in}}%
\pgfpathcurveto{\pgfqpoint{2.074053in}{1.883363in}}{\pgfqpoint{2.070781in}{1.875463in}}{\pgfqpoint{2.070781in}{1.867226in}}%
\pgfpathcurveto{\pgfqpoint{2.070781in}{1.858990in}}{\pgfqpoint{2.074053in}{1.851090in}}{\pgfqpoint{2.079877in}{1.845266in}}%
\pgfpathcurveto{\pgfqpoint{2.085701in}{1.839442in}}{\pgfqpoint{2.093601in}{1.836170in}}{\pgfqpoint{2.101837in}{1.836170in}}%
\pgfpathclose%
\pgfusepath{stroke,fill}%
\end{pgfscope}%
\begin{pgfscope}%
\pgfpathrectangle{\pgfqpoint{0.100000in}{0.212622in}}{\pgfqpoint{3.696000in}{3.696000in}}%
\pgfusepath{clip}%
\pgfsetbuttcap%
\pgfsetroundjoin%
\definecolor{currentfill}{rgb}{0.121569,0.466667,0.705882}%
\pgfsetfillcolor{currentfill}%
\pgfsetfillopacity{0.592576}%
\pgfsetlinewidth{1.003750pt}%
\definecolor{currentstroke}{rgb}{0.121569,0.466667,0.705882}%
\pgfsetstrokecolor{currentstroke}%
\pgfsetstrokeopacity{0.592576}%
\pgfsetdash{}{0pt}%
\pgfpathmoveto{\pgfqpoint{0.983883in}{1.385780in}}%
\pgfpathcurveto{\pgfqpoint{0.992120in}{1.385780in}}{\pgfqpoint{1.000020in}{1.389052in}}{\pgfqpoint{1.005844in}{1.394876in}}%
\pgfpathcurveto{\pgfqpoint{1.011667in}{1.400700in}}{\pgfqpoint{1.014940in}{1.408600in}}{\pgfqpoint{1.014940in}{1.416836in}}%
\pgfpathcurveto{\pgfqpoint{1.014940in}{1.425073in}}{\pgfqpoint{1.011667in}{1.432973in}}{\pgfqpoint{1.005844in}{1.438797in}}%
\pgfpathcurveto{\pgfqpoint{1.000020in}{1.444621in}}{\pgfqpoint{0.992120in}{1.447893in}}{\pgfqpoint{0.983883in}{1.447893in}}%
\pgfpathcurveto{\pgfqpoint{0.975647in}{1.447893in}}{\pgfqpoint{0.967747in}{1.444621in}}{\pgfqpoint{0.961923in}{1.438797in}}%
\pgfpathcurveto{\pgfqpoint{0.956099in}{1.432973in}}{\pgfqpoint{0.952827in}{1.425073in}}{\pgfqpoint{0.952827in}{1.416836in}}%
\pgfpathcurveto{\pgfqpoint{0.952827in}{1.408600in}}{\pgfqpoint{0.956099in}{1.400700in}}{\pgfqpoint{0.961923in}{1.394876in}}%
\pgfpathcurveto{\pgfqpoint{0.967747in}{1.389052in}}{\pgfqpoint{0.975647in}{1.385780in}}{\pgfqpoint{0.983883in}{1.385780in}}%
\pgfpathclose%
\pgfusepath{stroke,fill}%
\end{pgfscope}%
\begin{pgfscope}%
\pgfpathrectangle{\pgfqpoint{0.100000in}{0.212622in}}{\pgfqpoint{3.696000in}{3.696000in}}%
\pgfusepath{clip}%
\pgfsetbuttcap%
\pgfsetroundjoin%
\definecolor{currentfill}{rgb}{0.121569,0.466667,0.705882}%
\pgfsetfillcolor{currentfill}%
\pgfsetfillopacity{0.594995}%
\pgfsetlinewidth{1.003750pt}%
\definecolor{currentstroke}{rgb}{0.121569,0.466667,0.705882}%
\pgfsetstrokecolor{currentstroke}%
\pgfsetstrokeopacity{0.594995}%
\pgfsetdash{}{0pt}%
\pgfpathmoveto{\pgfqpoint{2.104544in}{1.832165in}}%
\pgfpathcurveto{\pgfqpoint{2.112780in}{1.832165in}}{\pgfqpoint{2.120680in}{1.835438in}}{\pgfqpoint{2.126504in}{1.841261in}}%
\pgfpathcurveto{\pgfqpoint{2.132328in}{1.847085in}}{\pgfqpoint{2.135601in}{1.854985in}}{\pgfqpoint{2.135601in}{1.863222in}}%
\pgfpathcurveto{\pgfqpoint{2.135601in}{1.871458in}}{\pgfqpoint{2.132328in}{1.879358in}}{\pgfqpoint{2.126504in}{1.885182in}}%
\pgfpathcurveto{\pgfqpoint{2.120680in}{1.891006in}}{\pgfqpoint{2.112780in}{1.894278in}}{\pgfqpoint{2.104544in}{1.894278in}}%
\pgfpathcurveto{\pgfqpoint{2.096308in}{1.894278in}}{\pgfqpoint{2.088408in}{1.891006in}}{\pgfqpoint{2.082584in}{1.885182in}}%
\pgfpathcurveto{\pgfqpoint{2.076760in}{1.879358in}}{\pgfqpoint{2.073488in}{1.871458in}}{\pgfqpoint{2.073488in}{1.863222in}}%
\pgfpathcurveto{\pgfqpoint{2.073488in}{1.854985in}}{\pgfqpoint{2.076760in}{1.847085in}}{\pgfqpoint{2.082584in}{1.841261in}}%
\pgfpathcurveto{\pgfqpoint{2.088408in}{1.835438in}}{\pgfqpoint{2.096308in}{1.832165in}}{\pgfqpoint{2.104544in}{1.832165in}}%
\pgfpathclose%
\pgfusepath{stroke,fill}%
\end{pgfscope}%
\begin{pgfscope}%
\pgfpathrectangle{\pgfqpoint{0.100000in}{0.212622in}}{\pgfqpoint{3.696000in}{3.696000in}}%
\pgfusepath{clip}%
\pgfsetbuttcap%
\pgfsetroundjoin%
\definecolor{currentfill}{rgb}{0.121569,0.466667,0.705882}%
\pgfsetfillcolor{currentfill}%
\pgfsetfillopacity{0.595215}%
\pgfsetlinewidth{1.003750pt}%
\definecolor{currentstroke}{rgb}{0.121569,0.466667,0.705882}%
\pgfsetstrokecolor{currentstroke}%
\pgfsetstrokeopacity{0.595215}%
\pgfsetdash{}{0pt}%
\pgfpathmoveto{\pgfqpoint{0.976199in}{1.377739in}}%
\pgfpathcurveto{\pgfqpoint{0.984435in}{1.377739in}}{\pgfqpoint{0.992335in}{1.381012in}}{\pgfqpoint{0.998159in}{1.386836in}}%
\pgfpathcurveto{\pgfqpoint{1.003983in}{1.392659in}}{\pgfqpoint{1.007255in}{1.400560in}}{\pgfqpoint{1.007255in}{1.408796in}}%
\pgfpathcurveto{\pgfqpoint{1.007255in}{1.417032in}}{\pgfqpoint{1.003983in}{1.424932in}}{\pgfqpoint{0.998159in}{1.430756in}}%
\pgfpathcurveto{\pgfqpoint{0.992335in}{1.436580in}}{\pgfqpoint{0.984435in}{1.439852in}}{\pgfqpoint{0.976199in}{1.439852in}}%
\pgfpathcurveto{\pgfqpoint{0.967962in}{1.439852in}}{\pgfqpoint{0.960062in}{1.436580in}}{\pgfqpoint{0.954238in}{1.430756in}}%
\pgfpathcurveto{\pgfqpoint{0.948414in}{1.424932in}}{\pgfqpoint{0.945142in}{1.417032in}}{\pgfqpoint{0.945142in}{1.408796in}}%
\pgfpathcurveto{\pgfqpoint{0.945142in}{1.400560in}}{\pgfqpoint{0.948414in}{1.392659in}}{\pgfqpoint{0.954238in}{1.386836in}}%
\pgfpathcurveto{\pgfqpoint{0.960062in}{1.381012in}}{\pgfqpoint{0.967962in}{1.377739in}}{\pgfqpoint{0.976199in}{1.377739in}}%
\pgfpathclose%
\pgfusepath{stroke,fill}%
\end{pgfscope}%
\begin{pgfscope}%
\pgfpathrectangle{\pgfqpoint{0.100000in}{0.212622in}}{\pgfqpoint{3.696000in}{3.696000in}}%
\pgfusepath{clip}%
\pgfsetbuttcap%
\pgfsetroundjoin%
\definecolor{currentfill}{rgb}{0.121569,0.466667,0.705882}%
\pgfsetfillcolor{currentfill}%
\pgfsetfillopacity{0.597244}%
\pgfsetlinewidth{1.003750pt}%
\definecolor{currentstroke}{rgb}{0.121569,0.466667,0.705882}%
\pgfsetstrokecolor{currentstroke}%
\pgfsetstrokeopacity{0.597244}%
\pgfsetdash{}{0pt}%
\pgfpathmoveto{\pgfqpoint{0.970947in}{1.367272in}}%
\pgfpathcurveto{\pgfqpoint{0.979184in}{1.367272in}}{\pgfqpoint{0.987084in}{1.370545in}}{\pgfqpoint{0.992908in}{1.376369in}}%
\pgfpathcurveto{\pgfqpoint{0.998732in}{1.382193in}}{\pgfqpoint{1.002004in}{1.390093in}}{\pgfqpoint{1.002004in}{1.398329in}}%
\pgfpathcurveto{\pgfqpoint{1.002004in}{1.406565in}}{\pgfqpoint{0.998732in}{1.414465in}}{\pgfqpoint{0.992908in}{1.420289in}}%
\pgfpathcurveto{\pgfqpoint{0.987084in}{1.426113in}}{\pgfqpoint{0.979184in}{1.429385in}}{\pgfqpoint{0.970947in}{1.429385in}}%
\pgfpathcurveto{\pgfqpoint{0.962711in}{1.429385in}}{\pgfqpoint{0.954811in}{1.426113in}}{\pgfqpoint{0.948987in}{1.420289in}}%
\pgfpathcurveto{\pgfqpoint{0.943163in}{1.414465in}}{\pgfqpoint{0.939891in}{1.406565in}}{\pgfqpoint{0.939891in}{1.398329in}}%
\pgfpathcurveto{\pgfqpoint{0.939891in}{1.390093in}}{\pgfqpoint{0.943163in}{1.382193in}}{\pgfqpoint{0.948987in}{1.376369in}}%
\pgfpathcurveto{\pgfqpoint{0.954811in}{1.370545in}}{\pgfqpoint{0.962711in}{1.367272in}}{\pgfqpoint{0.970947in}{1.367272in}}%
\pgfpathclose%
\pgfusepath{stroke,fill}%
\end{pgfscope}%
\begin{pgfscope}%
\pgfpathrectangle{\pgfqpoint{0.100000in}{0.212622in}}{\pgfqpoint{3.696000in}{3.696000in}}%
\pgfusepath{clip}%
\pgfsetbuttcap%
\pgfsetroundjoin%
\definecolor{currentfill}{rgb}{0.121569,0.466667,0.705882}%
\pgfsetfillcolor{currentfill}%
\pgfsetfillopacity{0.598974}%
\pgfsetlinewidth{1.003750pt}%
\definecolor{currentstroke}{rgb}{0.121569,0.466667,0.705882}%
\pgfsetstrokecolor{currentstroke}%
\pgfsetstrokeopacity{0.598974}%
\pgfsetdash{}{0pt}%
\pgfpathmoveto{\pgfqpoint{0.965523in}{1.363111in}}%
\pgfpathcurveto{\pgfqpoint{0.973759in}{1.363111in}}{\pgfqpoint{0.981659in}{1.366384in}}{\pgfqpoint{0.987483in}{1.372208in}}%
\pgfpathcurveto{\pgfqpoint{0.993307in}{1.378032in}}{\pgfqpoint{0.996580in}{1.385932in}}{\pgfqpoint{0.996580in}{1.394168in}}%
\pgfpathcurveto{\pgfqpoint{0.996580in}{1.402404in}}{\pgfqpoint{0.993307in}{1.410304in}}{\pgfqpoint{0.987483in}{1.416128in}}%
\pgfpathcurveto{\pgfqpoint{0.981659in}{1.421952in}}{\pgfqpoint{0.973759in}{1.425224in}}{\pgfqpoint{0.965523in}{1.425224in}}%
\pgfpathcurveto{\pgfqpoint{0.957287in}{1.425224in}}{\pgfqpoint{0.949387in}{1.421952in}}{\pgfqpoint{0.943563in}{1.416128in}}%
\pgfpathcurveto{\pgfqpoint{0.937739in}{1.410304in}}{\pgfqpoint{0.934467in}{1.402404in}}{\pgfqpoint{0.934467in}{1.394168in}}%
\pgfpathcurveto{\pgfqpoint{0.934467in}{1.385932in}}{\pgfqpoint{0.937739in}{1.378032in}}{\pgfqpoint{0.943563in}{1.372208in}}%
\pgfpathcurveto{\pgfqpoint{0.949387in}{1.366384in}}{\pgfqpoint{0.957287in}{1.363111in}}{\pgfqpoint{0.965523in}{1.363111in}}%
\pgfpathclose%
\pgfusepath{stroke,fill}%
\end{pgfscope}%
\begin{pgfscope}%
\pgfpathrectangle{\pgfqpoint{0.100000in}{0.212622in}}{\pgfqpoint{3.696000in}{3.696000in}}%
\pgfusepath{clip}%
\pgfsetbuttcap%
\pgfsetroundjoin%
\definecolor{currentfill}{rgb}{0.121569,0.466667,0.705882}%
\pgfsetfillcolor{currentfill}%
\pgfsetfillopacity{0.599006}%
\pgfsetlinewidth{1.003750pt}%
\definecolor{currentstroke}{rgb}{0.121569,0.466667,0.705882}%
\pgfsetstrokecolor{currentstroke}%
\pgfsetstrokeopacity{0.599006}%
\pgfsetdash{}{0pt}%
\pgfpathmoveto{\pgfqpoint{2.107045in}{1.827132in}}%
\pgfpathcurveto{\pgfqpoint{2.115281in}{1.827132in}}{\pgfqpoint{2.123181in}{1.830404in}}{\pgfqpoint{2.129005in}{1.836228in}}%
\pgfpathcurveto{\pgfqpoint{2.134829in}{1.842052in}}{\pgfqpoint{2.138102in}{1.849952in}}{\pgfqpoint{2.138102in}{1.858188in}}%
\pgfpathcurveto{\pgfqpoint{2.138102in}{1.866425in}}{\pgfqpoint{2.134829in}{1.874325in}}{\pgfqpoint{2.129005in}{1.880149in}}%
\pgfpathcurveto{\pgfqpoint{2.123181in}{1.885973in}}{\pgfqpoint{2.115281in}{1.889245in}}{\pgfqpoint{2.107045in}{1.889245in}}%
\pgfpathcurveto{\pgfqpoint{2.098809in}{1.889245in}}{\pgfqpoint{2.090909in}{1.885973in}}{\pgfqpoint{2.085085in}{1.880149in}}%
\pgfpathcurveto{\pgfqpoint{2.079261in}{1.874325in}}{\pgfqpoint{2.075989in}{1.866425in}}{\pgfqpoint{2.075989in}{1.858188in}}%
\pgfpathcurveto{\pgfqpoint{2.075989in}{1.849952in}}{\pgfqpoint{2.079261in}{1.842052in}}{\pgfqpoint{2.085085in}{1.836228in}}%
\pgfpathcurveto{\pgfqpoint{2.090909in}{1.830404in}}{\pgfqpoint{2.098809in}{1.827132in}}{\pgfqpoint{2.107045in}{1.827132in}}%
\pgfpathclose%
\pgfusepath{stroke,fill}%
\end{pgfscope}%
\begin{pgfscope}%
\pgfpathrectangle{\pgfqpoint{0.100000in}{0.212622in}}{\pgfqpoint{3.696000in}{3.696000in}}%
\pgfusepath{clip}%
\pgfsetbuttcap%
\pgfsetroundjoin%
\definecolor{currentfill}{rgb}{0.121569,0.466667,0.705882}%
\pgfsetfillcolor{currentfill}%
\pgfsetfillopacity{0.600004}%
\pgfsetlinewidth{1.003750pt}%
\definecolor{currentstroke}{rgb}{0.121569,0.466667,0.705882}%
\pgfsetstrokecolor{currentstroke}%
\pgfsetstrokeopacity{0.600004}%
\pgfsetdash{}{0pt}%
\pgfpathmoveto{\pgfqpoint{0.963138in}{1.357581in}}%
\pgfpathcurveto{\pgfqpoint{0.971375in}{1.357581in}}{\pgfqpoint{0.979275in}{1.360853in}}{\pgfqpoint{0.985099in}{1.366677in}}%
\pgfpathcurveto{\pgfqpoint{0.990923in}{1.372501in}}{\pgfqpoint{0.994195in}{1.380401in}}{\pgfqpoint{0.994195in}{1.388637in}}%
\pgfpathcurveto{\pgfqpoint{0.994195in}{1.396874in}}{\pgfqpoint{0.990923in}{1.404774in}}{\pgfqpoint{0.985099in}{1.410598in}}%
\pgfpathcurveto{\pgfqpoint{0.979275in}{1.416422in}}{\pgfqpoint{0.971375in}{1.419694in}}{\pgfqpoint{0.963138in}{1.419694in}}%
\pgfpathcurveto{\pgfqpoint{0.954902in}{1.419694in}}{\pgfqpoint{0.947002in}{1.416422in}}{\pgfqpoint{0.941178in}{1.410598in}}%
\pgfpathcurveto{\pgfqpoint{0.935354in}{1.404774in}}{\pgfqpoint{0.932082in}{1.396874in}}{\pgfqpoint{0.932082in}{1.388637in}}%
\pgfpathcurveto{\pgfqpoint{0.932082in}{1.380401in}}{\pgfqpoint{0.935354in}{1.372501in}}{\pgfqpoint{0.941178in}{1.366677in}}%
\pgfpathcurveto{\pgfqpoint{0.947002in}{1.360853in}}{\pgfqpoint{0.954902in}{1.357581in}}{\pgfqpoint{0.963138in}{1.357581in}}%
\pgfpathclose%
\pgfusepath{stroke,fill}%
\end{pgfscope}%
\begin{pgfscope}%
\pgfpathrectangle{\pgfqpoint{0.100000in}{0.212622in}}{\pgfqpoint{3.696000in}{3.696000in}}%
\pgfusepath{clip}%
\pgfsetbuttcap%
\pgfsetroundjoin%
\definecolor{currentfill}{rgb}{0.121569,0.466667,0.705882}%
\pgfsetfillcolor{currentfill}%
\pgfsetfillopacity{0.600994}%
\pgfsetlinewidth{1.003750pt}%
\definecolor{currentstroke}{rgb}{0.121569,0.466667,0.705882}%
\pgfsetstrokecolor{currentstroke}%
\pgfsetstrokeopacity{0.600994}%
\pgfsetdash{}{0pt}%
\pgfpathmoveto{\pgfqpoint{0.960177in}{1.356085in}}%
\pgfpathcurveto{\pgfqpoint{0.968414in}{1.356085in}}{\pgfqpoint{0.976314in}{1.359357in}}{\pgfqpoint{0.982138in}{1.365181in}}%
\pgfpathcurveto{\pgfqpoint{0.987962in}{1.371005in}}{\pgfqpoint{0.991234in}{1.378905in}}{\pgfqpoint{0.991234in}{1.387141in}}%
\pgfpathcurveto{\pgfqpoint{0.991234in}{1.395377in}}{\pgfqpoint{0.987962in}{1.403277in}}{\pgfqpoint{0.982138in}{1.409101in}}%
\pgfpathcurveto{\pgfqpoint{0.976314in}{1.414925in}}{\pgfqpoint{0.968414in}{1.418198in}}{\pgfqpoint{0.960177in}{1.418198in}}%
\pgfpathcurveto{\pgfqpoint{0.951941in}{1.418198in}}{\pgfqpoint{0.944041in}{1.414925in}}{\pgfqpoint{0.938217in}{1.409101in}}%
\pgfpathcurveto{\pgfqpoint{0.932393in}{1.403277in}}{\pgfqpoint{0.929121in}{1.395377in}}{\pgfqpoint{0.929121in}{1.387141in}}%
\pgfpathcurveto{\pgfqpoint{0.929121in}{1.378905in}}{\pgfqpoint{0.932393in}{1.371005in}}{\pgfqpoint{0.938217in}{1.365181in}}%
\pgfpathcurveto{\pgfqpoint{0.944041in}{1.359357in}}{\pgfqpoint{0.951941in}{1.356085in}}{\pgfqpoint{0.960177in}{1.356085in}}%
\pgfpathclose%
\pgfusepath{stroke,fill}%
\end{pgfscope}%
\begin{pgfscope}%
\pgfpathrectangle{\pgfqpoint{0.100000in}{0.212622in}}{\pgfqpoint{3.696000in}{3.696000in}}%
\pgfusepath{clip}%
\pgfsetbuttcap%
\pgfsetroundjoin%
\definecolor{currentfill}{rgb}{0.121569,0.466667,0.705882}%
\pgfsetfillcolor{currentfill}%
\pgfsetfillopacity{0.602536}%
\pgfsetlinewidth{1.003750pt}%
\definecolor{currentstroke}{rgb}{0.121569,0.466667,0.705882}%
\pgfsetstrokecolor{currentstroke}%
\pgfsetstrokeopacity{0.602536}%
\pgfsetdash{}{0pt}%
\pgfpathmoveto{\pgfqpoint{0.955442in}{1.350900in}}%
\pgfpathcurveto{\pgfqpoint{0.963679in}{1.350900in}}{\pgfqpoint{0.971579in}{1.354173in}}{\pgfqpoint{0.977403in}{1.359997in}}%
\pgfpathcurveto{\pgfqpoint{0.983227in}{1.365821in}}{\pgfqpoint{0.986499in}{1.373721in}}{\pgfqpoint{0.986499in}{1.381957in}}%
\pgfpathcurveto{\pgfqpoint{0.986499in}{1.390193in}}{\pgfqpoint{0.983227in}{1.398093in}}{\pgfqpoint{0.977403in}{1.403917in}}%
\pgfpathcurveto{\pgfqpoint{0.971579in}{1.409741in}}{\pgfqpoint{0.963679in}{1.413013in}}{\pgfqpoint{0.955442in}{1.413013in}}%
\pgfpathcurveto{\pgfqpoint{0.947206in}{1.413013in}}{\pgfqpoint{0.939306in}{1.409741in}}{\pgfqpoint{0.933482in}{1.403917in}}%
\pgfpathcurveto{\pgfqpoint{0.927658in}{1.398093in}}{\pgfqpoint{0.924386in}{1.390193in}}{\pgfqpoint{0.924386in}{1.381957in}}%
\pgfpathcurveto{\pgfqpoint{0.924386in}{1.373721in}}{\pgfqpoint{0.927658in}{1.365821in}}{\pgfqpoint{0.933482in}{1.359997in}}%
\pgfpathcurveto{\pgfqpoint{0.939306in}{1.354173in}}{\pgfqpoint{0.947206in}{1.350900in}}{\pgfqpoint{0.955442in}{1.350900in}}%
\pgfpathclose%
\pgfusepath{stroke,fill}%
\end{pgfscope}%
\begin{pgfscope}%
\pgfpathrectangle{\pgfqpoint{0.100000in}{0.212622in}}{\pgfqpoint{3.696000in}{3.696000in}}%
\pgfusepath{clip}%
\pgfsetbuttcap%
\pgfsetroundjoin%
\definecolor{currentfill}{rgb}{0.121569,0.466667,0.705882}%
\pgfsetfillcolor{currentfill}%
\pgfsetfillopacity{0.603353}%
\pgfsetlinewidth{1.003750pt}%
\definecolor{currentstroke}{rgb}{0.121569,0.466667,0.705882}%
\pgfsetstrokecolor{currentstroke}%
\pgfsetstrokeopacity{0.603353}%
\pgfsetdash{}{0pt}%
\pgfpathmoveto{\pgfqpoint{2.109802in}{1.820543in}}%
\pgfpathcurveto{\pgfqpoint{2.118038in}{1.820543in}}{\pgfqpoint{2.125939in}{1.823816in}}{\pgfqpoint{2.131762in}{1.829640in}}%
\pgfpathcurveto{\pgfqpoint{2.137586in}{1.835463in}}{\pgfqpoint{2.140859in}{1.843364in}}{\pgfqpoint{2.140859in}{1.851600in}}%
\pgfpathcurveto{\pgfqpoint{2.140859in}{1.859836in}}{\pgfqpoint{2.137586in}{1.867736in}}{\pgfqpoint{2.131762in}{1.873560in}}%
\pgfpathcurveto{\pgfqpoint{2.125939in}{1.879384in}}{\pgfqpoint{2.118038in}{1.882656in}}{\pgfqpoint{2.109802in}{1.882656in}}%
\pgfpathcurveto{\pgfqpoint{2.101566in}{1.882656in}}{\pgfqpoint{2.093666in}{1.879384in}}{\pgfqpoint{2.087842in}{1.873560in}}%
\pgfpathcurveto{\pgfqpoint{2.082018in}{1.867736in}}{\pgfqpoint{2.078746in}{1.859836in}}{\pgfqpoint{2.078746in}{1.851600in}}%
\pgfpathcurveto{\pgfqpoint{2.078746in}{1.843364in}}{\pgfqpoint{2.082018in}{1.835463in}}{\pgfqpoint{2.087842in}{1.829640in}}%
\pgfpathcurveto{\pgfqpoint{2.093666in}{1.823816in}}{\pgfqpoint{2.101566in}{1.820543in}}{\pgfqpoint{2.109802in}{1.820543in}}%
\pgfpathclose%
\pgfusepath{stroke,fill}%
\end{pgfscope}%
\begin{pgfscope}%
\pgfpathrectangle{\pgfqpoint{0.100000in}{0.212622in}}{\pgfqpoint{3.696000in}{3.696000in}}%
\pgfusepath{clip}%
\pgfsetbuttcap%
\pgfsetroundjoin%
\definecolor{currentfill}{rgb}{0.121569,0.466667,0.705882}%
\pgfsetfillcolor{currentfill}%
\pgfsetfillopacity{0.605582}%
\pgfsetlinewidth{1.003750pt}%
\definecolor{currentstroke}{rgb}{0.121569,0.466667,0.705882}%
\pgfsetstrokecolor{currentstroke}%
\pgfsetstrokeopacity{0.605582}%
\pgfsetdash{}{0pt}%
\pgfpathmoveto{\pgfqpoint{0.946675in}{1.343147in}}%
\pgfpathcurveto{\pgfqpoint{0.954911in}{1.343147in}}{\pgfqpoint{0.962811in}{1.346420in}}{\pgfqpoint{0.968635in}{1.352244in}}%
\pgfpathcurveto{\pgfqpoint{0.974459in}{1.358068in}}{\pgfqpoint{0.977731in}{1.365968in}}{\pgfqpoint{0.977731in}{1.374204in}}%
\pgfpathcurveto{\pgfqpoint{0.977731in}{1.382440in}}{\pgfqpoint{0.974459in}{1.390340in}}{\pgfqpoint{0.968635in}{1.396164in}}%
\pgfpathcurveto{\pgfqpoint{0.962811in}{1.401988in}}{\pgfqpoint{0.954911in}{1.405260in}}{\pgfqpoint{0.946675in}{1.405260in}}%
\pgfpathcurveto{\pgfqpoint{0.938438in}{1.405260in}}{\pgfqpoint{0.930538in}{1.401988in}}{\pgfqpoint{0.924714in}{1.396164in}}%
\pgfpathcurveto{\pgfqpoint{0.918890in}{1.390340in}}{\pgfqpoint{0.915618in}{1.382440in}}{\pgfqpoint{0.915618in}{1.374204in}}%
\pgfpathcurveto{\pgfqpoint{0.915618in}{1.365968in}}{\pgfqpoint{0.918890in}{1.358068in}}{\pgfqpoint{0.924714in}{1.352244in}}%
\pgfpathcurveto{\pgfqpoint{0.930538in}{1.346420in}}{\pgfqpoint{0.938438in}{1.343147in}}{\pgfqpoint{0.946675in}{1.343147in}}%
\pgfpathclose%
\pgfusepath{stroke,fill}%
\end{pgfscope}%
\begin{pgfscope}%
\pgfpathrectangle{\pgfqpoint{0.100000in}{0.212622in}}{\pgfqpoint{3.696000in}{3.696000in}}%
\pgfusepath{clip}%
\pgfsetbuttcap%
\pgfsetroundjoin%
\definecolor{currentfill}{rgb}{0.121569,0.466667,0.705882}%
\pgfsetfillcolor{currentfill}%
\pgfsetfillopacity{0.605905}%
\pgfsetlinewidth{1.003750pt}%
\definecolor{currentstroke}{rgb}{0.121569,0.466667,0.705882}%
\pgfsetstrokecolor{currentstroke}%
\pgfsetstrokeopacity{0.605905}%
\pgfsetdash{}{0pt}%
\pgfpathmoveto{\pgfqpoint{2.111854in}{1.818241in}}%
\pgfpathcurveto{\pgfqpoint{2.120090in}{1.818241in}}{\pgfqpoint{2.127990in}{1.821513in}}{\pgfqpoint{2.133814in}{1.827337in}}%
\pgfpathcurveto{\pgfqpoint{2.139638in}{1.833161in}}{\pgfqpoint{2.142910in}{1.841061in}}{\pgfqpoint{2.142910in}{1.849297in}}%
\pgfpathcurveto{\pgfqpoint{2.142910in}{1.857533in}}{\pgfqpoint{2.139638in}{1.865434in}}{\pgfqpoint{2.133814in}{1.871257in}}%
\pgfpathcurveto{\pgfqpoint{2.127990in}{1.877081in}}{\pgfqpoint{2.120090in}{1.880354in}}{\pgfqpoint{2.111854in}{1.880354in}}%
\pgfpathcurveto{\pgfqpoint{2.103618in}{1.880354in}}{\pgfqpoint{2.095718in}{1.877081in}}{\pgfqpoint{2.089894in}{1.871257in}}%
\pgfpathcurveto{\pgfqpoint{2.084070in}{1.865434in}}{\pgfqpoint{2.080797in}{1.857533in}}{\pgfqpoint{2.080797in}{1.849297in}}%
\pgfpathcurveto{\pgfqpoint{2.080797in}{1.841061in}}{\pgfqpoint{2.084070in}{1.833161in}}{\pgfqpoint{2.089894in}{1.827337in}}%
\pgfpathcurveto{\pgfqpoint{2.095718in}{1.821513in}}{\pgfqpoint{2.103618in}{1.818241in}}{\pgfqpoint{2.111854in}{1.818241in}}%
\pgfpathclose%
\pgfusepath{stroke,fill}%
\end{pgfscope}%
\begin{pgfscope}%
\pgfpathrectangle{\pgfqpoint{0.100000in}{0.212622in}}{\pgfqpoint{3.696000in}{3.696000in}}%
\pgfusepath{clip}%
\pgfsetbuttcap%
\pgfsetroundjoin%
\definecolor{currentfill}{rgb}{0.121569,0.466667,0.705882}%
\pgfsetfillcolor{currentfill}%
\pgfsetfillopacity{0.608043}%
\pgfsetlinewidth{1.003750pt}%
\definecolor{currentstroke}{rgb}{0.121569,0.466667,0.705882}%
\pgfsetstrokecolor{currentstroke}%
\pgfsetstrokeopacity{0.608043}%
\pgfsetdash{}{0pt}%
\pgfpathmoveto{\pgfqpoint{0.937851in}{1.336552in}}%
\pgfpathcurveto{\pgfqpoint{0.946087in}{1.336552in}}{\pgfqpoint{0.953987in}{1.339824in}}{\pgfqpoint{0.959811in}{1.345648in}}%
\pgfpathcurveto{\pgfqpoint{0.965635in}{1.351472in}}{\pgfqpoint{0.968907in}{1.359372in}}{\pgfqpoint{0.968907in}{1.367608in}}%
\pgfpathcurveto{\pgfqpoint{0.968907in}{1.375844in}}{\pgfqpoint{0.965635in}{1.383744in}}{\pgfqpoint{0.959811in}{1.389568in}}%
\pgfpathcurveto{\pgfqpoint{0.953987in}{1.395392in}}{\pgfqpoint{0.946087in}{1.398665in}}{\pgfqpoint{0.937851in}{1.398665in}}%
\pgfpathcurveto{\pgfqpoint{0.929615in}{1.398665in}}{\pgfqpoint{0.921715in}{1.395392in}}{\pgfqpoint{0.915891in}{1.389568in}}%
\pgfpathcurveto{\pgfqpoint{0.910067in}{1.383744in}}{\pgfqpoint{0.906794in}{1.375844in}}{\pgfqpoint{0.906794in}{1.367608in}}%
\pgfpathcurveto{\pgfqpoint{0.906794in}{1.359372in}}{\pgfqpoint{0.910067in}{1.351472in}}{\pgfqpoint{0.915891in}{1.345648in}}%
\pgfpathcurveto{\pgfqpoint{0.921715in}{1.339824in}}{\pgfqpoint{0.929615in}{1.336552in}}{\pgfqpoint{0.937851in}{1.336552in}}%
\pgfpathclose%
\pgfusepath{stroke,fill}%
\end{pgfscope}%
\begin{pgfscope}%
\pgfpathrectangle{\pgfqpoint{0.100000in}{0.212622in}}{\pgfqpoint{3.696000in}{3.696000in}}%
\pgfusepath{clip}%
\pgfsetbuttcap%
\pgfsetroundjoin%
\definecolor{currentfill}{rgb}{0.121569,0.466667,0.705882}%
\pgfsetfillcolor{currentfill}%
\pgfsetfillopacity{0.608594}%
\pgfsetlinewidth{1.003750pt}%
\definecolor{currentstroke}{rgb}{0.121569,0.466667,0.705882}%
\pgfsetstrokecolor{currentstroke}%
\pgfsetstrokeopacity{0.608594}%
\pgfsetdash{}{0pt}%
\pgfpathmoveto{\pgfqpoint{2.113386in}{1.814236in}}%
\pgfpathcurveto{\pgfqpoint{2.121623in}{1.814236in}}{\pgfqpoint{2.129523in}{1.817508in}}{\pgfqpoint{2.135347in}{1.823332in}}%
\pgfpathcurveto{\pgfqpoint{2.141170in}{1.829156in}}{\pgfqpoint{2.144443in}{1.837056in}}{\pgfqpoint{2.144443in}{1.845292in}}%
\pgfpathcurveto{\pgfqpoint{2.144443in}{1.853529in}}{\pgfqpoint{2.141170in}{1.861429in}}{\pgfqpoint{2.135347in}{1.867253in}}%
\pgfpathcurveto{\pgfqpoint{2.129523in}{1.873077in}}{\pgfqpoint{2.121623in}{1.876349in}}{\pgfqpoint{2.113386in}{1.876349in}}%
\pgfpathcurveto{\pgfqpoint{2.105150in}{1.876349in}}{\pgfqpoint{2.097250in}{1.873077in}}{\pgfqpoint{2.091426in}{1.867253in}}%
\pgfpathcurveto{\pgfqpoint{2.085602in}{1.861429in}}{\pgfqpoint{2.082330in}{1.853529in}}{\pgfqpoint{2.082330in}{1.845292in}}%
\pgfpathcurveto{\pgfqpoint{2.082330in}{1.837056in}}{\pgfqpoint{2.085602in}{1.829156in}}{\pgfqpoint{2.091426in}{1.823332in}}%
\pgfpathcurveto{\pgfqpoint{2.097250in}{1.817508in}}{\pgfqpoint{2.105150in}{1.814236in}}{\pgfqpoint{2.113386in}{1.814236in}}%
\pgfpathclose%
\pgfusepath{stroke,fill}%
\end{pgfscope}%
\begin{pgfscope}%
\pgfpathrectangle{\pgfqpoint{0.100000in}{0.212622in}}{\pgfqpoint{3.696000in}{3.696000in}}%
\pgfusepath{clip}%
\pgfsetbuttcap%
\pgfsetroundjoin%
\definecolor{currentfill}{rgb}{0.121569,0.466667,0.705882}%
\pgfsetfillcolor{currentfill}%
\pgfsetfillopacity{0.610226}%
\pgfsetlinewidth{1.003750pt}%
\definecolor{currentstroke}{rgb}{0.121569,0.466667,0.705882}%
\pgfsetstrokecolor{currentstroke}%
\pgfsetstrokeopacity{0.610226}%
\pgfsetdash{}{0pt}%
\pgfpathmoveto{\pgfqpoint{0.933224in}{1.325868in}}%
\pgfpathcurveto{\pgfqpoint{0.941460in}{1.325868in}}{\pgfqpoint{0.949360in}{1.329140in}}{\pgfqpoint{0.955184in}{1.334964in}}%
\pgfpathcurveto{\pgfqpoint{0.961008in}{1.340788in}}{\pgfqpoint{0.964280in}{1.348688in}}{\pgfqpoint{0.964280in}{1.356924in}}%
\pgfpathcurveto{\pgfqpoint{0.964280in}{1.365161in}}{\pgfqpoint{0.961008in}{1.373061in}}{\pgfqpoint{0.955184in}{1.378885in}}%
\pgfpathcurveto{\pgfqpoint{0.949360in}{1.384709in}}{\pgfqpoint{0.941460in}{1.387981in}}{\pgfqpoint{0.933224in}{1.387981in}}%
\pgfpathcurveto{\pgfqpoint{0.924987in}{1.387981in}}{\pgfqpoint{0.917087in}{1.384709in}}{\pgfqpoint{0.911263in}{1.378885in}}%
\pgfpathcurveto{\pgfqpoint{0.905439in}{1.373061in}}{\pgfqpoint{0.902167in}{1.365161in}}{\pgfqpoint{0.902167in}{1.356924in}}%
\pgfpathcurveto{\pgfqpoint{0.902167in}{1.348688in}}{\pgfqpoint{0.905439in}{1.340788in}}{\pgfqpoint{0.911263in}{1.334964in}}%
\pgfpathcurveto{\pgfqpoint{0.917087in}{1.329140in}}{\pgfqpoint{0.924987in}{1.325868in}}{\pgfqpoint{0.933224in}{1.325868in}}%
\pgfpathclose%
\pgfusepath{stroke,fill}%
\end{pgfscope}%
\begin{pgfscope}%
\pgfpathrectangle{\pgfqpoint{0.100000in}{0.212622in}}{\pgfqpoint{3.696000in}{3.696000in}}%
\pgfusepath{clip}%
\pgfsetbuttcap%
\pgfsetroundjoin%
\definecolor{currentfill}{rgb}{0.121569,0.466667,0.705882}%
\pgfsetfillcolor{currentfill}%
\pgfsetfillopacity{0.611643}%
\pgfsetlinewidth{1.003750pt}%
\definecolor{currentstroke}{rgb}{0.121569,0.466667,0.705882}%
\pgfsetstrokecolor{currentstroke}%
\pgfsetstrokeopacity{0.611643}%
\pgfsetdash{}{0pt}%
\pgfpathmoveto{\pgfqpoint{2.115446in}{1.810615in}}%
\pgfpathcurveto{\pgfqpoint{2.123682in}{1.810615in}}{\pgfqpoint{2.131583in}{1.813887in}}{\pgfqpoint{2.137406in}{1.819711in}}%
\pgfpathcurveto{\pgfqpoint{2.143230in}{1.825535in}}{\pgfqpoint{2.146503in}{1.833435in}}{\pgfqpoint{2.146503in}{1.841671in}}%
\pgfpathcurveto{\pgfqpoint{2.146503in}{1.849908in}}{\pgfqpoint{2.143230in}{1.857808in}}{\pgfqpoint{2.137406in}{1.863632in}}%
\pgfpathcurveto{\pgfqpoint{2.131583in}{1.869456in}}{\pgfqpoint{2.123682in}{1.872728in}}{\pgfqpoint{2.115446in}{1.872728in}}%
\pgfpathcurveto{\pgfqpoint{2.107210in}{1.872728in}}{\pgfqpoint{2.099310in}{1.869456in}}{\pgfqpoint{2.093486in}{1.863632in}}%
\pgfpathcurveto{\pgfqpoint{2.087662in}{1.857808in}}{\pgfqpoint{2.084390in}{1.849908in}}{\pgfqpoint{2.084390in}{1.841671in}}%
\pgfpathcurveto{\pgfqpoint{2.084390in}{1.833435in}}{\pgfqpoint{2.087662in}{1.825535in}}{\pgfqpoint{2.093486in}{1.819711in}}%
\pgfpathcurveto{\pgfqpoint{2.099310in}{1.813887in}}{\pgfqpoint{2.107210in}{1.810615in}}{\pgfqpoint{2.115446in}{1.810615in}}%
\pgfpathclose%
\pgfusepath{stroke,fill}%
\end{pgfscope}%
\begin{pgfscope}%
\pgfpathrectangle{\pgfqpoint{0.100000in}{0.212622in}}{\pgfqpoint{3.696000in}{3.696000in}}%
\pgfusepath{clip}%
\pgfsetbuttcap%
\pgfsetroundjoin%
\definecolor{currentfill}{rgb}{0.121569,0.466667,0.705882}%
\pgfsetfillcolor{currentfill}%
\pgfsetfillopacity{0.611904}%
\pgfsetlinewidth{1.003750pt}%
\definecolor{currentstroke}{rgb}{0.121569,0.466667,0.705882}%
\pgfsetstrokecolor{currentstroke}%
\pgfsetstrokeopacity{0.611904}%
\pgfsetdash{}{0pt}%
\pgfpathmoveto{\pgfqpoint{0.927694in}{1.322232in}}%
\pgfpathcurveto{\pgfqpoint{0.935930in}{1.322232in}}{\pgfqpoint{0.943830in}{1.325504in}}{\pgfqpoint{0.949654in}{1.331328in}}%
\pgfpathcurveto{\pgfqpoint{0.955478in}{1.337152in}}{\pgfqpoint{0.958750in}{1.345052in}}{\pgfqpoint{0.958750in}{1.353288in}}%
\pgfpathcurveto{\pgfqpoint{0.958750in}{1.361525in}}{\pgfqpoint{0.955478in}{1.369425in}}{\pgfqpoint{0.949654in}{1.375249in}}%
\pgfpathcurveto{\pgfqpoint{0.943830in}{1.381072in}}{\pgfqpoint{0.935930in}{1.384345in}}{\pgfqpoint{0.927694in}{1.384345in}}%
\pgfpathcurveto{\pgfqpoint{0.919458in}{1.384345in}}{\pgfqpoint{0.911558in}{1.381072in}}{\pgfqpoint{0.905734in}{1.375249in}}%
\pgfpathcurveto{\pgfqpoint{0.899910in}{1.369425in}}{\pgfqpoint{0.896638in}{1.361525in}}{\pgfqpoint{0.896638in}{1.353288in}}%
\pgfpathcurveto{\pgfqpoint{0.896638in}{1.345052in}}{\pgfqpoint{0.899910in}{1.337152in}}{\pgfqpoint{0.905734in}{1.331328in}}%
\pgfpathcurveto{\pgfqpoint{0.911558in}{1.325504in}}{\pgfqpoint{0.919458in}{1.322232in}}{\pgfqpoint{0.927694in}{1.322232in}}%
\pgfpathclose%
\pgfusepath{stroke,fill}%
\end{pgfscope}%
\begin{pgfscope}%
\pgfpathrectangle{\pgfqpoint{0.100000in}{0.212622in}}{\pgfqpoint{3.696000in}{3.696000in}}%
\pgfusepath{clip}%
\pgfsetbuttcap%
\pgfsetroundjoin%
\definecolor{currentfill}{rgb}{0.121569,0.466667,0.705882}%
\pgfsetfillcolor{currentfill}%
\pgfsetfillopacity{0.613186}%
\pgfsetlinewidth{1.003750pt}%
\definecolor{currentstroke}{rgb}{0.121569,0.466667,0.705882}%
\pgfsetstrokecolor{currentstroke}%
\pgfsetstrokeopacity{0.613186}%
\pgfsetdash{}{0pt}%
\pgfpathmoveto{\pgfqpoint{0.924409in}{1.315336in}}%
\pgfpathcurveto{\pgfqpoint{0.932645in}{1.315336in}}{\pgfqpoint{0.940545in}{1.318608in}}{\pgfqpoint{0.946369in}{1.324432in}}%
\pgfpathcurveto{\pgfqpoint{0.952193in}{1.330256in}}{\pgfqpoint{0.955466in}{1.338156in}}{\pgfqpoint{0.955466in}{1.346392in}}%
\pgfpathcurveto{\pgfqpoint{0.955466in}{1.354628in}}{\pgfqpoint{0.952193in}{1.362528in}}{\pgfqpoint{0.946369in}{1.368352in}}%
\pgfpathcurveto{\pgfqpoint{0.940545in}{1.374176in}}{\pgfqpoint{0.932645in}{1.377449in}}{\pgfqpoint{0.924409in}{1.377449in}}%
\pgfpathcurveto{\pgfqpoint{0.916173in}{1.377449in}}{\pgfqpoint{0.908273in}{1.374176in}}{\pgfqpoint{0.902449in}{1.368352in}}%
\pgfpathcurveto{\pgfqpoint{0.896625in}{1.362528in}}{\pgfqpoint{0.893353in}{1.354628in}}{\pgfqpoint{0.893353in}{1.346392in}}%
\pgfpathcurveto{\pgfqpoint{0.893353in}{1.338156in}}{\pgfqpoint{0.896625in}{1.330256in}}{\pgfqpoint{0.902449in}{1.324432in}}%
\pgfpathcurveto{\pgfqpoint{0.908273in}{1.318608in}}{\pgfqpoint{0.916173in}{1.315336in}}{\pgfqpoint{0.924409in}{1.315336in}}%
\pgfpathclose%
\pgfusepath{stroke,fill}%
\end{pgfscope}%
\begin{pgfscope}%
\pgfpathrectangle{\pgfqpoint{0.100000in}{0.212622in}}{\pgfqpoint{3.696000in}{3.696000in}}%
\pgfusepath{clip}%
\pgfsetbuttcap%
\pgfsetroundjoin%
\definecolor{currentfill}{rgb}{0.121569,0.466667,0.705882}%
\pgfsetfillcolor{currentfill}%
\pgfsetfillopacity{0.614082}%
\pgfsetlinewidth{1.003750pt}%
\definecolor{currentstroke}{rgb}{0.121569,0.466667,0.705882}%
\pgfsetstrokecolor{currentstroke}%
\pgfsetstrokeopacity{0.614082}%
\pgfsetdash{}{0pt}%
\pgfpathmoveto{\pgfqpoint{0.921854in}{1.314010in}}%
\pgfpathcurveto{\pgfqpoint{0.930090in}{1.314010in}}{\pgfqpoint{0.937990in}{1.317282in}}{\pgfqpoint{0.943814in}{1.323106in}}%
\pgfpathcurveto{\pgfqpoint{0.949638in}{1.328930in}}{\pgfqpoint{0.952910in}{1.336830in}}{\pgfqpoint{0.952910in}{1.345066in}}%
\pgfpathcurveto{\pgfqpoint{0.952910in}{1.353302in}}{\pgfqpoint{0.949638in}{1.361203in}}{\pgfqpoint{0.943814in}{1.367026in}}%
\pgfpathcurveto{\pgfqpoint{0.937990in}{1.372850in}}{\pgfqpoint{0.930090in}{1.376123in}}{\pgfqpoint{0.921854in}{1.376123in}}%
\pgfpathcurveto{\pgfqpoint{0.913618in}{1.376123in}}{\pgfqpoint{0.905717in}{1.372850in}}{\pgfqpoint{0.899894in}{1.367026in}}%
\pgfpathcurveto{\pgfqpoint{0.894070in}{1.361203in}}{\pgfqpoint{0.890797in}{1.353302in}}{\pgfqpoint{0.890797in}{1.345066in}}%
\pgfpathcurveto{\pgfqpoint{0.890797in}{1.336830in}}{\pgfqpoint{0.894070in}{1.328930in}}{\pgfqpoint{0.899894in}{1.323106in}}%
\pgfpathcurveto{\pgfqpoint{0.905717in}{1.317282in}}{\pgfqpoint{0.913618in}{1.314010in}}{\pgfqpoint{0.921854in}{1.314010in}}%
\pgfpathclose%
\pgfusepath{stroke,fill}%
\end{pgfscope}%
\begin{pgfscope}%
\pgfpathrectangle{\pgfqpoint{0.100000in}{0.212622in}}{\pgfqpoint{3.696000in}{3.696000in}}%
\pgfusepath{clip}%
\pgfsetbuttcap%
\pgfsetroundjoin%
\definecolor{currentfill}{rgb}{0.121569,0.466667,0.705882}%
\pgfsetfillcolor{currentfill}%
\pgfsetfillopacity{0.615022}%
\pgfsetlinewidth{1.003750pt}%
\definecolor{currentstroke}{rgb}{0.121569,0.466667,0.705882}%
\pgfsetstrokecolor{currentstroke}%
\pgfsetstrokeopacity{0.615022}%
\pgfsetdash{}{0pt}%
\pgfpathmoveto{\pgfqpoint{2.117745in}{1.806112in}}%
\pgfpathcurveto{\pgfqpoint{2.125981in}{1.806112in}}{\pgfqpoint{2.133881in}{1.809385in}}{\pgfqpoint{2.139705in}{1.815209in}}%
\pgfpathcurveto{\pgfqpoint{2.145529in}{1.821033in}}{\pgfqpoint{2.148801in}{1.828933in}}{\pgfqpoint{2.148801in}{1.837169in}}%
\pgfpathcurveto{\pgfqpoint{2.148801in}{1.845405in}}{\pgfqpoint{2.145529in}{1.853305in}}{\pgfqpoint{2.139705in}{1.859129in}}%
\pgfpathcurveto{\pgfqpoint{2.133881in}{1.864953in}}{\pgfqpoint{2.125981in}{1.868225in}}{\pgfqpoint{2.117745in}{1.868225in}}%
\pgfpathcurveto{\pgfqpoint{2.109508in}{1.868225in}}{\pgfqpoint{2.101608in}{1.864953in}}{\pgfqpoint{2.095784in}{1.859129in}}%
\pgfpathcurveto{\pgfqpoint{2.089960in}{1.853305in}}{\pgfqpoint{2.086688in}{1.845405in}}{\pgfqpoint{2.086688in}{1.837169in}}%
\pgfpathcurveto{\pgfqpoint{2.086688in}{1.828933in}}{\pgfqpoint{2.089960in}{1.821033in}}{\pgfqpoint{2.095784in}{1.815209in}}%
\pgfpathcurveto{\pgfqpoint{2.101608in}{1.809385in}}{\pgfqpoint{2.109508in}{1.806112in}}{\pgfqpoint{2.117745in}{1.806112in}}%
\pgfpathclose%
\pgfusepath{stroke,fill}%
\end{pgfscope}%
\begin{pgfscope}%
\pgfpathrectangle{\pgfqpoint{0.100000in}{0.212622in}}{\pgfqpoint{3.696000in}{3.696000in}}%
\pgfusepath{clip}%
\pgfsetbuttcap%
\pgfsetroundjoin%
\definecolor{currentfill}{rgb}{0.121569,0.466667,0.705882}%
\pgfsetfillcolor{currentfill}%
\pgfsetfillopacity{0.615616}%
\pgfsetlinewidth{1.003750pt}%
\definecolor{currentstroke}{rgb}{0.121569,0.466667,0.705882}%
\pgfsetstrokecolor{currentstroke}%
\pgfsetstrokeopacity{0.615616}%
\pgfsetdash{}{0pt}%
\pgfpathmoveto{\pgfqpoint{0.917750in}{1.310330in}}%
\pgfpathcurveto{\pgfqpoint{0.925987in}{1.310330in}}{\pgfqpoint{0.933887in}{1.313602in}}{\pgfqpoint{0.939711in}{1.319426in}}%
\pgfpathcurveto{\pgfqpoint{0.945534in}{1.325250in}}{\pgfqpoint{0.948807in}{1.333150in}}{\pgfqpoint{0.948807in}{1.341386in}}%
\pgfpathcurveto{\pgfqpoint{0.948807in}{1.349622in}}{\pgfqpoint{0.945534in}{1.357522in}}{\pgfqpoint{0.939711in}{1.363346in}}%
\pgfpathcurveto{\pgfqpoint{0.933887in}{1.369170in}}{\pgfqpoint{0.925987in}{1.372443in}}{\pgfqpoint{0.917750in}{1.372443in}}%
\pgfpathcurveto{\pgfqpoint{0.909514in}{1.372443in}}{\pgfqpoint{0.901614in}{1.369170in}}{\pgfqpoint{0.895790in}{1.363346in}}%
\pgfpathcurveto{\pgfqpoint{0.889966in}{1.357522in}}{\pgfqpoint{0.886694in}{1.349622in}}{\pgfqpoint{0.886694in}{1.341386in}}%
\pgfpathcurveto{\pgfqpoint{0.886694in}{1.333150in}}{\pgfqpoint{0.889966in}{1.325250in}}{\pgfqpoint{0.895790in}{1.319426in}}%
\pgfpathcurveto{\pgfqpoint{0.901614in}{1.313602in}}{\pgfqpoint{0.909514in}{1.310330in}}{\pgfqpoint{0.917750in}{1.310330in}}%
\pgfpathclose%
\pgfusepath{stroke,fill}%
\end{pgfscope}%
\begin{pgfscope}%
\pgfpathrectangle{\pgfqpoint{0.100000in}{0.212622in}}{\pgfqpoint{3.696000in}{3.696000in}}%
\pgfusepath{clip}%
\pgfsetbuttcap%
\pgfsetroundjoin%
\definecolor{currentfill}{rgb}{0.121569,0.466667,0.705882}%
\pgfsetfillcolor{currentfill}%
\pgfsetfillopacity{0.618191}%
\pgfsetlinewidth{1.003750pt}%
\definecolor{currentstroke}{rgb}{0.121569,0.466667,0.705882}%
\pgfsetstrokecolor{currentstroke}%
\pgfsetstrokeopacity{0.618191}%
\pgfsetdash{}{0pt}%
\pgfpathmoveto{\pgfqpoint{0.909591in}{1.303149in}}%
\pgfpathcurveto{\pgfqpoint{0.917827in}{1.303149in}}{\pgfqpoint{0.925727in}{1.306421in}}{\pgfqpoint{0.931551in}{1.312245in}}%
\pgfpathcurveto{\pgfqpoint{0.937375in}{1.318069in}}{\pgfqpoint{0.940647in}{1.325969in}}{\pgfqpoint{0.940647in}{1.334205in}}%
\pgfpathcurveto{\pgfqpoint{0.940647in}{1.342441in}}{\pgfqpoint{0.937375in}{1.350341in}}{\pgfqpoint{0.931551in}{1.356165in}}%
\pgfpathcurveto{\pgfqpoint{0.925727in}{1.361989in}}{\pgfqpoint{0.917827in}{1.365262in}}{\pgfqpoint{0.909591in}{1.365262in}}%
\pgfpathcurveto{\pgfqpoint{0.901355in}{1.365262in}}{\pgfqpoint{0.893454in}{1.361989in}}{\pgfqpoint{0.887631in}{1.356165in}}%
\pgfpathcurveto{\pgfqpoint{0.881807in}{1.350341in}}{\pgfqpoint{0.878534in}{1.342441in}}{\pgfqpoint{0.878534in}{1.334205in}}%
\pgfpathcurveto{\pgfqpoint{0.878534in}{1.325969in}}{\pgfqpoint{0.881807in}{1.318069in}}{\pgfqpoint{0.887631in}{1.312245in}}%
\pgfpathcurveto{\pgfqpoint{0.893454in}{1.306421in}}{\pgfqpoint{0.901355in}{1.303149in}}{\pgfqpoint{0.909591in}{1.303149in}}%
\pgfpathclose%
\pgfusepath{stroke,fill}%
\end{pgfscope}%
\begin{pgfscope}%
\pgfpathrectangle{\pgfqpoint{0.100000in}{0.212622in}}{\pgfqpoint{3.696000in}{3.696000in}}%
\pgfusepath{clip}%
\pgfsetbuttcap%
\pgfsetroundjoin%
\definecolor{currentfill}{rgb}{0.121569,0.466667,0.705882}%
\pgfsetfillcolor{currentfill}%
\pgfsetfillopacity{0.618962}%
\pgfsetlinewidth{1.003750pt}%
\definecolor{currentstroke}{rgb}{0.121569,0.466667,0.705882}%
\pgfsetstrokecolor{currentstroke}%
\pgfsetstrokeopacity{0.618962}%
\pgfsetdash{}{0pt}%
\pgfpathmoveto{\pgfqpoint{2.120504in}{1.802850in}}%
\pgfpathcurveto{\pgfqpoint{2.128741in}{1.802850in}}{\pgfqpoint{2.136641in}{1.806123in}}{\pgfqpoint{2.142465in}{1.811946in}}%
\pgfpathcurveto{\pgfqpoint{2.148289in}{1.817770in}}{\pgfqpoint{2.151561in}{1.825670in}}{\pgfqpoint{2.151561in}{1.833907in}}%
\pgfpathcurveto{\pgfqpoint{2.151561in}{1.842143in}}{\pgfqpoint{2.148289in}{1.850043in}}{\pgfqpoint{2.142465in}{1.855867in}}%
\pgfpathcurveto{\pgfqpoint{2.136641in}{1.861691in}}{\pgfqpoint{2.128741in}{1.864963in}}{\pgfqpoint{2.120504in}{1.864963in}}%
\pgfpathcurveto{\pgfqpoint{2.112268in}{1.864963in}}{\pgfqpoint{2.104368in}{1.861691in}}{\pgfqpoint{2.098544in}{1.855867in}}%
\pgfpathcurveto{\pgfqpoint{2.092720in}{1.850043in}}{\pgfqpoint{2.089448in}{1.842143in}}{\pgfqpoint{2.089448in}{1.833907in}}%
\pgfpathcurveto{\pgfqpoint{2.089448in}{1.825670in}}{\pgfqpoint{2.092720in}{1.817770in}}{\pgfqpoint{2.098544in}{1.811946in}}%
\pgfpathcurveto{\pgfqpoint{2.104368in}{1.806123in}}{\pgfqpoint{2.112268in}{1.802850in}}{\pgfqpoint{2.120504in}{1.802850in}}%
\pgfpathclose%
\pgfusepath{stroke,fill}%
\end{pgfscope}%
\begin{pgfscope}%
\pgfpathrectangle{\pgfqpoint{0.100000in}{0.212622in}}{\pgfqpoint{3.696000in}{3.696000in}}%
\pgfusepath{clip}%
\pgfsetbuttcap%
\pgfsetroundjoin%
\definecolor{currentfill}{rgb}{0.121569,0.466667,0.705882}%
\pgfsetfillcolor{currentfill}%
\pgfsetfillopacity{0.620174}%
\pgfsetlinewidth{1.003750pt}%
\definecolor{currentstroke}{rgb}{0.121569,0.466667,0.705882}%
\pgfsetstrokecolor{currentstroke}%
\pgfsetstrokeopacity{0.620174}%
\pgfsetdash{}{0pt}%
\pgfpathmoveto{\pgfqpoint{0.902359in}{1.297422in}}%
\pgfpathcurveto{\pgfqpoint{0.910595in}{1.297422in}}{\pgfqpoint{0.918495in}{1.300695in}}{\pgfqpoint{0.924319in}{1.306519in}}%
\pgfpathcurveto{\pgfqpoint{0.930143in}{1.312343in}}{\pgfqpoint{0.933415in}{1.320243in}}{\pgfqpoint{0.933415in}{1.328479in}}%
\pgfpathcurveto{\pgfqpoint{0.933415in}{1.336715in}}{\pgfqpoint{0.930143in}{1.344615in}}{\pgfqpoint{0.924319in}{1.350439in}}%
\pgfpathcurveto{\pgfqpoint{0.918495in}{1.356263in}}{\pgfqpoint{0.910595in}{1.359535in}}{\pgfqpoint{0.902359in}{1.359535in}}%
\pgfpathcurveto{\pgfqpoint{0.894123in}{1.359535in}}{\pgfqpoint{0.886223in}{1.356263in}}{\pgfqpoint{0.880399in}{1.350439in}}%
\pgfpathcurveto{\pgfqpoint{0.874575in}{1.344615in}}{\pgfqpoint{0.871302in}{1.336715in}}{\pgfqpoint{0.871302in}{1.328479in}}%
\pgfpathcurveto{\pgfqpoint{0.871302in}{1.320243in}}{\pgfqpoint{0.874575in}{1.312343in}}{\pgfqpoint{0.880399in}{1.306519in}}%
\pgfpathcurveto{\pgfqpoint{0.886223in}{1.300695in}}{\pgfqpoint{0.894123in}{1.297422in}}{\pgfqpoint{0.902359in}{1.297422in}}%
\pgfpathclose%
\pgfusepath{stroke,fill}%
\end{pgfscope}%
\begin{pgfscope}%
\pgfpathrectangle{\pgfqpoint{0.100000in}{0.212622in}}{\pgfqpoint{3.696000in}{3.696000in}}%
\pgfusepath{clip}%
\pgfsetbuttcap%
\pgfsetroundjoin%
\definecolor{currentfill}{rgb}{0.121569,0.466667,0.705882}%
\pgfsetfillcolor{currentfill}%
\pgfsetfillopacity{0.621748}%
\pgfsetlinewidth{1.003750pt}%
\definecolor{currentstroke}{rgb}{0.121569,0.466667,0.705882}%
\pgfsetstrokecolor{currentstroke}%
\pgfsetstrokeopacity{0.621748}%
\pgfsetdash{}{0pt}%
\pgfpathmoveto{\pgfqpoint{0.897808in}{1.292766in}}%
\pgfpathcurveto{\pgfqpoint{0.906044in}{1.292766in}}{\pgfqpoint{0.913944in}{1.296038in}}{\pgfqpoint{0.919768in}{1.301862in}}%
\pgfpathcurveto{\pgfqpoint{0.925592in}{1.307686in}}{\pgfqpoint{0.928864in}{1.315586in}}{\pgfqpoint{0.928864in}{1.323822in}}%
\pgfpathcurveto{\pgfqpoint{0.928864in}{1.332059in}}{\pgfqpoint{0.925592in}{1.339959in}}{\pgfqpoint{0.919768in}{1.345783in}}%
\pgfpathcurveto{\pgfqpoint{0.913944in}{1.351607in}}{\pgfqpoint{0.906044in}{1.354879in}}{\pgfqpoint{0.897808in}{1.354879in}}%
\pgfpathcurveto{\pgfqpoint{0.889571in}{1.354879in}}{\pgfqpoint{0.881671in}{1.351607in}}{\pgfqpoint{0.875847in}{1.345783in}}%
\pgfpathcurveto{\pgfqpoint{0.870023in}{1.339959in}}{\pgfqpoint{0.866751in}{1.332059in}}{\pgfqpoint{0.866751in}{1.323822in}}%
\pgfpathcurveto{\pgfqpoint{0.866751in}{1.315586in}}{\pgfqpoint{0.870023in}{1.307686in}}{\pgfqpoint{0.875847in}{1.301862in}}%
\pgfpathcurveto{\pgfqpoint{0.881671in}{1.296038in}}{\pgfqpoint{0.889571in}{1.292766in}}{\pgfqpoint{0.897808in}{1.292766in}}%
\pgfpathclose%
\pgfusepath{stroke,fill}%
\end{pgfscope}%
\begin{pgfscope}%
\pgfpathrectangle{\pgfqpoint{0.100000in}{0.212622in}}{\pgfqpoint{3.696000in}{3.696000in}}%
\pgfusepath{clip}%
\pgfsetbuttcap%
\pgfsetroundjoin%
\definecolor{currentfill}{rgb}{0.121569,0.466667,0.705882}%
\pgfsetfillcolor{currentfill}%
\pgfsetfillopacity{0.622094}%
\pgfsetlinewidth{1.003750pt}%
\definecolor{currentstroke}{rgb}{0.121569,0.466667,0.705882}%
\pgfsetstrokecolor{currentstroke}%
\pgfsetstrokeopacity{0.622094}%
\pgfsetdash{}{0pt}%
\pgfpathmoveto{\pgfqpoint{0.913757in}{1.254265in}}%
\pgfpathcurveto{\pgfqpoint{0.921993in}{1.254265in}}{\pgfqpoint{0.929893in}{1.257538in}}{\pgfqpoint{0.935717in}{1.263362in}}%
\pgfpathcurveto{\pgfqpoint{0.941541in}{1.269186in}}{\pgfqpoint{0.944813in}{1.277086in}}{\pgfqpoint{0.944813in}{1.285322in}}%
\pgfpathcurveto{\pgfqpoint{0.944813in}{1.293558in}}{\pgfqpoint{0.941541in}{1.301458in}}{\pgfqpoint{0.935717in}{1.307282in}}%
\pgfpathcurveto{\pgfqpoint{0.929893in}{1.313106in}}{\pgfqpoint{0.921993in}{1.316378in}}{\pgfqpoint{0.913757in}{1.316378in}}%
\pgfpathcurveto{\pgfqpoint{0.905520in}{1.316378in}}{\pgfqpoint{0.897620in}{1.313106in}}{\pgfqpoint{0.891796in}{1.307282in}}%
\pgfpathcurveto{\pgfqpoint{0.885972in}{1.301458in}}{\pgfqpoint{0.882700in}{1.293558in}}{\pgfqpoint{0.882700in}{1.285322in}}%
\pgfpathcurveto{\pgfqpoint{0.882700in}{1.277086in}}{\pgfqpoint{0.885972in}{1.269186in}}{\pgfqpoint{0.891796in}{1.263362in}}%
\pgfpathcurveto{\pgfqpoint{0.897620in}{1.257538in}}{\pgfqpoint{0.905520in}{1.254265in}}{\pgfqpoint{0.913757in}{1.254265in}}%
\pgfpathclose%
\pgfusepath{stroke,fill}%
\end{pgfscope}%
\begin{pgfscope}%
\pgfpathrectangle{\pgfqpoint{0.100000in}{0.212622in}}{\pgfqpoint{3.696000in}{3.696000in}}%
\pgfusepath{clip}%
\pgfsetbuttcap%
\pgfsetroundjoin%
\definecolor{currentfill}{rgb}{0.121569,0.466667,0.705882}%
\pgfsetfillcolor{currentfill}%
\pgfsetfillopacity{0.622215}%
\pgfsetlinewidth{1.003750pt}%
\definecolor{currentstroke}{rgb}{0.121569,0.466667,0.705882}%
\pgfsetstrokecolor{currentstroke}%
\pgfsetstrokeopacity{0.622215}%
\pgfsetdash{}{0pt}%
\pgfpathmoveto{\pgfqpoint{0.913535in}{1.254352in}}%
\pgfpathcurveto{\pgfqpoint{0.921772in}{1.254352in}}{\pgfqpoint{0.929672in}{1.257624in}}{\pgfqpoint{0.935496in}{1.263448in}}%
\pgfpathcurveto{\pgfqpoint{0.941320in}{1.269272in}}{\pgfqpoint{0.944592in}{1.277172in}}{\pgfqpoint{0.944592in}{1.285408in}}%
\pgfpathcurveto{\pgfqpoint{0.944592in}{1.293644in}}{\pgfqpoint{0.941320in}{1.301544in}}{\pgfqpoint{0.935496in}{1.307368in}}%
\pgfpathcurveto{\pgfqpoint{0.929672in}{1.313192in}}{\pgfqpoint{0.921772in}{1.316465in}}{\pgfqpoint{0.913535in}{1.316465in}}%
\pgfpathcurveto{\pgfqpoint{0.905299in}{1.316465in}}{\pgfqpoint{0.897399in}{1.313192in}}{\pgfqpoint{0.891575in}{1.307368in}}%
\pgfpathcurveto{\pgfqpoint{0.885751in}{1.301544in}}{\pgfqpoint{0.882479in}{1.293644in}}{\pgfqpoint{0.882479in}{1.285408in}}%
\pgfpathcurveto{\pgfqpoint{0.882479in}{1.277172in}}{\pgfqpoint{0.885751in}{1.269272in}}{\pgfqpoint{0.891575in}{1.263448in}}%
\pgfpathcurveto{\pgfqpoint{0.897399in}{1.257624in}}{\pgfqpoint{0.905299in}{1.254352in}}{\pgfqpoint{0.913535in}{1.254352in}}%
\pgfpathclose%
\pgfusepath{stroke,fill}%
\end{pgfscope}%
\begin{pgfscope}%
\pgfpathrectangle{\pgfqpoint{0.100000in}{0.212622in}}{\pgfqpoint{3.696000in}{3.696000in}}%
\pgfusepath{clip}%
\pgfsetbuttcap%
\pgfsetroundjoin%
\definecolor{currentfill}{rgb}{0.121569,0.466667,0.705882}%
\pgfsetfillcolor{currentfill}%
\pgfsetfillopacity{0.622555}%
\pgfsetlinewidth{1.003750pt}%
\definecolor{currentstroke}{rgb}{0.121569,0.466667,0.705882}%
\pgfsetstrokecolor{currentstroke}%
\pgfsetstrokeopacity{0.622555}%
\pgfsetdash{}{0pt}%
\pgfpathmoveto{\pgfqpoint{0.912907in}{1.254601in}}%
\pgfpathcurveto{\pgfqpoint{0.921144in}{1.254601in}}{\pgfqpoint{0.929044in}{1.257874in}}{\pgfqpoint{0.934868in}{1.263698in}}%
\pgfpathcurveto{\pgfqpoint{0.940692in}{1.269522in}}{\pgfqpoint{0.943964in}{1.277422in}}{\pgfqpoint{0.943964in}{1.285658in}}%
\pgfpathcurveto{\pgfqpoint{0.943964in}{1.293894in}}{\pgfqpoint{0.940692in}{1.301794in}}{\pgfqpoint{0.934868in}{1.307618in}}%
\pgfpathcurveto{\pgfqpoint{0.929044in}{1.313442in}}{\pgfqpoint{0.921144in}{1.316714in}}{\pgfqpoint{0.912907in}{1.316714in}}%
\pgfpathcurveto{\pgfqpoint{0.904671in}{1.316714in}}{\pgfqpoint{0.896771in}{1.313442in}}{\pgfqpoint{0.890947in}{1.307618in}}%
\pgfpathcurveto{\pgfqpoint{0.885123in}{1.301794in}}{\pgfqpoint{0.881851in}{1.293894in}}{\pgfqpoint{0.881851in}{1.285658in}}%
\pgfpathcurveto{\pgfqpoint{0.881851in}{1.277422in}}{\pgfqpoint{0.885123in}{1.269522in}}{\pgfqpoint{0.890947in}{1.263698in}}%
\pgfpathcurveto{\pgfqpoint{0.896771in}{1.257874in}}{\pgfqpoint{0.904671in}{1.254601in}}{\pgfqpoint{0.912907in}{1.254601in}}%
\pgfpathclose%
\pgfusepath{stroke,fill}%
\end{pgfscope}%
\begin{pgfscope}%
\pgfpathrectangle{\pgfqpoint{0.100000in}{0.212622in}}{\pgfqpoint{3.696000in}{3.696000in}}%
\pgfusepath{clip}%
\pgfsetbuttcap%
\pgfsetroundjoin%
\definecolor{currentfill}{rgb}{0.121569,0.466667,0.705882}%
\pgfsetfillcolor{currentfill}%
\pgfsetfillopacity{0.622781}%
\pgfsetlinewidth{1.003750pt}%
\definecolor{currentstroke}{rgb}{0.121569,0.466667,0.705882}%
\pgfsetstrokecolor{currentstroke}%
\pgfsetstrokeopacity{0.622781}%
\pgfsetdash{}{0pt}%
\pgfpathmoveto{\pgfqpoint{0.894630in}{1.291939in}}%
\pgfpathcurveto{\pgfqpoint{0.902866in}{1.291939in}}{\pgfqpoint{0.910766in}{1.295211in}}{\pgfqpoint{0.916590in}{1.301035in}}%
\pgfpathcurveto{\pgfqpoint{0.922414in}{1.306859in}}{\pgfqpoint{0.925686in}{1.314759in}}{\pgfqpoint{0.925686in}{1.322996in}}%
\pgfpathcurveto{\pgfqpoint{0.925686in}{1.331232in}}{\pgfqpoint{0.922414in}{1.339132in}}{\pgfqpoint{0.916590in}{1.344956in}}%
\pgfpathcurveto{\pgfqpoint{0.910766in}{1.350780in}}{\pgfqpoint{0.902866in}{1.354052in}}{\pgfqpoint{0.894630in}{1.354052in}}%
\pgfpathcurveto{\pgfqpoint{0.886394in}{1.354052in}}{\pgfqpoint{0.878494in}{1.350780in}}{\pgfqpoint{0.872670in}{1.344956in}}%
\pgfpathcurveto{\pgfqpoint{0.866846in}{1.339132in}}{\pgfqpoint{0.863573in}{1.331232in}}{\pgfqpoint{0.863573in}{1.322996in}}%
\pgfpathcurveto{\pgfqpoint{0.863573in}{1.314759in}}{\pgfqpoint{0.866846in}{1.306859in}}{\pgfqpoint{0.872670in}{1.301035in}}%
\pgfpathcurveto{\pgfqpoint{0.878494in}{1.295211in}}{\pgfqpoint{0.886394in}{1.291939in}}{\pgfqpoint{0.894630in}{1.291939in}}%
\pgfpathclose%
\pgfusepath{stroke,fill}%
\end{pgfscope}%
\begin{pgfscope}%
\pgfpathrectangle{\pgfqpoint{0.100000in}{0.212622in}}{\pgfqpoint{3.696000in}{3.696000in}}%
\pgfusepath{clip}%
\pgfsetbuttcap%
\pgfsetroundjoin%
\definecolor{currentfill}{rgb}{0.121569,0.466667,0.705882}%
\pgfsetfillcolor{currentfill}%
\pgfsetfillopacity{0.623107}%
\pgfsetlinewidth{1.003750pt}%
\definecolor{currentstroke}{rgb}{0.121569,0.466667,0.705882}%
\pgfsetstrokecolor{currentstroke}%
\pgfsetstrokeopacity{0.623107}%
\pgfsetdash{}{0pt}%
\pgfpathmoveto{\pgfqpoint{2.122341in}{1.799056in}}%
\pgfpathcurveto{\pgfqpoint{2.130577in}{1.799056in}}{\pgfqpoint{2.138477in}{1.802329in}}{\pgfqpoint{2.144301in}{1.808153in}}%
\pgfpathcurveto{\pgfqpoint{2.150125in}{1.813977in}}{\pgfqpoint{2.153397in}{1.821877in}}{\pgfqpoint{2.153397in}{1.830113in}}%
\pgfpathcurveto{\pgfqpoint{2.153397in}{1.838349in}}{\pgfqpoint{2.150125in}{1.846249in}}{\pgfqpoint{2.144301in}{1.852073in}}%
\pgfpathcurveto{\pgfqpoint{2.138477in}{1.857897in}}{\pgfqpoint{2.130577in}{1.861169in}}{\pgfqpoint{2.122341in}{1.861169in}}%
\pgfpathcurveto{\pgfqpoint{2.114104in}{1.861169in}}{\pgfqpoint{2.106204in}{1.857897in}}{\pgfqpoint{2.100380in}{1.852073in}}%
\pgfpathcurveto{\pgfqpoint{2.094557in}{1.846249in}}{\pgfqpoint{2.091284in}{1.838349in}}{\pgfqpoint{2.091284in}{1.830113in}}%
\pgfpathcurveto{\pgfqpoint{2.091284in}{1.821877in}}{\pgfqpoint{2.094557in}{1.813977in}}{\pgfqpoint{2.100380in}{1.808153in}}%
\pgfpathcurveto{\pgfqpoint{2.106204in}{1.802329in}}{\pgfqpoint{2.114104in}{1.799056in}}{\pgfqpoint{2.122341in}{1.799056in}}%
\pgfpathclose%
\pgfusepath{stroke,fill}%
\end{pgfscope}%
\begin{pgfscope}%
\pgfpathrectangle{\pgfqpoint{0.100000in}{0.212622in}}{\pgfqpoint{3.696000in}{3.696000in}}%
\pgfusepath{clip}%
\pgfsetbuttcap%
\pgfsetroundjoin%
\definecolor{currentfill}{rgb}{0.121569,0.466667,0.705882}%
\pgfsetfillcolor{currentfill}%
\pgfsetfillopacity{0.623181}%
\pgfsetlinewidth{1.003750pt}%
\definecolor{currentstroke}{rgb}{0.121569,0.466667,0.705882}%
\pgfsetstrokecolor{currentstroke}%
\pgfsetstrokeopacity{0.623181}%
\pgfsetdash{}{0pt}%
\pgfpathmoveto{\pgfqpoint{0.893467in}{1.290280in}}%
\pgfpathcurveto{\pgfqpoint{0.901704in}{1.290280in}}{\pgfqpoint{0.909604in}{1.293553in}}{\pgfqpoint{0.915428in}{1.299376in}}%
\pgfpathcurveto{\pgfqpoint{0.921252in}{1.305200in}}{\pgfqpoint{0.924524in}{1.313100in}}{\pgfqpoint{0.924524in}{1.321337in}}%
\pgfpathcurveto{\pgfqpoint{0.924524in}{1.329573in}}{\pgfqpoint{0.921252in}{1.337473in}}{\pgfqpoint{0.915428in}{1.343297in}}%
\pgfpathcurveto{\pgfqpoint{0.909604in}{1.349121in}}{\pgfqpoint{0.901704in}{1.352393in}}{\pgfqpoint{0.893467in}{1.352393in}}%
\pgfpathcurveto{\pgfqpoint{0.885231in}{1.352393in}}{\pgfqpoint{0.877331in}{1.349121in}}{\pgfqpoint{0.871507in}{1.343297in}}%
\pgfpathcurveto{\pgfqpoint{0.865683in}{1.337473in}}{\pgfqpoint{0.862411in}{1.329573in}}{\pgfqpoint{0.862411in}{1.321337in}}%
\pgfpathcurveto{\pgfqpoint{0.862411in}{1.313100in}}{\pgfqpoint{0.865683in}{1.305200in}}{\pgfqpoint{0.871507in}{1.299376in}}%
\pgfpathcurveto{\pgfqpoint{0.877331in}{1.293553in}}{\pgfqpoint{0.885231in}{1.290280in}}{\pgfqpoint{0.893467in}{1.290280in}}%
\pgfpathclose%
\pgfusepath{stroke,fill}%
\end{pgfscope}%
\begin{pgfscope}%
\pgfpathrectangle{\pgfqpoint{0.100000in}{0.212622in}}{\pgfqpoint{3.696000in}{3.696000in}}%
\pgfusepath{clip}%
\pgfsetbuttcap%
\pgfsetroundjoin%
\definecolor{currentfill}{rgb}{0.121569,0.466667,0.705882}%
\pgfsetfillcolor{currentfill}%
\pgfsetfillopacity{0.623181}%
\pgfsetlinewidth{1.003750pt}%
\definecolor{currentstroke}{rgb}{0.121569,0.466667,0.705882}%
\pgfsetstrokecolor{currentstroke}%
\pgfsetstrokeopacity{0.623181}%
\pgfsetdash{}{0pt}%
\pgfpathmoveto{\pgfqpoint{0.893467in}{1.290280in}}%
\pgfpathcurveto{\pgfqpoint{0.901704in}{1.290280in}}{\pgfqpoint{0.909604in}{1.293552in}}{\pgfqpoint{0.915428in}{1.299376in}}%
\pgfpathcurveto{\pgfqpoint{0.921252in}{1.305200in}}{\pgfqpoint{0.924524in}{1.313100in}}{\pgfqpoint{0.924524in}{1.321337in}}%
\pgfpathcurveto{\pgfqpoint{0.924524in}{1.329573in}}{\pgfqpoint{0.921252in}{1.337473in}}{\pgfqpoint{0.915428in}{1.343297in}}%
\pgfpathcurveto{\pgfqpoint{0.909604in}{1.349121in}}{\pgfqpoint{0.901704in}{1.352393in}}{\pgfqpoint{0.893467in}{1.352393in}}%
\pgfpathcurveto{\pgfqpoint{0.885231in}{1.352393in}}{\pgfqpoint{0.877331in}{1.349121in}}{\pgfqpoint{0.871507in}{1.343297in}}%
\pgfpathcurveto{\pgfqpoint{0.865683in}{1.337473in}}{\pgfqpoint{0.862411in}{1.329573in}}{\pgfqpoint{0.862411in}{1.321337in}}%
\pgfpathcurveto{\pgfqpoint{0.862411in}{1.313100in}}{\pgfqpoint{0.865683in}{1.305200in}}{\pgfqpoint{0.871507in}{1.299376in}}%
\pgfpathcurveto{\pgfqpoint{0.877331in}{1.293552in}}{\pgfqpoint{0.885231in}{1.290280in}}{\pgfqpoint{0.893467in}{1.290280in}}%
\pgfpathclose%
\pgfusepath{stroke,fill}%
\end{pgfscope}%
\begin{pgfscope}%
\pgfpathrectangle{\pgfqpoint{0.100000in}{0.212622in}}{\pgfqpoint{3.696000in}{3.696000in}}%
\pgfusepath{clip}%
\pgfsetbuttcap%
\pgfsetroundjoin%
\definecolor{currentfill}{rgb}{0.121569,0.466667,0.705882}%
\pgfsetfillcolor{currentfill}%
\pgfsetfillopacity{0.623181}%
\pgfsetlinewidth{1.003750pt}%
\definecolor{currentstroke}{rgb}{0.121569,0.466667,0.705882}%
\pgfsetstrokecolor{currentstroke}%
\pgfsetstrokeopacity{0.623181}%
\pgfsetdash{}{0pt}%
\pgfpathmoveto{\pgfqpoint{0.893467in}{1.290280in}}%
\pgfpathcurveto{\pgfqpoint{0.901704in}{1.290280in}}{\pgfqpoint{0.909604in}{1.293552in}}{\pgfqpoint{0.915428in}{1.299376in}}%
\pgfpathcurveto{\pgfqpoint{0.921252in}{1.305200in}}{\pgfqpoint{0.924524in}{1.313100in}}{\pgfqpoint{0.924524in}{1.321337in}}%
\pgfpathcurveto{\pgfqpoint{0.924524in}{1.329573in}}{\pgfqpoint{0.921252in}{1.337473in}}{\pgfqpoint{0.915428in}{1.343297in}}%
\pgfpathcurveto{\pgfqpoint{0.909604in}{1.349121in}}{\pgfqpoint{0.901704in}{1.352393in}}{\pgfqpoint{0.893467in}{1.352393in}}%
\pgfpathcurveto{\pgfqpoint{0.885231in}{1.352393in}}{\pgfqpoint{0.877331in}{1.349121in}}{\pgfqpoint{0.871507in}{1.343297in}}%
\pgfpathcurveto{\pgfqpoint{0.865683in}{1.337473in}}{\pgfqpoint{0.862411in}{1.329573in}}{\pgfqpoint{0.862411in}{1.321337in}}%
\pgfpathcurveto{\pgfqpoint{0.862411in}{1.313100in}}{\pgfqpoint{0.865683in}{1.305200in}}{\pgfqpoint{0.871507in}{1.299376in}}%
\pgfpathcurveto{\pgfqpoint{0.877331in}{1.293552in}}{\pgfqpoint{0.885231in}{1.290280in}}{\pgfqpoint{0.893467in}{1.290280in}}%
\pgfpathclose%
\pgfusepath{stroke,fill}%
\end{pgfscope}%
\begin{pgfscope}%
\pgfpathrectangle{\pgfqpoint{0.100000in}{0.212622in}}{\pgfqpoint{3.696000in}{3.696000in}}%
\pgfusepath{clip}%
\pgfsetbuttcap%
\pgfsetroundjoin%
\definecolor{currentfill}{rgb}{0.121569,0.466667,0.705882}%
\pgfsetfillcolor{currentfill}%
\pgfsetfillopacity{0.623181}%
\pgfsetlinewidth{1.003750pt}%
\definecolor{currentstroke}{rgb}{0.121569,0.466667,0.705882}%
\pgfsetstrokecolor{currentstroke}%
\pgfsetstrokeopacity{0.623181}%
\pgfsetdash{}{0pt}%
\pgfpathmoveto{\pgfqpoint{0.893467in}{1.290280in}}%
\pgfpathcurveto{\pgfqpoint{0.901703in}{1.290280in}}{\pgfqpoint{0.909603in}{1.293552in}}{\pgfqpoint{0.915427in}{1.299376in}}%
\pgfpathcurveto{\pgfqpoint{0.921251in}{1.305200in}}{\pgfqpoint{0.924524in}{1.313100in}}{\pgfqpoint{0.924524in}{1.321336in}}%
\pgfpathcurveto{\pgfqpoint{0.924524in}{1.329573in}}{\pgfqpoint{0.921251in}{1.337473in}}{\pgfqpoint{0.915427in}{1.343297in}}%
\pgfpathcurveto{\pgfqpoint{0.909603in}{1.349121in}}{\pgfqpoint{0.901703in}{1.352393in}}{\pgfqpoint{0.893467in}{1.352393in}}%
\pgfpathcurveto{\pgfqpoint{0.885231in}{1.352393in}}{\pgfqpoint{0.877331in}{1.349121in}}{\pgfqpoint{0.871507in}{1.343297in}}%
\pgfpathcurveto{\pgfqpoint{0.865683in}{1.337473in}}{\pgfqpoint{0.862411in}{1.329573in}}{\pgfqpoint{0.862411in}{1.321336in}}%
\pgfpathcurveto{\pgfqpoint{0.862411in}{1.313100in}}{\pgfqpoint{0.865683in}{1.305200in}}{\pgfqpoint{0.871507in}{1.299376in}}%
\pgfpathcurveto{\pgfqpoint{0.877331in}{1.293552in}}{\pgfqpoint{0.885231in}{1.290280in}}{\pgfqpoint{0.893467in}{1.290280in}}%
\pgfpathclose%
\pgfusepath{stroke,fill}%
\end{pgfscope}%
\begin{pgfscope}%
\pgfpathrectangle{\pgfqpoint{0.100000in}{0.212622in}}{\pgfqpoint{3.696000in}{3.696000in}}%
\pgfusepath{clip}%
\pgfsetbuttcap%
\pgfsetroundjoin%
\definecolor{currentfill}{rgb}{0.121569,0.466667,0.705882}%
\pgfsetfillcolor{currentfill}%
\pgfsetfillopacity{0.623181}%
\pgfsetlinewidth{1.003750pt}%
\definecolor{currentstroke}{rgb}{0.121569,0.466667,0.705882}%
\pgfsetstrokecolor{currentstroke}%
\pgfsetstrokeopacity{0.623181}%
\pgfsetdash{}{0pt}%
\pgfpathmoveto{\pgfqpoint{0.893467in}{1.290280in}}%
\pgfpathcurveto{\pgfqpoint{0.901703in}{1.290280in}}{\pgfqpoint{0.909603in}{1.293552in}}{\pgfqpoint{0.915427in}{1.299376in}}%
\pgfpathcurveto{\pgfqpoint{0.921251in}{1.305200in}}{\pgfqpoint{0.924523in}{1.313100in}}{\pgfqpoint{0.924523in}{1.321336in}}%
\pgfpathcurveto{\pgfqpoint{0.924523in}{1.329573in}}{\pgfqpoint{0.921251in}{1.337473in}}{\pgfqpoint{0.915427in}{1.343297in}}%
\pgfpathcurveto{\pgfqpoint{0.909603in}{1.349120in}}{\pgfqpoint{0.901703in}{1.352393in}}{\pgfqpoint{0.893467in}{1.352393in}}%
\pgfpathcurveto{\pgfqpoint{0.885231in}{1.352393in}}{\pgfqpoint{0.877331in}{1.349120in}}{\pgfqpoint{0.871507in}{1.343297in}}%
\pgfpathcurveto{\pgfqpoint{0.865683in}{1.337473in}}{\pgfqpoint{0.862410in}{1.329573in}}{\pgfqpoint{0.862410in}{1.321336in}}%
\pgfpathcurveto{\pgfqpoint{0.862410in}{1.313100in}}{\pgfqpoint{0.865683in}{1.305200in}}{\pgfqpoint{0.871507in}{1.299376in}}%
\pgfpathcurveto{\pgfqpoint{0.877331in}{1.293552in}}{\pgfqpoint{0.885231in}{1.290280in}}{\pgfqpoint{0.893467in}{1.290280in}}%
\pgfpathclose%
\pgfusepath{stroke,fill}%
\end{pgfscope}%
\begin{pgfscope}%
\pgfpathrectangle{\pgfqpoint{0.100000in}{0.212622in}}{\pgfqpoint{3.696000in}{3.696000in}}%
\pgfusepath{clip}%
\pgfsetbuttcap%
\pgfsetroundjoin%
\definecolor{currentfill}{rgb}{0.121569,0.466667,0.705882}%
\pgfsetfillcolor{currentfill}%
\pgfsetfillopacity{0.623181}%
\pgfsetlinewidth{1.003750pt}%
\definecolor{currentstroke}{rgb}{0.121569,0.466667,0.705882}%
\pgfsetstrokecolor{currentstroke}%
\pgfsetstrokeopacity{0.623181}%
\pgfsetdash{}{0pt}%
\pgfpathmoveto{\pgfqpoint{0.893466in}{1.290280in}}%
\pgfpathcurveto{\pgfqpoint{0.901703in}{1.290280in}}{\pgfqpoint{0.909603in}{1.293552in}}{\pgfqpoint{0.915427in}{1.299376in}}%
\pgfpathcurveto{\pgfqpoint{0.921250in}{1.305200in}}{\pgfqpoint{0.924523in}{1.313100in}}{\pgfqpoint{0.924523in}{1.321336in}}%
\pgfpathcurveto{\pgfqpoint{0.924523in}{1.329572in}}{\pgfqpoint{0.921250in}{1.337472in}}{\pgfqpoint{0.915427in}{1.343296in}}%
\pgfpathcurveto{\pgfqpoint{0.909603in}{1.349120in}}{\pgfqpoint{0.901703in}{1.352393in}}{\pgfqpoint{0.893466in}{1.352393in}}%
\pgfpathcurveto{\pgfqpoint{0.885230in}{1.352393in}}{\pgfqpoint{0.877330in}{1.349120in}}{\pgfqpoint{0.871506in}{1.343296in}}%
\pgfpathcurveto{\pgfqpoint{0.865682in}{1.337472in}}{\pgfqpoint{0.862410in}{1.329572in}}{\pgfqpoint{0.862410in}{1.321336in}}%
\pgfpathcurveto{\pgfqpoint{0.862410in}{1.313100in}}{\pgfqpoint{0.865682in}{1.305200in}}{\pgfqpoint{0.871506in}{1.299376in}}%
\pgfpathcurveto{\pgfqpoint{0.877330in}{1.293552in}}{\pgfqpoint{0.885230in}{1.290280in}}{\pgfqpoint{0.893466in}{1.290280in}}%
\pgfpathclose%
\pgfusepath{stroke,fill}%
\end{pgfscope}%
\begin{pgfscope}%
\pgfpathrectangle{\pgfqpoint{0.100000in}{0.212622in}}{\pgfqpoint{3.696000in}{3.696000in}}%
\pgfusepath{clip}%
\pgfsetbuttcap%
\pgfsetroundjoin%
\definecolor{currentfill}{rgb}{0.121569,0.466667,0.705882}%
\pgfsetfillcolor{currentfill}%
\pgfsetfillopacity{0.623181}%
\pgfsetlinewidth{1.003750pt}%
\definecolor{currentstroke}{rgb}{0.121569,0.466667,0.705882}%
\pgfsetstrokecolor{currentstroke}%
\pgfsetstrokeopacity{0.623181}%
\pgfsetdash{}{0pt}%
\pgfpathmoveto{\pgfqpoint{0.893465in}{1.290279in}}%
\pgfpathcurveto{\pgfqpoint{0.901702in}{1.290279in}}{\pgfqpoint{0.909602in}{1.293551in}}{\pgfqpoint{0.915426in}{1.299375in}}%
\pgfpathcurveto{\pgfqpoint{0.921249in}{1.305199in}}{\pgfqpoint{0.924522in}{1.313099in}}{\pgfqpoint{0.924522in}{1.321335in}}%
\pgfpathcurveto{\pgfqpoint{0.924522in}{1.329572in}}{\pgfqpoint{0.921249in}{1.337472in}}{\pgfqpoint{0.915426in}{1.343296in}}%
\pgfpathcurveto{\pgfqpoint{0.909602in}{1.349120in}}{\pgfqpoint{0.901702in}{1.352392in}}{\pgfqpoint{0.893465in}{1.352392in}}%
\pgfpathcurveto{\pgfqpoint{0.885229in}{1.352392in}}{\pgfqpoint{0.877329in}{1.349120in}}{\pgfqpoint{0.871505in}{1.343296in}}%
\pgfpathcurveto{\pgfqpoint{0.865681in}{1.337472in}}{\pgfqpoint{0.862409in}{1.329572in}}{\pgfqpoint{0.862409in}{1.321335in}}%
\pgfpathcurveto{\pgfqpoint{0.862409in}{1.313099in}}{\pgfqpoint{0.865681in}{1.305199in}}{\pgfqpoint{0.871505in}{1.299375in}}%
\pgfpathcurveto{\pgfqpoint{0.877329in}{1.293551in}}{\pgfqpoint{0.885229in}{1.290279in}}{\pgfqpoint{0.893465in}{1.290279in}}%
\pgfpathclose%
\pgfusepath{stroke,fill}%
\end{pgfscope}%
\begin{pgfscope}%
\pgfpathrectangle{\pgfqpoint{0.100000in}{0.212622in}}{\pgfqpoint{3.696000in}{3.696000in}}%
\pgfusepath{clip}%
\pgfsetbuttcap%
\pgfsetroundjoin%
\definecolor{currentfill}{rgb}{0.121569,0.466667,0.705882}%
\pgfsetfillcolor{currentfill}%
\pgfsetfillopacity{0.623182}%
\pgfsetlinewidth{1.003750pt}%
\definecolor{currentstroke}{rgb}{0.121569,0.466667,0.705882}%
\pgfsetstrokecolor{currentstroke}%
\pgfsetstrokeopacity{0.623182}%
\pgfsetdash{}{0pt}%
\pgfpathmoveto{\pgfqpoint{0.893463in}{1.290277in}}%
\pgfpathcurveto{\pgfqpoint{0.901700in}{1.290277in}}{\pgfqpoint{0.909600in}{1.293550in}}{\pgfqpoint{0.915424in}{1.299374in}}%
\pgfpathcurveto{\pgfqpoint{0.921247in}{1.305198in}}{\pgfqpoint{0.924520in}{1.313098in}}{\pgfqpoint{0.924520in}{1.321334in}}%
\pgfpathcurveto{\pgfqpoint{0.924520in}{1.329570in}}{\pgfqpoint{0.921247in}{1.337470in}}{\pgfqpoint{0.915424in}{1.343294in}}%
\pgfpathcurveto{\pgfqpoint{0.909600in}{1.349118in}}{\pgfqpoint{0.901700in}{1.352390in}}{\pgfqpoint{0.893463in}{1.352390in}}%
\pgfpathcurveto{\pgfqpoint{0.885227in}{1.352390in}}{\pgfqpoint{0.877327in}{1.349118in}}{\pgfqpoint{0.871503in}{1.343294in}}%
\pgfpathcurveto{\pgfqpoint{0.865679in}{1.337470in}}{\pgfqpoint{0.862407in}{1.329570in}}{\pgfqpoint{0.862407in}{1.321334in}}%
\pgfpathcurveto{\pgfqpoint{0.862407in}{1.313098in}}{\pgfqpoint{0.865679in}{1.305198in}}{\pgfqpoint{0.871503in}{1.299374in}}%
\pgfpathcurveto{\pgfqpoint{0.877327in}{1.293550in}}{\pgfqpoint{0.885227in}{1.290277in}}{\pgfqpoint{0.893463in}{1.290277in}}%
\pgfpathclose%
\pgfusepath{stroke,fill}%
\end{pgfscope}%
\begin{pgfscope}%
\pgfpathrectangle{\pgfqpoint{0.100000in}{0.212622in}}{\pgfqpoint{3.696000in}{3.696000in}}%
\pgfusepath{clip}%
\pgfsetbuttcap%
\pgfsetroundjoin%
\definecolor{currentfill}{rgb}{0.121569,0.466667,0.705882}%
\pgfsetfillcolor{currentfill}%
\pgfsetfillopacity{0.623183}%
\pgfsetlinewidth{1.003750pt}%
\definecolor{currentstroke}{rgb}{0.121569,0.466667,0.705882}%
\pgfsetstrokecolor{currentstroke}%
\pgfsetstrokeopacity{0.623183}%
\pgfsetdash{}{0pt}%
\pgfpathmoveto{\pgfqpoint{0.893460in}{1.290275in}}%
\pgfpathcurveto{\pgfqpoint{0.901696in}{1.290275in}}{\pgfqpoint{0.909596in}{1.293547in}}{\pgfqpoint{0.915420in}{1.299371in}}%
\pgfpathcurveto{\pgfqpoint{0.921244in}{1.305195in}}{\pgfqpoint{0.924516in}{1.313095in}}{\pgfqpoint{0.924516in}{1.321331in}}%
\pgfpathcurveto{\pgfqpoint{0.924516in}{1.329567in}}{\pgfqpoint{0.921244in}{1.337468in}}{\pgfqpoint{0.915420in}{1.343291in}}%
\pgfpathcurveto{\pgfqpoint{0.909596in}{1.349115in}}{\pgfqpoint{0.901696in}{1.352388in}}{\pgfqpoint{0.893460in}{1.352388in}}%
\pgfpathcurveto{\pgfqpoint{0.885224in}{1.352388in}}{\pgfqpoint{0.877324in}{1.349115in}}{\pgfqpoint{0.871500in}{1.343291in}}%
\pgfpathcurveto{\pgfqpoint{0.865676in}{1.337468in}}{\pgfqpoint{0.862403in}{1.329567in}}{\pgfqpoint{0.862403in}{1.321331in}}%
\pgfpathcurveto{\pgfqpoint{0.862403in}{1.313095in}}{\pgfqpoint{0.865676in}{1.305195in}}{\pgfqpoint{0.871500in}{1.299371in}}%
\pgfpathcurveto{\pgfqpoint{0.877324in}{1.293547in}}{\pgfqpoint{0.885224in}{1.290275in}}{\pgfqpoint{0.893460in}{1.290275in}}%
\pgfpathclose%
\pgfusepath{stroke,fill}%
\end{pgfscope}%
\begin{pgfscope}%
\pgfpathrectangle{\pgfqpoint{0.100000in}{0.212622in}}{\pgfqpoint{3.696000in}{3.696000in}}%
\pgfusepath{clip}%
\pgfsetbuttcap%
\pgfsetroundjoin%
\definecolor{currentfill}{rgb}{0.121569,0.466667,0.705882}%
\pgfsetfillcolor{currentfill}%
\pgfsetfillopacity{0.623185}%
\pgfsetlinewidth{1.003750pt}%
\definecolor{currentstroke}{rgb}{0.121569,0.466667,0.705882}%
\pgfsetstrokecolor{currentstroke}%
\pgfsetstrokeopacity{0.623185}%
\pgfsetdash{}{0pt}%
\pgfpathmoveto{\pgfqpoint{0.893453in}{1.290271in}}%
\pgfpathcurveto{\pgfqpoint{0.901690in}{1.290271in}}{\pgfqpoint{0.909590in}{1.293544in}}{\pgfqpoint{0.915413in}{1.299368in}}%
\pgfpathcurveto{\pgfqpoint{0.921237in}{1.305192in}}{\pgfqpoint{0.924510in}{1.313092in}}{\pgfqpoint{0.924510in}{1.321328in}}%
\pgfpathcurveto{\pgfqpoint{0.924510in}{1.329564in}}{\pgfqpoint{0.921237in}{1.337464in}}{\pgfqpoint{0.915413in}{1.343288in}}%
\pgfpathcurveto{\pgfqpoint{0.909590in}{1.349112in}}{\pgfqpoint{0.901690in}{1.352384in}}{\pgfqpoint{0.893453in}{1.352384in}}%
\pgfpathcurveto{\pgfqpoint{0.885217in}{1.352384in}}{\pgfqpoint{0.877317in}{1.349112in}}{\pgfqpoint{0.871493in}{1.343288in}}%
\pgfpathcurveto{\pgfqpoint{0.865669in}{1.337464in}}{\pgfqpoint{0.862397in}{1.329564in}}{\pgfqpoint{0.862397in}{1.321328in}}%
\pgfpathcurveto{\pgfqpoint{0.862397in}{1.313092in}}{\pgfqpoint{0.865669in}{1.305192in}}{\pgfqpoint{0.871493in}{1.299368in}}%
\pgfpathcurveto{\pgfqpoint{0.877317in}{1.293544in}}{\pgfqpoint{0.885217in}{1.290271in}}{\pgfqpoint{0.893453in}{1.290271in}}%
\pgfpathclose%
\pgfusepath{stroke,fill}%
\end{pgfscope}%
\begin{pgfscope}%
\pgfpathrectangle{\pgfqpoint{0.100000in}{0.212622in}}{\pgfqpoint{3.696000in}{3.696000in}}%
\pgfusepath{clip}%
\pgfsetbuttcap%
\pgfsetroundjoin%
\definecolor{currentfill}{rgb}{0.121569,0.466667,0.705882}%
\pgfsetfillcolor{currentfill}%
\pgfsetfillopacity{0.623189}%
\pgfsetlinewidth{1.003750pt}%
\definecolor{currentstroke}{rgb}{0.121569,0.466667,0.705882}%
\pgfsetstrokecolor{currentstroke}%
\pgfsetstrokeopacity{0.623189}%
\pgfsetdash{}{0pt}%
\pgfpathmoveto{\pgfqpoint{0.893442in}{1.290263in}}%
\pgfpathcurveto{\pgfqpoint{0.901678in}{1.290263in}}{\pgfqpoint{0.909578in}{1.293535in}}{\pgfqpoint{0.915402in}{1.299359in}}%
\pgfpathcurveto{\pgfqpoint{0.921226in}{1.305183in}}{\pgfqpoint{0.924498in}{1.313083in}}{\pgfqpoint{0.924498in}{1.321319in}}%
\pgfpathcurveto{\pgfqpoint{0.924498in}{1.329556in}}{\pgfqpoint{0.921226in}{1.337456in}}{\pgfqpoint{0.915402in}{1.343280in}}%
\pgfpathcurveto{\pgfqpoint{0.909578in}{1.349104in}}{\pgfqpoint{0.901678in}{1.352376in}}{\pgfqpoint{0.893442in}{1.352376in}}%
\pgfpathcurveto{\pgfqpoint{0.885205in}{1.352376in}}{\pgfqpoint{0.877305in}{1.349104in}}{\pgfqpoint{0.871481in}{1.343280in}}%
\pgfpathcurveto{\pgfqpoint{0.865657in}{1.337456in}}{\pgfqpoint{0.862385in}{1.329556in}}{\pgfqpoint{0.862385in}{1.321319in}}%
\pgfpathcurveto{\pgfqpoint{0.862385in}{1.313083in}}{\pgfqpoint{0.865657in}{1.305183in}}{\pgfqpoint{0.871481in}{1.299359in}}%
\pgfpathcurveto{\pgfqpoint{0.877305in}{1.293535in}}{\pgfqpoint{0.885205in}{1.290263in}}{\pgfqpoint{0.893442in}{1.290263in}}%
\pgfpathclose%
\pgfusepath{stroke,fill}%
\end{pgfscope}%
\begin{pgfscope}%
\pgfpathrectangle{\pgfqpoint{0.100000in}{0.212622in}}{\pgfqpoint{3.696000in}{3.696000in}}%
\pgfusepath{clip}%
\pgfsetbuttcap%
\pgfsetroundjoin%
\definecolor{currentfill}{rgb}{0.121569,0.466667,0.705882}%
\pgfsetfillcolor{currentfill}%
\pgfsetfillopacity{0.623196}%
\pgfsetlinewidth{1.003750pt}%
\definecolor{currentstroke}{rgb}{0.121569,0.466667,0.705882}%
\pgfsetstrokecolor{currentstroke}%
\pgfsetstrokeopacity{0.623196}%
\pgfsetdash{}{0pt}%
\pgfpathmoveto{\pgfqpoint{0.893421in}{1.290252in}}%
\pgfpathcurveto{\pgfqpoint{0.901657in}{1.290252in}}{\pgfqpoint{0.909557in}{1.293524in}}{\pgfqpoint{0.915381in}{1.299348in}}%
\pgfpathcurveto{\pgfqpoint{0.921205in}{1.305172in}}{\pgfqpoint{0.924477in}{1.313072in}}{\pgfqpoint{0.924477in}{1.321308in}}%
\pgfpathcurveto{\pgfqpoint{0.924477in}{1.329544in}}{\pgfqpoint{0.921205in}{1.337444in}}{\pgfqpoint{0.915381in}{1.343268in}}%
\pgfpathcurveto{\pgfqpoint{0.909557in}{1.349092in}}{\pgfqpoint{0.901657in}{1.352365in}}{\pgfqpoint{0.893421in}{1.352365in}}%
\pgfpathcurveto{\pgfqpoint{0.885184in}{1.352365in}}{\pgfqpoint{0.877284in}{1.349092in}}{\pgfqpoint{0.871460in}{1.343268in}}%
\pgfpathcurveto{\pgfqpoint{0.865637in}{1.337444in}}{\pgfqpoint{0.862364in}{1.329544in}}{\pgfqpoint{0.862364in}{1.321308in}}%
\pgfpathcurveto{\pgfqpoint{0.862364in}{1.313072in}}{\pgfqpoint{0.865637in}{1.305172in}}{\pgfqpoint{0.871460in}{1.299348in}}%
\pgfpathcurveto{\pgfqpoint{0.877284in}{1.293524in}}{\pgfqpoint{0.885184in}{1.290252in}}{\pgfqpoint{0.893421in}{1.290252in}}%
\pgfpathclose%
\pgfusepath{stroke,fill}%
\end{pgfscope}%
\begin{pgfscope}%
\pgfpathrectangle{\pgfqpoint{0.100000in}{0.212622in}}{\pgfqpoint{3.696000in}{3.696000in}}%
\pgfusepath{clip}%
\pgfsetbuttcap%
\pgfsetroundjoin%
\definecolor{currentfill}{rgb}{0.121569,0.466667,0.705882}%
\pgfsetfillcolor{currentfill}%
\pgfsetfillopacity{0.623208}%
\pgfsetlinewidth{1.003750pt}%
\definecolor{currentstroke}{rgb}{0.121569,0.466667,0.705882}%
\pgfsetstrokecolor{currentstroke}%
\pgfsetstrokeopacity{0.623208}%
\pgfsetdash{}{0pt}%
\pgfpathmoveto{\pgfqpoint{0.893379in}{1.290234in}}%
\pgfpathcurveto{\pgfqpoint{0.901616in}{1.290234in}}{\pgfqpoint{0.909516in}{1.293506in}}{\pgfqpoint{0.915340in}{1.299330in}}%
\pgfpathcurveto{\pgfqpoint{0.921163in}{1.305154in}}{\pgfqpoint{0.924436in}{1.313054in}}{\pgfqpoint{0.924436in}{1.321290in}}%
\pgfpathcurveto{\pgfqpoint{0.924436in}{1.329526in}}{\pgfqpoint{0.921163in}{1.337426in}}{\pgfqpoint{0.915340in}{1.343250in}}%
\pgfpathcurveto{\pgfqpoint{0.909516in}{1.349074in}}{\pgfqpoint{0.901616in}{1.352347in}}{\pgfqpoint{0.893379in}{1.352347in}}%
\pgfpathcurveto{\pgfqpoint{0.885143in}{1.352347in}}{\pgfqpoint{0.877243in}{1.349074in}}{\pgfqpoint{0.871419in}{1.343250in}}%
\pgfpathcurveto{\pgfqpoint{0.865595in}{1.337426in}}{\pgfqpoint{0.862323in}{1.329526in}}{\pgfqpoint{0.862323in}{1.321290in}}%
\pgfpathcurveto{\pgfqpoint{0.862323in}{1.313054in}}{\pgfqpoint{0.865595in}{1.305154in}}{\pgfqpoint{0.871419in}{1.299330in}}%
\pgfpathcurveto{\pgfqpoint{0.877243in}{1.293506in}}{\pgfqpoint{0.885143in}{1.290234in}}{\pgfqpoint{0.893379in}{1.290234in}}%
\pgfpathclose%
\pgfusepath{stroke,fill}%
\end{pgfscope}%
\begin{pgfscope}%
\pgfpathrectangle{\pgfqpoint{0.100000in}{0.212622in}}{\pgfqpoint{3.696000in}{3.696000in}}%
\pgfusepath{clip}%
\pgfsetbuttcap%
\pgfsetroundjoin%
\definecolor{currentfill}{rgb}{0.121569,0.466667,0.705882}%
\pgfsetfillcolor{currentfill}%
\pgfsetfillopacity{0.623227}%
\pgfsetlinewidth{1.003750pt}%
\definecolor{currentstroke}{rgb}{0.121569,0.466667,0.705882}%
\pgfsetstrokecolor{currentstroke}%
\pgfsetstrokeopacity{0.623227}%
\pgfsetdash{}{0pt}%
\pgfpathmoveto{\pgfqpoint{0.893307in}{1.290172in}}%
\pgfpathcurveto{\pgfqpoint{0.901543in}{1.290172in}}{\pgfqpoint{0.909443in}{1.293445in}}{\pgfqpoint{0.915267in}{1.299269in}}%
\pgfpathcurveto{\pgfqpoint{0.921091in}{1.305092in}}{\pgfqpoint{0.924363in}{1.312993in}}{\pgfqpoint{0.924363in}{1.321229in}}%
\pgfpathcurveto{\pgfqpoint{0.924363in}{1.329465in}}{\pgfqpoint{0.921091in}{1.337365in}}{\pgfqpoint{0.915267in}{1.343189in}}%
\pgfpathcurveto{\pgfqpoint{0.909443in}{1.349013in}}{\pgfqpoint{0.901543in}{1.352285in}}{\pgfqpoint{0.893307in}{1.352285in}}%
\pgfpathcurveto{\pgfqpoint{0.885070in}{1.352285in}}{\pgfqpoint{0.877170in}{1.349013in}}{\pgfqpoint{0.871346in}{1.343189in}}%
\pgfpathcurveto{\pgfqpoint{0.865522in}{1.337365in}}{\pgfqpoint{0.862250in}{1.329465in}}{\pgfqpoint{0.862250in}{1.321229in}}%
\pgfpathcurveto{\pgfqpoint{0.862250in}{1.312993in}}{\pgfqpoint{0.865522in}{1.305092in}}{\pgfqpoint{0.871346in}{1.299269in}}%
\pgfpathcurveto{\pgfqpoint{0.877170in}{1.293445in}}{\pgfqpoint{0.885070in}{1.290172in}}{\pgfqpoint{0.893307in}{1.290172in}}%
\pgfpathclose%
\pgfusepath{stroke,fill}%
\end{pgfscope}%
\begin{pgfscope}%
\pgfpathrectangle{\pgfqpoint{0.100000in}{0.212622in}}{\pgfqpoint{3.696000in}{3.696000in}}%
\pgfusepath{clip}%
\pgfsetbuttcap%
\pgfsetroundjoin%
\definecolor{currentfill}{rgb}{0.121569,0.466667,0.705882}%
\pgfsetfillcolor{currentfill}%
\pgfsetfillopacity{0.623261}%
\pgfsetlinewidth{1.003750pt}%
\definecolor{currentstroke}{rgb}{0.121569,0.466667,0.705882}%
\pgfsetstrokecolor{currentstroke}%
\pgfsetstrokeopacity{0.623261}%
\pgfsetdash{}{0pt}%
\pgfpathmoveto{\pgfqpoint{0.911548in}{1.255141in}}%
\pgfpathcurveto{\pgfqpoint{0.919785in}{1.255141in}}{\pgfqpoint{0.927685in}{1.258413in}}{\pgfqpoint{0.933509in}{1.264237in}}%
\pgfpathcurveto{\pgfqpoint{0.939333in}{1.270061in}}{\pgfqpoint{0.942605in}{1.277961in}}{\pgfqpoint{0.942605in}{1.286197in}}%
\pgfpathcurveto{\pgfqpoint{0.942605in}{1.294434in}}{\pgfqpoint{0.939333in}{1.302334in}}{\pgfqpoint{0.933509in}{1.308158in}}%
\pgfpathcurveto{\pgfqpoint{0.927685in}{1.313982in}}{\pgfqpoint{0.919785in}{1.317254in}}{\pgfqpoint{0.911548in}{1.317254in}}%
\pgfpathcurveto{\pgfqpoint{0.903312in}{1.317254in}}{\pgfqpoint{0.895412in}{1.313982in}}{\pgfqpoint{0.889588in}{1.308158in}}%
\pgfpathcurveto{\pgfqpoint{0.883764in}{1.302334in}}{\pgfqpoint{0.880492in}{1.294434in}}{\pgfqpoint{0.880492in}{1.286197in}}%
\pgfpathcurveto{\pgfqpoint{0.880492in}{1.277961in}}{\pgfqpoint{0.883764in}{1.270061in}}{\pgfqpoint{0.889588in}{1.264237in}}%
\pgfpathcurveto{\pgfqpoint{0.895412in}{1.258413in}}{\pgfqpoint{0.903312in}{1.255141in}}{\pgfqpoint{0.911548in}{1.255141in}}%
\pgfpathclose%
\pgfusepath{stroke,fill}%
\end{pgfscope}%
\begin{pgfscope}%
\pgfpathrectangle{\pgfqpoint{0.100000in}{0.212622in}}{\pgfqpoint{3.696000in}{3.696000in}}%
\pgfusepath{clip}%
\pgfsetbuttcap%
\pgfsetroundjoin%
\definecolor{currentfill}{rgb}{0.121569,0.466667,0.705882}%
\pgfsetfillcolor{currentfill}%
\pgfsetfillopacity{0.623269}%
\pgfsetlinewidth{1.003750pt}%
\definecolor{currentstroke}{rgb}{0.121569,0.466667,0.705882}%
\pgfsetstrokecolor{currentstroke}%
\pgfsetstrokeopacity{0.623269}%
\pgfsetdash{}{0pt}%
\pgfpathmoveto{\pgfqpoint{0.893181in}{1.290095in}}%
\pgfpathcurveto{\pgfqpoint{0.901417in}{1.290095in}}{\pgfqpoint{0.909318in}{1.293367in}}{\pgfqpoint{0.915141in}{1.299191in}}%
\pgfpathcurveto{\pgfqpoint{0.920965in}{1.305015in}}{\pgfqpoint{0.924238in}{1.312915in}}{\pgfqpoint{0.924238in}{1.321152in}}%
\pgfpathcurveto{\pgfqpoint{0.924238in}{1.329388in}}{\pgfqpoint{0.920965in}{1.337288in}}{\pgfqpoint{0.915141in}{1.343112in}}%
\pgfpathcurveto{\pgfqpoint{0.909318in}{1.348936in}}{\pgfqpoint{0.901417in}{1.352208in}}{\pgfqpoint{0.893181in}{1.352208in}}%
\pgfpathcurveto{\pgfqpoint{0.884945in}{1.352208in}}{\pgfqpoint{0.877045in}{1.348936in}}{\pgfqpoint{0.871221in}{1.343112in}}%
\pgfpathcurveto{\pgfqpoint{0.865397in}{1.337288in}}{\pgfqpoint{0.862125in}{1.329388in}}{\pgfqpoint{0.862125in}{1.321152in}}%
\pgfpathcurveto{\pgfqpoint{0.862125in}{1.312915in}}{\pgfqpoint{0.865397in}{1.305015in}}{\pgfqpoint{0.871221in}{1.299191in}}%
\pgfpathcurveto{\pgfqpoint{0.877045in}{1.293367in}}{\pgfqpoint{0.884945in}{1.290095in}}{\pgfqpoint{0.893181in}{1.290095in}}%
\pgfpathclose%
\pgfusepath{stroke,fill}%
\end{pgfscope}%
\begin{pgfscope}%
\pgfpathrectangle{\pgfqpoint{0.100000in}{0.212622in}}{\pgfqpoint{3.696000in}{3.696000in}}%
\pgfusepath{clip}%
\pgfsetbuttcap%
\pgfsetroundjoin%
\definecolor{currentfill}{rgb}{0.121569,0.466667,0.705882}%
\pgfsetfillcolor{currentfill}%
\pgfsetfillopacity{0.623340}%
\pgfsetlinewidth{1.003750pt}%
\definecolor{currentstroke}{rgb}{0.121569,0.466667,0.705882}%
\pgfsetstrokecolor{currentstroke}%
\pgfsetstrokeopacity{0.623340}%
\pgfsetdash{}{0pt}%
\pgfpathmoveto{\pgfqpoint{0.892947in}{1.289933in}}%
\pgfpathcurveto{\pgfqpoint{0.901183in}{1.289933in}}{\pgfqpoint{0.909083in}{1.293205in}}{\pgfqpoint{0.914907in}{1.299029in}}%
\pgfpathcurveto{\pgfqpoint{0.920731in}{1.304853in}}{\pgfqpoint{0.924003in}{1.312753in}}{\pgfqpoint{0.924003in}{1.320989in}}%
\pgfpathcurveto{\pgfqpoint{0.924003in}{1.329226in}}{\pgfqpoint{0.920731in}{1.337126in}}{\pgfqpoint{0.914907in}{1.342950in}}%
\pgfpathcurveto{\pgfqpoint{0.909083in}{1.348773in}}{\pgfqpoint{0.901183in}{1.352046in}}{\pgfqpoint{0.892947in}{1.352046in}}%
\pgfpathcurveto{\pgfqpoint{0.884710in}{1.352046in}}{\pgfqpoint{0.876810in}{1.348773in}}{\pgfqpoint{0.870986in}{1.342950in}}%
\pgfpathcurveto{\pgfqpoint{0.865162in}{1.337126in}}{\pgfqpoint{0.861890in}{1.329226in}}{\pgfqpoint{0.861890in}{1.320989in}}%
\pgfpathcurveto{\pgfqpoint{0.861890in}{1.312753in}}{\pgfqpoint{0.865162in}{1.304853in}}{\pgfqpoint{0.870986in}{1.299029in}}%
\pgfpathcurveto{\pgfqpoint{0.876810in}{1.293205in}}{\pgfqpoint{0.884710in}{1.289933in}}{\pgfqpoint{0.892947in}{1.289933in}}%
\pgfpathclose%
\pgfusepath{stroke,fill}%
\end{pgfscope}%
\begin{pgfscope}%
\pgfpathrectangle{\pgfqpoint{0.100000in}{0.212622in}}{\pgfqpoint{3.696000in}{3.696000in}}%
\pgfusepath{clip}%
\pgfsetbuttcap%
\pgfsetroundjoin%
\definecolor{currentfill}{rgb}{0.121569,0.466667,0.705882}%
\pgfsetfillcolor{currentfill}%
\pgfsetfillopacity{0.623479}%
\pgfsetlinewidth{1.003750pt}%
\definecolor{currentstroke}{rgb}{0.121569,0.466667,0.705882}%
\pgfsetstrokecolor{currentstroke}%
\pgfsetstrokeopacity{0.623479}%
\pgfsetdash{}{0pt}%
\pgfpathmoveto{\pgfqpoint{0.892543in}{1.289670in}}%
\pgfpathcurveto{\pgfqpoint{0.900780in}{1.289670in}}{\pgfqpoint{0.908680in}{1.292942in}}{\pgfqpoint{0.914504in}{1.298766in}}%
\pgfpathcurveto{\pgfqpoint{0.920327in}{1.304590in}}{\pgfqpoint{0.923600in}{1.312490in}}{\pgfqpoint{0.923600in}{1.320726in}}%
\pgfpathcurveto{\pgfqpoint{0.923600in}{1.328963in}}{\pgfqpoint{0.920327in}{1.336863in}}{\pgfqpoint{0.914504in}{1.342687in}}%
\pgfpathcurveto{\pgfqpoint{0.908680in}{1.348510in}}{\pgfqpoint{0.900780in}{1.351783in}}{\pgfqpoint{0.892543in}{1.351783in}}%
\pgfpathcurveto{\pgfqpoint{0.884307in}{1.351783in}}{\pgfqpoint{0.876407in}{1.348510in}}{\pgfqpoint{0.870583in}{1.342687in}}%
\pgfpathcurveto{\pgfqpoint{0.864759in}{1.336863in}}{\pgfqpoint{0.861487in}{1.328963in}}{\pgfqpoint{0.861487in}{1.320726in}}%
\pgfpathcurveto{\pgfqpoint{0.861487in}{1.312490in}}{\pgfqpoint{0.864759in}{1.304590in}}{\pgfqpoint{0.870583in}{1.298766in}}%
\pgfpathcurveto{\pgfqpoint{0.876407in}{1.292942in}}{\pgfqpoint{0.884307in}{1.289670in}}{\pgfqpoint{0.892543in}{1.289670in}}%
\pgfpathclose%
\pgfusepath{stroke,fill}%
\end{pgfscope}%
\begin{pgfscope}%
\pgfpathrectangle{\pgfqpoint{0.100000in}{0.212622in}}{\pgfqpoint{3.696000in}{3.696000in}}%
\pgfusepath{clip}%
\pgfsetbuttcap%
\pgfsetroundjoin%
\definecolor{currentfill}{rgb}{0.121569,0.466667,0.705882}%
\pgfsetfillcolor{currentfill}%
\pgfsetfillopacity{0.623710}%
\pgfsetlinewidth{1.003750pt}%
\definecolor{currentstroke}{rgb}{0.121569,0.466667,0.705882}%
\pgfsetstrokecolor{currentstroke}%
\pgfsetstrokeopacity{0.623710}%
\pgfsetdash{}{0pt}%
\pgfpathmoveto{\pgfqpoint{0.891797in}{1.289073in}}%
\pgfpathcurveto{\pgfqpoint{0.900033in}{1.289073in}}{\pgfqpoint{0.907933in}{1.292346in}}{\pgfqpoint{0.913757in}{1.298169in}}%
\pgfpathcurveto{\pgfqpoint{0.919581in}{1.303993in}}{\pgfqpoint{0.922853in}{1.311893in}}{\pgfqpoint{0.922853in}{1.320130in}}%
\pgfpathcurveto{\pgfqpoint{0.922853in}{1.328366in}}{\pgfqpoint{0.919581in}{1.336266in}}{\pgfqpoint{0.913757in}{1.342090in}}%
\pgfpathcurveto{\pgfqpoint{0.907933in}{1.347914in}}{\pgfqpoint{0.900033in}{1.351186in}}{\pgfqpoint{0.891797in}{1.351186in}}%
\pgfpathcurveto{\pgfqpoint{0.883560in}{1.351186in}}{\pgfqpoint{0.875660in}{1.347914in}}{\pgfqpoint{0.869836in}{1.342090in}}%
\pgfpathcurveto{\pgfqpoint{0.864012in}{1.336266in}}{\pgfqpoint{0.860740in}{1.328366in}}{\pgfqpoint{0.860740in}{1.320130in}}%
\pgfpathcurveto{\pgfqpoint{0.860740in}{1.311893in}}{\pgfqpoint{0.864012in}{1.303993in}}{\pgfqpoint{0.869836in}{1.298169in}}%
\pgfpathcurveto{\pgfqpoint{0.875660in}{1.292346in}}{\pgfqpoint{0.883560in}{1.289073in}}{\pgfqpoint{0.891797in}{1.289073in}}%
\pgfpathclose%
\pgfusepath{stroke,fill}%
\end{pgfscope}%
\begin{pgfscope}%
\pgfpathrectangle{\pgfqpoint{0.100000in}{0.212622in}}{\pgfqpoint{3.696000in}{3.696000in}}%
\pgfusepath{clip}%
\pgfsetbuttcap%
\pgfsetroundjoin%
\definecolor{currentfill}{rgb}{0.121569,0.466667,0.705882}%
\pgfsetfillcolor{currentfill}%
\pgfsetfillopacity{0.624155}%
\pgfsetlinewidth{1.003750pt}%
\definecolor{currentstroke}{rgb}{0.121569,0.466667,0.705882}%
\pgfsetstrokecolor{currentstroke}%
\pgfsetstrokeopacity{0.624155}%
\pgfsetdash{}{0pt}%
\pgfpathmoveto{\pgfqpoint{0.890353in}{1.288267in}}%
\pgfpathcurveto{\pgfqpoint{0.898589in}{1.288267in}}{\pgfqpoint{0.906489in}{1.291539in}}{\pgfqpoint{0.912313in}{1.297363in}}%
\pgfpathcurveto{\pgfqpoint{0.918137in}{1.303187in}}{\pgfqpoint{0.921409in}{1.311087in}}{\pgfqpoint{0.921409in}{1.319324in}}%
\pgfpathcurveto{\pgfqpoint{0.921409in}{1.327560in}}{\pgfqpoint{0.918137in}{1.335460in}}{\pgfqpoint{0.912313in}{1.341284in}}%
\pgfpathcurveto{\pgfqpoint{0.906489in}{1.347108in}}{\pgfqpoint{0.898589in}{1.350380in}}{\pgfqpoint{0.890353in}{1.350380in}}%
\pgfpathcurveto{\pgfqpoint{0.882117in}{1.350380in}}{\pgfqpoint{0.874217in}{1.347108in}}{\pgfqpoint{0.868393in}{1.341284in}}%
\pgfpathcurveto{\pgfqpoint{0.862569in}{1.335460in}}{\pgfqpoint{0.859296in}{1.327560in}}{\pgfqpoint{0.859296in}{1.319324in}}%
\pgfpathcurveto{\pgfqpoint{0.859296in}{1.311087in}}{\pgfqpoint{0.862569in}{1.303187in}}{\pgfqpoint{0.868393in}{1.297363in}}%
\pgfpathcurveto{\pgfqpoint{0.874217in}{1.291539in}}{\pgfqpoint{0.882117in}{1.288267in}}{\pgfqpoint{0.890353in}{1.288267in}}%
\pgfpathclose%
\pgfusepath{stroke,fill}%
\end{pgfscope}%
\begin{pgfscope}%
\pgfpathrectangle{\pgfqpoint{0.100000in}{0.212622in}}{\pgfqpoint{3.696000in}{3.696000in}}%
\pgfusepath{clip}%
\pgfsetbuttcap%
\pgfsetroundjoin%
\definecolor{currentfill}{rgb}{0.121569,0.466667,0.705882}%
\pgfsetfillcolor{currentfill}%
\pgfsetfillopacity{0.624436}%
\pgfsetlinewidth{1.003750pt}%
\definecolor{currentstroke}{rgb}{0.121569,0.466667,0.705882}%
\pgfsetstrokecolor{currentstroke}%
\pgfsetstrokeopacity{0.624436}%
\pgfsetdash{}{0pt}%
\pgfpathmoveto{\pgfqpoint{0.909677in}{1.255886in}}%
\pgfpathcurveto{\pgfqpoint{0.917913in}{1.255886in}}{\pgfqpoint{0.925813in}{1.259158in}}{\pgfqpoint{0.931637in}{1.264982in}}%
\pgfpathcurveto{\pgfqpoint{0.937461in}{1.270806in}}{\pgfqpoint{0.940733in}{1.278706in}}{\pgfqpoint{0.940733in}{1.286942in}}%
\pgfpathcurveto{\pgfqpoint{0.940733in}{1.295178in}}{\pgfqpoint{0.937461in}{1.303078in}}{\pgfqpoint{0.931637in}{1.308902in}}%
\pgfpathcurveto{\pgfqpoint{0.925813in}{1.314726in}}{\pgfqpoint{0.917913in}{1.317999in}}{\pgfqpoint{0.909677in}{1.317999in}}%
\pgfpathcurveto{\pgfqpoint{0.901441in}{1.317999in}}{\pgfqpoint{0.893541in}{1.314726in}}{\pgfqpoint{0.887717in}{1.308902in}}%
\pgfpathcurveto{\pgfqpoint{0.881893in}{1.303078in}}{\pgfqpoint{0.878620in}{1.295178in}}{\pgfqpoint{0.878620in}{1.286942in}}%
\pgfpathcurveto{\pgfqpoint{0.878620in}{1.278706in}}{\pgfqpoint{0.881893in}{1.270806in}}{\pgfqpoint{0.887717in}{1.264982in}}%
\pgfpathcurveto{\pgfqpoint{0.893541in}{1.259158in}}{\pgfqpoint{0.901441in}{1.255886in}}{\pgfqpoint{0.909677in}{1.255886in}}%
\pgfpathclose%
\pgfusepath{stroke,fill}%
\end{pgfscope}%
\begin{pgfscope}%
\pgfpathrectangle{\pgfqpoint{0.100000in}{0.212622in}}{\pgfqpoint{3.696000in}{3.696000in}}%
\pgfusepath{clip}%
\pgfsetbuttcap%
\pgfsetroundjoin%
\definecolor{currentfill}{rgb}{0.121569,0.466667,0.705882}%
\pgfsetfillcolor{currentfill}%
\pgfsetfillopacity{0.625023}%
\pgfsetlinewidth{1.003750pt}%
\definecolor{currentstroke}{rgb}{0.121569,0.466667,0.705882}%
\pgfsetstrokecolor{currentstroke}%
\pgfsetstrokeopacity{0.625023}%
\pgfsetdash{}{0pt}%
\pgfpathmoveto{\pgfqpoint{0.887698in}{1.287210in}}%
\pgfpathcurveto{\pgfqpoint{0.895934in}{1.287210in}}{\pgfqpoint{0.903834in}{1.290482in}}{\pgfqpoint{0.909658in}{1.296306in}}%
\pgfpathcurveto{\pgfqpoint{0.915482in}{1.302130in}}{\pgfqpoint{0.918755in}{1.310030in}}{\pgfqpoint{0.918755in}{1.318266in}}%
\pgfpathcurveto{\pgfqpoint{0.918755in}{1.326503in}}{\pgfqpoint{0.915482in}{1.334403in}}{\pgfqpoint{0.909658in}{1.340227in}}%
\pgfpathcurveto{\pgfqpoint{0.903834in}{1.346051in}}{\pgfqpoint{0.895934in}{1.349323in}}{\pgfqpoint{0.887698in}{1.349323in}}%
\pgfpathcurveto{\pgfqpoint{0.879462in}{1.349323in}}{\pgfqpoint{0.871562in}{1.346051in}}{\pgfqpoint{0.865738in}{1.340227in}}%
\pgfpathcurveto{\pgfqpoint{0.859914in}{1.334403in}}{\pgfqpoint{0.856642in}{1.326503in}}{\pgfqpoint{0.856642in}{1.318266in}}%
\pgfpathcurveto{\pgfqpoint{0.856642in}{1.310030in}}{\pgfqpoint{0.859914in}{1.302130in}}{\pgfqpoint{0.865738in}{1.296306in}}%
\pgfpathcurveto{\pgfqpoint{0.871562in}{1.290482in}}{\pgfqpoint{0.879462in}{1.287210in}}{\pgfqpoint{0.887698in}{1.287210in}}%
\pgfpathclose%
\pgfusepath{stroke,fill}%
\end{pgfscope}%
\begin{pgfscope}%
\pgfpathrectangle{\pgfqpoint{0.100000in}{0.212622in}}{\pgfqpoint{3.696000in}{3.696000in}}%
\pgfusepath{clip}%
\pgfsetbuttcap%
\pgfsetroundjoin%
\definecolor{currentfill}{rgb}{0.121569,0.466667,0.705882}%
\pgfsetfillcolor{currentfill}%
\pgfsetfillopacity{0.626006}%
\pgfsetlinewidth{1.003750pt}%
\definecolor{currentstroke}{rgb}{0.121569,0.466667,0.705882}%
\pgfsetstrokecolor{currentstroke}%
\pgfsetstrokeopacity{0.626006}%
\pgfsetdash{}{0pt}%
\pgfpathmoveto{\pgfqpoint{0.906892in}{1.257017in}}%
\pgfpathcurveto{\pgfqpoint{0.915128in}{1.257017in}}{\pgfqpoint{0.923028in}{1.260290in}}{\pgfqpoint{0.928852in}{1.266114in}}%
\pgfpathcurveto{\pgfqpoint{0.934676in}{1.271937in}}{\pgfqpoint{0.937948in}{1.279837in}}{\pgfqpoint{0.937948in}{1.288074in}}%
\pgfpathcurveto{\pgfqpoint{0.937948in}{1.296310in}}{\pgfqpoint{0.934676in}{1.304210in}}{\pgfqpoint{0.928852in}{1.310034in}}%
\pgfpathcurveto{\pgfqpoint{0.923028in}{1.315858in}}{\pgfqpoint{0.915128in}{1.319130in}}{\pgfqpoint{0.906892in}{1.319130in}}%
\pgfpathcurveto{\pgfqpoint{0.898656in}{1.319130in}}{\pgfqpoint{0.890756in}{1.315858in}}{\pgfqpoint{0.884932in}{1.310034in}}%
\pgfpathcurveto{\pgfqpoint{0.879108in}{1.304210in}}{\pgfqpoint{0.875835in}{1.296310in}}{\pgfqpoint{0.875835in}{1.288074in}}%
\pgfpathcurveto{\pgfqpoint{0.875835in}{1.279837in}}{\pgfqpoint{0.879108in}{1.271937in}}{\pgfqpoint{0.884932in}{1.266114in}}%
\pgfpathcurveto{\pgfqpoint{0.890756in}{1.260290in}}{\pgfqpoint{0.898656in}{1.257017in}}{\pgfqpoint{0.906892in}{1.257017in}}%
\pgfpathclose%
\pgfusepath{stroke,fill}%
\end{pgfscope}%
\begin{pgfscope}%
\pgfpathrectangle{\pgfqpoint{0.100000in}{0.212622in}}{\pgfqpoint{3.696000in}{3.696000in}}%
\pgfusepath{clip}%
\pgfsetbuttcap%
\pgfsetroundjoin%
\definecolor{currentfill}{rgb}{0.121569,0.466667,0.705882}%
\pgfsetfillcolor{currentfill}%
\pgfsetfillopacity{0.626273}%
\pgfsetlinewidth{1.003750pt}%
\definecolor{currentstroke}{rgb}{0.121569,0.466667,0.705882}%
\pgfsetstrokecolor{currentstroke}%
\pgfsetstrokeopacity{0.626273}%
\pgfsetdash{}{0pt}%
\pgfpathmoveto{\pgfqpoint{0.883044in}{1.283018in}}%
\pgfpathcurveto{\pgfqpoint{0.891280in}{1.283018in}}{\pgfqpoint{0.899180in}{1.286290in}}{\pgfqpoint{0.905004in}{1.292114in}}%
\pgfpathcurveto{\pgfqpoint{0.910828in}{1.297938in}}{\pgfqpoint{0.914100in}{1.305838in}}{\pgfqpoint{0.914100in}{1.314074in}}%
\pgfpathcurveto{\pgfqpoint{0.914100in}{1.322311in}}{\pgfqpoint{0.910828in}{1.330211in}}{\pgfqpoint{0.905004in}{1.336035in}}%
\pgfpathcurveto{\pgfqpoint{0.899180in}{1.341859in}}{\pgfqpoint{0.891280in}{1.345131in}}{\pgfqpoint{0.883044in}{1.345131in}}%
\pgfpathcurveto{\pgfqpoint{0.874807in}{1.345131in}}{\pgfqpoint{0.866907in}{1.341859in}}{\pgfqpoint{0.861083in}{1.336035in}}%
\pgfpathcurveto{\pgfqpoint{0.855260in}{1.330211in}}{\pgfqpoint{0.851987in}{1.322311in}}{\pgfqpoint{0.851987in}{1.314074in}}%
\pgfpathcurveto{\pgfqpoint{0.851987in}{1.305838in}}{\pgfqpoint{0.855260in}{1.297938in}}{\pgfqpoint{0.861083in}{1.292114in}}%
\pgfpathcurveto{\pgfqpoint{0.866907in}{1.286290in}}{\pgfqpoint{0.874807in}{1.283018in}}{\pgfqpoint{0.883044in}{1.283018in}}%
\pgfpathclose%
\pgfusepath{stroke,fill}%
\end{pgfscope}%
\begin{pgfscope}%
\pgfpathrectangle{\pgfqpoint{0.100000in}{0.212622in}}{\pgfqpoint{3.696000in}{3.696000in}}%
\pgfusepath{clip}%
\pgfsetbuttcap%
\pgfsetroundjoin%
\definecolor{currentfill}{rgb}{0.121569,0.466667,0.705882}%
\pgfsetfillcolor{currentfill}%
\pgfsetfillopacity{0.627595}%
\pgfsetlinewidth{1.003750pt}%
\definecolor{currentstroke}{rgb}{0.121569,0.466667,0.705882}%
\pgfsetstrokecolor{currentstroke}%
\pgfsetstrokeopacity{0.627595}%
\pgfsetdash{}{0pt}%
\pgfpathmoveto{\pgfqpoint{0.878826in}{1.281282in}}%
\pgfpathcurveto{\pgfqpoint{0.887063in}{1.281282in}}{\pgfqpoint{0.894963in}{1.284554in}}{\pgfqpoint{0.900787in}{1.290378in}}%
\pgfpathcurveto{\pgfqpoint{0.906610in}{1.296202in}}{\pgfqpoint{0.909883in}{1.304102in}}{\pgfqpoint{0.909883in}{1.312338in}}%
\pgfpathcurveto{\pgfqpoint{0.909883in}{1.320574in}}{\pgfqpoint{0.906610in}{1.328475in}}{\pgfqpoint{0.900787in}{1.334298in}}%
\pgfpathcurveto{\pgfqpoint{0.894963in}{1.340122in}}{\pgfqpoint{0.887063in}{1.343395in}}{\pgfqpoint{0.878826in}{1.343395in}}%
\pgfpathcurveto{\pgfqpoint{0.870590in}{1.343395in}}{\pgfqpoint{0.862690in}{1.340122in}}{\pgfqpoint{0.856866in}{1.334298in}}%
\pgfpathcurveto{\pgfqpoint{0.851042in}{1.328475in}}{\pgfqpoint{0.847770in}{1.320574in}}{\pgfqpoint{0.847770in}{1.312338in}}%
\pgfpathcurveto{\pgfqpoint{0.847770in}{1.304102in}}{\pgfqpoint{0.851042in}{1.296202in}}{\pgfqpoint{0.856866in}{1.290378in}}%
\pgfpathcurveto{\pgfqpoint{0.862690in}{1.284554in}}{\pgfqpoint{0.870590in}{1.281282in}}{\pgfqpoint{0.878826in}{1.281282in}}%
\pgfpathclose%
\pgfusepath{stroke,fill}%
\end{pgfscope}%
\begin{pgfscope}%
\pgfpathrectangle{\pgfqpoint{0.100000in}{0.212622in}}{\pgfqpoint{3.696000in}{3.696000in}}%
\pgfusepath{clip}%
\pgfsetbuttcap%
\pgfsetroundjoin%
\definecolor{currentfill}{rgb}{0.121569,0.466667,0.705882}%
\pgfsetfillcolor{currentfill}%
\pgfsetfillopacity{0.627638}%
\pgfsetlinewidth{1.003750pt}%
\definecolor{currentstroke}{rgb}{0.121569,0.466667,0.705882}%
\pgfsetstrokecolor{currentstroke}%
\pgfsetstrokeopacity{0.627638}%
\pgfsetdash{}{0pt}%
\pgfpathmoveto{\pgfqpoint{2.125770in}{1.796527in}}%
\pgfpathcurveto{\pgfqpoint{2.134006in}{1.796527in}}{\pgfqpoint{2.141906in}{1.799799in}}{\pgfqpoint{2.147730in}{1.805623in}}%
\pgfpathcurveto{\pgfqpoint{2.153554in}{1.811447in}}{\pgfqpoint{2.156826in}{1.819347in}}{\pgfqpoint{2.156826in}{1.827584in}}%
\pgfpathcurveto{\pgfqpoint{2.156826in}{1.835820in}}{\pgfqpoint{2.153554in}{1.843720in}}{\pgfqpoint{2.147730in}{1.849544in}}%
\pgfpathcurveto{\pgfqpoint{2.141906in}{1.855368in}}{\pgfqpoint{2.134006in}{1.858640in}}{\pgfqpoint{2.125770in}{1.858640in}}%
\pgfpathcurveto{\pgfqpoint{2.117534in}{1.858640in}}{\pgfqpoint{2.109634in}{1.855368in}}{\pgfqpoint{2.103810in}{1.849544in}}%
\pgfpathcurveto{\pgfqpoint{2.097986in}{1.843720in}}{\pgfqpoint{2.094713in}{1.835820in}}{\pgfqpoint{2.094713in}{1.827584in}}%
\pgfpathcurveto{\pgfqpoint{2.094713in}{1.819347in}}{\pgfqpoint{2.097986in}{1.811447in}}{\pgfqpoint{2.103810in}{1.805623in}}%
\pgfpathcurveto{\pgfqpoint{2.109634in}{1.799799in}}{\pgfqpoint{2.117534in}{1.796527in}}{\pgfqpoint{2.125770in}{1.796527in}}%
\pgfpathclose%
\pgfusepath{stroke,fill}%
\end{pgfscope}%
\begin{pgfscope}%
\pgfpathrectangle{\pgfqpoint{0.100000in}{0.212622in}}{\pgfqpoint{3.696000in}{3.696000in}}%
\pgfusepath{clip}%
\pgfsetbuttcap%
\pgfsetroundjoin%
\definecolor{currentfill}{rgb}{0.121569,0.466667,0.705882}%
\pgfsetfillcolor{currentfill}%
\pgfsetfillopacity{0.627920}%
\pgfsetlinewidth{1.003750pt}%
\definecolor{currentstroke}{rgb}{0.121569,0.466667,0.705882}%
\pgfsetstrokecolor{currentstroke}%
\pgfsetstrokeopacity{0.627920}%
\pgfsetdash{}{0pt}%
\pgfpathmoveto{\pgfqpoint{0.903610in}{1.258392in}}%
\pgfpathcurveto{\pgfqpoint{0.911846in}{1.258392in}}{\pgfqpoint{0.919746in}{1.261664in}}{\pgfqpoint{0.925570in}{1.267488in}}%
\pgfpathcurveto{\pgfqpoint{0.931394in}{1.273312in}}{\pgfqpoint{0.934666in}{1.281212in}}{\pgfqpoint{0.934666in}{1.289448in}}%
\pgfpathcurveto{\pgfqpoint{0.934666in}{1.297685in}}{\pgfqpoint{0.931394in}{1.305585in}}{\pgfqpoint{0.925570in}{1.311409in}}%
\pgfpathcurveto{\pgfqpoint{0.919746in}{1.317233in}}{\pgfqpoint{0.911846in}{1.320505in}}{\pgfqpoint{0.903610in}{1.320505in}}%
\pgfpathcurveto{\pgfqpoint{0.895374in}{1.320505in}}{\pgfqpoint{0.887474in}{1.317233in}}{\pgfqpoint{0.881650in}{1.311409in}}%
\pgfpathcurveto{\pgfqpoint{0.875826in}{1.305585in}}{\pgfqpoint{0.872553in}{1.297685in}}{\pgfqpoint{0.872553in}{1.289448in}}%
\pgfpathcurveto{\pgfqpoint{0.872553in}{1.281212in}}{\pgfqpoint{0.875826in}{1.273312in}}{\pgfqpoint{0.881650in}{1.267488in}}%
\pgfpathcurveto{\pgfqpoint{0.887474in}{1.261664in}}{\pgfqpoint{0.895374in}{1.258392in}}{\pgfqpoint{0.903610in}{1.258392in}}%
\pgfpathclose%
\pgfusepath{stroke,fill}%
\end{pgfscope}%
\begin{pgfscope}%
\pgfpathrectangle{\pgfqpoint{0.100000in}{0.212622in}}{\pgfqpoint{3.696000in}{3.696000in}}%
\pgfusepath{clip}%
\pgfsetbuttcap%
\pgfsetroundjoin%
\definecolor{currentfill}{rgb}{0.121569,0.466667,0.705882}%
\pgfsetfillcolor{currentfill}%
\pgfsetfillopacity{0.628739}%
\pgfsetlinewidth{1.003750pt}%
\definecolor{currentstroke}{rgb}{0.121569,0.466667,0.705882}%
\pgfsetstrokecolor{currentstroke}%
\pgfsetstrokeopacity{0.628739}%
\pgfsetdash{}{0pt}%
\pgfpathmoveto{\pgfqpoint{0.875418in}{1.279945in}}%
\pgfpathcurveto{\pgfqpoint{0.883654in}{1.279945in}}{\pgfqpoint{0.891554in}{1.283217in}}{\pgfqpoint{0.897378in}{1.289041in}}%
\pgfpathcurveto{\pgfqpoint{0.903202in}{1.294865in}}{\pgfqpoint{0.906474in}{1.302765in}}{\pgfqpoint{0.906474in}{1.311002in}}%
\pgfpathcurveto{\pgfqpoint{0.906474in}{1.319238in}}{\pgfqpoint{0.903202in}{1.327138in}}{\pgfqpoint{0.897378in}{1.332962in}}%
\pgfpathcurveto{\pgfqpoint{0.891554in}{1.338786in}}{\pgfqpoint{0.883654in}{1.342058in}}{\pgfqpoint{0.875418in}{1.342058in}}%
\pgfpathcurveto{\pgfqpoint{0.867182in}{1.342058in}}{\pgfqpoint{0.859282in}{1.338786in}}{\pgfqpoint{0.853458in}{1.332962in}}%
\pgfpathcurveto{\pgfqpoint{0.847634in}{1.327138in}}{\pgfqpoint{0.844361in}{1.319238in}}{\pgfqpoint{0.844361in}{1.311002in}}%
\pgfpathcurveto{\pgfqpoint{0.844361in}{1.302765in}}{\pgfqpoint{0.847634in}{1.294865in}}{\pgfqpoint{0.853458in}{1.289041in}}%
\pgfpathcurveto{\pgfqpoint{0.859282in}{1.283217in}}{\pgfqpoint{0.867182in}{1.279945in}}{\pgfqpoint{0.875418in}{1.279945in}}%
\pgfpathclose%
\pgfusepath{stroke,fill}%
\end{pgfscope}%
\begin{pgfscope}%
\pgfpathrectangle{\pgfqpoint{0.100000in}{0.212622in}}{\pgfqpoint{3.696000in}{3.696000in}}%
\pgfusepath{clip}%
\pgfsetbuttcap%
\pgfsetroundjoin%
\definecolor{currentfill}{rgb}{0.121569,0.466667,0.705882}%
\pgfsetfillcolor{currentfill}%
\pgfsetfillopacity{0.629732}%
\pgfsetlinewidth{1.003750pt}%
\definecolor{currentstroke}{rgb}{0.121569,0.466667,0.705882}%
\pgfsetstrokecolor{currentstroke}%
\pgfsetstrokeopacity{0.629732}%
\pgfsetdash{}{0pt}%
\pgfpathmoveto{\pgfqpoint{0.872755in}{1.278842in}}%
\pgfpathcurveto{\pgfqpoint{0.880992in}{1.278842in}}{\pgfqpoint{0.888892in}{1.282115in}}{\pgfqpoint{0.894716in}{1.287938in}}%
\pgfpathcurveto{\pgfqpoint{0.900540in}{1.293762in}}{\pgfqpoint{0.903812in}{1.301662in}}{\pgfqpoint{0.903812in}{1.309899in}}%
\pgfpathcurveto{\pgfqpoint{0.903812in}{1.318135in}}{\pgfqpoint{0.900540in}{1.326035in}}{\pgfqpoint{0.894716in}{1.331859in}}%
\pgfpathcurveto{\pgfqpoint{0.888892in}{1.337683in}}{\pgfqpoint{0.880992in}{1.340955in}}{\pgfqpoint{0.872755in}{1.340955in}}%
\pgfpathcurveto{\pgfqpoint{0.864519in}{1.340955in}}{\pgfqpoint{0.856619in}{1.337683in}}{\pgfqpoint{0.850795in}{1.331859in}}%
\pgfpathcurveto{\pgfqpoint{0.844971in}{1.326035in}}{\pgfqpoint{0.841699in}{1.318135in}}{\pgfqpoint{0.841699in}{1.309899in}}%
\pgfpathcurveto{\pgfqpoint{0.841699in}{1.301662in}}{\pgfqpoint{0.844971in}{1.293762in}}{\pgfqpoint{0.850795in}{1.287938in}}%
\pgfpathcurveto{\pgfqpoint{0.856619in}{1.282115in}}{\pgfqpoint{0.864519in}{1.278842in}}{\pgfqpoint{0.872755in}{1.278842in}}%
\pgfpathclose%
\pgfusepath{stroke,fill}%
\end{pgfscope}%
\begin{pgfscope}%
\pgfpathrectangle{\pgfqpoint{0.100000in}{0.212622in}}{\pgfqpoint{3.696000in}{3.696000in}}%
\pgfusepath{clip}%
\pgfsetbuttcap%
\pgfsetroundjoin%
\definecolor{currentfill}{rgb}{0.121569,0.466667,0.705882}%
\pgfsetfillcolor{currentfill}%
\pgfsetfillopacity{0.630008}%
\pgfsetlinewidth{1.003750pt}%
\definecolor{currentstroke}{rgb}{0.121569,0.466667,0.705882}%
\pgfsetstrokecolor{currentstroke}%
\pgfsetstrokeopacity{0.630008}%
\pgfsetdash{}{0pt}%
\pgfpathmoveto{\pgfqpoint{0.899897in}{1.259847in}}%
\pgfpathcurveto{\pgfqpoint{0.908134in}{1.259847in}}{\pgfqpoint{0.916034in}{1.263120in}}{\pgfqpoint{0.921858in}{1.268944in}}%
\pgfpathcurveto{\pgfqpoint{0.927682in}{1.274767in}}{\pgfqpoint{0.930954in}{1.282667in}}{\pgfqpoint{0.930954in}{1.290904in}}%
\pgfpathcurveto{\pgfqpoint{0.930954in}{1.299140in}}{\pgfqpoint{0.927682in}{1.307040in}}{\pgfqpoint{0.921858in}{1.312864in}}%
\pgfpathcurveto{\pgfqpoint{0.916034in}{1.318688in}}{\pgfqpoint{0.908134in}{1.321960in}}{\pgfqpoint{0.899897in}{1.321960in}}%
\pgfpathcurveto{\pgfqpoint{0.891661in}{1.321960in}}{\pgfqpoint{0.883761in}{1.318688in}}{\pgfqpoint{0.877937in}{1.312864in}}%
\pgfpathcurveto{\pgfqpoint{0.872113in}{1.307040in}}{\pgfqpoint{0.868841in}{1.299140in}}{\pgfqpoint{0.868841in}{1.290904in}}%
\pgfpathcurveto{\pgfqpoint{0.868841in}{1.282667in}}{\pgfqpoint{0.872113in}{1.274767in}}{\pgfqpoint{0.877937in}{1.268944in}}%
\pgfpathcurveto{\pgfqpoint{0.883761in}{1.263120in}}{\pgfqpoint{0.891661in}{1.259847in}}{\pgfqpoint{0.899897in}{1.259847in}}%
\pgfpathclose%
\pgfusepath{stroke,fill}%
\end{pgfscope}%
\begin{pgfscope}%
\pgfpathrectangle{\pgfqpoint{0.100000in}{0.212622in}}{\pgfqpoint{3.696000in}{3.696000in}}%
\pgfusepath{clip}%
\pgfsetbuttcap%
\pgfsetroundjoin%
\definecolor{currentfill}{rgb}{0.121569,0.466667,0.705882}%
\pgfsetfillcolor{currentfill}%
\pgfsetfillopacity{0.630173}%
\pgfsetlinewidth{1.003750pt}%
\definecolor{currentstroke}{rgb}{0.121569,0.466667,0.705882}%
\pgfsetstrokecolor{currentstroke}%
\pgfsetstrokeopacity{0.630173}%
\pgfsetdash{}{0pt}%
\pgfpathmoveto{\pgfqpoint{2.127084in}{1.795085in}}%
\pgfpathcurveto{\pgfqpoint{2.135321in}{1.795085in}}{\pgfqpoint{2.143221in}{1.798357in}}{\pgfqpoint{2.149045in}{1.804181in}}%
\pgfpathcurveto{\pgfqpoint{2.154869in}{1.810005in}}{\pgfqpoint{2.158141in}{1.817905in}}{\pgfqpoint{2.158141in}{1.826141in}}%
\pgfpathcurveto{\pgfqpoint{2.158141in}{1.834377in}}{\pgfqpoint{2.154869in}{1.842277in}}{\pgfqpoint{2.149045in}{1.848101in}}%
\pgfpathcurveto{\pgfqpoint{2.143221in}{1.853925in}}{\pgfqpoint{2.135321in}{1.857198in}}{\pgfqpoint{2.127084in}{1.857198in}}%
\pgfpathcurveto{\pgfqpoint{2.118848in}{1.857198in}}{\pgfqpoint{2.110948in}{1.853925in}}{\pgfqpoint{2.105124in}{1.848101in}}%
\pgfpathcurveto{\pgfqpoint{2.099300in}{1.842277in}}{\pgfqpoint{2.096028in}{1.834377in}}{\pgfqpoint{2.096028in}{1.826141in}}%
\pgfpathcurveto{\pgfqpoint{2.096028in}{1.817905in}}{\pgfqpoint{2.099300in}{1.810005in}}{\pgfqpoint{2.105124in}{1.804181in}}%
\pgfpathcurveto{\pgfqpoint{2.110948in}{1.798357in}}{\pgfqpoint{2.118848in}{1.795085in}}{\pgfqpoint{2.127084in}{1.795085in}}%
\pgfpathclose%
\pgfusepath{stroke,fill}%
\end{pgfscope}%
\begin{pgfscope}%
\pgfpathrectangle{\pgfqpoint{0.100000in}{0.212622in}}{\pgfqpoint{3.696000in}{3.696000in}}%
\pgfusepath{clip}%
\pgfsetbuttcap%
\pgfsetroundjoin%
\definecolor{currentfill}{rgb}{0.121569,0.466667,0.705882}%
\pgfsetfillcolor{currentfill}%
\pgfsetfillopacity{0.630313}%
\pgfsetlinewidth{1.003750pt}%
\definecolor{currentstroke}{rgb}{0.121569,0.466667,0.705882}%
\pgfsetstrokecolor{currentstroke}%
\pgfsetstrokeopacity{0.630313}%
\pgfsetdash{}{0pt}%
\pgfpathmoveto{\pgfqpoint{0.871633in}{1.277328in}}%
\pgfpathcurveto{\pgfqpoint{0.879869in}{1.277328in}}{\pgfqpoint{0.887769in}{1.280601in}}{\pgfqpoint{0.893593in}{1.286424in}}%
\pgfpathcurveto{\pgfqpoint{0.899417in}{1.292248in}}{\pgfqpoint{0.902690in}{1.300148in}}{\pgfqpoint{0.902690in}{1.308385in}}%
\pgfpathcurveto{\pgfqpoint{0.902690in}{1.316621in}}{\pgfqpoint{0.899417in}{1.324521in}}{\pgfqpoint{0.893593in}{1.330345in}}%
\pgfpathcurveto{\pgfqpoint{0.887769in}{1.336169in}}{\pgfqpoint{0.879869in}{1.339441in}}{\pgfqpoint{0.871633in}{1.339441in}}%
\pgfpathcurveto{\pgfqpoint{0.863397in}{1.339441in}}{\pgfqpoint{0.855497in}{1.336169in}}{\pgfqpoint{0.849673in}{1.330345in}}%
\pgfpathcurveto{\pgfqpoint{0.843849in}{1.324521in}}{\pgfqpoint{0.840577in}{1.316621in}}{\pgfqpoint{0.840577in}{1.308385in}}%
\pgfpathcurveto{\pgfqpoint{0.840577in}{1.300148in}}{\pgfqpoint{0.843849in}{1.292248in}}{\pgfqpoint{0.849673in}{1.286424in}}%
\pgfpathcurveto{\pgfqpoint{0.855497in}{1.280601in}}{\pgfqpoint{0.863397in}{1.277328in}}{\pgfqpoint{0.871633in}{1.277328in}}%
\pgfpathclose%
\pgfusepath{stroke,fill}%
\end{pgfscope}%
\begin{pgfscope}%
\pgfpathrectangle{\pgfqpoint{0.100000in}{0.212622in}}{\pgfqpoint{3.696000in}{3.696000in}}%
\pgfusepath{clip}%
\pgfsetbuttcap%
\pgfsetroundjoin%
\definecolor{currentfill}{rgb}{0.121569,0.466667,0.705882}%
\pgfsetfillcolor{currentfill}%
\pgfsetfillopacity{0.631119}%
\pgfsetlinewidth{1.003750pt}%
\definecolor{currentstroke}{rgb}{0.121569,0.466667,0.705882}%
\pgfsetstrokecolor{currentstroke}%
\pgfsetstrokeopacity{0.631119}%
\pgfsetdash{}{0pt}%
\pgfpathmoveto{\pgfqpoint{0.897784in}{1.260490in}}%
\pgfpathcurveto{\pgfqpoint{0.906020in}{1.260490in}}{\pgfqpoint{0.913920in}{1.263762in}}{\pgfqpoint{0.919744in}{1.269586in}}%
\pgfpathcurveto{\pgfqpoint{0.925568in}{1.275410in}}{\pgfqpoint{0.928840in}{1.283310in}}{\pgfqpoint{0.928840in}{1.291547in}}%
\pgfpathcurveto{\pgfqpoint{0.928840in}{1.299783in}}{\pgfqpoint{0.925568in}{1.307683in}}{\pgfqpoint{0.919744in}{1.313507in}}%
\pgfpathcurveto{\pgfqpoint{0.913920in}{1.319331in}}{\pgfqpoint{0.906020in}{1.322603in}}{\pgfqpoint{0.897784in}{1.322603in}}%
\pgfpathcurveto{\pgfqpoint{0.889548in}{1.322603in}}{\pgfqpoint{0.881647in}{1.319331in}}{\pgfqpoint{0.875824in}{1.313507in}}%
\pgfpathcurveto{\pgfqpoint{0.870000in}{1.307683in}}{\pgfqpoint{0.866727in}{1.299783in}}{\pgfqpoint{0.866727in}{1.291547in}}%
\pgfpathcurveto{\pgfqpoint{0.866727in}{1.283310in}}{\pgfqpoint{0.870000in}{1.275410in}}{\pgfqpoint{0.875824in}{1.269586in}}%
\pgfpathcurveto{\pgfqpoint{0.881647in}{1.263762in}}{\pgfqpoint{0.889548in}{1.260490in}}{\pgfqpoint{0.897784in}{1.260490in}}%
\pgfpathclose%
\pgfusepath{stroke,fill}%
\end{pgfscope}%
\begin{pgfscope}%
\pgfpathrectangle{\pgfqpoint{0.100000in}{0.212622in}}{\pgfqpoint{3.696000in}{3.696000in}}%
\pgfusepath{clip}%
\pgfsetbuttcap%
\pgfsetroundjoin%
\definecolor{currentfill}{rgb}{0.121569,0.466667,0.705882}%
\pgfsetfillcolor{currentfill}%
\pgfsetfillopacity{0.631357}%
\pgfsetlinewidth{1.003750pt}%
\definecolor{currentstroke}{rgb}{0.121569,0.466667,0.705882}%
\pgfsetstrokecolor{currentstroke}%
\pgfsetstrokeopacity{0.631357}%
\pgfsetdash{}{0pt}%
\pgfpathmoveto{\pgfqpoint{0.871442in}{1.274041in}}%
\pgfpathcurveto{\pgfqpoint{0.879678in}{1.274041in}}{\pgfqpoint{0.887578in}{1.277314in}}{\pgfqpoint{0.893402in}{1.283137in}}%
\pgfpathcurveto{\pgfqpoint{0.899226in}{1.288961in}}{\pgfqpoint{0.902499in}{1.296861in}}{\pgfqpoint{0.902499in}{1.305098in}}%
\pgfpathcurveto{\pgfqpoint{0.902499in}{1.313334in}}{\pgfqpoint{0.899226in}{1.321234in}}{\pgfqpoint{0.893402in}{1.327058in}}%
\pgfpathcurveto{\pgfqpoint{0.887578in}{1.332882in}}{\pgfqpoint{0.879678in}{1.336154in}}{\pgfqpoint{0.871442in}{1.336154in}}%
\pgfpathcurveto{\pgfqpoint{0.863206in}{1.336154in}}{\pgfqpoint{0.855306in}{1.332882in}}{\pgfqpoint{0.849482in}{1.327058in}}%
\pgfpathcurveto{\pgfqpoint{0.843658in}{1.321234in}}{\pgfqpoint{0.840386in}{1.313334in}}{\pgfqpoint{0.840386in}{1.305098in}}%
\pgfpathcurveto{\pgfqpoint{0.840386in}{1.296861in}}{\pgfqpoint{0.843658in}{1.288961in}}{\pgfqpoint{0.849482in}{1.283137in}}%
\pgfpathcurveto{\pgfqpoint{0.855306in}{1.277314in}}{\pgfqpoint{0.863206in}{1.274041in}}{\pgfqpoint{0.871442in}{1.274041in}}%
\pgfpathclose%
\pgfusepath{stroke,fill}%
\end{pgfscope}%
\begin{pgfscope}%
\pgfpathrectangle{\pgfqpoint{0.100000in}{0.212622in}}{\pgfqpoint{3.696000in}{3.696000in}}%
\pgfusepath{clip}%
\pgfsetbuttcap%
\pgfsetroundjoin%
\definecolor{currentfill}{rgb}{0.121569,0.466667,0.705882}%
\pgfsetfillcolor{currentfill}%
\pgfsetfillopacity{0.631427}%
\pgfsetlinewidth{1.003750pt}%
\definecolor{currentstroke}{rgb}{0.121569,0.466667,0.705882}%
\pgfsetstrokecolor{currentstroke}%
\pgfsetstrokeopacity{0.631427}%
\pgfsetdash{}{0pt}%
\pgfpathmoveto{\pgfqpoint{2.128117in}{1.793540in}}%
\pgfpathcurveto{\pgfqpoint{2.136353in}{1.793540in}}{\pgfqpoint{2.144254in}{1.796813in}}{\pgfqpoint{2.150077in}{1.802637in}}%
\pgfpathcurveto{\pgfqpoint{2.155901in}{1.808461in}}{\pgfqpoint{2.159174in}{1.816361in}}{\pgfqpoint{2.159174in}{1.824597in}}%
\pgfpathcurveto{\pgfqpoint{2.159174in}{1.832833in}}{\pgfqpoint{2.155901in}{1.840733in}}{\pgfqpoint{2.150077in}{1.846557in}}%
\pgfpathcurveto{\pgfqpoint{2.144254in}{1.852381in}}{\pgfqpoint{2.136353in}{1.855653in}}{\pgfqpoint{2.128117in}{1.855653in}}%
\pgfpathcurveto{\pgfqpoint{2.119881in}{1.855653in}}{\pgfqpoint{2.111981in}{1.852381in}}{\pgfqpoint{2.106157in}{1.846557in}}%
\pgfpathcurveto{\pgfqpoint{2.100333in}{1.840733in}}{\pgfqpoint{2.097061in}{1.832833in}}{\pgfqpoint{2.097061in}{1.824597in}}%
\pgfpathcurveto{\pgfqpoint{2.097061in}{1.816361in}}{\pgfqpoint{2.100333in}{1.808461in}}{\pgfqpoint{2.106157in}{1.802637in}}%
\pgfpathcurveto{\pgfqpoint{2.111981in}{1.796813in}}{\pgfqpoint{2.119881in}{1.793540in}}{\pgfqpoint{2.128117in}{1.793540in}}%
\pgfpathclose%
\pgfusepath{stroke,fill}%
\end{pgfscope}%
\begin{pgfscope}%
\pgfpathrectangle{\pgfqpoint{0.100000in}{0.212622in}}{\pgfqpoint{3.696000in}{3.696000in}}%
\pgfusepath{clip}%
\pgfsetbuttcap%
\pgfsetroundjoin%
\definecolor{currentfill}{rgb}{0.121569,0.466667,0.705882}%
\pgfsetfillcolor{currentfill}%
\pgfsetfillopacity{0.631754}%
\pgfsetlinewidth{1.003750pt}%
\definecolor{currentstroke}{rgb}{0.121569,0.466667,0.705882}%
\pgfsetstrokecolor{currentstroke}%
\pgfsetstrokeopacity{0.631754}%
\pgfsetdash{}{0pt}%
\pgfpathmoveto{\pgfqpoint{0.896626in}{1.260984in}}%
\pgfpathcurveto{\pgfqpoint{0.904862in}{1.260984in}}{\pgfqpoint{0.912762in}{1.264256in}}{\pgfqpoint{0.918586in}{1.270080in}}%
\pgfpathcurveto{\pgfqpoint{0.924410in}{1.275904in}}{\pgfqpoint{0.927682in}{1.283804in}}{\pgfqpoint{0.927682in}{1.292041in}}%
\pgfpathcurveto{\pgfqpoint{0.927682in}{1.300277in}}{\pgfqpoint{0.924410in}{1.308177in}}{\pgfqpoint{0.918586in}{1.314001in}}%
\pgfpathcurveto{\pgfqpoint{0.912762in}{1.319825in}}{\pgfqpoint{0.904862in}{1.323097in}}{\pgfqpoint{0.896626in}{1.323097in}}%
\pgfpathcurveto{\pgfqpoint{0.888389in}{1.323097in}}{\pgfqpoint{0.880489in}{1.319825in}}{\pgfqpoint{0.874665in}{1.314001in}}%
\pgfpathcurveto{\pgfqpoint{0.868841in}{1.308177in}}{\pgfqpoint{0.865569in}{1.300277in}}{\pgfqpoint{0.865569in}{1.292041in}}%
\pgfpathcurveto{\pgfqpoint{0.865569in}{1.283804in}}{\pgfqpoint{0.868841in}{1.275904in}}{\pgfqpoint{0.874665in}{1.270080in}}%
\pgfpathcurveto{\pgfqpoint{0.880489in}{1.264256in}}{\pgfqpoint{0.888389in}{1.260984in}}{\pgfqpoint{0.896626in}{1.260984in}}%
\pgfpathclose%
\pgfusepath{stroke,fill}%
\end{pgfscope}%
\begin{pgfscope}%
\pgfpathrectangle{\pgfqpoint{0.100000in}{0.212622in}}{\pgfqpoint{3.696000in}{3.696000in}}%
\pgfusepath{clip}%
\pgfsetbuttcap%
\pgfsetroundjoin%
\definecolor{currentfill}{rgb}{0.121569,0.466667,0.705882}%
\pgfsetfillcolor{currentfill}%
\pgfsetfillopacity{0.632103}%
\pgfsetlinewidth{1.003750pt}%
\definecolor{currentstroke}{rgb}{0.121569,0.466667,0.705882}%
\pgfsetstrokecolor{currentstroke}%
\pgfsetstrokeopacity{0.632103}%
\pgfsetdash{}{0pt}%
\pgfpathmoveto{\pgfqpoint{0.895996in}{1.261248in}}%
\pgfpathcurveto{\pgfqpoint{0.904233in}{1.261248in}}{\pgfqpoint{0.912133in}{1.264521in}}{\pgfqpoint{0.917957in}{1.270344in}}%
\pgfpathcurveto{\pgfqpoint{0.923781in}{1.276168in}}{\pgfqpoint{0.927053in}{1.284068in}}{\pgfqpoint{0.927053in}{1.292305in}}%
\pgfpathcurveto{\pgfqpoint{0.927053in}{1.300541in}}{\pgfqpoint{0.923781in}{1.308441in}}{\pgfqpoint{0.917957in}{1.314265in}}%
\pgfpathcurveto{\pgfqpoint{0.912133in}{1.320089in}}{\pgfqpoint{0.904233in}{1.323361in}}{\pgfqpoint{0.895996in}{1.323361in}}%
\pgfpathcurveto{\pgfqpoint{0.887760in}{1.323361in}}{\pgfqpoint{0.879860in}{1.320089in}}{\pgfqpoint{0.874036in}{1.314265in}}%
\pgfpathcurveto{\pgfqpoint{0.868212in}{1.308441in}}{\pgfqpoint{0.864940in}{1.300541in}}{\pgfqpoint{0.864940in}{1.292305in}}%
\pgfpathcurveto{\pgfqpoint{0.864940in}{1.284068in}}{\pgfqpoint{0.868212in}{1.276168in}}{\pgfqpoint{0.874036in}{1.270344in}}%
\pgfpathcurveto{\pgfqpoint{0.879860in}{1.264521in}}{\pgfqpoint{0.887760in}{1.261248in}}{\pgfqpoint{0.895996in}{1.261248in}}%
\pgfpathclose%
\pgfusepath{stroke,fill}%
\end{pgfscope}%
\begin{pgfscope}%
\pgfpathrectangle{\pgfqpoint{0.100000in}{0.212622in}}{\pgfqpoint{3.696000in}{3.696000in}}%
\pgfusepath{clip}%
\pgfsetbuttcap%
\pgfsetroundjoin%
\definecolor{currentfill}{rgb}{0.121569,0.466667,0.705882}%
\pgfsetfillcolor{currentfill}%
\pgfsetfillopacity{0.632298}%
\pgfsetlinewidth{1.003750pt}%
\definecolor{currentstroke}{rgb}{0.121569,0.466667,0.705882}%
\pgfsetstrokecolor{currentstroke}%
\pgfsetstrokeopacity{0.632298}%
\pgfsetdash{}{0pt}%
\pgfpathmoveto{\pgfqpoint{0.895661in}{1.261401in}}%
\pgfpathcurveto{\pgfqpoint{0.903897in}{1.261401in}}{\pgfqpoint{0.911797in}{1.264673in}}{\pgfqpoint{0.917621in}{1.270497in}}%
\pgfpathcurveto{\pgfqpoint{0.923445in}{1.276321in}}{\pgfqpoint{0.926717in}{1.284221in}}{\pgfqpoint{0.926717in}{1.292457in}}%
\pgfpathcurveto{\pgfqpoint{0.926717in}{1.300693in}}{\pgfqpoint{0.923445in}{1.308593in}}{\pgfqpoint{0.917621in}{1.314417in}}%
\pgfpathcurveto{\pgfqpoint{0.911797in}{1.320241in}}{\pgfqpoint{0.903897in}{1.323514in}}{\pgfqpoint{0.895661in}{1.323514in}}%
\pgfpathcurveto{\pgfqpoint{0.887425in}{1.323514in}}{\pgfqpoint{0.879525in}{1.320241in}}{\pgfqpoint{0.873701in}{1.314417in}}%
\pgfpathcurveto{\pgfqpoint{0.867877in}{1.308593in}}{\pgfqpoint{0.864604in}{1.300693in}}{\pgfqpoint{0.864604in}{1.292457in}}%
\pgfpathcurveto{\pgfqpoint{0.864604in}{1.284221in}}{\pgfqpoint{0.867877in}{1.276321in}}{\pgfqpoint{0.873701in}{1.270497in}}%
\pgfpathcurveto{\pgfqpoint{0.879525in}{1.264673in}}{\pgfqpoint{0.887425in}{1.261401in}}{\pgfqpoint{0.895661in}{1.261401in}}%
\pgfpathclose%
\pgfusepath{stroke,fill}%
\end{pgfscope}%
\begin{pgfscope}%
\pgfpathrectangle{\pgfqpoint{0.100000in}{0.212622in}}{\pgfqpoint{3.696000in}{3.696000in}}%
\pgfusepath{clip}%
\pgfsetbuttcap%
\pgfsetroundjoin%
\definecolor{currentfill}{rgb}{0.121569,0.466667,0.705882}%
\pgfsetfillcolor{currentfill}%
\pgfsetfillopacity{0.632400}%
\pgfsetlinewidth{1.003750pt}%
\definecolor{currentstroke}{rgb}{0.121569,0.466667,0.705882}%
\pgfsetstrokecolor{currentstroke}%
\pgfsetstrokeopacity{0.632400}%
\pgfsetdash{}{0pt}%
\pgfpathmoveto{\pgfqpoint{0.895467in}{1.261461in}}%
\pgfpathcurveto{\pgfqpoint{0.903704in}{1.261461in}}{\pgfqpoint{0.911604in}{1.264733in}}{\pgfqpoint{0.917428in}{1.270557in}}%
\pgfpathcurveto{\pgfqpoint{0.923252in}{1.276381in}}{\pgfqpoint{0.926524in}{1.284281in}}{\pgfqpoint{0.926524in}{1.292517in}}%
\pgfpathcurveto{\pgfqpoint{0.926524in}{1.300754in}}{\pgfqpoint{0.923252in}{1.308654in}}{\pgfqpoint{0.917428in}{1.314478in}}%
\pgfpathcurveto{\pgfqpoint{0.911604in}{1.320302in}}{\pgfqpoint{0.903704in}{1.323574in}}{\pgfqpoint{0.895467in}{1.323574in}}%
\pgfpathcurveto{\pgfqpoint{0.887231in}{1.323574in}}{\pgfqpoint{0.879331in}{1.320302in}}{\pgfqpoint{0.873507in}{1.314478in}}%
\pgfpathcurveto{\pgfqpoint{0.867683in}{1.308654in}}{\pgfqpoint{0.864411in}{1.300754in}}{\pgfqpoint{0.864411in}{1.292517in}}%
\pgfpathcurveto{\pgfqpoint{0.864411in}{1.284281in}}{\pgfqpoint{0.867683in}{1.276381in}}{\pgfqpoint{0.873507in}{1.270557in}}%
\pgfpathcurveto{\pgfqpoint{0.879331in}{1.264733in}}{\pgfqpoint{0.887231in}{1.261461in}}{\pgfqpoint{0.895467in}{1.261461in}}%
\pgfpathclose%
\pgfusepath{stroke,fill}%
\end{pgfscope}%
\begin{pgfscope}%
\pgfpathrectangle{\pgfqpoint{0.100000in}{0.212622in}}{\pgfqpoint{3.696000in}{3.696000in}}%
\pgfusepath{clip}%
\pgfsetbuttcap%
\pgfsetroundjoin%
\definecolor{currentfill}{rgb}{0.121569,0.466667,0.705882}%
\pgfsetfillcolor{currentfill}%
\pgfsetfillopacity{0.632459}%
\pgfsetlinewidth{1.003750pt}%
\definecolor{currentstroke}{rgb}{0.121569,0.466667,0.705882}%
\pgfsetstrokecolor{currentstroke}%
\pgfsetstrokeopacity{0.632459}%
\pgfsetdash{}{0pt}%
\pgfpathmoveto{\pgfqpoint{0.895365in}{1.261506in}}%
\pgfpathcurveto{\pgfqpoint{0.903601in}{1.261506in}}{\pgfqpoint{0.911501in}{1.264779in}}{\pgfqpoint{0.917325in}{1.270603in}}%
\pgfpathcurveto{\pgfqpoint{0.923149in}{1.276427in}}{\pgfqpoint{0.926421in}{1.284327in}}{\pgfqpoint{0.926421in}{1.292563in}}%
\pgfpathcurveto{\pgfqpoint{0.926421in}{1.300799in}}{\pgfqpoint{0.923149in}{1.308699in}}{\pgfqpoint{0.917325in}{1.314523in}}%
\pgfpathcurveto{\pgfqpoint{0.911501in}{1.320347in}}{\pgfqpoint{0.903601in}{1.323619in}}{\pgfqpoint{0.895365in}{1.323619in}}%
\pgfpathcurveto{\pgfqpoint{0.887129in}{1.323619in}}{\pgfqpoint{0.879229in}{1.320347in}}{\pgfqpoint{0.873405in}{1.314523in}}%
\pgfpathcurveto{\pgfqpoint{0.867581in}{1.308699in}}{\pgfqpoint{0.864308in}{1.300799in}}{\pgfqpoint{0.864308in}{1.292563in}}%
\pgfpathcurveto{\pgfqpoint{0.864308in}{1.284327in}}{\pgfqpoint{0.867581in}{1.276427in}}{\pgfqpoint{0.873405in}{1.270603in}}%
\pgfpathcurveto{\pgfqpoint{0.879229in}{1.264779in}}{\pgfqpoint{0.887129in}{1.261506in}}{\pgfqpoint{0.895365in}{1.261506in}}%
\pgfpathclose%
\pgfusepath{stroke,fill}%
\end{pgfscope}%
\begin{pgfscope}%
\pgfpathrectangle{\pgfqpoint{0.100000in}{0.212622in}}{\pgfqpoint{3.696000in}{3.696000in}}%
\pgfusepath{clip}%
\pgfsetbuttcap%
\pgfsetroundjoin%
\definecolor{currentfill}{rgb}{0.121569,0.466667,0.705882}%
\pgfsetfillcolor{currentfill}%
\pgfsetfillopacity{0.632491}%
\pgfsetlinewidth{1.003750pt}%
\definecolor{currentstroke}{rgb}{0.121569,0.466667,0.705882}%
\pgfsetstrokecolor{currentstroke}%
\pgfsetstrokeopacity{0.632491}%
\pgfsetdash{}{0pt}%
\pgfpathmoveto{\pgfqpoint{0.895310in}{1.261527in}}%
\pgfpathcurveto{\pgfqpoint{0.903546in}{1.261527in}}{\pgfqpoint{0.911446in}{1.264799in}}{\pgfqpoint{0.917270in}{1.270623in}}%
\pgfpathcurveto{\pgfqpoint{0.923094in}{1.276447in}}{\pgfqpoint{0.926366in}{1.284347in}}{\pgfqpoint{0.926366in}{1.292583in}}%
\pgfpathcurveto{\pgfqpoint{0.926366in}{1.300820in}}{\pgfqpoint{0.923094in}{1.308720in}}{\pgfqpoint{0.917270in}{1.314544in}}%
\pgfpathcurveto{\pgfqpoint{0.911446in}{1.320367in}}{\pgfqpoint{0.903546in}{1.323640in}}{\pgfqpoint{0.895310in}{1.323640in}}%
\pgfpathcurveto{\pgfqpoint{0.887073in}{1.323640in}}{\pgfqpoint{0.879173in}{1.320367in}}{\pgfqpoint{0.873349in}{1.314544in}}%
\pgfpathcurveto{\pgfqpoint{0.867525in}{1.308720in}}{\pgfqpoint{0.864253in}{1.300820in}}{\pgfqpoint{0.864253in}{1.292583in}}%
\pgfpathcurveto{\pgfqpoint{0.864253in}{1.284347in}}{\pgfqpoint{0.867525in}{1.276447in}}{\pgfqpoint{0.873349in}{1.270623in}}%
\pgfpathcurveto{\pgfqpoint{0.879173in}{1.264799in}}{\pgfqpoint{0.887073in}{1.261527in}}{\pgfqpoint{0.895310in}{1.261527in}}%
\pgfpathclose%
\pgfusepath{stroke,fill}%
\end{pgfscope}%
\begin{pgfscope}%
\pgfpathrectangle{\pgfqpoint{0.100000in}{0.212622in}}{\pgfqpoint{3.696000in}{3.696000in}}%
\pgfusepath{clip}%
\pgfsetbuttcap%
\pgfsetroundjoin%
\definecolor{currentfill}{rgb}{0.121569,0.466667,0.705882}%
\pgfsetfillcolor{currentfill}%
\pgfsetfillopacity{0.632509}%
\pgfsetlinewidth{1.003750pt}%
\definecolor{currentstroke}{rgb}{0.121569,0.466667,0.705882}%
\pgfsetstrokecolor{currentstroke}%
\pgfsetstrokeopacity{0.632509}%
\pgfsetdash{}{0pt}%
\pgfpathmoveto{\pgfqpoint{0.895278in}{1.261541in}}%
\pgfpathcurveto{\pgfqpoint{0.903514in}{1.261541in}}{\pgfqpoint{0.911414in}{1.264813in}}{\pgfqpoint{0.917238in}{1.270637in}}%
\pgfpathcurveto{\pgfqpoint{0.923062in}{1.276461in}}{\pgfqpoint{0.926334in}{1.284361in}}{\pgfqpoint{0.926334in}{1.292597in}}%
\pgfpathcurveto{\pgfqpoint{0.926334in}{1.300834in}}{\pgfqpoint{0.923062in}{1.308734in}}{\pgfqpoint{0.917238in}{1.314558in}}%
\pgfpathcurveto{\pgfqpoint{0.911414in}{1.320382in}}{\pgfqpoint{0.903514in}{1.323654in}}{\pgfqpoint{0.895278in}{1.323654in}}%
\pgfpathcurveto{\pgfqpoint{0.887041in}{1.323654in}}{\pgfqpoint{0.879141in}{1.320382in}}{\pgfqpoint{0.873317in}{1.314558in}}%
\pgfpathcurveto{\pgfqpoint{0.867493in}{1.308734in}}{\pgfqpoint{0.864221in}{1.300834in}}{\pgfqpoint{0.864221in}{1.292597in}}%
\pgfpathcurveto{\pgfqpoint{0.864221in}{1.284361in}}{\pgfqpoint{0.867493in}{1.276461in}}{\pgfqpoint{0.873317in}{1.270637in}}%
\pgfpathcurveto{\pgfqpoint{0.879141in}{1.264813in}}{\pgfqpoint{0.887041in}{1.261541in}}{\pgfqpoint{0.895278in}{1.261541in}}%
\pgfpathclose%
\pgfusepath{stroke,fill}%
\end{pgfscope}%
\begin{pgfscope}%
\pgfpathrectangle{\pgfqpoint{0.100000in}{0.212622in}}{\pgfqpoint{3.696000in}{3.696000in}}%
\pgfusepath{clip}%
\pgfsetbuttcap%
\pgfsetroundjoin%
\definecolor{currentfill}{rgb}{0.121569,0.466667,0.705882}%
\pgfsetfillcolor{currentfill}%
\pgfsetfillopacity{0.632519}%
\pgfsetlinewidth{1.003750pt}%
\definecolor{currentstroke}{rgb}{0.121569,0.466667,0.705882}%
\pgfsetstrokecolor{currentstroke}%
\pgfsetstrokeopacity{0.632519}%
\pgfsetdash{}{0pt}%
\pgfpathmoveto{\pgfqpoint{0.895261in}{1.261549in}}%
\pgfpathcurveto{\pgfqpoint{0.903497in}{1.261549in}}{\pgfqpoint{0.911397in}{1.264821in}}{\pgfqpoint{0.917221in}{1.270645in}}%
\pgfpathcurveto{\pgfqpoint{0.923045in}{1.276469in}}{\pgfqpoint{0.926317in}{1.284369in}}{\pgfqpoint{0.926317in}{1.292605in}}%
\pgfpathcurveto{\pgfqpoint{0.926317in}{1.300842in}}{\pgfqpoint{0.923045in}{1.308742in}}{\pgfqpoint{0.917221in}{1.314566in}}%
\pgfpathcurveto{\pgfqpoint{0.911397in}{1.320390in}}{\pgfqpoint{0.903497in}{1.323662in}}{\pgfqpoint{0.895261in}{1.323662in}}%
\pgfpathcurveto{\pgfqpoint{0.887025in}{1.323662in}}{\pgfqpoint{0.879125in}{1.320390in}}{\pgfqpoint{0.873301in}{1.314566in}}%
\pgfpathcurveto{\pgfqpoint{0.867477in}{1.308742in}}{\pgfqpoint{0.864204in}{1.300842in}}{\pgfqpoint{0.864204in}{1.292605in}}%
\pgfpathcurveto{\pgfqpoint{0.864204in}{1.284369in}}{\pgfqpoint{0.867477in}{1.276469in}}{\pgfqpoint{0.873301in}{1.270645in}}%
\pgfpathcurveto{\pgfqpoint{0.879125in}{1.264821in}}{\pgfqpoint{0.887025in}{1.261549in}}{\pgfqpoint{0.895261in}{1.261549in}}%
\pgfpathclose%
\pgfusepath{stroke,fill}%
\end{pgfscope}%
\begin{pgfscope}%
\pgfpathrectangle{\pgfqpoint{0.100000in}{0.212622in}}{\pgfqpoint{3.696000in}{3.696000in}}%
\pgfusepath{clip}%
\pgfsetbuttcap%
\pgfsetroundjoin%
\definecolor{currentfill}{rgb}{0.121569,0.466667,0.705882}%
\pgfsetfillcolor{currentfill}%
\pgfsetfillopacity{0.632524}%
\pgfsetlinewidth{1.003750pt}%
\definecolor{currentstroke}{rgb}{0.121569,0.466667,0.705882}%
\pgfsetstrokecolor{currentstroke}%
\pgfsetstrokeopacity{0.632524}%
\pgfsetdash{}{0pt}%
\pgfpathmoveto{\pgfqpoint{0.895251in}{1.261554in}}%
\pgfpathcurveto{\pgfqpoint{0.903487in}{1.261554in}}{\pgfqpoint{0.911387in}{1.264827in}}{\pgfqpoint{0.917211in}{1.270651in}}%
\pgfpathcurveto{\pgfqpoint{0.923035in}{1.276475in}}{\pgfqpoint{0.926307in}{1.284375in}}{\pgfqpoint{0.926307in}{1.292611in}}%
\pgfpathcurveto{\pgfqpoint{0.926307in}{1.300847in}}{\pgfqpoint{0.923035in}{1.308747in}}{\pgfqpoint{0.917211in}{1.314571in}}%
\pgfpathcurveto{\pgfqpoint{0.911387in}{1.320395in}}{\pgfqpoint{0.903487in}{1.323667in}}{\pgfqpoint{0.895251in}{1.323667in}}%
\pgfpathcurveto{\pgfqpoint{0.887015in}{1.323667in}}{\pgfqpoint{0.879115in}{1.320395in}}{\pgfqpoint{0.873291in}{1.314571in}}%
\pgfpathcurveto{\pgfqpoint{0.867467in}{1.308747in}}{\pgfqpoint{0.864194in}{1.300847in}}{\pgfqpoint{0.864194in}{1.292611in}}%
\pgfpathcurveto{\pgfqpoint{0.864194in}{1.284375in}}{\pgfqpoint{0.867467in}{1.276475in}}{\pgfqpoint{0.873291in}{1.270651in}}%
\pgfpathcurveto{\pgfqpoint{0.879115in}{1.264827in}}{\pgfqpoint{0.887015in}{1.261554in}}{\pgfqpoint{0.895251in}{1.261554in}}%
\pgfpathclose%
\pgfusepath{stroke,fill}%
\end{pgfscope}%
\begin{pgfscope}%
\pgfpathrectangle{\pgfqpoint{0.100000in}{0.212622in}}{\pgfqpoint{3.696000in}{3.696000in}}%
\pgfusepath{clip}%
\pgfsetbuttcap%
\pgfsetroundjoin%
\definecolor{currentfill}{rgb}{0.121569,0.466667,0.705882}%
\pgfsetfillcolor{currentfill}%
\pgfsetfillopacity{0.632527}%
\pgfsetlinewidth{1.003750pt}%
\definecolor{currentstroke}{rgb}{0.121569,0.466667,0.705882}%
\pgfsetstrokecolor{currentstroke}%
\pgfsetstrokeopacity{0.632527}%
\pgfsetdash{}{0pt}%
\pgfpathmoveto{\pgfqpoint{0.895246in}{1.261557in}}%
\pgfpathcurveto{\pgfqpoint{0.903482in}{1.261557in}}{\pgfqpoint{0.911382in}{1.264829in}}{\pgfqpoint{0.917206in}{1.270653in}}%
\pgfpathcurveto{\pgfqpoint{0.923030in}{1.276477in}}{\pgfqpoint{0.926302in}{1.284377in}}{\pgfqpoint{0.926302in}{1.292613in}}%
\pgfpathcurveto{\pgfqpoint{0.926302in}{1.300849in}}{\pgfqpoint{0.923030in}{1.308749in}}{\pgfqpoint{0.917206in}{1.314573in}}%
\pgfpathcurveto{\pgfqpoint{0.911382in}{1.320397in}}{\pgfqpoint{0.903482in}{1.323670in}}{\pgfqpoint{0.895246in}{1.323670in}}%
\pgfpathcurveto{\pgfqpoint{0.887009in}{1.323670in}}{\pgfqpoint{0.879109in}{1.320397in}}{\pgfqpoint{0.873285in}{1.314573in}}%
\pgfpathcurveto{\pgfqpoint{0.867461in}{1.308749in}}{\pgfqpoint{0.864189in}{1.300849in}}{\pgfqpoint{0.864189in}{1.292613in}}%
\pgfpathcurveto{\pgfqpoint{0.864189in}{1.284377in}}{\pgfqpoint{0.867461in}{1.276477in}}{\pgfqpoint{0.873285in}{1.270653in}}%
\pgfpathcurveto{\pgfqpoint{0.879109in}{1.264829in}}{\pgfqpoint{0.887009in}{1.261557in}}{\pgfqpoint{0.895246in}{1.261557in}}%
\pgfpathclose%
\pgfusepath{stroke,fill}%
\end{pgfscope}%
\begin{pgfscope}%
\pgfpathrectangle{\pgfqpoint{0.100000in}{0.212622in}}{\pgfqpoint{3.696000in}{3.696000in}}%
\pgfusepath{clip}%
\pgfsetbuttcap%
\pgfsetroundjoin%
\definecolor{currentfill}{rgb}{0.121569,0.466667,0.705882}%
\pgfsetfillcolor{currentfill}%
\pgfsetfillopacity{0.632529}%
\pgfsetlinewidth{1.003750pt}%
\definecolor{currentstroke}{rgb}{0.121569,0.466667,0.705882}%
\pgfsetstrokecolor{currentstroke}%
\pgfsetstrokeopacity{0.632529}%
\pgfsetdash{}{0pt}%
\pgfpathmoveto{\pgfqpoint{0.895243in}{1.261558in}}%
\pgfpathcurveto{\pgfqpoint{0.903479in}{1.261558in}}{\pgfqpoint{0.911379in}{1.264830in}}{\pgfqpoint{0.917203in}{1.270654in}}%
\pgfpathcurveto{\pgfqpoint{0.923027in}{1.276478in}}{\pgfqpoint{0.926299in}{1.284378in}}{\pgfqpoint{0.926299in}{1.292615in}}%
\pgfpathcurveto{\pgfqpoint{0.926299in}{1.300851in}}{\pgfqpoint{0.923027in}{1.308751in}}{\pgfqpoint{0.917203in}{1.314575in}}%
\pgfpathcurveto{\pgfqpoint{0.911379in}{1.320399in}}{\pgfqpoint{0.903479in}{1.323671in}}{\pgfqpoint{0.895243in}{1.323671in}}%
\pgfpathcurveto{\pgfqpoint{0.887006in}{1.323671in}}{\pgfqpoint{0.879106in}{1.320399in}}{\pgfqpoint{0.873282in}{1.314575in}}%
\pgfpathcurveto{\pgfqpoint{0.867458in}{1.308751in}}{\pgfqpoint{0.864186in}{1.300851in}}{\pgfqpoint{0.864186in}{1.292615in}}%
\pgfpathcurveto{\pgfqpoint{0.864186in}{1.284378in}}{\pgfqpoint{0.867458in}{1.276478in}}{\pgfqpoint{0.873282in}{1.270654in}}%
\pgfpathcurveto{\pgfqpoint{0.879106in}{1.264830in}}{\pgfqpoint{0.887006in}{1.261558in}}{\pgfqpoint{0.895243in}{1.261558in}}%
\pgfpathclose%
\pgfusepath{stroke,fill}%
\end{pgfscope}%
\begin{pgfscope}%
\pgfpathrectangle{\pgfqpoint{0.100000in}{0.212622in}}{\pgfqpoint{3.696000in}{3.696000in}}%
\pgfusepath{clip}%
\pgfsetbuttcap%
\pgfsetroundjoin%
\definecolor{currentfill}{rgb}{0.121569,0.466667,0.705882}%
\pgfsetfillcolor{currentfill}%
\pgfsetfillopacity{0.632529}%
\pgfsetlinewidth{1.003750pt}%
\definecolor{currentstroke}{rgb}{0.121569,0.466667,0.705882}%
\pgfsetstrokecolor{currentstroke}%
\pgfsetstrokeopacity{0.632529}%
\pgfsetdash{}{0pt}%
\pgfpathmoveto{\pgfqpoint{0.895241in}{1.261559in}}%
\pgfpathcurveto{\pgfqpoint{0.903477in}{1.261559in}}{\pgfqpoint{0.911377in}{1.264831in}}{\pgfqpoint{0.917201in}{1.270655in}}%
\pgfpathcurveto{\pgfqpoint{0.923025in}{1.276479in}}{\pgfqpoint{0.926298in}{1.284379in}}{\pgfqpoint{0.926298in}{1.292615in}}%
\pgfpathcurveto{\pgfqpoint{0.926298in}{1.300851in}}{\pgfqpoint{0.923025in}{1.308751in}}{\pgfqpoint{0.917201in}{1.314575in}}%
\pgfpathcurveto{\pgfqpoint{0.911377in}{1.320399in}}{\pgfqpoint{0.903477in}{1.323672in}}{\pgfqpoint{0.895241in}{1.323672in}}%
\pgfpathcurveto{\pgfqpoint{0.887005in}{1.323672in}}{\pgfqpoint{0.879105in}{1.320399in}}{\pgfqpoint{0.873281in}{1.314575in}}%
\pgfpathcurveto{\pgfqpoint{0.867457in}{1.308751in}}{\pgfqpoint{0.864185in}{1.300851in}}{\pgfqpoint{0.864185in}{1.292615in}}%
\pgfpathcurveto{\pgfqpoint{0.864185in}{1.284379in}}{\pgfqpoint{0.867457in}{1.276479in}}{\pgfqpoint{0.873281in}{1.270655in}}%
\pgfpathcurveto{\pgfqpoint{0.879105in}{1.264831in}}{\pgfqpoint{0.887005in}{1.261559in}}{\pgfqpoint{0.895241in}{1.261559in}}%
\pgfpathclose%
\pgfusepath{stroke,fill}%
\end{pgfscope}%
\begin{pgfscope}%
\pgfpathrectangle{\pgfqpoint{0.100000in}{0.212622in}}{\pgfqpoint{3.696000in}{3.696000in}}%
\pgfusepath{clip}%
\pgfsetbuttcap%
\pgfsetroundjoin%
\definecolor{currentfill}{rgb}{0.121569,0.466667,0.705882}%
\pgfsetfillcolor{currentfill}%
\pgfsetfillopacity{0.632530}%
\pgfsetlinewidth{1.003750pt}%
\definecolor{currentstroke}{rgb}{0.121569,0.466667,0.705882}%
\pgfsetstrokecolor{currentstroke}%
\pgfsetstrokeopacity{0.632530}%
\pgfsetdash{}{0pt}%
\pgfpathmoveto{\pgfqpoint{0.895240in}{1.261559in}}%
\pgfpathcurveto{\pgfqpoint{0.903476in}{1.261559in}}{\pgfqpoint{0.911376in}{1.264831in}}{\pgfqpoint{0.917200in}{1.270655in}}%
\pgfpathcurveto{\pgfqpoint{0.923024in}{1.276479in}}{\pgfqpoint{0.926297in}{1.284379in}}{\pgfqpoint{0.926297in}{1.292615in}}%
\pgfpathcurveto{\pgfqpoint{0.926297in}{1.300852in}}{\pgfqpoint{0.923024in}{1.308752in}}{\pgfqpoint{0.917200in}{1.314576in}}%
\pgfpathcurveto{\pgfqpoint{0.911376in}{1.320400in}}{\pgfqpoint{0.903476in}{1.323672in}}{\pgfqpoint{0.895240in}{1.323672in}}%
\pgfpathcurveto{\pgfqpoint{0.887004in}{1.323672in}}{\pgfqpoint{0.879104in}{1.320400in}}{\pgfqpoint{0.873280in}{1.314576in}}%
\pgfpathcurveto{\pgfqpoint{0.867456in}{1.308752in}}{\pgfqpoint{0.864184in}{1.300852in}}{\pgfqpoint{0.864184in}{1.292615in}}%
\pgfpathcurveto{\pgfqpoint{0.864184in}{1.284379in}}{\pgfqpoint{0.867456in}{1.276479in}}{\pgfqpoint{0.873280in}{1.270655in}}%
\pgfpathcurveto{\pgfqpoint{0.879104in}{1.264831in}}{\pgfqpoint{0.887004in}{1.261559in}}{\pgfqpoint{0.895240in}{1.261559in}}%
\pgfpathclose%
\pgfusepath{stroke,fill}%
\end{pgfscope}%
\begin{pgfscope}%
\pgfpathrectangle{\pgfqpoint{0.100000in}{0.212622in}}{\pgfqpoint{3.696000in}{3.696000in}}%
\pgfusepath{clip}%
\pgfsetbuttcap%
\pgfsetroundjoin%
\definecolor{currentfill}{rgb}{0.121569,0.466667,0.705882}%
\pgfsetfillcolor{currentfill}%
\pgfsetfillopacity{0.632530}%
\pgfsetlinewidth{1.003750pt}%
\definecolor{currentstroke}{rgb}{0.121569,0.466667,0.705882}%
\pgfsetstrokecolor{currentstroke}%
\pgfsetstrokeopacity{0.632530}%
\pgfsetdash{}{0pt}%
\pgfpathmoveto{\pgfqpoint{0.895240in}{1.261559in}}%
\pgfpathcurveto{\pgfqpoint{0.903476in}{1.261559in}}{\pgfqpoint{0.911376in}{1.264831in}}{\pgfqpoint{0.917200in}{1.270655in}}%
\pgfpathcurveto{\pgfqpoint{0.923024in}{1.276479in}}{\pgfqpoint{0.926296in}{1.284379in}}{\pgfqpoint{0.926296in}{1.292616in}}%
\pgfpathcurveto{\pgfqpoint{0.926296in}{1.300852in}}{\pgfqpoint{0.923024in}{1.308752in}}{\pgfqpoint{0.917200in}{1.314576in}}%
\pgfpathcurveto{\pgfqpoint{0.911376in}{1.320400in}}{\pgfqpoint{0.903476in}{1.323672in}}{\pgfqpoint{0.895240in}{1.323672in}}%
\pgfpathcurveto{\pgfqpoint{0.887003in}{1.323672in}}{\pgfqpoint{0.879103in}{1.320400in}}{\pgfqpoint{0.873279in}{1.314576in}}%
\pgfpathcurveto{\pgfqpoint{0.867455in}{1.308752in}}{\pgfqpoint{0.864183in}{1.300852in}}{\pgfqpoint{0.864183in}{1.292616in}}%
\pgfpathcurveto{\pgfqpoint{0.864183in}{1.284379in}}{\pgfqpoint{0.867455in}{1.276479in}}{\pgfqpoint{0.873279in}{1.270655in}}%
\pgfpathcurveto{\pgfqpoint{0.879103in}{1.264831in}}{\pgfqpoint{0.887003in}{1.261559in}}{\pgfqpoint{0.895240in}{1.261559in}}%
\pgfpathclose%
\pgfusepath{stroke,fill}%
\end{pgfscope}%
\begin{pgfscope}%
\pgfpathrectangle{\pgfqpoint{0.100000in}{0.212622in}}{\pgfqpoint{3.696000in}{3.696000in}}%
\pgfusepath{clip}%
\pgfsetbuttcap%
\pgfsetroundjoin%
\definecolor{currentfill}{rgb}{0.121569,0.466667,0.705882}%
\pgfsetfillcolor{currentfill}%
\pgfsetfillopacity{0.632530}%
\pgfsetlinewidth{1.003750pt}%
\definecolor{currentstroke}{rgb}{0.121569,0.466667,0.705882}%
\pgfsetstrokecolor{currentstroke}%
\pgfsetstrokeopacity{0.632530}%
\pgfsetdash{}{0pt}%
\pgfpathmoveto{\pgfqpoint{0.895239in}{1.261559in}}%
\pgfpathcurveto{\pgfqpoint{0.903476in}{1.261559in}}{\pgfqpoint{0.911376in}{1.264831in}}{\pgfqpoint{0.917200in}{1.270655in}}%
\pgfpathcurveto{\pgfqpoint{0.923024in}{1.276479in}}{\pgfqpoint{0.926296in}{1.284379in}}{\pgfqpoint{0.926296in}{1.292616in}}%
\pgfpathcurveto{\pgfqpoint{0.926296in}{1.300852in}}{\pgfqpoint{0.923024in}{1.308752in}}{\pgfqpoint{0.917200in}{1.314576in}}%
\pgfpathcurveto{\pgfqpoint{0.911376in}{1.320400in}}{\pgfqpoint{0.903476in}{1.323672in}}{\pgfqpoint{0.895239in}{1.323672in}}%
\pgfpathcurveto{\pgfqpoint{0.887003in}{1.323672in}}{\pgfqpoint{0.879103in}{1.320400in}}{\pgfqpoint{0.873279in}{1.314576in}}%
\pgfpathcurveto{\pgfqpoint{0.867455in}{1.308752in}}{\pgfqpoint{0.864183in}{1.300852in}}{\pgfqpoint{0.864183in}{1.292616in}}%
\pgfpathcurveto{\pgfqpoint{0.864183in}{1.284379in}}{\pgfqpoint{0.867455in}{1.276479in}}{\pgfqpoint{0.873279in}{1.270655in}}%
\pgfpathcurveto{\pgfqpoint{0.879103in}{1.264831in}}{\pgfqpoint{0.887003in}{1.261559in}}{\pgfqpoint{0.895239in}{1.261559in}}%
\pgfpathclose%
\pgfusepath{stroke,fill}%
\end{pgfscope}%
\begin{pgfscope}%
\pgfpathrectangle{\pgfqpoint{0.100000in}{0.212622in}}{\pgfqpoint{3.696000in}{3.696000in}}%
\pgfusepath{clip}%
\pgfsetbuttcap%
\pgfsetroundjoin%
\definecolor{currentfill}{rgb}{0.121569,0.466667,0.705882}%
\pgfsetfillcolor{currentfill}%
\pgfsetfillopacity{0.632530}%
\pgfsetlinewidth{1.003750pt}%
\definecolor{currentstroke}{rgb}{0.121569,0.466667,0.705882}%
\pgfsetstrokecolor{currentstroke}%
\pgfsetstrokeopacity{0.632530}%
\pgfsetdash{}{0pt}%
\pgfpathmoveto{\pgfqpoint{0.895239in}{1.261559in}}%
\pgfpathcurveto{\pgfqpoint{0.903476in}{1.261559in}}{\pgfqpoint{0.911376in}{1.264831in}}{\pgfqpoint{0.917200in}{1.270655in}}%
\pgfpathcurveto{\pgfqpoint{0.923023in}{1.276479in}}{\pgfqpoint{0.926296in}{1.284379in}}{\pgfqpoint{0.926296in}{1.292616in}}%
\pgfpathcurveto{\pgfqpoint{0.926296in}{1.300852in}}{\pgfqpoint{0.923023in}{1.308752in}}{\pgfqpoint{0.917200in}{1.314576in}}%
\pgfpathcurveto{\pgfqpoint{0.911376in}{1.320400in}}{\pgfqpoint{0.903476in}{1.323672in}}{\pgfqpoint{0.895239in}{1.323672in}}%
\pgfpathcurveto{\pgfqpoint{0.887003in}{1.323672in}}{\pgfqpoint{0.879103in}{1.320400in}}{\pgfqpoint{0.873279in}{1.314576in}}%
\pgfpathcurveto{\pgfqpoint{0.867455in}{1.308752in}}{\pgfqpoint{0.864183in}{1.300852in}}{\pgfqpoint{0.864183in}{1.292616in}}%
\pgfpathcurveto{\pgfqpoint{0.864183in}{1.284379in}}{\pgfqpoint{0.867455in}{1.276479in}}{\pgfqpoint{0.873279in}{1.270655in}}%
\pgfpathcurveto{\pgfqpoint{0.879103in}{1.264831in}}{\pgfqpoint{0.887003in}{1.261559in}}{\pgfqpoint{0.895239in}{1.261559in}}%
\pgfpathclose%
\pgfusepath{stroke,fill}%
\end{pgfscope}%
\begin{pgfscope}%
\pgfpathrectangle{\pgfqpoint{0.100000in}{0.212622in}}{\pgfqpoint{3.696000in}{3.696000in}}%
\pgfusepath{clip}%
\pgfsetbuttcap%
\pgfsetroundjoin%
\definecolor{currentfill}{rgb}{0.121569,0.466667,0.705882}%
\pgfsetfillcolor{currentfill}%
\pgfsetfillopacity{0.632530}%
\pgfsetlinewidth{1.003750pt}%
\definecolor{currentstroke}{rgb}{0.121569,0.466667,0.705882}%
\pgfsetstrokecolor{currentstroke}%
\pgfsetstrokeopacity{0.632530}%
\pgfsetdash{}{0pt}%
\pgfpathmoveto{\pgfqpoint{0.895239in}{1.261559in}}%
\pgfpathcurveto{\pgfqpoint{0.903475in}{1.261559in}}{\pgfqpoint{0.911376in}{1.264831in}}{\pgfqpoint{0.917199in}{1.270655in}}%
\pgfpathcurveto{\pgfqpoint{0.923023in}{1.276479in}}{\pgfqpoint{0.926296in}{1.284379in}}{\pgfqpoint{0.926296in}{1.292616in}}%
\pgfpathcurveto{\pgfqpoint{0.926296in}{1.300852in}}{\pgfqpoint{0.923023in}{1.308752in}}{\pgfqpoint{0.917199in}{1.314576in}}%
\pgfpathcurveto{\pgfqpoint{0.911376in}{1.320400in}}{\pgfqpoint{0.903475in}{1.323672in}}{\pgfqpoint{0.895239in}{1.323672in}}%
\pgfpathcurveto{\pgfqpoint{0.887003in}{1.323672in}}{\pgfqpoint{0.879103in}{1.320400in}}{\pgfqpoint{0.873279in}{1.314576in}}%
\pgfpathcurveto{\pgfqpoint{0.867455in}{1.308752in}}{\pgfqpoint{0.864183in}{1.300852in}}{\pgfqpoint{0.864183in}{1.292616in}}%
\pgfpathcurveto{\pgfqpoint{0.864183in}{1.284379in}}{\pgfqpoint{0.867455in}{1.276479in}}{\pgfqpoint{0.873279in}{1.270655in}}%
\pgfpathcurveto{\pgfqpoint{0.879103in}{1.264831in}}{\pgfqpoint{0.887003in}{1.261559in}}{\pgfqpoint{0.895239in}{1.261559in}}%
\pgfpathclose%
\pgfusepath{stroke,fill}%
\end{pgfscope}%
\begin{pgfscope}%
\pgfpathrectangle{\pgfqpoint{0.100000in}{0.212622in}}{\pgfqpoint{3.696000in}{3.696000in}}%
\pgfusepath{clip}%
\pgfsetbuttcap%
\pgfsetroundjoin%
\definecolor{currentfill}{rgb}{0.121569,0.466667,0.705882}%
\pgfsetfillcolor{currentfill}%
\pgfsetfillopacity{0.632530}%
\pgfsetlinewidth{1.003750pt}%
\definecolor{currentstroke}{rgb}{0.121569,0.466667,0.705882}%
\pgfsetstrokecolor{currentstroke}%
\pgfsetstrokeopacity{0.632530}%
\pgfsetdash{}{0pt}%
\pgfpathmoveto{\pgfqpoint{0.895239in}{1.261559in}}%
\pgfpathcurveto{\pgfqpoint{0.903475in}{1.261559in}}{\pgfqpoint{0.911375in}{1.264831in}}{\pgfqpoint{0.917199in}{1.270655in}}%
\pgfpathcurveto{\pgfqpoint{0.923023in}{1.276479in}}{\pgfqpoint{0.926296in}{1.284379in}}{\pgfqpoint{0.926296in}{1.292616in}}%
\pgfpathcurveto{\pgfqpoint{0.926296in}{1.300852in}}{\pgfqpoint{0.923023in}{1.308752in}}{\pgfqpoint{0.917199in}{1.314576in}}%
\pgfpathcurveto{\pgfqpoint{0.911375in}{1.320400in}}{\pgfqpoint{0.903475in}{1.323672in}}{\pgfqpoint{0.895239in}{1.323672in}}%
\pgfpathcurveto{\pgfqpoint{0.887003in}{1.323672in}}{\pgfqpoint{0.879103in}{1.320400in}}{\pgfqpoint{0.873279in}{1.314576in}}%
\pgfpathcurveto{\pgfqpoint{0.867455in}{1.308752in}}{\pgfqpoint{0.864183in}{1.300852in}}{\pgfqpoint{0.864183in}{1.292616in}}%
\pgfpathcurveto{\pgfqpoint{0.864183in}{1.284379in}}{\pgfqpoint{0.867455in}{1.276479in}}{\pgfqpoint{0.873279in}{1.270655in}}%
\pgfpathcurveto{\pgfqpoint{0.879103in}{1.264831in}}{\pgfqpoint{0.887003in}{1.261559in}}{\pgfqpoint{0.895239in}{1.261559in}}%
\pgfpathclose%
\pgfusepath{stroke,fill}%
\end{pgfscope}%
\begin{pgfscope}%
\pgfpathrectangle{\pgfqpoint{0.100000in}{0.212622in}}{\pgfqpoint{3.696000in}{3.696000in}}%
\pgfusepath{clip}%
\pgfsetbuttcap%
\pgfsetroundjoin%
\definecolor{currentfill}{rgb}{0.121569,0.466667,0.705882}%
\pgfsetfillcolor{currentfill}%
\pgfsetfillopacity{0.632530}%
\pgfsetlinewidth{1.003750pt}%
\definecolor{currentstroke}{rgb}{0.121569,0.466667,0.705882}%
\pgfsetstrokecolor{currentstroke}%
\pgfsetstrokeopacity{0.632530}%
\pgfsetdash{}{0pt}%
\pgfpathmoveto{\pgfqpoint{0.895239in}{1.261559in}}%
\pgfpathcurveto{\pgfqpoint{0.903475in}{1.261559in}}{\pgfqpoint{0.911375in}{1.264831in}}{\pgfqpoint{0.917199in}{1.270655in}}%
\pgfpathcurveto{\pgfqpoint{0.923023in}{1.276479in}}{\pgfqpoint{0.926296in}{1.284379in}}{\pgfqpoint{0.926296in}{1.292616in}}%
\pgfpathcurveto{\pgfqpoint{0.926296in}{1.300852in}}{\pgfqpoint{0.923023in}{1.308752in}}{\pgfqpoint{0.917199in}{1.314576in}}%
\pgfpathcurveto{\pgfqpoint{0.911375in}{1.320400in}}{\pgfqpoint{0.903475in}{1.323672in}}{\pgfqpoint{0.895239in}{1.323672in}}%
\pgfpathcurveto{\pgfqpoint{0.887003in}{1.323672in}}{\pgfqpoint{0.879103in}{1.320400in}}{\pgfqpoint{0.873279in}{1.314576in}}%
\pgfpathcurveto{\pgfqpoint{0.867455in}{1.308752in}}{\pgfqpoint{0.864183in}{1.300852in}}{\pgfqpoint{0.864183in}{1.292616in}}%
\pgfpathcurveto{\pgfqpoint{0.864183in}{1.284379in}}{\pgfqpoint{0.867455in}{1.276479in}}{\pgfqpoint{0.873279in}{1.270655in}}%
\pgfpathcurveto{\pgfqpoint{0.879103in}{1.264831in}}{\pgfqpoint{0.887003in}{1.261559in}}{\pgfqpoint{0.895239in}{1.261559in}}%
\pgfpathclose%
\pgfusepath{stroke,fill}%
\end{pgfscope}%
\begin{pgfscope}%
\pgfpathrectangle{\pgfqpoint{0.100000in}{0.212622in}}{\pgfqpoint{3.696000in}{3.696000in}}%
\pgfusepath{clip}%
\pgfsetbuttcap%
\pgfsetroundjoin%
\definecolor{currentfill}{rgb}{0.121569,0.466667,0.705882}%
\pgfsetfillcolor{currentfill}%
\pgfsetfillopacity{0.632530}%
\pgfsetlinewidth{1.003750pt}%
\definecolor{currentstroke}{rgb}{0.121569,0.466667,0.705882}%
\pgfsetstrokecolor{currentstroke}%
\pgfsetstrokeopacity{0.632530}%
\pgfsetdash{}{0pt}%
\pgfpathmoveto{\pgfqpoint{0.895239in}{1.261559in}}%
\pgfpathcurveto{\pgfqpoint{0.903475in}{1.261559in}}{\pgfqpoint{0.911375in}{1.264831in}}{\pgfqpoint{0.917199in}{1.270655in}}%
\pgfpathcurveto{\pgfqpoint{0.923023in}{1.276479in}}{\pgfqpoint{0.926296in}{1.284379in}}{\pgfqpoint{0.926296in}{1.292616in}}%
\pgfpathcurveto{\pgfqpoint{0.926296in}{1.300852in}}{\pgfqpoint{0.923023in}{1.308752in}}{\pgfqpoint{0.917199in}{1.314576in}}%
\pgfpathcurveto{\pgfqpoint{0.911375in}{1.320400in}}{\pgfqpoint{0.903475in}{1.323672in}}{\pgfqpoint{0.895239in}{1.323672in}}%
\pgfpathcurveto{\pgfqpoint{0.887003in}{1.323672in}}{\pgfqpoint{0.879103in}{1.320400in}}{\pgfqpoint{0.873279in}{1.314576in}}%
\pgfpathcurveto{\pgfqpoint{0.867455in}{1.308752in}}{\pgfqpoint{0.864183in}{1.300852in}}{\pgfqpoint{0.864183in}{1.292616in}}%
\pgfpathcurveto{\pgfqpoint{0.864183in}{1.284379in}}{\pgfqpoint{0.867455in}{1.276479in}}{\pgfqpoint{0.873279in}{1.270655in}}%
\pgfpathcurveto{\pgfqpoint{0.879103in}{1.264831in}}{\pgfqpoint{0.887003in}{1.261559in}}{\pgfqpoint{0.895239in}{1.261559in}}%
\pgfpathclose%
\pgfusepath{stroke,fill}%
\end{pgfscope}%
\begin{pgfscope}%
\pgfpathrectangle{\pgfqpoint{0.100000in}{0.212622in}}{\pgfqpoint{3.696000in}{3.696000in}}%
\pgfusepath{clip}%
\pgfsetbuttcap%
\pgfsetroundjoin%
\definecolor{currentfill}{rgb}{0.121569,0.466667,0.705882}%
\pgfsetfillcolor{currentfill}%
\pgfsetfillopacity{0.632530}%
\pgfsetlinewidth{1.003750pt}%
\definecolor{currentstroke}{rgb}{0.121569,0.466667,0.705882}%
\pgfsetstrokecolor{currentstroke}%
\pgfsetstrokeopacity{0.632530}%
\pgfsetdash{}{0pt}%
\pgfpathmoveto{\pgfqpoint{0.895239in}{1.261559in}}%
\pgfpathcurveto{\pgfqpoint{0.903475in}{1.261559in}}{\pgfqpoint{0.911375in}{1.264831in}}{\pgfqpoint{0.917199in}{1.270655in}}%
\pgfpathcurveto{\pgfqpoint{0.923023in}{1.276479in}}{\pgfqpoint{0.926296in}{1.284379in}}{\pgfqpoint{0.926296in}{1.292616in}}%
\pgfpathcurveto{\pgfqpoint{0.926296in}{1.300852in}}{\pgfqpoint{0.923023in}{1.308752in}}{\pgfqpoint{0.917199in}{1.314576in}}%
\pgfpathcurveto{\pgfqpoint{0.911375in}{1.320400in}}{\pgfqpoint{0.903475in}{1.323672in}}{\pgfqpoint{0.895239in}{1.323672in}}%
\pgfpathcurveto{\pgfqpoint{0.887003in}{1.323672in}}{\pgfqpoint{0.879103in}{1.320400in}}{\pgfqpoint{0.873279in}{1.314576in}}%
\pgfpathcurveto{\pgfqpoint{0.867455in}{1.308752in}}{\pgfqpoint{0.864183in}{1.300852in}}{\pgfqpoint{0.864183in}{1.292616in}}%
\pgfpathcurveto{\pgfqpoint{0.864183in}{1.284379in}}{\pgfqpoint{0.867455in}{1.276479in}}{\pgfqpoint{0.873279in}{1.270655in}}%
\pgfpathcurveto{\pgfqpoint{0.879103in}{1.264831in}}{\pgfqpoint{0.887003in}{1.261559in}}{\pgfqpoint{0.895239in}{1.261559in}}%
\pgfpathclose%
\pgfusepath{stroke,fill}%
\end{pgfscope}%
\begin{pgfscope}%
\pgfpathrectangle{\pgfqpoint{0.100000in}{0.212622in}}{\pgfqpoint{3.696000in}{3.696000in}}%
\pgfusepath{clip}%
\pgfsetbuttcap%
\pgfsetroundjoin%
\definecolor{currentfill}{rgb}{0.121569,0.466667,0.705882}%
\pgfsetfillcolor{currentfill}%
\pgfsetfillopacity{0.632530}%
\pgfsetlinewidth{1.003750pt}%
\definecolor{currentstroke}{rgb}{0.121569,0.466667,0.705882}%
\pgfsetstrokecolor{currentstroke}%
\pgfsetstrokeopacity{0.632530}%
\pgfsetdash{}{0pt}%
\pgfpathmoveto{\pgfqpoint{0.895239in}{1.261559in}}%
\pgfpathcurveto{\pgfqpoint{0.903475in}{1.261559in}}{\pgfqpoint{0.911375in}{1.264831in}}{\pgfqpoint{0.917199in}{1.270655in}}%
\pgfpathcurveto{\pgfqpoint{0.923023in}{1.276479in}}{\pgfqpoint{0.926296in}{1.284379in}}{\pgfqpoint{0.926296in}{1.292616in}}%
\pgfpathcurveto{\pgfqpoint{0.926296in}{1.300852in}}{\pgfqpoint{0.923023in}{1.308752in}}{\pgfqpoint{0.917199in}{1.314576in}}%
\pgfpathcurveto{\pgfqpoint{0.911375in}{1.320400in}}{\pgfqpoint{0.903475in}{1.323672in}}{\pgfqpoint{0.895239in}{1.323672in}}%
\pgfpathcurveto{\pgfqpoint{0.887003in}{1.323672in}}{\pgfqpoint{0.879103in}{1.320400in}}{\pgfqpoint{0.873279in}{1.314576in}}%
\pgfpathcurveto{\pgfqpoint{0.867455in}{1.308752in}}{\pgfqpoint{0.864183in}{1.300852in}}{\pgfqpoint{0.864183in}{1.292616in}}%
\pgfpathcurveto{\pgfqpoint{0.864183in}{1.284379in}}{\pgfqpoint{0.867455in}{1.276479in}}{\pgfqpoint{0.873279in}{1.270655in}}%
\pgfpathcurveto{\pgfqpoint{0.879103in}{1.264831in}}{\pgfqpoint{0.887003in}{1.261559in}}{\pgfqpoint{0.895239in}{1.261559in}}%
\pgfpathclose%
\pgfusepath{stroke,fill}%
\end{pgfscope}%
\begin{pgfscope}%
\pgfpathrectangle{\pgfqpoint{0.100000in}{0.212622in}}{\pgfqpoint{3.696000in}{3.696000in}}%
\pgfusepath{clip}%
\pgfsetbuttcap%
\pgfsetroundjoin%
\definecolor{currentfill}{rgb}{0.121569,0.466667,0.705882}%
\pgfsetfillcolor{currentfill}%
\pgfsetfillopacity{0.632530}%
\pgfsetlinewidth{1.003750pt}%
\definecolor{currentstroke}{rgb}{0.121569,0.466667,0.705882}%
\pgfsetstrokecolor{currentstroke}%
\pgfsetstrokeopacity{0.632530}%
\pgfsetdash{}{0pt}%
\pgfpathmoveto{\pgfqpoint{0.895239in}{1.261559in}}%
\pgfpathcurveto{\pgfqpoint{0.903475in}{1.261559in}}{\pgfqpoint{0.911375in}{1.264831in}}{\pgfqpoint{0.917199in}{1.270655in}}%
\pgfpathcurveto{\pgfqpoint{0.923023in}{1.276479in}}{\pgfqpoint{0.926296in}{1.284379in}}{\pgfqpoint{0.926296in}{1.292616in}}%
\pgfpathcurveto{\pgfqpoint{0.926296in}{1.300852in}}{\pgfqpoint{0.923023in}{1.308752in}}{\pgfqpoint{0.917199in}{1.314576in}}%
\pgfpathcurveto{\pgfqpoint{0.911375in}{1.320400in}}{\pgfqpoint{0.903475in}{1.323672in}}{\pgfqpoint{0.895239in}{1.323672in}}%
\pgfpathcurveto{\pgfqpoint{0.887003in}{1.323672in}}{\pgfqpoint{0.879103in}{1.320400in}}{\pgfqpoint{0.873279in}{1.314576in}}%
\pgfpathcurveto{\pgfqpoint{0.867455in}{1.308752in}}{\pgfqpoint{0.864183in}{1.300852in}}{\pgfqpoint{0.864183in}{1.292616in}}%
\pgfpathcurveto{\pgfqpoint{0.864183in}{1.284379in}}{\pgfqpoint{0.867455in}{1.276479in}}{\pgfqpoint{0.873279in}{1.270655in}}%
\pgfpathcurveto{\pgfqpoint{0.879103in}{1.264831in}}{\pgfqpoint{0.887003in}{1.261559in}}{\pgfqpoint{0.895239in}{1.261559in}}%
\pgfpathclose%
\pgfusepath{stroke,fill}%
\end{pgfscope}%
\begin{pgfscope}%
\pgfpathrectangle{\pgfqpoint{0.100000in}{0.212622in}}{\pgfqpoint{3.696000in}{3.696000in}}%
\pgfusepath{clip}%
\pgfsetbuttcap%
\pgfsetroundjoin%
\definecolor{currentfill}{rgb}{0.121569,0.466667,0.705882}%
\pgfsetfillcolor{currentfill}%
\pgfsetfillopacity{0.632530}%
\pgfsetlinewidth{1.003750pt}%
\definecolor{currentstroke}{rgb}{0.121569,0.466667,0.705882}%
\pgfsetstrokecolor{currentstroke}%
\pgfsetstrokeopacity{0.632530}%
\pgfsetdash{}{0pt}%
\pgfpathmoveto{\pgfqpoint{0.895239in}{1.261559in}}%
\pgfpathcurveto{\pgfqpoint{0.903475in}{1.261559in}}{\pgfqpoint{0.911375in}{1.264831in}}{\pgfqpoint{0.917199in}{1.270655in}}%
\pgfpathcurveto{\pgfqpoint{0.923023in}{1.276479in}}{\pgfqpoint{0.926296in}{1.284379in}}{\pgfqpoint{0.926296in}{1.292616in}}%
\pgfpathcurveto{\pgfqpoint{0.926296in}{1.300852in}}{\pgfqpoint{0.923023in}{1.308752in}}{\pgfqpoint{0.917199in}{1.314576in}}%
\pgfpathcurveto{\pgfqpoint{0.911375in}{1.320400in}}{\pgfqpoint{0.903475in}{1.323672in}}{\pgfqpoint{0.895239in}{1.323672in}}%
\pgfpathcurveto{\pgfqpoint{0.887003in}{1.323672in}}{\pgfqpoint{0.879103in}{1.320400in}}{\pgfqpoint{0.873279in}{1.314576in}}%
\pgfpathcurveto{\pgfqpoint{0.867455in}{1.308752in}}{\pgfqpoint{0.864183in}{1.300852in}}{\pgfqpoint{0.864183in}{1.292616in}}%
\pgfpathcurveto{\pgfqpoint{0.864183in}{1.284379in}}{\pgfqpoint{0.867455in}{1.276479in}}{\pgfqpoint{0.873279in}{1.270655in}}%
\pgfpathcurveto{\pgfqpoint{0.879103in}{1.264831in}}{\pgfqpoint{0.887003in}{1.261559in}}{\pgfqpoint{0.895239in}{1.261559in}}%
\pgfpathclose%
\pgfusepath{stroke,fill}%
\end{pgfscope}%
\begin{pgfscope}%
\pgfpathrectangle{\pgfqpoint{0.100000in}{0.212622in}}{\pgfqpoint{3.696000in}{3.696000in}}%
\pgfusepath{clip}%
\pgfsetbuttcap%
\pgfsetroundjoin%
\definecolor{currentfill}{rgb}{0.121569,0.466667,0.705882}%
\pgfsetfillcolor{currentfill}%
\pgfsetfillopacity{0.632530}%
\pgfsetlinewidth{1.003750pt}%
\definecolor{currentstroke}{rgb}{0.121569,0.466667,0.705882}%
\pgfsetstrokecolor{currentstroke}%
\pgfsetstrokeopacity{0.632530}%
\pgfsetdash{}{0pt}%
\pgfpathmoveto{\pgfqpoint{0.895239in}{1.261559in}}%
\pgfpathcurveto{\pgfqpoint{0.903475in}{1.261559in}}{\pgfqpoint{0.911375in}{1.264831in}}{\pgfqpoint{0.917199in}{1.270655in}}%
\pgfpathcurveto{\pgfqpoint{0.923023in}{1.276479in}}{\pgfqpoint{0.926296in}{1.284379in}}{\pgfqpoint{0.926296in}{1.292616in}}%
\pgfpathcurveto{\pgfqpoint{0.926296in}{1.300852in}}{\pgfqpoint{0.923023in}{1.308752in}}{\pgfqpoint{0.917199in}{1.314576in}}%
\pgfpathcurveto{\pgfqpoint{0.911375in}{1.320400in}}{\pgfqpoint{0.903475in}{1.323672in}}{\pgfqpoint{0.895239in}{1.323672in}}%
\pgfpathcurveto{\pgfqpoint{0.887003in}{1.323672in}}{\pgfqpoint{0.879103in}{1.320400in}}{\pgfqpoint{0.873279in}{1.314576in}}%
\pgfpathcurveto{\pgfqpoint{0.867455in}{1.308752in}}{\pgfqpoint{0.864183in}{1.300852in}}{\pgfqpoint{0.864183in}{1.292616in}}%
\pgfpathcurveto{\pgfqpoint{0.864183in}{1.284379in}}{\pgfqpoint{0.867455in}{1.276479in}}{\pgfqpoint{0.873279in}{1.270655in}}%
\pgfpathcurveto{\pgfqpoint{0.879103in}{1.264831in}}{\pgfqpoint{0.887003in}{1.261559in}}{\pgfqpoint{0.895239in}{1.261559in}}%
\pgfpathclose%
\pgfusepath{stroke,fill}%
\end{pgfscope}%
\begin{pgfscope}%
\pgfpathrectangle{\pgfqpoint{0.100000in}{0.212622in}}{\pgfqpoint{3.696000in}{3.696000in}}%
\pgfusepath{clip}%
\pgfsetbuttcap%
\pgfsetroundjoin%
\definecolor{currentfill}{rgb}{0.121569,0.466667,0.705882}%
\pgfsetfillcolor{currentfill}%
\pgfsetfillopacity{0.632530}%
\pgfsetlinewidth{1.003750pt}%
\definecolor{currentstroke}{rgb}{0.121569,0.466667,0.705882}%
\pgfsetstrokecolor{currentstroke}%
\pgfsetstrokeopacity{0.632530}%
\pgfsetdash{}{0pt}%
\pgfpathmoveto{\pgfqpoint{0.895239in}{1.261559in}}%
\pgfpathcurveto{\pgfqpoint{0.903475in}{1.261559in}}{\pgfqpoint{0.911375in}{1.264831in}}{\pgfqpoint{0.917199in}{1.270655in}}%
\pgfpathcurveto{\pgfqpoint{0.923023in}{1.276479in}}{\pgfqpoint{0.926296in}{1.284379in}}{\pgfqpoint{0.926296in}{1.292616in}}%
\pgfpathcurveto{\pgfqpoint{0.926296in}{1.300852in}}{\pgfqpoint{0.923023in}{1.308752in}}{\pgfqpoint{0.917199in}{1.314576in}}%
\pgfpathcurveto{\pgfqpoint{0.911375in}{1.320400in}}{\pgfqpoint{0.903475in}{1.323672in}}{\pgfqpoint{0.895239in}{1.323672in}}%
\pgfpathcurveto{\pgfqpoint{0.887003in}{1.323672in}}{\pgfqpoint{0.879103in}{1.320400in}}{\pgfqpoint{0.873279in}{1.314576in}}%
\pgfpathcurveto{\pgfqpoint{0.867455in}{1.308752in}}{\pgfqpoint{0.864183in}{1.300852in}}{\pgfqpoint{0.864183in}{1.292616in}}%
\pgfpathcurveto{\pgfqpoint{0.864183in}{1.284379in}}{\pgfqpoint{0.867455in}{1.276479in}}{\pgfqpoint{0.873279in}{1.270655in}}%
\pgfpathcurveto{\pgfqpoint{0.879103in}{1.264831in}}{\pgfqpoint{0.887003in}{1.261559in}}{\pgfqpoint{0.895239in}{1.261559in}}%
\pgfpathclose%
\pgfusepath{stroke,fill}%
\end{pgfscope}%
\begin{pgfscope}%
\pgfpathrectangle{\pgfqpoint{0.100000in}{0.212622in}}{\pgfqpoint{3.696000in}{3.696000in}}%
\pgfusepath{clip}%
\pgfsetbuttcap%
\pgfsetroundjoin%
\definecolor{currentfill}{rgb}{0.121569,0.466667,0.705882}%
\pgfsetfillcolor{currentfill}%
\pgfsetfillopacity{0.632530}%
\pgfsetlinewidth{1.003750pt}%
\definecolor{currentstroke}{rgb}{0.121569,0.466667,0.705882}%
\pgfsetstrokecolor{currentstroke}%
\pgfsetstrokeopacity{0.632530}%
\pgfsetdash{}{0pt}%
\pgfpathmoveto{\pgfqpoint{0.895239in}{1.261559in}}%
\pgfpathcurveto{\pgfqpoint{0.903475in}{1.261559in}}{\pgfqpoint{0.911375in}{1.264831in}}{\pgfqpoint{0.917199in}{1.270655in}}%
\pgfpathcurveto{\pgfqpoint{0.923023in}{1.276479in}}{\pgfqpoint{0.926296in}{1.284379in}}{\pgfqpoint{0.926296in}{1.292616in}}%
\pgfpathcurveto{\pgfqpoint{0.926296in}{1.300852in}}{\pgfqpoint{0.923023in}{1.308752in}}{\pgfqpoint{0.917199in}{1.314576in}}%
\pgfpathcurveto{\pgfqpoint{0.911375in}{1.320400in}}{\pgfqpoint{0.903475in}{1.323672in}}{\pgfqpoint{0.895239in}{1.323672in}}%
\pgfpathcurveto{\pgfqpoint{0.887003in}{1.323672in}}{\pgfqpoint{0.879103in}{1.320400in}}{\pgfqpoint{0.873279in}{1.314576in}}%
\pgfpathcurveto{\pgfqpoint{0.867455in}{1.308752in}}{\pgfqpoint{0.864183in}{1.300852in}}{\pgfqpoint{0.864183in}{1.292616in}}%
\pgfpathcurveto{\pgfqpoint{0.864183in}{1.284379in}}{\pgfqpoint{0.867455in}{1.276479in}}{\pgfqpoint{0.873279in}{1.270655in}}%
\pgfpathcurveto{\pgfqpoint{0.879103in}{1.264831in}}{\pgfqpoint{0.887003in}{1.261559in}}{\pgfqpoint{0.895239in}{1.261559in}}%
\pgfpathclose%
\pgfusepath{stroke,fill}%
\end{pgfscope}%
\begin{pgfscope}%
\pgfpathrectangle{\pgfqpoint{0.100000in}{0.212622in}}{\pgfqpoint{3.696000in}{3.696000in}}%
\pgfusepath{clip}%
\pgfsetbuttcap%
\pgfsetroundjoin%
\definecolor{currentfill}{rgb}{0.121569,0.466667,0.705882}%
\pgfsetfillcolor{currentfill}%
\pgfsetfillopacity{0.632530}%
\pgfsetlinewidth{1.003750pt}%
\definecolor{currentstroke}{rgb}{0.121569,0.466667,0.705882}%
\pgfsetstrokecolor{currentstroke}%
\pgfsetstrokeopacity{0.632530}%
\pgfsetdash{}{0pt}%
\pgfpathmoveto{\pgfqpoint{0.895239in}{1.261559in}}%
\pgfpathcurveto{\pgfqpoint{0.903475in}{1.261559in}}{\pgfqpoint{0.911375in}{1.264831in}}{\pgfqpoint{0.917199in}{1.270655in}}%
\pgfpathcurveto{\pgfqpoint{0.923023in}{1.276479in}}{\pgfqpoint{0.926296in}{1.284379in}}{\pgfqpoint{0.926296in}{1.292616in}}%
\pgfpathcurveto{\pgfqpoint{0.926296in}{1.300852in}}{\pgfqpoint{0.923023in}{1.308752in}}{\pgfqpoint{0.917199in}{1.314576in}}%
\pgfpathcurveto{\pgfqpoint{0.911375in}{1.320400in}}{\pgfqpoint{0.903475in}{1.323672in}}{\pgfqpoint{0.895239in}{1.323672in}}%
\pgfpathcurveto{\pgfqpoint{0.887003in}{1.323672in}}{\pgfqpoint{0.879103in}{1.320400in}}{\pgfqpoint{0.873279in}{1.314576in}}%
\pgfpathcurveto{\pgfqpoint{0.867455in}{1.308752in}}{\pgfqpoint{0.864183in}{1.300852in}}{\pgfqpoint{0.864183in}{1.292616in}}%
\pgfpathcurveto{\pgfqpoint{0.864183in}{1.284379in}}{\pgfqpoint{0.867455in}{1.276479in}}{\pgfqpoint{0.873279in}{1.270655in}}%
\pgfpathcurveto{\pgfqpoint{0.879103in}{1.264831in}}{\pgfqpoint{0.887003in}{1.261559in}}{\pgfqpoint{0.895239in}{1.261559in}}%
\pgfpathclose%
\pgfusepath{stroke,fill}%
\end{pgfscope}%
\begin{pgfscope}%
\pgfpathrectangle{\pgfqpoint{0.100000in}{0.212622in}}{\pgfqpoint{3.696000in}{3.696000in}}%
\pgfusepath{clip}%
\pgfsetbuttcap%
\pgfsetroundjoin%
\definecolor{currentfill}{rgb}{0.121569,0.466667,0.705882}%
\pgfsetfillcolor{currentfill}%
\pgfsetfillopacity{0.632530}%
\pgfsetlinewidth{1.003750pt}%
\definecolor{currentstroke}{rgb}{0.121569,0.466667,0.705882}%
\pgfsetstrokecolor{currentstroke}%
\pgfsetstrokeopacity{0.632530}%
\pgfsetdash{}{0pt}%
\pgfpathmoveto{\pgfqpoint{0.895239in}{1.261559in}}%
\pgfpathcurveto{\pgfqpoint{0.903475in}{1.261559in}}{\pgfqpoint{0.911375in}{1.264831in}}{\pgfqpoint{0.917199in}{1.270655in}}%
\pgfpathcurveto{\pgfqpoint{0.923023in}{1.276479in}}{\pgfqpoint{0.926296in}{1.284379in}}{\pgfqpoint{0.926296in}{1.292616in}}%
\pgfpathcurveto{\pgfqpoint{0.926296in}{1.300852in}}{\pgfqpoint{0.923023in}{1.308752in}}{\pgfqpoint{0.917199in}{1.314576in}}%
\pgfpathcurveto{\pgfqpoint{0.911375in}{1.320400in}}{\pgfqpoint{0.903475in}{1.323672in}}{\pgfqpoint{0.895239in}{1.323672in}}%
\pgfpathcurveto{\pgfqpoint{0.887003in}{1.323672in}}{\pgfqpoint{0.879103in}{1.320400in}}{\pgfqpoint{0.873279in}{1.314576in}}%
\pgfpathcurveto{\pgfqpoint{0.867455in}{1.308752in}}{\pgfqpoint{0.864183in}{1.300852in}}{\pgfqpoint{0.864183in}{1.292616in}}%
\pgfpathcurveto{\pgfqpoint{0.864183in}{1.284379in}}{\pgfqpoint{0.867455in}{1.276479in}}{\pgfqpoint{0.873279in}{1.270655in}}%
\pgfpathcurveto{\pgfqpoint{0.879103in}{1.264831in}}{\pgfqpoint{0.887003in}{1.261559in}}{\pgfqpoint{0.895239in}{1.261559in}}%
\pgfpathclose%
\pgfusepath{stroke,fill}%
\end{pgfscope}%
\begin{pgfscope}%
\pgfpathrectangle{\pgfqpoint{0.100000in}{0.212622in}}{\pgfqpoint{3.696000in}{3.696000in}}%
\pgfusepath{clip}%
\pgfsetbuttcap%
\pgfsetroundjoin%
\definecolor{currentfill}{rgb}{0.121569,0.466667,0.705882}%
\pgfsetfillcolor{currentfill}%
\pgfsetfillopacity{0.632530}%
\pgfsetlinewidth{1.003750pt}%
\definecolor{currentstroke}{rgb}{0.121569,0.466667,0.705882}%
\pgfsetstrokecolor{currentstroke}%
\pgfsetstrokeopacity{0.632530}%
\pgfsetdash{}{0pt}%
\pgfpathmoveto{\pgfqpoint{0.895239in}{1.261559in}}%
\pgfpathcurveto{\pgfqpoint{0.903475in}{1.261559in}}{\pgfqpoint{0.911375in}{1.264831in}}{\pgfqpoint{0.917199in}{1.270655in}}%
\pgfpathcurveto{\pgfqpoint{0.923023in}{1.276479in}}{\pgfqpoint{0.926296in}{1.284379in}}{\pgfqpoint{0.926296in}{1.292616in}}%
\pgfpathcurveto{\pgfqpoint{0.926296in}{1.300852in}}{\pgfqpoint{0.923023in}{1.308752in}}{\pgfqpoint{0.917199in}{1.314576in}}%
\pgfpathcurveto{\pgfqpoint{0.911375in}{1.320400in}}{\pgfqpoint{0.903475in}{1.323672in}}{\pgfqpoint{0.895239in}{1.323672in}}%
\pgfpathcurveto{\pgfqpoint{0.887003in}{1.323672in}}{\pgfqpoint{0.879103in}{1.320400in}}{\pgfqpoint{0.873279in}{1.314576in}}%
\pgfpathcurveto{\pgfqpoint{0.867455in}{1.308752in}}{\pgfqpoint{0.864183in}{1.300852in}}{\pgfqpoint{0.864183in}{1.292616in}}%
\pgfpathcurveto{\pgfqpoint{0.864183in}{1.284379in}}{\pgfqpoint{0.867455in}{1.276479in}}{\pgfqpoint{0.873279in}{1.270655in}}%
\pgfpathcurveto{\pgfqpoint{0.879103in}{1.264831in}}{\pgfqpoint{0.887003in}{1.261559in}}{\pgfqpoint{0.895239in}{1.261559in}}%
\pgfpathclose%
\pgfusepath{stroke,fill}%
\end{pgfscope}%
\begin{pgfscope}%
\pgfpathrectangle{\pgfqpoint{0.100000in}{0.212622in}}{\pgfqpoint{3.696000in}{3.696000in}}%
\pgfusepath{clip}%
\pgfsetbuttcap%
\pgfsetroundjoin%
\definecolor{currentfill}{rgb}{0.121569,0.466667,0.705882}%
\pgfsetfillcolor{currentfill}%
\pgfsetfillopacity{0.632530}%
\pgfsetlinewidth{1.003750pt}%
\definecolor{currentstroke}{rgb}{0.121569,0.466667,0.705882}%
\pgfsetstrokecolor{currentstroke}%
\pgfsetstrokeopacity{0.632530}%
\pgfsetdash{}{0pt}%
\pgfpathmoveto{\pgfqpoint{0.895239in}{1.261559in}}%
\pgfpathcurveto{\pgfqpoint{0.903475in}{1.261559in}}{\pgfqpoint{0.911375in}{1.264831in}}{\pgfqpoint{0.917199in}{1.270655in}}%
\pgfpathcurveto{\pgfqpoint{0.923023in}{1.276479in}}{\pgfqpoint{0.926296in}{1.284379in}}{\pgfqpoint{0.926296in}{1.292616in}}%
\pgfpathcurveto{\pgfqpoint{0.926296in}{1.300852in}}{\pgfqpoint{0.923023in}{1.308752in}}{\pgfqpoint{0.917199in}{1.314576in}}%
\pgfpathcurveto{\pgfqpoint{0.911375in}{1.320400in}}{\pgfqpoint{0.903475in}{1.323672in}}{\pgfqpoint{0.895239in}{1.323672in}}%
\pgfpathcurveto{\pgfqpoint{0.887003in}{1.323672in}}{\pgfqpoint{0.879103in}{1.320400in}}{\pgfqpoint{0.873279in}{1.314576in}}%
\pgfpathcurveto{\pgfqpoint{0.867455in}{1.308752in}}{\pgfqpoint{0.864183in}{1.300852in}}{\pgfqpoint{0.864183in}{1.292616in}}%
\pgfpathcurveto{\pgfqpoint{0.864183in}{1.284379in}}{\pgfqpoint{0.867455in}{1.276479in}}{\pgfqpoint{0.873279in}{1.270655in}}%
\pgfpathcurveto{\pgfqpoint{0.879103in}{1.264831in}}{\pgfqpoint{0.887003in}{1.261559in}}{\pgfqpoint{0.895239in}{1.261559in}}%
\pgfpathclose%
\pgfusepath{stroke,fill}%
\end{pgfscope}%
\begin{pgfscope}%
\pgfpathrectangle{\pgfqpoint{0.100000in}{0.212622in}}{\pgfqpoint{3.696000in}{3.696000in}}%
\pgfusepath{clip}%
\pgfsetbuttcap%
\pgfsetroundjoin%
\definecolor{currentfill}{rgb}{0.121569,0.466667,0.705882}%
\pgfsetfillcolor{currentfill}%
\pgfsetfillopacity{0.632530}%
\pgfsetlinewidth{1.003750pt}%
\definecolor{currentstroke}{rgb}{0.121569,0.466667,0.705882}%
\pgfsetstrokecolor{currentstroke}%
\pgfsetstrokeopacity{0.632530}%
\pgfsetdash{}{0pt}%
\pgfpathmoveto{\pgfqpoint{0.895239in}{1.261559in}}%
\pgfpathcurveto{\pgfqpoint{0.903475in}{1.261559in}}{\pgfqpoint{0.911375in}{1.264831in}}{\pgfqpoint{0.917199in}{1.270655in}}%
\pgfpathcurveto{\pgfqpoint{0.923023in}{1.276479in}}{\pgfqpoint{0.926296in}{1.284379in}}{\pgfqpoint{0.926296in}{1.292616in}}%
\pgfpathcurveto{\pgfqpoint{0.926296in}{1.300852in}}{\pgfqpoint{0.923023in}{1.308752in}}{\pgfqpoint{0.917199in}{1.314576in}}%
\pgfpathcurveto{\pgfqpoint{0.911375in}{1.320400in}}{\pgfqpoint{0.903475in}{1.323672in}}{\pgfqpoint{0.895239in}{1.323672in}}%
\pgfpathcurveto{\pgfqpoint{0.887003in}{1.323672in}}{\pgfqpoint{0.879103in}{1.320400in}}{\pgfqpoint{0.873279in}{1.314576in}}%
\pgfpathcurveto{\pgfqpoint{0.867455in}{1.308752in}}{\pgfqpoint{0.864183in}{1.300852in}}{\pgfqpoint{0.864183in}{1.292616in}}%
\pgfpathcurveto{\pgfqpoint{0.864183in}{1.284379in}}{\pgfqpoint{0.867455in}{1.276479in}}{\pgfqpoint{0.873279in}{1.270655in}}%
\pgfpathcurveto{\pgfqpoint{0.879103in}{1.264831in}}{\pgfqpoint{0.887003in}{1.261559in}}{\pgfqpoint{0.895239in}{1.261559in}}%
\pgfpathclose%
\pgfusepath{stroke,fill}%
\end{pgfscope}%
\begin{pgfscope}%
\pgfpathrectangle{\pgfqpoint{0.100000in}{0.212622in}}{\pgfqpoint{3.696000in}{3.696000in}}%
\pgfusepath{clip}%
\pgfsetbuttcap%
\pgfsetroundjoin%
\definecolor{currentfill}{rgb}{0.121569,0.466667,0.705882}%
\pgfsetfillcolor{currentfill}%
\pgfsetfillopacity{0.632530}%
\pgfsetlinewidth{1.003750pt}%
\definecolor{currentstroke}{rgb}{0.121569,0.466667,0.705882}%
\pgfsetstrokecolor{currentstroke}%
\pgfsetstrokeopacity{0.632530}%
\pgfsetdash{}{0pt}%
\pgfpathmoveto{\pgfqpoint{0.895239in}{1.261559in}}%
\pgfpathcurveto{\pgfqpoint{0.903475in}{1.261559in}}{\pgfqpoint{0.911375in}{1.264831in}}{\pgfqpoint{0.917199in}{1.270655in}}%
\pgfpathcurveto{\pgfqpoint{0.923023in}{1.276479in}}{\pgfqpoint{0.926296in}{1.284379in}}{\pgfqpoint{0.926296in}{1.292616in}}%
\pgfpathcurveto{\pgfqpoint{0.926296in}{1.300852in}}{\pgfqpoint{0.923023in}{1.308752in}}{\pgfqpoint{0.917199in}{1.314576in}}%
\pgfpathcurveto{\pgfqpoint{0.911375in}{1.320400in}}{\pgfqpoint{0.903475in}{1.323672in}}{\pgfqpoint{0.895239in}{1.323672in}}%
\pgfpathcurveto{\pgfqpoint{0.887003in}{1.323672in}}{\pgfqpoint{0.879103in}{1.320400in}}{\pgfqpoint{0.873279in}{1.314576in}}%
\pgfpathcurveto{\pgfqpoint{0.867455in}{1.308752in}}{\pgfqpoint{0.864183in}{1.300852in}}{\pgfqpoint{0.864183in}{1.292616in}}%
\pgfpathcurveto{\pgfqpoint{0.864183in}{1.284379in}}{\pgfqpoint{0.867455in}{1.276479in}}{\pgfqpoint{0.873279in}{1.270655in}}%
\pgfpathcurveto{\pgfqpoint{0.879103in}{1.264831in}}{\pgfqpoint{0.887003in}{1.261559in}}{\pgfqpoint{0.895239in}{1.261559in}}%
\pgfpathclose%
\pgfusepath{stroke,fill}%
\end{pgfscope}%
\begin{pgfscope}%
\pgfpathrectangle{\pgfqpoint{0.100000in}{0.212622in}}{\pgfqpoint{3.696000in}{3.696000in}}%
\pgfusepath{clip}%
\pgfsetbuttcap%
\pgfsetroundjoin%
\definecolor{currentfill}{rgb}{0.121569,0.466667,0.705882}%
\pgfsetfillcolor{currentfill}%
\pgfsetfillopacity{0.632530}%
\pgfsetlinewidth{1.003750pt}%
\definecolor{currentstroke}{rgb}{0.121569,0.466667,0.705882}%
\pgfsetstrokecolor{currentstroke}%
\pgfsetstrokeopacity{0.632530}%
\pgfsetdash{}{0pt}%
\pgfpathmoveto{\pgfqpoint{0.895239in}{1.261559in}}%
\pgfpathcurveto{\pgfqpoint{0.903475in}{1.261559in}}{\pgfqpoint{0.911375in}{1.264831in}}{\pgfqpoint{0.917199in}{1.270655in}}%
\pgfpathcurveto{\pgfqpoint{0.923023in}{1.276479in}}{\pgfqpoint{0.926296in}{1.284379in}}{\pgfqpoint{0.926296in}{1.292616in}}%
\pgfpathcurveto{\pgfqpoint{0.926296in}{1.300852in}}{\pgfqpoint{0.923023in}{1.308752in}}{\pgfqpoint{0.917199in}{1.314576in}}%
\pgfpathcurveto{\pgfqpoint{0.911375in}{1.320400in}}{\pgfqpoint{0.903475in}{1.323672in}}{\pgfqpoint{0.895239in}{1.323672in}}%
\pgfpathcurveto{\pgfqpoint{0.887003in}{1.323672in}}{\pgfqpoint{0.879103in}{1.320400in}}{\pgfqpoint{0.873279in}{1.314576in}}%
\pgfpathcurveto{\pgfqpoint{0.867455in}{1.308752in}}{\pgfqpoint{0.864183in}{1.300852in}}{\pgfqpoint{0.864183in}{1.292616in}}%
\pgfpathcurveto{\pgfqpoint{0.864183in}{1.284379in}}{\pgfqpoint{0.867455in}{1.276479in}}{\pgfqpoint{0.873279in}{1.270655in}}%
\pgfpathcurveto{\pgfqpoint{0.879103in}{1.264831in}}{\pgfqpoint{0.887003in}{1.261559in}}{\pgfqpoint{0.895239in}{1.261559in}}%
\pgfpathclose%
\pgfusepath{stroke,fill}%
\end{pgfscope}%
\begin{pgfscope}%
\pgfpathrectangle{\pgfqpoint{0.100000in}{0.212622in}}{\pgfqpoint{3.696000in}{3.696000in}}%
\pgfusepath{clip}%
\pgfsetbuttcap%
\pgfsetroundjoin%
\definecolor{currentfill}{rgb}{0.121569,0.466667,0.705882}%
\pgfsetfillcolor{currentfill}%
\pgfsetfillopacity{0.632530}%
\pgfsetlinewidth{1.003750pt}%
\definecolor{currentstroke}{rgb}{0.121569,0.466667,0.705882}%
\pgfsetstrokecolor{currentstroke}%
\pgfsetstrokeopacity{0.632530}%
\pgfsetdash{}{0pt}%
\pgfpathmoveto{\pgfqpoint{0.895239in}{1.261559in}}%
\pgfpathcurveto{\pgfqpoint{0.903475in}{1.261559in}}{\pgfqpoint{0.911375in}{1.264831in}}{\pgfqpoint{0.917199in}{1.270655in}}%
\pgfpathcurveto{\pgfqpoint{0.923023in}{1.276479in}}{\pgfqpoint{0.926296in}{1.284379in}}{\pgfqpoint{0.926296in}{1.292616in}}%
\pgfpathcurveto{\pgfqpoint{0.926296in}{1.300852in}}{\pgfqpoint{0.923023in}{1.308752in}}{\pgfqpoint{0.917199in}{1.314576in}}%
\pgfpathcurveto{\pgfqpoint{0.911375in}{1.320400in}}{\pgfqpoint{0.903475in}{1.323672in}}{\pgfqpoint{0.895239in}{1.323672in}}%
\pgfpathcurveto{\pgfqpoint{0.887003in}{1.323672in}}{\pgfqpoint{0.879103in}{1.320400in}}{\pgfqpoint{0.873279in}{1.314576in}}%
\pgfpathcurveto{\pgfqpoint{0.867455in}{1.308752in}}{\pgfqpoint{0.864183in}{1.300852in}}{\pgfqpoint{0.864183in}{1.292616in}}%
\pgfpathcurveto{\pgfqpoint{0.864183in}{1.284379in}}{\pgfqpoint{0.867455in}{1.276479in}}{\pgfqpoint{0.873279in}{1.270655in}}%
\pgfpathcurveto{\pgfqpoint{0.879103in}{1.264831in}}{\pgfqpoint{0.887003in}{1.261559in}}{\pgfqpoint{0.895239in}{1.261559in}}%
\pgfpathclose%
\pgfusepath{stroke,fill}%
\end{pgfscope}%
\begin{pgfscope}%
\pgfpathrectangle{\pgfqpoint{0.100000in}{0.212622in}}{\pgfqpoint{3.696000in}{3.696000in}}%
\pgfusepath{clip}%
\pgfsetbuttcap%
\pgfsetroundjoin%
\definecolor{currentfill}{rgb}{0.121569,0.466667,0.705882}%
\pgfsetfillcolor{currentfill}%
\pgfsetfillopacity{0.632530}%
\pgfsetlinewidth{1.003750pt}%
\definecolor{currentstroke}{rgb}{0.121569,0.466667,0.705882}%
\pgfsetstrokecolor{currentstroke}%
\pgfsetstrokeopacity{0.632530}%
\pgfsetdash{}{0pt}%
\pgfpathmoveto{\pgfqpoint{0.895239in}{1.261559in}}%
\pgfpathcurveto{\pgfqpoint{0.903475in}{1.261559in}}{\pgfqpoint{0.911375in}{1.264831in}}{\pgfqpoint{0.917199in}{1.270655in}}%
\pgfpathcurveto{\pgfqpoint{0.923023in}{1.276479in}}{\pgfqpoint{0.926296in}{1.284379in}}{\pgfqpoint{0.926296in}{1.292616in}}%
\pgfpathcurveto{\pgfqpoint{0.926296in}{1.300852in}}{\pgfqpoint{0.923023in}{1.308752in}}{\pgfqpoint{0.917199in}{1.314576in}}%
\pgfpathcurveto{\pgfqpoint{0.911375in}{1.320400in}}{\pgfqpoint{0.903475in}{1.323672in}}{\pgfqpoint{0.895239in}{1.323672in}}%
\pgfpathcurveto{\pgfqpoint{0.887003in}{1.323672in}}{\pgfqpoint{0.879103in}{1.320400in}}{\pgfqpoint{0.873279in}{1.314576in}}%
\pgfpathcurveto{\pgfqpoint{0.867455in}{1.308752in}}{\pgfqpoint{0.864183in}{1.300852in}}{\pgfqpoint{0.864183in}{1.292616in}}%
\pgfpathcurveto{\pgfqpoint{0.864183in}{1.284379in}}{\pgfqpoint{0.867455in}{1.276479in}}{\pgfqpoint{0.873279in}{1.270655in}}%
\pgfpathcurveto{\pgfqpoint{0.879103in}{1.264831in}}{\pgfqpoint{0.887003in}{1.261559in}}{\pgfqpoint{0.895239in}{1.261559in}}%
\pgfpathclose%
\pgfusepath{stroke,fill}%
\end{pgfscope}%
\begin{pgfscope}%
\pgfpathrectangle{\pgfqpoint{0.100000in}{0.212622in}}{\pgfqpoint{3.696000in}{3.696000in}}%
\pgfusepath{clip}%
\pgfsetbuttcap%
\pgfsetroundjoin%
\definecolor{currentfill}{rgb}{0.121569,0.466667,0.705882}%
\pgfsetfillcolor{currentfill}%
\pgfsetfillopacity{0.632530}%
\pgfsetlinewidth{1.003750pt}%
\definecolor{currentstroke}{rgb}{0.121569,0.466667,0.705882}%
\pgfsetstrokecolor{currentstroke}%
\pgfsetstrokeopacity{0.632530}%
\pgfsetdash{}{0pt}%
\pgfpathmoveto{\pgfqpoint{0.895239in}{1.261559in}}%
\pgfpathcurveto{\pgfqpoint{0.903475in}{1.261559in}}{\pgfqpoint{0.911375in}{1.264831in}}{\pgfqpoint{0.917199in}{1.270655in}}%
\pgfpathcurveto{\pgfqpoint{0.923023in}{1.276479in}}{\pgfqpoint{0.926296in}{1.284379in}}{\pgfqpoint{0.926296in}{1.292616in}}%
\pgfpathcurveto{\pgfqpoint{0.926296in}{1.300852in}}{\pgfqpoint{0.923023in}{1.308752in}}{\pgfqpoint{0.917199in}{1.314576in}}%
\pgfpathcurveto{\pgfqpoint{0.911375in}{1.320400in}}{\pgfqpoint{0.903475in}{1.323672in}}{\pgfqpoint{0.895239in}{1.323672in}}%
\pgfpathcurveto{\pgfqpoint{0.887003in}{1.323672in}}{\pgfqpoint{0.879103in}{1.320400in}}{\pgfqpoint{0.873279in}{1.314576in}}%
\pgfpathcurveto{\pgfqpoint{0.867455in}{1.308752in}}{\pgfqpoint{0.864183in}{1.300852in}}{\pgfqpoint{0.864183in}{1.292616in}}%
\pgfpathcurveto{\pgfqpoint{0.864183in}{1.284379in}}{\pgfqpoint{0.867455in}{1.276479in}}{\pgfqpoint{0.873279in}{1.270655in}}%
\pgfpathcurveto{\pgfqpoint{0.879103in}{1.264831in}}{\pgfqpoint{0.887003in}{1.261559in}}{\pgfqpoint{0.895239in}{1.261559in}}%
\pgfpathclose%
\pgfusepath{stroke,fill}%
\end{pgfscope}%
\begin{pgfscope}%
\pgfpathrectangle{\pgfqpoint{0.100000in}{0.212622in}}{\pgfqpoint{3.696000in}{3.696000in}}%
\pgfusepath{clip}%
\pgfsetbuttcap%
\pgfsetroundjoin%
\definecolor{currentfill}{rgb}{0.121569,0.466667,0.705882}%
\pgfsetfillcolor{currentfill}%
\pgfsetfillopacity{0.632530}%
\pgfsetlinewidth{1.003750pt}%
\definecolor{currentstroke}{rgb}{0.121569,0.466667,0.705882}%
\pgfsetstrokecolor{currentstroke}%
\pgfsetstrokeopacity{0.632530}%
\pgfsetdash{}{0pt}%
\pgfpathmoveto{\pgfqpoint{0.895239in}{1.261559in}}%
\pgfpathcurveto{\pgfqpoint{0.903475in}{1.261559in}}{\pgfqpoint{0.911375in}{1.264831in}}{\pgfqpoint{0.917199in}{1.270655in}}%
\pgfpathcurveto{\pgfqpoint{0.923023in}{1.276479in}}{\pgfqpoint{0.926296in}{1.284379in}}{\pgfqpoint{0.926296in}{1.292616in}}%
\pgfpathcurveto{\pgfqpoint{0.926296in}{1.300852in}}{\pgfqpoint{0.923023in}{1.308752in}}{\pgfqpoint{0.917199in}{1.314576in}}%
\pgfpathcurveto{\pgfqpoint{0.911375in}{1.320400in}}{\pgfqpoint{0.903475in}{1.323672in}}{\pgfqpoint{0.895239in}{1.323672in}}%
\pgfpathcurveto{\pgfqpoint{0.887003in}{1.323672in}}{\pgfqpoint{0.879103in}{1.320400in}}{\pgfqpoint{0.873279in}{1.314576in}}%
\pgfpathcurveto{\pgfqpoint{0.867455in}{1.308752in}}{\pgfqpoint{0.864183in}{1.300852in}}{\pgfqpoint{0.864183in}{1.292616in}}%
\pgfpathcurveto{\pgfqpoint{0.864183in}{1.284379in}}{\pgfqpoint{0.867455in}{1.276479in}}{\pgfqpoint{0.873279in}{1.270655in}}%
\pgfpathcurveto{\pgfqpoint{0.879103in}{1.264831in}}{\pgfqpoint{0.887003in}{1.261559in}}{\pgfqpoint{0.895239in}{1.261559in}}%
\pgfpathclose%
\pgfusepath{stroke,fill}%
\end{pgfscope}%
\begin{pgfscope}%
\pgfpathrectangle{\pgfqpoint{0.100000in}{0.212622in}}{\pgfqpoint{3.696000in}{3.696000in}}%
\pgfusepath{clip}%
\pgfsetbuttcap%
\pgfsetroundjoin%
\definecolor{currentfill}{rgb}{0.121569,0.466667,0.705882}%
\pgfsetfillcolor{currentfill}%
\pgfsetfillopacity{0.632530}%
\pgfsetlinewidth{1.003750pt}%
\definecolor{currentstroke}{rgb}{0.121569,0.466667,0.705882}%
\pgfsetstrokecolor{currentstroke}%
\pgfsetstrokeopacity{0.632530}%
\pgfsetdash{}{0pt}%
\pgfpathmoveto{\pgfqpoint{0.895239in}{1.261559in}}%
\pgfpathcurveto{\pgfqpoint{0.903475in}{1.261559in}}{\pgfqpoint{0.911375in}{1.264831in}}{\pgfqpoint{0.917199in}{1.270655in}}%
\pgfpathcurveto{\pgfqpoint{0.923023in}{1.276479in}}{\pgfqpoint{0.926296in}{1.284379in}}{\pgfqpoint{0.926296in}{1.292616in}}%
\pgfpathcurveto{\pgfqpoint{0.926296in}{1.300852in}}{\pgfqpoint{0.923023in}{1.308752in}}{\pgfqpoint{0.917199in}{1.314576in}}%
\pgfpathcurveto{\pgfqpoint{0.911375in}{1.320400in}}{\pgfqpoint{0.903475in}{1.323672in}}{\pgfqpoint{0.895239in}{1.323672in}}%
\pgfpathcurveto{\pgfqpoint{0.887003in}{1.323672in}}{\pgfqpoint{0.879103in}{1.320400in}}{\pgfqpoint{0.873279in}{1.314576in}}%
\pgfpathcurveto{\pgfqpoint{0.867455in}{1.308752in}}{\pgfqpoint{0.864183in}{1.300852in}}{\pgfqpoint{0.864183in}{1.292616in}}%
\pgfpathcurveto{\pgfqpoint{0.864183in}{1.284379in}}{\pgfqpoint{0.867455in}{1.276479in}}{\pgfqpoint{0.873279in}{1.270655in}}%
\pgfpathcurveto{\pgfqpoint{0.879103in}{1.264831in}}{\pgfqpoint{0.887003in}{1.261559in}}{\pgfqpoint{0.895239in}{1.261559in}}%
\pgfpathclose%
\pgfusepath{stroke,fill}%
\end{pgfscope}%
\begin{pgfscope}%
\pgfpathrectangle{\pgfqpoint{0.100000in}{0.212622in}}{\pgfqpoint{3.696000in}{3.696000in}}%
\pgfusepath{clip}%
\pgfsetbuttcap%
\pgfsetroundjoin%
\definecolor{currentfill}{rgb}{0.121569,0.466667,0.705882}%
\pgfsetfillcolor{currentfill}%
\pgfsetfillopacity{0.632530}%
\pgfsetlinewidth{1.003750pt}%
\definecolor{currentstroke}{rgb}{0.121569,0.466667,0.705882}%
\pgfsetstrokecolor{currentstroke}%
\pgfsetstrokeopacity{0.632530}%
\pgfsetdash{}{0pt}%
\pgfpathmoveto{\pgfqpoint{0.895239in}{1.261559in}}%
\pgfpathcurveto{\pgfqpoint{0.903475in}{1.261559in}}{\pgfqpoint{0.911375in}{1.264831in}}{\pgfqpoint{0.917199in}{1.270655in}}%
\pgfpathcurveto{\pgfqpoint{0.923023in}{1.276479in}}{\pgfqpoint{0.926296in}{1.284379in}}{\pgfqpoint{0.926296in}{1.292616in}}%
\pgfpathcurveto{\pgfqpoint{0.926296in}{1.300852in}}{\pgfqpoint{0.923023in}{1.308752in}}{\pgfqpoint{0.917199in}{1.314576in}}%
\pgfpathcurveto{\pgfqpoint{0.911375in}{1.320400in}}{\pgfqpoint{0.903475in}{1.323672in}}{\pgfqpoint{0.895239in}{1.323672in}}%
\pgfpathcurveto{\pgfqpoint{0.887003in}{1.323672in}}{\pgfqpoint{0.879103in}{1.320400in}}{\pgfqpoint{0.873279in}{1.314576in}}%
\pgfpathcurveto{\pgfqpoint{0.867455in}{1.308752in}}{\pgfqpoint{0.864183in}{1.300852in}}{\pgfqpoint{0.864183in}{1.292616in}}%
\pgfpathcurveto{\pgfqpoint{0.864183in}{1.284379in}}{\pgfqpoint{0.867455in}{1.276479in}}{\pgfqpoint{0.873279in}{1.270655in}}%
\pgfpathcurveto{\pgfqpoint{0.879103in}{1.264831in}}{\pgfqpoint{0.887003in}{1.261559in}}{\pgfqpoint{0.895239in}{1.261559in}}%
\pgfpathclose%
\pgfusepath{stroke,fill}%
\end{pgfscope}%
\begin{pgfscope}%
\pgfpathrectangle{\pgfqpoint{0.100000in}{0.212622in}}{\pgfqpoint{3.696000in}{3.696000in}}%
\pgfusepath{clip}%
\pgfsetbuttcap%
\pgfsetroundjoin%
\definecolor{currentfill}{rgb}{0.121569,0.466667,0.705882}%
\pgfsetfillcolor{currentfill}%
\pgfsetfillopacity{0.632530}%
\pgfsetlinewidth{1.003750pt}%
\definecolor{currentstroke}{rgb}{0.121569,0.466667,0.705882}%
\pgfsetstrokecolor{currentstroke}%
\pgfsetstrokeopacity{0.632530}%
\pgfsetdash{}{0pt}%
\pgfpathmoveto{\pgfqpoint{0.895239in}{1.261559in}}%
\pgfpathcurveto{\pgfqpoint{0.903475in}{1.261559in}}{\pgfqpoint{0.911375in}{1.264831in}}{\pgfqpoint{0.917199in}{1.270655in}}%
\pgfpathcurveto{\pgfqpoint{0.923023in}{1.276479in}}{\pgfqpoint{0.926296in}{1.284379in}}{\pgfqpoint{0.926296in}{1.292616in}}%
\pgfpathcurveto{\pgfqpoint{0.926296in}{1.300852in}}{\pgfqpoint{0.923023in}{1.308752in}}{\pgfqpoint{0.917199in}{1.314576in}}%
\pgfpathcurveto{\pgfqpoint{0.911375in}{1.320400in}}{\pgfqpoint{0.903475in}{1.323672in}}{\pgfqpoint{0.895239in}{1.323672in}}%
\pgfpathcurveto{\pgfqpoint{0.887003in}{1.323672in}}{\pgfqpoint{0.879103in}{1.320400in}}{\pgfqpoint{0.873279in}{1.314576in}}%
\pgfpathcurveto{\pgfqpoint{0.867455in}{1.308752in}}{\pgfqpoint{0.864183in}{1.300852in}}{\pgfqpoint{0.864183in}{1.292616in}}%
\pgfpathcurveto{\pgfqpoint{0.864183in}{1.284379in}}{\pgfqpoint{0.867455in}{1.276479in}}{\pgfqpoint{0.873279in}{1.270655in}}%
\pgfpathcurveto{\pgfqpoint{0.879103in}{1.264831in}}{\pgfqpoint{0.887003in}{1.261559in}}{\pgfqpoint{0.895239in}{1.261559in}}%
\pgfpathclose%
\pgfusepath{stroke,fill}%
\end{pgfscope}%
\begin{pgfscope}%
\pgfpathrectangle{\pgfqpoint{0.100000in}{0.212622in}}{\pgfqpoint{3.696000in}{3.696000in}}%
\pgfusepath{clip}%
\pgfsetbuttcap%
\pgfsetroundjoin%
\definecolor{currentfill}{rgb}{0.121569,0.466667,0.705882}%
\pgfsetfillcolor{currentfill}%
\pgfsetfillopacity{0.632530}%
\pgfsetlinewidth{1.003750pt}%
\definecolor{currentstroke}{rgb}{0.121569,0.466667,0.705882}%
\pgfsetstrokecolor{currentstroke}%
\pgfsetstrokeopacity{0.632530}%
\pgfsetdash{}{0pt}%
\pgfpathmoveto{\pgfqpoint{0.895239in}{1.261559in}}%
\pgfpathcurveto{\pgfqpoint{0.903475in}{1.261559in}}{\pgfqpoint{0.911375in}{1.264831in}}{\pgfqpoint{0.917199in}{1.270655in}}%
\pgfpathcurveto{\pgfqpoint{0.923023in}{1.276479in}}{\pgfqpoint{0.926296in}{1.284379in}}{\pgfqpoint{0.926296in}{1.292616in}}%
\pgfpathcurveto{\pgfqpoint{0.926296in}{1.300852in}}{\pgfqpoint{0.923023in}{1.308752in}}{\pgfqpoint{0.917199in}{1.314576in}}%
\pgfpathcurveto{\pgfqpoint{0.911375in}{1.320400in}}{\pgfqpoint{0.903475in}{1.323672in}}{\pgfqpoint{0.895239in}{1.323672in}}%
\pgfpathcurveto{\pgfqpoint{0.887003in}{1.323672in}}{\pgfqpoint{0.879103in}{1.320400in}}{\pgfqpoint{0.873279in}{1.314576in}}%
\pgfpathcurveto{\pgfqpoint{0.867455in}{1.308752in}}{\pgfqpoint{0.864183in}{1.300852in}}{\pgfqpoint{0.864183in}{1.292616in}}%
\pgfpathcurveto{\pgfqpoint{0.864183in}{1.284379in}}{\pgfqpoint{0.867455in}{1.276479in}}{\pgfqpoint{0.873279in}{1.270655in}}%
\pgfpathcurveto{\pgfqpoint{0.879103in}{1.264831in}}{\pgfqpoint{0.887003in}{1.261559in}}{\pgfqpoint{0.895239in}{1.261559in}}%
\pgfpathclose%
\pgfusepath{stroke,fill}%
\end{pgfscope}%
\begin{pgfscope}%
\pgfpathrectangle{\pgfqpoint{0.100000in}{0.212622in}}{\pgfqpoint{3.696000in}{3.696000in}}%
\pgfusepath{clip}%
\pgfsetbuttcap%
\pgfsetroundjoin%
\definecolor{currentfill}{rgb}{0.121569,0.466667,0.705882}%
\pgfsetfillcolor{currentfill}%
\pgfsetfillopacity{0.632530}%
\pgfsetlinewidth{1.003750pt}%
\definecolor{currentstroke}{rgb}{0.121569,0.466667,0.705882}%
\pgfsetstrokecolor{currentstroke}%
\pgfsetstrokeopacity{0.632530}%
\pgfsetdash{}{0pt}%
\pgfpathmoveto{\pgfqpoint{0.895239in}{1.261559in}}%
\pgfpathcurveto{\pgfqpoint{0.903475in}{1.261559in}}{\pgfqpoint{0.911375in}{1.264831in}}{\pgfqpoint{0.917199in}{1.270655in}}%
\pgfpathcurveto{\pgfqpoint{0.923023in}{1.276479in}}{\pgfqpoint{0.926296in}{1.284379in}}{\pgfqpoint{0.926296in}{1.292616in}}%
\pgfpathcurveto{\pgfqpoint{0.926296in}{1.300852in}}{\pgfqpoint{0.923023in}{1.308752in}}{\pgfqpoint{0.917199in}{1.314576in}}%
\pgfpathcurveto{\pgfqpoint{0.911375in}{1.320400in}}{\pgfqpoint{0.903475in}{1.323672in}}{\pgfqpoint{0.895239in}{1.323672in}}%
\pgfpathcurveto{\pgfqpoint{0.887003in}{1.323672in}}{\pgfqpoint{0.879103in}{1.320400in}}{\pgfqpoint{0.873279in}{1.314576in}}%
\pgfpathcurveto{\pgfqpoint{0.867455in}{1.308752in}}{\pgfqpoint{0.864183in}{1.300852in}}{\pgfqpoint{0.864183in}{1.292616in}}%
\pgfpathcurveto{\pgfqpoint{0.864183in}{1.284379in}}{\pgfqpoint{0.867455in}{1.276479in}}{\pgfqpoint{0.873279in}{1.270655in}}%
\pgfpathcurveto{\pgfqpoint{0.879103in}{1.264831in}}{\pgfqpoint{0.887003in}{1.261559in}}{\pgfqpoint{0.895239in}{1.261559in}}%
\pgfpathclose%
\pgfusepath{stroke,fill}%
\end{pgfscope}%
\begin{pgfscope}%
\pgfpathrectangle{\pgfqpoint{0.100000in}{0.212622in}}{\pgfqpoint{3.696000in}{3.696000in}}%
\pgfusepath{clip}%
\pgfsetbuttcap%
\pgfsetroundjoin%
\definecolor{currentfill}{rgb}{0.121569,0.466667,0.705882}%
\pgfsetfillcolor{currentfill}%
\pgfsetfillopacity{0.632530}%
\pgfsetlinewidth{1.003750pt}%
\definecolor{currentstroke}{rgb}{0.121569,0.466667,0.705882}%
\pgfsetstrokecolor{currentstroke}%
\pgfsetstrokeopacity{0.632530}%
\pgfsetdash{}{0pt}%
\pgfpathmoveto{\pgfqpoint{0.895239in}{1.261559in}}%
\pgfpathcurveto{\pgfqpoint{0.903475in}{1.261559in}}{\pgfqpoint{0.911375in}{1.264831in}}{\pgfqpoint{0.917199in}{1.270655in}}%
\pgfpathcurveto{\pgfqpoint{0.923023in}{1.276479in}}{\pgfqpoint{0.926296in}{1.284379in}}{\pgfqpoint{0.926296in}{1.292616in}}%
\pgfpathcurveto{\pgfqpoint{0.926296in}{1.300852in}}{\pgfqpoint{0.923023in}{1.308752in}}{\pgfqpoint{0.917199in}{1.314576in}}%
\pgfpathcurveto{\pgfqpoint{0.911375in}{1.320400in}}{\pgfqpoint{0.903475in}{1.323672in}}{\pgfqpoint{0.895239in}{1.323672in}}%
\pgfpathcurveto{\pgfqpoint{0.887003in}{1.323672in}}{\pgfqpoint{0.879103in}{1.320400in}}{\pgfqpoint{0.873279in}{1.314576in}}%
\pgfpathcurveto{\pgfqpoint{0.867455in}{1.308752in}}{\pgfqpoint{0.864183in}{1.300852in}}{\pgfqpoint{0.864183in}{1.292616in}}%
\pgfpathcurveto{\pgfqpoint{0.864183in}{1.284379in}}{\pgfqpoint{0.867455in}{1.276479in}}{\pgfqpoint{0.873279in}{1.270655in}}%
\pgfpathcurveto{\pgfqpoint{0.879103in}{1.264831in}}{\pgfqpoint{0.887003in}{1.261559in}}{\pgfqpoint{0.895239in}{1.261559in}}%
\pgfpathclose%
\pgfusepath{stroke,fill}%
\end{pgfscope}%
\begin{pgfscope}%
\pgfpathrectangle{\pgfqpoint{0.100000in}{0.212622in}}{\pgfqpoint{3.696000in}{3.696000in}}%
\pgfusepath{clip}%
\pgfsetbuttcap%
\pgfsetroundjoin%
\definecolor{currentfill}{rgb}{0.121569,0.466667,0.705882}%
\pgfsetfillcolor{currentfill}%
\pgfsetfillopacity{0.632530}%
\pgfsetlinewidth{1.003750pt}%
\definecolor{currentstroke}{rgb}{0.121569,0.466667,0.705882}%
\pgfsetstrokecolor{currentstroke}%
\pgfsetstrokeopacity{0.632530}%
\pgfsetdash{}{0pt}%
\pgfpathmoveto{\pgfqpoint{0.895239in}{1.261559in}}%
\pgfpathcurveto{\pgfqpoint{0.903475in}{1.261559in}}{\pgfqpoint{0.911375in}{1.264831in}}{\pgfqpoint{0.917199in}{1.270655in}}%
\pgfpathcurveto{\pgfqpoint{0.923023in}{1.276479in}}{\pgfqpoint{0.926296in}{1.284379in}}{\pgfqpoint{0.926296in}{1.292616in}}%
\pgfpathcurveto{\pgfqpoint{0.926296in}{1.300852in}}{\pgfqpoint{0.923023in}{1.308752in}}{\pgfqpoint{0.917199in}{1.314576in}}%
\pgfpathcurveto{\pgfqpoint{0.911375in}{1.320400in}}{\pgfqpoint{0.903475in}{1.323672in}}{\pgfqpoint{0.895239in}{1.323672in}}%
\pgfpathcurveto{\pgfqpoint{0.887003in}{1.323672in}}{\pgfqpoint{0.879103in}{1.320400in}}{\pgfqpoint{0.873279in}{1.314576in}}%
\pgfpathcurveto{\pgfqpoint{0.867455in}{1.308752in}}{\pgfqpoint{0.864183in}{1.300852in}}{\pgfqpoint{0.864183in}{1.292616in}}%
\pgfpathcurveto{\pgfqpoint{0.864183in}{1.284379in}}{\pgfqpoint{0.867455in}{1.276479in}}{\pgfqpoint{0.873279in}{1.270655in}}%
\pgfpathcurveto{\pgfqpoint{0.879103in}{1.264831in}}{\pgfqpoint{0.887003in}{1.261559in}}{\pgfqpoint{0.895239in}{1.261559in}}%
\pgfpathclose%
\pgfusepath{stroke,fill}%
\end{pgfscope}%
\begin{pgfscope}%
\pgfpathrectangle{\pgfqpoint{0.100000in}{0.212622in}}{\pgfqpoint{3.696000in}{3.696000in}}%
\pgfusepath{clip}%
\pgfsetbuttcap%
\pgfsetroundjoin%
\definecolor{currentfill}{rgb}{0.121569,0.466667,0.705882}%
\pgfsetfillcolor{currentfill}%
\pgfsetfillopacity{0.632530}%
\pgfsetlinewidth{1.003750pt}%
\definecolor{currentstroke}{rgb}{0.121569,0.466667,0.705882}%
\pgfsetstrokecolor{currentstroke}%
\pgfsetstrokeopacity{0.632530}%
\pgfsetdash{}{0pt}%
\pgfpathmoveto{\pgfqpoint{0.895239in}{1.261559in}}%
\pgfpathcurveto{\pgfqpoint{0.903475in}{1.261559in}}{\pgfqpoint{0.911375in}{1.264831in}}{\pgfqpoint{0.917199in}{1.270655in}}%
\pgfpathcurveto{\pgfqpoint{0.923023in}{1.276479in}}{\pgfqpoint{0.926296in}{1.284379in}}{\pgfqpoint{0.926296in}{1.292616in}}%
\pgfpathcurveto{\pgfqpoint{0.926296in}{1.300852in}}{\pgfqpoint{0.923023in}{1.308752in}}{\pgfqpoint{0.917199in}{1.314576in}}%
\pgfpathcurveto{\pgfqpoint{0.911375in}{1.320400in}}{\pgfqpoint{0.903475in}{1.323672in}}{\pgfqpoint{0.895239in}{1.323672in}}%
\pgfpathcurveto{\pgfqpoint{0.887003in}{1.323672in}}{\pgfqpoint{0.879103in}{1.320400in}}{\pgfqpoint{0.873279in}{1.314576in}}%
\pgfpathcurveto{\pgfqpoint{0.867455in}{1.308752in}}{\pgfqpoint{0.864183in}{1.300852in}}{\pgfqpoint{0.864183in}{1.292616in}}%
\pgfpathcurveto{\pgfqpoint{0.864183in}{1.284379in}}{\pgfqpoint{0.867455in}{1.276479in}}{\pgfqpoint{0.873279in}{1.270655in}}%
\pgfpathcurveto{\pgfqpoint{0.879103in}{1.264831in}}{\pgfqpoint{0.887003in}{1.261559in}}{\pgfqpoint{0.895239in}{1.261559in}}%
\pgfpathclose%
\pgfusepath{stroke,fill}%
\end{pgfscope}%
\begin{pgfscope}%
\pgfpathrectangle{\pgfqpoint{0.100000in}{0.212622in}}{\pgfqpoint{3.696000in}{3.696000in}}%
\pgfusepath{clip}%
\pgfsetbuttcap%
\pgfsetroundjoin%
\definecolor{currentfill}{rgb}{0.121569,0.466667,0.705882}%
\pgfsetfillcolor{currentfill}%
\pgfsetfillopacity{0.632530}%
\pgfsetlinewidth{1.003750pt}%
\definecolor{currentstroke}{rgb}{0.121569,0.466667,0.705882}%
\pgfsetstrokecolor{currentstroke}%
\pgfsetstrokeopacity{0.632530}%
\pgfsetdash{}{0pt}%
\pgfpathmoveto{\pgfqpoint{0.895239in}{1.261559in}}%
\pgfpathcurveto{\pgfqpoint{0.903475in}{1.261559in}}{\pgfqpoint{0.911375in}{1.264831in}}{\pgfqpoint{0.917199in}{1.270655in}}%
\pgfpathcurveto{\pgfqpoint{0.923023in}{1.276479in}}{\pgfqpoint{0.926296in}{1.284379in}}{\pgfqpoint{0.926296in}{1.292616in}}%
\pgfpathcurveto{\pgfqpoint{0.926296in}{1.300852in}}{\pgfqpoint{0.923023in}{1.308752in}}{\pgfqpoint{0.917199in}{1.314576in}}%
\pgfpathcurveto{\pgfqpoint{0.911375in}{1.320400in}}{\pgfqpoint{0.903475in}{1.323672in}}{\pgfqpoint{0.895239in}{1.323672in}}%
\pgfpathcurveto{\pgfqpoint{0.887003in}{1.323672in}}{\pgfqpoint{0.879103in}{1.320400in}}{\pgfqpoint{0.873279in}{1.314576in}}%
\pgfpathcurveto{\pgfqpoint{0.867455in}{1.308752in}}{\pgfqpoint{0.864183in}{1.300852in}}{\pgfqpoint{0.864183in}{1.292616in}}%
\pgfpathcurveto{\pgfqpoint{0.864183in}{1.284379in}}{\pgfqpoint{0.867455in}{1.276479in}}{\pgfqpoint{0.873279in}{1.270655in}}%
\pgfpathcurveto{\pgfqpoint{0.879103in}{1.264831in}}{\pgfqpoint{0.887003in}{1.261559in}}{\pgfqpoint{0.895239in}{1.261559in}}%
\pgfpathclose%
\pgfusepath{stroke,fill}%
\end{pgfscope}%
\begin{pgfscope}%
\pgfpathrectangle{\pgfqpoint{0.100000in}{0.212622in}}{\pgfqpoint{3.696000in}{3.696000in}}%
\pgfusepath{clip}%
\pgfsetbuttcap%
\pgfsetroundjoin%
\definecolor{currentfill}{rgb}{0.121569,0.466667,0.705882}%
\pgfsetfillcolor{currentfill}%
\pgfsetfillopacity{0.632530}%
\pgfsetlinewidth{1.003750pt}%
\definecolor{currentstroke}{rgb}{0.121569,0.466667,0.705882}%
\pgfsetstrokecolor{currentstroke}%
\pgfsetstrokeopacity{0.632530}%
\pgfsetdash{}{0pt}%
\pgfpathmoveto{\pgfqpoint{0.895239in}{1.261559in}}%
\pgfpathcurveto{\pgfqpoint{0.903475in}{1.261559in}}{\pgfqpoint{0.911375in}{1.264831in}}{\pgfqpoint{0.917199in}{1.270655in}}%
\pgfpathcurveto{\pgfqpoint{0.923023in}{1.276479in}}{\pgfqpoint{0.926296in}{1.284379in}}{\pgfqpoint{0.926296in}{1.292616in}}%
\pgfpathcurveto{\pgfqpoint{0.926296in}{1.300852in}}{\pgfqpoint{0.923023in}{1.308752in}}{\pgfqpoint{0.917199in}{1.314576in}}%
\pgfpathcurveto{\pgfqpoint{0.911375in}{1.320400in}}{\pgfqpoint{0.903475in}{1.323672in}}{\pgfqpoint{0.895239in}{1.323672in}}%
\pgfpathcurveto{\pgfqpoint{0.887003in}{1.323672in}}{\pgfqpoint{0.879103in}{1.320400in}}{\pgfqpoint{0.873279in}{1.314576in}}%
\pgfpathcurveto{\pgfqpoint{0.867455in}{1.308752in}}{\pgfqpoint{0.864183in}{1.300852in}}{\pgfqpoint{0.864183in}{1.292616in}}%
\pgfpathcurveto{\pgfqpoint{0.864183in}{1.284379in}}{\pgfqpoint{0.867455in}{1.276479in}}{\pgfqpoint{0.873279in}{1.270655in}}%
\pgfpathcurveto{\pgfqpoint{0.879103in}{1.264831in}}{\pgfqpoint{0.887003in}{1.261559in}}{\pgfqpoint{0.895239in}{1.261559in}}%
\pgfpathclose%
\pgfusepath{stroke,fill}%
\end{pgfscope}%
\begin{pgfscope}%
\pgfpathrectangle{\pgfqpoint{0.100000in}{0.212622in}}{\pgfqpoint{3.696000in}{3.696000in}}%
\pgfusepath{clip}%
\pgfsetbuttcap%
\pgfsetroundjoin%
\definecolor{currentfill}{rgb}{0.121569,0.466667,0.705882}%
\pgfsetfillcolor{currentfill}%
\pgfsetfillopacity{0.632530}%
\pgfsetlinewidth{1.003750pt}%
\definecolor{currentstroke}{rgb}{0.121569,0.466667,0.705882}%
\pgfsetstrokecolor{currentstroke}%
\pgfsetstrokeopacity{0.632530}%
\pgfsetdash{}{0pt}%
\pgfpathmoveto{\pgfqpoint{0.895239in}{1.261559in}}%
\pgfpathcurveto{\pgfqpoint{0.903475in}{1.261559in}}{\pgfqpoint{0.911375in}{1.264831in}}{\pgfqpoint{0.917199in}{1.270655in}}%
\pgfpathcurveto{\pgfqpoint{0.923023in}{1.276479in}}{\pgfqpoint{0.926296in}{1.284379in}}{\pgfqpoint{0.926296in}{1.292616in}}%
\pgfpathcurveto{\pgfqpoint{0.926296in}{1.300852in}}{\pgfqpoint{0.923023in}{1.308752in}}{\pgfqpoint{0.917199in}{1.314576in}}%
\pgfpathcurveto{\pgfqpoint{0.911375in}{1.320400in}}{\pgfqpoint{0.903475in}{1.323672in}}{\pgfqpoint{0.895239in}{1.323672in}}%
\pgfpathcurveto{\pgfqpoint{0.887003in}{1.323672in}}{\pgfqpoint{0.879103in}{1.320400in}}{\pgfqpoint{0.873279in}{1.314576in}}%
\pgfpathcurveto{\pgfqpoint{0.867455in}{1.308752in}}{\pgfqpoint{0.864183in}{1.300852in}}{\pgfqpoint{0.864183in}{1.292616in}}%
\pgfpathcurveto{\pgfqpoint{0.864183in}{1.284379in}}{\pgfqpoint{0.867455in}{1.276479in}}{\pgfqpoint{0.873279in}{1.270655in}}%
\pgfpathcurveto{\pgfqpoint{0.879103in}{1.264831in}}{\pgfqpoint{0.887003in}{1.261559in}}{\pgfqpoint{0.895239in}{1.261559in}}%
\pgfpathclose%
\pgfusepath{stroke,fill}%
\end{pgfscope}%
\begin{pgfscope}%
\pgfpathrectangle{\pgfqpoint{0.100000in}{0.212622in}}{\pgfqpoint{3.696000in}{3.696000in}}%
\pgfusepath{clip}%
\pgfsetbuttcap%
\pgfsetroundjoin%
\definecolor{currentfill}{rgb}{0.121569,0.466667,0.705882}%
\pgfsetfillcolor{currentfill}%
\pgfsetfillopacity{0.632530}%
\pgfsetlinewidth{1.003750pt}%
\definecolor{currentstroke}{rgb}{0.121569,0.466667,0.705882}%
\pgfsetstrokecolor{currentstroke}%
\pgfsetstrokeopacity{0.632530}%
\pgfsetdash{}{0pt}%
\pgfpathmoveto{\pgfqpoint{0.895239in}{1.261559in}}%
\pgfpathcurveto{\pgfqpoint{0.903475in}{1.261559in}}{\pgfqpoint{0.911375in}{1.264831in}}{\pgfqpoint{0.917199in}{1.270655in}}%
\pgfpathcurveto{\pgfqpoint{0.923023in}{1.276479in}}{\pgfqpoint{0.926296in}{1.284379in}}{\pgfqpoint{0.926296in}{1.292616in}}%
\pgfpathcurveto{\pgfqpoint{0.926296in}{1.300852in}}{\pgfqpoint{0.923023in}{1.308752in}}{\pgfqpoint{0.917199in}{1.314576in}}%
\pgfpathcurveto{\pgfqpoint{0.911375in}{1.320400in}}{\pgfqpoint{0.903475in}{1.323672in}}{\pgfqpoint{0.895239in}{1.323672in}}%
\pgfpathcurveto{\pgfqpoint{0.887003in}{1.323672in}}{\pgfqpoint{0.879103in}{1.320400in}}{\pgfqpoint{0.873279in}{1.314576in}}%
\pgfpathcurveto{\pgfqpoint{0.867455in}{1.308752in}}{\pgfqpoint{0.864183in}{1.300852in}}{\pgfqpoint{0.864183in}{1.292616in}}%
\pgfpathcurveto{\pgfqpoint{0.864183in}{1.284379in}}{\pgfqpoint{0.867455in}{1.276479in}}{\pgfqpoint{0.873279in}{1.270655in}}%
\pgfpathcurveto{\pgfqpoint{0.879103in}{1.264831in}}{\pgfqpoint{0.887003in}{1.261559in}}{\pgfqpoint{0.895239in}{1.261559in}}%
\pgfpathclose%
\pgfusepath{stroke,fill}%
\end{pgfscope}%
\begin{pgfscope}%
\pgfpathrectangle{\pgfqpoint{0.100000in}{0.212622in}}{\pgfqpoint{3.696000in}{3.696000in}}%
\pgfusepath{clip}%
\pgfsetbuttcap%
\pgfsetroundjoin%
\definecolor{currentfill}{rgb}{0.121569,0.466667,0.705882}%
\pgfsetfillcolor{currentfill}%
\pgfsetfillopacity{0.632530}%
\pgfsetlinewidth{1.003750pt}%
\definecolor{currentstroke}{rgb}{0.121569,0.466667,0.705882}%
\pgfsetstrokecolor{currentstroke}%
\pgfsetstrokeopacity{0.632530}%
\pgfsetdash{}{0pt}%
\pgfpathmoveto{\pgfqpoint{0.895239in}{1.261559in}}%
\pgfpathcurveto{\pgfqpoint{0.903475in}{1.261559in}}{\pgfqpoint{0.911375in}{1.264831in}}{\pgfqpoint{0.917199in}{1.270655in}}%
\pgfpathcurveto{\pgfqpoint{0.923023in}{1.276479in}}{\pgfqpoint{0.926296in}{1.284379in}}{\pgfqpoint{0.926296in}{1.292616in}}%
\pgfpathcurveto{\pgfqpoint{0.926296in}{1.300852in}}{\pgfqpoint{0.923023in}{1.308752in}}{\pgfqpoint{0.917199in}{1.314576in}}%
\pgfpathcurveto{\pgfqpoint{0.911375in}{1.320400in}}{\pgfqpoint{0.903475in}{1.323672in}}{\pgfqpoint{0.895239in}{1.323672in}}%
\pgfpathcurveto{\pgfqpoint{0.887003in}{1.323672in}}{\pgfqpoint{0.879103in}{1.320400in}}{\pgfqpoint{0.873279in}{1.314576in}}%
\pgfpathcurveto{\pgfqpoint{0.867455in}{1.308752in}}{\pgfqpoint{0.864183in}{1.300852in}}{\pgfqpoint{0.864183in}{1.292616in}}%
\pgfpathcurveto{\pgfqpoint{0.864183in}{1.284379in}}{\pgfqpoint{0.867455in}{1.276479in}}{\pgfqpoint{0.873279in}{1.270655in}}%
\pgfpathcurveto{\pgfqpoint{0.879103in}{1.264831in}}{\pgfqpoint{0.887003in}{1.261559in}}{\pgfqpoint{0.895239in}{1.261559in}}%
\pgfpathclose%
\pgfusepath{stroke,fill}%
\end{pgfscope}%
\begin{pgfscope}%
\pgfpathrectangle{\pgfqpoint{0.100000in}{0.212622in}}{\pgfqpoint{3.696000in}{3.696000in}}%
\pgfusepath{clip}%
\pgfsetbuttcap%
\pgfsetroundjoin%
\definecolor{currentfill}{rgb}{0.121569,0.466667,0.705882}%
\pgfsetfillcolor{currentfill}%
\pgfsetfillopacity{0.632530}%
\pgfsetlinewidth{1.003750pt}%
\definecolor{currentstroke}{rgb}{0.121569,0.466667,0.705882}%
\pgfsetstrokecolor{currentstroke}%
\pgfsetstrokeopacity{0.632530}%
\pgfsetdash{}{0pt}%
\pgfpathmoveto{\pgfqpoint{0.895239in}{1.261559in}}%
\pgfpathcurveto{\pgfqpoint{0.903475in}{1.261559in}}{\pgfqpoint{0.911375in}{1.264831in}}{\pgfqpoint{0.917199in}{1.270655in}}%
\pgfpathcurveto{\pgfqpoint{0.923023in}{1.276479in}}{\pgfqpoint{0.926296in}{1.284379in}}{\pgfqpoint{0.926296in}{1.292616in}}%
\pgfpathcurveto{\pgfqpoint{0.926296in}{1.300852in}}{\pgfqpoint{0.923023in}{1.308752in}}{\pgfqpoint{0.917199in}{1.314576in}}%
\pgfpathcurveto{\pgfqpoint{0.911375in}{1.320400in}}{\pgfqpoint{0.903475in}{1.323672in}}{\pgfqpoint{0.895239in}{1.323672in}}%
\pgfpathcurveto{\pgfqpoint{0.887003in}{1.323672in}}{\pgfqpoint{0.879103in}{1.320400in}}{\pgfqpoint{0.873279in}{1.314576in}}%
\pgfpathcurveto{\pgfqpoint{0.867455in}{1.308752in}}{\pgfqpoint{0.864183in}{1.300852in}}{\pgfqpoint{0.864183in}{1.292616in}}%
\pgfpathcurveto{\pgfqpoint{0.864183in}{1.284379in}}{\pgfqpoint{0.867455in}{1.276479in}}{\pgfqpoint{0.873279in}{1.270655in}}%
\pgfpathcurveto{\pgfqpoint{0.879103in}{1.264831in}}{\pgfqpoint{0.887003in}{1.261559in}}{\pgfqpoint{0.895239in}{1.261559in}}%
\pgfpathclose%
\pgfusepath{stroke,fill}%
\end{pgfscope}%
\begin{pgfscope}%
\pgfpathrectangle{\pgfqpoint{0.100000in}{0.212622in}}{\pgfqpoint{3.696000in}{3.696000in}}%
\pgfusepath{clip}%
\pgfsetbuttcap%
\pgfsetroundjoin%
\definecolor{currentfill}{rgb}{0.121569,0.466667,0.705882}%
\pgfsetfillcolor{currentfill}%
\pgfsetfillopacity{0.632530}%
\pgfsetlinewidth{1.003750pt}%
\definecolor{currentstroke}{rgb}{0.121569,0.466667,0.705882}%
\pgfsetstrokecolor{currentstroke}%
\pgfsetstrokeopacity{0.632530}%
\pgfsetdash{}{0pt}%
\pgfpathmoveto{\pgfqpoint{0.895239in}{1.261559in}}%
\pgfpathcurveto{\pgfqpoint{0.903475in}{1.261559in}}{\pgfqpoint{0.911375in}{1.264831in}}{\pgfqpoint{0.917199in}{1.270655in}}%
\pgfpathcurveto{\pgfqpoint{0.923023in}{1.276479in}}{\pgfqpoint{0.926296in}{1.284379in}}{\pgfqpoint{0.926296in}{1.292616in}}%
\pgfpathcurveto{\pgfqpoint{0.926296in}{1.300852in}}{\pgfqpoint{0.923023in}{1.308752in}}{\pgfqpoint{0.917199in}{1.314576in}}%
\pgfpathcurveto{\pgfqpoint{0.911375in}{1.320400in}}{\pgfqpoint{0.903475in}{1.323672in}}{\pgfqpoint{0.895239in}{1.323672in}}%
\pgfpathcurveto{\pgfqpoint{0.887003in}{1.323672in}}{\pgfqpoint{0.879103in}{1.320400in}}{\pgfqpoint{0.873279in}{1.314576in}}%
\pgfpathcurveto{\pgfqpoint{0.867455in}{1.308752in}}{\pgfqpoint{0.864183in}{1.300852in}}{\pgfqpoint{0.864183in}{1.292616in}}%
\pgfpathcurveto{\pgfqpoint{0.864183in}{1.284379in}}{\pgfqpoint{0.867455in}{1.276479in}}{\pgfqpoint{0.873279in}{1.270655in}}%
\pgfpathcurveto{\pgfqpoint{0.879103in}{1.264831in}}{\pgfqpoint{0.887003in}{1.261559in}}{\pgfqpoint{0.895239in}{1.261559in}}%
\pgfpathclose%
\pgfusepath{stroke,fill}%
\end{pgfscope}%
\begin{pgfscope}%
\pgfpathrectangle{\pgfqpoint{0.100000in}{0.212622in}}{\pgfqpoint{3.696000in}{3.696000in}}%
\pgfusepath{clip}%
\pgfsetbuttcap%
\pgfsetroundjoin%
\definecolor{currentfill}{rgb}{0.121569,0.466667,0.705882}%
\pgfsetfillcolor{currentfill}%
\pgfsetfillopacity{0.632530}%
\pgfsetlinewidth{1.003750pt}%
\definecolor{currentstroke}{rgb}{0.121569,0.466667,0.705882}%
\pgfsetstrokecolor{currentstroke}%
\pgfsetstrokeopacity{0.632530}%
\pgfsetdash{}{0pt}%
\pgfpathmoveto{\pgfqpoint{0.895239in}{1.261559in}}%
\pgfpathcurveto{\pgfqpoint{0.903475in}{1.261559in}}{\pgfqpoint{0.911375in}{1.264831in}}{\pgfqpoint{0.917199in}{1.270655in}}%
\pgfpathcurveto{\pgfqpoint{0.923023in}{1.276479in}}{\pgfqpoint{0.926296in}{1.284379in}}{\pgfqpoint{0.926296in}{1.292616in}}%
\pgfpathcurveto{\pgfqpoint{0.926296in}{1.300852in}}{\pgfqpoint{0.923023in}{1.308752in}}{\pgfqpoint{0.917199in}{1.314576in}}%
\pgfpathcurveto{\pgfqpoint{0.911375in}{1.320400in}}{\pgfqpoint{0.903475in}{1.323672in}}{\pgfqpoint{0.895239in}{1.323672in}}%
\pgfpathcurveto{\pgfqpoint{0.887003in}{1.323672in}}{\pgfqpoint{0.879103in}{1.320400in}}{\pgfqpoint{0.873279in}{1.314576in}}%
\pgfpathcurveto{\pgfqpoint{0.867455in}{1.308752in}}{\pgfqpoint{0.864183in}{1.300852in}}{\pgfqpoint{0.864183in}{1.292616in}}%
\pgfpathcurveto{\pgfqpoint{0.864183in}{1.284379in}}{\pgfqpoint{0.867455in}{1.276479in}}{\pgfqpoint{0.873279in}{1.270655in}}%
\pgfpathcurveto{\pgfqpoint{0.879103in}{1.264831in}}{\pgfqpoint{0.887003in}{1.261559in}}{\pgfqpoint{0.895239in}{1.261559in}}%
\pgfpathclose%
\pgfusepath{stroke,fill}%
\end{pgfscope}%
\begin{pgfscope}%
\pgfpathrectangle{\pgfqpoint{0.100000in}{0.212622in}}{\pgfqpoint{3.696000in}{3.696000in}}%
\pgfusepath{clip}%
\pgfsetbuttcap%
\pgfsetroundjoin%
\definecolor{currentfill}{rgb}{0.121569,0.466667,0.705882}%
\pgfsetfillcolor{currentfill}%
\pgfsetfillopacity{0.632530}%
\pgfsetlinewidth{1.003750pt}%
\definecolor{currentstroke}{rgb}{0.121569,0.466667,0.705882}%
\pgfsetstrokecolor{currentstroke}%
\pgfsetstrokeopacity{0.632530}%
\pgfsetdash{}{0pt}%
\pgfpathmoveto{\pgfqpoint{0.895239in}{1.261559in}}%
\pgfpathcurveto{\pgfqpoint{0.903475in}{1.261559in}}{\pgfqpoint{0.911375in}{1.264831in}}{\pgfqpoint{0.917199in}{1.270655in}}%
\pgfpathcurveto{\pgfqpoint{0.923023in}{1.276479in}}{\pgfqpoint{0.926296in}{1.284379in}}{\pgfqpoint{0.926296in}{1.292616in}}%
\pgfpathcurveto{\pgfqpoint{0.926296in}{1.300852in}}{\pgfqpoint{0.923023in}{1.308752in}}{\pgfqpoint{0.917199in}{1.314576in}}%
\pgfpathcurveto{\pgfqpoint{0.911375in}{1.320400in}}{\pgfqpoint{0.903475in}{1.323672in}}{\pgfqpoint{0.895239in}{1.323672in}}%
\pgfpathcurveto{\pgfqpoint{0.887003in}{1.323672in}}{\pgfqpoint{0.879103in}{1.320400in}}{\pgfqpoint{0.873279in}{1.314576in}}%
\pgfpathcurveto{\pgfqpoint{0.867455in}{1.308752in}}{\pgfqpoint{0.864183in}{1.300852in}}{\pgfqpoint{0.864183in}{1.292616in}}%
\pgfpathcurveto{\pgfqpoint{0.864183in}{1.284379in}}{\pgfqpoint{0.867455in}{1.276479in}}{\pgfqpoint{0.873279in}{1.270655in}}%
\pgfpathcurveto{\pgfqpoint{0.879103in}{1.264831in}}{\pgfqpoint{0.887003in}{1.261559in}}{\pgfqpoint{0.895239in}{1.261559in}}%
\pgfpathclose%
\pgfusepath{stroke,fill}%
\end{pgfscope}%
\begin{pgfscope}%
\pgfpathrectangle{\pgfqpoint{0.100000in}{0.212622in}}{\pgfqpoint{3.696000in}{3.696000in}}%
\pgfusepath{clip}%
\pgfsetbuttcap%
\pgfsetroundjoin%
\definecolor{currentfill}{rgb}{0.121569,0.466667,0.705882}%
\pgfsetfillcolor{currentfill}%
\pgfsetfillopacity{0.632530}%
\pgfsetlinewidth{1.003750pt}%
\definecolor{currentstroke}{rgb}{0.121569,0.466667,0.705882}%
\pgfsetstrokecolor{currentstroke}%
\pgfsetstrokeopacity{0.632530}%
\pgfsetdash{}{0pt}%
\pgfpathmoveto{\pgfqpoint{0.895239in}{1.261559in}}%
\pgfpathcurveto{\pgfqpoint{0.903475in}{1.261559in}}{\pgfqpoint{0.911375in}{1.264831in}}{\pgfqpoint{0.917199in}{1.270655in}}%
\pgfpathcurveto{\pgfqpoint{0.923023in}{1.276479in}}{\pgfqpoint{0.926296in}{1.284379in}}{\pgfqpoint{0.926296in}{1.292616in}}%
\pgfpathcurveto{\pgfqpoint{0.926296in}{1.300852in}}{\pgfqpoint{0.923023in}{1.308752in}}{\pgfqpoint{0.917199in}{1.314576in}}%
\pgfpathcurveto{\pgfqpoint{0.911375in}{1.320400in}}{\pgfqpoint{0.903475in}{1.323672in}}{\pgfqpoint{0.895239in}{1.323672in}}%
\pgfpathcurveto{\pgfqpoint{0.887003in}{1.323672in}}{\pgfqpoint{0.879103in}{1.320400in}}{\pgfqpoint{0.873279in}{1.314576in}}%
\pgfpathcurveto{\pgfqpoint{0.867455in}{1.308752in}}{\pgfqpoint{0.864183in}{1.300852in}}{\pgfqpoint{0.864183in}{1.292616in}}%
\pgfpathcurveto{\pgfqpoint{0.864183in}{1.284379in}}{\pgfqpoint{0.867455in}{1.276479in}}{\pgfqpoint{0.873279in}{1.270655in}}%
\pgfpathcurveto{\pgfqpoint{0.879103in}{1.264831in}}{\pgfqpoint{0.887003in}{1.261559in}}{\pgfqpoint{0.895239in}{1.261559in}}%
\pgfpathclose%
\pgfusepath{stroke,fill}%
\end{pgfscope}%
\begin{pgfscope}%
\pgfpathrectangle{\pgfqpoint{0.100000in}{0.212622in}}{\pgfqpoint{3.696000in}{3.696000in}}%
\pgfusepath{clip}%
\pgfsetbuttcap%
\pgfsetroundjoin%
\definecolor{currentfill}{rgb}{0.121569,0.466667,0.705882}%
\pgfsetfillcolor{currentfill}%
\pgfsetfillopacity{0.632530}%
\pgfsetlinewidth{1.003750pt}%
\definecolor{currentstroke}{rgb}{0.121569,0.466667,0.705882}%
\pgfsetstrokecolor{currentstroke}%
\pgfsetstrokeopacity{0.632530}%
\pgfsetdash{}{0pt}%
\pgfpathmoveto{\pgfqpoint{0.895239in}{1.261559in}}%
\pgfpathcurveto{\pgfqpoint{0.903475in}{1.261559in}}{\pgfqpoint{0.911375in}{1.264831in}}{\pgfqpoint{0.917199in}{1.270655in}}%
\pgfpathcurveto{\pgfqpoint{0.923023in}{1.276479in}}{\pgfqpoint{0.926296in}{1.284379in}}{\pgfqpoint{0.926296in}{1.292616in}}%
\pgfpathcurveto{\pgfqpoint{0.926296in}{1.300852in}}{\pgfqpoint{0.923023in}{1.308752in}}{\pgfqpoint{0.917199in}{1.314576in}}%
\pgfpathcurveto{\pgfqpoint{0.911375in}{1.320400in}}{\pgfqpoint{0.903475in}{1.323672in}}{\pgfqpoint{0.895239in}{1.323672in}}%
\pgfpathcurveto{\pgfqpoint{0.887003in}{1.323672in}}{\pgfqpoint{0.879103in}{1.320400in}}{\pgfqpoint{0.873279in}{1.314576in}}%
\pgfpathcurveto{\pgfqpoint{0.867455in}{1.308752in}}{\pgfqpoint{0.864183in}{1.300852in}}{\pgfqpoint{0.864183in}{1.292616in}}%
\pgfpathcurveto{\pgfqpoint{0.864183in}{1.284379in}}{\pgfqpoint{0.867455in}{1.276479in}}{\pgfqpoint{0.873279in}{1.270655in}}%
\pgfpathcurveto{\pgfqpoint{0.879103in}{1.264831in}}{\pgfqpoint{0.887003in}{1.261559in}}{\pgfqpoint{0.895239in}{1.261559in}}%
\pgfpathclose%
\pgfusepath{stroke,fill}%
\end{pgfscope}%
\begin{pgfscope}%
\pgfpathrectangle{\pgfqpoint{0.100000in}{0.212622in}}{\pgfqpoint{3.696000in}{3.696000in}}%
\pgfusepath{clip}%
\pgfsetbuttcap%
\pgfsetroundjoin%
\definecolor{currentfill}{rgb}{0.121569,0.466667,0.705882}%
\pgfsetfillcolor{currentfill}%
\pgfsetfillopacity{0.632530}%
\pgfsetlinewidth{1.003750pt}%
\definecolor{currentstroke}{rgb}{0.121569,0.466667,0.705882}%
\pgfsetstrokecolor{currentstroke}%
\pgfsetstrokeopacity{0.632530}%
\pgfsetdash{}{0pt}%
\pgfpathmoveto{\pgfqpoint{0.895239in}{1.261559in}}%
\pgfpathcurveto{\pgfqpoint{0.903475in}{1.261559in}}{\pgfqpoint{0.911375in}{1.264831in}}{\pgfqpoint{0.917199in}{1.270655in}}%
\pgfpathcurveto{\pgfqpoint{0.923023in}{1.276479in}}{\pgfqpoint{0.926296in}{1.284379in}}{\pgfqpoint{0.926296in}{1.292616in}}%
\pgfpathcurveto{\pgfqpoint{0.926296in}{1.300852in}}{\pgfqpoint{0.923023in}{1.308752in}}{\pgfqpoint{0.917199in}{1.314576in}}%
\pgfpathcurveto{\pgfqpoint{0.911375in}{1.320400in}}{\pgfqpoint{0.903475in}{1.323672in}}{\pgfqpoint{0.895239in}{1.323672in}}%
\pgfpathcurveto{\pgfqpoint{0.887003in}{1.323672in}}{\pgfqpoint{0.879103in}{1.320400in}}{\pgfqpoint{0.873279in}{1.314576in}}%
\pgfpathcurveto{\pgfqpoint{0.867455in}{1.308752in}}{\pgfqpoint{0.864183in}{1.300852in}}{\pgfqpoint{0.864183in}{1.292616in}}%
\pgfpathcurveto{\pgfqpoint{0.864183in}{1.284379in}}{\pgfqpoint{0.867455in}{1.276479in}}{\pgfqpoint{0.873279in}{1.270655in}}%
\pgfpathcurveto{\pgfqpoint{0.879103in}{1.264831in}}{\pgfqpoint{0.887003in}{1.261559in}}{\pgfqpoint{0.895239in}{1.261559in}}%
\pgfpathclose%
\pgfusepath{stroke,fill}%
\end{pgfscope}%
\begin{pgfscope}%
\pgfpathrectangle{\pgfqpoint{0.100000in}{0.212622in}}{\pgfqpoint{3.696000in}{3.696000in}}%
\pgfusepath{clip}%
\pgfsetbuttcap%
\pgfsetroundjoin%
\definecolor{currentfill}{rgb}{0.121569,0.466667,0.705882}%
\pgfsetfillcolor{currentfill}%
\pgfsetfillopacity{0.632530}%
\pgfsetlinewidth{1.003750pt}%
\definecolor{currentstroke}{rgb}{0.121569,0.466667,0.705882}%
\pgfsetstrokecolor{currentstroke}%
\pgfsetstrokeopacity{0.632530}%
\pgfsetdash{}{0pt}%
\pgfpathmoveto{\pgfqpoint{0.895239in}{1.261559in}}%
\pgfpathcurveto{\pgfqpoint{0.903475in}{1.261559in}}{\pgfqpoint{0.911375in}{1.264831in}}{\pgfqpoint{0.917199in}{1.270655in}}%
\pgfpathcurveto{\pgfqpoint{0.923023in}{1.276479in}}{\pgfqpoint{0.926296in}{1.284379in}}{\pgfqpoint{0.926296in}{1.292616in}}%
\pgfpathcurveto{\pgfqpoint{0.926296in}{1.300852in}}{\pgfqpoint{0.923023in}{1.308752in}}{\pgfqpoint{0.917199in}{1.314576in}}%
\pgfpathcurveto{\pgfqpoint{0.911375in}{1.320400in}}{\pgfqpoint{0.903475in}{1.323672in}}{\pgfqpoint{0.895239in}{1.323672in}}%
\pgfpathcurveto{\pgfqpoint{0.887003in}{1.323672in}}{\pgfqpoint{0.879103in}{1.320400in}}{\pgfqpoint{0.873279in}{1.314576in}}%
\pgfpathcurveto{\pgfqpoint{0.867455in}{1.308752in}}{\pgfqpoint{0.864183in}{1.300852in}}{\pgfqpoint{0.864183in}{1.292616in}}%
\pgfpathcurveto{\pgfqpoint{0.864183in}{1.284379in}}{\pgfqpoint{0.867455in}{1.276479in}}{\pgfqpoint{0.873279in}{1.270655in}}%
\pgfpathcurveto{\pgfqpoint{0.879103in}{1.264831in}}{\pgfqpoint{0.887003in}{1.261559in}}{\pgfqpoint{0.895239in}{1.261559in}}%
\pgfpathclose%
\pgfusepath{stroke,fill}%
\end{pgfscope}%
\begin{pgfscope}%
\pgfpathrectangle{\pgfqpoint{0.100000in}{0.212622in}}{\pgfqpoint{3.696000in}{3.696000in}}%
\pgfusepath{clip}%
\pgfsetbuttcap%
\pgfsetroundjoin%
\definecolor{currentfill}{rgb}{0.121569,0.466667,0.705882}%
\pgfsetfillcolor{currentfill}%
\pgfsetfillopacity{0.632530}%
\pgfsetlinewidth{1.003750pt}%
\definecolor{currentstroke}{rgb}{0.121569,0.466667,0.705882}%
\pgfsetstrokecolor{currentstroke}%
\pgfsetstrokeopacity{0.632530}%
\pgfsetdash{}{0pt}%
\pgfpathmoveto{\pgfqpoint{0.895239in}{1.261559in}}%
\pgfpathcurveto{\pgfqpoint{0.903475in}{1.261559in}}{\pgfqpoint{0.911375in}{1.264831in}}{\pgfqpoint{0.917199in}{1.270655in}}%
\pgfpathcurveto{\pgfqpoint{0.923023in}{1.276479in}}{\pgfqpoint{0.926296in}{1.284379in}}{\pgfqpoint{0.926296in}{1.292616in}}%
\pgfpathcurveto{\pgfqpoint{0.926296in}{1.300852in}}{\pgfqpoint{0.923023in}{1.308752in}}{\pgfqpoint{0.917199in}{1.314576in}}%
\pgfpathcurveto{\pgfqpoint{0.911375in}{1.320400in}}{\pgfqpoint{0.903475in}{1.323672in}}{\pgfqpoint{0.895239in}{1.323672in}}%
\pgfpathcurveto{\pgfqpoint{0.887003in}{1.323672in}}{\pgfqpoint{0.879103in}{1.320400in}}{\pgfqpoint{0.873279in}{1.314576in}}%
\pgfpathcurveto{\pgfqpoint{0.867455in}{1.308752in}}{\pgfqpoint{0.864183in}{1.300852in}}{\pgfqpoint{0.864183in}{1.292616in}}%
\pgfpathcurveto{\pgfqpoint{0.864183in}{1.284379in}}{\pgfqpoint{0.867455in}{1.276479in}}{\pgfqpoint{0.873279in}{1.270655in}}%
\pgfpathcurveto{\pgfqpoint{0.879103in}{1.264831in}}{\pgfqpoint{0.887003in}{1.261559in}}{\pgfqpoint{0.895239in}{1.261559in}}%
\pgfpathclose%
\pgfusepath{stroke,fill}%
\end{pgfscope}%
\begin{pgfscope}%
\pgfpathrectangle{\pgfqpoint{0.100000in}{0.212622in}}{\pgfqpoint{3.696000in}{3.696000in}}%
\pgfusepath{clip}%
\pgfsetbuttcap%
\pgfsetroundjoin%
\definecolor{currentfill}{rgb}{0.121569,0.466667,0.705882}%
\pgfsetfillcolor{currentfill}%
\pgfsetfillopacity{0.632530}%
\pgfsetlinewidth{1.003750pt}%
\definecolor{currentstroke}{rgb}{0.121569,0.466667,0.705882}%
\pgfsetstrokecolor{currentstroke}%
\pgfsetstrokeopacity{0.632530}%
\pgfsetdash{}{0pt}%
\pgfpathmoveto{\pgfqpoint{0.895239in}{1.261559in}}%
\pgfpathcurveto{\pgfqpoint{0.903475in}{1.261559in}}{\pgfqpoint{0.911375in}{1.264831in}}{\pgfqpoint{0.917199in}{1.270655in}}%
\pgfpathcurveto{\pgfqpoint{0.923023in}{1.276479in}}{\pgfqpoint{0.926296in}{1.284379in}}{\pgfqpoint{0.926296in}{1.292616in}}%
\pgfpathcurveto{\pgfqpoint{0.926296in}{1.300852in}}{\pgfqpoint{0.923023in}{1.308752in}}{\pgfqpoint{0.917199in}{1.314576in}}%
\pgfpathcurveto{\pgfqpoint{0.911375in}{1.320400in}}{\pgfqpoint{0.903475in}{1.323672in}}{\pgfqpoint{0.895239in}{1.323672in}}%
\pgfpathcurveto{\pgfqpoint{0.887003in}{1.323672in}}{\pgfqpoint{0.879103in}{1.320400in}}{\pgfqpoint{0.873279in}{1.314576in}}%
\pgfpathcurveto{\pgfqpoint{0.867455in}{1.308752in}}{\pgfqpoint{0.864183in}{1.300852in}}{\pgfqpoint{0.864183in}{1.292616in}}%
\pgfpathcurveto{\pgfqpoint{0.864183in}{1.284379in}}{\pgfqpoint{0.867455in}{1.276479in}}{\pgfqpoint{0.873279in}{1.270655in}}%
\pgfpathcurveto{\pgfqpoint{0.879103in}{1.264831in}}{\pgfqpoint{0.887003in}{1.261559in}}{\pgfqpoint{0.895239in}{1.261559in}}%
\pgfpathclose%
\pgfusepath{stroke,fill}%
\end{pgfscope}%
\begin{pgfscope}%
\pgfpathrectangle{\pgfqpoint{0.100000in}{0.212622in}}{\pgfqpoint{3.696000in}{3.696000in}}%
\pgfusepath{clip}%
\pgfsetbuttcap%
\pgfsetroundjoin%
\definecolor{currentfill}{rgb}{0.121569,0.466667,0.705882}%
\pgfsetfillcolor{currentfill}%
\pgfsetfillopacity{0.632530}%
\pgfsetlinewidth{1.003750pt}%
\definecolor{currentstroke}{rgb}{0.121569,0.466667,0.705882}%
\pgfsetstrokecolor{currentstroke}%
\pgfsetstrokeopacity{0.632530}%
\pgfsetdash{}{0pt}%
\pgfpathmoveto{\pgfqpoint{0.895239in}{1.261559in}}%
\pgfpathcurveto{\pgfqpoint{0.903475in}{1.261559in}}{\pgfqpoint{0.911375in}{1.264831in}}{\pgfqpoint{0.917199in}{1.270655in}}%
\pgfpathcurveto{\pgfqpoint{0.923023in}{1.276479in}}{\pgfqpoint{0.926296in}{1.284379in}}{\pgfqpoint{0.926296in}{1.292616in}}%
\pgfpathcurveto{\pgfqpoint{0.926296in}{1.300852in}}{\pgfqpoint{0.923023in}{1.308752in}}{\pgfqpoint{0.917199in}{1.314576in}}%
\pgfpathcurveto{\pgfqpoint{0.911375in}{1.320400in}}{\pgfqpoint{0.903475in}{1.323672in}}{\pgfqpoint{0.895239in}{1.323672in}}%
\pgfpathcurveto{\pgfqpoint{0.887003in}{1.323672in}}{\pgfqpoint{0.879103in}{1.320400in}}{\pgfqpoint{0.873279in}{1.314576in}}%
\pgfpathcurveto{\pgfqpoint{0.867455in}{1.308752in}}{\pgfqpoint{0.864183in}{1.300852in}}{\pgfqpoint{0.864183in}{1.292616in}}%
\pgfpathcurveto{\pgfqpoint{0.864183in}{1.284379in}}{\pgfqpoint{0.867455in}{1.276479in}}{\pgfqpoint{0.873279in}{1.270655in}}%
\pgfpathcurveto{\pgfqpoint{0.879103in}{1.264831in}}{\pgfqpoint{0.887003in}{1.261559in}}{\pgfqpoint{0.895239in}{1.261559in}}%
\pgfpathclose%
\pgfusepath{stroke,fill}%
\end{pgfscope}%
\begin{pgfscope}%
\pgfpathrectangle{\pgfqpoint{0.100000in}{0.212622in}}{\pgfqpoint{3.696000in}{3.696000in}}%
\pgfusepath{clip}%
\pgfsetbuttcap%
\pgfsetroundjoin%
\definecolor{currentfill}{rgb}{0.121569,0.466667,0.705882}%
\pgfsetfillcolor{currentfill}%
\pgfsetfillopacity{0.632530}%
\pgfsetlinewidth{1.003750pt}%
\definecolor{currentstroke}{rgb}{0.121569,0.466667,0.705882}%
\pgfsetstrokecolor{currentstroke}%
\pgfsetstrokeopacity{0.632530}%
\pgfsetdash{}{0pt}%
\pgfpathmoveto{\pgfqpoint{0.895239in}{1.261559in}}%
\pgfpathcurveto{\pgfqpoint{0.903475in}{1.261559in}}{\pgfqpoint{0.911375in}{1.264831in}}{\pgfqpoint{0.917199in}{1.270655in}}%
\pgfpathcurveto{\pgfqpoint{0.923023in}{1.276479in}}{\pgfqpoint{0.926296in}{1.284379in}}{\pgfqpoint{0.926296in}{1.292616in}}%
\pgfpathcurveto{\pgfqpoint{0.926296in}{1.300852in}}{\pgfqpoint{0.923023in}{1.308752in}}{\pgfqpoint{0.917199in}{1.314576in}}%
\pgfpathcurveto{\pgfqpoint{0.911375in}{1.320400in}}{\pgfqpoint{0.903475in}{1.323672in}}{\pgfqpoint{0.895239in}{1.323672in}}%
\pgfpathcurveto{\pgfqpoint{0.887003in}{1.323672in}}{\pgfqpoint{0.879103in}{1.320400in}}{\pgfqpoint{0.873279in}{1.314576in}}%
\pgfpathcurveto{\pgfqpoint{0.867455in}{1.308752in}}{\pgfqpoint{0.864183in}{1.300852in}}{\pgfqpoint{0.864183in}{1.292616in}}%
\pgfpathcurveto{\pgfqpoint{0.864183in}{1.284379in}}{\pgfqpoint{0.867455in}{1.276479in}}{\pgfqpoint{0.873279in}{1.270655in}}%
\pgfpathcurveto{\pgfqpoint{0.879103in}{1.264831in}}{\pgfqpoint{0.887003in}{1.261559in}}{\pgfqpoint{0.895239in}{1.261559in}}%
\pgfpathclose%
\pgfusepath{stroke,fill}%
\end{pgfscope}%
\begin{pgfscope}%
\pgfpathrectangle{\pgfqpoint{0.100000in}{0.212622in}}{\pgfqpoint{3.696000in}{3.696000in}}%
\pgfusepath{clip}%
\pgfsetbuttcap%
\pgfsetroundjoin%
\definecolor{currentfill}{rgb}{0.121569,0.466667,0.705882}%
\pgfsetfillcolor{currentfill}%
\pgfsetfillopacity{0.632530}%
\pgfsetlinewidth{1.003750pt}%
\definecolor{currentstroke}{rgb}{0.121569,0.466667,0.705882}%
\pgfsetstrokecolor{currentstroke}%
\pgfsetstrokeopacity{0.632530}%
\pgfsetdash{}{0pt}%
\pgfpathmoveto{\pgfqpoint{0.895239in}{1.261559in}}%
\pgfpathcurveto{\pgfqpoint{0.903475in}{1.261559in}}{\pgfqpoint{0.911375in}{1.264831in}}{\pgfqpoint{0.917199in}{1.270655in}}%
\pgfpathcurveto{\pgfqpoint{0.923023in}{1.276479in}}{\pgfqpoint{0.926296in}{1.284379in}}{\pgfqpoint{0.926296in}{1.292616in}}%
\pgfpathcurveto{\pgfqpoint{0.926296in}{1.300852in}}{\pgfqpoint{0.923023in}{1.308752in}}{\pgfqpoint{0.917199in}{1.314576in}}%
\pgfpathcurveto{\pgfqpoint{0.911375in}{1.320400in}}{\pgfqpoint{0.903475in}{1.323672in}}{\pgfqpoint{0.895239in}{1.323672in}}%
\pgfpathcurveto{\pgfqpoint{0.887003in}{1.323672in}}{\pgfqpoint{0.879103in}{1.320400in}}{\pgfqpoint{0.873279in}{1.314576in}}%
\pgfpathcurveto{\pgfqpoint{0.867455in}{1.308752in}}{\pgfqpoint{0.864183in}{1.300852in}}{\pgfqpoint{0.864183in}{1.292616in}}%
\pgfpathcurveto{\pgfqpoint{0.864183in}{1.284379in}}{\pgfqpoint{0.867455in}{1.276479in}}{\pgfqpoint{0.873279in}{1.270655in}}%
\pgfpathcurveto{\pgfqpoint{0.879103in}{1.264831in}}{\pgfqpoint{0.887003in}{1.261559in}}{\pgfqpoint{0.895239in}{1.261559in}}%
\pgfpathclose%
\pgfusepath{stroke,fill}%
\end{pgfscope}%
\begin{pgfscope}%
\pgfpathrectangle{\pgfqpoint{0.100000in}{0.212622in}}{\pgfqpoint{3.696000in}{3.696000in}}%
\pgfusepath{clip}%
\pgfsetbuttcap%
\pgfsetroundjoin%
\definecolor{currentfill}{rgb}{0.121569,0.466667,0.705882}%
\pgfsetfillcolor{currentfill}%
\pgfsetfillopacity{0.632530}%
\pgfsetlinewidth{1.003750pt}%
\definecolor{currentstroke}{rgb}{0.121569,0.466667,0.705882}%
\pgfsetstrokecolor{currentstroke}%
\pgfsetstrokeopacity{0.632530}%
\pgfsetdash{}{0pt}%
\pgfpathmoveto{\pgfqpoint{0.895239in}{1.261559in}}%
\pgfpathcurveto{\pgfqpoint{0.903475in}{1.261559in}}{\pgfqpoint{0.911375in}{1.264831in}}{\pgfqpoint{0.917199in}{1.270655in}}%
\pgfpathcurveto{\pgfqpoint{0.923023in}{1.276479in}}{\pgfqpoint{0.926296in}{1.284379in}}{\pgfqpoint{0.926296in}{1.292616in}}%
\pgfpathcurveto{\pgfqpoint{0.926296in}{1.300852in}}{\pgfqpoint{0.923023in}{1.308752in}}{\pgfqpoint{0.917199in}{1.314576in}}%
\pgfpathcurveto{\pgfqpoint{0.911375in}{1.320400in}}{\pgfqpoint{0.903475in}{1.323672in}}{\pgfqpoint{0.895239in}{1.323672in}}%
\pgfpathcurveto{\pgfqpoint{0.887003in}{1.323672in}}{\pgfqpoint{0.879103in}{1.320400in}}{\pgfqpoint{0.873279in}{1.314576in}}%
\pgfpathcurveto{\pgfqpoint{0.867455in}{1.308752in}}{\pgfqpoint{0.864183in}{1.300852in}}{\pgfqpoint{0.864183in}{1.292616in}}%
\pgfpathcurveto{\pgfqpoint{0.864183in}{1.284379in}}{\pgfqpoint{0.867455in}{1.276479in}}{\pgfqpoint{0.873279in}{1.270655in}}%
\pgfpathcurveto{\pgfqpoint{0.879103in}{1.264831in}}{\pgfqpoint{0.887003in}{1.261559in}}{\pgfqpoint{0.895239in}{1.261559in}}%
\pgfpathclose%
\pgfusepath{stroke,fill}%
\end{pgfscope}%
\begin{pgfscope}%
\pgfpathrectangle{\pgfqpoint{0.100000in}{0.212622in}}{\pgfqpoint{3.696000in}{3.696000in}}%
\pgfusepath{clip}%
\pgfsetbuttcap%
\pgfsetroundjoin%
\definecolor{currentfill}{rgb}{0.121569,0.466667,0.705882}%
\pgfsetfillcolor{currentfill}%
\pgfsetfillopacity{0.632530}%
\pgfsetlinewidth{1.003750pt}%
\definecolor{currentstroke}{rgb}{0.121569,0.466667,0.705882}%
\pgfsetstrokecolor{currentstroke}%
\pgfsetstrokeopacity{0.632530}%
\pgfsetdash{}{0pt}%
\pgfpathmoveto{\pgfqpoint{0.895239in}{1.261559in}}%
\pgfpathcurveto{\pgfqpoint{0.903475in}{1.261559in}}{\pgfqpoint{0.911375in}{1.264831in}}{\pgfqpoint{0.917199in}{1.270655in}}%
\pgfpathcurveto{\pgfqpoint{0.923023in}{1.276479in}}{\pgfqpoint{0.926296in}{1.284379in}}{\pgfqpoint{0.926296in}{1.292616in}}%
\pgfpathcurveto{\pgfqpoint{0.926296in}{1.300852in}}{\pgfqpoint{0.923023in}{1.308752in}}{\pgfqpoint{0.917199in}{1.314576in}}%
\pgfpathcurveto{\pgfqpoint{0.911375in}{1.320400in}}{\pgfqpoint{0.903475in}{1.323672in}}{\pgfqpoint{0.895239in}{1.323672in}}%
\pgfpathcurveto{\pgfqpoint{0.887003in}{1.323672in}}{\pgfqpoint{0.879103in}{1.320400in}}{\pgfqpoint{0.873279in}{1.314576in}}%
\pgfpathcurveto{\pgfqpoint{0.867455in}{1.308752in}}{\pgfqpoint{0.864183in}{1.300852in}}{\pgfqpoint{0.864183in}{1.292616in}}%
\pgfpathcurveto{\pgfqpoint{0.864183in}{1.284379in}}{\pgfqpoint{0.867455in}{1.276479in}}{\pgfqpoint{0.873279in}{1.270655in}}%
\pgfpathcurveto{\pgfqpoint{0.879103in}{1.264831in}}{\pgfqpoint{0.887003in}{1.261559in}}{\pgfqpoint{0.895239in}{1.261559in}}%
\pgfpathclose%
\pgfusepath{stroke,fill}%
\end{pgfscope}%
\begin{pgfscope}%
\pgfpathrectangle{\pgfqpoint{0.100000in}{0.212622in}}{\pgfqpoint{3.696000in}{3.696000in}}%
\pgfusepath{clip}%
\pgfsetbuttcap%
\pgfsetroundjoin%
\definecolor{currentfill}{rgb}{0.121569,0.466667,0.705882}%
\pgfsetfillcolor{currentfill}%
\pgfsetfillopacity{0.632530}%
\pgfsetlinewidth{1.003750pt}%
\definecolor{currentstroke}{rgb}{0.121569,0.466667,0.705882}%
\pgfsetstrokecolor{currentstroke}%
\pgfsetstrokeopacity{0.632530}%
\pgfsetdash{}{0pt}%
\pgfpathmoveto{\pgfqpoint{0.895239in}{1.261559in}}%
\pgfpathcurveto{\pgfqpoint{0.903475in}{1.261559in}}{\pgfqpoint{0.911375in}{1.264831in}}{\pgfqpoint{0.917199in}{1.270655in}}%
\pgfpathcurveto{\pgfqpoint{0.923023in}{1.276479in}}{\pgfqpoint{0.926296in}{1.284379in}}{\pgfqpoint{0.926296in}{1.292616in}}%
\pgfpathcurveto{\pgfqpoint{0.926296in}{1.300852in}}{\pgfqpoint{0.923023in}{1.308752in}}{\pgfqpoint{0.917199in}{1.314576in}}%
\pgfpathcurveto{\pgfqpoint{0.911375in}{1.320400in}}{\pgfqpoint{0.903475in}{1.323672in}}{\pgfqpoint{0.895239in}{1.323672in}}%
\pgfpathcurveto{\pgfqpoint{0.887003in}{1.323672in}}{\pgfqpoint{0.879103in}{1.320400in}}{\pgfqpoint{0.873279in}{1.314576in}}%
\pgfpathcurveto{\pgfqpoint{0.867455in}{1.308752in}}{\pgfqpoint{0.864183in}{1.300852in}}{\pgfqpoint{0.864183in}{1.292616in}}%
\pgfpathcurveto{\pgfqpoint{0.864183in}{1.284379in}}{\pgfqpoint{0.867455in}{1.276479in}}{\pgfqpoint{0.873279in}{1.270655in}}%
\pgfpathcurveto{\pgfqpoint{0.879103in}{1.264831in}}{\pgfqpoint{0.887003in}{1.261559in}}{\pgfqpoint{0.895239in}{1.261559in}}%
\pgfpathclose%
\pgfusepath{stroke,fill}%
\end{pgfscope}%
\begin{pgfscope}%
\pgfpathrectangle{\pgfqpoint{0.100000in}{0.212622in}}{\pgfqpoint{3.696000in}{3.696000in}}%
\pgfusepath{clip}%
\pgfsetbuttcap%
\pgfsetroundjoin%
\definecolor{currentfill}{rgb}{0.121569,0.466667,0.705882}%
\pgfsetfillcolor{currentfill}%
\pgfsetfillopacity{0.632530}%
\pgfsetlinewidth{1.003750pt}%
\definecolor{currentstroke}{rgb}{0.121569,0.466667,0.705882}%
\pgfsetstrokecolor{currentstroke}%
\pgfsetstrokeopacity{0.632530}%
\pgfsetdash{}{0pt}%
\pgfpathmoveto{\pgfqpoint{0.895239in}{1.261559in}}%
\pgfpathcurveto{\pgfqpoint{0.903475in}{1.261559in}}{\pgfqpoint{0.911375in}{1.264831in}}{\pgfqpoint{0.917199in}{1.270655in}}%
\pgfpathcurveto{\pgfqpoint{0.923023in}{1.276479in}}{\pgfqpoint{0.926296in}{1.284379in}}{\pgfqpoint{0.926296in}{1.292616in}}%
\pgfpathcurveto{\pgfqpoint{0.926296in}{1.300852in}}{\pgfqpoint{0.923023in}{1.308752in}}{\pgfqpoint{0.917199in}{1.314576in}}%
\pgfpathcurveto{\pgfqpoint{0.911375in}{1.320400in}}{\pgfqpoint{0.903475in}{1.323672in}}{\pgfqpoint{0.895239in}{1.323672in}}%
\pgfpathcurveto{\pgfqpoint{0.887003in}{1.323672in}}{\pgfqpoint{0.879103in}{1.320400in}}{\pgfqpoint{0.873279in}{1.314576in}}%
\pgfpathcurveto{\pgfqpoint{0.867455in}{1.308752in}}{\pgfqpoint{0.864183in}{1.300852in}}{\pgfqpoint{0.864183in}{1.292616in}}%
\pgfpathcurveto{\pgfqpoint{0.864183in}{1.284379in}}{\pgfqpoint{0.867455in}{1.276479in}}{\pgfqpoint{0.873279in}{1.270655in}}%
\pgfpathcurveto{\pgfqpoint{0.879103in}{1.264831in}}{\pgfqpoint{0.887003in}{1.261559in}}{\pgfqpoint{0.895239in}{1.261559in}}%
\pgfpathclose%
\pgfusepath{stroke,fill}%
\end{pgfscope}%
\begin{pgfscope}%
\pgfpathrectangle{\pgfqpoint{0.100000in}{0.212622in}}{\pgfqpoint{3.696000in}{3.696000in}}%
\pgfusepath{clip}%
\pgfsetbuttcap%
\pgfsetroundjoin%
\definecolor{currentfill}{rgb}{0.121569,0.466667,0.705882}%
\pgfsetfillcolor{currentfill}%
\pgfsetfillopacity{0.632530}%
\pgfsetlinewidth{1.003750pt}%
\definecolor{currentstroke}{rgb}{0.121569,0.466667,0.705882}%
\pgfsetstrokecolor{currentstroke}%
\pgfsetstrokeopacity{0.632530}%
\pgfsetdash{}{0pt}%
\pgfpathmoveto{\pgfqpoint{0.895239in}{1.261559in}}%
\pgfpathcurveto{\pgfqpoint{0.903475in}{1.261559in}}{\pgfqpoint{0.911375in}{1.264831in}}{\pgfqpoint{0.917199in}{1.270655in}}%
\pgfpathcurveto{\pgfqpoint{0.923023in}{1.276479in}}{\pgfqpoint{0.926296in}{1.284379in}}{\pgfqpoint{0.926296in}{1.292616in}}%
\pgfpathcurveto{\pgfqpoint{0.926296in}{1.300852in}}{\pgfqpoint{0.923023in}{1.308752in}}{\pgfqpoint{0.917199in}{1.314576in}}%
\pgfpathcurveto{\pgfqpoint{0.911375in}{1.320400in}}{\pgfqpoint{0.903475in}{1.323672in}}{\pgfqpoint{0.895239in}{1.323672in}}%
\pgfpathcurveto{\pgfqpoint{0.887003in}{1.323672in}}{\pgfqpoint{0.879103in}{1.320400in}}{\pgfqpoint{0.873279in}{1.314576in}}%
\pgfpathcurveto{\pgfqpoint{0.867455in}{1.308752in}}{\pgfqpoint{0.864183in}{1.300852in}}{\pgfqpoint{0.864183in}{1.292616in}}%
\pgfpathcurveto{\pgfqpoint{0.864183in}{1.284379in}}{\pgfqpoint{0.867455in}{1.276479in}}{\pgfqpoint{0.873279in}{1.270655in}}%
\pgfpathcurveto{\pgfqpoint{0.879103in}{1.264831in}}{\pgfqpoint{0.887003in}{1.261559in}}{\pgfqpoint{0.895239in}{1.261559in}}%
\pgfpathclose%
\pgfusepath{stroke,fill}%
\end{pgfscope}%
\begin{pgfscope}%
\pgfpathrectangle{\pgfqpoint{0.100000in}{0.212622in}}{\pgfqpoint{3.696000in}{3.696000in}}%
\pgfusepath{clip}%
\pgfsetbuttcap%
\pgfsetroundjoin%
\definecolor{currentfill}{rgb}{0.121569,0.466667,0.705882}%
\pgfsetfillcolor{currentfill}%
\pgfsetfillopacity{0.632530}%
\pgfsetlinewidth{1.003750pt}%
\definecolor{currentstroke}{rgb}{0.121569,0.466667,0.705882}%
\pgfsetstrokecolor{currentstroke}%
\pgfsetstrokeopacity{0.632530}%
\pgfsetdash{}{0pt}%
\pgfpathmoveto{\pgfqpoint{0.895239in}{1.261559in}}%
\pgfpathcurveto{\pgfqpoint{0.903475in}{1.261559in}}{\pgfqpoint{0.911375in}{1.264831in}}{\pgfqpoint{0.917199in}{1.270655in}}%
\pgfpathcurveto{\pgfqpoint{0.923023in}{1.276479in}}{\pgfqpoint{0.926296in}{1.284379in}}{\pgfqpoint{0.926296in}{1.292616in}}%
\pgfpathcurveto{\pgfqpoint{0.926296in}{1.300852in}}{\pgfqpoint{0.923023in}{1.308752in}}{\pgfqpoint{0.917199in}{1.314576in}}%
\pgfpathcurveto{\pgfqpoint{0.911375in}{1.320400in}}{\pgfqpoint{0.903475in}{1.323672in}}{\pgfqpoint{0.895239in}{1.323672in}}%
\pgfpathcurveto{\pgfqpoint{0.887003in}{1.323672in}}{\pgfqpoint{0.879103in}{1.320400in}}{\pgfqpoint{0.873279in}{1.314576in}}%
\pgfpathcurveto{\pgfqpoint{0.867455in}{1.308752in}}{\pgfqpoint{0.864183in}{1.300852in}}{\pgfqpoint{0.864183in}{1.292616in}}%
\pgfpathcurveto{\pgfqpoint{0.864183in}{1.284379in}}{\pgfqpoint{0.867455in}{1.276479in}}{\pgfqpoint{0.873279in}{1.270655in}}%
\pgfpathcurveto{\pgfqpoint{0.879103in}{1.264831in}}{\pgfqpoint{0.887003in}{1.261559in}}{\pgfqpoint{0.895239in}{1.261559in}}%
\pgfpathclose%
\pgfusepath{stroke,fill}%
\end{pgfscope}%
\begin{pgfscope}%
\pgfpathrectangle{\pgfqpoint{0.100000in}{0.212622in}}{\pgfqpoint{3.696000in}{3.696000in}}%
\pgfusepath{clip}%
\pgfsetbuttcap%
\pgfsetroundjoin%
\definecolor{currentfill}{rgb}{0.121569,0.466667,0.705882}%
\pgfsetfillcolor{currentfill}%
\pgfsetfillopacity{0.632530}%
\pgfsetlinewidth{1.003750pt}%
\definecolor{currentstroke}{rgb}{0.121569,0.466667,0.705882}%
\pgfsetstrokecolor{currentstroke}%
\pgfsetstrokeopacity{0.632530}%
\pgfsetdash{}{0pt}%
\pgfpathmoveto{\pgfqpoint{0.895239in}{1.261559in}}%
\pgfpathcurveto{\pgfqpoint{0.903475in}{1.261559in}}{\pgfqpoint{0.911375in}{1.264831in}}{\pgfqpoint{0.917199in}{1.270655in}}%
\pgfpathcurveto{\pgfqpoint{0.923023in}{1.276479in}}{\pgfqpoint{0.926296in}{1.284379in}}{\pgfqpoint{0.926296in}{1.292616in}}%
\pgfpathcurveto{\pgfqpoint{0.926296in}{1.300852in}}{\pgfqpoint{0.923023in}{1.308752in}}{\pgfqpoint{0.917199in}{1.314576in}}%
\pgfpathcurveto{\pgfqpoint{0.911375in}{1.320400in}}{\pgfqpoint{0.903475in}{1.323672in}}{\pgfqpoint{0.895239in}{1.323672in}}%
\pgfpathcurveto{\pgfqpoint{0.887003in}{1.323672in}}{\pgfqpoint{0.879103in}{1.320400in}}{\pgfqpoint{0.873279in}{1.314576in}}%
\pgfpathcurveto{\pgfqpoint{0.867455in}{1.308752in}}{\pgfqpoint{0.864183in}{1.300852in}}{\pgfqpoint{0.864183in}{1.292616in}}%
\pgfpathcurveto{\pgfqpoint{0.864183in}{1.284379in}}{\pgfqpoint{0.867455in}{1.276479in}}{\pgfqpoint{0.873279in}{1.270655in}}%
\pgfpathcurveto{\pgfqpoint{0.879103in}{1.264831in}}{\pgfqpoint{0.887003in}{1.261559in}}{\pgfqpoint{0.895239in}{1.261559in}}%
\pgfpathclose%
\pgfusepath{stroke,fill}%
\end{pgfscope}%
\begin{pgfscope}%
\pgfpathrectangle{\pgfqpoint{0.100000in}{0.212622in}}{\pgfqpoint{3.696000in}{3.696000in}}%
\pgfusepath{clip}%
\pgfsetbuttcap%
\pgfsetroundjoin%
\definecolor{currentfill}{rgb}{0.121569,0.466667,0.705882}%
\pgfsetfillcolor{currentfill}%
\pgfsetfillopacity{0.632530}%
\pgfsetlinewidth{1.003750pt}%
\definecolor{currentstroke}{rgb}{0.121569,0.466667,0.705882}%
\pgfsetstrokecolor{currentstroke}%
\pgfsetstrokeopacity{0.632530}%
\pgfsetdash{}{0pt}%
\pgfpathmoveto{\pgfqpoint{0.895239in}{1.261559in}}%
\pgfpathcurveto{\pgfqpoint{0.903475in}{1.261559in}}{\pgfqpoint{0.911375in}{1.264831in}}{\pgfqpoint{0.917199in}{1.270655in}}%
\pgfpathcurveto{\pgfqpoint{0.923023in}{1.276479in}}{\pgfqpoint{0.926296in}{1.284379in}}{\pgfqpoint{0.926296in}{1.292616in}}%
\pgfpathcurveto{\pgfqpoint{0.926296in}{1.300852in}}{\pgfqpoint{0.923023in}{1.308752in}}{\pgfqpoint{0.917199in}{1.314576in}}%
\pgfpathcurveto{\pgfqpoint{0.911375in}{1.320400in}}{\pgfqpoint{0.903475in}{1.323672in}}{\pgfqpoint{0.895239in}{1.323672in}}%
\pgfpathcurveto{\pgfqpoint{0.887003in}{1.323672in}}{\pgfqpoint{0.879103in}{1.320400in}}{\pgfqpoint{0.873279in}{1.314576in}}%
\pgfpathcurveto{\pgfqpoint{0.867455in}{1.308752in}}{\pgfqpoint{0.864183in}{1.300852in}}{\pgfqpoint{0.864183in}{1.292616in}}%
\pgfpathcurveto{\pgfqpoint{0.864183in}{1.284379in}}{\pgfqpoint{0.867455in}{1.276479in}}{\pgfqpoint{0.873279in}{1.270655in}}%
\pgfpathcurveto{\pgfqpoint{0.879103in}{1.264831in}}{\pgfqpoint{0.887003in}{1.261559in}}{\pgfqpoint{0.895239in}{1.261559in}}%
\pgfpathclose%
\pgfusepath{stroke,fill}%
\end{pgfscope}%
\begin{pgfscope}%
\pgfpathrectangle{\pgfqpoint{0.100000in}{0.212622in}}{\pgfqpoint{3.696000in}{3.696000in}}%
\pgfusepath{clip}%
\pgfsetbuttcap%
\pgfsetroundjoin%
\definecolor{currentfill}{rgb}{0.121569,0.466667,0.705882}%
\pgfsetfillcolor{currentfill}%
\pgfsetfillopacity{0.632530}%
\pgfsetlinewidth{1.003750pt}%
\definecolor{currentstroke}{rgb}{0.121569,0.466667,0.705882}%
\pgfsetstrokecolor{currentstroke}%
\pgfsetstrokeopacity{0.632530}%
\pgfsetdash{}{0pt}%
\pgfpathmoveto{\pgfqpoint{0.895239in}{1.261559in}}%
\pgfpathcurveto{\pgfqpoint{0.903475in}{1.261559in}}{\pgfqpoint{0.911375in}{1.264831in}}{\pgfqpoint{0.917199in}{1.270655in}}%
\pgfpathcurveto{\pgfqpoint{0.923023in}{1.276479in}}{\pgfqpoint{0.926296in}{1.284379in}}{\pgfqpoint{0.926296in}{1.292616in}}%
\pgfpathcurveto{\pgfqpoint{0.926296in}{1.300852in}}{\pgfqpoint{0.923023in}{1.308752in}}{\pgfqpoint{0.917199in}{1.314576in}}%
\pgfpathcurveto{\pgfqpoint{0.911375in}{1.320400in}}{\pgfqpoint{0.903475in}{1.323672in}}{\pgfqpoint{0.895239in}{1.323672in}}%
\pgfpathcurveto{\pgfqpoint{0.887003in}{1.323672in}}{\pgfqpoint{0.879103in}{1.320400in}}{\pgfqpoint{0.873279in}{1.314576in}}%
\pgfpathcurveto{\pgfqpoint{0.867455in}{1.308752in}}{\pgfqpoint{0.864183in}{1.300852in}}{\pgfqpoint{0.864183in}{1.292616in}}%
\pgfpathcurveto{\pgfqpoint{0.864183in}{1.284379in}}{\pgfqpoint{0.867455in}{1.276479in}}{\pgfqpoint{0.873279in}{1.270655in}}%
\pgfpathcurveto{\pgfqpoint{0.879103in}{1.264831in}}{\pgfqpoint{0.887003in}{1.261559in}}{\pgfqpoint{0.895239in}{1.261559in}}%
\pgfpathclose%
\pgfusepath{stroke,fill}%
\end{pgfscope}%
\begin{pgfscope}%
\pgfpathrectangle{\pgfqpoint{0.100000in}{0.212622in}}{\pgfqpoint{3.696000in}{3.696000in}}%
\pgfusepath{clip}%
\pgfsetbuttcap%
\pgfsetroundjoin%
\definecolor{currentfill}{rgb}{0.121569,0.466667,0.705882}%
\pgfsetfillcolor{currentfill}%
\pgfsetfillopacity{0.632530}%
\pgfsetlinewidth{1.003750pt}%
\definecolor{currentstroke}{rgb}{0.121569,0.466667,0.705882}%
\pgfsetstrokecolor{currentstroke}%
\pgfsetstrokeopacity{0.632530}%
\pgfsetdash{}{0pt}%
\pgfpathmoveto{\pgfqpoint{0.895239in}{1.261559in}}%
\pgfpathcurveto{\pgfqpoint{0.903475in}{1.261559in}}{\pgfqpoint{0.911375in}{1.264831in}}{\pgfqpoint{0.917199in}{1.270655in}}%
\pgfpathcurveto{\pgfqpoint{0.923023in}{1.276479in}}{\pgfqpoint{0.926296in}{1.284379in}}{\pgfqpoint{0.926296in}{1.292616in}}%
\pgfpathcurveto{\pgfqpoint{0.926296in}{1.300852in}}{\pgfqpoint{0.923023in}{1.308752in}}{\pgfqpoint{0.917199in}{1.314576in}}%
\pgfpathcurveto{\pgfqpoint{0.911375in}{1.320400in}}{\pgfqpoint{0.903475in}{1.323672in}}{\pgfqpoint{0.895239in}{1.323672in}}%
\pgfpathcurveto{\pgfqpoint{0.887003in}{1.323672in}}{\pgfqpoint{0.879103in}{1.320400in}}{\pgfqpoint{0.873279in}{1.314576in}}%
\pgfpathcurveto{\pgfqpoint{0.867455in}{1.308752in}}{\pgfqpoint{0.864183in}{1.300852in}}{\pgfqpoint{0.864183in}{1.292616in}}%
\pgfpathcurveto{\pgfqpoint{0.864183in}{1.284379in}}{\pgfqpoint{0.867455in}{1.276479in}}{\pgfqpoint{0.873279in}{1.270655in}}%
\pgfpathcurveto{\pgfqpoint{0.879103in}{1.264831in}}{\pgfqpoint{0.887003in}{1.261559in}}{\pgfqpoint{0.895239in}{1.261559in}}%
\pgfpathclose%
\pgfusepath{stroke,fill}%
\end{pgfscope}%
\begin{pgfscope}%
\pgfpathrectangle{\pgfqpoint{0.100000in}{0.212622in}}{\pgfqpoint{3.696000in}{3.696000in}}%
\pgfusepath{clip}%
\pgfsetbuttcap%
\pgfsetroundjoin%
\definecolor{currentfill}{rgb}{0.121569,0.466667,0.705882}%
\pgfsetfillcolor{currentfill}%
\pgfsetfillopacity{0.632530}%
\pgfsetlinewidth{1.003750pt}%
\definecolor{currentstroke}{rgb}{0.121569,0.466667,0.705882}%
\pgfsetstrokecolor{currentstroke}%
\pgfsetstrokeopacity{0.632530}%
\pgfsetdash{}{0pt}%
\pgfpathmoveto{\pgfqpoint{0.895239in}{1.261559in}}%
\pgfpathcurveto{\pgfqpoint{0.903475in}{1.261559in}}{\pgfqpoint{0.911375in}{1.264831in}}{\pgfqpoint{0.917199in}{1.270655in}}%
\pgfpathcurveto{\pgfqpoint{0.923023in}{1.276479in}}{\pgfqpoint{0.926296in}{1.284379in}}{\pgfqpoint{0.926296in}{1.292616in}}%
\pgfpathcurveto{\pgfqpoint{0.926296in}{1.300852in}}{\pgfqpoint{0.923023in}{1.308752in}}{\pgfqpoint{0.917199in}{1.314576in}}%
\pgfpathcurveto{\pgfqpoint{0.911375in}{1.320400in}}{\pgfqpoint{0.903475in}{1.323672in}}{\pgfqpoint{0.895239in}{1.323672in}}%
\pgfpathcurveto{\pgfqpoint{0.887003in}{1.323672in}}{\pgfqpoint{0.879103in}{1.320400in}}{\pgfqpoint{0.873279in}{1.314576in}}%
\pgfpathcurveto{\pgfqpoint{0.867455in}{1.308752in}}{\pgfqpoint{0.864183in}{1.300852in}}{\pgfqpoint{0.864183in}{1.292616in}}%
\pgfpathcurveto{\pgfqpoint{0.864183in}{1.284379in}}{\pgfqpoint{0.867455in}{1.276479in}}{\pgfqpoint{0.873279in}{1.270655in}}%
\pgfpathcurveto{\pgfqpoint{0.879103in}{1.264831in}}{\pgfqpoint{0.887003in}{1.261559in}}{\pgfqpoint{0.895239in}{1.261559in}}%
\pgfpathclose%
\pgfusepath{stroke,fill}%
\end{pgfscope}%
\begin{pgfscope}%
\pgfpathrectangle{\pgfqpoint{0.100000in}{0.212622in}}{\pgfqpoint{3.696000in}{3.696000in}}%
\pgfusepath{clip}%
\pgfsetbuttcap%
\pgfsetroundjoin%
\definecolor{currentfill}{rgb}{0.121569,0.466667,0.705882}%
\pgfsetfillcolor{currentfill}%
\pgfsetfillopacity{0.632530}%
\pgfsetlinewidth{1.003750pt}%
\definecolor{currentstroke}{rgb}{0.121569,0.466667,0.705882}%
\pgfsetstrokecolor{currentstroke}%
\pgfsetstrokeopacity{0.632530}%
\pgfsetdash{}{0pt}%
\pgfpathmoveto{\pgfqpoint{0.895239in}{1.261559in}}%
\pgfpathcurveto{\pgfqpoint{0.903475in}{1.261559in}}{\pgfqpoint{0.911375in}{1.264831in}}{\pgfqpoint{0.917199in}{1.270655in}}%
\pgfpathcurveto{\pgfqpoint{0.923023in}{1.276479in}}{\pgfqpoint{0.926296in}{1.284379in}}{\pgfqpoint{0.926296in}{1.292616in}}%
\pgfpathcurveto{\pgfqpoint{0.926296in}{1.300852in}}{\pgfqpoint{0.923023in}{1.308752in}}{\pgfqpoint{0.917199in}{1.314576in}}%
\pgfpathcurveto{\pgfqpoint{0.911375in}{1.320400in}}{\pgfqpoint{0.903475in}{1.323672in}}{\pgfqpoint{0.895239in}{1.323672in}}%
\pgfpathcurveto{\pgfqpoint{0.887003in}{1.323672in}}{\pgfqpoint{0.879103in}{1.320400in}}{\pgfqpoint{0.873279in}{1.314576in}}%
\pgfpathcurveto{\pgfqpoint{0.867455in}{1.308752in}}{\pgfqpoint{0.864183in}{1.300852in}}{\pgfqpoint{0.864183in}{1.292616in}}%
\pgfpathcurveto{\pgfqpoint{0.864183in}{1.284379in}}{\pgfqpoint{0.867455in}{1.276479in}}{\pgfqpoint{0.873279in}{1.270655in}}%
\pgfpathcurveto{\pgfqpoint{0.879103in}{1.264831in}}{\pgfqpoint{0.887003in}{1.261559in}}{\pgfqpoint{0.895239in}{1.261559in}}%
\pgfpathclose%
\pgfusepath{stroke,fill}%
\end{pgfscope}%
\begin{pgfscope}%
\pgfpathrectangle{\pgfqpoint{0.100000in}{0.212622in}}{\pgfqpoint{3.696000in}{3.696000in}}%
\pgfusepath{clip}%
\pgfsetbuttcap%
\pgfsetroundjoin%
\definecolor{currentfill}{rgb}{0.121569,0.466667,0.705882}%
\pgfsetfillcolor{currentfill}%
\pgfsetfillopacity{0.632530}%
\pgfsetlinewidth{1.003750pt}%
\definecolor{currentstroke}{rgb}{0.121569,0.466667,0.705882}%
\pgfsetstrokecolor{currentstroke}%
\pgfsetstrokeopacity{0.632530}%
\pgfsetdash{}{0pt}%
\pgfpathmoveto{\pgfqpoint{0.895239in}{1.261559in}}%
\pgfpathcurveto{\pgfqpoint{0.903475in}{1.261559in}}{\pgfqpoint{0.911375in}{1.264831in}}{\pgfqpoint{0.917199in}{1.270655in}}%
\pgfpathcurveto{\pgfqpoint{0.923023in}{1.276479in}}{\pgfqpoint{0.926296in}{1.284379in}}{\pgfqpoint{0.926296in}{1.292616in}}%
\pgfpathcurveto{\pgfqpoint{0.926296in}{1.300852in}}{\pgfqpoint{0.923023in}{1.308752in}}{\pgfqpoint{0.917199in}{1.314576in}}%
\pgfpathcurveto{\pgfqpoint{0.911375in}{1.320400in}}{\pgfqpoint{0.903475in}{1.323672in}}{\pgfqpoint{0.895239in}{1.323672in}}%
\pgfpathcurveto{\pgfqpoint{0.887003in}{1.323672in}}{\pgfqpoint{0.879103in}{1.320400in}}{\pgfqpoint{0.873279in}{1.314576in}}%
\pgfpathcurveto{\pgfqpoint{0.867455in}{1.308752in}}{\pgfqpoint{0.864183in}{1.300852in}}{\pgfqpoint{0.864183in}{1.292616in}}%
\pgfpathcurveto{\pgfqpoint{0.864183in}{1.284379in}}{\pgfqpoint{0.867455in}{1.276479in}}{\pgfqpoint{0.873279in}{1.270655in}}%
\pgfpathcurveto{\pgfqpoint{0.879103in}{1.264831in}}{\pgfqpoint{0.887003in}{1.261559in}}{\pgfqpoint{0.895239in}{1.261559in}}%
\pgfpathclose%
\pgfusepath{stroke,fill}%
\end{pgfscope}%
\begin{pgfscope}%
\pgfpathrectangle{\pgfqpoint{0.100000in}{0.212622in}}{\pgfqpoint{3.696000in}{3.696000in}}%
\pgfusepath{clip}%
\pgfsetbuttcap%
\pgfsetroundjoin%
\definecolor{currentfill}{rgb}{0.121569,0.466667,0.705882}%
\pgfsetfillcolor{currentfill}%
\pgfsetfillopacity{0.632530}%
\pgfsetlinewidth{1.003750pt}%
\definecolor{currentstroke}{rgb}{0.121569,0.466667,0.705882}%
\pgfsetstrokecolor{currentstroke}%
\pgfsetstrokeopacity{0.632530}%
\pgfsetdash{}{0pt}%
\pgfpathmoveto{\pgfqpoint{0.895239in}{1.261559in}}%
\pgfpathcurveto{\pgfqpoint{0.903475in}{1.261559in}}{\pgfqpoint{0.911375in}{1.264831in}}{\pgfqpoint{0.917199in}{1.270655in}}%
\pgfpathcurveto{\pgfqpoint{0.923023in}{1.276479in}}{\pgfqpoint{0.926296in}{1.284379in}}{\pgfqpoint{0.926296in}{1.292616in}}%
\pgfpathcurveto{\pgfqpoint{0.926296in}{1.300852in}}{\pgfqpoint{0.923023in}{1.308752in}}{\pgfqpoint{0.917199in}{1.314576in}}%
\pgfpathcurveto{\pgfqpoint{0.911375in}{1.320400in}}{\pgfqpoint{0.903475in}{1.323672in}}{\pgfqpoint{0.895239in}{1.323672in}}%
\pgfpathcurveto{\pgfqpoint{0.887003in}{1.323672in}}{\pgfqpoint{0.879103in}{1.320400in}}{\pgfqpoint{0.873279in}{1.314576in}}%
\pgfpathcurveto{\pgfqpoint{0.867455in}{1.308752in}}{\pgfqpoint{0.864183in}{1.300852in}}{\pgfqpoint{0.864183in}{1.292616in}}%
\pgfpathcurveto{\pgfqpoint{0.864183in}{1.284379in}}{\pgfqpoint{0.867455in}{1.276479in}}{\pgfqpoint{0.873279in}{1.270655in}}%
\pgfpathcurveto{\pgfqpoint{0.879103in}{1.264831in}}{\pgfqpoint{0.887003in}{1.261559in}}{\pgfqpoint{0.895239in}{1.261559in}}%
\pgfpathclose%
\pgfusepath{stroke,fill}%
\end{pgfscope}%
\begin{pgfscope}%
\pgfpathrectangle{\pgfqpoint{0.100000in}{0.212622in}}{\pgfqpoint{3.696000in}{3.696000in}}%
\pgfusepath{clip}%
\pgfsetbuttcap%
\pgfsetroundjoin%
\definecolor{currentfill}{rgb}{0.121569,0.466667,0.705882}%
\pgfsetfillcolor{currentfill}%
\pgfsetfillopacity{0.632530}%
\pgfsetlinewidth{1.003750pt}%
\definecolor{currentstroke}{rgb}{0.121569,0.466667,0.705882}%
\pgfsetstrokecolor{currentstroke}%
\pgfsetstrokeopacity{0.632530}%
\pgfsetdash{}{0pt}%
\pgfpathmoveto{\pgfqpoint{0.895239in}{1.261559in}}%
\pgfpathcurveto{\pgfqpoint{0.903475in}{1.261559in}}{\pgfqpoint{0.911375in}{1.264831in}}{\pgfqpoint{0.917199in}{1.270655in}}%
\pgfpathcurveto{\pgfqpoint{0.923023in}{1.276479in}}{\pgfqpoint{0.926296in}{1.284379in}}{\pgfqpoint{0.926296in}{1.292616in}}%
\pgfpathcurveto{\pgfqpoint{0.926296in}{1.300852in}}{\pgfqpoint{0.923023in}{1.308752in}}{\pgfqpoint{0.917199in}{1.314576in}}%
\pgfpathcurveto{\pgfqpoint{0.911375in}{1.320400in}}{\pgfqpoint{0.903475in}{1.323672in}}{\pgfqpoint{0.895239in}{1.323672in}}%
\pgfpathcurveto{\pgfqpoint{0.887003in}{1.323672in}}{\pgfqpoint{0.879103in}{1.320400in}}{\pgfqpoint{0.873279in}{1.314576in}}%
\pgfpathcurveto{\pgfqpoint{0.867455in}{1.308752in}}{\pgfqpoint{0.864183in}{1.300852in}}{\pgfqpoint{0.864183in}{1.292616in}}%
\pgfpathcurveto{\pgfqpoint{0.864183in}{1.284379in}}{\pgfqpoint{0.867455in}{1.276479in}}{\pgfqpoint{0.873279in}{1.270655in}}%
\pgfpathcurveto{\pgfqpoint{0.879103in}{1.264831in}}{\pgfqpoint{0.887003in}{1.261559in}}{\pgfqpoint{0.895239in}{1.261559in}}%
\pgfpathclose%
\pgfusepath{stroke,fill}%
\end{pgfscope}%
\begin{pgfscope}%
\pgfpathrectangle{\pgfqpoint{0.100000in}{0.212622in}}{\pgfqpoint{3.696000in}{3.696000in}}%
\pgfusepath{clip}%
\pgfsetbuttcap%
\pgfsetroundjoin%
\definecolor{currentfill}{rgb}{0.121569,0.466667,0.705882}%
\pgfsetfillcolor{currentfill}%
\pgfsetfillopacity{0.632530}%
\pgfsetlinewidth{1.003750pt}%
\definecolor{currentstroke}{rgb}{0.121569,0.466667,0.705882}%
\pgfsetstrokecolor{currentstroke}%
\pgfsetstrokeopacity{0.632530}%
\pgfsetdash{}{0pt}%
\pgfpathmoveto{\pgfqpoint{0.895239in}{1.261559in}}%
\pgfpathcurveto{\pgfqpoint{0.903475in}{1.261559in}}{\pgfqpoint{0.911375in}{1.264831in}}{\pgfqpoint{0.917199in}{1.270655in}}%
\pgfpathcurveto{\pgfqpoint{0.923023in}{1.276479in}}{\pgfqpoint{0.926296in}{1.284379in}}{\pgfqpoint{0.926296in}{1.292616in}}%
\pgfpathcurveto{\pgfqpoint{0.926296in}{1.300852in}}{\pgfqpoint{0.923023in}{1.308752in}}{\pgfqpoint{0.917199in}{1.314576in}}%
\pgfpathcurveto{\pgfqpoint{0.911375in}{1.320400in}}{\pgfqpoint{0.903475in}{1.323672in}}{\pgfqpoint{0.895239in}{1.323672in}}%
\pgfpathcurveto{\pgfqpoint{0.887003in}{1.323672in}}{\pgfqpoint{0.879103in}{1.320400in}}{\pgfqpoint{0.873279in}{1.314576in}}%
\pgfpathcurveto{\pgfqpoint{0.867455in}{1.308752in}}{\pgfqpoint{0.864183in}{1.300852in}}{\pgfqpoint{0.864183in}{1.292616in}}%
\pgfpathcurveto{\pgfqpoint{0.864183in}{1.284379in}}{\pgfqpoint{0.867455in}{1.276479in}}{\pgfqpoint{0.873279in}{1.270655in}}%
\pgfpathcurveto{\pgfqpoint{0.879103in}{1.264831in}}{\pgfqpoint{0.887003in}{1.261559in}}{\pgfqpoint{0.895239in}{1.261559in}}%
\pgfpathclose%
\pgfusepath{stroke,fill}%
\end{pgfscope}%
\begin{pgfscope}%
\pgfpathrectangle{\pgfqpoint{0.100000in}{0.212622in}}{\pgfqpoint{3.696000in}{3.696000in}}%
\pgfusepath{clip}%
\pgfsetbuttcap%
\pgfsetroundjoin%
\definecolor{currentfill}{rgb}{0.121569,0.466667,0.705882}%
\pgfsetfillcolor{currentfill}%
\pgfsetfillopacity{0.632530}%
\pgfsetlinewidth{1.003750pt}%
\definecolor{currentstroke}{rgb}{0.121569,0.466667,0.705882}%
\pgfsetstrokecolor{currentstroke}%
\pgfsetstrokeopacity{0.632530}%
\pgfsetdash{}{0pt}%
\pgfpathmoveto{\pgfqpoint{0.895239in}{1.261559in}}%
\pgfpathcurveto{\pgfqpoint{0.903475in}{1.261559in}}{\pgfqpoint{0.911375in}{1.264831in}}{\pgfqpoint{0.917199in}{1.270655in}}%
\pgfpathcurveto{\pgfqpoint{0.923023in}{1.276479in}}{\pgfqpoint{0.926296in}{1.284379in}}{\pgfqpoint{0.926296in}{1.292616in}}%
\pgfpathcurveto{\pgfqpoint{0.926296in}{1.300852in}}{\pgfqpoint{0.923023in}{1.308752in}}{\pgfqpoint{0.917199in}{1.314576in}}%
\pgfpathcurveto{\pgfqpoint{0.911375in}{1.320400in}}{\pgfqpoint{0.903475in}{1.323672in}}{\pgfqpoint{0.895239in}{1.323672in}}%
\pgfpathcurveto{\pgfqpoint{0.887003in}{1.323672in}}{\pgfqpoint{0.879103in}{1.320400in}}{\pgfqpoint{0.873279in}{1.314576in}}%
\pgfpathcurveto{\pgfqpoint{0.867455in}{1.308752in}}{\pgfqpoint{0.864183in}{1.300852in}}{\pgfqpoint{0.864183in}{1.292616in}}%
\pgfpathcurveto{\pgfqpoint{0.864183in}{1.284379in}}{\pgfqpoint{0.867455in}{1.276479in}}{\pgfqpoint{0.873279in}{1.270655in}}%
\pgfpathcurveto{\pgfqpoint{0.879103in}{1.264831in}}{\pgfqpoint{0.887003in}{1.261559in}}{\pgfqpoint{0.895239in}{1.261559in}}%
\pgfpathclose%
\pgfusepath{stroke,fill}%
\end{pgfscope}%
\begin{pgfscope}%
\pgfpathrectangle{\pgfqpoint{0.100000in}{0.212622in}}{\pgfqpoint{3.696000in}{3.696000in}}%
\pgfusepath{clip}%
\pgfsetbuttcap%
\pgfsetroundjoin%
\definecolor{currentfill}{rgb}{0.121569,0.466667,0.705882}%
\pgfsetfillcolor{currentfill}%
\pgfsetfillopacity{0.632530}%
\pgfsetlinewidth{1.003750pt}%
\definecolor{currentstroke}{rgb}{0.121569,0.466667,0.705882}%
\pgfsetstrokecolor{currentstroke}%
\pgfsetstrokeopacity{0.632530}%
\pgfsetdash{}{0pt}%
\pgfpathmoveto{\pgfqpoint{0.895239in}{1.261559in}}%
\pgfpathcurveto{\pgfqpoint{0.903475in}{1.261559in}}{\pgfqpoint{0.911375in}{1.264831in}}{\pgfqpoint{0.917199in}{1.270655in}}%
\pgfpathcurveto{\pgfqpoint{0.923023in}{1.276479in}}{\pgfqpoint{0.926296in}{1.284379in}}{\pgfqpoint{0.926296in}{1.292616in}}%
\pgfpathcurveto{\pgfqpoint{0.926296in}{1.300852in}}{\pgfqpoint{0.923023in}{1.308752in}}{\pgfqpoint{0.917199in}{1.314576in}}%
\pgfpathcurveto{\pgfqpoint{0.911375in}{1.320400in}}{\pgfqpoint{0.903475in}{1.323672in}}{\pgfqpoint{0.895239in}{1.323672in}}%
\pgfpathcurveto{\pgfqpoint{0.887003in}{1.323672in}}{\pgfqpoint{0.879103in}{1.320400in}}{\pgfqpoint{0.873279in}{1.314576in}}%
\pgfpathcurveto{\pgfqpoint{0.867455in}{1.308752in}}{\pgfqpoint{0.864183in}{1.300852in}}{\pgfqpoint{0.864183in}{1.292616in}}%
\pgfpathcurveto{\pgfqpoint{0.864183in}{1.284379in}}{\pgfqpoint{0.867455in}{1.276479in}}{\pgfqpoint{0.873279in}{1.270655in}}%
\pgfpathcurveto{\pgfqpoint{0.879103in}{1.264831in}}{\pgfqpoint{0.887003in}{1.261559in}}{\pgfqpoint{0.895239in}{1.261559in}}%
\pgfpathclose%
\pgfusepath{stroke,fill}%
\end{pgfscope}%
\begin{pgfscope}%
\pgfpathrectangle{\pgfqpoint{0.100000in}{0.212622in}}{\pgfqpoint{3.696000in}{3.696000in}}%
\pgfusepath{clip}%
\pgfsetbuttcap%
\pgfsetroundjoin%
\definecolor{currentfill}{rgb}{0.121569,0.466667,0.705882}%
\pgfsetfillcolor{currentfill}%
\pgfsetfillopacity{0.632530}%
\pgfsetlinewidth{1.003750pt}%
\definecolor{currentstroke}{rgb}{0.121569,0.466667,0.705882}%
\pgfsetstrokecolor{currentstroke}%
\pgfsetstrokeopacity{0.632530}%
\pgfsetdash{}{0pt}%
\pgfpathmoveto{\pgfqpoint{0.895239in}{1.261559in}}%
\pgfpathcurveto{\pgfqpoint{0.903475in}{1.261559in}}{\pgfqpoint{0.911375in}{1.264831in}}{\pgfqpoint{0.917199in}{1.270655in}}%
\pgfpathcurveto{\pgfqpoint{0.923023in}{1.276479in}}{\pgfqpoint{0.926296in}{1.284379in}}{\pgfqpoint{0.926296in}{1.292616in}}%
\pgfpathcurveto{\pgfqpoint{0.926296in}{1.300852in}}{\pgfqpoint{0.923023in}{1.308752in}}{\pgfqpoint{0.917199in}{1.314576in}}%
\pgfpathcurveto{\pgfqpoint{0.911375in}{1.320400in}}{\pgfqpoint{0.903475in}{1.323672in}}{\pgfqpoint{0.895239in}{1.323672in}}%
\pgfpathcurveto{\pgfqpoint{0.887003in}{1.323672in}}{\pgfqpoint{0.879103in}{1.320400in}}{\pgfqpoint{0.873279in}{1.314576in}}%
\pgfpathcurveto{\pgfqpoint{0.867455in}{1.308752in}}{\pgfqpoint{0.864183in}{1.300852in}}{\pgfqpoint{0.864183in}{1.292616in}}%
\pgfpathcurveto{\pgfqpoint{0.864183in}{1.284379in}}{\pgfqpoint{0.867455in}{1.276479in}}{\pgfqpoint{0.873279in}{1.270655in}}%
\pgfpathcurveto{\pgfqpoint{0.879103in}{1.264831in}}{\pgfqpoint{0.887003in}{1.261559in}}{\pgfqpoint{0.895239in}{1.261559in}}%
\pgfpathclose%
\pgfusepath{stroke,fill}%
\end{pgfscope}%
\begin{pgfscope}%
\pgfpathrectangle{\pgfqpoint{0.100000in}{0.212622in}}{\pgfqpoint{3.696000in}{3.696000in}}%
\pgfusepath{clip}%
\pgfsetbuttcap%
\pgfsetroundjoin%
\definecolor{currentfill}{rgb}{0.121569,0.466667,0.705882}%
\pgfsetfillcolor{currentfill}%
\pgfsetfillopacity{0.632530}%
\pgfsetlinewidth{1.003750pt}%
\definecolor{currentstroke}{rgb}{0.121569,0.466667,0.705882}%
\pgfsetstrokecolor{currentstroke}%
\pgfsetstrokeopacity{0.632530}%
\pgfsetdash{}{0pt}%
\pgfpathmoveto{\pgfqpoint{0.895239in}{1.261559in}}%
\pgfpathcurveto{\pgfqpoint{0.903475in}{1.261559in}}{\pgfqpoint{0.911375in}{1.264831in}}{\pgfqpoint{0.917199in}{1.270655in}}%
\pgfpathcurveto{\pgfqpoint{0.923023in}{1.276479in}}{\pgfqpoint{0.926296in}{1.284379in}}{\pgfqpoint{0.926296in}{1.292616in}}%
\pgfpathcurveto{\pgfqpoint{0.926296in}{1.300852in}}{\pgfqpoint{0.923023in}{1.308752in}}{\pgfqpoint{0.917199in}{1.314576in}}%
\pgfpathcurveto{\pgfqpoint{0.911375in}{1.320400in}}{\pgfqpoint{0.903475in}{1.323672in}}{\pgfqpoint{0.895239in}{1.323672in}}%
\pgfpathcurveto{\pgfqpoint{0.887003in}{1.323672in}}{\pgfqpoint{0.879103in}{1.320400in}}{\pgfqpoint{0.873279in}{1.314576in}}%
\pgfpathcurveto{\pgfqpoint{0.867455in}{1.308752in}}{\pgfqpoint{0.864183in}{1.300852in}}{\pgfqpoint{0.864183in}{1.292616in}}%
\pgfpathcurveto{\pgfqpoint{0.864183in}{1.284379in}}{\pgfqpoint{0.867455in}{1.276479in}}{\pgfqpoint{0.873279in}{1.270655in}}%
\pgfpathcurveto{\pgfqpoint{0.879103in}{1.264831in}}{\pgfqpoint{0.887003in}{1.261559in}}{\pgfqpoint{0.895239in}{1.261559in}}%
\pgfpathclose%
\pgfusepath{stroke,fill}%
\end{pgfscope}%
\begin{pgfscope}%
\pgfpathrectangle{\pgfqpoint{0.100000in}{0.212622in}}{\pgfqpoint{3.696000in}{3.696000in}}%
\pgfusepath{clip}%
\pgfsetbuttcap%
\pgfsetroundjoin%
\definecolor{currentfill}{rgb}{0.121569,0.466667,0.705882}%
\pgfsetfillcolor{currentfill}%
\pgfsetfillopacity{0.632530}%
\pgfsetlinewidth{1.003750pt}%
\definecolor{currentstroke}{rgb}{0.121569,0.466667,0.705882}%
\pgfsetstrokecolor{currentstroke}%
\pgfsetstrokeopacity{0.632530}%
\pgfsetdash{}{0pt}%
\pgfpathmoveto{\pgfqpoint{0.895239in}{1.261559in}}%
\pgfpathcurveto{\pgfqpoint{0.903475in}{1.261559in}}{\pgfqpoint{0.911375in}{1.264831in}}{\pgfqpoint{0.917199in}{1.270655in}}%
\pgfpathcurveto{\pgfqpoint{0.923023in}{1.276479in}}{\pgfqpoint{0.926296in}{1.284379in}}{\pgfqpoint{0.926296in}{1.292616in}}%
\pgfpathcurveto{\pgfqpoint{0.926296in}{1.300852in}}{\pgfqpoint{0.923023in}{1.308752in}}{\pgfqpoint{0.917199in}{1.314576in}}%
\pgfpathcurveto{\pgfqpoint{0.911375in}{1.320400in}}{\pgfqpoint{0.903475in}{1.323672in}}{\pgfqpoint{0.895239in}{1.323672in}}%
\pgfpathcurveto{\pgfqpoint{0.887003in}{1.323672in}}{\pgfqpoint{0.879103in}{1.320400in}}{\pgfqpoint{0.873279in}{1.314576in}}%
\pgfpathcurveto{\pgfqpoint{0.867455in}{1.308752in}}{\pgfqpoint{0.864183in}{1.300852in}}{\pgfqpoint{0.864183in}{1.292616in}}%
\pgfpathcurveto{\pgfqpoint{0.864183in}{1.284379in}}{\pgfqpoint{0.867455in}{1.276479in}}{\pgfqpoint{0.873279in}{1.270655in}}%
\pgfpathcurveto{\pgfqpoint{0.879103in}{1.264831in}}{\pgfqpoint{0.887003in}{1.261559in}}{\pgfqpoint{0.895239in}{1.261559in}}%
\pgfpathclose%
\pgfusepath{stroke,fill}%
\end{pgfscope}%
\begin{pgfscope}%
\pgfpathrectangle{\pgfqpoint{0.100000in}{0.212622in}}{\pgfqpoint{3.696000in}{3.696000in}}%
\pgfusepath{clip}%
\pgfsetbuttcap%
\pgfsetroundjoin%
\definecolor{currentfill}{rgb}{0.121569,0.466667,0.705882}%
\pgfsetfillcolor{currentfill}%
\pgfsetfillopacity{0.632530}%
\pgfsetlinewidth{1.003750pt}%
\definecolor{currentstroke}{rgb}{0.121569,0.466667,0.705882}%
\pgfsetstrokecolor{currentstroke}%
\pgfsetstrokeopacity{0.632530}%
\pgfsetdash{}{0pt}%
\pgfpathmoveto{\pgfqpoint{0.895239in}{1.261559in}}%
\pgfpathcurveto{\pgfqpoint{0.903475in}{1.261559in}}{\pgfqpoint{0.911375in}{1.264831in}}{\pgfqpoint{0.917199in}{1.270655in}}%
\pgfpathcurveto{\pgfqpoint{0.923023in}{1.276479in}}{\pgfqpoint{0.926296in}{1.284379in}}{\pgfqpoint{0.926296in}{1.292616in}}%
\pgfpathcurveto{\pgfqpoint{0.926296in}{1.300852in}}{\pgfqpoint{0.923023in}{1.308752in}}{\pgfqpoint{0.917199in}{1.314576in}}%
\pgfpathcurveto{\pgfqpoint{0.911375in}{1.320400in}}{\pgfqpoint{0.903475in}{1.323672in}}{\pgfqpoint{0.895239in}{1.323672in}}%
\pgfpathcurveto{\pgfqpoint{0.887003in}{1.323672in}}{\pgfqpoint{0.879103in}{1.320400in}}{\pgfqpoint{0.873279in}{1.314576in}}%
\pgfpathcurveto{\pgfqpoint{0.867455in}{1.308752in}}{\pgfqpoint{0.864183in}{1.300852in}}{\pgfqpoint{0.864183in}{1.292616in}}%
\pgfpathcurveto{\pgfqpoint{0.864183in}{1.284379in}}{\pgfqpoint{0.867455in}{1.276479in}}{\pgfqpoint{0.873279in}{1.270655in}}%
\pgfpathcurveto{\pgfqpoint{0.879103in}{1.264831in}}{\pgfqpoint{0.887003in}{1.261559in}}{\pgfqpoint{0.895239in}{1.261559in}}%
\pgfpathclose%
\pgfusepath{stroke,fill}%
\end{pgfscope}%
\begin{pgfscope}%
\pgfpathrectangle{\pgfqpoint{0.100000in}{0.212622in}}{\pgfqpoint{3.696000in}{3.696000in}}%
\pgfusepath{clip}%
\pgfsetbuttcap%
\pgfsetroundjoin%
\definecolor{currentfill}{rgb}{0.121569,0.466667,0.705882}%
\pgfsetfillcolor{currentfill}%
\pgfsetfillopacity{0.632530}%
\pgfsetlinewidth{1.003750pt}%
\definecolor{currentstroke}{rgb}{0.121569,0.466667,0.705882}%
\pgfsetstrokecolor{currentstroke}%
\pgfsetstrokeopacity{0.632530}%
\pgfsetdash{}{0pt}%
\pgfpathmoveto{\pgfqpoint{0.895239in}{1.261559in}}%
\pgfpathcurveto{\pgfqpoint{0.903475in}{1.261559in}}{\pgfqpoint{0.911375in}{1.264831in}}{\pgfqpoint{0.917199in}{1.270655in}}%
\pgfpathcurveto{\pgfqpoint{0.923023in}{1.276479in}}{\pgfqpoint{0.926296in}{1.284379in}}{\pgfqpoint{0.926296in}{1.292616in}}%
\pgfpathcurveto{\pgfqpoint{0.926296in}{1.300852in}}{\pgfqpoint{0.923023in}{1.308752in}}{\pgfqpoint{0.917199in}{1.314576in}}%
\pgfpathcurveto{\pgfqpoint{0.911375in}{1.320400in}}{\pgfqpoint{0.903475in}{1.323672in}}{\pgfqpoint{0.895239in}{1.323672in}}%
\pgfpathcurveto{\pgfqpoint{0.887003in}{1.323672in}}{\pgfqpoint{0.879103in}{1.320400in}}{\pgfqpoint{0.873279in}{1.314576in}}%
\pgfpathcurveto{\pgfqpoint{0.867455in}{1.308752in}}{\pgfqpoint{0.864183in}{1.300852in}}{\pgfqpoint{0.864183in}{1.292616in}}%
\pgfpathcurveto{\pgfqpoint{0.864183in}{1.284379in}}{\pgfqpoint{0.867455in}{1.276479in}}{\pgfqpoint{0.873279in}{1.270655in}}%
\pgfpathcurveto{\pgfqpoint{0.879103in}{1.264831in}}{\pgfqpoint{0.887003in}{1.261559in}}{\pgfqpoint{0.895239in}{1.261559in}}%
\pgfpathclose%
\pgfusepath{stroke,fill}%
\end{pgfscope}%
\begin{pgfscope}%
\pgfpathrectangle{\pgfqpoint{0.100000in}{0.212622in}}{\pgfqpoint{3.696000in}{3.696000in}}%
\pgfusepath{clip}%
\pgfsetbuttcap%
\pgfsetroundjoin%
\definecolor{currentfill}{rgb}{0.121569,0.466667,0.705882}%
\pgfsetfillcolor{currentfill}%
\pgfsetfillopacity{0.632530}%
\pgfsetlinewidth{1.003750pt}%
\definecolor{currentstroke}{rgb}{0.121569,0.466667,0.705882}%
\pgfsetstrokecolor{currentstroke}%
\pgfsetstrokeopacity{0.632530}%
\pgfsetdash{}{0pt}%
\pgfpathmoveto{\pgfqpoint{0.895239in}{1.261559in}}%
\pgfpathcurveto{\pgfqpoint{0.903475in}{1.261559in}}{\pgfqpoint{0.911375in}{1.264831in}}{\pgfqpoint{0.917199in}{1.270655in}}%
\pgfpathcurveto{\pgfqpoint{0.923023in}{1.276479in}}{\pgfqpoint{0.926296in}{1.284379in}}{\pgfqpoint{0.926296in}{1.292616in}}%
\pgfpathcurveto{\pgfqpoint{0.926296in}{1.300852in}}{\pgfqpoint{0.923023in}{1.308752in}}{\pgfqpoint{0.917199in}{1.314576in}}%
\pgfpathcurveto{\pgfqpoint{0.911375in}{1.320400in}}{\pgfqpoint{0.903475in}{1.323672in}}{\pgfqpoint{0.895239in}{1.323672in}}%
\pgfpathcurveto{\pgfqpoint{0.887003in}{1.323672in}}{\pgfqpoint{0.879103in}{1.320400in}}{\pgfqpoint{0.873279in}{1.314576in}}%
\pgfpathcurveto{\pgfqpoint{0.867455in}{1.308752in}}{\pgfqpoint{0.864183in}{1.300852in}}{\pgfqpoint{0.864183in}{1.292616in}}%
\pgfpathcurveto{\pgfqpoint{0.864183in}{1.284379in}}{\pgfqpoint{0.867455in}{1.276479in}}{\pgfqpoint{0.873279in}{1.270655in}}%
\pgfpathcurveto{\pgfqpoint{0.879103in}{1.264831in}}{\pgfqpoint{0.887003in}{1.261559in}}{\pgfqpoint{0.895239in}{1.261559in}}%
\pgfpathclose%
\pgfusepath{stroke,fill}%
\end{pgfscope}%
\begin{pgfscope}%
\pgfpathrectangle{\pgfqpoint{0.100000in}{0.212622in}}{\pgfqpoint{3.696000in}{3.696000in}}%
\pgfusepath{clip}%
\pgfsetbuttcap%
\pgfsetroundjoin%
\definecolor{currentfill}{rgb}{0.121569,0.466667,0.705882}%
\pgfsetfillcolor{currentfill}%
\pgfsetfillopacity{0.632530}%
\pgfsetlinewidth{1.003750pt}%
\definecolor{currentstroke}{rgb}{0.121569,0.466667,0.705882}%
\pgfsetstrokecolor{currentstroke}%
\pgfsetstrokeopacity{0.632530}%
\pgfsetdash{}{0pt}%
\pgfpathmoveto{\pgfqpoint{0.895239in}{1.261559in}}%
\pgfpathcurveto{\pgfqpoint{0.903475in}{1.261559in}}{\pgfqpoint{0.911375in}{1.264831in}}{\pgfqpoint{0.917199in}{1.270655in}}%
\pgfpathcurveto{\pgfqpoint{0.923023in}{1.276479in}}{\pgfqpoint{0.926296in}{1.284379in}}{\pgfqpoint{0.926296in}{1.292616in}}%
\pgfpathcurveto{\pgfqpoint{0.926296in}{1.300852in}}{\pgfqpoint{0.923023in}{1.308752in}}{\pgfqpoint{0.917199in}{1.314576in}}%
\pgfpathcurveto{\pgfqpoint{0.911375in}{1.320400in}}{\pgfqpoint{0.903475in}{1.323672in}}{\pgfqpoint{0.895239in}{1.323672in}}%
\pgfpathcurveto{\pgfqpoint{0.887003in}{1.323672in}}{\pgfqpoint{0.879103in}{1.320400in}}{\pgfqpoint{0.873279in}{1.314576in}}%
\pgfpathcurveto{\pgfqpoint{0.867455in}{1.308752in}}{\pgfqpoint{0.864183in}{1.300852in}}{\pgfqpoint{0.864183in}{1.292616in}}%
\pgfpathcurveto{\pgfqpoint{0.864183in}{1.284379in}}{\pgfqpoint{0.867455in}{1.276479in}}{\pgfqpoint{0.873279in}{1.270655in}}%
\pgfpathcurveto{\pgfqpoint{0.879103in}{1.264831in}}{\pgfqpoint{0.887003in}{1.261559in}}{\pgfqpoint{0.895239in}{1.261559in}}%
\pgfpathclose%
\pgfusepath{stroke,fill}%
\end{pgfscope}%
\begin{pgfscope}%
\pgfpathrectangle{\pgfqpoint{0.100000in}{0.212622in}}{\pgfqpoint{3.696000in}{3.696000in}}%
\pgfusepath{clip}%
\pgfsetbuttcap%
\pgfsetroundjoin%
\definecolor{currentfill}{rgb}{0.121569,0.466667,0.705882}%
\pgfsetfillcolor{currentfill}%
\pgfsetfillopacity{0.632689}%
\pgfsetlinewidth{1.003750pt}%
\definecolor{currentstroke}{rgb}{0.121569,0.466667,0.705882}%
\pgfsetstrokecolor{currentstroke}%
\pgfsetstrokeopacity{0.632689}%
\pgfsetdash{}{0pt}%
\pgfpathmoveto{\pgfqpoint{0.894771in}{1.261477in}}%
\pgfpathcurveto{\pgfqpoint{0.903007in}{1.261477in}}{\pgfqpoint{0.910908in}{1.264749in}}{\pgfqpoint{0.916731in}{1.270573in}}%
\pgfpathcurveto{\pgfqpoint{0.922555in}{1.276397in}}{\pgfqpoint{0.925828in}{1.284297in}}{\pgfqpoint{0.925828in}{1.292534in}}%
\pgfpathcurveto{\pgfqpoint{0.925828in}{1.300770in}}{\pgfqpoint{0.922555in}{1.308670in}}{\pgfqpoint{0.916731in}{1.314494in}}%
\pgfpathcurveto{\pgfqpoint{0.910908in}{1.320318in}}{\pgfqpoint{0.903007in}{1.323590in}}{\pgfqpoint{0.894771in}{1.323590in}}%
\pgfpathcurveto{\pgfqpoint{0.886535in}{1.323590in}}{\pgfqpoint{0.878635in}{1.320318in}}{\pgfqpoint{0.872811in}{1.314494in}}%
\pgfpathcurveto{\pgfqpoint{0.866987in}{1.308670in}}{\pgfqpoint{0.863715in}{1.300770in}}{\pgfqpoint{0.863715in}{1.292534in}}%
\pgfpathcurveto{\pgfqpoint{0.863715in}{1.284297in}}{\pgfqpoint{0.866987in}{1.276397in}}{\pgfqpoint{0.872811in}{1.270573in}}%
\pgfpathcurveto{\pgfqpoint{0.878635in}{1.264749in}}{\pgfqpoint{0.886535in}{1.261477in}}{\pgfqpoint{0.894771in}{1.261477in}}%
\pgfpathclose%
\pgfusepath{stroke,fill}%
\end{pgfscope}%
\begin{pgfscope}%
\pgfpathrectangle{\pgfqpoint{0.100000in}{0.212622in}}{\pgfqpoint{3.696000in}{3.696000in}}%
\pgfusepath{clip}%
\pgfsetbuttcap%
\pgfsetroundjoin%
\definecolor{currentfill}{rgb}{0.121569,0.466667,0.705882}%
\pgfsetfillcolor{currentfill}%
\pgfsetfillopacity{0.632713}%
\pgfsetlinewidth{1.003750pt}%
\definecolor{currentstroke}{rgb}{0.121569,0.466667,0.705882}%
\pgfsetstrokecolor{currentstroke}%
\pgfsetstrokeopacity{0.632713}%
\pgfsetdash{}{0pt}%
\pgfpathmoveto{\pgfqpoint{0.875248in}{1.267411in}}%
\pgfpathcurveto{\pgfqpoint{0.883484in}{1.267411in}}{\pgfqpoint{0.891384in}{1.270683in}}{\pgfqpoint{0.897208in}{1.276507in}}%
\pgfpathcurveto{\pgfqpoint{0.903032in}{1.282331in}}{\pgfqpoint{0.906304in}{1.290231in}}{\pgfqpoint{0.906304in}{1.298468in}}%
\pgfpathcurveto{\pgfqpoint{0.906304in}{1.306704in}}{\pgfqpoint{0.903032in}{1.314604in}}{\pgfqpoint{0.897208in}{1.320428in}}%
\pgfpathcurveto{\pgfqpoint{0.891384in}{1.326252in}}{\pgfqpoint{0.883484in}{1.329524in}}{\pgfqpoint{0.875248in}{1.329524in}}%
\pgfpathcurveto{\pgfqpoint{0.867011in}{1.329524in}}{\pgfqpoint{0.859111in}{1.326252in}}{\pgfqpoint{0.853287in}{1.320428in}}%
\pgfpathcurveto{\pgfqpoint{0.847464in}{1.314604in}}{\pgfqpoint{0.844191in}{1.306704in}}{\pgfqpoint{0.844191in}{1.298468in}}%
\pgfpathcurveto{\pgfqpoint{0.844191in}{1.290231in}}{\pgfqpoint{0.847464in}{1.282331in}}{\pgfqpoint{0.853287in}{1.276507in}}%
\pgfpathcurveto{\pgfqpoint{0.859111in}{1.270683in}}{\pgfqpoint{0.867011in}{1.267411in}}{\pgfqpoint{0.875248in}{1.267411in}}%
\pgfpathclose%
\pgfusepath{stroke,fill}%
\end{pgfscope}%
\begin{pgfscope}%
\pgfpathrectangle{\pgfqpoint{0.100000in}{0.212622in}}{\pgfqpoint{3.696000in}{3.696000in}}%
\pgfusepath{clip}%
\pgfsetbuttcap%
\pgfsetroundjoin%
\definecolor{currentfill}{rgb}{0.121569,0.466667,0.705882}%
\pgfsetfillcolor{currentfill}%
\pgfsetfillopacity{0.632831}%
\pgfsetlinewidth{1.003750pt}%
\definecolor{currentstroke}{rgb}{0.121569,0.466667,0.705882}%
\pgfsetstrokecolor{currentstroke}%
\pgfsetstrokeopacity{0.632831}%
\pgfsetdash{}{0pt}%
\pgfpathmoveto{\pgfqpoint{2.129281in}{1.791383in}}%
\pgfpathcurveto{\pgfqpoint{2.137517in}{1.791383in}}{\pgfqpoint{2.145417in}{1.794655in}}{\pgfqpoint{2.151241in}{1.800479in}}%
\pgfpathcurveto{\pgfqpoint{2.157065in}{1.806303in}}{\pgfqpoint{2.160337in}{1.814203in}}{\pgfqpoint{2.160337in}{1.822439in}}%
\pgfpathcurveto{\pgfqpoint{2.160337in}{1.830676in}}{\pgfqpoint{2.157065in}{1.838576in}}{\pgfqpoint{2.151241in}{1.844400in}}%
\pgfpathcurveto{\pgfqpoint{2.145417in}{1.850224in}}{\pgfqpoint{2.137517in}{1.853496in}}{\pgfqpoint{2.129281in}{1.853496in}}%
\pgfpathcurveto{\pgfqpoint{2.121044in}{1.853496in}}{\pgfqpoint{2.113144in}{1.850224in}}{\pgfqpoint{2.107320in}{1.844400in}}%
\pgfpathcurveto{\pgfqpoint{2.101496in}{1.838576in}}{\pgfqpoint{2.098224in}{1.830676in}}{\pgfqpoint{2.098224in}{1.822439in}}%
\pgfpathcurveto{\pgfqpoint{2.098224in}{1.814203in}}{\pgfqpoint{2.101496in}{1.806303in}}{\pgfqpoint{2.107320in}{1.800479in}}%
\pgfpathcurveto{\pgfqpoint{2.113144in}{1.794655in}}{\pgfqpoint{2.121044in}{1.791383in}}{\pgfqpoint{2.129281in}{1.791383in}}%
\pgfpathclose%
\pgfusepath{stroke,fill}%
\end{pgfscope}%
\begin{pgfscope}%
\pgfpathrectangle{\pgfqpoint{0.100000in}{0.212622in}}{\pgfqpoint{3.696000in}{3.696000in}}%
\pgfusepath{clip}%
\pgfsetbuttcap%
\pgfsetroundjoin%
\definecolor{currentfill}{rgb}{0.121569,0.466667,0.705882}%
\pgfsetfillcolor{currentfill}%
\pgfsetfillopacity{0.632954}%
\pgfsetlinewidth{1.003750pt}%
\definecolor{currentstroke}{rgb}{0.121569,0.466667,0.705882}%
\pgfsetstrokecolor{currentstroke}%
\pgfsetstrokeopacity{0.632954}%
\pgfsetdash{}{0pt}%
\pgfpathmoveto{\pgfqpoint{0.893528in}{1.261178in}}%
\pgfpathcurveto{\pgfqpoint{0.901764in}{1.261178in}}{\pgfqpoint{0.909664in}{1.264450in}}{\pgfqpoint{0.915488in}{1.270274in}}%
\pgfpathcurveto{\pgfqpoint{0.921312in}{1.276098in}}{\pgfqpoint{0.924585in}{1.283998in}}{\pgfqpoint{0.924585in}{1.292234in}}%
\pgfpathcurveto{\pgfqpoint{0.924585in}{1.300471in}}{\pgfqpoint{0.921312in}{1.308371in}}{\pgfqpoint{0.915488in}{1.314195in}}%
\pgfpathcurveto{\pgfqpoint{0.909664in}{1.320018in}}{\pgfqpoint{0.901764in}{1.323291in}}{\pgfqpoint{0.893528in}{1.323291in}}%
\pgfpathcurveto{\pgfqpoint{0.885292in}{1.323291in}}{\pgfqpoint{0.877392in}{1.320018in}}{\pgfqpoint{0.871568in}{1.314195in}}%
\pgfpathcurveto{\pgfqpoint{0.865744in}{1.308371in}}{\pgfqpoint{0.862472in}{1.300471in}}{\pgfqpoint{0.862472in}{1.292234in}}%
\pgfpathcurveto{\pgfqpoint{0.862472in}{1.283998in}}{\pgfqpoint{0.865744in}{1.276098in}}{\pgfqpoint{0.871568in}{1.270274in}}%
\pgfpathcurveto{\pgfqpoint{0.877392in}{1.264450in}}{\pgfqpoint{0.885292in}{1.261178in}}{\pgfqpoint{0.893528in}{1.261178in}}%
\pgfpathclose%
\pgfusepath{stroke,fill}%
\end{pgfscope}%
\begin{pgfscope}%
\pgfpathrectangle{\pgfqpoint{0.100000in}{0.212622in}}{\pgfqpoint{3.696000in}{3.696000in}}%
\pgfusepath{clip}%
\pgfsetbuttcap%
\pgfsetroundjoin%
\definecolor{currentfill}{rgb}{0.121569,0.466667,0.705882}%
\pgfsetfillcolor{currentfill}%
\pgfsetfillopacity{0.633323}%
\pgfsetlinewidth{1.003750pt}%
\definecolor{currentstroke}{rgb}{0.121569,0.466667,0.705882}%
\pgfsetstrokecolor{currentstroke}%
\pgfsetstrokeopacity{0.633323}%
\pgfsetdash{}{0pt}%
\pgfpathmoveto{\pgfqpoint{0.891066in}{1.261071in}}%
\pgfpathcurveto{\pgfqpoint{0.899302in}{1.261071in}}{\pgfqpoint{0.907202in}{1.264344in}}{\pgfqpoint{0.913026in}{1.270168in}}%
\pgfpathcurveto{\pgfqpoint{0.918850in}{1.275992in}}{\pgfqpoint{0.922122in}{1.283892in}}{\pgfqpoint{0.922122in}{1.292128in}}%
\pgfpathcurveto{\pgfqpoint{0.922122in}{1.300364in}}{\pgfqpoint{0.918850in}{1.308264in}}{\pgfqpoint{0.913026in}{1.314088in}}%
\pgfpathcurveto{\pgfqpoint{0.907202in}{1.319912in}}{\pgfqpoint{0.899302in}{1.323184in}}{\pgfqpoint{0.891066in}{1.323184in}}%
\pgfpathcurveto{\pgfqpoint{0.882829in}{1.323184in}}{\pgfqpoint{0.874929in}{1.319912in}}{\pgfqpoint{0.869105in}{1.314088in}}%
\pgfpathcurveto{\pgfqpoint{0.863282in}{1.308264in}}{\pgfqpoint{0.860009in}{1.300364in}}{\pgfqpoint{0.860009in}{1.292128in}}%
\pgfpathcurveto{\pgfqpoint{0.860009in}{1.283892in}}{\pgfqpoint{0.863282in}{1.275992in}}{\pgfqpoint{0.869105in}{1.270168in}}%
\pgfpathcurveto{\pgfqpoint{0.874929in}{1.264344in}}{\pgfqpoint{0.882829in}{1.261071in}}{\pgfqpoint{0.891066in}{1.261071in}}%
\pgfpathclose%
\pgfusepath{stroke,fill}%
\end{pgfscope}%
\begin{pgfscope}%
\pgfpathrectangle{\pgfqpoint{0.100000in}{0.212622in}}{\pgfqpoint{3.696000in}{3.696000in}}%
\pgfusepath{clip}%
\pgfsetbuttcap%
\pgfsetroundjoin%
\definecolor{currentfill}{rgb}{0.121569,0.466667,0.705882}%
\pgfsetfillcolor{currentfill}%
\pgfsetfillopacity{0.633523}%
\pgfsetlinewidth{1.003750pt}%
\definecolor{currentstroke}{rgb}{0.121569,0.466667,0.705882}%
\pgfsetstrokecolor{currentstroke}%
\pgfsetstrokeopacity{0.633523}%
\pgfsetdash{}{0pt}%
\pgfpathmoveto{\pgfqpoint{0.881123in}{1.263164in}}%
\pgfpathcurveto{\pgfqpoint{0.889359in}{1.263164in}}{\pgfqpoint{0.897259in}{1.266436in}}{\pgfqpoint{0.903083in}{1.272260in}}%
\pgfpathcurveto{\pgfqpoint{0.908907in}{1.278084in}}{\pgfqpoint{0.912179in}{1.285984in}}{\pgfqpoint{0.912179in}{1.294221in}}%
\pgfpathcurveto{\pgfqpoint{0.912179in}{1.302457in}}{\pgfqpoint{0.908907in}{1.310357in}}{\pgfqpoint{0.903083in}{1.316181in}}%
\pgfpathcurveto{\pgfqpoint{0.897259in}{1.322005in}}{\pgfqpoint{0.889359in}{1.325277in}}{\pgfqpoint{0.881123in}{1.325277in}}%
\pgfpathcurveto{\pgfqpoint{0.872887in}{1.325277in}}{\pgfqpoint{0.864987in}{1.322005in}}{\pgfqpoint{0.859163in}{1.316181in}}%
\pgfpathcurveto{\pgfqpoint{0.853339in}{1.310357in}}{\pgfqpoint{0.850066in}{1.302457in}}{\pgfqpoint{0.850066in}{1.294221in}}%
\pgfpathcurveto{\pgfqpoint{0.850066in}{1.285984in}}{\pgfqpoint{0.853339in}{1.278084in}}{\pgfqpoint{0.859163in}{1.272260in}}%
\pgfpathcurveto{\pgfqpoint{0.864987in}{1.266436in}}{\pgfqpoint{0.872887in}{1.263164in}}{\pgfqpoint{0.881123in}{1.263164in}}%
\pgfpathclose%
\pgfusepath{stroke,fill}%
\end{pgfscope}%
\begin{pgfscope}%
\pgfpathrectangle{\pgfqpoint{0.100000in}{0.212622in}}{\pgfqpoint{3.696000in}{3.696000in}}%
\pgfusepath{clip}%
\pgfsetbuttcap%
\pgfsetroundjoin%
\definecolor{currentfill}{rgb}{0.121569,0.466667,0.705882}%
\pgfsetfillcolor{currentfill}%
\pgfsetfillopacity{0.633700}%
\pgfsetlinewidth{1.003750pt}%
\definecolor{currentstroke}{rgb}{0.121569,0.466667,0.705882}%
\pgfsetstrokecolor{currentstroke}%
\pgfsetstrokeopacity{0.633700}%
\pgfsetdash{}{0pt}%
\pgfpathmoveto{\pgfqpoint{0.886624in}{1.261963in}}%
\pgfpathcurveto{\pgfqpoint{0.894860in}{1.261963in}}{\pgfqpoint{0.902760in}{1.265235in}}{\pgfqpoint{0.908584in}{1.271059in}}%
\pgfpathcurveto{\pgfqpoint{0.914408in}{1.276883in}}{\pgfqpoint{0.917680in}{1.284783in}}{\pgfqpoint{0.917680in}{1.293019in}}%
\pgfpathcurveto{\pgfqpoint{0.917680in}{1.301256in}}{\pgfqpoint{0.914408in}{1.309156in}}{\pgfqpoint{0.908584in}{1.314980in}}%
\pgfpathcurveto{\pgfqpoint{0.902760in}{1.320804in}}{\pgfqpoint{0.894860in}{1.324076in}}{\pgfqpoint{0.886624in}{1.324076in}}%
\pgfpathcurveto{\pgfqpoint{0.878387in}{1.324076in}}{\pgfqpoint{0.870487in}{1.320804in}}{\pgfqpoint{0.864663in}{1.314980in}}%
\pgfpathcurveto{\pgfqpoint{0.858839in}{1.309156in}}{\pgfqpoint{0.855567in}{1.301256in}}{\pgfqpoint{0.855567in}{1.293019in}}%
\pgfpathcurveto{\pgfqpoint{0.855567in}{1.284783in}}{\pgfqpoint{0.858839in}{1.276883in}}{\pgfqpoint{0.864663in}{1.271059in}}%
\pgfpathcurveto{\pgfqpoint{0.870487in}{1.265235in}}{\pgfqpoint{0.878387in}{1.261963in}}{\pgfqpoint{0.886624in}{1.261963in}}%
\pgfpathclose%
\pgfusepath{stroke,fill}%
\end{pgfscope}%
\begin{pgfscope}%
\pgfpathrectangle{\pgfqpoint{0.100000in}{0.212622in}}{\pgfqpoint{3.696000in}{3.696000in}}%
\pgfusepath{clip}%
\pgfsetbuttcap%
\pgfsetroundjoin%
\definecolor{currentfill}{rgb}{0.121569,0.466667,0.705882}%
\pgfsetfillcolor{currentfill}%
\pgfsetfillopacity{0.634664}%
\pgfsetlinewidth{1.003750pt}%
\definecolor{currentstroke}{rgb}{0.121569,0.466667,0.705882}%
\pgfsetstrokecolor{currentstroke}%
\pgfsetstrokeopacity{0.634664}%
\pgfsetdash{}{0pt}%
\pgfpathmoveto{\pgfqpoint{2.130528in}{1.789586in}}%
\pgfpathcurveto{\pgfqpoint{2.138765in}{1.789586in}}{\pgfqpoint{2.146665in}{1.792858in}}{\pgfqpoint{2.152489in}{1.798682in}}%
\pgfpathcurveto{\pgfqpoint{2.158313in}{1.804506in}}{\pgfqpoint{2.161585in}{1.812406in}}{\pgfqpoint{2.161585in}{1.820642in}}%
\pgfpathcurveto{\pgfqpoint{2.161585in}{1.828879in}}{\pgfqpoint{2.158313in}{1.836779in}}{\pgfqpoint{2.152489in}{1.842603in}}%
\pgfpathcurveto{\pgfqpoint{2.146665in}{1.848426in}}{\pgfqpoint{2.138765in}{1.851699in}}{\pgfqpoint{2.130528in}{1.851699in}}%
\pgfpathcurveto{\pgfqpoint{2.122292in}{1.851699in}}{\pgfqpoint{2.114392in}{1.848426in}}{\pgfqpoint{2.108568in}{1.842603in}}%
\pgfpathcurveto{\pgfqpoint{2.102744in}{1.836779in}}{\pgfqpoint{2.099472in}{1.828879in}}{\pgfqpoint{2.099472in}{1.820642in}}%
\pgfpathcurveto{\pgfqpoint{2.099472in}{1.812406in}}{\pgfqpoint{2.102744in}{1.804506in}}{\pgfqpoint{2.108568in}{1.798682in}}%
\pgfpathcurveto{\pgfqpoint{2.114392in}{1.792858in}}{\pgfqpoint{2.122292in}{1.789586in}}{\pgfqpoint{2.130528in}{1.789586in}}%
\pgfpathclose%
\pgfusepath{stroke,fill}%
\end{pgfscope}%
\begin{pgfscope}%
\pgfpathrectangle{\pgfqpoint{0.100000in}{0.212622in}}{\pgfqpoint{3.696000in}{3.696000in}}%
\pgfusepath{clip}%
\pgfsetbuttcap%
\pgfsetroundjoin%
\definecolor{currentfill}{rgb}{0.121569,0.466667,0.705882}%
\pgfsetfillcolor{currentfill}%
\pgfsetfillopacity{0.635730}%
\pgfsetlinewidth{1.003750pt}%
\definecolor{currentstroke}{rgb}{0.121569,0.466667,0.705882}%
\pgfsetstrokecolor{currentstroke}%
\pgfsetstrokeopacity{0.635730}%
\pgfsetdash{}{0pt}%
\pgfpathmoveto{\pgfqpoint{2.131117in}{1.788919in}}%
\pgfpathcurveto{\pgfqpoint{2.139353in}{1.788919in}}{\pgfqpoint{2.147253in}{1.792192in}}{\pgfqpoint{2.153077in}{1.798015in}}%
\pgfpathcurveto{\pgfqpoint{2.158901in}{1.803839in}}{\pgfqpoint{2.162173in}{1.811739in}}{\pgfqpoint{2.162173in}{1.819976in}}%
\pgfpathcurveto{\pgfqpoint{2.162173in}{1.828212in}}{\pgfqpoint{2.158901in}{1.836112in}}{\pgfqpoint{2.153077in}{1.841936in}}%
\pgfpathcurveto{\pgfqpoint{2.147253in}{1.847760in}}{\pgfqpoint{2.139353in}{1.851032in}}{\pgfqpoint{2.131117in}{1.851032in}}%
\pgfpathcurveto{\pgfqpoint{2.122881in}{1.851032in}}{\pgfqpoint{2.114981in}{1.847760in}}{\pgfqpoint{2.109157in}{1.841936in}}%
\pgfpathcurveto{\pgfqpoint{2.103333in}{1.836112in}}{\pgfqpoint{2.100060in}{1.828212in}}{\pgfqpoint{2.100060in}{1.819976in}}%
\pgfpathcurveto{\pgfqpoint{2.100060in}{1.811739in}}{\pgfqpoint{2.103333in}{1.803839in}}{\pgfqpoint{2.109157in}{1.798015in}}%
\pgfpathcurveto{\pgfqpoint{2.114981in}{1.792192in}}{\pgfqpoint{2.122881in}{1.788919in}}{\pgfqpoint{2.131117in}{1.788919in}}%
\pgfpathclose%
\pgfusepath{stroke,fill}%
\end{pgfscope}%
\begin{pgfscope}%
\pgfpathrectangle{\pgfqpoint{0.100000in}{0.212622in}}{\pgfqpoint{3.696000in}{3.696000in}}%
\pgfusepath{clip}%
\pgfsetbuttcap%
\pgfsetroundjoin%
\definecolor{currentfill}{rgb}{0.121569,0.466667,0.705882}%
\pgfsetfillcolor{currentfill}%
\pgfsetfillopacity{0.637155}%
\pgfsetlinewidth{1.003750pt}%
\definecolor{currentstroke}{rgb}{0.121569,0.466667,0.705882}%
\pgfsetstrokecolor{currentstroke}%
\pgfsetstrokeopacity{0.637155}%
\pgfsetdash{}{0pt}%
\pgfpathmoveto{\pgfqpoint{2.132581in}{1.787095in}}%
\pgfpathcurveto{\pgfqpoint{2.140817in}{1.787095in}}{\pgfqpoint{2.148718in}{1.790367in}}{\pgfqpoint{2.154541in}{1.796191in}}%
\pgfpathcurveto{\pgfqpoint{2.160365in}{1.802015in}}{\pgfqpoint{2.163638in}{1.809915in}}{\pgfqpoint{2.163638in}{1.818151in}}%
\pgfpathcurveto{\pgfqpoint{2.163638in}{1.826388in}}{\pgfqpoint{2.160365in}{1.834288in}}{\pgfqpoint{2.154541in}{1.840111in}}%
\pgfpathcurveto{\pgfqpoint{2.148718in}{1.845935in}}{\pgfqpoint{2.140817in}{1.849208in}}{\pgfqpoint{2.132581in}{1.849208in}}%
\pgfpathcurveto{\pgfqpoint{2.124345in}{1.849208in}}{\pgfqpoint{2.116445in}{1.845935in}}{\pgfqpoint{2.110621in}{1.840111in}}%
\pgfpathcurveto{\pgfqpoint{2.104797in}{1.834288in}}{\pgfqpoint{2.101525in}{1.826388in}}{\pgfqpoint{2.101525in}{1.818151in}}%
\pgfpathcurveto{\pgfqpoint{2.101525in}{1.809915in}}{\pgfqpoint{2.104797in}{1.802015in}}{\pgfqpoint{2.110621in}{1.796191in}}%
\pgfpathcurveto{\pgfqpoint{2.116445in}{1.790367in}}{\pgfqpoint{2.124345in}{1.787095in}}{\pgfqpoint{2.132581in}{1.787095in}}%
\pgfpathclose%
\pgfusepath{stroke,fill}%
\end{pgfscope}%
\begin{pgfscope}%
\pgfpathrectangle{\pgfqpoint{0.100000in}{0.212622in}}{\pgfqpoint{3.696000in}{3.696000in}}%
\pgfusepath{clip}%
\pgfsetbuttcap%
\pgfsetroundjoin%
\definecolor{currentfill}{rgb}{0.121569,0.466667,0.705882}%
\pgfsetfillcolor{currentfill}%
\pgfsetfillopacity{0.638912}%
\pgfsetlinewidth{1.003750pt}%
\definecolor{currentstroke}{rgb}{0.121569,0.466667,0.705882}%
\pgfsetstrokecolor{currentstroke}%
\pgfsetstrokeopacity{0.638912}%
\pgfsetdash{}{0pt}%
\pgfpathmoveto{\pgfqpoint{2.133693in}{1.784593in}}%
\pgfpathcurveto{\pgfqpoint{2.141929in}{1.784593in}}{\pgfqpoint{2.149829in}{1.787865in}}{\pgfqpoint{2.155653in}{1.793689in}}%
\pgfpathcurveto{\pgfqpoint{2.161477in}{1.799513in}}{\pgfqpoint{2.164749in}{1.807413in}}{\pgfqpoint{2.164749in}{1.815649in}}%
\pgfpathcurveto{\pgfqpoint{2.164749in}{1.823886in}}{\pgfqpoint{2.161477in}{1.831786in}}{\pgfqpoint{2.155653in}{1.837610in}}%
\pgfpathcurveto{\pgfqpoint{2.149829in}{1.843434in}}{\pgfqpoint{2.141929in}{1.846706in}}{\pgfqpoint{2.133693in}{1.846706in}}%
\pgfpathcurveto{\pgfqpoint{2.125457in}{1.846706in}}{\pgfqpoint{2.117557in}{1.843434in}}{\pgfqpoint{2.111733in}{1.837610in}}%
\pgfpathcurveto{\pgfqpoint{2.105909in}{1.831786in}}{\pgfqpoint{2.102636in}{1.823886in}}{\pgfqpoint{2.102636in}{1.815649in}}%
\pgfpathcurveto{\pgfqpoint{2.102636in}{1.807413in}}{\pgfqpoint{2.105909in}{1.799513in}}{\pgfqpoint{2.111733in}{1.793689in}}%
\pgfpathcurveto{\pgfqpoint{2.117557in}{1.787865in}}{\pgfqpoint{2.125457in}{1.784593in}}{\pgfqpoint{2.133693in}{1.784593in}}%
\pgfpathclose%
\pgfusepath{stroke,fill}%
\end{pgfscope}%
\begin{pgfscope}%
\pgfpathrectangle{\pgfqpoint{0.100000in}{0.212622in}}{\pgfqpoint{3.696000in}{3.696000in}}%
\pgfusepath{clip}%
\pgfsetbuttcap%
\pgfsetroundjoin%
\definecolor{currentfill}{rgb}{0.121569,0.466667,0.705882}%
\pgfsetfillcolor{currentfill}%
\pgfsetfillopacity{0.641414}%
\pgfsetlinewidth{1.003750pt}%
\definecolor{currentstroke}{rgb}{0.121569,0.466667,0.705882}%
\pgfsetstrokecolor{currentstroke}%
\pgfsetstrokeopacity{0.641414}%
\pgfsetdash{}{0pt}%
\pgfpathmoveto{\pgfqpoint{2.135208in}{1.782311in}}%
\pgfpathcurveto{\pgfqpoint{2.143445in}{1.782311in}}{\pgfqpoint{2.151345in}{1.785584in}}{\pgfqpoint{2.157169in}{1.791407in}}%
\pgfpathcurveto{\pgfqpoint{2.162993in}{1.797231in}}{\pgfqpoint{2.166265in}{1.805131in}}{\pgfqpoint{2.166265in}{1.813368in}}%
\pgfpathcurveto{\pgfqpoint{2.166265in}{1.821604in}}{\pgfqpoint{2.162993in}{1.829504in}}{\pgfqpoint{2.157169in}{1.835328in}}%
\pgfpathcurveto{\pgfqpoint{2.151345in}{1.841152in}}{\pgfqpoint{2.143445in}{1.844424in}}{\pgfqpoint{2.135208in}{1.844424in}}%
\pgfpathcurveto{\pgfqpoint{2.126972in}{1.844424in}}{\pgfqpoint{2.119072in}{1.841152in}}{\pgfqpoint{2.113248in}{1.835328in}}%
\pgfpathcurveto{\pgfqpoint{2.107424in}{1.829504in}}{\pgfqpoint{2.104152in}{1.821604in}}{\pgfqpoint{2.104152in}{1.813368in}}%
\pgfpathcurveto{\pgfqpoint{2.104152in}{1.805131in}}{\pgfqpoint{2.107424in}{1.797231in}}{\pgfqpoint{2.113248in}{1.791407in}}%
\pgfpathcurveto{\pgfqpoint{2.119072in}{1.785584in}}{\pgfqpoint{2.126972in}{1.782311in}}{\pgfqpoint{2.135208in}{1.782311in}}%
\pgfpathclose%
\pgfusepath{stroke,fill}%
\end{pgfscope}%
\begin{pgfscope}%
\pgfpathrectangle{\pgfqpoint{0.100000in}{0.212622in}}{\pgfqpoint{3.696000in}{3.696000in}}%
\pgfusepath{clip}%
\pgfsetbuttcap%
\pgfsetroundjoin%
\definecolor{currentfill}{rgb}{0.121569,0.466667,0.705882}%
\pgfsetfillcolor{currentfill}%
\pgfsetfillopacity{0.644354}%
\pgfsetlinewidth{1.003750pt}%
\definecolor{currentstroke}{rgb}{0.121569,0.466667,0.705882}%
\pgfsetstrokecolor{currentstroke}%
\pgfsetstrokeopacity{0.644354}%
\pgfsetdash{}{0pt}%
\pgfpathmoveto{\pgfqpoint{2.137212in}{1.781498in}}%
\pgfpathcurveto{\pgfqpoint{2.145448in}{1.781498in}}{\pgfqpoint{2.153348in}{1.784770in}}{\pgfqpoint{2.159172in}{1.790594in}}%
\pgfpathcurveto{\pgfqpoint{2.164996in}{1.796418in}}{\pgfqpoint{2.168269in}{1.804318in}}{\pgfqpoint{2.168269in}{1.812554in}}%
\pgfpathcurveto{\pgfqpoint{2.168269in}{1.820790in}}{\pgfqpoint{2.164996in}{1.828690in}}{\pgfqpoint{2.159172in}{1.834514in}}%
\pgfpathcurveto{\pgfqpoint{2.153348in}{1.840338in}}{\pgfqpoint{2.145448in}{1.843611in}}{\pgfqpoint{2.137212in}{1.843611in}}%
\pgfpathcurveto{\pgfqpoint{2.128976in}{1.843611in}}{\pgfqpoint{2.121076in}{1.840338in}}{\pgfqpoint{2.115252in}{1.834514in}}%
\pgfpathcurveto{\pgfqpoint{2.109428in}{1.828690in}}{\pgfqpoint{2.106156in}{1.820790in}}{\pgfqpoint{2.106156in}{1.812554in}}%
\pgfpathcurveto{\pgfqpoint{2.106156in}{1.804318in}}{\pgfqpoint{2.109428in}{1.796418in}}{\pgfqpoint{2.115252in}{1.790594in}}%
\pgfpathcurveto{\pgfqpoint{2.121076in}{1.784770in}}{\pgfqpoint{2.128976in}{1.781498in}}{\pgfqpoint{2.137212in}{1.781498in}}%
\pgfpathclose%
\pgfusepath{stroke,fill}%
\end{pgfscope}%
\begin{pgfscope}%
\pgfpathrectangle{\pgfqpoint{0.100000in}{0.212622in}}{\pgfqpoint{3.696000in}{3.696000in}}%
\pgfusepath{clip}%
\pgfsetbuttcap%
\pgfsetroundjoin%
\definecolor{currentfill}{rgb}{0.121569,0.466667,0.705882}%
\pgfsetfillcolor{currentfill}%
\pgfsetfillopacity{0.647499}%
\pgfsetlinewidth{1.003750pt}%
\definecolor{currentstroke}{rgb}{0.121569,0.466667,0.705882}%
\pgfsetstrokecolor{currentstroke}%
\pgfsetstrokeopacity{0.647499}%
\pgfsetdash{}{0pt}%
\pgfpathmoveto{\pgfqpoint{2.139628in}{1.777754in}}%
\pgfpathcurveto{\pgfqpoint{2.147864in}{1.777754in}}{\pgfqpoint{2.155764in}{1.781027in}}{\pgfqpoint{2.161588in}{1.786851in}}%
\pgfpathcurveto{\pgfqpoint{2.167412in}{1.792675in}}{\pgfqpoint{2.170685in}{1.800575in}}{\pgfqpoint{2.170685in}{1.808811in}}%
\pgfpathcurveto{\pgfqpoint{2.170685in}{1.817047in}}{\pgfqpoint{2.167412in}{1.824947in}}{\pgfqpoint{2.161588in}{1.830771in}}%
\pgfpathcurveto{\pgfqpoint{2.155764in}{1.836595in}}{\pgfqpoint{2.147864in}{1.839867in}}{\pgfqpoint{2.139628in}{1.839867in}}%
\pgfpathcurveto{\pgfqpoint{2.131392in}{1.839867in}}{\pgfqpoint{2.123492in}{1.836595in}}{\pgfqpoint{2.117668in}{1.830771in}}%
\pgfpathcurveto{\pgfqpoint{2.111844in}{1.824947in}}{\pgfqpoint{2.108572in}{1.817047in}}{\pgfqpoint{2.108572in}{1.808811in}}%
\pgfpathcurveto{\pgfqpoint{2.108572in}{1.800575in}}{\pgfqpoint{2.111844in}{1.792675in}}{\pgfqpoint{2.117668in}{1.786851in}}%
\pgfpathcurveto{\pgfqpoint{2.123492in}{1.781027in}}{\pgfqpoint{2.131392in}{1.777754in}}{\pgfqpoint{2.139628in}{1.777754in}}%
\pgfpathclose%
\pgfusepath{stroke,fill}%
\end{pgfscope}%
\begin{pgfscope}%
\pgfpathrectangle{\pgfqpoint{0.100000in}{0.212622in}}{\pgfqpoint{3.696000in}{3.696000in}}%
\pgfusepath{clip}%
\pgfsetbuttcap%
\pgfsetroundjoin%
\definecolor{currentfill}{rgb}{0.121569,0.466667,0.705882}%
\pgfsetfillcolor{currentfill}%
\pgfsetfillopacity{0.650604}%
\pgfsetlinewidth{1.003750pt}%
\definecolor{currentstroke}{rgb}{0.121569,0.466667,0.705882}%
\pgfsetstrokecolor{currentstroke}%
\pgfsetstrokeopacity{0.650604}%
\pgfsetdash{}{0pt}%
\pgfpathmoveto{\pgfqpoint{2.141981in}{1.772097in}}%
\pgfpathcurveto{\pgfqpoint{2.150218in}{1.772097in}}{\pgfqpoint{2.158118in}{1.775369in}}{\pgfqpoint{2.163942in}{1.781193in}}%
\pgfpathcurveto{\pgfqpoint{2.169766in}{1.787017in}}{\pgfqpoint{2.173038in}{1.794917in}}{\pgfqpoint{2.173038in}{1.803153in}}%
\pgfpathcurveto{\pgfqpoint{2.173038in}{1.811390in}}{\pgfqpoint{2.169766in}{1.819290in}}{\pgfqpoint{2.163942in}{1.825114in}}%
\pgfpathcurveto{\pgfqpoint{2.158118in}{1.830937in}}{\pgfqpoint{2.150218in}{1.834210in}}{\pgfqpoint{2.141981in}{1.834210in}}%
\pgfpathcurveto{\pgfqpoint{2.133745in}{1.834210in}}{\pgfqpoint{2.125845in}{1.830937in}}{\pgfqpoint{2.120021in}{1.825114in}}%
\pgfpathcurveto{\pgfqpoint{2.114197in}{1.819290in}}{\pgfqpoint{2.110925in}{1.811390in}}{\pgfqpoint{2.110925in}{1.803153in}}%
\pgfpathcurveto{\pgfqpoint{2.110925in}{1.794917in}}{\pgfqpoint{2.114197in}{1.787017in}}{\pgfqpoint{2.120021in}{1.781193in}}%
\pgfpathcurveto{\pgfqpoint{2.125845in}{1.775369in}}{\pgfqpoint{2.133745in}{1.772097in}}{\pgfqpoint{2.141981in}{1.772097in}}%
\pgfpathclose%
\pgfusepath{stroke,fill}%
\end{pgfscope}%
\begin{pgfscope}%
\pgfpathrectangle{\pgfqpoint{0.100000in}{0.212622in}}{\pgfqpoint{3.696000in}{3.696000in}}%
\pgfusepath{clip}%
\pgfsetbuttcap%
\pgfsetroundjoin%
\definecolor{currentfill}{rgb}{0.121569,0.466667,0.705882}%
\pgfsetfillcolor{currentfill}%
\pgfsetfillopacity{0.655020}%
\pgfsetlinewidth{1.003750pt}%
\definecolor{currentstroke}{rgb}{0.121569,0.466667,0.705882}%
\pgfsetstrokecolor{currentstroke}%
\pgfsetstrokeopacity{0.655020}%
\pgfsetdash{}{0pt}%
\pgfpathmoveto{\pgfqpoint{2.144415in}{1.768939in}}%
\pgfpathcurveto{\pgfqpoint{2.152651in}{1.768939in}}{\pgfqpoint{2.160551in}{1.772212in}}{\pgfqpoint{2.166375in}{1.778036in}}%
\pgfpathcurveto{\pgfqpoint{2.172199in}{1.783860in}}{\pgfqpoint{2.175471in}{1.791760in}}{\pgfqpoint{2.175471in}{1.799996in}}%
\pgfpathcurveto{\pgfqpoint{2.175471in}{1.808232in}}{\pgfqpoint{2.172199in}{1.816132in}}{\pgfqpoint{2.166375in}{1.821956in}}%
\pgfpathcurveto{\pgfqpoint{2.160551in}{1.827780in}}{\pgfqpoint{2.152651in}{1.831052in}}{\pgfqpoint{2.144415in}{1.831052in}}%
\pgfpathcurveto{\pgfqpoint{2.136178in}{1.831052in}}{\pgfqpoint{2.128278in}{1.827780in}}{\pgfqpoint{2.122454in}{1.821956in}}%
\pgfpathcurveto{\pgfqpoint{2.116631in}{1.816132in}}{\pgfqpoint{2.113358in}{1.808232in}}{\pgfqpoint{2.113358in}{1.799996in}}%
\pgfpathcurveto{\pgfqpoint{2.113358in}{1.791760in}}{\pgfqpoint{2.116631in}{1.783860in}}{\pgfqpoint{2.122454in}{1.778036in}}%
\pgfpathcurveto{\pgfqpoint{2.128278in}{1.772212in}}{\pgfqpoint{2.136178in}{1.768939in}}{\pgfqpoint{2.144415in}{1.768939in}}%
\pgfpathclose%
\pgfusepath{stroke,fill}%
\end{pgfscope}%
\begin{pgfscope}%
\pgfpathrectangle{\pgfqpoint{0.100000in}{0.212622in}}{\pgfqpoint{3.696000in}{3.696000in}}%
\pgfusepath{clip}%
\pgfsetbuttcap%
\pgfsetroundjoin%
\definecolor{currentfill}{rgb}{0.121569,0.466667,0.705882}%
\pgfsetfillcolor{currentfill}%
\pgfsetfillopacity{0.660013}%
\pgfsetlinewidth{1.003750pt}%
\definecolor{currentstroke}{rgb}{0.121569,0.466667,0.705882}%
\pgfsetstrokecolor{currentstroke}%
\pgfsetstrokeopacity{0.660013}%
\pgfsetdash{}{0pt}%
\pgfpathmoveto{\pgfqpoint{2.146556in}{1.766948in}}%
\pgfpathcurveto{\pgfqpoint{2.154793in}{1.766948in}}{\pgfqpoint{2.162693in}{1.770221in}}{\pgfqpoint{2.168517in}{1.776045in}}%
\pgfpathcurveto{\pgfqpoint{2.174341in}{1.781869in}}{\pgfqpoint{2.177613in}{1.789769in}}{\pgfqpoint{2.177613in}{1.798005in}}%
\pgfpathcurveto{\pgfqpoint{2.177613in}{1.806241in}}{\pgfqpoint{2.174341in}{1.814141in}}{\pgfqpoint{2.168517in}{1.819965in}}%
\pgfpathcurveto{\pgfqpoint{2.162693in}{1.825789in}}{\pgfqpoint{2.154793in}{1.829061in}}{\pgfqpoint{2.146556in}{1.829061in}}%
\pgfpathcurveto{\pgfqpoint{2.138320in}{1.829061in}}{\pgfqpoint{2.130420in}{1.825789in}}{\pgfqpoint{2.124596in}{1.819965in}}%
\pgfpathcurveto{\pgfqpoint{2.118772in}{1.814141in}}{\pgfqpoint{2.115500in}{1.806241in}}{\pgfqpoint{2.115500in}{1.798005in}}%
\pgfpathcurveto{\pgfqpoint{2.115500in}{1.789769in}}{\pgfqpoint{2.118772in}{1.781869in}}{\pgfqpoint{2.124596in}{1.776045in}}%
\pgfpathcurveto{\pgfqpoint{2.130420in}{1.770221in}}{\pgfqpoint{2.138320in}{1.766948in}}{\pgfqpoint{2.146556in}{1.766948in}}%
\pgfpathclose%
\pgfusepath{stroke,fill}%
\end{pgfscope}%
\begin{pgfscope}%
\pgfpathrectangle{\pgfqpoint{0.100000in}{0.212622in}}{\pgfqpoint{3.696000in}{3.696000in}}%
\pgfusepath{clip}%
\pgfsetbuttcap%
\pgfsetroundjoin%
\definecolor{currentfill}{rgb}{0.121569,0.466667,0.705882}%
\pgfsetfillcolor{currentfill}%
\pgfsetfillopacity{0.665210}%
\pgfsetlinewidth{1.003750pt}%
\definecolor{currentstroke}{rgb}{0.121569,0.466667,0.705882}%
\pgfsetstrokecolor{currentstroke}%
\pgfsetstrokeopacity{0.665210}%
\pgfsetdash{}{0pt}%
\pgfpathmoveto{\pgfqpoint{2.150170in}{1.762055in}}%
\pgfpathcurveto{\pgfqpoint{2.158406in}{1.762055in}}{\pgfqpoint{2.166306in}{1.765327in}}{\pgfqpoint{2.172130in}{1.771151in}}%
\pgfpathcurveto{\pgfqpoint{2.177954in}{1.776975in}}{\pgfqpoint{2.181226in}{1.784875in}}{\pgfqpoint{2.181226in}{1.793112in}}%
\pgfpathcurveto{\pgfqpoint{2.181226in}{1.801348in}}{\pgfqpoint{2.177954in}{1.809248in}}{\pgfqpoint{2.172130in}{1.815072in}}%
\pgfpathcurveto{\pgfqpoint{2.166306in}{1.820896in}}{\pgfqpoint{2.158406in}{1.824168in}}{\pgfqpoint{2.150170in}{1.824168in}}%
\pgfpathcurveto{\pgfqpoint{2.141933in}{1.824168in}}{\pgfqpoint{2.134033in}{1.820896in}}{\pgfqpoint{2.128209in}{1.815072in}}%
\pgfpathcurveto{\pgfqpoint{2.122386in}{1.809248in}}{\pgfqpoint{2.119113in}{1.801348in}}{\pgfqpoint{2.119113in}{1.793112in}}%
\pgfpathcurveto{\pgfqpoint{2.119113in}{1.784875in}}{\pgfqpoint{2.122386in}{1.776975in}}{\pgfqpoint{2.128209in}{1.771151in}}%
\pgfpathcurveto{\pgfqpoint{2.134033in}{1.765327in}}{\pgfqpoint{2.141933in}{1.762055in}}{\pgfqpoint{2.150170in}{1.762055in}}%
\pgfpathclose%
\pgfusepath{stroke,fill}%
\end{pgfscope}%
\begin{pgfscope}%
\pgfpathrectangle{\pgfqpoint{0.100000in}{0.212622in}}{\pgfqpoint{3.696000in}{3.696000in}}%
\pgfusepath{clip}%
\pgfsetbuttcap%
\pgfsetroundjoin%
\definecolor{currentfill}{rgb}{0.121569,0.466667,0.705882}%
\pgfsetfillcolor{currentfill}%
\pgfsetfillopacity{0.670508}%
\pgfsetlinewidth{1.003750pt}%
\definecolor{currentstroke}{rgb}{0.121569,0.466667,0.705882}%
\pgfsetstrokecolor{currentstroke}%
\pgfsetstrokeopacity{0.670508}%
\pgfsetdash{}{0pt}%
\pgfpathmoveto{\pgfqpoint{2.153918in}{1.754811in}}%
\pgfpathcurveto{\pgfqpoint{2.162154in}{1.754811in}}{\pgfqpoint{2.170054in}{1.758084in}}{\pgfqpoint{2.175878in}{1.763907in}}%
\pgfpathcurveto{\pgfqpoint{2.181702in}{1.769731in}}{\pgfqpoint{2.184974in}{1.777631in}}{\pgfqpoint{2.184974in}{1.785868in}}%
\pgfpathcurveto{\pgfqpoint{2.184974in}{1.794104in}}{\pgfqpoint{2.181702in}{1.802004in}}{\pgfqpoint{2.175878in}{1.807828in}}%
\pgfpathcurveto{\pgfqpoint{2.170054in}{1.813652in}}{\pgfqpoint{2.162154in}{1.816924in}}{\pgfqpoint{2.153918in}{1.816924in}}%
\pgfpathcurveto{\pgfqpoint{2.145681in}{1.816924in}}{\pgfqpoint{2.137781in}{1.813652in}}{\pgfqpoint{2.131957in}{1.807828in}}%
\pgfpathcurveto{\pgfqpoint{2.126134in}{1.802004in}}{\pgfqpoint{2.122861in}{1.794104in}}{\pgfqpoint{2.122861in}{1.785868in}}%
\pgfpathcurveto{\pgfqpoint{2.122861in}{1.777631in}}{\pgfqpoint{2.126134in}{1.769731in}}{\pgfqpoint{2.131957in}{1.763907in}}%
\pgfpathcurveto{\pgfqpoint{2.137781in}{1.758084in}}{\pgfqpoint{2.145681in}{1.754811in}}{\pgfqpoint{2.153918in}{1.754811in}}%
\pgfpathclose%
\pgfusepath{stroke,fill}%
\end{pgfscope}%
\begin{pgfscope}%
\pgfpathrectangle{\pgfqpoint{0.100000in}{0.212622in}}{\pgfqpoint{3.696000in}{3.696000in}}%
\pgfusepath{clip}%
\pgfsetbuttcap%
\pgfsetroundjoin%
\definecolor{currentfill}{rgb}{0.121569,0.466667,0.705882}%
\pgfsetfillcolor{currentfill}%
\pgfsetfillopacity{0.676803}%
\pgfsetlinewidth{1.003750pt}%
\definecolor{currentstroke}{rgb}{0.121569,0.466667,0.705882}%
\pgfsetstrokecolor{currentstroke}%
\pgfsetstrokeopacity{0.676803}%
\pgfsetdash{}{0pt}%
\pgfpathmoveto{\pgfqpoint{2.158878in}{1.748746in}}%
\pgfpathcurveto{\pgfqpoint{2.167114in}{1.748746in}}{\pgfqpoint{2.175015in}{1.752018in}}{\pgfqpoint{2.180838in}{1.757842in}}%
\pgfpathcurveto{\pgfqpoint{2.186662in}{1.763666in}}{\pgfqpoint{2.189935in}{1.771566in}}{\pgfqpoint{2.189935in}{1.779802in}}%
\pgfpathcurveto{\pgfqpoint{2.189935in}{1.788039in}}{\pgfqpoint{2.186662in}{1.795939in}}{\pgfqpoint{2.180838in}{1.801763in}}%
\pgfpathcurveto{\pgfqpoint{2.175015in}{1.807587in}}{\pgfqpoint{2.167114in}{1.810859in}}{\pgfqpoint{2.158878in}{1.810859in}}%
\pgfpathcurveto{\pgfqpoint{2.150642in}{1.810859in}}{\pgfqpoint{2.142742in}{1.807587in}}{\pgfqpoint{2.136918in}{1.801763in}}%
\pgfpathcurveto{\pgfqpoint{2.131094in}{1.795939in}}{\pgfqpoint{2.127822in}{1.788039in}}{\pgfqpoint{2.127822in}{1.779802in}}%
\pgfpathcurveto{\pgfqpoint{2.127822in}{1.771566in}}{\pgfqpoint{2.131094in}{1.763666in}}{\pgfqpoint{2.136918in}{1.757842in}}%
\pgfpathcurveto{\pgfqpoint{2.142742in}{1.752018in}}{\pgfqpoint{2.150642in}{1.748746in}}{\pgfqpoint{2.158878in}{1.748746in}}%
\pgfpathclose%
\pgfusepath{stroke,fill}%
\end{pgfscope}%
\begin{pgfscope}%
\pgfpathrectangle{\pgfqpoint{0.100000in}{0.212622in}}{\pgfqpoint{3.696000in}{3.696000in}}%
\pgfusepath{clip}%
\pgfsetbuttcap%
\pgfsetroundjoin%
\definecolor{currentfill}{rgb}{0.121569,0.466667,0.705882}%
\pgfsetfillcolor{currentfill}%
\pgfsetfillopacity{0.683863}%
\pgfsetlinewidth{1.003750pt}%
\definecolor{currentstroke}{rgb}{0.121569,0.466667,0.705882}%
\pgfsetstrokecolor{currentstroke}%
\pgfsetstrokeopacity{0.683863}%
\pgfsetdash{}{0pt}%
\pgfpathmoveto{\pgfqpoint{2.163397in}{1.744262in}}%
\pgfpathcurveto{\pgfqpoint{2.171634in}{1.744262in}}{\pgfqpoint{2.179534in}{1.747535in}}{\pgfqpoint{2.185358in}{1.753359in}}%
\pgfpathcurveto{\pgfqpoint{2.191182in}{1.759183in}}{\pgfqpoint{2.194454in}{1.767083in}}{\pgfqpoint{2.194454in}{1.775319in}}%
\pgfpathcurveto{\pgfqpoint{2.194454in}{1.783555in}}{\pgfqpoint{2.191182in}{1.791455in}}{\pgfqpoint{2.185358in}{1.797279in}}%
\pgfpathcurveto{\pgfqpoint{2.179534in}{1.803103in}}{\pgfqpoint{2.171634in}{1.806375in}}{\pgfqpoint{2.163397in}{1.806375in}}%
\pgfpathcurveto{\pgfqpoint{2.155161in}{1.806375in}}{\pgfqpoint{2.147261in}{1.803103in}}{\pgfqpoint{2.141437in}{1.797279in}}%
\pgfpathcurveto{\pgfqpoint{2.135613in}{1.791455in}}{\pgfqpoint{2.132341in}{1.783555in}}{\pgfqpoint{2.132341in}{1.775319in}}%
\pgfpathcurveto{\pgfqpoint{2.132341in}{1.767083in}}{\pgfqpoint{2.135613in}{1.759183in}}{\pgfqpoint{2.141437in}{1.753359in}}%
\pgfpathcurveto{\pgfqpoint{2.147261in}{1.747535in}}{\pgfqpoint{2.155161in}{1.744262in}}{\pgfqpoint{2.163397in}{1.744262in}}%
\pgfpathclose%
\pgfusepath{stroke,fill}%
\end{pgfscope}%
\begin{pgfscope}%
\pgfpathrectangle{\pgfqpoint{0.100000in}{0.212622in}}{\pgfqpoint{3.696000in}{3.696000in}}%
\pgfusepath{clip}%
\pgfsetbuttcap%
\pgfsetroundjoin%
\definecolor{currentfill}{rgb}{0.121569,0.466667,0.705882}%
\pgfsetfillcolor{currentfill}%
\pgfsetfillopacity{0.690389}%
\pgfsetlinewidth{1.003750pt}%
\definecolor{currentstroke}{rgb}{0.121569,0.466667,0.705882}%
\pgfsetstrokecolor{currentstroke}%
\pgfsetstrokeopacity{0.690389}%
\pgfsetdash{}{0pt}%
\pgfpathmoveto{\pgfqpoint{2.167401in}{1.732142in}}%
\pgfpathcurveto{\pgfqpoint{2.175638in}{1.732142in}}{\pgfqpoint{2.183538in}{1.735414in}}{\pgfqpoint{2.189362in}{1.741238in}}%
\pgfpathcurveto{\pgfqpoint{2.195185in}{1.747062in}}{\pgfqpoint{2.198458in}{1.754962in}}{\pgfqpoint{2.198458in}{1.763198in}}%
\pgfpathcurveto{\pgfqpoint{2.198458in}{1.771434in}}{\pgfqpoint{2.195185in}{1.779335in}}{\pgfqpoint{2.189362in}{1.785158in}}%
\pgfpathcurveto{\pgfqpoint{2.183538in}{1.790982in}}{\pgfqpoint{2.175638in}{1.794255in}}{\pgfqpoint{2.167401in}{1.794255in}}%
\pgfpathcurveto{\pgfqpoint{2.159165in}{1.794255in}}{\pgfqpoint{2.151265in}{1.790982in}}{\pgfqpoint{2.145441in}{1.785158in}}%
\pgfpathcurveto{\pgfqpoint{2.139617in}{1.779335in}}{\pgfqpoint{2.136345in}{1.771434in}}{\pgfqpoint{2.136345in}{1.763198in}}%
\pgfpathcurveto{\pgfqpoint{2.136345in}{1.754962in}}{\pgfqpoint{2.139617in}{1.747062in}}{\pgfqpoint{2.145441in}{1.741238in}}%
\pgfpathcurveto{\pgfqpoint{2.151265in}{1.735414in}}{\pgfqpoint{2.159165in}{1.732142in}}{\pgfqpoint{2.167401in}{1.732142in}}%
\pgfpathclose%
\pgfusepath{stroke,fill}%
\end{pgfscope}%
\begin{pgfscope}%
\pgfpathrectangle{\pgfqpoint{0.100000in}{0.212622in}}{\pgfqpoint{3.696000in}{3.696000in}}%
\pgfusepath{clip}%
\pgfsetbuttcap%
\pgfsetroundjoin%
\definecolor{currentfill}{rgb}{0.121569,0.466667,0.705882}%
\pgfsetfillcolor{currentfill}%
\pgfsetfillopacity{0.696981}%
\pgfsetlinewidth{1.003750pt}%
\definecolor{currentstroke}{rgb}{0.121569,0.466667,0.705882}%
\pgfsetstrokecolor{currentstroke}%
\pgfsetstrokeopacity{0.696981}%
\pgfsetdash{}{0pt}%
\pgfpathmoveto{\pgfqpoint{2.171736in}{1.719305in}}%
\pgfpathcurveto{\pgfqpoint{2.179972in}{1.719305in}}{\pgfqpoint{2.187872in}{1.722578in}}{\pgfqpoint{2.193696in}{1.728402in}}%
\pgfpathcurveto{\pgfqpoint{2.199520in}{1.734225in}}{\pgfqpoint{2.202792in}{1.742126in}}{\pgfqpoint{2.202792in}{1.750362in}}%
\pgfpathcurveto{\pgfqpoint{2.202792in}{1.758598in}}{\pgfqpoint{2.199520in}{1.766498in}}{\pgfqpoint{2.193696in}{1.772322in}}%
\pgfpathcurveto{\pgfqpoint{2.187872in}{1.778146in}}{\pgfqpoint{2.179972in}{1.781418in}}{\pgfqpoint{2.171736in}{1.781418in}}%
\pgfpathcurveto{\pgfqpoint{2.163500in}{1.781418in}}{\pgfqpoint{2.155600in}{1.778146in}}{\pgfqpoint{2.149776in}{1.772322in}}%
\pgfpathcurveto{\pgfqpoint{2.143952in}{1.766498in}}{\pgfqpoint{2.140679in}{1.758598in}}{\pgfqpoint{2.140679in}{1.750362in}}%
\pgfpathcurveto{\pgfqpoint{2.140679in}{1.742126in}}{\pgfqpoint{2.143952in}{1.734225in}}{\pgfqpoint{2.149776in}{1.728402in}}%
\pgfpathcurveto{\pgfqpoint{2.155600in}{1.722578in}}{\pgfqpoint{2.163500in}{1.719305in}}{\pgfqpoint{2.171736in}{1.719305in}}%
\pgfpathclose%
\pgfusepath{stroke,fill}%
\end{pgfscope}%
\begin{pgfscope}%
\pgfpathrectangle{\pgfqpoint{0.100000in}{0.212622in}}{\pgfqpoint{3.696000in}{3.696000in}}%
\pgfusepath{clip}%
\pgfsetbuttcap%
\pgfsetroundjoin%
\definecolor{currentfill}{rgb}{0.121569,0.466667,0.705882}%
\pgfsetfillcolor{currentfill}%
\pgfsetfillopacity{0.701312}%
\pgfsetlinewidth{1.003750pt}%
\definecolor{currentstroke}{rgb}{0.121569,0.466667,0.705882}%
\pgfsetstrokecolor{currentstroke}%
\pgfsetstrokeopacity{0.701312}%
\pgfsetdash{}{0pt}%
\pgfpathmoveto{\pgfqpoint{2.175177in}{1.717263in}}%
\pgfpathcurveto{\pgfqpoint{2.183413in}{1.717263in}}{\pgfqpoint{2.191313in}{1.720535in}}{\pgfqpoint{2.197137in}{1.726359in}}%
\pgfpathcurveto{\pgfqpoint{2.202961in}{1.732183in}}{\pgfqpoint{2.206234in}{1.740083in}}{\pgfqpoint{2.206234in}{1.748319in}}%
\pgfpathcurveto{\pgfqpoint{2.206234in}{1.756556in}}{\pgfqpoint{2.202961in}{1.764456in}}{\pgfqpoint{2.197137in}{1.770280in}}%
\pgfpathcurveto{\pgfqpoint{2.191313in}{1.776104in}}{\pgfqpoint{2.183413in}{1.779376in}}{\pgfqpoint{2.175177in}{1.779376in}}%
\pgfpathcurveto{\pgfqpoint{2.166941in}{1.779376in}}{\pgfqpoint{2.159041in}{1.776104in}}{\pgfqpoint{2.153217in}{1.770280in}}%
\pgfpathcurveto{\pgfqpoint{2.147393in}{1.764456in}}{\pgfqpoint{2.144121in}{1.756556in}}{\pgfqpoint{2.144121in}{1.748319in}}%
\pgfpathcurveto{\pgfqpoint{2.144121in}{1.740083in}}{\pgfqpoint{2.147393in}{1.732183in}}{\pgfqpoint{2.153217in}{1.726359in}}%
\pgfpathcurveto{\pgfqpoint{2.159041in}{1.720535in}}{\pgfqpoint{2.166941in}{1.717263in}}{\pgfqpoint{2.175177in}{1.717263in}}%
\pgfpathclose%
\pgfusepath{stroke,fill}%
\end{pgfscope}%
\begin{pgfscope}%
\pgfpathrectangle{\pgfqpoint{0.100000in}{0.212622in}}{\pgfqpoint{3.696000in}{3.696000in}}%
\pgfusepath{clip}%
\pgfsetbuttcap%
\pgfsetroundjoin%
\definecolor{currentfill}{rgb}{0.121569,0.466667,0.705882}%
\pgfsetfillcolor{currentfill}%
\pgfsetfillopacity{0.705604}%
\pgfsetlinewidth{1.003750pt}%
\definecolor{currentstroke}{rgb}{0.121569,0.466667,0.705882}%
\pgfsetstrokecolor{currentstroke}%
\pgfsetstrokeopacity{0.705604}%
\pgfsetdash{}{0pt}%
\pgfpathmoveto{\pgfqpoint{2.178600in}{1.713794in}}%
\pgfpathcurveto{\pgfqpoint{2.186836in}{1.713794in}}{\pgfqpoint{2.194736in}{1.717067in}}{\pgfqpoint{2.200560in}{1.722890in}}%
\pgfpathcurveto{\pgfqpoint{2.206384in}{1.728714in}}{\pgfqpoint{2.209656in}{1.736614in}}{\pgfqpoint{2.209656in}{1.744851in}}%
\pgfpathcurveto{\pgfqpoint{2.209656in}{1.753087in}}{\pgfqpoint{2.206384in}{1.760987in}}{\pgfqpoint{2.200560in}{1.766811in}}%
\pgfpathcurveto{\pgfqpoint{2.194736in}{1.772635in}}{\pgfqpoint{2.186836in}{1.775907in}}{\pgfqpoint{2.178600in}{1.775907in}}%
\pgfpathcurveto{\pgfqpoint{2.170364in}{1.775907in}}{\pgfqpoint{2.162464in}{1.772635in}}{\pgfqpoint{2.156640in}{1.766811in}}%
\pgfpathcurveto{\pgfqpoint{2.150816in}{1.760987in}}{\pgfqpoint{2.147543in}{1.753087in}}{\pgfqpoint{2.147543in}{1.744851in}}%
\pgfpathcurveto{\pgfqpoint{2.147543in}{1.736614in}}{\pgfqpoint{2.150816in}{1.728714in}}{\pgfqpoint{2.156640in}{1.722890in}}%
\pgfpathcurveto{\pgfqpoint{2.162464in}{1.717067in}}{\pgfqpoint{2.170364in}{1.713794in}}{\pgfqpoint{2.178600in}{1.713794in}}%
\pgfpathclose%
\pgfusepath{stroke,fill}%
\end{pgfscope}%
\begin{pgfscope}%
\pgfpathrectangle{\pgfqpoint{0.100000in}{0.212622in}}{\pgfqpoint{3.696000in}{3.696000in}}%
\pgfusepath{clip}%
\pgfsetbuttcap%
\pgfsetroundjoin%
\definecolor{currentfill}{rgb}{0.121569,0.466667,0.705882}%
\pgfsetfillcolor{currentfill}%
\pgfsetfillopacity{0.707539}%
\pgfsetlinewidth{1.003750pt}%
\definecolor{currentstroke}{rgb}{0.121569,0.466667,0.705882}%
\pgfsetstrokecolor{currentstroke}%
\pgfsetstrokeopacity{0.707539}%
\pgfsetdash{}{0pt}%
\pgfpathmoveto{\pgfqpoint{2.180508in}{1.709173in}}%
\pgfpathcurveto{\pgfqpoint{2.188744in}{1.709173in}}{\pgfqpoint{2.196644in}{1.712446in}}{\pgfqpoint{2.202468in}{1.718270in}}%
\pgfpathcurveto{\pgfqpoint{2.208292in}{1.724094in}}{\pgfqpoint{2.211564in}{1.731994in}}{\pgfqpoint{2.211564in}{1.740230in}}%
\pgfpathcurveto{\pgfqpoint{2.211564in}{1.748466in}}{\pgfqpoint{2.208292in}{1.756366in}}{\pgfqpoint{2.202468in}{1.762190in}}%
\pgfpathcurveto{\pgfqpoint{2.196644in}{1.768014in}}{\pgfqpoint{2.188744in}{1.771286in}}{\pgfqpoint{2.180508in}{1.771286in}}%
\pgfpathcurveto{\pgfqpoint{2.172272in}{1.771286in}}{\pgfqpoint{2.164372in}{1.768014in}}{\pgfqpoint{2.158548in}{1.762190in}}%
\pgfpathcurveto{\pgfqpoint{2.152724in}{1.756366in}}{\pgfqpoint{2.149451in}{1.748466in}}{\pgfqpoint{2.149451in}{1.740230in}}%
\pgfpathcurveto{\pgfqpoint{2.149451in}{1.731994in}}{\pgfqpoint{2.152724in}{1.724094in}}{\pgfqpoint{2.158548in}{1.718270in}}%
\pgfpathcurveto{\pgfqpoint{2.164372in}{1.712446in}}{\pgfqpoint{2.172272in}{1.709173in}}{\pgfqpoint{2.180508in}{1.709173in}}%
\pgfpathclose%
\pgfusepath{stroke,fill}%
\end{pgfscope}%
\begin{pgfscope}%
\pgfpathrectangle{\pgfqpoint{0.100000in}{0.212622in}}{\pgfqpoint{3.696000in}{3.696000in}}%
\pgfusepath{clip}%
\pgfsetbuttcap%
\pgfsetroundjoin%
\definecolor{currentfill}{rgb}{0.121569,0.466667,0.705882}%
\pgfsetfillcolor{currentfill}%
\pgfsetfillopacity{0.708668}%
\pgfsetlinewidth{1.003750pt}%
\definecolor{currentstroke}{rgb}{0.121569,0.466667,0.705882}%
\pgfsetstrokecolor{currentstroke}%
\pgfsetstrokeopacity{0.708668}%
\pgfsetdash{}{0pt}%
\pgfpathmoveto{\pgfqpoint{2.181577in}{1.707050in}}%
\pgfpathcurveto{\pgfqpoint{2.189813in}{1.707050in}}{\pgfqpoint{2.197713in}{1.710322in}}{\pgfqpoint{2.203537in}{1.716146in}}%
\pgfpathcurveto{\pgfqpoint{2.209361in}{1.721970in}}{\pgfqpoint{2.212634in}{1.729870in}}{\pgfqpoint{2.212634in}{1.738106in}}%
\pgfpathcurveto{\pgfqpoint{2.212634in}{1.746343in}}{\pgfqpoint{2.209361in}{1.754243in}}{\pgfqpoint{2.203537in}{1.760067in}}%
\pgfpathcurveto{\pgfqpoint{2.197713in}{1.765891in}}{\pgfqpoint{2.189813in}{1.769163in}}{\pgfqpoint{2.181577in}{1.769163in}}%
\pgfpathcurveto{\pgfqpoint{2.173341in}{1.769163in}}{\pgfqpoint{2.165441in}{1.765891in}}{\pgfqpoint{2.159617in}{1.760067in}}%
\pgfpathcurveto{\pgfqpoint{2.153793in}{1.754243in}}{\pgfqpoint{2.150521in}{1.746343in}}{\pgfqpoint{2.150521in}{1.738106in}}%
\pgfpathcurveto{\pgfqpoint{2.150521in}{1.729870in}}{\pgfqpoint{2.153793in}{1.721970in}}{\pgfqpoint{2.159617in}{1.716146in}}%
\pgfpathcurveto{\pgfqpoint{2.165441in}{1.710322in}}{\pgfqpoint{2.173341in}{1.707050in}}{\pgfqpoint{2.181577in}{1.707050in}}%
\pgfpathclose%
\pgfusepath{stroke,fill}%
\end{pgfscope}%
\begin{pgfscope}%
\pgfpathrectangle{\pgfqpoint{0.100000in}{0.212622in}}{\pgfqpoint{3.696000in}{3.696000in}}%
\pgfusepath{clip}%
\pgfsetbuttcap%
\pgfsetroundjoin%
\definecolor{currentfill}{rgb}{0.121569,0.466667,0.705882}%
\pgfsetfillcolor{currentfill}%
\pgfsetfillopacity{0.709420}%
\pgfsetlinewidth{1.003750pt}%
\definecolor{currentstroke}{rgb}{0.121569,0.466667,0.705882}%
\pgfsetstrokecolor{currentstroke}%
\pgfsetstrokeopacity{0.709420}%
\pgfsetdash{}{0pt}%
\pgfpathmoveto{\pgfqpoint{2.182106in}{1.706675in}}%
\pgfpathcurveto{\pgfqpoint{2.190342in}{1.706675in}}{\pgfqpoint{2.198242in}{1.709948in}}{\pgfqpoint{2.204066in}{1.715772in}}%
\pgfpathcurveto{\pgfqpoint{2.209890in}{1.721596in}}{\pgfqpoint{2.213163in}{1.729496in}}{\pgfqpoint{2.213163in}{1.737732in}}%
\pgfpathcurveto{\pgfqpoint{2.213163in}{1.745968in}}{\pgfqpoint{2.209890in}{1.753868in}}{\pgfqpoint{2.204066in}{1.759692in}}%
\pgfpathcurveto{\pgfqpoint{2.198242in}{1.765516in}}{\pgfqpoint{2.190342in}{1.768788in}}{\pgfqpoint{2.182106in}{1.768788in}}%
\pgfpathcurveto{\pgfqpoint{2.173870in}{1.768788in}}{\pgfqpoint{2.165970in}{1.765516in}}{\pgfqpoint{2.160146in}{1.759692in}}%
\pgfpathcurveto{\pgfqpoint{2.154322in}{1.753868in}}{\pgfqpoint{2.151050in}{1.745968in}}{\pgfqpoint{2.151050in}{1.737732in}}%
\pgfpathcurveto{\pgfqpoint{2.151050in}{1.729496in}}{\pgfqpoint{2.154322in}{1.721596in}}{\pgfqpoint{2.160146in}{1.715772in}}%
\pgfpathcurveto{\pgfqpoint{2.165970in}{1.709948in}}{\pgfqpoint{2.173870in}{1.706675in}}{\pgfqpoint{2.182106in}{1.706675in}}%
\pgfpathclose%
\pgfusepath{stroke,fill}%
\end{pgfscope}%
\begin{pgfscope}%
\pgfpathrectangle{\pgfqpoint{0.100000in}{0.212622in}}{\pgfqpoint{3.696000in}{3.696000in}}%
\pgfusepath{clip}%
\pgfsetbuttcap%
\pgfsetroundjoin%
\definecolor{currentfill}{rgb}{0.121569,0.466667,0.705882}%
\pgfsetfillcolor{currentfill}%
\pgfsetfillopacity{0.710310}%
\pgfsetlinewidth{1.003750pt}%
\definecolor{currentstroke}{rgb}{0.121569,0.466667,0.705882}%
\pgfsetstrokecolor{currentstroke}%
\pgfsetstrokeopacity{0.710310}%
\pgfsetdash{}{0pt}%
\pgfpathmoveto{\pgfqpoint{2.182912in}{1.705126in}}%
\pgfpathcurveto{\pgfqpoint{2.191148in}{1.705126in}}{\pgfqpoint{2.199048in}{1.708398in}}{\pgfqpoint{2.204872in}{1.714222in}}%
\pgfpathcurveto{\pgfqpoint{2.210696in}{1.720046in}}{\pgfqpoint{2.213968in}{1.727946in}}{\pgfqpoint{2.213968in}{1.736183in}}%
\pgfpathcurveto{\pgfqpoint{2.213968in}{1.744419in}}{\pgfqpoint{2.210696in}{1.752319in}}{\pgfqpoint{2.204872in}{1.758143in}}%
\pgfpathcurveto{\pgfqpoint{2.199048in}{1.763967in}}{\pgfqpoint{2.191148in}{1.767239in}}{\pgfqpoint{2.182912in}{1.767239in}}%
\pgfpathcurveto{\pgfqpoint{2.174676in}{1.767239in}}{\pgfqpoint{2.166776in}{1.763967in}}{\pgfqpoint{2.160952in}{1.758143in}}%
\pgfpathcurveto{\pgfqpoint{2.155128in}{1.752319in}}{\pgfqpoint{2.151855in}{1.744419in}}{\pgfqpoint{2.151855in}{1.736183in}}%
\pgfpathcurveto{\pgfqpoint{2.151855in}{1.727946in}}{\pgfqpoint{2.155128in}{1.720046in}}{\pgfqpoint{2.160952in}{1.714222in}}%
\pgfpathcurveto{\pgfqpoint{2.166776in}{1.708398in}}{\pgfqpoint{2.174676in}{1.705126in}}{\pgfqpoint{2.182912in}{1.705126in}}%
\pgfpathclose%
\pgfusepath{stroke,fill}%
\end{pgfscope}%
\begin{pgfscope}%
\pgfpathrectangle{\pgfqpoint{0.100000in}{0.212622in}}{\pgfqpoint{3.696000in}{3.696000in}}%
\pgfusepath{clip}%
\pgfsetbuttcap%
\pgfsetroundjoin%
\definecolor{currentfill}{rgb}{0.121569,0.466667,0.705882}%
\pgfsetfillcolor{currentfill}%
\pgfsetfillopacity{0.710777}%
\pgfsetlinewidth{1.003750pt}%
\definecolor{currentstroke}{rgb}{0.121569,0.466667,0.705882}%
\pgfsetstrokecolor{currentstroke}%
\pgfsetstrokeopacity{0.710777}%
\pgfsetdash{}{0pt}%
\pgfpathmoveto{\pgfqpoint{2.183241in}{1.704065in}}%
\pgfpathcurveto{\pgfqpoint{2.191477in}{1.704065in}}{\pgfqpoint{2.199377in}{1.707337in}}{\pgfqpoint{2.205201in}{1.713161in}}%
\pgfpathcurveto{\pgfqpoint{2.211025in}{1.718985in}}{\pgfqpoint{2.214297in}{1.726885in}}{\pgfqpoint{2.214297in}{1.735121in}}%
\pgfpathcurveto{\pgfqpoint{2.214297in}{1.743357in}}{\pgfqpoint{2.211025in}{1.751258in}}{\pgfqpoint{2.205201in}{1.757081in}}%
\pgfpathcurveto{\pgfqpoint{2.199377in}{1.762905in}}{\pgfqpoint{2.191477in}{1.766178in}}{\pgfqpoint{2.183241in}{1.766178in}}%
\pgfpathcurveto{\pgfqpoint{2.175004in}{1.766178in}}{\pgfqpoint{2.167104in}{1.762905in}}{\pgfqpoint{2.161280in}{1.757081in}}%
\pgfpathcurveto{\pgfqpoint{2.155457in}{1.751258in}}{\pgfqpoint{2.152184in}{1.743357in}}{\pgfqpoint{2.152184in}{1.735121in}}%
\pgfpathcurveto{\pgfqpoint{2.152184in}{1.726885in}}{\pgfqpoint{2.155457in}{1.718985in}}{\pgfqpoint{2.161280in}{1.713161in}}%
\pgfpathcurveto{\pgfqpoint{2.167104in}{1.707337in}}{\pgfqpoint{2.175004in}{1.704065in}}{\pgfqpoint{2.183241in}{1.704065in}}%
\pgfpathclose%
\pgfusepath{stroke,fill}%
\end{pgfscope}%
\begin{pgfscope}%
\pgfpathrectangle{\pgfqpoint{0.100000in}{0.212622in}}{\pgfqpoint{3.696000in}{3.696000in}}%
\pgfusepath{clip}%
\pgfsetbuttcap%
\pgfsetroundjoin%
\definecolor{currentfill}{rgb}{0.121569,0.466667,0.705882}%
\pgfsetfillcolor{currentfill}%
\pgfsetfillopacity{0.711462}%
\pgfsetlinewidth{1.003750pt}%
\definecolor{currentstroke}{rgb}{0.121569,0.466667,0.705882}%
\pgfsetstrokecolor{currentstroke}%
\pgfsetstrokeopacity{0.711462}%
\pgfsetdash{}{0pt}%
\pgfpathmoveto{\pgfqpoint{2.183830in}{1.702908in}}%
\pgfpathcurveto{\pgfqpoint{2.192066in}{1.702908in}}{\pgfqpoint{2.199966in}{1.706180in}}{\pgfqpoint{2.205790in}{1.712004in}}%
\pgfpathcurveto{\pgfqpoint{2.211614in}{1.717828in}}{\pgfqpoint{2.214886in}{1.725728in}}{\pgfqpoint{2.214886in}{1.733964in}}%
\pgfpathcurveto{\pgfqpoint{2.214886in}{1.742201in}}{\pgfqpoint{2.211614in}{1.750101in}}{\pgfqpoint{2.205790in}{1.755925in}}%
\pgfpathcurveto{\pgfqpoint{2.199966in}{1.761749in}}{\pgfqpoint{2.192066in}{1.765021in}}{\pgfqpoint{2.183830in}{1.765021in}}%
\pgfpathcurveto{\pgfqpoint{2.175594in}{1.765021in}}{\pgfqpoint{2.167694in}{1.761749in}}{\pgfqpoint{2.161870in}{1.755925in}}%
\pgfpathcurveto{\pgfqpoint{2.156046in}{1.750101in}}{\pgfqpoint{2.152773in}{1.742201in}}{\pgfqpoint{2.152773in}{1.733964in}}%
\pgfpathcurveto{\pgfqpoint{2.152773in}{1.725728in}}{\pgfqpoint{2.156046in}{1.717828in}}{\pgfqpoint{2.161870in}{1.712004in}}%
\pgfpathcurveto{\pgfqpoint{2.167694in}{1.706180in}}{\pgfqpoint{2.175594in}{1.702908in}}{\pgfqpoint{2.183830in}{1.702908in}}%
\pgfpathclose%
\pgfusepath{stroke,fill}%
\end{pgfscope}%
\begin{pgfscope}%
\pgfpathrectangle{\pgfqpoint{0.100000in}{0.212622in}}{\pgfqpoint{3.696000in}{3.696000in}}%
\pgfusepath{clip}%
\pgfsetbuttcap%
\pgfsetroundjoin%
\definecolor{currentfill}{rgb}{0.121569,0.466667,0.705882}%
\pgfsetfillcolor{currentfill}%
\pgfsetfillopacity{0.711926}%
\pgfsetlinewidth{1.003750pt}%
\definecolor{currentstroke}{rgb}{0.121569,0.466667,0.705882}%
\pgfsetstrokecolor{currentstroke}%
\pgfsetstrokeopacity{0.711926}%
\pgfsetdash{}{0pt}%
\pgfpathmoveto{\pgfqpoint{2.184033in}{1.702768in}}%
\pgfpathcurveto{\pgfqpoint{2.192270in}{1.702768in}}{\pgfqpoint{2.200170in}{1.706040in}}{\pgfqpoint{2.205994in}{1.711864in}}%
\pgfpathcurveto{\pgfqpoint{2.211818in}{1.717688in}}{\pgfqpoint{2.215090in}{1.725588in}}{\pgfqpoint{2.215090in}{1.733824in}}%
\pgfpathcurveto{\pgfqpoint{2.215090in}{1.742060in}}{\pgfqpoint{2.211818in}{1.749960in}}{\pgfqpoint{2.205994in}{1.755784in}}%
\pgfpathcurveto{\pgfqpoint{2.200170in}{1.761608in}}{\pgfqpoint{2.192270in}{1.764881in}}{\pgfqpoint{2.184033in}{1.764881in}}%
\pgfpathcurveto{\pgfqpoint{2.175797in}{1.764881in}}{\pgfqpoint{2.167897in}{1.761608in}}{\pgfqpoint{2.162073in}{1.755784in}}%
\pgfpathcurveto{\pgfqpoint{2.156249in}{1.749960in}}{\pgfqpoint{2.152977in}{1.742060in}}{\pgfqpoint{2.152977in}{1.733824in}}%
\pgfpathcurveto{\pgfqpoint{2.152977in}{1.725588in}}{\pgfqpoint{2.156249in}{1.717688in}}{\pgfqpoint{2.162073in}{1.711864in}}%
\pgfpathcurveto{\pgfqpoint{2.167897in}{1.706040in}}{\pgfqpoint{2.175797in}{1.702768in}}{\pgfqpoint{2.184033in}{1.702768in}}%
\pgfpathclose%
\pgfusepath{stroke,fill}%
\end{pgfscope}%
\begin{pgfscope}%
\pgfpathrectangle{\pgfqpoint{0.100000in}{0.212622in}}{\pgfqpoint{3.696000in}{3.696000in}}%
\pgfusepath{clip}%
\pgfsetbuttcap%
\pgfsetroundjoin%
\definecolor{currentfill}{rgb}{0.121569,0.466667,0.705882}%
\pgfsetfillcolor{currentfill}%
\pgfsetfillopacity{0.712640}%
\pgfsetlinewidth{1.003750pt}%
\definecolor{currentstroke}{rgb}{0.121569,0.466667,0.705882}%
\pgfsetstrokecolor{currentstroke}%
\pgfsetstrokeopacity{0.712640}%
\pgfsetdash{}{0pt}%
\pgfpathmoveto{\pgfqpoint{2.184686in}{1.701169in}}%
\pgfpathcurveto{\pgfqpoint{2.192922in}{1.701169in}}{\pgfqpoint{2.200822in}{1.704441in}}{\pgfqpoint{2.206646in}{1.710265in}}%
\pgfpathcurveto{\pgfqpoint{2.212470in}{1.716089in}}{\pgfqpoint{2.215743in}{1.723989in}}{\pgfqpoint{2.215743in}{1.732225in}}%
\pgfpathcurveto{\pgfqpoint{2.215743in}{1.740461in}}{\pgfqpoint{2.212470in}{1.748361in}}{\pgfqpoint{2.206646in}{1.754185in}}%
\pgfpathcurveto{\pgfqpoint{2.200822in}{1.760009in}}{\pgfqpoint{2.192922in}{1.763282in}}{\pgfqpoint{2.184686in}{1.763282in}}%
\pgfpathcurveto{\pgfqpoint{2.176450in}{1.763282in}}{\pgfqpoint{2.168550in}{1.760009in}}{\pgfqpoint{2.162726in}{1.754185in}}%
\pgfpathcurveto{\pgfqpoint{2.156902in}{1.748361in}}{\pgfqpoint{2.153630in}{1.740461in}}{\pgfqpoint{2.153630in}{1.732225in}}%
\pgfpathcurveto{\pgfqpoint{2.153630in}{1.723989in}}{\pgfqpoint{2.156902in}{1.716089in}}{\pgfqpoint{2.162726in}{1.710265in}}%
\pgfpathcurveto{\pgfqpoint{2.168550in}{1.704441in}}{\pgfqpoint{2.176450in}{1.701169in}}{\pgfqpoint{2.184686in}{1.701169in}}%
\pgfpathclose%
\pgfusepath{stroke,fill}%
\end{pgfscope}%
\begin{pgfscope}%
\pgfpathrectangle{\pgfqpoint{0.100000in}{0.212622in}}{\pgfqpoint{3.696000in}{3.696000in}}%
\pgfusepath{clip}%
\pgfsetbuttcap%
\pgfsetroundjoin%
\definecolor{currentfill}{rgb}{0.121569,0.466667,0.705882}%
\pgfsetfillcolor{currentfill}%
\pgfsetfillopacity{0.713010}%
\pgfsetlinewidth{1.003750pt}%
\definecolor{currentstroke}{rgb}{0.121569,0.466667,0.705882}%
\pgfsetstrokecolor{currentstroke}%
\pgfsetstrokeopacity{0.713010}%
\pgfsetdash{}{0pt}%
\pgfpathmoveto{\pgfqpoint{2.185121in}{1.700199in}}%
\pgfpathcurveto{\pgfqpoint{2.193357in}{1.700199in}}{\pgfqpoint{2.201257in}{1.703471in}}{\pgfqpoint{2.207081in}{1.709295in}}%
\pgfpathcurveto{\pgfqpoint{2.212905in}{1.715119in}}{\pgfqpoint{2.216177in}{1.723019in}}{\pgfqpoint{2.216177in}{1.731255in}}%
\pgfpathcurveto{\pgfqpoint{2.216177in}{1.739492in}}{\pgfqpoint{2.212905in}{1.747392in}}{\pgfqpoint{2.207081in}{1.753216in}}%
\pgfpathcurveto{\pgfqpoint{2.201257in}{1.759040in}}{\pgfqpoint{2.193357in}{1.762312in}}{\pgfqpoint{2.185121in}{1.762312in}}%
\pgfpathcurveto{\pgfqpoint{2.176884in}{1.762312in}}{\pgfqpoint{2.168984in}{1.759040in}}{\pgfqpoint{2.163161in}{1.753216in}}%
\pgfpathcurveto{\pgfqpoint{2.157337in}{1.747392in}}{\pgfqpoint{2.154064in}{1.739492in}}{\pgfqpoint{2.154064in}{1.731255in}}%
\pgfpathcurveto{\pgfqpoint{2.154064in}{1.723019in}}{\pgfqpoint{2.157337in}{1.715119in}}{\pgfqpoint{2.163161in}{1.709295in}}%
\pgfpathcurveto{\pgfqpoint{2.168984in}{1.703471in}}{\pgfqpoint{2.176884in}{1.700199in}}{\pgfqpoint{2.185121in}{1.700199in}}%
\pgfpathclose%
\pgfusepath{stroke,fill}%
\end{pgfscope}%
\begin{pgfscope}%
\pgfpathrectangle{\pgfqpoint{0.100000in}{0.212622in}}{\pgfqpoint{3.696000in}{3.696000in}}%
\pgfusepath{clip}%
\pgfsetbuttcap%
\pgfsetroundjoin%
\definecolor{currentfill}{rgb}{0.121569,0.466667,0.705882}%
\pgfsetfillcolor{currentfill}%
\pgfsetfillopacity{0.714022}%
\pgfsetlinewidth{1.003750pt}%
\definecolor{currentstroke}{rgb}{0.121569,0.466667,0.705882}%
\pgfsetstrokecolor{currentstroke}%
\pgfsetstrokeopacity{0.714022}%
\pgfsetdash{}{0pt}%
\pgfpathmoveto{\pgfqpoint{2.185731in}{1.699251in}}%
\pgfpathcurveto{\pgfqpoint{2.193968in}{1.699251in}}{\pgfqpoint{2.201868in}{1.702523in}}{\pgfqpoint{2.207692in}{1.708347in}}%
\pgfpathcurveto{\pgfqpoint{2.213516in}{1.714171in}}{\pgfqpoint{2.216788in}{1.722071in}}{\pgfqpoint{2.216788in}{1.730307in}}%
\pgfpathcurveto{\pgfqpoint{2.216788in}{1.738543in}}{\pgfqpoint{2.213516in}{1.746444in}}{\pgfqpoint{2.207692in}{1.752267in}}%
\pgfpathcurveto{\pgfqpoint{2.201868in}{1.758091in}}{\pgfqpoint{2.193968in}{1.761364in}}{\pgfqpoint{2.185731in}{1.761364in}}%
\pgfpathcurveto{\pgfqpoint{2.177495in}{1.761364in}}{\pgfqpoint{2.169595in}{1.758091in}}{\pgfqpoint{2.163771in}{1.752267in}}%
\pgfpathcurveto{\pgfqpoint{2.157947in}{1.746444in}}{\pgfqpoint{2.154675in}{1.738543in}}{\pgfqpoint{2.154675in}{1.730307in}}%
\pgfpathcurveto{\pgfqpoint{2.154675in}{1.722071in}}{\pgfqpoint{2.157947in}{1.714171in}}{\pgfqpoint{2.163771in}{1.708347in}}%
\pgfpathcurveto{\pgfqpoint{2.169595in}{1.702523in}}{\pgfqpoint{2.177495in}{1.699251in}}{\pgfqpoint{2.185731in}{1.699251in}}%
\pgfpathclose%
\pgfusepath{stroke,fill}%
\end{pgfscope}%
\begin{pgfscope}%
\pgfpathrectangle{\pgfqpoint{0.100000in}{0.212622in}}{\pgfqpoint{3.696000in}{3.696000in}}%
\pgfusepath{clip}%
\pgfsetbuttcap%
\pgfsetroundjoin%
\definecolor{currentfill}{rgb}{0.121569,0.466667,0.705882}%
\pgfsetfillcolor{currentfill}%
\pgfsetfillopacity{0.715340}%
\pgfsetlinewidth{1.003750pt}%
\definecolor{currentstroke}{rgb}{0.121569,0.466667,0.705882}%
\pgfsetstrokecolor{currentstroke}%
\pgfsetstrokeopacity{0.715340}%
\pgfsetdash{}{0pt}%
\pgfpathmoveto{\pgfqpoint{2.186478in}{1.698988in}}%
\pgfpathcurveto{\pgfqpoint{2.194714in}{1.698988in}}{\pgfqpoint{2.202614in}{1.702261in}}{\pgfqpoint{2.208438in}{1.708085in}}%
\pgfpathcurveto{\pgfqpoint{2.214262in}{1.713909in}}{\pgfqpoint{2.217535in}{1.721809in}}{\pgfqpoint{2.217535in}{1.730045in}}%
\pgfpathcurveto{\pgfqpoint{2.217535in}{1.738281in}}{\pgfqpoint{2.214262in}{1.746181in}}{\pgfqpoint{2.208438in}{1.752005in}}%
\pgfpathcurveto{\pgfqpoint{2.202614in}{1.757829in}}{\pgfqpoint{2.194714in}{1.761101in}}{\pgfqpoint{2.186478in}{1.761101in}}%
\pgfpathcurveto{\pgfqpoint{2.178242in}{1.761101in}}{\pgfqpoint{2.170342in}{1.757829in}}{\pgfqpoint{2.164518in}{1.752005in}}%
\pgfpathcurveto{\pgfqpoint{2.158694in}{1.746181in}}{\pgfqpoint{2.155422in}{1.738281in}}{\pgfqpoint{2.155422in}{1.730045in}}%
\pgfpathcurveto{\pgfqpoint{2.155422in}{1.721809in}}{\pgfqpoint{2.158694in}{1.713909in}}{\pgfqpoint{2.164518in}{1.708085in}}%
\pgfpathcurveto{\pgfqpoint{2.170342in}{1.702261in}}{\pgfqpoint{2.178242in}{1.698988in}}{\pgfqpoint{2.186478in}{1.698988in}}%
\pgfpathclose%
\pgfusepath{stroke,fill}%
\end{pgfscope}%
\begin{pgfscope}%
\pgfpathrectangle{\pgfqpoint{0.100000in}{0.212622in}}{\pgfqpoint{3.696000in}{3.696000in}}%
\pgfusepath{clip}%
\pgfsetbuttcap%
\pgfsetroundjoin%
\definecolor{currentfill}{rgb}{0.121569,0.466667,0.705882}%
\pgfsetfillcolor{currentfill}%
\pgfsetfillopacity{0.717189}%
\pgfsetlinewidth{1.003750pt}%
\definecolor{currentstroke}{rgb}{0.121569,0.466667,0.705882}%
\pgfsetstrokecolor{currentstroke}%
\pgfsetstrokeopacity{0.717189}%
\pgfsetdash{}{0pt}%
\pgfpathmoveto{\pgfqpoint{2.188410in}{1.696014in}}%
\pgfpathcurveto{\pgfqpoint{2.196647in}{1.696014in}}{\pgfqpoint{2.204547in}{1.699287in}}{\pgfqpoint{2.210371in}{1.705111in}}%
\pgfpathcurveto{\pgfqpoint{2.216195in}{1.710935in}}{\pgfqpoint{2.219467in}{1.718835in}}{\pgfqpoint{2.219467in}{1.727071in}}%
\pgfpathcurveto{\pgfqpoint{2.219467in}{1.735307in}}{\pgfqpoint{2.216195in}{1.743207in}}{\pgfqpoint{2.210371in}{1.749031in}}%
\pgfpathcurveto{\pgfqpoint{2.204547in}{1.754855in}}{\pgfqpoint{2.196647in}{1.758127in}}{\pgfqpoint{2.188410in}{1.758127in}}%
\pgfpathcurveto{\pgfqpoint{2.180174in}{1.758127in}}{\pgfqpoint{2.172274in}{1.754855in}}{\pgfqpoint{2.166450in}{1.749031in}}%
\pgfpathcurveto{\pgfqpoint{2.160626in}{1.743207in}}{\pgfqpoint{2.157354in}{1.735307in}}{\pgfqpoint{2.157354in}{1.727071in}}%
\pgfpathcurveto{\pgfqpoint{2.157354in}{1.718835in}}{\pgfqpoint{2.160626in}{1.710935in}}{\pgfqpoint{2.166450in}{1.705111in}}%
\pgfpathcurveto{\pgfqpoint{2.172274in}{1.699287in}}{\pgfqpoint{2.180174in}{1.696014in}}{\pgfqpoint{2.188410in}{1.696014in}}%
\pgfpathclose%
\pgfusepath{stroke,fill}%
\end{pgfscope}%
\begin{pgfscope}%
\pgfpathrectangle{\pgfqpoint{0.100000in}{0.212622in}}{\pgfqpoint{3.696000in}{3.696000in}}%
\pgfusepath{clip}%
\pgfsetbuttcap%
\pgfsetroundjoin%
\definecolor{currentfill}{rgb}{0.121569,0.466667,0.705882}%
\pgfsetfillcolor{currentfill}%
\pgfsetfillopacity{0.718210}%
\pgfsetlinewidth{1.003750pt}%
\definecolor{currentstroke}{rgb}{0.121569,0.466667,0.705882}%
\pgfsetstrokecolor{currentstroke}%
\pgfsetstrokeopacity{0.718210}%
\pgfsetdash{}{0pt}%
\pgfpathmoveto{\pgfqpoint{2.189103in}{1.694176in}}%
\pgfpathcurveto{\pgfqpoint{2.197340in}{1.694176in}}{\pgfqpoint{2.205240in}{1.697449in}}{\pgfqpoint{2.211063in}{1.703273in}}%
\pgfpathcurveto{\pgfqpoint{2.216887in}{1.709096in}}{\pgfqpoint{2.220160in}{1.716997in}}{\pgfqpoint{2.220160in}{1.725233in}}%
\pgfpathcurveto{\pgfqpoint{2.220160in}{1.733469in}}{\pgfqpoint{2.216887in}{1.741369in}}{\pgfqpoint{2.211063in}{1.747193in}}%
\pgfpathcurveto{\pgfqpoint{2.205240in}{1.753017in}}{\pgfqpoint{2.197340in}{1.756289in}}{\pgfqpoint{2.189103in}{1.756289in}}%
\pgfpathcurveto{\pgfqpoint{2.180867in}{1.756289in}}{\pgfqpoint{2.172967in}{1.753017in}}{\pgfqpoint{2.167143in}{1.747193in}}%
\pgfpathcurveto{\pgfqpoint{2.161319in}{1.741369in}}{\pgfqpoint{2.158047in}{1.733469in}}{\pgfqpoint{2.158047in}{1.725233in}}%
\pgfpathcurveto{\pgfqpoint{2.158047in}{1.716997in}}{\pgfqpoint{2.161319in}{1.709096in}}{\pgfqpoint{2.167143in}{1.703273in}}%
\pgfpathcurveto{\pgfqpoint{2.172967in}{1.697449in}}{\pgfqpoint{2.180867in}{1.694176in}}{\pgfqpoint{2.189103in}{1.694176in}}%
\pgfpathclose%
\pgfusepath{stroke,fill}%
\end{pgfscope}%
\begin{pgfscope}%
\pgfpathrectangle{\pgfqpoint{0.100000in}{0.212622in}}{\pgfqpoint{3.696000in}{3.696000in}}%
\pgfusepath{clip}%
\pgfsetbuttcap%
\pgfsetroundjoin%
\definecolor{currentfill}{rgb}{0.121569,0.466667,0.705882}%
\pgfsetfillcolor{currentfill}%
\pgfsetfillopacity{0.720426}%
\pgfsetlinewidth{1.003750pt}%
\definecolor{currentstroke}{rgb}{0.121569,0.466667,0.705882}%
\pgfsetstrokecolor{currentstroke}%
\pgfsetstrokeopacity{0.720426}%
\pgfsetdash{}{0pt}%
\pgfpathmoveto{\pgfqpoint{2.190291in}{1.693728in}}%
\pgfpathcurveto{\pgfqpoint{2.198528in}{1.693728in}}{\pgfqpoint{2.206428in}{1.697000in}}{\pgfqpoint{2.212252in}{1.702824in}}%
\pgfpathcurveto{\pgfqpoint{2.218076in}{1.708648in}}{\pgfqpoint{2.221348in}{1.716548in}}{\pgfqpoint{2.221348in}{1.724785in}}%
\pgfpathcurveto{\pgfqpoint{2.221348in}{1.733021in}}{\pgfqpoint{2.218076in}{1.740921in}}{\pgfqpoint{2.212252in}{1.746745in}}%
\pgfpathcurveto{\pgfqpoint{2.206428in}{1.752569in}}{\pgfqpoint{2.198528in}{1.755841in}}{\pgfqpoint{2.190291in}{1.755841in}}%
\pgfpathcurveto{\pgfqpoint{2.182055in}{1.755841in}}{\pgfqpoint{2.174155in}{1.752569in}}{\pgfqpoint{2.168331in}{1.746745in}}%
\pgfpathcurveto{\pgfqpoint{2.162507in}{1.740921in}}{\pgfqpoint{2.159235in}{1.733021in}}{\pgfqpoint{2.159235in}{1.724785in}}%
\pgfpathcurveto{\pgfqpoint{2.159235in}{1.716548in}}{\pgfqpoint{2.162507in}{1.708648in}}{\pgfqpoint{2.168331in}{1.702824in}}%
\pgfpathcurveto{\pgfqpoint{2.174155in}{1.697000in}}{\pgfqpoint{2.182055in}{1.693728in}}{\pgfqpoint{2.190291in}{1.693728in}}%
\pgfpathclose%
\pgfusepath{stroke,fill}%
\end{pgfscope}%
\begin{pgfscope}%
\pgfpathrectangle{\pgfqpoint{0.100000in}{0.212622in}}{\pgfqpoint{3.696000in}{3.696000in}}%
\pgfusepath{clip}%
\pgfsetbuttcap%
\pgfsetroundjoin%
\definecolor{currentfill}{rgb}{0.121569,0.466667,0.705882}%
\pgfsetfillcolor{currentfill}%
\pgfsetfillopacity{0.722916}%
\pgfsetlinewidth{1.003750pt}%
\definecolor{currentstroke}{rgb}{0.121569,0.466667,0.705882}%
\pgfsetstrokecolor{currentstroke}%
\pgfsetstrokeopacity{0.722916}%
\pgfsetdash{}{0pt}%
\pgfpathmoveto{\pgfqpoint{2.192096in}{1.692286in}}%
\pgfpathcurveto{\pgfqpoint{2.200332in}{1.692286in}}{\pgfqpoint{2.208232in}{1.695558in}}{\pgfqpoint{2.214056in}{1.701382in}}%
\pgfpathcurveto{\pgfqpoint{2.219880in}{1.707206in}}{\pgfqpoint{2.223152in}{1.715106in}}{\pgfqpoint{2.223152in}{1.723343in}}%
\pgfpathcurveto{\pgfqpoint{2.223152in}{1.731579in}}{\pgfqpoint{2.219880in}{1.739479in}}{\pgfqpoint{2.214056in}{1.745303in}}%
\pgfpathcurveto{\pgfqpoint{2.208232in}{1.751127in}}{\pgfqpoint{2.200332in}{1.754399in}}{\pgfqpoint{2.192096in}{1.754399in}}%
\pgfpathcurveto{\pgfqpoint{2.183860in}{1.754399in}}{\pgfqpoint{2.175959in}{1.751127in}}{\pgfqpoint{2.170136in}{1.745303in}}%
\pgfpathcurveto{\pgfqpoint{2.164312in}{1.739479in}}{\pgfqpoint{2.161039in}{1.731579in}}{\pgfqpoint{2.161039in}{1.723343in}}%
\pgfpathcurveto{\pgfqpoint{2.161039in}{1.715106in}}{\pgfqpoint{2.164312in}{1.707206in}}{\pgfqpoint{2.170136in}{1.701382in}}%
\pgfpathcurveto{\pgfqpoint{2.175959in}{1.695558in}}{\pgfqpoint{2.183860in}{1.692286in}}{\pgfqpoint{2.192096in}{1.692286in}}%
\pgfpathclose%
\pgfusepath{stroke,fill}%
\end{pgfscope}%
\begin{pgfscope}%
\pgfpathrectangle{\pgfqpoint{0.100000in}{0.212622in}}{\pgfqpoint{3.696000in}{3.696000in}}%
\pgfusepath{clip}%
\pgfsetbuttcap%
\pgfsetroundjoin%
\definecolor{currentfill}{rgb}{0.121569,0.466667,0.705882}%
\pgfsetfillcolor{currentfill}%
\pgfsetfillopacity{0.725357}%
\pgfsetlinewidth{1.003750pt}%
\definecolor{currentstroke}{rgb}{0.121569,0.466667,0.705882}%
\pgfsetstrokecolor{currentstroke}%
\pgfsetstrokeopacity{0.725357}%
\pgfsetdash{}{0pt}%
\pgfpathmoveto{\pgfqpoint{2.194646in}{1.686944in}}%
\pgfpathcurveto{\pgfqpoint{2.202882in}{1.686944in}}{\pgfqpoint{2.210782in}{1.690217in}}{\pgfqpoint{2.216606in}{1.696040in}}%
\pgfpathcurveto{\pgfqpoint{2.222430in}{1.701864in}}{\pgfqpoint{2.225702in}{1.709764in}}{\pgfqpoint{2.225702in}{1.718001in}}%
\pgfpathcurveto{\pgfqpoint{2.225702in}{1.726237in}}{\pgfqpoint{2.222430in}{1.734137in}}{\pgfqpoint{2.216606in}{1.739961in}}%
\pgfpathcurveto{\pgfqpoint{2.210782in}{1.745785in}}{\pgfqpoint{2.202882in}{1.749057in}}{\pgfqpoint{2.194646in}{1.749057in}}%
\pgfpathcurveto{\pgfqpoint{2.186410in}{1.749057in}}{\pgfqpoint{2.178510in}{1.745785in}}{\pgfqpoint{2.172686in}{1.739961in}}%
\pgfpathcurveto{\pgfqpoint{2.166862in}{1.734137in}}{\pgfqpoint{2.163589in}{1.726237in}}{\pgfqpoint{2.163589in}{1.718001in}}%
\pgfpathcurveto{\pgfqpoint{2.163589in}{1.709764in}}{\pgfqpoint{2.166862in}{1.701864in}}{\pgfqpoint{2.172686in}{1.696040in}}%
\pgfpathcurveto{\pgfqpoint{2.178510in}{1.690217in}}{\pgfqpoint{2.186410in}{1.686944in}}{\pgfqpoint{2.194646in}{1.686944in}}%
\pgfpathclose%
\pgfusepath{stroke,fill}%
\end{pgfscope}%
\begin{pgfscope}%
\pgfpathrectangle{\pgfqpoint{0.100000in}{0.212622in}}{\pgfqpoint{3.696000in}{3.696000in}}%
\pgfusepath{clip}%
\pgfsetbuttcap%
\pgfsetroundjoin%
\definecolor{currentfill}{rgb}{0.121569,0.466667,0.705882}%
\pgfsetfillcolor{currentfill}%
\pgfsetfillopacity{0.727993}%
\pgfsetlinewidth{1.003750pt}%
\definecolor{currentstroke}{rgb}{0.121569,0.466667,0.705882}%
\pgfsetstrokecolor{currentstroke}%
\pgfsetstrokeopacity{0.727993}%
\pgfsetdash{}{0pt}%
\pgfpathmoveto{\pgfqpoint{2.196628in}{1.681333in}}%
\pgfpathcurveto{\pgfqpoint{2.204864in}{1.681333in}}{\pgfqpoint{2.212764in}{1.684606in}}{\pgfqpoint{2.218588in}{1.690429in}}%
\pgfpathcurveto{\pgfqpoint{2.224412in}{1.696253in}}{\pgfqpoint{2.227684in}{1.704153in}}{\pgfqpoint{2.227684in}{1.712390in}}%
\pgfpathcurveto{\pgfqpoint{2.227684in}{1.720626in}}{\pgfqpoint{2.224412in}{1.728526in}}{\pgfqpoint{2.218588in}{1.734350in}}%
\pgfpathcurveto{\pgfqpoint{2.212764in}{1.740174in}}{\pgfqpoint{2.204864in}{1.743446in}}{\pgfqpoint{2.196628in}{1.743446in}}%
\pgfpathcurveto{\pgfqpoint{2.188392in}{1.743446in}}{\pgfqpoint{2.180492in}{1.740174in}}{\pgfqpoint{2.174668in}{1.734350in}}%
\pgfpathcurveto{\pgfqpoint{2.168844in}{1.728526in}}{\pgfqpoint{2.165571in}{1.720626in}}{\pgfqpoint{2.165571in}{1.712390in}}%
\pgfpathcurveto{\pgfqpoint{2.165571in}{1.704153in}}{\pgfqpoint{2.168844in}{1.696253in}}{\pgfqpoint{2.174668in}{1.690429in}}%
\pgfpathcurveto{\pgfqpoint{2.180492in}{1.684606in}}{\pgfqpoint{2.188392in}{1.681333in}}{\pgfqpoint{2.196628in}{1.681333in}}%
\pgfpathclose%
\pgfusepath{stroke,fill}%
\end{pgfscope}%
\begin{pgfscope}%
\pgfpathrectangle{\pgfqpoint{0.100000in}{0.212622in}}{\pgfqpoint{3.696000in}{3.696000in}}%
\pgfusepath{clip}%
\pgfsetbuttcap%
\pgfsetroundjoin%
\definecolor{currentfill}{rgb}{0.121569,0.466667,0.705882}%
\pgfsetfillcolor{currentfill}%
\pgfsetfillopacity{0.729912}%
\pgfsetlinewidth{1.003750pt}%
\definecolor{currentstroke}{rgb}{0.121569,0.466667,0.705882}%
\pgfsetstrokecolor{currentstroke}%
\pgfsetstrokeopacity{0.729912}%
\pgfsetdash{}{0pt}%
\pgfpathmoveto{\pgfqpoint{2.197704in}{1.681196in}}%
\pgfpathcurveto{\pgfqpoint{2.205940in}{1.681196in}}{\pgfqpoint{2.213840in}{1.684468in}}{\pgfqpoint{2.219664in}{1.690292in}}%
\pgfpathcurveto{\pgfqpoint{2.225488in}{1.696116in}}{\pgfqpoint{2.228760in}{1.704016in}}{\pgfqpoint{2.228760in}{1.712252in}}%
\pgfpathcurveto{\pgfqpoint{2.228760in}{1.720489in}}{\pgfqpoint{2.225488in}{1.728389in}}{\pgfqpoint{2.219664in}{1.734212in}}%
\pgfpathcurveto{\pgfqpoint{2.213840in}{1.740036in}}{\pgfqpoint{2.205940in}{1.743309in}}{\pgfqpoint{2.197704in}{1.743309in}}%
\pgfpathcurveto{\pgfqpoint{2.189468in}{1.743309in}}{\pgfqpoint{2.181567in}{1.740036in}}{\pgfqpoint{2.175744in}{1.734212in}}%
\pgfpathcurveto{\pgfqpoint{2.169920in}{1.728389in}}{\pgfqpoint{2.166647in}{1.720489in}}{\pgfqpoint{2.166647in}{1.712252in}}%
\pgfpathcurveto{\pgfqpoint{2.166647in}{1.704016in}}{\pgfqpoint{2.169920in}{1.696116in}}{\pgfqpoint{2.175744in}{1.690292in}}%
\pgfpathcurveto{\pgfqpoint{2.181567in}{1.684468in}}{\pgfqpoint{2.189468in}{1.681196in}}{\pgfqpoint{2.197704in}{1.681196in}}%
\pgfpathclose%
\pgfusepath{stroke,fill}%
\end{pgfscope}%
\begin{pgfscope}%
\pgfpathrectangle{\pgfqpoint{0.100000in}{0.212622in}}{\pgfqpoint{3.696000in}{3.696000in}}%
\pgfusepath{clip}%
\pgfsetbuttcap%
\pgfsetroundjoin%
\definecolor{currentfill}{rgb}{0.121569,0.466667,0.705882}%
\pgfsetfillcolor{currentfill}%
\pgfsetfillopacity{0.731887}%
\pgfsetlinewidth{1.003750pt}%
\definecolor{currentstroke}{rgb}{0.121569,0.466667,0.705882}%
\pgfsetstrokecolor{currentstroke}%
\pgfsetstrokeopacity{0.731887}%
\pgfsetdash{}{0pt}%
\pgfpathmoveto{\pgfqpoint{2.199035in}{1.679114in}}%
\pgfpathcurveto{\pgfqpoint{2.207271in}{1.679114in}}{\pgfqpoint{2.215171in}{1.682387in}}{\pgfqpoint{2.220995in}{1.688211in}}%
\pgfpathcurveto{\pgfqpoint{2.226819in}{1.694035in}}{\pgfqpoint{2.230091in}{1.701935in}}{\pgfqpoint{2.230091in}{1.710171in}}%
\pgfpathcurveto{\pgfqpoint{2.230091in}{1.718407in}}{\pgfqpoint{2.226819in}{1.726307in}}{\pgfqpoint{2.220995in}{1.732131in}}%
\pgfpathcurveto{\pgfqpoint{2.215171in}{1.737955in}}{\pgfqpoint{2.207271in}{1.741227in}}{\pgfqpoint{2.199035in}{1.741227in}}%
\pgfpathcurveto{\pgfqpoint{2.190799in}{1.741227in}}{\pgfqpoint{2.182898in}{1.737955in}}{\pgfqpoint{2.177075in}{1.732131in}}%
\pgfpathcurveto{\pgfqpoint{2.171251in}{1.726307in}}{\pgfqpoint{2.167978in}{1.718407in}}{\pgfqpoint{2.167978in}{1.710171in}}%
\pgfpathcurveto{\pgfqpoint{2.167978in}{1.701935in}}{\pgfqpoint{2.171251in}{1.694035in}}{\pgfqpoint{2.177075in}{1.688211in}}%
\pgfpathcurveto{\pgfqpoint{2.182898in}{1.682387in}}{\pgfqpoint{2.190799in}{1.679114in}}{\pgfqpoint{2.199035in}{1.679114in}}%
\pgfpathclose%
\pgfusepath{stroke,fill}%
\end{pgfscope}%
\begin{pgfscope}%
\pgfpathrectangle{\pgfqpoint{0.100000in}{0.212622in}}{\pgfqpoint{3.696000in}{3.696000in}}%
\pgfusepath{clip}%
\pgfsetbuttcap%
\pgfsetroundjoin%
\definecolor{currentfill}{rgb}{0.121569,0.466667,0.705882}%
\pgfsetfillcolor{currentfill}%
\pgfsetfillopacity{0.732821}%
\pgfsetlinewidth{1.003750pt}%
\definecolor{currentstroke}{rgb}{0.121569,0.466667,0.705882}%
\pgfsetstrokecolor{currentstroke}%
\pgfsetstrokeopacity{0.732821}%
\pgfsetdash{}{0pt}%
\pgfpathmoveto{\pgfqpoint{2.199756in}{1.676995in}}%
\pgfpathcurveto{\pgfqpoint{2.207992in}{1.676995in}}{\pgfqpoint{2.215892in}{1.680268in}}{\pgfqpoint{2.221716in}{1.686092in}}%
\pgfpathcurveto{\pgfqpoint{2.227540in}{1.691916in}}{\pgfqpoint{2.230812in}{1.699816in}}{\pgfqpoint{2.230812in}{1.708052in}}%
\pgfpathcurveto{\pgfqpoint{2.230812in}{1.716288in}}{\pgfqpoint{2.227540in}{1.724188in}}{\pgfqpoint{2.221716in}{1.730012in}}%
\pgfpathcurveto{\pgfqpoint{2.215892in}{1.735836in}}{\pgfqpoint{2.207992in}{1.739108in}}{\pgfqpoint{2.199756in}{1.739108in}}%
\pgfpathcurveto{\pgfqpoint{2.191520in}{1.739108in}}{\pgfqpoint{2.183620in}{1.735836in}}{\pgfqpoint{2.177796in}{1.730012in}}%
\pgfpathcurveto{\pgfqpoint{2.171972in}{1.724188in}}{\pgfqpoint{2.168699in}{1.716288in}}{\pgfqpoint{2.168699in}{1.708052in}}%
\pgfpathcurveto{\pgfqpoint{2.168699in}{1.699816in}}{\pgfqpoint{2.171972in}{1.691916in}}{\pgfqpoint{2.177796in}{1.686092in}}%
\pgfpathcurveto{\pgfqpoint{2.183620in}{1.680268in}}{\pgfqpoint{2.191520in}{1.676995in}}{\pgfqpoint{2.199756in}{1.676995in}}%
\pgfpathclose%
\pgfusepath{stroke,fill}%
\end{pgfscope}%
\begin{pgfscope}%
\pgfpathrectangle{\pgfqpoint{0.100000in}{0.212622in}}{\pgfqpoint{3.696000in}{3.696000in}}%
\pgfusepath{clip}%
\pgfsetbuttcap%
\pgfsetroundjoin%
\definecolor{currentfill}{rgb}{0.121569,0.466667,0.705882}%
\pgfsetfillcolor{currentfill}%
\pgfsetfillopacity{0.734182}%
\pgfsetlinewidth{1.003750pt}%
\definecolor{currentstroke}{rgb}{0.121569,0.466667,0.705882}%
\pgfsetstrokecolor{currentstroke}%
\pgfsetstrokeopacity{0.734182}%
\pgfsetdash{}{0pt}%
\pgfpathmoveto{\pgfqpoint{2.200619in}{1.675226in}}%
\pgfpathcurveto{\pgfqpoint{2.208856in}{1.675226in}}{\pgfqpoint{2.216756in}{1.678498in}}{\pgfqpoint{2.222580in}{1.684322in}}%
\pgfpathcurveto{\pgfqpoint{2.228404in}{1.690146in}}{\pgfqpoint{2.231676in}{1.698046in}}{\pgfqpoint{2.231676in}{1.706282in}}%
\pgfpathcurveto{\pgfqpoint{2.231676in}{1.714519in}}{\pgfqpoint{2.228404in}{1.722419in}}{\pgfqpoint{2.222580in}{1.728243in}}%
\pgfpathcurveto{\pgfqpoint{2.216756in}{1.734067in}}{\pgfqpoint{2.208856in}{1.737339in}}{\pgfqpoint{2.200619in}{1.737339in}}%
\pgfpathcurveto{\pgfqpoint{2.192383in}{1.737339in}}{\pgfqpoint{2.184483in}{1.734067in}}{\pgfqpoint{2.178659in}{1.728243in}}%
\pgfpathcurveto{\pgfqpoint{2.172835in}{1.722419in}}{\pgfqpoint{2.169563in}{1.714519in}}{\pgfqpoint{2.169563in}{1.706282in}}%
\pgfpathcurveto{\pgfqpoint{2.169563in}{1.698046in}}{\pgfqpoint{2.172835in}{1.690146in}}{\pgfqpoint{2.178659in}{1.684322in}}%
\pgfpathcurveto{\pgfqpoint{2.184483in}{1.678498in}}{\pgfqpoint{2.192383in}{1.675226in}}{\pgfqpoint{2.200619in}{1.675226in}}%
\pgfpathclose%
\pgfusepath{stroke,fill}%
\end{pgfscope}%
\begin{pgfscope}%
\pgfpathrectangle{\pgfqpoint{0.100000in}{0.212622in}}{\pgfqpoint{3.696000in}{3.696000in}}%
\pgfusepath{clip}%
\pgfsetbuttcap%
\pgfsetroundjoin%
\definecolor{currentfill}{rgb}{0.121569,0.466667,0.705882}%
\pgfsetfillcolor{currentfill}%
\pgfsetfillopacity{0.735077}%
\pgfsetlinewidth{1.003750pt}%
\definecolor{currentstroke}{rgb}{0.121569,0.466667,0.705882}%
\pgfsetstrokecolor{currentstroke}%
\pgfsetstrokeopacity{0.735077}%
\pgfsetdash{}{0pt}%
\pgfpathmoveto{\pgfqpoint{2.201292in}{1.675291in}}%
\pgfpathcurveto{\pgfqpoint{2.209528in}{1.675291in}}{\pgfqpoint{2.217428in}{1.678563in}}{\pgfqpoint{2.223252in}{1.684387in}}%
\pgfpathcurveto{\pgfqpoint{2.229076in}{1.690211in}}{\pgfqpoint{2.232348in}{1.698111in}}{\pgfqpoint{2.232348in}{1.706347in}}%
\pgfpathcurveto{\pgfqpoint{2.232348in}{1.714583in}}{\pgfqpoint{2.229076in}{1.722483in}}{\pgfqpoint{2.223252in}{1.728307in}}%
\pgfpathcurveto{\pgfqpoint{2.217428in}{1.734131in}}{\pgfqpoint{2.209528in}{1.737404in}}{\pgfqpoint{2.201292in}{1.737404in}}%
\pgfpathcurveto{\pgfqpoint{2.193056in}{1.737404in}}{\pgfqpoint{2.185156in}{1.734131in}}{\pgfqpoint{2.179332in}{1.728307in}}%
\pgfpathcurveto{\pgfqpoint{2.173508in}{1.722483in}}{\pgfqpoint{2.170235in}{1.714583in}}{\pgfqpoint{2.170235in}{1.706347in}}%
\pgfpathcurveto{\pgfqpoint{2.170235in}{1.698111in}}{\pgfqpoint{2.173508in}{1.690211in}}{\pgfqpoint{2.179332in}{1.684387in}}%
\pgfpathcurveto{\pgfqpoint{2.185156in}{1.678563in}}{\pgfqpoint{2.193056in}{1.675291in}}{\pgfqpoint{2.201292in}{1.675291in}}%
\pgfpathclose%
\pgfusepath{stroke,fill}%
\end{pgfscope}%
\begin{pgfscope}%
\pgfpathrectangle{\pgfqpoint{0.100000in}{0.212622in}}{\pgfqpoint{3.696000in}{3.696000in}}%
\pgfusepath{clip}%
\pgfsetbuttcap%
\pgfsetroundjoin%
\definecolor{currentfill}{rgb}{0.121569,0.466667,0.705882}%
\pgfsetfillcolor{currentfill}%
\pgfsetfillopacity{0.736201}%
\pgfsetlinewidth{1.003750pt}%
\definecolor{currentstroke}{rgb}{0.121569,0.466667,0.705882}%
\pgfsetstrokecolor{currentstroke}%
\pgfsetstrokeopacity{0.736201}%
\pgfsetdash{}{0pt}%
\pgfpathmoveto{\pgfqpoint{2.202153in}{1.673253in}}%
\pgfpathcurveto{\pgfqpoint{2.210389in}{1.673253in}}{\pgfqpoint{2.218289in}{1.676525in}}{\pgfqpoint{2.224113in}{1.682349in}}%
\pgfpathcurveto{\pgfqpoint{2.229937in}{1.688173in}}{\pgfqpoint{2.233209in}{1.696073in}}{\pgfqpoint{2.233209in}{1.704309in}}%
\pgfpathcurveto{\pgfqpoint{2.233209in}{1.712545in}}{\pgfqpoint{2.229937in}{1.720445in}}{\pgfqpoint{2.224113in}{1.726269in}}%
\pgfpathcurveto{\pgfqpoint{2.218289in}{1.732093in}}{\pgfqpoint{2.210389in}{1.735366in}}{\pgfqpoint{2.202153in}{1.735366in}}%
\pgfpathcurveto{\pgfqpoint{2.193917in}{1.735366in}}{\pgfqpoint{2.186016in}{1.732093in}}{\pgfqpoint{2.180193in}{1.726269in}}%
\pgfpathcurveto{\pgfqpoint{2.174369in}{1.720445in}}{\pgfqpoint{2.171096in}{1.712545in}}{\pgfqpoint{2.171096in}{1.704309in}}%
\pgfpathcurveto{\pgfqpoint{2.171096in}{1.696073in}}{\pgfqpoint{2.174369in}{1.688173in}}{\pgfqpoint{2.180193in}{1.682349in}}%
\pgfpathcurveto{\pgfqpoint{2.186016in}{1.676525in}}{\pgfqpoint{2.193917in}{1.673253in}}{\pgfqpoint{2.202153in}{1.673253in}}%
\pgfpathclose%
\pgfusepath{stroke,fill}%
\end{pgfscope}%
\begin{pgfscope}%
\pgfpathrectangle{\pgfqpoint{0.100000in}{0.212622in}}{\pgfqpoint{3.696000in}{3.696000in}}%
\pgfusepath{clip}%
\pgfsetbuttcap%
\pgfsetroundjoin%
\definecolor{currentfill}{rgb}{0.121569,0.466667,0.705882}%
\pgfsetfillcolor{currentfill}%
\pgfsetfillopacity{0.737461}%
\pgfsetlinewidth{1.003750pt}%
\definecolor{currentstroke}{rgb}{0.121569,0.466667,0.705882}%
\pgfsetstrokecolor{currentstroke}%
\pgfsetstrokeopacity{0.737461}%
\pgfsetdash{}{0pt}%
\pgfpathmoveto{\pgfqpoint{2.203339in}{1.670682in}}%
\pgfpathcurveto{\pgfqpoint{2.211576in}{1.670682in}}{\pgfqpoint{2.219476in}{1.673955in}}{\pgfqpoint{2.225300in}{1.679779in}}%
\pgfpathcurveto{\pgfqpoint{2.231124in}{1.685603in}}{\pgfqpoint{2.234396in}{1.693503in}}{\pgfqpoint{2.234396in}{1.701739in}}%
\pgfpathcurveto{\pgfqpoint{2.234396in}{1.709975in}}{\pgfqpoint{2.231124in}{1.717875in}}{\pgfqpoint{2.225300in}{1.723699in}}%
\pgfpathcurveto{\pgfqpoint{2.219476in}{1.729523in}}{\pgfqpoint{2.211576in}{1.732795in}}{\pgfqpoint{2.203339in}{1.732795in}}%
\pgfpathcurveto{\pgfqpoint{2.195103in}{1.732795in}}{\pgfqpoint{2.187203in}{1.729523in}}{\pgfqpoint{2.181379in}{1.723699in}}%
\pgfpathcurveto{\pgfqpoint{2.175555in}{1.717875in}}{\pgfqpoint{2.172283in}{1.709975in}}{\pgfqpoint{2.172283in}{1.701739in}}%
\pgfpathcurveto{\pgfqpoint{2.172283in}{1.693503in}}{\pgfqpoint{2.175555in}{1.685603in}}{\pgfqpoint{2.181379in}{1.679779in}}%
\pgfpathcurveto{\pgfqpoint{2.187203in}{1.673955in}}{\pgfqpoint{2.195103in}{1.670682in}}{\pgfqpoint{2.203339in}{1.670682in}}%
\pgfpathclose%
\pgfusepath{stroke,fill}%
\end{pgfscope}%
\begin{pgfscope}%
\pgfpathrectangle{\pgfqpoint{0.100000in}{0.212622in}}{\pgfqpoint{3.696000in}{3.696000in}}%
\pgfusepath{clip}%
\pgfsetbuttcap%
\pgfsetroundjoin%
\definecolor{currentfill}{rgb}{0.121569,0.466667,0.705882}%
\pgfsetfillcolor{currentfill}%
\pgfsetfillopacity{0.738940}%
\pgfsetlinewidth{1.003750pt}%
\definecolor{currentstroke}{rgb}{0.121569,0.466667,0.705882}%
\pgfsetstrokecolor{currentstroke}%
\pgfsetstrokeopacity{0.738940}%
\pgfsetdash{}{0pt}%
\pgfpathmoveto{\pgfqpoint{2.204901in}{1.666168in}}%
\pgfpathcurveto{\pgfqpoint{2.213138in}{1.666168in}}{\pgfqpoint{2.221038in}{1.669440in}}{\pgfqpoint{2.226862in}{1.675264in}}%
\pgfpathcurveto{\pgfqpoint{2.232686in}{1.681088in}}{\pgfqpoint{2.235958in}{1.688988in}}{\pgfqpoint{2.235958in}{1.697224in}}%
\pgfpathcurveto{\pgfqpoint{2.235958in}{1.705460in}}{\pgfqpoint{2.232686in}{1.713360in}}{\pgfqpoint{2.226862in}{1.719184in}}%
\pgfpathcurveto{\pgfqpoint{2.221038in}{1.725008in}}{\pgfqpoint{2.213138in}{1.728281in}}{\pgfqpoint{2.204901in}{1.728281in}}%
\pgfpathcurveto{\pgfqpoint{2.196665in}{1.728281in}}{\pgfqpoint{2.188765in}{1.725008in}}{\pgfqpoint{2.182941in}{1.719184in}}%
\pgfpathcurveto{\pgfqpoint{2.177117in}{1.713360in}}{\pgfqpoint{2.173845in}{1.705460in}}{\pgfqpoint{2.173845in}{1.697224in}}%
\pgfpathcurveto{\pgfqpoint{2.173845in}{1.688988in}}{\pgfqpoint{2.177117in}{1.681088in}}{\pgfqpoint{2.182941in}{1.675264in}}%
\pgfpathcurveto{\pgfqpoint{2.188765in}{1.669440in}}{\pgfqpoint{2.196665in}{1.666168in}}{\pgfqpoint{2.204901in}{1.666168in}}%
\pgfpathclose%
\pgfusepath{stroke,fill}%
\end{pgfscope}%
\begin{pgfscope}%
\pgfpathrectangle{\pgfqpoint{0.100000in}{0.212622in}}{\pgfqpoint{3.696000in}{3.696000in}}%
\pgfusepath{clip}%
\pgfsetbuttcap%
\pgfsetroundjoin%
\definecolor{currentfill}{rgb}{0.121569,0.466667,0.705882}%
\pgfsetfillcolor{currentfill}%
\pgfsetfillopacity{0.740128}%
\pgfsetlinewidth{1.003750pt}%
\definecolor{currentstroke}{rgb}{0.121569,0.466667,0.705882}%
\pgfsetstrokecolor{currentstroke}%
\pgfsetstrokeopacity{0.740128}%
\pgfsetdash{}{0pt}%
\pgfpathmoveto{\pgfqpoint{2.205769in}{1.666048in}}%
\pgfpathcurveto{\pgfqpoint{2.214005in}{1.666048in}}{\pgfqpoint{2.221905in}{1.669320in}}{\pgfqpoint{2.227729in}{1.675144in}}%
\pgfpathcurveto{\pgfqpoint{2.233553in}{1.680968in}}{\pgfqpoint{2.236825in}{1.688868in}}{\pgfqpoint{2.236825in}{1.697104in}}%
\pgfpathcurveto{\pgfqpoint{2.236825in}{1.705341in}}{\pgfqpoint{2.233553in}{1.713241in}}{\pgfqpoint{2.227729in}{1.719065in}}%
\pgfpathcurveto{\pgfqpoint{2.221905in}{1.724888in}}{\pgfqpoint{2.214005in}{1.728161in}}{\pgfqpoint{2.205769in}{1.728161in}}%
\pgfpathcurveto{\pgfqpoint{2.197533in}{1.728161in}}{\pgfqpoint{2.189632in}{1.724888in}}{\pgfqpoint{2.183809in}{1.719065in}}%
\pgfpathcurveto{\pgfqpoint{2.177985in}{1.713241in}}{\pgfqpoint{2.174712in}{1.705341in}}{\pgfqpoint{2.174712in}{1.697104in}}%
\pgfpathcurveto{\pgfqpoint{2.174712in}{1.688868in}}{\pgfqpoint{2.177985in}{1.680968in}}{\pgfqpoint{2.183809in}{1.675144in}}%
\pgfpathcurveto{\pgfqpoint{2.189632in}{1.669320in}}{\pgfqpoint{2.197533in}{1.666048in}}{\pgfqpoint{2.205769in}{1.666048in}}%
\pgfpathclose%
\pgfusepath{stroke,fill}%
\end{pgfscope}%
\begin{pgfscope}%
\pgfpathrectangle{\pgfqpoint{0.100000in}{0.212622in}}{\pgfqpoint{3.696000in}{3.696000in}}%
\pgfusepath{clip}%
\pgfsetbuttcap%
\pgfsetroundjoin%
\definecolor{currentfill}{rgb}{0.121569,0.466667,0.705882}%
\pgfsetfillcolor{currentfill}%
\pgfsetfillopacity{0.741453}%
\pgfsetlinewidth{1.003750pt}%
\definecolor{currentstroke}{rgb}{0.121569,0.466667,0.705882}%
\pgfsetstrokecolor{currentstroke}%
\pgfsetstrokeopacity{0.741453}%
\pgfsetdash{}{0pt}%
\pgfpathmoveto{\pgfqpoint{2.207221in}{1.665100in}}%
\pgfpathcurveto{\pgfqpoint{2.215458in}{1.665100in}}{\pgfqpoint{2.223358in}{1.668372in}}{\pgfqpoint{2.229182in}{1.674196in}}%
\pgfpathcurveto{\pgfqpoint{2.235006in}{1.680020in}}{\pgfqpoint{2.238278in}{1.687920in}}{\pgfqpoint{2.238278in}{1.696156in}}%
\pgfpathcurveto{\pgfqpoint{2.238278in}{1.704392in}}{\pgfqpoint{2.235006in}{1.712292in}}{\pgfqpoint{2.229182in}{1.718116in}}%
\pgfpathcurveto{\pgfqpoint{2.223358in}{1.723940in}}{\pgfqpoint{2.215458in}{1.727213in}}{\pgfqpoint{2.207221in}{1.727213in}}%
\pgfpathcurveto{\pgfqpoint{2.198985in}{1.727213in}}{\pgfqpoint{2.191085in}{1.723940in}}{\pgfqpoint{2.185261in}{1.718116in}}%
\pgfpathcurveto{\pgfqpoint{2.179437in}{1.712292in}}{\pgfqpoint{2.176165in}{1.704392in}}{\pgfqpoint{2.176165in}{1.696156in}}%
\pgfpathcurveto{\pgfqpoint{2.176165in}{1.687920in}}{\pgfqpoint{2.179437in}{1.680020in}}{\pgfqpoint{2.185261in}{1.674196in}}%
\pgfpathcurveto{\pgfqpoint{2.191085in}{1.668372in}}{\pgfqpoint{2.198985in}{1.665100in}}{\pgfqpoint{2.207221in}{1.665100in}}%
\pgfpathclose%
\pgfusepath{stroke,fill}%
\end{pgfscope}%
\begin{pgfscope}%
\pgfpathrectangle{\pgfqpoint{0.100000in}{0.212622in}}{\pgfqpoint{3.696000in}{3.696000in}}%
\pgfusepath{clip}%
\pgfsetbuttcap%
\pgfsetroundjoin%
\definecolor{currentfill}{rgb}{0.121569,0.466667,0.705882}%
\pgfsetfillcolor{currentfill}%
\pgfsetfillopacity{0.742181}%
\pgfsetlinewidth{1.003750pt}%
\definecolor{currentstroke}{rgb}{0.121569,0.466667,0.705882}%
\pgfsetstrokecolor{currentstroke}%
\pgfsetstrokeopacity{0.742181}%
\pgfsetdash{}{0pt}%
\pgfpathmoveto{\pgfqpoint{2.207720in}{1.664370in}}%
\pgfpathcurveto{\pgfqpoint{2.215957in}{1.664370in}}{\pgfqpoint{2.223857in}{1.667642in}}{\pgfqpoint{2.229681in}{1.673466in}}%
\pgfpathcurveto{\pgfqpoint{2.235504in}{1.679290in}}{\pgfqpoint{2.238777in}{1.687190in}}{\pgfqpoint{2.238777in}{1.695427in}}%
\pgfpathcurveto{\pgfqpoint{2.238777in}{1.703663in}}{\pgfqpoint{2.235504in}{1.711563in}}{\pgfqpoint{2.229681in}{1.717387in}}%
\pgfpathcurveto{\pgfqpoint{2.223857in}{1.723211in}}{\pgfqpoint{2.215957in}{1.726483in}}{\pgfqpoint{2.207720in}{1.726483in}}%
\pgfpathcurveto{\pgfqpoint{2.199484in}{1.726483in}}{\pgfqpoint{2.191584in}{1.723211in}}{\pgfqpoint{2.185760in}{1.717387in}}%
\pgfpathcurveto{\pgfqpoint{2.179936in}{1.711563in}}{\pgfqpoint{2.176664in}{1.703663in}}{\pgfqpoint{2.176664in}{1.695427in}}%
\pgfpathcurveto{\pgfqpoint{2.176664in}{1.687190in}}{\pgfqpoint{2.179936in}{1.679290in}}{\pgfqpoint{2.185760in}{1.673466in}}%
\pgfpathcurveto{\pgfqpoint{2.191584in}{1.667642in}}{\pgfqpoint{2.199484in}{1.664370in}}{\pgfqpoint{2.207720in}{1.664370in}}%
\pgfpathclose%
\pgfusepath{stroke,fill}%
\end{pgfscope}%
\begin{pgfscope}%
\pgfpathrectangle{\pgfqpoint{0.100000in}{0.212622in}}{\pgfqpoint{3.696000in}{3.696000in}}%
\pgfusepath{clip}%
\pgfsetbuttcap%
\pgfsetroundjoin%
\definecolor{currentfill}{rgb}{0.121569,0.466667,0.705882}%
\pgfsetfillcolor{currentfill}%
\pgfsetfillopacity{0.742922}%
\pgfsetlinewidth{1.003750pt}%
\definecolor{currentstroke}{rgb}{0.121569,0.466667,0.705882}%
\pgfsetstrokecolor{currentstroke}%
\pgfsetstrokeopacity{0.742922}%
\pgfsetdash{}{0pt}%
\pgfpathmoveto{\pgfqpoint{2.208473in}{1.662751in}}%
\pgfpathcurveto{\pgfqpoint{2.216709in}{1.662751in}}{\pgfqpoint{2.224610in}{1.666023in}}{\pgfqpoint{2.230433in}{1.671847in}}%
\pgfpathcurveto{\pgfqpoint{2.236257in}{1.677671in}}{\pgfqpoint{2.239530in}{1.685571in}}{\pgfqpoint{2.239530in}{1.693807in}}%
\pgfpathcurveto{\pgfqpoint{2.239530in}{1.702044in}}{\pgfqpoint{2.236257in}{1.709944in}}{\pgfqpoint{2.230433in}{1.715768in}}%
\pgfpathcurveto{\pgfqpoint{2.224610in}{1.721592in}}{\pgfqpoint{2.216709in}{1.724864in}}{\pgfqpoint{2.208473in}{1.724864in}}%
\pgfpathcurveto{\pgfqpoint{2.200237in}{1.724864in}}{\pgfqpoint{2.192337in}{1.721592in}}{\pgfqpoint{2.186513in}{1.715768in}}%
\pgfpathcurveto{\pgfqpoint{2.180689in}{1.709944in}}{\pgfqpoint{2.177417in}{1.702044in}}{\pgfqpoint{2.177417in}{1.693807in}}%
\pgfpathcurveto{\pgfqpoint{2.177417in}{1.685571in}}{\pgfqpoint{2.180689in}{1.677671in}}{\pgfqpoint{2.186513in}{1.671847in}}%
\pgfpathcurveto{\pgfqpoint{2.192337in}{1.666023in}}{\pgfqpoint{2.200237in}{1.662751in}}{\pgfqpoint{2.208473in}{1.662751in}}%
\pgfpathclose%
\pgfusepath{stroke,fill}%
\end{pgfscope}%
\begin{pgfscope}%
\pgfpathrectangle{\pgfqpoint{0.100000in}{0.212622in}}{\pgfqpoint{3.696000in}{3.696000in}}%
\pgfusepath{clip}%
\pgfsetbuttcap%
\pgfsetroundjoin%
\definecolor{currentfill}{rgb}{0.121569,0.466667,0.705882}%
\pgfsetfillcolor{currentfill}%
\pgfsetfillopacity{0.743457}%
\pgfsetlinewidth{1.003750pt}%
\definecolor{currentstroke}{rgb}{0.121569,0.466667,0.705882}%
\pgfsetstrokecolor{currentstroke}%
\pgfsetstrokeopacity{0.743457}%
\pgfsetdash{}{0pt}%
\pgfpathmoveto{\pgfqpoint{2.208849in}{1.662646in}}%
\pgfpathcurveto{\pgfqpoint{2.217085in}{1.662646in}}{\pgfqpoint{2.224986in}{1.665918in}}{\pgfqpoint{2.230809in}{1.671742in}}%
\pgfpathcurveto{\pgfqpoint{2.236633in}{1.677566in}}{\pgfqpoint{2.239906in}{1.685466in}}{\pgfqpoint{2.239906in}{1.693702in}}%
\pgfpathcurveto{\pgfqpoint{2.239906in}{1.701938in}}{\pgfqpoint{2.236633in}{1.709838in}}{\pgfqpoint{2.230809in}{1.715662in}}%
\pgfpathcurveto{\pgfqpoint{2.224986in}{1.721486in}}{\pgfqpoint{2.217085in}{1.724759in}}{\pgfqpoint{2.208849in}{1.724759in}}%
\pgfpathcurveto{\pgfqpoint{2.200613in}{1.724759in}}{\pgfqpoint{2.192713in}{1.721486in}}{\pgfqpoint{2.186889in}{1.715662in}}%
\pgfpathcurveto{\pgfqpoint{2.181065in}{1.709838in}}{\pgfqpoint{2.177793in}{1.701938in}}{\pgfqpoint{2.177793in}{1.693702in}}%
\pgfpathcurveto{\pgfqpoint{2.177793in}{1.685466in}}{\pgfqpoint{2.181065in}{1.677566in}}{\pgfqpoint{2.186889in}{1.671742in}}%
\pgfpathcurveto{\pgfqpoint{2.192713in}{1.665918in}}{\pgfqpoint{2.200613in}{1.662646in}}{\pgfqpoint{2.208849in}{1.662646in}}%
\pgfpathclose%
\pgfusepath{stroke,fill}%
\end{pgfscope}%
\begin{pgfscope}%
\pgfpathrectangle{\pgfqpoint{0.100000in}{0.212622in}}{\pgfqpoint{3.696000in}{3.696000in}}%
\pgfusepath{clip}%
\pgfsetbuttcap%
\pgfsetroundjoin%
\definecolor{currentfill}{rgb}{0.121569,0.466667,0.705882}%
\pgfsetfillcolor{currentfill}%
\pgfsetfillopacity{0.744097}%
\pgfsetlinewidth{1.003750pt}%
\definecolor{currentstroke}{rgb}{0.121569,0.466667,0.705882}%
\pgfsetstrokecolor{currentstroke}%
\pgfsetstrokeopacity{0.744097}%
\pgfsetdash{}{0pt}%
\pgfpathmoveto{\pgfqpoint{2.209550in}{1.661358in}}%
\pgfpathcurveto{\pgfqpoint{2.217787in}{1.661358in}}{\pgfqpoint{2.225687in}{1.664630in}}{\pgfqpoint{2.231511in}{1.670454in}}%
\pgfpathcurveto{\pgfqpoint{2.237334in}{1.676278in}}{\pgfqpoint{2.240607in}{1.684178in}}{\pgfqpoint{2.240607in}{1.692414in}}%
\pgfpathcurveto{\pgfqpoint{2.240607in}{1.700651in}}{\pgfqpoint{2.237334in}{1.708551in}}{\pgfqpoint{2.231511in}{1.714375in}}%
\pgfpathcurveto{\pgfqpoint{2.225687in}{1.720198in}}{\pgfqpoint{2.217787in}{1.723471in}}{\pgfqpoint{2.209550in}{1.723471in}}%
\pgfpathcurveto{\pgfqpoint{2.201314in}{1.723471in}}{\pgfqpoint{2.193414in}{1.720198in}}{\pgfqpoint{2.187590in}{1.714375in}}%
\pgfpathcurveto{\pgfqpoint{2.181766in}{1.708551in}}{\pgfqpoint{2.178494in}{1.700651in}}{\pgfqpoint{2.178494in}{1.692414in}}%
\pgfpathcurveto{\pgfqpoint{2.178494in}{1.684178in}}{\pgfqpoint{2.181766in}{1.676278in}}{\pgfqpoint{2.187590in}{1.670454in}}%
\pgfpathcurveto{\pgfqpoint{2.193414in}{1.664630in}}{\pgfqpoint{2.201314in}{1.661358in}}{\pgfqpoint{2.209550in}{1.661358in}}%
\pgfpathclose%
\pgfusepath{stroke,fill}%
\end{pgfscope}%
\begin{pgfscope}%
\pgfpathrectangle{\pgfqpoint{0.100000in}{0.212622in}}{\pgfqpoint{3.696000in}{3.696000in}}%
\pgfusepath{clip}%
\pgfsetbuttcap%
\pgfsetroundjoin%
\definecolor{currentfill}{rgb}{0.121569,0.466667,0.705882}%
\pgfsetfillcolor{currentfill}%
\pgfsetfillopacity{0.744487}%
\pgfsetlinewidth{1.003750pt}%
\definecolor{currentstroke}{rgb}{0.121569,0.466667,0.705882}%
\pgfsetstrokecolor{currentstroke}%
\pgfsetstrokeopacity{0.744487}%
\pgfsetdash{}{0pt}%
\pgfpathmoveto{\pgfqpoint{2.209902in}{1.660863in}}%
\pgfpathcurveto{\pgfqpoint{2.218138in}{1.660863in}}{\pgfqpoint{2.226038in}{1.664136in}}{\pgfqpoint{2.231862in}{1.669960in}}%
\pgfpathcurveto{\pgfqpoint{2.237686in}{1.675783in}}{\pgfqpoint{2.240958in}{1.683684in}}{\pgfqpoint{2.240958in}{1.691920in}}%
\pgfpathcurveto{\pgfqpoint{2.240958in}{1.700156in}}{\pgfqpoint{2.237686in}{1.708056in}}{\pgfqpoint{2.231862in}{1.713880in}}%
\pgfpathcurveto{\pgfqpoint{2.226038in}{1.719704in}}{\pgfqpoint{2.218138in}{1.722976in}}{\pgfqpoint{2.209902in}{1.722976in}}%
\pgfpathcurveto{\pgfqpoint{2.201666in}{1.722976in}}{\pgfqpoint{2.193766in}{1.719704in}}{\pgfqpoint{2.187942in}{1.713880in}}%
\pgfpathcurveto{\pgfqpoint{2.182118in}{1.708056in}}{\pgfqpoint{2.178845in}{1.700156in}}{\pgfqpoint{2.178845in}{1.691920in}}%
\pgfpathcurveto{\pgfqpoint{2.178845in}{1.683684in}}{\pgfqpoint{2.182118in}{1.675783in}}{\pgfqpoint{2.187942in}{1.669960in}}%
\pgfpathcurveto{\pgfqpoint{2.193766in}{1.664136in}}{\pgfqpoint{2.201666in}{1.660863in}}{\pgfqpoint{2.209902in}{1.660863in}}%
\pgfpathclose%
\pgfusepath{stroke,fill}%
\end{pgfscope}%
\begin{pgfscope}%
\pgfpathrectangle{\pgfqpoint{0.100000in}{0.212622in}}{\pgfqpoint{3.696000in}{3.696000in}}%
\pgfusepath{clip}%
\pgfsetbuttcap%
\pgfsetroundjoin%
\definecolor{currentfill}{rgb}{0.121569,0.466667,0.705882}%
\pgfsetfillcolor{currentfill}%
\pgfsetfillopacity{0.745262}%
\pgfsetlinewidth{1.003750pt}%
\definecolor{currentstroke}{rgb}{0.121569,0.466667,0.705882}%
\pgfsetstrokecolor{currentstroke}%
\pgfsetstrokeopacity{0.745262}%
\pgfsetdash{}{0pt}%
\pgfpathmoveto{\pgfqpoint{2.210377in}{1.660616in}}%
\pgfpathcurveto{\pgfqpoint{2.218614in}{1.660616in}}{\pgfqpoint{2.226514in}{1.663888in}}{\pgfqpoint{2.232338in}{1.669712in}}%
\pgfpathcurveto{\pgfqpoint{2.238162in}{1.675536in}}{\pgfqpoint{2.241434in}{1.683436in}}{\pgfqpoint{2.241434in}{1.691672in}}%
\pgfpathcurveto{\pgfqpoint{2.241434in}{1.699908in}}{\pgfqpoint{2.238162in}{1.707808in}}{\pgfqpoint{2.232338in}{1.713632in}}%
\pgfpathcurveto{\pgfqpoint{2.226514in}{1.719456in}}{\pgfqpoint{2.218614in}{1.722729in}}{\pgfqpoint{2.210377in}{1.722729in}}%
\pgfpathcurveto{\pgfqpoint{2.202141in}{1.722729in}}{\pgfqpoint{2.194241in}{1.719456in}}{\pgfqpoint{2.188417in}{1.713632in}}%
\pgfpathcurveto{\pgfqpoint{2.182593in}{1.707808in}}{\pgfqpoint{2.179321in}{1.699908in}}{\pgfqpoint{2.179321in}{1.691672in}}%
\pgfpathcurveto{\pgfqpoint{2.179321in}{1.683436in}}{\pgfqpoint{2.182593in}{1.675536in}}{\pgfqpoint{2.188417in}{1.669712in}}%
\pgfpathcurveto{\pgfqpoint{2.194241in}{1.663888in}}{\pgfqpoint{2.202141in}{1.660616in}}{\pgfqpoint{2.210377in}{1.660616in}}%
\pgfpathclose%
\pgfusepath{stroke,fill}%
\end{pgfscope}%
\begin{pgfscope}%
\pgfpathrectangle{\pgfqpoint{0.100000in}{0.212622in}}{\pgfqpoint{3.696000in}{3.696000in}}%
\pgfusepath{clip}%
\pgfsetbuttcap%
\pgfsetroundjoin%
\definecolor{currentfill}{rgb}{0.121569,0.466667,0.705882}%
\pgfsetfillcolor{currentfill}%
\pgfsetfillopacity{0.746193}%
\pgfsetlinewidth{1.003750pt}%
\definecolor{currentstroke}{rgb}{0.121569,0.466667,0.705882}%
\pgfsetstrokecolor{currentstroke}%
\pgfsetstrokeopacity{0.746193}%
\pgfsetdash{}{0pt}%
\pgfpathmoveto{\pgfqpoint{2.211206in}{1.660109in}}%
\pgfpathcurveto{\pgfqpoint{2.219442in}{1.660109in}}{\pgfqpoint{2.227342in}{1.663382in}}{\pgfqpoint{2.233166in}{1.669206in}}%
\pgfpathcurveto{\pgfqpoint{2.238990in}{1.675030in}}{\pgfqpoint{2.242262in}{1.682930in}}{\pgfqpoint{2.242262in}{1.691166in}}%
\pgfpathcurveto{\pgfqpoint{2.242262in}{1.699402in}}{\pgfqpoint{2.238990in}{1.707302in}}{\pgfqpoint{2.233166in}{1.713126in}}%
\pgfpathcurveto{\pgfqpoint{2.227342in}{1.718950in}}{\pgfqpoint{2.219442in}{1.722222in}}{\pgfqpoint{2.211206in}{1.722222in}}%
\pgfpathcurveto{\pgfqpoint{2.202970in}{1.722222in}}{\pgfqpoint{2.195070in}{1.718950in}}{\pgfqpoint{2.189246in}{1.713126in}}%
\pgfpathcurveto{\pgfqpoint{2.183422in}{1.707302in}}{\pgfqpoint{2.180149in}{1.699402in}}{\pgfqpoint{2.180149in}{1.691166in}}%
\pgfpathcurveto{\pgfqpoint{2.180149in}{1.682930in}}{\pgfqpoint{2.183422in}{1.675030in}}{\pgfqpoint{2.189246in}{1.669206in}}%
\pgfpathcurveto{\pgfqpoint{2.195070in}{1.663382in}}{\pgfqpoint{2.202970in}{1.660109in}}{\pgfqpoint{2.211206in}{1.660109in}}%
\pgfpathclose%
\pgfusepath{stroke,fill}%
\end{pgfscope}%
\begin{pgfscope}%
\pgfpathrectangle{\pgfqpoint{0.100000in}{0.212622in}}{\pgfqpoint{3.696000in}{3.696000in}}%
\pgfusepath{clip}%
\pgfsetbuttcap%
\pgfsetroundjoin%
\definecolor{currentfill}{rgb}{0.121569,0.466667,0.705882}%
\pgfsetfillcolor{currentfill}%
\pgfsetfillopacity{0.747744}%
\pgfsetlinewidth{1.003750pt}%
\definecolor{currentstroke}{rgb}{0.121569,0.466667,0.705882}%
\pgfsetstrokecolor{currentstroke}%
\pgfsetstrokeopacity{0.747744}%
\pgfsetdash{}{0pt}%
\pgfpathmoveto{\pgfqpoint{2.212465in}{1.658055in}}%
\pgfpathcurveto{\pgfqpoint{2.220701in}{1.658055in}}{\pgfqpoint{2.228601in}{1.661327in}}{\pgfqpoint{2.234425in}{1.667151in}}%
\pgfpathcurveto{\pgfqpoint{2.240249in}{1.672975in}}{\pgfqpoint{2.243521in}{1.680875in}}{\pgfqpoint{2.243521in}{1.689111in}}%
\pgfpathcurveto{\pgfqpoint{2.243521in}{1.697348in}}{\pgfqpoint{2.240249in}{1.705248in}}{\pgfqpoint{2.234425in}{1.711072in}}%
\pgfpathcurveto{\pgfqpoint{2.228601in}{1.716896in}}{\pgfqpoint{2.220701in}{1.720168in}}{\pgfqpoint{2.212465in}{1.720168in}}%
\pgfpathcurveto{\pgfqpoint{2.204229in}{1.720168in}}{\pgfqpoint{2.196328in}{1.716896in}}{\pgfqpoint{2.190505in}{1.711072in}}%
\pgfpathcurveto{\pgfqpoint{2.184681in}{1.705248in}}{\pgfqpoint{2.181408in}{1.697348in}}{\pgfqpoint{2.181408in}{1.689111in}}%
\pgfpathcurveto{\pgfqpoint{2.181408in}{1.680875in}}{\pgfqpoint{2.184681in}{1.672975in}}{\pgfqpoint{2.190505in}{1.667151in}}%
\pgfpathcurveto{\pgfqpoint{2.196328in}{1.661327in}}{\pgfqpoint{2.204229in}{1.658055in}}{\pgfqpoint{2.212465in}{1.658055in}}%
\pgfpathclose%
\pgfusepath{stroke,fill}%
\end{pgfscope}%
\begin{pgfscope}%
\pgfpathrectangle{\pgfqpoint{0.100000in}{0.212622in}}{\pgfqpoint{3.696000in}{3.696000in}}%
\pgfusepath{clip}%
\pgfsetbuttcap%
\pgfsetroundjoin%
\definecolor{currentfill}{rgb}{0.121569,0.466667,0.705882}%
\pgfsetfillcolor{currentfill}%
\pgfsetfillopacity{0.749267}%
\pgfsetlinewidth{1.003750pt}%
\definecolor{currentstroke}{rgb}{0.121569,0.466667,0.705882}%
\pgfsetstrokecolor{currentstroke}%
\pgfsetstrokeopacity{0.749267}%
\pgfsetdash{}{0pt}%
\pgfpathmoveto{\pgfqpoint{2.213988in}{1.653775in}}%
\pgfpathcurveto{\pgfqpoint{2.222225in}{1.653775in}}{\pgfqpoint{2.230125in}{1.657047in}}{\pgfqpoint{2.235949in}{1.662871in}}%
\pgfpathcurveto{\pgfqpoint{2.241773in}{1.668695in}}{\pgfqpoint{2.245045in}{1.676595in}}{\pgfqpoint{2.245045in}{1.684831in}}%
\pgfpathcurveto{\pgfqpoint{2.245045in}{1.693068in}}{\pgfqpoint{2.241773in}{1.700968in}}{\pgfqpoint{2.235949in}{1.706792in}}%
\pgfpathcurveto{\pgfqpoint{2.230125in}{1.712616in}}{\pgfqpoint{2.222225in}{1.715888in}}{\pgfqpoint{2.213988in}{1.715888in}}%
\pgfpathcurveto{\pgfqpoint{2.205752in}{1.715888in}}{\pgfqpoint{2.197852in}{1.712616in}}{\pgfqpoint{2.192028in}{1.706792in}}%
\pgfpathcurveto{\pgfqpoint{2.186204in}{1.700968in}}{\pgfqpoint{2.182932in}{1.693068in}}{\pgfqpoint{2.182932in}{1.684831in}}%
\pgfpathcurveto{\pgfqpoint{2.182932in}{1.676595in}}{\pgfqpoint{2.186204in}{1.668695in}}{\pgfqpoint{2.192028in}{1.662871in}}%
\pgfpathcurveto{\pgfqpoint{2.197852in}{1.657047in}}{\pgfqpoint{2.205752in}{1.653775in}}{\pgfqpoint{2.213988in}{1.653775in}}%
\pgfpathclose%
\pgfusepath{stroke,fill}%
\end{pgfscope}%
\begin{pgfscope}%
\pgfpathrectangle{\pgfqpoint{0.100000in}{0.212622in}}{\pgfqpoint{3.696000in}{3.696000in}}%
\pgfusepath{clip}%
\pgfsetbuttcap%
\pgfsetroundjoin%
\definecolor{currentfill}{rgb}{0.121569,0.466667,0.705882}%
\pgfsetfillcolor{currentfill}%
\pgfsetfillopacity{0.751831}%
\pgfsetlinewidth{1.003750pt}%
\definecolor{currentstroke}{rgb}{0.121569,0.466667,0.705882}%
\pgfsetstrokecolor{currentstroke}%
\pgfsetstrokeopacity{0.751831}%
\pgfsetdash{}{0pt}%
\pgfpathmoveto{\pgfqpoint{2.216340in}{1.652296in}}%
\pgfpathcurveto{\pgfqpoint{2.224577in}{1.652296in}}{\pgfqpoint{2.232477in}{1.655568in}}{\pgfqpoint{2.238301in}{1.661392in}}%
\pgfpathcurveto{\pgfqpoint{2.244124in}{1.667216in}}{\pgfqpoint{2.247397in}{1.675116in}}{\pgfqpoint{2.247397in}{1.683353in}}%
\pgfpathcurveto{\pgfqpoint{2.247397in}{1.691589in}}{\pgfqpoint{2.244124in}{1.699489in}}{\pgfqpoint{2.238301in}{1.705313in}}%
\pgfpathcurveto{\pgfqpoint{2.232477in}{1.711137in}}{\pgfqpoint{2.224577in}{1.714409in}}{\pgfqpoint{2.216340in}{1.714409in}}%
\pgfpathcurveto{\pgfqpoint{2.208104in}{1.714409in}}{\pgfqpoint{2.200204in}{1.711137in}}{\pgfqpoint{2.194380in}{1.705313in}}%
\pgfpathcurveto{\pgfqpoint{2.188556in}{1.699489in}}{\pgfqpoint{2.185284in}{1.691589in}}{\pgfqpoint{2.185284in}{1.683353in}}%
\pgfpathcurveto{\pgfqpoint{2.185284in}{1.675116in}}{\pgfqpoint{2.188556in}{1.667216in}}{\pgfqpoint{2.194380in}{1.661392in}}%
\pgfpathcurveto{\pgfqpoint{2.200204in}{1.655568in}}{\pgfqpoint{2.208104in}{1.652296in}}{\pgfqpoint{2.216340in}{1.652296in}}%
\pgfpathclose%
\pgfusepath{stroke,fill}%
\end{pgfscope}%
\begin{pgfscope}%
\pgfpathrectangle{\pgfqpoint{0.100000in}{0.212622in}}{\pgfqpoint{3.696000in}{3.696000in}}%
\pgfusepath{clip}%
\pgfsetbuttcap%
\pgfsetroundjoin%
\definecolor{currentfill}{rgb}{0.121569,0.466667,0.705882}%
\pgfsetfillcolor{currentfill}%
\pgfsetfillopacity{0.754978}%
\pgfsetlinewidth{1.003750pt}%
\definecolor{currentstroke}{rgb}{0.121569,0.466667,0.705882}%
\pgfsetstrokecolor{currentstroke}%
\pgfsetstrokeopacity{0.754978}%
\pgfsetdash{}{0pt}%
\pgfpathmoveto{\pgfqpoint{2.218597in}{1.650903in}}%
\pgfpathcurveto{\pgfqpoint{2.226833in}{1.650903in}}{\pgfqpoint{2.234733in}{1.654175in}}{\pgfqpoint{2.240557in}{1.659999in}}%
\pgfpathcurveto{\pgfqpoint{2.246381in}{1.665823in}}{\pgfqpoint{2.249653in}{1.673723in}}{\pgfqpoint{2.249653in}{1.681959in}}%
\pgfpathcurveto{\pgfqpoint{2.249653in}{1.690195in}}{\pgfqpoint{2.246381in}{1.698095in}}{\pgfqpoint{2.240557in}{1.703919in}}%
\pgfpathcurveto{\pgfqpoint{2.234733in}{1.709743in}}{\pgfqpoint{2.226833in}{1.713016in}}{\pgfqpoint{2.218597in}{1.713016in}}%
\pgfpathcurveto{\pgfqpoint{2.210360in}{1.713016in}}{\pgfqpoint{2.202460in}{1.709743in}}{\pgfqpoint{2.196636in}{1.703919in}}%
\pgfpathcurveto{\pgfqpoint{2.190812in}{1.698095in}}{\pgfqpoint{2.187540in}{1.690195in}}{\pgfqpoint{2.187540in}{1.681959in}}%
\pgfpathcurveto{\pgfqpoint{2.187540in}{1.673723in}}{\pgfqpoint{2.190812in}{1.665823in}}{\pgfqpoint{2.196636in}{1.659999in}}%
\pgfpathcurveto{\pgfqpoint{2.202460in}{1.654175in}}{\pgfqpoint{2.210360in}{1.650903in}}{\pgfqpoint{2.218597in}{1.650903in}}%
\pgfpathclose%
\pgfusepath{stroke,fill}%
\end{pgfscope}%
\begin{pgfscope}%
\pgfpathrectangle{\pgfqpoint{0.100000in}{0.212622in}}{\pgfqpoint{3.696000in}{3.696000in}}%
\pgfusepath{clip}%
\pgfsetbuttcap%
\pgfsetroundjoin%
\definecolor{currentfill}{rgb}{0.121569,0.466667,0.705882}%
\pgfsetfillcolor{currentfill}%
\pgfsetfillopacity{0.758381}%
\pgfsetlinewidth{1.003750pt}%
\definecolor{currentstroke}{rgb}{0.121569,0.466667,0.705882}%
\pgfsetstrokecolor{currentstroke}%
\pgfsetstrokeopacity{0.758381}%
\pgfsetdash{}{0pt}%
\pgfpathmoveto{\pgfqpoint{2.220438in}{1.647701in}}%
\pgfpathcurveto{\pgfqpoint{2.228675in}{1.647701in}}{\pgfqpoint{2.236575in}{1.650973in}}{\pgfqpoint{2.242399in}{1.656797in}}%
\pgfpathcurveto{\pgfqpoint{2.248222in}{1.662621in}}{\pgfqpoint{2.251495in}{1.670521in}}{\pgfqpoint{2.251495in}{1.678758in}}%
\pgfpathcurveto{\pgfqpoint{2.251495in}{1.686994in}}{\pgfqpoint{2.248222in}{1.694894in}}{\pgfqpoint{2.242399in}{1.700718in}}%
\pgfpathcurveto{\pgfqpoint{2.236575in}{1.706542in}}{\pgfqpoint{2.228675in}{1.709814in}}{\pgfqpoint{2.220438in}{1.709814in}}%
\pgfpathcurveto{\pgfqpoint{2.212202in}{1.709814in}}{\pgfqpoint{2.204302in}{1.706542in}}{\pgfqpoint{2.198478in}{1.700718in}}%
\pgfpathcurveto{\pgfqpoint{2.192654in}{1.694894in}}{\pgfqpoint{2.189382in}{1.686994in}}{\pgfqpoint{2.189382in}{1.678758in}}%
\pgfpathcurveto{\pgfqpoint{2.189382in}{1.670521in}}{\pgfqpoint{2.192654in}{1.662621in}}{\pgfqpoint{2.198478in}{1.656797in}}%
\pgfpathcurveto{\pgfqpoint{2.204302in}{1.650973in}}{\pgfqpoint{2.212202in}{1.647701in}}{\pgfqpoint{2.220438in}{1.647701in}}%
\pgfpathclose%
\pgfusepath{stroke,fill}%
\end{pgfscope}%
\begin{pgfscope}%
\pgfpathrectangle{\pgfqpoint{0.100000in}{0.212622in}}{\pgfqpoint{3.696000in}{3.696000in}}%
\pgfusepath{clip}%
\pgfsetbuttcap%
\pgfsetroundjoin%
\definecolor{currentfill}{rgb}{0.121569,0.466667,0.705882}%
\pgfsetfillcolor{currentfill}%
\pgfsetfillopacity{0.761578}%
\pgfsetlinewidth{1.003750pt}%
\definecolor{currentstroke}{rgb}{0.121569,0.466667,0.705882}%
\pgfsetstrokecolor{currentstroke}%
\pgfsetstrokeopacity{0.761578}%
\pgfsetdash{}{0pt}%
\pgfpathmoveto{\pgfqpoint{2.223342in}{1.642457in}}%
\pgfpathcurveto{\pgfqpoint{2.231578in}{1.642457in}}{\pgfqpoint{2.239478in}{1.645730in}}{\pgfqpoint{2.245302in}{1.651554in}}%
\pgfpathcurveto{\pgfqpoint{2.251126in}{1.657377in}}{\pgfqpoint{2.254398in}{1.665278in}}{\pgfqpoint{2.254398in}{1.673514in}}%
\pgfpathcurveto{\pgfqpoint{2.254398in}{1.681750in}}{\pgfqpoint{2.251126in}{1.689650in}}{\pgfqpoint{2.245302in}{1.695474in}}%
\pgfpathcurveto{\pgfqpoint{2.239478in}{1.701298in}}{\pgfqpoint{2.231578in}{1.704570in}}{\pgfqpoint{2.223342in}{1.704570in}}%
\pgfpathcurveto{\pgfqpoint{2.215105in}{1.704570in}}{\pgfqpoint{2.207205in}{1.701298in}}{\pgfqpoint{2.201381in}{1.695474in}}%
\pgfpathcurveto{\pgfqpoint{2.195557in}{1.689650in}}{\pgfqpoint{2.192285in}{1.681750in}}{\pgfqpoint{2.192285in}{1.673514in}}%
\pgfpathcurveto{\pgfqpoint{2.192285in}{1.665278in}}{\pgfqpoint{2.195557in}{1.657377in}}{\pgfqpoint{2.201381in}{1.651554in}}%
\pgfpathcurveto{\pgfqpoint{2.207205in}{1.645730in}}{\pgfqpoint{2.215105in}{1.642457in}}{\pgfqpoint{2.223342in}{1.642457in}}%
\pgfpathclose%
\pgfusepath{stroke,fill}%
\end{pgfscope}%
\begin{pgfscope}%
\pgfpathrectangle{\pgfqpoint{0.100000in}{0.212622in}}{\pgfqpoint{3.696000in}{3.696000in}}%
\pgfusepath{clip}%
\pgfsetbuttcap%
\pgfsetroundjoin%
\definecolor{currentfill}{rgb}{0.121569,0.466667,0.705882}%
\pgfsetfillcolor{currentfill}%
\pgfsetfillopacity{0.765729}%
\pgfsetlinewidth{1.003750pt}%
\definecolor{currentstroke}{rgb}{0.121569,0.466667,0.705882}%
\pgfsetstrokecolor{currentstroke}%
\pgfsetstrokeopacity{0.765729}%
\pgfsetdash{}{0pt}%
\pgfpathmoveto{\pgfqpoint{2.226135in}{1.641099in}}%
\pgfpathcurveto{\pgfqpoint{2.234371in}{1.641099in}}{\pgfqpoint{2.242271in}{1.644371in}}{\pgfqpoint{2.248095in}{1.650195in}}%
\pgfpathcurveto{\pgfqpoint{2.253919in}{1.656019in}}{\pgfqpoint{2.257191in}{1.663919in}}{\pgfqpoint{2.257191in}{1.672155in}}%
\pgfpathcurveto{\pgfqpoint{2.257191in}{1.680392in}}{\pgfqpoint{2.253919in}{1.688292in}}{\pgfqpoint{2.248095in}{1.694116in}}%
\pgfpathcurveto{\pgfqpoint{2.242271in}{1.699940in}}{\pgfqpoint{2.234371in}{1.703212in}}{\pgfqpoint{2.226135in}{1.703212in}}%
\pgfpathcurveto{\pgfqpoint{2.217898in}{1.703212in}}{\pgfqpoint{2.209998in}{1.699940in}}{\pgfqpoint{2.204174in}{1.694116in}}%
\pgfpathcurveto{\pgfqpoint{2.198351in}{1.688292in}}{\pgfqpoint{2.195078in}{1.680392in}}{\pgfqpoint{2.195078in}{1.672155in}}%
\pgfpathcurveto{\pgfqpoint{2.195078in}{1.663919in}}{\pgfqpoint{2.198351in}{1.656019in}}{\pgfqpoint{2.204174in}{1.650195in}}%
\pgfpathcurveto{\pgfqpoint{2.209998in}{1.644371in}}{\pgfqpoint{2.217898in}{1.641099in}}{\pgfqpoint{2.226135in}{1.641099in}}%
\pgfpathclose%
\pgfusepath{stroke,fill}%
\end{pgfscope}%
\begin{pgfscope}%
\pgfpathrectangle{\pgfqpoint{0.100000in}{0.212622in}}{\pgfqpoint{3.696000in}{3.696000in}}%
\pgfusepath{clip}%
\pgfsetbuttcap%
\pgfsetroundjoin%
\definecolor{currentfill}{rgb}{0.121569,0.466667,0.705882}%
\pgfsetfillcolor{currentfill}%
\pgfsetfillopacity{0.767780}%
\pgfsetlinewidth{1.003750pt}%
\definecolor{currentstroke}{rgb}{0.121569,0.466667,0.705882}%
\pgfsetstrokecolor{currentstroke}%
\pgfsetstrokeopacity{0.767780}%
\pgfsetdash{}{0pt}%
\pgfpathmoveto{\pgfqpoint{2.227981in}{1.639068in}}%
\pgfpathcurveto{\pgfqpoint{2.236217in}{1.639068in}}{\pgfqpoint{2.244117in}{1.642340in}}{\pgfqpoint{2.249941in}{1.648164in}}%
\pgfpathcurveto{\pgfqpoint{2.255765in}{1.653988in}}{\pgfqpoint{2.259038in}{1.661888in}}{\pgfqpoint{2.259038in}{1.670124in}}%
\pgfpathcurveto{\pgfqpoint{2.259038in}{1.678361in}}{\pgfqpoint{2.255765in}{1.686261in}}{\pgfqpoint{2.249941in}{1.692085in}}%
\pgfpathcurveto{\pgfqpoint{2.244117in}{1.697908in}}{\pgfqpoint{2.236217in}{1.701181in}}{\pgfqpoint{2.227981in}{1.701181in}}%
\pgfpathcurveto{\pgfqpoint{2.219745in}{1.701181in}}{\pgfqpoint{2.211845in}{1.697908in}}{\pgfqpoint{2.206021in}{1.692085in}}%
\pgfpathcurveto{\pgfqpoint{2.200197in}{1.686261in}}{\pgfqpoint{2.196925in}{1.678361in}}{\pgfqpoint{2.196925in}{1.670124in}}%
\pgfpathcurveto{\pgfqpoint{2.196925in}{1.661888in}}{\pgfqpoint{2.200197in}{1.653988in}}{\pgfqpoint{2.206021in}{1.648164in}}%
\pgfpathcurveto{\pgfqpoint{2.211845in}{1.642340in}}{\pgfqpoint{2.219745in}{1.639068in}}{\pgfqpoint{2.227981in}{1.639068in}}%
\pgfpathclose%
\pgfusepath{stroke,fill}%
\end{pgfscope}%
\begin{pgfscope}%
\pgfpathrectangle{\pgfqpoint{0.100000in}{0.212622in}}{\pgfqpoint{3.696000in}{3.696000in}}%
\pgfusepath{clip}%
\pgfsetbuttcap%
\pgfsetroundjoin%
\definecolor{currentfill}{rgb}{0.121569,0.466667,0.705882}%
\pgfsetfillcolor{currentfill}%
\pgfsetfillopacity{0.769790}%
\pgfsetlinewidth{1.003750pt}%
\definecolor{currentstroke}{rgb}{0.121569,0.466667,0.705882}%
\pgfsetstrokecolor{currentstroke}%
\pgfsetstrokeopacity{0.769790}%
\pgfsetdash{}{0pt}%
\pgfpathmoveto{\pgfqpoint{2.230187in}{1.634916in}}%
\pgfpathcurveto{\pgfqpoint{2.238423in}{1.634916in}}{\pgfqpoint{2.246323in}{1.638189in}}{\pgfqpoint{2.252147in}{1.644013in}}%
\pgfpathcurveto{\pgfqpoint{2.257971in}{1.649837in}}{\pgfqpoint{2.261243in}{1.657737in}}{\pgfqpoint{2.261243in}{1.665973in}}%
\pgfpathcurveto{\pgfqpoint{2.261243in}{1.674209in}}{\pgfqpoint{2.257971in}{1.682109in}}{\pgfqpoint{2.252147in}{1.687933in}}%
\pgfpathcurveto{\pgfqpoint{2.246323in}{1.693757in}}{\pgfqpoint{2.238423in}{1.697029in}}{\pgfqpoint{2.230187in}{1.697029in}}%
\pgfpathcurveto{\pgfqpoint{2.221951in}{1.697029in}}{\pgfqpoint{2.214051in}{1.693757in}}{\pgfqpoint{2.208227in}{1.687933in}}%
\pgfpathcurveto{\pgfqpoint{2.202403in}{1.682109in}}{\pgfqpoint{2.199130in}{1.674209in}}{\pgfqpoint{2.199130in}{1.665973in}}%
\pgfpathcurveto{\pgfqpoint{2.199130in}{1.657737in}}{\pgfqpoint{2.202403in}{1.649837in}}{\pgfqpoint{2.208227in}{1.644013in}}%
\pgfpathcurveto{\pgfqpoint{2.214051in}{1.638189in}}{\pgfqpoint{2.221951in}{1.634916in}}{\pgfqpoint{2.230187in}{1.634916in}}%
\pgfpathclose%
\pgfusepath{stroke,fill}%
\end{pgfscope}%
\begin{pgfscope}%
\pgfpathrectangle{\pgfqpoint{0.100000in}{0.212622in}}{\pgfqpoint{3.696000in}{3.696000in}}%
\pgfusepath{clip}%
\pgfsetbuttcap%
\pgfsetroundjoin%
\definecolor{currentfill}{rgb}{0.121569,0.466667,0.705882}%
\pgfsetfillcolor{currentfill}%
\pgfsetfillopacity{0.770998}%
\pgfsetlinewidth{1.003750pt}%
\definecolor{currentstroke}{rgb}{0.121569,0.466667,0.705882}%
\pgfsetstrokecolor{currentstroke}%
\pgfsetstrokeopacity{0.770998}%
\pgfsetdash{}{0pt}%
\pgfpathmoveto{\pgfqpoint{2.231231in}{1.633152in}}%
\pgfpathcurveto{\pgfqpoint{2.239467in}{1.633152in}}{\pgfqpoint{2.247367in}{1.636425in}}{\pgfqpoint{2.253191in}{1.642249in}}%
\pgfpathcurveto{\pgfqpoint{2.259015in}{1.648073in}}{\pgfqpoint{2.262287in}{1.655973in}}{\pgfqpoint{2.262287in}{1.664209in}}%
\pgfpathcurveto{\pgfqpoint{2.262287in}{1.672445in}}{\pgfqpoint{2.259015in}{1.680345in}}{\pgfqpoint{2.253191in}{1.686169in}}%
\pgfpathcurveto{\pgfqpoint{2.247367in}{1.691993in}}{\pgfqpoint{2.239467in}{1.695265in}}{\pgfqpoint{2.231231in}{1.695265in}}%
\pgfpathcurveto{\pgfqpoint{2.222995in}{1.695265in}}{\pgfqpoint{2.215095in}{1.691993in}}{\pgfqpoint{2.209271in}{1.686169in}}%
\pgfpathcurveto{\pgfqpoint{2.203447in}{1.680345in}}{\pgfqpoint{2.200174in}{1.672445in}}{\pgfqpoint{2.200174in}{1.664209in}}%
\pgfpathcurveto{\pgfqpoint{2.200174in}{1.655973in}}{\pgfqpoint{2.203447in}{1.648073in}}{\pgfqpoint{2.209271in}{1.642249in}}%
\pgfpathcurveto{\pgfqpoint{2.215095in}{1.636425in}}{\pgfqpoint{2.222995in}{1.633152in}}{\pgfqpoint{2.231231in}{1.633152in}}%
\pgfpathclose%
\pgfusepath{stroke,fill}%
\end{pgfscope}%
\begin{pgfscope}%
\pgfpathrectangle{\pgfqpoint{0.100000in}{0.212622in}}{\pgfqpoint{3.696000in}{3.696000in}}%
\pgfusepath{clip}%
\pgfsetbuttcap%
\pgfsetroundjoin%
\definecolor{currentfill}{rgb}{0.121569,0.466667,0.705882}%
\pgfsetfillcolor{currentfill}%
\pgfsetfillopacity{0.771765}%
\pgfsetlinewidth{1.003750pt}%
\definecolor{currentstroke}{rgb}{0.121569,0.466667,0.705882}%
\pgfsetstrokecolor{currentstroke}%
\pgfsetstrokeopacity{0.771765}%
\pgfsetdash{}{0pt}%
\pgfpathmoveto{\pgfqpoint{2.231759in}{1.632795in}}%
\pgfpathcurveto{\pgfqpoint{2.239995in}{1.632795in}}{\pgfqpoint{2.247896in}{1.636068in}}{\pgfqpoint{2.253719in}{1.641892in}}%
\pgfpathcurveto{\pgfqpoint{2.259543in}{1.647716in}}{\pgfqpoint{2.262816in}{1.655616in}}{\pgfqpoint{2.262816in}{1.663852in}}%
\pgfpathcurveto{\pgfqpoint{2.262816in}{1.672088in}}{\pgfqpoint{2.259543in}{1.679988in}}{\pgfqpoint{2.253719in}{1.685812in}}%
\pgfpathcurveto{\pgfqpoint{2.247896in}{1.691636in}}{\pgfqpoint{2.239995in}{1.694908in}}{\pgfqpoint{2.231759in}{1.694908in}}%
\pgfpathcurveto{\pgfqpoint{2.223523in}{1.694908in}}{\pgfqpoint{2.215623in}{1.691636in}}{\pgfqpoint{2.209799in}{1.685812in}}%
\pgfpathcurveto{\pgfqpoint{2.203975in}{1.679988in}}{\pgfqpoint{2.200703in}{1.672088in}}{\pgfqpoint{2.200703in}{1.663852in}}%
\pgfpathcurveto{\pgfqpoint{2.200703in}{1.655616in}}{\pgfqpoint{2.203975in}{1.647716in}}{\pgfqpoint{2.209799in}{1.641892in}}%
\pgfpathcurveto{\pgfqpoint{2.215623in}{1.636068in}}{\pgfqpoint{2.223523in}{1.632795in}}{\pgfqpoint{2.231759in}{1.632795in}}%
\pgfpathclose%
\pgfusepath{stroke,fill}%
\end{pgfscope}%
\begin{pgfscope}%
\pgfpathrectangle{\pgfqpoint{0.100000in}{0.212622in}}{\pgfqpoint{3.696000in}{3.696000in}}%
\pgfusepath{clip}%
\pgfsetbuttcap%
\pgfsetroundjoin%
\definecolor{currentfill}{rgb}{0.121569,0.466667,0.705882}%
\pgfsetfillcolor{currentfill}%
\pgfsetfillopacity{0.772758}%
\pgfsetlinewidth{1.003750pt}%
\definecolor{currentstroke}{rgb}{0.121569,0.466667,0.705882}%
\pgfsetstrokecolor{currentstroke}%
\pgfsetstrokeopacity{0.772758}%
\pgfsetdash{}{0pt}%
\pgfpathmoveto{\pgfqpoint{2.232792in}{1.630890in}}%
\pgfpathcurveto{\pgfqpoint{2.241028in}{1.630890in}}{\pgfqpoint{2.248929in}{1.634163in}}{\pgfqpoint{2.254752in}{1.639987in}}%
\pgfpathcurveto{\pgfqpoint{2.260576in}{1.645810in}}{\pgfqpoint{2.263849in}{1.653711in}}{\pgfqpoint{2.263849in}{1.661947in}}%
\pgfpathcurveto{\pgfqpoint{2.263849in}{1.670183in}}{\pgfqpoint{2.260576in}{1.678083in}}{\pgfqpoint{2.254752in}{1.683907in}}%
\pgfpathcurveto{\pgfqpoint{2.248929in}{1.689731in}}{\pgfqpoint{2.241028in}{1.693003in}}{\pgfqpoint{2.232792in}{1.693003in}}%
\pgfpathcurveto{\pgfqpoint{2.224556in}{1.693003in}}{\pgfqpoint{2.216656in}{1.689731in}}{\pgfqpoint{2.210832in}{1.683907in}}%
\pgfpathcurveto{\pgfqpoint{2.205008in}{1.678083in}}{\pgfqpoint{2.201736in}{1.670183in}}{\pgfqpoint{2.201736in}{1.661947in}}%
\pgfpathcurveto{\pgfqpoint{2.201736in}{1.653711in}}{\pgfqpoint{2.205008in}{1.645810in}}{\pgfqpoint{2.210832in}{1.639987in}}%
\pgfpathcurveto{\pgfqpoint{2.216656in}{1.634163in}}{\pgfqpoint{2.224556in}{1.630890in}}{\pgfqpoint{2.232792in}{1.630890in}}%
\pgfpathclose%
\pgfusepath{stroke,fill}%
\end{pgfscope}%
\begin{pgfscope}%
\pgfpathrectangle{\pgfqpoint{0.100000in}{0.212622in}}{\pgfqpoint{3.696000in}{3.696000in}}%
\pgfusepath{clip}%
\pgfsetbuttcap%
\pgfsetroundjoin%
\definecolor{currentfill}{rgb}{0.121569,0.466667,0.705882}%
\pgfsetfillcolor{currentfill}%
\pgfsetfillopacity{0.773271}%
\pgfsetlinewidth{1.003750pt}%
\definecolor{currentstroke}{rgb}{0.121569,0.466667,0.705882}%
\pgfsetstrokecolor{currentstroke}%
\pgfsetstrokeopacity{0.773271}%
\pgfsetdash{}{0pt}%
\pgfpathmoveto{\pgfqpoint{2.233364in}{1.629631in}}%
\pgfpathcurveto{\pgfqpoint{2.241600in}{1.629631in}}{\pgfqpoint{2.249500in}{1.632903in}}{\pgfqpoint{2.255324in}{1.638727in}}%
\pgfpathcurveto{\pgfqpoint{2.261148in}{1.644551in}}{\pgfqpoint{2.264421in}{1.652451in}}{\pgfqpoint{2.264421in}{1.660687in}}%
\pgfpathcurveto{\pgfqpoint{2.264421in}{1.668923in}}{\pgfqpoint{2.261148in}{1.676823in}}{\pgfqpoint{2.255324in}{1.682647in}}%
\pgfpathcurveto{\pgfqpoint{2.249500in}{1.688471in}}{\pgfqpoint{2.241600in}{1.691744in}}{\pgfqpoint{2.233364in}{1.691744in}}%
\pgfpathcurveto{\pgfqpoint{2.225128in}{1.691744in}}{\pgfqpoint{2.217228in}{1.688471in}}{\pgfqpoint{2.211404in}{1.682647in}}%
\pgfpathcurveto{\pgfqpoint{2.205580in}{1.676823in}}{\pgfqpoint{2.202308in}{1.668923in}}{\pgfqpoint{2.202308in}{1.660687in}}%
\pgfpathcurveto{\pgfqpoint{2.202308in}{1.652451in}}{\pgfqpoint{2.205580in}{1.644551in}}{\pgfqpoint{2.211404in}{1.638727in}}%
\pgfpathcurveto{\pgfqpoint{2.217228in}{1.632903in}}{\pgfqpoint{2.225128in}{1.629631in}}{\pgfqpoint{2.233364in}{1.629631in}}%
\pgfpathclose%
\pgfusepath{stroke,fill}%
\end{pgfscope}%
\begin{pgfscope}%
\pgfpathrectangle{\pgfqpoint{0.100000in}{0.212622in}}{\pgfqpoint{3.696000in}{3.696000in}}%
\pgfusepath{clip}%
\pgfsetbuttcap%
\pgfsetroundjoin%
\definecolor{currentfill}{rgb}{0.121569,0.466667,0.705882}%
\pgfsetfillcolor{currentfill}%
\pgfsetfillopacity{0.774032}%
\pgfsetlinewidth{1.003750pt}%
\definecolor{currentstroke}{rgb}{0.121569,0.466667,0.705882}%
\pgfsetstrokecolor{currentstroke}%
\pgfsetstrokeopacity{0.774032}%
\pgfsetdash{}{0pt}%
\pgfpathmoveto{\pgfqpoint{2.233981in}{1.628460in}}%
\pgfpathcurveto{\pgfqpoint{2.242217in}{1.628460in}}{\pgfqpoint{2.250118in}{1.631732in}}{\pgfqpoint{2.255941in}{1.637556in}}%
\pgfpathcurveto{\pgfqpoint{2.261765in}{1.643380in}}{\pgfqpoint{2.265038in}{1.651280in}}{\pgfqpoint{2.265038in}{1.659516in}}%
\pgfpathcurveto{\pgfqpoint{2.265038in}{1.667752in}}{\pgfqpoint{2.261765in}{1.675653in}}{\pgfqpoint{2.255941in}{1.681476in}}%
\pgfpathcurveto{\pgfqpoint{2.250118in}{1.687300in}}{\pgfqpoint{2.242217in}{1.690573in}}{\pgfqpoint{2.233981in}{1.690573in}}%
\pgfpathcurveto{\pgfqpoint{2.225745in}{1.690573in}}{\pgfqpoint{2.217845in}{1.687300in}}{\pgfqpoint{2.212021in}{1.681476in}}%
\pgfpathcurveto{\pgfqpoint{2.206197in}{1.675653in}}{\pgfqpoint{2.202925in}{1.667752in}}{\pgfqpoint{2.202925in}{1.659516in}}%
\pgfpathcurveto{\pgfqpoint{2.202925in}{1.651280in}}{\pgfqpoint{2.206197in}{1.643380in}}{\pgfqpoint{2.212021in}{1.637556in}}%
\pgfpathcurveto{\pgfqpoint{2.217845in}{1.631732in}}{\pgfqpoint{2.225745in}{1.628460in}}{\pgfqpoint{2.233981in}{1.628460in}}%
\pgfpathclose%
\pgfusepath{stroke,fill}%
\end{pgfscope}%
\begin{pgfscope}%
\pgfpathrectangle{\pgfqpoint{0.100000in}{0.212622in}}{\pgfqpoint{3.696000in}{3.696000in}}%
\pgfusepath{clip}%
\pgfsetbuttcap%
\pgfsetroundjoin%
\definecolor{currentfill}{rgb}{0.121569,0.466667,0.705882}%
\pgfsetfillcolor{currentfill}%
\pgfsetfillopacity{0.774500}%
\pgfsetlinewidth{1.003750pt}%
\definecolor{currentstroke}{rgb}{0.121569,0.466667,0.705882}%
\pgfsetstrokecolor{currentstroke}%
\pgfsetstrokeopacity{0.774500}%
\pgfsetdash{}{0pt}%
\pgfpathmoveto{\pgfqpoint{2.234380in}{1.628156in}}%
\pgfpathcurveto{\pgfqpoint{2.242616in}{1.628156in}}{\pgfqpoint{2.250516in}{1.631428in}}{\pgfqpoint{2.256340in}{1.637252in}}%
\pgfpathcurveto{\pgfqpoint{2.262164in}{1.643076in}}{\pgfqpoint{2.265436in}{1.650976in}}{\pgfqpoint{2.265436in}{1.659212in}}%
\pgfpathcurveto{\pgfqpoint{2.265436in}{1.667449in}}{\pgfqpoint{2.262164in}{1.675349in}}{\pgfqpoint{2.256340in}{1.681173in}}%
\pgfpathcurveto{\pgfqpoint{2.250516in}{1.686997in}}{\pgfqpoint{2.242616in}{1.690269in}}{\pgfqpoint{2.234380in}{1.690269in}}%
\pgfpathcurveto{\pgfqpoint{2.226143in}{1.690269in}}{\pgfqpoint{2.218243in}{1.686997in}}{\pgfqpoint{2.212420in}{1.681173in}}%
\pgfpathcurveto{\pgfqpoint{2.206596in}{1.675349in}}{\pgfqpoint{2.203323in}{1.667449in}}{\pgfqpoint{2.203323in}{1.659212in}}%
\pgfpathcurveto{\pgfqpoint{2.203323in}{1.650976in}}{\pgfqpoint{2.206596in}{1.643076in}}{\pgfqpoint{2.212420in}{1.637252in}}%
\pgfpathcurveto{\pgfqpoint{2.218243in}{1.631428in}}{\pgfqpoint{2.226143in}{1.628156in}}{\pgfqpoint{2.234380in}{1.628156in}}%
\pgfpathclose%
\pgfusepath{stroke,fill}%
\end{pgfscope}%
\begin{pgfscope}%
\pgfpathrectangle{\pgfqpoint{0.100000in}{0.212622in}}{\pgfqpoint{3.696000in}{3.696000in}}%
\pgfusepath{clip}%
\pgfsetbuttcap%
\pgfsetroundjoin%
\definecolor{currentfill}{rgb}{0.121569,0.466667,0.705882}%
\pgfsetfillcolor{currentfill}%
\pgfsetfillopacity{0.775546}%
\pgfsetlinewidth{1.003750pt}%
\definecolor{currentstroke}{rgb}{0.121569,0.466667,0.705882}%
\pgfsetstrokecolor{currentstroke}%
\pgfsetstrokeopacity{0.775546}%
\pgfsetdash{}{0pt}%
\pgfpathmoveto{\pgfqpoint{2.235527in}{1.626418in}}%
\pgfpathcurveto{\pgfqpoint{2.243763in}{1.626418in}}{\pgfqpoint{2.251663in}{1.629690in}}{\pgfqpoint{2.257487in}{1.635514in}}%
\pgfpathcurveto{\pgfqpoint{2.263311in}{1.641338in}}{\pgfqpoint{2.266584in}{1.649238in}}{\pgfqpoint{2.266584in}{1.657474in}}%
\pgfpathcurveto{\pgfqpoint{2.266584in}{1.665711in}}{\pgfqpoint{2.263311in}{1.673611in}}{\pgfqpoint{2.257487in}{1.679435in}}%
\pgfpathcurveto{\pgfqpoint{2.251663in}{1.685258in}}{\pgfqpoint{2.243763in}{1.688531in}}{\pgfqpoint{2.235527in}{1.688531in}}%
\pgfpathcurveto{\pgfqpoint{2.227291in}{1.688531in}}{\pgfqpoint{2.219391in}{1.685258in}}{\pgfqpoint{2.213567in}{1.679435in}}%
\pgfpathcurveto{\pgfqpoint{2.207743in}{1.673611in}}{\pgfqpoint{2.204471in}{1.665711in}}{\pgfqpoint{2.204471in}{1.657474in}}%
\pgfpathcurveto{\pgfqpoint{2.204471in}{1.649238in}}{\pgfqpoint{2.207743in}{1.641338in}}{\pgfqpoint{2.213567in}{1.635514in}}%
\pgfpathcurveto{\pgfqpoint{2.219391in}{1.629690in}}{\pgfqpoint{2.227291in}{1.626418in}}{\pgfqpoint{2.235527in}{1.626418in}}%
\pgfpathclose%
\pgfusepath{stroke,fill}%
\end{pgfscope}%
\begin{pgfscope}%
\pgfpathrectangle{\pgfqpoint{0.100000in}{0.212622in}}{\pgfqpoint{3.696000in}{3.696000in}}%
\pgfusepath{clip}%
\pgfsetbuttcap%
\pgfsetroundjoin%
\definecolor{currentfill}{rgb}{0.121569,0.466667,0.705882}%
\pgfsetfillcolor{currentfill}%
\pgfsetfillopacity{0.776601}%
\pgfsetlinewidth{1.003750pt}%
\definecolor{currentstroke}{rgb}{0.121569,0.466667,0.705882}%
\pgfsetstrokecolor{currentstroke}%
\pgfsetstrokeopacity{0.776601}%
\pgfsetdash{}{0pt}%
\pgfpathmoveto{\pgfqpoint{2.236637in}{1.623337in}}%
\pgfpathcurveto{\pgfqpoint{2.244873in}{1.623337in}}{\pgfqpoint{2.252773in}{1.626610in}}{\pgfqpoint{2.258597in}{1.632433in}}%
\pgfpathcurveto{\pgfqpoint{2.264421in}{1.638257in}}{\pgfqpoint{2.267693in}{1.646157in}}{\pgfqpoint{2.267693in}{1.654394in}}%
\pgfpathcurveto{\pgfqpoint{2.267693in}{1.662630in}}{\pgfqpoint{2.264421in}{1.670530in}}{\pgfqpoint{2.258597in}{1.676354in}}%
\pgfpathcurveto{\pgfqpoint{2.252773in}{1.682178in}}{\pgfqpoint{2.244873in}{1.685450in}}{\pgfqpoint{2.236637in}{1.685450in}}%
\pgfpathcurveto{\pgfqpoint{2.228401in}{1.685450in}}{\pgfqpoint{2.220501in}{1.682178in}}{\pgfqpoint{2.214677in}{1.676354in}}%
\pgfpathcurveto{\pgfqpoint{2.208853in}{1.670530in}}{\pgfqpoint{2.205580in}{1.662630in}}{\pgfqpoint{2.205580in}{1.654394in}}%
\pgfpathcurveto{\pgfqpoint{2.205580in}{1.646157in}}{\pgfqpoint{2.208853in}{1.638257in}}{\pgfqpoint{2.214677in}{1.632433in}}%
\pgfpathcurveto{\pgfqpoint{2.220501in}{1.626610in}}{\pgfqpoint{2.228401in}{1.623337in}}{\pgfqpoint{2.236637in}{1.623337in}}%
\pgfpathclose%
\pgfusepath{stroke,fill}%
\end{pgfscope}%
\begin{pgfscope}%
\pgfpathrectangle{\pgfqpoint{0.100000in}{0.212622in}}{\pgfqpoint{3.696000in}{3.696000in}}%
\pgfusepath{clip}%
\pgfsetbuttcap%
\pgfsetroundjoin%
\definecolor{currentfill}{rgb}{0.121569,0.466667,0.705882}%
\pgfsetfillcolor{currentfill}%
\pgfsetfillopacity{0.778445}%
\pgfsetlinewidth{1.003750pt}%
\definecolor{currentstroke}{rgb}{0.121569,0.466667,0.705882}%
\pgfsetstrokecolor{currentstroke}%
\pgfsetstrokeopacity{0.778445}%
\pgfsetdash{}{0pt}%
\pgfpathmoveto{\pgfqpoint{2.237889in}{1.620584in}}%
\pgfpathcurveto{\pgfqpoint{2.246125in}{1.620584in}}{\pgfqpoint{2.254025in}{1.623856in}}{\pgfqpoint{2.259849in}{1.629680in}}%
\pgfpathcurveto{\pgfqpoint{2.265673in}{1.635504in}}{\pgfqpoint{2.268945in}{1.643404in}}{\pgfqpoint{2.268945in}{1.651640in}}%
\pgfpathcurveto{\pgfqpoint{2.268945in}{1.659877in}}{\pgfqpoint{2.265673in}{1.667777in}}{\pgfqpoint{2.259849in}{1.673601in}}%
\pgfpathcurveto{\pgfqpoint{2.254025in}{1.679425in}}{\pgfqpoint{2.246125in}{1.682697in}}{\pgfqpoint{2.237889in}{1.682697in}}%
\pgfpathcurveto{\pgfqpoint{2.229652in}{1.682697in}}{\pgfqpoint{2.221752in}{1.679425in}}{\pgfqpoint{2.215928in}{1.673601in}}%
\pgfpathcurveto{\pgfqpoint{2.210104in}{1.667777in}}{\pgfqpoint{2.206832in}{1.659877in}}{\pgfqpoint{2.206832in}{1.651640in}}%
\pgfpathcurveto{\pgfqpoint{2.206832in}{1.643404in}}{\pgfqpoint{2.210104in}{1.635504in}}{\pgfqpoint{2.215928in}{1.629680in}}%
\pgfpathcurveto{\pgfqpoint{2.221752in}{1.623856in}}{\pgfqpoint{2.229652in}{1.620584in}}{\pgfqpoint{2.237889in}{1.620584in}}%
\pgfpathclose%
\pgfusepath{stroke,fill}%
\end{pgfscope}%
\begin{pgfscope}%
\pgfpathrectangle{\pgfqpoint{0.100000in}{0.212622in}}{\pgfqpoint{3.696000in}{3.696000in}}%
\pgfusepath{clip}%
\pgfsetbuttcap%
\pgfsetroundjoin%
\definecolor{currentfill}{rgb}{0.121569,0.466667,0.705882}%
\pgfsetfillcolor{currentfill}%
\pgfsetfillopacity{0.779483}%
\pgfsetlinewidth{1.003750pt}%
\definecolor{currentstroke}{rgb}{0.121569,0.466667,0.705882}%
\pgfsetstrokecolor{currentstroke}%
\pgfsetstrokeopacity{0.779483}%
\pgfsetdash{}{0pt}%
\pgfpathmoveto{\pgfqpoint{2.238922in}{1.619424in}}%
\pgfpathcurveto{\pgfqpoint{2.247159in}{1.619424in}}{\pgfqpoint{2.255059in}{1.622696in}}{\pgfqpoint{2.260883in}{1.628520in}}%
\pgfpathcurveto{\pgfqpoint{2.266707in}{1.634344in}}{\pgfqpoint{2.269979in}{1.642244in}}{\pgfqpoint{2.269979in}{1.650481in}}%
\pgfpathcurveto{\pgfqpoint{2.269979in}{1.658717in}}{\pgfqpoint{2.266707in}{1.666617in}}{\pgfqpoint{2.260883in}{1.672441in}}%
\pgfpathcurveto{\pgfqpoint{2.255059in}{1.678265in}}{\pgfqpoint{2.247159in}{1.681537in}}{\pgfqpoint{2.238922in}{1.681537in}}%
\pgfpathcurveto{\pgfqpoint{2.230686in}{1.681537in}}{\pgfqpoint{2.222786in}{1.678265in}}{\pgfqpoint{2.216962in}{1.672441in}}%
\pgfpathcurveto{\pgfqpoint{2.211138in}{1.666617in}}{\pgfqpoint{2.207866in}{1.658717in}}{\pgfqpoint{2.207866in}{1.650481in}}%
\pgfpathcurveto{\pgfqpoint{2.207866in}{1.642244in}}{\pgfqpoint{2.211138in}{1.634344in}}{\pgfqpoint{2.216962in}{1.628520in}}%
\pgfpathcurveto{\pgfqpoint{2.222786in}{1.622696in}}{\pgfqpoint{2.230686in}{1.619424in}}{\pgfqpoint{2.238922in}{1.619424in}}%
\pgfpathclose%
\pgfusepath{stroke,fill}%
\end{pgfscope}%
\begin{pgfscope}%
\pgfpathrectangle{\pgfqpoint{0.100000in}{0.212622in}}{\pgfqpoint{3.696000in}{3.696000in}}%
\pgfusepath{clip}%
\pgfsetbuttcap%
\pgfsetroundjoin%
\definecolor{currentfill}{rgb}{0.121569,0.466667,0.705882}%
\pgfsetfillcolor{currentfill}%
\pgfsetfillopacity{0.781080}%
\pgfsetlinewidth{1.003750pt}%
\definecolor{currentstroke}{rgb}{0.121569,0.466667,0.705882}%
\pgfsetstrokecolor{currentstroke}%
\pgfsetstrokeopacity{0.781080}%
\pgfsetdash{}{0pt}%
\pgfpathmoveto{\pgfqpoint{2.240324in}{1.616622in}}%
\pgfpathcurveto{\pgfqpoint{2.248561in}{1.616622in}}{\pgfqpoint{2.256461in}{1.619894in}}{\pgfqpoint{2.262284in}{1.625718in}}%
\pgfpathcurveto{\pgfqpoint{2.268108in}{1.631542in}}{\pgfqpoint{2.271381in}{1.639442in}}{\pgfqpoint{2.271381in}{1.647678in}}%
\pgfpathcurveto{\pgfqpoint{2.271381in}{1.655914in}}{\pgfqpoint{2.268108in}{1.663814in}}{\pgfqpoint{2.262284in}{1.669638in}}%
\pgfpathcurveto{\pgfqpoint{2.256461in}{1.675462in}}{\pgfqpoint{2.248561in}{1.678735in}}{\pgfqpoint{2.240324in}{1.678735in}}%
\pgfpathcurveto{\pgfqpoint{2.232088in}{1.678735in}}{\pgfqpoint{2.224188in}{1.675462in}}{\pgfqpoint{2.218364in}{1.669638in}}%
\pgfpathcurveto{\pgfqpoint{2.212540in}{1.663814in}}{\pgfqpoint{2.209268in}{1.655914in}}{\pgfqpoint{2.209268in}{1.647678in}}%
\pgfpathcurveto{\pgfqpoint{2.209268in}{1.639442in}}{\pgfqpoint{2.212540in}{1.631542in}}{\pgfqpoint{2.218364in}{1.625718in}}%
\pgfpathcurveto{\pgfqpoint{2.224188in}{1.619894in}}{\pgfqpoint{2.232088in}{1.616622in}}{\pgfqpoint{2.240324in}{1.616622in}}%
\pgfpathclose%
\pgfusepath{stroke,fill}%
\end{pgfscope}%
\begin{pgfscope}%
\pgfpathrectangle{\pgfqpoint{0.100000in}{0.212622in}}{\pgfqpoint{3.696000in}{3.696000in}}%
\pgfusepath{clip}%
\pgfsetbuttcap%
\pgfsetroundjoin%
\definecolor{currentfill}{rgb}{0.121569,0.466667,0.705882}%
\pgfsetfillcolor{currentfill}%
\pgfsetfillopacity{0.781799}%
\pgfsetlinewidth{1.003750pt}%
\definecolor{currentstroke}{rgb}{0.121569,0.466667,0.705882}%
\pgfsetstrokecolor{currentstroke}%
\pgfsetstrokeopacity{0.781799}%
\pgfsetdash{}{0pt}%
\pgfpathmoveto{\pgfqpoint{2.240997in}{1.614021in}}%
\pgfpathcurveto{\pgfqpoint{2.249234in}{1.614021in}}{\pgfqpoint{2.257134in}{1.617294in}}{\pgfqpoint{2.262958in}{1.623118in}}%
\pgfpathcurveto{\pgfqpoint{2.268781in}{1.628941in}}{\pgfqpoint{2.272054in}{1.636841in}}{\pgfqpoint{2.272054in}{1.645078in}}%
\pgfpathcurveto{\pgfqpoint{2.272054in}{1.653314in}}{\pgfqpoint{2.268781in}{1.661214in}}{\pgfqpoint{2.262958in}{1.667038in}}%
\pgfpathcurveto{\pgfqpoint{2.257134in}{1.672862in}}{\pgfqpoint{2.249234in}{1.676134in}}{\pgfqpoint{2.240997in}{1.676134in}}%
\pgfpathcurveto{\pgfqpoint{2.232761in}{1.676134in}}{\pgfqpoint{2.224861in}{1.672862in}}{\pgfqpoint{2.219037in}{1.667038in}}%
\pgfpathcurveto{\pgfqpoint{2.213213in}{1.661214in}}{\pgfqpoint{2.209941in}{1.653314in}}{\pgfqpoint{2.209941in}{1.645078in}}%
\pgfpathcurveto{\pgfqpoint{2.209941in}{1.636841in}}{\pgfqpoint{2.213213in}{1.628941in}}{\pgfqpoint{2.219037in}{1.623118in}}%
\pgfpathcurveto{\pgfqpoint{2.224861in}{1.617294in}}{\pgfqpoint{2.232761in}{1.614021in}}{\pgfqpoint{2.240997in}{1.614021in}}%
\pgfpathclose%
\pgfusepath{stroke,fill}%
\end{pgfscope}%
\begin{pgfscope}%
\pgfpathrectangle{\pgfqpoint{0.100000in}{0.212622in}}{\pgfqpoint{3.696000in}{3.696000in}}%
\pgfusepath{clip}%
\pgfsetbuttcap%
\pgfsetroundjoin%
\definecolor{currentfill}{rgb}{0.121569,0.466667,0.705882}%
\pgfsetfillcolor{currentfill}%
\pgfsetfillopacity{0.782940}%
\pgfsetlinewidth{1.003750pt}%
\definecolor{currentstroke}{rgb}{0.121569,0.466667,0.705882}%
\pgfsetstrokecolor{currentstroke}%
\pgfsetstrokeopacity{0.782940}%
\pgfsetdash{}{0pt}%
\pgfpathmoveto{\pgfqpoint{2.241979in}{1.612504in}}%
\pgfpathcurveto{\pgfqpoint{2.250215in}{1.612504in}}{\pgfqpoint{2.258115in}{1.615776in}}{\pgfqpoint{2.263939in}{1.621600in}}%
\pgfpathcurveto{\pgfqpoint{2.269763in}{1.627424in}}{\pgfqpoint{2.273035in}{1.635324in}}{\pgfqpoint{2.273035in}{1.643561in}}%
\pgfpathcurveto{\pgfqpoint{2.273035in}{1.651797in}}{\pgfqpoint{2.269763in}{1.659697in}}{\pgfqpoint{2.263939in}{1.665521in}}%
\pgfpathcurveto{\pgfqpoint{2.258115in}{1.671345in}}{\pgfqpoint{2.250215in}{1.674617in}}{\pgfqpoint{2.241979in}{1.674617in}}%
\pgfpathcurveto{\pgfqpoint{2.233742in}{1.674617in}}{\pgfqpoint{2.225842in}{1.671345in}}{\pgfqpoint{2.220018in}{1.665521in}}%
\pgfpathcurveto{\pgfqpoint{2.214194in}{1.659697in}}{\pgfqpoint{2.210922in}{1.651797in}}{\pgfqpoint{2.210922in}{1.643561in}}%
\pgfpathcurveto{\pgfqpoint{2.210922in}{1.635324in}}{\pgfqpoint{2.214194in}{1.627424in}}{\pgfqpoint{2.220018in}{1.621600in}}%
\pgfpathcurveto{\pgfqpoint{2.225842in}{1.615776in}}{\pgfqpoint{2.233742in}{1.612504in}}{\pgfqpoint{2.241979in}{1.612504in}}%
\pgfpathclose%
\pgfusepath{stroke,fill}%
\end{pgfscope}%
\begin{pgfscope}%
\pgfpathrectangle{\pgfqpoint{0.100000in}{0.212622in}}{\pgfqpoint{3.696000in}{3.696000in}}%
\pgfusepath{clip}%
\pgfsetbuttcap%
\pgfsetroundjoin%
\definecolor{currentfill}{rgb}{0.121569,0.466667,0.705882}%
\pgfsetfillcolor{currentfill}%
\pgfsetfillopacity{0.783645}%
\pgfsetlinewidth{1.003750pt}%
\definecolor{currentstroke}{rgb}{0.121569,0.466667,0.705882}%
\pgfsetstrokecolor{currentstroke}%
\pgfsetstrokeopacity{0.783645}%
\pgfsetdash{}{0pt}%
\pgfpathmoveto{\pgfqpoint{2.242420in}{1.612094in}}%
\pgfpathcurveto{\pgfqpoint{2.250656in}{1.612094in}}{\pgfqpoint{2.258556in}{1.615366in}}{\pgfqpoint{2.264380in}{1.621190in}}%
\pgfpathcurveto{\pgfqpoint{2.270204in}{1.627014in}}{\pgfqpoint{2.273477in}{1.634914in}}{\pgfqpoint{2.273477in}{1.643150in}}%
\pgfpathcurveto{\pgfqpoint{2.273477in}{1.651387in}}{\pgfqpoint{2.270204in}{1.659287in}}{\pgfqpoint{2.264380in}{1.665111in}}%
\pgfpathcurveto{\pgfqpoint{2.258556in}{1.670935in}}{\pgfqpoint{2.250656in}{1.674207in}}{\pgfqpoint{2.242420in}{1.674207in}}%
\pgfpathcurveto{\pgfqpoint{2.234184in}{1.674207in}}{\pgfqpoint{2.226284in}{1.670935in}}{\pgfqpoint{2.220460in}{1.665111in}}%
\pgfpathcurveto{\pgfqpoint{2.214636in}{1.659287in}}{\pgfqpoint{2.211364in}{1.651387in}}{\pgfqpoint{2.211364in}{1.643150in}}%
\pgfpathcurveto{\pgfqpoint{2.211364in}{1.634914in}}{\pgfqpoint{2.214636in}{1.627014in}}{\pgfqpoint{2.220460in}{1.621190in}}%
\pgfpathcurveto{\pgfqpoint{2.226284in}{1.615366in}}{\pgfqpoint{2.234184in}{1.612094in}}{\pgfqpoint{2.242420in}{1.612094in}}%
\pgfpathclose%
\pgfusepath{stroke,fill}%
\end{pgfscope}%
\begin{pgfscope}%
\pgfpathrectangle{\pgfqpoint{0.100000in}{0.212622in}}{\pgfqpoint{3.696000in}{3.696000in}}%
\pgfusepath{clip}%
\pgfsetbuttcap%
\pgfsetroundjoin%
\definecolor{currentfill}{rgb}{0.121569,0.466667,0.705882}%
\pgfsetfillcolor{currentfill}%
\pgfsetfillopacity{0.784821}%
\pgfsetlinewidth{1.003750pt}%
\definecolor{currentstroke}{rgb}{0.121569,0.466667,0.705882}%
\pgfsetstrokecolor{currentstroke}%
\pgfsetstrokeopacity{0.784821}%
\pgfsetdash{}{0pt}%
\pgfpathmoveto{\pgfqpoint{2.243389in}{1.610785in}}%
\pgfpathcurveto{\pgfqpoint{2.251625in}{1.610785in}}{\pgfqpoint{2.259525in}{1.614058in}}{\pgfqpoint{2.265349in}{1.619882in}}%
\pgfpathcurveto{\pgfqpoint{2.271173in}{1.625706in}}{\pgfqpoint{2.274445in}{1.633606in}}{\pgfqpoint{2.274445in}{1.641842in}}%
\pgfpathcurveto{\pgfqpoint{2.274445in}{1.650078in}}{\pgfqpoint{2.271173in}{1.657978in}}{\pgfqpoint{2.265349in}{1.663802in}}%
\pgfpathcurveto{\pgfqpoint{2.259525in}{1.669626in}}{\pgfqpoint{2.251625in}{1.672898in}}{\pgfqpoint{2.243389in}{1.672898in}}%
\pgfpathcurveto{\pgfqpoint{2.235152in}{1.672898in}}{\pgfqpoint{2.227252in}{1.669626in}}{\pgfqpoint{2.221429in}{1.663802in}}%
\pgfpathcurveto{\pgfqpoint{2.215605in}{1.657978in}}{\pgfqpoint{2.212332in}{1.650078in}}{\pgfqpoint{2.212332in}{1.641842in}}%
\pgfpathcurveto{\pgfqpoint{2.212332in}{1.633606in}}{\pgfqpoint{2.215605in}{1.625706in}}{\pgfqpoint{2.221429in}{1.619882in}}%
\pgfpathcurveto{\pgfqpoint{2.227252in}{1.614058in}}{\pgfqpoint{2.235152in}{1.610785in}}{\pgfqpoint{2.243389in}{1.610785in}}%
\pgfpathclose%
\pgfusepath{stroke,fill}%
\end{pgfscope}%
\begin{pgfscope}%
\pgfpathrectangle{\pgfqpoint{0.100000in}{0.212622in}}{\pgfqpoint{3.696000in}{3.696000in}}%
\pgfusepath{clip}%
\pgfsetbuttcap%
\pgfsetroundjoin%
\definecolor{currentfill}{rgb}{0.121569,0.466667,0.705882}%
\pgfsetfillcolor{currentfill}%
\pgfsetfillopacity{0.785360}%
\pgfsetlinewidth{1.003750pt}%
\definecolor{currentstroke}{rgb}{0.121569,0.466667,0.705882}%
\pgfsetstrokecolor{currentstroke}%
\pgfsetstrokeopacity{0.785360}%
\pgfsetdash{}{0pt}%
\pgfpathmoveto{\pgfqpoint{2.243929in}{1.609393in}}%
\pgfpathcurveto{\pgfqpoint{2.252165in}{1.609393in}}{\pgfqpoint{2.260065in}{1.612665in}}{\pgfqpoint{2.265889in}{1.618489in}}%
\pgfpathcurveto{\pgfqpoint{2.271713in}{1.624313in}}{\pgfqpoint{2.274986in}{1.632213in}}{\pgfqpoint{2.274986in}{1.640449in}}%
\pgfpathcurveto{\pgfqpoint{2.274986in}{1.648685in}}{\pgfqpoint{2.271713in}{1.656585in}}{\pgfqpoint{2.265889in}{1.662409in}}%
\pgfpathcurveto{\pgfqpoint{2.260065in}{1.668233in}}{\pgfqpoint{2.252165in}{1.671506in}}{\pgfqpoint{2.243929in}{1.671506in}}%
\pgfpathcurveto{\pgfqpoint{2.235693in}{1.671506in}}{\pgfqpoint{2.227793in}{1.668233in}}{\pgfqpoint{2.221969in}{1.662409in}}%
\pgfpathcurveto{\pgfqpoint{2.216145in}{1.656585in}}{\pgfqpoint{2.212873in}{1.648685in}}{\pgfqpoint{2.212873in}{1.640449in}}%
\pgfpathcurveto{\pgfqpoint{2.212873in}{1.632213in}}{\pgfqpoint{2.216145in}{1.624313in}}{\pgfqpoint{2.221969in}{1.618489in}}%
\pgfpathcurveto{\pgfqpoint{2.227793in}{1.612665in}}{\pgfqpoint{2.235693in}{1.609393in}}{\pgfqpoint{2.243929in}{1.609393in}}%
\pgfpathclose%
\pgfusepath{stroke,fill}%
\end{pgfscope}%
\begin{pgfscope}%
\pgfpathrectangle{\pgfqpoint{0.100000in}{0.212622in}}{\pgfqpoint{3.696000in}{3.696000in}}%
\pgfusepath{clip}%
\pgfsetbuttcap%
\pgfsetroundjoin%
\definecolor{currentfill}{rgb}{0.121569,0.466667,0.705882}%
\pgfsetfillcolor{currentfill}%
\pgfsetfillopacity{0.786130}%
\pgfsetlinewidth{1.003750pt}%
\definecolor{currentstroke}{rgb}{0.121569,0.466667,0.705882}%
\pgfsetstrokecolor{currentstroke}%
\pgfsetstrokeopacity{0.786130}%
\pgfsetdash{}{0pt}%
\pgfpathmoveto{\pgfqpoint{2.244728in}{1.608237in}}%
\pgfpathcurveto{\pgfqpoint{2.252964in}{1.608237in}}{\pgfqpoint{2.260864in}{1.611509in}}{\pgfqpoint{2.266688in}{1.617333in}}%
\pgfpathcurveto{\pgfqpoint{2.272512in}{1.623157in}}{\pgfqpoint{2.275785in}{1.631057in}}{\pgfqpoint{2.275785in}{1.639293in}}%
\pgfpathcurveto{\pgfqpoint{2.275785in}{1.647530in}}{\pgfqpoint{2.272512in}{1.655430in}}{\pgfqpoint{2.266688in}{1.661254in}}%
\pgfpathcurveto{\pgfqpoint{2.260864in}{1.667078in}}{\pgfqpoint{2.252964in}{1.670350in}}{\pgfqpoint{2.244728in}{1.670350in}}%
\pgfpathcurveto{\pgfqpoint{2.236492in}{1.670350in}}{\pgfqpoint{2.228592in}{1.667078in}}{\pgfqpoint{2.222768in}{1.661254in}}%
\pgfpathcurveto{\pgfqpoint{2.216944in}{1.655430in}}{\pgfqpoint{2.213672in}{1.647530in}}{\pgfqpoint{2.213672in}{1.639293in}}%
\pgfpathcurveto{\pgfqpoint{2.213672in}{1.631057in}}{\pgfqpoint{2.216944in}{1.623157in}}{\pgfqpoint{2.222768in}{1.617333in}}%
\pgfpathcurveto{\pgfqpoint{2.228592in}{1.611509in}}{\pgfqpoint{2.236492in}{1.608237in}}{\pgfqpoint{2.244728in}{1.608237in}}%
\pgfpathclose%
\pgfusepath{stroke,fill}%
\end{pgfscope}%
\begin{pgfscope}%
\pgfpathrectangle{\pgfqpoint{0.100000in}{0.212622in}}{\pgfqpoint{3.696000in}{3.696000in}}%
\pgfusepath{clip}%
\pgfsetbuttcap%
\pgfsetroundjoin%
\definecolor{currentfill}{rgb}{0.121569,0.466667,0.705882}%
\pgfsetfillcolor{currentfill}%
\pgfsetfillopacity{0.786642}%
\pgfsetlinewidth{1.003750pt}%
\definecolor{currentstroke}{rgb}{0.121569,0.466667,0.705882}%
\pgfsetstrokecolor{currentstroke}%
\pgfsetstrokeopacity{0.786642}%
\pgfsetdash{}{0pt}%
\pgfpathmoveto{\pgfqpoint{2.245082in}{1.608095in}}%
\pgfpathcurveto{\pgfqpoint{2.253318in}{1.608095in}}{\pgfqpoint{2.261218in}{1.611368in}}{\pgfqpoint{2.267042in}{1.617191in}}%
\pgfpathcurveto{\pgfqpoint{2.272866in}{1.623015in}}{\pgfqpoint{2.276138in}{1.630915in}}{\pgfqpoint{2.276138in}{1.639152in}}%
\pgfpathcurveto{\pgfqpoint{2.276138in}{1.647388in}}{\pgfqpoint{2.272866in}{1.655288in}}{\pgfqpoint{2.267042in}{1.661112in}}%
\pgfpathcurveto{\pgfqpoint{2.261218in}{1.666936in}}{\pgfqpoint{2.253318in}{1.670208in}}{\pgfqpoint{2.245082in}{1.670208in}}%
\pgfpathcurveto{\pgfqpoint{2.236846in}{1.670208in}}{\pgfqpoint{2.228946in}{1.666936in}}{\pgfqpoint{2.223122in}{1.661112in}}%
\pgfpathcurveto{\pgfqpoint{2.217298in}{1.655288in}}{\pgfqpoint{2.214025in}{1.647388in}}{\pgfqpoint{2.214025in}{1.639152in}}%
\pgfpathcurveto{\pgfqpoint{2.214025in}{1.630915in}}{\pgfqpoint{2.217298in}{1.623015in}}{\pgfqpoint{2.223122in}{1.617191in}}%
\pgfpathcurveto{\pgfqpoint{2.228946in}{1.611368in}}{\pgfqpoint{2.236846in}{1.608095in}}{\pgfqpoint{2.245082in}{1.608095in}}%
\pgfpathclose%
\pgfusepath{stroke,fill}%
\end{pgfscope}%
\begin{pgfscope}%
\pgfpathrectangle{\pgfqpoint{0.100000in}{0.212622in}}{\pgfqpoint{3.696000in}{3.696000in}}%
\pgfusepath{clip}%
\pgfsetbuttcap%
\pgfsetroundjoin%
\definecolor{currentfill}{rgb}{0.121569,0.466667,0.705882}%
\pgfsetfillcolor{currentfill}%
\pgfsetfillopacity{0.787404}%
\pgfsetlinewidth{1.003750pt}%
\definecolor{currentstroke}{rgb}{0.121569,0.466667,0.705882}%
\pgfsetstrokecolor{currentstroke}%
\pgfsetstrokeopacity{0.787404}%
\pgfsetdash{}{0pt}%
\pgfpathmoveto{\pgfqpoint{2.245697in}{1.607249in}}%
\pgfpathcurveto{\pgfqpoint{2.253934in}{1.607249in}}{\pgfqpoint{2.261834in}{1.610522in}}{\pgfqpoint{2.267657in}{1.616345in}}%
\pgfpathcurveto{\pgfqpoint{2.273481in}{1.622169in}}{\pgfqpoint{2.276754in}{1.630069in}}{\pgfqpoint{2.276754in}{1.638306in}}%
\pgfpathcurveto{\pgfqpoint{2.276754in}{1.646542in}}{\pgfqpoint{2.273481in}{1.654442in}}{\pgfqpoint{2.267657in}{1.660266in}}%
\pgfpathcurveto{\pgfqpoint{2.261834in}{1.666090in}}{\pgfqpoint{2.253934in}{1.669362in}}{\pgfqpoint{2.245697in}{1.669362in}}%
\pgfpathcurveto{\pgfqpoint{2.237461in}{1.669362in}}{\pgfqpoint{2.229561in}{1.666090in}}{\pgfqpoint{2.223737in}{1.660266in}}%
\pgfpathcurveto{\pgfqpoint{2.217913in}{1.654442in}}{\pgfqpoint{2.214641in}{1.646542in}}{\pgfqpoint{2.214641in}{1.638306in}}%
\pgfpathcurveto{\pgfqpoint{2.214641in}{1.630069in}}{\pgfqpoint{2.217913in}{1.622169in}}{\pgfqpoint{2.223737in}{1.616345in}}%
\pgfpathcurveto{\pgfqpoint{2.229561in}{1.610522in}}{\pgfqpoint{2.237461in}{1.607249in}}{\pgfqpoint{2.245697in}{1.607249in}}%
\pgfpathclose%
\pgfusepath{stroke,fill}%
\end{pgfscope}%
\begin{pgfscope}%
\pgfpathrectangle{\pgfqpoint{0.100000in}{0.212622in}}{\pgfqpoint{3.696000in}{3.696000in}}%
\pgfusepath{clip}%
\pgfsetbuttcap%
\pgfsetroundjoin%
\definecolor{currentfill}{rgb}{0.121569,0.466667,0.705882}%
\pgfsetfillcolor{currentfill}%
\pgfsetfillopacity{0.788284}%
\pgfsetlinewidth{1.003750pt}%
\definecolor{currentstroke}{rgb}{0.121569,0.466667,0.705882}%
\pgfsetstrokecolor{currentstroke}%
\pgfsetstrokeopacity{0.788284}%
\pgfsetdash{}{0pt}%
\pgfpathmoveto{\pgfqpoint{2.246474in}{1.606040in}}%
\pgfpathcurveto{\pgfqpoint{2.254710in}{1.606040in}}{\pgfqpoint{2.262610in}{1.609313in}}{\pgfqpoint{2.268434in}{1.615136in}}%
\pgfpathcurveto{\pgfqpoint{2.274258in}{1.620960in}}{\pgfqpoint{2.277530in}{1.628860in}}{\pgfqpoint{2.277530in}{1.637097in}}%
\pgfpathcurveto{\pgfqpoint{2.277530in}{1.645333in}}{\pgfqpoint{2.274258in}{1.653233in}}{\pgfqpoint{2.268434in}{1.659057in}}%
\pgfpathcurveto{\pgfqpoint{2.262610in}{1.664881in}}{\pgfqpoint{2.254710in}{1.668153in}}{\pgfqpoint{2.246474in}{1.668153in}}%
\pgfpathcurveto{\pgfqpoint{2.238238in}{1.668153in}}{\pgfqpoint{2.230337in}{1.664881in}}{\pgfqpoint{2.224514in}{1.659057in}}%
\pgfpathcurveto{\pgfqpoint{2.218690in}{1.653233in}}{\pgfqpoint{2.215417in}{1.645333in}}{\pgfqpoint{2.215417in}{1.637097in}}%
\pgfpathcurveto{\pgfqpoint{2.215417in}{1.628860in}}{\pgfqpoint{2.218690in}{1.620960in}}{\pgfqpoint{2.224514in}{1.615136in}}%
\pgfpathcurveto{\pgfqpoint{2.230337in}{1.609313in}}{\pgfqpoint{2.238238in}{1.606040in}}{\pgfqpoint{2.246474in}{1.606040in}}%
\pgfpathclose%
\pgfusepath{stroke,fill}%
\end{pgfscope}%
\begin{pgfscope}%
\pgfpathrectangle{\pgfqpoint{0.100000in}{0.212622in}}{\pgfqpoint{3.696000in}{3.696000in}}%
\pgfusepath{clip}%
\pgfsetbuttcap%
\pgfsetroundjoin%
\definecolor{currentfill}{rgb}{0.121569,0.466667,0.705882}%
\pgfsetfillcolor{currentfill}%
\pgfsetfillopacity{0.788647}%
\pgfsetlinewidth{1.003750pt}%
\definecolor{currentstroke}{rgb}{0.121569,0.466667,0.705882}%
\pgfsetstrokecolor{currentstroke}%
\pgfsetstrokeopacity{0.788647}%
\pgfsetdash{}{0pt}%
\pgfpathmoveto{\pgfqpoint{2.247000in}{1.604686in}}%
\pgfpathcurveto{\pgfqpoint{2.255236in}{1.604686in}}{\pgfqpoint{2.263136in}{1.607958in}}{\pgfqpoint{2.268960in}{1.613782in}}%
\pgfpathcurveto{\pgfqpoint{2.274784in}{1.619606in}}{\pgfqpoint{2.278056in}{1.627506in}}{\pgfqpoint{2.278056in}{1.635742in}}%
\pgfpathcurveto{\pgfqpoint{2.278056in}{1.643978in}}{\pgfqpoint{2.274784in}{1.651878in}}{\pgfqpoint{2.268960in}{1.657702in}}%
\pgfpathcurveto{\pgfqpoint{2.263136in}{1.663526in}}{\pgfqpoint{2.255236in}{1.666799in}}{\pgfqpoint{2.247000in}{1.666799in}}%
\pgfpathcurveto{\pgfqpoint{2.238763in}{1.666799in}}{\pgfqpoint{2.230863in}{1.663526in}}{\pgfqpoint{2.225040in}{1.657702in}}%
\pgfpathcurveto{\pgfqpoint{2.219216in}{1.651878in}}{\pgfqpoint{2.215943in}{1.643978in}}{\pgfqpoint{2.215943in}{1.635742in}}%
\pgfpathcurveto{\pgfqpoint{2.215943in}{1.627506in}}{\pgfqpoint{2.219216in}{1.619606in}}{\pgfqpoint{2.225040in}{1.613782in}}%
\pgfpathcurveto{\pgfqpoint{2.230863in}{1.607958in}}{\pgfqpoint{2.238763in}{1.604686in}}{\pgfqpoint{2.247000in}{1.604686in}}%
\pgfpathclose%
\pgfusepath{stroke,fill}%
\end{pgfscope}%
\begin{pgfscope}%
\pgfpathrectangle{\pgfqpoint{0.100000in}{0.212622in}}{\pgfqpoint{3.696000in}{3.696000in}}%
\pgfusepath{clip}%
\pgfsetbuttcap%
\pgfsetroundjoin%
\definecolor{currentfill}{rgb}{0.121569,0.466667,0.705882}%
\pgfsetfillcolor{currentfill}%
\pgfsetfillopacity{0.789365}%
\pgfsetlinewidth{1.003750pt}%
\definecolor{currentstroke}{rgb}{0.121569,0.466667,0.705882}%
\pgfsetstrokecolor{currentstroke}%
\pgfsetstrokeopacity{0.789365}%
\pgfsetdash{}{0pt}%
\pgfpathmoveto{\pgfqpoint{2.247671in}{1.603922in}}%
\pgfpathcurveto{\pgfqpoint{2.255907in}{1.603922in}}{\pgfqpoint{2.263807in}{1.607194in}}{\pgfqpoint{2.269631in}{1.613018in}}%
\pgfpathcurveto{\pgfqpoint{2.275455in}{1.618842in}}{\pgfqpoint{2.278727in}{1.626742in}}{\pgfqpoint{2.278727in}{1.634978in}}%
\pgfpathcurveto{\pgfqpoint{2.278727in}{1.643215in}}{\pgfqpoint{2.275455in}{1.651115in}}{\pgfqpoint{2.269631in}{1.656939in}}%
\pgfpathcurveto{\pgfqpoint{2.263807in}{1.662763in}}{\pgfqpoint{2.255907in}{1.666035in}}{\pgfqpoint{2.247671in}{1.666035in}}%
\pgfpathcurveto{\pgfqpoint{2.239435in}{1.666035in}}{\pgfqpoint{2.231534in}{1.662763in}}{\pgfqpoint{2.225711in}{1.656939in}}%
\pgfpathcurveto{\pgfqpoint{2.219887in}{1.651115in}}{\pgfqpoint{2.216614in}{1.643215in}}{\pgfqpoint{2.216614in}{1.634978in}}%
\pgfpathcurveto{\pgfqpoint{2.216614in}{1.626742in}}{\pgfqpoint{2.219887in}{1.618842in}}{\pgfqpoint{2.225711in}{1.613018in}}%
\pgfpathcurveto{\pgfqpoint{2.231534in}{1.607194in}}{\pgfqpoint{2.239435in}{1.603922in}}{\pgfqpoint{2.247671in}{1.603922in}}%
\pgfpathclose%
\pgfusepath{stroke,fill}%
\end{pgfscope}%
\begin{pgfscope}%
\pgfpathrectangle{\pgfqpoint{0.100000in}{0.212622in}}{\pgfqpoint{3.696000in}{3.696000in}}%
\pgfusepath{clip}%
\pgfsetbuttcap%
\pgfsetroundjoin%
\definecolor{currentfill}{rgb}{0.121569,0.466667,0.705882}%
\pgfsetfillcolor{currentfill}%
\pgfsetfillopacity{0.790709}%
\pgfsetlinewidth{1.003750pt}%
\definecolor{currentstroke}{rgb}{0.121569,0.466667,0.705882}%
\pgfsetstrokecolor{currentstroke}%
\pgfsetstrokeopacity{0.790709}%
\pgfsetdash{}{0pt}%
\pgfpathmoveto{\pgfqpoint{2.248318in}{1.603624in}}%
\pgfpathcurveto{\pgfqpoint{2.256554in}{1.603624in}}{\pgfqpoint{2.264454in}{1.606896in}}{\pgfqpoint{2.270278in}{1.612720in}}%
\pgfpathcurveto{\pgfqpoint{2.276102in}{1.618544in}}{\pgfqpoint{2.279374in}{1.626444in}}{\pgfqpoint{2.279374in}{1.634680in}}%
\pgfpathcurveto{\pgfqpoint{2.279374in}{1.642916in}}{\pgfqpoint{2.276102in}{1.650816in}}{\pgfqpoint{2.270278in}{1.656640in}}%
\pgfpathcurveto{\pgfqpoint{2.264454in}{1.662464in}}{\pgfqpoint{2.256554in}{1.665737in}}{\pgfqpoint{2.248318in}{1.665737in}}%
\pgfpathcurveto{\pgfqpoint{2.240081in}{1.665737in}}{\pgfqpoint{2.232181in}{1.662464in}}{\pgfqpoint{2.226357in}{1.656640in}}%
\pgfpathcurveto{\pgfqpoint{2.220533in}{1.650816in}}{\pgfqpoint{2.217261in}{1.642916in}}{\pgfqpoint{2.217261in}{1.634680in}}%
\pgfpathcurveto{\pgfqpoint{2.217261in}{1.626444in}}{\pgfqpoint{2.220533in}{1.618544in}}{\pgfqpoint{2.226357in}{1.612720in}}%
\pgfpathcurveto{\pgfqpoint{2.232181in}{1.606896in}}{\pgfqpoint{2.240081in}{1.603624in}}{\pgfqpoint{2.248318in}{1.603624in}}%
\pgfpathclose%
\pgfusepath{stroke,fill}%
\end{pgfscope}%
\begin{pgfscope}%
\pgfpathrectangle{\pgfqpoint{0.100000in}{0.212622in}}{\pgfqpoint{3.696000in}{3.696000in}}%
\pgfusepath{clip}%
\pgfsetbuttcap%
\pgfsetroundjoin%
\definecolor{currentfill}{rgb}{0.121569,0.466667,0.705882}%
\pgfsetfillcolor{currentfill}%
\pgfsetfillopacity{0.792626}%
\pgfsetlinewidth{1.003750pt}%
\definecolor{currentstroke}{rgb}{0.121569,0.466667,0.705882}%
\pgfsetstrokecolor{currentstroke}%
\pgfsetstrokeopacity{0.792626}%
\pgfsetdash{}{0pt}%
\pgfpathmoveto{\pgfqpoint{2.249716in}{1.601333in}}%
\pgfpathcurveto{\pgfqpoint{2.257952in}{1.601333in}}{\pgfqpoint{2.265852in}{1.604605in}}{\pgfqpoint{2.271676in}{1.610429in}}%
\pgfpathcurveto{\pgfqpoint{2.277500in}{1.616253in}}{\pgfqpoint{2.280772in}{1.624153in}}{\pgfqpoint{2.280772in}{1.632389in}}%
\pgfpathcurveto{\pgfqpoint{2.280772in}{1.640626in}}{\pgfqpoint{2.277500in}{1.648526in}}{\pgfqpoint{2.271676in}{1.654350in}}%
\pgfpathcurveto{\pgfqpoint{2.265852in}{1.660174in}}{\pgfqpoint{2.257952in}{1.663446in}}{\pgfqpoint{2.249716in}{1.663446in}}%
\pgfpathcurveto{\pgfqpoint{2.241479in}{1.663446in}}{\pgfqpoint{2.233579in}{1.660174in}}{\pgfqpoint{2.227755in}{1.654350in}}%
\pgfpathcurveto{\pgfqpoint{2.221932in}{1.648526in}}{\pgfqpoint{2.218659in}{1.640626in}}{\pgfqpoint{2.218659in}{1.632389in}}%
\pgfpathcurveto{\pgfqpoint{2.218659in}{1.624153in}}{\pgfqpoint{2.221932in}{1.616253in}}{\pgfqpoint{2.227755in}{1.610429in}}%
\pgfpathcurveto{\pgfqpoint{2.233579in}{1.604605in}}{\pgfqpoint{2.241479in}{1.601333in}}{\pgfqpoint{2.249716in}{1.601333in}}%
\pgfpathclose%
\pgfusepath{stroke,fill}%
\end{pgfscope}%
\begin{pgfscope}%
\pgfpathrectangle{\pgfqpoint{0.100000in}{0.212622in}}{\pgfqpoint{3.696000in}{3.696000in}}%
\pgfusepath{clip}%
\pgfsetbuttcap%
\pgfsetroundjoin%
\definecolor{currentfill}{rgb}{0.121569,0.466667,0.705882}%
\pgfsetfillcolor{currentfill}%
\pgfsetfillopacity{0.793575}%
\pgfsetlinewidth{1.003750pt}%
\definecolor{currentstroke}{rgb}{0.121569,0.466667,0.705882}%
\pgfsetstrokecolor{currentstroke}%
\pgfsetstrokeopacity{0.793575}%
\pgfsetdash{}{0pt}%
\pgfpathmoveto{\pgfqpoint{2.250360in}{1.599347in}}%
\pgfpathcurveto{\pgfqpoint{2.258596in}{1.599347in}}{\pgfqpoint{2.266496in}{1.602619in}}{\pgfqpoint{2.272320in}{1.608443in}}%
\pgfpathcurveto{\pgfqpoint{2.278144in}{1.614267in}}{\pgfqpoint{2.281416in}{1.622167in}}{\pgfqpoint{2.281416in}{1.630403in}}%
\pgfpathcurveto{\pgfqpoint{2.281416in}{1.638639in}}{\pgfqpoint{2.278144in}{1.646540in}}{\pgfqpoint{2.272320in}{1.652363in}}%
\pgfpathcurveto{\pgfqpoint{2.266496in}{1.658187in}}{\pgfqpoint{2.258596in}{1.661460in}}{\pgfqpoint{2.250360in}{1.661460in}}%
\pgfpathcurveto{\pgfqpoint{2.242124in}{1.661460in}}{\pgfqpoint{2.234224in}{1.658187in}}{\pgfqpoint{2.228400in}{1.652363in}}%
\pgfpathcurveto{\pgfqpoint{2.222576in}{1.646540in}}{\pgfqpoint{2.219303in}{1.638639in}}{\pgfqpoint{2.219303in}{1.630403in}}%
\pgfpathcurveto{\pgfqpoint{2.219303in}{1.622167in}}{\pgfqpoint{2.222576in}{1.614267in}}{\pgfqpoint{2.228400in}{1.608443in}}%
\pgfpathcurveto{\pgfqpoint{2.234224in}{1.602619in}}{\pgfqpoint{2.242124in}{1.599347in}}{\pgfqpoint{2.250360in}{1.599347in}}%
\pgfpathclose%
\pgfusepath{stroke,fill}%
\end{pgfscope}%
\begin{pgfscope}%
\pgfpathrectangle{\pgfqpoint{0.100000in}{0.212622in}}{\pgfqpoint{3.696000in}{3.696000in}}%
\pgfusepath{clip}%
\pgfsetbuttcap%
\pgfsetroundjoin%
\definecolor{currentfill}{rgb}{0.121569,0.466667,0.705882}%
\pgfsetfillcolor{currentfill}%
\pgfsetfillopacity{0.795046}%
\pgfsetlinewidth{1.003750pt}%
\definecolor{currentstroke}{rgb}{0.121569,0.466667,0.705882}%
\pgfsetstrokecolor{currentstroke}%
\pgfsetstrokeopacity{0.795046}%
\pgfsetdash{}{0pt}%
\pgfpathmoveto{\pgfqpoint{2.251534in}{1.598331in}}%
\pgfpathcurveto{\pgfqpoint{2.259771in}{1.598331in}}{\pgfqpoint{2.267671in}{1.601603in}}{\pgfqpoint{2.273495in}{1.607427in}}%
\pgfpathcurveto{\pgfqpoint{2.279318in}{1.613251in}}{\pgfqpoint{2.282591in}{1.621151in}}{\pgfqpoint{2.282591in}{1.629388in}}%
\pgfpathcurveto{\pgfqpoint{2.282591in}{1.637624in}}{\pgfqpoint{2.279318in}{1.645524in}}{\pgfqpoint{2.273495in}{1.651348in}}%
\pgfpathcurveto{\pgfqpoint{2.267671in}{1.657172in}}{\pgfqpoint{2.259771in}{1.660444in}}{\pgfqpoint{2.251534in}{1.660444in}}%
\pgfpathcurveto{\pgfqpoint{2.243298in}{1.660444in}}{\pgfqpoint{2.235398in}{1.657172in}}{\pgfqpoint{2.229574in}{1.651348in}}%
\pgfpathcurveto{\pgfqpoint{2.223750in}{1.645524in}}{\pgfqpoint{2.220478in}{1.637624in}}{\pgfqpoint{2.220478in}{1.629388in}}%
\pgfpathcurveto{\pgfqpoint{2.220478in}{1.621151in}}{\pgfqpoint{2.223750in}{1.613251in}}{\pgfqpoint{2.229574in}{1.607427in}}%
\pgfpathcurveto{\pgfqpoint{2.235398in}{1.601603in}}{\pgfqpoint{2.243298in}{1.598331in}}{\pgfqpoint{2.251534in}{1.598331in}}%
\pgfpathclose%
\pgfusepath{stroke,fill}%
\end{pgfscope}%
\begin{pgfscope}%
\pgfpathrectangle{\pgfqpoint{0.100000in}{0.212622in}}{\pgfqpoint{3.696000in}{3.696000in}}%
\pgfusepath{clip}%
\pgfsetbuttcap%
\pgfsetroundjoin%
\definecolor{currentfill}{rgb}{0.121569,0.466667,0.705882}%
\pgfsetfillcolor{currentfill}%
\pgfsetfillopacity{0.796872}%
\pgfsetlinewidth{1.003750pt}%
\definecolor{currentstroke}{rgb}{0.121569,0.466667,0.705882}%
\pgfsetstrokecolor{currentstroke}%
\pgfsetstrokeopacity{0.796872}%
\pgfsetdash{}{0pt}%
\pgfpathmoveto{\pgfqpoint{2.252661in}{1.597634in}}%
\pgfpathcurveto{\pgfqpoint{2.260897in}{1.597634in}}{\pgfqpoint{2.268797in}{1.600907in}}{\pgfqpoint{2.274621in}{1.606730in}}%
\pgfpathcurveto{\pgfqpoint{2.280445in}{1.612554in}}{\pgfqpoint{2.283717in}{1.620454in}}{\pgfqpoint{2.283717in}{1.628691in}}%
\pgfpathcurveto{\pgfqpoint{2.283717in}{1.636927in}}{\pgfqpoint{2.280445in}{1.644827in}}{\pgfqpoint{2.274621in}{1.650651in}}%
\pgfpathcurveto{\pgfqpoint{2.268797in}{1.656475in}}{\pgfqpoint{2.260897in}{1.659747in}}{\pgfqpoint{2.252661in}{1.659747in}}%
\pgfpathcurveto{\pgfqpoint{2.244425in}{1.659747in}}{\pgfqpoint{2.236525in}{1.656475in}}{\pgfqpoint{2.230701in}{1.650651in}}%
\pgfpathcurveto{\pgfqpoint{2.224877in}{1.644827in}}{\pgfqpoint{2.221604in}{1.636927in}}{\pgfqpoint{2.221604in}{1.628691in}}%
\pgfpathcurveto{\pgfqpoint{2.221604in}{1.620454in}}{\pgfqpoint{2.224877in}{1.612554in}}{\pgfqpoint{2.230701in}{1.606730in}}%
\pgfpathcurveto{\pgfqpoint{2.236525in}{1.600907in}}{\pgfqpoint{2.244425in}{1.597634in}}{\pgfqpoint{2.252661in}{1.597634in}}%
\pgfpathclose%
\pgfusepath{stroke,fill}%
\end{pgfscope}%
\begin{pgfscope}%
\pgfpathrectangle{\pgfqpoint{0.100000in}{0.212622in}}{\pgfqpoint{3.696000in}{3.696000in}}%
\pgfusepath{clip}%
\pgfsetbuttcap%
\pgfsetroundjoin%
\definecolor{currentfill}{rgb}{0.121569,0.466667,0.705882}%
\pgfsetfillcolor{currentfill}%
\pgfsetfillopacity{0.798901}%
\pgfsetlinewidth{1.003750pt}%
\definecolor{currentstroke}{rgb}{0.121569,0.466667,0.705882}%
\pgfsetstrokecolor{currentstroke}%
\pgfsetstrokeopacity{0.798901}%
\pgfsetdash{}{0pt}%
\pgfpathmoveto{\pgfqpoint{2.254766in}{1.593755in}}%
\pgfpathcurveto{\pgfqpoint{2.263002in}{1.593755in}}{\pgfqpoint{2.270902in}{1.597028in}}{\pgfqpoint{2.276726in}{1.602852in}}%
\pgfpathcurveto{\pgfqpoint{2.282550in}{1.608676in}}{\pgfqpoint{2.285822in}{1.616576in}}{\pgfqpoint{2.285822in}{1.624812in}}%
\pgfpathcurveto{\pgfqpoint{2.285822in}{1.633048in}}{\pgfqpoint{2.282550in}{1.640948in}}{\pgfqpoint{2.276726in}{1.646772in}}%
\pgfpathcurveto{\pgfqpoint{2.270902in}{1.652596in}}{\pgfqpoint{2.263002in}{1.655868in}}{\pgfqpoint{2.254766in}{1.655868in}}%
\pgfpathcurveto{\pgfqpoint{2.246529in}{1.655868in}}{\pgfqpoint{2.238629in}{1.652596in}}{\pgfqpoint{2.232805in}{1.646772in}}%
\pgfpathcurveto{\pgfqpoint{2.226982in}{1.640948in}}{\pgfqpoint{2.223709in}{1.633048in}}{\pgfqpoint{2.223709in}{1.624812in}}%
\pgfpathcurveto{\pgfqpoint{2.223709in}{1.616576in}}{\pgfqpoint{2.226982in}{1.608676in}}{\pgfqpoint{2.232805in}{1.602852in}}%
\pgfpathcurveto{\pgfqpoint{2.238629in}{1.597028in}}{\pgfqpoint{2.246529in}{1.593755in}}{\pgfqpoint{2.254766in}{1.593755in}}%
\pgfpathclose%
\pgfusepath{stroke,fill}%
\end{pgfscope}%
\begin{pgfscope}%
\pgfpathrectangle{\pgfqpoint{0.100000in}{0.212622in}}{\pgfqpoint{3.696000in}{3.696000in}}%
\pgfusepath{clip}%
\pgfsetbuttcap%
\pgfsetroundjoin%
\definecolor{currentfill}{rgb}{0.121569,0.466667,0.705882}%
\pgfsetfillcolor{currentfill}%
\pgfsetfillopacity{0.800779}%
\pgfsetlinewidth{1.003750pt}%
\definecolor{currentstroke}{rgb}{0.121569,0.466667,0.705882}%
\pgfsetstrokecolor{currentstroke}%
\pgfsetstrokeopacity{0.800779}%
\pgfsetdash{}{0pt}%
\pgfpathmoveto{\pgfqpoint{2.257206in}{1.586747in}}%
\pgfpathcurveto{\pgfqpoint{2.265442in}{1.586747in}}{\pgfqpoint{2.273342in}{1.590020in}}{\pgfqpoint{2.279166in}{1.595844in}}%
\pgfpathcurveto{\pgfqpoint{2.284990in}{1.601667in}}{\pgfqpoint{2.288262in}{1.609568in}}{\pgfqpoint{2.288262in}{1.617804in}}%
\pgfpathcurveto{\pgfqpoint{2.288262in}{1.626040in}}{\pgfqpoint{2.284990in}{1.633940in}}{\pgfqpoint{2.279166in}{1.639764in}}%
\pgfpathcurveto{\pgfqpoint{2.273342in}{1.645588in}}{\pgfqpoint{2.265442in}{1.648860in}}{\pgfqpoint{2.257206in}{1.648860in}}%
\pgfpathcurveto{\pgfqpoint{2.248969in}{1.648860in}}{\pgfqpoint{2.241069in}{1.645588in}}{\pgfqpoint{2.235245in}{1.639764in}}%
\pgfpathcurveto{\pgfqpoint{2.229422in}{1.633940in}}{\pgfqpoint{2.226149in}{1.626040in}}{\pgfqpoint{2.226149in}{1.617804in}}%
\pgfpathcurveto{\pgfqpoint{2.226149in}{1.609568in}}{\pgfqpoint{2.229422in}{1.601667in}}{\pgfqpoint{2.235245in}{1.595844in}}%
\pgfpathcurveto{\pgfqpoint{2.241069in}{1.590020in}}{\pgfqpoint{2.248969in}{1.586747in}}{\pgfqpoint{2.257206in}{1.586747in}}%
\pgfpathclose%
\pgfusepath{stroke,fill}%
\end{pgfscope}%
\begin{pgfscope}%
\pgfpathrectangle{\pgfqpoint{0.100000in}{0.212622in}}{\pgfqpoint{3.696000in}{3.696000in}}%
\pgfusepath{clip}%
\pgfsetbuttcap%
\pgfsetroundjoin%
\definecolor{currentfill}{rgb}{0.121569,0.466667,0.705882}%
\pgfsetfillcolor{currentfill}%
\pgfsetfillopacity{0.803137}%
\pgfsetlinewidth{1.003750pt}%
\definecolor{currentstroke}{rgb}{0.121569,0.466667,0.705882}%
\pgfsetstrokecolor{currentstroke}%
\pgfsetstrokeopacity{0.803137}%
\pgfsetdash{}{0pt}%
\pgfpathmoveto{\pgfqpoint{2.259077in}{1.580863in}}%
\pgfpathcurveto{\pgfqpoint{2.267314in}{1.580863in}}{\pgfqpoint{2.275214in}{1.584136in}}{\pgfqpoint{2.281038in}{1.589960in}}%
\pgfpathcurveto{\pgfqpoint{2.286861in}{1.595784in}}{\pgfqpoint{2.290134in}{1.603684in}}{\pgfqpoint{2.290134in}{1.611920in}}%
\pgfpathcurveto{\pgfqpoint{2.290134in}{1.620156in}}{\pgfqpoint{2.286861in}{1.628056in}}{\pgfqpoint{2.281038in}{1.633880in}}%
\pgfpathcurveto{\pgfqpoint{2.275214in}{1.639704in}}{\pgfqpoint{2.267314in}{1.642976in}}{\pgfqpoint{2.259077in}{1.642976in}}%
\pgfpathcurveto{\pgfqpoint{2.250841in}{1.642976in}}{\pgfqpoint{2.242941in}{1.639704in}}{\pgfqpoint{2.237117in}{1.633880in}}%
\pgfpathcurveto{\pgfqpoint{2.231293in}{1.628056in}}{\pgfqpoint{2.228021in}{1.620156in}}{\pgfqpoint{2.228021in}{1.611920in}}%
\pgfpathcurveto{\pgfqpoint{2.228021in}{1.603684in}}{\pgfqpoint{2.231293in}{1.595784in}}{\pgfqpoint{2.237117in}{1.589960in}}%
\pgfpathcurveto{\pgfqpoint{2.242941in}{1.584136in}}{\pgfqpoint{2.250841in}{1.580863in}}{\pgfqpoint{2.259077in}{1.580863in}}%
\pgfpathclose%
\pgfusepath{stroke,fill}%
\end{pgfscope}%
\begin{pgfscope}%
\pgfpathrectangle{\pgfqpoint{0.100000in}{0.212622in}}{\pgfqpoint{3.696000in}{3.696000in}}%
\pgfusepath{clip}%
\pgfsetbuttcap%
\pgfsetroundjoin%
\definecolor{currentfill}{rgb}{0.121569,0.466667,0.705882}%
\pgfsetfillcolor{currentfill}%
\pgfsetfillopacity{0.804846}%
\pgfsetlinewidth{1.003750pt}%
\definecolor{currentstroke}{rgb}{0.121569,0.466667,0.705882}%
\pgfsetstrokecolor{currentstroke}%
\pgfsetstrokeopacity{0.804846}%
\pgfsetdash{}{0pt}%
\pgfpathmoveto{\pgfqpoint{2.260306in}{1.580295in}}%
\pgfpathcurveto{\pgfqpoint{2.268543in}{1.580295in}}{\pgfqpoint{2.276443in}{1.583568in}}{\pgfqpoint{2.282267in}{1.589392in}}%
\pgfpathcurveto{\pgfqpoint{2.288091in}{1.595216in}}{\pgfqpoint{2.291363in}{1.603116in}}{\pgfqpoint{2.291363in}{1.611352in}}%
\pgfpathcurveto{\pgfqpoint{2.291363in}{1.619588in}}{\pgfqpoint{2.288091in}{1.627488in}}{\pgfqpoint{2.282267in}{1.633312in}}%
\pgfpathcurveto{\pgfqpoint{2.276443in}{1.639136in}}{\pgfqpoint{2.268543in}{1.642408in}}{\pgfqpoint{2.260306in}{1.642408in}}%
\pgfpathcurveto{\pgfqpoint{2.252070in}{1.642408in}}{\pgfqpoint{2.244170in}{1.639136in}}{\pgfqpoint{2.238346in}{1.633312in}}%
\pgfpathcurveto{\pgfqpoint{2.232522in}{1.627488in}}{\pgfqpoint{2.229250in}{1.619588in}}{\pgfqpoint{2.229250in}{1.611352in}}%
\pgfpathcurveto{\pgfqpoint{2.229250in}{1.603116in}}{\pgfqpoint{2.232522in}{1.595216in}}{\pgfqpoint{2.238346in}{1.589392in}}%
\pgfpathcurveto{\pgfqpoint{2.244170in}{1.583568in}}{\pgfqpoint{2.252070in}{1.580295in}}{\pgfqpoint{2.260306in}{1.580295in}}%
\pgfpathclose%
\pgfusepath{stroke,fill}%
\end{pgfscope}%
\begin{pgfscope}%
\pgfpathrectangle{\pgfqpoint{0.100000in}{0.212622in}}{\pgfqpoint{3.696000in}{3.696000in}}%
\pgfusepath{clip}%
\pgfsetbuttcap%
\pgfsetroundjoin%
\definecolor{currentfill}{rgb}{0.121569,0.466667,0.705882}%
\pgfsetfillcolor{currentfill}%
\pgfsetfillopacity{0.806804}%
\pgfsetlinewidth{1.003750pt}%
\definecolor{currentstroke}{rgb}{0.121569,0.466667,0.705882}%
\pgfsetstrokecolor{currentstroke}%
\pgfsetstrokeopacity{0.806804}%
\pgfsetdash{}{0pt}%
\pgfpathmoveto{\pgfqpoint{2.261932in}{1.576971in}}%
\pgfpathcurveto{\pgfqpoint{2.270169in}{1.576971in}}{\pgfqpoint{2.278069in}{1.580244in}}{\pgfqpoint{2.283893in}{1.586067in}}%
\pgfpathcurveto{\pgfqpoint{2.289717in}{1.591891in}}{\pgfqpoint{2.292989in}{1.599791in}}{\pgfqpoint{2.292989in}{1.608028in}}%
\pgfpathcurveto{\pgfqpoint{2.292989in}{1.616264in}}{\pgfqpoint{2.289717in}{1.624164in}}{\pgfqpoint{2.283893in}{1.629988in}}%
\pgfpathcurveto{\pgfqpoint{2.278069in}{1.635812in}}{\pgfqpoint{2.270169in}{1.639084in}}{\pgfqpoint{2.261932in}{1.639084in}}%
\pgfpathcurveto{\pgfqpoint{2.253696in}{1.639084in}}{\pgfqpoint{2.245796in}{1.635812in}}{\pgfqpoint{2.239972in}{1.629988in}}%
\pgfpathcurveto{\pgfqpoint{2.234148in}{1.624164in}}{\pgfqpoint{2.230876in}{1.616264in}}{\pgfqpoint{2.230876in}{1.608028in}}%
\pgfpathcurveto{\pgfqpoint{2.230876in}{1.599791in}}{\pgfqpoint{2.234148in}{1.591891in}}{\pgfqpoint{2.239972in}{1.586067in}}%
\pgfpathcurveto{\pgfqpoint{2.245796in}{1.580244in}}{\pgfqpoint{2.253696in}{1.576971in}}{\pgfqpoint{2.261932in}{1.576971in}}%
\pgfpathclose%
\pgfusepath{stroke,fill}%
\end{pgfscope}%
\begin{pgfscope}%
\pgfpathrectangle{\pgfqpoint{0.100000in}{0.212622in}}{\pgfqpoint{3.696000in}{3.696000in}}%
\pgfusepath{clip}%
\pgfsetbuttcap%
\pgfsetroundjoin%
\definecolor{currentfill}{rgb}{0.121569,0.466667,0.705882}%
\pgfsetfillcolor{currentfill}%
\pgfsetfillopacity{0.808773}%
\pgfsetlinewidth{1.003750pt}%
\definecolor{currentstroke}{rgb}{0.121569,0.466667,0.705882}%
\pgfsetstrokecolor{currentstroke}%
\pgfsetstrokeopacity{0.808773}%
\pgfsetdash{}{0pt}%
\pgfpathmoveto{\pgfqpoint{2.264102in}{1.572549in}}%
\pgfpathcurveto{\pgfqpoint{2.272339in}{1.572549in}}{\pgfqpoint{2.280239in}{1.575821in}}{\pgfqpoint{2.286063in}{1.581645in}}%
\pgfpathcurveto{\pgfqpoint{2.291887in}{1.587469in}}{\pgfqpoint{2.295159in}{1.595369in}}{\pgfqpoint{2.295159in}{1.603605in}}%
\pgfpathcurveto{\pgfqpoint{2.295159in}{1.611841in}}{\pgfqpoint{2.291887in}{1.619741in}}{\pgfqpoint{2.286063in}{1.625565in}}%
\pgfpathcurveto{\pgfqpoint{2.280239in}{1.631389in}}{\pgfqpoint{2.272339in}{1.634662in}}{\pgfqpoint{2.264102in}{1.634662in}}%
\pgfpathcurveto{\pgfqpoint{2.255866in}{1.634662in}}{\pgfqpoint{2.247966in}{1.631389in}}{\pgfqpoint{2.242142in}{1.625565in}}%
\pgfpathcurveto{\pgfqpoint{2.236318in}{1.619741in}}{\pgfqpoint{2.233046in}{1.611841in}}{\pgfqpoint{2.233046in}{1.603605in}}%
\pgfpathcurveto{\pgfqpoint{2.233046in}{1.595369in}}{\pgfqpoint{2.236318in}{1.587469in}}{\pgfqpoint{2.242142in}{1.581645in}}%
\pgfpathcurveto{\pgfqpoint{2.247966in}{1.575821in}}{\pgfqpoint{2.255866in}{1.572549in}}{\pgfqpoint{2.264102in}{1.572549in}}%
\pgfpathclose%
\pgfusepath{stroke,fill}%
\end{pgfscope}%
\begin{pgfscope}%
\pgfpathrectangle{\pgfqpoint{0.100000in}{0.212622in}}{\pgfqpoint{3.696000in}{3.696000in}}%
\pgfusepath{clip}%
\pgfsetbuttcap%
\pgfsetroundjoin%
\definecolor{currentfill}{rgb}{0.121569,0.466667,0.705882}%
\pgfsetfillcolor{currentfill}%
\pgfsetfillopacity{0.809825}%
\pgfsetlinewidth{1.003750pt}%
\definecolor{currentstroke}{rgb}{0.121569,0.466667,0.705882}%
\pgfsetstrokecolor{currentstroke}%
\pgfsetstrokeopacity{0.809825}%
\pgfsetdash{}{0pt}%
\pgfpathmoveto{\pgfqpoint{2.264977in}{1.569727in}}%
\pgfpathcurveto{\pgfqpoint{2.273214in}{1.569727in}}{\pgfqpoint{2.281114in}{1.572999in}}{\pgfqpoint{2.286938in}{1.578823in}}%
\pgfpathcurveto{\pgfqpoint{2.292762in}{1.584647in}}{\pgfqpoint{2.296034in}{1.592547in}}{\pgfqpoint{2.296034in}{1.600783in}}%
\pgfpathcurveto{\pgfqpoint{2.296034in}{1.609020in}}{\pgfqpoint{2.292762in}{1.616920in}}{\pgfqpoint{2.286938in}{1.622744in}}%
\pgfpathcurveto{\pgfqpoint{2.281114in}{1.628568in}}{\pgfqpoint{2.273214in}{1.631840in}}{\pgfqpoint{2.264977in}{1.631840in}}%
\pgfpathcurveto{\pgfqpoint{2.256741in}{1.631840in}}{\pgfqpoint{2.248841in}{1.628568in}}{\pgfqpoint{2.243017in}{1.622744in}}%
\pgfpathcurveto{\pgfqpoint{2.237193in}{1.616920in}}{\pgfqpoint{2.233921in}{1.609020in}}{\pgfqpoint{2.233921in}{1.600783in}}%
\pgfpathcurveto{\pgfqpoint{2.233921in}{1.592547in}}{\pgfqpoint{2.237193in}{1.584647in}}{\pgfqpoint{2.243017in}{1.578823in}}%
\pgfpathcurveto{\pgfqpoint{2.248841in}{1.572999in}}{\pgfqpoint{2.256741in}{1.569727in}}{\pgfqpoint{2.264977in}{1.569727in}}%
\pgfpathclose%
\pgfusepath{stroke,fill}%
\end{pgfscope}%
\begin{pgfscope}%
\pgfpathrectangle{\pgfqpoint{0.100000in}{0.212622in}}{\pgfqpoint{3.696000in}{3.696000in}}%
\pgfusepath{clip}%
\pgfsetbuttcap%
\pgfsetroundjoin%
\definecolor{currentfill}{rgb}{0.121569,0.466667,0.705882}%
\pgfsetfillcolor{currentfill}%
\pgfsetfillopacity{0.810573}%
\pgfsetlinewidth{1.003750pt}%
\definecolor{currentstroke}{rgb}{0.121569,0.466667,0.705882}%
\pgfsetstrokecolor{currentstroke}%
\pgfsetstrokeopacity{0.810573}%
\pgfsetdash{}{0pt}%
\pgfpathmoveto{\pgfqpoint{2.265482in}{1.569235in}}%
\pgfpathcurveto{\pgfqpoint{2.273718in}{1.569235in}}{\pgfqpoint{2.281618in}{1.572507in}}{\pgfqpoint{2.287442in}{1.578331in}}%
\pgfpathcurveto{\pgfqpoint{2.293266in}{1.584155in}}{\pgfqpoint{2.296538in}{1.592055in}}{\pgfqpoint{2.296538in}{1.600291in}}%
\pgfpathcurveto{\pgfqpoint{2.296538in}{1.608527in}}{\pgfqpoint{2.293266in}{1.616427in}}{\pgfqpoint{2.287442in}{1.622251in}}%
\pgfpathcurveto{\pgfqpoint{2.281618in}{1.628075in}}{\pgfqpoint{2.273718in}{1.631348in}}{\pgfqpoint{2.265482in}{1.631348in}}%
\pgfpathcurveto{\pgfqpoint{2.257245in}{1.631348in}}{\pgfqpoint{2.249345in}{1.628075in}}{\pgfqpoint{2.243521in}{1.622251in}}%
\pgfpathcurveto{\pgfqpoint{2.237697in}{1.616427in}}{\pgfqpoint{2.234425in}{1.608527in}}{\pgfqpoint{2.234425in}{1.600291in}}%
\pgfpathcurveto{\pgfqpoint{2.234425in}{1.592055in}}{\pgfqpoint{2.237697in}{1.584155in}}{\pgfqpoint{2.243521in}{1.578331in}}%
\pgfpathcurveto{\pgfqpoint{2.249345in}{1.572507in}}{\pgfqpoint{2.257245in}{1.569235in}}{\pgfqpoint{2.265482in}{1.569235in}}%
\pgfpathclose%
\pgfusepath{stroke,fill}%
\end{pgfscope}%
\begin{pgfscope}%
\pgfpathrectangle{\pgfqpoint{0.100000in}{0.212622in}}{\pgfqpoint{3.696000in}{3.696000in}}%
\pgfusepath{clip}%
\pgfsetbuttcap%
\pgfsetroundjoin%
\definecolor{currentfill}{rgb}{0.121569,0.466667,0.705882}%
\pgfsetfillcolor{currentfill}%
\pgfsetfillopacity{0.811766}%
\pgfsetlinewidth{1.003750pt}%
\definecolor{currentstroke}{rgb}{0.121569,0.466667,0.705882}%
\pgfsetstrokecolor{currentstroke}%
\pgfsetstrokeopacity{0.811766}%
\pgfsetdash{}{0pt}%
\pgfpathmoveto{\pgfqpoint{2.266200in}{1.568776in}}%
\pgfpathcurveto{\pgfqpoint{2.274436in}{1.568776in}}{\pgfqpoint{2.282336in}{1.572048in}}{\pgfqpoint{2.288160in}{1.577872in}}%
\pgfpathcurveto{\pgfqpoint{2.293984in}{1.583696in}}{\pgfqpoint{2.297256in}{1.591596in}}{\pgfqpoint{2.297256in}{1.599833in}}%
\pgfpathcurveto{\pgfqpoint{2.297256in}{1.608069in}}{\pgfqpoint{2.293984in}{1.615969in}}{\pgfqpoint{2.288160in}{1.621793in}}%
\pgfpathcurveto{\pgfqpoint{2.282336in}{1.627617in}}{\pgfqpoint{2.274436in}{1.630889in}}{\pgfqpoint{2.266200in}{1.630889in}}%
\pgfpathcurveto{\pgfqpoint{2.257964in}{1.630889in}}{\pgfqpoint{2.250064in}{1.627617in}}{\pgfqpoint{2.244240in}{1.621793in}}%
\pgfpathcurveto{\pgfqpoint{2.238416in}{1.615969in}}{\pgfqpoint{2.235143in}{1.608069in}}{\pgfqpoint{2.235143in}{1.599833in}}%
\pgfpathcurveto{\pgfqpoint{2.235143in}{1.591596in}}{\pgfqpoint{2.238416in}{1.583696in}}{\pgfqpoint{2.244240in}{1.577872in}}%
\pgfpathcurveto{\pgfqpoint{2.250064in}{1.572048in}}{\pgfqpoint{2.257964in}{1.568776in}}{\pgfqpoint{2.266200in}{1.568776in}}%
\pgfpathclose%
\pgfusepath{stroke,fill}%
\end{pgfscope}%
\begin{pgfscope}%
\pgfpathrectangle{\pgfqpoint{0.100000in}{0.212622in}}{\pgfqpoint{3.696000in}{3.696000in}}%
\pgfusepath{clip}%
\pgfsetbuttcap%
\pgfsetroundjoin%
\definecolor{currentfill}{rgb}{0.121569,0.466667,0.705882}%
\pgfsetfillcolor{currentfill}%
\pgfsetfillopacity{0.812960}%
\pgfsetlinewidth{1.003750pt}%
\definecolor{currentstroke}{rgb}{0.121569,0.466667,0.705882}%
\pgfsetstrokecolor{currentstroke}%
\pgfsetstrokeopacity{0.812960}%
\pgfsetdash{}{0pt}%
\pgfpathmoveto{\pgfqpoint{2.267190in}{1.567176in}}%
\pgfpathcurveto{\pgfqpoint{2.275427in}{1.567176in}}{\pgfqpoint{2.283327in}{1.570448in}}{\pgfqpoint{2.289151in}{1.576272in}}%
\pgfpathcurveto{\pgfqpoint{2.294975in}{1.582096in}}{\pgfqpoint{2.298247in}{1.589996in}}{\pgfqpoint{2.298247in}{1.598232in}}%
\pgfpathcurveto{\pgfqpoint{2.298247in}{1.606469in}}{\pgfqpoint{2.294975in}{1.614369in}}{\pgfqpoint{2.289151in}{1.620193in}}%
\pgfpathcurveto{\pgfqpoint{2.283327in}{1.626017in}}{\pgfqpoint{2.275427in}{1.629289in}}{\pgfqpoint{2.267190in}{1.629289in}}%
\pgfpathcurveto{\pgfqpoint{2.258954in}{1.629289in}}{\pgfqpoint{2.251054in}{1.626017in}}{\pgfqpoint{2.245230in}{1.620193in}}%
\pgfpathcurveto{\pgfqpoint{2.239406in}{1.614369in}}{\pgfqpoint{2.236134in}{1.606469in}}{\pgfqpoint{2.236134in}{1.598232in}}%
\pgfpathcurveto{\pgfqpoint{2.236134in}{1.589996in}}{\pgfqpoint{2.239406in}{1.582096in}}{\pgfqpoint{2.245230in}{1.576272in}}%
\pgfpathcurveto{\pgfqpoint{2.251054in}{1.570448in}}{\pgfqpoint{2.258954in}{1.567176in}}{\pgfqpoint{2.267190in}{1.567176in}}%
\pgfpathclose%
\pgfusepath{stroke,fill}%
\end{pgfscope}%
\begin{pgfscope}%
\pgfpathrectangle{\pgfqpoint{0.100000in}{0.212622in}}{\pgfqpoint{3.696000in}{3.696000in}}%
\pgfusepath{clip}%
\pgfsetbuttcap%
\pgfsetroundjoin%
\definecolor{currentfill}{rgb}{0.121569,0.466667,0.705882}%
\pgfsetfillcolor{currentfill}%
\pgfsetfillopacity{0.814156}%
\pgfsetlinewidth{1.003750pt}%
\definecolor{currentstroke}{rgb}{0.121569,0.466667,0.705882}%
\pgfsetstrokecolor{currentstroke}%
\pgfsetstrokeopacity{0.814156}%
\pgfsetdash{}{0pt}%
\pgfpathmoveto{\pgfqpoint{2.268223in}{1.563959in}}%
\pgfpathcurveto{\pgfqpoint{2.276459in}{1.563959in}}{\pgfqpoint{2.284360in}{1.567232in}}{\pgfqpoint{2.290183in}{1.573056in}}%
\pgfpathcurveto{\pgfqpoint{2.296007in}{1.578880in}}{\pgfqpoint{2.299280in}{1.586780in}}{\pgfqpoint{2.299280in}{1.595016in}}%
\pgfpathcurveto{\pgfqpoint{2.299280in}{1.603252in}}{\pgfqpoint{2.296007in}{1.611152in}}{\pgfqpoint{2.290183in}{1.616976in}}%
\pgfpathcurveto{\pgfqpoint{2.284360in}{1.622800in}}{\pgfqpoint{2.276459in}{1.626072in}}{\pgfqpoint{2.268223in}{1.626072in}}%
\pgfpathcurveto{\pgfqpoint{2.259987in}{1.626072in}}{\pgfqpoint{2.252087in}{1.622800in}}{\pgfqpoint{2.246263in}{1.616976in}}%
\pgfpathcurveto{\pgfqpoint{2.240439in}{1.611152in}}{\pgfqpoint{2.237167in}{1.603252in}}{\pgfqpoint{2.237167in}{1.595016in}}%
\pgfpathcurveto{\pgfqpoint{2.237167in}{1.586780in}}{\pgfqpoint{2.240439in}{1.578880in}}{\pgfqpoint{2.246263in}{1.573056in}}%
\pgfpathcurveto{\pgfqpoint{2.252087in}{1.567232in}}{\pgfqpoint{2.259987in}{1.563959in}}{\pgfqpoint{2.268223in}{1.563959in}}%
\pgfpathclose%
\pgfusepath{stroke,fill}%
\end{pgfscope}%
\begin{pgfscope}%
\pgfpathrectangle{\pgfqpoint{0.100000in}{0.212622in}}{\pgfqpoint{3.696000in}{3.696000in}}%
\pgfusepath{clip}%
\pgfsetbuttcap%
\pgfsetroundjoin%
\definecolor{currentfill}{rgb}{0.121569,0.466667,0.705882}%
\pgfsetfillcolor{currentfill}%
\pgfsetfillopacity{0.814916}%
\pgfsetlinewidth{1.003750pt}%
\definecolor{currentstroke}{rgb}{0.121569,0.466667,0.705882}%
\pgfsetstrokecolor{currentstroke}%
\pgfsetstrokeopacity{0.814916}%
\pgfsetdash{}{0pt}%
\pgfpathmoveto{\pgfqpoint{2.268962in}{1.562924in}}%
\pgfpathcurveto{\pgfqpoint{2.277198in}{1.562924in}}{\pgfqpoint{2.285098in}{1.566196in}}{\pgfqpoint{2.290922in}{1.572020in}}%
\pgfpathcurveto{\pgfqpoint{2.296746in}{1.577844in}}{\pgfqpoint{2.300018in}{1.585744in}}{\pgfqpoint{2.300018in}{1.593980in}}%
\pgfpathcurveto{\pgfqpoint{2.300018in}{1.602216in}}{\pgfqpoint{2.296746in}{1.610117in}}{\pgfqpoint{2.290922in}{1.615940in}}%
\pgfpathcurveto{\pgfqpoint{2.285098in}{1.621764in}}{\pgfqpoint{2.277198in}{1.625037in}}{\pgfqpoint{2.268962in}{1.625037in}}%
\pgfpathcurveto{\pgfqpoint{2.260725in}{1.625037in}}{\pgfqpoint{2.252825in}{1.621764in}}{\pgfqpoint{2.247002in}{1.615940in}}%
\pgfpathcurveto{\pgfqpoint{2.241178in}{1.610117in}}{\pgfqpoint{2.237905in}{1.602216in}}{\pgfqpoint{2.237905in}{1.593980in}}%
\pgfpathcurveto{\pgfqpoint{2.237905in}{1.585744in}}{\pgfqpoint{2.241178in}{1.577844in}}{\pgfqpoint{2.247002in}{1.572020in}}%
\pgfpathcurveto{\pgfqpoint{2.252825in}{1.566196in}}{\pgfqpoint{2.260725in}{1.562924in}}{\pgfqpoint{2.268962in}{1.562924in}}%
\pgfpathclose%
\pgfusepath{stroke,fill}%
\end{pgfscope}%
\begin{pgfscope}%
\pgfpathrectangle{\pgfqpoint{0.100000in}{0.212622in}}{\pgfqpoint{3.696000in}{3.696000in}}%
\pgfusepath{clip}%
\pgfsetbuttcap%
\pgfsetroundjoin%
\definecolor{currentfill}{rgb}{0.121569,0.466667,0.705882}%
\pgfsetfillcolor{currentfill}%
\pgfsetfillopacity{0.816117}%
\pgfsetlinewidth{1.003750pt}%
\definecolor{currentstroke}{rgb}{0.121569,0.466667,0.705882}%
\pgfsetstrokecolor{currentstroke}%
\pgfsetstrokeopacity{0.816117}%
\pgfsetdash{}{0pt}%
\pgfpathmoveto{\pgfqpoint{2.269727in}{1.562323in}}%
\pgfpathcurveto{\pgfqpoint{2.277963in}{1.562323in}}{\pgfqpoint{2.285863in}{1.565596in}}{\pgfqpoint{2.291687in}{1.571419in}}%
\pgfpathcurveto{\pgfqpoint{2.297511in}{1.577243in}}{\pgfqpoint{2.300783in}{1.585143in}}{\pgfqpoint{2.300783in}{1.593380in}}%
\pgfpathcurveto{\pgfqpoint{2.300783in}{1.601616in}}{\pgfqpoint{2.297511in}{1.609516in}}{\pgfqpoint{2.291687in}{1.615340in}}%
\pgfpathcurveto{\pgfqpoint{2.285863in}{1.621164in}}{\pgfqpoint{2.277963in}{1.624436in}}{\pgfqpoint{2.269727in}{1.624436in}}%
\pgfpathcurveto{\pgfqpoint{2.261490in}{1.624436in}}{\pgfqpoint{2.253590in}{1.621164in}}{\pgfqpoint{2.247766in}{1.615340in}}%
\pgfpathcurveto{\pgfqpoint{2.241942in}{1.609516in}}{\pgfqpoint{2.238670in}{1.601616in}}{\pgfqpoint{2.238670in}{1.593380in}}%
\pgfpathcurveto{\pgfqpoint{2.238670in}{1.585143in}}{\pgfqpoint{2.241942in}{1.577243in}}{\pgfqpoint{2.247766in}{1.571419in}}%
\pgfpathcurveto{\pgfqpoint{2.253590in}{1.565596in}}{\pgfqpoint{2.261490in}{1.562323in}}{\pgfqpoint{2.269727in}{1.562323in}}%
\pgfpathclose%
\pgfusepath{stroke,fill}%
\end{pgfscope}%
\begin{pgfscope}%
\pgfpathrectangle{\pgfqpoint{0.100000in}{0.212622in}}{\pgfqpoint{3.696000in}{3.696000in}}%
\pgfusepath{clip}%
\pgfsetbuttcap%
\pgfsetroundjoin%
\definecolor{currentfill}{rgb}{0.121569,0.466667,0.705882}%
\pgfsetfillcolor{currentfill}%
\pgfsetfillopacity{0.817879}%
\pgfsetlinewidth{1.003750pt}%
\definecolor{currentstroke}{rgb}{0.121569,0.466667,0.705882}%
\pgfsetstrokecolor{currentstroke}%
\pgfsetstrokeopacity{0.817879}%
\pgfsetdash{}{0pt}%
\pgfpathmoveto{\pgfqpoint{2.271062in}{1.559851in}}%
\pgfpathcurveto{\pgfqpoint{2.279298in}{1.559851in}}{\pgfqpoint{2.287198in}{1.563123in}}{\pgfqpoint{2.293022in}{1.568947in}}%
\pgfpathcurveto{\pgfqpoint{2.298846in}{1.574771in}}{\pgfqpoint{2.302119in}{1.582671in}}{\pgfqpoint{2.302119in}{1.590907in}}%
\pgfpathcurveto{\pgfqpoint{2.302119in}{1.599144in}}{\pgfqpoint{2.298846in}{1.607044in}}{\pgfqpoint{2.293022in}{1.612868in}}%
\pgfpathcurveto{\pgfqpoint{2.287198in}{1.618692in}}{\pgfqpoint{2.279298in}{1.621964in}}{\pgfqpoint{2.271062in}{1.621964in}}%
\pgfpathcurveto{\pgfqpoint{2.262826in}{1.621964in}}{\pgfqpoint{2.254926in}{1.618692in}}{\pgfqpoint{2.249102in}{1.612868in}}%
\pgfpathcurveto{\pgfqpoint{2.243278in}{1.607044in}}{\pgfqpoint{2.240006in}{1.599144in}}{\pgfqpoint{2.240006in}{1.590907in}}%
\pgfpathcurveto{\pgfqpoint{2.240006in}{1.582671in}}{\pgfqpoint{2.243278in}{1.574771in}}{\pgfqpoint{2.249102in}{1.568947in}}%
\pgfpathcurveto{\pgfqpoint{2.254926in}{1.563123in}}{\pgfqpoint{2.262826in}{1.559851in}}{\pgfqpoint{2.271062in}{1.559851in}}%
\pgfpathclose%
\pgfusepath{stroke,fill}%
\end{pgfscope}%
\begin{pgfscope}%
\pgfpathrectangle{\pgfqpoint{0.100000in}{0.212622in}}{\pgfqpoint{3.696000in}{3.696000in}}%
\pgfusepath{clip}%
\pgfsetbuttcap%
\pgfsetroundjoin%
\definecolor{currentfill}{rgb}{0.121569,0.466667,0.705882}%
\pgfsetfillcolor{currentfill}%
\pgfsetfillopacity{0.818640}%
\pgfsetlinewidth{1.003750pt}%
\definecolor{currentstroke}{rgb}{0.121569,0.466667,0.705882}%
\pgfsetstrokecolor{currentstroke}%
\pgfsetstrokeopacity{0.818640}%
\pgfsetdash{}{0pt}%
\pgfpathmoveto{\pgfqpoint{2.271633in}{1.557120in}}%
\pgfpathcurveto{\pgfqpoint{2.279869in}{1.557120in}}{\pgfqpoint{2.287769in}{1.560392in}}{\pgfqpoint{2.293593in}{1.566216in}}%
\pgfpathcurveto{\pgfqpoint{2.299417in}{1.572040in}}{\pgfqpoint{2.302689in}{1.579940in}}{\pgfqpoint{2.302689in}{1.588176in}}%
\pgfpathcurveto{\pgfqpoint{2.302689in}{1.596413in}}{\pgfqpoint{2.299417in}{1.604313in}}{\pgfqpoint{2.293593in}{1.610137in}}%
\pgfpathcurveto{\pgfqpoint{2.287769in}{1.615961in}}{\pgfqpoint{2.279869in}{1.619233in}}{\pgfqpoint{2.271633in}{1.619233in}}%
\pgfpathcurveto{\pgfqpoint{2.263397in}{1.619233in}}{\pgfqpoint{2.255497in}{1.615961in}}{\pgfqpoint{2.249673in}{1.610137in}}%
\pgfpathcurveto{\pgfqpoint{2.243849in}{1.604313in}}{\pgfqpoint{2.240576in}{1.596413in}}{\pgfqpoint{2.240576in}{1.588176in}}%
\pgfpathcurveto{\pgfqpoint{2.240576in}{1.579940in}}{\pgfqpoint{2.243849in}{1.572040in}}{\pgfqpoint{2.249673in}{1.566216in}}%
\pgfpathcurveto{\pgfqpoint{2.255497in}{1.560392in}}{\pgfqpoint{2.263397in}{1.557120in}}{\pgfqpoint{2.271633in}{1.557120in}}%
\pgfpathclose%
\pgfusepath{stroke,fill}%
\end{pgfscope}%
\begin{pgfscope}%
\pgfpathrectangle{\pgfqpoint{0.100000in}{0.212622in}}{\pgfqpoint{3.696000in}{3.696000in}}%
\pgfusepath{clip}%
\pgfsetbuttcap%
\pgfsetroundjoin%
\definecolor{currentfill}{rgb}{0.121569,0.466667,0.705882}%
\pgfsetfillcolor{currentfill}%
\pgfsetfillopacity{0.819811}%
\pgfsetlinewidth{1.003750pt}%
\definecolor{currentstroke}{rgb}{0.121569,0.466667,0.705882}%
\pgfsetstrokecolor{currentstroke}%
\pgfsetstrokeopacity{0.819811}%
\pgfsetdash{}{0pt}%
\pgfpathmoveto{\pgfqpoint{2.272667in}{1.555243in}}%
\pgfpathcurveto{\pgfqpoint{2.280903in}{1.555243in}}{\pgfqpoint{2.288803in}{1.558515in}}{\pgfqpoint{2.294627in}{1.564339in}}%
\pgfpathcurveto{\pgfqpoint{2.300451in}{1.570163in}}{\pgfqpoint{2.303723in}{1.578063in}}{\pgfqpoint{2.303723in}{1.586300in}}%
\pgfpathcurveto{\pgfqpoint{2.303723in}{1.594536in}}{\pgfqpoint{2.300451in}{1.602436in}}{\pgfqpoint{2.294627in}{1.608260in}}%
\pgfpathcurveto{\pgfqpoint{2.288803in}{1.614084in}}{\pgfqpoint{2.280903in}{1.617356in}}{\pgfqpoint{2.272667in}{1.617356in}}%
\pgfpathcurveto{\pgfqpoint{2.264430in}{1.617356in}}{\pgfqpoint{2.256530in}{1.614084in}}{\pgfqpoint{2.250706in}{1.608260in}}%
\pgfpathcurveto{\pgfqpoint{2.244883in}{1.602436in}}{\pgfqpoint{2.241610in}{1.594536in}}{\pgfqpoint{2.241610in}{1.586300in}}%
\pgfpathcurveto{\pgfqpoint{2.241610in}{1.578063in}}{\pgfqpoint{2.244883in}{1.570163in}}{\pgfqpoint{2.250706in}{1.564339in}}%
\pgfpathcurveto{\pgfqpoint{2.256530in}{1.558515in}}{\pgfqpoint{2.264430in}{1.555243in}}{\pgfqpoint{2.272667in}{1.555243in}}%
\pgfpathclose%
\pgfusepath{stroke,fill}%
\end{pgfscope}%
\begin{pgfscope}%
\pgfpathrectangle{\pgfqpoint{0.100000in}{0.212622in}}{\pgfqpoint{3.696000in}{3.696000in}}%
\pgfusepath{clip}%
\pgfsetbuttcap%
\pgfsetroundjoin%
\definecolor{currentfill}{rgb}{0.121569,0.466667,0.705882}%
\pgfsetfillcolor{currentfill}%
\pgfsetfillopacity{0.820299}%
\pgfsetlinewidth{1.003750pt}%
\definecolor{currentstroke}{rgb}{0.121569,0.466667,0.705882}%
\pgfsetstrokecolor{currentstroke}%
\pgfsetstrokeopacity{0.820299}%
\pgfsetdash{}{0pt}%
\pgfpathmoveto{\pgfqpoint{0.634814in}{2.616437in}}%
\pgfpathcurveto{\pgfqpoint{0.643050in}{2.616437in}}{\pgfqpoint{0.650950in}{2.619709in}}{\pgfqpoint{0.656774in}{2.625533in}}%
\pgfpathcurveto{\pgfqpoint{0.662598in}{2.631357in}}{\pgfqpoint{0.665870in}{2.639257in}}{\pgfqpoint{0.665870in}{2.647493in}}%
\pgfpathcurveto{\pgfqpoint{0.665870in}{2.655730in}}{\pgfqpoint{0.662598in}{2.663630in}}{\pgfqpoint{0.656774in}{2.669454in}}%
\pgfpathcurveto{\pgfqpoint{0.650950in}{2.675278in}}{\pgfqpoint{0.643050in}{2.678550in}}{\pgfqpoint{0.634814in}{2.678550in}}%
\pgfpathcurveto{\pgfqpoint{0.626577in}{2.678550in}}{\pgfqpoint{0.618677in}{2.675278in}}{\pgfqpoint{0.612853in}{2.669454in}}%
\pgfpathcurveto{\pgfqpoint{0.607030in}{2.663630in}}{\pgfqpoint{0.603757in}{2.655730in}}{\pgfqpoint{0.603757in}{2.647493in}}%
\pgfpathcurveto{\pgfqpoint{0.603757in}{2.639257in}}{\pgfqpoint{0.607030in}{2.631357in}}{\pgfqpoint{0.612853in}{2.625533in}}%
\pgfpathcurveto{\pgfqpoint{0.618677in}{2.619709in}}{\pgfqpoint{0.626577in}{2.616437in}}{\pgfqpoint{0.634814in}{2.616437in}}%
\pgfpathclose%
\pgfusepath{stroke,fill}%
\end{pgfscope}%
\begin{pgfscope}%
\pgfpathrectangle{\pgfqpoint{0.100000in}{0.212622in}}{\pgfqpoint{3.696000in}{3.696000in}}%
\pgfusepath{clip}%
\pgfsetbuttcap%
\pgfsetroundjoin%
\definecolor{currentfill}{rgb}{0.121569,0.466667,0.705882}%
\pgfsetfillcolor{currentfill}%
\pgfsetfillopacity{0.820370}%
\pgfsetlinewidth{1.003750pt}%
\definecolor{currentstroke}{rgb}{0.121569,0.466667,0.705882}%
\pgfsetstrokecolor{currentstroke}%
\pgfsetstrokeopacity{0.820370}%
\pgfsetdash{}{0pt}%
\pgfpathmoveto{\pgfqpoint{0.624828in}{2.629289in}}%
\pgfpathcurveto{\pgfqpoint{0.633064in}{2.629289in}}{\pgfqpoint{0.640964in}{2.632561in}}{\pgfqpoint{0.646788in}{2.638385in}}%
\pgfpathcurveto{\pgfqpoint{0.652612in}{2.644209in}}{\pgfqpoint{0.655884in}{2.652109in}}{\pgfqpoint{0.655884in}{2.660345in}}%
\pgfpathcurveto{\pgfqpoint{0.655884in}{2.668581in}}{\pgfqpoint{0.652612in}{2.676481in}}{\pgfqpoint{0.646788in}{2.682305in}}%
\pgfpathcurveto{\pgfqpoint{0.640964in}{2.688129in}}{\pgfqpoint{0.633064in}{2.691402in}}{\pgfqpoint{0.624828in}{2.691402in}}%
\pgfpathcurveto{\pgfqpoint{0.616591in}{2.691402in}}{\pgfqpoint{0.608691in}{2.688129in}}{\pgfqpoint{0.602867in}{2.682305in}}%
\pgfpathcurveto{\pgfqpoint{0.597043in}{2.676481in}}{\pgfqpoint{0.593771in}{2.668581in}}{\pgfqpoint{0.593771in}{2.660345in}}%
\pgfpathcurveto{\pgfqpoint{0.593771in}{2.652109in}}{\pgfqpoint{0.597043in}{2.644209in}}{\pgfqpoint{0.602867in}{2.638385in}}%
\pgfpathcurveto{\pgfqpoint{0.608691in}{2.632561in}}{\pgfqpoint{0.616591in}{2.629289in}}{\pgfqpoint{0.624828in}{2.629289in}}%
\pgfpathclose%
\pgfusepath{stroke,fill}%
\end{pgfscope}%
\begin{pgfscope}%
\pgfpathrectangle{\pgfqpoint{0.100000in}{0.212622in}}{\pgfqpoint{3.696000in}{3.696000in}}%
\pgfusepath{clip}%
\pgfsetbuttcap%
\pgfsetroundjoin%
\definecolor{currentfill}{rgb}{0.121569,0.466667,0.705882}%
\pgfsetfillcolor{currentfill}%
\pgfsetfillopacity{0.820581}%
\pgfsetlinewidth{1.003750pt}%
\definecolor{currentstroke}{rgb}{0.121569,0.466667,0.705882}%
\pgfsetstrokecolor{currentstroke}%
\pgfsetstrokeopacity{0.820581}%
\pgfsetdash{}{0pt}%
\pgfpathmoveto{\pgfqpoint{0.618227in}{2.631525in}}%
\pgfpathcurveto{\pgfqpoint{0.626464in}{2.631525in}}{\pgfqpoint{0.634364in}{2.634797in}}{\pgfqpoint{0.640188in}{2.640621in}}%
\pgfpathcurveto{\pgfqpoint{0.646011in}{2.646445in}}{\pgfqpoint{0.649284in}{2.654345in}}{\pgfqpoint{0.649284in}{2.662581in}}%
\pgfpathcurveto{\pgfqpoint{0.649284in}{2.670818in}}{\pgfqpoint{0.646011in}{2.678718in}}{\pgfqpoint{0.640188in}{2.684542in}}%
\pgfpathcurveto{\pgfqpoint{0.634364in}{2.690366in}}{\pgfqpoint{0.626464in}{2.693638in}}{\pgfqpoint{0.618227in}{2.693638in}}%
\pgfpathcurveto{\pgfqpoint{0.609991in}{2.693638in}}{\pgfqpoint{0.602091in}{2.690366in}}{\pgfqpoint{0.596267in}{2.684542in}}%
\pgfpathcurveto{\pgfqpoint{0.590443in}{2.678718in}}{\pgfqpoint{0.587171in}{2.670818in}}{\pgfqpoint{0.587171in}{2.662581in}}%
\pgfpathcurveto{\pgfqpoint{0.587171in}{2.654345in}}{\pgfqpoint{0.590443in}{2.646445in}}{\pgfqpoint{0.596267in}{2.640621in}}%
\pgfpathcurveto{\pgfqpoint{0.602091in}{2.634797in}}{\pgfqpoint{0.609991in}{2.631525in}}{\pgfqpoint{0.618227in}{2.631525in}}%
\pgfpathclose%
\pgfusepath{stroke,fill}%
\end{pgfscope}%
\begin{pgfscope}%
\pgfpathrectangle{\pgfqpoint{0.100000in}{0.212622in}}{\pgfqpoint{3.696000in}{3.696000in}}%
\pgfusepath{clip}%
\pgfsetbuttcap%
\pgfsetroundjoin%
\definecolor{currentfill}{rgb}{0.121569,0.466667,0.705882}%
\pgfsetfillcolor{currentfill}%
\pgfsetfillopacity{0.820593}%
\pgfsetlinewidth{1.003750pt}%
\definecolor{currentstroke}{rgb}{0.121569,0.466667,0.705882}%
\pgfsetstrokecolor{currentstroke}%
\pgfsetstrokeopacity{0.820593}%
\pgfsetdash{}{0pt}%
\pgfpathmoveto{\pgfqpoint{2.273179in}{1.555024in}}%
\pgfpathcurveto{\pgfqpoint{2.281416in}{1.555024in}}{\pgfqpoint{2.289316in}{1.558297in}}{\pgfqpoint{2.295140in}{1.564121in}}%
\pgfpathcurveto{\pgfqpoint{2.300963in}{1.569944in}}{\pgfqpoint{2.304236in}{1.577844in}}{\pgfqpoint{2.304236in}{1.586081in}}%
\pgfpathcurveto{\pgfqpoint{2.304236in}{1.594317in}}{\pgfqpoint{2.300963in}{1.602217in}}{\pgfqpoint{2.295140in}{1.608041in}}%
\pgfpathcurveto{\pgfqpoint{2.289316in}{1.613865in}}{\pgfqpoint{2.281416in}{1.617137in}}{\pgfqpoint{2.273179in}{1.617137in}}%
\pgfpathcurveto{\pgfqpoint{2.264943in}{1.617137in}}{\pgfqpoint{2.257043in}{1.613865in}}{\pgfqpoint{2.251219in}{1.608041in}}%
\pgfpathcurveto{\pgfqpoint{2.245395in}{1.602217in}}{\pgfqpoint{2.242123in}{1.594317in}}{\pgfqpoint{2.242123in}{1.586081in}}%
\pgfpathcurveto{\pgfqpoint{2.242123in}{1.577844in}}{\pgfqpoint{2.245395in}{1.569944in}}{\pgfqpoint{2.251219in}{1.564121in}}%
\pgfpathcurveto{\pgfqpoint{2.257043in}{1.558297in}}{\pgfqpoint{2.264943in}{1.555024in}}{\pgfqpoint{2.273179in}{1.555024in}}%
\pgfpathclose%
\pgfusepath{stroke,fill}%
\end{pgfscope}%
\begin{pgfscope}%
\pgfpathrectangle{\pgfqpoint{0.100000in}{0.212622in}}{\pgfqpoint{3.696000in}{3.696000in}}%
\pgfusepath{clip}%
\pgfsetbuttcap%
\pgfsetroundjoin%
\definecolor{currentfill}{rgb}{0.121569,0.466667,0.705882}%
\pgfsetfillcolor{currentfill}%
\pgfsetfillopacity{0.821371}%
\pgfsetlinewidth{1.003750pt}%
\definecolor{currentstroke}{rgb}{0.121569,0.466667,0.705882}%
\pgfsetstrokecolor{currentstroke}%
\pgfsetstrokeopacity{0.821371}%
\pgfsetdash{}{0pt}%
\pgfpathmoveto{\pgfqpoint{0.614876in}{2.639866in}}%
\pgfpathcurveto{\pgfqpoint{0.623112in}{2.639866in}}{\pgfqpoint{0.631012in}{2.643138in}}{\pgfqpoint{0.636836in}{2.648962in}}%
\pgfpathcurveto{\pgfqpoint{0.642660in}{2.654786in}}{\pgfqpoint{0.645933in}{2.662686in}}{\pgfqpoint{0.645933in}{2.670922in}}%
\pgfpathcurveto{\pgfqpoint{0.645933in}{2.679159in}}{\pgfqpoint{0.642660in}{2.687059in}}{\pgfqpoint{0.636836in}{2.692883in}}%
\pgfpathcurveto{\pgfqpoint{0.631012in}{2.698706in}}{\pgfqpoint{0.623112in}{2.701979in}}{\pgfqpoint{0.614876in}{2.701979in}}%
\pgfpathcurveto{\pgfqpoint{0.606640in}{2.701979in}}{\pgfqpoint{0.598740in}{2.698706in}}{\pgfqpoint{0.592916in}{2.692883in}}%
\pgfpathcurveto{\pgfqpoint{0.587092in}{2.687059in}}{\pgfqpoint{0.583820in}{2.679159in}}{\pgfqpoint{0.583820in}{2.670922in}}%
\pgfpathcurveto{\pgfqpoint{0.583820in}{2.662686in}}{\pgfqpoint{0.587092in}{2.654786in}}{\pgfqpoint{0.592916in}{2.648962in}}%
\pgfpathcurveto{\pgfqpoint{0.598740in}{2.643138in}}{\pgfqpoint{0.606640in}{2.639866in}}{\pgfqpoint{0.614876in}{2.639866in}}%
\pgfpathclose%
\pgfusepath{stroke,fill}%
\end{pgfscope}%
\begin{pgfscope}%
\pgfpathrectangle{\pgfqpoint{0.100000in}{0.212622in}}{\pgfqpoint{3.696000in}{3.696000in}}%
\pgfusepath{clip}%
\pgfsetbuttcap%
\pgfsetroundjoin%
\definecolor{currentfill}{rgb}{0.121569,0.466667,0.705882}%
\pgfsetfillcolor{currentfill}%
\pgfsetfillopacity{0.821599}%
\pgfsetlinewidth{1.003750pt}%
\definecolor{currentstroke}{rgb}{0.121569,0.466667,0.705882}%
\pgfsetstrokecolor{currentstroke}%
\pgfsetstrokeopacity{0.821599}%
\pgfsetdash{}{0pt}%
\pgfpathmoveto{\pgfqpoint{2.274090in}{1.553595in}}%
\pgfpathcurveto{\pgfqpoint{2.282326in}{1.553595in}}{\pgfqpoint{2.290226in}{1.556867in}}{\pgfqpoint{2.296050in}{1.562691in}}%
\pgfpathcurveto{\pgfqpoint{2.301874in}{1.568515in}}{\pgfqpoint{2.305147in}{1.576415in}}{\pgfqpoint{2.305147in}{1.584652in}}%
\pgfpathcurveto{\pgfqpoint{2.305147in}{1.592888in}}{\pgfqpoint{2.301874in}{1.600788in}}{\pgfqpoint{2.296050in}{1.606612in}}%
\pgfpathcurveto{\pgfqpoint{2.290226in}{1.612436in}}{\pgfqpoint{2.282326in}{1.615708in}}{\pgfqpoint{2.274090in}{1.615708in}}%
\pgfpathcurveto{\pgfqpoint{2.265854in}{1.615708in}}{\pgfqpoint{2.257954in}{1.612436in}}{\pgfqpoint{2.252130in}{1.606612in}}%
\pgfpathcurveto{\pgfqpoint{2.246306in}{1.600788in}}{\pgfqpoint{2.243034in}{1.592888in}}{\pgfqpoint{2.243034in}{1.584652in}}%
\pgfpathcurveto{\pgfqpoint{2.243034in}{1.576415in}}{\pgfqpoint{2.246306in}{1.568515in}}{\pgfqpoint{2.252130in}{1.562691in}}%
\pgfpathcurveto{\pgfqpoint{2.257954in}{1.556867in}}{\pgfqpoint{2.265854in}{1.553595in}}{\pgfqpoint{2.274090in}{1.553595in}}%
\pgfpathclose%
\pgfusepath{stroke,fill}%
\end{pgfscope}%
\begin{pgfscope}%
\pgfpathrectangle{\pgfqpoint{0.100000in}{0.212622in}}{\pgfqpoint{3.696000in}{3.696000in}}%
\pgfusepath{clip}%
\pgfsetbuttcap%
\pgfsetroundjoin%
\definecolor{currentfill}{rgb}{0.121569,0.466667,0.705882}%
\pgfsetfillcolor{currentfill}%
\pgfsetfillopacity{0.821631}%
\pgfsetlinewidth{1.003750pt}%
\definecolor{currentstroke}{rgb}{0.121569,0.466667,0.705882}%
\pgfsetstrokecolor{currentstroke}%
\pgfsetstrokeopacity{0.821631}%
\pgfsetdash{}{0pt}%
\pgfpathmoveto{\pgfqpoint{0.612810in}{2.641103in}}%
\pgfpathcurveto{\pgfqpoint{0.621046in}{2.641103in}}{\pgfqpoint{0.628946in}{2.644376in}}{\pgfqpoint{0.634770in}{2.650200in}}%
\pgfpathcurveto{\pgfqpoint{0.640594in}{2.656024in}}{\pgfqpoint{0.643866in}{2.663924in}}{\pgfqpoint{0.643866in}{2.672160in}}%
\pgfpathcurveto{\pgfqpoint{0.643866in}{2.680396in}}{\pgfqpoint{0.640594in}{2.688296in}}{\pgfqpoint{0.634770in}{2.694120in}}%
\pgfpathcurveto{\pgfqpoint{0.628946in}{2.699944in}}{\pgfqpoint{0.621046in}{2.703216in}}{\pgfqpoint{0.612810in}{2.703216in}}%
\pgfpathcurveto{\pgfqpoint{0.604573in}{2.703216in}}{\pgfqpoint{0.596673in}{2.699944in}}{\pgfqpoint{0.590850in}{2.694120in}}%
\pgfpathcurveto{\pgfqpoint{0.585026in}{2.688296in}}{\pgfqpoint{0.581753in}{2.680396in}}{\pgfqpoint{0.581753in}{2.672160in}}%
\pgfpathcurveto{\pgfqpoint{0.581753in}{2.663924in}}{\pgfqpoint{0.585026in}{2.656024in}}{\pgfqpoint{0.590850in}{2.650200in}}%
\pgfpathcurveto{\pgfqpoint{0.596673in}{2.644376in}}{\pgfqpoint{0.604573in}{2.641103in}}{\pgfqpoint{0.612810in}{2.641103in}}%
\pgfpathclose%
\pgfusepath{stroke,fill}%
\end{pgfscope}%
\begin{pgfscope}%
\pgfpathrectangle{\pgfqpoint{0.100000in}{0.212622in}}{\pgfqpoint{3.696000in}{3.696000in}}%
\pgfusepath{clip}%
\pgfsetbuttcap%
\pgfsetroundjoin%
\definecolor{currentfill}{rgb}{0.121569,0.466667,0.705882}%
\pgfsetfillcolor{currentfill}%
\pgfsetfillopacity{0.821749}%
\pgfsetlinewidth{1.003750pt}%
\definecolor{currentstroke}{rgb}{0.121569,0.466667,0.705882}%
\pgfsetstrokecolor{currentstroke}%
\pgfsetstrokeopacity{0.821749}%
\pgfsetdash{}{0pt}%
\pgfpathmoveto{\pgfqpoint{0.611680in}{2.641613in}}%
\pgfpathcurveto{\pgfqpoint{0.619916in}{2.641613in}}{\pgfqpoint{0.627816in}{2.644885in}}{\pgfqpoint{0.633640in}{2.650709in}}%
\pgfpathcurveto{\pgfqpoint{0.639464in}{2.656533in}}{\pgfqpoint{0.642736in}{2.664433in}}{\pgfqpoint{0.642736in}{2.672669in}}%
\pgfpathcurveto{\pgfqpoint{0.642736in}{2.680905in}}{\pgfqpoint{0.639464in}{2.688805in}}{\pgfqpoint{0.633640in}{2.694629in}}%
\pgfpathcurveto{\pgfqpoint{0.627816in}{2.700453in}}{\pgfqpoint{0.619916in}{2.703726in}}{\pgfqpoint{0.611680in}{2.703726in}}%
\pgfpathcurveto{\pgfqpoint{0.603443in}{2.703726in}}{\pgfqpoint{0.595543in}{2.700453in}}{\pgfqpoint{0.589719in}{2.694629in}}%
\pgfpathcurveto{\pgfqpoint{0.583895in}{2.688805in}}{\pgfqpoint{0.580623in}{2.680905in}}{\pgfqpoint{0.580623in}{2.672669in}}%
\pgfpathcurveto{\pgfqpoint{0.580623in}{2.664433in}}{\pgfqpoint{0.583895in}{2.656533in}}{\pgfqpoint{0.589719in}{2.650709in}}%
\pgfpathcurveto{\pgfqpoint{0.595543in}{2.644885in}}{\pgfqpoint{0.603443in}{2.641613in}}{\pgfqpoint{0.611680in}{2.641613in}}%
\pgfpathclose%
\pgfusepath{stroke,fill}%
\end{pgfscope}%
\begin{pgfscope}%
\pgfpathrectangle{\pgfqpoint{0.100000in}{0.212622in}}{\pgfqpoint{3.696000in}{3.696000in}}%
\pgfusepath{clip}%
\pgfsetbuttcap%
\pgfsetroundjoin%
\definecolor{currentfill}{rgb}{0.121569,0.466667,0.705882}%
\pgfsetfillcolor{currentfill}%
\pgfsetfillopacity{0.821813}%
\pgfsetlinewidth{1.003750pt}%
\definecolor{currentstroke}{rgb}{0.121569,0.466667,0.705882}%
\pgfsetstrokecolor{currentstroke}%
\pgfsetstrokeopacity{0.821813}%
\pgfsetdash{}{0pt}%
\pgfpathmoveto{\pgfqpoint{0.611067in}{2.641710in}}%
\pgfpathcurveto{\pgfqpoint{0.619303in}{2.641710in}}{\pgfqpoint{0.627203in}{2.644983in}}{\pgfqpoint{0.633027in}{2.650807in}}%
\pgfpathcurveto{\pgfqpoint{0.638851in}{2.656630in}}{\pgfqpoint{0.642123in}{2.664531in}}{\pgfqpoint{0.642123in}{2.672767in}}%
\pgfpathcurveto{\pgfqpoint{0.642123in}{2.681003in}}{\pgfqpoint{0.638851in}{2.688903in}}{\pgfqpoint{0.633027in}{2.694727in}}%
\pgfpathcurveto{\pgfqpoint{0.627203in}{2.700551in}}{\pgfqpoint{0.619303in}{2.703823in}}{\pgfqpoint{0.611067in}{2.703823in}}%
\pgfpathcurveto{\pgfqpoint{0.602831in}{2.703823in}}{\pgfqpoint{0.594931in}{2.700551in}}{\pgfqpoint{0.589107in}{2.694727in}}%
\pgfpathcurveto{\pgfqpoint{0.583283in}{2.688903in}}{\pgfqpoint{0.580010in}{2.681003in}}{\pgfqpoint{0.580010in}{2.672767in}}%
\pgfpathcurveto{\pgfqpoint{0.580010in}{2.664531in}}{\pgfqpoint{0.583283in}{2.656630in}}{\pgfqpoint{0.589107in}{2.650807in}}%
\pgfpathcurveto{\pgfqpoint{0.594931in}{2.644983in}}{\pgfqpoint{0.602831in}{2.641710in}}{\pgfqpoint{0.611067in}{2.641710in}}%
\pgfpathclose%
\pgfusepath{stroke,fill}%
\end{pgfscope}%
\begin{pgfscope}%
\pgfpathrectangle{\pgfqpoint{0.100000in}{0.212622in}}{\pgfqpoint{3.696000in}{3.696000in}}%
\pgfusepath{clip}%
\pgfsetbuttcap%
\pgfsetroundjoin%
\definecolor{currentfill}{rgb}{0.121569,0.466667,0.705882}%
\pgfsetfillcolor{currentfill}%
\pgfsetfillopacity{0.822134}%
\pgfsetlinewidth{1.003750pt}%
\definecolor{currentstroke}{rgb}{0.121569,0.466667,0.705882}%
\pgfsetstrokecolor{currentstroke}%
\pgfsetstrokeopacity{0.822134}%
\pgfsetdash{}{0pt}%
\pgfpathmoveto{\pgfqpoint{0.608365in}{2.640646in}}%
\pgfpathcurveto{\pgfqpoint{0.616601in}{2.640646in}}{\pgfqpoint{0.624501in}{2.643919in}}{\pgfqpoint{0.630325in}{2.649743in}}%
\pgfpathcurveto{\pgfqpoint{0.636149in}{2.655567in}}{\pgfqpoint{0.639421in}{2.663467in}}{\pgfqpoint{0.639421in}{2.671703in}}%
\pgfpathcurveto{\pgfqpoint{0.639421in}{2.679939in}}{\pgfqpoint{0.636149in}{2.687839in}}{\pgfqpoint{0.630325in}{2.693663in}}%
\pgfpathcurveto{\pgfqpoint{0.624501in}{2.699487in}}{\pgfqpoint{0.616601in}{2.702759in}}{\pgfqpoint{0.608365in}{2.702759in}}%
\pgfpathcurveto{\pgfqpoint{0.600129in}{2.702759in}}{\pgfqpoint{0.592229in}{2.699487in}}{\pgfqpoint{0.586405in}{2.693663in}}%
\pgfpathcurveto{\pgfqpoint{0.580581in}{2.687839in}}{\pgfqpoint{0.577308in}{2.679939in}}{\pgfqpoint{0.577308in}{2.671703in}}%
\pgfpathcurveto{\pgfqpoint{0.577308in}{2.663467in}}{\pgfqpoint{0.580581in}{2.655567in}}{\pgfqpoint{0.586405in}{2.649743in}}%
\pgfpathcurveto{\pgfqpoint{0.592229in}{2.643919in}}{\pgfqpoint{0.600129in}{2.640646in}}{\pgfqpoint{0.608365in}{2.640646in}}%
\pgfpathclose%
\pgfusepath{stroke,fill}%
\end{pgfscope}%
\begin{pgfscope}%
\pgfpathrectangle{\pgfqpoint{0.100000in}{0.212622in}}{\pgfqpoint{3.696000in}{3.696000in}}%
\pgfusepath{clip}%
\pgfsetbuttcap%
\pgfsetroundjoin%
\definecolor{currentfill}{rgb}{0.121569,0.466667,0.705882}%
\pgfsetfillcolor{currentfill}%
\pgfsetfillopacity{0.822135}%
\pgfsetlinewidth{1.003750pt}%
\definecolor{currentstroke}{rgb}{0.121569,0.466667,0.705882}%
\pgfsetstrokecolor{currentstroke}%
\pgfsetstrokeopacity{0.822135}%
\pgfsetdash{}{0pt}%
\pgfpathmoveto{\pgfqpoint{0.642282in}{2.612643in}}%
\pgfpathcurveto{\pgfqpoint{0.650518in}{2.612643in}}{\pgfqpoint{0.658418in}{2.615915in}}{\pgfqpoint{0.664242in}{2.621739in}}%
\pgfpathcurveto{\pgfqpoint{0.670066in}{2.627563in}}{\pgfqpoint{0.673338in}{2.635463in}}{\pgfqpoint{0.673338in}{2.643699in}}%
\pgfpathcurveto{\pgfqpoint{0.673338in}{2.651936in}}{\pgfqpoint{0.670066in}{2.659836in}}{\pgfqpoint{0.664242in}{2.665660in}}%
\pgfpathcurveto{\pgfqpoint{0.658418in}{2.671483in}}{\pgfqpoint{0.650518in}{2.674756in}}{\pgfqpoint{0.642282in}{2.674756in}}%
\pgfpathcurveto{\pgfqpoint{0.634045in}{2.674756in}}{\pgfqpoint{0.626145in}{2.671483in}}{\pgfqpoint{0.620321in}{2.665660in}}%
\pgfpathcurveto{\pgfqpoint{0.614498in}{2.659836in}}{\pgfqpoint{0.611225in}{2.651936in}}{\pgfqpoint{0.611225in}{2.643699in}}%
\pgfpathcurveto{\pgfqpoint{0.611225in}{2.635463in}}{\pgfqpoint{0.614498in}{2.627563in}}{\pgfqpoint{0.620321in}{2.621739in}}%
\pgfpathcurveto{\pgfqpoint{0.626145in}{2.615915in}}{\pgfqpoint{0.634045in}{2.612643in}}{\pgfqpoint{0.642282in}{2.612643in}}%
\pgfpathclose%
\pgfusepath{stroke,fill}%
\end{pgfscope}%
\begin{pgfscope}%
\pgfpathrectangle{\pgfqpoint{0.100000in}{0.212622in}}{\pgfqpoint{3.696000in}{3.696000in}}%
\pgfusepath{clip}%
\pgfsetbuttcap%
\pgfsetroundjoin%
\definecolor{currentfill}{rgb}{0.121569,0.466667,0.705882}%
\pgfsetfillcolor{currentfill}%
\pgfsetfillopacity{0.822220}%
\pgfsetlinewidth{1.003750pt}%
\definecolor{currentstroke}{rgb}{0.121569,0.466667,0.705882}%
\pgfsetstrokecolor{currentstroke}%
\pgfsetstrokeopacity{0.822220}%
\pgfsetdash{}{0pt}%
\pgfpathmoveto{\pgfqpoint{2.274345in}{1.553092in}}%
\pgfpathcurveto{\pgfqpoint{2.282581in}{1.553092in}}{\pgfqpoint{2.290481in}{1.556364in}}{\pgfqpoint{2.296305in}{1.562188in}}%
\pgfpathcurveto{\pgfqpoint{2.302129in}{1.568012in}}{\pgfqpoint{2.305401in}{1.575912in}}{\pgfqpoint{2.305401in}{1.584148in}}%
\pgfpathcurveto{\pgfqpoint{2.305401in}{1.592385in}}{\pgfqpoint{2.302129in}{1.600285in}}{\pgfqpoint{2.296305in}{1.606109in}}%
\pgfpathcurveto{\pgfqpoint{2.290481in}{1.611932in}}{\pgfqpoint{2.282581in}{1.615205in}}{\pgfqpoint{2.274345in}{1.615205in}}%
\pgfpathcurveto{\pgfqpoint{2.266109in}{1.615205in}}{\pgfqpoint{2.258209in}{1.611932in}}{\pgfqpoint{2.252385in}{1.606109in}}%
\pgfpathcurveto{\pgfqpoint{2.246561in}{1.600285in}}{\pgfqpoint{2.243288in}{1.592385in}}{\pgfqpoint{2.243288in}{1.584148in}}%
\pgfpathcurveto{\pgfqpoint{2.243288in}{1.575912in}}{\pgfqpoint{2.246561in}{1.568012in}}{\pgfqpoint{2.252385in}{1.562188in}}%
\pgfpathcurveto{\pgfqpoint{2.258209in}{1.556364in}}{\pgfqpoint{2.266109in}{1.553092in}}{\pgfqpoint{2.274345in}{1.553092in}}%
\pgfpathclose%
\pgfusepath{stroke,fill}%
\end{pgfscope}%
\begin{pgfscope}%
\pgfpathrectangle{\pgfqpoint{0.100000in}{0.212622in}}{\pgfqpoint{3.696000in}{3.696000in}}%
\pgfusepath{clip}%
\pgfsetbuttcap%
\pgfsetroundjoin%
\definecolor{currentfill}{rgb}{0.121569,0.466667,0.705882}%
\pgfsetfillcolor{currentfill}%
\pgfsetfillopacity{0.822728}%
\pgfsetlinewidth{1.003750pt}%
\definecolor{currentstroke}{rgb}{0.121569,0.466667,0.705882}%
\pgfsetstrokecolor{currentstroke}%
\pgfsetstrokeopacity{0.822728}%
\pgfsetdash{}{0pt}%
\pgfpathmoveto{\pgfqpoint{0.604784in}{2.637979in}}%
\pgfpathcurveto{\pgfqpoint{0.613021in}{2.637979in}}{\pgfqpoint{0.620921in}{2.641251in}}{\pgfqpoint{0.626745in}{2.647075in}}%
\pgfpathcurveto{\pgfqpoint{0.632569in}{2.652899in}}{\pgfqpoint{0.635841in}{2.660799in}}{\pgfqpoint{0.635841in}{2.669036in}}%
\pgfpathcurveto{\pgfqpoint{0.635841in}{2.677272in}}{\pgfqpoint{0.632569in}{2.685172in}}{\pgfqpoint{0.626745in}{2.690996in}}%
\pgfpathcurveto{\pgfqpoint{0.620921in}{2.696820in}}{\pgfqpoint{0.613021in}{2.700092in}}{\pgfqpoint{0.604784in}{2.700092in}}%
\pgfpathcurveto{\pgfqpoint{0.596548in}{2.700092in}}{\pgfqpoint{0.588648in}{2.696820in}}{\pgfqpoint{0.582824in}{2.690996in}}%
\pgfpathcurveto{\pgfqpoint{0.577000in}{2.685172in}}{\pgfqpoint{0.573728in}{2.677272in}}{\pgfqpoint{0.573728in}{2.669036in}}%
\pgfpathcurveto{\pgfqpoint{0.573728in}{2.660799in}}{\pgfqpoint{0.577000in}{2.652899in}}{\pgfqpoint{0.582824in}{2.647075in}}%
\pgfpathcurveto{\pgfqpoint{0.588648in}{2.641251in}}{\pgfqpoint{0.596548in}{2.637979in}}{\pgfqpoint{0.604784in}{2.637979in}}%
\pgfpathclose%
\pgfusepath{stroke,fill}%
\end{pgfscope}%
\begin{pgfscope}%
\pgfpathrectangle{\pgfqpoint{0.100000in}{0.212622in}}{\pgfqpoint{3.696000in}{3.696000in}}%
\pgfusepath{clip}%
\pgfsetbuttcap%
\pgfsetroundjoin%
\definecolor{currentfill}{rgb}{0.121569,0.466667,0.705882}%
\pgfsetfillcolor{currentfill}%
\pgfsetfillopacity{0.822989}%
\pgfsetlinewidth{1.003750pt}%
\definecolor{currentstroke}{rgb}{0.121569,0.466667,0.705882}%
\pgfsetstrokecolor{currentstroke}%
\pgfsetstrokeopacity{0.822989}%
\pgfsetdash{}{0pt}%
\pgfpathmoveto{\pgfqpoint{2.275129in}{1.551880in}}%
\pgfpathcurveto{\pgfqpoint{2.283365in}{1.551880in}}{\pgfqpoint{2.291265in}{1.555152in}}{\pgfqpoint{2.297089in}{1.560976in}}%
\pgfpathcurveto{\pgfqpoint{2.302913in}{1.566800in}}{\pgfqpoint{2.306186in}{1.574700in}}{\pgfqpoint{2.306186in}{1.582936in}}%
\pgfpathcurveto{\pgfqpoint{2.306186in}{1.591172in}}{\pgfqpoint{2.302913in}{1.599073in}}{\pgfqpoint{2.297089in}{1.604896in}}%
\pgfpathcurveto{\pgfqpoint{2.291265in}{1.610720in}}{\pgfqpoint{2.283365in}{1.613993in}}{\pgfqpoint{2.275129in}{1.613993in}}%
\pgfpathcurveto{\pgfqpoint{2.266893in}{1.613993in}}{\pgfqpoint{2.258993in}{1.610720in}}{\pgfqpoint{2.253169in}{1.604896in}}%
\pgfpathcurveto{\pgfqpoint{2.247345in}{1.599073in}}{\pgfqpoint{2.244073in}{1.591172in}}{\pgfqpoint{2.244073in}{1.582936in}}%
\pgfpathcurveto{\pgfqpoint{2.244073in}{1.574700in}}{\pgfqpoint{2.247345in}{1.566800in}}{\pgfqpoint{2.253169in}{1.560976in}}%
\pgfpathcurveto{\pgfqpoint{2.258993in}{1.555152in}}{\pgfqpoint{2.266893in}{1.551880in}}{\pgfqpoint{2.275129in}{1.551880in}}%
\pgfpathclose%
\pgfusepath{stroke,fill}%
\end{pgfscope}%
\begin{pgfscope}%
\pgfpathrectangle{\pgfqpoint{0.100000in}{0.212622in}}{\pgfqpoint{3.696000in}{3.696000in}}%
\pgfusepath{clip}%
\pgfsetbuttcap%
\pgfsetroundjoin%
\definecolor{currentfill}{rgb}{0.121569,0.466667,0.705882}%
\pgfsetfillcolor{currentfill}%
\pgfsetfillopacity{0.823244}%
\pgfsetlinewidth{1.003750pt}%
\definecolor{currentstroke}{rgb}{0.121569,0.466667,0.705882}%
\pgfsetstrokecolor{currentstroke}%
\pgfsetstrokeopacity{0.823244}%
\pgfsetdash{}{0pt}%
\pgfpathmoveto{\pgfqpoint{0.649330in}{2.605832in}}%
\pgfpathcurveto{\pgfqpoint{0.657566in}{2.605832in}}{\pgfqpoint{0.665466in}{2.609104in}}{\pgfqpoint{0.671290in}{2.614928in}}%
\pgfpathcurveto{\pgfqpoint{0.677114in}{2.620752in}}{\pgfqpoint{0.680386in}{2.628652in}}{\pgfqpoint{0.680386in}{2.636888in}}%
\pgfpathcurveto{\pgfqpoint{0.680386in}{2.645124in}}{\pgfqpoint{0.677114in}{2.653025in}}{\pgfqpoint{0.671290in}{2.658848in}}%
\pgfpathcurveto{\pgfqpoint{0.665466in}{2.664672in}}{\pgfqpoint{0.657566in}{2.667945in}}{\pgfqpoint{0.649330in}{2.667945in}}%
\pgfpathcurveto{\pgfqpoint{0.641093in}{2.667945in}}{\pgfqpoint{0.633193in}{2.664672in}}{\pgfqpoint{0.627369in}{2.658848in}}%
\pgfpathcurveto{\pgfqpoint{0.621545in}{2.653025in}}{\pgfqpoint{0.618273in}{2.645124in}}{\pgfqpoint{0.618273in}{2.636888in}}%
\pgfpathcurveto{\pgfqpoint{0.618273in}{2.628652in}}{\pgfqpoint{0.621545in}{2.620752in}}{\pgfqpoint{0.627369in}{2.614928in}}%
\pgfpathcurveto{\pgfqpoint{0.633193in}{2.609104in}}{\pgfqpoint{0.641093in}{2.605832in}}{\pgfqpoint{0.649330in}{2.605832in}}%
\pgfpathclose%
\pgfusepath{stroke,fill}%
\end{pgfscope}%
\begin{pgfscope}%
\pgfpathrectangle{\pgfqpoint{0.100000in}{0.212622in}}{\pgfqpoint{3.696000in}{3.696000in}}%
\pgfusepath{clip}%
\pgfsetbuttcap%
\pgfsetroundjoin%
\definecolor{currentfill}{rgb}{0.121569,0.466667,0.705882}%
\pgfsetfillcolor{currentfill}%
\pgfsetfillopacity{0.823514}%
\pgfsetlinewidth{1.003750pt}%
\definecolor{currentstroke}{rgb}{0.121569,0.466667,0.705882}%
\pgfsetstrokecolor{currentstroke}%
\pgfsetstrokeopacity{0.823514}%
\pgfsetdash{}{0pt}%
\pgfpathmoveto{\pgfqpoint{2.275549in}{1.551835in}}%
\pgfpathcurveto{\pgfqpoint{2.283786in}{1.551835in}}{\pgfqpoint{2.291686in}{1.555107in}}{\pgfqpoint{2.297510in}{1.560931in}}%
\pgfpathcurveto{\pgfqpoint{2.303334in}{1.566755in}}{\pgfqpoint{2.306606in}{1.574655in}}{\pgfqpoint{2.306606in}{1.582891in}}%
\pgfpathcurveto{\pgfqpoint{2.306606in}{1.591128in}}{\pgfqpoint{2.303334in}{1.599028in}}{\pgfqpoint{2.297510in}{1.604852in}}%
\pgfpathcurveto{\pgfqpoint{2.291686in}{1.610676in}}{\pgfqpoint{2.283786in}{1.613948in}}{\pgfqpoint{2.275549in}{1.613948in}}%
\pgfpathcurveto{\pgfqpoint{2.267313in}{1.613948in}}{\pgfqpoint{2.259413in}{1.610676in}}{\pgfqpoint{2.253589in}{1.604852in}}%
\pgfpathcurveto{\pgfqpoint{2.247765in}{1.599028in}}{\pgfqpoint{2.244493in}{1.591128in}}{\pgfqpoint{2.244493in}{1.582891in}}%
\pgfpathcurveto{\pgfqpoint{2.244493in}{1.574655in}}{\pgfqpoint{2.247765in}{1.566755in}}{\pgfqpoint{2.253589in}{1.560931in}}%
\pgfpathcurveto{\pgfqpoint{2.259413in}{1.555107in}}{\pgfqpoint{2.267313in}{1.551835in}}{\pgfqpoint{2.275549in}{1.551835in}}%
\pgfpathclose%
\pgfusepath{stroke,fill}%
\end{pgfscope}%
\begin{pgfscope}%
\pgfpathrectangle{\pgfqpoint{0.100000in}{0.212622in}}{\pgfqpoint{3.696000in}{3.696000in}}%
\pgfusepath{clip}%
\pgfsetbuttcap%
\pgfsetroundjoin%
\definecolor{currentfill}{rgb}{0.121569,0.466667,0.705882}%
\pgfsetfillcolor{currentfill}%
\pgfsetfillopacity{0.824346}%
\pgfsetlinewidth{1.003750pt}%
\definecolor{currentstroke}{rgb}{0.121569,0.466667,0.705882}%
\pgfsetstrokecolor{currentstroke}%
\pgfsetstrokeopacity{0.824346}%
\pgfsetdash{}{0pt}%
\pgfpathmoveto{\pgfqpoint{2.276037in}{1.551452in}}%
\pgfpathcurveto{\pgfqpoint{2.284273in}{1.551452in}}{\pgfqpoint{2.292173in}{1.554724in}}{\pgfqpoint{2.297997in}{1.560548in}}%
\pgfpathcurveto{\pgfqpoint{2.303821in}{1.566372in}}{\pgfqpoint{2.307094in}{1.574272in}}{\pgfqpoint{2.307094in}{1.582508in}}%
\pgfpathcurveto{\pgfqpoint{2.307094in}{1.590744in}}{\pgfqpoint{2.303821in}{1.598645in}}{\pgfqpoint{2.297997in}{1.604468in}}%
\pgfpathcurveto{\pgfqpoint{2.292173in}{1.610292in}}{\pgfqpoint{2.284273in}{1.613565in}}{\pgfqpoint{2.276037in}{1.613565in}}%
\pgfpathcurveto{\pgfqpoint{2.267801in}{1.613565in}}{\pgfqpoint{2.259901in}{1.610292in}}{\pgfqpoint{2.254077in}{1.604468in}}%
\pgfpathcurveto{\pgfqpoint{2.248253in}{1.598645in}}{\pgfqpoint{2.244981in}{1.590744in}}{\pgfqpoint{2.244981in}{1.582508in}}%
\pgfpathcurveto{\pgfqpoint{2.244981in}{1.574272in}}{\pgfqpoint{2.248253in}{1.566372in}}{\pgfqpoint{2.254077in}{1.560548in}}%
\pgfpathcurveto{\pgfqpoint{2.259901in}{1.554724in}}{\pgfqpoint{2.267801in}{1.551452in}}{\pgfqpoint{2.276037in}{1.551452in}}%
\pgfpathclose%
\pgfusepath{stroke,fill}%
\end{pgfscope}%
\begin{pgfscope}%
\pgfpathrectangle{\pgfqpoint{0.100000in}{0.212622in}}{\pgfqpoint{3.696000in}{3.696000in}}%
\pgfusepath{clip}%
\pgfsetbuttcap%
\pgfsetroundjoin%
\definecolor{currentfill}{rgb}{0.121569,0.466667,0.705882}%
\pgfsetfillcolor{currentfill}%
\pgfsetfillopacity{0.824521}%
\pgfsetlinewidth{1.003750pt}%
\definecolor{currentstroke}{rgb}{0.121569,0.466667,0.705882}%
\pgfsetstrokecolor{currentstroke}%
\pgfsetstrokeopacity{0.824521}%
\pgfsetdash{}{0pt}%
\pgfpathmoveto{\pgfqpoint{0.655737in}{2.602109in}}%
\pgfpathcurveto{\pgfqpoint{0.663973in}{2.602109in}}{\pgfqpoint{0.671874in}{2.605382in}}{\pgfqpoint{0.677697in}{2.611206in}}%
\pgfpathcurveto{\pgfqpoint{0.683521in}{2.617030in}}{\pgfqpoint{0.686794in}{2.624930in}}{\pgfqpoint{0.686794in}{2.633166in}}%
\pgfpathcurveto{\pgfqpoint{0.686794in}{2.641402in}}{\pgfqpoint{0.683521in}{2.649302in}}{\pgfqpoint{0.677697in}{2.655126in}}%
\pgfpathcurveto{\pgfqpoint{0.671874in}{2.660950in}}{\pgfqpoint{0.663973in}{2.664222in}}{\pgfqpoint{0.655737in}{2.664222in}}%
\pgfpathcurveto{\pgfqpoint{0.647501in}{2.664222in}}{\pgfqpoint{0.639601in}{2.660950in}}{\pgfqpoint{0.633777in}{2.655126in}}%
\pgfpathcurveto{\pgfqpoint{0.627953in}{2.649302in}}{\pgfqpoint{0.624681in}{2.641402in}}{\pgfqpoint{0.624681in}{2.633166in}}%
\pgfpathcurveto{\pgfqpoint{0.624681in}{2.624930in}}{\pgfqpoint{0.627953in}{2.617030in}}{\pgfqpoint{0.633777in}{2.611206in}}%
\pgfpathcurveto{\pgfqpoint{0.639601in}{2.605382in}}{\pgfqpoint{0.647501in}{2.602109in}}{\pgfqpoint{0.655737in}{2.602109in}}%
\pgfpathclose%
\pgfusepath{stroke,fill}%
\end{pgfscope}%
\begin{pgfscope}%
\pgfpathrectangle{\pgfqpoint{0.100000in}{0.212622in}}{\pgfqpoint{3.696000in}{3.696000in}}%
\pgfusepath{clip}%
\pgfsetbuttcap%
\pgfsetroundjoin%
\definecolor{currentfill}{rgb}{0.121569,0.466667,0.705882}%
\pgfsetfillcolor{currentfill}%
\pgfsetfillopacity{0.825157}%
\pgfsetlinewidth{1.003750pt}%
\definecolor{currentstroke}{rgb}{0.121569,0.466667,0.705882}%
\pgfsetstrokecolor{currentstroke}%
\pgfsetstrokeopacity{0.825157}%
\pgfsetdash{}{0pt}%
\pgfpathmoveto{\pgfqpoint{2.276707in}{1.549811in}}%
\pgfpathcurveto{\pgfqpoint{2.284944in}{1.549811in}}{\pgfqpoint{2.292844in}{1.553083in}}{\pgfqpoint{2.298668in}{1.558907in}}%
\pgfpathcurveto{\pgfqpoint{2.304492in}{1.564731in}}{\pgfqpoint{2.307764in}{1.572631in}}{\pgfqpoint{2.307764in}{1.580867in}}%
\pgfpathcurveto{\pgfqpoint{2.307764in}{1.589103in}}{\pgfqpoint{2.304492in}{1.597003in}}{\pgfqpoint{2.298668in}{1.602827in}}%
\pgfpathcurveto{\pgfqpoint{2.292844in}{1.608651in}}{\pgfqpoint{2.284944in}{1.611924in}}{\pgfqpoint{2.276707in}{1.611924in}}%
\pgfpathcurveto{\pgfqpoint{2.268471in}{1.611924in}}{\pgfqpoint{2.260571in}{1.608651in}}{\pgfqpoint{2.254747in}{1.602827in}}%
\pgfpathcurveto{\pgfqpoint{2.248923in}{1.597003in}}{\pgfqpoint{2.245651in}{1.589103in}}{\pgfqpoint{2.245651in}{1.580867in}}%
\pgfpathcurveto{\pgfqpoint{2.245651in}{1.572631in}}{\pgfqpoint{2.248923in}{1.564731in}}{\pgfqpoint{2.254747in}{1.558907in}}%
\pgfpathcurveto{\pgfqpoint{2.260571in}{1.553083in}}{\pgfqpoint{2.268471in}{1.549811in}}{\pgfqpoint{2.276707in}{1.549811in}}%
\pgfpathclose%
\pgfusepath{stroke,fill}%
\end{pgfscope}%
\begin{pgfscope}%
\pgfpathrectangle{\pgfqpoint{0.100000in}{0.212622in}}{\pgfqpoint{3.696000in}{3.696000in}}%
\pgfusepath{clip}%
\pgfsetbuttcap%
\pgfsetroundjoin%
\definecolor{currentfill}{rgb}{0.121569,0.466667,0.705882}%
\pgfsetfillcolor{currentfill}%
\pgfsetfillopacity{0.825850}%
\pgfsetlinewidth{1.003750pt}%
\definecolor{currentstroke}{rgb}{0.121569,0.466667,0.705882}%
\pgfsetstrokecolor{currentstroke}%
\pgfsetstrokeopacity{0.825850}%
\pgfsetdash{}{0pt}%
\pgfpathmoveto{\pgfqpoint{0.668211in}{2.590425in}}%
\pgfpathcurveto{\pgfqpoint{0.676447in}{2.590425in}}{\pgfqpoint{0.684347in}{2.593697in}}{\pgfqpoint{0.690171in}{2.599521in}}%
\pgfpathcurveto{\pgfqpoint{0.695995in}{2.605345in}}{\pgfqpoint{0.699268in}{2.613245in}}{\pgfqpoint{0.699268in}{2.621481in}}%
\pgfpathcurveto{\pgfqpoint{0.699268in}{2.629718in}}{\pgfqpoint{0.695995in}{2.637618in}}{\pgfqpoint{0.690171in}{2.643442in}}%
\pgfpathcurveto{\pgfqpoint{0.684347in}{2.649266in}}{\pgfqpoint{0.676447in}{2.652538in}}{\pgfqpoint{0.668211in}{2.652538in}}%
\pgfpathcurveto{\pgfqpoint{0.659975in}{2.652538in}}{\pgfqpoint{0.652075in}{2.649266in}}{\pgfqpoint{0.646251in}{2.643442in}}%
\pgfpathcurveto{\pgfqpoint{0.640427in}{2.637618in}}{\pgfqpoint{0.637155in}{2.629718in}}{\pgfqpoint{0.637155in}{2.621481in}}%
\pgfpathcurveto{\pgfqpoint{0.637155in}{2.613245in}}{\pgfqpoint{0.640427in}{2.605345in}}{\pgfqpoint{0.646251in}{2.599521in}}%
\pgfpathcurveto{\pgfqpoint{0.652075in}{2.593697in}}{\pgfqpoint{0.659975in}{2.590425in}}{\pgfqpoint{0.668211in}{2.590425in}}%
\pgfpathclose%
\pgfusepath{stroke,fill}%
\end{pgfscope}%
\begin{pgfscope}%
\pgfpathrectangle{\pgfqpoint{0.100000in}{0.212622in}}{\pgfqpoint{3.696000in}{3.696000in}}%
\pgfusepath{clip}%
\pgfsetbuttcap%
\pgfsetroundjoin%
\definecolor{currentfill}{rgb}{0.121569,0.466667,0.705882}%
\pgfsetfillcolor{currentfill}%
\pgfsetfillopacity{0.826165}%
\pgfsetlinewidth{1.003750pt}%
\definecolor{currentstroke}{rgb}{0.121569,0.466667,0.705882}%
\pgfsetstrokecolor{currentstroke}%
\pgfsetstrokeopacity{0.826165}%
\pgfsetdash{}{0pt}%
\pgfpathmoveto{\pgfqpoint{2.278159in}{1.545868in}}%
\pgfpathcurveto{\pgfqpoint{2.286395in}{1.545868in}}{\pgfqpoint{2.294295in}{1.549140in}}{\pgfqpoint{2.300119in}{1.554964in}}%
\pgfpathcurveto{\pgfqpoint{2.305943in}{1.560788in}}{\pgfqpoint{2.309215in}{1.568688in}}{\pgfqpoint{2.309215in}{1.576925in}}%
\pgfpathcurveto{\pgfqpoint{2.309215in}{1.585161in}}{\pgfqpoint{2.305943in}{1.593061in}}{\pgfqpoint{2.300119in}{1.598885in}}%
\pgfpathcurveto{\pgfqpoint{2.294295in}{1.604709in}}{\pgfqpoint{2.286395in}{1.607981in}}{\pgfqpoint{2.278159in}{1.607981in}}%
\pgfpathcurveto{\pgfqpoint{2.269922in}{1.607981in}}{\pgfqpoint{2.262022in}{1.604709in}}{\pgfqpoint{2.256198in}{1.598885in}}%
\pgfpathcurveto{\pgfqpoint{2.250374in}{1.593061in}}{\pgfqpoint{2.247102in}{1.585161in}}{\pgfqpoint{2.247102in}{1.576925in}}%
\pgfpathcurveto{\pgfqpoint{2.247102in}{1.568688in}}{\pgfqpoint{2.250374in}{1.560788in}}{\pgfqpoint{2.256198in}{1.554964in}}%
\pgfpathcurveto{\pgfqpoint{2.262022in}{1.549140in}}{\pgfqpoint{2.269922in}{1.545868in}}{\pgfqpoint{2.278159in}{1.545868in}}%
\pgfpathclose%
\pgfusepath{stroke,fill}%
\end{pgfscope}%
\begin{pgfscope}%
\pgfpathrectangle{\pgfqpoint{0.100000in}{0.212622in}}{\pgfqpoint{3.696000in}{3.696000in}}%
\pgfusepath{clip}%
\pgfsetbuttcap%
\pgfsetroundjoin%
\definecolor{currentfill}{rgb}{0.121569,0.466667,0.705882}%
\pgfsetfillcolor{currentfill}%
\pgfsetfillopacity{0.828077}%
\pgfsetlinewidth{1.003750pt}%
\definecolor{currentstroke}{rgb}{0.121569,0.466667,0.705882}%
\pgfsetstrokecolor{currentstroke}%
\pgfsetstrokeopacity{0.828077}%
\pgfsetdash{}{0pt}%
\pgfpathmoveto{\pgfqpoint{0.679892in}{2.584867in}}%
\pgfpathcurveto{\pgfqpoint{0.688129in}{2.584867in}}{\pgfqpoint{0.696029in}{2.588139in}}{\pgfqpoint{0.701853in}{2.593963in}}%
\pgfpathcurveto{\pgfqpoint{0.707676in}{2.599787in}}{\pgfqpoint{0.710949in}{2.607687in}}{\pgfqpoint{0.710949in}{2.615924in}}%
\pgfpathcurveto{\pgfqpoint{0.710949in}{2.624160in}}{\pgfqpoint{0.707676in}{2.632060in}}{\pgfqpoint{0.701853in}{2.637884in}}%
\pgfpathcurveto{\pgfqpoint{0.696029in}{2.643708in}}{\pgfqpoint{0.688129in}{2.646980in}}{\pgfqpoint{0.679892in}{2.646980in}}%
\pgfpathcurveto{\pgfqpoint{0.671656in}{2.646980in}}{\pgfqpoint{0.663756in}{2.643708in}}{\pgfqpoint{0.657932in}{2.637884in}}%
\pgfpathcurveto{\pgfqpoint{0.652108in}{2.632060in}}{\pgfqpoint{0.648836in}{2.624160in}}{\pgfqpoint{0.648836in}{2.615924in}}%
\pgfpathcurveto{\pgfqpoint{0.648836in}{2.607687in}}{\pgfqpoint{0.652108in}{2.599787in}}{\pgfqpoint{0.657932in}{2.593963in}}%
\pgfpathcurveto{\pgfqpoint{0.663756in}{2.588139in}}{\pgfqpoint{0.671656in}{2.584867in}}{\pgfqpoint{0.679892in}{2.584867in}}%
\pgfpathclose%
\pgfusepath{stroke,fill}%
\end{pgfscope}%
\begin{pgfscope}%
\pgfpathrectangle{\pgfqpoint{0.100000in}{0.212622in}}{\pgfqpoint{3.696000in}{3.696000in}}%
\pgfusepath{clip}%
\pgfsetbuttcap%
\pgfsetroundjoin%
\definecolor{currentfill}{rgb}{0.121569,0.466667,0.705882}%
\pgfsetfillcolor{currentfill}%
\pgfsetfillopacity{0.828215}%
\pgfsetlinewidth{1.003750pt}%
\definecolor{currentstroke}{rgb}{0.121569,0.466667,0.705882}%
\pgfsetstrokecolor{currentstroke}%
\pgfsetstrokeopacity{0.828215}%
\pgfsetdash{}{0pt}%
\pgfpathmoveto{\pgfqpoint{2.279837in}{1.543018in}}%
\pgfpathcurveto{\pgfqpoint{2.288073in}{1.543018in}}{\pgfqpoint{2.295973in}{1.546290in}}{\pgfqpoint{2.301797in}{1.552114in}}%
\pgfpathcurveto{\pgfqpoint{2.307621in}{1.557938in}}{\pgfqpoint{2.310894in}{1.565838in}}{\pgfqpoint{2.310894in}{1.574074in}}%
\pgfpathcurveto{\pgfqpoint{2.310894in}{1.582311in}}{\pgfqpoint{2.307621in}{1.590211in}}{\pgfqpoint{2.301797in}{1.596035in}}%
\pgfpathcurveto{\pgfqpoint{2.295973in}{1.601858in}}{\pgfqpoint{2.288073in}{1.605131in}}{\pgfqpoint{2.279837in}{1.605131in}}%
\pgfpathcurveto{\pgfqpoint{2.271601in}{1.605131in}}{\pgfqpoint{2.263701in}{1.601858in}}{\pgfqpoint{2.257877in}{1.596035in}}%
\pgfpathcurveto{\pgfqpoint{2.252053in}{1.590211in}}{\pgfqpoint{2.248781in}{1.582311in}}{\pgfqpoint{2.248781in}{1.574074in}}%
\pgfpathcurveto{\pgfqpoint{2.248781in}{1.565838in}}{\pgfqpoint{2.252053in}{1.557938in}}{\pgfqpoint{2.257877in}{1.552114in}}%
\pgfpathcurveto{\pgfqpoint{2.263701in}{1.546290in}}{\pgfqpoint{2.271601in}{1.543018in}}{\pgfqpoint{2.279837in}{1.543018in}}%
\pgfpathclose%
\pgfusepath{stroke,fill}%
\end{pgfscope}%
\begin{pgfscope}%
\pgfpathrectangle{\pgfqpoint{0.100000in}{0.212622in}}{\pgfqpoint{3.696000in}{3.696000in}}%
\pgfusepath{clip}%
\pgfsetbuttcap%
\pgfsetroundjoin%
\definecolor{currentfill}{rgb}{0.121569,0.466667,0.705882}%
\pgfsetfillcolor{currentfill}%
\pgfsetfillopacity{0.829644}%
\pgfsetlinewidth{1.003750pt}%
\definecolor{currentstroke}{rgb}{0.121569,0.466667,0.705882}%
\pgfsetstrokecolor{currentstroke}%
\pgfsetstrokeopacity{0.829644}%
\pgfsetdash{}{0pt}%
\pgfpathmoveto{\pgfqpoint{0.702792in}{2.561761in}}%
\pgfpathcurveto{\pgfqpoint{0.711029in}{2.561761in}}{\pgfqpoint{0.718929in}{2.565033in}}{\pgfqpoint{0.724753in}{2.570857in}}%
\pgfpathcurveto{\pgfqpoint{0.730577in}{2.576681in}}{\pgfqpoint{0.733849in}{2.584581in}}{\pgfqpoint{0.733849in}{2.592817in}}%
\pgfpathcurveto{\pgfqpoint{0.733849in}{2.601054in}}{\pgfqpoint{0.730577in}{2.608954in}}{\pgfqpoint{0.724753in}{2.614778in}}%
\pgfpathcurveto{\pgfqpoint{0.718929in}{2.620602in}}{\pgfqpoint{0.711029in}{2.623874in}}{\pgfqpoint{0.702792in}{2.623874in}}%
\pgfpathcurveto{\pgfqpoint{0.694556in}{2.623874in}}{\pgfqpoint{0.686656in}{2.620602in}}{\pgfqpoint{0.680832in}{2.614778in}}%
\pgfpathcurveto{\pgfqpoint{0.675008in}{2.608954in}}{\pgfqpoint{0.671736in}{2.601054in}}{\pgfqpoint{0.671736in}{2.592817in}}%
\pgfpathcurveto{\pgfqpoint{0.671736in}{2.584581in}}{\pgfqpoint{0.675008in}{2.576681in}}{\pgfqpoint{0.680832in}{2.570857in}}%
\pgfpathcurveto{\pgfqpoint{0.686656in}{2.565033in}}{\pgfqpoint{0.694556in}{2.561761in}}{\pgfqpoint{0.702792in}{2.561761in}}%
\pgfpathclose%
\pgfusepath{stroke,fill}%
\end{pgfscope}%
\begin{pgfscope}%
\pgfpathrectangle{\pgfqpoint{0.100000in}{0.212622in}}{\pgfqpoint{3.696000in}{3.696000in}}%
\pgfusepath{clip}%
\pgfsetbuttcap%
\pgfsetroundjoin%
\definecolor{currentfill}{rgb}{0.121569,0.466667,0.705882}%
\pgfsetfillcolor{currentfill}%
\pgfsetfillopacity{0.831031}%
\pgfsetlinewidth{1.003750pt}%
\definecolor{currentstroke}{rgb}{0.121569,0.466667,0.705882}%
\pgfsetstrokecolor{currentstroke}%
\pgfsetstrokeopacity{0.831031}%
\pgfsetdash{}{0pt}%
\pgfpathmoveto{\pgfqpoint{2.280848in}{1.540940in}}%
\pgfpathcurveto{\pgfqpoint{2.289085in}{1.540940in}}{\pgfqpoint{2.296985in}{1.544212in}}{\pgfqpoint{2.302809in}{1.550036in}}%
\pgfpathcurveto{\pgfqpoint{2.308632in}{1.555860in}}{\pgfqpoint{2.311905in}{1.563760in}}{\pgfqpoint{2.311905in}{1.571996in}}%
\pgfpathcurveto{\pgfqpoint{2.311905in}{1.580232in}}{\pgfqpoint{2.308632in}{1.588132in}}{\pgfqpoint{2.302809in}{1.593956in}}%
\pgfpathcurveto{\pgfqpoint{2.296985in}{1.599780in}}{\pgfqpoint{2.289085in}{1.603053in}}{\pgfqpoint{2.280848in}{1.603053in}}%
\pgfpathcurveto{\pgfqpoint{2.272612in}{1.603053in}}{\pgfqpoint{2.264712in}{1.599780in}}{\pgfqpoint{2.258888in}{1.593956in}}%
\pgfpathcurveto{\pgfqpoint{2.253064in}{1.588132in}}{\pgfqpoint{2.249792in}{1.580232in}}{\pgfqpoint{2.249792in}{1.571996in}}%
\pgfpathcurveto{\pgfqpoint{2.249792in}{1.563760in}}{\pgfqpoint{2.253064in}{1.555860in}}{\pgfqpoint{2.258888in}{1.550036in}}%
\pgfpathcurveto{\pgfqpoint{2.264712in}{1.544212in}}{\pgfqpoint{2.272612in}{1.540940in}}{\pgfqpoint{2.280848in}{1.540940in}}%
\pgfpathclose%
\pgfusepath{stroke,fill}%
\end{pgfscope}%
\begin{pgfscope}%
\pgfpathrectangle{\pgfqpoint{0.100000in}{0.212622in}}{\pgfqpoint{3.696000in}{3.696000in}}%
\pgfusepath{clip}%
\pgfsetbuttcap%
\pgfsetroundjoin%
\definecolor{currentfill}{rgb}{0.121569,0.466667,0.705882}%
\pgfsetfillcolor{currentfill}%
\pgfsetfillopacity{0.832474}%
\pgfsetlinewidth{1.003750pt}%
\definecolor{currentstroke}{rgb}{0.121569,0.466667,0.705882}%
\pgfsetstrokecolor{currentstroke}%
\pgfsetstrokeopacity{0.832474}%
\pgfsetdash{}{0pt}%
\pgfpathmoveto{\pgfqpoint{2.282156in}{1.539519in}}%
\pgfpathcurveto{\pgfqpoint{2.290392in}{1.539519in}}{\pgfqpoint{2.298292in}{1.542791in}}{\pgfqpoint{2.304116in}{1.548615in}}%
\pgfpathcurveto{\pgfqpoint{2.309940in}{1.554439in}}{\pgfqpoint{2.313212in}{1.562339in}}{\pgfqpoint{2.313212in}{1.570575in}}%
\pgfpathcurveto{\pgfqpoint{2.313212in}{1.578812in}}{\pgfqpoint{2.309940in}{1.586712in}}{\pgfqpoint{2.304116in}{1.592536in}}%
\pgfpathcurveto{\pgfqpoint{2.298292in}{1.598360in}}{\pgfqpoint{2.290392in}{1.601632in}}{\pgfqpoint{2.282156in}{1.601632in}}%
\pgfpathcurveto{\pgfqpoint{2.273920in}{1.601632in}}{\pgfqpoint{2.266019in}{1.598360in}}{\pgfqpoint{2.260196in}{1.592536in}}%
\pgfpathcurveto{\pgfqpoint{2.254372in}{1.586712in}}{\pgfqpoint{2.251099in}{1.578812in}}{\pgfqpoint{2.251099in}{1.570575in}}%
\pgfpathcurveto{\pgfqpoint{2.251099in}{1.562339in}}{\pgfqpoint{2.254372in}{1.554439in}}{\pgfqpoint{2.260196in}{1.548615in}}%
\pgfpathcurveto{\pgfqpoint{2.266019in}{1.542791in}}{\pgfqpoint{2.273920in}{1.539519in}}{\pgfqpoint{2.282156in}{1.539519in}}%
\pgfpathclose%
\pgfusepath{stroke,fill}%
\end{pgfscope}%
\begin{pgfscope}%
\pgfpathrectangle{\pgfqpoint{0.100000in}{0.212622in}}{\pgfqpoint{3.696000in}{3.696000in}}%
\pgfusepath{clip}%
\pgfsetbuttcap%
\pgfsetroundjoin%
\definecolor{currentfill}{rgb}{0.121569,0.466667,0.705882}%
\pgfsetfillcolor{currentfill}%
\pgfsetfillopacity{0.832991}%
\pgfsetlinewidth{1.003750pt}%
\definecolor{currentstroke}{rgb}{0.121569,0.466667,0.705882}%
\pgfsetstrokecolor{currentstroke}%
\pgfsetstrokeopacity{0.832991}%
\pgfsetdash{}{0pt}%
\pgfpathmoveto{\pgfqpoint{0.743751in}{2.521171in}}%
\pgfpathcurveto{\pgfqpoint{0.751988in}{2.521171in}}{\pgfqpoint{0.759888in}{2.524443in}}{\pgfqpoint{0.765712in}{2.530267in}}%
\pgfpathcurveto{\pgfqpoint{0.771536in}{2.536091in}}{\pgfqpoint{0.774808in}{2.543991in}}{\pgfqpoint{0.774808in}{2.552228in}}%
\pgfpathcurveto{\pgfqpoint{0.774808in}{2.560464in}}{\pgfqpoint{0.771536in}{2.568364in}}{\pgfqpoint{0.765712in}{2.574188in}}%
\pgfpathcurveto{\pgfqpoint{0.759888in}{2.580012in}}{\pgfqpoint{0.751988in}{2.583284in}}{\pgfqpoint{0.743751in}{2.583284in}}%
\pgfpathcurveto{\pgfqpoint{0.735515in}{2.583284in}}{\pgfqpoint{0.727615in}{2.580012in}}{\pgfqpoint{0.721791in}{2.574188in}}%
\pgfpathcurveto{\pgfqpoint{0.715967in}{2.568364in}}{\pgfqpoint{0.712695in}{2.560464in}}{\pgfqpoint{0.712695in}{2.552228in}}%
\pgfpathcurveto{\pgfqpoint{0.712695in}{2.543991in}}{\pgfqpoint{0.715967in}{2.536091in}}{\pgfqpoint{0.721791in}{2.530267in}}%
\pgfpathcurveto{\pgfqpoint{0.727615in}{2.524443in}}{\pgfqpoint{0.735515in}{2.521171in}}{\pgfqpoint{0.743751in}{2.521171in}}%
\pgfpathclose%
\pgfusepath{stroke,fill}%
\end{pgfscope}%
\begin{pgfscope}%
\pgfpathrectangle{\pgfqpoint{0.100000in}{0.212622in}}{\pgfqpoint{3.696000in}{3.696000in}}%
\pgfusepath{clip}%
\pgfsetbuttcap%
\pgfsetroundjoin%
\definecolor{currentfill}{rgb}{0.121569,0.466667,0.705882}%
\pgfsetfillcolor{currentfill}%
\pgfsetfillopacity{0.834216}%
\pgfsetlinewidth{1.003750pt}%
\definecolor{currentstroke}{rgb}{0.121569,0.466667,0.705882}%
\pgfsetstrokecolor{currentstroke}%
\pgfsetstrokeopacity{0.834216}%
\pgfsetdash{}{0pt}%
\pgfpathmoveto{\pgfqpoint{2.283360in}{1.537150in}}%
\pgfpathcurveto{\pgfqpoint{2.291596in}{1.537150in}}{\pgfqpoint{2.299496in}{1.540422in}}{\pgfqpoint{2.305320in}{1.546246in}}%
\pgfpathcurveto{\pgfqpoint{2.311144in}{1.552070in}}{\pgfqpoint{2.314416in}{1.559970in}}{\pgfqpoint{2.314416in}{1.568206in}}%
\pgfpathcurveto{\pgfqpoint{2.314416in}{1.576443in}}{\pgfqpoint{2.311144in}{1.584343in}}{\pgfqpoint{2.305320in}{1.590167in}}%
\pgfpathcurveto{\pgfqpoint{2.299496in}{1.595991in}}{\pgfqpoint{2.291596in}{1.599263in}}{\pgfqpoint{2.283360in}{1.599263in}}%
\pgfpathcurveto{\pgfqpoint{2.275124in}{1.599263in}}{\pgfqpoint{2.267223in}{1.595991in}}{\pgfqpoint{2.261400in}{1.590167in}}%
\pgfpathcurveto{\pgfqpoint{2.255576in}{1.584343in}}{\pgfqpoint{2.252303in}{1.576443in}}{\pgfqpoint{2.252303in}{1.568206in}}%
\pgfpathcurveto{\pgfqpoint{2.252303in}{1.559970in}}{\pgfqpoint{2.255576in}{1.552070in}}{\pgfqpoint{2.261400in}{1.546246in}}%
\pgfpathcurveto{\pgfqpoint{2.267223in}{1.540422in}}{\pgfqpoint{2.275124in}{1.537150in}}{\pgfqpoint{2.283360in}{1.537150in}}%
\pgfpathclose%
\pgfusepath{stroke,fill}%
\end{pgfscope}%
\begin{pgfscope}%
\pgfpathrectangle{\pgfqpoint{0.100000in}{0.212622in}}{\pgfqpoint{3.696000in}{3.696000in}}%
\pgfusepath{clip}%
\pgfsetbuttcap%
\pgfsetroundjoin%
\definecolor{currentfill}{rgb}{0.121569,0.466667,0.705882}%
\pgfsetfillcolor{currentfill}%
\pgfsetfillopacity{0.835084}%
\pgfsetlinewidth{1.003750pt}%
\definecolor{currentstroke}{rgb}{0.121569,0.466667,0.705882}%
\pgfsetstrokecolor{currentstroke}%
\pgfsetstrokeopacity{0.835084}%
\pgfsetdash{}{0pt}%
\pgfpathmoveto{\pgfqpoint{2.283954in}{1.535256in}}%
\pgfpathcurveto{\pgfqpoint{2.292190in}{1.535256in}}{\pgfqpoint{2.300090in}{1.538529in}}{\pgfqpoint{2.305914in}{1.544353in}}%
\pgfpathcurveto{\pgfqpoint{2.311738in}{1.550177in}}{\pgfqpoint{2.315010in}{1.558077in}}{\pgfqpoint{2.315010in}{1.566313in}}%
\pgfpathcurveto{\pgfqpoint{2.315010in}{1.574549in}}{\pgfqpoint{2.311738in}{1.582449in}}{\pgfqpoint{2.305914in}{1.588273in}}%
\pgfpathcurveto{\pgfqpoint{2.300090in}{1.594097in}}{\pgfqpoint{2.292190in}{1.597369in}}{\pgfqpoint{2.283954in}{1.597369in}}%
\pgfpathcurveto{\pgfqpoint{2.275717in}{1.597369in}}{\pgfqpoint{2.267817in}{1.594097in}}{\pgfqpoint{2.261993in}{1.588273in}}%
\pgfpathcurveto{\pgfqpoint{2.256169in}{1.582449in}}{\pgfqpoint{2.252897in}{1.574549in}}{\pgfqpoint{2.252897in}{1.566313in}}%
\pgfpathcurveto{\pgfqpoint{2.252897in}{1.558077in}}{\pgfqpoint{2.256169in}{1.550177in}}{\pgfqpoint{2.261993in}{1.544353in}}%
\pgfpathcurveto{\pgfqpoint{2.267817in}{1.538529in}}{\pgfqpoint{2.275717in}{1.535256in}}{\pgfqpoint{2.283954in}{1.535256in}}%
\pgfpathclose%
\pgfusepath{stroke,fill}%
\end{pgfscope}%
\begin{pgfscope}%
\pgfpathrectangle{\pgfqpoint{0.100000in}{0.212622in}}{\pgfqpoint{3.696000in}{3.696000in}}%
\pgfusepath{clip}%
\pgfsetbuttcap%
\pgfsetroundjoin%
\definecolor{currentfill}{rgb}{0.121569,0.466667,0.705882}%
\pgfsetfillcolor{currentfill}%
\pgfsetfillopacity{0.836412}%
\pgfsetlinewidth{1.003750pt}%
\definecolor{currentstroke}{rgb}{0.121569,0.466667,0.705882}%
\pgfsetstrokecolor{currentstroke}%
\pgfsetstrokeopacity{0.836412}%
\pgfsetdash{}{0pt}%
\pgfpathmoveto{\pgfqpoint{2.284390in}{1.534508in}}%
\pgfpathcurveto{\pgfqpoint{2.292626in}{1.534508in}}{\pgfqpoint{2.300526in}{1.537781in}}{\pgfqpoint{2.306350in}{1.543605in}}%
\pgfpathcurveto{\pgfqpoint{2.312174in}{1.549428in}}{\pgfqpoint{2.315446in}{1.557329in}}{\pgfqpoint{2.315446in}{1.565565in}}%
\pgfpathcurveto{\pgfqpoint{2.315446in}{1.573801in}}{\pgfqpoint{2.312174in}{1.581701in}}{\pgfqpoint{2.306350in}{1.587525in}}%
\pgfpathcurveto{\pgfqpoint{2.300526in}{1.593349in}}{\pgfqpoint{2.292626in}{1.596621in}}{\pgfqpoint{2.284390in}{1.596621in}}%
\pgfpathcurveto{\pgfqpoint{2.276153in}{1.596621in}}{\pgfqpoint{2.268253in}{1.593349in}}{\pgfqpoint{2.262429in}{1.587525in}}%
\pgfpathcurveto{\pgfqpoint{2.256606in}{1.581701in}}{\pgfqpoint{2.253333in}{1.573801in}}{\pgfqpoint{2.253333in}{1.565565in}}%
\pgfpathcurveto{\pgfqpoint{2.253333in}{1.557329in}}{\pgfqpoint{2.256606in}{1.549428in}}{\pgfqpoint{2.262429in}{1.543605in}}%
\pgfpathcurveto{\pgfqpoint{2.268253in}{1.537781in}}{\pgfqpoint{2.276153in}{1.534508in}}{\pgfqpoint{2.284390in}{1.534508in}}%
\pgfpathclose%
\pgfusepath{stroke,fill}%
\end{pgfscope}%
\begin{pgfscope}%
\pgfpathrectangle{\pgfqpoint{0.100000in}{0.212622in}}{\pgfqpoint{3.696000in}{3.696000in}}%
\pgfusepath{clip}%
\pgfsetbuttcap%
\pgfsetroundjoin%
\definecolor{currentfill}{rgb}{0.121569,0.466667,0.705882}%
\pgfsetfillcolor{currentfill}%
\pgfsetfillopacity{0.836735}%
\pgfsetlinewidth{1.003750pt}%
\definecolor{currentstroke}{rgb}{0.121569,0.466667,0.705882}%
\pgfsetstrokecolor{currentstroke}%
\pgfsetstrokeopacity{0.836735}%
\pgfsetdash{}{0pt}%
\pgfpathmoveto{\pgfqpoint{0.782231in}{2.478509in}}%
\pgfpathcurveto{\pgfqpoint{0.790467in}{2.478509in}}{\pgfqpoint{0.798367in}{2.481781in}}{\pgfqpoint{0.804191in}{2.487605in}}%
\pgfpathcurveto{\pgfqpoint{0.810015in}{2.493429in}}{\pgfqpoint{0.813287in}{2.501329in}}{\pgfqpoint{0.813287in}{2.509565in}}%
\pgfpathcurveto{\pgfqpoint{0.813287in}{2.517801in}}{\pgfqpoint{0.810015in}{2.525701in}}{\pgfqpoint{0.804191in}{2.531525in}}%
\pgfpathcurveto{\pgfqpoint{0.798367in}{2.537349in}}{\pgfqpoint{0.790467in}{2.540622in}}{\pgfqpoint{0.782231in}{2.540622in}}%
\pgfpathcurveto{\pgfqpoint{0.773995in}{2.540622in}}{\pgfqpoint{0.766095in}{2.537349in}}{\pgfqpoint{0.760271in}{2.531525in}}%
\pgfpathcurveto{\pgfqpoint{0.754447in}{2.525701in}}{\pgfqpoint{0.751174in}{2.517801in}}{\pgfqpoint{0.751174in}{2.509565in}}%
\pgfpathcurveto{\pgfqpoint{0.751174in}{2.501329in}}{\pgfqpoint{0.754447in}{2.493429in}}{\pgfqpoint{0.760271in}{2.487605in}}%
\pgfpathcurveto{\pgfqpoint{0.766095in}{2.481781in}}{\pgfqpoint{0.773995in}{2.478509in}}{\pgfqpoint{0.782231in}{2.478509in}}%
\pgfpathclose%
\pgfusepath{stroke,fill}%
\end{pgfscope}%
\begin{pgfscope}%
\pgfpathrectangle{\pgfqpoint{0.100000in}{0.212622in}}{\pgfqpoint{3.696000in}{3.696000in}}%
\pgfusepath{clip}%
\pgfsetbuttcap%
\pgfsetroundjoin%
\definecolor{currentfill}{rgb}{0.121569,0.466667,0.705882}%
\pgfsetfillcolor{currentfill}%
\pgfsetfillopacity{0.837110}%
\pgfsetlinewidth{1.003750pt}%
\definecolor{currentstroke}{rgb}{0.121569,0.466667,0.705882}%
\pgfsetstrokecolor{currentstroke}%
\pgfsetstrokeopacity{0.837110}%
\pgfsetdash{}{0pt}%
\pgfpathmoveto{\pgfqpoint{2.284820in}{1.533973in}}%
\pgfpathcurveto{\pgfqpoint{2.293056in}{1.533973in}}{\pgfqpoint{2.300956in}{1.537245in}}{\pgfqpoint{2.306780in}{1.543069in}}%
\pgfpathcurveto{\pgfqpoint{2.312604in}{1.548893in}}{\pgfqpoint{2.315877in}{1.556793in}}{\pgfqpoint{2.315877in}{1.565029in}}%
\pgfpathcurveto{\pgfqpoint{2.315877in}{1.573266in}}{\pgfqpoint{2.312604in}{1.581166in}}{\pgfqpoint{2.306780in}{1.586990in}}%
\pgfpathcurveto{\pgfqpoint{2.300956in}{1.592814in}}{\pgfqpoint{2.293056in}{1.596086in}}{\pgfqpoint{2.284820in}{1.596086in}}%
\pgfpathcurveto{\pgfqpoint{2.276584in}{1.596086in}}{\pgfqpoint{2.268684in}{1.592814in}}{\pgfqpoint{2.262860in}{1.586990in}}%
\pgfpathcurveto{\pgfqpoint{2.257036in}{1.581166in}}{\pgfqpoint{2.253764in}{1.573266in}}{\pgfqpoint{2.253764in}{1.565029in}}%
\pgfpathcurveto{\pgfqpoint{2.253764in}{1.556793in}}{\pgfqpoint{2.257036in}{1.548893in}}{\pgfqpoint{2.262860in}{1.543069in}}%
\pgfpathcurveto{\pgfqpoint{2.268684in}{1.537245in}}{\pgfqpoint{2.276584in}{1.533973in}}{\pgfqpoint{2.284820in}{1.533973in}}%
\pgfpathclose%
\pgfusepath{stroke,fill}%
\end{pgfscope}%
\begin{pgfscope}%
\pgfpathrectangle{\pgfqpoint{0.100000in}{0.212622in}}{\pgfqpoint{3.696000in}{3.696000in}}%
\pgfusepath{clip}%
\pgfsetbuttcap%
\pgfsetroundjoin%
\definecolor{currentfill}{rgb}{0.121569,0.466667,0.705882}%
\pgfsetfillcolor{currentfill}%
\pgfsetfillopacity{0.837952}%
\pgfsetlinewidth{1.003750pt}%
\definecolor{currentstroke}{rgb}{0.121569,0.466667,0.705882}%
\pgfsetstrokecolor{currentstroke}%
\pgfsetstrokeopacity{0.837952}%
\pgfsetdash{}{0pt}%
\pgfpathmoveto{\pgfqpoint{2.285540in}{1.532352in}}%
\pgfpathcurveto{\pgfqpoint{2.293776in}{1.532352in}}{\pgfqpoint{2.301676in}{1.535624in}}{\pgfqpoint{2.307500in}{1.541448in}}%
\pgfpathcurveto{\pgfqpoint{2.313324in}{1.547272in}}{\pgfqpoint{2.316596in}{1.555172in}}{\pgfqpoint{2.316596in}{1.563409in}}%
\pgfpathcurveto{\pgfqpoint{2.316596in}{1.571645in}}{\pgfqpoint{2.313324in}{1.579545in}}{\pgfqpoint{2.307500in}{1.585369in}}%
\pgfpathcurveto{\pgfqpoint{2.301676in}{1.591193in}}{\pgfqpoint{2.293776in}{1.594465in}}{\pgfqpoint{2.285540in}{1.594465in}}%
\pgfpathcurveto{\pgfqpoint{2.277303in}{1.594465in}}{\pgfqpoint{2.269403in}{1.591193in}}{\pgfqpoint{2.263579in}{1.585369in}}%
\pgfpathcurveto{\pgfqpoint{2.257756in}{1.579545in}}{\pgfqpoint{2.254483in}{1.571645in}}{\pgfqpoint{2.254483in}{1.563409in}}%
\pgfpathcurveto{\pgfqpoint{2.254483in}{1.555172in}}{\pgfqpoint{2.257756in}{1.547272in}}{\pgfqpoint{2.263579in}{1.541448in}}%
\pgfpathcurveto{\pgfqpoint{2.269403in}{1.535624in}}{\pgfqpoint{2.277303in}{1.532352in}}{\pgfqpoint{2.285540in}{1.532352in}}%
\pgfpathclose%
\pgfusepath{stroke,fill}%
\end{pgfscope}%
\begin{pgfscope}%
\pgfpathrectangle{\pgfqpoint{0.100000in}{0.212622in}}{\pgfqpoint{3.696000in}{3.696000in}}%
\pgfusepath{clip}%
\pgfsetbuttcap%
\pgfsetroundjoin%
\definecolor{currentfill}{rgb}{0.121569,0.466667,0.705882}%
\pgfsetfillcolor{currentfill}%
\pgfsetfillopacity{0.838946}%
\pgfsetlinewidth{1.003750pt}%
\definecolor{currentstroke}{rgb}{0.121569,0.466667,0.705882}%
\pgfsetstrokecolor{currentstroke}%
\pgfsetstrokeopacity{0.838946}%
\pgfsetdash{}{0pt}%
\pgfpathmoveto{\pgfqpoint{2.286430in}{1.528971in}}%
\pgfpathcurveto{\pgfqpoint{2.294667in}{1.528971in}}{\pgfqpoint{2.302567in}{1.532243in}}{\pgfqpoint{2.308391in}{1.538067in}}%
\pgfpathcurveto{\pgfqpoint{2.314214in}{1.543891in}}{\pgfqpoint{2.317487in}{1.551791in}}{\pgfqpoint{2.317487in}{1.560028in}}%
\pgfpathcurveto{\pgfqpoint{2.317487in}{1.568264in}}{\pgfqpoint{2.314214in}{1.576164in}}{\pgfqpoint{2.308391in}{1.581988in}}%
\pgfpathcurveto{\pgfqpoint{2.302567in}{1.587812in}}{\pgfqpoint{2.294667in}{1.591084in}}{\pgfqpoint{2.286430in}{1.591084in}}%
\pgfpathcurveto{\pgfqpoint{2.278194in}{1.591084in}}{\pgfqpoint{2.270294in}{1.587812in}}{\pgfqpoint{2.264470in}{1.581988in}}%
\pgfpathcurveto{\pgfqpoint{2.258646in}{1.576164in}}{\pgfqpoint{2.255374in}{1.568264in}}{\pgfqpoint{2.255374in}{1.560028in}}%
\pgfpathcurveto{\pgfqpoint{2.255374in}{1.551791in}}{\pgfqpoint{2.258646in}{1.543891in}}{\pgfqpoint{2.264470in}{1.538067in}}%
\pgfpathcurveto{\pgfqpoint{2.270294in}{1.532243in}}{\pgfqpoint{2.278194in}{1.528971in}}{\pgfqpoint{2.286430in}{1.528971in}}%
\pgfpathclose%
\pgfusepath{stroke,fill}%
\end{pgfscope}%
\begin{pgfscope}%
\pgfpathrectangle{\pgfqpoint{0.100000in}{0.212622in}}{\pgfqpoint{3.696000in}{3.696000in}}%
\pgfusepath{clip}%
\pgfsetbuttcap%
\pgfsetroundjoin%
\definecolor{currentfill}{rgb}{0.121569,0.466667,0.705882}%
\pgfsetfillcolor{currentfill}%
\pgfsetfillopacity{0.841148}%
\pgfsetlinewidth{1.003750pt}%
\definecolor{currentstroke}{rgb}{0.121569,0.466667,0.705882}%
\pgfsetstrokecolor{currentstroke}%
\pgfsetstrokeopacity{0.841148}%
\pgfsetdash{}{0pt}%
\pgfpathmoveto{\pgfqpoint{0.823209in}{2.450807in}}%
\pgfpathcurveto{\pgfqpoint{0.831445in}{2.450807in}}{\pgfqpoint{0.839345in}{2.454080in}}{\pgfqpoint{0.845169in}{2.459904in}}%
\pgfpathcurveto{\pgfqpoint{0.850993in}{2.465727in}}{\pgfqpoint{0.854265in}{2.473628in}}{\pgfqpoint{0.854265in}{2.481864in}}%
\pgfpathcurveto{\pgfqpoint{0.854265in}{2.490100in}}{\pgfqpoint{0.850993in}{2.498000in}}{\pgfqpoint{0.845169in}{2.503824in}}%
\pgfpathcurveto{\pgfqpoint{0.839345in}{2.509648in}}{\pgfqpoint{0.831445in}{2.512920in}}{\pgfqpoint{0.823209in}{2.512920in}}%
\pgfpathcurveto{\pgfqpoint{0.814973in}{2.512920in}}{\pgfqpoint{0.807073in}{2.509648in}}{\pgfqpoint{0.801249in}{2.503824in}}%
\pgfpathcurveto{\pgfqpoint{0.795425in}{2.498000in}}{\pgfqpoint{0.792152in}{2.490100in}}{\pgfqpoint{0.792152in}{2.481864in}}%
\pgfpathcurveto{\pgfqpoint{0.792152in}{2.473628in}}{\pgfqpoint{0.795425in}{2.465727in}}{\pgfqpoint{0.801249in}{2.459904in}}%
\pgfpathcurveto{\pgfqpoint{0.807073in}{2.454080in}}{\pgfqpoint{0.814973in}{2.450807in}}{\pgfqpoint{0.823209in}{2.450807in}}%
\pgfpathclose%
\pgfusepath{stroke,fill}%
\end{pgfscope}%
\begin{pgfscope}%
\pgfpathrectangle{\pgfqpoint{0.100000in}{0.212622in}}{\pgfqpoint{3.696000in}{3.696000in}}%
\pgfusepath{clip}%
\pgfsetbuttcap%
\pgfsetroundjoin%
\definecolor{currentfill}{rgb}{0.121569,0.466667,0.705882}%
\pgfsetfillcolor{currentfill}%
\pgfsetfillopacity{0.841200}%
\pgfsetlinewidth{1.003750pt}%
\definecolor{currentstroke}{rgb}{0.121569,0.466667,0.705882}%
\pgfsetstrokecolor{currentstroke}%
\pgfsetstrokeopacity{0.841200}%
\pgfsetdash{}{0pt}%
\pgfpathmoveto{\pgfqpoint{2.288098in}{1.527617in}}%
\pgfpathcurveto{\pgfqpoint{2.296334in}{1.527617in}}{\pgfqpoint{2.304234in}{1.530889in}}{\pgfqpoint{2.310058in}{1.536713in}}%
\pgfpathcurveto{\pgfqpoint{2.315882in}{1.542537in}}{\pgfqpoint{2.319154in}{1.550437in}}{\pgfqpoint{2.319154in}{1.558673in}}%
\pgfpathcurveto{\pgfqpoint{2.319154in}{1.566910in}}{\pgfqpoint{2.315882in}{1.574810in}}{\pgfqpoint{2.310058in}{1.580634in}}%
\pgfpathcurveto{\pgfqpoint{2.304234in}{1.586458in}}{\pgfqpoint{2.296334in}{1.589730in}}{\pgfqpoint{2.288098in}{1.589730in}}%
\pgfpathcurveto{\pgfqpoint{2.279861in}{1.589730in}}{\pgfqpoint{2.271961in}{1.586458in}}{\pgfqpoint{2.266137in}{1.580634in}}%
\pgfpathcurveto{\pgfqpoint{2.260313in}{1.574810in}}{\pgfqpoint{2.257041in}{1.566910in}}{\pgfqpoint{2.257041in}{1.558673in}}%
\pgfpathcurveto{\pgfqpoint{2.257041in}{1.550437in}}{\pgfqpoint{2.260313in}{1.542537in}}{\pgfqpoint{2.266137in}{1.536713in}}%
\pgfpathcurveto{\pgfqpoint{2.271961in}{1.530889in}}{\pgfqpoint{2.279861in}{1.527617in}}{\pgfqpoint{2.288098in}{1.527617in}}%
\pgfpathclose%
\pgfusepath{stroke,fill}%
\end{pgfscope}%
\begin{pgfscope}%
\pgfpathrectangle{\pgfqpoint{0.100000in}{0.212622in}}{\pgfqpoint{3.696000in}{3.696000in}}%
\pgfusepath{clip}%
\pgfsetbuttcap%
\pgfsetroundjoin%
\definecolor{currentfill}{rgb}{0.121569,0.466667,0.705882}%
\pgfsetfillcolor{currentfill}%
\pgfsetfillopacity{0.843928}%
\pgfsetlinewidth{1.003750pt}%
\definecolor{currentstroke}{rgb}{0.121569,0.466667,0.705882}%
\pgfsetstrokecolor{currentstroke}%
\pgfsetstrokeopacity{0.843928}%
\pgfsetdash{}{0pt}%
\pgfpathmoveto{\pgfqpoint{2.289429in}{1.527027in}}%
\pgfpathcurveto{\pgfqpoint{2.297665in}{1.527027in}}{\pgfqpoint{2.305565in}{1.530300in}}{\pgfqpoint{2.311389in}{1.536124in}}%
\pgfpathcurveto{\pgfqpoint{2.317213in}{1.541947in}}{\pgfqpoint{2.320485in}{1.549848in}}{\pgfqpoint{2.320485in}{1.558084in}}%
\pgfpathcurveto{\pgfqpoint{2.320485in}{1.566320in}}{\pgfqpoint{2.317213in}{1.574220in}}{\pgfqpoint{2.311389in}{1.580044in}}%
\pgfpathcurveto{\pgfqpoint{2.305565in}{1.585868in}}{\pgfqpoint{2.297665in}{1.589140in}}{\pgfqpoint{2.289429in}{1.589140in}}%
\pgfpathcurveto{\pgfqpoint{2.281192in}{1.589140in}}{\pgfqpoint{2.273292in}{1.585868in}}{\pgfqpoint{2.267468in}{1.580044in}}%
\pgfpathcurveto{\pgfqpoint{2.261644in}{1.574220in}}{\pgfqpoint{2.258372in}{1.566320in}}{\pgfqpoint{2.258372in}{1.558084in}}%
\pgfpathcurveto{\pgfqpoint{2.258372in}{1.549848in}}{\pgfqpoint{2.261644in}{1.541947in}}{\pgfqpoint{2.267468in}{1.536124in}}%
\pgfpathcurveto{\pgfqpoint{2.273292in}{1.530300in}}{\pgfqpoint{2.281192in}{1.527027in}}{\pgfqpoint{2.289429in}{1.527027in}}%
\pgfpathclose%
\pgfusepath{stroke,fill}%
\end{pgfscope}%
\begin{pgfscope}%
\pgfpathrectangle{\pgfqpoint{0.100000in}{0.212622in}}{\pgfqpoint{3.696000in}{3.696000in}}%
\pgfusepath{clip}%
\pgfsetbuttcap%
\pgfsetroundjoin%
\definecolor{currentfill}{rgb}{0.121569,0.466667,0.705882}%
\pgfsetfillcolor{currentfill}%
\pgfsetfillopacity{0.845372}%
\pgfsetlinewidth{1.003750pt}%
\definecolor{currentstroke}{rgb}{0.121569,0.466667,0.705882}%
\pgfsetstrokecolor{currentstroke}%
\pgfsetstrokeopacity{0.845372}%
\pgfsetdash{}{0pt}%
\pgfpathmoveto{\pgfqpoint{0.862418in}{2.426058in}}%
\pgfpathcurveto{\pgfqpoint{0.870654in}{2.426058in}}{\pgfqpoint{0.878554in}{2.429330in}}{\pgfqpoint{0.884378in}{2.435154in}}%
\pgfpathcurveto{\pgfqpoint{0.890202in}{2.440978in}}{\pgfqpoint{0.893474in}{2.448878in}}{\pgfqpoint{0.893474in}{2.457114in}}%
\pgfpathcurveto{\pgfqpoint{0.893474in}{2.465350in}}{\pgfqpoint{0.890202in}{2.473250in}}{\pgfqpoint{0.884378in}{2.479074in}}%
\pgfpathcurveto{\pgfqpoint{0.878554in}{2.484898in}}{\pgfqpoint{0.870654in}{2.488171in}}{\pgfqpoint{0.862418in}{2.488171in}}%
\pgfpathcurveto{\pgfqpoint{0.854181in}{2.488171in}}{\pgfqpoint{0.846281in}{2.484898in}}{\pgfqpoint{0.840457in}{2.479074in}}%
\pgfpathcurveto{\pgfqpoint{0.834633in}{2.473250in}}{\pgfqpoint{0.831361in}{2.465350in}}{\pgfqpoint{0.831361in}{2.457114in}}%
\pgfpathcurveto{\pgfqpoint{0.831361in}{2.448878in}}{\pgfqpoint{0.834633in}{2.440978in}}{\pgfqpoint{0.840457in}{2.435154in}}%
\pgfpathcurveto{\pgfqpoint{0.846281in}{2.429330in}}{\pgfqpoint{0.854181in}{2.426058in}}{\pgfqpoint{0.862418in}{2.426058in}}%
\pgfpathclose%
\pgfusepath{stroke,fill}%
\end{pgfscope}%
\begin{pgfscope}%
\pgfpathrectangle{\pgfqpoint{0.100000in}{0.212622in}}{\pgfqpoint{3.696000in}{3.696000in}}%
\pgfusepath{clip}%
\pgfsetbuttcap%
\pgfsetroundjoin%
\definecolor{currentfill}{rgb}{0.121569,0.466667,0.705882}%
\pgfsetfillcolor{currentfill}%
\pgfsetfillopacity{0.846416}%
\pgfsetlinewidth{1.003750pt}%
\definecolor{currentstroke}{rgb}{0.121569,0.466667,0.705882}%
\pgfsetstrokecolor{currentstroke}%
\pgfsetstrokeopacity{0.846416}%
\pgfsetdash{}{0pt}%
\pgfpathmoveto{\pgfqpoint{2.291351in}{1.524045in}}%
\pgfpathcurveto{\pgfqpoint{2.299587in}{1.524045in}}{\pgfqpoint{2.307487in}{1.527318in}}{\pgfqpoint{2.313311in}{1.533142in}}%
\pgfpathcurveto{\pgfqpoint{2.319135in}{1.538966in}}{\pgfqpoint{2.322407in}{1.546866in}}{\pgfqpoint{2.322407in}{1.555102in}}%
\pgfpathcurveto{\pgfqpoint{2.322407in}{1.563338in}}{\pgfqpoint{2.319135in}{1.571238in}}{\pgfqpoint{2.313311in}{1.577062in}}%
\pgfpathcurveto{\pgfqpoint{2.307487in}{1.582886in}}{\pgfqpoint{2.299587in}{1.586158in}}{\pgfqpoint{2.291351in}{1.586158in}}%
\pgfpathcurveto{\pgfqpoint{2.283114in}{1.586158in}}{\pgfqpoint{2.275214in}{1.582886in}}{\pgfqpoint{2.269390in}{1.577062in}}%
\pgfpathcurveto{\pgfqpoint{2.263567in}{1.571238in}}{\pgfqpoint{2.260294in}{1.563338in}}{\pgfqpoint{2.260294in}{1.555102in}}%
\pgfpathcurveto{\pgfqpoint{2.260294in}{1.546866in}}{\pgfqpoint{2.263567in}{1.538966in}}{\pgfqpoint{2.269390in}{1.533142in}}%
\pgfpathcurveto{\pgfqpoint{2.275214in}{1.527318in}}{\pgfqpoint{2.283114in}{1.524045in}}{\pgfqpoint{2.291351in}{1.524045in}}%
\pgfpathclose%
\pgfusepath{stroke,fill}%
\end{pgfscope}%
\begin{pgfscope}%
\pgfpathrectangle{\pgfqpoint{0.100000in}{0.212622in}}{\pgfqpoint{3.696000in}{3.696000in}}%
\pgfusepath{clip}%
\pgfsetbuttcap%
\pgfsetroundjoin%
\definecolor{currentfill}{rgb}{0.121569,0.466667,0.705882}%
\pgfsetfillcolor{currentfill}%
\pgfsetfillopacity{0.849134}%
\pgfsetlinewidth{1.003750pt}%
\definecolor{currentstroke}{rgb}{0.121569,0.466667,0.705882}%
\pgfsetstrokecolor{currentstroke}%
\pgfsetstrokeopacity{0.849134}%
\pgfsetdash{}{0pt}%
\pgfpathmoveto{\pgfqpoint{2.293487in}{1.520671in}}%
\pgfpathcurveto{\pgfqpoint{2.301723in}{1.520671in}}{\pgfqpoint{2.309623in}{1.523944in}}{\pgfqpoint{2.315447in}{1.529768in}}%
\pgfpathcurveto{\pgfqpoint{2.321271in}{1.535591in}}{\pgfqpoint{2.324543in}{1.543492in}}{\pgfqpoint{2.324543in}{1.551728in}}%
\pgfpathcurveto{\pgfqpoint{2.324543in}{1.559964in}}{\pgfqpoint{2.321271in}{1.567864in}}{\pgfqpoint{2.315447in}{1.573688in}}%
\pgfpathcurveto{\pgfqpoint{2.309623in}{1.579512in}}{\pgfqpoint{2.301723in}{1.582784in}}{\pgfqpoint{2.293487in}{1.582784in}}%
\pgfpathcurveto{\pgfqpoint{2.285250in}{1.582784in}}{\pgfqpoint{2.277350in}{1.579512in}}{\pgfqpoint{2.271526in}{1.573688in}}%
\pgfpathcurveto{\pgfqpoint{2.265702in}{1.567864in}}{\pgfqpoint{2.262430in}{1.559964in}}{\pgfqpoint{2.262430in}{1.551728in}}%
\pgfpathcurveto{\pgfqpoint{2.262430in}{1.543492in}}{\pgfqpoint{2.265702in}{1.535591in}}{\pgfqpoint{2.271526in}{1.529768in}}%
\pgfpathcurveto{\pgfqpoint{2.277350in}{1.523944in}}{\pgfqpoint{2.285250in}{1.520671in}}{\pgfqpoint{2.293487in}{1.520671in}}%
\pgfpathclose%
\pgfusepath{stroke,fill}%
\end{pgfscope}%
\begin{pgfscope}%
\pgfpathrectangle{\pgfqpoint{0.100000in}{0.212622in}}{\pgfqpoint{3.696000in}{3.696000in}}%
\pgfusepath{clip}%
\pgfsetbuttcap%
\pgfsetroundjoin%
\definecolor{currentfill}{rgb}{0.121569,0.466667,0.705882}%
\pgfsetfillcolor{currentfill}%
\pgfsetfillopacity{0.849237}%
\pgfsetlinewidth{1.003750pt}%
\definecolor{currentstroke}{rgb}{0.121569,0.466667,0.705882}%
\pgfsetstrokecolor{currentstroke}%
\pgfsetstrokeopacity{0.849237}%
\pgfsetdash{}{0pt}%
\pgfpathmoveto{\pgfqpoint{0.895559in}{2.394320in}}%
\pgfpathcurveto{\pgfqpoint{0.903795in}{2.394320in}}{\pgfqpoint{0.911695in}{2.397592in}}{\pgfqpoint{0.917519in}{2.403416in}}%
\pgfpathcurveto{\pgfqpoint{0.923343in}{2.409240in}}{\pgfqpoint{0.926616in}{2.417140in}}{\pgfqpoint{0.926616in}{2.425376in}}%
\pgfpathcurveto{\pgfqpoint{0.926616in}{2.433613in}}{\pgfqpoint{0.923343in}{2.441513in}}{\pgfqpoint{0.917519in}{2.447337in}}%
\pgfpathcurveto{\pgfqpoint{0.911695in}{2.453160in}}{\pgfqpoint{0.903795in}{2.456433in}}{\pgfqpoint{0.895559in}{2.456433in}}%
\pgfpathcurveto{\pgfqpoint{0.887323in}{2.456433in}}{\pgfqpoint{0.879423in}{2.453160in}}{\pgfqpoint{0.873599in}{2.447337in}}%
\pgfpathcurveto{\pgfqpoint{0.867775in}{2.441513in}}{\pgfqpoint{0.864503in}{2.433613in}}{\pgfqpoint{0.864503in}{2.425376in}}%
\pgfpathcurveto{\pgfqpoint{0.864503in}{2.417140in}}{\pgfqpoint{0.867775in}{2.409240in}}{\pgfqpoint{0.873599in}{2.403416in}}%
\pgfpathcurveto{\pgfqpoint{0.879423in}{2.397592in}}{\pgfqpoint{0.887323in}{2.394320in}}{\pgfqpoint{0.895559in}{2.394320in}}%
\pgfpathclose%
\pgfusepath{stroke,fill}%
\end{pgfscope}%
\begin{pgfscope}%
\pgfpathrectangle{\pgfqpoint{0.100000in}{0.212622in}}{\pgfqpoint{3.696000in}{3.696000in}}%
\pgfusepath{clip}%
\pgfsetbuttcap%
\pgfsetroundjoin%
\definecolor{currentfill}{rgb}{0.121569,0.466667,0.705882}%
\pgfsetfillcolor{currentfill}%
\pgfsetfillopacity{0.851924}%
\pgfsetlinewidth{1.003750pt}%
\definecolor{currentstroke}{rgb}{0.121569,0.466667,0.705882}%
\pgfsetstrokecolor{currentstroke}%
\pgfsetstrokeopacity{0.851924}%
\pgfsetdash{}{0pt}%
\pgfpathmoveto{\pgfqpoint{2.295865in}{1.516210in}}%
\pgfpathcurveto{\pgfqpoint{2.304101in}{1.516210in}}{\pgfqpoint{2.312001in}{1.519482in}}{\pgfqpoint{2.317825in}{1.525306in}}%
\pgfpathcurveto{\pgfqpoint{2.323649in}{1.531130in}}{\pgfqpoint{2.326921in}{1.539030in}}{\pgfqpoint{2.326921in}{1.547266in}}%
\pgfpathcurveto{\pgfqpoint{2.326921in}{1.555502in}}{\pgfqpoint{2.323649in}{1.563402in}}{\pgfqpoint{2.317825in}{1.569226in}}%
\pgfpathcurveto{\pgfqpoint{2.312001in}{1.575050in}}{\pgfqpoint{2.304101in}{1.578323in}}{\pgfqpoint{2.295865in}{1.578323in}}%
\pgfpathcurveto{\pgfqpoint{2.287628in}{1.578323in}}{\pgfqpoint{2.279728in}{1.575050in}}{\pgfqpoint{2.273904in}{1.569226in}}%
\pgfpathcurveto{\pgfqpoint{2.268081in}{1.563402in}}{\pgfqpoint{2.264808in}{1.555502in}}{\pgfqpoint{2.264808in}{1.547266in}}%
\pgfpathcurveto{\pgfqpoint{2.264808in}{1.539030in}}{\pgfqpoint{2.268081in}{1.531130in}}{\pgfqpoint{2.273904in}{1.525306in}}%
\pgfpathcurveto{\pgfqpoint{2.279728in}{1.519482in}}{\pgfqpoint{2.287628in}{1.516210in}}{\pgfqpoint{2.295865in}{1.516210in}}%
\pgfpathclose%
\pgfusepath{stroke,fill}%
\end{pgfscope}%
\begin{pgfscope}%
\pgfpathrectangle{\pgfqpoint{0.100000in}{0.212622in}}{\pgfqpoint{3.696000in}{3.696000in}}%
\pgfusepath{clip}%
\pgfsetbuttcap%
\pgfsetroundjoin%
\definecolor{currentfill}{rgb}{0.121569,0.466667,0.705882}%
\pgfsetfillcolor{currentfill}%
\pgfsetfillopacity{0.852852}%
\pgfsetlinewidth{1.003750pt}%
\definecolor{currentstroke}{rgb}{0.121569,0.466667,0.705882}%
\pgfsetstrokecolor{currentstroke}%
\pgfsetstrokeopacity{0.852852}%
\pgfsetdash{}{0pt}%
\pgfpathmoveto{\pgfqpoint{0.931179in}{2.371063in}}%
\pgfpathcurveto{\pgfqpoint{0.939416in}{2.371063in}}{\pgfqpoint{0.947316in}{2.374336in}}{\pgfqpoint{0.953140in}{2.380159in}}%
\pgfpathcurveto{\pgfqpoint{0.958964in}{2.385983in}}{\pgfqpoint{0.962236in}{2.393883in}}{\pgfqpoint{0.962236in}{2.402120in}}%
\pgfpathcurveto{\pgfqpoint{0.962236in}{2.410356in}}{\pgfqpoint{0.958964in}{2.418256in}}{\pgfqpoint{0.953140in}{2.424080in}}%
\pgfpathcurveto{\pgfqpoint{0.947316in}{2.429904in}}{\pgfqpoint{0.939416in}{2.433176in}}{\pgfqpoint{0.931179in}{2.433176in}}%
\pgfpathcurveto{\pgfqpoint{0.922943in}{2.433176in}}{\pgfqpoint{0.915043in}{2.429904in}}{\pgfqpoint{0.909219in}{2.424080in}}%
\pgfpathcurveto{\pgfqpoint{0.903395in}{2.418256in}}{\pgfqpoint{0.900123in}{2.410356in}}{\pgfqpoint{0.900123in}{2.402120in}}%
\pgfpathcurveto{\pgfqpoint{0.900123in}{2.393883in}}{\pgfqpoint{0.903395in}{2.385983in}}{\pgfqpoint{0.909219in}{2.380159in}}%
\pgfpathcurveto{\pgfqpoint{0.915043in}{2.374336in}}{\pgfqpoint{0.922943in}{2.371063in}}{\pgfqpoint{0.931179in}{2.371063in}}%
\pgfpathclose%
\pgfusepath{stroke,fill}%
\end{pgfscope}%
\begin{pgfscope}%
\pgfpathrectangle{\pgfqpoint{0.100000in}{0.212622in}}{\pgfqpoint{3.696000in}{3.696000in}}%
\pgfusepath{clip}%
\pgfsetbuttcap%
\pgfsetroundjoin%
\definecolor{currentfill}{rgb}{0.121569,0.466667,0.705882}%
\pgfsetfillcolor{currentfill}%
\pgfsetfillopacity{0.855218}%
\pgfsetlinewidth{1.003750pt}%
\definecolor{currentstroke}{rgb}{0.121569,0.466667,0.705882}%
\pgfsetstrokecolor{currentstroke}%
\pgfsetstrokeopacity{0.855218}%
\pgfsetdash{}{0pt}%
\pgfpathmoveto{\pgfqpoint{0.963437in}{2.343324in}}%
\pgfpathcurveto{\pgfqpoint{0.971673in}{2.343324in}}{\pgfqpoint{0.979573in}{2.346596in}}{\pgfqpoint{0.985397in}{2.352420in}}%
\pgfpathcurveto{\pgfqpoint{0.991221in}{2.358244in}}{\pgfqpoint{0.994493in}{2.366144in}}{\pgfqpoint{0.994493in}{2.374380in}}%
\pgfpathcurveto{\pgfqpoint{0.994493in}{2.382616in}}{\pgfqpoint{0.991221in}{2.390516in}}{\pgfqpoint{0.985397in}{2.396340in}}%
\pgfpathcurveto{\pgfqpoint{0.979573in}{2.402164in}}{\pgfqpoint{0.971673in}{2.405437in}}{\pgfqpoint{0.963437in}{2.405437in}}%
\pgfpathcurveto{\pgfqpoint{0.955200in}{2.405437in}}{\pgfqpoint{0.947300in}{2.402164in}}{\pgfqpoint{0.941476in}{2.396340in}}%
\pgfpathcurveto{\pgfqpoint{0.935652in}{2.390516in}}{\pgfqpoint{0.932380in}{2.382616in}}{\pgfqpoint{0.932380in}{2.374380in}}%
\pgfpathcurveto{\pgfqpoint{0.932380in}{2.366144in}}{\pgfqpoint{0.935652in}{2.358244in}}{\pgfqpoint{0.941476in}{2.352420in}}%
\pgfpathcurveto{\pgfqpoint{0.947300in}{2.346596in}}{\pgfqpoint{0.955200in}{2.343324in}}{\pgfqpoint{0.963437in}{2.343324in}}%
\pgfpathclose%
\pgfusepath{stroke,fill}%
\end{pgfscope}%
\begin{pgfscope}%
\pgfpathrectangle{\pgfqpoint{0.100000in}{0.212622in}}{\pgfqpoint{3.696000in}{3.696000in}}%
\pgfusepath{clip}%
\pgfsetbuttcap%
\pgfsetroundjoin%
\definecolor{currentfill}{rgb}{0.121569,0.466667,0.705882}%
\pgfsetfillcolor{currentfill}%
\pgfsetfillopacity{0.855897}%
\pgfsetlinewidth{1.003750pt}%
\definecolor{currentstroke}{rgb}{0.121569,0.466667,0.705882}%
\pgfsetstrokecolor{currentstroke}%
\pgfsetstrokeopacity{0.855897}%
\pgfsetdash{}{0pt}%
\pgfpathmoveto{\pgfqpoint{2.299196in}{1.516555in}}%
\pgfpathcurveto{\pgfqpoint{2.307432in}{1.516555in}}{\pgfqpoint{2.315332in}{1.519827in}}{\pgfqpoint{2.321156in}{1.525651in}}%
\pgfpathcurveto{\pgfqpoint{2.326980in}{1.531475in}}{\pgfqpoint{2.330252in}{1.539375in}}{\pgfqpoint{2.330252in}{1.547611in}}%
\pgfpathcurveto{\pgfqpoint{2.330252in}{1.555848in}}{\pgfqpoint{2.326980in}{1.563748in}}{\pgfqpoint{2.321156in}{1.569572in}}%
\pgfpathcurveto{\pgfqpoint{2.315332in}{1.575396in}}{\pgfqpoint{2.307432in}{1.578668in}}{\pgfqpoint{2.299196in}{1.578668in}}%
\pgfpathcurveto{\pgfqpoint{2.290960in}{1.578668in}}{\pgfqpoint{2.283059in}{1.575396in}}{\pgfqpoint{2.277236in}{1.569572in}}%
\pgfpathcurveto{\pgfqpoint{2.271412in}{1.563748in}}{\pgfqpoint{2.268139in}{1.555848in}}{\pgfqpoint{2.268139in}{1.547611in}}%
\pgfpathcurveto{\pgfqpoint{2.268139in}{1.539375in}}{\pgfqpoint{2.271412in}{1.531475in}}{\pgfqpoint{2.277236in}{1.525651in}}%
\pgfpathcurveto{\pgfqpoint{2.283059in}{1.519827in}}{\pgfqpoint{2.290960in}{1.516555in}}{\pgfqpoint{2.299196in}{1.516555in}}%
\pgfpathclose%
\pgfusepath{stroke,fill}%
\end{pgfscope}%
\begin{pgfscope}%
\pgfpathrectangle{\pgfqpoint{0.100000in}{0.212622in}}{\pgfqpoint{3.696000in}{3.696000in}}%
\pgfusepath{clip}%
\pgfsetbuttcap%
\pgfsetroundjoin%
\definecolor{currentfill}{rgb}{0.121569,0.466667,0.705882}%
\pgfsetfillcolor{currentfill}%
\pgfsetfillopacity{0.859815}%
\pgfsetlinewidth{1.003750pt}%
\definecolor{currentstroke}{rgb}{0.121569,0.466667,0.705882}%
\pgfsetstrokecolor{currentstroke}%
\pgfsetstrokeopacity{0.859815}%
\pgfsetdash{}{0pt}%
\pgfpathmoveto{\pgfqpoint{2.301830in}{1.513732in}}%
\pgfpathcurveto{\pgfqpoint{2.310066in}{1.513732in}}{\pgfqpoint{2.317966in}{1.517005in}}{\pgfqpoint{2.323790in}{1.522829in}}%
\pgfpathcurveto{\pgfqpoint{2.329614in}{1.528653in}}{\pgfqpoint{2.332887in}{1.536553in}}{\pgfqpoint{2.332887in}{1.544789in}}%
\pgfpathcurveto{\pgfqpoint{2.332887in}{1.553025in}}{\pgfqpoint{2.329614in}{1.560925in}}{\pgfqpoint{2.323790in}{1.566749in}}%
\pgfpathcurveto{\pgfqpoint{2.317966in}{1.572573in}}{\pgfqpoint{2.310066in}{1.575845in}}{\pgfqpoint{2.301830in}{1.575845in}}%
\pgfpathcurveto{\pgfqpoint{2.293594in}{1.575845in}}{\pgfqpoint{2.285694in}{1.572573in}}{\pgfqpoint{2.279870in}{1.566749in}}%
\pgfpathcurveto{\pgfqpoint{2.274046in}{1.560925in}}{\pgfqpoint{2.270774in}{1.553025in}}{\pgfqpoint{2.270774in}{1.544789in}}%
\pgfpathcurveto{\pgfqpoint{2.270774in}{1.536553in}}{\pgfqpoint{2.274046in}{1.528653in}}{\pgfqpoint{2.279870in}{1.522829in}}%
\pgfpathcurveto{\pgfqpoint{2.285694in}{1.517005in}}{\pgfqpoint{2.293594in}{1.513732in}}{\pgfqpoint{2.301830in}{1.513732in}}%
\pgfpathclose%
\pgfusepath{stroke,fill}%
\end{pgfscope}%
\begin{pgfscope}%
\pgfpathrectangle{\pgfqpoint{0.100000in}{0.212622in}}{\pgfqpoint{3.696000in}{3.696000in}}%
\pgfusepath{clip}%
\pgfsetbuttcap%
\pgfsetroundjoin%
\definecolor{currentfill}{rgb}{0.121569,0.466667,0.705882}%
\pgfsetfillcolor{currentfill}%
\pgfsetfillopacity{0.860819}%
\pgfsetlinewidth{1.003750pt}%
\definecolor{currentstroke}{rgb}{0.121569,0.466667,0.705882}%
\pgfsetstrokecolor{currentstroke}%
\pgfsetstrokeopacity{0.860819}%
\pgfsetdash{}{0pt}%
\pgfpathmoveto{\pgfqpoint{1.020175in}{2.295716in}}%
\pgfpathcurveto{\pgfqpoint{1.028411in}{2.295716in}}{\pgfqpoint{1.036311in}{2.298988in}}{\pgfqpoint{1.042135in}{2.304812in}}%
\pgfpathcurveto{\pgfqpoint{1.047959in}{2.310636in}}{\pgfqpoint{1.051231in}{2.318536in}}{\pgfqpoint{1.051231in}{2.326772in}}%
\pgfpathcurveto{\pgfqpoint{1.051231in}{2.335008in}}{\pgfqpoint{1.047959in}{2.342908in}}{\pgfqpoint{1.042135in}{2.348732in}}%
\pgfpathcurveto{\pgfqpoint{1.036311in}{2.354556in}}{\pgfqpoint{1.028411in}{2.357829in}}{\pgfqpoint{1.020175in}{2.357829in}}%
\pgfpathcurveto{\pgfqpoint{1.011939in}{2.357829in}}{\pgfqpoint{1.004039in}{2.354556in}}{\pgfqpoint{0.998215in}{2.348732in}}%
\pgfpathcurveto{\pgfqpoint{0.992391in}{2.342908in}}{\pgfqpoint{0.989118in}{2.335008in}}{\pgfqpoint{0.989118in}{2.326772in}}%
\pgfpathcurveto{\pgfqpoint{0.989118in}{2.318536in}}{\pgfqpoint{0.992391in}{2.310636in}}{\pgfqpoint{0.998215in}{2.304812in}}%
\pgfpathcurveto{\pgfqpoint{1.004039in}{2.298988in}}{\pgfqpoint{1.011939in}{2.295716in}}{\pgfqpoint{1.020175in}{2.295716in}}%
\pgfpathclose%
\pgfusepath{stroke,fill}%
\end{pgfscope}%
\begin{pgfscope}%
\pgfpathrectangle{\pgfqpoint{0.100000in}{0.212622in}}{\pgfqpoint{3.696000in}{3.696000in}}%
\pgfusepath{clip}%
\pgfsetbuttcap%
\pgfsetroundjoin%
\definecolor{currentfill}{rgb}{0.121569,0.466667,0.705882}%
\pgfsetfillcolor{currentfill}%
\pgfsetfillopacity{0.863791}%
\pgfsetlinewidth{1.003750pt}%
\definecolor{currentstroke}{rgb}{0.121569,0.466667,0.705882}%
\pgfsetstrokecolor{currentstroke}%
\pgfsetstrokeopacity{0.863791}%
\pgfsetdash{}{0pt}%
\pgfpathmoveto{\pgfqpoint{2.305441in}{1.509743in}}%
\pgfpathcurveto{\pgfqpoint{2.313677in}{1.509743in}}{\pgfqpoint{2.321577in}{1.513015in}}{\pgfqpoint{2.327401in}{1.518839in}}%
\pgfpathcurveto{\pgfqpoint{2.333225in}{1.524663in}}{\pgfqpoint{2.336497in}{1.532563in}}{\pgfqpoint{2.336497in}{1.540799in}}%
\pgfpathcurveto{\pgfqpoint{2.336497in}{1.549036in}}{\pgfqpoint{2.333225in}{1.556936in}}{\pgfqpoint{2.327401in}{1.562759in}}%
\pgfpathcurveto{\pgfqpoint{2.321577in}{1.568583in}}{\pgfqpoint{2.313677in}{1.571856in}}{\pgfqpoint{2.305441in}{1.571856in}}%
\pgfpathcurveto{\pgfqpoint{2.297204in}{1.571856in}}{\pgfqpoint{2.289304in}{1.568583in}}{\pgfqpoint{2.283480in}{1.562759in}}%
\pgfpathcurveto{\pgfqpoint{2.277657in}{1.556936in}}{\pgfqpoint{2.274384in}{1.549036in}}{\pgfqpoint{2.274384in}{1.540799in}}%
\pgfpathcurveto{\pgfqpoint{2.274384in}{1.532563in}}{\pgfqpoint{2.277657in}{1.524663in}}{\pgfqpoint{2.283480in}{1.518839in}}%
\pgfpathcurveto{\pgfqpoint{2.289304in}{1.513015in}}{\pgfqpoint{2.297204in}{1.509743in}}{\pgfqpoint{2.305441in}{1.509743in}}%
\pgfpathclose%
\pgfusepath{stroke,fill}%
\end{pgfscope}%
\begin{pgfscope}%
\pgfpathrectangle{\pgfqpoint{0.100000in}{0.212622in}}{\pgfqpoint{3.696000in}{3.696000in}}%
\pgfusepath{clip}%
\pgfsetbuttcap%
\pgfsetroundjoin%
\definecolor{currentfill}{rgb}{0.121569,0.466667,0.705882}%
\pgfsetfillcolor{currentfill}%
\pgfsetfillopacity{0.865360}%
\pgfsetlinewidth{1.003750pt}%
\definecolor{currentstroke}{rgb}{0.121569,0.466667,0.705882}%
\pgfsetstrokecolor{currentstroke}%
\pgfsetstrokeopacity{0.865360}%
\pgfsetdash{}{0pt}%
\pgfpathmoveto{\pgfqpoint{2.307120in}{1.503551in}}%
\pgfpathcurveto{\pgfqpoint{2.315356in}{1.503551in}}{\pgfqpoint{2.323256in}{1.506824in}}{\pgfqpoint{2.329080in}{1.512648in}}%
\pgfpathcurveto{\pgfqpoint{2.334904in}{1.518472in}}{\pgfqpoint{2.338176in}{1.526372in}}{\pgfqpoint{2.338176in}{1.534608in}}%
\pgfpathcurveto{\pgfqpoint{2.338176in}{1.542844in}}{\pgfqpoint{2.334904in}{1.550744in}}{\pgfqpoint{2.329080in}{1.556568in}}%
\pgfpathcurveto{\pgfqpoint{2.323256in}{1.562392in}}{\pgfqpoint{2.315356in}{1.565664in}}{\pgfqpoint{2.307120in}{1.565664in}}%
\pgfpathcurveto{\pgfqpoint{2.298883in}{1.565664in}}{\pgfqpoint{2.290983in}{1.562392in}}{\pgfqpoint{2.285159in}{1.556568in}}%
\pgfpathcurveto{\pgfqpoint{2.279336in}{1.550744in}}{\pgfqpoint{2.276063in}{1.542844in}}{\pgfqpoint{2.276063in}{1.534608in}}%
\pgfpathcurveto{\pgfqpoint{2.276063in}{1.526372in}}{\pgfqpoint{2.279336in}{1.518472in}}{\pgfqpoint{2.285159in}{1.512648in}}%
\pgfpathcurveto{\pgfqpoint{2.290983in}{1.506824in}}{\pgfqpoint{2.298883in}{1.503551in}}{\pgfqpoint{2.307120in}{1.503551in}}%
\pgfpathclose%
\pgfusepath{stroke,fill}%
\end{pgfscope}%
\begin{pgfscope}%
\pgfpathrectangle{\pgfqpoint{0.100000in}{0.212622in}}{\pgfqpoint{3.696000in}{3.696000in}}%
\pgfusepath{clip}%
\pgfsetbuttcap%
\pgfsetroundjoin%
\definecolor{currentfill}{rgb}{0.121569,0.466667,0.705882}%
\pgfsetfillcolor{currentfill}%
\pgfsetfillopacity{0.865986}%
\pgfsetlinewidth{1.003750pt}%
\definecolor{currentstroke}{rgb}{0.121569,0.466667,0.705882}%
\pgfsetstrokecolor{currentstroke}%
\pgfsetstrokeopacity{0.865986}%
\pgfsetdash{}{0pt}%
\pgfpathmoveto{\pgfqpoint{1.073716in}{2.250940in}}%
\pgfpathcurveto{\pgfqpoint{1.081952in}{2.250940in}}{\pgfqpoint{1.089852in}{2.254213in}}{\pgfqpoint{1.095676in}{2.260036in}}%
\pgfpathcurveto{\pgfqpoint{1.101500in}{2.265860in}}{\pgfqpoint{1.104772in}{2.273760in}}{\pgfqpoint{1.104772in}{2.281997in}}%
\pgfpathcurveto{\pgfqpoint{1.104772in}{2.290233in}}{\pgfqpoint{1.101500in}{2.298133in}}{\pgfqpoint{1.095676in}{2.303957in}}%
\pgfpathcurveto{\pgfqpoint{1.089852in}{2.309781in}}{\pgfqpoint{1.081952in}{2.313053in}}{\pgfqpoint{1.073716in}{2.313053in}}%
\pgfpathcurveto{\pgfqpoint{1.065480in}{2.313053in}}{\pgfqpoint{1.057580in}{2.309781in}}{\pgfqpoint{1.051756in}{2.303957in}}%
\pgfpathcurveto{\pgfqpoint{1.045932in}{2.298133in}}{\pgfqpoint{1.042659in}{2.290233in}}{\pgfqpoint{1.042659in}{2.281997in}}%
\pgfpathcurveto{\pgfqpoint{1.042659in}{2.273760in}}{\pgfqpoint{1.045932in}{2.265860in}}{\pgfqpoint{1.051756in}{2.260036in}}%
\pgfpathcurveto{\pgfqpoint{1.057580in}{2.254213in}}{\pgfqpoint{1.065480in}{2.250940in}}{\pgfqpoint{1.073716in}{2.250940in}}%
\pgfpathclose%
\pgfusepath{stroke,fill}%
\end{pgfscope}%
\begin{pgfscope}%
\pgfpathrectangle{\pgfqpoint{0.100000in}{0.212622in}}{\pgfqpoint{3.696000in}{3.696000in}}%
\pgfusepath{clip}%
\pgfsetbuttcap%
\pgfsetroundjoin%
\definecolor{currentfill}{rgb}{0.121569,0.466667,0.705882}%
\pgfsetfillcolor{currentfill}%
\pgfsetfillopacity{0.867771}%
\pgfsetlinewidth{1.003750pt}%
\definecolor{currentstroke}{rgb}{0.121569,0.466667,0.705882}%
\pgfsetstrokecolor{currentstroke}%
\pgfsetstrokeopacity{0.867771}%
\pgfsetdash{}{0pt}%
\pgfpathmoveto{\pgfqpoint{2.309210in}{1.500601in}}%
\pgfpathcurveto{\pgfqpoint{2.317446in}{1.500601in}}{\pgfqpoint{2.325346in}{1.503874in}}{\pgfqpoint{2.331170in}{1.509698in}}%
\pgfpathcurveto{\pgfqpoint{2.336994in}{1.515522in}}{\pgfqpoint{2.340266in}{1.523422in}}{\pgfqpoint{2.340266in}{1.531658in}}%
\pgfpathcurveto{\pgfqpoint{2.340266in}{1.539894in}}{\pgfqpoint{2.336994in}{1.547794in}}{\pgfqpoint{2.331170in}{1.553618in}}%
\pgfpathcurveto{\pgfqpoint{2.325346in}{1.559442in}}{\pgfqpoint{2.317446in}{1.562714in}}{\pgfqpoint{2.309210in}{1.562714in}}%
\pgfpathcurveto{\pgfqpoint{2.300973in}{1.562714in}}{\pgfqpoint{2.293073in}{1.559442in}}{\pgfqpoint{2.287249in}{1.553618in}}%
\pgfpathcurveto{\pgfqpoint{2.281425in}{1.547794in}}{\pgfqpoint{2.278153in}{1.539894in}}{\pgfqpoint{2.278153in}{1.531658in}}%
\pgfpathcurveto{\pgfqpoint{2.278153in}{1.523422in}}{\pgfqpoint{2.281425in}{1.515522in}}{\pgfqpoint{2.287249in}{1.509698in}}%
\pgfpathcurveto{\pgfqpoint{2.293073in}{1.503874in}}{\pgfqpoint{2.300973in}{1.500601in}}{\pgfqpoint{2.309210in}{1.500601in}}%
\pgfpathclose%
\pgfusepath{stroke,fill}%
\end{pgfscope}%
\begin{pgfscope}%
\pgfpathrectangle{\pgfqpoint{0.100000in}{0.212622in}}{\pgfqpoint{3.696000in}{3.696000in}}%
\pgfusepath{clip}%
\pgfsetbuttcap%
\pgfsetroundjoin%
\definecolor{currentfill}{rgb}{0.121569,0.466667,0.705882}%
\pgfsetfillcolor{currentfill}%
\pgfsetfillopacity{0.870631}%
\pgfsetlinewidth{1.003750pt}%
\definecolor{currentstroke}{rgb}{0.121569,0.466667,0.705882}%
\pgfsetstrokecolor{currentstroke}%
\pgfsetstrokeopacity{0.870631}%
\pgfsetdash{}{0pt}%
\pgfpathmoveto{\pgfqpoint{2.311490in}{1.499325in}}%
\pgfpathcurveto{\pgfqpoint{2.319727in}{1.499325in}}{\pgfqpoint{2.327627in}{1.502597in}}{\pgfqpoint{2.333450in}{1.508421in}}%
\pgfpathcurveto{\pgfqpoint{2.339274in}{1.514245in}}{\pgfqpoint{2.342547in}{1.522145in}}{\pgfqpoint{2.342547in}{1.530382in}}%
\pgfpathcurveto{\pgfqpoint{2.342547in}{1.538618in}}{\pgfqpoint{2.339274in}{1.546518in}}{\pgfqpoint{2.333450in}{1.552342in}}%
\pgfpathcurveto{\pgfqpoint{2.327627in}{1.558166in}}{\pgfqpoint{2.319727in}{1.561438in}}{\pgfqpoint{2.311490in}{1.561438in}}%
\pgfpathcurveto{\pgfqpoint{2.303254in}{1.561438in}}{\pgfqpoint{2.295354in}{1.558166in}}{\pgfqpoint{2.289530in}{1.552342in}}%
\pgfpathcurveto{\pgfqpoint{2.283706in}{1.546518in}}{\pgfqpoint{2.280434in}{1.538618in}}{\pgfqpoint{2.280434in}{1.530382in}}%
\pgfpathcurveto{\pgfqpoint{2.280434in}{1.522145in}}{\pgfqpoint{2.283706in}{1.514245in}}{\pgfqpoint{2.289530in}{1.508421in}}%
\pgfpathcurveto{\pgfqpoint{2.295354in}{1.502597in}}{\pgfqpoint{2.303254in}{1.499325in}}{\pgfqpoint{2.311490in}{1.499325in}}%
\pgfpathclose%
\pgfusepath{stroke,fill}%
\end{pgfscope}%
\begin{pgfscope}%
\pgfpathrectangle{\pgfqpoint{0.100000in}{0.212622in}}{\pgfqpoint{3.696000in}{3.696000in}}%
\pgfusepath{clip}%
\pgfsetbuttcap%
\pgfsetroundjoin%
\definecolor{currentfill}{rgb}{0.121569,0.466667,0.705882}%
\pgfsetfillcolor{currentfill}%
\pgfsetfillopacity{0.872047}%
\pgfsetlinewidth{1.003750pt}%
\definecolor{currentstroke}{rgb}{0.121569,0.466667,0.705882}%
\pgfsetstrokecolor{currentstroke}%
\pgfsetstrokeopacity{0.872047}%
\pgfsetdash{}{0pt}%
\pgfpathmoveto{\pgfqpoint{2.312621in}{1.497582in}}%
\pgfpathcurveto{\pgfqpoint{2.320857in}{1.497582in}}{\pgfqpoint{2.328757in}{1.500855in}}{\pgfqpoint{2.334581in}{1.506679in}}%
\pgfpathcurveto{\pgfqpoint{2.340405in}{1.512503in}}{\pgfqpoint{2.343677in}{1.520403in}}{\pgfqpoint{2.343677in}{1.528639in}}%
\pgfpathcurveto{\pgfqpoint{2.343677in}{1.536875in}}{\pgfqpoint{2.340405in}{1.544775in}}{\pgfqpoint{2.334581in}{1.550599in}}%
\pgfpathcurveto{\pgfqpoint{2.328757in}{1.556423in}}{\pgfqpoint{2.320857in}{1.559695in}}{\pgfqpoint{2.312621in}{1.559695in}}%
\pgfpathcurveto{\pgfqpoint{2.304384in}{1.559695in}}{\pgfqpoint{2.296484in}{1.556423in}}{\pgfqpoint{2.290661in}{1.550599in}}%
\pgfpathcurveto{\pgfqpoint{2.284837in}{1.544775in}}{\pgfqpoint{2.281564in}{1.536875in}}{\pgfqpoint{2.281564in}{1.528639in}}%
\pgfpathcurveto{\pgfqpoint{2.281564in}{1.520403in}}{\pgfqpoint{2.284837in}{1.512503in}}{\pgfqpoint{2.290661in}{1.506679in}}%
\pgfpathcurveto{\pgfqpoint{2.296484in}{1.500855in}}{\pgfqpoint{2.304384in}{1.497582in}}{\pgfqpoint{2.312621in}{1.497582in}}%
\pgfpathclose%
\pgfusepath{stroke,fill}%
\end{pgfscope}%
\begin{pgfscope}%
\pgfpathrectangle{\pgfqpoint{0.100000in}{0.212622in}}{\pgfqpoint{3.696000in}{3.696000in}}%
\pgfusepath{clip}%
\pgfsetbuttcap%
\pgfsetroundjoin%
\definecolor{currentfill}{rgb}{0.121569,0.466667,0.705882}%
\pgfsetfillcolor{currentfill}%
\pgfsetfillopacity{0.872395}%
\pgfsetlinewidth{1.003750pt}%
\definecolor{currentstroke}{rgb}{0.121569,0.466667,0.705882}%
\pgfsetstrokecolor{currentstroke}%
\pgfsetstrokeopacity{0.872395}%
\pgfsetdash{}{0pt}%
\pgfpathmoveto{\pgfqpoint{1.125519in}{2.211248in}}%
\pgfpathcurveto{\pgfqpoint{1.133755in}{2.211248in}}{\pgfqpoint{1.141655in}{2.214521in}}{\pgfqpoint{1.147479in}{2.220345in}}%
\pgfpathcurveto{\pgfqpoint{1.153303in}{2.226168in}}{\pgfqpoint{1.156575in}{2.234068in}}{\pgfqpoint{1.156575in}{2.242305in}}%
\pgfpathcurveto{\pgfqpoint{1.156575in}{2.250541in}}{\pgfqpoint{1.153303in}{2.258441in}}{\pgfqpoint{1.147479in}{2.264265in}}%
\pgfpathcurveto{\pgfqpoint{1.141655in}{2.270089in}}{\pgfqpoint{1.133755in}{2.273361in}}{\pgfqpoint{1.125519in}{2.273361in}}%
\pgfpathcurveto{\pgfqpoint{1.117282in}{2.273361in}}{\pgfqpoint{1.109382in}{2.270089in}}{\pgfqpoint{1.103558in}{2.264265in}}%
\pgfpathcurveto{\pgfqpoint{1.097735in}{2.258441in}}{\pgfqpoint{1.094462in}{2.250541in}}{\pgfqpoint{1.094462in}{2.242305in}}%
\pgfpathcurveto{\pgfqpoint{1.094462in}{2.234068in}}{\pgfqpoint{1.097735in}{2.226168in}}{\pgfqpoint{1.103558in}{2.220345in}}%
\pgfpathcurveto{\pgfqpoint{1.109382in}{2.214521in}}{\pgfqpoint{1.117282in}{2.211248in}}{\pgfqpoint{1.125519in}{2.211248in}}%
\pgfpathclose%
\pgfusepath{stroke,fill}%
\end{pgfscope}%
\begin{pgfscope}%
\pgfpathrectangle{\pgfqpoint{0.100000in}{0.212622in}}{\pgfqpoint{3.696000in}{3.696000in}}%
\pgfusepath{clip}%
\pgfsetbuttcap%
\pgfsetroundjoin%
\definecolor{currentfill}{rgb}{0.121569,0.466667,0.705882}%
\pgfsetfillcolor{currentfill}%
\pgfsetfillopacity{0.872857}%
\pgfsetlinewidth{1.003750pt}%
\definecolor{currentstroke}{rgb}{0.121569,0.466667,0.705882}%
\pgfsetstrokecolor{currentstroke}%
\pgfsetstrokeopacity{0.872857}%
\pgfsetdash{}{0pt}%
\pgfpathmoveto{\pgfqpoint{2.313249in}{1.496809in}}%
\pgfpathcurveto{\pgfqpoint{2.321485in}{1.496809in}}{\pgfqpoint{2.329385in}{1.500082in}}{\pgfqpoint{2.335209in}{1.505906in}}%
\pgfpathcurveto{\pgfqpoint{2.341033in}{1.511730in}}{\pgfqpoint{2.344305in}{1.519630in}}{\pgfqpoint{2.344305in}{1.527866in}}%
\pgfpathcurveto{\pgfqpoint{2.344305in}{1.536102in}}{\pgfqpoint{2.341033in}{1.544002in}}{\pgfqpoint{2.335209in}{1.549826in}}%
\pgfpathcurveto{\pgfqpoint{2.329385in}{1.555650in}}{\pgfqpoint{2.321485in}{1.558922in}}{\pgfqpoint{2.313249in}{1.558922in}}%
\pgfpathcurveto{\pgfqpoint{2.305012in}{1.558922in}}{\pgfqpoint{2.297112in}{1.555650in}}{\pgfqpoint{2.291288in}{1.549826in}}%
\pgfpathcurveto{\pgfqpoint{2.285465in}{1.544002in}}{\pgfqpoint{2.282192in}{1.536102in}}{\pgfqpoint{2.282192in}{1.527866in}}%
\pgfpathcurveto{\pgfqpoint{2.282192in}{1.519630in}}{\pgfqpoint{2.285465in}{1.511730in}}{\pgfqpoint{2.291288in}{1.505906in}}%
\pgfpathcurveto{\pgfqpoint{2.297112in}{1.500082in}}{\pgfqpoint{2.305012in}{1.496809in}}{\pgfqpoint{2.313249in}{1.496809in}}%
\pgfpathclose%
\pgfusepath{stroke,fill}%
\end{pgfscope}%
\begin{pgfscope}%
\pgfpathrectangle{\pgfqpoint{0.100000in}{0.212622in}}{\pgfqpoint{3.696000in}{3.696000in}}%
\pgfusepath{clip}%
\pgfsetbuttcap%
\pgfsetroundjoin%
\definecolor{currentfill}{rgb}{0.121569,0.466667,0.705882}%
\pgfsetfillcolor{currentfill}%
\pgfsetfillopacity{0.873198}%
\pgfsetlinewidth{1.003750pt}%
\definecolor{currentstroke}{rgb}{0.121569,0.466667,0.705882}%
\pgfsetstrokecolor{currentstroke}%
\pgfsetstrokeopacity{0.873198}%
\pgfsetdash{}{0pt}%
\pgfpathmoveto{\pgfqpoint{2.313617in}{1.495766in}}%
\pgfpathcurveto{\pgfqpoint{2.321853in}{1.495766in}}{\pgfqpoint{2.329753in}{1.499038in}}{\pgfqpoint{2.335577in}{1.504862in}}%
\pgfpathcurveto{\pgfqpoint{2.341401in}{1.510686in}}{\pgfqpoint{2.344673in}{1.518586in}}{\pgfqpoint{2.344673in}{1.526822in}}%
\pgfpathcurveto{\pgfqpoint{2.344673in}{1.535059in}}{\pgfqpoint{2.341401in}{1.542959in}}{\pgfqpoint{2.335577in}{1.548783in}}%
\pgfpathcurveto{\pgfqpoint{2.329753in}{1.554607in}}{\pgfqpoint{2.321853in}{1.557879in}}{\pgfqpoint{2.313617in}{1.557879in}}%
\pgfpathcurveto{\pgfqpoint{2.305380in}{1.557879in}}{\pgfqpoint{2.297480in}{1.554607in}}{\pgfqpoint{2.291657in}{1.548783in}}%
\pgfpathcurveto{\pgfqpoint{2.285833in}{1.542959in}}{\pgfqpoint{2.282560in}{1.535059in}}{\pgfqpoint{2.282560in}{1.526822in}}%
\pgfpathcurveto{\pgfqpoint{2.282560in}{1.518586in}}{\pgfqpoint{2.285833in}{1.510686in}}{\pgfqpoint{2.291657in}{1.504862in}}%
\pgfpathcurveto{\pgfqpoint{2.297480in}{1.499038in}}{\pgfqpoint{2.305380in}{1.495766in}}{\pgfqpoint{2.313617in}{1.495766in}}%
\pgfpathclose%
\pgfusepath{stroke,fill}%
\end{pgfscope}%
\begin{pgfscope}%
\pgfpathrectangle{\pgfqpoint{0.100000in}{0.212622in}}{\pgfqpoint{3.696000in}{3.696000in}}%
\pgfusepath{clip}%
\pgfsetbuttcap%
\pgfsetroundjoin%
\definecolor{currentfill}{rgb}{0.121569,0.466667,0.705882}%
\pgfsetfillcolor{currentfill}%
\pgfsetfillopacity{0.874176}%
\pgfsetlinewidth{1.003750pt}%
\definecolor{currentstroke}{rgb}{0.121569,0.466667,0.705882}%
\pgfsetstrokecolor{currentstroke}%
\pgfsetstrokeopacity{0.874176}%
\pgfsetdash{}{0pt}%
\pgfpathmoveto{\pgfqpoint{2.314193in}{1.494892in}}%
\pgfpathcurveto{\pgfqpoint{2.322429in}{1.494892in}}{\pgfqpoint{2.330329in}{1.498164in}}{\pgfqpoint{2.336153in}{1.503988in}}%
\pgfpathcurveto{\pgfqpoint{2.341977in}{1.509812in}}{\pgfqpoint{2.345249in}{1.517712in}}{\pgfqpoint{2.345249in}{1.525948in}}%
\pgfpathcurveto{\pgfqpoint{2.345249in}{1.534185in}}{\pgfqpoint{2.341977in}{1.542085in}}{\pgfqpoint{2.336153in}{1.547909in}}%
\pgfpathcurveto{\pgfqpoint{2.330329in}{1.553732in}}{\pgfqpoint{2.322429in}{1.557005in}}{\pgfqpoint{2.314193in}{1.557005in}}%
\pgfpathcurveto{\pgfqpoint{2.305956in}{1.557005in}}{\pgfqpoint{2.298056in}{1.553732in}}{\pgfqpoint{2.292232in}{1.547909in}}%
\pgfpathcurveto{\pgfqpoint{2.286408in}{1.542085in}}{\pgfqpoint{2.283136in}{1.534185in}}{\pgfqpoint{2.283136in}{1.525948in}}%
\pgfpathcurveto{\pgfqpoint{2.283136in}{1.517712in}}{\pgfqpoint{2.286408in}{1.509812in}}{\pgfqpoint{2.292232in}{1.503988in}}%
\pgfpathcurveto{\pgfqpoint{2.298056in}{1.498164in}}{\pgfqpoint{2.305956in}{1.494892in}}{\pgfqpoint{2.314193in}{1.494892in}}%
\pgfpathclose%
\pgfusepath{stroke,fill}%
\end{pgfscope}%
\begin{pgfscope}%
\pgfpathrectangle{\pgfqpoint{0.100000in}{0.212622in}}{\pgfqpoint{3.696000in}{3.696000in}}%
\pgfusepath{clip}%
\pgfsetbuttcap%
\pgfsetroundjoin%
\definecolor{currentfill}{rgb}{0.121569,0.466667,0.705882}%
\pgfsetfillcolor{currentfill}%
\pgfsetfillopacity{0.875798}%
\pgfsetlinewidth{1.003750pt}%
\definecolor{currentstroke}{rgb}{0.121569,0.466667,0.705882}%
\pgfsetstrokecolor{currentstroke}%
\pgfsetstrokeopacity{0.875798}%
\pgfsetdash{}{0pt}%
\pgfpathmoveto{\pgfqpoint{2.315229in}{1.494158in}}%
\pgfpathcurveto{\pgfqpoint{2.323466in}{1.494158in}}{\pgfqpoint{2.331366in}{1.497430in}}{\pgfqpoint{2.337190in}{1.503254in}}%
\pgfpathcurveto{\pgfqpoint{2.343013in}{1.509078in}}{\pgfqpoint{2.346286in}{1.516978in}}{\pgfqpoint{2.346286in}{1.525214in}}%
\pgfpathcurveto{\pgfqpoint{2.346286in}{1.533450in}}{\pgfqpoint{2.343013in}{1.541350in}}{\pgfqpoint{2.337190in}{1.547174in}}%
\pgfpathcurveto{\pgfqpoint{2.331366in}{1.552998in}}{\pgfqpoint{2.323466in}{1.556271in}}{\pgfqpoint{2.315229in}{1.556271in}}%
\pgfpathcurveto{\pgfqpoint{2.306993in}{1.556271in}}{\pgfqpoint{2.299093in}{1.552998in}}{\pgfqpoint{2.293269in}{1.547174in}}%
\pgfpathcurveto{\pgfqpoint{2.287445in}{1.541350in}}{\pgfqpoint{2.284173in}{1.533450in}}{\pgfqpoint{2.284173in}{1.525214in}}%
\pgfpathcurveto{\pgfqpoint{2.284173in}{1.516978in}}{\pgfqpoint{2.287445in}{1.509078in}}{\pgfqpoint{2.293269in}{1.503254in}}%
\pgfpathcurveto{\pgfqpoint{2.299093in}{1.497430in}}{\pgfqpoint{2.306993in}{1.494158in}}{\pgfqpoint{2.315229in}{1.494158in}}%
\pgfpathclose%
\pgfusepath{stroke,fill}%
\end{pgfscope}%
\begin{pgfscope}%
\pgfpathrectangle{\pgfqpoint{0.100000in}{0.212622in}}{\pgfqpoint{3.696000in}{3.696000in}}%
\pgfusepath{clip}%
\pgfsetbuttcap%
\pgfsetroundjoin%
\definecolor{currentfill}{rgb}{0.121569,0.466667,0.705882}%
\pgfsetfillcolor{currentfill}%
\pgfsetfillopacity{0.876976}%
\pgfsetlinewidth{1.003750pt}%
\definecolor{currentstroke}{rgb}{0.121569,0.466667,0.705882}%
\pgfsetstrokecolor{currentstroke}%
\pgfsetstrokeopacity{0.876976}%
\pgfsetdash{}{0pt}%
\pgfpathmoveto{\pgfqpoint{1.177501in}{2.169494in}}%
\pgfpathcurveto{\pgfqpoint{1.185738in}{2.169494in}}{\pgfqpoint{1.193638in}{2.172766in}}{\pgfqpoint{1.199462in}{2.178590in}}%
\pgfpathcurveto{\pgfqpoint{1.205286in}{2.184414in}}{\pgfqpoint{1.208558in}{2.192314in}}{\pgfqpoint{1.208558in}{2.200550in}}%
\pgfpathcurveto{\pgfqpoint{1.208558in}{2.208787in}}{\pgfqpoint{1.205286in}{2.216687in}}{\pgfqpoint{1.199462in}{2.222511in}}%
\pgfpathcurveto{\pgfqpoint{1.193638in}{2.228335in}}{\pgfqpoint{1.185738in}{2.231607in}}{\pgfqpoint{1.177501in}{2.231607in}}%
\pgfpathcurveto{\pgfqpoint{1.169265in}{2.231607in}}{\pgfqpoint{1.161365in}{2.228335in}}{\pgfqpoint{1.155541in}{2.222511in}}%
\pgfpathcurveto{\pgfqpoint{1.149717in}{2.216687in}}{\pgfqpoint{1.146445in}{2.208787in}}{\pgfqpoint{1.146445in}{2.200550in}}%
\pgfpathcurveto{\pgfqpoint{1.146445in}{2.192314in}}{\pgfqpoint{1.149717in}{2.184414in}}{\pgfqpoint{1.155541in}{2.178590in}}%
\pgfpathcurveto{\pgfqpoint{1.161365in}{2.172766in}}{\pgfqpoint{1.169265in}{2.169494in}}{\pgfqpoint{1.177501in}{2.169494in}}%
\pgfpathclose%
\pgfusepath{stroke,fill}%
\end{pgfscope}%
\begin{pgfscope}%
\pgfpathrectangle{\pgfqpoint{0.100000in}{0.212622in}}{\pgfqpoint{3.696000in}{3.696000in}}%
\pgfusepath{clip}%
\pgfsetbuttcap%
\pgfsetroundjoin%
\definecolor{currentfill}{rgb}{0.121569,0.466667,0.705882}%
\pgfsetfillcolor{currentfill}%
\pgfsetfillopacity{0.877854}%
\pgfsetlinewidth{1.003750pt}%
\definecolor{currentstroke}{rgb}{0.121569,0.466667,0.705882}%
\pgfsetstrokecolor{currentstroke}%
\pgfsetstrokeopacity{0.877854}%
\pgfsetdash{}{0pt}%
\pgfpathmoveto{\pgfqpoint{2.317063in}{1.490749in}}%
\pgfpathcurveto{\pgfqpoint{2.325299in}{1.490749in}}{\pgfqpoint{2.333199in}{1.494022in}}{\pgfqpoint{2.339023in}{1.499845in}}%
\pgfpathcurveto{\pgfqpoint{2.344847in}{1.505669in}}{\pgfqpoint{2.348119in}{1.513569in}}{\pgfqpoint{2.348119in}{1.521806in}}%
\pgfpathcurveto{\pgfqpoint{2.348119in}{1.530042in}}{\pgfqpoint{2.344847in}{1.537942in}}{\pgfqpoint{2.339023in}{1.543766in}}%
\pgfpathcurveto{\pgfqpoint{2.333199in}{1.549590in}}{\pgfqpoint{2.325299in}{1.552862in}}{\pgfqpoint{2.317063in}{1.552862in}}%
\pgfpathcurveto{\pgfqpoint{2.308826in}{1.552862in}}{\pgfqpoint{2.300926in}{1.549590in}}{\pgfqpoint{2.295102in}{1.543766in}}%
\pgfpathcurveto{\pgfqpoint{2.289278in}{1.537942in}}{\pgfqpoint{2.286006in}{1.530042in}}{\pgfqpoint{2.286006in}{1.521806in}}%
\pgfpathcurveto{\pgfqpoint{2.286006in}{1.513569in}}{\pgfqpoint{2.289278in}{1.505669in}}{\pgfqpoint{2.295102in}{1.499845in}}%
\pgfpathcurveto{\pgfqpoint{2.300926in}{1.494022in}}{\pgfqpoint{2.308826in}{1.490749in}}{\pgfqpoint{2.317063in}{1.490749in}}%
\pgfpathclose%
\pgfusepath{stroke,fill}%
\end{pgfscope}%
\begin{pgfscope}%
\pgfpathrectangle{\pgfqpoint{0.100000in}{0.212622in}}{\pgfqpoint{3.696000in}{3.696000in}}%
\pgfusepath{clip}%
\pgfsetbuttcap%
\pgfsetroundjoin%
\definecolor{currentfill}{rgb}{0.121569,0.466667,0.705882}%
\pgfsetfillcolor{currentfill}%
\pgfsetfillopacity{0.878986}%
\pgfsetlinewidth{1.003750pt}%
\definecolor{currentstroke}{rgb}{0.121569,0.466667,0.705882}%
\pgfsetstrokecolor{currentstroke}%
\pgfsetstrokeopacity{0.878986}%
\pgfsetdash{}{0pt}%
\pgfpathmoveto{\pgfqpoint{2.317771in}{1.488707in}}%
\pgfpathcurveto{\pgfqpoint{2.326007in}{1.488707in}}{\pgfqpoint{2.333907in}{1.491979in}}{\pgfqpoint{2.339731in}{1.497803in}}%
\pgfpathcurveto{\pgfqpoint{2.345555in}{1.503627in}}{\pgfqpoint{2.348827in}{1.511527in}}{\pgfqpoint{2.348827in}{1.519764in}}%
\pgfpathcurveto{\pgfqpoint{2.348827in}{1.528000in}}{\pgfqpoint{2.345555in}{1.535900in}}{\pgfqpoint{2.339731in}{1.541724in}}%
\pgfpathcurveto{\pgfqpoint{2.333907in}{1.547548in}}{\pgfqpoint{2.326007in}{1.550820in}}{\pgfqpoint{2.317771in}{1.550820in}}%
\pgfpathcurveto{\pgfqpoint{2.309534in}{1.550820in}}{\pgfqpoint{2.301634in}{1.547548in}}{\pgfqpoint{2.295810in}{1.541724in}}%
\pgfpathcurveto{\pgfqpoint{2.289986in}{1.535900in}}{\pgfqpoint{2.286714in}{1.528000in}}{\pgfqpoint{2.286714in}{1.519764in}}%
\pgfpathcurveto{\pgfqpoint{2.286714in}{1.511527in}}{\pgfqpoint{2.289986in}{1.503627in}}{\pgfqpoint{2.295810in}{1.497803in}}%
\pgfpathcurveto{\pgfqpoint{2.301634in}{1.491979in}}{\pgfqpoint{2.309534in}{1.488707in}}{\pgfqpoint{2.317771in}{1.488707in}}%
\pgfpathclose%
\pgfusepath{stroke,fill}%
\end{pgfscope}%
\begin{pgfscope}%
\pgfpathrectangle{\pgfqpoint{0.100000in}{0.212622in}}{\pgfqpoint{3.696000in}{3.696000in}}%
\pgfusepath{clip}%
\pgfsetbuttcap%
\pgfsetroundjoin%
\definecolor{currentfill}{rgb}{0.121569,0.466667,0.705882}%
\pgfsetfillcolor{currentfill}%
\pgfsetfillopacity{0.879650}%
\pgfsetlinewidth{1.003750pt}%
\definecolor{currentstroke}{rgb}{0.121569,0.466667,0.705882}%
\pgfsetstrokecolor{currentstroke}%
\pgfsetstrokeopacity{0.879650}%
\pgfsetdash{}{0pt}%
\pgfpathmoveto{\pgfqpoint{2.318177in}{1.487846in}}%
\pgfpathcurveto{\pgfqpoint{2.326414in}{1.487846in}}{\pgfqpoint{2.334314in}{1.491119in}}{\pgfqpoint{2.340138in}{1.496943in}}%
\pgfpathcurveto{\pgfqpoint{2.345962in}{1.502767in}}{\pgfqpoint{2.349234in}{1.510667in}}{\pgfqpoint{2.349234in}{1.518903in}}%
\pgfpathcurveto{\pgfqpoint{2.349234in}{1.527139in}}{\pgfqpoint{2.345962in}{1.535039in}}{\pgfqpoint{2.340138in}{1.540863in}}%
\pgfpathcurveto{\pgfqpoint{2.334314in}{1.546687in}}{\pgfqpoint{2.326414in}{1.549959in}}{\pgfqpoint{2.318177in}{1.549959in}}%
\pgfpathcurveto{\pgfqpoint{2.309941in}{1.549959in}}{\pgfqpoint{2.302041in}{1.546687in}}{\pgfqpoint{2.296217in}{1.540863in}}%
\pgfpathcurveto{\pgfqpoint{2.290393in}{1.535039in}}{\pgfqpoint{2.287121in}{1.527139in}}{\pgfqpoint{2.287121in}{1.518903in}}%
\pgfpathcurveto{\pgfqpoint{2.287121in}{1.510667in}}{\pgfqpoint{2.290393in}{1.502767in}}{\pgfqpoint{2.296217in}{1.496943in}}%
\pgfpathcurveto{\pgfqpoint{2.302041in}{1.491119in}}{\pgfqpoint{2.309941in}{1.487846in}}{\pgfqpoint{2.318177in}{1.487846in}}%
\pgfpathclose%
\pgfusepath{stroke,fill}%
\end{pgfscope}%
\begin{pgfscope}%
\pgfpathrectangle{\pgfqpoint{0.100000in}{0.212622in}}{\pgfqpoint{3.696000in}{3.696000in}}%
\pgfusepath{clip}%
\pgfsetbuttcap%
\pgfsetroundjoin%
\definecolor{currentfill}{rgb}{0.121569,0.466667,0.705882}%
\pgfsetfillcolor{currentfill}%
\pgfsetfillopacity{0.880019}%
\pgfsetlinewidth{1.003750pt}%
\definecolor{currentstroke}{rgb}{0.121569,0.466667,0.705882}%
\pgfsetstrokecolor{currentstroke}%
\pgfsetstrokeopacity{0.880019}%
\pgfsetdash{}{0pt}%
\pgfpathmoveto{\pgfqpoint{2.318392in}{1.487394in}}%
\pgfpathcurveto{\pgfqpoint{2.326628in}{1.487394in}}{\pgfqpoint{2.334528in}{1.490666in}}{\pgfqpoint{2.340352in}{1.496490in}}%
\pgfpathcurveto{\pgfqpoint{2.346176in}{1.502314in}}{\pgfqpoint{2.349448in}{1.510214in}}{\pgfqpoint{2.349448in}{1.518450in}}%
\pgfpathcurveto{\pgfqpoint{2.349448in}{1.526687in}}{\pgfqpoint{2.346176in}{1.534587in}}{\pgfqpoint{2.340352in}{1.540411in}}%
\pgfpathcurveto{\pgfqpoint{2.334528in}{1.546235in}}{\pgfqpoint{2.326628in}{1.549507in}}{\pgfqpoint{2.318392in}{1.549507in}}%
\pgfpathcurveto{\pgfqpoint{2.310155in}{1.549507in}}{\pgfqpoint{2.302255in}{1.546235in}}{\pgfqpoint{2.296431in}{1.540411in}}%
\pgfpathcurveto{\pgfqpoint{2.290607in}{1.534587in}}{\pgfqpoint{2.287335in}{1.526687in}}{\pgfqpoint{2.287335in}{1.518450in}}%
\pgfpathcurveto{\pgfqpoint{2.287335in}{1.510214in}}{\pgfqpoint{2.290607in}{1.502314in}}{\pgfqpoint{2.296431in}{1.496490in}}%
\pgfpathcurveto{\pgfqpoint{2.302255in}{1.490666in}}{\pgfqpoint{2.310155in}{1.487394in}}{\pgfqpoint{2.318392in}{1.487394in}}%
\pgfpathclose%
\pgfusepath{stroke,fill}%
\end{pgfscope}%
\begin{pgfscope}%
\pgfpathrectangle{\pgfqpoint{0.100000in}{0.212622in}}{\pgfqpoint{3.696000in}{3.696000in}}%
\pgfusepath{clip}%
\pgfsetbuttcap%
\pgfsetroundjoin%
\definecolor{currentfill}{rgb}{0.121569,0.466667,0.705882}%
\pgfsetfillcolor{currentfill}%
\pgfsetfillopacity{0.880698}%
\pgfsetlinewidth{1.003750pt}%
\definecolor{currentstroke}{rgb}{0.121569,0.466667,0.705882}%
\pgfsetstrokecolor{currentstroke}%
\pgfsetstrokeopacity{0.880698}%
\pgfsetdash{}{0pt}%
\pgfpathmoveto{\pgfqpoint{2.319006in}{1.486506in}}%
\pgfpathcurveto{\pgfqpoint{2.327242in}{1.486506in}}{\pgfqpoint{2.335142in}{1.489778in}}{\pgfqpoint{2.340966in}{1.495602in}}%
\pgfpathcurveto{\pgfqpoint{2.346790in}{1.501426in}}{\pgfqpoint{2.350062in}{1.509326in}}{\pgfqpoint{2.350062in}{1.517562in}}%
\pgfpathcurveto{\pgfqpoint{2.350062in}{1.525799in}}{\pgfqpoint{2.346790in}{1.533699in}}{\pgfqpoint{2.340966in}{1.539523in}}%
\pgfpathcurveto{\pgfqpoint{2.335142in}{1.545347in}}{\pgfqpoint{2.327242in}{1.548619in}}{\pgfqpoint{2.319006in}{1.548619in}}%
\pgfpathcurveto{\pgfqpoint{2.310769in}{1.548619in}}{\pgfqpoint{2.302869in}{1.545347in}}{\pgfqpoint{2.297045in}{1.539523in}}%
\pgfpathcurveto{\pgfqpoint{2.291221in}{1.533699in}}{\pgfqpoint{2.287949in}{1.525799in}}{\pgfqpoint{2.287949in}{1.517562in}}%
\pgfpathcurveto{\pgfqpoint{2.287949in}{1.509326in}}{\pgfqpoint{2.291221in}{1.501426in}}{\pgfqpoint{2.297045in}{1.495602in}}%
\pgfpathcurveto{\pgfqpoint{2.302869in}{1.489778in}}{\pgfqpoint{2.310769in}{1.486506in}}{\pgfqpoint{2.319006in}{1.486506in}}%
\pgfpathclose%
\pgfusepath{stroke,fill}%
\end{pgfscope}%
\begin{pgfscope}%
\pgfpathrectangle{\pgfqpoint{0.100000in}{0.212622in}}{\pgfqpoint{3.696000in}{3.696000in}}%
\pgfusepath{clip}%
\pgfsetbuttcap%
\pgfsetroundjoin%
\definecolor{currentfill}{rgb}{0.121569,0.466667,0.705882}%
\pgfsetfillcolor{currentfill}%
\pgfsetfillopacity{0.881095}%
\pgfsetlinewidth{1.003750pt}%
\definecolor{currentstroke}{rgb}{0.121569,0.466667,0.705882}%
\pgfsetstrokecolor{currentstroke}%
\pgfsetstrokeopacity{0.881095}%
\pgfsetdash{}{0pt}%
\pgfpathmoveto{\pgfqpoint{2.319359in}{1.486173in}}%
\pgfpathcurveto{\pgfqpoint{2.327595in}{1.486173in}}{\pgfqpoint{2.335495in}{1.489445in}}{\pgfqpoint{2.341319in}{1.495269in}}%
\pgfpathcurveto{\pgfqpoint{2.347143in}{1.501093in}}{\pgfqpoint{2.350415in}{1.508993in}}{\pgfqpoint{2.350415in}{1.517229in}}%
\pgfpathcurveto{\pgfqpoint{2.350415in}{1.525465in}}{\pgfqpoint{2.347143in}{1.533365in}}{\pgfqpoint{2.341319in}{1.539189in}}%
\pgfpathcurveto{\pgfqpoint{2.335495in}{1.545013in}}{\pgfqpoint{2.327595in}{1.548286in}}{\pgfqpoint{2.319359in}{1.548286in}}%
\pgfpathcurveto{\pgfqpoint{2.311123in}{1.548286in}}{\pgfqpoint{2.303223in}{1.545013in}}{\pgfqpoint{2.297399in}{1.539189in}}%
\pgfpathcurveto{\pgfqpoint{2.291575in}{1.533365in}}{\pgfqpoint{2.288302in}{1.525465in}}{\pgfqpoint{2.288302in}{1.517229in}}%
\pgfpathcurveto{\pgfqpoint{2.288302in}{1.508993in}}{\pgfqpoint{2.291575in}{1.501093in}}{\pgfqpoint{2.297399in}{1.495269in}}%
\pgfpathcurveto{\pgfqpoint{2.303223in}{1.489445in}}{\pgfqpoint{2.311123in}{1.486173in}}{\pgfqpoint{2.319359in}{1.486173in}}%
\pgfpathclose%
\pgfusepath{stroke,fill}%
\end{pgfscope}%
\begin{pgfscope}%
\pgfpathrectangle{\pgfqpoint{0.100000in}{0.212622in}}{\pgfqpoint{3.696000in}{3.696000in}}%
\pgfusepath{clip}%
\pgfsetbuttcap%
\pgfsetroundjoin%
\definecolor{currentfill}{rgb}{0.121569,0.466667,0.705882}%
\pgfsetfillcolor{currentfill}%
\pgfsetfillopacity{0.881631}%
\pgfsetlinewidth{1.003750pt}%
\definecolor{currentstroke}{rgb}{0.121569,0.466667,0.705882}%
\pgfsetstrokecolor{currentstroke}%
\pgfsetstrokeopacity{0.881631}%
\pgfsetdash{}{0pt}%
\pgfpathmoveto{\pgfqpoint{2.319749in}{1.484840in}}%
\pgfpathcurveto{\pgfqpoint{2.327985in}{1.484840in}}{\pgfqpoint{2.335885in}{1.488112in}}{\pgfqpoint{2.341709in}{1.493936in}}%
\pgfpathcurveto{\pgfqpoint{2.347533in}{1.499760in}}{\pgfqpoint{2.350805in}{1.507660in}}{\pgfqpoint{2.350805in}{1.515896in}}%
\pgfpathcurveto{\pgfqpoint{2.350805in}{1.524133in}}{\pgfqpoint{2.347533in}{1.532033in}}{\pgfqpoint{2.341709in}{1.537856in}}%
\pgfpathcurveto{\pgfqpoint{2.335885in}{1.543680in}}{\pgfqpoint{2.327985in}{1.546953in}}{\pgfqpoint{2.319749in}{1.546953in}}%
\pgfpathcurveto{\pgfqpoint{2.311512in}{1.546953in}}{\pgfqpoint{2.303612in}{1.543680in}}{\pgfqpoint{2.297788in}{1.537856in}}%
\pgfpathcurveto{\pgfqpoint{2.291964in}{1.532033in}}{\pgfqpoint{2.288692in}{1.524133in}}{\pgfqpoint{2.288692in}{1.515896in}}%
\pgfpathcurveto{\pgfqpoint{2.288692in}{1.507660in}}{\pgfqpoint{2.291964in}{1.499760in}}{\pgfqpoint{2.297788in}{1.493936in}}%
\pgfpathcurveto{\pgfqpoint{2.303612in}{1.488112in}}{\pgfqpoint{2.311512in}{1.484840in}}{\pgfqpoint{2.319749in}{1.484840in}}%
\pgfpathclose%
\pgfusepath{stroke,fill}%
\end{pgfscope}%
\begin{pgfscope}%
\pgfpathrectangle{\pgfqpoint{0.100000in}{0.212622in}}{\pgfqpoint{3.696000in}{3.696000in}}%
\pgfusepath{clip}%
\pgfsetbuttcap%
\pgfsetroundjoin%
\definecolor{currentfill}{rgb}{0.121569,0.466667,0.705882}%
\pgfsetfillcolor{currentfill}%
\pgfsetfillopacity{0.881945}%
\pgfsetlinewidth{1.003750pt}%
\definecolor{currentstroke}{rgb}{0.121569,0.466667,0.705882}%
\pgfsetstrokecolor{currentstroke}%
\pgfsetstrokeopacity{0.881945}%
\pgfsetdash{}{0pt}%
\pgfpathmoveto{\pgfqpoint{1.230234in}{2.139198in}}%
\pgfpathcurveto{\pgfqpoint{1.238470in}{2.139198in}}{\pgfqpoint{1.246370in}{2.142470in}}{\pgfqpoint{1.252194in}{2.148294in}}%
\pgfpathcurveto{\pgfqpoint{1.258018in}{2.154118in}}{\pgfqpoint{1.261291in}{2.162018in}}{\pgfqpoint{1.261291in}{2.170254in}}%
\pgfpathcurveto{\pgfqpoint{1.261291in}{2.178491in}}{\pgfqpoint{1.258018in}{2.186391in}}{\pgfqpoint{1.252194in}{2.192215in}}%
\pgfpathcurveto{\pgfqpoint{1.246370in}{2.198039in}}{\pgfqpoint{1.238470in}{2.201311in}}{\pgfqpoint{1.230234in}{2.201311in}}%
\pgfpathcurveto{\pgfqpoint{1.221998in}{2.201311in}}{\pgfqpoint{1.214098in}{2.198039in}}{\pgfqpoint{1.208274in}{2.192215in}}%
\pgfpathcurveto{\pgfqpoint{1.202450in}{2.186391in}}{\pgfqpoint{1.199178in}{2.178491in}}{\pgfqpoint{1.199178in}{2.170254in}}%
\pgfpathcurveto{\pgfqpoint{1.199178in}{2.162018in}}{\pgfqpoint{1.202450in}{2.154118in}}{\pgfqpoint{1.208274in}{2.148294in}}%
\pgfpathcurveto{\pgfqpoint{1.214098in}{2.142470in}}{\pgfqpoint{1.221998in}{2.139198in}}{\pgfqpoint{1.230234in}{2.139198in}}%
\pgfpathclose%
\pgfusepath{stroke,fill}%
\end{pgfscope}%
\begin{pgfscope}%
\pgfpathrectangle{\pgfqpoint{0.100000in}{0.212622in}}{\pgfqpoint{3.696000in}{3.696000in}}%
\pgfusepath{clip}%
\pgfsetbuttcap%
\pgfsetroundjoin%
\definecolor{currentfill}{rgb}{0.121569,0.466667,0.705882}%
\pgfsetfillcolor{currentfill}%
\pgfsetfillopacity{0.881997}%
\pgfsetlinewidth{1.003750pt}%
\definecolor{currentstroke}{rgb}{0.121569,0.466667,0.705882}%
\pgfsetstrokecolor{currentstroke}%
\pgfsetstrokeopacity{0.881997}%
\pgfsetdash{}{0pt}%
\pgfpathmoveto{\pgfqpoint{2.319953in}{1.484538in}}%
\pgfpathcurveto{\pgfqpoint{2.328190in}{1.484538in}}{\pgfqpoint{2.336090in}{1.487811in}}{\pgfqpoint{2.341914in}{1.493635in}}%
\pgfpathcurveto{\pgfqpoint{2.347737in}{1.499458in}}{\pgfqpoint{2.351010in}{1.507358in}}{\pgfqpoint{2.351010in}{1.515595in}}%
\pgfpathcurveto{\pgfqpoint{2.351010in}{1.523831in}}{\pgfqpoint{2.347737in}{1.531731in}}{\pgfqpoint{2.341914in}{1.537555in}}%
\pgfpathcurveto{\pgfqpoint{2.336090in}{1.543379in}}{\pgfqpoint{2.328190in}{1.546651in}}{\pgfqpoint{2.319953in}{1.546651in}}%
\pgfpathcurveto{\pgfqpoint{2.311717in}{1.546651in}}{\pgfqpoint{2.303817in}{1.543379in}}{\pgfqpoint{2.297993in}{1.537555in}}%
\pgfpathcurveto{\pgfqpoint{2.292169in}{1.531731in}}{\pgfqpoint{2.288897in}{1.523831in}}{\pgfqpoint{2.288897in}{1.515595in}}%
\pgfpathcurveto{\pgfqpoint{2.288897in}{1.507358in}}{\pgfqpoint{2.292169in}{1.499458in}}{\pgfqpoint{2.297993in}{1.493635in}}%
\pgfpathcurveto{\pgfqpoint{2.303817in}{1.487811in}}{\pgfqpoint{2.311717in}{1.484538in}}{\pgfqpoint{2.319953in}{1.484538in}}%
\pgfpathclose%
\pgfusepath{stroke,fill}%
\end{pgfscope}%
\begin{pgfscope}%
\pgfpathrectangle{\pgfqpoint{0.100000in}{0.212622in}}{\pgfqpoint{3.696000in}{3.696000in}}%
\pgfusepath{clip}%
\pgfsetbuttcap%
\pgfsetroundjoin%
\definecolor{currentfill}{rgb}{0.121569,0.466667,0.705882}%
\pgfsetfillcolor{currentfill}%
\pgfsetfillopacity{0.882205}%
\pgfsetlinewidth{1.003750pt}%
\definecolor{currentstroke}{rgb}{0.121569,0.466667,0.705882}%
\pgfsetstrokecolor{currentstroke}%
\pgfsetstrokeopacity{0.882205}%
\pgfsetdash{}{0pt}%
\pgfpathmoveto{\pgfqpoint{2.320093in}{1.484421in}}%
\pgfpathcurveto{\pgfqpoint{2.328329in}{1.484421in}}{\pgfqpoint{2.336229in}{1.487693in}}{\pgfqpoint{2.342053in}{1.493517in}}%
\pgfpathcurveto{\pgfqpoint{2.347877in}{1.499341in}}{\pgfqpoint{2.351149in}{1.507241in}}{\pgfqpoint{2.351149in}{1.515478in}}%
\pgfpathcurveto{\pgfqpoint{2.351149in}{1.523714in}}{\pgfqpoint{2.347877in}{1.531614in}}{\pgfqpoint{2.342053in}{1.537438in}}%
\pgfpathcurveto{\pgfqpoint{2.336229in}{1.543262in}}{\pgfqpoint{2.328329in}{1.546534in}}{\pgfqpoint{2.320093in}{1.546534in}}%
\pgfpathcurveto{\pgfqpoint{2.311857in}{1.546534in}}{\pgfqpoint{2.303957in}{1.543262in}}{\pgfqpoint{2.298133in}{1.537438in}}%
\pgfpathcurveto{\pgfqpoint{2.292309in}{1.531614in}}{\pgfqpoint{2.289036in}{1.523714in}}{\pgfqpoint{2.289036in}{1.515478in}}%
\pgfpathcurveto{\pgfqpoint{2.289036in}{1.507241in}}{\pgfqpoint{2.292309in}{1.499341in}}{\pgfqpoint{2.298133in}{1.493517in}}%
\pgfpathcurveto{\pgfqpoint{2.303957in}{1.487693in}}{\pgfqpoint{2.311857in}{1.484421in}}{\pgfqpoint{2.320093in}{1.484421in}}%
\pgfpathclose%
\pgfusepath{stroke,fill}%
\end{pgfscope}%
\begin{pgfscope}%
\pgfpathrectangle{\pgfqpoint{0.100000in}{0.212622in}}{\pgfqpoint{3.696000in}{3.696000in}}%
\pgfusepath{clip}%
\pgfsetbuttcap%
\pgfsetroundjoin%
\definecolor{currentfill}{rgb}{0.121569,0.466667,0.705882}%
\pgfsetfillcolor{currentfill}%
\pgfsetfillopacity{0.882832}%
\pgfsetlinewidth{1.003750pt}%
\definecolor{currentstroke}{rgb}{0.121569,0.466667,0.705882}%
\pgfsetstrokecolor{currentstroke}%
\pgfsetstrokeopacity{0.882832}%
\pgfsetdash{}{0pt}%
\pgfpathmoveto{\pgfqpoint{2.320615in}{1.483043in}}%
\pgfpathcurveto{\pgfqpoint{2.328851in}{1.483043in}}{\pgfqpoint{2.336751in}{1.486316in}}{\pgfqpoint{2.342575in}{1.492140in}}%
\pgfpathcurveto{\pgfqpoint{2.348399in}{1.497964in}}{\pgfqpoint{2.351671in}{1.505864in}}{\pgfqpoint{2.351671in}{1.514100in}}%
\pgfpathcurveto{\pgfqpoint{2.351671in}{1.522336in}}{\pgfqpoint{2.348399in}{1.530236in}}{\pgfqpoint{2.342575in}{1.536060in}}%
\pgfpathcurveto{\pgfqpoint{2.336751in}{1.541884in}}{\pgfqpoint{2.328851in}{1.545156in}}{\pgfqpoint{2.320615in}{1.545156in}}%
\pgfpathcurveto{\pgfqpoint{2.312378in}{1.545156in}}{\pgfqpoint{2.304478in}{1.541884in}}{\pgfqpoint{2.298654in}{1.536060in}}%
\pgfpathcurveto{\pgfqpoint{2.292830in}{1.530236in}}{\pgfqpoint{2.289558in}{1.522336in}}{\pgfqpoint{2.289558in}{1.514100in}}%
\pgfpathcurveto{\pgfqpoint{2.289558in}{1.505864in}}{\pgfqpoint{2.292830in}{1.497964in}}{\pgfqpoint{2.298654in}{1.492140in}}%
\pgfpathcurveto{\pgfqpoint{2.304478in}{1.486316in}}{\pgfqpoint{2.312378in}{1.483043in}}{\pgfqpoint{2.320615in}{1.483043in}}%
\pgfpathclose%
\pgfusepath{stroke,fill}%
\end{pgfscope}%
\begin{pgfscope}%
\pgfpathrectangle{\pgfqpoint{0.100000in}{0.212622in}}{\pgfqpoint{3.696000in}{3.696000in}}%
\pgfusepath{clip}%
\pgfsetbuttcap%
\pgfsetroundjoin%
\definecolor{currentfill}{rgb}{0.121569,0.466667,0.705882}%
\pgfsetfillcolor{currentfill}%
\pgfsetfillopacity{0.883565}%
\pgfsetlinewidth{1.003750pt}%
\definecolor{currentstroke}{rgb}{0.121569,0.466667,0.705882}%
\pgfsetstrokecolor{currentstroke}%
\pgfsetstrokeopacity{0.883565}%
\pgfsetdash{}{0pt}%
\pgfpathmoveto{\pgfqpoint{2.321332in}{1.480606in}}%
\pgfpathcurveto{\pgfqpoint{2.329569in}{1.480606in}}{\pgfqpoint{2.337469in}{1.483878in}}{\pgfqpoint{2.343293in}{1.489702in}}%
\pgfpathcurveto{\pgfqpoint{2.349117in}{1.495526in}}{\pgfqpoint{2.352389in}{1.503426in}}{\pgfqpoint{2.352389in}{1.511662in}}%
\pgfpathcurveto{\pgfqpoint{2.352389in}{1.519899in}}{\pgfqpoint{2.349117in}{1.527799in}}{\pgfqpoint{2.343293in}{1.533623in}}%
\pgfpathcurveto{\pgfqpoint{2.337469in}{1.539446in}}{\pgfqpoint{2.329569in}{1.542719in}}{\pgfqpoint{2.321332in}{1.542719in}}%
\pgfpathcurveto{\pgfqpoint{2.313096in}{1.542719in}}{\pgfqpoint{2.305196in}{1.539446in}}{\pgfqpoint{2.299372in}{1.533623in}}%
\pgfpathcurveto{\pgfqpoint{2.293548in}{1.527799in}}{\pgfqpoint{2.290276in}{1.519899in}}{\pgfqpoint{2.290276in}{1.511662in}}%
\pgfpathcurveto{\pgfqpoint{2.290276in}{1.503426in}}{\pgfqpoint{2.293548in}{1.495526in}}{\pgfqpoint{2.299372in}{1.489702in}}%
\pgfpathcurveto{\pgfqpoint{2.305196in}{1.483878in}}{\pgfqpoint{2.313096in}{1.480606in}}{\pgfqpoint{2.321332in}{1.480606in}}%
\pgfpathclose%
\pgfusepath{stroke,fill}%
\end{pgfscope}%
\begin{pgfscope}%
\pgfpathrectangle{\pgfqpoint{0.100000in}{0.212622in}}{\pgfqpoint{3.696000in}{3.696000in}}%
\pgfusepath{clip}%
\pgfsetbuttcap%
\pgfsetroundjoin%
\definecolor{currentfill}{rgb}{0.121569,0.466667,0.705882}%
\pgfsetfillcolor{currentfill}%
\pgfsetfillopacity{0.885011}%
\pgfsetlinewidth{1.003750pt}%
\definecolor{currentstroke}{rgb}{0.121569,0.466667,0.705882}%
\pgfsetstrokecolor{currentstroke}%
\pgfsetstrokeopacity{0.885011}%
\pgfsetdash{}{0pt}%
\pgfpathmoveto{\pgfqpoint{2.322289in}{1.478816in}}%
\pgfpathcurveto{\pgfqpoint{2.330526in}{1.478816in}}{\pgfqpoint{2.338426in}{1.482089in}}{\pgfqpoint{2.344250in}{1.487913in}}%
\pgfpathcurveto{\pgfqpoint{2.350073in}{1.493737in}}{\pgfqpoint{2.353346in}{1.501637in}}{\pgfqpoint{2.353346in}{1.509873in}}%
\pgfpathcurveto{\pgfqpoint{2.353346in}{1.518109in}}{\pgfqpoint{2.350073in}{1.526009in}}{\pgfqpoint{2.344250in}{1.531833in}}%
\pgfpathcurveto{\pgfqpoint{2.338426in}{1.537657in}}{\pgfqpoint{2.330526in}{1.540929in}}{\pgfqpoint{2.322289in}{1.540929in}}%
\pgfpathcurveto{\pgfqpoint{2.314053in}{1.540929in}}{\pgfqpoint{2.306153in}{1.537657in}}{\pgfqpoint{2.300329in}{1.531833in}}%
\pgfpathcurveto{\pgfqpoint{2.294505in}{1.526009in}}{\pgfqpoint{2.291233in}{1.518109in}}{\pgfqpoint{2.291233in}{1.509873in}}%
\pgfpathcurveto{\pgfqpoint{2.291233in}{1.501637in}}{\pgfqpoint{2.294505in}{1.493737in}}{\pgfqpoint{2.300329in}{1.487913in}}%
\pgfpathcurveto{\pgfqpoint{2.306153in}{1.482089in}}{\pgfqpoint{2.314053in}{1.478816in}}{\pgfqpoint{2.322289in}{1.478816in}}%
\pgfpathclose%
\pgfusepath{stroke,fill}%
\end{pgfscope}%
\begin{pgfscope}%
\pgfpathrectangle{\pgfqpoint{0.100000in}{0.212622in}}{\pgfqpoint{3.696000in}{3.696000in}}%
\pgfusepath{clip}%
\pgfsetbuttcap%
\pgfsetroundjoin%
\definecolor{currentfill}{rgb}{0.121569,0.466667,0.705882}%
\pgfsetfillcolor{currentfill}%
\pgfsetfillopacity{0.886792}%
\pgfsetlinewidth{1.003750pt}%
\definecolor{currentstroke}{rgb}{0.121569,0.466667,0.705882}%
\pgfsetstrokecolor{currentstroke}%
\pgfsetstrokeopacity{0.886792}%
\pgfsetdash{}{0pt}%
\pgfpathmoveto{\pgfqpoint{2.323272in}{1.477241in}}%
\pgfpathcurveto{\pgfqpoint{2.331508in}{1.477241in}}{\pgfqpoint{2.339408in}{1.480514in}}{\pgfqpoint{2.345232in}{1.486338in}}%
\pgfpathcurveto{\pgfqpoint{2.351056in}{1.492162in}}{\pgfqpoint{2.354329in}{1.500062in}}{\pgfqpoint{2.354329in}{1.508298in}}%
\pgfpathcurveto{\pgfqpoint{2.354329in}{1.516534in}}{\pgfqpoint{2.351056in}{1.524434in}}{\pgfqpoint{2.345232in}{1.530258in}}%
\pgfpathcurveto{\pgfqpoint{2.339408in}{1.536082in}}{\pgfqpoint{2.331508in}{1.539354in}}{\pgfqpoint{2.323272in}{1.539354in}}%
\pgfpathcurveto{\pgfqpoint{2.315036in}{1.539354in}}{\pgfqpoint{2.307136in}{1.536082in}}{\pgfqpoint{2.301312in}{1.530258in}}%
\pgfpathcurveto{\pgfqpoint{2.295488in}{1.524434in}}{\pgfqpoint{2.292216in}{1.516534in}}{\pgfqpoint{2.292216in}{1.508298in}}%
\pgfpathcurveto{\pgfqpoint{2.292216in}{1.500062in}}{\pgfqpoint{2.295488in}{1.492162in}}{\pgfqpoint{2.301312in}{1.486338in}}%
\pgfpathcurveto{\pgfqpoint{2.307136in}{1.480514in}}{\pgfqpoint{2.315036in}{1.477241in}}{\pgfqpoint{2.323272in}{1.477241in}}%
\pgfpathclose%
\pgfusepath{stroke,fill}%
\end{pgfscope}%
\begin{pgfscope}%
\pgfpathrectangle{\pgfqpoint{0.100000in}{0.212622in}}{\pgfqpoint{3.696000in}{3.696000in}}%
\pgfusepath{clip}%
\pgfsetbuttcap%
\pgfsetroundjoin%
\definecolor{currentfill}{rgb}{0.121569,0.466667,0.705882}%
\pgfsetfillcolor{currentfill}%
\pgfsetfillopacity{0.887800}%
\pgfsetlinewidth{1.003750pt}%
\definecolor{currentstroke}{rgb}{0.121569,0.466667,0.705882}%
\pgfsetstrokecolor{currentstroke}%
\pgfsetstrokeopacity{0.887800}%
\pgfsetdash{}{0pt}%
\pgfpathmoveto{\pgfqpoint{2.323894in}{1.476583in}}%
\pgfpathcurveto{\pgfqpoint{2.332130in}{1.476583in}}{\pgfqpoint{2.340030in}{1.479855in}}{\pgfqpoint{2.345854in}{1.485679in}}%
\pgfpathcurveto{\pgfqpoint{2.351678in}{1.491503in}}{\pgfqpoint{2.354950in}{1.499403in}}{\pgfqpoint{2.354950in}{1.507639in}}%
\pgfpathcurveto{\pgfqpoint{2.354950in}{1.515875in}}{\pgfqpoint{2.351678in}{1.523775in}}{\pgfqpoint{2.345854in}{1.529599in}}%
\pgfpathcurveto{\pgfqpoint{2.340030in}{1.535423in}}{\pgfqpoint{2.332130in}{1.538696in}}{\pgfqpoint{2.323894in}{1.538696in}}%
\pgfpathcurveto{\pgfqpoint{2.315657in}{1.538696in}}{\pgfqpoint{2.307757in}{1.535423in}}{\pgfqpoint{2.301933in}{1.529599in}}%
\pgfpathcurveto{\pgfqpoint{2.296109in}{1.523775in}}{\pgfqpoint{2.292837in}{1.515875in}}{\pgfqpoint{2.292837in}{1.507639in}}%
\pgfpathcurveto{\pgfqpoint{2.292837in}{1.499403in}}{\pgfqpoint{2.296109in}{1.491503in}}{\pgfqpoint{2.301933in}{1.485679in}}%
\pgfpathcurveto{\pgfqpoint{2.307757in}{1.479855in}}{\pgfqpoint{2.315657in}{1.476583in}}{\pgfqpoint{2.323894in}{1.476583in}}%
\pgfpathclose%
\pgfusepath{stroke,fill}%
\end{pgfscope}%
\begin{pgfscope}%
\pgfpathrectangle{\pgfqpoint{0.100000in}{0.212622in}}{\pgfqpoint{3.696000in}{3.696000in}}%
\pgfusepath{clip}%
\pgfsetbuttcap%
\pgfsetroundjoin%
\definecolor{currentfill}{rgb}{0.121569,0.466667,0.705882}%
\pgfsetfillcolor{currentfill}%
\pgfsetfillopacity{0.887833}%
\pgfsetlinewidth{1.003750pt}%
\definecolor{currentstroke}{rgb}{0.121569,0.466667,0.705882}%
\pgfsetstrokecolor{currentstroke}%
\pgfsetstrokeopacity{0.887833}%
\pgfsetdash{}{0pt}%
\pgfpathmoveto{\pgfqpoint{1.277585in}{2.106482in}}%
\pgfpathcurveto{\pgfqpoint{1.285821in}{2.106482in}}{\pgfqpoint{1.293721in}{2.109754in}}{\pgfqpoint{1.299545in}{2.115578in}}%
\pgfpathcurveto{\pgfqpoint{1.305369in}{2.121402in}}{\pgfqpoint{1.308641in}{2.129302in}}{\pgfqpoint{1.308641in}{2.137538in}}%
\pgfpathcurveto{\pgfqpoint{1.308641in}{2.145774in}}{\pgfqpoint{1.305369in}{2.153675in}}{\pgfqpoint{1.299545in}{2.159498in}}%
\pgfpathcurveto{\pgfqpoint{1.293721in}{2.165322in}}{\pgfqpoint{1.285821in}{2.168595in}}{\pgfqpoint{1.277585in}{2.168595in}}%
\pgfpathcurveto{\pgfqpoint{1.269349in}{2.168595in}}{\pgfqpoint{1.261448in}{2.165322in}}{\pgfqpoint{1.255625in}{2.159498in}}%
\pgfpathcurveto{\pgfqpoint{1.249801in}{2.153675in}}{\pgfqpoint{1.246528in}{2.145774in}}{\pgfqpoint{1.246528in}{2.137538in}}%
\pgfpathcurveto{\pgfqpoint{1.246528in}{2.129302in}}{\pgfqpoint{1.249801in}{2.121402in}}{\pgfqpoint{1.255625in}{2.115578in}}%
\pgfpathcurveto{\pgfqpoint{1.261448in}{2.109754in}}{\pgfqpoint{1.269349in}{2.106482in}}{\pgfqpoint{1.277585in}{2.106482in}}%
\pgfpathclose%
\pgfusepath{stroke,fill}%
\end{pgfscope}%
\begin{pgfscope}%
\pgfpathrectangle{\pgfqpoint{0.100000in}{0.212622in}}{\pgfqpoint{3.696000in}{3.696000in}}%
\pgfusepath{clip}%
\pgfsetbuttcap%
\pgfsetroundjoin%
\definecolor{currentfill}{rgb}{0.121569,0.466667,0.705882}%
\pgfsetfillcolor{currentfill}%
\pgfsetfillopacity{0.889006}%
\pgfsetlinewidth{1.003750pt}%
\definecolor{currentstroke}{rgb}{0.121569,0.466667,0.705882}%
\pgfsetstrokecolor{currentstroke}%
\pgfsetstrokeopacity{0.889006}%
\pgfsetdash{}{0pt}%
\pgfpathmoveto{\pgfqpoint{2.324612in}{1.475443in}}%
\pgfpathcurveto{\pgfqpoint{2.332848in}{1.475443in}}{\pgfqpoint{2.340748in}{1.478715in}}{\pgfqpoint{2.346572in}{1.484539in}}%
\pgfpathcurveto{\pgfqpoint{2.352396in}{1.490363in}}{\pgfqpoint{2.355668in}{1.498263in}}{\pgfqpoint{2.355668in}{1.506499in}}%
\pgfpathcurveto{\pgfqpoint{2.355668in}{1.514735in}}{\pgfqpoint{2.352396in}{1.522635in}}{\pgfqpoint{2.346572in}{1.528459in}}%
\pgfpathcurveto{\pgfqpoint{2.340748in}{1.534283in}}{\pgfqpoint{2.332848in}{1.537556in}}{\pgfqpoint{2.324612in}{1.537556in}}%
\pgfpathcurveto{\pgfqpoint{2.316376in}{1.537556in}}{\pgfqpoint{2.308476in}{1.534283in}}{\pgfqpoint{2.302652in}{1.528459in}}%
\pgfpathcurveto{\pgfqpoint{2.296828in}{1.522635in}}{\pgfqpoint{2.293555in}{1.514735in}}{\pgfqpoint{2.293555in}{1.506499in}}%
\pgfpathcurveto{\pgfqpoint{2.293555in}{1.498263in}}{\pgfqpoint{2.296828in}{1.490363in}}{\pgfqpoint{2.302652in}{1.484539in}}%
\pgfpathcurveto{\pgfqpoint{2.308476in}{1.478715in}}{\pgfqpoint{2.316376in}{1.475443in}}{\pgfqpoint{2.324612in}{1.475443in}}%
\pgfpathclose%
\pgfusepath{stroke,fill}%
\end{pgfscope}%
\begin{pgfscope}%
\pgfpathrectangle{\pgfqpoint{0.100000in}{0.212622in}}{\pgfqpoint{3.696000in}{3.696000in}}%
\pgfusepath{clip}%
\pgfsetbuttcap%
\pgfsetroundjoin%
\definecolor{currentfill}{rgb}{0.121569,0.466667,0.705882}%
\pgfsetfillcolor{currentfill}%
\pgfsetfillopacity{0.889580}%
\pgfsetlinewidth{1.003750pt}%
\definecolor{currentstroke}{rgb}{0.121569,0.466667,0.705882}%
\pgfsetstrokecolor{currentstroke}%
\pgfsetstrokeopacity{0.889580}%
\pgfsetdash{}{0pt}%
\pgfpathmoveto{\pgfqpoint{2.325207in}{1.474385in}}%
\pgfpathcurveto{\pgfqpoint{2.333444in}{1.474385in}}{\pgfqpoint{2.341344in}{1.477658in}}{\pgfqpoint{2.347168in}{1.483482in}}%
\pgfpathcurveto{\pgfqpoint{2.352992in}{1.489306in}}{\pgfqpoint{2.356264in}{1.497206in}}{\pgfqpoint{2.356264in}{1.505442in}}%
\pgfpathcurveto{\pgfqpoint{2.356264in}{1.513678in}}{\pgfqpoint{2.352992in}{1.521578in}}{\pgfqpoint{2.347168in}{1.527402in}}%
\pgfpathcurveto{\pgfqpoint{2.341344in}{1.533226in}}{\pgfqpoint{2.333444in}{1.536498in}}{\pgfqpoint{2.325207in}{1.536498in}}%
\pgfpathcurveto{\pgfqpoint{2.316971in}{1.536498in}}{\pgfqpoint{2.309071in}{1.533226in}}{\pgfqpoint{2.303247in}{1.527402in}}%
\pgfpathcurveto{\pgfqpoint{2.297423in}{1.521578in}}{\pgfqpoint{2.294151in}{1.513678in}}{\pgfqpoint{2.294151in}{1.505442in}}%
\pgfpathcurveto{\pgfqpoint{2.294151in}{1.497206in}}{\pgfqpoint{2.297423in}{1.489306in}}{\pgfqpoint{2.303247in}{1.483482in}}%
\pgfpathcurveto{\pgfqpoint{2.309071in}{1.477658in}}{\pgfqpoint{2.316971in}{1.474385in}}{\pgfqpoint{2.325207in}{1.474385in}}%
\pgfpathclose%
\pgfusepath{stroke,fill}%
\end{pgfscope}%
\begin{pgfscope}%
\pgfpathrectangle{\pgfqpoint{0.100000in}{0.212622in}}{\pgfqpoint{3.696000in}{3.696000in}}%
\pgfusepath{clip}%
\pgfsetbuttcap%
\pgfsetroundjoin%
\definecolor{currentfill}{rgb}{0.121569,0.466667,0.705882}%
\pgfsetfillcolor{currentfill}%
\pgfsetfillopacity{0.890649}%
\pgfsetlinewidth{1.003750pt}%
\definecolor{currentstroke}{rgb}{0.121569,0.466667,0.705882}%
\pgfsetstrokecolor{currentstroke}%
\pgfsetstrokeopacity{0.890649}%
\pgfsetdash{}{0pt}%
\pgfpathmoveto{\pgfqpoint{2.325826in}{1.474141in}}%
\pgfpathcurveto{\pgfqpoint{2.334062in}{1.474141in}}{\pgfqpoint{2.341962in}{1.477413in}}{\pgfqpoint{2.347786in}{1.483237in}}%
\pgfpathcurveto{\pgfqpoint{2.353610in}{1.489061in}}{\pgfqpoint{2.356882in}{1.496961in}}{\pgfqpoint{2.356882in}{1.505197in}}%
\pgfpathcurveto{\pgfqpoint{2.356882in}{1.513434in}}{\pgfqpoint{2.353610in}{1.521334in}}{\pgfqpoint{2.347786in}{1.527158in}}%
\pgfpathcurveto{\pgfqpoint{2.341962in}{1.532982in}}{\pgfqpoint{2.334062in}{1.536254in}}{\pgfqpoint{2.325826in}{1.536254in}}%
\pgfpathcurveto{\pgfqpoint{2.317589in}{1.536254in}}{\pgfqpoint{2.309689in}{1.532982in}}{\pgfqpoint{2.303865in}{1.527158in}}%
\pgfpathcurveto{\pgfqpoint{2.298041in}{1.521334in}}{\pgfqpoint{2.294769in}{1.513434in}}{\pgfqpoint{2.294769in}{1.505197in}}%
\pgfpathcurveto{\pgfqpoint{2.294769in}{1.496961in}}{\pgfqpoint{2.298041in}{1.489061in}}{\pgfqpoint{2.303865in}{1.483237in}}%
\pgfpathcurveto{\pgfqpoint{2.309689in}{1.477413in}}{\pgfqpoint{2.317589in}{1.474141in}}{\pgfqpoint{2.325826in}{1.474141in}}%
\pgfpathclose%
\pgfusepath{stroke,fill}%
\end{pgfscope}%
\begin{pgfscope}%
\pgfpathrectangle{\pgfqpoint{0.100000in}{0.212622in}}{\pgfqpoint{3.696000in}{3.696000in}}%
\pgfusepath{clip}%
\pgfsetbuttcap%
\pgfsetroundjoin%
\definecolor{currentfill}{rgb}{0.121569,0.466667,0.705882}%
\pgfsetfillcolor{currentfill}%
\pgfsetfillopacity{0.891246}%
\pgfsetlinewidth{1.003750pt}%
\definecolor{currentstroke}{rgb}{0.121569,0.466667,0.705882}%
\pgfsetstrokecolor{currentstroke}%
\pgfsetstrokeopacity{0.891246}%
\pgfsetdash{}{0pt}%
\pgfpathmoveto{\pgfqpoint{2.326154in}{1.474060in}}%
\pgfpathcurveto{\pgfqpoint{2.334391in}{1.474060in}}{\pgfqpoint{2.342291in}{1.477333in}}{\pgfqpoint{2.348115in}{1.483156in}}%
\pgfpathcurveto{\pgfqpoint{2.353939in}{1.488980in}}{\pgfqpoint{2.357211in}{1.496880in}}{\pgfqpoint{2.357211in}{1.505117in}}%
\pgfpathcurveto{\pgfqpoint{2.357211in}{1.513353in}}{\pgfqpoint{2.353939in}{1.521253in}}{\pgfqpoint{2.348115in}{1.527077in}}%
\pgfpathcurveto{\pgfqpoint{2.342291in}{1.532901in}}{\pgfqpoint{2.334391in}{1.536173in}}{\pgfqpoint{2.326154in}{1.536173in}}%
\pgfpathcurveto{\pgfqpoint{2.317918in}{1.536173in}}{\pgfqpoint{2.310018in}{1.532901in}}{\pgfqpoint{2.304194in}{1.527077in}}%
\pgfpathcurveto{\pgfqpoint{2.298370in}{1.521253in}}{\pgfqpoint{2.295098in}{1.513353in}}{\pgfqpoint{2.295098in}{1.505117in}}%
\pgfpathcurveto{\pgfqpoint{2.295098in}{1.496880in}}{\pgfqpoint{2.298370in}{1.488980in}}{\pgfqpoint{2.304194in}{1.483156in}}%
\pgfpathcurveto{\pgfqpoint{2.310018in}{1.477333in}}{\pgfqpoint{2.317918in}{1.474060in}}{\pgfqpoint{2.326154in}{1.474060in}}%
\pgfpathclose%
\pgfusepath{stroke,fill}%
\end{pgfscope}%
\begin{pgfscope}%
\pgfpathrectangle{\pgfqpoint{0.100000in}{0.212622in}}{\pgfqpoint{3.696000in}{3.696000in}}%
\pgfusepath{clip}%
\pgfsetbuttcap%
\pgfsetroundjoin%
\definecolor{currentfill}{rgb}{0.121569,0.466667,0.705882}%
\pgfsetfillcolor{currentfill}%
\pgfsetfillopacity{0.891297}%
\pgfsetlinewidth{1.003750pt}%
\definecolor{currentstroke}{rgb}{0.121569,0.466667,0.705882}%
\pgfsetstrokecolor{currentstroke}%
\pgfsetstrokeopacity{0.891297}%
\pgfsetdash{}{0pt}%
\pgfpathmoveto{\pgfqpoint{1.324012in}{2.065687in}}%
\pgfpathcurveto{\pgfqpoint{1.332248in}{2.065687in}}{\pgfqpoint{1.340148in}{2.068959in}}{\pgfqpoint{1.345972in}{2.074783in}}%
\pgfpathcurveto{\pgfqpoint{1.351796in}{2.080607in}}{\pgfqpoint{1.355068in}{2.088507in}}{\pgfqpoint{1.355068in}{2.096743in}}%
\pgfpathcurveto{\pgfqpoint{1.355068in}{2.104980in}}{\pgfqpoint{1.351796in}{2.112880in}}{\pgfqpoint{1.345972in}{2.118704in}}%
\pgfpathcurveto{\pgfqpoint{1.340148in}{2.124528in}}{\pgfqpoint{1.332248in}{2.127800in}}{\pgfqpoint{1.324012in}{2.127800in}}%
\pgfpathcurveto{\pgfqpoint{1.315775in}{2.127800in}}{\pgfqpoint{1.307875in}{2.124528in}}{\pgfqpoint{1.302051in}{2.118704in}}%
\pgfpathcurveto{\pgfqpoint{1.296227in}{2.112880in}}{\pgfqpoint{1.292955in}{2.104980in}}{\pgfqpoint{1.292955in}{2.096743in}}%
\pgfpathcurveto{\pgfqpoint{1.292955in}{2.088507in}}{\pgfqpoint{1.296227in}{2.080607in}}{\pgfqpoint{1.302051in}{2.074783in}}%
\pgfpathcurveto{\pgfqpoint{1.307875in}{2.068959in}}{\pgfqpoint{1.315775in}{2.065687in}}{\pgfqpoint{1.324012in}{2.065687in}}%
\pgfpathclose%
\pgfusepath{stroke,fill}%
\end{pgfscope}%
\begin{pgfscope}%
\pgfpathrectangle{\pgfqpoint{0.100000in}{0.212622in}}{\pgfqpoint{3.696000in}{3.696000in}}%
\pgfusepath{clip}%
\pgfsetbuttcap%
\pgfsetroundjoin%
\definecolor{currentfill}{rgb}{0.121569,0.466667,0.705882}%
\pgfsetfillcolor{currentfill}%
\pgfsetfillopacity{0.891859}%
\pgfsetlinewidth{1.003750pt}%
\definecolor{currentstroke}{rgb}{0.121569,0.466667,0.705882}%
\pgfsetstrokecolor{currentstroke}%
\pgfsetstrokeopacity{0.891859}%
\pgfsetdash{}{0pt}%
\pgfpathmoveto{\pgfqpoint{2.326766in}{1.472930in}}%
\pgfpathcurveto{\pgfqpoint{2.335002in}{1.472930in}}{\pgfqpoint{2.342902in}{1.476202in}}{\pgfqpoint{2.348726in}{1.482026in}}%
\pgfpathcurveto{\pgfqpoint{2.354550in}{1.487850in}}{\pgfqpoint{2.357822in}{1.495750in}}{\pgfqpoint{2.357822in}{1.503986in}}%
\pgfpathcurveto{\pgfqpoint{2.357822in}{1.512222in}}{\pgfqpoint{2.354550in}{1.520122in}}{\pgfqpoint{2.348726in}{1.525946in}}%
\pgfpathcurveto{\pgfqpoint{2.342902in}{1.531770in}}{\pgfqpoint{2.335002in}{1.535043in}}{\pgfqpoint{2.326766in}{1.535043in}}%
\pgfpathcurveto{\pgfqpoint{2.318529in}{1.535043in}}{\pgfqpoint{2.310629in}{1.531770in}}{\pgfqpoint{2.304805in}{1.525946in}}%
\pgfpathcurveto{\pgfqpoint{2.298981in}{1.520122in}}{\pgfqpoint{2.295709in}{1.512222in}}{\pgfqpoint{2.295709in}{1.503986in}}%
\pgfpathcurveto{\pgfqpoint{2.295709in}{1.495750in}}{\pgfqpoint{2.298981in}{1.487850in}}{\pgfqpoint{2.304805in}{1.482026in}}%
\pgfpathcurveto{\pgfqpoint{2.310629in}{1.476202in}}{\pgfqpoint{2.318529in}{1.472930in}}{\pgfqpoint{2.326766in}{1.472930in}}%
\pgfpathclose%
\pgfusepath{stroke,fill}%
\end{pgfscope}%
\begin{pgfscope}%
\pgfpathrectangle{\pgfqpoint{0.100000in}{0.212622in}}{\pgfqpoint{3.696000in}{3.696000in}}%
\pgfusepath{clip}%
\pgfsetbuttcap%
\pgfsetroundjoin%
\definecolor{currentfill}{rgb}{0.121569,0.466667,0.705882}%
\pgfsetfillcolor{currentfill}%
\pgfsetfillopacity{0.892682}%
\pgfsetlinewidth{1.003750pt}%
\definecolor{currentstroke}{rgb}{0.121569,0.466667,0.705882}%
\pgfsetstrokecolor{currentstroke}%
\pgfsetstrokeopacity{0.892682}%
\pgfsetdash{}{0pt}%
\pgfpathmoveto{\pgfqpoint{2.327892in}{1.469829in}}%
\pgfpathcurveto{\pgfqpoint{2.336128in}{1.469829in}}{\pgfqpoint{2.344028in}{1.473101in}}{\pgfqpoint{2.349852in}{1.478925in}}%
\pgfpathcurveto{\pgfqpoint{2.355676in}{1.484749in}}{\pgfqpoint{2.358948in}{1.492649in}}{\pgfqpoint{2.358948in}{1.500885in}}%
\pgfpathcurveto{\pgfqpoint{2.358948in}{1.509121in}}{\pgfqpoint{2.355676in}{1.517021in}}{\pgfqpoint{2.349852in}{1.522845in}}%
\pgfpathcurveto{\pgfqpoint{2.344028in}{1.528669in}}{\pgfqpoint{2.336128in}{1.531942in}}{\pgfqpoint{2.327892in}{1.531942in}}%
\pgfpathcurveto{\pgfqpoint{2.319655in}{1.531942in}}{\pgfqpoint{2.311755in}{1.528669in}}{\pgfqpoint{2.305931in}{1.522845in}}%
\pgfpathcurveto{\pgfqpoint{2.300107in}{1.517021in}}{\pgfqpoint{2.296835in}{1.509121in}}{\pgfqpoint{2.296835in}{1.500885in}}%
\pgfpathcurveto{\pgfqpoint{2.296835in}{1.492649in}}{\pgfqpoint{2.300107in}{1.484749in}}{\pgfqpoint{2.305931in}{1.478925in}}%
\pgfpathcurveto{\pgfqpoint{2.311755in}{1.473101in}}{\pgfqpoint{2.319655in}{1.469829in}}{\pgfqpoint{2.327892in}{1.469829in}}%
\pgfpathclose%
\pgfusepath{stroke,fill}%
\end{pgfscope}%
\begin{pgfscope}%
\pgfpathrectangle{\pgfqpoint{0.100000in}{0.212622in}}{\pgfqpoint{3.696000in}{3.696000in}}%
\pgfusepath{clip}%
\pgfsetbuttcap%
\pgfsetroundjoin%
\definecolor{currentfill}{rgb}{0.121569,0.466667,0.705882}%
\pgfsetfillcolor{currentfill}%
\pgfsetfillopacity{0.894739}%
\pgfsetlinewidth{1.003750pt}%
\definecolor{currentstroke}{rgb}{0.121569,0.466667,0.705882}%
\pgfsetstrokecolor{currentstroke}%
\pgfsetstrokeopacity{0.894739}%
\pgfsetdash{}{0pt}%
\pgfpathmoveto{\pgfqpoint{2.329384in}{1.468248in}}%
\pgfpathcurveto{\pgfqpoint{2.337620in}{1.468248in}}{\pgfqpoint{2.345520in}{1.471520in}}{\pgfqpoint{2.351344in}{1.477344in}}%
\pgfpathcurveto{\pgfqpoint{2.357168in}{1.483168in}}{\pgfqpoint{2.360441in}{1.491068in}}{\pgfqpoint{2.360441in}{1.499304in}}%
\pgfpathcurveto{\pgfqpoint{2.360441in}{1.507540in}}{\pgfqpoint{2.357168in}{1.515440in}}{\pgfqpoint{2.351344in}{1.521264in}}%
\pgfpathcurveto{\pgfqpoint{2.345520in}{1.527088in}}{\pgfqpoint{2.337620in}{1.530361in}}{\pgfqpoint{2.329384in}{1.530361in}}%
\pgfpathcurveto{\pgfqpoint{2.321148in}{1.530361in}}{\pgfqpoint{2.313248in}{1.527088in}}{\pgfqpoint{2.307424in}{1.521264in}}%
\pgfpathcurveto{\pgfqpoint{2.301600in}{1.515440in}}{\pgfqpoint{2.298328in}{1.507540in}}{\pgfqpoint{2.298328in}{1.499304in}}%
\pgfpathcurveto{\pgfqpoint{2.298328in}{1.491068in}}{\pgfqpoint{2.301600in}{1.483168in}}{\pgfqpoint{2.307424in}{1.477344in}}%
\pgfpathcurveto{\pgfqpoint{2.313248in}{1.471520in}}{\pgfqpoint{2.321148in}{1.468248in}}{\pgfqpoint{2.329384in}{1.468248in}}%
\pgfpathclose%
\pgfusepath{stroke,fill}%
\end{pgfscope}%
\begin{pgfscope}%
\pgfpathrectangle{\pgfqpoint{0.100000in}{0.212622in}}{\pgfqpoint{3.696000in}{3.696000in}}%
\pgfusepath{clip}%
\pgfsetbuttcap%
\pgfsetroundjoin%
\definecolor{currentfill}{rgb}{0.121569,0.466667,0.705882}%
\pgfsetfillcolor{currentfill}%
\pgfsetfillopacity{0.896241}%
\pgfsetlinewidth{1.003750pt}%
\definecolor{currentstroke}{rgb}{0.121569,0.466667,0.705882}%
\pgfsetstrokecolor{currentstroke}%
\pgfsetstrokeopacity{0.896241}%
\pgfsetdash{}{0pt}%
\pgfpathmoveto{\pgfqpoint{1.366645in}{2.030412in}}%
\pgfpathcurveto{\pgfqpoint{1.374881in}{2.030412in}}{\pgfqpoint{1.382781in}{2.033684in}}{\pgfqpoint{1.388605in}{2.039508in}}%
\pgfpathcurveto{\pgfqpoint{1.394429in}{2.045332in}}{\pgfqpoint{1.397701in}{2.053232in}}{\pgfqpoint{1.397701in}{2.061469in}}%
\pgfpathcurveto{\pgfqpoint{1.397701in}{2.069705in}}{\pgfqpoint{1.394429in}{2.077605in}}{\pgfqpoint{1.388605in}{2.083429in}}%
\pgfpathcurveto{\pgfqpoint{1.382781in}{2.089253in}}{\pgfqpoint{1.374881in}{2.092525in}}{\pgfqpoint{1.366645in}{2.092525in}}%
\pgfpathcurveto{\pgfqpoint{1.358408in}{2.092525in}}{\pgfqpoint{1.350508in}{2.089253in}}{\pgfqpoint{1.344684in}{2.083429in}}%
\pgfpathcurveto{\pgfqpoint{1.338861in}{2.077605in}}{\pgfqpoint{1.335588in}{2.069705in}}{\pgfqpoint{1.335588in}{2.061469in}}%
\pgfpathcurveto{\pgfqpoint{1.335588in}{2.053232in}}{\pgfqpoint{1.338861in}{2.045332in}}{\pgfqpoint{1.344684in}{2.039508in}}%
\pgfpathcurveto{\pgfqpoint{1.350508in}{2.033684in}}{\pgfqpoint{1.358408in}{2.030412in}}{\pgfqpoint{1.366645in}{2.030412in}}%
\pgfpathclose%
\pgfusepath{stroke,fill}%
\end{pgfscope}%
\begin{pgfscope}%
\pgfpathrectangle{\pgfqpoint{0.100000in}{0.212622in}}{\pgfqpoint{3.696000in}{3.696000in}}%
\pgfusepath{clip}%
\pgfsetbuttcap%
\pgfsetroundjoin%
\definecolor{currentfill}{rgb}{0.121569,0.466667,0.705882}%
\pgfsetfillcolor{currentfill}%
\pgfsetfillopacity{0.896931}%
\pgfsetlinewidth{1.003750pt}%
\definecolor{currentstroke}{rgb}{0.121569,0.466667,0.705882}%
\pgfsetstrokecolor{currentstroke}%
\pgfsetstrokeopacity{0.896931}%
\pgfsetdash{}{0pt}%
\pgfpathmoveto{\pgfqpoint{2.331348in}{1.466630in}}%
\pgfpathcurveto{\pgfqpoint{2.339584in}{1.466630in}}{\pgfqpoint{2.347484in}{1.469902in}}{\pgfqpoint{2.353308in}{1.475726in}}%
\pgfpathcurveto{\pgfqpoint{2.359132in}{1.481550in}}{\pgfqpoint{2.362404in}{1.489450in}}{\pgfqpoint{2.362404in}{1.497686in}}%
\pgfpathcurveto{\pgfqpoint{2.362404in}{1.505922in}}{\pgfqpoint{2.359132in}{1.513823in}}{\pgfqpoint{2.353308in}{1.519646in}}%
\pgfpathcurveto{\pgfqpoint{2.347484in}{1.525470in}}{\pgfqpoint{2.339584in}{1.528743in}}{\pgfqpoint{2.331348in}{1.528743in}}%
\pgfpathcurveto{\pgfqpoint{2.323112in}{1.528743in}}{\pgfqpoint{2.315211in}{1.525470in}}{\pgfqpoint{2.309388in}{1.519646in}}%
\pgfpathcurveto{\pgfqpoint{2.303564in}{1.513823in}}{\pgfqpoint{2.300291in}{1.505922in}}{\pgfqpoint{2.300291in}{1.497686in}}%
\pgfpathcurveto{\pgfqpoint{2.300291in}{1.489450in}}{\pgfqpoint{2.303564in}{1.481550in}}{\pgfqpoint{2.309388in}{1.475726in}}%
\pgfpathcurveto{\pgfqpoint{2.315211in}{1.469902in}}{\pgfqpoint{2.323112in}{1.466630in}}{\pgfqpoint{2.331348in}{1.466630in}}%
\pgfpathclose%
\pgfusepath{stroke,fill}%
\end{pgfscope}%
\begin{pgfscope}%
\pgfpathrectangle{\pgfqpoint{0.100000in}{0.212622in}}{\pgfqpoint{3.696000in}{3.696000in}}%
\pgfusepath{clip}%
\pgfsetbuttcap%
\pgfsetroundjoin%
\definecolor{currentfill}{rgb}{0.121569,0.466667,0.705882}%
\pgfsetfillcolor{currentfill}%
\pgfsetfillopacity{0.899219}%
\pgfsetlinewidth{1.003750pt}%
\definecolor{currentstroke}{rgb}{0.121569,0.466667,0.705882}%
\pgfsetstrokecolor{currentstroke}%
\pgfsetstrokeopacity{0.899219}%
\pgfsetdash{}{0pt}%
\pgfpathmoveto{\pgfqpoint{2.333613in}{1.463275in}}%
\pgfpathcurveto{\pgfqpoint{2.341850in}{1.463275in}}{\pgfqpoint{2.349750in}{1.466547in}}{\pgfqpoint{2.355574in}{1.472371in}}%
\pgfpathcurveto{\pgfqpoint{2.361397in}{1.478195in}}{\pgfqpoint{2.364670in}{1.486095in}}{\pgfqpoint{2.364670in}{1.494331in}}%
\pgfpathcurveto{\pgfqpoint{2.364670in}{1.502568in}}{\pgfqpoint{2.361397in}{1.510468in}}{\pgfqpoint{2.355574in}{1.516292in}}%
\pgfpathcurveto{\pgfqpoint{2.349750in}{1.522116in}}{\pgfqpoint{2.341850in}{1.525388in}}{\pgfqpoint{2.333613in}{1.525388in}}%
\pgfpathcurveto{\pgfqpoint{2.325377in}{1.525388in}}{\pgfqpoint{2.317477in}{1.522116in}}{\pgfqpoint{2.311653in}{1.516292in}}%
\pgfpathcurveto{\pgfqpoint{2.305829in}{1.510468in}}{\pgfqpoint{2.302557in}{1.502568in}}{\pgfqpoint{2.302557in}{1.494331in}}%
\pgfpathcurveto{\pgfqpoint{2.302557in}{1.486095in}}{\pgfqpoint{2.305829in}{1.478195in}}{\pgfqpoint{2.311653in}{1.472371in}}%
\pgfpathcurveto{\pgfqpoint{2.317477in}{1.466547in}}{\pgfqpoint{2.325377in}{1.463275in}}{\pgfqpoint{2.333613in}{1.463275in}}%
\pgfpathclose%
\pgfusepath{stroke,fill}%
\end{pgfscope}%
\begin{pgfscope}%
\pgfpathrectangle{\pgfqpoint{0.100000in}{0.212622in}}{\pgfqpoint{3.696000in}{3.696000in}}%
\pgfusepath{clip}%
\pgfsetbuttcap%
\pgfsetroundjoin%
\definecolor{currentfill}{rgb}{0.121569,0.466667,0.705882}%
\pgfsetfillcolor{currentfill}%
\pgfsetfillopacity{0.900856}%
\pgfsetlinewidth{1.003750pt}%
\definecolor{currentstroke}{rgb}{0.121569,0.466667,0.705882}%
\pgfsetstrokecolor{currentstroke}%
\pgfsetstrokeopacity{0.900856}%
\pgfsetdash{}{0pt}%
\pgfpathmoveto{\pgfqpoint{1.408997in}{2.007079in}}%
\pgfpathcurveto{\pgfqpoint{1.417234in}{2.007079in}}{\pgfqpoint{1.425134in}{2.010351in}}{\pgfqpoint{1.430958in}{2.016175in}}%
\pgfpathcurveto{\pgfqpoint{1.436781in}{2.021999in}}{\pgfqpoint{1.440054in}{2.029899in}}{\pgfqpoint{1.440054in}{2.038135in}}%
\pgfpathcurveto{\pgfqpoint{1.440054in}{2.046372in}}{\pgfqpoint{1.436781in}{2.054272in}}{\pgfqpoint{1.430958in}{2.060096in}}%
\pgfpathcurveto{\pgfqpoint{1.425134in}{2.065920in}}{\pgfqpoint{1.417234in}{2.069192in}}{\pgfqpoint{1.408997in}{2.069192in}}%
\pgfpathcurveto{\pgfqpoint{1.400761in}{2.069192in}}{\pgfqpoint{1.392861in}{2.065920in}}{\pgfqpoint{1.387037in}{2.060096in}}%
\pgfpathcurveto{\pgfqpoint{1.381213in}{2.054272in}}{\pgfqpoint{1.377941in}{2.046372in}}{\pgfqpoint{1.377941in}{2.038135in}}%
\pgfpathcurveto{\pgfqpoint{1.377941in}{2.029899in}}{\pgfqpoint{1.381213in}{2.021999in}}{\pgfqpoint{1.387037in}{2.016175in}}%
\pgfpathcurveto{\pgfqpoint{1.392861in}{2.010351in}}{\pgfqpoint{1.400761in}{2.007079in}}{\pgfqpoint{1.408997in}{2.007079in}}%
\pgfpathclose%
\pgfusepath{stroke,fill}%
\end{pgfscope}%
\begin{pgfscope}%
\pgfpathrectangle{\pgfqpoint{0.100000in}{0.212622in}}{\pgfqpoint{3.696000in}{3.696000in}}%
\pgfusepath{clip}%
\pgfsetbuttcap%
\pgfsetroundjoin%
\definecolor{currentfill}{rgb}{0.121569,0.466667,0.705882}%
\pgfsetfillcolor{currentfill}%
\pgfsetfillopacity{0.901602}%
\pgfsetlinewidth{1.003750pt}%
\definecolor{currentstroke}{rgb}{0.121569,0.466667,0.705882}%
\pgfsetstrokecolor{currentstroke}%
\pgfsetstrokeopacity{0.901602}%
\pgfsetdash{}{0pt}%
\pgfpathmoveto{\pgfqpoint{2.335638in}{1.458560in}}%
\pgfpathcurveto{\pgfqpoint{2.343874in}{1.458560in}}{\pgfqpoint{2.351774in}{1.461833in}}{\pgfqpoint{2.357598in}{1.467656in}}%
\pgfpathcurveto{\pgfqpoint{2.363422in}{1.473480in}}{\pgfqpoint{2.366694in}{1.481380in}}{\pgfqpoint{2.366694in}{1.489617in}}%
\pgfpathcurveto{\pgfqpoint{2.366694in}{1.497853in}}{\pgfqpoint{2.363422in}{1.505753in}}{\pgfqpoint{2.357598in}{1.511577in}}%
\pgfpathcurveto{\pgfqpoint{2.351774in}{1.517401in}}{\pgfqpoint{2.343874in}{1.520673in}}{\pgfqpoint{2.335638in}{1.520673in}}%
\pgfpathcurveto{\pgfqpoint{2.327401in}{1.520673in}}{\pgfqpoint{2.319501in}{1.517401in}}{\pgfqpoint{2.313677in}{1.511577in}}%
\pgfpathcurveto{\pgfqpoint{2.307853in}{1.505753in}}{\pgfqpoint{2.304581in}{1.497853in}}{\pgfqpoint{2.304581in}{1.489617in}}%
\pgfpathcurveto{\pgfqpoint{2.304581in}{1.481380in}}{\pgfqpoint{2.307853in}{1.473480in}}{\pgfqpoint{2.313677in}{1.467656in}}%
\pgfpathcurveto{\pgfqpoint{2.319501in}{1.461833in}}{\pgfqpoint{2.327401in}{1.458560in}}{\pgfqpoint{2.335638in}{1.458560in}}%
\pgfpathclose%
\pgfusepath{stroke,fill}%
\end{pgfscope}%
\begin{pgfscope}%
\pgfpathrectangle{\pgfqpoint{0.100000in}{0.212622in}}{\pgfqpoint{3.696000in}{3.696000in}}%
\pgfusepath{clip}%
\pgfsetbuttcap%
\pgfsetroundjoin%
\definecolor{currentfill}{rgb}{0.121569,0.466667,0.705882}%
\pgfsetfillcolor{currentfill}%
\pgfsetfillopacity{0.902995}%
\pgfsetlinewidth{1.003750pt}%
\definecolor{currentstroke}{rgb}{0.121569,0.466667,0.705882}%
\pgfsetstrokecolor{currentstroke}%
\pgfsetstrokeopacity{0.902995}%
\pgfsetdash{}{0pt}%
\pgfpathmoveto{\pgfqpoint{2.336697in}{1.456427in}}%
\pgfpathcurveto{\pgfqpoint{2.344933in}{1.456427in}}{\pgfqpoint{2.352833in}{1.459699in}}{\pgfqpoint{2.358657in}{1.465523in}}%
\pgfpathcurveto{\pgfqpoint{2.364481in}{1.471347in}}{\pgfqpoint{2.367753in}{1.479247in}}{\pgfqpoint{2.367753in}{1.487483in}}%
\pgfpathcurveto{\pgfqpoint{2.367753in}{1.495719in}}{\pgfqpoint{2.364481in}{1.503619in}}{\pgfqpoint{2.358657in}{1.509443in}}%
\pgfpathcurveto{\pgfqpoint{2.352833in}{1.515267in}}{\pgfqpoint{2.344933in}{1.518540in}}{\pgfqpoint{2.336697in}{1.518540in}}%
\pgfpathcurveto{\pgfqpoint{2.328461in}{1.518540in}}{\pgfqpoint{2.320560in}{1.515267in}}{\pgfqpoint{2.314737in}{1.509443in}}%
\pgfpathcurveto{\pgfqpoint{2.308913in}{1.503619in}}{\pgfqpoint{2.305640in}{1.495719in}}{\pgfqpoint{2.305640in}{1.487483in}}%
\pgfpathcurveto{\pgfqpoint{2.305640in}{1.479247in}}{\pgfqpoint{2.308913in}{1.471347in}}{\pgfqpoint{2.314737in}{1.465523in}}%
\pgfpathcurveto{\pgfqpoint{2.320560in}{1.459699in}}{\pgfqpoint{2.328461in}{1.456427in}}{\pgfqpoint{2.336697in}{1.456427in}}%
\pgfpathclose%
\pgfusepath{stroke,fill}%
\end{pgfscope}%
\begin{pgfscope}%
\pgfpathrectangle{\pgfqpoint{0.100000in}{0.212622in}}{\pgfqpoint{3.696000in}{3.696000in}}%
\pgfusepath{clip}%
\pgfsetbuttcap%
\pgfsetroundjoin%
\definecolor{currentfill}{rgb}{0.121569,0.466667,0.705882}%
\pgfsetfillcolor{currentfill}%
\pgfsetfillopacity{0.903568}%
\pgfsetlinewidth{1.003750pt}%
\definecolor{currentstroke}{rgb}{0.121569,0.466667,0.705882}%
\pgfsetstrokecolor{currentstroke}%
\pgfsetstrokeopacity{0.903568}%
\pgfsetdash{}{0pt}%
\pgfpathmoveto{\pgfqpoint{1.450582in}{1.971127in}}%
\pgfpathcurveto{\pgfqpoint{1.458819in}{1.971127in}}{\pgfqpoint{1.466719in}{1.974399in}}{\pgfqpoint{1.472543in}{1.980223in}}%
\pgfpathcurveto{\pgfqpoint{1.478367in}{1.986047in}}{\pgfqpoint{1.481639in}{1.993947in}}{\pgfqpoint{1.481639in}{2.002183in}}%
\pgfpathcurveto{\pgfqpoint{1.481639in}{2.010420in}}{\pgfqpoint{1.478367in}{2.018320in}}{\pgfqpoint{1.472543in}{2.024144in}}%
\pgfpathcurveto{\pgfqpoint{1.466719in}{2.029968in}}{\pgfqpoint{1.458819in}{2.033240in}}{\pgfqpoint{1.450582in}{2.033240in}}%
\pgfpathcurveto{\pgfqpoint{1.442346in}{2.033240in}}{\pgfqpoint{1.434446in}{2.029968in}}{\pgfqpoint{1.428622in}{2.024144in}}%
\pgfpathcurveto{\pgfqpoint{1.422798in}{2.018320in}}{\pgfqpoint{1.419526in}{2.010420in}}{\pgfqpoint{1.419526in}{2.002183in}}%
\pgfpathcurveto{\pgfqpoint{1.419526in}{1.993947in}}{\pgfqpoint{1.422798in}{1.986047in}}{\pgfqpoint{1.428622in}{1.980223in}}%
\pgfpathcurveto{\pgfqpoint{1.434446in}{1.974399in}}{\pgfqpoint{1.442346in}{1.971127in}}{\pgfqpoint{1.450582in}{1.971127in}}%
\pgfpathclose%
\pgfusepath{stroke,fill}%
\end{pgfscope}%
\begin{pgfscope}%
\pgfpathrectangle{\pgfqpoint{0.100000in}{0.212622in}}{\pgfqpoint{3.696000in}{3.696000in}}%
\pgfusepath{clip}%
\pgfsetbuttcap%
\pgfsetroundjoin%
\definecolor{currentfill}{rgb}{0.121569,0.466667,0.705882}%
\pgfsetfillcolor{currentfill}%
\pgfsetfillopacity{0.904803}%
\pgfsetlinewidth{1.003750pt}%
\definecolor{currentstroke}{rgb}{0.121569,0.466667,0.705882}%
\pgfsetstrokecolor{currentstroke}%
\pgfsetstrokeopacity{0.904803}%
\pgfsetdash{}{0pt}%
\pgfpathmoveto{\pgfqpoint{2.337999in}{1.455048in}}%
\pgfpathcurveto{\pgfqpoint{2.346235in}{1.455048in}}{\pgfqpoint{2.354135in}{1.458321in}}{\pgfqpoint{2.359959in}{1.464144in}}%
\pgfpathcurveto{\pgfqpoint{2.365783in}{1.469968in}}{\pgfqpoint{2.369056in}{1.477868in}}{\pgfqpoint{2.369056in}{1.486105in}}%
\pgfpathcurveto{\pgfqpoint{2.369056in}{1.494341in}}{\pgfqpoint{2.365783in}{1.502241in}}{\pgfqpoint{2.359959in}{1.508065in}}%
\pgfpathcurveto{\pgfqpoint{2.354135in}{1.513889in}}{\pgfqpoint{2.346235in}{1.517161in}}{\pgfqpoint{2.337999in}{1.517161in}}%
\pgfpathcurveto{\pgfqpoint{2.329763in}{1.517161in}}{\pgfqpoint{2.321863in}{1.513889in}}{\pgfqpoint{2.316039in}{1.508065in}}%
\pgfpathcurveto{\pgfqpoint{2.310215in}{1.502241in}}{\pgfqpoint{2.306943in}{1.494341in}}{\pgfqpoint{2.306943in}{1.486105in}}%
\pgfpathcurveto{\pgfqpoint{2.306943in}{1.477868in}}{\pgfqpoint{2.310215in}{1.469968in}}{\pgfqpoint{2.316039in}{1.464144in}}%
\pgfpathcurveto{\pgfqpoint{2.321863in}{1.458321in}}{\pgfqpoint{2.329763in}{1.455048in}}{\pgfqpoint{2.337999in}{1.455048in}}%
\pgfpathclose%
\pgfusepath{stroke,fill}%
\end{pgfscope}%
\begin{pgfscope}%
\pgfpathrectangle{\pgfqpoint{0.100000in}{0.212622in}}{\pgfqpoint{3.696000in}{3.696000in}}%
\pgfusepath{clip}%
\pgfsetbuttcap%
\pgfsetroundjoin%
\definecolor{currentfill}{rgb}{0.121569,0.466667,0.705882}%
\pgfsetfillcolor{currentfill}%
\pgfsetfillopacity{0.906786}%
\pgfsetlinewidth{1.003750pt}%
\definecolor{currentstroke}{rgb}{0.121569,0.466667,0.705882}%
\pgfsetstrokecolor{currentstroke}%
\pgfsetstrokeopacity{0.906786}%
\pgfsetdash{}{0pt}%
\pgfpathmoveto{\pgfqpoint{2.339673in}{1.453008in}}%
\pgfpathcurveto{\pgfqpoint{2.347909in}{1.453008in}}{\pgfqpoint{2.355809in}{1.456280in}}{\pgfqpoint{2.361633in}{1.462104in}}%
\pgfpathcurveto{\pgfqpoint{2.367457in}{1.467928in}}{\pgfqpoint{2.370729in}{1.475828in}}{\pgfqpoint{2.370729in}{1.484064in}}%
\pgfpathcurveto{\pgfqpoint{2.370729in}{1.492301in}}{\pgfqpoint{2.367457in}{1.500201in}}{\pgfqpoint{2.361633in}{1.506025in}}%
\pgfpathcurveto{\pgfqpoint{2.355809in}{1.511849in}}{\pgfqpoint{2.347909in}{1.515121in}}{\pgfqpoint{2.339673in}{1.515121in}}%
\pgfpathcurveto{\pgfqpoint{2.331436in}{1.515121in}}{\pgfqpoint{2.323536in}{1.511849in}}{\pgfqpoint{2.317712in}{1.506025in}}%
\pgfpathcurveto{\pgfqpoint{2.311888in}{1.500201in}}{\pgfqpoint{2.308616in}{1.492301in}}{\pgfqpoint{2.308616in}{1.484064in}}%
\pgfpathcurveto{\pgfqpoint{2.308616in}{1.475828in}}{\pgfqpoint{2.311888in}{1.467928in}}{\pgfqpoint{2.317712in}{1.462104in}}%
\pgfpathcurveto{\pgfqpoint{2.323536in}{1.456280in}}{\pgfqpoint{2.331436in}{1.453008in}}{\pgfqpoint{2.339673in}{1.453008in}}%
\pgfpathclose%
\pgfusepath{stroke,fill}%
\end{pgfscope}%
\begin{pgfscope}%
\pgfpathrectangle{\pgfqpoint{0.100000in}{0.212622in}}{\pgfqpoint{3.696000in}{3.696000in}}%
\pgfusepath{clip}%
\pgfsetbuttcap%
\pgfsetroundjoin%
\definecolor{currentfill}{rgb}{0.121569,0.466667,0.705882}%
\pgfsetfillcolor{currentfill}%
\pgfsetfillopacity{0.907232}%
\pgfsetlinewidth{1.003750pt}%
\definecolor{currentstroke}{rgb}{0.121569,0.466667,0.705882}%
\pgfsetstrokecolor{currentstroke}%
\pgfsetstrokeopacity{0.907232}%
\pgfsetdash{}{0pt}%
\pgfpathmoveto{\pgfqpoint{1.488898in}{1.947819in}}%
\pgfpathcurveto{\pgfqpoint{1.497135in}{1.947819in}}{\pgfqpoint{1.505035in}{1.951091in}}{\pgfqpoint{1.510859in}{1.956915in}}%
\pgfpathcurveto{\pgfqpoint{1.516682in}{1.962739in}}{\pgfqpoint{1.519955in}{1.970639in}}{\pgfqpoint{1.519955in}{1.978876in}}%
\pgfpathcurveto{\pgfqpoint{1.519955in}{1.987112in}}{\pgfqpoint{1.516682in}{1.995012in}}{\pgfqpoint{1.510859in}{2.000836in}}%
\pgfpathcurveto{\pgfqpoint{1.505035in}{2.006660in}}{\pgfqpoint{1.497135in}{2.009932in}}{\pgfqpoint{1.488898in}{2.009932in}}%
\pgfpathcurveto{\pgfqpoint{1.480662in}{2.009932in}}{\pgfqpoint{1.472762in}{2.006660in}}{\pgfqpoint{1.466938in}{2.000836in}}%
\pgfpathcurveto{\pgfqpoint{1.461114in}{1.995012in}}{\pgfqpoint{1.457842in}{1.987112in}}{\pgfqpoint{1.457842in}{1.978876in}}%
\pgfpathcurveto{\pgfqpoint{1.457842in}{1.970639in}}{\pgfqpoint{1.461114in}{1.962739in}}{\pgfqpoint{1.466938in}{1.956915in}}%
\pgfpathcurveto{\pgfqpoint{1.472762in}{1.951091in}}{\pgfqpoint{1.480662in}{1.947819in}}{\pgfqpoint{1.488898in}{1.947819in}}%
\pgfpathclose%
\pgfusepath{stroke,fill}%
\end{pgfscope}%
\begin{pgfscope}%
\pgfpathrectangle{\pgfqpoint{0.100000in}{0.212622in}}{\pgfqpoint{3.696000in}{3.696000in}}%
\pgfusepath{clip}%
\pgfsetbuttcap%
\pgfsetroundjoin%
\definecolor{currentfill}{rgb}{0.121569,0.466667,0.705882}%
\pgfsetfillcolor{currentfill}%
\pgfsetfillopacity{0.908793}%
\pgfsetlinewidth{1.003750pt}%
\definecolor{currentstroke}{rgb}{0.121569,0.466667,0.705882}%
\pgfsetstrokecolor{currentstroke}%
\pgfsetstrokeopacity{0.908793}%
\pgfsetdash{}{0pt}%
\pgfpathmoveto{\pgfqpoint{2.341883in}{1.449484in}}%
\pgfpathcurveto{\pgfqpoint{2.350119in}{1.449484in}}{\pgfqpoint{2.358019in}{1.452757in}}{\pgfqpoint{2.363843in}{1.458581in}}%
\pgfpathcurveto{\pgfqpoint{2.369667in}{1.464405in}}{\pgfqpoint{2.372939in}{1.472305in}}{\pgfqpoint{2.372939in}{1.480541in}}%
\pgfpathcurveto{\pgfqpoint{2.372939in}{1.488777in}}{\pgfqpoint{2.369667in}{1.496677in}}{\pgfqpoint{2.363843in}{1.502501in}}%
\pgfpathcurveto{\pgfqpoint{2.358019in}{1.508325in}}{\pgfqpoint{2.350119in}{1.511597in}}{\pgfqpoint{2.341883in}{1.511597in}}%
\pgfpathcurveto{\pgfqpoint{2.333646in}{1.511597in}}{\pgfqpoint{2.325746in}{1.508325in}}{\pgfqpoint{2.319922in}{1.502501in}}%
\pgfpathcurveto{\pgfqpoint{2.314098in}{1.496677in}}{\pgfqpoint{2.310826in}{1.488777in}}{\pgfqpoint{2.310826in}{1.480541in}}%
\pgfpathcurveto{\pgfqpoint{2.310826in}{1.472305in}}{\pgfqpoint{2.314098in}{1.464405in}}{\pgfqpoint{2.319922in}{1.458581in}}%
\pgfpathcurveto{\pgfqpoint{2.325746in}{1.452757in}}{\pgfqpoint{2.333646in}{1.449484in}}{\pgfqpoint{2.341883in}{1.449484in}}%
\pgfpathclose%
\pgfusepath{stroke,fill}%
\end{pgfscope}%
\begin{pgfscope}%
\pgfpathrectangle{\pgfqpoint{0.100000in}{0.212622in}}{\pgfqpoint{3.696000in}{3.696000in}}%
\pgfusepath{clip}%
\pgfsetbuttcap%
\pgfsetroundjoin%
\definecolor{currentfill}{rgb}{0.121569,0.466667,0.705882}%
\pgfsetfillcolor{currentfill}%
\pgfsetfillopacity{0.910767}%
\pgfsetlinewidth{1.003750pt}%
\definecolor{currentstroke}{rgb}{0.121569,0.466667,0.705882}%
\pgfsetstrokecolor{currentstroke}%
\pgfsetstrokeopacity{0.910767}%
\pgfsetdash{}{0pt}%
\pgfpathmoveto{\pgfqpoint{1.525720in}{1.922795in}}%
\pgfpathcurveto{\pgfqpoint{1.533956in}{1.922795in}}{\pgfqpoint{1.541856in}{1.926067in}}{\pgfqpoint{1.547680in}{1.931891in}}%
\pgfpathcurveto{\pgfqpoint{1.553504in}{1.937715in}}{\pgfqpoint{1.556777in}{1.945615in}}{\pgfqpoint{1.556777in}{1.953851in}}%
\pgfpathcurveto{\pgfqpoint{1.556777in}{1.962087in}}{\pgfqpoint{1.553504in}{1.969987in}}{\pgfqpoint{1.547680in}{1.975811in}}%
\pgfpathcurveto{\pgfqpoint{1.541856in}{1.981635in}}{\pgfqpoint{1.533956in}{1.984908in}}{\pgfqpoint{1.525720in}{1.984908in}}%
\pgfpathcurveto{\pgfqpoint{1.517484in}{1.984908in}}{\pgfqpoint{1.509584in}{1.981635in}}{\pgfqpoint{1.503760in}{1.975811in}}%
\pgfpathcurveto{\pgfqpoint{1.497936in}{1.969987in}}{\pgfqpoint{1.494664in}{1.962087in}}{\pgfqpoint{1.494664in}{1.953851in}}%
\pgfpathcurveto{\pgfqpoint{1.494664in}{1.945615in}}{\pgfqpoint{1.497936in}{1.937715in}}{\pgfqpoint{1.503760in}{1.931891in}}%
\pgfpathcurveto{\pgfqpoint{1.509584in}{1.926067in}}{\pgfqpoint{1.517484in}{1.922795in}}{\pgfqpoint{1.525720in}{1.922795in}}%
\pgfpathclose%
\pgfusepath{stroke,fill}%
\end{pgfscope}%
\begin{pgfscope}%
\pgfpathrectangle{\pgfqpoint{0.100000in}{0.212622in}}{\pgfqpoint{3.696000in}{3.696000in}}%
\pgfusepath{clip}%
\pgfsetbuttcap%
\pgfsetroundjoin%
\definecolor{currentfill}{rgb}{0.121569,0.466667,0.705882}%
\pgfsetfillcolor{currentfill}%
\pgfsetfillopacity{0.911001}%
\pgfsetlinewidth{1.003750pt}%
\definecolor{currentstroke}{rgb}{0.121569,0.466667,0.705882}%
\pgfsetstrokecolor{currentstroke}%
\pgfsetstrokeopacity{0.911001}%
\pgfsetdash{}{0pt}%
\pgfpathmoveto{\pgfqpoint{2.344316in}{1.442377in}}%
\pgfpathcurveto{\pgfqpoint{2.352553in}{1.442377in}}{\pgfqpoint{2.360453in}{1.445649in}}{\pgfqpoint{2.366277in}{1.451473in}}%
\pgfpathcurveto{\pgfqpoint{2.372101in}{1.457297in}}{\pgfqpoint{2.375373in}{1.465197in}}{\pgfqpoint{2.375373in}{1.473433in}}%
\pgfpathcurveto{\pgfqpoint{2.375373in}{1.481670in}}{\pgfqpoint{2.372101in}{1.489570in}}{\pgfqpoint{2.366277in}{1.495394in}}%
\pgfpathcurveto{\pgfqpoint{2.360453in}{1.501218in}}{\pgfqpoint{2.352553in}{1.504490in}}{\pgfqpoint{2.344316in}{1.504490in}}%
\pgfpathcurveto{\pgfqpoint{2.336080in}{1.504490in}}{\pgfqpoint{2.328180in}{1.501218in}}{\pgfqpoint{2.322356in}{1.495394in}}%
\pgfpathcurveto{\pgfqpoint{2.316532in}{1.489570in}}{\pgfqpoint{2.313260in}{1.481670in}}{\pgfqpoint{2.313260in}{1.473433in}}%
\pgfpathcurveto{\pgfqpoint{2.313260in}{1.465197in}}{\pgfqpoint{2.316532in}{1.457297in}}{\pgfqpoint{2.322356in}{1.451473in}}%
\pgfpathcurveto{\pgfqpoint{2.328180in}{1.445649in}}{\pgfqpoint{2.336080in}{1.442377in}}{\pgfqpoint{2.344316in}{1.442377in}}%
\pgfpathclose%
\pgfusepath{stroke,fill}%
\end{pgfscope}%
\begin{pgfscope}%
\pgfpathrectangle{\pgfqpoint{0.100000in}{0.212622in}}{\pgfqpoint{3.696000in}{3.696000in}}%
\pgfusepath{clip}%
\pgfsetbuttcap%
\pgfsetroundjoin%
\definecolor{currentfill}{rgb}{0.121569,0.466667,0.705882}%
\pgfsetfillcolor{currentfill}%
\pgfsetfillopacity{0.914031}%
\pgfsetlinewidth{1.003750pt}%
\definecolor{currentstroke}{rgb}{0.121569,0.466667,0.705882}%
\pgfsetstrokecolor{currentstroke}%
\pgfsetstrokeopacity{0.914031}%
\pgfsetdash{}{0pt}%
\pgfpathmoveto{\pgfqpoint{1.560248in}{1.898332in}}%
\pgfpathcurveto{\pgfqpoint{1.568484in}{1.898332in}}{\pgfqpoint{1.576384in}{1.901604in}}{\pgfqpoint{1.582208in}{1.907428in}}%
\pgfpathcurveto{\pgfqpoint{1.588032in}{1.913252in}}{\pgfqpoint{1.591304in}{1.921152in}}{\pgfqpoint{1.591304in}{1.929388in}}%
\pgfpathcurveto{\pgfqpoint{1.591304in}{1.937624in}}{\pgfqpoint{1.588032in}{1.945524in}}{\pgfqpoint{1.582208in}{1.951348in}}%
\pgfpathcurveto{\pgfqpoint{1.576384in}{1.957172in}}{\pgfqpoint{1.568484in}{1.960445in}}{\pgfqpoint{1.560248in}{1.960445in}}%
\pgfpathcurveto{\pgfqpoint{1.552012in}{1.960445in}}{\pgfqpoint{1.544111in}{1.957172in}}{\pgfqpoint{1.538288in}{1.951348in}}%
\pgfpathcurveto{\pgfqpoint{1.532464in}{1.945524in}}{\pgfqpoint{1.529191in}{1.937624in}}{\pgfqpoint{1.529191in}{1.929388in}}%
\pgfpathcurveto{\pgfqpoint{1.529191in}{1.921152in}}{\pgfqpoint{1.532464in}{1.913252in}}{\pgfqpoint{1.538288in}{1.907428in}}%
\pgfpathcurveto{\pgfqpoint{1.544111in}{1.901604in}}{\pgfqpoint{1.552012in}{1.898332in}}{\pgfqpoint{1.560248in}{1.898332in}}%
\pgfpathclose%
\pgfusepath{stroke,fill}%
\end{pgfscope}%
\begin{pgfscope}%
\pgfpathrectangle{\pgfqpoint{0.100000in}{0.212622in}}{\pgfqpoint{3.696000in}{3.696000in}}%
\pgfusepath{clip}%
\pgfsetbuttcap%
\pgfsetroundjoin%
\definecolor{currentfill}{rgb}{0.121569,0.466667,0.705882}%
\pgfsetfillcolor{currentfill}%
\pgfsetfillopacity{0.914575}%
\pgfsetlinewidth{1.003750pt}%
\definecolor{currentstroke}{rgb}{0.121569,0.466667,0.705882}%
\pgfsetstrokecolor{currentstroke}%
\pgfsetstrokeopacity{0.914575}%
\pgfsetdash{}{0pt}%
\pgfpathmoveto{\pgfqpoint{2.346754in}{1.438424in}}%
\pgfpathcurveto{\pgfqpoint{2.354990in}{1.438424in}}{\pgfqpoint{2.362890in}{1.441696in}}{\pgfqpoint{2.368714in}{1.447520in}}%
\pgfpathcurveto{\pgfqpoint{2.374538in}{1.453344in}}{\pgfqpoint{2.377810in}{1.461244in}}{\pgfqpoint{2.377810in}{1.469481in}}%
\pgfpathcurveto{\pgfqpoint{2.377810in}{1.477717in}}{\pgfqpoint{2.374538in}{1.485617in}}{\pgfqpoint{2.368714in}{1.491441in}}%
\pgfpathcurveto{\pgfqpoint{2.362890in}{1.497265in}}{\pgfqpoint{2.354990in}{1.500537in}}{\pgfqpoint{2.346754in}{1.500537in}}%
\pgfpathcurveto{\pgfqpoint{2.338517in}{1.500537in}}{\pgfqpoint{2.330617in}{1.497265in}}{\pgfqpoint{2.324793in}{1.491441in}}%
\pgfpathcurveto{\pgfqpoint{2.318969in}{1.485617in}}{\pgfqpoint{2.315697in}{1.477717in}}{\pgfqpoint{2.315697in}{1.469481in}}%
\pgfpathcurveto{\pgfqpoint{2.315697in}{1.461244in}}{\pgfqpoint{2.318969in}{1.453344in}}{\pgfqpoint{2.324793in}{1.447520in}}%
\pgfpathcurveto{\pgfqpoint{2.330617in}{1.441696in}}{\pgfqpoint{2.338517in}{1.438424in}}{\pgfqpoint{2.346754in}{1.438424in}}%
\pgfpathclose%
\pgfusepath{stroke,fill}%
\end{pgfscope}%
\begin{pgfscope}%
\pgfpathrectangle{\pgfqpoint{0.100000in}{0.212622in}}{\pgfqpoint{3.696000in}{3.696000in}}%
\pgfusepath{clip}%
\pgfsetbuttcap%
\pgfsetroundjoin%
\definecolor{currentfill}{rgb}{0.121569,0.466667,0.705882}%
\pgfsetfillcolor{currentfill}%
\pgfsetfillopacity{0.916324}%
\pgfsetlinewidth{1.003750pt}%
\definecolor{currentstroke}{rgb}{0.121569,0.466667,0.705882}%
\pgfsetstrokecolor{currentstroke}%
\pgfsetstrokeopacity{0.916324}%
\pgfsetdash{}{0pt}%
\pgfpathmoveto{\pgfqpoint{1.592585in}{1.871522in}}%
\pgfpathcurveto{\pgfqpoint{1.600821in}{1.871522in}}{\pgfqpoint{1.608721in}{1.874794in}}{\pgfqpoint{1.614545in}{1.880618in}}%
\pgfpathcurveto{\pgfqpoint{1.620369in}{1.886442in}}{\pgfqpoint{1.623641in}{1.894342in}}{\pgfqpoint{1.623641in}{1.902579in}}%
\pgfpathcurveto{\pgfqpoint{1.623641in}{1.910815in}}{\pgfqpoint{1.620369in}{1.918715in}}{\pgfqpoint{1.614545in}{1.924539in}}%
\pgfpathcurveto{\pgfqpoint{1.608721in}{1.930363in}}{\pgfqpoint{1.600821in}{1.933635in}}{\pgfqpoint{1.592585in}{1.933635in}}%
\pgfpathcurveto{\pgfqpoint{1.584348in}{1.933635in}}{\pgfqpoint{1.576448in}{1.930363in}}{\pgfqpoint{1.570624in}{1.924539in}}%
\pgfpathcurveto{\pgfqpoint{1.564800in}{1.918715in}}{\pgfqpoint{1.561528in}{1.910815in}}{\pgfqpoint{1.561528in}{1.902579in}}%
\pgfpathcurveto{\pgfqpoint{1.561528in}{1.894342in}}{\pgfqpoint{1.564800in}{1.886442in}}{\pgfqpoint{1.570624in}{1.880618in}}%
\pgfpathcurveto{\pgfqpoint{1.576448in}{1.874794in}}{\pgfqpoint{1.584348in}{1.871522in}}{\pgfqpoint{1.592585in}{1.871522in}}%
\pgfpathclose%
\pgfusepath{stroke,fill}%
\end{pgfscope}%
\begin{pgfscope}%
\pgfpathrectangle{\pgfqpoint{0.100000in}{0.212622in}}{\pgfqpoint{3.696000in}{3.696000in}}%
\pgfusepath{clip}%
\pgfsetbuttcap%
\pgfsetroundjoin%
\definecolor{currentfill}{rgb}{0.121569,0.466667,0.705882}%
\pgfsetfillcolor{currentfill}%
\pgfsetfillopacity{0.918666}%
\pgfsetlinewidth{1.003750pt}%
\definecolor{currentstroke}{rgb}{0.121569,0.466667,0.705882}%
\pgfsetstrokecolor{currentstroke}%
\pgfsetstrokeopacity{0.918666}%
\pgfsetdash{}{0pt}%
\pgfpathmoveto{\pgfqpoint{2.350004in}{1.434740in}}%
\pgfpathcurveto{\pgfqpoint{2.358240in}{1.434740in}}{\pgfqpoint{2.366140in}{1.438012in}}{\pgfqpoint{2.371964in}{1.443836in}}%
\pgfpathcurveto{\pgfqpoint{2.377788in}{1.449660in}}{\pgfqpoint{2.381060in}{1.457560in}}{\pgfqpoint{2.381060in}{1.465797in}}%
\pgfpathcurveto{\pgfqpoint{2.381060in}{1.474033in}}{\pgfqpoint{2.377788in}{1.481933in}}{\pgfqpoint{2.371964in}{1.487757in}}%
\pgfpathcurveto{\pgfqpoint{2.366140in}{1.493581in}}{\pgfqpoint{2.358240in}{1.496853in}}{\pgfqpoint{2.350004in}{1.496853in}}%
\pgfpathcurveto{\pgfqpoint{2.341767in}{1.496853in}}{\pgfqpoint{2.333867in}{1.493581in}}{\pgfqpoint{2.328043in}{1.487757in}}%
\pgfpathcurveto{\pgfqpoint{2.322220in}{1.481933in}}{\pgfqpoint{2.318947in}{1.474033in}}{\pgfqpoint{2.318947in}{1.465797in}}%
\pgfpathcurveto{\pgfqpoint{2.318947in}{1.457560in}}{\pgfqpoint{2.322220in}{1.449660in}}{\pgfqpoint{2.328043in}{1.443836in}}%
\pgfpathcurveto{\pgfqpoint{2.333867in}{1.438012in}}{\pgfqpoint{2.341767in}{1.434740in}}{\pgfqpoint{2.350004in}{1.434740in}}%
\pgfpathclose%
\pgfusepath{stroke,fill}%
\end{pgfscope}%
\begin{pgfscope}%
\pgfpathrectangle{\pgfqpoint{0.100000in}{0.212622in}}{\pgfqpoint{3.696000in}{3.696000in}}%
\pgfusepath{clip}%
\pgfsetbuttcap%
\pgfsetroundjoin%
\definecolor{currentfill}{rgb}{0.121569,0.466667,0.705882}%
\pgfsetfillcolor{currentfill}%
\pgfsetfillopacity{0.920516}%
\pgfsetlinewidth{1.003750pt}%
\definecolor{currentstroke}{rgb}{0.121569,0.466667,0.705882}%
\pgfsetstrokecolor{currentstroke}%
\pgfsetstrokeopacity{0.920516}%
\pgfsetdash{}{0pt}%
\pgfpathmoveto{\pgfqpoint{1.621565in}{1.854677in}}%
\pgfpathcurveto{\pgfqpoint{1.629802in}{1.854677in}}{\pgfqpoint{1.637702in}{1.857950in}}{\pgfqpoint{1.643526in}{1.863773in}}%
\pgfpathcurveto{\pgfqpoint{1.649350in}{1.869597in}}{\pgfqpoint{1.652622in}{1.877497in}}{\pgfqpoint{1.652622in}{1.885734in}}%
\pgfpathcurveto{\pgfqpoint{1.652622in}{1.893970in}}{\pgfqpoint{1.649350in}{1.901870in}}{\pgfqpoint{1.643526in}{1.907694in}}%
\pgfpathcurveto{\pgfqpoint{1.637702in}{1.913518in}}{\pgfqpoint{1.629802in}{1.916790in}}{\pgfqpoint{1.621565in}{1.916790in}}%
\pgfpathcurveto{\pgfqpoint{1.613329in}{1.916790in}}{\pgfqpoint{1.605429in}{1.913518in}}{\pgfqpoint{1.599605in}{1.907694in}}%
\pgfpathcurveto{\pgfqpoint{1.593781in}{1.901870in}}{\pgfqpoint{1.590509in}{1.893970in}}{\pgfqpoint{1.590509in}{1.885734in}}%
\pgfpathcurveto{\pgfqpoint{1.590509in}{1.877497in}}{\pgfqpoint{1.593781in}{1.869597in}}{\pgfqpoint{1.599605in}{1.863773in}}%
\pgfpathcurveto{\pgfqpoint{1.605429in}{1.857950in}}{\pgfqpoint{1.613329in}{1.854677in}}{\pgfqpoint{1.621565in}{1.854677in}}%
\pgfpathclose%
\pgfusepath{stroke,fill}%
\end{pgfscope}%
\begin{pgfscope}%
\pgfpathrectangle{\pgfqpoint{0.100000in}{0.212622in}}{\pgfqpoint{3.696000in}{3.696000in}}%
\pgfusepath{clip}%
\pgfsetbuttcap%
\pgfsetroundjoin%
\definecolor{currentfill}{rgb}{0.121569,0.466667,0.705882}%
\pgfsetfillcolor{currentfill}%
\pgfsetfillopacity{0.922784}%
\pgfsetlinewidth{1.003750pt}%
\definecolor{currentstroke}{rgb}{0.121569,0.466667,0.705882}%
\pgfsetstrokecolor{currentstroke}%
\pgfsetstrokeopacity{0.922784}%
\pgfsetdash{}{0pt}%
\pgfpathmoveto{\pgfqpoint{1.646815in}{1.829981in}}%
\pgfpathcurveto{\pgfqpoint{1.655051in}{1.829981in}}{\pgfqpoint{1.662951in}{1.833253in}}{\pgfqpoint{1.668775in}{1.839077in}}%
\pgfpathcurveto{\pgfqpoint{1.674599in}{1.844901in}}{\pgfqpoint{1.677871in}{1.852801in}}{\pgfqpoint{1.677871in}{1.861037in}}%
\pgfpathcurveto{\pgfqpoint{1.677871in}{1.869273in}}{\pgfqpoint{1.674599in}{1.877173in}}{\pgfqpoint{1.668775in}{1.882997in}}%
\pgfpathcurveto{\pgfqpoint{1.662951in}{1.888821in}}{\pgfqpoint{1.655051in}{1.892094in}}{\pgfqpoint{1.646815in}{1.892094in}}%
\pgfpathcurveto{\pgfqpoint{1.638579in}{1.892094in}}{\pgfqpoint{1.630679in}{1.888821in}}{\pgfqpoint{1.624855in}{1.882997in}}%
\pgfpathcurveto{\pgfqpoint{1.619031in}{1.877173in}}{\pgfqpoint{1.615758in}{1.869273in}}{\pgfqpoint{1.615758in}{1.861037in}}%
\pgfpathcurveto{\pgfqpoint{1.615758in}{1.852801in}}{\pgfqpoint{1.619031in}{1.844901in}}{\pgfqpoint{1.624855in}{1.839077in}}%
\pgfpathcurveto{\pgfqpoint{1.630679in}{1.833253in}}{\pgfqpoint{1.638579in}{1.829981in}}{\pgfqpoint{1.646815in}{1.829981in}}%
\pgfpathclose%
\pgfusepath{stroke,fill}%
\end{pgfscope}%
\begin{pgfscope}%
\pgfpathrectangle{\pgfqpoint{0.100000in}{0.212622in}}{\pgfqpoint{3.696000in}{3.696000in}}%
\pgfusepath{clip}%
\pgfsetbuttcap%
\pgfsetroundjoin%
\definecolor{currentfill}{rgb}{0.121569,0.466667,0.705882}%
\pgfsetfillcolor{currentfill}%
\pgfsetfillopacity{0.923299}%
\pgfsetlinewidth{1.003750pt}%
\definecolor{currentstroke}{rgb}{0.121569,0.466667,0.705882}%
\pgfsetstrokecolor{currentstroke}%
\pgfsetstrokeopacity{0.923299}%
\pgfsetdash{}{0pt}%
\pgfpathmoveto{\pgfqpoint{2.353605in}{1.432088in}}%
\pgfpathcurveto{\pgfqpoint{2.361841in}{1.432088in}}{\pgfqpoint{2.369741in}{1.435360in}}{\pgfqpoint{2.375565in}{1.441184in}}%
\pgfpathcurveto{\pgfqpoint{2.381389in}{1.447008in}}{\pgfqpoint{2.384661in}{1.454908in}}{\pgfqpoint{2.384661in}{1.463145in}}%
\pgfpathcurveto{\pgfqpoint{2.384661in}{1.471381in}}{\pgfqpoint{2.381389in}{1.479281in}}{\pgfqpoint{2.375565in}{1.485105in}}%
\pgfpathcurveto{\pgfqpoint{2.369741in}{1.490929in}}{\pgfqpoint{2.361841in}{1.494201in}}{\pgfqpoint{2.353605in}{1.494201in}}%
\pgfpathcurveto{\pgfqpoint{2.345369in}{1.494201in}}{\pgfqpoint{2.337469in}{1.490929in}}{\pgfqpoint{2.331645in}{1.485105in}}%
\pgfpathcurveto{\pgfqpoint{2.325821in}{1.479281in}}{\pgfqpoint{2.322548in}{1.471381in}}{\pgfqpoint{2.322548in}{1.463145in}}%
\pgfpathcurveto{\pgfqpoint{2.322548in}{1.454908in}}{\pgfqpoint{2.325821in}{1.447008in}}{\pgfqpoint{2.331645in}{1.441184in}}%
\pgfpathcurveto{\pgfqpoint{2.337469in}{1.435360in}}{\pgfqpoint{2.345369in}{1.432088in}}{\pgfqpoint{2.353605in}{1.432088in}}%
\pgfpathclose%
\pgfusepath{stroke,fill}%
\end{pgfscope}%
\begin{pgfscope}%
\pgfpathrectangle{\pgfqpoint{0.100000in}{0.212622in}}{\pgfqpoint{3.696000in}{3.696000in}}%
\pgfusepath{clip}%
\pgfsetbuttcap%
\pgfsetroundjoin%
\definecolor{currentfill}{rgb}{0.121569,0.466667,0.705882}%
\pgfsetfillcolor{currentfill}%
\pgfsetfillopacity{0.925450}%
\pgfsetlinewidth{1.003750pt}%
\definecolor{currentstroke}{rgb}{0.121569,0.466667,0.705882}%
\pgfsetstrokecolor{currentstroke}%
\pgfsetstrokeopacity{0.925450}%
\pgfsetdash{}{0pt}%
\pgfpathmoveto{\pgfqpoint{1.671104in}{1.809160in}}%
\pgfpathcurveto{\pgfqpoint{1.679340in}{1.809160in}}{\pgfqpoint{1.687240in}{1.812432in}}{\pgfqpoint{1.693064in}{1.818256in}}%
\pgfpathcurveto{\pgfqpoint{1.698888in}{1.824080in}}{\pgfqpoint{1.702161in}{1.831980in}}{\pgfqpoint{1.702161in}{1.840216in}}%
\pgfpathcurveto{\pgfqpoint{1.702161in}{1.848453in}}{\pgfqpoint{1.698888in}{1.856353in}}{\pgfqpoint{1.693064in}{1.862177in}}%
\pgfpathcurveto{\pgfqpoint{1.687240in}{1.868001in}}{\pgfqpoint{1.679340in}{1.871273in}}{\pgfqpoint{1.671104in}{1.871273in}}%
\pgfpathcurveto{\pgfqpoint{1.662868in}{1.871273in}}{\pgfqpoint{1.654968in}{1.868001in}}{\pgfqpoint{1.649144in}{1.862177in}}%
\pgfpathcurveto{\pgfqpoint{1.643320in}{1.856353in}}{\pgfqpoint{1.640048in}{1.848453in}}{\pgfqpoint{1.640048in}{1.840216in}}%
\pgfpathcurveto{\pgfqpoint{1.640048in}{1.831980in}}{\pgfqpoint{1.643320in}{1.824080in}}{\pgfqpoint{1.649144in}{1.818256in}}%
\pgfpathcurveto{\pgfqpoint{1.654968in}{1.812432in}}{\pgfqpoint{1.662868in}{1.809160in}}{\pgfqpoint{1.671104in}{1.809160in}}%
\pgfpathclose%
\pgfusepath{stroke,fill}%
\end{pgfscope}%
\begin{pgfscope}%
\pgfpathrectangle{\pgfqpoint{0.100000in}{0.212622in}}{\pgfqpoint{3.696000in}{3.696000in}}%
\pgfusepath{clip}%
\pgfsetbuttcap%
\pgfsetroundjoin%
\definecolor{currentfill}{rgb}{0.121569,0.466667,0.705882}%
\pgfsetfillcolor{currentfill}%
\pgfsetfillopacity{0.927660}%
\pgfsetlinewidth{1.003750pt}%
\definecolor{currentstroke}{rgb}{0.121569,0.466667,0.705882}%
\pgfsetstrokecolor{currentstroke}%
\pgfsetstrokeopacity{0.927660}%
\pgfsetdash{}{0pt}%
\pgfpathmoveto{\pgfqpoint{2.357267in}{1.426372in}}%
\pgfpathcurveto{\pgfqpoint{2.365503in}{1.426372in}}{\pgfqpoint{2.373403in}{1.429645in}}{\pgfqpoint{2.379227in}{1.435469in}}%
\pgfpathcurveto{\pgfqpoint{2.385051in}{1.441292in}}{\pgfqpoint{2.388323in}{1.449193in}}{\pgfqpoint{2.388323in}{1.457429in}}%
\pgfpathcurveto{\pgfqpoint{2.388323in}{1.465665in}}{\pgfqpoint{2.385051in}{1.473565in}}{\pgfqpoint{2.379227in}{1.479389in}}%
\pgfpathcurveto{\pgfqpoint{2.373403in}{1.485213in}}{\pgfqpoint{2.365503in}{1.488485in}}{\pgfqpoint{2.357267in}{1.488485in}}%
\pgfpathcurveto{\pgfqpoint{2.349030in}{1.488485in}}{\pgfqpoint{2.341130in}{1.485213in}}{\pgfqpoint{2.335306in}{1.479389in}}%
\pgfpathcurveto{\pgfqpoint{2.329483in}{1.473565in}}{\pgfqpoint{2.326210in}{1.465665in}}{\pgfqpoint{2.326210in}{1.457429in}}%
\pgfpathcurveto{\pgfqpoint{2.326210in}{1.449193in}}{\pgfqpoint{2.329483in}{1.441292in}}{\pgfqpoint{2.335306in}{1.435469in}}%
\pgfpathcurveto{\pgfqpoint{2.341130in}{1.429645in}}{\pgfqpoint{2.349030in}{1.426372in}}{\pgfqpoint{2.357267in}{1.426372in}}%
\pgfpathclose%
\pgfusepath{stroke,fill}%
\end{pgfscope}%
\begin{pgfscope}%
\pgfpathrectangle{\pgfqpoint{0.100000in}{0.212622in}}{\pgfqpoint{3.696000in}{3.696000in}}%
\pgfusepath{clip}%
\pgfsetbuttcap%
\pgfsetroundjoin%
\definecolor{currentfill}{rgb}{0.121569,0.466667,0.705882}%
\pgfsetfillcolor{currentfill}%
\pgfsetfillopacity{0.928082}%
\pgfsetlinewidth{1.003750pt}%
\definecolor{currentstroke}{rgb}{0.121569,0.466667,0.705882}%
\pgfsetstrokecolor{currentstroke}%
\pgfsetstrokeopacity{0.928082}%
\pgfsetdash{}{0pt}%
\pgfpathmoveto{\pgfqpoint{1.692748in}{1.794553in}}%
\pgfpathcurveto{\pgfqpoint{1.700984in}{1.794553in}}{\pgfqpoint{1.708884in}{1.797825in}}{\pgfqpoint{1.714708in}{1.803649in}}%
\pgfpathcurveto{\pgfqpoint{1.720532in}{1.809473in}}{\pgfqpoint{1.723805in}{1.817373in}}{\pgfqpoint{1.723805in}{1.825609in}}%
\pgfpathcurveto{\pgfqpoint{1.723805in}{1.833845in}}{\pgfqpoint{1.720532in}{1.841745in}}{\pgfqpoint{1.714708in}{1.847569in}}%
\pgfpathcurveto{\pgfqpoint{1.708884in}{1.853393in}}{\pgfqpoint{1.700984in}{1.856666in}}{\pgfqpoint{1.692748in}{1.856666in}}%
\pgfpathcurveto{\pgfqpoint{1.684512in}{1.856666in}}{\pgfqpoint{1.676612in}{1.853393in}}{\pgfqpoint{1.670788in}{1.847569in}}%
\pgfpathcurveto{\pgfqpoint{1.664964in}{1.841745in}}{\pgfqpoint{1.661692in}{1.833845in}}{\pgfqpoint{1.661692in}{1.825609in}}%
\pgfpathcurveto{\pgfqpoint{1.661692in}{1.817373in}}{\pgfqpoint{1.664964in}{1.809473in}}{\pgfqpoint{1.670788in}{1.803649in}}%
\pgfpathcurveto{\pgfqpoint{1.676612in}{1.797825in}}{\pgfqpoint{1.684512in}{1.794553in}}{\pgfqpoint{1.692748in}{1.794553in}}%
\pgfpathclose%
\pgfusepath{stroke,fill}%
\end{pgfscope}%
\begin{pgfscope}%
\pgfpathrectangle{\pgfqpoint{0.100000in}{0.212622in}}{\pgfqpoint{3.696000in}{3.696000in}}%
\pgfusepath{clip}%
\pgfsetbuttcap%
\pgfsetroundjoin%
\definecolor{currentfill}{rgb}{0.121569,0.466667,0.705882}%
\pgfsetfillcolor{currentfill}%
\pgfsetfillopacity{0.931596}%
\pgfsetlinewidth{1.003750pt}%
\definecolor{currentstroke}{rgb}{0.121569,0.466667,0.705882}%
\pgfsetstrokecolor{currentstroke}%
\pgfsetstrokeopacity{0.931596}%
\pgfsetdash{}{0pt}%
\pgfpathmoveto{\pgfqpoint{2.360058in}{1.414951in}}%
\pgfpathcurveto{\pgfqpoint{2.368294in}{1.414951in}}{\pgfqpoint{2.376194in}{1.418223in}}{\pgfqpoint{2.382018in}{1.424047in}}%
\pgfpathcurveto{\pgfqpoint{2.387842in}{1.429871in}}{\pgfqpoint{2.391114in}{1.437771in}}{\pgfqpoint{2.391114in}{1.446007in}}%
\pgfpathcurveto{\pgfqpoint{2.391114in}{1.454243in}}{\pgfqpoint{2.387842in}{1.462143in}}{\pgfqpoint{2.382018in}{1.467967in}}%
\pgfpathcurveto{\pgfqpoint{2.376194in}{1.473791in}}{\pgfqpoint{2.368294in}{1.477064in}}{\pgfqpoint{2.360058in}{1.477064in}}%
\pgfpathcurveto{\pgfqpoint{2.351822in}{1.477064in}}{\pgfqpoint{2.343922in}{1.473791in}}{\pgfqpoint{2.338098in}{1.467967in}}%
\pgfpathcurveto{\pgfqpoint{2.332274in}{1.462143in}}{\pgfqpoint{2.329001in}{1.454243in}}{\pgfqpoint{2.329001in}{1.446007in}}%
\pgfpathcurveto{\pgfqpoint{2.329001in}{1.437771in}}{\pgfqpoint{2.332274in}{1.429871in}}{\pgfqpoint{2.338098in}{1.424047in}}%
\pgfpathcurveto{\pgfqpoint{2.343922in}{1.418223in}}{\pgfqpoint{2.351822in}{1.414951in}}{\pgfqpoint{2.360058in}{1.414951in}}%
\pgfpathclose%
\pgfusepath{stroke,fill}%
\end{pgfscope}%
\begin{pgfscope}%
\pgfpathrectangle{\pgfqpoint{0.100000in}{0.212622in}}{\pgfqpoint{3.696000in}{3.696000in}}%
\pgfusepath{clip}%
\pgfsetbuttcap%
\pgfsetroundjoin%
\definecolor{currentfill}{rgb}{0.121569,0.466667,0.705882}%
\pgfsetfillcolor{currentfill}%
\pgfsetfillopacity{0.932124}%
\pgfsetlinewidth{1.003750pt}%
\definecolor{currentstroke}{rgb}{0.121569,0.466667,0.705882}%
\pgfsetstrokecolor{currentstroke}%
\pgfsetstrokeopacity{0.932124}%
\pgfsetdash{}{0pt}%
\pgfpathmoveto{\pgfqpoint{1.734200in}{1.769675in}}%
\pgfpathcurveto{\pgfqpoint{1.742437in}{1.769675in}}{\pgfqpoint{1.750337in}{1.772947in}}{\pgfqpoint{1.756161in}{1.778771in}}%
\pgfpathcurveto{\pgfqpoint{1.761985in}{1.784595in}}{\pgfqpoint{1.765257in}{1.792495in}}{\pgfqpoint{1.765257in}{1.800731in}}%
\pgfpathcurveto{\pgfqpoint{1.765257in}{1.808967in}}{\pgfqpoint{1.761985in}{1.816868in}}{\pgfqpoint{1.756161in}{1.822691in}}%
\pgfpathcurveto{\pgfqpoint{1.750337in}{1.828515in}}{\pgfqpoint{1.742437in}{1.831788in}}{\pgfqpoint{1.734200in}{1.831788in}}%
\pgfpathcurveto{\pgfqpoint{1.725964in}{1.831788in}}{\pgfqpoint{1.718064in}{1.828515in}}{\pgfqpoint{1.712240in}{1.822691in}}%
\pgfpathcurveto{\pgfqpoint{1.706416in}{1.816868in}}{\pgfqpoint{1.703144in}{1.808967in}}{\pgfqpoint{1.703144in}{1.800731in}}%
\pgfpathcurveto{\pgfqpoint{1.703144in}{1.792495in}}{\pgfqpoint{1.706416in}{1.784595in}}{\pgfqpoint{1.712240in}{1.778771in}}%
\pgfpathcurveto{\pgfqpoint{1.718064in}{1.772947in}}{\pgfqpoint{1.725964in}{1.769675in}}{\pgfqpoint{1.734200in}{1.769675in}}%
\pgfpathclose%
\pgfusepath{stroke,fill}%
\end{pgfscope}%
\begin{pgfscope}%
\pgfpathrectangle{\pgfqpoint{0.100000in}{0.212622in}}{\pgfqpoint{3.696000in}{3.696000in}}%
\pgfusepath{clip}%
\pgfsetbuttcap%
\pgfsetroundjoin%
\definecolor{currentfill}{rgb}{0.121569,0.466667,0.705882}%
\pgfsetfillcolor{currentfill}%
\pgfsetfillopacity{0.936411}%
\pgfsetlinewidth{1.003750pt}%
\definecolor{currentstroke}{rgb}{0.121569,0.466667,0.705882}%
\pgfsetstrokecolor{currentstroke}%
\pgfsetstrokeopacity{0.936411}%
\pgfsetdash{}{0pt}%
\pgfpathmoveto{\pgfqpoint{2.364347in}{1.407210in}}%
\pgfpathcurveto{\pgfqpoint{2.372583in}{1.407210in}}{\pgfqpoint{2.380483in}{1.410482in}}{\pgfqpoint{2.386307in}{1.416306in}}%
\pgfpathcurveto{\pgfqpoint{2.392131in}{1.422130in}}{\pgfqpoint{2.395403in}{1.430030in}}{\pgfqpoint{2.395403in}{1.438266in}}%
\pgfpathcurveto{\pgfqpoint{2.395403in}{1.446502in}}{\pgfqpoint{2.392131in}{1.454402in}}{\pgfqpoint{2.386307in}{1.460226in}}%
\pgfpathcurveto{\pgfqpoint{2.380483in}{1.466050in}}{\pgfqpoint{2.372583in}{1.469323in}}{\pgfqpoint{2.364347in}{1.469323in}}%
\pgfpathcurveto{\pgfqpoint{2.356110in}{1.469323in}}{\pgfqpoint{2.348210in}{1.466050in}}{\pgfqpoint{2.342386in}{1.460226in}}%
\pgfpathcurveto{\pgfqpoint{2.336563in}{1.454402in}}{\pgfqpoint{2.333290in}{1.446502in}}{\pgfqpoint{2.333290in}{1.438266in}}%
\pgfpathcurveto{\pgfqpoint{2.333290in}{1.430030in}}{\pgfqpoint{2.336563in}{1.422130in}}{\pgfqpoint{2.342386in}{1.416306in}}%
\pgfpathcurveto{\pgfqpoint{2.348210in}{1.410482in}}{\pgfqpoint{2.356110in}{1.407210in}}{\pgfqpoint{2.364347in}{1.407210in}}%
\pgfpathclose%
\pgfusepath{stroke,fill}%
\end{pgfscope}%
\begin{pgfscope}%
\pgfpathrectangle{\pgfqpoint{0.100000in}{0.212622in}}{\pgfqpoint{3.696000in}{3.696000in}}%
\pgfusepath{clip}%
\pgfsetbuttcap%
\pgfsetroundjoin%
\definecolor{currentfill}{rgb}{0.121569,0.466667,0.705882}%
\pgfsetfillcolor{currentfill}%
\pgfsetfillopacity{0.936437}%
\pgfsetlinewidth{1.003750pt}%
\definecolor{currentstroke}{rgb}{0.121569,0.466667,0.705882}%
\pgfsetstrokecolor{currentstroke}%
\pgfsetstrokeopacity{0.936437}%
\pgfsetdash{}{0pt}%
\pgfpathmoveto{\pgfqpoint{1.771609in}{1.747384in}}%
\pgfpathcurveto{\pgfqpoint{1.779845in}{1.747384in}}{\pgfqpoint{1.787745in}{1.750656in}}{\pgfqpoint{1.793569in}{1.756480in}}%
\pgfpathcurveto{\pgfqpoint{1.799393in}{1.762304in}}{\pgfqpoint{1.802665in}{1.770204in}}{\pgfqpoint{1.802665in}{1.778441in}}%
\pgfpathcurveto{\pgfqpoint{1.802665in}{1.786677in}}{\pgfqpoint{1.799393in}{1.794577in}}{\pgfqpoint{1.793569in}{1.800401in}}%
\pgfpathcurveto{\pgfqpoint{1.787745in}{1.806225in}}{\pgfqpoint{1.779845in}{1.809497in}}{\pgfqpoint{1.771609in}{1.809497in}}%
\pgfpathcurveto{\pgfqpoint{1.763372in}{1.809497in}}{\pgfqpoint{1.755472in}{1.806225in}}{\pgfqpoint{1.749648in}{1.800401in}}%
\pgfpathcurveto{\pgfqpoint{1.743825in}{1.794577in}}{\pgfqpoint{1.740552in}{1.786677in}}{\pgfqpoint{1.740552in}{1.778441in}}%
\pgfpathcurveto{\pgfqpoint{1.740552in}{1.770204in}}{\pgfqpoint{1.743825in}{1.762304in}}{\pgfqpoint{1.749648in}{1.756480in}}%
\pgfpathcurveto{\pgfqpoint{1.755472in}{1.750656in}}{\pgfqpoint{1.763372in}{1.747384in}}{\pgfqpoint{1.771609in}{1.747384in}}%
\pgfpathclose%
\pgfusepath{stroke,fill}%
\end{pgfscope}%
\begin{pgfscope}%
\pgfpathrectangle{\pgfqpoint{0.100000in}{0.212622in}}{\pgfqpoint{3.696000in}{3.696000in}}%
\pgfusepath{clip}%
\pgfsetbuttcap%
\pgfsetroundjoin%
\definecolor{currentfill}{rgb}{0.121569,0.466667,0.705882}%
\pgfsetfillcolor{currentfill}%
\pgfsetfillopacity{0.938917}%
\pgfsetlinewidth{1.003750pt}%
\definecolor{currentstroke}{rgb}{0.121569,0.466667,0.705882}%
\pgfsetstrokecolor{currentstroke}%
\pgfsetstrokeopacity{0.938917}%
\pgfsetdash{}{0pt}%
\pgfpathmoveto{\pgfqpoint{1.808083in}{1.715568in}}%
\pgfpathcurveto{\pgfqpoint{1.816319in}{1.715568in}}{\pgfqpoint{1.824219in}{1.718841in}}{\pgfqpoint{1.830043in}{1.724665in}}%
\pgfpathcurveto{\pgfqpoint{1.835867in}{1.730489in}}{\pgfqpoint{1.839139in}{1.738389in}}{\pgfqpoint{1.839139in}{1.746625in}}%
\pgfpathcurveto{\pgfqpoint{1.839139in}{1.754861in}}{\pgfqpoint{1.835867in}{1.762761in}}{\pgfqpoint{1.830043in}{1.768585in}}%
\pgfpathcurveto{\pgfqpoint{1.824219in}{1.774409in}}{\pgfqpoint{1.816319in}{1.777681in}}{\pgfqpoint{1.808083in}{1.777681in}}%
\pgfpathcurveto{\pgfqpoint{1.799847in}{1.777681in}}{\pgfqpoint{1.791947in}{1.774409in}}{\pgfqpoint{1.786123in}{1.768585in}}%
\pgfpathcurveto{\pgfqpoint{1.780299in}{1.762761in}}{\pgfqpoint{1.777026in}{1.754861in}}{\pgfqpoint{1.777026in}{1.746625in}}%
\pgfpathcurveto{\pgfqpoint{1.777026in}{1.738389in}}{\pgfqpoint{1.780299in}{1.730489in}}{\pgfqpoint{1.786123in}{1.724665in}}%
\pgfpathcurveto{\pgfqpoint{1.791947in}{1.718841in}}{\pgfqpoint{1.799847in}{1.715568in}}{\pgfqpoint{1.808083in}{1.715568in}}%
\pgfpathclose%
\pgfusepath{stroke,fill}%
\end{pgfscope}%
\begin{pgfscope}%
\pgfpathrectangle{\pgfqpoint{0.100000in}{0.212622in}}{\pgfqpoint{3.696000in}{3.696000in}}%
\pgfusepath{clip}%
\pgfsetbuttcap%
\pgfsetroundjoin%
\definecolor{currentfill}{rgb}{0.121569,0.466667,0.705882}%
\pgfsetfillcolor{currentfill}%
\pgfsetfillopacity{0.941852}%
\pgfsetlinewidth{1.003750pt}%
\definecolor{currentstroke}{rgb}{0.121569,0.466667,0.705882}%
\pgfsetstrokecolor{currentstroke}%
\pgfsetstrokeopacity{0.941852}%
\pgfsetdash{}{0pt}%
\pgfpathmoveto{\pgfqpoint{2.368543in}{1.401636in}}%
\pgfpathcurveto{\pgfqpoint{2.376779in}{1.401636in}}{\pgfqpoint{2.384680in}{1.404908in}}{\pgfqpoint{2.390503in}{1.410732in}}%
\pgfpathcurveto{\pgfqpoint{2.396327in}{1.416556in}}{\pgfqpoint{2.399600in}{1.424456in}}{\pgfqpoint{2.399600in}{1.432692in}}%
\pgfpathcurveto{\pgfqpoint{2.399600in}{1.440929in}}{\pgfqpoint{2.396327in}{1.448829in}}{\pgfqpoint{2.390503in}{1.454653in}}%
\pgfpathcurveto{\pgfqpoint{2.384680in}{1.460477in}}{\pgfqpoint{2.376779in}{1.463749in}}{\pgfqpoint{2.368543in}{1.463749in}}%
\pgfpathcurveto{\pgfqpoint{2.360307in}{1.463749in}}{\pgfqpoint{2.352407in}{1.460477in}}{\pgfqpoint{2.346583in}{1.454653in}}%
\pgfpathcurveto{\pgfqpoint{2.340759in}{1.448829in}}{\pgfqpoint{2.337487in}{1.440929in}}{\pgfqpoint{2.337487in}{1.432692in}}%
\pgfpathcurveto{\pgfqpoint{2.337487in}{1.424456in}}{\pgfqpoint{2.340759in}{1.416556in}}{\pgfqpoint{2.346583in}{1.410732in}}%
\pgfpathcurveto{\pgfqpoint{2.352407in}{1.404908in}}{\pgfqpoint{2.360307in}{1.401636in}}{\pgfqpoint{2.368543in}{1.401636in}}%
\pgfpathclose%
\pgfusepath{stroke,fill}%
\end{pgfscope}%
\begin{pgfscope}%
\pgfpathrectangle{\pgfqpoint{0.100000in}{0.212622in}}{\pgfqpoint{3.696000in}{3.696000in}}%
\pgfusepath{clip}%
\pgfsetbuttcap%
\pgfsetroundjoin%
\definecolor{currentfill}{rgb}{0.121569,0.466667,0.705882}%
\pgfsetfillcolor{currentfill}%
\pgfsetfillopacity{0.942187}%
\pgfsetlinewidth{1.003750pt}%
\definecolor{currentstroke}{rgb}{0.121569,0.466667,0.705882}%
\pgfsetstrokecolor{currentstroke}%
\pgfsetstrokeopacity{0.942187}%
\pgfsetdash{}{0pt}%
\pgfpathmoveto{\pgfqpoint{1.841948in}{1.691897in}}%
\pgfpathcurveto{\pgfqpoint{1.850184in}{1.691897in}}{\pgfqpoint{1.858084in}{1.695170in}}{\pgfqpoint{1.863908in}{1.700994in}}%
\pgfpathcurveto{\pgfqpoint{1.869732in}{1.706818in}}{\pgfqpoint{1.873004in}{1.714718in}}{\pgfqpoint{1.873004in}{1.722954in}}%
\pgfpathcurveto{\pgfqpoint{1.873004in}{1.731190in}}{\pgfqpoint{1.869732in}{1.739090in}}{\pgfqpoint{1.863908in}{1.744914in}}%
\pgfpathcurveto{\pgfqpoint{1.858084in}{1.750738in}}{\pgfqpoint{1.850184in}{1.754010in}}{\pgfqpoint{1.841948in}{1.754010in}}%
\pgfpathcurveto{\pgfqpoint{1.833712in}{1.754010in}}{\pgfqpoint{1.825812in}{1.750738in}}{\pgfqpoint{1.819988in}{1.744914in}}%
\pgfpathcurveto{\pgfqpoint{1.814164in}{1.739090in}}{\pgfqpoint{1.810891in}{1.731190in}}{\pgfqpoint{1.810891in}{1.722954in}}%
\pgfpathcurveto{\pgfqpoint{1.810891in}{1.714718in}}{\pgfqpoint{1.814164in}{1.706818in}}{\pgfqpoint{1.819988in}{1.700994in}}%
\pgfpathcurveto{\pgfqpoint{1.825812in}{1.695170in}}{\pgfqpoint{1.833712in}{1.691897in}}{\pgfqpoint{1.841948in}{1.691897in}}%
\pgfpathclose%
\pgfusepath{stroke,fill}%
\end{pgfscope}%
\begin{pgfscope}%
\pgfpathrectangle{\pgfqpoint{0.100000in}{0.212622in}}{\pgfqpoint{3.696000in}{3.696000in}}%
\pgfusepath{clip}%
\pgfsetbuttcap%
\pgfsetroundjoin%
\definecolor{currentfill}{rgb}{0.121569,0.466667,0.705882}%
\pgfsetfillcolor{currentfill}%
\pgfsetfillopacity{0.945897}%
\pgfsetlinewidth{1.003750pt}%
\definecolor{currentstroke}{rgb}{0.121569,0.466667,0.705882}%
\pgfsetstrokecolor{currentstroke}%
\pgfsetstrokeopacity{0.945897}%
\pgfsetdash{}{0pt}%
\pgfpathmoveto{\pgfqpoint{1.873272in}{1.672055in}}%
\pgfpathcurveto{\pgfqpoint{1.881508in}{1.672055in}}{\pgfqpoint{1.889408in}{1.675327in}}{\pgfqpoint{1.895232in}{1.681151in}}%
\pgfpathcurveto{\pgfqpoint{1.901056in}{1.686975in}}{\pgfqpoint{1.904328in}{1.694875in}}{\pgfqpoint{1.904328in}{1.703112in}}%
\pgfpathcurveto{\pgfqpoint{1.904328in}{1.711348in}}{\pgfqpoint{1.901056in}{1.719248in}}{\pgfqpoint{1.895232in}{1.725072in}}%
\pgfpathcurveto{\pgfqpoint{1.889408in}{1.730896in}}{\pgfqpoint{1.881508in}{1.734168in}}{\pgfqpoint{1.873272in}{1.734168in}}%
\pgfpathcurveto{\pgfqpoint{1.865036in}{1.734168in}}{\pgfqpoint{1.857136in}{1.730896in}}{\pgfqpoint{1.851312in}{1.725072in}}%
\pgfpathcurveto{\pgfqpoint{1.845488in}{1.719248in}}{\pgfqpoint{1.842215in}{1.711348in}}{\pgfqpoint{1.842215in}{1.703112in}}%
\pgfpathcurveto{\pgfqpoint{1.842215in}{1.694875in}}{\pgfqpoint{1.845488in}{1.686975in}}{\pgfqpoint{1.851312in}{1.681151in}}%
\pgfpathcurveto{\pgfqpoint{1.857136in}{1.675327in}}{\pgfqpoint{1.865036in}{1.672055in}}{\pgfqpoint{1.873272in}{1.672055in}}%
\pgfpathclose%
\pgfusepath{stroke,fill}%
\end{pgfscope}%
\begin{pgfscope}%
\pgfpathrectangle{\pgfqpoint{0.100000in}{0.212622in}}{\pgfqpoint{3.696000in}{3.696000in}}%
\pgfusepath{clip}%
\pgfsetbuttcap%
\pgfsetroundjoin%
\definecolor{currentfill}{rgb}{0.121569,0.466667,0.705882}%
\pgfsetfillcolor{currentfill}%
\pgfsetfillopacity{0.948310}%
\pgfsetlinewidth{1.003750pt}%
\definecolor{currentstroke}{rgb}{0.121569,0.466667,0.705882}%
\pgfsetstrokecolor{currentstroke}%
\pgfsetstrokeopacity{0.948310}%
\pgfsetdash{}{0pt}%
\pgfpathmoveto{\pgfqpoint{2.372834in}{1.399154in}}%
\pgfpathcurveto{\pgfqpoint{2.381070in}{1.399154in}}{\pgfqpoint{2.388970in}{1.402426in}}{\pgfqpoint{2.394794in}{1.408250in}}%
\pgfpathcurveto{\pgfqpoint{2.400618in}{1.414074in}}{\pgfqpoint{2.403891in}{1.421974in}}{\pgfqpoint{2.403891in}{1.430210in}}%
\pgfpathcurveto{\pgfqpoint{2.403891in}{1.438447in}}{\pgfqpoint{2.400618in}{1.446347in}}{\pgfqpoint{2.394794in}{1.452170in}}%
\pgfpathcurveto{\pgfqpoint{2.388970in}{1.457994in}}{\pgfqpoint{2.381070in}{1.461267in}}{\pgfqpoint{2.372834in}{1.461267in}}%
\pgfpathcurveto{\pgfqpoint{2.364598in}{1.461267in}}{\pgfqpoint{2.356698in}{1.457994in}}{\pgfqpoint{2.350874in}{1.452170in}}%
\pgfpathcurveto{\pgfqpoint{2.345050in}{1.446347in}}{\pgfqpoint{2.341778in}{1.438447in}}{\pgfqpoint{2.341778in}{1.430210in}}%
\pgfpathcurveto{\pgfqpoint{2.341778in}{1.421974in}}{\pgfqpoint{2.345050in}{1.414074in}}{\pgfqpoint{2.350874in}{1.408250in}}%
\pgfpathcurveto{\pgfqpoint{2.356698in}{1.402426in}}{\pgfqpoint{2.364598in}{1.399154in}}{\pgfqpoint{2.372834in}{1.399154in}}%
\pgfpathclose%
\pgfusepath{stroke,fill}%
\end{pgfscope}%
\begin{pgfscope}%
\pgfpathrectangle{\pgfqpoint{0.100000in}{0.212622in}}{\pgfqpoint{3.696000in}{3.696000in}}%
\pgfusepath{clip}%
\pgfsetbuttcap%
\pgfsetroundjoin%
\definecolor{currentfill}{rgb}{0.121569,0.466667,0.705882}%
\pgfsetfillcolor{currentfill}%
\pgfsetfillopacity{0.948733}%
\pgfsetlinewidth{1.003750pt}%
\definecolor{currentstroke}{rgb}{0.121569,0.466667,0.705882}%
\pgfsetstrokecolor{currentstroke}%
\pgfsetstrokeopacity{0.948733}%
\pgfsetdash{}{0pt}%
\pgfpathmoveto{\pgfqpoint{1.903168in}{1.653243in}}%
\pgfpathcurveto{\pgfqpoint{1.911404in}{1.653243in}}{\pgfqpoint{1.919304in}{1.656516in}}{\pgfqpoint{1.925128in}{1.662340in}}%
\pgfpathcurveto{\pgfqpoint{1.930952in}{1.668164in}}{\pgfqpoint{1.934224in}{1.676064in}}{\pgfqpoint{1.934224in}{1.684300in}}%
\pgfpathcurveto{\pgfqpoint{1.934224in}{1.692536in}}{\pgfqpoint{1.930952in}{1.700436in}}{\pgfqpoint{1.925128in}{1.706260in}}%
\pgfpathcurveto{\pgfqpoint{1.919304in}{1.712084in}}{\pgfqpoint{1.911404in}{1.715356in}}{\pgfqpoint{1.903168in}{1.715356in}}%
\pgfpathcurveto{\pgfqpoint{1.894931in}{1.715356in}}{\pgfqpoint{1.887031in}{1.712084in}}{\pgfqpoint{1.881207in}{1.706260in}}%
\pgfpathcurveto{\pgfqpoint{1.875383in}{1.700436in}}{\pgfqpoint{1.872111in}{1.692536in}}{\pgfqpoint{1.872111in}{1.684300in}}%
\pgfpathcurveto{\pgfqpoint{1.872111in}{1.676064in}}{\pgfqpoint{1.875383in}{1.668164in}}{\pgfqpoint{1.881207in}{1.662340in}}%
\pgfpathcurveto{\pgfqpoint{1.887031in}{1.656516in}}{\pgfqpoint{1.894931in}{1.653243in}}{\pgfqpoint{1.903168in}{1.653243in}}%
\pgfpathclose%
\pgfusepath{stroke,fill}%
\end{pgfscope}%
\begin{pgfscope}%
\pgfpathrectangle{\pgfqpoint{0.100000in}{0.212622in}}{\pgfqpoint{3.696000in}{3.696000in}}%
\pgfusepath{clip}%
\pgfsetbuttcap%
\pgfsetroundjoin%
\definecolor{currentfill}{rgb}{0.121569,0.466667,0.705882}%
\pgfsetfillcolor{currentfill}%
\pgfsetfillopacity{0.950400}%
\pgfsetlinewidth{1.003750pt}%
\definecolor{currentstroke}{rgb}{0.121569,0.466667,0.705882}%
\pgfsetstrokecolor{currentstroke}%
\pgfsetstrokeopacity{0.950400}%
\pgfsetdash{}{0pt}%
\pgfpathmoveto{\pgfqpoint{1.930513in}{1.630215in}}%
\pgfpathcurveto{\pgfqpoint{1.938750in}{1.630215in}}{\pgfqpoint{1.946650in}{1.633487in}}{\pgfqpoint{1.952474in}{1.639311in}}%
\pgfpathcurveto{\pgfqpoint{1.958298in}{1.645135in}}{\pgfqpoint{1.961570in}{1.653035in}}{\pgfqpoint{1.961570in}{1.661271in}}%
\pgfpathcurveto{\pgfqpoint{1.961570in}{1.669508in}}{\pgfqpoint{1.958298in}{1.677408in}}{\pgfqpoint{1.952474in}{1.683232in}}%
\pgfpathcurveto{\pgfqpoint{1.946650in}{1.689056in}}{\pgfqpoint{1.938750in}{1.692328in}}{\pgfqpoint{1.930513in}{1.692328in}}%
\pgfpathcurveto{\pgfqpoint{1.922277in}{1.692328in}}{\pgfqpoint{1.914377in}{1.689056in}}{\pgfqpoint{1.908553in}{1.683232in}}%
\pgfpathcurveto{\pgfqpoint{1.902729in}{1.677408in}}{\pgfqpoint{1.899457in}{1.669508in}}{\pgfqpoint{1.899457in}{1.661271in}}%
\pgfpathcurveto{\pgfqpoint{1.899457in}{1.653035in}}{\pgfqpoint{1.902729in}{1.645135in}}{\pgfqpoint{1.908553in}{1.639311in}}%
\pgfpathcurveto{\pgfqpoint{1.914377in}{1.633487in}}{\pgfqpoint{1.922277in}{1.630215in}}{\pgfqpoint{1.930513in}{1.630215in}}%
\pgfpathclose%
\pgfusepath{stroke,fill}%
\end{pgfscope}%
\begin{pgfscope}%
\pgfpathrectangle{\pgfqpoint{0.100000in}{0.212622in}}{\pgfqpoint{3.696000in}{3.696000in}}%
\pgfusepath{clip}%
\pgfsetbuttcap%
\pgfsetroundjoin%
\definecolor{currentfill}{rgb}{0.121569,0.466667,0.705882}%
\pgfsetfillcolor{currentfill}%
\pgfsetfillopacity{0.952748}%
\pgfsetlinewidth{1.003750pt}%
\definecolor{currentstroke}{rgb}{0.121569,0.466667,0.705882}%
\pgfsetstrokecolor{currentstroke}%
\pgfsetstrokeopacity{0.952748}%
\pgfsetdash{}{0pt}%
\pgfpathmoveto{\pgfqpoint{1.955890in}{1.618172in}}%
\pgfpathcurveto{\pgfqpoint{1.964127in}{1.618172in}}{\pgfqpoint{1.972027in}{1.621445in}}{\pgfqpoint{1.977851in}{1.627268in}}%
\pgfpathcurveto{\pgfqpoint{1.983675in}{1.633092in}}{\pgfqpoint{1.986947in}{1.640992in}}{\pgfqpoint{1.986947in}{1.649229in}}%
\pgfpathcurveto{\pgfqpoint{1.986947in}{1.657465in}}{\pgfqpoint{1.983675in}{1.665365in}}{\pgfqpoint{1.977851in}{1.671189in}}%
\pgfpathcurveto{\pgfqpoint{1.972027in}{1.677013in}}{\pgfqpoint{1.964127in}{1.680285in}}{\pgfqpoint{1.955890in}{1.680285in}}%
\pgfpathcurveto{\pgfqpoint{1.947654in}{1.680285in}}{\pgfqpoint{1.939754in}{1.677013in}}{\pgfqpoint{1.933930in}{1.671189in}}%
\pgfpathcurveto{\pgfqpoint{1.928106in}{1.665365in}}{\pgfqpoint{1.924834in}{1.657465in}}{\pgfqpoint{1.924834in}{1.649229in}}%
\pgfpathcurveto{\pgfqpoint{1.924834in}{1.640992in}}{\pgfqpoint{1.928106in}{1.633092in}}{\pgfqpoint{1.933930in}{1.627268in}}%
\pgfpathcurveto{\pgfqpoint{1.939754in}{1.621445in}}{\pgfqpoint{1.947654in}{1.618172in}}{\pgfqpoint{1.955890in}{1.618172in}}%
\pgfpathclose%
\pgfusepath{stroke,fill}%
\end{pgfscope}%
\begin{pgfscope}%
\pgfpathrectangle{\pgfqpoint{0.100000in}{0.212622in}}{\pgfqpoint{3.696000in}{3.696000in}}%
\pgfusepath{clip}%
\pgfsetbuttcap%
\pgfsetroundjoin%
\definecolor{currentfill}{rgb}{0.121569,0.466667,0.705882}%
\pgfsetfillcolor{currentfill}%
\pgfsetfillopacity{0.953981}%
\pgfsetlinewidth{1.003750pt}%
\definecolor{currentstroke}{rgb}{0.121569,0.466667,0.705882}%
\pgfsetstrokecolor{currentstroke}%
\pgfsetstrokeopacity{0.953981}%
\pgfsetdash{}{0pt}%
\pgfpathmoveto{\pgfqpoint{2.377103in}{1.389337in}}%
\pgfpathcurveto{\pgfqpoint{2.385340in}{1.389337in}}{\pgfqpoint{2.393240in}{1.392609in}}{\pgfqpoint{2.399064in}{1.398433in}}%
\pgfpathcurveto{\pgfqpoint{2.404888in}{1.404257in}}{\pgfqpoint{2.408160in}{1.412157in}}{\pgfqpoint{2.408160in}{1.420394in}}%
\pgfpathcurveto{\pgfqpoint{2.408160in}{1.428630in}}{\pgfqpoint{2.404888in}{1.436530in}}{\pgfqpoint{2.399064in}{1.442354in}}%
\pgfpathcurveto{\pgfqpoint{2.393240in}{1.448178in}}{\pgfqpoint{2.385340in}{1.451450in}}{\pgfqpoint{2.377103in}{1.451450in}}%
\pgfpathcurveto{\pgfqpoint{2.368867in}{1.451450in}}{\pgfqpoint{2.360967in}{1.448178in}}{\pgfqpoint{2.355143in}{1.442354in}}%
\pgfpathcurveto{\pgfqpoint{2.349319in}{1.436530in}}{\pgfqpoint{2.346047in}{1.428630in}}{\pgfqpoint{2.346047in}{1.420394in}}%
\pgfpathcurveto{\pgfqpoint{2.346047in}{1.412157in}}{\pgfqpoint{2.349319in}{1.404257in}}{\pgfqpoint{2.355143in}{1.398433in}}%
\pgfpathcurveto{\pgfqpoint{2.360967in}{1.392609in}}{\pgfqpoint{2.368867in}{1.389337in}}{\pgfqpoint{2.377103in}{1.389337in}}%
\pgfpathclose%
\pgfusepath{stroke,fill}%
\end{pgfscope}%
\begin{pgfscope}%
\pgfpathrectangle{\pgfqpoint{0.100000in}{0.212622in}}{\pgfqpoint{3.696000in}{3.696000in}}%
\pgfusepath{clip}%
\pgfsetbuttcap%
\pgfsetroundjoin%
\definecolor{currentfill}{rgb}{0.121569,0.466667,0.705882}%
\pgfsetfillcolor{currentfill}%
\pgfsetfillopacity{0.954517}%
\pgfsetlinewidth{1.003750pt}%
\definecolor{currentstroke}{rgb}{0.121569,0.466667,0.705882}%
\pgfsetstrokecolor{currentstroke}%
\pgfsetstrokeopacity{0.954517}%
\pgfsetdash{}{0pt}%
\pgfpathmoveto{\pgfqpoint{1.976013in}{1.606557in}}%
\pgfpathcurveto{\pgfqpoint{1.984249in}{1.606557in}}{\pgfqpoint{1.992149in}{1.609830in}}{\pgfqpoint{1.997973in}{1.615654in}}%
\pgfpathcurveto{\pgfqpoint{2.003797in}{1.621478in}}{\pgfqpoint{2.007069in}{1.629378in}}{\pgfqpoint{2.007069in}{1.637614in}}%
\pgfpathcurveto{\pgfqpoint{2.007069in}{1.645850in}}{\pgfqpoint{2.003797in}{1.653750in}}{\pgfqpoint{1.997973in}{1.659574in}}%
\pgfpathcurveto{\pgfqpoint{1.992149in}{1.665398in}}{\pgfqpoint{1.984249in}{1.668670in}}{\pgfqpoint{1.976013in}{1.668670in}}%
\pgfpathcurveto{\pgfqpoint{1.967777in}{1.668670in}}{\pgfqpoint{1.959877in}{1.665398in}}{\pgfqpoint{1.954053in}{1.659574in}}%
\pgfpathcurveto{\pgfqpoint{1.948229in}{1.653750in}}{\pgfqpoint{1.944956in}{1.645850in}}{\pgfqpoint{1.944956in}{1.637614in}}%
\pgfpathcurveto{\pgfqpoint{1.944956in}{1.629378in}}{\pgfqpoint{1.948229in}{1.621478in}}{\pgfqpoint{1.954053in}{1.615654in}}%
\pgfpathcurveto{\pgfqpoint{1.959877in}{1.609830in}}{\pgfqpoint{1.967777in}{1.606557in}}{\pgfqpoint{1.976013in}{1.606557in}}%
\pgfpathclose%
\pgfusepath{stroke,fill}%
\end{pgfscope}%
\begin{pgfscope}%
\pgfpathrectangle{\pgfqpoint{0.100000in}{0.212622in}}{\pgfqpoint{3.696000in}{3.696000in}}%
\pgfusepath{clip}%
\pgfsetbuttcap%
\pgfsetroundjoin%
\definecolor{currentfill}{rgb}{0.121569,0.466667,0.705882}%
\pgfsetfillcolor{currentfill}%
\pgfsetfillopacity{0.955680}%
\pgfsetlinewidth{1.003750pt}%
\definecolor{currentstroke}{rgb}{0.121569,0.466667,0.705882}%
\pgfsetstrokecolor{currentstroke}%
\pgfsetstrokeopacity{0.955680}%
\pgfsetdash{}{0pt}%
\pgfpathmoveto{\pgfqpoint{1.994301in}{1.593684in}}%
\pgfpathcurveto{\pgfqpoint{2.002538in}{1.593684in}}{\pgfqpoint{2.010438in}{1.596956in}}{\pgfqpoint{2.016262in}{1.602780in}}%
\pgfpathcurveto{\pgfqpoint{2.022085in}{1.608604in}}{\pgfqpoint{2.025358in}{1.616504in}}{\pgfqpoint{2.025358in}{1.624740in}}%
\pgfpathcurveto{\pgfqpoint{2.025358in}{1.632976in}}{\pgfqpoint{2.022085in}{1.640876in}}{\pgfqpoint{2.016262in}{1.646700in}}%
\pgfpathcurveto{\pgfqpoint{2.010438in}{1.652524in}}{\pgfqpoint{2.002538in}{1.655797in}}{\pgfqpoint{1.994301in}{1.655797in}}%
\pgfpathcurveto{\pgfqpoint{1.986065in}{1.655797in}}{\pgfqpoint{1.978165in}{1.652524in}}{\pgfqpoint{1.972341in}{1.646700in}}%
\pgfpathcurveto{\pgfqpoint{1.966517in}{1.640876in}}{\pgfqpoint{1.963245in}{1.632976in}}{\pgfqpoint{1.963245in}{1.624740in}}%
\pgfpathcurveto{\pgfqpoint{1.963245in}{1.616504in}}{\pgfqpoint{1.966517in}{1.608604in}}{\pgfqpoint{1.972341in}{1.602780in}}%
\pgfpathcurveto{\pgfqpoint{1.978165in}{1.596956in}}{\pgfqpoint{1.986065in}{1.593684in}}{\pgfqpoint{1.994301in}{1.593684in}}%
\pgfpathclose%
\pgfusepath{stroke,fill}%
\end{pgfscope}%
\begin{pgfscope}%
\pgfpathrectangle{\pgfqpoint{0.100000in}{0.212622in}}{\pgfqpoint{3.696000in}{3.696000in}}%
\pgfusepath{clip}%
\pgfsetbuttcap%
\pgfsetroundjoin%
\definecolor{currentfill}{rgb}{0.121569,0.466667,0.705882}%
\pgfsetfillcolor{currentfill}%
\pgfsetfillopacity{0.957052}%
\pgfsetlinewidth{1.003750pt}%
\definecolor{currentstroke}{rgb}{0.121569,0.466667,0.705882}%
\pgfsetstrokecolor{currentstroke}%
\pgfsetstrokeopacity{0.957052}%
\pgfsetdash{}{0pt}%
\pgfpathmoveto{\pgfqpoint{2.010186in}{1.582270in}}%
\pgfpathcurveto{\pgfqpoint{2.018422in}{1.582270in}}{\pgfqpoint{2.026322in}{1.585542in}}{\pgfqpoint{2.032146in}{1.591366in}}%
\pgfpathcurveto{\pgfqpoint{2.037970in}{1.597190in}}{\pgfqpoint{2.041242in}{1.605090in}}{\pgfqpoint{2.041242in}{1.613327in}}%
\pgfpathcurveto{\pgfqpoint{2.041242in}{1.621563in}}{\pgfqpoint{2.037970in}{1.629463in}}{\pgfqpoint{2.032146in}{1.635287in}}%
\pgfpathcurveto{\pgfqpoint{2.026322in}{1.641111in}}{\pgfqpoint{2.018422in}{1.644383in}}{\pgfqpoint{2.010186in}{1.644383in}}%
\pgfpathcurveto{\pgfqpoint{2.001949in}{1.644383in}}{\pgfqpoint{1.994049in}{1.641111in}}{\pgfqpoint{1.988225in}{1.635287in}}%
\pgfpathcurveto{\pgfqpoint{1.982402in}{1.629463in}}{\pgfqpoint{1.979129in}{1.621563in}}{\pgfqpoint{1.979129in}{1.613327in}}%
\pgfpathcurveto{\pgfqpoint{1.979129in}{1.605090in}}{\pgfqpoint{1.982402in}{1.597190in}}{\pgfqpoint{1.988225in}{1.591366in}}%
\pgfpathcurveto{\pgfqpoint{1.994049in}{1.585542in}}{\pgfqpoint{2.001949in}{1.582270in}}{\pgfqpoint{2.010186in}{1.582270in}}%
\pgfpathclose%
\pgfusepath{stroke,fill}%
\end{pgfscope}%
\begin{pgfscope}%
\pgfpathrectangle{\pgfqpoint{0.100000in}{0.212622in}}{\pgfqpoint{3.696000in}{3.696000in}}%
\pgfusepath{clip}%
\pgfsetbuttcap%
\pgfsetroundjoin%
\definecolor{currentfill}{rgb}{0.121569,0.466667,0.705882}%
\pgfsetfillcolor{currentfill}%
\pgfsetfillopacity{0.958204}%
\pgfsetlinewidth{1.003750pt}%
\definecolor{currentstroke}{rgb}{0.121569,0.466667,0.705882}%
\pgfsetstrokecolor{currentstroke}%
\pgfsetstrokeopacity{0.958204}%
\pgfsetdash{}{0pt}%
\pgfpathmoveto{\pgfqpoint{2.023278in}{1.570451in}}%
\pgfpathcurveto{\pgfqpoint{2.031515in}{1.570451in}}{\pgfqpoint{2.039415in}{1.573723in}}{\pgfqpoint{2.045239in}{1.579547in}}%
\pgfpathcurveto{\pgfqpoint{2.051062in}{1.585371in}}{\pgfqpoint{2.054335in}{1.593271in}}{\pgfqpoint{2.054335in}{1.601507in}}%
\pgfpathcurveto{\pgfqpoint{2.054335in}{1.609744in}}{\pgfqpoint{2.051062in}{1.617644in}}{\pgfqpoint{2.045239in}{1.623468in}}%
\pgfpathcurveto{\pgfqpoint{2.039415in}{1.629292in}}{\pgfqpoint{2.031515in}{1.632564in}}{\pgfqpoint{2.023278in}{1.632564in}}%
\pgfpathcurveto{\pgfqpoint{2.015042in}{1.632564in}}{\pgfqpoint{2.007142in}{1.629292in}}{\pgfqpoint{2.001318in}{1.623468in}}%
\pgfpathcurveto{\pgfqpoint{1.995494in}{1.617644in}}{\pgfqpoint{1.992222in}{1.609744in}}{\pgfqpoint{1.992222in}{1.601507in}}%
\pgfpathcurveto{\pgfqpoint{1.992222in}{1.593271in}}{\pgfqpoint{1.995494in}{1.585371in}}{\pgfqpoint{2.001318in}{1.579547in}}%
\pgfpathcurveto{\pgfqpoint{2.007142in}{1.573723in}}{\pgfqpoint{2.015042in}{1.570451in}}{\pgfqpoint{2.023278in}{1.570451in}}%
\pgfpathclose%
\pgfusepath{stroke,fill}%
\end{pgfscope}%
\begin{pgfscope}%
\pgfpathrectangle{\pgfqpoint{0.100000in}{0.212622in}}{\pgfqpoint{3.696000in}{3.696000in}}%
\pgfusepath{clip}%
\pgfsetbuttcap%
\pgfsetroundjoin%
\definecolor{currentfill}{rgb}{0.121569,0.466667,0.705882}%
\pgfsetfillcolor{currentfill}%
\pgfsetfillopacity{0.959318}%
\pgfsetlinewidth{1.003750pt}%
\definecolor{currentstroke}{rgb}{0.121569,0.466667,0.705882}%
\pgfsetstrokecolor{currentstroke}%
\pgfsetstrokeopacity{0.959318}%
\pgfsetdash{}{0pt}%
\pgfpathmoveto{\pgfqpoint{2.036184in}{1.561120in}}%
\pgfpathcurveto{\pgfqpoint{2.044420in}{1.561120in}}{\pgfqpoint{2.052320in}{1.564392in}}{\pgfqpoint{2.058144in}{1.570216in}}%
\pgfpathcurveto{\pgfqpoint{2.063968in}{1.576040in}}{\pgfqpoint{2.067240in}{1.583940in}}{\pgfqpoint{2.067240in}{1.592176in}}%
\pgfpathcurveto{\pgfqpoint{2.067240in}{1.600413in}}{\pgfqpoint{2.063968in}{1.608313in}}{\pgfqpoint{2.058144in}{1.614137in}}%
\pgfpathcurveto{\pgfqpoint{2.052320in}{1.619961in}}{\pgfqpoint{2.044420in}{1.623233in}}{\pgfqpoint{2.036184in}{1.623233in}}%
\pgfpathcurveto{\pgfqpoint{2.027948in}{1.623233in}}{\pgfqpoint{2.020048in}{1.619961in}}{\pgfqpoint{2.014224in}{1.614137in}}%
\pgfpathcurveto{\pgfqpoint{2.008400in}{1.608313in}}{\pgfqpoint{2.005127in}{1.600413in}}{\pgfqpoint{2.005127in}{1.592176in}}%
\pgfpathcurveto{\pgfqpoint{2.005127in}{1.583940in}}{\pgfqpoint{2.008400in}{1.576040in}}{\pgfqpoint{2.014224in}{1.570216in}}%
\pgfpathcurveto{\pgfqpoint{2.020048in}{1.564392in}}{\pgfqpoint{2.027948in}{1.561120in}}{\pgfqpoint{2.036184in}{1.561120in}}%
\pgfpathclose%
\pgfusepath{stroke,fill}%
\end{pgfscope}%
\begin{pgfscope}%
\pgfpathrectangle{\pgfqpoint{0.100000in}{0.212622in}}{\pgfqpoint{3.696000in}{3.696000in}}%
\pgfusepath{clip}%
\pgfsetbuttcap%
\pgfsetroundjoin%
\definecolor{currentfill}{rgb}{0.121569,0.466667,0.705882}%
\pgfsetfillcolor{currentfill}%
\pgfsetfillopacity{0.959796}%
\pgfsetlinewidth{1.003750pt}%
\definecolor{currentstroke}{rgb}{0.121569,0.466667,0.705882}%
\pgfsetstrokecolor{currentstroke}%
\pgfsetstrokeopacity{0.959796}%
\pgfsetdash{}{0pt}%
\pgfpathmoveto{\pgfqpoint{2.381758in}{1.378398in}}%
\pgfpathcurveto{\pgfqpoint{2.389994in}{1.378398in}}{\pgfqpoint{2.397894in}{1.381671in}}{\pgfqpoint{2.403718in}{1.387494in}}%
\pgfpathcurveto{\pgfqpoint{2.409542in}{1.393318in}}{\pgfqpoint{2.412814in}{1.401218in}}{\pgfqpoint{2.412814in}{1.409455in}}%
\pgfpathcurveto{\pgfqpoint{2.412814in}{1.417691in}}{\pgfqpoint{2.409542in}{1.425591in}}{\pgfqpoint{2.403718in}{1.431415in}}%
\pgfpathcurveto{\pgfqpoint{2.397894in}{1.437239in}}{\pgfqpoint{2.389994in}{1.440511in}}{\pgfqpoint{2.381758in}{1.440511in}}%
\pgfpathcurveto{\pgfqpoint{2.373521in}{1.440511in}}{\pgfqpoint{2.365621in}{1.437239in}}{\pgfqpoint{2.359797in}{1.431415in}}%
\pgfpathcurveto{\pgfqpoint{2.353973in}{1.425591in}}{\pgfqpoint{2.350701in}{1.417691in}}{\pgfqpoint{2.350701in}{1.409455in}}%
\pgfpathcurveto{\pgfqpoint{2.350701in}{1.401218in}}{\pgfqpoint{2.353973in}{1.393318in}}{\pgfqpoint{2.359797in}{1.387494in}}%
\pgfpathcurveto{\pgfqpoint{2.365621in}{1.381671in}}{\pgfqpoint{2.373521in}{1.378398in}}{\pgfqpoint{2.381758in}{1.378398in}}%
\pgfpathclose%
\pgfusepath{stroke,fill}%
\end{pgfscope}%
\begin{pgfscope}%
\pgfpathrectangle{\pgfqpoint{0.100000in}{0.212622in}}{\pgfqpoint{3.696000in}{3.696000in}}%
\pgfusepath{clip}%
\pgfsetbuttcap%
\pgfsetroundjoin%
\definecolor{currentfill}{rgb}{0.121569,0.466667,0.705882}%
\pgfsetfillcolor{currentfill}%
\pgfsetfillopacity{0.960520}%
\pgfsetlinewidth{1.003750pt}%
\definecolor{currentstroke}{rgb}{0.121569,0.466667,0.705882}%
\pgfsetstrokecolor{currentstroke}%
\pgfsetstrokeopacity{0.960520}%
\pgfsetdash{}{0pt}%
\pgfpathmoveto{\pgfqpoint{2.046074in}{1.555341in}}%
\pgfpathcurveto{\pgfqpoint{2.054310in}{1.555341in}}{\pgfqpoint{2.062210in}{1.558613in}}{\pgfqpoint{2.068034in}{1.564437in}}%
\pgfpathcurveto{\pgfqpoint{2.073858in}{1.570261in}}{\pgfqpoint{2.077130in}{1.578161in}}{\pgfqpoint{2.077130in}{1.586398in}}%
\pgfpathcurveto{\pgfqpoint{2.077130in}{1.594634in}}{\pgfqpoint{2.073858in}{1.602534in}}{\pgfqpoint{2.068034in}{1.608358in}}%
\pgfpathcurveto{\pgfqpoint{2.062210in}{1.614182in}}{\pgfqpoint{2.054310in}{1.617454in}}{\pgfqpoint{2.046074in}{1.617454in}}%
\pgfpathcurveto{\pgfqpoint{2.037837in}{1.617454in}}{\pgfqpoint{2.029937in}{1.614182in}}{\pgfqpoint{2.024113in}{1.608358in}}%
\pgfpathcurveto{\pgfqpoint{2.018289in}{1.602534in}}{\pgfqpoint{2.015017in}{1.594634in}}{\pgfqpoint{2.015017in}{1.586398in}}%
\pgfpathcurveto{\pgfqpoint{2.015017in}{1.578161in}}{\pgfqpoint{2.018289in}{1.570261in}}{\pgfqpoint{2.024113in}{1.564437in}}%
\pgfpathcurveto{\pgfqpoint{2.029937in}{1.558613in}}{\pgfqpoint{2.037837in}{1.555341in}}{\pgfqpoint{2.046074in}{1.555341in}}%
\pgfpathclose%
\pgfusepath{stroke,fill}%
\end{pgfscope}%
\begin{pgfscope}%
\pgfpathrectangle{\pgfqpoint{0.100000in}{0.212622in}}{\pgfqpoint{3.696000in}{3.696000in}}%
\pgfusepath{clip}%
\pgfsetbuttcap%
\pgfsetroundjoin%
\definecolor{currentfill}{rgb}{0.121569,0.466667,0.705882}%
\pgfsetfillcolor{currentfill}%
\pgfsetfillopacity{0.961180}%
\pgfsetlinewidth{1.003750pt}%
\definecolor{currentstroke}{rgb}{0.121569,0.466667,0.705882}%
\pgfsetstrokecolor{currentstroke}%
\pgfsetstrokeopacity{0.961180}%
\pgfsetdash{}{0pt}%
\pgfpathmoveto{\pgfqpoint{2.054763in}{1.548152in}}%
\pgfpathcurveto{\pgfqpoint{2.062999in}{1.548152in}}{\pgfqpoint{2.070899in}{1.551425in}}{\pgfqpoint{2.076723in}{1.557248in}}%
\pgfpathcurveto{\pgfqpoint{2.082547in}{1.563072in}}{\pgfqpoint{2.085819in}{1.570972in}}{\pgfqpoint{2.085819in}{1.579209in}}%
\pgfpathcurveto{\pgfqpoint{2.085819in}{1.587445in}}{\pgfqpoint{2.082547in}{1.595345in}}{\pgfqpoint{2.076723in}{1.601169in}}%
\pgfpathcurveto{\pgfqpoint{2.070899in}{1.606993in}}{\pgfqpoint{2.062999in}{1.610265in}}{\pgfqpoint{2.054763in}{1.610265in}}%
\pgfpathcurveto{\pgfqpoint{2.046526in}{1.610265in}}{\pgfqpoint{2.038626in}{1.606993in}}{\pgfqpoint{2.032803in}{1.601169in}}%
\pgfpathcurveto{\pgfqpoint{2.026979in}{1.595345in}}{\pgfqpoint{2.023706in}{1.587445in}}{\pgfqpoint{2.023706in}{1.579209in}}%
\pgfpathcurveto{\pgfqpoint{2.023706in}{1.570972in}}{\pgfqpoint{2.026979in}{1.563072in}}{\pgfqpoint{2.032803in}{1.557248in}}%
\pgfpathcurveto{\pgfqpoint{2.038626in}{1.551425in}}{\pgfqpoint{2.046526in}{1.548152in}}{\pgfqpoint{2.054763in}{1.548152in}}%
\pgfpathclose%
\pgfusepath{stroke,fill}%
\end{pgfscope}%
\begin{pgfscope}%
\pgfpathrectangle{\pgfqpoint{0.100000in}{0.212622in}}{\pgfqpoint{3.696000in}{3.696000in}}%
\pgfusepath{clip}%
\pgfsetbuttcap%
\pgfsetroundjoin%
\definecolor{currentfill}{rgb}{0.121569,0.466667,0.705882}%
\pgfsetfillcolor{currentfill}%
\pgfsetfillopacity{0.961814}%
\pgfsetlinewidth{1.003750pt}%
\definecolor{currentstroke}{rgb}{0.121569,0.466667,0.705882}%
\pgfsetstrokecolor{currentstroke}%
\pgfsetstrokeopacity{0.961814}%
\pgfsetdash{}{0pt}%
\pgfpathmoveto{\pgfqpoint{2.061253in}{1.543721in}}%
\pgfpathcurveto{\pgfqpoint{2.069489in}{1.543721in}}{\pgfqpoint{2.077389in}{1.546994in}}{\pgfqpoint{2.083213in}{1.552817in}}%
\pgfpathcurveto{\pgfqpoint{2.089037in}{1.558641in}}{\pgfqpoint{2.092309in}{1.566541in}}{\pgfqpoint{2.092309in}{1.574778in}}%
\pgfpathcurveto{\pgfqpoint{2.092309in}{1.583014in}}{\pgfqpoint{2.089037in}{1.590914in}}{\pgfqpoint{2.083213in}{1.596738in}}%
\pgfpathcurveto{\pgfqpoint{2.077389in}{1.602562in}}{\pgfqpoint{2.069489in}{1.605834in}}{\pgfqpoint{2.061253in}{1.605834in}}%
\pgfpathcurveto{\pgfqpoint{2.053016in}{1.605834in}}{\pgfqpoint{2.045116in}{1.602562in}}{\pgfqpoint{2.039292in}{1.596738in}}%
\pgfpathcurveto{\pgfqpoint{2.033468in}{1.590914in}}{\pgfqpoint{2.030196in}{1.583014in}}{\pgfqpoint{2.030196in}{1.574778in}}%
\pgfpathcurveto{\pgfqpoint{2.030196in}{1.566541in}}{\pgfqpoint{2.033468in}{1.558641in}}{\pgfqpoint{2.039292in}{1.552817in}}%
\pgfpathcurveto{\pgfqpoint{2.045116in}{1.546994in}}{\pgfqpoint{2.053016in}{1.543721in}}{\pgfqpoint{2.061253in}{1.543721in}}%
\pgfpathclose%
\pgfusepath{stroke,fill}%
\end{pgfscope}%
\begin{pgfscope}%
\pgfpathrectangle{\pgfqpoint{0.100000in}{0.212622in}}{\pgfqpoint{3.696000in}{3.696000in}}%
\pgfusepath{clip}%
\pgfsetbuttcap%
\pgfsetroundjoin%
\definecolor{currentfill}{rgb}{0.121569,0.466667,0.705882}%
\pgfsetfillcolor{currentfill}%
\pgfsetfillopacity{0.962737}%
\pgfsetlinewidth{1.003750pt}%
\definecolor{currentstroke}{rgb}{0.121569,0.466667,0.705882}%
\pgfsetstrokecolor{currentstroke}%
\pgfsetstrokeopacity{0.962737}%
\pgfsetdash{}{0pt}%
\pgfpathmoveto{\pgfqpoint{2.384587in}{1.370954in}}%
\pgfpathcurveto{\pgfqpoint{2.392824in}{1.370954in}}{\pgfqpoint{2.400724in}{1.374226in}}{\pgfqpoint{2.406548in}{1.380050in}}%
\pgfpathcurveto{\pgfqpoint{2.412371in}{1.385874in}}{\pgfqpoint{2.415644in}{1.393774in}}{\pgfqpoint{2.415644in}{1.402011in}}%
\pgfpathcurveto{\pgfqpoint{2.415644in}{1.410247in}}{\pgfqpoint{2.412371in}{1.418147in}}{\pgfqpoint{2.406548in}{1.423971in}}%
\pgfpathcurveto{\pgfqpoint{2.400724in}{1.429795in}}{\pgfqpoint{2.392824in}{1.433067in}}{\pgfqpoint{2.384587in}{1.433067in}}%
\pgfpathcurveto{\pgfqpoint{2.376351in}{1.433067in}}{\pgfqpoint{2.368451in}{1.429795in}}{\pgfqpoint{2.362627in}{1.423971in}}%
\pgfpathcurveto{\pgfqpoint{2.356803in}{1.418147in}}{\pgfqpoint{2.353531in}{1.410247in}}{\pgfqpoint{2.353531in}{1.402011in}}%
\pgfpathcurveto{\pgfqpoint{2.353531in}{1.393774in}}{\pgfqpoint{2.356803in}{1.385874in}}{\pgfqpoint{2.362627in}{1.380050in}}%
\pgfpathcurveto{\pgfqpoint{2.368451in}{1.374226in}}{\pgfqpoint{2.376351in}{1.370954in}}{\pgfqpoint{2.384587in}{1.370954in}}%
\pgfpathclose%
\pgfusepath{stroke,fill}%
\end{pgfscope}%
\begin{pgfscope}%
\pgfpathrectangle{\pgfqpoint{0.100000in}{0.212622in}}{\pgfqpoint{3.696000in}{3.696000in}}%
\pgfusepath{clip}%
\pgfsetbuttcap%
\pgfsetroundjoin%
\definecolor{currentfill}{rgb}{0.121569,0.466667,0.705882}%
\pgfsetfillcolor{currentfill}%
\pgfsetfillopacity{0.963377}%
\pgfsetlinewidth{1.003750pt}%
\definecolor{currentstroke}{rgb}{0.121569,0.466667,0.705882}%
\pgfsetstrokecolor{currentstroke}%
\pgfsetstrokeopacity{0.963377}%
\pgfsetdash{}{0pt}%
\pgfpathmoveto{\pgfqpoint{2.073719in}{1.540424in}}%
\pgfpathcurveto{\pgfqpoint{2.081955in}{1.540424in}}{\pgfqpoint{2.089855in}{1.543697in}}{\pgfqpoint{2.095679in}{1.549521in}}%
\pgfpathcurveto{\pgfqpoint{2.101503in}{1.555345in}}{\pgfqpoint{2.104776in}{1.563245in}}{\pgfqpoint{2.104776in}{1.571481in}}%
\pgfpathcurveto{\pgfqpoint{2.104776in}{1.579717in}}{\pgfqpoint{2.101503in}{1.587617in}}{\pgfqpoint{2.095679in}{1.593441in}}%
\pgfpathcurveto{\pgfqpoint{2.089855in}{1.599265in}}{\pgfqpoint{2.081955in}{1.602537in}}{\pgfqpoint{2.073719in}{1.602537in}}%
\pgfpathcurveto{\pgfqpoint{2.065483in}{1.602537in}}{\pgfqpoint{2.057583in}{1.599265in}}{\pgfqpoint{2.051759in}{1.593441in}}%
\pgfpathcurveto{\pgfqpoint{2.045935in}{1.587617in}}{\pgfqpoint{2.042663in}{1.579717in}}{\pgfqpoint{2.042663in}{1.571481in}}%
\pgfpathcurveto{\pgfqpoint{2.042663in}{1.563245in}}{\pgfqpoint{2.045935in}{1.555345in}}{\pgfqpoint{2.051759in}{1.549521in}}%
\pgfpathcurveto{\pgfqpoint{2.057583in}{1.543697in}}{\pgfqpoint{2.065483in}{1.540424in}}{\pgfqpoint{2.073719in}{1.540424in}}%
\pgfpathclose%
\pgfusepath{stroke,fill}%
\end{pgfscope}%
\begin{pgfscope}%
\pgfpathrectangle{\pgfqpoint{0.100000in}{0.212622in}}{\pgfqpoint{3.696000in}{3.696000in}}%
\pgfusepath{clip}%
\pgfsetbuttcap%
\pgfsetroundjoin%
\definecolor{currentfill}{rgb}{0.121569,0.466667,0.705882}%
\pgfsetfillcolor{currentfill}%
\pgfsetfillopacity{0.964038}%
\pgfsetlinewidth{1.003750pt}%
\definecolor{currentstroke}{rgb}{0.121569,0.466667,0.705882}%
\pgfsetstrokecolor{currentstroke}%
\pgfsetstrokeopacity{0.964038}%
\pgfsetdash{}{0pt}%
\pgfpathmoveto{\pgfqpoint{2.081739in}{1.534080in}}%
\pgfpathcurveto{\pgfqpoint{2.089975in}{1.534080in}}{\pgfqpoint{2.097875in}{1.537353in}}{\pgfqpoint{2.103699in}{1.543176in}}%
\pgfpathcurveto{\pgfqpoint{2.109523in}{1.549000in}}{\pgfqpoint{2.112795in}{1.556900in}}{\pgfqpoint{2.112795in}{1.565137in}}%
\pgfpathcurveto{\pgfqpoint{2.112795in}{1.573373in}}{\pgfqpoint{2.109523in}{1.581273in}}{\pgfqpoint{2.103699in}{1.587097in}}%
\pgfpathcurveto{\pgfqpoint{2.097875in}{1.592921in}}{\pgfqpoint{2.089975in}{1.596193in}}{\pgfqpoint{2.081739in}{1.596193in}}%
\pgfpathcurveto{\pgfqpoint{2.073502in}{1.596193in}}{\pgfqpoint{2.065602in}{1.592921in}}{\pgfqpoint{2.059778in}{1.587097in}}%
\pgfpathcurveto{\pgfqpoint{2.053954in}{1.581273in}}{\pgfqpoint{2.050682in}{1.573373in}}{\pgfqpoint{2.050682in}{1.565137in}}%
\pgfpathcurveto{\pgfqpoint{2.050682in}{1.556900in}}{\pgfqpoint{2.053954in}{1.549000in}}{\pgfqpoint{2.059778in}{1.543176in}}%
\pgfpathcurveto{\pgfqpoint{2.065602in}{1.537353in}}{\pgfqpoint{2.073502in}{1.534080in}}{\pgfqpoint{2.081739in}{1.534080in}}%
\pgfpathclose%
\pgfusepath{stroke,fill}%
\end{pgfscope}%
\begin{pgfscope}%
\pgfpathrectangle{\pgfqpoint{0.100000in}{0.212622in}}{\pgfqpoint{3.696000in}{3.696000in}}%
\pgfusepath{clip}%
\pgfsetbuttcap%
\pgfsetroundjoin%
\definecolor{currentfill}{rgb}{0.121569,0.466667,0.705882}%
\pgfsetfillcolor{currentfill}%
\pgfsetfillopacity{0.965910}%
\pgfsetlinewidth{1.003750pt}%
\definecolor{currentstroke}{rgb}{0.121569,0.466667,0.705882}%
\pgfsetstrokecolor{currentstroke}%
\pgfsetstrokeopacity{0.965910}%
\pgfsetdash{}{0pt}%
\pgfpathmoveto{\pgfqpoint{2.096012in}{1.525611in}}%
\pgfpathcurveto{\pgfqpoint{2.104249in}{1.525611in}}{\pgfqpoint{2.112149in}{1.528884in}}{\pgfqpoint{2.117973in}{1.534708in}}%
\pgfpathcurveto{\pgfqpoint{2.123797in}{1.540532in}}{\pgfqpoint{2.127069in}{1.548432in}}{\pgfqpoint{2.127069in}{1.556668in}}%
\pgfpathcurveto{\pgfqpoint{2.127069in}{1.564904in}}{\pgfqpoint{2.123797in}{1.572804in}}{\pgfqpoint{2.117973in}{1.578628in}}%
\pgfpathcurveto{\pgfqpoint{2.112149in}{1.584452in}}{\pgfqpoint{2.104249in}{1.587724in}}{\pgfqpoint{2.096012in}{1.587724in}}%
\pgfpathcurveto{\pgfqpoint{2.087776in}{1.587724in}}{\pgfqpoint{2.079876in}{1.584452in}}{\pgfqpoint{2.074052in}{1.578628in}}%
\pgfpathcurveto{\pgfqpoint{2.068228in}{1.572804in}}{\pgfqpoint{2.064956in}{1.564904in}}{\pgfqpoint{2.064956in}{1.556668in}}%
\pgfpathcurveto{\pgfqpoint{2.064956in}{1.548432in}}{\pgfqpoint{2.068228in}{1.540532in}}{\pgfqpoint{2.074052in}{1.534708in}}%
\pgfpathcurveto{\pgfqpoint{2.079876in}{1.528884in}}{\pgfqpoint{2.087776in}{1.525611in}}{\pgfqpoint{2.096012in}{1.525611in}}%
\pgfpathclose%
\pgfusepath{stroke,fill}%
\end{pgfscope}%
\begin{pgfscope}%
\pgfpathrectangle{\pgfqpoint{0.100000in}{0.212622in}}{\pgfqpoint{3.696000in}{3.696000in}}%
\pgfusepath{clip}%
\pgfsetbuttcap%
\pgfsetroundjoin%
\definecolor{currentfill}{rgb}{0.121569,0.466667,0.705882}%
\pgfsetfillcolor{currentfill}%
\pgfsetfillopacity{0.966086}%
\pgfsetlinewidth{1.003750pt}%
\definecolor{currentstroke}{rgb}{0.121569,0.466667,0.705882}%
\pgfsetstrokecolor{currentstroke}%
\pgfsetstrokeopacity{0.966086}%
\pgfsetdash{}{0pt}%
\pgfpathmoveto{\pgfqpoint{2.387147in}{1.364234in}}%
\pgfpathcurveto{\pgfqpoint{2.395383in}{1.364234in}}{\pgfqpoint{2.403283in}{1.367506in}}{\pgfqpoint{2.409107in}{1.373330in}}%
\pgfpathcurveto{\pgfqpoint{2.414931in}{1.379154in}}{\pgfqpoint{2.418203in}{1.387054in}}{\pgfqpoint{2.418203in}{1.395291in}}%
\pgfpathcurveto{\pgfqpoint{2.418203in}{1.403527in}}{\pgfqpoint{2.414931in}{1.411427in}}{\pgfqpoint{2.409107in}{1.417251in}}%
\pgfpathcurveto{\pgfqpoint{2.403283in}{1.423075in}}{\pgfqpoint{2.395383in}{1.426347in}}{\pgfqpoint{2.387147in}{1.426347in}}%
\pgfpathcurveto{\pgfqpoint{2.378911in}{1.426347in}}{\pgfqpoint{2.371011in}{1.423075in}}{\pgfqpoint{2.365187in}{1.417251in}}%
\pgfpathcurveto{\pgfqpoint{2.359363in}{1.411427in}}{\pgfqpoint{2.356090in}{1.403527in}}{\pgfqpoint{2.356090in}{1.395291in}}%
\pgfpathcurveto{\pgfqpoint{2.356090in}{1.387054in}}{\pgfqpoint{2.359363in}{1.379154in}}{\pgfqpoint{2.365187in}{1.373330in}}%
\pgfpathcurveto{\pgfqpoint{2.371011in}{1.367506in}}{\pgfqpoint{2.378911in}{1.364234in}}{\pgfqpoint{2.387147in}{1.364234in}}%
\pgfpathclose%
\pgfusepath{stroke,fill}%
\end{pgfscope}%
\begin{pgfscope}%
\pgfpathrectangle{\pgfqpoint{0.100000in}{0.212622in}}{\pgfqpoint{3.696000in}{3.696000in}}%
\pgfusepath{clip}%
\pgfsetbuttcap%
\pgfsetroundjoin%
\definecolor{currentfill}{rgb}{0.121569,0.466667,0.705882}%
\pgfsetfillcolor{currentfill}%
\pgfsetfillopacity{0.966666}%
\pgfsetlinewidth{1.003750pt}%
\definecolor{currentstroke}{rgb}{0.121569,0.466667,0.705882}%
\pgfsetstrokecolor{currentstroke}%
\pgfsetstrokeopacity{0.966666}%
\pgfsetdash{}{0pt}%
\pgfpathmoveto{\pgfqpoint{2.107363in}{1.513517in}}%
\pgfpathcurveto{\pgfqpoint{2.115600in}{1.513517in}}{\pgfqpoint{2.123500in}{1.516790in}}{\pgfqpoint{2.129323in}{1.522614in}}%
\pgfpathcurveto{\pgfqpoint{2.135147in}{1.528438in}}{\pgfqpoint{2.138420in}{1.536338in}}{\pgfqpoint{2.138420in}{1.544574in}}%
\pgfpathcurveto{\pgfqpoint{2.138420in}{1.552810in}}{\pgfqpoint{2.135147in}{1.560710in}}{\pgfqpoint{2.129323in}{1.566534in}}%
\pgfpathcurveto{\pgfqpoint{2.123500in}{1.572358in}}{\pgfqpoint{2.115600in}{1.575630in}}{\pgfqpoint{2.107363in}{1.575630in}}%
\pgfpathcurveto{\pgfqpoint{2.099127in}{1.575630in}}{\pgfqpoint{2.091227in}{1.572358in}}{\pgfqpoint{2.085403in}{1.566534in}}%
\pgfpathcurveto{\pgfqpoint{2.079579in}{1.560710in}}{\pgfqpoint{2.076307in}{1.552810in}}{\pgfqpoint{2.076307in}{1.544574in}}%
\pgfpathcurveto{\pgfqpoint{2.076307in}{1.536338in}}{\pgfqpoint{2.079579in}{1.528438in}}{\pgfqpoint{2.085403in}{1.522614in}}%
\pgfpathcurveto{\pgfqpoint{2.091227in}{1.516790in}}{\pgfqpoint{2.099127in}{1.513517in}}{\pgfqpoint{2.107363in}{1.513517in}}%
\pgfpathclose%
\pgfusepath{stroke,fill}%
\end{pgfscope}%
\begin{pgfscope}%
\pgfpathrectangle{\pgfqpoint{0.100000in}{0.212622in}}{\pgfqpoint{3.696000in}{3.696000in}}%
\pgfusepath{clip}%
\pgfsetbuttcap%
\pgfsetroundjoin%
\definecolor{currentfill}{rgb}{0.121569,0.466667,0.705882}%
\pgfsetfillcolor{currentfill}%
\pgfsetfillopacity{0.969160}%
\pgfsetlinewidth{1.003750pt}%
\definecolor{currentstroke}{rgb}{0.121569,0.466667,0.705882}%
\pgfsetstrokecolor{currentstroke}%
\pgfsetstrokeopacity{0.969160}%
\pgfsetdash{}{0pt}%
\pgfpathmoveto{\pgfqpoint{2.128525in}{1.499702in}}%
\pgfpathcurveto{\pgfqpoint{2.136761in}{1.499702in}}{\pgfqpoint{2.144661in}{1.502974in}}{\pgfqpoint{2.150485in}{1.508798in}}%
\pgfpathcurveto{\pgfqpoint{2.156309in}{1.514622in}}{\pgfqpoint{2.159582in}{1.522522in}}{\pgfqpoint{2.159582in}{1.530758in}}%
\pgfpathcurveto{\pgfqpoint{2.159582in}{1.538994in}}{\pgfqpoint{2.156309in}{1.546894in}}{\pgfqpoint{2.150485in}{1.552718in}}%
\pgfpathcurveto{\pgfqpoint{2.144661in}{1.558542in}}{\pgfqpoint{2.136761in}{1.561815in}}{\pgfqpoint{2.128525in}{1.561815in}}%
\pgfpathcurveto{\pgfqpoint{2.120289in}{1.561815in}}{\pgfqpoint{2.112389in}{1.558542in}}{\pgfqpoint{2.106565in}{1.552718in}}%
\pgfpathcurveto{\pgfqpoint{2.100741in}{1.546894in}}{\pgfqpoint{2.097469in}{1.538994in}}{\pgfqpoint{2.097469in}{1.530758in}}%
\pgfpathcurveto{\pgfqpoint{2.097469in}{1.522522in}}{\pgfqpoint{2.100741in}{1.514622in}}{\pgfqpoint{2.106565in}{1.508798in}}%
\pgfpathcurveto{\pgfqpoint{2.112389in}{1.502974in}}{\pgfqpoint{2.120289in}{1.499702in}}{\pgfqpoint{2.128525in}{1.499702in}}%
\pgfpathclose%
\pgfusepath{stroke,fill}%
\end{pgfscope}%
\begin{pgfscope}%
\pgfpathrectangle{\pgfqpoint{0.100000in}{0.212622in}}{\pgfqpoint{3.696000in}{3.696000in}}%
\pgfusepath{clip}%
\pgfsetbuttcap%
\pgfsetroundjoin%
\definecolor{currentfill}{rgb}{0.121569,0.466667,0.705882}%
\pgfsetfillcolor{currentfill}%
\pgfsetfillopacity{0.969969}%
\pgfsetlinewidth{1.003750pt}%
\definecolor{currentstroke}{rgb}{0.121569,0.466667,0.705882}%
\pgfsetstrokecolor{currentstroke}%
\pgfsetstrokeopacity{0.969969}%
\pgfsetdash{}{0pt}%
\pgfpathmoveto{\pgfqpoint{2.390982in}{1.357469in}}%
\pgfpathcurveto{\pgfqpoint{2.399219in}{1.357469in}}{\pgfqpoint{2.407119in}{1.360741in}}{\pgfqpoint{2.412943in}{1.366565in}}%
\pgfpathcurveto{\pgfqpoint{2.418766in}{1.372389in}}{\pgfqpoint{2.422039in}{1.380289in}}{\pgfqpoint{2.422039in}{1.388525in}}%
\pgfpathcurveto{\pgfqpoint{2.422039in}{1.396762in}}{\pgfqpoint{2.418766in}{1.404662in}}{\pgfqpoint{2.412943in}{1.410486in}}%
\pgfpathcurveto{\pgfqpoint{2.407119in}{1.416310in}}{\pgfqpoint{2.399219in}{1.419582in}}{\pgfqpoint{2.390982in}{1.419582in}}%
\pgfpathcurveto{\pgfqpoint{2.382746in}{1.419582in}}{\pgfqpoint{2.374846in}{1.416310in}}{\pgfqpoint{2.369022in}{1.410486in}}%
\pgfpathcurveto{\pgfqpoint{2.363198in}{1.404662in}}{\pgfqpoint{2.359926in}{1.396762in}}{\pgfqpoint{2.359926in}{1.388525in}}%
\pgfpathcurveto{\pgfqpoint{2.359926in}{1.380289in}}{\pgfqpoint{2.363198in}{1.372389in}}{\pgfqpoint{2.369022in}{1.366565in}}%
\pgfpathcurveto{\pgfqpoint{2.374846in}{1.360741in}}{\pgfqpoint{2.382746in}{1.357469in}}{\pgfqpoint{2.390982in}{1.357469in}}%
\pgfpathclose%
\pgfusepath{stroke,fill}%
\end{pgfscope}%
\begin{pgfscope}%
\pgfpathrectangle{\pgfqpoint{0.100000in}{0.212622in}}{\pgfqpoint{3.696000in}{3.696000in}}%
\pgfusepath{clip}%
\pgfsetbuttcap%
\pgfsetroundjoin%
\definecolor{currentfill}{rgb}{0.121569,0.466667,0.705882}%
\pgfsetfillcolor{currentfill}%
\pgfsetfillopacity{0.971428}%
\pgfsetlinewidth{1.003750pt}%
\definecolor{currentstroke}{rgb}{0.121569,0.466667,0.705882}%
\pgfsetstrokecolor{currentstroke}%
\pgfsetstrokeopacity{0.971428}%
\pgfsetdash{}{0pt}%
\pgfpathmoveto{\pgfqpoint{2.145462in}{1.487243in}}%
\pgfpathcurveto{\pgfqpoint{2.153698in}{1.487243in}}{\pgfqpoint{2.161598in}{1.490515in}}{\pgfqpoint{2.167422in}{1.496339in}}%
\pgfpathcurveto{\pgfqpoint{2.173246in}{1.502163in}}{\pgfqpoint{2.176518in}{1.510063in}}{\pgfqpoint{2.176518in}{1.518299in}}%
\pgfpathcurveto{\pgfqpoint{2.176518in}{1.526535in}}{\pgfqpoint{2.173246in}{1.534435in}}{\pgfqpoint{2.167422in}{1.540259in}}%
\pgfpathcurveto{\pgfqpoint{2.161598in}{1.546083in}}{\pgfqpoint{2.153698in}{1.549356in}}{\pgfqpoint{2.145462in}{1.549356in}}%
\pgfpathcurveto{\pgfqpoint{2.137226in}{1.549356in}}{\pgfqpoint{2.129326in}{1.546083in}}{\pgfqpoint{2.123502in}{1.540259in}}%
\pgfpathcurveto{\pgfqpoint{2.117678in}{1.534435in}}{\pgfqpoint{2.114405in}{1.526535in}}{\pgfqpoint{2.114405in}{1.518299in}}%
\pgfpathcurveto{\pgfqpoint{2.114405in}{1.510063in}}{\pgfqpoint{2.117678in}{1.502163in}}{\pgfqpoint{2.123502in}{1.496339in}}%
\pgfpathcurveto{\pgfqpoint{2.129326in}{1.490515in}}{\pgfqpoint{2.137226in}{1.487243in}}{\pgfqpoint{2.145462in}{1.487243in}}%
\pgfpathclose%
\pgfusepath{stroke,fill}%
\end{pgfscope}%
\begin{pgfscope}%
\pgfpathrectangle{\pgfqpoint{0.100000in}{0.212622in}}{\pgfqpoint{3.696000in}{3.696000in}}%
\pgfusepath{clip}%
\pgfsetbuttcap%
\pgfsetroundjoin%
\definecolor{currentfill}{rgb}{0.121569,0.466667,0.705882}%
\pgfsetfillcolor{currentfill}%
\pgfsetfillopacity{0.973360}%
\pgfsetlinewidth{1.003750pt}%
\definecolor{currentstroke}{rgb}{0.121569,0.466667,0.705882}%
\pgfsetstrokecolor{currentstroke}%
\pgfsetstrokeopacity{0.973360}%
\pgfsetdash{}{0pt}%
\pgfpathmoveto{\pgfqpoint{2.162849in}{1.475937in}}%
\pgfpathcurveto{\pgfqpoint{2.171085in}{1.475937in}}{\pgfqpoint{2.178985in}{1.479209in}}{\pgfqpoint{2.184809in}{1.485033in}}%
\pgfpathcurveto{\pgfqpoint{2.190633in}{1.490857in}}{\pgfqpoint{2.193905in}{1.498757in}}{\pgfqpoint{2.193905in}{1.506993in}}%
\pgfpathcurveto{\pgfqpoint{2.193905in}{1.515230in}}{\pgfqpoint{2.190633in}{1.523130in}}{\pgfqpoint{2.184809in}{1.528954in}}%
\pgfpathcurveto{\pgfqpoint{2.178985in}{1.534778in}}{\pgfqpoint{2.171085in}{1.538050in}}{\pgfqpoint{2.162849in}{1.538050in}}%
\pgfpathcurveto{\pgfqpoint{2.154613in}{1.538050in}}{\pgfqpoint{2.146712in}{1.534778in}}{\pgfqpoint{2.140889in}{1.528954in}}%
\pgfpathcurveto{\pgfqpoint{2.135065in}{1.523130in}}{\pgfqpoint{2.131792in}{1.515230in}}{\pgfqpoint{2.131792in}{1.506993in}}%
\pgfpathcurveto{\pgfqpoint{2.131792in}{1.498757in}}{\pgfqpoint{2.135065in}{1.490857in}}{\pgfqpoint{2.140889in}{1.485033in}}%
\pgfpathcurveto{\pgfqpoint{2.146712in}{1.479209in}}{\pgfqpoint{2.154613in}{1.475937in}}{\pgfqpoint{2.162849in}{1.475937in}}%
\pgfpathclose%
\pgfusepath{stroke,fill}%
\end{pgfscope}%
\begin{pgfscope}%
\pgfpathrectangle{\pgfqpoint{0.100000in}{0.212622in}}{\pgfqpoint{3.696000in}{3.696000in}}%
\pgfusepath{clip}%
\pgfsetbuttcap%
\pgfsetroundjoin%
\definecolor{currentfill}{rgb}{0.121569,0.466667,0.705882}%
\pgfsetfillcolor{currentfill}%
\pgfsetfillopacity{0.974571}%
\pgfsetlinewidth{1.003750pt}%
\definecolor{currentstroke}{rgb}{0.121569,0.466667,0.705882}%
\pgfsetstrokecolor{currentstroke}%
\pgfsetstrokeopacity{0.974571}%
\pgfsetdash{}{0pt}%
\pgfpathmoveto{\pgfqpoint{2.393770in}{1.352182in}}%
\pgfpathcurveto{\pgfqpoint{2.402006in}{1.352182in}}{\pgfqpoint{2.409906in}{1.355454in}}{\pgfqpoint{2.415730in}{1.361278in}}%
\pgfpathcurveto{\pgfqpoint{2.421554in}{1.367102in}}{\pgfqpoint{2.424826in}{1.375002in}}{\pgfqpoint{2.424826in}{1.383239in}}%
\pgfpathcurveto{\pgfqpoint{2.424826in}{1.391475in}}{\pgfqpoint{2.421554in}{1.399375in}}{\pgfqpoint{2.415730in}{1.405199in}}%
\pgfpathcurveto{\pgfqpoint{2.409906in}{1.411023in}}{\pgfqpoint{2.402006in}{1.414295in}}{\pgfqpoint{2.393770in}{1.414295in}}%
\pgfpathcurveto{\pgfqpoint{2.385534in}{1.414295in}}{\pgfqpoint{2.377634in}{1.411023in}}{\pgfqpoint{2.371810in}{1.405199in}}%
\pgfpathcurveto{\pgfqpoint{2.365986in}{1.399375in}}{\pgfqpoint{2.362713in}{1.391475in}}{\pgfqpoint{2.362713in}{1.383239in}}%
\pgfpathcurveto{\pgfqpoint{2.362713in}{1.375002in}}{\pgfqpoint{2.365986in}{1.367102in}}{\pgfqpoint{2.371810in}{1.361278in}}%
\pgfpathcurveto{\pgfqpoint{2.377634in}{1.355454in}}{\pgfqpoint{2.385534in}{1.352182in}}{\pgfqpoint{2.393770in}{1.352182in}}%
\pgfpathclose%
\pgfusepath{stroke,fill}%
\end{pgfscope}%
\begin{pgfscope}%
\pgfpathrectangle{\pgfqpoint{0.100000in}{0.212622in}}{\pgfqpoint{3.696000in}{3.696000in}}%
\pgfusepath{clip}%
\pgfsetbuttcap%
\pgfsetroundjoin%
\definecolor{currentfill}{rgb}{0.121569,0.466667,0.705882}%
\pgfsetfillcolor{currentfill}%
\pgfsetfillopacity{0.975599}%
\pgfsetlinewidth{1.003750pt}%
\definecolor{currentstroke}{rgb}{0.121569,0.466667,0.705882}%
\pgfsetstrokecolor{currentstroke}%
\pgfsetstrokeopacity{0.975599}%
\pgfsetdash{}{0pt}%
\pgfpathmoveto{\pgfqpoint{2.178016in}{1.468598in}}%
\pgfpathcurveto{\pgfqpoint{2.186252in}{1.468598in}}{\pgfqpoint{2.194152in}{1.471871in}}{\pgfqpoint{2.199976in}{1.477695in}}%
\pgfpathcurveto{\pgfqpoint{2.205800in}{1.483519in}}{\pgfqpoint{2.209072in}{1.491419in}}{\pgfqpoint{2.209072in}{1.499655in}}%
\pgfpathcurveto{\pgfqpoint{2.209072in}{1.507891in}}{\pgfqpoint{2.205800in}{1.515791in}}{\pgfqpoint{2.199976in}{1.521615in}}%
\pgfpathcurveto{\pgfqpoint{2.194152in}{1.527439in}}{\pgfqpoint{2.186252in}{1.530711in}}{\pgfqpoint{2.178016in}{1.530711in}}%
\pgfpathcurveto{\pgfqpoint{2.169779in}{1.530711in}}{\pgfqpoint{2.161879in}{1.527439in}}{\pgfqpoint{2.156055in}{1.521615in}}%
\pgfpathcurveto{\pgfqpoint{2.150231in}{1.515791in}}{\pgfqpoint{2.146959in}{1.507891in}}{\pgfqpoint{2.146959in}{1.499655in}}%
\pgfpathcurveto{\pgfqpoint{2.146959in}{1.491419in}}{\pgfqpoint{2.150231in}{1.483519in}}{\pgfqpoint{2.156055in}{1.477695in}}%
\pgfpathcurveto{\pgfqpoint{2.161879in}{1.471871in}}{\pgfqpoint{2.169779in}{1.468598in}}{\pgfqpoint{2.178016in}{1.468598in}}%
\pgfpathclose%
\pgfusepath{stroke,fill}%
\end{pgfscope}%
\begin{pgfscope}%
\pgfpathrectangle{\pgfqpoint{0.100000in}{0.212622in}}{\pgfqpoint{3.696000in}{3.696000in}}%
\pgfusepath{clip}%
\pgfsetbuttcap%
\pgfsetroundjoin%
\definecolor{currentfill}{rgb}{0.121569,0.466667,0.705882}%
\pgfsetfillcolor{currentfill}%
\pgfsetfillopacity{0.977403}%
\pgfsetlinewidth{1.003750pt}%
\definecolor{currentstroke}{rgb}{0.121569,0.466667,0.705882}%
\pgfsetstrokecolor{currentstroke}%
\pgfsetstrokeopacity{0.977403}%
\pgfsetdash{}{0pt}%
\pgfpathmoveto{\pgfqpoint{2.192424in}{1.458667in}}%
\pgfpathcurveto{\pgfqpoint{2.200660in}{1.458667in}}{\pgfqpoint{2.208560in}{1.461939in}}{\pgfqpoint{2.214384in}{1.467763in}}%
\pgfpathcurveto{\pgfqpoint{2.220208in}{1.473587in}}{\pgfqpoint{2.223480in}{1.481487in}}{\pgfqpoint{2.223480in}{1.489724in}}%
\pgfpathcurveto{\pgfqpoint{2.223480in}{1.497960in}}{\pgfqpoint{2.220208in}{1.505860in}}{\pgfqpoint{2.214384in}{1.511684in}}%
\pgfpathcurveto{\pgfqpoint{2.208560in}{1.517508in}}{\pgfqpoint{2.200660in}{1.520780in}}{\pgfqpoint{2.192424in}{1.520780in}}%
\pgfpathcurveto{\pgfqpoint{2.184187in}{1.520780in}}{\pgfqpoint{2.176287in}{1.517508in}}{\pgfqpoint{2.170463in}{1.511684in}}%
\pgfpathcurveto{\pgfqpoint{2.164639in}{1.505860in}}{\pgfqpoint{2.161367in}{1.497960in}}{\pgfqpoint{2.161367in}{1.489724in}}%
\pgfpathcurveto{\pgfqpoint{2.161367in}{1.481487in}}{\pgfqpoint{2.164639in}{1.473587in}}{\pgfqpoint{2.170463in}{1.467763in}}%
\pgfpathcurveto{\pgfqpoint{2.176287in}{1.461939in}}{\pgfqpoint{2.184187in}{1.458667in}}{\pgfqpoint{2.192424in}{1.458667in}}%
\pgfpathclose%
\pgfusepath{stroke,fill}%
\end{pgfscope}%
\begin{pgfscope}%
\pgfpathrectangle{\pgfqpoint{0.100000in}{0.212622in}}{\pgfqpoint{3.696000in}{3.696000in}}%
\pgfusepath{clip}%
\pgfsetbuttcap%
\pgfsetroundjoin%
\definecolor{currentfill}{rgb}{0.121569,0.466667,0.705882}%
\pgfsetfillcolor{currentfill}%
\pgfsetfillopacity{0.978744}%
\pgfsetlinewidth{1.003750pt}%
\definecolor{currentstroke}{rgb}{0.121569,0.466667,0.705882}%
\pgfsetstrokecolor{currentstroke}%
\pgfsetstrokeopacity{0.978744}%
\pgfsetdash{}{0pt}%
\pgfpathmoveto{\pgfqpoint{2.204667in}{1.451614in}}%
\pgfpathcurveto{\pgfqpoint{2.212903in}{1.451614in}}{\pgfqpoint{2.220804in}{1.454887in}}{\pgfqpoint{2.226627in}{1.460711in}}%
\pgfpathcurveto{\pgfqpoint{2.232451in}{1.466535in}}{\pgfqpoint{2.235724in}{1.474435in}}{\pgfqpoint{2.235724in}{1.482671in}}%
\pgfpathcurveto{\pgfqpoint{2.235724in}{1.490907in}}{\pgfqpoint{2.232451in}{1.498807in}}{\pgfqpoint{2.226627in}{1.504631in}}%
\pgfpathcurveto{\pgfqpoint{2.220804in}{1.510455in}}{\pgfqpoint{2.212903in}{1.513727in}}{\pgfqpoint{2.204667in}{1.513727in}}%
\pgfpathcurveto{\pgfqpoint{2.196431in}{1.513727in}}{\pgfqpoint{2.188531in}{1.510455in}}{\pgfqpoint{2.182707in}{1.504631in}}%
\pgfpathcurveto{\pgfqpoint{2.176883in}{1.498807in}}{\pgfqpoint{2.173611in}{1.490907in}}{\pgfqpoint{2.173611in}{1.482671in}}%
\pgfpathcurveto{\pgfqpoint{2.173611in}{1.474435in}}{\pgfqpoint{2.176883in}{1.466535in}}{\pgfqpoint{2.182707in}{1.460711in}}%
\pgfpathcurveto{\pgfqpoint{2.188531in}{1.454887in}}{\pgfqpoint{2.196431in}{1.451614in}}{\pgfqpoint{2.204667in}{1.451614in}}%
\pgfpathclose%
\pgfusepath{stroke,fill}%
\end{pgfscope}%
\begin{pgfscope}%
\pgfpathrectangle{\pgfqpoint{0.100000in}{0.212622in}}{\pgfqpoint{3.696000in}{3.696000in}}%
\pgfusepath{clip}%
\pgfsetbuttcap%
\pgfsetroundjoin%
\definecolor{currentfill}{rgb}{0.121569,0.466667,0.705882}%
\pgfsetfillcolor{currentfill}%
\pgfsetfillopacity{0.979203}%
\pgfsetlinewidth{1.003750pt}%
\definecolor{currentstroke}{rgb}{0.121569,0.466667,0.705882}%
\pgfsetstrokecolor{currentstroke}%
\pgfsetstrokeopacity{0.979203}%
\pgfsetdash{}{0pt}%
\pgfpathmoveto{\pgfqpoint{2.396824in}{1.345580in}}%
\pgfpathcurveto{\pgfqpoint{2.405060in}{1.345580in}}{\pgfqpoint{2.412960in}{1.348852in}}{\pgfqpoint{2.418784in}{1.354676in}}%
\pgfpathcurveto{\pgfqpoint{2.424608in}{1.360500in}}{\pgfqpoint{2.427880in}{1.368400in}}{\pgfqpoint{2.427880in}{1.376636in}}%
\pgfpathcurveto{\pgfqpoint{2.427880in}{1.384872in}}{\pgfqpoint{2.424608in}{1.392772in}}{\pgfqpoint{2.418784in}{1.398596in}}%
\pgfpathcurveto{\pgfqpoint{2.412960in}{1.404420in}}{\pgfqpoint{2.405060in}{1.407693in}}{\pgfqpoint{2.396824in}{1.407693in}}%
\pgfpathcurveto{\pgfqpoint{2.388587in}{1.407693in}}{\pgfqpoint{2.380687in}{1.404420in}}{\pgfqpoint{2.374863in}{1.398596in}}%
\pgfpathcurveto{\pgfqpoint{2.369039in}{1.392772in}}{\pgfqpoint{2.365767in}{1.384872in}}{\pgfqpoint{2.365767in}{1.376636in}}%
\pgfpathcurveto{\pgfqpoint{2.365767in}{1.368400in}}{\pgfqpoint{2.369039in}{1.360500in}}{\pgfqpoint{2.374863in}{1.354676in}}%
\pgfpathcurveto{\pgfqpoint{2.380687in}{1.348852in}}{\pgfqpoint{2.388587in}{1.345580in}}{\pgfqpoint{2.396824in}{1.345580in}}%
\pgfpathclose%
\pgfusepath{stroke,fill}%
\end{pgfscope}%
\begin{pgfscope}%
\pgfpathrectangle{\pgfqpoint{0.100000in}{0.212622in}}{\pgfqpoint{3.696000in}{3.696000in}}%
\pgfusepath{clip}%
\pgfsetbuttcap%
\pgfsetroundjoin%
\definecolor{currentfill}{rgb}{0.121569,0.466667,0.705882}%
\pgfsetfillcolor{currentfill}%
\pgfsetfillopacity{0.979867}%
\pgfsetlinewidth{1.003750pt}%
\definecolor{currentstroke}{rgb}{0.121569,0.466667,0.705882}%
\pgfsetstrokecolor{currentstroke}%
\pgfsetstrokeopacity{0.979867}%
\pgfsetdash{}{0pt}%
\pgfpathmoveto{\pgfqpoint{2.214057in}{1.442797in}}%
\pgfpathcurveto{\pgfqpoint{2.222293in}{1.442797in}}{\pgfqpoint{2.230193in}{1.446069in}}{\pgfqpoint{2.236017in}{1.451893in}}%
\pgfpathcurveto{\pgfqpoint{2.241841in}{1.457717in}}{\pgfqpoint{2.245113in}{1.465617in}}{\pgfqpoint{2.245113in}{1.473854in}}%
\pgfpathcurveto{\pgfqpoint{2.245113in}{1.482090in}}{\pgfqpoint{2.241841in}{1.489990in}}{\pgfqpoint{2.236017in}{1.495814in}}%
\pgfpathcurveto{\pgfqpoint{2.230193in}{1.501638in}}{\pgfqpoint{2.222293in}{1.504910in}}{\pgfqpoint{2.214057in}{1.504910in}}%
\pgfpathcurveto{\pgfqpoint{2.205820in}{1.504910in}}{\pgfqpoint{2.197920in}{1.501638in}}{\pgfqpoint{2.192096in}{1.495814in}}%
\pgfpathcurveto{\pgfqpoint{2.186272in}{1.489990in}}{\pgfqpoint{2.183000in}{1.482090in}}{\pgfqpoint{2.183000in}{1.473854in}}%
\pgfpathcurveto{\pgfqpoint{2.183000in}{1.465617in}}{\pgfqpoint{2.186272in}{1.457717in}}{\pgfqpoint{2.192096in}{1.451893in}}%
\pgfpathcurveto{\pgfqpoint{2.197920in}{1.446069in}}{\pgfqpoint{2.205820in}{1.442797in}}{\pgfqpoint{2.214057in}{1.442797in}}%
\pgfpathclose%
\pgfusepath{stroke,fill}%
\end{pgfscope}%
\begin{pgfscope}%
\pgfpathrectangle{\pgfqpoint{0.100000in}{0.212622in}}{\pgfqpoint{3.696000in}{3.696000in}}%
\pgfusepath{clip}%
\pgfsetbuttcap%
\pgfsetroundjoin%
\definecolor{currentfill}{rgb}{0.121569,0.466667,0.705882}%
\pgfsetfillcolor{currentfill}%
\pgfsetfillopacity{0.981867}%
\pgfsetlinewidth{1.003750pt}%
\definecolor{currentstroke}{rgb}{0.121569,0.466667,0.705882}%
\pgfsetstrokecolor{currentstroke}%
\pgfsetstrokeopacity{0.981867}%
\pgfsetdash{}{0pt}%
\pgfpathmoveto{\pgfqpoint{2.231541in}{1.427555in}}%
\pgfpathcurveto{\pgfqpoint{2.239778in}{1.427555in}}{\pgfqpoint{2.247678in}{1.430827in}}{\pgfqpoint{2.253502in}{1.436651in}}%
\pgfpathcurveto{\pgfqpoint{2.259326in}{1.442475in}}{\pgfqpoint{2.262598in}{1.450375in}}{\pgfqpoint{2.262598in}{1.458612in}}%
\pgfpathcurveto{\pgfqpoint{2.262598in}{1.466848in}}{\pgfqpoint{2.259326in}{1.474748in}}{\pgfqpoint{2.253502in}{1.480572in}}%
\pgfpathcurveto{\pgfqpoint{2.247678in}{1.486396in}}{\pgfqpoint{2.239778in}{1.489668in}}{\pgfqpoint{2.231541in}{1.489668in}}%
\pgfpathcurveto{\pgfqpoint{2.223305in}{1.489668in}}{\pgfqpoint{2.215405in}{1.486396in}}{\pgfqpoint{2.209581in}{1.480572in}}%
\pgfpathcurveto{\pgfqpoint{2.203757in}{1.474748in}}{\pgfqpoint{2.200485in}{1.466848in}}{\pgfqpoint{2.200485in}{1.458612in}}%
\pgfpathcurveto{\pgfqpoint{2.200485in}{1.450375in}}{\pgfqpoint{2.203757in}{1.442475in}}{\pgfqpoint{2.209581in}{1.436651in}}%
\pgfpathcurveto{\pgfqpoint{2.215405in}{1.430827in}}{\pgfqpoint{2.223305in}{1.427555in}}{\pgfqpoint{2.231541in}{1.427555in}}%
\pgfpathclose%
\pgfusepath{stroke,fill}%
\end{pgfscope}%
\begin{pgfscope}%
\pgfpathrectangle{\pgfqpoint{0.100000in}{0.212622in}}{\pgfqpoint{3.696000in}{3.696000in}}%
\pgfusepath{clip}%
\pgfsetbuttcap%
\pgfsetroundjoin%
\definecolor{currentfill}{rgb}{0.121569,0.466667,0.705882}%
\pgfsetfillcolor{currentfill}%
\pgfsetfillopacity{0.983282}%
\pgfsetlinewidth{1.003750pt}%
\definecolor{currentstroke}{rgb}{0.121569,0.466667,0.705882}%
\pgfsetstrokecolor{currentstroke}%
\pgfsetstrokeopacity{0.983282}%
\pgfsetdash{}{0pt}%
\pgfpathmoveto{\pgfqpoint{2.249412in}{1.415770in}}%
\pgfpathcurveto{\pgfqpoint{2.257649in}{1.415770in}}{\pgfqpoint{2.265549in}{1.419043in}}{\pgfqpoint{2.271373in}{1.424866in}}%
\pgfpathcurveto{\pgfqpoint{2.277197in}{1.430690in}}{\pgfqpoint{2.280469in}{1.438590in}}{\pgfqpoint{2.280469in}{1.446827in}}%
\pgfpathcurveto{\pgfqpoint{2.280469in}{1.455063in}}{\pgfqpoint{2.277197in}{1.462963in}}{\pgfqpoint{2.271373in}{1.468787in}}%
\pgfpathcurveto{\pgfqpoint{2.265549in}{1.474611in}}{\pgfqpoint{2.257649in}{1.477883in}}{\pgfqpoint{2.249412in}{1.477883in}}%
\pgfpathcurveto{\pgfqpoint{2.241176in}{1.477883in}}{\pgfqpoint{2.233276in}{1.474611in}}{\pgfqpoint{2.227452in}{1.468787in}}%
\pgfpathcurveto{\pgfqpoint{2.221628in}{1.462963in}}{\pgfqpoint{2.218356in}{1.455063in}}{\pgfqpoint{2.218356in}{1.446827in}}%
\pgfpathcurveto{\pgfqpoint{2.218356in}{1.438590in}}{\pgfqpoint{2.221628in}{1.430690in}}{\pgfqpoint{2.227452in}{1.424866in}}%
\pgfpathcurveto{\pgfqpoint{2.233276in}{1.419043in}}{\pgfqpoint{2.241176in}{1.415770in}}{\pgfqpoint{2.249412in}{1.415770in}}%
\pgfpathclose%
\pgfusepath{stroke,fill}%
\end{pgfscope}%
\begin{pgfscope}%
\pgfpathrectangle{\pgfqpoint{0.100000in}{0.212622in}}{\pgfqpoint{3.696000in}{3.696000in}}%
\pgfusepath{clip}%
\pgfsetbuttcap%
\pgfsetroundjoin%
\definecolor{currentfill}{rgb}{0.121569,0.466667,0.705882}%
\pgfsetfillcolor{currentfill}%
\pgfsetfillopacity{0.984400}%
\pgfsetlinewidth{1.003750pt}%
\definecolor{currentstroke}{rgb}{0.121569,0.466667,0.705882}%
\pgfsetstrokecolor{currentstroke}%
\pgfsetstrokeopacity{0.984400}%
\pgfsetdash{}{0pt}%
\pgfpathmoveto{\pgfqpoint{2.400853in}{1.340714in}}%
\pgfpathcurveto{\pgfqpoint{2.409090in}{1.340714in}}{\pgfqpoint{2.416990in}{1.343986in}}{\pgfqpoint{2.422814in}{1.349810in}}%
\pgfpathcurveto{\pgfqpoint{2.428638in}{1.355634in}}{\pgfqpoint{2.431910in}{1.363534in}}{\pgfqpoint{2.431910in}{1.371771in}}%
\pgfpathcurveto{\pgfqpoint{2.431910in}{1.380007in}}{\pgfqpoint{2.428638in}{1.387907in}}{\pgfqpoint{2.422814in}{1.393731in}}%
\pgfpathcurveto{\pgfqpoint{2.416990in}{1.399555in}}{\pgfqpoint{2.409090in}{1.402827in}}{\pgfqpoint{2.400853in}{1.402827in}}%
\pgfpathcurveto{\pgfqpoint{2.392617in}{1.402827in}}{\pgfqpoint{2.384717in}{1.399555in}}{\pgfqpoint{2.378893in}{1.393731in}}%
\pgfpathcurveto{\pgfqpoint{2.373069in}{1.387907in}}{\pgfqpoint{2.369797in}{1.380007in}}{\pgfqpoint{2.369797in}{1.371771in}}%
\pgfpathcurveto{\pgfqpoint{2.369797in}{1.363534in}}{\pgfqpoint{2.373069in}{1.355634in}}{\pgfqpoint{2.378893in}{1.349810in}}%
\pgfpathcurveto{\pgfqpoint{2.384717in}{1.343986in}}{\pgfqpoint{2.392617in}{1.340714in}}{\pgfqpoint{2.400853in}{1.340714in}}%
\pgfpathclose%
\pgfusepath{stroke,fill}%
\end{pgfscope}%
\begin{pgfscope}%
\pgfpathrectangle{\pgfqpoint{0.100000in}{0.212622in}}{\pgfqpoint{3.696000in}{3.696000in}}%
\pgfusepath{clip}%
\pgfsetbuttcap%
\pgfsetroundjoin%
\definecolor{currentfill}{rgb}{0.121569,0.466667,0.705882}%
\pgfsetfillcolor{currentfill}%
\pgfsetfillopacity{0.985856}%
\pgfsetlinewidth{1.003750pt}%
\definecolor{currentstroke}{rgb}{0.121569,0.466667,0.705882}%
\pgfsetstrokecolor{currentstroke}%
\pgfsetstrokeopacity{0.985856}%
\pgfsetdash{}{0pt}%
\pgfpathmoveto{\pgfqpoint{2.265430in}{1.409540in}}%
\pgfpathcurveto{\pgfqpoint{2.273666in}{1.409540in}}{\pgfqpoint{2.281566in}{1.412812in}}{\pgfqpoint{2.287390in}{1.418636in}}%
\pgfpathcurveto{\pgfqpoint{2.293214in}{1.424460in}}{\pgfqpoint{2.296486in}{1.432360in}}{\pgfqpoint{2.296486in}{1.440597in}}%
\pgfpathcurveto{\pgfqpoint{2.296486in}{1.448833in}}{\pgfqpoint{2.293214in}{1.456733in}}{\pgfqpoint{2.287390in}{1.462557in}}%
\pgfpathcurveto{\pgfqpoint{2.281566in}{1.468381in}}{\pgfqpoint{2.273666in}{1.471653in}}{\pgfqpoint{2.265430in}{1.471653in}}%
\pgfpathcurveto{\pgfqpoint{2.257193in}{1.471653in}}{\pgfqpoint{2.249293in}{1.468381in}}{\pgfqpoint{2.243469in}{1.462557in}}%
\pgfpathcurveto{\pgfqpoint{2.237645in}{1.456733in}}{\pgfqpoint{2.234373in}{1.448833in}}{\pgfqpoint{2.234373in}{1.440597in}}%
\pgfpathcurveto{\pgfqpoint{2.234373in}{1.432360in}}{\pgfqpoint{2.237645in}{1.424460in}}{\pgfqpoint{2.243469in}{1.418636in}}%
\pgfpathcurveto{\pgfqpoint{2.249293in}{1.412812in}}{\pgfqpoint{2.257193in}{1.409540in}}{\pgfqpoint{2.265430in}{1.409540in}}%
\pgfpathclose%
\pgfusepath{stroke,fill}%
\end{pgfscope}%
\begin{pgfscope}%
\pgfpathrectangle{\pgfqpoint{0.100000in}{0.212622in}}{\pgfqpoint{3.696000in}{3.696000in}}%
\pgfusepath{clip}%
\pgfsetbuttcap%
\pgfsetroundjoin%
\definecolor{currentfill}{rgb}{0.121569,0.466667,0.705882}%
\pgfsetfillcolor{currentfill}%
\pgfsetfillopacity{0.987609}%
\pgfsetlinewidth{1.003750pt}%
\definecolor{currentstroke}{rgb}{0.121569,0.466667,0.705882}%
\pgfsetstrokecolor{currentstroke}%
\pgfsetstrokeopacity{0.987609}%
\pgfsetdash{}{0pt}%
\pgfpathmoveto{\pgfqpoint{2.278934in}{1.401680in}}%
\pgfpathcurveto{\pgfqpoint{2.287170in}{1.401680in}}{\pgfqpoint{2.295070in}{1.404952in}}{\pgfqpoint{2.300894in}{1.410776in}}%
\pgfpathcurveto{\pgfqpoint{2.306718in}{1.416600in}}{\pgfqpoint{2.309991in}{1.424500in}}{\pgfqpoint{2.309991in}{1.432736in}}%
\pgfpathcurveto{\pgfqpoint{2.309991in}{1.440973in}}{\pgfqpoint{2.306718in}{1.448873in}}{\pgfqpoint{2.300894in}{1.454697in}}%
\pgfpathcurveto{\pgfqpoint{2.295070in}{1.460521in}}{\pgfqpoint{2.287170in}{1.463793in}}{\pgfqpoint{2.278934in}{1.463793in}}%
\pgfpathcurveto{\pgfqpoint{2.270698in}{1.463793in}}{\pgfqpoint{2.262798in}{1.460521in}}{\pgfqpoint{2.256974in}{1.454697in}}%
\pgfpathcurveto{\pgfqpoint{2.251150in}{1.448873in}}{\pgfqpoint{2.247878in}{1.440973in}}{\pgfqpoint{2.247878in}{1.432736in}}%
\pgfpathcurveto{\pgfqpoint{2.247878in}{1.424500in}}{\pgfqpoint{2.251150in}{1.416600in}}{\pgfqpoint{2.256974in}{1.410776in}}%
\pgfpathcurveto{\pgfqpoint{2.262798in}{1.404952in}}{\pgfqpoint{2.270698in}{1.401680in}}{\pgfqpoint{2.278934in}{1.401680in}}%
\pgfpathclose%
\pgfusepath{stroke,fill}%
\end{pgfscope}%
\begin{pgfscope}%
\pgfpathrectangle{\pgfqpoint{0.100000in}{0.212622in}}{\pgfqpoint{3.696000in}{3.696000in}}%
\pgfusepath{clip}%
\pgfsetbuttcap%
\pgfsetroundjoin%
\definecolor{currentfill}{rgb}{0.121569,0.466667,0.705882}%
\pgfsetfillcolor{currentfill}%
\pgfsetfillopacity{0.988636}%
\pgfsetlinewidth{1.003750pt}%
\definecolor{currentstroke}{rgb}{0.121569,0.466667,0.705882}%
\pgfsetstrokecolor{currentstroke}%
\pgfsetstrokeopacity{0.988636}%
\pgfsetdash{}{0pt}%
\pgfpathmoveto{\pgfqpoint{2.290850in}{1.390279in}}%
\pgfpathcurveto{\pgfqpoint{2.299086in}{1.390279in}}{\pgfqpoint{2.306986in}{1.393551in}}{\pgfqpoint{2.312810in}{1.399375in}}%
\pgfpathcurveto{\pgfqpoint{2.318634in}{1.405199in}}{\pgfqpoint{2.321906in}{1.413099in}}{\pgfqpoint{2.321906in}{1.421335in}}%
\pgfpathcurveto{\pgfqpoint{2.321906in}{1.429572in}}{\pgfqpoint{2.318634in}{1.437472in}}{\pgfqpoint{2.312810in}{1.443296in}}%
\pgfpathcurveto{\pgfqpoint{2.306986in}{1.449120in}}{\pgfqpoint{2.299086in}{1.452392in}}{\pgfqpoint{2.290850in}{1.452392in}}%
\pgfpathcurveto{\pgfqpoint{2.282613in}{1.452392in}}{\pgfqpoint{2.274713in}{1.449120in}}{\pgfqpoint{2.268889in}{1.443296in}}%
\pgfpathcurveto{\pgfqpoint{2.263065in}{1.437472in}}{\pgfqpoint{2.259793in}{1.429572in}}{\pgfqpoint{2.259793in}{1.421335in}}%
\pgfpathcurveto{\pgfqpoint{2.259793in}{1.413099in}}{\pgfqpoint{2.263065in}{1.405199in}}{\pgfqpoint{2.268889in}{1.399375in}}%
\pgfpathcurveto{\pgfqpoint{2.274713in}{1.393551in}}{\pgfqpoint{2.282613in}{1.390279in}}{\pgfqpoint{2.290850in}{1.390279in}}%
\pgfpathclose%
\pgfusepath{stroke,fill}%
\end{pgfscope}%
\begin{pgfscope}%
\pgfpathrectangle{\pgfqpoint{0.100000in}{0.212622in}}{\pgfqpoint{3.696000in}{3.696000in}}%
\pgfusepath{clip}%
\pgfsetbuttcap%
\pgfsetroundjoin%
\definecolor{currentfill}{rgb}{0.121569,0.466667,0.705882}%
\pgfsetfillcolor{currentfill}%
\pgfsetfillopacity{0.989904}%
\pgfsetlinewidth{1.003750pt}%
\definecolor{currentstroke}{rgb}{0.121569,0.466667,0.705882}%
\pgfsetstrokecolor{currentstroke}%
\pgfsetstrokeopacity{0.989904}%
\pgfsetdash{}{0pt}%
\pgfpathmoveto{\pgfqpoint{2.404450in}{1.335339in}}%
\pgfpathcurveto{\pgfqpoint{2.412686in}{1.335339in}}{\pgfqpoint{2.420586in}{1.338611in}}{\pgfqpoint{2.426410in}{1.344435in}}%
\pgfpathcurveto{\pgfqpoint{2.432234in}{1.350259in}}{\pgfqpoint{2.435507in}{1.358159in}}{\pgfqpoint{2.435507in}{1.366395in}}%
\pgfpathcurveto{\pgfqpoint{2.435507in}{1.374631in}}{\pgfqpoint{2.432234in}{1.382531in}}{\pgfqpoint{2.426410in}{1.388355in}}%
\pgfpathcurveto{\pgfqpoint{2.420586in}{1.394179in}}{\pgfqpoint{2.412686in}{1.397452in}}{\pgfqpoint{2.404450in}{1.397452in}}%
\pgfpathcurveto{\pgfqpoint{2.396214in}{1.397452in}}{\pgfqpoint{2.388314in}{1.394179in}}{\pgfqpoint{2.382490in}{1.388355in}}%
\pgfpathcurveto{\pgfqpoint{2.376666in}{1.382531in}}{\pgfqpoint{2.373394in}{1.374631in}}{\pgfqpoint{2.373394in}{1.366395in}}%
\pgfpathcurveto{\pgfqpoint{2.373394in}{1.358159in}}{\pgfqpoint{2.376666in}{1.350259in}}{\pgfqpoint{2.382490in}{1.344435in}}%
\pgfpathcurveto{\pgfqpoint{2.388314in}{1.338611in}}{\pgfqpoint{2.396214in}{1.335339in}}{\pgfqpoint{2.404450in}{1.335339in}}%
\pgfpathclose%
\pgfusepath{stroke,fill}%
\end{pgfscope}%
\begin{pgfscope}%
\pgfpathrectangle{\pgfqpoint{0.100000in}{0.212622in}}{\pgfqpoint{3.696000in}{3.696000in}}%
\pgfusepath{clip}%
\pgfsetbuttcap%
\pgfsetroundjoin%
\definecolor{currentfill}{rgb}{0.121569,0.466667,0.705882}%
\pgfsetfillcolor{currentfill}%
\pgfsetfillopacity{0.989984}%
\pgfsetlinewidth{1.003750pt}%
\definecolor{currentstroke}{rgb}{0.121569,0.466667,0.705882}%
\pgfsetstrokecolor{currentstroke}%
\pgfsetstrokeopacity{0.989984}%
\pgfsetdash{}{0pt}%
\pgfpathmoveto{\pgfqpoint{2.302294in}{1.383857in}}%
\pgfpathcurveto{\pgfqpoint{2.310530in}{1.383857in}}{\pgfqpoint{2.318430in}{1.387130in}}{\pgfqpoint{2.324254in}{1.392954in}}%
\pgfpathcurveto{\pgfqpoint{2.330078in}{1.398778in}}{\pgfqpoint{2.333350in}{1.406678in}}{\pgfqpoint{2.333350in}{1.414914in}}%
\pgfpathcurveto{\pgfqpoint{2.333350in}{1.423150in}}{\pgfqpoint{2.330078in}{1.431050in}}{\pgfqpoint{2.324254in}{1.436874in}}%
\pgfpathcurveto{\pgfqpoint{2.318430in}{1.442698in}}{\pgfqpoint{2.310530in}{1.445970in}}{\pgfqpoint{2.302294in}{1.445970in}}%
\pgfpathcurveto{\pgfqpoint{2.294057in}{1.445970in}}{\pgfqpoint{2.286157in}{1.442698in}}{\pgfqpoint{2.280333in}{1.436874in}}%
\pgfpathcurveto{\pgfqpoint{2.274510in}{1.431050in}}{\pgfqpoint{2.271237in}{1.423150in}}{\pgfqpoint{2.271237in}{1.414914in}}%
\pgfpathcurveto{\pgfqpoint{2.271237in}{1.406678in}}{\pgfqpoint{2.274510in}{1.398778in}}{\pgfqpoint{2.280333in}{1.392954in}}%
\pgfpathcurveto{\pgfqpoint{2.286157in}{1.387130in}}{\pgfqpoint{2.294057in}{1.383857in}}{\pgfqpoint{2.302294in}{1.383857in}}%
\pgfpathclose%
\pgfusepath{stroke,fill}%
\end{pgfscope}%
\begin{pgfscope}%
\pgfpathrectangle{\pgfqpoint{0.100000in}{0.212622in}}{\pgfqpoint{3.696000in}{3.696000in}}%
\pgfusepath{clip}%
\pgfsetbuttcap%
\pgfsetroundjoin%
\definecolor{currentfill}{rgb}{0.121569,0.466667,0.705882}%
\pgfsetfillcolor{currentfill}%
\pgfsetfillopacity{0.991218}%
\pgfsetlinewidth{1.003750pt}%
\definecolor{currentstroke}{rgb}{0.121569,0.466667,0.705882}%
\pgfsetstrokecolor{currentstroke}%
\pgfsetstrokeopacity{0.991218}%
\pgfsetdash{}{0pt}%
\pgfpathmoveto{\pgfqpoint{2.312040in}{1.378408in}}%
\pgfpathcurveto{\pgfqpoint{2.320276in}{1.378408in}}{\pgfqpoint{2.328176in}{1.381681in}}{\pgfqpoint{2.334000in}{1.387505in}}%
\pgfpathcurveto{\pgfqpoint{2.339824in}{1.393329in}}{\pgfqpoint{2.343096in}{1.401229in}}{\pgfqpoint{2.343096in}{1.409465in}}%
\pgfpathcurveto{\pgfqpoint{2.343096in}{1.417701in}}{\pgfqpoint{2.339824in}{1.425601in}}{\pgfqpoint{2.334000in}{1.431425in}}%
\pgfpathcurveto{\pgfqpoint{2.328176in}{1.437249in}}{\pgfqpoint{2.320276in}{1.440521in}}{\pgfqpoint{2.312040in}{1.440521in}}%
\pgfpathcurveto{\pgfqpoint{2.303803in}{1.440521in}}{\pgfqpoint{2.295903in}{1.437249in}}{\pgfqpoint{2.290079in}{1.431425in}}%
\pgfpathcurveto{\pgfqpoint{2.284255in}{1.425601in}}{\pgfqpoint{2.280983in}{1.417701in}}{\pgfqpoint{2.280983in}{1.409465in}}%
\pgfpathcurveto{\pgfqpoint{2.280983in}{1.401229in}}{\pgfqpoint{2.284255in}{1.393329in}}{\pgfqpoint{2.290079in}{1.387505in}}%
\pgfpathcurveto{\pgfqpoint{2.295903in}{1.381681in}}{\pgfqpoint{2.303803in}{1.378408in}}{\pgfqpoint{2.312040in}{1.378408in}}%
\pgfpathclose%
\pgfusepath{stroke,fill}%
\end{pgfscope}%
\begin{pgfscope}%
\pgfpathrectangle{\pgfqpoint{0.100000in}{0.212622in}}{\pgfqpoint{3.696000in}{3.696000in}}%
\pgfusepath{clip}%
\pgfsetbuttcap%
\pgfsetroundjoin%
\definecolor{currentfill}{rgb}{0.121569,0.466667,0.705882}%
\pgfsetfillcolor{currentfill}%
\pgfsetfillopacity{0.991976}%
\pgfsetlinewidth{1.003750pt}%
\definecolor{currentstroke}{rgb}{0.121569,0.466667,0.705882}%
\pgfsetstrokecolor{currentstroke}%
\pgfsetstrokeopacity{0.991976}%
\pgfsetdash{}{0pt}%
\pgfpathmoveto{\pgfqpoint{2.318684in}{1.374155in}}%
\pgfpathcurveto{\pgfqpoint{2.326920in}{1.374155in}}{\pgfqpoint{2.334820in}{1.377428in}}{\pgfqpoint{2.340644in}{1.383252in}}%
\pgfpathcurveto{\pgfqpoint{2.346468in}{1.389076in}}{\pgfqpoint{2.349740in}{1.396976in}}{\pgfqpoint{2.349740in}{1.405212in}}%
\pgfpathcurveto{\pgfqpoint{2.349740in}{1.413448in}}{\pgfqpoint{2.346468in}{1.421348in}}{\pgfqpoint{2.340644in}{1.427172in}}%
\pgfpathcurveto{\pgfqpoint{2.334820in}{1.432996in}}{\pgfqpoint{2.326920in}{1.436268in}}{\pgfqpoint{2.318684in}{1.436268in}}%
\pgfpathcurveto{\pgfqpoint{2.310448in}{1.436268in}}{\pgfqpoint{2.302548in}{1.432996in}}{\pgfqpoint{2.296724in}{1.427172in}}%
\pgfpathcurveto{\pgfqpoint{2.290900in}{1.421348in}}{\pgfqpoint{2.287627in}{1.413448in}}{\pgfqpoint{2.287627in}{1.405212in}}%
\pgfpathcurveto{\pgfqpoint{2.287627in}{1.396976in}}{\pgfqpoint{2.290900in}{1.389076in}}{\pgfqpoint{2.296724in}{1.383252in}}%
\pgfpathcurveto{\pgfqpoint{2.302548in}{1.377428in}}{\pgfqpoint{2.310448in}{1.374155in}}{\pgfqpoint{2.318684in}{1.374155in}}%
\pgfpathclose%
\pgfusepath{stroke,fill}%
\end{pgfscope}%
\begin{pgfscope}%
\pgfpathrectangle{\pgfqpoint{0.100000in}{0.212622in}}{\pgfqpoint{3.696000in}{3.696000in}}%
\pgfusepath{clip}%
\pgfsetbuttcap%
\pgfsetroundjoin%
\definecolor{currentfill}{rgb}{0.121569,0.466667,0.705882}%
\pgfsetfillcolor{currentfill}%
\pgfsetfillopacity{0.992578}%
\pgfsetlinewidth{1.003750pt}%
\definecolor{currentstroke}{rgb}{0.121569,0.466667,0.705882}%
\pgfsetstrokecolor{currentstroke}%
\pgfsetstrokeopacity{0.992578}%
\pgfsetdash{}{0pt}%
\pgfpathmoveto{\pgfqpoint{2.324688in}{1.370191in}}%
\pgfpathcurveto{\pgfqpoint{2.332924in}{1.370191in}}{\pgfqpoint{2.340824in}{1.373463in}}{\pgfqpoint{2.346648in}{1.379287in}}%
\pgfpathcurveto{\pgfqpoint{2.352472in}{1.385111in}}{\pgfqpoint{2.355744in}{1.393011in}}{\pgfqpoint{2.355744in}{1.401247in}}%
\pgfpathcurveto{\pgfqpoint{2.355744in}{1.409484in}}{\pgfqpoint{2.352472in}{1.417384in}}{\pgfqpoint{2.346648in}{1.423208in}}%
\pgfpathcurveto{\pgfqpoint{2.340824in}{1.429031in}}{\pgfqpoint{2.332924in}{1.432304in}}{\pgfqpoint{2.324688in}{1.432304in}}%
\pgfpathcurveto{\pgfqpoint{2.316451in}{1.432304in}}{\pgfqpoint{2.308551in}{1.429031in}}{\pgfqpoint{2.302727in}{1.423208in}}%
\pgfpathcurveto{\pgfqpoint{2.296903in}{1.417384in}}{\pgfqpoint{2.293631in}{1.409484in}}{\pgfqpoint{2.293631in}{1.401247in}}%
\pgfpathcurveto{\pgfqpoint{2.293631in}{1.393011in}}{\pgfqpoint{2.296903in}{1.385111in}}{\pgfqpoint{2.302727in}{1.379287in}}%
\pgfpathcurveto{\pgfqpoint{2.308551in}{1.373463in}}{\pgfqpoint{2.316451in}{1.370191in}}{\pgfqpoint{2.324688in}{1.370191in}}%
\pgfpathclose%
\pgfusepath{stroke,fill}%
\end{pgfscope}%
\begin{pgfscope}%
\pgfpathrectangle{\pgfqpoint{0.100000in}{0.212622in}}{\pgfqpoint{3.696000in}{3.696000in}}%
\pgfusepath{clip}%
\pgfsetbuttcap%
\pgfsetroundjoin%
\definecolor{currentfill}{rgb}{0.121569,0.466667,0.705882}%
\pgfsetfillcolor{currentfill}%
\pgfsetfillopacity{0.992649}%
\pgfsetlinewidth{1.003750pt}%
\definecolor{currentstroke}{rgb}{0.121569,0.466667,0.705882}%
\pgfsetstrokecolor{currentstroke}%
\pgfsetstrokeopacity{0.992649}%
\pgfsetdash{}{0pt}%
\pgfpathmoveto{\pgfqpoint{2.406384in}{1.330645in}}%
\pgfpathcurveto{\pgfqpoint{2.414620in}{1.330645in}}{\pgfqpoint{2.422520in}{1.333918in}}{\pgfqpoint{2.428344in}{1.339742in}}%
\pgfpathcurveto{\pgfqpoint{2.434168in}{1.345566in}}{\pgfqpoint{2.437440in}{1.353466in}}{\pgfqpoint{2.437440in}{1.361702in}}%
\pgfpathcurveto{\pgfqpoint{2.437440in}{1.369938in}}{\pgfqpoint{2.434168in}{1.377838in}}{\pgfqpoint{2.428344in}{1.383662in}}%
\pgfpathcurveto{\pgfqpoint{2.422520in}{1.389486in}}{\pgfqpoint{2.414620in}{1.392758in}}{\pgfqpoint{2.406384in}{1.392758in}}%
\pgfpathcurveto{\pgfqpoint{2.398148in}{1.392758in}}{\pgfqpoint{2.390248in}{1.389486in}}{\pgfqpoint{2.384424in}{1.383662in}}%
\pgfpathcurveto{\pgfqpoint{2.378600in}{1.377838in}}{\pgfqpoint{2.375327in}{1.369938in}}{\pgfqpoint{2.375327in}{1.361702in}}%
\pgfpathcurveto{\pgfqpoint{2.375327in}{1.353466in}}{\pgfqpoint{2.378600in}{1.345566in}}{\pgfqpoint{2.384424in}{1.339742in}}%
\pgfpathcurveto{\pgfqpoint{2.390248in}{1.333918in}}{\pgfqpoint{2.398148in}{1.330645in}}{\pgfqpoint{2.406384in}{1.330645in}}%
\pgfpathclose%
\pgfusepath{stroke,fill}%
\end{pgfscope}%
\begin{pgfscope}%
\pgfpathrectangle{\pgfqpoint{0.100000in}{0.212622in}}{\pgfqpoint{3.696000in}{3.696000in}}%
\pgfusepath{clip}%
\pgfsetbuttcap%
\pgfsetroundjoin%
\definecolor{currentfill}{rgb}{0.121569,0.466667,0.705882}%
\pgfsetfillcolor{currentfill}%
\pgfsetfillopacity{0.993011}%
\pgfsetlinewidth{1.003750pt}%
\definecolor{currentstroke}{rgb}{0.121569,0.466667,0.705882}%
\pgfsetstrokecolor{currentstroke}%
\pgfsetstrokeopacity{0.993011}%
\pgfsetdash{}{0pt}%
\pgfpathmoveto{\pgfqpoint{2.329200in}{1.367358in}}%
\pgfpathcurveto{\pgfqpoint{2.337437in}{1.367358in}}{\pgfqpoint{2.345337in}{1.370630in}}{\pgfqpoint{2.351161in}{1.376454in}}%
\pgfpathcurveto{\pgfqpoint{2.356985in}{1.382278in}}{\pgfqpoint{2.360257in}{1.390178in}}{\pgfqpoint{2.360257in}{1.398414in}}%
\pgfpathcurveto{\pgfqpoint{2.360257in}{1.406651in}}{\pgfqpoint{2.356985in}{1.414551in}}{\pgfqpoint{2.351161in}{1.420375in}}%
\pgfpathcurveto{\pgfqpoint{2.345337in}{1.426199in}}{\pgfqpoint{2.337437in}{1.429471in}}{\pgfqpoint{2.329200in}{1.429471in}}%
\pgfpathcurveto{\pgfqpoint{2.320964in}{1.429471in}}{\pgfqpoint{2.313064in}{1.426199in}}{\pgfqpoint{2.307240in}{1.420375in}}%
\pgfpathcurveto{\pgfqpoint{2.301416in}{1.414551in}}{\pgfqpoint{2.298144in}{1.406651in}}{\pgfqpoint{2.298144in}{1.398414in}}%
\pgfpathcurveto{\pgfqpoint{2.298144in}{1.390178in}}{\pgfqpoint{2.301416in}{1.382278in}}{\pgfqpoint{2.307240in}{1.376454in}}%
\pgfpathcurveto{\pgfqpoint{2.313064in}{1.370630in}}{\pgfqpoint{2.320964in}{1.367358in}}{\pgfqpoint{2.329200in}{1.367358in}}%
\pgfpathclose%
\pgfusepath{stroke,fill}%
\end{pgfscope}%
\begin{pgfscope}%
\pgfpathrectangle{\pgfqpoint{0.100000in}{0.212622in}}{\pgfqpoint{3.696000in}{3.696000in}}%
\pgfusepath{clip}%
\pgfsetbuttcap%
\pgfsetroundjoin%
\definecolor{currentfill}{rgb}{0.121569,0.466667,0.705882}%
\pgfsetfillcolor{currentfill}%
\pgfsetfillopacity{0.994120}%
\pgfsetlinewidth{1.003750pt}%
\definecolor{currentstroke}{rgb}{0.121569,0.466667,0.705882}%
\pgfsetstrokecolor{currentstroke}%
\pgfsetstrokeopacity{0.994120}%
\pgfsetdash{}{0pt}%
\pgfpathmoveto{\pgfqpoint{2.337604in}{1.364747in}}%
\pgfpathcurveto{\pgfqpoint{2.345840in}{1.364747in}}{\pgfqpoint{2.353741in}{1.368020in}}{\pgfqpoint{2.359564in}{1.373844in}}%
\pgfpathcurveto{\pgfqpoint{2.365388in}{1.379668in}}{\pgfqpoint{2.368661in}{1.387568in}}{\pgfqpoint{2.368661in}{1.395804in}}%
\pgfpathcurveto{\pgfqpoint{2.368661in}{1.404040in}}{\pgfqpoint{2.365388in}{1.411940in}}{\pgfqpoint{2.359564in}{1.417764in}}%
\pgfpathcurveto{\pgfqpoint{2.353741in}{1.423588in}}{\pgfqpoint{2.345840in}{1.426860in}}{\pgfqpoint{2.337604in}{1.426860in}}%
\pgfpathcurveto{\pgfqpoint{2.329368in}{1.426860in}}{\pgfqpoint{2.321468in}{1.423588in}}{\pgfqpoint{2.315644in}{1.417764in}}%
\pgfpathcurveto{\pgfqpoint{2.309820in}{1.411940in}}{\pgfqpoint{2.306548in}{1.404040in}}{\pgfqpoint{2.306548in}{1.395804in}}%
\pgfpathcurveto{\pgfqpoint{2.306548in}{1.387568in}}{\pgfqpoint{2.309820in}{1.379668in}}{\pgfqpoint{2.315644in}{1.373844in}}%
\pgfpathcurveto{\pgfqpoint{2.321468in}{1.368020in}}{\pgfqpoint{2.329368in}{1.364747in}}{\pgfqpoint{2.337604in}{1.364747in}}%
\pgfpathclose%
\pgfusepath{stroke,fill}%
\end{pgfscope}%
\begin{pgfscope}%
\pgfpathrectangle{\pgfqpoint{0.100000in}{0.212622in}}{\pgfqpoint{3.696000in}{3.696000in}}%
\pgfusepath{clip}%
\pgfsetbuttcap%
\pgfsetroundjoin%
\definecolor{currentfill}{rgb}{0.121569,0.466667,0.705882}%
\pgfsetfillcolor{currentfill}%
\pgfsetfillopacity{0.994538}%
\pgfsetlinewidth{1.003750pt}%
\definecolor{currentstroke}{rgb}{0.121569,0.466667,0.705882}%
\pgfsetstrokecolor{currentstroke}%
\pgfsetstrokeopacity{0.994538}%
\pgfsetdash{}{0pt}%
\pgfpathmoveto{\pgfqpoint{2.342137in}{1.360800in}}%
\pgfpathcurveto{\pgfqpoint{2.350374in}{1.360800in}}{\pgfqpoint{2.358274in}{1.364072in}}{\pgfqpoint{2.364097in}{1.369896in}}%
\pgfpathcurveto{\pgfqpoint{2.369921in}{1.375720in}}{\pgfqpoint{2.373194in}{1.383620in}}{\pgfqpoint{2.373194in}{1.391857in}}%
\pgfpathcurveto{\pgfqpoint{2.373194in}{1.400093in}}{\pgfqpoint{2.369921in}{1.407993in}}{\pgfqpoint{2.364097in}{1.413817in}}%
\pgfpathcurveto{\pgfqpoint{2.358274in}{1.419641in}}{\pgfqpoint{2.350374in}{1.422913in}}{\pgfqpoint{2.342137in}{1.422913in}}%
\pgfpathcurveto{\pgfqpoint{2.333901in}{1.422913in}}{\pgfqpoint{2.326001in}{1.419641in}}{\pgfqpoint{2.320177in}{1.413817in}}%
\pgfpathcurveto{\pgfqpoint{2.314353in}{1.407993in}}{\pgfqpoint{2.311081in}{1.400093in}}{\pgfqpoint{2.311081in}{1.391857in}}%
\pgfpathcurveto{\pgfqpoint{2.311081in}{1.383620in}}{\pgfqpoint{2.314353in}{1.375720in}}{\pgfqpoint{2.320177in}{1.369896in}}%
\pgfpathcurveto{\pgfqpoint{2.326001in}{1.364072in}}{\pgfqpoint{2.333901in}{1.360800in}}{\pgfqpoint{2.342137in}{1.360800in}}%
\pgfpathclose%
\pgfusepath{stroke,fill}%
\end{pgfscope}%
\begin{pgfscope}%
\pgfpathrectangle{\pgfqpoint{0.100000in}{0.212622in}}{\pgfqpoint{3.696000in}{3.696000in}}%
\pgfusepath{clip}%
\pgfsetbuttcap%
\pgfsetroundjoin%
\definecolor{currentfill}{rgb}{0.121569,0.466667,0.705882}%
\pgfsetfillcolor{currentfill}%
\pgfsetfillopacity{0.994852}%
\pgfsetlinewidth{1.003750pt}%
\definecolor{currentstroke}{rgb}{0.121569,0.466667,0.705882}%
\pgfsetstrokecolor{currentstroke}%
\pgfsetstrokeopacity{0.994852}%
\pgfsetdash{}{0pt}%
\pgfpathmoveto{\pgfqpoint{2.346332in}{1.357830in}}%
\pgfpathcurveto{\pgfqpoint{2.354569in}{1.357830in}}{\pgfqpoint{2.362469in}{1.361103in}}{\pgfqpoint{2.368293in}{1.366927in}}%
\pgfpathcurveto{\pgfqpoint{2.374117in}{1.372751in}}{\pgfqpoint{2.377389in}{1.380651in}}{\pgfqpoint{2.377389in}{1.388887in}}%
\pgfpathcurveto{\pgfqpoint{2.377389in}{1.397123in}}{\pgfqpoint{2.374117in}{1.405023in}}{\pgfqpoint{2.368293in}{1.410847in}}%
\pgfpathcurveto{\pgfqpoint{2.362469in}{1.416671in}}{\pgfqpoint{2.354569in}{1.419943in}}{\pgfqpoint{2.346332in}{1.419943in}}%
\pgfpathcurveto{\pgfqpoint{2.338096in}{1.419943in}}{\pgfqpoint{2.330196in}{1.416671in}}{\pgfqpoint{2.324372in}{1.410847in}}%
\pgfpathcurveto{\pgfqpoint{2.318548in}{1.405023in}}{\pgfqpoint{2.315276in}{1.397123in}}{\pgfqpoint{2.315276in}{1.388887in}}%
\pgfpathcurveto{\pgfqpoint{2.315276in}{1.380651in}}{\pgfqpoint{2.318548in}{1.372751in}}{\pgfqpoint{2.324372in}{1.366927in}}%
\pgfpathcurveto{\pgfqpoint{2.330196in}{1.361103in}}{\pgfqpoint{2.338096in}{1.357830in}}{\pgfqpoint{2.346332in}{1.357830in}}%
\pgfpathclose%
\pgfusepath{stroke,fill}%
\end{pgfscope}%
\begin{pgfscope}%
\pgfpathrectangle{\pgfqpoint{0.100000in}{0.212622in}}{\pgfqpoint{3.696000in}{3.696000in}}%
\pgfusepath{clip}%
\pgfsetbuttcap%
\pgfsetroundjoin%
\definecolor{currentfill}{rgb}{0.121569,0.466667,0.705882}%
\pgfsetfillcolor{currentfill}%
\pgfsetfillopacity{0.995052}%
\pgfsetlinewidth{1.003750pt}%
\definecolor{currentstroke}{rgb}{0.121569,0.466667,0.705882}%
\pgfsetstrokecolor{currentstroke}%
\pgfsetstrokeopacity{0.995052}%
\pgfsetdash{}{0pt}%
\pgfpathmoveto{\pgfqpoint{2.348124in}{1.356969in}}%
\pgfpathcurveto{\pgfqpoint{2.356360in}{1.356969in}}{\pgfqpoint{2.364260in}{1.360241in}}{\pgfqpoint{2.370084in}{1.366065in}}%
\pgfpathcurveto{\pgfqpoint{2.375908in}{1.371889in}}{\pgfqpoint{2.379181in}{1.379789in}}{\pgfqpoint{2.379181in}{1.388025in}}%
\pgfpathcurveto{\pgfqpoint{2.379181in}{1.396262in}}{\pgfqpoint{2.375908in}{1.404162in}}{\pgfqpoint{2.370084in}{1.409986in}}%
\pgfpathcurveto{\pgfqpoint{2.364260in}{1.415809in}}{\pgfqpoint{2.356360in}{1.419082in}}{\pgfqpoint{2.348124in}{1.419082in}}%
\pgfpathcurveto{\pgfqpoint{2.339888in}{1.419082in}}{\pgfqpoint{2.331988in}{1.415809in}}{\pgfqpoint{2.326164in}{1.409986in}}%
\pgfpathcurveto{\pgfqpoint{2.320340in}{1.404162in}}{\pgfqpoint{2.317068in}{1.396262in}}{\pgfqpoint{2.317068in}{1.388025in}}%
\pgfpathcurveto{\pgfqpoint{2.317068in}{1.379789in}}{\pgfqpoint{2.320340in}{1.371889in}}{\pgfqpoint{2.326164in}{1.366065in}}%
\pgfpathcurveto{\pgfqpoint{2.331988in}{1.360241in}}{\pgfqpoint{2.339888in}{1.356969in}}{\pgfqpoint{2.348124in}{1.356969in}}%
\pgfpathclose%
\pgfusepath{stroke,fill}%
\end{pgfscope}%
\begin{pgfscope}%
\pgfpathrectangle{\pgfqpoint{0.100000in}{0.212622in}}{\pgfqpoint{3.696000in}{3.696000in}}%
\pgfusepath{clip}%
\pgfsetbuttcap%
\pgfsetroundjoin%
\definecolor{currentfill}{rgb}{0.121569,0.466667,0.705882}%
\pgfsetfillcolor{currentfill}%
\pgfsetfillopacity{0.995403}%
\pgfsetlinewidth{1.003750pt}%
\definecolor{currentstroke}{rgb}{0.121569,0.466667,0.705882}%
\pgfsetstrokecolor{currentstroke}%
\pgfsetstrokeopacity{0.995403}%
\pgfsetdash{}{0pt}%
\pgfpathmoveto{\pgfqpoint{2.351352in}{1.355222in}}%
\pgfpathcurveto{\pgfqpoint{2.359588in}{1.355222in}}{\pgfqpoint{2.367488in}{1.358494in}}{\pgfqpoint{2.373312in}{1.364318in}}%
\pgfpathcurveto{\pgfqpoint{2.379136in}{1.370142in}}{\pgfqpoint{2.382408in}{1.378042in}}{\pgfqpoint{2.382408in}{1.386279in}}%
\pgfpathcurveto{\pgfqpoint{2.382408in}{1.394515in}}{\pgfqpoint{2.379136in}{1.402415in}}{\pgfqpoint{2.373312in}{1.408239in}}%
\pgfpathcurveto{\pgfqpoint{2.367488in}{1.414063in}}{\pgfqpoint{2.359588in}{1.417335in}}{\pgfqpoint{2.351352in}{1.417335in}}%
\pgfpathcurveto{\pgfqpoint{2.343116in}{1.417335in}}{\pgfqpoint{2.335216in}{1.414063in}}{\pgfqpoint{2.329392in}{1.408239in}}%
\pgfpathcurveto{\pgfqpoint{2.323568in}{1.402415in}}{\pgfqpoint{2.320295in}{1.394515in}}{\pgfqpoint{2.320295in}{1.386279in}}%
\pgfpathcurveto{\pgfqpoint{2.320295in}{1.378042in}}{\pgfqpoint{2.323568in}{1.370142in}}{\pgfqpoint{2.329392in}{1.364318in}}%
\pgfpathcurveto{\pgfqpoint{2.335216in}{1.358494in}}{\pgfqpoint{2.343116in}{1.355222in}}{\pgfqpoint{2.351352in}{1.355222in}}%
\pgfpathclose%
\pgfusepath{stroke,fill}%
\end{pgfscope}%
\begin{pgfscope}%
\pgfpathrectangle{\pgfqpoint{0.100000in}{0.212622in}}{\pgfqpoint{3.696000in}{3.696000in}}%
\pgfusepath{clip}%
\pgfsetbuttcap%
\pgfsetroundjoin%
\definecolor{currentfill}{rgb}{0.121569,0.466667,0.705882}%
\pgfsetfillcolor{currentfill}%
\pgfsetfillopacity{0.995515}%
\pgfsetlinewidth{1.003750pt}%
\definecolor{currentstroke}{rgb}{0.121569,0.466667,0.705882}%
\pgfsetstrokecolor{currentstroke}%
\pgfsetstrokeopacity{0.995515}%
\pgfsetdash{}{0pt}%
\pgfpathmoveto{\pgfqpoint{2.352482in}{1.354499in}}%
\pgfpathcurveto{\pgfqpoint{2.360718in}{1.354499in}}{\pgfqpoint{2.368618in}{1.357771in}}{\pgfqpoint{2.374442in}{1.363595in}}%
\pgfpathcurveto{\pgfqpoint{2.380266in}{1.369419in}}{\pgfqpoint{2.383538in}{1.377319in}}{\pgfqpoint{2.383538in}{1.385555in}}%
\pgfpathcurveto{\pgfqpoint{2.383538in}{1.393792in}}{\pgfqpoint{2.380266in}{1.401692in}}{\pgfqpoint{2.374442in}{1.407516in}}%
\pgfpathcurveto{\pgfqpoint{2.368618in}{1.413340in}}{\pgfqpoint{2.360718in}{1.416612in}}{\pgfqpoint{2.352482in}{1.416612in}}%
\pgfpathcurveto{\pgfqpoint{2.344246in}{1.416612in}}{\pgfqpoint{2.336346in}{1.413340in}}{\pgfqpoint{2.330522in}{1.407516in}}%
\pgfpathcurveto{\pgfqpoint{2.324698in}{1.401692in}}{\pgfqpoint{2.321425in}{1.393792in}}{\pgfqpoint{2.321425in}{1.385555in}}%
\pgfpathcurveto{\pgfqpoint{2.321425in}{1.377319in}}{\pgfqpoint{2.324698in}{1.369419in}}{\pgfqpoint{2.330522in}{1.363595in}}%
\pgfpathcurveto{\pgfqpoint{2.336346in}{1.357771in}}{\pgfqpoint{2.344246in}{1.354499in}}{\pgfqpoint{2.352482in}{1.354499in}}%
\pgfpathclose%
\pgfusepath{stroke,fill}%
\end{pgfscope}%
\begin{pgfscope}%
\pgfpathrectangle{\pgfqpoint{0.100000in}{0.212622in}}{\pgfqpoint{3.696000in}{3.696000in}}%
\pgfusepath{clip}%
\pgfsetbuttcap%
\pgfsetroundjoin%
\definecolor{currentfill}{rgb}{0.121569,0.466667,0.705882}%
\pgfsetfillcolor{currentfill}%
\pgfsetfillopacity{0.995546}%
\pgfsetlinewidth{1.003750pt}%
\definecolor{currentstroke}{rgb}{0.121569,0.466667,0.705882}%
\pgfsetstrokecolor{currentstroke}%
\pgfsetstrokeopacity{0.995546}%
\pgfsetdash{}{0pt}%
\pgfpathmoveto{\pgfqpoint{2.352711in}{1.354367in}}%
\pgfpathcurveto{\pgfqpoint{2.360947in}{1.354367in}}{\pgfqpoint{2.368847in}{1.357639in}}{\pgfqpoint{2.374671in}{1.363463in}}%
\pgfpathcurveto{\pgfqpoint{2.380495in}{1.369287in}}{\pgfqpoint{2.383767in}{1.377187in}}{\pgfqpoint{2.383767in}{1.385424in}}%
\pgfpathcurveto{\pgfqpoint{2.383767in}{1.393660in}}{\pgfqpoint{2.380495in}{1.401560in}}{\pgfqpoint{2.374671in}{1.407384in}}%
\pgfpathcurveto{\pgfqpoint{2.368847in}{1.413208in}}{\pgfqpoint{2.360947in}{1.416480in}}{\pgfqpoint{2.352711in}{1.416480in}}%
\pgfpathcurveto{\pgfqpoint{2.344474in}{1.416480in}}{\pgfqpoint{2.336574in}{1.413208in}}{\pgfqpoint{2.330750in}{1.407384in}}%
\pgfpathcurveto{\pgfqpoint{2.324927in}{1.401560in}}{\pgfqpoint{2.321654in}{1.393660in}}{\pgfqpoint{2.321654in}{1.385424in}}%
\pgfpathcurveto{\pgfqpoint{2.321654in}{1.377187in}}{\pgfqpoint{2.324927in}{1.369287in}}{\pgfqpoint{2.330750in}{1.363463in}}%
\pgfpathcurveto{\pgfqpoint{2.336574in}{1.357639in}}{\pgfqpoint{2.344474in}{1.354367in}}{\pgfqpoint{2.352711in}{1.354367in}}%
\pgfpathclose%
\pgfusepath{stroke,fill}%
\end{pgfscope}%
\begin{pgfscope}%
\pgfpathrectangle{\pgfqpoint{0.100000in}{0.212622in}}{\pgfqpoint{3.696000in}{3.696000in}}%
\pgfusepath{clip}%
\pgfsetbuttcap%
\pgfsetroundjoin%
\definecolor{currentfill}{rgb}{0.121569,0.466667,0.705882}%
\pgfsetfillcolor{currentfill}%
\pgfsetfillopacity{0.995601}%
\pgfsetlinewidth{1.003750pt}%
\definecolor{currentstroke}{rgb}{0.121569,0.466667,0.705882}%
\pgfsetstrokecolor{currentstroke}%
\pgfsetstrokeopacity{0.995601}%
\pgfsetdash{}{0pt}%
\pgfpathmoveto{\pgfqpoint{2.353118in}{1.354097in}}%
\pgfpathcurveto{\pgfqpoint{2.361354in}{1.354097in}}{\pgfqpoint{2.369254in}{1.357370in}}{\pgfqpoint{2.375078in}{1.363194in}}%
\pgfpathcurveto{\pgfqpoint{2.380902in}{1.369018in}}{\pgfqpoint{2.384174in}{1.376918in}}{\pgfqpoint{2.384174in}{1.385154in}}%
\pgfpathcurveto{\pgfqpoint{2.384174in}{1.393390in}}{\pgfqpoint{2.380902in}{1.401290in}}{\pgfqpoint{2.375078in}{1.407114in}}%
\pgfpathcurveto{\pgfqpoint{2.369254in}{1.412938in}}{\pgfqpoint{2.361354in}{1.416210in}}{\pgfqpoint{2.353118in}{1.416210in}}%
\pgfpathcurveto{\pgfqpoint{2.344882in}{1.416210in}}{\pgfqpoint{2.336982in}{1.412938in}}{\pgfqpoint{2.331158in}{1.407114in}}%
\pgfpathcurveto{\pgfqpoint{2.325334in}{1.401290in}}{\pgfqpoint{2.322061in}{1.393390in}}{\pgfqpoint{2.322061in}{1.385154in}}%
\pgfpathcurveto{\pgfqpoint{2.322061in}{1.376918in}}{\pgfqpoint{2.325334in}{1.369018in}}{\pgfqpoint{2.331158in}{1.363194in}}%
\pgfpathcurveto{\pgfqpoint{2.336982in}{1.357370in}}{\pgfqpoint{2.344882in}{1.354097in}}{\pgfqpoint{2.353118in}{1.354097in}}%
\pgfpathclose%
\pgfusepath{stroke,fill}%
\end{pgfscope}%
\begin{pgfscope}%
\pgfpathrectangle{\pgfqpoint{0.100000in}{0.212622in}}{\pgfqpoint{3.696000in}{3.696000in}}%
\pgfusepath{clip}%
\pgfsetbuttcap%
\pgfsetroundjoin%
\definecolor{currentfill}{rgb}{0.121569,0.466667,0.705882}%
\pgfsetfillcolor{currentfill}%
\pgfsetfillopacity{0.995692}%
\pgfsetlinewidth{1.003750pt}%
\definecolor{currentstroke}{rgb}{0.121569,0.466667,0.705882}%
\pgfsetstrokecolor{currentstroke}%
\pgfsetstrokeopacity{0.995692}%
\pgfsetdash{}{0pt}%
\pgfpathmoveto{\pgfqpoint{2.353848in}{1.353525in}}%
\pgfpathcurveto{\pgfqpoint{2.362084in}{1.353525in}}{\pgfqpoint{2.369984in}{1.356797in}}{\pgfqpoint{2.375808in}{1.362621in}}%
\pgfpathcurveto{\pgfqpoint{2.381632in}{1.368445in}}{\pgfqpoint{2.384904in}{1.376345in}}{\pgfqpoint{2.384904in}{1.384581in}}%
\pgfpathcurveto{\pgfqpoint{2.384904in}{1.392818in}}{\pgfqpoint{2.381632in}{1.400718in}}{\pgfqpoint{2.375808in}{1.406542in}}%
\pgfpathcurveto{\pgfqpoint{2.369984in}{1.412366in}}{\pgfqpoint{2.362084in}{1.415638in}}{\pgfqpoint{2.353848in}{1.415638in}}%
\pgfpathcurveto{\pgfqpoint{2.345612in}{1.415638in}}{\pgfqpoint{2.337712in}{1.412366in}}{\pgfqpoint{2.331888in}{1.406542in}}%
\pgfpathcurveto{\pgfqpoint{2.326064in}{1.400718in}}{\pgfqpoint{2.322791in}{1.392818in}}{\pgfqpoint{2.322791in}{1.384581in}}%
\pgfpathcurveto{\pgfqpoint{2.322791in}{1.376345in}}{\pgfqpoint{2.326064in}{1.368445in}}{\pgfqpoint{2.331888in}{1.362621in}}%
\pgfpathcurveto{\pgfqpoint{2.337712in}{1.356797in}}{\pgfqpoint{2.345612in}{1.353525in}}{\pgfqpoint{2.353848in}{1.353525in}}%
\pgfpathclose%
\pgfusepath{stroke,fill}%
\end{pgfscope}%
\begin{pgfscope}%
\pgfpathrectangle{\pgfqpoint{0.100000in}{0.212622in}}{\pgfqpoint{3.696000in}{3.696000in}}%
\pgfusepath{clip}%
\pgfsetbuttcap%
\pgfsetroundjoin%
\definecolor{currentfill}{rgb}{0.121569,0.466667,0.705882}%
\pgfsetfillcolor{currentfill}%
\pgfsetfillopacity{0.995699}%
\pgfsetlinewidth{1.003750pt}%
\definecolor{currentstroke}{rgb}{0.121569,0.466667,0.705882}%
\pgfsetstrokecolor{currentstroke}%
\pgfsetstrokeopacity{0.995699}%
\pgfsetdash{}{0pt}%
\pgfpathmoveto{\pgfqpoint{2.407960in}{1.326216in}}%
\pgfpathcurveto{\pgfqpoint{2.416196in}{1.326216in}}{\pgfqpoint{2.424096in}{1.329489in}}{\pgfqpoint{2.429920in}{1.335313in}}%
\pgfpathcurveto{\pgfqpoint{2.435744in}{1.341137in}}{\pgfqpoint{2.439017in}{1.349037in}}{\pgfqpoint{2.439017in}{1.357273in}}%
\pgfpathcurveto{\pgfqpoint{2.439017in}{1.365509in}}{\pgfqpoint{2.435744in}{1.373409in}}{\pgfqpoint{2.429920in}{1.379233in}}%
\pgfpathcurveto{\pgfqpoint{2.424096in}{1.385057in}}{\pgfqpoint{2.416196in}{1.388329in}}{\pgfqpoint{2.407960in}{1.388329in}}%
\pgfpathcurveto{\pgfqpoint{2.399724in}{1.388329in}}{\pgfqpoint{2.391824in}{1.385057in}}{\pgfqpoint{2.386000in}{1.379233in}}%
\pgfpathcurveto{\pgfqpoint{2.380176in}{1.373409in}}{\pgfqpoint{2.376904in}{1.365509in}}{\pgfqpoint{2.376904in}{1.357273in}}%
\pgfpathcurveto{\pgfqpoint{2.376904in}{1.349037in}}{\pgfqpoint{2.380176in}{1.341137in}}{\pgfqpoint{2.386000in}{1.335313in}}%
\pgfpathcurveto{\pgfqpoint{2.391824in}{1.329489in}}{\pgfqpoint{2.399724in}{1.326216in}}{\pgfqpoint{2.407960in}{1.326216in}}%
\pgfpathclose%
\pgfusepath{stroke,fill}%
\end{pgfscope}%
\begin{pgfscope}%
\pgfpathrectangle{\pgfqpoint{0.100000in}{0.212622in}}{\pgfqpoint{3.696000in}{3.696000in}}%
\pgfusepath{clip}%
\pgfsetbuttcap%
\pgfsetroundjoin%
\definecolor{currentfill}{rgb}{0.121569,0.466667,0.705882}%
\pgfsetfillcolor{currentfill}%
\pgfsetfillopacity{0.995847}%
\pgfsetlinewidth{1.003750pt}%
\definecolor{currentstroke}{rgb}{0.121569,0.466667,0.705882}%
\pgfsetstrokecolor{currentstroke}%
\pgfsetstrokeopacity{0.995847}%
\pgfsetdash{}{0pt}%
\pgfpathmoveto{\pgfqpoint{2.355185in}{1.352455in}}%
\pgfpathcurveto{\pgfqpoint{2.363421in}{1.352455in}}{\pgfqpoint{2.371321in}{1.355728in}}{\pgfqpoint{2.377145in}{1.361551in}}%
\pgfpathcurveto{\pgfqpoint{2.382969in}{1.367375in}}{\pgfqpoint{2.386241in}{1.375275in}}{\pgfqpoint{2.386241in}{1.383512in}}%
\pgfpathcurveto{\pgfqpoint{2.386241in}{1.391748in}}{\pgfqpoint{2.382969in}{1.399648in}}{\pgfqpoint{2.377145in}{1.405472in}}%
\pgfpathcurveto{\pgfqpoint{2.371321in}{1.411296in}}{\pgfqpoint{2.363421in}{1.414568in}}{\pgfqpoint{2.355185in}{1.414568in}}%
\pgfpathcurveto{\pgfqpoint{2.346948in}{1.414568in}}{\pgfqpoint{2.339048in}{1.411296in}}{\pgfqpoint{2.333224in}{1.405472in}}%
\pgfpathcurveto{\pgfqpoint{2.327401in}{1.399648in}}{\pgfqpoint{2.324128in}{1.391748in}}{\pgfqpoint{2.324128in}{1.383512in}}%
\pgfpathcurveto{\pgfqpoint{2.324128in}{1.375275in}}{\pgfqpoint{2.327401in}{1.367375in}}{\pgfqpoint{2.333224in}{1.361551in}}%
\pgfpathcurveto{\pgfqpoint{2.339048in}{1.355728in}}{\pgfqpoint{2.346948in}{1.352455in}}{\pgfqpoint{2.355185in}{1.352455in}}%
\pgfpathclose%
\pgfusepath{stroke,fill}%
\end{pgfscope}%
\begin{pgfscope}%
\pgfpathrectangle{\pgfqpoint{0.100000in}{0.212622in}}{\pgfqpoint{3.696000in}{3.696000in}}%
\pgfusepath{clip}%
\pgfsetbuttcap%
\pgfsetroundjoin%
\definecolor{currentfill}{rgb}{0.121569,0.466667,0.705882}%
\pgfsetfillcolor{currentfill}%
\pgfsetfillopacity{0.996126}%
\pgfsetlinewidth{1.003750pt}%
\definecolor{currentstroke}{rgb}{0.121569,0.466667,0.705882}%
\pgfsetstrokecolor{currentstroke}%
\pgfsetstrokeopacity{0.996126}%
\pgfsetdash{}{0pt}%
\pgfpathmoveto{\pgfqpoint{2.357629in}{1.350523in}}%
\pgfpathcurveto{\pgfqpoint{2.365866in}{1.350523in}}{\pgfqpoint{2.373766in}{1.353795in}}{\pgfqpoint{2.379590in}{1.359619in}}%
\pgfpathcurveto{\pgfqpoint{2.385413in}{1.365443in}}{\pgfqpoint{2.388686in}{1.373343in}}{\pgfqpoint{2.388686in}{1.381580in}}%
\pgfpathcurveto{\pgfqpoint{2.388686in}{1.389816in}}{\pgfqpoint{2.385413in}{1.397716in}}{\pgfqpoint{2.379590in}{1.403540in}}%
\pgfpathcurveto{\pgfqpoint{2.373766in}{1.409364in}}{\pgfqpoint{2.365866in}{1.412636in}}{\pgfqpoint{2.357629in}{1.412636in}}%
\pgfpathcurveto{\pgfqpoint{2.349393in}{1.412636in}}{\pgfqpoint{2.341493in}{1.409364in}}{\pgfqpoint{2.335669in}{1.403540in}}%
\pgfpathcurveto{\pgfqpoint{2.329845in}{1.397716in}}{\pgfqpoint{2.326573in}{1.389816in}}{\pgfqpoint{2.326573in}{1.381580in}}%
\pgfpathcurveto{\pgfqpoint{2.326573in}{1.373343in}}{\pgfqpoint{2.329845in}{1.365443in}}{\pgfqpoint{2.335669in}{1.359619in}}%
\pgfpathcurveto{\pgfqpoint{2.341493in}{1.353795in}}{\pgfqpoint{2.349393in}{1.350523in}}{\pgfqpoint{2.357629in}{1.350523in}}%
\pgfpathclose%
\pgfusepath{stroke,fill}%
\end{pgfscope}%
\begin{pgfscope}%
\pgfpathrectangle{\pgfqpoint{0.100000in}{0.212622in}}{\pgfqpoint{3.696000in}{3.696000in}}%
\pgfusepath{clip}%
\pgfsetbuttcap%
\pgfsetroundjoin%
\definecolor{currentfill}{rgb}{0.121569,0.466667,0.705882}%
\pgfsetfillcolor{currentfill}%
\pgfsetfillopacity{0.996609}%
\pgfsetlinewidth{1.003750pt}%
\definecolor{currentstroke}{rgb}{0.121569,0.466667,0.705882}%
\pgfsetstrokecolor{currentstroke}%
\pgfsetstrokeopacity{0.996609}%
\pgfsetdash{}{0pt}%
\pgfpathmoveto{\pgfqpoint{2.362080in}{1.346892in}}%
\pgfpathcurveto{\pgfqpoint{2.370316in}{1.346892in}}{\pgfqpoint{2.378216in}{1.350165in}}{\pgfqpoint{2.384040in}{1.355989in}}%
\pgfpathcurveto{\pgfqpoint{2.389864in}{1.361812in}}{\pgfqpoint{2.393136in}{1.369713in}}{\pgfqpoint{2.393136in}{1.377949in}}%
\pgfpathcurveto{\pgfqpoint{2.393136in}{1.386185in}}{\pgfqpoint{2.389864in}{1.394085in}}{\pgfqpoint{2.384040in}{1.399909in}}%
\pgfpathcurveto{\pgfqpoint{2.378216in}{1.405733in}}{\pgfqpoint{2.370316in}{1.409005in}}{\pgfqpoint{2.362080in}{1.409005in}}%
\pgfpathcurveto{\pgfqpoint{2.353843in}{1.409005in}}{\pgfqpoint{2.345943in}{1.405733in}}{\pgfqpoint{2.340119in}{1.399909in}}%
\pgfpathcurveto{\pgfqpoint{2.334295in}{1.394085in}}{\pgfqpoint{2.331023in}{1.386185in}}{\pgfqpoint{2.331023in}{1.377949in}}%
\pgfpathcurveto{\pgfqpoint{2.331023in}{1.369713in}}{\pgfqpoint{2.334295in}{1.361812in}}{\pgfqpoint{2.340119in}{1.355989in}}%
\pgfpathcurveto{\pgfqpoint{2.345943in}{1.350165in}}{\pgfqpoint{2.353843in}{1.346892in}}{\pgfqpoint{2.362080in}{1.346892in}}%
\pgfpathclose%
\pgfusepath{stroke,fill}%
\end{pgfscope}%
\begin{pgfscope}%
\pgfpathrectangle{\pgfqpoint{0.100000in}{0.212622in}}{\pgfqpoint{3.696000in}{3.696000in}}%
\pgfusepath{clip}%
\pgfsetbuttcap%
\pgfsetroundjoin%
\definecolor{currentfill}{rgb}{0.121569,0.466667,0.705882}%
\pgfsetfillcolor{currentfill}%
\pgfsetfillopacity{0.997225}%
\pgfsetlinewidth{1.003750pt}%
\definecolor{currentstroke}{rgb}{0.121569,0.466667,0.705882}%
\pgfsetstrokecolor{currentstroke}%
\pgfsetstrokeopacity{0.997225}%
\pgfsetdash{}{0pt}%
\pgfpathmoveto{\pgfqpoint{2.407914in}{1.322630in}}%
\pgfpathcurveto{\pgfqpoint{2.416150in}{1.322630in}}{\pgfqpoint{2.424050in}{1.325902in}}{\pgfqpoint{2.429874in}{1.331726in}}%
\pgfpathcurveto{\pgfqpoint{2.435698in}{1.337550in}}{\pgfqpoint{2.438970in}{1.345450in}}{\pgfqpoint{2.438970in}{1.353686in}}%
\pgfpathcurveto{\pgfqpoint{2.438970in}{1.361923in}}{\pgfqpoint{2.435698in}{1.369823in}}{\pgfqpoint{2.429874in}{1.375647in}}%
\pgfpathcurveto{\pgfqpoint{2.424050in}{1.381471in}}{\pgfqpoint{2.416150in}{1.384743in}}{\pgfqpoint{2.407914in}{1.384743in}}%
\pgfpathcurveto{\pgfqpoint{2.399677in}{1.384743in}}{\pgfqpoint{2.391777in}{1.381471in}}{\pgfqpoint{2.385953in}{1.375647in}}%
\pgfpathcurveto{\pgfqpoint{2.380129in}{1.369823in}}{\pgfqpoint{2.376857in}{1.361923in}}{\pgfqpoint{2.376857in}{1.353686in}}%
\pgfpathcurveto{\pgfqpoint{2.376857in}{1.345450in}}{\pgfqpoint{2.380129in}{1.337550in}}{\pgfqpoint{2.385953in}{1.331726in}}%
\pgfpathcurveto{\pgfqpoint{2.391777in}{1.325902in}}{\pgfqpoint{2.399677in}{1.322630in}}{\pgfqpoint{2.407914in}{1.322630in}}%
\pgfpathclose%
\pgfusepath{stroke,fill}%
\end{pgfscope}%
\begin{pgfscope}%
\pgfpathrectangle{\pgfqpoint{0.100000in}{0.212622in}}{\pgfqpoint{3.696000in}{3.696000in}}%
\pgfusepath{clip}%
\pgfsetbuttcap%
\pgfsetroundjoin%
\definecolor{currentfill}{rgb}{0.121569,0.466667,0.705882}%
\pgfsetfillcolor{currentfill}%
\pgfsetfillopacity{0.997489}%
\pgfsetlinewidth{1.003750pt}%
\definecolor{currentstroke}{rgb}{0.121569,0.466667,0.705882}%
\pgfsetstrokecolor{currentstroke}%
\pgfsetstrokeopacity{0.997489}%
\pgfsetdash{}{0pt}%
\pgfpathmoveto{\pgfqpoint{2.370049in}{1.339954in}}%
\pgfpathcurveto{\pgfqpoint{2.378286in}{1.339954in}}{\pgfqpoint{2.386186in}{1.343226in}}{\pgfqpoint{2.392010in}{1.349050in}}%
\pgfpathcurveto{\pgfqpoint{2.397833in}{1.354874in}}{\pgfqpoint{2.401106in}{1.362774in}}{\pgfqpoint{2.401106in}{1.371010in}}%
\pgfpathcurveto{\pgfqpoint{2.401106in}{1.379246in}}{\pgfqpoint{2.397833in}{1.387146in}}{\pgfqpoint{2.392010in}{1.392970in}}%
\pgfpathcurveto{\pgfqpoint{2.386186in}{1.398794in}}{\pgfqpoint{2.378286in}{1.402067in}}{\pgfqpoint{2.370049in}{1.402067in}}%
\pgfpathcurveto{\pgfqpoint{2.361813in}{1.402067in}}{\pgfqpoint{2.353913in}{1.398794in}}{\pgfqpoint{2.348089in}{1.392970in}}%
\pgfpathcurveto{\pgfqpoint{2.342265in}{1.387146in}}{\pgfqpoint{2.338993in}{1.379246in}}{\pgfqpoint{2.338993in}{1.371010in}}%
\pgfpathcurveto{\pgfqpoint{2.338993in}{1.362774in}}{\pgfqpoint{2.342265in}{1.354874in}}{\pgfqpoint{2.348089in}{1.349050in}}%
\pgfpathcurveto{\pgfqpoint{2.353913in}{1.343226in}}{\pgfqpoint{2.361813in}{1.339954in}}{\pgfqpoint{2.370049in}{1.339954in}}%
\pgfpathclose%
\pgfusepath{stroke,fill}%
\end{pgfscope}%
\begin{pgfscope}%
\pgfpathrectangle{\pgfqpoint{0.100000in}{0.212622in}}{\pgfqpoint{3.696000in}{3.696000in}}%
\pgfusepath{clip}%
\pgfsetbuttcap%
\pgfsetroundjoin%
\definecolor{currentfill}{rgb}{0.121569,0.466667,0.705882}%
\pgfsetfillcolor{currentfill}%
\pgfsetfillopacity{0.998000}%
\pgfsetlinewidth{1.003750pt}%
\definecolor{currentstroke}{rgb}{0.121569,0.466667,0.705882}%
\pgfsetstrokecolor{currentstroke}%
\pgfsetstrokeopacity{0.998000}%
\pgfsetdash{}{0pt}%
\pgfpathmoveto{\pgfqpoint{2.377666in}{1.331930in}}%
\pgfpathcurveto{\pgfqpoint{2.385902in}{1.331930in}}{\pgfqpoint{2.393802in}{1.335202in}}{\pgfqpoint{2.399626in}{1.341026in}}%
\pgfpathcurveto{\pgfqpoint{2.405450in}{1.346850in}}{\pgfqpoint{2.408723in}{1.354750in}}{\pgfqpoint{2.408723in}{1.362986in}}%
\pgfpathcurveto{\pgfqpoint{2.408723in}{1.371223in}}{\pgfqpoint{2.405450in}{1.379123in}}{\pgfqpoint{2.399626in}{1.384947in}}%
\pgfpathcurveto{\pgfqpoint{2.393802in}{1.390770in}}{\pgfqpoint{2.385902in}{1.394043in}}{\pgfqpoint{2.377666in}{1.394043in}}%
\pgfpathcurveto{\pgfqpoint{2.369430in}{1.394043in}}{\pgfqpoint{2.361530in}{1.390770in}}{\pgfqpoint{2.355706in}{1.384947in}}%
\pgfpathcurveto{\pgfqpoint{2.349882in}{1.379123in}}{\pgfqpoint{2.346610in}{1.371223in}}{\pgfqpoint{2.346610in}{1.362986in}}%
\pgfpathcurveto{\pgfqpoint{2.346610in}{1.354750in}}{\pgfqpoint{2.349882in}{1.346850in}}{\pgfqpoint{2.355706in}{1.341026in}}%
\pgfpathcurveto{\pgfqpoint{2.361530in}{1.335202in}}{\pgfqpoint{2.369430in}{1.331930in}}{\pgfqpoint{2.377666in}{1.331930in}}%
\pgfpathclose%
\pgfusepath{stroke,fill}%
\end{pgfscope}%
\begin{pgfscope}%
\pgfpathrectangle{\pgfqpoint{0.100000in}{0.212622in}}{\pgfqpoint{3.696000in}{3.696000in}}%
\pgfusepath{clip}%
\pgfsetbuttcap%
\pgfsetroundjoin%
\definecolor{currentfill}{rgb}{0.121569,0.466667,0.705882}%
\pgfsetfillcolor{currentfill}%
\pgfsetfillopacity{0.998198}%
\pgfsetlinewidth{1.003750pt}%
\definecolor{currentstroke}{rgb}{0.121569,0.466667,0.705882}%
\pgfsetstrokecolor{currentstroke}%
\pgfsetstrokeopacity{0.998198}%
\pgfsetdash{}{0pt}%
\pgfpathmoveto{\pgfqpoint{2.407420in}{1.321469in}}%
\pgfpathcurveto{\pgfqpoint{2.415656in}{1.321469in}}{\pgfqpoint{2.423556in}{1.324741in}}{\pgfqpoint{2.429380in}{1.330565in}}%
\pgfpathcurveto{\pgfqpoint{2.435204in}{1.336389in}}{\pgfqpoint{2.438476in}{1.344289in}}{\pgfqpoint{2.438476in}{1.352526in}}%
\pgfpathcurveto{\pgfqpoint{2.438476in}{1.360762in}}{\pgfqpoint{2.435204in}{1.368662in}}{\pgfqpoint{2.429380in}{1.374486in}}%
\pgfpathcurveto{\pgfqpoint{2.423556in}{1.380310in}}{\pgfqpoint{2.415656in}{1.383582in}}{\pgfqpoint{2.407420in}{1.383582in}}%
\pgfpathcurveto{\pgfqpoint{2.399183in}{1.383582in}}{\pgfqpoint{2.391283in}{1.380310in}}{\pgfqpoint{2.385459in}{1.374486in}}%
\pgfpathcurveto{\pgfqpoint{2.379636in}{1.368662in}}{\pgfqpoint{2.376363in}{1.360762in}}{\pgfqpoint{2.376363in}{1.352526in}}%
\pgfpathcurveto{\pgfqpoint{2.376363in}{1.344289in}}{\pgfqpoint{2.379636in}{1.336389in}}{\pgfqpoint{2.385459in}{1.330565in}}%
\pgfpathcurveto{\pgfqpoint{2.391283in}{1.324741in}}{\pgfqpoint{2.399183in}{1.321469in}}{\pgfqpoint{2.407420in}{1.321469in}}%
\pgfpathclose%
\pgfusepath{stroke,fill}%
\end{pgfscope}%
\begin{pgfscope}%
\pgfpathrectangle{\pgfqpoint{0.100000in}{0.212622in}}{\pgfqpoint{3.696000in}{3.696000in}}%
\pgfusepath{clip}%
\pgfsetbuttcap%
\pgfsetroundjoin%
\definecolor{currentfill}{rgb}{0.121569,0.466667,0.705882}%
\pgfsetfillcolor{currentfill}%
\pgfsetfillopacity{0.998691}%
\pgfsetlinewidth{1.003750pt}%
\definecolor{currentstroke}{rgb}{0.121569,0.466667,0.705882}%
\pgfsetstrokecolor{currentstroke}%
\pgfsetstrokeopacity{0.998691}%
\pgfsetdash{}{0pt}%
\pgfpathmoveto{\pgfqpoint{2.406668in}{1.320812in}}%
\pgfpathcurveto{\pgfqpoint{2.414904in}{1.320812in}}{\pgfqpoint{2.422804in}{1.324084in}}{\pgfqpoint{2.428628in}{1.329908in}}%
\pgfpathcurveto{\pgfqpoint{2.434452in}{1.335732in}}{\pgfqpoint{2.437724in}{1.343632in}}{\pgfqpoint{2.437724in}{1.351868in}}%
\pgfpathcurveto{\pgfqpoint{2.437724in}{1.360104in}}{\pgfqpoint{2.434452in}{1.368004in}}{\pgfqpoint{2.428628in}{1.373828in}}%
\pgfpathcurveto{\pgfqpoint{2.422804in}{1.379652in}}{\pgfqpoint{2.414904in}{1.382925in}}{\pgfqpoint{2.406668in}{1.382925in}}%
\pgfpathcurveto{\pgfqpoint{2.398432in}{1.382925in}}{\pgfqpoint{2.390531in}{1.379652in}}{\pgfqpoint{2.384708in}{1.373828in}}%
\pgfpathcurveto{\pgfqpoint{2.378884in}{1.368004in}}{\pgfqpoint{2.375611in}{1.360104in}}{\pgfqpoint{2.375611in}{1.351868in}}%
\pgfpathcurveto{\pgfqpoint{2.375611in}{1.343632in}}{\pgfqpoint{2.378884in}{1.335732in}}{\pgfqpoint{2.384708in}{1.329908in}}%
\pgfpathcurveto{\pgfqpoint{2.390531in}{1.324084in}}{\pgfqpoint{2.398432in}{1.320812in}}{\pgfqpoint{2.406668in}{1.320812in}}%
\pgfpathclose%
\pgfusepath{stroke,fill}%
\end{pgfscope}%
\begin{pgfscope}%
\pgfpathrectangle{\pgfqpoint{0.100000in}{0.212622in}}{\pgfqpoint{3.696000in}{3.696000in}}%
\pgfusepath{clip}%
\pgfsetbuttcap%
\pgfsetroundjoin%
\definecolor{currentfill}{rgb}{0.121569,0.466667,0.705882}%
\pgfsetfillcolor{currentfill}%
\pgfsetfillopacity{0.999214}%
\pgfsetlinewidth{1.003750pt}%
\definecolor{currentstroke}{rgb}{0.121569,0.466667,0.705882}%
\pgfsetstrokecolor{currentstroke}%
\pgfsetstrokeopacity{0.999214}%
\pgfsetdash{}{0pt}%
\pgfpathmoveto{\pgfqpoint{2.384933in}{1.329938in}}%
\pgfpathcurveto{\pgfqpoint{2.393170in}{1.329938in}}{\pgfqpoint{2.401070in}{1.333210in}}{\pgfqpoint{2.406894in}{1.339034in}}%
\pgfpathcurveto{\pgfqpoint{2.412718in}{1.344858in}}{\pgfqpoint{2.415990in}{1.352758in}}{\pgfqpoint{2.415990in}{1.360994in}}%
\pgfpathcurveto{\pgfqpoint{2.415990in}{1.369231in}}{\pgfqpoint{2.412718in}{1.377131in}}{\pgfqpoint{2.406894in}{1.382955in}}%
\pgfpathcurveto{\pgfqpoint{2.401070in}{1.388779in}}{\pgfqpoint{2.393170in}{1.392051in}}{\pgfqpoint{2.384933in}{1.392051in}}%
\pgfpathcurveto{\pgfqpoint{2.376697in}{1.392051in}}{\pgfqpoint{2.368797in}{1.388779in}}{\pgfqpoint{2.362973in}{1.382955in}}%
\pgfpathcurveto{\pgfqpoint{2.357149in}{1.377131in}}{\pgfqpoint{2.353877in}{1.369231in}}{\pgfqpoint{2.353877in}{1.360994in}}%
\pgfpathcurveto{\pgfqpoint{2.353877in}{1.352758in}}{\pgfqpoint{2.357149in}{1.344858in}}{\pgfqpoint{2.362973in}{1.339034in}}%
\pgfpathcurveto{\pgfqpoint{2.368797in}{1.333210in}}{\pgfqpoint{2.376697in}{1.329938in}}{\pgfqpoint{2.384933in}{1.329938in}}%
\pgfpathclose%
\pgfusepath{stroke,fill}%
\end{pgfscope}%
\begin{pgfscope}%
\pgfpathrectangle{\pgfqpoint{0.100000in}{0.212622in}}{\pgfqpoint{3.696000in}{3.696000in}}%
\pgfusepath{clip}%
\pgfsetbuttcap%
\pgfsetroundjoin%
\definecolor{currentfill}{rgb}{0.121569,0.466667,0.705882}%
\pgfsetfillcolor{currentfill}%
\pgfsetfillopacity{0.999388}%
\pgfsetlinewidth{1.003750pt}%
\definecolor{currentstroke}{rgb}{0.121569,0.466667,0.705882}%
\pgfsetstrokecolor{currentstroke}%
\pgfsetstrokeopacity{0.999388}%
\pgfsetdash{}{0pt}%
\pgfpathmoveto{\pgfqpoint{2.404987in}{1.320609in}}%
\pgfpathcurveto{\pgfqpoint{2.413223in}{1.320609in}}{\pgfqpoint{2.421123in}{1.323881in}}{\pgfqpoint{2.426947in}{1.329705in}}%
\pgfpathcurveto{\pgfqpoint{2.432771in}{1.335529in}}{\pgfqpoint{2.436043in}{1.343429in}}{\pgfqpoint{2.436043in}{1.351665in}}%
\pgfpathcurveto{\pgfqpoint{2.436043in}{1.359901in}}{\pgfqpoint{2.432771in}{1.367802in}}{\pgfqpoint{2.426947in}{1.373625in}}%
\pgfpathcurveto{\pgfqpoint{2.421123in}{1.379449in}}{\pgfqpoint{2.413223in}{1.382722in}}{\pgfqpoint{2.404987in}{1.382722in}}%
\pgfpathcurveto{\pgfqpoint{2.396751in}{1.382722in}}{\pgfqpoint{2.388851in}{1.379449in}}{\pgfqpoint{2.383027in}{1.373625in}}%
\pgfpathcurveto{\pgfqpoint{2.377203in}{1.367802in}}{\pgfqpoint{2.373930in}{1.359901in}}{\pgfqpoint{2.373930in}{1.351665in}}%
\pgfpathcurveto{\pgfqpoint{2.373930in}{1.343429in}}{\pgfqpoint{2.377203in}{1.335529in}}{\pgfqpoint{2.383027in}{1.329705in}}%
\pgfpathcurveto{\pgfqpoint{2.388851in}{1.323881in}}{\pgfqpoint{2.396751in}{1.320609in}}{\pgfqpoint{2.404987in}{1.320609in}}%
\pgfpathclose%
\pgfusepath{stroke,fill}%
\end{pgfscope}%
\begin{pgfscope}%
\pgfpathrectangle{\pgfqpoint{0.100000in}{0.212622in}}{\pgfqpoint{3.696000in}{3.696000in}}%
\pgfusepath{clip}%
\pgfsetbuttcap%
\pgfsetroundjoin%
\definecolor{currentfill}{rgb}{0.121569,0.466667,0.705882}%
\pgfsetfillcolor{currentfill}%
\pgfsetfillopacity{0.999735}%
\pgfsetlinewidth{1.003750pt}%
\definecolor{currentstroke}{rgb}{0.121569,0.466667,0.705882}%
\pgfsetstrokecolor{currentstroke}%
\pgfsetstrokeopacity{0.999735}%
\pgfsetdash{}{0pt}%
\pgfpathmoveto{\pgfqpoint{2.391363in}{1.325471in}}%
\pgfpathcurveto{\pgfqpoint{2.399599in}{1.325471in}}{\pgfqpoint{2.407499in}{1.328743in}}{\pgfqpoint{2.413323in}{1.334567in}}%
\pgfpathcurveto{\pgfqpoint{2.419147in}{1.340391in}}{\pgfqpoint{2.422419in}{1.348291in}}{\pgfqpoint{2.422419in}{1.356527in}}%
\pgfpathcurveto{\pgfqpoint{2.422419in}{1.364764in}}{\pgfqpoint{2.419147in}{1.372664in}}{\pgfqpoint{2.413323in}{1.378488in}}%
\pgfpathcurveto{\pgfqpoint{2.407499in}{1.384312in}}{\pgfqpoint{2.399599in}{1.387584in}}{\pgfqpoint{2.391363in}{1.387584in}}%
\pgfpathcurveto{\pgfqpoint{2.383126in}{1.387584in}}{\pgfqpoint{2.375226in}{1.384312in}}{\pgfqpoint{2.369402in}{1.378488in}}%
\pgfpathcurveto{\pgfqpoint{2.363579in}{1.372664in}}{\pgfqpoint{2.360306in}{1.364764in}}{\pgfqpoint{2.360306in}{1.356527in}}%
\pgfpathcurveto{\pgfqpoint{2.360306in}{1.348291in}}{\pgfqpoint{2.363579in}{1.340391in}}{\pgfqpoint{2.369402in}{1.334567in}}%
\pgfpathcurveto{\pgfqpoint{2.375226in}{1.328743in}}{\pgfqpoint{2.383126in}{1.325471in}}{\pgfqpoint{2.391363in}{1.325471in}}%
\pgfpathclose%
\pgfusepath{stroke,fill}%
\end{pgfscope}%
\begin{pgfscope}%
\pgfpathrectangle{\pgfqpoint{0.100000in}{0.212622in}}{\pgfqpoint{3.696000in}{3.696000in}}%
\pgfusepath{clip}%
\pgfsetbuttcap%
\pgfsetroundjoin%
\definecolor{currentfill}{rgb}{0.121569,0.466667,0.705882}%
\pgfsetfillcolor{currentfill}%
\pgfsetfillopacity{0.999861}%
\pgfsetlinewidth{1.003750pt}%
\definecolor{currentstroke}{rgb}{0.121569,0.466667,0.705882}%
\pgfsetstrokecolor{currentstroke}%
\pgfsetstrokeopacity{0.999861}%
\pgfsetdash{}{0pt}%
\pgfpathmoveto{\pgfqpoint{2.401906in}{1.320183in}}%
\pgfpathcurveto{\pgfqpoint{2.410142in}{1.320183in}}{\pgfqpoint{2.418043in}{1.323455in}}{\pgfqpoint{2.423866in}{1.329279in}}%
\pgfpathcurveto{\pgfqpoint{2.429690in}{1.335103in}}{\pgfqpoint{2.432963in}{1.343003in}}{\pgfqpoint{2.432963in}{1.351240in}}%
\pgfpathcurveto{\pgfqpoint{2.432963in}{1.359476in}}{\pgfqpoint{2.429690in}{1.367376in}}{\pgfqpoint{2.423866in}{1.373200in}}%
\pgfpathcurveto{\pgfqpoint{2.418043in}{1.379024in}}{\pgfqpoint{2.410142in}{1.382296in}}{\pgfqpoint{2.401906in}{1.382296in}}%
\pgfpathcurveto{\pgfqpoint{2.393670in}{1.382296in}}{\pgfqpoint{2.385770in}{1.379024in}}{\pgfqpoint{2.379946in}{1.373200in}}%
\pgfpathcurveto{\pgfqpoint{2.374122in}{1.367376in}}{\pgfqpoint{2.370850in}{1.359476in}}{\pgfqpoint{2.370850in}{1.351240in}}%
\pgfpathcurveto{\pgfqpoint{2.370850in}{1.343003in}}{\pgfqpoint{2.374122in}{1.335103in}}{\pgfqpoint{2.379946in}{1.329279in}}%
\pgfpathcurveto{\pgfqpoint{2.385770in}{1.323455in}}{\pgfqpoint{2.393670in}{1.320183in}}{\pgfqpoint{2.401906in}{1.320183in}}%
\pgfpathclose%
\pgfusepath{stroke,fill}%
\end{pgfscope}%
\begin{pgfscope}%
\pgfpathrectangle{\pgfqpoint{0.100000in}{0.212622in}}{\pgfqpoint{3.696000in}{3.696000in}}%
\pgfusepath{clip}%
\pgfsetbuttcap%
\pgfsetroundjoin%
\definecolor{currentfill}{rgb}{0.121569,0.466667,0.705882}%
\pgfsetfillcolor{currentfill}%
\pgfsetfillopacity{0.999920}%
\pgfsetlinewidth{1.003750pt}%
\definecolor{currentstroke}{rgb}{0.121569,0.466667,0.705882}%
\pgfsetstrokecolor{currentstroke}%
\pgfsetstrokeopacity{0.999920}%
\pgfsetdash{}{0pt}%
\pgfpathmoveto{\pgfqpoint{2.397839in}{1.321109in}}%
\pgfpathcurveto{\pgfqpoint{2.406076in}{1.321109in}}{\pgfqpoint{2.413976in}{1.324382in}}{\pgfqpoint{2.419800in}{1.330205in}}%
\pgfpathcurveto{\pgfqpoint{2.425624in}{1.336029in}}{\pgfqpoint{2.428896in}{1.343929in}}{\pgfqpoint{2.428896in}{1.352166in}}%
\pgfpathcurveto{\pgfqpoint{2.428896in}{1.360402in}}{\pgfqpoint{2.425624in}{1.368302in}}{\pgfqpoint{2.419800in}{1.374126in}}%
\pgfpathcurveto{\pgfqpoint{2.413976in}{1.379950in}}{\pgfqpoint{2.406076in}{1.383222in}}{\pgfqpoint{2.397839in}{1.383222in}}%
\pgfpathcurveto{\pgfqpoint{2.389603in}{1.383222in}}{\pgfqpoint{2.381703in}{1.379950in}}{\pgfqpoint{2.375879in}{1.374126in}}%
\pgfpathcurveto{\pgfqpoint{2.370055in}{1.368302in}}{\pgfqpoint{2.366783in}{1.360402in}}{\pgfqpoint{2.366783in}{1.352166in}}%
\pgfpathcurveto{\pgfqpoint{2.366783in}{1.343929in}}{\pgfqpoint{2.370055in}{1.336029in}}{\pgfqpoint{2.375879in}{1.330205in}}%
\pgfpathcurveto{\pgfqpoint{2.381703in}{1.324382in}}{\pgfqpoint{2.389603in}{1.321109in}}{\pgfqpoint{2.397839in}{1.321109in}}%
\pgfpathclose%
\pgfusepath{stroke,fill}%
\end{pgfscope}%
\begin{pgfscope}%
\pgfpathrectangle{\pgfqpoint{0.100000in}{0.212622in}}{\pgfqpoint{3.696000in}{3.696000in}}%
\pgfusepath{clip}%
\pgfsetbuttcap%
\pgfsetroundjoin%
\definecolor{currentfill}{rgb}{0.121569,0.466667,0.705882}%
\pgfsetfillcolor{currentfill}%
\pgfsetlinewidth{1.003750pt}%
\definecolor{currentstroke}{rgb}{0.121569,0.466667,0.705882}%
\pgfsetstrokecolor{currentstroke}%
\pgfsetdash{}{0pt}%
\pgfpathmoveto{\pgfqpoint{2.395697in}{1.322720in}}%
\pgfpathcurveto{\pgfqpoint{2.403933in}{1.322720in}}{\pgfqpoint{2.411833in}{1.325992in}}{\pgfqpoint{2.417657in}{1.331816in}}%
\pgfpathcurveto{\pgfqpoint{2.423481in}{1.337640in}}{\pgfqpoint{2.426753in}{1.345540in}}{\pgfqpoint{2.426753in}{1.353776in}}%
\pgfpathcurveto{\pgfqpoint{2.426753in}{1.362013in}}{\pgfqpoint{2.423481in}{1.369913in}}{\pgfqpoint{2.417657in}{1.375737in}}%
\pgfpathcurveto{\pgfqpoint{2.411833in}{1.381561in}}{\pgfqpoint{2.403933in}{1.384833in}}{\pgfqpoint{2.395697in}{1.384833in}}%
\pgfpathcurveto{\pgfqpoint{2.387460in}{1.384833in}}{\pgfqpoint{2.379560in}{1.381561in}}{\pgfqpoint{2.373736in}{1.375737in}}%
\pgfpathcurveto{\pgfqpoint{2.367912in}{1.369913in}}{\pgfqpoint{2.364640in}{1.362013in}}{\pgfqpoint{2.364640in}{1.353776in}}%
\pgfpathcurveto{\pgfqpoint{2.364640in}{1.345540in}}{\pgfqpoint{2.367912in}{1.337640in}}{\pgfqpoint{2.373736in}{1.331816in}}%
\pgfpathcurveto{\pgfqpoint{2.379560in}{1.325992in}}{\pgfqpoint{2.387460in}{1.322720in}}{\pgfqpoint{2.395697in}{1.322720in}}%
\pgfpathclose%
\pgfusepath{stroke,fill}%
\end{pgfscope}%
\begin{pgfscope}%
\pgfsetbuttcap%
\pgfsetmiterjoin%
\definecolor{currentfill}{rgb}{1.000000,1.000000,1.000000}%
\pgfsetfillcolor{currentfill}%
\pgfsetfillopacity{0.800000}%
\pgfsetlinewidth{1.003750pt}%
\definecolor{currentstroke}{rgb}{0.800000,0.800000,0.800000}%
\pgfsetstrokecolor{currentstroke}%
\pgfsetstrokeopacity{0.800000}%
\pgfsetdash{}{0pt}%
\pgfpathmoveto{\pgfqpoint{2.104889in}{3.216678in}}%
\pgfpathlineto{\pgfqpoint{3.698778in}{3.216678in}}%
\pgfpathquadraticcurveto{\pgfqpoint{3.726556in}{3.216678in}}{\pgfqpoint{3.726556in}{3.244456in}}%
\pgfpathlineto{\pgfqpoint{3.726556in}{3.811400in}}%
\pgfpathquadraticcurveto{\pgfqpoint{3.726556in}{3.839178in}}{\pgfqpoint{3.698778in}{3.839178in}}%
\pgfpathlineto{\pgfqpoint{2.104889in}{3.839178in}}%
\pgfpathquadraticcurveto{\pgfqpoint{2.077111in}{3.839178in}}{\pgfqpoint{2.077111in}{3.811400in}}%
\pgfpathlineto{\pgfqpoint{2.077111in}{3.244456in}}%
\pgfpathquadraticcurveto{\pgfqpoint{2.077111in}{3.216678in}}{\pgfqpoint{2.104889in}{3.216678in}}%
\pgfpathclose%
\pgfusepath{stroke,fill}%
\end{pgfscope}%
\begin{pgfscope}%
\pgfsetrectcap%
\pgfsetroundjoin%
\pgfsetlinewidth{1.505625pt}%
\definecolor{currentstroke}{rgb}{0.121569,0.466667,0.705882}%
\pgfsetstrokecolor{currentstroke}%
\pgfsetdash{}{0pt}%
\pgfpathmoveto{\pgfqpoint{2.132667in}{3.735011in}}%
\pgfpathlineto{\pgfqpoint{2.410444in}{3.735011in}}%
\pgfusepath{stroke}%
\end{pgfscope}%
\begin{pgfscope}%
\definecolor{textcolor}{rgb}{0.000000,0.000000,0.000000}%
\pgfsetstrokecolor{textcolor}%
\pgfsetfillcolor{textcolor}%
\pgftext[x=2.521555in,y=3.686400in,left,base]{\color{textcolor}\rmfamily\fontsize{10.000000}{12.000000}\selectfont Ground truth}%
\end{pgfscope}%
\begin{pgfscope}%
\pgfsetbuttcap%
\pgfsetroundjoin%
\definecolor{currentfill}{rgb}{0.121569,0.466667,0.705882}%
\pgfsetfillcolor{currentfill}%
\pgfsetlinewidth{1.003750pt}%
\definecolor{currentstroke}{rgb}{0.121569,0.466667,0.705882}%
\pgfsetstrokecolor{currentstroke}%
\pgfsetdash{}{0pt}%
\pgfsys@defobject{currentmarker}{\pgfqpoint{-0.031056in}{-0.031056in}}{\pgfqpoint{0.031056in}{0.031056in}}{%
\pgfpathmoveto{\pgfqpoint{0.000000in}{-0.031056in}}%
\pgfpathcurveto{\pgfqpoint{0.008236in}{-0.031056in}}{\pgfqpoint{0.016136in}{-0.027784in}}{\pgfqpoint{0.021960in}{-0.021960in}}%
\pgfpathcurveto{\pgfqpoint{0.027784in}{-0.016136in}}{\pgfqpoint{0.031056in}{-0.008236in}}{\pgfqpoint{0.031056in}{0.000000in}}%
\pgfpathcurveto{\pgfqpoint{0.031056in}{0.008236in}}{\pgfqpoint{0.027784in}{0.016136in}}{\pgfqpoint{0.021960in}{0.021960in}}%
\pgfpathcurveto{\pgfqpoint{0.016136in}{0.027784in}}{\pgfqpoint{0.008236in}{0.031056in}}{\pgfqpoint{0.000000in}{0.031056in}}%
\pgfpathcurveto{\pgfqpoint{-0.008236in}{0.031056in}}{\pgfqpoint{-0.016136in}{0.027784in}}{\pgfqpoint{-0.021960in}{0.021960in}}%
\pgfpathcurveto{\pgfqpoint{-0.027784in}{0.016136in}}{\pgfqpoint{-0.031056in}{0.008236in}}{\pgfqpoint{-0.031056in}{0.000000in}}%
\pgfpathcurveto{\pgfqpoint{-0.031056in}{-0.008236in}}{\pgfqpoint{-0.027784in}{-0.016136in}}{\pgfqpoint{-0.021960in}{-0.021960in}}%
\pgfpathcurveto{\pgfqpoint{-0.016136in}{-0.027784in}}{\pgfqpoint{-0.008236in}{-0.031056in}}{\pgfqpoint{0.000000in}{-0.031056in}}%
\pgfpathclose%
\pgfusepath{stroke,fill}%
}%
\begin{pgfscope}%
\pgfsys@transformshift{2.271555in}{3.529248in}%
\pgfsys@useobject{currentmarker}{}%
\end{pgfscope}%
\end{pgfscope}%
\begin{pgfscope}%
\definecolor{textcolor}{rgb}{0.000000,0.000000,0.000000}%
\pgfsetstrokecolor{textcolor}%
\pgfsetfillcolor{textcolor}%
\pgftext[x=2.521555in,y=3.492789in,left,base]{\color{textcolor}\rmfamily\fontsize{10.000000}{12.000000}\selectfont Estimated position}%
\end{pgfscope}%
\begin{pgfscope}%
\pgfsetbuttcap%
\pgfsetroundjoin%
\definecolor{currentfill}{rgb}{1.000000,0.498039,0.054902}%
\pgfsetfillcolor{currentfill}%
\pgfsetlinewidth{1.003750pt}%
\definecolor{currentstroke}{rgb}{1.000000,0.498039,0.054902}%
\pgfsetstrokecolor{currentstroke}%
\pgfsetdash{}{0pt}%
\pgfsys@defobject{currentmarker}{\pgfqpoint{-0.031056in}{-0.031056in}}{\pgfqpoint{0.031056in}{0.031056in}}{%
\pgfpathmoveto{\pgfqpoint{0.000000in}{-0.031056in}}%
\pgfpathcurveto{\pgfqpoint{0.008236in}{-0.031056in}}{\pgfqpoint{0.016136in}{-0.027784in}}{\pgfqpoint{0.021960in}{-0.021960in}}%
\pgfpathcurveto{\pgfqpoint{0.027784in}{-0.016136in}}{\pgfqpoint{0.031056in}{-0.008236in}}{\pgfqpoint{0.031056in}{0.000000in}}%
\pgfpathcurveto{\pgfqpoint{0.031056in}{0.008236in}}{\pgfqpoint{0.027784in}{0.016136in}}{\pgfqpoint{0.021960in}{0.021960in}}%
\pgfpathcurveto{\pgfqpoint{0.016136in}{0.027784in}}{\pgfqpoint{0.008236in}{0.031056in}}{\pgfqpoint{0.000000in}{0.031056in}}%
\pgfpathcurveto{\pgfqpoint{-0.008236in}{0.031056in}}{\pgfqpoint{-0.016136in}{0.027784in}}{\pgfqpoint{-0.021960in}{0.021960in}}%
\pgfpathcurveto{\pgfqpoint{-0.027784in}{0.016136in}}{\pgfqpoint{-0.031056in}{0.008236in}}{\pgfqpoint{-0.031056in}{0.000000in}}%
\pgfpathcurveto{\pgfqpoint{-0.031056in}{-0.008236in}}{\pgfqpoint{-0.027784in}{-0.016136in}}{\pgfqpoint{-0.021960in}{-0.021960in}}%
\pgfpathcurveto{\pgfqpoint{-0.016136in}{-0.027784in}}{\pgfqpoint{-0.008236in}{-0.031056in}}{\pgfqpoint{0.000000in}{-0.031056in}}%
\pgfpathclose%
\pgfusepath{stroke,fill}%
}%
\begin{pgfscope}%
\pgfsys@transformshift{2.271555in}{3.335637in}%
\pgfsys@useobject{currentmarker}{}%
\end{pgfscope}%
\end{pgfscope}%
\begin{pgfscope}%
\definecolor{textcolor}{rgb}{0.000000,0.000000,0.000000}%
\pgfsetstrokecolor{textcolor}%
\pgfsetfillcolor{textcolor}%
\pgftext[x=2.521555in,y=3.299178in,left,base]{\color{textcolor}\rmfamily\fontsize{10.000000}{12.000000}\selectfont Estimated turn}%
\end{pgfscope}%
\end{pgfpicture}%
\makeatother%
\endgroup%
}
%         \caption{EKF's 3D position estimation had the lowest turn error for the 16-meter side triangle experiment.}
%         \label{fig:triangle16_3D}
%     \end{subfigure}
%     \caption{Position estimation by the best performing algorithms in the 16-meter side triangle experiment.}
%     \label{fig:triangle16}
% \end{figure}

% \subsubsection{28 meter}

% For the 28-meter triangle experiment, the Mahony algorithm which had the lowest displacement error with an average of 2.93 meters (3.49\% of error margin), and EKF with an average of 5.72 meters of turn error (6.81\% of error margin).

% \begin{figure}[!h]
%     \centering
%     \begin{table}[H]
    \begin{center}
        \begin{tabular}[t]{lcccc}
            \hline
            Algorithm                   & Displacement Error[$m$] & Displacement Error[\%]      & Turn Error[$m$]  & Turn Error[\%]             \\
            \hline 
            AngularRate            & 21.96  & 26.15 & 38.76 & 46.15              \\            AQUA            & 6.64  & 7.91 & 15.91 & 18.94              \\            Complementary            & 14.51  & 17.27 & 17.72 & 21.09              \\            Davenport            & 9.67  & 11.51 & 8.61 & 10.25              \\            EKF            & 2.95  & 3.52 & 5.72 & 6.81              \\            FAMC            & 20.40  & 24.28 & 30.85 & 36.72              \\            FLAE            & 9.67  & 11.51 & 8.34 & 9.93              \\            Fourati            & 17.07  & 20.32 & 41.31 & 49.17              \\            Madgwick            & 9.29  & 11.06 & 9.42 & 11.22              \\            Mahony            & 2.85  & 3.39 & 5.48 & 6.52              \\            OLEQ            & 3.40  & 4.04 & 7.37 & 8.77              \\            QUEST            & 13.11  & 15.60 & 36.75 & 43.75              \\            ROLEQ            & 3.08  & 3.66 & 5.70 & 6.78              \\            SAAM            & 10.01  & 11.91 & 8.50 & 10.11              \\            Tilt            & 10.01  & 11.91 & 8.50 & 10.11              \\
            \hline
            Average & 10.31 & 12.27 & 16.59 & 19.75
        \end{tabular}
        \caption{Accelerometer Specifications. }
        \label{tab:accelerometer_specification}
    \end{center}
\end{table}
% \end{figure}

% \begin{figure}[!h]
%     \centering
%     \begin{subfigure}{0.49\textwidth}
%         \centering
%         \resizebox{1\linewidth}{!}{%% Creator: Matplotlib, PGF backend
%%
%% To include the figure in your LaTeX document, write
%%   \input{<filename>.pgf}
%%
%% Make sure the required packages are loaded in your preamble
%%   \usepackage{pgf}
%%
%% and, on pdftex
%%   \usepackage[utf8]{inputenc}\DeclareUnicodeCharacter{2212}{-}
%%
%% or, on luatex and xetex
%%   \usepackage{unicode-math}
%%
%% Figures using additional raster images can only be included by \input if
%% they are in the same directory as the main LaTeX file. For loading figures
%% from other directories you can use the `import` package
%%   \usepackage{import}
%%
%% and then include the figures with
%%   \import{<path to file>}{<filename>.pgf}
%%
%% Matplotlib used the following preamble
%%   \usepackage{fontspec}
%%
\begingroup%
\makeatletter%
\begin{pgfpicture}%
\pgfpathrectangle{\pgfpointorigin}{\pgfqpoint{4.342355in}{4.207622in}}%
\pgfusepath{use as bounding box, clip}%
\begin{pgfscope}%
\pgfsetbuttcap%
\pgfsetmiterjoin%
\definecolor{currentfill}{rgb}{1.000000,1.000000,1.000000}%
\pgfsetfillcolor{currentfill}%
\pgfsetlinewidth{0.000000pt}%
\definecolor{currentstroke}{rgb}{1.000000,1.000000,1.000000}%
\pgfsetstrokecolor{currentstroke}%
\pgfsetdash{}{0pt}%
\pgfpathmoveto{\pgfqpoint{0.000000in}{0.000000in}}%
\pgfpathlineto{\pgfqpoint{4.342355in}{0.000000in}}%
\pgfpathlineto{\pgfqpoint{4.342355in}{4.207622in}}%
\pgfpathlineto{\pgfqpoint{0.000000in}{4.207622in}}%
\pgfpathclose%
\pgfusepath{fill}%
\end{pgfscope}%
\begin{pgfscope}%
\pgfsetbuttcap%
\pgfsetmiterjoin%
\definecolor{currentfill}{rgb}{1.000000,1.000000,1.000000}%
\pgfsetfillcolor{currentfill}%
\pgfsetlinewidth{0.000000pt}%
\definecolor{currentstroke}{rgb}{0.000000,0.000000,0.000000}%
\pgfsetstrokecolor{currentstroke}%
\pgfsetstrokeopacity{0.000000}%
\pgfsetdash{}{0pt}%
\pgfpathmoveto{\pgfqpoint{0.100000in}{0.212622in}}%
\pgfpathlineto{\pgfqpoint{3.796000in}{0.212622in}}%
\pgfpathlineto{\pgfqpoint{3.796000in}{3.908622in}}%
\pgfpathlineto{\pgfqpoint{0.100000in}{3.908622in}}%
\pgfpathclose%
\pgfusepath{fill}%
\end{pgfscope}%
\begin{pgfscope}%
\pgfsetbuttcap%
\pgfsetmiterjoin%
\definecolor{currentfill}{rgb}{0.950000,0.950000,0.950000}%
\pgfsetfillcolor{currentfill}%
\pgfsetfillopacity{0.500000}%
\pgfsetlinewidth{1.003750pt}%
\definecolor{currentstroke}{rgb}{0.950000,0.950000,0.950000}%
\pgfsetstrokecolor{currentstroke}%
\pgfsetstrokeopacity{0.500000}%
\pgfsetdash{}{0pt}%
\pgfpathmoveto{\pgfqpoint{0.379073in}{1.123938in}}%
\pgfpathlineto{\pgfqpoint{1.599613in}{2.147018in}}%
\pgfpathlineto{\pgfqpoint{1.582647in}{3.622484in}}%
\pgfpathlineto{\pgfqpoint{0.303698in}{2.689165in}}%
\pgfusepath{stroke,fill}%
\end{pgfscope}%
\begin{pgfscope}%
\pgfsetbuttcap%
\pgfsetmiterjoin%
\definecolor{currentfill}{rgb}{0.900000,0.900000,0.900000}%
\pgfsetfillcolor{currentfill}%
\pgfsetfillopacity{0.500000}%
\pgfsetlinewidth{1.003750pt}%
\definecolor{currentstroke}{rgb}{0.900000,0.900000,0.900000}%
\pgfsetstrokecolor{currentstroke}%
\pgfsetstrokeopacity{0.500000}%
\pgfsetdash{}{0pt}%
\pgfpathmoveto{\pgfqpoint{1.599613in}{2.147018in}}%
\pgfpathlineto{\pgfqpoint{3.558144in}{1.577751in}}%
\pgfpathlineto{\pgfqpoint{3.628038in}{3.104037in}}%
\pgfpathlineto{\pgfqpoint{1.582647in}{3.622484in}}%
\pgfusepath{stroke,fill}%
\end{pgfscope}%
\begin{pgfscope}%
\pgfsetbuttcap%
\pgfsetmiterjoin%
\definecolor{currentfill}{rgb}{0.925000,0.925000,0.925000}%
\pgfsetfillcolor{currentfill}%
\pgfsetfillopacity{0.500000}%
\pgfsetlinewidth{1.003750pt}%
\definecolor{currentstroke}{rgb}{0.925000,0.925000,0.925000}%
\pgfsetstrokecolor{currentstroke}%
\pgfsetstrokeopacity{0.500000}%
\pgfsetdash{}{0pt}%
\pgfpathmoveto{\pgfqpoint{0.379073in}{1.123938in}}%
\pgfpathlineto{\pgfqpoint{2.455212in}{0.445871in}}%
\pgfpathlineto{\pgfqpoint{3.558144in}{1.577751in}}%
\pgfpathlineto{\pgfqpoint{1.599613in}{2.147018in}}%
\pgfusepath{stroke,fill}%
\end{pgfscope}%
\begin{pgfscope}%
\pgfsetrectcap%
\pgfsetroundjoin%
\pgfsetlinewidth{0.803000pt}%
\definecolor{currentstroke}{rgb}{0.000000,0.000000,0.000000}%
\pgfsetstrokecolor{currentstroke}%
\pgfsetdash{}{0pt}%
\pgfpathmoveto{\pgfqpoint{0.379073in}{1.123938in}}%
\pgfpathlineto{\pgfqpoint{2.455212in}{0.445871in}}%
\pgfusepath{stroke}%
\end{pgfscope}%
\begin{pgfscope}%
\definecolor{textcolor}{rgb}{0.000000,0.000000,0.000000}%
\pgfsetstrokecolor{textcolor}%
\pgfsetfillcolor{textcolor}%
\pgftext[x=0.730374in, y=0.408886in, left, base,rotate=341.912962]{\color{textcolor}\rmfamily\fontsize{10.000000}{12.000000}\selectfont Position X [\(\displaystyle m\)]}%
\end{pgfscope}%
\begin{pgfscope}%
\pgfsetbuttcap%
\pgfsetroundjoin%
\pgfsetlinewidth{0.803000pt}%
\definecolor{currentstroke}{rgb}{0.690196,0.690196,0.690196}%
\pgfsetstrokecolor{currentstroke}%
\pgfsetdash{}{0pt}%
\pgfpathmoveto{\pgfqpoint{0.631375in}{1.041536in}}%
\pgfpathlineto{\pgfqpoint{1.838547in}{2.077570in}}%
\pgfpathlineto{\pgfqpoint{1.831715in}{3.559352in}}%
\pgfusepath{stroke}%
\end{pgfscope}%
\begin{pgfscope}%
\pgfsetbuttcap%
\pgfsetroundjoin%
\pgfsetlinewidth{0.803000pt}%
\definecolor{currentstroke}{rgb}{0.690196,0.690196,0.690196}%
\pgfsetstrokecolor{currentstroke}%
\pgfsetdash{}{0pt}%
\pgfpathmoveto{\pgfqpoint{0.921857in}{0.946665in}}%
\pgfpathlineto{\pgfqpoint{2.113321in}{1.997704in}}%
\pgfpathlineto{\pgfqpoint{2.118302in}{3.486711in}}%
\pgfusepath{stroke}%
\end{pgfscope}%
\begin{pgfscope}%
\pgfsetbuttcap%
\pgfsetroundjoin%
\pgfsetlinewidth{0.803000pt}%
\definecolor{currentstroke}{rgb}{0.690196,0.690196,0.690196}%
\pgfsetstrokecolor{currentstroke}%
\pgfsetdash{}{0pt}%
\pgfpathmoveto{\pgfqpoint{1.216744in}{0.850354in}}%
\pgfpathlineto{\pgfqpoint{2.391915in}{1.916728in}}%
\pgfpathlineto{\pgfqpoint{2.409046in}{3.413015in}}%
\pgfusepath{stroke}%
\end{pgfscope}%
\begin{pgfscope}%
\pgfsetbuttcap%
\pgfsetroundjoin%
\pgfsetlinewidth{0.803000pt}%
\definecolor{currentstroke}{rgb}{0.690196,0.690196,0.690196}%
\pgfsetstrokecolor{currentstroke}%
\pgfsetdash{}{0pt}%
\pgfpathmoveto{\pgfqpoint{1.516140in}{0.752572in}}%
\pgfpathlineto{\pgfqpoint{2.674409in}{1.834618in}}%
\pgfpathlineto{\pgfqpoint{2.704040in}{3.338243in}}%
\pgfusepath{stroke}%
\end{pgfscope}%
\begin{pgfscope}%
\pgfsetbuttcap%
\pgfsetroundjoin%
\pgfsetlinewidth{0.803000pt}%
\definecolor{currentstroke}{rgb}{0.690196,0.690196,0.690196}%
\pgfsetstrokecolor{currentstroke}%
\pgfsetdash{}{0pt}%
\pgfpathmoveto{\pgfqpoint{1.820146in}{0.653283in}}%
\pgfpathlineto{\pgfqpoint{2.960888in}{1.751350in}}%
\pgfpathlineto{\pgfqpoint{3.003377in}{3.262370in}}%
\pgfusepath{stroke}%
\end{pgfscope}%
\begin{pgfscope}%
\pgfsetbuttcap%
\pgfsetroundjoin%
\pgfsetlinewidth{0.803000pt}%
\definecolor{currentstroke}{rgb}{0.690196,0.690196,0.690196}%
\pgfsetstrokecolor{currentstroke}%
\pgfsetdash{}{0pt}%
\pgfpathmoveto{\pgfqpoint{2.128872in}{0.552454in}}%
\pgfpathlineto{\pgfqpoint{3.251434in}{1.666899in}}%
\pgfpathlineto{\pgfqpoint{3.307154in}{3.185371in}}%
\pgfusepath{stroke}%
\end{pgfscope}%
\begin{pgfscope}%
\pgfsetrectcap%
\pgfsetroundjoin%
\pgfsetlinewidth{0.803000pt}%
\definecolor{currentstroke}{rgb}{0.000000,0.000000,0.000000}%
\pgfsetstrokecolor{currentstroke}%
\pgfsetdash{}{0pt}%
\pgfpathmoveto{\pgfqpoint{0.641890in}{1.050560in}}%
\pgfpathlineto{\pgfqpoint{0.610300in}{1.023449in}}%
\pgfusepath{stroke}%
\end{pgfscope}%
\begin{pgfscope}%
\definecolor{textcolor}{rgb}{0.000000,0.000000,0.000000}%
\pgfsetstrokecolor{textcolor}%
\pgfsetfillcolor{textcolor}%
\pgftext[x=0.526932in,y=0.822457in,,top]{\color{textcolor}\rmfamily\fontsize{10.000000}{12.000000}\selectfont \(\displaystyle {0}\)}%
\end{pgfscope}%
\begin{pgfscope}%
\pgfsetrectcap%
\pgfsetroundjoin%
\pgfsetlinewidth{0.803000pt}%
\definecolor{currentstroke}{rgb}{0.000000,0.000000,0.000000}%
\pgfsetstrokecolor{currentstroke}%
\pgfsetdash{}{0pt}%
\pgfpathmoveto{\pgfqpoint{0.932241in}{0.955825in}}%
\pgfpathlineto{\pgfqpoint{0.901043in}{0.928304in}}%
\pgfusepath{stroke}%
\end{pgfscope}%
\begin{pgfscope}%
\definecolor{textcolor}{rgb}{0.000000,0.000000,0.000000}%
\pgfsetstrokecolor{textcolor}%
\pgfsetfillcolor{textcolor}%
\pgftext[x=0.817716in,y=0.725574in,,top]{\color{textcolor}\rmfamily\fontsize{10.000000}{12.000000}\selectfont \(\displaystyle {5}\)}%
\end{pgfscope}%
\begin{pgfscope}%
\pgfsetrectcap%
\pgfsetroundjoin%
\pgfsetlinewidth{0.803000pt}%
\definecolor{currentstroke}{rgb}{0.000000,0.000000,0.000000}%
\pgfsetstrokecolor{currentstroke}%
\pgfsetdash{}{0pt}%
\pgfpathmoveto{\pgfqpoint{1.226993in}{0.859654in}}%
\pgfpathlineto{\pgfqpoint{1.196202in}{0.831714in}}%
\pgfusepath{stroke}%
\end{pgfscope}%
\begin{pgfscope}%
\definecolor{textcolor}{rgb}{0.000000,0.000000,0.000000}%
\pgfsetstrokecolor{textcolor}%
\pgfsetfillcolor{textcolor}%
\pgftext[x=1.112927in,y=0.627215in,,top]{\color{textcolor}\rmfamily\fontsize{10.000000}{12.000000}\selectfont \(\displaystyle {10}\)}%
\end{pgfscope}%
\begin{pgfscope}%
\pgfsetrectcap%
\pgfsetroundjoin%
\pgfsetlinewidth{0.803000pt}%
\definecolor{currentstroke}{rgb}{0.000000,0.000000,0.000000}%
\pgfsetstrokecolor{currentstroke}%
\pgfsetdash{}{0pt}%
\pgfpathmoveto{\pgfqpoint{1.526247in}{0.762014in}}%
\pgfpathlineto{\pgfqpoint{1.495880in}{0.733645in}}%
\pgfusepath{stroke}%
\end{pgfscope}%
\begin{pgfscope}%
\definecolor{textcolor}{rgb}{0.000000,0.000000,0.000000}%
\pgfsetstrokecolor{textcolor}%
\pgfsetfillcolor{textcolor}%
\pgftext[x=1.412668in,y=0.527348in,,top]{\color{textcolor}\rmfamily\fontsize{10.000000}{12.000000}\selectfont \(\displaystyle {15}\)}%
\end{pgfscope}%
\begin{pgfscope}%
\pgfsetrectcap%
\pgfsetroundjoin%
\pgfsetlinewidth{0.803000pt}%
\definecolor{currentstroke}{rgb}{0.000000,0.000000,0.000000}%
\pgfsetstrokecolor{currentstroke}%
\pgfsetdash{}{0pt}%
\pgfpathmoveto{\pgfqpoint{1.830107in}{0.662872in}}%
\pgfpathlineto{\pgfqpoint{1.800180in}{0.634064in}}%
\pgfusepath{stroke}%
\end{pgfscope}%
\begin{pgfscope}%
\definecolor{textcolor}{rgb}{0.000000,0.000000,0.000000}%
\pgfsetstrokecolor{textcolor}%
\pgfsetfillcolor{textcolor}%
\pgftext[x=1.717044in,y=0.425937in,,top]{\color{textcolor}\rmfamily\fontsize{10.000000}{12.000000}\selectfont \(\displaystyle {20}\)}%
\end{pgfscope}%
\begin{pgfscope}%
\pgfsetrectcap%
\pgfsetroundjoin%
\pgfsetlinewidth{0.803000pt}%
\definecolor{currentstroke}{rgb}{0.000000,0.000000,0.000000}%
\pgfsetstrokecolor{currentstroke}%
\pgfsetdash{}{0pt}%
\pgfpathmoveto{\pgfqpoint{2.138680in}{0.562191in}}%
\pgfpathlineto{\pgfqpoint{2.109210in}{0.532934in}}%
\pgfusepath{stroke}%
\end{pgfscope}%
\begin{pgfscope}%
\definecolor{textcolor}{rgb}{0.000000,0.000000,0.000000}%
\pgfsetstrokecolor{textcolor}%
\pgfsetfillcolor{textcolor}%
\pgftext[x=2.026162in,y=0.322945in,,top]{\color{textcolor}\rmfamily\fontsize{10.000000}{12.000000}\selectfont \(\displaystyle {25}\)}%
\end{pgfscope}%
\begin{pgfscope}%
\pgfsetrectcap%
\pgfsetroundjoin%
\pgfsetlinewidth{0.803000pt}%
\definecolor{currentstroke}{rgb}{0.000000,0.000000,0.000000}%
\pgfsetstrokecolor{currentstroke}%
\pgfsetdash{}{0pt}%
\pgfpathmoveto{\pgfqpoint{3.558144in}{1.577751in}}%
\pgfpathlineto{\pgfqpoint{2.455212in}{0.445871in}}%
\pgfusepath{stroke}%
\end{pgfscope}%
\begin{pgfscope}%
\definecolor{textcolor}{rgb}{0.000000,0.000000,0.000000}%
\pgfsetstrokecolor{textcolor}%
\pgfsetfillcolor{textcolor}%
\pgftext[x=3.120747in, y=0.305657in, left, base,rotate=45.742112]{\color{textcolor}\rmfamily\fontsize{10.000000}{12.000000}\selectfont Position Y [\(\displaystyle m\)]}%
\end{pgfscope}%
\begin{pgfscope}%
\pgfsetbuttcap%
\pgfsetroundjoin%
\pgfsetlinewidth{0.803000pt}%
\definecolor{currentstroke}{rgb}{0.690196,0.690196,0.690196}%
\pgfsetstrokecolor{currentstroke}%
\pgfsetdash{}{0pt}%
\pgfpathmoveto{\pgfqpoint{0.473082in}{2.812774in}}%
\pgfpathlineto{\pgfqpoint{0.540190in}{1.258989in}}%
\pgfpathlineto{\pgfqpoint{2.601363in}{0.595857in}}%
\pgfusepath{stroke}%
\end{pgfscope}%
\begin{pgfscope}%
\pgfsetbuttcap%
\pgfsetroundjoin%
\pgfsetlinewidth{0.803000pt}%
\definecolor{currentstroke}{rgb}{0.690196,0.690196,0.690196}%
\pgfsetstrokecolor{currentstroke}%
\pgfsetdash{}{0pt}%
\pgfpathmoveto{\pgfqpoint{0.669073in}{2.955799in}}%
\pgfpathlineto{\pgfqpoint{0.726817in}{1.415423in}}%
\pgfpathlineto{\pgfqpoint{2.770441in}{0.769373in}}%
\pgfusepath{stroke}%
\end{pgfscope}%
\begin{pgfscope}%
\pgfsetbuttcap%
\pgfsetroundjoin%
\pgfsetlinewidth{0.803000pt}%
\definecolor{currentstroke}{rgb}{0.690196,0.690196,0.690196}%
\pgfsetstrokecolor{currentstroke}%
\pgfsetdash{}{0pt}%
\pgfpathmoveto{\pgfqpoint{0.859955in}{3.095096in}}%
\pgfpathlineto{\pgfqpoint{0.908788in}{1.567955in}}%
\pgfpathlineto{\pgfqpoint{2.935082in}{0.938335in}}%
\pgfusepath{stroke}%
\end{pgfscope}%
\begin{pgfscope}%
\pgfsetbuttcap%
\pgfsetroundjoin%
\pgfsetlinewidth{0.803000pt}%
\definecolor{currentstroke}{rgb}{0.690196,0.690196,0.690196}%
\pgfsetstrokecolor{currentstroke}%
\pgfsetdash{}{0pt}%
\pgfpathmoveto{\pgfqpoint{1.045927in}{3.230810in}}%
\pgfpathlineto{\pgfqpoint{1.086277in}{1.716729in}}%
\pgfpathlineto{\pgfqpoint{3.095457in}{1.102920in}}%
\pgfusepath{stroke}%
\end{pgfscope}%
\begin{pgfscope}%
\pgfsetbuttcap%
\pgfsetroundjoin%
\pgfsetlinewidth{0.803000pt}%
\definecolor{currentstroke}{rgb}{0.690196,0.690196,0.690196}%
\pgfsetstrokecolor{currentstroke}%
\pgfsetdash{}{0pt}%
\pgfpathmoveto{\pgfqpoint{1.227175in}{3.363077in}}%
\pgfpathlineto{\pgfqpoint{1.259445in}{1.861883in}}%
\pgfpathlineto{\pgfqpoint{3.251732in}{1.263296in}}%
\pgfusepath{stroke}%
\end{pgfscope}%
\begin{pgfscope}%
\pgfsetbuttcap%
\pgfsetroundjoin%
\pgfsetlinewidth{0.803000pt}%
\definecolor{currentstroke}{rgb}{0.690196,0.690196,0.690196}%
\pgfsetstrokecolor{currentstroke}%
\pgfsetdash{}{0pt}%
\pgfpathmoveto{\pgfqpoint{1.403877in}{3.492026in}}%
\pgfpathlineto{\pgfqpoint{1.428450in}{2.003546in}}%
\pgfpathlineto{\pgfqpoint{3.404060in}{1.419622in}}%
\pgfusepath{stroke}%
\end{pgfscope}%
\begin{pgfscope}%
\pgfsetrectcap%
\pgfsetroundjoin%
\pgfsetlinewidth{0.803000pt}%
\definecolor{currentstroke}{rgb}{0.000000,0.000000,0.000000}%
\pgfsetstrokecolor{currentstroke}%
\pgfsetdash{}{0pt}%
\pgfpathmoveto{\pgfqpoint{2.583998in}{0.601444in}}%
\pgfpathlineto{\pgfqpoint{2.636137in}{0.584670in}}%
\pgfusepath{stroke}%
\end{pgfscope}%
\begin{pgfscope}%
\definecolor{textcolor}{rgb}{0.000000,0.000000,0.000000}%
\pgfsetstrokecolor{textcolor}%
\pgfsetfillcolor{textcolor}%
\pgftext[x=2.779324in,y=0.410484in,,top]{\color{textcolor}\rmfamily\fontsize{10.000000}{12.000000}\selectfont \(\displaystyle {0}\)}%
\end{pgfscope}%
\begin{pgfscope}%
\pgfsetrectcap%
\pgfsetroundjoin%
\pgfsetlinewidth{0.803000pt}%
\definecolor{currentstroke}{rgb}{0.000000,0.000000,0.000000}%
\pgfsetstrokecolor{currentstroke}%
\pgfsetdash{}{0pt}%
\pgfpathmoveto{\pgfqpoint{2.753236in}{0.774812in}}%
\pgfpathlineto{\pgfqpoint{2.804895in}{0.758481in}}%
\pgfusepath{stroke}%
\end{pgfscope}%
\begin{pgfscope}%
\definecolor{textcolor}{rgb}{0.000000,0.000000,0.000000}%
\pgfsetstrokecolor{textcolor}%
\pgfsetfillcolor{textcolor}%
\pgftext[x=2.946134in,y=0.586569in,,top]{\color{textcolor}\rmfamily\fontsize{10.000000}{12.000000}\selectfont \(\displaystyle {5}\)}%
\end{pgfscope}%
\begin{pgfscope}%
\pgfsetrectcap%
\pgfsetroundjoin%
\pgfsetlinewidth{0.803000pt}%
\definecolor{currentstroke}{rgb}{0.000000,0.000000,0.000000}%
\pgfsetstrokecolor{currentstroke}%
\pgfsetdash{}{0pt}%
\pgfpathmoveto{\pgfqpoint{2.918033in}{0.943632in}}%
\pgfpathlineto{\pgfqpoint{2.969221in}{0.927727in}}%
\pgfusepath{stroke}%
\end{pgfscope}%
\begin{pgfscope}%
\definecolor{textcolor}{rgb}{0.000000,0.000000,0.000000}%
\pgfsetstrokecolor{textcolor}%
\pgfsetfillcolor{textcolor}%
\pgftext[x=3.108563in,y=0.758029in,,top]{\color{textcolor}\rmfamily\fontsize{10.000000}{12.000000}\selectfont \(\displaystyle {10}\)}%
\end{pgfscope}%
\begin{pgfscope}%
\pgfsetrectcap%
\pgfsetroundjoin%
\pgfsetlinewidth{0.803000pt}%
\definecolor{currentstroke}{rgb}{0.000000,0.000000,0.000000}%
\pgfsetstrokecolor{currentstroke}%
\pgfsetdash{}{0pt}%
\pgfpathmoveto{\pgfqpoint{3.078564in}{1.108081in}}%
\pgfpathlineto{\pgfqpoint{3.129286in}{1.092585in}}%
\pgfusepath{stroke}%
\end{pgfscope}%
\begin{pgfscope}%
\definecolor{textcolor}{rgb}{0.000000,0.000000,0.000000}%
\pgfsetstrokecolor{textcolor}%
\pgfsetfillcolor{textcolor}%
\pgftext[x=3.266783in,y=0.925045in,,top]{\color{textcolor}\rmfamily\fontsize{10.000000}{12.000000}\selectfont \(\displaystyle {15}\)}%
\end{pgfscope}%
\begin{pgfscope}%
\pgfsetrectcap%
\pgfsetroundjoin%
\pgfsetlinewidth{0.803000pt}%
\definecolor{currentstroke}{rgb}{0.000000,0.000000,0.000000}%
\pgfsetstrokecolor{currentstroke}%
\pgfsetdash{}{0pt}%
\pgfpathmoveto{\pgfqpoint{3.234991in}{1.268326in}}%
\pgfpathlineto{\pgfqpoint{3.285255in}{1.253224in}}%
\pgfusepath{stroke}%
\end{pgfscope}%
\begin{pgfscope}%
\definecolor{textcolor}{rgb}{0.000000,0.000000,0.000000}%
\pgfsetstrokecolor{textcolor}%
\pgfsetfillcolor{textcolor}%
\pgftext[x=3.420953in,y=1.087787in,,top]{\color{textcolor}\rmfamily\fontsize{10.000000}{12.000000}\selectfont \(\displaystyle {20}\)}%
\end{pgfscope}%
\begin{pgfscope}%
\pgfsetrectcap%
\pgfsetroundjoin%
\pgfsetlinewidth{0.803000pt}%
\definecolor{currentstroke}{rgb}{0.000000,0.000000,0.000000}%
\pgfsetstrokecolor{currentstroke}%
\pgfsetdash{}{0pt}%
\pgfpathmoveto{\pgfqpoint{3.387469in}{1.424526in}}%
\pgfpathlineto{\pgfqpoint{3.437282in}{1.409803in}}%
\pgfusepath{stroke}%
\end{pgfscope}%
\begin{pgfscope}%
\definecolor{textcolor}{rgb}{0.000000,0.000000,0.000000}%
\pgfsetstrokecolor{textcolor}%
\pgfsetfillcolor{textcolor}%
\pgftext[x=3.571229in,y=1.246417in,,top]{\color{textcolor}\rmfamily\fontsize{10.000000}{12.000000}\selectfont \(\displaystyle {25}\)}%
\end{pgfscope}%
\begin{pgfscope}%
\pgfsetrectcap%
\pgfsetroundjoin%
\pgfsetlinewidth{0.803000pt}%
\definecolor{currentstroke}{rgb}{0.000000,0.000000,0.000000}%
\pgfsetstrokecolor{currentstroke}%
\pgfsetdash{}{0pt}%
\pgfpathmoveto{\pgfqpoint{3.558144in}{1.577751in}}%
\pgfpathlineto{\pgfqpoint{3.628038in}{3.104037in}}%
\pgfusepath{stroke}%
\end{pgfscope}%
\begin{pgfscope}%
\definecolor{textcolor}{rgb}{0.000000,0.000000,0.000000}%
\pgfsetstrokecolor{textcolor}%
\pgfsetfillcolor{textcolor}%
\pgftext[x=4.167903in, y=1.963517in, left, base,rotate=87.378092]{\color{textcolor}\rmfamily\fontsize{10.000000}{12.000000}\selectfont Position Z [\(\displaystyle m\)]}%
\end{pgfscope}%
\begin{pgfscope}%
\pgfsetbuttcap%
\pgfsetroundjoin%
\pgfsetlinewidth{0.803000pt}%
\definecolor{currentstroke}{rgb}{0.690196,0.690196,0.690196}%
\pgfsetstrokecolor{currentstroke}%
\pgfsetdash{}{0pt}%
\pgfpathmoveto{\pgfqpoint{3.568957in}{1.813878in}}%
\pgfpathlineto{\pgfqpoint{1.596984in}{2.375691in}}%
\pgfpathlineto{\pgfqpoint{0.367429in}{1.365745in}}%
\pgfusepath{stroke}%
\end{pgfscope}%
\begin{pgfscope}%
\pgfsetbuttcap%
\pgfsetroundjoin%
\pgfsetlinewidth{0.803000pt}%
\definecolor{currentstroke}{rgb}{0.690196,0.690196,0.690196}%
\pgfsetstrokecolor{currentstroke}%
\pgfsetdash{}{0pt}%
\pgfpathmoveto{\pgfqpoint{3.582824in}{2.116684in}}%
\pgfpathlineto{\pgfqpoint{1.593614in}{2.668718in}}%
\pgfpathlineto{\pgfqpoint{0.352487in}{1.676018in}}%
\pgfusepath{stroke}%
\end{pgfscope}%
\begin{pgfscope}%
\pgfsetbuttcap%
\pgfsetroundjoin%
\pgfsetlinewidth{0.803000pt}%
\definecolor{currentstroke}{rgb}{0.690196,0.690196,0.690196}%
\pgfsetstrokecolor{currentstroke}%
\pgfsetdash{}{0pt}%
\pgfpathmoveto{\pgfqpoint{3.596937in}{2.424882in}}%
\pgfpathlineto{\pgfqpoint{1.590188in}{2.966712in}}%
\pgfpathlineto{\pgfqpoint{0.337269in}{1.992031in}}%
\pgfusepath{stroke}%
\end{pgfscope}%
\begin{pgfscope}%
\pgfsetbuttcap%
\pgfsetroundjoin%
\pgfsetlinewidth{0.803000pt}%
\definecolor{currentstroke}{rgb}{0.690196,0.690196,0.690196}%
\pgfsetstrokecolor{currentstroke}%
\pgfsetdash{}{0pt}%
\pgfpathmoveto{\pgfqpoint{3.611304in}{2.738619in}}%
\pgfpathlineto{\pgfqpoint{1.586702in}{3.269800in}}%
\pgfpathlineto{\pgfqpoint{0.321767in}{2.313943in}}%
\pgfusepath{stroke}%
\end{pgfscope}%
\begin{pgfscope}%
\pgfsetbuttcap%
\pgfsetroundjoin%
\pgfsetlinewidth{0.803000pt}%
\definecolor{currentstroke}{rgb}{0.690196,0.690196,0.690196}%
\pgfsetstrokecolor{currentstroke}%
\pgfsetdash{}{0pt}%
\pgfpathmoveto{\pgfqpoint{3.625931in}{3.058045in}}%
\pgfpathlineto{\pgfqpoint{1.583157in}{3.578114in}}%
\pgfpathlineto{\pgfqpoint{0.305973in}{2.641922in}}%
\pgfusepath{stroke}%
\end{pgfscope}%
\begin{pgfscope}%
\pgfsetrectcap%
\pgfsetroundjoin%
\pgfsetlinewidth{0.803000pt}%
\definecolor{currentstroke}{rgb}{0.000000,0.000000,0.000000}%
\pgfsetstrokecolor{currentstroke}%
\pgfsetdash{}{0pt}%
\pgfpathmoveto{\pgfqpoint{3.552402in}{1.818595in}}%
\pgfpathlineto{\pgfqpoint{3.602108in}{1.804434in}}%
\pgfusepath{stroke}%
\end{pgfscope}%
\begin{pgfscope}%
\definecolor{textcolor}{rgb}{0.000000,0.000000,0.000000}%
\pgfsetstrokecolor{textcolor}%
\pgfsetfillcolor{textcolor}%
\pgftext[x=3.824129in,y=1.849596in,,top]{\color{textcolor}\rmfamily\fontsize{10.000000}{12.000000}\selectfont \(\displaystyle {0}\)}%
\end{pgfscope}%
\begin{pgfscope}%
\pgfsetrectcap%
\pgfsetroundjoin%
\pgfsetlinewidth{0.803000pt}%
\definecolor{currentstroke}{rgb}{0.000000,0.000000,0.000000}%
\pgfsetstrokecolor{currentstroke}%
\pgfsetdash{}{0pt}%
\pgfpathmoveto{\pgfqpoint{3.566116in}{2.121320in}}%
\pgfpathlineto{\pgfqpoint{3.616279in}{2.107400in}}%
\pgfusepath{stroke}%
\end{pgfscope}%
\begin{pgfscope}%
\definecolor{textcolor}{rgb}{0.000000,0.000000,0.000000}%
\pgfsetstrokecolor{textcolor}%
\pgfsetfillcolor{textcolor}%
\pgftext[x=3.840199in,y=2.151794in,,top]{\color{textcolor}\rmfamily\fontsize{10.000000}{12.000000}\selectfont \(\displaystyle {1}\)}%
\end{pgfscope}%
\begin{pgfscope}%
\pgfsetrectcap%
\pgfsetroundjoin%
\pgfsetlinewidth{0.803000pt}%
\definecolor{currentstroke}{rgb}{0.000000,0.000000,0.000000}%
\pgfsetstrokecolor{currentstroke}%
\pgfsetdash{}{0pt}%
\pgfpathmoveto{\pgfqpoint{3.580075in}{2.429435in}}%
\pgfpathlineto{\pgfqpoint{3.630702in}{2.415766in}}%
\pgfusepath{stroke}%
\end{pgfscope}%
\begin{pgfscope}%
\definecolor{textcolor}{rgb}{0.000000,0.000000,0.000000}%
\pgfsetstrokecolor{textcolor}%
\pgfsetfillcolor{textcolor}%
\pgftext[x=3.856554in,y=2.459357in,,top]{\color{textcolor}\rmfamily\fontsize{10.000000}{12.000000}\selectfont \(\displaystyle {2}\)}%
\end{pgfscope}%
\begin{pgfscope}%
\pgfsetrectcap%
\pgfsetroundjoin%
\pgfsetlinewidth{0.803000pt}%
\definecolor{currentstroke}{rgb}{0.000000,0.000000,0.000000}%
\pgfsetstrokecolor{currentstroke}%
\pgfsetdash{}{0pt}%
\pgfpathmoveto{\pgfqpoint{3.594285in}{2.743084in}}%
\pgfpathlineto{\pgfqpoint{3.645384in}{2.729678in}}%
\pgfusepath{stroke}%
\end{pgfscope}%
\begin{pgfscope}%
\definecolor{textcolor}{rgb}{0.000000,0.000000,0.000000}%
\pgfsetstrokecolor{textcolor}%
\pgfsetfillcolor{textcolor}%
\pgftext[x=3.873201in,y=2.772430in,,top]{\color{textcolor}\rmfamily\fontsize{10.000000}{12.000000}\selectfont \(\displaystyle {3}\)}%
\end{pgfscope}%
\begin{pgfscope}%
\pgfsetrectcap%
\pgfsetroundjoin%
\pgfsetlinewidth{0.803000pt}%
\definecolor{currentstroke}{rgb}{0.000000,0.000000,0.000000}%
\pgfsetstrokecolor{currentstroke}%
\pgfsetdash{}{0pt}%
\pgfpathmoveto{\pgfqpoint{3.608752in}{3.062418in}}%
\pgfpathlineto{\pgfqpoint{3.660333in}{3.049286in}}%
\pgfusepath{stroke}%
\end{pgfscope}%
\begin{pgfscope}%
\definecolor{textcolor}{rgb}{0.000000,0.000000,0.000000}%
\pgfsetstrokecolor{textcolor}%
\pgfsetfillcolor{textcolor}%
\pgftext[x=3.890150in,y=3.091162in,,top]{\color{textcolor}\rmfamily\fontsize{10.000000}{12.000000}\selectfont \(\displaystyle {4}\)}%
\end{pgfscope}%
\begin{pgfscope}%
\pgfpathrectangle{\pgfqpoint{0.100000in}{0.212622in}}{\pgfqpoint{3.696000in}{3.696000in}}%
\pgfusepath{clip}%
\pgfsetrectcap%
\pgfsetroundjoin%
\pgfsetlinewidth{1.505625pt}%
\definecolor{currentstroke}{rgb}{0.121569,0.466667,0.705882}%
\pgfsetstrokecolor{currentstroke}%
\pgfsetdash{}{0pt}%
\pgfpathmoveto{\pgfqpoint{0.782156in}{1.419499in}}%
\pgfpathlineto{\pgfqpoint{1.766134in}{2.247153in}}%
\pgfpathlineto{\pgfqpoint{2.467124in}{0.887769in}}%
\pgfpathlineto{\pgfqpoint{0.782156in}{1.419499in}}%
\pgfusepath{stroke}%
\end{pgfscope}%
\begin{pgfscope}%
\pgfpathrectangle{\pgfqpoint{0.100000in}{0.212622in}}{\pgfqpoint{3.696000in}{3.696000in}}%
\pgfusepath{clip}%
\pgfsetrectcap%
\pgfsetroundjoin%
\pgfsetlinewidth{1.505625pt}%
\definecolor{currentstroke}{rgb}{1.000000,0.000000,0.000000}%
\pgfsetstrokecolor{currentstroke}%
\pgfsetdash{}{0pt}%
\pgfpathmoveto{\pgfqpoint{0.781782in}{1.419170in}}%
\pgfpathlineto{\pgfqpoint{0.782156in}{1.419499in}}%
\pgfusepath{stroke}%
\end{pgfscope}%
\begin{pgfscope}%
\pgfpathrectangle{\pgfqpoint{0.100000in}{0.212622in}}{\pgfqpoint{3.696000in}{3.696000in}}%
\pgfusepath{clip}%
\pgfsetrectcap%
\pgfsetroundjoin%
\pgfsetlinewidth{1.505625pt}%
\definecolor{currentstroke}{rgb}{1.000000,0.000000,0.000000}%
\pgfsetstrokecolor{currentstroke}%
\pgfsetdash{}{0pt}%
\pgfpathmoveto{\pgfqpoint{0.781659in}{1.419435in}}%
\pgfpathlineto{\pgfqpoint{0.782156in}{1.419499in}}%
\pgfusepath{stroke}%
\end{pgfscope}%
\begin{pgfscope}%
\pgfpathrectangle{\pgfqpoint{0.100000in}{0.212622in}}{\pgfqpoint{3.696000in}{3.696000in}}%
\pgfusepath{clip}%
\pgfsetrectcap%
\pgfsetroundjoin%
\pgfsetlinewidth{1.505625pt}%
\definecolor{currentstroke}{rgb}{1.000000,0.000000,0.000000}%
\pgfsetstrokecolor{currentstroke}%
\pgfsetdash{}{0pt}%
\pgfpathmoveto{\pgfqpoint{0.781227in}{1.419828in}}%
\pgfpathlineto{\pgfqpoint{0.782156in}{1.419499in}}%
\pgfusepath{stroke}%
\end{pgfscope}%
\begin{pgfscope}%
\pgfpathrectangle{\pgfqpoint{0.100000in}{0.212622in}}{\pgfqpoint{3.696000in}{3.696000in}}%
\pgfusepath{clip}%
\pgfsetrectcap%
\pgfsetroundjoin%
\pgfsetlinewidth{1.505625pt}%
\definecolor{currentstroke}{rgb}{1.000000,0.000000,0.000000}%
\pgfsetstrokecolor{currentstroke}%
\pgfsetdash{}{0pt}%
\pgfpathmoveto{\pgfqpoint{0.780536in}{1.420679in}}%
\pgfpathlineto{\pgfqpoint{0.782156in}{1.419499in}}%
\pgfusepath{stroke}%
\end{pgfscope}%
\begin{pgfscope}%
\pgfpathrectangle{\pgfqpoint{0.100000in}{0.212622in}}{\pgfqpoint{3.696000in}{3.696000in}}%
\pgfusepath{clip}%
\pgfsetrectcap%
\pgfsetroundjoin%
\pgfsetlinewidth{1.505625pt}%
\definecolor{currentstroke}{rgb}{1.000000,0.000000,0.000000}%
\pgfsetstrokecolor{currentstroke}%
\pgfsetdash{}{0pt}%
\pgfpathmoveto{\pgfqpoint{0.779027in}{1.422111in}}%
\pgfpathlineto{\pgfqpoint{0.782156in}{1.419499in}}%
\pgfusepath{stroke}%
\end{pgfscope}%
\begin{pgfscope}%
\pgfpathrectangle{\pgfqpoint{0.100000in}{0.212622in}}{\pgfqpoint{3.696000in}{3.696000in}}%
\pgfusepath{clip}%
\pgfsetrectcap%
\pgfsetroundjoin%
\pgfsetlinewidth{1.505625pt}%
\definecolor{currentstroke}{rgb}{1.000000,0.000000,0.000000}%
\pgfsetstrokecolor{currentstroke}%
\pgfsetdash{}{0pt}%
\pgfpathmoveto{\pgfqpoint{0.777427in}{1.424046in}}%
\pgfpathlineto{\pgfqpoint{0.782156in}{1.419499in}}%
\pgfusepath{stroke}%
\end{pgfscope}%
\begin{pgfscope}%
\pgfpathrectangle{\pgfqpoint{0.100000in}{0.212622in}}{\pgfqpoint{3.696000in}{3.696000in}}%
\pgfusepath{clip}%
\pgfsetrectcap%
\pgfsetroundjoin%
\pgfsetlinewidth{1.505625pt}%
\definecolor{currentstroke}{rgb}{1.000000,0.000000,0.000000}%
\pgfsetstrokecolor{currentstroke}%
\pgfsetdash{}{0pt}%
\pgfpathmoveto{\pgfqpoint{0.775057in}{1.426507in}}%
\pgfpathlineto{\pgfqpoint{0.782156in}{1.419499in}}%
\pgfusepath{stroke}%
\end{pgfscope}%
\begin{pgfscope}%
\pgfpathrectangle{\pgfqpoint{0.100000in}{0.212622in}}{\pgfqpoint{3.696000in}{3.696000in}}%
\pgfusepath{clip}%
\pgfsetrectcap%
\pgfsetroundjoin%
\pgfsetlinewidth{1.505625pt}%
\definecolor{currentstroke}{rgb}{1.000000,0.000000,0.000000}%
\pgfsetstrokecolor{currentstroke}%
\pgfsetdash{}{0pt}%
\pgfpathmoveto{\pgfqpoint{0.772718in}{1.428743in}}%
\pgfpathlineto{\pgfqpoint{0.782156in}{1.419499in}}%
\pgfusepath{stroke}%
\end{pgfscope}%
\begin{pgfscope}%
\pgfpathrectangle{\pgfqpoint{0.100000in}{0.212622in}}{\pgfqpoint{3.696000in}{3.696000in}}%
\pgfusepath{clip}%
\pgfsetrectcap%
\pgfsetroundjoin%
\pgfsetlinewidth{1.505625pt}%
\definecolor{currentstroke}{rgb}{1.000000,0.000000,0.000000}%
\pgfsetstrokecolor{currentstroke}%
\pgfsetdash{}{0pt}%
\pgfpathmoveto{\pgfqpoint{0.771210in}{1.430418in}}%
\pgfpathlineto{\pgfqpoint{0.782156in}{1.419499in}}%
\pgfusepath{stroke}%
\end{pgfscope}%
\begin{pgfscope}%
\pgfpathrectangle{\pgfqpoint{0.100000in}{0.212622in}}{\pgfqpoint{3.696000in}{3.696000in}}%
\pgfusepath{clip}%
\pgfsetrectcap%
\pgfsetroundjoin%
\pgfsetlinewidth{1.505625pt}%
\definecolor{currentstroke}{rgb}{1.000000,0.000000,0.000000}%
\pgfsetstrokecolor{currentstroke}%
\pgfsetdash{}{0pt}%
\pgfpathmoveto{\pgfqpoint{0.770532in}{1.431185in}}%
\pgfpathlineto{\pgfqpoint{0.782156in}{1.419499in}}%
\pgfusepath{stroke}%
\end{pgfscope}%
\begin{pgfscope}%
\pgfpathrectangle{\pgfqpoint{0.100000in}{0.212622in}}{\pgfqpoint{3.696000in}{3.696000in}}%
\pgfusepath{clip}%
\pgfsetrectcap%
\pgfsetroundjoin%
\pgfsetlinewidth{1.505625pt}%
\definecolor{currentstroke}{rgb}{1.000000,0.000000,0.000000}%
\pgfsetstrokecolor{currentstroke}%
\pgfsetdash{}{0pt}%
\pgfpathmoveto{\pgfqpoint{0.770098in}{1.431696in}}%
\pgfpathlineto{\pgfqpoint{0.782156in}{1.419499in}}%
\pgfusepath{stroke}%
\end{pgfscope}%
\begin{pgfscope}%
\pgfpathrectangle{\pgfqpoint{0.100000in}{0.212622in}}{\pgfqpoint{3.696000in}{3.696000in}}%
\pgfusepath{clip}%
\pgfsetrectcap%
\pgfsetroundjoin%
\pgfsetlinewidth{1.505625pt}%
\definecolor{currentstroke}{rgb}{1.000000,0.000000,0.000000}%
\pgfsetstrokecolor{currentstroke}%
\pgfsetdash{}{0pt}%
\pgfpathmoveto{\pgfqpoint{0.769874in}{1.431909in}}%
\pgfpathlineto{\pgfqpoint{0.782156in}{1.419499in}}%
\pgfusepath{stroke}%
\end{pgfscope}%
\begin{pgfscope}%
\pgfpathrectangle{\pgfqpoint{0.100000in}{0.212622in}}{\pgfqpoint{3.696000in}{3.696000in}}%
\pgfusepath{clip}%
\pgfsetrectcap%
\pgfsetroundjoin%
\pgfsetlinewidth{1.505625pt}%
\definecolor{currentstroke}{rgb}{1.000000,0.000000,0.000000}%
\pgfsetstrokecolor{currentstroke}%
\pgfsetdash{}{0pt}%
\pgfpathmoveto{\pgfqpoint{0.769750in}{1.432071in}}%
\pgfpathlineto{\pgfqpoint{0.782156in}{1.419499in}}%
\pgfusepath{stroke}%
\end{pgfscope}%
\begin{pgfscope}%
\pgfpathrectangle{\pgfqpoint{0.100000in}{0.212622in}}{\pgfqpoint{3.696000in}{3.696000in}}%
\pgfusepath{clip}%
\pgfsetrectcap%
\pgfsetroundjoin%
\pgfsetlinewidth{1.505625pt}%
\definecolor{currentstroke}{rgb}{1.000000,0.000000,0.000000}%
\pgfsetstrokecolor{currentstroke}%
\pgfsetdash{}{0pt}%
\pgfpathmoveto{\pgfqpoint{0.769684in}{1.432140in}}%
\pgfpathlineto{\pgfqpoint{0.782156in}{1.419499in}}%
\pgfusepath{stroke}%
\end{pgfscope}%
\begin{pgfscope}%
\pgfpathrectangle{\pgfqpoint{0.100000in}{0.212622in}}{\pgfqpoint{3.696000in}{3.696000in}}%
\pgfusepath{clip}%
\pgfsetrectcap%
\pgfsetroundjoin%
\pgfsetlinewidth{1.505625pt}%
\definecolor{currentstroke}{rgb}{1.000000,0.000000,0.000000}%
\pgfsetstrokecolor{currentstroke}%
\pgfsetdash{}{0pt}%
\pgfpathmoveto{\pgfqpoint{0.769644in}{1.432188in}}%
\pgfpathlineto{\pgfqpoint{0.782156in}{1.419499in}}%
\pgfusepath{stroke}%
\end{pgfscope}%
\begin{pgfscope}%
\pgfpathrectangle{\pgfqpoint{0.100000in}{0.212622in}}{\pgfqpoint{3.696000in}{3.696000in}}%
\pgfusepath{clip}%
\pgfsetrectcap%
\pgfsetroundjoin%
\pgfsetlinewidth{1.505625pt}%
\definecolor{currentstroke}{rgb}{1.000000,0.000000,0.000000}%
\pgfsetstrokecolor{currentstroke}%
\pgfsetdash{}{0pt}%
\pgfpathmoveto{\pgfqpoint{0.769625in}{1.432214in}}%
\pgfpathlineto{\pgfqpoint{0.782156in}{1.419499in}}%
\pgfusepath{stroke}%
\end{pgfscope}%
\begin{pgfscope}%
\pgfpathrectangle{\pgfqpoint{0.100000in}{0.212622in}}{\pgfqpoint{3.696000in}{3.696000in}}%
\pgfusepath{clip}%
\pgfsetrectcap%
\pgfsetroundjoin%
\pgfsetlinewidth{1.505625pt}%
\definecolor{currentstroke}{rgb}{1.000000,0.000000,0.000000}%
\pgfsetstrokecolor{currentstroke}%
\pgfsetdash{}{0pt}%
\pgfpathmoveto{\pgfqpoint{0.769613in}{1.432230in}}%
\pgfpathlineto{\pgfqpoint{0.782156in}{1.419499in}}%
\pgfusepath{stroke}%
\end{pgfscope}%
\begin{pgfscope}%
\pgfpathrectangle{\pgfqpoint{0.100000in}{0.212622in}}{\pgfqpoint{3.696000in}{3.696000in}}%
\pgfusepath{clip}%
\pgfsetrectcap%
\pgfsetroundjoin%
\pgfsetlinewidth{1.505625pt}%
\definecolor{currentstroke}{rgb}{1.000000,0.000000,0.000000}%
\pgfsetstrokecolor{currentstroke}%
\pgfsetdash{}{0pt}%
\pgfpathmoveto{\pgfqpoint{0.769606in}{1.432237in}}%
\pgfpathlineto{\pgfqpoint{0.782156in}{1.419499in}}%
\pgfusepath{stroke}%
\end{pgfscope}%
\begin{pgfscope}%
\pgfpathrectangle{\pgfqpoint{0.100000in}{0.212622in}}{\pgfqpoint{3.696000in}{3.696000in}}%
\pgfusepath{clip}%
\pgfsetrectcap%
\pgfsetroundjoin%
\pgfsetlinewidth{1.505625pt}%
\definecolor{currentstroke}{rgb}{1.000000,0.000000,0.000000}%
\pgfsetstrokecolor{currentstroke}%
\pgfsetdash{}{0pt}%
\pgfpathmoveto{\pgfqpoint{0.769603in}{1.432242in}}%
\pgfpathlineto{\pgfqpoint{0.782156in}{1.419499in}}%
\pgfusepath{stroke}%
\end{pgfscope}%
\begin{pgfscope}%
\pgfpathrectangle{\pgfqpoint{0.100000in}{0.212622in}}{\pgfqpoint{3.696000in}{3.696000in}}%
\pgfusepath{clip}%
\pgfsetrectcap%
\pgfsetroundjoin%
\pgfsetlinewidth{1.505625pt}%
\definecolor{currentstroke}{rgb}{1.000000,0.000000,0.000000}%
\pgfsetstrokecolor{currentstroke}%
\pgfsetdash{}{0pt}%
\pgfpathmoveto{\pgfqpoint{0.769601in}{1.432244in}}%
\pgfpathlineto{\pgfqpoint{0.782156in}{1.419499in}}%
\pgfusepath{stroke}%
\end{pgfscope}%
\begin{pgfscope}%
\pgfpathrectangle{\pgfqpoint{0.100000in}{0.212622in}}{\pgfqpoint{3.696000in}{3.696000in}}%
\pgfusepath{clip}%
\pgfsetrectcap%
\pgfsetroundjoin%
\pgfsetlinewidth{1.505625pt}%
\definecolor{currentstroke}{rgb}{1.000000,0.000000,0.000000}%
\pgfsetstrokecolor{currentstroke}%
\pgfsetdash{}{0pt}%
\pgfpathmoveto{\pgfqpoint{0.769600in}{1.432245in}}%
\pgfpathlineto{\pgfqpoint{0.782156in}{1.419499in}}%
\pgfusepath{stroke}%
\end{pgfscope}%
\begin{pgfscope}%
\pgfpathrectangle{\pgfqpoint{0.100000in}{0.212622in}}{\pgfqpoint{3.696000in}{3.696000in}}%
\pgfusepath{clip}%
\pgfsetrectcap%
\pgfsetroundjoin%
\pgfsetlinewidth{1.505625pt}%
\definecolor{currentstroke}{rgb}{1.000000,0.000000,0.000000}%
\pgfsetstrokecolor{currentstroke}%
\pgfsetdash{}{0pt}%
\pgfpathmoveto{\pgfqpoint{0.769599in}{1.432245in}}%
\pgfpathlineto{\pgfqpoint{0.782156in}{1.419499in}}%
\pgfusepath{stroke}%
\end{pgfscope}%
\begin{pgfscope}%
\pgfpathrectangle{\pgfqpoint{0.100000in}{0.212622in}}{\pgfqpoint{3.696000in}{3.696000in}}%
\pgfusepath{clip}%
\pgfsetrectcap%
\pgfsetroundjoin%
\pgfsetlinewidth{1.505625pt}%
\definecolor{currentstroke}{rgb}{1.000000,0.000000,0.000000}%
\pgfsetstrokecolor{currentstroke}%
\pgfsetdash{}{0pt}%
\pgfpathmoveto{\pgfqpoint{0.769599in}{1.432245in}}%
\pgfpathlineto{\pgfqpoint{0.782156in}{1.419499in}}%
\pgfusepath{stroke}%
\end{pgfscope}%
\begin{pgfscope}%
\pgfpathrectangle{\pgfqpoint{0.100000in}{0.212622in}}{\pgfqpoint{3.696000in}{3.696000in}}%
\pgfusepath{clip}%
\pgfsetrectcap%
\pgfsetroundjoin%
\pgfsetlinewidth{1.505625pt}%
\definecolor{currentstroke}{rgb}{1.000000,0.000000,0.000000}%
\pgfsetstrokecolor{currentstroke}%
\pgfsetdash{}{0pt}%
\pgfpathmoveto{\pgfqpoint{0.769599in}{1.432246in}}%
\pgfpathlineto{\pgfqpoint{0.782156in}{1.419499in}}%
\pgfusepath{stroke}%
\end{pgfscope}%
\begin{pgfscope}%
\pgfpathrectangle{\pgfqpoint{0.100000in}{0.212622in}}{\pgfqpoint{3.696000in}{3.696000in}}%
\pgfusepath{clip}%
\pgfsetrectcap%
\pgfsetroundjoin%
\pgfsetlinewidth{1.505625pt}%
\definecolor{currentstroke}{rgb}{1.000000,0.000000,0.000000}%
\pgfsetstrokecolor{currentstroke}%
\pgfsetdash{}{0pt}%
\pgfpathmoveto{\pgfqpoint{0.769598in}{1.432246in}}%
\pgfpathlineto{\pgfqpoint{0.782156in}{1.419499in}}%
\pgfusepath{stroke}%
\end{pgfscope}%
\begin{pgfscope}%
\pgfpathrectangle{\pgfqpoint{0.100000in}{0.212622in}}{\pgfqpoint{3.696000in}{3.696000in}}%
\pgfusepath{clip}%
\pgfsetrectcap%
\pgfsetroundjoin%
\pgfsetlinewidth{1.505625pt}%
\definecolor{currentstroke}{rgb}{1.000000,0.000000,0.000000}%
\pgfsetstrokecolor{currentstroke}%
\pgfsetdash{}{0pt}%
\pgfpathmoveto{\pgfqpoint{0.769598in}{1.432246in}}%
\pgfpathlineto{\pgfqpoint{0.782156in}{1.419499in}}%
\pgfusepath{stroke}%
\end{pgfscope}%
\begin{pgfscope}%
\pgfpathrectangle{\pgfqpoint{0.100000in}{0.212622in}}{\pgfqpoint{3.696000in}{3.696000in}}%
\pgfusepath{clip}%
\pgfsetrectcap%
\pgfsetroundjoin%
\pgfsetlinewidth{1.505625pt}%
\definecolor{currentstroke}{rgb}{1.000000,0.000000,0.000000}%
\pgfsetstrokecolor{currentstroke}%
\pgfsetdash{}{0pt}%
\pgfpathmoveto{\pgfqpoint{0.769598in}{1.432246in}}%
\pgfpathlineto{\pgfqpoint{0.782156in}{1.419499in}}%
\pgfusepath{stroke}%
\end{pgfscope}%
\begin{pgfscope}%
\pgfpathrectangle{\pgfqpoint{0.100000in}{0.212622in}}{\pgfqpoint{3.696000in}{3.696000in}}%
\pgfusepath{clip}%
\pgfsetrectcap%
\pgfsetroundjoin%
\pgfsetlinewidth{1.505625pt}%
\definecolor{currentstroke}{rgb}{1.000000,0.000000,0.000000}%
\pgfsetstrokecolor{currentstroke}%
\pgfsetdash{}{0pt}%
\pgfpathmoveto{\pgfqpoint{0.769598in}{1.432246in}}%
\pgfpathlineto{\pgfqpoint{0.782156in}{1.419499in}}%
\pgfusepath{stroke}%
\end{pgfscope}%
\begin{pgfscope}%
\pgfpathrectangle{\pgfqpoint{0.100000in}{0.212622in}}{\pgfqpoint{3.696000in}{3.696000in}}%
\pgfusepath{clip}%
\pgfsetrectcap%
\pgfsetroundjoin%
\pgfsetlinewidth{1.505625pt}%
\definecolor{currentstroke}{rgb}{1.000000,0.000000,0.000000}%
\pgfsetstrokecolor{currentstroke}%
\pgfsetdash{}{0pt}%
\pgfpathmoveto{\pgfqpoint{0.769598in}{1.432246in}}%
\pgfpathlineto{\pgfqpoint{0.782156in}{1.419499in}}%
\pgfusepath{stroke}%
\end{pgfscope}%
\begin{pgfscope}%
\pgfpathrectangle{\pgfqpoint{0.100000in}{0.212622in}}{\pgfqpoint{3.696000in}{3.696000in}}%
\pgfusepath{clip}%
\pgfsetrectcap%
\pgfsetroundjoin%
\pgfsetlinewidth{1.505625pt}%
\definecolor{currentstroke}{rgb}{1.000000,0.000000,0.000000}%
\pgfsetstrokecolor{currentstroke}%
\pgfsetdash{}{0pt}%
\pgfpathmoveto{\pgfqpoint{0.769598in}{1.432246in}}%
\pgfpathlineto{\pgfqpoint{0.782156in}{1.419499in}}%
\pgfusepath{stroke}%
\end{pgfscope}%
\begin{pgfscope}%
\pgfpathrectangle{\pgfqpoint{0.100000in}{0.212622in}}{\pgfqpoint{3.696000in}{3.696000in}}%
\pgfusepath{clip}%
\pgfsetrectcap%
\pgfsetroundjoin%
\pgfsetlinewidth{1.505625pt}%
\definecolor{currentstroke}{rgb}{1.000000,0.000000,0.000000}%
\pgfsetstrokecolor{currentstroke}%
\pgfsetdash{}{0pt}%
\pgfpathmoveto{\pgfqpoint{0.769598in}{1.432246in}}%
\pgfpathlineto{\pgfqpoint{0.782156in}{1.419499in}}%
\pgfusepath{stroke}%
\end{pgfscope}%
\begin{pgfscope}%
\pgfpathrectangle{\pgfqpoint{0.100000in}{0.212622in}}{\pgfqpoint{3.696000in}{3.696000in}}%
\pgfusepath{clip}%
\pgfsetrectcap%
\pgfsetroundjoin%
\pgfsetlinewidth{1.505625pt}%
\definecolor{currentstroke}{rgb}{1.000000,0.000000,0.000000}%
\pgfsetstrokecolor{currentstroke}%
\pgfsetdash{}{0pt}%
\pgfpathmoveto{\pgfqpoint{0.769598in}{1.432246in}}%
\pgfpathlineto{\pgfqpoint{0.782156in}{1.419499in}}%
\pgfusepath{stroke}%
\end{pgfscope}%
\begin{pgfscope}%
\pgfpathrectangle{\pgfqpoint{0.100000in}{0.212622in}}{\pgfqpoint{3.696000in}{3.696000in}}%
\pgfusepath{clip}%
\pgfsetrectcap%
\pgfsetroundjoin%
\pgfsetlinewidth{1.505625pt}%
\definecolor{currentstroke}{rgb}{1.000000,0.000000,0.000000}%
\pgfsetstrokecolor{currentstroke}%
\pgfsetdash{}{0pt}%
\pgfpathmoveto{\pgfqpoint{0.769598in}{1.432246in}}%
\pgfpathlineto{\pgfqpoint{0.782156in}{1.419499in}}%
\pgfusepath{stroke}%
\end{pgfscope}%
\begin{pgfscope}%
\pgfpathrectangle{\pgfqpoint{0.100000in}{0.212622in}}{\pgfqpoint{3.696000in}{3.696000in}}%
\pgfusepath{clip}%
\pgfsetrectcap%
\pgfsetroundjoin%
\pgfsetlinewidth{1.505625pt}%
\definecolor{currentstroke}{rgb}{1.000000,0.000000,0.000000}%
\pgfsetstrokecolor{currentstroke}%
\pgfsetdash{}{0pt}%
\pgfpathmoveto{\pgfqpoint{0.769598in}{1.432246in}}%
\pgfpathlineto{\pgfqpoint{0.782156in}{1.419499in}}%
\pgfusepath{stroke}%
\end{pgfscope}%
\begin{pgfscope}%
\pgfpathrectangle{\pgfqpoint{0.100000in}{0.212622in}}{\pgfqpoint{3.696000in}{3.696000in}}%
\pgfusepath{clip}%
\pgfsetrectcap%
\pgfsetroundjoin%
\pgfsetlinewidth{1.505625pt}%
\definecolor{currentstroke}{rgb}{1.000000,0.000000,0.000000}%
\pgfsetstrokecolor{currentstroke}%
\pgfsetdash{}{0pt}%
\pgfpathmoveto{\pgfqpoint{0.769598in}{1.432246in}}%
\pgfpathlineto{\pgfqpoint{0.782156in}{1.419499in}}%
\pgfusepath{stroke}%
\end{pgfscope}%
\begin{pgfscope}%
\pgfpathrectangle{\pgfqpoint{0.100000in}{0.212622in}}{\pgfqpoint{3.696000in}{3.696000in}}%
\pgfusepath{clip}%
\pgfsetrectcap%
\pgfsetroundjoin%
\pgfsetlinewidth{1.505625pt}%
\definecolor{currentstroke}{rgb}{1.000000,0.000000,0.000000}%
\pgfsetstrokecolor{currentstroke}%
\pgfsetdash{}{0pt}%
\pgfpathmoveto{\pgfqpoint{0.769598in}{1.432246in}}%
\pgfpathlineto{\pgfqpoint{0.782156in}{1.419499in}}%
\pgfusepath{stroke}%
\end{pgfscope}%
\begin{pgfscope}%
\pgfpathrectangle{\pgfqpoint{0.100000in}{0.212622in}}{\pgfqpoint{3.696000in}{3.696000in}}%
\pgfusepath{clip}%
\pgfsetrectcap%
\pgfsetroundjoin%
\pgfsetlinewidth{1.505625pt}%
\definecolor{currentstroke}{rgb}{1.000000,0.000000,0.000000}%
\pgfsetstrokecolor{currentstroke}%
\pgfsetdash{}{0pt}%
\pgfpathmoveto{\pgfqpoint{0.769598in}{1.432246in}}%
\pgfpathlineto{\pgfqpoint{0.782156in}{1.419499in}}%
\pgfusepath{stroke}%
\end{pgfscope}%
\begin{pgfscope}%
\pgfpathrectangle{\pgfqpoint{0.100000in}{0.212622in}}{\pgfqpoint{3.696000in}{3.696000in}}%
\pgfusepath{clip}%
\pgfsetrectcap%
\pgfsetroundjoin%
\pgfsetlinewidth{1.505625pt}%
\definecolor{currentstroke}{rgb}{1.000000,0.000000,0.000000}%
\pgfsetstrokecolor{currentstroke}%
\pgfsetdash{}{0pt}%
\pgfpathmoveto{\pgfqpoint{0.769598in}{1.432246in}}%
\pgfpathlineto{\pgfqpoint{0.782156in}{1.419499in}}%
\pgfusepath{stroke}%
\end{pgfscope}%
\begin{pgfscope}%
\pgfpathrectangle{\pgfqpoint{0.100000in}{0.212622in}}{\pgfqpoint{3.696000in}{3.696000in}}%
\pgfusepath{clip}%
\pgfsetrectcap%
\pgfsetroundjoin%
\pgfsetlinewidth{1.505625pt}%
\definecolor{currentstroke}{rgb}{1.000000,0.000000,0.000000}%
\pgfsetstrokecolor{currentstroke}%
\pgfsetdash{}{0pt}%
\pgfpathmoveto{\pgfqpoint{0.769598in}{1.432246in}}%
\pgfpathlineto{\pgfqpoint{0.782156in}{1.419499in}}%
\pgfusepath{stroke}%
\end{pgfscope}%
\begin{pgfscope}%
\pgfpathrectangle{\pgfqpoint{0.100000in}{0.212622in}}{\pgfqpoint{3.696000in}{3.696000in}}%
\pgfusepath{clip}%
\pgfsetrectcap%
\pgfsetroundjoin%
\pgfsetlinewidth{1.505625pt}%
\definecolor{currentstroke}{rgb}{1.000000,0.000000,0.000000}%
\pgfsetstrokecolor{currentstroke}%
\pgfsetdash{}{0pt}%
\pgfpathmoveto{\pgfqpoint{0.769598in}{1.432246in}}%
\pgfpathlineto{\pgfqpoint{0.782156in}{1.419499in}}%
\pgfusepath{stroke}%
\end{pgfscope}%
\begin{pgfscope}%
\pgfpathrectangle{\pgfqpoint{0.100000in}{0.212622in}}{\pgfqpoint{3.696000in}{3.696000in}}%
\pgfusepath{clip}%
\pgfsetrectcap%
\pgfsetroundjoin%
\pgfsetlinewidth{1.505625pt}%
\definecolor{currentstroke}{rgb}{1.000000,0.000000,0.000000}%
\pgfsetstrokecolor{currentstroke}%
\pgfsetdash{}{0pt}%
\pgfpathmoveto{\pgfqpoint{0.769598in}{1.432246in}}%
\pgfpathlineto{\pgfqpoint{0.782156in}{1.419499in}}%
\pgfusepath{stroke}%
\end{pgfscope}%
\begin{pgfscope}%
\pgfpathrectangle{\pgfqpoint{0.100000in}{0.212622in}}{\pgfqpoint{3.696000in}{3.696000in}}%
\pgfusepath{clip}%
\pgfsetrectcap%
\pgfsetroundjoin%
\pgfsetlinewidth{1.505625pt}%
\definecolor{currentstroke}{rgb}{1.000000,0.000000,0.000000}%
\pgfsetstrokecolor{currentstroke}%
\pgfsetdash{}{0pt}%
\pgfpathmoveto{\pgfqpoint{0.769598in}{1.432246in}}%
\pgfpathlineto{\pgfqpoint{0.782156in}{1.419499in}}%
\pgfusepath{stroke}%
\end{pgfscope}%
\begin{pgfscope}%
\pgfpathrectangle{\pgfqpoint{0.100000in}{0.212622in}}{\pgfqpoint{3.696000in}{3.696000in}}%
\pgfusepath{clip}%
\pgfsetrectcap%
\pgfsetroundjoin%
\pgfsetlinewidth{1.505625pt}%
\definecolor{currentstroke}{rgb}{1.000000,0.000000,0.000000}%
\pgfsetstrokecolor{currentstroke}%
\pgfsetdash{}{0pt}%
\pgfpathmoveto{\pgfqpoint{0.769598in}{1.432246in}}%
\pgfpathlineto{\pgfqpoint{0.782156in}{1.419499in}}%
\pgfusepath{stroke}%
\end{pgfscope}%
\begin{pgfscope}%
\pgfpathrectangle{\pgfqpoint{0.100000in}{0.212622in}}{\pgfqpoint{3.696000in}{3.696000in}}%
\pgfusepath{clip}%
\pgfsetrectcap%
\pgfsetroundjoin%
\pgfsetlinewidth{1.505625pt}%
\definecolor{currentstroke}{rgb}{1.000000,0.000000,0.000000}%
\pgfsetstrokecolor{currentstroke}%
\pgfsetdash{}{0pt}%
\pgfpathmoveto{\pgfqpoint{0.769598in}{1.432246in}}%
\pgfpathlineto{\pgfqpoint{0.782156in}{1.419499in}}%
\pgfusepath{stroke}%
\end{pgfscope}%
\begin{pgfscope}%
\pgfpathrectangle{\pgfqpoint{0.100000in}{0.212622in}}{\pgfqpoint{3.696000in}{3.696000in}}%
\pgfusepath{clip}%
\pgfsetrectcap%
\pgfsetroundjoin%
\pgfsetlinewidth{1.505625pt}%
\definecolor{currentstroke}{rgb}{1.000000,0.000000,0.000000}%
\pgfsetstrokecolor{currentstroke}%
\pgfsetdash{}{0pt}%
\pgfpathmoveto{\pgfqpoint{0.769598in}{1.432246in}}%
\pgfpathlineto{\pgfqpoint{0.782156in}{1.419499in}}%
\pgfusepath{stroke}%
\end{pgfscope}%
\begin{pgfscope}%
\pgfpathrectangle{\pgfqpoint{0.100000in}{0.212622in}}{\pgfqpoint{3.696000in}{3.696000in}}%
\pgfusepath{clip}%
\pgfsetrectcap%
\pgfsetroundjoin%
\pgfsetlinewidth{1.505625pt}%
\definecolor{currentstroke}{rgb}{1.000000,0.000000,0.000000}%
\pgfsetstrokecolor{currentstroke}%
\pgfsetdash{}{0pt}%
\pgfpathmoveto{\pgfqpoint{0.769598in}{1.432246in}}%
\pgfpathlineto{\pgfqpoint{0.782156in}{1.419499in}}%
\pgfusepath{stroke}%
\end{pgfscope}%
\begin{pgfscope}%
\pgfpathrectangle{\pgfqpoint{0.100000in}{0.212622in}}{\pgfqpoint{3.696000in}{3.696000in}}%
\pgfusepath{clip}%
\pgfsetrectcap%
\pgfsetroundjoin%
\pgfsetlinewidth{1.505625pt}%
\definecolor{currentstroke}{rgb}{1.000000,0.000000,0.000000}%
\pgfsetstrokecolor{currentstroke}%
\pgfsetdash{}{0pt}%
\pgfpathmoveto{\pgfqpoint{0.769598in}{1.432246in}}%
\pgfpathlineto{\pgfqpoint{0.782156in}{1.419499in}}%
\pgfusepath{stroke}%
\end{pgfscope}%
\begin{pgfscope}%
\pgfpathrectangle{\pgfqpoint{0.100000in}{0.212622in}}{\pgfqpoint{3.696000in}{3.696000in}}%
\pgfusepath{clip}%
\pgfsetrectcap%
\pgfsetroundjoin%
\pgfsetlinewidth{1.505625pt}%
\definecolor{currentstroke}{rgb}{1.000000,0.000000,0.000000}%
\pgfsetstrokecolor{currentstroke}%
\pgfsetdash{}{0pt}%
\pgfpathmoveto{\pgfqpoint{0.769598in}{1.432246in}}%
\pgfpathlineto{\pgfqpoint{0.782156in}{1.419499in}}%
\pgfusepath{stroke}%
\end{pgfscope}%
\begin{pgfscope}%
\pgfpathrectangle{\pgfqpoint{0.100000in}{0.212622in}}{\pgfqpoint{3.696000in}{3.696000in}}%
\pgfusepath{clip}%
\pgfsetrectcap%
\pgfsetroundjoin%
\pgfsetlinewidth{1.505625pt}%
\definecolor{currentstroke}{rgb}{1.000000,0.000000,0.000000}%
\pgfsetstrokecolor{currentstroke}%
\pgfsetdash{}{0pt}%
\pgfpathmoveto{\pgfqpoint{0.769598in}{1.432246in}}%
\pgfpathlineto{\pgfqpoint{0.782156in}{1.419499in}}%
\pgfusepath{stroke}%
\end{pgfscope}%
\begin{pgfscope}%
\pgfpathrectangle{\pgfqpoint{0.100000in}{0.212622in}}{\pgfqpoint{3.696000in}{3.696000in}}%
\pgfusepath{clip}%
\pgfsetrectcap%
\pgfsetroundjoin%
\pgfsetlinewidth{1.505625pt}%
\definecolor{currentstroke}{rgb}{1.000000,0.000000,0.000000}%
\pgfsetstrokecolor{currentstroke}%
\pgfsetdash{}{0pt}%
\pgfpathmoveto{\pgfqpoint{0.769598in}{1.432246in}}%
\pgfpathlineto{\pgfqpoint{0.782156in}{1.419499in}}%
\pgfusepath{stroke}%
\end{pgfscope}%
\begin{pgfscope}%
\pgfpathrectangle{\pgfqpoint{0.100000in}{0.212622in}}{\pgfqpoint{3.696000in}{3.696000in}}%
\pgfusepath{clip}%
\pgfsetrectcap%
\pgfsetroundjoin%
\pgfsetlinewidth{1.505625pt}%
\definecolor{currentstroke}{rgb}{1.000000,0.000000,0.000000}%
\pgfsetstrokecolor{currentstroke}%
\pgfsetdash{}{0pt}%
\pgfpathmoveto{\pgfqpoint{0.769598in}{1.432246in}}%
\pgfpathlineto{\pgfqpoint{0.782156in}{1.419499in}}%
\pgfusepath{stroke}%
\end{pgfscope}%
\begin{pgfscope}%
\pgfpathrectangle{\pgfqpoint{0.100000in}{0.212622in}}{\pgfqpoint{3.696000in}{3.696000in}}%
\pgfusepath{clip}%
\pgfsetrectcap%
\pgfsetroundjoin%
\pgfsetlinewidth{1.505625pt}%
\definecolor{currentstroke}{rgb}{1.000000,0.000000,0.000000}%
\pgfsetstrokecolor{currentstroke}%
\pgfsetdash{}{0pt}%
\pgfpathmoveto{\pgfqpoint{0.769598in}{1.432246in}}%
\pgfpathlineto{\pgfqpoint{0.782156in}{1.419499in}}%
\pgfusepath{stroke}%
\end{pgfscope}%
\begin{pgfscope}%
\pgfpathrectangle{\pgfqpoint{0.100000in}{0.212622in}}{\pgfqpoint{3.696000in}{3.696000in}}%
\pgfusepath{clip}%
\pgfsetrectcap%
\pgfsetroundjoin%
\pgfsetlinewidth{1.505625pt}%
\definecolor{currentstroke}{rgb}{1.000000,0.000000,0.000000}%
\pgfsetstrokecolor{currentstroke}%
\pgfsetdash{}{0pt}%
\pgfpathmoveto{\pgfqpoint{0.769598in}{1.432246in}}%
\pgfpathlineto{\pgfqpoint{0.782156in}{1.419499in}}%
\pgfusepath{stroke}%
\end{pgfscope}%
\begin{pgfscope}%
\pgfpathrectangle{\pgfqpoint{0.100000in}{0.212622in}}{\pgfqpoint{3.696000in}{3.696000in}}%
\pgfusepath{clip}%
\pgfsetrectcap%
\pgfsetroundjoin%
\pgfsetlinewidth{1.505625pt}%
\definecolor{currentstroke}{rgb}{1.000000,0.000000,0.000000}%
\pgfsetstrokecolor{currentstroke}%
\pgfsetdash{}{0pt}%
\pgfpathmoveto{\pgfqpoint{0.769598in}{1.432246in}}%
\pgfpathlineto{\pgfqpoint{0.782156in}{1.419499in}}%
\pgfusepath{stroke}%
\end{pgfscope}%
\begin{pgfscope}%
\pgfpathrectangle{\pgfqpoint{0.100000in}{0.212622in}}{\pgfqpoint{3.696000in}{3.696000in}}%
\pgfusepath{clip}%
\pgfsetrectcap%
\pgfsetroundjoin%
\pgfsetlinewidth{1.505625pt}%
\definecolor{currentstroke}{rgb}{1.000000,0.000000,0.000000}%
\pgfsetstrokecolor{currentstroke}%
\pgfsetdash{}{0pt}%
\pgfpathmoveto{\pgfqpoint{0.769598in}{1.432246in}}%
\pgfpathlineto{\pgfqpoint{0.782156in}{1.419499in}}%
\pgfusepath{stroke}%
\end{pgfscope}%
\begin{pgfscope}%
\pgfpathrectangle{\pgfqpoint{0.100000in}{0.212622in}}{\pgfqpoint{3.696000in}{3.696000in}}%
\pgfusepath{clip}%
\pgfsetrectcap%
\pgfsetroundjoin%
\pgfsetlinewidth{1.505625pt}%
\definecolor{currentstroke}{rgb}{1.000000,0.000000,0.000000}%
\pgfsetstrokecolor{currentstroke}%
\pgfsetdash{}{0pt}%
\pgfpathmoveto{\pgfqpoint{0.769598in}{1.432246in}}%
\pgfpathlineto{\pgfqpoint{0.782156in}{1.419499in}}%
\pgfusepath{stroke}%
\end{pgfscope}%
\begin{pgfscope}%
\pgfpathrectangle{\pgfqpoint{0.100000in}{0.212622in}}{\pgfqpoint{3.696000in}{3.696000in}}%
\pgfusepath{clip}%
\pgfsetrectcap%
\pgfsetroundjoin%
\pgfsetlinewidth{1.505625pt}%
\definecolor{currentstroke}{rgb}{1.000000,0.000000,0.000000}%
\pgfsetstrokecolor{currentstroke}%
\pgfsetdash{}{0pt}%
\pgfpathmoveto{\pgfqpoint{0.769598in}{1.432246in}}%
\pgfpathlineto{\pgfqpoint{0.782156in}{1.419499in}}%
\pgfusepath{stroke}%
\end{pgfscope}%
\begin{pgfscope}%
\pgfpathrectangle{\pgfqpoint{0.100000in}{0.212622in}}{\pgfqpoint{3.696000in}{3.696000in}}%
\pgfusepath{clip}%
\pgfsetrectcap%
\pgfsetroundjoin%
\pgfsetlinewidth{1.505625pt}%
\definecolor{currentstroke}{rgb}{1.000000,0.000000,0.000000}%
\pgfsetstrokecolor{currentstroke}%
\pgfsetdash{}{0pt}%
\pgfpathmoveto{\pgfqpoint{0.769598in}{1.432246in}}%
\pgfpathlineto{\pgfqpoint{0.782156in}{1.419499in}}%
\pgfusepath{stroke}%
\end{pgfscope}%
\begin{pgfscope}%
\pgfpathrectangle{\pgfqpoint{0.100000in}{0.212622in}}{\pgfqpoint{3.696000in}{3.696000in}}%
\pgfusepath{clip}%
\pgfsetrectcap%
\pgfsetroundjoin%
\pgfsetlinewidth{1.505625pt}%
\definecolor{currentstroke}{rgb}{1.000000,0.000000,0.000000}%
\pgfsetstrokecolor{currentstroke}%
\pgfsetdash{}{0pt}%
\pgfpathmoveto{\pgfqpoint{0.769598in}{1.432246in}}%
\pgfpathlineto{\pgfqpoint{0.782156in}{1.419499in}}%
\pgfusepath{stroke}%
\end{pgfscope}%
\begin{pgfscope}%
\pgfpathrectangle{\pgfqpoint{0.100000in}{0.212622in}}{\pgfqpoint{3.696000in}{3.696000in}}%
\pgfusepath{clip}%
\pgfsetrectcap%
\pgfsetroundjoin%
\pgfsetlinewidth{1.505625pt}%
\definecolor{currentstroke}{rgb}{1.000000,0.000000,0.000000}%
\pgfsetstrokecolor{currentstroke}%
\pgfsetdash{}{0pt}%
\pgfpathmoveto{\pgfqpoint{0.769598in}{1.432246in}}%
\pgfpathlineto{\pgfqpoint{0.782156in}{1.419499in}}%
\pgfusepath{stroke}%
\end{pgfscope}%
\begin{pgfscope}%
\pgfpathrectangle{\pgfqpoint{0.100000in}{0.212622in}}{\pgfqpoint{3.696000in}{3.696000in}}%
\pgfusepath{clip}%
\pgfsetrectcap%
\pgfsetroundjoin%
\pgfsetlinewidth{1.505625pt}%
\definecolor{currentstroke}{rgb}{1.000000,0.000000,0.000000}%
\pgfsetstrokecolor{currentstroke}%
\pgfsetdash{}{0pt}%
\pgfpathmoveto{\pgfqpoint{0.769598in}{1.432246in}}%
\pgfpathlineto{\pgfqpoint{0.782156in}{1.419499in}}%
\pgfusepath{stroke}%
\end{pgfscope}%
\begin{pgfscope}%
\pgfpathrectangle{\pgfqpoint{0.100000in}{0.212622in}}{\pgfqpoint{3.696000in}{3.696000in}}%
\pgfusepath{clip}%
\pgfsetrectcap%
\pgfsetroundjoin%
\pgfsetlinewidth{1.505625pt}%
\definecolor{currentstroke}{rgb}{1.000000,0.000000,0.000000}%
\pgfsetstrokecolor{currentstroke}%
\pgfsetdash{}{0pt}%
\pgfpathmoveto{\pgfqpoint{0.769598in}{1.432246in}}%
\pgfpathlineto{\pgfqpoint{0.782156in}{1.419499in}}%
\pgfusepath{stroke}%
\end{pgfscope}%
\begin{pgfscope}%
\pgfpathrectangle{\pgfqpoint{0.100000in}{0.212622in}}{\pgfqpoint{3.696000in}{3.696000in}}%
\pgfusepath{clip}%
\pgfsetrectcap%
\pgfsetroundjoin%
\pgfsetlinewidth{1.505625pt}%
\definecolor{currentstroke}{rgb}{1.000000,0.000000,0.000000}%
\pgfsetstrokecolor{currentstroke}%
\pgfsetdash{}{0pt}%
\pgfpathmoveto{\pgfqpoint{0.769598in}{1.432246in}}%
\pgfpathlineto{\pgfqpoint{0.782156in}{1.419499in}}%
\pgfusepath{stroke}%
\end{pgfscope}%
\begin{pgfscope}%
\pgfpathrectangle{\pgfqpoint{0.100000in}{0.212622in}}{\pgfqpoint{3.696000in}{3.696000in}}%
\pgfusepath{clip}%
\pgfsetrectcap%
\pgfsetroundjoin%
\pgfsetlinewidth{1.505625pt}%
\definecolor{currentstroke}{rgb}{1.000000,0.000000,0.000000}%
\pgfsetstrokecolor{currentstroke}%
\pgfsetdash{}{0pt}%
\pgfpathmoveto{\pgfqpoint{0.769598in}{1.432246in}}%
\pgfpathlineto{\pgfqpoint{0.782156in}{1.419499in}}%
\pgfusepath{stroke}%
\end{pgfscope}%
\begin{pgfscope}%
\pgfpathrectangle{\pgfqpoint{0.100000in}{0.212622in}}{\pgfqpoint{3.696000in}{3.696000in}}%
\pgfusepath{clip}%
\pgfsetrectcap%
\pgfsetroundjoin%
\pgfsetlinewidth{1.505625pt}%
\definecolor{currentstroke}{rgb}{1.000000,0.000000,0.000000}%
\pgfsetstrokecolor{currentstroke}%
\pgfsetdash{}{0pt}%
\pgfpathmoveto{\pgfqpoint{0.769598in}{1.432246in}}%
\pgfpathlineto{\pgfqpoint{0.782156in}{1.419499in}}%
\pgfusepath{stroke}%
\end{pgfscope}%
\begin{pgfscope}%
\pgfpathrectangle{\pgfqpoint{0.100000in}{0.212622in}}{\pgfqpoint{3.696000in}{3.696000in}}%
\pgfusepath{clip}%
\pgfsetrectcap%
\pgfsetroundjoin%
\pgfsetlinewidth{1.505625pt}%
\definecolor{currentstroke}{rgb}{1.000000,0.000000,0.000000}%
\pgfsetstrokecolor{currentstroke}%
\pgfsetdash{}{0pt}%
\pgfpathmoveto{\pgfqpoint{0.769598in}{1.432246in}}%
\pgfpathlineto{\pgfqpoint{0.782156in}{1.419499in}}%
\pgfusepath{stroke}%
\end{pgfscope}%
\begin{pgfscope}%
\pgfpathrectangle{\pgfqpoint{0.100000in}{0.212622in}}{\pgfqpoint{3.696000in}{3.696000in}}%
\pgfusepath{clip}%
\pgfsetrectcap%
\pgfsetroundjoin%
\pgfsetlinewidth{1.505625pt}%
\definecolor{currentstroke}{rgb}{1.000000,0.000000,0.000000}%
\pgfsetstrokecolor{currentstroke}%
\pgfsetdash{}{0pt}%
\pgfpathmoveto{\pgfqpoint{0.769598in}{1.432246in}}%
\pgfpathlineto{\pgfqpoint{0.782156in}{1.419499in}}%
\pgfusepath{stroke}%
\end{pgfscope}%
\begin{pgfscope}%
\pgfpathrectangle{\pgfqpoint{0.100000in}{0.212622in}}{\pgfqpoint{3.696000in}{3.696000in}}%
\pgfusepath{clip}%
\pgfsetrectcap%
\pgfsetroundjoin%
\pgfsetlinewidth{1.505625pt}%
\definecolor{currentstroke}{rgb}{1.000000,0.000000,0.000000}%
\pgfsetstrokecolor{currentstroke}%
\pgfsetdash{}{0pt}%
\pgfpathmoveto{\pgfqpoint{0.769598in}{1.432246in}}%
\pgfpathlineto{\pgfqpoint{0.782156in}{1.419499in}}%
\pgfusepath{stroke}%
\end{pgfscope}%
\begin{pgfscope}%
\pgfpathrectangle{\pgfqpoint{0.100000in}{0.212622in}}{\pgfqpoint{3.696000in}{3.696000in}}%
\pgfusepath{clip}%
\pgfsetrectcap%
\pgfsetroundjoin%
\pgfsetlinewidth{1.505625pt}%
\definecolor{currentstroke}{rgb}{1.000000,0.000000,0.000000}%
\pgfsetstrokecolor{currentstroke}%
\pgfsetdash{}{0pt}%
\pgfpathmoveto{\pgfqpoint{0.769598in}{1.432246in}}%
\pgfpathlineto{\pgfqpoint{0.782156in}{1.419499in}}%
\pgfusepath{stroke}%
\end{pgfscope}%
\begin{pgfscope}%
\pgfpathrectangle{\pgfqpoint{0.100000in}{0.212622in}}{\pgfqpoint{3.696000in}{3.696000in}}%
\pgfusepath{clip}%
\pgfsetrectcap%
\pgfsetroundjoin%
\pgfsetlinewidth{1.505625pt}%
\definecolor{currentstroke}{rgb}{1.000000,0.000000,0.000000}%
\pgfsetstrokecolor{currentstroke}%
\pgfsetdash{}{0pt}%
\pgfpathmoveto{\pgfqpoint{0.769598in}{1.432246in}}%
\pgfpathlineto{\pgfqpoint{0.782156in}{1.419499in}}%
\pgfusepath{stroke}%
\end{pgfscope}%
\begin{pgfscope}%
\pgfpathrectangle{\pgfqpoint{0.100000in}{0.212622in}}{\pgfqpoint{3.696000in}{3.696000in}}%
\pgfusepath{clip}%
\pgfsetrectcap%
\pgfsetroundjoin%
\pgfsetlinewidth{1.505625pt}%
\definecolor{currentstroke}{rgb}{1.000000,0.000000,0.000000}%
\pgfsetstrokecolor{currentstroke}%
\pgfsetdash{}{0pt}%
\pgfpathmoveto{\pgfqpoint{0.769598in}{1.432246in}}%
\pgfpathlineto{\pgfqpoint{0.782156in}{1.419499in}}%
\pgfusepath{stroke}%
\end{pgfscope}%
\begin{pgfscope}%
\pgfpathrectangle{\pgfqpoint{0.100000in}{0.212622in}}{\pgfqpoint{3.696000in}{3.696000in}}%
\pgfusepath{clip}%
\pgfsetrectcap%
\pgfsetroundjoin%
\pgfsetlinewidth{1.505625pt}%
\definecolor{currentstroke}{rgb}{1.000000,0.000000,0.000000}%
\pgfsetstrokecolor{currentstroke}%
\pgfsetdash{}{0pt}%
\pgfpathmoveto{\pgfqpoint{0.769598in}{1.432246in}}%
\pgfpathlineto{\pgfqpoint{0.782156in}{1.419499in}}%
\pgfusepath{stroke}%
\end{pgfscope}%
\begin{pgfscope}%
\pgfpathrectangle{\pgfqpoint{0.100000in}{0.212622in}}{\pgfqpoint{3.696000in}{3.696000in}}%
\pgfusepath{clip}%
\pgfsetrectcap%
\pgfsetroundjoin%
\pgfsetlinewidth{1.505625pt}%
\definecolor{currentstroke}{rgb}{1.000000,0.000000,0.000000}%
\pgfsetstrokecolor{currentstroke}%
\pgfsetdash{}{0pt}%
\pgfpathmoveto{\pgfqpoint{0.769598in}{1.432246in}}%
\pgfpathlineto{\pgfqpoint{0.782156in}{1.419499in}}%
\pgfusepath{stroke}%
\end{pgfscope}%
\begin{pgfscope}%
\pgfpathrectangle{\pgfqpoint{0.100000in}{0.212622in}}{\pgfqpoint{3.696000in}{3.696000in}}%
\pgfusepath{clip}%
\pgfsetrectcap%
\pgfsetroundjoin%
\pgfsetlinewidth{1.505625pt}%
\definecolor{currentstroke}{rgb}{1.000000,0.000000,0.000000}%
\pgfsetstrokecolor{currentstroke}%
\pgfsetdash{}{0pt}%
\pgfpathmoveto{\pgfqpoint{0.769598in}{1.432246in}}%
\pgfpathlineto{\pgfqpoint{0.782156in}{1.419499in}}%
\pgfusepath{stroke}%
\end{pgfscope}%
\begin{pgfscope}%
\pgfpathrectangle{\pgfqpoint{0.100000in}{0.212622in}}{\pgfqpoint{3.696000in}{3.696000in}}%
\pgfusepath{clip}%
\pgfsetrectcap%
\pgfsetroundjoin%
\pgfsetlinewidth{1.505625pt}%
\definecolor{currentstroke}{rgb}{1.000000,0.000000,0.000000}%
\pgfsetstrokecolor{currentstroke}%
\pgfsetdash{}{0pt}%
\pgfpathmoveto{\pgfqpoint{0.769598in}{1.432246in}}%
\pgfpathlineto{\pgfqpoint{0.782156in}{1.419499in}}%
\pgfusepath{stroke}%
\end{pgfscope}%
\begin{pgfscope}%
\pgfpathrectangle{\pgfqpoint{0.100000in}{0.212622in}}{\pgfqpoint{3.696000in}{3.696000in}}%
\pgfusepath{clip}%
\pgfsetrectcap%
\pgfsetroundjoin%
\pgfsetlinewidth{1.505625pt}%
\definecolor{currentstroke}{rgb}{1.000000,0.000000,0.000000}%
\pgfsetstrokecolor{currentstroke}%
\pgfsetdash{}{0pt}%
\pgfpathmoveto{\pgfqpoint{0.769598in}{1.432246in}}%
\pgfpathlineto{\pgfqpoint{0.782156in}{1.419499in}}%
\pgfusepath{stroke}%
\end{pgfscope}%
\begin{pgfscope}%
\pgfpathrectangle{\pgfqpoint{0.100000in}{0.212622in}}{\pgfqpoint{3.696000in}{3.696000in}}%
\pgfusepath{clip}%
\pgfsetrectcap%
\pgfsetroundjoin%
\pgfsetlinewidth{1.505625pt}%
\definecolor{currentstroke}{rgb}{1.000000,0.000000,0.000000}%
\pgfsetstrokecolor{currentstroke}%
\pgfsetdash{}{0pt}%
\pgfpathmoveto{\pgfqpoint{0.769598in}{1.432246in}}%
\pgfpathlineto{\pgfqpoint{0.782156in}{1.419499in}}%
\pgfusepath{stroke}%
\end{pgfscope}%
\begin{pgfscope}%
\pgfpathrectangle{\pgfqpoint{0.100000in}{0.212622in}}{\pgfqpoint{3.696000in}{3.696000in}}%
\pgfusepath{clip}%
\pgfsetrectcap%
\pgfsetroundjoin%
\pgfsetlinewidth{1.505625pt}%
\definecolor{currentstroke}{rgb}{1.000000,0.000000,0.000000}%
\pgfsetstrokecolor{currentstroke}%
\pgfsetdash{}{0pt}%
\pgfpathmoveto{\pgfqpoint{0.769598in}{1.432246in}}%
\pgfpathlineto{\pgfqpoint{0.782156in}{1.419499in}}%
\pgfusepath{stroke}%
\end{pgfscope}%
\begin{pgfscope}%
\pgfpathrectangle{\pgfqpoint{0.100000in}{0.212622in}}{\pgfqpoint{3.696000in}{3.696000in}}%
\pgfusepath{clip}%
\pgfsetrectcap%
\pgfsetroundjoin%
\pgfsetlinewidth{1.505625pt}%
\definecolor{currentstroke}{rgb}{1.000000,0.000000,0.000000}%
\pgfsetstrokecolor{currentstroke}%
\pgfsetdash{}{0pt}%
\pgfpathmoveto{\pgfqpoint{0.769598in}{1.432246in}}%
\pgfpathlineto{\pgfqpoint{0.782156in}{1.419499in}}%
\pgfusepath{stroke}%
\end{pgfscope}%
\begin{pgfscope}%
\pgfpathrectangle{\pgfqpoint{0.100000in}{0.212622in}}{\pgfqpoint{3.696000in}{3.696000in}}%
\pgfusepath{clip}%
\pgfsetrectcap%
\pgfsetroundjoin%
\pgfsetlinewidth{1.505625pt}%
\definecolor{currentstroke}{rgb}{1.000000,0.000000,0.000000}%
\pgfsetstrokecolor{currentstroke}%
\pgfsetdash{}{0pt}%
\pgfpathmoveto{\pgfqpoint{0.769598in}{1.432246in}}%
\pgfpathlineto{\pgfqpoint{0.782156in}{1.419499in}}%
\pgfusepath{stroke}%
\end{pgfscope}%
\begin{pgfscope}%
\pgfpathrectangle{\pgfqpoint{0.100000in}{0.212622in}}{\pgfqpoint{3.696000in}{3.696000in}}%
\pgfusepath{clip}%
\pgfsetrectcap%
\pgfsetroundjoin%
\pgfsetlinewidth{1.505625pt}%
\definecolor{currentstroke}{rgb}{1.000000,0.000000,0.000000}%
\pgfsetstrokecolor{currentstroke}%
\pgfsetdash{}{0pt}%
\pgfpathmoveto{\pgfqpoint{0.769598in}{1.432246in}}%
\pgfpathlineto{\pgfqpoint{0.782156in}{1.419499in}}%
\pgfusepath{stroke}%
\end{pgfscope}%
\begin{pgfscope}%
\pgfpathrectangle{\pgfqpoint{0.100000in}{0.212622in}}{\pgfqpoint{3.696000in}{3.696000in}}%
\pgfusepath{clip}%
\pgfsetrectcap%
\pgfsetroundjoin%
\pgfsetlinewidth{1.505625pt}%
\definecolor{currentstroke}{rgb}{1.000000,0.000000,0.000000}%
\pgfsetstrokecolor{currentstroke}%
\pgfsetdash{}{0pt}%
\pgfpathmoveto{\pgfqpoint{0.769598in}{1.432246in}}%
\pgfpathlineto{\pgfqpoint{0.782156in}{1.419499in}}%
\pgfusepath{stroke}%
\end{pgfscope}%
\begin{pgfscope}%
\pgfpathrectangle{\pgfqpoint{0.100000in}{0.212622in}}{\pgfqpoint{3.696000in}{3.696000in}}%
\pgfusepath{clip}%
\pgfsetrectcap%
\pgfsetroundjoin%
\pgfsetlinewidth{1.505625pt}%
\definecolor{currentstroke}{rgb}{1.000000,0.000000,0.000000}%
\pgfsetstrokecolor{currentstroke}%
\pgfsetdash{}{0pt}%
\pgfpathmoveto{\pgfqpoint{0.769598in}{1.432246in}}%
\pgfpathlineto{\pgfqpoint{0.782156in}{1.419499in}}%
\pgfusepath{stroke}%
\end{pgfscope}%
\begin{pgfscope}%
\pgfpathrectangle{\pgfqpoint{0.100000in}{0.212622in}}{\pgfqpoint{3.696000in}{3.696000in}}%
\pgfusepath{clip}%
\pgfsetrectcap%
\pgfsetroundjoin%
\pgfsetlinewidth{1.505625pt}%
\definecolor{currentstroke}{rgb}{1.000000,0.000000,0.000000}%
\pgfsetstrokecolor{currentstroke}%
\pgfsetdash{}{0pt}%
\pgfpathmoveto{\pgfqpoint{0.769598in}{1.432246in}}%
\pgfpathlineto{\pgfqpoint{0.782156in}{1.419499in}}%
\pgfusepath{stroke}%
\end{pgfscope}%
\begin{pgfscope}%
\pgfpathrectangle{\pgfqpoint{0.100000in}{0.212622in}}{\pgfqpoint{3.696000in}{3.696000in}}%
\pgfusepath{clip}%
\pgfsetrectcap%
\pgfsetroundjoin%
\pgfsetlinewidth{1.505625pt}%
\definecolor{currentstroke}{rgb}{1.000000,0.000000,0.000000}%
\pgfsetstrokecolor{currentstroke}%
\pgfsetdash{}{0pt}%
\pgfpathmoveto{\pgfqpoint{0.769598in}{1.432246in}}%
\pgfpathlineto{\pgfqpoint{0.782156in}{1.419499in}}%
\pgfusepath{stroke}%
\end{pgfscope}%
\begin{pgfscope}%
\pgfpathrectangle{\pgfqpoint{0.100000in}{0.212622in}}{\pgfqpoint{3.696000in}{3.696000in}}%
\pgfusepath{clip}%
\pgfsetrectcap%
\pgfsetroundjoin%
\pgfsetlinewidth{1.505625pt}%
\definecolor{currentstroke}{rgb}{1.000000,0.000000,0.000000}%
\pgfsetstrokecolor{currentstroke}%
\pgfsetdash{}{0pt}%
\pgfpathmoveto{\pgfqpoint{0.769598in}{1.432246in}}%
\pgfpathlineto{\pgfqpoint{0.782156in}{1.419499in}}%
\pgfusepath{stroke}%
\end{pgfscope}%
\begin{pgfscope}%
\pgfpathrectangle{\pgfqpoint{0.100000in}{0.212622in}}{\pgfqpoint{3.696000in}{3.696000in}}%
\pgfusepath{clip}%
\pgfsetrectcap%
\pgfsetroundjoin%
\pgfsetlinewidth{1.505625pt}%
\definecolor{currentstroke}{rgb}{1.000000,0.000000,0.000000}%
\pgfsetstrokecolor{currentstroke}%
\pgfsetdash{}{0pt}%
\pgfpathmoveto{\pgfqpoint{0.769598in}{1.432246in}}%
\pgfpathlineto{\pgfqpoint{0.782156in}{1.419499in}}%
\pgfusepath{stroke}%
\end{pgfscope}%
\begin{pgfscope}%
\pgfpathrectangle{\pgfqpoint{0.100000in}{0.212622in}}{\pgfqpoint{3.696000in}{3.696000in}}%
\pgfusepath{clip}%
\pgfsetrectcap%
\pgfsetroundjoin%
\pgfsetlinewidth{1.505625pt}%
\definecolor{currentstroke}{rgb}{1.000000,0.000000,0.000000}%
\pgfsetstrokecolor{currentstroke}%
\pgfsetdash{}{0pt}%
\pgfpathmoveto{\pgfqpoint{0.769598in}{1.432246in}}%
\pgfpathlineto{\pgfqpoint{0.782156in}{1.419499in}}%
\pgfusepath{stroke}%
\end{pgfscope}%
\begin{pgfscope}%
\pgfpathrectangle{\pgfqpoint{0.100000in}{0.212622in}}{\pgfqpoint{3.696000in}{3.696000in}}%
\pgfusepath{clip}%
\pgfsetrectcap%
\pgfsetroundjoin%
\pgfsetlinewidth{1.505625pt}%
\definecolor{currentstroke}{rgb}{1.000000,0.000000,0.000000}%
\pgfsetstrokecolor{currentstroke}%
\pgfsetdash{}{0pt}%
\pgfpathmoveto{\pgfqpoint{0.769598in}{1.432246in}}%
\pgfpathlineto{\pgfqpoint{0.782156in}{1.419499in}}%
\pgfusepath{stroke}%
\end{pgfscope}%
\begin{pgfscope}%
\pgfpathrectangle{\pgfqpoint{0.100000in}{0.212622in}}{\pgfqpoint{3.696000in}{3.696000in}}%
\pgfusepath{clip}%
\pgfsetrectcap%
\pgfsetroundjoin%
\pgfsetlinewidth{1.505625pt}%
\definecolor{currentstroke}{rgb}{1.000000,0.000000,0.000000}%
\pgfsetstrokecolor{currentstroke}%
\pgfsetdash{}{0pt}%
\pgfpathmoveto{\pgfqpoint{0.769598in}{1.432246in}}%
\pgfpathlineto{\pgfqpoint{0.782156in}{1.419499in}}%
\pgfusepath{stroke}%
\end{pgfscope}%
\begin{pgfscope}%
\pgfpathrectangle{\pgfqpoint{0.100000in}{0.212622in}}{\pgfqpoint{3.696000in}{3.696000in}}%
\pgfusepath{clip}%
\pgfsetrectcap%
\pgfsetroundjoin%
\pgfsetlinewidth{1.505625pt}%
\definecolor{currentstroke}{rgb}{1.000000,0.000000,0.000000}%
\pgfsetstrokecolor{currentstroke}%
\pgfsetdash{}{0pt}%
\pgfpathmoveto{\pgfqpoint{0.769598in}{1.432246in}}%
\pgfpathlineto{\pgfqpoint{0.782156in}{1.419499in}}%
\pgfusepath{stroke}%
\end{pgfscope}%
\begin{pgfscope}%
\pgfpathrectangle{\pgfqpoint{0.100000in}{0.212622in}}{\pgfqpoint{3.696000in}{3.696000in}}%
\pgfusepath{clip}%
\pgfsetrectcap%
\pgfsetroundjoin%
\pgfsetlinewidth{1.505625pt}%
\definecolor{currentstroke}{rgb}{1.000000,0.000000,0.000000}%
\pgfsetstrokecolor{currentstroke}%
\pgfsetdash{}{0pt}%
\pgfpathmoveto{\pgfqpoint{0.769598in}{1.432246in}}%
\pgfpathlineto{\pgfqpoint{0.782156in}{1.419499in}}%
\pgfusepath{stroke}%
\end{pgfscope}%
\begin{pgfscope}%
\pgfpathrectangle{\pgfqpoint{0.100000in}{0.212622in}}{\pgfqpoint{3.696000in}{3.696000in}}%
\pgfusepath{clip}%
\pgfsetrectcap%
\pgfsetroundjoin%
\pgfsetlinewidth{1.505625pt}%
\definecolor{currentstroke}{rgb}{1.000000,0.000000,0.000000}%
\pgfsetstrokecolor{currentstroke}%
\pgfsetdash{}{0pt}%
\pgfpathmoveto{\pgfqpoint{0.769598in}{1.432246in}}%
\pgfpathlineto{\pgfqpoint{0.782156in}{1.419499in}}%
\pgfusepath{stroke}%
\end{pgfscope}%
\begin{pgfscope}%
\pgfpathrectangle{\pgfqpoint{0.100000in}{0.212622in}}{\pgfqpoint{3.696000in}{3.696000in}}%
\pgfusepath{clip}%
\pgfsetrectcap%
\pgfsetroundjoin%
\pgfsetlinewidth{1.505625pt}%
\definecolor{currentstroke}{rgb}{1.000000,0.000000,0.000000}%
\pgfsetstrokecolor{currentstroke}%
\pgfsetdash{}{0pt}%
\pgfpathmoveto{\pgfqpoint{0.769598in}{1.432246in}}%
\pgfpathlineto{\pgfqpoint{0.782156in}{1.419499in}}%
\pgfusepath{stroke}%
\end{pgfscope}%
\begin{pgfscope}%
\pgfpathrectangle{\pgfqpoint{0.100000in}{0.212622in}}{\pgfqpoint{3.696000in}{3.696000in}}%
\pgfusepath{clip}%
\pgfsetrectcap%
\pgfsetroundjoin%
\pgfsetlinewidth{1.505625pt}%
\definecolor{currentstroke}{rgb}{1.000000,0.000000,0.000000}%
\pgfsetstrokecolor{currentstroke}%
\pgfsetdash{}{0pt}%
\pgfpathmoveto{\pgfqpoint{0.769598in}{1.432246in}}%
\pgfpathlineto{\pgfqpoint{0.782156in}{1.419499in}}%
\pgfusepath{stroke}%
\end{pgfscope}%
\begin{pgfscope}%
\pgfpathrectangle{\pgfqpoint{0.100000in}{0.212622in}}{\pgfqpoint{3.696000in}{3.696000in}}%
\pgfusepath{clip}%
\pgfsetrectcap%
\pgfsetroundjoin%
\pgfsetlinewidth{1.505625pt}%
\definecolor{currentstroke}{rgb}{1.000000,0.000000,0.000000}%
\pgfsetstrokecolor{currentstroke}%
\pgfsetdash{}{0pt}%
\pgfpathmoveto{\pgfqpoint{0.769598in}{1.432246in}}%
\pgfpathlineto{\pgfqpoint{0.782156in}{1.419499in}}%
\pgfusepath{stroke}%
\end{pgfscope}%
\begin{pgfscope}%
\pgfpathrectangle{\pgfqpoint{0.100000in}{0.212622in}}{\pgfqpoint{3.696000in}{3.696000in}}%
\pgfusepath{clip}%
\pgfsetrectcap%
\pgfsetroundjoin%
\pgfsetlinewidth{1.505625pt}%
\definecolor{currentstroke}{rgb}{1.000000,0.000000,0.000000}%
\pgfsetstrokecolor{currentstroke}%
\pgfsetdash{}{0pt}%
\pgfpathmoveto{\pgfqpoint{0.769598in}{1.432246in}}%
\pgfpathlineto{\pgfqpoint{0.782156in}{1.419499in}}%
\pgfusepath{stroke}%
\end{pgfscope}%
\begin{pgfscope}%
\pgfpathrectangle{\pgfqpoint{0.100000in}{0.212622in}}{\pgfqpoint{3.696000in}{3.696000in}}%
\pgfusepath{clip}%
\pgfsetrectcap%
\pgfsetroundjoin%
\pgfsetlinewidth{1.505625pt}%
\definecolor{currentstroke}{rgb}{1.000000,0.000000,0.000000}%
\pgfsetstrokecolor{currentstroke}%
\pgfsetdash{}{0pt}%
\pgfpathmoveto{\pgfqpoint{0.769598in}{1.432246in}}%
\pgfpathlineto{\pgfqpoint{0.782156in}{1.419499in}}%
\pgfusepath{stroke}%
\end{pgfscope}%
\begin{pgfscope}%
\pgfpathrectangle{\pgfqpoint{0.100000in}{0.212622in}}{\pgfqpoint{3.696000in}{3.696000in}}%
\pgfusepath{clip}%
\pgfsetrectcap%
\pgfsetroundjoin%
\pgfsetlinewidth{1.505625pt}%
\definecolor{currentstroke}{rgb}{1.000000,0.000000,0.000000}%
\pgfsetstrokecolor{currentstroke}%
\pgfsetdash{}{0pt}%
\pgfpathmoveto{\pgfqpoint{0.769235in}{1.432157in}}%
\pgfpathlineto{\pgfqpoint{0.782156in}{1.419499in}}%
\pgfusepath{stroke}%
\end{pgfscope}%
\begin{pgfscope}%
\pgfpathrectangle{\pgfqpoint{0.100000in}{0.212622in}}{\pgfqpoint{3.696000in}{3.696000in}}%
\pgfusepath{clip}%
\pgfsetrectcap%
\pgfsetroundjoin%
\pgfsetlinewidth{1.505625pt}%
\definecolor{currentstroke}{rgb}{1.000000,0.000000,0.000000}%
\pgfsetstrokecolor{currentstroke}%
\pgfsetdash{}{0pt}%
\pgfpathmoveto{\pgfqpoint{0.768309in}{1.432064in}}%
\pgfpathlineto{\pgfqpoint{0.782156in}{1.419499in}}%
\pgfusepath{stroke}%
\end{pgfscope}%
\begin{pgfscope}%
\pgfpathrectangle{\pgfqpoint{0.100000in}{0.212622in}}{\pgfqpoint{3.696000in}{3.696000in}}%
\pgfusepath{clip}%
\pgfsetrectcap%
\pgfsetroundjoin%
\pgfsetlinewidth{1.505625pt}%
\definecolor{currentstroke}{rgb}{1.000000,0.000000,0.000000}%
\pgfsetstrokecolor{currentstroke}%
\pgfsetdash{}{0pt}%
\pgfpathmoveto{\pgfqpoint{0.766562in}{1.432106in}}%
\pgfpathlineto{\pgfqpoint{0.782156in}{1.419499in}}%
\pgfusepath{stroke}%
\end{pgfscope}%
\begin{pgfscope}%
\pgfpathrectangle{\pgfqpoint{0.100000in}{0.212622in}}{\pgfqpoint{3.696000in}{3.696000in}}%
\pgfusepath{clip}%
\pgfsetrectcap%
\pgfsetroundjoin%
\pgfsetlinewidth{1.505625pt}%
\definecolor{currentstroke}{rgb}{1.000000,0.000000,0.000000}%
\pgfsetstrokecolor{currentstroke}%
\pgfsetdash{}{0pt}%
\pgfpathmoveto{\pgfqpoint{0.763742in}{1.432546in}}%
\pgfpathlineto{\pgfqpoint{0.782156in}{1.419499in}}%
\pgfusepath{stroke}%
\end{pgfscope}%
\begin{pgfscope}%
\pgfpathrectangle{\pgfqpoint{0.100000in}{0.212622in}}{\pgfqpoint{3.696000in}{3.696000in}}%
\pgfusepath{clip}%
\pgfsetrectcap%
\pgfsetroundjoin%
\pgfsetlinewidth{1.505625pt}%
\definecolor{currentstroke}{rgb}{1.000000,0.000000,0.000000}%
\pgfsetstrokecolor{currentstroke}%
\pgfsetdash{}{0pt}%
\pgfpathmoveto{\pgfqpoint{0.760931in}{1.433809in}}%
\pgfpathlineto{\pgfqpoint{0.782156in}{1.419499in}}%
\pgfusepath{stroke}%
\end{pgfscope}%
\begin{pgfscope}%
\pgfpathrectangle{\pgfqpoint{0.100000in}{0.212622in}}{\pgfqpoint{3.696000in}{3.696000in}}%
\pgfusepath{clip}%
\pgfsetrectcap%
\pgfsetroundjoin%
\pgfsetlinewidth{1.505625pt}%
\definecolor{currentstroke}{rgb}{1.000000,0.000000,0.000000}%
\pgfsetstrokecolor{currentstroke}%
\pgfsetdash{}{0pt}%
\pgfpathmoveto{\pgfqpoint{0.758641in}{1.439250in}}%
\pgfpathlineto{\pgfqpoint{0.782156in}{1.419499in}}%
\pgfusepath{stroke}%
\end{pgfscope}%
\begin{pgfscope}%
\pgfpathrectangle{\pgfqpoint{0.100000in}{0.212622in}}{\pgfqpoint{3.696000in}{3.696000in}}%
\pgfusepath{clip}%
\pgfsetrectcap%
\pgfsetroundjoin%
\pgfsetlinewidth{1.505625pt}%
\definecolor{currentstroke}{rgb}{1.000000,0.000000,0.000000}%
\pgfsetstrokecolor{currentstroke}%
\pgfsetdash{}{0pt}%
\pgfpathmoveto{\pgfqpoint{0.758797in}{1.444456in}}%
\pgfpathlineto{\pgfqpoint{0.782156in}{1.419499in}}%
\pgfusepath{stroke}%
\end{pgfscope}%
\begin{pgfscope}%
\pgfpathrectangle{\pgfqpoint{0.100000in}{0.212622in}}{\pgfqpoint{3.696000in}{3.696000in}}%
\pgfusepath{clip}%
\pgfsetrectcap%
\pgfsetroundjoin%
\pgfsetlinewidth{1.505625pt}%
\definecolor{currentstroke}{rgb}{1.000000,0.000000,0.000000}%
\pgfsetstrokecolor{currentstroke}%
\pgfsetdash{}{0pt}%
\pgfpathmoveto{\pgfqpoint{0.759843in}{1.447313in}}%
\pgfpathlineto{\pgfqpoint{0.782156in}{1.419499in}}%
\pgfusepath{stroke}%
\end{pgfscope}%
\begin{pgfscope}%
\pgfpathrectangle{\pgfqpoint{0.100000in}{0.212622in}}{\pgfqpoint{3.696000in}{3.696000in}}%
\pgfusepath{clip}%
\pgfsetrectcap%
\pgfsetroundjoin%
\pgfsetlinewidth{1.505625pt}%
\definecolor{currentstroke}{rgb}{1.000000,0.000000,0.000000}%
\pgfsetstrokecolor{currentstroke}%
\pgfsetdash{}{0pt}%
\pgfpathmoveto{\pgfqpoint{0.761008in}{1.447216in}}%
\pgfpathlineto{\pgfqpoint{0.782156in}{1.419499in}}%
\pgfusepath{stroke}%
\end{pgfscope}%
\begin{pgfscope}%
\pgfpathrectangle{\pgfqpoint{0.100000in}{0.212622in}}{\pgfqpoint{3.696000in}{3.696000in}}%
\pgfusepath{clip}%
\pgfsetrectcap%
\pgfsetroundjoin%
\pgfsetlinewidth{1.505625pt}%
\definecolor{currentstroke}{rgb}{1.000000,0.000000,0.000000}%
\pgfsetstrokecolor{currentstroke}%
\pgfsetdash{}{0pt}%
\pgfpathmoveto{\pgfqpoint{0.762805in}{1.447783in}}%
\pgfpathlineto{\pgfqpoint{0.782156in}{1.419499in}}%
\pgfusepath{stroke}%
\end{pgfscope}%
\begin{pgfscope}%
\pgfpathrectangle{\pgfqpoint{0.100000in}{0.212622in}}{\pgfqpoint{3.696000in}{3.696000in}}%
\pgfusepath{clip}%
\pgfsetrectcap%
\pgfsetroundjoin%
\pgfsetlinewidth{1.505625pt}%
\definecolor{currentstroke}{rgb}{1.000000,0.000000,0.000000}%
\pgfsetstrokecolor{currentstroke}%
\pgfsetdash{}{0pt}%
\pgfpathmoveto{\pgfqpoint{0.765123in}{1.448300in}}%
\pgfpathlineto{\pgfqpoint{0.782156in}{1.419499in}}%
\pgfusepath{stroke}%
\end{pgfscope}%
\begin{pgfscope}%
\pgfpathrectangle{\pgfqpoint{0.100000in}{0.212622in}}{\pgfqpoint{3.696000in}{3.696000in}}%
\pgfusepath{clip}%
\pgfsetrectcap%
\pgfsetroundjoin%
\pgfsetlinewidth{1.505625pt}%
\definecolor{currentstroke}{rgb}{1.000000,0.000000,0.000000}%
\pgfsetstrokecolor{currentstroke}%
\pgfsetdash{}{0pt}%
\pgfpathmoveto{\pgfqpoint{0.767645in}{1.451475in}}%
\pgfpathlineto{\pgfqpoint{0.782156in}{1.419499in}}%
\pgfusepath{stroke}%
\end{pgfscope}%
\begin{pgfscope}%
\pgfpathrectangle{\pgfqpoint{0.100000in}{0.212622in}}{\pgfqpoint{3.696000in}{3.696000in}}%
\pgfusepath{clip}%
\pgfsetrectcap%
\pgfsetroundjoin%
\pgfsetlinewidth{1.505625pt}%
\definecolor{currentstroke}{rgb}{1.000000,0.000000,0.000000}%
\pgfsetstrokecolor{currentstroke}%
\pgfsetdash{}{0pt}%
\pgfpathmoveto{\pgfqpoint{0.770783in}{1.451434in}}%
\pgfpathlineto{\pgfqpoint{0.782156in}{1.419499in}}%
\pgfusepath{stroke}%
\end{pgfscope}%
\begin{pgfscope}%
\pgfpathrectangle{\pgfqpoint{0.100000in}{0.212622in}}{\pgfqpoint{3.696000in}{3.696000in}}%
\pgfusepath{clip}%
\pgfsetrectcap%
\pgfsetroundjoin%
\pgfsetlinewidth{1.505625pt}%
\definecolor{currentstroke}{rgb}{1.000000,0.000000,0.000000}%
\pgfsetstrokecolor{currentstroke}%
\pgfsetdash{}{0pt}%
\pgfpathmoveto{\pgfqpoint{0.772484in}{1.452477in}}%
\pgfpathlineto{\pgfqpoint{0.782156in}{1.419499in}}%
\pgfusepath{stroke}%
\end{pgfscope}%
\begin{pgfscope}%
\pgfpathrectangle{\pgfqpoint{0.100000in}{0.212622in}}{\pgfqpoint{3.696000in}{3.696000in}}%
\pgfusepath{clip}%
\pgfsetrectcap%
\pgfsetroundjoin%
\pgfsetlinewidth{1.505625pt}%
\definecolor{currentstroke}{rgb}{1.000000,0.000000,0.000000}%
\pgfsetstrokecolor{currentstroke}%
\pgfsetdash{}{0pt}%
\pgfpathmoveto{\pgfqpoint{0.773338in}{1.453236in}}%
\pgfpathlineto{\pgfqpoint{0.782156in}{1.419499in}}%
\pgfusepath{stroke}%
\end{pgfscope}%
\begin{pgfscope}%
\pgfpathrectangle{\pgfqpoint{0.100000in}{0.212622in}}{\pgfqpoint{3.696000in}{3.696000in}}%
\pgfusepath{clip}%
\pgfsetrectcap%
\pgfsetroundjoin%
\pgfsetlinewidth{1.505625pt}%
\definecolor{currentstroke}{rgb}{1.000000,0.000000,0.000000}%
\pgfsetstrokecolor{currentstroke}%
\pgfsetdash{}{0pt}%
\pgfpathmoveto{\pgfqpoint{0.773816in}{1.453458in}}%
\pgfpathlineto{\pgfqpoint{0.782156in}{1.419499in}}%
\pgfusepath{stroke}%
\end{pgfscope}%
\begin{pgfscope}%
\pgfpathrectangle{\pgfqpoint{0.100000in}{0.212622in}}{\pgfqpoint{3.696000in}{3.696000in}}%
\pgfusepath{clip}%
\pgfsetrectcap%
\pgfsetroundjoin%
\pgfsetlinewidth{1.505625pt}%
\definecolor{currentstroke}{rgb}{1.000000,0.000000,0.000000}%
\pgfsetstrokecolor{currentstroke}%
\pgfsetdash{}{0pt}%
\pgfpathmoveto{\pgfqpoint{0.774096in}{1.453660in}}%
\pgfpathlineto{\pgfqpoint{0.782156in}{1.419499in}}%
\pgfusepath{stroke}%
\end{pgfscope}%
\begin{pgfscope}%
\pgfpathrectangle{\pgfqpoint{0.100000in}{0.212622in}}{\pgfqpoint{3.696000in}{3.696000in}}%
\pgfusepath{clip}%
\pgfsetrectcap%
\pgfsetroundjoin%
\pgfsetlinewidth{1.505625pt}%
\definecolor{currentstroke}{rgb}{1.000000,0.000000,0.000000}%
\pgfsetstrokecolor{currentstroke}%
\pgfsetdash{}{0pt}%
\pgfpathmoveto{\pgfqpoint{0.774244in}{1.453729in}}%
\pgfpathlineto{\pgfqpoint{0.782156in}{1.419499in}}%
\pgfusepath{stroke}%
\end{pgfscope}%
\begin{pgfscope}%
\pgfpathrectangle{\pgfqpoint{0.100000in}{0.212622in}}{\pgfqpoint{3.696000in}{3.696000in}}%
\pgfusepath{clip}%
\pgfsetrectcap%
\pgfsetroundjoin%
\pgfsetlinewidth{1.505625pt}%
\definecolor{currentstroke}{rgb}{1.000000,0.000000,0.000000}%
\pgfsetstrokecolor{currentstroke}%
\pgfsetdash{}{0pt}%
\pgfpathmoveto{\pgfqpoint{0.774323in}{1.453820in}}%
\pgfpathlineto{\pgfqpoint{0.782156in}{1.419499in}}%
\pgfusepath{stroke}%
\end{pgfscope}%
\begin{pgfscope}%
\pgfpathrectangle{\pgfqpoint{0.100000in}{0.212622in}}{\pgfqpoint{3.696000in}{3.696000in}}%
\pgfusepath{clip}%
\pgfsetrectcap%
\pgfsetroundjoin%
\pgfsetlinewidth{1.505625pt}%
\definecolor{currentstroke}{rgb}{1.000000,0.000000,0.000000}%
\pgfsetstrokecolor{currentstroke}%
\pgfsetdash{}{0pt}%
\pgfpathmoveto{\pgfqpoint{0.774378in}{1.453808in}}%
\pgfpathlineto{\pgfqpoint{0.782156in}{1.419499in}}%
\pgfusepath{stroke}%
\end{pgfscope}%
\begin{pgfscope}%
\pgfpathrectangle{\pgfqpoint{0.100000in}{0.212622in}}{\pgfqpoint{3.696000in}{3.696000in}}%
\pgfusepath{clip}%
\pgfsetrectcap%
\pgfsetroundjoin%
\pgfsetlinewidth{1.505625pt}%
\definecolor{currentstroke}{rgb}{1.000000,0.000000,0.000000}%
\pgfsetstrokecolor{currentstroke}%
\pgfsetdash{}{0pt}%
\pgfpathmoveto{\pgfqpoint{0.774400in}{1.453818in}}%
\pgfpathlineto{\pgfqpoint{0.782156in}{1.419499in}}%
\pgfusepath{stroke}%
\end{pgfscope}%
\begin{pgfscope}%
\pgfpathrectangle{\pgfqpoint{0.100000in}{0.212622in}}{\pgfqpoint{3.696000in}{3.696000in}}%
\pgfusepath{clip}%
\pgfsetrectcap%
\pgfsetroundjoin%
\pgfsetlinewidth{1.505625pt}%
\definecolor{currentstroke}{rgb}{1.000000,0.000000,0.000000}%
\pgfsetstrokecolor{currentstroke}%
\pgfsetdash{}{0pt}%
\pgfpathmoveto{\pgfqpoint{0.774415in}{1.453828in}}%
\pgfpathlineto{\pgfqpoint{0.782156in}{1.419499in}}%
\pgfusepath{stroke}%
\end{pgfscope}%
\begin{pgfscope}%
\pgfpathrectangle{\pgfqpoint{0.100000in}{0.212622in}}{\pgfqpoint{3.696000in}{3.696000in}}%
\pgfusepath{clip}%
\pgfsetrectcap%
\pgfsetroundjoin%
\pgfsetlinewidth{1.505625pt}%
\definecolor{currentstroke}{rgb}{1.000000,0.000000,0.000000}%
\pgfsetstrokecolor{currentstroke}%
\pgfsetdash{}{0pt}%
\pgfpathmoveto{\pgfqpoint{0.774422in}{1.453831in}}%
\pgfpathlineto{\pgfqpoint{0.782156in}{1.419499in}}%
\pgfusepath{stroke}%
\end{pgfscope}%
\begin{pgfscope}%
\pgfpathrectangle{\pgfqpoint{0.100000in}{0.212622in}}{\pgfqpoint{3.696000in}{3.696000in}}%
\pgfusepath{clip}%
\pgfsetrectcap%
\pgfsetroundjoin%
\pgfsetlinewidth{1.505625pt}%
\definecolor{currentstroke}{rgb}{1.000000,0.000000,0.000000}%
\pgfsetstrokecolor{currentstroke}%
\pgfsetdash{}{0pt}%
\pgfpathmoveto{\pgfqpoint{0.774426in}{1.453834in}}%
\pgfpathlineto{\pgfqpoint{0.782156in}{1.419499in}}%
\pgfusepath{stroke}%
\end{pgfscope}%
\begin{pgfscope}%
\pgfpathrectangle{\pgfqpoint{0.100000in}{0.212622in}}{\pgfqpoint{3.696000in}{3.696000in}}%
\pgfusepath{clip}%
\pgfsetrectcap%
\pgfsetroundjoin%
\pgfsetlinewidth{1.505625pt}%
\definecolor{currentstroke}{rgb}{1.000000,0.000000,0.000000}%
\pgfsetstrokecolor{currentstroke}%
\pgfsetdash{}{0pt}%
\pgfpathmoveto{\pgfqpoint{0.774429in}{1.453836in}}%
\pgfpathlineto{\pgfqpoint{0.782156in}{1.419499in}}%
\pgfusepath{stroke}%
\end{pgfscope}%
\begin{pgfscope}%
\pgfpathrectangle{\pgfqpoint{0.100000in}{0.212622in}}{\pgfqpoint{3.696000in}{3.696000in}}%
\pgfusepath{clip}%
\pgfsetrectcap%
\pgfsetroundjoin%
\pgfsetlinewidth{1.505625pt}%
\definecolor{currentstroke}{rgb}{1.000000,0.000000,0.000000}%
\pgfsetstrokecolor{currentstroke}%
\pgfsetdash{}{0pt}%
\pgfpathmoveto{\pgfqpoint{0.774430in}{1.453836in}}%
\pgfpathlineto{\pgfqpoint{0.782156in}{1.419499in}}%
\pgfusepath{stroke}%
\end{pgfscope}%
\begin{pgfscope}%
\pgfpathrectangle{\pgfqpoint{0.100000in}{0.212622in}}{\pgfqpoint{3.696000in}{3.696000in}}%
\pgfusepath{clip}%
\pgfsetrectcap%
\pgfsetroundjoin%
\pgfsetlinewidth{1.505625pt}%
\definecolor{currentstroke}{rgb}{1.000000,0.000000,0.000000}%
\pgfsetstrokecolor{currentstroke}%
\pgfsetdash{}{0pt}%
\pgfpathmoveto{\pgfqpoint{0.774431in}{1.453837in}}%
\pgfpathlineto{\pgfqpoint{0.782156in}{1.419499in}}%
\pgfusepath{stroke}%
\end{pgfscope}%
\begin{pgfscope}%
\pgfpathrectangle{\pgfqpoint{0.100000in}{0.212622in}}{\pgfqpoint{3.696000in}{3.696000in}}%
\pgfusepath{clip}%
\pgfsetrectcap%
\pgfsetroundjoin%
\pgfsetlinewidth{1.505625pt}%
\definecolor{currentstroke}{rgb}{1.000000,0.000000,0.000000}%
\pgfsetstrokecolor{currentstroke}%
\pgfsetdash{}{0pt}%
\pgfpathmoveto{\pgfqpoint{0.774431in}{1.453837in}}%
\pgfpathlineto{\pgfqpoint{0.782156in}{1.419499in}}%
\pgfusepath{stroke}%
\end{pgfscope}%
\begin{pgfscope}%
\pgfpathrectangle{\pgfqpoint{0.100000in}{0.212622in}}{\pgfqpoint{3.696000in}{3.696000in}}%
\pgfusepath{clip}%
\pgfsetrectcap%
\pgfsetroundjoin%
\pgfsetlinewidth{1.505625pt}%
\definecolor{currentstroke}{rgb}{1.000000,0.000000,0.000000}%
\pgfsetstrokecolor{currentstroke}%
\pgfsetdash{}{0pt}%
\pgfpathmoveto{\pgfqpoint{0.774431in}{1.453837in}}%
\pgfpathlineto{\pgfqpoint{0.782156in}{1.419499in}}%
\pgfusepath{stroke}%
\end{pgfscope}%
\begin{pgfscope}%
\pgfpathrectangle{\pgfqpoint{0.100000in}{0.212622in}}{\pgfqpoint{3.696000in}{3.696000in}}%
\pgfusepath{clip}%
\pgfsetrectcap%
\pgfsetroundjoin%
\pgfsetlinewidth{1.505625pt}%
\definecolor{currentstroke}{rgb}{1.000000,0.000000,0.000000}%
\pgfsetstrokecolor{currentstroke}%
\pgfsetdash{}{0pt}%
\pgfpathmoveto{\pgfqpoint{0.774431in}{1.453837in}}%
\pgfpathlineto{\pgfqpoint{0.782156in}{1.419499in}}%
\pgfusepath{stroke}%
\end{pgfscope}%
\begin{pgfscope}%
\pgfpathrectangle{\pgfqpoint{0.100000in}{0.212622in}}{\pgfqpoint{3.696000in}{3.696000in}}%
\pgfusepath{clip}%
\pgfsetrectcap%
\pgfsetroundjoin%
\pgfsetlinewidth{1.505625pt}%
\definecolor{currentstroke}{rgb}{1.000000,0.000000,0.000000}%
\pgfsetstrokecolor{currentstroke}%
\pgfsetdash{}{0pt}%
\pgfpathmoveto{\pgfqpoint{0.774431in}{1.453837in}}%
\pgfpathlineto{\pgfqpoint{0.782156in}{1.419499in}}%
\pgfusepath{stroke}%
\end{pgfscope}%
\begin{pgfscope}%
\pgfpathrectangle{\pgfqpoint{0.100000in}{0.212622in}}{\pgfqpoint{3.696000in}{3.696000in}}%
\pgfusepath{clip}%
\pgfsetrectcap%
\pgfsetroundjoin%
\pgfsetlinewidth{1.505625pt}%
\definecolor{currentstroke}{rgb}{1.000000,0.000000,0.000000}%
\pgfsetstrokecolor{currentstroke}%
\pgfsetdash{}{0pt}%
\pgfpathmoveto{\pgfqpoint{0.774431in}{1.453837in}}%
\pgfpathlineto{\pgfqpoint{0.782156in}{1.419499in}}%
\pgfusepath{stroke}%
\end{pgfscope}%
\begin{pgfscope}%
\pgfpathrectangle{\pgfqpoint{0.100000in}{0.212622in}}{\pgfqpoint{3.696000in}{3.696000in}}%
\pgfusepath{clip}%
\pgfsetrectcap%
\pgfsetroundjoin%
\pgfsetlinewidth{1.505625pt}%
\definecolor{currentstroke}{rgb}{1.000000,0.000000,0.000000}%
\pgfsetstrokecolor{currentstroke}%
\pgfsetdash{}{0pt}%
\pgfpathmoveto{\pgfqpoint{0.775134in}{1.452879in}}%
\pgfpathlineto{\pgfqpoint{0.792809in}{1.428460in}}%
\pgfusepath{stroke}%
\end{pgfscope}%
\begin{pgfscope}%
\pgfpathrectangle{\pgfqpoint{0.100000in}{0.212622in}}{\pgfqpoint{3.696000in}{3.696000in}}%
\pgfusepath{clip}%
\pgfsetrectcap%
\pgfsetroundjoin%
\pgfsetlinewidth{1.505625pt}%
\definecolor{currentstroke}{rgb}{1.000000,0.000000,0.000000}%
\pgfsetstrokecolor{currentstroke}%
\pgfsetdash{}{0pt}%
\pgfpathmoveto{\pgfqpoint{0.777394in}{1.452248in}}%
\pgfpathlineto{\pgfqpoint{0.792809in}{1.428460in}}%
\pgfusepath{stroke}%
\end{pgfscope}%
\begin{pgfscope}%
\pgfpathrectangle{\pgfqpoint{0.100000in}{0.212622in}}{\pgfqpoint{3.696000in}{3.696000in}}%
\pgfusepath{clip}%
\pgfsetrectcap%
\pgfsetroundjoin%
\pgfsetlinewidth{1.505625pt}%
\definecolor{currentstroke}{rgb}{1.000000,0.000000,0.000000}%
\pgfsetstrokecolor{currentstroke}%
\pgfsetdash{}{0pt}%
\pgfpathmoveto{\pgfqpoint{0.779574in}{1.451703in}}%
\pgfpathlineto{\pgfqpoint{0.792809in}{1.428460in}}%
\pgfusepath{stroke}%
\end{pgfscope}%
\begin{pgfscope}%
\pgfpathrectangle{\pgfqpoint{0.100000in}{0.212622in}}{\pgfqpoint{3.696000in}{3.696000in}}%
\pgfusepath{clip}%
\pgfsetrectcap%
\pgfsetroundjoin%
\pgfsetlinewidth{1.505625pt}%
\definecolor{currentstroke}{rgb}{1.000000,0.000000,0.000000}%
\pgfsetstrokecolor{currentstroke}%
\pgfsetdash{}{0pt}%
\pgfpathmoveto{\pgfqpoint{0.785098in}{1.450564in}}%
\pgfpathlineto{\pgfqpoint{0.792809in}{1.428460in}}%
\pgfusepath{stroke}%
\end{pgfscope}%
\begin{pgfscope}%
\pgfpathrectangle{\pgfqpoint{0.100000in}{0.212622in}}{\pgfqpoint{3.696000in}{3.696000in}}%
\pgfusepath{clip}%
\pgfsetrectcap%
\pgfsetroundjoin%
\pgfsetlinewidth{1.505625pt}%
\definecolor{currentstroke}{rgb}{1.000000,0.000000,0.000000}%
\pgfsetstrokecolor{currentstroke}%
\pgfsetdash{}{0pt}%
\pgfpathmoveto{\pgfqpoint{0.789252in}{1.455433in}}%
\pgfpathlineto{\pgfqpoint{0.803447in}{1.437407in}}%
\pgfusepath{stroke}%
\end{pgfscope}%
\begin{pgfscope}%
\pgfpathrectangle{\pgfqpoint{0.100000in}{0.212622in}}{\pgfqpoint{3.696000in}{3.696000in}}%
\pgfusepath{clip}%
\pgfsetrectcap%
\pgfsetroundjoin%
\pgfsetlinewidth{1.505625pt}%
\definecolor{currentstroke}{rgb}{1.000000,0.000000,0.000000}%
\pgfsetstrokecolor{currentstroke}%
\pgfsetdash{}{0pt}%
\pgfpathmoveto{\pgfqpoint{0.791927in}{1.456871in}}%
\pgfpathlineto{\pgfqpoint{0.803447in}{1.437407in}}%
\pgfusepath{stroke}%
\end{pgfscope}%
\begin{pgfscope}%
\pgfpathrectangle{\pgfqpoint{0.100000in}{0.212622in}}{\pgfqpoint{3.696000in}{3.696000in}}%
\pgfusepath{clip}%
\pgfsetrectcap%
\pgfsetroundjoin%
\pgfsetlinewidth{1.505625pt}%
\definecolor{currentstroke}{rgb}{1.000000,0.000000,0.000000}%
\pgfsetstrokecolor{currentstroke}%
\pgfsetdash{}{0pt}%
\pgfpathmoveto{\pgfqpoint{0.793698in}{1.458701in}}%
\pgfpathlineto{\pgfqpoint{0.803447in}{1.437407in}}%
\pgfusepath{stroke}%
\end{pgfscope}%
\begin{pgfscope}%
\pgfpathrectangle{\pgfqpoint{0.100000in}{0.212622in}}{\pgfqpoint{3.696000in}{3.696000in}}%
\pgfusepath{clip}%
\pgfsetrectcap%
\pgfsetroundjoin%
\pgfsetlinewidth{1.505625pt}%
\definecolor{currentstroke}{rgb}{1.000000,0.000000,0.000000}%
\pgfsetstrokecolor{currentstroke}%
\pgfsetdash{}{0pt}%
\pgfpathmoveto{\pgfqpoint{0.795462in}{1.456368in}}%
\pgfpathlineto{\pgfqpoint{0.803447in}{1.437407in}}%
\pgfusepath{stroke}%
\end{pgfscope}%
\begin{pgfscope}%
\pgfpathrectangle{\pgfqpoint{0.100000in}{0.212622in}}{\pgfqpoint{3.696000in}{3.696000in}}%
\pgfusepath{clip}%
\pgfsetrectcap%
\pgfsetroundjoin%
\pgfsetlinewidth{1.505625pt}%
\definecolor{currentstroke}{rgb}{1.000000,0.000000,0.000000}%
\pgfsetstrokecolor{currentstroke}%
\pgfsetdash{}{0pt}%
\pgfpathmoveto{\pgfqpoint{0.799346in}{1.456467in}}%
\pgfpathlineto{\pgfqpoint{0.814069in}{1.446342in}}%
\pgfusepath{stroke}%
\end{pgfscope}%
\begin{pgfscope}%
\pgfpathrectangle{\pgfqpoint{0.100000in}{0.212622in}}{\pgfqpoint{3.696000in}{3.696000in}}%
\pgfusepath{clip}%
\pgfsetrectcap%
\pgfsetroundjoin%
\pgfsetlinewidth{1.505625pt}%
\definecolor{currentstroke}{rgb}{1.000000,0.000000,0.000000}%
\pgfsetstrokecolor{currentstroke}%
\pgfsetdash{}{0pt}%
\pgfpathmoveto{\pgfqpoint{0.802763in}{1.456565in}}%
\pgfpathlineto{\pgfqpoint{0.814069in}{1.446342in}}%
\pgfusepath{stroke}%
\end{pgfscope}%
\begin{pgfscope}%
\pgfpathrectangle{\pgfqpoint{0.100000in}{0.212622in}}{\pgfqpoint{3.696000in}{3.696000in}}%
\pgfusepath{clip}%
\pgfsetrectcap%
\pgfsetroundjoin%
\pgfsetlinewidth{1.505625pt}%
\definecolor{currentstroke}{rgb}{1.000000,0.000000,0.000000}%
\pgfsetstrokecolor{currentstroke}%
\pgfsetdash{}{0pt}%
\pgfpathmoveto{\pgfqpoint{0.809277in}{1.456220in}}%
\pgfpathlineto{\pgfqpoint{0.824675in}{1.455263in}}%
\pgfusepath{stroke}%
\end{pgfscope}%
\begin{pgfscope}%
\pgfpathrectangle{\pgfqpoint{0.100000in}{0.212622in}}{\pgfqpoint{3.696000in}{3.696000in}}%
\pgfusepath{clip}%
\pgfsetrectcap%
\pgfsetroundjoin%
\pgfsetlinewidth{1.505625pt}%
\definecolor{currentstroke}{rgb}{1.000000,0.000000,0.000000}%
\pgfsetstrokecolor{currentstroke}%
\pgfsetdash{}{0pt}%
\pgfpathmoveto{\pgfqpoint{0.813931in}{1.463686in}}%
\pgfpathlineto{\pgfqpoint{0.824675in}{1.455263in}}%
\pgfusepath{stroke}%
\end{pgfscope}%
\begin{pgfscope}%
\pgfpathrectangle{\pgfqpoint{0.100000in}{0.212622in}}{\pgfqpoint{3.696000in}{3.696000in}}%
\pgfusepath{clip}%
\pgfsetrectcap%
\pgfsetroundjoin%
\pgfsetlinewidth{1.505625pt}%
\definecolor{currentstroke}{rgb}{1.000000,0.000000,0.000000}%
\pgfsetstrokecolor{currentstroke}%
\pgfsetdash{}{0pt}%
\pgfpathmoveto{\pgfqpoint{0.817038in}{1.464899in}}%
\pgfpathlineto{\pgfqpoint{0.824675in}{1.455263in}}%
\pgfusepath{stroke}%
\end{pgfscope}%
\begin{pgfscope}%
\pgfpathrectangle{\pgfqpoint{0.100000in}{0.212622in}}{\pgfqpoint{3.696000in}{3.696000in}}%
\pgfusepath{clip}%
\pgfsetrectcap%
\pgfsetroundjoin%
\pgfsetlinewidth{1.505625pt}%
\definecolor{currentstroke}{rgb}{1.000000,0.000000,0.000000}%
\pgfsetstrokecolor{currentstroke}%
\pgfsetdash{}{0pt}%
\pgfpathmoveto{\pgfqpoint{0.818985in}{1.467253in}}%
\pgfpathlineto{\pgfqpoint{0.835266in}{1.464171in}}%
\pgfusepath{stroke}%
\end{pgfscope}%
\begin{pgfscope}%
\pgfpathrectangle{\pgfqpoint{0.100000in}{0.212622in}}{\pgfqpoint{3.696000in}{3.696000in}}%
\pgfusepath{clip}%
\pgfsetrectcap%
\pgfsetroundjoin%
\pgfsetlinewidth{1.505625pt}%
\definecolor{currentstroke}{rgb}{1.000000,0.000000,0.000000}%
\pgfsetstrokecolor{currentstroke}%
\pgfsetdash{}{0pt}%
\pgfpathmoveto{\pgfqpoint{0.820401in}{1.465854in}}%
\pgfpathlineto{\pgfqpoint{0.835266in}{1.464171in}}%
\pgfusepath{stroke}%
\end{pgfscope}%
\begin{pgfscope}%
\pgfpathrectangle{\pgfqpoint{0.100000in}{0.212622in}}{\pgfqpoint{3.696000in}{3.696000in}}%
\pgfusepath{clip}%
\pgfsetrectcap%
\pgfsetroundjoin%
\pgfsetlinewidth{1.505625pt}%
\definecolor{currentstroke}{rgb}{1.000000,0.000000,0.000000}%
\pgfsetstrokecolor{currentstroke}%
\pgfsetdash{}{0pt}%
\pgfpathmoveto{\pgfqpoint{0.824622in}{1.465377in}}%
\pgfpathlineto{\pgfqpoint{0.835266in}{1.464171in}}%
\pgfusepath{stroke}%
\end{pgfscope}%
\begin{pgfscope}%
\pgfpathrectangle{\pgfqpoint{0.100000in}{0.212622in}}{\pgfqpoint{3.696000in}{3.696000in}}%
\pgfusepath{clip}%
\pgfsetrectcap%
\pgfsetroundjoin%
\pgfsetlinewidth{1.505625pt}%
\definecolor{currentstroke}{rgb}{1.000000,0.000000,0.000000}%
\pgfsetstrokecolor{currentstroke}%
\pgfsetdash{}{0pt}%
\pgfpathmoveto{\pgfqpoint{0.827902in}{1.466271in}}%
\pgfpathlineto{\pgfqpoint{0.835266in}{1.464171in}}%
\pgfusepath{stroke}%
\end{pgfscope}%
\begin{pgfscope}%
\pgfpathrectangle{\pgfqpoint{0.100000in}{0.212622in}}{\pgfqpoint{3.696000in}{3.696000in}}%
\pgfusepath{clip}%
\pgfsetrectcap%
\pgfsetroundjoin%
\pgfsetlinewidth{1.505625pt}%
\definecolor{currentstroke}{rgb}{1.000000,0.000000,0.000000}%
\pgfsetstrokecolor{currentstroke}%
\pgfsetdash{}{0pt}%
\pgfpathmoveto{\pgfqpoint{0.833516in}{1.470428in}}%
\pgfpathlineto{\pgfqpoint{0.845841in}{1.473067in}}%
\pgfusepath{stroke}%
\end{pgfscope}%
\begin{pgfscope}%
\pgfpathrectangle{\pgfqpoint{0.100000in}{0.212622in}}{\pgfqpoint{3.696000in}{3.696000in}}%
\pgfusepath{clip}%
\pgfsetrectcap%
\pgfsetroundjoin%
\pgfsetlinewidth{1.505625pt}%
\definecolor{currentstroke}{rgb}{1.000000,0.000000,0.000000}%
\pgfsetstrokecolor{currentstroke}%
\pgfsetdash{}{0pt}%
\pgfpathmoveto{\pgfqpoint{0.839043in}{1.480546in}}%
\pgfpathlineto{\pgfqpoint{0.845841in}{1.473067in}}%
\pgfusepath{stroke}%
\end{pgfscope}%
\begin{pgfscope}%
\pgfpathrectangle{\pgfqpoint{0.100000in}{0.212622in}}{\pgfqpoint{3.696000in}{3.696000in}}%
\pgfusepath{clip}%
\pgfsetrectcap%
\pgfsetroundjoin%
\pgfsetlinewidth{1.505625pt}%
\definecolor{currentstroke}{rgb}{1.000000,0.000000,0.000000}%
\pgfsetstrokecolor{currentstroke}%
\pgfsetdash{}{0pt}%
\pgfpathmoveto{\pgfqpoint{0.844202in}{1.480791in}}%
\pgfpathlineto{\pgfqpoint{0.856401in}{1.481949in}}%
\pgfusepath{stroke}%
\end{pgfscope}%
\begin{pgfscope}%
\pgfpathrectangle{\pgfqpoint{0.100000in}{0.212622in}}{\pgfqpoint{3.696000in}{3.696000in}}%
\pgfusepath{clip}%
\pgfsetrectcap%
\pgfsetroundjoin%
\pgfsetlinewidth{1.505625pt}%
\definecolor{currentstroke}{rgb}{1.000000,0.000000,0.000000}%
\pgfsetstrokecolor{currentstroke}%
\pgfsetdash{}{0pt}%
\pgfpathmoveto{\pgfqpoint{0.852975in}{1.485215in}}%
\pgfpathlineto{\pgfqpoint{0.866946in}{1.490818in}}%
\pgfusepath{stroke}%
\end{pgfscope}%
\begin{pgfscope}%
\pgfpathrectangle{\pgfqpoint{0.100000in}{0.212622in}}{\pgfqpoint{3.696000in}{3.696000in}}%
\pgfusepath{clip}%
\pgfsetrectcap%
\pgfsetroundjoin%
\pgfsetlinewidth{1.505625pt}%
\definecolor{currentstroke}{rgb}{1.000000,0.000000,0.000000}%
\pgfsetstrokecolor{currentstroke}%
\pgfsetdash{}{0pt}%
\pgfpathmoveto{\pgfqpoint{0.860926in}{1.489531in}}%
\pgfpathlineto{\pgfqpoint{0.866946in}{1.490818in}}%
\pgfusepath{stroke}%
\end{pgfscope}%
\begin{pgfscope}%
\pgfpathrectangle{\pgfqpoint{0.100000in}{0.212622in}}{\pgfqpoint{3.696000in}{3.696000in}}%
\pgfusepath{clip}%
\pgfsetrectcap%
\pgfsetroundjoin%
\pgfsetlinewidth{1.505625pt}%
\definecolor{currentstroke}{rgb}{1.000000,0.000000,0.000000}%
\pgfsetstrokecolor{currentstroke}%
\pgfsetdash{}{0pt}%
\pgfpathmoveto{\pgfqpoint{0.865137in}{1.492236in}}%
\pgfpathlineto{\pgfqpoint{0.877475in}{1.499675in}}%
\pgfusepath{stroke}%
\end{pgfscope}%
\begin{pgfscope}%
\pgfpathrectangle{\pgfqpoint{0.100000in}{0.212622in}}{\pgfqpoint{3.696000in}{3.696000in}}%
\pgfusepath{clip}%
\pgfsetrectcap%
\pgfsetroundjoin%
\pgfsetlinewidth{1.505625pt}%
\definecolor{currentstroke}{rgb}{1.000000,0.000000,0.000000}%
\pgfsetstrokecolor{currentstroke}%
\pgfsetdash{}{0pt}%
\pgfpathmoveto{\pgfqpoint{0.870562in}{1.497585in}}%
\pgfpathlineto{\pgfqpoint{0.877475in}{1.499675in}}%
\pgfusepath{stroke}%
\end{pgfscope}%
\begin{pgfscope}%
\pgfpathrectangle{\pgfqpoint{0.100000in}{0.212622in}}{\pgfqpoint{3.696000in}{3.696000in}}%
\pgfusepath{clip}%
\pgfsetrectcap%
\pgfsetroundjoin%
\pgfsetlinewidth{1.505625pt}%
\definecolor{currentstroke}{rgb}{1.000000,0.000000,0.000000}%
\pgfsetstrokecolor{currentstroke}%
\pgfsetdash{}{0pt}%
\pgfpathmoveto{\pgfqpoint{0.875815in}{1.499473in}}%
\pgfpathlineto{\pgfqpoint{0.887989in}{1.508518in}}%
\pgfusepath{stroke}%
\end{pgfscope}%
\begin{pgfscope}%
\pgfpathrectangle{\pgfqpoint{0.100000in}{0.212622in}}{\pgfqpoint{3.696000in}{3.696000in}}%
\pgfusepath{clip}%
\pgfsetrectcap%
\pgfsetroundjoin%
\pgfsetlinewidth{1.505625pt}%
\definecolor{currentstroke}{rgb}{1.000000,0.000000,0.000000}%
\pgfsetstrokecolor{currentstroke}%
\pgfsetdash{}{0pt}%
\pgfpathmoveto{\pgfqpoint{0.882708in}{1.506092in}}%
\pgfpathlineto{\pgfqpoint{0.887989in}{1.508518in}}%
\pgfusepath{stroke}%
\end{pgfscope}%
\begin{pgfscope}%
\pgfpathrectangle{\pgfqpoint{0.100000in}{0.212622in}}{\pgfqpoint{3.696000in}{3.696000in}}%
\pgfusepath{clip}%
\pgfsetrectcap%
\pgfsetroundjoin%
\pgfsetlinewidth{1.505625pt}%
\definecolor{currentstroke}{rgb}{1.000000,0.000000,0.000000}%
\pgfsetstrokecolor{currentstroke}%
\pgfsetdash{}{0pt}%
\pgfpathmoveto{\pgfqpoint{0.889663in}{1.517028in}}%
\pgfpathlineto{\pgfqpoint{0.898488in}{1.517349in}}%
\pgfusepath{stroke}%
\end{pgfscope}%
\begin{pgfscope}%
\pgfpathrectangle{\pgfqpoint{0.100000in}{0.212622in}}{\pgfqpoint{3.696000in}{3.696000in}}%
\pgfusepath{clip}%
\pgfsetrectcap%
\pgfsetroundjoin%
\pgfsetlinewidth{1.505625pt}%
\definecolor{currentstroke}{rgb}{1.000000,0.000000,0.000000}%
\pgfsetstrokecolor{currentstroke}%
\pgfsetdash{}{0pt}%
\pgfpathmoveto{\pgfqpoint{0.896393in}{1.519683in}}%
\pgfpathlineto{\pgfqpoint{0.898488in}{1.517349in}}%
\pgfusepath{stroke}%
\end{pgfscope}%
\begin{pgfscope}%
\pgfpathrectangle{\pgfqpoint{0.100000in}{0.212622in}}{\pgfqpoint{3.696000in}{3.696000in}}%
\pgfusepath{clip}%
\pgfsetrectcap%
\pgfsetroundjoin%
\pgfsetlinewidth{1.505625pt}%
\definecolor{currentstroke}{rgb}{1.000000,0.000000,0.000000}%
\pgfsetstrokecolor{currentstroke}%
\pgfsetdash{}{0pt}%
\pgfpathmoveto{\pgfqpoint{0.901096in}{1.525988in}}%
\pgfpathlineto{\pgfqpoint{0.908971in}{1.526167in}}%
\pgfusepath{stroke}%
\end{pgfscope}%
\begin{pgfscope}%
\pgfpathrectangle{\pgfqpoint{0.100000in}{0.212622in}}{\pgfqpoint{3.696000in}{3.696000in}}%
\pgfusepath{clip}%
\pgfsetrectcap%
\pgfsetroundjoin%
\pgfsetlinewidth{1.505625pt}%
\definecolor{currentstroke}{rgb}{1.000000,0.000000,0.000000}%
\pgfsetstrokecolor{currentstroke}%
\pgfsetdash{}{0pt}%
\pgfpathmoveto{\pgfqpoint{0.903962in}{1.529424in}}%
\pgfpathlineto{\pgfqpoint{0.908971in}{1.526167in}}%
\pgfusepath{stroke}%
\end{pgfscope}%
\begin{pgfscope}%
\pgfpathrectangle{\pgfqpoint{0.100000in}{0.212622in}}{\pgfqpoint{3.696000in}{3.696000in}}%
\pgfusepath{clip}%
\pgfsetrectcap%
\pgfsetroundjoin%
\pgfsetlinewidth{1.505625pt}%
\definecolor{currentstroke}{rgb}{1.000000,0.000000,0.000000}%
\pgfsetstrokecolor{currentstroke}%
\pgfsetdash{}{0pt}%
\pgfpathmoveto{\pgfqpoint{0.907266in}{1.529987in}}%
\pgfpathlineto{\pgfqpoint{0.908971in}{1.526167in}}%
\pgfusepath{stroke}%
\end{pgfscope}%
\begin{pgfscope}%
\pgfpathrectangle{\pgfqpoint{0.100000in}{0.212622in}}{\pgfqpoint{3.696000in}{3.696000in}}%
\pgfusepath{clip}%
\pgfsetrectcap%
\pgfsetroundjoin%
\pgfsetlinewidth{1.505625pt}%
\definecolor{currentstroke}{rgb}{1.000000,0.000000,0.000000}%
\pgfsetstrokecolor{currentstroke}%
\pgfsetdash{}{0pt}%
\pgfpathmoveto{\pgfqpoint{0.912380in}{1.541420in}}%
\pgfpathlineto{\pgfqpoint{0.919439in}{1.534972in}}%
\pgfusepath{stroke}%
\end{pgfscope}%
\begin{pgfscope}%
\pgfpathrectangle{\pgfqpoint{0.100000in}{0.212622in}}{\pgfqpoint{3.696000in}{3.696000in}}%
\pgfusepath{clip}%
\pgfsetrectcap%
\pgfsetroundjoin%
\pgfsetlinewidth{1.505625pt}%
\definecolor{currentstroke}{rgb}{1.000000,0.000000,0.000000}%
\pgfsetstrokecolor{currentstroke}%
\pgfsetdash{}{0pt}%
\pgfpathmoveto{\pgfqpoint{0.918328in}{1.546469in}}%
\pgfpathlineto{\pgfqpoint{0.919439in}{1.534972in}}%
\pgfusepath{stroke}%
\end{pgfscope}%
\begin{pgfscope}%
\pgfpathrectangle{\pgfqpoint{0.100000in}{0.212622in}}{\pgfqpoint{3.696000in}{3.696000in}}%
\pgfusepath{clip}%
\pgfsetrectcap%
\pgfsetroundjoin%
\pgfsetlinewidth{1.505625pt}%
\definecolor{currentstroke}{rgb}{1.000000,0.000000,0.000000}%
\pgfsetstrokecolor{currentstroke}%
\pgfsetdash{}{0pt}%
\pgfpathmoveto{\pgfqpoint{0.923145in}{1.544180in}}%
\pgfpathlineto{\pgfqpoint{0.929892in}{1.543764in}}%
\pgfusepath{stroke}%
\end{pgfscope}%
\begin{pgfscope}%
\pgfpathrectangle{\pgfqpoint{0.100000in}{0.212622in}}{\pgfqpoint{3.696000in}{3.696000in}}%
\pgfusepath{clip}%
\pgfsetrectcap%
\pgfsetroundjoin%
\pgfsetlinewidth{1.505625pt}%
\definecolor{currentstroke}{rgb}{1.000000,0.000000,0.000000}%
\pgfsetstrokecolor{currentstroke}%
\pgfsetdash{}{0pt}%
\pgfpathmoveto{\pgfqpoint{0.929570in}{1.550803in}}%
\pgfpathlineto{\pgfqpoint{0.929892in}{1.543764in}}%
\pgfusepath{stroke}%
\end{pgfscope}%
\begin{pgfscope}%
\pgfpathrectangle{\pgfqpoint{0.100000in}{0.212622in}}{\pgfqpoint{3.696000in}{3.696000in}}%
\pgfusepath{clip}%
\pgfsetrectcap%
\pgfsetroundjoin%
\pgfsetlinewidth{1.505625pt}%
\definecolor{currentstroke}{rgb}{1.000000,0.000000,0.000000}%
\pgfsetstrokecolor{currentstroke}%
\pgfsetdash{}{0pt}%
\pgfpathmoveto{\pgfqpoint{0.933661in}{1.554447in}}%
\pgfpathlineto{\pgfqpoint{0.929892in}{1.543764in}}%
\pgfusepath{stroke}%
\end{pgfscope}%
\begin{pgfscope}%
\pgfpathrectangle{\pgfqpoint{0.100000in}{0.212622in}}{\pgfqpoint{3.696000in}{3.696000in}}%
\pgfusepath{clip}%
\pgfsetrectcap%
\pgfsetroundjoin%
\pgfsetlinewidth{1.505625pt}%
\definecolor{currentstroke}{rgb}{1.000000,0.000000,0.000000}%
\pgfsetstrokecolor{currentstroke}%
\pgfsetdash{}{0pt}%
\pgfpathmoveto{\pgfqpoint{0.936373in}{1.550637in}}%
\pgfpathlineto{\pgfqpoint{0.940330in}{1.552543in}}%
\pgfusepath{stroke}%
\end{pgfscope}%
\begin{pgfscope}%
\pgfpathrectangle{\pgfqpoint{0.100000in}{0.212622in}}{\pgfqpoint{3.696000in}{3.696000in}}%
\pgfusepath{clip}%
\pgfsetrectcap%
\pgfsetroundjoin%
\pgfsetlinewidth{1.505625pt}%
\definecolor{currentstroke}{rgb}{1.000000,0.000000,0.000000}%
\pgfsetstrokecolor{currentstroke}%
\pgfsetdash{}{0pt}%
\pgfpathmoveto{\pgfqpoint{0.941856in}{1.553193in}}%
\pgfpathlineto{\pgfqpoint{0.940330in}{1.552543in}}%
\pgfusepath{stroke}%
\end{pgfscope}%
\begin{pgfscope}%
\pgfpathrectangle{\pgfqpoint{0.100000in}{0.212622in}}{\pgfqpoint{3.696000in}{3.696000in}}%
\pgfusepath{clip}%
\pgfsetrectcap%
\pgfsetroundjoin%
\pgfsetlinewidth{1.505625pt}%
\definecolor{currentstroke}{rgb}{1.000000,0.000000,0.000000}%
\pgfsetstrokecolor{currentstroke}%
\pgfsetdash{}{0pt}%
\pgfpathmoveto{\pgfqpoint{0.947016in}{1.557737in}}%
\pgfpathlineto{\pgfqpoint{0.950752in}{1.561310in}}%
\pgfusepath{stroke}%
\end{pgfscope}%
\begin{pgfscope}%
\pgfpathrectangle{\pgfqpoint{0.100000in}{0.212622in}}{\pgfqpoint{3.696000in}{3.696000in}}%
\pgfusepath{clip}%
\pgfsetrectcap%
\pgfsetroundjoin%
\pgfsetlinewidth{1.505625pt}%
\definecolor{currentstroke}{rgb}{1.000000,0.000000,0.000000}%
\pgfsetstrokecolor{currentstroke}%
\pgfsetdash{}{0pt}%
\pgfpathmoveto{\pgfqpoint{0.949926in}{1.560702in}}%
\pgfpathlineto{\pgfqpoint{0.950752in}{1.561310in}}%
\pgfusepath{stroke}%
\end{pgfscope}%
\begin{pgfscope}%
\pgfpathrectangle{\pgfqpoint{0.100000in}{0.212622in}}{\pgfqpoint{3.696000in}{3.696000in}}%
\pgfusepath{clip}%
\pgfsetrectcap%
\pgfsetroundjoin%
\pgfsetlinewidth{1.505625pt}%
\definecolor{currentstroke}{rgb}{1.000000,0.000000,0.000000}%
\pgfsetstrokecolor{currentstroke}%
\pgfsetdash{}{0pt}%
\pgfpathmoveto{\pgfqpoint{0.955894in}{1.561910in}}%
\pgfpathlineto{\pgfqpoint{0.950752in}{1.561310in}}%
\pgfusepath{stroke}%
\end{pgfscope}%
\begin{pgfscope}%
\pgfpathrectangle{\pgfqpoint{0.100000in}{0.212622in}}{\pgfqpoint{3.696000in}{3.696000in}}%
\pgfusepath{clip}%
\pgfsetrectcap%
\pgfsetroundjoin%
\pgfsetlinewidth{1.505625pt}%
\definecolor{currentstroke}{rgb}{1.000000,0.000000,0.000000}%
\pgfsetstrokecolor{currentstroke}%
\pgfsetdash{}{0pt}%
\pgfpathmoveto{\pgfqpoint{0.961776in}{1.561157in}}%
\pgfpathlineto{\pgfqpoint{0.961160in}{1.570064in}}%
\pgfusepath{stroke}%
\end{pgfscope}%
\begin{pgfscope}%
\pgfpathrectangle{\pgfqpoint{0.100000in}{0.212622in}}{\pgfqpoint{3.696000in}{3.696000in}}%
\pgfusepath{clip}%
\pgfsetrectcap%
\pgfsetroundjoin%
\pgfsetlinewidth{1.505625pt}%
\definecolor{currentstroke}{rgb}{1.000000,0.000000,0.000000}%
\pgfsetstrokecolor{currentstroke}%
\pgfsetdash{}{0pt}%
\pgfpathmoveto{\pgfqpoint{0.969951in}{1.577161in}}%
\pgfpathlineto{\pgfqpoint{0.961160in}{1.570064in}}%
\pgfusepath{stroke}%
\end{pgfscope}%
\begin{pgfscope}%
\pgfpathrectangle{\pgfqpoint{0.100000in}{0.212622in}}{\pgfqpoint{3.696000in}{3.696000in}}%
\pgfusepath{clip}%
\pgfsetrectcap%
\pgfsetroundjoin%
\pgfsetlinewidth{1.505625pt}%
\definecolor{currentstroke}{rgb}{1.000000,0.000000,0.000000}%
\pgfsetstrokecolor{currentstroke}%
\pgfsetdash{}{0pt}%
\pgfpathmoveto{\pgfqpoint{0.974880in}{1.580342in}}%
\pgfpathlineto{\pgfqpoint{0.971552in}{1.578806in}}%
\pgfusepath{stroke}%
\end{pgfscope}%
\begin{pgfscope}%
\pgfpathrectangle{\pgfqpoint{0.100000in}{0.212622in}}{\pgfqpoint{3.696000in}{3.696000in}}%
\pgfusepath{clip}%
\pgfsetrectcap%
\pgfsetroundjoin%
\pgfsetlinewidth{1.505625pt}%
\definecolor{currentstroke}{rgb}{1.000000,0.000000,0.000000}%
\pgfsetstrokecolor{currentstroke}%
\pgfsetdash{}{0pt}%
\pgfpathmoveto{\pgfqpoint{0.976616in}{1.578977in}}%
\pgfpathlineto{\pgfqpoint{0.971552in}{1.578806in}}%
\pgfusepath{stroke}%
\end{pgfscope}%
\begin{pgfscope}%
\pgfpathrectangle{\pgfqpoint{0.100000in}{0.212622in}}{\pgfqpoint{3.696000in}{3.696000in}}%
\pgfusepath{clip}%
\pgfsetrectcap%
\pgfsetroundjoin%
\pgfsetlinewidth{1.505625pt}%
\definecolor{currentstroke}{rgb}{1.000000,0.000000,0.000000}%
\pgfsetstrokecolor{currentstroke}%
\pgfsetdash{}{0pt}%
\pgfpathmoveto{\pgfqpoint{0.979723in}{1.582148in}}%
\pgfpathlineto{\pgfqpoint{0.971552in}{1.578806in}}%
\pgfusepath{stroke}%
\end{pgfscope}%
\begin{pgfscope}%
\pgfpathrectangle{\pgfqpoint{0.100000in}{0.212622in}}{\pgfqpoint{3.696000in}{3.696000in}}%
\pgfusepath{clip}%
\pgfsetrectcap%
\pgfsetroundjoin%
\pgfsetlinewidth{1.505625pt}%
\definecolor{currentstroke}{rgb}{1.000000,0.000000,0.000000}%
\pgfsetstrokecolor{currentstroke}%
\pgfsetdash{}{0pt}%
\pgfpathmoveto{\pgfqpoint{0.981704in}{1.583797in}}%
\pgfpathlineto{\pgfqpoint{0.971552in}{1.578806in}}%
\pgfusepath{stroke}%
\end{pgfscope}%
\begin{pgfscope}%
\pgfpathrectangle{\pgfqpoint{0.100000in}{0.212622in}}{\pgfqpoint{3.696000in}{3.696000in}}%
\pgfusepath{clip}%
\pgfsetrectcap%
\pgfsetroundjoin%
\pgfsetlinewidth{1.505625pt}%
\definecolor{currentstroke}{rgb}{1.000000,0.000000,0.000000}%
\pgfsetstrokecolor{currentstroke}%
\pgfsetdash{}{0pt}%
\pgfpathmoveto{\pgfqpoint{0.982606in}{1.583474in}}%
\pgfpathlineto{\pgfqpoint{0.971552in}{1.578806in}}%
\pgfusepath{stroke}%
\end{pgfscope}%
\begin{pgfscope}%
\pgfpathrectangle{\pgfqpoint{0.100000in}{0.212622in}}{\pgfqpoint{3.696000in}{3.696000in}}%
\pgfusepath{clip}%
\pgfsetrectcap%
\pgfsetroundjoin%
\pgfsetlinewidth{1.505625pt}%
\definecolor{currentstroke}{rgb}{1.000000,0.000000,0.000000}%
\pgfsetstrokecolor{currentstroke}%
\pgfsetdash{}{0pt}%
\pgfpathmoveto{\pgfqpoint{0.985089in}{1.588025in}}%
\pgfpathlineto{\pgfqpoint{0.971552in}{1.578806in}}%
\pgfusepath{stroke}%
\end{pgfscope}%
\begin{pgfscope}%
\pgfpathrectangle{\pgfqpoint{0.100000in}{0.212622in}}{\pgfqpoint{3.696000in}{3.696000in}}%
\pgfusepath{clip}%
\pgfsetrectcap%
\pgfsetroundjoin%
\pgfsetlinewidth{1.505625pt}%
\definecolor{currentstroke}{rgb}{1.000000,0.000000,0.000000}%
\pgfsetstrokecolor{currentstroke}%
\pgfsetdash{}{0pt}%
\pgfpathmoveto{\pgfqpoint{0.986508in}{1.589537in}}%
\pgfpathlineto{\pgfqpoint{0.981929in}{1.587534in}}%
\pgfusepath{stroke}%
\end{pgfscope}%
\begin{pgfscope}%
\pgfpathrectangle{\pgfqpoint{0.100000in}{0.212622in}}{\pgfqpoint{3.696000in}{3.696000in}}%
\pgfusepath{clip}%
\pgfsetrectcap%
\pgfsetroundjoin%
\pgfsetlinewidth{1.505625pt}%
\definecolor{currentstroke}{rgb}{1.000000,0.000000,0.000000}%
\pgfsetstrokecolor{currentstroke}%
\pgfsetdash{}{0pt}%
\pgfpathmoveto{\pgfqpoint{0.988107in}{1.589794in}}%
\pgfpathlineto{\pgfqpoint{0.981929in}{1.587534in}}%
\pgfusepath{stroke}%
\end{pgfscope}%
\begin{pgfscope}%
\pgfpathrectangle{\pgfqpoint{0.100000in}{0.212622in}}{\pgfqpoint{3.696000in}{3.696000in}}%
\pgfusepath{clip}%
\pgfsetrectcap%
\pgfsetroundjoin%
\pgfsetlinewidth{1.505625pt}%
\definecolor{currentstroke}{rgb}{1.000000,0.000000,0.000000}%
\pgfsetstrokecolor{currentstroke}%
\pgfsetdash{}{0pt}%
\pgfpathmoveto{\pgfqpoint{0.991945in}{1.596629in}}%
\pgfpathlineto{\pgfqpoint{0.981929in}{1.587534in}}%
\pgfusepath{stroke}%
\end{pgfscope}%
\begin{pgfscope}%
\pgfpathrectangle{\pgfqpoint{0.100000in}{0.212622in}}{\pgfqpoint{3.696000in}{3.696000in}}%
\pgfusepath{clip}%
\pgfsetrectcap%
\pgfsetroundjoin%
\pgfsetlinewidth{1.505625pt}%
\definecolor{currentstroke}{rgb}{1.000000,0.000000,0.000000}%
\pgfsetstrokecolor{currentstroke}%
\pgfsetdash{}{0pt}%
\pgfpathmoveto{\pgfqpoint{0.993865in}{1.599671in}}%
\pgfpathlineto{\pgfqpoint{0.981929in}{1.587534in}}%
\pgfusepath{stroke}%
\end{pgfscope}%
\begin{pgfscope}%
\pgfpathrectangle{\pgfqpoint{0.100000in}{0.212622in}}{\pgfqpoint{3.696000in}{3.696000in}}%
\pgfusepath{clip}%
\pgfsetrectcap%
\pgfsetroundjoin%
\pgfsetlinewidth{1.505625pt}%
\definecolor{currentstroke}{rgb}{1.000000,0.000000,0.000000}%
\pgfsetstrokecolor{currentstroke}%
\pgfsetdash{}{0pt}%
\pgfpathmoveto{\pgfqpoint{0.996389in}{1.601037in}}%
\pgfpathlineto{\pgfqpoint{0.981929in}{1.587534in}}%
\pgfusepath{stroke}%
\end{pgfscope}%
\begin{pgfscope}%
\pgfpathrectangle{\pgfqpoint{0.100000in}{0.212622in}}{\pgfqpoint{3.696000in}{3.696000in}}%
\pgfusepath{clip}%
\pgfsetrectcap%
\pgfsetroundjoin%
\pgfsetlinewidth{1.505625pt}%
\definecolor{currentstroke}{rgb}{1.000000,0.000000,0.000000}%
\pgfsetstrokecolor{currentstroke}%
\pgfsetdash{}{0pt}%
\pgfpathmoveto{\pgfqpoint{0.999446in}{1.603602in}}%
\pgfpathlineto{\pgfqpoint{0.981929in}{1.587534in}}%
\pgfusepath{stroke}%
\end{pgfscope}%
\begin{pgfscope}%
\pgfpathrectangle{\pgfqpoint{0.100000in}{0.212622in}}{\pgfqpoint{3.696000in}{3.696000in}}%
\pgfusepath{clip}%
\pgfsetrectcap%
\pgfsetroundjoin%
\pgfsetlinewidth{1.505625pt}%
\definecolor{currentstroke}{rgb}{1.000000,0.000000,0.000000}%
\pgfsetstrokecolor{currentstroke}%
\pgfsetdash{}{0pt}%
\pgfpathmoveto{\pgfqpoint{1.004414in}{1.608282in}}%
\pgfpathlineto{\pgfqpoint{0.992292in}{1.596250in}}%
\pgfusepath{stroke}%
\end{pgfscope}%
\begin{pgfscope}%
\pgfpathrectangle{\pgfqpoint{0.100000in}{0.212622in}}{\pgfqpoint{3.696000in}{3.696000in}}%
\pgfusepath{clip}%
\pgfsetrectcap%
\pgfsetroundjoin%
\pgfsetlinewidth{1.505625pt}%
\definecolor{currentstroke}{rgb}{1.000000,0.000000,0.000000}%
\pgfsetstrokecolor{currentstroke}%
\pgfsetdash{}{0pt}%
\pgfpathmoveto{\pgfqpoint{1.010095in}{1.608667in}}%
\pgfpathlineto{\pgfqpoint{0.992292in}{1.596250in}}%
\pgfusepath{stroke}%
\end{pgfscope}%
\begin{pgfscope}%
\pgfpathrectangle{\pgfqpoint{0.100000in}{0.212622in}}{\pgfqpoint{3.696000in}{3.696000in}}%
\pgfusepath{clip}%
\pgfsetrectcap%
\pgfsetroundjoin%
\pgfsetlinewidth{1.505625pt}%
\definecolor{currentstroke}{rgb}{1.000000,0.000000,0.000000}%
\pgfsetstrokecolor{currentstroke}%
\pgfsetdash{}{0pt}%
\pgfpathmoveto{\pgfqpoint{1.016974in}{1.612174in}}%
\pgfpathlineto{\pgfqpoint{1.002639in}{1.604954in}}%
\pgfusepath{stroke}%
\end{pgfscope}%
\begin{pgfscope}%
\pgfpathrectangle{\pgfqpoint{0.100000in}{0.212622in}}{\pgfqpoint{3.696000in}{3.696000in}}%
\pgfusepath{clip}%
\pgfsetrectcap%
\pgfsetroundjoin%
\pgfsetlinewidth{1.505625pt}%
\definecolor{currentstroke}{rgb}{1.000000,0.000000,0.000000}%
\pgfsetstrokecolor{currentstroke}%
\pgfsetdash{}{0pt}%
\pgfpathmoveto{\pgfqpoint{1.025981in}{1.616195in}}%
\pgfpathlineto{\pgfqpoint{1.012971in}{1.613645in}}%
\pgfusepath{stroke}%
\end{pgfscope}%
\begin{pgfscope}%
\pgfpathrectangle{\pgfqpoint{0.100000in}{0.212622in}}{\pgfqpoint{3.696000in}{3.696000in}}%
\pgfusepath{clip}%
\pgfsetrectcap%
\pgfsetroundjoin%
\pgfsetlinewidth{1.505625pt}%
\definecolor{currentstroke}{rgb}{1.000000,0.000000,0.000000}%
\pgfsetstrokecolor{currentstroke}%
\pgfsetdash{}{0pt}%
\pgfpathmoveto{\pgfqpoint{1.034671in}{1.616703in}}%
\pgfpathlineto{\pgfqpoint{1.012971in}{1.613645in}}%
\pgfusepath{stroke}%
\end{pgfscope}%
\begin{pgfscope}%
\pgfpathrectangle{\pgfqpoint{0.100000in}{0.212622in}}{\pgfqpoint{3.696000in}{3.696000in}}%
\pgfusepath{clip}%
\pgfsetrectcap%
\pgfsetroundjoin%
\pgfsetlinewidth{1.505625pt}%
\definecolor{currentstroke}{rgb}{1.000000,0.000000,0.000000}%
\pgfsetstrokecolor{currentstroke}%
\pgfsetdash{}{0pt}%
\pgfpathmoveto{\pgfqpoint{1.040740in}{1.618724in}}%
\pgfpathlineto{\pgfqpoint{1.023289in}{1.622323in}}%
\pgfusepath{stroke}%
\end{pgfscope}%
\begin{pgfscope}%
\pgfpathrectangle{\pgfqpoint{0.100000in}{0.212622in}}{\pgfqpoint{3.696000in}{3.696000in}}%
\pgfusepath{clip}%
\pgfsetrectcap%
\pgfsetroundjoin%
\pgfsetlinewidth{1.505625pt}%
\definecolor{currentstroke}{rgb}{1.000000,0.000000,0.000000}%
\pgfsetstrokecolor{currentstroke}%
\pgfsetdash{}{0pt}%
\pgfpathmoveto{\pgfqpoint{1.048328in}{1.624325in}}%
\pgfpathlineto{\pgfqpoint{1.023289in}{1.622323in}}%
\pgfusepath{stroke}%
\end{pgfscope}%
\begin{pgfscope}%
\pgfpathrectangle{\pgfqpoint{0.100000in}{0.212622in}}{\pgfqpoint{3.696000in}{3.696000in}}%
\pgfusepath{clip}%
\pgfsetrectcap%
\pgfsetroundjoin%
\pgfsetlinewidth{1.505625pt}%
\definecolor{currentstroke}{rgb}{1.000000,0.000000,0.000000}%
\pgfsetstrokecolor{currentstroke}%
\pgfsetdash{}{0pt}%
\pgfpathmoveto{\pgfqpoint{1.055553in}{1.626766in}}%
\pgfpathlineto{\pgfqpoint{1.033592in}{1.630989in}}%
\pgfusepath{stroke}%
\end{pgfscope}%
\begin{pgfscope}%
\pgfpathrectangle{\pgfqpoint{0.100000in}{0.212622in}}{\pgfqpoint{3.696000in}{3.696000in}}%
\pgfusepath{clip}%
\pgfsetrectcap%
\pgfsetroundjoin%
\pgfsetlinewidth{1.505625pt}%
\definecolor{currentstroke}{rgb}{1.000000,0.000000,0.000000}%
\pgfsetstrokecolor{currentstroke}%
\pgfsetdash{}{0pt}%
\pgfpathmoveto{\pgfqpoint{1.059972in}{1.627229in}}%
\pgfpathlineto{\pgfqpoint{1.033592in}{1.630989in}}%
\pgfusepath{stroke}%
\end{pgfscope}%
\begin{pgfscope}%
\pgfpathrectangle{\pgfqpoint{0.100000in}{0.212622in}}{\pgfqpoint{3.696000in}{3.696000in}}%
\pgfusepath{clip}%
\pgfsetrectcap%
\pgfsetroundjoin%
\pgfsetlinewidth{1.505625pt}%
\definecolor{currentstroke}{rgb}{1.000000,0.000000,0.000000}%
\pgfsetstrokecolor{currentstroke}%
\pgfsetdash{}{0pt}%
\pgfpathmoveto{\pgfqpoint{1.062524in}{1.629155in}}%
\pgfpathlineto{\pgfqpoint{1.043879in}{1.639643in}}%
\pgfusepath{stroke}%
\end{pgfscope}%
\begin{pgfscope}%
\pgfpathrectangle{\pgfqpoint{0.100000in}{0.212622in}}{\pgfqpoint{3.696000in}{3.696000in}}%
\pgfusepath{clip}%
\pgfsetrectcap%
\pgfsetroundjoin%
\pgfsetlinewidth{1.505625pt}%
\definecolor{currentstroke}{rgb}{1.000000,0.000000,0.000000}%
\pgfsetstrokecolor{currentstroke}%
\pgfsetdash{}{0pt}%
\pgfpathmoveto{\pgfqpoint{1.066123in}{1.629310in}}%
\pgfpathlineto{\pgfqpoint{1.043879in}{1.639643in}}%
\pgfusepath{stroke}%
\end{pgfscope}%
\begin{pgfscope}%
\pgfpathrectangle{\pgfqpoint{0.100000in}{0.212622in}}{\pgfqpoint{3.696000in}{3.696000in}}%
\pgfusepath{clip}%
\pgfsetrectcap%
\pgfsetroundjoin%
\pgfsetlinewidth{1.505625pt}%
\definecolor{currentstroke}{rgb}{1.000000,0.000000,0.000000}%
\pgfsetstrokecolor{currentstroke}%
\pgfsetdash{}{0pt}%
\pgfpathmoveto{\pgfqpoint{1.068174in}{1.631213in}}%
\pgfpathlineto{\pgfqpoint{1.043879in}{1.639643in}}%
\pgfusepath{stroke}%
\end{pgfscope}%
\begin{pgfscope}%
\pgfpathrectangle{\pgfqpoint{0.100000in}{0.212622in}}{\pgfqpoint{3.696000in}{3.696000in}}%
\pgfusepath{clip}%
\pgfsetrectcap%
\pgfsetroundjoin%
\pgfsetlinewidth{1.505625pt}%
\definecolor{currentstroke}{rgb}{1.000000,0.000000,0.000000}%
\pgfsetstrokecolor{currentstroke}%
\pgfsetdash{}{0pt}%
\pgfpathmoveto{\pgfqpoint{1.072239in}{1.633230in}}%
\pgfpathlineto{\pgfqpoint{1.043879in}{1.639643in}}%
\pgfusepath{stroke}%
\end{pgfscope}%
\begin{pgfscope}%
\pgfpathrectangle{\pgfqpoint{0.100000in}{0.212622in}}{\pgfqpoint{3.696000in}{3.696000in}}%
\pgfusepath{clip}%
\pgfsetrectcap%
\pgfsetroundjoin%
\pgfsetlinewidth{1.505625pt}%
\definecolor{currentstroke}{rgb}{1.000000,0.000000,0.000000}%
\pgfsetstrokecolor{currentstroke}%
\pgfsetdash{}{0pt}%
\pgfpathmoveto{\pgfqpoint{1.076611in}{1.632624in}}%
\pgfpathlineto{\pgfqpoint{1.054152in}{1.648283in}}%
\pgfusepath{stroke}%
\end{pgfscope}%
\begin{pgfscope}%
\pgfpathrectangle{\pgfqpoint{0.100000in}{0.212622in}}{\pgfqpoint{3.696000in}{3.696000in}}%
\pgfusepath{clip}%
\pgfsetrectcap%
\pgfsetroundjoin%
\pgfsetlinewidth{1.505625pt}%
\definecolor{currentstroke}{rgb}{1.000000,0.000000,0.000000}%
\pgfsetstrokecolor{currentstroke}%
\pgfsetdash{}{0pt}%
\pgfpathmoveto{\pgfqpoint{1.079381in}{1.635824in}}%
\pgfpathlineto{\pgfqpoint{1.054152in}{1.648283in}}%
\pgfusepath{stroke}%
\end{pgfscope}%
\begin{pgfscope}%
\pgfpathrectangle{\pgfqpoint{0.100000in}{0.212622in}}{\pgfqpoint{3.696000in}{3.696000in}}%
\pgfusepath{clip}%
\pgfsetrectcap%
\pgfsetroundjoin%
\pgfsetlinewidth{1.505625pt}%
\definecolor{currentstroke}{rgb}{1.000000,0.000000,0.000000}%
\pgfsetstrokecolor{currentstroke}%
\pgfsetdash{}{0pt}%
\pgfpathmoveto{\pgfqpoint{1.083712in}{1.636809in}}%
\pgfpathlineto{\pgfqpoint{1.054152in}{1.648283in}}%
\pgfusepath{stroke}%
\end{pgfscope}%
\begin{pgfscope}%
\pgfpathrectangle{\pgfqpoint{0.100000in}{0.212622in}}{\pgfqpoint{3.696000in}{3.696000in}}%
\pgfusepath{clip}%
\pgfsetrectcap%
\pgfsetroundjoin%
\pgfsetlinewidth{1.505625pt}%
\definecolor{currentstroke}{rgb}{1.000000,0.000000,0.000000}%
\pgfsetstrokecolor{currentstroke}%
\pgfsetdash{}{0pt}%
\pgfpathmoveto{\pgfqpoint{1.088828in}{1.637270in}}%
\pgfpathlineto{\pgfqpoint{1.064411in}{1.656912in}}%
\pgfusepath{stroke}%
\end{pgfscope}%
\begin{pgfscope}%
\pgfpathrectangle{\pgfqpoint{0.100000in}{0.212622in}}{\pgfqpoint{3.696000in}{3.696000in}}%
\pgfusepath{clip}%
\pgfsetrectcap%
\pgfsetroundjoin%
\pgfsetlinewidth{1.505625pt}%
\definecolor{currentstroke}{rgb}{1.000000,0.000000,0.000000}%
\pgfsetstrokecolor{currentstroke}%
\pgfsetdash{}{0pt}%
\pgfpathmoveto{\pgfqpoint{1.092230in}{1.639301in}}%
\pgfpathlineto{\pgfqpoint{1.064411in}{1.656912in}}%
\pgfusepath{stroke}%
\end{pgfscope}%
\begin{pgfscope}%
\pgfpathrectangle{\pgfqpoint{0.100000in}{0.212622in}}{\pgfqpoint{3.696000in}{3.696000in}}%
\pgfusepath{clip}%
\pgfsetrectcap%
\pgfsetroundjoin%
\pgfsetlinewidth{1.505625pt}%
\definecolor{currentstroke}{rgb}{1.000000,0.000000,0.000000}%
\pgfsetstrokecolor{currentstroke}%
\pgfsetdash{}{0pt}%
\pgfpathmoveto{\pgfqpoint{1.094090in}{1.640984in}}%
\pgfpathlineto{\pgfqpoint{1.064411in}{1.656912in}}%
\pgfusepath{stroke}%
\end{pgfscope}%
\begin{pgfscope}%
\pgfpathrectangle{\pgfqpoint{0.100000in}{0.212622in}}{\pgfqpoint{3.696000in}{3.696000in}}%
\pgfusepath{clip}%
\pgfsetrectcap%
\pgfsetroundjoin%
\pgfsetlinewidth{1.505625pt}%
\definecolor{currentstroke}{rgb}{1.000000,0.000000,0.000000}%
\pgfsetstrokecolor{currentstroke}%
\pgfsetdash{}{0pt}%
\pgfpathmoveto{\pgfqpoint{1.095863in}{1.641310in}}%
\pgfpathlineto{\pgfqpoint{1.064411in}{1.656912in}}%
\pgfusepath{stroke}%
\end{pgfscope}%
\begin{pgfscope}%
\pgfpathrectangle{\pgfqpoint{0.100000in}{0.212622in}}{\pgfqpoint{3.696000in}{3.696000in}}%
\pgfusepath{clip}%
\pgfsetrectcap%
\pgfsetroundjoin%
\pgfsetlinewidth{1.505625pt}%
\definecolor{currentstroke}{rgb}{1.000000,0.000000,0.000000}%
\pgfsetstrokecolor{currentstroke}%
\pgfsetdash{}{0pt}%
\pgfpathmoveto{\pgfqpoint{1.098424in}{1.644078in}}%
\pgfpathlineto{\pgfqpoint{1.064411in}{1.656912in}}%
\pgfusepath{stroke}%
\end{pgfscope}%
\begin{pgfscope}%
\pgfpathrectangle{\pgfqpoint{0.100000in}{0.212622in}}{\pgfqpoint{3.696000in}{3.696000in}}%
\pgfusepath{clip}%
\pgfsetrectcap%
\pgfsetroundjoin%
\pgfsetlinewidth{1.505625pt}%
\definecolor{currentstroke}{rgb}{1.000000,0.000000,0.000000}%
\pgfsetstrokecolor{currentstroke}%
\pgfsetdash{}{0pt}%
\pgfpathmoveto{\pgfqpoint{1.102022in}{1.646485in}}%
\pgfpathlineto{\pgfqpoint{1.074654in}{1.665528in}}%
\pgfusepath{stroke}%
\end{pgfscope}%
\begin{pgfscope}%
\pgfpathrectangle{\pgfqpoint{0.100000in}{0.212622in}}{\pgfqpoint{3.696000in}{3.696000in}}%
\pgfusepath{clip}%
\pgfsetrectcap%
\pgfsetroundjoin%
\pgfsetlinewidth{1.505625pt}%
\definecolor{currentstroke}{rgb}{1.000000,0.000000,0.000000}%
\pgfsetstrokecolor{currentstroke}%
\pgfsetdash{}{0pt}%
\pgfpathmoveto{\pgfqpoint{1.106402in}{1.648170in}}%
\pgfpathlineto{\pgfqpoint{1.074654in}{1.665528in}}%
\pgfusepath{stroke}%
\end{pgfscope}%
\begin{pgfscope}%
\pgfpathrectangle{\pgfqpoint{0.100000in}{0.212622in}}{\pgfqpoint{3.696000in}{3.696000in}}%
\pgfusepath{clip}%
\pgfsetrectcap%
\pgfsetroundjoin%
\pgfsetlinewidth{1.505625pt}%
\definecolor{currentstroke}{rgb}{1.000000,0.000000,0.000000}%
\pgfsetstrokecolor{currentstroke}%
\pgfsetdash{}{0pt}%
\pgfpathmoveto{\pgfqpoint{1.109371in}{1.649962in}}%
\pgfpathlineto{\pgfqpoint{1.074654in}{1.665528in}}%
\pgfusepath{stroke}%
\end{pgfscope}%
\begin{pgfscope}%
\pgfpathrectangle{\pgfqpoint{0.100000in}{0.212622in}}{\pgfqpoint{3.696000in}{3.696000in}}%
\pgfusepath{clip}%
\pgfsetrectcap%
\pgfsetroundjoin%
\pgfsetlinewidth{1.505625pt}%
\definecolor{currentstroke}{rgb}{1.000000,0.000000,0.000000}%
\pgfsetstrokecolor{currentstroke}%
\pgfsetdash{}{0pt}%
\pgfpathmoveto{\pgfqpoint{1.112832in}{1.655562in}}%
\pgfpathlineto{\pgfqpoint{1.084883in}{1.674132in}}%
\pgfusepath{stroke}%
\end{pgfscope}%
\begin{pgfscope}%
\pgfpathrectangle{\pgfqpoint{0.100000in}{0.212622in}}{\pgfqpoint{3.696000in}{3.696000in}}%
\pgfusepath{clip}%
\pgfsetrectcap%
\pgfsetroundjoin%
\pgfsetlinewidth{1.505625pt}%
\definecolor{currentstroke}{rgb}{1.000000,0.000000,0.000000}%
\pgfsetstrokecolor{currentstroke}%
\pgfsetdash{}{0pt}%
\pgfpathmoveto{\pgfqpoint{1.118096in}{1.657527in}}%
\pgfpathlineto{\pgfqpoint{1.084883in}{1.674132in}}%
\pgfusepath{stroke}%
\end{pgfscope}%
\begin{pgfscope}%
\pgfpathrectangle{\pgfqpoint{0.100000in}{0.212622in}}{\pgfqpoint{3.696000in}{3.696000in}}%
\pgfusepath{clip}%
\pgfsetrectcap%
\pgfsetroundjoin%
\pgfsetlinewidth{1.505625pt}%
\definecolor{currentstroke}{rgb}{1.000000,0.000000,0.000000}%
\pgfsetstrokecolor{currentstroke}%
\pgfsetdash{}{0pt}%
\pgfpathmoveto{\pgfqpoint{1.120888in}{1.659656in}}%
\pgfpathlineto{\pgfqpoint{1.084883in}{1.674132in}}%
\pgfusepath{stroke}%
\end{pgfscope}%
\begin{pgfscope}%
\pgfpathrectangle{\pgfqpoint{0.100000in}{0.212622in}}{\pgfqpoint{3.696000in}{3.696000in}}%
\pgfusepath{clip}%
\pgfsetrectcap%
\pgfsetroundjoin%
\pgfsetlinewidth{1.505625pt}%
\definecolor{currentstroke}{rgb}{1.000000,0.000000,0.000000}%
\pgfsetstrokecolor{currentstroke}%
\pgfsetdash{}{0pt}%
\pgfpathmoveto{\pgfqpoint{1.124400in}{1.662380in}}%
\pgfpathlineto{\pgfqpoint{1.095097in}{1.682723in}}%
\pgfusepath{stroke}%
\end{pgfscope}%
\begin{pgfscope}%
\pgfpathrectangle{\pgfqpoint{0.100000in}{0.212622in}}{\pgfqpoint{3.696000in}{3.696000in}}%
\pgfusepath{clip}%
\pgfsetrectcap%
\pgfsetroundjoin%
\pgfsetlinewidth{1.505625pt}%
\definecolor{currentstroke}{rgb}{1.000000,0.000000,0.000000}%
\pgfsetstrokecolor{currentstroke}%
\pgfsetdash{}{0pt}%
\pgfpathmoveto{\pgfqpoint{1.128209in}{1.662105in}}%
\pgfpathlineto{\pgfqpoint{1.095097in}{1.682723in}}%
\pgfusepath{stroke}%
\end{pgfscope}%
\begin{pgfscope}%
\pgfpathrectangle{\pgfqpoint{0.100000in}{0.212622in}}{\pgfqpoint{3.696000in}{3.696000in}}%
\pgfusepath{clip}%
\pgfsetrectcap%
\pgfsetroundjoin%
\pgfsetlinewidth{1.505625pt}%
\definecolor{currentstroke}{rgb}{1.000000,0.000000,0.000000}%
\pgfsetstrokecolor{currentstroke}%
\pgfsetdash{}{0pt}%
\pgfpathmoveto{\pgfqpoint{1.133115in}{1.664756in}}%
\pgfpathlineto{\pgfqpoint{1.095097in}{1.682723in}}%
\pgfusepath{stroke}%
\end{pgfscope}%
\begin{pgfscope}%
\pgfpathrectangle{\pgfqpoint{0.100000in}{0.212622in}}{\pgfqpoint{3.696000in}{3.696000in}}%
\pgfusepath{clip}%
\pgfsetrectcap%
\pgfsetroundjoin%
\pgfsetlinewidth{1.505625pt}%
\definecolor{currentstroke}{rgb}{1.000000,0.000000,0.000000}%
\pgfsetstrokecolor{currentstroke}%
\pgfsetdash{}{0pt}%
\pgfpathmoveto{\pgfqpoint{1.135957in}{1.668154in}}%
\pgfpathlineto{\pgfqpoint{1.105297in}{1.691302in}}%
\pgfusepath{stroke}%
\end{pgfscope}%
\begin{pgfscope}%
\pgfpathrectangle{\pgfqpoint{0.100000in}{0.212622in}}{\pgfqpoint{3.696000in}{3.696000in}}%
\pgfusepath{clip}%
\pgfsetrectcap%
\pgfsetroundjoin%
\pgfsetlinewidth{1.505625pt}%
\definecolor{currentstroke}{rgb}{1.000000,0.000000,0.000000}%
\pgfsetstrokecolor{currentstroke}%
\pgfsetdash{}{0pt}%
\pgfpathmoveto{\pgfqpoint{1.139188in}{1.668898in}}%
\pgfpathlineto{\pgfqpoint{1.105297in}{1.691302in}}%
\pgfusepath{stroke}%
\end{pgfscope}%
\begin{pgfscope}%
\pgfpathrectangle{\pgfqpoint{0.100000in}{0.212622in}}{\pgfqpoint{3.696000in}{3.696000in}}%
\pgfusepath{clip}%
\pgfsetrectcap%
\pgfsetroundjoin%
\pgfsetlinewidth{1.505625pt}%
\definecolor{currentstroke}{rgb}{1.000000,0.000000,0.000000}%
\pgfsetstrokecolor{currentstroke}%
\pgfsetdash{}{0pt}%
\pgfpathmoveto{\pgfqpoint{1.143055in}{1.671340in}}%
\pgfpathlineto{\pgfqpoint{1.105297in}{1.691302in}}%
\pgfusepath{stroke}%
\end{pgfscope}%
\begin{pgfscope}%
\pgfpathrectangle{\pgfqpoint{0.100000in}{0.212622in}}{\pgfqpoint{3.696000in}{3.696000in}}%
\pgfusepath{clip}%
\pgfsetrectcap%
\pgfsetroundjoin%
\pgfsetlinewidth{1.505625pt}%
\definecolor{currentstroke}{rgb}{1.000000,0.000000,0.000000}%
\pgfsetstrokecolor{currentstroke}%
\pgfsetdash{}{0pt}%
\pgfpathmoveto{\pgfqpoint{1.147498in}{1.675045in}}%
\pgfpathlineto{\pgfqpoint{1.115481in}{1.699869in}}%
\pgfusepath{stroke}%
\end{pgfscope}%
\begin{pgfscope}%
\pgfpathrectangle{\pgfqpoint{0.100000in}{0.212622in}}{\pgfqpoint{3.696000in}{3.696000in}}%
\pgfusepath{clip}%
\pgfsetrectcap%
\pgfsetroundjoin%
\pgfsetlinewidth{1.505625pt}%
\definecolor{currentstroke}{rgb}{1.000000,0.000000,0.000000}%
\pgfsetstrokecolor{currentstroke}%
\pgfsetdash{}{0pt}%
\pgfpathmoveto{\pgfqpoint{1.151798in}{1.675474in}}%
\pgfpathlineto{\pgfqpoint{1.115481in}{1.699869in}}%
\pgfusepath{stroke}%
\end{pgfscope}%
\begin{pgfscope}%
\pgfpathrectangle{\pgfqpoint{0.100000in}{0.212622in}}{\pgfqpoint{3.696000in}{3.696000in}}%
\pgfusepath{clip}%
\pgfsetrectcap%
\pgfsetroundjoin%
\pgfsetlinewidth{1.505625pt}%
\definecolor{currentstroke}{rgb}{1.000000,0.000000,0.000000}%
\pgfsetstrokecolor{currentstroke}%
\pgfsetdash{}{0pt}%
\pgfpathmoveto{\pgfqpoint{1.156214in}{1.656808in}}%
\pgfpathlineto{\pgfqpoint{1.115481in}{1.699869in}}%
\pgfusepath{stroke}%
\end{pgfscope}%
\begin{pgfscope}%
\pgfpathrectangle{\pgfqpoint{0.100000in}{0.212622in}}{\pgfqpoint{3.696000in}{3.696000in}}%
\pgfusepath{clip}%
\pgfsetrectcap%
\pgfsetroundjoin%
\pgfsetlinewidth{1.505625pt}%
\definecolor{currentstroke}{rgb}{1.000000,0.000000,0.000000}%
\pgfsetstrokecolor{currentstroke}%
\pgfsetdash{}{0pt}%
\pgfpathmoveto{\pgfqpoint{1.158668in}{1.659198in}}%
\pgfpathlineto{\pgfqpoint{1.125652in}{1.708424in}}%
\pgfusepath{stroke}%
\end{pgfscope}%
\begin{pgfscope}%
\pgfpathrectangle{\pgfqpoint{0.100000in}{0.212622in}}{\pgfqpoint{3.696000in}{3.696000in}}%
\pgfusepath{clip}%
\pgfsetrectcap%
\pgfsetroundjoin%
\pgfsetlinewidth{1.505625pt}%
\definecolor{currentstroke}{rgb}{1.000000,0.000000,0.000000}%
\pgfsetstrokecolor{currentstroke}%
\pgfsetdash{}{0pt}%
\pgfpathmoveto{\pgfqpoint{1.161840in}{1.658419in}}%
\pgfpathlineto{\pgfqpoint{1.125652in}{1.708424in}}%
\pgfusepath{stroke}%
\end{pgfscope}%
\begin{pgfscope}%
\pgfpathrectangle{\pgfqpoint{0.100000in}{0.212622in}}{\pgfqpoint{3.696000in}{3.696000in}}%
\pgfusepath{clip}%
\pgfsetrectcap%
\pgfsetroundjoin%
\pgfsetlinewidth{1.505625pt}%
\definecolor{currentstroke}{rgb}{1.000000,0.000000,0.000000}%
\pgfsetstrokecolor{currentstroke}%
\pgfsetdash{}{0pt}%
\pgfpathmoveto{\pgfqpoint{1.166545in}{1.651570in}}%
\pgfpathlineto{\pgfqpoint{1.125652in}{1.708424in}}%
\pgfusepath{stroke}%
\end{pgfscope}%
\begin{pgfscope}%
\pgfpathrectangle{\pgfqpoint{0.100000in}{0.212622in}}{\pgfqpoint{3.696000in}{3.696000in}}%
\pgfusepath{clip}%
\pgfsetrectcap%
\pgfsetroundjoin%
\pgfsetlinewidth{1.505625pt}%
\definecolor{currentstroke}{rgb}{1.000000,0.000000,0.000000}%
\pgfsetstrokecolor{currentstroke}%
\pgfsetdash{}{0pt}%
\pgfpathmoveto{\pgfqpoint{1.168789in}{1.655286in}}%
\pgfpathlineto{\pgfqpoint{1.135807in}{1.716966in}}%
\pgfusepath{stroke}%
\end{pgfscope}%
\begin{pgfscope}%
\pgfpathrectangle{\pgfqpoint{0.100000in}{0.212622in}}{\pgfqpoint{3.696000in}{3.696000in}}%
\pgfusepath{clip}%
\pgfsetrectcap%
\pgfsetroundjoin%
\pgfsetlinewidth{1.505625pt}%
\definecolor{currentstroke}{rgb}{1.000000,0.000000,0.000000}%
\pgfsetstrokecolor{currentstroke}%
\pgfsetdash{}{0pt}%
\pgfpathmoveto{\pgfqpoint{1.171293in}{1.652799in}}%
\pgfpathlineto{\pgfqpoint{1.135807in}{1.716966in}}%
\pgfusepath{stroke}%
\end{pgfscope}%
\begin{pgfscope}%
\pgfpathrectangle{\pgfqpoint{0.100000in}{0.212622in}}{\pgfqpoint{3.696000in}{3.696000in}}%
\pgfusepath{clip}%
\pgfsetrectcap%
\pgfsetroundjoin%
\pgfsetlinewidth{1.505625pt}%
\definecolor{currentstroke}{rgb}{1.000000,0.000000,0.000000}%
\pgfsetstrokecolor{currentstroke}%
\pgfsetdash{}{0pt}%
\pgfpathmoveto{\pgfqpoint{1.174479in}{1.656579in}}%
\pgfpathlineto{\pgfqpoint{1.135807in}{1.716966in}}%
\pgfusepath{stroke}%
\end{pgfscope}%
\begin{pgfscope}%
\pgfpathrectangle{\pgfqpoint{0.100000in}{0.212622in}}{\pgfqpoint{3.696000in}{3.696000in}}%
\pgfusepath{clip}%
\pgfsetrectcap%
\pgfsetroundjoin%
\pgfsetlinewidth{1.505625pt}%
\definecolor{currentstroke}{rgb}{1.000000,0.000000,0.000000}%
\pgfsetstrokecolor{currentstroke}%
\pgfsetdash{}{0pt}%
\pgfpathmoveto{\pgfqpoint{1.176361in}{1.657123in}}%
\pgfpathlineto{\pgfqpoint{1.135807in}{1.716966in}}%
\pgfusepath{stroke}%
\end{pgfscope}%
\begin{pgfscope}%
\pgfpathrectangle{\pgfqpoint{0.100000in}{0.212622in}}{\pgfqpoint{3.696000in}{3.696000in}}%
\pgfusepath{clip}%
\pgfsetrectcap%
\pgfsetroundjoin%
\pgfsetlinewidth{1.505625pt}%
\definecolor{currentstroke}{rgb}{1.000000,0.000000,0.000000}%
\pgfsetstrokecolor{currentstroke}%
\pgfsetdash{}{0pt}%
\pgfpathmoveto{\pgfqpoint{1.177306in}{1.656965in}}%
\pgfpathlineto{\pgfqpoint{1.135807in}{1.716966in}}%
\pgfusepath{stroke}%
\end{pgfscope}%
\begin{pgfscope}%
\pgfpathrectangle{\pgfqpoint{0.100000in}{0.212622in}}{\pgfqpoint{3.696000in}{3.696000in}}%
\pgfusepath{clip}%
\pgfsetrectcap%
\pgfsetroundjoin%
\pgfsetlinewidth{1.505625pt}%
\definecolor{currentstroke}{rgb}{1.000000,0.000000,0.000000}%
\pgfsetstrokecolor{currentstroke}%
\pgfsetdash{}{0pt}%
\pgfpathmoveto{\pgfqpoint{1.178892in}{1.657985in}}%
\pgfpathlineto{\pgfqpoint{1.135807in}{1.716966in}}%
\pgfusepath{stroke}%
\end{pgfscope}%
\begin{pgfscope}%
\pgfpathrectangle{\pgfqpoint{0.100000in}{0.212622in}}{\pgfqpoint{3.696000in}{3.696000in}}%
\pgfusepath{clip}%
\pgfsetrectcap%
\pgfsetroundjoin%
\pgfsetlinewidth{1.505625pt}%
\definecolor{currentstroke}{rgb}{1.000000,0.000000,0.000000}%
\pgfsetstrokecolor{currentstroke}%
\pgfsetdash{}{0pt}%
\pgfpathmoveto{\pgfqpoint{1.180897in}{1.658398in}}%
\pgfpathlineto{\pgfqpoint{1.145948in}{1.725496in}}%
\pgfusepath{stroke}%
\end{pgfscope}%
\begin{pgfscope}%
\pgfpathrectangle{\pgfqpoint{0.100000in}{0.212622in}}{\pgfqpoint{3.696000in}{3.696000in}}%
\pgfusepath{clip}%
\pgfsetrectcap%
\pgfsetroundjoin%
\pgfsetlinewidth{1.505625pt}%
\definecolor{currentstroke}{rgb}{1.000000,0.000000,0.000000}%
\pgfsetstrokecolor{currentstroke}%
\pgfsetdash{}{0pt}%
\pgfpathmoveto{\pgfqpoint{1.183381in}{1.659792in}}%
\pgfpathlineto{\pgfqpoint{1.145948in}{1.725496in}}%
\pgfusepath{stroke}%
\end{pgfscope}%
\begin{pgfscope}%
\pgfpathrectangle{\pgfqpoint{0.100000in}{0.212622in}}{\pgfqpoint{3.696000in}{3.696000in}}%
\pgfusepath{clip}%
\pgfsetrectcap%
\pgfsetroundjoin%
\pgfsetlinewidth{1.505625pt}%
\definecolor{currentstroke}{rgb}{1.000000,0.000000,0.000000}%
\pgfsetstrokecolor{currentstroke}%
\pgfsetdash{}{0pt}%
\pgfpathmoveto{\pgfqpoint{1.186882in}{1.661833in}}%
\pgfpathlineto{\pgfqpoint{1.145948in}{1.725496in}}%
\pgfusepath{stroke}%
\end{pgfscope}%
\begin{pgfscope}%
\pgfpathrectangle{\pgfqpoint{0.100000in}{0.212622in}}{\pgfqpoint{3.696000in}{3.696000in}}%
\pgfusepath{clip}%
\pgfsetrectcap%
\pgfsetroundjoin%
\pgfsetlinewidth{1.505625pt}%
\definecolor{currentstroke}{rgb}{1.000000,0.000000,0.000000}%
\pgfsetstrokecolor{currentstroke}%
\pgfsetdash{}{0pt}%
\pgfpathmoveto{\pgfqpoint{1.190499in}{1.660547in}}%
\pgfpathlineto{\pgfqpoint{1.145948in}{1.725496in}}%
\pgfusepath{stroke}%
\end{pgfscope}%
\begin{pgfscope}%
\pgfpathrectangle{\pgfqpoint{0.100000in}{0.212622in}}{\pgfqpoint{3.696000in}{3.696000in}}%
\pgfusepath{clip}%
\pgfsetrectcap%
\pgfsetroundjoin%
\pgfsetlinewidth{1.505625pt}%
\definecolor{currentstroke}{rgb}{1.000000,0.000000,0.000000}%
\pgfsetstrokecolor{currentstroke}%
\pgfsetdash{}{0pt}%
\pgfpathmoveto{\pgfqpoint{1.192777in}{1.660890in}}%
\pgfpathlineto{\pgfqpoint{1.156075in}{1.734014in}}%
\pgfusepath{stroke}%
\end{pgfscope}%
\begin{pgfscope}%
\pgfpathrectangle{\pgfqpoint{0.100000in}{0.212622in}}{\pgfqpoint{3.696000in}{3.696000in}}%
\pgfusepath{clip}%
\pgfsetrectcap%
\pgfsetroundjoin%
\pgfsetlinewidth{1.505625pt}%
\definecolor{currentstroke}{rgb}{1.000000,0.000000,0.000000}%
\pgfsetstrokecolor{currentstroke}%
\pgfsetdash{}{0pt}%
\pgfpathmoveto{\pgfqpoint{1.194068in}{1.662240in}}%
\pgfpathlineto{\pgfqpoint{1.156075in}{1.734014in}}%
\pgfusepath{stroke}%
\end{pgfscope}%
\begin{pgfscope}%
\pgfpathrectangle{\pgfqpoint{0.100000in}{0.212622in}}{\pgfqpoint{3.696000in}{3.696000in}}%
\pgfusepath{clip}%
\pgfsetrectcap%
\pgfsetroundjoin%
\pgfsetlinewidth{1.505625pt}%
\definecolor{currentstroke}{rgb}{1.000000,0.000000,0.000000}%
\pgfsetstrokecolor{currentstroke}%
\pgfsetdash{}{0pt}%
\pgfpathmoveto{\pgfqpoint{1.196490in}{1.661750in}}%
\pgfpathlineto{\pgfqpoint{1.156075in}{1.734014in}}%
\pgfusepath{stroke}%
\end{pgfscope}%
\begin{pgfscope}%
\pgfpathrectangle{\pgfqpoint{0.100000in}{0.212622in}}{\pgfqpoint{3.696000in}{3.696000in}}%
\pgfusepath{clip}%
\pgfsetrectcap%
\pgfsetroundjoin%
\pgfsetlinewidth{1.505625pt}%
\definecolor{currentstroke}{rgb}{1.000000,0.000000,0.000000}%
\pgfsetstrokecolor{currentstroke}%
\pgfsetdash{}{0pt}%
\pgfpathmoveto{\pgfqpoint{1.197990in}{1.663925in}}%
\pgfpathlineto{\pgfqpoint{1.156075in}{1.734014in}}%
\pgfusepath{stroke}%
\end{pgfscope}%
\begin{pgfscope}%
\pgfpathrectangle{\pgfqpoint{0.100000in}{0.212622in}}{\pgfqpoint{3.696000in}{3.696000in}}%
\pgfusepath{clip}%
\pgfsetrectcap%
\pgfsetroundjoin%
\pgfsetlinewidth{1.505625pt}%
\definecolor{currentstroke}{rgb}{1.000000,0.000000,0.000000}%
\pgfsetstrokecolor{currentstroke}%
\pgfsetdash{}{0pt}%
\pgfpathmoveto{\pgfqpoint{1.198948in}{1.664231in}}%
\pgfpathlineto{\pgfqpoint{1.156075in}{1.734014in}}%
\pgfusepath{stroke}%
\end{pgfscope}%
\begin{pgfscope}%
\pgfpathrectangle{\pgfqpoint{0.100000in}{0.212622in}}{\pgfqpoint{3.696000in}{3.696000in}}%
\pgfusepath{clip}%
\pgfsetrectcap%
\pgfsetroundjoin%
\pgfsetlinewidth{1.505625pt}%
\definecolor{currentstroke}{rgb}{1.000000,0.000000,0.000000}%
\pgfsetstrokecolor{currentstroke}%
\pgfsetdash{}{0pt}%
\pgfpathmoveto{\pgfqpoint{1.200942in}{1.663577in}}%
\pgfpathlineto{\pgfqpoint{1.156075in}{1.734014in}}%
\pgfusepath{stroke}%
\end{pgfscope}%
\begin{pgfscope}%
\pgfpathrectangle{\pgfqpoint{0.100000in}{0.212622in}}{\pgfqpoint{3.696000in}{3.696000in}}%
\pgfusepath{clip}%
\pgfsetrectcap%
\pgfsetroundjoin%
\pgfsetlinewidth{1.505625pt}%
\definecolor{currentstroke}{rgb}{1.000000,0.000000,0.000000}%
\pgfsetstrokecolor{currentstroke}%
\pgfsetdash{}{0pt}%
\pgfpathmoveto{\pgfqpoint{1.202190in}{1.663740in}}%
\pgfpathlineto{\pgfqpoint{1.156075in}{1.734014in}}%
\pgfusepath{stroke}%
\end{pgfscope}%
\begin{pgfscope}%
\pgfpathrectangle{\pgfqpoint{0.100000in}{0.212622in}}{\pgfqpoint{3.696000in}{3.696000in}}%
\pgfusepath{clip}%
\pgfsetrectcap%
\pgfsetroundjoin%
\pgfsetlinewidth{1.505625pt}%
\definecolor{currentstroke}{rgb}{1.000000,0.000000,0.000000}%
\pgfsetstrokecolor{currentstroke}%
\pgfsetdash{}{0pt}%
\pgfpathmoveto{\pgfqpoint{1.202950in}{1.664234in}}%
\pgfpathlineto{\pgfqpoint{1.156075in}{1.734014in}}%
\pgfusepath{stroke}%
\end{pgfscope}%
\begin{pgfscope}%
\pgfpathrectangle{\pgfqpoint{0.100000in}{0.212622in}}{\pgfqpoint{3.696000in}{3.696000in}}%
\pgfusepath{clip}%
\pgfsetrectcap%
\pgfsetroundjoin%
\pgfsetlinewidth{1.505625pt}%
\definecolor{currentstroke}{rgb}{1.000000,0.000000,0.000000}%
\pgfsetstrokecolor{currentstroke}%
\pgfsetdash{}{0pt}%
\pgfpathmoveto{\pgfqpoint{1.204533in}{1.663843in}}%
\pgfpathlineto{\pgfqpoint{1.166187in}{1.742519in}}%
\pgfusepath{stroke}%
\end{pgfscope}%
\begin{pgfscope}%
\pgfpathrectangle{\pgfqpoint{0.100000in}{0.212622in}}{\pgfqpoint{3.696000in}{3.696000in}}%
\pgfusepath{clip}%
\pgfsetrectcap%
\pgfsetroundjoin%
\pgfsetlinewidth{1.505625pt}%
\definecolor{currentstroke}{rgb}{1.000000,0.000000,0.000000}%
\pgfsetstrokecolor{currentstroke}%
\pgfsetdash{}{0pt}%
\pgfpathmoveto{\pgfqpoint{1.205539in}{1.663750in}}%
\pgfpathlineto{\pgfqpoint{1.166187in}{1.742519in}}%
\pgfusepath{stroke}%
\end{pgfscope}%
\begin{pgfscope}%
\pgfpathrectangle{\pgfqpoint{0.100000in}{0.212622in}}{\pgfqpoint{3.696000in}{3.696000in}}%
\pgfusepath{clip}%
\pgfsetrectcap%
\pgfsetroundjoin%
\pgfsetlinewidth{1.505625pt}%
\definecolor{currentstroke}{rgb}{1.000000,0.000000,0.000000}%
\pgfsetstrokecolor{currentstroke}%
\pgfsetdash{}{0pt}%
\pgfpathmoveto{\pgfqpoint{1.207041in}{1.664628in}}%
\pgfpathlineto{\pgfqpoint{1.166187in}{1.742519in}}%
\pgfusepath{stroke}%
\end{pgfscope}%
\begin{pgfscope}%
\pgfpathrectangle{\pgfqpoint{0.100000in}{0.212622in}}{\pgfqpoint{3.696000in}{3.696000in}}%
\pgfusepath{clip}%
\pgfsetrectcap%
\pgfsetroundjoin%
\pgfsetlinewidth{1.505625pt}%
\definecolor{currentstroke}{rgb}{1.000000,0.000000,0.000000}%
\pgfsetstrokecolor{currentstroke}%
\pgfsetdash{}{0pt}%
\pgfpathmoveto{\pgfqpoint{1.209344in}{1.665429in}}%
\pgfpathlineto{\pgfqpoint{1.166187in}{1.742519in}}%
\pgfusepath{stroke}%
\end{pgfscope}%
\begin{pgfscope}%
\pgfpathrectangle{\pgfqpoint{0.100000in}{0.212622in}}{\pgfqpoint{3.696000in}{3.696000in}}%
\pgfusepath{clip}%
\pgfsetrectcap%
\pgfsetroundjoin%
\pgfsetlinewidth{1.505625pt}%
\definecolor{currentstroke}{rgb}{1.000000,0.000000,0.000000}%
\pgfsetstrokecolor{currentstroke}%
\pgfsetdash{}{0pt}%
\pgfpathmoveto{\pgfqpoint{1.210755in}{1.665140in}}%
\pgfpathlineto{\pgfqpoint{1.166187in}{1.742519in}}%
\pgfusepath{stroke}%
\end{pgfscope}%
\begin{pgfscope}%
\pgfpathrectangle{\pgfqpoint{0.100000in}{0.212622in}}{\pgfqpoint{3.696000in}{3.696000in}}%
\pgfusepath{clip}%
\pgfsetrectcap%
\pgfsetroundjoin%
\pgfsetlinewidth{1.505625pt}%
\definecolor{currentstroke}{rgb}{1.000000,0.000000,0.000000}%
\pgfsetstrokecolor{currentstroke}%
\pgfsetdash{}{0pt}%
\pgfpathmoveto{\pgfqpoint{1.212859in}{1.665911in}}%
\pgfpathlineto{\pgfqpoint{1.166187in}{1.742519in}}%
\pgfusepath{stroke}%
\end{pgfscope}%
\begin{pgfscope}%
\pgfpathrectangle{\pgfqpoint{0.100000in}{0.212622in}}{\pgfqpoint{3.696000in}{3.696000in}}%
\pgfusepath{clip}%
\pgfsetrectcap%
\pgfsetroundjoin%
\pgfsetlinewidth{1.505625pt}%
\definecolor{currentstroke}{rgb}{1.000000,0.000000,0.000000}%
\pgfsetstrokecolor{currentstroke}%
\pgfsetdash{}{0pt}%
\pgfpathmoveto{\pgfqpoint{1.215469in}{1.665945in}}%
\pgfpathlineto{\pgfqpoint{1.166187in}{1.742519in}}%
\pgfusepath{stroke}%
\end{pgfscope}%
\begin{pgfscope}%
\pgfpathrectangle{\pgfqpoint{0.100000in}{0.212622in}}{\pgfqpoint{3.696000in}{3.696000in}}%
\pgfusepath{clip}%
\pgfsetrectcap%
\pgfsetroundjoin%
\pgfsetlinewidth{1.505625pt}%
\definecolor{currentstroke}{rgb}{1.000000,0.000000,0.000000}%
\pgfsetstrokecolor{currentstroke}%
\pgfsetdash{}{0pt}%
\pgfpathmoveto{\pgfqpoint{1.217134in}{1.665966in}}%
\pgfpathlineto{\pgfqpoint{1.176285in}{1.751013in}}%
\pgfusepath{stroke}%
\end{pgfscope}%
\begin{pgfscope}%
\pgfpathrectangle{\pgfqpoint{0.100000in}{0.212622in}}{\pgfqpoint{3.696000in}{3.696000in}}%
\pgfusepath{clip}%
\pgfsetrectcap%
\pgfsetroundjoin%
\pgfsetlinewidth{1.505625pt}%
\definecolor{currentstroke}{rgb}{1.000000,0.000000,0.000000}%
\pgfsetstrokecolor{currentstroke}%
\pgfsetdash{}{0pt}%
\pgfpathmoveto{\pgfqpoint{1.219135in}{1.667095in}}%
\pgfpathlineto{\pgfqpoint{1.176285in}{1.751013in}}%
\pgfusepath{stroke}%
\end{pgfscope}%
\begin{pgfscope}%
\pgfpathrectangle{\pgfqpoint{0.100000in}{0.212622in}}{\pgfqpoint{3.696000in}{3.696000in}}%
\pgfusepath{clip}%
\pgfsetrectcap%
\pgfsetroundjoin%
\pgfsetlinewidth{1.505625pt}%
\definecolor{currentstroke}{rgb}{1.000000,0.000000,0.000000}%
\pgfsetstrokecolor{currentstroke}%
\pgfsetdash{}{0pt}%
\pgfpathmoveto{\pgfqpoint{1.222191in}{1.665883in}}%
\pgfpathlineto{\pgfqpoint{1.176285in}{1.751013in}}%
\pgfusepath{stroke}%
\end{pgfscope}%
\begin{pgfscope}%
\pgfpathrectangle{\pgfqpoint{0.100000in}{0.212622in}}{\pgfqpoint{3.696000in}{3.696000in}}%
\pgfusepath{clip}%
\pgfsetrectcap%
\pgfsetroundjoin%
\pgfsetlinewidth{1.505625pt}%
\definecolor{currentstroke}{rgb}{1.000000,0.000000,0.000000}%
\pgfsetstrokecolor{currentstroke}%
\pgfsetdash{}{0pt}%
\pgfpathmoveto{\pgfqpoint{1.224487in}{1.669002in}}%
\pgfpathlineto{\pgfqpoint{1.176285in}{1.751013in}}%
\pgfusepath{stroke}%
\end{pgfscope}%
\begin{pgfscope}%
\pgfpathrectangle{\pgfqpoint{0.100000in}{0.212622in}}{\pgfqpoint{3.696000in}{3.696000in}}%
\pgfusepath{clip}%
\pgfsetrectcap%
\pgfsetroundjoin%
\pgfsetlinewidth{1.505625pt}%
\definecolor{currentstroke}{rgb}{1.000000,0.000000,0.000000}%
\pgfsetstrokecolor{currentstroke}%
\pgfsetdash{}{0pt}%
\pgfpathmoveto{\pgfqpoint{1.229124in}{1.671865in}}%
\pgfpathlineto{\pgfqpoint{1.186368in}{1.759494in}}%
\pgfusepath{stroke}%
\end{pgfscope}%
\begin{pgfscope}%
\pgfpathrectangle{\pgfqpoint{0.100000in}{0.212622in}}{\pgfqpoint{3.696000in}{3.696000in}}%
\pgfusepath{clip}%
\pgfsetrectcap%
\pgfsetroundjoin%
\pgfsetlinewidth{1.505625pt}%
\definecolor{currentstroke}{rgb}{1.000000,0.000000,0.000000}%
\pgfsetstrokecolor{currentstroke}%
\pgfsetdash{}{0pt}%
\pgfpathmoveto{\pgfqpoint{1.234365in}{1.667903in}}%
\pgfpathlineto{\pgfqpoint{1.186368in}{1.759494in}}%
\pgfusepath{stroke}%
\end{pgfscope}%
\begin{pgfscope}%
\pgfpathrectangle{\pgfqpoint{0.100000in}{0.212622in}}{\pgfqpoint{3.696000in}{3.696000in}}%
\pgfusepath{clip}%
\pgfsetrectcap%
\pgfsetroundjoin%
\pgfsetlinewidth{1.505625pt}%
\definecolor{currentstroke}{rgb}{1.000000,0.000000,0.000000}%
\pgfsetstrokecolor{currentstroke}%
\pgfsetdash{}{0pt}%
\pgfpathmoveto{\pgfqpoint{1.237758in}{1.672366in}}%
\pgfpathlineto{\pgfqpoint{1.186368in}{1.759494in}}%
\pgfusepath{stroke}%
\end{pgfscope}%
\begin{pgfscope}%
\pgfpathrectangle{\pgfqpoint{0.100000in}{0.212622in}}{\pgfqpoint{3.696000in}{3.696000in}}%
\pgfusepath{clip}%
\pgfsetrectcap%
\pgfsetroundjoin%
\pgfsetlinewidth{1.505625pt}%
\definecolor{currentstroke}{rgb}{1.000000,0.000000,0.000000}%
\pgfsetstrokecolor{currentstroke}%
\pgfsetdash{}{0pt}%
\pgfpathmoveto{\pgfqpoint{1.243545in}{1.674749in}}%
\pgfpathlineto{\pgfqpoint{1.196437in}{1.767964in}}%
\pgfusepath{stroke}%
\end{pgfscope}%
\begin{pgfscope}%
\pgfpathrectangle{\pgfqpoint{0.100000in}{0.212622in}}{\pgfqpoint{3.696000in}{3.696000in}}%
\pgfusepath{clip}%
\pgfsetrectcap%
\pgfsetroundjoin%
\pgfsetlinewidth{1.505625pt}%
\definecolor{currentstroke}{rgb}{1.000000,0.000000,0.000000}%
\pgfsetstrokecolor{currentstroke}%
\pgfsetdash{}{0pt}%
\pgfpathmoveto{\pgfqpoint{1.249501in}{1.670911in}}%
\pgfpathlineto{\pgfqpoint{1.196437in}{1.767964in}}%
\pgfusepath{stroke}%
\end{pgfscope}%
\begin{pgfscope}%
\pgfpathrectangle{\pgfqpoint{0.100000in}{0.212622in}}{\pgfqpoint{3.696000in}{3.696000in}}%
\pgfusepath{clip}%
\pgfsetrectcap%
\pgfsetroundjoin%
\pgfsetlinewidth{1.505625pt}%
\definecolor{currentstroke}{rgb}{1.000000,0.000000,0.000000}%
\pgfsetstrokecolor{currentstroke}%
\pgfsetdash{}{0pt}%
\pgfpathmoveto{\pgfqpoint{1.253940in}{1.673790in}}%
\pgfpathlineto{\pgfqpoint{1.206492in}{1.776421in}}%
\pgfusepath{stroke}%
\end{pgfscope}%
\begin{pgfscope}%
\pgfpathrectangle{\pgfqpoint{0.100000in}{0.212622in}}{\pgfqpoint{3.696000in}{3.696000in}}%
\pgfusepath{clip}%
\pgfsetrectcap%
\pgfsetroundjoin%
\pgfsetlinewidth{1.505625pt}%
\definecolor{currentstroke}{rgb}{1.000000,0.000000,0.000000}%
\pgfsetstrokecolor{currentstroke}%
\pgfsetdash{}{0pt}%
\pgfpathmoveto{\pgfqpoint{1.260556in}{1.678875in}}%
\pgfpathlineto{\pgfqpoint{1.206492in}{1.776421in}}%
\pgfusepath{stroke}%
\end{pgfscope}%
\begin{pgfscope}%
\pgfpathrectangle{\pgfqpoint{0.100000in}{0.212622in}}{\pgfqpoint{3.696000in}{3.696000in}}%
\pgfusepath{clip}%
\pgfsetrectcap%
\pgfsetroundjoin%
\pgfsetlinewidth{1.505625pt}%
\definecolor{currentstroke}{rgb}{1.000000,0.000000,0.000000}%
\pgfsetstrokecolor{currentstroke}%
\pgfsetdash{}{0pt}%
\pgfpathmoveto{\pgfqpoint{1.267610in}{1.674291in}}%
\pgfpathlineto{\pgfqpoint{1.216532in}{1.784866in}}%
\pgfusepath{stroke}%
\end{pgfscope}%
\begin{pgfscope}%
\pgfpathrectangle{\pgfqpoint{0.100000in}{0.212622in}}{\pgfqpoint{3.696000in}{3.696000in}}%
\pgfusepath{clip}%
\pgfsetrectcap%
\pgfsetroundjoin%
\pgfsetlinewidth{1.505625pt}%
\definecolor{currentstroke}{rgb}{1.000000,0.000000,0.000000}%
\pgfsetstrokecolor{currentstroke}%
\pgfsetdash{}{0pt}%
\pgfpathmoveto{\pgfqpoint{1.272074in}{1.677240in}}%
\pgfpathlineto{\pgfqpoint{1.216532in}{1.784866in}}%
\pgfusepath{stroke}%
\end{pgfscope}%
\begin{pgfscope}%
\pgfpathrectangle{\pgfqpoint{0.100000in}{0.212622in}}{\pgfqpoint{3.696000in}{3.696000in}}%
\pgfusepath{clip}%
\pgfsetrectcap%
\pgfsetroundjoin%
\pgfsetlinewidth{1.505625pt}%
\definecolor{currentstroke}{rgb}{1.000000,0.000000,0.000000}%
\pgfsetstrokecolor{currentstroke}%
\pgfsetdash{}{0pt}%
\pgfpathmoveto{\pgfqpoint{1.274449in}{1.679749in}}%
\pgfpathlineto{\pgfqpoint{1.216532in}{1.784866in}}%
\pgfusepath{stroke}%
\end{pgfscope}%
\begin{pgfscope}%
\pgfpathrectangle{\pgfqpoint{0.100000in}{0.212622in}}{\pgfqpoint{3.696000in}{3.696000in}}%
\pgfusepath{clip}%
\pgfsetrectcap%
\pgfsetroundjoin%
\pgfsetlinewidth{1.505625pt}%
\definecolor{currentstroke}{rgb}{1.000000,0.000000,0.000000}%
\pgfsetstrokecolor{currentstroke}%
\pgfsetdash{}{0pt}%
\pgfpathmoveto{\pgfqpoint{1.278507in}{1.681779in}}%
\pgfpathlineto{\pgfqpoint{1.226558in}{1.793299in}}%
\pgfusepath{stroke}%
\end{pgfscope}%
\begin{pgfscope}%
\pgfpathrectangle{\pgfqpoint{0.100000in}{0.212622in}}{\pgfqpoint{3.696000in}{3.696000in}}%
\pgfusepath{clip}%
\pgfsetrectcap%
\pgfsetroundjoin%
\pgfsetlinewidth{1.505625pt}%
\definecolor{currentstroke}{rgb}{1.000000,0.000000,0.000000}%
\pgfsetstrokecolor{currentstroke}%
\pgfsetdash{}{0pt}%
\pgfpathmoveto{\pgfqpoint{1.281266in}{1.681360in}}%
\pgfpathlineto{\pgfqpoint{1.226558in}{1.793299in}}%
\pgfusepath{stroke}%
\end{pgfscope}%
\begin{pgfscope}%
\pgfpathrectangle{\pgfqpoint{0.100000in}{0.212622in}}{\pgfqpoint{3.696000in}{3.696000in}}%
\pgfusepath{clip}%
\pgfsetrectcap%
\pgfsetroundjoin%
\pgfsetlinewidth{1.505625pt}%
\definecolor{currentstroke}{rgb}{1.000000,0.000000,0.000000}%
\pgfsetstrokecolor{currentstroke}%
\pgfsetdash{}{0pt}%
\pgfpathmoveto{\pgfqpoint{1.284484in}{1.684388in}}%
\pgfpathlineto{\pgfqpoint{1.226558in}{1.793299in}}%
\pgfusepath{stroke}%
\end{pgfscope}%
\begin{pgfscope}%
\pgfpathrectangle{\pgfqpoint{0.100000in}{0.212622in}}{\pgfqpoint{3.696000in}{3.696000in}}%
\pgfusepath{clip}%
\pgfsetrectcap%
\pgfsetroundjoin%
\pgfsetlinewidth{1.505625pt}%
\definecolor{currentstroke}{rgb}{1.000000,0.000000,0.000000}%
\pgfsetstrokecolor{currentstroke}%
\pgfsetdash{}{0pt}%
\pgfpathmoveto{\pgfqpoint{1.288532in}{1.684135in}}%
\pgfpathlineto{\pgfqpoint{1.226558in}{1.793299in}}%
\pgfusepath{stroke}%
\end{pgfscope}%
\begin{pgfscope}%
\pgfpathrectangle{\pgfqpoint{0.100000in}{0.212622in}}{\pgfqpoint{3.696000in}{3.696000in}}%
\pgfusepath{clip}%
\pgfsetrectcap%
\pgfsetroundjoin%
\pgfsetlinewidth{1.505625pt}%
\definecolor{currentstroke}{rgb}{1.000000,0.000000,0.000000}%
\pgfsetstrokecolor{currentstroke}%
\pgfsetdash{}{0pt}%
\pgfpathmoveto{\pgfqpoint{1.290449in}{1.683475in}}%
\pgfpathlineto{\pgfqpoint{1.236570in}{1.801721in}}%
\pgfusepath{stroke}%
\end{pgfscope}%
\begin{pgfscope}%
\pgfpathrectangle{\pgfqpoint{0.100000in}{0.212622in}}{\pgfqpoint{3.696000in}{3.696000in}}%
\pgfusepath{clip}%
\pgfsetrectcap%
\pgfsetroundjoin%
\pgfsetlinewidth{1.505625pt}%
\definecolor{currentstroke}{rgb}{1.000000,0.000000,0.000000}%
\pgfsetstrokecolor{currentstroke}%
\pgfsetdash{}{0pt}%
\pgfpathmoveto{\pgfqpoint{1.293779in}{1.685378in}}%
\pgfpathlineto{\pgfqpoint{1.236570in}{1.801721in}}%
\pgfusepath{stroke}%
\end{pgfscope}%
\begin{pgfscope}%
\pgfpathrectangle{\pgfqpoint{0.100000in}{0.212622in}}{\pgfqpoint{3.696000in}{3.696000in}}%
\pgfusepath{clip}%
\pgfsetrectcap%
\pgfsetroundjoin%
\pgfsetlinewidth{1.505625pt}%
\definecolor{currentstroke}{rgb}{1.000000,0.000000,0.000000}%
\pgfsetstrokecolor{currentstroke}%
\pgfsetdash{}{0pt}%
\pgfpathmoveto{\pgfqpoint{1.297372in}{1.688225in}}%
\pgfpathlineto{\pgfqpoint{1.236570in}{1.801721in}}%
\pgfusepath{stroke}%
\end{pgfscope}%
\begin{pgfscope}%
\pgfpathrectangle{\pgfqpoint{0.100000in}{0.212622in}}{\pgfqpoint{3.696000in}{3.696000in}}%
\pgfusepath{clip}%
\pgfsetrectcap%
\pgfsetroundjoin%
\pgfsetlinewidth{1.505625pt}%
\definecolor{currentstroke}{rgb}{1.000000,0.000000,0.000000}%
\pgfsetstrokecolor{currentstroke}%
\pgfsetdash{}{0pt}%
\pgfpathmoveto{\pgfqpoint{1.299861in}{1.690391in}}%
\pgfpathlineto{\pgfqpoint{1.236570in}{1.801721in}}%
\pgfusepath{stroke}%
\end{pgfscope}%
\begin{pgfscope}%
\pgfpathrectangle{\pgfqpoint{0.100000in}{0.212622in}}{\pgfqpoint{3.696000in}{3.696000in}}%
\pgfusepath{clip}%
\pgfsetrectcap%
\pgfsetroundjoin%
\pgfsetlinewidth{1.505625pt}%
\definecolor{currentstroke}{rgb}{1.000000,0.000000,0.000000}%
\pgfsetstrokecolor{currentstroke}%
\pgfsetdash{}{0pt}%
\pgfpathmoveto{\pgfqpoint{1.301384in}{1.691481in}}%
\pgfpathlineto{\pgfqpoint{1.246568in}{1.810130in}}%
\pgfusepath{stroke}%
\end{pgfscope}%
\begin{pgfscope}%
\pgfpathrectangle{\pgfqpoint{0.100000in}{0.212622in}}{\pgfqpoint{3.696000in}{3.696000in}}%
\pgfusepath{clip}%
\pgfsetrectcap%
\pgfsetroundjoin%
\pgfsetlinewidth{1.505625pt}%
\definecolor{currentstroke}{rgb}{1.000000,0.000000,0.000000}%
\pgfsetstrokecolor{currentstroke}%
\pgfsetdash{}{0pt}%
\pgfpathmoveto{\pgfqpoint{1.303826in}{1.691149in}}%
\pgfpathlineto{\pgfqpoint{1.246568in}{1.810130in}}%
\pgfusepath{stroke}%
\end{pgfscope}%
\begin{pgfscope}%
\pgfpathrectangle{\pgfqpoint{0.100000in}{0.212622in}}{\pgfqpoint{3.696000in}{3.696000in}}%
\pgfusepath{clip}%
\pgfsetrectcap%
\pgfsetroundjoin%
\pgfsetlinewidth{1.505625pt}%
\definecolor{currentstroke}{rgb}{1.000000,0.000000,0.000000}%
\pgfsetstrokecolor{currentstroke}%
\pgfsetdash{}{0pt}%
\pgfpathmoveto{\pgfqpoint{1.305514in}{1.692782in}}%
\pgfpathlineto{\pgfqpoint{1.246568in}{1.810130in}}%
\pgfusepath{stroke}%
\end{pgfscope}%
\begin{pgfscope}%
\pgfpathrectangle{\pgfqpoint{0.100000in}{0.212622in}}{\pgfqpoint{3.696000in}{3.696000in}}%
\pgfusepath{clip}%
\pgfsetrectcap%
\pgfsetroundjoin%
\pgfsetlinewidth{1.505625pt}%
\definecolor{currentstroke}{rgb}{1.000000,0.000000,0.000000}%
\pgfsetstrokecolor{currentstroke}%
\pgfsetdash{}{0pt}%
\pgfpathmoveto{\pgfqpoint{1.307688in}{1.695333in}}%
\pgfpathlineto{\pgfqpoint{1.246568in}{1.810130in}}%
\pgfusepath{stroke}%
\end{pgfscope}%
\begin{pgfscope}%
\pgfpathrectangle{\pgfqpoint{0.100000in}{0.212622in}}{\pgfqpoint{3.696000in}{3.696000in}}%
\pgfusepath{clip}%
\pgfsetrectcap%
\pgfsetroundjoin%
\pgfsetlinewidth{1.505625pt}%
\definecolor{currentstroke}{rgb}{1.000000,0.000000,0.000000}%
\pgfsetstrokecolor{currentstroke}%
\pgfsetdash{}{0pt}%
\pgfpathmoveto{\pgfqpoint{1.310478in}{1.694379in}}%
\pgfpathlineto{\pgfqpoint{1.246568in}{1.810130in}}%
\pgfusepath{stroke}%
\end{pgfscope}%
\begin{pgfscope}%
\pgfpathrectangle{\pgfqpoint{0.100000in}{0.212622in}}{\pgfqpoint{3.696000in}{3.696000in}}%
\pgfusepath{clip}%
\pgfsetrectcap%
\pgfsetroundjoin%
\pgfsetlinewidth{1.505625pt}%
\definecolor{currentstroke}{rgb}{1.000000,0.000000,0.000000}%
\pgfsetstrokecolor{currentstroke}%
\pgfsetdash{}{0pt}%
\pgfpathmoveto{\pgfqpoint{1.314887in}{1.696686in}}%
\pgfpathlineto{\pgfqpoint{1.256551in}{1.818527in}}%
\pgfusepath{stroke}%
\end{pgfscope}%
\begin{pgfscope}%
\pgfpathrectangle{\pgfqpoint{0.100000in}{0.212622in}}{\pgfqpoint{3.696000in}{3.696000in}}%
\pgfusepath{clip}%
\pgfsetrectcap%
\pgfsetroundjoin%
\pgfsetlinewidth{1.505625pt}%
\definecolor{currentstroke}{rgb}{1.000000,0.000000,0.000000}%
\pgfsetstrokecolor{currentstroke}%
\pgfsetdash{}{0pt}%
\pgfpathmoveto{\pgfqpoint{1.319578in}{1.701712in}}%
\pgfpathlineto{\pgfqpoint{1.256551in}{1.818527in}}%
\pgfusepath{stroke}%
\end{pgfscope}%
\begin{pgfscope}%
\pgfpathrectangle{\pgfqpoint{0.100000in}{0.212622in}}{\pgfqpoint{3.696000in}{3.696000in}}%
\pgfusepath{clip}%
\pgfsetrectcap%
\pgfsetroundjoin%
\pgfsetlinewidth{1.505625pt}%
\definecolor{currentstroke}{rgb}{1.000000,0.000000,0.000000}%
\pgfsetstrokecolor{currentstroke}%
\pgfsetdash{}{0pt}%
\pgfpathmoveto{\pgfqpoint{1.323553in}{1.697779in}}%
\pgfpathlineto{\pgfqpoint{1.266521in}{1.826913in}}%
\pgfusepath{stroke}%
\end{pgfscope}%
\begin{pgfscope}%
\pgfpathrectangle{\pgfqpoint{0.100000in}{0.212622in}}{\pgfqpoint{3.696000in}{3.696000in}}%
\pgfusepath{clip}%
\pgfsetrectcap%
\pgfsetroundjoin%
\pgfsetlinewidth{1.505625pt}%
\definecolor{currentstroke}{rgb}{1.000000,0.000000,0.000000}%
\pgfsetstrokecolor{currentstroke}%
\pgfsetdash{}{0pt}%
\pgfpathmoveto{\pgfqpoint{1.329480in}{1.701934in}}%
\pgfpathlineto{\pgfqpoint{1.266521in}{1.826913in}}%
\pgfusepath{stroke}%
\end{pgfscope}%
\begin{pgfscope}%
\pgfpathrectangle{\pgfqpoint{0.100000in}{0.212622in}}{\pgfqpoint{3.696000in}{3.696000in}}%
\pgfusepath{clip}%
\pgfsetrectcap%
\pgfsetroundjoin%
\pgfsetlinewidth{1.505625pt}%
\definecolor{currentstroke}{rgb}{1.000000,0.000000,0.000000}%
\pgfsetstrokecolor{currentstroke}%
\pgfsetdash{}{0pt}%
\pgfpathmoveto{\pgfqpoint{1.335112in}{1.703112in}}%
\pgfpathlineto{\pgfqpoint{1.266521in}{1.826913in}}%
\pgfusepath{stroke}%
\end{pgfscope}%
\begin{pgfscope}%
\pgfpathrectangle{\pgfqpoint{0.100000in}{0.212622in}}{\pgfqpoint{3.696000in}{3.696000in}}%
\pgfusepath{clip}%
\pgfsetrectcap%
\pgfsetroundjoin%
\pgfsetlinewidth{1.505625pt}%
\definecolor{currentstroke}{rgb}{1.000000,0.000000,0.000000}%
\pgfsetstrokecolor{currentstroke}%
\pgfsetdash{}{0pt}%
\pgfpathmoveto{\pgfqpoint{1.338103in}{1.704216in}}%
\pgfpathlineto{\pgfqpoint{1.276476in}{1.835286in}}%
\pgfusepath{stroke}%
\end{pgfscope}%
\begin{pgfscope}%
\pgfpathrectangle{\pgfqpoint{0.100000in}{0.212622in}}{\pgfqpoint{3.696000in}{3.696000in}}%
\pgfusepath{clip}%
\pgfsetrectcap%
\pgfsetroundjoin%
\pgfsetlinewidth{1.505625pt}%
\definecolor{currentstroke}{rgb}{1.000000,0.000000,0.000000}%
\pgfsetstrokecolor{currentstroke}%
\pgfsetdash{}{0pt}%
\pgfpathmoveto{\pgfqpoint{1.342535in}{1.704747in}}%
\pgfpathlineto{\pgfqpoint{1.276476in}{1.835286in}}%
\pgfusepath{stroke}%
\end{pgfscope}%
\begin{pgfscope}%
\pgfpathrectangle{\pgfqpoint{0.100000in}{0.212622in}}{\pgfqpoint{3.696000in}{3.696000in}}%
\pgfusepath{clip}%
\pgfsetrectcap%
\pgfsetroundjoin%
\pgfsetlinewidth{1.505625pt}%
\definecolor{currentstroke}{rgb}{1.000000,0.000000,0.000000}%
\pgfsetstrokecolor{currentstroke}%
\pgfsetdash{}{0pt}%
\pgfpathmoveto{\pgfqpoint{1.347300in}{1.705976in}}%
\pgfpathlineto{\pgfqpoint{1.286417in}{1.843648in}}%
\pgfusepath{stroke}%
\end{pgfscope}%
\begin{pgfscope}%
\pgfpathrectangle{\pgfqpoint{0.100000in}{0.212622in}}{\pgfqpoint{3.696000in}{3.696000in}}%
\pgfusepath{clip}%
\pgfsetrectcap%
\pgfsetroundjoin%
\pgfsetlinewidth{1.505625pt}%
\definecolor{currentstroke}{rgb}{1.000000,0.000000,0.000000}%
\pgfsetstrokecolor{currentstroke}%
\pgfsetdash{}{0pt}%
\pgfpathmoveto{\pgfqpoint{1.350118in}{1.706833in}}%
\pgfpathlineto{\pgfqpoint{1.286417in}{1.843648in}}%
\pgfusepath{stroke}%
\end{pgfscope}%
\begin{pgfscope}%
\pgfpathrectangle{\pgfqpoint{0.100000in}{0.212622in}}{\pgfqpoint{3.696000in}{3.696000in}}%
\pgfusepath{clip}%
\pgfsetrectcap%
\pgfsetroundjoin%
\pgfsetlinewidth{1.505625pt}%
\definecolor{currentstroke}{rgb}{1.000000,0.000000,0.000000}%
\pgfsetstrokecolor{currentstroke}%
\pgfsetdash{}{0pt}%
\pgfpathmoveto{\pgfqpoint{1.353536in}{1.708615in}}%
\pgfpathlineto{\pgfqpoint{1.286417in}{1.843648in}}%
\pgfusepath{stroke}%
\end{pgfscope}%
\begin{pgfscope}%
\pgfpathrectangle{\pgfqpoint{0.100000in}{0.212622in}}{\pgfqpoint{3.696000in}{3.696000in}}%
\pgfusepath{clip}%
\pgfsetrectcap%
\pgfsetroundjoin%
\pgfsetlinewidth{1.505625pt}%
\definecolor{currentstroke}{rgb}{1.000000,0.000000,0.000000}%
\pgfsetstrokecolor{currentstroke}%
\pgfsetdash{}{0pt}%
\pgfpathmoveto{\pgfqpoint{1.357736in}{1.708565in}}%
\pgfpathlineto{\pgfqpoint{1.286417in}{1.843648in}}%
\pgfusepath{stroke}%
\end{pgfscope}%
\begin{pgfscope}%
\pgfpathrectangle{\pgfqpoint{0.100000in}{0.212622in}}{\pgfqpoint{3.696000in}{3.696000in}}%
\pgfusepath{clip}%
\pgfsetrectcap%
\pgfsetroundjoin%
\pgfsetlinewidth{1.505625pt}%
\definecolor{currentstroke}{rgb}{1.000000,0.000000,0.000000}%
\pgfsetstrokecolor{currentstroke}%
\pgfsetdash{}{0pt}%
\pgfpathmoveto{\pgfqpoint{1.362916in}{1.715015in}}%
\pgfpathlineto{\pgfqpoint{1.296344in}{1.851998in}}%
\pgfusepath{stroke}%
\end{pgfscope}%
\begin{pgfscope}%
\pgfpathrectangle{\pgfqpoint{0.100000in}{0.212622in}}{\pgfqpoint{3.696000in}{3.696000in}}%
\pgfusepath{clip}%
\pgfsetrectcap%
\pgfsetroundjoin%
\pgfsetlinewidth{1.505625pt}%
\definecolor{currentstroke}{rgb}{1.000000,0.000000,0.000000}%
\pgfsetstrokecolor{currentstroke}%
\pgfsetdash{}{0pt}%
\pgfpathmoveto{\pgfqpoint{1.365910in}{1.717494in}}%
\pgfpathlineto{\pgfqpoint{1.296344in}{1.851998in}}%
\pgfusepath{stroke}%
\end{pgfscope}%
\begin{pgfscope}%
\pgfpathrectangle{\pgfqpoint{0.100000in}{0.212622in}}{\pgfqpoint{3.696000in}{3.696000in}}%
\pgfusepath{clip}%
\pgfsetrectcap%
\pgfsetroundjoin%
\pgfsetlinewidth{1.505625pt}%
\definecolor{currentstroke}{rgb}{1.000000,0.000000,0.000000}%
\pgfsetstrokecolor{currentstroke}%
\pgfsetdash{}{0pt}%
\pgfpathmoveto{\pgfqpoint{1.368447in}{1.715973in}}%
\pgfpathlineto{\pgfqpoint{1.296344in}{1.851998in}}%
\pgfusepath{stroke}%
\end{pgfscope}%
\begin{pgfscope}%
\pgfpathrectangle{\pgfqpoint{0.100000in}{0.212622in}}{\pgfqpoint{3.696000in}{3.696000in}}%
\pgfusepath{clip}%
\pgfsetrectcap%
\pgfsetroundjoin%
\pgfsetlinewidth{1.505625pt}%
\definecolor{currentstroke}{rgb}{1.000000,0.000000,0.000000}%
\pgfsetstrokecolor{currentstroke}%
\pgfsetdash{}{0pt}%
\pgfpathmoveto{\pgfqpoint{1.373928in}{1.726175in}}%
\pgfpathlineto{\pgfqpoint{1.306257in}{1.860336in}}%
\pgfusepath{stroke}%
\end{pgfscope}%
\begin{pgfscope}%
\pgfpathrectangle{\pgfqpoint{0.100000in}{0.212622in}}{\pgfqpoint{3.696000in}{3.696000in}}%
\pgfusepath{clip}%
\pgfsetrectcap%
\pgfsetroundjoin%
\pgfsetlinewidth{1.505625pt}%
\definecolor{currentstroke}{rgb}{1.000000,0.000000,0.000000}%
\pgfsetstrokecolor{currentstroke}%
\pgfsetdash{}{0pt}%
\pgfpathmoveto{\pgfqpoint{1.376874in}{1.728308in}}%
\pgfpathlineto{\pgfqpoint{1.306257in}{1.860336in}}%
\pgfusepath{stroke}%
\end{pgfscope}%
\begin{pgfscope}%
\pgfpathrectangle{\pgfqpoint{0.100000in}{0.212622in}}{\pgfqpoint{3.696000in}{3.696000in}}%
\pgfusepath{clip}%
\pgfsetrectcap%
\pgfsetroundjoin%
\pgfsetlinewidth{1.505625pt}%
\definecolor{currentstroke}{rgb}{1.000000,0.000000,0.000000}%
\pgfsetstrokecolor{currentstroke}%
\pgfsetdash{}{0pt}%
\pgfpathmoveto{\pgfqpoint{1.378015in}{1.729378in}}%
\pgfpathlineto{\pgfqpoint{1.306257in}{1.860336in}}%
\pgfusepath{stroke}%
\end{pgfscope}%
\begin{pgfscope}%
\pgfpathrectangle{\pgfqpoint{0.100000in}{0.212622in}}{\pgfqpoint{3.696000in}{3.696000in}}%
\pgfusepath{clip}%
\pgfsetrectcap%
\pgfsetroundjoin%
\pgfsetlinewidth{1.505625pt}%
\definecolor{currentstroke}{rgb}{1.000000,0.000000,0.000000}%
\pgfsetstrokecolor{currentstroke}%
\pgfsetdash{}{0pt}%
\pgfpathmoveto{\pgfqpoint{1.380227in}{1.730306in}}%
\pgfpathlineto{\pgfqpoint{1.306257in}{1.860336in}}%
\pgfusepath{stroke}%
\end{pgfscope}%
\begin{pgfscope}%
\pgfpathrectangle{\pgfqpoint{0.100000in}{0.212622in}}{\pgfqpoint{3.696000in}{3.696000in}}%
\pgfusepath{clip}%
\pgfsetrectcap%
\pgfsetroundjoin%
\pgfsetlinewidth{1.505625pt}%
\definecolor{currentstroke}{rgb}{1.000000,0.000000,0.000000}%
\pgfsetstrokecolor{currentstroke}%
\pgfsetdash{}{0pt}%
\pgfpathmoveto{\pgfqpoint{1.384420in}{1.736376in}}%
\pgfpathlineto{\pgfqpoint{1.306257in}{1.860336in}}%
\pgfusepath{stroke}%
\end{pgfscope}%
\begin{pgfscope}%
\pgfpathrectangle{\pgfqpoint{0.100000in}{0.212622in}}{\pgfqpoint{3.696000in}{3.696000in}}%
\pgfusepath{clip}%
\pgfsetrectcap%
\pgfsetroundjoin%
\pgfsetlinewidth{1.505625pt}%
\definecolor{currentstroke}{rgb}{1.000000,0.000000,0.000000}%
\pgfsetstrokecolor{currentstroke}%
\pgfsetdash{}{0pt}%
\pgfpathmoveto{\pgfqpoint{1.389625in}{1.737805in}}%
\pgfpathlineto{\pgfqpoint{1.316156in}{1.868662in}}%
\pgfusepath{stroke}%
\end{pgfscope}%
\begin{pgfscope}%
\pgfpathrectangle{\pgfqpoint{0.100000in}{0.212622in}}{\pgfqpoint{3.696000in}{3.696000in}}%
\pgfusepath{clip}%
\pgfsetrectcap%
\pgfsetroundjoin%
\pgfsetlinewidth{1.505625pt}%
\definecolor{currentstroke}{rgb}{1.000000,0.000000,0.000000}%
\pgfsetstrokecolor{currentstroke}%
\pgfsetdash{}{0pt}%
\pgfpathmoveto{\pgfqpoint{1.392939in}{1.738734in}}%
\pgfpathlineto{\pgfqpoint{1.316156in}{1.868662in}}%
\pgfusepath{stroke}%
\end{pgfscope}%
\begin{pgfscope}%
\pgfpathrectangle{\pgfqpoint{0.100000in}{0.212622in}}{\pgfqpoint{3.696000in}{3.696000in}}%
\pgfusepath{clip}%
\pgfsetrectcap%
\pgfsetroundjoin%
\pgfsetlinewidth{1.505625pt}%
\definecolor{currentstroke}{rgb}{1.000000,0.000000,0.000000}%
\pgfsetstrokecolor{currentstroke}%
\pgfsetdash{}{0pt}%
\pgfpathmoveto{\pgfqpoint{1.394946in}{1.740102in}}%
\pgfpathlineto{\pgfqpoint{1.316156in}{1.868662in}}%
\pgfusepath{stroke}%
\end{pgfscope}%
\begin{pgfscope}%
\pgfpathrectangle{\pgfqpoint{0.100000in}{0.212622in}}{\pgfqpoint{3.696000in}{3.696000in}}%
\pgfusepath{clip}%
\pgfsetrectcap%
\pgfsetroundjoin%
\pgfsetlinewidth{1.505625pt}%
\definecolor{currentstroke}{rgb}{1.000000,0.000000,0.000000}%
\pgfsetstrokecolor{currentstroke}%
\pgfsetdash{}{0pt}%
\pgfpathmoveto{\pgfqpoint{1.397559in}{1.740844in}}%
\pgfpathlineto{\pgfqpoint{1.326041in}{1.876977in}}%
\pgfusepath{stroke}%
\end{pgfscope}%
\begin{pgfscope}%
\pgfpathrectangle{\pgfqpoint{0.100000in}{0.212622in}}{\pgfqpoint{3.696000in}{3.696000in}}%
\pgfusepath{clip}%
\pgfsetrectcap%
\pgfsetroundjoin%
\pgfsetlinewidth{1.505625pt}%
\definecolor{currentstroke}{rgb}{1.000000,0.000000,0.000000}%
\pgfsetstrokecolor{currentstroke}%
\pgfsetdash{}{0pt}%
\pgfpathmoveto{\pgfqpoint{1.399313in}{1.742240in}}%
\pgfpathlineto{\pgfqpoint{1.326041in}{1.876977in}}%
\pgfusepath{stroke}%
\end{pgfscope}%
\begin{pgfscope}%
\pgfpathrectangle{\pgfqpoint{0.100000in}{0.212622in}}{\pgfqpoint{3.696000in}{3.696000in}}%
\pgfusepath{clip}%
\pgfsetrectcap%
\pgfsetroundjoin%
\pgfsetlinewidth{1.505625pt}%
\definecolor{currentstroke}{rgb}{1.000000,0.000000,0.000000}%
\pgfsetstrokecolor{currentstroke}%
\pgfsetdash{}{0pt}%
\pgfpathmoveto{\pgfqpoint{1.401991in}{1.743969in}}%
\pgfpathlineto{\pgfqpoint{1.326041in}{1.876977in}}%
\pgfusepath{stroke}%
\end{pgfscope}%
\begin{pgfscope}%
\pgfpathrectangle{\pgfqpoint{0.100000in}{0.212622in}}{\pgfqpoint{3.696000in}{3.696000in}}%
\pgfusepath{clip}%
\pgfsetrectcap%
\pgfsetroundjoin%
\pgfsetlinewidth{1.505625pt}%
\definecolor{currentstroke}{rgb}{1.000000,0.000000,0.000000}%
\pgfsetstrokecolor{currentstroke}%
\pgfsetdash{}{0pt}%
\pgfpathmoveto{\pgfqpoint{1.404761in}{1.743137in}}%
\pgfpathlineto{\pgfqpoint{1.326041in}{1.876977in}}%
\pgfusepath{stroke}%
\end{pgfscope}%
\begin{pgfscope}%
\pgfpathrectangle{\pgfqpoint{0.100000in}{0.212622in}}{\pgfqpoint{3.696000in}{3.696000in}}%
\pgfusepath{clip}%
\pgfsetrectcap%
\pgfsetroundjoin%
\pgfsetlinewidth{1.505625pt}%
\definecolor{currentstroke}{rgb}{1.000000,0.000000,0.000000}%
\pgfsetstrokecolor{currentstroke}%
\pgfsetdash{}{0pt}%
\pgfpathmoveto{\pgfqpoint{1.406822in}{1.743778in}}%
\pgfpathlineto{\pgfqpoint{1.326041in}{1.876977in}}%
\pgfusepath{stroke}%
\end{pgfscope}%
\begin{pgfscope}%
\pgfpathrectangle{\pgfqpoint{0.100000in}{0.212622in}}{\pgfqpoint{3.696000in}{3.696000in}}%
\pgfusepath{clip}%
\pgfsetrectcap%
\pgfsetroundjoin%
\pgfsetlinewidth{1.505625pt}%
\definecolor{currentstroke}{rgb}{1.000000,0.000000,0.000000}%
\pgfsetstrokecolor{currentstroke}%
\pgfsetdash{}{0pt}%
\pgfpathmoveto{\pgfqpoint{1.409269in}{1.745966in}}%
\pgfpathlineto{\pgfqpoint{1.335912in}{1.885280in}}%
\pgfusepath{stroke}%
\end{pgfscope}%
\begin{pgfscope}%
\pgfpathrectangle{\pgfqpoint{0.100000in}{0.212622in}}{\pgfqpoint{3.696000in}{3.696000in}}%
\pgfusepath{clip}%
\pgfsetrectcap%
\pgfsetroundjoin%
\pgfsetlinewidth{1.505625pt}%
\definecolor{currentstroke}{rgb}{1.000000,0.000000,0.000000}%
\pgfsetstrokecolor{currentstroke}%
\pgfsetdash{}{0pt}%
\pgfpathmoveto{\pgfqpoint{1.412336in}{1.744914in}}%
\pgfpathlineto{\pgfqpoint{1.335912in}{1.885280in}}%
\pgfusepath{stroke}%
\end{pgfscope}%
\begin{pgfscope}%
\pgfpathrectangle{\pgfqpoint{0.100000in}{0.212622in}}{\pgfqpoint{3.696000in}{3.696000in}}%
\pgfusepath{clip}%
\pgfsetrectcap%
\pgfsetroundjoin%
\pgfsetlinewidth{1.505625pt}%
\definecolor{currentstroke}{rgb}{1.000000,0.000000,0.000000}%
\pgfsetstrokecolor{currentstroke}%
\pgfsetdash{}{0pt}%
\pgfpathmoveto{\pgfqpoint{1.414185in}{1.746449in}}%
\pgfpathlineto{\pgfqpoint{1.335912in}{1.885280in}}%
\pgfusepath{stroke}%
\end{pgfscope}%
\begin{pgfscope}%
\pgfpathrectangle{\pgfqpoint{0.100000in}{0.212622in}}{\pgfqpoint{3.696000in}{3.696000in}}%
\pgfusepath{clip}%
\pgfsetrectcap%
\pgfsetroundjoin%
\pgfsetlinewidth{1.505625pt}%
\definecolor{currentstroke}{rgb}{1.000000,0.000000,0.000000}%
\pgfsetstrokecolor{currentstroke}%
\pgfsetdash{}{0pt}%
\pgfpathmoveto{\pgfqpoint{1.416841in}{1.746536in}}%
\pgfpathlineto{\pgfqpoint{1.335912in}{1.885280in}}%
\pgfusepath{stroke}%
\end{pgfscope}%
\begin{pgfscope}%
\pgfpathrectangle{\pgfqpoint{0.100000in}{0.212622in}}{\pgfqpoint{3.696000in}{3.696000in}}%
\pgfusepath{clip}%
\pgfsetrectcap%
\pgfsetroundjoin%
\pgfsetlinewidth{1.505625pt}%
\definecolor{currentstroke}{rgb}{1.000000,0.000000,0.000000}%
\pgfsetstrokecolor{currentstroke}%
\pgfsetdash{}{0pt}%
\pgfpathmoveto{\pgfqpoint{1.420278in}{1.747081in}}%
\pgfpathlineto{\pgfqpoint{1.345769in}{1.893571in}}%
\pgfusepath{stroke}%
\end{pgfscope}%
\begin{pgfscope}%
\pgfpathrectangle{\pgfqpoint{0.100000in}{0.212622in}}{\pgfqpoint{3.696000in}{3.696000in}}%
\pgfusepath{clip}%
\pgfsetrectcap%
\pgfsetroundjoin%
\pgfsetlinewidth{1.505625pt}%
\definecolor{currentstroke}{rgb}{1.000000,0.000000,0.000000}%
\pgfsetstrokecolor{currentstroke}%
\pgfsetdash{}{0pt}%
\pgfpathmoveto{\pgfqpoint{1.422395in}{1.748991in}}%
\pgfpathlineto{\pgfqpoint{1.345769in}{1.893571in}}%
\pgfusepath{stroke}%
\end{pgfscope}%
\begin{pgfscope}%
\pgfpathrectangle{\pgfqpoint{0.100000in}{0.212622in}}{\pgfqpoint{3.696000in}{3.696000in}}%
\pgfusepath{clip}%
\pgfsetrectcap%
\pgfsetroundjoin%
\pgfsetlinewidth{1.505625pt}%
\definecolor{currentstroke}{rgb}{1.000000,0.000000,0.000000}%
\pgfsetstrokecolor{currentstroke}%
\pgfsetdash{}{0pt}%
\pgfpathmoveto{\pgfqpoint{1.427038in}{1.747909in}}%
\pgfpathlineto{\pgfqpoint{1.345769in}{1.893571in}}%
\pgfusepath{stroke}%
\end{pgfscope}%
\begin{pgfscope}%
\pgfpathrectangle{\pgfqpoint{0.100000in}{0.212622in}}{\pgfqpoint{3.696000in}{3.696000in}}%
\pgfusepath{clip}%
\pgfsetrectcap%
\pgfsetroundjoin%
\pgfsetlinewidth{1.505625pt}%
\definecolor{currentstroke}{rgb}{1.000000,0.000000,0.000000}%
\pgfsetstrokecolor{currentstroke}%
\pgfsetdash{}{0pt}%
\pgfpathmoveto{\pgfqpoint{1.431530in}{1.747713in}}%
\pgfpathlineto{\pgfqpoint{1.355612in}{1.901850in}}%
\pgfusepath{stroke}%
\end{pgfscope}%
\begin{pgfscope}%
\pgfpathrectangle{\pgfqpoint{0.100000in}{0.212622in}}{\pgfqpoint{3.696000in}{3.696000in}}%
\pgfusepath{clip}%
\pgfsetrectcap%
\pgfsetroundjoin%
\pgfsetlinewidth{1.505625pt}%
\definecolor{currentstroke}{rgb}{1.000000,0.000000,0.000000}%
\pgfsetstrokecolor{currentstroke}%
\pgfsetdash{}{0pt}%
\pgfpathmoveto{\pgfqpoint{1.434682in}{1.749776in}}%
\pgfpathlineto{\pgfqpoint{1.355612in}{1.901850in}}%
\pgfusepath{stroke}%
\end{pgfscope}%
\begin{pgfscope}%
\pgfpathrectangle{\pgfqpoint{0.100000in}{0.212622in}}{\pgfqpoint{3.696000in}{3.696000in}}%
\pgfusepath{clip}%
\pgfsetrectcap%
\pgfsetroundjoin%
\pgfsetlinewidth{1.505625pt}%
\definecolor{currentstroke}{rgb}{1.000000,0.000000,0.000000}%
\pgfsetstrokecolor{currentstroke}%
\pgfsetdash{}{0pt}%
\pgfpathmoveto{\pgfqpoint{1.438406in}{1.752548in}}%
\pgfpathlineto{\pgfqpoint{1.355612in}{1.901850in}}%
\pgfusepath{stroke}%
\end{pgfscope}%
\begin{pgfscope}%
\pgfpathrectangle{\pgfqpoint{0.100000in}{0.212622in}}{\pgfqpoint{3.696000in}{3.696000in}}%
\pgfusepath{clip}%
\pgfsetrectcap%
\pgfsetroundjoin%
\pgfsetlinewidth{1.505625pt}%
\definecolor{currentstroke}{rgb}{1.000000,0.000000,0.000000}%
\pgfsetstrokecolor{currentstroke}%
\pgfsetdash{}{0pt}%
\pgfpathmoveto{\pgfqpoint{1.442558in}{1.754311in}}%
\pgfpathlineto{\pgfqpoint{1.355612in}{1.901850in}}%
\pgfusepath{stroke}%
\end{pgfscope}%
\begin{pgfscope}%
\pgfpathrectangle{\pgfqpoint{0.100000in}{0.212622in}}{\pgfqpoint{3.696000in}{3.696000in}}%
\pgfusepath{clip}%
\pgfsetrectcap%
\pgfsetroundjoin%
\pgfsetlinewidth{1.505625pt}%
\definecolor{currentstroke}{rgb}{1.000000,0.000000,0.000000}%
\pgfsetstrokecolor{currentstroke}%
\pgfsetdash{}{0pt}%
\pgfpathmoveto{\pgfqpoint{1.445343in}{1.755281in}}%
\pgfpathlineto{\pgfqpoint{1.365441in}{1.910118in}}%
\pgfusepath{stroke}%
\end{pgfscope}%
\begin{pgfscope}%
\pgfpathrectangle{\pgfqpoint{0.100000in}{0.212622in}}{\pgfqpoint{3.696000in}{3.696000in}}%
\pgfusepath{clip}%
\pgfsetrectcap%
\pgfsetroundjoin%
\pgfsetlinewidth{1.505625pt}%
\definecolor{currentstroke}{rgb}{1.000000,0.000000,0.000000}%
\pgfsetstrokecolor{currentstroke}%
\pgfsetdash{}{0pt}%
\pgfpathmoveto{\pgfqpoint{1.448548in}{1.758054in}}%
\pgfpathlineto{\pgfqpoint{1.365441in}{1.910118in}}%
\pgfusepath{stroke}%
\end{pgfscope}%
\begin{pgfscope}%
\pgfpathrectangle{\pgfqpoint{0.100000in}{0.212622in}}{\pgfqpoint{3.696000in}{3.696000in}}%
\pgfusepath{clip}%
\pgfsetrectcap%
\pgfsetroundjoin%
\pgfsetlinewidth{1.505625pt}%
\definecolor{currentstroke}{rgb}{1.000000,0.000000,0.000000}%
\pgfsetstrokecolor{currentstroke}%
\pgfsetdash{}{0pt}%
\pgfpathmoveto{\pgfqpoint{1.452614in}{1.760536in}}%
\pgfpathlineto{\pgfqpoint{1.365441in}{1.910118in}}%
\pgfusepath{stroke}%
\end{pgfscope}%
\begin{pgfscope}%
\pgfpathrectangle{\pgfqpoint{0.100000in}{0.212622in}}{\pgfqpoint{3.696000in}{3.696000in}}%
\pgfusepath{clip}%
\pgfsetrectcap%
\pgfsetroundjoin%
\pgfsetlinewidth{1.505625pt}%
\definecolor{currentstroke}{rgb}{1.000000,0.000000,0.000000}%
\pgfsetstrokecolor{currentstroke}%
\pgfsetdash{}{0pt}%
\pgfpathmoveto{\pgfqpoint{1.455243in}{1.762720in}}%
\pgfpathlineto{\pgfqpoint{1.375257in}{1.918374in}}%
\pgfusepath{stroke}%
\end{pgfscope}%
\begin{pgfscope}%
\pgfpathrectangle{\pgfqpoint{0.100000in}{0.212622in}}{\pgfqpoint{3.696000in}{3.696000in}}%
\pgfusepath{clip}%
\pgfsetrectcap%
\pgfsetroundjoin%
\pgfsetlinewidth{1.505625pt}%
\definecolor{currentstroke}{rgb}{1.000000,0.000000,0.000000}%
\pgfsetstrokecolor{currentstroke}%
\pgfsetdash{}{0pt}%
\pgfpathmoveto{\pgfqpoint{1.458321in}{1.764442in}}%
\pgfpathlineto{\pgfqpoint{1.375257in}{1.918374in}}%
\pgfusepath{stroke}%
\end{pgfscope}%
\begin{pgfscope}%
\pgfpathrectangle{\pgfqpoint{0.100000in}{0.212622in}}{\pgfqpoint{3.696000in}{3.696000in}}%
\pgfusepath{clip}%
\pgfsetrectcap%
\pgfsetroundjoin%
\pgfsetlinewidth{1.505625pt}%
\definecolor{currentstroke}{rgb}{1.000000,0.000000,0.000000}%
\pgfsetstrokecolor{currentstroke}%
\pgfsetdash{}{0pt}%
\pgfpathmoveto{\pgfqpoint{1.462967in}{1.763707in}}%
\pgfpathlineto{\pgfqpoint{1.375257in}{1.918374in}}%
\pgfusepath{stroke}%
\end{pgfscope}%
\begin{pgfscope}%
\pgfpathrectangle{\pgfqpoint{0.100000in}{0.212622in}}{\pgfqpoint{3.696000in}{3.696000in}}%
\pgfusepath{clip}%
\pgfsetrectcap%
\pgfsetroundjoin%
\pgfsetlinewidth{1.505625pt}%
\definecolor{currentstroke}{rgb}{1.000000,0.000000,0.000000}%
\pgfsetstrokecolor{currentstroke}%
\pgfsetdash{}{0pt}%
\pgfpathmoveto{\pgfqpoint{1.465771in}{1.765305in}}%
\pgfpathlineto{\pgfqpoint{1.375257in}{1.918374in}}%
\pgfusepath{stroke}%
\end{pgfscope}%
\begin{pgfscope}%
\pgfpathrectangle{\pgfqpoint{0.100000in}{0.212622in}}{\pgfqpoint{3.696000in}{3.696000in}}%
\pgfusepath{clip}%
\pgfsetrectcap%
\pgfsetroundjoin%
\pgfsetlinewidth{1.505625pt}%
\definecolor{currentstroke}{rgb}{1.000000,0.000000,0.000000}%
\pgfsetstrokecolor{currentstroke}%
\pgfsetdash{}{0pt}%
\pgfpathmoveto{\pgfqpoint{1.467369in}{1.766048in}}%
\pgfpathlineto{\pgfqpoint{1.375257in}{1.918374in}}%
\pgfusepath{stroke}%
\end{pgfscope}%
\begin{pgfscope}%
\pgfpathrectangle{\pgfqpoint{0.100000in}{0.212622in}}{\pgfqpoint{3.696000in}{3.696000in}}%
\pgfusepath{clip}%
\pgfsetrectcap%
\pgfsetroundjoin%
\pgfsetlinewidth{1.505625pt}%
\definecolor{currentstroke}{rgb}{1.000000,0.000000,0.000000}%
\pgfsetstrokecolor{currentstroke}%
\pgfsetdash{}{0pt}%
\pgfpathmoveto{\pgfqpoint{1.470443in}{1.766865in}}%
\pgfpathlineto{\pgfqpoint{1.385058in}{1.926619in}}%
\pgfusepath{stroke}%
\end{pgfscope}%
\begin{pgfscope}%
\pgfpathrectangle{\pgfqpoint{0.100000in}{0.212622in}}{\pgfqpoint{3.696000in}{3.696000in}}%
\pgfusepath{clip}%
\pgfsetrectcap%
\pgfsetroundjoin%
\pgfsetlinewidth{1.505625pt}%
\definecolor{currentstroke}{rgb}{1.000000,0.000000,0.000000}%
\pgfsetstrokecolor{currentstroke}%
\pgfsetdash{}{0pt}%
\pgfpathmoveto{\pgfqpoint{1.473505in}{1.766376in}}%
\pgfpathlineto{\pgfqpoint{1.385058in}{1.926619in}}%
\pgfusepath{stroke}%
\end{pgfscope}%
\begin{pgfscope}%
\pgfpathrectangle{\pgfqpoint{0.100000in}{0.212622in}}{\pgfqpoint{3.696000in}{3.696000in}}%
\pgfusepath{clip}%
\pgfsetrectcap%
\pgfsetroundjoin%
\pgfsetlinewidth{1.505625pt}%
\definecolor{currentstroke}{rgb}{1.000000,0.000000,0.000000}%
\pgfsetstrokecolor{currentstroke}%
\pgfsetdash{}{0pt}%
\pgfpathmoveto{\pgfqpoint{1.477339in}{1.766314in}}%
\pgfpathlineto{\pgfqpoint{1.385058in}{1.926619in}}%
\pgfusepath{stroke}%
\end{pgfscope}%
\begin{pgfscope}%
\pgfpathrectangle{\pgfqpoint{0.100000in}{0.212622in}}{\pgfqpoint{3.696000in}{3.696000in}}%
\pgfusepath{clip}%
\pgfsetrectcap%
\pgfsetroundjoin%
\pgfsetlinewidth{1.505625pt}%
\definecolor{currentstroke}{rgb}{1.000000,0.000000,0.000000}%
\pgfsetstrokecolor{currentstroke}%
\pgfsetdash{}{0pt}%
\pgfpathmoveto{\pgfqpoint{1.479516in}{1.765292in}}%
\pgfpathlineto{\pgfqpoint{1.385058in}{1.926619in}}%
\pgfusepath{stroke}%
\end{pgfscope}%
\begin{pgfscope}%
\pgfpathrectangle{\pgfqpoint{0.100000in}{0.212622in}}{\pgfqpoint{3.696000in}{3.696000in}}%
\pgfusepath{clip}%
\pgfsetrectcap%
\pgfsetroundjoin%
\pgfsetlinewidth{1.505625pt}%
\definecolor{currentstroke}{rgb}{1.000000,0.000000,0.000000}%
\pgfsetstrokecolor{currentstroke}%
\pgfsetdash{}{0pt}%
\pgfpathmoveto{\pgfqpoint{1.480714in}{1.766045in}}%
\pgfpathlineto{\pgfqpoint{1.394846in}{1.934851in}}%
\pgfusepath{stroke}%
\end{pgfscope}%
\begin{pgfscope}%
\pgfpathrectangle{\pgfqpoint{0.100000in}{0.212622in}}{\pgfqpoint{3.696000in}{3.696000in}}%
\pgfusepath{clip}%
\pgfsetrectcap%
\pgfsetroundjoin%
\pgfsetlinewidth{1.505625pt}%
\definecolor{currentstroke}{rgb}{1.000000,0.000000,0.000000}%
\pgfsetstrokecolor{currentstroke}%
\pgfsetdash{}{0pt}%
\pgfpathmoveto{\pgfqpoint{1.481408in}{1.766154in}}%
\pgfpathlineto{\pgfqpoint{1.394846in}{1.934851in}}%
\pgfusepath{stroke}%
\end{pgfscope}%
\begin{pgfscope}%
\pgfpathrectangle{\pgfqpoint{0.100000in}{0.212622in}}{\pgfqpoint{3.696000in}{3.696000in}}%
\pgfusepath{clip}%
\pgfsetrectcap%
\pgfsetroundjoin%
\pgfsetlinewidth{1.505625pt}%
\definecolor{currentstroke}{rgb}{1.000000,0.000000,0.000000}%
\pgfsetstrokecolor{currentstroke}%
\pgfsetdash{}{0pt}%
\pgfpathmoveto{\pgfqpoint{1.482442in}{1.766223in}}%
\pgfpathlineto{\pgfqpoint{1.394846in}{1.934851in}}%
\pgfusepath{stroke}%
\end{pgfscope}%
\begin{pgfscope}%
\pgfpathrectangle{\pgfqpoint{0.100000in}{0.212622in}}{\pgfqpoint{3.696000in}{3.696000in}}%
\pgfusepath{clip}%
\pgfsetrectcap%
\pgfsetroundjoin%
\pgfsetlinewidth{1.505625pt}%
\definecolor{currentstroke}{rgb}{1.000000,0.000000,0.000000}%
\pgfsetstrokecolor{currentstroke}%
\pgfsetdash{}{0pt}%
\pgfpathmoveto{\pgfqpoint{1.483688in}{1.766358in}}%
\pgfpathlineto{\pgfqpoint{1.394846in}{1.934851in}}%
\pgfusepath{stroke}%
\end{pgfscope}%
\begin{pgfscope}%
\pgfpathrectangle{\pgfqpoint{0.100000in}{0.212622in}}{\pgfqpoint{3.696000in}{3.696000in}}%
\pgfusepath{clip}%
\pgfsetrectcap%
\pgfsetroundjoin%
\pgfsetlinewidth{1.505625pt}%
\definecolor{currentstroke}{rgb}{1.000000,0.000000,0.000000}%
\pgfsetstrokecolor{currentstroke}%
\pgfsetdash{}{0pt}%
\pgfpathmoveto{\pgfqpoint{1.484435in}{1.766449in}}%
\pgfpathlineto{\pgfqpoint{1.394846in}{1.934851in}}%
\pgfusepath{stroke}%
\end{pgfscope}%
\begin{pgfscope}%
\pgfpathrectangle{\pgfqpoint{0.100000in}{0.212622in}}{\pgfqpoint{3.696000in}{3.696000in}}%
\pgfusepath{clip}%
\pgfsetrectcap%
\pgfsetroundjoin%
\pgfsetlinewidth{1.505625pt}%
\definecolor{currentstroke}{rgb}{1.000000,0.000000,0.000000}%
\pgfsetstrokecolor{currentstroke}%
\pgfsetdash{}{0pt}%
\pgfpathmoveto{\pgfqpoint{1.484832in}{1.766524in}}%
\pgfpathlineto{\pgfqpoint{1.394846in}{1.934851in}}%
\pgfusepath{stroke}%
\end{pgfscope}%
\begin{pgfscope}%
\pgfpathrectangle{\pgfqpoint{0.100000in}{0.212622in}}{\pgfqpoint{3.696000in}{3.696000in}}%
\pgfusepath{clip}%
\pgfsetrectcap%
\pgfsetroundjoin%
\pgfsetlinewidth{1.505625pt}%
\definecolor{currentstroke}{rgb}{1.000000,0.000000,0.000000}%
\pgfsetstrokecolor{currentstroke}%
\pgfsetdash{}{0pt}%
\pgfpathmoveto{\pgfqpoint{1.485052in}{1.766562in}}%
\pgfpathlineto{\pgfqpoint{1.394846in}{1.934851in}}%
\pgfusepath{stroke}%
\end{pgfscope}%
\begin{pgfscope}%
\pgfpathrectangle{\pgfqpoint{0.100000in}{0.212622in}}{\pgfqpoint{3.696000in}{3.696000in}}%
\pgfusepath{clip}%
\pgfsetrectcap%
\pgfsetroundjoin%
\pgfsetlinewidth{1.505625pt}%
\definecolor{currentstroke}{rgb}{1.000000,0.000000,0.000000}%
\pgfsetstrokecolor{currentstroke}%
\pgfsetdash{}{0pt}%
\pgfpathmoveto{\pgfqpoint{1.485174in}{1.766574in}}%
\pgfpathlineto{\pgfqpoint{1.394846in}{1.934851in}}%
\pgfusepath{stroke}%
\end{pgfscope}%
\begin{pgfscope}%
\pgfpathrectangle{\pgfqpoint{0.100000in}{0.212622in}}{\pgfqpoint{3.696000in}{3.696000in}}%
\pgfusepath{clip}%
\pgfsetrectcap%
\pgfsetroundjoin%
\pgfsetlinewidth{1.505625pt}%
\definecolor{currentstroke}{rgb}{1.000000,0.000000,0.000000}%
\pgfsetstrokecolor{currentstroke}%
\pgfsetdash{}{0pt}%
\pgfpathmoveto{\pgfqpoint{1.485236in}{1.766586in}}%
\pgfpathlineto{\pgfqpoint{1.394846in}{1.934851in}}%
\pgfusepath{stroke}%
\end{pgfscope}%
\begin{pgfscope}%
\pgfpathrectangle{\pgfqpoint{0.100000in}{0.212622in}}{\pgfqpoint{3.696000in}{3.696000in}}%
\pgfusepath{clip}%
\pgfsetrectcap%
\pgfsetroundjoin%
\pgfsetlinewidth{1.505625pt}%
\definecolor{currentstroke}{rgb}{1.000000,0.000000,0.000000}%
\pgfsetstrokecolor{currentstroke}%
\pgfsetdash{}{0pt}%
\pgfpathmoveto{\pgfqpoint{1.485277in}{1.766568in}}%
\pgfpathlineto{\pgfqpoint{1.394846in}{1.934851in}}%
\pgfusepath{stroke}%
\end{pgfscope}%
\begin{pgfscope}%
\pgfpathrectangle{\pgfqpoint{0.100000in}{0.212622in}}{\pgfqpoint{3.696000in}{3.696000in}}%
\pgfusepath{clip}%
\pgfsetrectcap%
\pgfsetroundjoin%
\pgfsetlinewidth{1.505625pt}%
\definecolor{currentstroke}{rgb}{1.000000,0.000000,0.000000}%
\pgfsetstrokecolor{currentstroke}%
\pgfsetdash{}{0pt}%
\pgfpathmoveto{\pgfqpoint{1.485706in}{1.765802in}}%
\pgfpathlineto{\pgfqpoint{1.394846in}{1.934851in}}%
\pgfusepath{stroke}%
\end{pgfscope}%
\begin{pgfscope}%
\pgfpathrectangle{\pgfqpoint{0.100000in}{0.212622in}}{\pgfqpoint{3.696000in}{3.696000in}}%
\pgfusepath{clip}%
\pgfsetrectcap%
\pgfsetroundjoin%
\pgfsetlinewidth{1.505625pt}%
\definecolor{currentstroke}{rgb}{1.000000,0.000000,0.000000}%
\pgfsetstrokecolor{currentstroke}%
\pgfsetdash{}{0pt}%
\pgfpathmoveto{\pgfqpoint{1.485954in}{1.765805in}}%
\pgfpathlineto{\pgfqpoint{1.394846in}{1.934851in}}%
\pgfusepath{stroke}%
\end{pgfscope}%
\begin{pgfscope}%
\pgfpathrectangle{\pgfqpoint{0.100000in}{0.212622in}}{\pgfqpoint{3.696000in}{3.696000in}}%
\pgfusepath{clip}%
\pgfsetrectcap%
\pgfsetroundjoin%
\pgfsetlinewidth{1.505625pt}%
\definecolor{currentstroke}{rgb}{1.000000,0.000000,0.000000}%
\pgfsetstrokecolor{currentstroke}%
\pgfsetdash{}{0pt}%
\pgfpathmoveto{\pgfqpoint{1.486099in}{1.765653in}}%
\pgfpathlineto{\pgfqpoint{1.394846in}{1.934851in}}%
\pgfusepath{stroke}%
\end{pgfscope}%
\begin{pgfscope}%
\pgfpathrectangle{\pgfqpoint{0.100000in}{0.212622in}}{\pgfqpoint{3.696000in}{3.696000in}}%
\pgfusepath{clip}%
\pgfsetrectcap%
\pgfsetroundjoin%
\pgfsetlinewidth{1.505625pt}%
\definecolor{currentstroke}{rgb}{1.000000,0.000000,0.000000}%
\pgfsetstrokecolor{currentstroke}%
\pgfsetdash{}{0pt}%
\pgfpathmoveto{\pgfqpoint{1.486617in}{1.765418in}}%
\pgfpathlineto{\pgfqpoint{1.394846in}{1.934851in}}%
\pgfusepath{stroke}%
\end{pgfscope}%
\begin{pgfscope}%
\pgfpathrectangle{\pgfqpoint{0.100000in}{0.212622in}}{\pgfqpoint{3.696000in}{3.696000in}}%
\pgfusepath{clip}%
\pgfsetrectcap%
\pgfsetroundjoin%
\pgfsetlinewidth{1.505625pt}%
\definecolor{currentstroke}{rgb}{1.000000,0.000000,0.000000}%
\pgfsetstrokecolor{currentstroke}%
\pgfsetdash{}{0pt}%
\pgfpathmoveto{\pgfqpoint{1.486867in}{1.765061in}}%
\pgfpathlineto{\pgfqpoint{1.394846in}{1.934851in}}%
\pgfusepath{stroke}%
\end{pgfscope}%
\begin{pgfscope}%
\pgfpathrectangle{\pgfqpoint{0.100000in}{0.212622in}}{\pgfqpoint{3.696000in}{3.696000in}}%
\pgfusepath{clip}%
\pgfsetrectcap%
\pgfsetroundjoin%
\pgfsetlinewidth{1.505625pt}%
\definecolor{currentstroke}{rgb}{1.000000,0.000000,0.000000}%
\pgfsetstrokecolor{currentstroke}%
\pgfsetdash{}{0pt}%
\pgfpathmoveto{\pgfqpoint{1.486992in}{1.765022in}}%
\pgfpathlineto{\pgfqpoint{1.394846in}{1.934851in}}%
\pgfusepath{stroke}%
\end{pgfscope}%
\begin{pgfscope}%
\pgfpathrectangle{\pgfqpoint{0.100000in}{0.212622in}}{\pgfqpoint{3.696000in}{3.696000in}}%
\pgfusepath{clip}%
\pgfsetrectcap%
\pgfsetroundjoin%
\pgfsetlinewidth{1.505625pt}%
\definecolor{currentstroke}{rgb}{1.000000,0.000000,0.000000}%
\pgfsetstrokecolor{currentstroke}%
\pgfsetdash{}{0pt}%
\pgfpathmoveto{\pgfqpoint{1.487043in}{1.765043in}}%
\pgfpathlineto{\pgfqpoint{1.394846in}{1.934851in}}%
\pgfusepath{stroke}%
\end{pgfscope}%
\begin{pgfscope}%
\pgfpathrectangle{\pgfqpoint{0.100000in}{0.212622in}}{\pgfqpoint{3.696000in}{3.696000in}}%
\pgfusepath{clip}%
\pgfsetrectcap%
\pgfsetroundjoin%
\pgfsetlinewidth{1.505625pt}%
\definecolor{currentstroke}{rgb}{1.000000,0.000000,0.000000}%
\pgfsetstrokecolor{currentstroke}%
\pgfsetdash{}{0pt}%
\pgfpathmoveto{\pgfqpoint{1.487062in}{1.765032in}}%
\pgfpathlineto{\pgfqpoint{1.394846in}{1.934851in}}%
\pgfusepath{stroke}%
\end{pgfscope}%
\begin{pgfscope}%
\pgfpathrectangle{\pgfqpoint{0.100000in}{0.212622in}}{\pgfqpoint{3.696000in}{3.696000in}}%
\pgfusepath{clip}%
\pgfsetrectcap%
\pgfsetroundjoin%
\pgfsetlinewidth{1.505625pt}%
\definecolor{currentstroke}{rgb}{1.000000,0.000000,0.000000}%
\pgfsetstrokecolor{currentstroke}%
\pgfsetdash{}{0pt}%
\pgfpathmoveto{\pgfqpoint{1.487070in}{1.765033in}}%
\pgfpathlineto{\pgfqpoint{1.394846in}{1.934851in}}%
\pgfusepath{stroke}%
\end{pgfscope}%
\begin{pgfscope}%
\pgfpathrectangle{\pgfqpoint{0.100000in}{0.212622in}}{\pgfqpoint{3.696000in}{3.696000in}}%
\pgfusepath{clip}%
\pgfsetrectcap%
\pgfsetroundjoin%
\pgfsetlinewidth{1.505625pt}%
\definecolor{currentstroke}{rgb}{1.000000,0.000000,0.000000}%
\pgfsetstrokecolor{currentstroke}%
\pgfsetdash{}{0pt}%
\pgfpathmoveto{\pgfqpoint{1.487121in}{1.764861in}}%
\pgfpathlineto{\pgfqpoint{1.394846in}{1.934851in}}%
\pgfusepath{stroke}%
\end{pgfscope}%
\begin{pgfscope}%
\pgfpathrectangle{\pgfqpoint{0.100000in}{0.212622in}}{\pgfqpoint{3.696000in}{3.696000in}}%
\pgfusepath{clip}%
\pgfsetrectcap%
\pgfsetroundjoin%
\pgfsetlinewidth{1.505625pt}%
\definecolor{currentstroke}{rgb}{1.000000,0.000000,0.000000}%
\pgfsetstrokecolor{currentstroke}%
\pgfsetdash{}{0pt}%
\pgfpathmoveto{\pgfqpoint{1.487168in}{1.765429in}}%
\pgfpathlineto{\pgfqpoint{1.394846in}{1.934851in}}%
\pgfusepath{stroke}%
\end{pgfscope}%
\begin{pgfscope}%
\pgfpathrectangle{\pgfqpoint{0.100000in}{0.212622in}}{\pgfqpoint{3.696000in}{3.696000in}}%
\pgfusepath{clip}%
\pgfsetrectcap%
\pgfsetroundjoin%
\pgfsetlinewidth{1.505625pt}%
\definecolor{currentstroke}{rgb}{1.000000,0.000000,0.000000}%
\pgfsetstrokecolor{currentstroke}%
\pgfsetdash{}{0pt}%
\pgfpathmoveto{\pgfqpoint{1.487172in}{1.765561in}}%
\pgfpathlineto{\pgfqpoint{1.394846in}{1.934851in}}%
\pgfusepath{stroke}%
\end{pgfscope}%
\begin{pgfscope}%
\pgfpathrectangle{\pgfqpoint{0.100000in}{0.212622in}}{\pgfqpoint{3.696000in}{3.696000in}}%
\pgfusepath{clip}%
\pgfsetrectcap%
\pgfsetroundjoin%
\pgfsetlinewidth{1.505625pt}%
\definecolor{currentstroke}{rgb}{1.000000,0.000000,0.000000}%
\pgfsetstrokecolor{currentstroke}%
\pgfsetdash{}{0pt}%
\pgfpathmoveto{\pgfqpoint{1.487164in}{1.765488in}}%
\pgfpathlineto{\pgfqpoint{1.394846in}{1.934851in}}%
\pgfusepath{stroke}%
\end{pgfscope}%
\begin{pgfscope}%
\pgfpathrectangle{\pgfqpoint{0.100000in}{0.212622in}}{\pgfqpoint{3.696000in}{3.696000in}}%
\pgfusepath{clip}%
\pgfsetrectcap%
\pgfsetroundjoin%
\pgfsetlinewidth{1.505625pt}%
\definecolor{currentstroke}{rgb}{1.000000,0.000000,0.000000}%
\pgfsetstrokecolor{currentstroke}%
\pgfsetdash{}{0pt}%
\pgfpathmoveto{\pgfqpoint{1.487131in}{1.765620in}}%
\pgfpathlineto{\pgfqpoint{1.394846in}{1.934851in}}%
\pgfusepath{stroke}%
\end{pgfscope}%
\begin{pgfscope}%
\pgfpathrectangle{\pgfqpoint{0.100000in}{0.212622in}}{\pgfqpoint{3.696000in}{3.696000in}}%
\pgfusepath{clip}%
\pgfsetrectcap%
\pgfsetroundjoin%
\pgfsetlinewidth{1.505625pt}%
\definecolor{currentstroke}{rgb}{1.000000,0.000000,0.000000}%
\pgfsetstrokecolor{currentstroke}%
\pgfsetdash{}{0pt}%
\pgfpathmoveto{\pgfqpoint{1.487082in}{1.765621in}}%
\pgfpathlineto{\pgfqpoint{1.394846in}{1.934851in}}%
\pgfusepath{stroke}%
\end{pgfscope}%
\begin{pgfscope}%
\pgfpathrectangle{\pgfqpoint{0.100000in}{0.212622in}}{\pgfqpoint{3.696000in}{3.696000in}}%
\pgfusepath{clip}%
\pgfsetrectcap%
\pgfsetroundjoin%
\pgfsetlinewidth{1.505625pt}%
\definecolor{currentstroke}{rgb}{1.000000,0.000000,0.000000}%
\pgfsetstrokecolor{currentstroke}%
\pgfsetdash{}{0pt}%
\pgfpathmoveto{\pgfqpoint{1.486974in}{1.766370in}}%
\pgfpathlineto{\pgfqpoint{1.394846in}{1.934851in}}%
\pgfusepath{stroke}%
\end{pgfscope}%
\begin{pgfscope}%
\pgfpathrectangle{\pgfqpoint{0.100000in}{0.212622in}}{\pgfqpoint{3.696000in}{3.696000in}}%
\pgfusepath{clip}%
\pgfsetrectcap%
\pgfsetroundjoin%
\pgfsetlinewidth{1.505625pt}%
\definecolor{currentstroke}{rgb}{1.000000,0.000000,0.000000}%
\pgfsetstrokecolor{currentstroke}%
\pgfsetdash{}{0pt}%
\pgfpathmoveto{\pgfqpoint{1.487289in}{1.764285in}}%
\pgfpathlineto{\pgfqpoint{1.394846in}{1.934851in}}%
\pgfusepath{stroke}%
\end{pgfscope}%
\begin{pgfscope}%
\pgfpathrectangle{\pgfqpoint{0.100000in}{0.212622in}}{\pgfqpoint{3.696000in}{3.696000in}}%
\pgfusepath{clip}%
\pgfsetrectcap%
\pgfsetroundjoin%
\pgfsetlinewidth{1.505625pt}%
\definecolor{currentstroke}{rgb}{1.000000,0.000000,0.000000}%
\pgfsetstrokecolor{currentstroke}%
\pgfsetdash{}{0pt}%
\pgfpathmoveto{\pgfqpoint{1.487901in}{1.762503in}}%
\pgfpathlineto{\pgfqpoint{1.394846in}{1.934851in}}%
\pgfusepath{stroke}%
\end{pgfscope}%
\begin{pgfscope}%
\pgfpathrectangle{\pgfqpoint{0.100000in}{0.212622in}}{\pgfqpoint{3.696000in}{3.696000in}}%
\pgfusepath{clip}%
\pgfsetrectcap%
\pgfsetroundjoin%
\pgfsetlinewidth{1.505625pt}%
\definecolor{currentstroke}{rgb}{1.000000,0.000000,0.000000}%
\pgfsetstrokecolor{currentstroke}%
\pgfsetdash{}{0pt}%
\pgfpathmoveto{\pgfqpoint{1.488876in}{1.765918in}}%
\pgfpathlineto{\pgfqpoint{1.385058in}{1.926619in}}%
\pgfusepath{stroke}%
\end{pgfscope}%
\begin{pgfscope}%
\pgfpathrectangle{\pgfqpoint{0.100000in}{0.212622in}}{\pgfqpoint{3.696000in}{3.696000in}}%
\pgfusepath{clip}%
\pgfsetrectcap%
\pgfsetroundjoin%
\pgfsetlinewidth{1.505625pt}%
\definecolor{currentstroke}{rgb}{1.000000,0.000000,0.000000}%
\pgfsetstrokecolor{currentstroke}%
\pgfsetdash{}{0pt}%
\pgfpathmoveto{\pgfqpoint{1.489334in}{1.763799in}}%
\pgfpathlineto{\pgfqpoint{1.385058in}{1.926619in}}%
\pgfusepath{stroke}%
\end{pgfscope}%
\begin{pgfscope}%
\pgfpathrectangle{\pgfqpoint{0.100000in}{0.212622in}}{\pgfqpoint{3.696000in}{3.696000in}}%
\pgfusepath{clip}%
\pgfsetrectcap%
\pgfsetroundjoin%
\pgfsetlinewidth{1.505625pt}%
\definecolor{currentstroke}{rgb}{1.000000,0.000000,0.000000}%
\pgfsetstrokecolor{currentstroke}%
\pgfsetdash{}{0pt}%
\pgfpathmoveto{\pgfqpoint{1.491236in}{1.761327in}}%
\pgfpathlineto{\pgfqpoint{1.385058in}{1.926619in}}%
\pgfusepath{stroke}%
\end{pgfscope}%
\begin{pgfscope}%
\pgfpathrectangle{\pgfqpoint{0.100000in}{0.212622in}}{\pgfqpoint{3.696000in}{3.696000in}}%
\pgfusepath{clip}%
\pgfsetrectcap%
\pgfsetroundjoin%
\pgfsetlinewidth{1.505625pt}%
\definecolor{currentstroke}{rgb}{1.000000,0.000000,0.000000}%
\pgfsetstrokecolor{currentstroke}%
\pgfsetdash{}{0pt}%
\pgfpathmoveto{\pgfqpoint{1.492235in}{1.763115in}}%
\pgfpathlineto{\pgfqpoint{1.385058in}{1.926619in}}%
\pgfusepath{stroke}%
\end{pgfscope}%
\begin{pgfscope}%
\pgfpathrectangle{\pgfqpoint{0.100000in}{0.212622in}}{\pgfqpoint{3.696000in}{3.696000in}}%
\pgfusepath{clip}%
\pgfsetrectcap%
\pgfsetroundjoin%
\pgfsetlinewidth{1.505625pt}%
\definecolor{currentstroke}{rgb}{1.000000,0.000000,0.000000}%
\pgfsetstrokecolor{currentstroke}%
\pgfsetdash{}{0pt}%
\pgfpathmoveto{\pgfqpoint{1.492606in}{1.761328in}}%
\pgfpathlineto{\pgfqpoint{1.375257in}{1.918374in}}%
\pgfusepath{stroke}%
\end{pgfscope}%
\begin{pgfscope}%
\pgfpathrectangle{\pgfqpoint{0.100000in}{0.212622in}}{\pgfqpoint{3.696000in}{3.696000in}}%
\pgfusepath{clip}%
\pgfsetrectcap%
\pgfsetroundjoin%
\pgfsetlinewidth{1.505625pt}%
\definecolor{currentstroke}{rgb}{1.000000,0.000000,0.000000}%
\pgfsetstrokecolor{currentstroke}%
\pgfsetdash{}{0pt}%
\pgfpathmoveto{\pgfqpoint{1.493361in}{1.761308in}}%
\pgfpathlineto{\pgfqpoint{1.375257in}{1.918374in}}%
\pgfusepath{stroke}%
\end{pgfscope}%
\begin{pgfscope}%
\pgfpathrectangle{\pgfqpoint{0.100000in}{0.212622in}}{\pgfqpoint{3.696000in}{3.696000in}}%
\pgfusepath{clip}%
\pgfsetrectcap%
\pgfsetroundjoin%
\pgfsetlinewidth{1.505625pt}%
\definecolor{currentstroke}{rgb}{1.000000,0.000000,0.000000}%
\pgfsetstrokecolor{currentstroke}%
\pgfsetdash{}{0pt}%
\pgfpathmoveto{\pgfqpoint{1.494621in}{1.763895in}}%
\pgfpathlineto{\pgfqpoint{1.375257in}{1.918374in}}%
\pgfusepath{stroke}%
\end{pgfscope}%
\begin{pgfscope}%
\pgfpathrectangle{\pgfqpoint{0.100000in}{0.212622in}}{\pgfqpoint{3.696000in}{3.696000in}}%
\pgfusepath{clip}%
\pgfsetrectcap%
\pgfsetroundjoin%
\pgfsetlinewidth{1.505625pt}%
\definecolor{currentstroke}{rgb}{1.000000,0.000000,0.000000}%
\pgfsetstrokecolor{currentstroke}%
\pgfsetdash{}{0pt}%
\pgfpathmoveto{\pgfqpoint{1.494776in}{1.763646in}}%
\pgfpathlineto{\pgfqpoint{1.375257in}{1.918374in}}%
\pgfusepath{stroke}%
\end{pgfscope}%
\begin{pgfscope}%
\pgfpathrectangle{\pgfqpoint{0.100000in}{0.212622in}}{\pgfqpoint{3.696000in}{3.696000in}}%
\pgfusepath{clip}%
\pgfsetrectcap%
\pgfsetroundjoin%
\pgfsetlinewidth{1.505625pt}%
\definecolor{currentstroke}{rgb}{1.000000,0.000000,0.000000}%
\pgfsetstrokecolor{currentstroke}%
\pgfsetdash{}{0pt}%
\pgfpathmoveto{\pgfqpoint{1.495820in}{1.763688in}}%
\pgfpathlineto{\pgfqpoint{1.375257in}{1.918374in}}%
\pgfusepath{stroke}%
\end{pgfscope}%
\begin{pgfscope}%
\pgfpathrectangle{\pgfqpoint{0.100000in}{0.212622in}}{\pgfqpoint{3.696000in}{3.696000in}}%
\pgfusepath{clip}%
\pgfsetrectcap%
\pgfsetroundjoin%
\pgfsetlinewidth{1.505625pt}%
\definecolor{currentstroke}{rgb}{1.000000,0.000000,0.000000}%
\pgfsetstrokecolor{currentstroke}%
\pgfsetdash{}{0pt}%
\pgfpathmoveto{\pgfqpoint{1.497157in}{1.765045in}}%
\pgfpathlineto{\pgfqpoint{1.375257in}{1.918374in}}%
\pgfusepath{stroke}%
\end{pgfscope}%
\begin{pgfscope}%
\pgfpathrectangle{\pgfqpoint{0.100000in}{0.212622in}}{\pgfqpoint{3.696000in}{3.696000in}}%
\pgfusepath{clip}%
\pgfsetrectcap%
\pgfsetroundjoin%
\pgfsetlinewidth{1.505625pt}%
\definecolor{currentstroke}{rgb}{1.000000,0.000000,0.000000}%
\pgfsetstrokecolor{currentstroke}%
\pgfsetdash{}{0pt}%
\pgfpathmoveto{\pgfqpoint{1.497737in}{1.763220in}}%
\pgfpathlineto{\pgfqpoint{1.365441in}{1.910118in}}%
\pgfusepath{stroke}%
\end{pgfscope}%
\begin{pgfscope}%
\pgfpathrectangle{\pgfqpoint{0.100000in}{0.212622in}}{\pgfqpoint{3.696000in}{3.696000in}}%
\pgfusepath{clip}%
\pgfsetrectcap%
\pgfsetroundjoin%
\pgfsetlinewidth{1.505625pt}%
\definecolor{currentstroke}{rgb}{1.000000,0.000000,0.000000}%
\pgfsetstrokecolor{currentstroke}%
\pgfsetdash{}{0pt}%
\pgfpathmoveto{\pgfqpoint{1.499384in}{1.764556in}}%
\pgfpathlineto{\pgfqpoint{1.365441in}{1.910118in}}%
\pgfusepath{stroke}%
\end{pgfscope}%
\begin{pgfscope}%
\pgfpathrectangle{\pgfqpoint{0.100000in}{0.212622in}}{\pgfqpoint{3.696000in}{3.696000in}}%
\pgfusepath{clip}%
\pgfsetrectcap%
\pgfsetroundjoin%
\pgfsetlinewidth{1.505625pt}%
\definecolor{currentstroke}{rgb}{1.000000,0.000000,0.000000}%
\pgfsetstrokecolor{currentstroke}%
\pgfsetdash{}{0pt}%
\pgfpathmoveto{\pgfqpoint{1.501311in}{1.767738in}}%
\pgfpathlineto{\pgfqpoint{1.365441in}{1.910118in}}%
\pgfusepath{stroke}%
\end{pgfscope}%
\begin{pgfscope}%
\pgfpathrectangle{\pgfqpoint{0.100000in}{0.212622in}}{\pgfqpoint{3.696000in}{3.696000in}}%
\pgfusepath{clip}%
\pgfsetrectcap%
\pgfsetroundjoin%
\pgfsetlinewidth{1.505625pt}%
\definecolor{currentstroke}{rgb}{1.000000,0.000000,0.000000}%
\pgfsetstrokecolor{currentstroke}%
\pgfsetdash{}{0pt}%
\pgfpathmoveto{\pgfqpoint{1.501962in}{1.767697in}}%
\pgfpathlineto{\pgfqpoint{1.365441in}{1.910118in}}%
\pgfusepath{stroke}%
\end{pgfscope}%
\begin{pgfscope}%
\pgfpathrectangle{\pgfqpoint{0.100000in}{0.212622in}}{\pgfqpoint{3.696000in}{3.696000in}}%
\pgfusepath{clip}%
\pgfsetrectcap%
\pgfsetroundjoin%
\pgfsetlinewidth{1.505625pt}%
\definecolor{currentstroke}{rgb}{1.000000,0.000000,0.000000}%
\pgfsetstrokecolor{currentstroke}%
\pgfsetdash{}{0pt}%
\pgfpathmoveto{\pgfqpoint{1.503268in}{1.768194in}}%
\pgfpathlineto{\pgfqpoint{1.365441in}{1.910118in}}%
\pgfusepath{stroke}%
\end{pgfscope}%
\begin{pgfscope}%
\pgfpathrectangle{\pgfqpoint{0.100000in}{0.212622in}}{\pgfqpoint{3.696000in}{3.696000in}}%
\pgfusepath{clip}%
\pgfsetrectcap%
\pgfsetroundjoin%
\pgfsetlinewidth{1.505625pt}%
\definecolor{currentstroke}{rgb}{1.000000,0.000000,0.000000}%
\pgfsetstrokecolor{currentstroke}%
\pgfsetdash{}{0pt}%
\pgfpathmoveto{\pgfqpoint{1.505103in}{1.771301in}}%
\pgfpathlineto{\pgfqpoint{1.355612in}{1.901850in}}%
\pgfusepath{stroke}%
\end{pgfscope}%
\begin{pgfscope}%
\pgfpathrectangle{\pgfqpoint{0.100000in}{0.212622in}}{\pgfqpoint{3.696000in}{3.696000in}}%
\pgfusepath{clip}%
\pgfsetrectcap%
\pgfsetroundjoin%
\pgfsetlinewidth{1.505625pt}%
\definecolor{currentstroke}{rgb}{1.000000,0.000000,0.000000}%
\pgfsetstrokecolor{currentstroke}%
\pgfsetdash{}{0pt}%
\pgfpathmoveto{\pgfqpoint{1.506324in}{1.771159in}}%
\pgfpathlineto{\pgfqpoint{1.355612in}{1.901850in}}%
\pgfusepath{stroke}%
\end{pgfscope}%
\begin{pgfscope}%
\pgfpathrectangle{\pgfqpoint{0.100000in}{0.212622in}}{\pgfqpoint{3.696000in}{3.696000in}}%
\pgfusepath{clip}%
\pgfsetrectcap%
\pgfsetroundjoin%
\pgfsetlinewidth{1.505625pt}%
\definecolor{currentstroke}{rgb}{1.000000,0.000000,0.000000}%
\pgfsetstrokecolor{currentstroke}%
\pgfsetdash{}{0pt}%
\pgfpathmoveto{\pgfqpoint{1.508196in}{1.767877in}}%
\pgfpathlineto{\pgfqpoint{1.355612in}{1.901850in}}%
\pgfusepath{stroke}%
\end{pgfscope}%
\begin{pgfscope}%
\pgfpathrectangle{\pgfqpoint{0.100000in}{0.212622in}}{\pgfqpoint{3.696000in}{3.696000in}}%
\pgfusepath{clip}%
\pgfsetrectcap%
\pgfsetroundjoin%
\pgfsetlinewidth{1.505625pt}%
\definecolor{currentstroke}{rgb}{1.000000,0.000000,0.000000}%
\pgfsetstrokecolor{currentstroke}%
\pgfsetdash{}{0pt}%
\pgfpathmoveto{\pgfqpoint{1.510922in}{1.772068in}}%
\pgfpathlineto{\pgfqpoint{1.345769in}{1.893571in}}%
\pgfusepath{stroke}%
\end{pgfscope}%
\begin{pgfscope}%
\pgfpathrectangle{\pgfqpoint{0.100000in}{0.212622in}}{\pgfqpoint{3.696000in}{3.696000in}}%
\pgfusepath{clip}%
\pgfsetrectcap%
\pgfsetroundjoin%
\pgfsetlinewidth{1.505625pt}%
\definecolor{currentstroke}{rgb}{1.000000,0.000000,0.000000}%
\pgfsetstrokecolor{currentstroke}%
\pgfsetdash{}{0pt}%
\pgfpathmoveto{\pgfqpoint{1.513834in}{1.763264in}}%
\pgfpathlineto{\pgfqpoint{1.345769in}{1.893571in}}%
\pgfusepath{stroke}%
\end{pgfscope}%
\begin{pgfscope}%
\pgfpathrectangle{\pgfqpoint{0.100000in}{0.212622in}}{\pgfqpoint{3.696000in}{3.696000in}}%
\pgfusepath{clip}%
\pgfsetrectcap%
\pgfsetroundjoin%
\pgfsetlinewidth{1.505625pt}%
\definecolor{currentstroke}{rgb}{1.000000,0.000000,0.000000}%
\pgfsetstrokecolor{currentstroke}%
\pgfsetdash{}{0pt}%
\pgfpathmoveto{\pgfqpoint{1.515030in}{1.761477in}}%
\pgfpathlineto{\pgfqpoint{1.335912in}{1.885280in}}%
\pgfusepath{stroke}%
\end{pgfscope}%
\begin{pgfscope}%
\pgfpathrectangle{\pgfqpoint{0.100000in}{0.212622in}}{\pgfqpoint{3.696000in}{3.696000in}}%
\pgfusepath{clip}%
\pgfsetrectcap%
\pgfsetroundjoin%
\pgfsetlinewidth{1.505625pt}%
\definecolor{currentstroke}{rgb}{1.000000,0.000000,0.000000}%
\pgfsetstrokecolor{currentstroke}%
\pgfsetdash{}{0pt}%
\pgfpathmoveto{\pgfqpoint{1.517688in}{1.761513in}}%
\pgfpathlineto{\pgfqpoint{1.335912in}{1.885280in}}%
\pgfusepath{stroke}%
\end{pgfscope}%
\begin{pgfscope}%
\pgfpathrectangle{\pgfqpoint{0.100000in}{0.212622in}}{\pgfqpoint{3.696000in}{3.696000in}}%
\pgfusepath{clip}%
\pgfsetrectcap%
\pgfsetroundjoin%
\pgfsetlinewidth{1.505625pt}%
\definecolor{currentstroke}{rgb}{1.000000,0.000000,0.000000}%
\pgfsetstrokecolor{currentstroke}%
\pgfsetdash{}{0pt}%
\pgfpathmoveto{\pgfqpoint{1.520574in}{1.760680in}}%
\pgfpathlineto{\pgfqpoint{1.326041in}{1.876977in}}%
\pgfusepath{stroke}%
\end{pgfscope}%
\begin{pgfscope}%
\pgfpathrectangle{\pgfqpoint{0.100000in}{0.212622in}}{\pgfqpoint{3.696000in}{3.696000in}}%
\pgfusepath{clip}%
\pgfsetrectcap%
\pgfsetroundjoin%
\pgfsetlinewidth{1.505625pt}%
\definecolor{currentstroke}{rgb}{1.000000,0.000000,0.000000}%
\pgfsetstrokecolor{currentstroke}%
\pgfsetdash{}{0pt}%
\pgfpathmoveto{\pgfqpoint{1.520674in}{1.766184in}}%
\pgfpathlineto{\pgfqpoint{1.326041in}{1.876977in}}%
\pgfusepath{stroke}%
\end{pgfscope}%
\begin{pgfscope}%
\pgfpathrectangle{\pgfqpoint{0.100000in}{0.212622in}}{\pgfqpoint{3.696000in}{3.696000in}}%
\pgfusepath{clip}%
\pgfsetrectcap%
\pgfsetroundjoin%
\pgfsetlinewidth{1.505625pt}%
\definecolor{currentstroke}{rgb}{1.000000,0.000000,0.000000}%
\pgfsetstrokecolor{currentstroke}%
\pgfsetdash{}{0pt}%
\pgfpathmoveto{\pgfqpoint{1.521804in}{1.759985in}}%
\pgfpathlineto{\pgfqpoint{1.316156in}{1.868662in}}%
\pgfusepath{stroke}%
\end{pgfscope}%
\begin{pgfscope}%
\pgfpathrectangle{\pgfqpoint{0.100000in}{0.212622in}}{\pgfqpoint{3.696000in}{3.696000in}}%
\pgfusepath{clip}%
\pgfsetrectcap%
\pgfsetroundjoin%
\pgfsetlinewidth{1.505625pt}%
\definecolor{currentstroke}{rgb}{1.000000,0.000000,0.000000}%
\pgfsetstrokecolor{currentstroke}%
\pgfsetdash{}{0pt}%
\pgfpathmoveto{\pgfqpoint{1.523274in}{1.756750in}}%
\pgfpathlineto{\pgfqpoint{1.306257in}{1.860336in}}%
\pgfusepath{stroke}%
\end{pgfscope}%
\begin{pgfscope}%
\pgfpathrectangle{\pgfqpoint{0.100000in}{0.212622in}}{\pgfqpoint{3.696000in}{3.696000in}}%
\pgfusepath{clip}%
\pgfsetrectcap%
\pgfsetroundjoin%
\pgfsetlinewidth{1.505625pt}%
\definecolor{currentstroke}{rgb}{1.000000,0.000000,0.000000}%
\pgfsetstrokecolor{currentstroke}%
\pgfsetdash{}{0pt}%
\pgfpathmoveto{\pgfqpoint{1.522848in}{1.753531in}}%
\pgfpathlineto{\pgfqpoint{1.296344in}{1.851998in}}%
\pgfusepath{stroke}%
\end{pgfscope}%
\begin{pgfscope}%
\pgfpathrectangle{\pgfqpoint{0.100000in}{0.212622in}}{\pgfqpoint{3.696000in}{3.696000in}}%
\pgfusepath{clip}%
\pgfsetrectcap%
\pgfsetroundjoin%
\pgfsetlinewidth{1.505625pt}%
\definecolor{currentstroke}{rgb}{1.000000,0.000000,0.000000}%
\pgfsetstrokecolor{currentstroke}%
\pgfsetdash{}{0pt}%
\pgfpathmoveto{\pgfqpoint{1.526158in}{1.755646in}}%
\pgfpathlineto{\pgfqpoint{1.296344in}{1.851998in}}%
\pgfusepath{stroke}%
\end{pgfscope}%
\begin{pgfscope}%
\pgfpathrectangle{\pgfqpoint{0.100000in}{0.212622in}}{\pgfqpoint{3.696000in}{3.696000in}}%
\pgfusepath{clip}%
\pgfsetrectcap%
\pgfsetroundjoin%
\pgfsetlinewidth{1.505625pt}%
\definecolor{currentstroke}{rgb}{1.000000,0.000000,0.000000}%
\pgfsetstrokecolor{currentstroke}%
\pgfsetdash{}{0pt}%
\pgfpathmoveto{\pgfqpoint{1.527463in}{1.755050in}}%
\pgfpathlineto{\pgfqpoint{1.286417in}{1.843648in}}%
\pgfusepath{stroke}%
\end{pgfscope}%
\begin{pgfscope}%
\pgfpathrectangle{\pgfqpoint{0.100000in}{0.212622in}}{\pgfqpoint{3.696000in}{3.696000in}}%
\pgfusepath{clip}%
\pgfsetrectcap%
\pgfsetroundjoin%
\pgfsetlinewidth{1.505625pt}%
\definecolor{currentstroke}{rgb}{1.000000,0.000000,0.000000}%
\pgfsetstrokecolor{currentstroke}%
\pgfsetdash{}{0pt}%
\pgfpathmoveto{\pgfqpoint{1.527548in}{1.753388in}}%
\pgfpathlineto{\pgfqpoint{1.286417in}{1.843648in}}%
\pgfusepath{stroke}%
\end{pgfscope}%
\begin{pgfscope}%
\pgfpathrectangle{\pgfqpoint{0.100000in}{0.212622in}}{\pgfqpoint{3.696000in}{3.696000in}}%
\pgfusepath{clip}%
\pgfsetrectcap%
\pgfsetroundjoin%
\pgfsetlinewidth{1.505625pt}%
\definecolor{currentstroke}{rgb}{1.000000,0.000000,0.000000}%
\pgfsetstrokecolor{currentstroke}%
\pgfsetdash{}{0pt}%
\pgfpathmoveto{\pgfqpoint{1.529751in}{1.753663in}}%
\pgfpathlineto{\pgfqpoint{1.276476in}{1.835286in}}%
\pgfusepath{stroke}%
\end{pgfscope}%
\begin{pgfscope}%
\pgfpathrectangle{\pgfqpoint{0.100000in}{0.212622in}}{\pgfqpoint{3.696000in}{3.696000in}}%
\pgfusepath{clip}%
\pgfsetrectcap%
\pgfsetroundjoin%
\pgfsetlinewidth{1.505625pt}%
\definecolor{currentstroke}{rgb}{1.000000,0.000000,0.000000}%
\pgfsetstrokecolor{currentstroke}%
\pgfsetdash{}{0pt}%
\pgfpathmoveto{\pgfqpoint{1.533074in}{1.754664in}}%
\pgfpathlineto{\pgfqpoint{1.266521in}{1.826913in}}%
\pgfusepath{stroke}%
\end{pgfscope}%
\begin{pgfscope}%
\pgfpathrectangle{\pgfqpoint{0.100000in}{0.212622in}}{\pgfqpoint{3.696000in}{3.696000in}}%
\pgfusepath{clip}%
\pgfsetrectcap%
\pgfsetroundjoin%
\pgfsetlinewidth{1.505625pt}%
\definecolor{currentstroke}{rgb}{1.000000,0.000000,0.000000}%
\pgfsetstrokecolor{currentstroke}%
\pgfsetdash{}{0pt}%
\pgfpathmoveto{\pgfqpoint{1.533500in}{1.752945in}}%
\pgfpathlineto{\pgfqpoint{1.266521in}{1.826913in}}%
\pgfusepath{stroke}%
\end{pgfscope}%
\begin{pgfscope}%
\pgfpathrectangle{\pgfqpoint{0.100000in}{0.212622in}}{\pgfqpoint{3.696000in}{3.696000in}}%
\pgfusepath{clip}%
\pgfsetrectcap%
\pgfsetroundjoin%
\pgfsetlinewidth{1.505625pt}%
\definecolor{currentstroke}{rgb}{1.000000,0.000000,0.000000}%
\pgfsetstrokecolor{currentstroke}%
\pgfsetdash{}{0pt}%
\pgfpathmoveto{\pgfqpoint{1.534411in}{1.752358in}}%
\pgfpathlineto{\pgfqpoint{1.266521in}{1.826913in}}%
\pgfusepath{stroke}%
\end{pgfscope}%
\begin{pgfscope}%
\pgfpathrectangle{\pgfqpoint{0.100000in}{0.212622in}}{\pgfqpoint{3.696000in}{3.696000in}}%
\pgfusepath{clip}%
\pgfsetrectcap%
\pgfsetroundjoin%
\pgfsetlinewidth{1.505625pt}%
\definecolor{currentstroke}{rgb}{1.000000,0.000000,0.000000}%
\pgfsetstrokecolor{currentstroke}%
\pgfsetdash{}{0pt}%
\pgfpathmoveto{\pgfqpoint{1.535501in}{1.752590in}}%
\pgfpathlineto{\pgfqpoint{1.266521in}{1.826913in}}%
\pgfusepath{stroke}%
\end{pgfscope}%
\begin{pgfscope}%
\pgfpathrectangle{\pgfqpoint{0.100000in}{0.212622in}}{\pgfqpoint{3.696000in}{3.696000in}}%
\pgfusepath{clip}%
\pgfsetrectcap%
\pgfsetroundjoin%
\pgfsetlinewidth{1.505625pt}%
\definecolor{currentstroke}{rgb}{1.000000,0.000000,0.000000}%
\pgfsetstrokecolor{currentstroke}%
\pgfsetdash{}{0pt}%
\pgfpathmoveto{\pgfqpoint{1.536069in}{1.749809in}}%
\pgfpathlineto{\pgfqpoint{1.256551in}{1.818527in}}%
\pgfusepath{stroke}%
\end{pgfscope}%
\begin{pgfscope}%
\pgfpathrectangle{\pgfqpoint{0.100000in}{0.212622in}}{\pgfqpoint{3.696000in}{3.696000in}}%
\pgfusepath{clip}%
\pgfsetrectcap%
\pgfsetroundjoin%
\pgfsetlinewidth{1.505625pt}%
\definecolor{currentstroke}{rgb}{1.000000,0.000000,0.000000}%
\pgfsetstrokecolor{currentstroke}%
\pgfsetdash{}{0pt}%
\pgfpathmoveto{\pgfqpoint{1.537107in}{1.748001in}}%
\pgfpathlineto{\pgfqpoint{1.256551in}{1.818527in}}%
\pgfusepath{stroke}%
\end{pgfscope}%
\begin{pgfscope}%
\pgfpathrectangle{\pgfqpoint{0.100000in}{0.212622in}}{\pgfqpoint{3.696000in}{3.696000in}}%
\pgfusepath{clip}%
\pgfsetrectcap%
\pgfsetroundjoin%
\pgfsetlinewidth{1.505625pt}%
\definecolor{currentstroke}{rgb}{1.000000,0.000000,0.000000}%
\pgfsetstrokecolor{currentstroke}%
\pgfsetdash{}{0pt}%
\pgfpathmoveto{\pgfqpoint{1.539008in}{1.746929in}}%
\pgfpathlineto{\pgfqpoint{1.256551in}{1.818527in}}%
\pgfusepath{stroke}%
\end{pgfscope}%
\begin{pgfscope}%
\pgfpathrectangle{\pgfqpoint{0.100000in}{0.212622in}}{\pgfqpoint{3.696000in}{3.696000in}}%
\pgfusepath{clip}%
\pgfsetrectcap%
\pgfsetroundjoin%
\pgfsetlinewidth{1.505625pt}%
\definecolor{currentstroke}{rgb}{1.000000,0.000000,0.000000}%
\pgfsetstrokecolor{currentstroke}%
\pgfsetdash{}{0pt}%
\pgfpathmoveto{\pgfqpoint{1.539774in}{1.748367in}}%
\pgfpathlineto{\pgfqpoint{1.256551in}{1.818527in}}%
\pgfusepath{stroke}%
\end{pgfscope}%
\begin{pgfscope}%
\pgfpathrectangle{\pgfqpoint{0.100000in}{0.212622in}}{\pgfqpoint{3.696000in}{3.696000in}}%
\pgfusepath{clip}%
\pgfsetrectcap%
\pgfsetroundjoin%
\pgfsetlinewidth{1.505625pt}%
\definecolor{currentstroke}{rgb}{1.000000,0.000000,0.000000}%
\pgfsetstrokecolor{currentstroke}%
\pgfsetdash{}{0pt}%
\pgfpathmoveto{\pgfqpoint{1.539924in}{1.748394in}}%
\pgfpathlineto{\pgfqpoint{1.246568in}{1.810130in}}%
\pgfusepath{stroke}%
\end{pgfscope}%
\begin{pgfscope}%
\pgfpathrectangle{\pgfqpoint{0.100000in}{0.212622in}}{\pgfqpoint{3.696000in}{3.696000in}}%
\pgfusepath{clip}%
\pgfsetrectcap%
\pgfsetroundjoin%
\pgfsetlinewidth{1.505625pt}%
\definecolor{currentstroke}{rgb}{1.000000,0.000000,0.000000}%
\pgfsetstrokecolor{currentstroke}%
\pgfsetdash{}{0pt}%
\pgfpathmoveto{\pgfqpoint{1.540324in}{1.747340in}}%
\pgfpathlineto{\pgfqpoint{1.246568in}{1.810130in}}%
\pgfusepath{stroke}%
\end{pgfscope}%
\begin{pgfscope}%
\pgfpathrectangle{\pgfqpoint{0.100000in}{0.212622in}}{\pgfqpoint{3.696000in}{3.696000in}}%
\pgfusepath{clip}%
\pgfsetrectcap%
\pgfsetroundjoin%
\pgfsetlinewidth{1.505625pt}%
\definecolor{currentstroke}{rgb}{1.000000,0.000000,0.000000}%
\pgfsetstrokecolor{currentstroke}%
\pgfsetdash{}{0pt}%
\pgfpathmoveto{\pgfqpoint{1.541074in}{1.747842in}}%
\pgfpathlineto{\pgfqpoint{1.246568in}{1.810130in}}%
\pgfusepath{stroke}%
\end{pgfscope}%
\begin{pgfscope}%
\pgfpathrectangle{\pgfqpoint{0.100000in}{0.212622in}}{\pgfqpoint{3.696000in}{3.696000in}}%
\pgfusepath{clip}%
\pgfsetrectcap%
\pgfsetroundjoin%
\pgfsetlinewidth{1.505625pt}%
\definecolor{currentstroke}{rgb}{1.000000,0.000000,0.000000}%
\pgfsetstrokecolor{currentstroke}%
\pgfsetdash{}{0pt}%
\pgfpathmoveto{\pgfqpoint{1.541630in}{1.747463in}}%
\pgfpathlineto{\pgfqpoint{1.246568in}{1.810130in}}%
\pgfusepath{stroke}%
\end{pgfscope}%
\begin{pgfscope}%
\pgfpathrectangle{\pgfqpoint{0.100000in}{0.212622in}}{\pgfqpoint{3.696000in}{3.696000in}}%
\pgfusepath{clip}%
\pgfsetrectcap%
\pgfsetroundjoin%
\pgfsetlinewidth{1.505625pt}%
\definecolor{currentstroke}{rgb}{1.000000,0.000000,0.000000}%
\pgfsetstrokecolor{currentstroke}%
\pgfsetdash{}{0pt}%
\pgfpathmoveto{\pgfqpoint{1.541902in}{1.747250in}}%
\pgfpathlineto{\pgfqpoint{1.246568in}{1.810130in}}%
\pgfusepath{stroke}%
\end{pgfscope}%
\begin{pgfscope}%
\pgfpathrectangle{\pgfqpoint{0.100000in}{0.212622in}}{\pgfqpoint{3.696000in}{3.696000in}}%
\pgfusepath{clip}%
\pgfsetrectcap%
\pgfsetroundjoin%
\pgfsetlinewidth{1.505625pt}%
\definecolor{currentstroke}{rgb}{1.000000,0.000000,0.000000}%
\pgfsetstrokecolor{currentstroke}%
\pgfsetdash{}{0pt}%
\pgfpathmoveto{\pgfqpoint{1.542181in}{1.747134in}}%
\pgfpathlineto{\pgfqpoint{1.246568in}{1.810130in}}%
\pgfusepath{stroke}%
\end{pgfscope}%
\begin{pgfscope}%
\pgfpathrectangle{\pgfqpoint{0.100000in}{0.212622in}}{\pgfqpoint{3.696000in}{3.696000in}}%
\pgfusepath{clip}%
\pgfsetrectcap%
\pgfsetroundjoin%
\pgfsetlinewidth{1.505625pt}%
\definecolor{currentstroke}{rgb}{1.000000,0.000000,0.000000}%
\pgfsetstrokecolor{currentstroke}%
\pgfsetdash{}{0pt}%
\pgfpathmoveto{\pgfqpoint{1.542298in}{1.747530in}}%
\pgfpathlineto{\pgfqpoint{1.246568in}{1.810130in}}%
\pgfusepath{stroke}%
\end{pgfscope}%
\begin{pgfscope}%
\pgfpathrectangle{\pgfqpoint{0.100000in}{0.212622in}}{\pgfqpoint{3.696000in}{3.696000in}}%
\pgfusepath{clip}%
\pgfsetrectcap%
\pgfsetroundjoin%
\pgfsetlinewidth{1.505625pt}%
\definecolor{currentstroke}{rgb}{1.000000,0.000000,0.000000}%
\pgfsetstrokecolor{currentstroke}%
\pgfsetdash{}{0pt}%
\pgfpathmoveto{\pgfqpoint{1.542343in}{1.747440in}}%
\pgfpathlineto{\pgfqpoint{1.236570in}{1.801721in}}%
\pgfusepath{stroke}%
\end{pgfscope}%
\begin{pgfscope}%
\pgfpathrectangle{\pgfqpoint{0.100000in}{0.212622in}}{\pgfqpoint{3.696000in}{3.696000in}}%
\pgfusepath{clip}%
\pgfsetrectcap%
\pgfsetroundjoin%
\pgfsetlinewidth{1.505625pt}%
\definecolor{currentstroke}{rgb}{1.000000,0.000000,0.000000}%
\pgfsetstrokecolor{currentstroke}%
\pgfsetdash{}{0pt}%
\pgfpathmoveto{\pgfqpoint{1.542646in}{1.746771in}}%
\pgfpathlineto{\pgfqpoint{1.236570in}{1.801721in}}%
\pgfusepath{stroke}%
\end{pgfscope}%
\begin{pgfscope}%
\pgfpathrectangle{\pgfqpoint{0.100000in}{0.212622in}}{\pgfqpoint{3.696000in}{3.696000in}}%
\pgfusepath{clip}%
\pgfsetrectcap%
\pgfsetroundjoin%
\pgfsetlinewidth{1.505625pt}%
\definecolor{currentstroke}{rgb}{1.000000,0.000000,0.000000}%
\pgfsetstrokecolor{currentstroke}%
\pgfsetdash{}{0pt}%
\pgfpathmoveto{\pgfqpoint{1.542871in}{1.747003in}}%
\pgfpathlineto{\pgfqpoint{1.236570in}{1.801721in}}%
\pgfusepath{stroke}%
\end{pgfscope}%
\begin{pgfscope}%
\pgfpathrectangle{\pgfqpoint{0.100000in}{0.212622in}}{\pgfqpoint{3.696000in}{3.696000in}}%
\pgfusepath{clip}%
\pgfsetrectcap%
\pgfsetroundjoin%
\pgfsetlinewidth{1.505625pt}%
\definecolor{currentstroke}{rgb}{1.000000,0.000000,0.000000}%
\pgfsetstrokecolor{currentstroke}%
\pgfsetdash{}{0pt}%
\pgfpathmoveto{\pgfqpoint{1.543207in}{1.747074in}}%
\pgfpathlineto{\pgfqpoint{1.236570in}{1.801721in}}%
\pgfusepath{stroke}%
\end{pgfscope}%
\begin{pgfscope}%
\pgfpathrectangle{\pgfqpoint{0.100000in}{0.212622in}}{\pgfqpoint{3.696000in}{3.696000in}}%
\pgfusepath{clip}%
\pgfsetrectcap%
\pgfsetroundjoin%
\pgfsetlinewidth{1.505625pt}%
\definecolor{currentstroke}{rgb}{1.000000,0.000000,0.000000}%
\pgfsetstrokecolor{currentstroke}%
\pgfsetdash{}{0pt}%
\pgfpathmoveto{\pgfqpoint{1.543540in}{1.746921in}}%
\pgfpathlineto{\pgfqpoint{1.236570in}{1.801721in}}%
\pgfusepath{stroke}%
\end{pgfscope}%
\begin{pgfscope}%
\pgfpathrectangle{\pgfqpoint{0.100000in}{0.212622in}}{\pgfqpoint{3.696000in}{3.696000in}}%
\pgfusepath{clip}%
\pgfsetrectcap%
\pgfsetroundjoin%
\pgfsetlinewidth{1.505625pt}%
\definecolor{currentstroke}{rgb}{1.000000,0.000000,0.000000}%
\pgfsetstrokecolor{currentstroke}%
\pgfsetdash{}{0pt}%
\pgfpathmoveto{\pgfqpoint{1.543998in}{1.746335in}}%
\pgfpathlineto{\pgfqpoint{1.236570in}{1.801721in}}%
\pgfusepath{stroke}%
\end{pgfscope}%
\begin{pgfscope}%
\pgfpathrectangle{\pgfqpoint{0.100000in}{0.212622in}}{\pgfqpoint{3.696000in}{3.696000in}}%
\pgfusepath{clip}%
\pgfsetrectcap%
\pgfsetroundjoin%
\pgfsetlinewidth{1.505625pt}%
\definecolor{currentstroke}{rgb}{1.000000,0.000000,0.000000}%
\pgfsetstrokecolor{currentstroke}%
\pgfsetdash{}{0pt}%
\pgfpathmoveto{\pgfqpoint{1.545008in}{1.747091in}}%
\pgfpathlineto{\pgfqpoint{1.236570in}{1.801721in}}%
\pgfusepath{stroke}%
\end{pgfscope}%
\begin{pgfscope}%
\pgfpathrectangle{\pgfqpoint{0.100000in}{0.212622in}}{\pgfqpoint{3.696000in}{3.696000in}}%
\pgfusepath{clip}%
\pgfsetrectcap%
\pgfsetroundjoin%
\pgfsetlinewidth{1.505625pt}%
\definecolor{currentstroke}{rgb}{1.000000,0.000000,0.000000}%
\pgfsetstrokecolor{currentstroke}%
\pgfsetdash{}{0pt}%
\pgfpathmoveto{\pgfqpoint{1.545364in}{1.745983in}}%
\pgfpathlineto{\pgfqpoint{1.236570in}{1.801721in}}%
\pgfusepath{stroke}%
\end{pgfscope}%
\begin{pgfscope}%
\pgfpathrectangle{\pgfqpoint{0.100000in}{0.212622in}}{\pgfqpoint{3.696000in}{3.696000in}}%
\pgfusepath{clip}%
\pgfsetrectcap%
\pgfsetroundjoin%
\pgfsetlinewidth{1.505625pt}%
\definecolor{currentstroke}{rgb}{1.000000,0.000000,0.000000}%
\pgfsetstrokecolor{currentstroke}%
\pgfsetdash{}{0pt}%
\pgfpathmoveto{\pgfqpoint{1.545903in}{1.749048in}}%
\pgfpathlineto{\pgfqpoint{1.226558in}{1.793299in}}%
\pgfusepath{stroke}%
\end{pgfscope}%
\begin{pgfscope}%
\pgfpathrectangle{\pgfqpoint{0.100000in}{0.212622in}}{\pgfqpoint{3.696000in}{3.696000in}}%
\pgfusepath{clip}%
\pgfsetrectcap%
\pgfsetroundjoin%
\pgfsetlinewidth{1.505625pt}%
\definecolor{currentstroke}{rgb}{1.000000,0.000000,0.000000}%
\pgfsetstrokecolor{currentstroke}%
\pgfsetdash{}{0pt}%
\pgfpathmoveto{\pgfqpoint{1.546798in}{1.749021in}}%
\pgfpathlineto{\pgfqpoint{1.226558in}{1.793299in}}%
\pgfusepath{stroke}%
\end{pgfscope}%
\begin{pgfscope}%
\pgfpathrectangle{\pgfqpoint{0.100000in}{0.212622in}}{\pgfqpoint{3.696000in}{3.696000in}}%
\pgfusepath{clip}%
\pgfsetrectcap%
\pgfsetroundjoin%
\pgfsetlinewidth{1.505625pt}%
\definecolor{currentstroke}{rgb}{1.000000,0.000000,0.000000}%
\pgfsetstrokecolor{currentstroke}%
\pgfsetdash{}{0pt}%
\pgfpathmoveto{\pgfqpoint{1.547478in}{1.749693in}}%
\pgfpathlineto{\pgfqpoint{1.226558in}{1.793299in}}%
\pgfusepath{stroke}%
\end{pgfscope}%
\begin{pgfscope}%
\pgfpathrectangle{\pgfqpoint{0.100000in}{0.212622in}}{\pgfqpoint{3.696000in}{3.696000in}}%
\pgfusepath{clip}%
\pgfsetrectcap%
\pgfsetroundjoin%
\pgfsetlinewidth{1.505625pt}%
\definecolor{currentstroke}{rgb}{1.000000,0.000000,0.000000}%
\pgfsetstrokecolor{currentstroke}%
\pgfsetdash{}{0pt}%
\pgfpathmoveto{\pgfqpoint{1.547743in}{1.749290in}}%
\pgfpathlineto{\pgfqpoint{1.226558in}{1.793299in}}%
\pgfusepath{stroke}%
\end{pgfscope}%
\begin{pgfscope}%
\pgfpathrectangle{\pgfqpoint{0.100000in}{0.212622in}}{\pgfqpoint{3.696000in}{3.696000in}}%
\pgfusepath{clip}%
\pgfsetrectcap%
\pgfsetroundjoin%
\pgfsetlinewidth{1.505625pt}%
\definecolor{currentstroke}{rgb}{1.000000,0.000000,0.000000}%
\pgfsetstrokecolor{currentstroke}%
\pgfsetdash{}{0pt}%
\pgfpathmoveto{\pgfqpoint{1.548190in}{1.749412in}}%
\pgfpathlineto{\pgfqpoint{1.226558in}{1.793299in}}%
\pgfusepath{stroke}%
\end{pgfscope}%
\begin{pgfscope}%
\pgfpathrectangle{\pgfqpoint{0.100000in}{0.212622in}}{\pgfqpoint{3.696000in}{3.696000in}}%
\pgfusepath{clip}%
\pgfsetrectcap%
\pgfsetroundjoin%
\pgfsetlinewidth{1.505625pt}%
\definecolor{currentstroke}{rgb}{1.000000,0.000000,0.000000}%
\pgfsetstrokecolor{currentstroke}%
\pgfsetdash{}{0pt}%
\pgfpathmoveto{\pgfqpoint{1.548567in}{1.750081in}}%
\pgfpathlineto{\pgfqpoint{1.226558in}{1.793299in}}%
\pgfusepath{stroke}%
\end{pgfscope}%
\begin{pgfscope}%
\pgfpathrectangle{\pgfqpoint{0.100000in}{0.212622in}}{\pgfqpoint{3.696000in}{3.696000in}}%
\pgfusepath{clip}%
\pgfsetrectcap%
\pgfsetroundjoin%
\pgfsetlinewidth{1.505625pt}%
\definecolor{currentstroke}{rgb}{1.000000,0.000000,0.000000}%
\pgfsetstrokecolor{currentstroke}%
\pgfsetdash{}{0pt}%
\pgfpathmoveto{\pgfqpoint{1.548515in}{1.750020in}}%
\pgfpathlineto{\pgfqpoint{1.226558in}{1.793299in}}%
\pgfusepath{stroke}%
\end{pgfscope}%
\begin{pgfscope}%
\pgfpathrectangle{\pgfqpoint{0.100000in}{0.212622in}}{\pgfqpoint{3.696000in}{3.696000in}}%
\pgfusepath{clip}%
\pgfsetrectcap%
\pgfsetroundjoin%
\pgfsetlinewidth{1.505625pt}%
\definecolor{currentstroke}{rgb}{1.000000,0.000000,0.000000}%
\pgfsetstrokecolor{currentstroke}%
\pgfsetdash{}{0pt}%
\pgfpathmoveto{\pgfqpoint{1.549329in}{1.748507in}}%
\pgfpathlineto{\pgfqpoint{1.216532in}{1.784866in}}%
\pgfusepath{stroke}%
\end{pgfscope}%
\begin{pgfscope}%
\pgfpathrectangle{\pgfqpoint{0.100000in}{0.212622in}}{\pgfqpoint{3.696000in}{3.696000in}}%
\pgfusepath{clip}%
\pgfsetrectcap%
\pgfsetroundjoin%
\pgfsetlinewidth{1.505625pt}%
\definecolor{currentstroke}{rgb}{1.000000,0.000000,0.000000}%
\pgfsetstrokecolor{currentstroke}%
\pgfsetdash{}{0pt}%
\pgfpathmoveto{\pgfqpoint{1.549744in}{1.749048in}}%
\pgfpathlineto{\pgfqpoint{1.216532in}{1.784866in}}%
\pgfusepath{stroke}%
\end{pgfscope}%
\begin{pgfscope}%
\pgfpathrectangle{\pgfqpoint{0.100000in}{0.212622in}}{\pgfqpoint{3.696000in}{3.696000in}}%
\pgfusepath{clip}%
\pgfsetrectcap%
\pgfsetroundjoin%
\pgfsetlinewidth{1.505625pt}%
\definecolor{currentstroke}{rgb}{1.000000,0.000000,0.000000}%
\pgfsetstrokecolor{currentstroke}%
\pgfsetdash{}{0pt}%
\pgfpathmoveto{\pgfqpoint{1.550036in}{1.748263in}}%
\pgfpathlineto{\pgfqpoint{1.216532in}{1.784866in}}%
\pgfusepath{stroke}%
\end{pgfscope}%
\begin{pgfscope}%
\pgfpathrectangle{\pgfqpoint{0.100000in}{0.212622in}}{\pgfqpoint{3.696000in}{3.696000in}}%
\pgfusepath{clip}%
\pgfsetrectcap%
\pgfsetroundjoin%
\pgfsetlinewidth{1.505625pt}%
\definecolor{currentstroke}{rgb}{1.000000,0.000000,0.000000}%
\pgfsetstrokecolor{currentstroke}%
\pgfsetdash{}{0pt}%
\pgfpathmoveto{\pgfqpoint{1.550493in}{1.749077in}}%
\pgfpathlineto{\pgfqpoint{1.216532in}{1.784866in}}%
\pgfusepath{stroke}%
\end{pgfscope}%
\begin{pgfscope}%
\pgfpathrectangle{\pgfqpoint{0.100000in}{0.212622in}}{\pgfqpoint{3.696000in}{3.696000in}}%
\pgfusepath{clip}%
\pgfsetrectcap%
\pgfsetroundjoin%
\pgfsetlinewidth{1.505625pt}%
\definecolor{currentstroke}{rgb}{1.000000,0.000000,0.000000}%
\pgfsetstrokecolor{currentstroke}%
\pgfsetdash{}{0pt}%
\pgfpathmoveto{\pgfqpoint{1.551183in}{1.748174in}}%
\pgfpathlineto{\pgfqpoint{1.216532in}{1.784866in}}%
\pgfusepath{stroke}%
\end{pgfscope}%
\begin{pgfscope}%
\pgfpathrectangle{\pgfqpoint{0.100000in}{0.212622in}}{\pgfqpoint{3.696000in}{3.696000in}}%
\pgfusepath{clip}%
\pgfsetrectcap%
\pgfsetroundjoin%
\pgfsetlinewidth{1.505625pt}%
\definecolor{currentstroke}{rgb}{1.000000,0.000000,0.000000}%
\pgfsetstrokecolor{currentstroke}%
\pgfsetdash{}{0pt}%
\pgfpathmoveto{\pgfqpoint{1.552271in}{1.748184in}}%
\pgfpathlineto{\pgfqpoint{1.216532in}{1.784866in}}%
\pgfusepath{stroke}%
\end{pgfscope}%
\begin{pgfscope}%
\pgfpathrectangle{\pgfqpoint{0.100000in}{0.212622in}}{\pgfqpoint{3.696000in}{3.696000in}}%
\pgfusepath{clip}%
\pgfsetrectcap%
\pgfsetroundjoin%
\pgfsetlinewidth{1.505625pt}%
\definecolor{currentstroke}{rgb}{1.000000,0.000000,0.000000}%
\pgfsetstrokecolor{currentstroke}%
\pgfsetdash{}{0pt}%
\pgfpathmoveto{\pgfqpoint{1.553639in}{1.745222in}}%
\pgfpathlineto{\pgfqpoint{1.206492in}{1.776421in}}%
\pgfusepath{stroke}%
\end{pgfscope}%
\begin{pgfscope}%
\pgfpathrectangle{\pgfqpoint{0.100000in}{0.212622in}}{\pgfqpoint{3.696000in}{3.696000in}}%
\pgfusepath{clip}%
\pgfsetrectcap%
\pgfsetroundjoin%
\pgfsetlinewidth{1.505625pt}%
\definecolor{currentstroke}{rgb}{1.000000,0.000000,0.000000}%
\pgfsetstrokecolor{currentstroke}%
\pgfsetdash{}{0pt}%
\pgfpathmoveto{\pgfqpoint{1.554065in}{1.746091in}}%
\pgfpathlineto{\pgfqpoint{1.206492in}{1.776421in}}%
\pgfusepath{stroke}%
\end{pgfscope}%
\begin{pgfscope}%
\pgfpathrectangle{\pgfqpoint{0.100000in}{0.212622in}}{\pgfqpoint{3.696000in}{3.696000in}}%
\pgfusepath{clip}%
\pgfsetrectcap%
\pgfsetroundjoin%
\pgfsetlinewidth{1.505625pt}%
\definecolor{currentstroke}{rgb}{1.000000,0.000000,0.000000}%
\pgfsetstrokecolor{currentstroke}%
\pgfsetdash{}{0pt}%
\pgfpathmoveto{\pgfqpoint{1.554839in}{1.745031in}}%
\pgfpathlineto{\pgfqpoint{1.206492in}{1.776421in}}%
\pgfusepath{stroke}%
\end{pgfscope}%
\begin{pgfscope}%
\pgfpathrectangle{\pgfqpoint{0.100000in}{0.212622in}}{\pgfqpoint{3.696000in}{3.696000in}}%
\pgfusepath{clip}%
\pgfsetrectcap%
\pgfsetroundjoin%
\pgfsetlinewidth{1.505625pt}%
\definecolor{currentstroke}{rgb}{1.000000,0.000000,0.000000}%
\pgfsetstrokecolor{currentstroke}%
\pgfsetdash{}{0pt}%
\pgfpathmoveto{\pgfqpoint{1.556228in}{1.746156in}}%
\pgfpathlineto{\pgfqpoint{1.206492in}{1.776421in}}%
\pgfusepath{stroke}%
\end{pgfscope}%
\begin{pgfscope}%
\pgfpathrectangle{\pgfqpoint{0.100000in}{0.212622in}}{\pgfqpoint{3.696000in}{3.696000in}}%
\pgfusepath{clip}%
\pgfsetrectcap%
\pgfsetroundjoin%
\pgfsetlinewidth{1.505625pt}%
\definecolor{currentstroke}{rgb}{1.000000,0.000000,0.000000}%
\pgfsetstrokecolor{currentstroke}%
\pgfsetdash{}{0pt}%
\pgfpathmoveto{\pgfqpoint{1.556615in}{1.748679in}}%
\pgfpathlineto{\pgfqpoint{1.196437in}{1.767964in}}%
\pgfusepath{stroke}%
\end{pgfscope}%
\begin{pgfscope}%
\pgfpathrectangle{\pgfqpoint{0.100000in}{0.212622in}}{\pgfqpoint{3.696000in}{3.696000in}}%
\pgfusepath{clip}%
\pgfsetrectcap%
\pgfsetroundjoin%
\pgfsetlinewidth{1.505625pt}%
\definecolor{currentstroke}{rgb}{1.000000,0.000000,0.000000}%
\pgfsetstrokecolor{currentstroke}%
\pgfsetdash{}{0pt}%
\pgfpathmoveto{\pgfqpoint{1.558219in}{1.746123in}}%
\pgfpathlineto{\pgfqpoint{1.196437in}{1.767964in}}%
\pgfusepath{stroke}%
\end{pgfscope}%
\begin{pgfscope}%
\pgfpathrectangle{\pgfqpoint{0.100000in}{0.212622in}}{\pgfqpoint{3.696000in}{3.696000in}}%
\pgfusepath{clip}%
\pgfsetrectcap%
\pgfsetroundjoin%
\pgfsetlinewidth{1.505625pt}%
\definecolor{currentstroke}{rgb}{1.000000,0.000000,0.000000}%
\pgfsetstrokecolor{currentstroke}%
\pgfsetdash{}{0pt}%
\pgfpathmoveto{\pgfqpoint{1.559784in}{1.744675in}}%
\pgfpathlineto{\pgfqpoint{1.196437in}{1.767964in}}%
\pgfusepath{stroke}%
\end{pgfscope}%
\begin{pgfscope}%
\pgfpathrectangle{\pgfqpoint{0.100000in}{0.212622in}}{\pgfqpoint{3.696000in}{3.696000in}}%
\pgfusepath{clip}%
\pgfsetrectcap%
\pgfsetroundjoin%
\pgfsetlinewidth{1.505625pt}%
\definecolor{currentstroke}{rgb}{1.000000,0.000000,0.000000}%
\pgfsetstrokecolor{currentstroke}%
\pgfsetdash{}{0pt}%
\pgfpathmoveto{\pgfqpoint{1.560427in}{1.742337in}}%
\pgfpathlineto{\pgfqpoint{1.186368in}{1.759494in}}%
\pgfusepath{stroke}%
\end{pgfscope}%
\begin{pgfscope}%
\pgfpathrectangle{\pgfqpoint{0.100000in}{0.212622in}}{\pgfqpoint{3.696000in}{3.696000in}}%
\pgfusepath{clip}%
\pgfsetrectcap%
\pgfsetroundjoin%
\pgfsetlinewidth{1.505625pt}%
\definecolor{currentstroke}{rgb}{1.000000,0.000000,0.000000}%
\pgfsetstrokecolor{currentstroke}%
\pgfsetdash{}{0pt}%
\pgfpathmoveto{\pgfqpoint{1.562478in}{1.739590in}}%
\pgfpathlineto{\pgfqpoint{1.186368in}{1.759494in}}%
\pgfusepath{stroke}%
\end{pgfscope}%
\begin{pgfscope}%
\pgfpathrectangle{\pgfqpoint{0.100000in}{0.212622in}}{\pgfqpoint{3.696000in}{3.696000in}}%
\pgfusepath{clip}%
\pgfsetrectcap%
\pgfsetroundjoin%
\pgfsetlinewidth{1.505625pt}%
\definecolor{currentstroke}{rgb}{1.000000,0.000000,0.000000}%
\pgfsetstrokecolor{currentstroke}%
\pgfsetdash{}{0pt}%
\pgfpathmoveto{\pgfqpoint{1.563427in}{1.739614in}}%
\pgfpathlineto{\pgfqpoint{1.176285in}{1.751013in}}%
\pgfusepath{stroke}%
\end{pgfscope}%
\begin{pgfscope}%
\pgfpathrectangle{\pgfqpoint{0.100000in}{0.212622in}}{\pgfqpoint{3.696000in}{3.696000in}}%
\pgfusepath{clip}%
\pgfsetrectcap%
\pgfsetroundjoin%
\pgfsetlinewidth{1.505625pt}%
\definecolor{currentstroke}{rgb}{1.000000,0.000000,0.000000}%
\pgfsetstrokecolor{currentstroke}%
\pgfsetdash{}{0pt}%
\pgfpathmoveto{\pgfqpoint{1.565078in}{1.740746in}}%
\pgfpathlineto{\pgfqpoint{1.176285in}{1.751013in}}%
\pgfusepath{stroke}%
\end{pgfscope}%
\begin{pgfscope}%
\pgfpathrectangle{\pgfqpoint{0.100000in}{0.212622in}}{\pgfqpoint{3.696000in}{3.696000in}}%
\pgfusepath{clip}%
\pgfsetrectcap%
\pgfsetroundjoin%
\pgfsetlinewidth{1.505625pt}%
\definecolor{currentstroke}{rgb}{1.000000,0.000000,0.000000}%
\pgfsetstrokecolor{currentstroke}%
\pgfsetdash{}{0pt}%
\pgfpathmoveto{\pgfqpoint{1.566419in}{1.740777in}}%
\pgfpathlineto{\pgfqpoint{1.176285in}{1.751013in}}%
\pgfusepath{stroke}%
\end{pgfscope}%
\begin{pgfscope}%
\pgfpathrectangle{\pgfqpoint{0.100000in}{0.212622in}}{\pgfqpoint{3.696000in}{3.696000in}}%
\pgfusepath{clip}%
\pgfsetrectcap%
\pgfsetroundjoin%
\pgfsetlinewidth{1.505625pt}%
\definecolor{currentstroke}{rgb}{1.000000,0.000000,0.000000}%
\pgfsetstrokecolor{currentstroke}%
\pgfsetdash{}{0pt}%
\pgfpathmoveto{\pgfqpoint{1.567748in}{1.742072in}}%
\pgfpathlineto{\pgfqpoint{1.166187in}{1.742519in}}%
\pgfusepath{stroke}%
\end{pgfscope}%
\begin{pgfscope}%
\pgfpathrectangle{\pgfqpoint{0.100000in}{0.212622in}}{\pgfqpoint{3.696000in}{3.696000in}}%
\pgfusepath{clip}%
\pgfsetrectcap%
\pgfsetroundjoin%
\pgfsetlinewidth{1.505625pt}%
\definecolor{currentstroke}{rgb}{1.000000,0.000000,0.000000}%
\pgfsetstrokecolor{currentstroke}%
\pgfsetdash{}{0pt}%
\pgfpathmoveto{\pgfqpoint{1.570980in}{1.745119in}}%
\pgfpathlineto{\pgfqpoint{1.166187in}{1.742519in}}%
\pgfusepath{stroke}%
\end{pgfscope}%
\begin{pgfscope}%
\pgfpathrectangle{\pgfqpoint{0.100000in}{0.212622in}}{\pgfqpoint{3.696000in}{3.696000in}}%
\pgfusepath{clip}%
\pgfsetrectcap%
\pgfsetroundjoin%
\pgfsetlinewidth{1.505625pt}%
\definecolor{currentstroke}{rgb}{1.000000,0.000000,0.000000}%
\pgfsetstrokecolor{currentstroke}%
\pgfsetdash{}{0pt}%
\pgfpathmoveto{\pgfqpoint{1.571628in}{1.742949in}}%
\pgfpathlineto{\pgfqpoint{1.156075in}{1.734014in}}%
\pgfusepath{stroke}%
\end{pgfscope}%
\begin{pgfscope}%
\pgfpathrectangle{\pgfqpoint{0.100000in}{0.212622in}}{\pgfqpoint{3.696000in}{3.696000in}}%
\pgfusepath{clip}%
\pgfsetrectcap%
\pgfsetroundjoin%
\pgfsetlinewidth{1.505625pt}%
\definecolor{currentstroke}{rgb}{1.000000,0.000000,0.000000}%
\pgfsetstrokecolor{currentstroke}%
\pgfsetdash{}{0pt}%
\pgfpathmoveto{\pgfqpoint{1.573459in}{1.739173in}}%
\pgfpathlineto{\pgfqpoint{1.156075in}{1.734014in}}%
\pgfusepath{stroke}%
\end{pgfscope}%
\begin{pgfscope}%
\pgfpathrectangle{\pgfqpoint{0.100000in}{0.212622in}}{\pgfqpoint{3.696000in}{3.696000in}}%
\pgfusepath{clip}%
\pgfsetrectcap%
\pgfsetroundjoin%
\pgfsetlinewidth{1.505625pt}%
\definecolor{currentstroke}{rgb}{1.000000,0.000000,0.000000}%
\pgfsetstrokecolor{currentstroke}%
\pgfsetdash{}{0pt}%
\pgfpathmoveto{\pgfqpoint{1.575235in}{1.740518in}}%
\pgfpathlineto{\pgfqpoint{1.145948in}{1.725496in}}%
\pgfusepath{stroke}%
\end{pgfscope}%
\begin{pgfscope}%
\pgfpathrectangle{\pgfqpoint{0.100000in}{0.212622in}}{\pgfqpoint{3.696000in}{3.696000in}}%
\pgfusepath{clip}%
\pgfsetrectcap%
\pgfsetroundjoin%
\pgfsetlinewidth{1.505625pt}%
\definecolor{currentstroke}{rgb}{1.000000,0.000000,0.000000}%
\pgfsetstrokecolor{currentstroke}%
\pgfsetdash{}{0pt}%
\pgfpathmoveto{\pgfqpoint{1.577128in}{1.740490in}}%
\pgfpathlineto{\pgfqpoint{1.145948in}{1.725496in}}%
\pgfusepath{stroke}%
\end{pgfscope}%
\begin{pgfscope}%
\pgfpathrectangle{\pgfqpoint{0.100000in}{0.212622in}}{\pgfqpoint{3.696000in}{3.696000in}}%
\pgfusepath{clip}%
\pgfsetrectcap%
\pgfsetroundjoin%
\pgfsetlinewidth{1.505625pt}%
\definecolor{currentstroke}{rgb}{1.000000,0.000000,0.000000}%
\pgfsetstrokecolor{currentstroke}%
\pgfsetdash{}{0pt}%
\pgfpathmoveto{\pgfqpoint{1.577977in}{1.738666in}}%
\pgfpathlineto{\pgfqpoint{1.145948in}{1.725496in}}%
\pgfusepath{stroke}%
\end{pgfscope}%
\begin{pgfscope}%
\pgfpathrectangle{\pgfqpoint{0.100000in}{0.212622in}}{\pgfqpoint{3.696000in}{3.696000in}}%
\pgfusepath{clip}%
\pgfsetrectcap%
\pgfsetroundjoin%
\pgfsetlinewidth{1.505625pt}%
\definecolor{currentstroke}{rgb}{1.000000,0.000000,0.000000}%
\pgfsetstrokecolor{currentstroke}%
\pgfsetdash{}{0pt}%
\pgfpathmoveto{\pgfqpoint{1.579788in}{1.734073in}}%
\pgfpathlineto{\pgfqpoint{1.135807in}{1.716966in}}%
\pgfusepath{stroke}%
\end{pgfscope}%
\begin{pgfscope}%
\pgfpathrectangle{\pgfqpoint{0.100000in}{0.212622in}}{\pgfqpoint{3.696000in}{3.696000in}}%
\pgfusepath{clip}%
\pgfsetrectcap%
\pgfsetroundjoin%
\pgfsetlinewidth{1.505625pt}%
\definecolor{currentstroke}{rgb}{1.000000,0.000000,0.000000}%
\pgfsetstrokecolor{currentstroke}%
\pgfsetdash{}{0pt}%
\pgfpathmoveto{\pgfqpoint{1.582122in}{1.735120in}}%
\pgfpathlineto{\pgfqpoint{1.135807in}{1.716966in}}%
\pgfusepath{stroke}%
\end{pgfscope}%
\begin{pgfscope}%
\pgfpathrectangle{\pgfqpoint{0.100000in}{0.212622in}}{\pgfqpoint{3.696000in}{3.696000in}}%
\pgfusepath{clip}%
\pgfsetrectcap%
\pgfsetroundjoin%
\pgfsetlinewidth{1.505625pt}%
\definecolor{currentstroke}{rgb}{1.000000,0.000000,0.000000}%
\pgfsetstrokecolor{currentstroke}%
\pgfsetdash{}{0pt}%
\pgfpathmoveto{\pgfqpoint{1.584679in}{1.734740in}}%
\pgfpathlineto{\pgfqpoint{1.125652in}{1.708424in}}%
\pgfusepath{stroke}%
\end{pgfscope}%
\begin{pgfscope}%
\pgfpathrectangle{\pgfqpoint{0.100000in}{0.212622in}}{\pgfqpoint{3.696000in}{3.696000in}}%
\pgfusepath{clip}%
\pgfsetrectcap%
\pgfsetroundjoin%
\pgfsetlinewidth{1.505625pt}%
\definecolor{currentstroke}{rgb}{1.000000,0.000000,0.000000}%
\pgfsetstrokecolor{currentstroke}%
\pgfsetdash{}{0pt}%
\pgfpathmoveto{\pgfqpoint{1.585524in}{1.734756in}}%
\pgfpathlineto{\pgfqpoint{1.125652in}{1.708424in}}%
\pgfusepath{stroke}%
\end{pgfscope}%
\begin{pgfscope}%
\pgfpathrectangle{\pgfqpoint{0.100000in}{0.212622in}}{\pgfqpoint{3.696000in}{3.696000in}}%
\pgfusepath{clip}%
\pgfsetrectcap%
\pgfsetroundjoin%
\pgfsetlinewidth{1.505625pt}%
\definecolor{currentstroke}{rgb}{1.000000,0.000000,0.000000}%
\pgfsetstrokecolor{currentstroke}%
\pgfsetdash{}{0pt}%
\pgfpathmoveto{\pgfqpoint{1.588696in}{1.733765in}}%
\pgfpathlineto{\pgfqpoint{1.115481in}{1.699869in}}%
\pgfusepath{stroke}%
\end{pgfscope}%
\begin{pgfscope}%
\pgfpathrectangle{\pgfqpoint{0.100000in}{0.212622in}}{\pgfqpoint{3.696000in}{3.696000in}}%
\pgfusepath{clip}%
\pgfsetrectcap%
\pgfsetroundjoin%
\pgfsetlinewidth{1.505625pt}%
\definecolor{currentstroke}{rgb}{1.000000,0.000000,0.000000}%
\pgfsetstrokecolor{currentstroke}%
\pgfsetdash{}{0pt}%
\pgfpathmoveto{\pgfqpoint{1.590071in}{1.732816in}}%
\pgfpathlineto{\pgfqpoint{1.115481in}{1.699869in}}%
\pgfusepath{stroke}%
\end{pgfscope}%
\begin{pgfscope}%
\pgfpathrectangle{\pgfqpoint{0.100000in}{0.212622in}}{\pgfqpoint{3.696000in}{3.696000in}}%
\pgfusepath{clip}%
\pgfsetrectcap%
\pgfsetroundjoin%
\pgfsetlinewidth{1.505625pt}%
\definecolor{currentstroke}{rgb}{1.000000,0.000000,0.000000}%
\pgfsetstrokecolor{currentstroke}%
\pgfsetdash{}{0pt}%
\pgfpathmoveto{\pgfqpoint{1.591811in}{1.733077in}}%
\pgfpathlineto{\pgfqpoint{1.115481in}{1.699869in}}%
\pgfusepath{stroke}%
\end{pgfscope}%
\begin{pgfscope}%
\pgfpathrectangle{\pgfqpoint{0.100000in}{0.212622in}}{\pgfqpoint{3.696000in}{3.696000in}}%
\pgfusepath{clip}%
\pgfsetrectcap%
\pgfsetroundjoin%
\pgfsetlinewidth{1.505625pt}%
\definecolor{currentstroke}{rgb}{1.000000,0.000000,0.000000}%
\pgfsetstrokecolor{currentstroke}%
\pgfsetdash{}{0pt}%
\pgfpathmoveto{\pgfqpoint{1.592646in}{1.732527in}}%
\pgfpathlineto{\pgfqpoint{1.115481in}{1.699869in}}%
\pgfusepath{stroke}%
\end{pgfscope}%
\begin{pgfscope}%
\pgfpathrectangle{\pgfqpoint{0.100000in}{0.212622in}}{\pgfqpoint{3.696000in}{3.696000in}}%
\pgfusepath{clip}%
\pgfsetrectcap%
\pgfsetroundjoin%
\pgfsetlinewidth{1.505625pt}%
\definecolor{currentstroke}{rgb}{1.000000,0.000000,0.000000}%
\pgfsetstrokecolor{currentstroke}%
\pgfsetdash{}{0pt}%
\pgfpathmoveto{\pgfqpoint{1.593095in}{1.731869in}}%
\pgfpathlineto{\pgfqpoint{1.105297in}{1.691302in}}%
\pgfusepath{stroke}%
\end{pgfscope}%
\begin{pgfscope}%
\pgfpathrectangle{\pgfqpoint{0.100000in}{0.212622in}}{\pgfqpoint{3.696000in}{3.696000in}}%
\pgfusepath{clip}%
\pgfsetrectcap%
\pgfsetroundjoin%
\pgfsetlinewidth{1.505625pt}%
\definecolor{currentstroke}{rgb}{1.000000,0.000000,0.000000}%
\pgfsetstrokecolor{currentstroke}%
\pgfsetdash{}{0pt}%
\pgfpathmoveto{\pgfqpoint{1.594026in}{1.731375in}}%
\pgfpathlineto{\pgfqpoint{1.105297in}{1.691302in}}%
\pgfusepath{stroke}%
\end{pgfscope}%
\begin{pgfscope}%
\pgfpathrectangle{\pgfqpoint{0.100000in}{0.212622in}}{\pgfqpoint{3.696000in}{3.696000in}}%
\pgfusepath{clip}%
\pgfsetrectcap%
\pgfsetroundjoin%
\pgfsetlinewidth{1.505625pt}%
\definecolor{currentstroke}{rgb}{1.000000,0.000000,0.000000}%
\pgfsetstrokecolor{currentstroke}%
\pgfsetdash{}{0pt}%
\pgfpathmoveto{\pgfqpoint{1.594326in}{1.730866in}}%
\pgfpathlineto{\pgfqpoint{1.105297in}{1.691302in}}%
\pgfusepath{stroke}%
\end{pgfscope}%
\begin{pgfscope}%
\pgfpathrectangle{\pgfqpoint{0.100000in}{0.212622in}}{\pgfqpoint{3.696000in}{3.696000in}}%
\pgfusepath{clip}%
\pgfsetrectcap%
\pgfsetroundjoin%
\pgfsetlinewidth{1.505625pt}%
\definecolor{currentstroke}{rgb}{1.000000,0.000000,0.000000}%
\pgfsetstrokecolor{currentstroke}%
\pgfsetdash{}{0pt}%
\pgfpathmoveto{\pgfqpoint{1.595711in}{1.730070in}}%
\pgfpathlineto{\pgfqpoint{1.105297in}{1.691302in}}%
\pgfusepath{stroke}%
\end{pgfscope}%
\begin{pgfscope}%
\pgfpathrectangle{\pgfqpoint{0.100000in}{0.212622in}}{\pgfqpoint{3.696000in}{3.696000in}}%
\pgfusepath{clip}%
\pgfsetrectcap%
\pgfsetroundjoin%
\pgfsetlinewidth{1.505625pt}%
\definecolor{currentstroke}{rgb}{1.000000,0.000000,0.000000}%
\pgfsetstrokecolor{currentstroke}%
\pgfsetdash{}{0pt}%
\pgfpathmoveto{\pgfqpoint{1.597008in}{1.730130in}}%
\pgfpathlineto{\pgfqpoint{1.294468in}{1.257828in}}%
\pgfusepath{stroke}%
\end{pgfscope}%
\begin{pgfscope}%
\pgfpathrectangle{\pgfqpoint{0.100000in}{0.212622in}}{\pgfqpoint{3.696000in}{3.696000in}}%
\pgfusepath{clip}%
\pgfsetrectcap%
\pgfsetroundjoin%
\pgfsetlinewidth{1.505625pt}%
\definecolor{currentstroke}{rgb}{1.000000,0.000000,0.000000}%
\pgfsetstrokecolor{currentstroke}%
\pgfsetdash{}{0pt}%
\pgfpathmoveto{\pgfqpoint{1.598416in}{1.728291in}}%
\pgfpathlineto{\pgfqpoint{1.294468in}{1.257828in}}%
\pgfusepath{stroke}%
\end{pgfscope}%
\begin{pgfscope}%
\pgfpathrectangle{\pgfqpoint{0.100000in}{0.212622in}}{\pgfqpoint{3.696000in}{3.696000in}}%
\pgfusepath{clip}%
\pgfsetrectcap%
\pgfsetroundjoin%
\pgfsetlinewidth{1.505625pt}%
\definecolor{currentstroke}{rgb}{1.000000,0.000000,0.000000}%
\pgfsetstrokecolor{currentstroke}%
\pgfsetdash{}{0pt}%
\pgfpathmoveto{\pgfqpoint{1.600175in}{1.726104in}}%
\pgfpathlineto{\pgfqpoint{1.294468in}{1.257828in}}%
\pgfusepath{stroke}%
\end{pgfscope}%
\begin{pgfscope}%
\pgfpathrectangle{\pgfqpoint{0.100000in}{0.212622in}}{\pgfqpoint{3.696000in}{3.696000in}}%
\pgfusepath{clip}%
\pgfsetrectcap%
\pgfsetroundjoin%
\pgfsetlinewidth{1.505625pt}%
\definecolor{currentstroke}{rgb}{1.000000,0.000000,0.000000}%
\pgfsetstrokecolor{currentstroke}%
\pgfsetdash{}{0pt}%
\pgfpathmoveto{\pgfqpoint{1.602032in}{1.722574in}}%
\pgfpathlineto{\pgfqpoint{1.311220in}{1.252541in}}%
\pgfusepath{stroke}%
\end{pgfscope}%
\begin{pgfscope}%
\pgfpathrectangle{\pgfqpoint{0.100000in}{0.212622in}}{\pgfqpoint{3.696000in}{3.696000in}}%
\pgfusepath{clip}%
\pgfsetrectcap%
\pgfsetroundjoin%
\pgfsetlinewidth{1.505625pt}%
\definecolor{currentstroke}{rgb}{1.000000,0.000000,0.000000}%
\pgfsetstrokecolor{currentstroke}%
\pgfsetdash{}{0pt}%
\pgfpathmoveto{\pgfqpoint{1.605844in}{1.724834in}}%
\pgfpathlineto{\pgfqpoint{1.311220in}{1.252541in}}%
\pgfusepath{stroke}%
\end{pgfscope}%
\begin{pgfscope}%
\pgfpathrectangle{\pgfqpoint{0.100000in}{0.212622in}}{\pgfqpoint{3.696000in}{3.696000in}}%
\pgfusepath{clip}%
\pgfsetrectcap%
\pgfsetroundjoin%
\pgfsetlinewidth{1.505625pt}%
\definecolor{currentstroke}{rgb}{1.000000,0.000000,0.000000}%
\pgfsetstrokecolor{currentstroke}%
\pgfsetdash{}{0pt}%
\pgfpathmoveto{\pgfqpoint{1.606169in}{1.723980in}}%
\pgfpathlineto{\pgfqpoint{1.311220in}{1.252541in}}%
\pgfusepath{stroke}%
\end{pgfscope}%
\begin{pgfscope}%
\pgfpathrectangle{\pgfqpoint{0.100000in}{0.212622in}}{\pgfqpoint{3.696000in}{3.696000in}}%
\pgfusepath{clip}%
\pgfsetrectcap%
\pgfsetroundjoin%
\pgfsetlinewidth{1.505625pt}%
\definecolor{currentstroke}{rgb}{1.000000,0.000000,0.000000}%
\pgfsetstrokecolor{currentstroke}%
\pgfsetdash{}{0pt}%
\pgfpathmoveto{\pgfqpoint{1.607679in}{1.722279in}}%
\pgfpathlineto{\pgfqpoint{1.311220in}{1.252541in}}%
\pgfusepath{stroke}%
\end{pgfscope}%
\begin{pgfscope}%
\pgfpathrectangle{\pgfqpoint{0.100000in}{0.212622in}}{\pgfqpoint{3.696000in}{3.696000in}}%
\pgfusepath{clip}%
\pgfsetrectcap%
\pgfsetroundjoin%
\pgfsetlinewidth{1.505625pt}%
\definecolor{currentstroke}{rgb}{1.000000,0.000000,0.000000}%
\pgfsetstrokecolor{currentstroke}%
\pgfsetdash{}{0pt}%
\pgfpathmoveto{\pgfqpoint{1.609629in}{1.720712in}}%
\pgfpathlineto{\pgfqpoint{1.327986in}{1.247250in}}%
\pgfusepath{stroke}%
\end{pgfscope}%
\begin{pgfscope}%
\pgfpathrectangle{\pgfqpoint{0.100000in}{0.212622in}}{\pgfqpoint{3.696000in}{3.696000in}}%
\pgfusepath{clip}%
\pgfsetrectcap%
\pgfsetroundjoin%
\pgfsetlinewidth{1.505625pt}%
\definecolor{currentstroke}{rgb}{1.000000,0.000000,0.000000}%
\pgfsetstrokecolor{currentstroke}%
\pgfsetdash{}{0pt}%
\pgfpathmoveto{\pgfqpoint{1.611030in}{1.720182in}}%
\pgfpathlineto{\pgfqpoint{1.327986in}{1.247250in}}%
\pgfusepath{stroke}%
\end{pgfscope}%
\begin{pgfscope}%
\pgfpathrectangle{\pgfqpoint{0.100000in}{0.212622in}}{\pgfqpoint{3.696000in}{3.696000in}}%
\pgfusepath{clip}%
\pgfsetrectcap%
\pgfsetroundjoin%
\pgfsetlinewidth{1.505625pt}%
\definecolor{currentstroke}{rgb}{1.000000,0.000000,0.000000}%
\pgfsetstrokecolor{currentstroke}%
\pgfsetdash{}{0pt}%
\pgfpathmoveto{\pgfqpoint{1.611876in}{1.719600in}}%
\pgfpathlineto{\pgfqpoint{1.327986in}{1.247250in}}%
\pgfusepath{stroke}%
\end{pgfscope}%
\begin{pgfscope}%
\pgfpathrectangle{\pgfqpoint{0.100000in}{0.212622in}}{\pgfqpoint{3.696000in}{3.696000in}}%
\pgfusepath{clip}%
\pgfsetrectcap%
\pgfsetroundjoin%
\pgfsetlinewidth{1.505625pt}%
\definecolor{currentstroke}{rgb}{1.000000,0.000000,0.000000}%
\pgfsetstrokecolor{currentstroke}%
\pgfsetdash{}{0pt}%
\pgfpathmoveto{\pgfqpoint{1.614397in}{1.720615in}}%
\pgfpathlineto{\pgfqpoint{1.344766in}{1.241955in}}%
\pgfusepath{stroke}%
\end{pgfscope}%
\begin{pgfscope}%
\pgfpathrectangle{\pgfqpoint{0.100000in}{0.212622in}}{\pgfqpoint{3.696000in}{3.696000in}}%
\pgfusepath{clip}%
\pgfsetrectcap%
\pgfsetroundjoin%
\pgfsetlinewidth{1.505625pt}%
\definecolor{currentstroke}{rgb}{1.000000,0.000000,0.000000}%
\pgfsetstrokecolor{currentstroke}%
\pgfsetdash{}{0pt}%
\pgfpathmoveto{\pgfqpoint{1.616770in}{1.720385in}}%
\pgfpathlineto{\pgfqpoint{1.344766in}{1.241955in}}%
\pgfusepath{stroke}%
\end{pgfscope}%
\begin{pgfscope}%
\pgfpathrectangle{\pgfqpoint{0.100000in}{0.212622in}}{\pgfqpoint{3.696000in}{3.696000in}}%
\pgfusepath{clip}%
\pgfsetrectcap%
\pgfsetroundjoin%
\pgfsetlinewidth{1.505625pt}%
\definecolor{currentstroke}{rgb}{1.000000,0.000000,0.000000}%
\pgfsetstrokecolor{currentstroke}%
\pgfsetdash{}{0pt}%
\pgfpathmoveto{\pgfqpoint{1.617596in}{1.718769in}}%
\pgfpathlineto{\pgfqpoint{1.344766in}{1.241955in}}%
\pgfusepath{stroke}%
\end{pgfscope}%
\begin{pgfscope}%
\pgfpathrectangle{\pgfqpoint{0.100000in}{0.212622in}}{\pgfqpoint{3.696000in}{3.696000in}}%
\pgfusepath{clip}%
\pgfsetrectcap%
\pgfsetroundjoin%
\pgfsetlinewidth{1.505625pt}%
\definecolor{currentstroke}{rgb}{1.000000,0.000000,0.000000}%
\pgfsetstrokecolor{currentstroke}%
\pgfsetdash{}{0pt}%
\pgfpathmoveto{\pgfqpoint{1.618278in}{1.717753in}}%
\pgfpathlineto{\pgfqpoint{1.344766in}{1.241955in}}%
\pgfusepath{stroke}%
\end{pgfscope}%
\begin{pgfscope}%
\pgfpathrectangle{\pgfqpoint{0.100000in}{0.212622in}}{\pgfqpoint{3.696000in}{3.696000in}}%
\pgfusepath{clip}%
\pgfsetrectcap%
\pgfsetroundjoin%
\pgfsetlinewidth{1.505625pt}%
\definecolor{currentstroke}{rgb}{1.000000,0.000000,0.000000}%
\pgfsetstrokecolor{currentstroke}%
\pgfsetdash{}{0pt}%
\pgfpathmoveto{\pgfqpoint{1.619009in}{1.717408in}}%
\pgfpathlineto{\pgfqpoint{1.361561in}{1.236655in}}%
\pgfusepath{stroke}%
\end{pgfscope}%
\begin{pgfscope}%
\pgfpathrectangle{\pgfqpoint{0.100000in}{0.212622in}}{\pgfqpoint{3.696000in}{3.696000in}}%
\pgfusepath{clip}%
\pgfsetrectcap%
\pgfsetroundjoin%
\pgfsetlinewidth{1.505625pt}%
\definecolor{currentstroke}{rgb}{1.000000,0.000000,0.000000}%
\pgfsetstrokecolor{currentstroke}%
\pgfsetdash{}{0pt}%
\pgfpathmoveto{\pgfqpoint{1.620046in}{1.717776in}}%
\pgfpathlineto{\pgfqpoint{1.361561in}{1.236655in}}%
\pgfusepath{stroke}%
\end{pgfscope}%
\begin{pgfscope}%
\pgfpathrectangle{\pgfqpoint{0.100000in}{0.212622in}}{\pgfqpoint{3.696000in}{3.696000in}}%
\pgfusepath{clip}%
\pgfsetrectcap%
\pgfsetroundjoin%
\pgfsetlinewidth{1.505625pt}%
\definecolor{currentstroke}{rgb}{1.000000,0.000000,0.000000}%
\pgfsetstrokecolor{currentstroke}%
\pgfsetdash{}{0pt}%
\pgfpathmoveto{\pgfqpoint{1.620527in}{1.717069in}}%
\pgfpathlineto{\pgfqpoint{1.361561in}{1.236655in}}%
\pgfusepath{stroke}%
\end{pgfscope}%
\begin{pgfscope}%
\pgfpathrectangle{\pgfqpoint{0.100000in}{0.212622in}}{\pgfqpoint{3.696000in}{3.696000in}}%
\pgfusepath{clip}%
\pgfsetrectcap%
\pgfsetroundjoin%
\pgfsetlinewidth{1.505625pt}%
\definecolor{currentstroke}{rgb}{1.000000,0.000000,0.000000}%
\pgfsetstrokecolor{currentstroke}%
\pgfsetdash{}{0pt}%
\pgfpathmoveto{\pgfqpoint{1.621144in}{1.716118in}}%
\pgfpathlineto{\pgfqpoint{1.361561in}{1.236655in}}%
\pgfusepath{stroke}%
\end{pgfscope}%
\begin{pgfscope}%
\pgfpathrectangle{\pgfqpoint{0.100000in}{0.212622in}}{\pgfqpoint{3.696000in}{3.696000in}}%
\pgfusepath{clip}%
\pgfsetrectcap%
\pgfsetroundjoin%
\pgfsetlinewidth{1.505625pt}%
\definecolor{currentstroke}{rgb}{1.000000,0.000000,0.000000}%
\pgfsetstrokecolor{currentstroke}%
\pgfsetdash{}{0pt}%
\pgfpathmoveto{\pgfqpoint{1.622081in}{1.715918in}}%
\pgfpathlineto{\pgfqpoint{1.361561in}{1.236655in}}%
\pgfusepath{stroke}%
\end{pgfscope}%
\begin{pgfscope}%
\pgfpathrectangle{\pgfqpoint{0.100000in}{0.212622in}}{\pgfqpoint{3.696000in}{3.696000in}}%
\pgfusepath{clip}%
\pgfsetrectcap%
\pgfsetroundjoin%
\pgfsetlinewidth{1.505625pt}%
\definecolor{currentstroke}{rgb}{1.000000,0.000000,0.000000}%
\pgfsetstrokecolor{currentstroke}%
\pgfsetdash{}{0pt}%
\pgfpathmoveto{\pgfqpoint{1.623727in}{1.717688in}}%
\pgfpathlineto{\pgfqpoint{1.361561in}{1.236655in}}%
\pgfusepath{stroke}%
\end{pgfscope}%
\begin{pgfscope}%
\pgfpathrectangle{\pgfqpoint{0.100000in}{0.212622in}}{\pgfqpoint{3.696000in}{3.696000in}}%
\pgfusepath{clip}%
\pgfsetrectcap%
\pgfsetroundjoin%
\pgfsetlinewidth{1.505625pt}%
\definecolor{currentstroke}{rgb}{1.000000,0.000000,0.000000}%
\pgfsetstrokecolor{currentstroke}%
\pgfsetdash{}{0pt}%
\pgfpathmoveto{\pgfqpoint{1.624506in}{1.716348in}}%
\pgfpathlineto{\pgfqpoint{1.378370in}{1.231350in}}%
\pgfusepath{stroke}%
\end{pgfscope}%
\begin{pgfscope}%
\pgfpathrectangle{\pgfqpoint{0.100000in}{0.212622in}}{\pgfqpoint{3.696000in}{3.696000in}}%
\pgfusepath{clip}%
\pgfsetrectcap%
\pgfsetroundjoin%
\pgfsetlinewidth{1.505625pt}%
\definecolor{currentstroke}{rgb}{1.000000,0.000000,0.000000}%
\pgfsetstrokecolor{currentstroke}%
\pgfsetdash{}{0pt}%
\pgfpathmoveto{\pgfqpoint{1.626271in}{1.712606in}}%
\pgfpathlineto{\pgfqpoint{1.378370in}{1.231350in}}%
\pgfusepath{stroke}%
\end{pgfscope}%
\begin{pgfscope}%
\pgfpathrectangle{\pgfqpoint{0.100000in}{0.212622in}}{\pgfqpoint{3.696000in}{3.696000in}}%
\pgfusepath{clip}%
\pgfsetrectcap%
\pgfsetroundjoin%
\pgfsetlinewidth{1.505625pt}%
\definecolor{currentstroke}{rgb}{1.000000,0.000000,0.000000}%
\pgfsetstrokecolor{currentstroke}%
\pgfsetdash{}{0pt}%
\pgfpathmoveto{\pgfqpoint{1.627707in}{1.712736in}}%
\pgfpathlineto{\pgfqpoint{1.378370in}{1.231350in}}%
\pgfusepath{stroke}%
\end{pgfscope}%
\begin{pgfscope}%
\pgfpathrectangle{\pgfqpoint{0.100000in}{0.212622in}}{\pgfqpoint{3.696000in}{3.696000in}}%
\pgfusepath{clip}%
\pgfsetrectcap%
\pgfsetroundjoin%
\pgfsetlinewidth{1.505625pt}%
\definecolor{currentstroke}{rgb}{1.000000,0.000000,0.000000}%
\pgfsetstrokecolor{currentstroke}%
\pgfsetdash{}{0pt}%
\pgfpathmoveto{\pgfqpoint{1.629366in}{1.714065in}}%
\pgfpathlineto{\pgfqpoint{1.378370in}{1.231350in}}%
\pgfusepath{stroke}%
\end{pgfscope}%
\begin{pgfscope}%
\pgfpathrectangle{\pgfqpoint{0.100000in}{0.212622in}}{\pgfqpoint{3.696000in}{3.696000in}}%
\pgfusepath{clip}%
\pgfsetrectcap%
\pgfsetroundjoin%
\pgfsetlinewidth{1.505625pt}%
\definecolor{currentstroke}{rgb}{1.000000,0.000000,0.000000}%
\pgfsetstrokecolor{currentstroke}%
\pgfsetdash{}{0pt}%
\pgfpathmoveto{\pgfqpoint{1.631013in}{1.712427in}}%
\pgfpathlineto{\pgfqpoint{1.395194in}{1.226041in}}%
\pgfusepath{stroke}%
\end{pgfscope}%
\begin{pgfscope}%
\pgfpathrectangle{\pgfqpoint{0.100000in}{0.212622in}}{\pgfqpoint{3.696000in}{3.696000in}}%
\pgfusepath{clip}%
\pgfsetrectcap%
\pgfsetroundjoin%
\pgfsetlinewidth{1.505625pt}%
\definecolor{currentstroke}{rgb}{1.000000,0.000000,0.000000}%
\pgfsetstrokecolor{currentstroke}%
\pgfsetdash{}{0pt}%
\pgfpathmoveto{\pgfqpoint{1.633433in}{1.708750in}}%
\pgfpathlineto{\pgfqpoint{1.395194in}{1.226041in}}%
\pgfusepath{stroke}%
\end{pgfscope}%
\begin{pgfscope}%
\pgfpathrectangle{\pgfqpoint{0.100000in}{0.212622in}}{\pgfqpoint{3.696000in}{3.696000in}}%
\pgfusepath{clip}%
\pgfsetrectcap%
\pgfsetroundjoin%
\pgfsetlinewidth{1.505625pt}%
\definecolor{currentstroke}{rgb}{1.000000,0.000000,0.000000}%
\pgfsetstrokecolor{currentstroke}%
\pgfsetdash{}{0pt}%
\pgfpathmoveto{\pgfqpoint{1.637885in}{1.710015in}}%
\pgfpathlineto{\pgfqpoint{1.412032in}{1.220728in}}%
\pgfusepath{stroke}%
\end{pgfscope}%
\begin{pgfscope}%
\pgfpathrectangle{\pgfqpoint{0.100000in}{0.212622in}}{\pgfqpoint{3.696000in}{3.696000in}}%
\pgfusepath{clip}%
\pgfsetrectcap%
\pgfsetroundjoin%
\pgfsetlinewidth{1.505625pt}%
\definecolor{currentstroke}{rgb}{1.000000,0.000000,0.000000}%
\pgfsetstrokecolor{currentstroke}%
\pgfsetdash{}{0pt}%
\pgfpathmoveto{\pgfqpoint{1.640163in}{1.707164in}}%
\pgfpathlineto{\pgfqpoint{1.412032in}{1.220728in}}%
\pgfusepath{stroke}%
\end{pgfscope}%
\begin{pgfscope}%
\pgfpathrectangle{\pgfqpoint{0.100000in}{0.212622in}}{\pgfqpoint{3.696000in}{3.696000in}}%
\pgfusepath{clip}%
\pgfsetrectcap%
\pgfsetroundjoin%
\pgfsetlinewidth{1.505625pt}%
\definecolor{currentstroke}{rgb}{1.000000,0.000000,0.000000}%
\pgfsetstrokecolor{currentstroke}%
\pgfsetdash{}{0pt}%
\pgfpathmoveto{\pgfqpoint{1.642123in}{1.699721in}}%
\pgfpathlineto{\pgfqpoint{1.428884in}{1.215410in}}%
\pgfusepath{stroke}%
\end{pgfscope}%
\begin{pgfscope}%
\pgfpathrectangle{\pgfqpoint{0.100000in}{0.212622in}}{\pgfqpoint{3.696000in}{3.696000in}}%
\pgfusepath{clip}%
\pgfsetrectcap%
\pgfsetroundjoin%
\pgfsetlinewidth{1.505625pt}%
\definecolor{currentstroke}{rgb}{1.000000,0.000000,0.000000}%
\pgfsetstrokecolor{currentstroke}%
\pgfsetdash{}{0pt}%
\pgfpathmoveto{\pgfqpoint{1.645407in}{1.692090in}}%
\pgfpathlineto{\pgfqpoint{1.428884in}{1.215410in}}%
\pgfusepath{stroke}%
\end{pgfscope}%
\begin{pgfscope}%
\pgfpathrectangle{\pgfqpoint{0.100000in}{0.212622in}}{\pgfqpoint{3.696000in}{3.696000in}}%
\pgfusepath{clip}%
\pgfsetrectcap%
\pgfsetroundjoin%
\pgfsetlinewidth{1.505625pt}%
\definecolor{currentstroke}{rgb}{1.000000,0.000000,0.000000}%
\pgfsetstrokecolor{currentstroke}%
\pgfsetdash{}{0pt}%
\pgfpathmoveto{\pgfqpoint{1.650963in}{1.697576in}}%
\pgfpathlineto{\pgfqpoint{1.445751in}{1.210087in}}%
\pgfusepath{stroke}%
\end{pgfscope}%
\begin{pgfscope}%
\pgfpathrectangle{\pgfqpoint{0.100000in}{0.212622in}}{\pgfqpoint{3.696000in}{3.696000in}}%
\pgfusepath{clip}%
\pgfsetrectcap%
\pgfsetroundjoin%
\pgfsetlinewidth{1.505625pt}%
\definecolor{currentstroke}{rgb}{1.000000,0.000000,0.000000}%
\pgfsetstrokecolor{currentstroke}%
\pgfsetdash{}{0pt}%
\pgfpathmoveto{\pgfqpoint{1.652700in}{1.695650in}}%
\pgfpathlineto{\pgfqpoint{1.445751in}{1.210087in}}%
\pgfusepath{stroke}%
\end{pgfscope}%
\begin{pgfscope}%
\pgfpathrectangle{\pgfqpoint{0.100000in}{0.212622in}}{\pgfqpoint{3.696000in}{3.696000in}}%
\pgfusepath{clip}%
\pgfsetrectcap%
\pgfsetroundjoin%
\pgfsetlinewidth{1.505625pt}%
\definecolor{currentstroke}{rgb}{1.000000,0.000000,0.000000}%
\pgfsetstrokecolor{currentstroke}%
\pgfsetdash{}{0pt}%
\pgfpathmoveto{\pgfqpoint{1.654800in}{1.688990in}}%
\pgfpathlineto{\pgfqpoint{1.462632in}{1.204760in}}%
\pgfusepath{stroke}%
\end{pgfscope}%
\begin{pgfscope}%
\pgfpathrectangle{\pgfqpoint{0.100000in}{0.212622in}}{\pgfqpoint{3.696000in}{3.696000in}}%
\pgfusepath{clip}%
\pgfsetrectcap%
\pgfsetroundjoin%
\pgfsetlinewidth{1.505625pt}%
\definecolor{currentstroke}{rgb}{1.000000,0.000000,0.000000}%
\pgfsetstrokecolor{currentstroke}%
\pgfsetdash{}{0pt}%
\pgfpathmoveto{\pgfqpoint{1.656311in}{1.687201in}}%
\pgfpathlineto{\pgfqpoint{1.462632in}{1.204760in}}%
\pgfusepath{stroke}%
\end{pgfscope}%
\begin{pgfscope}%
\pgfpathrectangle{\pgfqpoint{0.100000in}{0.212622in}}{\pgfqpoint{3.696000in}{3.696000in}}%
\pgfusepath{clip}%
\pgfsetrectcap%
\pgfsetroundjoin%
\pgfsetlinewidth{1.505625pt}%
\definecolor{currentstroke}{rgb}{1.000000,0.000000,0.000000}%
\pgfsetstrokecolor{currentstroke}%
\pgfsetdash{}{0pt}%
\pgfpathmoveto{\pgfqpoint{1.657124in}{1.687634in}}%
\pgfpathlineto{\pgfqpoint{1.462632in}{1.204760in}}%
\pgfusepath{stroke}%
\end{pgfscope}%
\begin{pgfscope}%
\pgfpathrectangle{\pgfqpoint{0.100000in}{0.212622in}}{\pgfqpoint{3.696000in}{3.696000in}}%
\pgfusepath{clip}%
\pgfsetrectcap%
\pgfsetroundjoin%
\pgfsetlinewidth{1.505625pt}%
\definecolor{currentstroke}{rgb}{1.000000,0.000000,0.000000}%
\pgfsetstrokecolor{currentstroke}%
\pgfsetdash{}{0pt}%
\pgfpathmoveto{\pgfqpoint{1.657329in}{1.686801in}}%
\pgfpathlineto{\pgfqpoint{1.462632in}{1.204760in}}%
\pgfusepath{stroke}%
\end{pgfscope}%
\begin{pgfscope}%
\pgfpathrectangle{\pgfqpoint{0.100000in}{0.212622in}}{\pgfqpoint{3.696000in}{3.696000in}}%
\pgfusepath{clip}%
\pgfsetrectcap%
\pgfsetroundjoin%
\pgfsetlinewidth{1.505625pt}%
\definecolor{currentstroke}{rgb}{1.000000,0.000000,0.000000}%
\pgfsetstrokecolor{currentstroke}%
\pgfsetdash{}{0pt}%
\pgfpathmoveto{\pgfqpoint{1.657806in}{1.685413in}}%
\pgfpathlineto{\pgfqpoint{1.462632in}{1.204760in}}%
\pgfusepath{stroke}%
\end{pgfscope}%
\begin{pgfscope}%
\pgfpathrectangle{\pgfqpoint{0.100000in}{0.212622in}}{\pgfqpoint{3.696000in}{3.696000in}}%
\pgfusepath{clip}%
\pgfsetrectcap%
\pgfsetroundjoin%
\pgfsetlinewidth{1.505625pt}%
\definecolor{currentstroke}{rgb}{1.000000,0.000000,0.000000}%
\pgfsetstrokecolor{currentstroke}%
\pgfsetdash{}{0pt}%
\pgfpathmoveto{\pgfqpoint{1.658083in}{1.684996in}}%
\pgfpathlineto{\pgfqpoint{1.462632in}{1.204760in}}%
\pgfusepath{stroke}%
\end{pgfscope}%
\begin{pgfscope}%
\pgfpathrectangle{\pgfqpoint{0.100000in}{0.212622in}}{\pgfqpoint{3.696000in}{3.696000in}}%
\pgfusepath{clip}%
\pgfsetrectcap%
\pgfsetroundjoin%
\pgfsetlinewidth{1.505625pt}%
\definecolor{currentstroke}{rgb}{1.000000,0.000000,0.000000}%
\pgfsetstrokecolor{currentstroke}%
\pgfsetdash{}{0pt}%
\pgfpathmoveto{\pgfqpoint{1.658531in}{1.685562in}}%
\pgfpathlineto{\pgfqpoint{1.462632in}{1.204760in}}%
\pgfusepath{stroke}%
\end{pgfscope}%
\begin{pgfscope}%
\pgfpathrectangle{\pgfqpoint{0.100000in}{0.212622in}}{\pgfqpoint{3.696000in}{3.696000in}}%
\pgfusepath{clip}%
\pgfsetrectcap%
\pgfsetroundjoin%
\pgfsetlinewidth{1.505625pt}%
\definecolor{currentstroke}{rgb}{1.000000,0.000000,0.000000}%
\pgfsetstrokecolor{currentstroke}%
\pgfsetdash{}{0pt}%
\pgfpathmoveto{\pgfqpoint{1.658660in}{1.684909in}}%
\pgfpathlineto{\pgfqpoint{1.462632in}{1.204760in}}%
\pgfusepath{stroke}%
\end{pgfscope}%
\begin{pgfscope}%
\pgfpathrectangle{\pgfqpoint{0.100000in}{0.212622in}}{\pgfqpoint{3.696000in}{3.696000in}}%
\pgfusepath{clip}%
\pgfsetrectcap%
\pgfsetroundjoin%
\pgfsetlinewidth{1.505625pt}%
\definecolor{currentstroke}{rgb}{1.000000,0.000000,0.000000}%
\pgfsetstrokecolor{currentstroke}%
\pgfsetdash{}{0pt}%
\pgfpathmoveto{\pgfqpoint{1.659298in}{1.683610in}}%
\pgfpathlineto{\pgfqpoint{1.479528in}{1.199428in}}%
\pgfusepath{stroke}%
\end{pgfscope}%
\begin{pgfscope}%
\pgfpathrectangle{\pgfqpoint{0.100000in}{0.212622in}}{\pgfqpoint{3.696000in}{3.696000in}}%
\pgfusepath{clip}%
\pgfsetrectcap%
\pgfsetroundjoin%
\pgfsetlinewidth{1.505625pt}%
\definecolor{currentstroke}{rgb}{1.000000,0.000000,0.000000}%
\pgfsetstrokecolor{currentstroke}%
\pgfsetdash{}{0pt}%
\pgfpathmoveto{\pgfqpoint{1.659503in}{1.683490in}}%
\pgfpathlineto{\pgfqpoint{1.479528in}{1.199428in}}%
\pgfusepath{stroke}%
\end{pgfscope}%
\begin{pgfscope}%
\pgfpathrectangle{\pgfqpoint{0.100000in}{0.212622in}}{\pgfqpoint{3.696000in}{3.696000in}}%
\pgfusepath{clip}%
\pgfsetrectcap%
\pgfsetroundjoin%
\pgfsetlinewidth{1.505625pt}%
\definecolor{currentstroke}{rgb}{1.000000,0.000000,0.000000}%
\pgfsetstrokecolor{currentstroke}%
\pgfsetdash{}{0pt}%
\pgfpathmoveto{\pgfqpoint{1.660141in}{1.684417in}}%
\pgfpathlineto{\pgfqpoint{1.479528in}{1.199428in}}%
\pgfusepath{stroke}%
\end{pgfscope}%
\begin{pgfscope}%
\pgfpathrectangle{\pgfqpoint{0.100000in}{0.212622in}}{\pgfqpoint{3.696000in}{3.696000in}}%
\pgfusepath{clip}%
\pgfsetrectcap%
\pgfsetroundjoin%
\pgfsetlinewidth{1.505625pt}%
\definecolor{currentstroke}{rgb}{1.000000,0.000000,0.000000}%
\pgfsetstrokecolor{currentstroke}%
\pgfsetdash{}{0pt}%
\pgfpathmoveto{\pgfqpoint{1.660661in}{1.683238in}}%
\pgfpathlineto{\pgfqpoint{1.479528in}{1.199428in}}%
\pgfusepath{stroke}%
\end{pgfscope}%
\begin{pgfscope}%
\pgfpathrectangle{\pgfqpoint{0.100000in}{0.212622in}}{\pgfqpoint{3.696000in}{3.696000in}}%
\pgfusepath{clip}%
\pgfsetrectcap%
\pgfsetroundjoin%
\pgfsetlinewidth{1.505625pt}%
\definecolor{currentstroke}{rgb}{1.000000,0.000000,0.000000}%
\pgfsetstrokecolor{currentstroke}%
\pgfsetdash{}{0pt}%
\pgfpathmoveto{\pgfqpoint{1.661537in}{1.681348in}}%
\pgfpathlineto{\pgfqpoint{1.479528in}{1.199428in}}%
\pgfusepath{stroke}%
\end{pgfscope}%
\begin{pgfscope}%
\pgfpathrectangle{\pgfqpoint{0.100000in}{0.212622in}}{\pgfqpoint{3.696000in}{3.696000in}}%
\pgfusepath{clip}%
\pgfsetrectcap%
\pgfsetroundjoin%
\pgfsetlinewidth{1.505625pt}%
\definecolor{currentstroke}{rgb}{1.000000,0.000000,0.000000}%
\pgfsetstrokecolor{currentstroke}%
\pgfsetdash{}{0pt}%
\pgfpathmoveto{\pgfqpoint{1.662222in}{1.682432in}}%
\pgfpathlineto{\pgfqpoint{1.479528in}{1.199428in}}%
\pgfusepath{stroke}%
\end{pgfscope}%
\begin{pgfscope}%
\pgfpathrectangle{\pgfqpoint{0.100000in}{0.212622in}}{\pgfqpoint{3.696000in}{3.696000in}}%
\pgfusepath{clip}%
\pgfsetrectcap%
\pgfsetroundjoin%
\pgfsetlinewidth{1.505625pt}%
\definecolor{currentstroke}{rgb}{1.000000,0.000000,0.000000}%
\pgfsetstrokecolor{currentstroke}%
\pgfsetdash{}{0pt}%
\pgfpathmoveto{\pgfqpoint{1.663186in}{1.682017in}}%
\pgfpathlineto{\pgfqpoint{1.479528in}{1.199428in}}%
\pgfusepath{stroke}%
\end{pgfscope}%
\begin{pgfscope}%
\pgfpathrectangle{\pgfqpoint{0.100000in}{0.212622in}}{\pgfqpoint{3.696000in}{3.696000in}}%
\pgfusepath{clip}%
\pgfsetrectcap%
\pgfsetroundjoin%
\pgfsetlinewidth{1.505625pt}%
\definecolor{currentstroke}{rgb}{1.000000,0.000000,0.000000}%
\pgfsetstrokecolor{currentstroke}%
\pgfsetdash{}{0pt}%
\pgfpathmoveto{\pgfqpoint{1.664286in}{1.678685in}}%
\pgfpathlineto{\pgfqpoint{1.479528in}{1.199428in}}%
\pgfusepath{stroke}%
\end{pgfscope}%
\begin{pgfscope}%
\pgfpathrectangle{\pgfqpoint{0.100000in}{0.212622in}}{\pgfqpoint{3.696000in}{3.696000in}}%
\pgfusepath{clip}%
\pgfsetrectcap%
\pgfsetroundjoin%
\pgfsetlinewidth{1.505625pt}%
\definecolor{currentstroke}{rgb}{1.000000,0.000000,0.000000}%
\pgfsetstrokecolor{currentstroke}%
\pgfsetdash{}{0pt}%
\pgfpathmoveto{\pgfqpoint{1.665552in}{1.675416in}}%
\pgfpathlineto{\pgfqpoint{1.496438in}{1.194091in}}%
\pgfusepath{stroke}%
\end{pgfscope}%
\begin{pgfscope}%
\pgfpathrectangle{\pgfqpoint{0.100000in}{0.212622in}}{\pgfqpoint{3.696000in}{3.696000in}}%
\pgfusepath{clip}%
\pgfsetrectcap%
\pgfsetroundjoin%
\pgfsetlinewidth{1.505625pt}%
\definecolor{currentstroke}{rgb}{1.000000,0.000000,0.000000}%
\pgfsetstrokecolor{currentstroke}%
\pgfsetdash{}{0pt}%
\pgfpathmoveto{\pgfqpoint{1.667842in}{1.679788in}}%
\pgfpathlineto{\pgfqpoint{1.496438in}{1.194091in}}%
\pgfusepath{stroke}%
\end{pgfscope}%
\begin{pgfscope}%
\pgfpathrectangle{\pgfqpoint{0.100000in}{0.212622in}}{\pgfqpoint{3.696000in}{3.696000in}}%
\pgfusepath{clip}%
\pgfsetrectcap%
\pgfsetroundjoin%
\pgfsetlinewidth{1.505625pt}%
\definecolor{currentstroke}{rgb}{1.000000,0.000000,0.000000}%
\pgfsetstrokecolor{currentstroke}%
\pgfsetdash{}{0pt}%
\pgfpathmoveto{\pgfqpoint{1.668188in}{1.678559in}}%
\pgfpathlineto{\pgfqpoint{1.496438in}{1.194091in}}%
\pgfusepath{stroke}%
\end{pgfscope}%
\begin{pgfscope}%
\pgfpathrectangle{\pgfqpoint{0.100000in}{0.212622in}}{\pgfqpoint{3.696000in}{3.696000in}}%
\pgfusepath{clip}%
\pgfsetrectcap%
\pgfsetroundjoin%
\pgfsetlinewidth{1.505625pt}%
\definecolor{currentstroke}{rgb}{1.000000,0.000000,0.000000}%
\pgfsetstrokecolor{currentstroke}%
\pgfsetdash{}{0pt}%
\pgfpathmoveto{\pgfqpoint{1.669251in}{1.675439in}}%
\pgfpathlineto{\pgfqpoint{1.496438in}{1.194091in}}%
\pgfusepath{stroke}%
\end{pgfscope}%
\begin{pgfscope}%
\pgfpathrectangle{\pgfqpoint{0.100000in}{0.212622in}}{\pgfqpoint{3.696000in}{3.696000in}}%
\pgfusepath{clip}%
\pgfsetrectcap%
\pgfsetroundjoin%
\pgfsetlinewidth{1.505625pt}%
\definecolor{currentstroke}{rgb}{1.000000,0.000000,0.000000}%
\pgfsetstrokecolor{currentstroke}%
\pgfsetdash{}{0pt}%
\pgfpathmoveto{\pgfqpoint{1.669718in}{1.674997in}}%
\pgfpathlineto{\pgfqpoint{1.496438in}{1.194091in}}%
\pgfusepath{stroke}%
\end{pgfscope}%
\begin{pgfscope}%
\pgfpathrectangle{\pgfqpoint{0.100000in}{0.212622in}}{\pgfqpoint{3.696000in}{3.696000in}}%
\pgfusepath{clip}%
\pgfsetrectcap%
\pgfsetroundjoin%
\pgfsetlinewidth{1.505625pt}%
\definecolor{currentstroke}{rgb}{1.000000,0.000000,0.000000}%
\pgfsetstrokecolor{currentstroke}%
\pgfsetdash{}{0pt}%
\pgfpathmoveto{\pgfqpoint{1.670721in}{1.676827in}}%
\pgfpathlineto{\pgfqpoint{1.496438in}{1.194091in}}%
\pgfusepath{stroke}%
\end{pgfscope}%
\begin{pgfscope}%
\pgfpathrectangle{\pgfqpoint{0.100000in}{0.212622in}}{\pgfqpoint{3.696000in}{3.696000in}}%
\pgfusepath{clip}%
\pgfsetrectcap%
\pgfsetroundjoin%
\pgfsetlinewidth{1.505625pt}%
\definecolor{currentstroke}{rgb}{1.000000,0.000000,0.000000}%
\pgfsetstrokecolor{currentstroke}%
\pgfsetdash{}{0pt}%
\pgfpathmoveto{\pgfqpoint{1.670919in}{1.675941in}}%
\pgfpathlineto{\pgfqpoint{1.496438in}{1.194091in}}%
\pgfusepath{stroke}%
\end{pgfscope}%
\begin{pgfscope}%
\pgfpathrectangle{\pgfqpoint{0.100000in}{0.212622in}}{\pgfqpoint{3.696000in}{3.696000in}}%
\pgfusepath{clip}%
\pgfsetrectcap%
\pgfsetroundjoin%
\pgfsetlinewidth{1.505625pt}%
\definecolor{currentstroke}{rgb}{1.000000,0.000000,0.000000}%
\pgfsetstrokecolor{currentstroke}%
\pgfsetdash{}{0pt}%
\pgfpathmoveto{\pgfqpoint{1.671831in}{1.674153in}}%
\pgfpathlineto{\pgfqpoint{1.513363in}{1.188750in}}%
\pgfusepath{stroke}%
\end{pgfscope}%
\begin{pgfscope}%
\pgfpathrectangle{\pgfqpoint{0.100000in}{0.212622in}}{\pgfqpoint{3.696000in}{3.696000in}}%
\pgfusepath{clip}%
\pgfsetrectcap%
\pgfsetroundjoin%
\pgfsetlinewidth{1.505625pt}%
\definecolor{currentstroke}{rgb}{1.000000,0.000000,0.000000}%
\pgfsetstrokecolor{currentstroke}%
\pgfsetdash{}{0pt}%
\pgfpathmoveto{\pgfqpoint{1.672588in}{1.671548in}}%
\pgfpathlineto{\pgfqpoint{1.513363in}{1.188750in}}%
\pgfusepath{stroke}%
\end{pgfscope}%
\begin{pgfscope}%
\pgfpathrectangle{\pgfqpoint{0.100000in}{0.212622in}}{\pgfqpoint{3.696000in}{3.696000in}}%
\pgfusepath{clip}%
\pgfsetrectcap%
\pgfsetroundjoin%
\pgfsetlinewidth{1.505625pt}%
\definecolor{currentstroke}{rgb}{1.000000,0.000000,0.000000}%
\pgfsetstrokecolor{currentstroke}%
\pgfsetdash{}{0pt}%
\pgfpathmoveto{\pgfqpoint{1.674134in}{1.674376in}}%
\pgfpathlineto{\pgfqpoint{1.513363in}{1.188750in}}%
\pgfusepath{stroke}%
\end{pgfscope}%
\begin{pgfscope}%
\pgfpathrectangle{\pgfqpoint{0.100000in}{0.212622in}}{\pgfqpoint{3.696000in}{3.696000in}}%
\pgfusepath{clip}%
\pgfsetrectcap%
\pgfsetroundjoin%
\pgfsetlinewidth{1.505625pt}%
\definecolor{currentstroke}{rgb}{1.000000,0.000000,0.000000}%
\pgfsetstrokecolor{currentstroke}%
\pgfsetdash{}{0pt}%
\pgfpathmoveto{\pgfqpoint{1.674847in}{1.673823in}}%
\pgfpathlineto{\pgfqpoint{1.513363in}{1.188750in}}%
\pgfusepath{stroke}%
\end{pgfscope}%
\begin{pgfscope}%
\pgfpathrectangle{\pgfqpoint{0.100000in}{0.212622in}}{\pgfqpoint{3.696000in}{3.696000in}}%
\pgfusepath{clip}%
\pgfsetrectcap%
\pgfsetroundjoin%
\pgfsetlinewidth{1.505625pt}%
\definecolor{currentstroke}{rgb}{1.000000,0.000000,0.000000}%
\pgfsetstrokecolor{currentstroke}%
\pgfsetdash{}{0pt}%
\pgfpathmoveto{\pgfqpoint{1.675219in}{1.673743in}}%
\pgfpathlineto{\pgfqpoint{1.513363in}{1.188750in}}%
\pgfusepath{stroke}%
\end{pgfscope}%
\begin{pgfscope}%
\pgfpathrectangle{\pgfqpoint{0.100000in}{0.212622in}}{\pgfqpoint{3.696000in}{3.696000in}}%
\pgfusepath{clip}%
\pgfsetrectcap%
\pgfsetroundjoin%
\pgfsetlinewidth{1.505625pt}%
\definecolor{currentstroke}{rgb}{1.000000,0.000000,0.000000}%
\pgfsetstrokecolor{currentstroke}%
\pgfsetdash{}{0pt}%
\pgfpathmoveto{\pgfqpoint{1.675699in}{1.672647in}}%
\pgfpathlineto{\pgfqpoint{1.513363in}{1.188750in}}%
\pgfusepath{stroke}%
\end{pgfscope}%
\begin{pgfscope}%
\pgfpathrectangle{\pgfqpoint{0.100000in}{0.212622in}}{\pgfqpoint{3.696000in}{3.696000in}}%
\pgfusepath{clip}%
\pgfsetrectcap%
\pgfsetroundjoin%
\pgfsetlinewidth{1.505625pt}%
\definecolor{currentstroke}{rgb}{1.000000,0.000000,0.000000}%
\pgfsetstrokecolor{currentstroke}%
\pgfsetdash{}{0pt}%
\pgfpathmoveto{\pgfqpoint{1.676396in}{1.673599in}}%
\pgfpathlineto{\pgfqpoint{1.513363in}{1.188750in}}%
\pgfusepath{stroke}%
\end{pgfscope}%
\begin{pgfscope}%
\pgfpathrectangle{\pgfqpoint{0.100000in}{0.212622in}}{\pgfqpoint{3.696000in}{3.696000in}}%
\pgfusepath{clip}%
\pgfsetrectcap%
\pgfsetroundjoin%
\pgfsetlinewidth{1.505625pt}%
\definecolor{currentstroke}{rgb}{1.000000,0.000000,0.000000}%
\pgfsetstrokecolor{currentstroke}%
\pgfsetdash{}{0pt}%
\pgfpathmoveto{\pgfqpoint{1.676587in}{1.672781in}}%
\pgfpathlineto{\pgfqpoint{1.513363in}{1.188750in}}%
\pgfusepath{stroke}%
\end{pgfscope}%
\begin{pgfscope}%
\pgfpathrectangle{\pgfqpoint{0.100000in}{0.212622in}}{\pgfqpoint{3.696000in}{3.696000in}}%
\pgfusepath{clip}%
\pgfsetrectcap%
\pgfsetroundjoin%
\pgfsetlinewidth{1.505625pt}%
\definecolor{currentstroke}{rgb}{1.000000,0.000000,0.000000}%
\pgfsetstrokecolor{currentstroke}%
\pgfsetdash{}{0pt}%
\pgfpathmoveto{\pgfqpoint{1.677036in}{1.672577in}}%
\pgfpathlineto{\pgfqpoint{1.513363in}{1.188750in}}%
\pgfusepath{stroke}%
\end{pgfscope}%
\begin{pgfscope}%
\pgfpathrectangle{\pgfqpoint{0.100000in}{0.212622in}}{\pgfqpoint{3.696000in}{3.696000in}}%
\pgfusepath{clip}%
\pgfsetrectcap%
\pgfsetroundjoin%
\pgfsetlinewidth{1.505625pt}%
\definecolor{currentstroke}{rgb}{1.000000,0.000000,0.000000}%
\pgfsetstrokecolor{currentstroke}%
\pgfsetdash{}{0pt}%
\pgfpathmoveto{\pgfqpoint{1.677205in}{1.672716in}}%
\pgfpathlineto{\pgfqpoint{1.513363in}{1.188750in}}%
\pgfusepath{stroke}%
\end{pgfscope}%
\begin{pgfscope}%
\pgfpathrectangle{\pgfqpoint{0.100000in}{0.212622in}}{\pgfqpoint{3.696000in}{3.696000in}}%
\pgfusepath{clip}%
\pgfsetrectcap%
\pgfsetroundjoin%
\pgfsetlinewidth{1.505625pt}%
\definecolor{currentstroke}{rgb}{1.000000,0.000000,0.000000}%
\pgfsetstrokecolor{currentstroke}%
\pgfsetdash{}{0pt}%
\pgfpathmoveto{\pgfqpoint{1.677787in}{1.673094in}}%
\pgfpathlineto{\pgfqpoint{1.513363in}{1.188750in}}%
\pgfusepath{stroke}%
\end{pgfscope}%
\begin{pgfscope}%
\pgfpathrectangle{\pgfqpoint{0.100000in}{0.212622in}}{\pgfqpoint{3.696000in}{3.696000in}}%
\pgfusepath{clip}%
\pgfsetrectcap%
\pgfsetroundjoin%
\pgfsetlinewidth{1.505625pt}%
\definecolor{currentstroke}{rgb}{1.000000,0.000000,0.000000}%
\pgfsetstrokecolor{currentstroke}%
\pgfsetdash{}{0pt}%
\pgfpathmoveto{\pgfqpoint{1.678063in}{1.672746in}}%
\pgfpathlineto{\pgfqpoint{1.513363in}{1.188750in}}%
\pgfusepath{stroke}%
\end{pgfscope}%
\begin{pgfscope}%
\pgfpathrectangle{\pgfqpoint{0.100000in}{0.212622in}}{\pgfqpoint{3.696000in}{3.696000in}}%
\pgfusepath{clip}%
\pgfsetrectcap%
\pgfsetroundjoin%
\pgfsetlinewidth{1.505625pt}%
\definecolor{currentstroke}{rgb}{1.000000,0.000000,0.000000}%
\pgfsetstrokecolor{currentstroke}%
\pgfsetdash{}{0pt}%
\pgfpathmoveto{\pgfqpoint{1.679083in}{1.669591in}}%
\pgfpathlineto{\pgfqpoint{1.530302in}{1.183405in}}%
\pgfusepath{stroke}%
\end{pgfscope}%
\begin{pgfscope}%
\pgfpathrectangle{\pgfqpoint{0.100000in}{0.212622in}}{\pgfqpoint{3.696000in}{3.696000in}}%
\pgfusepath{clip}%
\pgfsetrectcap%
\pgfsetroundjoin%
\pgfsetlinewidth{1.505625pt}%
\definecolor{currentstroke}{rgb}{1.000000,0.000000,0.000000}%
\pgfsetstrokecolor{currentstroke}%
\pgfsetdash{}{0pt}%
\pgfpathmoveto{\pgfqpoint{1.680679in}{1.671103in}}%
\pgfpathlineto{\pgfqpoint{1.530302in}{1.183405in}}%
\pgfusepath{stroke}%
\end{pgfscope}%
\begin{pgfscope}%
\pgfpathrectangle{\pgfqpoint{0.100000in}{0.212622in}}{\pgfqpoint{3.696000in}{3.696000in}}%
\pgfusepath{clip}%
\pgfsetrectcap%
\pgfsetroundjoin%
\pgfsetlinewidth{1.505625pt}%
\definecolor{currentstroke}{rgb}{1.000000,0.000000,0.000000}%
\pgfsetstrokecolor{currentstroke}%
\pgfsetdash{}{0pt}%
\pgfpathmoveto{\pgfqpoint{1.681784in}{1.670682in}}%
\pgfpathlineto{\pgfqpoint{1.530302in}{1.183405in}}%
\pgfusepath{stroke}%
\end{pgfscope}%
\begin{pgfscope}%
\pgfpathrectangle{\pgfqpoint{0.100000in}{0.212622in}}{\pgfqpoint{3.696000in}{3.696000in}}%
\pgfusepath{clip}%
\pgfsetrectcap%
\pgfsetroundjoin%
\pgfsetlinewidth{1.505625pt}%
\definecolor{currentstroke}{rgb}{1.000000,0.000000,0.000000}%
\pgfsetstrokecolor{currentstroke}%
\pgfsetdash{}{0pt}%
\pgfpathmoveto{\pgfqpoint{1.682872in}{1.670065in}}%
\pgfpathlineto{\pgfqpoint{1.530302in}{1.183405in}}%
\pgfusepath{stroke}%
\end{pgfscope}%
\begin{pgfscope}%
\pgfpathrectangle{\pgfqpoint{0.100000in}{0.212622in}}{\pgfqpoint{3.696000in}{3.696000in}}%
\pgfusepath{clip}%
\pgfsetrectcap%
\pgfsetroundjoin%
\pgfsetlinewidth{1.505625pt}%
\definecolor{currentstroke}{rgb}{1.000000,0.000000,0.000000}%
\pgfsetstrokecolor{currentstroke}%
\pgfsetdash{}{0pt}%
\pgfpathmoveto{\pgfqpoint{1.684980in}{1.666683in}}%
\pgfpathlineto{\pgfqpoint{1.547256in}{1.178055in}}%
\pgfusepath{stroke}%
\end{pgfscope}%
\begin{pgfscope}%
\pgfpathrectangle{\pgfqpoint{0.100000in}{0.212622in}}{\pgfqpoint{3.696000in}{3.696000in}}%
\pgfusepath{clip}%
\pgfsetrectcap%
\pgfsetroundjoin%
\pgfsetlinewidth{1.505625pt}%
\definecolor{currentstroke}{rgb}{1.000000,0.000000,0.000000}%
\pgfsetstrokecolor{currentstroke}%
\pgfsetdash{}{0pt}%
\pgfpathmoveto{\pgfqpoint{1.687679in}{1.669694in}}%
\pgfpathlineto{\pgfqpoint{1.547256in}{1.178055in}}%
\pgfusepath{stroke}%
\end{pgfscope}%
\begin{pgfscope}%
\pgfpathrectangle{\pgfqpoint{0.100000in}{0.212622in}}{\pgfqpoint{3.696000in}{3.696000in}}%
\pgfusepath{clip}%
\pgfsetrectcap%
\pgfsetroundjoin%
\pgfsetlinewidth{1.505625pt}%
\definecolor{currentstroke}{rgb}{1.000000,0.000000,0.000000}%
\pgfsetstrokecolor{currentstroke}%
\pgfsetdash{}{0pt}%
\pgfpathmoveto{\pgfqpoint{1.689606in}{1.667108in}}%
\pgfpathlineto{\pgfqpoint{1.547256in}{1.178055in}}%
\pgfusepath{stroke}%
\end{pgfscope}%
\begin{pgfscope}%
\pgfpathrectangle{\pgfqpoint{0.100000in}{0.212622in}}{\pgfqpoint{3.696000in}{3.696000in}}%
\pgfusepath{clip}%
\pgfsetrectcap%
\pgfsetroundjoin%
\pgfsetlinewidth{1.505625pt}%
\definecolor{currentstroke}{rgb}{1.000000,0.000000,0.000000}%
\pgfsetstrokecolor{currentstroke}%
\pgfsetdash{}{0pt}%
\pgfpathmoveto{\pgfqpoint{1.690905in}{1.664465in}}%
\pgfpathlineto{\pgfqpoint{1.547256in}{1.178055in}}%
\pgfusepath{stroke}%
\end{pgfscope}%
\begin{pgfscope}%
\pgfpathrectangle{\pgfqpoint{0.100000in}{0.212622in}}{\pgfqpoint{3.696000in}{3.696000in}}%
\pgfusepath{clip}%
\pgfsetrectcap%
\pgfsetroundjoin%
\pgfsetlinewidth{1.505625pt}%
\definecolor{currentstroke}{rgb}{1.000000,0.000000,0.000000}%
\pgfsetstrokecolor{currentstroke}%
\pgfsetdash{}{0pt}%
\pgfpathmoveto{\pgfqpoint{1.692215in}{1.663083in}}%
\pgfpathlineto{\pgfqpoint{1.564224in}{1.172700in}}%
\pgfusepath{stroke}%
\end{pgfscope}%
\begin{pgfscope}%
\pgfpathrectangle{\pgfqpoint{0.100000in}{0.212622in}}{\pgfqpoint{3.696000in}{3.696000in}}%
\pgfusepath{clip}%
\pgfsetrectcap%
\pgfsetroundjoin%
\pgfsetlinewidth{1.505625pt}%
\definecolor{currentstroke}{rgb}{1.000000,0.000000,0.000000}%
\pgfsetstrokecolor{currentstroke}%
\pgfsetdash{}{0pt}%
\pgfpathmoveto{\pgfqpoint{1.693129in}{1.663805in}}%
\pgfpathlineto{\pgfqpoint{1.564224in}{1.172700in}}%
\pgfusepath{stroke}%
\end{pgfscope}%
\begin{pgfscope}%
\pgfpathrectangle{\pgfqpoint{0.100000in}{0.212622in}}{\pgfqpoint{3.696000in}{3.696000in}}%
\pgfusepath{clip}%
\pgfsetrectcap%
\pgfsetroundjoin%
\pgfsetlinewidth{1.505625pt}%
\definecolor{currentstroke}{rgb}{1.000000,0.000000,0.000000}%
\pgfsetstrokecolor{currentstroke}%
\pgfsetdash{}{0pt}%
\pgfpathmoveto{\pgfqpoint{1.693380in}{1.662810in}}%
\pgfpathlineto{\pgfqpoint{1.564224in}{1.172700in}}%
\pgfusepath{stroke}%
\end{pgfscope}%
\begin{pgfscope}%
\pgfpathrectangle{\pgfqpoint{0.100000in}{0.212622in}}{\pgfqpoint{3.696000in}{3.696000in}}%
\pgfusepath{clip}%
\pgfsetrectcap%
\pgfsetroundjoin%
\pgfsetlinewidth{1.505625pt}%
\definecolor{currentstroke}{rgb}{1.000000,0.000000,0.000000}%
\pgfsetstrokecolor{currentstroke}%
\pgfsetdash{}{0pt}%
\pgfpathmoveto{\pgfqpoint{1.694209in}{1.661089in}}%
\pgfpathlineto{\pgfqpoint{1.564224in}{1.172700in}}%
\pgfusepath{stroke}%
\end{pgfscope}%
\begin{pgfscope}%
\pgfpathrectangle{\pgfqpoint{0.100000in}{0.212622in}}{\pgfqpoint{3.696000in}{3.696000in}}%
\pgfusepath{clip}%
\pgfsetrectcap%
\pgfsetroundjoin%
\pgfsetlinewidth{1.505625pt}%
\definecolor{currentstroke}{rgb}{1.000000,0.000000,0.000000}%
\pgfsetstrokecolor{currentstroke}%
\pgfsetdash{}{0pt}%
\pgfpathmoveto{\pgfqpoint{1.694544in}{1.660632in}}%
\pgfpathlineto{\pgfqpoint{1.564224in}{1.172700in}}%
\pgfusepath{stroke}%
\end{pgfscope}%
\begin{pgfscope}%
\pgfpathrectangle{\pgfqpoint{0.100000in}{0.212622in}}{\pgfqpoint{3.696000in}{3.696000in}}%
\pgfusepath{clip}%
\pgfsetrectcap%
\pgfsetroundjoin%
\pgfsetlinewidth{1.505625pt}%
\definecolor{currentstroke}{rgb}{1.000000,0.000000,0.000000}%
\pgfsetstrokecolor{currentstroke}%
\pgfsetdash{}{0pt}%
\pgfpathmoveto{\pgfqpoint{1.695176in}{1.660993in}}%
\pgfpathlineto{\pgfqpoint{1.564224in}{1.172700in}}%
\pgfusepath{stroke}%
\end{pgfscope}%
\begin{pgfscope}%
\pgfpathrectangle{\pgfqpoint{0.100000in}{0.212622in}}{\pgfqpoint{3.696000in}{3.696000in}}%
\pgfusepath{clip}%
\pgfsetrectcap%
\pgfsetroundjoin%
\pgfsetlinewidth{1.505625pt}%
\definecolor{currentstroke}{rgb}{1.000000,0.000000,0.000000}%
\pgfsetstrokecolor{currentstroke}%
\pgfsetdash{}{0pt}%
\pgfpathmoveto{\pgfqpoint{1.695406in}{1.660569in}}%
\pgfpathlineto{\pgfqpoint{1.564224in}{1.172700in}}%
\pgfusepath{stroke}%
\end{pgfscope}%
\begin{pgfscope}%
\pgfpathrectangle{\pgfqpoint{0.100000in}{0.212622in}}{\pgfqpoint{3.696000in}{3.696000in}}%
\pgfusepath{clip}%
\pgfsetrectcap%
\pgfsetroundjoin%
\pgfsetlinewidth{1.505625pt}%
\definecolor{currentstroke}{rgb}{1.000000,0.000000,0.000000}%
\pgfsetstrokecolor{currentstroke}%
\pgfsetdash{}{0pt}%
\pgfpathmoveto{\pgfqpoint{1.696092in}{1.658451in}}%
\pgfpathlineto{\pgfqpoint{1.564224in}{1.172700in}}%
\pgfusepath{stroke}%
\end{pgfscope}%
\begin{pgfscope}%
\pgfpathrectangle{\pgfqpoint{0.100000in}{0.212622in}}{\pgfqpoint{3.696000in}{3.696000in}}%
\pgfusepath{clip}%
\pgfsetrectcap%
\pgfsetroundjoin%
\pgfsetlinewidth{1.505625pt}%
\definecolor{currentstroke}{rgb}{1.000000,0.000000,0.000000}%
\pgfsetstrokecolor{currentstroke}%
\pgfsetdash{}{0pt}%
\pgfpathmoveto{\pgfqpoint{1.696833in}{1.657806in}}%
\pgfpathlineto{\pgfqpoint{1.564224in}{1.172700in}}%
\pgfusepath{stroke}%
\end{pgfscope}%
\begin{pgfscope}%
\pgfpathrectangle{\pgfqpoint{0.100000in}{0.212622in}}{\pgfqpoint{3.696000in}{3.696000in}}%
\pgfusepath{clip}%
\pgfsetrectcap%
\pgfsetroundjoin%
\pgfsetlinewidth{1.505625pt}%
\definecolor{currentstroke}{rgb}{1.000000,0.000000,0.000000}%
\pgfsetstrokecolor{currentstroke}%
\pgfsetdash{}{0pt}%
\pgfpathmoveto{\pgfqpoint{1.698492in}{1.657789in}}%
\pgfpathlineto{\pgfqpoint{1.564224in}{1.172700in}}%
\pgfusepath{stroke}%
\end{pgfscope}%
\begin{pgfscope}%
\pgfpathrectangle{\pgfqpoint{0.100000in}{0.212622in}}{\pgfqpoint{3.696000in}{3.696000in}}%
\pgfusepath{clip}%
\pgfsetrectcap%
\pgfsetroundjoin%
\pgfsetlinewidth{1.505625pt}%
\definecolor{currentstroke}{rgb}{1.000000,0.000000,0.000000}%
\pgfsetstrokecolor{currentstroke}%
\pgfsetdash{}{0pt}%
\pgfpathmoveto{\pgfqpoint{1.698711in}{1.656885in}}%
\pgfpathlineto{\pgfqpoint{1.581207in}{1.167340in}}%
\pgfusepath{stroke}%
\end{pgfscope}%
\begin{pgfscope}%
\pgfpathrectangle{\pgfqpoint{0.100000in}{0.212622in}}{\pgfqpoint{3.696000in}{3.696000in}}%
\pgfusepath{clip}%
\pgfsetrectcap%
\pgfsetroundjoin%
\pgfsetlinewidth{1.505625pt}%
\definecolor{currentstroke}{rgb}{1.000000,0.000000,0.000000}%
\pgfsetstrokecolor{currentstroke}%
\pgfsetdash{}{0pt}%
\pgfpathmoveto{\pgfqpoint{1.699733in}{1.652652in}}%
\pgfpathlineto{\pgfqpoint{1.581207in}{1.167340in}}%
\pgfusepath{stroke}%
\end{pgfscope}%
\begin{pgfscope}%
\pgfpathrectangle{\pgfqpoint{0.100000in}{0.212622in}}{\pgfqpoint{3.696000in}{3.696000in}}%
\pgfusepath{clip}%
\pgfsetrectcap%
\pgfsetroundjoin%
\pgfsetlinewidth{1.505625pt}%
\definecolor{currentstroke}{rgb}{1.000000,0.000000,0.000000}%
\pgfsetstrokecolor{currentstroke}%
\pgfsetdash{}{0pt}%
\pgfpathmoveto{\pgfqpoint{1.700388in}{1.652608in}}%
\pgfpathlineto{\pgfqpoint{1.581207in}{1.167340in}}%
\pgfusepath{stroke}%
\end{pgfscope}%
\begin{pgfscope}%
\pgfpathrectangle{\pgfqpoint{0.100000in}{0.212622in}}{\pgfqpoint{3.696000in}{3.696000in}}%
\pgfusepath{clip}%
\pgfsetrectcap%
\pgfsetroundjoin%
\pgfsetlinewidth{1.505625pt}%
\definecolor{currentstroke}{rgb}{1.000000,0.000000,0.000000}%
\pgfsetstrokecolor{currentstroke}%
\pgfsetdash{}{0pt}%
\pgfpathmoveto{\pgfqpoint{1.701087in}{1.652610in}}%
\pgfpathlineto{\pgfqpoint{1.581207in}{1.167340in}}%
\pgfusepath{stroke}%
\end{pgfscope}%
\begin{pgfscope}%
\pgfpathrectangle{\pgfqpoint{0.100000in}{0.212622in}}{\pgfqpoint{3.696000in}{3.696000in}}%
\pgfusepath{clip}%
\pgfsetrectcap%
\pgfsetroundjoin%
\pgfsetlinewidth{1.505625pt}%
\definecolor{currentstroke}{rgb}{1.000000,0.000000,0.000000}%
\pgfsetstrokecolor{currentstroke}%
\pgfsetdash{}{0pt}%
\pgfpathmoveto{\pgfqpoint{1.701303in}{1.652336in}}%
\pgfpathlineto{\pgfqpoint{1.581207in}{1.167340in}}%
\pgfusepath{stroke}%
\end{pgfscope}%
\begin{pgfscope}%
\pgfpathrectangle{\pgfqpoint{0.100000in}{0.212622in}}{\pgfqpoint{3.696000in}{3.696000in}}%
\pgfusepath{clip}%
\pgfsetrectcap%
\pgfsetroundjoin%
\pgfsetlinewidth{1.505625pt}%
\definecolor{currentstroke}{rgb}{1.000000,0.000000,0.000000}%
\pgfsetstrokecolor{currentstroke}%
\pgfsetdash{}{0pt}%
\pgfpathmoveto{\pgfqpoint{1.702045in}{1.650231in}}%
\pgfpathlineto{\pgfqpoint{1.581207in}{1.167340in}}%
\pgfusepath{stroke}%
\end{pgfscope}%
\begin{pgfscope}%
\pgfpathrectangle{\pgfqpoint{0.100000in}{0.212622in}}{\pgfqpoint{3.696000in}{3.696000in}}%
\pgfusepath{clip}%
\pgfsetrectcap%
\pgfsetroundjoin%
\pgfsetlinewidth{1.505625pt}%
\definecolor{currentstroke}{rgb}{1.000000,0.000000,0.000000}%
\pgfsetstrokecolor{currentstroke}%
\pgfsetdash{}{0pt}%
\pgfpathmoveto{\pgfqpoint{1.702505in}{1.649961in}}%
\pgfpathlineto{\pgfqpoint{1.581207in}{1.167340in}}%
\pgfusepath{stroke}%
\end{pgfscope}%
\begin{pgfscope}%
\pgfpathrectangle{\pgfqpoint{0.100000in}{0.212622in}}{\pgfqpoint{3.696000in}{3.696000in}}%
\pgfusepath{clip}%
\pgfsetrectcap%
\pgfsetroundjoin%
\pgfsetlinewidth{1.505625pt}%
\definecolor{currentstroke}{rgb}{1.000000,0.000000,0.000000}%
\pgfsetstrokecolor{currentstroke}%
\pgfsetdash{}{0pt}%
\pgfpathmoveto{\pgfqpoint{1.703055in}{1.650391in}}%
\pgfpathlineto{\pgfqpoint{1.581207in}{1.167340in}}%
\pgfusepath{stroke}%
\end{pgfscope}%
\begin{pgfscope}%
\pgfpathrectangle{\pgfqpoint{0.100000in}{0.212622in}}{\pgfqpoint{3.696000in}{3.696000in}}%
\pgfusepath{clip}%
\pgfsetrectcap%
\pgfsetroundjoin%
\pgfsetlinewidth{1.505625pt}%
\definecolor{currentstroke}{rgb}{1.000000,0.000000,0.000000}%
\pgfsetstrokecolor{currentstroke}%
\pgfsetdash{}{0pt}%
\pgfpathmoveto{\pgfqpoint{1.703230in}{1.650094in}}%
\pgfpathlineto{\pgfqpoint{1.581207in}{1.167340in}}%
\pgfusepath{stroke}%
\end{pgfscope}%
\begin{pgfscope}%
\pgfpathrectangle{\pgfqpoint{0.100000in}{0.212622in}}{\pgfqpoint{3.696000in}{3.696000in}}%
\pgfusepath{clip}%
\pgfsetrectcap%
\pgfsetroundjoin%
\pgfsetlinewidth{1.505625pt}%
\definecolor{currentstroke}{rgb}{1.000000,0.000000,0.000000}%
\pgfsetstrokecolor{currentstroke}%
\pgfsetdash{}{0pt}%
\pgfpathmoveto{\pgfqpoint{1.703672in}{1.649584in}}%
\pgfpathlineto{\pgfqpoint{1.581207in}{1.167340in}}%
\pgfusepath{stroke}%
\end{pgfscope}%
\begin{pgfscope}%
\pgfpathrectangle{\pgfqpoint{0.100000in}{0.212622in}}{\pgfqpoint{3.696000in}{3.696000in}}%
\pgfusepath{clip}%
\pgfsetrectcap%
\pgfsetroundjoin%
\pgfsetlinewidth{1.505625pt}%
\definecolor{currentstroke}{rgb}{1.000000,0.000000,0.000000}%
\pgfsetstrokecolor{currentstroke}%
\pgfsetdash{}{0pt}%
\pgfpathmoveto{\pgfqpoint{1.704390in}{1.647349in}}%
\pgfpathlineto{\pgfqpoint{1.581207in}{1.167340in}}%
\pgfusepath{stroke}%
\end{pgfscope}%
\begin{pgfscope}%
\pgfpathrectangle{\pgfqpoint{0.100000in}{0.212622in}}{\pgfqpoint{3.696000in}{3.696000in}}%
\pgfusepath{clip}%
\pgfsetrectcap%
\pgfsetroundjoin%
\pgfsetlinewidth{1.505625pt}%
\definecolor{currentstroke}{rgb}{1.000000,0.000000,0.000000}%
\pgfsetstrokecolor{currentstroke}%
\pgfsetdash{}{0pt}%
\pgfpathmoveto{\pgfqpoint{1.704803in}{1.647424in}}%
\pgfpathlineto{\pgfqpoint{1.581207in}{1.167340in}}%
\pgfusepath{stroke}%
\end{pgfscope}%
\begin{pgfscope}%
\pgfpathrectangle{\pgfqpoint{0.100000in}{0.212622in}}{\pgfqpoint{3.696000in}{3.696000in}}%
\pgfusepath{clip}%
\pgfsetrectcap%
\pgfsetroundjoin%
\pgfsetlinewidth{1.505625pt}%
\definecolor{currentstroke}{rgb}{1.000000,0.000000,0.000000}%
\pgfsetstrokecolor{currentstroke}%
\pgfsetdash{}{0pt}%
\pgfpathmoveto{\pgfqpoint{1.705067in}{1.647512in}}%
\pgfpathlineto{\pgfqpoint{1.581207in}{1.167340in}}%
\pgfusepath{stroke}%
\end{pgfscope}%
\begin{pgfscope}%
\pgfpathrectangle{\pgfqpoint{0.100000in}{0.212622in}}{\pgfqpoint{3.696000in}{3.696000in}}%
\pgfusepath{clip}%
\pgfsetrectcap%
\pgfsetroundjoin%
\pgfsetlinewidth{1.505625pt}%
\definecolor{currentstroke}{rgb}{1.000000,0.000000,0.000000}%
\pgfsetstrokecolor{currentstroke}%
\pgfsetdash{}{0pt}%
\pgfpathmoveto{\pgfqpoint{1.705486in}{1.646760in}}%
\pgfpathlineto{\pgfqpoint{1.598205in}{1.161977in}}%
\pgfusepath{stroke}%
\end{pgfscope}%
\begin{pgfscope}%
\pgfpathrectangle{\pgfqpoint{0.100000in}{0.212622in}}{\pgfqpoint{3.696000in}{3.696000in}}%
\pgfusepath{clip}%
\pgfsetrectcap%
\pgfsetroundjoin%
\pgfsetlinewidth{1.505625pt}%
\definecolor{currentstroke}{rgb}{1.000000,0.000000,0.000000}%
\pgfsetstrokecolor{currentstroke}%
\pgfsetdash{}{0pt}%
\pgfpathmoveto{\pgfqpoint{1.706350in}{1.644205in}}%
\pgfpathlineto{\pgfqpoint{1.598205in}{1.161977in}}%
\pgfusepath{stroke}%
\end{pgfscope}%
\begin{pgfscope}%
\pgfpathrectangle{\pgfqpoint{0.100000in}{0.212622in}}{\pgfqpoint{3.696000in}{3.696000in}}%
\pgfusepath{clip}%
\pgfsetrectcap%
\pgfsetroundjoin%
\pgfsetlinewidth{1.505625pt}%
\definecolor{currentstroke}{rgb}{1.000000,0.000000,0.000000}%
\pgfsetstrokecolor{currentstroke}%
\pgfsetdash{}{0pt}%
\pgfpathmoveto{\pgfqpoint{1.707091in}{1.644744in}}%
\pgfpathlineto{\pgfqpoint{1.598205in}{1.161977in}}%
\pgfusepath{stroke}%
\end{pgfscope}%
\begin{pgfscope}%
\pgfpathrectangle{\pgfqpoint{0.100000in}{0.212622in}}{\pgfqpoint{3.696000in}{3.696000in}}%
\pgfusepath{clip}%
\pgfsetrectcap%
\pgfsetroundjoin%
\pgfsetlinewidth{1.505625pt}%
\definecolor{currentstroke}{rgb}{1.000000,0.000000,0.000000}%
\pgfsetstrokecolor{currentstroke}%
\pgfsetdash{}{0pt}%
\pgfpathmoveto{\pgfqpoint{1.707844in}{1.644847in}}%
\pgfpathlineto{\pgfqpoint{1.598205in}{1.161977in}}%
\pgfusepath{stroke}%
\end{pgfscope}%
\begin{pgfscope}%
\pgfpathrectangle{\pgfqpoint{0.100000in}{0.212622in}}{\pgfqpoint{3.696000in}{3.696000in}}%
\pgfusepath{clip}%
\pgfsetrectcap%
\pgfsetroundjoin%
\pgfsetlinewidth{1.505625pt}%
\definecolor{currentstroke}{rgb}{1.000000,0.000000,0.000000}%
\pgfsetstrokecolor{currentstroke}%
\pgfsetdash{}{0pt}%
\pgfpathmoveto{\pgfqpoint{1.708480in}{1.642773in}}%
\pgfpathlineto{\pgfqpoint{1.598205in}{1.161977in}}%
\pgfusepath{stroke}%
\end{pgfscope}%
\begin{pgfscope}%
\pgfpathrectangle{\pgfqpoint{0.100000in}{0.212622in}}{\pgfqpoint{3.696000in}{3.696000in}}%
\pgfusepath{clip}%
\pgfsetrectcap%
\pgfsetroundjoin%
\pgfsetlinewidth{1.505625pt}%
\definecolor{currentstroke}{rgb}{1.000000,0.000000,0.000000}%
\pgfsetstrokecolor{currentstroke}%
\pgfsetdash{}{0pt}%
\pgfpathmoveto{\pgfqpoint{1.710177in}{1.637298in}}%
\pgfpathlineto{\pgfqpoint{1.598205in}{1.161977in}}%
\pgfusepath{stroke}%
\end{pgfscope}%
\begin{pgfscope}%
\pgfpathrectangle{\pgfqpoint{0.100000in}{0.212622in}}{\pgfqpoint{3.696000in}{3.696000in}}%
\pgfusepath{clip}%
\pgfsetrectcap%
\pgfsetroundjoin%
\pgfsetlinewidth{1.505625pt}%
\definecolor{currentstroke}{rgb}{1.000000,0.000000,0.000000}%
\pgfsetstrokecolor{currentstroke}%
\pgfsetdash{}{0pt}%
\pgfpathmoveto{\pgfqpoint{1.711604in}{1.633822in}}%
\pgfpathlineto{\pgfqpoint{1.598205in}{1.161977in}}%
\pgfusepath{stroke}%
\end{pgfscope}%
\begin{pgfscope}%
\pgfpathrectangle{\pgfqpoint{0.100000in}{0.212622in}}{\pgfqpoint{3.696000in}{3.696000in}}%
\pgfusepath{clip}%
\pgfsetrectcap%
\pgfsetroundjoin%
\pgfsetlinewidth{1.505625pt}%
\definecolor{currentstroke}{rgb}{1.000000,0.000000,0.000000}%
\pgfsetstrokecolor{currentstroke}%
\pgfsetdash{}{0pt}%
\pgfpathmoveto{\pgfqpoint{1.714066in}{1.637230in}}%
\pgfpathlineto{\pgfqpoint{1.615217in}{1.156608in}}%
\pgfusepath{stroke}%
\end{pgfscope}%
\begin{pgfscope}%
\pgfpathrectangle{\pgfqpoint{0.100000in}{0.212622in}}{\pgfqpoint{3.696000in}{3.696000in}}%
\pgfusepath{clip}%
\pgfsetrectcap%
\pgfsetroundjoin%
\pgfsetlinewidth{1.505625pt}%
\definecolor{currentstroke}{rgb}{1.000000,0.000000,0.000000}%
\pgfsetstrokecolor{currentstroke}%
\pgfsetdash{}{0pt}%
\pgfpathmoveto{\pgfqpoint{1.714475in}{1.635490in}}%
\pgfpathlineto{\pgfqpoint{1.615217in}{1.156608in}}%
\pgfusepath{stroke}%
\end{pgfscope}%
\begin{pgfscope}%
\pgfpathrectangle{\pgfqpoint{0.100000in}{0.212622in}}{\pgfqpoint{3.696000in}{3.696000in}}%
\pgfusepath{clip}%
\pgfsetrectcap%
\pgfsetroundjoin%
\pgfsetlinewidth{1.505625pt}%
\definecolor{currentstroke}{rgb}{1.000000,0.000000,0.000000}%
\pgfsetstrokecolor{currentstroke}%
\pgfsetdash{}{0pt}%
\pgfpathmoveto{\pgfqpoint{1.716051in}{1.632645in}}%
\pgfpathlineto{\pgfqpoint{1.615217in}{1.156608in}}%
\pgfusepath{stroke}%
\end{pgfscope}%
\begin{pgfscope}%
\pgfpathrectangle{\pgfqpoint{0.100000in}{0.212622in}}{\pgfqpoint{3.696000in}{3.696000in}}%
\pgfusepath{clip}%
\pgfsetrectcap%
\pgfsetroundjoin%
\pgfsetlinewidth{1.505625pt}%
\definecolor{currentstroke}{rgb}{1.000000,0.000000,0.000000}%
\pgfsetstrokecolor{currentstroke}%
\pgfsetdash{}{0pt}%
\pgfpathmoveto{\pgfqpoint{1.717015in}{1.629123in}}%
\pgfpathlineto{\pgfqpoint{1.615217in}{1.156608in}}%
\pgfusepath{stroke}%
\end{pgfscope}%
\begin{pgfscope}%
\pgfpathrectangle{\pgfqpoint{0.100000in}{0.212622in}}{\pgfqpoint{3.696000in}{3.696000in}}%
\pgfusepath{clip}%
\pgfsetrectcap%
\pgfsetroundjoin%
\pgfsetlinewidth{1.505625pt}%
\definecolor{currentstroke}{rgb}{1.000000,0.000000,0.000000}%
\pgfsetstrokecolor{currentstroke}%
\pgfsetdash{}{0pt}%
\pgfpathmoveto{\pgfqpoint{1.717959in}{1.630055in}}%
\pgfpathlineto{\pgfqpoint{1.615217in}{1.156608in}}%
\pgfusepath{stroke}%
\end{pgfscope}%
\begin{pgfscope}%
\pgfpathrectangle{\pgfqpoint{0.100000in}{0.212622in}}{\pgfqpoint{3.696000in}{3.696000in}}%
\pgfusepath{clip}%
\pgfsetrectcap%
\pgfsetroundjoin%
\pgfsetlinewidth{1.505625pt}%
\definecolor{currentstroke}{rgb}{1.000000,0.000000,0.000000}%
\pgfsetstrokecolor{currentstroke}%
\pgfsetdash{}{0pt}%
\pgfpathmoveto{\pgfqpoint{1.718220in}{1.630053in}}%
\pgfpathlineto{\pgfqpoint{1.615217in}{1.156608in}}%
\pgfusepath{stroke}%
\end{pgfscope}%
\begin{pgfscope}%
\pgfpathrectangle{\pgfqpoint{0.100000in}{0.212622in}}{\pgfqpoint{3.696000in}{3.696000in}}%
\pgfusepath{clip}%
\pgfsetrectcap%
\pgfsetroundjoin%
\pgfsetlinewidth{1.505625pt}%
\definecolor{currentstroke}{rgb}{1.000000,0.000000,0.000000}%
\pgfsetstrokecolor{currentstroke}%
\pgfsetdash{}{0pt}%
\pgfpathmoveto{\pgfqpoint{1.718802in}{1.629147in}}%
\pgfpathlineto{\pgfqpoint{1.615217in}{1.156608in}}%
\pgfusepath{stroke}%
\end{pgfscope}%
\begin{pgfscope}%
\pgfpathrectangle{\pgfqpoint{0.100000in}{0.212622in}}{\pgfqpoint{3.696000in}{3.696000in}}%
\pgfusepath{clip}%
\pgfsetrectcap%
\pgfsetroundjoin%
\pgfsetlinewidth{1.505625pt}%
\definecolor{currentstroke}{rgb}{1.000000,0.000000,0.000000}%
\pgfsetstrokecolor{currentstroke}%
\pgfsetdash{}{0pt}%
\pgfpathmoveto{\pgfqpoint{1.719409in}{1.627075in}}%
\pgfpathlineto{\pgfqpoint{1.632244in}{1.151235in}}%
\pgfusepath{stroke}%
\end{pgfscope}%
\begin{pgfscope}%
\pgfpathrectangle{\pgfqpoint{0.100000in}{0.212622in}}{\pgfqpoint{3.696000in}{3.696000in}}%
\pgfusepath{clip}%
\pgfsetrectcap%
\pgfsetroundjoin%
\pgfsetlinewidth{1.505625pt}%
\definecolor{currentstroke}{rgb}{1.000000,0.000000,0.000000}%
\pgfsetstrokecolor{currentstroke}%
\pgfsetdash{}{0pt}%
\pgfpathmoveto{\pgfqpoint{1.720535in}{1.626907in}}%
\pgfpathlineto{\pgfqpoint{1.632244in}{1.151235in}}%
\pgfusepath{stroke}%
\end{pgfscope}%
\begin{pgfscope}%
\pgfpathrectangle{\pgfqpoint{0.100000in}{0.212622in}}{\pgfqpoint{3.696000in}{3.696000in}}%
\pgfusepath{clip}%
\pgfsetrectcap%
\pgfsetroundjoin%
\pgfsetlinewidth{1.505625pt}%
\definecolor{currentstroke}{rgb}{1.000000,0.000000,0.000000}%
\pgfsetstrokecolor{currentstroke}%
\pgfsetdash{}{0pt}%
\pgfpathmoveto{\pgfqpoint{1.720933in}{1.626900in}}%
\pgfpathlineto{\pgfqpoint{1.632244in}{1.151235in}}%
\pgfusepath{stroke}%
\end{pgfscope}%
\begin{pgfscope}%
\pgfpathrectangle{\pgfqpoint{0.100000in}{0.212622in}}{\pgfqpoint{3.696000in}{3.696000in}}%
\pgfusepath{clip}%
\pgfsetrectcap%
\pgfsetroundjoin%
\pgfsetlinewidth{1.505625pt}%
\definecolor{currentstroke}{rgb}{1.000000,0.000000,0.000000}%
\pgfsetstrokecolor{currentstroke}%
\pgfsetdash{}{0pt}%
\pgfpathmoveto{\pgfqpoint{1.721225in}{1.626093in}}%
\pgfpathlineto{\pgfqpoint{1.632244in}{1.151235in}}%
\pgfusepath{stroke}%
\end{pgfscope}%
\begin{pgfscope}%
\pgfpathrectangle{\pgfqpoint{0.100000in}{0.212622in}}{\pgfqpoint{3.696000in}{3.696000in}}%
\pgfusepath{clip}%
\pgfsetrectcap%
\pgfsetroundjoin%
\pgfsetlinewidth{1.505625pt}%
\definecolor{currentstroke}{rgb}{1.000000,0.000000,0.000000}%
\pgfsetstrokecolor{currentstroke}%
\pgfsetdash{}{0pt}%
\pgfpathmoveto{\pgfqpoint{1.722043in}{1.621697in}}%
\pgfpathlineto{\pgfqpoint{1.632244in}{1.151235in}}%
\pgfusepath{stroke}%
\end{pgfscope}%
\begin{pgfscope}%
\pgfpathrectangle{\pgfqpoint{0.100000in}{0.212622in}}{\pgfqpoint{3.696000in}{3.696000in}}%
\pgfusepath{clip}%
\pgfsetrectcap%
\pgfsetroundjoin%
\pgfsetlinewidth{1.505625pt}%
\definecolor{currentstroke}{rgb}{1.000000,0.000000,0.000000}%
\pgfsetstrokecolor{currentstroke}%
\pgfsetdash{}{0pt}%
\pgfpathmoveto{\pgfqpoint{1.722680in}{1.621399in}}%
\pgfpathlineto{\pgfqpoint{1.632244in}{1.151235in}}%
\pgfusepath{stroke}%
\end{pgfscope}%
\begin{pgfscope}%
\pgfpathrectangle{\pgfqpoint{0.100000in}{0.212622in}}{\pgfqpoint{3.696000in}{3.696000in}}%
\pgfusepath{clip}%
\pgfsetrectcap%
\pgfsetroundjoin%
\pgfsetlinewidth{1.505625pt}%
\definecolor{currentstroke}{rgb}{1.000000,0.000000,0.000000}%
\pgfsetstrokecolor{currentstroke}%
\pgfsetdash{}{0pt}%
\pgfpathmoveto{\pgfqpoint{1.723575in}{1.622219in}}%
\pgfpathlineto{\pgfqpoint{1.632244in}{1.151235in}}%
\pgfusepath{stroke}%
\end{pgfscope}%
\begin{pgfscope}%
\pgfpathrectangle{\pgfqpoint{0.100000in}{0.212622in}}{\pgfqpoint{3.696000in}{3.696000in}}%
\pgfusepath{clip}%
\pgfsetrectcap%
\pgfsetroundjoin%
\pgfsetlinewidth{1.505625pt}%
\definecolor{currentstroke}{rgb}{1.000000,0.000000,0.000000}%
\pgfsetstrokecolor{currentstroke}%
\pgfsetdash{}{0pt}%
\pgfpathmoveto{\pgfqpoint{1.723808in}{1.621491in}}%
\pgfpathlineto{\pgfqpoint{1.632244in}{1.151235in}}%
\pgfusepath{stroke}%
\end{pgfscope}%
\begin{pgfscope}%
\pgfpathrectangle{\pgfqpoint{0.100000in}{0.212622in}}{\pgfqpoint{3.696000in}{3.696000in}}%
\pgfusepath{clip}%
\pgfsetrectcap%
\pgfsetroundjoin%
\pgfsetlinewidth{1.505625pt}%
\definecolor{currentstroke}{rgb}{1.000000,0.000000,0.000000}%
\pgfsetstrokecolor{currentstroke}%
\pgfsetdash{}{0pt}%
\pgfpathmoveto{\pgfqpoint{1.724114in}{1.621630in}}%
\pgfpathlineto{\pgfqpoint{1.632244in}{1.151235in}}%
\pgfusepath{stroke}%
\end{pgfscope}%
\begin{pgfscope}%
\pgfpathrectangle{\pgfqpoint{0.100000in}{0.212622in}}{\pgfqpoint{3.696000in}{3.696000in}}%
\pgfusepath{clip}%
\pgfsetrectcap%
\pgfsetroundjoin%
\pgfsetlinewidth{1.505625pt}%
\definecolor{currentstroke}{rgb}{1.000000,0.000000,0.000000}%
\pgfsetstrokecolor{currentstroke}%
\pgfsetdash{}{0pt}%
\pgfpathmoveto{\pgfqpoint{1.724546in}{1.620995in}}%
\pgfpathlineto{\pgfqpoint{1.632244in}{1.151235in}}%
\pgfusepath{stroke}%
\end{pgfscope}%
\begin{pgfscope}%
\pgfpathrectangle{\pgfqpoint{0.100000in}{0.212622in}}{\pgfqpoint{3.696000in}{3.696000in}}%
\pgfusepath{clip}%
\pgfsetrectcap%
\pgfsetroundjoin%
\pgfsetlinewidth{1.505625pt}%
\definecolor{currentstroke}{rgb}{1.000000,0.000000,0.000000}%
\pgfsetstrokecolor{currentstroke}%
\pgfsetdash{}{0pt}%
\pgfpathmoveto{\pgfqpoint{1.725205in}{1.621862in}}%
\pgfpathlineto{\pgfqpoint{1.632244in}{1.151235in}}%
\pgfusepath{stroke}%
\end{pgfscope}%
\begin{pgfscope}%
\pgfpathrectangle{\pgfqpoint{0.100000in}{0.212622in}}{\pgfqpoint{3.696000in}{3.696000in}}%
\pgfusepath{clip}%
\pgfsetrectcap%
\pgfsetroundjoin%
\pgfsetlinewidth{1.505625pt}%
\definecolor{currentstroke}{rgb}{1.000000,0.000000,0.000000}%
\pgfsetstrokecolor{currentstroke}%
\pgfsetdash{}{0pt}%
\pgfpathmoveto{\pgfqpoint{1.725352in}{1.621752in}}%
\pgfpathlineto{\pgfqpoint{1.649285in}{1.145857in}}%
\pgfusepath{stroke}%
\end{pgfscope}%
\begin{pgfscope}%
\pgfpathrectangle{\pgfqpoint{0.100000in}{0.212622in}}{\pgfqpoint{3.696000in}{3.696000in}}%
\pgfusepath{clip}%
\pgfsetrectcap%
\pgfsetroundjoin%
\pgfsetlinewidth{1.505625pt}%
\definecolor{currentstroke}{rgb}{1.000000,0.000000,0.000000}%
\pgfsetstrokecolor{currentstroke}%
\pgfsetdash{}{0pt}%
\pgfpathmoveto{\pgfqpoint{1.725670in}{1.620722in}}%
\pgfpathlineto{\pgfqpoint{1.649285in}{1.145857in}}%
\pgfusepath{stroke}%
\end{pgfscope}%
\begin{pgfscope}%
\pgfpathrectangle{\pgfqpoint{0.100000in}{0.212622in}}{\pgfqpoint{3.696000in}{3.696000in}}%
\pgfusepath{clip}%
\pgfsetrectcap%
\pgfsetroundjoin%
\pgfsetlinewidth{1.505625pt}%
\definecolor{currentstroke}{rgb}{1.000000,0.000000,0.000000}%
\pgfsetstrokecolor{currentstroke}%
\pgfsetdash{}{0pt}%
\pgfpathmoveto{\pgfqpoint{1.726384in}{1.618577in}}%
\pgfpathlineto{\pgfqpoint{1.649285in}{1.145857in}}%
\pgfusepath{stroke}%
\end{pgfscope}%
\begin{pgfscope}%
\pgfpathrectangle{\pgfqpoint{0.100000in}{0.212622in}}{\pgfqpoint{3.696000in}{3.696000in}}%
\pgfusepath{clip}%
\pgfsetrectcap%
\pgfsetroundjoin%
\pgfsetlinewidth{1.505625pt}%
\definecolor{currentstroke}{rgb}{1.000000,0.000000,0.000000}%
\pgfsetstrokecolor{currentstroke}%
\pgfsetdash{}{0pt}%
\pgfpathmoveto{\pgfqpoint{1.727118in}{1.618144in}}%
\pgfpathlineto{\pgfqpoint{1.649285in}{1.145857in}}%
\pgfusepath{stroke}%
\end{pgfscope}%
\begin{pgfscope}%
\pgfpathrectangle{\pgfqpoint{0.100000in}{0.212622in}}{\pgfqpoint{3.696000in}{3.696000in}}%
\pgfusepath{clip}%
\pgfsetrectcap%
\pgfsetroundjoin%
\pgfsetlinewidth{1.505625pt}%
\definecolor{currentstroke}{rgb}{1.000000,0.000000,0.000000}%
\pgfsetstrokecolor{currentstroke}%
\pgfsetdash{}{0pt}%
\pgfpathmoveto{\pgfqpoint{1.727937in}{1.617767in}}%
\pgfpathlineto{\pgfqpoint{1.649285in}{1.145857in}}%
\pgfusepath{stroke}%
\end{pgfscope}%
\begin{pgfscope}%
\pgfpathrectangle{\pgfqpoint{0.100000in}{0.212622in}}{\pgfqpoint{3.696000in}{3.696000in}}%
\pgfusepath{clip}%
\pgfsetrectcap%
\pgfsetroundjoin%
\pgfsetlinewidth{1.505625pt}%
\definecolor{currentstroke}{rgb}{1.000000,0.000000,0.000000}%
\pgfsetstrokecolor{currentstroke}%
\pgfsetdash{}{0pt}%
\pgfpathmoveto{\pgfqpoint{1.729415in}{1.616918in}}%
\pgfpathlineto{\pgfqpoint{1.649285in}{1.145857in}}%
\pgfusepath{stroke}%
\end{pgfscope}%
\begin{pgfscope}%
\pgfpathrectangle{\pgfqpoint{0.100000in}{0.212622in}}{\pgfqpoint{3.696000in}{3.696000in}}%
\pgfusepath{clip}%
\pgfsetrectcap%
\pgfsetroundjoin%
\pgfsetlinewidth{1.505625pt}%
\definecolor{currentstroke}{rgb}{1.000000,0.000000,0.000000}%
\pgfsetstrokecolor{currentstroke}%
\pgfsetdash{}{0pt}%
\pgfpathmoveto{\pgfqpoint{1.730016in}{1.616086in}}%
\pgfpathlineto{\pgfqpoint{1.649285in}{1.145857in}}%
\pgfusepath{stroke}%
\end{pgfscope}%
\begin{pgfscope}%
\pgfpathrectangle{\pgfqpoint{0.100000in}{0.212622in}}{\pgfqpoint{3.696000in}{3.696000in}}%
\pgfusepath{clip}%
\pgfsetrectcap%
\pgfsetroundjoin%
\pgfsetlinewidth{1.505625pt}%
\definecolor{currentstroke}{rgb}{1.000000,0.000000,0.000000}%
\pgfsetstrokecolor{currentstroke}%
\pgfsetdash{}{0pt}%
\pgfpathmoveto{\pgfqpoint{1.730762in}{1.613546in}}%
\pgfpathlineto{\pgfqpoint{1.649285in}{1.145857in}}%
\pgfusepath{stroke}%
\end{pgfscope}%
\begin{pgfscope}%
\pgfpathrectangle{\pgfqpoint{0.100000in}{0.212622in}}{\pgfqpoint{3.696000in}{3.696000in}}%
\pgfusepath{clip}%
\pgfsetrectcap%
\pgfsetroundjoin%
\pgfsetlinewidth{1.505625pt}%
\definecolor{currentstroke}{rgb}{1.000000,0.000000,0.000000}%
\pgfsetstrokecolor{currentstroke}%
\pgfsetdash{}{0pt}%
\pgfpathmoveto{\pgfqpoint{1.731228in}{1.614057in}}%
\pgfpathlineto{\pgfqpoint{1.649285in}{1.145857in}}%
\pgfusepath{stroke}%
\end{pgfscope}%
\begin{pgfscope}%
\pgfpathrectangle{\pgfqpoint{0.100000in}{0.212622in}}{\pgfqpoint{3.696000in}{3.696000in}}%
\pgfusepath{clip}%
\pgfsetrectcap%
\pgfsetroundjoin%
\pgfsetlinewidth{1.505625pt}%
\definecolor{currentstroke}{rgb}{1.000000,0.000000,0.000000}%
\pgfsetstrokecolor{currentstroke}%
\pgfsetdash{}{0pt}%
\pgfpathmoveto{\pgfqpoint{1.731771in}{1.613768in}}%
\pgfpathlineto{\pgfqpoint{1.666341in}{1.140474in}}%
\pgfusepath{stroke}%
\end{pgfscope}%
\begin{pgfscope}%
\pgfpathrectangle{\pgfqpoint{0.100000in}{0.212622in}}{\pgfqpoint{3.696000in}{3.696000in}}%
\pgfusepath{clip}%
\pgfsetrectcap%
\pgfsetroundjoin%
\pgfsetlinewidth{1.505625pt}%
\definecolor{currentstroke}{rgb}{1.000000,0.000000,0.000000}%
\pgfsetstrokecolor{currentstroke}%
\pgfsetdash{}{0pt}%
\pgfpathmoveto{\pgfqpoint{1.731970in}{1.612966in}}%
\pgfpathlineto{\pgfqpoint{1.666341in}{1.140474in}}%
\pgfusepath{stroke}%
\end{pgfscope}%
\begin{pgfscope}%
\pgfpathrectangle{\pgfqpoint{0.100000in}{0.212622in}}{\pgfqpoint{3.696000in}{3.696000in}}%
\pgfusepath{clip}%
\pgfsetrectcap%
\pgfsetroundjoin%
\pgfsetlinewidth{1.505625pt}%
\definecolor{currentstroke}{rgb}{1.000000,0.000000,0.000000}%
\pgfsetstrokecolor{currentstroke}%
\pgfsetdash{}{0pt}%
\pgfpathmoveto{\pgfqpoint{1.732479in}{1.610942in}}%
\pgfpathlineto{\pgfqpoint{1.666341in}{1.140474in}}%
\pgfusepath{stroke}%
\end{pgfscope}%
\begin{pgfscope}%
\pgfpathrectangle{\pgfqpoint{0.100000in}{0.212622in}}{\pgfqpoint{3.696000in}{3.696000in}}%
\pgfusepath{clip}%
\pgfsetrectcap%
\pgfsetroundjoin%
\pgfsetlinewidth{1.505625pt}%
\definecolor{currentstroke}{rgb}{1.000000,0.000000,0.000000}%
\pgfsetstrokecolor{currentstroke}%
\pgfsetdash{}{0pt}%
\pgfpathmoveto{\pgfqpoint{1.733287in}{1.611807in}}%
\pgfpathlineto{\pgfqpoint{1.666341in}{1.140474in}}%
\pgfusepath{stroke}%
\end{pgfscope}%
\begin{pgfscope}%
\pgfpathrectangle{\pgfqpoint{0.100000in}{0.212622in}}{\pgfqpoint{3.696000in}{3.696000in}}%
\pgfusepath{clip}%
\pgfsetrectcap%
\pgfsetroundjoin%
\pgfsetlinewidth{1.505625pt}%
\definecolor{currentstroke}{rgb}{1.000000,0.000000,0.000000}%
\pgfsetstrokecolor{currentstroke}%
\pgfsetdash{}{0pt}%
\pgfpathmoveto{\pgfqpoint{1.734365in}{1.612474in}}%
\pgfpathlineto{\pgfqpoint{1.666341in}{1.140474in}}%
\pgfusepath{stroke}%
\end{pgfscope}%
\begin{pgfscope}%
\pgfpathrectangle{\pgfqpoint{0.100000in}{0.212622in}}{\pgfqpoint{3.696000in}{3.696000in}}%
\pgfusepath{clip}%
\pgfsetrectcap%
\pgfsetroundjoin%
\pgfsetlinewidth{1.505625pt}%
\definecolor{currentstroke}{rgb}{1.000000,0.000000,0.000000}%
\pgfsetstrokecolor{currentstroke}%
\pgfsetdash{}{0pt}%
\pgfpathmoveto{\pgfqpoint{1.734883in}{1.610863in}}%
\pgfpathlineto{\pgfqpoint{1.666341in}{1.140474in}}%
\pgfusepath{stroke}%
\end{pgfscope}%
\begin{pgfscope}%
\pgfpathrectangle{\pgfqpoint{0.100000in}{0.212622in}}{\pgfqpoint{3.696000in}{3.696000in}}%
\pgfusepath{clip}%
\pgfsetrectcap%
\pgfsetroundjoin%
\pgfsetlinewidth{1.505625pt}%
\definecolor{currentstroke}{rgb}{1.000000,0.000000,0.000000}%
\pgfsetstrokecolor{currentstroke}%
\pgfsetdash{}{0pt}%
\pgfpathmoveto{\pgfqpoint{1.736419in}{1.609253in}}%
\pgfpathlineto{\pgfqpoint{1.666341in}{1.140474in}}%
\pgfusepath{stroke}%
\end{pgfscope}%
\begin{pgfscope}%
\pgfpathrectangle{\pgfqpoint{0.100000in}{0.212622in}}{\pgfqpoint{3.696000in}{3.696000in}}%
\pgfusepath{clip}%
\pgfsetrectcap%
\pgfsetroundjoin%
\pgfsetlinewidth{1.505625pt}%
\definecolor{currentstroke}{rgb}{1.000000,0.000000,0.000000}%
\pgfsetstrokecolor{currentstroke}%
\pgfsetdash{}{0pt}%
\pgfpathmoveto{\pgfqpoint{1.737836in}{1.606746in}}%
\pgfpathlineto{\pgfqpoint{1.683412in}{1.135087in}}%
\pgfusepath{stroke}%
\end{pgfscope}%
\begin{pgfscope}%
\pgfpathrectangle{\pgfqpoint{0.100000in}{0.212622in}}{\pgfqpoint{3.696000in}{3.696000in}}%
\pgfusepath{clip}%
\pgfsetrectcap%
\pgfsetroundjoin%
\pgfsetlinewidth{1.505625pt}%
\definecolor{currentstroke}{rgb}{1.000000,0.000000,0.000000}%
\pgfsetstrokecolor{currentstroke}%
\pgfsetdash{}{0pt}%
\pgfpathmoveto{\pgfqpoint{1.740444in}{1.611416in}}%
\pgfpathlineto{\pgfqpoint{1.683412in}{1.135087in}}%
\pgfusepath{stroke}%
\end{pgfscope}%
\begin{pgfscope}%
\pgfpathrectangle{\pgfqpoint{0.100000in}{0.212622in}}{\pgfqpoint{3.696000in}{3.696000in}}%
\pgfusepath{clip}%
\pgfsetrectcap%
\pgfsetroundjoin%
\pgfsetlinewidth{1.505625pt}%
\definecolor{currentstroke}{rgb}{1.000000,0.000000,0.000000}%
\pgfsetstrokecolor{currentstroke}%
\pgfsetdash{}{0pt}%
\pgfpathmoveto{\pgfqpoint{1.741504in}{1.609768in}}%
\pgfpathlineto{\pgfqpoint{1.683412in}{1.135087in}}%
\pgfusepath{stroke}%
\end{pgfscope}%
\begin{pgfscope}%
\pgfpathrectangle{\pgfqpoint{0.100000in}{0.212622in}}{\pgfqpoint{3.696000in}{3.696000in}}%
\pgfusepath{clip}%
\pgfsetrectcap%
\pgfsetroundjoin%
\pgfsetlinewidth{1.505625pt}%
\definecolor{currentstroke}{rgb}{1.000000,0.000000,0.000000}%
\pgfsetstrokecolor{currentstroke}%
\pgfsetdash{}{0pt}%
\pgfpathmoveto{\pgfqpoint{1.743675in}{1.608335in}}%
\pgfpathlineto{\pgfqpoint{1.683412in}{1.135087in}}%
\pgfusepath{stroke}%
\end{pgfscope}%
\begin{pgfscope}%
\pgfpathrectangle{\pgfqpoint{0.100000in}{0.212622in}}{\pgfqpoint{3.696000in}{3.696000in}}%
\pgfusepath{clip}%
\pgfsetrectcap%
\pgfsetroundjoin%
\pgfsetlinewidth{1.505625pt}%
\definecolor{currentstroke}{rgb}{1.000000,0.000000,0.000000}%
\pgfsetstrokecolor{currentstroke}%
\pgfsetdash{}{0pt}%
\pgfpathmoveto{\pgfqpoint{1.745755in}{1.598247in}}%
\pgfpathlineto{\pgfqpoint{1.700498in}{1.129696in}}%
\pgfusepath{stroke}%
\end{pgfscope}%
\begin{pgfscope}%
\pgfpathrectangle{\pgfqpoint{0.100000in}{0.212622in}}{\pgfqpoint{3.696000in}{3.696000in}}%
\pgfusepath{clip}%
\pgfsetrectcap%
\pgfsetroundjoin%
\pgfsetlinewidth{1.505625pt}%
\definecolor{currentstroke}{rgb}{1.000000,0.000000,0.000000}%
\pgfsetstrokecolor{currentstroke}%
\pgfsetdash{}{0pt}%
\pgfpathmoveto{\pgfqpoint{1.747098in}{1.598450in}}%
\pgfpathlineto{\pgfqpoint{1.700498in}{1.129696in}}%
\pgfusepath{stroke}%
\end{pgfscope}%
\begin{pgfscope}%
\pgfpathrectangle{\pgfqpoint{0.100000in}{0.212622in}}{\pgfqpoint{3.696000in}{3.696000in}}%
\pgfusepath{clip}%
\pgfsetrectcap%
\pgfsetroundjoin%
\pgfsetlinewidth{1.505625pt}%
\definecolor{currentstroke}{rgb}{1.000000,0.000000,0.000000}%
\pgfsetstrokecolor{currentstroke}%
\pgfsetdash{}{0pt}%
\pgfpathmoveto{\pgfqpoint{1.748908in}{1.599335in}}%
\pgfpathlineto{\pgfqpoint{1.700498in}{1.129696in}}%
\pgfusepath{stroke}%
\end{pgfscope}%
\begin{pgfscope}%
\pgfpathrectangle{\pgfqpoint{0.100000in}{0.212622in}}{\pgfqpoint{3.696000in}{3.696000in}}%
\pgfusepath{clip}%
\pgfsetrectcap%
\pgfsetroundjoin%
\pgfsetlinewidth{1.505625pt}%
\definecolor{currentstroke}{rgb}{1.000000,0.000000,0.000000}%
\pgfsetstrokecolor{currentstroke}%
\pgfsetdash{}{0pt}%
\pgfpathmoveto{\pgfqpoint{1.749872in}{1.597290in}}%
\pgfpathlineto{\pgfqpoint{1.700498in}{1.129696in}}%
\pgfusepath{stroke}%
\end{pgfscope}%
\begin{pgfscope}%
\pgfpathrectangle{\pgfqpoint{0.100000in}{0.212622in}}{\pgfqpoint{3.696000in}{3.696000in}}%
\pgfusepath{clip}%
\pgfsetrectcap%
\pgfsetroundjoin%
\pgfsetlinewidth{1.505625pt}%
\definecolor{currentstroke}{rgb}{1.000000,0.000000,0.000000}%
\pgfsetstrokecolor{currentstroke}%
\pgfsetdash{}{0pt}%
\pgfpathmoveto{\pgfqpoint{1.750573in}{1.597053in}}%
\pgfpathlineto{\pgfqpoint{1.717598in}{1.124299in}}%
\pgfusepath{stroke}%
\end{pgfscope}%
\begin{pgfscope}%
\pgfpathrectangle{\pgfqpoint{0.100000in}{0.212622in}}{\pgfqpoint{3.696000in}{3.696000in}}%
\pgfusepath{clip}%
\pgfsetrectcap%
\pgfsetroundjoin%
\pgfsetlinewidth{1.505625pt}%
\definecolor{currentstroke}{rgb}{1.000000,0.000000,0.000000}%
\pgfsetstrokecolor{currentstroke}%
\pgfsetdash{}{0pt}%
\pgfpathmoveto{\pgfqpoint{1.751058in}{1.595346in}}%
\pgfpathlineto{\pgfqpoint{1.717598in}{1.124299in}}%
\pgfusepath{stroke}%
\end{pgfscope}%
\begin{pgfscope}%
\pgfpathrectangle{\pgfqpoint{0.100000in}{0.212622in}}{\pgfqpoint{3.696000in}{3.696000in}}%
\pgfusepath{clip}%
\pgfsetrectcap%
\pgfsetroundjoin%
\pgfsetlinewidth{1.505625pt}%
\definecolor{currentstroke}{rgb}{1.000000,0.000000,0.000000}%
\pgfsetstrokecolor{currentstroke}%
\pgfsetdash{}{0pt}%
\pgfpathmoveto{\pgfqpoint{1.751302in}{1.595395in}}%
\pgfpathlineto{\pgfqpoint{1.717598in}{1.124299in}}%
\pgfusepath{stroke}%
\end{pgfscope}%
\begin{pgfscope}%
\pgfpathrectangle{\pgfqpoint{0.100000in}{0.212622in}}{\pgfqpoint{3.696000in}{3.696000in}}%
\pgfusepath{clip}%
\pgfsetrectcap%
\pgfsetroundjoin%
\pgfsetlinewidth{1.505625pt}%
\definecolor{currentstroke}{rgb}{1.000000,0.000000,0.000000}%
\pgfsetstrokecolor{currentstroke}%
\pgfsetdash{}{0pt}%
\pgfpathmoveto{\pgfqpoint{1.751808in}{1.595624in}}%
\pgfpathlineto{\pgfqpoint{1.717598in}{1.124299in}}%
\pgfusepath{stroke}%
\end{pgfscope}%
\begin{pgfscope}%
\pgfpathrectangle{\pgfqpoint{0.100000in}{0.212622in}}{\pgfqpoint{3.696000in}{3.696000in}}%
\pgfusepath{clip}%
\pgfsetrectcap%
\pgfsetroundjoin%
\pgfsetlinewidth{1.505625pt}%
\definecolor{currentstroke}{rgb}{1.000000,0.000000,0.000000}%
\pgfsetstrokecolor{currentstroke}%
\pgfsetdash{}{0pt}%
\pgfpathmoveto{\pgfqpoint{1.752375in}{1.594055in}}%
\pgfpathlineto{\pgfqpoint{1.717598in}{1.124299in}}%
\pgfusepath{stroke}%
\end{pgfscope}%
\begin{pgfscope}%
\pgfpathrectangle{\pgfqpoint{0.100000in}{0.212622in}}{\pgfqpoint{3.696000in}{3.696000in}}%
\pgfusepath{clip}%
\pgfsetrectcap%
\pgfsetroundjoin%
\pgfsetlinewidth{1.505625pt}%
\definecolor{currentstroke}{rgb}{1.000000,0.000000,0.000000}%
\pgfsetstrokecolor{currentstroke}%
\pgfsetdash{}{0pt}%
\pgfpathmoveto{\pgfqpoint{1.753257in}{1.592020in}}%
\pgfpathlineto{\pgfqpoint{1.717598in}{1.124299in}}%
\pgfusepath{stroke}%
\end{pgfscope}%
\begin{pgfscope}%
\pgfpathrectangle{\pgfqpoint{0.100000in}{0.212622in}}{\pgfqpoint{3.696000in}{3.696000in}}%
\pgfusepath{clip}%
\pgfsetrectcap%
\pgfsetroundjoin%
\pgfsetlinewidth{1.505625pt}%
\definecolor{currentstroke}{rgb}{1.000000,0.000000,0.000000}%
\pgfsetstrokecolor{currentstroke}%
\pgfsetdash{}{0pt}%
\pgfpathmoveto{\pgfqpoint{1.753782in}{1.591399in}}%
\pgfpathlineto{\pgfqpoint{1.717598in}{1.124299in}}%
\pgfusepath{stroke}%
\end{pgfscope}%
\begin{pgfscope}%
\pgfpathrectangle{\pgfqpoint{0.100000in}{0.212622in}}{\pgfqpoint{3.696000in}{3.696000in}}%
\pgfusepath{clip}%
\pgfsetrectcap%
\pgfsetroundjoin%
\pgfsetlinewidth{1.505625pt}%
\definecolor{currentstroke}{rgb}{1.000000,0.000000,0.000000}%
\pgfsetstrokecolor{currentstroke}%
\pgfsetdash{}{0pt}%
\pgfpathmoveto{\pgfqpoint{1.754153in}{1.591138in}}%
\pgfpathlineto{\pgfqpoint{1.717598in}{1.124299in}}%
\pgfusepath{stroke}%
\end{pgfscope}%
\begin{pgfscope}%
\pgfpathrectangle{\pgfqpoint{0.100000in}{0.212622in}}{\pgfqpoint{3.696000in}{3.696000in}}%
\pgfusepath{clip}%
\pgfsetrectcap%
\pgfsetroundjoin%
\pgfsetlinewidth{1.505625pt}%
\definecolor{currentstroke}{rgb}{1.000000,0.000000,0.000000}%
\pgfsetstrokecolor{currentstroke}%
\pgfsetdash{}{0pt}%
\pgfpathmoveto{\pgfqpoint{1.754320in}{1.590888in}}%
\pgfpathlineto{\pgfqpoint{1.717598in}{1.124299in}}%
\pgfusepath{stroke}%
\end{pgfscope}%
\begin{pgfscope}%
\pgfpathrectangle{\pgfqpoint{0.100000in}{0.212622in}}{\pgfqpoint{3.696000in}{3.696000in}}%
\pgfusepath{clip}%
\pgfsetrectcap%
\pgfsetroundjoin%
\pgfsetlinewidth{1.505625pt}%
\definecolor{currentstroke}{rgb}{1.000000,0.000000,0.000000}%
\pgfsetstrokecolor{currentstroke}%
\pgfsetdash{}{0pt}%
\pgfpathmoveto{\pgfqpoint{1.754794in}{1.590752in}}%
\pgfpathlineto{\pgfqpoint{1.717598in}{1.124299in}}%
\pgfusepath{stroke}%
\end{pgfscope}%
\begin{pgfscope}%
\pgfpathrectangle{\pgfqpoint{0.100000in}{0.212622in}}{\pgfqpoint{3.696000in}{3.696000in}}%
\pgfusepath{clip}%
\pgfsetrectcap%
\pgfsetroundjoin%
\pgfsetlinewidth{1.505625pt}%
\definecolor{currentstroke}{rgb}{1.000000,0.000000,0.000000}%
\pgfsetstrokecolor{currentstroke}%
\pgfsetdash{}{0pt}%
\pgfpathmoveto{\pgfqpoint{1.754935in}{1.590107in}}%
\pgfpathlineto{\pgfqpoint{1.717598in}{1.124299in}}%
\pgfusepath{stroke}%
\end{pgfscope}%
\begin{pgfscope}%
\pgfpathrectangle{\pgfqpoint{0.100000in}{0.212622in}}{\pgfqpoint{3.696000in}{3.696000in}}%
\pgfusepath{clip}%
\pgfsetrectcap%
\pgfsetroundjoin%
\pgfsetlinewidth{1.505625pt}%
\definecolor{currentstroke}{rgb}{1.000000,0.000000,0.000000}%
\pgfsetstrokecolor{currentstroke}%
\pgfsetdash{}{0pt}%
\pgfpathmoveto{\pgfqpoint{1.755336in}{1.590440in}}%
\pgfpathlineto{\pgfqpoint{1.717598in}{1.124299in}}%
\pgfusepath{stroke}%
\end{pgfscope}%
\begin{pgfscope}%
\pgfpathrectangle{\pgfqpoint{0.100000in}{0.212622in}}{\pgfqpoint{3.696000in}{3.696000in}}%
\pgfusepath{clip}%
\pgfsetrectcap%
\pgfsetroundjoin%
\pgfsetlinewidth{1.505625pt}%
\definecolor{currentstroke}{rgb}{1.000000,0.000000,0.000000}%
\pgfsetstrokecolor{currentstroke}%
\pgfsetdash{}{0pt}%
\pgfpathmoveto{\pgfqpoint{1.755520in}{1.590412in}}%
\pgfpathlineto{\pgfqpoint{1.717598in}{1.124299in}}%
\pgfusepath{stroke}%
\end{pgfscope}%
\begin{pgfscope}%
\pgfpathrectangle{\pgfqpoint{0.100000in}{0.212622in}}{\pgfqpoint{3.696000in}{3.696000in}}%
\pgfusepath{clip}%
\pgfsetrectcap%
\pgfsetroundjoin%
\pgfsetlinewidth{1.505625pt}%
\definecolor{currentstroke}{rgb}{1.000000,0.000000,0.000000}%
\pgfsetstrokecolor{currentstroke}%
\pgfsetdash{}{0pt}%
\pgfpathmoveto{\pgfqpoint{1.755568in}{1.590152in}}%
\pgfpathlineto{\pgfqpoint{1.717598in}{1.124299in}}%
\pgfusepath{stroke}%
\end{pgfscope}%
\begin{pgfscope}%
\pgfpathrectangle{\pgfqpoint{0.100000in}{0.212622in}}{\pgfqpoint{3.696000in}{3.696000in}}%
\pgfusepath{clip}%
\pgfsetrectcap%
\pgfsetroundjoin%
\pgfsetlinewidth{1.505625pt}%
\definecolor{currentstroke}{rgb}{1.000000,0.000000,0.000000}%
\pgfsetstrokecolor{currentstroke}%
\pgfsetdash{}{0pt}%
\pgfpathmoveto{\pgfqpoint{1.755991in}{1.588701in}}%
\pgfpathlineto{\pgfqpoint{1.717598in}{1.124299in}}%
\pgfusepath{stroke}%
\end{pgfscope}%
\begin{pgfscope}%
\pgfpathrectangle{\pgfqpoint{0.100000in}{0.212622in}}{\pgfqpoint{3.696000in}{3.696000in}}%
\pgfusepath{clip}%
\pgfsetrectcap%
\pgfsetroundjoin%
\pgfsetlinewidth{1.505625pt}%
\definecolor{currentstroke}{rgb}{1.000000,0.000000,0.000000}%
\pgfsetstrokecolor{currentstroke}%
\pgfsetdash{}{0pt}%
\pgfpathmoveto{\pgfqpoint{1.756420in}{1.588477in}}%
\pgfpathlineto{\pgfqpoint{1.734713in}{1.118898in}}%
\pgfusepath{stroke}%
\end{pgfscope}%
\begin{pgfscope}%
\pgfpathrectangle{\pgfqpoint{0.100000in}{0.212622in}}{\pgfqpoint{3.696000in}{3.696000in}}%
\pgfusepath{clip}%
\pgfsetrectcap%
\pgfsetroundjoin%
\pgfsetlinewidth{1.505625pt}%
\definecolor{currentstroke}{rgb}{1.000000,0.000000,0.000000}%
\pgfsetstrokecolor{currentstroke}%
\pgfsetdash{}{0pt}%
\pgfpathmoveto{\pgfqpoint{1.757223in}{1.588367in}}%
\pgfpathlineto{\pgfqpoint{1.734713in}{1.118898in}}%
\pgfusepath{stroke}%
\end{pgfscope}%
\begin{pgfscope}%
\pgfpathrectangle{\pgfqpoint{0.100000in}{0.212622in}}{\pgfqpoint{3.696000in}{3.696000in}}%
\pgfusepath{clip}%
\pgfsetrectcap%
\pgfsetroundjoin%
\pgfsetlinewidth{1.505625pt}%
\definecolor{currentstroke}{rgb}{1.000000,0.000000,0.000000}%
\pgfsetstrokecolor{currentstroke}%
\pgfsetdash{}{0pt}%
\pgfpathmoveto{\pgfqpoint{1.757644in}{1.588407in}}%
\pgfpathlineto{\pgfqpoint{1.734713in}{1.118898in}}%
\pgfusepath{stroke}%
\end{pgfscope}%
\begin{pgfscope}%
\pgfpathrectangle{\pgfqpoint{0.100000in}{0.212622in}}{\pgfqpoint{3.696000in}{3.696000in}}%
\pgfusepath{clip}%
\pgfsetrectcap%
\pgfsetroundjoin%
\pgfsetlinewidth{1.505625pt}%
\definecolor{currentstroke}{rgb}{1.000000,0.000000,0.000000}%
\pgfsetstrokecolor{currentstroke}%
\pgfsetdash{}{0pt}%
\pgfpathmoveto{\pgfqpoint{1.758103in}{1.587949in}}%
\pgfpathlineto{\pgfqpoint{1.734713in}{1.118898in}}%
\pgfusepath{stroke}%
\end{pgfscope}%
\begin{pgfscope}%
\pgfpathrectangle{\pgfqpoint{0.100000in}{0.212622in}}{\pgfqpoint{3.696000in}{3.696000in}}%
\pgfusepath{clip}%
\pgfsetrectcap%
\pgfsetroundjoin%
\pgfsetlinewidth{1.505625pt}%
\definecolor{currentstroke}{rgb}{1.000000,0.000000,0.000000}%
\pgfsetstrokecolor{currentstroke}%
\pgfsetdash{}{0pt}%
\pgfpathmoveto{\pgfqpoint{1.758870in}{1.586586in}}%
\pgfpathlineto{\pgfqpoint{1.734713in}{1.118898in}}%
\pgfusepath{stroke}%
\end{pgfscope}%
\begin{pgfscope}%
\pgfpathrectangle{\pgfqpoint{0.100000in}{0.212622in}}{\pgfqpoint{3.696000in}{3.696000in}}%
\pgfusepath{clip}%
\pgfsetrectcap%
\pgfsetroundjoin%
\pgfsetlinewidth{1.505625pt}%
\definecolor{currentstroke}{rgb}{1.000000,0.000000,0.000000}%
\pgfsetstrokecolor{currentstroke}%
\pgfsetdash{}{0pt}%
\pgfpathmoveto{\pgfqpoint{1.759269in}{1.587061in}}%
\pgfpathlineto{\pgfqpoint{1.734713in}{1.118898in}}%
\pgfusepath{stroke}%
\end{pgfscope}%
\begin{pgfscope}%
\pgfpathrectangle{\pgfqpoint{0.100000in}{0.212622in}}{\pgfqpoint{3.696000in}{3.696000in}}%
\pgfusepath{clip}%
\pgfsetrectcap%
\pgfsetroundjoin%
\pgfsetlinewidth{1.505625pt}%
\definecolor{currentstroke}{rgb}{1.000000,0.000000,0.000000}%
\pgfsetstrokecolor{currentstroke}%
\pgfsetdash{}{0pt}%
\pgfpathmoveto{\pgfqpoint{1.759635in}{1.587508in}}%
\pgfpathlineto{\pgfqpoint{1.734713in}{1.118898in}}%
\pgfusepath{stroke}%
\end{pgfscope}%
\begin{pgfscope}%
\pgfpathrectangle{\pgfqpoint{0.100000in}{0.212622in}}{\pgfqpoint{3.696000in}{3.696000in}}%
\pgfusepath{clip}%
\pgfsetrectcap%
\pgfsetroundjoin%
\pgfsetlinewidth{1.505625pt}%
\definecolor{currentstroke}{rgb}{1.000000,0.000000,0.000000}%
\pgfsetstrokecolor{currentstroke}%
\pgfsetdash{}{0pt}%
\pgfpathmoveto{\pgfqpoint{1.759918in}{1.586781in}}%
\pgfpathlineto{\pgfqpoint{1.734713in}{1.118898in}}%
\pgfusepath{stroke}%
\end{pgfscope}%
\begin{pgfscope}%
\pgfpathrectangle{\pgfqpoint{0.100000in}{0.212622in}}{\pgfqpoint{3.696000in}{3.696000in}}%
\pgfusepath{clip}%
\pgfsetrectcap%
\pgfsetroundjoin%
\pgfsetlinewidth{1.505625pt}%
\definecolor{currentstroke}{rgb}{1.000000,0.000000,0.000000}%
\pgfsetstrokecolor{currentstroke}%
\pgfsetdash{}{0pt}%
\pgfpathmoveto{\pgfqpoint{1.760409in}{1.585259in}}%
\pgfpathlineto{\pgfqpoint{1.734713in}{1.118898in}}%
\pgfusepath{stroke}%
\end{pgfscope}%
\begin{pgfscope}%
\pgfpathrectangle{\pgfqpoint{0.100000in}{0.212622in}}{\pgfqpoint{3.696000in}{3.696000in}}%
\pgfusepath{clip}%
\pgfsetrectcap%
\pgfsetroundjoin%
\pgfsetlinewidth{1.505625pt}%
\definecolor{currentstroke}{rgb}{1.000000,0.000000,0.000000}%
\pgfsetstrokecolor{currentstroke}%
\pgfsetdash{}{0pt}%
\pgfpathmoveto{\pgfqpoint{1.760975in}{1.585522in}}%
\pgfpathlineto{\pgfqpoint{1.734713in}{1.118898in}}%
\pgfusepath{stroke}%
\end{pgfscope}%
\begin{pgfscope}%
\pgfpathrectangle{\pgfqpoint{0.100000in}{0.212622in}}{\pgfqpoint{3.696000in}{3.696000in}}%
\pgfusepath{clip}%
\pgfsetrectcap%
\pgfsetroundjoin%
\pgfsetlinewidth{1.505625pt}%
\definecolor{currentstroke}{rgb}{1.000000,0.000000,0.000000}%
\pgfsetstrokecolor{currentstroke}%
\pgfsetdash{}{0pt}%
\pgfpathmoveto{\pgfqpoint{1.762606in}{1.586108in}}%
\pgfpathlineto{\pgfqpoint{1.734713in}{1.118898in}}%
\pgfusepath{stroke}%
\end{pgfscope}%
\begin{pgfscope}%
\pgfpathrectangle{\pgfqpoint{0.100000in}{0.212622in}}{\pgfqpoint{3.696000in}{3.696000in}}%
\pgfusepath{clip}%
\pgfsetrectcap%
\pgfsetroundjoin%
\pgfsetlinewidth{1.505625pt}%
\definecolor{currentstroke}{rgb}{1.000000,0.000000,0.000000}%
\pgfsetstrokecolor{currentstroke}%
\pgfsetdash{}{0pt}%
\pgfpathmoveto{\pgfqpoint{1.763410in}{1.584437in}}%
\pgfpathlineto{\pgfqpoint{1.751843in}{1.113493in}}%
\pgfusepath{stroke}%
\end{pgfscope}%
\begin{pgfscope}%
\pgfpathrectangle{\pgfqpoint{0.100000in}{0.212622in}}{\pgfqpoint{3.696000in}{3.696000in}}%
\pgfusepath{clip}%
\pgfsetrectcap%
\pgfsetroundjoin%
\pgfsetlinewidth{1.505625pt}%
\definecolor{currentstroke}{rgb}{1.000000,0.000000,0.000000}%
\pgfsetstrokecolor{currentstroke}%
\pgfsetdash{}{0pt}%
\pgfpathmoveto{\pgfqpoint{1.764790in}{1.581482in}}%
\pgfpathlineto{\pgfqpoint{1.751843in}{1.113493in}}%
\pgfusepath{stroke}%
\end{pgfscope}%
\begin{pgfscope}%
\pgfpathrectangle{\pgfqpoint{0.100000in}{0.212622in}}{\pgfqpoint{3.696000in}{3.696000in}}%
\pgfusepath{clip}%
\pgfsetrectcap%
\pgfsetroundjoin%
\pgfsetlinewidth{1.505625pt}%
\definecolor{currentstroke}{rgb}{1.000000,0.000000,0.000000}%
\pgfsetstrokecolor{currentstroke}%
\pgfsetdash{}{0pt}%
\pgfpathmoveto{\pgfqpoint{1.766113in}{1.579677in}}%
\pgfpathlineto{\pgfqpoint{1.751843in}{1.113493in}}%
\pgfusepath{stroke}%
\end{pgfscope}%
\begin{pgfscope}%
\pgfpathrectangle{\pgfqpoint{0.100000in}{0.212622in}}{\pgfqpoint{3.696000in}{3.696000in}}%
\pgfusepath{clip}%
\pgfsetrectcap%
\pgfsetroundjoin%
\pgfsetlinewidth{1.505625pt}%
\definecolor{currentstroke}{rgb}{1.000000,0.000000,0.000000}%
\pgfsetstrokecolor{currentstroke}%
\pgfsetdash{}{0pt}%
\pgfpathmoveto{\pgfqpoint{1.767198in}{1.580176in}}%
\pgfpathlineto{\pgfqpoint{1.751843in}{1.113493in}}%
\pgfusepath{stroke}%
\end{pgfscope}%
\begin{pgfscope}%
\pgfpathrectangle{\pgfqpoint{0.100000in}{0.212622in}}{\pgfqpoint{3.696000in}{3.696000in}}%
\pgfusepath{clip}%
\pgfsetrectcap%
\pgfsetroundjoin%
\pgfsetlinewidth{1.505625pt}%
\definecolor{currentstroke}{rgb}{1.000000,0.000000,0.000000}%
\pgfsetstrokecolor{currentstroke}%
\pgfsetdash{}{0pt}%
\pgfpathmoveto{\pgfqpoint{1.768025in}{1.579183in}}%
\pgfpathlineto{\pgfqpoint{1.751843in}{1.113493in}}%
\pgfusepath{stroke}%
\end{pgfscope}%
\begin{pgfscope}%
\pgfpathrectangle{\pgfqpoint{0.100000in}{0.212622in}}{\pgfqpoint{3.696000in}{3.696000in}}%
\pgfusepath{clip}%
\pgfsetrectcap%
\pgfsetroundjoin%
\pgfsetlinewidth{1.505625pt}%
\definecolor{currentstroke}{rgb}{1.000000,0.000000,0.000000}%
\pgfsetstrokecolor{currentstroke}%
\pgfsetdash{}{0pt}%
\pgfpathmoveto{\pgfqpoint{1.769140in}{1.577190in}}%
\pgfpathlineto{\pgfqpoint{1.751843in}{1.113493in}}%
\pgfusepath{stroke}%
\end{pgfscope}%
\begin{pgfscope}%
\pgfpathrectangle{\pgfqpoint{0.100000in}{0.212622in}}{\pgfqpoint{3.696000in}{3.696000in}}%
\pgfusepath{clip}%
\pgfsetrectcap%
\pgfsetroundjoin%
\pgfsetlinewidth{1.505625pt}%
\definecolor{currentstroke}{rgb}{1.000000,0.000000,0.000000}%
\pgfsetstrokecolor{currentstroke}%
\pgfsetdash{}{0pt}%
\pgfpathmoveto{\pgfqpoint{1.770463in}{1.572746in}}%
\pgfpathlineto{\pgfqpoint{1.768988in}{1.108082in}}%
\pgfusepath{stroke}%
\end{pgfscope}%
\begin{pgfscope}%
\pgfpathrectangle{\pgfqpoint{0.100000in}{0.212622in}}{\pgfqpoint{3.696000in}{3.696000in}}%
\pgfusepath{clip}%
\pgfsetrectcap%
\pgfsetroundjoin%
\pgfsetlinewidth{1.505625pt}%
\definecolor{currentstroke}{rgb}{1.000000,0.000000,0.000000}%
\pgfsetstrokecolor{currentstroke}%
\pgfsetdash{}{0pt}%
\pgfpathmoveto{\pgfqpoint{1.771874in}{1.572631in}}%
\pgfpathlineto{\pgfqpoint{1.768988in}{1.108082in}}%
\pgfusepath{stroke}%
\end{pgfscope}%
\begin{pgfscope}%
\pgfpathrectangle{\pgfqpoint{0.100000in}{0.212622in}}{\pgfqpoint{3.696000in}{3.696000in}}%
\pgfusepath{clip}%
\pgfsetrectcap%
\pgfsetroundjoin%
\pgfsetlinewidth{1.505625pt}%
\definecolor{currentstroke}{rgb}{1.000000,0.000000,0.000000}%
\pgfsetstrokecolor{currentstroke}%
\pgfsetdash{}{0pt}%
\pgfpathmoveto{\pgfqpoint{1.774385in}{1.572474in}}%
\pgfpathlineto{\pgfqpoint{1.768988in}{1.108082in}}%
\pgfusepath{stroke}%
\end{pgfscope}%
\begin{pgfscope}%
\pgfpathrectangle{\pgfqpoint{0.100000in}{0.212622in}}{\pgfqpoint{3.696000in}{3.696000in}}%
\pgfusepath{clip}%
\pgfsetrectcap%
\pgfsetroundjoin%
\pgfsetlinewidth{1.505625pt}%
\definecolor{currentstroke}{rgb}{1.000000,0.000000,0.000000}%
\pgfsetstrokecolor{currentstroke}%
\pgfsetdash{}{0pt}%
\pgfpathmoveto{\pgfqpoint{1.776345in}{1.572444in}}%
\pgfpathlineto{\pgfqpoint{1.768988in}{1.108082in}}%
\pgfusepath{stroke}%
\end{pgfscope}%
\begin{pgfscope}%
\pgfpathrectangle{\pgfqpoint{0.100000in}{0.212622in}}{\pgfqpoint{3.696000in}{3.696000in}}%
\pgfusepath{clip}%
\pgfsetrectcap%
\pgfsetroundjoin%
\pgfsetlinewidth{1.505625pt}%
\definecolor{currentstroke}{rgb}{1.000000,0.000000,0.000000}%
\pgfsetstrokecolor{currentstroke}%
\pgfsetdash{}{0pt}%
\pgfpathmoveto{\pgfqpoint{1.778622in}{1.569238in}}%
\pgfpathlineto{\pgfqpoint{1.786147in}{1.102667in}}%
\pgfusepath{stroke}%
\end{pgfscope}%
\begin{pgfscope}%
\pgfpathrectangle{\pgfqpoint{0.100000in}{0.212622in}}{\pgfqpoint{3.696000in}{3.696000in}}%
\pgfusepath{clip}%
\pgfsetrectcap%
\pgfsetroundjoin%
\pgfsetlinewidth{1.505625pt}%
\definecolor{currentstroke}{rgb}{1.000000,0.000000,0.000000}%
\pgfsetstrokecolor{currentstroke}%
\pgfsetdash{}{0pt}%
\pgfpathmoveto{\pgfqpoint{1.780308in}{1.561195in}}%
\pgfpathlineto{\pgfqpoint{1.786147in}{1.102667in}}%
\pgfusepath{stroke}%
\end{pgfscope}%
\begin{pgfscope}%
\pgfpathrectangle{\pgfqpoint{0.100000in}{0.212622in}}{\pgfqpoint{3.696000in}{3.696000in}}%
\pgfusepath{clip}%
\pgfsetrectcap%
\pgfsetroundjoin%
\pgfsetlinewidth{1.505625pt}%
\definecolor{currentstroke}{rgb}{1.000000,0.000000,0.000000}%
\pgfsetstrokecolor{currentstroke}%
\pgfsetdash{}{0pt}%
\pgfpathmoveto{\pgfqpoint{1.783348in}{1.559063in}}%
\pgfpathlineto{\pgfqpoint{1.803321in}{1.097247in}}%
\pgfusepath{stroke}%
\end{pgfscope}%
\begin{pgfscope}%
\pgfpathrectangle{\pgfqpoint{0.100000in}{0.212622in}}{\pgfqpoint{3.696000in}{3.696000in}}%
\pgfusepath{clip}%
\pgfsetrectcap%
\pgfsetroundjoin%
\pgfsetlinewidth{1.505625pt}%
\definecolor{currentstroke}{rgb}{1.000000,0.000000,0.000000}%
\pgfsetstrokecolor{currentstroke}%
\pgfsetdash{}{0pt}%
\pgfpathmoveto{\pgfqpoint{1.786469in}{1.557295in}}%
\pgfpathlineto{\pgfqpoint{1.803321in}{1.097247in}}%
\pgfusepath{stroke}%
\end{pgfscope}%
\begin{pgfscope}%
\pgfpathrectangle{\pgfqpoint{0.100000in}{0.212622in}}{\pgfqpoint{3.696000in}{3.696000in}}%
\pgfusepath{clip}%
\pgfsetrectcap%
\pgfsetroundjoin%
\pgfsetlinewidth{1.505625pt}%
\definecolor{currentstroke}{rgb}{1.000000,0.000000,0.000000}%
\pgfsetstrokecolor{currentstroke}%
\pgfsetdash{}{0pt}%
\pgfpathmoveto{\pgfqpoint{1.789241in}{1.559356in}}%
\pgfpathlineto{\pgfqpoint{1.803321in}{1.097247in}}%
\pgfusepath{stroke}%
\end{pgfscope}%
\begin{pgfscope}%
\pgfpathrectangle{\pgfqpoint{0.100000in}{0.212622in}}{\pgfqpoint{3.696000in}{3.696000in}}%
\pgfusepath{clip}%
\pgfsetrectcap%
\pgfsetroundjoin%
\pgfsetlinewidth{1.505625pt}%
\definecolor{currentstroke}{rgb}{1.000000,0.000000,0.000000}%
\pgfsetstrokecolor{currentstroke}%
\pgfsetdash{}{0pt}%
\pgfpathmoveto{\pgfqpoint{1.791448in}{1.552642in}}%
\pgfpathlineto{\pgfqpoint{1.820510in}{1.091823in}}%
\pgfusepath{stroke}%
\end{pgfscope}%
\begin{pgfscope}%
\pgfpathrectangle{\pgfqpoint{0.100000in}{0.212622in}}{\pgfqpoint{3.696000in}{3.696000in}}%
\pgfusepath{clip}%
\pgfsetrectcap%
\pgfsetroundjoin%
\pgfsetlinewidth{1.505625pt}%
\definecolor{currentstroke}{rgb}{1.000000,0.000000,0.000000}%
\pgfsetstrokecolor{currentstroke}%
\pgfsetdash{}{0pt}%
\pgfpathmoveto{\pgfqpoint{1.794566in}{1.547062in}}%
\pgfpathlineto{\pgfqpoint{1.820510in}{1.091823in}}%
\pgfusepath{stroke}%
\end{pgfscope}%
\begin{pgfscope}%
\pgfpathrectangle{\pgfqpoint{0.100000in}{0.212622in}}{\pgfqpoint{3.696000in}{3.696000in}}%
\pgfusepath{clip}%
\pgfsetrectcap%
\pgfsetroundjoin%
\pgfsetlinewidth{1.505625pt}%
\definecolor{currentstroke}{rgb}{1.000000,0.000000,0.000000}%
\pgfsetstrokecolor{currentstroke}%
\pgfsetdash{}{0pt}%
\pgfpathmoveto{\pgfqpoint{1.798178in}{1.536989in}}%
\pgfpathlineto{\pgfqpoint{1.837714in}{1.086394in}}%
\pgfusepath{stroke}%
\end{pgfscope}%
\begin{pgfscope}%
\pgfpathrectangle{\pgfqpoint{0.100000in}{0.212622in}}{\pgfqpoint{3.696000in}{3.696000in}}%
\pgfusepath{clip}%
\pgfsetrectcap%
\pgfsetroundjoin%
\pgfsetlinewidth{1.505625pt}%
\definecolor{currentstroke}{rgb}{1.000000,0.000000,0.000000}%
\pgfsetstrokecolor{currentstroke}%
\pgfsetdash{}{0pt}%
\pgfpathmoveto{\pgfqpoint{1.799963in}{1.533356in}}%
\pgfpathlineto{\pgfqpoint{1.837714in}{1.086394in}}%
\pgfusepath{stroke}%
\end{pgfscope}%
\begin{pgfscope}%
\pgfpathrectangle{\pgfqpoint{0.100000in}{0.212622in}}{\pgfqpoint{3.696000in}{3.696000in}}%
\pgfusepath{clip}%
\pgfsetrectcap%
\pgfsetroundjoin%
\pgfsetlinewidth{1.505625pt}%
\definecolor{currentstroke}{rgb}{1.000000,0.000000,0.000000}%
\pgfsetstrokecolor{currentstroke}%
\pgfsetdash{}{0pt}%
\pgfpathmoveto{\pgfqpoint{1.802246in}{1.530327in}}%
\pgfpathlineto{\pgfqpoint{1.837714in}{1.086394in}}%
\pgfusepath{stroke}%
\end{pgfscope}%
\begin{pgfscope}%
\pgfpathrectangle{\pgfqpoint{0.100000in}{0.212622in}}{\pgfqpoint{3.696000in}{3.696000in}}%
\pgfusepath{clip}%
\pgfsetrectcap%
\pgfsetroundjoin%
\pgfsetlinewidth{1.505625pt}%
\definecolor{currentstroke}{rgb}{1.000000,0.000000,0.000000}%
\pgfsetstrokecolor{currentstroke}%
\pgfsetdash{}{0pt}%
\pgfpathmoveto{\pgfqpoint{1.804030in}{1.528327in}}%
\pgfpathlineto{\pgfqpoint{1.854933in}{1.080960in}}%
\pgfusepath{stroke}%
\end{pgfscope}%
\begin{pgfscope}%
\pgfpathrectangle{\pgfqpoint{0.100000in}{0.212622in}}{\pgfqpoint{3.696000in}{3.696000in}}%
\pgfusepath{clip}%
\pgfsetrectcap%
\pgfsetroundjoin%
\pgfsetlinewidth{1.505625pt}%
\definecolor{currentstroke}{rgb}{1.000000,0.000000,0.000000}%
\pgfsetstrokecolor{currentstroke}%
\pgfsetdash{}{0pt}%
\pgfpathmoveto{\pgfqpoint{1.806027in}{1.524979in}}%
\pgfpathlineto{\pgfqpoint{1.854933in}{1.080960in}}%
\pgfusepath{stroke}%
\end{pgfscope}%
\begin{pgfscope}%
\pgfpathrectangle{\pgfqpoint{0.100000in}{0.212622in}}{\pgfqpoint{3.696000in}{3.696000in}}%
\pgfusepath{clip}%
\pgfsetrectcap%
\pgfsetroundjoin%
\pgfsetlinewidth{1.505625pt}%
\definecolor{currentstroke}{rgb}{1.000000,0.000000,0.000000}%
\pgfsetstrokecolor{currentstroke}%
\pgfsetdash{}{0pt}%
\pgfpathmoveto{\pgfqpoint{1.808760in}{1.524512in}}%
\pgfpathlineto{\pgfqpoint{1.872167in}{1.075522in}}%
\pgfusepath{stroke}%
\end{pgfscope}%
\begin{pgfscope}%
\pgfpathrectangle{\pgfqpoint{0.100000in}{0.212622in}}{\pgfqpoint{3.696000in}{3.696000in}}%
\pgfusepath{clip}%
\pgfsetrectcap%
\pgfsetroundjoin%
\pgfsetlinewidth{1.505625pt}%
\definecolor{currentstroke}{rgb}{1.000000,0.000000,0.000000}%
\pgfsetstrokecolor{currentstroke}%
\pgfsetdash{}{0pt}%
\pgfpathmoveto{\pgfqpoint{1.811001in}{1.522505in}}%
\pgfpathlineto{\pgfqpoint{1.872167in}{1.075522in}}%
\pgfusepath{stroke}%
\end{pgfscope}%
\begin{pgfscope}%
\pgfpathrectangle{\pgfqpoint{0.100000in}{0.212622in}}{\pgfqpoint{3.696000in}{3.696000in}}%
\pgfusepath{clip}%
\pgfsetrectcap%
\pgfsetroundjoin%
\pgfsetlinewidth{1.505625pt}%
\definecolor{currentstroke}{rgb}{1.000000,0.000000,0.000000}%
\pgfsetstrokecolor{currentstroke}%
\pgfsetdash{}{0pt}%
\pgfpathmoveto{\pgfqpoint{1.813358in}{1.516955in}}%
\pgfpathlineto{\pgfqpoint{1.872167in}{1.075522in}}%
\pgfusepath{stroke}%
\end{pgfscope}%
\begin{pgfscope}%
\pgfpathrectangle{\pgfqpoint{0.100000in}{0.212622in}}{\pgfqpoint{3.696000in}{3.696000in}}%
\pgfusepath{clip}%
\pgfsetrectcap%
\pgfsetroundjoin%
\pgfsetlinewidth{1.505625pt}%
\definecolor{currentstroke}{rgb}{1.000000,0.000000,0.000000}%
\pgfsetstrokecolor{currentstroke}%
\pgfsetdash{}{0pt}%
\pgfpathmoveto{\pgfqpoint{1.814428in}{1.515136in}}%
\pgfpathlineto{\pgfqpoint{1.889416in}{1.070078in}}%
\pgfusepath{stroke}%
\end{pgfscope}%
\begin{pgfscope}%
\pgfpathrectangle{\pgfqpoint{0.100000in}{0.212622in}}{\pgfqpoint{3.696000in}{3.696000in}}%
\pgfusepath{clip}%
\pgfsetrectcap%
\pgfsetroundjoin%
\pgfsetlinewidth{1.505625pt}%
\definecolor{currentstroke}{rgb}{1.000000,0.000000,0.000000}%
\pgfsetstrokecolor{currentstroke}%
\pgfsetdash{}{0pt}%
\pgfpathmoveto{\pgfqpoint{1.814152in}{1.510088in}}%
\pgfpathlineto{\pgfqpoint{1.889416in}{1.070078in}}%
\pgfusepath{stroke}%
\end{pgfscope}%
\begin{pgfscope}%
\pgfpathrectangle{\pgfqpoint{0.100000in}{0.212622in}}{\pgfqpoint{3.696000in}{3.696000in}}%
\pgfusepath{clip}%
\pgfsetrectcap%
\pgfsetroundjoin%
\pgfsetlinewidth{1.505625pt}%
\definecolor{currentstroke}{rgb}{1.000000,0.000000,0.000000}%
\pgfsetstrokecolor{currentstroke}%
\pgfsetdash{}{0pt}%
\pgfpathmoveto{\pgfqpoint{1.813497in}{1.509868in}}%
\pgfpathlineto{\pgfqpoint{1.889416in}{1.070078in}}%
\pgfusepath{stroke}%
\end{pgfscope}%
\begin{pgfscope}%
\pgfpathrectangle{\pgfqpoint{0.100000in}{0.212622in}}{\pgfqpoint{3.696000in}{3.696000in}}%
\pgfusepath{clip}%
\pgfsetrectcap%
\pgfsetroundjoin%
\pgfsetlinewidth{1.505625pt}%
\definecolor{currentstroke}{rgb}{1.000000,0.000000,0.000000}%
\pgfsetstrokecolor{currentstroke}%
\pgfsetdash{}{0pt}%
\pgfpathmoveto{\pgfqpoint{1.812391in}{1.508695in}}%
\pgfpathlineto{\pgfqpoint{1.889416in}{1.070078in}}%
\pgfusepath{stroke}%
\end{pgfscope}%
\begin{pgfscope}%
\pgfpathrectangle{\pgfqpoint{0.100000in}{0.212622in}}{\pgfqpoint{3.696000in}{3.696000in}}%
\pgfusepath{clip}%
\pgfsetrectcap%
\pgfsetroundjoin%
\pgfsetlinewidth{1.505625pt}%
\definecolor{currentstroke}{rgb}{1.000000,0.000000,0.000000}%
\pgfsetstrokecolor{currentstroke}%
\pgfsetdash{}{0pt}%
\pgfpathmoveto{\pgfqpoint{1.811600in}{1.509126in}}%
\pgfpathlineto{\pgfqpoint{1.889416in}{1.070078in}}%
\pgfusepath{stroke}%
\end{pgfscope}%
\begin{pgfscope}%
\pgfpathrectangle{\pgfqpoint{0.100000in}{0.212622in}}{\pgfqpoint{3.696000in}{3.696000in}}%
\pgfusepath{clip}%
\pgfsetrectcap%
\pgfsetroundjoin%
\pgfsetlinewidth{1.505625pt}%
\definecolor{currentstroke}{rgb}{1.000000,0.000000,0.000000}%
\pgfsetstrokecolor{currentstroke}%
\pgfsetdash{}{0pt}%
\pgfpathmoveto{\pgfqpoint{1.810144in}{1.508635in}}%
\pgfpathlineto{\pgfqpoint{1.889416in}{1.070078in}}%
\pgfusepath{stroke}%
\end{pgfscope}%
\begin{pgfscope}%
\pgfpathrectangle{\pgfqpoint{0.100000in}{0.212622in}}{\pgfqpoint{3.696000in}{3.696000in}}%
\pgfusepath{clip}%
\pgfsetrectcap%
\pgfsetroundjoin%
\pgfsetlinewidth{1.505625pt}%
\definecolor{currentstroke}{rgb}{1.000000,0.000000,0.000000}%
\pgfsetstrokecolor{currentstroke}%
\pgfsetdash{}{0pt}%
\pgfpathmoveto{\pgfqpoint{1.807946in}{1.507820in}}%
\pgfpathlineto{\pgfqpoint{1.889416in}{1.070078in}}%
\pgfusepath{stroke}%
\end{pgfscope}%
\begin{pgfscope}%
\pgfpathrectangle{\pgfqpoint{0.100000in}{0.212622in}}{\pgfqpoint{3.696000in}{3.696000in}}%
\pgfusepath{clip}%
\pgfsetrectcap%
\pgfsetroundjoin%
\pgfsetlinewidth{1.505625pt}%
\definecolor{currentstroke}{rgb}{1.000000,0.000000,0.000000}%
\pgfsetstrokecolor{currentstroke}%
\pgfsetdash{}{0pt}%
\pgfpathmoveto{\pgfqpoint{1.805262in}{1.511297in}}%
\pgfpathlineto{\pgfqpoint{1.889416in}{1.070078in}}%
\pgfusepath{stroke}%
\end{pgfscope}%
\begin{pgfscope}%
\pgfpathrectangle{\pgfqpoint{0.100000in}{0.212622in}}{\pgfqpoint{3.696000in}{3.696000in}}%
\pgfusepath{clip}%
\pgfsetrectcap%
\pgfsetroundjoin%
\pgfsetlinewidth{1.505625pt}%
\definecolor{currentstroke}{rgb}{1.000000,0.000000,0.000000}%
\pgfsetstrokecolor{currentstroke}%
\pgfsetdash{}{0pt}%
\pgfpathmoveto{\pgfqpoint{1.803981in}{1.511989in}}%
\pgfpathlineto{\pgfqpoint{1.872167in}{1.075522in}}%
\pgfusepath{stroke}%
\end{pgfscope}%
\begin{pgfscope}%
\pgfpathrectangle{\pgfqpoint{0.100000in}{0.212622in}}{\pgfqpoint{3.696000in}{3.696000in}}%
\pgfusepath{clip}%
\pgfsetrectcap%
\pgfsetroundjoin%
\pgfsetlinewidth{1.505625pt}%
\definecolor{currentstroke}{rgb}{1.000000,0.000000,0.000000}%
\pgfsetstrokecolor{currentstroke}%
\pgfsetdash{}{0pt}%
\pgfpathmoveto{\pgfqpoint{1.801151in}{1.514674in}}%
\pgfpathlineto{\pgfqpoint{1.872167in}{1.075522in}}%
\pgfusepath{stroke}%
\end{pgfscope}%
\begin{pgfscope}%
\pgfpathrectangle{\pgfqpoint{0.100000in}{0.212622in}}{\pgfqpoint{3.696000in}{3.696000in}}%
\pgfusepath{clip}%
\pgfsetrectcap%
\pgfsetroundjoin%
\pgfsetlinewidth{1.505625pt}%
\definecolor{currentstroke}{rgb}{1.000000,0.000000,0.000000}%
\pgfsetstrokecolor{currentstroke}%
\pgfsetdash{}{0pt}%
\pgfpathmoveto{\pgfqpoint{1.797012in}{1.514710in}}%
\pgfpathlineto{\pgfqpoint{1.872167in}{1.075522in}}%
\pgfusepath{stroke}%
\end{pgfscope}%
\begin{pgfscope}%
\pgfpathrectangle{\pgfqpoint{0.100000in}{0.212622in}}{\pgfqpoint{3.696000in}{3.696000in}}%
\pgfusepath{clip}%
\pgfsetrectcap%
\pgfsetroundjoin%
\pgfsetlinewidth{1.505625pt}%
\definecolor{currentstroke}{rgb}{1.000000,0.000000,0.000000}%
\pgfsetstrokecolor{currentstroke}%
\pgfsetdash{}{0pt}%
\pgfpathmoveto{\pgfqpoint{1.792189in}{1.522960in}}%
\pgfpathlineto{\pgfqpoint{1.872167in}{1.075522in}}%
\pgfusepath{stroke}%
\end{pgfscope}%
\begin{pgfscope}%
\pgfpathrectangle{\pgfqpoint{0.100000in}{0.212622in}}{\pgfqpoint{3.696000in}{3.696000in}}%
\pgfusepath{clip}%
\pgfsetrectcap%
\pgfsetroundjoin%
\pgfsetlinewidth{1.505625pt}%
\definecolor{currentstroke}{rgb}{1.000000,0.000000,0.000000}%
\pgfsetstrokecolor{currentstroke}%
\pgfsetdash{}{0pt}%
\pgfpathmoveto{\pgfqpoint{1.787211in}{1.527364in}}%
\pgfpathlineto{\pgfqpoint{1.854933in}{1.080960in}}%
\pgfusepath{stroke}%
\end{pgfscope}%
\begin{pgfscope}%
\pgfpathrectangle{\pgfqpoint{0.100000in}{0.212622in}}{\pgfqpoint{3.696000in}{3.696000in}}%
\pgfusepath{clip}%
\pgfsetrectcap%
\pgfsetroundjoin%
\pgfsetlinewidth{1.505625pt}%
\definecolor{currentstroke}{rgb}{1.000000,0.000000,0.000000}%
\pgfsetstrokecolor{currentstroke}%
\pgfsetdash{}{0pt}%
\pgfpathmoveto{\pgfqpoint{1.781662in}{1.533422in}}%
\pgfpathlineto{\pgfqpoint{1.854933in}{1.080960in}}%
\pgfusepath{stroke}%
\end{pgfscope}%
\begin{pgfscope}%
\pgfpathrectangle{\pgfqpoint{0.100000in}{0.212622in}}{\pgfqpoint{3.696000in}{3.696000in}}%
\pgfusepath{clip}%
\pgfsetrectcap%
\pgfsetroundjoin%
\pgfsetlinewidth{1.505625pt}%
\definecolor{currentstroke}{rgb}{1.000000,0.000000,0.000000}%
\pgfsetstrokecolor{currentstroke}%
\pgfsetdash{}{0pt}%
\pgfpathmoveto{\pgfqpoint{1.778681in}{1.536207in}}%
\pgfpathlineto{\pgfqpoint{1.854933in}{1.080960in}}%
\pgfusepath{stroke}%
\end{pgfscope}%
\begin{pgfscope}%
\pgfpathrectangle{\pgfqpoint{0.100000in}{0.212622in}}{\pgfqpoint{3.696000in}{3.696000in}}%
\pgfusepath{clip}%
\pgfsetrectcap%
\pgfsetroundjoin%
\pgfsetlinewidth{1.505625pt}%
\definecolor{currentstroke}{rgb}{1.000000,0.000000,0.000000}%
\pgfsetstrokecolor{currentstroke}%
\pgfsetdash{}{0pt}%
\pgfpathmoveto{\pgfqpoint{1.777033in}{1.537877in}}%
\pgfpathlineto{\pgfqpoint{1.854933in}{1.080960in}}%
\pgfusepath{stroke}%
\end{pgfscope}%
\begin{pgfscope}%
\pgfpathrectangle{\pgfqpoint{0.100000in}{0.212622in}}{\pgfqpoint{3.696000in}{3.696000in}}%
\pgfusepath{clip}%
\pgfsetrectcap%
\pgfsetroundjoin%
\pgfsetlinewidth{1.505625pt}%
\definecolor{currentstroke}{rgb}{1.000000,0.000000,0.000000}%
\pgfsetstrokecolor{currentstroke}%
\pgfsetdash{}{0pt}%
\pgfpathmoveto{\pgfqpoint{1.776141in}{1.538702in}}%
\pgfpathlineto{\pgfqpoint{1.854933in}{1.080960in}}%
\pgfusepath{stroke}%
\end{pgfscope}%
\begin{pgfscope}%
\pgfpathrectangle{\pgfqpoint{0.100000in}{0.212622in}}{\pgfqpoint{3.696000in}{3.696000in}}%
\pgfusepath{clip}%
\pgfsetrectcap%
\pgfsetroundjoin%
\pgfsetlinewidth{1.505625pt}%
\definecolor{currentstroke}{rgb}{1.000000,0.000000,0.000000}%
\pgfsetstrokecolor{currentstroke}%
\pgfsetdash{}{0pt}%
\pgfpathmoveto{\pgfqpoint{1.775643in}{1.539093in}}%
\pgfpathlineto{\pgfqpoint{1.854933in}{1.080960in}}%
\pgfusepath{stroke}%
\end{pgfscope}%
\begin{pgfscope}%
\pgfpathrectangle{\pgfqpoint{0.100000in}{0.212622in}}{\pgfqpoint{3.696000in}{3.696000in}}%
\pgfusepath{clip}%
\pgfsetrectcap%
\pgfsetroundjoin%
\pgfsetlinewidth{1.505625pt}%
\definecolor{currentstroke}{rgb}{1.000000,0.000000,0.000000}%
\pgfsetstrokecolor{currentstroke}%
\pgfsetdash{}{0pt}%
\pgfpathmoveto{\pgfqpoint{1.775359in}{1.539308in}}%
\pgfpathlineto{\pgfqpoint{1.854933in}{1.080960in}}%
\pgfusepath{stroke}%
\end{pgfscope}%
\begin{pgfscope}%
\pgfpathrectangle{\pgfqpoint{0.100000in}{0.212622in}}{\pgfqpoint{3.696000in}{3.696000in}}%
\pgfusepath{clip}%
\pgfsetrectcap%
\pgfsetroundjoin%
\pgfsetlinewidth{1.505625pt}%
\definecolor{currentstroke}{rgb}{1.000000,0.000000,0.000000}%
\pgfsetstrokecolor{currentstroke}%
\pgfsetdash{}{0pt}%
\pgfpathmoveto{\pgfqpoint{1.775201in}{1.539439in}}%
\pgfpathlineto{\pgfqpoint{1.854933in}{1.080960in}}%
\pgfusepath{stroke}%
\end{pgfscope}%
\begin{pgfscope}%
\pgfpathrectangle{\pgfqpoint{0.100000in}{0.212622in}}{\pgfqpoint{3.696000in}{3.696000in}}%
\pgfusepath{clip}%
\pgfsetrectcap%
\pgfsetroundjoin%
\pgfsetlinewidth{1.505625pt}%
\definecolor{currentstroke}{rgb}{1.000000,0.000000,0.000000}%
\pgfsetstrokecolor{currentstroke}%
\pgfsetdash{}{0pt}%
\pgfpathmoveto{\pgfqpoint{1.774434in}{1.539926in}}%
\pgfpathlineto{\pgfqpoint{1.854933in}{1.080960in}}%
\pgfusepath{stroke}%
\end{pgfscope}%
\begin{pgfscope}%
\pgfpathrectangle{\pgfqpoint{0.100000in}{0.212622in}}{\pgfqpoint{3.696000in}{3.696000in}}%
\pgfusepath{clip}%
\pgfsetrectcap%
\pgfsetroundjoin%
\pgfsetlinewidth{1.505625pt}%
\definecolor{currentstroke}{rgb}{1.000000,0.000000,0.000000}%
\pgfsetstrokecolor{currentstroke}%
\pgfsetdash{}{0pt}%
\pgfpathmoveto{\pgfqpoint{1.772250in}{1.541260in}}%
\pgfpathlineto{\pgfqpoint{1.854933in}{1.080960in}}%
\pgfusepath{stroke}%
\end{pgfscope}%
\begin{pgfscope}%
\pgfpathrectangle{\pgfqpoint{0.100000in}{0.212622in}}{\pgfqpoint{3.696000in}{3.696000in}}%
\pgfusepath{clip}%
\pgfsetrectcap%
\pgfsetroundjoin%
\pgfsetlinewidth{1.505625pt}%
\definecolor{currentstroke}{rgb}{1.000000,0.000000,0.000000}%
\pgfsetstrokecolor{currentstroke}%
\pgfsetdash{}{0pt}%
\pgfpathmoveto{\pgfqpoint{1.771038in}{1.542402in}}%
\pgfpathlineto{\pgfqpoint{1.837714in}{1.086394in}}%
\pgfusepath{stroke}%
\end{pgfscope}%
\begin{pgfscope}%
\pgfpathrectangle{\pgfqpoint{0.100000in}{0.212622in}}{\pgfqpoint{3.696000in}{3.696000in}}%
\pgfusepath{clip}%
\pgfsetrectcap%
\pgfsetroundjoin%
\pgfsetlinewidth{1.505625pt}%
\definecolor{currentstroke}{rgb}{1.000000,0.000000,0.000000}%
\pgfsetstrokecolor{currentstroke}%
\pgfsetdash{}{0pt}%
\pgfpathmoveto{\pgfqpoint{1.768422in}{1.545988in}}%
\pgfpathlineto{\pgfqpoint{1.837714in}{1.086394in}}%
\pgfusepath{stroke}%
\end{pgfscope}%
\begin{pgfscope}%
\pgfpathrectangle{\pgfqpoint{0.100000in}{0.212622in}}{\pgfqpoint{3.696000in}{3.696000in}}%
\pgfusepath{clip}%
\pgfsetrectcap%
\pgfsetroundjoin%
\pgfsetlinewidth{1.505625pt}%
\definecolor{currentstroke}{rgb}{1.000000,0.000000,0.000000}%
\pgfsetstrokecolor{currentstroke}%
\pgfsetdash{}{0pt}%
\pgfpathmoveto{\pgfqpoint{1.765148in}{1.546893in}}%
\pgfpathlineto{\pgfqpoint{1.837714in}{1.086394in}}%
\pgfusepath{stroke}%
\end{pgfscope}%
\begin{pgfscope}%
\pgfpathrectangle{\pgfqpoint{0.100000in}{0.212622in}}{\pgfqpoint{3.696000in}{3.696000in}}%
\pgfusepath{clip}%
\pgfsetrectcap%
\pgfsetroundjoin%
\pgfsetlinewidth{1.505625pt}%
\definecolor{currentstroke}{rgb}{1.000000,0.000000,0.000000}%
\pgfsetstrokecolor{currentstroke}%
\pgfsetdash{}{0pt}%
\pgfpathmoveto{\pgfqpoint{1.759516in}{1.551379in}}%
\pgfpathlineto{\pgfqpoint{1.837714in}{1.086394in}}%
\pgfusepath{stroke}%
\end{pgfscope}%
\begin{pgfscope}%
\pgfpathrectangle{\pgfqpoint{0.100000in}{0.212622in}}{\pgfqpoint{3.696000in}{3.696000in}}%
\pgfusepath{clip}%
\pgfsetrectcap%
\pgfsetroundjoin%
\pgfsetlinewidth{1.505625pt}%
\definecolor{currentstroke}{rgb}{1.000000,0.000000,0.000000}%
\pgfsetstrokecolor{currentstroke}%
\pgfsetdash{}{0pt}%
\pgfpathmoveto{\pgfqpoint{1.756525in}{1.553860in}}%
\pgfpathlineto{\pgfqpoint{1.837714in}{1.086394in}}%
\pgfusepath{stroke}%
\end{pgfscope}%
\begin{pgfscope}%
\pgfpathrectangle{\pgfqpoint{0.100000in}{0.212622in}}{\pgfqpoint{3.696000in}{3.696000in}}%
\pgfusepath{clip}%
\pgfsetrectcap%
\pgfsetroundjoin%
\pgfsetlinewidth{1.505625pt}%
\definecolor{currentstroke}{rgb}{1.000000,0.000000,0.000000}%
\pgfsetstrokecolor{currentstroke}%
\pgfsetdash{}{0pt}%
\pgfpathmoveto{\pgfqpoint{1.752546in}{1.557077in}}%
\pgfpathlineto{\pgfqpoint{1.820510in}{1.091823in}}%
\pgfusepath{stroke}%
\end{pgfscope}%
\begin{pgfscope}%
\pgfpathrectangle{\pgfqpoint{0.100000in}{0.212622in}}{\pgfqpoint{3.696000in}{3.696000in}}%
\pgfusepath{clip}%
\pgfsetrectcap%
\pgfsetroundjoin%
\pgfsetlinewidth{1.505625pt}%
\definecolor{currentstroke}{rgb}{1.000000,0.000000,0.000000}%
\pgfsetstrokecolor{currentstroke}%
\pgfsetdash{}{0pt}%
\pgfpathmoveto{\pgfqpoint{1.748059in}{1.560134in}}%
\pgfpathlineto{\pgfqpoint{1.820510in}{1.091823in}}%
\pgfusepath{stroke}%
\end{pgfscope}%
\begin{pgfscope}%
\pgfpathrectangle{\pgfqpoint{0.100000in}{0.212622in}}{\pgfqpoint{3.696000in}{3.696000in}}%
\pgfusepath{clip}%
\pgfsetrectcap%
\pgfsetroundjoin%
\pgfsetlinewidth{1.505625pt}%
\definecolor{currentstroke}{rgb}{1.000000,0.000000,0.000000}%
\pgfsetstrokecolor{currentstroke}%
\pgfsetdash{}{0pt}%
\pgfpathmoveto{\pgfqpoint{1.741462in}{1.564633in}}%
\pgfpathlineto{\pgfqpoint{1.820510in}{1.091823in}}%
\pgfusepath{stroke}%
\end{pgfscope}%
\begin{pgfscope}%
\pgfpathrectangle{\pgfqpoint{0.100000in}{0.212622in}}{\pgfqpoint{3.696000in}{3.696000in}}%
\pgfusepath{clip}%
\pgfsetrectcap%
\pgfsetroundjoin%
\pgfsetlinewidth{1.505625pt}%
\definecolor{currentstroke}{rgb}{1.000000,0.000000,0.000000}%
\pgfsetstrokecolor{currentstroke}%
\pgfsetdash{}{0pt}%
\pgfpathmoveto{\pgfqpoint{1.734259in}{1.575387in}}%
\pgfpathlineto{\pgfqpoint{1.803321in}{1.097247in}}%
\pgfusepath{stroke}%
\end{pgfscope}%
\begin{pgfscope}%
\pgfpathrectangle{\pgfqpoint{0.100000in}{0.212622in}}{\pgfqpoint{3.696000in}{3.696000in}}%
\pgfusepath{clip}%
\pgfsetrectcap%
\pgfsetroundjoin%
\pgfsetlinewidth{1.505625pt}%
\definecolor{currentstroke}{rgb}{1.000000,0.000000,0.000000}%
\pgfsetstrokecolor{currentstroke}%
\pgfsetdash{}{0pt}%
\pgfpathmoveto{\pgfqpoint{1.725817in}{1.579074in}}%
\pgfpathlineto{\pgfqpoint{1.803321in}{1.097247in}}%
\pgfusepath{stroke}%
\end{pgfscope}%
\begin{pgfscope}%
\pgfpathrectangle{\pgfqpoint{0.100000in}{0.212622in}}{\pgfqpoint{3.696000in}{3.696000in}}%
\pgfusepath{clip}%
\pgfsetrectcap%
\pgfsetroundjoin%
\pgfsetlinewidth{1.505625pt}%
\definecolor{currentstroke}{rgb}{1.000000,0.000000,0.000000}%
\pgfsetstrokecolor{currentstroke}%
\pgfsetdash{}{0pt}%
\pgfpathmoveto{\pgfqpoint{1.716755in}{1.582393in}}%
\pgfpathlineto{\pgfqpoint{1.786147in}{1.102667in}}%
\pgfusepath{stroke}%
\end{pgfscope}%
\begin{pgfscope}%
\pgfpathrectangle{\pgfqpoint{0.100000in}{0.212622in}}{\pgfqpoint{3.696000in}{3.696000in}}%
\pgfusepath{clip}%
\pgfsetrectcap%
\pgfsetroundjoin%
\pgfsetlinewidth{1.505625pt}%
\definecolor{currentstroke}{rgb}{1.000000,0.000000,0.000000}%
\pgfsetstrokecolor{currentstroke}%
\pgfsetdash{}{0pt}%
\pgfpathmoveto{\pgfqpoint{1.705097in}{1.595195in}}%
\pgfpathlineto{\pgfqpoint{1.786147in}{1.102667in}}%
\pgfusepath{stroke}%
\end{pgfscope}%
\begin{pgfscope}%
\pgfpathrectangle{\pgfqpoint{0.100000in}{0.212622in}}{\pgfqpoint{3.696000in}{3.696000in}}%
\pgfusepath{clip}%
\pgfsetrectcap%
\pgfsetroundjoin%
\pgfsetlinewidth{1.505625pt}%
\definecolor{currentstroke}{rgb}{1.000000,0.000000,0.000000}%
\pgfsetstrokecolor{currentstroke}%
\pgfsetdash{}{0pt}%
\pgfpathmoveto{\pgfqpoint{1.694617in}{1.604181in}}%
\pgfpathlineto{\pgfqpoint{1.768988in}{1.108082in}}%
\pgfusepath{stroke}%
\end{pgfscope}%
\begin{pgfscope}%
\pgfpathrectangle{\pgfqpoint{0.100000in}{0.212622in}}{\pgfqpoint{3.696000in}{3.696000in}}%
\pgfusepath{clip}%
\pgfsetrectcap%
\pgfsetroundjoin%
\pgfsetlinewidth{1.505625pt}%
\definecolor{currentstroke}{rgb}{1.000000,0.000000,0.000000}%
\pgfsetstrokecolor{currentstroke}%
\pgfsetdash{}{0pt}%
\pgfpathmoveto{\pgfqpoint{1.682923in}{1.618004in}}%
\pgfpathlineto{\pgfqpoint{1.751843in}{1.113493in}}%
\pgfusepath{stroke}%
\end{pgfscope}%
\begin{pgfscope}%
\pgfpathrectangle{\pgfqpoint{0.100000in}{0.212622in}}{\pgfqpoint{3.696000in}{3.696000in}}%
\pgfusepath{clip}%
\pgfsetrectcap%
\pgfsetroundjoin%
\pgfsetlinewidth{1.505625pt}%
\definecolor{currentstroke}{rgb}{1.000000,0.000000,0.000000}%
\pgfsetstrokecolor{currentstroke}%
\pgfsetdash{}{0pt}%
\pgfpathmoveto{\pgfqpoint{1.675750in}{1.620308in}}%
\pgfpathlineto{\pgfqpoint{1.751843in}{1.113493in}}%
\pgfusepath{stroke}%
\end{pgfscope}%
\begin{pgfscope}%
\pgfpathrectangle{\pgfqpoint{0.100000in}{0.212622in}}{\pgfqpoint{3.696000in}{3.696000in}}%
\pgfusepath{clip}%
\pgfsetrectcap%
\pgfsetroundjoin%
\pgfsetlinewidth{1.505625pt}%
\definecolor{currentstroke}{rgb}{1.000000,0.000000,0.000000}%
\pgfsetstrokecolor{currentstroke}%
\pgfsetdash{}{0pt}%
\pgfpathmoveto{\pgfqpoint{1.667857in}{1.626325in}}%
\pgfpathlineto{\pgfqpoint{1.734713in}{1.118898in}}%
\pgfusepath{stroke}%
\end{pgfscope}%
\begin{pgfscope}%
\pgfpathrectangle{\pgfqpoint{0.100000in}{0.212622in}}{\pgfqpoint{3.696000in}{3.696000in}}%
\pgfusepath{clip}%
\pgfsetrectcap%
\pgfsetroundjoin%
\pgfsetlinewidth{1.505625pt}%
\definecolor{currentstroke}{rgb}{1.000000,0.000000,0.000000}%
\pgfsetstrokecolor{currentstroke}%
\pgfsetdash{}{0pt}%
\pgfpathmoveto{\pgfqpoint{1.658227in}{1.631491in}}%
\pgfpathlineto{\pgfqpoint{1.734713in}{1.118898in}}%
\pgfusepath{stroke}%
\end{pgfscope}%
\begin{pgfscope}%
\pgfpathrectangle{\pgfqpoint{0.100000in}{0.212622in}}{\pgfqpoint{3.696000in}{3.696000in}}%
\pgfusepath{clip}%
\pgfsetrectcap%
\pgfsetroundjoin%
\pgfsetlinewidth{1.505625pt}%
\definecolor{currentstroke}{rgb}{1.000000,0.000000,0.000000}%
\pgfsetstrokecolor{currentstroke}%
\pgfsetdash{}{0pt}%
\pgfpathmoveto{\pgfqpoint{1.647825in}{1.640308in}}%
\pgfpathlineto{\pgfqpoint{1.717598in}{1.124299in}}%
\pgfusepath{stroke}%
\end{pgfscope}%
\begin{pgfscope}%
\pgfpathrectangle{\pgfqpoint{0.100000in}{0.212622in}}{\pgfqpoint{3.696000in}{3.696000in}}%
\pgfusepath{clip}%
\pgfsetrectcap%
\pgfsetroundjoin%
\pgfsetlinewidth{1.505625pt}%
\definecolor{currentstroke}{rgb}{1.000000,0.000000,0.000000}%
\pgfsetstrokecolor{currentstroke}%
\pgfsetdash{}{0pt}%
\pgfpathmoveto{\pgfqpoint{1.636824in}{1.652345in}}%
\pgfpathlineto{\pgfqpoint{1.700498in}{1.129696in}}%
\pgfusepath{stroke}%
\end{pgfscope}%
\begin{pgfscope}%
\pgfpathrectangle{\pgfqpoint{0.100000in}{0.212622in}}{\pgfqpoint{3.696000in}{3.696000in}}%
\pgfusepath{clip}%
\pgfsetrectcap%
\pgfsetroundjoin%
\pgfsetlinewidth{1.505625pt}%
\definecolor{currentstroke}{rgb}{1.000000,0.000000,0.000000}%
\pgfsetstrokecolor{currentstroke}%
\pgfsetdash{}{0pt}%
\pgfpathmoveto{\pgfqpoint{1.624687in}{1.662650in}}%
\pgfpathlineto{\pgfqpoint{1.700498in}{1.129696in}}%
\pgfusepath{stroke}%
\end{pgfscope}%
\begin{pgfscope}%
\pgfpathrectangle{\pgfqpoint{0.100000in}{0.212622in}}{\pgfqpoint{3.696000in}{3.696000in}}%
\pgfusepath{clip}%
\pgfsetrectcap%
\pgfsetroundjoin%
\pgfsetlinewidth{1.505625pt}%
\definecolor{currentstroke}{rgb}{1.000000,0.000000,0.000000}%
\pgfsetstrokecolor{currentstroke}%
\pgfsetdash{}{0pt}%
\pgfpathmoveto{\pgfqpoint{1.610224in}{1.681428in}}%
\pgfpathlineto{\pgfqpoint{1.683412in}{1.135087in}}%
\pgfusepath{stroke}%
\end{pgfscope}%
\begin{pgfscope}%
\pgfpathrectangle{\pgfqpoint{0.100000in}{0.212622in}}{\pgfqpoint{3.696000in}{3.696000in}}%
\pgfusepath{clip}%
\pgfsetrectcap%
\pgfsetroundjoin%
\pgfsetlinewidth{1.505625pt}%
\definecolor{currentstroke}{rgb}{1.000000,0.000000,0.000000}%
\pgfsetstrokecolor{currentstroke}%
\pgfsetdash{}{0pt}%
\pgfpathmoveto{\pgfqpoint{1.602312in}{1.686366in}}%
\pgfpathlineto{\pgfqpoint{1.666341in}{1.140474in}}%
\pgfusepath{stroke}%
\end{pgfscope}%
\begin{pgfscope}%
\pgfpathrectangle{\pgfqpoint{0.100000in}{0.212622in}}{\pgfqpoint{3.696000in}{3.696000in}}%
\pgfusepath{clip}%
\pgfsetrectcap%
\pgfsetroundjoin%
\pgfsetlinewidth{1.505625pt}%
\definecolor{currentstroke}{rgb}{1.000000,0.000000,0.000000}%
\pgfsetstrokecolor{currentstroke}%
\pgfsetdash{}{0pt}%
\pgfpathmoveto{\pgfqpoint{1.592658in}{1.697194in}}%
\pgfpathlineto{\pgfqpoint{1.666341in}{1.140474in}}%
\pgfusepath{stroke}%
\end{pgfscope}%
\begin{pgfscope}%
\pgfpathrectangle{\pgfqpoint{0.100000in}{0.212622in}}{\pgfqpoint{3.696000in}{3.696000in}}%
\pgfusepath{clip}%
\pgfsetrectcap%
\pgfsetroundjoin%
\pgfsetlinewidth{1.505625pt}%
\definecolor{currentstroke}{rgb}{1.000000,0.000000,0.000000}%
\pgfsetstrokecolor{currentstroke}%
\pgfsetdash{}{0pt}%
\pgfpathmoveto{\pgfqpoint{1.586971in}{1.699264in}}%
\pgfpathlineto{\pgfqpoint{1.666341in}{1.140474in}}%
\pgfusepath{stroke}%
\end{pgfscope}%
\begin{pgfscope}%
\pgfpathrectangle{\pgfqpoint{0.100000in}{0.212622in}}{\pgfqpoint{3.696000in}{3.696000in}}%
\pgfusepath{clip}%
\pgfsetrectcap%
\pgfsetroundjoin%
\pgfsetlinewidth{1.505625pt}%
\definecolor{currentstroke}{rgb}{1.000000,0.000000,0.000000}%
\pgfsetstrokecolor{currentstroke}%
\pgfsetdash{}{0pt}%
\pgfpathmoveto{\pgfqpoint{1.578811in}{1.705334in}}%
\pgfpathlineto{\pgfqpoint{1.649285in}{1.145857in}}%
\pgfusepath{stroke}%
\end{pgfscope}%
\begin{pgfscope}%
\pgfpathrectangle{\pgfqpoint{0.100000in}{0.212622in}}{\pgfqpoint{3.696000in}{3.696000in}}%
\pgfusepath{clip}%
\pgfsetrectcap%
\pgfsetroundjoin%
\pgfsetlinewidth{1.505625pt}%
\definecolor{currentstroke}{rgb}{1.000000,0.000000,0.000000}%
\pgfsetstrokecolor{currentstroke}%
\pgfsetdash{}{0pt}%
\pgfpathmoveto{\pgfqpoint{1.574436in}{1.710220in}}%
\pgfpathlineto{\pgfqpoint{1.649285in}{1.145857in}}%
\pgfusepath{stroke}%
\end{pgfscope}%
\begin{pgfscope}%
\pgfpathrectangle{\pgfqpoint{0.100000in}{0.212622in}}{\pgfqpoint{3.696000in}{3.696000in}}%
\pgfusepath{clip}%
\pgfsetrectcap%
\pgfsetroundjoin%
\pgfsetlinewidth{1.505625pt}%
\definecolor{currentstroke}{rgb}{1.000000,0.000000,0.000000}%
\pgfsetstrokecolor{currentstroke}%
\pgfsetdash{}{0pt}%
\pgfpathmoveto{\pgfqpoint{1.568543in}{1.713944in}}%
\pgfpathlineto{\pgfqpoint{1.632244in}{1.151235in}}%
\pgfusepath{stroke}%
\end{pgfscope}%
\begin{pgfscope}%
\pgfpathrectangle{\pgfqpoint{0.100000in}{0.212622in}}{\pgfqpoint{3.696000in}{3.696000in}}%
\pgfusepath{clip}%
\pgfsetrectcap%
\pgfsetroundjoin%
\pgfsetlinewidth{1.505625pt}%
\definecolor{currentstroke}{rgb}{1.000000,0.000000,0.000000}%
\pgfsetstrokecolor{currentstroke}%
\pgfsetdash{}{0pt}%
\pgfpathmoveto{\pgfqpoint{1.561585in}{1.720654in}}%
\pgfpathlineto{\pgfqpoint{1.632244in}{1.151235in}}%
\pgfusepath{stroke}%
\end{pgfscope}%
\begin{pgfscope}%
\pgfpathrectangle{\pgfqpoint{0.100000in}{0.212622in}}{\pgfqpoint{3.696000in}{3.696000in}}%
\pgfusepath{clip}%
\pgfsetrectcap%
\pgfsetroundjoin%
\pgfsetlinewidth{1.505625pt}%
\definecolor{currentstroke}{rgb}{1.000000,0.000000,0.000000}%
\pgfsetstrokecolor{currentstroke}%
\pgfsetdash{}{0pt}%
\pgfpathmoveto{\pgfqpoint{1.553242in}{1.728013in}}%
\pgfpathlineto{\pgfqpoint{1.632244in}{1.151235in}}%
\pgfusepath{stroke}%
\end{pgfscope}%
\begin{pgfscope}%
\pgfpathrectangle{\pgfqpoint{0.100000in}{0.212622in}}{\pgfqpoint{3.696000in}{3.696000in}}%
\pgfusepath{clip}%
\pgfsetrectcap%
\pgfsetroundjoin%
\pgfsetlinewidth{1.505625pt}%
\definecolor{currentstroke}{rgb}{1.000000,0.000000,0.000000}%
\pgfsetstrokecolor{currentstroke}%
\pgfsetdash{}{0pt}%
\pgfpathmoveto{\pgfqpoint{1.543918in}{1.736580in}}%
\pgfpathlineto{\pgfqpoint{1.615217in}{1.156608in}}%
\pgfusepath{stroke}%
\end{pgfscope}%
\begin{pgfscope}%
\pgfpathrectangle{\pgfqpoint{0.100000in}{0.212622in}}{\pgfqpoint{3.696000in}{3.696000in}}%
\pgfusepath{clip}%
\pgfsetrectcap%
\pgfsetroundjoin%
\pgfsetlinewidth{1.505625pt}%
\definecolor{currentstroke}{rgb}{1.000000,0.000000,0.000000}%
\pgfsetstrokecolor{currentstroke}%
\pgfsetdash{}{0pt}%
\pgfpathmoveto{\pgfqpoint{1.532746in}{1.746435in}}%
\pgfpathlineto{\pgfqpoint{1.598205in}{1.161977in}}%
\pgfusepath{stroke}%
\end{pgfscope}%
\begin{pgfscope}%
\pgfpathrectangle{\pgfqpoint{0.100000in}{0.212622in}}{\pgfqpoint{3.696000in}{3.696000in}}%
\pgfusepath{clip}%
\pgfsetrectcap%
\pgfsetroundjoin%
\pgfsetlinewidth{1.505625pt}%
\definecolor{currentstroke}{rgb}{1.000000,0.000000,0.000000}%
\pgfsetstrokecolor{currentstroke}%
\pgfsetdash{}{0pt}%
\pgfpathmoveto{\pgfqpoint{1.520677in}{1.754156in}}%
\pgfpathlineto{\pgfqpoint{1.598205in}{1.161977in}}%
\pgfusepath{stroke}%
\end{pgfscope}%
\begin{pgfscope}%
\pgfpathrectangle{\pgfqpoint{0.100000in}{0.212622in}}{\pgfqpoint{3.696000in}{3.696000in}}%
\pgfusepath{clip}%
\pgfsetrectcap%
\pgfsetroundjoin%
\pgfsetlinewidth{1.505625pt}%
\definecolor{currentstroke}{rgb}{1.000000,0.000000,0.000000}%
\pgfsetstrokecolor{currentstroke}%
\pgfsetdash{}{0pt}%
\pgfpathmoveto{\pgfqpoint{1.506892in}{1.763676in}}%
\pgfpathlineto{\pgfqpoint{1.581207in}{1.167340in}}%
\pgfusepath{stroke}%
\end{pgfscope}%
\begin{pgfscope}%
\pgfpathrectangle{\pgfqpoint{0.100000in}{0.212622in}}{\pgfqpoint{3.696000in}{3.696000in}}%
\pgfusepath{clip}%
\pgfsetrectcap%
\pgfsetroundjoin%
\pgfsetlinewidth{1.505625pt}%
\definecolor{currentstroke}{rgb}{1.000000,0.000000,0.000000}%
\pgfsetstrokecolor{currentstroke}%
\pgfsetdash{}{0pt}%
\pgfpathmoveto{\pgfqpoint{1.490805in}{1.786709in}}%
\pgfpathlineto{\pgfqpoint{1.564224in}{1.172700in}}%
\pgfusepath{stroke}%
\end{pgfscope}%
\begin{pgfscope}%
\pgfpathrectangle{\pgfqpoint{0.100000in}{0.212622in}}{\pgfqpoint{3.696000in}{3.696000in}}%
\pgfusepath{clip}%
\pgfsetrectcap%
\pgfsetroundjoin%
\pgfsetlinewidth{1.505625pt}%
\definecolor{currentstroke}{rgb}{1.000000,0.000000,0.000000}%
\pgfsetstrokecolor{currentstroke}%
\pgfsetdash{}{0pt}%
\pgfpathmoveto{\pgfqpoint{1.471950in}{1.798405in}}%
\pgfpathlineto{\pgfqpoint{1.547256in}{1.178055in}}%
\pgfusepath{stroke}%
\end{pgfscope}%
\begin{pgfscope}%
\pgfpathrectangle{\pgfqpoint{0.100000in}{0.212622in}}{\pgfqpoint{3.696000in}{3.696000in}}%
\pgfusepath{clip}%
\pgfsetrectcap%
\pgfsetroundjoin%
\pgfsetlinewidth{1.505625pt}%
\definecolor{currentstroke}{rgb}{1.000000,0.000000,0.000000}%
\pgfsetstrokecolor{currentstroke}%
\pgfsetdash{}{0pt}%
\pgfpathmoveto{\pgfqpoint{1.452406in}{1.809576in}}%
\pgfpathlineto{\pgfqpoint{1.530302in}{1.183405in}}%
\pgfusepath{stroke}%
\end{pgfscope}%
\begin{pgfscope}%
\pgfpathrectangle{\pgfqpoint{0.100000in}{0.212622in}}{\pgfqpoint{3.696000in}{3.696000in}}%
\pgfusepath{clip}%
\pgfsetrectcap%
\pgfsetroundjoin%
\pgfsetlinewidth{1.505625pt}%
\definecolor{currentstroke}{rgb}{1.000000,0.000000,0.000000}%
\pgfsetstrokecolor{currentstroke}%
\pgfsetdash{}{0pt}%
\pgfpathmoveto{\pgfqpoint{1.430500in}{1.830262in}}%
\pgfpathlineto{\pgfqpoint{1.496438in}{1.194091in}}%
\pgfusepath{stroke}%
\end{pgfscope}%
\begin{pgfscope}%
\pgfpathrectangle{\pgfqpoint{0.100000in}{0.212622in}}{\pgfqpoint{3.696000in}{3.696000in}}%
\pgfusepath{clip}%
\pgfsetrectcap%
\pgfsetroundjoin%
\pgfsetlinewidth{1.505625pt}%
\definecolor{currentstroke}{rgb}{1.000000,0.000000,0.000000}%
\pgfsetstrokecolor{currentstroke}%
\pgfsetdash{}{0pt}%
\pgfpathmoveto{\pgfqpoint{1.407433in}{1.857772in}}%
\pgfpathlineto{\pgfqpoint{1.479528in}{1.199428in}}%
\pgfusepath{stroke}%
\end{pgfscope}%
\begin{pgfscope}%
\pgfpathrectangle{\pgfqpoint{0.100000in}{0.212622in}}{\pgfqpoint{3.696000in}{3.696000in}}%
\pgfusepath{clip}%
\pgfsetrectcap%
\pgfsetroundjoin%
\pgfsetlinewidth{1.505625pt}%
\definecolor{currentstroke}{rgb}{1.000000,0.000000,0.000000}%
\pgfsetstrokecolor{currentstroke}%
\pgfsetdash{}{0pt}%
\pgfpathmoveto{\pgfqpoint{1.382100in}{1.873278in}}%
\pgfpathlineto{\pgfqpoint{1.462632in}{1.204760in}}%
\pgfusepath{stroke}%
\end{pgfscope}%
\begin{pgfscope}%
\pgfpathrectangle{\pgfqpoint{0.100000in}{0.212622in}}{\pgfqpoint{3.696000in}{3.696000in}}%
\pgfusepath{clip}%
\pgfsetrectcap%
\pgfsetroundjoin%
\pgfsetlinewidth{1.505625pt}%
\definecolor{currentstroke}{rgb}{1.000000,0.000000,0.000000}%
\pgfsetstrokecolor{currentstroke}%
\pgfsetdash{}{0pt}%
\pgfpathmoveto{\pgfqpoint{1.355910in}{1.892734in}}%
\pgfpathlineto{\pgfqpoint{1.428884in}{1.215410in}}%
\pgfusepath{stroke}%
\end{pgfscope}%
\begin{pgfscope}%
\pgfpathrectangle{\pgfqpoint{0.100000in}{0.212622in}}{\pgfqpoint{3.696000in}{3.696000in}}%
\pgfusepath{clip}%
\pgfsetrectcap%
\pgfsetroundjoin%
\pgfsetlinewidth{1.505625pt}%
\definecolor{currentstroke}{rgb}{1.000000,0.000000,0.000000}%
\pgfsetstrokecolor{currentstroke}%
\pgfsetdash{}{0pt}%
\pgfpathmoveto{\pgfqpoint{1.329438in}{1.907596in}}%
\pgfpathlineto{\pgfqpoint{1.395194in}{1.226041in}}%
\pgfusepath{stroke}%
\end{pgfscope}%
\begin{pgfscope}%
\pgfpathrectangle{\pgfqpoint{0.100000in}{0.212622in}}{\pgfqpoint{3.696000in}{3.696000in}}%
\pgfusepath{clip}%
\pgfsetrectcap%
\pgfsetroundjoin%
\pgfsetlinewidth{1.505625pt}%
\definecolor{currentstroke}{rgb}{1.000000,0.000000,0.000000}%
\pgfsetstrokecolor{currentstroke}%
\pgfsetdash{}{0pt}%
\pgfpathmoveto{\pgfqpoint{1.315047in}{1.923610in}}%
\pgfpathlineto{\pgfqpoint{1.395194in}{1.226041in}}%
\pgfusepath{stroke}%
\end{pgfscope}%
\begin{pgfscope}%
\pgfpathrectangle{\pgfqpoint{0.100000in}{0.212622in}}{\pgfqpoint{3.696000in}{3.696000in}}%
\pgfusepath{clip}%
\pgfsetrectcap%
\pgfsetroundjoin%
\pgfsetlinewidth{1.505625pt}%
\definecolor{currentstroke}{rgb}{1.000000,0.000000,0.000000}%
\pgfsetstrokecolor{currentstroke}%
\pgfsetdash{}{0pt}%
\pgfpathmoveto{\pgfqpoint{1.298429in}{1.944383in}}%
\pgfpathlineto{\pgfqpoint{1.378370in}{1.231350in}}%
\pgfusepath{stroke}%
\end{pgfscope}%
\begin{pgfscope}%
\pgfpathrectangle{\pgfqpoint{0.100000in}{0.212622in}}{\pgfqpoint{3.696000in}{3.696000in}}%
\pgfusepath{clip}%
\pgfsetrectcap%
\pgfsetroundjoin%
\pgfsetlinewidth{1.505625pt}%
\definecolor{currentstroke}{rgb}{1.000000,0.000000,0.000000}%
\pgfsetstrokecolor{currentstroke}%
\pgfsetdash{}{0pt}%
\pgfpathmoveto{\pgfqpoint{1.281019in}{1.955735in}}%
\pgfpathlineto{\pgfqpoint{1.361561in}{1.236655in}}%
\pgfusepath{stroke}%
\end{pgfscope}%
\begin{pgfscope}%
\pgfpathrectangle{\pgfqpoint{0.100000in}{0.212622in}}{\pgfqpoint{3.696000in}{3.696000in}}%
\pgfusepath{clip}%
\pgfsetrectcap%
\pgfsetroundjoin%
\pgfsetlinewidth{1.505625pt}%
\definecolor{currentstroke}{rgb}{1.000000,0.000000,0.000000}%
\pgfsetstrokecolor{currentstroke}%
\pgfsetdash{}{0pt}%
\pgfpathmoveto{\pgfqpoint{1.258736in}{1.986589in}}%
\pgfpathlineto{\pgfqpoint{1.327986in}{1.247250in}}%
\pgfusepath{stroke}%
\end{pgfscope}%
\begin{pgfscope}%
\pgfpathrectangle{\pgfqpoint{0.100000in}{0.212622in}}{\pgfqpoint{3.696000in}{3.696000in}}%
\pgfusepath{clip}%
\pgfsetrectcap%
\pgfsetroundjoin%
\pgfsetlinewidth{1.505625pt}%
\definecolor{currentstroke}{rgb}{1.000000,0.000000,0.000000}%
\pgfsetstrokecolor{currentstroke}%
\pgfsetdash{}{0pt}%
\pgfpathmoveto{\pgfqpoint{1.237589in}{1.999289in}}%
\pgfpathlineto{\pgfqpoint{1.311220in}{1.252541in}}%
\pgfusepath{stroke}%
\end{pgfscope}%
\begin{pgfscope}%
\pgfpathrectangle{\pgfqpoint{0.100000in}{0.212622in}}{\pgfqpoint{3.696000in}{3.696000in}}%
\pgfusepath{clip}%
\pgfsetrectcap%
\pgfsetroundjoin%
\pgfsetlinewidth{1.505625pt}%
\definecolor{currentstroke}{rgb}{1.000000,0.000000,0.000000}%
\pgfsetstrokecolor{currentstroke}%
\pgfsetdash{}{0pt}%
\pgfpathmoveto{\pgfqpoint{1.213752in}{2.016926in}}%
\pgfpathlineto{\pgfqpoint{1.294468in}{1.257828in}}%
\pgfusepath{stroke}%
\end{pgfscope}%
\begin{pgfscope}%
\pgfpathrectangle{\pgfqpoint{0.100000in}{0.212622in}}{\pgfqpoint{3.696000in}{3.696000in}}%
\pgfusepath{clip}%
\pgfsetrectcap%
\pgfsetroundjoin%
\pgfsetlinewidth{1.505625pt}%
\definecolor{currentstroke}{rgb}{1.000000,0.000000,0.000000}%
\pgfsetstrokecolor{currentstroke}%
\pgfsetdash{}{0pt}%
\pgfpathmoveto{\pgfqpoint{1.188278in}{2.036063in}}%
\pgfpathlineto{\pgfqpoint{1.261007in}{1.268387in}}%
\pgfusepath{stroke}%
\end{pgfscope}%
\begin{pgfscope}%
\pgfpathrectangle{\pgfqpoint{0.100000in}{0.212622in}}{\pgfqpoint{3.696000in}{3.696000in}}%
\pgfusepath{clip}%
\pgfsetrectcap%
\pgfsetroundjoin%
\pgfsetlinewidth{1.505625pt}%
\definecolor{currentstroke}{rgb}{1.000000,0.000000,0.000000}%
\pgfsetstrokecolor{currentstroke}%
\pgfsetdash{}{0pt}%
\pgfpathmoveto{\pgfqpoint{1.161978in}{2.065363in}}%
\pgfpathlineto{\pgfqpoint{1.244298in}{1.273660in}}%
\pgfusepath{stroke}%
\end{pgfscope}%
\begin{pgfscope}%
\pgfpathrectangle{\pgfqpoint{0.100000in}{0.212622in}}{\pgfqpoint{3.696000in}{3.696000in}}%
\pgfusepath{clip}%
\pgfsetrectcap%
\pgfsetroundjoin%
\pgfsetlinewidth{1.505625pt}%
\definecolor{currentstroke}{rgb}{1.000000,0.000000,0.000000}%
\pgfsetstrokecolor{currentstroke}%
\pgfsetdash{}{0pt}%
\pgfpathmoveto{\pgfqpoint{1.134677in}{2.080723in}}%
\pgfpathlineto{\pgfqpoint{1.210922in}{1.284192in}}%
\pgfusepath{stroke}%
\end{pgfscope}%
\begin{pgfscope}%
\pgfpathrectangle{\pgfqpoint{0.100000in}{0.212622in}}{\pgfqpoint{3.696000in}{3.696000in}}%
\pgfusepath{clip}%
\pgfsetrectcap%
\pgfsetroundjoin%
\pgfsetlinewidth{1.505625pt}%
\definecolor{currentstroke}{rgb}{1.000000,0.000000,0.000000}%
\pgfsetstrokecolor{currentstroke}%
\pgfsetdash{}{0pt}%
\pgfpathmoveto{\pgfqpoint{1.107687in}{2.109243in}}%
\pgfpathlineto{\pgfqpoint{1.177604in}{1.294707in}}%
\pgfusepath{stroke}%
\end{pgfscope}%
\begin{pgfscope}%
\pgfpathrectangle{\pgfqpoint{0.100000in}{0.212622in}}{\pgfqpoint{3.696000in}{3.696000in}}%
\pgfusepath{clip}%
\pgfsetrectcap%
\pgfsetroundjoin%
\pgfsetlinewidth{1.505625pt}%
\definecolor{currentstroke}{rgb}{1.000000,0.000000,0.000000}%
\pgfsetstrokecolor{currentstroke}%
\pgfsetdash{}{0pt}%
\pgfpathmoveto{\pgfqpoint{1.076587in}{2.138843in}}%
\pgfpathlineto{\pgfqpoint{1.144342in}{1.305203in}}%
\pgfusepath{stroke}%
\end{pgfscope}%
\begin{pgfscope}%
\pgfpathrectangle{\pgfqpoint{0.100000in}{0.212622in}}{\pgfqpoint{3.696000in}{3.696000in}}%
\pgfusepath{clip}%
\pgfsetrectcap%
\pgfsetroundjoin%
\pgfsetlinewidth{1.505625pt}%
\definecolor{currentstroke}{rgb}{1.000000,0.000000,0.000000}%
\pgfsetstrokecolor{currentstroke}%
\pgfsetdash{}{0pt}%
\pgfpathmoveto{\pgfqpoint{1.045823in}{2.165283in}}%
\pgfpathlineto{\pgfqpoint{1.111136in}{1.315682in}}%
\pgfusepath{stroke}%
\end{pgfscope}%
\begin{pgfscope}%
\pgfpathrectangle{\pgfqpoint{0.100000in}{0.212622in}}{\pgfqpoint{3.696000in}{3.696000in}}%
\pgfusepath{clip}%
\pgfsetrectcap%
\pgfsetroundjoin%
\pgfsetlinewidth{1.505625pt}%
\definecolor{currentstroke}{rgb}{1.000000,0.000000,0.000000}%
\pgfsetstrokecolor{currentstroke}%
\pgfsetdash{}{0pt}%
\pgfpathmoveto{\pgfqpoint{1.009767in}{2.199859in}}%
\pgfpathlineto{\pgfqpoint{1.077987in}{1.326143in}}%
\pgfusepath{stroke}%
\end{pgfscope}%
\begin{pgfscope}%
\pgfpathrectangle{\pgfqpoint{0.100000in}{0.212622in}}{\pgfqpoint{3.696000in}{3.696000in}}%
\pgfusepath{clip}%
\pgfsetrectcap%
\pgfsetroundjoin%
\pgfsetlinewidth{1.505625pt}%
\definecolor{currentstroke}{rgb}{1.000000,0.000000,0.000000}%
\pgfsetstrokecolor{currentstroke}%
\pgfsetdash{}{0pt}%
\pgfpathmoveto{\pgfqpoint{0.975593in}{2.223708in}}%
\pgfpathlineto{\pgfqpoint{1.044894in}{1.336586in}}%
\pgfusepath{stroke}%
\end{pgfscope}%
\begin{pgfscope}%
\pgfpathrectangle{\pgfqpoint{0.100000in}{0.212622in}}{\pgfqpoint{3.696000in}{3.696000in}}%
\pgfusepath{clip}%
\pgfsetrectcap%
\pgfsetroundjoin%
\pgfsetlinewidth{1.505625pt}%
\definecolor{currentstroke}{rgb}{1.000000,0.000000,0.000000}%
\pgfsetstrokecolor{currentstroke}%
\pgfsetdash{}{0pt}%
\pgfpathmoveto{\pgfqpoint{0.941400in}{2.258725in}}%
\pgfpathlineto{\pgfqpoint{1.011857in}{1.347012in}}%
\pgfusepath{stroke}%
\end{pgfscope}%
\begin{pgfscope}%
\pgfpathrectangle{\pgfqpoint{0.100000in}{0.212622in}}{\pgfqpoint{3.696000in}{3.696000in}}%
\pgfusepath{clip}%
\pgfsetrectcap%
\pgfsetroundjoin%
\pgfsetlinewidth{1.505625pt}%
\definecolor{currentstroke}{rgb}{1.000000,0.000000,0.000000}%
\pgfsetstrokecolor{currentstroke}%
\pgfsetdash{}{0pt}%
\pgfpathmoveto{\pgfqpoint{0.904909in}{2.295915in}}%
\pgfpathlineto{\pgfqpoint{0.978876in}{1.357420in}}%
\pgfusepath{stroke}%
\end{pgfscope}%
\begin{pgfscope}%
\pgfpathrectangle{\pgfqpoint{0.100000in}{0.212622in}}{\pgfqpoint{3.696000in}{3.696000in}}%
\pgfusepath{clip}%
\pgfsetrectcap%
\pgfsetroundjoin%
\pgfsetlinewidth{1.505625pt}%
\definecolor{currentstroke}{rgb}{1.000000,0.000000,0.000000}%
\pgfsetstrokecolor{currentstroke}%
\pgfsetdash{}{0pt}%
\pgfpathmoveto{\pgfqpoint{0.867404in}{2.335911in}}%
\pgfpathlineto{\pgfqpoint{0.945951in}{1.367810in}}%
\pgfusepath{stroke}%
\end{pgfscope}%
\begin{pgfscope}%
\pgfpathrectangle{\pgfqpoint{0.100000in}{0.212622in}}{\pgfqpoint{3.696000in}{3.696000in}}%
\pgfusepath{clip}%
\pgfsetrectcap%
\pgfsetroundjoin%
\pgfsetlinewidth{1.505625pt}%
\definecolor{currentstroke}{rgb}{1.000000,0.000000,0.000000}%
\pgfsetstrokecolor{currentstroke}%
\pgfsetdash{}{0pt}%
\pgfpathmoveto{\pgfqpoint{0.826315in}{2.385025in}}%
\pgfpathlineto{\pgfqpoint{0.896667in}{1.383363in}}%
\pgfusepath{stroke}%
\end{pgfscope}%
\begin{pgfscope}%
\pgfpathrectangle{\pgfqpoint{0.100000in}{0.212622in}}{\pgfqpoint{3.696000in}{3.696000in}}%
\pgfusepath{clip}%
\pgfsetrectcap%
\pgfsetroundjoin%
\pgfsetlinewidth{1.505625pt}%
\definecolor{currentstroke}{rgb}{1.000000,0.000000,0.000000}%
\pgfsetstrokecolor{currentstroke}%
\pgfsetdash{}{0pt}%
\pgfpathmoveto{\pgfqpoint{0.804043in}{2.402885in}}%
\pgfpathlineto{\pgfqpoint{0.880267in}{1.388538in}}%
\pgfusepath{stroke}%
\end{pgfscope}%
\begin{pgfscope}%
\pgfpathrectangle{\pgfqpoint{0.100000in}{0.212622in}}{\pgfqpoint{3.696000in}{3.696000in}}%
\pgfusepath{clip}%
\pgfsetrectcap%
\pgfsetroundjoin%
\pgfsetlinewidth{1.505625pt}%
\definecolor{currentstroke}{rgb}{1.000000,0.000000,0.000000}%
\pgfsetstrokecolor{currentstroke}%
\pgfsetdash{}{0pt}%
\pgfpathmoveto{\pgfqpoint{0.782277in}{2.440819in}}%
\pgfpathlineto{\pgfqpoint{0.863881in}{1.393709in}}%
\pgfusepath{stroke}%
\end{pgfscope}%
\begin{pgfscope}%
\pgfpathrectangle{\pgfqpoint{0.100000in}{0.212622in}}{\pgfqpoint{3.696000in}{3.696000in}}%
\pgfusepath{clip}%
\pgfsetrectcap%
\pgfsetroundjoin%
\pgfsetlinewidth{1.505625pt}%
\definecolor{currentstroke}{rgb}{1.000000,0.000000,0.000000}%
\pgfsetstrokecolor{currentstroke}%
\pgfsetdash{}{0pt}%
\pgfpathmoveto{\pgfqpoint{0.757001in}{2.454536in}}%
\pgfpathlineto{\pgfqpoint{0.831150in}{1.404038in}}%
\pgfusepath{stroke}%
\end{pgfscope}%
\begin{pgfscope}%
\pgfpathrectangle{\pgfqpoint{0.100000in}{0.212622in}}{\pgfqpoint{3.696000in}{3.696000in}}%
\pgfusepath{clip}%
\pgfsetrectcap%
\pgfsetroundjoin%
\pgfsetlinewidth{1.505625pt}%
\definecolor{currentstroke}{rgb}{1.000000,0.000000,0.000000}%
\pgfsetstrokecolor{currentstroke}%
\pgfsetdash{}{0pt}%
\pgfpathmoveto{\pgfqpoint{0.729556in}{2.477049in}}%
\pgfpathlineto{\pgfqpoint{0.798474in}{1.414350in}}%
\pgfusepath{stroke}%
\end{pgfscope}%
\begin{pgfscope}%
\pgfpathrectangle{\pgfqpoint{0.100000in}{0.212622in}}{\pgfqpoint{3.696000in}{3.696000in}}%
\pgfusepath{clip}%
\pgfsetrectcap%
\pgfsetroundjoin%
\pgfsetlinewidth{1.505625pt}%
\definecolor{currentstroke}{rgb}{1.000000,0.000000,0.000000}%
\pgfsetstrokecolor{currentstroke}%
\pgfsetdash{}{0pt}%
\pgfpathmoveto{\pgfqpoint{0.702905in}{2.503814in}}%
\pgfpathlineto{\pgfqpoint{0.782156in}{1.419499in}}%
\pgfusepath{stroke}%
\end{pgfscope}%
\begin{pgfscope}%
\pgfpathrectangle{\pgfqpoint{0.100000in}{0.212622in}}{\pgfqpoint{3.696000in}{3.696000in}}%
\pgfusepath{clip}%
\pgfsetrectcap%
\pgfsetroundjoin%
\pgfsetlinewidth{1.505625pt}%
\definecolor{currentstroke}{rgb}{1.000000,0.000000,0.000000}%
\pgfsetstrokecolor{currentstroke}%
\pgfsetdash{}{0pt}%
\pgfpathmoveto{\pgfqpoint{0.673803in}{2.544394in}}%
\pgfpathlineto{\pgfqpoint{0.782156in}{1.419499in}}%
\pgfusepath{stroke}%
\end{pgfscope}%
\begin{pgfscope}%
\pgfpathrectangle{\pgfqpoint{0.100000in}{0.212622in}}{\pgfqpoint{3.696000in}{3.696000in}}%
\pgfusepath{clip}%
\pgfsetrectcap%
\pgfsetroundjoin%
\pgfsetlinewidth{1.505625pt}%
\definecolor{currentstroke}{rgb}{1.000000,0.000000,0.000000}%
\pgfsetstrokecolor{currentstroke}%
\pgfsetdash{}{0pt}%
\pgfpathmoveto{\pgfqpoint{0.644791in}{2.581793in}}%
\pgfpathlineto{\pgfqpoint{0.782156in}{1.419499in}}%
\pgfusepath{stroke}%
\end{pgfscope}%
\begin{pgfscope}%
\pgfpathrectangle{\pgfqpoint{0.100000in}{0.212622in}}{\pgfqpoint{3.696000in}{3.696000in}}%
\pgfusepath{clip}%
\pgfsetrectcap%
\pgfsetroundjoin%
\pgfsetlinewidth{1.505625pt}%
\definecolor{currentstroke}{rgb}{1.000000,0.000000,0.000000}%
\pgfsetstrokecolor{currentstroke}%
\pgfsetdash{}{0pt}%
\pgfpathmoveto{\pgfqpoint{0.630329in}{2.585943in}}%
\pgfpathlineto{\pgfqpoint{0.782156in}{1.419499in}}%
\pgfusepath{stroke}%
\end{pgfscope}%
\begin{pgfscope}%
\pgfpathrectangle{\pgfqpoint{0.100000in}{0.212622in}}{\pgfqpoint{3.696000in}{3.696000in}}%
\pgfusepath{clip}%
\pgfsetrectcap%
\pgfsetroundjoin%
\pgfsetlinewidth{1.505625pt}%
\definecolor{currentstroke}{rgb}{1.000000,0.000000,0.000000}%
\pgfsetstrokecolor{currentstroke}%
\pgfsetdash{}{0pt}%
\pgfpathmoveto{\pgfqpoint{0.621618in}{2.597954in}}%
\pgfpathlineto{\pgfqpoint{0.782156in}{1.419499in}}%
\pgfusepath{stroke}%
\end{pgfscope}%
\begin{pgfscope}%
\pgfpathrectangle{\pgfqpoint{0.100000in}{0.212622in}}{\pgfqpoint{3.696000in}{3.696000in}}%
\pgfusepath{clip}%
\pgfsetrectcap%
\pgfsetroundjoin%
\pgfsetlinewidth{1.505625pt}%
\definecolor{currentstroke}{rgb}{1.000000,0.000000,0.000000}%
\pgfsetstrokecolor{currentstroke}%
\pgfsetdash{}{0pt}%
\pgfpathmoveto{\pgfqpoint{0.613555in}{2.601675in}}%
\pgfpathlineto{\pgfqpoint{0.782156in}{1.419499in}}%
\pgfusepath{stroke}%
\end{pgfscope}%
\begin{pgfscope}%
\pgfpathrectangle{\pgfqpoint{0.100000in}{0.212622in}}{\pgfqpoint{3.696000in}{3.696000in}}%
\pgfusepath{clip}%
\pgfsetrectcap%
\pgfsetroundjoin%
\pgfsetlinewidth{1.505625pt}%
\definecolor{currentstroke}{rgb}{1.000000,0.000000,0.000000}%
\pgfsetstrokecolor{currentstroke}%
\pgfsetdash{}{0pt}%
\pgfpathmoveto{\pgfqpoint{0.609115in}{2.607926in}}%
\pgfpathlineto{\pgfqpoint{0.782156in}{1.419499in}}%
\pgfusepath{stroke}%
\end{pgfscope}%
\begin{pgfscope}%
\pgfpathrectangle{\pgfqpoint{0.100000in}{0.212622in}}{\pgfqpoint{3.696000in}{3.696000in}}%
\pgfusepath{clip}%
\pgfsetrectcap%
\pgfsetroundjoin%
\pgfsetlinewidth{1.505625pt}%
\definecolor{currentstroke}{rgb}{1.000000,0.000000,0.000000}%
\pgfsetstrokecolor{currentstroke}%
\pgfsetdash{}{0pt}%
\pgfpathmoveto{\pgfqpoint{0.604691in}{2.608367in}}%
\pgfpathlineto{\pgfqpoint{0.782156in}{1.419499in}}%
\pgfusepath{stroke}%
\end{pgfscope}%
\begin{pgfscope}%
\pgfpathrectangle{\pgfqpoint{0.100000in}{0.212622in}}{\pgfqpoint{3.696000in}{3.696000in}}%
\pgfusepath{clip}%
\pgfsetrectcap%
\pgfsetroundjoin%
\pgfsetlinewidth{1.505625pt}%
\definecolor{currentstroke}{rgb}{1.000000,0.000000,0.000000}%
\pgfsetstrokecolor{currentstroke}%
\pgfsetdash{}{0pt}%
\pgfpathmoveto{\pgfqpoint{0.597554in}{2.626447in}}%
\pgfpathlineto{\pgfqpoint{0.782156in}{1.419499in}}%
\pgfusepath{stroke}%
\end{pgfscope}%
\begin{pgfscope}%
\pgfpathrectangle{\pgfqpoint{0.100000in}{0.212622in}}{\pgfqpoint{3.696000in}{3.696000in}}%
\pgfusepath{clip}%
\pgfsetrectcap%
\pgfsetroundjoin%
\pgfsetlinewidth{1.505625pt}%
\definecolor{currentstroke}{rgb}{1.000000,0.000000,0.000000}%
\pgfsetstrokecolor{currentstroke}%
\pgfsetdash{}{0pt}%
\pgfpathmoveto{\pgfqpoint{0.589719in}{2.632077in}}%
\pgfpathlineto{\pgfqpoint{0.782156in}{1.419499in}}%
\pgfusepath{stroke}%
\end{pgfscope}%
\begin{pgfscope}%
\pgfpathrectangle{\pgfqpoint{0.100000in}{0.212622in}}{\pgfqpoint{3.696000in}{3.696000in}}%
\pgfusepath{clip}%
\pgfsetrectcap%
\pgfsetroundjoin%
\pgfsetlinewidth{1.505625pt}%
\definecolor{currentstroke}{rgb}{1.000000,0.000000,0.000000}%
\pgfsetstrokecolor{currentstroke}%
\pgfsetdash{}{0pt}%
\pgfpathmoveto{\pgfqpoint{0.584634in}{2.648831in}}%
\pgfpathlineto{\pgfqpoint{0.782156in}{1.419499in}}%
\pgfusepath{stroke}%
\end{pgfscope}%
\begin{pgfscope}%
\pgfpathrectangle{\pgfqpoint{0.100000in}{0.212622in}}{\pgfqpoint{3.696000in}{3.696000in}}%
\pgfusepath{clip}%
\pgfsetrectcap%
\pgfsetroundjoin%
\pgfsetlinewidth{1.505625pt}%
\definecolor{currentstroke}{rgb}{1.000000,0.000000,0.000000}%
\pgfsetstrokecolor{currentstroke}%
\pgfsetdash{}{0pt}%
\pgfpathmoveto{\pgfqpoint{0.582540in}{2.647816in}}%
\pgfpathlineto{\pgfqpoint{0.782156in}{1.419499in}}%
\pgfusepath{stroke}%
\end{pgfscope}%
\begin{pgfscope}%
\pgfpathrectangle{\pgfqpoint{0.100000in}{0.212622in}}{\pgfqpoint{3.696000in}{3.696000in}}%
\pgfusepath{clip}%
\pgfsetrectcap%
\pgfsetroundjoin%
\pgfsetlinewidth{1.505625pt}%
\definecolor{currentstroke}{rgb}{1.000000,0.000000,0.000000}%
\pgfsetstrokecolor{currentstroke}%
\pgfsetdash{}{0pt}%
\pgfpathmoveto{\pgfqpoint{0.581083in}{2.650390in}}%
\pgfpathlineto{\pgfqpoint{0.782156in}{1.419499in}}%
\pgfusepath{stroke}%
\end{pgfscope}%
\begin{pgfscope}%
\pgfpathrectangle{\pgfqpoint{0.100000in}{0.212622in}}{\pgfqpoint{3.696000in}{3.696000in}}%
\pgfusepath{clip}%
\pgfsetrectcap%
\pgfsetroundjoin%
\pgfsetlinewidth{1.505625pt}%
\definecolor{currentstroke}{rgb}{1.000000,0.000000,0.000000}%
\pgfsetstrokecolor{currentstroke}%
\pgfsetdash{}{0pt}%
\pgfpathmoveto{\pgfqpoint{0.580361in}{2.649986in}}%
\pgfpathlineto{\pgfqpoint{0.782156in}{1.419499in}}%
\pgfusepath{stroke}%
\end{pgfscope}%
\begin{pgfscope}%
\pgfpathrectangle{\pgfqpoint{0.100000in}{0.212622in}}{\pgfqpoint{3.696000in}{3.696000in}}%
\pgfusepath{clip}%
\pgfsetrectcap%
\pgfsetroundjoin%
\pgfsetlinewidth{1.505625pt}%
\definecolor{currentstroke}{rgb}{1.000000,0.000000,0.000000}%
\pgfsetstrokecolor{currentstroke}%
\pgfsetdash{}{0pt}%
\pgfpathmoveto{\pgfqpoint{0.579994in}{2.650062in}}%
\pgfpathlineto{\pgfqpoint{0.782156in}{1.419499in}}%
\pgfusepath{stroke}%
\end{pgfscope}%
\begin{pgfscope}%
\pgfpathrectangle{\pgfqpoint{0.100000in}{0.212622in}}{\pgfqpoint{3.696000in}{3.696000in}}%
\pgfusepath{clip}%
\pgfsetrectcap%
\pgfsetroundjoin%
\pgfsetlinewidth{1.505625pt}%
\definecolor{currentstroke}{rgb}{1.000000,0.000000,0.000000}%
\pgfsetstrokecolor{currentstroke}%
\pgfsetdash{}{0pt}%
\pgfpathmoveto{\pgfqpoint{0.578363in}{2.648667in}}%
\pgfpathlineto{\pgfqpoint{0.782156in}{1.419499in}}%
\pgfusepath{stroke}%
\end{pgfscope}%
\begin{pgfscope}%
\pgfpathrectangle{\pgfqpoint{0.100000in}{0.212622in}}{\pgfqpoint{3.696000in}{3.696000in}}%
\pgfusepath{clip}%
\pgfsetrectcap%
\pgfsetroundjoin%
\pgfsetlinewidth{1.505625pt}%
\definecolor{currentstroke}{rgb}{1.000000,0.000000,0.000000}%
\pgfsetstrokecolor{currentstroke}%
\pgfsetdash{}{0pt}%
\pgfpathmoveto{\pgfqpoint{0.575879in}{2.646544in}}%
\pgfpathlineto{\pgfqpoint{0.782156in}{1.419499in}}%
\pgfusepath{stroke}%
\end{pgfscope}%
\begin{pgfscope}%
\pgfpathrectangle{\pgfqpoint{0.100000in}{0.212622in}}{\pgfqpoint{3.696000in}{3.696000in}}%
\pgfusepath{clip}%
\pgfsetbuttcap%
\pgfsetroundjoin%
\definecolor{currentfill}{rgb}{0.121569,0.466667,0.705882}%
\pgfsetfillcolor{currentfill}%
\pgfsetfillopacity{0.300000}%
\pgfsetlinewidth{1.003750pt}%
\definecolor{currentstroke}{rgb}{0.121569,0.466667,0.705882}%
\pgfsetstrokecolor{currentstroke}%
\pgfsetstrokeopacity{0.300000}%
\pgfsetdash{}{0pt}%
\pgfpathmoveto{\pgfqpoint{1.486867in}{1.734005in}}%
\pgfpathcurveto{\pgfqpoint{1.495103in}{1.734005in}}{\pgfqpoint{1.503003in}{1.737277in}}{\pgfqpoint{1.508827in}{1.743101in}}%
\pgfpathcurveto{\pgfqpoint{1.514651in}{1.748925in}}{\pgfqpoint{1.517924in}{1.756825in}}{\pgfqpoint{1.517924in}{1.765061in}}%
\pgfpathcurveto{\pgfqpoint{1.517924in}{1.773298in}}{\pgfqpoint{1.514651in}{1.781198in}}{\pgfqpoint{1.508827in}{1.787022in}}%
\pgfpathcurveto{\pgfqpoint{1.503003in}{1.792846in}}{\pgfqpoint{1.495103in}{1.796118in}}{\pgfqpoint{1.486867in}{1.796118in}}%
\pgfpathcurveto{\pgfqpoint{1.478631in}{1.796118in}}{\pgfqpoint{1.470731in}{1.792846in}}{\pgfqpoint{1.464907in}{1.787022in}}%
\pgfpathcurveto{\pgfqpoint{1.459083in}{1.781198in}}{\pgfqpoint{1.455811in}{1.773298in}}{\pgfqpoint{1.455811in}{1.765061in}}%
\pgfpathcurveto{\pgfqpoint{1.455811in}{1.756825in}}{\pgfqpoint{1.459083in}{1.748925in}}{\pgfqpoint{1.464907in}{1.743101in}}%
\pgfpathcurveto{\pgfqpoint{1.470731in}{1.737277in}}{\pgfqpoint{1.478631in}{1.734005in}}{\pgfqpoint{1.486867in}{1.734005in}}%
\pgfpathclose%
\pgfusepath{stroke,fill}%
\end{pgfscope}%
\begin{pgfscope}%
\pgfpathrectangle{\pgfqpoint{0.100000in}{0.212622in}}{\pgfqpoint{3.696000in}{3.696000in}}%
\pgfusepath{clip}%
\pgfsetbuttcap%
\pgfsetroundjoin%
\definecolor{currentfill}{rgb}{0.121569,0.466667,0.705882}%
\pgfsetfillcolor{currentfill}%
\pgfsetfillopacity{0.300016}%
\pgfsetlinewidth{1.003750pt}%
\definecolor{currentstroke}{rgb}{0.121569,0.466667,0.705882}%
\pgfsetstrokecolor{currentstroke}%
\pgfsetstrokeopacity{0.300016}%
\pgfsetdash{}{0pt}%
\pgfpathmoveto{\pgfqpoint{1.486617in}{1.734362in}}%
\pgfpathcurveto{\pgfqpoint{1.494853in}{1.734362in}}{\pgfqpoint{1.502753in}{1.737634in}}{\pgfqpoint{1.508577in}{1.743458in}}%
\pgfpathcurveto{\pgfqpoint{1.514401in}{1.749282in}}{\pgfqpoint{1.517673in}{1.757182in}}{\pgfqpoint{1.517673in}{1.765418in}}%
\pgfpathcurveto{\pgfqpoint{1.517673in}{1.773655in}}{\pgfqpoint{1.514401in}{1.781555in}}{\pgfqpoint{1.508577in}{1.787379in}}%
\pgfpathcurveto{\pgfqpoint{1.502753in}{1.793202in}}{\pgfqpoint{1.494853in}{1.796475in}}{\pgfqpoint{1.486617in}{1.796475in}}%
\pgfpathcurveto{\pgfqpoint{1.478380in}{1.796475in}}{\pgfqpoint{1.470480in}{1.793202in}}{\pgfqpoint{1.464656in}{1.787379in}}%
\pgfpathcurveto{\pgfqpoint{1.458833in}{1.781555in}}{\pgfqpoint{1.455560in}{1.773655in}}{\pgfqpoint{1.455560in}{1.765418in}}%
\pgfpathcurveto{\pgfqpoint{1.455560in}{1.757182in}}{\pgfqpoint{1.458833in}{1.749282in}}{\pgfqpoint{1.464656in}{1.743458in}}%
\pgfpathcurveto{\pgfqpoint{1.470480in}{1.737634in}}{\pgfqpoint{1.478380in}{1.734362in}}{\pgfqpoint{1.486617in}{1.734362in}}%
\pgfpathclose%
\pgfusepath{stroke,fill}%
\end{pgfscope}%
\begin{pgfscope}%
\pgfpathrectangle{\pgfqpoint{0.100000in}{0.212622in}}{\pgfqpoint{3.696000in}{3.696000in}}%
\pgfusepath{clip}%
\pgfsetbuttcap%
\pgfsetroundjoin%
\definecolor{currentfill}{rgb}{0.121569,0.466667,0.705882}%
\pgfsetfillcolor{currentfill}%
\pgfsetfillopacity{0.300031}%
\pgfsetlinewidth{1.003750pt}%
\definecolor{currentstroke}{rgb}{0.121569,0.466667,0.705882}%
\pgfsetstrokecolor{currentstroke}%
\pgfsetstrokeopacity{0.300031}%
\pgfsetdash{}{0pt}%
\pgfpathmoveto{\pgfqpoint{1.486992in}{1.733966in}}%
\pgfpathcurveto{\pgfqpoint{1.495228in}{1.733966in}}{\pgfqpoint{1.503128in}{1.737238in}}{\pgfqpoint{1.508952in}{1.743062in}}%
\pgfpathcurveto{\pgfqpoint{1.514776in}{1.748886in}}{\pgfqpoint{1.518048in}{1.756786in}}{\pgfqpoint{1.518048in}{1.765022in}}%
\pgfpathcurveto{\pgfqpoint{1.518048in}{1.773259in}}{\pgfqpoint{1.514776in}{1.781159in}}{\pgfqpoint{1.508952in}{1.786983in}}%
\pgfpathcurveto{\pgfqpoint{1.503128in}{1.792807in}}{\pgfqpoint{1.495228in}{1.796079in}}{\pgfqpoint{1.486992in}{1.796079in}}%
\pgfpathcurveto{\pgfqpoint{1.478755in}{1.796079in}}{\pgfqpoint{1.470855in}{1.792807in}}{\pgfqpoint{1.465031in}{1.786983in}}%
\pgfpathcurveto{\pgfqpoint{1.459207in}{1.781159in}}{\pgfqpoint{1.455935in}{1.773259in}}{\pgfqpoint{1.455935in}{1.765022in}}%
\pgfpathcurveto{\pgfqpoint{1.455935in}{1.756786in}}{\pgfqpoint{1.459207in}{1.748886in}}{\pgfqpoint{1.465031in}{1.743062in}}%
\pgfpathcurveto{\pgfqpoint{1.470855in}{1.737238in}}{\pgfqpoint{1.478755in}{1.733966in}}{\pgfqpoint{1.486992in}{1.733966in}}%
\pgfpathclose%
\pgfusepath{stroke,fill}%
\end{pgfscope}%
\begin{pgfscope}%
\pgfpathrectangle{\pgfqpoint{0.100000in}{0.212622in}}{\pgfqpoint{3.696000in}{3.696000in}}%
\pgfusepath{clip}%
\pgfsetbuttcap%
\pgfsetroundjoin%
\definecolor{currentfill}{rgb}{0.121569,0.466667,0.705882}%
\pgfsetfillcolor{currentfill}%
\pgfsetfillopacity{0.300036}%
\pgfsetlinewidth{1.003750pt}%
\definecolor{currentstroke}{rgb}{0.121569,0.466667,0.705882}%
\pgfsetstrokecolor{currentstroke}%
\pgfsetstrokeopacity{0.300036}%
\pgfsetdash{}{0pt}%
\pgfpathmoveto{\pgfqpoint{1.486099in}{1.734597in}}%
\pgfpathcurveto{\pgfqpoint{1.494335in}{1.734597in}}{\pgfqpoint{1.502235in}{1.737869in}}{\pgfqpoint{1.508059in}{1.743693in}}%
\pgfpathcurveto{\pgfqpoint{1.513883in}{1.749517in}}{\pgfqpoint{1.517155in}{1.757417in}}{\pgfqpoint{1.517155in}{1.765653in}}%
\pgfpathcurveto{\pgfqpoint{1.517155in}{1.773890in}}{\pgfqpoint{1.513883in}{1.781790in}}{\pgfqpoint{1.508059in}{1.787614in}}%
\pgfpathcurveto{\pgfqpoint{1.502235in}{1.793438in}}{\pgfqpoint{1.494335in}{1.796710in}}{\pgfqpoint{1.486099in}{1.796710in}}%
\pgfpathcurveto{\pgfqpoint{1.477862in}{1.796710in}}{\pgfqpoint{1.469962in}{1.793438in}}{\pgfqpoint{1.464138in}{1.787614in}}%
\pgfpathcurveto{\pgfqpoint{1.458314in}{1.781790in}}{\pgfqpoint{1.455042in}{1.773890in}}{\pgfqpoint{1.455042in}{1.765653in}}%
\pgfpathcurveto{\pgfqpoint{1.455042in}{1.757417in}}{\pgfqpoint{1.458314in}{1.749517in}}{\pgfqpoint{1.464138in}{1.743693in}}%
\pgfpathcurveto{\pgfqpoint{1.469962in}{1.737869in}}{\pgfqpoint{1.477862in}{1.734597in}}{\pgfqpoint{1.486099in}{1.734597in}}%
\pgfpathclose%
\pgfusepath{stroke,fill}%
\end{pgfscope}%
\begin{pgfscope}%
\pgfpathrectangle{\pgfqpoint{0.100000in}{0.212622in}}{\pgfqpoint{3.696000in}{3.696000in}}%
\pgfusepath{clip}%
\pgfsetbuttcap%
\pgfsetroundjoin%
\definecolor{currentfill}{rgb}{0.121569,0.466667,0.705882}%
\pgfsetfillcolor{currentfill}%
\pgfsetfillopacity{0.300064}%
\pgfsetlinewidth{1.003750pt}%
\definecolor{currentstroke}{rgb}{0.121569,0.466667,0.705882}%
\pgfsetstrokecolor{currentstroke}%
\pgfsetstrokeopacity{0.300064}%
\pgfsetdash{}{0pt}%
\pgfpathmoveto{\pgfqpoint{1.487043in}{1.733986in}}%
\pgfpathcurveto{\pgfqpoint{1.495280in}{1.733986in}}{\pgfqpoint{1.503180in}{1.737258in}}{\pgfqpoint{1.509004in}{1.743082in}}%
\pgfpathcurveto{\pgfqpoint{1.514827in}{1.748906in}}{\pgfqpoint{1.518100in}{1.756806in}}{\pgfqpoint{1.518100in}{1.765043in}}%
\pgfpathcurveto{\pgfqpoint{1.518100in}{1.773279in}}{\pgfqpoint{1.514827in}{1.781179in}}{\pgfqpoint{1.509004in}{1.787003in}}%
\pgfpathcurveto{\pgfqpoint{1.503180in}{1.792827in}}{\pgfqpoint{1.495280in}{1.796099in}}{\pgfqpoint{1.487043in}{1.796099in}}%
\pgfpathcurveto{\pgfqpoint{1.478807in}{1.796099in}}{\pgfqpoint{1.470907in}{1.792827in}}{\pgfqpoint{1.465083in}{1.787003in}}%
\pgfpathcurveto{\pgfqpoint{1.459259in}{1.781179in}}{\pgfqpoint{1.455987in}{1.773279in}}{\pgfqpoint{1.455987in}{1.765043in}}%
\pgfpathcurveto{\pgfqpoint{1.455987in}{1.756806in}}{\pgfqpoint{1.459259in}{1.748906in}}{\pgfqpoint{1.465083in}{1.743082in}}%
\pgfpathcurveto{\pgfqpoint{1.470907in}{1.737258in}}{\pgfqpoint{1.478807in}{1.733986in}}{\pgfqpoint{1.487043in}{1.733986in}}%
\pgfpathclose%
\pgfusepath{stroke,fill}%
\end{pgfscope}%
\begin{pgfscope}%
\pgfpathrectangle{\pgfqpoint{0.100000in}{0.212622in}}{\pgfqpoint{3.696000in}{3.696000in}}%
\pgfusepath{clip}%
\pgfsetbuttcap%
\pgfsetroundjoin%
\definecolor{currentfill}{rgb}{0.121569,0.466667,0.705882}%
\pgfsetfillcolor{currentfill}%
\pgfsetfillopacity{0.300069}%
\pgfsetlinewidth{1.003750pt}%
\definecolor{currentstroke}{rgb}{0.121569,0.466667,0.705882}%
\pgfsetstrokecolor{currentstroke}%
\pgfsetstrokeopacity{0.300069}%
\pgfsetdash{}{0pt}%
\pgfpathmoveto{\pgfqpoint{1.485954in}{1.734748in}}%
\pgfpathcurveto{\pgfqpoint{1.494191in}{1.734748in}}{\pgfqpoint{1.502091in}{1.738021in}}{\pgfqpoint{1.507915in}{1.743844in}}%
\pgfpathcurveto{\pgfqpoint{1.513738in}{1.749668in}}{\pgfqpoint{1.517011in}{1.757568in}}{\pgfqpoint{1.517011in}{1.765805in}}%
\pgfpathcurveto{\pgfqpoint{1.517011in}{1.774041in}}{\pgfqpoint{1.513738in}{1.781941in}}{\pgfqpoint{1.507915in}{1.787765in}}%
\pgfpathcurveto{\pgfqpoint{1.502091in}{1.793589in}}{\pgfqpoint{1.494191in}{1.796861in}}{\pgfqpoint{1.485954in}{1.796861in}}%
\pgfpathcurveto{\pgfqpoint{1.477718in}{1.796861in}}{\pgfqpoint{1.469818in}{1.793589in}}{\pgfqpoint{1.463994in}{1.787765in}}%
\pgfpathcurveto{\pgfqpoint{1.458170in}{1.781941in}}{\pgfqpoint{1.454898in}{1.774041in}}{\pgfqpoint{1.454898in}{1.765805in}}%
\pgfpathcurveto{\pgfqpoint{1.454898in}{1.757568in}}{\pgfqpoint{1.458170in}{1.749668in}}{\pgfqpoint{1.463994in}{1.743844in}}%
\pgfpathcurveto{\pgfqpoint{1.469818in}{1.738021in}}{\pgfqpoint{1.477718in}{1.734748in}}{\pgfqpoint{1.485954in}{1.734748in}}%
\pgfpathclose%
\pgfusepath{stroke,fill}%
\end{pgfscope}%
\begin{pgfscope}%
\pgfpathrectangle{\pgfqpoint{0.100000in}{0.212622in}}{\pgfqpoint{3.696000in}{3.696000in}}%
\pgfusepath{clip}%
\pgfsetbuttcap%
\pgfsetroundjoin%
\definecolor{currentfill}{rgb}{0.121569,0.466667,0.705882}%
\pgfsetfillcolor{currentfill}%
\pgfsetfillopacity{0.300079}%
\pgfsetlinewidth{1.003750pt}%
\definecolor{currentstroke}{rgb}{0.121569,0.466667,0.705882}%
\pgfsetstrokecolor{currentstroke}%
\pgfsetstrokeopacity{0.300079}%
\pgfsetdash{}{0pt}%
\pgfpathmoveto{\pgfqpoint{1.487062in}{1.733975in}}%
\pgfpathcurveto{\pgfqpoint{1.495299in}{1.733975in}}{\pgfqpoint{1.503199in}{1.737248in}}{\pgfqpoint{1.509023in}{1.743072in}}%
\pgfpathcurveto{\pgfqpoint{1.514847in}{1.748896in}}{\pgfqpoint{1.518119in}{1.756796in}}{\pgfqpoint{1.518119in}{1.765032in}}%
\pgfpathcurveto{\pgfqpoint{1.518119in}{1.773268in}}{\pgfqpoint{1.514847in}{1.781168in}}{\pgfqpoint{1.509023in}{1.786992in}}%
\pgfpathcurveto{\pgfqpoint{1.503199in}{1.792816in}}{\pgfqpoint{1.495299in}{1.796088in}}{\pgfqpoint{1.487062in}{1.796088in}}%
\pgfpathcurveto{\pgfqpoint{1.478826in}{1.796088in}}{\pgfqpoint{1.470926in}{1.792816in}}{\pgfqpoint{1.465102in}{1.786992in}}%
\pgfpathcurveto{\pgfqpoint{1.459278in}{1.781168in}}{\pgfqpoint{1.456006in}{1.773268in}}{\pgfqpoint{1.456006in}{1.765032in}}%
\pgfpathcurveto{\pgfqpoint{1.456006in}{1.756796in}}{\pgfqpoint{1.459278in}{1.748896in}}{\pgfqpoint{1.465102in}{1.743072in}}%
\pgfpathcurveto{\pgfqpoint{1.470926in}{1.737248in}}{\pgfqpoint{1.478826in}{1.733975in}}{\pgfqpoint{1.487062in}{1.733975in}}%
\pgfpathclose%
\pgfusepath{stroke,fill}%
\end{pgfscope}%
\begin{pgfscope}%
\pgfpathrectangle{\pgfqpoint{0.100000in}{0.212622in}}{\pgfqpoint{3.696000in}{3.696000in}}%
\pgfusepath{clip}%
\pgfsetbuttcap%
\pgfsetroundjoin%
\definecolor{currentfill}{rgb}{0.121569,0.466667,0.705882}%
\pgfsetfillcolor{currentfill}%
\pgfsetfillopacity{0.300090}%
\pgfsetlinewidth{1.003750pt}%
\definecolor{currentstroke}{rgb}{0.121569,0.466667,0.705882}%
\pgfsetstrokecolor{currentstroke}%
\pgfsetstrokeopacity{0.300090}%
\pgfsetdash{}{0pt}%
\pgfpathmoveto{\pgfqpoint{1.487070in}{1.733976in}}%
\pgfpathcurveto{\pgfqpoint{1.495307in}{1.733976in}}{\pgfqpoint{1.503207in}{1.737248in}}{\pgfqpoint{1.509031in}{1.743072in}}%
\pgfpathcurveto{\pgfqpoint{1.514855in}{1.748896in}}{\pgfqpoint{1.518127in}{1.756796in}}{\pgfqpoint{1.518127in}{1.765033in}}%
\pgfpathcurveto{\pgfqpoint{1.518127in}{1.773269in}}{\pgfqpoint{1.514855in}{1.781169in}}{\pgfqpoint{1.509031in}{1.786993in}}%
\pgfpathcurveto{\pgfqpoint{1.503207in}{1.792817in}}{\pgfqpoint{1.495307in}{1.796089in}}{\pgfqpoint{1.487070in}{1.796089in}}%
\pgfpathcurveto{\pgfqpoint{1.478834in}{1.796089in}}{\pgfqpoint{1.470934in}{1.792817in}}{\pgfqpoint{1.465110in}{1.786993in}}%
\pgfpathcurveto{\pgfqpoint{1.459286in}{1.781169in}}{\pgfqpoint{1.456014in}{1.773269in}}{\pgfqpoint{1.456014in}{1.765033in}}%
\pgfpathcurveto{\pgfqpoint{1.456014in}{1.756796in}}{\pgfqpoint{1.459286in}{1.748896in}}{\pgfqpoint{1.465110in}{1.743072in}}%
\pgfpathcurveto{\pgfqpoint{1.470934in}{1.737248in}}{\pgfqpoint{1.478834in}{1.733976in}}{\pgfqpoint{1.487070in}{1.733976in}}%
\pgfpathclose%
\pgfusepath{stroke,fill}%
\end{pgfscope}%
\begin{pgfscope}%
\pgfpathrectangle{\pgfqpoint{0.100000in}{0.212622in}}{\pgfqpoint{3.696000in}{3.696000in}}%
\pgfusepath{clip}%
\pgfsetbuttcap%
\pgfsetroundjoin%
\definecolor{currentfill}{rgb}{0.121569,0.466667,0.705882}%
\pgfsetfillcolor{currentfill}%
\pgfsetfillopacity{0.300103}%
\pgfsetlinewidth{1.003750pt}%
\definecolor{currentstroke}{rgb}{0.121569,0.466667,0.705882}%
\pgfsetstrokecolor{currentstroke}%
\pgfsetstrokeopacity{0.300103}%
\pgfsetdash{}{0pt}%
\pgfpathmoveto{\pgfqpoint{1.485706in}{1.734746in}}%
\pgfpathcurveto{\pgfqpoint{1.493942in}{1.734746in}}{\pgfqpoint{1.501842in}{1.738018in}}{\pgfqpoint{1.507666in}{1.743842in}}%
\pgfpathcurveto{\pgfqpoint{1.513490in}{1.749666in}}{\pgfqpoint{1.516763in}{1.757566in}}{\pgfqpoint{1.516763in}{1.765802in}}%
\pgfpathcurveto{\pgfqpoint{1.516763in}{1.774039in}}{\pgfqpoint{1.513490in}{1.781939in}}{\pgfqpoint{1.507666in}{1.787763in}}%
\pgfpathcurveto{\pgfqpoint{1.501842in}{1.793587in}}{\pgfqpoint{1.493942in}{1.796859in}}{\pgfqpoint{1.485706in}{1.796859in}}%
\pgfpathcurveto{\pgfqpoint{1.477470in}{1.796859in}}{\pgfqpoint{1.469570in}{1.793587in}}{\pgfqpoint{1.463746in}{1.787763in}}%
\pgfpathcurveto{\pgfqpoint{1.457922in}{1.781939in}}{\pgfqpoint{1.454650in}{1.774039in}}{\pgfqpoint{1.454650in}{1.765802in}}%
\pgfpathcurveto{\pgfqpoint{1.454650in}{1.757566in}}{\pgfqpoint{1.457922in}{1.749666in}}{\pgfqpoint{1.463746in}{1.743842in}}%
\pgfpathcurveto{\pgfqpoint{1.469570in}{1.738018in}}{\pgfqpoint{1.477470in}{1.734746in}}{\pgfqpoint{1.485706in}{1.734746in}}%
\pgfpathclose%
\pgfusepath{stroke,fill}%
\end{pgfscope}%
\begin{pgfscope}%
\pgfpathrectangle{\pgfqpoint{0.100000in}{0.212622in}}{\pgfqpoint{3.696000in}{3.696000in}}%
\pgfusepath{clip}%
\pgfsetbuttcap%
\pgfsetroundjoin%
\definecolor{currentfill}{rgb}{0.121569,0.466667,0.705882}%
\pgfsetfillcolor{currentfill}%
\pgfsetfillopacity{0.300234}%
\pgfsetlinewidth{1.003750pt}%
\definecolor{currentstroke}{rgb}{0.121569,0.466667,0.705882}%
\pgfsetstrokecolor{currentstroke}%
\pgfsetstrokeopacity{0.300234}%
\pgfsetdash{}{0pt}%
\pgfpathmoveto{\pgfqpoint{1.487121in}{1.733804in}}%
\pgfpathcurveto{\pgfqpoint{1.495358in}{1.733804in}}{\pgfqpoint{1.503258in}{1.737076in}}{\pgfqpoint{1.509081in}{1.742900in}}%
\pgfpathcurveto{\pgfqpoint{1.514905in}{1.748724in}}{\pgfqpoint{1.518178in}{1.756624in}}{\pgfqpoint{1.518178in}{1.764861in}}%
\pgfpathcurveto{\pgfqpoint{1.518178in}{1.773097in}}{\pgfqpoint{1.514905in}{1.780997in}}{\pgfqpoint{1.509081in}{1.786821in}}%
\pgfpathcurveto{\pgfqpoint{1.503258in}{1.792645in}}{\pgfqpoint{1.495358in}{1.795917in}}{\pgfqpoint{1.487121in}{1.795917in}}%
\pgfpathcurveto{\pgfqpoint{1.478885in}{1.795917in}}{\pgfqpoint{1.470985in}{1.792645in}}{\pgfqpoint{1.465161in}{1.786821in}}%
\pgfpathcurveto{\pgfqpoint{1.459337in}{1.780997in}}{\pgfqpoint{1.456065in}{1.773097in}}{\pgfqpoint{1.456065in}{1.764861in}}%
\pgfpathcurveto{\pgfqpoint{1.456065in}{1.756624in}}{\pgfqpoint{1.459337in}{1.748724in}}{\pgfqpoint{1.465161in}{1.742900in}}%
\pgfpathcurveto{\pgfqpoint{1.470985in}{1.737076in}}{\pgfqpoint{1.478885in}{1.733804in}}{\pgfqpoint{1.487121in}{1.733804in}}%
\pgfpathclose%
\pgfusepath{stroke,fill}%
\end{pgfscope}%
\begin{pgfscope}%
\pgfpathrectangle{\pgfqpoint{0.100000in}{0.212622in}}{\pgfqpoint{3.696000in}{3.696000in}}%
\pgfusepath{clip}%
\pgfsetbuttcap%
\pgfsetroundjoin%
\definecolor{currentfill}{rgb}{0.121569,0.466667,0.705882}%
\pgfsetfillcolor{currentfill}%
\pgfsetfillopacity{0.300359}%
\pgfsetlinewidth{1.003750pt}%
\definecolor{currentstroke}{rgb}{0.121569,0.466667,0.705882}%
\pgfsetstrokecolor{currentstroke}%
\pgfsetstrokeopacity{0.300359}%
\pgfsetdash{}{0pt}%
\pgfpathmoveto{\pgfqpoint{1.485277in}{1.735512in}}%
\pgfpathcurveto{\pgfqpoint{1.493513in}{1.735512in}}{\pgfqpoint{1.501413in}{1.738784in}}{\pgfqpoint{1.507237in}{1.744608in}}%
\pgfpathcurveto{\pgfqpoint{1.513061in}{1.750432in}}{\pgfqpoint{1.516334in}{1.758332in}}{\pgfqpoint{1.516334in}{1.766568in}}%
\pgfpathcurveto{\pgfqpoint{1.516334in}{1.774805in}}{\pgfqpoint{1.513061in}{1.782705in}}{\pgfqpoint{1.507237in}{1.788529in}}%
\pgfpathcurveto{\pgfqpoint{1.501413in}{1.794353in}}{\pgfqpoint{1.493513in}{1.797625in}}{\pgfqpoint{1.485277in}{1.797625in}}%
\pgfpathcurveto{\pgfqpoint{1.477041in}{1.797625in}}{\pgfqpoint{1.469141in}{1.794353in}}{\pgfqpoint{1.463317in}{1.788529in}}%
\pgfpathcurveto{\pgfqpoint{1.457493in}{1.782705in}}{\pgfqpoint{1.454221in}{1.774805in}}{\pgfqpoint{1.454221in}{1.766568in}}%
\pgfpathcurveto{\pgfqpoint{1.454221in}{1.758332in}}{\pgfqpoint{1.457493in}{1.750432in}}{\pgfqpoint{1.463317in}{1.744608in}}%
\pgfpathcurveto{\pgfqpoint{1.469141in}{1.738784in}}{\pgfqpoint{1.477041in}{1.735512in}}{\pgfqpoint{1.485277in}{1.735512in}}%
\pgfpathclose%
\pgfusepath{stroke,fill}%
\end{pgfscope}%
\begin{pgfscope}%
\pgfpathrectangle{\pgfqpoint{0.100000in}{0.212622in}}{\pgfqpoint{3.696000in}{3.696000in}}%
\pgfusepath{clip}%
\pgfsetbuttcap%
\pgfsetroundjoin%
\definecolor{currentfill}{rgb}{0.121569,0.466667,0.705882}%
\pgfsetfillcolor{currentfill}%
\pgfsetfillopacity{0.300375}%
\pgfsetlinewidth{1.003750pt}%
\definecolor{currentstroke}{rgb}{0.121569,0.466667,0.705882}%
\pgfsetstrokecolor{currentstroke}%
\pgfsetstrokeopacity{0.300375}%
\pgfsetdash{}{0pt}%
\pgfpathmoveto{\pgfqpoint{1.485236in}{1.735529in}}%
\pgfpathcurveto{\pgfqpoint{1.493472in}{1.735529in}}{\pgfqpoint{1.501372in}{1.738802in}}{\pgfqpoint{1.507196in}{1.744626in}}%
\pgfpathcurveto{\pgfqpoint{1.513020in}{1.750450in}}{\pgfqpoint{1.516293in}{1.758350in}}{\pgfqpoint{1.516293in}{1.766586in}}%
\pgfpathcurveto{\pgfqpoint{1.516293in}{1.774822in}}{\pgfqpoint{1.513020in}{1.782722in}}{\pgfqpoint{1.507196in}{1.788546in}}%
\pgfpathcurveto{\pgfqpoint{1.501372in}{1.794370in}}{\pgfqpoint{1.493472in}{1.797642in}}{\pgfqpoint{1.485236in}{1.797642in}}%
\pgfpathcurveto{\pgfqpoint{1.477000in}{1.797642in}}{\pgfqpoint{1.469100in}{1.794370in}}{\pgfqpoint{1.463276in}{1.788546in}}%
\pgfpathcurveto{\pgfqpoint{1.457452in}{1.782722in}}{\pgfqpoint{1.454180in}{1.774822in}}{\pgfqpoint{1.454180in}{1.766586in}}%
\pgfpathcurveto{\pgfqpoint{1.454180in}{1.758350in}}{\pgfqpoint{1.457452in}{1.750450in}}{\pgfqpoint{1.463276in}{1.744626in}}%
\pgfpathcurveto{\pgfqpoint{1.469100in}{1.738802in}}{\pgfqpoint{1.477000in}{1.735529in}}{\pgfqpoint{1.485236in}{1.735529in}}%
\pgfpathclose%
\pgfusepath{stroke,fill}%
\end{pgfscope}%
\begin{pgfscope}%
\pgfpathrectangle{\pgfqpoint{0.100000in}{0.212622in}}{\pgfqpoint{3.696000in}{3.696000in}}%
\pgfusepath{clip}%
\pgfsetbuttcap%
\pgfsetroundjoin%
\definecolor{currentfill}{rgb}{0.121569,0.466667,0.705882}%
\pgfsetfillcolor{currentfill}%
\pgfsetfillopacity{0.300400}%
\pgfsetlinewidth{1.003750pt}%
\definecolor{currentstroke}{rgb}{0.121569,0.466667,0.705882}%
\pgfsetstrokecolor{currentstroke}%
\pgfsetstrokeopacity{0.300400}%
\pgfsetdash{}{0pt}%
\pgfpathmoveto{\pgfqpoint{1.485174in}{1.735518in}}%
\pgfpathcurveto{\pgfqpoint{1.493410in}{1.735518in}}{\pgfqpoint{1.501310in}{1.738790in}}{\pgfqpoint{1.507134in}{1.744614in}}%
\pgfpathcurveto{\pgfqpoint{1.512958in}{1.750438in}}{\pgfqpoint{1.516231in}{1.758338in}}{\pgfqpoint{1.516231in}{1.766574in}}%
\pgfpathcurveto{\pgfqpoint{1.516231in}{1.774811in}}{\pgfqpoint{1.512958in}{1.782711in}}{\pgfqpoint{1.507134in}{1.788535in}}%
\pgfpathcurveto{\pgfqpoint{1.501310in}{1.794359in}}{\pgfqpoint{1.493410in}{1.797631in}}{\pgfqpoint{1.485174in}{1.797631in}}%
\pgfpathcurveto{\pgfqpoint{1.476938in}{1.797631in}}{\pgfqpoint{1.469038in}{1.794359in}}{\pgfqpoint{1.463214in}{1.788535in}}%
\pgfpathcurveto{\pgfqpoint{1.457390in}{1.782711in}}{\pgfqpoint{1.454118in}{1.774811in}}{\pgfqpoint{1.454118in}{1.766574in}}%
\pgfpathcurveto{\pgfqpoint{1.454118in}{1.758338in}}{\pgfqpoint{1.457390in}{1.750438in}}{\pgfqpoint{1.463214in}{1.744614in}}%
\pgfpathcurveto{\pgfqpoint{1.469038in}{1.738790in}}{\pgfqpoint{1.476938in}{1.735518in}}{\pgfqpoint{1.485174in}{1.735518in}}%
\pgfpathclose%
\pgfusepath{stroke,fill}%
\end{pgfscope}%
\begin{pgfscope}%
\pgfpathrectangle{\pgfqpoint{0.100000in}{0.212622in}}{\pgfqpoint{3.696000in}{3.696000in}}%
\pgfusepath{clip}%
\pgfsetbuttcap%
\pgfsetroundjoin%
\definecolor{currentfill}{rgb}{0.121569,0.466667,0.705882}%
\pgfsetfillcolor{currentfill}%
\pgfsetfillopacity{0.300443}%
\pgfsetlinewidth{1.003750pt}%
\definecolor{currentstroke}{rgb}{0.121569,0.466667,0.705882}%
\pgfsetstrokecolor{currentstroke}%
\pgfsetstrokeopacity{0.300443}%
\pgfsetdash{}{0pt}%
\pgfpathmoveto{\pgfqpoint{1.485052in}{1.735506in}}%
\pgfpathcurveto{\pgfqpoint{1.493288in}{1.735506in}}{\pgfqpoint{1.501188in}{1.738778in}}{\pgfqpoint{1.507012in}{1.744602in}}%
\pgfpathcurveto{\pgfqpoint{1.512836in}{1.750426in}}{\pgfqpoint{1.516109in}{1.758326in}}{\pgfqpoint{1.516109in}{1.766562in}}%
\pgfpathcurveto{\pgfqpoint{1.516109in}{1.774799in}}{\pgfqpoint{1.512836in}{1.782699in}}{\pgfqpoint{1.507012in}{1.788523in}}%
\pgfpathcurveto{\pgfqpoint{1.501188in}{1.794346in}}{\pgfqpoint{1.493288in}{1.797619in}}{\pgfqpoint{1.485052in}{1.797619in}}%
\pgfpathcurveto{\pgfqpoint{1.476816in}{1.797619in}}{\pgfqpoint{1.468916in}{1.794346in}}{\pgfqpoint{1.463092in}{1.788523in}}%
\pgfpathcurveto{\pgfqpoint{1.457268in}{1.782699in}}{\pgfqpoint{1.453996in}{1.774799in}}{\pgfqpoint{1.453996in}{1.766562in}}%
\pgfpathcurveto{\pgfqpoint{1.453996in}{1.758326in}}{\pgfqpoint{1.457268in}{1.750426in}}{\pgfqpoint{1.463092in}{1.744602in}}%
\pgfpathcurveto{\pgfqpoint{1.468916in}{1.738778in}}{\pgfqpoint{1.476816in}{1.735506in}}{\pgfqpoint{1.485052in}{1.735506in}}%
\pgfpathclose%
\pgfusepath{stroke,fill}%
\end{pgfscope}%
\begin{pgfscope}%
\pgfpathrectangle{\pgfqpoint{0.100000in}{0.212622in}}{\pgfqpoint{3.696000in}{3.696000in}}%
\pgfusepath{clip}%
\pgfsetbuttcap%
\pgfsetroundjoin%
\definecolor{currentfill}{rgb}{0.121569,0.466667,0.705882}%
\pgfsetfillcolor{currentfill}%
\pgfsetfillopacity{0.300519}%
\pgfsetlinewidth{1.003750pt}%
\definecolor{currentstroke}{rgb}{0.121569,0.466667,0.705882}%
\pgfsetstrokecolor{currentstroke}%
\pgfsetstrokeopacity{0.300519}%
\pgfsetdash{}{0pt}%
\pgfpathmoveto{\pgfqpoint{1.484832in}{1.735467in}}%
\pgfpathcurveto{\pgfqpoint{1.493068in}{1.735467in}}{\pgfqpoint{1.500968in}{1.738740in}}{\pgfqpoint{1.506792in}{1.744563in}}%
\pgfpathcurveto{\pgfqpoint{1.512616in}{1.750387in}}{\pgfqpoint{1.515888in}{1.758287in}}{\pgfqpoint{1.515888in}{1.766524in}}%
\pgfpathcurveto{\pgfqpoint{1.515888in}{1.774760in}}{\pgfqpoint{1.512616in}{1.782660in}}{\pgfqpoint{1.506792in}{1.788484in}}%
\pgfpathcurveto{\pgfqpoint{1.500968in}{1.794308in}}{\pgfqpoint{1.493068in}{1.797580in}}{\pgfqpoint{1.484832in}{1.797580in}}%
\pgfpathcurveto{\pgfqpoint{1.476596in}{1.797580in}}{\pgfqpoint{1.468696in}{1.794308in}}{\pgfqpoint{1.462872in}{1.788484in}}%
\pgfpathcurveto{\pgfqpoint{1.457048in}{1.782660in}}{\pgfqpoint{1.453775in}{1.774760in}}{\pgfqpoint{1.453775in}{1.766524in}}%
\pgfpathcurveto{\pgfqpoint{1.453775in}{1.758287in}}{\pgfqpoint{1.457048in}{1.750387in}}{\pgfqpoint{1.462872in}{1.744563in}}%
\pgfpathcurveto{\pgfqpoint{1.468696in}{1.738740in}}{\pgfqpoint{1.476596in}{1.735467in}}{\pgfqpoint{1.484832in}{1.735467in}}%
\pgfpathclose%
\pgfusepath{stroke,fill}%
\end{pgfscope}%
\begin{pgfscope}%
\pgfpathrectangle{\pgfqpoint{0.100000in}{0.212622in}}{\pgfqpoint{3.696000in}{3.696000in}}%
\pgfusepath{clip}%
\pgfsetbuttcap%
\pgfsetroundjoin%
\definecolor{currentfill}{rgb}{0.121569,0.466667,0.705882}%
\pgfsetfillcolor{currentfill}%
\pgfsetfillopacity{0.300658}%
\pgfsetlinewidth{1.003750pt}%
\definecolor{currentstroke}{rgb}{0.121569,0.466667,0.705882}%
\pgfsetstrokecolor{currentstroke}%
\pgfsetstrokeopacity{0.300658}%
\pgfsetdash{}{0pt}%
\pgfpathmoveto{\pgfqpoint{1.484435in}{1.735392in}}%
\pgfpathcurveto{\pgfqpoint{1.492671in}{1.735392in}}{\pgfqpoint{1.500571in}{1.738664in}}{\pgfqpoint{1.506395in}{1.744488in}}%
\pgfpathcurveto{\pgfqpoint{1.512219in}{1.750312in}}{\pgfqpoint{1.515491in}{1.758212in}}{\pgfqpoint{1.515491in}{1.766449in}}%
\pgfpathcurveto{\pgfqpoint{1.515491in}{1.774685in}}{\pgfqpoint{1.512219in}{1.782585in}}{\pgfqpoint{1.506395in}{1.788409in}}%
\pgfpathcurveto{\pgfqpoint{1.500571in}{1.794233in}}{\pgfqpoint{1.492671in}{1.797505in}}{\pgfqpoint{1.484435in}{1.797505in}}%
\pgfpathcurveto{\pgfqpoint{1.476198in}{1.797505in}}{\pgfqpoint{1.468298in}{1.794233in}}{\pgfqpoint{1.462474in}{1.788409in}}%
\pgfpathcurveto{\pgfqpoint{1.456651in}{1.782585in}}{\pgfqpoint{1.453378in}{1.774685in}}{\pgfqpoint{1.453378in}{1.766449in}}%
\pgfpathcurveto{\pgfqpoint{1.453378in}{1.758212in}}{\pgfqpoint{1.456651in}{1.750312in}}{\pgfqpoint{1.462474in}{1.744488in}}%
\pgfpathcurveto{\pgfqpoint{1.468298in}{1.738664in}}{\pgfqpoint{1.476198in}{1.735392in}}{\pgfqpoint{1.484435in}{1.735392in}}%
\pgfpathclose%
\pgfusepath{stroke,fill}%
\end{pgfscope}%
\begin{pgfscope}%
\pgfpathrectangle{\pgfqpoint{0.100000in}{0.212622in}}{\pgfqpoint{3.696000in}{3.696000in}}%
\pgfusepath{clip}%
\pgfsetbuttcap%
\pgfsetroundjoin%
\definecolor{currentfill}{rgb}{0.121569,0.466667,0.705882}%
\pgfsetfillcolor{currentfill}%
\pgfsetfillopacity{0.300754}%
\pgfsetlinewidth{1.003750pt}%
\definecolor{currentstroke}{rgb}{0.121569,0.466667,0.705882}%
\pgfsetstrokecolor{currentstroke}%
\pgfsetstrokeopacity{0.300754}%
\pgfsetdash{}{0pt}%
\pgfpathmoveto{\pgfqpoint{1.487168in}{1.734372in}}%
\pgfpathcurveto{\pgfqpoint{1.495405in}{1.734372in}}{\pgfqpoint{1.503305in}{1.737645in}}{\pgfqpoint{1.509129in}{1.743469in}}%
\pgfpathcurveto{\pgfqpoint{1.514953in}{1.749293in}}{\pgfqpoint{1.518225in}{1.757193in}}{\pgfqpoint{1.518225in}{1.765429in}}%
\pgfpathcurveto{\pgfqpoint{1.518225in}{1.773665in}}{\pgfqpoint{1.514953in}{1.781565in}}{\pgfqpoint{1.509129in}{1.787389in}}%
\pgfpathcurveto{\pgfqpoint{1.503305in}{1.793213in}}{\pgfqpoint{1.495405in}{1.796485in}}{\pgfqpoint{1.487168in}{1.796485in}}%
\pgfpathcurveto{\pgfqpoint{1.478932in}{1.796485in}}{\pgfqpoint{1.471032in}{1.793213in}}{\pgfqpoint{1.465208in}{1.787389in}}%
\pgfpathcurveto{\pgfqpoint{1.459384in}{1.781565in}}{\pgfqpoint{1.456112in}{1.773665in}}{\pgfqpoint{1.456112in}{1.765429in}}%
\pgfpathcurveto{\pgfqpoint{1.456112in}{1.757193in}}{\pgfqpoint{1.459384in}{1.749293in}}{\pgfqpoint{1.465208in}{1.743469in}}%
\pgfpathcurveto{\pgfqpoint{1.471032in}{1.737645in}}{\pgfqpoint{1.478932in}{1.734372in}}{\pgfqpoint{1.487168in}{1.734372in}}%
\pgfpathclose%
\pgfusepath{stroke,fill}%
\end{pgfscope}%
\begin{pgfscope}%
\pgfpathrectangle{\pgfqpoint{0.100000in}{0.212622in}}{\pgfqpoint{3.696000in}{3.696000in}}%
\pgfusepath{clip}%
\pgfsetbuttcap%
\pgfsetroundjoin%
\definecolor{currentfill}{rgb}{0.121569,0.466667,0.705882}%
\pgfsetfillcolor{currentfill}%
\pgfsetfillopacity{0.300909}%
\pgfsetlinewidth{1.003750pt}%
\definecolor{currentstroke}{rgb}{0.121569,0.466667,0.705882}%
\pgfsetstrokecolor{currentstroke}%
\pgfsetstrokeopacity{0.300909}%
\pgfsetdash{}{0pt}%
\pgfpathmoveto{\pgfqpoint{1.483688in}{1.735302in}}%
\pgfpathcurveto{\pgfqpoint{1.491924in}{1.735302in}}{\pgfqpoint{1.499824in}{1.738574in}}{\pgfqpoint{1.505648in}{1.744398in}}%
\pgfpathcurveto{\pgfqpoint{1.511472in}{1.750222in}}{\pgfqpoint{1.514744in}{1.758122in}}{\pgfqpoint{1.514744in}{1.766358in}}%
\pgfpathcurveto{\pgfqpoint{1.514744in}{1.774595in}}{\pgfqpoint{1.511472in}{1.782495in}}{\pgfqpoint{1.505648in}{1.788319in}}%
\pgfpathcurveto{\pgfqpoint{1.499824in}{1.794142in}}{\pgfqpoint{1.491924in}{1.797415in}}{\pgfqpoint{1.483688in}{1.797415in}}%
\pgfpathcurveto{\pgfqpoint{1.475451in}{1.797415in}}{\pgfqpoint{1.467551in}{1.794142in}}{\pgfqpoint{1.461727in}{1.788319in}}%
\pgfpathcurveto{\pgfqpoint{1.455903in}{1.782495in}}{\pgfqpoint{1.452631in}{1.774595in}}{\pgfqpoint{1.452631in}{1.766358in}}%
\pgfpathcurveto{\pgfqpoint{1.452631in}{1.758122in}}{\pgfqpoint{1.455903in}{1.750222in}}{\pgfqpoint{1.461727in}{1.744398in}}%
\pgfpathcurveto{\pgfqpoint{1.467551in}{1.738574in}}{\pgfqpoint{1.475451in}{1.735302in}}{\pgfqpoint{1.483688in}{1.735302in}}%
\pgfpathclose%
\pgfusepath{stroke,fill}%
\end{pgfscope}%
\begin{pgfscope}%
\pgfpathrectangle{\pgfqpoint{0.100000in}{0.212622in}}{\pgfqpoint{3.696000in}{3.696000in}}%
\pgfusepath{clip}%
\pgfsetbuttcap%
\pgfsetroundjoin%
\definecolor{currentfill}{rgb}{0.121569,0.466667,0.705882}%
\pgfsetfillcolor{currentfill}%
\pgfsetfillopacity{0.301003}%
\pgfsetlinewidth{1.003750pt}%
\definecolor{currentstroke}{rgb}{0.121569,0.466667,0.705882}%
\pgfsetstrokecolor{currentstroke}%
\pgfsetstrokeopacity{0.301003}%
\pgfsetdash{}{0pt}%
\pgfpathmoveto{\pgfqpoint{1.487172in}{1.734505in}}%
\pgfpathcurveto{\pgfqpoint{1.495409in}{1.734505in}}{\pgfqpoint{1.503309in}{1.737777in}}{\pgfqpoint{1.509133in}{1.743601in}}%
\pgfpathcurveto{\pgfqpoint{1.514957in}{1.749425in}}{\pgfqpoint{1.518229in}{1.757325in}}{\pgfqpoint{1.518229in}{1.765561in}}%
\pgfpathcurveto{\pgfqpoint{1.518229in}{1.773798in}}{\pgfqpoint{1.514957in}{1.781698in}}{\pgfqpoint{1.509133in}{1.787522in}}%
\pgfpathcurveto{\pgfqpoint{1.503309in}{1.793345in}}{\pgfqpoint{1.495409in}{1.796618in}}{\pgfqpoint{1.487172in}{1.796618in}}%
\pgfpathcurveto{\pgfqpoint{1.478936in}{1.796618in}}{\pgfqpoint{1.471036in}{1.793345in}}{\pgfqpoint{1.465212in}{1.787522in}}%
\pgfpathcurveto{\pgfqpoint{1.459388in}{1.781698in}}{\pgfqpoint{1.456116in}{1.773798in}}{\pgfqpoint{1.456116in}{1.765561in}}%
\pgfpathcurveto{\pgfqpoint{1.456116in}{1.757325in}}{\pgfqpoint{1.459388in}{1.749425in}}{\pgfqpoint{1.465212in}{1.743601in}}%
\pgfpathcurveto{\pgfqpoint{1.471036in}{1.737777in}}{\pgfqpoint{1.478936in}{1.734505in}}{\pgfqpoint{1.487172in}{1.734505in}}%
\pgfpathclose%
\pgfusepath{stroke,fill}%
\end{pgfscope}%
\begin{pgfscope}%
\pgfpathrectangle{\pgfqpoint{0.100000in}{0.212622in}}{\pgfqpoint{3.696000in}{3.696000in}}%
\pgfusepath{clip}%
\pgfsetbuttcap%
\pgfsetroundjoin%
\definecolor{currentfill}{rgb}{0.121569,0.466667,0.705882}%
\pgfsetfillcolor{currentfill}%
\pgfsetfillopacity{0.301109}%
\pgfsetlinewidth{1.003750pt}%
\definecolor{currentstroke}{rgb}{0.121569,0.466667,0.705882}%
\pgfsetstrokecolor{currentstroke}%
\pgfsetstrokeopacity{0.301109}%
\pgfsetdash{}{0pt}%
\pgfpathmoveto{\pgfqpoint{1.487164in}{1.734431in}}%
\pgfpathcurveto{\pgfqpoint{1.495400in}{1.734431in}}{\pgfqpoint{1.503300in}{1.737704in}}{\pgfqpoint{1.509124in}{1.743528in}}%
\pgfpathcurveto{\pgfqpoint{1.514948in}{1.749351in}}{\pgfqpoint{1.518220in}{1.757252in}}{\pgfqpoint{1.518220in}{1.765488in}}%
\pgfpathcurveto{\pgfqpoint{1.518220in}{1.773724in}}{\pgfqpoint{1.514948in}{1.781624in}}{\pgfqpoint{1.509124in}{1.787448in}}%
\pgfpathcurveto{\pgfqpoint{1.503300in}{1.793272in}}{\pgfqpoint{1.495400in}{1.796544in}}{\pgfqpoint{1.487164in}{1.796544in}}%
\pgfpathcurveto{\pgfqpoint{1.478927in}{1.796544in}}{\pgfqpoint{1.471027in}{1.793272in}}{\pgfqpoint{1.465203in}{1.787448in}}%
\pgfpathcurveto{\pgfqpoint{1.459379in}{1.781624in}}{\pgfqpoint{1.456107in}{1.773724in}}{\pgfqpoint{1.456107in}{1.765488in}}%
\pgfpathcurveto{\pgfqpoint{1.456107in}{1.757252in}}{\pgfqpoint{1.459379in}{1.749351in}}{\pgfqpoint{1.465203in}{1.743528in}}%
\pgfpathcurveto{\pgfqpoint{1.471027in}{1.737704in}}{\pgfqpoint{1.478927in}{1.734431in}}{\pgfqpoint{1.487164in}{1.734431in}}%
\pgfpathclose%
\pgfusepath{stroke,fill}%
\end{pgfscope}%
\begin{pgfscope}%
\pgfpathrectangle{\pgfqpoint{0.100000in}{0.212622in}}{\pgfqpoint{3.696000in}{3.696000in}}%
\pgfusepath{clip}%
\pgfsetbuttcap%
\pgfsetroundjoin%
\definecolor{currentfill}{rgb}{0.121569,0.466667,0.705882}%
\pgfsetfillcolor{currentfill}%
\pgfsetfillopacity{0.301422}%
\pgfsetlinewidth{1.003750pt}%
\definecolor{currentstroke}{rgb}{0.121569,0.466667,0.705882}%
\pgfsetstrokecolor{currentstroke}%
\pgfsetstrokeopacity{0.301422}%
\pgfsetdash{}{0pt}%
\pgfpathmoveto{\pgfqpoint{1.482442in}{1.735167in}}%
\pgfpathcurveto{\pgfqpoint{1.490678in}{1.735167in}}{\pgfqpoint{1.498578in}{1.738439in}}{\pgfqpoint{1.504402in}{1.744263in}}%
\pgfpathcurveto{\pgfqpoint{1.510226in}{1.750087in}}{\pgfqpoint{1.513498in}{1.757987in}}{\pgfqpoint{1.513498in}{1.766223in}}%
\pgfpathcurveto{\pgfqpoint{1.513498in}{1.774460in}}{\pgfqpoint{1.510226in}{1.782360in}}{\pgfqpoint{1.504402in}{1.788183in}}%
\pgfpathcurveto{\pgfqpoint{1.498578in}{1.794007in}}{\pgfqpoint{1.490678in}{1.797280in}}{\pgfqpoint{1.482442in}{1.797280in}}%
\pgfpathcurveto{\pgfqpoint{1.474205in}{1.797280in}}{\pgfqpoint{1.466305in}{1.794007in}}{\pgfqpoint{1.460481in}{1.788183in}}%
\pgfpathcurveto{\pgfqpoint{1.454657in}{1.782360in}}{\pgfqpoint{1.451385in}{1.774460in}}{\pgfqpoint{1.451385in}{1.766223in}}%
\pgfpathcurveto{\pgfqpoint{1.451385in}{1.757987in}}{\pgfqpoint{1.454657in}{1.750087in}}{\pgfqpoint{1.460481in}{1.744263in}}%
\pgfpathcurveto{\pgfqpoint{1.466305in}{1.738439in}}{\pgfqpoint{1.474205in}{1.735167in}}{\pgfqpoint{1.482442in}{1.735167in}}%
\pgfpathclose%
\pgfusepath{stroke,fill}%
\end{pgfscope}%
\begin{pgfscope}%
\pgfpathrectangle{\pgfqpoint{0.100000in}{0.212622in}}{\pgfqpoint{3.696000in}{3.696000in}}%
\pgfusepath{clip}%
\pgfsetbuttcap%
\pgfsetroundjoin%
\definecolor{currentfill}{rgb}{0.121569,0.466667,0.705882}%
\pgfsetfillcolor{currentfill}%
\pgfsetfillopacity{0.301475}%
\pgfsetlinewidth{1.003750pt}%
\definecolor{currentstroke}{rgb}{0.121569,0.466667,0.705882}%
\pgfsetstrokecolor{currentstroke}%
\pgfsetstrokeopacity{0.301475}%
\pgfsetdash{}{0pt}%
\pgfpathmoveto{\pgfqpoint{1.487131in}{1.734563in}}%
\pgfpathcurveto{\pgfqpoint{1.495367in}{1.734563in}}{\pgfqpoint{1.503267in}{1.737836in}}{\pgfqpoint{1.509091in}{1.743660in}}%
\pgfpathcurveto{\pgfqpoint{1.514915in}{1.749484in}}{\pgfqpoint{1.518187in}{1.757384in}}{\pgfqpoint{1.518187in}{1.765620in}}%
\pgfpathcurveto{\pgfqpoint{1.518187in}{1.773856in}}{\pgfqpoint{1.514915in}{1.781756in}}{\pgfqpoint{1.509091in}{1.787580in}}%
\pgfpathcurveto{\pgfqpoint{1.503267in}{1.793404in}}{\pgfqpoint{1.495367in}{1.796676in}}{\pgfqpoint{1.487131in}{1.796676in}}%
\pgfpathcurveto{\pgfqpoint{1.478894in}{1.796676in}}{\pgfqpoint{1.470994in}{1.793404in}}{\pgfqpoint{1.465170in}{1.787580in}}%
\pgfpathcurveto{\pgfqpoint{1.459346in}{1.781756in}}{\pgfqpoint{1.456074in}{1.773856in}}{\pgfqpoint{1.456074in}{1.765620in}}%
\pgfpathcurveto{\pgfqpoint{1.456074in}{1.757384in}}{\pgfqpoint{1.459346in}{1.749484in}}{\pgfqpoint{1.465170in}{1.743660in}}%
\pgfpathcurveto{\pgfqpoint{1.470994in}{1.737836in}}{\pgfqpoint{1.478894in}{1.734563in}}{\pgfqpoint{1.487131in}{1.734563in}}%
\pgfpathclose%
\pgfusepath{stroke,fill}%
\end{pgfscope}%
\begin{pgfscope}%
\pgfpathrectangle{\pgfqpoint{0.100000in}{0.212622in}}{\pgfqpoint{3.696000in}{3.696000in}}%
\pgfusepath{clip}%
\pgfsetbuttcap%
\pgfsetroundjoin%
\definecolor{currentfill}{rgb}{0.121569,0.466667,0.705882}%
\pgfsetfillcolor{currentfill}%
\pgfsetfillopacity{0.301661}%
\pgfsetlinewidth{1.003750pt}%
\definecolor{currentstroke}{rgb}{0.121569,0.466667,0.705882}%
\pgfsetstrokecolor{currentstroke}%
\pgfsetstrokeopacity{0.301661}%
\pgfsetdash{}{0pt}%
\pgfpathmoveto{\pgfqpoint{1.487082in}{1.734564in}}%
\pgfpathcurveto{\pgfqpoint{1.495318in}{1.734564in}}{\pgfqpoint{1.503218in}{1.737837in}}{\pgfqpoint{1.509042in}{1.743661in}}%
\pgfpathcurveto{\pgfqpoint{1.514866in}{1.749485in}}{\pgfqpoint{1.518138in}{1.757385in}}{\pgfqpoint{1.518138in}{1.765621in}}%
\pgfpathcurveto{\pgfqpoint{1.518138in}{1.773857in}}{\pgfqpoint{1.514866in}{1.781757in}}{\pgfqpoint{1.509042in}{1.787581in}}%
\pgfpathcurveto{\pgfqpoint{1.503218in}{1.793405in}}{\pgfqpoint{1.495318in}{1.796677in}}{\pgfqpoint{1.487082in}{1.796677in}}%
\pgfpathcurveto{\pgfqpoint{1.478845in}{1.796677in}}{\pgfqpoint{1.470945in}{1.793405in}}{\pgfqpoint{1.465121in}{1.787581in}}%
\pgfpathcurveto{\pgfqpoint{1.459297in}{1.781757in}}{\pgfqpoint{1.456025in}{1.773857in}}{\pgfqpoint{1.456025in}{1.765621in}}%
\pgfpathcurveto{\pgfqpoint{1.456025in}{1.757385in}}{\pgfqpoint{1.459297in}{1.749485in}}{\pgfqpoint{1.465121in}{1.743661in}}%
\pgfpathcurveto{\pgfqpoint{1.470945in}{1.737837in}}{\pgfqpoint{1.478845in}{1.734564in}}{\pgfqpoint{1.487082in}{1.734564in}}%
\pgfpathclose%
\pgfusepath{stroke,fill}%
\end{pgfscope}%
\begin{pgfscope}%
\pgfpathrectangle{\pgfqpoint{0.100000in}{0.212622in}}{\pgfqpoint{3.696000in}{3.696000in}}%
\pgfusepath{clip}%
\pgfsetbuttcap%
\pgfsetroundjoin%
\definecolor{currentfill}{rgb}{0.121569,0.466667,0.705882}%
\pgfsetfillcolor{currentfill}%
\pgfsetfillopacity{0.301819}%
\pgfsetlinewidth{1.003750pt}%
\definecolor{currentstroke}{rgb}{0.121569,0.466667,0.705882}%
\pgfsetstrokecolor{currentstroke}%
\pgfsetstrokeopacity{0.301819}%
\pgfsetdash{}{0pt}%
\pgfpathmoveto{\pgfqpoint{1.481408in}{1.735098in}}%
\pgfpathcurveto{\pgfqpoint{1.489645in}{1.735098in}}{\pgfqpoint{1.497545in}{1.738370in}}{\pgfqpoint{1.503369in}{1.744194in}}%
\pgfpathcurveto{\pgfqpoint{1.509193in}{1.750018in}}{\pgfqpoint{1.512465in}{1.757918in}}{\pgfqpoint{1.512465in}{1.766154in}}%
\pgfpathcurveto{\pgfqpoint{1.512465in}{1.774391in}}{\pgfqpoint{1.509193in}{1.782291in}}{\pgfqpoint{1.503369in}{1.788115in}}%
\pgfpathcurveto{\pgfqpoint{1.497545in}{1.793939in}}{\pgfqpoint{1.489645in}{1.797211in}}{\pgfqpoint{1.481408in}{1.797211in}}%
\pgfpathcurveto{\pgfqpoint{1.473172in}{1.797211in}}{\pgfqpoint{1.465272in}{1.793939in}}{\pgfqpoint{1.459448in}{1.788115in}}%
\pgfpathcurveto{\pgfqpoint{1.453624in}{1.782291in}}{\pgfqpoint{1.450352in}{1.774391in}}{\pgfqpoint{1.450352in}{1.766154in}}%
\pgfpathcurveto{\pgfqpoint{1.450352in}{1.757918in}}{\pgfqpoint{1.453624in}{1.750018in}}{\pgfqpoint{1.459448in}{1.744194in}}%
\pgfpathcurveto{\pgfqpoint{1.465272in}{1.738370in}}{\pgfqpoint{1.473172in}{1.735098in}}{\pgfqpoint{1.481408in}{1.735098in}}%
\pgfpathclose%
\pgfusepath{stroke,fill}%
\end{pgfscope}%
\begin{pgfscope}%
\pgfpathrectangle{\pgfqpoint{0.100000in}{0.212622in}}{\pgfqpoint{3.696000in}{3.696000in}}%
\pgfusepath{clip}%
\pgfsetbuttcap%
\pgfsetroundjoin%
\definecolor{currentfill}{rgb}{0.121569,0.466667,0.705882}%
\pgfsetfillcolor{currentfill}%
\pgfsetfillopacity{0.302077}%
\pgfsetlinewidth{1.003750pt}%
\definecolor{currentstroke}{rgb}{0.121569,0.466667,0.705882}%
\pgfsetstrokecolor{currentstroke}%
\pgfsetstrokeopacity{0.302077}%
\pgfsetdash{}{0pt}%
\pgfpathmoveto{\pgfqpoint{1.480714in}{1.734988in}}%
\pgfpathcurveto{\pgfqpoint{1.488950in}{1.734988in}}{\pgfqpoint{1.496850in}{1.738260in}}{\pgfqpoint{1.502674in}{1.744084in}}%
\pgfpathcurveto{\pgfqpoint{1.508498in}{1.749908in}}{\pgfqpoint{1.511770in}{1.757808in}}{\pgfqpoint{1.511770in}{1.766045in}}%
\pgfpathcurveto{\pgfqpoint{1.511770in}{1.774281in}}{\pgfqpoint{1.508498in}{1.782181in}}{\pgfqpoint{1.502674in}{1.788005in}}%
\pgfpathcurveto{\pgfqpoint{1.496850in}{1.793829in}}{\pgfqpoint{1.488950in}{1.797101in}}{\pgfqpoint{1.480714in}{1.797101in}}%
\pgfpathcurveto{\pgfqpoint{1.472477in}{1.797101in}}{\pgfqpoint{1.464577in}{1.793829in}}{\pgfqpoint{1.458753in}{1.788005in}}%
\pgfpathcurveto{\pgfqpoint{1.452929in}{1.782181in}}{\pgfqpoint{1.449657in}{1.774281in}}{\pgfqpoint{1.449657in}{1.766045in}}%
\pgfpathcurveto{\pgfqpoint{1.449657in}{1.757808in}}{\pgfqpoint{1.452929in}{1.749908in}}{\pgfqpoint{1.458753in}{1.744084in}}%
\pgfpathcurveto{\pgfqpoint{1.464577in}{1.738260in}}{\pgfqpoint{1.472477in}{1.734988in}}{\pgfqpoint{1.480714in}{1.734988in}}%
\pgfpathclose%
\pgfusepath{stroke,fill}%
\end{pgfscope}%
\begin{pgfscope}%
\pgfpathrectangle{\pgfqpoint{0.100000in}{0.212622in}}{\pgfqpoint{3.696000in}{3.696000in}}%
\pgfusepath{clip}%
\pgfsetbuttcap%
\pgfsetroundjoin%
\definecolor{currentfill}{rgb}{0.121569,0.466667,0.705882}%
\pgfsetfillcolor{currentfill}%
\pgfsetfillopacity{0.302142}%
\pgfsetlinewidth{1.003750pt}%
\definecolor{currentstroke}{rgb}{0.121569,0.466667,0.705882}%
\pgfsetstrokecolor{currentstroke}%
\pgfsetstrokeopacity{0.302142}%
\pgfsetdash{}{0pt}%
\pgfpathmoveto{\pgfqpoint{1.486974in}{1.735314in}}%
\pgfpathcurveto{\pgfqpoint{1.495211in}{1.735314in}}{\pgfqpoint{1.503111in}{1.738586in}}{\pgfqpoint{1.508935in}{1.744410in}}%
\pgfpathcurveto{\pgfqpoint{1.514758in}{1.750234in}}{\pgfqpoint{1.518031in}{1.758134in}}{\pgfqpoint{1.518031in}{1.766370in}}%
\pgfpathcurveto{\pgfqpoint{1.518031in}{1.774606in}}{\pgfqpoint{1.514758in}{1.782507in}}{\pgfqpoint{1.508935in}{1.788330in}}%
\pgfpathcurveto{\pgfqpoint{1.503111in}{1.794154in}}{\pgfqpoint{1.495211in}{1.797427in}}{\pgfqpoint{1.486974in}{1.797427in}}%
\pgfpathcurveto{\pgfqpoint{1.478738in}{1.797427in}}{\pgfqpoint{1.470838in}{1.794154in}}{\pgfqpoint{1.465014in}{1.788330in}}%
\pgfpathcurveto{\pgfqpoint{1.459190in}{1.782507in}}{\pgfqpoint{1.455918in}{1.774606in}}{\pgfqpoint{1.455918in}{1.766370in}}%
\pgfpathcurveto{\pgfqpoint{1.455918in}{1.758134in}}{\pgfqpoint{1.459190in}{1.750234in}}{\pgfqpoint{1.465014in}{1.744410in}}%
\pgfpathcurveto{\pgfqpoint{1.470838in}{1.738586in}}{\pgfqpoint{1.478738in}{1.735314in}}{\pgfqpoint{1.486974in}{1.735314in}}%
\pgfpathclose%
\pgfusepath{stroke,fill}%
\end{pgfscope}%
\begin{pgfscope}%
\pgfpathrectangle{\pgfqpoint{0.100000in}{0.212622in}}{\pgfqpoint{3.696000in}{3.696000in}}%
\pgfusepath{clip}%
\pgfsetbuttcap%
\pgfsetroundjoin%
\definecolor{currentfill}{rgb}{0.121569,0.466667,0.705882}%
\pgfsetfillcolor{currentfill}%
\pgfsetfillopacity{0.302454}%
\pgfsetlinewidth{1.003750pt}%
\definecolor{currentstroke}{rgb}{0.121569,0.466667,0.705882}%
\pgfsetstrokecolor{currentstroke}%
\pgfsetstrokeopacity{0.302454}%
\pgfsetdash{}{0pt}%
\pgfpathmoveto{\pgfqpoint{1.479516in}{1.734236in}}%
\pgfpathcurveto{\pgfqpoint{1.487753in}{1.734236in}}{\pgfqpoint{1.495653in}{1.737508in}}{\pgfqpoint{1.501477in}{1.743332in}}%
\pgfpathcurveto{\pgfqpoint{1.507300in}{1.749156in}}{\pgfqpoint{1.510573in}{1.757056in}}{\pgfqpoint{1.510573in}{1.765292in}}%
\pgfpathcurveto{\pgfqpoint{1.510573in}{1.773528in}}{\pgfqpoint{1.507300in}{1.781428in}}{\pgfqpoint{1.501477in}{1.787252in}}%
\pgfpathcurveto{\pgfqpoint{1.495653in}{1.793076in}}{\pgfqpoint{1.487753in}{1.796349in}}{\pgfqpoint{1.479516in}{1.796349in}}%
\pgfpathcurveto{\pgfqpoint{1.471280in}{1.796349in}}{\pgfqpoint{1.463380in}{1.793076in}}{\pgfqpoint{1.457556in}{1.787252in}}%
\pgfpathcurveto{\pgfqpoint{1.451732in}{1.781428in}}{\pgfqpoint{1.448460in}{1.773528in}}{\pgfqpoint{1.448460in}{1.765292in}}%
\pgfpathcurveto{\pgfqpoint{1.448460in}{1.757056in}}{\pgfqpoint{1.451732in}{1.749156in}}{\pgfqpoint{1.457556in}{1.743332in}}%
\pgfpathcurveto{\pgfqpoint{1.463380in}{1.737508in}}{\pgfqpoint{1.471280in}{1.734236in}}{\pgfqpoint{1.479516in}{1.734236in}}%
\pgfpathclose%
\pgfusepath{stroke,fill}%
\end{pgfscope}%
\begin{pgfscope}%
\pgfpathrectangle{\pgfqpoint{0.100000in}{0.212622in}}{\pgfqpoint{3.696000in}{3.696000in}}%
\pgfusepath{clip}%
\pgfsetbuttcap%
\pgfsetroundjoin%
\definecolor{currentfill}{rgb}{0.121569,0.466667,0.705882}%
\pgfsetfillcolor{currentfill}%
\pgfsetfillopacity{0.302617}%
\pgfsetlinewidth{1.003750pt}%
\definecolor{currentstroke}{rgb}{0.121569,0.466667,0.705882}%
\pgfsetstrokecolor{currentstroke}%
\pgfsetstrokeopacity{0.302617}%
\pgfsetdash{}{0pt}%
\pgfpathmoveto{\pgfqpoint{1.487289in}{1.733228in}}%
\pgfpathcurveto{\pgfqpoint{1.495525in}{1.733228in}}{\pgfqpoint{1.503425in}{1.736501in}}{\pgfqpoint{1.509249in}{1.742325in}}%
\pgfpathcurveto{\pgfqpoint{1.515073in}{1.748149in}}{\pgfqpoint{1.518346in}{1.756049in}}{\pgfqpoint{1.518346in}{1.764285in}}%
\pgfpathcurveto{\pgfqpoint{1.518346in}{1.772521in}}{\pgfqpoint{1.515073in}{1.780421in}}{\pgfqpoint{1.509249in}{1.786245in}}%
\pgfpathcurveto{\pgfqpoint{1.503425in}{1.792069in}}{\pgfqpoint{1.495525in}{1.795341in}}{\pgfqpoint{1.487289in}{1.795341in}}%
\pgfpathcurveto{\pgfqpoint{1.479053in}{1.795341in}}{\pgfqpoint{1.471153in}{1.792069in}}{\pgfqpoint{1.465329in}{1.786245in}}%
\pgfpathcurveto{\pgfqpoint{1.459505in}{1.780421in}}{\pgfqpoint{1.456233in}{1.772521in}}{\pgfqpoint{1.456233in}{1.764285in}}%
\pgfpathcurveto{\pgfqpoint{1.456233in}{1.756049in}}{\pgfqpoint{1.459505in}{1.748149in}}{\pgfqpoint{1.465329in}{1.742325in}}%
\pgfpathcurveto{\pgfqpoint{1.471153in}{1.736501in}}{\pgfqpoint{1.479053in}{1.733228in}}{\pgfqpoint{1.487289in}{1.733228in}}%
\pgfpathclose%
\pgfusepath{stroke,fill}%
\end{pgfscope}%
\begin{pgfscope}%
\pgfpathrectangle{\pgfqpoint{0.100000in}{0.212622in}}{\pgfqpoint{3.696000in}{3.696000in}}%
\pgfusepath{clip}%
\pgfsetbuttcap%
\pgfsetroundjoin%
\definecolor{currentfill}{rgb}{0.121569,0.466667,0.705882}%
\pgfsetfillcolor{currentfill}%
\pgfsetfillopacity{0.303546}%
\pgfsetlinewidth{1.003750pt}%
\definecolor{currentstroke}{rgb}{0.121569,0.466667,0.705882}%
\pgfsetstrokecolor{currentstroke}%
\pgfsetstrokeopacity{0.303546}%
\pgfsetdash{}{0pt}%
\pgfpathmoveto{\pgfqpoint{1.487901in}{1.731446in}}%
\pgfpathcurveto{\pgfqpoint{1.496138in}{1.731446in}}{\pgfqpoint{1.504038in}{1.734719in}}{\pgfqpoint{1.509862in}{1.740543in}}%
\pgfpathcurveto{\pgfqpoint{1.515686in}{1.746367in}}{\pgfqpoint{1.518958in}{1.754267in}}{\pgfqpoint{1.518958in}{1.762503in}}%
\pgfpathcurveto{\pgfqpoint{1.518958in}{1.770739in}}{\pgfqpoint{1.515686in}{1.778639in}}{\pgfqpoint{1.509862in}{1.784463in}}%
\pgfpathcurveto{\pgfqpoint{1.504038in}{1.790287in}}{\pgfqpoint{1.496138in}{1.793559in}}{\pgfqpoint{1.487901in}{1.793559in}}%
\pgfpathcurveto{\pgfqpoint{1.479665in}{1.793559in}}{\pgfqpoint{1.471765in}{1.790287in}}{\pgfqpoint{1.465941in}{1.784463in}}%
\pgfpathcurveto{\pgfqpoint{1.460117in}{1.778639in}}{\pgfqpoint{1.456845in}{1.770739in}}{\pgfqpoint{1.456845in}{1.762503in}}%
\pgfpathcurveto{\pgfqpoint{1.456845in}{1.754267in}}{\pgfqpoint{1.460117in}{1.746367in}}{\pgfqpoint{1.465941in}{1.740543in}}%
\pgfpathcurveto{\pgfqpoint{1.471765in}{1.734719in}}{\pgfqpoint{1.479665in}{1.731446in}}{\pgfqpoint{1.487901in}{1.731446in}}%
\pgfpathclose%
\pgfusepath{stroke,fill}%
\end{pgfscope}%
\begin{pgfscope}%
\pgfpathrectangle{\pgfqpoint{0.100000in}{0.212622in}}{\pgfqpoint{3.696000in}{3.696000in}}%
\pgfusepath{clip}%
\pgfsetbuttcap%
\pgfsetroundjoin%
\definecolor{currentfill}{rgb}{0.121569,0.466667,0.705882}%
\pgfsetfillcolor{currentfill}%
\pgfsetfillopacity{0.303655}%
\pgfsetlinewidth{1.003750pt}%
\definecolor{currentstroke}{rgb}{0.121569,0.466667,0.705882}%
\pgfsetstrokecolor{currentstroke}%
\pgfsetstrokeopacity{0.303655}%
\pgfsetdash{}{0pt}%
\pgfpathmoveto{\pgfqpoint{1.477339in}{1.735257in}}%
\pgfpathcurveto{\pgfqpoint{1.485576in}{1.735257in}}{\pgfqpoint{1.493476in}{1.738530in}}{\pgfqpoint{1.499300in}{1.744354in}}%
\pgfpathcurveto{\pgfqpoint{1.505124in}{1.750177in}}{\pgfqpoint{1.508396in}{1.758077in}}{\pgfqpoint{1.508396in}{1.766314in}}%
\pgfpathcurveto{\pgfqpoint{1.508396in}{1.774550in}}{\pgfqpoint{1.505124in}{1.782450in}}{\pgfqpoint{1.499300in}{1.788274in}}%
\pgfpathcurveto{\pgfqpoint{1.493476in}{1.794098in}}{\pgfqpoint{1.485576in}{1.797370in}}{\pgfqpoint{1.477339in}{1.797370in}}%
\pgfpathcurveto{\pgfqpoint{1.469103in}{1.797370in}}{\pgfqpoint{1.461203in}{1.794098in}}{\pgfqpoint{1.455379in}{1.788274in}}%
\pgfpathcurveto{\pgfqpoint{1.449555in}{1.782450in}}{\pgfqpoint{1.446283in}{1.774550in}}{\pgfqpoint{1.446283in}{1.766314in}}%
\pgfpathcurveto{\pgfqpoint{1.446283in}{1.758077in}}{\pgfqpoint{1.449555in}{1.750177in}}{\pgfqpoint{1.455379in}{1.744354in}}%
\pgfpathcurveto{\pgfqpoint{1.461203in}{1.738530in}}{\pgfqpoint{1.469103in}{1.735257in}}{\pgfqpoint{1.477339in}{1.735257in}}%
\pgfpathclose%
\pgfusepath{stroke,fill}%
\end{pgfscope}%
\begin{pgfscope}%
\pgfpathrectangle{\pgfqpoint{0.100000in}{0.212622in}}{\pgfqpoint{3.696000in}{3.696000in}}%
\pgfusepath{clip}%
\pgfsetbuttcap%
\pgfsetroundjoin%
\definecolor{currentfill}{rgb}{0.121569,0.466667,0.705882}%
\pgfsetfillcolor{currentfill}%
\pgfsetfillopacity{0.305499}%
\pgfsetlinewidth{1.003750pt}%
\definecolor{currentstroke}{rgb}{0.121569,0.466667,0.705882}%
\pgfsetstrokecolor{currentstroke}%
\pgfsetstrokeopacity{0.305499}%
\pgfsetdash{}{0pt}%
\pgfpathmoveto{\pgfqpoint{1.473505in}{1.735319in}}%
\pgfpathcurveto{\pgfqpoint{1.481741in}{1.735319in}}{\pgfqpoint{1.489641in}{1.738592in}}{\pgfqpoint{1.495465in}{1.744416in}}%
\pgfpathcurveto{\pgfqpoint{1.501289in}{1.750240in}}{\pgfqpoint{1.504562in}{1.758140in}}{\pgfqpoint{1.504562in}{1.766376in}}%
\pgfpathcurveto{\pgfqpoint{1.504562in}{1.774612in}}{\pgfqpoint{1.501289in}{1.782512in}}{\pgfqpoint{1.495465in}{1.788336in}}%
\pgfpathcurveto{\pgfqpoint{1.489641in}{1.794160in}}{\pgfqpoint{1.481741in}{1.797432in}}{\pgfqpoint{1.473505in}{1.797432in}}%
\pgfpathcurveto{\pgfqpoint{1.465269in}{1.797432in}}{\pgfqpoint{1.457369in}{1.794160in}}{\pgfqpoint{1.451545in}{1.788336in}}%
\pgfpathcurveto{\pgfqpoint{1.445721in}{1.782512in}}{\pgfqpoint{1.442449in}{1.774612in}}{\pgfqpoint{1.442449in}{1.766376in}}%
\pgfpathcurveto{\pgfqpoint{1.442449in}{1.758140in}}{\pgfqpoint{1.445721in}{1.750240in}}{\pgfqpoint{1.451545in}{1.744416in}}%
\pgfpathcurveto{\pgfqpoint{1.457369in}{1.738592in}}{\pgfqpoint{1.465269in}{1.735319in}}{\pgfqpoint{1.473505in}{1.735319in}}%
\pgfpathclose%
\pgfusepath{stroke,fill}%
\end{pgfscope}%
\begin{pgfscope}%
\pgfpathrectangle{\pgfqpoint{0.100000in}{0.212622in}}{\pgfqpoint{3.696000in}{3.696000in}}%
\pgfusepath{clip}%
\pgfsetbuttcap%
\pgfsetroundjoin%
\definecolor{currentfill}{rgb}{0.121569,0.466667,0.705882}%
\pgfsetfillcolor{currentfill}%
\pgfsetfillopacity{0.305700}%
\pgfsetlinewidth{1.003750pt}%
\definecolor{currentstroke}{rgb}{0.121569,0.466667,0.705882}%
\pgfsetstrokecolor{currentstroke}%
\pgfsetstrokeopacity{0.305700}%
\pgfsetdash{}{0pt}%
\pgfpathmoveto{\pgfqpoint{1.488876in}{1.734862in}}%
\pgfpathcurveto{\pgfqpoint{1.497112in}{1.734862in}}{\pgfqpoint{1.505012in}{1.738134in}}{\pgfqpoint{1.510836in}{1.743958in}}%
\pgfpathcurveto{\pgfqpoint{1.516660in}{1.749782in}}{\pgfqpoint{1.519932in}{1.757682in}}{\pgfqpoint{1.519932in}{1.765918in}}%
\pgfpathcurveto{\pgfqpoint{1.519932in}{1.774155in}}{\pgfqpoint{1.516660in}{1.782055in}}{\pgfqpoint{1.510836in}{1.787879in}}%
\pgfpathcurveto{\pgfqpoint{1.505012in}{1.793703in}}{\pgfqpoint{1.497112in}{1.796975in}}{\pgfqpoint{1.488876in}{1.796975in}}%
\pgfpathcurveto{\pgfqpoint{1.480639in}{1.796975in}}{\pgfqpoint{1.472739in}{1.793703in}}{\pgfqpoint{1.466915in}{1.787879in}}%
\pgfpathcurveto{\pgfqpoint{1.461092in}{1.782055in}}{\pgfqpoint{1.457819in}{1.774155in}}{\pgfqpoint{1.457819in}{1.765918in}}%
\pgfpathcurveto{\pgfqpoint{1.457819in}{1.757682in}}{\pgfqpoint{1.461092in}{1.749782in}}{\pgfqpoint{1.466915in}{1.743958in}}%
\pgfpathcurveto{\pgfqpoint{1.472739in}{1.738134in}}{\pgfqpoint{1.480639in}{1.734862in}}{\pgfqpoint{1.488876in}{1.734862in}}%
\pgfpathclose%
\pgfusepath{stroke,fill}%
\end{pgfscope}%
\begin{pgfscope}%
\pgfpathrectangle{\pgfqpoint{0.100000in}{0.212622in}}{\pgfqpoint{3.696000in}{3.696000in}}%
\pgfusepath{clip}%
\pgfsetbuttcap%
\pgfsetroundjoin%
\definecolor{currentfill}{rgb}{0.121569,0.466667,0.705882}%
\pgfsetfillcolor{currentfill}%
\pgfsetfillopacity{0.307309}%
\pgfsetlinewidth{1.003750pt}%
\definecolor{currentstroke}{rgb}{0.121569,0.466667,0.705882}%
\pgfsetstrokecolor{currentstroke}%
\pgfsetstrokeopacity{0.307309}%
\pgfsetdash{}{0pt}%
\pgfpathmoveto{\pgfqpoint{1.470443in}{1.735808in}}%
\pgfpathcurveto{\pgfqpoint{1.478680in}{1.735808in}}{\pgfqpoint{1.486580in}{1.739080in}}{\pgfqpoint{1.492404in}{1.744904in}}%
\pgfpathcurveto{\pgfqpoint{1.498228in}{1.750728in}}{\pgfqpoint{1.501500in}{1.758628in}}{\pgfqpoint{1.501500in}{1.766865in}}%
\pgfpathcurveto{\pgfqpoint{1.501500in}{1.775101in}}{\pgfqpoint{1.498228in}{1.783001in}}{\pgfqpoint{1.492404in}{1.788825in}}%
\pgfpathcurveto{\pgfqpoint{1.486580in}{1.794649in}}{\pgfqpoint{1.478680in}{1.797921in}}{\pgfqpoint{1.470443in}{1.797921in}}%
\pgfpathcurveto{\pgfqpoint{1.462207in}{1.797921in}}{\pgfqpoint{1.454307in}{1.794649in}}{\pgfqpoint{1.448483in}{1.788825in}}%
\pgfpathcurveto{\pgfqpoint{1.442659in}{1.783001in}}{\pgfqpoint{1.439387in}{1.775101in}}{\pgfqpoint{1.439387in}{1.766865in}}%
\pgfpathcurveto{\pgfqpoint{1.439387in}{1.758628in}}{\pgfqpoint{1.442659in}{1.750728in}}{\pgfqpoint{1.448483in}{1.744904in}}%
\pgfpathcurveto{\pgfqpoint{1.454307in}{1.739080in}}{\pgfqpoint{1.462207in}{1.735808in}}{\pgfqpoint{1.470443in}{1.735808in}}%
\pgfpathclose%
\pgfusepath{stroke,fill}%
\end{pgfscope}%
\begin{pgfscope}%
\pgfpathrectangle{\pgfqpoint{0.100000in}{0.212622in}}{\pgfqpoint{3.696000in}{3.696000in}}%
\pgfusepath{clip}%
\pgfsetbuttcap%
\pgfsetroundjoin%
\definecolor{currentfill}{rgb}{0.121569,0.466667,0.705882}%
\pgfsetfillcolor{currentfill}%
\pgfsetfillopacity{0.307963}%
\pgfsetlinewidth{1.003750pt}%
\definecolor{currentstroke}{rgb}{0.121569,0.466667,0.705882}%
\pgfsetstrokecolor{currentstroke}%
\pgfsetstrokeopacity{0.307963}%
\pgfsetdash{}{0pt}%
\pgfpathmoveto{\pgfqpoint{1.489334in}{1.732742in}}%
\pgfpathcurveto{\pgfqpoint{1.497570in}{1.732742in}}{\pgfqpoint{1.505470in}{1.736014in}}{\pgfqpoint{1.511294in}{1.741838in}}%
\pgfpathcurveto{\pgfqpoint{1.517118in}{1.747662in}}{\pgfqpoint{1.520390in}{1.755562in}}{\pgfqpoint{1.520390in}{1.763799in}}%
\pgfpathcurveto{\pgfqpoint{1.520390in}{1.772035in}}{\pgfqpoint{1.517118in}{1.779935in}}{\pgfqpoint{1.511294in}{1.785759in}}%
\pgfpathcurveto{\pgfqpoint{1.505470in}{1.791583in}}{\pgfqpoint{1.497570in}{1.794855in}}{\pgfqpoint{1.489334in}{1.794855in}}%
\pgfpathcurveto{\pgfqpoint{1.481098in}{1.794855in}}{\pgfqpoint{1.473198in}{1.791583in}}{\pgfqpoint{1.467374in}{1.785759in}}%
\pgfpathcurveto{\pgfqpoint{1.461550in}{1.779935in}}{\pgfqpoint{1.458277in}{1.772035in}}{\pgfqpoint{1.458277in}{1.763799in}}%
\pgfpathcurveto{\pgfqpoint{1.458277in}{1.755562in}}{\pgfqpoint{1.461550in}{1.747662in}}{\pgfqpoint{1.467374in}{1.741838in}}%
\pgfpathcurveto{\pgfqpoint{1.473198in}{1.736014in}}{\pgfqpoint{1.481098in}{1.732742in}}{\pgfqpoint{1.489334in}{1.732742in}}%
\pgfpathclose%
\pgfusepath{stroke,fill}%
\end{pgfscope}%
\begin{pgfscope}%
\pgfpathrectangle{\pgfqpoint{0.100000in}{0.212622in}}{\pgfqpoint{3.696000in}{3.696000in}}%
\pgfusepath{clip}%
\pgfsetbuttcap%
\pgfsetroundjoin%
\definecolor{currentfill}{rgb}{0.121569,0.466667,0.705882}%
\pgfsetfillcolor{currentfill}%
\pgfsetfillopacity{0.308589}%
\pgfsetlinewidth{1.003750pt}%
\definecolor{currentstroke}{rgb}{0.121569,0.466667,0.705882}%
\pgfsetstrokecolor{currentstroke}%
\pgfsetstrokeopacity{0.308589}%
\pgfsetdash{}{0pt}%
\pgfpathmoveto{\pgfqpoint{1.467369in}{1.734992in}}%
\pgfpathcurveto{\pgfqpoint{1.475606in}{1.734992in}}{\pgfqpoint{1.483506in}{1.738264in}}{\pgfqpoint{1.489330in}{1.744088in}}%
\pgfpathcurveto{\pgfqpoint{1.495153in}{1.749912in}}{\pgfqpoint{1.498426in}{1.757812in}}{\pgfqpoint{1.498426in}{1.766048in}}%
\pgfpathcurveto{\pgfqpoint{1.498426in}{1.774285in}}{\pgfqpoint{1.495153in}{1.782185in}}{\pgfqpoint{1.489330in}{1.788009in}}%
\pgfpathcurveto{\pgfqpoint{1.483506in}{1.793833in}}{\pgfqpoint{1.475606in}{1.797105in}}{\pgfqpoint{1.467369in}{1.797105in}}%
\pgfpathcurveto{\pgfqpoint{1.459133in}{1.797105in}}{\pgfqpoint{1.451233in}{1.793833in}}{\pgfqpoint{1.445409in}{1.788009in}}%
\pgfpathcurveto{\pgfqpoint{1.439585in}{1.782185in}}{\pgfqpoint{1.436313in}{1.774285in}}{\pgfqpoint{1.436313in}{1.766048in}}%
\pgfpathcurveto{\pgfqpoint{1.436313in}{1.757812in}}{\pgfqpoint{1.439585in}{1.749912in}}{\pgfqpoint{1.445409in}{1.744088in}}%
\pgfpathcurveto{\pgfqpoint{1.451233in}{1.738264in}}{\pgfqpoint{1.459133in}{1.734992in}}{\pgfqpoint{1.467369in}{1.734992in}}%
\pgfpathclose%
\pgfusepath{stroke,fill}%
\end{pgfscope}%
\begin{pgfscope}%
\pgfpathrectangle{\pgfqpoint{0.100000in}{0.212622in}}{\pgfqpoint{3.696000in}{3.696000in}}%
\pgfusepath{clip}%
\pgfsetbuttcap%
\pgfsetroundjoin%
\definecolor{currentfill}{rgb}{0.121569,0.466667,0.705882}%
\pgfsetfillcolor{currentfill}%
\pgfsetfillopacity{0.309128}%
\pgfsetlinewidth{1.003750pt}%
\definecolor{currentstroke}{rgb}{0.121569,0.466667,0.705882}%
\pgfsetstrokecolor{currentstroke}%
\pgfsetstrokeopacity{0.309128}%
\pgfsetdash{}{0pt}%
\pgfpathmoveto{\pgfqpoint{1.465771in}{1.734249in}}%
\pgfpathcurveto{\pgfqpoint{1.474007in}{1.734249in}}{\pgfqpoint{1.481907in}{1.737521in}}{\pgfqpoint{1.487731in}{1.743345in}}%
\pgfpathcurveto{\pgfqpoint{1.493555in}{1.749169in}}{\pgfqpoint{1.496828in}{1.757069in}}{\pgfqpoint{1.496828in}{1.765305in}}%
\pgfpathcurveto{\pgfqpoint{1.496828in}{1.773542in}}{\pgfqpoint{1.493555in}{1.781442in}}{\pgfqpoint{1.487731in}{1.787266in}}%
\pgfpathcurveto{\pgfqpoint{1.481907in}{1.793090in}}{\pgfqpoint{1.474007in}{1.796362in}}{\pgfqpoint{1.465771in}{1.796362in}}%
\pgfpathcurveto{\pgfqpoint{1.457535in}{1.796362in}}{\pgfqpoint{1.449635in}{1.793090in}}{\pgfqpoint{1.443811in}{1.787266in}}%
\pgfpathcurveto{\pgfqpoint{1.437987in}{1.781442in}}{\pgfqpoint{1.434715in}{1.773542in}}{\pgfqpoint{1.434715in}{1.765305in}}%
\pgfpathcurveto{\pgfqpoint{1.434715in}{1.757069in}}{\pgfqpoint{1.437987in}{1.749169in}}{\pgfqpoint{1.443811in}{1.743345in}}%
\pgfpathcurveto{\pgfqpoint{1.449635in}{1.737521in}}{\pgfqpoint{1.457535in}{1.734249in}}{\pgfqpoint{1.465771in}{1.734249in}}%
\pgfpathclose%
\pgfusepath{stroke,fill}%
\end{pgfscope}%
\begin{pgfscope}%
\pgfpathrectangle{\pgfqpoint{0.100000in}{0.212622in}}{\pgfqpoint{3.696000in}{3.696000in}}%
\pgfusepath{clip}%
\pgfsetbuttcap%
\pgfsetroundjoin%
\definecolor{currentfill}{rgb}{0.121569,0.466667,0.705882}%
\pgfsetfillcolor{currentfill}%
\pgfsetfillopacity{0.310097}%
\pgfsetlinewidth{1.003750pt}%
\definecolor{currentstroke}{rgb}{0.121569,0.466667,0.705882}%
\pgfsetstrokecolor{currentstroke}%
\pgfsetstrokeopacity{0.310097}%
\pgfsetdash{}{0pt}%
\pgfpathmoveto{\pgfqpoint{1.462967in}{1.732651in}}%
\pgfpathcurveto{\pgfqpoint{1.471203in}{1.732651in}}{\pgfqpoint{1.479103in}{1.735923in}}{\pgfqpoint{1.484927in}{1.741747in}}%
\pgfpathcurveto{\pgfqpoint{1.490751in}{1.747571in}}{\pgfqpoint{1.494024in}{1.755471in}}{\pgfqpoint{1.494024in}{1.763707in}}%
\pgfpathcurveto{\pgfqpoint{1.494024in}{1.771943in}}{\pgfqpoint{1.490751in}{1.779844in}}{\pgfqpoint{1.484927in}{1.785667in}}%
\pgfpathcurveto{\pgfqpoint{1.479103in}{1.791491in}}{\pgfqpoint{1.471203in}{1.794764in}}{\pgfqpoint{1.462967in}{1.794764in}}%
\pgfpathcurveto{\pgfqpoint{1.454731in}{1.794764in}}{\pgfqpoint{1.446831in}{1.791491in}}{\pgfqpoint{1.441007in}{1.785667in}}%
\pgfpathcurveto{\pgfqpoint{1.435183in}{1.779844in}}{\pgfqpoint{1.431911in}{1.771943in}}{\pgfqpoint{1.431911in}{1.763707in}}%
\pgfpathcurveto{\pgfqpoint{1.431911in}{1.755471in}}{\pgfqpoint{1.435183in}{1.747571in}}{\pgfqpoint{1.441007in}{1.741747in}}%
\pgfpathcurveto{\pgfqpoint{1.446831in}{1.735923in}}{\pgfqpoint{1.454731in}{1.732651in}}{\pgfqpoint{1.462967in}{1.732651in}}%
\pgfpathclose%
\pgfusepath{stroke,fill}%
\end{pgfscope}%
\begin{pgfscope}%
\pgfpathrectangle{\pgfqpoint{0.100000in}{0.212622in}}{\pgfqpoint{3.696000in}{3.696000in}}%
\pgfusepath{clip}%
\pgfsetbuttcap%
\pgfsetroundjoin%
\definecolor{currentfill}{rgb}{0.121569,0.466667,0.705882}%
\pgfsetfillcolor{currentfill}%
\pgfsetfillopacity{0.310417}%
\pgfsetlinewidth{1.003750pt}%
\definecolor{currentstroke}{rgb}{0.121569,0.466667,0.705882}%
\pgfsetstrokecolor{currentstroke}%
\pgfsetstrokeopacity{0.310417}%
\pgfsetdash{}{0pt}%
\pgfpathmoveto{\pgfqpoint{1.491236in}{1.730271in}}%
\pgfpathcurveto{\pgfqpoint{1.499473in}{1.730271in}}{\pgfqpoint{1.507373in}{1.733543in}}{\pgfqpoint{1.513197in}{1.739367in}}%
\pgfpathcurveto{\pgfqpoint{1.519020in}{1.745191in}}{\pgfqpoint{1.522293in}{1.753091in}}{\pgfqpoint{1.522293in}{1.761327in}}%
\pgfpathcurveto{\pgfqpoint{1.522293in}{1.769563in}}{\pgfqpoint{1.519020in}{1.777463in}}{\pgfqpoint{1.513197in}{1.783287in}}%
\pgfpathcurveto{\pgfqpoint{1.507373in}{1.789111in}}{\pgfqpoint{1.499473in}{1.792384in}}{\pgfqpoint{1.491236in}{1.792384in}}%
\pgfpathcurveto{\pgfqpoint{1.483000in}{1.792384in}}{\pgfqpoint{1.475100in}{1.789111in}}{\pgfqpoint{1.469276in}{1.783287in}}%
\pgfpathcurveto{\pgfqpoint{1.463452in}{1.777463in}}{\pgfqpoint{1.460180in}{1.769563in}}{\pgfqpoint{1.460180in}{1.761327in}}%
\pgfpathcurveto{\pgfqpoint{1.460180in}{1.753091in}}{\pgfqpoint{1.463452in}{1.745191in}}{\pgfqpoint{1.469276in}{1.739367in}}%
\pgfpathcurveto{\pgfqpoint{1.475100in}{1.733543in}}{\pgfqpoint{1.483000in}{1.730271in}}{\pgfqpoint{1.491236in}{1.730271in}}%
\pgfpathclose%
\pgfusepath{stroke,fill}%
\end{pgfscope}%
\begin{pgfscope}%
\pgfpathrectangle{\pgfqpoint{0.100000in}{0.212622in}}{\pgfqpoint{3.696000in}{3.696000in}}%
\pgfusepath{clip}%
\pgfsetbuttcap%
\pgfsetroundjoin%
\definecolor{currentfill}{rgb}{0.121569,0.466667,0.705882}%
\pgfsetfillcolor{currentfill}%
\pgfsetfillopacity{0.312426}%
\pgfsetlinewidth{1.003750pt}%
\definecolor{currentstroke}{rgb}{0.121569,0.466667,0.705882}%
\pgfsetstrokecolor{currentstroke}%
\pgfsetstrokeopacity{0.312426}%
\pgfsetdash{}{0pt}%
\pgfpathmoveto{\pgfqpoint{1.492235in}{1.732058in}}%
\pgfpathcurveto{\pgfqpoint{1.500471in}{1.732058in}}{\pgfqpoint{1.508371in}{1.735330in}}{\pgfqpoint{1.514195in}{1.741154in}}%
\pgfpathcurveto{\pgfqpoint{1.520019in}{1.746978in}}{\pgfqpoint{1.523291in}{1.754878in}}{\pgfqpoint{1.523291in}{1.763115in}}%
\pgfpathcurveto{\pgfqpoint{1.523291in}{1.771351in}}{\pgfqpoint{1.520019in}{1.779251in}}{\pgfqpoint{1.514195in}{1.785075in}}%
\pgfpathcurveto{\pgfqpoint{1.508371in}{1.790899in}}{\pgfqpoint{1.500471in}{1.794171in}}{\pgfqpoint{1.492235in}{1.794171in}}%
\pgfpathcurveto{\pgfqpoint{1.483998in}{1.794171in}}{\pgfqpoint{1.476098in}{1.790899in}}{\pgfqpoint{1.470274in}{1.785075in}}%
\pgfpathcurveto{\pgfqpoint{1.464451in}{1.779251in}}{\pgfqpoint{1.461178in}{1.771351in}}{\pgfqpoint{1.461178in}{1.763115in}}%
\pgfpathcurveto{\pgfqpoint{1.461178in}{1.754878in}}{\pgfqpoint{1.464451in}{1.746978in}}{\pgfqpoint{1.470274in}{1.741154in}}%
\pgfpathcurveto{\pgfqpoint{1.476098in}{1.735330in}}{\pgfqpoint{1.483998in}{1.732058in}}{\pgfqpoint{1.492235in}{1.732058in}}%
\pgfpathclose%
\pgfusepath{stroke,fill}%
\end{pgfscope}%
\begin{pgfscope}%
\pgfpathrectangle{\pgfqpoint{0.100000in}{0.212622in}}{\pgfqpoint{3.696000in}{3.696000in}}%
\pgfusepath{clip}%
\pgfsetbuttcap%
\pgfsetroundjoin%
\definecolor{currentfill}{rgb}{0.121569,0.466667,0.705882}%
\pgfsetfillcolor{currentfill}%
\pgfsetfillopacity{0.312797}%
\pgfsetlinewidth{1.003750pt}%
\definecolor{currentstroke}{rgb}{0.121569,0.466667,0.705882}%
\pgfsetstrokecolor{currentstroke}%
\pgfsetstrokeopacity{0.312797}%
\pgfsetdash{}{0pt}%
\pgfpathmoveto{\pgfqpoint{1.458321in}{1.733385in}}%
\pgfpathcurveto{\pgfqpoint{1.466558in}{1.733385in}}{\pgfqpoint{1.474458in}{1.736657in}}{\pgfqpoint{1.480282in}{1.742481in}}%
\pgfpathcurveto{\pgfqpoint{1.486106in}{1.748305in}}{\pgfqpoint{1.489378in}{1.756205in}}{\pgfqpoint{1.489378in}{1.764442in}}%
\pgfpathcurveto{\pgfqpoint{1.489378in}{1.772678in}}{\pgfqpoint{1.486106in}{1.780578in}}{\pgfqpoint{1.480282in}{1.786402in}}%
\pgfpathcurveto{\pgfqpoint{1.474458in}{1.792226in}}{\pgfqpoint{1.466558in}{1.795498in}}{\pgfqpoint{1.458321in}{1.795498in}}%
\pgfpathcurveto{\pgfqpoint{1.450085in}{1.795498in}}{\pgfqpoint{1.442185in}{1.792226in}}{\pgfqpoint{1.436361in}{1.786402in}}%
\pgfpathcurveto{\pgfqpoint{1.430537in}{1.780578in}}{\pgfqpoint{1.427265in}{1.772678in}}{\pgfqpoint{1.427265in}{1.764442in}}%
\pgfpathcurveto{\pgfqpoint{1.427265in}{1.756205in}}{\pgfqpoint{1.430537in}{1.748305in}}{\pgfqpoint{1.436361in}{1.742481in}}%
\pgfpathcurveto{\pgfqpoint{1.442185in}{1.736657in}}{\pgfqpoint{1.450085in}{1.733385in}}{\pgfqpoint{1.458321in}{1.733385in}}%
\pgfpathclose%
\pgfusepath{stroke,fill}%
\end{pgfscope}%
\begin{pgfscope}%
\pgfpathrectangle{\pgfqpoint{0.100000in}{0.212622in}}{\pgfqpoint{3.696000in}{3.696000in}}%
\pgfusepath{clip}%
\pgfsetbuttcap%
\pgfsetroundjoin%
\definecolor{currentfill}{rgb}{0.121569,0.466667,0.705882}%
\pgfsetfillcolor{currentfill}%
\pgfsetfillopacity{0.313749}%
\pgfsetlinewidth{1.003750pt}%
\definecolor{currentstroke}{rgb}{0.121569,0.466667,0.705882}%
\pgfsetstrokecolor{currentstroke}%
\pgfsetstrokeopacity{0.313749}%
\pgfsetdash{}{0pt}%
\pgfpathmoveto{\pgfqpoint{1.455243in}{1.731664in}}%
\pgfpathcurveto{\pgfqpoint{1.463480in}{1.731664in}}{\pgfqpoint{1.471380in}{1.734936in}}{\pgfqpoint{1.477204in}{1.740760in}}%
\pgfpathcurveto{\pgfqpoint{1.483028in}{1.746584in}}{\pgfqpoint{1.486300in}{1.754484in}}{\pgfqpoint{1.486300in}{1.762720in}}%
\pgfpathcurveto{\pgfqpoint{1.486300in}{1.770957in}}{\pgfqpoint{1.483028in}{1.778857in}}{\pgfqpoint{1.477204in}{1.784681in}}%
\pgfpathcurveto{\pgfqpoint{1.471380in}{1.790504in}}{\pgfqpoint{1.463480in}{1.793777in}}{\pgfqpoint{1.455243in}{1.793777in}}%
\pgfpathcurveto{\pgfqpoint{1.447007in}{1.793777in}}{\pgfqpoint{1.439107in}{1.790504in}}{\pgfqpoint{1.433283in}{1.784681in}}%
\pgfpathcurveto{\pgfqpoint{1.427459in}{1.778857in}}{\pgfqpoint{1.424187in}{1.770957in}}{\pgfqpoint{1.424187in}{1.762720in}}%
\pgfpathcurveto{\pgfqpoint{1.424187in}{1.754484in}}{\pgfqpoint{1.427459in}{1.746584in}}{\pgfqpoint{1.433283in}{1.740760in}}%
\pgfpathcurveto{\pgfqpoint{1.439107in}{1.734936in}}{\pgfqpoint{1.447007in}{1.731664in}}{\pgfqpoint{1.455243in}{1.731664in}}%
\pgfpathclose%
\pgfusepath{stroke,fill}%
\end{pgfscope}%
\begin{pgfscope}%
\pgfpathrectangle{\pgfqpoint{0.100000in}{0.212622in}}{\pgfqpoint{3.696000in}{3.696000in}}%
\pgfusepath{clip}%
\pgfsetbuttcap%
\pgfsetroundjoin%
\definecolor{currentfill}{rgb}{0.121569,0.466667,0.705882}%
\pgfsetfillcolor{currentfill}%
\pgfsetfillopacity{0.314419}%
\pgfsetlinewidth{1.003750pt}%
\definecolor{currentstroke}{rgb}{0.121569,0.466667,0.705882}%
\pgfsetstrokecolor{currentstroke}%
\pgfsetstrokeopacity{0.314419}%
\pgfsetdash{}{0pt}%
\pgfpathmoveto{\pgfqpoint{1.492606in}{1.730272in}}%
\pgfpathcurveto{\pgfqpoint{1.500843in}{1.730272in}}{\pgfqpoint{1.508743in}{1.733544in}}{\pgfqpoint{1.514567in}{1.739368in}}%
\pgfpathcurveto{\pgfqpoint{1.520391in}{1.745192in}}{\pgfqpoint{1.523663in}{1.753092in}}{\pgfqpoint{1.523663in}{1.761328in}}%
\pgfpathcurveto{\pgfqpoint{1.523663in}{1.769564in}}{\pgfqpoint{1.520391in}{1.777464in}}{\pgfqpoint{1.514567in}{1.783288in}}%
\pgfpathcurveto{\pgfqpoint{1.508743in}{1.789112in}}{\pgfqpoint{1.500843in}{1.792385in}}{\pgfqpoint{1.492606in}{1.792385in}}%
\pgfpathcurveto{\pgfqpoint{1.484370in}{1.792385in}}{\pgfqpoint{1.476470in}{1.789112in}}{\pgfqpoint{1.470646in}{1.783288in}}%
\pgfpathcurveto{\pgfqpoint{1.464822in}{1.777464in}}{\pgfqpoint{1.461550in}{1.769564in}}{\pgfqpoint{1.461550in}{1.761328in}}%
\pgfpathcurveto{\pgfqpoint{1.461550in}{1.753092in}}{\pgfqpoint{1.464822in}{1.745192in}}{\pgfqpoint{1.470646in}{1.739368in}}%
\pgfpathcurveto{\pgfqpoint{1.476470in}{1.733544in}}{\pgfqpoint{1.484370in}{1.730272in}}{\pgfqpoint{1.492606in}{1.730272in}}%
\pgfpathclose%
\pgfusepath{stroke,fill}%
\end{pgfscope}%
\begin{pgfscope}%
\pgfpathrectangle{\pgfqpoint{0.100000in}{0.212622in}}{\pgfqpoint{3.696000in}{3.696000in}}%
\pgfusepath{clip}%
\pgfsetbuttcap%
\pgfsetroundjoin%
\definecolor{currentfill}{rgb}{0.121569,0.466667,0.705882}%
\pgfsetfillcolor{currentfill}%
\pgfsetfillopacity{0.314534}%
\pgfsetlinewidth{1.003750pt}%
\definecolor{currentstroke}{rgb}{0.121569,0.466667,0.705882}%
\pgfsetstrokecolor{currentstroke}%
\pgfsetstrokeopacity{0.314534}%
\pgfsetdash{}{0pt}%
\pgfpathmoveto{\pgfqpoint{1.452614in}{1.729480in}}%
\pgfpathcurveto{\pgfqpoint{1.460850in}{1.729480in}}{\pgfqpoint{1.468750in}{1.732752in}}{\pgfqpoint{1.474574in}{1.738576in}}%
\pgfpathcurveto{\pgfqpoint{1.480398in}{1.744400in}}{\pgfqpoint{1.483670in}{1.752300in}}{\pgfqpoint{1.483670in}{1.760536in}}%
\pgfpathcurveto{\pgfqpoint{1.483670in}{1.768772in}}{\pgfqpoint{1.480398in}{1.776672in}}{\pgfqpoint{1.474574in}{1.782496in}}%
\pgfpathcurveto{\pgfqpoint{1.468750in}{1.788320in}}{\pgfqpoint{1.460850in}{1.791593in}}{\pgfqpoint{1.452614in}{1.791593in}}%
\pgfpathcurveto{\pgfqpoint{1.444378in}{1.791593in}}{\pgfqpoint{1.436478in}{1.788320in}}{\pgfqpoint{1.430654in}{1.782496in}}%
\pgfpathcurveto{\pgfqpoint{1.424830in}{1.776672in}}{\pgfqpoint{1.421557in}{1.768772in}}{\pgfqpoint{1.421557in}{1.760536in}}%
\pgfpathcurveto{\pgfqpoint{1.421557in}{1.752300in}}{\pgfqpoint{1.424830in}{1.744400in}}{\pgfqpoint{1.430654in}{1.738576in}}%
\pgfpathcurveto{\pgfqpoint{1.436478in}{1.732752in}}{\pgfqpoint{1.444378in}{1.729480in}}{\pgfqpoint{1.452614in}{1.729480in}}%
\pgfpathclose%
\pgfusepath{stroke,fill}%
\end{pgfscope}%
\begin{pgfscope}%
\pgfpathrectangle{\pgfqpoint{0.100000in}{0.212622in}}{\pgfqpoint{3.696000in}{3.696000in}}%
\pgfusepath{clip}%
\pgfsetbuttcap%
\pgfsetroundjoin%
\definecolor{currentfill}{rgb}{0.121569,0.466667,0.705882}%
\pgfsetfillcolor{currentfill}%
\pgfsetfillopacity{0.315672}%
\pgfsetlinewidth{1.003750pt}%
\definecolor{currentstroke}{rgb}{0.121569,0.466667,0.705882}%
\pgfsetstrokecolor{currentstroke}%
\pgfsetstrokeopacity{0.315672}%
\pgfsetdash{}{0pt}%
\pgfpathmoveto{\pgfqpoint{1.493361in}{1.730251in}}%
\pgfpathcurveto{\pgfqpoint{1.501597in}{1.730251in}}{\pgfqpoint{1.509497in}{1.733524in}}{\pgfqpoint{1.515321in}{1.739348in}}%
\pgfpathcurveto{\pgfqpoint{1.521145in}{1.745172in}}{\pgfqpoint{1.524417in}{1.753072in}}{\pgfqpoint{1.524417in}{1.761308in}}%
\pgfpathcurveto{\pgfqpoint{1.524417in}{1.769544in}}{\pgfqpoint{1.521145in}{1.777444in}}{\pgfqpoint{1.515321in}{1.783268in}}%
\pgfpathcurveto{\pgfqpoint{1.509497in}{1.789092in}}{\pgfqpoint{1.501597in}{1.792364in}}{\pgfqpoint{1.493361in}{1.792364in}}%
\pgfpathcurveto{\pgfqpoint{1.485125in}{1.792364in}}{\pgfqpoint{1.477224in}{1.789092in}}{\pgfqpoint{1.471401in}{1.783268in}}%
\pgfpathcurveto{\pgfqpoint{1.465577in}{1.777444in}}{\pgfqpoint{1.462304in}{1.769544in}}{\pgfqpoint{1.462304in}{1.761308in}}%
\pgfpathcurveto{\pgfqpoint{1.462304in}{1.753072in}}{\pgfqpoint{1.465577in}{1.745172in}}{\pgfqpoint{1.471401in}{1.739348in}}%
\pgfpathcurveto{\pgfqpoint{1.477224in}{1.733524in}}{\pgfqpoint{1.485125in}{1.730251in}}{\pgfqpoint{1.493361in}{1.730251in}}%
\pgfpathclose%
\pgfusepath{stroke,fill}%
\end{pgfscope}%
\begin{pgfscope}%
\pgfpathrectangle{\pgfqpoint{0.100000in}{0.212622in}}{\pgfqpoint{3.696000in}{3.696000in}}%
\pgfusepath{clip}%
\pgfsetbuttcap%
\pgfsetroundjoin%
\definecolor{currentfill}{rgb}{0.121569,0.466667,0.705882}%
\pgfsetfillcolor{currentfill}%
\pgfsetfillopacity{0.316516}%
\pgfsetlinewidth{1.003750pt}%
\definecolor{currentstroke}{rgb}{0.121569,0.466667,0.705882}%
\pgfsetstrokecolor{currentstroke}%
\pgfsetstrokeopacity{0.316516}%
\pgfsetdash{}{0pt}%
\pgfpathmoveto{\pgfqpoint{1.448548in}{1.726998in}}%
\pgfpathcurveto{\pgfqpoint{1.456784in}{1.726998in}}{\pgfqpoint{1.464684in}{1.730270in}}{\pgfqpoint{1.470508in}{1.736094in}}%
\pgfpathcurveto{\pgfqpoint{1.476332in}{1.741918in}}{\pgfqpoint{1.479605in}{1.749818in}}{\pgfqpoint{1.479605in}{1.758054in}}%
\pgfpathcurveto{\pgfqpoint{1.479605in}{1.766290in}}{\pgfqpoint{1.476332in}{1.774190in}}{\pgfqpoint{1.470508in}{1.780014in}}%
\pgfpathcurveto{\pgfqpoint{1.464684in}{1.785838in}}{\pgfqpoint{1.456784in}{1.789111in}}{\pgfqpoint{1.448548in}{1.789111in}}%
\pgfpathcurveto{\pgfqpoint{1.440312in}{1.789111in}}{\pgfqpoint{1.432412in}{1.785838in}}{\pgfqpoint{1.426588in}{1.780014in}}%
\pgfpathcurveto{\pgfqpoint{1.420764in}{1.774190in}}{\pgfqpoint{1.417492in}{1.766290in}}{\pgfqpoint{1.417492in}{1.758054in}}%
\pgfpathcurveto{\pgfqpoint{1.417492in}{1.749818in}}{\pgfqpoint{1.420764in}{1.741918in}}{\pgfqpoint{1.426588in}{1.736094in}}%
\pgfpathcurveto{\pgfqpoint{1.432412in}{1.730270in}}{\pgfqpoint{1.440312in}{1.726998in}}{\pgfqpoint{1.448548in}{1.726998in}}%
\pgfpathclose%
\pgfusepath{stroke,fill}%
\end{pgfscope}%
\begin{pgfscope}%
\pgfpathrectangle{\pgfqpoint{0.100000in}{0.212622in}}{\pgfqpoint{3.696000in}{3.696000in}}%
\pgfusepath{clip}%
\pgfsetbuttcap%
\pgfsetroundjoin%
\definecolor{currentfill}{rgb}{0.121569,0.466667,0.705882}%
\pgfsetfillcolor{currentfill}%
\pgfsetfillopacity{0.317430}%
\pgfsetlinewidth{1.003750pt}%
\definecolor{currentstroke}{rgb}{0.121569,0.466667,0.705882}%
\pgfsetstrokecolor{currentstroke}%
\pgfsetstrokeopacity{0.317430}%
\pgfsetdash{}{0pt}%
\pgfpathmoveto{\pgfqpoint{1.445343in}{1.724224in}}%
\pgfpathcurveto{\pgfqpoint{1.453579in}{1.724224in}}{\pgfqpoint{1.461479in}{1.727496in}}{\pgfqpoint{1.467303in}{1.733320in}}%
\pgfpathcurveto{\pgfqpoint{1.473127in}{1.739144in}}{\pgfqpoint{1.476399in}{1.747044in}}{\pgfqpoint{1.476399in}{1.755281in}}%
\pgfpathcurveto{\pgfqpoint{1.476399in}{1.763517in}}{\pgfqpoint{1.473127in}{1.771417in}}{\pgfqpoint{1.467303in}{1.777241in}}%
\pgfpathcurveto{\pgfqpoint{1.461479in}{1.783065in}}{\pgfqpoint{1.453579in}{1.786337in}}{\pgfqpoint{1.445343in}{1.786337in}}%
\pgfpathcurveto{\pgfqpoint{1.437107in}{1.786337in}}{\pgfqpoint{1.429207in}{1.783065in}}{\pgfqpoint{1.423383in}{1.777241in}}%
\pgfpathcurveto{\pgfqpoint{1.417559in}{1.771417in}}{\pgfqpoint{1.414286in}{1.763517in}}{\pgfqpoint{1.414286in}{1.755281in}}%
\pgfpathcurveto{\pgfqpoint{1.414286in}{1.747044in}}{\pgfqpoint{1.417559in}{1.739144in}}{\pgfqpoint{1.423383in}{1.733320in}}%
\pgfpathcurveto{\pgfqpoint{1.429207in}{1.727496in}}{\pgfqpoint{1.437107in}{1.724224in}}{\pgfqpoint{1.445343in}{1.724224in}}%
\pgfpathclose%
\pgfusepath{stroke,fill}%
\end{pgfscope}%
\begin{pgfscope}%
\pgfpathrectangle{\pgfqpoint{0.100000in}{0.212622in}}{\pgfqpoint{3.696000in}{3.696000in}}%
\pgfusepath{clip}%
\pgfsetbuttcap%
\pgfsetroundjoin%
\definecolor{currentfill}{rgb}{0.121569,0.466667,0.705882}%
\pgfsetfillcolor{currentfill}%
\pgfsetfillopacity{0.317853}%
\pgfsetlinewidth{1.003750pt}%
\definecolor{currentstroke}{rgb}{0.121569,0.466667,0.705882}%
\pgfsetstrokecolor{currentstroke}%
\pgfsetstrokeopacity{0.317853}%
\pgfsetdash{}{0pt}%
\pgfpathmoveto{\pgfqpoint{1.494621in}{1.732838in}}%
\pgfpathcurveto{\pgfqpoint{1.502857in}{1.732838in}}{\pgfqpoint{1.510757in}{1.736111in}}{\pgfqpoint{1.516581in}{1.741935in}}%
\pgfpathcurveto{\pgfqpoint{1.522405in}{1.747759in}}{\pgfqpoint{1.525677in}{1.755659in}}{\pgfqpoint{1.525677in}{1.763895in}}%
\pgfpathcurveto{\pgfqpoint{1.525677in}{1.772131in}}{\pgfqpoint{1.522405in}{1.780031in}}{\pgfqpoint{1.516581in}{1.785855in}}%
\pgfpathcurveto{\pgfqpoint{1.510757in}{1.791679in}}{\pgfqpoint{1.502857in}{1.794951in}}{\pgfqpoint{1.494621in}{1.794951in}}%
\pgfpathcurveto{\pgfqpoint{1.486384in}{1.794951in}}{\pgfqpoint{1.478484in}{1.791679in}}{\pgfqpoint{1.472660in}{1.785855in}}%
\pgfpathcurveto{\pgfqpoint{1.466836in}{1.780031in}}{\pgfqpoint{1.463564in}{1.772131in}}{\pgfqpoint{1.463564in}{1.763895in}}%
\pgfpathcurveto{\pgfqpoint{1.463564in}{1.755659in}}{\pgfqpoint{1.466836in}{1.747759in}}{\pgfqpoint{1.472660in}{1.741935in}}%
\pgfpathcurveto{\pgfqpoint{1.478484in}{1.736111in}}{\pgfqpoint{1.486384in}{1.732838in}}{\pgfqpoint{1.494621in}{1.732838in}}%
\pgfpathclose%
\pgfusepath{stroke,fill}%
\end{pgfscope}%
\begin{pgfscope}%
\pgfpathrectangle{\pgfqpoint{0.100000in}{0.212622in}}{\pgfqpoint{3.696000in}{3.696000in}}%
\pgfusepath{clip}%
\pgfsetbuttcap%
\pgfsetroundjoin%
\definecolor{currentfill}{rgb}{0.121569,0.466667,0.705882}%
\pgfsetfillcolor{currentfill}%
\pgfsetfillopacity{0.318688}%
\pgfsetlinewidth{1.003750pt}%
\definecolor{currentstroke}{rgb}{0.121569,0.466667,0.705882}%
\pgfsetstrokecolor{currentstroke}%
\pgfsetstrokeopacity{0.318688}%
\pgfsetdash{}{0pt}%
\pgfpathmoveto{\pgfqpoint{1.442558in}{1.723254in}}%
\pgfpathcurveto{\pgfqpoint{1.450795in}{1.723254in}}{\pgfqpoint{1.458695in}{1.726527in}}{\pgfqpoint{1.464518in}{1.732351in}}%
\pgfpathcurveto{\pgfqpoint{1.470342in}{1.738175in}}{\pgfqpoint{1.473615in}{1.746075in}}{\pgfqpoint{1.473615in}{1.754311in}}%
\pgfpathcurveto{\pgfqpoint{1.473615in}{1.762547in}}{\pgfqpoint{1.470342in}{1.770447in}}{\pgfqpoint{1.464518in}{1.776271in}}%
\pgfpathcurveto{\pgfqpoint{1.458695in}{1.782095in}}{\pgfqpoint{1.450795in}{1.785367in}}{\pgfqpoint{1.442558in}{1.785367in}}%
\pgfpathcurveto{\pgfqpoint{1.434322in}{1.785367in}}{\pgfqpoint{1.426422in}{1.782095in}}{\pgfqpoint{1.420598in}{1.776271in}}%
\pgfpathcurveto{\pgfqpoint{1.414774in}{1.770447in}}{\pgfqpoint{1.411502in}{1.762547in}}{\pgfqpoint{1.411502in}{1.754311in}}%
\pgfpathcurveto{\pgfqpoint{1.411502in}{1.746075in}}{\pgfqpoint{1.414774in}{1.738175in}}{\pgfqpoint{1.420598in}{1.732351in}}%
\pgfpathcurveto{\pgfqpoint{1.426422in}{1.726527in}}{\pgfqpoint{1.434322in}{1.723254in}}{\pgfqpoint{1.442558in}{1.723254in}}%
\pgfpathclose%
\pgfusepath{stroke,fill}%
\end{pgfscope}%
\begin{pgfscope}%
\pgfpathrectangle{\pgfqpoint{0.100000in}{0.212622in}}{\pgfqpoint{3.696000in}{3.696000in}}%
\pgfusepath{clip}%
\pgfsetbuttcap%
\pgfsetroundjoin%
\definecolor{currentfill}{rgb}{0.121569,0.466667,0.705882}%
\pgfsetfillcolor{currentfill}%
\pgfsetfillopacity{0.318756}%
\pgfsetlinewidth{1.003750pt}%
\definecolor{currentstroke}{rgb}{0.121569,0.466667,0.705882}%
\pgfsetstrokecolor{currentstroke}%
\pgfsetstrokeopacity{0.318756}%
\pgfsetdash{}{0pt}%
\pgfpathmoveto{\pgfqpoint{1.494776in}{1.732590in}}%
\pgfpathcurveto{\pgfqpoint{1.503012in}{1.732590in}}{\pgfqpoint{1.510912in}{1.735862in}}{\pgfqpoint{1.516736in}{1.741686in}}%
\pgfpathcurveto{\pgfqpoint{1.522560in}{1.747510in}}{\pgfqpoint{1.525832in}{1.755410in}}{\pgfqpoint{1.525832in}{1.763646in}}%
\pgfpathcurveto{\pgfqpoint{1.525832in}{1.771883in}}{\pgfqpoint{1.522560in}{1.779783in}}{\pgfqpoint{1.516736in}{1.785607in}}%
\pgfpathcurveto{\pgfqpoint{1.510912in}{1.791431in}}{\pgfqpoint{1.503012in}{1.794703in}}{\pgfqpoint{1.494776in}{1.794703in}}%
\pgfpathcurveto{\pgfqpoint{1.486539in}{1.794703in}}{\pgfqpoint{1.478639in}{1.791431in}}{\pgfqpoint{1.472815in}{1.785607in}}%
\pgfpathcurveto{\pgfqpoint{1.466991in}{1.779783in}}{\pgfqpoint{1.463719in}{1.771883in}}{\pgfqpoint{1.463719in}{1.763646in}}%
\pgfpathcurveto{\pgfqpoint{1.463719in}{1.755410in}}{\pgfqpoint{1.466991in}{1.747510in}}{\pgfqpoint{1.472815in}{1.741686in}}%
\pgfpathcurveto{\pgfqpoint{1.478639in}{1.735862in}}{\pgfqpoint{1.486539in}{1.732590in}}{\pgfqpoint{1.494776in}{1.732590in}}%
\pgfpathclose%
\pgfusepath{stroke,fill}%
\end{pgfscope}%
\begin{pgfscope}%
\pgfpathrectangle{\pgfqpoint{0.100000in}{0.212622in}}{\pgfqpoint{3.696000in}{3.696000in}}%
\pgfusepath{clip}%
\pgfsetbuttcap%
\pgfsetroundjoin%
\definecolor{currentfill}{rgb}{0.121569,0.466667,0.705882}%
\pgfsetfillcolor{currentfill}%
\pgfsetfillopacity{0.320399}%
\pgfsetlinewidth{1.003750pt}%
\definecolor{currentstroke}{rgb}{0.121569,0.466667,0.705882}%
\pgfsetstrokecolor{currentstroke}%
\pgfsetstrokeopacity{0.320399}%
\pgfsetdash{}{0pt}%
\pgfpathmoveto{\pgfqpoint{1.495820in}{1.732632in}}%
\pgfpathcurveto{\pgfqpoint{1.504057in}{1.732632in}}{\pgfqpoint{1.511957in}{1.735904in}}{\pgfqpoint{1.517781in}{1.741728in}}%
\pgfpathcurveto{\pgfqpoint{1.523605in}{1.747552in}}{\pgfqpoint{1.526877in}{1.755452in}}{\pgfqpoint{1.526877in}{1.763688in}}%
\pgfpathcurveto{\pgfqpoint{1.526877in}{1.771925in}}{\pgfqpoint{1.523605in}{1.779825in}}{\pgfqpoint{1.517781in}{1.785649in}}%
\pgfpathcurveto{\pgfqpoint{1.511957in}{1.791473in}}{\pgfqpoint{1.504057in}{1.794745in}}{\pgfqpoint{1.495820in}{1.794745in}}%
\pgfpathcurveto{\pgfqpoint{1.487584in}{1.794745in}}{\pgfqpoint{1.479684in}{1.791473in}}{\pgfqpoint{1.473860in}{1.785649in}}%
\pgfpathcurveto{\pgfqpoint{1.468036in}{1.779825in}}{\pgfqpoint{1.464764in}{1.771925in}}{\pgfqpoint{1.464764in}{1.763688in}}%
\pgfpathcurveto{\pgfqpoint{1.464764in}{1.755452in}}{\pgfqpoint{1.468036in}{1.747552in}}{\pgfqpoint{1.473860in}{1.741728in}}%
\pgfpathcurveto{\pgfqpoint{1.479684in}{1.735904in}}{\pgfqpoint{1.487584in}{1.732632in}}{\pgfqpoint{1.495820in}{1.732632in}}%
\pgfpathclose%
\pgfusepath{stroke,fill}%
\end{pgfscope}%
\begin{pgfscope}%
\pgfpathrectangle{\pgfqpoint{0.100000in}{0.212622in}}{\pgfqpoint{3.696000in}{3.696000in}}%
\pgfusepath{clip}%
\pgfsetbuttcap%
\pgfsetroundjoin%
\definecolor{currentfill}{rgb}{0.121569,0.466667,0.705882}%
\pgfsetfillcolor{currentfill}%
\pgfsetfillopacity{0.321238}%
\pgfsetlinewidth{1.003750pt}%
\definecolor{currentstroke}{rgb}{0.121569,0.466667,0.705882}%
\pgfsetstrokecolor{currentstroke}%
\pgfsetstrokeopacity{0.321238}%
\pgfsetdash{}{0pt}%
\pgfpathmoveto{\pgfqpoint{1.438406in}{1.721491in}}%
\pgfpathcurveto{\pgfqpoint{1.446642in}{1.721491in}}{\pgfqpoint{1.454542in}{1.724764in}}{\pgfqpoint{1.460366in}{1.730588in}}%
\pgfpathcurveto{\pgfqpoint{1.466190in}{1.736412in}}{\pgfqpoint{1.469462in}{1.744312in}}{\pgfqpoint{1.469462in}{1.752548in}}%
\pgfpathcurveto{\pgfqpoint{1.469462in}{1.760784in}}{\pgfqpoint{1.466190in}{1.768684in}}{\pgfqpoint{1.460366in}{1.774508in}}%
\pgfpathcurveto{\pgfqpoint{1.454542in}{1.780332in}}{\pgfqpoint{1.446642in}{1.783604in}}{\pgfqpoint{1.438406in}{1.783604in}}%
\pgfpathcurveto{\pgfqpoint{1.430169in}{1.783604in}}{\pgfqpoint{1.422269in}{1.780332in}}{\pgfqpoint{1.416445in}{1.774508in}}%
\pgfpathcurveto{\pgfqpoint{1.410622in}{1.768684in}}{\pgfqpoint{1.407349in}{1.760784in}}{\pgfqpoint{1.407349in}{1.752548in}}%
\pgfpathcurveto{\pgfqpoint{1.407349in}{1.744312in}}{\pgfqpoint{1.410622in}{1.736412in}}{\pgfqpoint{1.416445in}{1.730588in}}%
\pgfpathcurveto{\pgfqpoint{1.422269in}{1.724764in}}{\pgfqpoint{1.430169in}{1.721491in}}{\pgfqpoint{1.438406in}{1.721491in}}%
\pgfpathclose%
\pgfusepath{stroke,fill}%
\end{pgfscope}%
\begin{pgfscope}%
\pgfpathrectangle{\pgfqpoint{0.100000in}{0.212622in}}{\pgfqpoint{3.696000in}{3.696000in}}%
\pgfusepath{clip}%
\pgfsetbuttcap%
\pgfsetroundjoin%
\definecolor{currentfill}{rgb}{0.121569,0.466667,0.705882}%
\pgfsetfillcolor{currentfill}%
\pgfsetfillopacity{0.322235}%
\pgfsetlinewidth{1.003750pt}%
\definecolor{currentstroke}{rgb}{0.121569,0.466667,0.705882}%
\pgfsetstrokecolor{currentstroke}%
\pgfsetstrokeopacity{0.322235}%
\pgfsetdash{}{0pt}%
\pgfpathmoveto{\pgfqpoint{1.434682in}{1.718719in}}%
\pgfpathcurveto{\pgfqpoint{1.442918in}{1.718719in}}{\pgfqpoint{1.450818in}{1.721992in}}{\pgfqpoint{1.456642in}{1.727816in}}%
\pgfpathcurveto{\pgfqpoint{1.462466in}{1.733640in}}{\pgfqpoint{1.465738in}{1.741540in}}{\pgfqpoint{1.465738in}{1.749776in}}%
\pgfpathcurveto{\pgfqpoint{1.465738in}{1.758012in}}{\pgfqpoint{1.462466in}{1.765912in}}{\pgfqpoint{1.456642in}{1.771736in}}%
\pgfpathcurveto{\pgfqpoint{1.450818in}{1.777560in}}{\pgfqpoint{1.442918in}{1.780832in}}{\pgfqpoint{1.434682in}{1.780832in}}%
\pgfpathcurveto{\pgfqpoint{1.426446in}{1.780832in}}{\pgfqpoint{1.418546in}{1.777560in}}{\pgfqpoint{1.412722in}{1.771736in}}%
\pgfpathcurveto{\pgfqpoint{1.406898in}{1.765912in}}{\pgfqpoint{1.403625in}{1.758012in}}{\pgfqpoint{1.403625in}{1.749776in}}%
\pgfpathcurveto{\pgfqpoint{1.403625in}{1.741540in}}{\pgfqpoint{1.406898in}{1.733640in}}{\pgfqpoint{1.412722in}{1.727816in}}%
\pgfpathcurveto{\pgfqpoint{1.418546in}{1.721992in}}{\pgfqpoint{1.426446in}{1.718719in}}{\pgfqpoint{1.434682in}{1.718719in}}%
\pgfpathclose%
\pgfusepath{stroke,fill}%
\end{pgfscope}%
\begin{pgfscope}%
\pgfpathrectangle{\pgfqpoint{0.100000in}{0.212622in}}{\pgfqpoint{3.696000in}{3.696000in}}%
\pgfusepath{clip}%
\pgfsetbuttcap%
\pgfsetroundjoin%
\definecolor{currentfill}{rgb}{0.121569,0.466667,0.705882}%
\pgfsetfillcolor{currentfill}%
\pgfsetfillopacity{0.322683}%
\pgfsetlinewidth{1.003750pt}%
\definecolor{currentstroke}{rgb}{0.121569,0.466667,0.705882}%
\pgfsetstrokecolor{currentstroke}%
\pgfsetstrokeopacity{0.322683}%
\pgfsetdash{}{0pt}%
\pgfpathmoveto{\pgfqpoint{1.497157in}{1.733988in}}%
\pgfpathcurveto{\pgfqpoint{1.505393in}{1.733988in}}{\pgfqpoint{1.513293in}{1.737261in}}{\pgfqpoint{1.519117in}{1.743085in}}%
\pgfpathcurveto{\pgfqpoint{1.524941in}{1.748908in}}{\pgfqpoint{1.528214in}{1.756809in}}{\pgfqpoint{1.528214in}{1.765045in}}%
\pgfpathcurveto{\pgfqpoint{1.528214in}{1.773281in}}{\pgfqpoint{1.524941in}{1.781181in}}{\pgfqpoint{1.519117in}{1.787005in}}%
\pgfpathcurveto{\pgfqpoint{1.513293in}{1.792829in}}{\pgfqpoint{1.505393in}{1.796101in}}{\pgfqpoint{1.497157in}{1.796101in}}%
\pgfpathcurveto{\pgfqpoint{1.488921in}{1.796101in}}{\pgfqpoint{1.481021in}{1.792829in}}{\pgfqpoint{1.475197in}{1.787005in}}%
\pgfpathcurveto{\pgfqpoint{1.469373in}{1.781181in}}{\pgfqpoint{1.466101in}{1.773281in}}{\pgfqpoint{1.466101in}{1.765045in}}%
\pgfpathcurveto{\pgfqpoint{1.466101in}{1.756809in}}{\pgfqpoint{1.469373in}{1.748908in}}{\pgfqpoint{1.475197in}{1.743085in}}%
\pgfpathcurveto{\pgfqpoint{1.481021in}{1.737261in}}{\pgfqpoint{1.488921in}{1.733988in}}{\pgfqpoint{1.497157in}{1.733988in}}%
\pgfpathclose%
\pgfusepath{stroke,fill}%
\end{pgfscope}%
\begin{pgfscope}%
\pgfpathrectangle{\pgfqpoint{0.100000in}{0.212622in}}{\pgfqpoint{3.696000in}{3.696000in}}%
\pgfusepath{clip}%
\pgfsetbuttcap%
\pgfsetroundjoin%
\definecolor{currentfill}{rgb}{0.121569,0.466667,0.705882}%
\pgfsetfillcolor{currentfill}%
\pgfsetfillopacity{0.323314}%
\pgfsetlinewidth{1.003750pt}%
\definecolor{currentstroke}{rgb}{0.121569,0.466667,0.705882}%
\pgfsetstrokecolor{currentstroke}%
\pgfsetstrokeopacity{0.323314}%
\pgfsetdash{}{0pt}%
\pgfpathmoveto{\pgfqpoint{1.431530in}{1.716657in}}%
\pgfpathcurveto{\pgfqpoint{1.439766in}{1.716657in}}{\pgfqpoint{1.447666in}{1.719929in}}{\pgfqpoint{1.453490in}{1.725753in}}%
\pgfpathcurveto{\pgfqpoint{1.459314in}{1.731577in}}{\pgfqpoint{1.462586in}{1.739477in}}{\pgfqpoint{1.462586in}{1.747713in}}%
\pgfpathcurveto{\pgfqpoint{1.462586in}{1.755950in}}{\pgfqpoint{1.459314in}{1.763850in}}{\pgfqpoint{1.453490in}{1.769674in}}%
\pgfpathcurveto{\pgfqpoint{1.447666in}{1.775498in}}{\pgfqpoint{1.439766in}{1.778770in}}{\pgfqpoint{1.431530in}{1.778770in}}%
\pgfpathcurveto{\pgfqpoint{1.423294in}{1.778770in}}{\pgfqpoint{1.415394in}{1.775498in}}{\pgfqpoint{1.409570in}{1.769674in}}%
\pgfpathcurveto{\pgfqpoint{1.403746in}{1.763850in}}{\pgfqpoint{1.400473in}{1.755950in}}{\pgfqpoint{1.400473in}{1.747713in}}%
\pgfpathcurveto{\pgfqpoint{1.400473in}{1.739477in}}{\pgfqpoint{1.403746in}{1.731577in}}{\pgfqpoint{1.409570in}{1.725753in}}%
\pgfpathcurveto{\pgfqpoint{1.415394in}{1.719929in}}{\pgfqpoint{1.423294in}{1.716657in}}{\pgfqpoint{1.431530in}{1.716657in}}%
\pgfpathclose%
\pgfusepath{stroke,fill}%
\end{pgfscope}%
\begin{pgfscope}%
\pgfpathrectangle{\pgfqpoint{0.100000in}{0.212622in}}{\pgfqpoint{3.696000in}{3.696000in}}%
\pgfusepath{clip}%
\pgfsetbuttcap%
\pgfsetroundjoin%
\definecolor{currentfill}{rgb}{0.121569,0.466667,0.705882}%
\pgfsetfillcolor{currentfill}%
\pgfsetfillopacity{0.324830}%
\pgfsetlinewidth{1.003750pt}%
\definecolor{currentstroke}{rgb}{0.121569,0.466667,0.705882}%
\pgfsetstrokecolor{currentstroke}%
\pgfsetstrokeopacity{0.324830}%
\pgfsetdash{}{0pt}%
\pgfpathmoveto{\pgfqpoint{1.497737in}{1.732164in}}%
\pgfpathcurveto{\pgfqpoint{1.505973in}{1.732164in}}{\pgfqpoint{1.513873in}{1.735436in}}{\pgfqpoint{1.519697in}{1.741260in}}%
\pgfpathcurveto{\pgfqpoint{1.525521in}{1.747084in}}{\pgfqpoint{1.528793in}{1.754984in}}{\pgfqpoint{1.528793in}{1.763220in}}%
\pgfpathcurveto{\pgfqpoint{1.528793in}{1.771457in}}{\pgfqpoint{1.525521in}{1.779357in}}{\pgfqpoint{1.519697in}{1.785181in}}%
\pgfpathcurveto{\pgfqpoint{1.513873in}{1.791005in}}{\pgfqpoint{1.505973in}{1.794277in}}{\pgfqpoint{1.497737in}{1.794277in}}%
\pgfpathcurveto{\pgfqpoint{1.489501in}{1.794277in}}{\pgfqpoint{1.481601in}{1.791005in}}{\pgfqpoint{1.475777in}{1.785181in}}%
\pgfpathcurveto{\pgfqpoint{1.469953in}{1.779357in}}{\pgfqpoint{1.466680in}{1.771457in}}{\pgfqpoint{1.466680in}{1.763220in}}%
\pgfpathcurveto{\pgfqpoint{1.466680in}{1.754984in}}{\pgfqpoint{1.469953in}{1.747084in}}{\pgfqpoint{1.475777in}{1.741260in}}%
\pgfpathcurveto{\pgfqpoint{1.481601in}{1.735436in}}{\pgfqpoint{1.489501in}{1.732164in}}{\pgfqpoint{1.497737in}{1.732164in}}%
\pgfpathclose%
\pgfusepath{stroke,fill}%
\end{pgfscope}%
\begin{pgfscope}%
\pgfpathrectangle{\pgfqpoint{0.100000in}{0.212622in}}{\pgfqpoint{3.696000in}{3.696000in}}%
\pgfusepath{clip}%
\pgfsetbuttcap%
\pgfsetroundjoin%
\definecolor{currentfill}{rgb}{0.121569,0.466667,0.705882}%
\pgfsetfillcolor{currentfill}%
\pgfsetfillopacity{0.326500}%
\pgfsetlinewidth{1.003750pt}%
\definecolor{currentstroke}{rgb}{0.121569,0.466667,0.705882}%
\pgfsetstrokecolor{currentstroke}%
\pgfsetstrokeopacity{0.326500}%
\pgfsetdash{}{0pt}%
\pgfpathmoveto{\pgfqpoint{1.427038in}{1.716853in}}%
\pgfpathcurveto{\pgfqpoint{1.435274in}{1.716853in}}{\pgfqpoint{1.443174in}{1.720125in}}{\pgfqpoint{1.448998in}{1.725949in}}%
\pgfpathcurveto{\pgfqpoint{1.454822in}{1.731773in}}{\pgfqpoint{1.458094in}{1.739673in}}{\pgfqpoint{1.458094in}{1.747909in}}%
\pgfpathcurveto{\pgfqpoint{1.458094in}{1.756146in}}{\pgfqpoint{1.454822in}{1.764046in}}{\pgfqpoint{1.448998in}{1.769870in}}%
\pgfpathcurveto{\pgfqpoint{1.443174in}{1.775694in}}{\pgfqpoint{1.435274in}{1.778966in}}{\pgfqpoint{1.427038in}{1.778966in}}%
\pgfpathcurveto{\pgfqpoint{1.418802in}{1.778966in}}{\pgfqpoint{1.410902in}{1.775694in}}{\pgfqpoint{1.405078in}{1.769870in}}%
\pgfpathcurveto{\pgfqpoint{1.399254in}{1.764046in}}{\pgfqpoint{1.395981in}{1.756146in}}{\pgfqpoint{1.395981in}{1.747909in}}%
\pgfpathcurveto{\pgfqpoint{1.395981in}{1.739673in}}{\pgfqpoint{1.399254in}{1.731773in}}{\pgfqpoint{1.405078in}{1.725949in}}%
\pgfpathcurveto{\pgfqpoint{1.410902in}{1.720125in}}{\pgfqpoint{1.418802in}{1.716853in}}{\pgfqpoint{1.427038in}{1.716853in}}%
\pgfpathclose%
\pgfusepath{stroke,fill}%
\end{pgfscope}%
\begin{pgfscope}%
\pgfpathrectangle{\pgfqpoint{0.100000in}{0.212622in}}{\pgfqpoint{3.696000in}{3.696000in}}%
\pgfusepath{clip}%
\pgfsetbuttcap%
\pgfsetroundjoin%
\definecolor{currentfill}{rgb}{0.121569,0.466667,0.705882}%
\pgfsetfillcolor{currentfill}%
\pgfsetfillopacity{0.327721}%
\pgfsetlinewidth{1.003750pt}%
\definecolor{currentstroke}{rgb}{0.121569,0.466667,0.705882}%
\pgfsetstrokecolor{currentstroke}%
\pgfsetstrokeopacity{0.327721}%
\pgfsetdash{}{0pt}%
\pgfpathmoveto{\pgfqpoint{1.499384in}{1.733499in}}%
\pgfpathcurveto{\pgfqpoint{1.507620in}{1.733499in}}{\pgfqpoint{1.515521in}{1.736772in}}{\pgfqpoint{1.521344in}{1.742596in}}%
\pgfpathcurveto{\pgfqpoint{1.527168in}{1.748420in}}{\pgfqpoint{1.530441in}{1.756320in}}{\pgfqpoint{1.530441in}{1.764556in}}%
\pgfpathcurveto{\pgfqpoint{1.530441in}{1.772792in}}{\pgfqpoint{1.527168in}{1.780692in}}{\pgfqpoint{1.521344in}{1.786516in}}%
\pgfpathcurveto{\pgfqpoint{1.515521in}{1.792340in}}{\pgfqpoint{1.507620in}{1.795612in}}{\pgfqpoint{1.499384in}{1.795612in}}%
\pgfpathcurveto{\pgfqpoint{1.491148in}{1.795612in}}{\pgfqpoint{1.483248in}{1.792340in}}{\pgfqpoint{1.477424in}{1.786516in}}%
\pgfpathcurveto{\pgfqpoint{1.471600in}{1.780692in}}{\pgfqpoint{1.468328in}{1.772792in}}{\pgfqpoint{1.468328in}{1.764556in}}%
\pgfpathcurveto{\pgfqpoint{1.468328in}{1.756320in}}{\pgfqpoint{1.471600in}{1.748420in}}{\pgfqpoint{1.477424in}{1.742596in}}%
\pgfpathcurveto{\pgfqpoint{1.483248in}{1.736772in}}{\pgfqpoint{1.491148in}{1.733499in}}{\pgfqpoint{1.499384in}{1.733499in}}%
\pgfpathclose%
\pgfusepath{stroke,fill}%
\end{pgfscope}%
\begin{pgfscope}%
\pgfpathrectangle{\pgfqpoint{0.100000in}{0.212622in}}{\pgfqpoint{3.696000in}{3.696000in}}%
\pgfusepath{clip}%
\pgfsetbuttcap%
\pgfsetroundjoin%
\definecolor{currentfill}{rgb}{0.121569,0.466667,0.705882}%
\pgfsetfillcolor{currentfill}%
\pgfsetfillopacity{0.328173}%
\pgfsetlinewidth{1.003750pt}%
\definecolor{currentstroke}{rgb}{0.121569,0.466667,0.705882}%
\pgfsetstrokecolor{currentstroke}%
\pgfsetstrokeopacity{0.328173}%
\pgfsetdash{}{0pt}%
\pgfpathmoveto{\pgfqpoint{1.422395in}{1.717934in}}%
\pgfpathcurveto{\pgfqpoint{1.430632in}{1.717934in}}{\pgfqpoint{1.438532in}{1.721207in}}{\pgfqpoint{1.444356in}{1.727030in}}%
\pgfpathcurveto{\pgfqpoint{1.450180in}{1.732854in}}{\pgfqpoint{1.453452in}{1.740754in}}{\pgfqpoint{1.453452in}{1.748991in}}%
\pgfpathcurveto{\pgfqpoint{1.453452in}{1.757227in}}{\pgfqpoint{1.450180in}{1.765127in}}{\pgfqpoint{1.444356in}{1.770951in}}%
\pgfpathcurveto{\pgfqpoint{1.438532in}{1.776775in}}{\pgfqpoint{1.430632in}{1.780047in}}{\pgfqpoint{1.422395in}{1.780047in}}%
\pgfpathcurveto{\pgfqpoint{1.414159in}{1.780047in}}{\pgfqpoint{1.406259in}{1.776775in}}{\pgfqpoint{1.400435in}{1.770951in}}%
\pgfpathcurveto{\pgfqpoint{1.394611in}{1.765127in}}{\pgfqpoint{1.391339in}{1.757227in}}{\pgfqpoint{1.391339in}{1.748991in}}%
\pgfpathcurveto{\pgfqpoint{1.391339in}{1.740754in}}{\pgfqpoint{1.394611in}{1.732854in}}{\pgfqpoint{1.400435in}{1.727030in}}%
\pgfpathcurveto{\pgfqpoint{1.406259in}{1.721207in}}{\pgfqpoint{1.414159in}{1.717934in}}{\pgfqpoint{1.422395in}{1.717934in}}%
\pgfpathclose%
\pgfusepath{stroke,fill}%
\end{pgfscope}%
\begin{pgfscope}%
\pgfpathrectangle{\pgfqpoint{0.100000in}{0.212622in}}{\pgfqpoint{3.696000in}{3.696000in}}%
\pgfusepath{clip}%
\pgfsetbuttcap%
\pgfsetroundjoin%
\definecolor{currentfill}{rgb}{0.121569,0.466667,0.705882}%
\pgfsetfillcolor{currentfill}%
\pgfsetfillopacity{0.328865}%
\pgfsetlinewidth{1.003750pt}%
\definecolor{currentstroke}{rgb}{0.121569,0.466667,0.705882}%
\pgfsetstrokecolor{currentstroke}%
\pgfsetstrokeopacity{0.328865}%
\pgfsetdash{}{0pt}%
\pgfpathmoveto{\pgfqpoint{1.420278in}{1.716025in}}%
\pgfpathcurveto{\pgfqpoint{1.428515in}{1.716025in}}{\pgfqpoint{1.436415in}{1.719297in}}{\pgfqpoint{1.442239in}{1.725121in}}%
\pgfpathcurveto{\pgfqpoint{1.448063in}{1.730945in}}{\pgfqpoint{1.451335in}{1.738845in}}{\pgfqpoint{1.451335in}{1.747081in}}%
\pgfpathcurveto{\pgfqpoint{1.451335in}{1.755318in}}{\pgfqpoint{1.448063in}{1.763218in}}{\pgfqpoint{1.442239in}{1.769042in}}%
\pgfpathcurveto{\pgfqpoint{1.436415in}{1.774865in}}{\pgfqpoint{1.428515in}{1.778138in}}{\pgfqpoint{1.420278in}{1.778138in}}%
\pgfpathcurveto{\pgfqpoint{1.412042in}{1.778138in}}{\pgfqpoint{1.404142in}{1.774865in}}{\pgfqpoint{1.398318in}{1.769042in}}%
\pgfpathcurveto{\pgfqpoint{1.392494in}{1.763218in}}{\pgfqpoint{1.389222in}{1.755318in}}{\pgfqpoint{1.389222in}{1.747081in}}%
\pgfpathcurveto{\pgfqpoint{1.389222in}{1.738845in}}{\pgfqpoint{1.392494in}{1.730945in}}{\pgfqpoint{1.398318in}{1.725121in}}%
\pgfpathcurveto{\pgfqpoint{1.404142in}{1.719297in}}{\pgfqpoint{1.412042in}{1.716025in}}{\pgfqpoint{1.420278in}{1.716025in}}%
\pgfpathclose%
\pgfusepath{stroke,fill}%
\end{pgfscope}%
\begin{pgfscope}%
\pgfpathrectangle{\pgfqpoint{0.100000in}{0.212622in}}{\pgfqpoint{3.696000in}{3.696000in}}%
\pgfusepath{clip}%
\pgfsetbuttcap%
\pgfsetroundjoin%
\definecolor{currentfill}{rgb}{0.121569,0.466667,0.705882}%
\pgfsetfillcolor{currentfill}%
\pgfsetfillopacity{0.330881}%
\pgfsetlinewidth{1.003750pt}%
\definecolor{currentstroke}{rgb}{0.121569,0.466667,0.705882}%
\pgfsetstrokecolor{currentstroke}%
\pgfsetstrokeopacity{0.330881}%
\pgfsetdash{}{0pt}%
\pgfpathmoveto{\pgfqpoint{1.416841in}{1.715480in}}%
\pgfpathcurveto{\pgfqpoint{1.425077in}{1.715480in}}{\pgfqpoint{1.432977in}{1.718752in}}{\pgfqpoint{1.438801in}{1.724576in}}%
\pgfpathcurveto{\pgfqpoint{1.444625in}{1.730400in}}{\pgfqpoint{1.447897in}{1.738300in}}{\pgfqpoint{1.447897in}{1.746536in}}%
\pgfpathcurveto{\pgfqpoint{1.447897in}{1.754772in}}{\pgfqpoint{1.444625in}{1.762673in}}{\pgfqpoint{1.438801in}{1.768496in}}%
\pgfpathcurveto{\pgfqpoint{1.432977in}{1.774320in}}{\pgfqpoint{1.425077in}{1.777593in}}{\pgfqpoint{1.416841in}{1.777593in}}%
\pgfpathcurveto{\pgfqpoint{1.408604in}{1.777593in}}{\pgfqpoint{1.400704in}{1.774320in}}{\pgfqpoint{1.394880in}{1.768496in}}%
\pgfpathcurveto{\pgfqpoint{1.389056in}{1.762673in}}{\pgfqpoint{1.385784in}{1.754772in}}{\pgfqpoint{1.385784in}{1.746536in}}%
\pgfpathcurveto{\pgfqpoint{1.385784in}{1.738300in}}{\pgfqpoint{1.389056in}{1.730400in}}{\pgfqpoint{1.394880in}{1.724576in}}%
\pgfpathcurveto{\pgfqpoint{1.400704in}{1.718752in}}{\pgfqpoint{1.408604in}{1.715480in}}{\pgfqpoint{1.416841in}{1.715480in}}%
\pgfpathclose%
\pgfusepath{stroke,fill}%
\end{pgfscope}%
\begin{pgfscope}%
\pgfpathrectangle{\pgfqpoint{0.100000in}{0.212622in}}{\pgfqpoint{3.696000in}{3.696000in}}%
\pgfusepath{clip}%
\pgfsetbuttcap%
\pgfsetroundjoin%
\definecolor{currentfill}{rgb}{0.121569,0.466667,0.705882}%
\pgfsetfillcolor{currentfill}%
\pgfsetfillopacity{0.331199}%
\pgfsetlinewidth{1.003750pt}%
\definecolor{currentstroke}{rgb}{0.121569,0.466667,0.705882}%
\pgfsetstrokecolor{currentstroke}%
\pgfsetstrokeopacity{0.331199}%
\pgfsetdash{}{0pt}%
\pgfpathmoveto{\pgfqpoint{1.501311in}{1.736682in}}%
\pgfpathcurveto{\pgfqpoint{1.509547in}{1.736682in}}{\pgfqpoint{1.517447in}{1.739954in}}{\pgfqpoint{1.523271in}{1.745778in}}%
\pgfpathcurveto{\pgfqpoint{1.529095in}{1.751602in}}{\pgfqpoint{1.532368in}{1.759502in}}{\pgfqpoint{1.532368in}{1.767738in}}%
\pgfpathcurveto{\pgfqpoint{1.532368in}{1.775975in}}{\pgfqpoint{1.529095in}{1.783875in}}{\pgfqpoint{1.523271in}{1.789699in}}%
\pgfpathcurveto{\pgfqpoint{1.517447in}{1.795522in}}{\pgfqpoint{1.509547in}{1.798795in}}{\pgfqpoint{1.501311in}{1.798795in}}%
\pgfpathcurveto{\pgfqpoint{1.493075in}{1.798795in}}{\pgfqpoint{1.485175in}{1.795522in}}{\pgfqpoint{1.479351in}{1.789699in}}%
\pgfpathcurveto{\pgfqpoint{1.473527in}{1.783875in}}{\pgfqpoint{1.470255in}{1.775975in}}{\pgfqpoint{1.470255in}{1.767738in}}%
\pgfpathcurveto{\pgfqpoint{1.470255in}{1.759502in}}{\pgfqpoint{1.473527in}{1.751602in}}{\pgfqpoint{1.479351in}{1.745778in}}%
\pgfpathcurveto{\pgfqpoint{1.485175in}{1.739954in}}{\pgfqpoint{1.493075in}{1.736682in}}{\pgfqpoint{1.501311in}{1.736682in}}%
\pgfpathclose%
\pgfusepath{stroke,fill}%
\end{pgfscope}%
\begin{pgfscope}%
\pgfpathrectangle{\pgfqpoint{0.100000in}{0.212622in}}{\pgfqpoint{3.696000in}{3.696000in}}%
\pgfusepath{clip}%
\pgfsetbuttcap%
\pgfsetroundjoin%
\definecolor{currentfill}{rgb}{0.121569,0.466667,0.705882}%
\pgfsetfillcolor{currentfill}%
\pgfsetfillopacity{0.331870}%
\pgfsetlinewidth{1.003750pt}%
\definecolor{currentstroke}{rgb}{0.121569,0.466667,0.705882}%
\pgfsetstrokecolor{currentstroke}%
\pgfsetstrokeopacity{0.331870}%
\pgfsetdash{}{0pt}%
\pgfpathmoveto{\pgfqpoint{1.414185in}{1.715392in}}%
\pgfpathcurveto{\pgfqpoint{1.422421in}{1.715392in}}{\pgfqpoint{1.430321in}{1.718665in}}{\pgfqpoint{1.436145in}{1.724489in}}%
\pgfpathcurveto{\pgfqpoint{1.441969in}{1.730313in}}{\pgfqpoint{1.445241in}{1.738213in}}{\pgfqpoint{1.445241in}{1.746449in}}%
\pgfpathcurveto{\pgfqpoint{1.445241in}{1.754685in}}{\pgfqpoint{1.441969in}{1.762585in}}{\pgfqpoint{1.436145in}{1.768409in}}%
\pgfpathcurveto{\pgfqpoint{1.430321in}{1.774233in}}{\pgfqpoint{1.422421in}{1.777505in}}{\pgfqpoint{1.414185in}{1.777505in}}%
\pgfpathcurveto{\pgfqpoint{1.405949in}{1.777505in}}{\pgfqpoint{1.398049in}{1.774233in}}{\pgfqpoint{1.392225in}{1.768409in}}%
\pgfpathcurveto{\pgfqpoint{1.386401in}{1.762585in}}{\pgfqpoint{1.383128in}{1.754685in}}{\pgfqpoint{1.383128in}{1.746449in}}%
\pgfpathcurveto{\pgfqpoint{1.383128in}{1.738213in}}{\pgfqpoint{1.386401in}{1.730313in}}{\pgfqpoint{1.392225in}{1.724489in}}%
\pgfpathcurveto{\pgfqpoint{1.398049in}{1.718665in}}{\pgfqpoint{1.405949in}{1.715392in}}{\pgfqpoint{1.414185in}{1.715392in}}%
\pgfpathclose%
\pgfusepath{stroke,fill}%
\end{pgfscope}%
\begin{pgfscope}%
\pgfpathrectangle{\pgfqpoint{0.100000in}{0.212622in}}{\pgfqpoint{3.696000in}{3.696000in}}%
\pgfusepath{clip}%
\pgfsetbuttcap%
\pgfsetroundjoin%
\definecolor{currentfill}{rgb}{0.121569,0.466667,0.705882}%
\pgfsetfillcolor{currentfill}%
\pgfsetfillopacity{0.332545}%
\pgfsetlinewidth{1.003750pt}%
\definecolor{currentstroke}{rgb}{0.121569,0.466667,0.705882}%
\pgfsetstrokecolor{currentstroke}%
\pgfsetstrokeopacity{0.332545}%
\pgfsetdash{}{0pt}%
\pgfpathmoveto{\pgfqpoint{1.412336in}{1.713858in}}%
\pgfpathcurveto{\pgfqpoint{1.420573in}{1.713858in}}{\pgfqpoint{1.428473in}{1.717130in}}{\pgfqpoint{1.434297in}{1.722954in}}%
\pgfpathcurveto{\pgfqpoint{1.440121in}{1.728778in}}{\pgfqpoint{1.443393in}{1.736678in}}{\pgfqpoint{1.443393in}{1.744914in}}%
\pgfpathcurveto{\pgfqpoint{1.443393in}{1.753151in}}{\pgfqpoint{1.440121in}{1.761051in}}{\pgfqpoint{1.434297in}{1.766875in}}%
\pgfpathcurveto{\pgfqpoint{1.428473in}{1.772698in}}{\pgfqpoint{1.420573in}{1.775971in}}{\pgfqpoint{1.412336in}{1.775971in}}%
\pgfpathcurveto{\pgfqpoint{1.404100in}{1.775971in}}{\pgfqpoint{1.396200in}{1.772698in}}{\pgfqpoint{1.390376in}{1.766875in}}%
\pgfpathcurveto{\pgfqpoint{1.384552in}{1.761051in}}{\pgfqpoint{1.381280in}{1.753151in}}{\pgfqpoint{1.381280in}{1.744914in}}%
\pgfpathcurveto{\pgfqpoint{1.381280in}{1.736678in}}{\pgfqpoint{1.384552in}{1.728778in}}{\pgfqpoint{1.390376in}{1.722954in}}%
\pgfpathcurveto{\pgfqpoint{1.396200in}{1.717130in}}{\pgfqpoint{1.404100in}{1.713858in}}{\pgfqpoint{1.412336in}{1.713858in}}%
\pgfpathclose%
\pgfusepath{stroke,fill}%
\end{pgfscope}%
\begin{pgfscope}%
\pgfpathrectangle{\pgfqpoint{0.100000in}{0.212622in}}{\pgfqpoint{3.696000in}{3.696000in}}%
\pgfusepath{clip}%
\pgfsetbuttcap%
\pgfsetroundjoin%
\definecolor{currentfill}{rgb}{0.121569,0.466667,0.705882}%
\pgfsetfillcolor{currentfill}%
\pgfsetfillopacity{0.332786}%
\pgfsetlinewidth{1.003750pt}%
\definecolor{currentstroke}{rgb}{0.121569,0.466667,0.705882}%
\pgfsetstrokecolor{currentstroke}%
\pgfsetstrokeopacity{0.332786}%
\pgfsetdash{}{0pt}%
\pgfpathmoveto{\pgfqpoint{1.501962in}{1.736641in}}%
\pgfpathcurveto{\pgfqpoint{1.510198in}{1.736641in}}{\pgfqpoint{1.518098in}{1.739913in}}{\pgfqpoint{1.523922in}{1.745737in}}%
\pgfpathcurveto{\pgfqpoint{1.529746in}{1.751561in}}{\pgfqpoint{1.533019in}{1.759461in}}{\pgfqpoint{1.533019in}{1.767697in}}%
\pgfpathcurveto{\pgfqpoint{1.533019in}{1.775934in}}{\pgfqpoint{1.529746in}{1.783834in}}{\pgfqpoint{1.523922in}{1.789658in}}%
\pgfpathcurveto{\pgfqpoint{1.518098in}{1.795481in}}{\pgfqpoint{1.510198in}{1.798754in}}{\pgfqpoint{1.501962in}{1.798754in}}%
\pgfpathcurveto{\pgfqpoint{1.493726in}{1.798754in}}{\pgfqpoint{1.485826in}{1.795481in}}{\pgfqpoint{1.480002in}{1.789658in}}%
\pgfpathcurveto{\pgfqpoint{1.474178in}{1.783834in}}{\pgfqpoint{1.470906in}{1.775934in}}{\pgfqpoint{1.470906in}{1.767697in}}%
\pgfpathcurveto{\pgfqpoint{1.470906in}{1.759461in}}{\pgfqpoint{1.474178in}{1.751561in}}{\pgfqpoint{1.480002in}{1.745737in}}%
\pgfpathcurveto{\pgfqpoint{1.485826in}{1.739913in}}{\pgfqpoint{1.493726in}{1.736641in}}{\pgfqpoint{1.501962in}{1.736641in}}%
\pgfpathclose%
\pgfusepath{stroke,fill}%
\end{pgfscope}%
\begin{pgfscope}%
\pgfpathrectangle{\pgfqpoint{0.100000in}{0.212622in}}{\pgfqpoint{3.696000in}{3.696000in}}%
\pgfusepath{clip}%
\pgfsetbuttcap%
\pgfsetroundjoin%
\definecolor{currentfill}{rgb}{0.121569,0.466667,0.705882}%
\pgfsetfillcolor{currentfill}%
\pgfsetfillopacity{0.334687}%
\pgfsetlinewidth{1.003750pt}%
\definecolor{currentstroke}{rgb}{0.121569,0.466667,0.705882}%
\pgfsetstrokecolor{currentstroke}%
\pgfsetstrokeopacity{0.334687}%
\pgfsetdash{}{0pt}%
\pgfpathmoveto{\pgfqpoint{1.409269in}{1.714909in}}%
\pgfpathcurveto{\pgfqpoint{1.417506in}{1.714909in}}{\pgfqpoint{1.425406in}{1.718182in}}{\pgfqpoint{1.431230in}{1.724005in}}%
\pgfpathcurveto{\pgfqpoint{1.437053in}{1.729829in}}{\pgfqpoint{1.440326in}{1.737729in}}{\pgfqpoint{1.440326in}{1.745966in}}%
\pgfpathcurveto{\pgfqpoint{1.440326in}{1.754202in}}{\pgfqpoint{1.437053in}{1.762102in}}{\pgfqpoint{1.431230in}{1.767926in}}%
\pgfpathcurveto{\pgfqpoint{1.425406in}{1.773750in}}{\pgfqpoint{1.417506in}{1.777022in}}{\pgfqpoint{1.409269in}{1.777022in}}%
\pgfpathcurveto{\pgfqpoint{1.401033in}{1.777022in}}{\pgfqpoint{1.393133in}{1.773750in}}{\pgfqpoint{1.387309in}{1.767926in}}%
\pgfpathcurveto{\pgfqpoint{1.381485in}{1.762102in}}{\pgfqpoint{1.378213in}{1.754202in}}{\pgfqpoint{1.378213in}{1.745966in}}%
\pgfpathcurveto{\pgfqpoint{1.378213in}{1.737729in}}{\pgfqpoint{1.381485in}{1.729829in}}{\pgfqpoint{1.387309in}{1.724005in}}%
\pgfpathcurveto{\pgfqpoint{1.393133in}{1.718182in}}{\pgfqpoint{1.401033in}{1.714909in}}{\pgfqpoint{1.409269in}{1.714909in}}%
\pgfpathclose%
\pgfusepath{stroke,fill}%
\end{pgfscope}%
\begin{pgfscope}%
\pgfpathrectangle{\pgfqpoint{0.100000in}{0.212622in}}{\pgfqpoint{3.696000in}{3.696000in}}%
\pgfusepath{clip}%
\pgfsetbuttcap%
\pgfsetroundjoin%
\definecolor{currentfill}{rgb}{0.121569,0.466667,0.705882}%
\pgfsetfillcolor{currentfill}%
\pgfsetfillopacity{0.334829}%
\pgfsetlinewidth{1.003750pt}%
\definecolor{currentstroke}{rgb}{0.121569,0.466667,0.705882}%
\pgfsetstrokecolor{currentstroke}%
\pgfsetstrokeopacity{0.334829}%
\pgfsetdash{}{0pt}%
\pgfpathmoveto{\pgfqpoint{1.503268in}{1.737138in}}%
\pgfpathcurveto{\pgfqpoint{1.511504in}{1.737138in}}{\pgfqpoint{1.519404in}{1.740410in}}{\pgfqpoint{1.525228in}{1.746234in}}%
\pgfpathcurveto{\pgfqpoint{1.531052in}{1.752058in}}{\pgfqpoint{1.534325in}{1.759958in}}{\pgfqpoint{1.534325in}{1.768194in}}%
\pgfpathcurveto{\pgfqpoint{1.534325in}{1.776431in}}{\pgfqpoint{1.531052in}{1.784331in}}{\pgfqpoint{1.525228in}{1.790155in}}%
\pgfpathcurveto{\pgfqpoint{1.519404in}{1.795979in}}{\pgfqpoint{1.511504in}{1.799251in}}{\pgfqpoint{1.503268in}{1.799251in}}%
\pgfpathcurveto{\pgfqpoint{1.495032in}{1.799251in}}{\pgfqpoint{1.487132in}{1.795979in}}{\pgfqpoint{1.481308in}{1.790155in}}%
\pgfpathcurveto{\pgfqpoint{1.475484in}{1.784331in}}{\pgfqpoint{1.472212in}{1.776431in}}{\pgfqpoint{1.472212in}{1.768194in}}%
\pgfpathcurveto{\pgfqpoint{1.472212in}{1.759958in}}{\pgfqpoint{1.475484in}{1.752058in}}{\pgfqpoint{1.481308in}{1.746234in}}%
\pgfpathcurveto{\pgfqpoint{1.487132in}{1.740410in}}{\pgfqpoint{1.495032in}{1.737138in}}{\pgfqpoint{1.503268in}{1.737138in}}%
\pgfpathclose%
\pgfusepath{stroke,fill}%
\end{pgfscope}%
\begin{pgfscope}%
\pgfpathrectangle{\pgfqpoint{0.100000in}{0.212622in}}{\pgfqpoint{3.696000in}{3.696000in}}%
\pgfusepath{clip}%
\pgfsetbuttcap%
\pgfsetroundjoin%
\definecolor{currentfill}{rgb}{0.121569,0.466667,0.705882}%
\pgfsetfillcolor{currentfill}%
\pgfsetfillopacity{0.335410}%
\pgfsetlinewidth{1.003750pt}%
\definecolor{currentstroke}{rgb}{0.121569,0.466667,0.705882}%
\pgfsetstrokecolor{currentstroke}%
\pgfsetstrokeopacity{0.335410}%
\pgfsetdash{}{0pt}%
\pgfpathmoveto{\pgfqpoint{1.406822in}{1.712721in}}%
\pgfpathcurveto{\pgfqpoint{1.415058in}{1.712721in}}{\pgfqpoint{1.422958in}{1.715994in}}{\pgfqpoint{1.428782in}{1.721818in}}%
\pgfpathcurveto{\pgfqpoint{1.434606in}{1.727642in}}{\pgfqpoint{1.437878in}{1.735542in}}{\pgfqpoint{1.437878in}{1.743778in}}%
\pgfpathcurveto{\pgfqpoint{1.437878in}{1.752014in}}{\pgfqpoint{1.434606in}{1.759914in}}{\pgfqpoint{1.428782in}{1.765738in}}%
\pgfpathcurveto{\pgfqpoint{1.422958in}{1.771562in}}{\pgfqpoint{1.415058in}{1.774834in}}{\pgfqpoint{1.406822in}{1.774834in}}%
\pgfpathcurveto{\pgfqpoint{1.398585in}{1.774834in}}{\pgfqpoint{1.390685in}{1.771562in}}{\pgfqpoint{1.384861in}{1.765738in}}%
\pgfpathcurveto{\pgfqpoint{1.379037in}{1.759914in}}{\pgfqpoint{1.375765in}{1.752014in}}{\pgfqpoint{1.375765in}{1.743778in}}%
\pgfpathcurveto{\pgfqpoint{1.375765in}{1.735542in}}{\pgfqpoint{1.379037in}{1.727642in}}{\pgfqpoint{1.384861in}{1.721818in}}%
\pgfpathcurveto{\pgfqpoint{1.390685in}{1.715994in}}{\pgfqpoint{1.398585in}{1.712721in}}{\pgfqpoint{1.406822in}{1.712721in}}%
\pgfpathclose%
\pgfusepath{stroke,fill}%
\end{pgfscope}%
\begin{pgfscope}%
\pgfpathrectangle{\pgfqpoint{0.100000in}{0.212622in}}{\pgfqpoint{3.696000in}{3.696000in}}%
\pgfusepath{clip}%
\pgfsetbuttcap%
\pgfsetroundjoin%
\definecolor{currentfill}{rgb}{0.121569,0.466667,0.705882}%
\pgfsetfillcolor{currentfill}%
\pgfsetfillopacity{0.336219}%
\pgfsetlinewidth{1.003750pt}%
\definecolor{currentstroke}{rgb}{0.121569,0.466667,0.705882}%
\pgfsetstrokecolor{currentstroke}%
\pgfsetstrokeopacity{0.336219}%
\pgfsetdash{}{0pt}%
\pgfpathmoveto{\pgfqpoint{1.404761in}{1.712080in}}%
\pgfpathcurveto{\pgfqpoint{1.412997in}{1.712080in}}{\pgfqpoint{1.420898in}{1.715353in}}{\pgfqpoint{1.426721in}{1.721177in}}%
\pgfpathcurveto{\pgfqpoint{1.432545in}{1.727001in}}{\pgfqpoint{1.435818in}{1.734901in}}{\pgfqpoint{1.435818in}{1.743137in}}%
\pgfpathcurveto{\pgfqpoint{1.435818in}{1.751373in}}{\pgfqpoint{1.432545in}{1.759273in}}{\pgfqpoint{1.426721in}{1.765097in}}%
\pgfpathcurveto{\pgfqpoint{1.420898in}{1.770921in}}{\pgfqpoint{1.412997in}{1.774193in}}{\pgfqpoint{1.404761in}{1.774193in}}%
\pgfpathcurveto{\pgfqpoint{1.396525in}{1.774193in}}{\pgfqpoint{1.388625in}{1.770921in}}{\pgfqpoint{1.382801in}{1.765097in}}%
\pgfpathcurveto{\pgfqpoint{1.376977in}{1.759273in}}{\pgfqpoint{1.373705in}{1.751373in}}{\pgfqpoint{1.373705in}{1.743137in}}%
\pgfpathcurveto{\pgfqpoint{1.373705in}{1.734901in}}{\pgfqpoint{1.376977in}{1.727001in}}{\pgfqpoint{1.382801in}{1.721177in}}%
\pgfpathcurveto{\pgfqpoint{1.388625in}{1.715353in}}{\pgfqpoint{1.396525in}{1.712080in}}{\pgfqpoint{1.404761in}{1.712080in}}%
\pgfpathclose%
\pgfusepath{stroke,fill}%
\end{pgfscope}%
\begin{pgfscope}%
\pgfpathrectangle{\pgfqpoint{0.100000in}{0.212622in}}{\pgfqpoint{3.696000in}{3.696000in}}%
\pgfusepath{clip}%
\pgfsetbuttcap%
\pgfsetroundjoin%
\definecolor{currentfill}{rgb}{0.121569,0.466667,0.705882}%
\pgfsetfillcolor{currentfill}%
\pgfsetfillopacity{0.337850}%
\pgfsetlinewidth{1.003750pt}%
\definecolor{currentstroke}{rgb}{0.121569,0.466667,0.705882}%
\pgfsetstrokecolor{currentstroke}%
\pgfsetstrokeopacity{0.337850}%
\pgfsetdash{}{0pt}%
\pgfpathmoveto{\pgfqpoint{1.505103in}{1.740244in}}%
\pgfpathcurveto{\pgfqpoint{1.513340in}{1.740244in}}{\pgfqpoint{1.521240in}{1.743516in}}{\pgfqpoint{1.527064in}{1.749340in}}%
\pgfpathcurveto{\pgfqpoint{1.532887in}{1.755164in}}{\pgfqpoint{1.536160in}{1.763064in}}{\pgfqpoint{1.536160in}{1.771301in}}%
\pgfpathcurveto{\pgfqpoint{1.536160in}{1.779537in}}{\pgfqpoint{1.532887in}{1.787437in}}{\pgfqpoint{1.527064in}{1.793261in}}%
\pgfpathcurveto{\pgfqpoint{1.521240in}{1.799085in}}{\pgfqpoint{1.513340in}{1.802357in}}{\pgfqpoint{1.505103in}{1.802357in}}%
\pgfpathcurveto{\pgfqpoint{1.496867in}{1.802357in}}{\pgfqpoint{1.488967in}{1.799085in}}{\pgfqpoint{1.483143in}{1.793261in}}%
\pgfpathcurveto{\pgfqpoint{1.477319in}{1.787437in}}{\pgfqpoint{1.474047in}{1.779537in}}{\pgfqpoint{1.474047in}{1.771301in}}%
\pgfpathcurveto{\pgfqpoint{1.474047in}{1.763064in}}{\pgfqpoint{1.477319in}{1.755164in}}{\pgfqpoint{1.483143in}{1.749340in}}%
\pgfpathcurveto{\pgfqpoint{1.488967in}{1.743516in}}{\pgfqpoint{1.496867in}{1.740244in}}{\pgfqpoint{1.505103in}{1.740244in}}%
\pgfpathclose%
\pgfusepath{stroke,fill}%
\end{pgfscope}%
\begin{pgfscope}%
\pgfpathrectangle{\pgfqpoint{0.100000in}{0.212622in}}{\pgfqpoint{3.696000in}{3.696000in}}%
\pgfusepath{clip}%
\pgfsetbuttcap%
\pgfsetroundjoin%
\definecolor{currentfill}{rgb}{0.121569,0.466667,0.705882}%
\pgfsetfillcolor{currentfill}%
\pgfsetfillopacity{0.338417}%
\pgfsetlinewidth{1.003750pt}%
\definecolor{currentstroke}{rgb}{0.121569,0.466667,0.705882}%
\pgfsetstrokecolor{currentstroke}%
\pgfsetstrokeopacity{0.338417}%
\pgfsetdash{}{0pt}%
\pgfpathmoveto{\pgfqpoint{1.401991in}{1.712913in}}%
\pgfpathcurveto{\pgfqpoint{1.410228in}{1.712913in}}{\pgfqpoint{1.418128in}{1.716185in}}{\pgfqpoint{1.423952in}{1.722009in}}%
\pgfpathcurveto{\pgfqpoint{1.429775in}{1.727833in}}{\pgfqpoint{1.433048in}{1.735733in}}{\pgfqpoint{1.433048in}{1.743969in}}%
\pgfpathcurveto{\pgfqpoint{1.433048in}{1.752206in}}{\pgfqpoint{1.429775in}{1.760106in}}{\pgfqpoint{1.423952in}{1.765930in}}%
\pgfpathcurveto{\pgfqpoint{1.418128in}{1.771754in}}{\pgfqpoint{1.410228in}{1.775026in}}{\pgfqpoint{1.401991in}{1.775026in}}%
\pgfpathcurveto{\pgfqpoint{1.393755in}{1.775026in}}{\pgfqpoint{1.385855in}{1.771754in}}{\pgfqpoint{1.380031in}{1.765930in}}%
\pgfpathcurveto{\pgfqpoint{1.374207in}{1.760106in}}{\pgfqpoint{1.370935in}{1.752206in}}{\pgfqpoint{1.370935in}{1.743969in}}%
\pgfpathcurveto{\pgfqpoint{1.370935in}{1.735733in}}{\pgfqpoint{1.374207in}{1.727833in}}{\pgfqpoint{1.380031in}{1.722009in}}%
\pgfpathcurveto{\pgfqpoint{1.385855in}{1.716185in}}{\pgfqpoint{1.393755in}{1.712913in}}{\pgfqpoint{1.401991in}{1.712913in}}%
\pgfpathclose%
\pgfusepath{stroke,fill}%
\end{pgfscope}%
\begin{pgfscope}%
\pgfpathrectangle{\pgfqpoint{0.100000in}{0.212622in}}{\pgfqpoint{3.696000in}{3.696000in}}%
\pgfusepath{clip}%
\pgfsetbuttcap%
\pgfsetroundjoin%
\definecolor{currentfill}{rgb}{0.121569,0.466667,0.705882}%
\pgfsetfillcolor{currentfill}%
\pgfsetfillopacity{0.339191}%
\pgfsetlinewidth{1.003750pt}%
\definecolor{currentstroke}{rgb}{0.121569,0.466667,0.705882}%
\pgfsetstrokecolor{currentstroke}%
\pgfsetstrokeopacity{0.339191}%
\pgfsetdash{}{0pt}%
\pgfpathmoveto{\pgfqpoint{1.399313in}{1.711183in}}%
\pgfpathcurveto{\pgfqpoint{1.407549in}{1.711183in}}{\pgfqpoint{1.415449in}{1.714456in}}{\pgfqpoint{1.421273in}{1.720280in}}%
\pgfpathcurveto{\pgfqpoint{1.427097in}{1.726103in}}{\pgfqpoint{1.430369in}{1.734003in}}{\pgfqpoint{1.430369in}{1.742240in}}%
\pgfpathcurveto{\pgfqpoint{1.430369in}{1.750476in}}{\pgfqpoint{1.427097in}{1.758376in}}{\pgfqpoint{1.421273in}{1.764200in}}%
\pgfpathcurveto{\pgfqpoint{1.415449in}{1.770024in}}{\pgfqpoint{1.407549in}{1.773296in}}{\pgfqpoint{1.399313in}{1.773296in}}%
\pgfpathcurveto{\pgfqpoint{1.391077in}{1.773296in}}{\pgfqpoint{1.383176in}{1.770024in}}{\pgfqpoint{1.377353in}{1.764200in}}%
\pgfpathcurveto{\pgfqpoint{1.371529in}{1.758376in}}{\pgfqpoint{1.368256in}{1.750476in}}{\pgfqpoint{1.368256in}{1.742240in}}%
\pgfpathcurveto{\pgfqpoint{1.368256in}{1.734003in}}{\pgfqpoint{1.371529in}{1.726103in}}{\pgfqpoint{1.377353in}{1.720280in}}%
\pgfpathcurveto{\pgfqpoint{1.383176in}{1.714456in}}{\pgfqpoint{1.391077in}{1.711183in}}{\pgfqpoint{1.399313in}{1.711183in}}%
\pgfpathclose%
\pgfusepath{stroke,fill}%
\end{pgfscope}%
\begin{pgfscope}%
\pgfpathrectangle{\pgfqpoint{0.100000in}{0.212622in}}{\pgfqpoint{3.696000in}{3.696000in}}%
\pgfusepath{clip}%
\pgfsetbuttcap%
\pgfsetroundjoin%
\definecolor{currentfill}{rgb}{0.121569,0.466667,0.705882}%
\pgfsetfillcolor{currentfill}%
\pgfsetfillopacity{0.339843}%
\pgfsetlinewidth{1.003750pt}%
\definecolor{currentstroke}{rgb}{0.121569,0.466667,0.705882}%
\pgfsetstrokecolor{currentstroke}%
\pgfsetstrokeopacity{0.339843}%
\pgfsetdash{}{0pt}%
\pgfpathmoveto{\pgfqpoint{1.397559in}{1.709787in}}%
\pgfpathcurveto{\pgfqpoint{1.405796in}{1.709787in}}{\pgfqpoint{1.413696in}{1.713060in}}{\pgfqpoint{1.419520in}{1.718884in}}%
\pgfpathcurveto{\pgfqpoint{1.425343in}{1.724708in}}{\pgfqpoint{1.428616in}{1.732608in}}{\pgfqpoint{1.428616in}{1.740844in}}%
\pgfpathcurveto{\pgfqpoint{1.428616in}{1.749080in}}{\pgfqpoint{1.425343in}{1.756980in}}{\pgfqpoint{1.419520in}{1.762804in}}%
\pgfpathcurveto{\pgfqpoint{1.413696in}{1.768628in}}{\pgfqpoint{1.405796in}{1.771900in}}{\pgfqpoint{1.397559in}{1.771900in}}%
\pgfpathcurveto{\pgfqpoint{1.389323in}{1.771900in}}{\pgfqpoint{1.381423in}{1.768628in}}{\pgfqpoint{1.375599in}{1.762804in}}%
\pgfpathcurveto{\pgfqpoint{1.369775in}{1.756980in}}{\pgfqpoint{1.366503in}{1.749080in}}{\pgfqpoint{1.366503in}{1.740844in}}%
\pgfpathcurveto{\pgfqpoint{1.366503in}{1.732608in}}{\pgfqpoint{1.369775in}{1.724708in}}{\pgfqpoint{1.375599in}{1.718884in}}%
\pgfpathcurveto{\pgfqpoint{1.381423in}{1.713060in}}{\pgfqpoint{1.389323in}{1.709787in}}{\pgfqpoint{1.397559in}{1.709787in}}%
\pgfpathclose%
\pgfusepath{stroke,fill}%
\end{pgfscope}%
\begin{pgfscope}%
\pgfpathrectangle{\pgfqpoint{0.100000in}{0.212622in}}{\pgfqpoint{3.696000in}{3.696000in}}%
\pgfusepath{clip}%
\pgfsetbuttcap%
\pgfsetroundjoin%
\definecolor{currentfill}{rgb}{0.121569,0.466667,0.705882}%
\pgfsetfillcolor{currentfill}%
\pgfsetfillopacity{0.341054}%
\pgfsetlinewidth{1.003750pt}%
\definecolor{currentstroke}{rgb}{0.121569,0.466667,0.705882}%
\pgfsetstrokecolor{currentstroke}%
\pgfsetstrokeopacity{0.341054}%
\pgfsetdash{}{0pt}%
\pgfpathmoveto{\pgfqpoint{1.506324in}{1.740102in}}%
\pgfpathcurveto{\pgfqpoint{1.514560in}{1.740102in}}{\pgfqpoint{1.522461in}{1.743374in}}{\pgfqpoint{1.528284in}{1.749198in}}%
\pgfpathcurveto{\pgfqpoint{1.534108in}{1.755022in}}{\pgfqpoint{1.537381in}{1.762922in}}{\pgfqpoint{1.537381in}{1.771159in}}%
\pgfpathcurveto{\pgfqpoint{1.537381in}{1.779395in}}{\pgfqpoint{1.534108in}{1.787295in}}{\pgfqpoint{1.528284in}{1.793119in}}%
\pgfpathcurveto{\pgfqpoint{1.522461in}{1.798943in}}{\pgfqpoint{1.514560in}{1.802215in}}{\pgfqpoint{1.506324in}{1.802215in}}%
\pgfpathcurveto{\pgfqpoint{1.498088in}{1.802215in}}{\pgfqpoint{1.490188in}{1.798943in}}{\pgfqpoint{1.484364in}{1.793119in}}%
\pgfpathcurveto{\pgfqpoint{1.478540in}{1.787295in}}{\pgfqpoint{1.475268in}{1.779395in}}{\pgfqpoint{1.475268in}{1.771159in}}%
\pgfpathcurveto{\pgfqpoint{1.475268in}{1.762922in}}{\pgfqpoint{1.478540in}{1.755022in}}{\pgfqpoint{1.484364in}{1.749198in}}%
\pgfpathcurveto{\pgfqpoint{1.490188in}{1.743374in}}{\pgfqpoint{1.498088in}{1.740102in}}{\pgfqpoint{1.506324in}{1.740102in}}%
\pgfpathclose%
\pgfusepath{stroke,fill}%
\end{pgfscope}%
\begin{pgfscope}%
\pgfpathrectangle{\pgfqpoint{0.100000in}{0.212622in}}{\pgfqpoint{3.696000in}{3.696000in}}%
\pgfusepath{clip}%
\pgfsetbuttcap%
\pgfsetroundjoin%
\definecolor{currentfill}{rgb}{0.121569,0.466667,0.705882}%
\pgfsetfillcolor{currentfill}%
\pgfsetfillopacity{0.341576}%
\pgfsetlinewidth{1.003750pt}%
\definecolor{currentstroke}{rgb}{0.121569,0.466667,0.705882}%
\pgfsetstrokecolor{currentstroke}%
\pgfsetstrokeopacity{0.341576}%
\pgfsetdash{}{0pt}%
\pgfpathmoveto{\pgfqpoint{1.394946in}{1.709046in}}%
\pgfpathcurveto{\pgfqpoint{1.403182in}{1.709046in}}{\pgfqpoint{1.411082in}{1.712318in}}{\pgfqpoint{1.416906in}{1.718142in}}%
\pgfpathcurveto{\pgfqpoint{1.422730in}{1.723966in}}{\pgfqpoint{1.426003in}{1.731866in}}{\pgfqpoint{1.426003in}{1.740102in}}%
\pgfpathcurveto{\pgfqpoint{1.426003in}{1.748338in}}{\pgfqpoint{1.422730in}{1.756238in}}{\pgfqpoint{1.416906in}{1.762062in}}%
\pgfpathcurveto{\pgfqpoint{1.411082in}{1.767886in}}{\pgfqpoint{1.403182in}{1.771159in}}{\pgfqpoint{1.394946in}{1.771159in}}%
\pgfpathcurveto{\pgfqpoint{1.386710in}{1.771159in}}{\pgfqpoint{1.378810in}{1.767886in}}{\pgfqpoint{1.372986in}{1.762062in}}%
\pgfpathcurveto{\pgfqpoint{1.367162in}{1.756238in}}{\pgfqpoint{1.363890in}{1.748338in}}{\pgfqpoint{1.363890in}{1.740102in}}%
\pgfpathcurveto{\pgfqpoint{1.363890in}{1.731866in}}{\pgfqpoint{1.367162in}{1.723966in}}{\pgfqpoint{1.372986in}{1.718142in}}%
\pgfpathcurveto{\pgfqpoint{1.378810in}{1.712318in}}{\pgfqpoint{1.386710in}{1.709046in}}{\pgfqpoint{1.394946in}{1.709046in}}%
\pgfpathclose%
\pgfusepath{stroke,fill}%
\end{pgfscope}%
\begin{pgfscope}%
\pgfpathrectangle{\pgfqpoint{0.100000in}{0.212622in}}{\pgfqpoint{3.696000in}{3.696000in}}%
\pgfusepath{clip}%
\pgfsetbuttcap%
\pgfsetroundjoin%
\definecolor{currentfill}{rgb}{0.121569,0.466667,0.705882}%
\pgfsetfillcolor{currentfill}%
\pgfsetfillopacity{0.342201}%
\pgfsetlinewidth{1.003750pt}%
\definecolor{currentstroke}{rgb}{0.121569,0.466667,0.705882}%
\pgfsetstrokecolor{currentstroke}%
\pgfsetstrokeopacity{0.342201}%
\pgfsetdash{}{0pt}%
\pgfpathmoveto{\pgfqpoint{1.392939in}{1.707677in}}%
\pgfpathcurveto{\pgfqpoint{1.401175in}{1.707677in}}{\pgfqpoint{1.409075in}{1.710949in}}{\pgfqpoint{1.414899in}{1.716773in}}%
\pgfpathcurveto{\pgfqpoint{1.420723in}{1.722597in}}{\pgfqpoint{1.423995in}{1.730497in}}{\pgfqpoint{1.423995in}{1.738734in}}%
\pgfpathcurveto{\pgfqpoint{1.423995in}{1.746970in}}{\pgfqpoint{1.420723in}{1.754870in}}{\pgfqpoint{1.414899in}{1.760694in}}%
\pgfpathcurveto{\pgfqpoint{1.409075in}{1.766518in}}{\pgfqpoint{1.401175in}{1.769790in}}{\pgfqpoint{1.392939in}{1.769790in}}%
\pgfpathcurveto{\pgfqpoint{1.384702in}{1.769790in}}{\pgfqpoint{1.376802in}{1.766518in}}{\pgfqpoint{1.370978in}{1.760694in}}%
\pgfpathcurveto{\pgfqpoint{1.365154in}{1.754870in}}{\pgfqpoint{1.361882in}{1.746970in}}{\pgfqpoint{1.361882in}{1.738734in}}%
\pgfpathcurveto{\pgfqpoint{1.361882in}{1.730497in}}{\pgfqpoint{1.365154in}{1.722597in}}{\pgfqpoint{1.370978in}{1.716773in}}%
\pgfpathcurveto{\pgfqpoint{1.376802in}{1.710949in}}{\pgfqpoint{1.384702in}{1.707677in}}{\pgfqpoint{1.392939in}{1.707677in}}%
\pgfpathclose%
\pgfusepath{stroke,fill}%
\end{pgfscope}%
\begin{pgfscope}%
\pgfpathrectangle{\pgfqpoint{0.100000in}{0.212622in}}{\pgfqpoint{3.696000in}{3.696000in}}%
\pgfusepath{clip}%
\pgfsetbuttcap%
\pgfsetroundjoin%
\definecolor{currentfill}{rgb}{0.121569,0.466667,0.705882}%
\pgfsetfillcolor{currentfill}%
\pgfsetfillopacity{0.343797}%
\pgfsetlinewidth{1.003750pt}%
\definecolor{currentstroke}{rgb}{0.121569,0.466667,0.705882}%
\pgfsetstrokecolor{currentstroke}%
\pgfsetstrokeopacity{0.343797}%
\pgfsetdash{}{0pt}%
\pgfpathmoveto{\pgfqpoint{1.389625in}{1.706748in}}%
\pgfpathcurveto{\pgfqpoint{1.397862in}{1.706748in}}{\pgfqpoint{1.405762in}{1.710021in}}{\pgfqpoint{1.411586in}{1.715845in}}%
\pgfpathcurveto{\pgfqpoint{1.417410in}{1.721669in}}{\pgfqpoint{1.420682in}{1.729569in}}{\pgfqpoint{1.420682in}{1.737805in}}%
\pgfpathcurveto{\pgfqpoint{1.420682in}{1.746041in}}{\pgfqpoint{1.417410in}{1.753941in}}{\pgfqpoint{1.411586in}{1.759765in}}%
\pgfpathcurveto{\pgfqpoint{1.405762in}{1.765589in}}{\pgfqpoint{1.397862in}{1.768861in}}{\pgfqpoint{1.389625in}{1.768861in}}%
\pgfpathcurveto{\pgfqpoint{1.381389in}{1.768861in}}{\pgfqpoint{1.373489in}{1.765589in}}{\pgfqpoint{1.367665in}{1.759765in}}%
\pgfpathcurveto{\pgfqpoint{1.361841in}{1.753941in}}{\pgfqpoint{1.358569in}{1.746041in}}{\pgfqpoint{1.358569in}{1.737805in}}%
\pgfpathcurveto{\pgfqpoint{1.358569in}{1.729569in}}{\pgfqpoint{1.361841in}{1.721669in}}{\pgfqpoint{1.367665in}{1.715845in}}%
\pgfpathcurveto{\pgfqpoint{1.373489in}{1.710021in}}{\pgfqpoint{1.381389in}{1.706748in}}{\pgfqpoint{1.389625in}{1.706748in}}%
\pgfpathclose%
\pgfusepath{stroke,fill}%
\end{pgfscope}%
\begin{pgfscope}%
\pgfpathrectangle{\pgfqpoint{0.100000in}{0.212622in}}{\pgfqpoint{3.696000in}{3.696000in}}%
\pgfusepath{clip}%
\pgfsetbuttcap%
\pgfsetroundjoin%
\definecolor{currentfill}{rgb}{0.121569,0.466667,0.705882}%
\pgfsetfillcolor{currentfill}%
\pgfsetfillopacity{0.343855}%
\pgfsetlinewidth{1.003750pt}%
\definecolor{currentstroke}{rgb}{0.121569,0.466667,0.705882}%
\pgfsetstrokecolor{currentstroke}%
\pgfsetstrokeopacity{0.343855}%
\pgfsetdash{}{0pt}%
\pgfpathmoveto{\pgfqpoint{1.508196in}{1.736821in}}%
\pgfpathcurveto{\pgfqpoint{1.516432in}{1.736821in}}{\pgfqpoint{1.524332in}{1.740093in}}{\pgfqpoint{1.530156in}{1.745917in}}%
\pgfpathcurveto{\pgfqpoint{1.535980in}{1.751741in}}{\pgfqpoint{1.539252in}{1.759641in}}{\pgfqpoint{1.539252in}{1.767877in}}%
\pgfpathcurveto{\pgfqpoint{1.539252in}{1.776114in}}{\pgfqpoint{1.535980in}{1.784014in}}{\pgfqpoint{1.530156in}{1.789838in}}%
\pgfpathcurveto{\pgfqpoint{1.524332in}{1.795662in}}{\pgfqpoint{1.516432in}{1.798934in}}{\pgfqpoint{1.508196in}{1.798934in}}%
\pgfpathcurveto{\pgfqpoint{1.499959in}{1.798934in}}{\pgfqpoint{1.492059in}{1.795662in}}{\pgfqpoint{1.486235in}{1.789838in}}%
\pgfpathcurveto{\pgfqpoint{1.480411in}{1.784014in}}{\pgfqpoint{1.477139in}{1.776114in}}{\pgfqpoint{1.477139in}{1.767877in}}%
\pgfpathcurveto{\pgfqpoint{1.477139in}{1.759641in}}{\pgfqpoint{1.480411in}{1.751741in}}{\pgfqpoint{1.486235in}{1.745917in}}%
\pgfpathcurveto{\pgfqpoint{1.492059in}{1.740093in}}{\pgfqpoint{1.499959in}{1.736821in}}{\pgfqpoint{1.508196in}{1.736821in}}%
\pgfpathclose%
\pgfusepath{stroke,fill}%
\end{pgfscope}%
\begin{pgfscope}%
\pgfpathrectangle{\pgfqpoint{0.100000in}{0.212622in}}{\pgfqpoint{3.696000in}{3.696000in}}%
\pgfusepath{clip}%
\pgfsetbuttcap%
\pgfsetroundjoin%
\definecolor{currentfill}{rgb}{0.121569,0.466667,0.705882}%
\pgfsetfillcolor{currentfill}%
\pgfsetfillopacity{0.346993}%
\pgfsetlinewidth{1.003750pt}%
\definecolor{currentstroke}{rgb}{0.121569,0.466667,0.705882}%
\pgfsetstrokecolor{currentstroke}%
\pgfsetstrokeopacity{0.346993}%
\pgfsetdash{}{0pt}%
\pgfpathmoveto{\pgfqpoint{1.384420in}{1.705319in}}%
\pgfpathcurveto{\pgfqpoint{1.392656in}{1.705319in}}{\pgfqpoint{1.400556in}{1.708592in}}{\pgfqpoint{1.406380in}{1.714416in}}%
\pgfpathcurveto{\pgfqpoint{1.412204in}{1.720240in}}{\pgfqpoint{1.415476in}{1.728140in}}{\pgfqpoint{1.415476in}{1.736376in}}%
\pgfpathcurveto{\pgfqpoint{1.415476in}{1.744612in}}{\pgfqpoint{1.412204in}{1.752512in}}{\pgfqpoint{1.406380in}{1.758336in}}%
\pgfpathcurveto{\pgfqpoint{1.400556in}{1.764160in}}{\pgfqpoint{1.392656in}{1.767432in}}{\pgfqpoint{1.384420in}{1.767432in}}%
\pgfpathcurveto{\pgfqpoint{1.376184in}{1.767432in}}{\pgfqpoint{1.368284in}{1.764160in}}{\pgfqpoint{1.362460in}{1.758336in}}%
\pgfpathcurveto{\pgfqpoint{1.356636in}{1.752512in}}{\pgfqpoint{1.353363in}{1.744612in}}{\pgfqpoint{1.353363in}{1.736376in}}%
\pgfpathcurveto{\pgfqpoint{1.353363in}{1.728140in}}{\pgfqpoint{1.356636in}{1.720240in}}{\pgfqpoint{1.362460in}{1.714416in}}%
\pgfpathcurveto{\pgfqpoint{1.368284in}{1.708592in}}{\pgfqpoint{1.376184in}{1.705319in}}{\pgfqpoint{1.384420in}{1.705319in}}%
\pgfpathclose%
\pgfusepath{stroke,fill}%
\end{pgfscope}%
\begin{pgfscope}%
\pgfpathrectangle{\pgfqpoint{0.100000in}{0.212622in}}{\pgfqpoint{3.696000in}{3.696000in}}%
\pgfusepath{clip}%
\pgfsetbuttcap%
\pgfsetroundjoin%
\definecolor{currentfill}{rgb}{0.121569,0.466667,0.705882}%
\pgfsetfillcolor{currentfill}%
\pgfsetfillopacity{0.348140}%
\pgfsetlinewidth{1.003750pt}%
\definecolor{currentstroke}{rgb}{0.121569,0.466667,0.705882}%
\pgfsetstrokecolor{currentstroke}%
\pgfsetstrokeopacity{0.348140}%
\pgfsetdash{}{0pt}%
\pgfpathmoveto{\pgfqpoint{1.380227in}{1.699249in}}%
\pgfpathcurveto{\pgfqpoint{1.388464in}{1.699249in}}{\pgfqpoint{1.396364in}{1.702522in}}{\pgfqpoint{1.402188in}{1.708346in}}%
\pgfpathcurveto{\pgfqpoint{1.408012in}{1.714170in}}{\pgfqpoint{1.411284in}{1.722070in}}{\pgfqpoint{1.411284in}{1.730306in}}%
\pgfpathcurveto{\pgfqpoint{1.411284in}{1.738542in}}{\pgfqpoint{1.408012in}{1.746442in}}{\pgfqpoint{1.402188in}{1.752266in}}%
\pgfpathcurveto{\pgfqpoint{1.396364in}{1.758090in}}{\pgfqpoint{1.388464in}{1.761362in}}{\pgfqpoint{1.380227in}{1.761362in}}%
\pgfpathcurveto{\pgfqpoint{1.371991in}{1.761362in}}{\pgfqpoint{1.364091in}{1.758090in}}{\pgfqpoint{1.358267in}{1.752266in}}%
\pgfpathcurveto{\pgfqpoint{1.352443in}{1.746442in}}{\pgfqpoint{1.349171in}{1.738542in}}{\pgfqpoint{1.349171in}{1.730306in}}%
\pgfpathcurveto{\pgfqpoint{1.349171in}{1.722070in}}{\pgfqpoint{1.352443in}{1.714170in}}{\pgfqpoint{1.358267in}{1.708346in}}%
\pgfpathcurveto{\pgfqpoint{1.364091in}{1.702522in}}{\pgfqpoint{1.371991in}{1.699249in}}{\pgfqpoint{1.380227in}{1.699249in}}%
\pgfpathclose%
\pgfusepath{stroke,fill}%
\end{pgfscope}%
\begin{pgfscope}%
\pgfpathrectangle{\pgfqpoint{0.100000in}{0.212622in}}{\pgfqpoint{3.696000in}{3.696000in}}%
\pgfusepath{clip}%
\pgfsetbuttcap%
\pgfsetroundjoin%
\definecolor{currentfill}{rgb}{0.121569,0.466667,0.705882}%
\pgfsetfillcolor{currentfill}%
\pgfsetfillopacity{0.348640}%
\pgfsetlinewidth{1.003750pt}%
\definecolor{currentstroke}{rgb}{0.121569,0.466667,0.705882}%
\pgfsetstrokecolor{currentstroke}%
\pgfsetstrokeopacity{0.348640}%
\pgfsetdash{}{0pt}%
\pgfpathmoveto{\pgfqpoint{1.510922in}{1.741011in}}%
\pgfpathcurveto{\pgfqpoint{1.519159in}{1.741011in}}{\pgfqpoint{1.527059in}{1.744283in}}{\pgfqpoint{1.532882in}{1.750107in}}%
\pgfpathcurveto{\pgfqpoint{1.538706in}{1.755931in}}{\pgfqpoint{1.541979in}{1.763831in}}{\pgfqpoint{1.541979in}{1.772068in}}%
\pgfpathcurveto{\pgfqpoint{1.541979in}{1.780304in}}{\pgfqpoint{1.538706in}{1.788204in}}{\pgfqpoint{1.532882in}{1.794028in}}%
\pgfpathcurveto{\pgfqpoint{1.527059in}{1.799852in}}{\pgfqpoint{1.519159in}{1.803124in}}{\pgfqpoint{1.510922in}{1.803124in}}%
\pgfpathcurveto{\pgfqpoint{1.502686in}{1.803124in}}{\pgfqpoint{1.494786in}{1.799852in}}{\pgfqpoint{1.488962in}{1.794028in}}%
\pgfpathcurveto{\pgfqpoint{1.483138in}{1.788204in}}{\pgfqpoint{1.479866in}{1.780304in}}{\pgfqpoint{1.479866in}{1.772068in}}%
\pgfpathcurveto{\pgfqpoint{1.479866in}{1.763831in}}{\pgfqpoint{1.483138in}{1.755931in}}{\pgfqpoint{1.488962in}{1.750107in}}%
\pgfpathcurveto{\pgfqpoint{1.494786in}{1.744283in}}{\pgfqpoint{1.502686in}{1.741011in}}{\pgfqpoint{1.510922in}{1.741011in}}%
\pgfpathclose%
\pgfusepath{stroke,fill}%
\end{pgfscope}%
\begin{pgfscope}%
\pgfpathrectangle{\pgfqpoint{0.100000in}{0.212622in}}{\pgfqpoint{3.696000in}{3.696000in}}%
\pgfusepath{clip}%
\pgfsetbuttcap%
\pgfsetroundjoin%
\definecolor{currentfill}{rgb}{0.121569,0.466667,0.705882}%
\pgfsetfillcolor{currentfill}%
\pgfsetfillopacity{0.349157}%
\pgfsetlinewidth{1.003750pt}%
\definecolor{currentstroke}{rgb}{0.121569,0.466667,0.705882}%
\pgfsetstrokecolor{currentstroke}%
\pgfsetstrokeopacity{0.349157}%
\pgfsetdash{}{0pt}%
\pgfpathmoveto{\pgfqpoint{1.378015in}{1.698322in}}%
\pgfpathcurveto{\pgfqpoint{1.386251in}{1.698322in}}{\pgfqpoint{1.394151in}{1.701594in}}{\pgfqpoint{1.399975in}{1.707418in}}%
\pgfpathcurveto{\pgfqpoint{1.405799in}{1.713242in}}{\pgfqpoint{1.409071in}{1.721142in}}{\pgfqpoint{1.409071in}{1.729378in}}%
\pgfpathcurveto{\pgfqpoint{1.409071in}{1.737614in}}{\pgfqpoint{1.405799in}{1.745514in}}{\pgfqpoint{1.399975in}{1.751338in}}%
\pgfpathcurveto{\pgfqpoint{1.394151in}{1.757162in}}{\pgfqpoint{1.386251in}{1.760435in}}{\pgfqpoint{1.378015in}{1.760435in}}%
\pgfpathcurveto{\pgfqpoint{1.369778in}{1.760435in}}{\pgfqpoint{1.361878in}{1.757162in}}{\pgfqpoint{1.356054in}{1.751338in}}%
\pgfpathcurveto{\pgfqpoint{1.350230in}{1.745514in}}{\pgfqpoint{1.346958in}{1.737614in}}{\pgfqpoint{1.346958in}{1.729378in}}%
\pgfpathcurveto{\pgfqpoint{1.346958in}{1.721142in}}{\pgfqpoint{1.350230in}{1.713242in}}{\pgfqpoint{1.356054in}{1.707418in}}%
\pgfpathcurveto{\pgfqpoint{1.361878in}{1.701594in}}{\pgfqpoint{1.369778in}{1.698322in}}{\pgfqpoint{1.378015in}{1.698322in}}%
\pgfpathclose%
\pgfusepath{stroke,fill}%
\end{pgfscope}%
\begin{pgfscope}%
\pgfpathrectangle{\pgfqpoint{0.100000in}{0.212622in}}{\pgfqpoint{3.696000in}{3.696000in}}%
\pgfusepath{clip}%
\pgfsetbuttcap%
\pgfsetroundjoin%
\definecolor{currentfill}{rgb}{0.121569,0.466667,0.705882}%
\pgfsetfillcolor{currentfill}%
\pgfsetfillopacity{0.349263}%
\pgfsetlinewidth{1.003750pt}%
\definecolor{currentstroke}{rgb}{0.121569,0.466667,0.705882}%
\pgfsetstrokecolor{currentstroke}%
\pgfsetstrokeopacity{0.349263}%
\pgfsetdash{}{0pt}%
\pgfpathmoveto{\pgfqpoint{1.368447in}{1.684917in}}%
\pgfpathcurveto{\pgfqpoint{1.376683in}{1.684917in}}{\pgfqpoint{1.384583in}{1.688189in}}{\pgfqpoint{1.390407in}{1.694013in}}%
\pgfpathcurveto{\pgfqpoint{1.396231in}{1.699837in}}{\pgfqpoint{1.399504in}{1.707737in}}{\pgfqpoint{1.399504in}{1.715973in}}%
\pgfpathcurveto{\pgfqpoint{1.399504in}{1.724210in}}{\pgfqpoint{1.396231in}{1.732110in}}{\pgfqpoint{1.390407in}{1.737934in}}%
\pgfpathcurveto{\pgfqpoint{1.384583in}{1.743758in}}{\pgfqpoint{1.376683in}{1.747030in}}{\pgfqpoint{1.368447in}{1.747030in}}%
\pgfpathcurveto{\pgfqpoint{1.360211in}{1.747030in}}{\pgfqpoint{1.352311in}{1.743758in}}{\pgfqpoint{1.346487in}{1.737934in}}%
\pgfpathcurveto{\pgfqpoint{1.340663in}{1.732110in}}{\pgfqpoint{1.337391in}{1.724210in}}{\pgfqpoint{1.337391in}{1.715973in}}%
\pgfpathcurveto{\pgfqpoint{1.337391in}{1.707737in}}{\pgfqpoint{1.340663in}{1.699837in}}{\pgfqpoint{1.346487in}{1.694013in}}%
\pgfpathcurveto{\pgfqpoint{1.352311in}{1.688189in}}{\pgfqpoint{1.360211in}{1.684917in}}{\pgfqpoint{1.368447in}{1.684917in}}%
\pgfpathclose%
\pgfusepath{stroke,fill}%
\end{pgfscope}%
\begin{pgfscope}%
\pgfpathrectangle{\pgfqpoint{0.100000in}{0.212622in}}{\pgfqpoint{3.696000in}{3.696000in}}%
\pgfusepath{clip}%
\pgfsetbuttcap%
\pgfsetroundjoin%
\definecolor{currentfill}{rgb}{0.121569,0.466667,0.705882}%
\pgfsetfillcolor{currentfill}%
\pgfsetfillopacity{0.349612}%
\pgfsetlinewidth{1.003750pt}%
\definecolor{currentstroke}{rgb}{0.121569,0.466667,0.705882}%
\pgfsetstrokecolor{currentstroke}%
\pgfsetstrokeopacity{0.349612}%
\pgfsetdash{}{0pt}%
\pgfpathmoveto{\pgfqpoint{1.376874in}{1.697251in}}%
\pgfpathcurveto{\pgfqpoint{1.385110in}{1.697251in}}{\pgfqpoint{1.393010in}{1.700524in}}{\pgfqpoint{1.398834in}{1.706348in}}%
\pgfpathcurveto{\pgfqpoint{1.404658in}{1.712172in}}{\pgfqpoint{1.407931in}{1.720072in}}{\pgfqpoint{1.407931in}{1.728308in}}%
\pgfpathcurveto{\pgfqpoint{1.407931in}{1.736544in}}{\pgfqpoint{1.404658in}{1.744444in}}{\pgfqpoint{1.398834in}{1.750268in}}%
\pgfpathcurveto{\pgfqpoint{1.393010in}{1.756092in}}{\pgfqpoint{1.385110in}{1.759364in}}{\pgfqpoint{1.376874in}{1.759364in}}%
\pgfpathcurveto{\pgfqpoint{1.368638in}{1.759364in}}{\pgfqpoint{1.360738in}{1.756092in}}{\pgfqpoint{1.354914in}{1.750268in}}%
\pgfpathcurveto{\pgfqpoint{1.349090in}{1.744444in}}{\pgfqpoint{1.345818in}{1.736544in}}{\pgfqpoint{1.345818in}{1.728308in}}%
\pgfpathcurveto{\pgfqpoint{1.345818in}{1.720072in}}{\pgfqpoint{1.349090in}{1.712172in}}{\pgfqpoint{1.354914in}{1.706348in}}%
\pgfpathcurveto{\pgfqpoint{1.360738in}{1.700524in}}{\pgfqpoint{1.368638in}{1.697251in}}{\pgfqpoint{1.376874in}{1.697251in}}%
\pgfpathclose%
\pgfusepath{stroke,fill}%
\end{pgfscope}%
\begin{pgfscope}%
\pgfpathrectangle{\pgfqpoint{0.100000in}{0.212622in}}{\pgfqpoint{3.696000in}{3.696000in}}%
\pgfusepath{clip}%
\pgfsetbuttcap%
\pgfsetroundjoin%
\definecolor{currentfill}{rgb}{0.121569,0.466667,0.705882}%
\pgfsetfillcolor{currentfill}%
\pgfsetfillopacity{0.350020}%
\pgfsetlinewidth{1.003750pt}%
\definecolor{currentstroke}{rgb}{0.121569,0.466667,0.705882}%
\pgfsetstrokecolor{currentstroke}%
\pgfsetstrokeopacity{0.350020}%
\pgfsetdash{}{0pt}%
\pgfpathmoveto{\pgfqpoint{1.373928in}{1.695118in}}%
\pgfpathcurveto{\pgfqpoint{1.382164in}{1.695118in}}{\pgfqpoint{1.390064in}{1.698391in}}{\pgfqpoint{1.395888in}{1.704215in}}%
\pgfpathcurveto{\pgfqpoint{1.401712in}{1.710039in}}{\pgfqpoint{1.404984in}{1.717939in}}{\pgfqpoint{1.404984in}{1.726175in}}%
\pgfpathcurveto{\pgfqpoint{1.404984in}{1.734411in}}{\pgfqpoint{1.401712in}{1.742311in}}{\pgfqpoint{1.395888in}{1.748135in}}%
\pgfpathcurveto{\pgfqpoint{1.390064in}{1.753959in}}{\pgfqpoint{1.382164in}{1.757231in}}{\pgfqpoint{1.373928in}{1.757231in}}%
\pgfpathcurveto{\pgfqpoint{1.365691in}{1.757231in}}{\pgfqpoint{1.357791in}{1.753959in}}{\pgfqpoint{1.351967in}{1.748135in}}%
\pgfpathcurveto{\pgfqpoint{1.346143in}{1.742311in}}{\pgfqpoint{1.342871in}{1.734411in}}{\pgfqpoint{1.342871in}{1.726175in}}%
\pgfpathcurveto{\pgfqpoint{1.342871in}{1.717939in}}{\pgfqpoint{1.346143in}{1.710039in}}{\pgfqpoint{1.351967in}{1.704215in}}%
\pgfpathcurveto{\pgfqpoint{1.357791in}{1.698391in}}{\pgfqpoint{1.365691in}{1.695118in}}{\pgfqpoint{1.373928in}{1.695118in}}%
\pgfpathclose%
\pgfusepath{stroke,fill}%
\end{pgfscope}%
\begin{pgfscope}%
\pgfpathrectangle{\pgfqpoint{0.100000in}{0.212622in}}{\pgfqpoint{3.696000in}{3.696000in}}%
\pgfusepath{clip}%
\pgfsetbuttcap%
\pgfsetroundjoin%
\definecolor{currentfill}{rgb}{0.121569,0.466667,0.705882}%
\pgfsetfillcolor{currentfill}%
\pgfsetfillopacity{0.351501}%
\pgfsetlinewidth{1.003750pt}%
\definecolor{currentstroke}{rgb}{0.121569,0.466667,0.705882}%
\pgfsetstrokecolor{currentstroke}%
\pgfsetstrokeopacity{0.351501}%
\pgfsetdash{}{0pt}%
\pgfpathmoveto{\pgfqpoint{1.365910in}{1.686437in}}%
\pgfpathcurveto{\pgfqpoint{1.374146in}{1.686437in}}{\pgfqpoint{1.382046in}{1.689710in}}{\pgfqpoint{1.387870in}{1.695534in}}%
\pgfpathcurveto{\pgfqpoint{1.393694in}{1.701357in}}{\pgfqpoint{1.396966in}{1.709257in}}{\pgfqpoint{1.396966in}{1.717494in}}%
\pgfpathcurveto{\pgfqpoint{1.396966in}{1.725730in}}{\pgfqpoint{1.393694in}{1.733630in}}{\pgfqpoint{1.387870in}{1.739454in}}%
\pgfpathcurveto{\pgfqpoint{1.382046in}{1.745278in}}{\pgfqpoint{1.374146in}{1.748550in}}{\pgfqpoint{1.365910in}{1.748550in}}%
\pgfpathcurveto{\pgfqpoint{1.357673in}{1.748550in}}{\pgfqpoint{1.349773in}{1.745278in}}{\pgfqpoint{1.343949in}{1.739454in}}%
\pgfpathcurveto{\pgfqpoint{1.338125in}{1.733630in}}{\pgfqpoint{1.334853in}{1.725730in}}{\pgfqpoint{1.334853in}{1.717494in}}%
\pgfpathcurveto{\pgfqpoint{1.334853in}{1.709257in}}{\pgfqpoint{1.338125in}{1.701357in}}{\pgfqpoint{1.343949in}{1.695534in}}%
\pgfpathcurveto{\pgfqpoint{1.349773in}{1.689710in}}{\pgfqpoint{1.357673in}{1.686437in}}{\pgfqpoint{1.365910in}{1.686437in}}%
\pgfpathclose%
\pgfusepath{stroke,fill}%
\end{pgfscope}%
\begin{pgfscope}%
\pgfpathrectangle{\pgfqpoint{0.100000in}{0.212622in}}{\pgfqpoint{3.696000in}{3.696000in}}%
\pgfusepath{clip}%
\pgfsetbuttcap%
\pgfsetroundjoin%
\definecolor{currentfill}{rgb}{0.121569,0.466667,0.705882}%
\pgfsetfillcolor{currentfill}%
\pgfsetfillopacity{0.351577}%
\pgfsetlinewidth{1.003750pt}%
\definecolor{currentstroke}{rgb}{0.121569,0.466667,0.705882}%
\pgfsetstrokecolor{currentstroke}%
\pgfsetstrokeopacity{0.351577}%
\pgfsetdash{}{0pt}%
\pgfpathmoveto{\pgfqpoint{1.513834in}{1.732208in}}%
\pgfpathcurveto{\pgfqpoint{1.522070in}{1.732208in}}{\pgfqpoint{1.529970in}{1.735480in}}{\pgfqpoint{1.535794in}{1.741304in}}%
\pgfpathcurveto{\pgfqpoint{1.541618in}{1.747128in}}{\pgfqpoint{1.544890in}{1.755028in}}{\pgfqpoint{1.544890in}{1.763264in}}%
\pgfpathcurveto{\pgfqpoint{1.544890in}{1.771501in}}{\pgfqpoint{1.541618in}{1.779401in}}{\pgfqpoint{1.535794in}{1.785225in}}%
\pgfpathcurveto{\pgfqpoint{1.529970in}{1.791048in}}{\pgfqpoint{1.522070in}{1.794321in}}{\pgfqpoint{1.513834in}{1.794321in}}%
\pgfpathcurveto{\pgfqpoint{1.505597in}{1.794321in}}{\pgfqpoint{1.497697in}{1.791048in}}{\pgfqpoint{1.491873in}{1.785225in}}%
\pgfpathcurveto{\pgfqpoint{1.486050in}{1.779401in}}{\pgfqpoint{1.482777in}{1.771501in}}{\pgfqpoint{1.482777in}{1.763264in}}%
\pgfpathcurveto{\pgfqpoint{1.482777in}{1.755028in}}{\pgfqpoint{1.486050in}{1.747128in}}{\pgfqpoint{1.491873in}{1.741304in}}%
\pgfpathcurveto{\pgfqpoint{1.497697in}{1.735480in}}{\pgfqpoint{1.505597in}{1.732208in}}{\pgfqpoint{1.513834in}{1.732208in}}%
\pgfpathclose%
\pgfusepath{stroke,fill}%
\end{pgfscope}%
\begin{pgfscope}%
\pgfpathrectangle{\pgfqpoint{0.100000in}{0.212622in}}{\pgfqpoint{3.696000in}{3.696000in}}%
\pgfusepath{clip}%
\pgfsetbuttcap%
\pgfsetroundjoin%
\definecolor{currentfill}{rgb}{0.121569,0.466667,0.705882}%
\pgfsetfillcolor{currentfill}%
\pgfsetfillopacity{0.352332}%
\pgfsetlinewidth{1.003750pt}%
\definecolor{currentstroke}{rgb}{0.121569,0.466667,0.705882}%
\pgfsetstrokecolor{currentstroke}%
\pgfsetstrokeopacity{0.352332}%
\pgfsetdash{}{0pt}%
\pgfpathmoveto{\pgfqpoint{1.362916in}{1.683958in}}%
\pgfpathcurveto{\pgfqpoint{1.371152in}{1.683958in}}{\pgfqpoint{1.379052in}{1.687230in}}{\pgfqpoint{1.384876in}{1.693054in}}%
\pgfpathcurveto{\pgfqpoint{1.390700in}{1.698878in}}{\pgfqpoint{1.393972in}{1.706778in}}{\pgfqpoint{1.393972in}{1.715015in}}%
\pgfpathcurveto{\pgfqpoint{1.393972in}{1.723251in}}{\pgfqpoint{1.390700in}{1.731151in}}{\pgfqpoint{1.384876in}{1.736975in}}%
\pgfpathcurveto{\pgfqpoint{1.379052in}{1.742799in}}{\pgfqpoint{1.371152in}{1.746071in}}{\pgfqpoint{1.362916in}{1.746071in}}%
\pgfpathcurveto{\pgfqpoint{1.354680in}{1.746071in}}{\pgfqpoint{1.346779in}{1.742799in}}{\pgfqpoint{1.340956in}{1.736975in}}%
\pgfpathcurveto{\pgfqpoint{1.335132in}{1.731151in}}{\pgfqpoint{1.331859in}{1.723251in}}{\pgfqpoint{1.331859in}{1.715015in}}%
\pgfpathcurveto{\pgfqpoint{1.331859in}{1.706778in}}{\pgfqpoint{1.335132in}{1.698878in}}{\pgfqpoint{1.340956in}{1.693054in}}%
\pgfpathcurveto{\pgfqpoint{1.346779in}{1.687230in}}{\pgfqpoint{1.354680in}{1.683958in}}{\pgfqpoint{1.362916in}{1.683958in}}%
\pgfpathclose%
\pgfusepath{stroke,fill}%
\end{pgfscope}%
\begin{pgfscope}%
\pgfpathrectangle{\pgfqpoint{0.100000in}{0.212622in}}{\pgfqpoint{3.696000in}{3.696000in}}%
\pgfusepath{clip}%
\pgfsetbuttcap%
\pgfsetroundjoin%
\definecolor{currentfill}{rgb}{0.121569,0.466667,0.705882}%
\pgfsetfillcolor{currentfill}%
\pgfsetfillopacity{0.353521}%
\pgfsetlinewidth{1.003750pt}%
\definecolor{currentstroke}{rgb}{0.121569,0.466667,0.705882}%
\pgfsetstrokecolor{currentstroke}%
\pgfsetstrokeopacity{0.353521}%
\pgfsetdash{}{0pt}%
\pgfpathmoveto{\pgfqpoint{1.357736in}{1.677509in}}%
\pgfpathcurveto{\pgfqpoint{1.365972in}{1.677509in}}{\pgfqpoint{1.373872in}{1.680781in}}{\pgfqpoint{1.379696in}{1.686605in}}%
\pgfpathcurveto{\pgfqpoint{1.385520in}{1.692429in}}{\pgfqpoint{1.388792in}{1.700329in}}{\pgfqpoint{1.388792in}{1.708565in}}%
\pgfpathcurveto{\pgfqpoint{1.388792in}{1.716802in}}{\pgfqpoint{1.385520in}{1.724702in}}{\pgfqpoint{1.379696in}{1.730526in}}%
\pgfpathcurveto{\pgfqpoint{1.373872in}{1.736350in}}{\pgfqpoint{1.365972in}{1.739622in}}{\pgfqpoint{1.357736in}{1.739622in}}%
\pgfpathcurveto{\pgfqpoint{1.349500in}{1.739622in}}{\pgfqpoint{1.341599in}{1.736350in}}{\pgfqpoint{1.335776in}{1.730526in}}%
\pgfpathcurveto{\pgfqpoint{1.329952in}{1.724702in}}{\pgfqpoint{1.326679in}{1.716802in}}{\pgfqpoint{1.326679in}{1.708565in}}%
\pgfpathcurveto{\pgfqpoint{1.326679in}{1.700329in}}{\pgfqpoint{1.329952in}{1.692429in}}{\pgfqpoint{1.335776in}{1.686605in}}%
\pgfpathcurveto{\pgfqpoint{1.341599in}{1.680781in}}{\pgfqpoint{1.349500in}{1.677509in}}{\pgfqpoint{1.357736in}{1.677509in}}%
\pgfpathclose%
\pgfusepath{stroke,fill}%
\end{pgfscope}%
\begin{pgfscope}%
\pgfpathrectangle{\pgfqpoint{0.100000in}{0.212622in}}{\pgfqpoint{3.696000in}{3.696000in}}%
\pgfusepath{clip}%
\pgfsetbuttcap%
\pgfsetroundjoin%
\definecolor{currentfill}{rgb}{0.121569,0.466667,0.705882}%
\pgfsetfillcolor{currentfill}%
\pgfsetfillopacity{0.356246}%
\pgfsetlinewidth{1.003750pt}%
\definecolor{currentstroke}{rgb}{0.121569,0.466667,0.705882}%
\pgfsetstrokecolor{currentstroke}%
\pgfsetstrokeopacity{0.356246}%
\pgfsetdash{}{0pt}%
\pgfpathmoveto{\pgfqpoint{1.353536in}{1.677559in}}%
\pgfpathcurveto{\pgfqpoint{1.361772in}{1.677559in}}{\pgfqpoint{1.369672in}{1.680831in}}{\pgfqpoint{1.375496in}{1.686655in}}%
\pgfpathcurveto{\pgfqpoint{1.381320in}{1.692479in}}{\pgfqpoint{1.384593in}{1.700379in}}{\pgfqpoint{1.384593in}{1.708615in}}%
\pgfpathcurveto{\pgfqpoint{1.384593in}{1.716852in}}{\pgfqpoint{1.381320in}{1.724752in}}{\pgfqpoint{1.375496in}{1.730576in}}%
\pgfpathcurveto{\pgfqpoint{1.369672in}{1.736400in}}{\pgfqpoint{1.361772in}{1.739672in}}{\pgfqpoint{1.353536in}{1.739672in}}%
\pgfpathcurveto{\pgfqpoint{1.345300in}{1.739672in}}{\pgfqpoint{1.337400in}{1.736400in}}{\pgfqpoint{1.331576in}{1.730576in}}%
\pgfpathcurveto{\pgfqpoint{1.325752in}{1.724752in}}{\pgfqpoint{1.322480in}{1.716852in}}{\pgfqpoint{1.322480in}{1.708615in}}%
\pgfpathcurveto{\pgfqpoint{1.322480in}{1.700379in}}{\pgfqpoint{1.325752in}{1.692479in}}{\pgfqpoint{1.331576in}{1.686655in}}%
\pgfpathcurveto{\pgfqpoint{1.337400in}{1.680831in}}{\pgfqpoint{1.345300in}{1.677559in}}{\pgfqpoint{1.353536in}{1.677559in}}%
\pgfpathclose%
\pgfusepath{stroke,fill}%
\end{pgfscope}%
\begin{pgfscope}%
\pgfpathrectangle{\pgfqpoint{0.100000in}{0.212622in}}{\pgfqpoint{3.696000in}{3.696000in}}%
\pgfusepath{clip}%
\pgfsetbuttcap%
\pgfsetroundjoin%
\definecolor{currentfill}{rgb}{0.121569,0.466667,0.705882}%
\pgfsetfillcolor{currentfill}%
\pgfsetfillopacity{0.356407}%
\pgfsetlinewidth{1.003750pt}%
\definecolor{currentstroke}{rgb}{0.121569,0.466667,0.705882}%
\pgfsetstrokecolor{currentstroke}%
\pgfsetstrokeopacity{0.356407}%
\pgfsetdash{}{0pt}%
\pgfpathmoveto{\pgfqpoint{1.515030in}{1.730420in}}%
\pgfpathcurveto{\pgfqpoint{1.523266in}{1.730420in}}{\pgfqpoint{1.531166in}{1.733692in}}{\pgfqpoint{1.536990in}{1.739516in}}%
\pgfpathcurveto{\pgfqpoint{1.542814in}{1.745340in}}{\pgfqpoint{1.546087in}{1.753240in}}{\pgfqpoint{1.546087in}{1.761477in}}%
\pgfpathcurveto{\pgfqpoint{1.546087in}{1.769713in}}{\pgfqpoint{1.542814in}{1.777613in}}{\pgfqpoint{1.536990in}{1.783437in}}%
\pgfpathcurveto{\pgfqpoint{1.531166in}{1.789261in}}{\pgfqpoint{1.523266in}{1.792533in}}{\pgfqpoint{1.515030in}{1.792533in}}%
\pgfpathcurveto{\pgfqpoint{1.506794in}{1.792533in}}{\pgfqpoint{1.498894in}{1.789261in}}{\pgfqpoint{1.493070in}{1.783437in}}%
\pgfpathcurveto{\pgfqpoint{1.487246in}{1.777613in}}{\pgfqpoint{1.483974in}{1.769713in}}{\pgfqpoint{1.483974in}{1.761477in}}%
\pgfpathcurveto{\pgfqpoint{1.483974in}{1.753240in}}{\pgfqpoint{1.487246in}{1.745340in}}{\pgfqpoint{1.493070in}{1.739516in}}%
\pgfpathcurveto{\pgfqpoint{1.498894in}{1.733692in}}{\pgfqpoint{1.506794in}{1.730420in}}{\pgfqpoint{1.515030in}{1.730420in}}%
\pgfpathclose%
\pgfusepath{stroke,fill}%
\end{pgfscope}%
\begin{pgfscope}%
\pgfpathrectangle{\pgfqpoint{0.100000in}{0.212622in}}{\pgfqpoint{3.696000in}{3.696000in}}%
\pgfusepath{clip}%
\pgfsetbuttcap%
\pgfsetroundjoin%
\definecolor{currentfill}{rgb}{0.121569,0.466667,0.705882}%
\pgfsetfillcolor{currentfill}%
\pgfsetfillopacity{0.357465}%
\pgfsetlinewidth{1.003750pt}%
\definecolor{currentstroke}{rgb}{0.121569,0.466667,0.705882}%
\pgfsetstrokecolor{currentstroke}%
\pgfsetstrokeopacity{0.357465}%
\pgfsetdash{}{0pt}%
\pgfpathmoveto{\pgfqpoint{1.350118in}{1.675776in}}%
\pgfpathcurveto{\pgfqpoint{1.358355in}{1.675776in}}{\pgfqpoint{1.366255in}{1.679049in}}{\pgfqpoint{1.372078in}{1.684873in}}%
\pgfpathcurveto{\pgfqpoint{1.377902in}{1.690697in}}{\pgfqpoint{1.381175in}{1.698597in}}{\pgfqpoint{1.381175in}{1.706833in}}%
\pgfpathcurveto{\pgfqpoint{1.381175in}{1.715069in}}{\pgfqpoint{1.377902in}{1.722969in}}{\pgfqpoint{1.372078in}{1.728793in}}%
\pgfpathcurveto{\pgfqpoint{1.366255in}{1.734617in}}{\pgfqpoint{1.358355in}{1.737889in}}{\pgfqpoint{1.350118in}{1.737889in}}%
\pgfpathcurveto{\pgfqpoint{1.341882in}{1.737889in}}{\pgfqpoint{1.333982in}{1.734617in}}{\pgfqpoint{1.328158in}{1.728793in}}%
\pgfpathcurveto{\pgfqpoint{1.322334in}{1.722969in}}{\pgfqpoint{1.319062in}{1.715069in}}{\pgfqpoint{1.319062in}{1.706833in}}%
\pgfpathcurveto{\pgfqpoint{1.319062in}{1.698597in}}{\pgfqpoint{1.322334in}{1.690697in}}{\pgfqpoint{1.328158in}{1.684873in}}%
\pgfpathcurveto{\pgfqpoint{1.333982in}{1.679049in}}{\pgfqpoint{1.341882in}{1.675776in}}{\pgfqpoint{1.350118in}{1.675776in}}%
\pgfpathclose%
\pgfusepath{stroke,fill}%
\end{pgfscope}%
\begin{pgfscope}%
\pgfpathrectangle{\pgfqpoint{0.100000in}{0.212622in}}{\pgfqpoint{3.696000in}{3.696000in}}%
\pgfusepath{clip}%
\pgfsetbuttcap%
\pgfsetroundjoin%
\definecolor{currentfill}{rgb}{0.121569,0.466667,0.705882}%
\pgfsetfillcolor{currentfill}%
\pgfsetfillopacity{0.358807}%
\pgfsetlinewidth{1.003750pt}%
\definecolor{currentstroke}{rgb}{0.121569,0.466667,0.705882}%
\pgfsetstrokecolor{currentstroke}%
\pgfsetstrokeopacity{0.358807}%
\pgfsetdash{}{0pt}%
\pgfpathmoveto{\pgfqpoint{1.347300in}{1.674919in}}%
\pgfpathcurveto{\pgfqpoint{1.355537in}{1.674919in}}{\pgfqpoint{1.363437in}{1.678191in}}{\pgfqpoint{1.369260in}{1.684015in}}%
\pgfpathcurveto{\pgfqpoint{1.375084in}{1.689839in}}{\pgfqpoint{1.378357in}{1.697739in}}{\pgfqpoint{1.378357in}{1.705976in}}%
\pgfpathcurveto{\pgfqpoint{1.378357in}{1.714212in}}{\pgfqpoint{1.375084in}{1.722112in}}{\pgfqpoint{1.369260in}{1.727936in}}%
\pgfpathcurveto{\pgfqpoint{1.363437in}{1.733760in}}{\pgfqpoint{1.355537in}{1.737032in}}{\pgfqpoint{1.347300in}{1.737032in}}%
\pgfpathcurveto{\pgfqpoint{1.339064in}{1.737032in}}{\pgfqpoint{1.331164in}{1.733760in}}{\pgfqpoint{1.325340in}{1.727936in}}%
\pgfpathcurveto{\pgfqpoint{1.319516in}{1.722112in}}{\pgfqpoint{1.316244in}{1.714212in}}{\pgfqpoint{1.316244in}{1.705976in}}%
\pgfpathcurveto{\pgfqpoint{1.316244in}{1.697739in}}{\pgfqpoint{1.319516in}{1.689839in}}{\pgfqpoint{1.325340in}{1.684015in}}%
\pgfpathcurveto{\pgfqpoint{1.331164in}{1.678191in}}{\pgfqpoint{1.339064in}{1.674919in}}{\pgfqpoint{1.347300in}{1.674919in}}%
\pgfpathclose%
\pgfusepath{stroke,fill}%
\end{pgfscope}%
\begin{pgfscope}%
\pgfpathrectangle{\pgfqpoint{0.100000in}{0.212622in}}{\pgfqpoint{3.696000in}{3.696000in}}%
\pgfusepath{clip}%
\pgfsetbuttcap%
\pgfsetroundjoin%
\definecolor{currentfill}{rgb}{0.121569,0.466667,0.705882}%
\pgfsetfillcolor{currentfill}%
\pgfsetfillopacity{0.361433}%
\pgfsetlinewidth{1.003750pt}%
\definecolor{currentstroke}{rgb}{0.121569,0.466667,0.705882}%
\pgfsetstrokecolor{currentstroke}%
\pgfsetstrokeopacity{0.361433}%
\pgfsetdash{}{0pt}%
\pgfpathmoveto{\pgfqpoint{1.342535in}{1.673690in}}%
\pgfpathcurveto{\pgfqpoint{1.350771in}{1.673690in}}{\pgfqpoint{1.358671in}{1.676963in}}{\pgfqpoint{1.364495in}{1.682787in}}%
\pgfpathcurveto{\pgfqpoint{1.370319in}{1.688611in}}{\pgfqpoint{1.373592in}{1.696511in}}{\pgfqpoint{1.373592in}{1.704747in}}%
\pgfpathcurveto{\pgfqpoint{1.373592in}{1.712983in}}{\pgfqpoint{1.370319in}{1.720883in}}{\pgfqpoint{1.364495in}{1.726707in}}%
\pgfpathcurveto{\pgfqpoint{1.358671in}{1.732531in}}{\pgfqpoint{1.350771in}{1.735803in}}{\pgfqpoint{1.342535in}{1.735803in}}%
\pgfpathcurveto{\pgfqpoint{1.334299in}{1.735803in}}{\pgfqpoint{1.326399in}{1.732531in}}{\pgfqpoint{1.320575in}{1.726707in}}%
\pgfpathcurveto{\pgfqpoint{1.314751in}{1.720883in}}{\pgfqpoint{1.311479in}{1.712983in}}{\pgfqpoint{1.311479in}{1.704747in}}%
\pgfpathcurveto{\pgfqpoint{1.311479in}{1.696511in}}{\pgfqpoint{1.314751in}{1.688611in}}{\pgfqpoint{1.320575in}{1.682787in}}%
\pgfpathcurveto{\pgfqpoint{1.326399in}{1.676963in}}{\pgfqpoint{1.334299in}{1.673690in}}{\pgfqpoint{1.342535in}{1.673690in}}%
\pgfpathclose%
\pgfusepath{stroke,fill}%
\end{pgfscope}%
\begin{pgfscope}%
\pgfpathrectangle{\pgfqpoint{0.100000in}{0.212622in}}{\pgfqpoint{3.696000in}{3.696000in}}%
\pgfusepath{clip}%
\pgfsetbuttcap%
\pgfsetroundjoin%
\definecolor{currentfill}{rgb}{0.121569,0.466667,0.705882}%
\pgfsetfillcolor{currentfill}%
\pgfsetfillopacity{0.361943}%
\pgfsetlinewidth{1.003750pt}%
\definecolor{currentstroke}{rgb}{0.121569,0.466667,0.705882}%
\pgfsetstrokecolor{currentstroke}%
\pgfsetstrokeopacity{0.361943}%
\pgfsetdash{}{0pt}%
\pgfpathmoveto{\pgfqpoint{1.517688in}{1.730457in}}%
\pgfpathcurveto{\pgfqpoint{1.525924in}{1.730457in}}{\pgfqpoint{1.533825in}{1.733729in}}{\pgfqpoint{1.539648in}{1.739553in}}%
\pgfpathcurveto{\pgfqpoint{1.545472in}{1.745377in}}{\pgfqpoint{1.548745in}{1.753277in}}{\pgfqpoint{1.548745in}{1.761513in}}%
\pgfpathcurveto{\pgfqpoint{1.548745in}{1.769750in}}{\pgfqpoint{1.545472in}{1.777650in}}{\pgfqpoint{1.539648in}{1.783474in}}%
\pgfpathcurveto{\pgfqpoint{1.533825in}{1.789298in}}{\pgfqpoint{1.525924in}{1.792570in}}{\pgfqpoint{1.517688in}{1.792570in}}%
\pgfpathcurveto{\pgfqpoint{1.509452in}{1.792570in}}{\pgfqpoint{1.501552in}{1.789298in}}{\pgfqpoint{1.495728in}{1.783474in}}%
\pgfpathcurveto{\pgfqpoint{1.489904in}{1.777650in}}{\pgfqpoint{1.486632in}{1.769750in}}{\pgfqpoint{1.486632in}{1.761513in}}%
\pgfpathcurveto{\pgfqpoint{1.486632in}{1.753277in}}{\pgfqpoint{1.489904in}{1.745377in}}{\pgfqpoint{1.495728in}{1.739553in}}%
\pgfpathcurveto{\pgfqpoint{1.501552in}{1.733729in}}{\pgfqpoint{1.509452in}{1.730457in}}{\pgfqpoint{1.517688in}{1.730457in}}%
\pgfpathclose%
\pgfusepath{stroke,fill}%
\end{pgfscope}%
\begin{pgfscope}%
\pgfpathrectangle{\pgfqpoint{0.100000in}{0.212622in}}{\pgfqpoint{3.696000in}{3.696000in}}%
\pgfusepath{clip}%
\pgfsetbuttcap%
\pgfsetroundjoin%
\definecolor{currentfill}{rgb}{0.121569,0.466667,0.705882}%
\pgfsetfillcolor{currentfill}%
\pgfsetfillopacity{0.363097}%
\pgfsetlinewidth{1.003750pt}%
\definecolor{currentstroke}{rgb}{0.121569,0.466667,0.705882}%
\pgfsetstrokecolor{currentstroke}%
\pgfsetstrokeopacity{0.363097}%
\pgfsetdash{}{0pt}%
\pgfpathmoveto{\pgfqpoint{1.338103in}{1.673159in}}%
\pgfpathcurveto{\pgfqpoint{1.346339in}{1.673159in}}{\pgfqpoint{1.354239in}{1.676432in}}{\pgfqpoint{1.360063in}{1.682256in}}%
\pgfpathcurveto{\pgfqpoint{1.365887in}{1.688079in}}{\pgfqpoint{1.369159in}{1.695980in}}{\pgfqpoint{1.369159in}{1.704216in}}%
\pgfpathcurveto{\pgfqpoint{1.369159in}{1.712452in}}{\pgfqpoint{1.365887in}{1.720352in}}{\pgfqpoint{1.360063in}{1.726176in}}%
\pgfpathcurveto{\pgfqpoint{1.354239in}{1.732000in}}{\pgfqpoint{1.346339in}{1.735272in}}{\pgfqpoint{1.338103in}{1.735272in}}%
\pgfpathcurveto{\pgfqpoint{1.329866in}{1.735272in}}{\pgfqpoint{1.321966in}{1.732000in}}{\pgfqpoint{1.316142in}{1.726176in}}%
\pgfpathcurveto{\pgfqpoint{1.310318in}{1.720352in}}{\pgfqpoint{1.307046in}{1.712452in}}{\pgfqpoint{1.307046in}{1.704216in}}%
\pgfpathcurveto{\pgfqpoint{1.307046in}{1.695980in}}{\pgfqpoint{1.310318in}{1.688079in}}{\pgfqpoint{1.316142in}{1.682256in}}%
\pgfpathcurveto{\pgfqpoint{1.321966in}{1.676432in}}{\pgfqpoint{1.329866in}{1.673159in}}{\pgfqpoint{1.338103in}{1.673159in}}%
\pgfpathclose%
\pgfusepath{stroke,fill}%
\end{pgfscope}%
\begin{pgfscope}%
\pgfpathrectangle{\pgfqpoint{0.100000in}{0.212622in}}{\pgfqpoint{3.696000in}{3.696000in}}%
\pgfusepath{clip}%
\pgfsetbuttcap%
\pgfsetroundjoin%
\definecolor{currentfill}{rgb}{0.121569,0.466667,0.705882}%
\pgfsetfillcolor{currentfill}%
\pgfsetfillopacity{0.364715}%
\pgfsetlinewidth{1.003750pt}%
\definecolor{currentstroke}{rgb}{0.121569,0.466667,0.705882}%
\pgfsetstrokecolor{currentstroke}%
\pgfsetstrokeopacity{0.364715}%
\pgfsetdash{}{0pt}%
\pgfpathmoveto{\pgfqpoint{1.335112in}{1.672055in}}%
\pgfpathcurveto{\pgfqpoint{1.343348in}{1.672055in}}{\pgfqpoint{1.351248in}{1.675328in}}{\pgfqpoint{1.357072in}{1.681152in}}%
\pgfpathcurveto{\pgfqpoint{1.362896in}{1.686975in}}{\pgfqpoint{1.366169in}{1.694876in}}{\pgfqpoint{1.366169in}{1.703112in}}%
\pgfpathcurveto{\pgfqpoint{1.366169in}{1.711348in}}{\pgfqpoint{1.362896in}{1.719248in}}{\pgfqpoint{1.357072in}{1.725072in}}%
\pgfpathcurveto{\pgfqpoint{1.351248in}{1.730896in}}{\pgfqpoint{1.343348in}{1.734168in}}{\pgfqpoint{1.335112in}{1.734168in}}%
\pgfpathcurveto{\pgfqpoint{1.326876in}{1.734168in}}{\pgfqpoint{1.318976in}{1.730896in}}{\pgfqpoint{1.313152in}{1.725072in}}%
\pgfpathcurveto{\pgfqpoint{1.307328in}{1.719248in}}{\pgfqpoint{1.304056in}{1.711348in}}{\pgfqpoint{1.304056in}{1.703112in}}%
\pgfpathcurveto{\pgfqpoint{1.304056in}{1.694876in}}{\pgfqpoint{1.307328in}{1.686975in}}{\pgfqpoint{1.313152in}{1.681152in}}%
\pgfpathcurveto{\pgfqpoint{1.318976in}{1.675328in}}{\pgfqpoint{1.326876in}{1.672055in}}{\pgfqpoint{1.335112in}{1.672055in}}%
\pgfpathclose%
\pgfusepath{stroke,fill}%
\end{pgfscope}%
\begin{pgfscope}%
\pgfpathrectangle{\pgfqpoint{0.100000in}{0.212622in}}{\pgfqpoint{3.696000in}{3.696000in}}%
\pgfusepath{clip}%
\pgfsetbuttcap%
\pgfsetroundjoin%
\definecolor{currentfill}{rgb}{0.121569,0.466667,0.705882}%
\pgfsetfillcolor{currentfill}%
\pgfsetfillopacity{0.367784}%
\pgfsetlinewidth{1.003750pt}%
\definecolor{currentstroke}{rgb}{0.121569,0.466667,0.705882}%
\pgfsetstrokecolor{currentstroke}%
\pgfsetstrokeopacity{0.367784}%
\pgfsetdash{}{0pt}%
\pgfpathmoveto{\pgfqpoint{1.329480in}{1.670877in}}%
\pgfpathcurveto{\pgfqpoint{1.337717in}{1.670877in}}{\pgfqpoint{1.345617in}{1.674150in}}{\pgfqpoint{1.351441in}{1.679974in}}%
\pgfpathcurveto{\pgfqpoint{1.357265in}{1.685798in}}{\pgfqpoint{1.360537in}{1.693698in}}{\pgfqpoint{1.360537in}{1.701934in}}%
\pgfpathcurveto{\pgfqpoint{1.360537in}{1.710170in}}{\pgfqpoint{1.357265in}{1.718070in}}{\pgfqpoint{1.351441in}{1.723894in}}%
\pgfpathcurveto{\pgfqpoint{1.345617in}{1.729718in}}{\pgfqpoint{1.337717in}{1.732990in}}{\pgfqpoint{1.329480in}{1.732990in}}%
\pgfpathcurveto{\pgfqpoint{1.321244in}{1.732990in}}{\pgfqpoint{1.313344in}{1.729718in}}{\pgfqpoint{1.307520in}{1.723894in}}%
\pgfpathcurveto{\pgfqpoint{1.301696in}{1.718070in}}{\pgfqpoint{1.298424in}{1.710170in}}{\pgfqpoint{1.298424in}{1.701934in}}%
\pgfpathcurveto{\pgfqpoint{1.298424in}{1.693698in}}{\pgfqpoint{1.301696in}{1.685798in}}{\pgfqpoint{1.307520in}{1.679974in}}%
\pgfpathcurveto{\pgfqpoint{1.313344in}{1.674150in}}{\pgfqpoint{1.321244in}{1.670877in}}{\pgfqpoint{1.329480in}{1.670877in}}%
\pgfpathclose%
\pgfusepath{stroke,fill}%
\end{pgfscope}%
\begin{pgfscope}%
\pgfpathrectangle{\pgfqpoint{0.100000in}{0.212622in}}{\pgfqpoint{3.696000in}{3.696000in}}%
\pgfusepath{clip}%
\pgfsetbuttcap%
\pgfsetroundjoin%
\definecolor{currentfill}{rgb}{0.121569,0.466667,0.705882}%
\pgfsetfillcolor{currentfill}%
\pgfsetfillopacity{0.367814}%
\pgfsetlinewidth{1.003750pt}%
\definecolor{currentstroke}{rgb}{0.121569,0.466667,0.705882}%
\pgfsetstrokecolor{currentstroke}%
\pgfsetstrokeopacity{0.367814}%
\pgfsetdash{}{0pt}%
\pgfpathmoveto{\pgfqpoint{1.520574in}{1.729624in}}%
\pgfpathcurveto{\pgfqpoint{1.528810in}{1.729624in}}{\pgfqpoint{1.536710in}{1.732896in}}{\pgfqpoint{1.542534in}{1.738720in}}%
\pgfpathcurveto{\pgfqpoint{1.548358in}{1.744544in}}{\pgfqpoint{1.551630in}{1.752444in}}{\pgfqpoint{1.551630in}{1.760680in}}%
\pgfpathcurveto{\pgfqpoint{1.551630in}{1.768916in}}{\pgfqpoint{1.548358in}{1.776816in}}{\pgfqpoint{1.542534in}{1.782640in}}%
\pgfpathcurveto{\pgfqpoint{1.536710in}{1.788464in}}{\pgfqpoint{1.528810in}{1.791737in}}{\pgfqpoint{1.520574in}{1.791737in}}%
\pgfpathcurveto{\pgfqpoint{1.512338in}{1.791737in}}{\pgfqpoint{1.504438in}{1.788464in}}{\pgfqpoint{1.498614in}{1.782640in}}%
\pgfpathcurveto{\pgfqpoint{1.492790in}{1.776816in}}{\pgfqpoint{1.489517in}{1.768916in}}{\pgfqpoint{1.489517in}{1.760680in}}%
\pgfpathcurveto{\pgfqpoint{1.489517in}{1.752444in}}{\pgfqpoint{1.492790in}{1.744544in}}{\pgfqpoint{1.498614in}{1.738720in}}%
\pgfpathcurveto{\pgfqpoint{1.504438in}{1.732896in}}{\pgfqpoint{1.512338in}{1.729624in}}{\pgfqpoint{1.520574in}{1.729624in}}%
\pgfpathclose%
\pgfusepath{stroke,fill}%
\end{pgfscope}%
\begin{pgfscope}%
\pgfpathrectangle{\pgfqpoint{0.100000in}{0.212622in}}{\pgfqpoint{3.696000in}{3.696000in}}%
\pgfusepath{clip}%
\pgfsetbuttcap%
\pgfsetroundjoin%
\definecolor{currentfill}{rgb}{0.121569,0.466667,0.705882}%
\pgfsetfillcolor{currentfill}%
\pgfsetfillopacity{0.369790}%
\pgfsetlinewidth{1.003750pt}%
\definecolor{currentstroke}{rgb}{0.121569,0.466667,0.705882}%
\pgfsetstrokecolor{currentstroke}%
\pgfsetstrokeopacity{0.369790}%
\pgfsetdash{}{0pt}%
\pgfpathmoveto{\pgfqpoint{1.323553in}{1.666722in}}%
\pgfpathcurveto{\pgfqpoint{1.331789in}{1.666722in}}{\pgfqpoint{1.339689in}{1.669994in}}{\pgfqpoint{1.345513in}{1.675818in}}%
\pgfpathcurveto{\pgfqpoint{1.351337in}{1.681642in}}{\pgfqpoint{1.354609in}{1.689542in}}{\pgfqpoint{1.354609in}{1.697779in}}%
\pgfpathcurveto{\pgfqpoint{1.354609in}{1.706015in}}{\pgfqpoint{1.351337in}{1.713915in}}{\pgfqpoint{1.345513in}{1.719739in}}%
\pgfpathcurveto{\pgfqpoint{1.339689in}{1.725563in}}{\pgfqpoint{1.331789in}{1.728835in}}{\pgfqpoint{1.323553in}{1.728835in}}%
\pgfpathcurveto{\pgfqpoint{1.315317in}{1.728835in}}{\pgfqpoint{1.307417in}{1.725563in}}{\pgfqpoint{1.301593in}{1.719739in}}%
\pgfpathcurveto{\pgfqpoint{1.295769in}{1.713915in}}{\pgfqpoint{1.292496in}{1.706015in}}{\pgfqpoint{1.292496in}{1.697779in}}%
\pgfpathcurveto{\pgfqpoint{1.292496in}{1.689542in}}{\pgfqpoint{1.295769in}{1.681642in}}{\pgfqpoint{1.301593in}{1.675818in}}%
\pgfpathcurveto{\pgfqpoint{1.307417in}{1.669994in}}{\pgfqpoint{1.315317in}{1.666722in}}{\pgfqpoint{1.323553in}{1.666722in}}%
\pgfpathclose%
\pgfusepath{stroke,fill}%
\end{pgfscope}%
\begin{pgfscope}%
\pgfpathrectangle{\pgfqpoint{0.100000in}{0.212622in}}{\pgfqpoint{3.696000in}{3.696000in}}%
\pgfusepath{clip}%
\pgfsetbuttcap%
\pgfsetroundjoin%
\definecolor{currentfill}{rgb}{0.121569,0.466667,0.705882}%
\pgfsetfillcolor{currentfill}%
\pgfsetfillopacity{0.373459}%
\pgfsetlinewidth{1.003750pt}%
\definecolor{currentstroke}{rgb}{0.121569,0.466667,0.705882}%
\pgfsetstrokecolor{currentstroke}%
\pgfsetstrokeopacity{0.373459}%
\pgfsetdash{}{0pt}%
\pgfpathmoveto{\pgfqpoint{1.319578in}{1.670655in}}%
\pgfpathcurveto{\pgfqpoint{1.327814in}{1.670655in}}{\pgfqpoint{1.335714in}{1.673928in}}{\pgfqpoint{1.341538in}{1.679752in}}%
\pgfpathcurveto{\pgfqpoint{1.347362in}{1.685576in}}{\pgfqpoint{1.350635in}{1.693476in}}{\pgfqpoint{1.350635in}{1.701712in}}%
\pgfpathcurveto{\pgfqpoint{1.350635in}{1.709948in}}{\pgfqpoint{1.347362in}{1.717848in}}{\pgfqpoint{1.341538in}{1.723672in}}%
\pgfpathcurveto{\pgfqpoint{1.335714in}{1.729496in}}{\pgfqpoint{1.327814in}{1.732768in}}{\pgfqpoint{1.319578in}{1.732768in}}%
\pgfpathcurveto{\pgfqpoint{1.311342in}{1.732768in}}{\pgfqpoint{1.303442in}{1.729496in}}{\pgfqpoint{1.297618in}{1.723672in}}%
\pgfpathcurveto{\pgfqpoint{1.291794in}{1.717848in}}{\pgfqpoint{1.288522in}{1.709948in}}{\pgfqpoint{1.288522in}{1.701712in}}%
\pgfpathcurveto{\pgfqpoint{1.288522in}{1.693476in}}{\pgfqpoint{1.291794in}{1.685576in}}{\pgfqpoint{1.297618in}{1.679752in}}%
\pgfpathcurveto{\pgfqpoint{1.303442in}{1.673928in}}{\pgfqpoint{1.311342in}{1.670655in}}{\pgfqpoint{1.319578in}{1.670655in}}%
\pgfpathclose%
\pgfusepath{stroke,fill}%
\end{pgfscope}%
\begin{pgfscope}%
\pgfpathrectangle{\pgfqpoint{0.100000in}{0.212622in}}{\pgfqpoint{3.696000in}{3.696000in}}%
\pgfusepath{clip}%
\pgfsetbuttcap%
\pgfsetroundjoin%
\definecolor{currentfill}{rgb}{0.121569,0.466667,0.705882}%
\pgfsetfillcolor{currentfill}%
\pgfsetfillopacity{0.374789}%
\pgfsetlinewidth{1.003750pt}%
\definecolor{currentstroke}{rgb}{0.121569,0.466667,0.705882}%
\pgfsetstrokecolor{currentstroke}%
\pgfsetstrokeopacity{0.374789}%
\pgfsetdash{}{0pt}%
\pgfpathmoveto{\pgfqpoint{1.314887in}{1.665630in}}%
\pgfpathcurveto{\pgfqpoint{1.323123in}{1.665630in}}{\pgfqpoint{1.331023in}{1.668902in}}{\pgfqpoint{1.336847in}{1.674726in}}%
\pgfpathcurveto{\pgfqpoint{1.342671in}{1.680550in}}{\pgfqpoint{1.345943in}{1.688450in}}{\pgfqpoint{1.345943in}{1.696686in}}%
\pgfpathcurveto{\pgfqpoint{1.345943in}{1.704923in}}{\pgfqpoint{1.342671in}{1.712823in}}{\pgfqpoint{1.336847in}{1.718647in}}%
\pgfpathcurveto{\pgfqpoint{1.331023in}{1.724471in}}{\pgfqpoint{1.323123in}{1.727743in}}{\pgfqpoint{1.314887in}{1.727743in}}%
\pgfpathcurveto{\pgfqpoint{1.306650in}{1.727743in}}{\pgfqpoint{1.298750in}{1.724471in}}{\pgfqpoint{1.292926in}{1.718647in}}%
\pgfpathcurveto{\pgfqpoint{1.287102in}{1.712823in}}{\pgfqpoint{1.283830in}{1.704923in}}{\pgfqpoint{1.283830in}{1.696686in}}%
\pgfpathcurveto{\pgfqpoint{1.283830in}{1.688450in}}{\pgfqpoint{1.287102in}{1.680550in}}{\pgfqpoint{1.292926in}{1.674726in}}%
\pgfpathcurveto{\pgfqpoint{1.298750in}{1.668902in}}{\pgfqpoint{1.306650in}{1.665630in}}{\pgfqpoint{1.314887in}{1.665630in}}%
\pgfpathclose%
\pgfusepath{stroke,fill}%
\end{pgfscope}%
\begin{pgfscope}%
\pgfpathrectangle{\pgfqpoint{0.100000in}{0.212622in}}{\pgfqpoint{3.696000in}{3.696000in}}%
\pgfusepath{clip}%
\pgfsetbuttcap%
\pgfsetroundjoin%
\definecolor{currentfill}{rgb}{0.121569,0.466667,0.705882}%
\pgfsetfillcolor{currentfill}%
\pgfsetfillopacity{0.375497}%
\pgfsetlinewidth{1.003750pt}%
\definecolor{currentstroke}{rgb}{0.121569,0.466667,0.705882}%
\pgfsetstrokecolor{currentstroke}%
\pgfsetstrokeopacity{0.375497}%
\pgfsetdash{}{0pt}%
\pgfpathmoveto{\pgfqpoint{1.520674in}{1.735128in}}%
\pgfpathcurveto{\pgfqpoint{1.528911in}{1.735128in}}{\pgfqpoint{1.536811in}{1.738400in}}{\pgfqpoint{1.542635in}{1.744224in}}%
\pgfpathcurveto{\pgfqpoint{1.548459in}{1.750048in}}{\pgfqpoint{1.551731in}{1.757948in}}{\pgfqpoint{1.551731in}{1.766184in}}%
\pgfpathcurveto{\pgfqpoint{1.551731in}{1.774421in}}{\pgfqpoint{1.548459in}{1.782321in}}{\pgfqpoint{1.542635in}{1.788145in}}%
\pgfpathcurveto{\pgfqpoint{1.536811in}{1.793968in}}{\pgfqpoint{1.528911in}{1.797241in}}{\pgfqpoint{1.520674in}{1.797241in}}%
\pgfpathcurveto{\pgfqpoint{1.512438in}{1.797241in}}{\pgfqpoint{1.504538in}{1.793968in}}{\pgfqpoint{1.498714in}{1.788145in}}%
\pgfpathcurveto{\pgfqpoint{1.492890in}{1.782321in}}{\pgfqpoint{1.489618in}{1.774421in}}{\pgfqpoint{1.489618in}{1.766184in}}%
\pgfpathcurveto{\pgfqpoint{1.489618in}{1.757948in}}{\pgfqpoint{1.492890in}{1.750048in}}{\pgfqpoint{1.498714in}{1.744224in}}%
\pgfpathcurveto{\pgfqpoint{1.504538in}{1.738400in}}{\pgfqpoint{1.512438in}{1.735128in}}{\pgfqpoint{1.520674in}{1.735128in}}%
\pgfpathclose%
\pgfusepath{stroke,fill}%
\end{pgfscope}%
\begin{pgfscope}%
\pgfpathrectangle{\pgfqpoint{0.100000in}{0.212622in}}{\pgfqpoint{3.696000in}{3.696000in}}%
\pgfusepath{clip}%
\pgfsetbuttcap%
\pgfsetroundjoin%
\definecolor{currentfill}{rgb}{0.121569,0.466667,0.705882}%
\pgfsetfillcolor{currentfill}%
\pgfsetfillopacity{0.376359}%
\pgfsetlinewidth{1.003750pt}%
\definecolor{currentstroke}{rgb}{0.121569,0.466667,0.705882}%
\pgfsetstrokecolor{currentstroke}%
\pgfsetstrokeopacity{0.376359}%
\pgfsetdash{}{0pt}%
\pgfpathmoveto{\pgfqpoint{1.310478in}{1.663322in}}%
\pgfpathcurveto{\pgfqpoint{1.318714in}{1.663322in}}{\pgfqpoint{1.326614in}{1.666595in}}{\pgfqpoint{1.332438in}{1.672419in}}%
\pgfpathcurveto{\pgfqpoint{1.338262in}{1.678242in}}{\pgfqpoint{1.341535in}{1.686143in}}{\pgfqpoint{1.341535in}{1.694379in}}%
\pgfpathcurveto{\pgfqpoint{1.341535in}{1.702615in}}{\pgfqpoint{1.338262in}{1.710515in}}{\pgfqpoint{1.332438in}{1.716339in}}%
\pgfpathcurveto{\pgfqpoint{1.326614in}{1.722163in}}{\pgfqpoint{1.318714in}{1.725435in}}{\pgfqpoint{1.310478in}{1.725435in}}%
\pgfpathcurveto{\pgfqpoint{1.302242in}{1.725435in}}{\pgfqpoint{1.294342in}{1.722163in}}{\pgfqpoint{1.288518in}{1.716339in}}%
\pgfpathcurveto{\pgfqpoint{1.282694in}{1.710515in}}{\pgfqpoint{1.279422in}{1.702615in}}{\pgfqpoint{1.279422in}{1.694379in}}%
\pgfpathcurveto{\pgfqpoint{1.279422in}{1.686143in}}{\pgfqpoint{1.282694in}{1.678242in}}{\pgfqpoint{1.288518in}{1.672419in}}%
\pgfpathcurveto{\pgfqpoint{1.294342in}{1.666595in}}{\pgfqpoint{1.302242in}{1.663322in}}{\pgfqpoint{1.310478in}{1.663322in}}%
\pgfpathclose%
\pgfusepath{stroke,fill}%
\end{pgfscope}%
\begin{pgfscope}%
\pgfpathrectangle{\pgfqpoint{0.100000in}{0.212622in}}{\pgfqpoint{3.696000in}{3.696000in}}%
\pgfusepath{clip}%
\pgfsetbuttcap%
\pgfsetroundjoin%
\definecolor{currentfill}{rgb}{0.121569,0.466667,0.705882}%
\pgfsetfillcolor{currentfill}%
\pgfsetfillopacity{0.378684}%
\pgfsetlinewidth{1.003750pt}%
\definecolor{currentstroke}{rgb}{0.121569,0.466667,0.705882}%
\pgfsetstrokecolor{currentstroke}%
\pgfsetstrokeopacity{0.378684}%
\pgfsetdash{}{0pt}%
\pgfpathmoveto{\pgfqpoint{1.307688in}{1.664277in}}%
\pgfpathcurveto{\pgfqpoint{1.315925in}{1.664277in}}{\pgfqpoint{1.323825in}{1.667549in}}{\pgfqpoint{1.329649in}{1.673373in}}%
\pgfpathcurveto{\pgfqpoint{1.335473in}{1.679197in}}{\pgfqpoint{1.338745in}{1.687097in}}{\pgfqpoint{1.338745in}{1.695333in}}%
\pgfpathcurveto{\pgfqpoint{1.338745in}{1.703569in}}{\pgfqpoint{1.335473in}{1.711469in}}{\pgfqpoint{1.329649in}{1.717293in}}%
\pgfpathcurveto{\pgfqpoint{1.323825in}{1.723117in}}{\pgfqpoint{1.315925in}{1.726390in}}{\pgfqpoint{1.307688in}{1.726390in}}%
\pgfpathcurveto{\pgfqpoint{1.299452in}{1.726390in}}{\pgfqpoint{1.291552in}{1.723117in}}{\pgfqpoint{1.285728in}{1.717293in}}%
\pgfpathcurveto{\pgfqpoint{1.279904in}{1.711469in}}{\pgfqpoint{1.276632in}{1.703569in}}{\pgfqpoint{1.276632in}{1.695333in}}%
\pgfpathcurveto{\pgfqpoint{1.276632in}{1.687097in}}{\pgfqpoint{1.279904in}{1.679197in}}{\pgfqpoint{1.285728in}{1.673373in}}%
\pgfpathcurveto{\pgfqpoint{1.291552in}{1.667549in}}{\pgfqpoint{1.299452in}{1.664277in}}{\pgfqpoint{1.307688in}{1.664277in}}%
\pgfpathclose%
\pgfusepath{stroke,fill}%
\end{pgfscope}%
\begin{pgfscope}%
\pgfpathrectangle{\pgfqpoint{0.100000in}{0.212622in}}{\pgfqpoint{3.696000in}{3.696000in}}%
\pgfusepath{clip}%
\pgfsetbuttcap%
\pgfsetroundjoin%
\definecolor{currentfill}{rgb}{0.121569,0.466667,0.705882}%
\pgfsetfillcolor{currentfill}%
\pgfsetfillopacity{0.379195}%
\pgfsetlinewidth{1.003750pt}%
\definecolor{currentstroke}{rgb}{0.121569,0.466667,0.705882}%
\pgfsetstrokecolor{currentstroke}%
\pgfsetstrokeopacity{0.379195}%
\pgfsetdash{}{0pt}%
\pgfpathmoveto{\pgfqpoint{1.305514in}{1.661726in}}%
\pgfpathcurveto{\pgfqpoint{1.313750in}{1.661726in}}{\pgfqpoint{1.321650in}{1.664998in}}{\pgfqpoint{1.327474in}{1.670822in}}%
\pgfpathcurveto{\pgfqpoint{1.333298in}{1.676646in}}{\pgfqpoint{1.336571in}{1.684546in}}{\pgfqpoint{1.336571in}{1.692782in}}%
\pgfpathcurveto{\pgfqpoint{1.336571in}{1.701019in}}{\pgfqpoint{1.333298in}{1.708919in}}{\pgfqpoint{1.327474in}{1.714743in}}%
\pgfpathcurveto{\pgfqpoint{1.321650in}{1.720567in}}{\pgfqpoint{1.313750in}{1.723839in}}{\pgfqpoint{1.305514in}{1.723839in}}%
\pgfpathcurveto{\pgfqpoint{1.297278in}{1.723839in}}{\pgfqpoint{1.289378in}{1.720567in}}{\pgfqpoint{1.283554in}{1.714743in}}%
\pgfpathcurveto{\pgfqpoint{1.277730in}{1.708919in}}{\pgfqpoint{1.274458in}{1.701019in}}{\pgfqpoint{1.274458in}{1.692782in}}%
\pgfpathcurveto{\pgfqpoint{1.274458in}{1.684546in}}{\pgfqpoint{1.277730in}{1.676646in}}{\pgfqpoint{1.283554in}{1.670822in}}%
\pgfpathcurveto{\pgfqpoint{1.289378in}{1.664998in}}{\pgfqpoint{1.297278in}{1.661726in}}{\pgfqpoint{1.305514in}{1.661726in}}%
\pgfpathclose%
\pgfusepath{stroke,fill}%
\end{pgfscope}%
\begin{pgfscope}%
\pgfpathrectangle{\pgfqpoint{0.100000in}{0.212622in}}{\pgfqpoint{3.696000in}{3.696000in}}%
\pgfusepath{clip}%
\pgfsetbuttcap%
\pgfsetroundjoin%
\definecolor{currentfill}{rgb}{0.121569,0.466667,0.705882}%
\pgfsetfillcolor{currentfill}%
\pgfsetfillopacity{0.379758}%
\pgfsetlinewidth{1.003750pt}%
\definecolor{currentstroke}{rgb}{0.121569,0.466667,0.705882}%
\pgfsetstrokecolor{currentstroke}%
\pgfsetstrokeopacity{0.379758}%
\pgfsetdash{}{0pt}%
\pgfpathmoveto{\pgfqpoint{1.303826in}{1.660092in}}%
\pgfpathcurveto{\pgfqpoint{1.312063in}{1.660092in}}{\pgfqpoint{1.319963in}{1.663364in}}{\pgfqpoint{1.325787in}{1.669188in}}%
\pgfpathcurveto{\pgfqpoint{1.331611in}{1.675012in}}{\pgfqpoint{1.334883in}{1.682912in}}{\pgfqpoint{1.334883in}{1.691149in}}%
\pgfpathcurveto{\pgfqpoint{1.334883in}{1.699385in}}{\pgfqpoint{1.331611in}{1.707285in}}{\pgfqpoint{1.325787in}{1.713109in}}%
\pgfpathcurveto{\pgfqpoint{1.319963in}{1.718933in}}{\pgfqpoint{1.312063in}{1.722205in}}{\pgfqpoint{1.303826in}{1.722205in}}%
\pgfpathcurveto{\pgfqpoint{1.295590in}{1.722205in}}{\pgfqpoint{1.287690in}{1.718933in}}{\pgfqpoint{1.281866in}{1.713109in}}%
\pgfpathcurveto{\pgfqpoint{1.276042in}{1.707285in}}{\pgfqpoint{1.272770in}{1.699385in}}{\pgfqpoint{1.272770in}{1.691149in}}%
\pgfpathcurveto{\pgfqpoint{1.272770in}{1.682912in}}{\pgfqpoint{1.276042in}{1.675012in}}{\pgfqpoint{1.281866in}{1.669188in}}%
\pgfpathcurveto{\pgfqpoint{1.287690in}{1.663364in}}{\pgfqpoint{1.295590in}{1.660092in}}{\pgfqpoint{1.303826in}{1.660092in}}%
\pgfpathclose%
\pgfusepath{stroke,fill}%
\end{pgfscope}%
\begin{pgfscope}%
\pgfpathrectangle{\pgfqpoint{0.100000in}{0.212622in}}{\pgfqpoint{3.696000in}{3.696000in}}%
\pgfusepath{clip}%
\pgfsetbuttcap%
\pgfsetroundjoin%
\definecolor{currentfill}{rgb}{0.121569,0.466667,0.705882}%
\pgfsetfillcolor{currentfill}%
\pgfsetfillopacity{0.380894}%
\pgfsetlinewidth{1.003750pt}%
\definecolor{currentstroke}{rgb}{0.121569,0.466667,0.705882}%
\pgfsetstrokecolor{currentstroke}%
\pgfsetstrokeopacity{0.380894}%
\pgfsetdash{}{0pt}%
\pgfpathmoveto{\pgfqpoint{1.521804in}{1.728929in}}%
\pgfpathcurveto{\pgfqpoint{1.530040in}{1.728929in}}{\pgfqpoint{1.537940in}{1.732201in}}{\pgfqpoint{1.543764in}{1.738025in}}%
\pgfpathcurveto{\pgfqpoint{1.549588in}{1.743849in}}{\pgfqpoint{1.552860in}{1.751749in}}{\pgfqpoint{1.552860in}{1.759985in}}%
\pgfpathcurveto{\pgfqpoint{1.552860in}{1.768221in}}{\pgfqpoint{1.549588in}{1.776121in}}{\pgfqpoint{1.543764in}{1.781945in}}%
\pgfpathcurveto{\pgfqpoint{1.537940in}{1.787769in}}{\pgfqpoint{1.530040in}{1.791042in}}{\pgfqpoint{1.521804in}{1.791042in}}%
\pgfpathcurveto{\pgfqpoint{1.513568in}{1.791042in}}{\pgfqpoint{1.505668in}{1.787769in}}{\pgfqpoint{1.499844in}{1.781945in}}%
\pgfpathcurveto{\pgfqpoint{1.494020in}{1.776121in}}{\pgfqpoint{1.490747in}{1.768221in}}{\pgfqpoint{1.490747in}{1.759985in}}%
\pgfpathcurveto{\pgfqpoint{1.490747in}{1.751749in}}{\pgfqpoint{1.494020in}{1.743849in}}{\pgfqpoint{1.499844in}{1.738025in}}%
\pgfpathcurveto{\pgfqpoint{1.505668in}{1.732201in}}{\pgfqpoint{1.513568in}{1.728929in}}{\pgfqpoint{1.521804in}{1.728929in}}%
\pgfpathclose%
\pgfusepath{stroke,fill}%
\end{pgfscope}%
\begin{pgfscope}%
\pgfpathrectangle{\pgfqpoint{0.100000in}{0.212622in}}{\pgfqpoint{3.696000in}{3.696000in}}%
\pgfusepath{clip}%
\pgfsetbuttcap%
\pgfsetroundjoin%
\definecolor{currentfill}{rgb}{0.121569,0.466667,0.705882}%
\pgfsetfillcolor{currentfill}%
\pgfsetfillopacity{0.381673}%
\pgfsetlinewidth{1.003750pt}%
\definecolor{currentstroke}{rgb}{0.121569,0.466667,0.705882}%
\pgfsetstrokecolor{currentstroke}%
\pgfsetstrokeopacity{0.381673}%
\pgfsetdash{}{0pt}%
\pgfpathmoveto{\pgfqpoint{1.301384in}{1.660424in}}%
\pgfpathcurveto{\pgfqpoint{1.309620in}{1.660424in}}{\pgfqpoint{1.317520in}{1.663697in}}{\pgfqpoint{1.323344in}{1.669521in}}%
\pgfpathcurveto{\pgfqpoint{1.329168in}{1.675344in}}{\pgfqpoint{1.332441in}{1.683245in}}{\pgfqpoint{1.332441in}{1.691481in}}%
\pgfpathcurveto{\pgfqpoint{1.332441in}{1.699717in}}{\pgfqpoint{1.329168in}{1.707617in}}{\pgfqpoint{1.323344in}{1.713441in}}%
\pgfpathcurveto{\pgfqpoint{1.317520in}{1.719265in}}{\pgfqpoint{1.309620in}{1.722537in}}{\pgfqpoint{1.301384in}{1.722537in}}%
\pgfpathcurveto{\pgfqpoint{1.293148in}{1.722537in}}{\pgfqpoint{1.285248in}{1.719265in}}{\pgfqpoint{1.279424in}{1.713441in}}%
\pgfpathcurveto{\pgfqpoint{1.273600in}{1.707617in}}{\pgfqpoint{1.270328in}{1.699717in}}{\pgfqpoint{1.270328in}{1.691481in}}%
\pgfpathcurveto{\pgfqpoint{1.270328in}{1.683245in}}{\pgfqpoint{1.273600in}{1.675344in}}{\pgfqpoint{1.279424in}{1.669521in}}%
\pgfpathcurveto{\pgfqpoint{1.285248in}{1.663697in}}{\pgfqpoint{1.293148in}{1.660424in}}{\pgfqpoint{1.301384in}{1.660424in}}%
\pgfpathclose%
\pgfusepath{stroke,fill}%
\end{pgfscope}%
\begin{pgfscope}%
\pgfpathrectangle{\pgfqpoint{0.100000in}{0.212622in}}{\pgfqpoint{3.696000in}{3.696000in}}%
\pgfusepath{clip}%
\pgfsetbuttcap%
\pgfsetroundjoin%
\definecolor{currentfill}{rgb}{0.121569,0.466667,0.705882}%
\pgfsetfillcolor{currentfill}%
\pgfsetfillopacity{0.382042}%
\pgfsetlinewidth{1.003750pt}%
\definecolor{currentstroke}{rgb}{0.121569,0.466667,0.705882}%
\pgfsetstrokecolor{currentstroke}%
\pgfsetstrokeopacity{0.382042}%
\pgfsetdash{}{0pt}%
\pgfpathmoveto{\pgfqpoint{1.299861in}{1.659334in}}%
\pgfpathcurveto{\pgfqpoint{1.308097in}{1.659334in}}{\pgfqpoint{1.315997in}{1.662607in}}{\pgfqpoint{1.321821in}{1.668431in}}%
\pgfpathcurveto{\pgfqpoint{1.327645in}{1.674255in}}{\pgfqpoint{1.330918in}{1.682155in}}{\pgfqpoint{1.330918in}{1.690391in}}%
\pgfpathcurveto{\pgfqpoint{1.330918in}{1.698627in}}{\pgfqpoint{1.327645in}{1.706527in}}{\pgfqpoint{1.321821in}{1.712351in}}%
\pgfpathcurveto{\pgfqpoint{1.315997in}{1.718175in}}{\pgfqpoint{1.308097in}{1.721447in}}{\pgfqpoint{1.299861in}{1.721447in}}%
\pgfpathcurveto{\pgfqpoint{1.291625in}{1.721447in}}{\pgfqpoint{1.283725in}{1.718175in}}{\pgfqpoint{1.277901in}{1.712351in}}%
\pgfpathcurveto{\pgfqpoint{1.272077in}{1.706527in}}{\pgfqpoint{1.268805in}{1.698627in}}{\pgfqpoint{1.268805in}{1.690391in}}%
\pgfpathcurveto{\pgfqpoint{1.268805in}{1.682155in}}{\pgfqpoint{1.272077in}{1.674255in}}{\pgfqpoint{1.277901in}{1.668431in}}%
\pgfpathcurveto{\pgfqpoint{1.283725in}{1.662607in}}{\pgfqpoint{1.291625in}{1.659334in}}{\pgfqpoint{1.299861in}{1.659334in}}%
\pgfpathclose%
\pgfusepath{stroke,fill}%
\end{pgfscope}%
\begin{pgfscope}%
\pgfpathrectangle{\pgfqpoint{0.100000in}{0.212622in}}{\pgfqpoint{3.696000in}{3.696000in}}%
\pgfusepath{clip}%
\pgfsetbuttcap%
\pgfsetroundjoin%
\definecolor{currentfill}{rgb}{0.121569,0.466667,0.705882}%
\pgfsetfillcolor{currentfill}%
\pgfsetfillopacity{0.382789}%
\pgfsetlinewidth{1.003750pt}%
\definecolor{currentstroke}{rgb}{0.121569,0.466667,0.705882}%
\pgfsetstrokecolor{currentstroke}%
\pgfsetstrokeopacity{0.382789}%
\pgfsetdash{}{0pt}%
\pgfpathmoveto{\pgfqpoint{1.297372in}{1.657168in}}%
\pgfpathcurveto{\pgfqpoint{1.305608in}{1.657168in}}{\pgfqpoint{1.313508in}{1.660440in}}{\pgfqpoint{1.319332in}{1.666264in}}%
\pgfpathcurveto{\pgfqpoint{1.325156in}{1.672088in}}{\pgfqpoint{1.328428in}{1.679988in}}{\pgfqpoint{1.328428in}{1.688225in}}%
\pgfpathcurveto{\pgfqpoint{1.328428in}{1.696461in}}{\pgfqpoint{1.325156in}{1.704361in}}{\pgfqpoint{1.319332in}{1.710185in}}%
\pgfpathcurveto{\pgfqpoint{1.313508in}{1.716009in}}{\pgfqpoint{1.305608in}{1.719281in}}{\pgfqpoint{1.297372in}{1.719281in}}%
\pgfpathcurveto{\pgfqpoint{1.289135in}{1.719281in}}{\pgfqpoint{1.281235in}{1.716009in}}{\pgfqpoint{1.275411in}{1.710185in}}%
\pgfpathcurveto{\pgfqpoint{1.269587in}{1.704361in}}{\pgfqpoint{1.266315in}{1.696461in}}{\pgfqpoint{1.266315in}{1.688225in}}%
\pgfpathcurveto{\pgfqpoint{1.266315in}{1.679988in}}{\pgfqpoint{1.269587in}{1.672088in}}{\pgfqpoint{1.275411in}{1.666264in}}%
\pgfpathcurveto{\pgfqpoint{1.281235in}{1.660440in}}{\pgfqpoint{1.289135in}{1.657168in}}{\pgfqpoint{1.297372in}{1.657168in}}%
\pgfpathclose%
\pgfusepath{stroke,fill}%
\end{pgfscope}%
\begin{pgfscope}%
\pgfpathrectangle{\pgfqpoint{0.100000in}{0.212622in}}{\pgfqpoint{3.696000in}{3.696000in}}%
\pgfusepath{clip}%
\pgfsetbuttcap%
\pgfsetroundjoin%
\definecolor{currentfill}{rgb}{0.121569,0.466667,0.705882}%
\pgfsetfillcolor{currentfill}%
\pgfsetfillopacity{0.384676}%
\pgfsetlinewidth{1.003750pt}%
\definecolor{currentstroke}{rgb}{0.121569,0.466667,0.705882}%
\pgfsetstrokecolor{currentstroke}%
\pgfsetstrokeopacity{0.384676}%
\pgfsetdash{}{0pt}%
\pgfpathmoveto{\pgfqpoint{1.293779in}{1.654321in}}%
\pgfpathcurveto{\pgfqpoint{1.302016in}{1.654321in}}{\pgfqpoint{1.309916in}{1.657594in}}{\pgfqpoint{1.315740in}{1.663417in}}%
\pgfpathcurveto{\pgfqpoint{1.321563in}{1.669241in}}{\pgfqpoint{1.324836in}{1.677141in}}{\pgfqpoint{1.324836in}{1.685378in}}%
\pgfpathcurveto{\pgfqpoint{1.324836in}{1.693614in}}{\pgfqpoint{1.321563in}{1.701514in}}{\pgfqpoint{1.315740in}{1.707338in}}%
\pgfpathcurveto{\pgfqpoint{1.309916in}{1.713162in}}{\pgfqpoint{1.302016in}{1.716434in}}{\pgfqpoint{1.293779in}{1.716434in}}%
\pgfpathcurveto{\pgfqpoint{1.285543in}{1.716434in}}{\pgfqpoint{1.277643in}{1.713162in}}{\pgfqpoint{1.271819in}{1.707338in}}%
\pgfpathcurveto{\pgfqpoint{1.265995in}{1.701514in}}{\pgfqpoint{1.262723in}{1.693614in}}{\pgfqpoint{1.262723in}{1.685378in}}%
\pgfpathcurveto{\pgfqpoint{1.262723in}{1.677141in}}{\pgfqpoint{1.265995in}{1.669241in}}{\pgfqpoint{1.271819in}{1.663417in}}%
\pgfpathcurveto{\pgfqpoint{1.277643in}{1.657594in}}{\pgfqpoint{1.285543in}{1.654321in}}{\pgfqpoint{1.293779in}{1.654321in}}%
\pgfpathclose%
\pgfusepath{stroke,fill}%
\end{pgfscope}%
\begin{pgfscope}%
\pgfpathrectangle{\pgfqpoint{0.100000in}{0.212622in}}{\pgfqpoint{3.696000in}{3.696000in}}%
\pgfusepath{clip}%
\pgfsetbuttcap%
\pgfsetroundjoin%
\definecolor{currentfill}{rgb}{0.121569,0.466667,0.705882}%
\pgfsetfillcolor{currentfill}%
\pgfsetfillopacity{0.385628}%
\pgfsetlinewidth{1.003750pt}%
\definecolor{currentstroke}{rgb}{0.121569,0.466667,0.705882}%
\pgfsetstrokecolor{currentstroke}%
\pgfsetstrokeopacity{0.385628}%
\pgfsetdash{}{0pt}%
\pgfpathmoveto{\pgfqpoint{1.290449in}{1.652419in}}%
\pgfpathcurveto{\pgfqpoint{1.298685in}{1.652419in}}{\pgfqpoint{1.306585in}{1.655691in}}{\pgfqpoint{1.312409in}{1.661515in}}%
\pgfpathcurveto{\pgfqpoint{1.318233in}{1.667339in}}{\pgfqpoint{1.321505in}{1.675239in}}{\pgfqpoint{1.321505in}{1.683475in}}%
\pgfpathcurveto{\pgfqpoint{1.321505in}{1.691712in}}{\pgfqpoint{1.318233in}{1.699612in}}{\pgfqpoint{1.312409in}{1.705436in}}%
\pgfpathcurveto{\pgfqpoint{1.306585in}{1.711260in}}{\pgfqpoint{1.298685in}{1.714532in}}{\pgfqpoint{1.290449in}{1.714532in}}%
\pgfpathcurveto{\pgfqpoint{1.282212in}{1.714532in}}{\pgfqpoint{1.274312in}{1.711260in}}{\pgfqpoint{1.268488in}{1.705436in}}%
\pgfpathcurveto{\pgfqpoint{1.262664in}{1.699612in}}{\pgfqpoint{1.259392in}{1.691712in}}{\pgfqpoint{1.259392in}{1.683475in}}%
\pgfpathcurveto{\pgfqpoint{1.259392in}{1.675239in}}{\pgfqpoint{1.262664in}{1.667339in}}{\pgfqpoint{1.268488in}{1.661515in}}%
\pgfpathcurveto{\pgfqpoint{1.274312in}{1.655691in}}{\pgfqpoint{1.282212in}{1.652419in}}{\pgfqpoint{1.290449in}{1.652419in}}%
\pgfpathclose%
\pgfusepath{stroke,fill}%
\end{pgfscope}%
\begin{pgfscope}%
\pgfpathrectangle{\pgfqpoint{0.100000in}{0.212622in}}{\pgfqpoint{3.696000in}{3.696000in}}%
\pgfusepath{clip}%
\pgfsetbuttcap%
\pgfsetroundjoin%
\definecolor{currentfill}{rgb}{0.121569,0.466667,0.705882}%
\pgfsetfillcolor{currentfill}%
\pgfsetfillopacity{0.387027}%
\pgfsetlinewidth{1.003750pt}%
\definecolor{currentstroke}{rgb}{0.121569,0.466667,0.705882}%
\pgfsetstrokecolor{currentstroke}%
\pgfsetstrokeopacity{0.387027}%
\pgfsetdash{}{0pt}%
\pgfpathmoveto{\pgfqpoint{1.288532in}{1.653078in}}%
\pgfpathcurveto{\pgfqpoint{1.296768in}{1.653078in}}{\pgfqpoint{1.304669in}{1.656351in}}{\pgfqpoint{1.310492in}{1.662175in}}%
\pgfpathcurveto{\pgfqpoint{1.316316in}{1.667999in}}{\pgfqpoint{1.319589in}{1.675899in}}{\pgfqpoint{1.319589in}{1.684135in}}%
\pgfpathcurveto{\pgfqpoint{1.319589in}{1.692371in}}{\pgfqpoint{1.316316in}{1.700271in}}{\pgfqpoint{1.310492in}{1.706095in}}%
\pgfpathcurveto{\pgfqpoint{1.304669in}{1.711919in}}{\pgfqpoint{1.296768in}{1.715191in}}{\pgfqpoint{1.288532in}{1.715191in}}%
\pgfpathcurveto{\pgfqpoint{1.280296in}{1.715191in}}{\pgfqpoint{1.272396in}{1.711919in}}{\pgfqpoint{1.266572in}{1.706095in}}%
\pgfpathcurveto{\pgfqpoint{1.260748in}{1.700271in}}{\pgfqpoint{1.257476in}{1.692371in}}{\pgfqpoint{1.257476in}{1.684135in}}%
\pgfpathcurveto{\pgfqpoint{1.257476in}{1.675899in}}{\pgfqpoint{1.260748in}{1.667999in}}{\pgfqpoint{1.266572in}{1.662175in}}%
\pgfpathcurveto{\pgfqpoint{1.272396in}{1.656351in}}{\pgfqpoint{1.280296in}{1.653078in}}{\pgfqpoint{1.288532in}{1.653078in}}%
\pgfpathclose%
\pgfusepath{stroke,fill}%
\end{pgfscope}%
\begin{pgfscope}%
\pgfpathrectangle{\pgfqpoint{0.100000in}{0.212622in}}{\pgfqpoint{3.696000in}{3.696000in}}%
\pgfusepath{clip}%
\pgfsetbuttcap%
\pgfsetroundjoin%
\definecolor{currentfill}{rgb}{0.121569,0.466667,0.705882}%
\pgfsetfillcolor{currentfill}%
\pgfsetfillopacity{0.387794}%
\pgfsetlinewidth{1.003750pt}%
\definecolor{currentstroke}{rgb}{0.121569,0.466667,0.705882}%
\pgfsetstrokecolor{currentstroke}%
\pgfsetstrokeopacity{0.387794}%
\pgfsetdash{}{0pt}%
\pgfpathmoveto{\pgfqpoint{1.523274in}{1.725693in}}%
\pgfpathcurveto{\pgfqpoint{1.531511in}{1.725693in}}{\pgfqpoint{1.539411in}{1.728965in}}{\pgfqpoint{1.545235in}{1.734789in}}%
\pgfpathcurveto{\pgfqpoint{1.551059in}{1.740613in}}{\pgfqpoint{1.554331in}{1.748513in}}{\pgfqpoint{1.554331in}{1.756750in}}%
\pgfpathcurveto{\pgfqpoint{1.554331in}{1.764986in}}{\pgfqpoint{1.551059in}{1.772886in}}{\pgfqpoint{1.545235in}{1.778710in}}%
\pgfpathcurveto{\pgfqpoint{1.539411in}{1.784534in}}{\pgfqpoint{1.531511in}{1.787806in}}{\pgfqpoint{1.523274in}{1.787806in}}%
\pgfpathcurveto{\pgfqpoint{1.515038in}{1.787806in}}{\pgfqpoint{1.507138in}{1.784534in}}{\pgfqpoint{1.501314in}{1.778710in}}%
\pgfpathcurveto{\pgfqpoint{1.495490in}{1.772886in}}{\pgfqpoint{1.492218in}{1.764986in}}{\pgfqpoint{1.492218in}{1.756750in}}%
\pgfpathcurveto{\pgfqpoint{1.492218in}{1.748513in}}{\pgfqpoint{1.495490in}{1.740613in}}{\pgfqpoint{1.501314in}{1.734789in}}%
\pgfpathcurveto{\pgfqpoint{1.507138in}{1.728965in}}{\pgfqpoint{1.515038in}{1.725693in}}{\pgfqpoint{1.523274in}{1.725693in}}%
\pgfpathclose%
\pgfusepath{stroke,fill}%
\end{pgfscope}%
\begin{pgfscope}%
\pgfpathrectangle{\pgfqpoint{0.100000in}{0.212622in}}{\pgfqpoint{3.696000in}{3.696000in}}%
\pgfusepath{clip}%
\pgfsetbuttcap%
\pgfsetroundjoin%
\definecolor{currentfill}{rgb}{0.121569,0.466667,0.705882}%
\pgfsetfillcolor{currentfill}%
\pgfsetfillopacity{0.389196}%
\pgfsetlinewidth{1.003750pt}%
\definecolor{currentstroke}{rgb}{0.121569,0.466667,0.705882}%
\pgfsetstrokecolor{currentstroke}%
\pgfsetstrokeopacity{0.389196}%
\pgfsetdash{}{0pt}%
\pgfpathmoveto{\pgfqpoint{1.284484in}{1.653331in}}%
\pgfpathcurveto{\pgfqpoint{1.292720in}{1.653331in}}{\pgfqpoint{1.300620in}{1.656604in}}{\pgfqpoint{1.306444in}{1.662428in}}%
\pgfpathcurveto{\pgfqpoint{1.312268in}{1.668252in}}{\pgfqpoint{1.315540in}{1.676152in}}{\pgfqpoint{1.315540in}{1.684388in}}%
\pgfpathcurveto{\pgfqpoint{1.315540in}{1.692624in}}{\pgfqpoint{1.312268in}{1.700524in}}{\pgfqpoint{1.306444in}{1.706348in}}%
\pgfpathcurveto{\pgfqpoint{1.300620in}{1.712172in}}{\pgfqpoint{1.292720in}{1.715444in}}{\pgfqpoint{1.284484in}{1.715444in}}%
\pgfpathcurveto{\pgfqpoint{1.276247in}{1.715444in}}{\pgfqpoint{1.268347in}{1.712172in}}{\pgfqpoint{1.262524in}{1.706348in}}%
\pgfpathcurveto{\pgfqpoint{1.256700in}{1.700524in}}{\pgfqpoint{1.253427in}{1.692624in}}{\pgfqpoint{1.253427in}{1.684388in}}%
\pgfpathcurveto{\pgfqpoint{1.253427in}{1.676152in}}{\pgfqpoint{1.256700in}{1.668252in}}{\pgfqpoint{1.262524in}{1.662428in}}%
\pgfpathcurveto{\pgfqpoint{1.268347in}{1.656604in}}{\pgfqpoint{1.276247in}{1.653331in}}{\pgfqpoint{1.284484in}{1.653331in}}%
\pgfpathclose%
\pgfusepath{stroke,fill}%
\end{pgfscope}%
\begin{pgfscope}%
\pgfpathrectangle{\pgfqpoint{0.100000in}{0.212622in}}{\pgfqpoint{3.696000in}{3.696000in}}%
\pgfusepath{clip}%
\pgfsetbuttcap%
\pgfsetroundjoin%
\definecolor{currentfill}{rgb}{0.121569,0.466667,0.705882}%
\pgfsetfillcolor{currentfill}%
\pgfsetfillopacity{0.390147}%
\pgfsetlinewidth{1.003750pt}%
\definecolor{currentstroke}{rgb}{0.121569,0.466667,0.705882}%
\pgfsetstrokecolor{currentstroke}%
\pgfsetstrokeopacity{0.390147}%
\pgfsetdash{}{0pt}%
\pgfpathmoveto{\pgfqpoint{1.281266in}{1.650304in}}%
\pgfpathcurveto{\pgfqpoint{1.289503in}{1.650304in}}{\pgfqpoint{1.297403in}{1.653576in}}{\pgfqpoint{1.303227in}{1.659400in}}%
\pgfpathcurveto{\pgfqpoint{1.309051in}{1.665224in}}{\pgfqpoint{1.312323in}{1.673124in}}{\pgfqpoint{1.312323in}{1.681360in}}%
\pgfpathcurveto{\pgfqpoint{1.312323in}{1.689596in}}{\pgfqpoint{1.309051in}{1.697497in}}{\pgfqpoint{1.303227in}{1.703320in}}%
\pgfpathcurveto{\pgfqpoint{1.297403in}{1.709144in}}{\pgfqpoint{1.289503in}{1.712417in}}{\pgfqpoint{1.281266in}{1.712417in}}%
\pgfpathcurveto{\pgfqpoint{1.273030in}{1.712417in}}{\pgfqpoint{1.265130in}{1.709144in}}{\pgfqpoint{1.259306in}{1.703320in}}%
\pgfpathcurveto{\pgfqpoint{1.253482in}{1.697497in}}{\pgfqpoint{1.250210in}{1.689596in}}{\pgfqpoint{1.250210in}{1.681360in}}%
\pgfpathcurveto{\pgfqpoint{1.250210in}{1.673124in}}{\pgfqpoint{1.253482in}{1.665224in}}{\pgfqpoint{1.259306in}{1.659400in}}%
\pgfpathcurveto{\pgfqpoint{1.265130in}{1.653576in}}{\pgfqpoint{1.273030in}{1.650304in}}{\pgfqpoint{1.281266in}{1.650304in}}%
\pgfpathclose%
\pgfusepath{stroke,fill}%
\end{pgfscope}%
\begin{pgfscope}%
\pgfpathrectangle{\pgfqpoint{0.100000in}{0.212622in}}{\pgfqpoint{3.696000in}{3.696000in}}%
\pgfusepath{clip}%
\pgfsetbuttcap%
\pgfsetroundjoin%
\definecolor{currentfill}{rgb}{0.121569,0.466667,0.705882}%
\pgfsetfillcolor{currentfill}%
\pgfsetfillopacity{0.391561}%
\pgfsetlinewidth{1.003750pt}%
\definecolor{currentstroke}{rgb}{0.121569,0.466667,0.705882}%
\pgfsetstrokecolor{currentstroke}%
\pgfsetstrokeopacity{0.391561}%
\pgfsetdash{}{0pt}%
\pgfpathmoveto{\pgfqpoint{1.278507in}{1.650722in}}%
\pgfpathcurveto{\pgfqpoint{1.286744in}{1.650722in}}{\pgfqpoint{1.294644in}{1.653994in}}{\pgfqpoint{1.300468in}{1.659818in}}%
\pgfpathcurveto{\pgfqpoint{1.306292in}{1.665642in}}{\pgfqpoint{1.309564in}{1.673542in}}{\pgfqpoint{1.309564in}{1.681779in}}%
\pgfpathcurveto{\pgfqpoint{1.309564in}{1.690015in}}{\pgfqpoint{1.306292in}{1.697915in}}{\pgfqpoint{1.300468in}{1.703739in}}%
\pgfpathcurveto{\pgfqpoint{1.294644in}{1.709563in}}{\pgfqpoint{1.286744in}{1.712835in}}{\pgfqpoint{1.278507in}{1.712835in}}%
\pgfpathcurveto{\pgfqpoint{1.270271in}{1.712835in}}{\pgfqpoint{1.262371in}{1.709563in}}{\pgfqpoint{1.256547in}{1.703739in}}%
\pgfpathcurveto{\pgfqpoint{1.250723in}{1.697915in}}{\pgfqpoint{1.247451in}{1.690015in}}{\pgfqpoint{1.247451in}{1.681779in}}%
\pgfpathcurveto{\pgfqpoint{1.247451in}{1.673542in}}{\pgfqpoint{1.250723in}{1.665642in}}{\pgfqpoint{1.256547in}{1.659818in}}%
\pgfpathcurveto{\pgfqpoint{1.262371in}{1.653994in}}{\pgfqpoint{1.270271in}{1.650722in}}{\pgfqpoint{1.278507in}{1.650722in}}%
\pgfpathclose%
\pgfusepath{stroke,fill}%
\end{pgfscope}%
\begin{pgfscope}%
\pgfpathrectangle{\pgfqpoint{0.100000in}{0.212622in}}{\pgfqpoint{3.696000in}{3.696000in}}%
\pgfusepath{clip}%
\pgfsetbuttcap%
\pgfsetroundjoin%
\definecolor{currentfill}{rgb}{0.121569,0.466667,0.705882}%
\pgfsetfillcolor{currentfill}%
\pgfsetfillopacity{0.393823}%
\pgfsetlinewidth{1.003750pt}%
\definecolor{currentstroke}{rgb}{0.121569,0.466667,0.705882}%
\pgfsetstrokecolor{currentstroke}%
\pgfsetstrokeopacity{0.393823}%
\pgfsetdash{}{0pt}%
\pgfpathmoveto{\pgfqpoint{1.274449in}{1.648692in}}%
\pgfpathcurveto{\pgfqpoint{1.282685in}{1.648692in}}{\pgfqpoint{1.290585in}{1.651965in}}{\pgfqpoint{1.296409in}{1.657789in}}%
\pgfpathcurveto{\pgfqpoint{1.302233in}{1.663613in}}{\pgfqpoint{1.305505in}{1.671513in}}{\pgfqpoint{1.305505in}{1.679749in}}%
\pgfpathcurveto{\pgfqpoint{1.305505in}{1.687985in}}{\pgfqpoint{1.302233in}{1.695885in}}{\pgfqpoint{1.296409in}{1.701709in}}%
\pgfpathcurveto{\pgfqpoint{1.290585in}{1.707533in}}{\pgfqpoint{1.282685in}{1.710805in}}{\pgfqpoint{1.274449in}{1.710805in}}%
\pgfpathcurveto{\pgfqpoint{1.266213in}{1.710805in}}{\pgfqpoint{1.258313in}{1.707533in}}{\pgfqpoint{1.252489in}{1.701709in}}%
\pgfpathcurveto{\pgfqpoint{1.246665in}{1.695885in}}{\pgfqpoint{1.243392in}{1.687985in}}{\pgfqpoint{1.243392in}{1.679749in}}%
\pgfpathcurveto{\pgfqpoint{1.243392in}{1.671513in}}{\pgfqpoint{1.246665in}{1.663613in}}{\pgfqpoint{1.252489in}{1.657789in}}%
\pgfpathcurveto{\pgfqpoint{1.258313in}{1.651965in}}{\pgfqpoint{1.266213in}{1.648692in}}{\pgfqpoint{1.274449in}{1.648692in}}%
\pgfpathclose%
\pgfusepath{stroke,fill}%
\end{pgfscope}%
\begin{pgfscope}%
\pgfpathrectangle{\pgfqpoint{0.100000in}{0.212622in}}{\pgfqpoint{3.696000in}{3.696000in}}%
\pgfusepath{clip}%
\pgfsetbuttcap%
\pgfsetroundjoin%
\definecolor{currentfill}{rgb}{0.121569,0.466667,0.705882}%
\pgfsetfillcolor{currentfill}%
\pgfsetfillopacity{0.394533}%
\pgfsetlinewidth{1.003750pt}%
\definecolor{currentstroke}{rgb}{0.121569,0.466667,0.705882}%
\pgfsetstrokecolor{currentstroke}%
\pgfsetstrokeopacity{0.394533}%
\pgfsetdash{}{0pt}%
\pgfpathmoveto{\pgfqpoint{1.272074in}{1.646184in}}%
\pgfpathcurveto{\pgfqpoint{1.280310in}{1.646184in}}{\pgfqpoint{1.288211in}{1.649456in}}{\pgfqpoint{1.294034in}{1.655280in}}%
\pgfpathcurveto{\pgfqpoint{1.299858in}{1.661104in}}{\pgfqpoint{1.303131in}{1.669004in}}{\pgfqpoint{1.303131in}{1.677240in}}%
\pgfpathcurveto{\pgfqpoint{1.303131in}{1.685477in}}{\pgfqpoint{1.299858in}{1.693377in}}{\pgfqpoint{1.294034in}{1.699201in}}%
\pgfpathcurveto{\pgfqpoint{1.288211in}{1.705025in}}{\pgfqpoint{1.280310in}{1.708297in}}{\pgfqpoint{1.272074in}{1.708297in}}%
\pgfpathcurveto{\pgfqpoint{1.263838in}{1.708297in}}{\pgfqpoint{1.255938in}{1.705025in}}{\pgfqpoint{1.250114in}{1.699201in}}%
\pgfpathcurveto{\pgfqpoint{1.244290in}{1.693377in}}{\pgfqpoint{1.241018in}{1.685477in}}{\pgfqpoint{1.241018in}{1.677240in}}%
\pgfpathcurveto{\pgfqpoint{1.241018in}{1.669004in}}{\pgfqpoint{1.244290in}{1.661104in}}{\pgfqpoint{1.250114in}{1.655280in}}%
\pgfpathcurveto{\pgfqpoint{1.255938in}{1.649456in}}{\pgfqpoint{1.263838in}{1.646184in}}{\pgfqpoint{1.272074in}{1.646184in}}%
\pgfpathclose%
\pgfusepath{stroke,fill}%
\end{pgfscope}%
\begin{pgfscope}%
\pgfpathrectangle{\pgfqpoint{0.100000in}{0.212622in}}{\pgfqpoint{3.696000in}{3.696000in}}%
\pgfusepath{clip}%
\pgfsetbuttcap%
\pgfsetroundjoin%
\definecolor{currentfill}{rgb}{0.121569,0.466667,0.705882}%
\pgfsetfillcolor{currentfill}%
\pgfsetfillopacity{0.395083}%
\pgfsetlinewidth{1.003750pt}%
\definecolor{currentstroke}{rgb}{0.121569,0.466667,0.705882}%
\pgfsetstrokecolor{currentstroke}%
\pgfsetstrokeopacity{0.395083}%
\pgfsetdash{}{0pt}%
\pgfpathmoveto{\pgfqpoint{1.522848in}{1.722474in}}%
\pgfpathcurveto{\pgfqpoint{1.531084in}{1.722474in}}{\pgfqpoint{1.538985in}{1.725747in}}{\pgfqpoint{1.544808in}{1.731571in}}%
\pgfpathcurveto{\pgfqpoint{1.550632in}{1.737394in}}{\pgfqpoint{1.553905in}{1.745294in}}{\pgfqpoint{1.553905in}{1.753531in}}%
\pgfpathcurveto{\pgfqpoint{1.553905in}{1.761767in}}{\pgfqpoint{1.550632in}{1.769667in}}{\pgfqpoint{1.544808in}{1.775491in}}%
\pgfpathcurveto{\pgfqpoint{1.538985in}{1.781315in}}{\pgfqpoint{1.531084in}{1.784587in}}{\pgfqpoint{1.522848in}{1.784587in}}%
\pgfpathcurveto{\pgfqpoint{1.514612in}{1.784587in}}{\pgfqpoint{1.506712in}{1.781315in}}{\pgfqpoint{1.500888in}{1.775491in}}%
\pgfpathcurveto{\pgfqpoint{1.495064in}{1.769667in}}{\pgfqpoint{1.491792in}{1.761767in}}{\pgfqpoint{1.491792in}{1.753531in}}%
\pgfpathcurveto{\pgfqpoint{1.491792in}{1.745294in}}{\pgfqpoint{1.495064in}{1.737394in}}{\pgfqpoint{1.500888in}{1.731571in}}%
\pgfpathcurveto{\pgfqpoint{1.506712in}{1.725747in}}{\pgfqpoint{1.514612in}{1.722474in}}{\pgfqpoint{1.522848in}{1.722474in}}%
\pgfpathclose%
\pgfusepath{stroke,fill}%
\end{pgfscope}%
\begin{pgfscope}%
\pgfpathrectangle{\pgfqpoint{0.100000in}{0.212622in}}{\pgfqpoint{3.696000in}{3.696000in}}%
\pgfusepath{clip}%
\pgfsetbuttcap%
\pgfsetroundjoin%
\definecolor{currentfill}{rgb}{0.121569,0.466667,0.705882}%
\pgfsetfillcolor{currentfill}%
\pgfsetfillopacity{0.396126}%
\pgfsetlinewidth{1.003750pt}%
\definecolor{currentstroke}{rgb}{0.121569,0.466667,0.705882}%
\pgfsetstrokecolor{currentstroke}%
\pgfsetstrokeopacity{0.396126}%
\pgfsetdash{}{0pt}%
\pgfpathmoveto{\pgfqpoint{1.267610in}{1.643235in}}%
\pgfpathcurveto{\pgfqpoint{1.275846in}{1.643235in}}{\pgfqpoint{1.283746in}{1.646507in}}{\pgfqpoint{1.289570in}{1.652331in}}%
\pgfpathcurveto{\pgfqpoint{1.295394in}{1.658155in}}{\pgfqpoint{1.298667in}{1.666055in}}{\pgfqpoint{1.298667in}{1.674291in}}%
\pgfpathcurveto{\pgfqpoint{1.298667in}{1.682527in}}{\pgfqpoint{1.295394in}{1.690427in}}{\pgfqpoint{1.289570in}{1.696251in}}%
\pgfpathcurveto{\pgfqpoint{1.283746in}{1.702075in}}{\pgfqpoint{1.275846in}{1.705348in}}{\pgfqpoint{1.267610in}{1.705348in}}%
\pgfpathcurveto{\pgfqpoint{1.259374in}{1.705348in}}{\pgfqpoint{1.251474in}{1.702075in}}{\pgfqpoint{1.245650in}{1.696251in}}%
\pgfpathcurveto{\pgfqpoint{1.239826in}{1.690427in}}{\pgfqpoint{1.236554in}{1.682527in}}{\pgfqpoint{1.236554in}{1.674291in}}%
\pgfpathcurveto{\pgfqpoint{1.236554in}{1.666055in}}{\pgfqpoint{1.239826in}{1.658155in}}{\pgfqpoint{1.245650in}{1.652331in}}%
\pgfpathcurveto{\pgfqpoint{1.251474in}{1.646507in}}{\pgfqpoint{1.259374in}{1.643235in}}{\pgfqpoint{1.267610in}{1.643235in}}%
\pgfpathclose%
\pgfusepath{stroke,fill}%
\end{pgfscope}%
\begin{pgfscope}%
\pgfpathrectangle{\pgfqpoint{0.100000in}{0.212622in}}{\pgfqpoint{3.696000in}{3.696000in}}%
\pgfusepath{clip}%
\pgfsetbuttcap%
\pgfsetroundjoin%
\definecolor{currentfill}{rgb}{0.121569,0.466667,0.705882}%
\pgfsetfillcolor{currentfill}%
\pgfsetfillopacity{0.401545}%
\pgfsetlinewidth{1.003750pt}%
\definecolor{currentstroke}{rgb}{0.121569,0.466667,0.705882}%
\pgfsetstrokecolor{currentstroke}%
\pgfsetstrokeopacity{0.401545}%
\pgfsetdash{}{0pt}%
\pgfpathmoveto{\pgfqpoint{1.260556in}{1.647818in}}%
\pgfpathcurveto{\pgfqpoint{1.268792in}{1.647818in}}{\pgfqpoint{1.276692in}{1.651091in}}{\pgfqpoint{1.282516in}{1.656915in}}%
\pgfpathcurveto{\pgfqpoint{1.288340in}{1.662739in}}{\pgfqpoint{1.291612in}{1.670639in}}{\pgfqpoint{1.291612in}{1.678875in}}%
\pgfpathcurveto{\pgfqpoint{1.291612in}{1.687111in}}{\pgfqpoint{1.288340in}{1.695011in}}{\pgfqpoint{1.282516in}{1.700835in}}%
\pgfpathcurveto{\pgfqpoint{1.276692in}{1.706659in}}{\pgfqpoint{1.268792in}{1.709931in}}{\pgfqpoint{1.260556in}{1.709931in}}%
\pgfpathcurveto{\pgfqpoint{1.252319in}{1.709931in}}{\pgfqpoint{1.244419in}{1.706659in}}{\pgfqpoint{1.238595in}{1.700835in}}%
\pgfpathcurveto{\pgfqpoint{1.232771in}{1.695011in}}{\pgfqpoint{1.229499in}{1.687111in}}{\pgfqpoint{1.229499in}{1.678875in}}%
\pgfpathcurveto{\pgfqpoint{1.229499in}{1.670639in}}{\pgfqpoint{1.232771in}{1.662739in}}{\pgfqpoint{1.238595in}{1.656915in}}%
\pgfpathcurveto{\pgfqpoint{1.244419in}{1.651091in}}{\pgfqpoint{1.252319in}{1.647818in}}{\pgfqpoint{1.260556in}{1.647818in}}%
\pgfpathclose%
\pgfusepath{stroke,fill}%
\end{pgfscope}%
\begin{pgfscope}%
\pgfpathrectangle{\pgfqpoint{0.100000in}{0.212622in}}{\pgfqpoint{3.696000in}{3.696000in}}%
\pgfusepath{clip}%
\pgfsetbuttcap%
\pgfsetroundjoin%
\definecolor{currentfill}{rgb}{0.121569,0.466667,0.705882}%
\pgfsetfillcolor{currentfill}%
\pgfsetfillopacity{0.403451}%
\pgfsetlinewidth{1.003750pt}%
\definecolor{currentstroke}{rgb}{0.121569,0.466667,0.705882}%
\pgfsetstrokecolor{currentstroke}%
\pgfsetstrokeopacity{0.403451}%
\pgfsetdash{}{0pt}%
\pgfpathmoveto{\pgfqpoint{1.253940in}{1.642733in}}%
\pgfpathcurveto{\pgfqpoint{1.262176in}{1.642733in}}{\pgfqpoint{1.270076in}{1.646005in}}{\pgfqpoint{1.275900in}{1.651829in}}%
\pgfpathcurveto{\pgfqpoint{1.281724in}{1.657653in}}{\pgfqpoint{1.284996in}{1.665553in}}{\pgfqpoint{1.284996in}{1.673790in}}%
\pgfpathcurveto{\pgfqpoint{1.284996in}{1.682026in}}{\pgfqpoint{1.281724in}{1.689926in}}{\pgfqpoint{1.275900in}{1.695750in}}%
\pgfpathcurveto{\pgfqpoint{1.270076in}{1.701574in}}{\pgfqpoint{1.262176in}{1.704846in}}{\pgfqpoint{1.253940in}{1.704846in}}%
\pgfpathcurveto{\pgfqpoint{1.245704in}{1.704846in}}{\pgfqpoint{1.237804in}{1.701574in}}{\pgfqpoint{1.231980in}{1.695750in}}%
\pgfpathcurveto{\pgfqpoint{1.226156in}{1.689926in}}{\pgfqpoint{1.222883in}{1.682026in}}{\pgfqpoint{1.222883in}{1.673790in}}%
\pgfpathcurveto{\pgfqpoint{1.222883in}{1.665553in}}{\pgfqpoint{1.226156in}{1.657653in}}{\pgfqpoint{1.231980in}{1.651829in}}%
\pgfpathcurveto{\pgfqpoint{1.237804in}{1.646005in}}{\pgfqpoint{1.245704in}{1.642733in}}{\pgfqpoint{1.253940in}{1.642733in}}%
\pgfpathclose%
\pgfusepath{stroke,fill}%
\end{pgfscope}%
\begin{pgfscope}%
\pgfpathrectangle{\pgfqpoint{0.100000in}{0.212622in}}{\pgfqpoint{3.696000in}{3.696000in}}%
\pgfusepath{clip}%
\pgfsetbuttcap%
\pgfsetroundjoin%
\definecolor{currentfill}{rgb}{0.121569,0.466667,0.705882}%
\pgfsetfillcolor{currentfill}%
\pgfsetfillopacity{0.404021}%
\pgfsetlinewidth{1.003750pt}%
\definecolor{currentstroke}{rgb}{0.121569,0.466667,0.705882}%
\pgfsetstrokecolor{currentstroke}%
\pgfsetstrokeopacity{0.404021}%
\pgfsetdash{}{0pt}%
\pgfpathmoveto{\pgfqpoint{1.526158in}{1.724590in}}%
\pgfpathcurveto{\pgfqpoint{1.534395in}{1.724590in}}{\pgfqpoint{1.542295in}{1.727862in}}{\pgfqpoint{1.548119in}{1.733686in}}%
\pgfpathcurveto{\pgfqpoint{1.553943in}{1.739510in}}{\pgfqpoint{1.557215in}{1.747410in}}{\pgfqpoint{1.557215in}{1.755646in}}%
\pgfpathcurveto{\pgfqpoint{1.557215in}{1.763883in}}{\pgfqpoint{1.553943in}{1.771783in}}{\pgfqpoint{1.548119in}{1.777607in}}%
\pgfpathcurveto{\pgfqpoint{1.542295in}{1.783431in}}{\pgfqpoint{1.534395in}{1.786703in}}{\pgfqpoint{1.526158in}{1.786703in}}%
\pgfpathcurveto{\pgfqpoint{1.517922in}{1.786703in}}{\pgfqpoint{1.510022in}{1.783431in}}{\pgfqpoint{1.504198in}{1.777607in}}%
\pgfpathcurveto{\pgfqpoint{1.498374in}{1.771783in}}{\pgfqpoint{1.495102in}{1.763883in}}{\pgfqpoint{1.495102in}{1.755646in}}%
\pgfpathcurveto{\pgfqpoint{1.495102in}{1.747410in}}{\pgfqpoint{1.498374in}{1.739510in}}{\pgfqpoint{1.504198in}{1.733686in}}%
\pgfpathcurveto{\pgfqpoint{1.510022in}{1.727862in}}{\pgfqpoint{1.517922in}{1.724590in}}{\pgfqpoint{1.526158in}{1.724590in}}%
\pgfpathclose%
\pgfusepath{stroke,fill}%
\end{pgfscope}%
\begin{pgfscope}%
\pgfpathrectangle{\pgfqpoint{0.100000in}{0.212622in}}{\pgfqpoint{3.696000in}{3.696000in}}%
\pgfusepath{clip}%
\pgfsetbuttcap%
\pgfsetroundjoin%
\definecolor{currentfill}{rgb}{0.121569,0.466667,0.705882}%
\pgfsetfillcolor{currentfill}%
\pgfsetfillopacity{0.404682}%
\pgfsetlinewidth{1.003750pt}%
\definecolor{currentstroke}{rgb}{0.121569,0.466667,0.705882}%
\pgfsetstrokecolor{currentstroke}%
\pgfsetstrokeopacity{0.404682}%
\pgfsetdash{}{0pt}%
\pgfpathmoveto{\pgfqpoint{1.249501in}{1.639854in}}%
\pgfpathcurveto{\pgfqpoint{1.257737in}{1.639854in}}{\pgfqpoint{1.265637in}{1.643127in}}{\pgfqpoint{1.271461in}{1.648951in}}%
\pgfpathcurveto{\pgfqpoint{1.277285in}{1.654775in}}{\pgfqpoint{1.280557in}{1.662675in}}{\pgfqpoint{1.280557in}{1.670911in}}%
\pgfpathcurveto{\pgfqpoint{1.280557in}{1.679147in}}{\pgfqpoint{1.277285in}{1.687047in}}{\pgfqpoint{1.271461in}{1.692871in}}%
\pgfpathcurveto{\pgfqpoint{1.265637in}{1.698695in}}{\pgfqpoint{1.257737in}{1.701967in}}{\pgfqpoint{1.249501in}{1.701967in}}%
\pgfpathcurveto{\pgfqpoint{1.241264in}{1.701967in}}{\pgfqpoint{1.233364in}{1.698695in}}{\pgfqpoint{1.227540in}{1.692871in}}%
\pgfpathcurveto{\pgfqpoint{1.221716in}{1.687047in}}{\pgfqpoint{1.218444in}{1.679147in}}{\pgfqpoint{1.218444in}{1.670911in}}%
\pgfpathcurveto{\pgfqpoint{1.218444in}{1.662675in}}{\pgfqpoint{1.221716in}{1.654775in}}{\pgfqpoint{1.227540in}{1.648951in}}%
\pgfpathcurveto{\pgfqpoint{1.233364in}{1.643127in}}{\pgfqpoint{1.241264in}{1.639854in}}{\pgfqpoint{1.249501in}{1.639854in}}%
\pgfpathclose%
\pgfusepath{stroke,fill}%
\end{pgfscope}%
\begin{pgfscope}%
\pgfpathrectangle{\pgfqpoint{0.100000in}{0.212622in}}{\pgfqpoint{3.696000in}{3.696000in}}%
\pgfusepath{clip}%
\pgfsetbuttcap%
\pgfsetroundjoin%
\definecolor{currentfill}{rgb}{0.121569,0.466667,0.705882}%
\pgfsetfillcolor{currentfill}%
\pgfsetfillopacity{0.408613}%
\pgfsetlinewidth{1.003750pt}%
\definecolor{currentstroke}{rgb}{0.121569,0.466667,0.705882}%
\pgfsetstrokecolor{currentstroke}%
\pgfsetstrokeopacity{0.408613}%
\pgfsetdash{}{0pt}%
\pgfpathmoveto{\pgfqpoint{1.527463in}{1.723994in}}%
\pgfpathcurveto{\pgfqpoint{1.535699in}{1.723994in}}{\pgfqpoint{1.543599in}{1.727266in}}{\pgfqpoint{1.549423in}{1.733090in}}%
\pgfpathcurveto{\pgfqpoint{1.555247in}{1.738914in}}{\pgfqpoint{1.558519in}{1.746814in}}{\pgfqpoint{1.558519in}{1.755050in}}%
\pgfpathcurveto{\pgfqpoint{1.558519in}{1.763287in}}{\pgfqpoint{1.555247in}{1.771187in}}{\pgfqpoint{1.549423in}{1.777011in}}%
\pgfpathcurveto{\pgfqpoint{1.543599in}{1.782834in}}{\pgfqpoint{1.535699in}{1.786107in}}{\pgfqpoint{1.527463in}{1.786107in}}%
\pgfpathcurveto{\pgfqpoint{1.519226in}{1.786107in}}{\pgfqpoint{1.511326in}{1.782834in}}{\pgfqpoint{1.505502in}{1.777011in}}%
\pgfpathcurveto{\pgfqpoint{1.499678in}{1.771187in}}{\pgfqpoint{1.496406in}{1.763287in}}{\pgfqpoint{1.496406in}{1.755050in}}%
\pgfpathcurveto{\pgfqpoint{1.496406in}{1.746814in}}{\pgfqpoint{1.499678in}{1.738914in}}{\pgfqpoint{1.505502in}{1.733090in}}%
\pgfpathcurveto{\pgfqpoint{1.511326in}{1.727266in}}{\pgfqpoint{1.519226in}{1.723994in}}{\pgfqpoint{1.527463in}{1.723994in}}%
\pgfpathclose%
\pgfusepath{stroke,fill}%
\end{pgfscope}%
\begin{pgfscope}%
\pgfpathrectangle{\pgfqpoint{0.100000in}{0.212622in}}{\pgfqpoint{3.696000in}{3.696000in}}%
\pgfusepath{clip}%
\pgfsetbuttcap%
\pgfsetroundjoin%
\definecolor{currentfill}{rgb}{0.121569,0.466667,0.705882}%
\pgfsetfillcolor{currentfill}%
\pgfsetfillopacity{0.409627}%
\pgfsetlinewidth{1.003750pt}%
\definecolor{currentstroke}{rgb}{0.121569,0.466667,0.705882}%
\pgfsetstrokecolor{currentstroke}%
\pgfsetstrokeopacity{0.409627}%
\pgfsetdash{}{0pt}%
\pgfpathmoveto{\pgfqpoint{1.243545in}{1.643692in}}%
\pgfpathcurveto{\pgfqpoint{1.251781in}{1.643692in}}{\pgfqpoint{1.259681in}{1.646965in}}{\pgfqpoint{1.265505in}{1.652788in}}%
\pgfpathcurveto{\pgfqpoint{1.271329in}{1.658612in}}{\pgfqpoint{1.274601in}{1.666512in}}{\pgfqpoint{1.274601in}{1.674749in}}%
\pgfpathcurveto{\pgfqpoint{1.274601in}{1.682985in}}{\pgfqpoint{1.271329in}{1.690885in}}{\pgfqpoint{1.265505in}{1.696709in}}%
\pgfpathcurveto{\pgfqpoint{1.259681in}{1.702533in}}{\pgfqpoint{1.251781in}{1.705805in}}{\pgfqpoint{1.243545in}{1.705805in}}%
\pgfpathcurveto{\pgfqpoint{1.235308in}{1.705805in}}{\pgfqpoint{1.227408in}{1.702533in}}{\pgfqpoint{1.221584in}{1.696709in}}%
\pgfpathcurveto{\pgfqpoint{1.215760in}{1.690885in}}{\pgfqpoint{1.212488in}{1.682985in}}{\pgfqpoint{1.212488in}{1.674749in}}%
\pgfpathcurveto{\pgfqpoint{1.212488in}{1.666512in}}{\pgfqpoint{1.215760in}{1.658612in}}{\pgfqpoint{1.221584in}{1.652788in}}%
\pgfpathcurveto{\pgfqpoint{1.227408in}{1.646965in}}{\pgfqpoint{1.235308in}{1.643692in}}{\pgfqpoint{1.243545in}{1.643692in}}%
\pgfpathclose%
\pgfusepath{stroke,fill}%
\end{pgfscope}%
\begin{pgfscope}%
\pgfpathrectangle{\pgfqpoint{0.100000in}{0.212622in}}{\pgfqpoint{3.696000in}{3.696000in}}%
\pgfusepath{clip}%
\pgfsetbuttcap%
\pgfsetroundjoin%
\definecolor{currentfill}{rgb}{0.121569,0.466667,0.705882}%
\pgfsetfillcolor{currentfill}%
\pgfsetfillopacity{0.411522}%
\pgfsetlinewidth{1.003750pt}%
\definecolor{currentstroke}{rgb}{0.121569,0.466667,0.705882}%
\pgfsetstrokecolor{currentstroke}%
\pgfsetstrokeopacity{0.411522}%
\pgfsetdash{}{0pt}%
\pgfpathmoveto{\pgfqpoint{1.237758in}{1.641310in}}%
\pgfpathcurveto{\pgfqpoint{1.245994in}{1.641310in}}{\pgfqpoint{1.253894in}{1.644582in}}{\pgfqpoint{1.259718in}{1.650406in}}%
\pgfpathcurveto{\pgfqpoint{1.265542in}{1.656230in}}{\pgfqpoint{1.268814in}{1.664130in}}{\pgfqpoint{1.268814in}{1.672366in}}%
\pgfpathcurveto{\pgfqpoint{1.268814in}{1.680602in}}{\pgfqpoint{1.265542in}{1.688502in}}{\pgfqpoint{1.259718in}{1.694326in}}%
\pgfpathcurveto{\pgfqpoint{1.253894in}{1.700150in}}{\pgfqpoint{1.245994in}{1.703423in}}{\pgfqpoint{1.237758in}{1.703423in}}%
\pgfpathcurveto{\pgfqpoint{1.229522in}{1.703423in}}{\pgfqpoint{1.221622in}{1.700150in}}{\pgfqpoint{1.215798in}{1.694326in}}%
\pgfpathcurveto{\pgfqpoint{1.209974in}{1.688502in}}{\pgfqpoint{1.206701in}{1.680602in}}{\pgfqpoint{1.206701in}{1.672366in}}%
\pgfpathcurveto{\pgfqpoint{1.206701in}{1.664130in}}{\pgfqpoint{1.209974in}{1.656230in}}{\pgfqpoint{1.215798in}{1.650406in}}%
\pgfpathcurveto{\pgfqpoint{1.221622in}{1.644582in}}{\pgfqpoint{1.229522in}{1.641310in}}{\pgfqpoint{1.237758in}{1.641310in}}%
\pgfpathclose%
\pgfusepath{stroke,fill}%
\end{pgfscope}%
\begin{pgfscope}%
\pgfpathrectangle{\pgfqpoint{0.100000in}{0.212622in}}{\pgfqpoint{3.696000in}{3.696000in}}%
\pgfusepath{clip}%
\pgfsetbuttcap%
\pgfsetroundjoin%
\definecolor{currentfill}{rgb}{0.121569,0.466667,0.705882}%
\pgfsetfillcolor{currentfill}%
\pgfsetfillopacity{0.412176}%
\pgfsetlinewidth{1.003750pt}%
\definecolor{currentstroke}{rgb}{0.121569,0.466667,0.705882}%
\pgfsetstrokecolor{currentstroke}%
\pgfsetstrokeopacity{0.412176}%
\pgfsetdash{}{0pt}%
\pgfpathmoveto{\pgfqpoint{1.234365in}{1.636847in}}%
\pgfpathcurveto{\pgfqpoint{1.242601in}{1.636847in}}{\pgfqpoint{1.250501in}{1.640119in}}{\pgfqpoint{1.256325in}{1.645943in}}%
\pgfpathcurveto{\pgfqpoint{1.262149in}{1.651767in}}{\pgfqpoint{1.265422in}{1.659667in}}{\pgfqpoint{1.265422in}{1.667903in}}%
\pgfpathcurveto{\pgfqpoint{1.265422in}{1.676139in}}{\pgfqpoint{1.262149in}{1.684039in}}{\pgfqpoint{1.256325in}{1.689863in}}%
\pgfpathcurveto{\pgfqpoint{1.250501in}{1.695687in}}{\pgfqpoint{1.242601in}{1.698960in}}{\pgfqpoint{1.234365in}{1.698960in}}%
\pgfpathcurveto{\pgfqpoint{1.226129in}{1.698960in}}{\pgfqpoint{1.218229in}{1.695687in}}{\pgfqpoint{1.212405in}{1.689863in}}%
\pgfpathcurveto{\pgfqpoint{1.206581in}{1.684039in}}{\pgfqpoint{1.203309in}{1.676139in}}{\pgfqpoint{1.203309in}{1.667903in}}%
\pgfpathcurveto{\pgfqpoint{1.203309in}{1.659667in}}{\pgfqpoint{1.206581in}{1.651767in}}{\pgfqpoint{1.212405in}{1.645943in}}%
\pgfpathcurveto{\pgfqpoint{1.218229in}{1.640119in}}{\pgfqpoint{1.226129in}{1.636847in}}{\pgfqpoint{1.234365in}{1.636847in}}%
\pgfpathclose%
\pgfusepath{stroke,fill}%
\end{pgfscope}%
\begin{pgfscope}%
\pgfpathrectangle{\pgfqpoint{0.100000in}{0.212622in}}{\pgfqpoint{3.696000in}{3.696000in}}%
\pgfusepath{clip}%
\pgfsetbuttcap%
\pgfsetroundjoin%
\definecolor{currentfill}{rgb}{0.121569,0.466667,0.705882}%
\pgfsetfillcolor{currentfill}%
\pgfsetfillopacity{0.413437}%
\pgfsetlinewidth{1.003750pt}%
\definecolor{currentstroke}{rgb}{0.121569,0.466667,0.705882}%
\pgfsetstrokecolor{currentstroke}%
\pgfsetstrokeopacity{0.413437}%
\pgfsetdash{}{0pt}%
\pgfpathmoveto{\pgfqpoint{1.527548in}{1.722332in}}%
\pgfpathcurveto{\pgfqpoint{1.535784in}{1.722332in}}{\pgfqpoint{1.543684in}{1.725604in}}{\pgfqpoint{1.549508in}{1.731428in}}%
\pgfpathcurveto{\pgfqpoint{1.555332in}{1.737252in}}{\pgfqpoint{1.558605in}{1.745152in}}{\pgfqpoint{1.558605in}{1.753388in}}%
\pgfpathcurveto{\pgfqpoint{1.558605in}{1.761624in}}{\pgfqpoint{1.555332in}{1.769524in}}{\pgfqpoint{1.549508in}{1.775348in}}%
\pgfpathcurveto{\pgfqpoint{1.543684in}{1.781172in}}{\pgfqpoint{1.535784in}{1.784445in}}{\pgfqpoint{1.527548in}{1.784445in}}%
\pgfpathcurveto{\pgfqpoint{1.519312in}{1.784445in}}{\pgfqpoint{1.511412in}{1.781172in}}{\pgfqpoint{1.505588in}{1.775348in}}%
\pgfpathcurveto{\pgfqpoint{1.499764in}{1.769524in}}{\pgfqpoint{1.496492in}{1.761624in}}{\pgfqpoint{1.496492in}{1.753388in}}%
\pgfpathcurveto{\pgfqpoint{1.496492in}{1.745152in}}{\pgfqpoint{1.499764in}{1.737252in}}{\pgfqpoint{1.505588in}{1.731428in}}%
\pgfpathcurveto{\pgfqpoint{1.511412in}{1.725604in}}{\pgfqpoint{1.519312in}{1.722332in}}{\pgfqpoint{1.527548in}{1.722332in}}%
\pgfpathclose%
\pgfusepath{stroke,fill}%
\end{pgfscope}%
\begin{pgfscope}%
\pgfpathrectangle{\pgfqpoint{0.100000in}{0.212622in}}{\pgfqpoint{3.696000in}{3.696000in}}%
\pgfusepath{clip}%
\pgfsetbuttcap%
\pgfsetroundjoin%
\definecolor{currentfill}{rgb}{0.121569,0.466667,0.705882}%
\pgfsetfillcolor{currentfill}%
\pgfsetfillopacity{0.416332}%
\pgfsetlinewidth{1.003750pt}%
\definecolor{currentstroke}{rgb}{0.121569,0.466667,0.705882}%
\pgfsetstrokecolor{currentstroke}%
\pgfsetstrokeopacity{0.416332}%
\pgfsetdash{}{0pt}%
\pgfpathmoveto{\pgfqpoint{1.229124in}{1.640809in}}%
\pgfpathcurveto{\pgfqpoint{1.237360in}{1.640809in}}{\pgfqpoint{1.245260in}{1.644081in}}{\pgfqpoint{1.251084in}{1.649905in}}%
\pgfpathcurveto{\pgfqpoint{1.256908in}{1.655729in}}{\pgfqpoint{1.260181in}{1.663629in}}{\pgfqpoint{1.260181in}{1.671865in}}%
\pgfpathcurveto{\pgfqpoint{1.260181in}{1.680101in}}{\pgfqpoint{1.256908in}{1.688001in}}{\pgfqpoint{1.251084in}{1.693825in}}%
\pgfpathcurveto{\pgfqpoint{1.245260in}{1.699649in}}{\pgfqpoint{1.237360in}{1.702922in}}{\pgfqpoint{1.229124in}{1.702922in}}%
\pgfpathcurveto{\pgfqpoint{1.220888in}{1.702922in}}{\pgfqpoint{1.212988in}{1.699649in}}{\pgfqpoint{1.207164in}{1.693825in}}%
\pgfpathcurveto{\pgfqpoint{1.201340in}{1.688001in}}{\pgfqpoint{1.198068in}{1.680101in}}{\pgfqpoint{1.198068in}{1.671865in}}%
\pgfpathcurveto{\pgfqpoint{1.198068in}{1.663629in}}{\pgfqpoint{1.201340in}{1.655729in}}{\pgfqpoint{1.207164in}{1.649905in}}%
\pgfpathcurveto{\pgfqpoint{1.212988in}{1.644081in}}{\pgfqpoint{1.220888in}{1.640809in}}{\pgfqpoint{1.229124in}{1.640809in}}%
\pgfpathclose%
\pgfusepath{stroke,fill}%
\end{pgfscope}%
\begin{pgfscope}%
\pgfpathrectangle{\pgfqpoint{0.100000in}{0.212622in}}{\pgfqpoint{3.696000in}{3.696000in}}%
\pgfusepath{clip}%
\pgfsetbuttcap%
\pgfsetroundjoin%
\definecolor{currentfill}{rgb}{0.121569,0.466667,0.705882}%
\pgfsetfillcolor{currentfill}%
\pgfsetfillopacity{0.417428}%
\pgfsetlinewidth{1.003750pt}%
\definecolor{currentstroke}{rgb}{0.121569,0.466667,0.705882}%
\pgfsetstrokecolor{currentstroke}%
\pgfsetstrokeopacity{0.417428}%
\pgfsetdash{}{0pt}%
\pgfpathmoveto{\pgfqpoint{1.224487in}{1.637945in}}%
\pgfpathcurveto{\pgfqpoint{1.232724in}{1.637945in}}{\pgfqpoint{1.240624in}{1.641217in}}{\pgfqpoint{1.246448in}{1.647041in}}%
\pgfpathcurveto{\pgfqpoint{1.252271in}{1.652865in}}{\pgfqpoint{1.255544in}{1.660765in}}{\pgfqpoint{1.255544in}{1.669002in}}%
\pgfpathcurveto{\pgfqpoint{1.255544in}{1.677238in}}{\pgfqpoint{1.252271in}{1.685138in}}{\pgfqpoint{1.246448in}{1.690962in}}%
\pgfpathcurveto{\pgfqpoint{1.240624in}{1.696786in}}{\pgfqpoint{1.232724in}{1.700058in}}{\pgfqpoint{1.224487in}{1.700058in}}%
\pgfpathcurveto{\pgfqpoint{1.216251in}{1.700058in}}{\pgfqpoint{1.208351in}{1.696786in}}{\pgfqpoint{1.202527in}{1.690962in}}%
\pgfpathcurveto{\pgfqpoint{1.196703in}{1.685138in}}{\pgfqpoint{1.193431in}{1.677238in}}{\pgfqpoint{1.193431in}{1.669002in}}%
\pgfpathcurveto{\pgfqpoint{1.193431in}{1.660765in}}{\pgfqpoint{1.196703in}{1.652865in}}{\pgfqpoint{1.202527in}{1.647041in}}%
\pgfpathcurveto{\pgfqpoint{1.208351in}{1.641217in}}{\pgfqpoint{1.216251in}{1.637945in}}{\pgfqpoint{1.224487in}{1.637945in}}%
\pgfpathclose%
\pgfusepath{stroke,fill}%
\end{pgfscope}%
\begin{pgfscope}%
\pgfpathrectangle{\pgfqpoint{0.100000in}{0.212622in}}{\pgfqpoint{3.696000in}{3.696000in}}%
\pgfusepath{clip}%
\pgfsetbuttcap%
\pgfsetroundjoin%
\definecolor{currentfill}{rgb}{0.121569,0.466667,0.705882}%
\pgfsetfillcolor{currentfill}%
\pgfsetfillopacity{0.417694}%
\pgfsetlinewidth{1.003750pt}%
\definecolor{currentstroke}{rgb}{0.121569,0.466667,0.705882}%
\pgfsetstrokecolor{currentstroke}%
\pgfsetstrokeopacity{0.417694}%
\pgfsetdash{}{0pt}%
\pgfpathmoveto{\pgfqpoint{1.222191in}{1.634826in}}%
\pgfpathcurveto{\pgfqpoint{1.230427in}{1.634826in}}{\pgfqpoint{1.238328in}{1.638099in}}{\pgfqpoint{1.244151in}{1.643922in}}%
\pgfpathcurveto{\pgfqpoint{1.249975in}{1.649746in}}{\pgfqpoint{1.253248in}{1.657646in}}{\pgfqpoint{1.253248in}{1.665883in}}%
\pgfpathcurveto{\pgfqpoint{1.253248in}{1.674119in}}{\pgfqpoint{1.249975in}{1.682019in}}{\pgfqpoint{1.244151in}{1.687843in}}%
\pgfpathcurveto{\pgfqpoint{1.238328in}{1.693667in}}{\pgfqpoint{1.230427in}{1.696939in}}{\pgfqpoint{1.222191in}{1.696939in}}%
\pgfpathcurveto{\pgfqpoint{1.213955in}{1.696939in}}{\pgfqpoint{1.206055in}{1.693667in}}{\pgfqpoint{1.200231in}{1.687843in}}%
\pgfpathcurveto{\pgfqpoint{1.194407in}{1.682019in}}{\pgfqpoint{1.191135in}{1.674119in}}{\pgfqpoint{1.191135in}{1.665883in}}%
\pgfpathcurveto{\pgfqpoint{1.191135in}{1.657646in}}{\pgfqpoint{1.194407in}{1.649746in}}{\pgfqpoint{1.200231in}{1.643922in}}%
\pgfpathcurveto{\pgfqpoint{1.206055in}{1.638099in}}{\pgfqpoint{1.213955in}{1.634826in}}{\pgfqpoint{1.222191in}{1.634826in}}%
\pgfpathclose%
\pgfusepath{stroke,fill}%
\end{pgfscope}%
\begin{pgfscope}%
\pgfpathrectangle{\pgfqpoint{0.100000in}{0.212622in}}{\pgfqpoint{3.696000in}{3.696000in}}%
\pgfusepath{clip}%
\pgfsetbuttcap%
\pgfsetroundjoin%
\definecolor{currentfill}{rgb}{0.121569,0.466667,0.705882}%
\pgfsetfillcolor{currentfill}%
\pgfsetfillopacity{0.419264}%
\pgfsetlinewidth{1.003750pt}%
\definecolor{currentstroke}{rgb}{0.121569,0.466667,0.705882}%
\pgfsetstrokecolor{currentstroke}%
\pgfsetstrokeopacity{0.419264}%
\pgfsetdash{}{0pt}%
\pgfpathmoveto{\pgfqpoint{1.529751in}{1.722607in}}%
\pgfpathcurveto{\pgfqpoint{1.537987in}{1.722607in}}{\pgfqpoint{1.545887in}{1.725879in}}{\pgfqpoint{1.551711in}{1.731703in}}%
\pgfpathcurveto{\pgfqpoint{1.557535in}{1.737527in}}{\pgfqpoint{1.560808in}{1.745427in}}{\pgfqpoint{1.560808in}{1.753663in}}%
\pgfpathcurveto{\pgfqpoint{1.560808in}{1.761899in}}{\pgfqpoint{1.557535in}{1.769799in}}{\pgfqpoint{1.551711in}{1.775623in}}%
\pgfpathcurveto{\pgfqpoint{1.545887in}{1.781447in}}{\pgfqpoint{1.537987in}{1.784720in}}{\pgfqpoint{1.529751in}{1.784720in}}%
\pgfpathcurveto{\pgfqpoint{1.521515in}{1.784720in}}{\pgfqpoint{1.513615in}{1.781447in}}{\pgfqpoint{1.507791in}{1.775623in}}%
\pgfpathcurveto{\pgfqpoint{1.501967in}{1.769799in}}{\pgfqpoint{1.498695in}{1.761899in}}{\pgfqpoint{1.498695in}{1.753663in}}%
\pgfpathcurveto{\pgfqpoint{1.498695in}{1.745427in}}{\pgfqpoint{1.501967in}{1.737527in}}{\pgfqpoint{1.507791in}{1.731703in}}%
\pgfpathcurveto{\pgfqpoint{1.513615in}{1.725879in}}{\pgfqpoint{1.521515in}{1.722607in}}{\pgfqpoint{1.529751in}{1.722607in}}%
\pgfpathclose%
\pgfusepath{stroke,fill}%
\end{pgfscope}%
\begin{pgfscope}%
\pgfpathrectangle{\pgfqpoint{0.100000in}{0.212622in}}{\pgfqpoint{3.696000in}{3.696000in}}%
\pgfusepath{clip}%
\pgfsetbuttcap%
\pgfsetroundjoin%
\definecolor{currentfill}{rgb}{0.121569,0.466667,0.705882}%
\pgfsetfillcolor{currentfill}%
\pgfsetfillopacity{0.420083}%
\pgfsetlinewidth{1.003750pt}%
\definecolor{currentstroke}{rgb}{0.121569,0.466667,0.705882}%
\pgfsetstrokecolor{currentstroke}%
\pgfsetstrokeopacity{0.420083}%
\pgfsetdash{}{0pt}%
\pgfpathmoveto{\pgfqpoint{1.219135in}{1.636039in}}%
\pgfpathcurveto{\pgfqpoint{1.227371in}{1.636039in}}{\pgfqpoint{1.235271in}{1.639311in}}{\pgfqpoint{1.241095in}{1.645135in}}%
\pgfpathcurveto{\pgfqpoint{1.246919in}{1.650959in}}{\pgfqpoint{1.250192in}{1.658859in}}{\pgfqpoint{1.250192in}{1.667095in}}%
\pgfpathcurveto{\pgfqpoint{1.250192in}{1.675332in}}{\pgfqpoint{1.246919in}{1.683232in}}{\pgfqpoint{1.241095in}{1.689056in}}%
\pgfpathcurveto{\pgfqpoint{1.235271in}{1.694880in}}{\pgfqpoint{1.227371in}{1.698152in}}{\pgfqpoint{1.219135in}{1.698152in}}%
\pgfpathcurveto{\pgfqpoint{1.210899in}{1.698152in}}{\pgfqpoint{1.202999in}{1.694880in}}{\pgfqpoint{1.197175in}{1.689056in}}%
\pgfpathcurveto{\pgfqpoint{1.191351in}{1.683232in}}{\pgfqpoint{1.188079in}{1.675332in}}{\pgfqpoint{1.188079in}{1.667095in}}%
\pgfpathcurveto{\pgfqpoint{1.188079in}{1.658859in}}{\pgfqpoint{1.191351in}{1.650959in}}{\pgfqpoint{1.197175in}{1.645135in}}%
\pgfpathcurveto{\pgfqpoint{1.202999in}{1.639311in}}{\pgfqpoint{1.210899in}{1.636039in}}{\pgfqpoint{1.219135in}{1.636039in}}%
\pgfpathclose%
\pgfusepath{stroke,fill}%
\end{pgfscope}%
\begin{pgfscope}%
\pgfpathrectangle{\pgfqpoint{0.100000in}{0.212622in}}{\pgfqpoint{3.696000in}{3.696000in}}%
\pgfusepath{clip}%
\pgfsetbuttcap%
\pgfsetroundjoin%
\definecolor{currentfill}{rgb}{0.121569,0.466667,0.705882}%
\pgfsetfillcolor{currentfill}%
\pgfsetfillopacity{0.420736}%
\pgfsetlinewidth{1.003750pt}%
\definecolor{currentstroke}{rgb}{0.121569,0.466667,0.705882}%
\pgfsetstrokecolor{currentstroke}%
\pgfsetstrokeopacity{0.420736}%
\pgfsetdash{}{0pt}%
\pgfpathmoveto{\pgfqpoint{1.217134in}{1.634909in}}%
\pgfpathcurveto{\pgfqpoint{1.225370in}{1.634909in}}{\pgfqpoint{1.233270in}{1.638181in}}{\pgfqpoint{1.239094in}{1.644005in}}%
\pgfpathcurveto{\pgfqpoint{1.244918in}{1.649829in}}{\pgfqpoint{1.248190in}{1.657729in}}{\pgfqpoint{1.248190in}{1.665966in}}%
\pgfpathcurveto{\pgfqpoint{1.248190in}{1.674202in}}{\pgfqpoint{1.244918in}{1.682102in}}{\pgfqpoint{1.239094in}{1.687926in}}%
\pgfpathcurveto{\pgfqpoint{1.233270in}{1.693750in}}{\pgfqpoint{1.225370in}{1.697022in}}{\pgfqpoint{1.217134in}{1.697022in}}%
\pgfpathcurveto{\pgfqpoint{1.208898in}{1.697022in}}{\pgfqpoint{1.200997in}{1.693750in}}{\pgfqpoint{1.195174in}{1.687926in}}%
\pgfpathcurveto{\pgfqpoint{1.189350in}{1.682102in}}{\pgfqpoint{1.186077in}{1.674202in}}{\pgfqpoint{1.186077in}{1.665966in}}%
\pgfpathcurveto{\pgfqpoint{1.186077in}{1.657729in}}{\pgfqpoint{1.189350in}{1.649829in}}{\pgfqpoint{1.195174in}{1.644005in}}%
\pgfpathcurveto{\pgfqpoint{1.200997in}{1.638181in}}{\pgfqpoint{1.208898in}{1.634909in}}{\pgfqpoint{1.217134in}{1.634909in}}%
\pgfpathclose%
\pgfusepath{stroke,fill}%
\end{pgfscope}%
\begin{pgfscope}%
\pgfpathrectangle{\pgfqpoint{0.100000in}{0.212622in}}{\pgfqpoint{3.696000in}{3.696000in}}%
\pgfusepath{clip}%
\pgfsetbuttcap%
\pgfsetroundjoin%
\definecolor{currentfill}{rgb}{0.121569,0.466667,0.705882}%
\pgfsetfillcolor{currentfill}%
\pgfsetfillopacity{0.421551}%
\pgfsetlinewidth{1.003750pt}%
\definecolor{currentstroke}{rgb}{0.121569,0.466667,0.705882}%
\pgfsetstrokecolor{currentstroke}%
\pgfsetstrokeopacity{0.421551}%
\pgfsetdash{}{0pt}%
\pgfpathmoveto{\pgfqpoint{1.215469in}{1.634889in}}%
\pgfpathcurveto{\pgfqpoint{1.223706in}{1.634889in}}{\pgfqpoint{1.231606in}{1.638161in}}{\pgfqpoint{1.237430in}{1.643985in}}%
\pgfpathcurveto{\pgfqpoint{1.243254in}{1.649809in}}{\pgfqpoint{1.246526in}{1.657709in}}{\pgfqpoint{1.246526in}{1.665945in}}%
\pgfpathcurveto{\pgfqpoint{1.246526in}{1.674182in}}{\pgfqpoint{1.243254in}{1.682082in}}{\pgfqpoint{1.237430in}{1.687906in}}%
\pgfpathcurveto{\pgfqpoint{1.231606in}{1.693730in}}{\pgfqpoint{1.223706in}{1.697002in}}{\pgfqpoint{1.215469in}{1.697002in}}%
\pgfpathcurveto{\pgfqpoint{1.207233in}{1.697002in}}{\pgfqpoint{1.199333in}{1.693730in}}{\pgfqpoint{1.193509in}{1.687906in}}%
\pgfpathcurveto{\pgfqpoint{1.187685in}{1.682082in}}{\pgfqpoint{1.184413in}{1.674182in}}{\pgfqpoint{1.184413in}{1.665945in}}%
\pgfpathcurveto{\pgfqpoint{1.184413in}{1.657709in}}{\pgfqpoint{1.187685in}{1.649809in}}{\pgfqpoint{1.193509in}{1.643985in}}%
\pgfpathcurveto{\pgfqpoint{1.199333in}{1.638161in}}{\pgfqpoint{1.207233in}{1.634889in}}{\pgfqpoint{1.215469in}{1.634889in}}%
\pgfpathclose%
\pgfusepath{stroke,fill}%
\end{pgfscope}%
\begin{pgfscope}%
\pgfpathrectangle{\pgfqpoint{0.100000in}{0.212622in}}{\pgfqpoint{3.696000in}{3.696000in}}%
\pgfusepath{clip}%
\pgfsetbuttcap%
\pgfsetroundjoin%
\definecolor{currentfill}{rgb}{0.121569,0.466667,0.705882}%
\pgfsetfillcolor{currentfill}%
\pgfsetfillopacity{0.423169}%
\pgfsetlinewidth{1.003750pt}%
\definecolor{currentstroke}{rgb}{0.121569,0.466667,0.705882}%
\pgfsetstrokecolor{currentstroke}%
\pgfsetstrokeopacity{0.423169}%
\pgfsetdash{}{0pt}%
\pgfpathmoveto{\pgfqpoint{1.212859in}{1.634855in}}%
\pgfpathcurveto{\pgfqpoint{1.221095in}{1.634855in}}{\pgfqpoint{1.228995in}{1.638127in}}{\pgfqpoint{1.234819in}{1.643951in}}%
\pgfpathcurveto{\pgfqpoint{1.240643in}{1.649775in}}{\pgfqpoint{1.243915in}{1.657675in}}{\pgfqpoint{1.243915in}{1.665911in}}%
\pgfpathcurveto{\pgfqpoint{1.243915in}{1.674147in}}{\pgfqpoint{1.240643in}{1.682048in}}{\pgfqpoint{1.234819in}{1.687871in}}%
\pgfpathcurveto{\pgfqpoint{1.228995in}{1.693695in}}{\pgfqpoint{1.221095in}{1.696968in}}{\pgfqpoint{1.212859in}{1.696968in}}%
\pgfpathcurveto{\pgfqpoint{1.204622in}{1.696968in}}{\pgfqpoint{1.196722in}{1.693695in}}{\pgfqpoint{1.190898in}{1.687871in}}%
\pgfpathcurveto{\pgfqpoint{1.185074in}{1.682048in}}{\pgfqpoint{1.181802in}{1.674147in}}{\pgfqpoint{1.181802in}{1.665911in}}%
\pgfpathcurveto{\pgfqpoint{1.181802in}{1.657675in}}{\pgfqpoint{1.185074in}{1.649775in}}{\pgfqpoint{1.190898in}{1.643951in}}%
\pgfpathcurveto{\pgfqpoint{1.196722in}{1.638127in}}{\pgfqpoint{1.204622in}{1.634855in}}{\pgfqpoint{1.212859in}{1.634855in}}%
\pgfpathclose%
\pgfusepath{stroke,fill}%
\end{pgfscope}%
\begin{pgfscope}%
\pgfpathrectangle{\pgfqpoint{0.100000in}{0.212622in}}{\pgfqpoint{3.696000in}{3.696000in}}%
\pgfusepath{clip}%
\pgfsetbuttcap%
\pgfsetroundjoin%
\definecolor{currentfill}{rgb}{0.121569,0.466667,0.705882}%
\pgfsetfillcolor{currentfill}%
\pgfsetfillopacity{0.423912}%
\pgfsetlinewidth{1.003750pt}%
\definecolor{currentstroke}{rgb}{0.121569,0.466667,0.705882}%
\pgfsetstrokecolor{currentstroke}%
\pgfsetstrokeopacity{0.423912}%
\pgfsetdash{}{0pt}%
\pgfpathmoveto{\pgfqpoint{1.210755in}{1.634084in}}%
\pgfpathcurveto{\pgfqpoint{1.218992in}{1.634084in}}{\pgfqpoint{1.226892in}{1.637356in}}{\pgfqpoint{1.232716in}{1.643180in}}%
\pgfpathcurveto{\pgfqpoint{1.238540in}{1.649004in}}{\pgfqpoint{1.241812in}{1.656904in}}{\pgfqpoint{1.241812in}{1.665140in}}%
\pgfpathcurveto{\pgfqpoint{1.241812in}{1.673377in}}{\pgfqpoint{1.238540in}{1.681277in}}{\pgfqpoint{1.232716in}{1.687101in}}%
\pgfpathcurveto{\pgfqpoint{1.226892in}{1.692925in}}{\pgfqpoint{1.218992in}{1.696197in}}{\pgfqpoint{1.210755in}{1.696197in}}%
\pgfpathcurveto{\pgfqpoint{1.202519in}{1.696197in}}{\pgfqpoint{1.194619in}{1.692925in}}{\pgfqpoint{1.188795in}{1.687101in}}%
\pgfpathcurveto{\pgfqpoint{1.182971in}{1.681277in}}{\pgfqpoint{1.179699in}{1.673377in}}{\pgfqpoint{1.179699in}{1.665140in}}%
\pgfpathcurveto{\pgfqpoint{1.179699in}{1.656904in}}{\pgfqpoint{1.182971in}{1.649004in}}{\pgfqpoint{1.188795in}{1.643180in}}%
\pgfpathcurveto{\pgfqpoint{1.194619in}{1.637356in}}{\pgfqpoint{1.202519in}{1.634084in}}{\pgfqpoint{1.210755in}{1.634084in}}%
\pgfpathclose%
\pgfusepath{stroke,fill}%
\end{pgfscope}%
\begin{pgfscope}%
\pgfpathrectangle{\pgfqpoint{0.100000in}{0.212622in}}{\pgfqpoint{3.696000in}{3.696000in}}%
\pgfusepath{clip}%
\pgfsetbuttcap%
\pgfsetroundjoin%
\definecolor{currentfill}{rgb}{0.121569,0.466667,0.705882}%
\pgfsetfillcolor{currentfill}%
\pgfsetfillopacity{0.424787}%
\pgfsetlinewidth{1.003750pt}%
\definecolor{currentstroke}{rgb}{0.121569,0.466667,0.705882}%
\pgfsetstrokecolor{currentstroke}%
\pgfsetstrokeopacity{0.424787}%
\pgfsetdash{}{0pt}%
\pgfpathmoveto{\pgfqpoint{1.209344in}{1.634373in}}%
\pgfpathcurveto{\pgfqpoint{1.217580in}{1.634373in}}{\pgfqpoint{1.225480in}{1.637645in}}{\pgfqpoint{1.231304in}{1.643469in}}%
\pgfpathcurveto{\pgfqpoint{1.237128in}{1.649293in}}{\pgfqpoint{1.240401in}{1.657193in}}{\pgfqpoint{1.240401in}{1.665429in}}%
\pgfpathcurveto{\pgfqpoint{1.240401in}{1.673665in}}{\pgfqpoint{1.237128in}{1.681566in}}{\pgfqpoint{1.231304in}{1.687389in}}%
\pgfpathcurveto{\pgfqpoint{1.225480in}{1.693213in}}{\pgfqpoint{1.217580in}{1.696486in}}{\pgfqpoint{1.209344in}{1.696486in}}%
\pgfpathcurveto{\pgfqpoint{1.201108in}{1.696486in}}{\pgfqpoint{1.193208in}{1.693213in}}{\pgfqpoint{1.187384in}{1.687389in}}%
\pgfpathcurveto{\pgfqpoint{1.181560in}{1.681566in}}{\pgfqpoint{1.178288in}{1.673665in}}{\pgfqpoint{1.178288in}{1.665429in}}%
\pgfpathcurveto{\pgfqpoint{1.178288in}{1.657193in}}{\pgfqpoint{1.181560in}{1.649293in}}{\pgfqpoint{1.187384in}{1.643469in}}%
\pgfpathcurveto{\pgfqpoint{1.193208in}{1.637645in}}{\pgfqpoint{1.201108in}{1.634373in}}{\pgfqpoint{1.209344in}{1.634373in}}%
\pgfpathclose%
\pgfusepath{stroke,fill}%
\end{pgfscope}%
\begin{pgfscope}%
\pgfpathrectangle{\pgfqpoint{0.100000in}{0.212622in}}{\pgfqpoint{3.696000in}{3.696000in}}%
\pgfusepath{clip}%
\pgfsetbuttcap%
\pgfsetroundjoin%
\definecolor{currentfill}{rgb}{0.121569,0.466667,0.705882}%
\pgfsetfillcolor{currentfill}%
\pgfsetfillopacity{0.425371}%
\pgfsetlinewidth{1.003750pt}%
\definecolor{currentstroke}{rgb}{0.121569,0.466667,0.705882}%
\pgfsetstrokecolor{currentstroke}%
\pgfsetstrokeopacity{0.425371}%
\pgfsetdash{}{0pt}%
\pgfpathmoveto{\pgfqpoint{1.533074in}{1.723607in}}%
\pgfpathcurveto{\pgfqpoint{1.541310in}{1.723607in}}{\pgfqpoint{1.549210in}{1.726879in}}{\pgfqpoint{1.555034in}{1.732703in}}%
\pgfpathcurveto{\pgfqpoint{1.560858in}{1.738527in}}{\pgfqpoint{1.564131in}{1.746427in}}{\pgfqpoint{1.564131in}{1.754664in}}%
\pgfpathcurveto{\pgfqpoint{1.564131in}{1.762900in}}{\pgfqpoint{1.560858in}{1.770800in}}{\pgfqpoint{1.555034in}{1.776624in}}%
\pgfpathcurveto{\pgfqpoint{1.549210in}{1.782448in}}{\pgfqpoint{1.541310in}{1.785720in}}{\pgfqpoint{1.533074in}{1.785720in}}%
\pgfpathcurveto{\pgfqpoint{1.524838in}{1.785720in}}{\pgfqpoint{1.516938in}{1.782448in}}{\pgfqpoint{1.511114in}{1.776624in}}%
\pgfpathcurveto{\pgfqpoint{1.505290in}{1.770800in}}{\pgfqpoint{1.502018in}{1.762900in}}{\pgfqpoint{1.502018in}{1.754664in}}%
\pgfpathcurveto{\pgfqpoint{1.502018in}{1.746427in}}{\pgfqpoint{1.505290in}{1.738527in}}{\pgfqpoint{1.511114in}{1.732703in}}%
\pgfpathcurveto{\pgfqpoint{1.516938in}{1.726879in}}{\pgfqpoint{1.524838in}{1.723607in}}{\pgfqpoint{1.533074in}{1.723607in}}%
\pgfpathclose%
\pgfusepath{stroke,fill}%
\end{pgfscope}%
\begin{pgfscope}%
\pgfpathrectangle{\pgfqpoint{0.100000in}{0.212622in}}{\pgfqpoint{3.696000in}{3.696000in}}%
\pgfusepath{clip}%
\pgfsetbuttcap%
\pgfsetroundjoin%
\definecolor{currentfill}{rgb}{0.121569,0.466667,0.705882}%
\pgfsetfillcolor{currentfill}%
\pgfsetfillopacity{0.426162}%
\pgfsetlinewidth{1.003750pt}%
\definecolor{currentstroke}{rgb}{0.121569,0.466667,0.705882}%
\pgfsetstrokecolor{currentstroke}%
\pgfsetstrokeopacity{0.426162}%
\pgfsetdash{}{0pt}%
\pgfpathmoveto{\pgfqpoint{1.207041in}{1.633571in}}%
\pgfpathcurveto{\pgfqpoint{1.215278in}{1.633571in}}{\pgfqpoint{1.223178in}{1.636844in}}{\pgfqpoint{1.229001in}{1.642668in}}%
\pgfpathcurveto{\pgfqpoint{1.234825in}{1.648491in}}{\pgfqpoint{1.238098in}{1.656392in}}{\pgfqpoint{1.238098in}{1.664628in}}%
\pgfpathcurveto{\pgfqpoint{1.238098in}{1.672864in}}{\pgfqpoint{1.234825in}{1.680764in}}{\pgfqpoint{1.229001in}{1.686588in}}%
\pgfpathcurveto{\pgfqpoint{1.223178in}{1.692412in}}{\pgfqpoint{1.215278in}{1.695684in}}{\pgfqpoint{1.207041in}{1.695684in}}%
\pgfpathcurveto{\pgfqpoint{1.198805in}{1.695684in}}{\pgfqpoint{1.190905in}{1.692412in}}{\pgfqpoint{1.185081in}{1.686588in}}%
\pgfpathcurveto{\pgfqpoint{1.179257in}{1.680764in}}{\pgfqpoint{1.175985in}{1.672864in}}{\pgfqpoint{1.175985in}{1.664628in}}%
\pgfpathcurveto{\pgfqpoint{1.175985in}{1.656392in}}{\pgfqpoint{1.179257in}{1.648491in}}{\pgfqpoint{1.185081in}{1.642668in}}%
\pgfpathcurveto{\pgfqpoint{1.190905in}{1.636844in}}{\pgfqpoint{1.198805in}{1.633571in}}{\pgfqpoint{1.207041in}{1.633571in}}%
\pgfpathclose%
\pgfusepath{stroke,fill}%
\end{pgfscope}%
\begin{pgfscope}%
\pgfpathrectangle{\pgfqpoint{0.100000in}{0.212622in}}{\pgfqpoint{3.696000in}{3.696000in}}%
\pgfusepath{clip}%
\pgfsetbuttcap%
\pgfsetroundjoin%
\definecolor{currentfill}{rgb}{0.121569,0.466667,0.705882}%
\pgfsetfillcolor{currentfill}%
\pgfsetfillopacity{0.426625}%
\pgfsetlinewidth{1.003750pt}%
\definecolor{currentstroke}{rgb}{0.121569,0.466667,0.705882}%
\pgfsetstrokecolor{currentstroke}%
\pgfsetstrokeopacity{0.426625}%
\pgfsetdash{}{0pt}%
\pgfpathmoveto{\pgfqpoint{1.205539in}{1.632693in}}%
\pgfpathcurveto{\pgfqpoint{1.213775in}{1.632693in}}{\pgfqpoint{1.221675in}{1.635965in}}{\pgfqpoint{1.227499in}{1.641789in}}%
\pgfpathcurveto{\pgfqpoint{1.233323in}{1.647613in}}{\pgfqpoint{1.236595in}{1.655513in}}{\pgfqpoint{1.236595in}{1.663750in}}%
\pgfpathcurveto{\pgfqpoint{1.236595in}{1.671986in}}{\pgfqpoint{1.233323in}{1.679886in}}{\pgfqpoint{1.227499in}{1.685710in}}%
\pgfpathcurveto{\pgfqpoint{1.221675in}{1.691534in}}{\pgfqpoint{1.213775in}{1.694806in}}{\pgfqpoint{1.205539in}{1.694806in}}%
\pgfpathcurveto{\pgfqpoint{1.197302in}{1.694806in}}{\pgfqpoint{1.189402in}{1.691534in}}{\pgfqpoint{1.183578in}{1.685710in}}%
\pgfpathcurveto{\pgfqpoint{1.177754in}{1.679886in}}{\pgfqpoint{1.174482in}{1.671986in}}{\pgfqpoint{1.174482in}{1.663750in}}%
\pgfpathcurveto{\pgfqpoint{1.174482in}{1.655513in}}{\pgfqpoint{1.177754in}{1.647613in}}{\pgfqpoint{1.183578in}{1.641789in}}%
\pgfpathcurveto{\pgfqpoint{1.189402in}{1.635965in}}{\pgfqpoint{1.197302in}{1.632693in}}{\pgfqpoint{1.205539in}{1.632693in}}%
\pgfpathclose%
\pgfusepath{stroke,fill}%
\end{pgfscope}%
\begin{pgfscope}%
\pgfpathrectangle{\pgfqpoint{0.100000in}{0.212622in}}{\pgfqpoint{3.696000in}{3.696000in}}%
\pgfusepath{clip}%
\pgfsetbuttcap%
\pgfsetroundjoin%
\definecolor{currentfill}{rgb}{0.121569,0.466667,0.705882}%
\pgfsetfillcolor{currentfill}%
\pgfsetfillopacity{0.427192}%
\pgfsetlinewidth{1.003750pt}%
\definecolor{currentstroke}{rgb}{0.121569,0.466667,0.705882}%
\pgfsetstrokecolor{currentstroke}%
\pgfsetstrokeopacity{0.427192}%
\pgfsetdash{}{0pt}%
\pgfpathmoveto{\pgfqpoint{1.204533in}{1.632786in}}%
\pgfpathcurveto{\pgfqpoint{1.212769in}{1.632786in}}{\pgfqpoint{1.220669in}{1.636058in}}{\pgfqpoint{1.226493in}{1.641882in}}%
\pgfpathcurveto{\pgfqpoint{1.232317in}{1.647706in}}{\pgfqpoint{1.235589in}{1.655606in}}{\pgfqpoint{1.235589in}{1.663843in}}%
\pgfpathcurveto{\pgfqpoint{1.235589in}{1.672079in}}{\pgfqpoint{1.232317in}{1.679979in}}{\pgfqpoint{1.226493in}{1.685803in}}%
\pgfpathcurveto{\pgfqpoint{1.220669in}{1.691627in}}{\pgfqpoint{1.212769in}{1.694899in}}{\pgfqpoint{1.204533in}{1.694899in}}%
\pgfpathcurveto{\pgfqpoint{1.196297in}{1.694899in}}{\pgfqpoint{1.188397in}{1.691627in}}{\pgfqpoint{1.182573in}{1.685803in}}%
\pgfpathcurveto{\pgfqpoint{1.176749in}{1.679979in}}{\pgfqpoint{1.173476in}{1.672079in}}{\pgfqpoint{1.173476in}{1.663843in}}%
\pgfpathcurveto{\pgfqpoint{1.173476in}{1.655606in}}{\pgfqpoint{1.176749in}{1.647706in}}{\pgfqpoint{1.182573in}{1.641882in}}%
\pgfpathcurveto{\pgfqpoint{1.188397in}{1.636058in}}{\pgfqpoint{1.196297in}{1.632786in}}{\pgfqpoint{1.204533in}{1.632786in}}%
\pgfpathclose%
\pgfusepath{stroke,fill}%
\end{pgfscope}%
\begin{pgfscope}%
\pgfpathrectangle{\pgfqpoint{0.100000in}{0.212622in}}{\pgfqpoint{3.696000in}{3.696000in}}%
\pgfusepath{clip}%
\pgfsetbuttcap%
\pgfsetroundjoin%
\definecolor{currentfill}{rgb}{0.121569,0.466667,0.705882}%
\pgfsetfillcolor{currentfill}%
\pgfsetfillopacity{0.428345}%
\pgfsetlinewidth{1.003750pt}%
\definecolor{currentstroke}{rgb}{0.121569,0.466667,0.705882}%
\pgfsetstrokecolor{currentstroke}%
\pgfsetstrokeopacity{0.428345}%
\pgfsetdash{}{0pt}%
\pgfpathmoveto{\pgfqpoint{1.202950in}{1.633177in}}%
\pgfpathcurveto{\pgfqpoint{1.211186in}{1.633177in}}{\pgfqpoint{1.219086in}{1.636449in}}{\pgfqpoint{1.224910in}{1.642273in}}%
\pgfpathcurveto{\pgfqpoint{1.230734in}{1.648097in}}{\pgfqpoint{1.234006in}{1.655997in}}{\pgfqpoint{1.234006in}{1.664234in}}%
\pgfpathcurveto{\pgfqpoint{1.234006in}{1.672470in}}{\pgfqpoint{1.230734in}{1.680370in}}{\pgfqpoint{1.224910in}{1.686194in}}%
\pgfpathcurveto{\pgfqpoint{1.219086in}{1.692018in}}{\pgfqpoint{1.211186in}{1.695290in}}{\pgfqpoint{1.202950in}{1.695290in}}%
\pgfpathcurveto{\pgfqpoint{1.194713in}{1.695290in}}{\pgfqpoint{1.186813in}{1.692018in}}{\pgfqpoint{1.180989in}{1.686194in}}%
\pgfpathcurveto{\pgfqpoint{1.175166in}{1.680370in}}{\pgfqpoint{1.171893in}{1.672470in}}{\pgfqpoint{1.171893in}{1.664234in}}%
\pgfpathcurveto{\pgfqpoint{1.171893in}{1.655997in}}{\pgfqpoint{1.175166in}{1.648097in}}{\pgfqpoint{1.180989in}{1.642273in}}%
\pgfpathcurveto{\pgfqpoint{1.186813in}{1.636449in}}{\pgfqpoint{1.194713in}{1.633177in}}{\pgfqpoint{1.202950in}{1.633177in}}%
\pgfpathclose%
\pgfusepath{stroke,fill}%
\end{pgfscope}%
\begin{pgfscope}%
\pgfpathrectangle{\pgfqpoint{0.100000in}{0.212622in}}{\pgfqpoint{3.696000in}{3.696000in}}%
\pgfusepath{clip}%
\pgfsetbuttcap%
\pgfsetroundjoin%
\definecolor{currentfill}{rgb}{0.121569,0.466667,0.705882}%
\pgfsetfillcolor{currentfill}%
\pgfsetfillopacity{0.428359}%
\pgfsetlinewidth{1.003750pt}%
\definecolor{currentstroke}{rgb}{0.121569,0.466667,0.705882}%
\pgfsetstrokecolor{currentstroke}%
\pgfsetstrokeopacity{0.428359}%
\pgfsetdash{}{0pt}%
\pgfpathmoveto{\pgfqpoint{1.533500in}{1.721888in}}%
\pgfpathcurveto{\pgfqpoint{1.541737in}{1.721888in}}{\pgfqpoint{1.549637in}{1.725161in}}{\pgfqpoint{1.555461in}{1.730985in}}%
\pgfpathcurveto{\pgfqpoint{1.561285in}{1.736809in}}{\pgfqpoint{1.564557in}{1.744709in}}{\pgfqpoint{1.564557in}{1.752945in}}%
\pgfpathcurveto{\pgfqpoint{1.564557in}{1.761181in}}{\pgfqpoint{1.561285in}{1.769081in}}{\pgfqpoint{1.555461in}{1.774905in}}%
\pgfpathcurveto{\pgfqpoint{1.549637in}{1.780729in}}{\pgfqpoint{1.541737in}{1.784001in}}{\pgfqpoint{1.533500in}{1.784001in}}%
\pgfpathcurveto{\pgfqpoint{1.525264in}{1.784001in}}{\pgfqpoint{1.517364in}{1.780729in}}{\pgfqpoint{1.511540in}{1.774905in}}%
\pgfpathcurveto{\pgfqpoint{1.505716in}{1.769081in}}{\pgfqpoint{1.502444in}{1.761181in}}{\pgfqpoint{1.502444in}{1.752945in}}%
\pgfpathcurveto{\pgfqpoint{1.502444in}{1.744709in}}{\pgfqpoint{1.505716in}{1.736809in}}{\pgfqpoint{1.511540in}{1.730985in}}%
\pgfpathcurveto{\pgfqpoint{1.517364in}{1.725161in}}{\pgfqpoint{1.525264in}{1.721888in}}{\pgfqpoint{1.533500in}{1.721888in}}%
\pgfpathclose%
\pgfusepath{stroke,fill}%
\end{pgfscope}%
\begin{pgfscope}%
\pgfpathrectangle{\pgfqpoint{0.100000in}{0.212622in}}{\pgfqpoint{3.696000in}{3.696000in}}%
\pgfusepath{clip}%
\pgfsetbuttcap%
\pgfsetroundjoin%
\definecolor{currentfill}{rgb}{0.121569,0.466667,0.705882}%
\pgfsetfillcolor{currentfill}%
\pgfsetfillopacity{0.428578}%
\pgfsetlinewidth{1.003750pt}%
\definecolor{currentstroke}{rgb}{0.121569,0.466667,0.705882}%
\pgfsetstrokecolor{currentstroke}%
\pgfsetstrokeopacity{0.428578}%
\pgfsetdash{}{0pt}%
\pgfpathmoveto{\pgfqpoint{1.202190in}{1.632684in}}%
\pgfpathcurveto{\pgfqpoint{1.210427in}{1.632684in}}{\pgfqpoint{1.218327in}{1.635956in}}{\pgfqpoint{1.224151in}{1.641780in}}%
\pgfpathcurveto{\pgfqpoint{1.229974in}{1.647604in}}{\pgfqpoint{1.233247in}{1.655504in}}{\pgfqpoint{1.233247in}{1.663740in}}%
\pgfpathcurveto{\pgfqpoint{1.233247in}{1.671977in}}{\pgfqpoint{1.229974in}{1.679877in}}{\pgfqpoint{1.224151in}{1.685701in}}%
\pgfpathcurveto{\pgfqpoint{1.218327in}{1.691524in}}{\pgfqpoint{1.210427in}{1.694797in}}{\pgfqpoint{1.202190in}{1.694797in}}%
\pgfpathcurveto{\pgfqpoint{1.193954in}{1.694797in}}{\pgfqpoint{1.186054in}{1.691524in}}{\pgfqpoint{1.180230in}{1.685701in}}%
\pgfpathcurveto{\pgfqpoint{1.174406in}{1.679877in}}{\pgfqpoint{1.171134in}{1.671977in}}{\pgfqpoint{1.171134in}{1.663740in}}%
\pgfpathcurveto{\pgfqpoint{1.171134in}{1.655504in}}{\pgfqpoint{1.174406in}{1.647604in}}{\pgfqpoint{1.180230in}{1.641780in}}%
\pgfpathcurveto{\pgfqpoint{1.186054in}{1.635956in}}{\pgfqpoint{1.193954in}{1.632684in}}{\pgfqpoint{1.202190in}{1.632684in}}%
\pgfpathclose%
\pgfusepath{stroke,fill}%
\end{pgfscope}%
\begin{pgfscope}%
\pgfpathrectangle{\pgfqpoint{0.100000in}{0.212622in}}{\pgfqpoint{3.696000in}{3.696000in}}%
\pgfusepath{clip}%
\pgfsetbuttcap%
\pgfsetroundjoin%
\definecolor{currentfill}{rgb}{0.121569,0.466667,0.705882}%
\pgfsetfillcolor{currentfill}%
\pgfsetfillopacity{0.429215}%
\pgfsetlinewidth{1.003750pt}%
\definecolor{currentstroke}{rgb}{0.121569,0.466667,0.705882}%
\pgfsetstrokecolor{currentstroke}%
\pgfsetstrokeopacity{0.429215}%
\pgfsetdash{}{0pt}%
\pgfpathmoveto{\pgfqpoint{1.200942in}{1.632520in}}%
\pgfpathcurveto{\pgfqpoint{1.209179in}{1.632520in}}{\pgfqpoint{1.217079in}{1.635792in}}{\pgfqpoint{1.222903in}{1.641616in}}%
\pgfpathcurveto{\pgfqpoint{1.228726in}{1.647440in}}{\pgfqpoint{1.231999in}{1.655340in}}{\pgfqpoint{1.231999in}{1.663577in}}%
\pgfpathcurveto{\pgfqpoint{1.231999in}{1.671813in}}{\pgfqpoint{1.228726in}{1.679713in}}{\pgfqpoint{1.222903in}{1.685537in}}%
\pgfpathcurveto{\pgfqpoint{1.217079in}{1.691361in}}{\pgfqpoint{1.209179in}{1.694633in}}{\pgfqpoint{1.200942in}{1.694633in}}%
\pgfpathcurveto{\pgfqpoint{1.192706in}{1.694633in}}{\pgfqpoint{1.184806in}{1.691361in}}{\pgfqpoint{1.178982in}{1.685537in}}%
\pgfpathcurveto{\pgfqpoint{1.173158in}{1.679713in}}{\pgfqpoint{1.169886in}{1.671813in}}{\pgfqpoint{1.169886in}{1.663577in}}%
\pgfpathcurveto{\pgfqpoint{1.169886in}{1.655340in}}{\pgfqpoint{1.173158in}{1.647440in}}{\pgfqpoint{1.178982in}{1.641616in}}%
\pgfpathcurveto{\pgfqpoint{1.184806in}{1.635792in}}{\pgfqpoint{1.192706in}{1.632520in}}{\pgfqpoint{1.200942in}{1.632520in}}%
\pgfpathclose%
\pgfusepath{stroke,fill}%
\end{pgfscope}%
\begin{pgfscope}%
\pgfpathrectangle{\pgfqpoint{0.100000in}{0.212622in}}{\pgfqpoint{3.696000in}{3.696000in}}%
\pgfusepath{clip}%
\pgfsetbuttcap%
\pgfsetroundjoin%
\definecolor{currentfill}{rgb}{0.121569,0.466667,0.705882}%
\pgfsetfillcolor{currentfill}%
\pgfsetfillopacity{0.430033}%
\pgfsetlinewidth{1.003750pt}%
\definecolor{currentstroke}{rgb}{0.121569,0.466667,0.705882}%
\pgfsetstrokecolor{currentstroke}%
\pgfsetstrokeopacity{0.430033}%
\pgfsetdash{}{0pt}%
\pgfpathmoveto{\pgfqpoint{1.534411in}{1.721301in}}%
\pgfpathcurveto{\pgfqpoint{1.542647in}{1.721301in}}{\pgfqpoint{1.550547in}{1.724573in}}{\pgfqpoint{1.556371in}{1.730397in}}%
\pgfpathcurveto{\pgfqpoint{1.562195in}{1.736221in}}{\pgfqpoint{1.565467in}{1.744121in}}{\pgfqpoint{1.565467in}{1.752358in}}%
\pgfpathcurveto{\pgfqpoint{1.565467in}{1.760594in}}{\pgfqpoint{1.562195in}{1.768494in}}{\pgfqpoint{1.556371in}{1.774318in}}%
\pgfpathcurveto{\pgfqpoint{1.550547in}{1.780142in}}{\pgfqpoint{1.542647in}{1.783414in}}{\pgfqpoint{1.534411in}{1.783414in}}%
\pgfpathcurveto{\pgfqpoint{1.526174in}{1.783414in}}{\pgfqpoint{1.518274in}{1.780142in}}{\pgfqpoint{1.512450in}{1.774318in}}%
\pgfpathcurveto{\pgfqpoint{1.506626in}{1.768494in}}{\pgfqpoint{1.503354in}{1.760594in}}{\pgfqpoint{1.503354in}{1.752358in}}%
\pgfpathcurveto{\pgfqpoint{1.503354in}{1.744121in}}{\pgfqpoint{1.506626in}{1.736221in}}{\pgfqpoint{1.512450in}{1.730397in}}%
\pgfpathcurveto{\pgfqpoint{1.518274in}{1.724573in}}{\pgfqpoint{1.526174in}{1.721301in}}{\pgfqpoint{1.534411in}{1.721301in}}%
\pgfpathclose%
\pgfusepath{stroke,fill}%
\end{pgfscope}%
\begin{pgfscope}%
\pgfpathrectangle{\pgfqpoint{0.100000in}{0.212622in}}{\pgfqpoint{3.696000in}{3.696000in}}%
\pgfusepath{clip}%
\pgfsetbuttcap%
\pgfsetroundjoin%
\definecolor{currentfill}{rgb}{0.121569,0.466667,0.705882}%
\pgfsetfillcolor{currentfill}%
\pgfsetfillopacity{0.430664}%
\pgfsetlinewidth{1.003750pt}%
\definecolor{currentstroke}{rgb}{0.121569,0.466667,0.705882}%
\pgfsetstrokecolor{currentstroke}%
\pgfsetstrokeopacity{0.430664}%
\pgfsetdash{}{0pt}%
\pgfpathmoveto{\pgfqpoint{1.198948in}{1.633175in}}%
\pgfpathcurveto{\pgfqpoint{1.207185in}{1.633175in}}{\pgfqpoint{1.215085in}{1.636447in}}{\pgfqpoint{1.220909in}{1.642271in}}%
\pgfpathcurveto{\pgfqpoint{1.226733in}{1.648095in}}{\pgfqpoint{1.230005in}{1.655995in}}{\pgfqpoint{1.230005in}{1.664231in}}%
\pgfpathcurveto{\pgfqpoint{1.230005in}{1.672468in}}{\pgfqpoint{1.226733in}{1.680368in}}{\pgfqpoint{1.220909in}{1.686192in}}%
\pgfpathcurveto{\pgfqpoint{1.215085in}{1.692016in}}{\pgfqpoint{1.207185in}{1.695288in}}{\pgfqpoint{1.198948in}{1.695288in}}%
\pgfpathcurveto{\pgfqpoint{1.190712in}{1.695288in}}{\pgfqpoint{1.182812in}{1.692016in}}{\pgfqpoint{1.176988in}{1.686192in}}%
\pgfpathcurveto{\pgfqpoint{1.171164in}{1.680368in}}{\pgfqpoint{1.167892in}{1.672468in}}{\pgfqpoint{1.167892in}{1.664231in}}%
\pgfpathcurveto{\pgfqpoint{1.167892in}{1.655995in}}{\pgfqpoint{1.171164in}{1.648095in}}{\pgfqpoint{1.176988in}{1.642271in}}%
\pgfpathcurveto{\pgfqpoint{1.182812in}{1.636447in}}{\pgfqpoint{1.190712in}{1.633175in}}{\pgfqpoint{1.198948in}{1.633175in}}%
\pgfpathclose%
\pgfusepath{stroke,fill}%
\end{pgfscope}%
\begin{pgfscope}%
\pgfpathrectangle{\pgfqpoint{0.100000in}{0.212622in}}{\pgfqpoint{3.696000in}{3.696000in}}%
\pgfusepath{clip}%
\pgfsetbuttcap%
\pgfsetroundjoin%
\definecolor{currentfill}{rgb}{0.121569,0.466667,0.705882}%
\pgfsetfillcolor{currentfill}%
\pgfsetfillopacity{0.430948}%
\pgfsetlinewidth{1.003750pt}%
\definecolor{currentstroke}{rgb}{0.121569,0.466667,0.705882}%
\pgfsetstrokecolor{currentstroke}%
\pgfsetstrokeopacity{0.430948}%
\pgfsetdash{}{0pt}%
\pgfpathmoveto{\pgfqpoint{1.197990in}{1.632868in}}%
\pgfpathcurveto{\pgfqpoint{1.206226in}{1.632868in}}{\pgfqpoint{1.214126in}{1.636141in}}{\pgfqpoint{1.219950in}{1.641965in}}%
\pgfpathcurveto{\pgfqpoint{1.225774in}{1.647788in}}{\pgfqpoint{1.229046in}{1.655688in}}{\pgfqpoint{1.229046in}{1.663925in}}%
\pgfpathcurveto{\pgfqpoint{1.229046in}{1.672161in}}{\pgfqpoint{1.225774in}{1.680061in}}{\pgfqpoint{1.219950in}{1.685885in}}%
\pgfpathcurveto{\pgfqpoint{1.214126in}{1.691709in}}{\pgfqpoint{1.206226in}{1.694981in}}{\pgfqpoint{1.197990in}{1.694981in}}%
\pgfpathcurveto{\pgfqpoint{1.189754in}{1.694981in}}{\pgfqpoint{1.181854in}{1.691709in}}{\pgfqpoint{1.176030in}{1.685885in}}%
\pgfpathcurveto{\pgfqpoint{1.170206in}{1.680061in}}{\pgfqpoint{1.166933in}{1.672161in}}{\pgfqpoint{1.166933in}{1.663925in}}%
\pgfpathcurveto{\pgfqpoint{1.166933in}{1.655688in}}{\pgfqpoint{1.170206in}{1.647788in}}{\pgfqpoint{1.176030in}{1.641965in}}%
\pgfpathcurveto{\pgfqpoint{1.181854in}{1.636141in}}{\pgfqpoint{1.189754in}{1.632868in}}{\pgfqpoint{1.197990in}{1.632868in}}%
\pgfpathclose%
\pgfusepath{stroke,fill}%
\end{pgfscope}%
\begin{pgfscope}%
\pgfpathrectangle{\pgfqpoint{0.100000in}{0.212622in}}{\pgfqpoint{3.696000in}{3.696000in}}%
\pgfusepath{clip}%
\pgfsetbuttcap%
\pgfsetroundjoin%
\definecolor{currentfill}{rgb}{0.121569,0.466667,0.705882}%
\pgfsetfillcolor{currentfill}%
\pgfsetfillopacity{0.431209}%
\pgfsetlinewidth{1.003750pt}%
\definecolor{currentstroke}{rgb}{0.121569,0.466667,0.705882}%
\pgfsetstrokecolor{currentstroke}%
\pgfsetstrokeopacity{0.431209}%
\pgfsetdash{}{0pt}%
\pgfpathmoveto{\pgfqpoint{1.196490in}{1.630694in}}%
\pgfpathcurveto{\pgfqpoint{1.204726in}{1.630694in}}{\pgfqpoint{1.212627in}{1.633966in}}{\pgfqpoint{1.218450in}{1.639790in}}%
\pgfpathcurveto{\pgfqpoint{1.224274in}{1.645614in}}{\pgfqpoint{1.227547in}{1.653514in}}{\pgfqpoint{1.227547in}{1.661750in}}%
\pgfpathcurveto{\pgfqpoint{1.227547in}{1.669987in}}{\pgfqpoint{1.224274in}{1.677887in}}{\pgfqpoint{1.218450in}{1.683711in}}%
\pgfpathcurveto{\pgfqpoint{1.212627in}{1.689534in}}{\pgfqpoint{1.204726in}{1.692807in}}{\pgfqpoint{1.196490in}{1.692807in}}%
\pgfpathcurveto{\pgfqpoint{1.188254in}{1.692807in}}{\pgfqpoint{1.180354in}{1.689534in}}{\pgfqpoint{1.174530in}{1.683711in}}%
\pgfpathcurveto{\pgfqpoint{1.168706in}{1.677887in}}{\pgfqpoint{1.165434in}{1.669987in}}{\pgfqpoint{1.165434in}{1.661750in}}%
\pgfpathcurveto{\pgfqpoint{1.165434in}{1.653514in}}{\pgfqpoint{1.168706in}{1.645614in}}{\pgfqpoint{1.174530in}{1.639790in}}%
\pgfpathcurveto{\pgfqpoint{1.180354in}{1.633966in}}{\pgfqpoint{1.188254in}{1.630694in}}{\pgfqpoint{1.196490in}{1.630694in}}%
\pgfpathclose%
\pgfusepath{stroke,fill}%
\end{pgfscope}%
\begin{pgfscope}%
\pgfpathrectangle{\pgfqpoint{0.100000in}{0.212622in}}{\pgfqpoint{3.696000in}{3.696000in}}%
\pgfusepath{clip}%
\pgfsetbuttcap%
\pgfsetroundjoin%
\definecolor{currentfill}{rgb}{0.121569,0.466667,0.705882}%
\pgfsetfillcolor{currentfill}%
\pgfsetfillopacity{0.432232}%
\pgfsetlinewidth{1.003750pt}%
\definecolor{currentstroke}{rgb}{0.121569,0.466667,0.705882}%
\pgfsetstrokecolor{currentstroke}%
\pgfsetstrokeopacity{0.432232}%
\pgfsetdash{}{0pt}%
\pgfpathmoveto{\pgfqpoint{1.535501in}{1.721533in}}%
\pgfpathcurveto{\pgfqpoint{1.543738in}{1.721533in}}{\pgfqpoint{1.551638in}{1.724806in}}{\pgfqpoint{1.557462in}{1.730630in}}%
\pgfpathcurveto{\pgfqpoint{1.563286in}{1.736453in}}{\pgfqpoint{1.566558in}{1.744353in}}{\pgfqpoint{1.566558in}{1.752590in}}%
\pgfpathcurveto{\pgfqpoint{1.566558in}{1.760826in}}{\pgfqpoint{1.563286in}{1.768726in}}{\pgfqpoint{1.557462in}{1.774550in}}%
\pgfpathcurveto{\pgfqpoint{1.551638in}{1.780374in}}{\pgfqpoint{1.543738in}{1.783646in}}{\pgfqpoint{1.535501in}{1.783646in}}%
\pgfpathcurveto{\pgfqpoint{1.527265in}{1.783646in}}{\pgfqpoint{1.519365in}{1.780374in}}{\pgfqpoint{1.513541in}{1.774550in}}%
\pgfpathcurveto{\pgfqpoint{1.507717in}{1.768726in}}{\pgfqpoint{1.504445in}{1.760826in}}{\pgfqpoint{1.504445in}{1.752590in}}%
\pgfpathcurveto{\pgfqpoint{1.504445in}{1.744353in}}{\pgfqpoint{1.507717in}{1.736453in}}{\pgfqpoint{1.513541in}{1.730630in}}%
\pgfpathcurveto{\pgfqpoint{1.519365in}{1.724806in}}{\pgfqpoint{1.527265in}{1.721533in}}{\pgfqpoint{1.535501in}{1.721533in}}%
\pgfpathclose%
\pgfusepath{stroke,fill}%
\end{pgfscope}%
\begin{pgfscope}%
\pgfpathrectangle{\pgfqpoint{0.100000in}{0.212622in}}{\pgfqpoint{3.696000in}{3.696000in}}%
\pgfusepath{clip}%
\pgfsetbuttcap%
\pgfsetroundjoin%
\definecolor{currentfill}{rgb}{0.121569,0.466667,0.705882}%
\pgfsetfillcolor{currentfill}%
\pgfsetfillopacity{0.432771}%
\pgfsetlinewidth{1.003750pt}%
\definecolor{currentstroke}{rgb}{0.121569,0.466667,0.705882}%
\pgfsetstrokecolor{currentstroke}%
\pgfsetstrokeopacity{0.432771}%
\pgfsetdash{}{0pt}%
\pgfpathmoveto{\pgfqpoint{1.194068in}{1.631184in}}%
\pgfpathcurveto{\pgfqpoint{1.202304in}{1.631184in}}{\pgfqpoint{1.210204in}{1.634456in}}{\pgfqpoint{1.216028in}{1.640280in}}%
\pgfpathcurveto{\pgfqpoint{1.221852in}{1.646104in}}{\pgfqpoint{1.225125in}{1.654004in}}{\pgfqpoint{1.225125in}{1.662240in}}%
\pgfpathcurveto{\pgfqpoint{1.225125in}{1.670476in}}{\pgfqpoint{1.221852in}{1.678376in}}{\pgfqpoint{1.216028in}{1.684200in}}%
\pgfpathcurveto{\pgfqpoint{1.210204in}{1.690024in}}{\pgfqpoint{1.202304in}{1.693297in}}{\pgfqpoint{1.194068in}{1.693297in}}%
\pgfpathcurveto{\pgfqpoint{1.185832in}{1.693297in}}{\pgfqpoint{1.177932in}{1.690024in}}{\pgfqpoint{1.172108in}{1.684200in}}%
\pgfpathcurveto{\pgfqpoint{1.166284in}{1.678376in}}{\pgfqpoint{1.163012in}{1.670476in}}{\pgfqpoint{1.163012in}{1.662240in}}%
\pgfpathcurveto{\pgfqpoint{1.163012in}{1.654004in}}{\pgfqpoint{1.166284in}{1.646104in}}{\pgfqpoint{1.172108in}{1.640280in}}%
\pgfpathcurveto{\pgfqpoint{1.177932in}{1.634456in}}{\pgfqpoint{1.185832in}{1.631184in}}{\pgfqpoint{1.194068in}{1.631184in}}%
\pgfpathclose%
\pgfusepath{stroke,fill}%
\end{pgfscope}%
\begin{pgfscope}%
\pgfpathrectangle{\pgfqpoint{0.100000in}{0.212622in}}{\pgfqpoint{3.696000in}{3.696000in}}%
\pgfusepath{clip}%
\pgfsetbuttcap%
\pgfsetroundjoin%
\definecolor{currentfill}{rgb}{0.121569,0.466667,0.705882}%
\pgfsetfillcolor{currentfill}%
\pgfsetfillopacity{0.433106}%
\pgfsetlinewidth{1.003750pt}%
\definecolor{currentstroke}{rgb}{0.121569,0.466667,0.705882}%
\pgfsetstrokecolor{currentstroke}%
\pgfsetstrokeopacity{0.433106}%
\pgfsetdash{}{0pt}%
\pgfpathmoveto{\pgfqpoint{1.192777in}{1.629834in}}%
\pgfpathcurveto{\pgfqpoint{1.201013in}{1.629834in}}{\pgfqpoint{1.208913in}{1.633106in}}{\pgfqpoint{1.214737in}{1.638930in}}%
\pgfpathcurveto{\pgfqpoint{1.220561in}{1.644754in}}{\pgfqpoint{1.223833in}{1.652654in}}{\pgfqpoint{1.223833in}{1.660890in}}%
\pgfpathcurveto{\pgfqpoint{1.223833in}{1.669127in}}{\pgfqpoint{1.220561in}{1.677027in}}{\pgfqpoint{1.214737in}{1.682851in}}%
\pgfpathcurveto{\pgfqpoint{1.208913in}{1.688674in}}{\pgfqpoint{1.201013in}{1.691947in}}{\pgfqpoint{1.192777in}{1.691947in}}%
\pgfpathcurveto{\pgfqpoint{1.184540in}{1.691947in}}{\pgfqpoint{1.176640in}{1.688674in}}{\pgfqpoint{1.170816in}{1.682851in}}%
\pgfpathcurveto{\pgfqpoint{1.164993in}{1.677027in}}{\pgfqpoint{1.161720in}{1.669127in}}{\pgfqpoint{1.161720in}{1.660890in}}%
\pgfpathcurveto{\pgfqpoint{1.161720in}{1.652654in}}{\pgfqpoint{1.164993in}{1.644754in}}{\pgfqpoint{1.170816in}{1.638930in}}%
\pgfpathcurveto{\pgfqpoint{1.176640in}{1.633106in}}{\pgfqpoint{1.184540in}{1.629834in}}{\pgfqpoint{1.192777in}{1.629834in}}%
\pgfpathclose%
\pgfusepath{stroke,fill}%
\end{pgfscope}%
\begin{pgfscope}%
\pgfpathrectangle{\pgfqpoint{0.100000in}{0.212622in}}{\pgfqpoint{3.696000in}{3.696000in}}%
\pgfusepath{clip}%
\pgfsetbuttcap%
\pgfsetroundjoin%
\definecolor{currentfill}{rgb}{0.121569,0.466667,0.705882}%
\pgfsetfillcolor{currentfill}%
\pgfsetfillopacity{0.434210}%
\pgfsetlinewidth{1.003750pt}%
\definecolor{currentstroke}{rgb}{0.121569,0.466667,0.705882}%
\pgfsetstrokecolor{currentstroke}%
\pgfsetstrokeopacity{0.434210}%
\pgfsetdash{}{0pt}%
\pgfpathmoveto{\pgfqpoint{1.190499in}{1.629491in}}%
\pgfpathcurveto{\pgfqpoint{1.198735in}{1.629491in}}{\pgfqpoint{1.206635in}{1.632763in}}{\pgfqpoint{1.212459in}{1.638587in}}%
\pgfpathcurveto{\pgfqpoint{1.218283in}{1.644411in}}{\pgfqpoint{1.221555in}{1.652311in}}{\pgfqpoint{1.221555in}{1.660547in}}%
\pgfpathcurveto{\pgfqpoint{1.221555in}{1.668783in}}{\pgfqpoint{1.218283in}{1.676683in}}{\pgfqpoint{1.212459in}{1.682507in}}%
\pgfpathcurveto{\pgfqpoint{1.206635in}{1.688331in}}{\pgfqpoint{1.198735in}{1.691604in}}{\pgfqpoint{1.190499in}{1.691604in}}%
\pgfpathcurveto{\pgfqpoint{1.182263in}{1.691604in}}{\pgfqpoint{1.174363in}{1.688331in}}{\pgfqpoint{1.168539in}{1.682507in}}%
\pgfpathcurveto{\pgfqpoint{1.162715in}{1.676683in}}{\pgfqpoint{1.159442in}{1.668783in}}{\pgfqpoint{1.159442in}{1.660547in}}%
\pgfpathcurveto{\pgfqpoint{1.159442in}{1.652311in}}{\pgfqpoint{1.162715in}{1.644411in}}{\pgfqpoint{1.168539in}{1.638587in}}%
\pgfpathcurveto{\pgfqpoint{1.174363in}{1.632763in}}{\pgfqpoint{1.182263in}{1.629491in}}{\pgfqpoint{1.190499in}{1.629491in}}%
\pgfpathclose%
\pgfusepath{stroke,fill}%
\end{pgfscope}%
\begin{pgfscope}%
\pgfpathrectangle{\pgfqpoint{0.100000in}{0.212622in}}{\pgfqpoint{3.696000in}{3.696000in}}%
\pgfusepath{clip}%
\pgfsetbuttcap%
\pgfsetroundjoin%
\definecolor{currentfill}{rgb}{0.121569,0.466667,0.705882}%
\pgfsetfillcolor{currentfill}%
\pgfsetfillopacity{0.434678}%
\pgfsetlinewidth{1.003750pt}%
\definecolor{currentstroke}{rgb}{0.121569,0.466667,0.705882}%
\pgfsetstrokecolor{currentstroke}%
\pgfsetstrokeopacity{0.434678}%
\pgfsetdash{}{0pt}%
\pgfpathmoveto{\pgfqpoint{1.536069in}{1.718752in}}%
\pgfpathcurveto{\pgfqpoint{1.544306in}{1.718752in}}{\pgfqpoint{1.552206in}{1.722025in}}{\pgfqpoint{1.558030in}{1.727849in}}%
\pgfpathcurveto{\pgfqpoint{1.563854in}{1.733672in}}{\pgfqpoint{1.567126in}{1.741572in}}{\pgfqpoint{1.567126in}{1.749809in}}%
\pgfpathcurveto{\pgfqpoint{1.567126in}{1.758045in}}{\pgfqpoint{1.563854in}{1.765945in}}{\pgfqpoint{1.558030in}{1.771769in}}%
\pgfpathcurveto{\pgfqpoint{1.552206in}{1.777593in}}{\pgfqpoint{1.544306in}{1.780865in}}{\pgfqpoint{1.536069in}{1.780865in}}%
\pgfpathcurveto{\pgfqpoint{1.527833in}{1.780865in}}{\pgfqpoint{1.519933in}{1.777593in}}{\pgfqpoint{1.514109in}{1.771769in}}%
\pgfpathcurveto{\pgfqpoint{1.508285in}{1.765945in}}{\pgfqpoint{1.505013in}{1.758045in}}{\pgfqpoint{1.505013in}{1.749809in}}%
\pgfpathcurveto{\pgfqpoint{1.505013in}{1.741572in}}{\pgfqpoint{1.508285in}{1.733672in}}{\pgfqpoint{1.514109in}{1.727849in}}%
\pgfpathcurveto{\pgfqpoint{1.519933in}{1.722025in}}{\pgfqpoint{1.527833in}{1.718752in}}{\pgfqpoint{1.536069in}{1.718752in}}%
\pgfpathclose%
\pgfusepath{stroke,fill}%
\end{pgfscope}%
\begin{pgfscope}%
\pgfpathrectangle{\pgfqpoint{0.100000in}{0.212622in}}{\pgfqpoint{3.696000in}{3.696000in}}%
\pgfusepath{clip}%
\pgfsetbuttcap%
\pgfsetroundjoin%
\definecolor{currentfill}{rgb}{0.121569,0.466667,0.705882}%
\pgfsetfillcolor{currentfill}%
\pgfsetfillopacity{0.436801}%
\pgfsetlinewidth{1.003750pt}%
\definecolor{currentstroke}{rgb}{0.121569,0.466667,0.705882}%
\pgfsetstrokecolor{currentstroke}%
\pgfsetstrokeopacity{0.436801}%
\pgfsetdash{}{0pt}%
\pgfpathmoveto{\pgfqpoint{1.186882in}{1.630776in}}%
\pgfpathcurveto{\pgfqpoint{1.195118in}{1.630776in}}{\pgfqpoint{1.203018in}{1.634049in}}{\pgfqpoint{1.208842in}{1.639873in}}%
\pgfpathcurveto{\pgfqpoint{1.214666in}{1.645697in}}{\pgfqpoint{1.217938in}{1.653597in}}{\pgfqpoint{1.217938in}{1.661833in}}%
\pgfpathcurveto{\pgfqpoint{1.217938in}{1.670069in}}{\pgfqpoint{1.214666in}{1.677969in}}{\pgfqpoint{1.208842in}{1.683793in}}%
\pgfpathcurveto{\pgfqpoint{1.203018in}{1.689617in}}{\pgfqpoint{1.195118in}{1.692889in}}{\pgfqpoint{1.186882in}{1.692889in}}%
\pgfpathcurveto{\pgfqpoint{1.178645in}{1.692889in}}{\pgfqpoint{1.170745in}{1.689617in}}{\pgfqpoint{1.164921in}{1.683793in}}%
\pgfpathcurveto{\pgfqpoint{1.159097in}{1.677969in}}{\pgfqpoint{1.155825in}{1.670069in}}{\pgfqpoint{1.155825in}{1.661833in}}%
\pgfpathcurveto{\pgfqpoint{1.155825in}{1.653597in}}{\pgfqpoint{1.159097in}{1.645697in}}{\pgfqpoint{1.164921in}{1.639873in}}%
\pgfpathcurveto{\pgfqpoint{1.170745in}{1.634049in}}{\pgfqpoint{1.178645in}{1.630776in}}{\pgfqpoint{1.186882in}{1.630776in}}%
\pgfpathclose%
\pgfusepath{stroke,fill}%
\end{pgfscope}%
\begin{pgfscope}%
\pgfpathrectangle{\pgfqpoint{0.100000in}{0.212622in}}{\pgfqpoint{3.696000in}{3.696000in}}%
\pgfusepath{clip}%
\pgfsetbuttcap%
\pgfsetroundjoin%
\definecolor{currentfill}{rgb}{0.121569,0.466667,0.705882}%
\pgfsetfillcolor{currentfill}%
\pgfsetfillopacity{0.437580}%
\pgfsetlinewidth{1.003750pt}%
\definecolor{currentstroke}{rgb}{0.121569,0.466667,0.705882}%
\pgfsetstrokecolor{currentstroke}%
\pgfsetstrokeopacity{0.437580}%
\pgfsetdash{}{0pt}%
\pgfpathmoveto{\pgfqpoint{1.537107in}{1.716944in}}%
\pgfpathcurveto{\pgfqpoint{1.545343in}{1.716944in}}{\pgfqpoint{1.553243in}{1.720216in}}{\pgfqpoint{1.559067in}{1.726040in}}%
\pgfpathcurveto{\pgfqpoint{1.564891in}{1.731864in}}{\pgfqpoint{1.568164in}{1.739764in}}{\pgfqpoint{1.568164in}{1.748001in}}%
\pgfpathcurveto{\pgfqpoint{1.568164in}{1.756237in}}{\pgfqpoint{1.564891in}{1.764137in}}{\pgfqpoint{1.559067in}{1.769961in}}%
\pgfpathcurveto{\pgfqpoint{1.553243in}{1.775785in}}{\pgfqpoint{1.545343in}{1.779057in}}{\pgfqpoint{1.537107in}{1.779057in}}%
\pgfpathcurveto{\pgfqpoint{1.528871in}{1.779057in}}{\pgfqpoint{1.520971in}{1.775785in}}{\pgfqpoint{1.515147in}{1.769961in}}%
\pgfpathcurveto{\pgfqpoint{1.509323in}{1.764137in}}{\pgfqpoint{1.506051in}{1.756237in}}{\pgfqpoint{1.506051in}{1.748001in}}%
\pgfpathcurveto{\pgfqpoint{1.506051in}{1.739764in}}{\pgfqpoint{1.509323in}{1.731864in}}{\pgfqpoint{1.515147in}{1.726040in}}%
\pgfpathcurveto{\pgfqpoint{1.520971in}{1.720216in}}{\pgfqpoint{1.528871in}{1.716944in}}{\pgfqpoint{1.537107in}{1.716944in}}%
\pgfpathclose%
\pgfusepath{stroke,fill}%
\end{pgfscope}%
\begin{pgfscope}%
\pgfpathrectangle{\pgfqpoint{0.100000in}{0.212622in}}{\pgfqpoint{3.696000in}{3.696000in}}%
\pgfusepath{clip}%
\pgfsetbuttcap%
\pgfsetroundjoin%
\definecolor{currentfill}{rgb}{0.121569,0.466667,0.705882}%
\pgfsetfillcolor{currentfill}%
\pgfsetfillopacity{0.437833}%
\pgfsetlinewidth{1.003750pt}%
\definecolor{currentstroke}{rgb}{0.121569,0.466667,0.705882}%
\pgfsetstrokecolor{currentstroke}%
\pgfsetstrokeopacity{0.437833}%
\pgfsetdash{}{0pt}%
\pgfpathmoveto{\pgfqpoint{1.183381in}{1.628736in}}%
\pgfpathcurveto{\pgfqpoint{1.191617in}{1.628736in}}{\pgfqpoint{1.199517in}{1.632008in}}{\pgfqpoint{1.205341in}{1.637832in}}%
\pgfpathcurveto{\pgfqpoint{1.211165in}{1.643656in}}{\pgfqpoint{1.214438in}{1.651556in}}{\pgfqpoint{1.214438in}{1.659792in}}%
\pgfpathcurveto{\pgfqpoint{1.214438in}{1.668029in}}{\pgfqpoint{1.211165in}{1.675929in}}{\pgfqpoint{1.205341in}{1.681753in}}%
\pgfpathcurveto{\pgfqpoint{1.199517in}{1.687576in}}{\pgfqpoint{1.191617in}{1.690849in}}{\pgfqpoint{1.183381in}{1.690849in}}%
\pgfpathcurveto{\pgfqpoint{1.175145in}{1.690849in}}{\pgfqpoint{1.167245in}{1.687576in}}{\pgfqpoint{1.161421in}{1.681753in}}%
\pgfpathcurveto{\pgfqpoint{1.155597in}{1.675929in}}{\pgfqpoint{1.152325in}{1.668029in}}{\pgfqpoint{1.152325in}{1.659792in}}%
\pgfpathcurveto{\pgfqpoint{1.152325in}{1.651556in}}{\pgfqpoint{1.155597in}{1.643656in}}{\pgfqpoint{1.161421in}{1.637832in}}%
\pgfpathcurveto{\pgfqpoint{1.167245in}{1.632008in}}{\pgfqpoint{1.175145in}{1.628736in}}{\pgfqpoint{1.183381in}{1.628736in}}%
\pgfpathclose%
\pgfusepath{stroke,fill}%
\end{pgfscope}%
\begin{pgfscope}%
\pgfpathrectangle{\pgfqpoint{0.100000in}{0.212622in}}{\pgfqpoint{3.696000in}{3.696000in}}%
\pgfusepath{clip}%
\pgfsetbuttcap%
\pgfsetroundjoin%
\definecolor{currentfill}{rgb}{0.121569,0.466667,0.705882}%
\pgfsetfillcolor{currentfill}%
\pgfsetfillopacity{0.438872}%
\pgfsetlinewidth{1.003750pt}%
\definecolor{currentstroke}{rgb}{0.121569,0.466667,0.705882}%
\pgfsetstrokecolor{currentstroke}%
\pgfsetstrokeopacity{0.438872}%
\pgfsetdash{}{0pt}%
\pgfpathmoveto{\pgfqpoint{1.180897in}{1.627341in}}%
\pgfpathcurveto{\pgfqpoint{1.189134in}{1.627341in}}{\pgfqpoint{1.197034in}{1.630614in}}{\pgfqpoint{1.202858in}{1.636438in}}%
\pgfpathcurveto{\pgfqpoint{1.208682in}{1.642262in}}{\pgfqpoint{1.211954in}{1.650162in}}{\pgfqpoint{1.211954in}{1.658398in}}%
\pgfpathcurveto{\pgfqpoint{1.211954in}{1.666634in}}{\pgfqpoint{1.208682in}{1.674534in}}{\pgfqpoint{1.202858in}{1.680358in}}%
\pgfpathcurveto{\pgfqpoint{1.197034in}{1.686182in}}{\pgfqpoint{1.189134in}{1.689454in}}{\pgfqpoint{1.180897in}{1.689454in}}%
\pgfpathcurveto{\pgfqpoint{1.172661in}{1.689454in}}{\pgfqpoint{1.164761in}{1.686182in}}{\pgfqpoint{1.158937in}{1.680358in}}%
\pgfpathcurveto{\pgfqpoint{1.153113in}{1.674534in}}{\pgfqpoint{1.149841in}{1.666634in}}{\pgfqpoint{1.149841in}{1.658398in}}%
\pgfpathcurveto{\pgfqpoint{1.149841in}{1.650162in}}{\pgfqpoint{1.153113in}{1.642262in}}{\pgfqpoint{1.158937in}{1.636438in}}%
\pgfpathcurveto{\pgfqpoint{1.164761in}{1.630614in}}{\pgfqpoint{1.172661in}{1.627341in}}{\pgfqpoint{1.180897in}{1.627341in}}%
\pgfpathclose%
\pgfusepath{stroke,fill}%
\end{pgfscope}%
\begin{pgfscope}%
\pgfpathrectangle{\pgfqpoint{0.100000in}{0.212622in}}{\pgfqpoint{3.696000in}{3.696000in}}%
\pgfusepath{clip}%
\pgfsetbuttcap%
\pgfsetroundjoin%
\definecolor{currentfill}{rgb}{0.121569,0.466667,0.705882}%
\pgfsetfillcolor{currentfill}%
\pgfsetfillopacity{0.440064}%
\pgfsetlinewidth{1.003750pt}%
\definecolor{currentstroke}{rgb}{0.121569,0.466667,0.705882}%
\pgfsetstrokecolor{currentstroke}%
\pgfsetstrokeopacity{0.440064}%
\pgfsetdash{}{0pt}%
\pgfpathmoveto{\pgfqpoint{1.178892in}{1.626929in}}%
\pgfpathcurveto{\pgfqpoint{1.187129in}{1.626929in}}{\pgfqpoint{1.195029in}{1.630201in}}{\pgfqpoint{1.200853in}{1.636025in}}%
\pgfpathcurveto{\pgfqpoint{1.206676in}{1.641849in}}{\pgfqpoint{1.209949in}{1.649749in}}{\pgfqpoint{1.209949in}{1.657985in}}%
\pgfpathcurveto{\pgfqpoint{1.209949in}{1.666221in}}{\pgfqpoint{1.206676in}{1.674121in}}{\pgfqpoint{1.200853in}{1.679945in}}%
\pgfpathcurveto{\pgfqpoint{1.195029in}{1.685769in}}{\pgfqpoint{1.187129in}{1.689042in}}{\pgfqpoint{1.178892in}{1.689042in}}%
\pgfpathcurveto{\pgfqpoint{1.170656in}{1.689042in}}{\pgfqpoint{1.162756in}{1.685769in}}{\pgfqpoint{1.156932in}{1.679945in}}%
\pgfpathcurveto{\pgfqpoint{1.151108in}{1.674121in}}{\pgfqpoint{1.147836in}{1.666221in}}{\pgfqpoint{1.147836in}{1.657985in}}%
\pgfpathcurveto{\pgfqpoint{1.147836in}{1.649749in}}{\pgfqpoint{1.151108in}{1.641849in}}{\pgfqpoint{1.156932in}{1.636025in}}%
\pgfpathcurveto{\pgfqpoint{1.162756in}{1.630201in}}{\pgfqpoint{1.170656in}{1.626929in}}{\pgfqpoint{1.178892in}{1.626929in}}%
\pgfpathclose%
\pgfusepath{stroke,fill}%
\end{pgfscope}%
\begin{pgfscope}%
\pgfpathrectangle{\pgfqpoint{0.100000in}{0.212622in}}{\pgfqpoint{3.696000in}{3.696000in}}%
\pgfusepath{clip}%
\pgfsetbuttcap%
\pgfsetroundjoin%
\definecolor{currentfill}{rgb}{0.121569,0.466667,0.705882}%
\pgfsetfillcolor{currentfill}%
\pgfsetfillopacity{0.440612}%
\pgfsetlinewidth{1.003750pt}%
\definecolor{currentstroke}{rgb}{0.121569,0.466667,0.705882}%
\pgfsetstrokecolor{currentstroke}%
\pgfsetstrokeopacity{0.440612}%
\pgfsetdash{}{0pt}%
\pgfpathmoveto{\pgfqpoint{1.177306in}{1.625908in}}%
\pgfpathcurveto{\pgfqpoint{1.185542in}{1.625908in}}{\pgfqpoint{1.193442in}{1.629180in}}{\pgfqpoint{1.199266in}{1.635004in}}%
\pgfpathcurveto{\pgfqpoint{1.205090in}{1.640828in}}{\pgfqpoint{1.208362in}{1.648728in}}{\pgfqpoint{1.208362in}{1.656965in}}%
\pgfpathcurveto{\pgfqpoint{1.208362in}{1.665201in}}{\pgfqpoint{1.205090in}{1.673101in}}{\pgfqpoint{1.199266in}{1.678925in}}%
\pgfpathcurveto{\pgfqpoint{1.193442in}{1.684749in}}{\pgfqpoint{1.185542in}{1.688021in}}{\pgfqpoint{1.177306in}{1.688021in}}%
\pgfpathcurveto{\pgfqpoint{1.169069in}{1.688021in}}{\pgfqpoint{1.161169in}{1.684749in}}{\pgfqpoint{1.155345in}{1.678925in}}%
\pgfpathcurveto{\pgfqpoint{1.149522in}{1.673101in}}{\pgfqpoint{1.146249in}{1.665201in}}{\pgfqpoint{1.146249in}{1.656965in}}%
\pgfpathcurveto{\pgfqpoint{1.146249in}{1.648728in}}{\pgfqpoint{1.149522in}{1.640828in}}{\pgfqpoint{1.155345in}{1.635004in}}%
\pgfpathcurveto{\pgfqpoint{1.161169in}{1.629180in}}{\pgfqpoint{1.169069in}{1.625908in}}{\pgfqpoint{1.177306in}{1.625908in}}%
\pgfpathclose%
\pgfusepath{stroke,fill}%
\end{pgfscope}%
\begin{pgfscope}%
\pgfpathrectangle{\pgfqpoint{0.100000in}{0.212622in}}{\pgfqpoint{3.696000in}{3.696000in}}%
\pgfusepath{clip}%
\pgfsetbuttcap%
\pgfsetroundjoin%
\definecolor{currentfill}{rgb}{0.121569,0.466667,0.705882}%
\pgfsetfillcolor{currentfill}%
\pgfsetfillopacity{0.440944}%
\pgfsetlinewidth{1.003750pt}%
\definecolor{currentstroke}{rgb}{0.121569,0.466667,0.705882}%
\pgfsetstrokecolor{currentstroke}%
\pgfsetstrokeopacity{0.440944}%
\pgfsetdash{}{0pt}%
\pgfpathmoveto{\pgfqpoint{1.539008in}{1.715872in}}%
\pgfpathcurveto{\pgfqpoint{1.547244in}{1.715872in}}{\pgfqpoint{1.555144in}{1.719144in}}{\pgfqpoint{1.560968in}{1.724968in}}%
\pgfpathcurveto{\pgfqpoint{1.566792in}{1.730792in}}{\pgfqpoint{1.570065in}{1.738692in}}{\pgfqpoint{1.570065in}{1.746929in}}%
\pgfpathcurveto{\pgfqpoint{1.570065in}{1.755165in}}{\pgfqpoint{1.566792in}{1.763065in}}{\pgfqpoint{1.560968in}{1.768889in}}%
\pgfpathcurveto{\pgfqpoint{1.555144in}{1.774713in}}{\pgfqpoint{1.547244in}{1.777985in}}{\pgfqpoint{1.539008in}{1.777985in}}%
\pgfpathcurveto{\pgfqpoint{1.530772in}{1.777985in}}{\pgfqpoint{1.522872in}{1.774713in}}{\pgfqpoint{1.517048in}{1.768889in}}%
\pgfpathcurveto{\pgfqpoint{1.511224in}{1.763065in}}{\pgfqpoint{1.507952in}{1.755165in}}{\pgfqpoint{1.507952in}{1.746929in}}%
\pgfpathcurveto{\pgfqpoint{1.507952in}{1.738692in}}{\pgfqpoint{1.511224in}{1.730792in}}{\pgfqpoint{1.517048in}{1.724968in}}%
\pgfpathcurveto{\pgfqpoint{1.522872in}{1.719144in}}{\pgfqpoint{1.530772in}{1.715872in}}{\pgfqpoint{1.539008in}{1.715872in}}%
\pgfpathclose%
\pgfusepath{stroke,fill}%
\end{pgfscope}%
\begin{pgfscope}%
\pgfpathrectangle{\pgfqpoint{0.100000in}{0.212622in}}{\pgfqpoint{3.696000in}{3.696000in}}%
\pgfusepath{clip}%
\pgfsetbuttcap%
\pgfsetroundjoin%
\definecolor{currentfill}{rgb}{0.121569,0.466667,0.705882}%
\pgfsetfillcolor{currentfill}%
\pgfsetfillopacity{0.441238}%
\pgfsetlinewidth{1.003750pt}%
\definecolor{currentstroke}{rgb}{0.121569,0.466667,0.705882}%
\pgfsetstrokecolor{currentstroke}%
\pgfsetstrokeopacity{0.441238}%
\pgfsetdash{}{0pt}%
\pgfpathmoveto{\pgfqpoint{1.176361in}{1.626067in}}%
\pgfpathcurveto{\pgfqpoint{1.184597in}{1.626067in}}{\pgfqpoint{1.192497in}{1.629339in}}{\pgfqpoint{1.198321in}{1.635163in}}%
\pgfpathcurveto{\pgfqpoint{1.204145in}{1.640987in}}{\pgfqpoint{1.207418in}{1.648887in}}{\pgfqpoint{1.207418in}{1.657123in}}%
\pgfpathcurveto{\pgfqpoint{1.207418in}{1.665360in}}{\pgfqpoint{1.204145in}{1.673260in}}{\pgfqpoint{1.198321in}{1.679083in}}%
\pgfpathcurveto{\pgfqpoint{1.192497in}{1.684907in}}{\pgfqpoint{1.184597in}{1.688180in}}{\pgfqpoint{1.176361in}{1.688180in}}%
\pgfpathcurveto{\pgfqpoint{1.168125in}{1.688180in}}{\pgfqpoint{1.160225in}{1.684907in}}{\pgfqpoint{1.154401in}{1.679083in}}%
\pgfpathcurveto{\pgfqpoint{1.148577in}{1.673260in}}{\pgfqpoint{1.145305in}{1.665360in}}{\pgfqpoint{1.145305in}{1.657123in}}%
\pgfpathcurveto{\pgfqpoint{1.145305in}{1.648887in}}{\pgfqpoint{1.148577in}{1.640987in}}{\pgfqpoint{1.154401in}{1.635163in}}%
\pgfpathcurveto{\pgfqpoint{1.160225in}{1.629339in}}{\pgfqpoint{1.168125in}{1.626067in}}{\pgfqpoint{1.176361in}{1.626067in}}%
\pgfpathclose%
\pgfusepath{stroke,fill}%
\end{pgfscope}%
\begin{pgfscope}%
\pgfpathrectangle{\pgfqpoint{0.100000in}{0.212622in}}{\pgfqpoint{3.696000in}{3.696000in}}%
\pgfusepath{clip}%
\pgfsetbuttcap%
\pgfsetroundjoin%
\definecolor{currentfill}{rgb}{0.121569,0.466667,0.705882}%
\pgfsetfillcolor{currentfill}%
\pgfsetfillopacity{0.442141}%
\pgfsetlinewidth{1.003750pt}%
\definecolor{currentstroke}{rgb}{0.121569,0.466667,0.705882}%
\pgfsetstrokecolor{currentstroke}%
\pgfsetstrokeopacity{0.442141}%
\pgfsetdash{}{0pt}%
\pgfpathmoveto{\pgfqpoint{1.174479in}{1.625523in}}%
\pgfpathcurveto{\pgfqpoint{1.182715in}{1.625523in}}{\pgfqpoint{1.190615in}{1.628795in}}{\pgfqpoint{1.196439in}{1.634619in}}%
\pgfpathcurveto{\pgfqpoint{1.202263in}{1.640443in}}{\pgfqpoint{1.205535in}{1.648343in}}{\pgfqpoint{1.205535in}{1.656579in}}%
\pgfpathcurveto{\pgfqpoint{1.205535in}{1.664816in}}{\pgfqpoint{1.202263in}{1.672716in}}{\pgfqpoint{1.196439in}{1.678540in}}%
\pgfpathcurveto{\pgfqpoint{1.190615in}{1.684364in}}{\pgfqpoint{1.182715in}{1.687636in}}{\pgfqpoint{1.174479in}{1.687636in}}%
\pgfpathcurveto{\pgfqpoint{1.166242in}{1.687636in}}{\pgfqpoint{1.158342in}{1.684364in}}{\pgfqpoint{1.152518in}{1.678540in}}%
\pgfpathcurveto{\pgfqpoint{1.146694in}{1.672716in}}{\pgfqpoint{1.143422in}{1.664816in}}{\pgfqpoint{1.143422in}{1.656579in}}%
\pgfpathcurveto{\pgfqpoint{1.143422in}{1.648343in}}{\pgfqpoint{1.146694in}{1.640443in}}{\pgfqpoint{1.152518in}{1.634619in}}%
\pgfpathcurveto{\pgfqpoint{1.158342in}{1.628795in}}{\pgfqpoint{1.166242in}{1.625523in}}{\pgfqpoint{1.174479in}{1.625523in}}%
\pgfpathclose%
\pgfusepath{stroke,fill}%
\end{pgfscope}%
\begin{pgfscope}%
\pgfpathrectangle{\pgfqpoint{0.100000in}{0.212622in}}{\pgfqpoint{3.696000in}{3.696000in}}%
\pgfusepath{clip}%
\pgfsetbuttcap%
\pgfsetroundjoin%
\definecolor{currentfill}{rgb}{0.121569,0.466667,0.705882}%
\pgfsetfillcolor{currentfill}%
\pgfsetfillopacity{0.443243}%
\pgfsetlinewidth{1.003750pt}%
\definecolor{currentstroke}{rgb}{0.121569,0.466667,0.705882}%
\pgfsetstrokecolor{currentstroke}%
\pgfsetstrokeopacity{0.443243}%
\pgfsetdash{}{0pt}%
\pgfpathmoveto{\pgfqpoint{1.171293in}{1.621743in}}%
\pgfpathcurveto{\pgfqpoint{1.179529in}{1.621743in}}{\pgfqpoint{1.187429in}{1.625015in}}{\pgfqpoint{1.193253in}{1.630839in}}%
\pgfpathcurveto{\pgfqpoint{1.199077in}{1.636663in}}{\pgfqpoint{1.202350in}{1.644563in}}{\pgfqpoint{1.202350in}{1.652799in}}%
\pgfpathcurveto{\pgfqpoint{1.202350in}{1.661036in}}{\pgfqpoint{1.199077in}{1.668936in}}{\pgfqpoint{1.193253in}{1.674760in}}%
\pgfpathcurveto{\pgfqpoint{1.187429in}{1.680584in}}{\pgfqpoint{1.179529in}{1.683856in}}{\pgfqpoint{1.171293in}{1.683856in}}%
\pgfpathcurveto{\pgfqpoint{1.163057in}{1.683856in}}{\pgfqpoint{1.155157in}{1.680584in}}{\pgfqpoint{1.149333in}{1.674760in}}%
\pgfpathcurveto{\pgfqpoint{1.143509in}{1.668936in}}{\pgfqpoint{1.140237in}{1.661036in}}{\pgfqpoint{1.140237in}{1.652799in}}%
\pgfpathcurveto{\pgfqpoint{1.140237in}{1.644563in}}{\pgfqpoint{1.143509in}{1.636663in}}{\pgfqpoint{1.149333in}{1.630839in}}%
\pgfpathcurveto{\pgfqpoint{1.155157in}{1.625015in}}{\pgfqpoint{1.163057in}{1.621743in}}{\pgfqpoint{1.171293in}{1.621743in}}%
\pgfpathclose%
\pgfusepath{stroke,fill}%
\end{pgfscope}%
\begin{pgfscope}%
\pgfpathrectangle{\pgfqpoint{0.100000in}{0.212622in}}{\pgfqpoint{3.696000in}{3.696000in}}%
\pgfusepath{clip}%
\pgfsetbuttcap%
\pgfsetroundjoin%
\definecolor{currentfill}{rgb}{0.121569,0.466667,0.705882}%
\pgfsetfillcolor{currentfill}%
\pgfsetfillopacity{0.443253}%
\pgfsetlinewidth{1.003750pt}%
\definecolor{currentstroke}{rgb}{0.121569,0.466667,0.705882}%
\pgfsetstrokecolor{currentstroke}%
\pgfsetstrokeopacity{0.443253}%
\pgfsetdash{}{0pt}%
\pgfpathmoveto{\pgfqpoint{1.539774in}{1.717310in}}%
\pgfpathcurveto{\pgfqpoint{1.548011in}{1.717310in}}{\pgfqpoint{1.555911in}{1.720582in}}{\pgfqpoint{1.561735in}{1.726406in}}%
\pgfpathcurveto{\pgfqpoint{1.567559in}{1.732230in}}{\pgfqpoint{1.570831in}{1.740130in}}{\pgfqpoint{1.570831in}{1.748367in}}%
\pgfpathcurveto{\pgfqpoint{1.570831in}{1.756603in}}{\pgfqpoint{1.567559in}{1.764503in}}{\pgfqpoint{1.561735in}{1.770327in}}%
\pgfpathcurveto{\pgfqpoint{1.555911in}{1.776151in}}{\pgfqpoint{1.548011in}{1.779423in}}{\pgfqpoint{1.539774in}{1.779423in}}%
\pgfpathcurveto{\pgfqpoint{1.531538in}{1.779423in}}{\pgfqpoint{1.523638in}{1.776151in}}{\pgfqpoint{1.517814in}{1.770327in}}%
\pgfpathcurveto{\pgfqpoint{1.511990in}{1.764503in}}{\pgfqpoint{1.508718in}{1.756603in}}{\pgfqpoint{1.508718in}{1.748367in}}%
\pgfpathcurveto{\pgfqpoint{1.508718in}{1.740130in}}{\pgfqpoint{1.511990in}{1.732230in}}{\pgfqpoint{1.517814in}{1.726406in}}%
\pgfpathcurveto{\pgfqpoint{1.523638in}{1.720582in}}{\pgfqpoint{1.531538in}{1.717310in}}{\pgfqpoint{1.539774in}{1.717310in}}%
\pgfpathclose%
\pgfusepath{stroke,fill}%
\end{pgfscope}%
\begin{pgfscope}%
\pgfpathrectangle{\pgfqpoint{0.100000in}{0.212622in}}{\pgfqpoint{3.696000in}{3.696000in}}%
\pgfusepath{clip}%
\pgfsetbuttcap%
\pgfsetroundjoin%
\definecolor{currentfill}{rgb}{0.121569,0.466667,0.705882}%
\pgfsetfillcolor{currentfill}%
\pgfsetfillopacity{0.445446}%
\pgfsetlinewidth{1.003750pt}%
\definecolor{currentstroke}{rgb}{0.121569,0.466667,0.705882}%
\pgfsetstrokecolor{currentstroke}%
\pgfsetstrokeopacity{0.445446}%
\pgfsetdash{}{0pt}%
\pgfpathmoveto{\pgfqpoint{1.168789in}{1.624230in}}%
\pgfpathcurveto{\pgfqpoint{1.177026in}{1.624230in}}{\pgfqpoint{1.184926in}{1.627502in}}{\pgfqpoint{1.190750in}{1.633326in}}%
\pgfpathcurveto{\pgfqpoint{1.196573in}{1.639150in}}{\pgfqpoint{1.199846in}{1.647050in}}{\pgfqpoint{1.199846in}{1.655286in}}%
\pgfpathcurveto{\pgfqpoint{1.199846in}{1.663522in}}{\pgfqpoint{1.196573in}{1.671423in}}{\pgfqpoint{1.190750in}{1.677246in}}%
\pgfpathcurveto{\pgfqpoint{1.184926in}{1.683070in}}{\pgfqpoint{1.177026in}{1.686343in}}{\pgfqpoint{1.168789in}{1.686343in}}%
\pgfpathcurveto{\pgfqpoint{1.160553in}{1.686343in}}{\pgfqpoint{1.152653in}{1.683070in}}{\pgfqpoint{1.146829in}{1.677246in}}%
\pgfpathcurveto{\pgfqpoint{1.141005in}{1.671423in}}{\pgfqpoint{1.137733in}{1.663522in}}{\pgfqpoint{1.137733in}{1.655286in}}%
\pgfpathcurveto{\pgfqpoint{1.137733in}{1.647050in}}{\pgfqpoint{1.141005in}{1.639150in}}{\pgfqpoint{1.146829in}{1.633326in}}%
\pgfpathcurveto{\pgfqpoint{1.152653in}{1.627502in}}{\pgfqpoint{1.160553in}{1.624230in}}{\pgfqpoint{1.168789in}{1.624230in}}%
\pgfpathclose%
\pgfusepath{stroke,fill}%
\end{pgfscope}%
\begin{pgfscope}%
\pgfpathrectangle{\pgfqpoint{0.100000in}{0.212622in}}{\pgfqpoint{3.696000in}{3.696000in}}%
\pgfusepath{clip}%
\pgfsetbuttcap%
\pgfsetroundjoin%
\definecolor{currentfill}{rgb}{0.121569,0.466667,0.705882}%
\pgfsetfillcolor{currentfill}%
\pgfsetfillopacity{0.445655}%
\pgfsetlinewidth{1.003750pt}%
\definecolor{currentstroke}{rgb}{0.121569,0.466667,0.705882}%
\pgfsetstrokecolor{currentstroke}%
\pgfsetstrokeopacity{0.445655}%
\pgfsetdash{}{0pt}%
\pgfpathmoveto{\pgfqpoint{1.539924in}{1.717337in}}%
\pgfpathcurveto{\pgfqpoint{1.548160in}{1.717337in}}{\pgfqpoint{1.556060in}{1.720610in}}{\pgfqpoint{1.561884in}{1.726434in}}%
\pgfpathcurveto{\pgfqpoint{1.567708in}{1.732258in}}{\pgfqpoint{1.570980in}{1.740158in}}{\pgfqpoint{1.570980in}{1.748394in}}%
\pgfpathcurveto{\pgfqpoint{1.570980in}{1.756630in}}{\pgfqpoint{1.567708in}{1.764530in}}{\pgfqpoint{1.561884in}{1.770354in}}%
\pgfpathcurveto{\pgfqpoint{1.556060in}{1.776178in}}{\pgfqpoint{1.548160in}{1.779450in}}{\pgfqpoint{1.539924in}{1.779450in}}%
\pgfpathcurveto{\pgfqpoint{1.531687in}{1.779450in}}{\pgfqpoint{1.523787in}{1.776178in}}{\pgfqpoint{1.517964in}{1.770354in}}%
\pgfpathcurveto{\pgfqpoint{1.512140in}{1.764530in}}{\pgfqpoint{1.508867in}{1.756630in}}{\pgfqpoint{1.508867in}{1.748394in}}%
\pgfpathcurveto{\pgfqpoint{1.508867in}{1.740158in}}{\pgfqpoint{1.512140in}{1.732258in}}{\pgfqpoint{1.517964in}{1.726434in}}%
\pgfpathcurveto{\pgfqpoint{1.523787in}{1.720610in}}{\pgfqpoint{1.531687in}{1.717337in}}{\pgfqpoint{1.539924in}{1.717337in}}%
\pgfpathclose%
\pgfusepath{stroke,fill}%
\end{pgfscope}%
\begin{pgfscope}%
\pgfpathrectangle{\pgfqpoint{0.100000in}{0.212622in}}{\pgfqpoint{3.696000in}{3.696000in}}%
\pgfusepath{clip}%
\pgfsetbuttcap%
\pgfsetroundjoin%
\definecolor{currentfill}{rgb}{0.121569,0.466667,0.705882}%
\pgfsetfillcolor{currentfill}%
\pgfsetfillopacity{0.445982}%
\pgfsetlinewidth{1.003750pt}%
\definecolor{currentstroke}{rgb}{0.121569,0.466667,0.705882}%
\pgfsetstrokecolor{currentstroke}%
\pgfsetstrokeopacity{0.445982}%
\pgfsetdash{}{0pt}%
\pgfpathmoveto{\pgfqpoint{1.166545in}{1.620513in}}%
\pgfpathcurveto{\pgfqpoint{1.174782in}{1.620513in}}{\pgfqpoint{1.182682in}{1.623786in}}{\pgfqpoint{1.188506in}{1.629610in}}%
\pgfpathcurveto{\pgfqpoint{1.194329in}{1.635434in}}{\pgfqpoint{1.197602in}{1.643334in}}{\pgfqpoint{1.197602in}{1.651570in}}%
\pgfpathcurveto{\pgfqpoint{1.197602in}{1.659806in}}{\pgfqpoint{1.194329in}{1.667706in}}{\pgfqpoint{1.188506in}{1.673530in}}%
\pgfpathcurveto{\pgfqpoint{1.182682in}{1.679354in}}{\pgfqpoint{1.174782in}{1.682626in}}{\pgfqpoint{1.166545in}{1.682626in}}%
\pgfpathcurveto{\pgfqpoint{1.158309in}{1.682626in}}{\pgfqpoint{1.150409in}{1.679354in}}{\pgfqpoint{1.144585in}{1.673530in}}%
\pgfpathcurveto{\pgfqpoint{1.138761in}{1.667706in}}{\pgfqpoint{1.135489in}{1.659806in}}{\pgfqpoint{1.135489in}{1.651570in}}%
\pgfpathcurveto{\pgfqpoint{1.135489in}{1.643334in}}{\pgfqpoint{1.138761in}{1.635434in}}{\pgfqpoint{1.144585in}{1.629610in}}%
\pgfpathcurveto{\pgfqpoint{1.150409in}{1.623786in}}{\pgfqpoint{1.158309in}{1.620513in}}{\pgfqpoint{1.166545in}{1.620513in}}%
\pgfpathclose%
\pgfusepath{stroke,fill}%
\end{pgfscope}%
\begin{pgfscope}%
\pgfpathrectangle{\pgfqpoint{0.100000in}{0.212622in}}{\pgfqpoint{3.696000in}{3.696000in}}%
\pgfusepath{clip}%
\pgfsetbuttcap%
\pgfsetroundjoin%
\definecolor{currentfill}{rgb}{0.121569,0.466667,0.705882}%
\pgfsetfillcolor{currentfill}%
\pgfsetfillopacity{0.448320}%
\pgfsetlinewidth{1.003750pt}%
\definecolor{currentstroke}{rgb}{0.121569,0.466667,0.705882}%
\pgfsetstrokecolor{currentstroke}%
\pgfsetstrokeopacity{0.448320}%
\pgfsetdash{}{0pt}%
\pgfpathmoveto{\pgfqpoint{1.540324in}{1.716283in}}%
\pgfpathcurveto{\pgfqpoint{1.548561in}{1.716283in}}{\pgfqpoint{1.556461in}{1.719556in}}{\pgfqpoint{1.562285in}{1.725380in}}%
\pgfpathcurveto{\pgfqpoint{1.568109in}{1.731203in}}{\pgfqpoint{1.571381in}{1.739104in}}{\pgfqpoint{1.571381in}{1.747340in}}%
\pgfpathcurveto{\pgfqpoint{1.571381in}{1.755576in}}{\pgfqpoint{1.568109in}{1.763476in}}{\pgfqpoint{1.562285in}{1.769300in}}%
\pgfpathcurveto{\pgfqpoint{1.556461in}{1.775124in}}{\pgfqpoint{1.548561in}{1.778396in}}{\pgfqpoint{1.540324in}{1.778396in}}%
\pgfpathcurveto{\pgfqpoint{1.532088in}{1.778396in}}{\pgfqpoint{1.524188in}{1.775124in}}{\pgfqpoint{1.518364in}{1.769300in}}%
\pgfpathcurveto{\pgfqpoint{1.512540in}{1.763476in}}{\pgfqpoint{1.509268in}{1.755576in}}{\pgfqpoint{1.509268in}{1.747340in}}%
\pgfpathcurveto{\pgfqpoint{1.509268in}{1.739104in}}{\pgfqpoint{1.512540in}{1.731203in}}{\pgfqpoint{1.518364in}{1.725380in}}%
\pgfpathcurveto{\pgfqpoint{1.524188in}{1.719556in}}{\pgfqpoint{1.532088in}{1.716283in}}{\pgfqpoint{1.540324in}{1.716283in}}%
\pgfpathclose%
\pgfusepath{stroke,fill}%
\end{pgfscope}%
\begin{pgfscope}%
\pgfpathrectangle{\pgfqpoint{0.100000in}{0.212622in}}{\pgfqpoint{3.696000in}{3.696000in}}%
\pgfusepath{clip}%
\pgfsetbuttcap%
\pgfsetroundjoin%
\definecolor{currentfill}{rgb}{0.121569,0.466667,0.705882}%
\pgfsetfillcolor{currentfill}%
\pgfsetfillopacity{0.449772}%
\pgfsetlinewidth{1.003750pt}%
\definecolor{currentstroke}{rgb}{0.121569,0.466667,0.705882}%
\pgfsetstrokecolor{currentstroke}%
\pgfsetstrokeopacity{0.449772}%
\pgfsetdash{}{0pt}%
\pgfpathmoveto{\pgfqpoint{1.161840in}{1.627362in}}%
\pgfpathcurveto{\pgfqpoint{1.170076in}{1.627362in}}{\pgfqpoint{1.177977in}{1.630635in}}{\pgfqpoint{1.183800in}{1.636459in}}%
\pgfpathcurveto{\pgfqpoint{1.189624in}{1.642282in}}{\pgfqpoint{1.192897in}{1.650182in}}{\pgfqpoint{1.192897in}{1.658419in}}%
\pgfpathcurveto{\pgfqpoint{1.192897in}{1.666655in}}{\pgfqpoint{1.189624in}{1.674555in}}{\pgfqpoint{1.183800in}{1.680379in}}%
\pgfpathcurveto{\pgfqpoint{1.177977in}{1.686203in}}{\pgfqpoint{1.170076in}{1.689475in}}{\pgfqpoint{1.161840in}{1.689475in}}%
\pgfpathcurveto{\pgfqpoint{1.153604in}{1.689475in}}{\pgfqpoint{1.145704in}{1.686203in}}{\pgfqpoint{1.139880in}{1.680379in}}%
\pgfpathcurveto{\pgfqpoint{1.134056in}{1.674555in}}{\pgfqpoint{1.130784in}{1.666655in}}{\pgfqpoint{1.130784in}{1.658419in}}%
\pgfpathcurveto{\pgfqpoint{1.130784in}{1.650182in}}{\pgfqpoint{1.134056in}{1.642282in}}{\pgfqpoint{1.139880in}{1.636459in}}%
\pgfpathcurveto{\pgfqpoint{1.145704in}{1.630635in}}{\pgfqpoint{1.153604in}{1.627362in}}{\pgfqpoint{1.161840in}{1.627362in}}%
\pgfpathclose%
\pgfusepath{stroke,fill}%
\end{pgfscope}%
\begin{pgfscope}%
\pgfpathrectangle{\pgfqpoint{0.100000in}{0.212622in}}{\pgfqpoint{3.696000in}{3.696000in}}%
\pgfusepath{clip}%
\pgfsetbuttcap%
\pgfsetroundjoin%
\definecolor{currentfill}{rgb}{0.121569,0.466667,0.705882}%
\pgfsetfillcolor{currentfill}%
\pgfsetfillopacity{0.449980}%
\pgfsetlinewidth{1.003750pt}%
\definecolor{currentstroke}{rgb}{0.121569,0.466667,0.705882}%
\pgfsetstrokecolor{currentstroke}%
\pgfsetstrokeopacity{0.449980}%
\pgfsetdash{}{0pt}%
\pgfpathmoveto{\pgfqpoint{1.541074in}{1.716785in}}%
\pgfpathcurveto{\pgfqpoint{1.549310in}{1.716785in}}{\pgfqpoint{1.557210in}{1.720058in}}{\pgfqpoint{1.563034in}{1.725882in}}%
\pgfpathcurveto{\pgfqpoint{1.568858in}{1.731705in}}{\pgfqpoint{1.572130in}{1.739606in}}{\pgfqpoint{1.572130in}{1.747842in}}%
\pgfpathcurveto{\pgfqpoint{1.572130in}{1.756078in}}{\pgfqpoint{1.568858in}{1.763978in}}{\pgfqpoint{1.563034in}{1.769802in}}%
\pgfpathcurveto{\pgfqpoint{1.557210in}{1.775626in}}{\pgfqpoint{1.549310in}{1.778898in}}{\pgfqpoint{1.541074in}{1.778898in}}%
\pgfpathcurveto{\pgfqpoint{1.532837in}{1.778898in}}{\pgfqpoint{1.524937in}{1.775626in}}{\pgfqpoint{1.519113in}{1.769802in}}%
\pgfpathcurveto{\pgfqpoint{1.513289in}{1.763978in}}{\pgfqpoint{1.510017in}{1.756078in}}{\pgfqpoint{1.510017in}{1.747842in}}%
\pgfpathcurveto{\pgfqpoint{1.510017in}{1.739606in}}{\pgfqpoint{1.513289in}{1.731705in}}{\pgfqpoint{1.519113in}{1.725882in}}%
\pgfpathcurveto{\pgfqpoint{1.524937in}{1.720058in}}{\pgfqpoint{1.532837in}{1.716785in}}{\pgfqpoint{1.541074in}{1.716785in}}%
\pgfpathclose%
\pgfusepath{stroke,fill}%
\end{pgfscope}%
\begin{pgfscope}%
\pgfpathrectangle{\pgfqpoint{0.100000in}{0.212622in}}{\pgfqpoint{3.696000in}{3.696000in}}%
\pgfusepath{clip}%
\pgfsetbuttcap%
\pgfsetroundjoin%
\definecolor{currentfill}{rgb}{0.121569,0.466667,0.705882}%
\pgfsetfillcolor{currentfill}%
\pgfsetfillopacity{0.451783}%
\pgfsetlinewidth{1.003750pt}%
\definecolor{currentstroke}{rgb}{0.121569,0.466667,0.705882}%
\pgfsetstrokecolor{currentstroke}%
\pgfsetstrokeopacity{0.451783}%
\pgfsetdash{}{0pt}%
\pgfpathmoveto{\pgfqpoint{1.158668in}{1.628142in}}%
\pgfpathcurveto{\pgfqpoint{1.166904in}{1.628142in}}{\pgfqpoint{1.174804in}{1.631414in}}{\pgfqpoint{1.180628in}{1.637238in}}%
\pgfpathcurveto{\pgfqpoint{1.186452in}{1.643062in}}{\pgfqpoint{1.189725in}{1.650962in}}{\pgfqpoint{1.189725in}{1.659198in}}%
\pgfpathcurveto{\pgfqpoint{1.189725in}{1.667435in}}{\pgfqpoint{1.186452in}{1.675335in}}{\pgfqpoint{1.180628in}{1.681159in}}%
\pgfpathcurveto{\pgfqpoint{1.174804in}{1.686982in}}{\pgfqpoint{1.166904in}{1.690255in}}{\pgfqpoint{1.158668in}{1.690255in}}%
\pgfpathcurveto{\pgfqpoint{1.150432in}{1.690255in}}{\pgfqpoint{1.142532in}{1.686982in}}{\pgfqpoint{1.136708in}{1.681159in}}%
\pgfpathcurveto{\pgfqpoint{1.130884in}{1.675335in}}{\pgfqpoint{1.127612in}{1.667435in}}{\pgfqpoint{1.127612in}{1.659198in}}%
\pgfpathcurveto{\pgfqpoint{1.127612in}{1.650962in}}{\pgfqpoint{1.130884in}{1.643062in}}{\pgfqpoint{1.136708in}{1.637238in}}%
\pgfpathcurveto{\pgfqpoint{1.142532in}{1.631414in}}{\pgfqpoint{1.150432in}{1.628142in}}{\pgfqpoint{1.158668in}{1.628142in}}%
\pgfpathclose%
\pgfusepath{stroke,fill}%
\end{pgfscope}%
\begin{pgfscope}%
\pgfpathrectangle{\pgfqpoint{0.100000in}{0.212622in}}{\pgfqpoint{3.696000in}{3.696000in}}%
\pgfusepath{clip}%
\pgfsetbuttcap%
\pgfsetroundjoin%
\definecolor{currentfill}{rgb}{0.121569,0.466667,0.705882}%
\pgfsetfillcolor{currentfill}%
\pgfsetfillopacity{0.452125}%
\pgfsetlinewidth{1.003750pt}%
\definecolor{currentstroke}{rgb}{0.121569,0.466667,0.705882}%
\pgfsetstrokecolor{currentstroke}%
\pgfsetstrokeopacity{0.452125}%
\pgfsetdash{}{0pt}%
\pgfpathmoveto{\pgfqpoint{1.541630in}{1.716407in}}%
\pgfpathcurveto{\pgfqpoint{1.549866in}{1.716407in}}{\pgfqpoint{1.557766in}{1.719679in}}{\pgfqpoint{1.563590in}{1.725503in}}%
\pgfpathcurveto{\pgfqpoint{1.569414in}{1.731327in}}{\pgfqpoint{1.572686in}{1.739227in}}{\pgfqpoint{1.572686in}{1.747463in}}%
\pgfpathcurveto{\pgfqpoint{1.572686in}{1.755699in}}{\pgfqpoint{1.569414in}{1.763599in}}{\pgfqpoint{1.563590in}{1.769423in}}%
\pgfpathcurveto{\pgfqpoint{1.557766in}{1.775247in}}{\pgfqpoint{1.549866in}{1.778520in}}{\pgfqpoint{1.541630in}{1.778520in}}%
\pgfpathcurveto{\pgfqpoint{1.533394in}{1.778520in}}{\pgfqpoint{1.525494in}{1.775247in}}{\pgfqpoint{1.519670in}{1.769423in}}%
\pgfpathcurveto{\pgfqpoint{1.513846in}{1.763599in}}{\pgfqpoint{1.510573in}{1.755699in}}{\pgfqpoint{1.510573in}{1.747463in}}%
\pgfpathcurveto{\pgfqpoint{1.510573in}{1.739227in}}{\pgfqpoint{1.513846in}{1.731327in}}{\pgfqpoint{1.519670in}{1.725503in}}%
\pgfpathcurveto{\pgfqpoint{1.525494in}{1.719679in}}{\pgfqpoint{1.533394in}{1.716407in}}{\pgfqpoint{1.541630in}{1.716407in}}%
\pgfpathclose%
\pgfusepath{stroke,fill}%
\end{pgfscope}%
\begin{pgfscope}%
\pgfpathrectangle{\pgfqpoint{0.100000in}{0.212622in}}{\pgfqpoint{3.696000in}{3.696000in}}%
\pgfusepath{clip}%
\pgfsetbuttcap%
\pgfsetroundjoin%
\definecolor{currentfill}{rgb}{0.121569,0.466667,0.705882}%
\pgfsetfillcolor{currentfill}%
\pgfsetfillopacity{0.453099}%
\pgfsetlinewidth{1.003750pt}%
\definecolor{currentstroke}{rgb}{0.121569,0.466667,0.705882}%
\pgfsetstrokecolor{currentstroke}%
\pgfsetstrokeopacity{0.453099}%
\pgfsetdash{}{0pt}%
\pgfpathmoveto{\pgfqpoint{1.156214in}{1.625752in}}%
\pgfpathcurveto{\pgfqpoint{1.164450in}{1.625752in}}{\pgfqpoint{1.172350in}{1.629024in}}{\pgfqpoint{1.178174in}{1.634848in}}%
\pgfpathcurveto{\pgfqpoint{1.183998in}{1.640672in}}{\pgfqpoint{1.187270in}{1.648572in}}{\pgfqpoint{1.187270in}{1.656808in}}%
\pgfpathcurveto{\pgfqpoint{1.187270in}{1.665045in}}{\pgfqpoint{1.183998in}{1.672945in}}{\pgfqpoint{1.178174in}{1.678769in}}%
\pgfpathcurveto{\pgfqpoint{1.172350in}{1.684593in}}{\pgfqpoint{1.164450in}{1.687865in}}{\pgfqpoint{1.156214in}{1.687865in}}%
\pgfpathcurveto{\pgfqpoint{1.147977in}{1.687865in}}{\pgfqpoint{1.140077in}{1.684593in}}{\pgfqpoint{1.134253in}{1.678769in}}%
\pgfpathcurveto{\pgfqpoint{1.128429in}{1.672945in}}{\pgfqpoint{1.125157in}{1.665045in}}{\pgfqpoint{1.125157in}{1.656808in}}%
\pgfpathcurveto{\pgfqpoint{1.125157in}{1.648572in}}{\pgfqpoint{1.128429in}{1.640672in}}{\pgfqpoint{1.134253in}{1.634848in}}%
\pgfpathcurveto{\pgfqpoint{1.140077in}{1.629024in}}{\pgfqpoint{1.147977in}{1.625752in}}{\pgfqpoint{1.156214in}{1.625752in}}%
\pgfpathclose%
\pgfusepath{stroke,fill}%
\end{pgfscope}%
\begin{pgfscope}%
\pgfpathrectangle{\pgfqpoint{0.100000in}{0.212622in}}{\pgfqpoint{3.696000in}{3.696000in}}%
\pgfusepath{clip}%
\pgfsetbuttcap%
\pgfsetroundjoin%
\definecolor{currentfill}{rgb}{0.121569,0.466667,0.705882}%
\pgfsetfillcolor{currentfill}%
\pgfsetfillopacity{0.453307}%
\pgfsetlinewidth{1.003750pt}%
\definecolor{currentstroke}{rgb}{0.121569,0.466667,0.705882}%
\pgfsetstrokecolor{currentstroke}%
\pgfsetstrokeopacity{0.453307}%
\pgfsetdash{}{0pt}%
\pgfpathmoveto{\pgfqpoint{1.541902in}{1.716193in}}%
\pgfpathcurveto{\pgfqpoint{1.550138in}{1.716193in}}{\pgfqpoint{1.558038in}{1.719466in}}{\pgfqpoint{1.563862in}{1.725289in}}%
\pgfpathcurveto{\pgfqpoint{1.569686in}{1.731113in}}{\pgfqpoint{1.572959in}{1.739013in}}{\pgfqpoint{1.572959in}{1.747250in}}%
\pgfpathcurveto{\pgfqpoint{1.572959in}{1.755486in}}{\pgfqpoint{1.569686in}{1.763386in}}{\pgfqpoint{1.563862in}{1.769210in}}%
\pgfpathcurveto{\pgfqpoint{1.558038in}{1.775034in}}{\pgfqpoint{1.550138in}{1.778306in}}{\pgfqpoint{1.541902in}{1.778306in}}%
\pgfpathcurveto{\pgfqpoint{1.533666in}{1.778306in}}{\pgfqpoint{1.525766in}{1.775034in}}{\pgfqpoint{1.519942in}{1.769210in}}%
\pgfpathcurveto{\pgfqpoint{1.514118in}{1.763386in}}{\pgfqpoint{1.510846in}{1.755486in}}{\pgfqpoint{1.510846in}{1.747250in}}%
\pgfpathcurveto{\pgfqpoint{1.510846in}{1.739013in}}{\pgfqpoint{1.514118in}{1.731113in}}{\pgfqpoint{1.519942in}{1.725289in}}%
\pgfpathcurveto{\pgfqpoint{1.525766in}{1.719466in}}{\pgfqpoint{1.533666in}{1.716193in}}{\pgfqpoint{1.541902in}{1.716193in}}%
\pgfpathclose%
\pgfusepath{stroke,fill}%
\end{pgfscope}%
\begin{pgfscope}%
\pgfpathrectangle{\pgfqpoint{0.100000in}{0.212622in}}{\pgfqpoint{3.696000in}{3.696000in}}%
\pgfusepath{clip}%
\pgfsetbuttcap%
\pgfsetroundjoin%
\definecolor{currentfill}{rgb}{0.121569,0.466667,0.705882}%
\pgfsetfillcolor{currentfill}%
\pgfsetfillopacity{0.453948}%
\pgfsetlinewidth{1.003750pt}%
\definecolor{currentstroke}{rgb}{0.121569,0.466667,0.705882}%
\pgfsetstrokecolor{currentstroke}%
\pgfsetstrokeopacity{0.453948}%
\pgfsetdash{}{0pt}%
\pgfpathmoveto{\pgfqpoint{1.542181in}{1.716078in}}%
\pgfpathcurveto{\pgfqpoint{1.550418in}{1.716078in}}{\pgfqpoint{1.558318in}{1.719350in}}{\pgfqpoint{1.564142in}{1.725174in}}%
\pgfpathcurveto{\pgfqpoint{1.569966in}{1.730998in}}{\pgfqpoint{1.573238in}{1.738898in}}{\pgfqpoint{1.573238in}{1.747134in}}%
\pgfpathcurveto{\pgfqpoint{1.573238in}{1.755370in}}{\pgfqpoint{1.569966in}{1.763270in}}{\pgfqpoint{1.564142in}{1.769094in}}%
\pgfpathcurveto{\pgfqpoint{1.558318in}{1.774918in}}{\pgfqpoint{1.550418in}{1.778191in}}{\pgfqpoint{1.542181in}{1.778191in}}%
\pgfpathcurveto{\pgfqpoint{1.533945in}{1.778191in}}{\pgfqpoint{1.526045in}{1.774918in}}{\pgfqpoint{1.520221in}{1.769094in}}%
\pgfpathcurveto{\pgfqpoint{1.514397in}{1.763270in}}{\pgfqpoint{1.511125in}{1.755370in}}{\pgfqpoint{1.511125in}{1.747134in}}%
\pgfpathcurveto{\pgfqpoint{1.511125in}{1.738898in}}{\pgfqpoint{1.514397in}{1.730998in}}{\pgfqpoint{1.520221in}{1.725174in}}%
\pgfpathcurveto{\pgfqpoint{1.526045in}{1.719350in}}{\pgfqpoint{1.533945in}{1.716078in}}{\pgfqpoint{1.542181in}{1.716078in}}%
\pgfpathclose%
\pgfusepath{stroke,fill}%
\end{pgfscope}%
\begin{pgfscope}%
\pgfpathrectangle{\pgfqpoint{0.100000in}{0.212622in}}{\pgfqpoint{3.696000in}{3.696000in}}%
\pgfusepath{clip}%
\pgfsetbuttcap%
\pgfsetroundjoin%
\definecolor{currentfill}{rgb}{0.121569,0.466667,0.705882}%
\pgfsetfillcolor{currentfill}%
\pgfsetfillopacity{0.454401}%
\pgfsetlinewidth{1.003750pt}%
\definecolor{currentstroke}{rgb}{0.121569,0.466667,0.705882}%
\pgfsetstrokecolor{currentstroke}%
\pgfsetstrokeopacity{0.454401}%
\pgfsetdash{}{0pt}%
\pgfpathmoveto{\pgfqpoint{1.542298in}{1.716473in}}%
\pgfpathcurveto{\pgfqpoint{1.550534in}{1.716473in}}{\pgfqpoint{1.558434in}{1.719745in}}{\pgfqpoint{1.564258in}{1.725569in}}%
\pgfpathcurveto{\pgfqpoint{1.570082in}{1.731393in}}{\pgfqpoint{1.573354in}{1.739293in}}{\pgfqpoint{1.573354in}{1.747530in}}%
\pgfpathcurveto{\pgfqpoint{1.573354in}{1.755766in}}{\pgfqpoint{1.570082in}{1.763666in}}{\pgfqpoint{1.564258in}{1.769490in}}%
\pgfpathcurveto{\pgfqpoint{1.558434in}{1.775314in}}{\pgfqpoint{1.550534in}{1.778586in}}{\pgfqpoint{1.542298in}{1.778586in}}%
\pgfpathcurveto{\pgfqpoint{1.534062in}{1.778586in}}{\pgfqpoint{1.526162in}{1.775314in}}{\pgfqpoint{1.520338in}{1.769490in}}%
\pgfpathcurveto{\pgfqpoint{1.514514in}{1.763666in}}{\pgfqpoint{1.511241in}{1.755766in}}{\pgfqpoint{1.511241in}{1.747530in}}%
\pgfpathcurveto{\pgfqpoint{1.511241in}{1.739293in}}{\pgfqpoint{1.514514in}{1.731393in}}{\pgfqpoint{1.520338in}{1.725569in}}%
\pgfpathcurveto{\pgfqpoint{1.526162in}{1.719745in}}{\pgfqpoint{1.534062in}{1.716473in}}{\pgfqpoint{1.542298in}{1.716473in}}%
\pgfpathclose%
\pgfusepath{stroke,fill}%
\end{pgfscope}%
\begin{pgfscope}%
\pgfpathrectangle{\pgfqpoint{0.100000in}{0.212622in}}{\pgfqpoint{3.696000in}{3.696000in}}%
\pgfusepath{clip}%
\pgfsetbuttcap%
\pgfsetroundjoin%
\definecolor{currentfill}{rgb}{0.121569,0.466667,0.705882}%
\pgfsetfillcolor{currentfill}%
\pgfsetfillopacity{0.454586}%
\pgfsetlinewidth{1.003750pt}%
\definecolor{currentstroke}{rgb}{0.121569,0.466667,0.705882}%
\pgfsetstrokecolor{currentstroke}%
\pgfsetstrokeopacity{0.454586}%
\pgfsetdash{}{0pt}%
\pgfpathmoveto{\pgfqpoint{1.542343in}{1.716383in}}%
\pgfpathcurveto{\pgfqpoint{1.550580in}{1.716383in}}{\pgfqpoint{1.558480in}{1.719656in}}{\pgfqpoint{1.564304in}{1.725479in}}%
\pgfpathcurveto{\pgfqpoint{1.570128in}{1.731303in}}{\pgfqpoint{1.573400in}{1.739203in}}{\pgfqpoint{1.573400in}{1.747440in}}%
\pgfpathcurveto{\pgfqpoint{1.573400in}{1.755676in}}{\pgfqpoint{1.570128in}{1.763576in}}{\pgfqpoint{1.564304in}{1.769400in}}%
\pgfpathcurveto{\pgfqpoint{1.558480in}{1.775224in}}{\pgfqpoint{1.550580in}{1.778496in}}{\pgfqpoint{1.542343in}{1.778496in}}%
\pgfpathcurveto{\pgfqpoint{1.534107in}{1.778496in}}{\pgfqpoint{1.526207in}{1.775224in}}{\pgfqpoint{1.520383in}{1.769400in}}%
\pgfpathcurveto{\pgfqpoint{1.514559in}{1.763576in}}{\pgfqpoint{1.511287in}{1.755676in}}{\pgfqpoint{1.511287in}{1.747440in}}%
\pgfpathcurveto{\pgfqpoint{1.511287in}{1.739203in}}{\pgfqpoint{1.514559in}{1.731303in}}{\pgfqpoint{1.520383in}{1.725479in}}%
\pgfpathcurveto{\pgfqpoint{1.526207in}{1.719656in}}{\pgfqpoint{1.534107in}{1.716383in}}{\pgfqpoint{1.542343in}{1.716383in}}%
\pgfpathclose%
\pgfusepath{stroke,fill}%
\end{pgfscope}%
\begin{pgfscope}%
\pgfpathrectangle{\pgfqpoint{0.100000in}{0.212622in}}{\pgfqpoint{3.696000in}{3.696000in}}%
\pgfusepath{clip}%
\pgfsetbuttcap%
\pgfsetroundjoin%
\definecolor{currentfill}{rgb}{0.121569,0.466667,0.705882}%
\pgfsetfillcolor{currentfill}%
\pgfsetfillopacity{0.455096}%
\pgfsetlinewidth{1.003750pt}%
\definecolor{currentstroke}{rgb}{0.121569,0.466667,0.705882}%
\pgfsetstrokecolor{currentstroke}%
\pgfsetstrokeopacity{0.455096}%
\pgfsetdash{}{0pt}%
\pgfpathmoveto{\pgfqpoint{1.542646in}{1.715714in}}%
\pgfpathcurveto{\pgfqpoint{1.550882in}{1.715714in}}{\pgfqpoint{1.558782in}{1.718986in}}{\pgfqpoint{1.564606in}{1.724810in}}%
\pgfpathcurveto{\pgfqpoint{1.570430in}{1.730634in}}{\pgfqpoint{1.573703in}{1.738534in}}{\pgfqpoint{1.573703in}{1.746771in}}%
\pgfpathcurveto{\pgfqpoint{1.573703in}{1.755007in}}{\pgfqpoint{1.570430in}{1.762907in}}{\pgfqpoint{1.564606in}{1.768731in}}%
\pgfpathcurveto{\pgfqpoint{1.558782in}{1.774555in}}{\pgfqpoint{1.550882in}{1.777827in}}{\pgfqpoint{1.542646in}{1.777827in}}%
\pgfpathcurveto{\pgfqpoint{1.534410in}{1.777827in}}{\pgfqpoint{1.526510in}{1.774555in}}{\pgfqpoint{1.520686in}{1.768731in}}%
\pgfpathcurveto{\pgfqpoint{1.514862in}{1.762907in}}{\pgfqpoint{1.511590in}{1.755007in}}{\pgfqpoint{1.511590in}{1.746771in}}%
\pgfpathcurveto{\pgfqpoint{1.511590in}{1.738534in}}{\pgfqpoint{1.514862in}{1.730634in}}{\pgfqpoint{1.520686in}{1.724810in}}%
\pgfpathcurveto{\pgfqpoint{1.526510in}{1.718986in}}{\pgfqpoint{1.534410in}{1.715714in}}{\pgfqpoint{1.542646in}{1.715714in}}%
\pgfpathclose%
\pgfusepath{stroke,fill}%
\end{pgfscope}%
\begin{pgfscope}%
\pgfpathrectangle{\pgfqpoint{0.100000in}{0.212622in}}{\pgfqpoint{3.696000in}{3.696000in}}%
\pgfusepath{clip}%
\pgfsetbuttcap%
\pgfsetroundjoin%
\definecolor{currentfill}{rgb}{0.121569,0.466667,0.705882}%
\pgfsetfillcolor{currentfill}%
\pgfsetfillopacity{0.455497}%
\pgfsetlinewidth{1.003750pt}%
\definecolor{currentstroke}{rgb}{0.121569,0.466667,0.705882}%
\pgfsetstrokecolor{currentstroke}%
\pgfsetstrokeopacity{0.455497}%
\pgfsetdash{}{0pt}%
\pgfpathmoveto{\pgfqpoint{1.542871in}{1.715946in}}%
\pgfpathcurveto{\pgfqpoint{1.551108in}{1.715946in}}{\pgfqpoint{1.559008in}{1.719218in}}{\pgfqpoint{1.564832in}{1.725042in}}%
\pgfpathcurveto{\pgfqpoint{1.570655in}{1.730866in}}{\pgfqpoint{1.573928in}{1.738766in}}{\pgfqpoint{1.573928in}{1.747003in}}%
\pgfpathcurveto{\pgfqpoint{1.573928in}{1.755239in}}{\pgfqpoint{1.570655in}{1.763139in}}{\pgfqpoint{1.564832in}{1.768963in}}%
\pgfpathcurveto{\pgfqpoint{1.559008in}{1.774787in}}{\pgfqpoint{1.551108in}{1.778059in}}{\pgfqpoint{1.542871in}{1.778059in}}%
\pgfpathcurveto{\pgfqpoint{1.534635in}{1.778059in}}{\pgfqpoint{1.526735in}{1.774787in}}{\pgfqpoint{1.520911in}{1.768963in}}%
\pgfpathcurveto{\pgfqpoint{1.515087in}{1.763139in}}{\pgfqpoint{1.511815in}{1.755239in}}{\pgfqpoint{1.511815in}{1.747003in}}%
\pgfpathcurveto{\pgfqpoint{1.511815in}{1.738766in}}{\pgfqpoint{1.515087in}{1.730866in}}{\pgfqpoint{1.520911in}{1.725042in}}%
\pgfpathcurveto{\pgfqpoint{1.526735in}{1.719218in}}{\pgfqpoint{1.534635in}{1.715946in}}{\pgfqpoint{1.542871in}{1.715946in}}%
\pgfpathclose%
\pgfusepath{stroke,fill}%
\end{pgfscope}%
\begin{pgfscope}%
\pgfpathrectangle{\pgfqpoint{0.100000in}{0.212622in}}{\pgfqpoint{3.696000in}{3.696000in}}%
\pgfusepath{clip}%
\pgfsetbuttcap%
\pgfsetroundjoin%
\definecolor{currentfill}{rgb}{0.121569,0.466667,0.705882}%
\pgfsetfillcolor{currentfill}%
\pgfsetfillopacity{0.456551}%
\pgfsetlinewidth{1.003750pt}%
\definecolor{currentstroke}{rgb}{0.121569,0.466667,0.705882}%
\pgfsetstrokecolor{currentstroke}%
\pgfsetstrokeopacity{0.456551}%
\pgfsetdash{}{0pt}%
\pgfpathmoveto{\pgfqpoint{1.543207in}{1.716017in}}%
\pgfpathcurveto{\pgfqpoint{1.551443in}{1.716017in}}{\pgfqpoint{1.559343in}{1.719290in}}{\pgfqpoint{1.565167in}{1.725114in}}%
\pgfpathcurveto{\pgfqpoint{1.570991in}{1.730938in}}{\pgfqpoint{1.574263in}{1.738838in}}{\pgfqpoint{1.574263in}{1.747074in}}%
\pgfpathcurveto{\pgfqpoint{1.574263in}{1.755310in}}{\pgfqpoint{1.570991in}{1.763210in}}{\pgfqpoint{1.565167in}{1.769034in}}%
\pgfpathcurveto{\pgfqpoint{1.559343in}{1.774858in}}{\pgfqpoint{1.551443in}{1.778130in}}{\pgfqpoint{1.543207in}{1.778130in}}%
\pgfpathcurveto{\pgfqpoint{1.534971in}{1.778130in}}{\pgfqpoint{1.527071in}{1.774858in}}{\pgfqpoint{1.521247in}{1.769034in}}%
\pgfpathcurveto{\pgfqpoint{1.515423in}{1.763210in}}{\pgfqpoint{1.512150in}{1.755310in}}{\pgfqpoint{1.512150in}{1.747074in}}%
\pgfpathcurveto{\pgfqpoint{1.512150in}{1.738838in}}{\pgfqpoint{1.515423in}{1.730938in}}{\pgfqpoint{1.521247in}{1.725114in}}%
\pgfpathcurveto{\pgfqpoint{1.527071in}{1.719290in}}{\pgfqpoint{1.534971in}{1.716017in}}{\pgfqpoint{1.543207in}{1.716017in}}%
\pgfpathclose%
\pgfusepath{stroke,fill}%
\end{pgfscope}%
\begin{pgfscope}%
\pgfpathrectangle{\pgfqpoint{0.100000in}{0.212622in}}{\pgfqpoint{3.696000in}{3.696000in}}%
\pgfusepath{clip}%
\pgfsetbuttcap%
\pgfsetroundjoin%
\definecolor{currentfill}{rgb}{0.121569,0.466667,0.705882}%
\pgfsetfillcolor{currentfill}%
\pgfsetfillopacity{0.457766}%
\pgfsetlinewidth{1.003750pt}%
\definecolor{currentstroke}{rgb}{0.121569,0.466667,0.705882}%
\pgfsetstrokecolor{currentstroke}%
\pgfsetstrokeopacity{0.457766}%
\pgfsetdash{}{0pt}%
\pgfpathmoveto{\pgfqpoint{1.543540in}{1.715864in}}%
\pgfpathcurveto{\pgfqpoint{1.551776in}{1.715864in}}{\pgfqpoint{1.559676in}{1.719136in}}{\pgfqpoint{1.565500in}{1.724960in}}%
\pgfpathcurveto{\pgfqpoint{1.571324in}{1.730784in}}{\pgfqpoint{1.574596in}{1.738684in}}{\pgfqpoint{1.574596in}{1.746921in}}%
\pgfpathcurveto{\pgfqpoint{1.574596in}{1.755157in}}{\pgfqpoint{1.571324in}{1.763057in}}{\pgfqpoint{1.565500in}{1.768881in}}%
\pgfpathcurveto{\pgfqpoint{1.559676in}{1.774705in}}{\pgfqpoint{1.551776in}{1.777977in}}{\pgfqpoint{1.543540in}{1.777977in}}%
\pgfpathcurveto{\pgfqpoint{1.535304in}{1.777977in}}{\pgfqpoint{1.527404in}{1.774705in}}{\pgfqpoint{1.521580in}{1.768881in}}%
\pgfpathcurveto{\pgfqpoint{1.515756in}{1.763057in}}{\pgfqpoint{1.512483in}{1.755157in}}{\pgfqpoint{1.512483in}{1.746921in}}%
\pgfpathcurveto{\pgfqpoint{1.512483in}{1.738684in}}{\pgfqpoint{1.515756in}{1.730784in}}{\pgfqpoint{1.521580in}{1.724960in}}%
\pgfpathcurveto{\pgfqpoint{1.527404in}{1.719136in}}{\pgfqpoint{1.535304in}{1.715864in}}{\pgfqpoint{1.543540in}{1.715864in}}%
\pgfpathclose%
\pgfusepath{stroke,fill}%
\end{pgfscope}%
\begin{pgfscope}%
\pgfpathrectangle{\pgfqpoint{0.100000in}{0.212622in}}{\pgfqpoint{3.696000in}{3.696000in}}%
\pgfusepath{clip}%
\pgfsetbuttcap%
\pgfsetroundjoin%
\definecolor{currentfill}{rgb}{0.121569,0.466667,0.705882}%
\pgfsetfillcolor{currentfill}%
\pgfsetfillopacity{0.459202}%
\pgfsetlinewidth{1.003750pt}%
\definecolor{currentstroke}{rgb}{0.121569,0.466667,0.705882}%
\pgfsetstrokecolor{currentstroke}%
\pgfsetstrokeopacity{0.459202}%
\pgfsetdash{}{0pt}%
\pgfpathmoveto{\pgfqpoint{1.543998in}{1.715278in}}%
\pgfpathcurveto{\pgfqpoint{1.552234in}{1.715278in}}{\pgfqpoint{1.560134in}{1.718551in}}{\pgfqpoint{1.565958in}{1.724375in}}%
\pgfpathcurveto{\pgfqpoint{1.571782in}{1.730199in}}{\pgfqpoint{1.575054in}{1.738099in}}{\pgfqpoint{1.575054in}{1.746335in}}%
\pgfpathcurveto{\pgfqpoint{1.575054in}{1.754571in}}{\pgfqpoint{1.571782in}{1.762471in}}{\pgfqpoint{1.565958in}{1.768295in}}%
\pgfpathcurveto{\pgfqpoint{1.560134in}{1.774119in}}{\pgfqpoint{1.552234in}{1.777391in}}{\pgfqpoint{1.543998in}{1.777391in}}%
\pgfpathcurveto{\pgfqpoint{1.535762in}{1.777391in}}{\pgfqpoint{1.527862in}{1.774119in}}{\pgfqpoint{1.522038in}{1.768295in}}%
\pgfpathcurveto{\pgfqpoint{1.516214in}{1.762471in}}{\pgfqpoint{1.512941in}{1.754571in}}{\pgfqpoint{1.512941in}{1.746335in}}%
\pgfpathcurveto{\pgfqpoint{1.512941in}{1.738099in}}{\pgfqpoint{1.516214in}{1.730199in}}{\pgfqpoint{1.522038in}{1.724375in}}%
\pgfpathcurveto{\pgfqpoint{1.527862in}{1.718551in}}{\pgfqpoint{1.535762in}{1.715278in}}{\pgfqpoint{1.543998in}{1.715278in}}%
\pgfpathclose%
\pgfusepath{stroke,fill}%
\end{pgfscope}%
\begin{pgfscope}%
\pgfpathrectangle{\pgfqpoint{0.100000in}{0.212622in}}{\pgfqpoint{3.696000in}{3.696000in}}%
\pgfusepath{clip}%
\pgfsetbuttcap%
\pgfsetroundjoin%
\definecolor{currentfill}{rgb}{0.121569,0.466667,0.705882}%
\pgfsetfillcolor{currentfill}%
\pgfsetfillopacity{0.460559}%
\pgfsetlinewidth{1.003750pt}%
\definecolor{currentstroke}{rgb}{0.121569,0.466667,0.705882}%
\pgfsetstrokecolor{currentstroke}%
\pgfsetstrokeopacity{0.460559}%
\pgfsetdash{}{0pt}%
\pgfpathmoveto{\pgfqpoint{1.151798in}{1.644418in}}%
\pgfpathcurveto{\pgfqpoint{1.160035in}{1.644418in}}{\pgfqpoint{1.167935in}{1.647690in}}{\pgfqpoint{1.173759in}{1.653514in}}%
\pgfpathcurveto{\pgfqpoint{1.179583in}{1.659338in}}{\pgfqpoint{1.182855in}{1.667238in}}{\pgfqpoint{1.182855in}{1.675474in}}%
\pgfpathcurveto{\pgfqpoint{1.182855in}{1.683711in}}{\pgfqpoint{1.179583in}{1.691611in}}{\pgfqpoint{1.173759in}{1.697435in}}%
\pgfpathcurveto{\pgfqpoint{1.167935in}{1.703259in}}{\pgfqpoint{1.160035in}{1.706531in}}{\pgfqpoint{1.151798in}{1.706531in}}%
\pgfpathcurveto{\pgfqpoint{1.143562in}{1.706531in}}{\pgfqpoint{1.135662in}{1.703259in}}{\pgfqpoint{1.129838in}{1.697435in}}%
\pgfpathcurveto{\pgfqpoint{1.124014in}{1.691611in}}{\pgfqpoint{1.120742in}{1.683711in}}{\pgfqpoint{1.120742in}{1.675474in}}%
\pgfpathcurveto{\pgfqpoint{1.120742in}{1.667238in}}{\pgfqpoint{1.124014in}{1.659338in}}{\pgfqpoint{1.129838in}{1.653514in}}%
\pgfpathcurveto{\pgfqpoint{1.135662in}{1.647690in}}{\pgfqpoint{1.143562in}{1.644418in}}{\pgfqpoint{1.151798in}{1.644418in}}%
\pgfpathclose%
\pgfusepath{stroke,fill}%
\end{pgfscope}%
\begin{pgfscope}%
\pgfpathrectangle{\pgfqpoint{0.100000in}{0.212622in}}{\pgfqpoint{3.696000in}{3.696000in}}%
\pgfusepath{clip}%
\pgfsetbuttcap%
\pgfsetroundjoin%
\definecolor{currentfill}{rgb}{0.121569,0.466667,0.705882}%
\pgfsetfillcolor{currentfill}%
\pgfsetfillopacity{0.461229}%
\pgfsetlinewidth{1.003750pt}%
\definecolor{currentstroke}{rgb}{0.121569,0.466667,0.705882}%
\pgfsetstrokecolor{currentstroke}%
\pgfsetstrokeopacity{0.461229}%
\pgfsetdash{}{0pt}%
\pgfpathmoveto{\pgfqpoint{1.545008in}{1.716034in}}%
\pgfpathcurveto{\pgfqpoint{1.553244in}{1.716034in}}{\pgfqpoint{1.561144in}{1.719306in}}{\pgfqpoint{1.566968in}{1.725130in}}%
\pgfpathcurveto{\pgfqpoint{1.572792in}{1.730954in}}{\pgfqpoint{1.576064in}{1.738854in}}{\pgfqpoint{1.576064in}{1.747091in}}%
\pgfpathcurveto{\pgfqpoint{1.576064in}{1.755327in}}{\pgfqpoint{1.572792in}{1.763227in}}{\pgfqpoint{1.566968in}{1.769051in}}%
\pgfpathcurveto{\pgfqpoint{1.561144in}{1.774875in}}{\pgfqpoint{1.553244in}{1.778147in}}{\pgfqpoint{1.545008in}{1.778147in}}%
\pgfpathcurveto{\pgfqpoint{1.536772in}{1.778147in}}{\pgfqpoint{1.528872in}{1.774875in}}{\pgfqpoint{1.523048in}{1.769051in}}%
\pgfpathcurveto{\pgfqpoint{1.517224in}{1.763227in}}{\pgfqpoint{1.513951in}{1.755327in}}{\pgfqpoint{1.513951in}{1.747091in}}%
\pgfpathcurveto{\pgfqpoint{1.513951in}{1.738854in}}{\pgfqpoint{1.517224in}{1.730954in}}{\pgfqpoint{1.523048in}{1.725130in}}%
\pgfpathcurveto{\pgfqpoint{1.528872in}{1.719306in}}{\pgfqpoint{1.536772in}{1.716034in}}{\pgfqpoint{1.545008in}{1.716034in}}%
\pgfpathclose%
\pgfusepath{stroke,fill}%
\end{pgfscope}%
\begin{pgfscope}%
\pgfpathrectangle{\pgfqpoint{0.100000in}{0.212622in}}{\pgfqpoint{3.696000in}{3.696000in}}%
\pgfusepath{clip}%
\pgfsetbuttcap%
\pgfsetroundjoin%
\definecolor{currentfill}{rgb}{0.121569,0.466667,0.705882}%
\pgfsetfillcolor{currentfill}%
\pgfsetfillopacity{0.463239}%
\pgfsetlinewidth{1.003750pt}%
\definecolor{currentstroke}{rgb}{0.121569,0.466667,0.705882}%
\pgfsetstrokecolor{currentstroke}%
\pgfsetstrokeopacity{0.463239}%
\pgfsetdash{}{0pt}%
\pgfpathmoveto{\pgfqpoint{1.545364in}{1.714926in}}%
\pgfpathcurveto{\pgfqpoint{1.553601in}{1.714926in}}{\pgfqpoint{1.561501in}{1.718199in}}{\pgfqpoint{1.567324in}{1.724023in}}%
\pgfpathcurveto{\pgfqpoint{1.573148in}{1.729846in}}{\pgfqpoint{1.576421in}{1.737747in}}{\pgfqpoint{1.576421in}{1.745983in}}%
\pgfpathcurveto{\pgfqpoint{1.576421in}{1.754219in}}{\pgfqpoint{1.573148in}{1.762119in}}{\pgfqpoint{1.567324in}{1.767943in}}%
\pgfpathcurveto{\pgfqpoint{1.561501in}{1.773767in}}{\pgfqpoint{1.553601in}{1.777039in}}{\pgfqpoint{1.545364in}{1.777039in}}%
\pgfpathcurveto{\pgfqpoint{1.537128in}{1.777039in}}{\pgfqpoint{1.529228in}{1.773767in}}{\pgfqpoint{1.523404in}{1.767943in}}%
\pgfpathcurveto{\pgfqpoint{1.517580in}{1.762119in}}{\pgfqpoint{1.514308in}{1.754219in}}{\pgfqpoint{1.514308in}{1.745983in}}%
\pgfpathcurveto{\pgfqpoint{1.514308in}{1.737747in}}{\pgfqpoint{1.517580in}{1.729846in}}{\pgfqpoint{1.523404in}{1.724023in}}%
\pgfpathcurveto{\pgfqpoint{1.529228in}{1.718199in}}{\pgfqpoint{1.537128in}{1.714926in}}{\pgfqpoint{1.545364in}{1.714926in}}%
\pgfpathclose%
\pgfusepath{stroke,fill}%
\end{pgfscope}%
\begin{pgfscope}%
\pgfpathrectangle{\pgfqpoint{0.100000in}{0.212622in}}{\pgfqpoint{3.696000in}{3.696000in}}%
\pgfusepath{clip}%
\pgfsetbuttcap%
\pgfsetroundjoin%
\definecolor{currentfill}{rgb}{0.121569,0.466667,0.705882}%
\pgfsetfillcolor{currentfill}%
\pgfsetfillopacity{0.463451}%
\pgfsetlinewidth{1.003750pt}%
\definecolor{currentstroke}{rgb}{0.121569,0.466667,0.705882}%
\pgfsetstrokecolor{currentstroke}%
\pgfsetstrokeopacity{0.463451}%
\pgfsetdash{}{0pt}%
\pgfpathmoveto{\pgfqpoint{1.147498in}{1.643988in}}%
\pgfpathcurveto{\pgfqpoint{1.155735in}{1.643988in}}{\pgfqpoint{1.163635in}{1.647261in}}{\pgfqpoint{1.169459in}{1.653084in}}%
\pgfpathcurveto{\pgfqpoint{1.175283in}{1.658908in}}{\pgfqpoint{1.178555in}{1.666808in}}{\pgfqpoint{1.178555in}{1.675045in}}%
\pgfpathcurveto{\pgfqpoint{1.178555in}{1.683281in}}{\pgfqpoint{1.175283in}{1.691181in}}{\pgfqpoint{1.169459in}{1.697005in}}%
\pgfpathcurveto{\pgfqpoint{1.163635in}{1.702829in}}{\pgfqpoint{1.155735in}{1.706101in}}{\pgfqpoint{1.147498in}{1.706101in}}%
\pgfpathcurveto{\pgfqpoint{1.139262in}{1.706101in}}{\pgfqpoint{1.131362in}{1.702829in}}{\pgfqpoint{1.125538in}{1.697005in}}%
\pgfpathcurveto{\pgfqpoint{1.119714in}{1.691181in}}{\pgfqpoint{1.116442in}{1.683281in}}{\pgfqpoint{1.116442in}{1.675045in}}%
\pgfpathcurveto{\pgfqpoint{1.116442in}{1.666808in}}{\pgfqpoint{1.119714in}{1.658908in}}{\pgfqpoint{1.125538in}{1.653084in}}%
\pgfpathcurveto{\pgfqpoint{1.131362in}{1.647261in}}{\pgfqpoint{1.139262in}{1.643988in}}{\pgfqpoint{1.147498in}{1.643988in}}%
\pgfpathclose%
\pgfusepath{stroke,fill}%
\end{pgfscope}%
\begin{pgfscope}%
\pgfpathrectangle{\pgfqpoint{0.100000in}{0.212622in}}{\pgfqpoint{3.696000in}{3.696000in}}%
\pgfusepath{clip}%
\pgfsetbuttcap%
\pgfsetroundjoin%
\definecolor{currentfill}{rgb}{0.121569,0.466667,0.705882}%
\pgfsetfillcolor{currentfill}%
\pgfsetfillopacity{0.464728}%
\pgfsetlinewidth{1.003750pt}%
\definecolor{currentstroke}{rgb}{0.121569,0.466667,0.705882}%
\pgfsetstrokecolor{currentstroke}%
\pgfsetstrokeopacity{0.464728}%
\pgfsetdash{}{0pt}%
\pgfpathmoveto{\pgfqpoint{1.143055in}{1.640283in}}%
\pgfpathcurveto{\pgfqpoint{1.151291in}{1.640283in}}{\pgfqpoint{1.159191in}{1.643556in}}{\pgfqpoint{1.165015in}{1.649380in}}%
\pgfpathcurveto{\pgfqpoint{1.170839in}{1.655204in}}{\pgfqpoint{1.174111in}{1.663104in}}{\pgfqpoint{1.174111in}{1.671340in}}%
\pgfpathcurveto{\pgfqpoint{1.174111in}{1.679576in}}{\pgfqpoint{1.170839in}{1.687476in}}{\pgfqpoint{1.165015in}{1.693300in}}%
\pgfpathcurveto{\pgfqpoint{1.159191in}{1.699124in}}{\pgfqpoint{1.151291in}{1.702396in}}{\pgfqpoint{1.143055in}{1.702396in}}%
\pgfpathcurveto{\pgfqpoint{1.134819in}{1.702396in}}{\pgfqpoint{1.126918in}{1.699124in}}{\pgfqpoint{1.121095in}{1.693300in}}%
\pgfpathcurveto{\pgfqpoint{1.115271in}{1.687476in}}{\pgfqpoint{1.111998in}{1.679576in}}{\pgfqpoint{1.111998in}{1.671340in}}%
\pgfpathcurveto{\pgfqpoint{1.111998in}{1.663104in}}{\pgfqpoint{1.115271in}{1.655204in}}{\pgfqpoint{1.121095in}{1.649380in}}%
\pgfpathcurveto{\pgfqpoint{1.126918in}{1.643556in}}{\pgfqpoint{1.134819in}{1.640283in}}{\pgfqpoint{1.143055in}{1.640283in}}%
\pgfpathclose%
\pgfusepath{stroke,fill}%
\end{pgfscope}%
\begin{pgfscope}%
\pgfpathrectangle{\pgfqpoint{0.100000in}{0.212622in}}{\pgfqpoint{3.696000in}{3.696000in}}%
\pgfusepath{clip}%
\pgfsetbuttcap%
\pgfsetroundjoin%
\definecolor{currentfill}{rgb}{0.121569,0.466667,0.705882}%
\pgfsetfillcolor{currentfill}%
\pgfsetfillopacity{0.466288}%
\pgfsetlinewidth{1.003750pt}%
\definecolor{currentstroke}{rgb}{0.121569,0.466667,0.705882}%
\pgfsetstrokecolor{currentstroke}%
\pgfsetstrokeopacity{0.466288}%
\pgfsetdash{}{0pt}%
\pgfpathmoveto{\pgfqpoint{1.139188in}{1.637841in}}%
\pgfpathcurveto{\pgfqpoint{1.147424in}{1.637841in}}{\pgfqpoint{1.155324in}{1.641114in}}{\pgfqpoint{1.161148in}{1.646938in}}%
\pgfpathcurveto{\pgfqpoint{1.166972in}{1.652761in}}{\pgfqpoint{1.170244in}{1.660661in}}{\pgfqpoint{1.170244in}{1.668898in}}%
\pgfpathcurveto{\pgfqpoint{1.170244in}{1.677134in}}{\pgfqpoint{1.166972in}{1.685034in}}{\pgfqpoint{1.161148in}{1.690858in}}%
\pgfpathcurveto{\pgfqpoint{1.155324in}{1.696682in}}{\pgfqpoint{1.147424in}{1.699954in}}{\pgfqpoint{1.139188in}{1.699954in}}%
\pgfpathcurveto{\pgfqpoint{1.130951in}{1.699954in}}{\pgfqpoint{1.123051in}{1.696682in}}{\pgfqpoint{1.117227in}{1.690858in}}%
\pgfpathcurveto{\pgfqpoint{1.111403in}{1.685034in}}{\pgfqpoint{1.108131in}{1.677134in}}{\pgfqpoint{1.108131in}{1.668898in}}%
\pgfpathcurveto{\pgfqpoint{1.108131in}{1.660661in}}{\pgfqpoint{1.111403in}{1.652761in}}{\pgfqpoint{1.117227in}{1.646938in}}%
\pgfpathcurveto{\pgfqpoint{1.123051in}{1.641114in}}{\pgfqpoint{1.130951in}{1.637841in}}{\pgfqpoint{1.139188in}{1.637841in}}%
\pgfpathclose%
\pgfusepath{stroke,fill}%
\end{pgfscope}%
\begin{pgfscope}%
\pgfpathrectangle{\pgfqpoint{0.100000in}{0.212622in}}{\pgfqpoint{3.696000in}{3.696000in}}%
\pgfusepath{clip}%
\pgfsetbuttcap%
\pgfsetroundjoin%
\definecolor{currentfill}{rgb}{0.121569,0.466667,0.705882}%
\pgfsetfillcolor{currentfill}%
\pgfsetfillopacity{0.466662}%
\pgfsetlinewidth{1.003750pt}%
\definecolor{currentstroke}{rgb}{0.121569,0.466667,0.705882}%
\pgfsetstrokecolor{currentstroke}%
\pgfsetstrokeopacity{0.466662}%
\pgfsetdash{}{0pt}%
\pgfpathmoveto{\pgfqpoint{1.545903in}{1.717991in}}%
\pgfpathcurveto{\pgfqpoint{1.554140in}{1.717991in}}{\pgfqpoint{1.562040in}{1.721264in}}{\pgfqpoint{1.567864in}{1.727087in}}%
\pgfpathcurveto{\pgfqpoint{1.573688in}{1.732911in}}{\pgfqpoint{1.576960in}{1.740811in}}{\pgfqpoint{1.576960in}{1.749048in}}%
\pgfpathcurveto{\pgfqpoint{1.576960in}{1.757284in}}{\pgfqpoint{1.573688in}{1.765184in}}{\pgfqpoint{1.567864in}{1.771008in}}%
\pgfpathcurveto{\pgfqpoint{1.562040in}{1.776832in}}{\pgfqpoint{1.554140in}{1.780104in}}{\pgfqpoint{1.545903in}{1.780104in}}%
\pgfpathcurveto{\pgfqpoint{1.537667in}{1.780104in}}{\pgfqpoint{1.529767in}{1.776832in}}{\pgfqpoint{1.523943in}{1.771008in}}%
\pgfpathcurveto{\pgfqpoint{1.518119in}{1.765184in}}{\pgfqpoint{1.514847in}{1.757284in}}{\pgfqpoint{1.514847in}{1.749048in}}%
\pgfpathcurveto{\pgfqpoint{1.514847in}{1.740811in}}{\pgfqpoint{1.518119in}{1.732911in}}{\pgfqpoint{1.523943in}{1.727087in}}%
\pgfpathcurveto{\pgfqpoint{1.529767in}{1.721264in}}{\pgfqpoint{1.537667in}{1.717991in}}{\pgfqpoint{1.545903in}{1.717991in}}%
\pgfpathclose%
\pgfusepath{stroke,fill}%
\end{pgfscope}%
\begin{pgfscope}%
\pgfpathrectangle{\pgfqpoint{0.100000in}{0.212622in}}{\pgfqpoint{3.696000in}{3.696000in}}%
\pgfusepath{clip}%
\pgfsetbuttcap%
\pgfsetroundjoin%
\definecolor{currentfill}{rgb}{0.121569,0.466667,0.705882}%
\pgfsetfillcolor{currentfill}%
\pgfsetfillopacity{0.468129}%
\pgfsetlinewidth{1.003750pt}%
\definecolor{currentstroke}{rgb}{0.121569,0.466667,0.705882}%
\pgfsetstrokecolor{currentstroke}%
\pgfsetstrokeopacity{0.468129}%
\pgfsetdash{}{0pt}%
\pgfpathmoveto{\pgfqpoint{1.546798in}{1.717964in}}%
\pgfpathcurveto{\pgfqpoint{1.555035in}{1.717964in}}{\pgfqpoint{1.562935in}{1.721237in}}{\pgfqpoint{1.568759in}{1.727061in}}%
\pgfpathcurveto{\pgfqpoint{1.574582in}{1.732885in}}{\pgfqpoint{1.577855in}{1.740785in}}{\pgfqpoint{1.577855in}{1.749021in}}%
\pgfpathcurveto{\pgfqpoint{1.577855in}{1.757257in}}{\pgfqpoint{1.574582in}{1.765157in}}{\pgfqpoint{1.568759in}{1.770981in}}%
\pgfpathcurveto{\pgfqpoint{1.562935in}{1.776805in}}{\pgfqpoint{1.555035in}{1.780077in}}{\pgfqpoint{1.546798in}{1.780077in}}%
\pgfpathcurveto{\pgfqpoint{1.538562in}{1.780077in}}{\pgfqpoint{1.530662in}{1.776805in}}{\pgfqpoint{1.524838in}{1.770981in}}%
\pgfpathcurveto{\pgfqpoint{1.519014in}{1.765157in}}{\pgfqpoint{1.515742in}{1.757257in}}{\pgfqpoint{1.515742in}{1.749021in}}%
\pgfpathcurveto{\pgfqpoint{1.515742in}{1.740785in}}{\pgfqpoint{1.519014in}{1.732885in}}{\pgfqpoint{1.524838in}{1.727061in}}%
\pgfpathcurveto{\pgfqpoint{1.530662in}{1.721237in}}{\pgfqpoint{1.538562in}{1.717964in}}{\pgfqpoint{1.546798in}{1.717964in}}%
\pgfpathclose%
\pgfusepath{stroke,fill}%
\end{pgfscope}%
\begin{pgfscope}%
\pgfpathrectangle{\pgfqpoint{0.100000in}{0.212622in}}{\pgfqpoint{3.696000in}{3.696000in}}%
\pgfusepath{clip}%
\pgfsetbuttcap%
\pgfsetroundjoin%
\definecolor{currentfill}{rgb}{0.121569,0.466667,0.705882}%
\pgfsetfillcolor{currentfill}%
\pgfsetfillopacity{0.468155}%
\pgfsetlinewidth{1.003750pt}%
\definecolor{currentstroke}{rgb}{0.121569,0.466667,0.705882}%
\pgfsetstrokecolor{currentstroke}%
\pgfsetstrokeopacity{0.468155}%
\pgfsetdash{}{0pt}%
\pgfpathmoveto{\pgfqpoint{1.135957in}{1.637098in}}%
\pgfpathcurveto{\pgfqpoint{1.144193in}{1.637098in}}{\pgfqpoint{1.152093in}{1.640370in}}{\pgfqpoint{1.157917in}{1.646194in}}%
\pgfpathcurveto{\pgfqpoint{1.163741in}{1.652018in}}{\pgfqpoint{1.167013in}{1.659918in}}{\pgfqpoint{1.167013in}{1.668154in}}%
\pgfpathcurveto{\pgfqpoint{1.167013in}{1.676390in}}{\pgfqpoint{1.163741in}{1.684290in}}{\pgfqpoint{1.157917in}{1.690114in}}%
\pgfpathcurveto{\pgfqpoint{1.152093in}{1.695938in}}{\pgfqpoint{1.144193in}{1.699211in}}{\pgfqpoint{1.135957in}{1.699211in}}%
\pgfpathcurveto{\pgfqpoint{1.127720in}{1.699211in}}{\pgfqpoint{1.119820in}{1.695938in}}{\pgfqpoint{1.113996in}{1.690114in}}%
\pgfpathcurveto{\pgfqpoint{1.108173in}{1.684290in}}{\pgfqpoint{1.104900in}{1.676390in}}{\pgfqpoint{1.104900in}{1.668154in}}%
\pgfpathcurveto{\pgfqpoint{1.104900in}{1.659918in}}{\pgfqpoint{1.108173in}{1.652018in}}{\pgfqpoint{1.113996in}{1.646194in}}%
\pgfpathcurveto{\pgfqpoint{1.119820in}{1.640370in}}{\pgfqpoint{1.127720in}{1.637098in}}{\pgfqpoint{1.135957in}{1.637098in}}%
\pgfpathclose%
\pgfusepath{stroke,fill}%
\end{pgfscope}%
\begin{pgfscope}%
\pgfpathrectangle{\pgfqpoint{0.100000in}{0.212622in}}{\pgfqpoint{3.696000in}{3.696000in}}%
\pgfusepath{clip}%
\pgfsetbuttcap%
\pgfsetroundjoin%
\definecolor{currentfill}{rgb}{0.121569,0.466667,0.705882}%
\pgfsetfillcolor{currentfill}%
\pgfsetfillopacity{0.468883}%
\pgfsetlinewidth{1.003750pt}%
\definecolor{currentstroke}{rgb}{0.121569,0.466667,0.705882}%
\pgfsetstrokecolor{currentstroke}%
\pgfsetstrokeopacity{0.468883}%
\pgfsetdash{}{0pt}%
\pgfpathmoveto{\pgfqpoint{1.133115in}{1.633700in}}%
\pgfpathcurveto{\pgfqpoint{1.141351in}{1.633700in}}{\pgfqpoint{1.149251in}{1.636972in}}{\pgfqpoint{1.155075in}{1.642796in}}%
\pgfpathcurveto{\pgfqpoint{1.160899in}{1.648620in}}{\pgfqpoint{1.164171in}{1.656520in}}{\pgfqpoint{1.164171in}{1.664756in}}%
\pgfpathcurveto{\pgfqpoint{1.164171in}{1.672993in}}{\pgfqpoint{1.160899in}{1.680893in}}{\pgfqpoint{1.155075in}{1.686717in}}%
\pgfpathcurveto{\pgfqpoint{1.149251in}{1.692540in}}{\pgfqpoint{1.141351in}{1.695813in}}{\pgfqpoint{1.133115in}{1.695813in}}%
\pgfpathcurveto{\pgfqpoint{1.124878in}{1.695813in}}{\pgfqpoint{1.116978in}{1.692540in}}{\pgfqpoint{1.111154in}{1.686717in}}%
\pgfpathcurveto{\pgfqpoint{1.105330in}{1.680893in}}{\pgfqpoint{1.102058in}{1.672993in}}{\pgfqpoint{1.102058in}{1.664756in}}%
\pgfpathcurveto{\pgfqpoint{1.102058in}{1.656520in}}{\pgfqpoint{1.105330in}{1.648620in}}{\pgfqpoint{1.111154in}{1.642796in}}%
\pgfpathcurveto{\pgfqpoint{1.116978in}{1.636972in}}{\pgfqpoint{1.124878in}{1.633700in}}{\pgfqpoint{1.133115in}{1.633700in}}%
\pgfpathclose%
\pgfusepath{stroke,fill}%
\end{pgfscope}%
\begin{pgfscope}%
\pgfpathrectangle{\pgfqpoint{0.100000in}{0.212622in}}{\pgfqpoint{3.696000in}{3.696000in}}%
\pgfusepath{clip}%
\pgfsetbuttcap%
\pgfsetroundjoin%
\definecolor{currentfill}{rgb}{0.121569,0.466667,0.705882}%
\pgfsetfillcolor{currentfill}%
\pgfsetfillopacity{0.470321}%
\pgfsetlinewidth{1.003750pt}%
\definecolor{currentstroke}{rgb}{0.121569,0.466667,0.705882}%
\pgfsetstrokecolor{currentstroke}%
\pgfsetstrokeopacity{0.470321}%
\pgfsetdash{}{0pt}%
\pgfpathmoveto{\pgfqpoint{1.547478in}{1.718636in}}%
\pgfpathcurveto{\pgfqpoint{1.555715in}{1.718636in}}{\pgfqpoint{1.563615in}{1.721909in}}{\pgfqpoint{1.569439in}{1.727733in}}%
\pgfpathcurveto{\pgfqpoint{1.575263in}{1.733556in}}{\pgfqpoint{1.578535in}{1.741457in}}{\pgfqpoint{1.578535in}{1.749693in}}%
\pgfpathcurveto{\pgfqpoint{1.578535in}{1.757929in}}{\pgfqpoint{1.575263in}{1.765829in}}{\pgfqpoint{1.569439in}{1.771653in}}%
\pgfpathcurveto{\pgfqpoint{1.563615in}{1.777477in}}{\pgfqpoint{1.555715in}{1.780749in}}{\pgfqpoint{1.547478in}{1.780749in}}%
\pgfpathcurveto{\pgfqpoint{1.539242in}{1.780749in}}{\pgfqpoint{1.531342in}{1.777477in}}{\pgfqpoint{1.525518in}{1.771653in}}%
\pgfpathcurveto{\pgfqpoint{1.519694in}{1.765829in}}{\pgfqpoint{1.516422in}{1.757929in}}{\pgfqpoint{1.516422in}{1.749693in}}%
\pgfpathcurveto{\pgfqpoint{1.516422in}{1.741457in}}{\pgfqpoint{1.519694in}{1.733556in}}{\pgfqpoint{1.525518in}{1.727733in}}%
\pgfpathcurveto{\pgfqpoint{1.531342in}{1.721909in}}{\pgfqpoint{1.539242in}{1.718636in}}{\pgfqpoint{1.547478in}{1.718636in}}%
\pgfpathclose%
\pgfusepath{stroke,fill}%
\end{pgfscope}%
\begin{pgfscope}%
\pgfpathrectangle{\pgfqpoint{0.100000in}{0.212622in}}{\pgfqpoint{3.696000in}{3.696000in}}%
\pgfusepath{clip}%
\pgfsetbuttcap%
\pgfsetroundjoin%
\definecolor{currentfill}{rgb}{0.121569,0.466667,0.705882}%
\pgfsetfillcolor{currentfill}%
\pgfsetfillopacity{0.471084}%
\pgfsetlinewidth{1.003750pt}%
\definecolor{currentstroke}{rgb}{0.121569,0.466667,0.705882}%
\pgfsetstrokecolor{currentstroke}%
\pgfsetstrokeopacity{0.471084}%
\pgfsetdash{}{0pt}%
\pgfpathmoveto{\pgfqpoint{1.128209in}{1.631048in}}%
\pgfpathcurveto{\pgfqpoint{1.136445in}{1.631048in}}{\pgfqpoint{1.144345in}{1.634321in}}{\pgfqpoint{1.150169in}{1.640145in}}%
\pgfpathcurveto{\pgfqpoint{1.155993in}{1.645968in}}{\pgfqpoint{1.159266in}{1.653869in}}{\pgfqpoint{1.159266in}{1.662105in}}%
\pgfpathcurveto{\pgfqpoint{1.159266in}{1.670341in}}{\pgfqpoint{1.155993in}{1.678241in}}{\pgfqpoint{1.150169in}{1.684065in}}%
\pgfpathcurveto{\pgfqpoint{1.144345in}{1.689889in}}{\pgfqpoint{1.136445in}{1.693161in}}{\pgfqpoint{1.128209in}{1.693161in}}%
\pgfpathcurveto{\pgfqpoint{1.119973in}{1.693161in}}{\pgfqpoint{1.112073in}{1.689889in}}{\pgfqpoint{1.106249in}{1.684065in}}%
\pgfpathcurveto{\pgfqpoint{1.100425in}{1.678241in}}{\pgfqpoint{1.097153in}{1.670341in}}{\pgfqpoint{1.097153in}{1.662105in}}%
\pgfpathcurveto{\pgfqpoint{1.097153in}{1.653869in}}{\pgfqpoint{1.100425in}{1.645968in}}{\pgfqpoint{1.106249in}{1.640145in}}%
\pgfpathcurveto{\pgfqpoint{1.112073in}{1.634321in}}{\pgfqpoint{1.119973in}{1.631048in}}{\pgfqpoint{1.128209in}{1.631048in}}%
\pgfpathclose%
\pgfusepath{stroke,fill}%
\end{pgfscope}%
\begin{pgfscope}%
\pgfpathrectangle{\pgfqpoint{0.100000in}{0.212622in}}{\pgfqpoint{3.696000in}{3.696000in}}%
\pgfusepath{clip}%
\pgfsetbuttcap%
\pgfsetroundjoin%
\definecolor{currentfill}{rgb}{0.121569,0.466667,0.705882}%
\pgfsetfillcolor{currentfill}%
\pgfsetfillopacity{0.471370}%
\pgfsetlinewidth{1.003750pt}%
\definecolor{currentstroke}{rgb}{0.121569,0.466667,0.705882}%
\pgfsetstrokecolor{currentstroke}%
\pgfsetstrokeopacity{0.471370}%
\pgfsetdash{}{0pt}%
\pgfpathmoveto{\pgfqpoint{1.547743in}{1.718234in}}%
\pgfpathcurveto{\pgfqpoint{1.555979in}{1.718234in}}{\pgfqpoint{1.563879in}{1.721506in}}{\pgfqpoint{1.569703in}{1.727330in}}%
\pgfpathcurveto{\pgfqpoint{1.575527in}{1.733154in}}{\pgfqpoint{1.578799in}{1.741054in}}{\pgfqpoint{1.578799in}{1.749290in}}%
\pgfpathcurveto{\pgfqpoint{1.578799in}{1.757526in}}{\pgfqpoint{1.575527in}{1.765426in}}{\pgfqpoint{1.569703in}{1.771250in}}%
\pgfpathcurveto{\pgfqpoint{1.563879in}{1.777074in}}{\pgfqpoint{1.555979in}{1.780347in}}{\pgfqpoint{1.547743in}{1.780347in}}%
\pgfpathcurveto{\pgfqpoint{1.539507in}{1.780347in}}{\pgfqpoint{1.531607in}{1.777074in}}{\pgfqpoint{1.525783in}{1.771250in}}%
\pgfpathcurveto{\pgfqpoint{1.519959in}{1.765426in}}{\pgfqpoint{1.516686in}{1.757526in}}{\pgfqpoint{1.516686in}{1.749290in}}%
\pgfpathcurveto{\pgfqpoint{1.516686in}{1.741054in}}{\pgfqpoint{1.519959in}{1.733154in}}{\pgfqpoint{1.525783in}{1.727330in}}%
\pgfpathcurveto{\pgfqpoint{1.531607in}{1.721506in}}{\pgfqpoint{1.539507in}{1.718234in}}{\pgfqpoint{1.547743in}{1.718234in}}%
\pgfpathclose%
\pgfusepath{stroke,fill}%
\end{pgfscope}%
\begin{pgfscope}%
\pgfpathrectangle{\pgfqpoint{0.100000in}{0.212622in}}{\pgfqpoint{3.696000in}{3.696000in}}%
\pgfusepath{clip}%
\pgfsetbuttcap%
\pgfsetroundjoin%
\definecolor{currentfill}{rgb}{0.121569,0.466667,0.705882}%
\pgfsetfillcolor{currentfill}%
\pgfsetfillopacity{0.472666}%
\pgfsetlinewidth{1.003750pt}%
\definecolor{currentstroke}{rgb}{0.121569,0.466667,0.705882}%
\pgfsetstrokecolor{currentstroke}%
\pgfsetstrokeopacity{0.472666}%
\pgfsetdash{}{0pt}%
\pgfpathmoveto{\pgfqpoint{1.548190in}{1.718356in}}%
\pgfpathcurveto{\pgfqpoint{1.556426in}{1.718356in}}{\pgfqpoint{1.564326in}{1.721628in}}{\pgfqpoint{1.570150in}{1.727452in}}%
\pgfpathcurveto{\pgfqpoint{1.575974in}{1.733276in}}{\pgfqpoint{1.579246in}{1.741176in}}{\pgfqpoint{1.579246in}{1.749412in}}%
\pgfpathcurveto{\pgfqpoint{1.579246in}{1.757648in}}{\pgfqpoint{1.575974in}{1.765549in}}{\pgfqpoint{1.570150in}{1.771372in}}%
\pgfpathcurveto{\pgfqpoint{1.564326in}{1.777196in}}{\pgfqpoint{1.556426in}{1.780469in}}{\pgfqpoint{1.548190in}{1.780469in}}%
\pgfpathcurveto{\pgfqpoint{1.539954in}{1.780469in}}{\pgfqpoint{1.532054in}{1.777196in}}{\pgfqpoint{1.526230in}{1.771372in}}%
\pgfpathcurveto{\pgfqpoint{1.520406in}{1.765549in}}{\pgfqpoint{1.517133in}{1.757648in}}{\pgfqpoint{1.517133in}{1.749412in}}%
\pgfpathcurveto{\pgfqpoint{1.517133in}{1.741176in}}{\pgfqpoint{1.520406in}{1.733276in}}{\pgfqpoint{1.526230in}{1.727452in}}%
\pgfpathcurveto{\pgfqpoint{1.532054in}{1.721628in}}{\pgfqpoint{1.539954in}{1.718356in}}{\pgfqpoint{1.548190in}{1.718356in}}%
\pgfpathclose%
\pgfusepath{stroke,fill}%
\end{pgfscope}%
\begin{pgfscope}%
\pgfpathrectangle{\pgfqpoint{0.100000in}{0.212622in}}{\pgfqpoint{3.696000in}{3.696000in}}%
\pgfusepath{clip}%
\pgfsetbuttcap%
\pgfsetroundjoin%
\definecolor{currentfill}{rgb}{0.121569,0.466667,0.705882}%
\pgfsetfillcolor{currentfill}%
\pgfsetfillopacity{0.473494}%
\pgfsetlinewidth{1.003750pt}%
\definecolor{currentstroke}{rgb}{0.121569,0.466667,0.705882}%
\pgfsetstrokecolor{currentstroke}%
\pgfsetstrokeopacity{0.473494}%
\pgfsetdash{}{0pt}%
\pgfpathmoveto{\pgfqpoint{1.548567in}{1.719025in}}%
\pgfpathcurveto{\pgfqpoint{1.556804in}{1.719025in}}{\pgfqpoint{1.564704in}{1.722297in}}{\pgfqpoint{1.570528in}{1.728121in}}%
\pgfpathcurveto{\pgfqpoint{1.576352in}{1.733945in}}{\pgfqpoint{1.579624in}{1.741845in}}{\pgfqpoint{1.579624in}{1.750081in}}%
\pgfpathcurveto{\pgfqpoint{1.579624in}{1.758317in}}{\pgfqpoint{1.576352in}{1.766217in}}{\pgfqpoint{1.570528in}{1.772041in}}%
\pgfpathcurveto{\pgfqpoint{1.564704in}{1.777865in}}{\pgfqpoint{1.556804in}{1.781138in}}{\pgfqpoint{1.548567in}{1.781138in}}%
\pgfpathcurveto{\pgfqpoint{1.540331in}{1.781138in}}{\pgfqpoint{1.532431in}{1.777865in}}{\pgfqpoint{1.526607in}{1.772041in}}%
\pgfpathcurveto{\pgfqpoint{1.520783in}{1.766217in}}{\pgfqpoint{1.517511in}{1.758317in}}{\pgfqpoint{1.517511in}{1.750081in}}%
\pgfpathcurveto{\pgfqpoint{1.517511in}{1.741845in}}{\pgfqpoint{1.520783in}{1.733945in}}{\pgfqpoint{1.526607in}{1.728121in}}%
\pgfpathcurveto{\pgfqpoint{1.532431in}{1.722297in}}{\pgfqpoint{1.540331in}{1.719025in}}{\pgfqpoint{1.548567in}{1.719025in}}%
\pgfpathclose%
\pgfusepath{stroke,fill}%
\end{pgfscope}%
\begin{pgfscope}%
\pgfpathrectangle{\pgfqpoint{0.100000in}{0.212622in}}{\pgfqpoint{3.696000in}{3.696000in}}%
\pgfusepath{clip}%
\pgfsetbuttcap%
\pgfsetroundjoin%
\definecolor{currentfill}{rgb}{0.121569,0.466667,0.705882}%
\pgfsetfillcolor{currentfill}%
\pgfsetfillopacity{0.473730}%
\pgfsetlinewidth{1.003750pt}%
\definecolor{currentstroke}{rgb}{0.121569,0.466667,0.705882}%
\pgfsetstrokecolor{currentstroke}%
\pgfsetstrokeopacity{0.473730}%
\pgfsetdash{}{0pt}%
\pgfpathmoveto{\pgfqpoint{1.124400in}{1.631323in}}%
\pgfpathcurveto{\pgfqpoint{1.132636in}{1.631323in}}{\pgfqpoint{1.140536in}{1.634596in}}{\pgfqpoint{1.146360in}{1.640420in}}%
\pgfpathcurveto{\pgfqpoint{1.152184in}{1.646244in}}{\pgfqpoint{1.155456in}{1.654144in}}{\pgfqpoint{1.155456in}{1.662380in}}%
\pgfpathcurveto{\pgfqpoint{1.155456in}{1.670616in}}{\pgfqpoint{1.152184in}{1.678516in}}{\pgfqpoint{1.146360in}{1.684340in}}%
\pgfpathcurveto{\pgfqpoint{1.140536in}{1.690164in}}{\pgfqpoint{1.132636in}{1.693436in}}{\pgfqpoint{1.124400in}{1.693436in}}%
\pgfpathcurveto{\pgfqpoint{1.116164in}{1.693436in}}{\pgfqpoint{1.108264in}{1.690164in}}{\pgfqpoint{1.102440in}{1.684340in}}%
\pgfpathcurveto{\pgfqpoint{1.096616in}{1.678516in}}{\pgfqpoint{1.093343in}{1.670616in}}{\pgfqpoint{1.093343in}{1.662380in}}%
\pgfpathcurveto{\pgfqpoint{1.093343in}{1.654144in}}{\pgfqpoint{1.096616in}{1.646244in}}{\pgfqpoint{1.102440in}{1.640420in}}%
\pgfpathcurveto{\pgfqpoint{1.108264in}{1.634596in}}{\pgfqpoint{1.116164in}{1.631323in}}{\pgfqpoint{1.124400in}{1.631323in}}%
\pgfpathclose%
\pgfusepath{stroke,fill}%
\end{pgfscope}%
\begin{pgfscope}%
\pgfpathrectangle{\pgfqpoint{0.100000in}{0.212622in}}{\pgfqpoint{3.696000in}{3.696000in}}%
\pgfusepath{clip}%
\pgfsetbuttcap%
\pgfsetroundjoin%
\definecolor{currentfill}{rgb}{0.121569,0.466667,0.705882}%
\pgfsetfillcolor{currentfill}%
\pgfsetfillopacity{0.474399}%
\pgfsetlinewidth{1.003750pt}%
\definecolor{currentstroke}{rgb}{0.121569,0.466667,0.705882}%
\pgfsetstrokecolor{currentstroke}%
\pgfsetstrokeopacity{0.474399}%
\pgfsetdash{}{0pt}%
\pgfpathmoveto{\pgfqpoint{1.548515in}{1.718964in}}%
\pgfpathcurveto{\pgfqpoint{1.556752in}{1.718964in}}{\pgfqpoint{1.564652in}{1.722236in}}{\pgfqpoint{1.570476in}{1.728060in}}%
\pgfpathcurveto{\pgfqpoint{1.576300in}{1.733884in}}{\pgfqpoint{1.579572in}{1.741784in}}{\pgfqpoint{1.579572in}{1.750020in}}%
\pgfpathcurveto{\pgfqpoint{1.579572in}{1.758257in}}{\pgfqpoint{1.576300in}{1.766157in}}{\pgfqpoint{1.570476in}{1.771981in}}%
\pgfpathcurveto{\pgfqpoint{1.564652in}{1.777805in}}{\pgfqpoint{1.556752in}{1.781077in}}{\pgfqpoint{1.548515in}{1.781077in}}%
\pgfpathcurveto{\pgfqpoint{1.540279in}{1.781077in}}{\pgfqpoint{1.532379in}{1.777805in}}{\pgfqpoint{1.526555in}{1.771981in}}%
\pgfpathcurveto{\pgfqpoint{1.520731in}{1.766157in}}{\pgfqpoint{1.517459in}{1.758257in}}{\pgfqpoint{1.517459in}{1.750020in}}%
\pgfpathcurveto{\pgfqpoint{1.517459in}{1.741784in}}{\pgfqpoint{1.520731in}{1.733884in}}{\pgfqpoint{1.526555in}{1.728060in}}%
\pgfpathcurveto{\pgfqpoint{1.532379in}{1.722236in}}{\pgfqpoint{1.540279in}{1.718964in}}{\pgfqpoint{1.548515in}{1.718964in}}%
\pgfpathclose%
\pgfusepath{stroke,fill}%
\end{pgfscope}%
\begin{pgfscope}%
\pgfpathrectangle{\pgfqpoint{0.100000in}{0.212622in}}{\pgfqpoint{3.696000in}{3.696000in}}%
\pgfusepath{clip}%
\pgfsetbuttcap%
\pgfsetroundjoin%
\definecolor{currentfill}{rgb}{0.121569,0.466667,0.705882}%
\pgfsetfillcolor{currentfill}%
\pgfsetfillopacity{0.474813}%
\pgfsetlinewidth{1.003750pt}%
\definecolor{currentstroke}{rgb}{0.121569,0.466667,0.705882}%
\pgfsetstrokecolor{currentstroke}%
\pgfsetstrokeopacity{0.474813}%
\pgfsetdash{}{0pt}%
\pgfpathmoveto{\pgfqpoint{1.120888in}{1.628599in}}%
\pgfpathcurveto{\pgfqpoint{1.129125in}{1.628599in}}{\pgfqpoint{1.137025in}{1.631872in}}{\pgfqpoint{1.142849in}{1.637696in}}%
\pgfpathcurveto{\pgfqpoint{1.148673in}{1.643520in}}{\pgfqpoint{1.151945in}{1.651420in}}{\pgfqpoint{1.151945in}{1.659656in}}%
\pgfpathcurveto{\pgfqpoint{1.151945in}{1.667892in}}{\pgfqpoint{1.148673in}{1.675792in}}{\pgfqpoint{1.142849in}{1.681616in}}%
\pgfpathcurveto{\pgfqpoint{1.137025in}{1.687440in}}{\pgfqpoint{1.129125in}{1.690712in}}{\pgfqpoint{1.120888in}{1.690712in}}%
\pgfpathcurveto{\pgfqpoint{1.112652in}{1.690712in}}{\pgfqpoint{1.104752in}{1.687440in}}{\pgfqpoint{1.098928in}{1.681616in}}%
\pgfpathcurveto{\pgfqpoint{1.093104in}{1.675792in}}{\pgfqpoint{1.089832in}{1.667892in}}{\pgfqpoint{1.089832in}{1.659656in}}%
\pgfpathcurveto{\pgfqpoint{1.089832in}{1.651420in}}{\pgfqpoint{1.093104in}{1.643520in}}{\pgfqpoint{1.098928in}{1.637696in}}%
\pgfpathcurveto{\pgfqpoint{1.104752in}{1.631872in}}{\pgfqpoint{1.112652in}{1.628599in}}{\pgfqpoint{1.120888in}{1.628599in}}%
\pgfpathclose%
\pgfusepath{stroke,fill}%
\end{pgfscope}%
\begin{pgfscope}%
\pgfpathrectangle{\pgfqpoint{0.100000in}{0.212622in}}{\pgfqpoint{3.696000in}{3.696000in}}%
\pgfusepath{clip}%
\pgfsetbuttcap%
\pgfsetroundjoin%
\definecolor{currentfill}{rgb}{0.121569,0.466667,0.705882}%
\pgfsetfillcolor{currentfill}%
\pgfsetfillopacity{0.475647}%
\pgfsetlinewidth{1.003750pt}%
\definecolor{currentstroke}{rgb}{0.121569,0.466667,0.705882}%
\pgfsetstrokecolor{currentstroke}%
\pgfsetstrokeopacity{0.475647}%
\pgfsetdash{}{0pt}%
\pgfpathmoveto{\pgfqpoint{1.549329in}{1.717450in}}%
\pgfpathcurveto{\pgfqpoint{1.557565in}{1.717450in}}{\pgfqpoint{1.565465in}{1.720723in}}{\pgfqpoint{1.571289in}{1.726547in}}%
\pgfpathcurveto{\pgfqpoint{1.577113in}{1.732371in}}{\pgfqpoint{1.580385in}{1.740271in}}{\pgfqpoint{1.580385in}{1.748507in}}%
\pgfpathcurveto{\pgfqpoint{1.580385in}{1.756743in}}{\pgfqpoint{1.577113in}{1.764643in}}{\pgfqpoint{1.571289in}{1.770467in}}%
\pgfpathcurveto{\pgfqpoint{1.565465in}{1.776291in}}{\pgfqpoint{1.557565in}{1.779563in}}{\pgfqpoint{1.549329in}{1.779563in}}%
\pgfpathcurveto{\pgfqpoint{1.541093in}{1.779563in}}{\pgfqpoint{1.533192in}{1.776291in}}{\pgfqpoint{1.527369in}{1.770467in}}%
\pgfpathcurveto{\pgfqpoint{1.521545in}{1.764643in}}{\pgfqpoint{1.518272in}{1.756743in}}{\pgfqpoint{1.518272in}{1.748507in}}%
\pgfpathcurveto{\pgfqpoint{1.518272in}{1.740271in}}{\pgfqpoint{1.521545in}{1.732371in}}{\pgfqpoint{1.527369in}{1.726547in}}%
\pgfpathcurveto{\pgfqpoint{1.533192in}{1.720723in}}{\pgfqpoint{1.541093in}{1.717450in}}{\pgfqpoint{1.549329in}{1.717450in}}%
\pgfpathclose%
\pgfusepath{stroke,fill}%
\end{pgfscope}%
\begin{pgfscope}%
\pgfpathrectangle{\pgfqpoint{0.100000in}{0.212622in}}{\pgfqpoint{3.696000in}{3.696000in}}%
\pgfusepath{clip}%
\pgfsetbuttcap%
\pgfsetroundjoin%
\definecolor{currentfill}{rgb}{0.121569,0.466667,0.705882}%
\pgfsetfillcolor{currentfill}%
\pgfsetfillopacity{0.476075}%
\pgfsetlinewidth{1.003750pt}%
\definecolor{currentstroke}{rgb}{0.121569,0.466667,0.705882}%
\pgfsetstrokecolor{currentstroke}%
\pgfsetstrokeopacity{0.476075}%
\pgfsetdash{}{0pt}%
\pgfpathmoveto{\pgfqpoint{1.118096in}{1.626471in}}%
\pgfpathcurveto{\pgfqpoint{1.126332in}{1.626471in}}{\pgfqpoint{1.134232in}{1.629743in}}{\pgfqpoint{1.140056in}{1.635567in}}%
\pgfpathcurveto{\pgfqpoint{1.145880in}{1.641391in}}{\pgfqpoint{1.149152in}{1.649291in}}{\pgfqpoint{1.149152in}{1.657527in}}%
\pgfpathcurveto{\pgfqpoint{1.149152in}{1.665764in}}{\pgfqpoint{1.145880in}{1.673664in}}{\pgfqpoint{1.140056in}{1.679488in}}%
\pgfpathcurveto{\pgfqpoint{1.134232in}{1.685312in}}{\pgfqpoint{1.126332in}{1.688584in}}{\pgfqpoint{1.118096in}{1.688584in}}%
\pgfpathcurveto{\pgfqpoint{1.109859in}{1.688584in}}{\pgfqpoint{1.101959in}{1.685312in}}{\pgfqpoint{1.096135in}{1.679488in}}%
\pgfpathcurveto{\pgfqpoint{1.090311in}{1.673664in}}{\pgfqpoint{1.087039in}{1.665764in}}{\pgfqpoint{1.087039in}{1.657527in}}%
\pgfpathcurveto{\pgfqpoint{1.087039in}{1.649291in}}{\pgfqpoint{1.090311in}{1.641391in}}{\pgfqpoint{1.096135in}{1.635567in}}%
\pgfpathcurveto{\pgfqpoint{1.101959in}{1.629743in}}{\pgfqpoint{1.109859in}{1.626471in}}{\pgfqpoint{1.118096in}{1.626471in}}%
\pgfpathclose%
\pgfusepath{stroke,fill}%
\end{pgfscope}%
\begin{pgfscope}%
\pgfpathrectangle{\pgfqpoint{0.100000in}{0.212622in}}{\pgfqpoint{3.696000in}{3.696000in}}%
\pgfusepath{clip}%
\pgfsetbuttcap%
\pgfsetroundjoin%
\definecolor{currentfill}{rgb}{0.121569,0.466667,0.705882}%
\pgfsetfillcolor{currentfill}%
\pgfsetfillopacity{0.476630}%
\pgfsetlinewidth{1.003750pt}%
\definecolor{currentstroke}{rgb}{0.121569,0.466667,0.705882}%
\pgfsetstrokecolor{currentstroke}%
\pgfsetstrokeopacity{0.476630}%
\pgfsetdash{}{0pt}%
\pgfpathmoveto{\pgfqpoint{1.549744in}{1.717992in}}%
\pgfpathcurveto{\pgfqpoint{1.557980in}{1.717992in}}{\pgfqpoint{1.565880in}{1.721264in}}{\pgfqpoint{1.571704in}{1.727088in}}%
\pgfpathcurveto{\pgfqpoint{1.577528in}{1.732912in}}{\pgfqpoint{1.580801in}{1.740812in}}{\pgfqpoint{1.580801in}{1.749048in}}%
\pgfpathcurveto{\pgfqpoint{1.580801in}{1.757285in}}{\pgfqpoint{1.577528in}{1.765185in}}{\pgfqpoint{1.571704in}{1.771009in}}%
\pgfpathcurveto{\pgfqpoint{1.565880in}{1.776833in}}{\pgfqpoint{1.557980in}{1.780105in}}{\pgfqpoint{1.549744in}{1.780105in}}%
\pgfpathcurveto{\pgfqpoint{1.541508in}{1.780105in}}{\pgfqpoint{1.533608in}{1.776833in}}{\pgfqpoint{1.527784in}{1.771009in}}%
\pgfpathcurveto{\pgfqpoint{1.521960in}{1.765185in}}{\pgfqpoint{1.518688in}{1.757285in}}{\pgfqpoint{1.518688in}{1.749048in}}%
\pgfpathcurveto{\pgfqpoint{1.518688in}{1.740812in}}{\pgfqpoint{1.521960in}{1.732912in}}{\pgfqpoint{1.527784in}{1.727088in}}%
\pgfpathcurveto{\pgfqpoint{1.533608in}{1.721264in}}{\pgfqpoint{1.541508in}{1.717992in}}{\pgfqpoint{1.549744in}{1.717992in}}%
\pgfpathclose%
\pgfusepath{stroke,fill}%
\end{pgfscope}%
\begin{pgfscope}%
\pgfpathrectangle{\pgfqpoint{0.100000in}{0.212622in}}{\pgfqpoint{3.696000in}{3.696000in}}%
\pgfusepath{clip}%
\pgfsetbuttcap%
\pgfsetroundjoin%
\definecolor{currentfill}{rgb}{0.121569,0.466667,0.705882}%
\pgfsetfillcolor{currentfill}%
\pgfsetfillopacity{0.477947}%
\pgfsetlinewidth{1.003750pt}%
\definecolor{currentstroke}{rgb}{0.121569,0.466667,0.705882}%
\pgfsetstrokecolor{currentstroke}%
\pgfsetstrokeopacity{0.477947}%
\pgfsetdash{}{0pt}%
\pgfpathmoveto{\pgfqpoint{1.550036in}{1.717207in}}%
\pgfpathcurveto{\pgfqpoint{1.558272in}{1.717207in}}{\pgfqpoint{1.566172in}{1.720479in}}{\pgfqpoint{1.571996in}{1.726303in}}%
\pgfpathcurveto{\pgfqpoint{1.577820in}{1.732127in}}{\pgfqpoint{1.581092in}{1.740027in}}{\pgfqpoint{1.581092in}{1.748263in}}%
\pgfpathcurveto{\pgfqpoint{1.581092in}{1.756500in}}{\pgfqpoint{1.577820in}{1.764400in}}{\pgfqpoint{1.571996in}{1.770223in}}%
\pgfpathcurveto{\pgfqpoint{1.566172in}{1.776047in}}{\pgfqpoint{1.558272in}{1.779320in}}{\pgfqpoint{1.550036in}{1.779320in}}%
\pgfpathcurveto{\pgfqpoint{1.541800in}{1.779320in}}{\pgfqpoint{1.533900in}{1.776047in}}{\pgfqpoint{1.528076in}{1.770223in}}%
\pgfpathcurveto{\pgfqpoint{1.522252in}{1.764400in}}{\pgfqpoint{1.518979in}{1.756500in}}{\pgfqpoint{1.518979in}{1.748263in}}%
\pgfpathcurveto{\pgfqpoint{1.518979in}{1.740027in}}{\pgfqpoint{1.522252in}{1.732127in}}{\pgfqpoint{1.528076in}{1.726303in}}%
\pgfpathcurveto{\pgfqpoint{1.533900in}{1.720479in}}{\pgfqpoint{1.541800in}{1.717207in}}{\pgfqpoint{1.550036in}{1.717207in}}%
\pgfpathclose%
\pgfusepath{stroke,fill}%
\end{pgfscope}%
\begin{pgfscope}%
\pgfpathrectangle{\pgfqpoint{0.100000in}{0.212622in}}{\pgfqpoint{3.696000in}{3.696000in}}%
\pgfusepath{clip}%
\pgfsetbuttcap%
\pgfsetroundjoin%
\definecolor{currentfill}{rgb}{0.121569,0.466667,0.705882}%
\pgfsetfillcolor{currentfill}%
\pgfsetfillopacity{0.478740}%
\pgfsetlinewidth{1.003750pt}%
\definecolor{currentstroke}{rgb}{0.121569,0.466667,0.705882}%
\pgfsetstrokecolor{currentstroke}%
\pgfsetstrokeopacity{0.478740}%
\pgfsetdash{}{0pt}%
\pgfpathmoveto{\pgfqpoint{1.112832in}{1.624506in}}%
\pgfpathcurveto{\pgfqpoint{1.121068in}{1.624506in}}{\pgfqpoint{1.128968in}{1.627778in}}{\pgfqpoint{1.134792in}{1.633602in}}%
\pgfpathcurveto{\pgfqpoint{1.140616in}{1.639426in}}{\pgfqpoint{1.143888in}{1.647326in}}{\pgfqpoint{1.143888in}{1.655562in}}%
\pgfpathcurveto{\pgfqpoint{1.143888in}{1.663799in}}{\pgfqpoint{1.140616in}{1.671699in}}{\pgfqpoint{1.134792in}{1.677523in}}%
\pgfpathcurveto{\pgfqpoint{1.128968in}{1.683347in}}{\pgfqpoint{1.121068in}{1.686619in}}{\pgfqpoint{1.112832in}{1.686619in}}%
\pgfpathcurveto{\pgfqpoint{1.104595in}{1.686619in}}{\pgfqpoint{1.096695in}{1.683347in}}{\pgfqpoint{1.090871in}{1.677523in}}%
\pgfpathcurveto{\pgfqpoint{1.085047in}{1.671699in}}{\pgfqpoint{1.081775in}{1.663799in}}{\pgfqpoint{1.081775in}{1.655562in}}%
\pgfpathcurveto{\pgfqpoint{1.081775in}{1.647326in}}{\pgfqpoint{1.085047in}{1.639426in}}{\pgfqpoint{1.090871in}{1.633602in}}%
\pgfpathcurveto{\pgfqpoint{1.096695in}{1.627778in}}{\pgfqpoint{1.104595in}{1.624506in}}{\pgfqpoint{1.112832in}{1.624506in}}%
\pgfpathclose%
\pgfusepath{stroke,fill}%
\end{pgfscope}%
\begin{pgfscope}%
\pgfpathrectangle{\pgfqpoint{0.100000in}{0.212622in}}{\pgfqpoint{3.696000in}{3.696000in}}%
\pgfusepath{clip}%
\pgfsetbuttcap%
\pgfsetroundjoin%
\definecolor{currentfill}{rgb}{0.121569,0.466667,0.705882}%
\pgfsetfillcolor{currentfill}%
\pgfsetfillopacity{0.479650}%
\pgfsetlinewidth{1.003750pt}%
\definecolor{currentstroke}{rgb}{0.121569,0.466667,0.705882}%
\pgfsetstrokecolor{currentstroke}%
\pgfsetstrokeopacity{0.479650}%
\pgfsetdash{}{0pt}%
\pgfpathmoveto{\pgfqpoint{1.109371in}{1.618906in}}%
\pgfpathcurveto{\pgfqpoint{1.117607in}{1.618906in}}{\pgfqpoint{1.125507in}{1.622178in}}{\pgfqpoint{1.131331in}{1.628002in}}%
\pgfpathcurveto{\pgfqpoint{1.137155in}{1.633826in}}{\pgfqpoint{1.140427in}{1.641726in}}{\pgfqpoint{1.140427in}{1.649962in}}%
\pgfpathcurveto{\pgfqpoint{1.140427in}{1.658199in}}{\pgfqpoint{1.137155in}{1.666099in}}{\pgfqpoint{1.131331in}{1.671923in}}%
\pgfpathcurveto{\pgfqpoint{1.125507in}{1.677747in}}{\pgfqpoint{1.117607in}{1.681019in}}{\pgfqpoint{1.109371in}{1.681019in}}%
\pgfpathcurveto{\pgfqpoint{1.101135in}{1.681019in}}{\pgfqpoint{1.093234in}{1.677747in}}{\pgfqpoint{1.087411in}{1.671923in}}%
\pgfpathcurveto{\pgfqpoint{1.081587in}{1.666099in}}{\pgfqpoint{1.078314in}{1.658199in}}{\pgfqpoint{1.078314in}{1.649962in}}%
\pgfpathcurveto{\pgfqpoint{1.078314in}{1.641726in}}{\pgfqpoint{1.081587in}{1.633826in}}{\pgfqpoint{1.087411in}{1.628002in}}%
\pgfpathcurveto{\pgfqpoint{1.093234in}{1.622178in}}{\pgfqpoint{1.101135in}{1.618906in}}{\pgfqpoint{1.109371in}{1.618906in}}%
\pgfpathclose%
\pgfusepath{stroke,fill}%
\end{pgfscope}%
\begin{pgfscope}%
\pgfpathrectangle{\pgfqpoint{0.100000in}{0.212622in}}{\pgfqpoint{3.696000in}{3.696000in}}%
\pgfusepath{clip}%
\pgfsetbuttcap%
\pgfsetroundjoin%
\definecolor{currentfill}{rgb}{0.121569,0.466667,0.705882}%
\pgfsetfillcolor{currentfill}%
\pgfsetfillopacity{0.480096}%
\pgfsetlinewidth{1.003750pt}%
\definecolor{currentstroke}{rgb}{0.121569,0.466667,0.705882}%
\pgfsetstrokecolor{currentstroke}%
\pgfsetstrokeopacity{0.480096}%
\pgfsetdash{}{0pt}%
\pgfpathmoveto{\pgfqpoint{1.550493in}{1.718021in}}%
\pgfpathcurveto{\pgfqpoint{1.558729in}{1.718021in}}{\pgfqpoint{1.566629in}{1.721293in}}{\pgfqpoint{1.572453in}{1.727117in}}%
\pgfpathcurveto{\pgfqpoint{1.578277in}{1.732941in}}{\pgfqpoint{1.581549in}{1.740841in}}{\pgfqpoint{1.581549in}{1.749077in}}%
\pgfpathcurveto{\pgfqpoint{1.581549in}{1.757314in}}{\pgfqpoint{1.578277in}{1.765214in}}{\pgfqpoint{1.572453in}{1.771038in}}%
\pgfpathcurveto{\pgfqpoint{1.566629in}{1.776862in}}{\pgfqpoint{1.558729in}{1.780134in}}{\pgfqpoint{1.550493in}{1.780134in}}%
\pgfpathcurveto{\pgfqpoint{1.542257in}{1.780134in}}{\pgfqpoint{1.534357in}{1.776862in}}{\pgfqpoint{1.528533in}{1.771038in}}%
\pgfpathcurveto{\pgfqpoint{1.522709in}{1.765214in}}{\pgfqpoint{1.519436in}{1.757314in}}{\pgfqpoint{1.519436in}{1.749077in}}%
\pgfpathcurveto{\pgfqpoint{1.519436in}{1.740841in}}{\pgfqpoint{1.522709in}{1.732941in}}{\pgfqpoint{1.528533in}{1.727117in}}%
\pgfpathcurveto{\pgfqpoint{1.534357in}{1.721293in}}{\pgfqpoint{1.542257in}{1.718021in}}{\pgfqpoint{1.550493in}{1.718021in}}%
\pgfpathclose%
\pgfusepath{stroke,fill}%
\end{pgfscope}%
\begin{pgfscope}%
\pgfpathrectangle{\pgfqpoint{0.100000in}{0.212622in}}{\pgfqpoint{3.696000in}{3.696000in}}%
\pgfusepath{clip}%
\pgfsetbuttcap%
\pgfsetroundjoin%
\definecolor{currentfill}{rgb}{0.121569,0.466667,0.705882}%
\pgfsetfillcolor{currentfill}%
\pgfsetfillopacity{0.480629}%
\pgfsetlinewidth{1.003750pt}%
\definecolor{currentstroke}{rgb}{0.121569,0.466667,0.705882}%
\pgfsetstrokecolor{currentstroke}%
\pgfsetstrokeopacity{0.480629}%
\pgfsetdash{}{0pt}%
\pgfpathmoveto{\pgfqpoint{1.106402in}{1.617114in}}%
\pgfpathcurveto{\pgfqpoint{1.114639in}{1.617114in}}{\pgfqpoint{1.122539in}{1.620386in}}{\pgfqpoint{1.128363in}{1.626210in}}%
\pgfpathcurveto{\pgfqpoint{1.134187in}{1.632034in}}{\pgfqpoint{1.137459in}{1.639934in}}{\pgfqpoint{1.137459in}{1.648170in}}%
\pgfpathcurveto{\pgfqpoint{1.137459in}{1.656406in}}{\pgfqpoint{1.134187in}{1.664306in}}{\pgfqpoint{1.128363in}{1.670130in}}%
\pgfpathcurveto{\pgfqpoint{1.122539in}{1.675954in}}{\pgfqpoint{1.114639in}{1.679227in}}{\pgfqpoint{1.106402in}{1.679227in}}%
\pgfpathcurveto{\pgfqpoint{1.098166in}{1.679227in}}{\pgfqpoint{1.090266in}{1.675954in}}{\pgfqpoint{1.084442in}{1.670130in}}%
\pgfpathcurveto{\pgfqpoint{1.078618in}{1.664306in}}{\pgfqpoint{1.075346in}{1.656406in}}{\pgfqpoint{1.075346in}{1.648170in}}%
\pgfpathcurveto{\pgfqpoint{1.075346in}{1.639934in}}{\pgfqpoint{1.078618in}{1.632034in}}{\pgfqpoint{1.084442in}{1.626210in}}%
\pgfpathcurveto{\pgfqpoint{1.090266in}{1.620386in}}{\pgfqpoint{1.098166in}{1.617114in}}{\pgfqpoint{1.106402in}{1.617114in}}%
\pgfpathclose%
\pgfusepath{stroke,fill}%
\end{pgfscope}%
\begin{pgfscope}%
\pgfpathrectangle{\pgfqpoint{0.100000in}{0.212622in}}{\pgfqpoint{3.696000in}{3.696000in}}%
\pgfusepath{clip}%
\pgfsetbuttcap%
\pgfsetroundjoin%
\definecolor{currentfill}{rgb}{0.121569,0.466667,0.705882}%
\pgfsetfillcolor{currentfill}%
\pgfsetfillopacity{0.482273}%
\pgfsetlinewidth{1.003750pt}%
\definecolor{currentstroke}{rgb}{0.121569,0.466667,0.705882}%
\pgfsetstrokecolor{currentstroke}%
\pgfsetstrokeopacity{0.482273}%
\pgfsetdash{}{0pt}%
\pgfpathmoveto{\pgfqpoint{1.551183in}{1.717117in}}%
\pgfpathcurveto{\pgfqpoint{1.559419in}{1.717117in}}{\pgfqpoint{1.567319in}{1.720390in}}{\pgfqpoint{1.573143in}{1.726214in}}%
\pgfpathcurveto{\pgfqpoint{1.578967in}{1.732037in}}{\pgfqpoint{1.582240in}{1.739938in}}{\pgfqpoint{1.582240in}{1.748174in}}%
\pgfpathcurveto{\pgfqpoint{1.582240in}{1.756410in}}{\pgfqpoint{1.578967in}{1.764310in}}{\pgfqpoint{1.573143in}{1.770134in}}%
\pgfpathcurveto{\pgfqpoint{1.567319in}{1.775958in}}{\pgfqpoint{1.559419in}{1.779230in}}{\pgfqpoint{1.551183in}{1.779230in}}%
\pgfpathcurveto{\pgfqpoint{1.542947in}{1.779230in}}{\pgfqpoint{1.535047in}{1.775958in}}{\pgfqpoint{1.529223in}{1.770134in}}%
\pgfpathcurveto{\pgfqpoint{1.523399in}{1.764310in}}{\pgfqpoint{1.520127in}{1.756410in}}{\pgfqpoint{1.520127in}{1.748174in}}%
\pgfpathcurveto{\pgfqpoint{1.520127in}{1.739938in}}{\pgfqpoint{1.523399in}{1.732037in}}{\pgfqpoint{1.529223in}{1.726214in}}%
\pgfpathcurveto{\pgfqpoint{1.535047in}{1.720390in}}{\pgfqpoint{1.542947in}{1.717117in}}{\pgfqpoint{1.551183in}{1.717117in}}%
\pgfpathclose%
\pgfusepath{stroke,fill}%
\end{pgfscope}%
\begin{pgfscope}%
\pgfpathrectangle{\pgfqpoint{0.100000in}{0.212622in}}{\pgfqpoint{3.696000in}{3.696000in}}%
\pgfusepath{clip}%
\pgfsetbuttcap%
\pgfsetroundjoin%
\definecolor{currentfill}{rgb}{0.121569,0.466667,0.705882}%
\pgfsetfillcolor{currentfill}%
\pgfsetfillopacity{0.483105}%
\pgfsetlinewidth{1.003750pt}%
\definecolor{currentstroke}{rgb}{0.121569,0.466667,0.705882}%
\pgfsetstrokecolor{currentstroke}%
\pgfsetstrokeopacity{0.483105}%
\pgfsetdash{}{0pt}%
\pgfpathmoveto{\pgfqpoint{1.102022in}{1.615428in}}%
\pgfpathcurveto{\pgfqpoint{1.110258in}{1.615428in}}{\pgfqpoint{1.118158in}{1.618701in}}{\pgfqpoint{1.123982in}{1.624525in}}%
\pgfpathcurveto{\pgfqpoint{1.129806in}{1.630349in}}{\pgfqpoint{1.133078in}{1.638249in}}{\pgfqpoint{1.133078in}{1.646485in}}%
\pgfpathcurveto{\pgfqpoint{1.133078in}{1.654721in}}{\pgfqpoint{1.129806in}{1.662621in}}{\pgfqpoint{1.123982in}{1.668445in}}%
\pgfpathcurveto{\pgfqpoint{1.118158in}{1.674269in}}{\pgfqpoint{1.110258in}{1.677541in}}{\pgfqpoint{1.102022in}{1.677541in}}%
\pgfpathcurveto{\pgfqpoint{1.093785in}{1.677541in}}{\pgfqpoint{1.085885in}{1.674269in}}{\pgfqpoint{1.080061in}{1.668445in}}%
\pgfpathcurveto{\pgfqpoint{1.074237in}{1.662621in}}{\pgfqpoint{1.070965in}{1.654721in}}{\pgfqpoint{1.070965in}{1.646485in}}%
\pgfpathcurveto{\pgfqpoint{1.070965in}{1.638249in}}{\pgfqpoint{1.074237in}{1.630349in}}{\pgfqpoint{1.080061in}{1.624525in}}%
\pgfpathcurveto{\pgfqpoint{1.085885in}{1.618701in}}{\pgfqpoint{1.093785in}{1.615428in}}{\pgfqpoint{1.102022in}{1.615428in}}%
\pgfpathclose%
\pgfusepath{stroke,fill}%
\end{pgfscope}%
\begin{pgfscope}%
\pgfpathrectangle{\pgfqpoint{0.100000in}{0.212622in}}{\pgfqpoint{3.696000in}{3.696000in}}%
\pgfusepath{clip}%
\pgfsetbuttcap%
\pgfsetroundjoin%
\definecolor{currentfill}{rgb}{0.121569,0.466667,0.705882}%
\pgfsetfillcolor{currentfill}%
\pgfsetfillopacity{0.484394}%
\pgfsetlinewidth{1.003750pt}%
\definecolor{currentstroke}{rgb}{0.121569,0.466667,0.705882}%
\pgfsetstrokecolor{currentstroke}%
\pgfsetstrokeopacity{0.484394}%
\pgfsetdash{}{0pt}%
\pgfpathmoveto{\pgfqpoint{1.098424in}{1.613021in}}%
\pgfpathcurveto{\pgfqpoint{1.106660in}{1.613021in}}{\pgfqpoint{1.114560in}{1.616294in}}{\pgfqpoint{1.120384in}{1.622118in}}%
\pgfpathcurveto{\pgfqpoint{1.126208in}{1.627942in}}{\pgfqpoint{1.129481in}{1.635842in}}{\pgfqpoint{1.129481in}{1.644078in}}%
\pgfpathcurveto{\pgfqpoint{1.129481in}{1.652314in}}{\pgfqpoint{1.126208in}{1.660214in}}{\pgfqpoint{1.120384in}{1.666038in}}%
\pgfpathcurveto{\pgfqpoint{1.114560in}{1.671862in}}{\pgfqpoint{1.106660in}{1.675134in}}{\pgfqpoint{1.098424in}{1.675134in}}%
\pgfpathcurveto{\pgfqpoint{1.090188in}{1.675134in}}{\pgfqpoint{1.082288in}{1.671862in}}{\pgfqpoint{1.076464in}{1.666038in}}%
\pgfpathcurveto{\pgfqpoint{1.070640in}{1.660214in}}{\pgfqpoint{1.067368in}{1.652314in}}{\pgfqpoint{1.067368in}{1.644078in}}%
\pgfpathcurveto{\pgfqpoint{1.067368in}{1.635842in}}{\pgfqpoint{1.070640in}{1.627942in}}{\pgfqpoint{1.076464in}{1.622118in}}%
\pgfpathcurveto{\pgfqpoint{1.082288in}{1.616294in}}{\pgfqpoint{1.090188in}{1.613021in}}{\pgfqpoint{1.098424in}{1.613021in}}%
\pgfpathclose%
\pgfusepath{stroke,fill}%
\end{pgfscope}%
\begin{pgfscope}%
\pgfpathrectangle{\pgfqpoint{0.100000in}{0.212622in}}{\pgfqpoint{3.696000in}{3.696000in}}%
\pgfusepath{clip}%
\pgfsetbuttcap%
\pgfsetroundjoin%
\definecolor{currentfill}{rgb}{0.121569,0.466667,0.705882}%
\pgfsetfillcolor{currentfill}%
\pgfsetfillopacity{0.484910}%
\pgfsetlinewidth{1.003750pt}%
\definecolor{currentstroke}{rgb}{0.121569,0.466667,0.705882}%
\pgfsetstrokecolor{currentstroke}%
\pgfsetstrokeopacity{0.484910}%
\pgfsetdash{}{0pt}%
\pgfpathmoveto{\pgfqpoint{1.552271in}{1.717128in}}%
\pgfpathcurveto{\pgfqpoint{1.560507in}{1.717128in}}{\pgfqpoint{1.568407in}{1.720400in}}{\pgfqpoint{1.574231in}{1.726224in}}%
\pgfpathcurveto{\pgfqpoint{1.580055in}{1.732048in}}{\pgfqpoint{1.583328in}{1.739948in}}{\pgfqpoint{1.583328in}{1.748184in}}%
\pgfpathcurveto{\pgfqpoint{1.583328in}{1.756421in}}{\pgfqpoint{1.580055in}{1.764321in}}{\pgfqpoint{1.574231in}{1.770145in}}%
\pgfpathcurveto{\pgfqpoint{1.568407in}{1.775969in}}{\pgfqpoint{1.560507in}{1.779241in}}{\pgfqpoint{1.552271in}{1.779241in}}%
\pgfpathcurveto{\pgfqpoint{1.544035in}{1.779241in}}{\pgfqpoint{1.536135in}{1.775969in}}{\pgfqpoint{1.530311in}{1.770145in}}%
\pgfpathcurveto{\pgfqpoint{1.524487in}{1.764321in}}{\pgfqpoint{1.521215in}{1.756421in}}{\pgfqpoint{1.521215in}{1.748184in}}%
\pgfpathcurveto{\pgfqpoint{1.521215in}{1.739948in}}{\pgfqpoint{1.524487in}{1.732048in}}{\pgfqpoint{1.530311in}{1.726224in}}%
\pgfpathcurveto{\pgfqpoint{1.536135in}{1.720400in}}{\pgfqpoint{1.544035in}{1.717128in}}{\pgfqpoint{1.552271in}{1.717128in}}%
\pgfpathclose%
\pgfusepath{stroke,fill}%
\end{pgfscope}%
\begin{pgfscope}%
\pgfpathrectangle{\pgfqpoint{0.100000in}{0.212622in}}{\pgfqpoint{3.696000in}{3.696000in}}%
\pgfusepath{clip}%
\pgfsetbuttcap%
\pgfsetroundjoin%
\definecolor{currentfill}{rgb}{0.121569,0.466667,0.705882}%
\pgfsetfillcolor{currentfill}%
\pgfsetfillopacity{0.485190}%
\pgfsetlinewidth{1.003750pt}%
\definecolor{currentstroke}{rgb}{0.121569,0.466667,0.705882}%
\pgfsetstrokecolor{currentstroke}%
\pgfsetstrokeopacity{0.485190}%
\pgfsetdash{}{0pt}%
\pgfpathmoveto{\pgfqpoint{1.095863in}{1.610254in}}%
\pgfpathcurveto{\pgfqpoint{1.104099in}{1.610254in}}{\pgfqpoint{1.111999in}{1.613526in}}{\pgfqpoint{1.117823in}{1.619350in}}%
\pgfpathcurveto{\pgfqpoint{1.123647in}{1.625174in}}{\pgfqpoint{1.126919in}{1.633074in}}{\pgfqpoint{1.126919in}{1.641310in}}%
\pgfpathcurveto{\pgfqpoint{1.126919in}{1.649546in}}{\pgfqpoint{1.123647in}{1.657446in}}{\pgfqpoint{1.117823in}{1.663270in}}%
\pgfpathcurveto{\pgfqpoint{1.111999in}{1.669094in}}{\pgfqpoint{1.104099in}{1.672367in}}{\pgfqpoint{1.095863in}{1.672367in}}%
\pgfpathcurveto{\pgfqpoint{1.087626in}{1.672367in}}{\pgfqpoint{1.079726in}{1.669094in}}{\pgfqpoint{1.073902in}{1.663270in}}%
\pgfpathcurveto{\pgfqpoint{1.068078in}{1.657446in}}{\pgfqpoint{1.064806in}{1.649546in}}{\pgfqpoint{1.064806in}{1.641310in}}%
\pgfpathcurveto{\pgfqpoint{1.064806in}{1.633074in}}{\pgfqpoint{1.068078in}{1.625174in}}{\pgfqpoint{1.073902in}{1.619350in}}%
\pgfpathcurveto{\pgfqpoint{1.079726in}{1.613526in}}{\pgfqpoint{1.087626in}{1.610254in}}{\pgfqpoint{1.095863in}{1.610254in}}%
\pgfpathclose%
\pgfusepath{stroke,fill}%
\end{pgfscope}%
\begin{pgfscope}%
\pgfpathrectangle{\pgfqpoint{0.100000in}{0.212622in}}{\pgfqpoint{3.696000in}{3.696000in}}%
\pgfusepath{clip}%
\pgfsetbuttcap%
\pgfsetroundjoin%
\definecolor{currentfill}{rgb}{0.121569,0.466667,0.705882}%
\pgfsetfillcolor{currentfill}%
\pgfsetfillopacity{0.486301}%
\pgfsetlinewidth{1.003750pt}%
\definecolor{currentstroke}{rgb}{0.121569,0.466667,0.705882}%
\pgfsetstrokecolor{currentstroke}%
\pgfsetstrokeopacity{0.486301}%
\pgfsetdash{}{0pt}%
\pgfpathmoveto{\pgfqpoint{1.094090in}{1.609928in}}%
\pgfpathcurveto{\pgfqpoint{1.102326in}{1.609928in}}{\pgfqpoint{1.110226in}{1.613200in}}{\pgfqpoint{1.116050in}{1.619024in}}%
\pgfpathcurveto{\pgfqpoint{1.121874in}{1.624848in}}{\pgfqpoint{1.125146in}{1.632748in}}{\pgfqpoint{1.125146in}{1.640984in}}%
\pgfpathcurveto{\pgfqpoint{1.125146in}{1.649221in}}{\pgfqpoint{1.121874in}{1.657121in}}{\pgfqpoint{1.116050in}{1.662945in}}%
\pgfpathcurveto{\pgfqpoint{1.110226in}{1.668769in}}{\pgfqpoint{1.102326in}{1.672041in}}{\pgfqpoint{1.094090in}{1.672041in}}%
\pgfpathcurveto{\pgfqpoint{1.085854in}{1.672041in}}{\pgfqpoint{1.077953in}{1.668769in}}{\pgfqpoint{1.072130in}{1.662945in}}%
\pgfpathcurveto{\pgfqpoint{1.066306in}{1.657121in}}{\pgfqpoint{1.063033in}{1.649221in}}{\pgfqpoint{1.063033in}{1.640984in}}%
\pgfpathcurveto{\pgfqpoint{1.063033in}{1.632748in}}{\pgfqpoint{1.066306in}{1.624848in}}{\pgfqpoint{1.072130in}{1.619024in}}%
\pgfpathcurveto{\pgfqpoint{1.077953in}{1.613200in}}{\pgfqpoint{1.085854in}{1.609928in}}{\pgfqpoint{1.094090in}{1.609928in}}%
\pgfpathclose%
\pgfusepath{stroke,fill}%
\end{pgfscope}%
\begin{pgfscope}%
\pgfpathrectangle{\pgfqpoint{0.100000in}{0.212622in}}{\pgfqpoint{3.696000in}{3.696000in}}%
\pgfusepath{clip}%
\pgfsetbuttcap%
\pgfsetroundjoin%
\definecolor{currentfill}{rgb}{0.121569,0.466667,0.705882}%
\pgfsetfillcolor{currentfill}%
\pgfsetfillopacity{0.486825}%
\pgfsetlinewidth{1.003750pt}%
\definecolor{currentstroke}{rgb}{0.121569,0.466667,0.705882}%
\pgfsetstrokecolor{currentstroke}%
\pgfsetstrokeopacity{0.486825}%
\pgfsetdash{}{0pt}%
\pgfpathmoveto{\pgfqpoint{1.092230in}{1.608244in}}%
\pgfpathcurveto{\pgfqpoint{1.100467in}{1.608244in}}{\pgfqpoint{1.108367in}{1.611517in}}{\pgfqpoint{1.114191in}{1.617341in}}%
\pgfpathcurveto{\pgfqpoint{1.120015in}{1.623165in}}{\pgfqpoint{1.123287in}{1.631065in}}{\pgfqpoint{1.123287in}{1.639301in}}%
\pgfpathcurveto{\pgfqpoint{1.123287in}{1.647537in}}{\pgfqpoint{1.120015in}{1.655437in}}{\pgfqpoint{1.114191in}{1.661261in}}%
\pgfpathcurveto{\pgfqpoint{1.108367in}{1.667085in}}{\pgfqpoint{1.100467in}{1.670357in}}{\pgfqpoint{1.092230in}{1.670357in}}%
\pgfpathcurveto{\pgfqpoint{1.083994in}{1.670357in}}{\pgfqpoint{1.076094in}{1.667085in}}{\pgfqpoint{1.070270in}{1.661261in}}%
\pgfpathcurveto{\pgfqpoint{1.064446in}{1.655437in}}{\pgfqpoint{1.061174in}{1.647537in}}{\pgfqpoint{1.061174in}{1.639301in}}%
\pgfpathcurveto{\pgfqpoint{1.061174in}{1.631065in}}{\pgfqpoint{1.064446in}{1.623165in}}{\pgfqpoint{1.070270in}{1.617341in}}%
\pgfpathcurveto{\pgfqpoint{1.076094in}{1.611517in}}{\pgfqpoint{1.083994in}{1.608244in}}{\pgfqpoint{1.092230in}{1.608244in}}%
\pgfpathclose%
\pgfusepath{stroke,fill}%
\end{pgfscope}%
\begin{pgfscope}%
\pgfpathrectangle{\pgfqpoint{0.100000in}{0.212622in}}{\pgfqpoint{3.696000in}{3.696000in}}%
\pgfusepath{clip}%
\pgfsetbuttcap%
\pgfsetroundjoin%
\definecolor{currentfill}{rgb}{0.121569,0.466667,0.705882}%
\pgfsetfillcolor{currentfill}%
\pgfsetfillopacity{0.487276}%
\pgfsetlinewidth{1.003750pt}%
\definecolor{currentstroke}{rgb}{0.121569,0.466667,0.705882}%
\pgfsetstrokecolor{currentstroke}%
\pgfsetstrokeopacity{0.487276}%
\pgfsetdash{}{0pt}%
\pgfpathmoveto{\pgfqpoint{1.553639in}{1.714166in}}%
\pgfpathcurveto{\pgfqpoint{1.561875in}{1.714166in}}{\pgfqpoint{1.569775in}{1.717438in}}{\pgfqpoint{1.575599in}{1.723262in}}%
\pgfpathcurveto{\pgfqpoint{1.581423in}{1.729086in}}{\pgfqpoint{1.584695in}{1.736986in}}{\pgfqpoint{1.584695in}{1.745222in}}%
\pgfpathcurveto{\pgfqpoint{1.584695in}{1.753458in}}{\pgfqpoint{1.581423in}{1.761359in}}{\pgfqpoint{1.575599in}{1.767182in}}%
\pgfpathcurveto{\pgfqpoint{1.569775in}{1.773006in}}{\pgfqpoint{1.561875in}{1.776279in}}{\pgfqpoint{1.553639in}{1.776279in}}%
\pgfpathcurveto{\pgfqpoint{1.545402in}{1.776279in}}{\pgfqpoint{1.537502in}{1.773006in}}{\pgfqpoint{1.531678in}{1.767182in}}%
\pgfpathcurveto{\pgfqpoint{1.525854in}{1.761359in}}{\pgfqpoint{1.522582in}{1.753458in}}{\pgfqpoint{1.522582in}{1.745222in}}%
\pgfpathcurveto{\pgfqpoint{1.522582in}{1.736986in}}{\pgfqpoint{1.525854in}{1.729086in}}{\pgfqpoint{1.531678in}{1.723262in}}%
\pgfpathcurveto{\pgfqpoint{1.537502in}{1.717438in}}{\pgfqpoint{1.545402in}{1.714166in}}{\pgfqpoint{1.553639in}{1.714166in}}%
\pgfpathclose%
\pgfusepath{stroke,fill}%
\end{pgfscope}%
\begin{pgfscope}%
\pgfpathrectangle{\pgfqpoint{0.100000in}{0.212622in}}{\pgfqpoint{3.696000in}{3.696000in}}%
\pgfusepath{clip}%
\pgfsetbuttcap%
\pgfsetroundjoin%
\definecolor{currentfill}{rgb}{0.121569,0.466667,0.705882}%
\pgfsetfillcolor{currentfill}%
\pgfsetfillopacity{0.488003}%
\pgfsetlinewidth{1.003750pt}%
\definecolor{currentstroke}{rgb}{0.121569,0.466667,0.705882}%
\pgfsetstrokecolor{currentstroke}%
\pgfsetstrokeopacity{0.488003}%
\pgfsetdash{}{0pt}%
\pgfpathmoveto{\pgfqpoint{1.088828in}{1.606213in}}%
\pgfpathcurveto{\pgfqpoint{1.097064in}{1.606213in}}{\pgfqpoint{1.104964in}{1.609485in}}{\pgfqpoint{1.110788in}{1.615309in}}%
\pgfpathcurveto{\pgfqpoint{1.116612in}{1.621133in}}{\pgfqpoint{1.119884in}{1.629033in}}{\pgfqpoint{1.119884in}{1.637270in}}%
\pgfpathcurveto{\pgfqpoint{1.119884in}{1.645506in}}{\pgfqpoint{1.116612in}{1.653406in}}{\pgfqpoint{1.110788in}{1.659230in}}%
\pgfpathcurveto{\pgfqpoint{1.104964in}{1.665054in}}{\pgfqpoint{1.097064in}{1.668326in}}{\pgfqpoint{1.088828in}{1.668326in}}%
\pgfpathcurveto{\pgfqpoint{1.080591in}{1.668326in}}{\pgfqpoint{1.072691in}{1.665054in}}{\pgfqpoint{1.066867in}{1.659230in}}%
\pgfpathcurveto{\pgfqpoint{1.061043in}{1.653406in}}{\pgfqpoint{1.057771in}{1.645506in}}{\pgfqpoint{1.057771in}{1.637270in}}%
\pgfpathcurveto{\pgfqpoint{1.057771in}{1.629033in}}{\pgfqpoint{1.061043in}{1.621133in}}{\pgfqpoint{1.066867in}{1.615309in}}%
\pgfpathcurveto{\pgfqpoint{1.072691in}{1.609485in}}{\pgfqpoint{1.080591in}{1.606213in}}{\pgfqpoint{1.088828in}{1.606213in}}%
\pgfpathclose%
\pgfusepath{stroke,fill}%
\end{pgfscope}%
\begin{pgfscope}%
\pgfpathrectangle{\pgfqpoint{0.100000in}{0.212622in}}{\pgfqpoint{3.696000in}{3.696000in}}%
\pgfusepath{clip}%
\pgfsetbuttcap%
\pgfsetroundjoin%
\definecolor{currentfill}{rgb}{0.121569,0.466667,0.705882}%
\pgfsetfillcolor{currentfill}%
\pgfsetfillopacity{0.489138}%
\pgfsetlinewidth{1.003750pt}%
\definecolor{currentstroke}{rgb}{0.121569,0.466667,0.705882}%
\pgfsetstrokecolor{currentstroke}%
\pgfsetstrokeopacity{0.489138}%
\pgfsetdash{}{0pt}%
\pgfpathmoveto{\pgfqpoint{1.554065in}{1.715034in}}%
\pgfpathcurveto{\pgfqpoint{1.562301in}{1.715034in}}{\pgfqpoint{1.570202in}{1.718306in}}{\pgfqpoint{1.576025in}{1.724130in}}%
\pgfpathcurveto{\pgfqpoint{1.581849in}{1.729954in}}{\pgfqpoint{1.585122in}{1.737854in}}{\pgfqpoint{1.585122in}{1.746091in}}%
\pgfpathcurveto{\pgfqpoint{1.585122in}{1.754327in}}{\pgfqpoint{1.581849in}{1.762227in}}{\pgfqpoint{1.576025in}{1.768051in}}%
\pgfpathcurveto{\pgfqpoint{1.570202in}{1.773875in}}{\pgfqpoint{1.562301in}{1.777147in}}{\pgfqpoint{1.554065in}{1.777147in}}%
\pgfpathcurveto{\pgfqpoint{1.545829in}{1.777147in}}{\pgfqpoint{1.537929in}{1.773875in}}{\pgfqpoint{1.532105in}{1.768051in}}%
\pgfpathcurveto{\pgfqpoint{1.526281in}{1.762227in}}{\pgfqpoint{1.523009in}{1.754327in}}{\pgfqpoint{1.523009in}{1.746091in}}%
\pgfpathcurveto{\pgfqpoint{1.523009in}{1.737854in}}{\pgfqpoint{1.526281in}{1.729954in}}{\pgfqpoint{1.532105in}{1.724130in}}%
\pgfpathcurveto{\pgfqpoint{1.537929in}{1.718306in}}{\pgfqpoint{1.545829in}{1.715034in}}{\pgfqpoint{1.554065in}{1.715034in}}%
\pgfpathclose%
\pgfusepath{stroke,fill}%
\end{pgfscope}%
\begin{pgfscope}%
\pgfpathrectangle{\pgfqpoint{0.100000in}{0.212622in}}{\pgfqpoint{3.696000in}{3.696000in}}%
\pgfusepath{clip}%
\pgfsetbuttcap%
\pgfsetroundjoin%
\definecolor{currentfill}{rgb}{0.121569,0.466667,0.705882}%
\pgfsetfillcolor{currentfill}%
\pgfsetfillopacity{0.490968}%
\pgfsetlinewidth{1.003750pt}%
\definecolor{currentstroke}{rgb}{0.121569,0.466667,0.705882}%
\pgfsetstrokecolor{currentstroke}%
\pgfsetstrokeopacity{0.490968}%
\pgfsetdash{}{0pt}%
\pgfpathmoveto{\pgfqpoint{1.554839in}{1.713974in}}%
\pgfpathcurveto{\pgfqpoint{1.563075in}{1.713974in}}{\pgfqpoint{1.570975in}{1.717247in}}{\pgfqpoint{1.576799in}{1.723071in}}%
\pgfpathcurveto{\pgfqpoint{1.582623in}{1.728895in}}{\pgfqpoint{1.585896in}{1.736795in}}{\pgfqpoint{1.585896in}{1.745031in}}%
\pgfpathcurveto{\pgfqpoint{1.585896in}{1.753267in}}{\pgfqpoint{1.582623in}{1.761167in}}{\pgfqpoint{1.576799in}{1.766991in}}%
\pgfpathcurveto{\pgfqpoint{1.570975in}{1.772815in}}{\pgfqpoint{1.563075in}{1.776087in}}{\pgfqpoint{1.554839in}{1.776087in}}%
\pgfpathcurveto{\pgfqpoint{1.546603in}{1.776087in}}{\pgfqpoint{1.538703in}{1.772815in}}{\pgfqpoint{1.532879in}{1.766991in}}%
\pgfpathcurveto{\pgfqpoint{1.527055in}{1.761167in}}{\pgfqpoint{1.523783in}{1.753267in}}{\pgfqpoint{1.523783in}{1.745031in}}%
\pgfpathcurveto{\pgfqpoint{1.523783in}{1.736795in}}{\pgfqpoint{1.527055in}{1.728895in}}{\pgfqpoint{1.532879in}{1.723071in}}%
\pgfpathcurveto{\pgfqpoint{1.538703in}{1.717247in}}{\pgfqpoint{1.546603in}{1.713974in}}{\pgfqpoint{1.554839in}{1.713974in}}%
\pgfpathclose%
\pgfusepath{stroke,fill}%
\end{pgfscope}%
\begin{pgfscope}%
\pgfpathrectangle{\pgfqpoint{0.100000in}{0.212622in}}{\pgfqpoint{3.696000in}{3.696000in}}%
\pgfusepath{clip}%
\pgfsetbuttcap%
\pgfsetroundjoin%
\definecolor{currentfill}{rgb}{0.121569,0.466667,0.705882}%
\pgfsetfillcolor{currentfill}%
\pgfsetfillopacity{0.491223}%
\pgfsetlinewidth{1.003750pt}%
\definecolor{currentstroke}{rgb}{0.121569,0.466667,0.705882}%
\pgfsetstrokecolor{currentstroke}%
\pgfsetstrokeopacity{0.491223}%
\pgfsetdash{}{0pt}%
\pgfpathmoveto{\pgfqpoint{1.083712in}{1.605752in}}%
\pgfpathcurveto{\pgfqpoint{1.091949in}{1.605752in}}{\pgfqpoint{1.099849in}{1.609024in}}{\pgfqpoint{1.105673in}{1.614848in}}%
\pgfpathcurveto{\pgfqpoint{1.111497in}{1.620672in}}{\pgfqpoint{1.114769in}{1.628572in}}{\pgfqpoint{1.114769in}{1.636809in}}%
\pgfpathcurveto{\pgfqpoint{1.114769in}{1.645045in}}{\pgfqpoint{1.111497in}{1.652945in}}{\pgfqpoint{1.105673in}{1.658769in}}%
\pgfpathcurveto{\pgfqpoint{1.099849in}{1.664593in}}{\pgfqpoint{1.091949in}{1.667865in}}{\pgfqpoint{1.083712in}{1.667865in}}%
\pgfpathcurveto{\pgfqpoint{1.075476in}{1.667865in}}{\pgfqpoint{1.067576in}{1.664593in}}{\pgfqpoint{1.061752in}{1.658769in}}%
\pgfpathcurveto{\pgfqpoint{1.055928in}{1.652945in}}{\pgfqpoint{1.052656in}{1.645045in}}{\pgfqpoint{1.052656in}{1.636809in}}%
\pgfpathcurveto{\pgfqpoint{1.052656in}{1.628572in}}{\pgfqpoint{1.055928in}{1.620672in}}{\pgfqpoint{1.061752in}{1.614848in}}%
\pgfpathcurveto{\pgfqpoint{1.067576in}{1.609024in}}{\pgfqpoint{1.075476in}{1.605752in}}{\pgfqpoint{1.083712in}{1.605752in}}%
\pgfpathclose%
\pgfusepath{stroke,fill}%
\end{pgfscope}%
\begin{pgfscope}%
\pgfpathrectangle{\pgfqpoint{0.100000in}{0.212622in}}{\pgfqpoint{3.696000in}{3.696000in}}%
\pgfusepath{clip}%
\pgfsetbuttcap%
\pgfsetroundjoin%
\definecolor{currentfill}{rgb}{0.121569,0.466667,0.705882}%
\pgfsetfillcolor{currentfill}%
\pgfsetfillopacity{0.492765}%
\pgfsetlinewidth{1.003750pt}%
\definecolor{currentstroke}{rgb}{0.121569,0.466667,0.705882}%
\pgfsetstrokecolor{currentstroke}%
\pgfsetstrokeopacity{0.492765}%
\pgfsetdash{}{0pt}%
\pgfpathmoveto{\pgfqpoint{1.079381in}{1.604768in}}%
\pgfpathcurveto{\pgfqpoint{1.087617in}{1.604768in}}{\pgfqpoint{1.095517in}{1.608040in}}{\pgfqpoint{1.101341in}{1.613864in}}%
\pgfpathcurveto{\pgfqpoint{1.107165in}{1.619688in}}{\pgfqpoint{1.110438in}{1.627588in}}{\pgfqpoint{1.110438in}{1.635824in}}%
\pgfpathcurveto{\pgfqpoint{1.110438in}{1.644060in}}{\pgfqpoint{1.107165in}{1.651960in}}{\pgfqpoint{1.101341in}{1.657784in}}%
\pgfpathcurveto{\pgfqpoint{1.095517in}{1.663608in}}{\pgfqpoint{1.087617in}{1.666881in}}{\pgfqpoint{1.079381in}{1.666881in}}%
\pgfpathcurveto{\pgfqpoint{1.071145in}{1.666881in}}{\pgfqpoint{1.063245in}{1.663608in}}{\pgfqpoint{1.057421in}{1.657784in}}%
\pgfpathcurveto{\pgfqpoint{1.051597in}{1.651960in}}{\pgfqpoint{1.048325in}{1.644060in}}{\pgfqpoint{1.048325in}{1.635824in}}%
\pgfpathcurveto{\pgfqpoint{1.048325in}{1.627588in}}{\pgfqpoint{1.051597in}{1.619688in}}{\pgfqpoint{1.057421in}{1.613864in}}%
\pgfpathcurveto{\pgfqpoint{1.063245in}{1.608040in}}{\pgfqpoint{1.071145in}{1.604768in}}{\pgfqpoint{1.079381in}{1.604768in}}%
\pgfpathclose%
\pgfusepath{stroke,fill}%
\end{pgfscope}%
\begin{pgfscope}%
\pgfpathrectangle{\pgfqpoint{0.100000in}{0.212622in}}{\pgfqpoint{3.696000in}{3.696000in}}%
\pgfusepath{clip}%
\pgfsetbuttcap%
\pgfsetroundjoin%
\definecolor{currentfill}{rgb}{0.121569,0.466667,0.705882}%
\pgfsetfillcolor{currentfill}%
\pgfsetfillopacity{0.493604}%
\pgfsetlinewidth{1.003750pt}%
\definecolor{currentstroke}{rgb}{0.121569,0.466667,0.705882}%
\pgfsetstrokecolor{currentstroke}%
\pgfsetstrokeopacity{0.493604}%
\pgfsetdash{}{0pt}%
\pgfpathmoveto{\pgfqpoint{1.076611in}{1.601568in}}%
\pgfpathcurveto{\pgfqpoint{1.084848in}{1.601568in}}{\pgfqpoint{1.092748in}{1.604840in}}{\pgfqpoint{1.098572in}{1.610664in}}%
\pgfpathcurveto{\pgfqpoint{1.104396in}{1.616488in}}{\pgfqpoint{1.107668in}{1.624388in}}{\pgfqpoint{1.107668in}{1.632624in}}%
\pgfpathcurveto{\pgfqpoint{1.107668in}{1.640861in}}{\pgfqpoint{1.104396in}{1.648761in}}{\pgfqpoint{1.098572in}{1.654585in}}%
\pgfpathcurveto{\pgfqpoint{1.092748in}{1.660409in}}{\pgfqpoint{1.084848in}{1.663681in}}{\pgfqpoint{1.076611in}{1.663681in}}%
\pgfpathcurveto{\pgfqpoint{1.068375in}{1.663681in}}{\pgfqpoint{1.060475in}{1.660409in}}{\pgfqpoint{1.054651in}{1.654585in}}%
\pgfpathcurveto{\pgfqpoint{1.048827in}{1.648761in}}{\pgfqpoint{1.045555in}{1.640861in}}{\pgfqpoint{1.045555in}{1.632624in}}%
\pgfpathcurveto{\pgfqpoint{1.045555in}{1.624388in}}{\pgfqpoint{1.048827in}{1.616488in}}{\pgfqpoint{1.054651in}{1.610664in}}%
\pgfpathcurveto{\pgfqpoint{1.060475in}{1.604840in}}{\pgfqpoint{1.068375in}{1.601568in}}{\pgfqpoint{1.076611in}{1.601568in}}%
\pgfpathclose%
\pgfusepath{stroke,fill}%
\end{pgfscope}%
\begin{pgfscope}%
\pgfpathrectangle{\pgfqpoint{0.100000in}{0.212622in}}{\pgfqpoint{3.696000in}{3.696000in}}%
\pgfusepath{clip}%
\pgfsetbuttcap%
\pgfsetroundjoin%
\definecolor{currentfill}{rgb}{0.121569,0.466667,0.705882}%
\pgfsetfillcolor{currentfill}%
\pgfsetfillopacity{0.494023}%
\pgfsetlinewidth{1.003750pt}%
\definecolor{currentstroke}{rgb}{0.121569,0.466667,0.705882}%
\pgfsetstrokecolor{currentstroke}%
\pgfsetstrokeopacity{0.494023}%
\pgfsetdash{}{0pt}%
\pgfpathmoveto{\pgfqpoint{1.556228in}{1.715099in}}%
\pgfpathcurveto{\pgfqpoint{1.564464in}{1.715099in}}{\pgfqpoint{1.572364in}{1.718371in}}{\pgfqpoint{1.578188in}{1.724195in}}%
\pgfpathcurveto{\pgfqpoint{1.584012in}{1.730019in}}{\pgfqpoint{1.587284in}{1.737919in}}{\pgfqpoint{1.587284in}{1.746156in}}%
\pgfpathcurveto{\pgfqpoint{1.587284in}{1.754392in}}{\pgfqpoint{1.584012in}{1.762292in}}{\pgfqpoint{1.578188in}{1.768116in}}%
\pgfpathcurveto{\pgfqpoint{1.572364in}{1.773940in}}{\pgfqpoint{1.564464in}{1.777212in}}{\pgfqpoint{1.556228in}{1.777212in}}%
\pgfpathcurveto{\pgfqpoint{1.547991in}{1.777212in}}{\pgfqpoint{1.540091in}{1.773940in}}{\pgfqpoint{1.534267in}{1.768116in}}%
\pgfpathcurveto{\pgfqpoint{1.528443in}{1.762292in}}{\pgfqpoint{1.525171in}{1.754392in}}{\pgfqpoint{1.525171in}{1.746156in}}%
\pgfpathcurveto{\pgfqpoint{1.525171in}{1.737919in}}{\pgfqpoint{1.528443in}{1.730019in}}{\pgfqpoint{1.534267in}{1.724195in}}%
\pgfpathcurveto{\pgfqpoint{1.540091in}{1.718371in}}{\pgfqpoint{1.547991in}{1.715099in}}{\pgfqpoint{1.556228in}{1.715099in}}%
\pgfpathclose%
\pgfusepath{stroke,fill}%
\end{pgfscope}%
\begin{pgfscope}%
\pgfpathrectangle{\pgfqpoint{0.100000in}{0.212622in}}{\pgfqpoint{3.696000in}{3.696000in}}%
\pgfusepath{clip}%
\pgfsetbuttcap%
\pgfsetroundjoin%
\definecolor{currentfill}{rgb}{0.121569,0.466667,0.705882}%
\pgfsetfillcolor{currentfill}%
\pgfsetfillopacity{0.496762}%
\pgfsetlinewidth{1.003750pt}%
\definecolor{currentstroke}{rgb}{0.121569,0.466667,0.705882}%
\pgfsetstrokecolor{currentstroke}%
\pgfsetstrokeopacity{0.496762}%
\pgfsetdash{}{0pt}%
\pgfpathmoveto{\pgfqpoint{1.072239in}{1.602173in}}%
\pgfpathcurveto{\pgfqpoint{1.080475in}{1.602173in}}{\pgfqpoint{1.088375in}{1.605445in}}{\pgfqpoint{1.094199in}{1.611269in}}%
\pgfpathcurveto{\pgfqpoint{1.100023in}{1.617093in}}{\pgfqpoint{1.103296in}{1.624993in}}{\pgfqpoint{1.103296in}{1.633230in}}%
\pgfpathcurveto{\pgfqpoint{1.103296in}{1.641466in}}{\pgfqpoint{1.100023in}{1.649366in}}{\pgfqpoint{1.094199in}{1.655190in}}%
\pgfpathcurveto{\pgfqpoint{1.088375in}{1.661014in}}{\pgfqpoint{1.080475in}{1.664286in}}{\pgfqpoint{1.072239in}{1.664286in}}%
\pgfpathcurveto{\pgfqpoint{1.064003in}{1.664286in}}{\pgfqpoint{1.056103in}{1.661014in}}{\pgfqpoint{1.050279in}{1.655190in}}%
\pgfpathcurveto{\pgfqpoint{1.044455in}{1.649366in}}{\pgfqpoint{1.041183in}{1.641466in}}{\pgfqpoint{1.041183in}{1.633230in}}%
\pgfpathcurveto{\pgfqpoint{1.041183in}{1.624993in}}{\pgfqpoint{1.044455in}{1.617093in}}{\pgfqpoint{1.050279in}{1.611269in}}%
\pgfpathcurveto{\pgfqpoint{1.056103in}{1.605445in}}{\pgfqpoint{1.064003in}{1.602173in}}{\pgfqpoint{1.072239in}{1.602173in}}%
\pgfpathclose%
\pgfusepath{stroke,fill}%
\end{pgfscope}%
\begin{pgfscope}%
\pgfpathrectangle{\pgfqpoint{0.100000in}{0.212622in}}{\pgfqpoint{3.696000in}{3.696000in}}%
\pgfusepath{clip}%
\pgfsetbuttcap%
\pgfsetroundjoin%
\definecolor{currentfill}{rgb}{0.121569,0.466667,0.705882}%
\pgfsetfillcolor{currentfill}%
\pgfsetfillopacity{0.497738}%
\pgfsetlinewidth{1.003750pt}%
\definecolor{currentstroke}{rgb}{0.121569,0.466667,0.705882}%
\pgfsetstrokecolor{currentstroke}%
\pgfsetstrokeopacity{0.497738}%
\pgfsetdash{}{0pt}%
\pgfpathmoveto{\pgfqpoint{1.556615in}{1.717622in}}%
\pgfpathcurveto{\pgfqpoint{1.564851in}{1.717622in}}{\pgfqpoint{1.572751in}{1.720894in}}{\pgfqpoint{1.578575in}{1.726718in}}%
\pgfpathcurveto{\pgfqpoint{1.584399in}{1.732542in}}{\pgfqpoint{1.587672in}{1.740442in}}{\pgfqpoint{1.587672in}{1.748679in}}%
\pgfpathcurveto{\pgfqpoint{1.587672in}{1.756915in}}{\pgfqpoint{1.584399in}{1.764815in}}{\pgfqpoint{1.578575in}{1.770639in}}%
\pgfpathcurveto{\pgfqpoint{1.572751in}{1.776463in}}{\pgfqpoint{1.564851in}{1.779735in}}{\pgfqpoint{1.556615in}{1.779735in}}%
\pgfpathcurveto{\pgfqpoint{1.548379in}{1.779735in}}{\pgfqpoint{1.540479in}{1.776463in}}{\pgfqpoint{1.534655in}{1.770639in}}%
\pgfpathcurveto{\pgfqpoint{1.528831in}{1.764815in}}{\pgfqpoint{1.525559in}{1.756915in}}{\pgfqpoint{1.525559in}{1.748679in}}%
\pgfpathcurveto{\pgfqpoint{1.525559in}{1.740442in}}{\pgfqpoint{1.528831in}{1.732542in}}{\pgfqpoint{1.534655in}{1.726718in}}%
\pgfpathcurveto{\pgfqpoint{1.540479in}{1.720894in}}{\pgfqpoint{1.548379in}{1.717622in}}{\pgfqpoint{1.556615in}{1.717622in}}%
\pgfpathclose%
\pgfusepath{stroke,fill}%
\end{pgfscope}%
\begin{pgfscope}%
\pgfpathrectangle{\pgfqpoint{0.100000in}{0.212622in}}{\pgfqpoint{3.696000in}{3.696000in}}%
\pgfusepath{clip}%
\pgfsetbuttcap%
\pgfsetroundjoin%
\definecolor{currentfill}{rgb}{0.121569,0.466667,0.705882}%
\pgfsetfillcolor{currentfill}%
\pgfsetfillopacity{0.498035}%
\pgfsetlinewidth{1.003750pt}%
\definecolor{currentstroke}{rgb}{0.121569,0.466667,0.705882}%
\pgfsetstrokecolor{currentstroke}%
\pgfsetstrokeopacity{0.498035}%
\pgfsetdash{}{0pt}%
\pgfpathmoveto{\pgfqpoint{1.068174in}{1.600156in}}%
\pgfpathcurveto{\pgfqpoint{1.076410in}{1.600156in}}{\pgfqpoint{1.084310in}{1.603429in}}{\pgfqpoint{1.090134in}{1.609253in}}%
\pgfpathcurveto{\pgfqpoint{1.095958in}{1.615076in}}{\pgfqpoint{1.099231in}{1.622977in}}{\pgfqpoint{1.099231in}{1.631213in}}%
\pgfpathcurveto{\pgfqpoint{1.099231in}{1.639449in}}{\pgfqpoint{1.095958in}{1.647349in}}{\pgfqpoint{1.090134in}{1.653173in}}%
\pgfpathcurveto{\pgfqpoint{1.084310in}{1.658997in}}{\pgfqpoint{1.076410in}{1.662269in}}{\pgfqpoint{1.068174in}{1.662269in}}%
\pgfpathcurveto{\pgfqpoint{1.059938in}{1.662269in}}{\pgfqpoint{1.052038in}{1.658997in}}{\pgfqpoint{1.046214in}{1.653173in}}%
\pgfpathcurveto{\pgfqpoint{1.040390in}{1.647349in}}{\pgfqpoint{1.037118in}{1.639449in}}{\pgfqpoint{1.037118in}{1.631213in}}%
\pgfpathcurveto{\pgfqpoint{1.037118in}{1.622977in}}{\pgfqpoint{1.040390in}{1.615076in}}{\pgfqpoint{1.046214in}{1.609253in}}%
\pgfpathcurveto{\pgfqpoint{1.052038in}{1.603429in}}{\pgfqpoint{1.059938in}{1.600156in}}{\pgfqpoint{1.068174in}{1.600156in}}%
\pgfpathclose%
\pgfusepath{stroke,fill}%
\end{pgfscope}%
\begin{pgfscope}%
\pgfpathrectangle{\pgfqpoint{0.100000in}{0.212622in}}{\pgfqpoint{3.696000in}{3.696000in}}%
\pgfusepath{clip}%
\pgfsetbuttcap%
\pgfsetroundjoin%
\definecolor{currentfill}{rgb}{0.121569,0.466667,0.705882}%
\pgfsetfillcolor{currentfill}%
\pgfsetfillopacity{0.498709}%
\pgfsetlinewidth{1.003750pt}%
\definecolor{currentstroke}{rgb}{0.121569,0.466667,0.705882}%
\pgfsetstrokecolor{currentstroke}%
\pgfsetstrokeopacity{0.498709}%
\pgfsetdash{}{0pt}%
\pgfpathmoveto{\pgfqpoint{1.066123in}{1.598254in}}%
\pgfpathcurveto{\pgfqpoint{1.074359in}{1.598254in}}{\pgfqpoint{1.082259in}{1.601526in}}{\pgfqpoint{1.088083in}{1.607350in}}%
\pgfpathcurveto{\pgfqpoint{1.093907in}{1.613174in}}{\pgfqpoint{1.097179in}{1.621074in}}{\pgfqpoint{1.097179in}{1.629310in}}%
\pgfpathcurveto{\pgfqpoint{1.097179in}{1.637546in}}{\pgfqpoint{1.093907in}{1.645447in}}{\pgfqpoint{1.088083in}{1.651270in}}%
\pgfpathcurveto{\pgfqpoint{1.082259in}{1.657094in}}{\pgfqpoint{1.074359in}{1.660367in}}{\pgfqpoint{1.066123in}{1.660367in}}%
\pgfpathcurveto{\pgfqpoint{1.057887in}{1.660367in}}{\pgfqpoint{1.049987in}{1.657094in}}{\pgfqpoint{1.044163in}{1.651270in}}%
\pgfpathcurveto{\pgfqpoint{1.038339in}{1.645447in}}{\pgfqpoint{1.035066in}{1.637546in}}{\pgfqpoint{1.035066in}{1.629310in}}%
\pgfpathcurveto{\pgfqpoint{1.035066in}{1.621074in}}{\pgfqpoint{1.038339in}{1.613174in}}{\pgfqpoint{1.044163in}{1.607350in}}%
\pgfpathcurveto{\pgfqpoint{1.049987in}{1.601526in}}{\pgfqpoint{1.057887in}{1.598254in}}{\pgfqpoint{1.066123in}{1.598254in}}%
\pgfpathclose%
\pgfusepath{stroke,fill}%
\end{pgfscope}%
\begin{pgfscope}%
\pgfpathrectangle{\pgfqpoint{0.100000in}{0.212622in}}{\pgfqpoint{3.696000in}{3.696000in}}%
\pgfusepath{clip}%
\pgfsetbuttcap%
\pgfsetroundjoin%
\definecolor{currentfill}{rgb}{0.121569,0.466667,0.705882}%
\pgfsetfillcolor{currentfill}%
\pgfsetfillopacity{0.500722}%
\pgfsetlinewidth{1.003750pt}%
\definecolor{currentstroke}{rgb}{0.121569,0.466667,0.705882}%
\pgfsetstrokecolor{currentstroke}%
\pgfsetstrokeopacity{0.500722}%
\pgfsetdash{}{0pt}%
\pgfpathmoveto{\pgfqpoint{1.062524in}{1.598098in}}%
\pgfpathcurveto{\pgfqpoint{1.070760in}{1.598098in}}{\pgfqpoint{1.078660in}{1.601370in}}{\pgfqpoint{1.084484in}{1.607194in}}%
\pgfpathcurveto{\pgfqpoint{1.090308in}{1.613018in}}{\pgfqpoint{1.093580in}{1.620918in}}{\pgfqpoint{1.093580in}{1.629155in}}%
\pgfpathcurveto{\pgfqpoint{1.093580in}{1.637391in}}{\pgfqpoint{1.090308in}{1.645291in}}{\pgfqpoint{1.084484in}{1.651115in}}%
\pgfpathcurveto{\pgfqpoint{1.078660in}{1.656939in}}{\pgfqpoint{1.070760in}{1.660211in}}{\pgfqpoint{1.062524in}{1.660211in}}%
\pgfpathcurveto{\pgfqpoint{1.054288in}{1.660211in}}{\pgfqpoint{1.046388in}{1.656939in}}{\pgfqpoint{1.040564in}{1.651115in}}%
\pgfpathcurveto{\pgfqpoint{1.034740in}{1.645291in}}{\pgfqpoint{1.031467in}{1.637391in}}{\pgfqpoint{1.031467in}{1.629155in}}%
\pgfpathcurveto{\pgfqpoint{1.031467in}{1.620918in}}{\pgfqpoint{1.034740in}{1.613018in}}{\pgfqpoint{1.040564in}{1.607194in}}%
\pgfpathcurveto{\pgfqpoint{1.046388in}{1.601370in}}{\pgfqpoint{1.054288in}{1.598098in}}{\pgfqpoint{1.062524in}{1.598098in}}%
\pgfpathclose%
\pgfusepath{stroke,fill}%
\end{pgfscope}%
\begin{pgfscope}%
\pgfpathrectangle{\pgfqpoint{0.100000in}{0.212622in}}{\pgfqpoint{3.696000in}{3.696000in}}%
\pgfusepath{clip}%
\pgfsetbuttcap%
\pgfsetroundjoin%
\definecolor{currentfill}{rgb}{0.121569,0.466667,0.705882}%
\pgfsetfillcolor{currentfill}%
\pgfsetfillopacity{0.500917}%
\pgfsetlinewidth{1.003750pt}%
\definecolor{currentstroke}{rgb}{0.121569,0.466667,0.705882}%
\pgfsetstrokecolor{currentstroke}%
\pgfsetstrokeopacity{0.500917}%
\pgfsetdash{}{0pt}%
\pgfpathmoveto{\pgfqpoint{1.558219in}{1.715067in}}%
\pgfpathcurveto{\pgfqpoint{1.566455in}{1.715067in}}{\pgfqpoint{1.574355in}{1.718339in}}{\pgfqpoint{1.580179in}{1.724163in}}%
\pgfpathcurveto{\pgfqpoint{1.586003in}{1.729987in}}{\pgfqpoint{1.589275in}{1.737887in}}{\pgfqpoint{1.589275in}{1.746123in}}%
\pgfpathcurveto{\pgfqpoint{1.589275in}{1.754359in}}{\pgfqpoint{1.586003in}{1.762259in}}{\pgfqpoint{1.580179in}{1.768083in}}%
\pgfpathcurveto{\pgfqpoint{1.574355in}{1.773907in}}{\pgfqpoint{1.566455in}{1.777180in}}{\pgfqpoint{1.558219in}{1.777180in}}%
\pgfpathcurveto{\pgfqpoint{1.549983in}{1.777180in}}{\pgfqpoint{1.542083in}{1.773907in}}{\pgfqpoint{1.536259in}{1.768083in}}%
\pgfpathcurveto{\pgfqpoint{1.530435in}{1.762259in}}{\pgfqpoint{1.527162in}{1.754359in}}{\pgfqpoint{1.527162in}{1.746123in}}%
\pgfpathcurveto{\pgfqpoint{1.527162in}{1.737887in}}{\pgfqpoint{1.530435in}{1.729987in}}{\pgfqpoint{1.536259in}{1.724163in}}%
\pgfpathcurveto{\pgfqpoint{1.542083in}{1.718339in}}{\pgfqpoint{1.549983in}{1.715067in}}{\pgfqpoint{1.558219in}{1.715067in}}%
\pgfpathclose%
\pgfusepath{stroke,fill}%
\end{pgfscope}%
\begin{pgfscope}%
\pgfpathrectangle{\pgfqpoint{0.100000in}{0.212622in}}{\pgfqpoint{3.696000in}{3.696000in}}%
\pgfusepath{clip}%
\pgfsetbuttcap%
\pgfsetroundjoin%
\definecolor{currentfill}{rgb}{0.121569,0.466667,0.705882}%
\pgfsetfillcolor{currentfill}%
\pgfsetfillopacity{0.501594}%
\pgfsetlinewidth{1.003750pt}%
\definecolor{currentstroke}{rgb}{0.121569,0.466667,0.705882}%
\pgfsetstrokecolor{currentstroke}%
\pgfsetstrokeopacity{0.501594}%
\pgfsetdash{}{0pt}%
\pgfpathmoveto{\pgfqpoint{1.059972in}{1.596172in}}%
\pgfpathcurveto{\pgfqpoint{1.068209in}{1.596172in}}{\pgfqpoint{1.076109in}{1.599444in}}{\pgfqpoint{1.081933in}{1.605268in}}%
\pgfpathcurveto{\pgfqpoint{1.087756in}{1.611092in}}{\pgfqpoint{1.091029in}{1.618992in}}{\pgfqpoint{1.091029in}{1.627229in}}%
\pgfpathcurveto{\pgfqpoint{1.091029in}{1.635465in}}{\pgfqpoint{1.087756in}{1.643365in}}{\pgfqpoint{1.081933in}{1.649189in}}%
\pgfpathcurveto{\pgfqpoint{1.076109in}{1.655013in}}{\pgfqpoint{1.068209in}{1.658285in}}{\pgfqpoint{1.059972in}{1.658285in}}%
\pgfpathcurveto{\pgfqpoint{1.051736in}{1.658285in}}{\pgfqpoint{1.043836in}{1.655013in}}{\pgfqpoint{1.038012in}{1.649189in}}%
\pgfpathcurveto{\pgfqpoint{1.032188in}{1.643365in}}{\pgfqpoint{1.028916in}{1.635465in}}{\pgfqpoint{1.028916in}{1.627229in}}%
\pgfpathcurveto{\pgfqpoint{1.028916in}{1.618992in}}{\pgfqpoint{1.032188in}{1.611092in}}{\pgfqpoint{1.038012in}{1.605268in}}%
\pgfpathcurveto{\pgfqpoint{1.043836in}{1.599444in}}{\pgfqpoint{1.051736in}{1.596172in}}{\pgfqpoint{1.059972in}{1.596172in}}%
\pgfpathclose%
\pgfusepath{stroke,fill}%
\end{pgfscope}%
\begin{pgfscope}%
\pgfpathrectangle{\pgfqpoint{0.100000in}{0.212622in}}{\pgfqpoint{3.696000in}{3.696000in}}%
\pgfusepath{clip}%
\pgfsetbuttcap%
\pgfsetroundjoin%
\definecolor{currentfill}{rgb}{0.121569,0.466667,0.705882}%
\pgfsetfillcolor{currentfill}%
\pgfsetfillopacity{0.503943}%
\pgfsetlinewidth{1.003750pt}%
\definecolor{currentstroke}{rgb}{0.121569,0.466667,0.705882}%
\pgfsetstrokecolor{currentstroke}%
\pgfsetstrokeopacity{0.503943}%
\pgfsetdash{}{0pt}%
\pgfpathmoveto{\pgfqpoint{1.055553in}{1.595710in}}%
\pgfpathcurveto{\pgfqpoint{1.063790in}{1.595710in}}{\pgfqpoint{1.071690in}{1.598982in}}{\pgfqpoint{1.077514in}{1.604806in}}%
\pgfpathcurveto{\pgfqpoint{1.083337in}{1.610630in}}{\pgfqpoint{1.086610in}{1.618530in}}{\pgfqpoint{1.086610in}{1.626766in}}%
\pgfpathcurveto{\pgfqpoint{1.086610in}{1.635003in}}{\pgfqpoint{1.083337in}{1.642903in}}{\pgfqpoint{1.077514in}{1.648727in}}%
\pgfpathcurveto{\pgfqpoint{1.071690in}{1.654551in}}{\pgfqpoint{1.063790in}{1.657823in}}{\pgfqpoint{1.055553in}{1.657823in}}%
\pgfpathcurveto{\pgfqpoint{1.047317in}{1.657823in}}{\pgfqpoint{1.039417in}{1.654551in}}{\pgfqpoint{1.033593in}{1.648727in}}%
\pgfpathcurveto{\pgfqpoint{1.027769in}{1.642903in}}{\pgfqpoint{1.024497in}{1.635003in}}{\pgfqpoint{1.024497in}{1.626766in}}%
\pgfpathcurveto{\pgfqpoint{1.024497in}{1.618530in}}{\pgfqpoint{1.027769in}{1.610630in}}{\pgfqpoint{1.033593in}{1.604806in}}%
\pgfpathcurveto{\pgfqpoint{1.039417in}{1.598982in}}{\pgfqpoint{1.047317in}{1.595710in}}{\pgfqpoint{1.055553in}{1.595710in}}%
\pgfpathclose%
\pgfusepath{stroke,fill}%
\end{pgfscope}%
\begin{pgfscope}%
\pgfpathrectangle{\pgfqpoint{0.100000in}{0.212622in}}{\pgfqpoint{3.696000in}{3.696000in}}%
\pgfusepath{clip}%
\pgfsetbuttcap%
\pgfsetroundjoin%
\definecolor{currentfill}{rgb}{0.121569,0.466667,0.705882}%
\pgfsetfillcolor{currentfill}%
\pgfsetfillopacity{0.504793}%
\pgfsetlinewidth{1.003750pt}%
\definecolor{currentstroke}{rgb}{0.121569,0.466667,0.705882}%
\pgfsetstrokecolor{currentstroke}%
\pgfsetstrokeopacity{0.504793}%
\pgfsetdash{}{0pt}%
\pgfpathmoveto{\pgfqpoint{1.559784in}{1.713619in}}%
\pgfpathcurveto{\pgfqpoint{1.568021in}{1.713619in}}{\pgfqpoint{1.575921in}{1.716891in}}{\pgfqpoint{1.581745in}{1.722715in}}%
\pgfpathcurveto{\pgfqpoint{1.587569in}{1.728539in}}{\pgfqpoint{1.590841in}{1.736439in}}{\pgfqpoint{1.590841in}{1.744675in}}%
\pgfpathcurveto{\pgfqpoint{1.590841in}{1.752912in}}{\pgfqpoint{1.587569in}{1.760812in}}{\pgfqpoint{1.581745in}{1.766636in}}%
\pgfpathcurveto{\pgfqpoint{1.575921in}{1.772460in}}{\pgfqpoint{1.568021in}{1.775732in}}{\pgfqpoint{1.559784in}{1.775732in}}%
\pgfpathcurveto{\pgfqpoint{1.551548in}{1.775732in}}{\pgfqpoint{1.543648in}{1.772460in}}{\pgfqpoint{1.537824in}{1.766636in}}%
\pgfpathcurveto{\pgfqpoint{1.532000in}{1.760812in}}{\pgfqpoint{1.528728in}{1.752912in}}{\pgfqpoint{1.528728in}{1.744675in}}%
\pgfpathcurveto{\pgfqpoint{1.528728in}{1.736439in}}{\pgfqpoint{1.532000in}{1.728539in}}{\pgfqpoint{1.537824in}{1.722715in}}%
\pgfpathcurveto{\pgfqpoint{1.543648in}{1.716891in}}{\pgfqpoint{1.551548in}{1.713619in}}{\pgfqpoint{1.559784in}{1.713619in}}%
\pgfpathclose%
\pgfusepath{stroke,fill}%
\end{pgfscope}%
\begin{pgfscope}%
\pgfpathrectangle{\pgfqpoint{0.100000in}{0.212622in}}{\pgfqpoint{3.696000in}{3.696000in}}%
\pgfusepath{clip}%
\pgfsetbuttcap%
\pgfsetroundjoin%
\definecolor{currentfill}{rgb}{0.121569,0.466667,0.705882}%
\pgfsetfillcolor{currentfill}%
\pgfsetfillopacity{0.508104}%
\pgfsetlinewidth{1.003750pt}%
\definecolor{currentstroke}{rgb}{0.121569,0.466667,0.705882}%
\pgfsetstrokecolor{currentstroke}%
\pgfsetstrokeopacity{0.508104}%
\pgfsetdash{}{0pt}%
\pgfpathmoveto{\pgfqpoint{1.048328in}{1.593268in}}%
\pgfpathcurveto{\pgfqpoint{1.056564in}{1.593268in}}{\pgfqpoint{1.064464in}{1.596540in}}{\pgfqpoint{1.070288in}{1.602364in}}%
\pgfpathcurveto{\pgfqpoint{1.076112in}{1.608188in}}{\pgfqpoint{1.079384in}{1.616088in}}{\pgfqpoint{1.079384in}{1.624325in}}%
\pgfpathcurveto{\pgfqpoint{1.079384in}{1.632561in}}{\pgfqpoint{1.076112in}{1.640461in}}{\pgfqpoint{1.070288in}{1.646285in}}%
\pgfpathcurveto{\pgfqpoint{1.064464in}{1.652109in}}{\pgfqpoint{1.056564in}{1.655381in}}{\pgfqpoint{1.048328in}{1.655381in}}%
\pgfpathcurveto{\pgfqpoint{1.040092in}{1.655381in}}{\pgfqpoint{1.032192in}{1.652109in}}{\pgfqpoint{1.026368in}{1.646285in}}%
\pgfpathcurveto{\pgfqpoint{1.020544in}{1.640461in}}{\pgfqpoint{1.017271in}{1.632561in}}{\pgfqpoint{1.017271in}{1.624325in}}%
\pgfpathcurveto{\pgfqpoint{1.017271in}{1.616088in}}{\pgfqpoint{1.020544in}{1.608188in}}{\pgfqpoint{1.026368in}{1.602364in}}%
\pgfpathcurveto{\pgfqpoint{1.032192in}{1.596540in}}{\pgfqpoint{1.040092in}{1.593268in}}{\pgfqpoint{1.048328in}{1.593268in}}%
\pgfpathclose%
\pgfusepath{stroke,fill}%
\end{pgfscope}%
\begin{pgfscope}%
\pgfpathrectangle{\pgfqpoint{0.100000in}{0.212622in}}{\pgfqpoint{3.696000in}{3.696000in}}%
\pgfusepath{clip}%
\pgfsetbuttcap%
\pgfsetroundjoin%
\definecolor{currentfill}{rgb}{0.121569,0.466667,0.705882}%
\pgfsetfillcolor{currentfill}%
\pgfsetfillopacity{0.509051}%
\pgfsetlinewidth{1.003750pt}%
\definecolor{currentstroke}{rgb}{0.121569,0.466667,0.705882}%
\pgfsetstrokecolor{currentstroke}%
\pgfsetstrokeopacity{0.509051}%
\pgfsetdash{}{0pt}%
\pgfpathmoveto{\pgfqpoint{1.560427in}{1.711281in}}%
\pgfpathcurveto{\pgfqpoint{1.568664in}{1.711281in}}{\pgfqpoint{1.576564in}{1.714553in}}{\pgfqpoint{1.582388in}{1.720377in}}%
\pgfpathcurveto{\pgfqpoint{1.588212in}{1.726201in}}{\pgfqpoint{1.591484in}{1.734101in}}{\pgfqpoint{1.591484in}{1.742337in}}%
\pgfpathcurveto{\pgfqpoint{1.591484in}{1.750574in}}{\pgfqpoint{1.588212in}{1.758474in}}{\pgfqpoint{1.582388in}{1.764298in}}%
\pgfpathcurveto{\pgfqpoint{1.576564in}{1.770122in}}{\pgfqpoint{1.568664in}{1.773394in}}{\pgfqpoint{1.560427in}{1.773394in}}%
\pgfpathcurveto{\pgfqpoint{1.552191in}{1.773394in}}{\pgfqpoint{1.544291in}{1.770122in}}{\pgfqpoint{1.538467in}{1.764298in}}%
\pgfpathcurveto{\pgfqpoint{1.532643in}{1.758474in}}{\pgfqpoint{1.529371in}{1.750574in}}{\pgfqpoint{1.529371in}{1.742337in}}%
\pgfpathcurveto{\pgfqpoint{1.529371in}{1.734101in}}{\pgfqpoint{1.532643in}{1.726201in}}{\pgfqpoint{1.538467in}{1.720377in}}%
\pgfpathcurveto{\pgfqpoint{1.544291in}{1.714553in}}{\pgfqpoint{1.552191in}{1.711281in}}{\pgfqpoint{1.560427in}{1.711281in}}%
\pgfpathclose%
\pgfusepath{stroke,fill}%
\end{pgfscope}%
\begin{pgfscope}%
\pgfpathrectangle{\pgfqpoint{0.100000in}{0.212622in}}{\pgfqpoint{3.696000in}{3.696000in}}%
\pgfusepath{clip}%
\pgfsetbuttcap%
\pgfsetroundjoin%
\definecolor{currentfill}{rgb}{0.121569,0.466667,0.705882}%
\pgfsetfillcolor{currentfill}%
\pgfsetfillopacity{0.510184}%
\pgfsetlinewidth{1.003750pt}%
\definecolor{currentstroke}{rgb}{0.121569,0.466667,0.705882}%
\pgfsetstrokecolor{currentstroke}%
\pgfsetstrokeopacity{0.510184}%
\pgfsetdash{}{0pt}%
\pgfpathmoveto{\pgfqpoint{1.040740in}{1.587668in}}%
\pgfpathcurveto{\pgfqpoint{1.048976in}{1.587668in}}{\pgfqpoint{1.056876in}{1.590940in}}{\pgfqpoint{1.062700in}{1.596764in}}%
\pgfpathcurveto{\pgfqpoint{1.068524in}{1.602588in}}{\pgfqpoint{1.071796in}{1.610488in}}{\pgfqpoint{1.071796in}{1.618724in}}%
\pgfpathcurveto{\pgfqpoint{1.071796in}{1.626960in}}{\pgfqpoint{1.068524in}{1.634860in}}{\pgfqpoint{1.062700in}{1.640684in}}%
\pgfpathcurveto{\pgfqpoint{1.056876in}{1.646508in}}{\pgfqpoint{1.048976in}{1.649781in}}{\pgfqpoint{1.040740in}{1.649781in}}%
\pgfpathcurveto{\pgfqpoint{1.032504in}{1.649781in}}{\pgfqpoint{1.024603in}{1.646508in}}{\pgfqpoint{1.018780in}{1.640684in}}%
\pgfpathcurveto{\pgfqpoint{1.012956in}{1.634860in}}{\pgfqpoint{1.009683in}{1.626960in}}{\pgfqpoint{1.009683in}{1.618724in}}%
\pgfpathcurveto{\pgfqpoint{1.009683in}{1.610488in}}{\pgfqpoint{1.012956in}{1.602588in}}{\pgfqpoint{1.018780in}{1.596764in}}%
\pgfpathcurveto{\pgfqpoint{1.024603in}{1.590940in}}{\pgfqpoint{1.032504in}{1.587668in}}{\pgfqpoint{1.040740in}{1.587668in}}%
\pgfpathclose%
\pgfusepath{stroke,fill}%
\end{pgfscope}%
\begin{pgfscope}%
\pgfpathrectangle{\pgfqpoint{0.100000in}{0.212622in}}{\pgfqpoint{3.696000in}{3.696000in}}%
\pgfusepath{clip}%
\pgfsetbuttcap%
\pgfsetroundjoin%
\definecolor{currentfill}{rgb}{0.121569,0.466667,0.705882}%
\pgfsetfillcolor{currentfill}%
\pgfsetfillopacity{0.512537}%
\pgfsetlinewidth{1.003750pt}%
\definecolor{currentstroke}{rgb}{0.121569,0.466667,0.705882}%
\pgfsetstrokecolor{currentstroke}%
\pgfsetstrokeopacity{0.512537}%
\pgfsetdash{}{0pt}%
\pgfpathmoveto{\pgfqpoint{1.034671in}{1.585646in}}%
\pgfpathcurveto{\pgfqpoint{1.042908in}{1.585646in}}{\pgfqpoint{1.050808in}{1.588919in}}{\pgfqpoint{1.056632in}{1.594743in}}%
\pgfpathcurveto{\pgfqpoint{1.062456in}{1.600567in}}{\pgfqpoint{1.065728in}{1.608467in}}{\pgfqpoint{1.065728in}{1.616703in}}%
\pgfpathcurveto{\pgfqpoint{1.065728in}{1.624939in}}{\pgfqpoint{1.062456in}{1.632839in}}{\pgfqpoint{1.056632in}{1.638663in}}%
\pgfpathcurveto{\pgfqpoint{1.050808in}{1.644487in}}{\pgfqpoint{1.042908in}{1.647759in}}{\pgfqpoint{1.034671in}{1.647759in}}%
\pgfpathcurveto{\pgfqpoint{1.026435in}{1.647759in}}{\pgfqpoint{1.018535in}{1.644487in}}{\pgfqpoint{1.012711in}{1.638663in}}%
\pgfpathcurveto{\pgfqpoint{1.006887in}{1.632839in}}{\pgfqpoint{1.003615in}{1.624939in}}{\pgfqpoint{1.003615in}{1.616703in}}%
\pgfpathcurveto{\pgfqpoint{1.003615in}{1.608467in}}{\pgfqpoint{1.006887in}{1.600567in}}{\pgfqpoint{1.012711in}{1.594743in}}%
\pgfpathcurveto{\pgfqpoint{1.018535in}{1.588919in}}{\pgfqpoint{1.026435in}{1.585646in}}{\pgfqpoint{1.034671in}{1.585646in}}%
\pgfpathclose%
\pgfusepath{stroke,fill}%
\end{pgfscope}%
\begin{pgfscope}%
\pgfpathrectangle{\pgfqpoint{0.100000in}{0.212622in}}{\pgfqpoint{3.696000in}{3.696000in}}%
\pgfusepath{clip}%
\pgfsetbuttcap%
\pgfsetroundjoin%
\definecolor{currentfill}{rgb}{0.121569,0.466667,0.705882}%
\pgfsetfillcolor{currentfill}%
\pgfsetfillopacity{0.514057}%
\pgfsetlinewidth{1.003750pt}%
\definecolor{currentstroke}{rgb}{0.121569,0.466667,0.705882}%
\pgfsetstrokecolor{currentstroke}%
\pgfsetstrokeopacity{0.514057}%
\pgfsetdash{}{0pt}%
\pgfpathmoveto{\pgfqpoint{1.562478in}{1.708533in}}%
\pgfpathcurveto{\pgfqpoint{1.570714in}{1.708533in}}{\pgfqpoint{1.578614in}{1.711806in}}{\pgfqpoint{1.584438in}{1.717629in}}%
\pgfpathcurveto{\pgfqpoint{1.590262in}{1.723453in}}{\pgfqpoint{1.593535in}{1.731353in}}{\pgfqpoint{1.593535in}{1.739590in}}%
\pgfpathcurveto{\pgfqpoint{1.593535in}{1.747826in}}{\pgfqpoint{1.590262in}{1.755726in}}{\pgfqpoint{1.584438in}{1.761550in}}%
\pgfpathcurveto{\pgfqpoint{1.578614in}{1.767374in}}{\pgfqpoint{1.570714in}{1.770646in}}{\pgfqpoint{1.562478in}{1.770646in}}%
\pgfpathcurveto{\pgfqpoint{1.554242in}{1.770646in}}{\pgfqpoint{1.546342in}{1.767374in}}{\pgfqpoint{1.540518in}{1.761550in}}%
\pgfpathcurveto{\pgfqpoint{1.534694in}{1.755726in}}{\pgfqpoint{1.531422in}{1.747826in}}{\pgfqpoint{1.531422in}{1.739590in}}%
\pgfpathcurveto{\pgfqpoint{1.531422in}{1.731353in}}{\pgfqpoint{1.534694in}{1.723453in}}{\pgfqpoint{1.540518in}{1.717629in}}%
\pgfpathcurveto{\pgfqpoint{1.546342in}{1.711806in}}{\pgfqpoint{1.554242in}{1.708533in}}{\pgfqpoint{1.562478in}{1.708533in}}%
\pgfpathclose%
\pgfusepath{stroke,fill}%
\end{pgfscope}%
\begin{pgfscope}%
\pgfpathrectangle{\pgfqpoint{0.100000in}{0.212622in}}{\pgfqpoint{3.696000in}{3.696000in}}%
\pgfusepath{clip}%
\pgfsetbuttcap%
\pgfsetroundjoin%
\definecolor{currentfill}{rgb}{0.121569,0.466667,0.705882}%
\pgfsetfillcolor{currentfill}%
\pgfsetfillopacity{0.517157}%
\pgfsetlinewidth{1.003750pt}%
\definecolor{currentstroke}{rgb}{0.121569,0.466667,0.705882}%
\pgfsetstrokecolor{currentstroke}%
\pgfsetstrokeopacity{0.517157}%
\pgfsetdash{}{0pt}%
\pgfpathmoveto{\pgfqpoint{1.563427in}{1.708558in}}%
\pgfpathcurveto{\pgfqpoint{1.571663in}{1.708558in}}{\pgfqpoint{1.579564in}{1.711830in}}{\pgfqpoint{1.585387in}{1.717654in}}%
\pgfpathcurveto{\pgfqpoint{1.591211in}{1.723478in}}{\pgfqpoint{1.594484in}{1.731378in}}{\pgfqpoint{1.594484in}{1.739614in}}%
\pgfpathcurveto{\pgfqpoint{1.594484in}{1.747850in}}{\pgfqpoint{1.591211in}{1.755750in}}{\pgfqpoint{1.585387in}{1.761574in}}%
\pgfpathcurveto{\pgfqpoint{1.579564in}{1.767398in}}{\pgfqpoint{1.571663in}{1.770671in}}{\pgfqpoint{1.563427in}{1.770671in}}%
\pgfpathcurveto{\pgfqpoint{1.555191in}{1.770671in}}{\pgfqpoint{1.547291in}{1.767398in}}{\pgfqpoint{1.541467in}{1.761574in}}%
\pgfpathcurveto{\pgfqpoint{1.535643in}{1.755750in}}{\pgfqpoint{1.532371in}{1.747850in}}{\pgfqpoint{1.532371in}{1.739614in}}%
\pgfpathcurveto{\pgfqpoint{1.532371in}{1.731378in}}{\pgfqpoint{1.535643in}{1.723478in}}{\pgfqpoint{1.541467in}{1.717654in}}%
\pgfpathcurveto{\pgfqpoint{1.547291in}{1.711830in}}{\pgfqpoint{1.555191in}{1.708558in}}{\pgfqpoint{1.563427in}{1.708558in}}%
\pgfpathclose%
\pgfusepath{stroke,fill}%
\end{pgfscope}%
\begin{pgfscope}%
\pgfpathrectangle{\pgfqpoint{0.100000in}{0.212622in}}{\pgfqpoint{3.696000in}{3.696000in}}%
\pgfusepath{clip}%
\pgfsetbuttcap%
\pgfsetroundjoin%
\definecolor{currentfill}{rgb}{0.121569,0.466667,0.705882}%
\pgfsetfillcolor{currentfill}%
\pgfsetfillopacity{0.518309}%
\pgfsetlinewidth{1.003750pt}%
\definecolor{currentstroke}{rgb}{0.121569,0.466667,0.705882}%
\pgfsetstrokecolor{currentstroke}%
\pgfsetstrokeopacity{0.518309}%
\pgfsetdash{}{0pt}%
\pgfpathmoveto{\pgfqpoint{1.025981in}{1.585138in}}%
\pgfpathcurveto{\pgfqpoint{1.034217in}{1.585138in}}{\pgfqpoint{1.042117in}{1.588411in}}{\pgfqpoint{1.047941in}{1.594235in}}%
\pgfpathcurveto{\pgfqpoint{1.053765in}{1.600059in}}{\pgfqpoint{1.057037in}{1.607959in}}{\pgfqpoint{1.057037in}{1.616195in}}%
\pgfpathcurveto{\pgfqpoint{1.057037in}{1.624431in}}{\pgfqpoint{1.053765in}{1.632331in}}{\pgfqpoint{1.047941in}{1.638155in}}%
\pgfpathcurveto{\pgfqpoint{1.042117in}{1.643979in}}{\pgfqpoint{1.034217in}{1.647251in}}{\pgfqpoint{1.025981in}{1.647251in}}%
\pgfpathcurveto{\pgfqpoint{1.017744in}{1.647251in}}{\pgfqpoint{1.009844in}{1.643979in}}{\pgfqpoint{1.004020in}{1.638155in}}%
\pgfpathcurveto{\pgfqpoint{0.998196in}{1.632331in}}{\pgfqpoint{0.994924in}{1.624431in}}{\pgfqpoint{0.994924in}{1.616195in}}%
\pgfpathcurveto{\pgfqpoint{0.994924in}{1.607959in}}{\pgfqpoint{0.998196in}{1.600059in}}{\pgfqpoint{1.004020in}{1.594235in}}%
\pgfpathcurveto{\pgfqpoint{1.009844in}{1.588411in}}{\pgfqpoint{1.017744in}{1.585138in}}{\pgfqpoint{1.025981in}{1.585138in}}%
\pgfpathclose%
\pgfusepath{stroke,fill}%
\end{pgfscope}%
\begin{pgfscope}%
\pgfpathrectangle{\pgfqpoint{0.100000in}{0.212622in}}{\pgfqpoint{3.696000in}{3.696000in}}%
\pgfusepath{clip}%
\pgfsetbuttcap%
\pgfsetroundjoin%
\definecolor{currentfill}{rgb}{0.121569,0.466667,0.705882}%
\pgfsetfillcolor{currentfill}%
\pgfsetfillopacity{0.521084}%
\pgfsetlinewidth{1.003750pt}%
\definecolor{currentstroke}{rgb}{0.121569,0.466667,0.705882}%
\pgfsetstrokecolor{currentstroke}%
\pgfsetstrokeopacity{0.521084}%
\pgfsetdash{}{0pt}%
\pgfpathmoveto{\pgfqpoint{1.565078in}{1.709690in}}%
\pgfpathcurveto{\pgfqpoint{1.573314in}{1.709690in}}{\pgfqpoint{1.581214in}{1.712962in}}{\pgfqpoint{1.587038in}{1.718786in}}%
\pgfpathcurveto{\pgfqpoint{1.592862in}{1.724610in}}{\pgfqpoint{1.596134in}{1.732510in}}{\pgfqpoint{1.596134in}{1.740746in}}%
\pgfpathcurveto{\pgfqpoint{1.596134in}{1.748982in}}{\pgfqpoint{1.592862in}{1.756882in}}{\pgfqpoint{1.587038in}{1.762706in}}%
\pgfpathcurveto{\pgfqpoint{1.581214in}{1.768530in}}{\pgfqpoint{1.573314in}{1.771803in}}{\pgfqpoint{1.565078in}{1.771803in}}%
\pgfpathcurveto{\pgfqpoint{1.556841in}{1.771803in}}{\pgfqpoint{1.548941in}{1.768530in}}{\pgfqpoint{1.543117in}{1.762706in}}%
\pgfpathcurveto{\pgfqpoint{1.537293in}{1.756882in}}{\pgfqpoint{1.534021in}{1.748982in}}{\pgfqpoint{1.534021in}{1.740746in}}%
\pgfpathcurveto{\pgfqpoint{1.534021in}{1.732510in}}{\pgfqpoint{1.537293in}{1.724610in}}{\pgfqpoint{1.543117in}{1.718786in}}%
\pgfpathcurveto{\pgfqpoint{1.548941in}{1.712962in}}{\pgfqpoint{1.556841in}{1.709690in}}{\pgfqpoint{1.565078in}{1.709690in}}%
\pgfpathclose%
\pgfusepath{stroke,fill}%
\end{pgfscope}%
\begin{pgfscope}%
\pgfpathrectangle{\pgfqpoint{0.100000in}{0.212622in}}{\pgfqpoint{3.696000in}{3.696000in}}%
\pgfusepath{clip}%
\pgfsetbuttcap%
\pgfsetroundjoin%
\definecolor{currentfill}{rgb}{0.121569,0.466667,0.705882}%
\pgfsetfillcolor{currentfill}%
\pgfsetfillopacity{0.521290}%
\pgfsetlinewidth{1.003750pt}%
\definecolor{currentstroke}{rgb}{0.121569,0.466667,0.705882}%
\pgfsetstrokecolor{currentstroke}%
\pgfsetstrokeopacity{0.521290}%
\pgfsetdash{}{0pt}%
\pgfpathmoveto{\pgfqpoint{1.016974in}{1.581117in}}%
\pgfpathcurveto{\pgfqpoint{1.025211in}{1.581117in}}{\pgfqpoint{1.033111in}{1.584390in}}{\pgfqpoint{1.038935in}{1.590214in}}%
\pgfpathcurveto{\pgfqpoint{1.044759in}{1.596038in}}{\pgfqpoint{1.048031in}{1.603938in}}{\pgfqpoint{1.048031in}{1.612174in}}%
\pgfpathcurveto{\pgfqpoint{1.048031in}{1.620410in}}{\pgfqpoint{1.044759in}{1.628310in}}{\pgfqpoint{1.038935in}{1.634134in}}%
\pgfpathcurveto{\pgfqpoint{1.033111in}{1.639958in}}{\pgfqpoint{1.025211in}{1.643230in}}{\pgfqpoint{1.016974in}{1.643230in}}%
\pgfpathcurveto{\pgfqpoint{1.008738in}{1.643230in}}{\pgfqpoint{1.000838in}{1.639958in}}{\pgfqpoint{0.995014in}{1.634134in}}%
\pgfpathcurveto{\pgfqpoint{0.989190in}{1.628310in}}{\pgfqpoint{0.985918in}{1.620410in}}{\pgfqpoint{0.985918in}{1.612174in}}%
\pgfpathcurveto{\pgfqpoint{0.985918in}{1.603938in}}{\pgfqpoint{0.989190in}{1.596038in}}{\pgfqpoint{0.995014in}{1.590214in}}%
\pgfpathcurveto{\pgfqpoint{1.000838in}{1.584390in}}{\pgfqpoint{1.008738in}{1.581117in}}{\pgfqpoint{1.016974in}{1.581117in}}%
\pgfpathclose%
\pgfusepath{stroke,fill}%
\end{pgfscope}%
\begin{pgfscope}%
\pgfpathrectangle{\pgfqpoint{0.100000in}{0.212622in}}{\pgfqpoint{3.696000in}{3.696000in}}%
\pgfusepath{clip}%
\pgfsetbuttcap%
\pgfsetroundjoin%
\definecolor{currentfill}{rgb}{0.121569,0.466667,0.705882}%
\pgfsetfillcolor{currentfill}%
\pgfsetfillopacity{0.524161}%
\pgfsetlinewidth{1.003750pt}%
\definecolor{currentstroke}{rgb}{0.121569,0.466667,0.705882}%
\pgfsetstrokecolor{currentstroke}%
\pgfsetstrokeopacity{0.524161}%
\pgfsetdash{}{0pt}%
\pgfpathmoveto{\pgfqpoint{1.010095in}{1.577611in}}%
\pgfpathcurveto{\pgfqpoint{1.018331in}{1.577611in}}{\pgfqpoint{1.026231in}{1.580883in}}{\pgfqpoint{1.032055in}{1.586707in}}%
\pgfpathcurveto{\pgfqpoint{1.037879in}{1.592531in}}{\pgfqpoint{1.041151in}{1.600431in}}{\pgfqpoint{1.041151in}{1.608667in}}%
\pgfpathcurveto{\pgfqpoint{1.041151in}{1.616904in}}{\pgfqpoint{1.037879in}{1.624804in}}{\pgfqpoint{1.032055in}{1.630628in}}%
\pgfpathcurveto{\pgfqpoint{1.026231in}{1.636452in}}{\pgfqpoint{1.018331in}{1.639724in}}{\pgfqpoint{1.010095in}{1.639724in}}%
\pgfpathcurveto{\pgfqpoint{1.001859in}{1.639724in}}{\pgfqpoint{0.993959in}{1.636452in}}{\pgfqpoint{0.988135in}{1.630628in}}%
\pgfpathcurveto{\pgfqpoint{0.982311in}{1.624804in}}{\pgfqpoint{0.979038in}{1.616904in}}{\pgfqpoint{0.979038in}{1.608667in}}%
\pgfpathcurveto{\pgfqpoint{0.979038in}{1.600431in}}{\pgfqpoint{0.982311in}{1.592531in}}{\pgfqpoint{0.988135in}{1.586707in}}%
\pgfpathcurveto{\pgfqpoint{0.993959in}{1.580883in}}{\pgfqpoint{1.001859in}{1.577611in}}{\pgfqpoint{1.010095in}{1.577611in}}%
\pgfpathclose%
\pgfusepath{stroke,fill}%
\end{pgfscope}%
\begin{pgfscope}%
\pgfpathrectangle{\pgfqpoint{0.100000in}{0.212622in}}{\pgfqpoint{3.696000in}{3.696000in}}%
\pgfusepath{clip}%
\pgfsetbuttcap%
\pgfsetroundjoin%
\definecolor{currentfill}{rgb}{0.121569,0.466667,0.705882}%
\pgfsetfillcolor{currentfill}%
\pgfsetfillopacity{0.525124}%
\pgfsetlinewidth{1.003750pt}%
\definecolor{currentstroke}{rgb}{0.121569,0.466667,0.705882}%
\pgfsetstrokecolor{currentstroke}%
\pgfsetstrokeopacity{0.525124}%
\pgfsetdash{}{0pt}%
\pgfpathmoveto{\pgfqpoint{1.566419in}{1.709721in}}%
\pgfpathcurveto{\pgfqpoint{1.574655in}{1.709721in}}{\pgfqpoint{1.582556in}{1.712993in}}{\pgfqpoint{1.588379in}{1.718817in}}%
\pgfpathcurveto{\pgfqpoint{1.594203in}{1.724641in}}{\pgfqpoint{1.597476in}{1.732541in}}{\pgfqpoint{1.597476in}{1.740777in}}%
\pgfpathcurveto{\pgfqpoint{1.597476in}{1.749014in}}{\pgfqpoint{1.594203in}{1.756914in}}{\pgfqpoint{1.588379in}{1.762738in}}%
\pgfpathcurveto{\pgfqpoint{1.582556in}{1.768562in}}{\pgfqpoint{1.574655in}{1.771834in}}{\pgfqpoint{1.566419in}{1.771834in}}%
\pgfpathcurveto{\pgfqpoint{1.558183in}{1.771834in}}{\pgfqpoint{1.550283in}{1.768562in}}{\pgfqpoint{1.544459in}{1.762738in}}%
\pgfpathcurveto{\pgfqpoint{1.538635in}{1.756914in}}{\pgfqpoint{1.535363in}{1.749014in}}{\pgfqpoint{1.535363in}{1.740777in}}%
\pgfpathcurveto{\pgfqpoint{1.535363in}{1.732541in}}{\pgfqpoint{1.538635in}{1.724641in}}{\pgfqpoint{1.544459in}{1.718817in}}%
\pgfpathcurveto{\pgfqpoint{1.550283in}{1.712993in}}{\pgfqpoint{1.558183in}{1.709721in}}{\pgfqpoint{1.566419in}{1.709721in}}%
\pgfpathclose%
\pgfusepath{stroke,fill}%
\end{pgfscope}%
\begin{pgfscope}%
\pgfpathrectangle{\pgfqpoint{0.100000in}{0.212622in}}{\pgfqpoint{3.696000in}{3.696000in}}%
\pgfusepath{clip}%
\pgfsetbuttcap%
\pgfsetroundjoin%
\definecolor{currentfill}{rgb}{0.121569,0.466667,0.705882}%
\pgfsetfillcolor{currentfill}%
\pgfsetfillopacity{0.527917}%
\pgfsetlinewidth{1.003750pt}%
\definecolor{currentstroke}{rgb}{0.121569,0.466667,0.705882}%
\pgfsetstrokecolor{currentstroke}%
\pgfsetstrokeopacity{0.527917}%
\pgfsetdash{}{0pt}%
\pgfpathmoveto{\pgfqpoint{1.004414in}{1.577226in}}%
\pgfpathcurveto{\pgfqpoint{1.012650in}{1.577226in}}{\pgfqpoint{1.020550in}{1.580498in}}{\pgfqpoint{1.026374in}{1.586322in}}%
\pgfpathcurveto{\pgfqpoint{1.032198in}{1.592146in}}{\pgfqpoint{1.035470in}{1.600046in}}{\pgfqpoint{1.035470in}{1.608282in}}%
\pgfpathcurveto{\pgfqpoint{1.035470in}{1.616518in}}{\pgfqpoint{1.032198in}{1.624418in}}{\pgfqpoint{1.026374in}{1.630242in}}%
\pgfpathcurveto{\pgfqpoint{1.020550in}{1.636066in}}{\pgfqpoint{1.012650in}{1.639339in}}{\pgfqpoint{1.004414in}{1.639339in}}%
\pgfpathcurveto{\pgfqpoint{0.996177in}{1.639339in}}{\pgfqpoint{0.988277in}{1.636066in}}{\pgfqpoint{0.982453in}{1.630242in}}%
\pgfpathcurveto{\pgfqpoint{0.976629in}{1.624418in}}{\pgfqpoint{0.973357in}{1.616518in}}{\pgfqpoint{0.973357in}{1.608282in}}%
\pgfpathcurveto{\pgfqpoint{0.973357in}{1.600046in}}{\pgfqpoint{0.976629in}{1.592146in}}{\pgfqpoint{0.982453in}{1.586322in}}%
\pgfpathcurveto{\pgfqpoint{0.988277in}{1.580498in}}{\pgfqpoint{0.996177in}{1.577226in}}{\pgfqpoint{1.004414in}{1.577226in}}%
\pgfpathclose%
\pgfusepath{stroke,fill}%
\end{pgfscope}%
\begin{pgfscope}%
\pgfpathrectangle{\pgfqpoint{0.100000in}{0.212622in}}{\pgfqpoint{3.696000in}{3.696000in}}%
\pgfusepath{clip}%
\pgfsetbuttcap%
\pgfsetroundjoin%
\definecolor{currentfill}{rgb}{0.121569,0.466667,0.705882}%
\pgfsetfillcolor{currentfill}%
\pgfsetfillopacity{0.529208}%
\pgfsetlinewidth{1.003750pt}%
\definecolor{currentstroke}{rgb}{0.121569,0.466667,0.705882}%
\pgfsetstrokecolor{currentstroke}%
\pgfsetstrokeopacity{0.529208}%
\pgfsetdash{}{0pt}%
\pgfpathmoveto{\pgfqpoint{0.999446in}{1.572545in}}%
\pgfpathcurveto{\pgfqpoint{1.007682in}{1.572545in}}{\pgfqpoint{1.015582in}{1.575817in}}{\pgfqpoint{1.021406in}{1.581641in}}%
\pgfpathcurveto{\pgfqpoint{1.027230in}{1.587465in}}{\pgfqpoint{1.030502in}{1.595365in}}{\pgfqpoint{1.030502in}{1.603602in}}%
\pgfpathcurveto{\pgfqpoint{1.030502in}{1.611838in}}{\pgfqpoint{1.027230in}{1.619738in}}{\pgfqpoint{1.021406in}{1.625562in}}%
\pgfpathcurveto{\pgfqpoint{1.015582in}{1.631386in}}{\pgfqpoint{1.007682in}{1.634658in}}{\pgfqpoint{0.999446in}{1.634658in}}%
\pgfpathcurveto{\pgfqpoint{0.991209in}{1.634658in}}{\pgfqpoint{0.983309in}{1.631386in}}{\pgfqpoint{0.977485in}{1.625562in}}%
\pgfpathcurveto{\pgfqpoint{0.971661in}{1.619738in}}{\pgfqpoint{0.968389in}{1.611838in}}{\pgfqpoint{0.968389in}{1.603602in}}%
\pgfpathcurveto{\pgfqpoint{0.968389in}{1.595365in}}{\pgfqpoint{0.971661in}{1.587465in}}{\pgfqpoint{0.977485in}{1.581641in}}%
\pgfpathcurveto{\pgfqpoint{0.983309in}{1.575817in}}{\pgfqpoint{0.991209in}{1.572545in}}{\pgfqpoint{0.999446in}{1.572545in}}%
\pgfpathclose%
\pgfusepath{stroke,fill}%
\end{pgfscope}%
\begin{pgfscope}%
\pgfpathrectangle{\pgfqpoint{0.100000in}{0.212622in}}{\pgfqpoint{3.696000in}{3.696000in}}%
\pgfusepath{clip}%
\pgfsetbuttcap%
\pgfsetroundjoin%
\definecolor{currentfill}{rgb}{0.121569,0.466667,0.705882}%
\pgfsetfillcolor{currentfill}%
\pgfsetfillopacity{0.530011}%
\pgfsetlinewidth{1.003750pt}%
\definecolor{currentstroke}{rgb}{0.121569,0.466667,0.705882}%
\pgfsetstrokecolor{currentstroke}%
\pgfsetstrokeopacity{0.530011}%
\pgfsetdash{}{0pt}%
\pgfpathmoveto{\pgfqpoint{1.567748in}{1.711015in}}%
\pgfpathcurveto{\pgfqpoint{1.575985in}{1.711015in}}{\pgfqpoint{1.583885in}{1.714288in}}{\pgfqpoint{1.589709in}{1.720112in}}%
\pgfpathcurveto{\pgfqpoint{1.595532in}{1.725935in}}{\pgfqpoint{1.598805in}{1.733836in}}{\pgfqpoint{1.598805in}{1.742072in}}%
\pgfpathcurveto{\pgfqpoint{1.598805in}{1.750308in}}{\pgfqpoint{1.595532in}{1.758208in}}{\pgfqpoint{1.589709in}{1.764032in}}%
\pgfpathcurveto{\pgfqpoint{1.583885in}{1.769856in}}{\pgfqpoint{1.575985in}{1.773128in}}{\pgfqpoint{1.567748in}{1.773128in}}%
\pgfpathcurveto{\pgfqpoint{1.559512in}{1.773128in}}{\pgfqpoint{1.551612in}{1.769856in}}{\pgfqpoint{1.545788in}{1.764032in}}%
\pgfpathcurveto{\pgfqpoint{1.539964in}{1.758208in}}{\pgfqpoint{1.536692in}{1.750308in}}{\pgfqpoint{1.536692in}{1.742072in}}%
\pgfpathcurveto{\pgfqpoint{1.536692in}{1.733836in}}{\pgfqpoint{1.539964in}{1.725935in}}{\pgfqpoint{1.545788in}{1.720112in}}%
\pgfpathcurveto{\pgfqpoint{1.551612in}{1.714288in}}{\pgfqpoint{1.559512in}{1.711015in}}{\pgfqpoint{1.567748in}{1.711015in}}%
\pgfpathclose%
\pgfusepath{stroke,fill}%
\end{pgfscope}%
\begin{pgfscope}%
\pgfpathrectangle{\pgfqpoint{0.100000in}{0.212622in}}{\pgfqpoint{3.696000in}{3.696000in}}%
\pgfusepath{clip}%
\pgfsetbuttcap%
\pgfsetroundjoin%
\definecolor{currentfill}{rgb}{0.121569,0.466667,0.705882}%
\pgfsetfillcolor{currentfill}%
\pgfsetfillopacity{0.530257}%
\pgfsetlinewidth{1.003750pt}%
\definecolor{currentstroke}{rgb}{0.121569,0.466667,0.705882}%
\pgfsetstrokecolor{currentstroke}%
\pgfsetstrokeopacity{0.530257}%
\pgfsetdash{}{0pt}%
\pgfpathmoveto{\pgfqpoint{0.996389in}{1.569981in}}%
\pgfpathcurveto{\pgfqpoint{1.004625in}{1.569981in}}{\pgfqpoint{1.012525in}{1.573253in}}{\pgfqpoint{1.018349in}{1.579077in}}%
\pgfpathcurveto{\pgfqpoint{1.024173in}{1.584901in}}{\pgfqpoint{1.027445in}{1.592801in}}{\pgfqpoint{1.027445in}{1.601037in}}%
\pgfpathcurveto{\pgfqpoint{1.027445in}{1.609274in}}{\pgfqpoint{1.024173in}{1.617174in}}{\pgfqpoint{1.018349in}{1.622998in}}%
\pgfpathcurveto{\pgfqpoint{1.012525in}{1.628822in}}{\pgfqpoint{1.004625in}{1.632094in}}{\pgfqpoint{0.996389in}{1.632094in}}%
\pgfpathcurveto{\pgfqpoint{0.988152in}{1.632094in}}{\pgfqpoint{0.980252in}{1.628822in}}{\pgfqpoint{0.974428in}{1.622998in}}%
\pgfpathcurveto{\pgfqpoint{0.968604in}{1.617174in}}{\pgfqpoint{0.965332in}{1.609274in}}{\pgfqpoint{0.965332in}{1.601037in}}%
\pgfpathcurveto{\pgfqpoint{0.965332in}{1.592801in}}{\pgfqpoint{0.968604in}{1.584901in}}{\pgfqpoint{0.974428in}{1.579077in}}%
\pgfpathcurveto{\pgfqpoint{0.980252in}{1.573253in}}{\pgfqpoint{0.988152in}{1.569981in}}{\pgfqpoint{0.996389in}{1.569981in}}%
\pgfpathclose%
\pgfusepath{stroke,fill}%
\end{pgfscope}%
\begin{pgfscope}%
\pgfpathrectangle{\pgfqpoint{0.100000in}{0.212622in}}{\pgfqpoint{3.696000in}{3.696000in}}%
\pgfusepath{clip}%
\pgfsetbuttcap%
\pgfsetroundjoin%
\definecolor{currentfill}{rgb}{0.121569,0.466667,0.705882}%
\pgfsetfillcolor{currentfill}%
\pgfsetfillopacity{0.531073}%
\pgfsetlinewidth{1.003750pt}%
\definecolor{currentstroke}{rgb}{0.121569,0.466667,0.705882}%
\pgfsetstrokecolor{currentstroke}%
\pgfsetstrokeopacity{0.531073}%
\pgfsetdash{}{0pt}%
\pgfpathmoveto{\pgfqpoint{0.988107in}{1.558737in}}%
\pgfpathcurveto{\pgfqpoint{0.996343in}{1.558737in}}{\pgfqpoint{1.004243in}{1.562010in}}{\pgfqpoint{1.010067in}{1.567834in}}%
\pgfpathcurveto{\pgfqpoint{1.015891in}{1.573657in}}{\pgfqpoint{1.019163in}{1.581558in}}{\pgfqpoint{1.019163in}{1.589794in}}%
\pgfpathcurveto{\pgfqpoint{1.019163in}{1.598030in}}{\pgfqpoint{1.015891in}{1.605930in}}{\pgfqpoint{1.010067in}{1.611754in}}%
\pgfpathcurveto{\pgfqpoint{1.004243in}{1.617578in}}{\pgfqpoint{0.996343in}{1.620850in}}{\pgfqpoint{0.988107in}{1.620850in}}%
\pgfpathcurveto{\pgfqpoint{0.979870in}{1.620850in}}{\pgfqpoint{0.971970in}{1.617578in}}{\pgfqpoint{0.966146in}{1.611754in}}%
\pgfpathcurveto{\pgfqpoint{0.960322in}{1.605930in}}{\pgfqpoint{0.957050in}{1.598030in}}{\pgfqpoint{0.957050in}{1.589794in}}%
\pgfpathcurveto{\pgfqpoint{0.957050in}{1.581558in}}{\pgfqpoint{0.960322in}{1.573657in}}{\pgfqpoint{0.966146in}{1.567834in}}%
\pgfpathcurveto{\pgfqpoint{0.971970in}{1.562010in}}{\pgfqpoint{0.979870in}{1.558737in}}{\pgfqpoint{0.988107in}{1.558737in}}%
\pgfpathclose%
\pgfusepath{stroke,fill}%
\end{pgfscope}%
\begin{pgfscope}%
\pgfpathrectangle{\pgfqpoint{0.100000in}{0.212622in}}{\pgfqpoint{3.696000in}{3.696000in}}%
\pgfusepath{clip}%
\pgfsetbuttcap%
\pgfsetroundjoin%
\definecolor{currentfill}{rgb}{0.121569,0.466667,0.705882}%
\pgfsetfillcolor{currentfill}%
\pgfsetfillopacity{0.531465}%
\pgfsetlinewidth{1.003750pt}%
\definecolor{currentstroke}{rgb}{0.121569,0.466667,0.705882}%
\pgfsetstrokecolor{currentstroke}%
\pgfsetstrokeopacity{0.531465}%
\pgfsetdash{}{0pt}%
\pgfpathmoveto{\pgfqpoint{0.993865in}{1.568615in}}%
\pgfpathcurveto{\pgfqpoint{1.002101in}{1.568615in}}{\pgfqpoint{1.010001in}{1.571887in}}{\pgfqpoint{1.015825in}{1.577711in}}%
\pgfpathcurveto{\pgfqpoint{1.021649in}{1.583535in}}{\pgfqpoint{1.024922in}{1.591435in}}{\pgfqpoint{1.024922in}{1.599671in}}%
\pgfpathcurveto{\pgfqpoint{1.024922in}{1.607908in}}{\pgfqpoint{1.021649in}{1.615808in}}{\pgfqpoint{1.015825in}{1.621632in}}%
\pgfpathcurveto{\pgfqpoint{1.010001in}{1.627456in}}{\pgfqpoint{1.002101in}{1.630728in}}{\pgfqpoint{0.993865in}{1.630728in}}%
\pgfpathcurveto{\pgfqpoint{0.985629in}{1.630728in}}{\pgfqpoint{0.977729in}{1.627456in}}{\pgfqpoint{0.971905in}{1.621632in}}%
\pgfpathcurveto{\pgfqpoint{0.966081in}{1.615808in}}{\pgfqpoint{0.962809in}{1.607908in}}{\pgfqpoint{0.962809in}{1.599671in}}%
\pgfpathcurveto{\pgfqpoint{0.962809in}{1.591435in}}{\pgfqpoint{0.966081in}{1.583535in}}{\pgfqpoint{0.971905in}{1.577711in}}%
\pgfpathcurveto{\pgfqpoint{0.977729in}{1.571887in}}{\pgfqpoint{0.985629in}{1.568615in}}{\pgfqpoint{0.993865in}{1.568615in}}%
\pgfpathclose%
\pgfusepath{stroke,fill}%
\end{pgfscope}%
\begin{pgfscope}%
\pgfpathrectangle{\pgfqpoint{0.100000in}{0.212622in}}{\pgfqpoint{3.696000in}{3.696000in}}%
\pgfusepath{clip}%
\pgfsetbuttcap%
\pgfsetroundjoin%
\definecolor{currentfill}{rgb}{0.121569,0.466667,0.705882}%
\pgfsetfillcolor{currentfill}%
\pgfsetfillopacity{0.531516}%
\pgfsetlinewidth{1.003750pt}%
\definecolor{currentstroke}{rgb}{0.121569,0.466667,0.705882}%
\pgfsetstrokecolor{currentstroke}%
\pgfsetstrokeopacity{0.531516}%
\pgfsetdash{}{0pt}%
\pgfpathmoveto{\pgfqpoint{0.991945in}{1.565572in}}%
\pgfpathcurveto{\pgfqpoint{1.000181in}{1.565572in}}{\pgfqpoint{1.008081in}{1.568845in}}{\pgfqpoint{1.013905in}{1.574669in}}%
\pgfpathcurveto{\pgfqpoint{1.019729in}{1.580493in}}{\pgfqpoint{1.023001in}{1.588393in}}{\pgfqpoint{1.023001in}{1.596629in}}%
\pgfpathcurveto{\pgfqpoint{1.023001in}{1.604865in}}{\pgfqpoint{1.019729in}{1.612765in}}{\pgfqpoint{1.013905in}{1.618589in}}%
\pgfpathcurveto{\pgfqpoint{1.008081in}{1.624413in}}{\pgfqpoint{1.000181in}{1.627685in}}{\pgfqpoint{0.991945in}{1.627685in}}%
\pgfpathcurveto{\pgfqpoint{0.983709in}{1.627685in}}{\pgfqpoint{0.975808in}{1.624413in}}{\pgfqpoint{0.969985in}{1.618589in}}%
\pgfpathcurveto{\pgfqpoint{0.964161in}{1.612765in}}{\pgfqpoint{0.960888in}{1.604865in}}{\pgfqpoint{0.960888in}{1.596629in}}%
\pgfpathcurveto{\pgfqpoint{0.960888in}{1.588393in}}{\pgfqpoint{0.964161in}{1.580493in}}{\pgfqpoint{0.969985in}{1.574669in}}%
\pgfpathcurveto{\pgfqpoint{0.975808in}{1.568845in}}{\pgfqpoint{0.983709in}{1.565572in}}{\pgfqpoint{0.991945in}{1.565572in}}%
\pgfpathclose%
\pgfusepath{stroke,fill}%
\end{pgfscope}%
\begin{pgfscope}%
\pgfpathrectangle{\pgfqpoint{0.100000in}{0.212622in}}{\pgfqpoint{3.696000in}{3.696000in}}%
\pgfusepath{clip}%
\pgfsetbuttcap%
\pgfsetroundjoin%
\definecolor{currentfill}{rgb}{0.121569,0.466667,0.705882}%
\pgfsetfillcolor{currentfill}%
\pgfsetfillopacity{0.532161}%
\pgfsetlinewidth{1.003750pt}%
\definecolor{currentstroke}{rgb}{0.121569,0.466667,0.705882}%
\pgfsetstrokecolor{currentstroke}%
\pgfsetstrokeopacity{0.532161}%
\pgfsetdash{}{0pt}%
\pgfpathmoveto{\pgfqpoint{0.986508in}{1.558481in}}%
\pgfpathcurveto{\pgfqpoint{0.994744in}{1.558481in}}{\pgfqpoint{1.002645in}{1.561753in}}{\pgfqpoint{1.008468in}{1.567577in}}%
\pgfpathcurveto{\pgfqpoint{1.014292in}{1.573401in}}{\pgfqpoint{1.017565in}{1.581301in}}{\pgfqpoint{1.017565in}{1.589537in}}%
\pgfpathcurveto{\pgfqpoint{1.017565in}{1.597773in}}{\pgfqpoint{1.014292in}{1.605674in}}{\pgfqpoint{1.008468in}{1.611497in}}%
\pgfpathcurveto{\pgfqpoint{1.002645in}{1.617321in}}{\pgfqpoint{0.994744in}{1.620594in}}{\pgfqpoint{0.986508in}{1.620594in}}%
\pgfpathcurveto{\pgfqpoint{0.978272in}{1.620594in}}{\pgfqpoint{0.970372in}{1.617321in}}{\pgfqpoint{0.964548in}{1.611497in}}%
\pgfpathcurveto{\pgfqpoint{0.958724in}{1.605674in}}{\pgfqpoint{0.955452in}{1.597773in}}{\pgfqpoint{0.955452in}{1.589537in}}%
\pgfpathcurveto{\pgfqpoint{0.955452in}{1.581301in}}{\pgfqpoint{0.958724in}{1.573401in}}{\pgfqpoint{0.964548in}{1.567577in}}%
\pgfpathcurveto{\pgfqpoint{0.970372in}{1.561753in}}{\pgfqpoint{0.978272in}{1.558481in}}{\pgfqpoint{0.986508in}{1.558481in}}%
\pgfpathclose%
\pgfusepath{stroke,fill}%
\end{pgfscope}%
\begin{pgfscope}%
\pgfpathrectangle{\pgfqpoint{0.100000in}{0.212622in}}{\pgfqpoint{3.696000in}{3.696000in}}%
\pgfusepath{clip}%
\pgfsetbuttcap%
\pgfsetroundjoin%
\definecolor{currentfill}{rgb}{0.121569,0.466667,0.705882}%
\pgfsetfillcolor{currentfill}%
\pgfsetfillopacity{0.532313}%
\pgfsetlinewidth{1.003750pt}%
\definecolor{currentstroke}{rgb}{0.121569,0.466667,0.705882}%
\pgfsetstrokecolor{currentstroke}%
\pgfsetstrokeopacity{0.532313}%
\pgfsetdash{}{0pt}%
\pgfpathmoveto{\pgfqpoint{0.982606in}{1.552417in}}%
\pgfpathcurveto{\pgfqpoint{0.990842in}{1.552417in}}{\pgfqpoint{0.998742in}{1.555690in}}{\pgfqpoint{1.004566in}{1.561514in}}%
\pgfpathcurveto{\pgfqpoint{1.010390in}{1.567338in}}{\pgfqpoint{1.013662in}{1.575238in}}{\pgfqpoint{1.013662in}{1.583474in}}%
\pgfpathcurveto{\pgfqpoint{1.013662in}{1.591710in}}{\pgfqpoint{1.010390in}{1.599610in}}{\pgfqpoint{1.004566in}{1.605434in}}%
\pgfpathcurveto{\pgfqpoint{0.998742in}{1.611258in}}{\pgfqpoint{0.990842in}{1.614530in}}{\pgfqpoint{0.982606in}{1.614530in}}%
\pgfpathcurveto{\pgfqpoint{0.974370in}{1.614530in}}{\pgfqpoint{0.966470in}{1.611258in}}{\pgfqpoint{0.960646in}{1.605434in}}%
\pgfpathcurveto{\pgfqpoint{0.954822in}{1.599610in}}{\pgfqpoint{0.951549in}{1.591710in}}{\pgfqpoint{0.951549in}{1.583474in}}%
\pgfpathcurveto{\pgfqpoint{0.951549in}{1.575238in}}{\pgfqpoint{0.954822in}{1.567338in}}{\pgfqpoint{0.960646in}{1.561514in}}%
\pgfpathcurveto{\pgfqpoint{0.966470in}{1.555690in}}{\pgfqpoint{0.974370in}{1.552417in}}{\pgfqpoint{0.982606in}{1.552417in}}%
\pgfpathclose%
\pgfusepath{stroke,fill}%
\end{pgfscope}%
\begin{pgfscope}%
\pgfpathrectangle{\pgfqpoint{0.100000in}{0.212622in}}{\pgfqpoint{3.696000in}{3.696000in}}%
\pgfusepath{clip}%
\pgfsetbuttcap%
\pgfsetroundjoin%
\definecolor{currentfill}{rgb}{0.121569,0.466667,0.705882}%
\pgfsetfillcolor{currentfill}%
\pgfsetfillopacity{0.532346}%
\pgfsetlinewidth{1.003750pt}%
\definecolor{currentstroke}{rgb}{0.121569,0.466667,0.705882}%
\pgfsetstrokecolor{currentstroke}%
\pgfsetstrokeopacity{0.532346}%
\pgfsetdash{}{0pt}%
\pgfpathmoveto{\pgfqpoint{0.985089in}{1.556969in}}%
\pgfpathcurveto{\pgfqpoint{0.993326in}{1.556969in}}{\pgfqpoint{1.001226in}{1.560241in}}{\pgfqpoint{1.007050in}{1.566065in}}%
\pgfpathcurveto{\pgfqpoint{1.012873in}{1.571889in}}{\pgfqpoint{1.016146in}{1.579789in}}{\pgfqpoint{1.016146in}{1.588025in}}%
\pgfpathcurveto{\pgfqpoint{1.016146in}{1.596261in}}{\pgfqpoint{1.012873in}{1.604162in}}{\pgfqpoint{1.007050in}{1.609985in}}%
\pgfpathcurveto{\pgfqpoint{1.001226in}{1.615809in}}{\pgfqpoint{0.993326in}{1.619082in}}{\pgfqpoint{0.985089in}{1.619082in}}%
\pgfpathcurveto{\pgfqpoint{0.976853in}{1.619082in}}{\pgfqpoint{0.968953in}{1.615809in}}{\pgfqpoint{0.963129in}{1.609985in}}%
\pgfpathcurveto{\pgfqpoint{0.957305in}{1.604162in}}{\pgfqpoint{0.954033in}{1.596261in}}{\pgfqpoint{0.954033in}{1.588025in}}%
\pgfpathcurveto{\pgfqpoint{0.954033in}{1.579789in}}{\pgfqpoint{0.957305in}{1.571889in}}{\pgfqpoint{0.963129in}{1.566065in}}%
\pgfpathcurveto{\pgfqpoint{0.968953in}{1.560241in}}{\pgfqpoint{0.976853in}{1.556969in}}{\pgfqpoint{0.985089in}{1.556969in}}%
\pgfpathclose%
\pgfusepath{stroke,fill}%
\end{pgfscope}%
\begin{pgfscope}%
\pgfpathrectangle{\pgfqpoint{0.100000in}{0.212622in}}{\pgfqpoint{3.696000in}{3.696000in}}%
\pgfusepath{clip}%
\pgfsetbuttcap%
\pgfsetroundjoin%
\definecolor{currentfill}{rgb}{0.121569,0.466667,0.705882}%
\pgfsetfillcolor{currentfill}%
\pgfsetfillopacity{0.532968}%
\pgfsetlinewidth{1.003750pt}%
\definecolor{currentstroke}{rgb}{0.121569,0.466667,0.705882}%
\pgfsetstrokecolor{currentstroke}%
\pgfsetstrokeopacity{0.532968}%
\pgfsetdash{}{0pt}%
\pgfpathmoveto{\pgfqpoint{0.981704in}{1.552741in}}%
\pgfpathcurveto{\pgfqpoint{0.989940in}{1.552741in}}{\pgfqpoint{0.997840in}{1.556013in}}{\pgfqpoint{1.003664in}{1.561837in}}%
\pgfpathcurveto{\pgfqpoint{1.009488in}{1.567661in}}{\pgfqpoint{1.012760in}{1.575561in}}{\pgfqpoint{1.012760in}{1.583797in}}%
\pgfpathcurveto{\pgfqpoint{1.012760in}{1.592034in}}{\pgfqpoint{1.009488in}{1.599934in}}{\pgfqpoint{1.003664in}{1.605758in}}%
\pgfpathcurveto{\pgfqpoint{0.997840in}{1.611582in}}{\pgfqpoint{0.989940in}{1.614854in}}{\pgfqpoint{0.981704in}{1.614854in}}%
\pgfpathcurveto{\pgfqpoint{0.973467in}{1.614854in}}{\pgfqpoint{0.965567in}{1.611582in}}{\pgfqpoint{0.959743in}{1.605758in}}%
\pgfpathcurveto{\pgfqpoint{0.953920in}{1.599934in}}{\pgfqpoint{0.950647in}{1.592034in}}{\pgfqpoint{0.950647in}{1.583797in}}%
\pgfpathcurveto{\pgfqpoint{0.950647in}{1.575561in}}{\pgfqpoint{0.953920in}{1.567661in}}{\pgfqpoint{0.959743in}{1.561837in}}%
\pgfpathcurveto{\pgfqpoint{0.965567in}{1.556013in}}{\pgfqpoint{0.973467in}{1.552741in}}{\pgfqpoint{0.981704in}{1.552741in}}%
\pgfpathclose%
\pgfusepath{stroke,fill}%
\end{pgfscope}%
\begin{pgfscope}%
\pgfpathrectangle{\pgfqpoint{0.100000in}{0.212622in}}{\pgfqpoint{3.696000in}{3.696000in}}%
\pgfusepath{clip}%
\pgfsetbuttcap%
\pgfsetroundjoin%
\definecolor{currentfill}{rgb}{0.121569,0.466667,0.705882}%
\pgfsetfillcolor{currentfill}%
\pgfsetfillopacity{0.533540}%
\pgfsetlinewidth{1.003750pt}%
\definecolor{currentstroke}{rgb}{0.121569,0.466667,0.705882}%
\pgfsetstrokecolor{currentstroke}%
\pgfsetstrokeopacity{0.533540}%
\pgfsetdash{}{0pt}%
\pgfpathmoveto{\pgfqpoint{0.979723in}{1.551092in}}%
\pgfpathcurveto{\pgfqpoint{0.987960in}{1.551092in}}{\pgfqpoint{0.995860in}{1.554364in}}{\pgfqpoint{1.001684in}{1.560188in}}%
\pgfpathcurveto{\pgfqpoint{1.007508in}{1.566012in}}{\pgfqpoint{1.010780in}{1.573912in}}{\pgfqpoint{1.010780in}{1.582148in}}%
\pgfpathcurveto{\pgfqpoint{1.010780in}{1.590385in}}{\pgfqpoint{1.007508in}{1.598285in}}{\pgfqpoint{1.001684in}{1.604109in}}%
\pgfpathcurveto{\pgfqpoint{0.995860in}{1.609932in}}{\pgfqpoint{0.987960in}{1.613205in}}{\pgfqpoint{0.979723in}{1.613205in}}%
\pgfpathcurveto{\pgfqpoint{0.971487in}{1.613205in}}{\pgfqpoint{0.963587in}{1.609932in}}{\pgfqpoint{0.957763in}{1.604109in}}%
\pgfpathcurveto{\pgfqpoint{0.951939in}{1.598285in}}{\pgfqpoint{0.948667in}{1.590385in}}{\pgfqpoint{0.948667in}{1.582148in}}%
\pgfpathcurveto{\pgfqpoint{0.948667in}{1.573912in}}{\pgfqpoint{0.951939in}{1.566012in}}{\pgfqpoint{0.957763in}{1.560188in}}%
\pgfpathcurveto{\pgfqpoint{0.963587in}{1.554364in}}{\pgfqpoint{0.971487in}{1.551092in}}{\pgfqpoint{0.979723in}{1.551092in}}%
\pgfpathclose%
\pgfusepath{stroke,fill}%
\end{pgfscope}%
\begin{pgfscope}%
\pgfpathrectangle{\pgfqpoint{0.100000in}{0.212622in}}{\pgfqpoint{3.696000in}{3.696000in}}%
\pgfusepath{clip}%
\pgfsetbuttcap%
\pgfsetroundjoin%
\definecolor{currentfill}{rgb}{0.121569,0.466667,0.705882}%
\pgfsetfillcolor{currentfill}%
\pgfsetfillopacity{0.534716}%
\pgfsetlinewidth{1.003750pt}%
\definecolor{currentstroke}{rgb}{0.121569,0.466667,0.705882}%
\pgfsetstrokecolor{currentstroke}%
\pgfsetstrokeopacity{0.534716}%
\pgfsetdash{}{0pt}%
\pgfpathmoveto{\pgfqpoint{0.976616in}{1.547921in}}%
\pgfpathcurveto{\pgfqpoint{0.984853in}{1.547921in}}{\pgfqpoint{0.992753in}{1.551193in}}{\pgfqpoint{0.998577in}{1.557017in}}%
\pgfpathcurveto{\pgfqpoint{1.004401in}{1.562841in}}{\pgfqpoint{1.007673in}{1.570741in}}{\pgfqpoint{1.007673in}{1.578977in}}%
\pgfpathcurveto{\pgfqpoint{1.007673in}{1.587213in}}{\pgfqpoint{1.004401in}{1.595113in}}{\pgfqpoint{0.998577in}{1.600937in}}%
\pgfpathcurveto{\pgfqpoint{0.992753in}{1.606761in}}{\pgfqpoint{0.984853in}{1.610034in}}{\pgfqpoint{0.976616in}{1.610034in}}%
\pgfpathcurveto{\pgfqpoint{0.968380in}{1.610034in}}{\pgfqpoint{0.960480in}{1.606761in}}{\pgfqpoint{0.954656in}{1.600937in}}%
\pgfpathcurveto{\pgfqpoint{0.948832in}{1.595113in}}{\pgfqpoint{0.945560in}{1.587213in}}{\pgfqpoint{0.945560in}{1.578977in}}%
\pgfpathcurveto{\pgfqpoint{0.945560in}{1.570741in}}{\pgfqpoint{0.948832in}{1.562841in}}{\pgfqpoint{0.954656in}{1.557017in}}%
\pgfpathcurveto{\pgfqpoint{0.960480in}{1.551193in}}{\pgfqpoint{0.968380in}{1.547921in}}{\pgfqpoint{0.976616in}{1.547921in}}%
\pgfpathclose%
\pgfusepath{stroke,fill}%
\end{pgfscope}%
\begin{pgfscope}%
\pgfpathrectangle{\pgfqpoint{0.100000in}{0.212622in}}{\pgfqpoint{3.696000in}{3.696000in}}%
\pgfusepath{clip}%
\pgfsetbuttcap%
\pgfsetroundjoin%
\definecolor{currentfill}{rgb}{0.121569,0.466667,0.705882}%
\pgfsetfillcolor{currentfill}%
\pgfsetfillopacity{0.535467}%
\pgfsetlinewidth{1.003750pt}%
\definecolor{currentstroke}{rgb}{0.121569,0.466667,0.705882}%
\pgfsetstrokecolor{currentstroke}%
\pgfsetstrokeopacity{0.535467}%
\pgfsetdash{}{0pt}%
\pgfpathmoveto{\pgfqpoint{1.570980in}{1.714063in}}%
\pgfpathcurveto{\pgfqpoint{1.579216in}{1.714063in}}{\pgfqpoint{1.587116in}{1.717335in}}{\pgfqpoint{1.592940in}{1.723159in}}%
\pgfpathcurveto{\pgfqpoint{1.598764in}{1.728983in}}{\pgfqpoint{1.602037in}{1.736883in}}{\pgfqpoint{1.602037in}{1.745119in}}%
\pgfpathcurveto{\pgfqpoint{1.602037in}{1.753356in}}{\pgfqpoint{1.598764in}{1.761256in}}{\pgfqpoint{1.592940in}{1.767080in}}%
\pgfpathcurveto{\pgfqpoint{1.587116in}{1.772904in}}{\pgfqpoint{1.579216in}{1.776176in}}{\pgfqpoint{1.570980in}{1.776176in}}%
\pgfpathcurveto{\pgfqpoint{1.562744in}{1.776176in}}{\pgfqpoint{1.554844in}{1.772904in}}{\pgfqpoint{1.549020in}{1.767080in}}%
\pgfpathcurveto{\pgfqpoint{1.543196in}{1.761256in}}{\pgfqpoint{1.539924in}{1.753356in}}{\pgfqpoint{1.539924in}{1.745119in}}%
\pgfpathcurveto{\pgfqpoint{1.539924in}{1.736883in}}{\pgfqpoint{1.543196in}{1.728983in}}{\pgfqpoint{1.549020in}{1.723159in}}%
\pgfpathcurveto{\pgfqpoint{1.554844in}{1.717335in}}{\pgfqpoint{1.562744in}{1.714063in}}{\pgfqpoint{1.570980in}{1.714063in}}%
\pgfpathclose%
\pgfusepath{stroke,fill}%
\end{pgfscope}%
\begin{pgfscope}%
\pgfpathrectangle{\pgfqpoint{0.100000in}{0.212622in}}{\pgfqpoint{3.696000in}{3.696000in}}%
\pgfusepath{clip}%
\pgfsetbuttcap%
\pgfsetroundjoin%
\definecolor{currentfill}{rgb}{0.121569,0.466667,0.705882}%
\pgfsetfillcolor{currentfill}%
\pgfsetfillopacity{0.536296}%
\pgfsetlinewidth{1.003750pt}%
\definecolor{currentstroke}{rgb}{0.121569,0.466667,0.705882}%
\pgfsetstrokecolor{currentstroke}%
\pgfsetstrokeopacity{0.536296}%
\pgfsetdash{}{0pt}%
\pgfpathmoveto{\pgfqpoint{0.974880in}{1.549285in}}%
\pgfpathcurveto{\pgfqpoint{0.983116in}{1.549285in}}{\pgfqpoint{0.991016in}{1.552558in}}{\pgfqpoint{0.996840in}{1.558382in}}%
\pgfpathcurveto{\pgfqpoint{1.002664in}{1.564205in}}{\pgfqpoint{1.005936in}{1.572106in}}{\pgfqpoint{1.005936in}{1.580342in}}%
\pgfpathcurveto{\pgfqpoint{1.005936in}{1.588578in}}{\pgfqpoint{1.002664in}{1.596478in}}{\pgfqpoint{0.996840in}{1.602302in}}%
\pgfpathcurveto{\pgfqpoint{0.991016in}{1.608126in}}{\pgfqpoint{0.983116in}{1.611398in}}{\pgfqpoint{0.974880in}{1.611398in}}%
\pgfpathcurveto{\pgfqpoint{0.966644in}{1.611398in}}{\pgfqpoint{0.958744in}{1.608126in}}{\pgfqpoint{0.952920in}{1.602302in}}%
\pgfpathcurveto{\pgfqpoint{0.947096in}{1.596478in}}{\pgfqpoint{0.943823in}{1.588578in}}{\pgfqpoint{0.943823in}{1.580342in}}%
\pgfpathcurveto{\pgfqpoint{0.943823in}{1.572106in}}{\pgfqpoint{0.947096in}{1.564205in}}{\pgfqpoint{0.952920in}{1.558382in}}%
\pgfpathcurveto{\pgfqpoint{0.958744in}{1.552558in}}{\pgfqpoint{0.966644in}{1.549285in}}{\pgfqpoint{0.974880in}{1.549285in}}%
\pgfpathclose%
\pgfusepath{stroke,fill}%
\end{pgfscope}%
\begin{pgfscope}%
\pgfpathrectangle{\pgfqpoint{0.100000in}{0.212622in}}{\pgfqpoint{3.696000in}{3.696000in}}%
\pgfusepath{clip}%
\pgfsetbuttcap%
\pgfsetroundjoin%
\definecolor{currentfill}{rgb}{0.121569,0.466667,0.705882}%
\pgfsetfillcolor{currentfill}%
\pgfsetfillopacity{0.536888}%
\pgfsetlinewidth{1.003750pt}%
\definecolor{currentstroke}{rgb}{0.121569,0.466667,0.705882}%
\pgfsetstrokecolor{currentstroke}%
\pgfsetstrokeopacity{0.536888}%
\pgfsetdash{}{0pt}%
\pgfpathmoveto{\pgfqpoint{0.961776in}{1.530100in}}%
\pgfpathcurveto{\pgfqpoint{0.970012in}{1.530100in}}{\pgfqpoint{0.977912in}{1.533372in}}{\pgfqpoint{0.983736in}{1.539196in}}%
\pgfpathcurveto{\pgfqpoint{0.989560in}{1.545020in}}{\pgfqpoint{0.992833in}{1.552920in}}{\pgfqpoint{0.992833in}{1.561157in}}%
\pgfpathcurveto{\pgfqpoint{0.992833in}{1.569393in}}{\pgfqpoint{0.989560in}{1.577293in}}{\pgfqpoint{0.983736in}{1.583117in}}%
\pgfpathcurveto{\pgfqpoint{0.977912in}{1.588941in}}{\pgfqpoint{0.970012in}{1.592213in}}{\pgfqpoint{0.961776in}{1.592213in}}%
\pgfpathcurveto{\pgfqpoint{0.953540in}{1.592213in}}{\pgfqpoint{0.945640in}{1.588941in}}{\pgfqpoint{0.939816in}{1.583117in}}%
\pgfpathcurveto{\pgfqpoint{0.933992in}{1.577293in}}{\pgfqpoint{0.930720in}{1.569393in}}{\pgfqpoint{0.930720in}{1.561157in}}%
\pgfpathcurveto{\pgfqpoint{0.930720in}{1.552920in}}{\pgfqpoint{0.933992in}{1.545020in}}{\pgfqpoint{0.939816in}{1.539196in}}%
\pgfpathcurveto{\pgfqpoint{0.945640in}{1.533372in}}{\pgfqpoint{0.953540in}{1.530100in}}{\pgfqpoint{0.961776in}{1.530100in}}%
\pgfpathclose%
\pgfusepath{stroke,fill}%
\end{pgfscope}%
\begin{pgfscope}%
\pgfpathrectangle{\pgfqpoint{0.100000in}{0.212622in}}{\pgfqpoint{3.696000in}{3.696000in}}%
\pgfusepath{clip}%
\pgfsetbuttcap%
\pgfsetroundjoin%
\definecolor{currentfill}{rgb}{0.121569,0.466667,0.705882}%
\pgfsetfillcolor{currentfill}%
\pgfsetfillopacity{0.537196}%
\pgfsetlinewidth{1.003750pt}%
\definecolor{currentstroke}{rgb}{0.121569,0.466667,0.705882}%
\pgfsetstrokecolor{currentstroke}%
\pgfsetstrokeopacity{0.537196}%
\pgfsetdash{}{0pt}%
\pgfpathmoveto{\pgfqpoint{0.969951in}{1.546104in}}%
\pgfpathcurveto{\pgfqpoint{0.978188in}{1.546104in}}{\pgfqpoint{0.986088in}{1.549376in}}{\pgfqpoint{0.991912in}{1.555200in}}%
\pgfpathcurveto{\pgfqpoint{0.997735in}{1.561024in}}{\pgfqpoint{1.001008in}{1.568924in}}{\pgfqpoint{1.001008in}{1.577161in}}%
\pgfpathcurveto{\pgfqpoint{1.001008in}{1.585397in}}{\pgfqpoint{0.997735in}{1.593297in}}{\pgfqpoint{0.991912in}{1.599121in}}%
\pgfpathcurveto{\pgfqpoint{0.986088in}{1.604945in}}{\pgfqpoint{0.978188in}{1.608217in}}{\pgfqpoint{0.969951in}{1.608217in}}%
\pgfpathcurveto{\pgfqpoint{0.961715in}{1.608217in}}{\pgfqpoint{0.953815in}{1.604945in}}{\pgfqpoint{0.947991in}{1.599121in}}%
\pgfpathcurveto{\pgfqpoint{0.942167in}{1.593297in}}{\pgfqpoint{0.938895in}{1.585397in}}{\pgfqpoint{0.938895in}{1.577161in}}%
\pgfpathcurveto{\pgfqpoint{0.938895in}{1.568924in}}{\pgfqpoint{0.942167in}{1.561024in}}{\pgfqpoint{0.947991in}{1.555200in}}%
\pgfpathcurveto{\pgfqpoint{0.953815in}{1.549376in}}{\pgfqpoint{0.961715in}{1.546104in}}{\pgfqpoint{0.969951in}{1.546104in}}%
\pgfpathclose%
\pgfusepath{stroke,fill}%
\end{pgfscope}%
\begin{pgfscope}%
\pgfpathrectangle{\pgfqpoint{0.100000in}{0.212622in}}{\pgfqpoint{3.696000in}{3.696000in}}%
\pgfusepath{clip}%
\pgfsetbuttcap%
\pgfsetroundjoin%
\definecolor{currentfill}{rgb}{0.121569,0.466667,0.705882}%
\pgfsetfillcolor{currentfill}%
\pgfsetfillopacity{0.540421}%
\pgfsetlinewidth{1.003750pt}%
\definecolor{currentstroke}{rgb}{0.121569,0.466667,0.705882}%
\pgfsetstrokecolor{currentstroke}%
\pgfsetstrokeopacity{0.540421}%
\pgfsetdash{}{0pt}%
\pgfpathmoveto{\pgfqpoint{1.571628in}{1.711893in}}%
\pgfpathcurveto{\pgfqpoint{1.579864in}{1.711893in}}{\pgfqpoint{1.587764in}{1.715165in}}{\pgfqpoint{1.593588in}{1.720989in}}%
\pgfpathcurveto{\pgfqpoint{1.599412in}{1.726813in}}{\pgfqpoint{1.602685in}{1.734713in}}{\pgfqpoint{1.602685in}{1.742949in}}%
\pgfpathcurveto{\pgfqpoint{1.602685in}{1.751185in}}{\pgfqpoint{1.599412in}{1.759085in}}{\pgfqpoint{1.593588in}{1.764909in}}%
\pgfpathcurveto{\pgfqpoint{1.587764in}{1.770733in}}{\pgfqpoint{1.579864in}{1.774006in}}{\pgfqpoint{1.571628in}{1.774006in}}%
\pgfpathcurveto{\pgfqpoint{1.563392in}{1.774006in}}{\pgfqpoint{1.555492in}{1.770733in}}{\pgfqpoint{1.549668in}{1.764909in}}%
\pgfpathcurveto{\pgfqpoint{1.543844in}{1.759085in}}{\pgfqpoint{1.540572in}{1.751185in}}{\pgfqpoint{1.540572in}{1.742949in}}%
\pgfpathcurveto{\pgfqpoint{1.540572in}{1.734713in}}{\pgfqpoint{1.543844in}{1.726813in}}{\pgfqpoint{1.549668in}{1.720989in}}%
\pgfpathcurveto{\pgfqpoint{1.555492in}{1.715165in}}{\pgfqpoint{1.563392in}{1.711893in}}{\pgfqpoint{1.571628in}{1.711893in}}%
\pgfpathclose%
\pgfusepath{stroke,fill}%
\end{pgfscope}%
\begin{pgfscope}%
\pgfpathrectangle{\pgfqpoint{0.100000in}{0.212622in}}{\pgfqpoint{3.696000in}{3.696000in}}%
\pgfusepath{clip}%
\pgfsetbuttcap%
\pgfsetroundjoin%
\definecolor{currentfill}{rgb}{0.121569,0.466667,0.705882}%
\pgfsetfillcolor{currentfill}%
\pgfsetfillopacity{0.541069}%
\pgfsetlinewidth{1.003750pt}%
\definecolor{currentstroke}{rgb}{0.121569,0.466667,0.705882}%
\pgfsetstrokecolor{currentstroke}%
\pgfsetstrokeopacity{0.541069}%
\pgfsetdash{}{0pt}%
\pgfpathmoveto{\pgfqpoint{0.955894in}{1.530854in}}%
\pgfpathcurveto{\pgfqpoint{0.964130in}{1.530854in}}{\pgfqpoint{0.972030in}{1.534126in}}{\pgfqpoint{0.977854in}{1.539950in}}%
\pgfpathcurveto{\pgfqpoint{0.983678in}{1.545774in}}{\pgfqpoint{0.986950in}{1.553674in}}{\pgfqpoint{0.986950in}{1.561910in}}%
\pgfpathcurveto{\pgfqpoint{0.986950in}{1.570147in}}{\pgfqpoint{0.983678in}{1.578047in}}{\pgfqpoint{0.977854in}{1.583871in}}%
\pgfpathcurveto{\pgfqpoint{0.972030in}{1.589694in}}{\pgfqpoint{0.964130in}{1.592967in}}{\pgfqpoint{0.955894in}{1.592967in}}%
\pgfpathcurveto{\pgfqpoint{0.947657in}{1.592967in}}{\pgfqpoint{0.939757in}{1.589694in}}{\pgfqpoint{0.933933in}{1.583871in}}%
\pgfpathcurveto{\pgfqpoint{0.928109in}{1.578047in}}{\pgfqpoint{0.924837in}{1.570147in}}{\pgfqpoint{0.924837in}{1.561910in}}%
\pgfpathcurveto{\pgfqpoint{0.924837in}{1.553674in}}{\pgfqpoint{0.928109in}{1.545774in}}{\pgfqpoint{0.933933in}{1.539950in}}%
\pgfpathcurveto{\pgfqpoint{0.939757in}{1.534126in}}{\pgfqpoint{0.947657in}{1.530854in}}{\pgfqpoint{0.955894in}{1.530854in}}%
\pgfpathclose%
\pgfusepath{stroke,fill}%
\end{pgfscope}%
\begin{pgfscope}%
\pgfpathrectangle{\pgfqpoint{0.100000in}{0.212622in}}{\pgfqpoint{3.696000in}{3.696000in}}%
\pgfusepath{clip}%
\pgfsetbuttcap%
\pgfsetroundjoin%
\definecolor{currentfill}{rgb}{0.121569,0.466667,0.705882}%
\pgfsetfillcolor{currentfill}%
\pgfsetfillopacity{0.543261}%
\pgfsetlinewidth{1.003750pt}%
\definecolor{currentstroke}{rgb}{0.121569,0.466667,0.705882}%
\pgfsetstrokecolor{currentstroke}%
\pgfsetstrokeopacity{0.543261}%
\pgfsetdash{}{0pt}%
\pgfpathmoveto{\pgfqpoint{0.949926in}{1.529646in}}%
\pgfpathcurveto{\pgfqpoint{0.958162in}{1.529646in}}{\pgfqpoint{0.966062in}{1.532918in}}{\pgfqpoint{0.971886in}{1.538742in}}%
\pgfpathcurveto{\pgfqpoint{0.977710in}{1.544566in}}{\pgfqpoint{0.980982in}{1.552466in}}{\pgfqpoint{0.980982in}{1.560702in}}%
\pgfpathcurveto{\pgfqpoint{0.980982in}{1.568939in}}{\pgfqpoint{0.977710in}{1.576839in}}{\pgfqpoint{0.971886in}{1.582663in}}%
\pgfpathcurveto{\pgfqpoint{0.966062in}{1.588487in}}{\pgfqpoint{0.958162in}{1.591759in}}{\pgfqpoint{0.949926in}{1.591759in}}%
\pgfpathcurveto{\pgfqpoint{0.941689in}{1.591759in}}{\pgfqpoint{0.933789in}{1.588487in}}{\pgfqpoint{0.927965in}{1.582663in}}%
\pgfpathcurveto{\pgfqpoint{0.922141in}{1.576839in}}{\pgfqpoint{0.918869in}{1.568939in}}{\pgfqpoint{0.918869in}{1.560702in}}%
\pgfpathcurveto{\pgfqpoint{0.918869in}{1.552466in}}{\pgfqpoint{0.922141in}{1.544566in}}{\pgfqpoint{0.927965in}{1.538742in}}%
\pgfpathcurveto{\pgfqpoint{0.933789in}{1.532918in}}{\pgfqpoint{0.941689in}{1.529646in}}{\pgfqpoint{0.949926in}{1.529646in}}%
\pgfpathclose%
\pgfusepath{stroke,fill}%
\end{pgfscope}%
\begin{pgfscope}%
\pgfpathrectangle{\pgfqpoint{0.100000in}{0.212622in}}{\pgfqpoint{3.696000in}{3.696000in}}%
\pgfusepath{clip}%
\pgfsetbuttcap%
\pgfsetroundjoin%
\definecolor{currentfill}{rgb}{0.121569,0.466667,0.705882}%
\pgfsetfillcolor{currentfill}%
\pgfsetfillopacity{0.544450}%
\pgfsetlinewidth{1.003750pt}%
\definecolor{currentstroke}{rgb}{0.121569,0.466667,0.705882}%
\pgfsetstrokecolor{currentstroke}%
\pgfsetstrokeopacity{0.544450}%
\pgfsetdash{}{0pt}%
\pgfpathmoveto{\pgfqpoint{0.947016in}{1.526680in}}%
\pgfpathcurveto{\pgfqpoint{0.955253in}{1.526680in}}{\pgfqpoint{0.963153in}{1.529953in}}{\pgfqpoint{0.968977in}{1.535776in}}%
\pgfpathcurveto{\pgfqpoint{0.974801in}{1.541600in}}{\pgfqpoint{0.978073in}{1.549500in}}{\pgfqpoint{0.978073in}{1.557737in}}%
\pgfpathcurveto{\pgfqpoint{0.978073in}{1.565973in}}{\pgfqpoint{0.974801in}{1.573873in}}{\pgfqpoint{0.968977in}{1.579697in}}%
\pgfpathcurveto{\pgfqpoint{0.963153in}{1.585521in}}{\pgfqpoint{0.955253in}{1.588793in}}{\pgfqpoint{0.947016in}{1.588793in}}%
\pgfpathcurveto{\pgfqpoint{0.938780in}{1.588793in}}{\pgfqpoint{0.930880in}{1.585521in}}{\pgfqpoint{0.925056in}{1.579697in}}%
\pgfpathcurveto{\pgfqpoint{0.919232in}{1.573873in}}{\pgfqpoint{0.915960in}{1.565973in}}{\pgfqpoint{0.915960in}{1.557737in}}%
\pgfpathcurveto{\pgfqpoint{0.915960in}{1.549500in}}{\pgfqpoint{0.919232in}{1.541600in}}{\pgfqpoint{0.925056in}{1.535776in}}%
\pgfpathcurveto{\pgfqpoint{0.930880in}{1.529953in}}{\pgfqpoint{0.938780in}{1.526680in}}{\pgfqpoint{0.947016in}{1.526680in}}%
\pgfpathclose%
\pgfusepath{stroke,fill}%
\end{pgfscope}%
\begin{pgfscope}%
\pgfpathrectangle{\pgfqpoint{0.100000in}{0.212622in}}{\pgfqpoint{3.696000in}{3.696000in}}%
\pgfusepath{clip}%
\pgfsetbuttcap%
\pgfsetroundjoin%
\definecolor{currentfill}{rgb}{0.121569,0.466667,0.705882}%
\pgfsetfillcolor{currentfill}%
\pgfsetfillopacity{0.545479}%
\pgfsetlinewidth{1.003750pt}%
\definecolor{currentstroke}{rgb}{0.121569,0.466667,0.705882}%
\pgfsetstrokecolor{currentstroke}%
\pgfsetstrokeopacity{0.545479}%
\pgfsetdash{}{0pt}%
\pgfpathmoveto{\pgfqpoint{1.573459in}{1.708116in}}%
\pgfpathcurveto{\pgfqpoint{1.581695in}{1.708116in}}{\pgfqpoint{1.589595in}{1.711389in}}{\pgfqpoint{1.595419in}{1.717213in}}%
\pgfpathcurveto{\pgfqpoint{1.601243in}{1.723037in}}{\pgfqpoint{1.604515in}{1.730937in}}{\pgfqpoint{1.604515in}{1.739173in}}%
\pgfpathcurveto{\pgfqpoint{1.604515in}{1.747409in}}{\pgfqpoint{1.601243in}{1.755309in}}{\pgfqpoint{1.595419in}{1.761133in}}%
\pgfpathcurveto{\pgfqpoint{1.589595in}{1.766957in}}{\pgfqpoint{1.581695in}{1.770229in}}{\pgfqpoint{1.573459in}{1.770229in}}%
\pgfpathcurveto{\pgfqpoint{1.565223in}{1.770229in}}{\pgfqpoint{1.557322in}{1.766957in}}{\pgfqpoint{1.551499in}{1.761133in}}%
\pgfpathcurveto{\pgfqpoint{1.545675in}{1.755309in}}{\pgfqpoint{1.542402in}{1.747409in}}{\pgfqpoint{1.542402in}{1.739173in}}%
\pgfpathcurveto{\pgfqpoint{1.542402in}{1.730937in}}{\pgfqpoint{1.545675in}{1.723037in}}{\pgfqpoint{1.551499in}{1.717213in}}%
\pgfpathcurveto{\pgfqpoint{1.557322in}{1.711389in}}{\pgfqpoint{1.565223in}{1.708116in}}{\pgfqpoint{1.573459in}{1.708116in}}%
\pgfpathclose%
\pgfusepath{stroke,fill}%
\end{pgfscope}%
\begin{pgfscope}%
\pgfpathrectangle{\pgfqpoint{0.100000in}{0.212622in}}{\pgfqpoint{3.696000in}{3.696000in}}%
\pgfusepath{clip}%
\pgfsetbuttcap%
\pgfsetroundjoin%
\definecolor{currentfill}{rgb}{0.121569,0.466667,0.705882}%
\pgfsetfillcolor{currentfill}%
\pgfsetfillopacity{0.546843}%
\pgfsetlinewidth{1.003750pt}%
\definecolor{currentstroke}{rgb}{0.121569,0.466667,0.705882}%
\pgfsetstrokecolor{currentstroke}%
\pgfsetstrokeopacity{0.546843}%
\pgfsetdash{}{0pt}%
\pgfpathmoveto{\pgfqpoint{0.941856in}{1.522136in}}%
\pgfpathcurveto{\pgfqpoint{0.950092in}{1.522136in}}{\pgfqpoint{0.957992in}{1.525409in}}{\pgfqpoint{0.963816in}{1.531233in}}%
\pgfpathcurveto{\pgfqpoint{0.969640in}{1.537056in}}{\pgfqpoint{0.972913in}{1.544957in}}{\pgfqpoint{0.972913in}{1.553193in}}%
\pgfpathcurveto{\pgfqpoint{0.972913in}{1.561429in}}{\pgfqpoint{0.969640in}{1.569329in}}{\pgfqpoint{0.963816in}{1.575153in}}%
\pgfpathcurveto{\pgfqpoint{0.957992in}{1.580977in}}{\pgfqpoint{0.950092in}{1.584249in}}{\pgfqpoint{0.941856in}{1.584249in}}%
\pgfpathcurveto{\pgfqpoint{0.933620in}{1.584249in}}{\pgfqpoint{0.925720in}{1.580977in}}{\pgfqpoint{0.919896in}{1.575153in}}%
\pgfpathcurveto{\pgfqpoint{0.914072in}{1.569329in}}{\pgfqpoint{0.910800in}{1.561429in}}{\pgfqpoint{0.910800in}{1.553193in}}%
\pgfpathcurveto{\pgfqpoint{0.910800in}{1.544957in}}{\pgfqpoint{0.914072in}{1.537056in}}{\pgfqpoint{0.919896in}{1.531233in}}%
\pgfpathcurveto{\pgfqpoint{0.925720in}{1.525409in}}{\pgfqpoint{0.933620in}{1.522136in}}{\pgfqpoint{0.941856in}{1.522136in}}%
\pgfpathclose%
\pgfusepath{stroke,fill}%
\end{pgfscope}%
\begin{pgfscope}%
\pgfpathrectangle{\pgfqpoint{0.100000in}{0.212622in}}{\pgfqpoint{3.696000in}{3.696000in}}%
\pgfusepath{clip}%
\pgfsetbuttcap%
\pgfsetroundjoin%
\definecolor{currentfill}{rgb}{0.121569,0.466667,0.705882}%
\pgfsetfillcolor{currentfill}%
\pgfsetfillopacity{0.548928}%
\pgfsetlinewidth{1.003750pt}%
\definecolor{currentstroke}{rgb}{0.121569,0.466667,0.705882}%
\pgfsetstrokecolor{currentstroke}%
\pgfsetstrokeopacity{0.548928}%
\pgfsetdash{}{0pt}%
\pgfpathmoveto{\pgfqpoint{1.575235in}{1.709461in}}%
\pgfpathcurveto{\pgfqpoint{1.583471in}{1.709461in}}{\pgfqpoint{1.591371in}{1.712734in}}{\pgfqpoint{1.597195in}{1.718558in}}%
\pgfpathcurveto{\pgfqpoint{1.603019in}{1.724382in}}{\pgfqpoint{1.606292in}{1.732282in}}{\pgfqpoint{1.606292in}{1.740518in}}%
\pgfpathcurveto{\pgfqpoint{1.606292in}{1.748754in}}{\pgfqpoint{1.603019in}{1.756654in}}{\pgfqpoint{1.597195in}{1.762478in}}%
\pgfpathcurveto{\pgfqpoint{1.591371in}{1.768302in}}{\pgfqpoint{1.583471in}{1.771574in}}{\pgfqpoint{1.575235in}{1.771574in}}%
\pgfpathcurveto{\pgfqpoint{1.566999in}{1.771574in}}{\pgfqpoint{1.559099in}{1.768302in}}{\pgfqpoint{1.553275in}{1.762478in}}%
\pgfpathcurveto{\pgfqpoint{1.547451in}{1.756654in}}{\pgfqpoint{1.544179in}{1.748754in}}{\pgfqpoint{1.544179in}{1.740518in}}%
\pgfpathcurveto{\pgfqpoint{1.544179in}{1.732282in}}{\pgfqpoint{1.547451in}{1.724382in}}{\pgfqpoint{1.553275in}{1.718558in}}%
\pgfpathcurveto{\pgfqpoint{1.559099in}{1.712734in}}{\pgfqpoint{1.566999in}{1.709461in}}{\pgfqpoint{1.575235in}{1.709461in}}%
\pgfpathclose%
\pgfusepath{stroke,fill}%
\end{pgfscope}%
\begin{pgfscope}%
\pgfpathrectangle{\pgfqpoint{0.100000in}{0.212622in}}{\pgfqpoint{3.696000in}{3.696000in}}%
\pgfusepath{clip}%
\pgfsetbuttcap%
\pgfsetroundjoin%
\definecolor{currentfill}{rgb}{0.121569,0.466667,0.705882}%
\pgfsetfillcolor{currentfill}%
\pgfsetfillopacity{0.548937}%
\pgfsetlinewidth{1.003750pt}%
\definecolor{currentstroke}{rgb}{0.121569,0.466667,0.705882}%
\pgfsetstrokecolor{currentstroke}%
\pgfsetstrokeopacity{0.548937}%
\pgfsetdash{}{0pt}%
\pgfpathmoveto{\pgfqpoint{0.936373in}{1.519581in}}%
\pgfpathcurveto{\pgfqpoint{0.944610in}{1.519581in}}{\pgfqpoint{0.952510in}{1.522853in}}{\pgfqpoint{0.958334in}{1.528677in}}%
\pgfpathcurveto{\pgfqpoint{0.964158in}{1.534501in}}{\pgfqpoint{0.967430in}{1.542401in}}{\pgfqpoint{0.967430in}{1.550637in}}%
\pgfpathcurveto{\pgfqpoint{0.967430in}{1.558874in}}{\pgfqpoint{0.964158in}{1.566774in}}{\pgfqpoint{0.958334in}{1.572598in}}%
\pgfpathcurveto{\pgfqpoint{0.952510in}{1.578422in}}{\pgfqpoint{0.944610in}{1.581694in}}{\pgfqpoint{0.936373in}{1.581694in}}%
\pgfpathcurveto{\pgfqpoint{0.928137in}{1.581694in}}{\pgfqpoint{0.920237in}{1.578422in}}{\pgfqpoint{0.914413in}{1.572598in}}%
\pgfpathcurveto{\pgfqpoint{0.908589in}{1.566774in}}{\pgfqpoint{0.905317in}{1.558874in}}{\pgfqpoint{0.905317in}{1.550637in}}%
\pgfpathcurveto{\pgfqpoint{0.905317in}{1.542401in}}{\pgfqpoint{0.908589in}{1.534501in}}{\pgfqpoint{0.914413in}{1.528677in}}%
\pgfpathcurveto{\pgfqpoint{0.920237in}{1.522853in}}{\pgfqpoint{0.928137in}{1.519581in}}{\pgfqpoint{0.936373in}{1.519581in}}%
\pgfpathclose%
\pgfusepath{stroke,fill}%
\end{pgfscope}%
\begin{pgfscope}%
\pgfpathrectangle{\pgfqpoint{0.100000in}{0.212622in}}{\pgfqpoint{3.696000in}{3.696000in}}%
\pgfusepath{clip}%
\pgfsetbuttcap%
\pgfsetroundjoin%
\definecolor{currentfill}{rgb}{0.121569,0.466667,0.705882}%
\pgfsetfillcolor{currentfill}%
\pgfsetfillopacity{0.552235}%
\pgfsetlinewidth{1.003750pt}%
\definecolor{currentstroke}{rgb}{0.121569,0.466667,0.705882}%
\pgfsetstrokecolor{currentstroke}%
\pgfsetstrokeopacity{0.552235}%
\pgfsetdash{}{0pt}%
\pgfpathmoveto{\pgfqpoint{0.933661in}{1.523391in}}%
\pgfpathcurveto{\pgfqpoint{0.941897in}{1.523391in}}{\pgfqpoint{0.949797in}{1.526663in}}{\pgfqpoint{0.955621in}{1.532487in}}%
\pgfpathcurveto{\pgfqpoint{0.961445in}{1.538311in}}{\pgfqpoint{0.964717in}{1.546211in}}{\pgfqpoint{0.964717in}{1.554447in}}%
\pgfpathcurveto{\pgfqpoint{0.964717in}{1.562684in}}{\pgfqpoint{0.961445in}{1.570584in}}{\pgfqpoint{0.955621in}{1.576408in}}%
\pgfpathcurveto{\pgfqpoint{0.949797in}{1.582232in}}{\pgfqpoint{0.941897in}{1.585504in}}{\pgfqpoint{0.933661in}{1.585504in}}%
\pgfpathcurveto{\pgfqpoint{0.925424in}{1.585504in}}{\pgfqpoint{0.917524in}{1.582232in}}{\pgfqpoint{0.911700in}{1.576408in}}%
\pgfpathcurveto{\pgfqpoint{0.905876in}{1.570584in}}{\pgfqpoint{0.902604in}{1.562684in}}{\pgfqpoint{0.902604in}{1.554447in}}%
\pgfpathcurveto{\pgfqpoint{0.902604in}{1.546211in}}{\pgfqpoint{0.905876in}{1.538311in}}{\pgfqpoint{0.911700in}{1.532487in}}%
\pgfpathcurveto{\pgfqpoint{0.917524in}{1.526663in}}{\pgfqpoint{0.925424in}{1.523391in}}{\pgfqpoint{0.933661in}{1.523391in}}%
\pgfpathclose%
\pgfusepath{stroke,fill}%
\end{pgfscope}%
\begin{pgfscope}%
\pgfpathrectangle{\pgfqpoint{0.100000in}{0.212622in}}{\pgfqpoint{3.696000in}{3.696000in}}%
\pgfusepath{clip}%
\pgfsetbuttcap%
\pgfsetroundjoin%
\definecolor{currentfill}{rgb}{0.121569,0.466667,0.705882}%
\pgfsetfillcolor{currentfill}%
\pgfsetfillopacity{0.552345}%
\pgfsetlinewidth{1.003750pt}%
\definecolor{currentstroke}{rgb}{0.121569,0.466667,0.705882}%
\pgfsetstrokecolor{currentstroke}%
\pgfsetstrokeopacity{0.552345}%
\pgfsetdash{}{0pt}%
\pgfpathmoveto{\pgfqpoint{1.577128in}{1.709433in}}%
\pgfpathcurveto{\pgfqpoint{1.585365in}{1.709433in}}{\pgfqpoint{1.593265in}{1.712706in}}{\pgfqpoint{1.599089in}{1.718530in}}%
\pgfpathcurveto{\pgfqpoint{1.604912in}{1.724353in}}{\pgfqpoint{1.608185in}{1.732254in}}{\pgfqpoint{1.608185in}{1.740490in}}%
\pgfpathcurveto{\pgfqpoint{1.608185in}{1.748726in}}{\pgfqpoint{1.604912in}{1.756626in}}{\pgfqpoint{1.599089in}{1.762450in}}%
\pgfpathcurveto{\pgfqpoint{1.593265in}{1.768274in}}{\pgfqpoint{1.585365in}{1.771546in}}{\pgfqpoint{1.577128in}{1.771546in}}%
\pgfpathcurveto{\pgfqpoint{1.568892in}{1.771546in}}{\pgfqpoint{1.560992in}{1.768274in}}{\pgfqpoint{1.555168in}{1.762450in}}%
\pgfpathcurveto{\pgfqpoint{1.549344in}{1.756626in}}{\pgfqpoint{1.546072in}{1.748726in}}{\pgfqpoint{1.546072in}{1.740490in}}%
\pgfpathcurveto{\pgfqpoint{1.546072in}{1.732254in}}{\pgfqpoint{1.549344in}{1.724353in}}{\pgfqpoint{1.555168in}{1.718530in}}%
\pgfpathcurveto{\pgfqpoint{1.560992in}{1.712706in}}{\pgfqpoint{1.568892in}{1.709433in}}{\pgfqpoint{1.577128in}{1.709433in}}%
\pgfpathclose%
\pgfusepath{stroke,fill}%
\end{pgfscope}%
\begin{pgfscope}%
\pgfpathrectangle{\pgfqpoint{0.100000in}{0.212622in}}{\pgfqpoint{3.696000in}{3.696000in}}%
\pgfusepath{clip}%
\pgfsetbuttcap%
\pgfsetroundjoin%
\definecolor{currentfill}{rgb}{0.121569,0.466667,0.705882}%
\pgfsetfillcolor{currentfill}%
\pgfsetfillopacity{0.553243}%
\pgfsetlinewidth{1.003750pt}%
\definecolor{currentstroke}{rgb}{0.121569,0.466667,0.705882}%
\pgfsetstrokecolor{currentstroke}%
\pgfsetstrokeopacity{0.553243}%
\pgfsetdash{}{0pt}%
\pgfpathmoveto{\pgfqpoint{0.929570in}{1.519747in}}%
\pgfpathcurveto{\pgfqpoint{0.937806in}{1.519747in}}{\pgfqpoint{0.945706in}{1.523019in}}{\pgfqpoint{0.951530in}{1.528843in}}%
\pgfpathcurveto{\pgfqpoint{0.957354in}{1.534667in}}{\pgfqpoint{0.960626in}{1.542567in}}{\pgfqpoint{0.960626in}{1.550803in}}%
\pgfpathcurveto{\pgfqpoint{0.960626in}{1.559040in}}{\pgfqpoint{0.957354in}{1.566940in}}{\pgfqpoint{0.951530in}{1.572764in}}%
\pgfpathcurveto{\pgfqpoint{0.945706in}{1.578588in}}{\pgfqpoint{0.937806in}{1.581860in}}{\pgfqpoint{0.929570in}{1.581860in}}%
\pgfpathcurveto{\pgfqpoint{0.921333in}{1.581860in}}{\pgfqpoint{0.913433in}{1.578588in}}{\pgfqpoint{0.907609in}{1.572764in}}%
\pgfpathcurveto{\pgfqpoint{0.901785in}{1.566940in}}{\pgfqpoint{0.898513in}{1.559040in}}{\pgfqpoint{0.898513in}{1.550803in}}%
\pgfpathcurveto{\pgfqpoint{0.898513in}{1.542567in}}{\pgfqpoint{0.901785in}{1.534667in}}{\pgfqpoint{0.907609in}{1.528843in}}%
\pgfpathcurveto{\pgfqpoint{0.913433in}{1.523019in}}{\pgfqpoint{0.921333in}{1.519747in}}{\pgfqpoint{0.929570in}{1.519747in}}%
\pgfpathclose%
\pgfusepath{stroke,fill}%
\end{pgfscope}%
\begin{pgfscope}%
\pgfpathrectangle{\pgfqpoint{0.100000in}{0.212622in}}{\pgfqpoint{3.696000in}{3.696000in}}%
\pgfusepath{clip}%
\pgfsetbuttcap%
\pgfsetroundjoin%
\definecolor{currentfill}{rgb}{0.121569,0.466667,0.705882}%
\pgfsetfillcolor{currentfill}%
\pgfsetfillopacity{0.555468}%
\pgfsetlinewidth{1.003750pt}%
\definecolor{currentstroke}{rgb}{0.121569,0.466667,0.705882}%
\pgfsetstrokecolor{currentstroke}%
\pgfsetstrokeopacity{0.555468}%
\pgfsetdash{}{0pt}%
\pgfpathmoveto{\pgfqpoint{0.923145in}{1.513123in}}%
\pgfpathcurveto{\pgfqpoint{0.931381in}{1.513123in}}{\pgfqpoint{0.939281in}{1.516395in}}{\pgfqpoint{0.945105in}{1.522219in}}%
\pgfpathcurveto{\pgfqpoint{0.950929in}{1.528043in}}{\pgfqpoint{0.954201in}{1.535943in}}{\pgfqpoint{0.954201in}{1.544180in}}%
\pgfpathcurveto{\pgfqpoint{0.954201in}{1.552416in}}{\pgfqpoint{0.950929in}{1.560316in}}{\pgfqpoint{0.945105in}{1.566140in}}%
\pgfpathcurveto{\pgfqpoint{0.939281in}{1.571964in}}{\pgfqpoint{0.931381in}{1.575236in}}{\pgfqpoint{0.923145in}{1.575236in}}%
\pgfpathcurveto{\pgfqpoint{0.914908in}{1.575236in}}{\pgfqpoint{0.907008in}{1.571964in}}{\pgfqpoint{0.901184in}{1.566140in}}%
\pgfpathcurveto{\pgfqpoint{0.895360in}{1.560316in}}{\pgfqpoint{0.892088in}{1.552416in}}{\pgfqpoint{0.892088in}{1.544180in}}%
\pgfpathcurveto{\pgfqpoint{0.892088in}{1.535943in}}{\pgfqpoint{0.895360in}{1.528043in}}{\pgfqpoint{0.901184in}{1.522219in}}%
\pgfpathcurveto{\pgfqpoint{0.907008in}{1.516395in}}{\pgfqpoint{0.914908in}{1.513123in}}{\pgfqpoint{0.923145in}{1.513123in}}%
\pgfpathclose%
\pgfusepath{stroke,fill}%
\end{pgfscope}%
\begin{pgfscope}%
\pgfpathrectangle{\pgfqpoint{0.100000in}{0.212622in}}{\pgfqpoint{3.696000in}{3.696000in}}%
\pgfusepath{clip}%
\pgfsetbuttcap%
\pgfsetroundjoin%
\definecolor{currentfill}{rgb}{0.121569,0.466667,0.705882}%
\pgfsetfillcolor{currentfill}%
\pgfsetfillopacity{0.555861}%
\pgfsetlinewidth{1.003750pt}%
\definecolor{currentstroke}{rgb}{0.121569,0.466667,0.705882}%
\pgfsetstrokecolor{currentstroke}%
\pgfsetstrokeopacity{0.555861}%
\pgfsetdash{}{0pt}%
\pgfpathmoveto{\pgfqpoint{1.577977in}{1.707609in}}%
\pgfpathcurveto{\pgfqpoint{1.586214in}{1.707609in}}{\pgfqpoint{1.594114in}{1.710882in}}{\pgfqpoint{1.599938in}{1.716705in}}%
\pgfpathcurveto{\pgfqpoint{1.605762in}{1.722529in}}{\pgfqpoint{1.609034in}{1.730429in}}{\pgfqpoint{1.609034in}{1.738666in}}%
\pgfpathcurveto{\pgfqpoint{1.609034in}{1.746902in}}{\pgfqpoint{1.605762in}{1.754802in}}{\pgfqpoint{1.599938in}{1.760626in}}%
\pgfpathcurveto{\pgfqpoint{1.594114in}{1.766450in}}{\pgfqpoint{1.586214in}{1.769722in}}{\pgfqpoint{1.577977in}{1.769722in}}%
\pgfpathcurveto{\pgfqpoint{1.569741in}{1.769722in}}{\pgfqpoint{1.561841in}{1.766450in}}{\pgfqpoint{1.556017in}{1.760626in}}%
\pgfpathcurveto{\pgfqpoint{1.550193in}{1.754802in}}{\pgfqpoint{1.546921in}{1.746902in}}{\pgfqpoint{1.546921in}{1.738666in}}%
\pgfpathcurveto{\pgfqpoint{1.546921in}{1.730429in}}{\pgfqpoint{1.550193in}{1.722529in}}{\pgfqpoint{1.556017in}{1.716705in}}%
\pgfpathcurveto{\pgfqpoint{1.561841in}{1.710882in}}{\pgfqpoint{1.569741in}{1.707609in}}{\pgfqpoint{1.577977in}{1.707609in}}%
\pgfpathclose%
\pgfusepath{stroke,fill}%
\end{pgfscope}%
\begin{pgfscope}%
\pgfpathrectangle{\pgfqpoint{0.100000in}{0.212622in}}{\pgfqpoint{3.696000in}{3.696000in}}%
\pgfusepath{clip}%
\pgfsetbuttcap%
\pgfsetroundjoin%
\definecolor{currentfill}{rgb}{0.121569,0.466667,0.705882}%
\pgfsetfillcolor{currentfill}%
\pgfsetfillopacity{0.559328}%
\pgfsetlinewidth{1.003750pt}%
\definecolor{currentstroke}{rgb}{0.121569,0.466667,0.705882}%
\pgfsetstrokecolor{currentstroke}%
\pgfsetstrokeopacity{0.559328}%
\pgfsetdash{}{0pt}%
\pgfpathmoveto{\pgfqpoint{1.579788in}{1.703016in}}%
\pgfpathcurveto{\pgfqpoint{1.588024in}{1.703016in}}{\pgfqpoint{1.595924in}{1.706289in}}{\pgfqpoint{1.601748in}{1.712113in}}%
\pgfpathcurveto{\pgfqpoint{1.607572in}{1.717937in}}{\pgfqpoint{1.610844in}{1.725837in}}{\pgfqpoint{1.610844in}{1.734073in}}%
\pgfpathcurveto{\pgfqpoint{1.610844in}{1.742309in}}{\pgfqpoint{1.607572in}{1.750209in}}{\pgfqpoint{1.601748in}{1.756033in}}%
\pgfpathcurveto{\pgfqpoint{1.595924in}{1.761857in}}{\pgfqpoint{1.588024in}{1.765129in}}{\pgfqpoint{1.579788in}{1.765129in}}%
\pgfpathcurveto{\pgfqpoint{1.571552in}{1.765129in}}{\pgfqpoint{1.563652in}{1.761857in}}{\pgfqpoint{1.557828in}{1.756033in}}%
\pgfpathcurveto{\pgfqpoint{1.552004in}{1.750209in}}{\pgfqpoint{1.548731in}{1.742309in}}{\pgfqpoint{1.548731in}{1.734073in}}%
\pgfpathcurveto{\pgfqpoint{1.548731in}{1.725837in}}{\pgfqpoint{1.552004in}{1.717937in}}{\pgfqpoint{1.557828in}{1.712113in}}%
\pgfpathcurveto{\pgfqpoint{1.563652in}{1.706289in}}{\pgfqpoint{1.571552in}{1.703016in}}{\pgfqpoint{1.579788in}{1.703016in}}%
\pgfpathclose%
\pgfusepath{stroke,fill}%
\end{pgfscope}%
\begin{pgfscope}%
\pgfpathrectangle{\pgfqpoint{0.100000in}{0.212622in}}{\pgfqpoint{3.696000in}{3.696000in}}%
\pgfusepath{clip}%
\pgfsetbuttcap%
\pgfsetroundjoin%
\definecolor{currentfill}{rgb}{0.121569,0.466667,0.705882}%
\pgfsetfillcolor{currentfill}%
\pgfsetfillopacity{0.559463}%
\pgfsetlinewidth{1.003750pt}%
\definecolor{currentstroke}{rgb}{0.121569,0.466667,0.705882}%
\pgfsetstrokecolor{currentstroke}%
\pgfsetstrokeopacity{0.559463}%
\pgfsetdash{}{0pt}%
\pgfpathmoveto{\pgfqpoint{0.918328in}{1.515413in}}%
\pgfpathcurveto{\pgfqpoint{0.926564in}{1.515413in}}{\pgfqpoint{0.934464in}{1.518685in}}{\pgfqpoint{0.940288in}{1.524509in}}%
\pgfpathcurveto{\pgfqpoint{0.946112in}{1.530333in}}{\pgfqpoint{0.949384in}{1.538233in}}{\pgfqpoint{0.949384in}{1.546469in}}%
\pgfpathcurveto{\pgfqpoint{0.949384in}{1.554706in}}{\pgfqpoint{0.946112in}{1.562606in}}{\pgfqpoint{0.940288in}{1.568430in}}%
\pgfpathcurveto{\pgfqpoint{0.934464in}{1.574253in}}{\pgfqpoint{0.926564in}{1.577526in}}{\pgfqpoint{0.918328in}{1.577526in}}%
\pgfpathcurveto{\pgfqpoint{0.910091in}{1.577526in}}{\pgfqpoint{0.902191in}{1.574253in}}{\pgfqpoint{0.896367in}{1.568430in}}%
\pgfpathcurveto{\pgfqpoint{0.890543in}{1.562606in}}{\pgfqpoint{0.887271in}{1.554706in}}{\pgfqpoint{0.887271in}{1.546469in}}%
\pgfpathcurveto{\pgfqpoint{0.887271in}{1.538233in}}{\pgfqpoint{0.890543in}{1.530333in}}{\pgfqpoint{0.896367in}{1.524509in}}%
\pgfpathcurveto{\pgfqpoint{0.902191in}{1.518685in}}{\pgfqpoint{0.910091in}{1.515413in}}{\pgfqpoint{0.918328in}{1.515413in}}%
\pgfpathclose%
\pgfusepath{stroke,fill}%
\end{pgfscope}%
\begin{pgfscope}%
\pgfpathrectangle{\pgfqpoint{0.100000in}{0.212622in}}{\pgfqpoint{3.696000in}{3.696000in}}%
\pgfusepath{clip}%
\pgfsetbuttcap%
\pgfsetroundjoin%
\definecolor{currentfill}{rgb}{0.121569,0.466667,0.705882}%
\pgfsetfillcolor{currentfill}%
\pgfsetfillopacity{0.559867}%
\pgfsetlinewidth{1.003750pt}%
\definecolor{currentstroke}{rgb}{0.121569,0.466667,0.705882}%
\pgfsetstrokecolor{currentstroke}%
\pgfsetstrokeopacity{0.559867}%
\pgfsetdash{}{0pt}%
\pgfpathmoveto{\pgfqpoint{0.907266in}{1.498930in}}%
\pgfpathcurveto{\pgfqpoint{0.915502in}{1.498930in}}{\pgfqpoint{0.923402in}{1.502203in}}{\pgfqpoint{0.929226in}{1.508027in}}%
\pgfpathcurveto{\pgfqpoint{0.935050in}{1.513851in}}{\pgfqpoint{0.938322in}{1.521751in}}{\pgfqpoint{0.938322in}{1.529987in}}%
\pgfpathcurveto{\pgfqpoint{0.938322in}{1.538223in}}{\pgfqpoint{0.935050in}{1.546123in}}{\pgfqpoint{0.929226in}{1.551947in}}%
\pgfpathcurveto{\pgfqpoint{0.923402in}{1.557771in}}{\pgfqpoint{0.915502in}{1.561043in}}{\pgfqpoint{0.907266in}{1.561043in}}%
\pgfpathcurveto{\pgfqpoint{0.899029in}{1.561043in}}{\pgfqpoint{0.891129in}{1.557771in}}{\pgfqpoint{0.885305in}{1.551947in}}%
\pgfpathcurveto{\pgfqpoint{0.879481in}{1.546123in}}{\pgfqpoint{0.876209in}{1.538223in}}{\pgfqpoint{0.876209in}{1.529987in}}%
\pgfpathcurveto{\pgfqpoint{0.876209in}{1.521751in}}{\pgfqpoint{0.879481in}{1.513851in}}{\pgfqpoint{0.885305in}{1.508027in}}%
\pgfpathcurveto{\pgfqpoint{0.891129in}{1.502203in}}{\pgfqpoint{0.899029in}{1.498930in}}{\pgfqpoint{0.907266in}{1.498930in}}%
\pgfpathclose%
\pgfusepath{stroke,fill}%
\end{pgfscope}%
\begin{pgfscope}%
\pgfpathrectangle{\pgfqpoint{0.100000in}{0.212622in}}{\pgfqpoint{3.696000in}{3.696000in}}%
\pgfusepath{clip}%
\pgfsetbuttcap%
\pgfsetroundjoin%
\definecolor{currentfill}{rgb}{0.121569,0.466667,0.705882}%
\pgfsetfillcolor{currentfill}%
\pgfsetfillopacity{0.560321}%
\pgfsetlinewidth{1.003750pt}%
\definecolor{currentstroke}{rgb}{0.121569,0.466667,0.705882}%
\pgfsetstrokecolor{currentstroke}%
\pgfsetstrokeopacity{0.560321}%
\pgfsetdash{}{0pt}%
\pgfpathmoveto{\pgfqpoint{0.912380in}{1.510363in}}%
\pgfpathcurveto{\pgfqpoint{0.920616in}{1.510363in}}{\pgfqpoint{0.928517in}{1.513636in}}{\pgfqpoint{0.934340in}{1.519460in}}%
\pgfpathcurveto{\pgfqpoint{0.940164in}{1.525284in}}{\pgfqpoint{0.943437in}{1.533184in}}{\pgfqpoint{0.943437in}{1.541420in}}%
\pgfpathcurveto{\pgfqpoint{0.943437in}{1.549656in}}{\pgfqpoint{0.940164in}{1.557556in}}{\pgfqpoint{0.934340in}{1.563380in}}%
\pgfpathcurveto{\pgfqpoint{0.928517in}{1.569204in}}{\pgfqpoint{0.920616in}{1.572476in}}{\pgfqpoint{0.912380in}{1.572476in}}%
\pgfpathcurveto{\pgfqpoint{0.904144in}{1.572476in}}{\pgfqpoint{0.896244in}{1.569204in}}{\pgfqpoint{0.890420in}{1.563380in}}%
\pgfpathcurveto{\pgfqpoint{0.884596in}{1.557556in}}{\pgfqpoint{0.881324in}{1.549656in}}{\pgfqpoint{0.881324in}{1.541420in}}%
\pgfpathcurveto{\pgfqpoint{0.881324in}{1.533184in}}{\pgfqpoint{0.884596in}{1.525284in}}{\pgfqpoint{0.890420in}{1.519460in}}%
\pgfpathcurveto{\pgfqpoint{0.896244in}{1.513636in}}{\pgfqpoint{0.904144in}{1.510363in}}{\pgfqpoint{0.912380in}{1.510363in}}%
\pgfpathclose%
\pgfusepath{stroke,fill}%
\end{pgfscope}%
\begin{pgfscope}%
\pgfpathrectangle{\pgfqpoint{0.100000in}{0.212622in}}{\pgfqpoint{3.696000in}{3.696000in}}%
\pgfusepath{clip}%
\pgfsetbuttcap%
\pgfsetroundjoin%
\definecolor{currentfill}{rgb}{0.121569,0.466667,0.705882}%
\pgfsetfillcolor{currentfill}%
\pgfsetfillopacity{0.562145}%
\pgfsetlinewidth{1.003750pt}%
\definecolor{currentstroke}{rgb}{0.121569,0.466667,0.705882}%
\pgfsetstrokecolor{currentstroke}%
\pgfsetstrokeopacity{0.562145}%
\pgfsetdash{}{0pt}%
\pgfpathmoveto{\pgfqpoint{0.903962in}{1.498368in}}%
\pgfpathcurveto{\pgfqpoint{0.912199in}{1.498368in}}{\pgfqpoint{0.920099in}{1.501640in}}{\pgfqpoint{0.925923in}{1.507464in}}%
\pgfpathcurveto{\pgfqpoint{0.931747in}{1.513288in}}{\pgfqpoint{0.935019in}{1.521188in}}{\pgfqpoint{0.935019in}{1.529424in}}%
\pgfpathcurveto{\pgfqpoint{0.935019in}{1.537661in}}{\pgfqpoint{0.931747in}{1.545561in}}{\pgfqpoint{0.925923in}{1.551385in}}%
\pgfpathcurveto{\pgfqpoint{0.920099in}{1.557209in}}{\pgfqpoint{0.912199in}{1.560481in}}{\pgfqpoint{0.903962in}{1.560481in}}%
\pgfpathcurveto{\pgfqpoint{0.895726in}{1.560481in}}{\pgfqpoint{0.887826in}{1.557209in}}{\pgfqpoint{0.882002in}{1.551385in}}%
\pgfpathcurveto{\pgfqpoint{0.876178in}{1.545561in}}{\pgfqpoint{0.872906in}{1.537661in}}{\pgfqpoint{0.872906in}{1.529424in}}%
\pgfpathcurveto{\pgfqpoint{0.872906in}{1.521188in}}{\pgfqpoint{0.876178in}{1.513288in}}{\pgfqpoint{0.882002in}{1.507464in}}%
\pgfpathcurveto{\pgfqpoint{0.887826in}{1.501640in}}{\pgfqpoint{0.895726in}{1.498368in}}{\pgfqpoint{0.903962in}{1.498368in}}%
\pgfpathclose%
\pgfusepath{stroke,fill}%
\end{pgfscope}%
\begin{pgfscope}%
\pgfpathrectangle{\pgfqpoint{0.100000in}{0.212622in}}{\pgfqpoint{3.696000in}{3.696000in}}%
\pgfusepath{clip}%
\pgfsetbuttcap%
\pgfsetroundjoin%
\definecolor{currentfill}{rgb}{0.121569,0.466667,0.705882}%
\pgfsetfillcolor{currentfill}%
\pgfsetfillopacity{0.562823}%
\pgfsetlinewidth{1.003750pt}%
\definecolor{currentstroke}{rgb}{0.121569,0.466667,0.705882}%
\pgfsetstrokecolor{currentstroke}%
\pgfsetstrokeopacity{0.562823}%
\pgfsetdash{}{0pt}%
\pgfpathmoveto{\pgfqpoint{0.901096in}{1.494932in}}%
\pgfpathcurveto{\pgfqpoint{0.909332in}{1.494932in}}{\pgfqpoint{0.917232in}{1.498204in}}{\pgfqpoint{0.923056in}{1.504028in}}%
\pgfpathcurveto{\pgfqpoint{0.928880in}{1.509852in}}{\pgfqpoint{0.932152in}{1.517752in}}{\pgfqpoint{0.932152in}{1.525988in}}%
\pgfpathcurveto{\pgfqpoint{0.932152in}{1.534225in}}{\pgfqpoint{0.928880in}{1.542125in}}{\pgfqpoint{0.923056in}{1.547949in}}%
\pgfpathcurveto{\pgfqpoint{0.917232in}{1.553773in}}{\pgfqpoint{0.909332in}{1.557045in}}{\pgfqpoint{0.901096in}{1.557045in}}%
\pgfpathcurveto{\pgfqpoint{0.892860in}{1.557045in}}{\pgfqpoint{0.884960in}{1.553773in}}{\pgfqpoint{0.879136in}{1.547949in}}%
\pgfpathcurveto{\pgfqpoint{0.873312in}{1.542125in}}{\pgfqpoint{0.870039in}{1.534225in}}{\pgfqpoint{0.870039in}{1.525988in}}%
\pgfpathcurveto{\pgfqpoint{0.870039in}{1.517752in}}{\pgfqpoint{0.873312in}{1.509852in}}{\pgfqpoint{0.879136in}{1.504028in}}%
\pgfpathcurveto{\pgfqpoint{0.884960in}{1.498204in}}{\pgfqpoint{0.892860in}{1.494932in}}{\pgfqpoint{0.901096in}{1.494932in}}%
\pgfpathclose%
\pgfusepath{stroke,fill}%
\end{pgfscope}%
\begin{pgfscope}%
\pgfpathrectangle{\pgfqpoint{0.100000in}{0.212622in}}{\pgfqpoint{3.696000in}{3.696000in}}%
\pgfusepath{clip}%
\pgfsetbuttcap%
\pgfsetroundjoin%
\definecolor{currentfill}{rgb}{0.121569,0.466667,0.705882}%
\pgfsetfillcolor{currentfill}%
\pgfsetfillopacity{0.564224}%
\pgfsetlinewidth{1.003750pt}%
\definecolor{currentstroke}{rgb}{0.121569,0.466667,0.705882}%
\pgfsetstrokecolor{currentstroke}%
\pgfsetstrokeopacity{0.564224}%
\pgfsetdash{}{0pt}%
\pgfpathmoveto{\pgfqpoint{0.896393in}{1.488627in}}%
\pgfpathcurveto{\pgfqpoint{0.904629in}{1.488627in}}{\pgfqpoint{0.912529in}{1.491899in}}{\pgfqpoint{0.918353in}{1.497723in}}%
\pgfpathcurveto{\pgfqpoint{0.924177in}{1.503547in}}{\pgfqpoint{0.927449in}{1.511447in}}{\pgfqpoint{0.927449in}{1.519683in}}%
\pgfpathcurveto{\pgfqpoint{0.927449in}{1.527920in}}{\pgfqpoint{0.924177in}{1.535820in}}{\pgfqpoint{0.918353in}{1.541644in}}%
\pgfpathcurveto{\pgfqpoint{0.912529in}{1.547468in}}{\pgfqpoint{0.904629in}{1.550740in}}{\pgfqpoint{0.896393in}{1.550740in}}%
\pgfpathcurveto{\pgfqpoint{0.888157in}{1.550740in}}{\pgfqpoint{0.880256in}{1.547468in}}{\pgfqpoint{0.874433in}{1.541644in}}%
\pgfpathcurveto{\pgfqpoint{0.868609in}{1.535820in}}{\pgfqpoint{0.865336in}{1.527920in}}{\pgfqpoint{0.865336in}{1.519683in}}%
\pgfpathcurveto{\pgfqpoint{0.865336in}{1.511447in}}{\pgfqpoint{0.868609in}{1.503547in}}{\pgfqpoint{0.874433in}{1.497723in}}%
\pgfpathcurveto{\pgfqpoint{0.880256in}{1.491899in}}{\pgfqpoint{0.888157in}{1.488627in}}{\pgfqpoint{0.896393in}{1.488627in}}%
\pgfpathclose%
\pgfusepath{stroke,fill}%
\end{pgfscope}%
\begin{pgfscope}%
\pgfpathrectangle{\pgfqpoint{0.100000in}{0.212622in}}{\pgfqpoint{3.696000in}{3.696000in}}%
\pgfusepath{clip}%
\pgfsetbuttcap%
\pgfsetroundjoin%
\definecolor{currentfill}{rgb}{0.121569,0.466667,0.705882}%
\pgfsetfillcolor{currentfill}%
\pgfsetfillopacity{0.564242}%
\pgfsetlinewidth{1.003750pt}%
\definecolor{currentstroke}{rgb}{0.121569,0.466667,0.705882}%
\pgfsetstrokecolor{currentstroke}%
\pgfsetstrokeopacity{0.564242}%
\pgfsetdash{}{0pt}%
\pgfpathmoveto{\pgfqpoint{1.582122in}{1.704064in}}%
\pgfpathcurveto{\pgfqpoint{1.590359in}{1.704064in}}{\pgfqpoint{1.598259in}{1.707336in}}{\pgfqpoint{1.604083in}{1.713160in}}%
\pgfpathcurveto{\pgfqpoint{1.609906in}{1.718984in}}{\pgfqpoint{1.613179in}{1.726884in}}{\pgfqpoint{1.613179in}{1.735120in}}%
\pgfpathcurveto{\pgfqpoint{1.613179in}{1.743357in}}{\pgfqpoint{1.609906in}{1.751257in}}{\pgfqpoint{1.604083in}{1.757081in}}%
\pgfpathcurveto{\pgfqpoint{1.598259in}{1.762905in}}{\pgfqpoint{1.590359in}{1.766177in}}{\pgfqpoint{1.582122in}{1.766177in}}%
\pgfpathcurveto{\pgfqpoint{1.573886in}{1.766177in}}{\pgfqpoint{1.565986in}{1.762905in}}{\pgfqpoint{1.560162in}{1.757081in}}%
\pgfpathcurveto{\pgfqpoint{1.554338in}{1.751257in}}{\pgfqpoint{1.551066in}{1.743357in}}{\pgfqpoint{1.551066in}{1.735120in}}%
\pgfpathcurveto{\pgfqpoint{1.551066in}{1.726884in}}{\pgfqpoint{1.554338in}{1.718984in}}{\pgfqpoint{1.560162in}{1.713160in}}%
\pgfpathcurveto{\pgfqpoint{1.565986in}{1.707336in}}{\pgfqpoint{1.573886in}{1.704064in}}{\pgfqpoint{1.582122in}{1.704064in}}%
\pgfpathclose%
\pgfusepath{stroke,fill}%
\end{pgfscope}%
\begin{pgfscope}%
\pgfpathrectangle{\pgfqpoint{0.100000in}{0.212622in}}{\pgfqpoint{3.696000in}{3.696000in}}%
\pgfusepath{clip}%
\pgfsetbuttcap%
\pgfsetroundjoin%
\definecolor{currentfill}{rgb}{0.121569,0.466667,0.705882}%
\pgfsetfillcolor{currentfill}%
\pgfsetfillopacity{0.569153}%
\pgfsetlinewidth{1.003750pt}%
\definecolor{currentstroke}{rgb}{0.121569,0.466667,0.705882}%
\pgfsetstrokecolor{currentstroke}%
\pgfsetstrokeopacity{0.569153}%
\pgfsetdash{}{0pt}%
\pgfpathmoveto{\pgfqpoint{1.584679in}{1.703684in}}%
\pgfpathcurveto{\pgfqpoint{1.592915in}{1.703684in}}{\pgfqpoint{1.600815in}{1.706956in}}{\pgfqpoint{1.606639in}{1.712780in}}%
\pgfpathcurveto{\pgfqpoint{1.612463in}{1.718604in}}{\pgfqpoint{1.615735in}{1.726504in}}{\pgfqpoint{1.615735in}{1.734740in}}%
\pgfpathcurveto{\pgfqpoint{1.615735in}{1.742977in}}{\pgfqpoint{1.612463in}{1.750877in}}{\pgfqpoint{1.606639in}{1.756700in}}%
\pgfpathcurveto{\pgfqpoint{1.600815in}{1.762524in}}{\pgfqpoint{1.592915in}{1.765797in}}{\pgfqpoint{1.584679in}{1.765797in}}%
\pgfpathcurveto{\pgfqpoint{1.576443in}{1.765797in}}{\pgfqpoint{1.568543in}{1.762524in}}{\pgfqpoint{1.562719in}{1.756700in}}%
\pgfpathcurveto{\pgfqpoint{1.556895in}{1.750877in}}{\pgfqpoint{1.553622in}{1.742977in}}{\pgfqpoint{1.553622in}{1.734740in}}%
\pgfpathcurveto{\pgfqpoint{1.553622in}{1.726504in}}{\pgfqpoint{1.556895in}{1.718604in}}{\pgfqpoint{1.562719in}{1.712780in}}%
\pgfpathcurveto{\pgfqpoint{1.568543in}{1.706956in}}{\pgfqpoint{1.576443in}{1.703684in}}{\pgfqpoint{1.584679in}{1.703684in}}%
\pgfpathclose%
\pgfusepath{stroke,fill}%
\end{pgfscope}%
\begin{pgfscope}%
\pgfpathrectangle{\pgfqpoint{0.100000in}{0.212622in}}{\pgfqpoint{3.696000in}{3.696000in}}%
\pgfusepath{clip}%
\pgfsetbuttcap%
\pgfsetroundjoin%
\definecolor{currentfill}{rgb}{0.121569,0.466667,0.705882}%
\pgfsetfillcolor{currentfill}%
\pgfsetfillopacity{0.569256}%
\pgfsetlinewidth{1.003750pt}%
\definecolor{currentstroke}{rgb}{0.121569,0.466667,0.705882}%
\pgfsetstrokecolor{currentstroke}%
\pgfsetstrokeopacity{0.569256}%
\pgfsetdash{}{0pt}%
\pgfpathmoveto{\pgfqpoint{0.889663in}{1.485971in}}%
\pgfpathcurveto{\pgfqpoint{0.897900in}{1.485971in}}{\pgfqpoint{0.905800in}{1.489243in}}{\pgfqpoint{0.911624in}{1.495067in}}%
\pgfpathcurveto{\pgfqpoint{0.917448in}{1.500891in}}{\pgfqpoint{0.920720in}{1.508791in}}{\pgfqpoint{0.920720in}{1.517028in}}%
\pgfpathcurveto{\pgfqpoint{0.920720in}{1.525264in}}{\pgfqpoint{0.917448in}{1.533164in}}{\pgfqpoint{0.911624in}{1.538988in}}%
\pgfpathcurveto{\pgfqpoint{0.905800in}{1.544812in}}{\pgfqpoint{0.897900in}{1.548084in}}{\pgfqpoint{0.889663in}{1.548084in}}%
\pgfpathcurveto{\pgfqpoint{0.881427in}{1.548084in}}{\pgfqpoint{0.873527in}{1.544812in}}{\pgfqpoint{0.867703in}{1.538988in}}%
\pgfpathcurveto{\pgfqpoint{0.861879in}{1.533164in}}{\pgfqpoint{0.858607in}{1.525264in}}{\pgfqpoint{0.858607in}{1.517028in}}%
\pgfpathcurveto{\pgfqpoint{0.858607in}{1.508791in}}{\pgfqpoint{0.861879in}{1.500891in}}{\pgfqpoint{0.867703in}{1.495067in}}%
\pgfpathcurveto{\pgfqpoint{0.873527in}{1.489243in}}{\pgfqpoint{0.881427in}{1.485971in}}{\pgfqpoint{0.889663in}{1.485971in}}%
\pgfpathclose%
\pgfusepath{stroke,fill}%
\end{pgfscope}%
\begin{pgfscope}%
\pgfpathrectangle{\pgfqpoint{0.100000in}{0.212622in}}{\pgfqpoint{3.696000in}{3.696000in}}%
\pgfusepath{clip}%
\pgfsetbuttcap%
\pgfsetroundjoin%
\definecolor{currentfill}{rgb}{0.121569,0.466667,0.705882}%
\pgfsetfillcolor{currentfill}%
\pgfsetfillopacity{0.570958}%
\pgfsetlinewidth{1.003750pt}%
\definecolor{currentstroke}{rgb}{0.121569,0.466667,0.705882}%
\pgfsetstrokecolor{currentstroke}%
\pgfsetstrokeopacity{0.570958}%
\pgfsetdash{}{0pt}%
\pgfpathmoveto{\pgfqpoint{0.882708in}{1.475035in}}%
\pgfpathcurveto{\pgfqpoint{0.890944in}{1.475035in}}{\pgfqpoint{0.898844in}{1.478308in}}{\pgfqpoint{0.904668in}{1.484131in}}%
\pgfpathcurveto{\pgfqpoint{0.910492in}{1.489955in}}{\pgfqpoint{0.913764in}{1.497855in}}{\pgfqpoint{0.913764in}{1.506092in}}%
\pgfpathcurveto{\pgfqpoint{0.913764in}{1.514328in}}{\pgfqpoint{0.910492in}{1.522228in}}{\pgfqpoint{0.904668in}{1.528052in}}%
\pgfpathcurveto{\pgfqpoint{0.898844in}{1.533876in}}{\pgfqpoint{0.890944in}{1.537148in}}{\pgfqpoint{0.882708in}{1.537148in}}%
\pgfpathcurveto{\pgfqpoint{0.874471in}{1.537148in}}{\pgfqpoint{0.866571in}{1.533876in}}{\pgfqpoint{0.860747in}{1.528052in}}%
\pgfpathcurveto{\pgfqpoint{0.854924in}{1.522228in}}{\pgfqpoint{0.851651in}{1.514328in}}{\pgfqpoint{0.851651in}{1.506092in}}%
\pgfpathcurveto{\pgfqpoint{0.851651in}{1.497855in}}{\pgfqpoint{0.854924in}{1.489955in}}{\pgfqpoint{0.860747in}{1.484131in}}%
\pgfpathcurveto{\pgfqpoint{0.866571in}{1.478308in}}{\pgfqpoint{0.874471in}{1.475035in}}{\pgfqpoint{0.882708in}{1.475035in}}%
\pgfpathclose%
\pgfusepath{stroke,fill}%
\end{pgfscope}%
\begin{pgfscope}%
\pgfpathrectangle{\pgfqpoint{0.100000in}{0.212622in}}{\pgfqpoint{3.696000in}{3.696000in}}%
\pgfusepath{clip}%
\pgfsetbuttcap%
\pgfsetroundjoin%
\definecolor{currentfill}{rgb}{0.121569,0.466667,0.705882}%
\pgfsetfillcolor{currentfill}%
\pgfsetfillopacity{0.573102}%
\pgfsetlinewidth{1.003750pt}%
\definecolor{currentstroke}{rgb}{0.121569,0.466667,0.705882}%
\pgfsetstrokecolor{currentstroke}%
\pgfsetstrokeopacity{0.573102}%
\pgfsetdash{}{0pt}%
\pgfpathmoveto{\pgfqpoint{0.875815in}{1.468417in}}%
\pgfpathcurveto{\pgfqpoint{0.884052in}{1.468417in}}{\pgfqpoint{0.891952in}{1.471689in}}{\pgfqpoint{0.897776in}{1.477513in}}%
\pgfpathcurveto{\pgfqpoint{0.903600in}{1.483337in}}{\pgfqpoint{0.906872in}{1.491237in}}{\pgfqpoint{0.906872in}{1.499473in}}%
\pgfpathcurveto{\pgfqpoint{0.906872in}{1.507709in}}{\pgfqpoint{0.903600in}{1.515609in}}{\pgfqpoint{0.897776in}{1.521433in}}%
\pgfpathcurveto{\pgfqpoint{0.891952in}{1.527257in}}{\pgfqpoint{0.884052in}{1.530530in}}{\pgfqpoint{0.875815in}{1.530530in}}%
\pgfpathcurveto{\pgfqpoint{0.867579in}{1.530530in}}{\pgfqpoint{0.859679in}{1.527257in}}{\pgfqpoint{0.853855in}{1.521433in}}%
\pgfpathcurveto{\pgfqpoint{0.848031in}{1.515609in}}{\pgfqpoint{0.844759in}{1.507709in}}{\pgfqpoint{0.844759in}{1.499473in}}%
\pgfpathcurveto{\pgfqpoint{0.844759in}{1.491237in}}{\pgfqpoint{0.848031in}{1.483337in}}{\pgfqpoint{0.853855in}{1.477513in}}%
\pgfpathcurveto{\pgfqpoint{0.859679in}{1.471689in}}{\pgfqpoint{0.867579in}{1.468417in}}{\pgfqpoint{0.875815in}{1.468417in}}%
\pgfpathclose%
\pgfusepath{stroke,fill}%
\end{pgfscope}%
\begin{pgfscope}%
\pgfpathrectangle{\pgfqpoint{0.100000in}{0.212622in}}{\pgfqpoint{3.696000in}{3.696000in}}%
\pgfusepath{clip}%
\pgfsetbuttcap%
\pgfsetroundjoin%
\definecolor{currentfill}{rgb}{0.121569,0.466667,0.705882}%
\pgfsetfillcolor{currentfill}%
\pgfsetfillopacity{0.574622}%
\pgfsetlinewidth{1.003750pt}%
\definecolor{currentstroke}{rgb}{0.121569,0.466667,0.705882}%
\pgfsetstrokecolor{currentstroke}%
\pgfsetstrokeopacity{0.574622}%
\pgfsetdash{}{0pt}%
\pgfpathmoveto{\pgfqpoint{1.585524in}{1.703699in}}%
\pgfpathcurveto{\pgfqpoint{1.593760in}{1.703699in}}{\pgfqpoint{1.601660in}{1.706972in}}{\pgfqpoint{1.607484in}{1.712796in}}%
\pgfpathcurveto{\pgfqpoint{1.613308in}{1.718620in}}{\pgfqpoint{1.616580in}{1.726520in}}{\pgfqpoint{1.616580in}{1.734756in}}%
\pgfpathcurveto{\pgfqpoint{1.616580in}{1.742992in}}{\pgfqpoint{1.613308in}{1.750892in}}{\pgfqpoint{1.607484in}{1.756716in}}%
\pgfpathcurveto{\pgfqpoint{1.601660in}{1.762540in}}{\pgfqpoint{1.593760in}{1.765812in}}{\pgfqpoint{1.585524in}{1.765812in}}%
\pgfpathcurveto{\pgfqpoint{1.577287in}{1.765812in}}{\pgfqpoint{1.569387in}{1.762540in}}{\pgfqpoint{1.563563in}{1.756716in}}%
\pgfpathcurveto{\pgfqpoint{1.557739in}{1.750892in}}{\pgfqpoint{1.554467in}{1.742992in}}{\pgfqpoint{1.554467in}{1.734756in}}%
\pgfpathcurveto{\pgfqpoint{1.554467in}{1.726520in}}{\pgfqpoint{1.557739in}{1.718620in}}{\pgfqpoint{1.563563in}{1.712796in}}%
\pgfpathcurveto{\pgfqpoint{1.569387in}{1.706972in}}{\pgfqpoint{1.577287in}{1.703699in}}{\pgfqpoint{1.585524in}{1.703699in}}%
\pgfpathclose%
\pgfusepath{stroke,fill}%
\end{pgfscope}%
\begin{pgfscope}%
\pgfpathrectangle{\pgfqpoint{0.100000in}{0.212622in}}{\pgfqpoint{3.696000in}{3.696000in}}%
\pgfusepath{clip}%
\pgfsetbuttcap%
\pgfsetroundjoin%
\definecolor{currentfill}{rgb}{0.121569,0.466667,0.705882}%
\pgfsetfillcolor{currentfill}%
\pgfsetfillopacity{0.576608}%
\pgfsetlinewidth{1.003750pt}%
\definecolor{currentstroke}{rgb}{0.121569,0.466667,0.705882}%
\pgfsetstrokecolor{currentstroke}%
\pgfsetstrokeopacity{0.576608}%
\pgfsetdash{}{0pt}%
\pgfpathmoveto{\pgfqpoint{0.870562in}{1.466529in}}%
\pgfpathcurveto{\pgfqpoint{0.878798in}{1.466529in}}{\pgfqpoint{0.886699in}{1.469801in}}{\pgfqpoint{0.892522in}{1.475625in}}%
\pgfpathcurveto{\pgfqpoint{0.898346in}{1.481449in}}{\pgfqpoint{0.901619in}{1.489349in}}{\pgfqpoint{0.901619in}{1.497585in}}%
\pgfpathcurveto{\pgfqpoint{0.901619in}{1.505821in}}{\pgfqpoint{0.898346in}{1.513721in}}{\pgfqpoint{0.892522in}{1.519545in}}%
\pgfpathcurveto{\pgfqpoint{0.886699in}{1.525369in}}{\pgfqpoint{0.878798in}{1.528642in}}{\pgfqpoint{0.870562in}{1.528642in}}%
\pgfpathcurveto{\pgfqpoint{0.862326in}{1.528642in}}{\pgfqpoint{0.854426in}{1.525369in}}{\pgfqpoint{0.848602in}{1.519545in}}%
\pgfpathcurveto{\pgfqpoint{0.842778in}{1.513721in}}{\pgfqpoint{0.839506in}{1.505821in}}{\pgfqpoint{0.839506in}{1.497585in}}%
\pgfpathcurveto{\pgfqpoint{0.839506in}{1.489349in}}{\pgfqpoint{0.842778in}{1.481449in}}{\pgfqpoint{0.848602in}{1.475625in}}%
\pgfpathcurveto{\pgfqpoint{0.854426in}{1.469801in}}{\pgfqpoint{0.862326in}{1.466529in}}{\pgfqpoint{0.870562in}{1.466529in}}%
\pgfpathclose%
\pgfusepath{stroke,fill}%
\end{pgfscope}%
\begin{pgfscope}%
\pgfpathrectangle{\pgfqpoint{0.100000in}{0.212622in}}{\pgfqpoint{3.696000in}{3.696000in}}%
\pgfusepath{clip}%
\pgfsetbuttcap%
\pgfsetroundjoin%
\definecolor{currentfill}{rgb}{0.121569,0.466667,0.705882}%
\pgfsetfillcolor{currentfill}%
\pgfsetfillopacity{0.578021}%
\pgfsetlinewidth{1.003750pt}%
\definecolor{currentstroke}{rgb}{0.121569,0.466667,0.705882}%
\pgfsetstrokecolor{currentstroke}%
\pgfsetstrokeopacity{0.578021}%
\pgfsetdash{}{0pt}%
\pgfpathmoveto{\pgfqpoint{0.865137in}{1.461180in}}%
\pgfpathcurveto{\pgfqpoint{0.873373in}{1.461180in}}{\pgfqpoint{0.881273in}{1.464452in}}{\pgfqpoint{0.887097in}{1.470276in}}%
\pgfpathcurveto{\pgfqpoint{0.892921in}{1.476100in}}{\pgfqpoint{0.896193in}{1.484000in}}{\pgfqpoint{0.896193in}{1.492236in}}%
\pgfpathcurveto{\pgfqpoint{0.896193in}{1.500473in}}{\pgfqpoint{0.892921in}{1.508373in}}{\pgfqpoint{0.887097in}{1.514197in}}%
\pgfpathcurveto{\pgfqpoint{0.881273in}{1.520020in}}{\pgfqpoint{0.873373in}{1.523293in}}{\pgfqpoint{0.865137in}{1.523293in}}%
\pgfpathcurveto{\pgfqpoint{0.856900in}{1.523293in}}{\pgfqpoint{0.849000in}{1.520020in}}{\pgfqpoint{0.843176in}{1.514197in}}%
\pgfpathcurveto{\pgfqpoint{0.837353in}{1.508373in}}{\pgfqpoint{0.834080in}{1.500473in}}{\pgfqpoint{0.834080in}{1.492236in}}%
\pgfpathcurveto{\pgfqpoint{0.834080in}{1.484000in}}{\pgfqpoint{0.837353in}{1.476100in}}{\pgfqpoint{0.843176in}{1.470276in}}%
\pgfpathcurveto{\pgfqpoint{0.849000in}{1.464452in}}{\pgfqpoint{0.856900in}{1.461180in}}{\pgfqpoint{0.865137in}{1.461180in}}%
\pgfpathclose%
\pgfusepath{stroke,fill}%
\end{pgfscope}%
\begin{pgfscope}%
\pgfpathrectangle{\pgfqpoint{0.100000in}{0.212622in}}{\pgfqpoint{3.696000in}{3.696000in}}%
\pgfusepath{clip}%
\pgfsetbuttcap%
\pgfsetroundjoin%
\definecolor{currentfill}{rgb}{0.121569,0.466667,0.705882}%
\pgfsetfillcolor{currentfill}%
\pgfsetfillopacity{0.580080}%
\pgfsetlinewidth{1.003750pt}%
\definecolor{currentstroke}{rgb}{0.121569,0.466667,0.705882}%
\pgfsetstrokecolor{currentstroke}%
\pgfsetstrokeopacity{0.580080}%
\pgfsetdash{}{0pt}%
\pgfpathmoveto{\pgfqpoint{0.860926in}{1.458475in}}%
\pgfpathcurveto{\pgfqpoint{0.869163in}{1.458475in}}{\pgfqpoint{0.877063in}{1.461747in}}{\pgfqpoint{0.882887in}{1.467571in}}%
\pgfpathcurveto{\pgfqpoint{0.888710in}{1.473395in}}{\pgfqpoint{0.891983in}{1.481295in}}{\pgfqpoint{0.891983in}{1.489531in}}%
\pgfpathcurveto{\pgfqpoint{0.891983in}{1.497768in}}{\pgfqpoint{0.888710in}{1.505668in}}{\pgfqpoint{0.882887in}{1.511492in}}%
\pgfpathcurveto{\pgfqpoint{0.877063in}{1.517316in}}{\pgfqpoint{0.869163in}{1.520588in}}{\pgfqpoint{0.860926in}{1.520588in}}%
\pgfpathcurveto{\pgfqpoint{0.852690in}{1.520588in}}{\pgfqpoint{0.844790in}{1.517316in}}{\pgfqpoint{0.838966in}{1.511492in}}%
\pgfpathcurveto{\pgfqpoint{0.833142in}{1.505668in}}{\pgfqpoint{0.829870in}{1.497768in}}{\pgfqpoint{0.829870in}{1.489531in}}%
\pgfpathcurveto{\pgfqpoint{0.829870in}{1.481295in}}{\pgfqpoint{0.833142in}{1.473395in}}{\pgfqpoint{0.838966in}{1.467571in}}%
\pgfpathcurveto{\pgfqpoint{0.844790in}{1.461747in}}{\pgfqpoint{0.852690in}{1.458475in}}{\pgfqpoint{0.860926in}{1.458475in}}%
\pgfpathclose%
\pgfusepath{stroke,fill}%
\end{pgfscope}%
\begin{pgfscope}%
\pgfpathrectangle{\pgfqpoint{0.100000in}{0.212622in}}{\pgfqpoint{3.696000in}{3.696000in}}%
\pgfusepath{clip}%
\pgfsetbuttcap%
\pgfsetroundjoin%
\definecolor{currentfill}{rgb}{0.121569,0.466667,0.705882}%
\pgfsetfillcolor{currentfill}%
\pgfsetfillopacity{0.580588}%
\pgfsetlinewidth{1.003750pt}%
\definecolor{currentstroke}{rgb}{0.121569,0.466667,0.705882}%
\pgfsetstrokecolor{currentstroke}%
\pgfsetstrokeopacity{0.580588}%
\pgfsetdash{}{0pt}%
\pgfpathmoveto{\pgfqpoint{1.588696in}{1.702708in}}%
\pgfpathcurveto{\pgfqpoint{1.596933in}{1.702708in}}{\pgfqpoint{1.604833in}{1.705981in}}{\pgfqpoint{1.610657in}{1.711805in}}%
\pgfpathcurveto{\pgfqpoint{1.616481in}{1.717629in}}{\pgfqpoint{1.619753in}{1.725529in}}{\pgfqpoint{1.619753in}{1.733765in}}%
\pgfpathcurveto{\pgfqpoint{1.619753in}{1.742001in}}{\pgfqpoint{1.616481in}{1.749901in}}{\pgfqpoint{1.610657in}{1.755725in}}%
\pgfpathcurveto{\pgfqpoint{1.604833in}{1.761549in}}{\pgfqpoint{1.596933in}{1.764821in}}{\pgfqpoint{1.588696in}{1.764821in}}%
\pgfpathcurveto{\pgfqpoint{1.580460in}{1.764821in}}{\pgfqpoint{1.572560in}{1.761549in}}{\pgfqpoint{1.566736in}{1.755725in}}%
\pgfpathcurveto{\pgfqpoint{1.560912in}{1.749901in}}{\pgfqpoint{1.557640in}{1.742001in}}{\pgfqpoint{1.557640in}{1.733765in}}%
\pgfpathcurveto{\pgfqpoint{1.557640in}{1.725529in}}{\pgfqpoint{1.560912in}{1.717629in}}{\pgfqpoint{1.566736in}{1.711805in}}%
\pgfpathcurveto{\pgfqpoint{1.572560in}{1.705981in}}{\pgfqpoint{1.580460in}{1.702708in}}{\pgfqpoint{1.588696in}{1.702708in}}%
\pgfpathclose%
\pgfusepath{stroke,fill}%
\end{pgfscope}%
\begin{pgfscope}%
\pgfpathrectangle{\pgfqpoint{0.100000in}{0.212622in}}{\pgfqpoint{3.696000in}{3.696000in}}%
\pgfusepath{clip}%
\pgfsetbuttcap%
\pgfsetroundjoin%
\definecolor{currentfill}{rgb}{0.121569,0.466667,0.705882}%
\pgfsetfillcolor{currentfill}%
\pgfsetfillopacity{0.583823}%
\pgfsetlinewidth{1.003750pt}%
\definecolor{currentstroke}{rgb}{0.121569,0.466667,0.705882}%
\pgfsetstrokecolor{currentstroke}%
\pgfsetstrokeopacity{0.583823}%
\pgfsetdash{}{0pt}%
\pgfpathmoveto{\pgfqpoint{1.590071in}{1.701759in}}%
\pgfpathcurveto{\pgfqpoint{1.598307in}{1.701759in}}{\pgfqpoint{1.606207in}{1.705031in}}{\pgfqpoint{1.612031in}{1.710855in}}%
\pgfpathcurveto{\pgfqpoint{1.617855in}{1.716679in}}{\pgfqpoint{1.621127in}{1.724579in}}{\pgfqpoint{1.621127in}{1.732816in}}%
\pgfpathcurveto{\pgfqpoint{1.621127in}{1.741052in}}{\pgfqpoint{1.617855in}{1.748952in}}{\pgfqpoint{1.612031in}{1.754776in}}%
\pgfpathcurveto{\pgfqpoint{1.606207in}{1.760600in}}{\pgfqpoint{1.598307in}{1.763872in}}{\pgfqpoint{1.590071in}{1.763872in}}%
\pgfpathcurveto{\pgfqpoint{1.581835in}{1.763872in}}{\pgfqpoint{1.573934in}{1.760600in}}{\pgfqpoint{1.568111in}{1.754776in}}%
\pgfpathcurveto{\pgfqpoint{1.562287in}{1.748952in}}{\pgfqpoint{1.559014in}{1.741052in}}{\pgfqpoint{1.559014in}{1.732816in}}%
\pgfpathcurveto{\pgfqpoint{1.559014in}{1.724579in}}{\pgfqpoint{1.562287in}{1.716679in}}{\pgfqpoint{1.568111in}{1.710855in}}%
\pgfpathcurveto{\pgfqpoint{1.573934in}{1.705031in}}{\pgfqpoint{1.581835in}{1.701759in}}{\pgfqpoint{1.590071in}{1.701759in}}%
\pgfpathclose%
\pgfusepath{stroke,fill}%
\end{pgfscope}%
\begin{pgfscope}%
\pgfpathrectangle{\pgfqpoint{0.100000in}{0.212622in}}{\pgfqpoint{3.696000in}{3.696000in}}%
\pgfusepath{clip}%
\pgfsetbuttcap%
\pgfsetroundjoin%
\definecolor{currentfill}{rgb}{0.121569,0.466667,0.705882}%
\pgfsetfillcolor{currentfill}%
\pgfsetfillopacity{0.583877}%
\pgfsetlinewidth{1.003750pt}%
\definecolor{currentstroke}{rgb}{0.121569,0.466667,0.705882}%
\pgfsetstrokecolor{currentstroke}%
\pgfsetstrokeopacity{0.583877}%
\pgfsetdash{}{0pt}%
\pgfpathmoveto{\pgfqpoint{0.852975in}{1.454158in}}%
\pgfpathcurveto{\pgfqpoint{0.861211in}{1.454158in}}{\pgfqpoint{0.869111in}{1.457431in}}{\pgfqpoint{0.874935in}{1.463255in}}%
\pgfpathcurveto{\pgfqpoint{0.880759in}{1.469078in}}{\pgfqpoint{0.884031in}{1.476979in}}{\pgfqpoint{0.884031in}{1.485215in}}%
\pgfpathcurveto{\pgfqpoint{0.884031in}{1.493451in}}{\pgfqpoint{0.880759in}{1.501351in}}{\pgfqpoint{0.874935in}{1.507175in}}%
\pgfpathcurveto{\pgfqpoint{0.869111in}{1.512999in}}{\pgfqpoint{0.861211in}{1.516271in}}{\pgfqpoint{0.852975in}{1.516271in}}%
\pgfpathcurveto{\pgfqpoint{0.844739in}{1.516271in}}{\pgfqpoint{0.836839in}{1.512999in}}{\pgfqpoint{0.831015in}{1.507175in}}%
\pgfpathcurveto{\pgfqpoint{0.825191in}{1.501351in}}{\pgfqpoint{0.821918in}{1.493451in}}{\pgfqpoint{0.821918in}{1.485215in}}%
\pgfpathcurveto{\pgfqpoint{0.821918in}{1.476979in}}{\pgfqpoint{0.825191in}{1.469078in}}{\pgfqpoint{0.831015in}{1.463255in}}%
\pgfpathcurveto{\pgfqpoint{0.836839in}{1.457431in}}{\pgfqpoint{0.844739in}{1.454158in}}{\pgfqpoint{0.852975in}{1.454158in}}%
\pgfpathclose%
\pgfusepath{stroke,fill}%
\end{pgfscope}%
\begin{pgfscope}%
\pgfpathrectangle{\pgfqpoint{0.100000in}{0.212622in}}{\pgfqpoint{3.696000in}{3.696000in}}%
\pgfusepath{clip}%
\pgfsetbuttcap%
\pgfsetroundjoin%
\definecolor{currentfill}{rgb}{0.121569,0.466667,0.705882}%
\pgfsetfillcolor{currentfill}%
\pgfsetfillopacity{0.587083}%
\pgfsetlinewidth{1.003750pt}%
\definecolor{currentstroke}{rgb}{0.121569,0.466667,0.705882}%
\pgfsetstrokecolor{currentstroke}%
\pgfsetstrokeopacity{0.587083}%
\pgfsetdash{}{0pt}%
\pgfpathmoveto{\pgfqpoint{0.844202in}{1.449734in}}%
\pgfpathcurveto{\pgfqpoint{0.852439in}{1.449734in}}{\pgfqpoint{0.860339in}{1.453007in}}{\pgfqpoint{0.866163in}{1.458831in}}%
\pgfpathcurveto{\pgfqpoint{0.871987in}{1.464654in}}{\pgfqpoint{0.875259in}{1.472555in}}{\pgfqpoint{0.875259in}{1.480791in}}%
\pgfpathcurveto{\pgfqpoint{0.875259in}{1.489027in}}{\pgfqpoint{0.871987in}{1.496927in}}{\pgfqpoint{0.866163in}{1.502751in}}%
\pgfpathcurveto{\pgfqpoint{0.860339in}{1.508575in}}{\pgfqpoint{0.852439in}{1.511847in}}{\pgfqpoint{0.844202in}{1.511847in}}%
\pgfpathcurveto{\pgfqpoint{0.835966in}{1.511847in}}{\pgfqpoint{0.828066in}{1.508575in}}{\pgfqpoint{0.822242in}{1.502751in}}%
\pgfpathcurveto{\pgfqpoint{0.816418in}{1.496927in}}{\pgfqpoint{0.813146in}{1.489027in}}{\pgfqpoint{0.813146in}{1.480791in}}%
\pgfpathcurveto{\pgfqpoint{0.813146in}{1.472555in}}{\pgfqpoint{0.816418in}{1.464654in}}{\pgfqpoint{0.822242in}{1.458831in}}%
\pgfpathcurveto{\pgfqpoint{0.828066in}{1.453007in}}{\pgfqpoint{0.835966in}{1.449734in}}{\pgfqpoint{0.844202in}{1.449734in}}%
\pgfpathclose%
\pgfusepath{stroke,fill}%
\end{pgfscope}%
\begin{pgfscope}%
\pgfpathrectangle{\pgfqpoint{0.100000in}{0.212622in}}{\pgfqpoint{3.696000in}{3.696000in}}%
\pgfusepath{clip}%
\pgfsetbuttcap%
\pgfsetroundjoin%
\definecolor{currentfill}{rgb}{0.121569,0.466667,0.705882}%
\pgfsetfillcolor{currentfill}%
\pgfsetfillopacity{0.587971}%
\pgfsetlinewidth{1.003750pt}%
\definecolor{currentstroke}{rgb}{0.121569,0.466667,0.705882}%
\pgfsetstrokecolor{currentstroke}%
\pgfsetstrokeopacity{0.587971}%
\pgfsetdash{}{0pt}%
\pgfpathmoveto{\pgfqpoint{1.591811in}{1.702021in}}%
\pgfpathcurveto{\pgfqpoint{1.600047in}{1.702021in}}{\pgfqpoint{1.607947in}{1.705293in}}{\pgfqpoint{1.613771in}{1.711117in}}%
\pgfpathcurveto{\pgfqpoint{1.619595in}{1.716941in}}{\pgfqpoint{1.622868in}{1.724841in}}{\pgfqpoint{1.622868in}{1.733077in}}%
\pgfpathcurveto{\pgfqpoint{1.622868in}{1.741314in}}{\pgfqpoint{1.619595in}{1.749214in}}{\pgfqpoint{1.613771in}{1.755038in}}%
\pgfpathcurveto{\pgfqpoint{1.607947in}{1.760861in}}{\pgfqpoint{1.600047in}{1.764134in}}{\pgfqpoint{1.591811in}{1.764134in}}%
\pgfpathcurveto{\pgfqpoint{1.583575in}{1.764134in}}{\pgfqpoint{1.575675in}{1.760861in}}{\pgfqpoint{1.569851in}{1.755038in}}%
\pgfpathcurveto{\pgfqpoint{1.564027in}{1.749214in}}{\pgfqpoint{1.560755in}{1.741314in}}{\pgfqpoint{1.560755in}{1.733077in}}%
\pgfpathcurveto{\pgfqpoint{1.560755in}{1.724841in}}{\pgfqpoint{1.564027in}{1.716941in}}{\pgfqpoint{1.569851in}{1.711117in}}%
\pgfpathcurveto{\pgfqpoint{1.575675in}{1.705293in}}{\pgfqpoint{1.583575in}{1.702021in}}{\pgfqpoint{1.591811in}{1.702021in}}%
\pgfpathclose%
\pgfusepath{stroke,fill}%
\end{pgfscope}%
\begin{pgfscope}%
\pgfpathrectangle{\pgfqpoint{0.100000in}{0.212622in}}{\pgfqpoint{3.696000in}{3.696000in}}%
\pgfusepath{clip}%
\pgfsetbuttcap%
\pgfsetroundjoin%
\definecolor{currentfill}{rgb}{0.121569,0.466667,0.705882}%
\pgfsetfillcolor{currentfill}%
\pgfsetfillopacity{0.590114}%
\pgfsetlinewidth{1.003750pt}%
\definecolor{currentstroke}{rgb}{0.121569,0.466667,0.705882}%
\pgfsetstrokecolor{currentstroke}%
\pgfsetstrokeopacity{0.590114}%
\pgfsetdash{}{0pt}%
\pgfpathmoveto{\pgfqpoint{1.592646in}{1.701471in}}%
\pgfpathcurveto{\pgfqpoint{1.600882in}{1.701471in}}{\pgfqpoint{1.608782in}{1.704743in}}{\pgfqpoint{1.614606in}{1.710567in}}%
\pgfpathcurveto{\pgfqpoint{1.620430in}{1.716391in}}{\pgfqpoint{1.623703in}{1.724291in}}{\pgfqpoint{1.623703in}{1.732527in}}%
\pgfpathcurveto{\pgfqpoint{1.623703in}{1.740763in}}{\pgfqpoint{1.620430in}{1.748664in}}{\pgfqpoint{1.614606in}{1.754487in}}%
\pgfpathcurveto{\pgfqpoint{1.608782in}{1.760311in}}{\pgfqpoint{1.600882in}{1.763584in}}{\pgfqpoint{1.592646in}{1.763584in}}%
\pgfpathcurveto{\pgfqpoint{1.584410in}{1.763584in}}{\pgfqpoint{1.576510in}{1.760311in}}{\pgfqpoint{1.570686in}{1.754487in}}%
\pgfpathcurveto{\pgfqpoint{1.564862in}{1.748664in}}{\pgfqpoint{1.561590in}{1.740763in}}{\pgfqpoint{1.561590in}{1.732527in}}%
\pgfpathcurveto{\pgfqpoint{1.561590in}{1.724291in}}{\pgfqpoint{1.564862in}{1.716391in}}{\pgfqpoint{1.570686in}{1.710567in}}%
\pgfpathcurveto{\pgfqpoint{1.576510in}{1.704743in}}{\pgfqpoint{1.584410in}{1.701471in}}{\pgfqpoint{1.592646in}{1.701471in}}%
\pgfpathclose%
\pgfusepath{stroke,fill}%
\end{pgfscope}%
\begin{pgfscope}%
\pgfpathrectangle{\pgfqpoint{0.100000in}{0.212622in}}{\pgfqpoint{3.696000in}{3.696000in}}%
\pgfusepath{clip}%
\pgfsetbuttcap%
\pgfsetroundjoin%
\definecolor{currentfill}{rgb}{0.121569,0.466667,0.705882}%
\pgfsetfillcolor{currentfill}%
\pgfsetfillopacity{0.591081}%
\pgfsetlinewidth{1.003750pt}%
\definecolor{currentstroke}{rgb}{0.121569,0.466667,0.705882}%
\pgfsetstrokecolor{currentstroke}%
\pgfsetstrokeopacity{0.591081}%
\pgfsetdash{}{0pt}%
\pgfpathmoveto{\pgfqpoint{0.839043in}{1.449489in}}%
\pgfpathcurveto{\pgfqpoint{0.847279in}{1.449489in}}{\pgfqpoint{0.855180in}{1.452761in}}{\pgfqpoint{0.861003in}{1.458585in}}%
\pgfpathcurveto{\pgfqpoint{0.866827in}{1.464409in}}{\pgfqpoint{0.870100in}{1.472309in}}{\pgfqpoint{0.870100in}{1.480546in}}%
\pgfpathcurveto{\pgfqpoint{0.870100in}{1.488782in}}{\pgfqpoint{0.866827in}{1.496682in}}{\pgfqpoint{0.861003in}{1.502506in}}%
\pgfpathcurveto{\pgfqpoint{0.855180in}{1.508330in}}{\pgfqpoint{0.847279in}{1.511602in}}{\pgfqpoint{0.839043in}{1.511602in}}%
\pgfpathcurveto{\pgfqpoint{0.830807in}{1.511602in}}{\pgfqpoint{0.822907in}{1.508330in}}{\pgfqpoint{0.817083in}{1.502506in}}%
\pgfpathcurveto{\pgfqpoint{0.811259in}{1.496682in}}{\pgfqpoint{0.807987in}{1.488782in}}{\pgfqpoint{0.807987in}{1.480546in}}%
\pgfpathcurveto{\pgfqpoint{0.807987in}{1.472309in}}{\pgfqpoint{0.811259in}{1.464409in}}{\pgfqpoint{0.817083in}{1.458585in}}%
\pgfpathcurveto{\pgfqpoint{0.822907in}{1.452761in}}{\pgfqpoint{0.830807in}{1.449489in}}{\pgfqpoint{0.839043in}{1.449489in}}%
\pgfpathclose%
\pgfusepath{stroke,fill}%
\end{pgfscope}%
\begin{pgfscope}%
\pgfpathrectangle{\pgfqpoint{0.100000in}{0.212622in}}{\pgfqpoint{3.696000in}{3.696000in}}%
\pgfusepath{clip}%
\pgfsetbuttcap%
\pgfsetroundjoin%
\definecolor{currentfill}{rgb}{0.121569,0.466667,0.705882}%
\pgfsetfillcolor{currentfill}%
\pgfsetfillopacity{0.591217}%
\pgfsetlinewidth{1.003750pt}%
\definecolor{currentstroke}{rgb}{0.121569,0.466667,0.705882}%
\pgfsetstrokecolor{currentstroke}%
\pgfsetstrokeopacity{0.591217}%
\pgfsetdash{}{0pt}%
\pgfpathmoveto{\pgfqpoint{1.593095in}{1.700812in}}%
\pgfpathcurveto{\pgfqpoint{1.601331in}{1.700812in}}{\pgfqpoint{1.609231in}{1.704085in}}{\pgfqpoint{1.615055in}{1.709909in}}%
\pgfpathcurveto{\pgfqpoint{1.620879in}{1.715733in}}{\pgfqpoint{1.624151in}{1.723633in}}{\pgfqpoint{1.624151in}{1.731869in}}%
\pgfpathcurveto{\pgfqpoint{1.624151in}{1.740105in}}{\pgfqpoint{1.620879in}{1.748005in}}{\pgfqpoint{1.615055in}{1.753829in}}%
\pgfpathcurveto{\pgfqpoint{1.609231in}{1.759653in}}{\pgfqpoint{1.601331in}{1.762925in}}{\pgfqpoint{1.593095in}{1.762925in}}%
\pgfpathcurveto{\pgfqpoint{1.584858in}{1.762925in}}{\pgfqpoint{1.576958in}{1.759653in}}{\pgfqpoint{1.571134in}{1.753829in}}%
\pgfpathcurveto{\pgfqpoint{1.565311in}{1.748005in}}{\pgfqpoint{1.562038in}{1.740105in}}{\pgfqpoint{1.562038in}{1.731869in}}%
\pgfpathcurveto{\pgfqpoint{1.562038in}{1.723633in}}{\pgfqpoint{1.565311in}{1.715733in}}{\pgfqpoint{1.571134in}{1.709909in}}%
\pgfpathcurveto{\pgfqpoint{1.576958in}{1.704085in}}{\pgfqpoint{1.584858in}{1.700812in}}{\pgfqpoint{1.593095in}{1.700812in}}%
\pgfpathclose%
\pgfusepath{stroke,fill}%
\end{pgfscope}%
\begin{pgfscope}%
\pgfpathrectangle{\pgfqpoint{0.100000in}{0.212622in}}{\pgfqpoint{3.696000in}{3.696000in}}%
\pgfusepath{clip}%
\pgfsetbuttcap%
\pgfsetroundjoin%
\definecolor{currentfill}{rgb}{0.121569,0.466667,0.705882}%
\pgfsetfillcolor{currentfill}%
\pgfsetfillopacity{0.592499}%
\pgfsetlinewidth{1.003750pt}%
\definecolor{currentstroke}{rgb}{0.121569,0.466667,0.705882}%
\pgfsetstrokecolor{currentstroke}%
\pgfsetstrokeopacity{0.592499}%
\pgfsetdash{}{0pt}%
\pgfpathmoveto{\pgfqpoint{0.833516in}{1.439372in}}%
\pgfpathcurveto{\pgfqpoint{0.841752in}{1.439372in}}{\pgfqpoint{0.849652in}{1.442644in}}{\pgfqpoint{0.855476in}{1.448468in}}%
\pgfpathcurveto{\pgfqpoint{0.861300in}{1.454292in}}{\pgfqpoint{0.864572in}{1.462192in}}{\pgfqpoint{0.864572in}{1.470428in}}%
\pgfpathcurveto{\pgfqpoint{0.864572in}{1.478665in}}{\pgfqpoint{0.861300in}{1.486565in}}{\pgfqpoint{0.855476in}{1.492389in}}%
\pgfpathcurveto{\pgfqpoint{0.849652in}{1.498212in}}{\pgfqpoint{0.841752in}{1.501485in}}{\pgfqpoint{0.833516in}{1.501485in}}%
\pgfpathcurveto{\pgfqpoint{0.825280in}{1.501485in}}{\pgfqpoint{0.817380in}{1.498212in}}{\pgfqpoint{0.811556in}{1.492389in}}%
\pgfpathcurveto{\pgfqpoint{0.805732in}{1.486565in}}{\pgfqpoint{0.802459in}{1.478665in}}{\pgfqpoint{0.802459in}{1.470428in}}%
\pgfpathcurveto{\pgfqpoint{0.802459in}{1.462192in}}{\pgfqpoint{0.805732in}{1.454292in}}{\pgfqpoint{0.811556in}{1.448468in}}%
\pgfpathcurveto{\pgfqpoint{0.817380in}{1.442644in}}{\pgfqpoint{0.825280in}{1.439372in}}{\pgfqpoint{0.833516in}{1.439372in}}%
\pgfpathclose%
\pgfusepath{stroke,fill}%
\end{pgfscope}%
\begin{pgfscope}%
\pgfpathrectangle{\pgfqpoint{0.100000in}{0.212622in}}{\pgfqpoint{3.696000in}{3.696000in}}%
\pgfusepath{clip}%
\pgfsetbuttcap%
\pgfsetroundjoin%
\definecolor{currentfill}{rgb}{0.121569,0.466667,0.705882}%
\pgfsetfillcolor{currentfill}%
\pgfsetfillopacity{0.592527}%
\pgfsetlinewidth{1.003750pt}%
\definecolor{currentstroke}{rgb}{0.121569,0.466667,0.705882}%
\pgfsetstrokecolor{currentstroke}%
\pgfsetstrokeopacity{0.592527}%
\pgfsetdash{}{0pt}%
\pgfpathmoveto{\pgfqpoint{1.594026in}{1.700318in}}%
\pgfpathcurveto{\pgfqpoint{1.602262in}{1.700318in}}{\pgfqpoint{1.610162in}{1.703590in}}{\pgfqpoint{1.615986in}{1.709414in}}%
\pgfpathcurveto{\pgfqpoint{1.621810in}{1.715238in}}{\pgfqpoint{1.625083in}{1.723138in}}{\pgfqpoint{1.625083in}{1.731375in}}%
\pgfpathcurveto{\pgfqpoint{1.625083in}{1.739611in}}{\pgfqpoint{1.621810in}{1.747511in}}{\pgfqpoint{1.615986in}{1.753335in}}%
\pgfpathcurveto{\pgfqpoint{1.610162in}{1.759159in}}{\pgfqpoint{1.602262in}{1.762431in}}{\pgfqpoint{1.594026in}{1.762431in}}%
\pgfpathcurveto{\pgfqpoint{1.585790in}{1.762431in}}{\pgfqpoint{1.577890in}{1.759159in}}{\pgfqpoint{1.572066in}{1.753335in}}%
\pgfpathcurveto{\pgfqpoint{1.566242in}{1.747511in}}{\pgfqpoint{1.562970in}{1.739611in}}{\pgfqpoint{1.562970in}{1.731375in}}%
\pgfpathcurveto{\pgfqpoint{1.562970in}{1.723138in}}{\pgfqpoint{1.566242in}{1.715238in}}{\pgfqpoint{1.572066in}{1.709414in}}%
\pgfpathcurveto{\pgfqpoint{1.577890in}{1.703590in}}{\pgfqpoint{1.585790in}{1.700318in}}{\pgfqpoint{1.594026in}{1.700318in}}%
\pgfpathclose%
\pgfusepath{stroke,fill}%
\end{pgfscope}%
\begin{pgfscope}%
\pgfpathrectangle{\pgfqpoint{0.100000in}{0.212622in}}{\pgfqpoint{3.696000in}{3.696000in}}%
\pgfusepath{clip}%
\pgfsetbuttcap%
\pgfsetroundjoin%
\definecolor{currentfill}{rgb}{0.121569,0.466667,0.705882}%
\pgfsetfillcolor{currentfill}%
\pgfsetfillopacity{0.594424}%
\pgfsetlinewidth{1.003750pt}%
\definecolor{currentstroke}{rgb}{0.121569,0.466667,0.705882}%
\pgfsetstrokecolor{currentstroke}%
\pgfsetstrokeopacity{0.594424}%
\pgfsetdash{}{0pt}%
\pgfpathmoveto{\pgfqpoint{1.594326in}{1.699810in}}%
\pgfpathcurveto{\pgfqpoint{1.602562in}{1.699810in}}{\pgfqpoint{1.610463in}{1.703082in}}{\pgfqpoint{1.616286in}{1.708906in}}%
\pgfpathcurveto{\pgfqpoint{1.622110in}{1.714730in}}{\pgfqpoint{1.625383in}{1.722630in}}{\pgfqpoint{1.625383in}{1.730866in}}%
\pgfpathcurveto{\pgfqpoint{1.625383in}{1.739103in}}{\pgfqpoint{1.622110in}{1.747003in}}{\pgfqpoint{1.616286in}{1.752827in}}%
\pgfpathcurveto{\pgfqpoint{1.610463in}{1.758651in}}{\pgfqpoint{1.602562in}{1.761923in}}{\pgfqpoint{1.594326in}{1.761923in}}%
\pgfpathcurveto{\pgfqpoint{1.586090in}{1.761923in}}{\pgfqpoint{1.578190in}{1.758651in}}{\pgfqpoint{1.572366in}{1.752827in}}%
\pgfpathcurveto{\pgfqpoint{1.566542in}{1.747003in}}{\pgfqpoint{1.563270in}{1.739103in}}{\pgfqpoint{1.563270in}{1.730866in}}%
\pgfpathcurveto{\pgfqpoint{1.563270in}{1.722630in}}{\pgfqpoint{1.566542in}{1.714730in}}{\pgfqpoint{1.572366in}{1.708906in}}%
\pgfpathcurveto{\pgfqpoint{1.578190in}{1.703082in}}{\pgfqpoint{1.586090in}{1.699810in}}{\pgfqpoint{1.594326in}{1.699810in}}%
\pgfpathclose%
\pgfusepath{stroke,fill}%
\end{pgfscope}%
\begin{pgfscope}%
\pgfpathrectangle{\pgfqpoint{0.100000in}{0.212622in}}{\pgfqpoint{3.696000in}{3.696000in}}%
\pgfusepath{clip}%
\pgfsetbuttcap%
\pgfsetroundjoin%
\definecolor{currentfill}{rgb}{0.121569,0.466667,0.705882}%
\pgfsetfillcolor{currentfill}%
\pgfsetfillopacity{0.594901}%
\pgfsetlinewidth{1.003750pt}%
\definecolor{currentstroke}{rgb}{0.121569,0.466667,0.705882}%
\pgfsetstrokecolor{currentstroke}%
\pgfsetstrokeopacity{0.594901}%
\pgfsetdash{}{0pt}%
\pgfpathmoveto{\pgfqpoint{0.827902in}{1.435214in}}%
\pgfpathcurveto{\pgfqpoint{0.836139in}{1.435214in}}{\pgfqpoint{0.844039in}{1.438486in}}{\pgfqpoint{0.849863in}{1.444310in}}%
\pgfpathcurveto{\pgfqpoint{0.855686in}{1.450134in}}{\pgfqpoint{0.858959in}{1.458034in}}{\pgfqpoint{0.858959in}{1.466271in}}%
\pgfpathcurveto{\pgfqpoint{0.858959in}{1.474507in}}{\pgfqpoint{0.855686in}{1.482407in}}{\pgfqpoint{0.849863in}{1.488231in}}%
\pgfpathcurveto{\pgfqpoint{0.844039in}{1.494055in}}{\pgfqpoint{0.836139in}{1.497327in}}{\pgfqpoint{0.827902in}{1.497327in}}%
\pgfpathcurveto{\pgfqpoint{0.819666in}{1.497327in}}{\pgfqpoint{0.811766in}{1.494055in}}{\pgfqpoint{0.805942in}{1.488231in}}%
\pgfpathcurveto{\pgfqpoint{0.800118in}{1.482407in}}{\pgfqpoint{0.796846in}{1.474507in}}{\pgfqpoint{0.796846in}{1.466271in}}%
\pgfpathcurveto{\pgfqpoint{0.796846in}{1.458034in}}{\pgfqpoint{0.800118in}{1.450134in}}{\pgfqpoint{0.805942in}{1.444310in}}%
\pgfpathcurveto{\pgfqpoint{0.811766in}{1.438486in}}{\pgfqpoint{0.819666in}{1.435214in}}{\pgfqpoint{0.827902in}{1.435214in}}%
\pgfpathclose%
\pgfusepath{stroke,fill}%
\end{pgfscope}%
\begin{pgfscope}%
\pgfpathrectangle{\pgfqpoint{0.100000in}{0.212622in}}{\pgfqpoint{3.696000in}{3.696000in}}%
\pgfusepath{clip}%
\pgfsetbuttcap%
\pgfsetroundjoin%
\definecolor{currentfill}{rgb}{0.121569,0.466667,0.705882}%
\pgfsetfillcolor{currentfill}%
\pgfsetfillopacity{0.596996}%
\pgfsetlinewidth{1.003750pt}%
\definecolor{currentstroke}{rgb}{0.121569,0.466667,0.705882}%
\pgfsetstrokecolor{currentstroke}%
\pgfsetstrokeopacity{0.596996}%
\pgfsetdash{}{0pt}%
\pgfpathmoveto{\pgfqpoint{1.595711in}{1.699013in}}%
\pgfpathcurveto{\pgfqpoint{1.603947in}{1.699013in}}{\pgfqpoint{1.611847in}{1.702285in}}{\pgfqpoint{1.617671in}{1.708109in}}%
\pgfpathcurveto{\pgfqpoint{1.623495in}{1.713933in}}{\pgfqpoint{1.626767in}{1.721833in}}{\pgfqpoint{1.626767in}{1.730070in}}%
\pgfpathcurveto{\pgfqpoint{1.626767in}{1.738306in}}{\pgfqpoint{1.623495in}{1.746206in}}{\pgfqpoint{1.617671in}{1.752030in}}%
\pgfpathcurveto{\pgfqpoint{1.611847in}{1.757854in}}{\pgfqpoint{1.603947in}{1.761126in}}{\pgfqpoint{1.595711in}{1.761126in}}%
\pgfpathcurveto{\pgfqpoint{1.587475in}{1.761126in}}{\pgfqpoint{1.579575in}{1.757854in}}{\pgfqpoint{1.573751in}{1.752030in}}%
\pgfpathcurveto{\pgfqpoint{1.567927in}{1.746206in}}{\pgfqpoint{1.564654in}{1.738306in}}{\pgfqpoint{1.564654in}{1.730070in}}%
\pgfpathcurveto{\pgfqpoint{1.564654in}{1.721833in}}{\pgfqpoint{1.567927in}{1.713933in}}{\pgfqpoint{1.573751in}{1.708109in}}%
\pgfpathcurveto{\pgfqpoint{1.579575in}{1.702285in}}{\pgfqpoint{1.587475in}{1.699013in}}{\pgfqpoint{1.595711in}{1.699013in}}%
\pgfpathclose%
\pgfusepath{stroke,fill}%
\end{pgfscope}%
\begin{pgfscope}%
\pgfpathrectangle{\pgfqpoint{0.100000in}{0.212622in}}{\pgfqpoint{3.696000in}{3.696000in}}%
\pgfusepath{clip}%
\pgfsetbuttcap%
\pgfsetroundjoin%
\definecolor{currentfill}{rgb}{0.121569,0.466667,0.705882}%
\pgfsetfillcolor{currentfill}%
\pgfsetfillopacity{0.598054}%
\pgfsetlinewidth{1.003750pt}%
\definecolor{currentstroke}{rgb}{0.121569,0.466667,0.705882}%
\pgfsetstrokecolor{currentstroke}%
\pgfsetstrokeopacity{0.598054}%
\pgfsetdash{}{0pt}%
\pgfpathmoveto{\pgfqpoint{0.824622in}{1.434320in}}%
\pgfpathcurveto{\pgfqpoint{0.832858in}{1.434320in}}{\pgfqpoint{0.840758in}{1.437592in}}{\pgfqpoint{0.846582in}{1.443416in}}%
\pgfpathcurveto{\pgfqpoint{0.852406in}{1.449240in}}{\pgfqpoint{0.855678in}{1.457140in}}{\pgfqpoint{0.855678in}{1.465377in}}%
\pgfpathcurveto{\pgfqpoint{0.855678in}{1.473613in}}{\pgfqpoint{0.852406in}{1.481513in}}{\pgfqpoint{0.846582in}{1.487337in}}%
\pgfpathcurveto{\pgfqpoint{0.840758in}{1.493161in}}{\pgfqpoint{0.832858in}{1.496433in}}{\pgfqpoint{0.824622in}{1.496433in}}%
\pgfpathcurveto{\pgfqpoint{0.816386in}{1.496433in}}{\pgfqpoint{0.808486in}{1.493161in}}{\pgfqpoint{0.802662in}{1.487337in}}%
\pgfpathcurveto{\pgfqpoint{0.796838in}{1.481513in}}{\pgfqpoint{0.793565in}{1.473613in}}{\pgfqpoint{0.793565in}{1.465377in}}%
\pgfpathcurveto{\pgfqpoint{0.793565in}{1.457140in}}{\pgfqpoint{0.796838in}{1.449240in}}{\pgfqpoint{0.802662in}{1.443416in}}%
\pgfpathcurveto{\pgfqpoint{0.808486in}{1.437592in}}{\pgfqpoint{0.816386in}{1.434320in}}{\pgfqpoint{0.824622in}{1.434320in}}%
\pgfpathclose%
\pgfusepath{stroke,fill}%
\end{pgfscope}%
\begin{pgfscope}%
\pgfpathrectangle{\pgfqpoint{0.100000in}{0.212622in}}{\pgfqpoint{3.696000in}{3.696000in}}%
\pgfusepath{clip}%
\pgfsetbuttcap%
\pgfsetroundjoin%
\definecolor{currentfill}{rgb}{0.121569,0.466667,0.705882}%
\pgfsetfillcolor{currentfill}%
\pgfsetfillopacity{0.600071}%
\pgfsetlinewidth{1.003750pt}%
\definecolor{currentstroke}{rgb}{0.121569,0.466667,0.705882}%
\pgfsetstrokecolor{currentstroke}%
\pgfsetstrokeopacity{0.600071}%
\pgfsetdash{}{0pt}%
\pgfpathmoveto{\pgfqpoint{0.820401in}{1.434798in}}%
\pgfpathcurveto{\pgfqpoint{0.828637in}{1.434798in}}{\pgfqpoint{0.836537in}{1.438070in}}{\pgfqpoint{0.842361in}{1.443894in}}%
\pgfpathcurveto{\pgfqpoint{0.848185in}{1.449718in}}{\pgfqpoint{0.851457in}{1.457618in}}{\pgfqpoint{0.851457in}{1.465854in}}%
\pgfpathcurveto{\pgfqpoint{0.851457in}{1.474090in}}{\pgfqpoint{0.848185in}{1.481991in}}{\pgfqpoint{0.842361in}{1.487814in}}%
\pgfpathcurveto{\pgfqpoint{0.836537in}{1.493638in}}{\pgfqpoint{0.828637in}{1.496911in}}{\pgfqpoint{0.820401in}{1.496911in}}%
\pgfpathcurveto{\pgfqpoint{0.812165in}{1.496911in}}{\pgfqpoint{0.804265in}{1.493638in}}{\pgfqpoint{0.798441in}{1.487814in}}%
\pgfpathcurveto{\pgfqpoint{0.792617in}{1.481991in}}{\pgfqpoint{0.789344in}{1.474090in}}{\pgfqpoint{0.789344in}{1.465854in}}%
\pgfpathcurveto{\pgfqpoint{0.789344in}{1.457618in}}{\pgfqpoint{0.792617in}{1.449718in}}{\pgfqpoint{0.798441in}{1.443894in}}%
\pgfpathcurveto{\pgfqpoint{0.804265in}{1.438070in}}{\pgfqpoint{0.812165in}{1.434798in}}{\pgfqpoint{0.820401in}{1.434798in}}%
\pgfpathclose%
\pgfusepath{stroke,fill}%
\end{pgfscope}%
\begin{pgfscope}%
\pgfpathrectangle{\pgfqpoint{0.100000in}{0.212622in}}{\pgfqpoint{3.696000in}{3.696000in}}%
\pgfusepath{clip}%
\pgfsetbuttcap%
\pgfsetroundjoin%
\definecolor{currentfill}{rgb}{0.121569,0.466667,0.705882}%
\pgfsetfillcolor{currentfill}%
\pgfsetfillopacity{0.600234}%
\pgfsetlinewidth{1.003750pt}%
\definecolor{currentstroke}{rgb}{0.121569,0.466667,0.705882}%
\pgfsetstrokecolor{currentstroke}%
\pgfsetstrokeopacity{0.600234}%
\pgfsetdash{}{0pt}%
\pgfpathmoveto{\pgfqpoint{1.597008in}{1.699074in}}%
\pgfpathcurveto{\pgfqpoint{1.605245in}{1.699074in}}{\pgfqpoint{1.613145in}{1.702346in}}{\pgfqpoint{1.618969in}{1.708170in}}%
\pgfpathcurveto{\pgfqpoint{1.624793in}{1.713994in}}{\pgfqpoint{1.628065in}{1.721894in}}{\pgfqpoint{1.628065in}{1.730130in}}%
\pgfpathcurveto{\pgfqpoint{1.628065in}{1.738367in}}{\pgfqpoint{1.624793in}{1.746267in}}{\pgfqpoint{1.618969in}{1.752091in}}%
\pgfpathcurveto{\pgfqpoint{1.613145in}{1.757914in}}{\pgfqpoint{1.605245in}{1.761187in}}{\pgfqpoint{1.597008in}{1.761187in}}%
\pgfpathcurveto{\pgfqpoint{1.588772in}{1.761187in}}{\pgfqpoint{1.580872in}{1.757914in}}{\pgfqpoint{1.575048in}{1.752091in}}%
\pgfpathcurveto{\pgfqpoint{1.569224in}{1.746267in}}{\pgfqpoint{1.565952in}{1.738367in}}{\pgfqpoint{1.565952in}{1.730130in}}%
\pgfpathcurveto{\pgfqpoint{1.565952in}{1.721894in}}{\pgfqpoint{1.569224in}{1.713994in}}{\pgfqpoint{1.575048in}{1.708170in}}%
\pgfpathcurveto{\pgfqpoint{1.580872in}{1.702346in}}{\pgfqpoint{1.588772in}{1.699074in}}{\pgfqpoint{1.597008in}{1.699074in}}%
\pgfpathclose%
\pgfusepath{stroke,fill}%
\end{pgfscope}%
\begin{pgfscope}%
\pgfpathrectangle{\pgfqpoint{0.100000in}{0.212622in}}{\pgfqpoint{3.696000in}{3.696000in}}%
\pgfusepath{clip}%
\pgfsetbuttcap%
\pgfsetroundjoin%
\definecolor{currentfill}{rgb}{0.121569,0.466667,0.705882}%
\pgfsetfillcolor{currentfill}%
\pgfsetfillopacity{0.602127}%
\pgfsetlinewidth{1.003750pt}%
\definecolor{currentstroke}{rgb}{0.121569,0.466667,0.705882}%
\pgfsetstrokecolor{currentstroke}%
\pgfsetstrokeopacity{0.602127}%
\pgfsetdash{}{0pt}%
\pgfpathmoveto{\pgfqpoint{0.818985in}{1.436196in}}%
\pgfpathcurveto{\pgfqpoint{0.827222in}{1.436196in}}{\pgfqpoint{0.835122in}{1.439468in}}{\pgfqpoint{0.840946in}{1.445292in}}%
\pgfpathcurveto{\pgfqpoint{0.846770in}{1.451116in}}{\pgfqpoint{0.850042in}{1.459016in}}{\pgfqpoint{0.850042in}{1.467253in}}%
\pgfpathcurveto{\pgfqpoint{0.850042in}{1.475489in}}{\pgfqpoint{0.846770in}{1.483389in}}{\pgfqpoint{0.840946in}{1.489213in}}%
\pgfpathcurveto{\pgfqpoint{0.835122in}{1.495037in}}{\pgfqpoint{0.827222in}{1.498309in}}{\pgfqpoint{0.818985in}{1.498309in}}%
\pgfpathcurveto{\pgfqpoint{0.810749in}{1.498309in}}{\pgfqpoint{0.802849in}{1.495037in}}{\pgfqpoint{0.797025in}{1.489213in}}%
\pgfpathcurveto{\pgfqpoint{0.791201in}{1.483389in}}{\pgfqpoint{0.787929in}{1.475489in}}{\pgfqpoint{0.787929in}{1.467253in}}%
\pgfpathcurveto{\pgfqpoint{0.787929in}{1.459016in}}{\pgfqpoint{0.791201in}{1.451116in}}{\pgfqpoint{0.797025in}{1.445292in}}%
\pgfpathcurveto{\pgfqpoint{0.802849in}{1.439468in}}{\pgfqpoint{0.810749in}{1.436196in}}{\pgfqpoint{0.818985in}{1.436196in}}%
\pgfpathclose%
\pgfusepath{stroke,fill}%
\end{pgfscope}%
\begin{pgfscope}%
\pgfpathrectangle{\pgfqpoint{0.100000in}{0.212622in}}{\pgfqpoint{3.696000in}{3.696000in}}%
\pgfusepath{clip}%
\pgfsetbuttcap%
\pgfsetroundjoin%
\definecolor{currentfill}{rgb}{0.121569,0.466667,0.705882}%
\pgfsetfillcolor{currentfill}%
\pgfsetfillopacity{0.602679}%
\pgfsetlinewidth{1.003750pt}%
\definecolor{currentstroke}{rgb}{0.121569,0.466667,0.705882}%
\pgfsetstrokecolor{currentstroke}%
\pgfsetstrokeopacity{0.602679}%
\pgfsetdash{}{0pt}%
\pgfpathmoveto{\pgfqpoint{0.817038in}{1.433843in}}%
\pgfpathcurveto{\pgfqpoint{0.825274in}{1.433843in}}{\pgfqpoint{0.833174in}{1.437115in}}{\pgfqpoint{0.838998in}{1.442939in}}%
\pgfpathcurveto{\pgfqpoint{0.844822in}{1.448763in}}{\pgfqpoint{0.848094in}{1.456663in}}{\pgfqpoint{0.848094in}{1.464899in}}%
\pgfpathcurveto{\pgfqpoint{0.848094in}{1.473136in}}{\pgfqpoint{0.844822in}{1.481036in}}{\pgfqpoint{0.838998in}{1.486860in}}%
\pgfpathcurveto{\pgfqpoint{0.833174in}{1.492684in}}{\pgfqpoint{0.825274in}{1.495956in}}{\pgfqpoint{0.817038in}{1.495956in}}%
\pgfpathcurveto{\pgfqpoint{0.808801in}{1.495956in}}{\pgfqpoint{0.800901in}{1.492684in}}{\pgfqpoint{0.795077in}{1.486860in}}%
\pgfpathcurveto{\pgfqpoint{0.789253in}{1.481036in}}{\pgfqpoint{0.785981in}{1.473136in}}{\pgfqpoint{0.785981in}{1.464899in}}%
\pgfpathcurveto{\pgfqpoint{0.785981in}{1.456663in}}{\pgfqpoint{0.789253in}{1.448763in}}{\pgfqpoint{0.795077in}{1.442939in}}%
\pgfpathcurveto{\pgfqpoint{0.800901in}{1.437115in}}{\pgfqpoint{0.808801in}{1.433843in}}{\pgfqpoint{0.817038in}{1.433843in}}%
\pgfpathclose%
\pgfusepath{stroke,fill}%
\end{pgfscope}%
\begin{pgfscope}%
\pgfpathrectangle{\pgfqpoint{0.100000in}{0.212622in}}{\pgfqpoint{3.696000in}{3.696000in}}%
\pgfusepath{clip}%
\pgfsetbuttcap%
\pgfsetroundjoin%
\definecolor{currentfill}{rgb}{0.121569,0.466667,0.705882}%
\pgfsetfillcolor{currentfill}%
\pgfsetfillopacity{0.603612}%
\pgfsetlinewidth{1.003750pt}%
\definecolor{currentstroke}{rgb}{0.121569,0.466667,0.705882}%
\pgfsetstrokecolor{currentstroke}%
\pgfsetstrokeopacity{0.603612}%
\pgfsetdash{}{0pt}%
\pgfpathmoveto{\pgfqpoint{1.598416in}{1.697234in}}%
\pgfpathcurveto{\pgfqpoint{1.606652in}{1.697234in}}{\pgfqpoint{1.614552in}{1.700506in}}{\pgfqpoint{1.620376in}{1.706330in}}%
\pgfpathcurveto{\pgfqpoint{1.626200in}{1.712154in}}{\pgfqpoint{1.629472in}{1.720054in}}{\pgfqpoint{1.629472in}{1.728291in}}%
\pgfpathcurveto{\pgfqpoint{1.629472in}{1.736527in}}{\pgfqpoint{1.626200in}{1.744427in}}{\pgfqpoint{1.620376in}{1.750251in}}%
\pgfpathcurveto{\pgfqpoint{1.614552in}{1.756075in}}{\pgfqpoint{1.606652in}{1.759347in}}{\pgfqpoint{1.598416in}{1.759347in}}%
\pgfpathcurveto{\pgfqpoint{1.590180in}{1.759347in}}{\pgfqpoint{1.582280in}{1.756075in}}{\pgfqpoint{1.576456in}{1.750251in}}%
\pgfpathcurveto{\pgfqpoint{1.570632in}{1.744427in}}{\pgfqpoint{1.567359in}{1.736527in}}{\pgfqpoint{1.567359in}{1.728291in}}%
\pgfpathcurveto{\pgfqpoint{1.567359in}{1.720054in}}{\pgfqpoint{1.570632in}{1.712154in}}{\pgfqpoint{1.576456in}{1.706330in}}%
\pgfpathcurveto{\pgfqpoint{1.582280in}{1.700506in}}{\pgfqpoint{1.590180in}{1.697234in}}{\pgfqpoint{1.598416in}{1.697234in}}%
\pgfpathclose%
\pgfusepath{stroke,fill}%
\end{pgfscope}%
\begin{pgfscope}%
\pgfpathrectangle{\pgfqpoint{0.100000in}{0.212622in}}{\pgfqpoint{3.696000in}{3.696000in}}%
\pgfusepath{clip}%
\pgfsetbuttcap%
\pgfsetroundjoin%
\definecolor{currentfill}{rgb}{0.121569,0.466667,0.705882}%
\pgfsetfillcolor{currentfill}%
\pgfsetfillopacity{0.604526}%
\pgfsetlinewidth{1.003750pt}%
\definecolor{currentstroke}{rgb}{0.121569,0.466667,0.705882}%
\pgfsetstrokecolor{currentstroke}%
\pgfsetstrokeopacity{0.604526}%
\pgfsetdash{}{0pt}%
\pgfpathmoveto{\pgfqpoint{0.813931in}{1.432630in}}%
\pgfpathcurveto{\pgfqpoint{0.822167in}{1.432630in}}{\pgfqpoint{0.830067in}{1.435902in}}{\pgfqpoint{0.835891in}{1.441726in}}%
\pgfpathcurveto{\pgfqpoint{0.841715in}{1.447550in}}{\pgfqpoint{0.844988in}{1.455450in}}{\pgfqpoint{0.844988in}{1.463686in}}%
\pgfpathcurveto{\pgfqpoint{0.844988in}{1.471923in}}{\pgfqpoint{0.841715in}{1.479823in}}{\pgfqpoint{0.835891in}{1.485647in}}%
\pgfpathcurveto{\pgfqpoint{0.830067in}{1.491471in}}{\pgfqpoint{0.822167in}{1.494743in}}{\pgfqpoint{0.813931in}{1.494743in}}%
\pgfpathcurveto{\pgfqpoint{0.805695in}{1.494743in}}{\pgfqpoint{0.797795in}{1.491471in}}{\pgfqpoint{0.791971in}{1.485647in}}%
\pgfpathcurveto{\pgfqpoint{0.786147in}{1.479823in}}{\pgfqpoint{0.782875in}{1.471923in}}{\pgfqpoint{0.782875in}{1.463686in}}%
\pgfpathcurveto{\pgfqpoint{0.782875in}{1.455450in}}{\pgfqpoint{0.786147in}{1.447550in}}{\pgfqpoint{0.791971in}{1.441726in}}%
\pgfpathcurveto{\pgfqpoint{0.797795in}{1.435902in}}{\pgfqpoint{0.805695in}{1.432630in}}{\pgfqpoint{0.813931in}{1.432630in}}%
\pgfpathclose%
\pgfusepath{stroke,fill}%
\end{pgfscope}%
\begin{pgfscope}%
\pgfpathrectangle{\pgfqpoint{0.100000in}{0.212622in}}{\pgfqpoint{3.696000in}{3.696000in}}%
\pgfusepath{clip}%
\pgfsetbuttcap%
\pgfsetroundjoin%
\definecolor{currentfill}{rgb}{0.121569,0.466667,0.705882}%
\pgfsetfillcolor{currentfill}%
\pgfsetfillopacity{0.606925}%
\pgfsetlinewidth{1.003750pt}%
\definecolor{currentstroke}{rgb}{0.121569,0.466667,0.705882}%
\pgfsetstrokecolor{currentstroke}%
\pgfsetstrokeopacity{0.606925}%
\pgfsetdash{}{0pt}%
\pgfpathmoveto{\pgfqpoint{0.809277in}{1.425163in}}%
\pgfpathcurveto{\pgfqpoint{0.817513in}{1.425163in}}{\pgfqpoint{0.825413in}{1.428436in}}{\pgfqpoint{0.831237in}{1.434260in}}%
\pgfpathcurveto{\pgfqpoint{0.837061in}{1.440084in}}{\pgfqpoint{0.840334in}{1.447984in}}{\pgfqpoint{0.840334in}{1.456220in}}%
\pgfpathcurveto{\pgfqpoint{0.840334in}{1.464456in}}{\pgfqpoint{0.837061in}{1.472356in}}{\pgfqpoint{0.831237in}{1.478180in}}%
\pgfpathcurveto{\pgfqpoint{0.825413in}{1.484004in}}{\pgfqpoint{0.817513in}{1.487276in}}{\pgfqpoint{0.809277in}{1.487276in}}%
\pgfpathcurveto{\pgfqpoint{0.801041in}{1.487276in}}{\pgfqpoint{0.793141in}{1.484004in}}{\pgfqpoint{0.787317in}{1.478180in}}%
\pgfpathcurveto{\pgfqpoint{0.781493in}{1.472356in}}{\pgfqpoint{0.778221in}{1.464456in}}{\pgfqpoint{0.778221in}{1.456220in}}%
\pgfpathcurveto{\pgfqpoint{0.778221in}{1.447984in}}{\pgfqpoint{0.781493in}{1.440084in}}{\pgfqpoint{0.787317in}{1.434260in}}%
\pgfpathcurveto{\pgfqpoint{0.793141in}{1.428436in}}{\pgfqpoint{0.801041in}{1.425163in}}{\pgfqpoint{0.809277in}{1.425163in}}%
\pgfpathclose%
\pgfusepath{stroke,fill}%
\end{pgfscope}%
\begin{pgfscope}%
\pgfpathrectangle{\pgfqpoint{0.100000in}{0.212622in}}{\pgfqpoint{3.696000in}{3.696000in}}%
\pgfusepath{clip}%
\pgfsetbuttcap%
\pgfsetroundjoin%
\definecolor{currentfill}{rgb}{0.121569,0.466667,0.705882}%
\pgfsetfillcolor{currentfill}%
\pgfsetfillopacity{0.607122}%
\pgfsetlinewidth{1.003750pt}%
\definecolor{currentstroke}{rgb}{0.121569,0.466667,0.705882}%
\pgfsetstrokecolor{currentstroke}%
\pgfsetstrokeopacity{0.607122}%
\pgfsetdash{}{0pt}%
\pgfpathmoveto{\pgfqpoint{1.600175in}{1.695047in}}%
\pgfpathcurveto{\pgfqpoint{1.608411in}{1.695047in}}{\pgfqpoint{1.616311in}{1.698320in}}{\pgfqpoint{1.622135in}{1.704144in}}%
\pgfpathcurveto{\pgfqpoint{1.627959in}{1.709968in}}{\pgfqpoint{1.631231in}{1.717868in}}{\pgfqpoint{1.631231in}{1.726104in}}%
\pgfpathcurveto{\pgfqpoint{1.631231in}{1.734340in}}{\pgfqpoint{1.627959in}{1.742240in}}{\pgfqpoint{1.622135in}{1.748064in}}%
\pgfpathcurveto{\pgfqpoint{1.616311in}{1.753888in}}{\pgfqpoint{1.608411in}{1.757160in}}{\pgfqpoint{1.600175in}{1.757160in}}%
\pgfpathcurveto{\pgfqpoint{1.591939in}{1.757160in}}{\pgfqpoint{1.584039in}{1.753888in}}{\pgfqpoint{1.578215in}{1.748064in}}%
\pgfpathcurveto{\pgfqpoint{1.572391in}{1.742240in}}{\pgfqpoint{1.569118in}{1.734340in}}{\pgfqpoint{1.569118in}{1.726104in}}%
\pgfpathcurveto{\pgfqpoint{1.569118in}{1.717868in}}{\pgfqpoint{1.572391in}{1.709968in}}{\pgfqpoint{1.578215in}{1.704144in}}%
\pgfpathcurveto{\pgfqpoint{1.584039in}{1.698320in}}{\pgfqpoint{1.591939in}{1.695047in}}{\pgfqpoint{1.600175in}{1.695047in}}%
\pgfpathclose%
\pgfusepath{stroke,fill}%
\end{pgfscope}%
\begin{pgfscope}%
\pgfpathrectangle{\pgfqpoint{0.100000in}{0.212622in}}{\pgfqpoint{3.696000in}{3.696000in}}%
\pgfusepath{clip}%
\pgfsetbuttcap%
\pgfsetroundjoin%
\definecolor{currentfill}{rgb}{0.121569,0.466667,0.705882}%
\pgfsetfillcolor{currentfill}%
\pgfsetfillopacity{0.609870}%
\pgfsetlinewidth{1.003750pt}%
\definecolor{currentstroke}{rgb}{0.121569,0.466667,0.705882}%
\pgfsetstrokecolor{currentstroke}%
\pgfsetstrokeopacity{0.609870}%
\pgfsetdash{}{0pt}%
\pgfpathmoveto{\pgfqpoint{0.802763in}{1.425509in}}%
\pgfpathcurveto{\pgfqpoint{0.810999in}{1.425509in}}{\pgfqpoint{0.818899in}{1.428781in}}{\pgfqpoint{0.824723in}{1.434605in}}%
\pgfpathcurveto{\pgfqpoint{0.830547in}{1.440429in}}{\pgfqpoint{0.833819in}{1.448329in}}{\pgfqpoint{0.833819in}{1.456565in}}%
\pgfpathcurveto{\pgfqpoint{0.833819in}{1.464801in}}{\pgfqpoint{0.830547in}{1.472702in}}{\pgfqpoint{0.824723in}{1.478525in}}%
\pgfpathcurveto{\pgfqpoint{0.818899in}{1.484349in}}{\pgfqpoint{0.810999in}{1.487622in}}{\pgfqpoint{0.802763in}{1.487622in}}%
\pgfpathcurveto{\pgfqpoint{0.794526in}{1.487622in}}{\pgfqpoint{0.786626in}{1.484349in}}{\pgfqpoint{0.780802in}{1.478525in}}%
\pgfpathcurveto{\pgfqpoint{0.774978in}{1.472702in}}{\pgfqpoint{0.771706in}{1.464801in}}{\pgfqpoint{0.771706in}{1.456565in}}%
\pgfpathcurveto{\pgfqpoint{0.771706in}{1.448329in}}{\pgfqpoint{0.774978in}{1.440429in}}{\pgfqpoint{0.780802in}{1.434605in}}%
\pgfpathcurveto{\pgfqpoint{0.786626in}{1.428781in}}{\pgfqpoint{0.794526in}{1.425509in}}{\pgfqpoint{0.802763in}{1.425509in}}%
\pgfpathclose%
\pgfusepath{stroke,fill}%
\end{pgfscope}%
\begin{pgfscope}%
\pgfpathrectangle{\pgfqpoint{0.100000in}{0.212622in}}{\pgfqpoint{3.696000in}{3.696000in}}%
\pgfusepath{clip}%
\pgfsetbuttcap%
\pgfsetroundjoin%
\definecolor{currentfill}{rgb}{0.121569,0.466667,0.705882}%
\pgfsetfillcolor{currentfill}%
\pgfsetfillopacity{0.610816}%
\pgfsetlinewidth{1.003750pt}%
\definecolor{currentstroke}{rgb}{0.121569,0.466667,0.705882}%
\pgfsetstrokecolor{currentstroke}%
\pgfsetstrokeopacity{0.610816}%
\pgfsetdash{}{0pt}%
\pgfpathmoveto{\pgfqpoint{1.602032in}{1.691518in}}%
\pgfpathcurveto{\pgfqpoint{1.610268in}{1.691518in}}{\pgfqpoint{1.618168in}{1.694790in}}{\pgfqpoint{1.623992in}{1.700614in}}%
\pgfpathcurveto{\pgfqpoint{1.629816in}{1.706438in}}{\pgfqpoint{1.633088in}{1.714338in}}{\pgfqpoint{1.633088in}{1.722574in}}%
\pgfpathcurveto{\pgfqpoint{1.633088in}{1.730811in}}{\pgfqpoint{1.629816in}{1.738711in}}{\pgfqpoint{1.623992in}{1.744535in}}%
\pgfpathcurveto{\pgfqpoint{1.618168in}{1.750358in}}{\pgfqpoint{1.610268in}{1.753631in}}{\pgfqpoint{1.602032in}{1.753631in}}%
\pgfpathcurveto{\pgfqpoint{1.593795in}{1.753631in}}{\pgfqpoint{1.585895in}{1.750358in}}{\pgfqpoint{1.580071in}{1.744535in}}%
\pgfpathcurveto{\pgfqpoint{1.574248in}{1.738711in}}{\pgfqpoint{1.570975in}{1.730811in}}{\pgfqpoint{1.570975in}{1.722574in}}%
\pgfpathcurveto{\pgfqpoint{1.570975in}{1.714338in}}{\pgfqpoint{1.574248in}{1.706438in}}{\pgfqpoint{1.580071in}{1.700614in}}%
\pgfpathcurveto{\pgfqpoint{1.585895in}{1.694790in}}{\pgfqpoint{1.593795in}{1.691518in}}{\pgfqpoint{1.602032in}{1.691518in}}%
\pgfpathclose%
\pgfusepath{stroke,fill}%
\end{pgfscope}%
\begin{pgfscope}%
\pgfpathrectangle{\pgfqpoint{0.100000in}{0.212622in}}{\pgfqpoint{3.696000in}{3.696000in}}%
\pgfusepath{clip}%
\pgfsetbuttcap%
\pgfsetroundjoin%
\definecolor{currentfill}{rgb}{0.121569,0.466667,0.705882}%
\pgfsetfillcolor{currentfill}%
\pgfsetfillopacity{0.613289}%
\pgfsetlinewidth{1.003750pt}%
\definecolor{currentstroke}{rgb}{0.121569,0.466667,0.705882}%
\pgfsetstrokecolor{currentstroke}%
\pgfsetstrokeopacity{0.613289}%
\pgfsetdash{}{0pt}%
\pgfpathmoveto{\pgfqpoint{0.799346in}{1.425410in}}%
\pgfpathcurveto{\pgfqpoint{0.807582in}{1.425410in}}{\pgfqpoint{0.815482in}{1.428683in}}{\pgfqpoint{0.821306in}{1.434507in}}%
\pgfpathcurveto{\pgfqpoint{0.827130in}{1.440331in}}{\pgfqpoint{0.830402in}{1.448231in}}{\pgfqpoint{0.830402in}{1.456467in}}%
\pgfpathcurveto{\pgfqpoint{0.830402in}{1.464703in}}{\pgfqpoint{0.827130in}{1.472603in}}{\pgfqpoint{0.821306in}{1.478427in}}%
\pgfpathcurveto{\pgfqpoint{0.815482in}{1.484251in}}{\pgfqpoint{0.807582in}{1.487523in}}{\pgfqpoint{0.799346in}{1.487523in}}%
\pgfpathcurveto{\pgfqpoint{0.791109in}{1.487523in}}{\pgfqpoint{0.783209in}{1.484251in}}{\pgfqpoint{0.777385in}{1.478427in}}%
\pgfpathcurveto{\pgfqpoint{0.771562in}{1.472603in}}{\pgfqpoint{0.768289in}{1.464703in}}{\pgfqpoint{0.768289in}{1.456467in}}%
\pgfpathcurveto{\pgfqpoint{0.768289in}{1.448231in}}{\pgfqpoint{0.771562in}{1.440331in}}{\pgfqpoint{0.777385in}{1.434507in}}%
\pgfpathcurveto{\pgfqpoint{0.783209in}{1.428683in}}{\pgfqpoint{0.791109in}{1.425410in}}{\pgfqpoint{0.799346in}{1.425410in}}%
\pgfpathclose%
\pgfusepath{stroke,fill}%
\end{pgfscope}%
\begin{pgfscope}%
\pgfpathrectangle{\pgfqpoint{0.100000in}{0.212622in}}{\pgfqpoint{3.696000in}{3.696000in}}%
\pgfusepath{clip}%
\pgfsetbuttcap%
\pgfsetroundjoin%
\definecolor{currentfill}{rgb}{0.121569,0.466667,0.705882}%
\pgfsetfillcolor{currentfill}%
\pgfsetfillopacity{0.615194}%
\pgfsetlinewidth{1.003750pt}%
\definecolor{currentstroke}{rgb}{0.121569,0.466667,0.705882}%
\pgfsetstrokecolor{currentstroke}%
\pgfsetstrokeopacity{0.615194}%
\pgfsetdash{}{0pt}%
\pgfpathmoveto{\pgfqpoint{0.795462in}{1.425311in}}%
\pgfpathcurveto{\pgfqpoint{0.803698in}{1.425311in}}{\pgfqpoint{0.811598in}{1.428584in}}{\pgfqpoint{0.817422in}{1.434408in}}%
\pgfpathcurveto{\pgfqpoint{0.823246in}{1.440232in}}{\pgfqpoint{0.826518in}{1.448132in}}{\pgfqpoint{0.826518in}{1.456368in}}%
\pgfpathcurveto{\pgfqpoint{0.826518in}{1.464604in}}{\pgfqpoint{0.823246in}{1.472504in}}{\pgfqpoint{0.817422in}{1.478328in}}%
\pgfpathcurveto{\pgfqpoint{0.811598in}{1.484152in}}{\pgfqpoint{0.803698in}{1.487424in}}{\pgfqpoint{0.795462in}{1.487424in}}%
\pgfpathcurveto{\pgfqpoint{0.787226in}{1.487424in}}{\pgfqpoint{0.779326in}{1.484152in}}{\pgfqpoint{0.773502in}{1.478328in}}%
\pgfpathcurveto{\pgfqpoint{0.767678in}{1.472504in}}{\pgfqpoint{0.764405in}{1.464604in}}{\pgfqpoint{0.764405in}{1.456368in}}%
\pgfpathcurveto{\pgfqpoint{0.764405in}{1.448132in}}{\pgfqpoint{0.767678in}{1.440232in}}{\pgfqpoint{0.773502in}{1.434408in}}%
\pgfpathcurveto{\pgfqpoint{0.779326in}{1.428584in}}{\pgfqpoint{0.787226in}{1.425311in}}{\pgfqpoint{0.795462in}{1.425311in}}%
\pgfpathclose%
\pgfusepath{stroke,fill}%
\end{pgfscope}%
\begin{pgfscope}%
\pgfpathrectangle{\pgfqpoint{0.100000in}{0.212622in}}{\pgfqpoint{3.696000in}{3.696000in}}%
\pgfusepath{clip}%
\pgfsetbuttcap%
\pgfsetroundjoin%
\definecolor{currentfill}{rgb}{0.121569,0.466667,0.705882}%
\pgfsetfillcolor{currentfill}%
\pgfsetfillopacity{0.616078}%
\pgfsetlinewidth{1.003750pt}%
\definecolor{currentstroke}{rgb}{0.121569,0.466667,0.705882}%
\pgfsetstrokecolor{currentstroke}%
\pgfsetstrokeopacity{0.616078}%
\pgfsetdash{}{0pt}%
\pgfpathmoveto{\pgfqpoint{1.605844in}{1.693777in}}%
\pgfpathcurveto{\pgfqpoint{1.614081in}{1.693777in}}{\pgfqpoint{1.621981in}{1.697049in}}{\pgfqpoint{1.627805in}{1.702873in}}%
\pgfpathcurveto{\pgfqpoint{1.633629in}{1.708697in}}{\pgfqpoint{1.636901in}{1.716597in}}{\pgfqpoint{1.636901in}{1.724834in}}%
\pgfpathcurveto{\pgfqpoint{1.636901in}{1.733070in}}{\pgfqpoint{1.633629in}{1.740970in}}{\pgfqpoint{1.627805in}{1.746794in}}%
\pgfpathcurveto{\pgfqpoint{1.621981in}{1.752618in}}{\pgfqpoint{1.614081in}{1.755890in}}{\pgfqpoint{1.605844in}{1.755890in}}%
\pgfpathcurveto{\pgfqpoint{1.597608in}{1.755890in}}{\pgfqpoint{1.589708in}{1.752618in}}{\pgfqpoint{1.583884in}{1.746794in}}%
\pgfpathcurveto{\pgfqpoint{1.578060in}{1.740970in}}{\pgfqpoint{1.574788in}{1.733070in}}{\pgfqpoint{1.574788in}{1.724834in}}%
\pgfpathcurveto{\pgfqpoint{1.574788in}{1.716597in}}{\pgfqpoint{1.578060in}{1.708697in}}{\pgfqpoint{1.583884in}{1.702873in}}%
\pgfpathcurveto{\pgfqpoint{1.589708in}{1.697049in}}{\pgfqpoint{1.597608in}{1.693777in}}{\pgfqpoint{1.605844in}{1.693777in}}%
\pgfpathclose%
\pgfusepath{stroke,fill}%
\end{pgfscope}%
\begin{pgfscope}%
\pgfpathrectangle{\pgfqpoint{0.100000in}{0.212622in}}{\pgfqpoint{3.696000in}{3.696000in}}%
\pgfusepath{clip}%
\pgfsetbuttcap%
\pgfsetroundjoin%
\definecolor{currentfill}{rgb}{0.121569,0.466667,0.705882}%
\pgfsetfillcolor{currentfill}%
\pgfsetfillopacity{0.617918}%
\pgfsetlinewidth{1.003750pt}%
\definecolor{currentstroke}{rgb}{0.121569,0.466667,0.705882}%
\pgfsetstrokecolor{currentstroke}%
\pgfsetstrokeopacity{0.617918}%
\pgfsetdash{}{0pt}%
\pgfpathmoveto{\pgfqpoint{0.793698in}{1.427644in}}%
\pgfpathcurveto{\pgfqpoint{0.801935in}{1.427644in}}{\pgfqpoint{0.809835in}{1.430917in}}{\pgfqpoint{0.815659in}{1.436741in}}%
\pgfpathcurveto{\pgfqpoint{0.821482in}{1.442565in}}{\pgfqpoint{0.824755in}{1.450465in}}{\pgfqpoint{0.824755in}{1.458701in}}%
\pgfpathcurveto{\pgfqpoint{0.824755in}{1.466937in}}{\pgfqpoint{0.821482in}{1.474837in}}{\pgfqpoint{0.815659in}{1.480661in}}%
\pgfpathcurveto{\pgfqpoint{0.809835in}{1.486485in}}{\pgfqpoint{0.801935in}{1.489757in}}{\pgfqpoint{0.793698in}{1.489757in}}%
\pgfpathcurveto{\pgfqpoint{0.785462in}{1.489757in}}{\pgfqpoint{0.777562in}{1.486485in}}{\pgfqpoint{0.771738in}{1.480661in}}%
\pgfpathcurveto{\pgfqpoint{0.765914in}{1.474837in}}{\pgfqpoint{0.762642in}{1.466937in}}{\pgfqpoint{0.762642in}{1.458701in}}%
\pgfpathcurveto{\pgfqpoint{0.762642in}{1.450465in}}{\pgfqpoint{0.765914in}{1.442565in}}{\pgfqpoint{0.771738in}{1.436741in}}%
\pgfpathcurveto{\pgfqpoint{0.777562in}{1.430917in}}{\pgfqpoint{0.785462in}{1.427644in}}{\pgfqpoint{0.793698in}{1.427644in}}%
\pgfpathclose%
\pgfusepath{stroke,fill}%
\end{pgfscope}%
\begin{pgfscope}%
\pgfpathrectangle{\pgfqpoint{0.100000in}{0.212622in}}{\pgfqpoint{3.696000in}{3.696000in}}%
\pgfusepath{clip}%
\pgfsetbuttcap%
\pgfsetroundjoin%
\definecolor{currentfill}{rgb}{0.121569,0.466667,0.705882}%
\pgfsetfillcolor{currentfill}%
\pgfsetfillopacity{0.618449}%
\pgfsetlinewidth{1.003750pt}%
\definecolor{currentstroke}{rgb}{0.121569,0.466667,0.705882}%
\pgfsetstrokecolor{currentstroke}%
\pgfsetstrokeopacity{0.618449}%
\pgfsetdash{}{0pt}%
\pgfpathmoveto{\pgfqpoint{0.791927in}{1.425814in}}%
\pgfpathcurveto{\pgfqpoint{0.800163in}{1.425814in}}{\pgfqpoint{0.808063in}{1.429087in}}{\pgfqpoint{0.813887in}{1.434911in}}%
\pgfpathcurveto{\pgfqpoint{0.819711in}{1.440735in}}{\pgfqpoint{0.822983in}{1.448635in}}{\pgfqpoint{0.822983in}{1.456871in}}%
\pgfpathcurveto{\pgfqpoint{0.822983in}{1.465107in}}{\pgfqpoint{0.819711in}{1.473007in}}{\pgfqpoint{0.813887in}{1.478831in}}%
\pgfpathcurveto{\pgfqpoint{0.808063in}{1.484655in}}{\pgfqpoint{0.800163in}{1.487927in}}{\pgfqpoint{0.791927in}{1.487927in}}%
\pgfpathcurveto{\pgfqpoint{0.783690in}{1.487927in}}{\pgfqpoint{0.775790in}{1.484655in}}{\pgfqpoint{0.769967in}{1.478831in}}%
\pgfpathcurveto{\pgfqpoint{0.764143in}{1.473007in}}{\pgfqpoint{0.760870in}{1.465107in}}{\pgfqpoint{0.760870in}{1.456871in}}%
\pgfpathcurveto{\pgfqpoint{0.760870in}{1.448635in}}{\pgfqpoint{0.764143in}{1.440735in}}{\pgfqpoint{0.769967in}{1.434911in}}%
\pgfpathcurveto{\pgfqpoint{0.775790in}{1.429087in}}{\pgfqpoint{0.783690in}{1.425814in}}{\pgfqpoint{0.791927in}{1.425814in}}%
\pgfpathclose%
\pgfusepath{stroke,fill}%
\end{pgfscope}%
\begin{pgfscope}%
\pgfpathrectangle{\pgfqpoint{0.100000in}{0.212622in}}{\pgfqpoint{3.696000in}{3.696000in}}%
\pgfusepath{clip}%
\pgfsetbuttcap%
\pgfsetroundjoin%
\definecolor{currentfill}{rgb}{0.121569,0.466667,0.705882}%
\pgfsetfillcolor{currentfill}%
\pgfsetfillopacity{0.618692}%
\pgfsetlinewidth{1.003750pt}%
\definecolor{currentstroke}{rgb}{0.121569,0.466667,0.705882}%
\pgfsetstrokecolor{currentstroke}%
\pgfsetstrokeopacity{0.618692}%
\pgfsetdash{}{0pt}%
\pgfpathmoveto{\pgfqpoint{1.606169in}{1.692923in}}%
\pgfpathcurveto{\pgfqpoint{1.614405in}{1.692923in}}{\pgfqpoint{1.622305in}{1.696195in}}{\pgfqpoint{1.628129in}{1.702019in}}%
\pgfpathcurveto{\pgfqpoint{1.633953in}{1.707843in}}{\pgfqpoint{1.637225in}{1.715743in}}{\pgfqpoint{1.637225in}{1.723980in}}%
\pgfpathcurveto{\pgfqpoint{1.637225in}{1.732216in}}{\pgfqpoint{1.633953in}{1.740116in}}{\pgfqpoint{1.628129in}{1.745940in}}%
\pgfpathcurveto{\pgfqpoint{1.622305in}{1.751764in}}{\pgfqpoint{1.614405in}{1.755036in}}{\pgfqpoint{1.606169in}{1.755036in}}%
\pgfpathcurveto{\pgfqpoint{1.597932in}{1.755036in}}{\pgfqpoint{1.590032in}{1.751764in}}{\pgfqpoint{1.584208in}{1.745940in}}%
\pgfpathcurveto{\pgfqpoint{1.578384in}{1.740116in}}{\pgfqpoint{1.575112in}{1.732216in}}{\pgfqpoint{1.575112in}{1.723980in}}%
\pgfpathcurveto{\pgfqpoint{1.575112in}{1.715743in}}{\pgfqpoint{1.578384in}{1.707843in}}{\pgfqpoint{1.584208in}{1.702019in}}%
\pgfpathcurveto{\pgfqpoint{1.590032in}{1.696195in}}{\pgfqpoint{1.597932in}{1.692923in}}{\pgfqpoint{1.606169in}{1.692923in}}%
\pgfpathclose%
\pgfusepath{stroke,fill}%
\end{pgfscope}%
\begin{pgfscope}%
\pgfpathrectangle{\pgfqpoint{0.100000in}{0.212622in}}{\pgfqpoint{3.696000in}{3.696000in}}%
\pgfusepath{clip}%
\pgfsetbuttcap%
\pgfsetroundjoin%
\definecolor{currentfill}{rgb}{0.121569,0.466667,0.705882}%
\pgfsetfillcolor{currentfill}%
\pgfsetfillopacity{0.620023}%
\pgfsetlinewidth{1.003750pt}%
\definecolor{currentstroke}{rgb}{0.121569,0.466667,0.705882}%
\pgfsetstrokecolor{currentstroke}%
\pgfsetstrokeopacity{0.620023}%
\pgfsetdash{}{0pt}%
\pgfpathmoveto{\pgfqpoint{0.789252in}{1.424377in}}%
\pgfpathcurveto{\pgfqpoint{0.797488in}{1.424377in}}{\pgfqpoint{0.805388in}{1.427649in}}{\pgfqpoint{0.811212in}{1.433473in}}%
\pgfpathcurveto{\pgfqpoint{0.817036in}{1.439297in}}{\pgfqpoint{0.820308in}{1.447197in}}{\pgfqpoint{0.820308in}{1.455433in}}%
\pgfpathcurveto{\pgfqpoint{0.820308in}{1.463669in}}{\pgfqpoint{0.817036in}{1.471570in}}{\pgfqpoint{0.811212in}{1.477393in}}%
\pgfpathcurveto{\pgfqpoint{0.805388in}{1.483217in}}{\pgfqpoint{0.797488in}{1.486490in}}{\pgfqpoint{0.789252in}{1.486490in}}%
\pgfpathcurveto{\pgfqpoint{0.781015in}{1.486490in}}{\pgfqpoint{0.773115in}{1.483217in}}{\pgfqpoint{0.767291in}{1.477393in}}%
\pgfpathcurveto{\pgfqpoint{0.761467in}{1.471570in}}{\pgfqpoint{0.758195in}{1.463669in}}{\pgfqpoint{0.758195in}{1.455433in}}%
\pgfpathcurveto{\pgfqpoint{0.758195in}{1.447197in}}{\pgfqpoint{0.761467in}{1.439297in}}{\pgfqpoint{0.767291in}{1.433473in}}%
\pgfpathcurveto{\pgfqpoint{0.773115in}{1.427649in}}{\pgfqpoint{0.781015in}{1.424377in}}{\pgfqpoint{0.789252in}{1.424377in}}%
\pgfpathclose%
\pgfusepath{stroke,fill}%
\end{pgfscope}%
\begin{pgfscope}%
\pgfpathrectangle{\pgfqpoint{0.100000in}{0.212622in}}{\pgfqpoint{3.696000in}{3.696000in}}%
\pgfusepath{clip}%
\pgfsetbuttcap%
\pgfsetroundjoin%
\definecolor{currentfill}{rgb}{0.121569,0.466667,0.705882}%
\pgfsetfillcolor{currentfill}%
\pgfsetfillopacity{0.621365}%
\pgfsetlinewidth{1.003750pt}%
\definecolor{currentstroke}{rgb}{0.121569,0.466667,0.705882}%
\pgfsetstrokecolor{currentstroke}%
\pgfsetstrokeopacity{0.621365}%
\pgfsetdash{}{0pt}%
\pgfpathmoveto{\pgfqpoint{1.607679in}{1.691222in}}%
\pgfpathcurveto{\pgfqpoint{1.615915in}{1.691222in}}{\pgfqpoint{1.623815in}{1.694494in}}{\pgfqpoint{1.629639in}{1.700318in}}%
\pgfpathcurveto{\pgfqpoint{1.635463in}{1.706142in}}{\pgfqpoint{1.638735in}{1.714042in}}{\pgfqpoint{1.638735in}{1.722279in}}%
\pgfpathcurveto{\pgfqpoint{1.638735in}{1.730515in}}{\pgfqpoint{1.635463in}{1.738415in}}{\pgfqpoint{1.629639in}{1.744239in}}%
\pgfpathcurveto{\pgfqpoint{1.623815in}{1.750063in}}{\pgfqpoint{1.615915in}{1.753335in}}{\pgfqpoint{1.607679in}{1.753335in}}%
\pgfpathcurveto{\pgfqpoint{1.599443in}{1.753335in}}{\pgfqpoint{1.591542in}{1.750063in}}{\pgfqpoint{1.585719in}{1.744239in}}%
\pgfpathcurveto{\pgfqpoint{1.579895in}{1.738415in}}{\pgfqpoint{1.576622in}{1.730515in}}{\pgfqpoint{1.576622in}{1.722279in}}%
\pgfpathcurveto{\pgfqpoint{1.576622in}{1.714042in}}{\pgfqpoint{1.579895in}{1.706142in}}{\pgfqpoint{1.585719in}{1.700318in}}%
\pgfpathcurveto{\pgfqpoint{1.591542in}{1.694494in}}{\pgfqpoint{1.599443in}{1.691222in}}{\pgfqpoint{1.607679in}{1.691222in}}%
\pgfpathclose%
\pgfusepath{stroke,fill}%
\end{pgfscope}%
\begin{pgfscope}%
\pgfpathrectangle{\pgfqpoint{0.100000in}{0.212622in}}{\pgfqpoint{3.696000in}{3.696000in}}%
\pgfusepath{clip}%
\pgfsetbuttcap%
\pgfsetroundjoin%
\definecolor{currentfill}{rgb}{0.121569,0.466667,0.705882}%
\pgfsetfillcolor{currentfill}%
\pgfsetfillopacity{0.622539}%
\pgfsetlinewidth{1.003750pt}%
\definecolor{currentstroke}{rgb}{0.121569,0.466667,0.705882}%
\pgfsetstrokecolor{currentstroke}%
\pgfsetstrokeopacity{0.622539}%
\pgfsetdash{}{0pt}%
\pgfpathmoveto{\pgfqpoint{0.785098in}{1.419507in}}%
\pgfpathcurveto{\pgfqpoint{0.793334in}{1.419507in}}{\pgfqpoint{0.801234in}{1.422780in}}{\pgfqpoint{0.807058in}{1.428604in}}%
\pgfpathcurveto{\pgfqpoint{0.812882in}{1.434427in}}{\pgfqpoint{0.816154in}{1.442328in}}{\pgfqpoint{0.816154in}{1.450564in}}%
\pgfpathcurveto{\pgfqpoint{0.816154in}{1.458800in}}{\pgfqpoint{0.812882in}{1.466700in}}{\pgfqpoint{0.807058in}{1.472524in}}%
\pgfpathcurveto{\pgfqpoint{0.801234in}{1.478348in}}{\pgfqpoint{0.793334in}{1.481620in}}{\pgfqpoint{0.785098in}{1.481620in}}%
\pgfpathcurveto{\pgfqpoint{0.776862in}{1.481620in}}{\pgfqpoint{0.768962in}{1.478348in}}{\pgfqpoint{0.763138in}{1.472524in}}%
\pgfpathcurveto{\pgfqpoint{0.757314in}{1.466700in}}{\pgfqpoint{0.754041in}{1.458800in}}{\pgfqpoint{0.754041in}{1.450564in}}%
\pgfpathcurveto{\pgfqpoint{0.754041in}{1.442328in}}{\pgfqpoint{0.757314in}{1.434427in}}{\pgfqpoint{0.763138in}{1.428604in}}%
\pgfpathcurveto{\pgfqpoint{0.768962in}{1.422780in}}{\pgfqpoint{0.776862in}{1.419507in}}{\pgfqpoint{0.785098in}{1.419507in}}%
\pgfpathclose%
\pgfusepath{stroke,fill}%
\end{pgfscope}%
\begin{pgfscope}%
\pgfpathrectangle{\pgfqpoint{0.100000in}{0.212622in}}{\pgfqpoint{3.696000in}{3.696000in}}%
\pgfusepath{clip}%
\pgfsetbuttcap%
\pgfsetroundjoin%
\definecolor{currentfill}{rgb}{0.121569,0.466667,0.705882}%
\pgfsetfillcolor{currentfill}%
\pgfsetfillopacity{0.624336}%
\pgfsetlinewidth{1.003750pt}%
\definecolor{currentstroke}{rgb}{0.121569,0.466667,0.705882}%
\pgfsetstrokecolor{currentstroke}%
\pgfsetstrokeopacity{0.624336}%
\pgfsetdash{}{0pt}%
\pgfpathmoveto{\pgfqpoint{1.609629in}{1.689655in}}%
\pgfpathcurveto{\pgfqpoint{1.617866in}{1.689655in}}{\pgfqpoint{1.625766in}{1.692927in}}{\pgfqpoint{1.631590in}{1.698751in}}%
\pgfpathcurveto{\pgfqpoint{1.637414in}{1.704575in}}{\pgfqpoint{1.640686in}{1.712475in}}{\pgfqpoint{1.640686in}{1.720712in}}%
\pgfpathcurveto{\pgfqpoint{1.640686in}{1.728948in}}{\pgfqpoint{1.637414in}{1.736848in}}{\pgfqpoint{1.631590in}{1.742672in}}%
\pgfpathcurveto{\pgfqpoint{1.625766in}{1.748496in}}{\pgfqpoint{1.617866in}{1.751768in}}{\pgfqpoint{1.609629in}{1.751768in}}%
\pgfpathcurveto{\pgfqpoint{1.601393in}{1.751768in}}{\pgfqpoint{1.593493in}{1.748496in}}{\pgfqpoint{1.587669in}{1.742672in}}%
\pgfpathcurveto{\pgfqpoint{1.581845in}{1.736848in}}{\pgfqpoint{1.578573in}{1.728948in}}{\pgfqpoint{1.578573in}{1.720712in}}%
\pgfpathcurveto{\pgfqpoint{1.578573in}{1.712475in}}{\pgfqpoint{1.581845in}{1.704575in}}{\pgfqpoint{1.587669in}{1.698751in}}%
\pgfpathcurveto{\pgfqpoint{1.593493in}{1.692927in}}{\pgfqpoint{1.601393in}{1.689655in}}{\pgfqpoint{1.609629in}{1.689655in}}%
\pgfpathclose%
\pgfusepath{stroke,fill}%
\end{pgfscope}%
\begin{pgfscope}%
\pgfpathrectangle{\pgfqpoint{0.100000in}{0.212622in}}{\pgfqpoint{3.696000in}{3.696000in}}%
\pgfusepath{clip}%
\pgfsetbuttcap%
\pgfsetroundjoin%
\definecolor{currentfill}{rgb}{0.121569,0.466667,0.705882}%
\pgfsetfillcolor{currentfill}%
\pgfsetfillopacity{0.625086}%
\pgfsetlinewidth{1.003750pt}%
\definecolor{currentstroke}{rgb}{0.121569,0.466667,0.705882}%
\pgfsetstrokecolor{currentstroke}%
\pgfsetstrokeopacity{0.625086}%
\pgfsetdash{}{0pt}%
\pgfpathmoveto{\pgfqpoint{0.779574in}{1.420646in}}%
\pgfpathcurveto{\pgfqpoint{0.787810in}{1.420646in}}{\pgfqpoint{0.795710in}{1.423919in}}{\pgfqpoint{0.801534in}{1.429743in}}%
\pgfpathcurveto{\pgfqpoint{0.807358in}{1.435567in}}{\pgfqpoint{0.810630in}{1.443467in}}{\pgfqpoint{0.810630in}{1.451703in}}%
\pgfpathcurveto{\pgfqpoint{0.810630in}{1.459939in}}{\pgfqpoint{0.807358in}{1.467839in}}{\pgfqpoint{0.801534in}{1.473663in}}%
\pgfpathcurveto{\pgfqpoint{0.795710in}{1.479487in}}{\pgfqpoint{0.787810in}{1.482759in}}{\pgfqpoint{0.779574in}{1.482759in}}%
\pgfpathcurveto{\pgfqpoint{0.771337in}{1.482759in}}{\pgfqpoint{0.763437in}{1.479487in}}{\pgfqpoint{0.757613in}{1.473663in}}%
\pgfpathcurveto{\pgfqpoint{0.751789in}{1.467839in}}{\pgfqpoint{0.748517in}{1.459939in}}{\pgfqpoint{0.748517in}{1.451703in}}%
\pgfpathcurveto{\pgfqpoint{0.748517in}{1.443467in}}{\pgfqpoint{0.751789in}{1.435567in}}{\pgfqpoint{0.757613in}{1.429743in}}%
\pgfpathcurveto{\pgfqpoint{0.763437in}{1.423919in}}{\pgfqpoint{0.771337in}{1.420646in}}{\pgfqpoint{0.779574in}{1.420646in}}%
\pgfpathclose%
\pgfusepath{stroke,fill}%
\end{pgfscope}%
\begin{pgfscope}%
\pgfpathrectangle{\pgfqpoint{0.100000in}{0.212622in}}{\pgfqpoint{3.696000in}{3.696000in}}%
\pgfusepath{clip}%
\pgfsetbuttcap%
\pgfsetroundjoin%
\definecolor{currentfill}{rgb}{0.121569,0.466667,0.705882}%
\pgfsetfillcolor{currentfill}%
\pgfsetfillopacity{0.627453}%
\pgfsetlinewidth{1.003750pt}%
\definecolor{currentstroke}{rgb}{0.121569,0.466667,0.705882}%
\pgfsetstrokecolor{currentstroke}%
\pgfsetstrokeopacity{0.627453}%
\pgfsetdash{}{0pt}%
\pgfpathmoveto{\pgfqpoint{0.777394in}{1.421191in}}%
\pgfpathcurveto{\pgfqpoint{0.785630in}{1.421191in}}{\pgfqpoint{0.793530in}{1.424463in}}{\pgfqpoint{0.799354in}{1.430287in}}%
\pgfpathcurveto{\pgfqpoint{0.805178in}{1.436111in}}{\pgfqpoint{0.808450in}{1.444011in}}{\pgfqpoint{0.808450in}{1.452248in}}%
\pgfpathcurveto{\pgfqpoint{0.808450in}{1.460484in}}{\pgfqpoint{0.805178in}{1.468384in}}{\pgfqpoint{0.799354in}{1.474208in}}%
\pgfpathcurveto{\pgfqpoint{0.793530in}{1.480032in}}{\pgfqpoint{0.785630in}{1.483304in}}{\pgfqpoint{0.777394in}{1.483304in}}%
\pgfpathcurveto{\pgfqpoint{0.769158in}{1.483304in}}{\pgfqpoint{0.761258in}{1.480032in}}{\pgfqpoint{0.755434in}{1.474208in}}%
\pgfpathcurveto{\pgfqpoint{0.749610in}{1.468384in}}{\pgfqpoint{0.746337in}{1.460484in}}{\pgfqpoint{0.746337in}{1.452248in}}%
\pgfpathcurveto{\pgfqpoint{0.746337in}{1.444011in}}{\pgfqpoint{0.749610in}{1.436111in}}{\pgfqpoint{0.755434in}{1.430287in}}%
\pgfpathcurveto{\pgfqpoint{0.761258in}{1.424463in}}{\pgfqpoint{0.769158in}{1.421191in}}{\pgfqpoint{0.777394in}{1.421191in}}%
\pgfpathclose%
\pgfusepath{stroke,fill}%
\end{pgfscope}%
\begin{pgfscope}%
\pgfpathrectangle{\pgfqpoint{0.100000in}{0.212622in}}{\pgfqpoint{3.696000in}{3.696000in}}%
\pgfusepath{clip}%
\pgfsetbuttcap%
\pgfsetroundjoin%
\definecolor{currentfill}{rgb}{0.121569,0.466667,0.705882}%
\pgfsetfillcolor{currentfill}%
\pgfsetfillopacity{0.628040}%
\pgfsetlinewidth{1.003750pt}%
\definecolor{currentstroke}{rgb}{0.121569,0.466667,0.705882}%
\pgfsetstrokecolor{currentstroke}%
\pgfsetstrokeopacity{0.628040}%
\pgfsetdash{}{0pt}%
\pgfpathmoveto{\pgfqpoint{1.611030in}{1.689125in}}%
\pgfpathcurveto{\pgfqpoint{1.619266in}{1.689125in}}{\pgfqpoint{1.627166in}{1.692398in}}{\pgfqpoint{1.632990in}{1.698221in}}%
\pgfpathcurveto{\pgfqpoint{1.638814in}{1.704045in}}{\pgfqpoint{1.642086in}{1.711945in}}{\pgfqpoint{1.642086in}{1.720182in}}%
\pgfpathcurveto{\pgfqpoint{1.642086in}{1.728418in}}{\pgfqpoint{1.638814in}{1.736318in}}{\pgfqpoint{1.632990in}{1.742142in}}%
\pgfpathcurveto{\pgfqpoint{1.627166in}{1.747966in}}{\pgfqpoint{1.619266in}{1.751238in}}{\pgfqpoint{1.611030in}{1.751238in}}%
\pgfpathcurveto{\pgfqpoint{1.602793in}{1.751238in}}{\pgfqpoint{1.594893in}{1.747966in}}{\pgfqpoint{1.589069in}{1.742142in}}%
\pgfpathcurveto{\pgfqpoint{1.583246in}{1.736318in}}{\pgfqpoint{1.579973in}{1.728418in}}{\pgfqpoint{1.579973in}{1.720182in}}%
\pgfpathcurveto{\pgfqpoint{1.579973in}{1.711945in}}{\pgfqpoint{1.583246in}{1.704045in}}{\pgfqpoint{1.589069in}{1.698221in}}%
\pgfpathcurveto{\pgfqpoint{1.594893in}{1.692398in}}{\pgfqpoint{1.602793in}{1.689125in}}{\pgfqpoint{1.611030in}{1.689125in}}%
\pgfpathclose%
\pgfusepath{stroke,fill}%
\end{pgfscope}%
\begin{pgfscope}%
\pgfpathrectangle{\pgfqpoint{0.100000in}{0.212622in}}{\pgfqpoint{3.696000in}{3.696000in}}%
\pgfusepath{clip}%
\pgfsetbuttcap%
\pgfsetroundjoin%
\definecolor{currentfill}{rgb}{0.121569,0.466667,0.705882}%
\pgfsetfillcolor{currentfill}%
\pgfsetfillopacity{0.628550}%
\pgfsetlinewidth{1.003750pt}%
\definecolor{currentstroke}{rgb}{0.121569,0.466667,0.705882}%
\pgfsetstrokecolor{currentstroke}%
\pgfsetstrokeopacity{0.628550}%
\pgfsetdash{}{0pt}%
\pgfpathmoveto{\pgfqpoint{0.781782in}{1.388113in}}%
\pgfpathcurveto{\pgfqpoint{0.790018in}{1.388113in}}{\pgfqpoint{0.797918in}{1.391386in}}{\pgfqpoint{0.803742in}{1.397210in}}%
\pgfpathcurveto{\pgfqpoint{0.809566in}{1.403034in}}{\pgfqpoint{0.812838in}{1.410934in}}{\pgfqpoint{0.812838in}{1.419170in}}%
\pgfpathcurveto{\pgfqpoint{0.812838in}{1.427406in}}{\pgfqpoint{0.809566in}{1.435306in}}{\pgfqpoint{0.803742in}{1.441130in}}%
\pgfpathcurveto{\pgfqpoint{0.797918in}{1.446954in}}{\pgfqpoint{0.790018in}{1.450226in}}{\pgfqpoint{0.781782in}{1.450226in}}%
\pgfpathcurveto{\pgfqpoint{0.773545in}{1.450226in}}{\pgfqpoint{0.765645in}{1.446954in}}{\pgfqpoint{0.759821in}{1.441130in}}%
\pgfpathcurveto{\pgfqpoint{0.753997in}{1.435306in}}{\pgfqpoint{0.750725in}{1.427406in}}{\pgfqpoint{0.750725in}{1.419170in}}%
\pgfpathcurveto{\pgfqpoint{0.750725in}{1.410934in}}{\pgfqpoint{0.753997in}{1.403034in}}{\pgfqpoint{0.759821in}{1.397210in}}%
\pgfpathcurveto{\pgfqpoint{0.765645in}{1.391386in}}{\pgfqpoint{0.773545in}{1.388113in}}{\pgfqpoint{0.781782in}{1.388113in}}%
\pgfpathclose%
\pgfusepath{stroke,fill}%
\end{pgfscope}%
\begin{pgfscope}%
\pgfpathrectangle{\pgfqpoint{0.100000in}{0.212622in}}{\pgfqpoint{3.696000in}{3.696000in}}%
\pgfusepath{clip}%
\pgfsetbuttcap%
\pgfsetroundjoin%
\definecolor{currentfill}{rgb}{0.121569,0.466667,0.705882}%
\pgfsetfillcolor{currentfill}%
\pgfsetfillopacity{0.628564}%
\pgfsetlinewidth{1.003750pt}%
\definecolor{currentstroke}{rgb}{0.121569,0.466667,0.705882}%
\pgfsetstrokecolor{currentstroke}%
\pgfsetstrokeopacity{0.628564}%
\pgfsetdash{}{0pt}%
\pgfpathmoveto{\pgfqpoint{0.775134in}{1.421822in}}%
\pgfpathcurveto{\pgfqpoint{0.783370in}{1.421822in}}{\pgfqpoint{0.791270in}{1.425095in}}{\pgfqpoint{0.797094in}{1.430919in}}%
\pgfpathcurveto{\pgfqpoint{0.802918in}{1.436742in}}{\pgfqpoint{0.806190in}{1.444643in}}{\pgfqpoint{0.806190in}{1.452879in}}%
\pgfpathcurveto{\pgfqpoint{0.806190in}{1.461115in}}{\pgfqpoint{0.802918in}{1.469015in}}{\pgfqpoint{0.797094in}{1.474839in}}%
\pgfpathcurveto{\pgfqpoint{0.791270in}{1.480663in}}{\pgfqpoint{0.783370in}{1.483935in}}{\pgfqpoint{0.775134in}{1.483935in}}%
\pgfpathcurveto{\pgfqpoint{0.766897in}{1.483935in}}{\pgfqpoint{0.758997in}{1.480663in}}{\pgfqpoint{0.753173in}{1.474839in}}%
\pgfpathcurveto{\pgfqpoint{0.747349in}{1.469015in}}{\pgfqpoint{0.744077in}{1.461115in}}{\pgfqpoint{0.744077in}{1.452879in}}%
\pgfpathcurveto{\pgfqpoint{0.744077in}{1.444643in}}{\pgfqpoint{0.747349in}{1.436742in}}{\pgfqpoint{0.753173in}{1.430919in}}%
\pgfpathcurveto{\pgfqpoint{0.758997in}{1.425095in}}{\pgfqpoint{0.766897in}{1.421822in}}{\pgfqpoint{0.775134in}{1.421822in}}%
\pgfpathclose%
\pgfusepath{stroke,fill}%
\end{pgfscope}%
\begin{pgfscope}%
\pgfpathrectangle{\pgfqpoint{0.100000in}{0.212622in}}{\pgfqpoint{3.696000in}{3.696000in}}%
\pgfusepath{clip}%
\pgfsetbuttcap%
\pgfsetroundjoin%
\definecolor{currentfill}{rgb}{0.121569,0.466667,0.705882}%
\pgfsetfillcolor{currentfill}%
\pgfsetfillopacity{0.628710}%
\pgfsetlinewidth{1.003750pt}%
\definecolor{currentstroke}{rgb}{0.121569,0.466667,0.705882}%
\pgfsetstrokecolor{currentstroke}%
\pgfsetstrokeopacity{0.628710}%
\pgfsetdash{}{0pt}%
\pgfpathmoveto{\pgfqpoint{0.781659in}{1.388378in}}%
\pgfpathcurveto{\pgfqpoint{0.789895in}{1.388378in}}{\pgfqpoint{0.797795in}{1.391650in}}{\pgfqpoint{0.803619in}{1.397474in}}%
\pgfpathcurveto{\pgfqpoint{0.809443in}{1.403298in}}{\pgfqpoint{0.812716in}{1.411198in}}{\pgfqpoint{0.812716in}{1.419435in}}%
\pgfpathcurveto{\pgfqpoint{0.812716in}{1.427671in}}{\pgfqpoint{0.809443in}{1.435571in}}{\pgfqpoint{0.803619in}{1.441395in}}%
\pgfpathcurveto{\pgfqpoint{0.797795in}{1.447219in}}{\pgfqpoint{0.789895in}{1.450491in}}{\pgfqpoint{0.781659in}{1.450491in}}%
\pgfpathcurveto{\pgfqpoint{0.773423in}{1.450491in}}{\pgfqpoint{0.765523in}{1.447219in}}{\pgfqpoint{0.759699in}{1.441395in}}%
\pgfpathcurveto{\pgfqpoint{0.753875in}{1.435571in}}{\pgfqpoint{0.750603in}{1.427671in}}{\pgfqpoint{0.750603in}{1.419435in}}%
\pgfpathcurveto{\pgfqpoint{0.750603in}{1.411198in}}{\pgfqpoint{0.753875in}{1.403298in}}{\pgfqpoint{0.759699in}{1.397474in}}%
\pgfpathcurveto{\pgfqpoint{0.765523in}{1.391650in}}{\pgfqpoint{0.773423in}{1.388378in}}{\pgfqpoint{0.781659in}{1.388378in}}%
\pgfpathclose%
\pgfusepath{stroke,fill}%
\end{pgfscope}%
\begin{pgfscope}%
\pgfpathrectangle{\pgfqpoint{0.100000in}{0.212622in}}{\pgfqpoint{3.696000in}{3.696000in}}%
\pgfusepath{clip}%
\pgfsetbuttcap%
\pgfsetroundjoin%
\definecolor{currentfill}{rgb}{0.121569,0.466667,0.705882}%
\pgfsetfillcolor{currentfill}%
\pgfsetfillopacity{0.629063}%
\pgfsetlinewidth{1.003750pt}%
\definecolor{currentstroke}{rgb}{0.121569,0.466667,0.705882}%
\pgfsetstrokecolor{currentstroke}%
\pgfsetstrokeopacity{0.629063}%
\pgfsetdash{}{0pt}%
\pgfpathmoveto{\pgfqpoint{0.781227in}{1.388772in}}%
\pgfpathcurveto{\pgfqpoint{0.789464in}{1.388772in}}{\pgfqpoint{0.797364in}{1.392044in}}{\pgfqpoint{0.803188in}{1.397868in}}%
\pgfpathcurveto{\pgfqpoint{0.809012in}{1.403692in}}{\pgfqpoint{0.812284in}{1.411592in}}{\pgfqpoint{0.812284in}{1.419828in}}%
\pgfpathcurveto{\pgfqpoint{0.812284in}{1.428064in}}{\pgfqpoint{0.809012in}{1.435964in}}{\pgfqpoint{0.803188in}{1.441788in}}%
\pgfpathcurveto{\pgfqpoint{0.797364in}{1.447612in}}{\pgfqpoint{0.789464in}{1.450885in}}{\pgfqpoint{0.781227in}{1.450885in}}%
\pgfpathcurveto{\pgfqpoint{0.772991in}{1.450885in}}{\pgfqpoint{0.765091in}{1.447612in}}{\pgfqpoint{0.759267in}{1.441788in}}%
\pgfpathcurveto{\pgfqpoint{0.753443in}{1.435964in}}{\pgfqpoint{0.750171in}{1.428064in}}{\pgfqpoint{0.750171in}{1.419828in}}%
\pgfpathcurveto{\pgfqpoint{0.750171in}{1.411592in}}{\pgfqpoint{0.753443in}{1.403692in}}{\pgfqpoint{0.759267in}{1.397868in}}%
\pgfpathcurveto{\pgfqpoint{0.765091in}{1.392044in}}{\pgfqpoint{0.772991in}{1.388772in}}{\pgfqpoint{0.781227in}{1.388772in}}%
\pgfpathclose%
\pgfusepath{stroke,fill}%
\end{pgfscope}%
\begin{pgfscope}%
\pgfpathrectangle{\pgfqpoint{0.100000in}{0.212622in}}{\pgfqpoint{3.696000in}{3.696000in}}%
\pgfusepath{clip}%
\pgfsetbuttcap%
\pgfsetroundjoin%
\definecolor{currentfill}{rgb}{0.121569,0.466667,0.705882}%
\pgfsetfillcolor{currentfill}%
\pgfsetfillopacity{0.629391}%
\pgfsetlinewidth{1.003750pt}%
\definecolor{currentstroke}{rgb}{0.121569,0.466667,0.705882}%
\pgfsetstrokecolor{currentstroke}%
\pgfsetstrokeopacity{0.629391}%
\pgfsetdash{}{0pt}%
\pgfpathmoveto{\pgfqpoint{0.774431in}{1.422781in}}%
\pgfpathcurveto{\pgfqpoint{0.782668in}{1.422781in}}{\pgfqpoint{0.790568in}{1.426053in}}{\pgfqpoint{0.796392in}{1.431877in}}%
\pgfpathcurveto{\pgfqpoint{0.802216in}{1.437701in}}{\pgfqpoint{0.805488in}{1.445601in}}{\pgfqpoint{0.805488in}{1.453837in}}%
\pgfpathcurveto{\pgfqpoint{0.805488in}{1.462073in}}{\pgfqpoint{0.802216in}{1.469973in}}{\pgfqpoint{0.796392in}{1.475797in}}%
\pgfpathcurveto{\pgfqpoint{0.790568in}{1.481621in}}{\pgfqpoint{0.782668in}{1.484894in}}{\pgfqpoint{0.774431in}{1.484894in}}%
\pgfpathcurveto{\pgfqpoint{0.766195in}{1.484894in}}{\pgfqpoint{0.758295in}{1.481621in}}{\pgfqpoint{0.752471in}{1.475797in}}%
\pgfpathcurveto{\pgfqpoint{0.746647in}{1.469973in}}{\pgfqpoint{0.743375in}{1.462073in}}{\pgfqpoint{0.743375in}{1.453837in}}%
\pgfpathcurveto{\pgfqpoint{0.743375in}{1.445601in}}{\pgfqpoint{0.746647in}{1.437701in}}{\pgfqpoint{0.752471in}{1.431877in}}%
\pgfpathcurveto{\pgfqpoint{0.758295in}{1.426053in}}{\pgfqpoint{0.766195in}{1.422781in}}{\pgfqpoint{0.774431in}{1.422781in}}%
\pgfpathclose%
\pgfusepath{stroke,fill}%
\end{pgfscope}%
\begin{pgfscope}%
\pgfpathrectangle{\pgfqpoint{0.100000in}{0.212622in}}{\pgfqpoint{3.696000in}{3.696000in}}%
\pgfusepath{clip}%
\pgfsetbuttcap%
\pgfsetroundjoin%
\definecolor{currentfill}{rgb}{0.121569,0.466667,0.705882}%
\pgfsetfillcolor{currentfill}%
\pgfsetfillopacity{0.629391}%
\pgfsetlinewidth{1.003750pt}%
\definecolor{currentstroke}{rgb}{0.121569,0.466667,0.705882}%
\pgfsetstrokecolor{currentstroke}%
\pgfsetstrokeopacity{0.629391}%
\pgfsetdash{}{0pt}%
\pgfpathmoveto{\pgfqpoint{0.774431in}{1.422781in}}%
\pgfpathcurveto{\pgfqpoint{0.782668in}{1.422781in}}{\pgfqpoint{0.790568in}{1.426053in}}{\pgfqpoint{0.796392in}{1.431877in}}%
\pgfpathcurveto{\pgfqpoint{0.802216in}{1.437701in}}{\pgfqpoint{0.805488in}{1.445601in}}{\pgfqpoint{0.805488in}{1.453837in}}%
\pgfpathcurveto{\pgfqpoint{0.805488in}{1.462073in}}{\pgfqpoint{0.802216in}{1.469973in}}{\pgfqpoint{0.796392in}{1.475797in}}%
\pgfpathcurveto{\pgfqpoint{0.790568in}{1.481621in}}{\pgfqpoint{0.782668in}{1.484894in}}{\pgfqpoint{0.774431in}{1.484894in}}%
\pgfpathcurveto{\pgfqpoint{0.766195in}{1.484894in}}{\pgfqpoint{0.758295in}{1.481621in}}{\pgfqpoint{0.752471in}{1.475797in}}%
\pgfpathcurveto{\pgfqpoint{0.746647in}{1.469973in}}{\pgfqpoint{0.743375in}{1.462073in}}{\pgfqpoint{0.743375in}{1.453837in}}%
\pgfpathcurveto{\pgfqpoint{0.743375in}{1.445601in}}{\pgfqpoint{0.746647in}{1.437701in}}{\pgfqpoint{0.752471in}{1.431877in}}%
\pgfpathcurveto{\pgfqpoint{0.758295in}{1.426053in}}{\pgfqpoint{0.766195in}{1.422781in}}{\pgfqpoint{0.774431in}{1.422781in}}%
\pgfpathclose%
\pgfusepath{stroke,fill}%
\end{pgfscope}%
\begin{pgfscope}%
\pgfpathrectangle{\pgfqpoint{0.100000in}{0.212622in}}{\pgfqpoint{3.696000in}{3.696000in}}%
\pgfusepath{clip}%
\pgfsetbuttcap%
\pgfsetroundjoin%
\definecolor{currentfill}{rgb}{0.121569,0.466667,0.705882}%
\pgfsetfillcolor{currentfill}%
\pgfsetfillopacity{0.629391}%
\pgfsetlinewidth{1.003750pt}%
\definecolor{currentstroke}{rgb}{0.121569,0.466667,0.705882}%
\pgfsetstrokecolor{currentstroke}%
\pgfsetstrokeopacity{0.629391}%
\pgfsetdash{}{0pt}%
\pgfpathmoveto{\pgfqpoint{0.774431in}{1.422781in}}%
\pgfpathcurveto{\pgfqpoint{0.782668in}{1.422781in}}{\pgfqpoint{0.790568in}{1.426053in}}{\pgfqpoint{0.796392in}{1.431877in}}%
\pgfpathcurveto{\pgfqpoint{0.802216in}{1.437701in}}{\pgfqpoint{0.805488in}{1.445601in}}{\pgfqpoint{0.805488in}{1.453837in}}%
\pgfpathcurveto{\pgfqpoint{0.805488in}{1.462073in}}{\pgfqpoint{0.802216in}{1.469973in}}{\pgfqpoint{0.796392in}{1.475797in}}%
\pgfpathcurveto{\pgfqpoint{0.790568in}{1.481621in}}{\pgfqpoint{0.782668in}{1.484894in}}{\pgfqpoint{0.774431in}{1.484894in}}%
\pgfpathcurveto{\pgfqpoint{0.766195in}{1.484894in}}{\pgfqpoint{0.758295in}{1.481621in}}{\pgfqpoint{0.752471in}{1.475797in}}%
\pgfpathcurveto{\pgfqpoint{0.746647in}{1.469973in}}{\pgfqpoint{0.743375in}{1.462073in}}{\pgfqpoint{0.743375in}{1.453837in}}%
\pgfpathcurveto{\pgfqpoint{0.743375in}{1.445601in}}{\pgfqpoint{0.746647in}{1.437701in}}{\pgfqpoint{0.752471in}{1.431877in}}%
\pgfpathcurveto{\pgfqpoint{0.758295in}{1.426053in}}{\pgfqpoint{0.766195in}{1.422781in}}{\pgfqpoint{0.774431in}{1.422781in}}%
\pgfpathclose%
\pgfusepath{stroke,fill}%
\end{pgfscope}%
\begin{pgfscope}%
\pgfpathrectangle{\pgfqpoint{0.100000in}{0.212622in}}{\pgfqpoint{3.696000in}{3.696000in}}%
\pgfusepath{clip}%
\pgfsetbuttcap%
\pgfsetroundjoin%
\definecolor{currentfill}{rgb}{0.121569,0.466667,0.705882}%
\pgfsetfillcolor{currentfill}%
\pgfsetfillopacity{0.629391}%
\pgfsetlinewidth{1.003750pt}%
\definecolor{currentstroke}{rgb}{0.121569,0.466667,0.705882}%
\pgfsetstrokecolor{currentstroke}%
\pgfsetstrokeopacity{0.629391}%
\pgfsetdash{}{0pt}%
\pgfpathmoveto{\pgfqpoint{0.774431in}{1.422780in}}%
\pgfpathcurveto{\pgfqpoint{0.782668in}{1.422780in}}{\pgfqpoint{0.790568in}{1.426053in}}{\pgfqpoint{0.796391in}{1.431877in}}%
\pgfpathcurveto{\pgfqpoint{0.802215in}{1.437701in}}{\pgfqpoint{0.805488in}{1.445601in}}{\pgfqpoint{0.805488in}{1.453837in}}%
\pgfpathcurveto{\pgfqpoint{0.805488in}{1.462073in}}{\pgfqpoint{0.802215in}{1.469973in}}{\pgfqpoint{0.796391in}{1.475797in}}%
\pgfpathcurveto{\pgfqpoint{0.790568in}{1.481621in}}{\pgfqpoint{0.782668in}{1.484893in}}{\pgfqpoint{0.774431in}{1.484893in}}%
\pgfpathcurveto{\pgfqpoint{0.766195in}{1.484893in}}{\pgfqpoint{0.758295in}{1.481621in}}{\pgfqpoint{0.752471in}{1.475797in}}%
\pgfpathcurveto{\pgfqpoint{0.746647in}{1.469973in}}{\pgfqpoint{0.743375in}{1.462073in}}{\pgfqpoint{0.743375in}{1.453837in}}%
\pgfpathcurveto{\pgfqpoint{0.743375in}{1.445601in}}{\pgfqpoint{0.746647in}{1.437701in}}{\pgfqpoint{0.752471in}{1.431877in}}%
\pgfpathcurveto{\pgfqpoint{0.758295in}{1.426053in}}{\pgfqpoint{0.766195in}{1.422780in}}{\pgfqpoint{0.774431in}{1.422780in}}%
\pgfpathclose%
\pgfusepath{stroke,fill}%
\end{pgfscope}%
\begin{pgfscope}%
\pgfpathrectangle{\pgfqpoint{0.100000in}{0.212622in}}{\pgfqpoint{3.696000in}{3.696000in}}%
\pgfusepath{clip}%
\pgfsetbuttcap%
\pgfsetroundjoin%
\definecolor{currentfill}{rgb}{0.121569,0.466667,0.705882}%
\pgfsetfillcolor{currentfill}%
\pgfsetfillopacity{0.629391}%
\pgfsetlinewidth{1.003750pt}%
\definecolor{currentstroke}{rgb}{0.121569,0.466667,0.705882}%
\pgfsetstrokecolor{currentstroke}%
\pgfsetstrokeopacity{0.629391}%
\pgfsetdash{}{0pt}%
\pgfpathmoveto{\pgfqpoint{0.774431in}{1.422780in}}%
\pgfpathcurveto{\pgfqpoint{0.782667in}{1.422780in}}{\pgfqpoint{0.790567in}{1.426053in}}{\pgfqpoint{0.796391in}{1.431877in}}%
\pgfpathcurveto{\pgfqpoint{0.802215in}{1.437701in}}{\pgfqpoint{0.805488in}{1.445601in}}{\pgfqpoint{0.805488in}{1.453837in}}%
\pgfpathcurveto{\pgfqpoint{0.805488in}{1.462073in}}{\pgfqpoint{0.802215in}{1.469973in}}{\pgfqpoint{0.796391in}{1.475797in}}%
\pgfpathcurveto{\pgfqpoint{0.790567in}{1.481621in}}{\pgfqpoint{0.782667in}{1.484893in}}{\pgfqpoint{0.774431in}{1.484893in}}%
\pgfpathcurveto{\pgfqpoint{0.766195in}{1.484893in}}{\pgfqpoint{0.758295in}{1.481621in}}{\pgfqpoint{0.752471in}{1.475797in}}%
\pgfpathcurveto{\pgfqpoint{0.746647in}{1.469973in}}{\pgfqpoint{0.743375in}{1.462073in}}{\pgfqpoint{0.743375in}{1.453837in}}%
\pgfpathcurveto{\pgfqpoint{0.743375in}{1.445601in}}{\pgfqpoint{0.746647in}{1.437701in}}{\pgfqpoint{0.752471in}{1.431877in}}%
\pgfpathcurveto{\pgfqpoint{0.758295in}{1.426053in}}{\pgfqpoint{0.766195in}{1.422780in}}{\pgfqpoint{0.774431in}{1.422780in}}%
\pgfpathclose%
\pgfusepath{stroke,fill}%
\end{pgfscope}%
\begin{pgfscope}%
\pgfpathrectangle{\pgfqpoint{0.100000in}{0.212622in}}{\pgfqpoint{3.696000in}{3.696000in}}%
\pgfusepath{clip}%
\pgfsetbuttcap%
\pgfsetroundjoin%
\definecolor{currentfill}{rgb}{0.121569,0.466667,0.705882}%
\pgfsetfillcolor{currentfill}%
\pgfsetfillopacity{0.629391}%
\pgfsetlinewidth{1.003750pt}%
\definecolor{currentstroke}{rgb}{0.121569,0.466667,0.705882}%
\pgfsetstrokecolor{currentstroke}%
\pgfsetstrokeopacity{0.629391}%
\pgfsetdash{}{0pt}%
\pgfpathmoveto{\pgfqpoint{0.774431in}{1.422780in}}%
\pgfpathcurveto{\pgfqpoint{0.782667in}{1.422780in}}{\pgfqpoint{0.790567in}{1.426052in}}{\pgfqpoint{0.796391in}{1.431876in}}%
\pgfpathcurveto{\pgfqpoint{0.802215in}{1.437700in}}{\pgfqpoint{0.805487in}{1.445600in}}{\pgfqpoint{0.805487in}{1.453837in}}%
\pgfpathcurveto{\pgfqpoint{0.805487in}{1.462073in}}{\pgfqpoint{0.802215in}{1.469973in}}{\pgfqpoint{0.796391in}{1.475797in}}%
\pgfpathcurveto{\pgfqpoint{0.790567in}{1.481621in}}{\pgfqpoint{0.782667in}{1.484893in}}{\pgfqpoint{0.774431in}{1.484893in}}%
\pgfpathcurveto{\pgfqpoint{0.766194in}{1.484893in}}{\pgfqpoint{0.758294in}{1.481621in}}{\pgfqpoint{0.752470in}{1.475797in}}%
\pgfpathcurveto{\pgfqpoint{0.746646in}{1.469973in}}{\pgfqpoint{0.743374in}{1.462073in}}{\pgfqpoint{0.743374in}{1.453837in}}%
\pgfpathcurveto{\pgfqpoint{0.743374in}{1.445600in}}{\pgfqpoint{0.746646in}{1.437700in}}{\pgfqpoint{0.752470in}{1.431876in}}%
\pgfpathcurveto{\pgfqpoint{0.758294in}{1.426052in}}{\pgfqpoint{0.766194in}{1.422780in}}{\pgfqpoint{0.774431in}{1.422780in}}%
\pgfpathclose%
\pgfusepath{stroke,fill}%
\end{pgfscope}%
\begin{pgfscope}%
\pgfpathrectangle{\pgfqpoint{0.100000in}{0.212622in}}{\pgfqpoint{3.696000in}{3.696000in}}%
\pgfusepath{clip}%
\pgfsetbuttcap%
\pgfsetroundjoin%
\definecolor{currentfill}{rgb}{0.121569,0.466667,0.705882}%
\pgfsetfillcolor{currentfill}%
\pgfsetfillopacity{0.629392}%
\pgfsetlinewidth{1.003750pt}%
\definecolor{currentstroke}{rgb}{0.121569,0.466667,0.705882}%
\pgfsetstrokecolor{currentstroke}%
\pgfsetstrokeopacity{0.629392}%
\pgfsetdash{}{0pt}%
\pgfpathmoveto{\pgfqpoint{0.774430in}{1.422780in}}%
\pgfpathcurveto{\pgfqpoint{0.782666in}{1.422780in}}{\pgfqpoint{0.790566in}{1.426052in}}{\pgfqpoint{0.796390in}{1.431876in}}%
\pgfpathcurveto{\pgfqpoint{0.802214in}{1.437700in}}{\pgfqpoint{0.805486in}{1.445600in}}{\pgfqpoint{0.805486in}{1.453836in}}%
\pgfpathcurveto{\pgfqpoint{0.805486in}{1.462073in}}{\pgfqpoint{0.802214in}{1.469973in}}{\pgfqpoint{0.796390in}{1.475797in}}%
\pgfpathcurveto{\pgfqpoint{0.790566in}{1.481621in}}{\pgfqpoint{0.782666in}{1.484893in}}{\pgfqpoint{0.774430in}{1.484893in}}%
\pgfpathcurveto{\pgfqpoint{0.766194in}{1.484893in}}{\pgfqpoint{0.758294in}{1.481621in}}{\pgfqpoint{0.752470in}{1.475797in}}%
\pgfpathcurveto{\pgfqpoint{0.746646in}{1.469973in}}{\pgfqpoint{0.743373in}{1.462073in}}{\pgfqpoint{0.743373in}{1.453836in}}%
\pgfpathcurveto{\pgfqpoint{0.743373in}{1.445600in}}{\pgfqpoint{0.746646in}{1.437700in}}{\pgfqpoint{0.752470in}{1.431876in}}%
\pgfpathcurveto{\pgfqpoint{0.758294in}{1.426052in}}{\pgfqpoint{0.766194in}{1.422780in}}{\pgfqpoint{0.774430in}{1.422780in}}%
\pgfpathclose%
\pgfusepath{stroke,fill}%
\end{pgfscope}%
\begin{pgfscope}%
\pgfpathrectangle{\pgfqpoint{0.100000in}{0.212622in}}{\pgfqpoint{3.696000in}{3.696000in}}%
\pgfusepath{clip}%
\pgfsetbuttcap%
\pgfsetroundjoin%
\definecolor{currentfill}{rgb}{0.121569,0.466667,0.705882}%
\pgfsetfillcolor{currentfill}%
\pgfsetfillopacity{0.629392}%
\pgfsetlinewidth{1.003750pt}%
\definecolor{currentstroke}{rgb}{0.121569,0.466667,0.705882}%
\pgfsetstrokecolor{currentstroke}%
\pgfsetstrokeopacity{0.629392}%
\pgfsetdash{}{0pt}%
\pgfpathmoveto{\pgfqpoint{0.774429in}{1.422779in}}%
\pgfpathcurveto{\pgfqpoint{0.782665in}{1.422779in}}{\pgfqpoint{0.790565in}{1.426052in}}{\pgfqpoint{0.796389in}{1.431876in}}%
\pgfpathcurveto{\pgfqpoint{0.802213in}{1.437700in}}{\pgfqpoint{0.805485in}{1.445600in}}{\pgfqpoint{0.805485in}{1.453836in}}%
\pgfpathcurveto{\pgfqpoint{0.805485in}{1.462072in}}{\pgfqpoint{0.802213in}{1.469972in}}{\pgfqpoint{0.796389in}{1.475796in}}%
\pgfpathcurveto{\pgfqpoint{0.790565in}{1.481620in}}{\pgfqpoint{0.782665in}{1.484892in}}{\pgfqpoint{0.774429in}{1.484892in}}%
\pgfpathcurveto{\pgfqpoint{0.766193in}{1.484892in}}{\pgfqpoint{0.758292in}{1.481620in}}{\pgfqpoint{0.752469in}{1.475796in}}%
\pgfpathcurveto{\pgfqpoint{0.746645in}{1.469972in}}{\pgfqpoint{0.743372in}{1.462072in}}{\pgfqpoint{0.743372in}{1.453836in}}%
\pgfpathcurveto{\pgfqpoint{0.743372in}{1.445600in}}{\pgfqpoint{0.746645in}{1.437700in}}{\pgfqpoint{0.752469in}{1.431876in}}%
\pgfpathcurveto{\pgfqpoint{0.758292in}{1.426052in}}{\pgfqpoint{0.766193in}{1.422779in}}{\pgfqpoint{0.774429in}{1.422779in}}%
\pgfpathclose%
\pgfusepath{stroke,fill}%
\end{pgfscope}%
\begin{pgfscope}%
\pgfpathrectangle{\pgfqpoint{0.100000in}{0.212622in}}{\pgfqpoint{3.696000in}{3.696000in}}%
\pgfusepath{clip}%
\pgfsetbuttcap%
\pgfsetroundjoin%
\definecolor{currentfill}{rgb}{0.121569,0.466667,0.705882}%
\pgfsetfillcolor{currentfill}%
\pgfsetfillopacity{0.629393}%
\pgfsetlinewidth{1.003750pt}%
\definecolor{currentstroke}{rgb}{0.121569,0.466667,0.705882}%
\pgfsetstrokecolor{currentstroke}%
\pgfsetstrokeopacity{0.629393}%
\pgfsetdash{}{0pt}%
\pgfpathmoveto{\pgfqpoint{0.774426in}{1.422778in}}%
\pgfpathcurveto{\pgfqpoint{0.782663in}{1.422778in}}{\pgfqpoint{0.790563in}{1.426050in}}{\pgfqpoint{0.796387in}{1.431874in}}%
\pgfpathcurveto{\pgfqpoint{0.802211in}{1.437698in}}{\pgfqpoint{0.805483in}{1.445598in}}{\pgfqpoint{0.805483in}{1.453834in}}%
\pgfpathcurveto{\pgfqpoint{0.805483in}{1.462071in}}{\pgfqpoint{0.802211in}{1.469971in}}{\pgfqpoint{0.796387in}{1.475795in}}%
\pgfpathcurveto{\pgfqpoint{0.790563in}{1.481619in}}{\pgfqpoint{0.782663in}{1.484891in}}{\pgfqpoint{0.774426in}{1.484891in}}%
\pgfpathcurveto{\pgfqpoint{0.766190in}{1.484891in}}{\pgfqpoint{0.758290in}{1.481619in}}{\pgfqpoint{0.752466in}{1.475795in}}%
\pgfpathcurveto{\pgfqpoint{0.746642in}{1.469971in}}{\pgfqpoint{0.743370in}{1.462071in}}{\pgfqpoint{0.743370in}{1.453834in}}%
\pgfpathcurveto{\pgfqpoint{0.743370in}{1.445598in}}{\pgfqpoint{0.746642in}{1.437698in}}{\pgfqpoint{0.752466in}{1.431874in}}%
\pgfpathcurveto{\pgfqpoint{0.758290in}{1.426050in}}{\pgfqpoint{0.766190in}{1.422778in}}{\pgfqpoint{0.774426in}{1.422778in}}%
\pgfpathclose%
\pgfusepath{stroke,fill}%
\end{pgfscope}%
\begin{pgfscope}%
\pgfpathrectangle{\pgfqpoint{0.100000in}{0.212622in}}{\pgfqpoint{3.696000in}{3.696000in}}%
\pgfusepath{clip}%
\pgfsetbuttcap%
\pgfsetroundjoin%
\definecolor{currentfill}{rgb}{0.121569,0.466667,0.705882}%
\pgfsetfillcolor{currentfill}%
\pgfsetfillopacity{0.629395}%
\pgfsetlinewidth{1.003750pt}%
\definecolor{currentstroke}{rgb}{0.121569,0.466667,0.705882}%
\pgfsetstrokecolor{currentstroke}%
\pgfsetstrokeopacity{0.629395}%
\pgfsetdash{}{0pt}%
\pgfpathmoveto{\pgfqpoint{0.774422in}{1.422774in}}%
\pgfpathcurveto{\pgfqpoint{0.782659in}{1.422774in}}{\pgfqpoint{0.790559in}{1.426046in}}{\pgfqpoint{0.796383in}{1.431870in}}%
\pgfpathcurveto{\pgfqpoint{0.802207in}{1.437694in}}{\pgfqpoint{0.805479in}{1.445594in}}{\pgfqpoint{0.805479in}{1.453831in}}%
\pgfpathcurveto{\pgfqpoint{0.805479in}{1.462067in}}{\pgfqpoint{0.802207in}{1.469967in}}{\pgfqpoint{0.796383in}{1.475791in}}%
\pgfpathcurveto{\pgfqpoint{0.790559in}{1.481615in}}{\pgfqpoint{0.782659in}{1.484887in}}{\pgfqpoint{0.774422in}{1.484887in}}%
\pgfpathcurveto{\pgfqpoint{0.766186in}{1.484887in}}{\pgfqpoint{0.758286in}{1.481615in}}{\pgfqpoint{0.752462in}{1.475791in}}%
\pgfpathcurveto{\pgfqpoint{0.746638in}{1.469967in}}{\pgfqpoint{0.743366in}{1.462067in}}{\pgfqpoint{0.743366in}{1.453831in}}%
\pgfpathcurveto{\pgfqpoint{0.743366in}{1.445594in}}{\pgfqpoint{0.746638in}{1.437694in}}{\pgfqpoint{0.752462in}{1.431870in}}%
\pgfpathcurveto{\pgfqpoint{0.758286in}{1.426046in}}{\pgfqpoint{0.766186in}{1.422774in}}{\pgfqpoint{0.774422in}{1.422774in}}%
\pgfpathclose%
\pgfusepath{stroke,fill}%
\end{pgfscope}%
\begin{pgfscope}%
\pgfpathrectangle{\pgfqpoint{0.100000in}{0.212622in}}{\pgfqpoint{3.696000in}{3.696000in}}%
\pgfusepath{clip}%
\pgfsetbuttcap%
\pgfsetroundjoin%
\definecolor{currentfill}{rgb}{0.121569,0.466667,0.705882}%
\pgfsetfillcolor{currentfill}%
\pgfsetfillopacity{0.629399}%
\pgfsetlinewidth{1.003750pt}%
\definecolor{currentstroke}{rgb}{0.121569,0.466667,0.705882}%
\pgfsetstrokecolor{currentstroke}%
\pgfsetstrokeopacity{0.629399}%
\pgfsetdash{}{0pt}%
\pgfpathmoveto{\pgfqpoint{0.774415in}{1.422772in}}%
\pgfpathcurveto{\pgfqpoint{0.782651in}{1.422772in}}{\pgfqpoint{0.790551in}{1.426044in}}{\pgfqpoint{0.796375in}{1.431868in}}%
\pgfpathcurveto{\pgfqpoint{0.802199in}{1.437692in}}{\pgfqpoint{0.805471in}{1.445592in}}{\pgfqpoint{0.805471in}{1.453828in}}%
\pgfpathcurveto{\pgfqpoint{0.805471in}{1.462065in}}{\pgfqpoint{0.802199in}{1.469965in}}{\pgfqpoint{0.796375in}{1.475789in}}%
\pgfpathcurveto{\pgfqpoint{0.790551in}{1.481613in}}{\pgfqpoint{0.782651in}{1.484885in}}{\pgfqpoint{0.774415in}{1.484885in}}%
\pgfpathcurveto{\pgfqpoint{0.766178in}{1.484885in}}{\pgfqpoint{0.758278in}{1.481613in}}{\pgfqpoint{0.752454in}{1.475789in}}%
\pgfpathcurveto{\pgfqpoint{0.746631in}{1.469965in}}{\pgfqpoint{0.743358in}{1.462065in}}{\pgfqpoint{0.743358in}{1.453828in}}%
\pgfpathcurveto{\pgfqpoint{0.743358in}{1.445592in}}{\pgfqpoint{0.746631in}{1.437692in}}{\pgfqpoint{0.752454in}{1.431868in}}%
\pgfpathcurveto{\pgfqpoint{0.758278in}{1.426044in}}{\pgfqpoint{0.766178in}{1.422772in}}{\pgfqpoint{0.774415in}{1.422772in}}%
\pgfpathclose%
\pgfusepath{stroke,fill}%
\end{pgfscope}%
\begin{pgfscope}%
\pgfpathrectangle{\pgfqpoint{0.100000in}{0.212622in}}{\pgfqpoint{3.696000in}{3.696000in}}%
\pgfusepath{clip}%
\pgfsetbuttcap%
\pgfsetroundjoin%
\definecolor{currentfill}{rgb}{0.121569,0.466667,0.705882}%
\pgfsetfillcolor{currentfill}%
\pgfsetfillopacity{0.629404}%
\pgfsetlinewidth{1.003750pt}%
\definecolor{currentstroke}{rgb}{0.121569,0.466667,0.705882}%
\pgfsetstrokecolor{currentstroke}%
\pgfsetstrokeopacity{0.629404}%
\pgfsetdash{}{0pt}%
\pgfpathmoveto{\pgfqpoint{0.774400in}{1.422762in}}%
\pgfpathcurveto{\pgfqpoint{0.782637in}{1.422762in}}{\pgfqpoint{0.790537in}{1.426034in}}{\pgfqpoint{0.796360in}{1.431858in}}%
\pgfpathcurveto{\pgfqpoint{0.802184in}{1.437682in}}{\pgfqpoint{0.805457in}{1.445582in}}{\pgfqpoint{0.805457in}{1.453818in}}%
\pgfpathcurveto{\pgfqpoint{0.805457in}{1.462055in}}{\pgfqpoint{0.802184in}{1.469955in}}{\pgfqpoint{0.796360in}{1.475779in}}%
\pgfpathcurveto{\pgfqpoint{0.790537in}{1.481603in}}{\pgfqpoint{0.782637in}{1.484875in}}{\pgfqpoint{0.774400in}{1.484875in}}%
\pgfpathcurveto{\pgfqpoint{0.766164in}{1.484875in}}{\pgfqpoint{0.758264in}{1.481603in}}{\pgfqpoint{0.752440in}{1.475779in}}%
\pgfpathcurveto{\pgfqpoint{0.746616in}{1.469955in}}{\pgfqpoint{0.743344in}{1.462055in}}{\pgfqpoint{0.743344in}{1.453818in}}%
\pgfpathcurveto{\pgfqpoint{0.743344in}{1.445582in}}{\pgfqpoint{0.746616in}{1.437682in}}{\pgfqpoint{0.752440in}{1.431858in}}%
\pgfpathcurveto{\pgfqpoint{0.758264in}{1.426034in}}{\pgfqpoint{0.766164in}{1.422762in}}{\pgfqpoint{0.774400in}{1.422762in}}%
\pgfpathclose%
\pgfusepath{stroke,fill}%
\end{pgfscope}%
\begin{pgfscope}%
\pgfpathrectangle{\pgfqpoint{0.100000in}{0.212622in}}{\pgfqpoint{3.696000in}{3.696000in}}%
\pgfusepath{clip}%
\pgfsetbuttcap%
\pgfsetroundjoin%
\definecolor{currentfill}{rgb}{0.121569,0.466667,0.705882}%
\pgfsetfillcolor{currentfill}%
\pgfsetfillopacity{0.629417}%
\pgfsetlinewidth{1.003750pt}%
\definecolor{currentstroke}{rgb}{0.121569,0.466667,0.705882}%
\pgfsetstrokecolor{currentstroke}%
\pgfsetstrokeopacity{0.629417}%
\pgfsetdash{}{0pt}%
\pgfpathmoveto{\pgfqpoint{0.774378in}{1.422751in}}%
\pgfpathcurveto{\pgfqpoint{0.782614in}{1.422751in}}{\pgfqpoint{0.790514in}{1.426024in}}{\pgfqpoint{0.796338in}{1.431848in}}%
\pgfpathcurveto{\pgfqpoint{0.802162in}{1.437671in}}{\pgfqpoint{0.805434in}{1.445571in}}{\pgfqpoint{0.805434in}{1.453808in}}%
\pgfpathcurveto{\pgfqpoint{0.805434in}{1.462044in}}{\pgfqpoint{0.802162in}{1.469944in}}{\pgfqpoint{0.796338in}{1.475768in}}%
\pgfpathcurveto{\pgfqpoint{0.790514in}{1.481592in}}{\pgfqpoint{0.782614in}{1.484864in}}{\pgfqpoint{0.774378in}{1.484864in}}%
\pgfpathcurveto{\pgfqpoint{0.766142in}{1.484864in}}{\pgfqpoint{0.758242in}{1.481592in}}{\pgfqpoint{0.752418in}{1.475768in}}%
\pgfpathcurveto{\pgfqpoint{0.746594in}{1.469944in}}{\pgfqpoint{0.743321in}{1.462044in}}{\pgfqpoint{0.743321in}{1.453808in}}%
\pgfpathcurveto{\pgfqpoint{0.743321in}{1.445571in}}{\pgfqpoint{0.746594in}{1.437671in}}{\pgfqpoint{0.752418in}{1.431848in}}%
\pgfpathcurveto{\pgfqpoint{0.758242in}{1.426024in}}{\pgfqpoint{0.766142in}{1.422751in}}{\pgfqpoint{0.774378in}{1.422751in}}%
\pgfpathclose%
\pgfusepath{stroke,fill}%
\end{pgfscope}%
\begin{pgfscope}%
\pgfpathrectangle{\pgfqpoint{0.100000in}{0.212622in}}{\pgfqpoint{3.696000in}{3.696000in}}%
\pgfusepath{clip}%
\pgfsetbuttcap%
\pgfsetroundjoin%
\definecolor{currentfill}{rgb}{0.121569,0.466667,0.705882}%
\pgfsetfillcolor{currentfill}%
\pgfsetfillopacity{0.629443}%
\pgfsetlinewidth{1.003750pt}%
\definecolor{currentstroke}{rgb}{0.121569,0.466667,0.705882}%
\pgfsetstrokecolor{currentstroke}%
\pgfsetstrokeopacity{0.629443}%
\pgfsetdash{}{0pt}%
\pgfpathmoveto{\pgfqpoint{0.774323in}{1.422763in}}%
\pgfpathcurveto{\pgfqpoint{0.782559in}{1.422763in}}{\pgfqpoint{0.790459in}{1.426035in}}{\pgfqpoint{0.796283in}{1.431859in}}%
\pgfpathcurveto{\pgfqpoint{0.802107in}{1.437683in}}{\pgfqpoint{0.805379in}{1.445583in}}{\pgfqpoint{0.805379in}{1.453820in}}%
\pgfpathcurveto{\pgfqpoint{0.805379in}{1.462056in}}{\pgfqpoint{0.802107in}{1.469956in}}{\pgfqpoint{0.796283in}{1.475780in}}%
\pgfpathcurveto{\pgfqpoint{0.790459in}{1.481604in}}{\pgfqpoint{0.782559in}{1.484876in}}{\pgfqpoint{0.774323in}{1.484876in}}%
\pgfpathcurveto{\pgfqpoint{0.766087in}{1.484876in}}{\pgfqpoint{0.758186in}{1.481604in}}{\pgfqpoint{0.752363in}{1.475780in}}%
\pgfpathcurveto{\pgfqpoint{0.746539in}{1.469956in}}{\pgfqpoint{0.743266in}{1.462056in}}{\pgfqpoint{0.743266in}{1.453820in}}%
\pgfpathcurveto{\pgfqpoint{0.743266in}{1.445583in}}{\pgfqpoint{0.746539in}{1.437683in}}{\pgfqpoint{0.752363in}{1.431859in}}%
\pgfpathcurveto{\pgfqpoint{0.758186in}{1.426035in}}{\pgfqpoint{0.766087in}{1.422763in}}{\pgfqpoint{0.774323in}{1.422763in}}%
\pgfpathclose%
\pgfusepath{stroke,fill}%
\end{pgfscope}%
\begin{pgfscope}%
\pgfpathrectangle{\pgfqpoint{0.100000in}{0.212622in}}{\pgfqpoint{3.696000in}{3.696000in}}%
\pgfusepath{clip}%
\pgfsetbuttcap%
\pgfsetroundjoin%
\definecolor{currentfill}{rgb}{0.121569,0.466667,0.705882}%
\pgfsetfillcolor{currentfill}%
\pgfsetfillopacity{0.629471}%
\pgfsetlinewidth{1.003750pt}%
\definecolor{currentstroke}{rgb}{0.121569,0.466667,0.705882}%
\pgfsetstrokecolor{currentstroke}%
\pgfsetstrokeopacity{0.629471}%
\pgfsetdash{}{0pt}%
\pgfpathmoveto{\pgfqpoint{0.774244in}{1.422672in}}%
\pgfpathcurveto{\pgfqpoint{0.782480in}{1.422672in}}{\pgfqpoint{0.790380in}{1.425944in}}{\pgfqpoint{0.796204in}{1.431768in}}%
\pgfpathcurveto{\pgfqpoint{0.802028in}{1.437592in}}{\pgfqpoint{0.805300in}{1.445492in}}{\pgfqpoint{0.805300in}{1.453729in}}%
\pgfpathcurveto{\pgfqpoint{0.805300in}{1.461965in}}{\pgfqpoint{0.802028in}{1.469865in}}{\pgfqpoint{0.796204in}{1.475689in}}%
\pgfpathcurveto{\pgfqpoint{0.790380in}{1.481513in}}{\pgfqpoint{0.782480in}{1.484785in}}{\pgfqpoint{0.774244in}{1.484785in}}%
\pgfpathcurveto{\pgfqpoint{0.766007in}{1.484785in}}{\pgfqpoint{0.758107in}{1.481513in}}{\pgfqpoint{0.752283in}{1.475689in}}%
\pgfpathcurveto{\pgfqpoint{0.746459in}{1.469865in}}{\pgfqpoint{0.743187in}{1.461965in}}{\pgfqpoint{0.743187in}{1.453729in}}%
\pgfpathcurveto{\pgfqpoint{0.743187in}{1.445492in}}{\pgfqpoint{0.746459in}{1.437592in}}{\pgfqpoint{0.752283in}{1.431768in}}%
\pgfpathcurveto{\pgfqpoint{0.758107in}{1.425944in}}{\pgfqpoint{0.766007in}{1.422672in}}{\pgfqpoint{0.774244in}{1.422672in}}%
\pgfpathclose%
\pgfusepath{stroke,fill}%
\end{pgfscope}%
\begin{pgfscope}%
\pgfpathrectangle{\pgfqpoint{0.100000in}{0.212622in}}{\pgfqpoint{3.696000in}{3.696000in}}%
\pgfusepath{clip}%
\pgfsetbuttcap%
\pgfsetroundjoin%
\definecolor{currentfill}{rgb}{0.121569,0.466667,0.705882}%
\pgfsetfillcolor{currentfill}%
\pgfsetfillopacity{0.629543}%
\pgfsetlinewidth{1.003750pt}%
\definecolor{currentstroke}{rgb}{0.121569,0.466667,0.705882}%
\pgfsetstrokecolor{currentstroke}%
\pgfsetstrokeopacity{0.629543}%
\pgfsetdash{}{0pt}%
\pgfpathmoveto{\pgfqpoint{0.774096in}{1.422604in}}%
\pgfpathcurveto{\pgfqpoint{0.782333in}{1.422604in}}{\pgfqpoint{0.790233in}{1.425876in}}{\pgfqpoint{0.796057in}{1.431700in}}%
\pgfpathcurveto{\pgfqpoint{0.801881in}{1.437524in}}{\pgfqpoint{0.805153in}{1.445424in}}{\pgfqpoint{0.805153in}{1.453660in}}%
\pgfpathcurveto{\pgfqpoint{0.805153in}{1.461897in}}{\pgfqpoint{0.801881in}{1.469797in}}{\pgfqpoint{0.796057in}{1.475621in}}%
\pgfpathcurveto{\pgfqpoint{0.790233in}{1.481445in}}{\pgfqpoint{0.782333in}{1.484717in}}{\pgfqpoint{0.774096in}{1.484717in}}%
\pgfpathcurveto{\pgfqpoint{0.765860in}{1.484717in}}{\pgfqpoint{0.757960in}{1.481445in}}{\pgfqpoint{0.752136in}{1.475621in}}%
\pgfpathcurveto{\pgfqpoint{0.746312in}{1.469797in}}{\pgfqpoint{0.743040in}{1.461897in}}{\pgfqpoint{0.743040in}{1.453660in}}%
\pgfpathcurveto{\pgfqpoint{0.743040in}{1.445424in}}{\pgfqpoint{0.746312in}{1.437524in}}{\pgfqpoint{0.752136in}{1.431700in}}%
\pgfpathcurveto{\pgfqpoint{0.757960in}{1.425876in}}{\pgfqpoint{0.765860in}{1.422604in}}{\pgfqpoint{0.774096in}{1.422604in}}%
\pgfpathclose%
\pgfusepath{stroke,fill}%
\end{pgfscope}%
\begin{pgfscope}%
\pgfpathrectangle{\pgfqpoint{0.100000in}{0.212622in}}{\pgfqpoint{3.696000in}{3.696000in}}%
\pgfusepath{clip}%
\pgfsetbuttcap%
\pgfsetroundjoin%
\definecolor{currentfill}{rgb}{0.121569,0.466667,0.705882}%
\pgfsetfillcolor{currentfill}%
\pgfsetfillopacity{0.629653}%
\pgfsetlinewidth{1.003750pt}%
\definecolor{currentstroke}{rgb}{0.121569,0.466667,0.705882}%
\pgfsetstrokecolor{currentstroke}%
\pgfsetstrokeopacity{0.629653}%
\pgfsetdash{}{0pt}%
\pgfpathmoveto{\pgfqpoint{0.773816in}{1.422402in}}%
\pgfpathcurveto{\pgfqpoint{0.782052in}{1.422402in}}{\pgfqpoint{0.789952in}{1.425674in}}{\pgfqpoint{0.795776in}{1.431498in}}%
\pgfpathcurveto{\pgfqpoint{0.801600in}{1.437322in}}{\pgfqpoint{0.804872in}{1.445222in}}{\pgfqpoint{0.804872in}{1.453458in}}%
\pgfpathcurveto{\pgfqpoint{0.804872in}{1.461694in}}{\pgfqpoint{0.801600in}{1.469594in}}{\pgfqpoint{0.795776in}{1.475418in}}%
\pgfpathcurveto{\pgfqpoint{0.789952in}{1.481242in}}{\pgfqpoint{0.782052in}{1.484515in}}{\pgfqpoint{0.773816in}{1.484515in}}%
\pgfpathcurveto{\pgfqpoint{0.765580in}{1.484515in}}{\pgfqpoint{0.757680in}{1.481242in}}{\pgfqpoint{0.751856in}{1.475418in}}%
\pgfpathcurveto{\pgfqpoint{0.746032in}{1.469594in}}{\pgfqpoint{0.742759in}{1.461694in}}{\pgfqpoint{0.742759in}{1.453458in}}%
\pgfpathcurveto{\pgfqpoint{0.742759in}{1.445222in}}{\pgfqpoint{0.746032in}{1.437322in}}{\pgfqpoint{0.751856in}{1.431498in}}%
\pgfpathcurveto{\pgfqpoint{0.757680in}{1.425674in}}{\pgfqpoint{0.765580in}{1.422402in}}{\pgfqpoint{0.773816in}{1.422402in}}%
\pgfpathclose%
\pgfusepath{stroke,fill}%
\end{pgfscope}%
\begin{pgfscope}%
\pgfpathrectangle{\pgfqpoint{0.100000in}{0.212622in}}{\pgfqpoint{3.696000in}{3.696000in}}%
\pgfusepath{clip}%
\pgfsetbuttcap%
\pgfsetroundjoin%
\definecolor{currentfill}{rgb}{0.121569,0.466667,0.705882}%
\pgfsetfillcolor{currentfill}%
\pgfsetfillopacity{0.629855}%
\pgfsetlinewidth{1.003750pt}%
\definecolor{currentstroke}{rgb}{0.121569,0.466667,0.705882}%
\pgfsetstrokecolor{currentstroke}%
\pgfsetstrokeopacity{0.629855}%
\pgfsetdash{}{0pt}%
\pgfpathmoveto{\pgfqpoint{0.780536in}{1.389622in}}%
\pgfpathcurveto{\pgfqpoint{0.788772in}{1.389622in}}{\pgfqpoint{0.796673in}{1.392894in}}{\pgfqpoint{0.802496in}{1.398718in}}%
\pgfpathcurveto{\pgfqpoint{0.808320in}{1.404542in}}{\pgfqpoint{0.811593in}{1.412442in}}{\pgfqpoint{0.811593in}{1.420679in}}%
\pgfpathcurveto{\pgfqpoint{0.811593in}{1.428915in}}{\pgfqpoint{0.808320in}{1.436815in}}{\pgfqpoint{0.802496in}{1.442639in}}%
\pgfpathcurveto{\pgfqpoint{0.796673in}{1.448463in}}{\pgfqpoint{0.788772in}{1.451735in}}{\pgfqpoint{0.780536in}{1.451735in}}%
\pgfpathcurveto{\pgfqpoint{0.772300in}{1.451735in}}{\pgfqpoint{0.764400in}{1.448463in}}{\pgfqpoint{0.758576in}{1.442639in}}%
\pgfpathcurveto{\pgfqpoint{0.752752in}{1.436815in}}{\pgfqpoint{0.749480in}{1.428915in}}{\pgfqpoint{0.749480in}{1.420679in}}%
\pgfpathcurveto{\pgfqpoint{0.749480in}{1.412442in}}{\pgfqpoint{0.752752in}{1.404542in}}{\pgfqpoint{0.758576in}{1.398718in}}%
\pgfpathcurveto{\pgfqpoint{0.764400in}{1.392894in}}{\pgfqpoint{0.772300in}{1.389622in}}{\pgfqpoint{0.780536in}{1.389622in}}%
\pgfpathclose%
\pgfusepath{stroke,fill}%
\end{pgfscope}%
\begin{pgfscope}%
\pgfpathrectangle{\pgfqpoint{0.100000in}{0.212622in}}{\pgfqpoint{3.696000in}{3.696000in}}%
\pgfusepath{clip}%
\pgfsetbuttcap%
\pgfsetroundjoin%
\definecolor{currentfill}{rgb}{0.121569,0.466667,0.705882}%
\pgfsetfillcolor{currentfill}%
\pgfsetfillopacity{0.629897}%
\pgfsetlinewidth{1.003750pt}%
\definecolor{currentstroke}{rgb}{0.121569,0.466667,0.705882}%
\pgfsetstrokecolor{currentstroke}%
\pgfsetstrokeopacity{0.629897}%
\pgfsetdash{}{0pt}%
\pgfpathmoveto{\pgfqpoint{0.773338in}{1.422179in}}%
\pgfpathcurveto{\pgfqpoint{0.781574in}{1.422179in}}{\pgfqpoint{0.789474in}{1.425452in}}{\pgfqpoint{0.795298in}{1.431276in}}%
\pgfpathcurveto{\pgfqpoint{0.801122in}{1.437100in}}{\pgfqpoint{0.804394in}{1.445000in}}{\pgfqpoint{0.804394in}{1.453236in}}%
\pgfpathcurveto{\pgfqpoint{0.804394in}{1.461472in}}{\pgfqpoint{0.801122in}{1.469372in}}{\pgfqpoint{0.795298in}{1.475196in}}%
\pgfpathcurveto{\pgfqpoint{0.789474in}{1.481020in}}{\pgfqpoint{0.781574in}{1.484292in}}{\pgfqpoint{0.773338in}{1.484292in}}%
\pgfpathcurveto{\pgfqpoint{0.765102in}{1.484292in}}{\pgfqpoint{0.757201in}{1.481020in}}{\pgfqpoint{0.751378in}{1.475196in}}%
\pgfpathcurveto{\pgfqpoint{0.745554in}{1.469372in}}{\pgfqpoint{0.742281in}{1.461472in}}{\pgfqpoint{0.742281in}{1.453236in}}%
\pgfpathcurveto{\pgfqpoint{0.742281in}{1.445000in}}{\pgfqpoint{0.745554in}{1.437100in}}{\pgfqpoint{0.751378in}{1.431276in}}%
\pgfpathcurveto{\pgfqpoint{0.757201in}{1.425452in}}{\pgfqpoint{0.765102in}{1.422179in}}{\pgfqpoint{0.773338in}{1.422179in}}%
\pgfpathclose%
\pgfusepath{stroke,fill}%
\end{pgfscope}%
\begin{pgfscope}%
\pgfpathrectangle{\pgfqpoint{0.100000in}{0.212622in}}{\pgfqpoint{3.696000in}{3.696000in}}%
\pgfusepath{clip}%
\pgfsetbuttcap%
\pgfsetroundjoin%
\definecolor{currentfill}{rgb}{0.121569,0.466667,0.705882}%
\pgfsetfillcolor{currentfill}%
\pgfsetfillopacity{0.630263}%
\pgfsetlinewidth{1.003750pt}%
\definecolor{currentstroke}{rgb}{0.121569,0.466667,0.705882}%
\pgfsetstrokecolor{currentstroke}%
\pgfsetstrokeopacity{0.630263}%
\pgfsetdash{}{0pt}%
\pgfpathmoveto{\pgfqpoint{0.772484in}{1.421421in}}%
\pgfpathcurveto{\pgfqpoint{0.780720in}{1.421421in}}{\pgfqpoint{0.788620in}{1.424693in}}{\pgfqpoint{0.794444in}{1.430517in}}%
\pgfpathcurveto{\pgfqpoint{0.800268in}{1.436341in}}{\pgfqpoint{0.803540in}{1.444241in}}{\pgfqpoint{0.803540in}{1.452477in}}%
\pgfpathcurveto{\pgfqpoint{0.803540in}{1.460713in}}{\pgfqpoint{0.800268in}{1.468614in}}{\pgfqpoint{0.794444in}{1.474437in}}%
\pgfpathcurveto{\pgfqpoint{0.788620in}{1.480261in}}{\pgfqpoint{0.780720in}{1.483534in}}{\pgfqpoint{0.772484in}{1.483534in}}%
\pgfpathcurveto{\pgfqpoint{0.764247in}{1.483534in}}{\pgfqpoint{0.756347in}{1.480261in}}{\pgfqpoint{0.750523in}{1.474437in}}%
\pgfpathcurveto{\pgfqpoint{0.744700in}{1.468614in}}{\pgfqpoint{0.741427in}{1.460713in}}{\pgfqpoint{0.741427in}{1.452477in}}%
\pgfpathcurveto{\pgfqpoint{0.741427in}{1.444241in}}{\pgfqpoint{0.744700in}{1.436341in}}{\pgfqpoint{0.750523in}{1.430517in}}%
\pgfpathcurveto{\pgfqpoint{0.756347in}{1.424693in}}{\pgfqpoint{0.764247in}{1.421421in}}{\pgfqpoint{0.772484in}{1.421421in}}%
\pgfpathclose%
\pgfusepath{stroke,fill}%
\end{pgfscope}%
\begin{pgfscope}%
\pgfpathrectangle{\pgfqpoint{0.100000in}{0.212622in}}{\pgfqpoint{3.696000in}{3.696000in}}%
\pgfusepath{clip}%
\pgfsetbuttcap%
\pgfsetroundjoin%
\definecolor{currentfill}{rgb}{0.121569,0.466667,0.705882}%
\pgfsetfillcolor{currentfill}%
\pgfsetfillopacity{0.630957}%
\pgfsetlinewidth{1.003750pt}%
\definecolor{currentstroke}{rgb}{0.121569,0.466667,0.705882}%
\pgfsetstrokecolor{currentstroke}%
\pgfsetstrokeopacity{0.630957}%
\pgfsetdash{}{0pt}%
\pgfpathmoveto{\pgfqpoint{0.770783in}{1.420377in}}%
\pgfpathcurveto{\pgfqpoint{0.779020in}{1.420377in}}{\pgfqpoint{0.786920in}{1.423649in}}{\pgfqpoint{0.792744in}{1.429473in}}%
\pgfpathcurveto{\pgfqpoint{0.798568in}{1.435297in}}{\pgfqpoint{0.801840in}{1.443197in}}{\pgfqpoint{0.801840in}{1.451434in}}%
\pgfpathcurveto{\pgfqpoint{0.801840in}{1.459670in}}{\pgfqpoint{0.798568in}{1.467570in}}{\pgfqpoint{0.792744in}{1.473394in}}%
\pgfpathcurveto{\pgfqpoint{0.786920in}{1.479218in}}{\pgfqpoint{0.779020in}{1.482490in}}{\pgfqpoint{0.770783in}{1.482490in}}%
\pgfpathcurveto{\pgfqpoint{0.762547in}{1.482490in}}{\pgfqpoint{0.754647in}{1.479218in}}{\pgfqpoint{0.748823in}{1.473394in}}%
\pgfpathcurveto{\pgfqpoint{0.742999in}{1.467570in}}{\pgfqpoint{0.739727in}{1.459670in}}{\pgfqpoint{0.739727in}{1.451434in}}%
\pgfpathcurveto{\pgfqpoint{0.739727in}{1.443197in}}{\pgfqpoint{0.742999in}{1.435297in}}{\pgfqpoint{0.748823in}{1.429473in}}%
\pgfpathcurveto{\pgfqpoint{0.754647in}{1.423649in}}{\pgfqpoint{0.762547in}{1.420377in}}{\pgfqpoint{0.770783in}{1.420377in}}%
\pgfpathclose%
\pgfusepath{stroke,fill}%
\end{pgfscope}%
\begin{pgfscope}%
\pgfpathrectangle{\pgfqpoint{0.100000in}{0.212622in}}{\pgfqpoint{3.696000in}{3.696000in}}%
\pgfusepath{clip}%
\pgfsetbuttcap%
\pgfsetroundjoin%
\definecolor{currentfill}{rgb}{0.121569,0.466667,0.705882}%
\pgfsetfillcolor{currentfill}%
\pgfsetfillopacity{0.631052}%
\pgfsetlinewidth{1.003750pt}%
\definecolor{currentstroke}{rgb}{0.121569,0.466667,0.705882}%
\pgfsetstrokecolor{currentstroke}%
\pgfsetstrokeopacity{0.631052}%
\pgfsetdash{}{0pt}%
\pgfpathmoveto{\pgfqpoint{0.779027in}{1.391055in}}%
\pgfpathcurveto{\pgfqpoint{0.787264in}{1.391055in}}{\pgfqpoint{0.795164in}{1.394327in}}{\pgfqpoint{0.800988in}{1.400151in}}%
\pgfpathcurveto{\pgfqpoint{0.806811in}{1.405975in}}{\pgfqpoint{0.810084in}{1.413875in}}{\pgfqpoint{0.810084in}{1.422111in}}%
\pgfpathcurveto{\pgfqpoint{0.810084in}{1.430347in}}{\pgfqpoint{0.806811in}{1.438247in}}{\pgfqpoint{0.800988in}{1.444071in}}%
\pgfpathcurveto{\pgfqpoint{0.795164in}{1.449895in}}{\pgfqpoint{0.787264in}{1.453168in}}{\pgfqpoint{0.779027in}{1.453168in}}%
\pgfpathcurveto{\pgfqpoint{0.770791in}{1.453168in}}{\pgfqpoint{0.762891in}{1.449895in}}{\pgfqpoint{0.757067in}{1.444071in}}%
\pgfpathcurveto{\pgfqpoint{0.751243in}{1.438247in}}{\pgfqpoint{0.747971in}{1.430347in}}{\pgfqpoint{0.747971in}{1.422111in}}%
\pgfpathcurveto{\pgfqpoint{0.747971in}{1.413875in}}{\pgfqpoint{0.751243in}{1.405975in}}{\pgfqpoint{0.757067in}{1.400151in}}%
\pgfpathcurveto{\pgfqpoint{0.762891in}{1.394327in}}{\pgfqpoint{0.770791in}{1.391055in}}{\pgfqpoint{0.779027in}{1.391055in}}%
\pgfpathclose%
\pgfusepath{stroke,fill}%
\end{pgfscope}%
\begin{pgfscope}%
\pgfpathrectangle{\pgfqpoint{0.100000in}{0.212622in}}{\pgfqpoint{3.696000in}{3.696000in}}%
\pgfusepath{clip}%
\pgfsetbuttcap%
\pgfsetroundjoin%
\definecolor{currentfill}{rgb}{0.121569,0.466667,0.705882}%
\pgfsetfillcolor{currentfill}%
\pgfsetfillopacity{0.632141}%
\pgfsetlinewidth{1.003750pt}%
\definecolor{currentstroke}{rgb}{0.121569,0.466667,0.705882}%
\pgfsetstrokecolor{currentstroke}%
\pgfsetstrokeopacity{0.632141}%
\pgfsetdash{}{0pt}%
\pgfpathmoveto{\pgfqpoint{1.611876in}{1.688544in}}%
\pgfpathcurveto{\pgfqpoint{1.620113in}{1.688544in}}{\pgfqpoint{1.628013in}{1.691816in}}{\pgfqpoint{1.633837in}{1.697640in}}%
\pgfpathcurveto{\pgfqpoint{1.639661in}{1.703464in}}{\pgfqpoint{1.642933in}{1.711364in}}{\pgfqpoint{1.642933in}{1.719600in}}%
\pgfpathcurveto{\pgfqpoint{1.642933in}{1.727837in}}{\pgfqpoint{1.639661in}{1.735737in}}{\pgfqpoint{1.633837in}{1.741561in}}%
\pgfpathcurveto{\pgfqpoint{1.628013in}{1.747385in}}{\pgfqpoint{1.620113in}{1.750657in}}{\pgfqpoint{1.611876in}{1.750657in}}%
\pgfpathcurveto{\pgfqpoint{1.603640in}{1.750657in}}{\pgfqpoint{1.595740in}{1.747385in}}{\pgfqpoint{1.589916in}{1.741561in}}%
\pgfpathcurveto{\pgfqpoint{1.584092in}{1.735737in}}{\pgfqpoint{1.580820in}{1.727837in}}{\pgfqpoint{1.580820in}{1.719600in}}%
\pgfpathcurveto{\pgfqpoint{1.580820in}{1.711364in}}{\pgfqpoint{1.584092in}{1.703464in}}{\pgfqpoint{1.589916in}{1.697640in}}%
\pgfpathcurveto{\pgfqpoint{1.595740in}{1.691816in}}{\pgfqpoint{1.603640in}{1.688544in}}{\pgfqpoint{1.611876in}{1.688544in}}%
\pgfpathclose%
\pgfusepath{stroke,fill}%
\end{pgfscope}%
\begin{pgfscope}%
\pgfpathrectangle{\pgfqpoint{0.100000in}{0.212622in}}{\pgfqpoint{3.696000in}{3.696000in}}%
\pgfusepath{clip}%
\pgfsetbuttcap%
\pgfsetroundjoin%
\definecolor{currentfill}{rgb}{0.121569,0.466667,0.705882}%
\pgfsetfillcolor{currentfill}%
\pgfsetfillopacity{0.632657}%
\pgfsetlinewidth{1.003750pt}%
\definecolor{currentstroke}{rgb}{0.121569,0.466667,0.705882}%
\pgfsetstrokecolor{currentstroke}%
\pgfsetstrokeopacity{0.632657}%
\pgfsetdash{}{0pt}%
\pgfpathmoveto{\pgfqpoint{0.767645in}{1.420418in}}%
\pgfpathcurveto{\pgfqpoint{0.775881in}{1.420418in}}{\pgfqpoint{0.783781in}{1.423690in}}{\pgfqpoint{0.789605in}{1.429514in}}%
\pgfpathcurveto{\pgfqpoint{0.795429in}{1.435338in}}{\pgfqpoint{0.798702in}{1.443238in}}{\pgfqpoint{0.798702in}{1.451475in}}%
\pgfpathcurveto{\pgfqpoint{0.798702in}{1.459711in}}{\pgfqpoint{0.795429in}{1.467611in}}{\pgfqpoint{0.789605in}{1.473435in}}%
\pgfpathcurveto{\pgfqpoint{0.783781in}{1.479259in}}{\pgfqpoint{0.775881in}{1.482531in}}{\pgfqpoint{0.767645in}{1.482531in}}%
\pgfpathcurveto{\pgfqpoint{0.759409in}{1.482531in}}{\pgfqpoint{0.751509in}{1.479259in}}{\pgfqpoint{0.745685in}{1.473435in}}%
\pgfpathcurveto{\pgfqpoint{0.739861in}{1.467611in}}{\pgfqpoint{0.736589in}{1.459711in}}{\pgfqpoint{0.736589in}{1.451475in}}%
\pgfpathcurveto{\pgfqpoint{0.736589in}{1.443238in}}{\pgfqpoint{0.739861in}{1.435338in}}{\pgfqpoint{0.745685in}{1.429514in}}%
\pgfpathcurveto{\pgfqpoint{0.751509in}{1.423690in}}{\pgfqpoint{0.759409in}{1.420418in}}{\pgfqpoint{0.767645in}{1.420418in}}%
\pgfpathclose%
\pgfusepath{stroke,fill}%
\end{pgfscope}%
\begin{pgfscope}%
\pgfpathrectangle{\pgfqpoint{0.100000in}{0.212622in}}{\pgfqpoint{3.696000in}{3.696000in}}%
\pgfusepath{clip}%
\pgfsetbuttcap%
\pgfsetroundjoin%
\definecolor{currentfill}{rgb}{0.121569,0.466667,0.705882}%
\pgfsetfillcolor{currentfill}%
\pgfsetfillopacity{0.632784}%
\pgfsetlinewidth{1.003750pt}%
\definecolor{currentstroke}{rgb}{0.121569,0.466667,0.705882}%
\pgfsetstrokecolor{currentstroke}%
\pgfsetstrokeopacity{0.632784}%
\pgfsetdash{}{0pt}%
\pgfpathmoveto{\pgfqpoint{0.777427in}{1.392990in}}%
\pgfpathcurveto{\pgfqpoint{0.785663in}{1.392990in}}{\pgfqpoint{0.793563in}{1.396262in}}{\pgfqpoint{0.799387in}{1.402086in}}%
\pgfpathcurveto{\pgfqpoint{0.805211in}{1.407910in}}{\pgfqpoint{0.808483in}{1.415810in}}{\pgfqpoint{0.808483in}{1.424046in}}%
\pgfpathcurveto{\pgfqpoint{0.808483in}{1.432283in}}{\pgfqpoint{0.805211in}{1.440183in}}{\pgfqpoint{0.799387in}{1.446007in}}%
\pgfpathcurveto{\pgfqpoint{0.793563in}{1.451831in}}{\pgfqpoint{0.785663in}{1.455103in}}{\pgfqpoint{0.777427in}{1.455103in}}%
\pgfpathcurveto{\pgfqpoint{0.769190in}{1.455103in}}{\pgfqpoint{0.761290in}{1.451831in}}{\pgfqpoint{0.755466in}{1.446007in}}%
\pgfpathcurveto{\pgfqpoint{0.749642in}{1.440183in}}{\pgfqpoint{0.746370in}{1.432283in}}{\pgfqpoint{0.746370in}{1.424046in}}%
\pgfpathcurveto{\pgfqpoint{0.746370in}{1.415810in}}{\pgfqpoint{0.749642in}{1.407910in}}{\pgfqpoint{0.755466in}{1.402086in}}%
\pgfpathcurveto{\pgfqpoint{0.761290in}{1.396262in}}{\pgfqpoint{0.769190in}{1.392990in}}{\pgfqpoint{0.777427in}{1.392990in}}%
\pgfpathclose%
\pgfusepath{stroke,fill}%
\end{pgfscope}%
\begin{pgfscope}%
\pgfpathrectangle{\pgfqpoint{0.100000in}{0.212622in}}{\pgfqpoint{3.696000in}{3.696000in}}%
\pgfusepath{clip}%
\pgfsetbuttcap%
\pgfsetroundjoin%
\definecolor{currentfill}{rgb}{0.121569,0.466667,0.705882}%
\pgfsetfillcolor{currentfill}%
\pgfsetfillopacity{0.633425}%
\pgfsetlinewidth{1.003750pt}%
\definecolor{currentstroke}{rgb}{0.121569,0.466667,0.705882}%
\pgfsetstrokecolor{currentstroke}%
\pgfsetstrokeopacity{0.633425}%
\pgfsetdash{}{0pt}%
\pgfpathmoveto{\pgfqpoint{0.765123in}{1.417244in}}%
\pgfpathcurveto{\pgfqpoint{0.773359in}{1.417244in}}{\pgfqpoint{0.781259in}{1.420516in}}{\pgfqpoint{0.787083in}{1.426340in}}%
\pgfpathcurveto{\pgfqpoint{0.792907in}{1.432164in}}{\pgfqpoint{0.796179in}{1.440064in}}{\pgfqpoint{0.796179in}{1.448300in}}%
\pgfpathcurveto{\pgfqpoint{0.796179in}{1.456536in}}{\pgfqpoint{0.792907in}{1.464436in}}{\pgfqpoint{0.787083in}{1.470260in}}%
\pgfpathcurveto{\pgfqpoint{0.781259in}{1.476084in}}{\pgfqpoint{0.773359in}{1.479357in}}{\pgfqpoint{0.765123in}{1.479357in}}%
\pgfpathcurveto{\pgfqpoint{0.756886in}{1.479357in}}{\pgfqpoint{0.748986in}{1.476084in}}{\pgfqpoint{0.743162in}{1.470260in}}%
\pgfpathcurveto{\pgfqpoint{0.737338in}{1.464436in}}{\pgfqpoint{0.734066in}{1.456536in}}{\pgfqpoint{0.734066in}{1.448300in}}%
\pgfpathcurveto{\pgfqpoint{0.734066in}{1.440064in}}{\pgfqpoint{0.737338in}{1.432164in}}{\pgfqpoint{0.743162in}{1.426340in}}%
\pgfpathcurveto{\pgfqpoint{0.748986in}{1.420516in}}{\pgfqpoint{0.756886in}{1.417244in}}{\pgfqpoint{0.765123in}{1.417244in}}%
\pgfpathclose%
\pgfusepath{stroke,fill}%
\end{pgfscope}%
\begin{pgfscope}%
\pgfpathrectangle{\pgfqpoint{0.100000in}{0.212622in}}{\pgfqpoint{3.696000in}{3.696000in}}%
\pgfusepath{clip}%
\pgfsetbuttcap%
\pgfsetroundjoin%
\definecolor{currentfill}{rgb}{0.121569,0.466667,0.705882}%
\pgfsetfillcolor{currentfill}%
\pgfsetfillopacity{0.634482}%
\pgfsetlinewidth{1.003750pt}%
\definecolor{currentstroke}{rgb}{0.121569,0.466667,0.705882}%
\pgfsetstrokecolor{currentstroke}%
\pgfsetstrokeopacity{0.634482}%
\pgfsetdash{}{0pt}%
\pgfpathmoveto{\pgfqpoint{0.762805in}{1.416726in}}%
\pgfpathcurveto{\pgfqpoint{0.771042in}{1.416726in}}{\pgfqpoint{0.778942in}{1.419999in}}{\pgfqpoint{0.784766in}{1.425822in}}%
\pgfpathcurveto{\pgfqpoint{0.790590in}{1.431646in}}{\pgfqpoint{0.793862in}{1.439546in}}{\pgfqpoint{0.793862in}{1.447783in}}%
\pgfpathcurveto{\pgfqpoint{0.793862in}{1.456019in}}{\pgfqpoint{0.790590in}{1.463919in}}{\pgfqpoint{0.784766in}{1.469743in}}%
\pgfpathcurveto{\pgfqpoint{0.778942in}{1.475567in}}{\pgfqpoint{0.771042in}{1.478839in}}{\pgfqpoint{0.762805in}{1.478839in}}%
\pgfpathcurveto{\pgfqpoint{0.754569in}{1.478839in}}{\pgfqpoint{0.746669in}{1.475567in}}{\pgfqpoint{0.740845in}{1.469743in}}%
\pgfpathcurveto{\pgfqpoint{0.735021in}{1.463919in}}{\pgfqpoint{0.731749in}{1.456019in}}{\pgfqpoint{0.731749in}{1.447783in}}%
\pgfpathcurveto{\pgfqpoint{0.731749in}{1.439546in}}{\pgfqpoint{0.735021in}{1.431646in}}{\pgfqpoint{0.740845in}{1.425822in}}%
\pgfpathcurveto{\pgfqpoint{0.746669in}{1.419999in}}{\pgfqpoint{0.754569in}{1.416726in}}{\pgfqpoint{0.762805in}{1.416726in}}%
\pgfpathclose%
\pgfusepath{stroke,fill}%
\end{pgfscope}%
\begin{pgfscope}%
\pgfpathrectangle{\pgfqpoint{0.100000in}{0.212622in}}{\pgfqpoint{3.696000in}{3.696000in}}%
\pgfusepath{clip}%
\pgfsetbuttcap%
\pgfsetroundjoin%
\definecolor{currentfill}{rgb}{0.121569,0.466667,0.705882}%
\pgfsetfillcolor{currentfill}%
\pgfsetfillopacity{0.634801}%
\pgfsetlinewidth{1.003750pt}%
\definecolor{currentstroke}{rgb}{0.121569,0.466667,0.705882}%
\pgfsetstrokecolor{currentstroke}%
\pgfsetstrokeopacity{0.634801}%
\pgfsetdash{}{0pt}%
\pgfpathmoveto{\pgfqpoint{0.775057in}{1.395451in}}%
\pgfpathcurveto{\pgfqpoint{0.783293in}{1.395451in}}{\pgfqpoint{0.791193in}{1.398723in}}{\pgfqpoint{0.797017in}{1.404547in}}%
\pgfpathcurveto{\pgfqpoint{0.802841in}{1.410371in}}{\pgfqpoint{0.806113in}{1.418271in}}{\pgfqpoint{0.806113in}{1.426507in}}%
\pgfpathcurveto{\pgfqpoint{0.806113in}{1.434744in}}{\pgfqpoint{0.802841in}{1.442644in}}{\pgfqpoint{0.797017in}{1.448468in}}%
\pgfpathcurveto{\pgfqpoint{0.791193in}{1.454292in}}{\pgfqpoint{0.783293in}{1.457564in}}{\pgfqpoint{0.775057in}{1.457564in}}%
\pgfpathcurveto{\pgfqpoint{0.766820in}{1.457564in}}{\pgfqpoint{0.758920in}{1.454292in}}{\pgfqpoint{0.753096in}{1.448468in}}%
\pgfpathcurveto{\pgfqpoint{0.747272in}{1.442644in}}{\pgfqpoint{0.744000in}{1.434744in}}{\pgfqpoint{0.744000in}{1.426507in}}%
\pgfpathcurveto{\pgfqpoint{0.744000in}{1.418271in}}{\pgfqpoint{0.747272in}{1.410371in}}{\pgfqpoint{0.753096in}{1.404547in}}%
\pgfpathcurveto{\pgfqpoint{0.758920in}{1.398723in}}{\pgfqpoint{0.766820in}{1.395451in}}{\pgfqpoint{0.775057in}{1.395451in}}%
\pgfpathclose%
\pgfusepath{stroke,fill}%
\end{pgfscope}%
\begin{pgfscope}%
\pgfpathrectangle{\pgfqpoint{0.100000in}{0.212622in}}{\pgfqpoint{3.696000in}{3.696000in}}%
\pgfusepath{clip}%
\pgfsetbuttcap%
\pgfsetroundjoin%
\definecolor{currentfill}{rgb}{0.121569,0.466667,0.705882}%
\pgfsetfillcolor{currentfill}%
\pgfsetfillopacity{0.635374}%
\pgfsetlinewidth{1.003750pt}%
\definecolor{currentstroke}{rgb}{0.121569,0.466667,0.705882}%
\pgfsetstrokecolor{currentstroke}%
\pgfsetstrokeopacity{0.635374}%
\pgfsetdash{}{0pt}%
\pgfpathmoveto{\pgfqpoint{0.761008in}{1.416159in}}%
\pgfpathcurveto{\pgfqpoint{0.769244in}{1.416159in}}{\pgfqpoint{0.777144in}{1.419432in}}{\pgfqpoint{0.782968in}{1.425255in}}%
\pgfpathcurveto{\pgfqpoint{0.788792in}{1.431079in}}{\pgfqpoint{0.792065in}{1.438979in}}{\pgfqpoint{0.792065in}{1.447216in}}%
\pgfpathcurveto{\pgfqpoint{0.792065in}{1.455452in}}{\pgfqpoint{0.788792in}{1.463352in}}{\pgfqpoint{0.782968in}{1.469176in}}%
\pgfpathcurveto{\pgfqpoint{0.777144in}{1.475000in}}{\pgfqpoint{0.769244in}{1.478272in}}{\pgfqpoint{0.761008in}{1.478272in}}%
\pgfpathcurveto{\pgfqpoint{0.752772in}{1.478272in}}{\pgfqpoint{0.744872in}{1.475000in}}{\pgfqpoint{0.739048in}{1.469176in}}%
\pgfpathcurveto{\pgfqpoint{0.733224in}{1.463352in}}{\pgfqpoint{0.729952in}{1.455452in}}{\pgfqpoint{0.729952in}{1.447216in}}%
\pgfpathcurveto{\pgfqpoint{0.729952in}{1.438979in}}{\pgfqpoint{0.733224in}{1.431079in}}{\pgfqpoint{0.739048in}{1.425255in}}%
\pgfpathcurveto{\pgfqpoint{0.744872in}{1.419432in}}{\pgfqpoint{0.752772in}{1.416159in}}{\pgfqpoint{0.761008in}{1.416159in}}%
\pgfpathclose%
\pgfusepath{stroke,fill}%
\end{pgfscope}%
\begin{pgfscope}%
\pgfpathrectangle{\pgfqpoint{0.100000in}{0.212622in}}{\pgfqpoint{3.696000in}{3.696000in}}%
\pgfusepath{clip}%
\pgfsetbuttcap%
\pgfsetroundjoin%
\definecolor{currentfill}{rgb}{0.121569,0.466667,0.705882}%
\pgfsetfillcolor{currentfill}%
\pgfsetfillopacity{0.636030}%
\pgfsetlinewidth{1.003750pt}%
\definecolor{currentstroke}{rgb}{0.121569,0.466667,0.705882}%
\pgfsetstrokecolor{currentstroke}%
\pgfsetstrokeopacity{0.636030}%
\pgfsetdash{}{0pt}%
\pgfpathmoveto{\pgfqpoint{0.759843in}{1.416256in}}%
\pgfpathcurveto{\pgfqpoint{0.768079in}{1.416256in}}{\pgfqpoint{0.775980in}{1.419529in}}{\pgfqpoint{0.781803in}{1.425353in}}%
\pgfpathcurveto{\pgfqpoint{0.787627in}{1.431177in}}{\pgfqpoint{0.790900in}{1.439077in}}{\pgfqpoint{0.790900in}{1.447313in}}%
\pgfpathcurveto{\pgfqpoint{0.790900in}{1.455549in}}{\pgfqpoint{0.787627in}{1.463449in}}{\pgfqpoint{0.781803in}{1.469273in}}%
\pgfpathcurveto{\pgfqpoint{0.775980in}{1.475097in}}{\pgfqpoint{0.768079in}{1.478369in}}{\pgfqpoint{0.759843in}{1.478369in}}%
\pgfpathcurveto{\pgfqpoint{0.751607in}{1.478369in}}{\pgfqpoint{0.743707in}{1.475097in}}{\pgfqpoint{0.737883in}{1.469273in}}%
\pgfpathcurveto{\pgfqpoint{0.732059in}{1.463449in}}{\pgfqpoint{0.728787in}{1.455549in}}{\pgfqpoint{0.728787in}{1.447313in}}%
\pgfpathcurveto{\pgfqpoint{0.728787in}{1.439077in}}{\pgfqpoint{0.732059in}{1.431177in}}{\pgfqpoint{0.737883in}{1.425353in}}%
\pgfpathcurveto{\pgfqpoint{0.743707in}{1.419529in}}{\pgfqpoint{0.751607in}{1.416256in}}{\pgfqpoint{0.759843in}{1.416256in}}%
\pgfpathclose%
\pgfusepath{stroke,fill}%
\end{pgfscope}%
\begin{pgfscope}%
\pgfpathrectangle{\pgfqpoint{0.100000in}{0.212622in}}{\pgfqpoint{3.696000in}{3.696000in}}%
\pgfusepath{clip}%
\pgfsetbuttcap%
\pgfsetroundjoin%
\definecolor{currentfill}{rgb}{0.121569,0.466667,0.705882}%
\pgfsetfillcolor{currentfill}%
\pgfsetfillopacity{0.636645}%
\pgfsetlinewidth{1.003750pt}%
\definecolor{currentstroke}{rgb}{0.121569,0.466667,0.705882}%
\pgfsetstrokecolor{currentstroke}%
\pgfsetstrokeopacity{0.636645}%
\pgfsetdash{}{0pt}%
\pgfpathmoveto{\pgfqpoint{1.614397in}{1.689559in}}%
\pgfpathcurveto{\pgfqpoint{1.622633in}{1.689559in}}{\pgfqpoint{1.630533in}{1.692831in}}{\pgfqpoint{1.636357in}{1.698655in}}%
\pgfpathcurveto{\pgfqpoint{1.642181in}{1.704479in}}{\pgfqpoint{1.645453in}{1.712379in}}{\pgfqpoint{1.645453in}{1.720615in}}%
\pgfpathcurveto{\pgfqpoint{1.645453in}{1.728852in}}{\pgfqpoint{1.642181in}{1.736752in}}{\pgfqpoint{1.636357in}{1.742576in}}%
\pgfpathcurveto{\pgfqpoint{1.630533in}{1.748400in}}{\pgfqpoint{1.622633in}{1.751672in}}{\pgfqpoint{1.614397in}{1.751672in}}%
\pgfpathcurveto{\pgfqpoint{1.606161in}{1.751672in}}{\pgfqpoint{1.598261in}{1.748400in}}{\pgfqpoint{1.592437in}{1.742576in}}%
\pgfpathcurveto{\pgfqpoint{1.586613in}{1.736752in}}{\pgfqpoint{1.583340in}{1.728852in}}{\pgfqpoint{1.583340in}{1.720615in}}%
\pgfpathcurveto{\pgfqpoint{1.583340in}{1.712379in}}{\pgfqpoint{1.586613in}{1.704479in}}{\pgfqpoint{1.592437in}{1.698655in}}%
\pgfpathcurveto{\pgfqpoint{1.598261in}{1.692831in}}{\pgfqpoint{1.606161in}{1.689559in}}{\pgfqpoint{1.614397in}{1.689559in}}%
\pgfpathclose%
\pgfusepath{stroke,fill}%
\end{pgfscope}%
\begin{pgfscope}%
\pgfpathrectangle{\pgfqpoint{0.100000in}{0.212622in}}{\pgfqpoint{3.696000in}{3.696000in}}%
\pgfusepath{clip}%
\pgfsetbuttcap%
\pgfsetroundjoin%
\definecolor{currentfill}{rgb}{0.121569,0.466667,0.705882}%
\pgfsetfillcolor{currentfill}%
\pgfsetfillopacity{0.636772}%
\pgfsetlinewidth{1.003750pt}%
\definecolor{currentstroke}{rgb}{0.121569,0.466667,0.705882}%
\pgfsetstrokecolor{currentstroke}%
\pgfsetstrokeopacity{0.636772}%
\pgfsetdash{}{0pt}%
\pgfpathmoveto{\pgfqpoint{0.758797in}{1.413400in}}%
\pgfpathcurveto{\pgfqpoint{0.767034in}{1.413400in}}{\pgfqpoint{0.774934in}{1.416672in}}{\pgfqpoint{0.780757in}{1.422496in}}%
\pgfpathcurveto{\pgfqpoint{0.786581in}{1.428320in}}{\pgfqpoint{0.789854in}{1.436220in}}{\pgfqpoint{0.789854in}{1.444456in}}%
\pgfpathcurveto{\pgfqpoint{0.789854in}{1.452692in}}{\pgfqpoint{0.786581in}{1.460593in}}{\pgfqpoint{0.780757in}{1.466416in}}%
\pgfpathcurveto{\pgfqpoint{0.774934in}{1.472240in}}{\pgfqpoint{0.767034in}{1.475513in}}{\pgfqpoint{0.758797in}{1.475513in}}%
\pgfpathcurveto{\pgfqpoint{0.750561in}{1.475513in}}{\pgfqpoint{0.742661in}{1.472240in}}{\pgfqpoint{0.736837in}{1.466416in}}%
\pgfpathcurveto{\pgfqpoint{0.731013in}{1.460593in}}{\pgfqpoint{0.727741in}{1.452692in}}{\pgfqpoint{0.727741in}{1.444456in}}%
\pgfpathcurveto{\pgfqpoint{0.727741in}{1.436220in}}{\pgfqpoint{0.731013in}{1.428320in}}{\pgfqpoint{0.736837in}{1.422496in}}%
\pgfpathcurveto{\pgfqpoint{0.742661in}{1.416672in}}{\pgfqpoint{0.750561in}{1.413400in}}{\pgfqpoint{0.758797in}{1.413400in}}%
\pgfpathclose%
\pgfusepath{stroke,fill}%
\end{pgfscope}%
\begin{pgfscope}%
\pgfpathrectangle{\pgfqpoint{0.100000in}{0.212622in}}{\pgfqpoint{3.696000in}{3.696000in}}%
\pgfusepath{clip}%
\pgfsetbuttcap%
\pgfsetroundjoin%
\definecolor{currentfill}{rgb}{0.121569,0.466667,0.705882}%
\pgfsetfillcolor{currentfill}%
\pgfsetfillopacity{0.636988}%
\pgfsetlinewidth{1.003750pt}%
\definecolor{currentstroke}{rgb}{0.121569,0.466667,0.705882}%
\pgfsetstrokecolor{currentstroke}%
\pgfsetstrokeopacity{0.636988}%
\pgfsetdash{}{0pt}%
\pgfpathmoveto{\pgfqpoint{0.772718in}{1.397686in}}%
\pgfpathcurveto{\pgfqpoint{0.780955in}{1.397686in}}{\pgfqpoint{0.788855in}{1.400958in}}{\pgfqpoint{0.794679in}{1.406782in}}%
\pgfpathcurveto{\pgfqpoint{0.800503in}{1.412606in}}{\pgfqpoint{0.803775in}{1.420506in}}{\pgfqpoint{0.803775in}{1.428743in}}%
\pgfpathcurveto{\pgfqpoint{0.803775in}{1.436979in}}{\pgfqpoint{0.800503in}{1.444879in}}{\pgfqpoint{0.794679in}{1.450703in}}%
\pgfpathcurveto{\pgfqpoint{0.788855in}{1.456527in}}{\pgfqpoint{0.780955in}{1.459799in}}{\pgfqpoint{0.772718in}{1.459799in}}%
\pgfpathcurveto{\pgfqpoint{0.764482in}{1.459799in}}{\pgfqpoint{0.756582in}{1.456527in}}{\pgfqpoint{0.750758in}{1.450703in}}%
\pgfpathcurveto{\pgfqpoint{0.744934in}{1.444879in}}{\pgfqpoint{0.741662in}{1.436979in}}{\pgfqpoint{0.741662in}{1.428743in}}%
\pgfpathcurveto{\pgfqpoint{0.741662in}{1.420506in}}{\pgfqpoint{0.744934in}{1.412606in}}{\pgfqpoint{0.750758in}{1.406782in}}%
\pgfpathcurveto{\pgfqpoint{0.756582in}{1.400958in}}{\pgfqpoint{0.764482in}{1.397686in}}{\pgfqpoint{0.772718in}{1.397686in}}%
\pgfpathclose%
\pgfusepath{stroke,fill}%
\end{pgfscope}%
\begin{pgfscope}%
\pgfpathrectangle{\pgfqpoint{0.100000in}{0.212622in}}{\pgfqpoint{3.696000in}{3.696000in}}%
\pgfusepath{clip}%
\pgfsetbuttcap%
\pgfsetroundjoin%
\definecolor{currentfill}{rgb}{0.121569,0.466667,0.705882}%
\pgfsetfillcolor{currentfill}%
\pgfsetfillopacity{0.638223}%
\pgfsetlinewidth{1.003750pt}%
\definecolor{currentstroke}{rgb}{0.121569,0.466667,0.705882}%
\pgfsetstrokecolor{currentstroke}%
\pgfsetstrokeopacity{0.638223}%
\pgfsetdash{}{0pt}%
\pgfpathmoveto{\pgfqpoint{0.758641in}{1.408194in}}%
\pgfpathcurveto{\pgfqpoint{0.766877in}{1.408194in}}{\pgfqpoint{0.774777in}{1.411466in}}{\pgfqpoint{0.780601in}{1.417290in}}%
\pgfpathcurveto{\pgfqpoint{0.786425in}{1.423114in}}{\pgfqpoint{0.789697in}{1.431014in}}{\pgfqpoint{0.789697in}{1.439250in}}%
\pgfpathcurveto{\pgfqpoint{0.789697in}{1.447487in}}{\pgfqpoint{0.786425in}{1.455387in}}{\pgfqpoint{0.780601in}{1.461211in}}%
\pgfpathcurveto{\pgfqpoint{0.774777in}{1.467035in}}{\pgfqpoint{0.766877in}{1.470307in}}{\pgfqpoint{0.758641in}{1.470307in}}%
\pgfpathcurveto{\pgfqpoint{0.750405in}{1.470307in}}{\pgfqpoint{0.742505in}{1.467035in}}{\pgfqpoint{0.736681in}{1.461211in}}%
\pgfpathcurveto{\pgfqpoint{0.730857in}{1.455387in}}{\pgfqpoint{0.727584in}{1.447487in}}{\pgfqpoint{0.727584in}{1.439250in}}%
\pgfpathcurveto{\pgfqpoint{0.727584in}{1.431014in}}{\pgfqpoint{0.730857in}{1.423114in}}{\pgfqpoint{0.736681in}{1.417290in}}%
\pgfpathcurveto{\pgfqpoint{0.742505in}{1.411466in}}{\pgfqpoint{0.750405in}{1.408194in}}{\pgfqpoint{0.758641in}{1.408194in}}%
\pgfpathclose%
\pgfusepath{stroke,fill}%
\end{pgfscope}%
\begin{pgfscope}%
\pgfpathrectangle{\pgfqpoint{0.100000in}{0.212622in}}{\pgfqpoint{3.696000in}{3.696000in}}%
\pgfusepath{clip}%
\pgfsetbuttcap%
\pgfsetroundjoin%
\definecolor{currentfill}{rgb}{0.121569,0.466667,0.705882}%
\pgfsetfillcolor{currentfill}%
\pgfsetfillopacity{0.638235}%
\pgfsetlinewidth{1.003750pt}%
\definecolor{currentstroke}{rgb}{0.121569,0.466667,0.705882}%
\pgfsetstrokecolor{currentstroke}%
\pgfsetstrokeopacity{0.638235}%
\pgfsetdash{}{0pt}%
\pgfpathmoveto{\pgfqpoint{0.771210in}{1.399362in}}%
\pgfpathcurveto{\pgfqpoint{0.779446in}{1.399362in}}{\pgfqpoint{0.787346in}{1.402634in}}{\pgfqpoint{0.793170in}{1.408458in}}%
\pgfpathcurveto{\pgfqpoint{0.798994in}{1.414282in}}{\pgfqpoint{0.802266in}{1.422182in}}{\pgfqpoint{0.802266in}{1.430418in}}%
\pgfpathcurveto{\pgfqpoint{0.802266in}{1.438654in}}{\pgfqpoint{0.798994in}{1.446554in}}{\pgfqpoint{0.793170in}{1.452378in}}%
\pgfpathcurveto{\pgfqpoint{0.787346in}{1.458202in}}{\pgfqpoint{0.779446in}{1.461475in}}{\pgfqpoint{0.771210in}{1.461475in}}%
\pgfpathcurveto{\pgfqpoint{0.762973in}{1.461475in}}{\pgfqpoint{0.755073in}{1.458202in}}{\pgfqpoint{0.749249in}{1.452378in}}%
\pgfpathcurveto{\pgfqpoint{0.743425in}{1.446554in}}{\pgfqpoint{0.740153in}{1.438654in}}{\pgfqpoint{0.740153in}{1.430418in}}%
\pgfpathcurveto{\pgfqpoint{0.740153in}{1.422182in}}{\pgfqpoint{0.743425in}{1.414282in}}{\pgfqpoint{0.749249in}{1.408458in}}%
\pgfpathcurveto{\pgfqpoint{0.755073in}{1.402634in}}{\pgfqpoint{0.762973in}{1.399362in}}{\pgfqpoint{0.771210in}{1.399362in}}%
\pgfpathclose%
\pgfusepath{stroke,fill}%
\end{pgfscope}%
\begin{pgfscope}%
\pgfpathrectangle{\pgfqpoint{0.100000in}{0.212622in}}{\pgfqpoint{3.696000in}{3.696000in}}%
\pgfusepath{clip}%
\pgfsetbuttcap%
\pgfsetroundjoin%
\definecolor{currentfill}{rgb}{0.121569,0.466667,0.705882}%
\pgfsetfillcolor{currentfill}%
\pgfsetfillopacity{0.638924}%
\pgfsetlinewidth{1.003750pt}%
\definecolor{currentstroke}{rgb}{0.121569,0.466667,0.705882}%
\pgfsetstrokecolor{currentstroke}%
\pgfsetstrokeopacity{0.638924}%
\pgfsetdash{}{0pt}%
\pgfpathmoveto{\pgfqpoint{0.770532in}{1.400128in}}%
\pgfpathcurveto{\pgfqpoint{0.778768in}{1.400128in}}{\pgfqpoint{0.786668in}{1.403401in}}{\pgfqpoint{0.792492in}{1.409225in}}%
\pgfpathcurveto{\pgfqpoint{0.798316in}{1.415049in}}{\pgfqpoint{0.801588in}{1.422949in}}{\pgfqpoint{0.801588in}{1.431185in}}%
\pgfpathcurveto{\pgfqpoint{0.801588in}{1.439421in}}{\pgfqpoint{0.798316in}{1.447321in}}{\pgfqpoint{0.792492in}{1.453145in}}%
\pgfpathcurveto{\pgfqpoint{0.786668in}{1.458969in}}{\pgfqpoint{0.778768in}{1.462241in}}{\pgfqpoint{0.770532in}{1.462241in}}%
\pgfpathcurveto{\pgfqpoint{0.762295in}{1.462241in}}{\pgfqpoint{0.754395in}{1.458969in}}{\pgfqpoint{0.748571in}{1.453145in}}%
\pgfpathcurveto{\pgfqpoint{0.742748in}{1.447321in}}{\pgfqpoint{0.739475in}{1.439421in}}{\pgfqpoint{0.739475in}{1.431185in}}%
\pgfpathcurveto{\pgfqpoint{0.739475in}{1.422949in}}{\pgfqpoint{0.742748in}{1.415049in}}{\pgfqpoint{0.748571in}{1.409225in}}%
\pgfpathcurveto{\pgfqpoint{0.754395in}{1.403401in}}{\pgfqpoint{0.762295in}{1.400128in}}{\pgfqpoint{0.770532in}{1.400128in}}%
\pgfpathclose%
\pgfusepath{stroke,fill}%
\end{pgfscope}%
\begin{pgfscope}%
\pgfpathrectangle{\pgfqpoint{0.100000in}{0.212622in}}{\pgfqpoint{3.696000in}{3.696000in}}%
\pgfusepath{clip}%
\pgfsetbuttcap%
\pgfsetroundjoin%
\definecolor{currentfill}{rgb}{0.121569,0.466667,0.705882}%
\pgfsetfillcolor{currentfill}%
\pgfsetfillopacity{0.639103}%
\pgfsetlinewidth{1.003750pt}%
\definecolor{currentstroke}{rgb}{0.121569,0.466667,0.705882}%
\pgfsetstrokecolor{currentstroke}%
\pgfsetstrokeopacity{0.639103}%
\pgfsetdash{}{0pt}%
\pgfpathmoveto{\pgfqpoint{0.760931in}{1.402752in}}%
\pgfpathcurveto{\pgfqpoint{0.769168in}{1.402752in}}{\pgfqpoint{0.777068in}{1.406025in}}{\pgfqpoint{0.782891in}{1.411849in}}%
\pgfpathcurveto{\pgfqpoint{0.788715in}{1.417673in}}{\pgfqpoint{0.791988in}{1.425573in}}{\pgfqpoint{0.791988in}{1.433809in}}%
\pgfpathcurveto{\pgfqpoint{0.791988in}{1.442045in}}{\pgfqpoint{0.788715in}{1.449945in}}{\pgfqpoint{0.782891in}{1.455769in}}%
\pgfpathcurveto{\pgfqpoint{0.777068in}{1.461593in}}{\pgfqpoint{0.769168in}{1.464865in}}{\pgfqpoint{0.760931in}{1.464865in}}%
\pgfpathcurveto{\pgfqpoint{0.752695in}{1.464865in}}{\pgfqpoint{0.744795in}{1.461593in}}{\pgfqpoint{0.738971in}{1.455769in}}%
\pgfpathcurveto{\pgfqpoint{0.733147in}{1.449945in}}{\pgfqpoint{0.729875in}{1.442045in}}{\pgfqpoint{0.729875in}{1.433809in}}%
\pgfpathcurveto{\pgfqpoint{0.729875in}{1.425573in}}{\pgfqpoint{0.733147in}{1.417673in}}{\pgfqpoint{0.738971in}{1.411849in}}%
\pgfpathcurveto{\pgfqpoint{0.744795in}{1.406025in}}{\pgfqpoint{0.752695in}{1.402752in}}{\pgfqpoint{0.760931in}{1.402752in}}%
\pgfpathclose%
\pgfusepath{stroke,fill}%
\end{pgfscope}%
\begin{pgfscope}%
\pgfpathrectangle{\pgfqpoint{0.100000in}{0.212622in}}{\pgfqpoint{3.696000in}{3.696000in}}%
\pgfusepath{clip}%
\pgfsetbuttcap%
\pgfsetroundjoin%
\definecolor{currentfill}{rgb}{0.121569,0.466667,0.705882}%
\pgfsetfillcolor{currentfill}%
\pgfsetfillopacity{0.639309}%
\pgfsetlinewidth{1.003750pt}%
\definecolor{currentstroke}{rgb}{0.121569,0.466667,0.705882}%
\pgfsetstrokecolor{currentstroke}%
\pgfsetstrokeopacity{0.639309}%
\pgfsetdash{}{0pt}%
\pgfpathmoveto{\pgfqpoint{0.770098in}{1.400639in}}%
\pgfpathcurveto{\pgfqpoint{0.778334in}{1.400639in}}{\pgfqpoint{0.786234in}{1.403912in}}{\pgfqpoint{0.792058in}{1.409736in}}%
\pgfpathcurveto{\pgfqpoint{0.797882in}{1.415560in}}{\pgfqpoint{0.801155in}{1.423460in}}{\pgfqpoint{0.801155in}{1.431696in}}%
\pgfpathcurveto{\pgfqpoint{0.801155in}{1.439932in}}{\pgfqpoint{0.797882in}{1.447832in}}{\pgfqpoint{0.792058in}{1.453656in}}%
\pgfpathcurveto{\pgfqpoint{0.786234in}{1.459480in}}{\pgfqpoint{0.778334in}{1.462752in}}{\pgfqpoint{0.770098in}{1.462752in}}%
\pgfpathcurveto{\pgfqpoint{0.761862in}{1.462752in}}{\pgfqpoint{0.753962in}{1.459480in}}{\pgfqpoint{0.748138in}{1.453656in}}%
\pgfpathcurveto{\pgfqpoint{0.742314in}{1.447832in}}{\pgfqpoint{0.739042in}{1.439932in}}{\pgfqpoint{0.739042in}{1.431696in}}%
\pgfpathcurveto{\pgfqpoint{0.739042in}{1.423460in}}{\pgfqpoint{0.742314in}{1.415560in}}{\pgfqpoint{0.748138in}{1.409736in}}%
\pgfpathcurveto{\pgfqpoint{0.753962in}{1.403912in}}{\pgfqpoint{0.761862in}{1.400639in}}{\pgfqpoint{0.770098in}{1.400639in}}%
\pgfpathclose%
\pgfusepath{stroke,fill}%
\end{pgfscope}%
\begin{pgfscope}%
\pgfpathrectangle{\pgfqpoint{0.100000in}{0.212622in}}{\pgfqpoint{3.696000in}{3.696000in}}%
\pgfusepath{clip}%
\pgfsetbuttcap%
\pgfsetroundjoin%
\definecolor{currentfill}{rgb}{0.121569,0.466667,0.705882}%
\pgfsetfillcolor{currentfill}%
\pgfsetfillopacity{0.639509}%
\pgfsetlinewidth{1.003750pt}%
\definecolor{currentstroke}{rgb}{0.121569,0.466667,0.705882}%
\pgfsetstrokecolor{currentstroke}%
\pgfsetstrokeopacity{0.639509}%
\pgfsetdash{}{0pt}%
\pgfpathmoveto{\pgfqpoint{0.769874in}{1.400852in}}%
\pgfpathcurveto{\pgfqpoint{0.778110in}{1.400852in}}{\pgfqpoint{0.786010in}{1.404125in}}{\pgfqpoint{0.791834in}{1.409949in}}%
\pgfpathcurveto{\pgfqpoint{0.797658in}{1.415773in}}{\pgfqpoint{0.800931in}{1.423673in}}{\pgfqpoint{0.800931in}{1.431909in}}%
\pgfpathcurveto{\pgfqpoint{0.800931in}{1.440145in}}{\pgfqpoint{0.797658in}{1.448045in}}{\pgfqpoint{0.791834in}{1.453869in}}%
\pgfpathcurveto{\pgfqpoint{0.786010in}{1.459693in}}{\pgfqpoint{0.778110in}{1.462965in}}{\pgfqpoint{0.769874in}{1.462965in}}%
\pgfpathcurveto{\pgfqpoint{0.761638in}{1.462965in}}{\pgfqpoint{0.753738in}{1.459693in}}{\pgfqpoint{0.747914in}{1.453869in}}%
\pgfpathcurveto{\pgfqpoint{0.742090in}{1.448045in}}{\pgfqpoint{0.738818in}{1.440145in}}{\pgfqpoint{0.738818in}{1.431909in}}%
\pgfpathcurveto{\pgfqpoint{0.738818in}{1.423673in}}{\pgfqpoint{0.742090in}{1.415773in}}{\pgfqpoint{0.747914in}{1.409949in}}%
\pgfpathcurveto{\pgfqpoint{0.753738in}{1.404125in}}{\pgfqpoint{0.761638in}{1.400852in}}{\pgfqpoint{0.769874in}{1.400852in}}%
\pgfpathclose%
\pgfusepath{stroke,fill}%
\end{pgfscope}%
\begin{pgfscope}%
\pgfpathrectangle{\pgfqpoint{0.100000in}{0.212622in}}{\pgfqpoint{3.696000in}{3.696000in}}%
\pgfusepath{clip}%
\pgfsetbuttcap%
\pgfsetroundjoin%
\definecolor{currentfill}{rgb}{0.121569,0.466667,0.705882}%
\pgfsetfillcolor{currentfill}%
\pgfsetfillopacity{0.639629}%
\pgfsetlinewidth{1.003750pt}%
\definecolor{currentstroke}{rgb}{0.121569,0.466667,0.705882}%
\pgfsetstrokecolor{currentstroke}%
\pgfsetstrokeopacity{0.639629}%
\pgfsetdash{}{0pt}%
\pgfpathmoveto{\pgfqpoint{0.769750in}{1.401014in}}%
\pgfpathcurveto{\pgfqpoint{0.777986in}{1.401014in}}{\pgfqpoint{0.785886in}{1.404286in}}{\pgfqpoint{0.791710in}{1.410110in}}%
\pgfpathcurveto{\pgfqpoint{0.797534in}{1.415934in}}{\pgfqpoint{0.800806in}{1.423834in}}{\pgfqpoint{0.800806in}{1.432071in}}%
\pgfpathcurveto{\pgfqpoint{0.800806in}{1.440307in}}{\pgfqpoint{0.797534in}{1.448207in}}{\pgfqpoint{0.791710in}{1.454031in}}%
\pgfpathcurveto{\pgfqpoint{0.785886in}{1.459855in}}{\pgfqpoint{0.777986in}{1.463127in}}{\pgfqpoint{0.769750in}{1.463127in}}%
\pgfpathcurveto{\pgfqpoint{0.761514in}{1.463127in}}{\pgfqpoint{0.753613in}{1.459855in}}{\pgfqpoint{0.747790in}{1.454031in}}%
\pgfpathcurveto{\pgfqpoint{0.741966in}{1.448207in}}{\pgfqpoint{0.738693in}{1.440307in}}{\pgfqpoint{0.738693in}{1.432071in}}%
\pgfpathcurveto{\pgfqpoint{0.738693in}{1.423834in}}{\pgfqpoint{0.741966in}{1.415934in}}{\pgfqpoint{0.747790in}{1.410110in}}%
\pgfpathcurveto{\pgfqpoint{0.753613in}{1.404286in}}{\pgfqpoint{0.761514in}{1.401014in}}{\pgfqpoint{0.769750in}{1.401014in}}%
\pgfpathclose%
\pgfusepath{stroke,fill}%
\end{pgfscope}%
\begin{pgfscope}%
\pgfpathrectangle{\pgfqpoint{0.100000in}{0.212622in}}{\pgfqpoint{3.696000in}{3.696000in}}%
\pgfusepath{clip}%
\pgfsetbuttcap%
\pgfsetroundjoin%
\definecolor{currentfill}{rgb}{0.121569,0.466667,0.705882}%
\pgfsetfillcolor{currentfill}%
\pgfsetfillopacity{0.639691}%
\pgfsetlinewidth{1.003750pt}%
\definecolor{currentstroke}{rgb}{0.121569,0.466667,0.705882}%
\pgfsetstrokecolor{currentstroke}%
\pgfsetstrokeopacity{0.639691}%
\pgfsetdash{}{0pt}%
\pgfpathmoveto{\pgfqpoint{0.769684in}{1.401083in}}%
\pgfpathcurveto{\pgfqpoint{0.777920in}{1.401083in}}{\pgfqpoint{0.785820in}{1.404355in}}{\pgfqpoint{0.791644in}{1.410179in}}%
\pgfpathcurveto{\pgfqpoint{0.797468in}{1.416003in}}{\pgfqpoint{0.800741in}{1.423903in}}{\pgfqpoint{0.800741in}{1.432140in}}%
\pgfpathcurveto{\pgfqpoint{0.800741in}{1.440376in}}{\pgfqpoint{0.797468in}{1.448276in}}{\pgfqpoint{0.791644in}{1.454100in}}%
\pgfpathcurveto{\pgfqpoint{0.785820in}{1.459924in}}{\pgfqpoint{0.777920in}{1.463196in}}{\pgfqpoint{0.769684in}{1.463196in}}%
\pgfpathcurveto{\pgfqpoint{0.761448in}{1.463196in}}{\pgfqpoint{0.753548in}{1.459924in}}{\pgfqpoint{0.747724in}{1.454100in}}%
\pgfpathcurveto{\pgfqpoint{0.741900in}{1.448276in}}{\pgfqpoint{0.738628in}{1.440376in}}{\pgfqpoint{0.738628in}{1.432140in}}%
\pgfpathcurveto{\pgfqpoint{0.738628in}{1.423903in}}{\pgfqpoint{0.741900in}{1.416003in}}{\pgfqpoint{0.747724in}{1.410179in}}%
\pgfpathcurveto{\pgfqpoint{0.753548in}{1.404355in}}{\pgfqpoint{0.761448in}{1.401083in}}{\pgfqpoint{0.769684in}{1.401083in}}%
\pgfpathclose%
\pgfusepath{stroke,fill}%
\end{pgfscope}%
\begin{pgfscope}%
\pgfpathrectangle{\pgfqpoint{0.100000in}{0.212622in}}{\pgfqpoint{3.696000in}{3.696000in}}%
\pgfusepath{clip}%
\pgfsetbuttcap%
\pgfsetroundjoin%
\definecolor{currentfill}{rgb}{0.121569,0.466667,0.705882}%
\pgfsetfillcolor{currentfill}%
\pgfsetfillopacity{0.639727}%
\pgfsetlinewidth{1.003750pt}%
\definecolor{currentstroke}{rgb}{0.121569,0.466667,0.705882}%
\pgfsetstrokecolor{currentstroke}%
\pgfsetstrokeopacity{0.639727}%
\pgfsetdash{}{0pt}%
\pgfpathmoveto{\pgfqpoint{0.769644in}{1.401131in}}%
\pgfpathcurveto{\pgfqpoint{0.777881in}{1.401131in}}{\pgfqpoint{0.785781in}{1.404404in}}{\pgfqpoint{0.791605in}{1.410227in}}%
\pgfpathcurveto{\pgfqpoint{0.797429in}{1.416051in}}{\pgfqpoint{0.800701in}{1.423951in}}{\pgfqpoint{0.800701in}{1.432188in}}%
\pgfpathcurveto{\pgfqpoint{0.800701in}{1.440424in}}{\pgfqpoint{0.797429in}{1.448324in}}{\pgfqpoint{0.791605in}{1.454148in}}%
\pgfpathcurveto{\pgfqpoint{0.785781in}{1.459972in}}{\pgfqpoint{0.777881in}{1.463244in}}{\pgfqpoint{0.769644in}{1.463244in}}%
\pgfpathcurveto{\pgfqpoint{0.761408in}{1.463244in}}{\pgfqpoint{0.753508in}{1.459972in}}{\pgfqpoint{0.747684in}{1.454148in}}%
\pgfpathcurveto{\pgfqpoint{0.741860in}{1.448324in}}{\pgfqpoint{0.738588in}{1.440424in}}{\pgfqpoint{0.738588in}{1.432188in}}%
\pgfpathcurveto{\pgfqpoint{0.738588in}{1.423951in}}{\pgfqpoint{0.741860in}{1.416051in}}{\pgfqpoint{0.747684in}{1.410227in}}%
\pgfpathcurveto{\pgfqpoint{0.753508in}{1.404404in}}{\pgfqpoint{0.761408in}{1.401131in}}{\pgfqpoint{0.769644in}{1.401131in}}%
\pgfpathclose%
\pgfusepath{stroke,fill}%
\end{pgfscope}%
\begin{pgfscope}%
\pgfpathrectangle{\pgfqpoint{0.100000in}{0.212622in}}{\pgfqpoint{3.696000in}{3.696000in}}%
\pgfusepath{clip}%
\pgfsetbuttcap%
\pgfsetroundjoin%
\definecolor{currentfill}{rgb}{0.121569,0.466667,0.705882}%
\pgfsetfillcolor{currentfill}%
\pgfsetfillopacity{0.639747}%
\pgfsetlinewidth{1.003750pt}%
\definecolor{currentstroke}{rgb}{0.121569,0.466667,0.705882}%
\pgfsetstrokecolor{currentstroke}%
\pgfsetstrokeopacity{0.639747}%
\pgfsetdash{}{0pt}%
\pgfpathmoveto{\pgfqpoint{0.769625in}{1.401157in}}%
\pgfpathcurveto{\pgfqpoint{0.777861in}{1.401157in}}{\pgfqpoint{0.785761in}{1.404429in}}{\pgfqpoint{0.791585in}{1.410253in}}%
\pgfpathcurveto{\pgfqpoint{0.797409in}{1.416077in}}{\pgfqpoint{0.800681in}{1.423977in}}{\pgfqpoint{0.800681in}{1.432214in}}%
\pgfpathcurveto{\pgfqpoint{0.800681in}{1.440450in}}{\pgfqpoint{0.797409in}{1.448350in}}{\pgfqpoint{0.791585in}{1.454174in}}%
\pgfpathcurveto{\pgfqpoint{0.785761in}{1.459998in}}{\pgfqpoint{0.777861in}{1.463270in}}{\pgfqpoint{0.769625in}{1.463270in}}%
\pgfpathcurveto{\pgfqpoint{0.761388in}{1.463270in}}{\pgfqpoint{0.753488in}{1.459998in}}{\pgfqpoint{0.747664in}{1.454174in}}%
\pgfpathcurveto{\pgfqpoint{0.741840in}{1.448350in}}{\pgfqpoint{0.738568in}{1.440450in}}{\pgfqpoint{0.738568in}{1.432214in}}%
\pgfpathcurveto{\pgfqpoint{0.738568in}{1.423977in}}{\pgfqpoint{0.741840in}{1.416077in}}{\pgfqpoint{0.747664in}{1.410253in}}%
\pgfpathcurveto{\pgfqpoint{0.753488in}{1.404429in}}{\pgfqpoint{0.761388in}{1.401157in}}{\pgfqpoint{0.769625in}{1.401157in}}%
\pgfpathclose%
\pgfusepath{stroke,fill}%
\end{pgfscope}%
\begin{pgfscope}%
\pgfpathrectangle{\pgfqpoint{0.100000in}{0.212622in}}{\pgfqpoint{3.696000in}{3.696000in}}%
\pgfusepath{clip}%
\pgfsetbuttcap%
\pgfsetroundjoin%
\definecolor{currentfill}{rgb}{0.121569,0.466667,0.705882}%
\pgfsetfillcolor{currentfill}%
\pgfsetfillopacity{0.639758}%
\pgfsetlinewidth{1.003750pt}%
\definecolor{currentstroke}{rgb}{0.121569,0.466667,0.705882}%
\pgfsetstrokecolor{currentstroke}%
\pgfsetstrokeopacity{0.639758}%
\pgfsetdash{}{0pt}%
\pgfpathmoveto{\pgfqpoint{0.769613in}{1.401174in}}%
\pgfpathcurveto{\pgfqpoint{0.777849in}{1.401174in}}{\pgfqpoint{0.785749in}{1.404446in}}{\pgfqpoint{0.791573in}{1.410270in}}%
\pgfpathcurveto{\pgfqpoint{0.797397in}{1.416094in}}{\pgfqpoint{0.800669in}{1.423994in}}{\pgfqpoint{0.800669in}{1.432230in}}%
\pgfpathcurveto{\pgfqpoint{0.800669in}{1.440466in}}{\pgfqpoint{0.797397in}{1.448366in}}{\pgfqpoint{0.791573in}{1.454190in}}%
\pgfpathcurveto{\pgfqpoint{0.785749in}{1.460014in}}{\pgfqpoint{0.777849in}{1.463287in}}{\pgfqpoint{0.769613in}{1.463287in}}%
\pgfpathcurveto{\pgfqpoint{0.761376in}{1.463287in}}{\pgfqpoint{0.753476in}{1.460014in}}{\pgfqpoint{0.747652in}{1.454190in}}%
\pgfpathcurveto{\pgfqpoint{0.741828in}{1.448366in}}{\pgfqpoint{0.738556in}{1.440466in}}{\pgfqpoint{0.738556in}{1.432230in}}%
\pgfpathcurveto{\pgfqpoint{0.738556in}{1.423994in}}{\pgfqpoint{0.741828in}{1.416094in}}{\pgfqpoint{0.747652in}{1.410270in}}%
\pgfpathcurveto{\pgfqpoint{0.753476in}{1.404446in}}{\pgfqpoint{0.761376in}{1.401174in}}{\pgfqpoint{0.769613in}{1.401174in}}%
\pgfpathclose%
\pgfusepath{stroke,fill}%
\end{pgfscope}%
\begin{pgfscope}%
\pgfpathrectangle{\pgfqpoint{0.100000in}{0.212622in}}{\pgfqpoint{3.696000in}{3.696000in}}%
\pgfusepath{clip}%
\pgfsetbuttcap%
\pgfsetroundjoin%
\definecolor{currentfill}{rgb}{0.121569,0.466667,0.705882}%
\pgfsetfillcolor{currentfill}%
\pgfsetfillopacity{0.639763}%
\pgfsetlinewidth{1.003750pt}%
\definecolor{currentstroke}{rgb}{0.121569,0.466667,0.705882}%
\pgfsetstrokecolor{currentstroke}%
\pgfsetstrokeopacity{0.639763}%
\pgfsetdash{}{0pt}%
\pgfpathmoveto{\pgfqpoint{0.769606in}{1.401180in}}%
\pgfpathcurveto{\pgfqpoint{0.777843in}{1.401180in}}{\pgfqpoint{0.785743in}{1.404453in}}{\pgfqpoint{0.791567in}{1.410277in}}%
\pgfpathcurveto{\pgfqpoint{0.797390in}{1.416101in}}{\pgfqpoint{0.800663in}{1.424001in}}{\pgfqpoint{0.800663in}{1.432237in}}%
\pgfpathcurveto{\pgfqpoint{0.800663in}{1.440473in}}{\pgfqpoint{0.797390in}{1.448373in}}{\pgfqpoint{0.791567in}{1.454197in}}%
\pgfpathcurveto{\pgfqpoint{0.785743in}{1.460021in}}{\pgfqpoint{0.777843in}{1.463293in}}{\pgfqpoint{0.769606in}{1.463293in}}%
\pgfpathcurveto{\pgfqpoint{0.761370in}{1.463293in}}{\pgfqpoint{0.753470in}{1.460021in}}{\pgfqpoint{0.747646in}{1.454197in}}%
\pgfpathcurveto{\pgfqpoint{0.741822in}{1.448373in}}{\pgfqpoint{0.738550in}{1.440473in}}{\pgfqpoint{0.738550in}{1.432237in}}%
\pgfpathcurveto{\pgfqpoint{0.738550in}{1.424001in}}{\pgfqpoint{0.741822in}{1.416101in}}{\pgfqpoint{0.747646in}{1.410277in}}%
\pgfpathcurveto{\pgfqpoint{0.753470in}{1.404453in}}{\pgfqpoint{0.761370in}{1.401180in}}{\pgfqpoint{0.769606in}{1.401180in}}%
\pgfpathclose%
\pgfusepath{stroke,fill}%
\end{pgfscope}%
\begin{pgfscope}%
\pgfpathrectangle{\pgfqpoint{0.100000in}{0.212622in}}{\pgfqpoint{3.696000in}{3.696000in}}%
\pgfusepath{clip}%
\pgfsetbuttcap%
\pgfsetroundjoin%
\definecolor{currentfill}{rgb}{0.121569,0.466667,0.705882}%
\pgfsetfillcolor{currentfill}%
\pgfsetfillopacity{0.639767}%
\pgfsetlinewidth{1.003750pt}%
\definecolor{currentstroke}{rgb}{0.121569,0.466667,0.705882}%
\pgfsetstrokecolor{currentstroke}%
\pgfsetstrokeopacity{0.639767}%
\pgfsetdash{}{0pt}%
\pgfpathmoveto{\pgfqpoint{0.769603in}{1.401185in}}%
\pgfpathcurveto{\pgfqpoint{0.777839in}{1.401185in}}{\pgfqpoint{0.785739in}{1.404457in}}{\pgfqpoint{0.791563in}{1.410281in}}%
\pgfpathcurveto{\pgfqpoint{0.797387in}{1.416105in}}{\pgfqpoint{0.800659in}{1.424005in}}{\pgfqpoint{0.800659in}{1.432242in}}%
\pgfpathcurveto{\pgfqpoint{0.800659in}{1.440478in}}{\pgfqpoint{0.797387in}{1.448378in}}{\pgfqpoint{0.791563in}{1.454202in}}%
\pgfpathcurveto{\pgfqpoint{0.785739in}{1.460026in}}{\pgfqpoint{0.777839in}{1.463298in}}{\pgfqpoint{0.769603in}{1.463298in}}%
\pgfpathcurveto{\pgfqpoint{0.761366in}{1.463298in}}{\pgfqpoint{0.753466in}{1.460026in}}{\pgfqpoint{0.747642in}{1.454202in}}%
\pgfpathcurveto{\pgfqpoint{0.741818in}{1.448378in}}{\pgfqpoint{0.738546in}{1.440478in}}{\pgfqpoint{0.738546in}{1.432242in}}%
\pgfpathcurveto{\pgfqpoint{0.738546in}{1.424005in}}{\pgfqpoint{0.741818in}{1.416105in}}{\pgfqpoint{0.747642in}{1.410281in}}%
\pgfpathcurveto{\pgfqpoint{0.753466in}{1.404457in}}{\pgfqpoint{0.761366in}{1.401185in}}{\pgfqpoint{0.769603in}{1.401185in}}%
\pgfpathclose%
\pgfusepath{stroke,fill}%
\end{pgfscope}%
\begin{pgfscope}%
\pgfpathrectangle{\pgfqpoint{0.100000in}{0.212622in}}{\pgfqpoint{3.696000in}{3.696000in}}%
\pgfusepath{clip}%
\pgfsetbuttcap%
\pgfsetroundjoin%
\definecolor{currentfill}{rgb}{0.121569,0.466667,0.705882}%
\pgfsetfillcolor{currentfill}%
\pgfsetfillopacity{0.639768}%
\pgfsetlinewidth{1.003750pt}%
\definecolor{currentstroke}{rgb}{0.121569,0.466667,0.705882}%
\pgfsetstrokecolor{currentstroke}%
\pgfsetstrokeopacity{0.639768}%
\pgfsetdash{}{0pt}%
\pgfpathmoveto{\pgfqpoint{0.769601in}{1.401187in}}%
\pgfpathcurveto{\pgfqpoint{0.777837in}{1.401187in}}{\pgfqpoint{0.785737in}{1.404459in}}{\pgfqpoint{0.791561in}{1.410283in}}%
\pgfpathcurveto{\pgfqpoint{0.797385in}{1.416107in}}{\pgfqpoint{0.800657in}{1.424007in}}{\pgfqpoint{0.800657in}{1.432244in}}%
\pgfpathcurveto{\pgfqpoint{0.800657in}{1.440480in}}{\pgfqpoint{0.797385in}{1.448380in}}{\pgfqpoint{0.791561in}{1.454204in}}%
\pgfpathcurveto{\pgfqpoint{0.785737in}{1.460028in}}{\pgfqpoint{0.777837in}{1.463300in}}{\pgfqpoint{0.769601in}{1.463300in}}%
\pgfpathcurveto{\pgfqpoint{0.761364in}{1.463300in}}{\pgfqpoint{0.753464in}{1.460028in}}{\pgfqpoint{0.747640in}{1.454204in}}%
\pgfpathcurveto{\pgfqpoint{0.741816in}{1.448380in}}{\pgfqpoint{0.738544in}{1.440480in}}{\pgfqpoint{0.738544in}{1.432244in}}%
\pgfpathcurveto{\pgfqpoint{0.738544in}{1.424007in}}{\pgfqpoint{0.741816in}{1.416107in}}{\pgfqpoint{0.747640in}{1.410283in}}%
\pgfpathcurveto{\pgfqpoint{0.753464in}{1.404459in}}{\pgfqpoint{0.761364in}{1.401187in}}{\pgfqpoint{0.769601in}{1.401187in}}%
\pgfpathclose%
\pgfusepath{stroke,fill}%
\end{pgfscope}%
\begin{pgfscope}%
\pgfpathrectangle{\pgfqpoint{0.100000in}{0.212622in}}{\pgfqpoint{3.696000in}{3.696000in}}%
\pgfusepath{clip}%
\pgfsetbuttcap%
\pgfsetroundjoin%
\definecolor{currentfill}{rgb}{0.121569,0.466667,0.705882}%
\pgfsetfillcolor{currentfill}%
\pgfsetfillopacity{0.639769}%
\pgfsetlinewidth{1.003750pt}%
\definecolor{currentstroke}{rgb}{0.121569,0.466667,0.705882}%
\pgfsetstrokecolor{currentstroke}%
\pgfsetstrokeopacity{0.639769}%
\pgfsetdash{}{0pt}%
\pgfpathmoveto{\pgfqpoint{0.769600in}{1.401188in}}%
\pgfpathcurveto{\pgfqpoint{0.777836in}{1.401188in}}{\pgfqpoint{0.785736in}{1.404461in}}{\pgfqpoint{0.791560in}{1.410284in}}%
\pgfpathcurveto{\pgfqpoint{0.797384in}{1.416108in}}{\pgfqpoint{0.800656in}{1.424008in}}{\pgfqpoint{0.800656in}{1.432245in}}%
\pgfpathcurveto{\pgfqpoint{0.800656in}{1.440481in}}{\pgfqpoint{0.797384in}{1.448381in}}{\pgfqpoint{0.791560in}{1.454205in}}%
\pgfpathcurveto{\pgfqpoint{0.785736in}{1.460029in}}{\pgfqpoint{0.777836in}{1.463301in}}{\pgfqpoint{0.769600in}{1.463301in}}%
\pgfpathcurveto{\pgfqpoint{0.761363in}{1.463301in}}{\pgfqpoint{0.753463in}{1.460029in}}{\pgfqpoint{0.747639in}{1.454205in}}%
\pgfpathcurveto{\pgfqpoint{0.741815in}{1.448381in}}{\pgfqpoint{0.738543in}{1.440481in}}{\pgfqpoint{0.738543in}{1.432245in}}%
\pgfpathcurveto{\pgfqpoint{0.738543in}{1.424008in}}{\pgfqpoint{0.741815in}{1.416108in}}{\pgfqpoint{0.747639in}{1.410284in}}%
\pgfpathcurveto{\pgfqpoint{0.753463in}{1.404461in}}{\pgfqpoint{0.761363in}{1.401188in}}{\pgfqpoint{0.769600in}{1.401188in}}%
\pgfpathclose%
\pgfusepath{stroke,fill}%
\end{pgfscope}%
\begin{pgfscope}%
\pgfpathrectangle{\pgfqpoint{0.100000in}{0.212622in}}{\pgfqpoint{3.696000in}{3.696000in}}%
\pgfusepath{clip}%
\pgfsetbuttcap%
\pgfsetroundjoin%
\definecolor{currentfill}{rgb}{0.121569,0.466667,0.705882}%
\pgfsetfillcolor{currentfill}%
\pgfsetfillopacity{0.639770}%
\pgfsetlinewidth{1.003750pt}%
\definecolor{currentstroke}{rgb}{0.121569,0.466667,0.705882}%
\pgfsetstrokecolor{currentstroke}%
\pgfsetstrokeopacity{0.639770}%
\pgfsetdash{}{0pt}%
\pgfpathmoveto{\pgfqpoint{0.769599in}{1.401189in}}%
\pgfpathcurveto{\pgfqpoint{0.777835in}{1.401189in}}{\pgfqpoint{0.785735in}{1.404461in}}{\pgfqpoint{0.791559in}{1.410285in}}%
\pgfpathcurveto{\pgfqpoint{0.797383in}{1.416109in}}{\pgfqpoint{0.800656in}{1.424009in}}{\pgfqpoint{0.800656in}{1.432245in}}%
\pgfpathcurveto{\pgfqpoint{0.800656in}{1.440482in}}{\pgfqpoint{0.797383in}{1.448382in}}{\pgfqpoint{0.791559in}{1.454206in}}%
\pgfpathcurveto{\pgfqpoint{0.785735in}{1.460030in}}{\pgfqpoint{0.777835in}{1.463302in}}{\pgfqpoint{0.769599in}{1.463302in}}%
\pgfpathcurveto{\pgfqpoint{0.761363in}{1.463302in}}{\pgfqpoint{0.753463in}{1.460030in}}{\pgfqpoint{0.747639in}{1.454206in}}%
\pgfpathcurveto{\pgfqpoint{0.741815in}{1.448382in}}{\pgfqpoint{0.738543in}{1.440482in}}{\pgfqpoint{0.738543in}{1.432245in}}%
\pgfpathcurveto{\pgfqpoint{0.738543in}{1.424009in}}{\pgfqpoint{0.741815in}{1.416109in}}{\pgfqpoint{0.747639in}{1.410285in}}%
\pgfpathcurveto{\pgfqpoint{0.753463in}{1.404461in}}{\pgfqpoint{0.761363in}{1.401189in}}{\pgfqpoint{0.769599in}{1.401189in}}%
\pgfpathclose%
\pgfusepath{stroke,fill}%
\end{pgfscope}%
\begin{pgfscope}%
\pgfpathrectangle{\pgfqpoint{0.100000in}{0.212622in}}{\pgfqpoint{3.696000in}{3.696000in}}%
\pgfusepath{clip}%
\pgfsetbuttcap%
\pgfsetroundjoin%
\definecolor{currentfill}{rgb}{0.121569,0.466667,0.705882}%
\pgfsetfillcolor{currentfill}%
\pgfsetfillopacity{0.639770}%
\pgfsetlinewidth{1.003750pt}%
\definecolor{currentstroke}{rgb}{0.121569,0.466667,0.705882}%
\pgfsetstrokecolor{currentstroke}%
\pgfsetstrokeopacity{0.639770}%
\pgfsetdash{}{0pt}%
\pgfpathmoveto{\pgfqpoint{0.769599in}{1.401189in}}%
\pgfpathcurveto{\pgfqpoint{0.777835in}{1.401189in}}{\pgfqpoint{0.785735in}{1.404461in}}{\pgfqpoint{0.791559in}{1.410285in}}%
\pgfpathcurveto{\pgfqpoint{0.797383in}{1.416109in}}{\pgfqpoint{0.800655in}{1.424009in}}{\pgfqpoint{0.800655in}{1.432245in}}%
\pgfpathcurveto{\pgfqpoint{0.800655in}{1.440482in}}{\pgfqpoint{0.797383in}{1.448382in}}{\pgfqpoint{0.791559in}{1.454206in}}%
\pgfpathcurveto{\pgfqpoint{0.785735in}{1.460030in}}{\pgfqpoint{0.777835in}{1.463302in}}{\pgfqpoint{0.769599in}{1.463302in}}%
\pgfpathcurveto{\pgfqpoint{0.761362in}{1.463302in}}{\pgfqpoint{0.753462in}{1.460030in}}{\pgfqpoint{0.747638in}{1.454206in}}%
\pgfpathcurveto{\pgfqpoint{0.741815in}{1.448382in}}{\pgfqpoint{0.738542in}{1.440482in}}{\pgfqpoint{0.738542in}{1.432245in}}%
\pgfpathcurveto{\pgfqpoint{0.738542in}{1.424009in}}{\pgfqpoint{0.741815in}{1.416109in}}{\pgfqpoint{0.747638in}{1.410285in}}%
\pgfpathcurveto{\pgfqpoint{0.753462in}{1.404461in}}{\pgfqpoint{0.761362in}{1.401189in}}{\pgfqpoint{0.769599in}{1.401189in}}%
\pgfpathclose%
\pgfusepath{stroke,fill}%
\end{pgfscope}%
\begin{pgfscope}%
\pgfpathrectangle{\pgfqpoint{0.100000in}{0.212622in}}{\pgfqpoint{3.696000in}{3.696000in}}%
\pgfusepath{clip}%
\pgfsetbuttcap%
\pgfsetroundjoin%
\definecolor{currentfill}{rgb}{0.121569,0.466667,0.705882}%
\pgfsetfillcolor{currentfill}%
\pgfsetfillopacity{0.639770}%
\pgfsetlinewidth{1.003750pt}%
\definecolor{currentstroke}{rgb}{0.121569,0.466667,0.705882}%
\pgfsetstrokecolor{currentstroke}%
\pgfsetstrokeopacity{0.639770}%
\pgfsetdash{}{0pt}%
\pgfpathmoveto{\pgfqpoint{0.769599in}{1.401189in}}%
\pgfpathcurveto{\pgfqpoint{0.777835in}{1.401189in}}{\pgfqpoint{0.785735in}{1.404461in}}{\pgfqpoint{0.791559in}{1.410285in}}%
\pgfpathcurveto{\pgfqpoint{0.797383in}{1.416109in}}{\pgfqpoint{0.800655in}{1.424009in}}{\pgfqpoint{0.800655in}{1.432246in}}%
\pgfpathcurveto{\pgfqpoint{0.800655in}{1.440482in}}{\pgfqpoint{0.797383in}{1.448382in}}{\pgfqpoint{0.791559in}{1.454206in}}%
\pgfpathcurveto{\pgfqpoint{0.785735in}{1.460030in}}{\pgfqpoint{0.777835in}{1.463302in}}{\pgfqpoint{0.769599in}{1.463302in}}%
\pgfpathcurveto{\pgfqpoint{0.761362in}{1.463302in}}{\pgfqpoint{0.753462in}{1.460030in}}{\pgfqpoint{0.747638in}{1.454206in}}%
\pgfpathcurveto{\pgfqpoint{0.741814in}{1.448382in}}{\pgfqpoint{0.738542in}{1.440482in}}{\pgfqpoint{0.738542in}{1.432246in}}%
\pgfpathcurveto{\pgfqpoint{0.738542in}{1.424009in}}{\pgfqpoint{0.741814in}{1.416109in}}{\pgfqpoint{0.747638in}{1.410285in}}%
\pgfpathcurveto{\pgfqpoint{0.753462in}{1.404461in}}{\pgfqpoint{0.761362in}{1.401189in}}{\pgfqpoint{0.769599in}{1.401189in}}%
\pgfpathclose%
\pgfusepath{stroke,fill}%
\end{pgfscope}%
\begin{pgfscope}%
\pgfpathrectangle{\pgfqpoint{0.100000in}{0.212622in}}{\pgfqpoint{3.696000in}{3.696000in}}%
\pgfusepath{clip}%
\pgfsetbuttcap%
\pgfsetroundjoin%
\definecolor{currentfill}{rgb}{0.121569,0.466667,0.705882}%
\pgfsetfillcolor{currentfill}%
\pgfsetfillopacity{0.639770}%
\pgfsetlinewidth{1.003750pt}%
\definecolor{currentstroke}{rgb}{0.121569,0.466667,0.705882}%
\pgfsetstrokecolor{currentstroke}%
\pgfsetstrokeopacity{0.639770}%
\pgfsetdash{}{0pt}%
\pgfpathmoveto{\pgfqpoint{0.769598in}{1.401189in}}%
\pgfpathcurveto{\pgfqpoint{0.777835in}{1.401189in}}{\pgfqpoint{0.785735in}{1.404461in}}{\pgfqpoint{0.791559in}{1.410285in}}%
\pgfpathcurveto{\pgfqpoint{0.797383in}{1.416109in}}{\pgfqpoint{0.800655in}{1.424009in}}{\pgfqpoint{0.800655in}{1.432246in}}%
\pgfpathcurveto{\pgfqpoint{0.800655in}{1.440482in}}{\pgfqpoint{0.797383in}{1.448382in}}{\pgfqpoint{0.791559in}{1.454206in}}%
\pgfpathcurveto{\pgfqpoint{0.785735in}{1.460030in}}{\pgfqpoint{0.777835in}{1.463302in}}{\pgfqpoint{0.769598in}{1.463302in}}%
\pgfpathcurveto{\pgfqpoint{0.761362in}{1.463302in}}{\pgfqpoint{0.753462in}{1.460030in}}{\pgfqpoint{0.747638in}{1.454206in}}%
\pgfpathcurveto{\pgfqpoint{0.741814in}{1.448382in}}{\pgfqpoint{0.738542in}{1.440482in}}{\pgfqpoint{0.738542in}{1.432246in}}%
\pgfpathcurveto{\pgfqpoint{0.738542in}{1.424009in}}{\pgfqpoint{0.741814in}{1.416109in}}{\pgfqpoint{0.747638in}{1.410285in}}%
\pgfpathcurveto{\pgfqpoint{0.753462in}{1.404461in}}{\pgfqpoint{0.761362in}{1.401189in}}{\pgfqpoint{0.769598in}{1.401189in}}%
\pgfpathclose%
\pgfusepath{stroke,fill}%
\end{pgfscope}%
\begin{pgfscope}%
\pgfpathrectangle{\pgfqpoint{0.100000in}{0.212622in}}{\pgfqpoint{3.696000in}{3.696000in}}%
\pgfusepath{clip}%
\pgfsetbuttcap%
\pgfsetroundjoin%
\definecolor{currentfill}{rgb}{0.121569,0.466667,0.705882}%
\pgfsetfillcolor{currentfill}%
\pgfsetfillopacity{0.639770}%
\pgfsetlinewidth{1.003750pt}%
\definecolor{currentstroke}{rgb}{0.121569,0.466667,0.705882}%
\pgfsetstrokecolor{currentstroke}%
\pgfsetstrokeopacity{0.639770}%
\pgfsetdash{}{0pt}%
\pgfpathmoveto{\pgfqpoint{0.769598in}{1.401189in}}%
\pgfpathcurveto{\pgfqpoint{0.777835in}{1.401189in}}{\pgfqpoint{0.785735in}{1.404461in}}{\pgfqpoint{0.791559in}{1.410285in}}%
\pgfpathcurveto{\pgfqpoint{0.797383in}{1.416109in}}{\pgfqpoint{0.800655in}{1.424009in}}{\pgfqpoint{0.800655in}{1.432246in}}%
\pgfpathcurveto{\pgfqpoint{0.800655in}{1.440482in}}{\pgfqpoint{0.797383in}{1.448382in}}{\pgfqpoint{0.791559in}{1.454206in}}%
\pgfpathcurveto{\pgfqpoint{0.785735in}{1.460030in}}{\pgfqpoint{0.777835in}{1.463302in}}{\pgfqpoint{0.769598in}{1.463302in}}%
\pgfpathcurveto{\pgfqpoint{0.761362in}{1.463302in}}{\pgfqpoint{0.753462in}{1.460030in}}{\pgfqpoint{0.747638in}{1.454206in}}%
\pgfpathcurveto{\pgfqpoint{0.741814in}{1.448382in}}{\pgfqpoint{0.738542in}{1.440482in}}{\pgfqpoint{0.738542in}{1.432246in}}%
\pgfpathcurveto{\pgfqpoint{0.738542in}{1.424009in}}{\pgfqpoint{0.741814in}{1.416109in}}{\pgfqpoint{0.747638in}{1.410285in}}%
\pgfpathcurveto{\pgfqpoint{0.753462in}{1.404461in}}{\pgfqpoint{0.761362in}{1.401189in}}{\pgfqpoint{0.769598in}{1.401189in}}%
\pgfpathclose%
\pgfusepath{stroke,fill}%
\end{pgfscope}%
\begin{pgfscope}%
\pgfpathrectangle{\pgfqpoint{0.100000in}{0.212622in}}{\pgfqpoint{3.696000in}{3.696000in}}%
\pgfusepath{clip}%
\pgfsetbuttcap%
\pgfsetroundjoin%
\definecolor{currentfill}{rgb}{0.121569,0.466667,0.705882}%
\pgfsetfillcolor{currentfill}%
\pgfsetfillopacity{0.639770}%
\pgfsetlinewidth{1.003750pt}%
\definecolor{currentstroke}{rgb}{0.121569,0.466667,0.705882}%
\pgfsetstrokecolor{currentstroke}%
\pgfsetstrokeopacity{0.639770}%
\pgfsetdash{}{0pt}%
\pgfpathmoveto{\pgfqpoint{0.769598in}{1.401189in}}%
\pgfpathcurveto{\pgfqpoint{0.777835in}{1.401189in}}{\pgfqpoint{0.785735in}{1.404461in}}{\pgfqpoint{0.791559in}{1.410285in}}%
\pgfpathcurveto{\pgfqpoint{0.797383in}{1.416109in}}{\pgfqpoint{0.800655in}{1.424009in}}{\pgfqpoint{0.800655in}{1.432246in}}%
\pgfpathcurveto{\pgfqpoint{0.800655in}{1.440482in}}{\pgfqpoint{0.797383in}{1.448382in}}{\pgfqpoint{0.791559in}{1.454206in}}%
\pgfpathcurveto{\pgfqpoint{0.785735in}{1.460030in}}{\pgfqpoint{0.777835in}{1.463302in}}{\pgfqpoint{0.769598in}{1.463302in}}%
\pgfpathcurveto{\pgfqpoint{0.761362in}{1.463302in}}{\pgfqpoint{0.753462in}{1.460030in}}{\pgfqpoint{0.747638in}{1.454206in}}%
\pgfpathcurveto{\pgfqpoint{0.741814in}{1.448382in}}{\pgfqpoint{0.738542in}{1.440482in}}{\pgfqpoint{0.738542in}{1.432246in}}%
\pgfpathcurveto{\pgfqpoint{0.738542in}{1.424009in}}{\pgfqpoint{0.741814in}{1.416109in}}{\pgfqpoint{0.747638in}{1.410285in}}%
\pgfpathcurveto{\pgfqpoint{0.753462in}{1.404461in}}{\pgfqpoint{0.761362in}{1.401189in}}{\pgfqpoint{0.769598in}{1.401189in}}%
\pgfpathclose%
\pgfusepath{stroke,fill}%
\end{pgfscope}%
\begin{pgfscope}%
\pgfpathrectangle{\pgfqpoint{0.100000in}{0.212622in}}{\pgfqpoint{3.696000in}{3.696000in}}%
\pgfusepath{clip}%
\pgfsetbuttcap%
\pgfsetroundjoin%
\definecolor{currentfill}{rgb}{0.121569,0.466667,0.705882}%
\pgfsetfillcolor{currentfill}%
\pgfsetfillopacity{0.639770}%
\pgfsetlinewidth{1.003750pt}%
\definecolor{currentstroke}{rgb}{0.121569,0.466667,0.705882}%
\pgfsetstrokecolor{currentstroke}%
\pgfsetstrokeopacity{0.639770}%
\pgfsetdash{}{0pt}%
\pgfpathmoveto{\pgfqpoint{0.769598in}{1.401189in}}%
\pgfpathcurveto{\pgfqpoint{0.777835in}{1.401189in}}{\pgfqpoint{0.785735in}{1.404461in}}{\pgfqpoint{0.791559in}{1.410285in}}%
\pgfpathcurveto{\pgfqpoint{0.797383in}{1.416109in}}{\pgfqpoint{0.800655in}{1.424009in}}{\pgfqpoint{0.800655in}{1.432246in}}%
\pgfpathcurveto{\pgfqpoint{0.800655in}{1.440482in}}{\pgfqpoint{0.797383in}{1.448382in}}{\pgfqpoint{0.791559in}{1.454206in}}%
\pgfpathcurveto{\pgfqpoint{0.785735in}{1.460030in}}{\pgfqpoint{0.777835in}{1.463302in}}{\pgfqpoint{0.769598in}{1.463302in}}%
\pgfpathcurveto{\pgfqpoint{0.761362in}{1.463302in}}{\pgfqpoint{0.753462in}{1.460030in}}{\pgfqpoint{0.747638in}{1.454206in}}%
\pgfpathcurveto{\pgfqpoint{0.741814in}{1.448382in}}{\pgfqpoint{0.738542in}{1.440482in}}{\pgfqpoint{0.738542in}{1.432246in}}%
\pgfpathcurveto{\pgfqpoint{0.738542in}{1.424009in}}{\pgfqpoint{0.741814in}{1.416109in}}{\pgfqpoint{0.747638in}{1.410285in}}%
\pgfpathcurveto{\pgfqpoint{0.753462in}{1.404461in}}{\pgfqpoint{0.761362in}{1.401189in}}{\pgfqpoint{0.769598in}{1.401189in}}%
\pgfpathclose%
\pgfusepath{stroke,fill}%
\end{pgfscope}%
\begin{pgfscope}%
\pgfpathrectangle{\pgfqpoint{0.100000in}{0.212622in}}{\pgfqpoint{3.696000in}{3.696000in}}%
\pgfusepath{clip}%
\pgfsetbuttcap%
\pgfsetroundjoin%
\definecolor{currentfill}{rgb}{0.121569,0.466667,0.705882}%
\pgfsetfillcolor{currentfill}%
\pgfsetfillopacity{0.639770}%
\pgfsetlinewidth{1.003750pt}%
\definecolor{currentstroke}{rgb}{0.121569,0.466667,0.705882}%
\pgfsetstrokecolor{currentstroke}%
\pgfsetstrokeopacity{0.639770}%
\pgfsetdash{}{0pt}%
\pgfpathmoveto{\pgfqpoint{0.769598in}{1.401189in}}%
\pgfpathcurveto{\pgfqpoint{0.777835in}{1.401189in}}{\pgfqpoint{0.785735in}{1.404461in}}{\pgfqpoint{0.791559in}{1.410285in}}%
\pgfpathcurveto{\pgfqpoint{0.797383in}{1.416109in}}{\pgfqpoint{0.800655in}{1.424009in}}{\pgfqpoint{0.800655in}{1.432246in}}%
\pgfpathcurveto{\pgfqpoint{0.800655in}{1.440482in}}{\pgfqpoint{0.797383in}{1.448382in}}{\pgfqpoint{0.791559in}{1.454206in}}%
\pgfpathcurveto{\pgfqpoint{0.785735in}{1.460030in}}{\pgfqpoint{0.777835in}{1.463302in}}{\pgfqpoint{0.769598in}{1.463302in}}%
\pgfpathcurveto{\pgfqpoint{0.761362in}{1.463302in}}{\pgfqpoint{0.753462in}{1.460030in}}{\pgfqpoint{0.747638in}{1.454206in}}%
\pgfpathcurveto{\pgfqpoint{0.741814in}{1.448382in}}{\pgfqpoint{0.738542in}{1.440482in}}{\pgfqpoint{0.738542in}{1.432246in}}%
\pgfpathcurveto{\pgfqpoint{0.738542in}{1.424009in}}{\pgfqpoint{0.741814in}{1.416109in}}{\pgfqpoint{0.747638in}{1.410285in}}%
\pgfpathcurveto{\pgfqpoint{0.753462in}{1.404461in}}{\pgfqpoint{0.761362in}{1.401189in}}{\pgfqpoint{0.769598in}{1.401189in}}%
\pgfpathclose%
\pgfusepath{stroke,fill}%
\end{pgfscope}%
\begin{pgfscope}%
\pgfpathrectangle{\pgfqpoint{0.100000in}{0.212622in}}{\pgfqpoint{3.696000in}{3.696000in}}%
\pgfusepath{clip}%
\pgfsetbuttcap%
\pgfsetroundjoin%
\definecolor{currentfill}{rgb}{0.121569,0.466667,0.705882}%
\pgfsetfillcolor{currentfill}%
\pgfsetfillopacity{0.639770}%
\pgfsetlinewidth{1.003750pt}%
\definecolor{currentstroke}{rgb}{0.121569,0.466667,0.705882}%
\pgfsetstrokecolor{currentstroke}%
\pgfsetstrokeopacity{0.639770}%
\pgfsetdash{}{0pt}%
\pgfpathmoveto{\pgfqpoint{0.769598in}{1.401189in}}%
\pgfpathcurveto{\pgfqpoint{0.777835in}{1.401189in}}{\pgfqpoint{0.785735in}{1.404461in}}{\pgfqpoint{0.791559in}{1.410285in}}%
\pgfpathcurveto{\pgfqpoint{0.797383in}{1.416109in}}{\pgfqpoint{0.800655in}{1.424009in}}{\pgfqpoint{0.800655in}{1.432246in}}%
\pgfpathcurveto{\pgfqpoint{0.800655in}{1.440482in}}{\pgfqpoint{0.797383in}{1.448382in}}{\pgfqpoint{0.791559in}{1.454206in}}%
\pgfpathcurveto{\pgfqpoint{0.785735in}{1.460030in}}{\pgfqpoint{0.777835in}{1.463302in}}{\pgfqpoint{0.769598in}{1.463302in}}%
\pgfpathcurveto{\pgfqpoint{0.761362in}{1.463302in}}{\pgfqpoint{0.753462in}{1.460030in}}{\pgfqpoint{0.747638in}{1.454206in}}%
\pgfpathcurveto{\pgfqpoint{0.741814in}{1.448382in}}{\pgfqpoint{0.738542in}{1.440482in}}{\pgfqpoint{0.738542in}{1.432246in}}%
\pgfpathcurveto{\pgfqpoint{0.738542in}{1.424009in}}{\pgfqpoint{0.741814in}{1.416109in}}{\pgfqpoint{0.747638in}{1.410285in}}%
\pgfpathcurveto{\pgfqpoint{0.753462in}{1.404461in}}{\pgfqpoint{0.761362in}{1.401189in}}{\pgfqpoint{0.769598in}{1.401189in}}%
\pgfpathclose%
\pgfusepath{stroke,fill}%
\end{pgfscope}%
\begin{pgfscope}%
\pgfpathrectangle{\pgfqpoint{0.100000in}{0.212622in}}{\pgfqpoint{3.696000in}{3.696000in}}%
\pgfusepath{clip}%
\pgfsetbuttcap%
\pgfsetroundjoin%
\definecolor{currentfill}{rgb}{0.121569,0.466667,0.705882}%
\pgfsetfillcolor{currentfill}%
\pgfsetfillopacity{0.639770}%
\pgfsetlinewidth{1.003750pt}%
\definecolor{currentstroke}{rgb}{0.121569,0.466667,0.705882}%
\pgfsetstrokecolor{currentstroke}%
\pgfsetstrokeopacity{0.639770}%
\pgfsetdash{}{0pt}%
\pgfpathmoveto{\pgfqpoint{0.769598in}{1.401189in}}%
\pgfpathcurveto{\pgfqpoint{0.777835in}{1.401189in}}{\pgfqpoint{0.785735in}{1.404461in}}{\pgfqpoint{0.791559in}{1.410285in}}%
\pgfpathcurveto{\pgfqpoint{0.797383in}{1.416109in}}{\pgfqpoint{0.800655in}{1.424009in}}{\pgfqpoint{0.800655in}{1.432246in}}%
\pgfpathcurveto{\pgfqpoint{0.800655in}{1.440482in}}{\pgfqpoint{0.797383in}{1.448382in}}{\pgfqpoint{0.791559in}{1.454206in}}%
\pgfpathcurveto{\pgfqpoint{0.785735in}{1.460030in}}{\pgfqpoint{0.777835in}{1.463302in}}{\pgfqpoint{0.769598in}{1.463302in}}%
\pgfpathcurveto{\pgfqpoint{0.761362in}{1.463302in}}{\pgfqpoint{0.753462in}{1.460030in}}{\pgfqpoint{0.747638in}{1.454206in}}%
\pgfpathcurveto{\pgfqpoint{0.741814in}{1.448382in}}{\pgfqpoint{0.738542in}{1.440482in}}{\pgfqpoint{0.738542in}{1.432246in}}%
\pgfpathcurveto{\pgfqpoint{0.738542in}{1.424009in}}{\pgfqpoint{0.741814in}{1.416109in}}{\pgfqpoint{0.747638in}{1.410285in}}%
\pgfpathcurveto{\pgfqpoint{0.753462in}{1.404461in}}{\pgfqpoint{0.761362in}{1.401189in}}{\pgfqpoint{0.769598in}{1.401189in}}%
\pgfpathclose%
\pgfusepath{stroke,fill}%
\end{pgfscope}%
\begin{pgfscope}%
\pgfpathrectangle{\pgfqpoint{0.100000in}{0.212622in}}{\pgfqpoint{3.696000in}{3.696000in}}%
\pgfusepath{clip}%
\pgfsetbuttcap%
\pgfsetroundjoin%
\definecolor{currentfill}{rgb}{0.121569,0.466667,0.705882}%
\pgfsetfillcolor{currentfill}%
\pgfsetfillopacity{0.639770}%
\pgfsetlinewidth{1.003750pt}%
\definecolor{currentstroke}{rgb}{0.121569,0.466667,0.705882}%
\pgfsetstrokecolor{currentstroke}%
\pgfsetstrokeopacity{0.639770}%
\pgfsetdash{}{0pt}%
\pgfpathmoveto{\pgfqpoint{0.769598in}{1.401189in}}%
\pgfpathcurveto{\pgfqpoint{0.777835in}{1.401189in}}{\pgfqpoint{0.785735in}{1.404461in}}{\pgfqpoint{0.791559in}{1.410285in}}%
\pgfpathcurveto{\pgfqpoint{0.797383in}{1.416109in}}{\pgfqpoint{0.800655in}{1.424009in}}{\pgfqpoint{0.800655in}{1.432246in}}%
\pgfpathcurveto{\pgfqpoint{0.800655in}{1.440482in}}{\pgfqpoint{0.797383in}{1.448382in}}{\pgfqpoint{0.791559in}{1.454206in}}%
\pgfpathcurveto{\pgfqpoint{0.785735in}{1.460030in}}{\pgfqpoint{0.777835in}{1.463302in}}{\pgfqpoint{0.769598in}{1.463302in}}%
\pgfpathcurveto{\pgfqpoint{0.761362in}{1.463302in}}{\pgfqpoint{0.753462in}{1.460030in}}{\pgfqpoint{0.747638in}{1.454206in}}%
\pgfpathcurveto{\pgfqpoint{0.741814in}{1.448382in}}{\pgfqpoint{0.738542in}{1.440482in}}{\pgfqpoint{0.738542in}{1.432246in}}%
\pgfpathcurveto{\pgfqpoint{0.738542in}{1.424009in}}{\pgfqpoint{0.741814in}{1.416109in}}{\pgfqpoint{0.747638in}{1.410285in}}%
\pgfpathcurveto{\pgfqpoint{0.753462in}{1.404461in}}{\pgfqpoint{0.761362in}{1.401189in}}{\pgfqpoint{0.769598in}{1.401189in}}%
\pgfpathclose%
\pgfusepath{stroke,fill}%
\end{pgfscope}%
\begin{pgfscope}%
\pgfpathrectangle{\pgfqpoint{0.100000in}{0.212622in}}{\pgfqpoint{3.696000in}{3.696000in}}%
\pgfusepath{clip}%
\pgfsetbuttcap%
\pgfsetroundjoin%
\definecolor{currentfill}{rgb}{0.121569,0.466667,0.705882}%
\pgfsetfillcolor{currentfill}%
\pgfsetfillopacity{0.639770}%
\pgfsetlinewidth{1.003750pt}%
\definecolor{currentstroke}{rgb}{0.121569,0.466667,0.705882}%
\pgfsetstrokecolor{currentstroke}%
\pgfsetstrokeopacity{0.639770}%
\pgfsetdash{}{0pt}%
\pgfpathmoveto{\pgfqpoint{0.769598in}{1.401189in}}%
\pgfpathcurveto{\pgfqpoint{0.777835in}{1.401189in}}{\pgfqpoint{0.785735in}{1.404461in}}{\pgfqpoint{0.791559in}{1.410285in}}%
\pgfpathcurveto{\pgfqpoint{0.797383in}{1.416109in}}{\pgfqpoint{0.800655in}{1.424009in}}{\pgfqpoint{0.800655in}{1.432246in}}%
\pgfpathcurveto{\pgfqpoint{0.800655in}{1.440482in}}{\pgfqpoint{0.797383in}{1.448382in}}{\pgfqpoint{0.791559in}{1.454206in}}%
\pgfpathcurveto{\pgfqpoint{0.785735in}{1.460030in}}{\pgfqpoint{0.777835in}{1.463302in}}{\pgfqpoint{0.769598in}{1.463302in}}%
\pgfpathcurveto{\pgfqpoint{0.761362in}{1.463302in}}{\pgfqpoint{0.753462in}{1.460030in}}{\pgfqpoint{0.747638in}{1.454206in}}%
\pgfpathcurveto{\pgfqpoint{0.741814in}{1.448382in}}{\pgfqpoint{0.738542in}{1.440482in}}{\pgfqpoint{0.738542in}{1.432246in}}%
\pgfpathcurveto{\pgfqpoint{0.738542in}{1.424009in}}{\pgfqpoint{0.741814in}{1.416109in}}{\pgfqpoint{0.747638in}{1.410285in}}%
\pgfpathcurveto{\pgfqpoint{0.753462in}{1.404461in}}{\pgfqpoint{0.761362in}{1.401189in}}{\pgfqpoint{0.769598in}{1.401189in}}%
\pgfpathclose%
\pgfusepath{stroke,fill}%
\end{pgfscope}%
\begin{pgfscope}%
\pgfpathrectangle{\pgfqpoint{0.100000in}{0.212622in}}{\pgfqpoint{3.696000in}{3.696000in}}%
\pgfusepath{clip}%
\pgfsetbuttcap%
\pgfsetroundjoin%
\definecolor{currentfill}{rgb}{0.121569,0.466667,0.705882}%
\pgfsetfillcolor{currentfill}%
\pgfsetfillopacity{0.639770}%
\pgfsetlinewidth{1.003750pt}%
\definecolor{currentstroke}{rgb}{0.121569,0.466667,0.705882}%
\pgfsetstrokecolor{currentstroke}%
\pgfsetstrokeopacity{0.639770}%
\pgfsetdash{}{0pt}%
\pgfpathmoveto{\pgfqpoint{0.769598in}{1.401189in}}%
\pgfpathcurveto{\pgfqpoint{0.777835in}{1.401189in}}{\pgfqpoint{0.785735in}{1.404461in}}{\pgfqpoint{0.791559in}{1.410285in}}%
\pgfpathcurveto{\pgfqpoint{0.797383in}{1.416109in}}{\pgfqpoint{0.800655in}{1.424009in}}{\pgfqpoint{0.800655in}{1.432246in}}%
\pgfpathcurveto{\pgfqpoint{0.800655in}{1.440482in}}{\pgfqpoint{0.797383in}{1.448382in}}{\pgfqpoint{0.791559in}{1.454206in}}%
\pgfpathcurveto{\pgfqpoint{0.785735in}{1.460030in}}{\pgfqpoint{0.777835in}{1.463302in}}{\pgfqpoint{0.769598in}{1.463302in}}%
\pgfpathcurveto{\pgfqpoint{0.761362in}{1.463302in}}{\pgfqpoint{0.753462in}{1.460030in}}{\pgfqpoint{0.747638in}{1.454206in}}%
\pgfpathcurveto{\pgfqpoint{0.741814in}{1.448382in}}{\pgfqpoint{0.738542in}{1.440482in}}{\pgfqpoint{0.738542in}{1.432246in}}%
\pgfpathcurveto{\pgfqpoint{0.738542in}{1.424009in}}{\pgfqpoint{0.741814in}{1.416109in}}{\pgfqpoint{0.747638in}{1.410285in}}%
\pgfpathcurveto{\pgfqpoint{0.753462in}{1.404461in}}{\pgfqpoint{0.761362in}{1.401189in}}{\pgfqpoint{0.769598in}{1.401189in}}%
\pgfpathclose%
\pgfusepath{stroke,fill}%
\end{pgfscope}%
\begin{pgfscope}%
\pgfpathrectangle{\pgfqpoint{0.100000in}{0.212622in}}{\pgfqpoint{3.696000in}{3.696000in}}%
\pgfusepath{clip}%
\pgfsetbuttcap%
\pgfsetroundjoin%
\definecolor{currentfill}{rgb}{0.121569,0.466667,0.705882}%
\pgfsetfillcolor{currentfill}%
\pgfsetfillopacity{0.639770}%
\pgfsetlinewidth{1.003750pt}%
\definecolor{currentstroke}{rgb}{0.121569,0.466667,0.705882}%
\pgfsetstrokecolor{currentstroke}%
\pgfsetstrokeopacity{0.639770}%
\pgfsetdash{}{0pt}%
\pgfpathmoveto{\pgfqpoint{0.769598in}{1.401189in}}%
\pgfpathcurveto{\pgfqpoint{0.777835in}{1.401189in}}{\pgfqpoint{0.785735in}{1.404461in}}{\pgfqpoint{0.791559in}{1.410285in}}%
\pgfpathcurveto{\pgfqpoint{0.797383in}{1.416109in}}{\pgfqpoint{0.800655in}{1.424009in}}{\pgfqpoint{0.800655in}{1.432246in}}%
\pgfpathcurveto{\pgfqpoint{0.800655in}{1.440482in}}{\pgfqpoint{0.797383in}{1.448382in}}{\pgfqpoint{0.791559in}{1.454206in}}%
\pgfpathcurveto{\pgfqpoint{0.785735in}{1.460030in}}{\pgfqpoint{0.777835in}{1.463302in}}{\pgfqpoint{0.769598in}{1.463302in}}%
\pgfpathcurveto{\pgfqpoint{0.761362in}{1.463302in}}{\pgfqpoint{0.753462in}{1.460030in}}{\pgfqpoint{0.747638in}{1.454206in}}%
\pgfpathcurveto{\pgfqpoint{0.741814in}{1.448382in}}{\pgfqpoint{0.738542in}{1.440482in}}{\pgfqpoint{0.738542in}{1.432246in}}%
\pgfpathcurveto{\pgfqpoint{0.738542in}{1.424009in}}{\pgfqpoint{0.741814in}{1.416109in}}{\pgfqpoint{0.747638in}{1.410285in}}%
\pgfpathcurveto{\pgfqpoint{0.753462in}{1.404461in}}{\pgfqpoint{0.761362in}{1.401189in}}{\pgfqpoint{0.769598in}{1.401189in}}%
\pgfpathclose%
\pgfusepath{stroke,fill}%
\end{pgfscope}%
\begin{pgfscope}%
\pgfpathrectangle{\pgfqpoint{0.100000in}{0.212622in}}{\pgfqpoint{3.696000in}{3.696000in}}%
\pgfusepath{clip}%
\pgfsetbuttcap%
\pgfsetroundjoin%
\definecolor{currentfill}{rgb}{0.121569,0.466667,0.705882}%
\pgfsetfillcolor{currentfill}%
\pgfsetfillopacity{0.639770}%
\pgfsetlinewidth{1.003750pt}%
\definecolor{currentstroke}{rgb}{0.121569,0.466667,0.705882}%
\pgfsetstrokecolor{currentstroke}%
\pgfsetstrokeopacity{0.639770}%
\pgfsetdash{}{0pt}%
\pgfpathmoveto{\pgfqpoint{0.769598in}{1.401189in}}%
\pgfpathcurveto{\pgfqpoint{0.777835in}{1.401189in}}{\pgfqpoint{0.785735in}{1.404461in}}{\pgfqpoint{0.791559in}{1.410285in}}%
\pgfpathcurveto{\pgfqpoint{0.797383in}{1.416109in}}{\pgfqpoint{0.800655in}{1.424009in}}{\pgfqpoint{0.800655in}{1.432246in}}%
\pgfpathcurveto{\pgfqpoint{0.800655in}{1.440482in}}{\pgfqpoint{0.797383in}{1.448382in}}{\pgfqpoint{0.791559in}{1.454206in}}%
\pgfpathcurveto{\pgfqpoint{0.785735in}{1.460030in}}{\pgfqpoint{0.777835in}{1.463302in}}{\pgfqpoint{0.769598in}{1.463302in}}%
\pgfpathcurveto{\pgfqpoint{0.761362in}{1.463302in}}{\pgfqpoint{0.753462in}{1.460030in}}{\pgfqpoint{0.747638in}{1.454206in}}%
\pgfpathcurveto{\pgfqpoint{0.741814in}{1.448382in}}{\pgfqpoint{0.738542in}{1.440482in}}{\pgfqpoint{0.738542in}{1.432246in}}%
\pgfpathcurveto{\pgfqpoint{0.738542in}{1.424009in}}{\pgfqpoint{0.741814in}{1.416109in}}{\pgfqpoint{0.747638in}{1.410285in}}%
\pgfpathcurveto{\pgfqpoint{0.753462in}{1.404461in}}{\pgfqpoint{0.761362in}{1.401189in}}{\pgfqpoint{0.769598in}{1.401189in}}%
\pgfpathclose%
\pgfusepath{stroke,fill}%
\end{pgfscope}%
\begin{pgfscope}%
\pgfpathrectangle{\pgfqpoint{0.100000in}{0.212622in}}{\pgfqpoint{3.696000in}{3.696000in}}%
\pgfusepath{clip}%
\pgfsetbuttcap%
\pgfsetroundjoin%
\definecolor{currentfill}{rgb}{0.121569,0.466667,0.705882}%
\pgfsetfillcolor{currentfill}%
\pgfsetfillopacity{0.639770}%
\pgfsetlinewidth{1.003750pt}%
\definecolor{currentstroke}{rgb}{0.121569,0.466667,0.705882}%
\pgfsetstrokecolor{currentstroke}%
\pgfsetstrokeopacity{0.639770}%
\pgfsetdash{}{0pt}%
\pgfpathmoveto{\pgfqpoint{0.769598in}{1.401189in}}%
\pgfpathcurveto{\pgfqpoint{0.777835in}{1.401189in}}{\pgfqpoint{0.785735in}{1.404461in}}{\pgfqpoint{0.791559in}{1.410285in}}%
\pgfpathcurveto{\pgfqpoint{0.797383in}{1.416109in}}{\pgfqpoint{0.800655in}{1.424009in}}{\pgfqpoint{0.800655in}{1.432246in}}%
\pgfpathcurveto{\pgfqpoint{0.800655in}{1.440482in}}{\pgfqpoint{0.797383in}{1.448382in}}{\pgfqpoint{0.791559in}{1.454206in}}%
\pgfpathcurveto{\pgfqpoint{0.785735in}{1.460030in}}{\pgfqpoint{0.777835in}{1.463302in}}{\pgfqpoint{0.769598in}{1.463302in}}%
\pgfpathcurveto{\pgfqpoint{0.761362in}{1.463302in}}{\pgfqpoint{0.753462in}{1.460030in}}{\pgfqpoint{0.747638in}{1.454206in}}%
\pgfpathcurveto{\pgfqpoint{0.741814in}{1.448382in}}{\pgfqpoint{0.738542in}{1.440482in}}{\pgfqpoint{0.738542in}{1.432246in}}%
\pgfpathcurveto{\pgfqpoint{0.738542in}{1.424009in}}{\pgfqpoint{0.741814in}{1.416109in}}{\pgfqpoint{0.747638in}{1.410285in}}%
\pgfpathcurveto{\pgfqpoint{0.753462in}{1.404461in}}{\pgfqpoint{0.761362in}{1.401189in}}{\pgfqpoint{0.769598in}{1.401189in}}%
\pgfpathclose%
\pgfusepath{stroke,fill}%
\end{pgfscope}%
\begin{pgfscope}%
\pgfpathrectangle{\pgfqpoint{0.100000in}{0.212622in}}{\pgfqpoint{3.696000in}{3.696000in}}%
\pgfusepath{clip}%
\pgfsetbuttcap%
\pgfsetroundjoin%
\definecolor{currentfill}{rgb}{0.121569,0.466667,0.705882}%
\pgfsetfillcolor{currentfill}%
\pgfsetfillopacity{0.639770}%
\pgfsetlinewidth{1.003750pt}%
\definecolor{currentstroke}{rgb}{0.121569,0.466667,0.705882}%
\pgfsetstrokecolor{currentstroke}%
\pgfsetstrokeopacity{0.639770}%
\pgfsetdash{}{0pt}%
\pgfpathmoveto{\pgfqpoint{0.769598in}{1.401189in}}%
\pgfpathcurveto{\pgfqpoint{0.777835in}{1.401189in}}{\pgfqpoint{0.785735in}{1.404461in}}{\pgfqpoint{0.791559in}{1.410285in}}%
\pgfpathcurveto{\pgfqpoint{0.797383in}{1.416109in}}{\pgfqpoint{0.800655in}{1.424009in}}{\pgfqpoint{0.800655in}{1.432246in}}%
\pgfpathcurveto{\pgfqpoint{0.800655in}{1.440482in}}{\pgfqpoint{0.797383in}{1.448382in}}{\pgfqpoint{0.791559in}{1.454206in}}%
\pgfpathcurveto{\pgfqpoint{0.785735in}{1.460030in}}{\pgfqpoint{0.777835in}{1.463302in}}{\pgfqpoint{0.769598in}{1.463302in}}%
\pgfpathcurveto{\pgfqpoint{0.761362in}{1.463302in}}{\pgfqpoint{0.753462in}{1.460030in}}{\pgfqpoint{0.747638in}{1.454206in}}%
\pgfpathcurveto{\pgfqpoint{0.741814in}{1.448382in}}{\pgfqpoint{0.738542in}{1.440482in}}{\pgfqpoint{0.738542in}{1.432246in}}%
\pgfpathcurveto{\pgfqpoint{0.738542in}{1.424009in}}{\pgfqpoint{0.741814in}{1.416109in}}{\pgfqpoint{0.747638in}{1.410285in}}%
\pgfpathcurveto{\pgfqpoint{0.753462in}{1.404461in}}{\pgfqpoint{0.761362in}{1.401189in}}{\pgfqpoint{0.769598in}{1.401189in}}%
\pgfpathclose%
\pgfusepath{stroke,fill}%
\end{pgfscope}%
\begin{pgfscope}%
\pgfpathrectangle{\pgfqpoint{0.100000in}{0.212622in}}{\pgfqpoint{3.696000in}{3.696000in}}%
\pgfusepath{clip}%
\pgfsetbuttcap%
\pgfsetroundjoin%
\definecolor{currentfill}{rgb}{0.121569,0.466667,0.705882}%
\pgfsetfillcolor{currentfill}%
\pgfsetfillopacity{0.639770}%
\pgfsetlinewidth{1.003750pt}%
\definecolor{currentstroke}{rgb}{0.121569,0.466667,0.705882}%
\pgfsetstrokecolor{currentstroke}%
\pgfsetstrokeopacity{0.639770}%
\pgfsetdash{}{0pt}%
\pgfpathmoveto{\pgfqpoint{0.769598in}{1.401189in}}%
\pgfpathcurveto{\pgfqpoint{0.777835in}{1.401189in}}{\pgfqpoint{0.785735in}{1.404461in}}{\pgfqpoint{0.791559in}{1.410285in}}%
\pgfpathcurveto{\pgfqpoint{0.797383in}{1.416109in}}{\pgfqpoint{0.800655in}{1.424009in}}{\pgfqpoint{0.800655in}{1.432246in}}%
\pgfpathcurveto{\pgfqpoint{0.800655in}{1.440482in}}{\pgfqpoint{0.797383in}{1.448382in}}{\pgfqpoint{0.791559in}{1.454206in}}%
\pgfpathcurveto{\pgfqpoint{0.785735in}{1.460030in}}{\pgfqpoint{0.777835in}{1.463302in}}{\pgfqpoint{0.769598in}{1.463302in}}%
\pgfpathcurveto{\pgfqpoint{0.761362in}{1.463302in}}{\pgfqpoint{0.753462in}{1.460030in}}{\pgfqpoint{0.747638in}{1.454206in}}%
\pgfpathcurveto{\pgfqpoint{0.741814in}{1.448382in}}{\pgfqpoint{0.738542in}{1.440482in}}{\pgfqpoint{0.738542in}{1.432246in}}%
\pgfpathcurveto{\pgfqpoint{0.738542in}{1.424009in}}{\pgfqpoint{0.741814in}{1.416109in}}{\pgfqpoint{0.747638in}{1.410285in}}%
\pgfpathcurveto{\pgfqpoint{0.753462in}{1.404461in}}{\pgfqpoint{0.761362in}{1.401189in}}{\pgfqpoint{0.769598in}{1.401189in}}%
\pgfpathclose%
\pgfusepath{stroke,fill}%
\end{pgfscope}%
\begin{pgfscope}%
\pgfpathrectangle{\pgfqpoint{0.100000in}{0.212622in}}{\pgfqpoint{3.696000in}{3.696000in}}%
\pgfusepath{clip}%
\pgfsetbuttcap%
\pgfsetroundjoin%
\definecolor{currentfill}{rgb}{0.121569,0.466667,0.705882}%
\pgfsetfillcolor{currentfill}%
\pgfsetfillopacity{0.639770}%
\pgfsetlinewidth{1.003750pt}%
\definecolor{currentstroke}{rgb}{0.121569,0.466667,0.705882}%
\pgfsetstrokecolor{currentstroke}%
\pgfsetstrokeopacity{0.639770}%
\pgfsetdash{}{0pt}%
\pgfpathmoveto{\pgfqpoint{0.769598in}{1.401189in}}%
\pgfpathcurveto{\pgfqpoint{0.777835in}{1.401189in}}{\pgfqpoint{0.785735in}{1.404461in}}{\pgfqpoint{0.791559in}{1.410285in}}%
\pgfpathcurveto{\pgfqpoint{0.797383in}{1.416109in}}{\pgfqpoint{0.800655in}{1.424009in}}{\pgfqpoint{0.800655in}{1.432246in}}%
\pgfpathcurveto{\pgfqpoint{0.800655in}{1.440482in}}{\pgfqpoint{0.797383in}{1.448382in}}{\pgfqpoint{0.791559in}{1.454206in}}%
\pgfpathcurveto{\pgfqpoint{0.785735in}{1.460030in}}{\pgfqpoint{0.777835in}{1.463302in}}{\pgfqpoint{0.769598in}{1.463302in}}%
\pgfpathcurveto{\pgfqpoint{0.761362in}{1.463302in}}{\pgfqpoint{0.753462in}{1.460030in}}{\pgfqpoint{0.747638in}{1.454206in}}%
\pgfpathcurveto{\pgfqpoint{0.741814in}{1.448382in}}{\pgfqpoint{0.738542in}{1.440482in}}{\pgfqpoint{0.738542in}{1.432246in}}%
\pgfpathcurveto{\pgfqpoint{0.738542in}{1.424009in}}{\pgfqpoint{0.741814in}{1.416109in}}{\pgfqpoint{0.747638in}{1.410285in}}%
\pgfpathcurveto{\pgfqpoint{0.753462in}{1.404461in}}{\pgfqpoint{0.761362in}{1.401189in}}{\pgfqpoint{0.769598in}{1.401189in}}%
\pgfpathclose%
\pgfusepath{stroke,fill}%
\end{pgfscope}%
\begin{pgfscope}%
\pgfpathrectangle{\pgfqpoint{0.100000in}{0.212622in}}{\pgfqpoint{3.696000in}{3.696000in}}%
\pgfusepath{clip}%
\pgfsetbuttcap%
\pgfsetroundjoin%
\definecolor{currentfill}{rgb}{0.121569,0.466667,0.705882}%
\pgfsetfillcolor{currentfill}%
\pgfsetfillopacity{0.639770}%
\pgfsetlinewidth{1.003750pt}%
\definecolor{currentstroke}{rgb}{0.121569,0.466667,0.705882}%
\pgfsetstrokecolor{currentstroke}%
\pgfsetstrokeopacity{0.639770}%
\pgfsetdash{}{0pt}%
\pgfpathmoveto{\pgfqpoint{0.769598in}{1.401189in}}%
\pgfpathcurveto{\pgfqpoint{0.777835in}{1.401189in}}{\pgfqpoint{0.785735in}{1.404461in}}{\pgfqpoint{0.791559in}{1.410285in}}%
\pgfpathcurveto{\pgfqpoint{0.797383in}{1.416109in}}{\pgfqpoint{0.800655in}{1.424009in}}{\pgfqpoint{0.800655in}{1.432246in}}%
\pgfpathcurveto{\pgfqpoint{0.800655in}{1.440482in}}{\pgfqpoint{0.797383in}{1.448382in}}{\pgfqpoint{0.791559in}{1.454206in}}%
\pgfpathcurveto{\pgfqpoint{0.785735in}{1.460030in}}{\pgfqpoint{0.777835in}{1.463302in}}{\pgfqpoint{0.769598in}{1.463302in}}%
\pgfpathcurveto{\pgfqpoint{0.761362in}{1.463302in}}{\pgfqpoint{0.753462in}{1.460030in}}{\pgfqpoint{0.747638in}{1.454206in}}%
\pgfpathcurveto{\pgfqpoint{0.741814in}{1.448382in}}{\pgfqpoint{0.738542in}{1.440482in}}{\pgfqpoint{0.738542in}{1.432246in}}%
\pgfpathcurveto{\pgfqpoint{0.738542in}{1.424009in}}{\pgfqpoint{0.741814in}{1.416109in}}{\pgfqpoint{0.747638in}{1.410285in}}%
\pgfpathcurveto{\pgfqpoint{0.753462in}{1.404461in}}{\pgfqpoint{0.761362in}{1.401189in}}{\pgfqpoint{0.769598in}{1.401189in}}%
\pgfpathclose%
\pgfusepath{stroke,fill}%
\end{pgfscope}%
\begin{pgfscope}%
\pgfpathrectangle{\pgfqpoint{0.100000in}{0.212622in}}{\pgfqpoint{3.696000in}{3.696000in}}%
\pgfusepath{clip}%
\pgfsetbuttcap%
\pgfsetroundjoin%
\definecolor{currentfill}{rgb}{0.121569,0.466667,0.705882}%
\pgfsetfillcolor{currentfill}%
\pgfsetfillopacity{0.639770}%
\pgfsetlinewidth{1.003750pt}%
\definecolor{currentstroke}{rgb}{0.121569,0.466667,0.705882}%
\pgfsetstrokecolor{currentstroke}%
\pgfsetstrokeopacity{0.639770}%
\pgfsetdash{}{0pt}%
\pgfpathmoveto{\pgfqpoint{0.769598in}{1.401189in}}%
\pgfpathcurveto{\pgfqpoint{0.777835in}{1.401189in}}{\pgfqpoint{0.785735in}{1.404461in}}{\pgfqpoint{0.791559in}{1.410285in}}%
\pgfpathcurveto{\pgfqpoint{0.797383in}{1.416109in}}{\pgfqpoint{0.800655in}{1.424009in}}{\pgfqpoint{0.800655in}{1.432246in}}%
\pgfpathcurveto{\pgfqpoint{0.800655in}{1.440482in}}{\pgfqpoint{0.797383in}{1.448382in}}{\pgfqpoint{0.791559in}{1.454206in}}%
\pgfpathcurveto{\pgfqpoint{0.785735in}{1.460030in}}{\pgfqpoint{0.777835in}{1.463302in}}{\pgfqpoint{0.769598in}{1.463302in}}%
\pgfpathcurveto{\pgfqpoint{0.761362in}{1.463302in}}{\pgfqpoint{0.753462in}{1.460030in}}{\pgfqpoint{0.747638in}{1.454206in}}%
\pgfpathcurveto{\pgfqpoint{0.741814in}{1.448382in}}{\pgfqpoint{0.738542in}{1.440482in}}{\pgfqpoint{0.738542in}{1.432246in}}%
\pgfpathcurveto{\pgfqpoint{0.738542in}{1.424009in}}{\pgfqpoint{0.741814in}{1.416109in}}{\pgfqpoint{0.747638in}{1.410285in}}%
\pgfpathcurveto{\pgfqpoint{0.753462in}{1.404461in}}{\pgfqpoint{0.761362in}{1.401189in}}{\pgfqpoint{0.769598in}{1.401189in}}%
\pgfpathclose%
\pgfusepath{stroke,fill}%
\end{pgfscope}%
\begin{pgfscope}%
\pgfpathrectangle{\pgfqpoint{0.100000in}{0.212622in}}{\pgfqpoint{3.696000in}{3.696000in}}%
\pgfusepath{clip}%
\pgfsetbuttcap%
\pgfsetroundjoin%
\definecolor{currentfill}{rgb}{0.121569,0.466667,0.705882}%
\pgfsetfillcolor{currentfill}%
\pgfsetfillopacity{0.639770}%
\pgfsetlinewidth{1.003750pt}%
\definecolor{currentstroke}{rgb}{0.121569,0.466667,0.705882}%
\pgfsetstrokecolor{currentstroke}%
\pgfsetstrokeopacity{0.639770}%
\pgfsetdash{}{0pt}%
\pgfpathmoveto{\pgfqpoint{0.769598in}{1.401189in}}%
\pgfpathcurveto{\pgfqpoint{0.777835in}{1.401189in}}{\pgfqpoint{0.785735in}{1.404461in}}{\pgfqpoint{0.791559in}{1.410285in}}%
\pgfpathcurveto{\pgfqpoint{0.797383in}{1.416109in}}{\pgfqpoint{0.800655in}{1.424009in}}{\pgfqpoint{0.800655in}{1.432246in}}%
\pgfpathcurveto{\pgfqpoint{0.800655in}{1.440482in}}{\pgfqpoint{0.797383in}{1.448382in}}{\pgfqpoint{0.791559in}{1.454206in}}%
\pgfpathcurveto{\pgfqpoint{0.785735in}{1.460030in}}{\pgfqpoint{0.777835in}{1.463302in}}{\pgfqpoint{0.769598in}{1.463302in}}%
\pgfpathcurveto{\pgfqpoint{0.761362in}{1.463302in}}{\pgfqpoint{0.753462in}{1.460030in}}{\pgfqpoint{0.747638in}{1.454206in}}%
\pgfpathcurveto{\pgfqpoint{0.741814in}{1.448382in}}{\pgfqpoint{0.738542in}{1.440482in}}{\pgfqpoint{0.738542in}{1.432246in}}%
\pgfpathcurveto{\pgfqpoint{0.738542in}{1.424009in}}{\pgfqpoint{0.741814in}{1.416109in}}{\pgfqpoint{0.747638in}{1.410285in}}%
\pgfpathcurveto{\pgfqpoint{0.753462in}{1.404461in}}{\pgfqpoint{0.761362in}{1.401189in}}{\pgfqpoint{0.769598in}{1.401189in}}%
\pgfpathclose%
\pgfusepath{stroke,fill}%
\end{pgfscope}%
\begin{pgfscope}%
\pgfpathrectangle{\pgfqpoint{0.100000in}{0.212622in}}{\pgfqpoint{3.696000in}{3.696000in}}%
\pgfusepath{clip}%
\pgfsetbuttcap%
\pgfsetroundjoin%
\definecolor{currentfill}{rgb}{0.121569,0.466667,0.705882}%
\pgfsetfillcolor{currentfill}%
\pgfsetfillopacity{0.639770}%
\pgfsetlinewidth{1.003750pt}%
\definecolor{currentstroke}{rgb}{0.121569,0.466667,0.705882}%
\pgfsetstrokecolor{currentstroke}%
\pgfsetstrokeopacity{0.639770}%
\pgfsetdash{}{0pt}%
\pgfpathmoveto{\pgfqpoint{0.769598in}{1.401189in}}%
\pgfpathcurveto{\pgfqpoint{0.777835in}{1.401189in}}{\pgfqpoint{0.785735in}{1.404461in}}{\pgfqpoint{0.791559in}{1.410285in}}%
\pgfpathcurveto{\pgfqpoint{0.797383in}{1.416109in}}{\pgfqpoint{0.800655in}{1.424009in}}{\pgfqpoint{0.800655in}{1.432246in}}%
\pgfpathcurveto{\pgfqpoint{0.800655in}{1.440482in}}{\pgfqpoint{0.797383in}{1.448382in}}{\pgfqpoint{0.791559in}{1.454206in}}%
\pgfpathcurveto{\pgfqpoint{0.785735in}{1.460030in}}{\pgfqpoint{0.777835in}{1.463302in}}{\pgfqpoint{0.769598in}{1.463302in}}%
\pgfpathcurveto{\pgfqpoint{0.761362in}{1.463302in}}{\pgfqpoint{0.753462in}{1.460030in}}{\pgfqpoint{0.747638in}{1.454206in}}%
\pgfpathcurveto{\pgfqpoint{0.741814in}{1.448382in}}{\pgfqpoint{0.738542in}{1.440482in}}{\pgfqpoint{0.738542in}{1.432246in}}%
\pgfpathcurveto{\pgfqpoint{0.738542in}{1.424009in}}{\pgfqpoint{0.741814in}{1.416109in}}{\pgfqpoint{0.747638in}{1.410285in}}%
\pgfpathcurveto{\pgfqpoint{0.753462in}{1.404461in}}{\pgfqpoint{0.761362in}{1.401189in}}{\pgfqpoint{0.769598in}{1.401189in}}%
\pgfpathclose%
\pgfusepath{stroke,fill}%
\end{pgfscope}%
\begin{pgfscope}%
\pgfpathrectangle{\pgfqpoint{0.100000in}{0.212622in}}{\pgfqpoint{3.696000in}{3.696000in}}%
\pgfusepath{clip}%
\pgfsetbuttcap%
\pgfsetroundjoin%
\definecolor{currentfill}{rgb}{0.121569,0.466667,0.705882}%
\pgfsetfillcolor{currentfill}%
\pgfsetfillopacity{0.639770}%
\pgfsetlinewidth{1.003750pt}%
\definecolor{currentstroke}{rgb}{0.121569,0.466667,0.705882}%
\pgfsetstrokecolor{currentstroke}%
\pgfsetstrokeopacity{0.639770}%
\pgfsetdash{}{0pt}%
\pgfpathmoveto{\pgfqpoint{0.769598in}{1.401189in}}%
\pgfpathcurveto{\pgfqpoint{0.777835in}{1.401189in}}{\pgfqpoint{0.785735in}{1.404461in}}{\pgfqpoint{0.791559in}{1.410285in}}%
\pgfpathcurveto{\pgfqpoint{0.797383in}{1.416109in}}{\pgfqpoint{0.800655in}{1.424009in}}{\pgfqpoint{0.800655in}{1.432246in}}%
\pgfpathcurveto{\pgfqpoint{0.800655in}{1.440482in}}{\pgfqpoint{0.797383in}{1.448382in}}{\pgfqpoint{0.791559in}{1.454206in}}%
\pgfpathcurveto{\pgfqpoint{0.785735in}{1.460030in}}{\pgfqpoint{0.777835in}{1.463302in}}{\pgfqpoint{0.769598in}{1.463302in}}%
\pgfpathcurveto{\pgfqpoint{0.761362in}{1.463302in}}{\pgfqpoint{0.753462in}{1.460030in}}{\pgfqpoint{0.747638in}{1.454206in}}%
\pgfpathcurveto{\pgfqpoint{0.741814in}{1.448382in}}{\pgfqpoint{0.738542in}{1.440482in}}{\pgfqpoint{0.738542in}{1.432246in}}%
\pgfpathcurveto{\pgfqpoint{0.738542in}{1.424009in}}{\pgfqpoint{0.741814in}{1.416109in}}{\pgfqpoint{0.747638in}{1.410285in}}%
\pgfpathcurveto{\pgfqpoint{0.753462in}{1.404461in}}{\pgfqpoint{0.761362in}{1.401189in}}{\pgfqpoint{0.769598in}{1.401189in}}%
\pgfpathclose%
\pgfusepath{stroke,fill}%
\end{pgfscope}%
\begin{pgfscope}%
\pgfpathrectangle{\pgfqpoint{0.100000in}{0.212622in}}{\pgfqpoint{3.696000in}{3.696000in}}%
\pgfusepath{clip}%
\pgfsetbuttcap%
\pgfsetroundjoin%
\definecolor{currentfill}{rgb}{0.121569,0.466667,0.705882}%
\pgfsetfillcolor{currentfill}%
\pgfsetfillopacity{0.639770}%
\pgfsetlinewidth{1.003750pt}%
\definecolor{currentstroke}{rgb}{0.121569,0.466667,0.705882}%
\pgfsetstrokecolor{currentstroke}%
\pgfsetstrokeopacity{0.639770}%
\pgfsetdash{}{0pt}%
\pgfpathmoveto{\pgfqpoint{0.769598in}{1.401189in}}%
\pgfpathcurveto{\pgfqpoint{0.777835in}{1.401189in}}{\pgfqpoint{0.785735in}{1.404461in}}{\pgfqpoint{0.791559in}{1.410285in}}%
\pgfpathcurveto{\pgfqpoint{0.797383in}{1.416109in}}{\pgfqpoint{0.800655in}{1.424009in}}{\pgfqpoint{0.800655in}{1.432246in}}%
\pgfpathcurveto{\pgfqpoint{0.800655in}{1.440482in}}{\pgfqpoint{0.797383in}{1.448382in}}{\pgfqpoint{0.791559in}{1.454206in}}%
\pgfpathcurveto{\pgfqpoint{0.785735in}{1.460030in}}{\pgfqpoint{0.777835in}{1.463302in}}{\pgfqpoint{0.769598in}{1.463302in}}%
\pgfpathcurveto{\pgfqpoint{0.761362in}{1.463302in}}{\pgfqpoint{0.753462in}{1.460030in}}{\pgfqpoint{0.747638in}{1.454206in}}%
\pgfpathcurveto{\pgfqpoint{0.741814in}{1.448382in}}{\pgfqpoint{0.738542in}{1.440482in}}{\pgfqpoint{0.738542in}{1.432246in}}%
\pgfpathcurveto{\pgfqpoint{0.738542in}{1.424009in}}{\pgfqpoint{0.741814in}{1.416109in}}{\pgfqpoint{0.747638in}{1.410285in}}%
\pgfpathcurveto{\pgfqpoint{0.753462in}{1.404461in}}{\pgfqpoint{0.761362in}{1.401189in}}{\pgfqpoint{0.769598in}{1.401189in}}%
\pgfpathclose%
\pgfusepath{stroke,fill}%
\end{pgfscope}%
\begin{pgfscope}%
\pgfpathrectangle{\pgfqpoint{0.100000in}{0.212622in}}{\pgfqpoint{3.696000in}{3.696000in}}%
\pgfusepath{clip}%
\pgfsetbuttcap%
\pgfsetroundjoin%
\definecolor{currentfill}{rgb}{0.121569,0.466667,0.705882}%
\pgfsetfillcolor{currentfill}%
\pgfsetfillopacity{0.639770}%
\pgfsetlinewidth{1.003750pt}%
\definecolor{currentstroke}{rgb}{0.121569,0.466667,0.705882}%
\pgfsetstrokecolor{currentstroke}%
\pgfsetstrokeopacity{0.639770}%
\pgfsetdash{}{0pt}%
\pgfpathmoveto{\pgfqpoint{0.769598in}{1.401189in}}%
\pgfpathcurveto{\pgfqpoint{0.777835in}{1.401189in}}{\pgfqpoint{0.785735in}{1.404461in}}{\pgfqpoint{0.791559in}{1.410285in}}%
\pgfpathcurveto{\pgfqpoint{0.797383in}{1.416109in}}{\pgfqpoint{0.800655in}{1.424009in}}{\pgfqpoint{0.800655in}{1.432246in}}%
\pgfpathcurveto{\pgfqpoint{0.800655in}{1.440482in}}{\pgfqpoint{0.797383in}{1.448382in}}{\pgfqpoint{0.791559in}{1.454206in}}%
\pgfpathcurveto{\pgfqpoint{0.785735in}{1.460030in}}{\pgfqpoint{0.777835in}{1.463302in}}{\pgfqpoint{0.769598in}{1.463302in}}%
\pgfpathcurveto{\pgfqpoint{0.761362in}{1.463302in}}{\pgfqpoint{0.753462in}{1.460030in}}{\pgfqpoint{0.747638in}{1.454206in}}%
\pgfpathcurveto{\pgfqpoint{0.741814in}{1.448382in}}{\pgfqpoint{0.738542in}{1.440482in}}{\pgfqpoint{0.738542in}{1.432246in}}%
\pgfpathcurveto{\pgfqpoint{0.738542in}{1.424009in}}{\pgfqpoint{0.741814in}{1.416109in}}{\pgfqpoint{0.747638in}{1.410285in}}%
\pgfpathcurveto{\pgfqpoint{0.753462in}{1.404461in}}{\pgfqpoint{0.761362in}{1.401189in}}{\pgfqpoint{0.769598in}{1.401189in}}%
\pgfpathclose%
\pgfusepath{stroke,fill}%
\end{pgfscope}%
\begin{pgfscope}%
\pgfpathrectangle{\pgfqpoint{0.100000in}{0.212622in}}{\pgfqpoint{3.696000in}{3.696000in}}%
\pgfusepath{clip}%
\pgfsetbuttcap%
\pgfsetroundjoin%
\definecolor{currentfill}{rgb}{0.121569,0.466667,0.705882}%
\pgfsetfillcolor{currentfill}%
\pgfsetfillopacity{0.639770}%
\pgfsetlinewidth{1.003750pt}%
\definecolor{currentstroke}{rgb}{0.121569,0.466667,0.705882}%
\pgfsetstrokecolor{currentstroke}%
\pgfsetstrokeopacity{0.639770}%
\pgfsetdash{}{0pt}%
\pgfpathmoveto{\pgfqpoint{0.769598in}{1.401189in}}%
\pgfpathcurveto{\pgfqpoint{0.777835in}{1.401189in}}{\pgfqpoint{0.785735in}{1.404461in}}{\pgfqpoint{0.791559in}{1.410285in}}%
\pgfpathcurveto{\pgfqpoint{0.797383in}{1.416109in}}{\pgfqpoint{0.800655in}{1.424009in}}{\pgfqpoint{0.800655in}{1.432246in}}%
\pgfpathcurveto{\pgfqpoint{0.800655in}{1.440482in}}{\pgfqpoint{0.797383in}{1.448382in}}{\pgfqpoint{0.791559in}{1.454206in}}%
\pgfpathcurveto{\pgfqpoint{0.785735in}{1.460030in}}{\pgfqpoint{0.777835in}{1.463302in}}{\pgfqpoint{0.769598in}{1.463302in}}%
\pgfpathcurveto{\pgfqpoint{0.761362in}{1.463302in}}{\pgfqpoint{0.753462in}{1.460030in}}{\pgfqpoint{0.747638in}{1.454206in}}%
\pgfpathcurveto{\pgfqpoint{0.741814in}{1.448382in}}{\pgfqpoint{0.738542in}{1.440482in}}{\pgfqpoint{0.738542in}{1.432246in}}%
\pgfpathcurveto{\pgfqpoint{0.738542in}{1.424009in}}{\pgfqpoint{0.741814in}{1.416109in}}{\pgfqpoint{0.747638in}{1.410285in}}%
\pgfpathcurveto{\pgfqpoint{0.753462in}{1.404461in}}{\pgfqpoint{0.761362in}{1.401189in}}{\pgfqpoint{0.769598in}{1.401189in}}%
\pgfpathclose%
\pgfusepath{stroke,fill}%
\end{pgfscope}%
\begin{pgfscope}%
\pgfpathrectangle{\pgfqpoint{0.100000in}{0.212622in}}{\pgfqpoint{3.696000in}{3.696000in}}%
\pgfusepath{clip}%
\pgfsetbuttcap%
\pgfsetroundjoin%
\definecolor{currentfill}{rgb}{0.121569,0.466667,0.705882}%
\pgfsetfillcolor{currentfill}%
\pgfsetfillopacity{0.639770}%
\pgfsetlinewidth{1.003750pt}%
\definecolor{currentstroke}{rgb}{0.121569,0.466667,0.705882}%
\pgfsetstrokecolor{currentstroke}%
\pgfsetstrokeopacity{0.639770}%
\pgfsetdash{}{0pt}%
\pgfpathmoveto{\pgfqpoint{0.769598in}{1.401189in}}%
\pgfpathcurveto{\pgfqpoint{0.777835in}{1.401189in}}{\pgfqpoint{0.785735in}{1.404461in}}{\pgfqpoint{0.791559in}{1.410285in}}%
\pgfpathcurveto{\pgfqpoint{0.797383in}{1.416109in}}{\pgfqpoint{0.800655in}{1.424009in}}{\pgfqpoint{0.800655in}{1.432246in}}%
\pgfpathcurveto{\pgfqpoint{0.800655in}{1.440482in}}{\pgfqpoint{0.797383in}{1.448382in}}{\pgfqpoint{0.791559in}{1.454206in}}%
\pgfpathcurveto{\pgfqpoint{0.785735in}{1.460030in}}{\pgfqpoint{0.777835in}{1.463302in}}{\pgfqpoint{0.769598in}{1.463302in}}%
\pgfpathcurveto{\pgfqpoint{0.761362in}{1.463302in}}{\pgfqpoint{0.753462in}{1.460030in}}{\pgfqpoint{0.747638in}{1.454206in}}%
\pgfpathcurveto{\pgfqpoint{0.741814in}{1.448382in}}{\pgfqpoint{0.738542in}{1.440482in}}{\pgfqpoint{0.738542in}{1.432246in}}%
\pgfpathcurveto{\pgfqpoint{0.738542in}{1.424009in}}{\pgfqpoint{0.741814in}{1.416109in}}{\pgfqpoint{0.747638in}{1.410285in}}%
\pgfpathcurveto{\pgfqpoint{0.753462in}{1.404461in}}{\pgfqpoint{0.761362in}{1.401189in}}{\pgfqpoint{0.769598in}{1.401189in}}%
\pgfpathclose%
\pgfusepath{stroke,fill}%
\end{pgfscope}%
\begin{pgfscope}%
\pgfpathrectangle{\pgfqpoint{0.100000in}{0.212622in}}{\pgfqpoint{3.696000in}{3.696000in}}%
\pgfusepath{clip}%
\pgfsetbuttcap%
\pgfsetroundjoin%
\definecolor{currentfill}{rgb}{0.121569,0.466667,0.705882}%
\pgfsetfillcolor{currentfill}%
\pgfsetfillopacity{0.639770}%
\pgfsetlinewidth{1.003750pt}%
\definecolor{currentstroke}{rgb}{0.121569,0.466667,0.705882}%
\pgfsetstrokecolor{currentstroke}%
\pgfsetstrokeopacity{0.639770}%
\pgfsetdash{}{0pt}%
\pgfpathmoveto{\pgfqpoint{0.769598in}{1.401189in}}%
\pgfpathcurveto{\pgfqpoint{0.777835in}{1.401189in}}{\pgfqpoint{0.785735in}{1.404461in}}{\pgfqpoint{0.791559in}{1.410285in}}%
\pgfpathcurveto{\pgfqpoint{0.797383in}{1.416109in}}{\pgfqpoint{0.800655in}{1.424009in}}{\pgfqpoint{0.800655in}{1.432246in}}%
\pgfpathcurveto{\pgfqpoint{0.800655in}{1.440482in}}{\pgfqpoint{0.797383in}{1.448382in}}{\pgfqpoint{0.791559in}{1.454206in}}%
\pgfpathcurveto{\pgfqpoint{0.785735in}{1.460030in}}{\pgfqpoint{0.777835in}{1.463302in}}{\pgfqpoint{0.769598in}{1.463302in}}%
\pgfpathcurveto{\pgfqpoint{0.761362in}{1.463302in}}{\pgfqpoint{0.753462in}{1.460030in}}{\pgfqpoint{0.747638in}{1.454206in}}%
\pgfpathcurveto{\pgfqpoint{0.741814in}{1.448382in}}{\pgfqpoint{0.738542in}{1.440482in}}{\pgfqpoint{0.738542in}{1.432246in}}%
\pgfpathcurveto{\pgfqpoint{0.738542in}{1.424009in}}{\pgfqpoint{0.741814in}{1.416109in}}{\pgfqpoint{0.747638in}{1.410285in}}%
\pgfpathcurveto{\pgfqpoint{0.753462in}{1.404461in}}{\pgfqpoint{0.761362in}{1.401189in}}{\pgfqpoint{0.769598in}{1.401189in}}%
\pgfpathclose%
\pgfusepath{stroke,fill}%
\end{pgfscope}%
\begin{pgfscope}%
\pgfpathrectangle{\pgfqpoint{0.100000in}{0.212622in}}{\pgfqpoint{3.696000in}{3.696000in}}%
\pgfusepath{clip}%
\pgfsetbuttcap%
\pgfsetroundjoin%
\definecolor{currentfill}{rgb}{0.121569,0.466667,0.705882}%
\pgfsetfillcolor{currentfill}%
\pgfsetfillopacity{0.639770}%
\pgfsetlinewidth{1.003750pt}%
\definecolor{currentstroke}{rgb}{0.121569,0.466667,0.705882}%
\pgfsetstrokecolor{currentstroke}%
\pgfsetstrokeopacity{0.639770}%
\pgfsetdash{}{0pt}%
\pgfpathmoveto{\pgfqpoint{0.769598in}{1.401189in}}%
\pgfpathcurveto{\pgfqpoint{0.777835in}{1.401189in}}{\pgfqpoint{0.785735in}{1.404461in}}{\pgfqpoint{0.791559in}{1.410285in}}%
\pgfpathcurveto{\pgfqpoint{0.797383in}{1.416109in}}{\pgfqpoint{0.800655in}{1.424009in}}{\pgfqpoint{0.800655in}{1.432246in}}%
\pgfpathcurveto{\pgfqpoint{0.800655in}{1.440482in}}{\pgfqpoint{0.797383in}{1.448382in}}{\pgfqpoint{0.791559in}{1.454206in}}%
\pgfpathcurveto{\pgfqpoint{0.785735in}{1.460030in}}{\pgfqpoint{0.777835in}{1.463302in}}{\pgfqpoint{0.769598in}{1.463302in}}%
\pgfpathcurveto{\pgfqpoint{0.761362in}{1.463302in}}{\pgfqpoint{0.753462in}{1.460030in}}{\pgfqpoint{0.747638in}{1.454206in}}%
\pgfpathcurveto{\pgfqpoint{0.741814in}{1.448382in}}{\pgfqpoint{0.738542in}{1.440482in}}{\pgfqpoint{0.738542in}{1.432246in}}%
\pgfpathcurveto{\pgfqpoint{0.738542in}{1.424009in}}{\pgfqpoint{0.741814in}{1.416109in}}{\pgfqpoint{0.747638in}{1.410285in}}%
\pgfpathcurveto{\pgfqpoint{0.753462in}{1.404461in}}{\pgfqpoint{0.761362in}{1.401189in}}{\pgfqpoint{0.769598in}{1.401189in}}%
\pgfpathclose%
\pgfusepath{stroke,fill}%
\end{pgfscope}%
\begin{pgfscope}%
\pgfpathrectangle{\pgfqpoint{0.100000in}{0.212622in}}{\pgfqpoint{3.696000in}{3.696000in}}%
\pgfusepath{clip}%
\pgfsetbuttcap%
\pgfsetroundjoin%
\definecolor{currentfill}{rgb}{0.121569,0.466667,0.705882}%
\pgfsetfillcolor{currentfill}%
\pgfsetfillopacity{0.639770}%
\pgfsetlinewidth{1.003750pt}%
\definecolor{currentstroke}{rgb}{0.121569,0.466667,0.705882}%
\pgfsetstrokecolor{currentstroke}%
\pgfsetstrokeopacity{0.639770}%
\pgfsetdash{}{0pt}%
\pgfpathmoveto{\pgfqpoint{0.769598in}{1.401189in}}%
\pgfpathcurveto{\pgfqpoint{0.777835in}{1.401189in}}{\pgfqpoint{0.785735in}{1.404461in}}{\pgfqpoint{0.791559in}{1.410285in}}%
\pgfpathcurveto{\pgfqpoint{0.797383in}{1.416109in}}{\pgfqpoint{0.800655in}{1.424009in}}{\pgfqpoint{0.800655in}{1.432246in}}%
\pgfpathcurveto{\pgfqpoint{0.800655in}{1.440482in}}{\pgfqpoint{0.797383in}{1.448382in}}{\pgfqpoint{0.791559in}{1.454206in}}%
\pgfpathcurveto{\pgfqpoint{0.785735in}{1.460030in}}{\pgfqpoint{0.777835in}{1.463302in}}{\pgfqpoint{0.769598in}{1.463302in}}%
\pgfpathcurveto{\pgfqpoint{0.761362in}{1.463302in}}{\pgfqpoint{0.753462in}{1.460030in}}{\pgfqpoint{0.747638in}{1.454206in}}%
\pgfpathcurveto{\pgfqpoint{0.741814in}{1.448382in}}{\pgfqpoint{0.738542in}{1.440482in}}{\pgfqpoint{0.738542in}{1.432246in}}%
\pgfpathcurveto{\pgfqpoint{0.738542in}{1.424009in}}{\pgfqpoint{0.741814in}{1.416109in}}{\pgfqpoint{0.747638in}{1.410285in}}%
\pgfpathcurveto{\pgfqpoint{0.753462in}{1.404461in}}{\pgfqpoint{0.761362in}{1.401189in}}{\pgfqpoint{0.769598in}{1.401189in}}%
\pgfpathclose%
\pgfusepath{stroke,fill}%
\end{pgfscope}%
\begin{pgfscope}%
\pgfpathrectangle{\pgfqpoint{0.100000in}{0.212622in}}{\pgfqpoint{3.696000in}{3.696000in}}%
\pgfusepath{clip}%
\pgfsetbuttcap%
\pgfsetroundjoin%
\definecolor{currentfill}{rgb}{0.121569,0.466667,0.705882}%
\pgfsetfillcolor{currentfill}%
\pgfsetfillopacity{0.639770}%
\pgfsetlinewidth{1.003750pt}%
\definecolor{currentstroke}{rgb}{0.121569,0.466667,0.705882}%
\pgfsetstrokecolor{currentstroke}%
\pgfsetstrokeopacity{0.639770}%
\pgfsetdash{}{0pt}%
\pgfpathmoveto{\pgfqpoint{0.769598in}{1.401189in}}%
\pgfpathcurveto{\pgfqpoint{0.777835in}{1.401189in}}{\pgfqpoint{0.785735in}{1.404461in}}{\pgfqpoint{0.791559in}{1.410285in}}%
\pgfpathcurveto{\pgfqpoint{0.797383in}{1.416109in}}{\pgfqpoint{0.800655in}{1.424009in}}{\pgfqpoint{0.800655in}{1.432246in}}%
\pgfpathcurveto{\pgfqpoint{0.800655in}{1.440482in}}{\pgfqpoint{0.797383in}{1.448382in}}{\pgfqpoint{0.791559in}{1.454206in}}%
\pgfpathcurveto{\pgfqpoint{0.785735in}{1.460030in}}{\pgfqpoint{0.777835in}{1.463302in}}{\pgfqpoint{0.769598in}{1.463302in}}%
\pgfpathcurveto{\pgfqpoint{0.761362in}{1.463302in}}{\pgfqpoint{0.753462in}{1.460030in}}{\pgfqpoint{0.747638in}{1.454206in}}%
\pgfpathcurveto{\pgfqpoint{0.741814in}{1.448382in}}{\pgfqpoint{0.738542in}{1.440482in}}{\pgfqpoint{0.738542in}{1.432246in}}%
\pgfpathcurveto{\pgfqpoint{0.738542in}{1.424009in}}{\pgfqpoint{0.741814in}{1.416109in}}{\pgfqpoint{0.747638in}{1.410285in}}%
\pgfpathcurveto{\pgfqpoint{0.753462in}{1.404461in}}{\pgfqpoint{0.761362in}{1.401189in}}{\pgfqpoint{0.769598in}{1.401189in}}%
\pgfpathclose%
\pgfusepath{stroke,fill}%
\end{pgfscope}%
\begin{pgfscope}%
\pgfpathrectangle{\pgfqpoint{0.100000in}{0.212622in}}{\pgfqpoint{3.696000in}{3.696000in}}%
\pgfusepath{clip}%
\pgfsetbuttcap%
\pgfsetroundjoin%
\definecolor{currentfill}{rgb}{0.121569,0.466667,0.705882}%
\pgfsetfillcolor{currentfill}%
\pgfsetfillopacity{0.639770}%
\pgfsetlinewidth{1.003750pt}%
\definecolor{currentstroke}{rgb}{0.121569,0.466667,0.705882}%
\pgfsetstrokecolor{currentstroke}%
\pgfsetstrokeopacity{0.639770}%
\pgfsetdash{}{0pt}%
\pgfpathmoveto{\pgfqpoint{0.769598in}{1.401189in}}%
\pgfpathcurveto{\pgfqpoint{0.777835in}{1.401189in}}{\pgfqpoint{0.785735in}{1.404461in}}{\pgfqpoint{0.791559in}{1.410285in}}%
\pgfpathcurveto{\pgfqpoint{0.797383in}{1.416109in}}{\pgfqpoint{0.800655in}{1.424009in}}{\pgfqpoint{0.800655in}{1.432246in}}%
\pgfpathcurveto{\pgfqpoint{0.800655in}{1.440482in}}{\pgfqpoint{0.797383in}{1.448382in}}{\pgfqpoint{0.791559in}{1.454206in}}%
\pgfpathcurveto{\pgfqpoint{0.785735in}{1.460030in}}{\pgfqpoint{0.777835in}{1.463302in}}{\pgfqpoint{0.769598in}{1.463302in}}%
\pgfpathcurveto{\pgfqpoint{0.761362in}{1.463302in}}{\pgfqpoint{0.753462in}{1.460030in}}{\pgfqpoint{0.747638in}{1.454206in}}%
\pgfpathcurveto{\pgfqpoint{0.741814in}{1.448382in}}{\pgfqpoint{0.738542in}{1.440482in}}{\pgfqpoint{0.738542in}{1.432246in}}%
\pgfpathcurveto{\pgfqpoint{0.738542in}{1.424009in}}{\pgfqpoint{0.741814in}{1.416109in}}{\pgfqpoint{0.747638in}{1.410285in}}%
\pgfpathcurveto{\pgfqpoint{0.753462in}{1.404461in}}{\pgfqpoint{0.761362in}{1.401189in}}{\pgfqpoint{0.769598in}{1.401189in}}%
\pgfpathclose%
\pgfusepath{stroke,fill}%
\end{pgfscope}%
\begin{pgfscope}%
\pgfpathrectangle{\pgfqpoint{0.100000in}{0.212622in}}{\pgfqpoint{3.696000in}{3.696000in}}%
\pgfusepath{clip}%
\pgfsetbuttcap%
\pgfsetroundjoin%
\definecolor{currentfill}{rgb}{0.121569,0.466667,0.705882}%
\pgfsetfillcolor{currentfill}%
\pgfsetfillopacity{0.639770}%
\pgfsetlinewidth{1.003750pt}%
\definecolor{currentstroke}{rgb}{0.121569,0.466667,0.705882}%
\pgfsetstrokecolor{currentstroke}%
\pgfsetstrokeopacity{0.639770}%
\pgfsetdash{}{0pt}%
\pgfpathmoveto{\pgfqpoint{0.769598in}{1.401189in}}%
\pgfpathcurveto{\pgfqpoint{0.777835in}{1.401189in}}{\pgfqpoint{0.785735in}{1.404461in}}{\pgfqpoint{0.791559in}{1.410285in}}%
\pgfpathcurveto{\pgfqpoint{0.797383in}{1.416109in}}{\pgfqpoint{0.800655in}{1.424009in}}{\pgfqpoint{0.800655in}{1.432246in}}%
\pgfpathcurveto{\pgfqpoint{0.800655in}{1.440482in}}{\pgfqpoint{0.797383in}{1.448382in}}{\pgfqpoint{0.791559in}{1.454206in}}%
\pgfpathcurveto{\pgfqpoint{0.785735in}{1.460030in}}{\pgfqpoint{0.777835in}{1.463302in}}{\pgfqpoint{0.769598in}{1.463302in}}%
\pgfpathcurveto{\pgfqpoint{0.761362in}{1.463302in}}{\pgfqpoint{0.753462in}{1.460030in}}{\pgfqpoint{0.747638in}{1.454206in}}%
\pgfpathcurveto{\pgfqpoint{0.741814in}{1.448382in}}{\pgfqpoint{0.738542in}{1.440482in}}{\pgfqpoint{0.738542in}{1.432246in}}%
\pgfpathcurveto{\pgfqpoint{0.738542in}{1.424009in}}{\pgfqpoint{0.741814in}{1.416109in}}{\pgfqpoint{0.747638in}{1.410285in}}%
\pgfpathcurveto{\pgfqpoint{0.753462in}{1.404461in}}{\pgfqpoint{0.761362in}{1.401189in}}{\pgfqpoint{0.769598in}{1.401189in}}%
\pgfpathclose%
\pgfusepath{stroke,fill}%
\end{pgfscope}%
\begin{pgfscope}%
\pgfpathrectangle{\pgfqpoint{0.100000in}{0.212622in}}{\pgfqpoint{3.696000in}{3.696000in}}%
\pgfusepath{clip}%
\pgfsetbuttcap%
\pgfsetroundjoin%
\definecolor{currentfill}{rgb}{0.121569,0.466667,0.705882}%
\pgfsetfillcolor{currentfill}%
\pgfsetfillopacity{0.639770}%
\pgfsetlinewidth{1.003750pt}%
\definecolor{currentstroke}{rgb}{0.121569,0.466667,0.705882}%
\pgfsetstrokecolor{currentstroke}%
\pgfsetstrokeopacity{0.639770}%
\pgfsetdash{}{0pt}%
\pgfpathmoveto{\pgfqpoint{0.769598in}{1.401189in}}%
\pgfpathcurveto{\pgfqpoint{0.777835in}{1.401189in}}{\pgfqpoint{0.785735in}{1.404461in}}{\pgfqpoint{0.791559in}{1.410285in}}%
\pgfpathcurveto{\pgfqpoint{0.797383in}{1.416109in}}{\pgfqpoint{0.800655in}{1.424009in}}{\pgfqpoint{0.800655in}{1.432246in}}%
\pgfpathcurveto{\pgfqpoint{0.800655in}{1.440482in}}{\pgfqpoint{0.797383in}{1.448382in}}{\pgfqpoint{0.791559in}{1.454206in}}%
\pgfpathcurveto{\pgfqpoint{0.785735in}{1.460030in}}{\pgfqpoint{0.777835in}{1.463302in}}{\pgfqpoint{0.769598in}{1.463302in}}%
\pgfpathcurveto{\pgfqpoint{0.761362in}{1.463302in}}{\pgfqpoint{0.753462in}{1.460030in}}{\pgfqpoint{0.747638in}{1.454206in}}%
\pgfpathcurveto{\pgfqpoint{0.741814in}{1.448382in}}{\pgfqpoint{0.738542in}{1.440482in}}{\pgfqpoint{0.738542in}{1.432246in}}%
\pgfpathcurveto{\pgfqpoint{0.738542in}{1.424009in}}{\pgfqpoint{0.741814in}{1.416109in}}{\pgfqpoint{0.747638in}{1.410285in}}%
\pgfpathcurveto{\pgfqpoint{0.753462in}{1.404461in}}{\pgfqpoint{0.761362in}{1.401189in}}{\pgfqpoint{0.769598in}{1.401189in}}%
\pgfpathclose%
\pgfusepath{stroke,fill}%
\end{pgfscope}%
\begin{pgfscope}%
\pgfpathrectangle{\pgfqpoint{0.100000in}{0.212622in}}{\pgfqpoint{3.696000in}{3.696000in}}%
\pgfusepath{clip}%
\pgfsetbuttcap%
\pgfsetroundjoin%
\definecolor{currentfill}{rgb}{0.121569,0.466667,0.705882}%
\pgfsetfillcolor{currentfill}%
\pgfsetfillopacity{0.639770}%
\pgfsetlinewidth{1.003750pt}%
\definecolor{currentstroke}{rgb}{0.121569,0.466667,0.705882}%
\pgfsetstrokecolor{currentstroke}%
\pgfsetstrokeopacity{0.639770}%
\pgfsetdash{}{0pt}%
\pgfpathmoveto{\pgfqpoint{0.769598in}{1.401189in}}%
\pgfpathcurveto{\pgfqpoint{0.777835in}{1.401189in}}{\pgfqpoint{0.785735in}{1.404461in}}{\pgfqpoint{0.791559in}{1.410285in}}%
\pgfpathcurveto{\pgfqpoint{0.797383in}{1.416109in}}{\pgfqpoint{0.800655in}{1.424009in}}{\pgfqpoint{0.800655in}{1.432246in}}%
\pgfpathcurveto{\pgfqpoint{0.800655in}{1.440482in}}{\pgfqpoint{0.797383in}{1.448382in}}{\pgfqpoint{0.791559in}{1.454206in}}%
\pgfpathcurveto{\pgfqpoint{0.785735in}{1.460030in}}{\pgfqpoint{0.777835in}{1.463302in}}{\pgfqpoint{0.769598in}{1.463302in}}%
\pgfpathcurveto{\pgfqpoint{0.761362in}{1.463302in}}{\pgfqpoint{0.753462in}{1.460030in}}{\pgfqpoint{0.747638in}{1.454206in}}%
\pgfpathcurveto{\pgfqpoint{0.741814in}{1.448382in}}{\pgfqpoint{0.738542in}{1.440482in}}{\pgfqpoint{0.738542in}{1.432246in}}%
\pgfpathcurveto{\pgfqpoint{0.738542in}{1.424009in}}{\pgfqpoint{0.741814in}{1.416109in}}{\pgfqpoint{0.747638in}{1.410285in}}%
\pgfpathcurveto{\pgfqpoint{0.753462in}{1.404461in}}{\pgfqpoint{0.761362in}{1.401189in}}{\pgfqpoint{0.769598in}{1.401189in}}%
\pgfpathclose%
\pgfusepath{stroke,fill}%
\end{pgfscope}%
\begin{pgfscope}%
\pgfpathrectangle{\pgfqpoint{0.100000in}{0.212622in}}{\pgfqpoint{3.696000in}{3.696000in}}%
\pgfusepath{clip}%
\pgfsetbuttcap%
\pgfsetroundjoin%
\definecolor{currentfill}{rgb}{0.121569,0.466667,0.705882}%
\pgfsetfillcolor{currentfill}%
\pgfsetfillopacity{0.639770}%
\pgfsetlinewidth{1.003750pt}%
\definecolor{currentstroke}{rgb}{0.121569,0.466667,0.705882}%
\pgfsetstrokecolor{currentstroke}%
\pgfsetstrokeopacity{0.639770}%
\pgfsetdash{}{0pt}%
\pgfpathmoveto{\pgfqpoint{0.769598in}{1.401189in}}%
\pgfpathcurveto{\pgfqpoint{0.777835in}{1.401189in}}{\pgfqpoint{0.785735in}{1.404461in}}{\pgfqpoint{0.791559in}{1.410285in}}%
\pgfpathcurveto{\pgfqpoint{0.797383in}{1.416109in}}{\pgfqpoint{0.800655in}{1.424009in}}{\pgfqpoint{0.800655in}{1.432246in}}%
\pgfpathcurveto{\pgfqpoint{0.800655in}{1.440482in}}{\pgfqpoint{0.797383in}{1.448382in}}{\pgfqpoint{0.791559in}{1.454206in}}%
\pgfpathcurveto{\pgfqpoint{0.785735in}{1.460030in}}{\pgfqpoint{0.777835in}{1.463302in}}{\pgfqpoint{0.769598in}{1.463302in}}%
\pgfpathcurveto{\pgfqpoint{0.761362in}{1.463302in}}{\pgfqpoint{0.753462in}{1.460030in}}{\pgfqpoint{0.747638in}{1.454206in}}%
\pgfpathcurveto{\pgfqpoint{0.741814in}{1.448382in}}{\pgfqpoint{0.738542in}{1.440482in}}{\pgfqpoint{0.738542in}{1.432246in}}%
\pgfpathcurveto{\pgfqpoint{0.738542in}{1.424009in}}{\pgfqpoint{0.741814in}{1.416109in}}{\pgfqpoint{0.747638in}{1.410285in}}%
\pgfpathcurveto{\pgfqpoint{0.753462in}{1.404461in}}{\pgfqpoint{0.761362in}{1.401189in}}{\pgfqpoint{0.769598in}{1.401189in}}%
\pgfpathclose%
\pgfusepath{stroke,fill}%
\end{pgfscope}%
\begin{pgfscope}%
\pgfpathrectangle{\pgfqpoint{0.100000in}{0.212622in}}{\pgfqpoint{3.696000in}{3.696000in}}%
\pgfusepath{clip}%
\pgfsetbuttcap%
\pgfsetroundjoin%
\definecolor{currentfill}{rgb}{0.121569,0.466667,0.705882}%
\pgfsetfillcolor{currentfill}%
\pgfsetfillopacity{0.639770}%
\pgfsetlinewidth{1.003750pt}%
\definecolor{currentstroke}{rgb}{0.121569,0.466667,0.705882}%
\pgfsetstrokecolor{currentstroke}%
\pgfsetstrokeopacity{0.639770}%
\pgfsetdash{}{0pt}%
\pgfpathmoveto{\pgfqpoint{0.769598in}{1.401189in}}%
\pgfpathcurveto{\pgfqpoint{0.777835in}{1.401189in}}{\pgfqpoint{0.785735in}{1.404461in}}{\pgfqpoint{0.791559in}{1.410285in}}%
\pgfpathcurveto{\pgfqpoint{0.797383in}{1.416109in}}{\pgfqpoint{0.800655in}{1.424009in}}{\pgfqpoint{0.800655in}{1.432246in}}%
\pgfpathcurveto{\pgfqpoint{0.800655in}{1.440482in}}{\pgfqpoint{0.797383in}{1.448382in}}{\pgfqpoint{0.791559in}{1.454206in}}%
\pgfpathcurveto{\pgfqpoint{0.785735in}{1.460030in}}{\pgfqpoint{0.777835in}{1.463302in}}{\pgfqpoint{0.769598in}{1.463302in}}%
\pgfpathcurveto{\pgfqpoint{0.761362in}{1.463302in}}{\pgfqpoint{0.753462in}{1.460030in}}{\pgfqpoint{0.747638in}{1.454206in}}%
\pgfpathcurveto{\pgfqpoint{0.741814in}{1.448382in}}{\pgfqpoint{0.738542in}{1.440482in}}{\pgfqpoint{0.738542in}{1.432246in}}%
\pgfpathcurveto{\pgfqpoint{0.738542in}{1.424009in}}{\pgfqpoint{0.741814in}{1.416109in}}{\pgfqpoint{0.747638in}{1.410285in}}%
\pgfpathcurveto{\pgfqpoint{0.753462in}{1.404461in}}{\pgfqpoint{0.761362in}{1.401189in}}{\pgfqpoint{0.769598in}{1.401189in}}%
\pgfpathclose%
\pgfusepath{stroke,fill}%
\end{pgfscope}%
\begin{pgfscope}%
\pgfpathrectangle{\pgfqpoint{0.100000in}{0.212622in}}{\pgfqpoint{3.696000in}{3.696000in}}%
\pgfusepath{clip}%
\pgfsetbuttcap%
\pgfsetroundjoin%
\definecolor{currentfill}{rgb}{0.121569,0.466667,0.705882}%
\pgfsetfillcolor{currentfill}%
\pgfsetfillopacity{0.639770}%
\pgfsetlinewidth{1.003750pt}%
\definecolor{currentstroke}{rgb}{0.121569,0.466667,0.705882}%
\pgfsetstrokecolor{currentstroke}%
\pgfsetstrokeopacity{0.639770}%
\pgfsetdash{}{0pt}%
\pgfpathmoveto{\pgfqpoint{0.769598in}{1.401189in}}%
\pgfpathcurveto{\pgfqpoint{0.777835in}{1.401189in}}{\pgfqpoint{0.785735in}{1.404461in}}{\pgfqpoint{0.791559in}{1.410285in}}%
\pgfpathcurveto{\pgfqpoint{0.797383in}{1.416109in}}{\pgfqpoint{0.800655in}{1.424009in}}{\pgfqpoint{0.800655in}{1.432246in}}%
\pgfpathcurveto{\pgfqpoint{0.800655in}{1.440482in}}{\pgfqpoint{0.797383in}{1.448382in}}{\pgfqpoint{0.791559in}{1.454206in}}%
\pgfpathcurveto{\pgfqpoint{0.785735in}{1.460030in}}{\pgfqpoint{0.777835in}{1.463302in}}{\pgfqpoint{0.769598in}{1.463302in}}%
\pgfpathcurveto{\pgfqpoint{0.761362in}{1.463302in}}{\pgfqpoint{0.753462in}{1.460030in}}{\pgfqpoint{0.747638in}{1.454206in}}%
\pgfpathcurveto{\pgfqpoint{0.741814in}{1.448382in}}{\pgfqpoint{0.738542in}{1.440482in}}{\pgfqpoint{0.738542in}{1.432246in}}%
\pgfpathcurveto{\pgfqpoint{0.738542in}{1.424009in}}{\pgfqpoint{0.741814in}{1.416109in}}{\pgfqpoint{0.747638in}{1.410285in}}%
\pgfpathcurveto{\pgfqpoint{0.753462in}{1.404461in}}{\pgfqpoint{0.761362in}{1.401189in}}{\pgfqpoint{0.769598in}{1.401189in}}%
\pgfpathclose%
\pgfusepath{stroke,fill}%
\end{pgfscope}%
\begin{pgfscope}%
\pgfpathrectangle{\pgfqpoint{0.100000in}{0.212622in}}{\pgfqpoint{3.696000in}{3.696000in}}%
\pgfusepath{clip}%
\pgfsetbuttcap%
\pgfsetroundjoin%
\definecolor{currentfill}{rgb}{0.121569,0.466667,0.705882}%
\pgfsetfillcolor{currentfill}%
\pgfsetfillopacity{0.639770}%
\pgfsetlinewidth{1.003750pt}%
\definecolor{currentstroke}{rgb}{0.121569,0.466667,0.705882}%
\pgfsetstrokecolor{currentstroke}%
\pgfsetstrokeopacity{0.639770}%
\pgfsetdash{}{0pt}%
\pgfpathmoveto{\pgfqpoint{0.769598in}{1.401189in}}%
\pgfpathcurveto{\pgfqpoint{0.777835in}{1.401189in}}{\pgfqpoint{0.785735in}{1.404461in}}{\pgfqpoint{0.791559in}{1.410285in}}%
\pgfpathcurveto{\pgfqpoint{0.797383in}{1.416109in}}{\pgfqpoint{0.800655in}{1.424009in}}{\pgfqpoint{0.800655in}{1.432246in}}%
\pgfpathcurveto{\pgfqpoint{0.800655in}{1.440482in}}{\pgfqpoint{0.797383in}{1.448382in}}{\pgfqpoint{0.791559in}{1.454206in}}%
\pgfpathcurveto{\pgfqpoint{0.785735in}{1.460030in}}{\pgfqpoint{0.777835in}{1.463302in}}{\pgfqpoint{0.769598in}{1.463302in}}%
\pgfpathcurveto{\pgfqpoint{0.761362in}{1.463302in}}{\pgfqpoint{0.753462in}{1.460030in}}{\pgfqpoint{0.747638in}{1.454206in}}%
\pgfpathcurveto{\pgfqpoint{0.741814in}{1.448382in}}{\pgfqpoint{0.738542in}{1.440482in}}{\pgfqpoint{0.738542in}{1.432246in}}%
\pgfpathcurveto{\pgfqpoint{0.738542in}{1.424009in}}{\pgfqpoint{0.741814in}{1.416109in}}{\pgfqpoint{0.747638in}{1.410285in}}%
\pgfpathcurveto{\pgfqpoint{0.753462in}{1.404461in}}{\pgfqpoint{0.761362in}{1.401189in}}{\pgfqpoint{0.769598in}{1.401189in}}%
\pgfpathclose%
\pgfusepath{stroke,fill}%
\end{pgfscope}%
\begin{pgfscope}%
\pgfpathrectangle{\pgfqpoint{0.100000in}{0.212622in}}{\pgfqpoint{3.696000in}{3.696000in}}%
\pgfusepath{clip}%
\pgfsetbuttcap%
\pgfsetroundjoin%
\definecolor{currentfill}{rgb}{0.121569,0.466667,0.705882}%
\pgfsetfillcolor{currentfill}%
\pgfsetfillopacity{0.639770}%
\pgfsetlinewidth{1.003750pt}%
\definecolor{currentstroke}{rgb}{0.121569,0.466667,0.705882}%
\pgfsetstrokecolor{currentstroke}%
\pgfsetstrokeopacity{0.639770}%
\pgfsetdash{}{0pt}%
\pgfpathmoveto{\pgfqpoint{0.769598in}{1.401189in}}%
\pgfpathcurveto{\pgfqpoint{0.777835in}{1.401189in}}{\pgfqpoint{0.785735in}{1.404461in}}{\pgfqpoint{0.791559in}{1.410285in}}%
\pgfpathcurveto{\pgfqpoint{0.797383in}{1.416109in}}{\pgfqpoint{0.800655in}{1.424009in}}{\pgfqpoint{0.800655in}{1.432246in}}%
\pgfpathcurveto{\pgfqpoint{0.800655in}{1.440482in}}{\pgfqpoint{0.797383in}{1.448382in}}{\pgfqpoint{0.791559in}{1.454206in}}%
\pgfpathcurveto{\pgfqpoint{0.785735in}{1.460030in}}{\pgfqpoint{0.777835in}{1.463302in}}{\pgfqpoint{0.769598in}{1.463302in}}%
\pgfpathcurveto{\pgfqpoint{0.761362in}{1.463302in}}{\pgfqpoint{0.753462in}{1.460030in}}{\pgfqpoint{0.747638in}{1.454206in}}%
\pgfpathcurveto{\pgfqpoint{0.741814in}{1.448382in}}{\pgfqpoint{0.738542in}{1.440482in}}{\pgfqpoint{0.738542in}{1.432246in}}%
\pgfpathcurveto{\pgfqpoint{0.738542in}{1.424009in}}{\pgfqpoint{0.741814in}{1.416109in}}{\pgfqpoint{0.747638in}{1.410285in}}%
\pgfpathcurveto{\pgfqpoint{0.753462in}{1.404461in}}{\pgfqpoint{0.761362in}{1.401189in}}{\pgfqpoint{0.769598in}{1.401189in}}%
\pgfpathclose%
\pgfusepath{stroke,fill}%
\end{pgfscope}%
\begin{pgfscope}%
\pgfpathrectangle{\pgfqpoint{0.100000in}{0.212622in}}{\pgfqpoint{3.696000in}{3.696000in}}%
\pgfusepath{clip}%
\pgfsetbuttcap%
\pgfsetroundjoin%
\definecolor{currentfill}{rgb}{0.121569,0.466667,0.705882}%
\pgfsetfillcolor{currentfill}%
\pgfsetfillopacity{0.639770}%
\pgfsetlinewidth{1.003750pt}%
\definecolor{currentstroke}{rgb}{0.121569,0.466667,0.705882}%
\pgfsetstrokecolor{currentstroke}%
\pgfsetstrokeopacity{0.639770}%
\pgfsetdash{}{0pt}%
\pgfpathmoveto{\pgfqpoint{0.769598in}{1.401189in}}%
\pgfpathcurveto{\pgfqpoint{0.777835in}{1.401189in}}{\pgfqpoint{0.785735in}{1.404461in}}{\pgfqpoint{0.791559in}{1.410285in}}%
\pgfpathcurveto{\pgfqpoint{0.797383in}{1.416109in}}{\pgfqpoint{0.800655in}{1.424009in}}{\pgfqpoint{0.800655in}{1.432246in}}%
\pgfpathcurveto{\pgfqpoint{0.800655in}{1.440482in}}{\pgfqpoint{0.797383in}{1.448382in}}{\pgfqpoint{0.791559in}{1.454206in}}%
\pgfpathcurveto{\pgfqpoint{0.785735in}{1.460030in}}{\pgfqpoint{0.777835in}{1.463302in}}{\pgfqpoint{0.769598in}{1.463302in}}%
\pgfpathcurveto{\pgfqpoint{0.761362in}{1.463302in}}{\pgfqpoint{0.753462in}{1.460030in}}{\pgfqpoint{0.747638in}{1.454206in}}%
\pgfpathcurveto{\pgfqpoint{0.741814in}{1.448382in}}{\pgfqpoint{0.738542in}{1.440482in}}{\pgfqpoint{0.738542in}{1.432246in}}%
\pgfpathcurveto{\pgfqpoint{0.738542in}{1.424009in}}{\pgfqpoint{0.741814in}{1.416109in}}{\pgfqpoint{0.747638in}{1.410285in}}%
\pgfpathcurveto{\pgfqpoint{0.753462in}{1.404461in}}{\pgfqpoint{0.761362in}{1.401189in}}{\pgfqpoint{0.769598in}{1.401189in}}%
\pgfpathclose%
\pgfusepath{stroke,fill}%
\end{pgfscope}%
\begin{pgfscope}%
\pgfpathrectangle{\pgfqpoint{0.100000in}{0.212622in}}{\pgfqpoint{3.696000in}{3.696000in}}%
\pgfusepath{clip}%
\pgfsetbuttcap%
\pgfsetroundjoin%
\definecolor{currentfill}{rgb}{0.121569,0.466667,0.705882}%
\pgfsetfillcolor{currentfill}%
\pgfsetfillopacity{0.639770}%
\pgfsetlinewidth{1.003750pt}%
\definecolor{currentstroke}{rgb}{0.121569,0.466667,0.705882}%
\pgfsetstrokecolor{currentstroke}%
\pgfsetstrokeopacity{0.639770}%
\pgfsetdash{}{0pt}%
\pgfpathmoveto{\pgfqpoint{0.769598in}{1.401189in}}%
\pgfpathcurveto{\pgfqpoint{0.777835in}{1.401189in}}{\pgfqpoint{0.785735in}{1.404461in}}{\pgfqpoint{0.791559in}{1.410285in}}%
\pgfpathcurveto{\pgfqpoint{0.797383in}{1.416109in}}{\pgfqpoint{0.800655in}{1.424009in}}{\pgfqpoint{0.800655in}{1.432246in}}%
\pgfpathcurveto{\pgfqpoint{0.800655in}{1.440482in}}{\pgfqpoint{0.797383in}{1.448382in}}{\pgfqpoint{0.791559in}{1.454206in}}%
\pgfpathcurveto{\pgfqpoint{0.785735in}{1.460030in}}{\pgfqpoint{0.777835in}{1.463302in}}{\pgfqpoint{0.769598in}{1.463302in}}%
\pgfpathcurveto{\pgfqpoint{0.761362in}{1.463302in}}{\pgfqpoint{0.753462in}{1.460030in}}{\pgfqpoint{0.747638in}{1.454206in}}%
\pgfpathcurveto{\pgfqpoint{0.741814in}{1.448382in}}{\pgfqpoint{0.738542in}{1.440482in}}{\pgfqpoint{0.738542in}{1.432246in}}%
\pgfpathcurveto{\pgfqpoint{0.738542in}{1.424009in}}{\pgfqpoint{0.741814in}{1.416109in}}{\pgfqpoint{0.747638in}{1.410285in}}%
\pgfpathcurveto{\pgfqpoint{0.753462in}{1.404461in}}{\pgfqpoint{0.761362in}{1.401189in}}{\pgfqpoint{0.769598in}{1.401189in}}%
\pgfpathclose%
\pgfusepath{stroke,fill}%
\end{pgfscope}%
\begin{pgfscope}%
\pgfpathrectangle{\pgfqpoint{0.100000in}{0.212622in}}{\pgfqpoint{3.696000in}{3.696000in}}%
\pgfusepath{clip}%
\pgfsetbuttcap%
\pgfsetroundjoin%
\definecolor{currentfill}{rgb}{0.121569,0.466667,0.705882}%
\pgfsetfillcolor{currentfill}%
\pgfsetfillopacity{0.639770}%
\pgfsetlinewidth{1.003750pt}%
\definecolor{currentstroke}{rgb}{0.121569,0.466667,0.705882}%
\pgfsetstrokecolor{currentstroke}%
\pgfsetstrokeopacity{0.639770}%
\pgfsetdash{}{0pt}%
\pgfpathmoveto{\pgfqpoint{0.769598in}{1.401189in}}%
\pgfpathcurveto{\pgfqpoint{0.777835in}{1.401189in}}{\pgfqpoint{0.785735in}{1.404461in}}{\pgfqpoint{0.791559in}{1.410285in}}%
\pgfpathcurveto{\pgfqpoint{0.797383in}{1.416109in}}{\pgfqpoint{0.800655in}{1.424009in}}{\pgfqpoint{0.800655in}{1.432246in}}%
\pgfpathcurveto{\pgfqpoint{0.800655in}{1.440482in}}{\pgfqpoint{0.797383in}{1.448382in}}{\pgfqpoint{0.791559in}{1.454206in}}%
\pgfpathcurveto{\pgfqpoint{0.785735in}{1.460030in}}{\pgfqpoint{0.777835in}{1.463302in}}{\pgfqpoint{0.769598in}{1.463302in}}%
\pgfpathcurveto{\pgfqpoint{0.761362in}{1.463302in}}{\pgfqpoint{0.753462in}{1.460030in}}{\pgfqpoint{0.747638in}{1.454206in}}%
\pgfpathcurveto{\pgfqpoint{0.741814in}{1.448382in}}{\pgfqpoint{0.738542in}{1.440482in}}{\pgfqpoint{0.738542in}{1.432246in}}%
\pgfpathcurveto{\pgfqpoint{0.738542in}{1.424009in}}{\pgfqpoint{0.741814in}{1.416109in}}{\pgfqpoint{0.747638in}{1.410285in}}%
\pgfpathcurveto{\pgfqpoint{0.753462in}{1.404461in}}{\pgfqpoint{0.761362in}{1.401189in}}{\pgfqpoint{0.769598in}{1.401189in}}%
\pgfpathclose%
\pgfusepath{stroke,fill}%
\end{pgfscope}%
\begin{pgfscope}%
\pgfpathrectangle{\pgfqpoint{0.100000in}{0.212622in}}{\pgfqpoint{3.696000in}{3.696000in}}%
\pgfusepath{clip}%
\pgfsetbuttcap%
\pgfsetroundjoin%
\definecolor{currentfill}{rgb}{0.121569,0.466667,0.705882}%
\pgfsetfillcolor{currentfill}%
\pgfsetfillopacity{0.639770}%
\pgfsetlinewidth{1.003750pt}%
\definecolor{currentstroke}{rgb}{0.121569,0.466667,0.705882}%
\pgfsetstrokecolor{currentstroke}%
\pgfsetstrokeopacity{0.639770}%
\pgfsetdash{}{0pt}%
\pgfpathmoveto{\pgfqpoint{0.769598in}{1.401189in}}%
\pgfpathcurveto{\pgfqpoint{0.777835in}{1.401189in}}{\pgfqpoint{0.785735in}{1.404461in}}{\pgfqpoint{0.791559in}{1.410285in}}%
\pgfpathcurveto{\pgfqpoint{0.797383in}{1.416109in}}{\pgfqpoint{0.800655in}{1.424009in}}{\pgfqpoint{0.800655in}{1.432246in}}%
\pgfpathcurveto{\pgfqpoint{0.800655in}{1.440482in}}{\pgfqpoint{0.797383in}{1.448382in}}{\pgfqpoint{0.791559in}{1.454206in}}%
\pgfpathcurveto{\pgfqpoint{0.785735in}{1.460030in}}{\pgfqpoint{0.777835in}{1.463302in}}{\pgfqpoint{0.769598in}{1.463302in}}%
\pgfpathcurveto{\pgfqpoint{0.761362in}{1.463302in}}{\pgfqpoint{0.753462in}{1.460030in}}{\pgfqpoint{0.747638in}{1.454206in}}%
\pgfpathcurveto{\pgfqpoint{0.741814in}{1.448382in}}{\pgfqpoint{0.738542in}{1.440482in}}{\pgfqpoint{0.738542in}{1.432246in}}%
\pgfpathcurveto{\pgfqpoint{0.738542in}{1.424009in}}{\pgfqpoint{0.741814in}{1.416109in}}{\pgfqpoint{0.747638in}{1.410285in}}%
\pgfpathcurveto{\pgfqpoint{0.753462in}{1.404461in}}{\pgfqpoint{0.761362in}{1.401189in}}{\pgfqpoint{0.769598in}{1.401189in}}%
\pgfpathclose%
\pgfusepath{stroke,fill}%
\end{pgfscope}%
\begin{pgfscope}%
\pgfpathrectangle{\pgfqpoint{0.100000in}{0.212622in}}{\pgfqpoint{3.696000in}{3.696000in}}%
\pgfusepath{clip}%
\pgfsetbuttcap%
\pgfsetroundjoin%
\definecolor{currentfill}{rgb}{0.121569,0.466667,0.705882}%
\pgfsetfillcolor{currentfill}%
\pgfsetfillopacity{0.639770}%
\pgfsetlinewidth{1.003750pt}%
\definecolor{currentstroke}{rgb}{0.121569,0.466667,0.705882}%
\pgfsetstrokecolor{currentstroke}%
\pgfsetstrokeopacity{0.639770}%
\pgfsetdash{}{0pt}%
\pgfpathmoveto{\pgfqpoint{0.769598in}{1.401189in}}%
\pgfpathcurveto{\pgfqpoint{0.777835in}{1.401189in}}{\pgfqpoint{0.785735in}{1.404461in}}{\pgfqpoint{0.791559in}{1.410285in}}%
\pgfpathcurveto{\pgfqpoint{0.797383in}{1.416109in}}{\pgfqpoint{0.800655in}{1.424009in}}{\pgfqpoint{0.800655in}{1.432246in}}%
\pgfpathcurveto{\pgfqpoint{0.800655in}{1.440482in}}{\pgfqpoint{0.797383in}{1.448382in}}{\pgfqpoint{0.791559in}{1.454206in}}%
\pgfpathcurveto{\pgfqpoint{0.785735in}{1.460030in}}{\pgfqpoint{0.777835in}{1.463302in}}{\pgfqpoint{0.769598in}{1.463302in}}%
\pgfpathcurveto{\pgfqpoint{0.761362in}{1.463302in}}{\pgfqpoint{0.753462in}{1.460030in}}{\pgfqpoint{0.747638in}{1.454206in}}%
\pgfpathcurveto{\pgfqpoint{0.741814in}{1.448382in}}{\pgfqpoint{0.738542in}{1.440482in}}{\pgfqpoint{0.738542in}{1.432246in}}%
\pgfpathcurveto{\pgfqpoint{0.738542in}{1.424009in}}{\pgfqpoint{0.741814in}{1.416109in}}{\pgfqpoint{0.747638in}{1.410285in}}%
\pgfpathcurveto{\pgfqpoint{0.753462in}{1.404461in}}{\pgfqpoint{0.761362in}{1.401189in}}{\pgfqpoint{0.769598in}{1.401189in}}%
\pgfpathclose%
\pgfusepath{stroke,fill}%
\end{pgfscope}%
\begin{pgfscope}%
\pgfpathrectangle{\pgfqpoint{0.100000in}{0.212622in}}{\pgfqpoint{3.696000in}{3.696000in}}%
\pgfusepath{clip}%
\pgfsetbuttcap%
\pgfsetroundjoin%
\definecolor{currentfill}{rgb}{0.121569,0.466667,0.705882}%
\pgfsetfillcolor{currentfill}%
\pgfsetfillopacity{0.639770}%
\pgfsetlinewidth{1.003750pt}%
\definecolor{currentstroke}{rgb}{0.121569,0.466667,0.705882}%
\pgfsetstrokecolor{currentstroke}%
\pgfsetstrokeopacity{0.639770}%
\pgfsetdash{}{0pt}%
\pgfpathmoveto{\pgfqpoint{0.769598in}{1.401189in}}%
\pgfpathcurveto{\pgfqpoint{0.777835in}{1.401189in}}{\pgfqpoint{0.785735in}{1.404461in}}{\pgfqpoint{0.791559in}{1.410285in}}%
\pgfpathcurveto{\pgfqpoint{0.797383in}{1.416109in}}{\pgfqpoint{0.800655in}{1.424009in}}{\pgfqpoint{0.800655in}{1.432246in}}%
\pgfpathcurveto{\pgfqpoint{0.800655in}{1.440482in}}{\pgfqpoint{0.797383in}{1.448382in}}{\pgfqpoint{0.791559in}{1.454206in}}%
\pgfpathcurveto{\pgfqpoint{0.785735in}{1.460030in}}{\pgfqpoint{0.777835in}{1.463302in}}{\pgfqpoint{0.769598in}{1.463302in}}%
\pgfpathcurveto{\pgfqpoint{0.761362in}{1.463302in}}{\pgfqpoint{0.753462in}{1.460030in}}{\pgfqpoint{0.747638in}{1.454206in}}%
\pgfpathcurveto{\pgfqpoint{0.741814in}{1.448382in}}{\pgfqpoint{0.738542in}{1.440482in}}{\pgfqpoint{0.738542in}{1.432246in}}%
\pgfpathcurveto{\pgfqpoint{0.738542in}{1.424009in}}{\pgfqpoint{0.741814in}{1.416109in}}{\pgfqpoint{0.747638in}{1.410285in}}%
\pgfpathcurveto{\pgfqpoint{0.753462in}{1.404461in}}{\pgfqpoint{0.761362in}{1.401189in}}{\pgfqpoint{0.769598in}{1.401189in}}%
\pgfpathclose%
\pgfusepath{stroke,fill}%
\end{pgfscope}%
\begin{pgfscope}%
\pgfpathrectangle{\pgfqpoint{0.100000in}{0.212622in}}{\pgfqpoint{3.696000in}{3.696000in}}%
\pgfusepath{clip}%
\pgfsetbuttcap%
\pgfsetroundjoin%
\definecolor{currentfill}{rgb}{0.121569,0.466667,0.705882}%
\pgfsetfillcolor{currentfill}%
\pgfsetfillopacity{0.639770}%
\pgfsetlinewidth{1.003750pt}%
\definecolor{currentstroke}{rgb}{0.121569,0.466667,0.705882}%
\pgfsetstrokecolor{currentstroke}%
\pgfsetstrokeopacity{0.639770}%
\pgfsetdash{}{0pt}%
\pgfpathmoveto{\pgfqpoint{0.769598in}{1.401189in}}%
\pgfpathcurveto{\pgfqpoint{0.777835in}{1.401189in}}{\pgfqpoint{0.785735in}{1.404461in}}{\pgfqpoint{0.791559in}{1.410285in}}%
\pgfpathcurveto{\pgfqpoint{0.797383in}{1.416109in}}{\pgfqpoint{0.800655in}{1.424009in}}{\pgfqpoint{0.800655in}{1.432246in}}%
\pgfpathcurveto{\pgfqpoint{0.800655in}{1.440482in}}{\pgfqpoint{0.797383in}{1.448382in}}{\pgfqpoint{0.791559in}{1.454206in}}%
\pgfpathcurveto{\pgfqpoint{0.785735in}{1.460030in}}{\pgfqpoint{0.777835in}{1.463302in}}{\pgfqpoint{0.769598in}{1.463302in}}%
\pgfpathcurveto{\pgfqpoint{0.761362in}{1.463302in}}{\pgfqpoint{0.753462in}{1.460030in}}{\pgfqpoint{0.747638in}{1.454206in}}%
\pgfpathcurveto{\pgfqpoint{0.741814in}{1.448382in}}{\pgfqpoint{0.738542in}{1.440482in}}{\pgfqpoint{0.738542in}{1.432246in}}%
\pgfpathcurveto{\pgfqpoint{0.738542in}{1.424009in}}{\pgfqpoint{0.741814in}{1.416109in}}{\pgfqpoint{0.747638in}{1.410285in}}%
\pgfpathcurveto{\pgfqpoint{0.753462in}{1.404461in}}{\pgfqpoint{0.761362in}{1.401189in}}{\pgfqpoint{0.769598in}{1.401189in}}%
\pgfpathclose%
\pgfusepath{stroke,fill}%
\end{pgfscope}%
\begin{pgfscope}%
\pgfpathrectangle{\pgfqpoint{0.100000in}{0.212622in}}{\pgfqpoint{3.696000in}{3.696000in}}%
\pgfusepath{clip}%
\pgfsetbuttcap%
\pgfsetroundjoin%
\definecolor{currentfill}{rgb}{0.121569,0.466667,0.705882}%
\pgfsetfillcolor{currentfill}%
\pgfsetfillopacity{0.639770}%
\pgfsetlinewidth{1.003750pt}%
\definecolor{currentstroke}{rgb}{0.121569,0.466667,0.705882}%
\pgfsetstrokecolor{currentstroke}%
\pgfsetstrokeopacity{0.639770}%
\pgfsetdash{}{0pt}%
\pgfpathmoveto{\pgfqpoint{0.769598in}{1.401189in}}%
\pgfpathcurveto{\pgfqpoint{0.777835in}{1.401189in}}{\pgfqpoint{0.785735in}{1.404461in}}{\pgfqpoint{0.791559in}{1.410285in}}%
\pgfpathcurveto{\pgfqpoint{0.797383in}{1.416109in}}{\pgfqpoint{0.800655in}{1.424009in}}{\pgfqpoint{0.800655in}{1.432246in}}%
\pgfpathcurveto{\pgfqpoint{0.800655in}{1.440482in}}{\pgfqpoint{0.797383in}{1.448382in}}{\pgfqpoint{0.791559in}{1.454206in}}%
\pgfpathcurveto{\pgfqpoint{0.785735in}{1.460030in}}{\pgfqpoint{0.777835in}{1.463302in}}{\pgfqpoint{0.769598in}{1.463302in}}%
\pgfpathcurveto{\pgfqpoint{0.761362in}{1.463302in}}{\pgfqpoint{0.753462in}{1.460030in}}{\pgfqpoint{0.747638in}{1.454206in}}%
\pgfpathcurveto{\pgfqpoint{0.741814in}{1.448382in}}{\pgfqpoint{0.738542in}{1.440482in}}{\pgfqpoint{0.738542in}{1.432246in}}%
\pgfpathcurveto{\pgfqpoint{0.738542in}{1.424009in}}{\pgfqpoint{0.741814in}{1.416109in}}{\pgfqpoint{0.747638in}{1.410285in}}%
\pgfpathcurveto{\pgfqpoint{0.753462in}{1.404461in}}{\pgfqpoint{0.761362in}{1.401189in}}{\pgfqpoint{0.769598in}{1.401189in}}%
\pgfpathclose%
\pgfusepath{stroke,fill}%
\end{pgfscope}%
\begin{pgfscope}%
\pgfpathrectangle{\pgfqpoint{0.100000in}{0.212622in}}{\pgfqpoint{3.696000in}{3.696000in}}%
\pgfusepath{clip}%
\pgfsetbuttcap%
\pgfsetroundjoin%
\definecolor{currentfill}{rgb}{0.121569,0.466667,0.705882}%
\pgfsetfillcolor{currentfill}%
\pgfsetfillopacity{0.639770}%
\pgfsetlinewidth{1.003750pt}%
\definecolor{currentstroke}{rgb}{0.121569,0.466667,0.705882}%
\pgfsetstrokecolor{currentstroke}%
\pgfsetstrokeopacity{0.639770}%
\pgfsetdash{}{0pt}%
\pgfpathmoveto{\pgfqpoint{0.769598in}{1.401189in}}%
\pgfpathcurveto{\pgfqpoint{0.777835in}{1.401189in}}{\pgfqpoint{0.785735in}{1.404461in}}{\pgfqpoint{0.791559in}{1.410285in}}%
\pgfpathcurveto{\pgfqpoint{0.797383in}{1.416109in}}{\pgfqpoint{0.800655in}{1.424009in}}{\pgfqpoint{0.800655in}{1.432246in}}%
\pgfpathcurveto{\pgfqpoint{0.800655in}{1.440482in}}{\pgfqpoint{0.797383in}{1.448382in}}{\pgfqpoint{0.791559in}{1.454206in}}%
\pgfpathcurveto{\pgfqpoint{0.785735in}{1.460030in}}{\pgfqpoint{0.777835in}{1.463302in}}{\pgfqpoint{0.769598in}{1.463302in}}%
\pgfpathcurveto{\pgfqpoint{0.761362in}{1.463302in}}{\pgfqpoint{0.753462in}{1.460030in}}{\pgfqpoint{0.747638in}{1.454206in}}%
\pgfpathcurveto{\pgfqpoint{0.741814in}{1.448382in}}{\pgfqpoint{0.738542in}{1.440482in}}{\pgfqpoint{0.738542in}{1.432246in}}%
\pgfpathcurveto{\pgfqpoint{0.738542in}{1.424009in}}{\pgfqpoint{0.741814in}{1.416109in}}{\pgfqpoint{0.747638in}{1.410285in}}%
\pgfpathcurveto{\pgfqpoint{0.753462in}{1.404461in}}{\pgfqpoint{0.761362in}{1.401189in}}{\pgfqpoint{0.769598in}{1.401189in}}%
\pgfpathclose%
\pgfusepath{stroke,fill}%
\end{pgfscope}%
\begin{pgfscope}%
\pgfpathrectangle{\pgfqpoint{0.100000in}{0.212622in}}{\pgfqpoint{3.696000in}{3.696000in}}%
\pgfusepath{clip}%
\pgfsetbuttcap%
\pgfsetroundjoin%
\definecolor{currentfill}{rgb}{0.121569,0.466667,0.705882}%
\pgfsetfillcolor{currentfill}%
\pgfsetfillopacity{0.639770}%
\pgfsetlinewidth{1.003750pt}%
\definecolor{currentstroke}{rgb}{0.121569,0.466667,0.705882}%
\pgfsetstrokecolor{currentstroke}%
\pgfsetstrokeopacity{0.639770}%
\pgfsetdash{}{0pt}%
\pgfpathmoveto{\pgfqpoint{0.769598in}{1.401189in}}%
\pgfpathcurveto{\pgfqpoint{0.777835in}{1.401189in}}{\pgfqpoint{0.785735in}{1.404461in}}{\pgfqpoint{0.791559in}{1.410285in}}%
\pgfpathcurveto{\pgfqpoint{0.797383in}{1.416109in}}{\pgfqpoint{0.800655in}{1.424009in}}{\pgfqpoint{0.800655in}{1.432246in}}%
\pgfpathcurveto{\pgfqpoint{0.800655in}{1.440482in}}{\pgfqpoint{0.797383in}{1.448382in}}{\pgfqpoint{0.791559in}{1.454206in}}%
\pgfpathcurveto{\pgfqpoint{0.785735in}{1.460030in}}{\pgfqpoint{0.777835in}{1.463302in}}{\pgfqpoint{0.769598in}{1.463302in}}%
\pgfpathcurveto{\pgfqpoint{0.761362in}{1.463302in}}{\pgfqpoint{0.753462in}{1.460030in}}{\pgfqpoint{0.747638in}{1.454206in}}%
\pgfpathcurveto{\pgfqpoint{0.741814in}{1.448382in}}{\pgfqpoint{0.738542in}{1.440482in}}{\pgfqpoint{0.738542in}{1.432246in}}%
\pgfpathcurveto{\pgfqpoint{0.738542in}{1.424009in}}{\pgfqpoint{0.741814in}{1.416109in}}{\pgfqpoint{0.747638in}{1.410285in}}%
\pgfpathcurveto{\pgfqpoint{0.753462in}{1.404461in}}{\pgfqpoint{0.761362in}{1.401189in}}{\pgfqpoint{0.769598in}{1.401189in}}%
\pgfpathclose%
\pgfusepath{stroke,fill}%
\end{pgfscope}%
\begin{pgfscope}%
\pgfpathrectangle{\pgfqpoint{0.100000in}{0.212622in}}{\pgfqpoint{3.696000in}{3.696000in}}%
\pgfusepath{clip}%
\pgfsetbuttcap%
\pgfsetroundjoin%
\definecolor{currentfill}{rgb}{0.121569,0.466667,0.705882}%
\pgfsetfillcolor{currentfill}%
\pgfsetfillopacity{0.639770}%
\pgfsetlinewidth{1.003750pt}%
\definecolor{currentstroke}{rgb}{0.121569,0.466667,0.705882}%
\pgfsetstrokecolor{currentstroke}%
\pgfsetstrokeopacity{0.639770}%
\pgfsetdash{}{0pt}%
\pgfpathmoveto{\pgfqpoint{0.769598in}{1.401189in}}%
\pgfpathcurveto{\pgfqpoint{0.777835in}{1.401189in}}{\pgfqpoint{0.785735in}{1.404461in}}{\pgfqpoint{0.791559in}{1.410285in}}%
\pgfpathcurveto{\pgfqpoint{0.797383in}{1.416109in}}{\pgfqpoint{0.800655in}{1.424009in}}{\pgfqpoint{0.800655in}{1.432246in}}%
\pgfpathcurveto{\pgfqpoint{0.800655in}{1.440482in}}{\pgfqpoint{0.797383in}{1.448382in}}{\pgfqpoint{0.791559in}{1.454206in}}%
\pgfpathcurveto{\pgfqpoint{0.785735in}{1.460030in}}{\pgfqpoint{0.777835in}{1.463302in}}{\pgfqpoint{0.769598in}{1.463302in}}%
\pgfpathcurveto{\pgfqpoint{0.761362in}{1.463302in}}{\pgfqpoint{0.753462in}{1.460030in}}{\pgfqpoint{0.747638in}{1.454206in}}%
\pgfpathcurveto{\pgfqpoint{0.741814in}{1.448382in}}{\pgfqpoint{0.738542in}{1.440482in}}{\pgfqpoint{0.738542in}{1.432246in}}%
\pgfpathcurveto{\pgfqpoint{0.738542in}{1.424009in}}{\pgfqpoint{0.741814in}{1.416109in}}{\pgfqpoint{0.747638in}{1.410285in}}%
\pgfpathcurveto{\pgfqpoint{0.753462in}{1.404461in}}{\pgfqpoint{0.761362in}{1.401189in}}{\pgfqpoint{0.769598in}{1.401189in}}%
\pgfpathclose%
\pgfusepath{stroke,fill}%
\end{pgfscope}%
\begin{pgfscope}%
\pgfpathrectangle{\pgfqpoint{0.100000in}{0.212622in}}{\pgfqpoint{3.696000in}{3.696000in}}%
\pgfusepath{clip}%
\pgfsetbuttcap%
\pgfsetroundjoin%
\definecolor{currentfill}{rgb}{0.121569,0.466667,0.705882}%
\pgfsetfillcolor{currentfill}%
\pgfsetfillopacity{0.639770}%
\pgfsetlinewidth{1.003750pt}%
\definecolor{currentstroke}{rgb}{0.121569,0.466667,0.705882}%
\pgfsetstrokecolor{currentstroke}%
\pgfsetstrokeopacity{0.639770}%
\pgfsetdash{}{0pt}%
\pgfpathmoveto{\pgfqpoint{0.769598in}{1.401189in}}%
\pgfpathcurveto{\pgfqpoint{0.777835in}{1.401189in}}{\pgfqpoint{0.785735in}{1.404461in}}{\pgfqpoint{0.791559in}{1.410285in}}%
\pgfpathcurveto{\pgfqpoint{0.797383in}{1.416109in}}{\pgfqpoint{0.800655in}{1.424009in}}{\pgfqpoint{0.800655in}{1.432246in}}%
\pgfpathcurveto{\pgfqpoint{0.800655in}{1.440482in}}{\pgfqpoint{0.797383in}{1.448382in}}{\pgfqpoint{0.791559in}{1.454206in}}%
\pgfpathcurveto{\pgfqpoint{0.785735in}{1.460030in}}{\pgfqpoint{0.777835in}{1.463302in}}{\pgfqpoint{0.769598in}{1.463302in}}%
\pgfpathcurveto{\pgfqpoint{0.761362in}{1.463302in}}{\pgfqpoint{0.753462in}{1.460030in}}{\pgfqpoint{0.747638in}{1.454206in}}%
\pgfpathcurveto{\pgfqpoint{0.741814in}{1.448382in}}{\pgfqpoint{0.738542in}{1.440482in}}{\pgfqpoint{0.738542in}{1.432246in}}%
\pgfpathcurveto{\pgfqpoint{0.738542in}{1.424009in}}{\pgfqpoint{0.741814in}{1.416109in}}{\pgfqpoint{0.747638in}{1.410285in}}%
\pgfpathcurveto{\pgfqpoint{0.753462in}{1.404461in}}{\pgfqpoint{0.761362in}{1.401189in}}{\pgfqpoint{0.769598in}{1.401189in}}%
\pgfpathclose%
\pgfusepath{stroke,fill}%
\end{pgfscope}%
\begin{pgfscope}%
\pgfpathrectangle{\pgfqpoint{0.100000in}{0.212622in}}{\pgfqpoint{3.696000in}{3.696000in}}%
\pgfusepath{clip}%
\pgfsetbuttcap%
\pgfsetroundjoin%
\definecolor{currentfill}{rgb}{0.121569,0.466667,0.705882}%
\pgfsetfillcolor{currentfill}%
\pgfsetfillopacity{0.639770}%
\pgfsetlinewidth{1.003750pt}%
\definecolor{currentstroke}{rgb}{0.121569,0.466667,0.705882}%
\pgfsetstrokecolor{currentstroke}%
\pgfsetstrokeopacity{0.639770}%
\pgfsetdash{}{0pt}%
\pgfpathmoveto{\pgfqpoint{0.769598in}{1.401189in}}%
\pgfpathcurveto{\pgfqpoint{0.777835in}{1.401189in}}{\pgfqpoint{0.785735in}{1.404461in}}{\pgfqpoint{0.791559in}{1.410285in}}%
\pgfpathcurveto{\pgfqpoint{0.797383in}{1.416109in}}{\pgfqpoint{0.800655in}{1.424009in}}{\pgfqpoint{0.800655in}{1.432246in}}%
\pgfpathcurveto{\pgfqpoint{0.800655in}{1.440482in}}{\pgfqpoint{0.797383in}{1.448382in}}{\pgfqpoint{0.791559in}{1.454206in}}%
\pgfpathcurveto{\pgfqpoint{0.785735in}{1.460030in}}{\pgfqpoint{0.777835in}{1.463302in}}{\pgfqpoint{0.769598in}{1.463302in}}%
\pgfpathcurveto{\pgfqpoint{0.761362in}{1.463302in}}{\pgfqpoint{0.753462in}{1.460030in}}{\pgfqpoint{0.747638in}{1.454206in}}%
\pgfpathcurveto{\pgfqpoint{0.741814in}{1.448382in}}{\pgfqpoint{0.738542in}{1.440482in}}{\pgfqpoint{0.738542in}{1.432246in}}%
\pgfpathcurveto{\pgfqpoint{0.738542in}{1.424009in}}{\pgfqpoint{0.741814in}{1.416109in}}{\pgfqpoint{0.747638in}{1.410285in}}%
\pgfpathcurveto{\pgfqpoint{0.753462in}{1.404461in}}{\pgfqpoint{0.761362in}{1.401189in}}{\pgfqpoint{0.769598in}{1.401189in}}%
\pgfpathclose%
\pgfusepath{stroke,fill}%
\end{pgfscope}%
\begin{pgfscope}%
\pgfpathrectangle{\pgfqpoint{0.100000in}{0.212622in}}{\pgfqpoint{3.696000in}{3.696000in}}%
\pgfusepath{clip}%
\pgfsetbuttcap%
\pgfsetroundjoin%
\definecolor{currentfill}{rgb}{0.121569,0.466667,0.705882}%
\pgfsetfillcolor{currentfill}%
\pgfsetfillopacity{0.639770}%
\pgfsetlinewidth{1.003750pt}%
\definecolor{currentstroke}{rgb}{0.121569,0.466667,0.705882}%
\pgfsetstrokecolor{currentstroke}%
\pgfsetstrokeopacity{0.639770}%
\pgfsetdash{}{0pt}%
\pgfpathmoveto{\pgfqpoint{0.769598in}{1.401189in}}%
\pgfpathcurveto{\pgfqpoint{0.777835in}{1.401189in}}{\pgfqpoint{0.785735in}{1.404461in}}{\pgfqpoint{0.791559in}{1.410285in}}%
\pgfpathcurveto{\pgfqpoint{0.797383in}{1.416109in}}{\pgfqpoint{0.800655in}{1.424009in}}{\pgfqpoint{0.800655in}{1.432246in}}%
\pgfpathcurveto{\pgfqpoint{0.800655in}{1.440482in}}{\pgfqpoint{0.797383in}{1.448382in}}{\pgfqpoint{0.791559in}{1.454206in}}%
\pgfpathcurveto{\pgfqpoint{0.785735in}{1.460030in}}{\pgfqpoint{0.777835in}{1.463302in}}{\pgfqpoint{0.769598in}{1.463302in}}%
\pgfpathcurveto{\pgfqpoint{0.761362in}{1.463302in}}{\pgfqpoint{0.753462in}{1.460030in}}{\pgfqpoint{0.747638in}{1.454206in}}%
\pgfpathcurveto{\pgfqpoint{0.741814in}{1.448382in}}{\pgfqpoint{0.738542in}{1.440482in}}{\pgfqpoint{0.738542in}{1.432246in}}%
\pgfpathcurveto{\pgfqpoint{0.738542in}{1.424009in}}{\pgfqpoint{0.741814in}{1.416109in}}{\pgfqpoint{0.747638in}{1.410285in}}%
\pgfpathcurveto{\pgfqpoint{0.753462in}{1.404461in}}{\pgfqpoint{0.761362in}{1.401189in}}{\pgfqpoint{0.769598in}{1.401189in}}%
\pgfpathclose%
\pgfusepath{stroke,fill}%
\end{pgfscope}%
\begin{pgfscope}%
\pgfpathrectangle{\pgfqpoint{0.100000in}{0.212622in}}{\pgfqpoint{3.696000in}{3.696000in}}%
\pgfusepath{clip}%
\pgfsetbuttcap%
\pgfsetroundjoin%
\definecolor{currentfill}{rgb}{0.121569,0.466667,0.705882}%
\pgfsetfillcolor{currentfill}%
\pgfsetfillopacity{0.639770}%
\pgfsetlinewidth{1.003750pt}%
\definecolor{currentstroke}{rgb}{0.121569,0.466667,0.705882}%
\pgfsetstrokecolor{currentstroke}%
\pgfsetstrokeopacity{0.639770}%
\pgfsetdash{}{0pt}%
\pgfpathmoveto{\pgfqpoint{0.769598in}{1.401189in}}%
\pgfpathcurveto{\pgfqpoint{0.777835in}{1.401189in}}{\pgfqpoint{0.785735in}{1.404461in}}{\pgfqpoint{0.791559in}{1.410285in}}%
\pgfpathcurveto{\pgfqpoint{0.797383in}{1.416109in}}{\pgfqpoint{0.800655in}{1.424009in}}{\pgfqpoint{0.800655in}{1.432246in}}%
\pgfpathcurveto{\pgfqpoint{0.800655in}{1.440482in}}{\pgfqpoint{0.797383in}{1.448382in}}{\pgfqpoint{0.791559in}{1.454206in}}%
\pgfpathcurveto{\pgfqpoint{0.785735in}{1.460030in}}{\pgfqpoint{0.777835in}{1.463302in}}{\pgfqpoint{0.769598in}{1.463302in}}%
\pgfpathcurveto{\pgfqpoint{0.761362in}{1.463302in}}{\pgfqpoint{0.753462in}{1.460030in}}{\pgfqpoint{0.747638in}{1.454206in}}%
\pgfpathcurveto{\pgfqpoint{0.741814in}{1.448382in}}{\pgfqpoint{0.738542in}{1.440482in}}{\pgfqpoint{0.738542in}{1.432246in}}%
\pgfpathcurveto{\pgfqpoint{0.738542in}{1.424009in}}{\pgfqpoint{0.741814in}{1.416109in}}{\pgfqpoint{0.747638in}{1.410285in}}%
\pgfpathcurveto{\pgfqpoint{0.753462in}{1.404461in}}{\pgfqpoint{0.761362in}{1.401189in}}{\pgfqpoint{0.769598in}{1.401189in}}%
\pgfpathclose%
\pgfusepath{stroke,fill}%
\end{pgfscope}%
\begin{pgfscope}%
\pgfpathrectangle{\pgfqpoint{0.100000in}{0.212622in}}{\pgfqpoint{3.696000in}{3.696000in}}%
\pgfusepath{clip}%
\pgfsetbuttcap%
\pgfsetroundjoin%
\definecolor{currentfill}{rgb}{0.121569,0.466667,0.705882}%
\pgfsetfillcolor{currentfill}%
\pgfsetfillopacity{0.639770}%
\pgfsetlinewidth{1.003750pt}%
\definecolor{currentstroke}{rgb}{0.121569,0.466667,0.705882}%
\pgfsetstrokecolor{currentstroke}%
\pgfsetstrokeopacity{0.639770}%
\pgfsetdash{}{0pt}%
\pgfpathmoveto{\pgfqpoint{0.769598in}{1.401189in}}%
\pgfpathcurveto{\pgfqpoint{0.777835in}{1.401189in}}{\pgfqpoint{0.785735in}{1.404461in}}{\pgfqpoint{0.791559in}{1.410285in}}%
\pgfpathcurveto{\pgfqpoint{0.797383in}{1.416109in}}{\pgfqpoint{0.800655in}{1.424009in}}{\pgfqpoint{0.800655in}{1.432246in}}%
\pgfpathcurveto{\pgfqpoint{0.800655in}{1.440482in}}{\pgfqpoint{0.797383in}{1.448382in}}{\pgfqpoint{0.791559in}{1.454206in}}%
\pgfpathcurveto{\pgfqpoint{0.785735in}{1.460030in}}{\pgfqpoint{0.777835in}{1.463302in}}{\pgfqpoint{0.769598in}{1.463302in}}%
\pgfpathcurveto{\pgfqpoint{0.761362in}{1.463302in}}{\pgfqpoint{0.753462in}{1.460030in}}{\pgfqpoint{0.747638in}{1.454206in}}%
\pgfpathcurveto{\pgfqpoint{0.741814in}{1.448382in}}{\pgfqpoint{0.738542in}{1.440482in}}{\pgfqpoint{0.738542in}{1.432246in}}%
\pgfpathcurveto{\pgfqpoint{0.738542in}{1.424009in}}{\pgfqpoint{0.741814in}{1.416109in}}{\pgfqpoint{0.747638in}{1.410285in}}%
\pgfpathcurveto{\pgfqpoint{0.753462in}{1.404461in}}{\pgfqpoint{0.761362in}{1.401189in}}{\pgfqpoint{0.769598in}{1.401189in}}%
\pgfpathclose%
\pgfusepath{stroke,fill}%
\end{pgfscope}%
\begin{pgfscope}%
\pgfpathrectangle{\pgfqpoint{0.100000in}{0.212622in}}{\pgfqpoint{3.696000in}{3.696000in}}%
\pgfusepath{clip}%
\pgfsetbuttcap%
\pgfsetroundjoin%
\definecolor{currentfill}{rgb}{0.121569,0.466667,0.705882}%
\pgfsetfillcolor{currentfill}%
\pgfsetfillopacity{0.639770}%
\pgfsetlinewidth{1.003750pt}%
\definecolor{currentstroke}{rgb}{0.121569,0.466667,0.705882}%
\pgfsetstrokecolor{currentstroke}%
\pgfsetstrokeopacity{0.639770}%
\pgfsetdash{}{0pt}%
\pgfpathmoveto{\pgfqpoint{0.769598in}{1.401189in}}%
\pgfpathcurveto{\pgfqpoint{0.777835in}{1.401189in}}{\pgfqpoint{0.785735in}{1.404461in}}{\pgfqpoint{0.791559in}{1.410285in}}%
\pgfpathcurveto{\pgfqpoint{0.797383in}{1.416109in}}{\pgfqpoint{0.800655in}{1.424009in}}{\pgfqpoint{0.800655in}{1.432246in}}%
\pgfpathcurveto{\pgfqpoint{0.800655in}{1.440482in}}{\pgfqpoint{0.797383in}{1.448382in}}{\pgfqpoint{0.791559in}{1.454206in}}%
\pgfpathcurveto{\pgfqpoint{0.785735in}{1.460030in}}{\pgfqpoint{0.777835in}{1.463302in}}{\pgfqpoint{0.769598in}{1.463302in}}%
\pgfpathcurveto{\pgfqpoint{0.761362in}{1.463302in}}{\pgfqpoint{0.753462in}{1.460030in}}{\pgfqpoint{0.747638in}{1.454206in}}%
\pgfpathcurveto{\pgfqpoint{0.741814in}{1.448382in}}{\pgfqpoint{0.738542in}{1.440482in}}{\pgfqpoint{0.738542in}{1.432246in}}%
\pgfpathcurveto{\pgfqpoint{0.738542in}{1.424009in}}{\pgfqpoint{0.741814in}{1.416109in}}{\pgfqpoint{0.747638in}{1.410285in}}%
\pgfpathcurveto{\pgfqpoint{0.753462in}{1.404461in}}{\pgfqpoint{0.761362in}{1.401189in}}{\pgfqpoint{0.769598in}{1.401189in}}%
\pgfpathclose%
\pgfusepath{stroke,fill}%
\end{pgfscope}%
\begin{pgfscope}%
\pgfpathrectangle{\pgfqpoint{0.100000in}{0.212622in}}{\pgfqpoint{3.696000in}{3.696000in}}%
\pgfusepath{clip}%
\pgfsetbuttcap%
\pgfsetroundjoin%
\definecolor{currentfill}{rgb}{0.121569,0.466667,0.705882}%
\pgfsetfillcolor{currentfill}%
\pgfsetfillopacity{0.639770}%
\pgfsetlinewidth{1.003750pt}%
\definecolor{currentstroke}{rgb}{0.121569,0.466667,0.705882}%
\pgfsetstrokecolor{currentstroke}%
\pgfsetstrokeopacity{0.639770}%
\pgfsetdash{}{0pt}%
\pgfpathmoveto{\pgfqpoint{0.769598in}{1.401189in}}%
\pgfpathcurveto{\pgfqpoint{0.777835in}{1.401189in}}{\pgfqpoint{0.785735in}{1.404461in}}{\pgfqpoint{0.791559in}{1.410285in}}%
\pgfpathcurveto{\pgfqpoint{0.797383in}{1.416109in}}{\pgfqpoint{0.800655in}{1.424009in}}{\pgfqpoint{0.800655in}{1.432246in}}%
\pgfpathcurveto{\pgfqpoint{0.800655in}{1.440482in}}{\pgfqpoint{0.797383in}{1.448382in}}{\pgfqpoint{0.791559in}{1.454206in}}%
\pgfpathcurveto{\pgfqpoint{0.785735in}{1.460030in}}{\pgfqpoint{0.777835in}{1.463302in}}{\pgfqpoint{0.769598in}{1.463302in}}%
\pgfpathcurveto{\pgfqpoint{0.761362in}{1.463302in}}{\pgfqpoint{0.753462in}{1.460030in}}{\pgfqpoint{0.747638in}{1.454206in}}%
\pgfpathcurveto{\pgfqpoint{0.741814in}{1.448382in}}{\pgfqpoint{0.738542in}{1.440482in}}{\pgfqpoint{0.738542in}{1.432246in}}%
\pgfpathcurveto{\pgfqpoint{0.738542in}{1.424009in}}{\pgfqpoint{0.741814in}{1.416109in}}{\pgfqpoint{0.747638in}{1.410285in}}%
\pgfpathcurveto{\pgfqpoint{0.753462in}{1.404461in}}{\pgfqpoint{0.761362in}{1.401189in}}{\pgfqpoint{0.769598in}{1.401189in}}%
\pgfpathclose%
\pgfusepath{stroke,fill}%
\end{pgfscope}%
\begin{pgfscope}%
\pgfpathrectangle{\pgfqpoint{0.100000in}{0.212622in}}{\pgfqpoint{3.696000in}{3.696000in}}%
\pgfusepath{clip}%
\pgfsetbuttcap%
\pgfsetroundjoin%
\definecolor{currentfill}{rgb}{0.121569,0.466667,0.705882}%
\pgfsetfillcolor{currentfill}%
\pgfsetfillopacity{0.639770}%
\pgfsetlinewidth{1.003750pt}%
\definecolor{currentstroke}{rgb}{0.121569,0.466667,0.705882}%
\pgfsetstrokecolor{currentstroke}%
\pgfsetstrokeopacity{0.639770}%
\pgfsetdash{}{0pt}%
\pgfpathmoveto{\pgfqpoint{0.769598in}{1.401189in}}%
\pgfpathcurveto{\pgfqpoint{0.777835in}{1.401189in}}{\pgfqpoint{0.785735in}{1.404461in}}{\pgfqpoint{0.791559in}{1.410285in}}%
\pgfpathcurveto{\pgfqpoint{0.797383in}{1.416109in}}{\pgfqpoint{0.800655in}{1.424009in}}{\pgfqpoint{0.800655in}{1.432246in}}%
\pgfpathcurveto{\pgfqpoint{0.800655in}{1.440482in}}{\pgfqpoint{0.797383in}{1.448382in}}{\pgfqpoint{0.791559in}{1.454206in}}%
\pgfpathcurveto{\pgfqpoint{0.785735in}{1.460030in}}{\pgfqpoint{0.777835in}{1.463302in}}{\pgfqpoint{0.769598in}{1.463302in}}%
\pgfpathcurveto{\pgfqpoint{0.761362in}{1.463302in}}{\pgfqpoint{0.753462in}{1.460030in}}{\pgfqpoint{0.747638in}{1.454206in}}%
\pgfpathcurveto{\pgfqpoint{0.741814in}{1.448382in}}{\pgfqpoint{0.738542in}{1.440482in}}{\pgfqpoint{0.738542in}{1.432246in}}%
\pgfpathcurveto{\pgfqpoint{0.738542in}{1.424009in}}{\pgfqpoint{0.741814in}{1.416109in}}{\pgfqpoint{0.747638in}{1.410285in}}%
\pgfpathcurveto{\pgfqpoint{0.753462in}{1.404461in}}{\pgfqpoint{0.761362in}{1.401189in}}{\pgfqpoint{0.769598in}{1.401189in}}%
\pgfpathclose%
\pgfusepath{stroke,fill}%
\end{pgfscope}%
\begin{pgfscope}%
\pgfpathrectangle{\pgfqpoint{0.100000in}{0.212622in}}{\pgfqpoint{3.696000in}{3.696000in}}%
\pgfusepath{clip}%
\pgfsetbuttcap%
\pgfsetroundjoin%
\definecolor{currentfill}{rgb}{0.121569,0.466667,0.705882}%
\pgfsetfillcolor{currentfill}%
\pgfsetfillopacity{0.639770}%
\pgfsetlinewidth{1.003750pt}%
\definecolor{currentstroke}{rgb}{0.121569,0.466667,0.705882}%
\pgfsetstrokecolor{currentstroke}%
\pgfsetstrokeopacity{0.639770}%
\pgfsetdash{}{0pt}%
\pgfpathmoveto{\pgfqpoint{0.769598in}{1.401189in}}%
\pgfpathcurveto{\pgfqpoint{0.777835in}{1.401189in}}{\pgfqpoint{0.785735in}{1.404461in}}{\pgfqpoint{0.791559in}{1.410285in}}%
\pgfpathcurveto{\pgfqpoint{0.797383in}{1.416109in}}{\pgfqpoint{0.800655in}{1.424009in}}{\pgfqpoint{0.800655in}{1.432246in}}%
\pgfpathcurveto{\pgfqpoint{0.800655in}{1.440482in}}{\pgfqpoint{0.797383in}{1.448382in}}{\pgfqpoint{0.791559in}{1.454206in}}%
\pgfpathcurveto{\pgfqpoint{0.785735in}{1.460030in}}{\pgfqpoint{0.777835in}{1.463302in}}{\pgfqpoint{0.769598in}{1.463302in}}%
\pgfpathcurveto{\pgfqpoint{0.761362in}{1.463302in}}{\pgfqpoint{0.753462in}{1.460030in}}{\pgfqpoint{0.747638in}{1.454206in}}%
\pgfpathcurveto{\pgfqpoint{0.741814in}{1.448382in}}{\pgfqpoint{0.738542in}{1.440482in}}{\pgfqpoint{0.738542in}{1.432246in}}%
\pgfpathcurveto{\pgfqpoint{0.738542in}{1.424009in}}{\pgfqpoint{0.741814in}{1.416109in}}{\pgfqpoint{0.747638in}{1.410285in}}%
\pgfpathcurveto{\pgfqpoint{0.753462in}{1.404461in}}{\pgfqpoint{0.761362in}{1.401189in}}{\pgfqpoint{0.769598in}{1.401189in}}%
\pgfpathclose%
\pgfusepath{stroke,fill}%
\end{pgfscope}%
\begin{pgfscope}%
\pgfpathrectangle{\pgfqpoint{0.100000in}{0.212622in}}{\pgfqpoint{3.696000in}{3.696000in}}%
\pgfusepath{clip}%
\pgfsetbuttcap%
\pgfsetroundjoin%
\definecolor{currentfill}{rgb}{0.121569,0.466667,0.705882}%
\pgfsetfillcolor{currentfill}%
\pgfsetfillopacity{0.639770}%
\pgfsetlinewidth{1.003750pt}%
\definecolor{currentstroke}{rgb}{0.121569,0.466667,0.705882}%
\pgfsetstrokecolor{currentstroke}%
\pgfsetstrokeopacity{0.639770}%
\pgfsetdash{}{0pt}%
\pgfpathmoveto{\pgfqpoint{0.769598in}{1.401189in}}%
\pgfpathcurveto{\pgfqpoint{0.777835in}{1.401189in}}{\pgfqpoint{0.785735in}{1.404461in}}{\pgfqpoint{0.791559in}{1.410285in}}%
\pgfpathcurveto{\pgfqpoint{0.797383in}{1.416109in}}{\pgfqpoint{0.800655in}{1.424009in}}{\pgfqpoint{0.800655in}{1.432246in}}%
\pgfpathcurveto{\pgfqpoint{0.800655in}{1.440482in}}{\pgfqpoint{0.797383in}{1.448382in}}{\pgfqpoint{0.791559in}{1.454206in}}%
\pgfpathcurveto{\pgfqpoint{0.785735in}{1.460030in}}{\pgfqpoint{0.777835in}{1.463302in}}{\pgfqpoint{0.769598in}{1.463302in}}%
\pgfpathcurveto{\pgfqpoint{0.761362in}{1.463302in}}{\pgfqpoint{0.753462in}{1.460030in}}{\pgfqpoint{0.747638in}{1.454206in}}%
\pgfpathcurveto{\pgfqpoint{0.741814in}{1.448382in}}{\pgfqpoint{0.738542in}{1.440482in}}{\pgfqpoint{0.738542in}{1.432246in}}%
\pgfpathcurveto{\pgfqpoint{0.738542in}{1.424009in}}{\pgfqpoint{0.741814in}{1.416109in}}{\pgfqpoint{0.747638in}{1.410285in}}%
\pgfpathcurveto{\pgfqpoint{0.753462in}{1.404461in}}{\pgfqpoint{0.761362in}{1.401189in}}{\pgfqpoint{0.769598in}{1.401189in}}%
\pgfpathclose%
\pgfusepath{stroke,fill}%
\end{pgfscope}%
\begin{pgfscope}%
\pgfpathrectangle{\pgfqpoint{0.100000in}{0.212622in}}{\pgfqpoint{3.696000in}{3.696000in}}%
\pgfusepath{clip}%
\pgfsetbuttcap%
\pgfsetroundjoin%
\definecolor{currentfill}{rgb}{0.121569,0.466667,0.705882}%
\pgfsetfillcolor{currentfill}%
\pgfsetfillopacity{0.639770}%
\pgfsetlinewidth{1.003750pt}%
\definecolor{currentstroke}{rgb}{0.121569,0.466667,0.705882}%
\pgfsetstrokecolor{currentstroke}%
\pgfsetstrokeopacity{0.639770}%
\pgfsetdash{}{0pt}%
\pgfpathmoveto{\pgfqpoint{0.769598in}{1.401189in}}%
\pgfpathcurveto{\pgfqpoint{0.777835in}{1.401189in}}{\pgfqpoint{0.785735in}{1.404461in}}{\pgfqpoint{0.791559in}{1.410285in}}%
\pgfpathcurveto{\pgfqpoint{0.797383in}{1.416109in}}{\pgfqpoint{0.800655in}{1.424009in}}{\pgfqpoint{0.800655in}{1.432246in}}%
\pgfpathcurveto{\pgfqpoint{0.800655in}{1.440482in}}{\pgfqpoint{0.797383in}{1.448382in}}{\pgfqpoint{0.791559in}{1.454206in}}%
\pgfpathcurveto{\pgfqpoint{0.785735in}{1.460030in}}{\pgfqpoint{0.777835in}{1.463302in}}{\pgfqpoint{0.769598in}{1.463302in}}%
\pgfpathcurveto{\pgfqpoint{0.761362in}{1.463302in}}{\pgfqpoint{0.753462in}{1.460030in}}{\pgfqpoint{0.747638in}{1.454206in}}%
\pgfpathcurveto{\pgfqpoint{0.741814in}{1.448382in}}{\pgfqpoint{0.738542in}{1.440482in}}{\pgfqpoint{0.738542in}{1.432246in}}%
\pgfpathcurveto{\pgfqpoint{0.738542in}{1.424009in}}{\pgfqpoint{0.741814in}{1.416109in}}{\pgfqpoint{0.747638in}{1.410285in}}%
\pgfpathcurveto{\pgfqpoint{0.753462in}{1.404461in}}{\pgfqpoint{0.761362in}{1.401189in}}{\pgfqpoint{0.769598in}{1.401189in}}%
\pgfpathclose%
\pgfusepath{stroke,fill}%
\end{pgfscope}%
\begin{pgfscope}%
\pgfpathrectangle{\pgfqpoint{0.100000in}{0.212622in}}{\pgfqpoint{3.696000in}{3.696000in}}%
\pgfusepath{clip}%
\pgfsetbuttcap%
\pgfsetroundjoin%
\definecolor{currentfill}{rgb}{0.121569,0.466667,0.705882}%
\pgfsetfillcolor{currentfill}%
\pgfsetfillopacity{0.639770}%
\pgfsetlinewidth{1.003750pt}%
\definecolor{currentstroke}{rgb}{0.121569,0.466667,0.705882}%
\pgfsetstrokecolor{currentstroke}%
\pgfsetstrokeopacity{0.639770}%
\pgfsetdash{}{0pt}%
\pgfpathmoveto{\pgfqpoint{0.769598in}{1.401189in}}%
\pgfpathcurveto{\pgfqpoint{0.777835in}{1.401189in}}{\pgfqpoint{0.785735in}{1.404461in}}{\pgfqpoint{0.791559in}{1.410285in}}%
\pgfpathcurveto{\pgfqpoint{0.797383in}{1.416109in}}{\pgfqpoint{0.800655in}{1.424009in}}{\pgfqpoint{0.800655in}{1.432246in}}%
\pgfpathcurveto{\pgfqpoint{0.800655in}{1.440482in}}{\pgfqpoint{0.797383in}{1.448382in}}{\pgfqpoint{0.791559in}{1.454206in}}%
\pgfpathcurveto{\pgfqpoint{0.785735in}{1.460030in}}{\pgfqpoint{0.777835in}{1.463302in}}{\pgfqpoint{0.769598in}{1.463302in}}%
\pgfpathcurveto{\pgfqpoint{0.761362in}{1.463302in}}{\pgfqpoint{0.753462in}{1.460030in}}{\pgfqpoint{0.747638in}{1.454206in}}%
\pgfpathcurveto{\pgfqpoint{0.741814in}{1.448382in}}{\pgfqpoint{0.738542in}{1.440482in}}{\pgfqpoint{0.738542in}{1.432246in}}%
\pgfpathcurveto{\pgfqpoint{0.738542in}{1.424009in}}{\pgfqpoint{0.741814in}{1.416109in}}{\pgfqpoint{0.747638in}{1.410285in}}%
\pgfpathcurveto{\pgfqpoint{0.753462in}{1.404461in}}{\pgfqpoint{0.761362in}{1.401189in}}{\pgfqpoint{0.769598in}{1.401189in}}%
\pgfpathclose%
\pgfusepath{stroke,fill}%
\end{pgfscope}%
\begin{pgfscope}%
\pgfpathrectangle{\pgfqpoint{0.100000in}{0.212622in}}{\pgfqpoint{3.696000in}{3.696000in}}%
\pgfusepath{clip}%
\pgfsetbuttcap%
\pgfsetroundjoin%
\definecolor{currentfill}{rgb}{0.121569,0.466667,0.705882}%
\pgfsetfillcolor{currentfill}%
\pgfsetfillopacity{0.639770}%
\pgfsetlinewidth{1.003750pt}%
\definecolor{currentstroke}{rgb}{0.121569,0.466667,0.705882}%
\pgfsetstrokecolor{currentstroke}%
\pgfsetstrokeopacity{0.639770}%
\pgfsetdash{}{0pt}%
\pgfpathmoveto{\pgfqpoint{0.769598in}{1.401189in}}%
\pgfpathcurveto{\pgfqpoint{0.777835in}{1.401189in}}{\pgfqpoint{0.785735in}{1.404461in}}{\pgfqpoint{0.791559in}{1.410285in}}%
\pgfpathcurveto{\pgfqpoint{0.797383in}{1.416109in}}{\pgfqpoint{0.800655in}{1.424009in}}{\pgfqpoint{0.800655in}{1.432246in}}%
\pgfpathcurveto{\pgfqpoint{0.800655in}{1.440482in}}{\pgfqpoint{0.797383in}{1.448382in}}{\pgfqpoint{0.791559in}{1.454206in}}%
\pgfpathcurveto{\pgfqpoint{0.785735in}{1.460030in}}{\pgfqpoint{0.777835in}{1.463302in}}{\pgfqpoint{0.769598in}{1.463302in}}%
\pgfpathcurveto{\pgfqpoint{0.761362in}{1.463302in}}{\pgfqpoint{0.753462in}{1.460030in}}{\pgfqpoint{0.747638in}{1.454206in}}%
\pgfpathcurveto{\pgfqpoint{0.741814in}{1.448382in}}{\pgfqpoint{0.738542in}{1.440482in}}{\pgfqpoint{0.738542in}{1.432246in}}%
\pgfpathcurveto{\pgfqpoint{0.738542in}{1.424009in}}{\pgfqpoint{0.741814in}{1.416109in}}{\pgfqpoint{0.747638in}{1.410285in}}%
\pgfpathcurveto{\pgfqpoint{0.753462in}{1.404461in}}{\pgfqpoint{0.761362in}{1.401189in}}{\pgfqpoint{0.769598in}{1.401189in}}%
\pgfpathclose%
\pgfusepath{stroke,fill}%
\end{pgfscope}%
\begin{pgfscope}%
\pgfpathrectangle{\pgfqpoint{0.100000in}{0.212622in}}{\pgfqpoint{3.696000in}{3.696000in}}%
\pgfusepath{clip}%
\pgfsetbuttcap%
\pgfsetroundjoin%
\definecolor{currentfill}{rgb}{0.121569,0.466667,0.705882}%
\pgfsetfillcolor{currentfill}%
\pgfsetfillopacity{0.639770}%
\pgfsetlinewidth{1.003750pt}%
\definecolor{currentstroke}{rgb}{0.121569,0.466667,0.705882}%
\pgfsetstrokecolor{currentstroke}%
\pgfsetstrokeopacity{0.639770}%
\pgfsetdash{}{0pt}%
\pgfpathmoveto{\pgfqpoint{0.769598in}{1.401189in}}%
\pgfpathcurveto{\pgfqpoint{0.777835in}{1.401189in}}{\pgfqpoint{0.785735in}{1.404461in}}{\pgfqpoint{0.791559in}{1.410285in}}%
\pgfpathcurveto{\pgfqpoint{0.797383in}{1.416109in}}{\pgfqpoint{0.800655in}{1.424009in}}{\pgfqpoint{0.800655in}{1.432246in}}%
\pgfpathcurveto{\pgfqpoint{0.800655in}{1.440482in}}{\pgfqpoint{0.797383in}{1.448382in}}{\pgfqpoint{0.791559in}{1.454206in}}%
\pgfpathcurveto{\pgfqpoint{0.785735in}{1.460030in}}{\pgfqpoint{0.777835in}{1.463302in}}{\pgfqpoint{0.769598in}{1.463302in}}%
\pgfpathcurveto{\pgfqpoint{0.761362in}{1.463302in}}{\pgfqpoint{0.753462in}{1.460030in}}{\pgfqpoint{0.747638in}{1.454206in}}%
\pgfpathcurveto{\pgfqpoint{0.741814in}{1.448382in}}{\pgfqpoint{0.738542in}{1.440482in}}{\pgfqpoint{0.738542in}{1.432246in}}%
\pgfpathcurveto{\pgfqpoint{0.738542in}{1.424009in}}{\pgfqpoint{0.741814in}{1.416109in}}{\pgfqpoint{0.747638in}{1.410285in}}%
\pgfpathcurveto{\pgfqpoint{0.753462in}{1.404461in}}{\pgfqpoint{0.761362in}{1.401189in}}{\pgfqpoint{0.769598in}{1.401189in}}%
\pgfpathclose%
\pgfusepath{stroke,fill}%
\end{pgfscope}%
\begin{pgfscope}%
\pgfpathrectangle{\pgfqpoint{0.100000in}{0.212622in}}{\pgfqpoint{3.696000in}{3.696000in}}%
\pgfusepath{clip}%
\pgfsetbuttcap%
\pgfsetroundjoin%
\definecolor{currentfill}{rgb}{0.121569,0.466667,0.705882}%
\pgfsetfillcolor{currentfill}%
\pgfsetfillopacity{0.639770}%
\pgfsetlinewidth{1.003750pt}%
\definecolor{currentstroke}{rgb}{0.121569,0.466667,0.705882}%
\pgfsetstrokecolor{currentstroke}%
\pgfsetstrokeopacity{0.639770}%
\pgfsetdash{}{0pt}%
\pgfpathmoveto{\pgfqpoint{0.769598in}{1.401189in}}%
\pgfpathcurveto{\pgfqpoint{0.777835in}{1.401189in}}{\pgfqpoint{0.785735in}{1.404461in}}{\pgfqpoint{0.791559in}{1.410285in}}%
\pgfpathcurveto{\pgfqpoint{0.797383in}{1.416109in}}{\pgfqpoint{0.800655in}{1.424009in}}{\pgfqpoint{0.800655in}{1.432246in}}%
\pgfpathcurveto{\pgfqpoint{0.800655in}{1.440482in}}{\pgfqpoint{0.797383in}{1.448382in}}{\pgfqpoint{0.791559in}{1.454206in}}%
\pgfpathcurveto{\pgfqpoint{0.785735in}{1.460030in}}{\pgfqpoint{0.777835in}{1.463302in}}{\pgfqpoint{0.769598in}{1.463302in}}%
\pgfpathcurveto{\pgfqpoint{0.761362in}{1.463302in}}{\pgfqpoint{0.753462in}{1.460030in}}{\pgfqpoint{0.747638in}{1.454206in}}%
\pgfpathcurveto{\pgfqpoint{0.741814in}{1.448382in}}{\pgfqpoint{0.738542in}{1.440482in}}{\pgfqpoint{0.738542in}{1.432246in}}%
\pgfpathcurveto{\pgfqpoint{0.738542in}{1.424009in}}{\pgfqpoint{0.741814in}{1.416109in}}{\pgfqpoint{0.747638in}{1.410285in}}%
\pgfpathcurveto{\pgfqpoint{0.753462in}{1.404461in}}{\pgfqpoint{0.761362in}{1.401189in}}{\pgfqpoint{0.769598in}{1.401189in}}%
\pgfpathclose%
\pgfusepath{stroke,fill}%
\end{pgfscope}%
\begin{pgfscope}%
\pgfpathrectangle{\pgfqpoint{0.100000in}{0.212622in}}{\pgfqpoint{3.696000in}{3.696000in}}%
\pgfusepath{clip}%
\pgfsetbuttcap%
\pgfsetroundjoin%
\definecolor{currentfill}{rgb}{0.121569,0.466667,0.705882}%
\pgfsetfillcolor{currentfill}%
\pgfsetfillopacity{0.639770}%
\pgfsetlinewidth{1.003750pt}%
\definecolor{currentstroke}{rgb}{0.121569,0.466667,0.705882}%
\pgfsetstrokecolor{currentstroke}%
\pgfsetstrokeopacity{0.639770}%
\pgfsetdash{}{0pt}%
\pgfpathmoveto{\pgfqpoint{0.769598in}{1.401189in}}%
\pgfpathcurveto{\pgfqpoint{0.777835in}{1.401189in}}{\pgfqpoint{0.785735in}{1.404461in}}{\pgfqpoint{0.791559in}{1.410285in}}%
\pgfpathcurveto{\pgfqpoint{0.797383in}{1.416109in}}{\pgfqpoint{0.800655in}{1.424009in}}{\pgfqpoint{0.800655in}{1.432246in}}%
\pgfpathcurveto{\pgfqpoint{0.800655in}{1.440482in}}{\pgfqpoint{0.797383in}{1.448382in}}{\pgfqpoint{0.791559in}{1.454206in}}%
\pgfpathcurveto{\pgfqpoint{0.785735in}{1.460030in}}{\pgfqpoint{0.777835in}{1.463302in}}{\pgfqpoint{0.769598in}{1.463302in}}%
\pgfpathcurveto{\pgfqpoint{0.761362in}{1.463302in}}{\pgfqpoint{0.753462in}{1.460030in}}{\pgfqpoint{0.747638in}{1.454206in}}%
\pgfpathcurveto{\pgfqpoint{0.741814in}{1.448382in}}{\pgfqpoint{0.738542in}{1.440482in}}{\pgfqpoint{0.738542in}{1.432246in}}%
\pgfpathcurveto{\pgfqpoint{0.738542in}{1.424009in}}{\pgfqpoint{0.741814in}{1.416109in}}{\pgfqpoint{0.747638in}{1.410285in}}%
\pgfpathcurveto{\pgfqpoint{0.753462in}{1.404461in}}{\pgfqpoint{0.761362in}{1.401189in}}{\pgfqpoint{0.769598in}{1.401189in}}%
\pgfpathclose%
\pgfusepath{stroke,fill}%
\end{pgfscope}%
\begin{pgfscope}%
\pgfpathrectangle{\pgfqpoint{0.100000in}{0.212622in}}{\pgfqpoint{3.696000in}{3.696000in}}%
\pgfusepath{clip}%
\pgfsetbuttcap%
\pgfsetroundjoin%
\definecolor{currentfill}{rgb}{0.121569,0.466667,0.705882}%
\pgfsetfillcolor{currentfill}%
\pgfsetfillopacity{0.639770}%
\pgfsetlinewidth{1.003750pt}%
\definecolor{currentstroke}{rgb}{0.121569,0.466667,0.705882}%
\pgfsetstrokecolor{currentstroke}%
\pgfsetstrokeopacity{0.639770}%
\pgfsetdash{}{0pt}%
\pgfpathmoveto{\pgfqpoint{0.769598in}{1.401189in}}%
\pgfpathcurveto{\pgfqpoint{0.777835in}{1.401189in}}{\pgfqpoint{0.785735in}{1.404461in}}{\pgfqpoint{0.791559in}{1.410285in}}%
\pgfpathcurveto{\pgfqpoint{0.797383in}{1.416109in}}{\pgfqpoint{0.800655in}{1.424009in}}{\pgfqpoint{0.800655in}{1.432246in}}%
\pgfpathcurveto{\pgfqpoint{0.800655in}{1.440482in}}{\pgfqpoint{0.797383in}{1.448382in}}{\pgfqpoint{0.791559in}{1.454206in}}%
\pgfpathcurveto{\pgfqpoint{0.785735in}{1.460030in}}{\pgfqpoint{0.777835in}{1.463302in}}{\pgfqpoint{0.769598in}{1.463302in}}%
\pgfpathcurveto{\pgfqpoint{0.761362in}{1.463302in}}{\pgfqpoint{0.753462in}{1.460030in}}{\pgfqpoint{0.747638in}{1.454206in}}%
\pgfpathcurveto{\pgfqpoint{0.741814in}{1.448382in}}{\pgfqpoint{0.738542in}{1.440482in}}{\pgfqpoint{0.738542in}{1.432246in}}%
\pgfpathcurveto{\pgfqpoint{0.738542in}{1.424009in}}{\pgfqpoint{0.741814in}{1.416109in}}{\pgfqpoint{0.747638in}{1.410285in}}%
\pgfpathcurveto{\pgfqpoint{0.753462in}{1.404461in}}{\pgfqpoint{0.761362in}{1.401189in}}{\pgfqpoint{0.769598in}{1.401189in}}%
\pgfpathclose%
\pgfusepath{stroke,fill}%
\end{pgfscope}%
\begin{pgfscope}%
\pgfpathrectangle{\pgfqpoint{0.100000in}{0.212622in}}{\pgfqpoint{3.696000in}{3.696000in}}%
\pgfusepath{clip}%
\pgfsetbuttcap%
\pgfsetroundjoin%
\definecolor{currentfill}{rgb}{0.121569,0.466667,0.705882}%
\pgfsetfillcolor{currentfill}%
\pgfsetfillopacity{0.639770}%
\pgfsetlinewidth{1.003750pt}%
\definecolor{currentstroke}{rgb}{0.121569,0.466667,0.705882}%
\pgfsetstrokecolor{currentstroke}%
\pgfsetstrokeopacity{0.639770}%
\pgfsetdash{}{0pt}%
\pgfpathmoveto{\pgfqpoint{0.769598in}{1.401189in}}%
\pgfpathcurveto{\pgfqpoint{0.777835in}{1.401189in}}{\pgfqpoint{0.785735in}{1.404461in}}{\pgfqpoint{0.791559in}{1.410285in}}%
\pgfpathcurveto{\pgfqpoint{0.797383in}{1.416109in}}{\pgfqpoint{0.800655in}{1.424009in}}{\pgfqpoint{0.800655in}{1.432246in}}%
\pgfpathcurveto{\pgfqpoint{0.800655in}{1.440482in}}{\pgfqpoint{0.797383in}{1.448382in}}{\pgfqpoint{0.791559in}{1.454206in}}%
\pgfpathcurveto{\pgfqpoint{0.785735in}{1.460030in}}{\pgfqpoint{0.777835in}{1.463302in}}{\pgfqpoint{0.769598in}{1.463302in}}%
\pgfpathcurveto{\pgfqpoint{0.761362in}{1.463302in}}{\pgfqpoint{0.753462in}{1.460030in}}{\pgfqpoint{0.747638in}{1.454206in}}%
\pgfpathcurveto{\pgfqpoint{0.741814in}{1.448382in}}{\pgfqpoint{0.738542in}{1.440482in}}{\pgfqpoint{0.738542in}{1.432246in}}%
\pgfpathcurveto{\pgfqpoint{0.738542in}{1.424009in}}{\pgfqpoint{0.741814in}{1.416109in}}{\pgfqpoint{0.747638in}{1.410285in}}%
\pgfpathcurveto{\pgfqpoint{0.753462in}{1.404461in}}{\pgfqpoint{0.761362in}{1.401189in}}{\pgfqpoint{0.769598in}{1.401189in}}%
\pgfpathclose%
\pgfusepath{stroke,fill}%
\end{pgfscope}%
\begin{pgfscope}%
\pgfpathrectangle{\pgfqpoint{0.100000in}{0.212622in}}{\pgfqpoint{3.696000in}{3.696000in}}%
\pgfusepath{clip}%
\pgfsetbuttcap%
\pgfsetroundjoin%
\definecolor{currentfill}{rgb}{0.121569,0.466667,0.705882}%
\pgfsetfillcolor{currentfill}%
\pgfsetfillopacity{0.639770}%
\pgfsetlinewidth{1.003750pt}%
\definecolor{currentstroke}{rgb}{0.121569,0.466667,0.705882}%
\pgfsetstrokecolor{currentstroke}%
\pgfsetstrokeopacity{0.639770}%
\pgfsetdash{}{0pt}%
\pgfpathmoveto{\pgfqpoint{0.769598in}{1.401189in}}%
\pgfpathcurveto{\pgfqpoint{0.777835in}{1.401189in}}{\pgfqpoint{0.785735in}{1.404461in}}{\pgfqpoint{0.791559in}{1.410285in}}%
\pgfpathcurveto{\pgfqpoint{0.797383in}{1.416109in}}{\pgfqpoint{0.800655in}{1.424009in}}{\pgfqpoint{0.800655in}{1.432246in}}%
\pgfpathcurveto{\pgfqpoint{0.800655in}{1.440482in}}{\pgfqpoint{0.797383in}{1.448382in}}{\pgfqpoint{0.791559in}{1.454206in}}%
\pgfpathcurveto{\pgfqpoint{0.785735in}{1.460030in}}{\pgfqpoint{0.777835in}{1.463302in}}{\pgfqpoint{0.769598in}{1.463302in}}%
\pgfpathcurveto{\pgfqpoint{0.761362in}{1.463302in}}{\pgfqpoint{0.753462in}{1.460030in}}{\pgfqpoint{0.747638in}{1.454206in}}%
\pgfpathcurveto{\pgfqpoint{0.741814in}{1.448382in}}{\pgfqpoint{0.738542in}{1.440482in}}{\pgfqpoint{0.738542in}{1.432246in}}%
\pgfpathcurveto{\pgfqpoint{0.738542in}{1.424009in}}{\pgfqpoint{0.741814in}{1.416109in}}{\pgfqpoint{0.747638in}{1.410285in}}%
\pgfpathcurveto{\pgfqpoint{0.753462in}{1.404461in}}{\pgfqpoint{0.761362in}{1.401189in}}{\pgfqpoint{0.769598in}{1.401189in}}%
\pgfpathclose%
\pgfusepath{stroke,fill}%
\end{pgfscope}%
\begin{pgfscope}%
\pgfpathrectangle{\pgfqpoint{0.100000in}{0.212622in}}{\pgfqpoint{3.696000in}{3.696000in}}%
\pgfusepath{clip}%
\pgfsetbuttcap%
\pgfsetroundjoin%
\definecolor{currentfill}{rgb}{0.121569,0.466667,0.705882}%
\pgfsetfillcolor{currentfill}%
\pgfsetfillopacity{0.639770}%
\pgfsetlinewidth{1.003750pt}%
\definecolor{currentstroke}{rgb}{0.121569,0.466667,0.705882}%
\pgfsetstrokecolor{currentstroke}%
\pgfsetstrokeopacity{0.639770}%
\pgfsetdash{}{0pt}%
\pgfpathmoveto{\pgfqpoint{0.769598in}{1.401189in}}%
\pgfpathcurveto{\pgfqpoint{0.777835in}{1.401189in}}{\pgfqpoint{0.785735in}{1.404461in}}{\pgfqpoint{0.791559in}{1.410285in}}%
\pgfpathcurveto{\pgfqpoint{0.797383in}{1.416109in}}{\pgfqpoint{0.800655in}{1.424009in}}{\pgfqpoint{0.800655in}{1.432246in}}%
\pgfpathcurveto{\pgfqpoint{0.800655in}{1.440482in}}{\pgfqpoint{0.797383in}{1.448382in}}{\pgfqpoint{0.791559in}{1.454206in}}%
\pgfpathcurveto{\pgfqpoint{0.785735in}{1.460030in}}{\pgfqpoint{0.777835in}{1.463302in}}{\pgfqpoint{0.769598in}{1.463302in}}%
\pgfpathcurveto{\pgfqpoint{0.761362in}{1.463302in}}{\pgfqpoint{0.753462in}{1.460030in}}{\pgfqpoint{0.747638in}{1.454206in}}%
\pgfpathcurveto{\pgfqpoint{0.741814in}{1.448382in}}{\pgfqpoint{0.738542in}{1.440482in}}{\pgfqpoint{0.738542in}{1.432246in}}%
\pgfpathcurveto{\pgfqpoint{0.738542in}{1.424009in}}{\pgfqpoint{0.741814in}{1.416109in}}{\pgfqpoint{0.747638in}{1.410285in}}%
\pgfpathcurveto{\pgfqpoint{0.753462in}{1.404461in}}{\pgfqpoint{0.761362in}{1.401189in}}{\pgfqpoint{0.769598in}{1.401189in}}%
\pgfpathclose%
\pgfusepath{stroke,fill}%
\end{pgfscope}%
\begin{pgfscope}%
\pgfpathrectangle{\pgfqpoint{0.100000in}{0.212622in}}{\pgfqpoint{3.696000in}{3.696000in}}%
\pgfusepath{clip}%
\pgfsetbuttcap%
\pgfsetroundjoin%
\definecolor{currentfill}{rgb}{0.121569,0.466667,0.705882}%
\pgfsetfillcolor{currentfill}%
\pgfsetfillopacity{0.639770}%
\pgfsetlinewidth{1.003750pt}%
\definecolor{currentstroke}{rgb}{0.121569,0.466667,0.705882}%
\pgfsetstrokecolor{currentstroke}%
\pgfsetstrokeopacity{0.639770}%
\pgfsetdash{}{0pt}%
\pgfpathmoveto{\pgfqpoint{0.769598in}{1.401189in}}%
\pgfpathcurveto{\pgfqpoint{0.777835in}{1.401189in}}{\pgfqpoint{0.785735in}{1.404461in}}{\pgfqpoint{0.791559in}{1.410285in}}%
\pgfpathcurveto{\pgfqpoint{0.797383in}{1.416109in}}{\pgfqpoint{0.800655in}{1.424009in}}{\pgfqpoint{0.800655in}{1.432246in}}%
\pgfpathcurveto{\pgfqpoint{0.800655in}{1.440482in}}{\pgfqpoint{0.797383in}{1.448382in}}{\pgfqpoint{0.791559in}{1.454206in}}%
\pgfpathcurveto{\pgfqpoint{0.785735in}{1.460030in}}{\pgfqpoint{0.777835in}{1.463302in}}{\pgfqpoint{0.769598in}{1.463302in}}%
\pgfpathcurveto{\pgfqpoint{0.761362in}{1.463302in}}{\pgfqpoint{0.753462in}{1.460030in}}{\pgfqpoint{0.747638in}{1.454206in}}%
\pgfpathcurveto{\pgfqpoint{0.741814in}{1.448382in}}{\pgfqpoint{0.738542in}{1.440482in}}{\pgfqpoint{0.738542in}{1.432246in}}%
\pgfpathcurveto{\pgfqpoint{0.738542in}{1.424009in}}{\pgfqpoint{0.741814in}{1.416109in}}{\pgfqpoint{0.747638in}{1.410285in}}%
\pgfpathcurveto{\pgfqpoint{0.753462in}{1.404461in}}{\pgfqpoint{0.761362in}{1.401189in}}{\pgfqpoint{0.769598in}{1.401189in}}%
\pgfpathclose%
\pgfusepath{stroke,fill}%
\end{pgfscope}%
\begin{pgfscope}%
\pgfpathrectangle{\pgfqpoint{0.100000in}{0.212622in}}{\pgfqpoint{3.696000in}{3.696000in}}%
\pgfusepath{clip}%
\pgfsetbuttcap%
\pgfsetroundjoin%
\definecolor{currentfill}{rgb}{0.121569,0.466667,0.705882}%
\pgfsetfillcolor{currentfill}%
\pgfsetfillopacity{0.639770}%
\pgfsetlinewidth{1.003750pt}%
\definecolor{currentstroke}{rgb}{0.121569,0.466667,0.705882}%
\pgfsetstrokecolor{currentstroke}%
\pgfsetstrokeopacity{0.639770}%
\pgfsetdash{}{0pt}%
\pgfpathmoveto{\pgfqpoint{0.769598in}{1.401189in}}%
\pgfpathcurveto{\pgfqpoint{0.777835in}{1.401189in}}{\pgfqpoint{0.785735in}{1.404461in}}{\pgfqpoint{0.791559in}{1.410285in}}%
\pgfpathcurveto{\pgfqpoint{0.797383in}{1.416109in}}{\pgfqpoint{0.800655in}{1.424009in}}{\pgfqpoint{0.800655in}{1.432246in}}%
\pgfpathcurveto{\pgfqpoint{0.800655in}{1.440482in}}{\pgfqpoint{0.797383in}{1.448382in}}{\pgfqpoint{0.791559in}{1.454206in}}%
\pgfpathcurveto{\pgfqpoint{0.785735in}{1.460030in}}{\pgfqpoint{0.777835in}{1.463302in}}{\pgfqpoint{0.769598in}{1.463302in}}%
\pgfpathcurveto{\pgfqpoint{0.761362in}{1.463302in}}{\pgfqpoint{0.753462in}{1.460030in}}{\pgfqpoint{0.747638in}{1.454206in}}%
\pgfpathcurveto{\pgfqpoint{0.741814in}{1.448382in}}{\pgfqpoint{0.738542in}{1.440482in}}{\pgfqpoint{0.738542in}{1.432246in}}%
\pgfpathcurveto{\pgfqpoint{0.738542in}{1.424009in}}{\pgfqpoint{0.741814in}{1.416109in}}{\pgfqpoint{0.747638in}{1.410285in}}%
\pgfpathcurveto{\pgfqpoint{0.753462in}{1.404461in}}{\pgfqpoint{0.761362in}{1.401189in}}{\pgfqpoint{0.769598in}{1.401189in}}%
\pgfpathclose%
\pgfusepath{stroke,fill}%
\end{pgfscope}%
\begin{pgfscope}%
\pgfpathrectangle{\pgfqpoint{0.100000in}{0.212622in}}{\pgfqpoint{3.696000in}{3.696000in}}%
\pgfusepath{clip}%
\pgfsetbuttcap%
\pgfsetroundjoin%
\definecolor{currentfill}{rgb}{0.121569,0.466667,0.705882}%
\pgfsetfillcolor{currentfill}%
\pgfsetfillopacity{0.639770}%
\pgfsetlinewidth{1.003750pt}%
\definecolor{currentstroke}{rgb}{0.121569,0.466667,0.705882}%
\pgfsetstrokecolor{currentstroke}%
\pgfsetstrokeopacity{0.639770}%
\pgfsetdash{}{0pt}%
\pgfpathmoveto{\pgfqpoint{0.769598in}{1.401189in}}%
\pgfpathcurveto{\pgfqpoint{0.777835in}{1.401189in}}{\pgfqpoint{0.785735in}{1.404461in}}{\pgfqpoint{0.791559in}{1.410285in}}%
\pgfpathcurveto{\pgfqpoint{0.797383in}{1.416109in}}{\pgfqpoint{0.800655in}{1.424009in}}{\pgfqpoint{0.800655in}{1.432246in}}%
\pgfpathcurveto{\pgfqpoint{0.800655in}{1.440482in}}{\pgfqpoint{0.797383in}{1.448382in}}{\pgfqpoint{0.791559in}{1.454206in}}%
\pgfpathcurveto{\pgfqpoint{0.785735in}{1.460030in}}{\pgfqpoint{0.777835in}{1.463302in}}{\pgfqpoint{0.769598in}{1.463302in}}%
\pgfpathcurveto{\pgfqpoint{0.761362in}{1.463302in}}{\pgfqpoint{0.753462in}{1.460030in}}{\pgfqpoint{0.747638in}{1.454206in}}%
\pgfpathcurveto{\pgfqpoint{0.741814in}{1.448382in}}{\pgfqpoint{0.738542in}{1.440482in}}{\pgfqpoint{0.738542in}{1.432246in}}%
\pgfpathcurveto{\pgfqpoint{0.738542in}{1.424009in}}{\pgfqpoint{0.741814in}{1.416109in}}{\pgfqpoint{0.747638in}{1.410285in}}%
\pgfpathcurveto{\pgfqpoint{0.753462in}{1.404461in}}{\pgfqpoint{0.761362in}{1.401189in}}{\pgfqpoint{0.769598in}{1.401189in}}%
\pgfpathclose%
\pgfusepath{stroke,fill}%
\end{pgfscope}%
\begin{pgfscope}%
\pgfpathrectangle{\pgfqpoint{0.100000in}{0.212622in}}{\pgfqpoint{3.696000in}{3.696000in}}%
\pgfusepath{clip}%
\pgfsetbuttcap%
\pgfsetroundjoin%
\definecolor{currentfill}{rgb}{0.121569,0.466667,0.705882}%
\pgfsetfillcolor{currentfill}%
\pgfsetfillopacity{0.639770}%
\pgfsetlinewidth{1.003750pt}%
\definecolor{currentstroke}{rgb}{0.121569,0.466667,0.705882}%
\pgfsetstrokecolor{currentstroke}%
\pgfsetstrokeopacity{0.639770}%
\pgfsetdash{}{0pt}%
\pgfpathmoveto{\pgfqpoint{0.769598in}{1.401189in}}%
\pgfpathcurveto{\pgfqpoint{0.777835in}{1.401189in}}{\pgfqpoint{0.785735in}{1.404461in}}{\pgfqpoint{0.791559in}{1.410285in}}%
\pgfpathcurveto{\pgfqpoint{0.797383in}{1.416109in}}{\pgfqpoint{0.800655in}{1.424009in}}{\pgfqpoint{0.800655in}{1.432246in}}%
\pgfpathcurveto{\pgfqpoint{0.800655in}{1.440482in}}{\pgfqpoint{0.797383in}{1.448382in}}{\pgfqpoint{0.791559in}{1.454206in}}%
\pgfpathcurveto{\pgfqpoint{0.785735in}{1.460030in}}{\pgfqpoint{0.777835in}{1.463302in}}{\pgfqpoint{0.769598in}{1.463302in}}%
\pgfpathcurveto{\pgfqpoint{0.761362in}{1.463302in}}{\pgfqpoint{0.753462in}{1.460030in}}{\pgfqpoint{0.747638in}{1.454206in}}%
\pgfpathcurveto{\pgfqpoint{0.741814in}{1.448382in}}{\pgfqpoint{0.738542in}{1.440482in}}{\pgfqpoint{0.738542in}{1.432246in}}%
\pgfpathcurveto{\pgfqpoint{0.738542in}{1.424009in}}{\pgfqpoint{0.741814in}{1.416109in}}{\pgfqpoint{0.747638in}{1.410285in}}%
\pgfpathcurveto{\pgfqpoint{0.753462in}{1.404461in}}{\pgfqpoint{0.761362in}{1.401189in}}{\pgfqpoint{0.769598in}{1.401189in}}%
\pgfpathclose%
\pgfusepath{stroke,fill}%
\end{pgfscope}%
\begin{pgfscope}%
\pgfpathrectangle{\pgfqpoint{0.100000in}{0.212622in}}{\pgfqpoint{3.696000in}{3.696000in}}%
\pgfusepath{clip}%
\pgfsetbuttcap%
\pgfsetroundjoin%
\definecolor{currentfill}{rgb}{0.121569,0.466667,0.705882}%
\pgfsetfillcolor{currentfill}%
\pgfsetfillopacity{0.639770}%
\pgfsetlinewidth{1.003750pt}%
\definecolor{currentstroke}{rgb}{0.121569,0.466667,0.705882}%
\pgfsetstrokecolor{currentstroke}%
\pgfsetstrokeopacity{0.639770}%
\pgfsetdash{}{0pt}%
\pgfpathmoveto{\pgfqpoint{0.769598in}{1.401189in}}%
\pgfpathcurveto{\pgfqpoint{0.777835in}{1.401189in}}{\pgfqpoint{0.785735in}{1.404461in}}{\pgfqpoint{0.791559in}{1.410285in}}%
\pgfpathcurveto{\pgfqpoint{0.797383in}{1.416109in}}{\pgfqpoint{0.800655in}{1.424009in}}{\pgfqpoint{0.800655in}{1.432246in}}%
\pgfpathcurveto{\pgfqpoint{0.800655in}{1.440482in}}{\pgfqpoint{0.797383in}{1.448382in}}{\pgfqpoint{0.791559in}{1.454206in}}%
\pgfpathcurveto{\pgfqpoint{0.785735in}{1.460030in}}{\pgfqpoint{0.777835in}{1.463302in}}{\pgfqpoint{0.769598in}{1.463302in}}%
\pgfpathcurveto{\pgfqpoint{0.761362in}{1.463302in}}{\pgfqpoint{0.753462in}{1.460030in}}{\pgfqpoint{0.747638in}{1.454206in}}%
\pgfpathcurveto{\pgfqpoint{0.741814in}{1.448382in}}{\pgfqpoint{0.738542in}{1.440482in}}{\pgfqpoint{0.738542in}{1.432246in}}%
\pgfpathcurveto{\pgfqpoint{0.738542in}{1.424009in}}{\pgfqpoint{0.741814in}{1.416109in}}{\pgfqpoint{0.747638in}{1.410285in}}%
\pgfpathcurveto{\pgfqpoint{0.753462in}{1.404461in}}{\pgfqpoint{0.761362in}{1.401189in}}{\pgfqpoint{0.769598in}{1.401189in}}%
\pgfpathclose%
\pgfusepath{stroke,fill}%
\end{pgfscope}%
\begin{pgfscope}%
\pgfpathrectangle{\pgfqpoint{0.100000in}{0.212622in}}{\pgfqpoint{3.696000in}{3.696000in}}%
\pgfusepath{clip}%
\pgfsetbuttcap%
\pgfsetroundjoin%
\definecolor{currentfill}{rgb}{0.121569,0.466667,0.705882}%
\pgfsetfillcolor{currentfill}%
\pgfsetfillopacity{0.639874}%
\pgfsetlinewidth{1.003750pt}%
\definecolor{currentstroke}{rgb}{0.121569,0.466667,0.705882}%
\pgfsetstrokecolor{currentstroke}%
\pgfsetstrokeopacity{0.639874}%
\pgfsetdash{}{0pt}%
\pgfpathmoveto{\pgfqpoint{0.769235in}{1.401101in}}%
\pgfpathcurveto{\pgfqpoint{0.777471in}{1.401101in}}{\pgfqpoint{0.785371in}{1.404373in}}{\pgfqpoint{0.791195in}{1.410197in}}%
\pgfpathcurveto{\pgfqpoint{0.797019in}{1.416021in}}{\pgfqpoint{0.800291in}{1.423921in}}{\pgfqpoint{0.800291in}{1.432157in}}%
\pgfpathcurveto{\pgfqpoint{0.800291in}{1.440394in}}{\pgfqpoint{0.797019in}{1.448294in}}{\pgfqpoint{0.791195in}{1.454118in}}%
\pgfpathcurveto{\pgfqpoint{0.785371in}{1.459942in}}{\pgfqpoint{0.777471in}{1.463214in}}{\pgfqpoint{0.769235in}{1.463214in}}%
\pgfpathcurveto{\pgfqpoint{0.760999in}{1.463214in}}{\pgfqpoint{0.753099in}{1.459942in}}{\pgfqpoint{0.747275in}{1.454118in}}%
\pgfpathcurveto{\pgfqpoint{0.741451in}{1.448294in}}{\pgfqpoint{0.738178in}{1.440394in}}{\pgfqpoint{0.738178in}{1.432157in}}%
\pgfpathcurveto{\pgfqpoint{0.738178in}{1.423921in}}{\pgfqpoint{0.741451in}{1.416021in}}{\pgfqpoint{0.747275in}{1.410197in}}%
\pgfpathcurveto{\pgfqpoint{0.753099in}{1.404373in}}{\pgfqpoint{0.760999in}{1.401101in}}{\pgfqpoint{0.769235in}{1.401101in}}%
\pgfpathclose%
\pgfusepath{stroke,fill}%
\end{pgfscope}%
\begin{pgfscope}%
\pgfpathrectangle{\pgfqpoint{0.100000in}{0.212622in}}{\pgfqpoint{3.696000in}{3.696000in}}%
\pgfusepath{clip}%
\pgfsetbuttcap%
\pgfsetroundjoin%
\definecolor{currentfill}{rgb}{0.121569,0.466667,0.705882}%
\pgfsetfillcolor{currentfill}%
\pgfsetfillopacity{0.639942}%
\pgfsetlinewidth{1.003750pt}%
\definecolor{currentstroke}{rgb}{0.121569,0.466667,0.705882}%
\pgfsetstrokecolor{currentstroke}%
\pgfsetstrokeopacity{0.639942}%
\pgfsetdash{}{0pt}%
\pgfpathmoveto{\pgfqpoint{0.763742in}{1.401490in}}%
\pgfpathcurveto{\pgfqpoint{0.771979in}{1.401490in}}{\pgfqpoint{0.779879in}{1.404762in}}{\pgfqpoint{0.785703in}{1.410586in}}%
\pgfpathcurveto{\pgfqpoint{0.791527in}{1.416410in}}{\pgfqpoint{0.794799in}{1.424310in}}{\pgfqpoint{0.794799in}{1.432546in}}%
\pgfpathcurveto{\pgfqpoint{0.794799in}{1.440783in}}{\pgfqpoint{0.791527in}{1.448683in}}{\pgfqpoint{0.785703in}{1.454507in}}%
\pgfpathcurveto{\pgfqpoint{0.779879in}{1.460331in}}{\pgfqpoint{0.771979in}{1.463603in}}{\pgfqpoint{0.763742in}{1.463603in}}%
\pgfpathcurveto{\pgfqpoint{0.755506in}{1.463603in}}{\pgfqpoint{0.747606in}{1.460331in}}{\pgfqpoint{0.741782in}{1.454507in}}%
\pgfpathcurveto{\pgfqpoint{0.735958in}{1.448683in}}{\pgfqpoint{0.732686in}{1.440783in}}{\pgfqpoint{0.732686in}{1.432546in}}%
\pgfpathcurveto{\pgfqpoint{0.732686in}{1.424310in}}{\pgfqpoint{0.735958in}{1.416410in}}{\pgfqpoint{0.741782in}{1.410586in}}%
\pgfpathcurveto{\pgfqpoint{0.747606in}{1.404762in}}{\pgfqpoint{0.755506in}{1.401490in}}{\pgfqpoint{0.763742in}{1.401490in}}%
\pgfpathclose%
\pgfusepath{stroke,fill}%
\end{pgfscope}%
\begin{pgfscope}%
\pgfpathrectangle{\pgfqpoint{0.100000in}{0.212622in}}{\pgfqpoint{3.696000in}{3.696000in}}%
\pgfusepath{clip}%
\pgfsetbuttcap%
\pgfsetroundjoin%
\definecolor{currentfill}{rgb}{0.121569,0.466667,0.705882}%
\pgfsetfillcolor{currentfill}%
\pgfsetfillopacity{0.640041}%
\pgfsetlinewidth{1.003750pt}%
\definecolor{currentstroke}{rgb}{0.121569,0.466667,0.705882}%
\pgfsetstrokecolor{currentstroke}%
\pgfsetstrokeopacity{0.640041}%
\pgfsetdash{}{0pt}%
\pgfpathmoveto{\pgfqpoint{0.768309in}{1.401008in}}%
\pgfpathcurveto{\pgfqpoint{0.776545in}{1.401008in}}{\pgfqpoint{0.784445in}{1.404280in}}{\pgfqpoint{0.790269in}{1.410104in}}%
\pgfpathcurveto{\pgfqpoint{0.796093in}{1.415928in}}{\pgfqpoint{0.799365in}{1.423828in}}{\pgfqpoint{0.799365in}{1.432064in}}%
\pgfpathcurveto{\pgfqpoint{0.799365in}{1.440300in}}{\pgfqpoint{0.796093in}{1.448200in}}{\pgfqpoint{0.790269in}{1.454024in}}%
\pgfpathcurveto{\pgfqpoint{0.784445in}{1.459848in}}{\pgfqpoint{0.776545in}{1.463121in}}{\pgfqpoint{0.768309in}{1.463121in}}%
\pgfpathcurveto{\pgfqpoint{0.760072in}{1.463121in}}{\pgfqpoint{0.752172in}{1.459848in}}{\pgfqpoint{0.746348in}{1.454024in}}%
\pgfpathcurveto{\pgfqpoint{0.740524in}{1.448200in}}{\pgfqpoint{0.737252in}{1.440300in}}{\pgfqpoint{0.737252in}{1.432064in}}%
\pgfpathcurveto{\pgfqpoint{0.737252in}{1.423828in}}{\pgfqpoint{0.740524in}{1.415928in}}{\pgfqpoint{0.746348in}{1.410104in}}%
\pgfpathcurveto{\pgfqpoint{0.752172in}{1.404280in}}{\pgfqpoint{0.760072in}{1.401008in}}{\pgfqpoint{0.768309in}{1.401008in}}%
\pgfpathclose%
\pgfusepath{stroke,fill}%
\end{pgfscope}%
\begin{pgfscope}%
\pgfpathrectangle{\pgfqpoint{0.100000in}{0.212622in}}{\pgfqpoint{3.696000in}{3.696000in}}%
\pgfusepath{clip}%
\pgfsetbuttcap%
\pgfsetroundjoin%
\definecolor{currentfill}{rgb}{0.121569,0.466667,0.705882}%
\pgfsetfillcolor{currentfill}%
\pgfsetfillopacity{0.640184}%
\pgfsetlinewidth{1.003750pt}%
\definecolor{currentstroke}{rgb}{0.121569,0.466667,0.705882}%
\pgfsetstrokecolor{currentstroke}%
\pgfsetstrokeopacity{0.640184}%
\pgfsetdash{}{0pt}%
\pgfpathmoveto{\pgfqpoint{0.766562in}{1.401050in}}%
\pgfpathcurveto{\pgfqpoint{0.774798in}{1.401050in}}{\pgfqpoint{0.782698in}{1.404322in}}{\pgfqpoint{0.788522in}{1.410146in}}%
\pgfpathcurveto{\pgfqpoint{0.794346in}{1.415970in}}{\pgfqpoint{0.797618in}{1.423870in}}{\pgfqpoint{0.797618in}{1.432106in}}%
\pgfpathcurveto{\pgfqpoint{0.797618in}{1.440342in}}{\pgfqpoint{0.794346in}{1.448242in}}{\pgfqpoint{0.788522in}{1.454066in}}%
\pgfpathcurveto{\pgfqpoint{0.782698in}{1.459890in}}{\pgfqpoint{0.774798in}{1.463163in}}{\pgfqpoint{0.766562in}{1.463163in}}%
\pgfpathcurveto{\pgfqpoint{0.758325in}{1.463163in}}{\pgfqpoint{0.750425in}{1.459890in}}{\pgfqpoint{0.744601in}{1.454066in}}%
\pgfpathcurveto{\pgfqpoint{0.738778in}{1.448242in}}{\pgfqpoint{0.735505in}{1.440342in}}{\pgfqpoint{0.735505in}{1.432106in}}%
\pgfpathcurveto{\pgfqpoint{0.735505in}{1.423870in}}{\pgfqpoint{0.738778in}{1.415970in}}{\pgfqpoint{0.744601in}{1.410146in}}%
\pgfpathcurveto{\pgfqpoint{0.750425in}{1.404322in}}{\pgfqpoint{0.758325in}{1.401050in}}{\pgfqpoint{0.766562in}{1.401050in}}%
\pgfpathclose%
\pgfusepath{stroke,fill}%
\end{pgfscope}%
\begin{pgfscope}%
\pgfpathrectangle{\pgfqpoint{0.100000in}{0.212622in}}{\pgfqpoint{3.696000in}{3.696000in}}%
\pgfusepath{clip}%
\pgfsetbuttcap%
\pgfsetroundjoin%
\definecolor{currentfill}{rgb}{0.121569,0.466667,0.705882}%
\pgfsetfillcolor{currentfill}%
\pgfsetfillopacity{0.641189}%
\pgfsetlinewidth{1.003750pt}%
\definecolor{currentstroke}{rgb}{0.121569,0.466667,0.705882}%
\pgfsetstrokecolor{currentstroke}%
\pgfsetstrokeopacity{0.641189}%
\pgfsetdash{}{0pt}%
\pgfpathmoveto{\pgfqpoint{1.616770in}{1.689328in}}%
\pgfpathcurveto{\pgfqpoint{1.625006in}{1.689328in}}{\pgfqpoint{1.632906in}{1.692601in}}{\pgfqpoint{1.638730in}{1.698425in}}%
\pgfpathcurveto{\pgfqpoint{1.644554in}{1.704248in}}{\pgfqpoint{1.647826in}{1.712148in}}{\pgfqpoint{1.647826in}{1.720385in}}%
\pgfpathcurveto{\pgfqpoint{1.647826in}{1.728621in}}{\pgfqpoint{1.644554in}{1.736521in}}{\pgfqpoint{1.638730in}{1.742345in}}%
\pgfpathcurveto{\pgfqpoint{1.632906in}{1.748169in}}{\pgfqpoint{1.625006in}{1.751441in}}{\pgfqpoint{1.616770in}{1.751441in}}%
\pgfpathcurveto{\pgfqpoint{1.608533in}{1.751441in}}{\pgfqpoint{1.600633in}{1.748169in}}{\pgfqpoint{1.594809in}{1.742345in}}%
\pgfpathcurveto{\pgfqpoint{1.588985in}{1.736521in}}{\pgfqpoint{1.585713in}{1.728621in}}{\pgfqpoint{1.585713in}{1.720385in}}%
\pgfpathcurveto{\pgfqpoint{1.585713in}{1.712148in}}{\pgfqpoint{1.588985in}{1.704248in}}{\pgfqpoint{1.594809in}{1.698425in}}%
\pgfpathcurveto{\pgfqpoint{1.600633in}{1.692601in}}{\pgfqpoint{1.608533in}{1.689328in}}{\pgfqpoint{1.616770in}{1.689328in}}%
\pgfpathclose%
\pgfusepath{stroke,fill}%
\end{pgfscope}%
\begin{pgfscope}%
\pgfpathrectangle{\pgfqpoint{0.100000in}{0.212622in}}{\pgfqpoint{3.696000in}{3.696000in}}%
\pgfusepath{clip}%
\pgfsetbuttcap%
\pgfsetroundjoin%
\definecolor{currentfill}{rgb}{0.121569,0.466667,0.705882}%
\pgfsetfillcolor{currentfill}%
\pgfsetfillopacity{0.643410}%
\pgfsetlinewidth{1.003750pt}%
\definecolor{currentstroke}{rgb}{0.121569,0.466667,0.705882}%
\pgfsetstrokecolor{currentstroke}%
\pgfsetstrokeopacity{0.643410}%
\pgfsetdash{}{0pt}%
\pgfpathmoveto{\pgfqpoint{1.617596in}{1.687712in}}%
\pgfpathcurveto{\pgfqpoint{1.625832in}{1.687712in}}{\pgfqpoint{1.633732in}{1.690985in}}{\pgfqpoint{1.639556in}{1.696809in}}%
\pgfpathcurveto{\pgfqpoint{1.645380in}{1.702633in}}{\pgfqpoint{1.648652in}{1.710533in}}{\pgfqpoint{1.648652in}{1.718769in}}%
\pgfpathcurveto{\pgfqpoint{1.648652in}{1.727005in}}{\pgfqpoint{1.645380in}{1.734905in}}{\pgfqpoint{1.639556in}{1.740729in}}%
\pgfpathcurveto{\pgfqpoint{1.633732in}{1.746553in}}{\pgfqpoint{1.625832in}{1.749825in}}{\pgfqpoint{1.617596in}{1.749825in}}%
\pgfpathcurveto{\pgfqpoint{1.609360in}{1.749825in}}{\pgfqpoint{1.601460in}{1.746553in}}{\pgfqpoint{1.595636in}{1.740729in}}%
\pgfpathcurveto{\pgfqpoint{1.589812in}{1.734905in}}{\pgfqpoint{1.586539in}{1.727005in}}{\pgfqpoint{1.586539in}{1.718769in}}%
\pgfpathcurveto{\pgfqpoint{1.586539in}{1.710533in}}{\pgfqpoint{1.589812in}{1.702633in}}{\pgfqpoint{1.595636in}{1.696809in}}%
\pgfpathcurveto{\pgfqpoint{1.601460in}{1.690985in}}{\pgfqpoint{1.609360in}{1.687712in}}{\pgfqpoint{1.617596in}{1.687712in}}%
\pgfpathclose%
\pgfusepath{stroke,fill}%
\end{pgfscope}%
\begin{pgfscope}%
\pgfpathrectangle{\pgfqpoint{0.100000in}{0.212622in}}{\pgfqpoint{3.696000in}{3.696000in}}%
\pgfusepath{clip}%
\pgfsetbuttcap%
\pgfsetroundjoin%
\definecolor{currentfill}{rgb}{0.121569,0.466667,0.705882}%
\pgfsetfillcolor{currentfill}%
\pgfsetfillopacity{0.644584}%
\pgfsetlinewidth{1.003750pt}%
\definecolor{currentstroke}{rgb}{0.121569,0.466667,0.705882}%
\pgfsetstrokecolor{currentstroke}%
\pgfsetstrokeopacity{0.644584}%
\pgfsetdash{}{0pt}%
\pgfpathmoveto{\pgfqpoint{1.618278in}{1.686696in}}%
\pgfpathcurveto{\pgfqpoint{1.626515in}{1.686696in}}{\pgfqpoint{1.634415in}{1.689968in}}{\pgfqpoint{1.640239in}{1.695792in}}%
\pgfpathcurveto{\pgfqpoint{1.646063in}{1.701616in}}{\pgfqpoint{1.649335in}{1.709516in}}{\pgfqpoint{1.649335in}{1.717753in}}%
\pgfpathcurveto{\pgfqpoint{1.649335in}{1.725989in}}{\pgfqpoint{1.646063in}{1.733889in}}{\pgfqpoint{1.640239in}{1.739713in}}%
\pgfpathcurveto{\pgfqpoint{1.634415in}{1.745537in}}{\pgfqpoint{1.626515in}{1.748809in}}{\pgfqpoint{1.618278in}{1.748809in}}%
\pgfpathcurveto{\pgfqpoint{1.610042in}{1.748809in}}{\pgfqpoint{1.602142in}{1.745537in}}{\pgfqpoint{1.596318in}{1.739713in}}%
\pgfpathcurveto{\pgfqpoint{1.590494in}{1.733889in}}{\pgfqpoint{1.587222in}{1.725989in}}{\pgfqpoint{1.587222in}{1.717753in}}%
\pgfpathcurveto{\pgfqpoint{1.587222in}{1.709516in}}{\pgfqpoint{1.590494in}{1.701616in}}{\pgfqpoint{1.596318in}{1.695792in}}%
\pgfpathcurveto{\pgfqpoint{1.602142in}{1.689968in}}{\pgfqpoint{1.610042in}{1.686696in}}{\pgfqpoint{1.618278in}{1.686696in}}%
\pgfpathclose%
\pgfusepath{stroke,fill}%
\end{pgfscope}%
\begin{pgfscope}%
\pgfpathrectangle{\pgfqpoint{0.100000in}{0.212622in}}{\pgfqpoint{3.696000in}{3.696000in}}%
\pgfusepath{clip}%
\pgfsetbuttcap%
\pgfsetroundjoin%
\definecolor{currentfill}{rgb}{0.121569,0.466667,0.705882}%
\pgfsetfillcolor{currentfill}%
\pgfsetfillopacity{0.646137}%
\pgfsetlinewidth{1.003750pt}%
\definecolor{currentstroke}{rgb}{0.121569,0.466667,0.705882}%
\pgfsetstrokecolor{currentstroke}%
\pgfsetstrokeopacity{0.646137}%
\pgfsetdash{}{0pt}%
\pgfpathmoveto{\pgfqpoint{1.619009in}{1.686351in}}%
\pgfpathcurveto{\pgfqpoint{1.627245in}{1.686351in}}{\pgfqpoint{1.635145in}{1.689623in}}{\pgfqpoint{1.640969in}{1.695447in}}%
\pgfpathcurveto{\pgfqpoint{1.646793in}{1.701271in}}{\pgfqpoint{1.650066in}{1.709171in}}{\pgfqpoint{1.650066in}{1.717408in}}%
\pgfpathcurveto{\pgfqpoint{1.650066in}{1.725644in}}{\pgfqpoint{1.646793in}{1.733544in}}{\pgfqpoint{1.640969in}{1.739368in}}%
\pgfpathcurveto{\pgfqpoint{1.635145in}{1.745192in}}{\pgfqpoint{1.627245in}{1.748464in}}{\pgfqpoint{1.619009in}{1.748464in}}%
\pgfpathcurveto{\pgfqpoint{1.610773in}{1.748464in}}{\pgfqpoint{1.602873in}{1.745192in}}{\pgfqpoint{1.597049in}{1.739368in}}%
\pgfpathcurveto{\pgfqpoint{1.591225in}{1.733544in}}{\pgfqpoint{1.587953in}{1.725644in}}{\pgfqpoint{1.587953in}{1.717408in}}%
\pgfpathcurveto{\pgfqpoint{1.587953in}{1.709171in}}{\pgfqpoint{1.591225in}{1.701271in}}{\pgfqpoint{1.597049in}{1.695447in}}%
\pgfpathcurveto{\pgfqpoint{1.602873in}{1.689623in}}{\pgfqpoint{1.610773in}{1.686351in}}{\pgfqpoint{1.619009in}{1.686351in}}%
\pgfpathclose%
\pgfusepath{stroke,fill}%
\end{pgfscope}%
\begin{pgfscope}%
\pgfpathrectangle{\pgfqpoint{0.100000in}{0.212622in}}{\pgfqpoint{3.696000in}{3.696000in}}%
\pgfusepath{clip}%
\pgfsetbuttcap%
\pgfsetroundjoin%
\definecolor{currentfill}{rgb}{0.121569,0.466667,0.705882}%
\pgfsetfillcolor{currentfill}%
\pgfsetfillopacity{0.648165}%
\pgfsetlinewidth{1.003750pt}%
\definecolor{currentstroke}{rgb}{0.121569,0.466667,0.705882}%
\pgfsetstrokecolor{currentstroke}%
\pgfsetstrokeopacity{0.648165}%
\pgfsetdash{}{0pt}%
\pgfpathmoveto{\pgfqpoint{1.620046in}{1.686719in}}%
\pgfpathcurveto{\pgfqpoint{1.628283in}{1.686719in}}{\pgfqpoint{1.636183in}{1.689991in}}{\pgfqpoint{1.642007in}{1.695815in}}%
\pgfpathcurveto{\pgfqpoint{1.647831in}{1.701639in}}{\pgfqpoint{1.651103in}{1.709539in}}{\pgfqpoint{1.651103in}{1.717776in}}%
\pgfpathcurveto{\pgfqpoint{1.651103in}{1.726012in}}{\pgfqpoint{1.647831in}{1.733912in}}{\pgfqpoint{1.642007in}{1.739736in}}%
\pgfpathcurveto{\pgfqpoint{1.636183in}{1.745560in}}{\pgfqpoint{1.628283in}{1.748832in}}{\pgfqpoint{1.620046in}{1.748832in}}%
\pgfpathcurveto{\pgfqpoint{1.611810in}{1.748832in}}{\pgfqpoint{1.603910in}{1.745560in}}{\pgfqpoint{1.598086in}{1.739736in}}%
\pgfpathcurveto{\pgfqpoint{1.592262in}{1.733912in}}{\pgfqpoint{1.588990in}{1.726012in}}{\pgfqpoint{1.588990in}{1.717776in}}%
\pgfpathcurveto{\pgfqpoint{1.588990in}{1.709539in}}{\pgfqpoint{1.592262in}{1.701639in}}{\pgfqpoint{1.598086in}{1.695815in}}%
\pgfpathcurveto{\pgfqpoint{1.603910in}{1.689991in}}{\pgfqpoint{1.611810in}{1.686719in}}{\pgfqpoint{1.620046in}{1.686719in}}%
\pgfpathclose%
\pgfusepath{stroke,fill}%
\end{pgfscope}%
\begin{pgfscope}%
\pgfpathrectangle{\pgfqpoint{0.100000in}{0.212622in}}{\pgfqpoint{3.696000in}{3.696000in}}%
\pgfusepath{clip}%
\pgfsetbuttcap%
\pgfsetroundjoin%
\definecolor{currentfill}{rgb}{0.121569,0.466667,0.705882}%
\pgfsetfillcolor{currentfill}%
\pgfsetfillopacity{0.649092}%
\pgfsetlinewidth{1.003750pt}%
\definecolor{currentstroke}{rgb}{0.121569,0.466667,0.705882}%
\pgfsetstrokecolor{currentstroke}%
\pgfsetstrokeopacity{0.649092}%
\pgfsetdash{}{0pt}%
\pgfpathmoveto{\pgfqpoint{1.620527in}{1.686012in}}%
\pgfpathcurveto{\pgfqpoint{1.628764in}{1.686012in}}{\pgfqpoint{1.636664in}{1.689284in}}{\pgfqpoint{1.642488in}{1.695108in}}%
\pgfpathcurveto{\pgfqpoint{1.648312in}{1.700932in}}{\pgfqpoint{1.651584in}{1.708832in}}{\pgfqpoint{1.651584in}{1.717069in}}%
\pgfpathcurveto{\pgfqpoint{1.651584in}{1.725305in}}{\pgfqpoint{1.648312in}{1.733205in}}{\pgfqpoint{1.642488in}{1.739029in}}%
\pgfpathcurveto{\pgfqpoint{1.636664in}{1.744853in}}{\pgfqpoint{1.628764in}{1.748125in}}{\pgfqpoint{1.620527in}{1.748125in}}%
\pgfpathcurveto{\pgfqpoint{1.612291in}{1.748125in}}{\pgfqpoint{1.604391in}{1.744853in}}{\pgfqpoint{1.598567in}{1.739029in}}%
\pgfpathcurveto{\pgfqpoint{1.592743in}{1.733205in}}{\pgfqpoint{1.589471in}{1.725305in}}{\pgfqpoint{1.589471in}{1.717069in}}%
\pgfpathcurveto{\pgfqpoint{1.589471in}{1.708832in}}{\pgfqpoint{1.592743in}{1.700932in}}{\pgfqpoint{1.598567in}{1.695108in}}%
\pgfpathcurveto{\pgfqpoint{1.604391in}{1.689284in}}{\pgfqpoint{1.612291in}{1.686012in}}{\pgfqpoint{1.620527in}{1.686012in}}%
\pgfpathclose%
\pgfusepath{stroke,fill}%
\end{pgfscope}%
\begin{pgfscope}%
\pgfpathrectangle{\pgfqpoint{0.100000in}{0.212622in}}{\pgfqpoint{3.696000in}{3.696000in}}%
\pgfusepath{clip}%
\pgfsetbuttcap%
\pgfsetroundjoin%
\definecolor{currentfill}{rgb}{0.121569,0.466667,0.705882}%
\pgfsetfillcolor{currentfill}%
\pgfsetfillopacity{0.650521}%
\pgfsetlinewidth{1.003750pt}%
\definecolor{currentstroke}{rgb}{0.121569,0.466667,0.705882}%
\pgfsetstrokecolor{currentstroke}%
\pgfsetstrokeopacity{0.650521}%
\pgfsetdash{}{0pt}%
\pgfpathmoveto{\pgfqpoint{1.621144in}{1.685061in}}%
\pgfpathcurveto{\pgfqpoint{1.629381in}{1.685061in}}{\pgfqpoint{1.637281in}{1.688334in}}{\pgfqpoint{1.643105in}{1.694158in}}%
\pgfpathcurveto{\pgfqpoint{1.648929in}{1.699981in}}{\pgfqpoint{1.652201in}{1.707882in}}{\pgfqpoint{1.652201in}{1.716118in}}%
\pgfpathcurveto{\pgfqpoint{1.652201in}{1.724354in}}{\pgfqpoint{1.648929in}{1.732254in}}{\pgfqpoint{1.643105in}{1.738078in}}%
\pgfpathcurveto{\pgfqpoint{1.637281in}{1.743902in}}{\pgfqpoint{1.629381in}{1.747174in}}{\pgfqpoint{1.621144in}{1.747174in}}%
\pgfpathcurveto{\pgfqpoint{1.612908in}{1.747174in}}{\pgfqpoint{1.605008in}{1.743902in}}{\pgfqpoint{1.599184in}{1.738078in}}%
\pgfpathcurveto{\pgfqpoint{1.593360in}{1.732254in}}{\pgfqpoint{1.590088in}{1.724354in}}{\pgfqpoint{1.590088in}{1.716118in}}%
\pgfpathcurveto{\pgfqpoint{1.590088in}{1.707882in}}{\pgfqpoint{1.593360in}{1.699981in}}{\pgfqpoint{1.599184in}{1.694158in}}%
\pgfpathcurveto{\pgfqpoint{1.605008in}{1.688334in}}{\pgfqpoint{1.612908in}{1.685061in}}{\pgfqpoint{1.621144in}{1.685061in}}%
\pgfpathclose%
\pgfusepath{stroke,fill}%
\end{pgfscope}%
\begin{pgfscope}%
\pgfpathrectangle{\pgfqpoint{0.100000in}{0.212622in}}{\pgfqpoint{3.696000in}{3.696000in}}%
\pgfusepath{clip}%
\pgfsetbuttcap%
\pgfsetroundjoin%
\definecolor{currentfill}{rgb}{0.121569,0.466667,0.705882}%
\pgfsetfillcolor{currentfill}%
\pgfsetfillopacity{0.652456}%
\pgfsetlinewidth{1.003750pt}%
\definecolor{currentstroke}{rgb}{0.121569,0.466667,0.705882}%
\pgfsetstrokecolor{currentstroke}%
\pgfsetstrokeopacity{0.652456}%
\pgfsetdash{}{0pt}%
\pgfpathmoveto{\pgfqpoint{1.622081in}{1.684861in}}%
\pgfpathcurveto{\pgfqpoint{1.630317in}{1.684861in}}{\pgfqpoint{1.638217in}{1.688133in}}{\pgfqpoint{1.644041in}{1.693957in}}%
\pgfpathcurveto{\pgfqpoint{1.649865in}{1.699781in}}{\pgfqpoint{1.653138in}{1.707681in}}{\pgfqpoint{1.653138in}{1.715918in}}%
\pgfpathcurveto{\pgfqpoint{1.653138in}{1.724154in}}{\pgfqpoint{1.649865in}{1.732054in}}{\pgfqpoint{1.644041in}{1.737878in}}%
\pgfpathcurveto{\pgfqpoint{1.638217in}{1.743702in}}{\pgfqpoint{1.630317in}{1.746974in}}{\pgfqpoint{1.622081in}{1.746974in}}%
\pgfpathcurveto{\pgfqpoint{1.613845in}{1.746974in}}{\pgfqpoint{1.605945in}{1.743702in}}{\pgfqpoint{1.600121in}{1.737878in}}%
\pgfpathcurveto{\pgfqpoint{1.594297in}{1.732054in}}{\pgfqpoint{1.591025in}{1.724154in}}{\pgfqpoint{1.591025in}{1.715918in}}%
\pgfpathcurveto{\pgfqpoint{1.591025in}{1.707681in}}{\pgfqpoint{1.594297in}{1.699781in}}{\pgfqpoint{1.600121in}{1.693957in}}%
\pgfpathcurveto{\pgfqpoint{1.605945in}{1.688133in}}{\pgfqpoint{1.613845in}{1.684861in}}{\pgfqpoint{1.622081in}{1.684861in}}%
\pgfpathclose%
\pgfusepath{stroke,fill}%
\end{pgfscope}%
\begin{pgfscope}%
\pgfpathrectangle{\pgfqpoint{0.100000in}{0.212622in}}{\pgfqpoint{3.696000in}{3.696000in}}%
\pgfusepath{clip}%
\pgfsetbuttcap%
\pgfsetroundjoin%
\definecolor{currentfill}{rgb}{0.121569,0.466667,0.705882}%
\pgfsetfillcolor{currentfill}%
\pgfsetfillopacity{0.655432}%
\pgfsetlinewidth{1.003750pt}%
\definecolor{currentstroke}{rgb}{0.121569,0.466667,0.705882}%
\pgfsetstrokecolor{currentstroke}%
\pgfsetstrokeopacity{0.655432}%
\pgfsetdash{}{0pt}%
\pgfpathmoveto{\pgfqpoint{1.623727in}{1.686631in}}%
\pgfpathcurveto{\pgfqpoint{1.631963in}{1.686631in}}{\pgfqpoint{1.639863in}{1.689904in}}{\pgfqpoint{1.645687in}{1.695727in}}%
\pgfpathcurveto{\pgfqpoint{1.651511in}{1.701551in}}{\pgfqpoint{1.654784in}{1.709451in}}{\pgfqpoint{1.654784in}{1.717688in}}%
\pgfpathcurveto{\pgfqpoint{1.654784in}{1.725924in}}{\pgfqpoint{1.651511in}{1.733824in}}{\pgfqpoint{1.645687in}{1.739648in}}%
\pgfpathcurveto{\pgfqpoint{1.639863in}{1.745472in}}{\pgfqpoint{1.631963in}{1.748744in}}{\pgfqpoint{1.623727in}{1.748744in}}%
\pgfpathcurveto{\pgfqpoint{1.615491in}{1.748744in}}{\pgfqpoint{1.607591in}{1.745472in}}{\pgfqpoint{1.601767in}{1.739648in}}%
\pgfpathcurveto{\pgfqpoint{1.595943in}{1.733824in}}{\pgfqpoint{1.592671in}{1.725924in}}{\pgfqpoint{1.592671in}{1.717688in}}%
\pgfpathcurveto{\pgfqpoint{1.592671in}{1.709451in}}{\pgfqpoint{1.595943in}{1.701551in}}{\pgfqpoint{1.601767in}{1.695727in}}%
\pgfpathcurveto{\pgfqpoint{1.607591in}{1.689904in}}{\pgfqpoint{1.615491in}{1.686631in}}{\pgfqpoint{1.623727in}{1.686631in}}%
\pgfpathclose%
\pgfusepath{stroke,fill}%
\end{pgfscope}%
\begin{pgfscope}%
\pgfpathrectangle{\pgfqpoint{0.100000in}{0.212622in}}{\pgfqpoint{3.696000in}{3.696000in}}%
\pgfusepath{clip}%
\pgfsetbuttcap%
\pgfsetroundjoin%
\definecolor{currentfill}{rgb}{0.121569,0.466667,0.705882}%
\pgfsetfillcolor{currentfill}%
\pgfsetfillopacity{0.658045}%
\pgfsetlinewidth{1.003750pt}%
\definecolor{currentstroke}{rgb}{0.121569,0.466667,0.705882}%
\pgfsetstrokecolor{currentstroke}%
\pgfsetstrokeopacity{0.658045}%
\pgfsetdash{}{0pt}%
\pgfpathmoveto{\pgfqpoint{1.624506in}{1.685292in}}%
\pgfpathcurveto{\pgfqpoint{1.632743in}{1.685292in}}{\pgfqpoint{1.640643in}{1.688564in}}{\pgfqpoint{1.646467in}{1.694388in}}%
\pgfpathcurveto{\pgfqpoint{1.652291in}{1.700212in}}{\pgfqpoint{1.655563in}{1.708112in}}{\pgfqpoint{1.655563in}{1.716348in}}%
\pgfpathcurveto{\pgfqpoint{1.655563in}{1.724585in}}{\pgfqpoint{1.652291in}{1.732485in}}{\pgfqpoint{1.646467in}{1.738309in}}%
\pgfpathcurveto{\pgfqpoint{1.640643in}{1.744133in}}{\pgfqpoint{1.632743in}{1.747405in}}{\pgfqpoint{1.624506in}{1.747405in}}%
\pgfpathcurveto{\pgfqpoint{1.616270in}{1.747405in}}{\pgfqpoint{1.608370in}{1.744133in}}{\pgfqpoint{1.602546in}{1.738309in}}%
\pgfpathcurveto{\pgfqpoint{1.596722in}{1.732485in}}{\pgfqpoint{1.593450in}{1.724585in}}{\pgfqpoint{1.593450in}{1.716348in}}%
\pgfpathcurveto{\pgfqpoint{1.593450in}{1.708112in}}{\pgfqpoint{1.596722in}{1.700212in}}{\pgfqpoint{1.602546in}{1.694388in}}%
\pgfpathcurveto{\pgfqpoint{1.608370in}{1.688564in}}{\pgfqpoint{1.616270in}{1.685292in}}{\pgfqpoint{1.624506in}{1.685292in}}%
\pgfpathclose%
\pgfusepath{stroke,fill}%
\end{pgfscope}%
\begin{pgfscope}%
\pgfpathrectangle{\pgfqpoint{0.100000in}{0.212622in}}{\pgfqpoint{3.696000in}{3.696000in}}%
\pgfusepath{clip}%
\pgfsetbuttcap%
\pgfsetroundjoin%
\definecolor{currentfill}{rgb}{0.121569,0.466667,0.705882}%
\pgfsetfillcolor{currentfill}%
\pgfsetfillopacity{0.660683}%
\pgfsetlinewidth{1.003750pt}%
\definecolor{currentstroke}{rgb}{0.121569,0.466667,0.705882}%
\pgfsetstrokecolor{currentstroke}%
\pgfsetstrokeopacity{0.660683}%
\pgfsetdash{}{0pt}%
\pgfpathmoveto{\pgfqpoint{1.626271in}{1.681549in}}%
\pgfpathcurveto{\pgfqpoint{1.634508in}{1.681549in}}{\pgfqpoint{1.642408in}{1.684822in}}{\pgfqpoint{1.648232in}{1.690646in}}%
\pgfpathcurveto{\pgfqpoint{1.654055in}{1.696470in}}{\pgfqpoint{1.657328in}{1.704370in}}{\pgfqpoint{1.657328in}{1.712606in}}%
\pgfpathcurveto{\pgfqpoint{1.657328in}{1.720842in}}{\pgfqpoint{1.654055in}{1.728742in}}{\pgfqpoint{1.648232in}{1.734566in}}%
\pgfpathcurveto{\pgfqpoint{1.642408in}{1.740390in}}{\pgfqpoint{1.634508in}{1.743662in}}{\pgfqpoint{1.626271in}{1.743662in}}%
\pgfpathcurveto{\pgfqpoint{1.618035in}{1.743662in}}{\pgfqpoint{1.610135in}{1.740390in}}{\pgfqpoint{1.604311in}{1.734566in}}%
\pgfpathcurveto{\pgfqpoint{1.598487in}{1.728742in}}{\pgfqpoint{1.595215in}{1.720842in}}{\pgfqpoint{1.595215in}{1.712606in}}%
\pgfpathcurveto{\pgfqpoint{1.595215in}{1.704370in}}{\pgfqpoint{1.598487in}{1.696470in}}{\pgfqpoint{1.604311in}{1.690646in}}%
\pgfpathcurveto{\pgfqpoint{1.610135in}{1.684822in}}{\pgfqpoint{1.618035in}{1.681549in}}{\pgfqpoint{1.626271in}{1.681549in}}%
\pgfpathclose%
\pgfusepath{stroke,fill}%
\end{pgfscope}%
\begin{pgfscope}%
\pgfpathrectangle{\pgfqpoint{0.100000in}{0.212622in}}{\pgfqpoint{3.696000in}{3.696000in}}%
\pgfusepath{clip}%
\pgfsetbuttcap%
\pgfsetroundjoin%
\definecolor{currentfill}{rgb}{0.121569,0.466667,0.705882}%
\pgfsetfillcolor{currentfill}%
\pgfsetfillopacity{0.664432}%
\pgfsetlinewidth{1.003750pt}%
\definecolor{currentstroke}{rgb}{0.121569,0.466667,0.705882}%
\pgfsetstrokecolor{currentstroke}%
\pgfsetstrokeopacity{0.664432}%
\pgfsetdash{}{0pt}%
\pgfpathmoveto{\pgfqpoint{1.627707in}{1.681679in}}%
\pgfpathcurveto{\pgfqpoint{1.635943in}{1.681679in}}{\pgfqpoint{1.643843in}{1.684952in}}{\pgfqpoint{1.649667in}{1.690776in}}%
\pgfpathcurveto{\pgfqpoint{1.655491in}{1.696599in}}{\pgfqpoint{1.658763in}{1.704499in}}{\pgfqpoint{1.658763in}{1.712736in}}%
\pgfpathcurveto{\pgfqpoint{1.658763in}{1.720972in}}{\pgfqpoint{1.655491in}{1.728872in}}{\pgfqpoint{1.649667in}{1.734696in}}%
\pgfpathcurveto{\pgfqpoint{1.643843in}{1.740520in}}{\pgfqpoint{1.635943in}{1.743792in}}{\pgfqpoint{1.627707in}{1.743792in}}%
\pgfpathcurveto{\pgfqpoint{1.619470in}{1.743792in}}{\pgfqpoint{1.611570in}{1.740520in}}{\pgfqpoint{1.605746in}{1.734696in}}%
\pgfpathcurveto{\pgfqpoint{1.599923in}{1.728872in}}{\pgfqpoint{1.596650in}{1.720972in}}{\pgfqpoint{1.596650in}{1.712736in}}%
\pgfpathcurveto{\pgfqpoint{1.596650in}{1.704499in}}{\pgfqpoint{1.599923in}{1.696599in}}{\pgfqpoint{1.605746in}{1.690776in}}%
\pgfpathcurveto{\pgfqpoint{1.611570in}{1.684952in}}{\pgfqpoint{1.619470in}{1.681679in}}{\pgfqpoint{1.627707in}{1.681679in}}%
\pgfpathclose%
\pgfusepath{stroke,fill}%
\end{pgfscope}%
\begin{pgfscope}%
\pgfpathrectangle{\pgfqpoint{0.100000in}{0.212622in}}{\pgfqpoint{3.696000in}{3.696000in}}%
\pgfusepath{clip}%
\pgfsetbuttcap%
\pgfsetroundjoin%
\definecolor{currentfill}{rgb}{0.121569,0.466667,0.705882}%
\pgfsetfillcolor{currentfill}%
\pgfsetfillopacity{0.669279}%
\pgfsetlinewidth{1.003750pt}%
\definecolor{currentstroke}{rgb}{0.121569,0.466667,0.705882}%
\pgfsetstrokecolor{currentstroke}%
\pgfsetstrokeopacity{0.669279}%
\pgfsetdash{}{0pt}%
\pgfpathmoveto{\pgfqpoint{1.629366in}{1.683008in}}%
\pgfpathcurveto{\pgfqpoint{1.637602in}{1.683008in}}{\pgfqpoint{1.645502in}{1.686280in}}{\pgfqpoint{1.651326in}{1.692104in}}%
\pgfpathcurveto{\pgfqpoint{1.657150in}{1.697928in}}{\pgfqpoint{1.660422in}{1.705828in}}{\pgfqpoint{1.660422in}{1.714065in}}%
\pgfpathcurveto{\pgfqpoint{1.660422in}{1.722301in}}{\pgfqpoint{1.657150in}{1.730201in}}{\pgfqpoint{1.651326in}{1.736025in}}%
\pgfpathcurveto{\pgfqpoint{1.645502in}{1.741849in}}{\pgfqpoint{1.637602in}{1.745121in}}{\pgfqpoint{1.629366in}{1.745121in}}%
\pgfpathcurveto{\pgfqpoint{1.621129in}{1.745121in}}{\pgfqpoint{1.613229in}{1.741849in}}{\pgfqpoint{1.607405in}{1.736025in}}%
\pgfpathcurveto{\pgfqpoint{1.601581in}{1.730201in}}{\pgfqpoint{1.598309in}{1.722301in}}{\pgfqpoint{1.598309in}{1.714065in}}%
\pgfpathcurveto{\pgfqpoint{1.598309in}{1.705828in}}{\pgfqpoint{1.601581in}{1.697928in}}{\pgfqpoint{1.607405in}{1.692104in}}%
\pgfpathcurveto{\pgfqpoint{1.613229in}{1.686280in}}{\pgfqpoint{1.621129in}{1.683008in}}{\pgfqpoint{1.629366in}{1.683008in}}%
\pgfpathclose%
\pgfusepath{stroke,fill}%
\end{pgfscope}%
\begin{pgfscope}%
\pgfpathrectangle{\pgfqpoint{0.100000in}{0.212622in}}{\pgfqpoint{3.696000in}{3.696000in}}%
\pgfusepath{clip}%
\pgfsetbuttcap%
\pgfsetroundjoin%
\definecolor{currentfill}{rgb}{0.121569,0.466667,0.705882}%
\pgfsetfillcolor{currentfill}%
\pgfsetfillopacity{0.673820}%
\pgfsetlinewidth{1.003750pt}%
\definecolor{currentstroke}{rgb}{0.121569,0.466667,0.705882}%
\pgfsetstrokecolor{currentstroke}%
\pgfsetstrokeopacity{0.673820}%
\pgfsetdash{}{0pt}%
\pgfpathmoveto{\pgfqpoint{1.631013in}{1.681370in}}%
\pgfpathcurveto{\pgfqpoint{1.639249in}{1.681370in}}{\pgfqpoint{1.647149in}{1.684642in}}{\pgfqpoint{1.652973in}{1.690466in}}%
\pgfpathcurveto{\pgfqpoint{1.658797in}{1.696290in}}{\pgfqpoint{1.662070in}{1.704190in}}{\pgfqpoint{1.662070in}{1.712427in}}%
\pgfpathcurveto{\pgfqpoint{1.662070in}{1.720663in}}{\pgfqpoint{1.658797in}{1.728563in}}{\pgfqpoint{1.652973in}{1.734387in}}%
\pgfpathcurveto{\pgfqpoint{1.647149in}{1.740211in}}{\pgfqpoint{1.639249in}{1.743483in}}{\pgfqpoint{1.631013in}{1.743483in}}%
\pgfpathcurveto{\pgfqpoint{1.622777in}{1.743483in}}{\pgfqpoint{1.614877in}{1.740211in}}{\pgfqpoint{1.609053in}{1.734387in}}%
\pgfpathcurveto{\pgfqpoint{1.603229in}{1.728563in}}{\pgfqpoint{1.599957in}{1.720663in}}{\pgfqpoint{1.599957in}{1.712427in}}%
\pgfpathcurveto{\pgfqpoint{1.599957in}{1.704190in}}{\pgfqpoint{1.603229in}{1.696290in}}{\pgfqpoint{1.609053in}{1.690466in}}%
\pgfpathcurveto{\pgfqpoint{1.614877in}{1.684642in}}{\pgfqpoint{1.622777in}{1.681370in}}{\pgfqpoint{1.631013in}{1.681370in}}%
\pgfpathclose%
\pgfusepath{stroke,fill}%
\end{pgfscope}%
\begin{pgfscope}%
\pgfpathrectangle{\pgfqpoint{0.100000in}{0.212622in}}{\pgfqpoint{3.696000in}{3.696000in}}%
\pgfusepath{clip}%
\pgfsetbuttcap%
\pgfsetroundjoin%
\definecolor{currentfill}{rgb}{0.121569,0.466667,0.705882}%
\pgfsetfillcolor{currentfill}%
\pgfsetfillopacity{0.678552}%
\pgfsetlinewidth{1.003750pt}%
\definecolor{currentstroke}{rgb}{0.121569,0.466667,0.705882}%
\pgfsetstrokecolor{currentstroke}%
\pgfsetstrokeopacity{0.678552}%
\pgfsetdash{}{0pt}%
\pgfpathmoveto{\pgfqpoint{1.633433in}{1.677694in}}%
\pgfpathcurveto{\pgfqpoint{1.641670in}{1.677694in}}{\pgfqpoint{1.649570in}{1.680966in}}{\pgfqpoint{1.655393in}{1.686790in}}%
\pgfpathcurveto{\pgfqpoint{1.661217in}{1.692614in}}{\pgfqpoint{1.664490in}{1.700514in}}{\pgfqpoint{1.664490in}{1.708750in}}%
\pgfpathcurveto{\pgfqpoint{1.664490in}{1.716986in}}{\pgfqpoint{1.661217in}{1.724886in}}{\pgfqpoint{1.655393in}{1.730710in}}%
\pgfpathcurveto{\pgfqpoint{1.649570in}{1.736534in}}{\pgfqpoint{1.641670in}{1.739807in}}{\pgfqpoint{1.633433in}{1.739807in}}%
\pgfpathcurveto{\pgfqpoint{1.625197in}{1.739807in}}{\pgfqpoint{1.617297in}{1.736534in}}{\pgfqpoint{1.611473in}{1.730710in}}%
\pgfpathcurveto{\pgfqpoint{1.605649in}{1.724886in}}{\pgfqpoint{1.602377in}{1.716986in}}{\pgfqpoint{1.602377in}{1.708750in}}%
\pgfpathcurveto{\pgfqpoint{1.602377in}{1.700514in}}{\pgfqpoint{1.605649in}{1.692614in}}{\pgfqpoint{1.611473in}{1.686790in}}%
\pgfpathcurveto{\pgfqpoint{1.617297in}{1.680966in}}{\pgfqpoint{1.625197in}{1.677694in}}{\pgfqpoint{1.633433in}{1.677694in}}%
\pgfpathclose%
\pgfusepath{stroke,fill}%
\end{pgfscope}%
\begin{pgfscope}%
\pgfpathrectangle{\pgfqpoint{0.100000in}{0.212622in}}{\pgfqpoint{3.696000in}{3.696000in}}%
\pgfusepath{clip}%
\pgfsetbuttcap%
\pgfsetroundjoin%
\definecolor{currentfill}{rgb}{0.121569,0.466667,0.705882}%
\pgfsetfillcolor{currentfill}%
\pgfsetfillopacity{0.684547}%
\pgfsetlinewidth{1.003750pt}%
\definecolor{currentstroke}{rgb}{0.121569,0.466667,0.705882}%
\pgfsetstrokecolor{currentstroke}%
\pgfsetstrokeopacity{0.684547}%
\pgfsetdash{}{0pt}%
\pgfpathmoveto{\pgfqpoint{1.637885in}{1.678958in}}%
\pgfpathcurveto{\pgfqpoint{1.646121in}{1.678958in}}{\pgfqpoint{1.654021in}{1.682230in}}{\pgfqpoint{1.659845in}{1.688054in}}%
\pgfpathcurveto{\pgfqpoint{1.665669in}{1.693878in}}{\pgfqpoint{1.668942in}{1.701778in}}{\pgfqpoint{1.668942in}{1.710015in}}%
\pgfpathcurveto{\pgfqpoint{1.668942in}{1.718251in}}{\pgfqpoint{1.665669in}{1.726151in}}{\pgfqpoint{1.659845in}{1.731975in}}%
\pgfpathcurveto{\pgfqpoint{1.654021in}{1.737799in}}{\pgfqpoint{1.646121in}{1.741071in}}{\pgfqpoint{1.637885in}{1.741071in}}%
\pgfpathcurveto{\pgfqpoint{1.629649in}{1.741071in}}{\pgfqpoint{1.621749in}{1.737799in}}{\pgfqpoint{1.615925in}{1.731975in}}%
\pgfpathcurveto{\pgfqpoint{1.610101in}{1.726151in}}{\pgfqpoint{1.606829in}{1.718251in}}{\pgfqpoint{1.606829in}{1.710015in}}%
\pgfpathcurveto{\pgfqpoint{1.606829in}{1.701778in}}{\pgfqpoint{1.610101in}{1.693878in}}{\pgfqpoint{1.615925in}{1.688054in}}%
\pgfpathcurveto{\pgfqpoint{1.621749in}{1.682230in}}{\pgfqpoint{1.629649in}{1.678958in}}{\pgfqpoint{1.637885in}{1.678958in}}%
\pgfpathclose%
\pgfusepath{stroke,fill}%
\end{pgfscope}%
\begin{pgfscope}%
\pgfpathrectangle{\pgfqpoint{0.100000in}{0.212622in}}{\pgfqpoint{3.696000in}{3.696000in}}%
\pgfusepath{clip}%
\pgfsetbuttcap%
\pgfsetroundjoin%
\definecolor{currentfill}{rgb}{0.121569,0.466667,0.705882}%
\pgfsetfillcolor{currentfill}%
\pgfsetfillopacity{0.690745}%
\pgfsetlinewidth{1.003750pt}%
\definecolor{currentstroke}{rgb}{0.121569,0.466667,0.705882}%
\pgfsetstrokecolor{currentstroke}%
\pgfsetstrokeopacity{0.690745}%
\pgfsetdash{}{0pt}%
\pgfpathmoveto{\pgfqpoint{1.640163in}{1.676108in}}%
\pgfpathcurveto{\pgfqpoint{1.648399in}{1.676108in}}{\pgfqpoint{1.656299in}{1.679380in}}{\pgfqpoint{1.662123in}{1.685204in}}%
\pgfpathcurveto{\pgfqpoint{1.667947in}{1.691028in}}{\pgfqpoint{1.671219in}{1.698928in}}{\pgfqpoint{1.671219in}{1.707164in}}%
\pgfpathcurveto{\pgfqpoint{1.671219in}{1.715400in}}{\pgfqpoint{1.667947in}{1.723300in}}{\pgfqpoint{1.662123in}{1.729124in}}%
\pgfpathcurveto{\pgfqpoint{1.656299in}{1.734948in}}{\pgfqpoint{1.648399in}{1.738221in}}{\pgfqpoint{1.640163in}{1.738221in}}%
\pgfpathcurveto{\pgfqpoint{1.631926in}{1.738221in}}{\pgfqpoint{1.624026in}{1.734948in}}{\pgfqpoint{1.618202in}{1.729124in}}%
\pgfpathcurveto{\pgfqpoint{1.612378in}{1.723300in}}{\pgfqpoint{1.609106in}{1.715400in}}{\pgfqpoint{1.609106in}{1.707164in}}%
\pgfpathcurveto{\pgfqpoint{1.609106in}{1.698928in}}{\pgfqpoint{1.612378in}{1.691028in}}{\pgfqpoint{1.618202in}{1.685204in}}%
\pgfpathcurveto{\pgfqpoint{1.624026in}{1.679380in}}{\pgfqpoint{1.631926in}{1.676108in}}{\pgfqpoint{1.640163in}{1.676108in}}%
\pgfpathclose%
\pgfusepath{stroke,fill}%
\end{pgfscope}%
\begin{pgfscope}%
\pgfpathrectangle{\pgfqpoint{0.100000in}{0.212622in}}{\pgfqpoint{3.696000in}{3.696000in}}%
\pgfusepath{clip}%
\pgfsetbuttcap%
\pgfsetroundjoin%
\definecolor{currentfill}{rgb}{0.121569,0.466667,0.705882}%
\pgfsetfillcolor{currentfill}%
\pgfsetfillopacity{0.696383}%
\pgfsetlinewidth{1.003750pt}%
\definecolor{currentstroke}{rgb}{0.121569,0.466667,0.705882}%
\pgfsetstrokecolor{currentstroke}%
\pgfsetstrokeopacity{0.696383}%
\pgfsetdash{}{0pt}%
\pgfpathmoveto{\pgfqpoint{1.642123in}{1.668665in}}%
\pgfpathcurveto{\pgfqpoint{1.650359in}{1.668665in}}{\pgfqpoint{1.658260in}{1.671937in}}{\pgfqpoint{1.664083in}{1.677761in}}%
\pgfpathcurveto{\pgfqpoint{1.669907in}{1.683585in}}{\pgfqpoint{1.673180in}{1.691485in}}{\pgfqpoint{1.673180in}{1.699721in}}%
\pgfpathcurveto{\pgfqpoint{1.673180in}{1.707958in}}{\pgfqpoint{1.669907in}{1.715858in}}{\pgfqpoint{1.664083in}{1.721682in}}%
\pgfpathcurveto{\pgfqpoint{1.658260in}{1.727506in}}{\pgfqpoint{1.650359in}{1.730778in}}{\pgfqpoint{1.642123in}{1.730778in}}%
\pgfpathcurveto{\pgfqpoint{1.633887in}{1.730778in}}{\pgfqpoint{1.625987in}{1.727506in}}{\pgfqpoint{1.620163in}{1.721682in}}%
\pgfpathcurveto{\pgfqpoint{1.614339in}{1.715858in}}{\pgfqpoint{1.611067in}{1.707958in}}{\pgfqpoint{1.611067in}{1.699721in}}%
\pgfpathcurveto{\pgfqpoint{1.611067in}{1.691485in}}{\pgfqpoint{1.614339in}{1.683585in}}{\pgfqpoint{1.620163in}{1.677761in}}%
\pgfpathcurveto{\pgfqpoint{1.625987in}{1.671937in}}{\pgfqpoint{1.633887in}{1.668665in}}{\pgfqpoint{1.642123in}{1.668665in}}%
\pgfpathclose%
\pgfusepath{stroke,fill}%
\end{pgfscope}%
\begin{pgfscope}%
\pgfpathrectangle{\pgfqpoint{0.100000in}{0.212622in}}{\pgfqpoint{3.696000in}{3.696000in}}%
\pgfusepath{clip}%
\pgfsetbuttcap%
\pgfsetroundjoin%
\definecolor{currentfill}{rgb}{0.121569,0.466667,0.705882}%
\pgfsetfillcolor{currentfill}%
\pgfsetfillopacity{0.702435}%
\pgfsetlinewidth{1.003750pt}%
\definecolor{currentstroke}{rgb}{0.121569,0.466667,0.705882}%
\pgfsetstrokecolor{currentstroke}%
\pgfsetstrokeopacity{0.702435}%
\pgfsetdash{}{0pt}%
\pgfpathmoveto{\pgfqpoint{1.645407in}{1.661034in}}%
\pgfpathcurveto{\pgfqpoint{1.653644in}{1.661034in}}{\pgfqpoint{1.661544in}{1.664306in}}{\pgfqpoint{1.667368in}{1.670130in}}%
\pgfpathcurveto{\pgfqpoint{1.673191in}{1.675954in}}{\pgfqpoint{1.676464in}{1.683854in}}{\pgfqpoint{1.676464in}{1.692090in}}%
\pgfpathcurveto{\pgfqpoint{1.676464in}{1.700327in}}{\pgfqpoint{1.673191in}{1.708227in}}{\pgfqpoint{1.667368in}{1.714051in}}%
\pgfpathcurveto{\pgfqpoint{1.661544in}{1.719875in}}{\pgfqpoint{1.653644in}{1.723147in}}{\pgfqpoint{1.645407in}{1.723147in}}%
\pgfpathcurveto{\pgfqpoint{1.637171in}{1.723147in}}{\pgfqpoint{1.629271in}{1.719875in}}{\pgfqpoint{1.623447in}{1.714051in}}%
\pgfpathcurveto{\pgfqpoint{1.617623in}{1.708227in}}{\pgfqpoint{1.614351in}{1.700327in}}{\pgfqpoint{1.614351in}{1.692090in}}%
\pgfpathcurveto{\pgfqpoint{1.614351in}{1.683854in}}{\pgfqpoint{1.617623in}{1.675954in}}{\pgfqpoint{1.623447in}{1.670130in}}%
\pgfpathcurveto{\pgfqpoint{1.629271in}{1.664306in}}{\pgfqpoint{1.637171in}{1.661034in}}{\pgfqpoint{1.645407in}{1.661034in}}%
\pgfpathclose%
\pgfusepath{stroke,fill}%
\end{pgfscope}%
\begin{pgfscope}%
\pgfpathrectangle{\pgfqpoint{0.100000in}{0.212622in}}{\pgfqpoint{3.696000in}{3.696000in}}%
\pgfusepath{clip}%
\pgfsetbuttcap%
\pgfsetroundjoin%
\definecolor{currentfill}{rgb}{0.121569,0.466667,0.705882}%
\pgfsetfillcolor{currentfill}%
\pgfsetfillopacity{0.711280}%
\pgfsetlinewidth{1.003750pt}%
\definecolor{currentstroke}{rgb}{0.121569,0.466667,0.705882}%
\pgfsetstrokecolor{currentstroke}%
\pgfsetstrokeopacity{0.711280}%
\pgfsetdash{}{0pt}%
\pgfpathmoveto{\pgfqpoint{1.650963in}{1.666520in}}%
\pgfpathcurveto{\pgfqpoint{1.659200in}{1.666520in}}{\pgfqpoint{1.667100in}{1.669792in}}{\pgfqpoint{1.672923in}{1.675616in}}%
\pgfpathcurveto{\pgfqpoint{1.678747in}{1.681440in}}{\pgfqpoint{1.682020in}{1.689340in}}{\pgfqpoint{1.682020in}{1.697576in}}%
\pgfpathcurveto{\pgfqpoint{1.682020in}{1.705813in}}{\pgfqpoint{1.678747in}{1.713713in}}{\pgfqpoint{1.672923in}{1.719536in}}%
\pgfpathcurveto{\pgfqpoint{1.667100in}{1.725360in}}{\pgfqpoint{1.659200in}{1.728633in}}{\pgfqpoint{1.650963in}{1.728633in}}%
\pgfpathcurveto{\pgfqpoint{1.642727in}{1.728633in}}{\pgfqpoint{1.634827in}{1.725360in}}{\pgfqpoint{1.629003in}{1.719536in}}%
\pgfpathcurveto{\pgfqpoint{1.623179in}{1.713713in}}{\pgfqpoint{1.619907in}{1.705813in}}{\pgfqpoint{1.619907in}{1.697576in}}%
\pgfpathcurveto{\pgfqpoint{1.619907in}{1.689340in}}{\pgfqpoint{1.623179in}{1.681440in}}{\pgfqpoint{1.629003in}{1.675616in}}%
\pgfpathcurveto{\pgfqpoint{1.634827in}{1.669792in}}{\pgfqpoint{1.642727in}{1.666520in}}{\pgfqpoint{1.650963in}{1.666520in}}%
\pgfpathclose%
\pgfusepath{stroke,fill}%
\end{pgfscope}%
\begin{pgfscope}%
\pgfpathrectangle{\pgfqpoint{0.100000in}{0.212622in}}{\pgfqpoint{3.696000in}{3.696000in}}%
\pgfusepath{clip}%
\pgfsetbuttcap%
\pgfsetroundjoin%
\definecolor{currentfill}{rgb}{0.121569,0.466667,0.705882}%
\pgfsetfillcolor{currentfill}%
\pgfsetfillopacity{0.715223}%
\pgfsetlinewidth{1.003750pt}%
\definecolor{currentstroke}{rgb}{0.121569,0.466667,0.705882}%
\pgfsetstrokecolor{currentstroke}%
\pgfsetstrokeopacity{0.715223}%
\pgfsetdash{}{0pt}%
\pgfpathmoveto{\pgfqpoint{1.652700in}{1.664594in}}%
\pgfpathcurveto{\pgfqpoint{1.660936in}{1.664594in}}{\pgfqpoint{1.668837in}{1.667866in}}{\pgfqpoint{1.674660in}{1.673690in}}%
\pgfpathcurveto{\pgfqpoint{1.680484in}{1.679514in}}{\pgfqpoint{1.683757in}{1.687414in}}{\pgfqpoint{1.683757in}{1.695650in}}%
\pgfpathcurveto{\pgfqpoint{1.683757in}{1.703886in}}{\pgfqpoint{1.680484in}{1.711786in}}{\pgfqpoint{1.674660in}{1.717610in}}%
\pgfpathcurveto{\pgfqpoint{1.668837in}{1.723434in}}{\pgfqpoint{1.660936in}{1.726707in}}{\pgfqpoint{1.652700in}{1.726707in}}%
\pgfpathcurveto{\pgfqpoint{1.644464in}{1.726707in}}{\pgfqpoint{1.636564in}{1.723434in}}{\pgfqpoint{1.630740in}{1.717610in}}%
\pgfpathcurveto{\pgfqpoint{1.624916in}{1.711786in}}{\pgfqpoint{1.621644in}{1.703886in}}{\pgfqpoint{1.621644in}{1.695650in}}%
\pgfpathcurveto{\pgfqpoint{1.621644in}{1.687414in}}{\pgfqpoint{1.624916in}{1.679514in}}{\pgfqpoint{1.630740in}{1.673690in}}%
\pgfpathcurveto{\pgfqpoint{1.636564in}{1.667866in}}{\pgfqpoint{1.644464in}{1.664594in}}{\pgfqpoint{1.652700in}{1.664594in}}%
\pgfpathclose%
\pgfusepath{stroke,fill}%
\end{pgfscope}%
\begin{pgfscope}%
\pgfpathrectangle{\pgfqpoint{0.100000in}{0.212622in}}{\pgfqpoint{3.696000in}{3.696000in}}%
\pgfusepath{clip}%
\pgfsetbuttcap%
\pgfsetroundjoin%
\definecolor{currentfill}{rgb}{0.121569,0.466667,0.705882}%
\pgfsetfillcolor{currentfill}%
\pgfsetfillopacity{0.718252}%
\pgfsetlinewidth{1.003750pt}%
\definecolor{currentstroke}{rgb}{0.121569,0.466667,0.705882}%
\pgfsetstrokecolor{currentstroke}%
\pgfsetstrokeopacity{0.718252}%
\pgfsetdash{}{0pt}%
\pgfpathmoveto{\pgfqpoint{1.654800in}{1.657933in}}%
\pgfpathcurveto{\pgfqpoint{1.663036in}{1.657933in}}{\pgfqpoint{1.670936in}{1.661205in}}{\pgfqpoint{1.676760in}{1.667029in}}%
\pgfpathcurveto{\pgfqpoint{1.682584in}{1.672853in}}{\pgfqpoint{1.685856in}{1.680753in}}{\pgfqpoint{1.685856in}{1.688990in}}%
\pgfpathcurveto{\pgfqpoint{1.685856in}{1.697226in}}{\pgfqpoint{1.682584in}{1.705126in}}{\pgfqpoint{1.676760in}{1.710950in}}%
\pgfpathcurveto{\pgfqpoint{1.670936in}{1.716774in}}{\pgfqpoint{1.663036in}{1.720046in}}{\pgfqpoint{1.654800in}{1.720046in}}%
\pgfpathcurveto{\pgfqpoint{1.646564in}{1.720046in}}{\pgfqpoint{1.638664in}{1.716774in}}{\pgfqpoint{1.632840in}{1.710950in}}%
\pgfpathcurveto{\pgfqpoint{1.627016in}{1.705126in}}{\pgfqpoint{1.623743in}{1.697226in}}{\pgfqpoint{1.623743in}{1.688990in}}%
\pgfpathcurveto{\pgfqpoint{1.623743in}{1.680753in}}{\pgfqpoint{1.627016in}{1.672853in}}{\pgfqpoint{1.632840in}{1.667029in}}%
\pgfpathcurveto{\pgfqpoint{1.638664in}{1.661205in}}{\pgfqpoint{1.646564in}{1.657933in}}{\pgfqpoint{1.654800in}{1.657933in}}%
\pgfpathclose%
\pgfusepath{stroke,fill}%
\end{pgfscope}%
\begin{pgfscope}%
\pgfpathrectangle{\pgfqpoint{0.100000in}{0.212622in}}{\pgfqpoint{3.696000in}{3.696000in}}%
\pgfusepath{clip}%
\pgfsetbuttcap%
\pgfsetroundjoin%
\definecolor{currentfill}{rgb}{0.121569,0.466667,0.705882}%
\pgfsetfillcolor{currentfill}%
\pgfsetfillopacity{0.720292}%
\pgfsetlinewidth{1.003750pt}%
\definecolor{currentstroke}{rgb}{0.121569,0.466667,0.705882}%
\pgfsetstrokecolor{currentstroke}%
\pgfsetstrokeopacity{0.720292}%
\pgfsetdash{}{0pt}%
\pgfpathmoveto{\pgfqpoint{1.656311in}{1.656145in}}%
\pgfpathcurveto{\pgfqpoint{1.664547in}{1.656145in}}{\pgfqpoint{1.672447in}{1.659417in}}{\pgfqpoint{1.678271in}{1.665241in}}%
\pgfpathcurveto{\pgfqpoint{1.684095in}{1.671065in}}{\pgfqpoint{1.687368in}{1.678965in}}{\pgfqpoint{1.687368in}{1.687201in}}%
\pgfpathcurveto{\pgfqpoint{1.687368in}{1.695438in}}{\pgfqpoint{1.684095in}{1.703338in}}{\pgfqpoint{1.678271in}{1.709162in}}%
\pgfpathcurveto{\pgfqpoint{1.672447in}{1.714986in}}{\pgfqpoint{1.664547in}{1.718258in}}{\pgfqpoint{1.656311in}{1.718258in}}%
\pgfpathcurveto{\pgfqpoint{1.648075in}{1.718258in}}{\pgfqpoint{1.640175in}{1.714986in}}{\pgfqpoint{1.634351in}{1.709162in}}%
\pgfpathcurveto{\pgfqpoint{1.628527in}{1.703338in}}{\pgfqpoint{1.625255in}{1.695438in}}{\pgfqpoint{1.625255in}{1.687201in}}%
\pgfpathcurveto{\pgfqpoint{1.625255in}{1.678965in}}{\pgfqpoint{1.628527in}{1.671065in}}{\pgfqpoint{1.634351in}{1.665241in}}%
\pgfpathcurveto{\pgfqpoint{1.640175in}{1.659417in}}{\pgfqpoint{1.648075in}{1.656145in}}{\pgfqpoint{1.656311in}{1.656145in}}%
\pgfpathclose%
\pgfusepath{stroke,fill}%
\end{pgfscope}%
\begin{pgfscope}%
\pgfpathrectangle{\pgfqpoint{0.100000in}{0.212622in}}{\pgfqpoint{3.696000in}{3.696000in}}%
\pgfusepath{clip}%
\pgfsetbuttcap%
\pgfsetroundjoin%
\definecolor{currentfill}{rgb}{0.121569,0.466667,0.705882}%
\pgfsetfillcolor{currentfill}%
\pgfsetfillopacity{0.721729}%
\pgfsetlinewidth{1.003750pt}%
\definecolor{currentstroke}{rgb}{0.121569,0.466667,0.705882}%
\pgfsetstrokecolor{currentstroke}%
\pgfsetstrokeopacity{0.721729}%
\pgfsetdash{}{0pt}%
\pgfpathmoveto{\pgfqpoint{1.657124in}{1.656577in}}%
\pgfpathcurveto{\pgfqpoint{1.665360in}{1.656577in}}{\pgfqpoint{1.673261in}{1.659850in}}{\pgfqpoint{1.679084in}{1.665674in}}%
\pgfpathcurveto{\pgfqpoint{1.684908in}{1.671498in}}{\pgfqpoint{1.688181in}{1.679398in}}{\pgfqpoint{1.688181in}{1.687634in}}%
\pgfpathcurveto{\pgfqpoint{1.688181in}{1.695870in}}{\pgfqpoint{1.684908in}{1.703770in}}{\pgfqpoint{1.679084in}{1.709594in}}%
\pgfpathcurveto{\pgfqpoint{1.673261in}{1.715418in}}{\pgfqpoint{1.665360in}{1.718690in}}{\pgfqpoint{1.657124in}{1.718690in}}%
\pgfpathcurveto{\pgfqpoint{1.648888in}{1.718690in}}{\pgfqpoint{1.640988in}{1.715418in}}{\pgfqpoint{1.635164in}{1.709594in}}%
\pgfpathcurveto{\pgfqpoint{1.629340in}{1.703770in}}{\pgfqpoint{1.626068in}{1.695870in}}{\pgfqpoint{1.626068in}{1.687634in}}%
\pgfpathcurveto{\pgfqpoint{1.626068in}{1.679398in}}{\pgfqpoint{1.629340in}{1.671498in}}{\pgfqpoint{1.635164in}{1.665674in}}%
\pgfpathcurveto{\pgfqpoint{1.640988in}{1.659850in}}{\pgfqpoint{1.648888in}{1.656577in}}{\pgfqpoint{1.657124in}{1.656577in}}%
\pgfpathclose%
\pgfusepath{stroke,fill}%
\end{pgfscope}%
\begin{pgfscope}%
\pgfpathrectangle{\pgfqpoint{0.100000in}{0.212622in}}{\pgfqpoint{3.696000in}{3.696000in}}%
\pgfusepath{clip}%
\pgfsetbuttcap%
\pgfsetroundjoin%
\definecolor{currentfill}{rgb}{0.121569,0.466667,0.705882}%
\pgfsetfillcolor{currentfill}%
\pgfsetfillopacity{0.722306}%
\pgfsetlinewidth{1.003750pt}%
\definecolor{currentstroke}{rgb}{0.121569,0.466667,0.705882}%
\pgfsetstrokecolor{currentstroke}%
\pgfsetstrokeopacity{0.722306}%
\pgfsetdash{}{0pt}%
\pgfpathmoveto{\pgfqpoint{1.657329in}{1.655745in}}%
\pgfpathcurveto{\pgfqpoint{1.665566in}{1.655745in}}{\pgfqpoint{1.673466in}{1.659017in}}{\pgfqpoint{1.679290in}{1.664841in}}%
\pgfpathcurveto{\pgfqpoint{1.685113in}{1.670665in}}{\pgfqpoint{1.688386in}{1.678565in}}{\pgfqpoint{1.688386in}{1.686801in}}%
\pgfpathcurveto{\pgfqpoint{1.688386in}{1.695037in}}{\pgfqpoint{1.685113in}{1.702937in}}{\pgfqpoint{1.679290in}{1.708761in}}%
\pgfpathcurveto{\pgfqpoint{1.673466in}{1.714585in}}{\pgfqpoint{1.665566in}{1.717858in}}{\pgfqpoint{1.657329in}{1.717858in}}%
\pgfpathcurveto{\pgfqpoint{1.649093in}{1.717858in}}{\pgfqpoint{1.641193in}{1.714585in}}{\pgfqpoint{1.635369in}{1.708761in}}%
\pgfpathcurveto{\pgfqpoint{1.629545in}{1.702937in}}{\pgfqpoint{1.626273in}{1.695037in}}{\pgfqpoint{1.626273in}{1.686801in}}%
\pgfpathcurveto{\pgfqpoint{1.626273in}{1.678565in}}{\pgfqpoint{1.629545in}{1.670665in}}{\pgfqpoint{1.635369in}{1.664841in}}%
\pgfpathcurveto{\pgfqpoint{1.641193in}{1.659017in}}{\pgfqpoint{1.649093in}{1.655745in}}{\pgfqpoint{1.657329in}{1.655745in}}%
\pgfpathclose%
\pgfusepath{stroke,fill}%
\end{pgfscope}%
\begin{pgfscope}%
\pgfpathrectangle{\pgfqpoint{0.100000in}{0.212622in}}{\pgfqpoint{3.696000in}{3.696000in}}%
\pgfusepath{clip}%
\pgfsetbuttcap%
\pgfsetroundjoin%
\definecolor{currentfill}{rgb}{0.121569,0.466667,0.705882}%
\pgfsetfillcolor{currentfill}%
\pgfsetfillopacity{0.723055}%
\pgfsetlinewidth{1.003750pt}%
\definecolor{currentstroke}{rgb}{0.121569,0.466667,0.705882}%
\pgfsetstrokecolor{currentstroke}%
\pgfsetstrokeopacity{0.723055}%
\pgfsetdash{}{0pt}%
\pgfpathmoveto{\pgfqpoint{1.657806in}{1.654357in}}%
\pgfpathcurveto{\pgfqpoint{1.666042in}{1.654357in}}{\pgfqpoint{1.673942in}{1.657629in}}{\pgfqpoint{1.679766in}{1.663453in}}%
\pgfpathcurveto{\pgfqpoint{1.685590in}{1.669277in}}{\pgfqpoint{1.688862in}{1.677177in}}{\pgfqpoint{1.688862in}{1.685413in}}%
\pgfpathcurveto{\pgfqpoint{1.688862in}{1.693649in}}{\pgfqpoint{1.685590in}{1.701549in}}{\pgfqpoint{1.679766in}{1.707373in}}%
\pgfpathcurveto{\pgfqpoint{1.673942in}{1.713197in}}{\pgfqpoint{1.666042in}{1.716470in}}{\pgfqpoint{1.657806in}{1.716470in}}%
\pgfpathcurveto{\pgfqpoint{1.649570in}{1.716470in}}{\pgfqpoint{1.641670in}{1.713197in}}{\pgfqpoint{1.635846in}{1.707373in}}%
\pgfpathcurveto{\pgfqpoint{1.630022in}{1.701549in}}{\pgfqpoint{1.626749in}{1.693649in}}{\pgfqpoint{1.626749in}{1.685413in}}%
\pgfpathcurveto{\pgfqpoint{1.626749in}{1.677177in}}{\pgfqpoint{1.630022in}{1.669277in}}{\pgfqpoint{1.635846in}{1.663453in}}%
\pgfpathcurveto{\pgfqpoint{1.641670in}{1.657629in}}{\pgfqpoint{1.649570in}{1.654357in}}{\pgfqpoint{1.657806in}{1.654357in}}%
\pgfpathclose%
\pgfusepath{stroke,fill}%
\end{pgfscope}%
\begin{pgfscope}%
\pgfpathrectangle{\pgfqpoint{0.100000in}{0.212622in}}{\pgfqpoint{3.696000in}{3.696000in}}%
\pgfusepath{clip}%
\pgfsetbuttcap%
\pgfsetroundjoin%
\definecolor{currentfill}{rgb}{0.121569,0.466667,0.705882}%
\pgfsetfillcolor{currentfill}%
\pgfsetfillopacity{0.723542}%
\pgfsetlinewidth{1.003750pt}%
\definecolor{currentstroke}{rgb}{0.121569,0.466667,0.705882}%
\pgfsetstrokecolor{currentstroke}%
\pgfsetstrokeopacity{0.723542}%
\pgfsetdash{}{0pt}%
\pgfpathmoveto{\pgfqpoint{1.658083in}{1.653939in}}%
\pgfpathcurveto{\pgfqpoint{1.666319in}{1.653939in}}{\pgfqpoint{1.674219in}{1.657212in}}{\pgfqpoint{1.680043in}{1.663036in}}%
\pgfpathcurveto{\pgfqpoint{1.685867in}{1.668860in}}{\pgfqpoint{1.689139in}{1.676760in}}{\pgfqpoint{1.689139in}{1.684996in}}%
\pgfpathcurveto{\pgfqpoint{1.689139in}{1.693232in}}{\pgfqpoint{1.685867in}{1.701132in}}{\pgfqpoint{1.680043in}{1.706956in}}%
\pgfpathcurveto{\pgfqpoint{1.674219in}{1.712780in}}{\pgfqpoint{1.666319in}{1.716052in}}{\pgfqpoint{1.658083in}{1.716052in}}%
\pgfpathcurveto{\pgfqpoint{1.649847in}{1.716052in}}{\pgfqpoint{1.641947in}{1.712780in}}{\pgfqpoint{1.636123in}{1.706956in}}%
\pgfpathcurveto{\pgfqpoint{1.630299in}{1.701132in}}{\pgfqpoint{1.627026in}{1.693232in}}{\pgfqpoint{1.627026in}{1.684996in}}%
\pgfpathcurveto{\pgfqpoint{1.627026in}{1.676760in}}{\pgfqpoint{1.630299in}{1.668860in}}{\pgfqpoint{1.636123in}{1.663036in}}%
\pgfpathcurveto{\pgfqpoint{1.641947in}{1.657212in}}{\pgfqpoint{1.649847in}{1.653939in}}{\pgfqpoint{1.658083in}{1.653939in}}%
\pgfpathclose%
\pgfusepath{stroke,fill}%
\end{pgfscope}%
\begin{pgfscope}%
\pgfpathrectangle{\pgfqpoint{0.100000in}{0.212622in}}{\pgfqpoint{3.696000in}{3.696000in}}%
\pgfusepath{clip}%
\pgfsetbuttcap%
\pgfsetroundjoin%
\definecolor{currentfill}{rgb}{0.121569,0.466667,0.705882}%
\pgfsetfillcolor{currentfill}%
\pgfsetfillopacity{0.724462}%
\pgfsetlinewidth{1.003750pt}%
\definecolor{currentstroke}{rgb}{0.121569,0.466667,0.705882}%
\pgfsetstrokecolor{currentstroke}%
\pgfsetstrokeopacity{0.724462}%
\pgfsetdash{}{0pt}%
\pgfpathmoveto{\pgfqpoint{1.658531in}{1.654506in}}%
\pgfpathcurveto{\pgfqpoint{1.666767in}{1.654506in}}{\pgfqpoint{1.674667in}{1.657778in}}{\pgfqpoint{1.680491in}{1.663602in}}%
\pgfpathcurveto{\pgfqpoint{1.686315in}{1.669426in}}{\pgfqpoint{1.689587in}{1.677326in}}{\pgfqpoint{1.689587in}{1.685562in}}%
\pgfpathcurveto{\pgfqpoint{1.689587in}{1.693799in}}{\pgfqpoint{1.686315in}{1.701699in}}{\pgfqpoint{1.680491in}{1.707523in}}%
\pgfpathcurveto{\pgfqpoint{1.674667in}{1.713346in}}{\pgfqpoint{1.666767in}{1.716619in}}{\pgfqpoint{1.658531in}{1.716619in}}%
\pgfpathcurveto{\pgfqpoint{1.650294in}{1.716619in}}{\pgfqpoint{1.642394in}{1.713346in}}{\pgfqpoint{1.636570in}{1.707523in}}%
\pgfpathcurveto{\pgfqpoint{1.630746in}{1.701699in}}{\pgfqpoint{1.627474in}{1.693799in}}{\pgfqpoint{1.627474in}{1.685562in}}%
\pgfpathcurveto{\pgfqpoint{1.627474in}{1.677326in}}{\pgfqpoint{1.630746in}{1.669426in}}{\pgfqpoint{1.636570in}{1.663602in}}%
\pgfpathcurveto{\pgfqpoint{1.642394in}{1.657778in}}{\pgfqpoint{1.650294in}{1.654506in}}{\pgfqpoint{1.658531in}{1.654506in}}%
\pgfpathclose%
\pgfusepath{stroke,fill}%
\end{pgfscope}%
\begin{pgfscope}%
\pgfpathrectangle{\pgfqpoint{0.100000in}{0.212622in}}{\pgfqpoint{3.696000in}{3.696000in}}%
\pgfusepath{clip}%
\pgfsetbuttcap%
\pgfsetroundjoin%
\definecolor{currentfill}{rgb}{0.121569,0.466667,0.705882}%
\pgfsetfillcolor{currentfill}%
\pgfsetfillopacity{0.724766}%
\pgfsetlinewidth{1.003750pt}%
\definecolor{currentstroke}{rgb}{0.121569,0.466667,0.705882}%
\pgfsetstrokecolor{currentstroke}%
\pgfsetstrokeopacity{0.724766}%
\pgfsetdash{}{0pt}%
\pgfpathmoveto{\pgfqpoint{1.658660in}{1.653853in}}%
\pgfpathcurveto{\pgfqpoint{1.666897in}{1.653853in}}{\pgfqpoint{1.674797in}{1.657125in}}{\pgfqpoint{1.680621in}{1.662949in}}%
\pgfpathcurveto{\pgfqpoint{1.686445in}{1.668773in}}{\pgfqpoint{1.689717in}{1.676673in}}{\pgfqpoint{1.689717in}{1.684909in}}%
\pgfpathcurveto{\pgfqpoint{1.689717in}{1.693145in}}{\pgfqpoint{1.686445in}{1.701045in}}{\pgfqpoint{1.680621in}{1.706869in}}%
\pgfpathcurveto{\pgfqpoint{1.674797in}{1.712693in}}{\pgfqpoint{1.666897in}{1.715966in}}{\pgfqpoint{1.658660in}{1.715966in}}%
\pgfpathcurveto{\pgfqpoint{1.650424in}{1.715966in}}{\pgfqpoint{1.642524in}{1.712693in}}{\pgfqpoint{1.636700in}{1.706869in}}%
\pgfpathcurveto{\pgfqpoint{1.630876in}{1.701045in}}{\pgfqpoint{1.627604in}{1.693145in}}{\pgfqpoint{1.627604in}{1.684909in}}%
\pgfpathcurveto{\pgfqpoint{1.627604in}{1.676673in}}{\pgfqpoint{1.630876in}{1.668773in}}{\pgfqpoint{1.636700in}{1.662949in}}%
\pgfpathcurveto{\pgfqpoint{1.642524in}{1.657125in}}{\pgfqpoint{1.650424in}{1.653853in}}{\pgfqpoint{1.658660in}{1.653853in}}%
\pgfpathclose%
\pgfusepath{stroke,fill}%
\end{pgfscope}%
\begin{pgfscope}%
\pgfpathrectangle{\pgfqpoint{0.100000in}{0.212622in}}{\pgfqpoint{3.696000in}{3.696000in}}%
\pgfusepath{clip}%
\pgfsetbuttcap%
\pgfsetroundjoin%
\definecolor{currentfill}{rgb}{0.121569,0.466667,0.705882}%
\pgfsetfillcolor{currentfill}%
\pgfsetfillopacity{0.725345}%
\pgfsetlinewidth{1.003750pt}%
\definecolor{currentstroke}{rgb}{0.121569,0.466667,0.705882}%
\pgfsetstrokecolor{currentstroke}%
\pgfsetstrokeopacity{0.725345}%
\pgfsetdash{}{0pt}%
\pgfpathmoveto{\pgfqpoint{1.659298in}{1.652554in}}%
\pgfpathcurveto{\pgfqpoint{1.667534in}{1.652554in}}{\pgfqpoint{1.675434in}{1.655826in}}{\pgfqpoint{1.681258in}{1.661650in}}%
\pgfpathcurveto{\pgfqpoint{1.687082in}{1.667474in}}{\pgfqpoint{1.690355in}{1.675374in}}{\pgfqpoint{1.690355in}{1.683610in}}%
\pgfpathcurveto{\pgfqpoint{1.690355in}{1.691846in}}{\pgfqpoint{1.687082in}{1.699746in}}{\pgfqpoint{1.681258in}{1.705570in}}%
\pgfpathcurveto{\pgfqpoint{1.675434in}{1.711394in}}{\pgfqpoint{1.667534in}{1.714667in}}{\pgfqpoint{1.659298in}{1.714667in}}%
\pgfpathcurveto{\pgfqpoint{1.651062in}{1.714667in}}{\pgfqpoint{1.643162in}{1.711394in}}{\pgfqpoint{1.637338in}{1.705570in}}%
\pgfpathcurveto{\pgfqpoint{1.631514in}{1.699746in}}{\pgfqpoint{1.628242in}{1.691846in}}{\pgfqpoint{1.628242in}{1.683610in}}%
\pgfpathcurveto{\pgfqpoint{1.628242in}{1.675374in}}{\pgfqpoint{1.631514in}{1.667474in}}{\pgfqpoint{1.637338in}{1.661650in}}%
\pgfpathcurveto{\pgfqpoint{1.643162in}{1.655826in}}{\pgfqpoint{1.651062in}{1.652554in}}{\pgfqpoint{1.659298in}{1.652554in}}%
\pgfpathclose%
\pgfusepath{stroke,fill}%
\end{pgfscope}%
\begin{pgfscope}%
\pgfpathrectangle{\pgfqpoint{0.100000in}{0.212622in}}{\pgfqpoint{3.696000in}{3.696000in}}%
\pgfusepath{clip}%
\pgfsetbuttcap%
\pgfsetroundjoin%
\definecolor{currentfill}{rgb}{0.121569,0.466667,0.705882}%
\pgfsetfillcolor{currentfill}%
\pgfsetfillopacity{0.725812}%
\pgfsetlinewidth{1.003750pt}%
\definecolor{currentstroke}{rgb}{0.121569,0.466667,0.705882}%
\pgfsetstrokecolor{currentstroke}%
\pgfsetstrokeopacity{0.725812}%
\pgfsetdash{}{0pt}%
\pgfpathmoveto{\pgfqpoint{1.659503in}{1.652433in}}%
\pgfpathcurveto{\pgfqpoint{1.667739in}{1.652433in}}{\pgfqpoint{1.675639in}{1.655705in}}{\pgfqpoint{1.681463in}{1.661529in}}%
\pgfpathcurveto{\pgfqpoint{1.687287in}{1.667353in}}{\pgfqpoint{1.690559in}{1.675253in}}{\pgfqpoint{1.690559in}{1.683490in}}%
\pgfpathcurveto{\pgfqpoint{1.690559in}{1.691726in}}{\pgfqpoint{1.687287in}{1.699626in}}{\pgfqpoint{1.681463in}{1.705450in}}%
\pgfpathcurveto{\pgfqpoint{1.675639in}{1.711274in}}{\pgfqpoint{1.667739in}{1.714546in}}{\pgfqpoint{1.659503in}{1.714546in}}%
\pgfpathcurveto{\pgfqpoint{1.651266in}{1.714546in}}{\pgfqpoint{1.643366in}{1.711274in}}{\pgfqpoint{1.637542in}{1.705450in}}%
\pgfpathcurveto{\pgfqpoint{1.631718in}{1.699626in}}{\pgfqpoint{1.628446in}{1.691726in}}{\pgfqpoint{1.628446in}{1.683490in}}%
\pgfpathcurveto{\pgfqpoint{1.628446in}{1.675253in}}{\pgfqpoint{1.631718in}{1.667353in}}{\pgfqpoint{1.637542in}{1.661529in}}%
\pgfpathcurveto{\pgfqpoint{1.643366in}{1.655705in}}{\pgfqpoint{1.651266in}{1.652433in}}{\pgfqpoint{1.659503in}{1.652433in}}%
\pgfpathclose%
\pgfusepath{stroke,fill}%
\end{pgfscope}%
\begin{pgfscope}%
\pgfpathrectangle{\pgfqpoint{0.100000in}{0.212622in}}{\pgfqpoint{3.696000in}{3.696000in}}%
\pgfusepath{clip}%
\pgfsetbuttcap%
\pgfsetroundjoin%
\definecolor{currentfill}{rgb}{0.121569,0.466667,0.705882}%
\pgfsetfillcolor{currentfill}%
\pgfsetfillopacity{0.727062}%
\pgfsetlinewidth{1.003750pt}%
\definecolor{currentstroke}{rgb}{0.121569,0.466667,0.705882}%
\pgfsetstrokecolor{currentstroke}%
\pgfsetstrokeopacity{0.727062}%
\pgfsetdash{}{0pt}%
\pgfpathmoveto{\pgfqpoint{1.660141in}{1.653361in}}%
\pgfpathcurveto{\pgfqpoint{1.668377in}{1.653361in}}{\pgfqpoint{1.676277in}{1.656633in}}{\pgfqpoint{1.682101in}{1.662457in}}%
\pgfpathcurveto{\pgfqpoint{1.687925in}{1.668281in}}{\pgfqpoint{1.691198in}{1.676181in}}{\pgfqpoint{1.691198in}{1.684417in}}%
\pgfpathcurveto{\pgfqpoint{1.691198in}{1.692653in}}{\pgfqpoint{1.687925in}{1.700554in}}{\pgfqpoint{1.682101in}{1.706377in}}%
\pgfpathcurveto{\pgfqpoint{1.676277in}{1.712201in}}{\pgfqpoint{1.668377in}{1.715474in}}{\pgfqpoint{1.660141in}{1.715474in}}%
\pgfpathcurveto{\pgfqpoint{1.651905in}{1.715474in}}{\pgfqpoint{1.644005in}{1.712201in}}{\pgfqpoint{1.638181in}{1.706377in}}%
\pgfpathcurveto{\pgfqpoint{1.632357in}{1.700554in}}{\pgfqpoint{1.629085in}{1.692653in}}{\pgfqpoint{1.629085in}{1.684417in}}%
\pgfpathcurveto{\pgfqpoint{1.629085in}{1.676181in}}{\pgfqpoint{1.632357in}{1.668281in}}{\pgfqpoint{1.638181in}{1.662457in}}%
\pgfpathcurveto{\pgfqpoint{1.644005in}{1.656633in}}{\pgfqpoint{1.651905in}{1.653361in}}{\pgfqpoint{1.660141in}{1.653361in}}%
\pgfpathclose%
\pgfusepath{stroke,fill}%
\end{pgfscope}%
\begin{pgfscope}%
\pgfpathrectangle{\pgfqpoint{0.100000in}{0.212622in}}{\pgfqpoint{3.696000in}{3.696000in}}%
\pgfusepath{clip}%
\pgfsetbuttcap%
\pgfsetroundjoin%
\definecolor{currentfill}{rgb}{0.121569,0.466667,0.705882}%
\pgfsetfillcolor{currentfill}%
\pgfsetfillopacity{0.728053}%
\pgfsetlinewidth{1.003750pt}%
\definecolor{currentstroke}{rgb}{0.121569,0.466667,0.705882}%
\pgfsetstrokecolor{currentstroke}%
\pgfsetstrokeopacity{0.728053}%
\pgfsetdash{}{0pt}%
\pgfpathmoveto{\pgfqpoint{1.660661in}{1.652182in}}%
\pgfpathcurveto{\pgfqpoint{1.668897in}{1.652182in}}{\pgfqpoint{1.676797in}{1.655454in}}{\pgfqpoint{1.682621in}{1.661278in}}%
\pgfpathcurveto{\pgfqpoint{1.688445in}{1.667102in}}{\pgfqpoint{1.691717in}{1.675002in}}{\pgfqpoint{1.691717in}{1.683238in}}%
\pgfpathcurveto{\pgfqpoint{1.691717in}{1.691475in}}{\pgfqpoint{1.688445in}{1.699375in}}{\pgfqpoint{1.682621in}{1.705199in}}%
\pgfpathcurveto{\pgfqpoint{1.676797in}{1.711023in}}{\pgfqpoint{1.668897in}{1.714295in}}{\pgfqpoint{1.660661in}{1.714295in}}%
\pgfpathcurveto{\pgfqpoint{1.652424in}{1.714295in}}{\pgfqpoint{1.644524in}{1.711023in}}{\pgfqpoint{1.638700in}{1.705199in}}%
\pgfpathcurveto{\pgfqpoint{1.632876in}{1.699375in}}{\pgfqpoint{1.629604in}{1.691475in}}{\pgfqpoint{1.629604in}{1.683238in}}%
\pgfpathcurveto{\pgfqpoint{1.629604in}{1.675002in}}{\pgfqpoint{1.632876in}{1.667102in}}{\pgfqpoint{1.638700in}{1.661278in}}%
\pgfpathcurveto{\pgfqpoint{1.644524in}{1.655454in}}{\pgfqpoint{1.652424in}{1.652182in}}{\pgfqpoint{1.660661in}{1.652182in}}%
\pgfpathclose%
\pgfusepath{stroke,fill}%
\end{pgfscope}%
\begin{pgfscope}%
\pgfpathrectangle{\pgfqpoint{0.100000in}{0.212622in}}{\pgfqpoint{3.696000in}{3.696000in}}%
\pgfusepath{clip}%
\pgfsetbuttcap%
\pgfsetroundjoin%
\definecolor{currentfill}{rgb}{0.121569,0.466667,0.705882}%
\pgfsetfillcolor{currentfill}%
\pgfsetfillopacity{0.729827}%
\pgfsetlinewidth{1.003750pt}%
\definecolor{currentstroke}{rgb}{0.121569,0.466667,0.705882}%
\pgfsetstrokecolor{currentstroke}%
\pgfsetstrokeopacity{0.729827}%
\pgfsetdash{}{0pt}%
\pgfpathmoveto{\pgfqpoint{1.661537in}{1.650291in}}%
\pgfpathcurveto{\pgfqpoint{1.669774in}{1.650291in}}{\pgfqpoint{1.677674in}{1.653564in}}{\pgfqpoint{1.683498in}{1.659388in}}%
\pgfpathcurveto{\pgfqpoint{1.689322in}{1.665212in}}{\pgfqpoint{1.692594in}{1.673112in}}{\pgfqpoint{1.692594in}{1.681348in}}%
\pgfpathcurveto{\pgfqpoint{1.692594in}{1.689584in}}{\pgfqpoint{1.689322in}{1.697484in}}{\pgfqpoint{1.683498in}{1.703308in}}%
\pgfpathcurveto{\pgfqpoint{1.677674in}{1.709132in}}{\pgfqpoint{1.669774in}{1.712404in}}{\pgfqpoint{1.661537in}{1.712404in}}%
\pgfpathcurveto{\pgfqpoint{1.653301in}{1.712404in}}{\pgfqpoint{1.645401in}{1.709132in}}{\pgfqpoint{1.639577in}{1.703308in}}%
\pgfpathcurveto{\pgfqpoint{1.633753in}{1.697484in}}{\pgfqpoint{1.630481in}{1.689584in}}{\pgfqpoint{1.630481in}{1.681348in}}%
\pgfpathcurveto{\pgfqpoint{1.630481in}{1.673112in}}{\pgfqpoint{1.633753in}{1.665212in}}{\pgfqpoint{1.639577in}{1.659388in}}%
\pgfpathcurveto{\pgfqpoint{1.645401in}{1.653564in}}{\pgfqpoint{1.653301in}{1.650291in}}{\pgfqpoint{1.661537in}{1.650291in}}%
\pgfpathclose%
\pgfusepath{stroke,fill}%
\end{pgfscope}%
\begin{pgfscope}%
\pgfpathrectangle{\pgfqpoint{0.100000in}{0.212622in}}{\pgfqpoint{3.696000in}{3.696000in}}%
\pgfusepath{clip}%
\pgfsetbuttcap%
\pgfsetroundjoin%
\definecolor{currentfill}{rgb}{0.121569,0.466667,0.705882}%
\pgfsetfillcolor{currentfill}%
\pgfsetfillopacity{0.731250}%
\pgfsetlinewidth{1.003750pt}%
\definecolor{currentstroke}{rgb}{0.121569,0.466667,0.705882}%
\pgfsetstrokecolor{currentstroke}%
\pgfsetstrokeopacity{0.731250}%
\pgfsetdash{}{0pt}%
\pgfpathmoveto{\pgfqpoint{1.662222in}{1.651376in}}%
\pgfpathcurveto{\pgfqpoint{1.670458in}{1.651376in}}{\pgfqpoint{1.678358in}{1.654648in}}{\pgfqpoint{1.684182in}{1.660472in}}%
\pgfpathcurveto{\pgfqpoint{1.690006in}{1.666296in}}{\pgfqpoint{1.693279in}{1.674196in}}{\pgfqpoint{1.693279in}{1.682432in}}%
\pgfpathcurveto{\pgfqpoint{1.693279in}{1.690669in}}{\pgfqpoint{1.690006in}{1.698569in}}{\pgfqpoint{1.684182in}{1.704393in}}%
\pgfpathcurveto{\pgfqpoint{1.678358in}{1.710216in}}{\pgfqpoint{1.670458in}{1.713489in}}{\pgfqpoint{1.662222in}{1.713489in}}%
\pgfpathcurveto{\pgfqpoint{1.653986in}{1.713489in}}{\pgfqpoint{1.646086in}{1.710216in}}{\pgfqpoint{1.640262in}{1.704393in}}%
\pgfpathcurveto{\pgfqpoint{1.634438in}{1.698569in}}{\pgfqpoint{1.631166in}{1.690669in}}{\pgfqpoint{1.631166in}{1.682432in}}%
\pgfpathcurveto{\pgfqpoint{1.631166in}{1.674196in}}{\pgfqpoint{1.634438in}{1.666296in}}{\pgfqpoint{1.640262in}{1.660472in}}%
\pgfpathcurveto{\pgfqpoint{1.646086in}{1.654648in}}{\pgfqpoint{1.653986in}{1.651376in}}{\pgfqpoint{1.662222in}{1.651376in}}%
\pgfpathclose%
\pgfusepath{stroke,fill}%
\end{pgfscope}%
\begin{pgfscope}%
\pgfpathrectangle{\pgfqpoint{0.100000in}{0.212622in}}{\pgfqpoint{3.696000in}{3.696000in}}%
\pgfusepath{clip}%
\pgfsetbuttcap%
\pgfsetroundjoin%
\definecolor{currentfill}{rgb}{0.121569,0.466667,0.705882}%
\pgfsetfillcolor{currentfill}%
\pgfsetfillopacity{0.733249}%
\pgfsetlinewidth{1.003750pt}%
\definecolor{currentstroke}{rgb}{0.121569,0.466667,0.705882}%
\pgfsetstrokecolor{currentstroke}%
\pgfsetstrokeopacity{0.733249}%
\pgfsetdash{}{0pt}%
\pgfpathmoveto{\pgfqpoint{1.663186in}{1.650961in}}%
\pgfpathcurveto{\pgfqpoint{1.671422in}{1.650961in}}{\pgfqpoint{1.679322in}{1.654233in}}{\pgfqpoint{1.685146in}{1.660057in}}%
\pgfpathcurveto{\pgfqpoint{1.690970in}{1.665881in}}{\pgfqpoint{1.694242in}{1.673781in}}{\pgfqpoint{1.694242in}{1.682017in}}%
\pgfpathcurveto{\pgfqpoint{1.694242in}{1.690254in}}{\pgfqpoint{1.690970in}{1.698154in}}{\pgfqpoint{1.685146in}{1.703978in}}%
\pgfpathcurveto{\pgfqpoint{1.679322in}{1.709802in}}{\pgfqpoint{1.671422in}{1.713074in}}{\pgfqpoint{1.663186in}{1.713074in}}%
\pgfpathcurveto{\pgfqpoint{1.654949in}{1.713074in}}{\pgfqpoint{1.647049in}{1.709802in}}{\pgfqpoint{1.641225in}{1.703978in}}%
\pgfpathcurveto{\pgfqpoint{1.635401in}{1.698154in}}{\pgfqpoint{1.632129in}{1.690254in}}{\pgfqpoint{1.632129in}{1.682017in}}%
\pgfpathcurveto{\pgfqpoint{1.632129in}{1.673781in}}{\pgfqpoint{1.635401in}{1.665881in}}{\pgfqpoint{1.641225in}{1.660057in}}%
\pgfpathcurveto{\pgfqpoint{1.647049in}{1.654233in}}{\pgfqpoint{1.654949in}{1.650961in}}{\pgfqpoint{1.663186in}{1.650961in}}%
\pgfpathclose%
\pgfusepath{stroke,fill}%
\end{pgfscope}%
\begin{pgfscope}%
\pgfpathrectangle{\pgfqpoint{0.100000in}{0.212622in}}{\pgfqpoint{3.696000in}{3.696000in}}%
\pgfusepath{clip}%
\pgfsetbuttcap%
\pgfsetroundjoin%
\definecolor{currentfill}{rgb}{0.121569,0.466667,0.705882}%
\pgfsetfillcolor{currentfill}%
\pgfsetfillopacity{0.735029}%
\pgfsetlinewidth{1.003750pt}%
\definecolor{currentstroke}{rgb}{0.121569,0.466667,0.705882}%
\pgfsetstrokecolor{currentstroke}%
\pgfsetstrokeopacity{0.735029}%
\pgfsetdash{}{0pt}%
\pgfpathmoveto{\pgfqpoint{1.664286in}{1.647628in}}%
\pgfpathcurveto{\pgfqpoint{1.672522in}{1.647628in}}{\pgfqpoint{1.680422in}{1.650901in}}{\pgfqpoint{1.686246in}{1.656725in}}%
\pgfpathcurveto{\pgfqpoint{1.692070in}{1.662549in}}{\pgfqpoint{1.695342in}{1.670449in}}{\pgfqpoint{1.695342in}{1.678685in}}%
\pgfpathcurveto{\pgfqpoint{1.695342in}{1.686921in}}{\pgfqpoint{1.692070in}{1.694821in}}{\pgfqpoint{1.686246in}{1.700645in}}%
\pgfpathcurveto{\pgfqpoint{1.680422in}{1.706469in}}{\pgfqpoint{1.672522in}{1.709741in}}{\pgfqpoint{1.664286in}{1.709741in}}%
\pgfpathcurveto{\pgfqpoint{1.656050in}{1.709741in}}{\pgfqpoint{1.648149in}{1.706469in}}{\pgfqpoint{1.642326in}{1.700645in}}%
\pgfpathcurveto{\pgfqpoint{1.636502in}{1.694821in}}{\pgfqpoint{1.633229in}{1.686921in}}{\pgfqpoint{1.633229in}{1.678685in}}%
\pgfpathcurveto{\pgfqpoint{1.633229in}{1.670449in}}{\pgfqpoint{1.636502in}{1.662549in}}{\pgfqpoint{1.642326in}{1.656725in}}%
\pgfpathcurveto{\pgfqpoint{1.648149in}{1.650901in}}{\pgfqpoint{1.656050in}{1.647628in}}{\pgfqpoint{1.664286in}{1.647628in}}%
\pgfpathclose%
\pgfusepath{stroke,fill}%
\end{pgfscope}%
\begin{pgfscope}%
\pgfpathrectangle{\pgfqpoint{0.100000in}{0.212622in}}{\pgfqpoint{3.696000in}{3.696000in}}%
\pgfusepath{clip}%
\pgfsetbuttcap%
\pgfsetroundjoin%
\definecolor{currentfill}{rgb}{0.121569,0.466667,0.705882}%
\pgfsetfillcolor{currentfill}%
\pgfsetfillopacity{0.737387}%
\pgfsetlinewidth{1.003750pt}%
\definecolor{currentstroke}{rgb}{0.121569,0.466667,0.705882}%
\pgfsetstrokecolor{currentstroke}%
\pgfsetstrokeopacity{0.737387}%
\pgfsetdash{}{0pt}%
\pgfpathmoveto{\pgfqpoint{1.665552in}{1.644359in}}%
\pgfpathcurveto{\pgfqpoint{1.673788in}{1.644359in}}{\pgfqpoint{1.681688in}{1.647632in}}{\pgfqpoint{1.687512in}{1.653456in}}%
\pgfpathcurveto{\pgfqpoint{1.693336in}{1.659279in}}{\pgfqpoint{1.696608in}{1.667180in}}{\pgfqpoint{1.696608in}{1.675416in}}%
\pgfpathcurveto{\pgfqpoint{1.696608in}{1.683652in}}{\pgfqpoint{1.693336in}{1.691552in}}{\pgfqpoint{1.687512in}{1.697376in}}%
\pgfpathcurveto{\pgfqpoint{1.681688in}{1.703200in}}{\pgfqpoint{1.673788in}{1.706472in}}{\pgfqpoint{1.665552in}{1.706472in}}%
\pgfpathcurveto{\pgfqpoint{1.657316in}{1.706472in}}{\pgfqpoint{1.649416in}{1.703200in}}{\pgfqpoint{1.643592in}{1.697376in}}%
\pgfpathcurveto{\pgfqpoint{1.637768in}{1.691552in}}{\pgfqpoint{1.634495in}{1.683652in}}{\pgfqpoint{1.634495in}{1.675416in}}%
\pgfpathcurveto{\pgfqpoint{1.634495in}{1.667180in}}{\pgfqpoint{1.637768in}{1.659279in}}{\pgfqpoint{1.643592in}{1.653456in}}%
\pgfpathcurveto{\pgfqpoint{1.649416in}{1.647632in}}{\pgfqpoint{1.657316in}{1.644359in}}{\pgfqpoint{1.665552in}{1.644359in}}%
\pgfpathclose%
\pgfusepath{stroke,fill}%
\end{pgfscope}%
\begin{pgfscope}%
\pgfpathrectangle{\pgfqpoint{0.100000in}{0.212622in}}{\pgfqpoint{3.696000in}{3.696000in}}%
\pgfusepath{clip}%
\pgfsetbuttcap%
\pgfsetroundjoin%
\definecolor{currentfill}{rgb}{0.121569,0.466667,0.705882}%
\pgfsetfillcolor{currentfill}%
\pgfsetfillopacity{0.741477}%
\pgfsetlinewidth{1.003750pt}%
\definecolor{currentstroke}{rgb}{0.121569,0.466667,0.705882}%
\pgfsetstrokecolor{currentstroke}%
\pgfsetstrokeopacity{0.741477}%
\pgfsetdash{}{0pt}%
\pgfpathmoveto{\pgfqpoint{1.667842in}{1.648732in}}%
\pgfpathcurveto{\pgfqpoint{1.676079in}{1.648732in}}{\pgfqpoint{1.683979in}{1.652004in}}{\pgfqpoint{1.689803in}{1.657828in}}%
\pgfpathcurveto{\pgfqpoint{1.695627in}{1.663652in}}{\pgfqpoint{1.698899in}{1.671552in}}{\pgfqpoint{1.698899in}{1.679788in}}%
\pgfpathcurveto{\pgfqpoint{1.698899in}{1.688025in}}{\pgfqpoint{1.695627in}{1.695925in}}{\pgfqpoint{1.689803in}{1.701749in}}%
\pgfpathcurveto{\pgfqpoint{1.683979in}{1.707572in}}{\pgfqpoint{1.676079in}{1.710845in}}{\pgfqpoint{1.667842in}{1.710845in}}%
\pgfpathcurveto{\pgfqpoint{1.659606in}{1.710845in}}{\pgfqpoint{1.651706in}{1.707572in}}{\pgfqpoint{1.645882in}{1.701749in}}%
\pgfpathcurveto{\pgfqpoint{1.640058in}{1.695925in}}{\pgfqpoint{1.636786in}{1.688025in}}{\pgfqpoint{1.636786in}{1.679788in}}%
\pgfpathcurveto{\pgfqpoint{1.636786in}{1.671552in}}{\pgfqpoint{1.640058in}{1.663652in}}{\pgfqpoint{1.645882in}{1.657828in}}%
\pgfpathcurveto{\pgfqpoint{1.651706in}{1.652004in}}{\pgfqpoint{1.659606in}{1.648732in}}{\pgfqpoint{1.667842in}{1.648732in}}%
\pgfpathclose%
\pgfusepath{stroke,fill}%
\end{pgfscope}%
\begin{pgfscope}%
\pgfpathrectangle{\pgfqpoint{0.100000in}{0.212622in}}{\pgfqpoint{3.696000in}{3.696000in}}%
\pgfusepath{clip}%
\pgfsetbuttcap%
\pgfsetroundjoin%
\definecolor{currentfill}{rgb}{0.121569,0.466667,0.705882}%
\pgfsetfillcolor{currentfill}%
\pgfsetfillopacity{0.743016}%
\pgfsetlinewidth{1.003750pt}%
\definecolor{currentstroke}{rgb}{0.121569,0.466667,0.705882}%
\pgfsetstrokecolor{currentstroke}%
\pgfsetstrokeopacity{0.743016}%
\pgfsetdash{}{0pt}%
\pgfpathmoveto{\pgfqpoint{1.668188in}{1.647503in}}%
\pgfpathcurveto{\pgfqpoint{1.676424in}{1.647503in}}{\pgfqpoint{1.684324in}{1.650775in}}{\pgfqpoint{1.690148in}{1.656599in}}%
\pgfpathcurveto{\pgfqpoint{1.695972in}{1.662423in}}{\pgfqpoint{1.699244in}{1.670323in}}{\pgfqpoint{1.699244in}{1.678559in}}%
\pgfpathcurveto{\pgfqpoint{1.699244in}{1.686796in}}{\pgfqpoint{1.695972in}{1.694696in}}{\pgfqpoint{1.690148in}{1.700520in}}%
\pgfpathcurveto{\pgfqpoint{1.684324in}{1.706343in}}{\pgfqpoint{1.676424in}{1.709616in}}{\pgfqpoint{1.668188in}{1.709616in}}%
\pgfpathcurveto{\pgfqpoint{1.659951in}{1.709616in}}{\pgfqpoint{1.652051in}{1.706343in}}{\pgfqpoint{1.646227in}{1.700520in}}%
\pgfpathcurveto{\pgfqpoint{1.640404in}{1.694696in}}{\pgfqpoint{1.637131in}{1.686796in}}{\pgfqpoint{1.637131in}{1.678559in}}%
\pgfpathcurveto{\pgfqpoint{1.637131in}{1.670323in}}{\pgfqpoint{1.640404in}{1.662423in}}{\pgfqpoint{1.646227in}{1.656599in}}%
\pgfpathcurveto{\pgfqpoint{1.652051in}{1.650775in}}{\pgfqpoint{1.659951in}{1.647503in}}{\pgfqpoint{1.668188in}{1.647503in}}%
\pgfpathclose%
\pgfusepath{stroke,fill}%
\end{pgfscope}%
\begin{pgfscope}%
\pgfpathrectangle{\pgfqpoint{0.100000in}{0.212622in}}{\pgfqpoint{3.696000in}{3.696000in}}%
\pgfusepath{clip}%
\pgfsetbuttcap%
\pgfsetroundjoin%
\definecolor{currentfill}{rgb}{0.121569,0.466667,0.705882}%
\pgfsetfillcolor{currentfill}%
\pgfsetfillopacity{0.744428}%
\pgfsetlinewidth{1.003750pt}%
\definecolor{currentstroke}{rgb}{0.121569,0.466667,0.705882}%
\pgfsetstrokecolor{currentstroke}%
\pgfsetstrokeopacity{0.744428}%
\pgfsetdash{}{0pt}%
\pgfpathmoveto{\pgfqpoint{1.669251in}{1.644383in}}%
\pgfpathcurveto{\pgfqpoint{1.677487in}{1.644383in}}{\pgfqpoint{1.685387in}{1.647655in}}{\pgfqpoint{1.691211in}{1.653479in}}%
\pgfpathcurveto{\pgfqpoint{1.697035in}{1.659303in}}{\pgfqpoint{1.700307in}{1.667203in}}{\pgfqpoint{1.700307in}{1.675439in}}%
\pgfpathcurveto{\pgfqpoint{1.700307in}{1.683676in}}{\pgfqpoint{1.697035in}{1.691576in}}{\pgfqpoint{1.691211in}{1.697400in}}%
\pgfpathcurveto{\pgfqpoint{1.685387in}{1.703224in}}{\pgfqpoint{1.677487in}{1.706496in}}{\pgfqpoint{1.669251in}{1.706496in}}%
\pgfpathcurveto{\pgfqpoint{1.661015in}{1.706496in}}{\pgfqpoint{1.653114in}{1.703224in}}{\pgfqpoint{1.647291in}{1.697400in}}%
\pgfpathcurveto{\pgfqpoint{1.641467in}{1.691576in}}{\pgfqpoint{1.638194in}{1.683676in}}{\pgfqpoint{1.638194in}{1.675439in}}%
\pgfpathcurveto{\pgfqpoint{1.638194in}{1.667203in}}{\pgfqpoint{1.641467in}{1.659303in}}{\pgfqpoint{1.647291in}{1.653479in}}%
\pgfpathcurveto{\pgfqpoint{1.653114in}{1.647655in}}{\pgfqpoint{1.661015in}{1.644383in}}{\pgfqpoint{1.669251in}{1.644383in}}%
\pgfpathclose%
\pgfusepath{stroke,fill}%
\end{pgfscope}%
\begin{pgfscope}%
\pgfpathrectangle{\pgfqpoint{0.100000in}{0.212622in}}{\pgfqpoint{3.696000in}{3.696000in}}%
\pgfusepath{clip}%
\pgfsetbuttcap%
\pgfsetroundjoin%
\definecolor{currentfill}{rgb}{0.121569,0.466667,0.705882}%
\pgfsetfillcolor{currentfill}%
\pgfsetfillopacity{0.745498}%
\pgfsetlinewidth{1.003750pt}%
\definecolor{currentstroke}{rgb}{0.121569,0.466667,0.705882}%
\pgfsetstrokecolor{currentstroke}%
\pgfsetstrokeopacity{0.745498}%
\pgfsetdash{}{0pt}%
\pgfpathmoveto{\pgfqpoint{1.669718in}{1.643941in}}%
\pgfpathcurveto{\pgfqpoint{1.677954in}{1.643941in}}{\pgfqpoint{1.685854in}{1.647213in}}{\pgfqpoint{1.691678in}{1.653037in}}%
\pgfpathcurveto{\pgfqpoint{1.697502in}{1.658861in}}{\pgfqpoint{1.700774in}{1.666761in}}{\pgfqpoint{1.700774in}{1.674997in}}%
\pgfpathcurveto{\pgfqpoint{1.700774in}{1.683234in}}{\pgfqpoint{1.697502in}{1.691134in}}{\pgfqpoint{1.691678in}{1.696957in}}%
\pgfpathcurveto{\pgfqpoint{1.685854in}{1.702781in}}{\pgfqpoint{1.677954in}{1.706054in}}{\pgfqpoint{1.669718in}{1.706054in}}%
\pgfpathcurveto{\pgfqpoint{1.661481in}{1.706054in}}{\pgfqpoint{1.653581in}{1.702781in}}{\pgfqpoint{1.647758in}{1.696957in}}%
\pgfpathcurveto{\pgfqpoint{1.641934in}{1.691134in}}{\pgfqpoint{1.638661in}{1.683234in}}{\pgfqpoint{1.638661in}{1.674997in}}%
\pgfpathcurveto{\pgfqpoint{1.638661in}{1.666761in}}{\pgfqpoint{1.641934in}{1.658861in}}{\pgfqpoint{1.647758in}{1.653037in}}%
\pgfpathcurveto{\pgfqpoint{1.653581in}{1.647213in}}{\pgfqpoint{1.661481in}{1.643941in}}{\pgfqpoint{1.669718in}{1.643941in}}%
\pgfpathclose%
\pgfusepath{stroke,fill}%
\end{pgfscope}%
\begin{pgfscope}%
\pgfpathrectangle{\pgfqpoint{0.100000in}{0.212622in}}{\pgfqpoint{3.696000in}{3.696000in}}%
\pgfusepath{clip}%
\pgfsetbuttcap%
\pgfsetroundjoin%
\definecolor{currentfill}{rgb}{0.121569,0.466667,0.705882}%
\pgfsetfillcolor{currentfill}%
\pgfsetfillopacity{0.747362}%
\pgfsetlinewidth{1.003750pt}%
\definecolor{currentstroke}{rgb}{0.121569,0.466667,0.705882}%
\pgfsetstrokecolor{currentstroke}%
\pgfsetstrokeopacity{0.747362}%
\pgfsetdash{}{0pt}%
\pgfpathmoveto{\pgfqpoint{1.670721in}{1.645770in}}%
\pgfpathcurveto{\pgfqpoint{1.678957in}{1.645770in}}{\pgfqpoint{1.686857in}{1.649042in}}{\pgfqpoint{1.692681in}{1.654866in}}%
\pgfpathcurveto{\pgfqpoint{1.698505in}{1.660690in}}{\pgfqpoint{1.701777in}{1.668590in}}{\pgfqpoint{1.701777in}{1.676827in}}%
\pgfpathcurveto{\pgfqpoint{1.701777in}{1.685063in}}{\pgfqpoint{1.698505in}{1.692963in}}{\pgfqpoint{1.692681in}{1.698787in}}%
\pgfpathcurveto{\pgfqpoint{1.686857in}{1.704611in}}{\pgfqpoint{1.678957in}{1.707883in}}{\pgfqpoint{1.670721in}{1.707883in}}%
\pgfpathcurveto{\pgfqpoint{1.662484in}{1.707883in}}{\pgfqpoint{1.654584in}{1.704611in}}{\pgfqpoint{1.648760in}{1.698787in}}%
\pgfpathcurveto{\pgfqpoint{1.642937in}{1.692963in}}{\pgfqpoint{1.639664in}{1.685063in}}{\pgfqpoint{1.639664in}{1.676827in}}%
\pgfpathcurveto{\pgfqpoint{1.639664in}{1.668590in}}{\pgfqpoint{1.642937in}{1.660690in}}{\pgfqpoint{1.648760in}{1.654866in}}%
\pgfpathcurveto{\pgfqpoint{1.654584in}{1.649042in}}{\pgfqpoint{1.662484in}{1.645770in}}{\pgfqpoint{1.670721in}{1.645770in}}%
\pgfpathclose%
\pgfusepath{stroke,fill}%
\end{pgfscope}%
\begin{pgfscope}%
\pgfpathrectangle{\pgfqpoint{0.100000in}{0.212622in}}{\pgfqpoint{3.696000in}{3.696000in}}%
\pgfusepath{clip}%
\pgfsetbuttcap%
\pgfsetroundjoin%
\definecolor{currentfill}{rgb}{0.121569,0.466667,0.705882}%
\pgfsetfillcolor{currentfill}%
\pgfsetfillopacity{0.748004}%
\pgfsetlinewidth{1.003750pt}%
\definecolor{currentstroke}{rgb}{0.121569,0.466667,0.705882}%
\pgfsetstrokecolor{currentstroke}%
\pgfsetstrokeopacity{0.748004}%
\pgfsetdash{}{0pt}%
\pgfpathmoveto{\pgfqpoint{1.670919in}{1.644884in}}%
\pgfpathcurveto{\pgfqpoint{1.679155in}{1.644884in}}{\pgfqpoint{1.687055in}{1.648157in}}{\pgfqpoint{1.692879in}{1.653981in}}%
\pgfpathcurveto{\pgfqpoint{1.698703in}{1.659805in}}{\pgfqpoint{1.701975in}{1.667705in}}{\pgfqpoint{1.701975in}{1.675941in}}%
\pgfpathcurveto{\pgfqpoint{1.701975in}{1.684177in}}{\pgfqpoint{1.698703in}{1.692077in}}{\pgfqpoint{1.692879in}{1.697901in}}%
\pgfpathcurveto{\pgfqpoint{1.687055in}{1.703725in}}{\pgfqpoint{1.679155in}{1.706997in}}{\pgfqpoint{1.670919in}{1.706997in}}%
\pgfpathcurveto{\pgfqpoint{1.662683in}{1.706997in}}{\pgfqpoint{1.654783in}{1.703725in}}{\pgfqpoint{1.648959in}{1.697901in}}%
\pgfpathcurveto{\pgfqpoint{1.643135in}{1.692077in}}{\pgfqpoint{1.639862in}{1.684177in}}{\pgfqpoint{1.639862in}{1.675941in}}%
\pgfpathcurveto{\pgfqpoint{1.639862in}{1.667705in}}{\pgfqpoint{1.643135in}{1.659805in}}{\pgfqpoint{1.648959in}{1.653981in}}%
\pgfpathcurveto{\pgfqpoint{1.654783in}{1.648157in}}{\pgfqpoint{1.662683in}{1.644884in}}{\pgfqpoint{1.670919in}{1.644884in}}%
\pgfpathclose%
\pgfusepath{stroke,fill}%
\end{pgfscope}%
\begin{pgfscope}%
\pgfpathrectangle{\pgfqpoint{0.100000in}{0.212622in}}{\pgfqpoint{3.696000in}{3.696000in}}%
\pgfusepath{clip}%
\pgfsetbuttcap%
\pgfsetroundjoin%
\definecolor{currentfill}{rgb}{0.121569,0.466667,0.705882}%
\pgfsetfillcolor{currentfill}%
\pgfsetfillopacity{0.748898}%
\pgfsetlinewidth{1.003750pt}%
\definecolor{currentstroke}{rgb}{0.121569,0.466667,0.705882}%
\pgfsetstrokecolor{currentstroke}%
\pgfsetstrokeopacity{0.748898}%
\pgfsetdash{}{0pt}%
\pgfpathmoveto{\pgfqpoint{1.671831in}{1.643096in}}%
\pgfpathcurveto{\pgfqpoint{1.680067in}{1.643096in}}{\pgfqpoint{1.687967in}{1.646369in}}{\pgfqpoint{1.693791in}{1.652193in}}%
\pgfpathcurveto{\pgfqpoint{1.699615in}{1.658017in}}{\pgfqpoint{1.702888in}{1.665917in}}{\pgfqpoint{1.702888in}{1.674153in}}%
\pgfpathcurveto{\pgfqpoint{1.702888in}{1.682389in}}{\pgfqpoint{1.699615in}{1.690289in}}{\pgfqpoint{1.693791in}{1.696113in}}%
\pgfpathcurveto{\pgfqpoint{1.687967in}{1.701937in}}{\pgfqpoint{1.680067in}{1.705209in}}{\pgfqpoint{1.671831in}{1.705209in}}%
\pgfpathcurveto{\pgfqpoint{1.663595in}{1.705209in}}{\pgfqpoint{1.655695in}{1.701937in}}{\pgfqpoint{1.649871in}{1.696113in}}%
\pgfpathcurveto{\pgfqpoint{1.644047in}{1.690289in}}{\pgfqpoint{1.640775in}{1.682389in}}{\pgfqpoint{1.640775in}{1.674153in}}%
\pgfpathcurveto{\pgfqpoint{1.640775in}{1.665917in}}{\pgfqpoint{1.644047in}{1.658017in}}{\pgfqpoint{1.649871in}{1.652193in}}%
\pgfpathcurveto{\pgfqpoint{1.655695in}{1.646369in}}{\pgfqpoint{1.663595in}{1.643096in}}{\pgfqpoint{1.671831in}{1.643096in}}%
\pgfpathclose%
\pgfusepath{stroke,fill}%
\end{pgfscope}%
\begin{pgfscope}%
\pgfpathrectangle{\pgfqpoint{0.100000in}{0.212622in}}{\pgfqpoint{3.696000in}{3.696000in}}%
\pgfusepath{clip}%
\pgfsetbuttcap%
\pgfsetroundjoin%
\definecolor{currentfill}{rgb}{0.121569,0.466667,0.705882}%
\pgfsetfillcolor{currentfill}%
\pgfsetfillopacity{0.749860}%
\pgfsetlinewidth{1.003750pt}%
\definecolor{currentstroke}{rgb}{0.121569,0.466667,0.705882}%
\pgfsetstrokecolor{currentstroke}%
\pgfsetstrokeopacity{0.749860}%
\pgfsetdash{}{0pt}%
\pgfpathmoveto{\pgfqpoint{1.672588in}{1.640491in}}%
\pgfpathcurveto{\pgfqpoint{1.680824in}{1.640491in}}{\pgfqpoint{1.688724in}{1.643763in}}{\pgfqpoint{1.694548in}{1.649587in}}%
\pgfpathcurveto{\pgfqpoint{1.700372in}{1.655411in}}{\pgfqpoint{1.703644in}{1.663311in}}{\pgfqpoint{1.703644in}{1.671548in}}%
\pgfpathcurveto{\pgfqpoint{1.703644in}{1.679784in}}{\pgfqpoint{1.700372in}{1.687684in}}{\pgfqpoint{1.694548in}{1.693508in}}%
\pgfpathcurveto{\pgfqpoint{1.688724in}{1.699332in}}{\pgfqpoint{1.680824in}{1.702604in}}{\pgfqpoint{1.672588in}{1.702604in}}%
\pgfpathcurveto{\pgfqpoint{1.664351in}{1.702604in}}{\pgfqpoint{1.656451in}{1.699332in}}{\pgfqpoint{1.650627in}{1.693508in}}%
\pgfpathcurveto{\pgfqpoint{1.644803in}{1.687684in}}{\pgfqpoint{1.641531in}{1.679784in}}{\pgfqpoint{1.641531in}{1.671548in}}%
\pgfpathcurveto{\pgfqpoint{1.641531in}{1.663311in}}{\pgfqpoint{1.644803in}{1.655411in}}{\pgfqpoint{1.650627in}{1.649587in}}%
\pgfpathcurveto{\pgfqpoint{1.656451in}{1.643763in}}{\pgfqpoint{1.664351in}{1.640491in}}{\pgfqpoint{1.672588in}{1.640491in}}%
\pgfpathclose%
\pgfusepath{stroke,fill}%
\end{pgfscope}%
\begin{pgfscope}%
\pgfpathrectangle{\pgfqpoint{0.100000in}{0.212622in}}{\pgfqpoint{3.696000in}{3.696000in}}%
\pgfusepath{clip}%
\pgfsetbuttcap%
\pgfsetroundjoin%
\definecolor{currentfill}{rgb}{0.121569,0.466667,0.705882}%
\pgfsetfillcolor{currentfill}%
\pgfsetfillopacity{0.752437}%
\pgfsetlinewidth{1.003750pt}%
\definecolor{currentstroke}{rgb}{0.121569,0.466667,0.705882}%
\pgfsetstrokecolor{currentstroke}%
\pgfsetstrokeopacity{0.752437}%
\pgfsetdash{}{0pt}%
\pgfpathmoveto{\pgfqpoint{1.674134in}{1.643320in}}%
\pgfpathcurveto{\pgfqpoint{1.682371in}{1.643320in}}{\pgfqpoint{1.690271in}{1.646592in}}{\pgfqpoint{1.696095in}{1.652416in}}%
\pgfpathcurveto{\pgfqpoint{1.701919in}{1.658240in}}{\pgfqpoint{1.705191in}{1.666140in}}{\pgfqpoint{1.705191in}{1.674376in}}%
\pgfpathcurveto{\pgfqpoint{1.705191in}{1.682612in}}{\pgfqpoint{1.701919in}{1.690512in}}{\pgfqpoint{1.696095in}{1.696336in}}%
\pgfpathcurveto{\pgfqpoint{1.690271in}{1.702160in}}{\pgfqpoint{1.682371in}{1.705433in}}{\pgfqpoint{1.674134in}{1.705433in}}%
\pgfpathcurveto{\pgfqpoint{1.665898in}{1.705433in}}{\pgfqpoint{1.657998in}{1.702160in}}{\pgfqpoint{1.652174in}{1.696336in}}%
\pgfpathcurveto{\pgfqpoint{1.646350in}{1.690512in}}{\pgfqpoint{1.643078in}{1.682612in}}{\pgfqpoint{1.643078in}{1.674376in}}%
\pgfpathcurveto{\pgfqpoint{1.643078in}{1.666140in}}{\pgfqpoint{1.646350in}{1.658240in}}{\pgfqpoint{1.652174in}{1.652416in}}%
\pgfpathcurveto{\pgfqpoint{1.657998in}{1.646592in}}{\pgfqpoint{1.665898in}{1.643320in}}{\pgfqpoint{1.674134in}{1.643320in}}%
\pgfpathclose%
\pgfusepath{stroke,fill}%
\end{pgfscope}%
\begin{pgfscope}%
\pgfpathrectangle{\pgfqpoint{0.100000in}{0.212622in}}{\pgfqpoint{3.696000in}{3.696000in}}%
\pgfusepath{clip}%
\pgfsetbuttcap%
\pgfsetroundjoin%
\definecolor{currentfill}{rgb}{0.121569,0.466667,0.705882}%
\pgfsetfillcolor{currentfill}%
\pgfsetfillopacity{0.753409}%
\pgfsetlinewidth{1.003750pt}%
\definecolor{currentstroke}{rgb}{0.121569,0.466667,0.705882}%
\pgfsetstrokecolor{currentstroke}%
\pgfsetstrokeopacity{0.753409}%
\pgfsetdash{}{0pt}%
\pgfpathmoveto{\pgfqpoint{1.674847in}{1.642767in}}%
\pgfpathcurveto{\pgfqpoint{1.683083in}{1.642767in}}{\pgfqpoint{1.690983in}{1.646039in}}{\pgfqpoint{1.696807in}{1.651863in}}%
\pgfpathcurveto{\pgfqpoint{1.702631in}{1.657687in}}{\pgfqpoint{1.705903in}{1.665587in}}{\pgfqpoint{1.705903in}{1.673823in}}%
\pgfpathcurveto{\pgfqpoint{1.705903in}{1.682059in}}{\pgfqpoint{1.702631in}{1.689959in}}{\pgfqpoint{1.696807in}{1.695783in}}%
\pgfpathcurveto{\pgfqpoint{1.690983in}{1.701607in}}{\pgfqpoint{1.683083in}{1.704880in}}{\pgfqpoint{1.674847in}{1.704880in}}%
\pgfpathcurveto{\pgfqpoint{1.666611in}{1.704880in}}{\pgfqpoint{1.658711in}{1.701607in}}{\pgfqpoint{1.652887in}{1.695783in}}%
\pgfpathcurveto{\pgfqpoint{1.647063in}{1.689959in}}{\pgfqpoint{1.643790in}{1.682059in}}{\pgfqpoint{1.643790in}{1.673823in}}%
\pgfpathcurveto{\pgfqpoint{1.643790in}{1.665587in}}{\pgfqpoint{1.647063in}{1.657687in}}{\pgfqpoint{1.652887in}{1.651863in}}%
\pgfpathcurveto{\pgfqpoint{1.658711in}{1.646039in}}{\pgfqpoint{1.666611in}{1.642767in}}{\pgfqpoint{1.674847in}{1.642767in}}%
\pgfpathclose%
\pgfusepath{stroke,fill}%
\end{pgfscope}%
\begin{pgfscope}%
\pgfpathrectangle{\pgfqpoint{0.100000in}{0.212622in}}{\pgfqpoint{3.696000in}{3.696000in}}%
\pgfusepath{clip}%
\pgfsetbuttcap%
\pgfsetroundjoin%
\definecolor{currentfill}{rgb}{0.121569,0.466667,0.705882}%
\pgfsetfillcolor{currentfill}%
\pgfsetfillopacity{0.754820}%
\pgfsetlinewidth{1.003750pt}%
\definecolor{currentstroke}{rgb}{0.121569,0.466667,0.705882}%
\pgfsetstrokecolor{currentstroke}%
\pgfsetstrokeopacity{0.754820}%
\pgfsetdash{}{0pt}%
\pgfpathmoveto{\pgfqpoint{1.675219in}{1.642687in}}%
\pgfpathcurveto{\pgfqpoint{1.683456in}{1.642687in}}{\pgfqpoint{1.691356in}{1.645959in}}{\pgfqpoint{1.697180in}{1.651783in}}%
\pgfpathcurveto{\pgfqpoint{1.703003in}{1.657607in}}{\pgfqpoint{1.706276in}{1.665507in}}{\pgfqpoint{1.706276in}{1.673743in}}%
\pgfpathcurveto{\pgfqpoint{1.706276in}{1.681980in}}{\pgfqpoint{1.703003in}{1.689880in}}{\pgfqpoint{1.697180in}{1.695704in}}%
\pgfpathcurveto{\pgfqpoint{1.691356in}{1.701527in}}{\pgfqpoint{1.683456in}{1.704800in}}{\pgfqpoint{1.675219in}{1.704800in}}%
\pgfpathcurveto{\pgfqpoint{1.666983in}{1.704800in}}{\pgfqpoint{1.659083in}{1.701527in}}{\pgfqpoint{1.653259in}{1.695704in}}%
\pgfpathcurveto{\pgfqpoint{1.647435in}{1.689880in}}{\pgfqpoint{1.644163in}{1.681980in}}{\pgfqpoint{1.644163in}{1.673743in}}%
\pgfpathcurveto{\pgfqpoint{1.644163in}{1.665507in}}{\pgfqpoint{1.647435in}{1.657607in}}{\pgfqpoint{1.653259in}{1.651783in}}%
\pgfpathcurveto{\pgfqpoint{1.659083in}{1.645959in}}{\pgfqpoint{1.666983in}{1.642687in}}{\pgfqpoint{1.675219in}{1.642687in}}%
\pgfpathclose%
\pgfusepath{stroke,fill}%
\end{pgfscope}%
\begin{pgfscope}%
\pgfpathrectangle{\pgfqpoint{0.100000in}{0.212622in}}{\pgfqpoint{3.696000in}{3.696000in}}%
\pgfusepath{clip}%
\pgfsetbuttcap%
\pgfsetroundjoin%
\definecolor{currentfill}{rgb}{0.121569,0.466667,0.705882}%
\pgfsetfillcolor{currentfill}%
\pgfsetfillopacity{0.755336}%
\pgfsetlinewidth{1.003750pt}%
\definecolor{currentstroke}{rgb}{0.121569,0.466667,0.705882}%
\pgfsetstrokecolor{currentstroke}%
\pgfsetstrokeopacity{0.755336}%
\pgfsetdash{}{0pt}%
\pgfpathmoveto{\pgfqpoint{1.675699in}{1.641591in}}%
\pgfpathcurveto{\pgfqpoint{1.683935in}{1.641591in}}{\pgfqpoint{1.691835in}{1.644863in}}{\pgfqpoint{1.697659in}{1.650687in}}%
\pgfpathcurveto{\pgfqpoint{1.703483in}{1.656511in}}{\pgfqpoint{1.706755in}{1.664411in}}{\pgfqpoint{1.706755in}{1.672647in}}%
\pgfpathcurveto{\pgfqpoint{1.706755in}{1.680883in}}{\pgfqpoint{1.703483in}{1.688784in}}{\pgfqpoint{1.697659in}{1.694607in}}%
\pgfpathcurveto{\pgfqpoint{1.691835in}{1.700431in}}{\pgfqpoint{1.683935in}{1.703704in}}{\pgfqpoint{1.675699in}{1.703704in}}%
\pgfpathcurveto{\pgfqpoint{1.667462in}{1.703704in}}{\pgfqpoint{1.659562in}{1.700431in}}{\pgfqpoint{1.653738in}{1.694607in}}%
\pgfpathcurveto{\pgfqpoint{1.647915in}{1.688784in}}{\pgfqpoint{1.644642in}{1.680883in}}{\pgfqpoint{1.644642in}{1.672647in}}%
\pgfpathcurveto{\pgfqpoint{1.644642in}{1.664411in}}{\pgfqpoint{1.647915in}{1.656511in}}{\pgfqpoint{1.653738in}{1.650687in}}%
\pgfpathcurveto{\pgfqpoint{1.659562in}{1.644863in}}{\pgfqpoint{1.667462in}{1.641591in}}{\pgfqpoint{1.675699in}{1.641591in}}%
\pgfpathclose%
\pgfusepath{stroke,fill}%
\end{pgfscope}%
\begin{pgfscope}%
\pgfpathrectangle{\pgfqpoint{0.100000in}{0.212622in}}{\pgfqpoint{3.696000in}{3.696000in}}%
\pgfusepath{clip}%
\pgfsetbuttcap%
\pgfsetroundjoin%
\definecolor{currentfill}{rgb}{0.121569,0.466667,0.705882}%
\pgfsetfillcolor{currentfill}%
\pgfsetfillopacity{0.756438}%
\pgfsetlinewidth{1.003750pt}%
\definecolor{currentstroke}{rgb}{0.121569,0.466667,0.705882}%
\pgfsetstrokecolor{currentstroke}%
\pgfsetstrokeopacity{0.756438}%
\pgfsetdash{}{0pt}%
\pgfpathmoveto{\pgfqpoint{1.676396in}{1.642542in}}%
\pgfpathcurveto{\pgfqpoint{1.684633in}{1.642542in}}{\pgfqpoint{1.692533in}{1.645815in}}{\pgfqpoint{1.698357in}{1.651639in}}%
\pgfpathcurveto{\pgfqpoint{1.704181in}{1.657463in}}{\pgfqpoint{1.707453in}{1.665363in}}{\pgfqpoint{1.707453in}{1.673599in}}%
\pgfpathcurveto{\pgfqpoint{1.707453in}{1.681835in}}{\pgfqpoint{1.704181in}{1.689735in}}{\pgfqpoint{1.698357in}{1.695559in}}%
\pgfpathcurveto{\pgfqpoint{1.692533in}{1.701383in}}{\pgfqpoint{1.684633in}{1.704655in}}{\pgfqpoint{1.676396in}{1.704655in}}%
\pgfpathcurveto{\pgfqpoint{1.668160in}{1.704655in}}{\pgfqpoint{1.660260in}{1.701383in}}{\pgfqpoint{1.654436in}{1.695559in}}%
\pgfpathcurveto{\pgfqpoint{1.648612in}{1.689735in}}{\pgfqpoint{1.645340in}{1.681835in}}{\pgfqpoint{1.645340in}{1.673599in}}%
\pgfpathcurveto{\pgfqpoint{1.645340in}{1.665363in}}{\pgfqpoint{1.648612in}{1.657463in}}{\pgfqpoint{1.654436in}{1.651639in}}%
\pgfpathcurveto{\pgfqpoint{1.660260in}{1.645815in}}{\pgfqpoint{1.668160in}{1.642542in}}{\pgfqpoint{1.676396in}{1.642542in}}%
\pgfpathclose%
\pgfusepath{stroke,fill}%
\end{pgfscope}%
\begin{pgfscope}%
\pgfpathrectangle{\pgfqpoint{0.100000in}{0.212622in}}{\pgfqpoint{3.696000in}{3.696000in}}%
\pgfusepath{clip}%
\pgfsetbuttcap%
\pgfsetroundjoin%
\definecolor{currentfill}{rgb}{0.121569,0.466667,0.705882}%
\pgfsetfillcolor{currentfill}%
\pgfsetfillopacity{0.756771}%
\pgfsetlinewidth{1.003750pt}%
\definecolor{currentstroke}{rgb}{0.121569,0.466667,0.705882}%
\pgfsetstrokecolor{currentstroke}%
\pgfsetstrokeopacity{0.756771}%
\pgfsetdash{}{0pt}%
\pgfpathmoveto{\pgfqpoint{1.676587in}{1.641725in}}%
\pgfpathcurveto{\pgfqpoint{1.684823in}{1.641725in}}{\pgfqpoint{1.692723in}{1.644997in}}{\pgfqpoint{1.698547in}{1.650821in}}%
\pgfpathcurveto{\pgfqpoint{1.704371in}{1.656645in}}{\pgfqpoint{1.707643in}{1.664545in}}{\pgfqpoint{1.707643in}{1.672781in}}%
\pgfpathcurveto{\pgfqpoint{1.707643in}{1.681018in}}{\pgfqpoint{1.704371in}{1.688918in}}{\pgfqpoint{1.698547in}{1.694742in}}%
\pgfpathcurveto{\pgfqpoint{1.692723in}{1.700565in}}{\pgfqpoint{1.684823in}{1.703838in}}{\pgfqpoint{1.676587in}{1.703838in}}%
\pgfpathcurveto{\pgfqpoint{1.668351in}{1.703838in}}{\pgfqpoint{1.660450in}{1.700565in}}{\pgfqpoint{1.654627in}{1.694742in}}%
\pgfpathcurveto{\pgfqpoint{1.648803in}{1.688918in}}{\pgfqpoint{1.645530in}{1.681018in}}{\pgfqpoint{1.645530in}{1.672781in}}%
\pgfpathcurveto{\pgfqpoint{1.645530in}{1.664545in}}{\pgfqpoint{1.648803in}{1.656645in}}{\pgfqpoint{1.654627in}{1.650821in}}%
\pgfpathcurveto{\pgfqpoint{1.660450in}{1.644997in}}{\pgfqpoint{1.668351in}{1.641725in}}{\pgfqpoint{1.676587in}{1.641725in}}%
\pgfpathclose%
\pgfusepath{stroke,fill}%
\end{pgfscope}%
\begin{pgfscope}%
\pgfpathrectangle{\pgfqpoint{0.100000in}{0.212622in}}{\pgfqpoint{3.696000in}{3.696000in}}%
\pgfusepath{clip}%
\pgfsetbuttcap%
\pgfsetroundjoin%
\definecolor{currentfill}{rgb}{0.121569,0.466667,0.705882}%
\pgfsetfillcolor{currentfill}%
\pgfsetfillopacity{0.757507}%
\pgfsetlinewidth{1.003750pt}%
\definecolor{currentstroke}{rgb}{0.121569,0.466667,0.705882}%
\pgfsetstrokecolor{currentstroke}%
\pgfsetstrokeopacity{0.757507}%
\pgfsetdash{}{0pt}%
\pgfpathmoveto{\pgfqpoint{1.677036in}{1.641521in}}%
\pgfpathcurveto{\pgfqpoint{1.685272in}{1.641521in}}{\pgfqpoint{1.693172in}{1.644793in}}{\pgfqpoint{1.698996in}{1.650617in}}%
\pgfpathcurveto{\pgfqpoint{1.704820in}{1.656441in}}{\pgfqpoint{1.708092in}{1.664341in}}{\pgfqpoint{1.708092in}{1.672577in}}%
\pgfpathcurveto{\pgfqpoint{1.708092in}{1.680814in}}{\pgfqpoint{1.704820in}{1.688714in}}{\pgfqpoint{1.698996in}{1.694538in}}%
\pgfpathcurveto{\pgfqpoint{1.693172in}{1.700362in}}{\pgfqpoint{1.685272in}{1.703634in}}{\pgfqpoint{1.677036in}{1.703634in}}%
\pgfpathcurveto{\pgfqpoint{1.668800in}{1.703634in}}{\pgfqpoint{1.660900in}{1.700362in}}{\pgfqpoint{1.655076in}{1.694538in}}%
\pgfpathcurveto{\pgfqpoint{1.649252in}{1.688714in}}{\pgfqpoint{1.645979in}{1.680814in}}{\pgfqpoint{1.645979in}{1.672577in}}%
\pgfpathcurveto{\pgfqpoint{1.645979in}{1.664341in}}{\pgfqpoint{1.649252in}{1.656441in}}{\pgfqpoint{1.655076in}{1.650617in}}%
\pgfpathcurveto{\pgfqpoint{1.660900in}{1.644793in}}{\pgfqpoint{1.668800in}{1.641521in}}{\pgfqpoint{1.677036in}{1.641521in}}%
\pgfpathclose%
\pgfusepath{stroke,fill}%
\end{pgfscope}%
\begin{pgfscope}%
\pgfpathrectangle{\pgfqpoint{0.100000in}{0.212622in}}{\pgfqpoint{3.696000in}{3.696000in}}%
\pgfusepath{clip}%
\pgfsetbuttcap%
\pgfsetroundjoin%
\definecolor{currentfill}{rgb}{0.121569,0.466667,0.705882}%
\pgfsetfillcolor{currentfill}%
\pgfsetfillopacity{0.757976}%
\pgfsetlinewidth{1.003750pt}%
\definecolor{currentstroke}{rgb}{0.121569,0.466667,0.705882}%
\pgfsetstrokecolor{currentstroke}%
\pgfsetstrokeopacity{0.757976}%
\pgfsetdash{}{0pt}%
\pgfpathmoveto{\pgfqpoint{1.677205in}{1.641660in}}%
\pgfpathcurveto{\pgfqpoint{1.685441in}{1.641660in}}{\pgfqpoint{1.693341in}{1.644932in}}{\pgfqpoint{1.699165in}{1.650756in}}%
\pgfpathcurveto{\pgfqpoint{1.704989in}{1.656580in}}{\pgfqpoint{1.708261in}{1.664480in}}{\pgfqpoint{1.708261in}{1.672716in}}%
\pgfpathcurveto{\pgfqpoint{1.708261in}{1.680953in}}{\pgfqpoint{1.704989in}{1.688853in}}{\pgfqpoint{1.699165in}{1.694677in}}%
\pgfpathcurveto{\pgfqpoint{1.693341in}{1.700501in}}{\pgfqpoint{1.685441in}{1.703773in}}{\pgfqpoint{1.677205in}{1.703773in}}%
\pgfpathcurveto{\pgfqpoint{1.668968in}{1.703773in}}{\pgfqpoint{1.661068in}{1.700501in}}{\pgfqpoint{1.655244in}{1.694677in}}%
\pgfpathcurveto{\pgfqpoint{1.649420in}{1.688853in}}{\pgfqpoint{1.646148in}{1.680953in}}{\pgfqpoint{1.646148in}{1.672716in}}%
\pgfpathcurveto{\pgfqpoint{1.646148in}{1.664480in}}{\pgfqpoint{1.649420in}{1.656580in}}{\pgfqpoint{1.655244in}{1.650756in}}%
\pgfpathcurveto{\pgfqpoint{1.661068in}{1.644932in}}{\pgfqpoint{1.668968in}{1.641660in}}{\pgfqpoint{1.677205in}{1.641660in}}%
\pgfpathclose%
\pgfusepath{stroke,fill}%
\end{pgfscope}%
\begin{pgfscope}%
\pgfpathrectangle{\pgfqpoint{0.100000in}{0.212622in}}{\pgfqpoint{3.696000in}{3.696000in}}%
\pgfusepath{clip}%
\pgfsetbuttcap%
\pgfsetroundjoin%
\definecolor{currentfill}{rgb}{0.121569,0.466667,0.705882}%
\pgfsetfillcolor{currentfill}%
\pgfsetfillopacity{0.758772}%
\pgfsetlinewidth{1.003750pt}%
\definecolor{currentstroke}{rgb}{0.121569,0.466667,0.705882}%
\pgfsetstrokecolor{currentstroke}%
\pgfsetstrokeopacity{0.758772}%
\pgfsetdash{}{0pt}%
\pgfpathmoveto{\pgfqpoint{1.677787in}{1.642038in}}%
\pgfpathcurveto{\pgfqpoint{1.686023in}{1.642038in}}{\pgfqpoint{1.693923in}{1.645310in}}{\pgfqpoint{1.699747in}{1.651134in}}%
\pgfpathcurveto{\pgfqpoint{1.705571in}{1.656958in}}{\pgfqpoint{1.708843in}{1.664858in}}{\pgfqpoint{1.708843in}{1.673094in}}%
\pgfpathcurveto{\pgfqpoint{1.708843in}{1.681330in}}{\pgfqpoint{1.705571in}{1.689230in}}{\pgfqpoint{1.699747in}{1.695054in}}%
\pgfpathcurveto{\pgfqpoint{1.693923in}{1.700878in}}{\pgfqpoint{1.686023in}{1.704151in}}{\pgfqpoint{1.677787in}{1.704151in}}%
\pgfpathcurveto{\pgfqpoint{1.669550in}{1.704151in}}{\pgfqpoint{1.661650in}{1.700878in}}{\pgfqpoint{1.655826in}{1.695054in}}%
\pgfpathcurveto{\pgfqpoint{1.650002in}{1.689230in}}{\pgfqpoint{1.646730in}{1.681330in}}{\pgfqpoint{1.646730in}{1.673094in}}%
\pgfpathcurveto{\pgfqpoint{1.646730in}{1.664858in}}{\pgfqpoint{1.650002in}{1.656958in}}{\pgfqpoint{1.655826in}{1.651134in}}%
\pgfpathcurveto{\pgfqpoint{1.661650in}{1.645310in}}{\pgfqpoint{1.669550in}{1.642038in}}{\pgfqpoint{1.677787in}{1.642038in}}%
\pgfpathclose%
\pgfusepath{stroke,fill}%
\end{pgfscope}%
\begin{pgfscope}%
\pgfpathrectangle{\pgfqpoint{0.100000in}{0.212622in}}{\pgfqpoint{3.696000in}{3.696000in}}%
\pgfusepath{clip}%
\pgfsetbuttcap%
\pgfsetroundjoin%
\definecolor{currentfill}{rgb}{0.121569,0.466667,0.705882}%
\pgfsetfillcolor{currentfill}%
\pgfsetfillopacity{0.759654}%
\pgfsetlinewidth{1.003750pt}%
\definecolor{currentstroke}{rgb}{0.121569,0.466667,0.705882}%
\pgfsetstrokecolor{currentstroke}%
\pgfsetstrokeopacity{0.759654}%
\pgfsetdash{}{0pt}%
\pgfpathmoveto{\pgfqpoint{1.678063in}{1.641690in}}%
\pgfpathcurveto{\pgfqpoint{1.686300in}{1.641690in}}{\pgfqpoint{1.694200in}{1.644962in}}{\pgfqpoint{1.700024in}{1.650786in}}%
\pgfpathcurveto{\pgfqpoint{1.705848in}{1.656610in}}{\pgfqpoint{1.709120in}{1.664510in}}{\pgfqpoint{1.709120in}{1.672746in}}%
\pgfpathcurveto{\pgfqpoint{1.709120in}{1.680982in}}{\pgfqpoint{1.705848in}{1.688882in}}{\pgfqpoint{1.700024in}{1.694706in}}%
\pgfpathcurveto{\pgfqpoint{1.694200in}{1.700530in}}{\pgfqpoint{1.686300in}{1.703803in}}{\pgfqpoint{1.678063in}{1.703803in}}%
\pgfpathcurveto{\pgfqpoint{1.669827in}{1.703803in}}{\pgfqpoint{1.661927in}{1.700530in}}{\pgfqpoint{1.656103in}{1.694706in}}%
\pgfpathcurveto{\pgfqpoint{1.650279in}{1.688882in}}{\pgfqpoint{1.647007in}{1.680982in}}{\pgfqpoint{1.647007in}{1.672746in}}%
\pgfpathcurveto{\pgfqpoint{1.647007in}{1.664510in}}{\pgfqpoint{1.650279in}{1.656610in}}{\pgfqpoint{1.656103in}{1.650786in}}%
\pgfpathcurveto{\pgfqpoint{1.661927in}{1.644962in}}{\pgfqpoint{1.669827in}{1.641690in}}{\pgfqpoint{1.678063in}{1.641690in}}%
\pgfpathclose%
\pgfusepath{stroke,fill}%
\end{pgfscope}%
\begin{pgfscope}%
\pgfpathrectangle{\pgfqpoint{0.100000in}{0.212622in}}{\pgfqpoint{3.696000in}{3.696000in}}%
\pgfusepath{clip}%
\pgfsetbuttcap%
\pgfsetroundjoin%
\definecolor{currentfill}{rgb}{0.121569,0.466667,0.705882}%
\pgfsetfillcolor{currentfill}%
\pgfsetfillopacity{0.760659}%
\pgfsetlinewidth{1.003750pt}%
\definecolor{currentstroke}{rgb}{0.121569,0.466667,0.705882}%
\pgfsetstrokecolor{currentstroke}%
\pgfsetstrokeopacity{0.760659}%
\pgfsetdash{}{0pt}%
\pgfpathmoveto{\pgfqpoint{1.679083in}{1.638535in}}%
\pgfpathcurveto{\pgfqpoint{1.687320in}{1.638535in}}{\pgfqpoint{1.695220in}{1.641807in}}{\pgfqpoint{1.701044in}{1.647631in}}%
\pgfpathcurveto{\pgfqpoint{1.706867in}{1.653455in}}{\pgfqpoint{1.710140in}{1.661355in}}{\pgfqpoint{1.710140in}{1.669591in}}%
\pgfpathcurveto{\pgfqpoint{1.710140in}{1.677827in}}{\pgfqpoint{1.706867in}{1.685728in}}{\pgfqpoint{1.701044in}{1.691551in}}%
\pgfpathcurveto{\pgfqpoint{1.695220in}{1.697375in}}{\pgfqpoint{1.687320in}{1.700648in}}{\pgfqpoint{1.679083in}{1.700648in}}%
\pgfpathcurveto{\pgfqpoint{1.670847in}{1.700648in}}{\pgfqpoint{1.662947in}{1.697375in}}{\pgfqpoint{1.657123in}{1.691551in}}%
\pgfpathcurveto{\pgfqpoint{1.651299in}{1.685728in}}{\pgfqpoint{1.648027in}{1.677827in}}{\pgfqpoint{1.648027in}{1.669591in}}%
\pgfpathcurveto{\pgfqpoint{1.648027in}{1.661355in}}{\pgfqpoint{1.651299in}{1.653455in}}{\pgfqpoint{1.657123in}{1.647631in}}%
\pgfpathcurveto{\pgfqpoint{1.662947in}{1.641807in}}{\pgfqpoint{1.670847in}{1.638535in}}{\pgfqpoint{1.679083in}{1.638535in}}%
\pgfpathclose%
\pgfusepath{stroke,fill}%
\end{pgfscope}%
\begin{pgfscope}%
\pgfpathrectangle{\pgfqpoint{0.100000in}{0.212622in}}{\pgfqpoint{3.696000in}{3.696000in}}%
\pgfusepath{clip}%
\pgfsetbuttcap%
\pgfsetroundjoin%
\definecolor{currentfill}{rgb}{0.121569,0.466667,0.705882}%
\pgfsetfillcolor{currentfill}%
\pgfsetfillopacity{0.762939}%
\pgfsetlinewidth{1.003750pt}%
\definecolor{currentstroke}{rgb}{0.121569,0.466667,0.705882}%
\pgfsetstrokecolor{currentstroke}%
\pgfsetstrokeopacity{0.762939}%
\pgfsetdash{}{0pt}%
\pgfpathmoveto{\pgfqpoint{1.680679in}{1.640047in}}%
\pgfpathcurveto{\pgfqpoint{1.688916in}{1.640047in}}{\pgfqpoint{1.696816in}{1.643319in}}{\pgfqpoint{1.702640in}{1.649143in}}%
\pgfpathcurveto{\pgfqpoint{1.708464in}{1.654967in}}{\pgfqpoint{1.711736in}{1.662867in}}{\pgfqpoint{1.711736in}{1.671103in}}%
\pgfpathcurveto{\pgfqpoint{1.711736in}{1.679339in}}{\pgfqpoint{1.708464in}{1.687239in}}{\pgfqpoint{1.702640in}{1.693063in}}%
\pgfpathcurveto{\pgfqpoint{1.696816in}{1.698887in}}{\pgfqpoint{1.688916in}{1.702160in}}{\pgfqpoint{1.680679in}{1.702160in}}%
\pgfpathcurveto{\pgfqpoint{1.672443in}{1.702160in}}{\pgfqpoint{1.664543in}{1.698887in}}{\pgfqpoint{1.658719in}{1.693063in}}%
\pgfpathcurveto{\pgfqpoint{1.652895in}{1.687239in}}{\pgfqpoint{1.649623in}{1.679339in}}{\pgfqpoint{1.649623in}{1.671103in}}%
\pgfpathcurveto{\pgfqpoint{1.649623in}{1.662867in}}{\pgfqpoint{1.652895in}{1.654967in}}{\pgfqpoint{1.658719in}{1.649143in}}%
\pgfpathcurveto{\pgfqpoint{1.664543in}{1.643319in}}{\pgfqpoint{1.672443in}{1.640047in}}{\pgfqpoint{1.680679in}{1.640047in}}%
\pgfpathclose%
\pgfusepath{stroke,fill}%
\end{pgfscope}%
\begin{pgfscope}%
\pgfpathrectangle{\pgfqpoint{0.100000in}{0.212622in}}{\pgfqpoint{3.696000in}{3.696000in}}%
\pgfusepath{clip}%
\pgfsetbuttcap%
\pgfsetroundjoin%
\definecolor{currentfill}{rgb}{0.121569,0.466667,0.705882}%
\pgfsetfillcolor{currentfill}%
\pgfsetfillopacity{0.765477}%
\pgfsetlinewidth{1.003750pt}%
\definecolor{currentstroke}{rgb}{0.121569,0.466667,0.705882}%
\pgfsetstrokecolor{currentstroke}%
\pgfsetstrokeopacity{0.765477}%
\pgfsetdash{}{0pt}%
\pgfpathmoveto{\pgfqpoint{1.681784in}{1.639626in}}%
\pgfpathcurveto{\pgfqpoint{1.690021in}{1.639626in}}{\pgfqpoint{1.697921in}{1.642898in}}{\pgfqpoint{1.703745in}{1.648722in}}%
\pgfpathcurveto{\pgfqpoint{1.709569in}{1.654546in}}{\pgfqpoint{1.712841in}{1.662446in}}{\pgfqpoint{1.712841in}{1.670682in}}%
\pgfpathcurveto{\pgfqpoint{1.712841in}{1.678918in}}{\pgfqpoint{1.709569in}{1.686818in}}{\pgfqpoint{1.703745in}{1.692642in}}%
\pgfpathcurveto{\pgfqpoint{1.697921in}{1.698466in}}{\pgfqpoint{1.690021in}{1.701739in}}{\pgfqpoint{1.681784in}{1.701739in}}%
\pgfpathcurveto{\pgfqpoint{1.673548in}{1.701739in}}{\pgfqpoint{1.665648in}{1.698466in}}{\pgfqpoint{1.659824in}{1.692642in}}%
\pgfpathcurveto{\pgfqpoint{1.654000in}{1.686818in}}{\pgfqpoint{1.650728in}{1.678918in}}{\pgfqpoint{1.650728in}{1.670682in}}%
\pgfpathcurveto{\pgfqpoint{1.650728in}{1.662446in}}{\pgfqpoint{1.654000in}{1.654546in}}{\pgfqpoint{1.659824in}{1.648722in}}%
\pgfpathcurveto{\pgfqpoint{1.665648in}{1.642898in}}{\pgfqpoint{1.673548in}{1.639626in}}{\pgfqpoint{1.681784in}{1.639626in}}%
\pgfpathclose%
\pgfusepath{stroke,fill}%
\end{pgfscope}%
\begin{pgfscope}%
\pgfpathrectangle{\pgfqpoint{0.100000in}{0.212622in}}{\pgfqpoint{3.696000in}{3.696000in}}%
\pgfusepath{clip}%
\pgfsetbuttcap%
\pgfsetroundjoin%
\definecolor{currentfill}{rgb}{0.121569,0.466667,0.705882}%
\pgfsetfillcolor{currentfill}%
\pgfsetfillopacity{0.768475}%
\pgfsetlinewidth{1.003750pt}%
\definecolor{currentstroke}{rgb}{0.121569,0.466667,0.705882}%
\pgfsetstrokecolor{currentstroke}%
\pgfsetstrokeopacity{0.768475}%
\pgfsetdash{}{0pt}%
\pgfpathmoveto{\pgfqpoint{1.682872in}{1.639009in}}%
\pgfpathcurveto{\pgfqpoint{1.691108in}{1.639009in}}{\pgfqpoint{1.699008in}{1.642281in}}{\pgfqpoint{1.704832in}{1.648105in}}%
\pgfpathcurveto{\pgfqpoint{1.710656in}{1.653929in}}{\pgfqpoint{1.713929in}{1.661829in}}{\pgfqpoint{1.713929in}{1.670065in}}%
\pgfpathcurveto{\pgfqpoint{1.713929in}{1.678301in}}{\pgfqpoint{1.710656in}{1.686201in}}{\pgfqpoint{1.704832in}{1.692025in}}%
\pgfpathcurveto{\pgfqpoint{1.699008in}{1.697849in}}{\pgfqpoint{1.691108in}{1.701122in}}{\pgfqpoint{1.682872in}{1.701122in}}%
\pgfpathcurveto{\pgfqpoint{1.674636in}{1.701122in}}{\pgfqpoint{1.666736in}{1.697849in}}{\pgfqpoint{1.660912in}{1.692025in}}%
\pgfpathcurveto{\pgfqpoint{1.655088in}{1.686201in}}{\pgfqpoint{1.651816in}{1.678301in}}{\pgfqpoint{1.651816in}{1.670065in}}%
\pgfpathcurveto{\pgfqpoint{1.651816in}{1.661829in}}{\pgfqpoint{1.655088in}{1.653929in}}{\pgfqpoint{1.660912in}{1.648105in}}%
\pgfpathcurveto{\pgfqpoint{1.666736in}{1.642281in}}{\pgfqpoint{1.674636in}{1.639009in}}{\pgfqpoint{1.682872in}{1.639009in}}%
\pgfpathclose%
\pgfusepath{stroke,fill}%
\end{pgfscope}%
\begin{pgfscope}%
\pgfpathrectangle{\pgfqpoint{0.100000in}{0.212622in}}{\pgfqpoint{3.696000in}{3.696000in}}%
\pgfusepath{clip}%
\pgfsetbuttcap%
\pgfsetroundjoin%
\definecolor{currentfill}{rgb}{0.121569,0.466667,0.705882}%
\pgfsetfillcolor{currentfill}%
\pgfsetfillopacity{0.771213}%
\pgfsetlinewidth{1.003750pt}%
\definecolor{currentstroke}{rgb}{0.121569,0.466667,0.705882}%
\pgfsetstrokecolor{currentstroke}%
\pgfsetstrokeopacity{0.771213}%
\pgfsetdash{}{0pt}%
\pgfpathmoveto{\pgfqpoint{1.684980in}{1.635627in}}%
\pgfpathcurveto{\pgfqpoint{1.693217in}{1.635627in}}{\pgfqpoint{1.701117in}{1.638899in}}{\pgfqpoint{1.706941in}{1.644723in}}%
\pgfpathcurveto{\pgfqpoint{1.712765in}{1.650547in}}{\pgfqpoint{1.716037in}{1.658447in}}{\pgfqpoint{1.716037in}{1.666683in}}%
\pgfpathcurveto{\pgfqpoint{1.716037in}{1.674920in}}{\pgfqpoint{1.712765in}{1.682820in}}{\pgfqpoint{1.706941in}{1.688644in}}%
\pgfpathcurveto{\pgfqpoint{1.701117in}{1.694468in}}{\pgfqpoint{1.693217in}{1.697740in}}{\pgfqpoint{1.684980in}{1.697740in}}%
\pgfpathcurveto{\pgfqpoint{1.676744in}{1.697740in}}{\pgfqpoint{1.668844in}{1.694468in}}{\pgfqpoint{1.663020in}{1.688644in}}%
\pgfpathcurveto{\pgfqpoint{1.657196in}{1.682820in}}{\pgfqpoint{1.653924in}{1.674920in}}{\pgfqpoint{1.653924in}{1.666683in}}%
\pgfpathcurveto{\pgfqpoint{1.653924in}{1.658447in}}{\pgfqpoint{1.657196in}{1.650547in}}{\pgfqpoint{1.663020in}{1.644723in}}%
\pgfpathcurveto{\pgfqpoint{1.668844in}{1.638899in}}{\pgfqpoint{1.676744in}{1.635627in}}{\pgfqpoint{1.684980in}{1.635627in}}%
\pgfpathclose%
\pgfusepath{stroke,fill}%
\end{pgfscope}%
\begin{pgfscope}%
\pgfpathrectangle{\pgfqpoint{0.100000in}{0.212622in}}{\pgfqpoint{3.696000in}{3.696000in}}%
\pgfusepath{clip}%
\pgfsetbuttcap%
\pgfsetroundjoin%
\definecolor{currentfill}{rgb}{0.121569,0.466667,0.705882}%
\pgfsetfillcolor{currentfill}%
\pgfsetfillopacity{0.775482}%
\pgfsetlinewidth{1.003750pt}%
\definecolor{currentstroke}{rgb}{0.121569,0.466667,0.705882}%
\pgfsetstrokecolor{currentstroke}%
\pgfsetstrokeopacity{0.775482}%
\pgfsetdash{}{0pt}%
\pgfpathmoveto{\pgfqpoint{1.687679in}{1.638638in}}%
\pgfpathcurveto{\pgfqpoint{1.695916in}{1.638638in}}{\pgfqpoint{1.703816in}{1.641910in}}{\pgfqpoint{1.709639in}{1.647734in}}%
\pgfpathcurveto{\pgfqpoint{1.715463in}{1.653558in}}{\pgfqpoint{1.718736in}{1.661458in}}{\pgfqpoint{1.718736in}{1.669694in}}%
\pgfpathcurveto{\pgfqpoint{1.718736in}{1.677930in}}{\pgfqpoint{1.715463in}{1.685830in}}{\pgfqpoint{1.709639in}{1.691654in}}%
\pgfpathcurveto{\pgfqpoint{1.703816in}{1.697478in}}{\pgfqpoint{1.695916in}{1.700751in}}{\pgfqpoint{1.687679in}{1.700751in}}%
\pgfpathcurveto{\pgfqpoint{1.679443in}{1.700751in}}{\pgfqpoint{1.671543in}{1.697478in}}{\pgfqpoint{1.665719in}{1.691654in}}%
\pgfpathcurveto{\pgfqpoint{1.659895in}{1.685830in}}{\pgfqpoint{1.656623in}{1.677930in}}{\pgfqpoint{1.656623in}{1.669694in}}%
\pgfpathcurveto{\pgfqpoint{1.656623in}{1.661458in}}{\pgfqpoint{1.659895in}{1.653558in}}{\pgfqpoint{1.665719in}{1.647734in}}%
\pgfpathcurveto{\pgfqpoint{1.671543in}{1.641910in}}{\pgfqpoint{1.679443in}{1.638638in}}{\pgfqpoint{1.687679in}{1.638638in}}%
\pgfpathclose%
\pgfusepath{stroke,fill}%
\end{pgfscope}%
\begin{pgfscope}%
\pgfpathrectangle{\pgfqpoint{0.100000in}{0.212622in}}{\pgfqpoint{3.696000in}{3.696000in}}%
\pgfusepath{clip}%
\pgfsetbuttcap%
\pgfsetroundjoin%
\definecolor{currentfill}{rgb}{0.121569,0.466667,0.705882}%
\pgfsetfillcolor{currentfill}%
\pgfsetfillopacity{0.778906}%
\pgfsetlinewidth{1.003750pt}%
\definecolor{currentstroke}{rgb}{0.121569,0.466667,0.705882}%
\pgfsetstrokecolor{currentstroke}%
\pgfsetstrokeopacity{0.778906}%
\pgfsetdash{}{0pt}%
\pgfpathmoveto{\pgfqpoint{1.689606in}{1.636052in}}%
\pgfpathcurveto{\pgfqpoint{1.697842in}{1.636052in}}{\pgfqpoint{1.705742in}{1.639324in}}{\pgfqpoint{1.711566in}{1.645148in}}%
\pgfpathcurveto{\pgfqpoint{1.717390in}{1.650972in}}{\pgfqpoint{1.720663in}{1.658872in}}{\pgfqpoint{1.720663in}{1.667108in}}%
\pgfpathcurveto{\pgfqpoint{1.720663in}{1.675344in}}{\pgfqpoint{1.717390in}{1.683245in}}{\pgfqpoint{1.711566in}{1.689068in}}%
\pgfpathcurveto{\pgfqpoint{1.705742in}{1.694892in}}{\pgfqpoint{1.697842in}{1.698165in}}{\pgfqpoint{1.689606in}{1.698165in}}%
\pgfpathcurveto{\pgfqpoint{1.681370in}{1.698165in}}{\pgfqpoint{1.673470in}{1.694892in}}{\pgfqpoint{1.667646in}{1.689068in}}%
\pgfpathcurveto{\pgfqpoint{1.661822in}{1.683245in}}{\pgfqpoint{1.658550in}{1.675344in}}{\pgfqpoint{1.658550in}{1.667108in}}%
\pgfpathcurveto{\pgfqpoint{1.658550in}{1.658872in}}{\pgfqpoint{1.661822in}{1.650972in}}{\pgfqpoint{1.667646in}{1.645148in}}%
\pgfpathcurveto{\pgfqpoint{1.673470in}{1.639324in}}{\pgfqpoint{1.681370in}{1.636052in}}{\pgfqpoint{1.689606in}{1.636052in}}%
\pgfpathclose%
\pgfusepath{stroke,fill}%
\end{pgfscope}%
\begin{pgfscope}%
\pgfpathrectangle{\pgfqpoint{0.100000in}{0.212622in}}{\pgfqpoint{3.696000in}{3.696000in}}%
\pgfusepath{clip}%
\pgfsetbuttcap%
\pgfsetroundjoin%
\definecolor{currentfill}{rgb}{0.121569,0.466667,0.705882}%
\pgfsetfillcolor{currentfill}%
\pgfsetfillopacity{0.780491}%
\pgfsetlinewidth{1.003750pt}%
\definecolor{currentstroke}{rgb}{0.121569,0.466667,0.705882}%
\pgfsetstrokecolor{currentstroke}%
\pgfsetstrokeopacity{0.780491}%
\pgfsetdash{}{0pt}%
\pgfpathmoveto{\pgfqpoint{1.690905in}{1.633409in}}%
\pgfpathcurveto{\pgfqpoint{1.699141in}{1.633409in}}{\pgfqpoint{1.707041in}{1.636681in}}{\pgfqpoint{1.712865in}{1.642505in}}%
\pgfpathcurveto{\pgfqpoint{1.718689in}{1.648329in}}{\pgfqpoint{1.721961in}{1.656229in}}{\pgfqpoint{1.721961in}{1.664465in}}%
\pgfpathcurveto{\pgfqpoint{1.721961in}{1.672701in}}{\pgfqpoint{1.718689in}{1.680602in}}{\pgfqpoint{1.712865in}{1.686425in}}%
\pgfpathcurveto{\pgfqpoint{1.707041in}{1.692249in}}{\pgfqpoint{1.699141in}{1.695522in}}{\pgfqpoint{1.690905in}{1.695522in}}%
\pgfpathcurveto{\pgfqpoint{1.682669in}{1.695522in}}{\pgfqpoint{1.674769in}{1.692249in}}{\pgfqpoint{1.668945in}{1.686425in}}%
\pgfpathcurveto{\pgfqpoint{1.663121in}{1.680602in}}{\pgfqpoint{1.659848in}{1.672701in}}{\pgfqpoint{1.659848in}{1.664465in}}%
\pgfpathcurveto{\pgfqpoint{1.659848in}{1.656229in}}{\pgfqpoint{1.663121in}{1.648329in}}{\pgfqpoint{1.668945in}{1.642505in}}%
\pgfpathcurveto{\pgfqpoint{1.674769in}{1.636681in}}{\pgfqpoint{1.682669in}{1.633409in}}{\pgfqpoint{1.690905in}{1.633409in}}%
\pgfpathclose%
\pgfusepath{stroke,fill}%
\end{pgfscope}%
\begin{pgfscope}%
\pgfpathrectangle{\pgfqpoint{0.100000in}{0.212622in}}{\pgfqpoint{3.696000in}{3.696000in}}%
\pgfusepath{clip}%
\pgfsetbuttcap%
\pgfsetroundjoin%
\definecolor{currentfill}{rgb}{0.121569,0.466667,0.705882}%
\pgfsetfillcolor{currentfill}%
\pgfsetfillopacity{0.782660}%
\pgfsetlinewidth{1.003750pt}%
\definecolor{currentstroke}{rgb}{0.121569,0.466667,0.705882}%
\pgfsetstrokecolor{currentstroke}%
\pgfsetstrokeopacity{0.782660}%
\pgfsetdash{}{0pt}%
\pgfpathmoveto{\pgfqpoint{1.692215in}{1.632026in}}%
\pgfpathcurveto{\pgfqpoint{1.700451in}{1.632026in}}{\pgfqpoint{1.708351in}{1.635298in}}{\pgfqpoint{1.714175in}{1.641122in}}%
\pgfpathcurveto{\pgfqpoint{1.719999in}{1.646946in}}{\pgfqpoint{1.723271in}{1.654846in}}{\pgfqpoint{1.723271in}{1.663083in}}%
\pgfpathcurveto{\pgfqpoint{1.723271in}{1.671319in}}{\pgfqpoint{1.719999in}{1.679219in}}{\pgfqpoint{1.714175in}{1.685043in}}%
\pgfpathcurveto{\pgfqpoint{1.708351in}{1.690867in}}{\pgfqpoint{1.700451in}{1.694139in}}{\pgfqpoint{1.692215in}{1.694139in}}%
\pgfpathcurveto{\pgfqpoint{1.683978in}{1.694139in}}{\pgfqpoint{1.676078in}{1.690867in}}{\pgfqpoint{1.670254in}{1.685043in}}%
\pgfpathcurveto{\pgfqpoint{1.664430in}{1.679219in}}{\pgfqpoint{1.661158in}{1.671319in}}{\pgfqpoint{1.661158in}{1.663083in}}%
\pgfpathcurveto{\pgfqpoint{1.661158in}{1.654846in}}{\pgfqpoint{1.664430in}{1.646946in}}{\pgfqpoint{1.670254in}{1.641122in}}%
\pgfpathcurveto{\pgfqpoint{1.676078in}{1.635298in}}{\pgfqpoint{1.683978in}{1.632026in}}{\pgfqpoint{1.692215in}{1.632026in}}%
\pgfpathclose%
\pgfusepath{stroke,fill}%
\end{pgfscope}%
\begin{pgfscope}%
\pgfpathrectangle{\pgfqpoint{0.100000in}{0.212622in}}{\pgfqpoint{3.696000in}{3.696000in}}%
\pgfusepath{clip}%
\pgfsetbuttcap%
\pgfsetroundjoin%
\definecolor{currentfill}{rgb}{0.121569,0.466667,0.705882}%
\pgfsetfillcolor{currentfill}%
\pgfsetfillopacity{0.784159}%
\pgfsetlinewidth{1.003750pt}%
\definecolor{currentstroke}{rgb}{0.121569,0.466667,0.705882}%
\pgfsetstrokecolor{currentstroke}%
\pgfsetstrokeopacity{0.784159}%
\pgfsetdash{}{0pt}%
\pgfpathmoveto{\pgfqpoint{1.693129in}{1.632749in}}%
\pgfpathcurveto{\pgfqpoint{1.701365in}{1.632749in}}{\pgfqpoint{1.709265in}{1.636021in}}{\pgfqpoint{1.715089in}{1.641845in}}%
\pgfpathcurveto{\pgfqpoint{1.720913in}{1.647669in}}{\pgfqpoint{1.724185in}{1.655569in}}{\pgfqpoint{1.724185in}{1.663805in}}%
\pgfpathcurveto{\pgfqpoint{1.724185in}{1.672042in}}{\pgfqpoint{1.720913in}{1.679942in}}{\pgfqpoint{1.715089in}{1.685766in}}%
\pgfpathcurveto{\pgfqpoint{1.709265in}{1.691590in}}{\pgfqpoint{1.701365in}{1.694862in}}{\pgfqpoint{1.693129in}{1.694862in}}%
\pgfpathcurveto{\pgfqpoint{1.684892in}{1.694862in}}{\pgfqpoint{1.676992in}{1.691590in}}{\pgfqpoint{1.671168in}{1.685766in}}%
\pgfpathcurveto{\pgfqpoint{1.665345in}{1.679942in}}{\pgfqpoint{1.662072in}{1.672042in}}{\pgfqpoint{1.662072in}{1.663805in}}%
\pgfpathcurveto{\pgfqpoint{1.662072in}{1.655569in}}{\pgfqpoint{1.665345in}{1.647669in}}{\pgfqpoint{1.671168in}{1.641845in}}%
\pgfpathcurveto{\pgfqpoint{1.676992in}{1.636021in}}{\pgfqpoint{1.684892in}{1.632749in}}{\pgfqpoint{1.693129in}{1.632749in}}%
\pgfpathclose%
\pgfusepath{stroke,fill}%
\end{pgfscope}%
\begin{pgfscope}%
\pgfpathrectangle{\pgfqpoint{0.100000in}{0.212622in}}{\pgfqpoint{3.696000in}{3.696000in}}%
\pgfusepath{clip}%
\pgfsetbuttcap%
\pgfsetroundjoin%
\definecolor{currentfill}{rgb}{0.121569,0.466667,0.705882}%
\pgfsetfillcolor{currentfill}%
\pgfsetfillopacity{0.784701}%
\pgfsetlinewidth{1.003750pt}%
\definecolor{currentstroke}{rgb}{0.121569,0.466667,0.705882}%
\pgfsetstrokecolor{currentstroke}%
\pgfsetstrokeopacity{0.784701}%
\pgfsetdash{}{0pt}%
\pgfpathmoveto{\pgfqpoint{1.693380in}{1.631753in}}%
\pgfpathcurveto{\pgfqpoint{1.701616in}{1.631753in}}{\pgfqpoint{1.709516in}{1.635026in}}{\pgfqpoint{1.715340in}{1.640850in}}%
\pgfpathcurveto{\pgfqpoint{1.721164in}{1.646674in}}{\pgfqpoint{1.724436in}{1.654574in}}{\pgfqpoint{1.724436in}{1.662810in}}%
\pgfpathcurveto{\pgfqpoint{1.724436in}{1.671046in}}{\pgfqpoint{1.721164in}{1.678946in}}{\pgfqpoint{1.715340in}{1.684770in}}%
\pgfpathcurveto{\pgfqpoint{1.709516in}{1.690594in}}{\pgfqpoint{1.701616in}{1.693866in}}{\pgfqpoint{1.693380in}{1.693866in}}%
\pgfpathcurveto{\pgfqpoint{1.685144in}{1.693866in}}{\pgfqpoint{1.677244in}{1.690594in}}{\pgfqpoint{1.671420in}{1.684770in}}%
\pgfpathcurveto{\pgfqpoint{1.665596in}{1.678946in}}{\pgfqpoint{1.662323in}{1.671046in}}{\pgfqpoint{1.662323in}{1.662810in}}%
\pgfpathcurveto{\pgfqpoint{1.662323in}{1.654574in}}{\pgfqpoint{1.665596in}{1.646674in}}{\pgfqpoint{1.671420in}{1.640850in}}%
\pgfpathcurveto{\pgfqpoint{1.677244in}{1.635026in}}{\pgfqpoint{1.685144in}{1.631753in}}{\pgfqpoint{1.693380in}{1.631753in}}%
\pgfpathclose%
\pgfusepath{stroke,fill}%
\end{pgfscope}%
\begin{pgfscope}%
\pgfpathrectangle{\pgfqpoint{0.100000in}{0.212622in}}{\pgfqpoint{3.696000in}{3.696000in}}%
\pgfusepath{clip}%
\pgfsetbuttcap%
\pgfsetroundjoin%
\definecolor{currentfill}{rgb}{0.121569,0.466667,0.705882}%
\pgfsetfillcolor{currentfill}%
\pgfsetfillopacity{0.785490}%
\pgfsetlinewidth{1.003750pt}%
\definecolor{currentstroke}{rgb}{0.121569,0.466667,0.705882}%
\pgfsetstrokecolor{currentstroke}%
\pgfsetstrokeopacity{0.785490}%
\pgfsetdash{}{0pt}%
\pgfpathmoveto{\pgfqpoint{1.694209in}{1.630032in}}%
\pgfpathcurveto{\pgfqpoint{1.702445in}{1.630032in}}{\pgfqpoint{1.710346in}{1.633304in}}{\pgfqpoint{1.716169in}{1.639128in}}%
\pgfpathcurveto{\pgfqpoint{1.721993in}{1.644952in}}{\pgfqpoint{1.725266in}{1.652852in}}{\pgfqpoint{1.725266in}{1.661089in}}%
\pgfpathcurveto{\pgfqpoint{1.725266in}{1.669325in}}{\pgfqpoint{1.721993in}{1.677225in}}{\pgfqpoint{1.716169in}{1.683049in}}%
\pgfpathcurveto{\pgfqpoint{1.710346in}{1.688873in}}{\pgfqpoint{1.702445in}{1.692145in}}{\pgfqpoint{1.694209in}{1.692145in}}%
\pgfpathcurveto{\pgfqpoint{1.685973in}{1.692145in}}{\pgfqpoint{1.678073in}{1.688873in}}{\pgfqpoint{1.672249in}{1.683049in}}%
\pgfpathcurveto{\pgfqpoint{1.666425in}{1.677225in}}{\pgfqpoint{1.663153in}{1.669325in}}{\pgfqpoint{1.663153in}{1.661089in}}%
\pgfpathcurveto{\pgfqpoint{1.663153in}{1.652852in}}{\pgfqpoint{1.666425in}{1.644952in}}{\pgfqpoint{1.672249in}{1.639128in}}%
\pgfpathcurveto{\pgfqpoint{1.678073in}{1.633304in}}{\pgfqpoint{1.685973in}{1.630032in}}{\pgfqpoint{1.694209in}{1.630032in}}%
\pgfpathclose%
\pgfusepath{stroke,fill}%
\end{pgfscope}%
\begin{pgfscope}%
\pgfpathrectangle{\pgfqpoint{0.100000in}{0.212622in}}{\pgfqpoint{3.696000in}{3.696000in}}%
\pgfusepath{clip}%
\pgfsetbuttcap%
\pgfsetroundjoin%
\definecolor{currentfill}{rgb}{0.121569,0.466667,0.705882}%
\pgfsetfillcolor{currentfill}%
\pgfsetfillopacity{0.786049}%
\pgfsetlinewidth{1.003750pt}%
\definecolor{currentstroke}{rgb}{0.121569,0.466667,0.705882}%
\pgfsetstrokecolor{currentstroke}%
\pgfsetstrokeopacity{0.786049}%
\pgfsetdash{}{0pt}%
\pgfpathmoveto{\pgfqpoint{1.694544in}{1.629576in}}%
\pgfpathcurveto{\pgfqpoint{1.702780in}{1.629576in}}{\pgfqpoint{1.710680in}{1.632848in}}{\pgfqpoint{1.716504in}{1.638672in}}%
\pgfpathcurveto{\pgfqpoint{1.722328in}{1.644496in}}{\pgfqpoint{1.725600in}{1.652396in}}{\pgfqpoint{1.725600in}{1.660632in}}%
\pgfpathcurveto{\pgfqpoint{1.725600in}{1.668868in}}{\pgfqpoint{1.722328in}{1.676768in}}{\pgfqpoint{1.716504in}{1.682592in}}%
\pgfpathcurveto{\pgfqpoint{1.710680in}{1.688416in}}{\pgfqpoint{1.702780in}{1.691689in}}{\pgfqpoint{1.694544in}{1.691689in}}%
\pgfpathcurveto{\pgfqpoint{1.686308in}{1.691689in}}{\pgfqpoint{1.678407in}{1.688416in}}{\pgfqpoint{1.672584in}{1.682592in}}%
\pgfpathcurveto{\pgfqpoint{1.666760in}{1.676768in}}{\pgfqpoint{1.663487in}{1.668868in}}{\pgfqpoint{1.663487in}{1.660632in}}%
\pgfpathcurveto{\pgfqpoint{1.663487in}{1.652396in}}{\pgfqpoint{1.666760in}{1.644496in}}{\pgfqpoint{1.672584in}{1.638672in}}%
\pgfpathcurveto{\pgfqpoint{1.678407in}{1.632848in}}{\pgfqpoint{1.686308in}{1.629576in}}{\pgfqpoint{1.694544in}{1.629576in}}%
\pgfpathclose%
\pgfusepath{stroke,fill}%
\end{pgfscope}%
\begin{pgfscope}%
\pgfpathrectangle{\pgfqpoint{0.100000in}{0.212622in}}{\pgfqpoint{3.696000in}{3.696000in}}%
\pgfusepath{clip}%
\pgfsetbuttcap%
\pgfsetroundjoin%
\definecolor{currentfill}{rgb}{0.121569,0.466667,0.705882}%
\pgfsetfillcolor{currentfill}%
\pgfsetfillopacity{0.786969}%
\pgfsetlinewidth{1.003750pt}%
\definecolor{currentstroke}{rgb}{0.121569,0.466667,0.705882}%
\pgfsetstrokecolor{currentstroke}%
\pgfsetstrokeopacity{0.786969}%
\pgfsetdash{}{0pt}%
\pgfpathmoveto{\pgfqpoint{1.695176in}{1.629936in}}%
\pgfpathcurveto{\pgfqpoint{1.703412in}{1.629936in}}{\pgfqpoint{1.711312in}{1.633208in}}{\pgfqpoint{1.717136in}{1.639032in}}%
\pgfpathcurveto{\pgfqpoint{1.722960in}{1.644856in}}{\pgfqpoint{1.726232in}{1.652756in}}{\pgfqpoint{1.726232in}{1.660993in}}%
\pgfpathcurveto{\pgfqpoint{1.726232in}{1.669229in}}{\pgfqpoint{1.722960in}{1.677129in}}{\pgfqpoint{1.717136in}{1.682953in}}%
\pgfpathcurveto{\pgfqpoint{1.711312in}{1.688777in}}{\pgfqpoint{1.703412in}{1.692049in}}{\pgfqpoint{1.695176in}{1.692049in}}%
\pgfpathcurveto{\pgfqpoint{1.686940in}{1.692049in}}{\pgfqpoint{1.679040in}{1.688777in}}{\pgfqpoint{1.673216in}{1.682953in}}%
\pgfpathcurveto{\pgfqpoint{1.667392in}{1.677129in}}{\pgfqpoint{1.664119in}{1.669229in}}{\pgfqpoint{1.664119in}{1.660993in}}%
\pgfpathcurveto{\pgfqpoint{1.664119in}{1.652756in}}{\pgfqpoint{1.667392in}{1.644856in}}{\pgfqpoint{1.673216in}{1.639032in}}%
\pgfpathcurveto{\pgfqpoint{1.679040in}{1.633208in}}{\pgfqpoint{1.686940in}{1.629936in}}{\pgfqpoint{1.695176in}{1.629936in}}%
\pgfpathclose%
\pgfusepath{stroke,fill}%
\end{pgfscope}%
\begin{pgfscope}%
\pgfpathrectangle{\pgfqpoint{0.100000in}{0.212622in}}{\pgfqpoint{3.696000in}{3.696000in}}%
\pgfusepath{clip}%
\pgfsetbuttcap%
\pgfsetroundjoin%
\definecolor{currentfill}{rgb}{0.121569,0.466667,0.705882}%
\pgfsetfillcolor{currentfill}%
\pgfsetfillopacity{0.787352}%
\pgfsetlinewidth{1.003750pt}%
\definecolor{currentstroke}{rgb}{0.121569,0.466667,0.705882}%
\pgfsetstrokecolor{currentstroke}%
\pgfsetstrokeopacity{0.787352}%
\pgfsetdash{}{0pt}%
\pgfpathmoveto{\pgfqpoint{1.695406in}{1.629513in}}%
\pgfpathcurveto{\pgfqpoint{1.703642in}{1.629513in}}{\pgfqpoint{1.711542in}{1.632785in}}{\pgfqpoint{1.717366in}{1.638609in}}%
\pgfpathcurveto{\pgfqpoint{1.723190in}{1.644433in}}{\pgfqpoint{1.726463in}{1.652333in}}{\pgfqpoint{1.726463in}{1.660569in}}%
\pgfpathcurveto{\pgfqpoint{1.726463in}{1.668805in}}{\pgfqpoint{1.723190in}{1.676705in}}{\pgfqpoint{1.717366in}{1.682529in}}%
\pgfpathcurveto{\pgfqpoint{1.711542in}{1.688353in}}{\pgfqpoint{1.703642in}{1.691626in}}{\pgfqpoint{1.695406in}{1.691626in}}%
\pgfpathcurveto{\pgfqpoint{1.687170in}{1.691626in}}{\pgfqpoint{1.679270in}{1.688353in}}{\pgfqpoint{1.673446in}{1.682529in}}%
\pgfpathcurveto{\pgfqpoint{1.667622in}{1.676705in}}{\pgfqpoint{1.664350in}{1.668805in}}{\pgfqpoint{1.664350in}{1.660569in}}%
\pgfpathcurveto{\pgfqpoint{1.664350in}{1.652333in}}{\pgfqpoint{1.667622in}{1.644433in}}{\pgfqpoint{1.673446in}{1.638609in}}%
\pgfpathcurveto{\pgfqpoint{1.679270in}{1.632785in}}{\pgfqpoint{1.687170in}{1.629513in}}{\pgfqpoint{1.695406in}{1.629513in}}%
\pgfpathclose%
\pgfusepath{stroke,fill}%
\end{pgfscope}%
\begin{pgfscope}%
\pgfpathrectangle{\pgfqpoint{0.100000in}{0.212622in}}{\pgfqpoint{3.696000in}{3.696000in}}%
\pgfusepath{clip}%
\pgfsetbuttcap%
\pgfsetroundjoin%
\definecolor{currentfill}{rgb}{0.121569,0.466667,0.705882}%
\pgfsetfillcolor{currentfill}%
\pgfsetfillopacity{0.788107}%
\pgfsetlinewidth{1.003750pt}%
\definecolor{currentstroke}{rgb}{0.121569,0.466667,0.705882}%
\pgfsetstrokecolor{currentstroke}%
\pgfsetstrokeopacity{0.788107}%
\pgfsetdash{}{0pt}%
\pgfpathmoveto{\pgfqpoint{1.696092in}{1.627394in}}%
\pgfpathcurveto{\pgfqpoint{1.704328in}{1.627394in}}{\pgfqpoint{1.712229in}{1.630666in}}{\pgfqpoint{1.718052in}{1.636490in}}%
\pgfpathcurveto{\pgfqpoint{1.723876in}{1.642314in}}{\pgfqpoint{1.727149in}{1.650214in}}{\pgfqpoint{1.727149in}{1.658451in}}%
\pgfpathcurveto{\pgfqpoint{1.727149in}{1.666687in}}{\pgfqpoint{1.723876in}{1.674587in}}{\pgfqpoint{1.718052in}{1.680411in}}%
\pgfpathcurveto{\pgfqpoint{1.712229in}{1.686235in}}{\pgfqpoint{1.704328in}{1.689507in}}{\pgfqpoint{1.696092in}{1.689507in}}%
\pgfpathcurveto{\pgfqpoint{1.687856in}{1.689507in}}{\pgfqpoint{1.679956in}{1.686235in}}{\pgfqpoint{1.674132in}{1.680411in}}%
\pgfpathcurveto{\pgfqpoint{1.668308in}{1.674587in}}{\pgfqpoint{1.665036in}{1.666687in}}{\pgfqpoint{1.665036in}{1.658451in}}%
\pgfpathcurveto{\pgfqpoint{1.665036in}{1.650214in}}{\pgfqpoint{1.668308in}{1.642314in}}{\pgfqpoint{1.674132in}{1.636490in}}%
\pgfpathcurveto{\pgfqpoint{1.679956in}{1.630666in}}{\pgfqpoint{1.687856in}{1.627394in}}{\pgfqpoint{1.696092in}{1.627394in}}%
\pgfpathclose%
\pgfusepath{stroke,fill}%
\end{pgfscope}%
\begin{pgfscope}%
\pgfpathrectangle{\pgfqpoint{0.100000in}{0.212622in}}{\pgfqpoint{3.696000in}{3.696000in}}%
\pgfusepath{clip}%
\pgfsetbuttcap%
\pgfsetroundjoin%
\definecolor{currentfill}{rgb}{0.121569,0.466667,0.705882}%
\pgfsetfillcolor{currentfill}%
\pgfsetfillopacity{0.789382}%
\pgfsetlinewidth{1.003750pt}%
\definecolor{currentstroke}{rgb}{0.121569,0.466667,0.705882}%
\pgfsetstrokecolor{currentstroke}%
\pgfsetstrokeopacity{0.789382}%
\pgfsetdash{}{0pt}%
\pgfpathmoveto{\pgfqpoint{1.696833in}{1.626749in}}%
\pgfpathcurveto{\pgfqpoint{1.705070in}{1.626749in}}{\pgfqpoint{1.712970in}{1.630021in}}{\pgfqpoint{1.718794in}{1.635845in}}%
\pgfpathcurveto{\pgfqpoint{1.724618in}{1.641669in}}{\pgfqpoint{1.727890in}{1.649569in}}{\pgfqpoint{1.727890in}{1.657806in}}%
\pgfpathcurveto{\pgfqpoint{1.727890in}{1.666042in}}{\pgfqpoint{1.724618in}{1.673942in}}{\pgfqpoint{1.718794in}{1.679766in}}%
\pgfpathcurveto{\pgfqpoint{1.712970in}{1.685590in}}{\pgfqpoint{1.705070in}{1.688862in}}{\pgfqpoint{1.696833in}{1.688862in}}%
\pgfpathcurveto{\pgfqpoint{1.688597in}{1.688862in}}{\pgfqpoint{1.680697in}{1.685590in}}{\pgfqpoint{1.674873in}{1.679766in}}%
\pgfpathcurveto{\pgfqpoint{1.669049in}{1.673942in}}{\pgfqpoint{1.665777in}{1.666042in}}{\pgfqpoint{1.665777in}{1.657806in}}%
\pgfpathcurveto{\pgfqpoint{1.665777in}{1.649569in}}{\pgfqpoint{1.669049in}{1.641669in}}{\pgfqpoint{1.674873in}{1.635845in}}%
\pgfpathcurveto{\pgfqpoint{1.680697in}{1.630021in}}{\pgfqpoint{1.688597in}{1.626749in}}{\pgfqpoint{1.696833in}{1.626749in}}%
\pgfpathclose%
\pgfusepath{stroke,fill}%
\end{pgfscope}%
\begin{pgfscope}%
\pgfpathrectangle{\pgfqpoint{0.100000in}{0.212622in}}{\pgfqpoint{3.696000in}{3.696000in}}%
\pgfusepath{clip}%
\pgfsetbuttcap%
\pgfsetroundjoin%
\definecolor{currentfill}{rgb}{0.121569,0.466667,0.705882}%
\pgfsetfillcolor{currentfill}%
\pgfsetfillopacity{0.791357}%
\pgfsetlinewidth{1.003750pt}%
\definecolor{currentstroke}{rgb}{0.121569,0.466667,0.705882}%
\pgfsetstrokecolor{currentstroke}%
\pgfsetstrokeopacity{0.791357}%
\pgfsetdash{}{0pt}%
\pgfpathmoveto{\pgfqpoint{1.698492in}{1.626733in}}%
\pgfpathcurveto{\pgfqpoint{1.706729in}{1.626733in}}{\pgfqpoint{1.714629in}{1.630005in}}{\pgfqpoint{1.720453in}{1.635829in}}%
\pgfpathcurveto{\pgfqpoint{1.726277in}{1.641653in}}{\pgfqpoint{1.729549in}{1.649553in}}{\pgfqpoint{1.729549in}{1.657789in}}%
\pgfpathcurveto{\pgfqpoint{1.729549in}{1.666025in}}{\pgfqpoint{1.726277in}{1.673925in}}{\pgfqpoint{1.720453in}{1.679749in}}%
\pgfpathcurveto{\pgfqpoint{1.714629in}{1.685573in}}{\pgfqpoint{1.706729in}{1.688846in}}{\pgfqpoint{1.698492in}{1.688846in}}%
\pgfpathcurveto{\pgfqpoint{1.690256in}{1.688846in}}{\pgfqpoint{1.682356in}{1.685573in}}{\pgfqpoint{1.676532in}{1.679749in}}%
\pgfpathcurveto{\pgfqpoint{1.670708in}{1.673925in}}{\pgfqpoint{1.667436in}{1.666025in}}{\pgfqpoint{1.667436in}{1.657789in}}%
\pgfpathcurveto{\pgfqpoint{1.667436in}{1.649553in}}{\pgfqpoint{1.670708in}{1.641653in}}{\pgfqpoint{1.676532in}{1.635829in}}%
\pgfpathcurveto{\pgfqpoint{1.682356in}{1.630005in}}{\pgfqpoint{1.690256in}{1.626733in}}{\pgfqpoint{1.698492in}{1.626733in}}%
\pgfpathclose%
\pgfusepath{stroke,fill}%
\end{pgfscope}%
\begin{pgfscope}%
\pgfpathrectangle{\pgfqpoint{0.100000in}{0.212622in}}{\pgfqpoint{3.696000in}{3.696000in}}%
\pgfusepath{clip}%
\pgfsetbuttcap%
\pgfsetroundjoin%
\definecolor{currentfill}{rgb}{0.121569,0.466667,0.705882}%
\pgfsetfillcolor{currentfill}%
\pgfsetfillopacity{0.792320}%
\pgfsetlinewidth{1.003750pt}%
\definecolor{currentstroke}{rgb}{0.121569,0.466667,0.705882}%
\pgfsetstrokecolor{currentstroke}%
\pgfsetstrokeopacity{0.792320}%
\pgfsetdash{}{0pt}%
\pgfpathmoveto{\pgfqpoint{1.698711in}{1.625829in}}%
\pgfpathcurveto{\pgfqpoint{1.706947in}{1.625829in}}{\pgfqpoint{1.714847in}{1.629101in}}{\pgfqpoint{1.720671in}{1.634925in}}%
\pgfpathcurveto{\pgfqpoint{1.726495in}{1.640749in}}{\pgfqpoint{1.729767in}{1.648649in}}{\pgfqpoint{1.729767in}{1.656885in}}%
\pgfpathcurveto{\pgfqpoint{1.729767in}{1.665121in}}{\pgfqpoint{1.726495in}{1.673022in}}{\pgfqpoint{1.720671in}{1.678845in}}%
\pgfpathcurveto{\pgfqpoint{1.714847in}{1.684669in}}{\pgfqpoint{1.706947in}{1.687942in}}{\pgfqpoint{1.698711in}{1.687942in}}%
\pgfpathcurveto{\pgfqpoint{1.690475in}{1.687942in}}{\pgfqpoint{1.682575in}{1.684669in}}{\pgfqpoint{1.676751in}{1.678845in}}%
\pgfpathcurveto{\pgfqpoint{1.670927in}{1.673022in}}{\pgfqpoint{1.667654in}{1.665121in}}{\pgfqpoint{1.667654in}{1.656885in}}%
\pgfpathcurveto{\pgfqpoint{1.667654in}{1.648649in}}{\pgfqpoint{1.670927in}{1.640749in}}{\pgfqpoint{1.676751in}{1.634925in}}%
\pgfpathcurveto{\pgfqpoint{1.682575in}{1.629101in}}{\pgfqpoint{1.690475in}{1.625829in}}{\pgfqpoint{1.698711in}{1.625829in}}%
\pgfpathclose%
\pgfusepath{stroke,fill}%
\end{pgfscope}%
\begin{pgfscope}%
\pgfpathrectangle{\pgfqpoint{0.100000in}{0.212622in}}{\pgfqpoint{3.696000in}{3.696000in}}%
\pgfusepath{clip}%
\pgfsetbuttcap%
\pgfsetroundjoin%
\definecolor{currentfill}{rgb}{0.121569,0.466667,0.705882}%
\pgfsetfillcolor{currentfill}%
\pgfsetfillopacity{0.793239}%
\pgfsetlinewidth{1.003750pt}%
\definecolor{currentstroke}{rgb}{0.121569,0.466667,0.705882}%
\pgfsetstrokecolor{currentstroke}%
\pgfsetstrokeopacity{0.793239}%
\pgfsetdash{}{0pt}%
\pgfpathmoveto{\pgfqpoint{1.699733in}{1.621595in}}%
\pgfpathcurveto{\pgfqpoint{1.707969in}{1.621595in}}{\pgfqpoint{1.715869in}{1.624867in}}{\pgfqpoint{1.721693in}{1.630691in}}%
\pgfpathcurveto{\pgfqpoint{1.727517in}{1.636515in}}{\pgfqpoint{1.730789in}{1.644415in}}{\pgfqpoint{1.730789in}{1.652652in}}%
\pgfpathcurveto{\pgfqpoint{1.730789in}{1.660888in}}{\pgfqpoint{1.727517in}{1.668788in}}{\pgfqpoint{1.721693in}{1.674612in}}%
\pgfpathcurveto{\pgfqpoint{1.715869in}{1.680436in}}{\pgfqpoint{1.707969in}{1.683708in}}{\pgfqpoint{1.699733in}{1.683708in}}%
\pgfpathcurveto{\pgfqpoint{1.691497in}{1.683708in}}{\pgfqpoint{1.683597in}{1.680436in}}{\pgfqpoint{1.677773in}{1.674612in}}%
\pgfpathcurveto{\pgfqpoint{1.671949in}{1.668788in}}{\pgfqpoint{1.668676in}{1.660888in}}{\pgfqpoint{1.668676in}{1.652652in}}%
\pgfpathcurveto{\pgfqpoint{1.668676in}{1.644415in}}{\pgfqpoint{1.671949in}{1.636515in}}{\pgfqpoint{1.677773in}{1.630691in}}%
\pgfpathcurveto{\pgfqpoint{1.683597in}{1.624867in}}{\pgfqpoint{1.691497in}{1.621595in}}{\pgfqpoint{1.699733in}{1.621595in}}%
\pgfpathclose%
\pgfusepath{stroke,fill}%
\end{pgfscope}%
\begin{pgfscope}%
\pgfpathrectangle{\pgfqpoint{0.100000in}{0.212622in}}{\pgfqpoint{3.696000in}{3.696000in}}%
\pgfusepath{clip}%
\pgfsetbuttcap%
\pgfsetroundjoin%
\definecolor{currentfill}{rgb}{0.121569,0.466667,0.705882}%
\pgfsetfillcolor{currentfill}%
\pgfsetfillopacity{0.794242}%
\pgfsetlinewidth{1.003750pt}%
\definecolor{currentstroke}{rgb}{0.121569,0.466667,0.705882}%
\pgfsetstrokecolor{currentstroke}%
\pgfsetstrokeopacity{0.794242}%
\pgfsetdash{}{0pt}%
\pgfpathmoveto{\pgfqpoint{1.700388in}{1.621551in}}%
\pgfpathcurveto{\pgfqpoint{1.708625in}{1.621551in}}{\pgfqpoint{1.716525in}{1.624823in}}{\pgfqpoint{1.722349in}{1.630647in}}%
\pgfpathcurveto{\pgfqpoint{1.728173in}{1.636471in}}{\pgfqpoint{1.731445in}{1.644371in}}{\pgfqpoint{1.731445in}{1.652608in}}%
\pgfpathcurveto{\pgfqpoint{1.731445in}{1.660844in}}{\pgfqpoint{1.728173in}{1.668744in}}{\pgfqpoint{1.722349in}{1.674568in}}%
\pgfpathcurveto{\pgfqpoint{1.716525in}{1.680392in}}{\pgfqpoint{1.708625in}{1.683664in}}{\pgfqpoint{1.700388in}{1.683664in}}%
\pgfpathcurveto{\pgfqpoint{1.692152in}{1.683664in}}{\pgfqpoint{1.684252in}{1.680392in}}{\pgfqpoint{1.678428in}{1.674568in}}%
\pgfpathcurveto{\pgfqpoint{1.672604in}{1.668744in}}{\pgfqpoint{1.669332in}{1.660844in}}{\pgfqpoint{1.669332in}{1.652608in}}%
\pgfpathcurveto{\pgfqpoint{1.669332in}{1.644371in}}{\pgfqpoint{1.672604in}{1.636471in}}{\pgfqpoint{1.678428in}{1.630647in}}%
\pgfpathcurveto{\pgfqpoint{1.684252in}{1.624823in}}{\pgfqpoint{1.692152in}{1.621551in}}{\pgfqpoint{1.700388in}{1.621551in}}%
\pgfpathclose%
\pgfusepath{stroke,fill}%
\end{pgfscope}%
\begin{pgfscope}%
\pgfpathrectangle{\pgfqpoint{0.100000in}{0.212622in}}{\pgfqpoint{3.696000in}{3.696000in}}%
\pgfusepath{clip}%
\pgfsetbuttcap%
\pgfsetroundjoin%
\definecolor{currentfill}{rgb}{0.121569,0.466667,0.705882}%
\pgfsetfillcolor{currentfill}%
\pgfsetfillopacity{0.795504}%
\pgfsetlinewidth{1.003750pt}%
\definecolor{currentstroke}{rgb}{0.121569,0.466667,0.705882}%
\pgfsetstrokecolor{currentstroke}%
\pgfsetstrokeopacity{0.795504}%
\pgfsetdash{}{0pt}%
\pgfpathmoveto{\pgfqpoint{1.701087in}{1.621553in}}%
\pgfpathcurveto{\pgfqpoint{1.709324in}{1.621553in}}{\pgfqpoint{1.717224in}{1.624826in}}{\pgfqpoint{1.723048in}{1.630650in}}%
\pgfpathcurveto{\pgfqpoint{1.728871in}{1.636474in}}{\pgfqpoint{1.732144in}{1.644374in}}{\pgfqpoint{1.732144in}{1.652610in}}%
\pgfpathcurveto{\pgfqpoint{1.732144in}{1.660846in}}{\pgfqpoint{1.728871in}{1.668746in}}{\pgfqpoint{1.723048in}{1.674570in}}%
\pgfpathcurveto{\pgfqpoint{1.717224in}{1.680394in}}{\pgfqpoint{1.709324in}{1.683666in}}{\pgfqpoint{1.701087in}{1.683666in}}%
\pgfpathcurveto{\pgfqpoint{1.692851in}{1.683666in}}{\pgfqpoint{1.684951in}{1.680394in}}{\pgfqpoint{1.679127in}{1.674570in}}%
\pgfpathcurveto{\pgfqpoint{1.673303in}{1.668746in}}{\pgfqpoint{1.670031in}{1.660846in}}{\pgfqpoint{1.670031in}{1.652610in}}%
\pgfpathcurveto{\pgfqpoint{1.670031in}{1.644374in}}{\pgfqpoint{1.673303in}{1.636474in}}{\pgfqpoint{1.679127in}{1.630650in}}%
\pgfpathcurveto{\pgfqpoint{1.684951in}{1.624826in}}{\pgfqpoint{1.692851in}{1.621553in}}{\pgfqpoint{1.701087in}{1.621553in}}%
\pgfpathclose%
\pgfusepath{stroke,fill}%
\end{pgfscope}%
\begin{pgfscope}%
\pgfpathrectangle{\pgfqpoint{0.100000in}{0.212622in}}{\pgfqpoint{3.696000in}{3.696000in}}%
\pgfusepath{clip}%
\pgfsetbuttcap%
\pgfsetroundjoin%
\definecolor{currentfill}{rgb}{0.121569,0.466667,0.705882}%
\pgfsetfillcolor{currentfill}%
\pgfsetfillopacity{0.796154}%
\pgfsetlinewidth{1.003750pt}%
\definecolor{currentstroke}{rgb}{0.121569,0.466667,0.705882}%
\pgfsetstrokecolor{currentstroke}%
\pgfsetstrokeopacity{0.796154}%
\pgfsetdash{}{0pt}%
\pgfpathmoveto{\pgfqpoint{1.701303in}{1.621280in}}%
\pgfpathcurveto{\pgfqpoint{1.709539in}{1.621280in}}{\pgfqpoint{1.717439in}{1.624552in}}{\pgfqpoint{1.723263in}{1.630376in}}%
\pgfpathcurveto{\pgfqpoint{1.729087in}{1.636200in}}{\pgfqpoint{1.732359in}{1.644100in}}{\pgfqpoint{1.732359in}{1.652336in}}%
\pgfpathcurveto{\pgfqpoint{1.732359in}{1.660572in}}{\pgfqpoint{1.729087in}{1.668473in}}{\pgfqpoint{1.723263in}{1.674296in}}%
\pgfpathcurveto{\pgfqpoint{1.717439in}{1.680120in}}{\pgfqpoint{1.709539in}{1.683393in}}{\pgfqpoint{1.701303in}{1.683393in}}%
\pgfpathcurveto{\pgfqpoint{1.693066in}{1.683393in}}{\pgfqpoint{1.685166in}{1.680120in}}{\pgfqpoint{1.679343in}{1.674296in}}%
\pgfpathcurveto{\pgfqpoint{1.673519in}{1.668473in}}{\pgfqpoint{1.670246in}{1.660572in}}{\pgfqpoint{1.670246in}{1.652336in}}%
\pgfpathcurveto{\pgfqpoint{1.670246in}{1.644100in}}{\pgfqpoint{1.673519in}{1.636200in}}{\pgfqpoint{1.679343in}{1.630376in}}%
\pgfpathcurveto{\pgfqpoint{1.685166in}{1.624552in}}{\pgfqpoint{1.693066in}{1.621280in}}{\pgfqpoint{1.701303in}{1.621280in}}%
\pgfpathclose%
\pgfusepath{stroke,fill}%
\end{pgfscope}%
\begin{pgfscope}%
\pgfpathrectangle{\pgfqpoint{0.100000in}{0.212622in}}{\pgfqpoint{3.696000in}{3.696000in}}%
\pgfusepath{clip}%
\pgfsetbuttcap%
\pgfsetroundjoin%
\definecolor{currentfill}{rgb}{0.121569,0.466667,0.705882}%
\pgfsetfillcolor{currentfill}%
\pgfsetfillopacity{0.796941}%
\pgfsetlinewidth{1.003750pt}%
\definecolor{currentstroke}{rgb}{0.121569,0.466667,0.705882}%
\pgfsetstrokecolor{currentstroke}%
\pgfsetstrokeopacity{0.796941}%
\pgfsetdash{}{0pt}%
\pgfpathmoveto{\pgfqpoint{1.702045in}{1.619175in}}%
\pgfpathcurveto{\pgfqpoint{1.710281in}{1.619175in}}{\pgfqpoint{1.718181in}{1.622447in}}{\pgfqpoint{1.724005in}{1.628271in}}%
\pgfpathcurveto{\pgfqpoint{1.729829in}{1.634095in}}{\pgfqpoint{1.733101in}{1.641995in}}{\pgfqpoint{1.733101in}{1.650231in}}%
\pgfpathcurveto{\pgfqpoint{1.733101in}{1.658467in}}{\pgfqpoint{1.729829in}{1.666367in}}{\pgfqpoint{1.724005in}{1.672191in}}%
\pgfpathcurveto{\pgfqpoint{1.718181in}{1.678015in}}{\pgfqpoint{1.710281in}{1.681288in}}{\pgfqpoint{1.702045in}{1.681288in}}%
\pgfpathcurveto{\pgfqpoint{1.693808in}{1.681288in}}{\pgfqpoint{1.685908in}{1.678015in}}{\pgfqpoint{1.680084in}{1.672191in}}%
\pgfpathcurveto{\pgfqpoint{1.674260in}{1.666367in}}{\pgfqpoint{1.670988in}{1.658467in}}{\pgfqpoint{1.670988in}{1.650231in}}%
\pgfpathcurveto{\pgfqpoint{1.670988in}{1.641995in}}{\pgfqpoint{1.674260in}{1.634095in}}{\pgfqpoint{1.680084in}{1.628271in}}%
\pgfpathcurveto{\pgfqpoint{1.685908in}{1.622447in}}{\pgfqpoint{1.693808in}{1.619175in}}{\pgfqpoint{1.702045in}{1.619175in}}%
\pgfpathclose%
\pgfusepath{stroke,fill}%
\end{pgfscope}%
\begin{pgfscope}%
\pgfpathrectangle{\pgfqpoint{0.100000in}{0.212622in}}{\pgfqpoint{3.696000in}{3.696000in}}%
\pgfusepath{clip}%
\pgfsetbuttcap%
\pgfsetroundjoin%
\definecolor{currentfill}{rgb}{0.121569,0.466667,0.705882}%
\pgfsetfillcolor{currentfill}%
\pgfsetfillopacity{0.797565}%
\pgfsetlinewidth{1.003750pt}%
\definecolor{currentstroke}{rgb}{0.121569,0.466667,0.705882}%
\pgfsetstrokecolor{currentstroke}%
\pgfsetstrokeopacity{0.797565}%
\pgfsetdash{}{0pt}%
\pgfpathmoveto{\pgfqpoint{1.702505in}{1.618905in}}%
\pgfpathcurveto{\pgfqpoint{1.710741in}{1.618905in}}{\pgfqpoint{1.718641in}{1.622177in}}{\pgfqpoint{1.724465in}{1.628001in}}%
\pgfpathcurveto{\pgfqpoint{1.730289in}{1.633825in}}{\pgfqpoint{1.733562in}{1.641725in}}{\pgfqpoint{1.733562in}{1.649961in}}%
\pgfpathcurveto{\pgfqpoint{1.733562in}{1.658198in}}{\pgfqpoint{1.730289in}{1.666098in}}{\pgfqpoint{1.724465in}{1.671922in}}%
\pgfpathcurveto{\pgfqpoint{1.718641in}{1.677746in}}{\pgfqpoint{1.710741in}{1.681018in}}{\pgfqpoint{1.702505in}{1.681018in}}%
\pgfpathcurveto{\pgfqpoint{1.694269in}{1.681018in}}{\pgfqpoint{1.686369in}{1.677746in}}{\pgfqpoint{1.680545in}{1.671922in}}%
\pgfpathcurveto{\pgfqpoint{1.674721in}{1.666098in}}{\pgfqpoint{1.671449in}{1.658198in}}{\pgfqpoint{1.671449in}{1.649961in}}%
\pgfpathcurveto{\pgfqpoint{1.671449in}{1.641725in}}{\pgfqpoint{1.674721in}{1.633825in}}{\pgfqpoint{1.680545in}{1.628001in}}%
\pgfpathcurveto{\pgfqpoint{1.686369in}{1.622177in}}{\pgfqpoint{1.694269in}{1.618905in}}{\pgfqpoint{1.702505in}{1.618905in}}%
\pgfpathclose%
\pgfusepath{stroke,fill}%
\end{pgfscope}%
\begin{pgfscope}%
\pgfpathrectangle{\pgfqpoint{0.100000in}{0.212622in}}{\pgfqpoint{3.696000in}{3.696000in}}%
\pgfusepath{clip}%
\pgfsetbuttcap%
\pgfsetroundjoin%
\definecolor{currentfill}{rgb}{0.121569,0.466667,0.705882}%
\pgfsetfillcolor{currentfill}%
\pgfsetfillopacity{0.798535}%
\pgfsetlinewidth{1.003750pt}%
\definecolor{currentstroke}{rgb}{0.121569,0.466667,0.705882}%
\pgfsetstrokecolor{currentstroke}%
\pgfsetstrokeopacity{0.798535}%
\pgfsetdash{}{0pt}%
\pgfpathmoveto{\pgfqpoint{1.703055in}{1.619335in}}%
\pgfpathcurveto{\pgfqpoint{1.711291in}{1.619335in}}{\pgfqpoint{1.719191in}{1.622607in}}{\pgfqpoint{1.725015in}{1.628431in}}%
\pgfpathcurveto{\pgfqpoint{1.730839in}{1.634255in}}{\pgfqpoint{1.734111in}{1.642155in}}{\pgfqpoint{1.734111in}{1.650391in}}%
\pgfpathcurveto{\pgfqpoint{1.734111in}{1.658627in}}{\pgfqpoint{1.730839in}{1.666527in}}{\pgfqpoint{1.725015in}{1.672351in}}%
\pgfpathcurveto{\pgfqpoint{1.719191in}{1.678175in}}{\pgfqpoint{1.711291in}{1.681448in}}{\pgfqpoint{1.703055in}{1.681448in}}%
\pgfpathcurveto{\pgfqpoint{1.694818in}{1.681448in}}{\pgfqpoint{1.686918in}{1.678175in}}{\pgfqpoint{1.681094in}{1.672351in}}%
\pgfpathcurveto{\pgfqpoint{1.675270in}{1.666527in}}{\pgfqpoint{1.671998in}{1.658627in}}{\pgfqpoint{1.671998in}{1.650391in}}%
\pgfpathcurveto{\pgfqpoint{1.671998in}{1.642155in}}{\pgfqpoint{1.675270in}{1.634255in}}{\pgfqpoint{1.681094in}{1.628431in}}%
\pgfpathcurveto{\pgfqpoint{1.686918in}{1.622607in}}{\pgfqpoint{1.694818in}{1.619335in}}{\pgfqpoint{1.703055in}{1.619335in}}%
\pgfpathclose%
\pgfusepath{stroke,fill}%
\end{pgfscope}%
\begin{pgfscope}%
\pgfpathrectangle{\pgfqpoint{0.100000in}{0.212622in}}{\pgfqpoint{3.696000in}{3.696000in}}%
\pgfusepath{clip}%
\pgfsetbuttcap%
\pgfsetroundjoin%
\definecolor{currentfill}{rgb}{0.121569,0.466667,0.705882}%
\pgfsetfillcolor{currentfill}%
\pgfsetfillopacity{0.798963}%
\pgfsetlinewidth{1.003750pt}%
\definecolor{currentstroke}{rgb}{0.121569,0.466667,0.705882}%
\pgfsetstrokecolor{currentstroke}%
\pgfsetstrokeopacity{0.798963}%
\pgfsetdash{}{0pt}%
\pgfpathmoveto{\pgfqpoint{1.703230in}{1.619038in}}%
\pgfpathcurveto{\pgfqpoint{1.711466in}{1.619038in}}{\pgfqpoint{1.719366in}{1.622310in}}{\pgfqpoint{1.725190in}{1.628134in}}%
\pgfpathcurveto{\pgfqpoint{1.731014in}{1.633958in}}{\pgfqpoint{1.734287in}{1.641858in}}{\pgfqpoint{1.734287in}{1.650094in}}%
\pgfpathcurveto{\pgfqpoint{1.734287in}{1.658330in}}{\pgfqpoint{1.731014in}{1.666230in}}{\pgfqpoint{1.725190in}{1.672054in}}%
\pgfpathcurveto{\pgfqpoint{1.719366in}{1.677878in}}{\pgfqpoint{1.711466in}{1.681151in}}{\pgfqpoint{1.703230in}{1.681151in}}%
\pgfpathcurveto{\pgfqpoint{1.694994in}{1.681151in}}{\pgfqpoint{1.687094in}{1.677878in}}{\pgfqpoint{1.681270in}{1.672054in}}%
\pgfpathcurveto{\pgfqpoint{1.675446in}{1.666230in}}{\pgfqpoint{1.672174in}{1.658330in}}{\pgfqpoint{1.672174in}{1.650094in}}%
\pgfpathcurveto{\pgfqpoint{1.672174in}{1.641858in}}{\pgfqpoint{1.675446in}{1.633958in}}{\pgfqpoint{1.681270in}{1.628134in}}%
\pgfpathcurveto{\pgfqpoint{1.687094in}{1.622310in}}{\pgfqpoint{1.694994in}{1.619038in}}{\pgfqpoint{1.703230in}{1.619038in}}%
\pgfpathclose%
\pgfusepath{stroke,fill}%
\end{pgfscope}%
\begin{pgfscope}%
\pgfpathrectangle{\pgfqpoint{0.100000in}{0.212622in}}{\pgfqpoint{3.696000in}{3.696000in}}%
\pgfusepath{clip}%
\pgfsetbuttcap%
\pgfsetroundjoin%
\definecolor{currentfill}{rgb}{0.121569,0.466667,0.705882}%
\pgfsetfillcolor{currentfill}%
\pgfsetfillopacity{0.799667}%
\pgfsetlinewidth{1.003750pt}%
\definecolor{currentstroke}{rgb}{0.121569,0.466667,0.705882}%
\pgfsetstrokecolor{currentstroke}%
\pgfsetstrokeopacity{0.799667}%
\pgfsetdash{}{0pt}%
\pgfpathmoveto{\pgfqpoint{1.703672in}{1.618527in}}%
\pgfpathcurveto{\pgfqpoint{1.711908in}{1.618527in}}{\pgfqpoint{1.719808in}{1.621799in}}{\pgfqpoint{1.725632in}{1.627623in}}%
\pgfpathcurveto{\pgfqpoint{1.731456in}{1.633447in}}{\pgfqpoint{1.734728in}{1.641347in}}{\pgfqpoint{1.734728in}{1.649584in}}%
\pgfpathcurveto{\pgfqpoint{1.734728in}{1.657820in}}{\pgfqpoint{1.731456in}{1.665720in}}{\pgfqpoint{1.725632in}{1.671544in}}%
\pgfpathcurveto{\pgfqpoint{1.719808in}{1.677368in}}{\pgfqpoint{1.711908in}{1.680640in}}{\pgfqpoint{1.703672in}{1.680640in}}%
\pgfpathcurveto{\pgfqpoint{1.695436in}{1.680640in}}{\pgfqpoint{1.687536in}{1.677368in}}{\pgfqpoint{1.681712in}{1.671544in}}%
\pgfpathcurveto{\pgfqpoint{1.675888in}{1.665720in}}{\pgfqpoint{1.672615in}{1.657820in}}{\pgfqpoint{1.672615in}{1.649584in}}%
\pgfpathcurveto{\pgfqpoint{1.672615in}{1.641347in}}{\pgfqpoint{1.675888in}{1.633447in}}{\pgfqpoint{1.681712in}{1.627623in}}%
\pgfpathcurveto{\pgfqpoint{1.687536in}{1.621799in}}{\pgfqpoint{1.695436in}{1.618527in}}{\pgfqpoint{1.703672in}{1.618527in}}%
\pgfpathclose%
\pgfusepath{stroke,fill}%
\end{pgfscope}%
\begin{pgfscope}%
\pgfpathrectangle{\pgfqpoint{0.100000in}{0.212622in}}{\pgfqpoint{3.696000in}{3.696000in}}%
\pgfusepath{clip}%
\pgfsetbuttcap%
\pgfsetroundjoin%
\definecolor{currentfill}{rgb}{0.121569,0.466667,0.705882}%
\pgfsetfillcolor{currentfill}%
\pgfsetfillopacity{0.800124}%
\pgfsetlinewidth{1.003750pt}%
\definecolor{currentstroke}{rgb}{0.121569,0.466667,0.705882}%
\pgfsetstrokecolor{currentstroke}%
\pgfsetstrokeopacity{0.800124}%
\pgfsetdash{}{0pt}%
\pgfpathmoveto{\pgfqpoint{1.704390in}{1.616293in}}%
\pgfpathcurveto{\pgfqpoint{1.712626in}{1.616293in}}{\pgfqpoint{1.720527in}{1.619565in}}{\pgfqpoint{1.726350in}{1.625389in}}%
\pgfpathcurveto{\pgfqpoint{1.732174in}{1.631213in}}{\pgfqpoint{1.735447in}{1.639113in}}{\pgfqpoint{1.735447in}{1.647349in}}%
\pgfpathcurveto{\pgfqpoint{1.735447in}{1.655585in}}{\pgfqpoint{1.732174in}{1.663485in}}{\pgfqpoint{1.726350in}{1.669309in}}%
\pgfpathcurveto{\pgfqpoint{1.720527in}{1.675133in}}{\pgfqpoint{1.712626in}{1.678406in}}{\pgfqpoint{1.704390in}{1.678406in}}%
\pgfpathcurveto{\pgfqpoint{1.696154in}{1.678406in}}{\pgfqpoint{1.688254in}{1.675133in}}{\pgfqpoint{1.682430in}{1.669309in}}%
\pgfpathcurveto{\pgfqpoint{1.676606in}{1.663485in}}{\pgfqpoint{1.673334in}{1.655585in}}{\pgfqpoint{1.673334in}{1.647349in}}%
\pgfpathcurveto{\pgfqpoint{1.673334in}{1.639113in}}{\pgfqpoint{1.676606in}{1.631213in}}{\pgfqpoint{1.682430in}{1.625389in}}%
\pgfpathcurveto{\pgfqpoint{1.688254in}{1.619565in}}{\pgfqpoint{1.696154in}{1.616293in}}{\pgfqpoint{1.704390in}{1.616293in}}%
\pgfpathclose%
\pgfusepath{stroke,fill}%
\end{pgfscope}%
\begin{pgfscope}%
\pgfpathrectangle{\pgfqpoint{0.100000in}{0.212622in}}{\pgfqpoint{3.696000in}{3.696000in}}%
\pgfusepath{clip}%
\pgfsetbuttcap%
\pgfsetroundjoin%
\definecolor{currentfill}{rgb}{0.121569,0.466667,0.705882}%
\pgfsetfillcolor{currentfill}%
\pgfsetfillopacity{0.800663}%
\pgfsetlinewidth{1.003750pt}%
\definecolor{currentstroke}{rgb}{0.121569,0.466667,0.705882}%
\pgfsetstrokecolor{currentstroke}%
\pgfsetstrokeopacity{0.800663}%
\pgfsetdash{}{0pt}%
\pgfpathmoveto{\pgfqpoint{1.704803in}{1.616367in}}%
\pgfpathcurveto{\pgfqpoint{1.713039in}{1.616367in}}{\pgfqpoint{1.720939in}{1.619639in}}{\pgfqpoint{1.726763in}{1.625463in}}%
\pgfpathcurveto{\pgfqpoint{1.732587in}{1.631287in}}{\pgfqpoint{1.735859in}{1.639187in}}{\pgfqpoint{1.735859in}{1.647424in}}%
\pgfpathcurveto{\pgfqpoint{1.735859in}{1.655660in}}{\pgfqpoint{1.732587in}{1.663560in}}{\pgfqpoint{1.726763in}{1.669384in}}%
\pgfpathcurveto{\pgfqpoint{1.720939in}{1.675208in}}{\pgfqpoint{1.713039in}{1.678480in}}{\pgfqpoint{1.704803in}{1.678480in}}%
\pgfpathcurveto{\pgfqpoint{1.696566in}{1.678480in}}{\pgfqpoint{1.688666in}{1.675208in}}{\pgfqpoint{1.682842in}{1.669384in}}%
\pgfpathcurveto{\pgfqpoint{1.677019in}{1.663560in}}{\pgfqpoint{1.673746in}{1.655660in}}{\pgfqpoint{1.673746in}{1.647424in}}%
\pgfpathcurveto{\pgfqpoint{1.673746in}{1.639187in}}{\pgfqpoint{1.677019in}{1.631287in}}{\pgfqpoint{1.682842in}{1.625463in}}%
\pgfpathcurveto{\pgfqpoint{1.688666in}{1.619639in}}{\pgfqpoint{1.696566in}{1.616367in}}{\pgfqpoint{1.704803in}{1.616367in}}%
\pgfpathclose%
\pgfusepath{stroke,fill}%
\end{pgfscope}%
\begin{pgfscope}%
\pgfpathrectangle{\pgfqpoint{0.100000in}{0.212622in}}{\pgfqpoint{3.696000in}{3.696000in}}%
\pgfusepath{clip}%
\pgfsetbuttcap%
\pgfsetroundjoin%
\definecolor{currentfill}{rgb}{0.121569,0.466667,0.705882}%
\pgfsetfillcolor{currentfill}%
\pgfsetfillopacity{0.801472}%
\pgfsetlinewidth{1.003750pt}%
\definecolor{currentstroke}{rgb}{0.121569,0.466667,0.705882}%
\pgfsetstrokecolor{currentstroke}%
\pgfsetstrokeopacity{0.801472}%
\pgfsetdash{}{0pt}%
\pgfpathmoveto{\pgfqpoint{1.705067in}{1.616455in}}%
\pgfpathcurveto{\pgfqpoint{1.713303in}{1.616455in}}{\pgfqpoint{1.721203in}{1.619728in}}{\pgfqpoint{1.727027in}{1.625552in}}%
\pgfpathcurveto{\pgfqpoint{1.732851in}{1.631376in}}{\pgfqpoint{1.736124in}{1.639276in}}{\pgfqpoint{1.736124in}{1.647512in}}%
\pgfpathcurveto{\pgfqpoint{1.736124in}{1.655748in}}{\pgfqpoint{1.732851in}{1.663648in}}{\pgfqpoint{1.727027in}{1.669472in}}%
\pgfpathcurveto{\pgfqpoint{1.721203in}{1.675296in}}{\pgfqpoint{1.713303in}{1.678568in}}{\pgfqpoint{1.705067in}{1.678568in}}%
\pgfpathcurveto{\pgfqpoint{1.696831in}{1.678568in}}{\pgfqpoint{1.688931in}{1.675296in}}{\pgfqpoint{1.683107in}{1.669472in}}%
\pgfpathcurveto{\pgfqpoint{1.677283in}{1.663648in}}{\pgfqpoint{1.674011in}{1.655748in}}{\pgfqpoint{1.674011in}{1.647512in}}%
\pgfpathcurveto{\pgfqpoint{1.674011in}{1.639276in}}{\pgfqpoint{1.677283in}{1.631376in}}{\pgfqpoint{1.683107in}{1.625552in}}%
\pgfpathcurveto{\pgfqpoint{1.688931in}{1.619728in}}{\pgfqpoint{1.696831in}{1.616455in}}{\pgfqpoint{1.705067in}{1.616455in}}%
\pgfpathclose%
\pgfusepath{stroke,fill}%
\end{pgfscope}%
\begin{pgfscope}%
\pgfpathrectangle{\pgfqpoint{0.100000in}{0.212622in}}{\pgfqpoint{3.696000in}{3.696000in}}%
\pgfusepath{clip}%
\pgfsetbuttcap%
\pgfsetroundjoin%
\definecolor{currentfill}{rgb}{0.121569,0.466667,0.705882}%
\pgfsetfillcolor{currentfill}%
\pgfsetfillopacity{0.802573}%
\pgfsetlinewidth{1.003750pt}%
\definecolor{currentstroke}{rgb}{0.121569,0.466667,0.705882}%
\pgfsetstrokecolor{currentstroke}%
\pgfsetstrokeopacity{0.802573}%
\pgfsetdash{}{0pt}%
\pgfpathmoveto{\pgfqpoint{1.705486in}{1.615703in}}%
\pgfpathcurveto{\pgfqpoint{1.713722in}{1.615703in}}{\pgfqpoint{1.721622in}{1.618975in}}{\pgfqpoint{1.727446in}{1.624799in}}%
\pgfpathcurveto{\pgfqpoint{1.733270in}{1.630623in}}{\pgfqpoint{1.736542in}{1.638523in}}{\pgfqpoint{1.736542in}{1.646760in}}%
\pgfpathcurveto{\pgfqpoint{1.736542in}{1.654996in}}{\pgfqpoint{1.733270in}{1.662896in}}{\pgfqpoint{1.727446in}{1.668720in}}%
\pgfpathcurveto{\pgfqpoint{1.721622in}{1.674544in}}{\pgfqpoint{1.713722in}{1.677816in}}{\pgfqpoint{1.705486in}{1.677816in}}%
\pgfpathcurveto{\pgfqpoint{1.697249in}{1.677816in}}{\pgfqpoint{1.689349in}{1.674544in}}{\pgfqpoint{1.683525in}{1.668720in}}%
\pgfpathcurveto{\pgfqpoint{1.677701in}{1.662896in}}{\pgfqpoint{1.674429in}{1.654996in}}{\pgfqpoint{1.674429in}{1.646760in}}%
\pgfpathcurveto{\pgfqpoint{1.674429in}{1.638523in}}{\pgfqpoint{1.677701in}{1.630623in}}{\pgfqpoint{1.683525in}{1.624799in}}%
\pgfpathcurveto{\pgfqpoint{1.689349in}{1.618975in}}{\pgfqpoint{1.697249in}{1.615703in}}{\pgfqpoint{1.705486in}{1.615703in}}%
\pgfpathclose%
\pgfusepath{stroke,fill}%
\end{pgfscope}%
\begin{pgfscope}%
\pgfpathrectangle{\pgfqpoint{0.100000in}{0.212622in}}{\pgfqpoint{3.696000in}{3.696000in}}%
\pgfusepath{clip}%
\pgfsetbuttcap%
\pgfsetroundjoin%
\definecolor{currentfill}{rgb}{0.121569,0.466667,0.705882}%
\pgfsetfillcolor{currentfill}%
\pgfsetfillopacity{0.804094}%
\pgfsetlinewidth{1.003750pt}%
\definecolor{currentstroke}{rgb}{0.121569,0.466667,0.705882}%
\pgfsetstrokecolor{currentstroke}%
\pgfsetstrokeopacity{0.804094}%
\pgfsetdash{}{0pt}%
\pgfpathmoveto{\pgfqpoint{1.706350in}{1.613148in}}%
\pgfpathcurveto{\pgfqpoint{1.714586in}{1.613148in}}{\pgfqpoint{1.722486in}{1.616420in}}{\pgfqpoint{1.728310in}{1.622244in}}%
\pgfpathcurveto{\pgfqpoint{1.734134in}{1.628068in}}{\pgfqpoint{1.737406in}{1.635968in}}{\pgfqpoint{1.737406in}{1.644205in}}%
\pgfpathcurveto{\pgfqpoint{1.737406in}{1.652441in}}{\pgfqpoint{1.734134in}{1.660341in}}{\pgfqpoint{1.728310in}{1.666165in}}%
\pgfpathcurveto{\pgfqpoint{1.722486in}{1.671989in}}{\pgfqpoint{1.714586in}{1.675261in}}{\pgfqpoint{1.706350in}{1.675261in}}%
\pgfpathcurveto{\pgfqpoint{1.698113in}{1.675261in}}{\pgfqpoint{1.690213in}{1.671989in}}{\pgfqpoint{1.684389in}{1.666165in}}%
\pgfpathcurveto{\pgfqpoint{1.678566in}{1.660341in}}{\pgfqpoint{1.675293in}{1.652441in}}{\pgfqpoint{1.675293in}{1.644205in}}%
\pgfpathcurveto{\pgfqpoint{1.675293in}{1.635968in}}{\pgfqpoint{1.678566in}{1.628068in}}{\pgfqpoint{1.684389in}{1.622244in}}%
\pgfpathcurveto{\pgfqpoint{1.690213in}{1.616420in}}{\pgfqpoint{1.698113in}{1.613148in}}{\pgfqpoint{1.706350in}{1.613148in}}%
\pgfpathclose%
\pgfusepath{stroke,fill}%
\end{pgfscope}%
\begin{pgfscope}%
\pgfpathrectangle{\pgfqpoint{0.100000in}{0.212622in}}{\pgfqpoint{3.696000in}{3.696000in}}%
\pgfusepath{clip}%
\pgfsetbuttcap%
\pgfsetroundjoin%
\definecolor{currentfill}{rgb}{0.121569,0.466667,0.705882}%
\pgfsetfillcolor{currentfill}%
\pgfsetfillopacity{0.805334}%
\pgfsetlinewidth{1.003750pt}%
\definecolor{currentstroke}{rgb}{0.121569,0.466667,0.705882}%
\pgfsetstrokecolor{currentstroke}%
\pgfsetstrokeopacity{0.805334}%
\pgfsetdash{}{0pt}%
\pgfpathmoveto{\pgfqpoint{1.707091in}{1.613687in}}%
\pgfpathcurveto{\pgfqpoint{1.715327in}{1.613687in}}{\pgfqpoint{1.723227in}{1.616959in}}{\pgfqpoint{1.729051in}{1.622783in}}%
\pgfpathcurveto{\pgfqpoint{1.734875in}{1.628607in}}{\pgfqpoint{1.738147in}{1.636507in}}{\pgfqpoint{1.738147in}{1.644744in}}%
\pgfpathcurveto{\pgfqpoint{1.738147in}{1.652980in}}{\pgfqpoint{1.734875in}{1.660880in}}{\pgfqpoint{1.729051in}{1.666704in}}%
\pgfpathcurveto{\pgfqpoint{1.723227in}{1.672528in}}{\pgfqpoint{1.715327in}{1.675800in}}{\pgfqpoint{1.707091in}{1.675800in}}%
\pgfpathcurveto{\pgfqpoint{1.698855in}{1.675800in}}{\pgfqpoint{1.690955in}{1.672528in}}{\pgfqpoint{1.685131in}{1.666704in}}%
\pgfpathcurveto{\pgfqpoint{1.679307in}{1.660880in}}{\pgfqpoint{1.676034in}{1.652980in}}{\pgfqpoint{1.676034in}{1.644744in}}%
\pgfpathcurveto{\pgfqpoint{1.676034in}{1.636507in}}{\pgfqpoint{1.679307in}{1.628607in}}{\pgfqpoint{1.685131in}{1.622783in}}%
\pgfpathcurveto{\pgfqpoint{1.690955in}{1.616959in}}{\pgfqpoint{1.698855in}{1.613687in}}{\pgfqpoint{1.707091in}{1.613687in}}%
\pgfpathclose%
\pgfusepath{stroke,fill}%
\end{pgfscope}%
\begin{pgfscope}%
\pgfpathrectangle{\pgfqpoint{0.100000in}{0.212622in}}{\pgfqpoint{3.696000in}{3.696000in}}%
\pgfusepath{clip}%
\pgfsetbuttcap%
\pgfsetroundjoin%
\definecolor{currentfill}{rgb}{0.121569,0.466667,0.705882}%
\pgfsetfillcolor{currentfill}%
\pgfsetfillopacity{0.806854}%
\pgfsetlinewidth{1.003750pt}%
\definecolor{currentstroke}{rgb}{0.121569,0.466667,0.705882}%
\pgfsetstrokecolor{currentstroke}%
\pgfsetstrokeopacity{0.806854}%
\pgfsetdash{}{0pt}%
\pgfpathmoveto{\pgfqpoint{1.707844in}{1.613790in}}%
\pgfpathcurveto{\pgfqpoint{1.716081in}{1.613790in}}{\pgfqpoint{1.723981in}{1.617062in}}{\pgfqpoint{1.729805in}{1.622886in}}%
\pgfpathcurveto{\pgfqpoint{1.735629in}{1.628710in}}{\pgfqpoint{1.738901in}{1.636610in}}{\pgfqpoint{1.738901in}{1.644847in}}%
\pgfpathcurveto{\pgfqpoint{1.738901in}{1.653083in}}{\pgfqpoint{1.735629in}{1.660983in}}{\pgfqpoint{1.729805in}{1.666807in}}%
\pgfpathcurveto{\pgfqpoint{1.723981in}{1.672631in}}{\pgfqpoint{1.716081in}{1.675903in}}{\pgfqpoint{1.707844in}{1.675903in}}%
\pgfpathcurveto{\pgfqpoint{1.699608in}{1.675903in}}{\pgfqpoint{1.691708in}{1.672631in}}{\pgfqpoint{1.685884in}{1.666807in}}%
\pgfpathcurveto{\pgfqpoint{1.680060in}{1.660983in}}{\pgfqpoint{1.676788in}{1.653083in}}{\pgfqpoint{1.676788in}{1.644847in}}%
\pgfpathcurveto{\pgfqpoint{1.676788in}{1.636610in}}{\pgfqpoint{1.680060in}{1.628710in}}{\pgfqpoint{1.685884in}{1.622886in}}%
\pgfpathcurveto{\pgfqpoint{1.691708in}{1.617062in}}{\pgfqpoint{1.699608in}{1.613790in}}{\pgfqpoint{1.707844in}{1.613790in}}%
\pgfpathclose%
\pgfusepath{stroke,fill}%
\end{pgfscope}%
\begin{pgfscope}%
\pgfpathrectangle{\pgfqpoint{0.100000in}{0.212622in}}{\pgfqpoint{3.696000in}{3.696000in}}%
\pgfusepath{clip}%
\pgfsetbuttcap%
\pgfsetroundjoin%
\definecolor{currentfill}{rgb}{0.121569,0.466667,0.705882}%
\pgfsetfillcolor{currentfill}%
\pgfsetfillopacity{0.808167}%
\pgfsetlinewidth{1.003750pt}%
\definecolor{currentstroke}{rgb}{0.121569,0.466667,0.705882}%
\pgfsetstrokecolor{currentstroke}%
\pgfsetstrokeopacity{0.808167}%
\pgfsetdash{}{0pt}%
\pgfpathmoveto{\pgfqpoint{1.708480in}{1.611716in}}%
\pgfpathcurveto{\pgfqpoint{1.716716in}{1.611716in}}{\pgfqpoint{1.724616in}{1.614989in}}{\pgfqpoint{1.730440in}{1.620812in}}%
\pgfpathcurveto{\pgfqpoint{1.736264in}{1.626636in}}{\pgfqpoint{1.739536in}{1.634536in}}{\pgfqpoint{1.739536in}{1.642773in}}%
\pgfpathcurveto{\pgfqpoint{1.739536in}{1.651009in}}{\pgfqpoint{1.736264in}{1.658909in}}{\pgfqpoint{1.730440in}{1.664733in}}%
\pgfpathcurveto{\pgfqpoint{1.724616in}{1.670557in}}{\pgfqpoint{1.716716in}{1.673829in}}{\pgfqpoint{1.708480in}{1.673829in}}%
\pgfpathcurveto{\pgfqpoint{1.700244in}{1.673829in}}{\pgfqpoint{1.692344in}{1.670557in}}{\pgfqpoint{1.686520in}{1.664733in}}%
\pgfpathcurveto{\pgfqpoint{1.680696in}{1.658909in}}{\pgfqpoint{1.677423in}{1.651009in}}{\pgfqpoint{1.677423in}{1.642773in}}%
\pgfpathcurveto{\pgfqpoint{1.677423in}{1.634536in}}{\pgfqpoint{1.680696in}{1.626636in}}{\pgfqpoint{1.686520in}{1.620812in}}%
\pgfpathcurveto{\pgfqpoint{1.692344in}{1.614989in}}{\pgfqpoint{1.700244in}{1.611716in}}{\pgfqpoint{1.708480in}{1.611716in}}%
\pgfpathclose%
\pgfusepath{stroke,fill}%
\end{pgfscope}%
\begin{pgfscope}%
\pgfpathrectangle{\pgfqpoint{0.100000in}{0.212622in}}{\pgfqpoint{3.696000in}{3.696000in}}%
\pgfusepath{clip}%
\pgfsetbuttcap%
\pgfsetroundjoin%
\definecolor{currentfill}{rgb}{0.121569,0.466667,0.705882}%
\pgfsetfillcolor{currentfill}%
\pgfsetfillopacity{0.809330}%
\pgfsetlinewidth{1.003750pt}%
\definecolor{currentstroke}{rgb}{0.121569,0.466667,0.705882}%
\pgfsetstrokecolor{currentstroke}%
\pgfsetstrokeopacity{0.809330}%
\pgfsetdash{}{0pt}%
\pgfpathmoveto{\pgfqpoint{1.710177in}{1.606242in}}%
\pgfpathcurveto{\pgfqpoint{1.718413in}{1.606242in}}{\pgfqpoint{1.726313in}{1.609514in}}{\pgfqpoint{1.732137in}{1.615338in}}%
\pgfpathcurveto{\pgfqpoint{1.737961in}{1.621162in}}{\pgfqpoint{1.741233in}{1.629062in}}{\pgfqpoint{1.741233in}{1.637298in}}%
\pgfpathcurveto{\pgfqpoint{1.741233in}{1.645535in}}{\pgfqpoint{1.737961in}{1.653435in}}{\pgfqpoint{1.732137in}{1.659259in}}%
\pgfpathcurveto{\pgfqpoint{1.726313in}{1.665083in}}{\pgfqpoint{1.718413in}{1.668355in}}{\pgfqpoint{1.710177in}{1.668355in}}%
\pgfpathcurveto{\pgfqpoint{1.701940in}{1.668355in}}{\pgfqpoint{1.694040in}{1.665083in}}{\pgfqpoint{1.688216in}{1.659259in}}%
\pgfpathcurveto{\pgfqpoint{1.682392in}{1.653435in}}{\pgfqpoint{1.679120in}{1.645535in}}{\pgfqpoint{1.679120in}{1.637298in}}%
\pgfpathcurveto{\pgfqpoint{1.679120in}{1.629062in}}{\pgfqpoint{1.682392in}{1.621162in}}{\pgfqpoint{1.688216in}{1.615338in}}%
\pgfpathcurveto{\pgfqpoint{1.694040in}{1.609514in}}{\pgfqpoint{1.701940in}{1.606242in}}{\pgfqpoint{1.710177in}{1.606242in}}%
\pgfpathclose%
\pgfusepath{stroke,fill}%
\end{pgfscope}%
\begin{pgfscope}%
\pgfpathrectangle{\pgfqpoint{0.100000in}{0.212622in}}{\pgfqpoint{3.696000in}{3.696000in}}%
\pgfusepath{clip}%
\pgfsetbuttcap%
\pgfsetroundjoin%
\definecolor{currentfill}{rgb}{0.121569,0.466667,0.705882}%
\pgfsetfillcolor{currentfill}%
\pgfsetfillopacity{0.811326}%
\pgfsetlinewidth{1.003750pt}%
\definecolor{currentstroke}{rgb}{0.121569,0.466667,0.705882}%
\pgfsetstrokecolor{currentstroke}%
\pgfsetstrokeopacity{0.811326}%
\pgfsetdash{}{0pt}%
\pgfpathmoveto{\pgfqpoint{1.711604in}{1.602766in}}%
\pgfpathcurveto{\pgfqpoint{1.719840in}{1.602766in}}{\pgfqpoint{1.727740in}{1.606038in}}{\pgfqpoint{1.733564in}{1.611862in}}%
\pgfpathcurveto{\pgfqpoint{1.739388in}{1.617686in}}{\pgfqpoint{1.742660in}{1.625586in}}{\pgfqpoint{1.742660in}{1.633822in}}%
\pgfpathcurveto{\pgfqpoint{1.742660in}{1.642059in}}{\pgfqpoint{1.739388in}{1.649959in}}{\pgfqpoint{1.733564in}{1.655783in}}%
\pgfpathcurveto{\pgfqpoint{1.727740in}{1.661606in}}{\pgfqpoint{1.719840in}{1.664879in}}{\pgfqpoint{1.711604in}{1.664879in}}%
\pgfpathcurveto{\pgfqpoint{1.703367in}{1.664879in}}{\pgfqpoint{1.695467in}{1.661606in}}{\pgfqpoint{1.689643in}{1.655783in}}%
\pgfpathcurveto{\pgfqpoint{1.683819in}{1.649959in}}{\pgfqpoint{1.680547in}{1.642059in}}{\pgfqpoint{1.680547in}{1.633822in}}%
\pgfpathcurveto{\pgfqpoint{1.680547in}{1.625586in}}{\pgfqpoint{1.683819in}{1.617686in}}{\pgfqpoint{1.689643in}{1.611862in}}%
\pgfpathcurveto{\pgfqpoint{1.695467in}{1.606038in}}{\pgfqpoint{1.703367in}{1.602766in}}{\pgfqpoint{1.711604in}{1.602766in}}%
\pgfpathclose%
\pgfusepath{stroke,fill}%
\end{pgfscope}%
\begin{pgfscope}%
\pgfpathrectangle{\pgfqpoint{0.100000in}{0.212622in}}{\pgfqpoint{3.696000in}{3.696000in}}%
\pgfusepath{clip}%
\pgfsetbuttcap%
\pgfsetroundjoin%
\definecolor{currentfill}{rgb}{0.121569,0.466667,0.705882}%
\pgfsetfillcolor{currentfill}%
\pgfsetfillopacity{0.814930}%
\pgfsetlinewidth{1.003750pt}%
\definecolor{currentstroke}{rgb}{0.121569,0.466667,0.705882}%
\pgfsetstrokecolor{currentstroke}%
\pgfsetstrokeopacity{0.814930}%
\pgfsetdash{}{0pt}%
\pgfpathmoveto{\pgfqpoint{1.714066in}{1.606174in}}%
\pgfpathcurveto{\pgfqpoint{1.722302in}{1.606174in}}{\pgfqpoint{1.730202in}{1.609446in}}{\pgfqpoint{1.736026in}{1.615270in}}%
\pgfpathcurveto{\pgfqpoint{1.741850in}{1.621094in}}{\pgfqpoint{1.745122in}{1.628994in}}{\pgfqpoint{1.745122in}{1.637230in}}%
\pgfpathcurveto{\pgfqpoint{1.745122in}{1.645467in}}{\pgfqpoint{1.741850in}{1.653367in}}{\pgfqpoint{1.736026in}{1.659191in}}%
\pgfpathcurveto{\pgfqpoint{1.730202in}{1.665014in}}{\pgfqpoint{1.722302in}{1.668287in}}{\pgfqpoint{1.714066in}{1.668287in}}%
\pgfpathcurveto{\pgfqpoint{1.705830in}{1.668287in}}{\pgfqpoint{1.697930in}{1.665014in}}{\pgfqpoint{1.692106in}{1.659191in}}%
\pgfpathcurveto{\pgfqpoint{1.686282in}{1.653367in}}{\pgfqpoint{1.683009in}{1.645467in}}{\pgfqpoint{1.683009in}{1.637230in}}%
\pgfpathcurveto{\pgfqpoint{1.683009in}{1.628994in}}{\pgfqpoint{1.686282in}{1.621094in}}{\pgfqpoint{1.692106in}{1.615270in}}%
\pgfpathcurveto{\pgfqpoint{1.697930in}{1.609446in}}{\pgfqpoint{1.705830in}{1.606174in}}{\pgfqpoint{1.714066in}{1.606174in}}%
\pgfpathclose%
\pgfusepath{stroke,fill}%
\end{pgfscope}%
\begin{pgfscope}%
\pgfpathrectangle{\pgfqpoint{0.100000in}{0.212622in}}{\pgfqpoint{3.696000in}{3.696000in}}%
\pgfusepath{clip}%
\pgfsetbuttcap%
\pgfsetroundjoin%
\definecolor{currentfill}{rgb}{0.121569,0.466667,0.705882}%
\pgfsetfillcolor{currentfill}%
\pgfsetfillopacity{0.816217}%
\pgfsetlinewidth{1.003750pt}%
\definecolor{currentstroke}{rgb}{0.121569,0.466667,0.705882}%
\pgfsetstrokecolor{currentstroke}%
\pgfsetstrokeopacity{0.816217}%
\pgfsetdash{}{0pt}%
\pgfpathmoveto{\pgfqpoint{1.714475in}{1.604434in}}%
\pgfpathcurveto{\pgfqpoint{1.722711in}{1.604434in}}{\pgfqpoint{1.730611in}{1.607706in}}{\pgfqpoint{1.736435in}{1.613530in}}%
\pgfpathcurveto{\pgfqpoint{1.742259in}{1.619354in}}{\pgfqpoint{1.745531in}{1.627254in}}{\pgfqpoint{1.745531in}{1.635490in}}%
\pgfpathcurveto{\pgfqpoint{1.745531in}{1.643727in}}{\pgfqpoint{1.742259in}{1.651627in}}{\pgfqpoint{1.736435in}{1.657451in}}%
\pgfpathcurveto{\pgfqpoint{1.730611in}{1.663274in}}{\pgfqpoint{1.722711in}{1.666547in}}{\pgfqpoint{1.714475in}{1.666547in}}%
\pgfpathcurveto{\pgfqpoint{1.706238in}{1.666547in}}{\pgfqpoint{1.698338in}{1.663274in}}{\pgfqpoint{1.692514in}{1.657451in}}%
\pgfpathcurveto{\pgfqpoint{1.686691in}{1.651627in}}{\pgfqpoint{1.683418in}{1.643727in}}{\pgfqpoint{1.683418in}{1.635490in}}%
\pgfpathcurveto{\pgfqpoint{1.683418in}{1.627254in}}{\pgfqpoint{1.686691in}{1.619354in}}{\pgfqpoint{1.692514in}{1.613530in}}%
\pgfpathcurveto{\pgfqpoint{1.698338in}{1.607706in}}{\pgfqpoint{1.706238in}{1.604434in}}{\pgfqpoint{1.714475in}{1.604434in}}%
\pgfpathclose%
\pgfusepath{stroke,fill}%
\end{pgfscope}%
\begin{pgfscope}%
\pgfpathrectangle{\pgfqpoint{0.100000in}{0.212622in}}{\pgfqpoint{3.696000in}{3.696000in}}%
\pgfusepath{clip}%
\pgfsetbuttcap%
\pgfsetroundjoin%
\definecolor{currentfill}{rgb}{0.121569,0.466667,0.705882}%
\pgfsetfillcolor{currentfill}%
\pgfsetfillopacity{0.817790}%
\pgfsetlinewidth{1.003750pt}%
\definecolor{currentstroke}{rgb}{0.121569,0.466667,0.705882}%
\pgfsetstrokecolor{currentstroke}%
\pgfsetstrokeopacity{0.817790}%
\pgfsetdash{}{0pt}%
\pgfpathmoveto{\pgfqpoint{1.716051in}{1.601588in}}%
\pgfpathcurveto{\pgfqpoint{1.724287in}{1.601588in}}{\pgfqpoint{1.732187in}{1.604861in}}{\pgfqpoint{1.738011in}{1.610685in}}%
\pgfpathcurveto{\pgfqpoint{1.743835in}{1.616509in}}{\pgfqpoint{1.747107in}{1.624409in}}{\pgfqpoint{1.747107in}{1.632645in}}%
\pgfpathcurveto{\pgfqpoint{1.747107in}{1.640881in}}{\pgfqpoint{1.743835in}{1.648781in}}{\pgfqpoint{1.738011in}{1.654605in}}%
\pgfpathcurveto{\pgfqpoint{1.732187in}{1.660429in}}{\pgfqpoint{1.724287in}{1.663701in}}{\pgfqpoint{1.716051in}{1.663701in}}%
\pgfpathcurveto{\pgfqpoint{1.707814in}{1.663701in}}{\pgfqpoint{1.699914in}{1.660429in}}{\pgfqpoint{1.694090in}{1.654605in}}%
\pgfpathcurveto{\pgfqpoint{1.688266in}{1.648781in}}{\pgfqpoint{1.684994in}{1.640881in}}{\pgfqpoint{1.684994in}{1.632645in}}%
\pgfpathcurveto{\pgfqpoint{1.684994in}{1.624409in}}{\pgfqpoint{1.688266in}{1.616509in}}{\pgfqpoint{1.694090in}{1.610685in}}%
\pgfpathcurveto{\pgfqpoint{1.699914in}{1.604861in}}{\pgfqpoint{1.707814in}{1.601588in}}{\pgfqpoint{1.716051in}{1.601588in}}%
\pgfpathclose%
\pgfusepath{stroke,fill}%
\end{pgfscope}%
\begin{pgfscope}%
\pgfpathrectangle{\pgfqpoint{0.100000in}{0.212622in}}{\pgfqpoint{3.696000in}{3.696000in}}%
\pgfusepath{clip}%
\pgfsetbuttcap%
\pgfsetroundjoin%
\definecolor{currentfill}{rgb}{0.121569,0.466667,0.705882}%
\pgfsetfillcolor{currentfill}%
\pgfsetfillopacity{0.819502}%
\pgfsetlinewidth{1.003750pt}%
\definecolor{currentstroke}{rgb}{0.121569,0.466667,0.705882}%
\pgfsetstrokecolor{currentstroke}%
\pgfsetstrokeopacity{0.819502}%
\pgfsetdash{}{0pt}%
\pgfpathmoveto{\pgfqpoint{1.717015in}{1.598066in}}%
\pgfpathcurveto{\pgfqpoint{1.725251in}{1.598066in}}{\pgfqpoint{1.733151in}{1.601339in}}{\pgfqpoint{1.738975in}{1.607163in}}%
\pgfpathcurveto{\pgfqpoint{1.744799in}{1.612986in}}{\pgfqpoint{1.748072in}{1.620887in}}{\pgfqpoint{1.748072in}{1.629123in}}%
\pgfpathcurveto{\pgfqpoint{1.748072in}{1.637359in}}{\pgfqpoint{1.744799in}{1.645259in}}{\pgfqpoint{1.738975in}{1.651083in}}%
\pgfpathcurveto{\pgfqpoint{1.733151in}{1.656907in}}{\pgfqpoint{1.725251in}{1.660179in}}{\pgfqpoint{1.717015in}{1.660179in}}%
\pgfpathcurveto{\pgfqpoint{1.708779in}{1.660179in}}{\pgfqpoint{1.700879in}{1.656907in}}{\pgfqpoint{1.695055in}{1.651083in}}%
\pgfpathcurveto{\pgfqpoint{1.689231in}{1.645259in}}{\pgfqpoint{1.685959in}{1.637359in}}{\pgfqpoint{1.685959in}{1.629123in}}%
\pgfpathcurveto{\pgfqpoint{1.685959in}{1.620887in}}{\pgfqpoint{1.689231in}{1.612986in}}{\pgfqpoint{1.695055in}{1.607163in}}%
\pgfpathcurveto{\pgfqpoint{1.700879in}{1.601339in}}{\pgfqpoint{1.708779in}{1.598066in}}{\pgfqpoint{1.717015in}{1.598066in}}%
\pgfpathclose%
\pgfusepath{stroke,fill}%
\end{pgfscope}%
\begin{pgfscope}%
\pgfpathrectangle{\pgfqpoint{0.100000in}{0.212622in}}{\pgfqpoint{3.696000in}{3.696000in}}%
\pgfusepath{clip}%
\pgfsetbuttcap%
\pgfsetroundjoin%
\definecolor{currentfill}{rgb}{0.121569,0.466667,0.705882}%
\pgfsetfillcolor{currentfill}%
\pgfsetfillopacity{0.821038}%
\pgfsetlinewidth{1.003750pt}%
\definecolor{currentstroke}{rgb}{0.121569,0.466667,0.705882}%
\pgfsetstrokecolor{currentstroke}%
\pgfsetstrokeopacity{0.821038}%
\pgfsetdash{}{0pt}%
\pgfpathmoveto{\pgfqpoint{1.717959in}{1.598999in}}%
\pgfpathcurveto{\pgfqpoint{1.726196in}{1.598999in}}{\pgfqpoint{1.734096in}{1.602271in}}{\pgfqpoint{1.739920in}{1.608095in}}%
\pgfpathcurveto{\pgfqpoint{1.745743in}{1.613919in}}{\pgfqpoint{1.749016in}{1.621819in}}{\pgfqpoint{1.749016in}{1.630055in}}%
\pgfpathcurveto{\pgfqpoint{1.749016in}{1.638292in}}{\pgfqpoint{1.745743in}{1.646192in}}{\pgfqpoint{1.739920in}{1.652016in}}%
\pgfpathcurveto{\pgfqpoint{1.734096in}{1.657840in}}{\pgfqpoint{1.726196in}{1.661112in}}{\pgfqpoint{1.717959in}{1.661112in}}%
\pgfpathcurveto{\pgfqpoint{1.709723in}{1.661112in}}{\pgfqpoint{1.701823in}{1.657840in}}{\pgfqpoint{1.695999in}{1.652016in}}%
\pgfpathcurveto{\pgfqpoint{1.690175in}{1.646192in}}{\pgfqpoint{1.686903in}{1.638292in}}{\pgfqpoint{1.686903in}{1.630055in}}%
\pgfpathcurveto{\pgfqpoint{1.686903in}{1.621819in}}{\pgfqpoint{1.690175in}{1.613919in}}{\pgfqpoint{1.695999in}{1.608095in}}%
\pgfpathcurveto{\pgfqpoint{1.701823in}{1.602271in}}{\pgfqpoint{1.709723in}{1.598999in}}{\pgfqpoint{1.717959in}{1.598999in}}%
\pgfpathclose%
\pgfusepath{stroke,fill}%
\end{pgfscope}%
\begin{pgfscope}%
\pgfpathrectangle{\pgfqpoint{0.100000in}{0.212622in}}{\pgfqpoint{3.696000in}{3.696000in}}%
\pgfusepath{clip}%
\pgfsetbuttcap%
\pgfsetroundjoin%
\definecolor{currentfill}{rgb}{0.121569,0.466667,0.705882}%
\pgfsetfillcolor{currentfill}%
\pgfsetfillopacity{0.821797}%
\pgfsetlinewidth{1.003750pt}%
\definecolor{currentstroke}{rgb}{0.121569,0.466667,0.705882}%
\pgfsetstrokecolor{currentstroke}%
\pgfsetstrokeopacity{0.821797}%
\pgfsetdash{}{0pt}%
\pgfpathmoveto{\pgfqpoint{1.718220in}{1.598996in}}%
\pgfpathcurveto{\pgfqpoint{1.726456in}{1.598996in}}{\pgfqpoint{1.734356in}{1.602269in}}{\pgfqpoint{1.740180in}{1.608093in}}%
\pgfpathcurveto{\pgfqpoint{1.746004in}{1.613917in}}{\pgfqpoint{1.749276in}{1.621817in}}{\pgfqpoint{1.749276in}{1.630053in}}%
\pgfpathcurveto{\pgfqpoint{1.749276in}{1.638289in}}{\pgfqpoint{1.746004in}{1.646189in}}{\pgfqpoint{1.740180in}{1.652013in}}%
\pgfpathcurveto{\pgfqpoint{1.734356in}{1.657837in}}{\pgfqpoint{1.726456in}{1.661109in}}{\pgfqpoint{1.718220in}{1.661109in}}%
\pgfpathcurveto{\pgfqpoint{1.709983in}{1.661109in}}{\pgfqpoint{1.702083in}{1.657837in}}{\pgfqpoint{1.696259in}{1.652013in}}%
\pgfpathcurveto{\pgfqpoint{1.690435in}{1.646189in}}{\pgfqpoint{1.687163in}{1.638289in}}{\pgfqpoint{1.687163in}{1.630053in}}%
\pgfpathcurveto{\pgfqpoint{1.687163in}{1.621817in}}{\pgfqpoint{1.690435in}{1.613917in}}{\pgfqpoint{1.696259in}{1.608093in}}%
\pgfpathcurveto{\pgfqpoint{1.702083in}{1.602269in}}{\pgfqpoint{1.709983in}{1.598996in}}{\pgfqpoint{1.718220in}{1.598996in}}%
\pgfpathclose%
\pgfusepath{stroke,fill}%
\end{pgfscope}%
\begin{pgfscope}%
\pgfpathrectangle{\pgfqpoint{0.100000in}{0.212622in}}{\pgfqpoint{3.696000in}{3.696000in}}%
\pgfusepath{clip}%
\pgfsetbuttcap%
\pgfsetroundjoin%
\definecolor{currentfill}{rgb}{0.121569,0.466667,0.705882}%
\pgfsetfillcolor{currentfill}%
\pgfsetfillopacity{0.822736}%
\pgfsetlinewidth{1.003750pt}%
\definecolor{currentstroke}{rgb}{0.121569,0.466667,0.705882}%
\pgfsetstrokecolor{currentstroke}%
\pgfsetstrokeopacity{0.822736}%
\pgfsetdash{}{0pt}%
\pgfpathmoveto{\pgfqpoint{1.718802in}{1.598090in}}%
\pgfpathcurveto{\pgfqpoint{1.727039in}{1.598090in}}{\pgfqpoint{1.734939in}{1.601362in}}{\pgfqpoint{1.740763in}{1.607186in}}%
\pgfpathcurveto{\pgfqpoint{1.746587in}{1.613010in}}{\pgfqpoint{1.749859in}{1.620910in}}{\pgfqpoint{1.749859in}{1.629147in}}%
\pgfpathcurveto{\pgfqpoint{1.749859in}{1.637383in}}{\pgfqpoint{1.746587in}{1.645283in}}{\pgfqpoint{1.740763in}{1.651107in}}%
\pgfpathcurveto{\pgfqpoint{1.734939in}{1.656931in}}{\pgfqpoint{1.727039in}{1.660203in}}{\pgfqpoint{1.718802in}{1.660203in}}%
\pgfpathcurveto{\pgfqpoint{1.710566in}{1.660203in}}{\pgfqpoint{1.702666in}{1.656931in}}{\pgfqpoint{1.696842in}{1.651107in}}%
\pgfpathcurveto{\pgfqpoint{1.691018in}{1.645283in}}{\pgfqpoint{1.687746in}{1.637383in}}{\pgfqpoint{1.687746in}{1.629147in}}%
\pgfpathcurveto{\pgfqpoint{1.687746in}{1.620910in}}{\pgfqpoint{1.691018in}{1.613010in}}{\pgfqpoint{1.696842in}{1.607186in}}%
\pgfpathcurveto{\pgfqpoint{1.702666in}{1.601362in}}{\pgfqpoint{1.710566in}{1.598090in}}{\pgfqpoint{1.718802in}{1.598090in}}%
\pgfpathclose%
\pgfusepath{stroke,fill}%
\end{pgfscope}%
\begin{pgfscope}%
\pgfpathrectangle{\pgfqpoint{0.100000in}{0.212622in}}{\pgfqpoint{3.696000in}{3.696000in}}%
\pgfusepath{clip}%
\pgfsetbuttcap%
\pgfsetroundjoin%
\definecolor{currentfill}{rgb}{0.121569,0.466667,0.705882}%
\pgfsetfillcolor{currentfill}%
\pgfsetfillopacity{0.823600}%
\pgfsetlinewidth{1.003750pt}%
\definecolor{currentstroke}{rgb}{0.121569,0.466667,0.705882}%
\pgfsetstrokecolor{currentstroke}%
\pgfsetstrokeopacity{0.823600}%
\pgfsetdash{}{0pt}%
\pgfpathmoveto{\pgfqpoint{1.719409in}{1.596019in}}%
\pgfpathcurveto{\pgfqpoint{1.727645in}{1.596019in}}{\pgfqpoint{1.735545in}{1.599291in}}{\pgfqpoint{1.741369in}{1.605115in}}%
\pgfpathcurveto{\pgfqpoint{1.747193in}{1.610939in}}{\pgfqpoint{1.750466in}{1.618839in}}{\pgfqpoint{1.750466in}{1.627075in}}%
\pgfpathcurveto{\pgfqpoint{1.750466in}{1.635312in}}{\pgfqpoint{1.747193in}{1.643212in}}{\pgfqpoint{1.741369in}{1.649036in}}%
\pgfpathcurveto{\pgfqpoint{1.735545in}{1.654859in}}{\pgfqpoint{1.727645in}{1.658132in}}{\pgfqpoint{1.719409in}{1.658132in}}%
\pgfpathcurveto{\pgfqpoint{1.711173in}{1.658132in}}{\pgfqpoint{1.703273in}{1.654859in}}{\pgfqpoint{1.697449in}{1.649036in}}%
\pgfpathcurveto{\pgfqpoint{1.691625in}{1.643212in}}{\pgfqpoint{1.688353in}{1.635312in}}{\pgfqpoint{1.688353in}{1.627075in}}%
\pgfpathcurveto{\pgfqpoint{1.688353in}{1.618839in}}{\pgfqpoint{1.691625in}{1.610939in}}{\pgfqpoint{1.697449in}{1.605115in}}%
\pgfpathcurveto{\pgfqpoint{1.703273in}{1.599291in}}{\pgfqpoint{1.711173in}{1.596019in}}{\pgfqpoint{1.719409in}{1.596019in}}%
\pgfpathclose%
\pgfusepath{stroke,fill}%
\end{pgfscope}%
\begin{pgfscope}%
\pgfpathrectangle{\pgfqpoint{0.100000in}{0.212622in}}{\pgfqpoint{3.696000in}{3.696000in}}%
\pgfusepath{clip}%
\pgfsetbuttcap%
\pgfsetroundjoin%
\definecolor{currentfill}{rgb}{0.121569,0.466667,0.705882}%
\pgfsetfillcolor{currentfill}%
\pgfsetfillopacity{0.825066}%
\pgfsetlinewidth{1.003750pt}%
\definecolor{currentstroke}{rgb}{0.121569,0.466667,0.705882}%
\pgfsetstrokecolor{currentstroke}%
\pgfsetstrokeopacity{0.825066}%
\pgfsetdash{}{0pt}%
\pgfpathmoveto{\pgfqpoint{1.720535in}{1.595850in}}%
\pgfpathcurveto{\pgfqpoint{1.728772in}{1.595850in}}{\pgfqpoint{1.736672in}{1.599122in}}{\pgfqpoint{1.742496in}{1.604946in}}%
\pgfpathcurveto{\pgfqpoint{1.748319in}{1.610770in}}{\pgfqpoint{1.751592in}{1.618670in}}{\pgfqpoint{1.751592in}{1.626907in}}%
\pgfpathcurveto{\pgfqpoint{1.751592in}{1.635143in}}{\pgfqpoint{1.748319in}{1.643043in}}{\pgfqpoint{1.742496in}{1.648867in}}%
\pgfpathcurveto{\pgfqpoint{1.736672in}{1.654691in}}{\pgfqpoint{1.728772in}{1.657963in}}{\pgfqpoint{1.720535in}{1.657963in}}%
\pgfpathcurveto{\pgfqpoint{1.712299in}{1.657963in}}{\pgfqpoint{1.704399in}{1.654691in}}{\pgfqpoint{1.698575in}{1.648867in}}%
\pgfpathcurveto{\pgfqpoint{1.692751in}{1.643043in}}{\pgfqpoint{1.689479in}{1.635143in}}{\pgfqpoint{1.689479in}{1.626907in}}%
\pgfpathcurveto{\pgfqpoint{1.689479in}{1.618670in}}{\pgfqpoint{1.692751in}{1.610770in}}{\pgfqpoint{1.698575in}{1.604946in}}%
\pgfpathcurveto{\pgfqpoint{1.704399in}{1.599122in}}{\pgfqpoint{1.712299in}{1.595850in}}{\pgfqpoint{1.720535in}{1.595850in}}%
\pgfpathclose%
\pgfusepath{stroke,fill}%
\end{pgfscope}%
\begin{pgfscope}%
\pgfpathrectangle{\pgfqpoint{0.100000in}{0.212622in}}{\pgfqpoint{3.696000in}{3.696000in}}%
\pgfusepath{clip}%
\pgfsetbuttcap%
\pgfsetroundjoin%
\definecolor{currentfill}{rgb}{0.121569,0.466667,0.705882}%
\pgfsetfillcolor{currentfill}%
\pgfsetfillopacity{0.825922}%
\pgfsetlinewidth{1.003750pt}%
\definecolor{currentstroke}{rgb}{0.121569,0.466667,0.705882}%
\pgfsetstrokecolor{currentstroke}%
\pgfsetstrokeopacity{0.825922}%
\pgfsetdash{}{0pt}%
\pgfpathmoveto{\pgfqpoint{1.720933in}{1.595844in}}%
\pgfpathcurveto{\pgfqpoint{1.729169in}{1.595844in}}{\pgfqpoint{1.737069in}{1.599116in}}{\pgfqpoint{1.742893in}{1.604940in}}%
\pgfpathcurveto{\pgfqpoint{1.748717in}{1.610764in}}{\pgfqpoint{1.751989in}{1.618664in}}{\pgfqpoint{1.751989in}{1.626900in}}%
\pgfpathcurveto{\pgfqpoint{1.751989in}{1.635136in}}{\pgfqpoint{1.748717in}{1.643036in}}{\pgfqpoint{1.742893in}{1.648860in}}%
\pgfpathcurveto{\pgfqpoint{1.737069in}{1.654684in}}{\pgfqpoint{1.729169in}{1.657957in}}{\pgfqpoint{1.720933in}{1.657957in}}%
\pgfpathcurveto{\pgfqpoint{1.712696in}{1.657957in}}{\pgfqpoint{1.704796in}{1.654684in}}{\pgfqpoint{1.698972in}{1.648860in}}%
\pgfpathcurveto{\pgfqpoint{1.693149in}{1.643036in}}{\pgfqpoint{1.689876in}{1.635136in}}{\pgfqpoint{1.689876in}{1.626900in}}%
\pgfpathcurveto{\pgfqpoint{1.689876in}{1.618664in}}{\pgfqpoint{1.693149in}{1.610764in}}{\pgfqpoint{1.698972in}{1.604940in}}%
\pgfpathcurveto{\pgfqpoint{1.704796in}{1.599116in}}{\pgfqpoint{1.712696in}{1.595844in}}{\pgfqpoint{1.720933in}{1.595844in}}%
\pgfpathclose%
\pgfusepath{stroke,fill}%
\end{pgfscope}%
\begin{pgfscope}%
\pgfpathrectangle{\pgfqpoint{0.100000in}{0.212622in}}{\pgfqpoint{3.696000in}{3.696000in}}%
\pgfusepath{clip}%
\pgfsetbuttcap%
\pgfsetroundjoin%
\definecolor{currentfill}{rgb}{0.121569,0.466667,0.705882}%
\pgfsetfillcolor{currentfill}%
\pgfsetfillopacity{0.826938}%
\pgfsetlinewidth{1.003750pt}%
\definecolor{currentstroke}{rgb}{0.121569,0.466667,0.705882}%
\pgfsetstrokecolor{currentstroke}%
\pgfsetstrokeopacity{0.826938}%
\pgfsetdash{}{0pt}%
\pgfpathmoveto{\pgfqpoint{1.721225in}{1.595037in}}%
\pgfpathcurveto{\pgfqpoint{1.729461in}{1.595037in}}{\pgfqpoint{1.737361in}{1.598309in}}{\pgfqpoint{1.743185in}{1.604133in}}%
\pgfpathcurveto{\pgfqpoint{1.749009in}{1.609957in}}{\pgfqpoint{1.752281in}{1.617857in}}{\pgfqpoint{1.752281in}{1.626093in}}%
\pgfpathcurveto{\pgfqpoint{1.752281in}{1.634330in}}{\pgfqpoint{1.749009in}{1.642230in}}{\pgfqpoint{1.743185in}{1.648054in}}%
\pgfpathcurveto{\pgfqpoint{1.737361in}{1.653878in}}{\pgfqpoint{1.729461in}{1.657150in}}{\pgfqpoint{1.721225in}{1.657150in}}%
\pgfpathcurveto{\pgfqpoint{1.712988in}{1.657150in}}{\pgfqpoint{1.705088in}{1.653878in}}{\pgfqpoint{1.699265in}{1.648054in}}%
\pgfpathcurveto{\pgfqpoint{1.693441in}{1.642230in}}{\pgfqpoint{1.690168in}{1.634330in}}{\pgfqpoint{1.690168in}{1.626093in}}%
\pgfpathcurveto{\pgfqpoint{1.690168in}{1.617857in}}{\pgfqpoint{1.693441in}{1.609957in}}{\pgfqpoint{1.699265in}{1.604133in}}%
\pgfpathcurveto{\pgfqpoint{1.705088in}{1.598309in}}{\pgfqpoint{1.712988in}{1.595037in}}{\pgfqpoint{1.721225in}{1.595037in}}%
\pgfpathclose%
\pgfusepath{stroke,fill}%
\end{pgfscope}%
\begin{pgfscope}%
\pgfpathrectangle{\pgfqpoint{0.100000in}{0.212622in}}{\pgfqpoint{3.696000in}{3.696000in}}%
\pgfusepath{clip}%
\pgfsetbuttcap%
\pgfsetroundjoin%
\definecolor{currentfill}{rgb}{0.121569,0.466667,0.705882}%
\pgfsetfillcolor{currentfill}%
\pgfsetfillopacity{0.827923}%
\pgfsetlinewidth{1.003750pt}%
\definecolor{currentstroke}{rgb}{0.121569,0.466667,0.705882}%
\pgfsetstrokecolor{currentstroke}%
\pgfsetstrokeopacity{0.827923}%
\pgfsetdash{}{0pt}%
\pgfpathmoveto{\pgfqpoint{1.722043in}{1.590640in}}%
\pgfpathcurveto{\pgfqpoint{1.730279in}{1.590640in}}{\pgfqpoint{1.738179in}{1.593913in}}{\pgfqpoint{1.744003in}{1.599737in}}%
\pgfpathcurveto{\pgfqpoint{1.749827in}{1.605560in}}{\pgfqpoint{1.753100in}{1.613461in}}{\pgfqpoint{1.753100in}{1.621697in}}%
\pgfpathcurveto{\pgfqpoint{1.753100in}{1.629933in}}{\pgfqpoint{1.749827in}{1.637833in}}{\pgfqpoint{1.744003in}{1.643657in}}%
\pgfpathcurveto{\pgfqpoint{1.738179in}{1.649481in}}{\pgfqpoint{1.730279in}{1.652753in}}{\pgfqpoint{1.722043in}{1.652753in}}%
\pgfpathcurveto{\pgfqpoint{1.713807in}{1.652753in}}{\pgfqpoint{1.705907in}{1.649481in}}{\pgfqpoint{1.700083in}{1.643657in}}%
\pgfpathcurveto{\pgfqpoint{1.694259in}{1.637833in}}{\pgfqpoint{1.690987in}{1.629933in}}{\pgfqpoint{1.690987in}{1.621697in}}%
\pgfpathcurveto{\pgfqpoint{1.690987in}{1.613461in}}{\pgfqpoint{1.694259in}{1.605560in}}{\pgfqpoint{1.700083in}{1.599737in}}%
\pgfpathcurveto{\pgfqpoint{1.705907in}{1.593913in}}{\pgfqpoint{1.713807in}{1.590640in}}{\pgfqpoint{1.722043in}{1.590640in}}%
\pgfpathclose%
\pgfusepath{stroke,fill}%
\end{pgfscope}%
\begin{pgfscope}%
\pgfpathrectangle{\pgfqpoint{0.100000in}{0.212622in}}{\pgfqpoint{3.696000in}{3.696000in}}%
\pgfusepath{clip}%
\pgfsetbuttcap%
\pgfsetroundjoin%
\definecolor{currentfill}{rgb}{0.121569,0.466667,0.705882}%
\pgfsetfillcolor{currentfill}%
\pgfsetfillopacity{0.828921}%
\pgfsetlinewidth{1.003750pt}%
\definecolor{currentstroke}{rgb}{0.121569,0.466667,0.705882}%
\pgfsetstrokecolor{currentstroke}%
\pgfsetstrokeopacity{0.828921}%
\pgfsetdash{}{0pt}%
\pgfpathmoveto{\pgfqpoint{1.722680in}{1.590343in}}%
\pgfpathcurveto{\pgfqpoint{1.730916in}{1.590343in}}{\pgfqpoint{1.738817in}{1.593615in}}{\pgfqpoint{1.744640in}{1.599439in}}%
\pgfpathcurveto{\pgfqpoint{1.750464in}{1.605263in}}{\pgfqpoint{1.753737in}{1.613163in}}{\pgfqpoint{1.753737in}{1.621399in}}%
\pgfpathcurveto{\pgfqpoint{1.753737in}{1.629635in}}{\pgfqpoint{1.750464in}{1.637535in}}{\pgfqpoint{1.744640in}{1.643359in}}%
\pgfpathcurveto{\pgfqpoint{1.738817in}{1.649183in}}{\pgfqpoint{1.730916in}{1.652456in}}{\pgfqpoint{1.722680in}{1.652456in}}%
\pgfpathcurveto{\pgfqpoint{1.714444in}{1.652456in}}{\pgfqpoint{1.706544in}{1.649183in}}{\pgfqpoint{1.700720in}{1.643359in}}%
\pgfpathcurveto{\pgfqpoint{1.694896in}{1.637535in}}{\pgfqpoint{1.691624in}{1.629635in}}{\pgfqpoint{1.691624in}{1.621399in}}%
\pgfpathcurveto{\pgfqpoint{1.691624in}{1.613163in}}{\pgfqpoint{1.694896in}{1.605263in}}{\pgfqpoint{1.700720in}{1.599439in}}%
\pgfpathcurveto{\pgfqpoint{1.706544in}{1.593615in}}{\pgfqpoint{1.714444in}{1.590343in}}{\pgfqpoint{1.722680in}{1.590343in}}%
\pgfpathclose%
\pgfusepath{stroke,fill}%
\end{pgfscope}%
\begin{pgfscope}%
\pgfpathrectangle{\pgfqpoint{0.100000in}{0.212622in}}{\pgfqpoint{3.696000in}{3.696000in}}%
\pgfusepath{clip}%
\pgfsetbuttcap%
\pgfsetroundjoin%
\definecolor{currentfill}{rgb}{0.121569,0.466667,0.705882}%
\pgfsetfillcolor{currentfill}%
\pgfsetfillopacity{0.830421}%
\pgfsetlinewidth{1.003750pt}%
\definecolor{currentstroke}{rgb}{0.121569,0.466667,0.705882}%
\pgfsetstrokecolor{currentstroke}%
\pgfsetstrokeopacity{0.830421}%
\pgfsetdash{}{0pt}%
\pgfpathmoveto{\pgfqpoint{1.723575in}{1.591162in}}%
\pgfpathcurveto{\pgfqpoint{1.731811in}{1.591162in}}{\pgfqpoint{1.739711in}{1.594435in}}{\pgfqpoint{1.745535in}{1.600259in}}%
\pgfpathcurveto{\pgfqpoint{1.751359in}{1.606083in}}{\pgfqpoint{1.754631in}{1.613983in}}{\pgfqpoint{1.754631in}{1.622219in}}%
\pgfpathcurveto{\pgfqpoint{1.754631in}{1.630455in}}{\pgfqpoint{1.751359in}{1.638355in}}{\pgfqpoint{1.745535in}{1.644179in}}%
\pgfpathcurveto{\pgfqpoint{1.739711in}{1.650003in}}{\pgfqpoint{1.731811in}{1.653275in}}{\pgfqpoint{1.723575in}{1.653275in}}%
\pgfpathcurveto{\pgfqpoint{1.715338in}{1.653275in}}{\pgfqpoint{1.707438in}{1.650003in}}{\pgfqpoint{1.701614in}{1.644179in}}%
\pgfpathcurveto{\pgfqpoint{1.695790in}{1.638355in}}{\pgfqpoint{1.692518in}{1.630455in}}{\pgfqpoint{1.692518in}{1.622219in}}%
\pgfpathcurveto{\pgfqpoint{1.692518in}{1.613983in}}{\pgfqpoint{1.695790in}{1.606083in}}{\pgfqpoint{1.701614in}{1.600259in}}%
\pgfpathcurveto{\pgfqpoint{1.707438in}{1.594435in}}{\pgfqpoint{1.715338in}{1.591162in}}{\pgfqpoint{1.723575in}{1.591162in}}%
\pgfpathclose%
\pgfusepath{stroke,fill}%
\end{pgfscope}%
\begin{pgfscope}%
\pgfpathrectangle{\pgfqpoint{0.100000in}{0.212622in}}{\pgfqpoint{3.696000in}{3.696000in}}%
\pgfusepath{clip}%
\pgfsetbuttcap%
\pgfsetroundjoin%
\definecolor{currentfill}{rgb}{0.121569,0.466667,0.705882}%
\pgfsetfillcolor{currentfill}%
\pgfsetfillopacity{0.831009}%
\pgfsetlinewidth{1.003750pt}%
\definecolor{currentstroke}{rgb}{0.121569,0.466667,0.705882}%
\pgfsetstrokecolor{currentstroke}%
\pgfsetstrokeopacity{0.831009}%
\pgfsetdash{}{0pt}%
\pgfpathmoveto{\pgfqpoint{1.723808in}{1.590435in}}%
\pgfpathcurveto{\pgfqpoint{1.732044in}{1.590435in}}{\pgfqpoint{1.739944in}{1.593707in}}{\pgfqpoint{1.745768in}{1.599531in}}%
\pgfpathcurveto{\pgfqpoint{1.751592in}{1.605355in}}{\pgfqpoint{1.754864in}{1.613255in}}{\pgfqpoint{1.754864in}{1.621491in}}%
\pgfpathcurveto{\pgfqpoint{1.754864in}{1.629727in}}{\pgfqpoint{1.751592in}{1.637627in}}{\pgfqpoint{1.745768in}{1.643451in}}%
\pgfpathcurveto{\pgfqpoint{1.739944in}{1.649275in}}{\pgfqpoint{1.732044in}{1.652548in}}{\pgfqpoint{1.723808in}{1.652548in}}%
\pgfpathcurveto{\pgfqpoint{1.715572in}{1.652548in}}{\pgfqpoint{1.707672in}{1.649275in}}{\pgfqpoint{1.701848in}{1.643451in}}%
\pgfpathcurveto{\pgfqpoint{1.696024in}{1.637627in}}{\pgfqpoint{1.692751in}{1.629727in}}{\pgfqpoint{1.692751in}{1.621491in}}%
\pgfpathcurveto{\pgfqpoint{1.692751in}{1.613255in}}{\pgfqpoint{1.696024in}{1.605355in}}{\pgfqpoint{1.701848in}{1.599531in}}%
\pgfpathcurveto{\pgfqpoint{1.707672in}{1.593707in}}{\pgfqpoint{1.715572in}{1.590435in}}{\pgfqpoint{1.723808in}{1.590435in}}%
\pgfpathclose%
\pgfusepath{stroke,fill}%
\end{pgfscope}%
\begin{pgfscope}%
\pgfpathrectangle{\pgfqpoint{0.100000in}{0.212622in}}{\pgfqpoint{3.696000in}{3.696000in}}%
\pgfusepath{clip}%
\pgfsetbuttcap%
\pgfsetroundjoin%
\definecolor{currentfill}{rgb}{0.121569,0.466667,0.705882}%
\pgfsetfillcolor{currentfill}%
\pgfsetfillopacity{0.832195}%
\pgfsetlinewidth{1.003750pt}%
\definecolor{currentstroke}{rgb}{0.121569,0.466667,0.705882}%
\pgfsetstrokecolor{currentstroke}%
\pgfsetstrokeopacity{0.832195}%
\pgfsetdash{}{0pt}%
\pgfpathmoveto{\pgfqpoint{1.724114in}{1.590573in}}%
\pgfpathcurveto{\pgfqpoint{1.732350in}{1.590573in}}{\pgfqpoint{1.740250in}{1.593845in}}{\pgfqpoint{1.746074in}{1.599669in}}%
\pgfpathcurveto{\pgfqpoint{1.751898in}{1.605493in}}{\pgfqpoint{1.755170in}{1.613393in}}{\pgfqpoint{1.755170in}{1.621630in}}%
\pgfpathcurveto{\pgfqpoint{1.755170in}{1.629866in}}{\pgfqpoint{1.751898in}{1.637766in}}{\pgfqpoint{1.746074in}{1.643590in}}%
\pgfpathcurveto{\pgfqpoint{1.740250in}{1.649414in}}{\pgfqpoint{1.732350in}{1.652686in}}{\pgfqpoint{1.724114in}{1.652686in}}%
\pgfpathcurveto{\pgfqpoint{1.715878in}{1.652686in}}{\pgfqpoint{1.707978in}{1.649414in}}{\pgfqpoint{1.702154in}{1.643590in}}%
\pgfpathcurveto{\pgfqpoint{1.696330in}{1.637766in}}{\pgfqpoint{1.693057in}{1.629866in}}{\pgfqpoint{1.693057in}{1.621630in}}%
\pgfpathcurveto{\pgfqpoint{1.693057in}{1.613393in}}{\pgfqpoint{1.696330in}{1.605493in}}{\pgfqpoint{1.702154in}{1.599669in}}%
\pgfpathcurveto{\pgfqpoint{1.707978in}{1.593845in}}{\pgfqpoint{1.715878in}{1.590573in}}{\pgfqpoint{1.724114in}{1.590573in}}%
\pgfpathclose%
\pgfusepath{stroke,fill}%
\end{pgfscope}%
\begin{pgfscope}%
\pgfpathrectangle{\pgfqpoint{0.100000in}{0.212622in}}{\pgfqpoint{3.696000in}{3.696000in}}%
\pgfusepath{clip}%
\pgfsetbuttcap%
\pgfsetroundjoin%
\definecolor{currentfill}{rgb}{0.121569,0.466667,0.705882}%
\pgfsetfillcolor{currentfill}%
\pgfsetfillopacity{0.832660}%
\pgfsetlinewidth{1.003750pt}%
\definecolor{currentstroke}{rgb}{0.121569,0.466667,0.705882}%
\pgfsetstrokecolor{currentstroke}%
\pgfsetstrokeopacity{0.832660}%
\pgfsetdash{}{0pt}%
\pgfpathmoveto{\pgfqpoint{1.724546in}{1.589939in}}%
\pgfpathcurveto{\pgfqpoint{1.732782in}{1.589939in}}{\pgfqpoint{1.740682in}{1.593211in}}{\pgfqpoint{1.746506in}{1.599035in}}%
\pgfpathcurveto{\pgfqpoint{1.752330in}{1.604859in}}{\pgfqpoint{1.755603in}{1.612759in}}{\pgfqpoint{1.755603in}{1.620995in}}%
\pgfpathcurveto{\pgfqpoint{1.755603in}{1.629232in}}{\pgfqpoint{1.752330in}{1.637132in}}{\pgfqpoint{1.746506in}{1.642956in}}%
\pgfpathcurveto{\pgfqpoint{1.740682in}{1.648780in}}{\pgfqpoint{1.732782in}{1.652052in}}{\pgfqpoint{1.724546in}{1.652052in}}%
\pgfpathcurveto{\pgfqpoint{1.716310in}{1.652052in}}{\pgfqpoint{1.708410in}{1.648780in}}{\pgfqpoint{1.702586in}{1.642956in}}%
\pgfpathcurveto{\pgfqpoint{1.696762in}{1.637132in}}{\pgfqpoint{1.693490in}{1.629232in}}{\pgfqpoint{1.693490in}{1.620995in}}%
\pgfpathcurveto{\pgfqpoint{1.693490in}{1.612759in}}{\pgfqpoint{1.696762in}{1.604859in}}{\pgfqpoint{1.702586in}{1.599035in}}%
\pgfpathcurveto{\pgfqpoint{1.708410in}{1.593211in}}{\pgfqpoint{1.716310in}{1.589939in}}{\pgfqpoint{1.724546in}{1.589939in}}%
\pgfpathclose%
\pgfusepath{stroke,fill}%
\end{pgfscope}%
\begin{pgfscope}%
\pgfpathrectangle{\pgfqpoint{0.100000in}{0.212622in}}{\pgfqpoint{3.696000in}{3.696000in}}%
\pgfusepath{clip}%
\pgfsetbuttcap%
\pgfsetroundjoin%
\definecolor{currentfill}{rgb}{0.121569,0.466667,0.705882}%
\pgfsetfillcolor{currentfill}%
\pgfsetfillopacity{0.833714}%
\pgfsetlinewidth{1.003750pt}%
\definecolor{currentstroke}{rgb}{0.121569,0.466667,0.705882}%
\pgfsetstrokecolor{currentstroke}%
\pgfsetstrokeopacity{0.833714}%
\pgfsetdash{}{0pt}%
\pgfpathmoveto{\pgfqpoint{1.725205in}{1.590805in}}%
\pgfpathcurveto{\pgfqpoint{1.733442in}{1.590805in}}{\pgfqpoint{1.741342in}{1.594078in}}{\pgfqpoint{1.747166in}{1.599902in}}%
\pgfpathcurveto{\pgfqpoint{1.752990in}{1.605726in}}{\pgfqpoint{1.756262in}{1.613626in}}{\pgfqpoint{1.756262in}{1.621862in}}%
\pgfpathcurveto{\pgfqpoint{1.756262in}{1.630098in}}{\pgfqpoint{1.752990in}{1.637998in}}{\pgfqpoint{1.747166in}{1.643822in}}%
\pgfpathcurveto{\pgfqpoint{1.741342in}{1.649646in}}{\pgfqpoint{1.733442in}{1.652918in}}{\pgfqpoint{1.725205in}{1.652918in}}%
\pgfpathcurveto{\pgfqpoint{1.716969in}{1.652918in}}{\pgfqpoint{1.709069in}{1.649646in}}{\pgfqpoint{1.703245in}{1.643822in}}%
\pgfpathcurveto{\pgfqpoint{1.697421in}{1.637998in}}{\pgfqpoint{1.694149in}{1.630098in}}{\pgfqpoint{1.694149in}{1.621862in}}%
\pgfpathcurveto{\pgfqpoint{1.694149in}{1.613626in}}{\pgfqpoint{1.697421in}{1.605726in}}{\pgfqpoint{1.703245in}{1.599902in}}%
\pgfpathcurveto{\pgfqpoint{1.709069in}{1.594078in}}{\pgfqpoint{1.716969in}{1.590805in}}{\pgfqpoint{1.725205in}{1.590805in}}%
\pgfpathclose%
\pgfusepath{stroke,fill}%
\end{pgfscope}%
\begin{pgfscope}%
\pgfpathrectangle{\pgfqpoint{0.100000in}{0.212622in}}{\pgfqpoint{3.696000in}{3.696000in}}%
\pgfusepath{clip}%
\pgfsetbuttcap%
\pgfsetroundjoin%
\definecolor{currentfill}{rgb}{0.121569,0.466667,0.705882}%
\pgfsetfillcolor{currentfill}%
\pgfsetfillopacity{0.834187}%
\pgfsetlinewidth{1.003750pt}%
\definecolor{currentstroke}{rgb}{0.121569,0.466667,0.705882}%
\pgfsetstrokecolor{currentstroke}%
\pgfsetstrokeopacity{0.834187}%
\pgfsetdash{}{0pt}%
\pgfpathmoveto{\pgfqpoint{1.725352in}{1.590695in}}%
\pgfpathcurveto{\pgfqpoint{1.733589in}{1.590695in}}{\pgfqpoint{1.741489in}{1.593968in}}{\pgfqpoint{1.747313in}{1.599792in}}%
\pgfpathcurveto{\pgfqpoint{1.753137in}{1.605615in}}{\pgfqpoint{1.756409in}{1.613515in}}{\pgfqpoint{1.756409in}{1.621752in}}%
\pgfpathcurveto{\pgfqpoint{1.756409in}{1.629988in}}{\pgfqpoint{1.753137in}{1.637888in}}{\pgfqpoint{1.747313in}{1.643712in}}%
\pgfpathcurveto{\pgfqpoint{1.741489in}{1.649536in}}{\pgfqpoint{1.733589in}{1.652808in}}{\pgfqpoint{1.725352in}{1.652808in}}%
\pgfpathcurveto{\pgfqpoint{1.717116in}{1.652808in}}{\pgfqpoint{1.709216in}{1.649536in}}{\pgfqpoint{1.703392in}{1.643712in}}%
\pgfpathcurveto{\pgfqpoint{1.697568in}{1.637888in}}{\pgfqpoint{1.694296in}{1.629988in}}{\pgfqpoint{1.694296in}{1.621752in}}%
\pgfpathcurveto{\pgfqpoint{1.694296in}{1.613515in}}{\pgfqpoint{1.697568in}{1.605615in}}{\pgfqpoint{1.703392in}{1.599792in}}%
\pgfpathcurveto{\pgfqpoint{1.709216in}{1.593968in}}{\pgfqpoint{1.717116in}{1.590695in}}{\pgfqpoint{1.725352in}{1.590695in}}%
\pgfpathclose%
\pgfusepath{stroke,fill}%
\end{pgfscope}%
\begin{pgfscope}%
\pgfpathrectangle{\pgfqpoint{0.100000in}{0.212622in}}{\pgfqpoint{3.696000in}{3.696000in}}%
\pgfusepath{clip}%
\pgfsetbuttcap%
\pgfsetroundjoin%
\definecolor{currentfill}{rgb}{0.121569,0.466667,0.705882}%
\pgfsetfillcolor{currentfill}%
\pgfsetfillopacity{0.834768}%
\pgfsetlinewidth{1.003750pt}%
\definecolor{currentstroke}{rgb}{0.121569,0.466667,0.705882}%
\pgfsetstrokecolor{currentstroke}%
\pgfsetstrokeopacity{0.834768}%
\pgfsetdash{}{0pt}%
\pgfpathmoveto{\pgfqpoint{1.725670in}{1.589666in}}%
\pgfpathcurveto{\pgfqpoint{1.733906in}{1.589666in}}{\pgfqpoint{1.741806in}{1.592938in}}{\pgfqpoint{1.747630in}{1.598762in}}%
\pgfpathcurveto{\pgfqpoint{1.753454in}{1.604586in}}{\pgfqpoint{1.756727in}{1.612486in}}{\pgfqpoint{1.756727in}{1.620722in}}%
\pgfpathcurveto{\pgfqpoint{1.756727in}{1.628959in}}{\pgfqpoint{1.753454in}{1.636859in}}{\pgfqpoint{1.747630in}{1.642683in}}%
\pgfpathcurveto{\pgfqpoint{1.741806in}{1.648507in}}{\pgfqpoint{1.733906in}{1.651779in}}{\pgfqpoint{1.725670in}{1.651779in}}%
\pgfpathcurveto{\pgfqpoint{1.717434in}{1.651779in}}{\pgfqpoint{1.709534in}{1.648507in}}{\pgfqpoint{1.703710in}{1.642683in}}%
\pgfpathcurveto{\pgfqpoint{1.697886in}{1.636859in}}{\pgfqpoint{1.694614in}{1.628959in}}{\pgfqpoint{1.694614in}{1.620722in}}%
\pgfpathcurveto{\pgfqpoint{1.694614in}{1.612486in}}{\pgfqpoint{1.697886in}{1.604586in}}{\pgfqpoint{1.703710in}{1.598762in}}%
\pgfpathcurveto{\pgfqpoint{1.709534in}{1.592938in}}{\pgfqpoint{1.717434in}{1.589666in}}{\pgfqpoint{1.725670in}{1.589666in}}%
\pgfpathclose%
\pgfusepath{stroke,fill}%
\end{pgfscope}%
\begin{pgfscope}%
\pgfpathrectangle{\pgfqpoint{0.100000in}{0.212622in}}{\pgfqpoint{3.696000in}{3.696000in}}%
\pgfusepath{clip}%
\pgfsetbuttcap%
\pgfsetroundjoin%
\definecolor{currentfill}{rgb}{0.121569,0.466667,0.705882}%
\pgfsetfillcolor{currentfill}%
\pgfsetfillopacity{0.835230}%
\pgfsetlinewidth{1.003750pt}%
\definecolor{currentstroke}{rgb}{0.121569,0.466667,0.705882}%
\pgfsetstrokecolor{currentstroke}%
\pgfsetstrokeopacity{0.835230}%
\pgfsetdash{}{0pt}%
\pgfpathmoveto{\pgfqpoint{1.726384in}{1.587521in}}%
\pgfpathcurveto{\pgfqpoint{1.734621in}{1.587521in}}{\pgfqpoint{1.742521in}{1.590793in}}{\pgfqpoint{1.748345in}{1.596617in}}%
\pgfpathcurveto{\pgfqpoint{1.754169in}{1.602441in}}{\pgfqpoint{1.757441in}{1.610341in}}{\pgfqpoint{1.757441in}{1.618577in}}%
\pgfpathcurveto{\pgfqpoint{1.757441in}{1.626813in}}{\pgfqpoint{1.754169in}{1.634714in}}{\pgfqpoint{1.748345in}{1.640537in}}%
\pgfpathcurveto{\pgfqpoint{1.742521in}{1.646361in}}{\pgfqpoint{1.734621in}{1.649634in}}{\pgfqpoint{1.726384in}{1.649634in}}%
\pgfpathcurveto{\pgfqpoint{1.718148in}{1.649634in}}{\pgfqpoint{1.710248in}{1.646361in}}{\pgfqpoint{1.704424in}{1.640537in}}%
\pgfpathcurveto{\pgfqpoint{1.698600in}{1.634714in}}{\pgfqpoint{1.695328in}{1.626813in}}{\pgfqpoint{1.695328in}{1.618577in}}%
\pgfpathcurveto{\pgfqpoint{1.695328in}{1.610341in}}{\pgfqpoint{1.698600in}{1.602441in}}{\pgfqpoint{1.704424in}{1.596617in}}%
\pgfpathcurveto{\pgfqpoint{1.710248in}{1.590793in}}{\pgfqpoint{1.718148in}{1.587521in}}{\pgfqpoint{1.726384in}{1.587521in}}%
\pgfpathclose%
\pgfusepath{stroke,fill}%
\end{pgfscope}%
\begin{pgfscope}%
\pgfpathrectangle{\pgfqpoint{0.100000in}{0.212622in}}{\pgfqpoint{3.696000in}{3.696000in}}%
\pgfusepath{clip}%
\pgfsetbuttcap%
\pgfsetroundjoin%
\definecolor{currentfill}{rgb}{0.121569,0.466667,0.705882}%
\pgfsetfillcolor{currentfill}%
\pgfsetfillopacity{0.836664}%
\pgfsetlinewidth{1.003750pt}%
\definecolor{currentstroke}{rgb}{0.121569,0.466667,0.705882}%
\pgfsetstrokecolor{currentstroke}%
\pgfsetstrokeopacity{0.836664}%
\pgfsetdash{}{0pt}%
\pgfpathmoveto{\pgfqpoint{1.727118in}{1.587087in}}%
\pgfpathcurveto{\pgfqpoint{1.735354in}{1.587087in}}{\pgfqpoint{1.743254in}{1.590360in}}{\pgfqpoint{1.749078in}{1.596184in}}%
\pgfpathcurveto{\pgfqpoint{1.754902in}{1.602008in}}{\pgfqpoint{1.758175in}{1.609908in}}{\pgfqpoint{1.758175in}{1.618144in}}%
\pgfpathcurveto{\pgfqpoint{1.758175in}{1.626380in}}{\pgfqpoint{1.754902in}{1.634280in}}{\pgfqpoint{1.749078in}{1.640104in}}%
\pgfpathcurveto{\pgfqpoint{1.743254in}{1.645928in}}{\pgfqpoint{1.735354in}{1.649200in}}{\pgfqpoint{1.727118in}{1.649200in}}%
\pgfpathcurveto{\pgfqpoint{1.718882in}{1.649200in}}{\pgfqpoint{1.710982in}{1.645928in}}{\pgfqpoint{1.705158in}{1.640104in}}%
\pgfpathcurveto{\pgfqpoint{1.699334in}{1.634280in}}{\pgfqpoint{1.696062in}{1.626380in}}{\pgfqpoint{1.696062in}{1.618144in}}%
\pgfpathcurveto{\pgfqpoint{1.696062in}{1.609908in}}{\pgfqpoint{1.699334in}{1.602008in}}{\pgfqpoint{1.705158in}{1.596184in}}%
\pgfpathcurveto{\pgfqpoint{1.710982in}{1.590360in}}{\pgfqpoint{1.718882in}{1.587087in}}{\pgfqpoint{1.727118in}{1.587087in}}%
\pgfpathclose%
\pgfusepath{stroke,fill}%
\end{pgfscope}%
\begin{pgfscope}%
\pgfpathrectangle{\pgfqpoint{0.100000in}{0.212622in}}{\pgfqpoint{3.696000in}{3.696000in}}%
\pgfusepath{clip}%
\pgfsetbuttcap%
\pgfsetroundjoin%
\definecolor{currentfill}{rgb}{0.121569,0.466667,0.705882}%
\pgfsetfillcolor{currentfill}%
\pgfsetfillopacity{0.838891}%
\pgfsetlinewidth{1.003750pt}%
\definecolor{currentstroke}{rgb}{0.121569,0.466667,0.705882}%
\pgfsetstrokecolor{currentstroke}%
\pgfsetstrokeopacity{0.838891}%
\pgfsetdash{}{0pt}%
\pgfpathmoveto{\pgfqpoint{1.727937in}{1.586711in}}%
\pgfpathcurveto{\pgfqpoint{1.736173in}{1.586711in}}{\pgfqpoint{1.744073in}{1.589983in}}{\pgfqpoint{1.749897in}{1.595807in}}%
\pgfpathcurveto{\pgfqpoint{1.755721in}{1.601631in}}{\pgfqpoint{1.758994in}{1.609531in}}{\pgfqpoint{1.758994in}{1.617767in}}%
\pgfpathcurveto{\pgfqpoint{1.758994in}{1.626003in}}{\pgfqpoint{1.755721in}{1.633904in}}{\pgfqpoint{1.749897in}{1.639727in}}%
\pgfpathcurveto{\pgfqpoint{1.744073in}{1.645551in}}{\pgfqpoint{1.736173in}{1.648824in}}{\pgfqpoint{1.727937in}{1.648824in}}%
\pgfpathcurveto{\pgfqpoint{1.719701in}{1.648824in}}{\pgfqpoint{1.711801in}{1.645551in}}{\pgfqpoint{1.705977in}{1.639727in}}%
\pgfpathcurveto{\pgfqpoint{1.700153in}{1.633904in}}{\pgfqpoint{1.696881in}{1.626003in}}{\pgfqpoint{1.696881in}{1.617767in}}%
\pgfpathcurveto{\pgfqpoint{1.696881in}{1.609531in}}{\pgfqpoint{1.700153in}{1.601631in}}{\pgfqpoint{1.705977in}{1.595807in}}%
\pgfpathcurveto{\pgfqpoint{1.711801in}{1.589983in}}{\pgfqpoint{1.719701in}{1.586711in}}{\pgfqpoint{1.727937in}{1.586711in}}%
\pgfpathclose%
\pgfusepath{stroke,fill}%
\end{pgfscope}%
\begin{pgfscope}%
\pgfpathrectangle{\pgfqpoint{0.100000in}{0.212622in}}{\pgfqpoint{3.696000in}{3.696000in}}%
\pgfusepath{clip}%
\pgfsetbuttcap%
\pgfsetroundjoin%
\definecolor{currentfill}{rgb}{0.121569,0.466667,0.705882}%
\pgfsetfillcolor{currentfill}%
\pgfsetfillopacity{0.841481}%
\pgfsetlinewidth{1.003750pt}%
\definecolor{currentstroke}{rgb}{0.121569,0.466667,0.705882}%
\pgfsetstrokecolor{currentstroke}%
\pgfsetstrokeopacity{0.841481}%
\pgfsetdash{}{0pt}%
\pgfpathmoveto{\pgfqpoint{1.729415in}{1.585861in}}%
\pgfpathcurveto{\pgfqpoint{1.737651in}{1.585861in}}{\pgfqpoint{1.745551in}{1.589134in}}{\pgfqpoint{1.751375in}{1.594958in}}%
\pgfpathcurveto{\pgfqpoint{1.757199in}{1.600782in}}{\pgfqpoint{1.760471in}{1.608682in}}{\pgfqpoint{1.760471in}{1.616918in}}%
\pgfpathcurveto{\pgfqpoint{1.760471in}{1.625154in}}{\pgfqpoint{1.757199in}{1.633054in}}{\pgfqpoint{1.751375in}{1.638878in}}%
\pgfpathcurveto{\pgfqpoint{1.745551in}{1.644702in}}{\pgfqpoint{1.737651in}{1.647974in}}{\pgfqpoint{1.729415in}{1.647974in}}%
\pgfpathcurveto{\pgfqpoint{1.721178in}{1.647974in}}{\pgfqpoint{1.713278in}{1.644702in}}{\pgfqpoint{1.707455in}{1.638878in}}%
\pgfpathcurveto{\pgfqpoint{1.701631in}{1.633054in}}{\pgfqpoint{1.698358in}{1.625154in}}{\pgfqpoint{1.698358in}{1.616918in}}%
\pgfpathcurveto{\pgfqpoint{1.698358in}{1.608682in}}{\pgfqpoint{1.701631in}{1.600782in}}{\pgfqpoint{1.707455in}{1.594958in}}%
\pgfpathcurveto{\pgfqpoint{1.713278in}{1.589134in}}{\pgfqpoint{1.721178in}{1.585861in}}{\pgfqpoint{1.729415in}{1.585861in}}%
\pgfpathclose%
\pgfusepath{stroke,fill}%
\end{pgfscope}%
\begin{pgfscope}%
\pgfpathrectangle{\pgfqpoint{0.100000in}{0.212622in}}{\pgfqpoint{3.696000in}{3.696000in}}%
\pgfusepath{clip}%
\pgfsetbuttcap%
\pgfsetroundjoin%
\definecolor{currentfill}{rgb}{0.121569,0.466667,0.705882}%
\pgfsetfillcolor{currentfill}%
\pgfsetfillopacity{0.842845}%
\pgfsetlinewidth{1.003750pt}%
\definecolor{currentstroke}{rgb}{0.121569,0.466667,0.705882}%
\pgfsetstrokecolor{currentstroke}%
\pgfsetstrokeopacity{0.842845}%
\pgfsetdash{}{0pt}%
\pgfpathmoveto{\pgfqpoint{1.730016in}{1.585029in}}%
\pgfpathcurveto{\pgfqpoint{1.738252in}{1.585029in}}{\pgfqpoint{1.746152in}{1.588301in}}{\pgfqpoint{1.751976in}{1.594125in}}%
\pgfpathcurveto{\pgfqpoint{1.757800in}{1.599949in}}{\pgfqpoint{1.761072in}{1.607849in}}{\pgfqpoint{1.761072in}{1.616086in}}%
\pgfpathcurveto{\pgfqpoint{1.761072in}{1.624322in}}{\pgfqpoint{1.757800in}{1.632222in}}{\pgfqpoint{1.751976in}{1.638046in}}%
\pgfpathcurveto{\pgfqpoint{1.746152in}{1.643870in}}{\pgfqpoint{1.738252in}{1.647142in}}{\pgfqpoint{1.730016in}{1.647142in}}%
\pgfpathcurveto{\pgfqpoint{1.721780in}{1.647142in}}{\pgfqpoint{1.713880in}{1.643870in}}{\pgfqpoint{1.708056in}{1.638046in}}%
\pgfpathcurveto{\pgfqpoint{1.702232in}{1.632222in}}{\pgfqpoint{1.698959in}{1.624322in}}{\pgfqpoint{1.698959in}{1.616086in}}%
\pgfpathcurveto{\pgfqpoint{1.698959in}{1.607849in}}{\pgfqpoint{1.702232in}{1.599949in}}{\pgfqpoint{1.708056in}{1.594125in}}%
\pgfpathcurveto{\pgfqpoint{1.713880in}{1.588301in}}{\pgfqpoint{1.721780in}{1.585029in}}{\pgfqpoint{1.730016in}{1.585029in}}%
\pgfpathclose%
\pgfusepath{stroke,fill}%
\end{pgfscope}%
\begin{pgfscope}%
\pgfpathrectangle{\pgfqpoint{0.100000in}{0.212622in}}{\pgfqpoint{3.696000in}{3.696000in}}%
\pgfusepath{clip}%
\pgfsetbuttcap%
\pgfsetroundjoin%
\definecolor{currentfill}{rgb}{0.121569,0.466667,0.705882}%
\pgfsetfillcolor{currentfill}%
\pgfsetfillopacity{0.844206}%
\pgfsetlinewidth{1.003750pt}%
\definecolor{currentstroke}{rgb}{0.121569,0.466667,0.705882}%
\pgfsetstrokecolor{currentstroke}%
\pgfsetstrokeopacity{0.844206}%
\pgfsetdash{}{0pt}%
\pgfpathmoveto{\pgfqpoint{1.730762in}{1.582490in}}%
\pgfpathcurveto{\pgfqpoint{1.738998in}{1.582490in}}{\pgfqpoint{1.746898in}{1.585762in}}{\pgfqpoint{1.752722in}{1.591586in}}%
\pgfpathcurveto{\pgfqpoint{1.758546in}{1.597410in}}{\pgfqpoint{1.761818in}{1.605310in}}{\pgfqpoint{1.761818in}{1.613546in}}%
\pgfpathcurveto{\pgfqpoint{1.761818in}{1.621783in}}{\pgfqpoint{1.758546in}{1.629683in}}{\pgfqpoint{1.752722in}{1.635507in}}%
\pgfpathcurveto{\pgfqpoint{1.746898in}{1.641331in}}{\pgfqpoint{1.738998in}{1.644603in}}{\pgfqpoint{1.730762in}{1.644603in}}%
\pgfpathcurveto{\pgfqpoint{1.722525in}{1.644603in}}{\pgfqpoint{1.714625in}{1.641331in}}{\pgfqpoint{1.708802in}{1.635507in}}%
\pgfpathcurveto{\pgfqpoint{1.702978in}{1.629683in}}{\pgfqpoint{1.699705in}{1.621783in}}{\pgfqpoint{1.699705in}{1.613546in}}%
\pgfpathcurveto{\pgfqpoint{1.699705in}{1.605310in}}{\pgfqpoint{1.702978in}{1.597410in}}{\pgfqpoint{1.708802in}{1.591586in}}%
\pgfpathcurveto{\pgfqpoint{1.714625in}{1.585762in}}{\pgfqpoint{1.722525in}{1.582490in}}{\pgfqpoint{1.730762in}{1.582490in}}%
\pgfpathclose%
\pgfusepath{stroke,fill}%
\end{pgfscope}%
\begin{pgfscope}%
\pgfpathrectangle{\pgfqpoint{0.100000in}{0.212622in}}{\pgfqpoint{3.696000in}{3.696000in}}%
\pgfusepath{clip}%
\pgfsetbuttcap%
\pgfsetroundjoin%
\definecolor{currentfill}{rgb}{0.121569,0.466667,0.705882}%
\pgfsetfillcolor{currentfill}%
\pgfsetfillopacity{0.845379}%
\pgfsetlinewidth{1.003750pt}%
\definecolor{currentstroke}{rgb}{0.121569,0.466667,0.705882}%
\pgfsetstrokecolor{currentstroke}%
\pgfsetstrokeopacity{0.845379}%
\pgfsetdash{}{0pt}%
\pgfpathmoveto{\pgfqpoint{1.731228in}{1.583001in}}%
\pgfpathcurveto{\pgfqpoint{1.739464in}{1.583001in}}{\pgfqpoint{1.747364in}{1.586273in}}{\pgfqpoint{1.753188in}{1.592097in}}%
\pgfpathcurveto{\pgfqpoint{1.759012in}{1.597921in}}{\pgfqpoint{1.762285in}{1.605821in}}{\pgfqpoint{1.762285in}{1.614057in}}%
\pgfpathcurveto{\pgfqpoint{1.762285in}{1.622294in}}{\pgfqpoint{1.759012in}{1.630194in}}{\pgfqpoint{1.753188in}{1.636018in}}%
\pgfpathcurveto{\pgfqpoint{1.747364in}{1.641841in}}{\pgfqpoint{1.739464in}{1.645114in}}{\pgfqpoint{1.731228in}{1.645114in}}%
\pgfpathcurveto{\pgfqpoint{1.722992in}{1.645114in}}{\pgfqpoint{1.715092in}{1.641841in}}{\pgfqpoint{1.709268in}{1.636018in}}%
\pgfpathcurveto{\pgfqpoint{1.703444in}{1.630194in}}{\pgfqpoint{1.700172in}{1.622294in}}{\pgfqpoint{1.700172in}{1.614057in}}%
\pgfpathcurveto{\pgfqpoint{1.700172in}{1.605821in}}{\pgfqpoint{1.703444in}{1.597921in}}{\pgfqpoint{1.709268in}{1.592097in}}%
\pgfpathcurveto{\pgfqpoint{1.715092in}{1.586273in}}{\pgfqpoint{1.722992in}{1.583001in}}{\pgfqpoint{1.731228in}{1.583001in}}%
\pgfpathclose%
\pgfusepath{stroke,fill}%
\end{pgfscope}%
\begin{pgfscope}%
\pgfpathrectangle{\pgfqpoint{0.100000in}{0.212622in}}{\pgfqpoint{3.696000in}{3.696000in}}%
\pgfusepath{clip}%
\pgfsetbuttcap%
\pgfsetroundjoin%
\definecolor{currentfill}{rgb}{0.121569,0.466667,0.705882}%
\pgfsetfillcolor{currentfill}%
\pgfsetfillopacity{0.846604}%
\pgfsetlinewidth{1.003750pt}%
\definecolor{currentstroke}{rgb}{0.121569,0.466667,0.705882}%
\pgfsetstrokecolor{currentstroke}%
\pgfsetstrokeopacity{0.846604}%
\pgfsetdash{}{0pt}%
\pgfpathmoveto{\pgfqpoint{1.731771in}{1.582712in}}%
\pgfpathcurveto{\pgfqpoint{1.740007in}{1.582712in}}{\pgfqpoint{1.747907in}{1.585984in}}{\pgfqpoint{1.753731in}{1.591808in}}%
\pgfpathcurveto{\pgfqpoint{1.759555in}{1.597632in}}{\pgfqpoint{1.762827in}{1.605532in}}{\pgfqpoint{1.762827in}{1.613768in}}%
\pgfpathcurveto{\pgfqpoint{1.762827in}{1.622004in}}{\pgfqpoint{1.759555in}{1.629904in}}{\pgfqpoint{1.753731in}{1.635728in}}%
\pgfpathcurveto{\pgfqpoint{1.747907in}{1.641552in}}{\pgfqpoint{1.740007in}{1.644825in}}{\pgfqpoint{1.731771in}{1.644825in}}%
\pgfpathcurveto{\pgfqpoint{1.723534in}{1.644825in}}{\pgfqpoint{1.715634in}{1.641552in}}{\pgfqpoint{1.709810in}{1.635728in}}%
\pgfpathcurveto{\pgfqpoint{1.703986in}{1.629904in}}{\pgfqpoint{1.700714in}{1.622004in}}{\pgfqpoint{1.700714in}{1.613768in}}%
\pgfpathcurveto{\pgfqpoint{1.700714in}{1.605532in}}{\pgfqpoint{1.703986in}{1.597632in}}{\pgfqpoint{1.709810in}{1.591808in}}%
\pgfpathcurveto{\pgfqpoint{1.715634in}{1.585984in}}{\pgfqpoint{1.723534in}{1.582712in}}{\pgfqpoint{1.731771in}{1.582712in}}%
\pgfpathclose%
\pgfusepath{stroke,fill}%
\end{pgfscope}%
\begin{pgfscope}%
\pgfpathrectangle{\pgfqpoint{0.100000in}{0.212622in}}{\pgfqpoint{3.696000in}{3.696000in}}%
\pgfusepath{clip}%
\pgfsetbuttcap%
\pgfsetroundjoin%
\definecolor{currentfill}{rgb}{0.121569,0.466667,0.705882}%
\pgfsetfillcolor{currentfill}%
\pgfsetfillopacity{0.847142}%
\pgfsetlinewidth{1.003750pt}%
\definecolor{currentstroke}{rgb}{0.121569,0.466667,0.705882}%
\pgfsetstrokecolor{currentstroke}%
\pgfsetstrokeopacity{0.847142}%
\pgfsetdash{}{0pt}%
\pgfpathmoveto{\pgfqpoint{1.731970in}{1.581910in}}%
\pgfpathcurveto{\pgfqpoint{1.740206in}{1.581910in}}{\pgfqpoint{1.748106in}{1.585182in}}{\pgfqpoint{1.753930in}{1.591006in}}%
\pgfpathcurveto{\pgfqpoint{1.759754in}{1.596830in}}{\pgfqpoint{1.763026in}{1.604730in}}{\pgfqpoint{1.763026in}{1.612966in}}%
\pgfpathcurveto{\pgfqpoint{1.763026in}{1.621202in}}{\pgfqpoint{1.759754in}{1.629102in}}{\pgfqpoint{1.753930in}{1.634926in}}%
\pgfpathcurveto{\pgfqpoint{1.748106in}{1.640750in}}{\pgfqpoint{1.740206in}{1.644023in}}{\pgfqpoint{1.731970in}{1.644023in}}%
\pgfpathcurveto{\pgfqpoint{1.723733in}{1.644023in}}{\pgfqpoint{1.715833in}{1.640750in}}{\pgfqpoint{1.710009in}{1.634926in}}%
\pgfpathcurveto{\pgfqpoint{1.704185in}{1.629102in}}{\pgfqpoint{1.700913in}{1.621202in}}{\pgfqpoint{1.700913in}{1.612966in}}%
\pgfpathcurveto{\pgfqpoint{1.700913in}{1.604730in}}{\pgfqpoint{1.704185in}{1.596830in}}{\pgfqpoint{1.710009in}{1.591006in}}%
\pgfpathcurveto{\pgfqpoint{1.715833in}{1.585182in}}{\pgfqpoint{1.723733in}{1.581910in}}{\pgfqpoint{1.731970in}{1.581910in}}%
\pgfpathclose%
\pgfusepath{stroke,fill}%
\end{pgfscope}%
\begin{pgfscope}%
\pgfpathrectangle{\pgfqpoint{0.100000in}{0.212622in}}{\pgfqpoint{3.696000in}{3.696000in}}%
\pgfusepath{clip}%
\pgfsetbuttcap%
\pgfsetroundjoin%
\definecolor{currentfill}{rgb}{0.121569,0.466667,0.705882}%
\pgfsetfillcolor{currentfill}%
\pgfsetfillopacity{0.847685}%
\pgfsetlinewidth{1.003750pt}%
\definecolor{currentstroke}{rgb}{0.121569,0.466667,0.705882}%
\pgfsetstrokecolor{currentstroke}%
\pgfsetstrokeopacity{0.847685}%
\pgfsetdash{}{0pt}%
\pgfpathmoveto{\pgfqpoint{1.732479in}{1.579885in}}%
\pgfpathcurveto{\pgfqpoint{1.740715in}{1.579885in}}{\pgfqpoint{1.748615in}{1.583157in}}{\pgfqpoint{1.754439in}{1.588981in}}%
\pgfpathcurveto{\pgfqpoint{1.760263in}{1.594805in}}{\pgfqpoint{1.763535in}{1.602705in}}{\pgfqpoint{1.763535in}{1.610942in}}%
\pgfpathcurveto{\pgfqpoint{1.763535in}{1.619178in}}{\pgfqpoint{1.760263in}{1.627078in}}{\pgfqpoint{1.754439in}{1.632902in}}%
\pgfpathcurveto{\pgfqpoint{1.748615in}{1.638726in}}{\pgfqpoint{1.740715in}{1.641998in}}{\pgfqpoint{1.732479in}{1.641998in}}%
\pgfpathcurveto{\pgfqpoint{1.724242in}{1.641998in}}{\pgfqpoint{1.716342in}{1.638726in}}{\pgfqpoint{1.710518in}{1.632902in}}%
\pgfpathcurveto{\pgfqpoint{1.704694in}{1.627078in}}{\pgfqpoint{1.701422in}{1.619178in}}{\pgfqpoint{1.701422in}{1.610942in}}%
\pgfpathcurveto{\pgfqpoint{1.701422in}{1.602705in}}{\pgfqpoint{1.704694in}{1.594805in}}{\pgfqpoint{1.710518in}{1.588981in}}%
\pgfpathcurveto{\pgfqpoint{1.716342in}{1.583157in}}{\pgfqpoint{1.724242in}{1.579885in}}{\pgfqpoint{1.732479in}{1.579885in}}%
\pgfpathclose%
\pgfusepath{stroke,fill}%
\end{pgfscope}%
\begin{pgfscope}%
\pgfpathrectangle{\pgfqpoint{0.100000in}{0.212622in}}{\pgfqpoint{3.696000in}{3.696000in}}%
\pgfusepath{clip}%
\pgfsetbuttcap%
\pgfsetroundjoin%
\definecolor{currentfill}{rgb}{0.121569,0.466667,0.705882}%
\pgfsetfillcolor{currentfill}%
\pgfsetfillopacity{0.849254}%
\pgfsetlinewidth{1.003750pt}%
\definecolor{currentstroke}{rgb}{0.121569,0.466667,0.705882}%
\pgfsetstrokecolor{currentstroke}%
\pgfsetstrokeopacity{0.849254}%
\pgfsetdash{}{0pt}%
\pgfpathmoveto{\pgfqpoint{1.733287in}{1.580750in}}%
\pgfpathcurveto{\pgfqpoint{1.741524in}{1.580750in}}{\pgfqpoint{1.749424in}{1.584022in}}{\pgfqpoint{1.755248in}{1.589846in}}%
\pgfpathcurveto{\pgfqpoint{1.761071in}{1.595670in}}{\pgfqpoint{1.764344in}{1.603570in}}{\pgfqpoint{1.764344in}{1.611807in}}%
\pgfpathcurveto{\pgfqpoint{1.764344in}{1.620043in}}{\pgfqpoint{1.761071in}{1.627943in}}{\pgfqpoint{1.755248in}{1.633767in}}%
\pgfpathcurveto{\pgfqpoint{1.749424in}{1.639591in}}{\pgfqpoint{1.741524in}{1.642863in}}{\pgfqpoint{1.733287in}{1.642863in}}%
\pgfpathcurveto{\pgfqpoint{1.725051in}{1.642863in}}{\pgfqpoint{1.717151in}{1.639591in}}{\pgfqpoint{1.711327in}{1.633767in}}%
\pgfpathcurveto{\pgfqpoint{1.705503in}{1.627943in}}{\pgfqpoint{1.702231in}{1.620043in}}{\pgfqpoint{1.702231in}{1.611807in}}%
\pgfpathcurveto{\pgfqpoint{1.702231in}{1.603570in}}{\pgfqpoint{1.705503in}{1.595670in}}{\pgfqpoint{1.711327in}{1.589846in}}%
\pgfpathcurveto{\pgfqpoint{1.717151in}{1.584022in}}{\pgfqpoint{1.725051in}{1.580750in}}{\pgfqpoint{1.733287in}{1.580750in}}%
\pgfpathclose%
\pgfusepath{stroke,fill}%
\end{pgfscope}%
\begin{pgfscope}%
\pgfpathrectangle{\pgfqpoint{0.100000in}{0.212622in}}{\pgfqpoint{3.696000in}{3.696000in}}%
\pgfusepath{clip}%
\pgfsetbuttcap%
\pgfsetroundjoin%
\definecolor{currentfill}{rgb}{0.121569,0.466667,0.705882}%
\pgfsetfillcolor{currentfill}%
\pgfsetfillopacity{0.851654}%
\pgfsetlinewidth{1.003750pt}%
\definecolor{currentstroke}{rgb}{0.121569,0.466667,0.705882}%
\pgfsetstrokecolor{currentstroke}%
\pgfsetstrokeopacity{0.851654}%
\pgfsetdash{}{0pt}%
\pgfpathmoveto{\pgfqpoint{1.734365in}{1.581418in}}%
\pgfpathcurveto{\pgfqpoint{1.742601in}{1.581418in}}{\pgfqpoint{1.750502in}{1.584690in}}{\pgfqpoint{1.756325in}{1.590514in}}%
\pgfpathcurveto{\pgfqpoint{1.762149in}{1.596338in}}{\pgfqpoint{1.765422in}{1.604238in}}{\pgfqpoint{1.765422in}{1.612474in}}%
\pgfpathcurveto{\pgfqpoint{1.765422in}{1.620711in}}{\pgfqpoint{1.762149in}{1.628611in}}{\pgfqpoint{1.756325in}{1.634435in}}%
\pgfpathcurveto{\pgfqpoint{1.750502in}{1.640259in}}{\pgfqpoint{1.742601in}{1.643531in}}{\pgfqpoint{1.734365in}{1.643531in}}%
\pgfpathcurveto{\pgfqpoint{1.726129in}{1.643531in}}{\pgfqpoint{1.718229in}{1.640259in}}{\pgfqpoint{1.712405in}{1.634435in}}%
\pgfpathcurveto{\pgfqpoint{1.706581in}{1.628611in}}{\pgfqpoint{1.703309in}{1.620711in}}{\pgfqpoint{1.703309in}{1.612474in}}%
\pgfpathcurveto{\pgfqpoint{1.703309in}{1.604238in}}{\pgfqpoint{1.706581in}{1.596338in}}{\pgfqpoint{1.712405in}{1.590514in}}%
\pgfpathcurveto{\pgfqpoint{1.718229in}{1.584690in}}{\pgfqpoint{1.726129in}{1.581418in}}{\pgfqpoint{1.734365in}{1.581418in}}%
\pgfpathclose%
\pgfusepath{stroke,fill}%
\end{pgfscope}%
\begin{pgfscope}%
\pgfpathrectangle{\pgfqpoint{0.100000in}{0.212622in}}{\pgfqpoint{3.696000in}{3.696000in}}%
\pgfusepath{clip}%
\pgfsetbuttcap%
\pgfsetroundjoin%
\definecolor{currentfill}{rgb}{0.121569,0.466667,0.705882}%
\pgfsetfillcolor{currentfill}%
\pgfsetfillopacity{0.853869}%
\pgfsetlinewidth{1.003750pt}%
\definecolor{currentstroke}{rgb}{0.121569,0.466667,0.705882}%
\pgfsetstrokecolor{currentstroke}%
\pgfsetstrokeopacity{0.853869}%
\pgfsetdash{}{0pt}%
\pgfpathmoveto{\pgfqpoint{1.734883in}{1.579806in}}%
\pgfpathcurveto{\pgfqpoint{1.743120in}{1.579806in}}{\pgfqpoint{1.751020in}{1.583079in}}{\pgfqpoint{1.756844in}{1.588903in}}%
\pgfpathcurveto{\pgfqpoint{1.762668in}{1.594726in}}{\pgfqpoint{1.765940in}{1.602627in}}{\pgfqpoint{1.765940in}{1.610863in}}%
\pgfpathcurveto{\pgfqpoint{1.765940in}{1.619099in}}{\pgfqpoint{1.762668in}{1.626999in}}{\pgfqpoint{1.756844in}{1.632823in}}%
\pgfpathcurveto{\pgfqpoint{1.751020in}{1.638647in}}{\pgfqpoint{1.743120in}{1.641919in}}{\pgfqpoint{1.734883in}{1.641919in}}%
\pgfpathcurveto{\pgfqpoint{1.726647in}{1.641919in}}{\pgfqpoint{1.718747in}{1.638647in}}{\pgfqpoint{1.712923in}{1.632823in}}%
\pgfpathcurveto{\pgfqpoint{1.707099in}{1.626999in}}{\pgfqpoint{1.703827in}{1.619099in}}{\pgfqpoint{1.703827in}{1.610863in}}%
\pgfpathcurveto{\pgfqpoint{1.703827in}{1.602627in}}{\pgfqpoint{1.707099in}{1.594726in}}{\pgfqpoint{1.712923in}{1.588903in}}%
\pgfpathcurveto{\pgfqpoint{1.718747in}{1.583079in}}{\pgfqpoint{1.726647in}{1.579806in}}{\pgfqpoint{1.734883in}{1.579806in}}%
\pgfpathclose%
\pgfusepath{stroke,fill}%
\end{pgfscope}%
\begin{pgfscope}%
\pgfpathrectangle{\pgfqpoint{0.100000in}{0.212622in}}{\pgfqpoint{3.696000in}{3.696000in}}%
\pgfusepath{clip}%
\pgfsetbuttcap%
\pgfsetroundjoin%
\definecolor{currentfill}{rgb}{0.121569,0.466667,0.705882}%
\pgfsetfillcolor{currentfill}%
\pgfsetfillopacity{0.856171}%
\pgfsetlinewidth{1.003750pt}%
\definecolor{currentstroke}{rgb}{0.121569,0.466667,0.705882}%
\pgfsetstrokecolor{currentstroke}%
\pgfsetstrokeopacity{0.856171}%
\pgfsetdash{}{0pt}%
\pgfpathmoveto{\pgfqpoint{1.736419in}{1.578197in}}%
\pgfpathcurveto{\pgfqpoint{1.744655in}{1.578197in}}{\pgfqpoint{1.752555in}{1.581469in}}{\pgfqpoint{1.758379in}{1.587293in}}%
\pgfpathcurveto{\pgfqpoint{1.764203in}{1.593117in}}{\pgfqpoint{1.767476in}{1.601017in}}{\pgfqpoint{1.767476in}{1.609253in}}%
\pgfpathcurveto{\pgfqpoint{1.767476in}{1.617490in}}{\pgfqpoint{1.764203in}{1.625390in}}{\pgfqpoint{1.758379in}{1.631214in}}%
\pgfpathcurveto{\pgfqpoint{1.752555in}{1.637038in}}{\pgfqpoint{1.744655in}{1.640310in}}{\pgfqpoint{1.736419in}{1.640310in}}%
\pgfpathcurveto{\pgfqpoint{1.728183in}{1.640310in}}{\pgfqpoint{1.720283in}{1.637038in}}{\pgfqpoint{1.714459in}{1.631214in}}%
\pgfpathcurveto{\pgfqpoint{1.708635in}{1.625390in}}{\pgfqpoint{1.705363in}{1.617490in}}{\pgfqpoint{1.705363in}{1.609253in}}%
\pgfpathcurveto{\pgfqpoint{1.705363in}{1.601017in}}{\pgfqpoint{1.708635in}{1.593117in}}{\pgfqpoint{1.714459in}{1.587293in}}%
\pgfpathcurveto{\pgfqpoint{1.720283in}{1.581469in}}{\pgfqpoint{1.728183in}{1.578197in}}{\pgfqpoint{1.736419in}{1.578197in}}%
\pgfpathclose%
\pgfusepath{stroke,fill}%
\end{pgfscope}%
\begin{pgfscope}%
\pgfpathrectangle{\pgfqpoint{0.100000in}{0.212622in}}{\pgfqpoint{3.696000in}{3.696000in}}%
\pgfusepath{clip}%
\pgfsetbuttcap%
\pgfsetroundjoin%
\definecolor{currentfill}{rgb}{0.121569,0.466667,0.705882}%
\pgfsetfillcolor{currentfill}%
\pgfsetfillopacity{0.858567}%
\pgfsetlinewidth{1.003750pt}%
\definecolor{currentstroke}{rgb}{0.121569,0.466667,0.705882}%
\pgfsetstrokecolor{currentstroke}%
\pgfsetstrokeopacity{0.858567}%
\pgfsetdash{}{0pt}%
\pgfpathmoveto{\pgfqpoint{1.737836in}{1.575690in}}%
\pgfpathcurveto{\pgfqpoint{1.746072in}{1.575690in}}{\pgfqpoint{1.753972in}{1.578962in}}{\pgfqpoint{1.759796in}{1.584786in}}%
\pgfpathcurveto{\pgfqpoint{1.765620in}{1.590610in}}{\pgfqpoint{1.768893in}{1.598510in}}{\pgfqpoint{1.768893in}{1.606746in}}%
\pgfpathcurveto{\pgfqpoint{1.768893in}{1.614983in}}{\pgfqpoint{1.765620in}{1.622883in}}{\pgfqpoint{1.759796in}{1.628707in}}%
\pgfpathcurveto{\pgfqpoint{1.753972in}{1.634531in}}{\pgfqpoint{1.746072in}{1.637803in}}{\pgfqpoint{1.737836in}{1.637803in}}%
\pgfpathcurveto{\pgfqpoint{1.729600in}{1.637803in}}{\pgfqpoint{1.721700in}{1.634531in}}{\pgfqpoint{1.715876in}{1.628707in}}%
\pgfpathcurveto{\pgfqpoint{1.710052in}{1.622883in}}{\pgfqpoint{1.706780in}{1.614983in}}{\pgfqpoint{1.706780in}{1.606746in}}%
\pgfpathcurveto{\pgfqpoint{1.706780in}{1.598510in}}{\pgfqpoint{1.710052in}{1.590610in}}{\pgfqpoint{1.715876in}{1.584786in}}%
\pgfpathcurveto{\pgfqpoint{1.721700in}{1.578962in}}{\pgfqpoint{1.729600in}{1.575690in}}{\pgfqpoint{1.737836in}{1.575690in}}%
\pgfpathclose%
\pgfusepath{stroke,fill}%
\end{pgfscope}%
\begin{pgfscope}%
\pgfpathrectangle{\pgfqpoint{0.100000in}{0.212622in}}{\pgfqpoint{3.696000in}{3.696000in}}%
\pgfusepath{clip}%
\pgfsetbuttcap%
\pgfsetroundjoin%
\definecolor{currentfill}{rgb}{0.121569,0.466667,0.705882}%
\pgfsetfillcolor{currentfill}%
\pgfsetfillopacity{0.862655}%
\pgfsetlinewidth{1.003750pt}%
\definecolor{currentstroke}{rgb}{0.121569,0.466667,0.705882}%
\pgfsetstrokecolor{currentstroke}%
\pgfsetstrokeopacity{0.862655}%
\pgfsetdash{}{0pt}%
\pgfpathmoveto{\pgfqpoint{1.740444in}{1.580359in}}%
\pgfpathcurveto{\pgfqpoint{1.748680in}{1.580359in}}{\pgfqpoint{1.756580in}{1.583631in}}{\pgfqpoint{1.762404in}{1.589455in}}%
\pgfpathcurveto{\pgfqpoint{1.768228in}{1.595279in}}{\pgfqpoint{1.771501in}{1.603179in}}{\pgfqpoint{1.771501in}{1.611416in}}%
\pgfpathcurveto{\pgfqpoint{1.771501in}{1.619652in}}{\pgfqpoint{1.768228in}{1.627552in}}{\pgfqpoint{1.762404in}{1.633376in}}%
\pgfpathcurveto{\pgfqpoint{1.756580in}{1.639200in}}{\pgfqpoint{1.748680in}{1.642472in}}{\pgfqpoint{1.740444in}{1.642472in}}%
\pgfpathcurveto{\pgfqpoint{1.732208in}{1.642472in}}{\pgfqpoint{1.724308in}{1.639200in}}{\pgfqpoint{1.718484in}{1.633376in}}%
\pgfpathcurveto{\pgfqpoint{1.712660in}{1.627552in}}{\pgfqpoint{1.709388in}{1.619652in}}{\pgfqpoint{1.709388in}{1.611416in}}%
\pgfpathcurveto{\pgfqpoint{1.709388in}{1.603179in}}{\pgfqpoint{1.712660in}{1.595279in}}{\pgfqpoint{1.718484in}{1.589455in}}%
\pgfpathcurveto{\pgfqpoint{1.724308in}{1.583631in}}{\pgfqpoint{1.732208in}{1.580359in}}{\pgfqpoint{1.740444in}{1.580359in}}%
\pgfpathclose%
\pgfusepath{stroke,fill}%
\end{pgfscope}%
\begin{pgfscope}%
\pgfpathrectangle{\pgfqpoint{0.100000in}{0.212622in}}{\pgfqpoint{3.696000in}{3.696000in}}%
\pgfusepath{clip}%
\pgfsetbuttcap%
\pgfsetroundjoin%
\definecolor{currentfill}{rgb}{0.121569,0.466667,0.705882}%
\pgfsetfillcolor{currentfill}%
\pgfsetfillopacity{0.865958}%
\pgfsetlinewidth{1.003750pt}%
\definecolor{currentstroke}{rgb}{0.121569,0.466667,0.705882}%
\pgfsetstrokecolor{currentstroke}%
\pgfsetstrokeopacity{0.865958}%
\pgfsetdash{}{0pt}%
\pgfpathmoveto{\pgfqpoint{1.741504in}{1.578712in}}%
\pgfpathcurveto{\pgfqpoint{1.749740in}{1.578712in}}{\pgfqpoint{1.757640in}{1.581984in}}{\pgfqpoint{1.763464in}{1.587808in}}%
\pgfpathcurveto{\pgfqpoint{1.769288in}{1.593632in}}{\pgfqpoint{1.772560in}{1.601532in}}{\pgfqpoint{1.772560in}{1.609768in}}%
\pgfpathcurveto{\pgfqpoint{1.772560in}{1.618005in}}{\pgfqpoint{1.769288in}{1.625905in}}{\pgfqpoint{1.763464in}{1.631729in}}%
\pgfpathcurveto{\pgfqpoint{1.757640in}{1.637553in}}{\pgfqpoint{1.749740in}{1.640825in}}{\pgfqpoint{1.741504in}{1.640825in}}%
\pgfpathcurveto{\pgfqpoint{1.733267in}{1.640825in}}{\pgfqpoint{1.725367in}{1.637553in}}{\pgfqpoint{1.719543in}{1.631729in}}%
\pgfpathcurveto{\pgfqpoint{1.713720in}{1.625905in}}{\pgfqpoint{1.710447in}{1.618005in}}{\pgfqpoint{1.710447in}{1.609768in}}%
\pgfpathcurveto{\pgfqpoint{1.710447in}{1.601532in}}{\pgfqpoint{1.713720in}{1.593632in}}{\pgfqpoint{1.719543in}{1.587808in}}%
\pgfpathcurveto{\pgfqpoint{1.725367in}{1.581984in}}{\pgfqpoint{1.733267in}{1.578712in}}{\pgfqpoint{1.741504in}{1.578712in}}%
\pgfpathclose%
\pgfusepath{stroke,fill}%
\end{pgfscope}%
\begin{pgfscope}%
\pgfpathrectangle{\pgfqpoint{0.100000in}{0.212622in}}{\pgfqpoint{3.696000in}{3.696000in}}%
\pgfusepath{clip}%
\pgfsetbuttcap%
\pgfsetroundjoin%
\definecolor{currentfill}{rgb}{0.121569,0.466667,0.705882}%
\pgfsetfillcolor{currentfill}%
\pgfsetfillopacity{0.869548}%
\pgfsetlinewidth{1.003750pt}%
\definecolor{currentstroke}{rgb}{0.121569,0.466667,0.705882}%
\pgfsetstrokecolor{currentstroke}%
\pgfsetstrokeopacity{0.869548}%
\pgfsetdash{}{0pt}%
\pgfpathmoveto{\pgfqpoint{1.743675in}{1.577279in}}%
\pgfpathcurveto{\pgfqpoint{1.751912in}{1.577279in}}{\pgfqpoint{1.759812in}{1.580551in}}{\pgfqpoint{1.765636in}{1.586375in}}%
\pgfpathcurveto{\pgfqpoint{1.771460in}{1.592199in}}{\pgfqpoint{1.774732in}{1.600099in}}{\pgfqpoint{1.774732in}{1.608335in}}%
\pgfpathcurveto{\pgfqpoint{1.774732in}{1.616572in}}{\pgfqpoint{1.771460in}{1.624472in}}{\pgfqpoint{1.765636in}{1.630296in}}%
\pgfpathcurveto{\pgfqpoint{1.759812in}{1.636120in}}{\pgfqpoint{1.751912in}{1.639392in}}{\pgfqpoint{1.743675in}{1.639392in}}%
\pgfpathcurveto{\pgfqpoint{1.735439in}{1.639392in}}{\pgfqpoint{1.727539in}{1.636120in}}{\pgfqpoint{1.721715in}{1.630296in}}%
\pgfpathcurveto{\pgfqpoint{1.715891in}{1.624472in}}{\pgfqpoint{1.712619in}{1.616572in}}{\pgfqpoint{1.712619in}{1.608335in}}%
\pgfpathcurveto{\pgfqpoint{1.712619in}{1.600099in}}{\pgfqpoint{1.715891in}{1.592199in}}{\pgfqpoint{1.721715in}{1.586375in}}%
\pgfpathcurveto{\pgfqpoint{1.727539in}{1.580551in}}{\pgfqpoint{1.735439in}{1.577279in}}{\pgfqpoint{1.743675in}{1.577279in}}%
\pgfpathclose%
\pgfusepath{stroke,fill}%
\end{pgfscope}%
\begin{pgfscope}%
\pgfpathrectangle{\pgfqpoint{0.100000in}{0.212622in}}{\pgfqpoint{3.696000in}{3.696000in}}%
\pgfusepath{clip}%
\pgfsetbuttcap%
\pgfsetroundjoin%
\definecolor{currentfill}{rgb}{0.121569,0.466667,0.705882}%
\pgfsetfillcolor{currentfill}%
\pgfsetfillopacity{0.871501}%
\pgfsetlinewidth{1.003750pt}%
\definecolor{currentstroke}{rgb}{0.121569,0.466667,0.705882}%
\pgfsetstrokecolor{currentstroke}%
\pgfsetstrokeopacity{0.871501}%
\pgfsetdash{}{0pt}%
\pgfpathmoveto{\pgfqpoint{1.745755in}{1.567191in}}%
\pgfpathcurveto{\pgfqpoint{1.753991in}{1.567191in}}{\pgfqpoint{1.761891in}{1.570463in}}{\pgfqpoint{1.767715in}{1.576287in}}%
\pgfpathcurveto{\pgfqpoint{1.773539in}{1.582111in}}{\pgfqpoint{1.776811in}{1.590011in}}{\pgfqpoint{1.776811in}{1.598247in}}%
\pgfpathcurveto{\pgfqpoint{1.776811in}{1.606483in}}{\pgfqpoint{1.773539in}{1.614383in}}{\pgfqpoint{1.767715in}{1.620207in}}%
\pgfpathcurveto{\pgfqpoint{1.761891in}{1.626031in}}{\pgfqpoint{1.753991in}{1.629304in}}{\pgfqpoint{1.745755in}{1.629304in}}%
\pgfpathcurveto{\pgfqpoint{1.737518in}{1.629304in}}{\pgfqpoint{1.729618in}{1.626031in}}{\pgfqpoint{1.723794in}{1.620207in}}%
\pgfpathcurveto{\pgfqpoint{1.717971in}{1.614383in}}{\pgfqpoint{1.714698in}{1.606483in}}{\pgfqpoint{1.714698in}{1.598247in}}%
\pgfpathcurveto{\pgfqpoint{1.714698in}{1.590011in}}{\pgfqpoint{1.717971in}{1.582111in}}{\pgfqpoint{1.723794in}{1.576287in}}%
\pgfpathcurveto{\pgfqpoint{1.729618in}{1.570463in}}{\pgfqpoint{1.737518in}{1.567191in}}{\pgfqpoint{1.745755in}{1.567191in}}%
\pgfpathclose%
\pgfusepath{stroke,fill}%
\end{pgfscope}%
\begin{pgfscope}%
\pgfpathrectangle{\pgfqpoint{0.100000in}{0.212622in}}{\pgfqpoint{3.696000in}{3.696000in}}%
\pgfusepath{clip}%
\pgfsetbuttcap%
\pgfsetroundjoin%
\definecolor{currentfill}{rgb}{0.121569,0.466667,0.705882}%
\pgfsetfillcolor{currentfill}%
\pgfsetfillopacity{0.873852}%
\pgfsetlinewidth{1.003750pt}%
\definecolor{currentstroke}{rgb}{0.121569,0.466667,0.705882}%
\pgfsetstrokecolor{currentstroke}%
\pgfsetstrokeopacity{0.873852}%
\pgfsetdash{}{0pt}%
\pgfpathmoveto{\pgfqpoint{1.747098in}{1.567394in}}%
\pgfpathcurveto{\pgfqpoint{1.755334in}{1.567394in}}{\pgfqpoint{1.763235in}{1.570666in}}{\pgfqpoint{1.769058in}{1.576490in}}%
\pgfpathcurveto{\pgfqpoint{1.774882in}{1.582314in}}{\pgfqpoint{1.778155in}{1.590214in}}{\pgfqpoint{1.778155in}{1.598450in}}%
\pgfpathcurveto{\pgfqpoint{1.778155in}{1.606686in}}{\pgfqpoint{1.774882in}{1.614587in}}{\pgfqpoint{1.769058in}{1.620410in}}%
\pgfpathcurveto{\pgfqpoint{1.763235in}{1.626234in}}{\pgfqpoint{1.755334in}{1.629507in}}{\pgfqpoint{1.747098in}{1.629507in}}%
\pgfpathcurveto{\pgfqpoint{1.738862in}{1.629507in}}{\pgfqpoint{1.730962in}{1.626234in}}{\pgfqpoint{1.725138in}{1.620410in}}%
\pgfpathcurveto{\pgfqpoint{1.719314in}{1.614587in}}{\pgfqpoint{1.716042in}{1.606686in}}{\pgfqpoint{1.716042in}{1.598450in}}%
\pgfpathcurveto{\pgfqpoint{1.716042in}{1.590214in}}{\pgfqpoint{1.719314in}{1.582314in}}{\pgfqpoint{1.725138in}{1.576490in}}%
\pgfpathcurveto{\pgfqpoint{1.730962in}{1.570666in}}{\pgfqpoint{1.738862in}{1.567394in}}{\pgfqpoint{1.747098in}{1.567394in}}%
\pgfpathclose%
\pgfusepath{stroke,fill}%
\end{pgfscope}%
\begin{pgfscope}%
\pgfpathrectangle{\pgfqpoint{0.100000in}{0.212622in}}{\pgfqpoint{3.696000in}{3.696000in}}%
\pgfusepath{clip}%
\pgfsetbuttcap%
\pgfsetroundjoin%
\definecolor{currentfill}{rgb}{0.121569,0.466667,0.705882}%
\pgfsetfillcolor{currentfill}%
\pgfsetfillopacity{0.876613}%
\pgfsetlinewidth{1.003750pt}%
\definecolor{currentstroke}{rgb}{0.121569,0.466667,0.705882}%
\pgfsetstrokecolor{currentstroke}%
\pgfsetstrokeopacity{0.876613}%
\pgfsetdash{}{0pt}%
\pgfpathmoveto{\pgfqpoint{1.748908in}{1.568278in}}%
\pgfpathcurveto{\pgfqpoint{1.757145in}{1.568278in}}{\pgfqpoint{1.765045in}{1.571551in}}{\pgfqpoint{1.770869in}{1.577375in}}%
\pgfpathcurveto{\pgfqpoint{1.776693in}{1.583199in}}{\pgfqpoint{1.779965in}{1.591099in}}{\pgfqpoint{1.779965in}{1.599335in}}%
\pgfpathcurveto{\pgfqpoint{1.779965in}{1.607571in}}{\pgfqpoint{1.776693in}{1.615471in}}{\pgfqpoint{1.770869in}{1.621295in}}%
\pgfpathcurveto{\pgfqpoint{1.765045in}{1.627119in}}{\pgfqpoint{1.757145in}{1.630391in}}{\pgfqpoint{1.748908in}{1.630391in}}%
\pgfpathcurveto{\pgfqpoint{1.740672in}{1.630391in}}{\pgfqpoint{1.732772in}{1.627119in}}{\pgfqpoint{1.726948in}{1.621295in}}%
\pgfpathcurveto{\pgfqpoint{1.721124in}{1.615471in}}{\pgfqpoint{1.717852in}{1.607571in}}{\pgfqpoint{1.717852in}{1.599335in}}%
\pgfpathcurveto{\pgfqpoint{1.717852in}{1.591099in}}{\pgfqpoint{1.721124in}{1.583199in}}{\pgfqpoint{1.726948in}{1.577375in}}%
\pgfpathcurveto{\pgfqpoint{1.732772in}{1.571551in}}{\pgfqpoint{1.740672in}{1.568278in}}{\pgfqpoint{1.748908in}{1.568278in}}%
\pgfpathclose%
\pgfusepath{stroke,fill}%
\end{pgfscope}%
\begin{pgfscope}%
\pgfpathrectangle{\pgfqpoint{0.100000in}{0.212622in}}{\pgfqpoint{3.696000in}{3.696000in}}%
\pgfusepath{clip}%
\pgfsetbuttcap%
\pgfsetroundjoin%
\definecolor{currentfill}{rgb}{0.121569,0.466667,0.705882}%
\pgfsetfillcolor{currentfill}%
\pgfsetfillopacity{0.878982}%
\pgfsetlinewidth{1.003750pt}%
\definecolor{currentstroke}{rgb}{0.121569,0.466667,0.705882}%
\pgfsetstrokecolor{currentstroke}%
\pgfsetstrokeopacity{0.878982}%
\pgfsetdash{}{0pt}%
\pgfpathmoveto{\pgfqpoint{1.749872in}{1.566233in}}%
\pgfpathcurveto{\pgfqpoint{1.758109in}{1.566233in}}{\pgfqpoint{1.766009in}{1.569506in}}{\pgfqpoint{1.771833in}{1.575329in}}%
\pgfpathcurveto{\pgfqpoint{1.777657in}{1.581153in}}{\pgfqpoint{1.780929in}{1.589053in}}{\pgfqpoint{1.780929in}{1.597290in}}%
\pgfpathcurveto{\pgfqpoint{1.780929in}{1.605526in}}{\pgfqpoint{1.777657in}{1.613426in}}{\pgfqpoint{1.771833in}{1.619250in}}%
\pgfpathcurveto{\pgfqpoint{1.766009in}{1.625074in}}{\pgfqpoint{1.758109in}{1.628346in}}{\pgfqpoint{1.749872in}{1.628346in}}%
\pgfpathcurveto{\pgfqpoint{1.741636in}{1.628346in}}{\pgfqpoint{1.733736in}{1.625074in}}{\pgfqpoint{1.727912in}{1.619250in}}%
\pgfpathcurveto{\pgfqpoint{1.722088in}{1.613426in}}{\pgfqpoint{1.718816in}{1.605526in}}{\pgfqpoint{1.718816in}{1.597290in}}%
\pgfpathcurveto{\pgfqpoint{1.718816in}{1.589053in}}{\pgfqpoint{1.722088in}{1.581153in}}{\pgfqpoint{1.727912in}{1.575329in}}%
\pgfpathcurveto{\pgfqpoint{1.733736in}{1.569506in}}{\pgfqpoint{1.741636in}{1.566233in}}{\pgfqpoint{1.749872in}{1.566233in}}%
\pgfpathclose%
\pgfusepath{stroke,fill}%
\end{pgfscope}%
\begin{pgfscope}%
\pgfpathrectangle{\pgfqpoint{0.100000in}{0.212622in}}{\pgfqpoint{3.696000in}{3.696000in}}%
\pgfusepath{clip}%
\pgfsetbuttcap%
\pgfsetroundjoin%
\definecolor{currentfill}{rgb}{0.121569,0.466667,0.705882}%
\pgfsetfillcolor{currentfill}%
\pgfsetfillopacity{0.880321}%
\pgfsetlinewidth{1.003750pt}%
\definecolor{currentstroke}{rgb}{0.121569,0.466667,0.705882}%
\pgfsetstrokecolor{currentstroke}%
\pgfsetstrokeopacity{0.880321}%
\pgfsetdash{}{0pt}%
\pgfpathmoveto{\pgfqpoint{0.604691in}{2.577311in}}%
\pgfpathcurveto{\pgfqpoint{0.612928in}{2.577311in}}{\pgfqpoint{0.620828in}{2.580583in}}{\pgfqpoint{0.626652in}{2.586407in}}%
\pgfpathcurveto{\pgfqpoint{0.632476in}{2.592231in}}{\pgfqpoint{0.635748in}{2.600131in}}{\pgfqpoint{0.635748in}{2.608367in}}%
\pgfpathcurveto{\pgfqpoint{0.635748in}{2.616604in}}{\pgfqpoint{0.632476in}{2.624504in}}{\pgfqpoint{0.626652in}{2.630328in}}%
\pgfpathcurveto{\pgfqpoint{0.620828in}{2.636152in}}{\pgfqpoint{0.612928in}{2.639424in}}{\pgfqpoint{0.604691in}{2.639424in}}%
\pgfpathcurveto{\pgfqpoint{0.596455in}{2.639424in}}{\pgfqpoint{0.588555in}{2.636152in}}{\pgfqpoint{0.582731in}{2.630328in}}%
\pgfpathcurveto{\pgfqpoint{0.576907in}{2.624504in}}{\pgfqpoint{0.573635in}{2.616604in}}{\pgfqpoint{0.573635in}{2.608367in}}%
\pgfpathcurveto{\pgfqpoint{0.573635in}{2.600131in}}{\pgfqpoint{0.576907in}{2.592231in}}{\pgfqpoint{0.582731in}{2.586407in}}%
\pgfpathcurveto{\pgfqpoint{0.588555in}{2.580583in}}{\pgfqpoint{0.596455in}{2.577311in}}{\pgfqpoint{0.604691in}{2.577311in}}%
\pgfpathclose%
\pgfusepath{stroke,fill}%
\end{pgfscope}%
\begin{pgfscope}%
\pgfpathrectangle{\pgfqpoint{0.100000in}{0.212622in}}{\pgfqpoint{3.696000in}{3.696000in}}%
\pgfusepath{clip}%
\pgfsetbuttcap%
\pgfsetroundjoin%
\definecolor{currentfill}{rgb}{0.121569,0.466667,0.705882}%
\pgfsetfillcolor{currentfill}%
\pgfsetfillopacity{0.880471}%
\pgfsetlinewidth{1.003750pt}%
\definecolor{currentstroke}{rgb}{0.121569,0.466667,0.705882}%
\pgfsetstrokecolor{currentstroke}%
\pgfsetstrokeopacity{0.880471}%
\pgfsetdash{}{0pt}%
\pgfpathmoveto{\pgfqpoint{1.750573in}{1.565996in}}%
\pgfpathcurveto{\pgfqpoint{1.758809in}{1.565996in}}{\pgfqpoint{1.766709in}{1.569269in}}{\pgfqpoint{1.772533in}{1.575093in}}%
\pgfpathcurveto{\pgfqpoint{1.778357in}{1.580916in}}{\pgfqpoint{1.781629in}{1.588817in}}{\pgfqpoint{1.781629in}{1.597053in}}%
\pgfpathcurveto{\pgfqpoint{1.781629in}{1.605289in}}{\pgfqpoint{1.778357in}{1.613189in}}{\pgfqpoint{1.772533in}{1.619013in}}%
\pgfpathcurveto{\pgfqpoint{1.766709in}{1.624837in}}{\pgfqpoint{1.758809in}{1.628109in}}{\pgfqpoint{1.750573in}{1.628109in}}%
\pgfpathcurveto{\pgfqpoint{1.742336in}{1.628109in}}{\pgfqpoint{1.734436in}{1.624837in}}{\pgfqpoint{1.728612in}{1.619013in}}%
\pgfpathcurveto{\pgfqpoint{1.722789in}{1.613189in}}{\pgfqpoint{1.719516in}{1.605289in}}{\pgfqpoint{1.719516in}{1.597053in}}%
\pgfpathcurveto{\pgfqpoint{1.719516in}{1.588817in}}{\pgfqpoint{1.722789in}{1.580916in}}{\pgfqpoint{1.728612in}{1.575093in}}%
\pgfpathcurveto{\pgfqpoint{1.734436in}{1.569269in}}{\pgfqpoint{1.742336in}{1.565996in}}{\pgfqpoint{1.750573in}{1.565996in}}%
\pgfpathclose%
\pgfusepath{stroke,fill}%
\end{pgfscope}%
\begin{pgfscope}%
\pgfpathrectangle{\pgfqpoint{0.100000in}{0.212622in}}{\pgfqpoint{3.696000in}{3.696000in}}%
\pgfusepath{clip}%
\pgfsetbuttcap%
\pgfsetroundjoin%
\definecolor{currentfill}{rgb}{0.121569,0.466667,0.705882}%
\pgfsetfillcolor{currentfill}%
\pgfsetfillopacity{0.880923}%
\pgfsetlinewidth{1.003750pt}%
\definecolor{currentstroke}{rgb}{0.121569,0.466667,0.705882}%
\pgfsetstrokecolor{currentstroke}%
\pgfsetstrokeopacity{0.880923}%
\pgfsetdash{}{0pt}%
\pgfpathmoveto{\pgfqpoint{1.751058in}{1.564290in}}%
\pgfpathcurveto{\pgfqpoint{1.759294in}{1.564290in}}{\pgfqpoint{1.767194in}{1.567562in}}{\pgfqpoint{1.773018in}{1.573386in}}%
\pgfpathcurveto{\pgfqpoint{1.778842in}{1.579210in}}{\pgfqpoint{1.782114in}{1.587110in}}{\pgfqpoint{1.782114in}{1.595346in}}%
\pgfpathcurveto{\pgfqpoint{1.782114in}{1.603583in}}{\pgfqpoint{1.778842in}{1.611483in}}{\pgfqpoint{1.773018in}{1.617307in}}%
\pgfpathcurveto{\pgfqpoint{1.767194in}{1.623131in}}{\pgfqpoint{1.759294in}{1.626403in}}{\pgfqpoint{1.751058in}{1.626403in}}%
\pgfpathcurveto{\pgfqpoint{1.742821in}{1.626403in}}{\pgfqpoint{1.734921in}{1.623131in}}{\pgfqpoint{1.729097in}{1.617307in}}%
\pgfpathcurveto{\pgfqpoint{1.723273in}{1.611483in}}{\pgfqpoint{1.720001in}{1.603583in}}{\pgfqpoint{1.720001in}{1.595346in}}%
\pgfpathcurveto{\pgfqpoint{1.720001in}{1.587110in}}{\pgfqpoint{1.723273in}{1.579210in}}{\pgfqpoint{1.729097in}{1.573386in}}%
\pgfpathcurveto{\pgfqpoint{1.734921in}{1.567562in}}{\pgfqpoint{1.742821in}{1.564290in}}{\pgfqpoint{1.751058in}{1.564290in}}%
\pgfpathclose%
\pgfusepath{stroke,fill}%
\end{pgfscope}%
\begin{pgfscope}%
\pgfpathrectangle{\pgfqpoint{0.100000in}{0.212622in}}{\pgfqpoint{3.696000in}{3.696000in}}%
\pgfusepath{clip}%
\pgfsetbuttcap%
\pgfsetroundjoin%
\definecolor{currentfill}{rgb}{0.121569,0.466667,0.705882}%
\pgfsetfillcolor{currentfill}%
\pgfsetfillopacity{0.881397}%
\pgfsetlinewidth{1.003750pt}%
\definecolor{currentstroke}{rgb}{0.121569,0.466667,0.705882}%
\pgfsetstrokecolor{currentstroke}%
\pgfsetstrokeopacity{0.881397}%
\pgfsetdash{}{0pt}%
\pgfpathmoveto{\pgfqpoint{1.751302in}{1.564338in}}%
\pgfpathcurveto{\pgfqpoint{1.759539in}{1.564338in}}{\pgfqpoint{1.767439in}{1.567610in}}{\pgfqpoint{1.773263in}{1.573434in}}%
\pgfpathcurveto{\pgfqpoint{1.779087in}{1.579258in}}{\pgfqpoint{1.782359in}{1.587158in}}{\pgfqpoint{1.782359in}{1.595395in}}%
\pgfpathcurveto{\pgfqpoint{1.782359in}{1.603631in}}{\pgfqpoint{1.779087in}{1.611531in}}{\pgfqpoint{1.773263in}{1.617355in}}%
\pgfpathcurveto{\pgfqpoint{1.767439in}{1.623179in}}{\pgfqpoint{1.759539in}{1.626451in}}{\pgfqpoint{1.751302in}{1.626451in}}%
\pgfpathcurveto{\pgfqpoint{1.743066in}{1.626451in}}{\pgfqpoint{1.735166in}{1.623179in}}{\pgfqpoint{1.729342in}{1.617355in}}%
\pgfpathcurveto{\pgfqpoint{1.723518in}{1.611531in}}{\pgfqpoint{1.720246in}{1.603631in}}{\pgfqpoint{1.720246in}{1.595395in}}%
\pgfpathcurveto{\pgfqpoint{1.720246in}{1.587158in}}{\pgfqpoint{1.723518in}{1.579258in}}{\pgfqpoint{1.729342in}{1.573434in}}%
\pgfpathcurveto{\pgfqpoint{1.735166in}{1.567610in}}{\pgfqpoint{1.743066in}{1.564338in}}{\pgfqpoint{1.751302in}{1.564338in}}%
\pgfpathclose%
\pgfusepath{stroke,fill}%
\end{pgfscope}%
\begin{pgfscope}%
\pgfpathrectangle{\pgfqpoint{0.100000in}{0.212622in}}{\pgfqpoint{3.696000in}{3.696000in}}%
\pgfusepath{clip}%
\pgfsetbuttcap%
\pgfsetroundjoin%
\definecolor{currentfill}{rgb}{0.121569,0.466667,0.705882}%
\pgfsetfillcolor{currentfill}%
\pgfsetfillopacity{0.882432}%
\pgfsetlinewidth{1.003750pt}%
\definecolor{currentstroke}{rgb}{0.121569,0.466667,0.705882}%
\pgfsetstrokecolor{currentstroke}%
\pgfsetstrokeopacity{0.882432}%
\pgfsetdash{}{0pt}%
\pgfpathmoveto{\pgfqpoint{0.609115in}{2.576869in}}%
\pgfpathcurveto{\pgfqpoint{0.617351in}{2.576869in}}{\pgfqpoint{0.625251in}{2.580142in}}{\pgfqpoint{0.631075in}{2.585966in}}%
\pgfpathcurveto{\pgfqpoint{0.636899in}{2.591789in}}{\pgfqpoint{0.640171in}{2.599690in}}{\pgfqpoint{0.640171in}{2.607926in}}%
\pgfpathcurveto{\pgfqpoint{0.640171in}{2.616162in}}{\pgfqpoint{0.636899in}{2.624062in}}{\pgfqpoint{0.631075in}{2.629886in}}%
\pgfpathcurveto{\pgfqpoint{0.625251in}{2.635710in}}{\pgfqpoint{0.617351in}{2.638982in}}{\pgfqpoint{0.609115in}{2.638982in}}%
\pgfpathcurveto{\pgfqpoint{0.600879in}{2.638982in}}{\pgfqpoint{0.592978in}{2.635710in}}{\pgfqpoint{0.587155in}{2.629886in}}%
\pgfpathcurveto{\pgfqpoint{0.581331in}{2.624062in}}{\pgfqpoint{0.578058in}{2.616162in}}{\pgfqpoint{0.578058in}{2.607926in}}%
\pgfpathcurveto{\pgfqpoint{0.578058in}{2.599690in}}{\pgfqpoint{0.581331in}{2.591789in}}{\pgfqpoint{0.587155in}{2.585966in}}%
\pgfpathcurveto{\pgfqpoint{0.592978in}{2.580142in}}{\pgfqpoint{0.600879in}{2.576869in}}{\pgfqpoint{0.609115in}{2.576869in}}%
\pgfpathclose%
\pgfusepath{stroke,fill}%
\end{pgfscope}%
\begin{pgfscope}%
\pgfpathrectangle{\pgfqpoint{0.100000in}{0.212622in}}{\pgfqpoint{3.696000in}{3.696000in}}%
\pgfusepath{clip}%
\pgfsetbuttcap%
\pgfsetroundjoin%
\definecolor{currentfill}{rgb}{0.121569,0.466667,0.705882}%
\pgfsetfillcolor{currentfill}%
\pgfsetfillopacity{0.882447}%
\pgfsetlinewidth{1.003750pt}%
\definecolor{currentstroke}{rgb}{0.121569,0.466667,0.705882}%
\pgfsetstrokecolor{currentstroke}%
\pgfsetstrokeopacity{0.882447}%
\pgfsetdash{}{0pt}%
\pgfpathmoveto{\pgfqpoint{1.751808in}{1.564568in}}%
\pgfpathcurveto{\pgfqpoint{1.760044in}{1.564568in}}{\pgfqpoint{1.767944in}{1.567840in}}{\pgfqpoint{1.773768in}{1.573664in}}%
\pgfpathcurveto{\pgfqpoint{1.779592in}{1.579488in}}{\pgfqpoint{1.782864in}{1.587388in}}{\pgfqpoint{1.782864in}{1.595624in}}%
\pgfpathcurveto{\pgfqpoint{1.782864in}{1.603861in}}{\pgfqpoint{1.779592in}{1.611761in}}{\pgfqpoint{1.773768in}{1.617585in}}%
\pgfpathcurveto{\pgfqpoint{1.767944in}{1.623409in}}{\pgfqpoint{1.760044in}{1.626681in}}{\pgfqpoint{1.751808in}{1.626681in}}%
\pgfpathcurveto{\pgfqpoint{1.743572in}{1.626681in}}{\pgfqpoint{1.735671in}{1.623409in}}{\pgfqpoint{1.729848in}{1.617585in}}%
\pgfpathcurveto{\pgfqpoint{1.724024in}{1.611761in}}{\pgfqpoint{1.720751in}{1.603861in}}{\pgfqpoint{1.720751in}{1.595624in}}%
\pgfpathcurveto{\pgfqpoint{1.720751in}{1.587388in}}{\pgfqpoint{1.724024in}{1.579488in}}{\pgfqpoint{1.729848in}{1.573664in}}%
\pgfpathcurveto{\pgfqpoint{1.735671in}{1.567840in}}{\pgfqpoint{1.743572in}{1.564568in}}{\pgfqpoint{1.751808in}{1.564568in}}%
\pgfpathclose%
\pgfusepath{stroke,fill}%
\end{pgfscope}%
\begin{pgfscope}%
\pgfpathrectangle{\pgfqpoint{0.100000in}{0.212622in}}{\pgfqpoint{3.696000in}{3.696000in}}%
\pgfusepath{clip}%
\pgfsetbuttcap%
\pgfsetroundjoin%
\definecolor{currentfill}{rgb}{0.121569,0.466667,0.705882}%
\pgfsetfillcolor{currentfill}%
\pgfsetfillopacity{0.882711}%
\pgfsetlinewidth{1.003750pt}%
\definecolor{currentstroke}{rgb}{0.121569,0.466667,0.705882}%
\pgfsetstrokecolor{currentstroke}%
\pgfsetstrokeopacity{0.882711}%
\pgfsetdash{}{0pt}%
\pgfpathmoveto{\pgfqpoint{0.597554in}{2.595391in}}%
\pgfpathcurveto{\pgfqpoint{0.605791in}{2.595391in}}{\pgfqpoint{0.613691in}{2.598663in}}{\pgfqpoint{0.619515in}{2.604487in}}%
\pgfpathcurveto{\pgfqpoint{0.625339in}{2.610311in}}{\pgfqpoint{0.628611in}{2.618211in}}{\pgfqpoint{0.628611in}{2.626447in}}%
\pgfpathcurveto{\pgfqpoint{0.628611in}{2.634683in}}{\pgfqpoint{0.625339in}{2.642583in}}{\pgfqpoint{0.619515in}{2.648407in}}%
\pgfpathcurveto{\pgfqpoint{0.613691in}{2.654231in}}{\pgfqpoint{0.605791in}{2.657504in}}{\pgfqpoint{0.597554in}{2.657504in}}%
\pgfpathcurveto{\pgfqpoint{0.589318in}{2.657504in}}{\pgfqpoint{0.581418in}{2.654231in}}{\pgfqpoint{0.575594in}{2.648407in}}%
\pgfpathcurveto{\pgfqpoint{0.569770in}{2.642583in}}{\pgfqpoint{0.566498in}{2.634683in}}{\pgfqpoint{0.566498in}{2.626447in}}%
\pgfpathcurveto{\pgfqpoint{0.566498in}{2.618211in}}{\pgfqpoint{0.569770in}{2.610311in}}{\pgfqpoint{0.575594in}{2.604487in}}%
\pgfpathcurveto{\pgfqpoint{0.581418in}{2.598663in}}{\pgfqpoint{0.589318in}{2.595391in}}{\pgfqpoint{0.597554in}{2.595391in}}%
\pgfpathclose%
\pgfusepath{stroke,fill}%
\end{pgfscope}%
\begin{pgfscope}%
\pgfpathrectangle{\pgfqpoint{0.100000in}{0.212622in}}{\pgfqpoint{3.696000in}{3.696000in}}%
\pgfusepath{clip}%
\pgfsetbuttcap%
\pgfsetroundjoin%
\definecolor{currentfill}{rgb}{0.121569,0.466667,0.705882}%
\pgfsetfillcolor{currentfill}%
\pgfsetfillopacity{0.882906}%
\pgfsetlinewidth{1.003750pt}%
\definecolor{currentstroke}{rgb}{0.121569,0.466667,0.705882}%
\pgfsetstrokecolor{currentstroke}%
\pgfsetstrokeopacity{0.882906}%
\pgfsetdash{}{0pt}%
\pgfpathmoveto{\pgfqpoint{0.613555in}{2.570618in}}%
\pgfpathcurveto{\pgfqpoint{0.621792in}{2.570618in}}{\pgfqpoint{0.629692in}{2.573891in}}{\pgfqpoint{0.635516in}{2.579715in}}%
\pgfpathcurveto{\pgfqpoint{0.641340in}{2.585539in}}{\pgfqpoint{0.644612in}{2.593439in}}{\pgfqpoint{0.644612in}{2.601675in}}%
\pgfpathcurveto{\pgfqpoint{0.644612in}{2.609911in}}{\pgfqpoint{0.641340in}{2.617811in}}{\pgfqpoint{0.635516in}{2.623635in}}%
\pgfpathcurveto{\pgfqpoint{0.629692in}{2.629459in}}{\pgfqpoint{0.621792in}{2.632731in}}{\pgfqpoint{0.613555in}{2.632731in}}%
\pgfpathcurveto{\pgfqpoint{0.605319in}{2.632731in}}{\pgfqpoint{0.597419in}{2.629459in}}{\pgfqpoint{0.591595in}{2.623635in}}%
\pgfpathcurveto{\pgfqpoint{0.585771in}{2.617811in}}{\pgfqpoint{0.582499in}{2.609911in}}{\pgfqpoint{0.582499in}{2.601675in}}%
\pgfpathcurveto{\pgfqpoint{0.582499in}{2.593439in}}{\pgfqpoint{0.585771in}{2.585539in}}{\pgfqpoint{0.591595in}{2.579715in}}%
\pgfpathcurveto{\pgfqpoint{0.597419in}{2.573891in}}{\pgfqpoint{0.605319in}{2.570618in}}{\pgfqpoint{0.613555in}{2.570618in}}%
\pgfpathclose%
\pgfusepath{stroke,fill}%
\end{pgfscope}%
\begin{pgfscope}%
\pgfpathrectangle{\pgfqpoint{0.100000in}{0.212622in}}{\pgfqpoint{3.696000in}{3.696000in}}%
\pgfusepath{clip}%
\pgfsetbuttcap%
\pgfsetroundjoin%
\definecolor{currentfill}{rgb}{0.121569,0.466667,0.705882}%
\pgfsetfillcolor{currentfill}%
\pgfsetfillopacity{0.882965}%
\pgfsetlinewidth{1.003750pt}%
\definecolor{currentstroke}{rgb}{0.121569,0.466667,0.705882}%
\pgfsetstrokecolor{currentstroke}%
\pgfsetstrokeopacity{0.882965}%
\pgfsetdash{}{0pt}%
\pgfpathmoveto{\pgfqpoint{0.589719in}{2.601020in}}%
\pgfpathcurveto{\pgfqpoint{0.597955in}{2.601020in}}{\pgfqpoint{0.605855in}{2.604293in}}{\pgfqpoint{0.611679in}{2.610117in}}%
\pgfpathcurveto{\pgfqpoint{0.617503in}{2.615940in}}{\pgfqpoint{0.620776in}{2.623841in}}{\pgfqpoint{0.620776in}{2.632077in}}%
\pgfpathcurveto{\pgfqpoint{0.620776in}{2.640313in}}{\pgfqpoint{0.617503in}{2.648213in}}{\pgfqpoint{0.611679in}{2.654037in}}%
\pgfpathcurveto{\pgfqpoint{0.605855in}{2.659861in}}{\pgfqpoint{0.597955in}{2.663133in}}{\pgfqpoint{0.589719in}{2.663133in}}%
\pgfpathcurveto{\pgfqpoint{0.581483in}{2.663133in}}{\pgfqpoint{0.573583in}{2.659861in}}{\pgfqpoint{0.567759in}{2.654037in}}%
\pgfpathcurveto{\pgfqpoint{0.561935in}{2.648213in}}{\pgfqpoint{0.558663in}{2.640313in}}{\pgfqpoint{0.558663in}{2.632077in}}%
\pgfpathcurveto{\pgfqpoint{0.558663in}{2.623841in}}{\pgfqpoint{0.561935in}{2.615940in}}{\pgfqpoint{0.567759in}{2.610117in}}%
\pgfpathcurveto{\pgfqpoint{0.573583in}{2.604293in}}{\pgfqpoint{0.581483in}{2.601020in}}{\pgfqpoint{0.589719in}{2.601020in}}%
\pgfpathclose%
\pgfusepath{stroke,fill}%
\end{pgfscope}%
\begin{pgfscope}%
\pgfpathrectangle{\pgfqpoint{0.100000in}{0.212622in}}{\pgfqpoint{3.696000in}{3.696000in}}%
\pgfusepath{clip}%
\pgfsetbuttcap%
\pgfsetroundjoin%
\definecolor{currentfill}{rgb}{0.121569,0.466667,0.705882}%
\pgfsetfillcolor{currentfill}%
\pgfsetfillopacity{0.883674}%
\pgfsetlinewidth{1.003750pt}%
\definecolor{currentstroke}{rgb}{0.121569,0.466667,0.705882}%
\pgfsetstrokecolor{currentstroke}%
\pgfsetstrokeopacity{0.883674}%
\pgfsetdash{}{0pt}%
\pgfpathmoveto{\pgfqpoint{1.752375in}{1.562999in}}%
\pgfpathcurveto{\pgfqpoint{1.760611in}{1.562999in}}{\pgfqpoint{1.768511in}{1.566271in}}{\pgfqpoint{1.774335in}{1.572095in}}%
\pgfpathcurveto{\pgfqpoint{1.780159in}{1.577919in}}{\pgfqpoint{1.783431in}{1.585819in}}{\pgfqpoint{1.783431in}{1.594055in}}%
\pgfpathcurveto{\pgfqpoint{1.783431in}{1.602291in}}{\pgfqpoint{1.780159in}{1.610192in}}{\pgfqpoint{1.774335in}{1.616015in}}%
\pgfpathcurveto{\pgfqpoint{1.768511in}{1.621839in}}{\pgfqpoint{1.760611in}{1.625112in}}{\pgfqpoint{1.752375in}{1.625112in}}%
\pgfpathcurveto{\pgfqpoint{1.744138in}{1.625112in}}{\pgfqpoint{1.736238in}{1.621839in}}{\pgfqpoint{1.730414in}{1.616015in}}%
\pgfpathcurveto{\pgfqpoint{1.724590in}{1.610192in}}{\pgfqpoint{1.721318in}{1.602291in}}{\pgfqpoint{1.721318in}{1.594055in}}%
\pgfpathcurveto{\pgfqpoint{1.721318in}{1.585819in}}{\pgfqpoint{1.724590in}{1.577919in}}{\pgfqpoint{1.730414in}{1.572095in}}%
\pgfpathcurveto{\pgfqpoint{1.736238in}{1.566271in}}{\pgfqpoint{1.744138in}{1.562999in}}{\pgfqpoint{1.752375in}{1.562999in}}%
\pgfpathclose%
\pgfusepath{stroke,fill}%
\end{pgfscope}%
\begin{pgfscope}%
\pgfpathrectangle{\pgfqpoint{0.100000in}{0.212622in}}{\pgfqpoint{3.696000in}{3.696000in}}%
\pgfusepath{clip}%
\pgfsetbuttcap%
\pgfsetroundjoin%
\definecolor{currentfill}{rgb}{0.121569,0.466667,0.705882}%
\pgfsetfillcolor{currentfill}%
\pgfsetfillopacity{0.885324}%
\pgfsetlinewidth{1.003750pt}%
\definecolor{currentstroke}{rgb}{0.121569,0.466667,0.705882}%
\pgfsetstrokecolor{currentstroke}%
\pgfsetstrokeopacity{0.885324}%
\pgfsetdash{}{0pt}%
\pgfpathmoveto{\pgfqpoint{0.621618in}{2.566898in}}%
\pgfpathcurveto{\pgfqpoint{0.629854in}{2.566898in}}{\pgfqpoint{0.637754in}{2.570170in}}{\pgfqpoint{0.643578in}{2.575994in}}%
\pgfpathcurveto{\pgfqpoint{0.649402in}{2.581818in}}{\pgfqpoint{0.652675in}{2.589718in}}{\pgfqpoint{0.652675in}{2.597954in}}%
\pgfpathcurveto{\pgfqpoint{0.652675in}{2.606191in}}{\pgfqpoint{0.649402in}{2.614091in}}{\pgfqpoint{0.643578in}{2.619915in}}%
\pgfpathcurveto{\pgfqpoint{0.637754in}{2.625739in}}{\pgfqpoint{0.629854in}{2.629011in}}{\pgfqpoint{0.621618in}{2.629011in}}%
\pgfpathcurveto{\pgfqpoint{0.613382in}{2.629011in}}{\pgfqpoint{0.605482in}{2.625739in}}{\pgfqpoint{0.599658in}{2.619915in}}%
\pgfpathcurveto{\pgfqpoint{0.593834in}{2.614091in}}{\pgfqpoint{0.590562in}{2.606191in}}{\pgfqpoint{0.590562in}{2.597954in}}%
\pgfpathcurveto{\pgfqpoint{0.590562in}{2.589718in}}{\pgfqpoint{0.593834in}{2.581818in}}{\pgfqpoint{0.599658in}{2.575994in}}%
\pgfpathcurveto{\pgfqpoint{0.605482in}{2.570170in}}{\pgfqpoint{0.613382in}{2.566898in}}{\pgfqpoint{0.621618in}{2.566898in}}%
\pgfpathclose%
\pgfusepath{stroke,fill}%
\end{pgfscope}%
\begin{pgfscope}%
\pgfpathrectangle{\pgfqpoint{0.100000in}{0.212622in}}{\pgfqpoint{3.696000in}{3.696000in}}%
\pgfusepath{clip}%
\pgfsetbuttcap%
\pgfsetroundjoin%
\definecolor{currentfill}{rgb}{0.121569,0.466667,0.705882}%
\pgfsetfillcolor{currentfill}%
\pgfsetfillopacity{0.885491}%
\pgfsetlinewidth{1.003750pt}%
\definecolor{currentstroke}{rgb}{0.121569,0.466667,0.705882}%
\pgfsetstrokecolor{currentstroke}%
\pgfsetstrokeopacity{0.885491}%
\pgfsetdash{}{0pt}%
\pgfpathmoveto{\pgfqpoint{0.630329in}{2.554887in}}%
\pgfpathcurveto{\pgfqpoint{0.638565in}{2.554887in}}{\pgfqpoint{0.646465in}{2.558159in}}{\pgfqpoint{0.652289in}{2.563983in}}%
\pgfpathcurveto{\pgfqpoint{0.658113in}{2.569807in}}{\pgfqpoint{0.661385in}{2.577707in}}{\pgfqpoint{0.661385in}{2.585943in}}%
\pgfpathcurveto{\pgfqpoint{0.661385in}{2.594179in}}{\pgfqpoint{0.658113in}{2.602080in}}{\pgfqpoint{0.652289in}{2.607903in}}%
\pgfpathcurveto{\pgfqpoint{0.646465in}{2.613727in}}{\pgfqpoint{0.638565in}{2.617000in}}{\pgfqpoint{0.630329in}{2.617000in}}%
\pgfpathcurveto{\pgfqpoint{0.622092in}{2.617000in}}{\pgfqpoint{0.614192in}{2.613727in}}{\pgfqpoint{0.608368in}{2.607903in}}%
\pgfpathcurveto{\pgfqpoint{0.602544in}{2.602080in}}{\pgfqpoint{0.599272in}{2.594179in}}{\pgfqpoint{0.599272in}{2.585943in}}%
\pgfpathcurveto{\pgfqpoint{0.599272in}{2.577707in}}{\pgfqpoint{0.602544in}{2.569807in}}{\pgfqpoint{0.608368in}{2.563983in}}%
\pgfpathcurveto{\pgfqpoint{0.614192in}{2.558159in}}{\pgfqpoint{0.622092in}{2.554887in}}{\pgfqpoint{0.630329in}{2.554887in}}%
\pgfpathclose%
\pgfusepath{stroke,fill}%
\end{pgfscope}%
\begin{pgfscope}%
\pgfpathrectangle{\pgfqpoint{0.100000in}{0.212622in}}{\pgfqpoint{3.696000in}{3.696000in}}%
\pgfusepath{clip}%
\pgfsetbuttcap%
\pgfsetroundjoin%
\definecolor{currentfill}{rgb}{0.121569,0.466667,0.705882}%
\pgfsetfillcolor{currentfill}%
\pgfsetfillopacity{0.885608}%
\pgfsetlinewidth{1.003750pt}%
\definecolor{currentstroke}{rgb}{0.121569,0.466667,0.705882}%
\pgfsetstrokecolor{currentstroke}%
\pgfsetstrokeopacity{0.885608}%
\pgfsetdash{}{0pt}%
\pgfpathmoveto{\pgfqpoint{1.753257in}{1.560963in}}%
\pgfpathcurveto{\pgfqpoint{1.761493in}{1.560963in}}{\pgfqpoint{1.769393in}{1.564236in}}{\pgfqpoint{1.775217in}{1.570060in}}%
\pgfpathcurveto{\pgfqpoint{1.781041in}{1.575884in}}{\pgfqpoint{1.784313in}{1.583784in}}{\pgfqpoint{1.784313in}{1.592020in}}%
\pgfpathcurveto{\pgfqpoint{1.784313in}{1.600256in}}{\pgfqpoint{1.781041in}{1.608156in}}{\pgfqpoint{1.775217in}{1.613980in}}%
\pgfpathcurveto{\pgfqpoint{1.769393in}{1.619804in}}{\pgfqpoint{1.761493in}{1.623076in}}{\pgfqpoint{1.753257in}{1.623076in}}%
\pgfpathcurveto{\pgfqpoint{1.745021in}{1.623076in}}{\pgfqpoint{1.737120in}{1.619804in}}{\pgfqpoint{1.731297in}{1.613980in}}%
\pgfpathcurveto{\pgfqpoint{1.725473in}{1.608156in}}{\pgfqpoint{1.722200in}{1.600256in}}{\pgfqpoint{1.722200in}{1.592020in}}%
\pgfpathcurveto{\pgfqpoint{1.722200in}{1.583784in}}{\pgfqpoint{1.725473in}{1.575884in}}{\pgfqpoint{1.731297in}{1.570060in}}%
\pgfpathcurveto{\pgfqpoint{1.737120in}{1.564236in}}{\pgfqpoint{1.745021in}{1.560963in}}{\pgfqpoint{1.753257in}{1.560963in}}%
\pgfpathclose%
\pgfusepath{stroke,fill}%
\end{pgfscope}%
\begin{pgfscope}%
\pgfpathrectangle{\pgfqpoint{0.100000in}{0.212622in}}{\pgfqpoint{3.696000in}{3.696000in}}%
\pgfusepath{clip}%
\pgfsetbuttcap%
\pgfsetroundjoin%
\definecolor{currentfill}{rgb}{0.121569,0.466667,0.705882}%
\pgfsetfillcolor{currentfill}%
\pgfsetfillopacity{0.885801}%
\pgfsetlinewidth{1.003750pt}%
\definecolor{currentstroke}{rgb}{0.121569,0.466667,0.705882}%
\pgfsetstrokecolor{currentstroke}%
\pgfsetstrokeopacity{0.885801}%
\pgfsetdash{}{0pt}%
\pgfpathmoveto{\pgfqpoint{0.582540in}{2.616759in}}%
\pgfpathcurveto{\pgfqpoint{0.590777in}{2.616759in}}{\pgfqpoint{0.598677in}{2.620032in}}{\pgfqpoint{0.604501in}{2.625855in}}%
\pgfpathcurveto{\pgfqpoint{0.610325in}{2.631679in}}{\pgfqpoint{0.613597in}{2.639579in}}{\pgfqpoint{0.613597in}{2.647816in}}%
\pgfpathcurveto{\pgfqpoint{0.613597in}{2.656052in}}{\pgfqpoint{0.610325in}{2.663952in}}{\pgfqpoint{0.604501in}{2.669776in}}%
\pgfpathcurveto{\pgfqpoint{0.598677in}{2.675600in}}{\pgfqpoint{0.590777in}{2.678872in}}{\pgfqpoint{0.582540in}{2.678872in}}%
\pgfpathcurveto{\pgfqpoint{0.574304in}{2.678872in}}{\pgfqpoint{0.566404in}{2.675600in}}{\pgfqpoint{0.560580in}{2.669776in}}%
\pgfpathcurveto{\pgfqpoint{0.554756in}{2.663952in}}{\pgfqpoint{0.551484in}{2.656052in}}{\pgfqpoint{0.551484in}{2.647816in}}%
\pgfpathcurveto{\pgfqpoint{0.551484in}{2.639579in}}{\pgfqpoint{0.554756in}{2.631679in}}{\pgfqpoint{0.560580in}{2.625855in}}%
\pgfpathcurveto{\pgfqpoint{0.566404in}{2.620032in}}{\pgfqpoint{0.574304in}{2.616759in}}{\pgfqpoint{0.582540in}{2.616759in}}%
\pgfpathclose%
\pgfusepath{stroke,fill}%
\end{pgfscope}%
\begin{pgfscope}%
\pgfpathrectangle{\pgfqpoint{0.100000in}{0.212622in}}{\pgfqpoint{3.696000in}{3.696000in}}%
\pgfusepath{clip}%
\pgfsetbuttcap%
\pgfsetroundjoin%
\definecolor{currentfill}{rgb}{0.121569,0.466667,0.705882}%
\pgfsetfillcolor{currentfill}%
\pgfsetfillopacity{0.886341}%
\pgfsetlinewidth{1.003750pt}%
\definecolor{currentstroke}{rgb}{0.121569,0.466667,0.705882}%
\pgfsetstrokecolor{currentstroke}%
\pgfsetstrokeopacity{0.886341}%
\pgfsetdash{}{0pt}%
\pgfpathmoveto{\pgfqpoint{0.580361in}{2.618930in}}%
\pgfpathcurveto{\pgfqpoint{0.588597in}{2.618930in}}{\pgfqpoint{0.596497in}{2.622202in}}{\pgfqpoint{0.602321in}{2.628026in}}%
\pgfpathcurveto{\pgfqpoint{0.608145in}{2.633850in}}{\pgfqpoint{0.611417in}{2.641750in}}{\pgfqpoint{0.611417in}{2.649986in}}%
\pgfpathcurveto{\pgfqpoint{0.611417in}{2.658223in}}{\pgfqpoint{0.608145in}{2.666123in}}{\pgfqpoint{0.602321in}{2.671947in}}%
\pgfpathcurveto{\pgfqpoint{0.596497in}{2.677771in}}{\pgfqpoint{0.588597in}{2.681043in}}{\pgfqpoint{0.580361in}{2.681043in}}%
\pgfpathcurveto{\pgfqpoint{0.572124in}{2.681043in}}{\pgfqpoint{0.564224in}{2.677771in}}{\pgfqpoint{0.558400in}{2.671947in}}%
\pgfpathcurveto{\pgfqpoint{0.552576in}{2.666123in}}{\pgfqpoint{0.549304in}{2.658223in}}{\pgfqpoint{0.549304in}{2.649986in}}%
\pgfpathcurveto{\pgfqpoint{0.549304in}{2.641750in}}{\pgfqpoint{0.552576in}{2.633850in}}{\pgfqpoint{0.558400in}{2.628026in}}%
\pgfpathcurveto{\pgfqpoint{0.564224in}{2.622202in}}{\pgfqpoint{0.572124in}{2.618930in}}{\pgfqpoint{0.580361in}{2.618930in}}%
\pgfpathclose%
\pgfusepath{stroke,fill}%
\end{pgfscope}%
\begin{pgfscope}%
\pgfpathrectangle{\pgfqpoint{0.100000in}{0.212622in}}{\pgfqpoint{3.696000in}{3.696000in}}%
\pgfusepath{clip}%
\pgfsetbuttcap%
\pgfsetroundjoin%
\definecolor{currentfill}{rgb}{0.121569,0.466667,0.705882}%
\pgfsetfillcolor{currentfill}%
\pgfsetfillopacity{0.886447}%
\pgfsetlinewidth{1.003750pt}%
\definecolor{currentstroke}{rgb}{0.121569,0.466667,0.705882}%
\pgfsetstrokecolor{currentstroke}%
\pgfsetstrokeopacity{0.886447}%
\pgfsetdash{}{0pt}%
\pgfpathmoveto{\pgfqpoint{0.581083in}{2.619334in}}%
\pgfpathcurveto{\pgfqpoint{0.589319in}{2.619334in}}{\pgfqpoint{0.597219in}{2.622606in}}{\pgfqpoint{0.603043in}{2.628430in}}%
\pgfpathcurveto{\pgfqpoint{0.608867in}{2.634254in}}{\pgfqpoint{0.612139in}{2.642154in}}{\pgfqpoint{0.612139in}{2.650390in}}%
\pgfpathcurveto{\pgfqpoint{0.612139in}{2.658626in}}{\pgfqpoint{0.608867in}{2.666526in}}{\pgfqpoint{0.603043in}{2.672350in}}%
\pgfpathcurveto{\pgfqpoint{0.597219in}{2.678174in}}{\pgfqpoint{0.589319in}{2.681447in}}{\pgfqpoint{0.581083in}{2.681447in}}%
\pgfpathcurveto{\pgfqpoint{0.572846in}{2.681447in}}{\pgfqpoint{0.564946in}{2.678174in}}{\pgfqpoint{0.559122in}{2.672350in}}%
\pgfpathcurveto{\pgfqpoint{0.553299in}{2.666526in}}{\pgfqpoint{0.550026in}{2.658626in}}{\pgfqpoint{0.550026in}{2.650390in}}%
\pgfpathcurveto{\pgfqpoint{0.550026in}{2.642154in}}{\pgfqpoint{0.553299in}{2.634254in}}{\pgfqpoint{0.559122in}{2.628430in}}%
\pgfpathcurveto{\pgfqpoint{0.564946in}{2.622606in}}{\pgfqpoint{0.572846in}{2.619334in}}{\pgfqpoint{0.581083in}{2.619334in}}%
\pgfpathclose%
\pgfusepath{stroke,fill}%
\end{pgfscope}%
\begin{pgfscope}%
\pgfpathrectangle{\pgfqpoint{0.100000in}{0.212622in}}{\pgfqpoint{3.696000in}{3.696000in}}%
\pgfusepath{clip}%
\pgfsetbuttcap%
\pgfsetroundjoin%
\definecolor{currentfill}{rgb}{0.121569,0.466667,0.705882}%
\pgfsetfillcolor{currentfill}%
\pgfsetfillopacity{0.886454}%
\pgfsetlinewidth{1.003750pt}%
\definecolor{currentstroke}{rgb}{0.121569,0.466667,0.705882}%
\pgfsetstrokecolor{currentstroke}%
\pgfsetstrokeopacity{0.886454}%
\pgfsetdash{}{0pt}%
\pgfpathmoveto{\pgfqpoint{0.579994in}{2.619005in}}%
\pgfpathcurveto{\pgfqpoint{0.588231in}{2.619005in}}{\pgfqpoint{0.596131in}{2.622278in}}{\pgfqpoint{0.601955in}{2.628102in}}%
\pgfpathcurveto{\pgfqpoint{0.607778in}{2.633925in}}{\pgfqpoint{0.611051in}{2.641825in}}{\pgfqpoint{0.611051in}{2.650062in}}%
\pgfpathcurveto{\pgfqpoint{0.611051in}{2.658298in}}{\pgfqpoint{0.607778in}{2.666198in}}{\pgfqpoint{0.601955in}{2.672022in}}%
\pgfpathcurveto{\pgfqpoint{0.596131in}{2.677846in}}{\pgfqpoint{0.588231in}{2.681118in}}{\pgfqpoint{0.579994in}{2.681118in}}%
\pgfpathcurveto{\pgfqpoint{0.571758in}{2.681118in}}{\pgfqpoint{0.563858in}{2.677846in}}{\pgfqpoint{0.558034in}{2.672022in}}%
\pgfpathcurveto{\pgfqpoint{0.552210in}{2.666198in}}{\pgfqpoint{0.548938in}{2.658298in}}{\pgfqpoint{0.548938in}{2.650062in}}%
\pgfpathcurveto{\pgfqpoint{0.548938in}{2.641825in}}{\pgfqpoint{0.552210in}{2.633925in}}{\pgfqpoint{0.558034in}{2.628102in}}%
\pgfpathcurveto{\pgfqpoint{0.563858in}{2.622278in}}{\pgfqpoint{0.571758in}{2.619005in}}{\pgfqpoint{0.579994in}{2.619005in}}%
\pgfpathclose%
\pgfusepath{stroke,fill}%
\end{pgfscope}%
\begin{pgfscope}%
\pgfpathrectangle{\pgfqpoint{0.100000in}{0.212622in}}{\pgfqpoint{3.696000in}{3.696000in}}%
\pgfusepath{clip}%
\pgfsetbuttcap%
\pgfsetroundjoin%
\definecolor{currentfill}{rgb}{0.121569,0.466667,0.705882}%
\pgfsetfillcolor{currentfill}%
\pgfsetfillopacity{0.886485}%
\pgfsetlinewidth{1.003750pt}%
\definecolor{currentstroke}{rgb}{0.121569,0.466667,0.705882}%
\pgfsetstrokecolor{currentstroke}%
\pgfsetstrokeopacity{0.886485}%
\pgfsetdash{}{0pt}%
\pgfpathmoveto{\pgfqpoint{0.584634in}{2.617775in}}%
\pgfpathcurveto{\pgfqpoint{0.592870in}{2.617775in}}{\pgfqpoint{0.600770in}{2.621047in}}{\pgfqpoint{0.606594in}{2.626871in}}%
\pgfpathcurveto{\pgfqpoint{0.612418in}{2.632695in}}{\pgfqpoint{0.615691in}{2.640595in}}{\pgfqpoint{0.615691in}{2.648831in}}%
\pgfpathcurveto{\pgfqpoint{0.615691in}{2.657068in}}{\pgfqpoint{0.612418in}{2.664968in}}{\pgfqpoint{0.606594in}{2.670792in}}%
\pgfpathcurveto{\pgfqpoint{0.600770in}{2.676616in}}{\pgfqpoint{0.592870in}{2.679888in}}{\pgfqpoint{0.584634in}{2.679888in}}%
\pgfpathcurveto{\pgfqpoint{0.576398in}{2.679888in}}{\pgfqpoint{0.568498in}{2.676616in}}{\pgfqpoint{0.562674in}{2.670792in}}%
\pgfpathcurveto{\pgfqpoint{0.556850in}{2.664968in}}{\pgfqpoint{0.553578in}{2.657068in}}{\pgfqpoint{0.553578in}{2.648831in}}%
\pgfpathcurveto{\pgfqpoint{0.553578in}{2.640595in}}{\pgfqpoint{0.556850in}{2.632695in}}{\pgfqpoint{0.562674in}{2.626871in}}%
\pgfpathcurveto{\pgfqpoint{0.568498in}{2.621047in}}{\pgfqpoint{0.576398in}{2.617775in}}{\pgfqpoint{0.584634in}{2.617775in}}%
\pgfpathclose%
\pgfusepath{stroke,fill}%
\end{pgfscope}%
\begin{pgfscope}%
\pgfpathrectangle{\pgfqpoint{0.100000in}{0.212622in}}{\pgfqpoint{3.696000in}{3.696000in}}%
\pgfusepath{clip}%
\pgfsetbuttcap%
\pgfsetroundjoin%
\definecolor{currentfill}{rgb}{0.121569,0.466667,0.705882}%
\pgfsetfillcolor{currentfill}%
\pgfsetfillopacity{0.886664}%
\pgfsetlinewidth{1.003750pt}%
\definecolor{currentstroke}{rgb}{0.121569,0.466667,0.705882}%
\pgfsetstrokecolor{currentstroke}%
\pgfsetstrokeopacity{0.886664}%
\pgfsetdash{}{0pt}%
\pgfpathmoveto{\pgfqpoint{0.578363in}{2.617611in}}%
\pgfpathcurveto{\pgfqpoint{0.586600in}{2.617611in}}{\pgfqpoint{0.594500in}{2.620883in}}{\pgfqpoint{0.600324in}{2.626707in}}%
\pgfpathcurveto{\pgfqpoint{0.606148in}{2.632531in}}{\pgfqpoint{0.609420in}{2.640431in}}{\pgfqpoint{0.609420in}{2.648667in}}%
\pgfpathcurveto{\pgfqpoint{0.609420in}{2.656904in}}{\pgfqpoint{0.606148in}{2.664804in}}{\pgfqpoint{0.600324in}{2.670628in}}%
\pgfpathcurveto{\pgfqpoint{0.594500in}{2.676452in}}{\pgfqpoint{0.586600in}{2.679724in}}{\pgfqpoint{0.578363in}{2.679724in}}%
\pgfpathcurveto{\pgfqpoint{0.570127in}{2.679724in}}{\pgfqpoint{0.562227in}{2.676452in}}{\pgfqpoint{0.556403in}{2.670628in}}%
\pgfpathcurveto{\pgfqpoint{0.550579in}{2.664804in}}{\pgfqpoint{0.547307in}{2.656904in}}{\pgfqpoint{0.547307in}{2.648667in}}%
\pgfpathcurveto{\pgfqpoint{0.547307in}{2.640431in}}{\pgfqpoint{0.550579in}{2.632531in}}{\pgfqpoint{0.556403in}{2.626707in}}%
\pgfpathcurveto{\pgfqpoint{0.562227in}{2.620883in}}{\pgfqpoint{0.570127in}{2.617611in}}{\pgfqpoint{0.578363in}{2.617611in}}%
\pgfpathclose%
\pgfusepath{stroke,fill}%
\end{pgfscope}%
\begin{pgfscope}%
\pgfpathrectangle{\pgfqpoint{0.100000in}{0.212622in}}{\pgfqpoint{3.696000in}{3.696000in}}%
\pgfusepath{clip}%
\pgfsetbuttcap%
\pgfsetroundjoin%
\definecolor{currentfill}{rgb}{0.121569,0.466667,0.705882}%
\pgfsetfillcolor{currentfill}%
\pgfsetfillopacity{0.886782}%
\pgfsetlinewidth{1.003750pt}%
\definecolor{currentstroke}{rgb}{0.121569,0.466667,0.705882}%
\pgfsetstrokecolor{currentstroke}%
\pgfsetstrokeopacity{0.886782}%
\pgfsetdash{}{0pt}%
\pgfpathmoveto{\pgfqpoint{1.753782in}{1.560342in}}%
\pgfpathcurveto{\pgfqpoint{1.762019in}{1.560342in}}{\pgfqpoint{1.769919in}{1.563614in}}{\pgfqpoint{1.775743in}{1.569438in}}%
\pgfpathcurveto{\pgfqpoint{1.781566in}{1.575262in}}{\pgfqpoint{1.784839in}{1.583162in}}{\pgfqpoint{1.784839in}{1.591399in}}%
\pgfpathcurveto{\pgfqpoint{1.784839in}{1.599635in}}{\pgfqpoint{1.781566in}{1.607535in}}{\pgfqpoint{1.775743in}{1.613359in}}%
\pgfpathcurveto{\pgfqpoint{1.769919in}{1.619183in}}{\pgfqpoint{1.762019in}{1.622455in}}{\pgfqpoint{1.753782in}{1.622455in}}%
\pgfpathcurveto{\pgfqpoint{1.745546in}{1.622455in}}{\pgfqpoint{1.737646in}{1.619183in}}{\pgfqpoint{1.731822in}{1.613359in}}%
\pgfpathcurveto{\pgfqpoint{1.725998in}{1.607535in}}{\pgfqpoint{1.722726in}{1.599635in}}{\pgfqpoint{1.722726in}{1.591399in}}%
\pgfpathcurveto{\pgfqpoint{1.722726in}{1.583162in}}{\pgfqpoint{1.725998in}{1.575262in}}{\pgfqpoint{1.731822in}{1.569438in}}%
\pgfpathcurveto{\pgfqpoint{1.737646in}{1.563614in}}{\pgfqpoint{1.745546in}{1.560342in}}{\pgfqpoint{1.753782in}{1.560342in}}%
\pgfpathclose%
\pgfusepath{stroke,fill}%
\end{pgfscope}%
\begin{pgfscope}%
\pgfpathrectangle{\pgfqpoint{0.100000in}{0.212622in}}{\pgfqpoint{3.696000in}{3.696000in}}%
\pgfusepath{clip}%
\pgfsetbuttcap%
\pgfsetroundjoin%
\definecolor{currentfill}{rgb}{0.121569,0.466667,0.705882}%
\pgfsetfillcolor{currentfill}%
\pgfsetfillopacity{0.886982}%
\pgfsetlinewidth{1.003750pt}%
\definecolor{currentstroke}{rgb}{0.121569,0.466667,0.705882}%
\pgfsetstrokecolor{currentstroke}%
\pgfsetstrokeopacity{0.886982}%
\pgfsetdash{}{0pt}%
\pgfpathmoveto{\pgfqpoint{0.575879in}{2.615487in}}%
\pgfpathcurveto{\pgfqpoint{0.584115in}{2.615487in}}{\pgfqpoint{0.592015in}{2.618760in}}{\pgfqpoint{0.597839in}{2.624583in}}%
\pgfpathcurveto{\pgfqpoint{0.603663in}{2.630407in}}{\pgfqpoint{0.606936in}{2.638307in}}{\pgfqpoint{0.606936in}{2.646544in}}%
\pgfpathcurveto{\pgfqpoint{0.606936in}{2.654780in}}{\pgfqpoint{0.603663in}{2.662680in}}{\pgfqpoint{0.597839in}{2.668504in}}%
\pgfpathcurveto{\pgfqpoint{0.592015in}{2.674328in}}{\pgfqpoint{0.584115in}{2.677600in}}{\pgfqpoint{0.575879in}{2.677600in}}%
\pgfpathcurveto{\pgfqpoint{0.567643in}{2.677600in}}{\pgfqpoint{0.559743in}{2.674328in}}{\pgfqpoint{0.553919in}{2.668504in}}%
\pgfpathcurveto{\pgfqpoint{0.548095in}{2.662680in}}{\pgfqpoint{0.544823in}{2.654780in}}{\pgfqpoint{0.544823in}{2.646544in}}%
\pgfpathcurveto{\pgfqpoint{0.544823in}{2.638307in}}{\pgfqpoint{0.548095in}{2.630407in}}{\pgfqpoint{0.553919in}{2.624583in}}%
\pgfpathcurveto{\pgfqpoint{0.559743in}{2.618760in}}{\pgfqpoint{0.567643in}{2.615487in}}{\pgfqpoint{0.575879in}{2.615487in}}%
\pgfpathclose%
\pgfusepath{stroke,fill}%
\end{pgfscope}%
\begin{pgfscope}%
\pgfpathrectangle{\pgfqpoint{0.100000in}{0.212622in}}{\pgfqpoint{3.696000in}{3.696000in}}%
\pgfusepath{clip}%
\pgfsetbuttcap%
\pgfsetroundjoin%
\definecolor{currentfill}{rgb}{0.121569,0.466667,0.705882}%
\pgfsetfillcolor{currentfill}%
\pgfsetfillopacity{0.887438}%
\pgfsetlinewidth{1.003750pt}%
\definecolor{currentstroke}{rgb}{0.121569,0.466667,0.705882}%
\pgfsetstrokecolor{currentstroke}%
\pgfsetstrokeopacity{0.887438}%
\pgfsetdash{}{0pt}%
\pgfpathmoveto{\pgfqpoint{1.754153in}{1.560081in}}%
\pgfpathcurveto{\pgfqpoint{1.762389in}{1.560081in}}{\pgfqpoint{1.770289in}{1.563353in}}{\pgfqpoint{1.776113in}{1.569177in}}%
\pgfpathcurveto{\pgfqpoint{1.781937in}{1.575001in}}{\pgfqpoint{1.785209in}{1.582901in}}{\pgfqpoint{1.785209in}{1.591138in}}%
\pgfpathcurveto{\pgfqpoint{1.785209in}{1.599374in}}{\pgfqpoint{1.781937in}{1.607274in}}{\pgfqpoint{1.776113in}{1.613098in}}%
\pgfpathcurveto{\pgfqpoint{1.770289in}{1.618922in}}{\pgfqpoint{1.762389in}{1.622194in}}{\pgfqpoint{1.754153in}{1.622194in}}%
\pgfpathcurveto{\pgfqpoint{1.745917in}{1.622194in}}{\pgfqpoint{1.738017in}{1.618922in}}{\pgfqpoint{1.732193in}{1.613098in}}%
\pgfpathcurveto{\pgfqpoint{1.726369in}{1.607274in}}{\pgfqpoint{1.723096in}{1.599374in}}{\pgfqpoint{1.723096in}{1.591138in}}%
\pgfpathcurveto{\pgfqpoint{1.723096in}{1.582901in}}{\pgfqpoint{1.726369in}{1.575001in}}{\pgfqpoint{1.732193in}{1.569177in}}%
\pgfpathcurveto{\pgfqpoint{1.738017in}{1.563353in}}{\pgfqpoint{1.745917in}{1.560081in}}{\pgfqpoint{1.754153in}{1.560081in}}%
\pgfpathclose%
\pgfusepath{stroke,fill}%
\end{pgfscope}%
\begin{pgfscope}%
\pgfpathrectangle{\pgfqpoint{0.100000in}{0.212622in}}{\pgfqpoint{3.696000in}{3.696000in}}%
\pgfusepath{clip}%
\pgfsetbuttcap%
\pgfsetroundjoin%
\definecolor{currentfill}{rgb}{0.121569,0.466667,0.705882}%
\pgfsetfillcolor{currentfill}%
\pgfsetfillopacity{0.887778}%
\pgfsetlinewidth{1.003750pt}%
\definecolor{currentstroke}{rgb}{0.121569,0.466667,0.705882}%
\pgfsetstrokecolor{currentstroke}%
\pgfsetstrokeopacity{0.887778}%
\pgfsetdash{}{0pt}%
\pgfpathmoveto{\pgfqpoint{1.754320in}{1.559832in}}%
\pgfpathcurveto{\pgfqpoint{1.762556in}{1.559832in}}{\pgfqpoint{1.770456in}{1.563104in}}{\pgfqpoint{1.776280in}{1.568928in}}%
\pgfpathcurveto{\pgfqpoint{1.782104in}{1.574752in}}{\pgfqpoint{1.785377in}{1.582652in}}{\pgfqpoint{1.785377in}{1.590888in}}%
\pgfpathcurveto{\pgfqpoint{1.785377in}{1.599124in}}{\pgfqpoint{1.782104in}{1.607024in}}{\pgfqpoint{1.776280in}{1.612848in}}%
\pgfpathcurveto{\pgfqpoint{1.770456in}{1.618672in}}{\pgfqpoint{1.762556in}{1.621945in}}{\pgfqpoint{1.754320in}{1.621945in}}%
\pgfpathcurveto{\pgfqpoint{1.746084in}{1.621945in}}{\pgfqpoint{1.738184in}{1.618672in}}{\pgfqpoint{1.732360in}{1.612848in}}%
\pgfpathcurveto{\pgfqpoint{1.726536in}{1.607024in}}{\pgfqpoint{1.723264in}{1.599124in}}{\pgfqpoint{1.723264in}{1.590888in}}%
\pgfpathcurveto{\pgfqpoint{1.723264in}{1.582652in}}{\pgfqpoint{1.726536in}{1.574752in}}{\pgfqpoint{1.732360in}{1.568928in}}%
\pgfpathcurveto{\pgfqpoint{1.738184in}{1.563104in}}{\pgfqpoint{1.746084in}{1.559832in}}{\pgfqpoint{1.754320in}{1.559832in}}%
\pgfpathclose%
\pgfusepath{stroke,fill}%
\end{pgfscope}%
\begin{pgfscope}%
\pgfpathrectangle{\pgfqpoint{0.100000in}{0.212622in}}{\pgfqpoint{3.696000in}{3.696000in}}%
\pgfusepath{clip}%
\pgfsetbuttcap%
\pgfsetroundjoin%
\definecolor{currentfill}{rgb}{0.121569,0.466667,0.705882}%
\pgfsetfillcolor{currentfill}%
\pgfsetfillopacity{0.888484}%
\pgfsetlinewidth{1.003750pt}%
\definecolor{currentstroke}{rgb}{0.121569,0.466667,0.705882}%
\pgfsetstrokecolor{currentstroke}%
\pgfsetstrokeopacity{0.888484}%
\pgfsetdash{}{0pt}%
\pgfpathmoveto{\pgfqpoint{1.754794in}{1.559696in}}%
\pgfpathcurveto{\pgfqpoint{1.763030in}{1.559696in}}{\pgfqpoint{1.770930in}{1.562968in}}{\pgfqpoint{1.776754in}{1.568792in}}%
\pgfpathcurveto{\pgfqpoint{1.782578in}{1.574616in}}{\pgfqpoint{1.785851in}{1.582516in}}{\pgfqpoint{1.785851in}{1.590752in}}%
\pgfpathcurveto{\pgfqpoint{1.785851in}{1.598988in}}{\pgfqpoint{1.782578in}{1.606888in}}{\pgfqpoint{1.776754in}{1.612712in}}%
\pgfpathcurveto{\pgfqpoint{1.770930in}{1.618536in}}{\pgfqpoint{1.763030in}{1.621809in}}{\pgfqpoint{1.754794in}{1.621809in}}%
\pgfpathcurveto{\pgfqpoint{1.746558in}{1.621809in}}{\pgfqpoint{1.738658in}{1.618536in}}{\pgfqpoint{1.732834in}{1.612712in}}%
\pgfpathcurveto{\pgfqpoint{1.727010in}{1.606888in}}{\pgfqpoint{1.723738in}{1.598988in}}{\pgfqpoint{1.723738in}{1.590752in}}%
\pgfpathcurveto{\pgfqpoint{1.723738in}{1.582516in}}{\pgfqpoint{1.727010in}{1.574616in}}{\pgfqpoint{1.732834in}{1.568792in}}%
\pgfpathcurveto{\pgfqpoint{1.738658in}{1.562968in}}{\pgfqpoint{1.746558in}{1.559696in}}{\pgfqpoint{1.754794in}{1.559696in}}%
\pgfpathclose%
\pgfusepath{stroke,fill}%
\end{pgfscope}%
\begin{pgfscope}%
\pgfpathrectangle{\pgfqpoint{0.100000in}{0.212622in}}{\pgfqpoint{3.696000in}{3.696000in}}%
\pgfusepath{clip}%
\pgfsetbuttcap%
\pgfsetroundjoin%
\definecolor{currentfill}{rgb}{0.121569,0.466667,0.705882}%
\pgfsetfillcolor{currentfill}%
\pgfsetfillopacity{0.888755}%
\pgfsetlinewidth{1.003750pt}%
\definecolor{currentstroke}{rgb}{0.121569,0.466667,0.705882}%
\pgfsetstrokecolor{currentstroke}%
\pgfsetstrokeopacity{0.888755}%
\pgfsetdash{}{0pt}%
\pgfpathmoveto{\pgfqpoint{1.754935in}{1.559051in}}%
\pgfpathcurveto{\pgfqpoint{1.763171in}{1.559051in}}{\pgfqpoint{1.771071in}{1.562323in}}{\pgfqpoint{1.776895in}{1.568147in}}%
\pgfpathcurveto{\pgfqpoint{1.782719in}{1.573971in}}{\pgfqpoint{1.785992in}{1.581871in}}{\pgfqpoint{1.785992in}{1.590107in}}%
\pgfpathcurveto{\pgfqpoint{1.785992in}{1.598344in}}{\pgfqpoint{1.782719in}{1.606244in}}{\pgfqpoint{1.776895in}{1.612068in}}%
\pgfpathcurveto{\pgfqpoint{1.771071in}{1.617892in}}{\pgfqpoint{1.763171in}{1.621164in}}{\pgfqpoint{1.754935in}{1.621164in}}%
\pgfpathcurveto{\pgfqpoint{1.746699in}{1.621164in}}{\pgfqpoint{1.738799in}{1.617892in}}{\pgfqpoint{1.732975in}{1.612068in}}%
\pgfpathcurveto{\pgfqpoint{1.727151in}{1.606244in}}{\pgfqpoint{1.723879in}{1.598344in}}{\pgfqpoint{1.723879in}{1.590107in}}%
\pgfpathcurveto{\pgfqpoint{1.723879in}{1.581871in}}{\pgfqpoint{1.727151in}{1.573971in}}{\pgfqpoint{1.732975in}{1.568147in}}%
\pgfpathcurveto{\pgfqpoint{1.738799in}{1.562323in}}{\pgfqpoint{1.746699in}{1.559051in}}{\pgfqpoint{1.754935in}{1.559051in}}%
\pgfpathclose%
\pgfusepath{stroke,fill}%
\end{pgfscope}%
\begin{pgfscope}%
\pgfpathrectangle{\pgfqpoint{0.100000in}{0.212622in}}{\pgfqpoint{3.696000in}{3.696000in}}%
\pgfusepath{clip}%
\pgfsetbuttcap%
\pgfsetroundjoin%
\definecolor{currentfill}{rgb}{0.121569,0.466667,0.705882}%
\pgfsetfillcolor{currentfill}%
\pgfsetfillopacity{0.889494}%
\pgfsetlinewidth{1.003750pt}%
\definecolor{currentstroke}{rgb}{0.121569,0.466667,0.705882}%
\pgfsetstrokecolor{currentstroke}%
\pgfsetstrokeopacity{0.889494}%
\pgfsetdash{}{0pt}%
\pgfpathmoveto{\pgfqpoint{1.755336in}{1.559384in}}%
\pgfpathcurveto{\pgfqpoint{1.763572in}{1.559384in}}{\pgfqpoint{1.771472in}{1.562656in}}{\pgfqpoint{1.777296in}{1.568480in}}%
\pgfpathcurveto{\pgfqpoint{1.783120in}{1.574304in}}{\pgfqpoint{1.786392in}{1.582204in}}{\pgfqpoint{1.786392in}{1.590440in}}%
\pgfpathcurveto{\pgfqpoint{1.786392in}{1.598676in}}{\pgfqpoint{1.783120in}{1.606576in}}{\pgfqpoint{1.777296in}{1.612400in}}%
\pgfpathcurveto{\pgfqpoint{1.771472in}{1.618224in}}{\pgfqpoint{1.763572in}{1.621497in}}{\pgfqpoint{1.755336in}{1.621497in}}%
\pgfpathcurveto{\pgfqpoint{1.747100in}{1.621497in}}{\pgfqpoint{1.739199in}{1.618224in}}{\pgfqpoint{1.733376in}{1.612400in}}%
\pgfpathcurveto{\pgfqpoint{1.727552in}{1.606576in}}{\pgfqpoint{1.724279in}{1.598676in}}{\pgfqpoint{1.724279in}{1.590440in}}%
\pgfpathcurveto{\pgfqpoint{1.724279in}{1.582204in}}{\pgfqpoint{1.727552in}{1.574304in}}{\pgfqpoint{1.733376in}{1.568480in}}%
\pgfpathcurveto{\pgfqpoint{1.739199in}{1.562656in}}{\pgfqpoint{1.747100in}{1.559384in}}{\pgfqpoint{1.755336in}{1.559384in}}%
\pgfpathclose%
\pgfusepath{stroke,fill}%
\end{pgfscope}%
\begin{pgfscope}%
\pgfpathrectangle{\pgfqpoint{0.100000in}{0.212622in}}{\pgfqpoint{3.696000in}{3.696000in}}%
\pgfusepath{clip}%
\pgfsetbuttcap%
\pgfsetroundjoin%
\definecolor{currentfill}{rgb}{0.121569,0.466667,0.705882}%
\pgfsetfillcolor{currentfill}%
\pgfsetfillopacity{0.889857}%
\pgfsetlinewidth{1.003750pt}%
\definecolor{currentstroke}{rgb}{0.121569,0.466667,0.705882}%
\pgfsetstrokecolor{currentstroke}%
\pgfsetstrokeopacity{0.889857}%
\pgfsetdash{}{0pt}%
\pgfpathmoveto{\pgfqpoint{1.755520in}{1.559355in}}%
\pgfpathcurveto{\pgfqpoint{1.763756in}{1.559355in}}{\pgfqpoint{1.771656in}{1.562627in}}{\pgfqpoint{1.777480in}{1.568451in}}%
\pgfpathcurveto{\pgfqpoint{1.783304in}{1.574275in}}{\pgfqpoint{1.786577in}{1.582175in}}{\pgfqpoint{1.786577in}{1.590412in}}%
\pgfpathcurveto{\pgfqpoint{1.786577in}{1.598648in}}{\pgfqpoint{1.783304in}{1.606548in}}{\pgfqpoint{1.777480in}{1.612372in}}%
\pgfpathcurveto{\pgfqpoint{1.771656in}{1.618196in}}{\pgfqpoint{1.763756in}{1.621468in}}{\pgfqpoint{1.755520in}{1.621468in}}%
\pgfpathcurveto{\pgfqpoint{1.747284in}{1.621468in}}{\pgfqpoint{1.739384in}{1.618196in}}{\pgfqpoint{1.733560in}{1.612372in}}%
\pgfpathcurveto{\pgfqpoint{1.727736in}{1.606548in}}{\pgfqpoint{1.724464in}{1.598648in}}{\pgfqpoint{1.724464in}{1.590412in}}%
\pgfpathcurveto{\pgfqpoint{1.724464in}{1.582175in}}{\pgfqpoint{1.727736in}{1.574275in}}{\pgfqpoint{1.733560in}{1.568451in}}%
\pgfpathcurveto{\pgfqpoint{1.739384in}{1.562627in}}{\pgfqpoint{1.747284in}{1.559355in}}{\pgfqpoint{1.755520in}{1.559355in}}%
\pgfpathclose%
\pgfusepath{stroke,fill}%
\end{pgfscope}%
\begin{pgfscope}%
\pgfpathrectangle{\pgfqpoint{0.100000in}{0.212622in}}{\pgfqpoint{3.696000in}{3.696000in}}%
\pgfusepath{clip}%
\pgfsetbuttcap%
\pgfsetroundjoin%
\definecolor{currentfill}{rgb}{0.121569,0.466667,0.705882}%
\pgfsetfillcolor{currentfill}%
\pgfsetfillopacity{0.890006}%
\pgfsetlinewidth{1.003750pt}%
\definecolor{currentstroke}{rgb}{0.121569,0.466667,0.705882}%
\pgfsetstrokecolor{currentstroke}%
\pgfsetstrokeopacity{0.890006}%
\pgfsetdash{}{0pt}%
\pgfpathmoveto{\pgfqpoint{1.755568in}{1.559096in}}%
\pgfpathcurveto{\pgfqpoint{1.763804in}{1.559096in}}{\pgfqpoint{1.771704in}{1.562368in}}{\pgfqpoint{1.777528in}{1.568192in}}%
\pgfpathcurveto{\pgfqpoint{1.783352in}{1.574016in}}{\pgfqpoint{1.786624in}{1.581916in}}{\pgfqpoint{1.786624in}{1.590152in}}%
\pgfpathcurveto{\pgfqpoint{1.786624in}{1.598389in}}{\pgfqpoint{1.783352in}{1.606289in}}{\pgfqpoint{1.777528in}{1.612113in}}%
\pgfpathcurveto{\pgfqpoint{1.771704in}{1.617936in}}{\pgfqpoint{1.763804in}{1.621209in}}{\pgfqpoint{1.755568in}{1.621209in}}%
\pgfpathcurveto{\pgfqpoint{1.747331in}{1.621209in}}{\pgfqpoint{1.739431in}{1.617936in}}{\pgfqpoint{1.733607in}{1.612113in}}%
\pgfpathcurveto{\pgfqpoint{1.727784in}{1.606289in}}{\pgfqpoint{1.724511in}{1.598389in}}{\pgfqpoint{1.724511in}{1.590152in}}%
\pgfpathcurveto{\pgfqpoint{1.724511in}{1.581916in}}{\pgfqpoint{1.727784in}{1.574016in}}{\pgfqpoint{1.733607in}{1.568192in}}%
\pgfpathcurveto{\pgfqpoint{1.739431in}{1.562368in}}{\pgfqpoint{1.747331in}{1.559096in}}{\pgfqpoint{1.755568in}{1.559096in}}%
\pgfpathclose%
\pgfusepath{stroke,fill}%
\end{pgfscope}%
\begin{pgfscope}%
\pgfpathrectangle{\pgfqpoint{0.100000in}{0.212622in}}{\pgfqpoint{3.696000in}{3.696000in}}%
\pgfusepath{clip}%
\pgfsetbuttcap%
\pgfsetroundjoin%
\definecolor{currentfill}{rgb}{0.121569,0.466667,0.705882}%
\pgfsetfillcolor{currentfill}%
\pgfsetfillopacity{0.890065}%
\pgfsetlinewidth{1.003750pt}%
\definecolor{currentstroke}{rgb}{0.121569,0.466667,0.705882}%
\pgfsetstrokecolor{currentstroke}%
\pgfsetstrokeopacity{0.890065}%
\pgfsetdash{}{0pt}%
\pgfpathmoveto{\pgfqpoint{0.644791in}{2.550737in}}%
\pgfpathcurveto{\pgfqpoint{0.653027in}{2.550737in}}{\pgfqpoint{0.660927in}{2.554009in}}{\pgfqpoint{0.666751in}{2.559833in}}%
\pgfpathcurveto{\pgfqpoint{0.672575in}{2.565657in}}{\pgfqpoint{0.675847in}{2.573557in}}{\pgfqpoint{0.675847in}{2.581793in}}%
\pgfpathcurveto{\pgfqpoint{0.675847in}{2.590030in}}{\pgfqpoint{0.672575in}{2.597930in}}{\pgfqpoint{0.666751in}{2.603754in}}%
\pgfpathcurveto{\pgfqpoint{0.660927in}{2.609577in}}{\pgfqpoint{0.653027in}{2.612850in}}{\pgfqpoint{0.644791in}{2.612850in}}%
\pgfpathcurveto{\pgfqpoint{0.636554in}{2.612850in}}{\pgfqpoint{0.628654in}{2.609577in}}{\pgfqpoint{0.622830in}{2.603754in}}%
\pgfpathcurveto{\pgfqpoint{0.617006in}{2.597930in}}{\pgfqpoint{0.613734in}{2.590030in}}{\pgfqpoint{0.613734in}{2.581793in}}%
\pgfpathcurveto{\pgfqpoint{0.613734in}{2.573557in}}{\pgfqpoint{0.617006in}{2.565657in}}{\pgfqpoint{0.622830in}{2.559833in}}%
\pgfpathcurveto{\pgfqpoint{0.628654in}{2.554009in}}{\pgfqpoint{0.636554in}{2.550737in}}{\pgfqpoint{0.644791in}{2.550737in}}%
\pgfpathclose%
\pgfusepath{stroke,fill}%
\end{pgfscope}%
\begin{pgfscope}%
\pgfpathrectangle{\pgfqpoint{0.100000in}{0.212622in}}{\pgfqpoint{3.696000in}{3.696000in}}%
\pgfusepath{clip}%
\pgfsetbuttcap%
\pgfsetroundjoin%
\definecolor{currentfill}{rgb}{0.121569,0.466667,0.705882}%
\pgfsetfillcolor{currentfill}%
\pgfsetfillopacity{0.890438}%
\pgfsetlinewidth{1.003750pt}%
\definecolor{currentstroke}{rgb}{0.121569,0.466667,0.705882}%
\pgfsetstrokecolor{currentstroke}%
\pgfsetstrokeopacity{0.890438}%
\pgfsetdash{}{0pt}%
\pgfpathmoveto{\pgfqpoint{1.755991in}{1.557644in}}%
\pgfpathcurveto{\pgfqpoint{1.764228in}{1.557644in}}{\pgfqpoint{1.772128in}{1.560917in}}{\pgfqpoint{1.777952in}{1.566741in}}%
\pgfpathcurveto{\pgfqpoint{1.783776in}{1.572565in}}{\pgfqpoint{1.787048in}{1.580465in}}{\pgfqpoint{1.787048in}{1.588701in}}%
\pgfpathcurveto{\pgfqpoint{1.787048in}{1.596937in}}{\pgfqpoint{1.783776in}{1.604837in}}{\pgfqpoint{1.777952in}{1.610661in}}%
\pgfpathcurveto{\pgfqpoint{1.772128in}{1.616485in}}{\pgfqpoint{1.764228in}{1.619757in}}{\pgfqpoint{1.755991in}{1.619757in}}%
\pgfpathcurveto{\pgfqpoint{1.747755in}{1.619757in}}{\pgfqpoint{1.739855in}{1.616485in}}{\pgfqpoint{1.734031in}{1.610661in}}%
\pgfpathcurveto{\pgfqpoint{1.728207in}{1.604837in}}{\pgfqpoint{1.724935in}{1.596937in}}{\pgfqpoint{1.724935in}{1.588701in}}%
\pgfpathcurveto{\pgfqpoint{1.724935in}{1.580465in}}{\pgfqpoint{1.728207in}{1.572565in}}{\pgfqpoint{1.734031in}{1.566741in}}%
\pgfpathcurveto{\pgfqpoint{1.739855in}{1.560917in}}{\pgfqpoint{1.747755in}{1.557644in}}{\pgfqpoint{1.755991in}{1.557644in}}%
\pgfpathclose%
\pgfusepath{stroke,fill}%
\end{pgfscope}%
\begin{pgfscope}%
\pgfpathrectangle{\pgfqpoint{0.100000in}{0.212622in}}{\pgfqpoint{3.696000in}{3.696000in}}%
\pgfusepath{clip}%
\pgfsetbuttcap%
\pgfsetroundjoin%
\definecolor{currentfill}{rgb}{0.121569,0.466667,0.705882}%
\pgfsetfillcolor{currentfill}%
\pgfsetfillopacity{0.890920}%
\pgfsetlinewidth{1.003750pt}%
\definecolor{currentstroke}{rgb}{0.121569,0.466667,0.705882}%
\pgfsetstrokecolor{currentstroke}%
\pgfsetstrokeopacity{0.890920}%
\pgfsetdash{}{0pt}%
\pgfpathmoveto{\pgfqpoint{0.702905in}{2.472758in}}%
\pgfpathcurveto{\pgfqpoint{0.711142in}{2.472758in}}{\pgfqpoint{0.719042in}{2.476030in}}{\pgfqpoint{0.724866in}{2.481854in}}%
\pgfpathcurveto{\pgfqpoint{0.730690in}{2.487678in}}{\pgfqpoint{0.733962in}{2.495578in}}{\pgfqpoint{0.733962in}{2.503814in}}%
\pgfpathcurveto{\pgfqpoint{0.733962in}{2.512051in}}{\pgfqpoint{0.730690in}{2.519951in}}{\pgfqpoint{0.724866in}{2.525775in}}%
\pgfpathcurveto{\pgfqpoint{0.719042in}{2.531599in}}{\pgfqpoint{0.711142in}{2.534871in}}{\pgfqpoint{0.702905in}{2.534871in}}%
\pgfpathcurveto{\pgfqpoint{0.694669in}{2.534871in}}{\pgfqpoint{0.686769in}{2.531599in}}{\pgfqpoint{0.680945in}{2.525775in}}%
\pgfpathcurveto{\pgfqpoint{0.675121in}{2.519951in}}{\pgfqpoint{0.671849in}{2.512051in}}{\pgfqpoint{0.671849in}{2.503814in}}%
\pgfpathcurveto{\pgfqpoint{0.671849in}{2.495578in}}{\pgfqpoint{0.675121in}{2.487678in}}{\pgfqpoint{0.680945in}{2.481854in}}%
\pgfpathcurveto{\pgfqpoint{0.686769in}{2.476030in}}{\pgfqpoint{0.694669in}{2.472758in}}{\pgfqpoint{0.702905in}{2.472758in}}%
\pgfpathclose%
\pgfusepath{stroke,fill}%
\end{pgfscope}%
\begin{pgfscope}%
\pgfpathrectangle{\pgfqpoint{0.100000in}{0.212622in}}{\pgfqpoint{3.696000in}{3.696000in}}%
\pgfusepath{clip}%
\pgfsetbuttcap%
\pgfsetroundjoin%
\definecolor{currentfill}{rgb}{0.121569,0.466667,0.705882}%
\pgfsetfillcolor{currentfill}%
\pgfsetfillopacity{0.890976}%
\pgfsetlinewidth{1.003750pt}%
\definecolor{currentstroke}{rgb}{0.121569,0.466667,0.705882}%
\pgfsetstrokecolor{currentstroke}%
\pgfsetstrokeopacity{0.890976}%
\pgfsetdash{}{0pt}%
\pgfpathmoveto{\pgfqpoint{0.673803in}{2.513337in}}%
\pgfpathcurveto{\pgfqpoint{0.682039in}{2.513337in}}{\pgfqpoint{0.689939in}{2.516610in}}{\pgfqpoint{0.695763in}{2.522434in}}%
\pgfpathcurveto{\pgfqpoint{0.701587in}{2.528257in}}{\pgfqpoint{0.704859in}{2.536158in}}{\pgfqpoint{0.704859in}{2.544394in}}%
\pgfpathcurveto{\pgfqpoint{0.704859in}{2.552630in}}{\pgfqpoint{0.701587in}{2.560530in}}{\pgfqpoint{0.695763in}{2.566354in}}%
\pgfpathcurveto{\pgfqpoint{0.689939in}{2.572178in}}{\pgfqpoint{0.682039in}{2.575450in}}{\pgfqpoint{0.673803in}{2.575450in}}%
\pgfpathcurveto{\pgfqpoint{0.665566in}{2.575450in}}{\pgfqpoint{0.657666in}{2.572178in}}{\pgfqpoint{0.651842in}{2.566354in}}%
\pgfpathcurveto{\pgfqpoint{0.646019in}{2.560530in}}{\pgfqpoint{0.642746in}{2.552630in}}{\pgfqpoint{0.642746in}{2.544394in}}%
\pgfpathcurveto{\pgfqpoint{0.642746in}{2.536158in}}{\pgfqpoint{0.646019in}{2.528257in}}{\pgfqpoint{0.651842in}{2.522434in}}%
\pgfpathcurveto{\pgfqpoint{0.657666in}{2.516610in}}{\pgfqpoint{0.665566in}{2.513337in}}{\pgfqpoint{0.673803in}{2.513337in}}%
\pgfpathclose%
\pgfusepath{stroke,fill}%
\end{pgfscope}%
\begin{pgfscope}%
\pgfpathrectangle{\pgfqpoint{0.100000in}{0.212622in}}{\pgfqpoint{3.696000in}{3.696000in}}%
\pgfusepath{clip}%
\pgfsetbuttcap%
\pgfsetroundjoin%
\definecolor{currentfill}{rgb}{0.121569,0.466667,0.705882}%
\pgfsetfillcolor{currentfill}%
\pgfsetfillopacity{0.891413}%
\pgfsetlinewidth{1.003750pt}%
\definecolor{currentstroke}{rgb}{0.121569,0.466667,0.705882}%
\pgfsetstrokecolor{currentstroke}%
\pgfsetstrokeopacity{0.891413}%
\pgfsetdash{}{0pt}%
\pgfpathmoveto{\pgfqpoint{1.756420in}{1.557421in}}%
\pgfpathcurveto{\pgfqpoint{1.764656in}{1.557421in}}{\pgfqpoint{1.772556in}{1.560693in}}{\pgfqpoint{1.778380in}{1.566517in}}%
\pgfpathcurveto{\pgfqpoint{1.784204in}{1.572341in}}{\pgfqpoint{1.787477in}{1.580241in}}{\pgfqpoint{1.787477in}{1.588477in}}%
\pgfpathcurveto{\pgfqpoint{1.787477in}{1.596713in}}{\pgfqpoint{1.784204in}{1.604613in}}{\pgfqpoint{1.778380in}{1.610437in}}%
\pgfpathcurveto{\pgfqpoint{1.772556in}{1.616261in}}{\pgfqpoint{1.764656in}{1.619534in}}{\pgfqpoint{1.756420in}{1.619534in}}%
\pgfpathcurveto{\pgfqpoint{1.748184in}{1.619534in}}{\pgfqpoint{1.740284in}{1.616261in}}{\pgfqpoint{1.734460in}{1.610437in}}%
\pgfpathcurveto{\pgfqpoint{1.728636in}{1.604613in}}{\pgfqpoint{1.725364in}{1.596713in}}{\pgfqpoint{1.725364in}{1.588477in}}%
\pgfpathcurveto{\pgfqpoint{1.725364in}{1.580241in}}{\pgfqpoint{1.728636in}{1.572341in}}{\pgfqpoint{1.734460in}{1.566517in}}%
\pgfpathcurveto{\pgfqpoint{1.740284in}{1.560693in}}{\pgfqpoint{1.748184in}{1.557421in}}{\pgfqpoint{1.756420in}{1.557421in}}%
\pgfpathclose%
\pgfusepath{stroke,fill}%
\end{pgfscope}%
\begin{pgfscope}%
\pgfpathrectangle{\pgfqpoint{0.100000in}{0.212622in}}{\pgfqpoint{3.696000in}{3.696000in}}%
\pgfusepath{clip}%
\pgfsetbuttcap%
\pgfsetroundjoin%
\definecolor{currentfill}{rgb}{0.121569,0.466667,0.705882}%
\pgfsetfillcolor{currentfill}%
\pgfsetfillopacity{0.892937}%
\pgfsetlinewidth{1.003750pt}%
\definecolor{currentstroke}{rgb}{0.121569,0.466667,0.705882}%
\pgfsetstrokecolor{currentstroke}%
\pgfsetstrokeopacity{0.892937}%
\pgfsetdash{}{0pt}%
\pgfpathmoveto{\pgfqpoint{1.757223in}{1.557311in}}%
\pgfpathcurveto{\pgfqpoint{1.765459in}{1.557311in}}{\pgfqpoint{1.773359in}{1.560583in}}{\pgfqpoint{1.779183in}{1.566407in}}%
\pgfpathcurveto{\pgfqpoint{1.785007in}{1.572231in}}{\pgfqpoint{1.788279in}{1.580131in}}{\pgfqpoint{1.788279in}{1.588367in}}%
\pgfpathcurveto{\pgfqpoint{1.788279in}{1.596604in}}{\pgfqpoint{1.785007in}{1.604504in}}{\pgfqpoint{1.779183in}{1.610328in}}%
\pgfpathcurveto{\pgfqpoint{1.773359in}{1.616152in}}{\pgfqpoint{1.765459in}{1.619424in}}{\pgfqpoint{1.757223in}{1.619424in}}%
\pgfpathcurveto{\pgfqpoint{1.748987in}{1.619424in}}{\pgfqpoint{1.741086in}{1.616152in}}{\pgfqpoint{1.735263in}{1.610328in}}%
\pgfpathcurveto{\pgfqpoint{1.729439in}{1.604504in}}{\pgfqpoint{1.726166in}{1.596604in}}{\pgfqpoint{1.726166in}{1.588367in}}%
\pgfpathcurveto{\pgfqpoint{1.726166in}{1.580131in}}{\pgfqpoint{1.729439in}{1.572231in}}{\pgfqpoint{1.735263in}{1.566407in}}%
\pgfpathcurveto{\pgfqpoint{1.741086in}{1.560583in}}{\pgfqpoint{1.748987in}{1.557311in}}{\pgfqpoint{1.757223in}{1.557311in}}%
\pgfpathclose%
\pgfusepath{stroke,fill}%
\end{pgfscope}%
\begin{pgfscope}%
\pgfpathrectangle{\pgfqpoint{0.100000in}{0.212622in}}{\pgfqpoint{3.696000in}{3.696000in}}%
\pgfusepath{clip}%
\pgfsetbuttcap%
\pgfsetroundjoin%
\definecolor{currentfill}{rgb}{0.121569,0.466667,0.705882}%
\pgfsetfillcolor{currentfill}%
\pgfsetfillopacity{0.894499}%
\pgfsetlinewidth{1.003750pt}%
\definecolor{currentstroke}{rgb}{0.121569,0.466667,0.705882}%
\pgfsetstrokecolor{currentstroke}%
\pgfsetstrokeopacity{0.894499}%
\pgfsetdash{}{0pt}%
\pgfpathmoveto{\pgfqpoint{0.729556in}{2.445993in}}%
\pgfpathcurveto{\pgfqpoint{0.737793in}{2.445993in}}{\pgfqpoint{0.745693in}{2.449265in}}{\pgfqpoint{0.751517in}{2.455089in}}%
\pgfpathcurveto{\pgfqpoint{0.757341in}{2.460913in}}{\pgfqpoint{0.760613in}{2.468813in}}{\pgfqpoint{0.760613in}{2.477049in}}%
\pgfpathcurveto{\pgfqpoint{0.760613in}{2.485285in}}{\pgfqpoint{0.757341in}{2.493185in}}{\pgfqpoint{0.751517in}{2.499009in}}%
\pgfpathcurveto{\pgfqpoint{0.745693in}{2.504833in}}{\pgfqpoint{0.737793in}{2.508106in}}{\pgfqpoint{0.729556in}{2.508106in}}%
\pgfpathcurveto{\pgfqpoint{0.721320in}{2.508106in}}{\pgfqpoint{0.713420in}{2.504833in}}{\pgfqpoint{0.707596in}{2.499009in}}%
\pgfpathcurveto{\pgfqpoint{0.701772in}{2.493185in}}{\pgfqpoint{0.698500in}{2.485285in}}{\pgfqpoint{0.698500in}{2.477049in}}%
\pgfpathcurveto{\pgfqpoint{0.698500in}{2.468813in}}{\pgfqpoint{0.701772in}{2.460913in}}{\pgfqpoint{0.707596in}{2.455089in}}%
\pgfpathcurveto{\pgfqpoint{0.713420in}{2.449265in}}{\pgfqpoint{0.721320in}{2.445993in}}{\pgfqpoint{0.729556in}{2.445993in}}%
\pgfpathclose%
\pgfusepath{stroke,fill}%
\end{pgfscope}%
\begin{pgfscope}%
\pgfpathrectangle{\pgfqpoint{0.100000in}{0.212622in}}{\pgfqpoint{3.696000in}{3.696000in}}%
\pgfusepath{clip}%
\pgfsetbuttcap%
\pgfsetroundjoin%
\definecolor{currentfill}{rgb}{0.121569,0.466667,0.705882}%
\pgfsetfillcolor{currentfill}%
\pgfsetfillopacity{0.894796}%
\pgfsetlinewidth{1.003750pt}%
\definecolor{currentstroke}{rgb}{0.121569,0.466667,0.705882}%
\pgfsetstrokecolor{currentstroke}%
\pgfsetstrokeopacity{0.894796}%
\pgfsetdash{}{0pt}%
\pgfpathmoveto{\pgfqpoint{1.757644in}{1.557351in}}%
\pgfpathcurveto{\pgfqpoint{1.765880in}{1.557351in}}{\pgfqpoint{1.773780in}{1.560623in}}{\pgfqpoint{1.779604in}{1.566447in}}%
\pgfpathcurveto{\pgfqpoint{1.785428in}{1.572271in}}{\pgfqpoint{1.788700in}{1.580171in}}{\pgfqpoint{1.788700in}{1.588407in}}%
\pgfpathcurveto{\pgfqpoint{1.788700in}{1.596644in}}{\pgfqpoint{1.785428in}{1.604544in}}{\pgfqpoint{1.779604in}{1.610368in}}%
\pgfpathcurveto{\pgfqpoint{1.773780in}{1.616192in}}{\pgfqpoint{1.765880in}{1.619464in}}{\pgfqpoint{1.757644in}{1.619464in}}%
\pgfpathcurveto{\pgfqpoint{1.749407in}{1.619464in}}{\pgfqpoint{1.741507in}{1.616192in}}{\pgfqpoint{1.735683in}{1.610368in}}%
\pgfpathcurveto{\pgfqpoint{1.729859in}{1.604544in}}{\pgfqpoint{1.726587in}{1.596644in}}{\pgfqpoint{1.726587in}{1.588407in}}%
\pgfpathcurveto{\pgfqpoint{1.726587in}{1.580171in}}{\pgfqpoint{1.729859in}{1.572271in}}{\pgfqpoint{1.735683in}{1.566447in}}%
\pgfpathcurveto{\pgfqpoint{1.741507in}{1.560623in}}{\pgfqpoint{1.749407in}{1.557351in}}{\pgfqpoint{1.757644in}{1.557351in}}%
\pgfpathclose%
\pgfusepath{stroke,fill}%
\end{pgfscope}%
\begin{pgfscope}%
\pgfpathrectangle{\pgfqpoint{0.100000in}{0.212622in}}{\pgfqpoint{3.696000in}{3.696000in}}%
\pgfusepath{clip}%
\pgfsetbuttcap%
\pgfsetroundjoin%
\definecolor{currentfill}{rgb}{0.121569,0.466667,0.705882}%
\pgfsetfillcolor{currentfill}%
\pgfsetfillopacity{0.895692}%
\pgfsetlinewidth{1.003750pt}%
\definecolor{currentstroke}{rgb}{0.121569,0.466667,0.705882}%
\pgfsetstrokecolor{currentstroke}%
\pgfsetstrokeopacity{0.895692}%
\pgfsetdash{}{0pt}%
\pgfpathmoveto{\pgfqpoint{1.758103in}{1.556892in}}%
\pgfpathcurveto{\pgfqpoint{1.766339in}{1.556892in}}{\pgfqpoint{1.774239in}{1.560165in}}{\pgfqpoint{1.780063in}{1.565989in}}%
\pgfpathcurveto{\pgfqpoint{1.785887in}{1.571812in}}{\pgfqpoint{1.789160in}{1.579713in}}{\pgfqpoint{1.789160in}{1.587949in}}%
\pgfpathcurveto{\pgfqpoint{1.789160in}{1.596185in}}{\pgfqpoint{1.785887in}{1.604085in}}{\pgfqpoint{1.780063in}{1.609909in}}%
\pgfpathcurveto{\pgfqpoint{1.774239in}{1.615733in}}{\pgfqpoint{1.766339in}{1.619005in}}{\pgfqpoint{1.758103in}{1.619005in}}%
\pgfpathcurveto{\pgfqpoint{1.749867in}{1.619005in}}{\pgfqpoint{1.741967in}{1.615733in}}{\pgfqpoint{1.736143in}{1.609909in}}%
\pgfpathcurveto{\pgfqpoint{1.730319in}{1.604085in}}{\pgfqpoint{1.727047in}{1.596185in}}{\pgfqpoint{1.727047in}{1.587949in}}%
\pgfpathcurveto{\pgfqpoint{1.727047in}{1.579713in}}{\pgfqpoint{1.730319in}{1.571812in}}{\pgfqpoint{1.736143in}{1.565989in}}%
\pgfpathcurveto{\pgfqpoint{1.741967in}{1.560165in}}{\pgfqpoint{1.749867in}{1.556892in}}{\pgfqpoint{1.758103in}{1.556892in}}%
\pgfpathclose%
\pgfusepath{stroke,fill}%
\end{pgfscope}%
\begin{pgfscope}%
\pgfpathrectangle{\pgfqpoint{0.100000in}{0.212622in}}{\pgfqpoint{3.696000in}{3.696000in}}%
\pgfusepath{clip}%
\pgfsetbuttcap%
\pgfsetroundjoin%
\definecolor{currentfill}{rgb}{0.121569,0.466667,0.705882}%
\pgfsetfillcolor{currentfill}%
\pgfsetfillopacity{0.896604}%
\pgfsetlinewidth{1.003750pt}%
\definecolor{currentstroke}{rgb}{0.121569,0.466667,0.705882}%
\pgfsetstrokecolor{currentstroke}%
\pgfsetstrokeopacity{0.896604}%
\pgfsetdash{}{0pt}%
\pgfpathmoveto{\pgfqpoint{1.758870in}{1.555529in}}%
\pgfpathcurveto{\pgfqpoint{1.767107in}{1.555529in}}{\pgfqpoint{1.775007in}{1.558802in}}{\pgfqpoint{1.780830in}{1.564626in}}%
\pgfpathcurveto{\pgfqpoint{1.786654in}{1.570450in}}{\pgfqpoint{1.789927in}{1.578350in}}{\pgfqpoint{1.789927in}{1.586586in}}%
\pgfpathcurveto{\pgfqpoint{1.789927in}{1.594822in}}{\pgfqpoint{1.786654in}{1.602722in}}{\pgfqpoint{1.780830in}{1.608546in}}%
\pgfpathcurveto{\pgfqpoint{1.775007in}{1.614370in}}{\pgfqpoint{1.767107in}{1.617642in}}{\pgfqpoint{1.758870in}{1.617642in}}%
\pgfpathcurveto{\pgfqpoint{1.750634in}{1.617642in}}{\pgfqpoint{1.742734in}{1.614370in}}{\pgfqpoint{1.736910in}{1.608546in}}%
\pgfpathcurveto{\pgfqpoint{1.731086in}{1.602722in}}{\pgfqpoint{1.727814in}{1.594822in}}{\pgfqpoint{1.727814in}{1.586586in}}%
\pgfpathcurveto{\pgfqpoint{1.727814in}{1.578350in}}{\pgfqpoint{1.731086in}{1.570450in}}{\pgfqpoint{1.736910in}{1.564626in}}%
\pgfpathcurveto{\pgfqpoint{1.742734in}{1.558802in}}{\pgfqpoint{1.750634in}{1.555529in}}{\pgfqpoint{1.758870in}{1.555529in}}%
\pgfpathclose%
\pgfusepath{stroke,fill}%
\end{pgfscope}%
\begin{pgfscope}%
\pgfpathrectangle{\pgfqpoint{0.100000in}{0.212622in}}{\pgfqpoint{3.696000in}{3.696000in}}%
\pgfusepath{clip}%
\pgfsetbuttcap%
\pgfsetroundjoin%
\definecolor{currentfill}{rgb}{0.121569,0.466667,0.705882}%
\pgfsetfillcolor{currentfill}%
\pgfsetfillopacity{0.897186}%
\pgfsetlinewidth{1.003750pt}%
\definecolor{currentstroke}{rgb}{0.121569,0.466667,0.705882}%
\pgfsetstrokecolor{currentstroke}%
\pgfsetstrokeopacity{0.897186}%
\pgfsetdash{}{0pt}%
\pgfpathmoveto{\pgfqpoint{0.757001in}{2.423479in}}%
\pgfpathcurveto{\pgfqpoint{0.765237in}{2.423479in}}{\pgfqpoint{0.773137in}{2.426752in}}{\pgfqpoint{0.778961in}{2.432575in}}%
\pgfpathcurveto{\pgfqpoint{0.784785in}{2.438399in}}{\pgfqpoint{0.788057in}{2.446299in}}{\pgfqpoint{0.788057in}{2.454536in}}%
\pgfpathcurveto{\pgfqpoint{0.788057in}{2.462772in}}{\pgfqpoint{0.784785in}{2.470672in}}{\pgfqpoint{0.778961in}{2.476496in}}%
\pgfpathcurveto{\pgfqpoint{0.773137in}{2.482320in}}{\pgfqpoint{0.765237in}{2.485592in}}{\pgfqpoint{0.757001in}{2.485592in}}%
\pgfpathcurveto{\pgfqpoint{0.748764in}{2.485592in}}{\pgfqpoint{0.740864in}{2.482320in}}{\pgfqpoint{0.735040in}{2.476496in}}%
\pgfpathcurveto{\pgfqpoint{0.729216in}{2.470672in}}{\pgfqpoint{0.725944in}{2.462772in}}{\pgfqpoint{0.725944in}{2.454536in}}%
\pgfpathcurveto{\pgfqpoint{0.725944in}{2.446299in}}{\pgfqpoint{0.729216in}{2.438399in}}{\pgfqpoint{0.735040in}{2.432575in}}%
\pgfpathcurveto{\pgfqpoint{0.740864in}{2.426752in}}{\pgfqpoint{0.748764in}{2.423479in}}{\pgfqpoint{0.757001in}{2.423479in}}%
\pgfpathclose%
\pgfusepath{stroke,fill}%
\end{pgfscope}%
\begin{pgfscope}%
\pgfpathrectangle{\pgfqpoint{0.100000in}{0.212622in}}{\pgfqpoint{3.696000in}{3.696000in}}%
\pgfusepath{clip}%
\pgfsetbuttcap%
\pgfsetroundjoin%
\definecolor{currentfill}{rgb}{0.121569,0.466667,0.705882}%
\pgfsetfillcolor{currentfill}%
\pgfsetfillopacity{0.897386}%
\pgfsetlinewidth{1.003750pt}%
\definecolor{currentstroke}{rgb}{0.121569,0.466667,0.705882}%
\pgfsetstrokecolor{currentstroke}%
\pgfsetstrokeopacity{0.897386}%
\pgfsetdash{}{0pt}%
\pgfpathmoveto{\pgfqpoint{1.759269in}{1.556004in}}%
\pgfpathcurveto{\pgfqpoint{1.767505in}{1.556004in}}{\pgfqpoint{1.775405in}{1.559277in}}{\pgfqpoint{1.781229in}{1.565101in}}%
\pgfpathcurveto{\pgfqpoint{1.787053in}{1.570925in}}{\pgfqpoint{1.790325in}{1.578825in}}{\pgfqpoint{1.790325in}{1.587061in}}%
\pgfpathcurveto{\pgfqpoint{1.790325in}{1.595297in}}{\pgfqpoint{1.787053in}{1.603197in}}{\pgfqpoint{1.781229in}{1.609021in}}%
\pgfpathcurveto{\pgfqpoint{1.775405in}{1.614845in}}{\pgfqpoint{1.767505in}{1.618117in}}{\pgfqpoint{1.759269in}{1.618117in}}%
\pgfpathcurveto{\pgfqpoint{1.751033in}{1.618117in}}{\pgfqpoint{1.743132in}{1.614845in}}{\pgfqpoint{1.737309in}{1.609021in}}%
\pgfpathcurveto{\pgfqpoint{1.731485in}{1.603197in}}{\pgfqpoint{1.728212in}{1.595297in}}{\pgfqpoint{1.728212in}{1.587061in}}%
\pgfpathcurveto{\pgfqpoint{1.728212in}{1.578825in}}{\pgfqpoint{1.731485in}{1.570925in}}{\pgfqpoint{1.737309in}{1.565101in}}%
\pgfpathcurveto{\pgfqpoint{1.743132in}{1.559277in}}{\pgfqpoint{1.751033in}{1.556004in}}{\pgfqpoint{1.759269in}{1.556004in}}%
\pgfpathclose%
\pgfusepath{stroke,fill}%
\end{pgfscope}%
\begin{pgfscope}%
\pgfpathrectangle{\pgfqpoint{0.100000in}{0.212622in}}{\pgfqpoint{3.696000in}{3.696000in}}%
\pgfusepath{clip}%
\pgfsetbuttcap%
\pgfsetroundjoin%
\definecolor{currentfill}{rgb}{0.121569,0.466667,0.705882}%
\pgfsetfillcolor{currentfill}%
\pgfsetfillopacity{0.898489}%
\pgfsetlinewidth{1.003750pt}%
\definecolor{currentstroke}{rgb}{0.121569,0.466667,0.705882}%
\pgfsetstrokecolor{currentstroke}%
\pgfsetstrokeopacity{0.898489}%
\pgfsetdash{}{0pt}%
\pgfpathmoveto{\pgfqpoint{1.759635in}{1.556452in}}%
\pgfpathcurveto{\pgfqpoint{1.767872in}{1.556452in}}{\pgfqpoint{1.775772in}{1.559724in}}{\pgfqpoint{1.781596in}{1.565548in}}%
\pgfpathcurveto{\pgfqpoint{1.787420in}{1.571372in}}{\pgfqpoint{1.790692in}{1.579272in}}{\pgfqpoint{1.790692in}{1.587508in}}%
\pgfpathcurveto{\pgfqpoint{1.790692in}{1.595745in}}{\pgfqpoint{1.787420in}{1.603645in}}{\pgfqpoint{1.781596in}{1.609469in}}%
\pgfpathcurveto{\pgfqpoint{1.775772in}{1.615293in}}{\pgfqpoint{1.767872in}{1.618565in}}{\pgfqpoint{1.759635in}{1.618565in}}%
\pgfpathcurveto{\pgfqpoint{1.751399in}{1.618565in}}{\pgfqpoint{1.743499in}{1.615293in}}{\pgfqpoint{1.737675in}{1.609469in}}%
\pgfpathcurveto{\pgfqpoint{1.731851in}{1.603645in}}{\pgfqpoint{1.728579in}{1.595745in}}{\pgfqpoint{1.728579in}{1.587508in}}%
\pgfpathcurveto{\pgfqpoint{1.728579in}{1.579272in}}{\pgfqpoint{1.731851in}{1.571372in}}{\pgfqpoint{1.737675in}{1.565548in}}%
\pgfpathcurveto{\pgfqpoint{1.743499in}{1.559724in}}{\pgfqpoint{1.751399in}{1.556452in}}{\pgfqpoint{1.759635in}{1.556452in}}%
\pgfpathclose%
\pgfusepath{stroke,fill}%
\end{pgfscope}%
\begin{pgfscope}%
\pgfpathrectangle{\pgfqpoint{0.100000in}{0.212622in}}{\pgfqpoint{3.696000in}{3.696000in}}%
\pgfusepath{clip}%
\pgfsetbuttcap%
\pgfsetroundjoin%
\definecolor{currentfill}{rgb}{0.121569,0.466667,0.705882}%
\pgfsetfillcolor{currentfill}%
\pgfsetfillopacity{0.898867}%
\pgfsetlinewidth{1.003750pt}%
\definecolor{currentstroke}{rgb}{0.121569,0.466667,0.705882}%
\pgfsetstrokecolor{currentstroke}%
\pgfsetstrokeopacity{0.898867}%
\pgfsetdash{}{0pt}%
\pgfpathmoveto{\pgfqpoint{1.759918in}{1.555725in}}%
\pgfpathcurveto{\pgfqpoint{1.768155in}{1.555725in}}{\pgfqpoint{1.776055in}{1.558997in}}{\pgfqpoint{1.781879in}{1.564821in}}%
\pgfpathcurveto{\pgfqpoint{1.787703in}{1.570645in}}{\pgfqpoint{1.790975in}{1.578545in}}{\pgfqpoint{1.790975in}{1.586781in}}%
\pgfpathcurveto{\pgfqpoint{1.790975in}{1.595018in}}{\pgfqpoint{1.787703in}{1.602918in}}{\pgfqpoint{1.781879in}{1.608742in}}%
\pgfpathcurveto{\pgfqpoint{1.776055in}{1.614565in}}{\pgfqpoint{1.768155in}{1.617838in}}{\pgfqpoint{1.759918in}{1.617838in}}%
\pgfpathcurveto{\pgfqpoint{1.751682in}{1.617838in}}{\pgfqpoint{1.743782in}{1.614565in}}{\pgfqpoint{1.737958in}{1.608742in}}%
\pgfpathcurveto{\pgfqpoint{1.732134in}{1.602918in}}{\pgfqpoint{1.728862in}{1.595018in}}{\pgfqpoint{1.728862in}{1.586781in}}%
\pgfpathcurveto{\pgfqpoint{1.728862in}{1.578545in}}{\pgfqpoint{1.732134in}{1.570645in}}{\pgfqpoint{1.737958in}{1.564821in}}%
\pgfpathcurveto{\pgfqpoint{1.743782in}{1.558997in}}{\pgfqpoint{1.751682in}{1.555725in}}{\pgfqpoint{1.759918in}{1.555725in}}%
\pgfpathclose%
\pgfusepath{stroke,fill}%
\end{pgfscope}%
\begin{pgfscope}%
\pgfpathrectangle{\pgfqpoint{0.100000in}{0.212622in}}{\pgfqpoint{3.696000in}{3.696000in}}%
\pgfusepath{clip}%
\pgfsetbuttcap%
\pgfsetroundjoin%
\definecolor{currentfill}{rgb}{0.121569,0.466667,0.705882}%
\pgfsetfillcolor{currentfill}%
\pgfsetfillopacity{0.899231}%
\pgfsetlinewidth{1.003750pt}%
\definecolor{currentstroke}{rgb}{0.121569,0.466667,0.705882}%
\pgfsetstrokecolor{currentstroke}%
\pgfsetstrokeopacity{0.899231}%
\pgfsetdash{}{0pt}%
\pgfpathmoveto{\pgfqpoint{1.760409in}{1.554202in}}%
\pgfpathcurveto{\pgfqpoint{1.768645in}{1.554202in}}{\pgfqpoint{1.776545in}{1.557474in}}{\pgfqpoint{1.782369in}{1.563298in}}%
\pgfpathcurveto{\pgfqpoint{1.788193in}{1.569122in}}{\pgfqpoint{1.791466in}{1.577022in}}{\pgfqpoint{1.791466in}{1.585259in}}%
\pgfpathcurveto{\pgfqpoint{1.791466in}{1.593495in}}{\pgfqpoint{1.788193in}{1.601395in}}{\pgfqpoint{1.782369in}{1.607219in}}%
\pgfpathcurveto{\pgfqpoint{1.776545in}{1.613043in}}{\pgfqpoint{1.768645in}{1.616315in}}{\pgfqpoint{1.760409in}{1.616315in}}%
\pgfpathcurveto{\pgfqpoint{1.752173in}{1.616315in}}{\pgfqpoint{1.744273in}{1.613043in}}{\pgfqpoint{1.738449in}{1.607219in}}%
\pgfpathcurveto{\pgfqpoint{1.732625in}{1.601395in}}{\pgfqpoint{1.729353in}{1.593495in}}{\pgfqpoint{1.729353in}{1.585259in}}%
\pgfpathcurveto{\pgfqpoint{1.729353in}{1.577022in}}{\pgfqpoint{1.732625in}{1.569122in}}{\pgfqpoint{1.738449in}{1.563298in}}%
\pgfpathcurveto{\pgfqpoint{1.744273in}{1.557474in}}{\pgfqpoint{1.752173in}{1.554202in}}{\pgfqpoint{1.760409in}{1.554202in}}%
\pgfpathclose%
\pgfusepath{stroke,fill}%
\end{pgfscope}%
\begin{pgfscope}%
\pgfpathrectangle{\pgfqpoint{0.100000in}{0.212622in}}{\pgfqpoint{3.696000in}{3.696000in}}%
\pgfusepath{clip}%
\pgfsetbuttcap%
\pgfsetroundjoin%
\definecolor{currentfill}{rgb}{0.121569,0.466667,0.705882}%
\pgfsetfillcolor{currentfill}%
\pgfsetfillopacity{0.900517}%
\pgfsetlinewidth{1.003750pt}%
\definecolor{currentstroke}{rgb}{0.121569,0.466667,0.705882}%
\pgfsetstrokecolor{currentstroke}%
\pgfsetstrokeopacity{0.900517}%
\pgfsetdash{}{0pt}%
\pgfpathmoveto{\pgfqpoint{1.760975in}{1.554466in}}%
\pgfpathcurveto{\pgfqpoint{1.769212in}{1.554466in}}{\pgfqpoint{1.777112in}{1.557738in}}{\pgfqpoint{1.782936in}{1.563562in}}%
\pgfpathcurveto{\pgfqpoint{1.788759in}{1.569386in}}{\pgfqpoint{1.792032in}{1.577286in}}{\pgfqpoint{1.792032in}{1.585522in}}%
\pgfpathcurveto{\pgfqpoint{1.792032in}{1.593759in}}{\pgfqpoint{1.788759in}{1.601659in}}{\pgfqpoint{1.782936in}{1.607483in}}%
\pgfpathcurveto{\pgfqpoint{1.777112in}{1.613307in}}{\pgfqpoint{1.769212in}{1.616579in}}{\pgfqpoint{1.760975in}{1.616579in}}%
\pgfpathcurveto{\pgfqpoint{1.752739in}{1.616579in}}{\pgfqpoint{1.744839in}{1.613307in}}{\pgfqpoint{1.739015in}{1.607483in}}%
\pgfpathcurveto{\pgfqpoint{1.733191in}{1.601659in}}{\pgfqpoint{1.729919in}{1.593759in}}{\pgfqpoint{1.729919in}{1.585522in}}%
\pgfpathcurveto{\pgfqpoint{1.729919in}{1.577286in}}{\pgfqpoint{1.733191in}{1.569386in}}{\pgfqpoint{1.739015in}{1.563562in}}%
\pgfpathcurveto{\pgfqpoint{1.744839in}{1.557738in}}{\pgfqpoint{1.752739in}{1.554466in}}{\pgfqpoint{1.760975in}{1.554466in}}%
\pgfpathclose%
\pgfusepath{stroke,fill}%
\end{pgfscope}%
\begin{pgfscope}%
\pgfpathrectangle{\pgfqpoint{0.100000in}{0.212622in}}{\pgfqpoint{3.696000in}{3.696000in}}%
\pgfusepath{clip}%
\pgfsetbuttcap%
\pgfsetroundjoin%
\definecolor{currentfill}{rgb}{0.121569,0.466667,0.705882}%
\pgfsetfillcolor{currentfill}%
\pgfsetfillopacity{0.901093}%
\pgfsetlinewidth{1.003750pt}%
\definecolor{currentstroke}{rgb}{0.121569,0.466667,0.705882}%
\pgfsetstrokecolor{currentstroke}%
\pgfsetstrokeopacity{0.901093}%
\pgfsetdash{}{0pt}%
\pgfpathmoveto{\pgfqpoint{0.782277in}{2.409763in}}%
\pgfpathcurveto{\pgfqpoint{0.790513in}{2.409763in}}{\pgfqpoint{0.798413in}{2.413035in}}{\pgfqpoint{0.804237in}{2.418859in}}%
\pgfpathcurveto{\pgfqpoint{0.810061in}{2.424683in}}{\pgfqpoint{0.813333in}{2.432583in}}{\pgfqpoint{0.813333in}{2.440819in}}%
\pgfpathcurveto{\pgfqpoint{0.813333in}{2.449055in}}{\pgfqpoint{0.810061in}{2.456955in}}{\pgfqpoint{0.804237in}{2.462779in}}%
\pgfpathcurveto{\pgfqpoint{0.798413in}{2.468603in}}{\pgfqpoint{0.790513in}{2.471876in}}{\pgfqpoint{0.782277in}{2.471876in}}%
\pgfpathcurveto{\pgfqpoint{0.774041in}{2.471876in}}{\pgfqpoint{0.766141in}{2.468603in}}{\pgfqpoint{0.760317in}{2.462779in}}%
\pgfpathcurveto{\pgfqpoint{0.754493in}{2.456955in}}{\pgfqpoint{0.751220in}{2.449055in}}{\pgfqpoint{0.751220in}{2.440819in}}%
\pgfpathcurveto{\pgfqpoint{0.751220in}{2.432583in}}{\pgfqpoint{0.754493in}{2.424683in}}{\pgfqpoint{0.760317in}{2.418859in}}%
\pgfpathcurveto{\pgfqpoint{0.766141in}{2.413035in}}{\pgfqpoint{0.774041in}{2.409763in}}{\pgfqpoint{0.782277in}{2.409763in}}%
\pgfpathclose%
\pgfusepath{stroke,fill}%
\end{pgfscope}%
\begin{pgfscope}%
\pgfpathrectangle{\pgfqpoint{0.100000in}{0.212622in}}{\pgfqpoint{3.696000in}{3.696000in}}%
\pgfusepath{clip}%
\pgfsetbuttcap%
\pgfsetroundjoin%
\definecolor{currentfill}{rgb}{0.121569,0.466667,0.705882}%
\pgfsetfillcolor{currentfill}%
\pgfsetfillopacity{0.901554}%
\pgfsetlinewidth{1.003750pt}%
\definecolor{currentstroke}{rgb}{0.121569,0.466667,0.705882}%
\pgfsetstrokecolor{currentstroke}%
\pgfsetstrokeopacity{0.901554}%
\pgfsetdash{}{0pt}%
\pgfpathmoveto{\pgfqpoint{0.804043in}{2.371828in}}%
\pgfpathcurveto{\pgfqpoint{0.812280in}{2.371828in}}{\pgfqpoint{0.820180in}{2.375101in}}{\pgfqpoint{0.826004in}{2.380925in}}%
\pgfpathcurveto{\pgfqpoint{0.831827in}{2.386749in}}{\pgfqpoint{0.835100in}{2.394649in}}{\pgfqpoint{0.835100in}{2.402885in}}%
\pgfpathcurveto{\pgfqpoint{0.835100in}{2.411121in}}{\pgfqpoint{0.831827in}{2.419021in}}{\pgfqpoint{0.826004in}{2.424845in}}%
\pgfpathcurveto{\pgfqpoint{0.820180in}{2.430669in}}{\pgfqpoint{0.812280in}{2.433941in}}{\pgfqpoint{0.804043in}{2.433941in}}%
\pgfpathcurveto{\pgfqpoint{0.795807in}{2.433941in}}{\pgfqpoint{0.787907in}{2.430669in}}{\pgfqpoint{0.782083in}{2.424845in}}%
\pgfpathcurveto{\pgfqpoint{0.776259in}{2.419021in}}{\pgfqpoint{0.772987in}{2.411121in}}{\pgfqpoint{0.772987in}{2.402885in}}%
\pgfpathcurveto{\pgfqpoint{0.772987in}{2.394649in}}{\pgfqpoint{0.776259in}{2.386749in}}{\pgfqpoint{0.782083in}{2.380925in}}%
\pgfpathcurveto{\pgfqpoint{0.787907in}{2.375101in}}{\pgfqpoint{0.795807in}{2.371828in}}{\pgfqpoint{0.804043in}{2.371828in}}%
\pgfpathclose%
\pgfusepath{stroke,fill}%
\end{pgfscope}%
\begin{pgfscope}%
\pgfpathrectangle{\pgfqpoint{0.100000in}{0.212622in}}{\pgfqpoint{3.696000in}{3.696000in}}%
\pgfusepath{clip}%
\pgfsetbuttcap%
\pgfsetroundjoin%
\definecolor{currentfill}{rgb}{0.121569,0.466667,0.705882}%
\pgfsetfillcolor{currentfill}%
\pgfsetfillopacity{0.902645}%
\pgfsetlinewidth{1.003750pt}%
\definecolor{currentstroke}{rgb}{0.121569,0.466667,0.705882}%
\pgfsetstrokecolor{currentstroke}%
\pgfsetstrokeopacity{0.902645}%
\pgfsetdash{}{0pt}%
\pgfpathmoveto{\pgfqpoint{1.762606in}{1.555052in}}%
\pgfpathcurveto{\pgfqpoint{1.770842in}{1.555052in}}{\pgfqpoint{1.778742in}{1.558324in}}{\pgfqpoint{1.784566in}{1.564148in}}%
\pgfpathcurveto{\pgfqpoint{1.790390in}{1.569972in}}{\pgfqpoint{1.793663in}{1.577872in}}{\pgfqpoint{1.793663in}{1.586108in}}%
\pgfpathcurveto{\pgfqpoint{1.793663in}{1.594344in}}{\pgfqpoint{1.790390in}{1.602244in}}{\pgfqpoint{1.784566in}{1.608068in}}%
\pgfpathcurveto{\pgfqpoint{1.778742in}{1.613892in}}{\pgfqpoint{1.770842in}{1.617165in}}{\pgfqpoint{1.762606in}{1.617165in}}%
\pgfpathcurveto{\pgfqpoint{1.754370in}{1.617165in}}{\pgfqpoint{1.746470in}{1.613892in}}{\pgfqpoint{1.740646in}{1.608068in}}%
\pgfpathcurveto{\pgfqpoint{1.734822in}{1.602244in}}{\pgfqpoint{1.731550in}{1.594344in}}{\pgfqpoint{1.731550in}{1.586108in}}%
\pgfpathcurveto{\pgfqpoint{1.731550in}{1.577872in}}{\pgfqpoint{1.734822in}{1.569972in}}{\pgfqpoint{1.740646in}{1.564148in}}%
\pgfpathcurveto{\pgfqpoint{1.746470in}{1.558324in}}{\pgfqpoint{1.754370in}{1.555052in}}{\pgfqpoint{1.762606in}{1.555052in}}%
\pgfpathclose%
\pgfusepath{stroke,fill}%
\end{pgfscope}%
\begin{pgfscope}%
\pgfpathrectangle{\pgfqpoint{0.100000in}{0.212622in}}{\pgfqpoint{3.696000in}{3.696000in}}%
\pgfusepath{clip}%
\pgfsetbuttcap%
\pgfsetroundjoin%
\definecolor{currentfill}{rgb}{0.121569,0.466667,0.705882}%
\pgfsetfillcolor{currentfill}%
\pgfsetfillopacity{0.904007}%
\pgfsetlinewidth{1.003750pt}%
\definecolor{currentstroke}{rgb}{0.121569,0.466667,0.705882}%
\pgfsetstrokecolor{currentstroke}%
\pgfsetstrokeopacity{0.904007}%
\pgfsetdash{}{0pt}%
\pgfpathmoveto{\pgfqpoint{0.826315in}{2.353968in}}%
\pgfpathcurveto{\pgfqpoint{0.834551in}{2.353968in}}{\pgfqpoint{0.842451in}{2.357240in}}{\pgfqpoint{0.848275in}{2.363064in}}%
\pgfpathcurveto{\pgfqpoint{0.854099in}{2.368888in}}{\pgfqpoint{0.857372in}{2.376788in}}{\pgfqpoint{0.857372in}{2.385025in}}%
\pgfpathcurveto{\pgfqpoint{0.857372in}{2.393261in}}{\pgfqpoint{0.854099in}{2.401161in}}{\pgfqpoint{0.848275in}{2.406985in}}%
\pgfpathcurveto{\pgfqpoint{0.842451in}{2.412809in}}{\pgfqpoint{0.834551in}{2.416081in}}{\pgfqpoint{0.826315in}{2.416081in}}%
\pgfpathcurveto{\pgfqpoint{0.818079in}{2.416081in}}{\pgfqpoint{0.810179in}{2.412809in}}{\pgfqpoint{0.804355in}{2.406985in}}%
\pgfpathcurveto{\pgfqpoint{0.798531in}{2.401161in}}{\pgfqpoint{0.795259in}{2.393261in}}{\pgfqpoint{0.795259in}{2.385025in}}%
\pgfpathcurveto{\pgfqpoint{0.795259in}{2.376788in}}{\pgfqpoint{0.798531in}{2.368888in}}{\pgfqpoint{0.804355in}{2.363064in}}%
\pgfpathcurveto{\pgfqpoint{0.810179in}{2.357240in}}{\pgfqpoint{0.818079in}{2.353968in}}{\pgfqpoint{0.826315in}{2.353968in}}%
\pgfpathclose%
\pgfusepath{stroke,fill}%
\end{pgfscope}%
\begin{pgfscope}%
\pgfpathrectangle{\pgfqpoint{0.100000in}{0.212622in}}{\pgfqpoint{3.696000in}{3.696000in}}%
\pgfusepath{clip}%
\pgfsetbuttcap%
\pgfsetroundjoin%
\definecolor{currentfill}{rgb}{0.121569,0.466667,0.705882}%
\pgfsetfillcolor{currentfill}%
\pgfsetfillopacity{0.904537}%
\pgfsetlinewidth{1.003750pt}%
\definecolor{currentstroke}{rgb}{0.121569,0.466667,0.705882}%
\pgfsetstrokecolor{currentstroke}%
\pgfsetstrokeopacity{0.904537}%
\pgfsetdash{}{0pt}%
\pgfpathmoveto{\pgfqpoint{1.763410in}{1.553381in}}%
\pgfpathcurveto{\pgfqpoint{1.771646in}{1.553381in}}{\pgfqpoint{1.779546in}{1.556653in}}{\pgfqpoint{1.785370in}{1.562477in}}%
\pgfpathcurveto{\pgfqpoint{1.791194in}{1.568301in}}{\pgfqpoint{1.794466in}{1.576201in}}{\pgfqpoint{1.794466in}{1.584437in}}%
\pgfpathcurveto{\pgfqpoint{1.794466in}{1.592674in}}{\pgfqpoint{1.791194in}{1.600574in}}{\pgfqpoint{1.785370in}{1.606398in}}%
\pgfpathcurveto{\pgfqpoint{1.779546in}{1.612222in}}{\pgfqpoint{1.771646in}{1.615494in}}{\pgfqpoint{1.763410in}{1.615494in}}%
\pgfpathcurveto{\pgfqpoint{1.755173in}{1.615494in}}{\pgfqpoint{1.747273in}{1.612222in}}{\pgfqpoint{1.741449in}{1.606398in}}%
\pgfpathcurveto{\pgfqpoint{1.735625in}{1.600574in}}{\pgfqpoint{1.732353in}{1.592674in}}{\pgfqpoint{1.732353in}{1.584437in}}%
\pgfpathcurveto{\pgfqpoint{1.732353in}{1.576201in}}{\pgfqpoint{1.735625in}{1.568301in}}{\pgfqpoint{1.741449in}{1.562477in}}%
\pgfpathcurveto{\pgfqpoint{1.747273in}{1.556653in}}{\pgfqpoint{1.755173in}{1.553381in}}{\pgfqpoint{1.763410in}{1.553381in}}%
\pgfpathclose%
\pgfusepath{stroke,fill}%
\end{pgfscope}%
\begin{pgfscope}%
\pgfpathrectangle{\pgfqpoint{0.100000in}{0.212622in}}{\pgfqpoint{3.696000in}{3.696000in}}%
\pgfusepath{clip}%
\pgfsetbuttcap%
\pgfsetroundjoin%
\definecolor{currentfill}{rgb}{0.121569,0.466667,0.705882}%
\pgfsetfillcolor{currentfill}%
\pgfsetfillopacity{0.904754}%
\pgfsetlinewidth{1.003750pt}%
\definecolor{currentstroke}{rgb}{0.121569,0.466667,0.705882}%
\pgfsetstrokecolor{currentstroke}%
\pgfsetstrokeopacity{0.904754}%
\pgfsetdash{}{0pt}%
\pgfpathmoveto{\pgfqpoint{0.867404in}{2.304855in}}%
\pgfpathcurveto{\pgfqpoint{0.875641in}{2.304855in}}{\pgfqpoint{0.883541in}{2.308127in}}{\pgfqpoint{0.889365in}{2.313951in}}%
\pgfpathcurveto{\pgfqpoint{0.895189in}{2.319775in}}{\pgfqpoint{0.898461in}{2.327675in}}{\pgfqpoint{0.898461in}{2.335911in}}%
\pgfpathcurveto{\pgfqpoint{0.898461in}{2.344148in}}{\pgfqpoint{0.895189in}{2.352048in}}{\pgfqpoint{0.889365in}{2.357872in}}%
\pgfpathcurveto{\pgfqpoint{0.883541in}{2.363695in}}{\pgfqpoint{0.875641in}{2.366968in}}{\pgfqpoint{0.867404in}{2.366968in}}%
\pgfpathcurveto{\pgfqpoint{0.859168in}{2.366968in}}{\pgfqpoint{0.851268in}{2.363695in}}{\pgfqpoint{0.845444in}{2.357872in}}%
\pgfpathcurveto{\pgfqpoint{0.839620in}{2.352048in}}{\pgfqpoint{0.836348in}{2.344148in}}{\pgfqpoint{0.836348in}{2.335911in}}%
\pgfpathcurveto{\pgfqpoint{0.836348in}{2.327675in}}{\pgfqpoint{0.839620in}{2.319775in}}{\pgfqpoint{0.845444in}{2.313951in}}%
\pgfpathcurveto{\pgfqpoint{0.851268in}{2.308127in}}{\pgfqpoint{0.859168in}{2.304855in}}{\pgfqpoint{0.867404in}{2.304855in}}%
\pgfpathclose%
\pgfusepath{stroke,fill}%
\end{pgfscope}%
\begin{pgfscope}%
\pgfpathrectangle{\pgfqpoint{0.100000in}{0.212622in}}{\pgfqpoint{3.696000in}{3.696000in}}%
\pgfusepath{clip}%
\pgfsetbuttcap%
\pgfsetroundjoin%
\definecolor{currentfill}{rgb}{0.121569,0.466667,0.705882}%
\pgfsetfillcolor{currentfill}%
\pgfsetfillopacity{0.906444}%
\pgfsetlinewidth{1.003750pt}%
\definecolor{currentstroke}{rgb}{0.121569,0.466667,0.705882}%
\pgfsetstrokecolor{currentstroke}%
\pgfsetstrokeopacity{0.906444}%
\pgfsetdash{}{0pt}%
\pgfpathmoveto{\pgfqpoint{1.764790in}{1.550426in}}%
\pgfpathcurveto{\pgfqpoint{1.773026in}{1.550426in}}{\pgfqpoint{1.780926in}{1.553698in}}{\pgfqpoint{1.786750in}{1.559522in}}%
\pgfpathcurveto{\pgfqpoint{1.792574in}{1.565346in}}{\pgfqpoint{1.795846in}{1.573246in}}{\pgfqpoint{1.795846in}{1.581482in}}%
\pgfpathcurveto{\pgfqpoint{1.795846in}{1.589719in}}{\pgfqpoint{1.792574in}{1.597619in}}{\pgfqpoint{1.786750in}{1.603443in}}%
\pgfpathcurveto{\pgfqpoint{1.780926in}{1.609267in}}{\pgfqpoint{1.773026in}{1.612539in}}{\pgfqpoint{1.764790in}{1.612539in}}%
\pgfpathcurveto{\pgfqpoint{1.756554in}{1.612539in}}{\pgfqpoint{1.748654in}{1.609267in}}{\pgfqpoint{1.742830in}{1.603443in}}%
\pgfpathcurveto{\pgfqpoint{1.737006in}{1.597619in}}{\pgfqpoint{1.733733in}{1.589719in}}{\pgfqpoint{1.733733in}{1.581482in}}%
\pgfpathcurveto{\pgfqpoint{1.733733in}{1.573246in}}{\pgfqpoint{1.737006in}{1.565346in}}{\pgfqpoint{1.742830in}{1.559522in}}%
\pgfpathcurveto{\pgfqpoint{1.748654in}{1.553698in}}{\pgfqpoint{1.756554in}{1.550426in}}{\pgfqpoint{1.764790in}{1.550426in}}%
\pgfpathclose%
\pgfusepath{stroke,fill}%
\end{pgfscope}%
\begin{pgfscope}%
\pgfpathrectangle{\pgfqpoint{0.100000in}{0.212622in}}{\pgfqpoint{3.696000in}{3.696000in}}%
\pgfusepath{clip}%
\pgfsetbuttcap%
\pgfsetroundjoin%
\definecolor{currentfill}{rgb}{0.121569,0.466667,0.705882}%
\pgfsetfillcolor{currentfill}%
\pgfsetfillopacity{0.907341}%
\pgfsetlinewidth{1.003750pt}%
\definecolor{currentstroke}{rgb}{0.121569,0.466667,0.705882}%
\pgfsetstrokecolor{currentstroke}%
\pgfsetstrokeopacity{0.907341}%
\pgfsetdash{}{0pt}%
\pgfpathmoveto{\pgfqpoint{0.904909in}{2.264859in}}%
\pgfpathcurveto{\pgfqpoint{0.913145in}{2.264859in}}{\pgfqpoint{0.921045in}{2.268131in}}{\pgfqpoint{0.926869in}{2.273955in}}%
\pgfpathcurveto{\pgfqpoint{0.932693in}{2.279779in}}{\pgfqpoint{0.935965in}{2.287679in}}{\pgfqpoint{0.935965in}{2.295915in}}%
\pgfpathcurveto{\pgfqpoint{0.935965in}{2.304151in}}{\pgfqpoint{0.932693in}{2.312051in}}{\pgfqpoint{0.926869in}{2.317875in}}%
\pgfpathcurveto{\pgfqpoint{0.921045in}{2.323699in}}{\pgfqpoint{0.913145in}{2.326972in}}{\pgfqpoint{0.904909in}{2.326972in}}%
\pgfpathcurveto{\pgfqpoint{0.896673in}{2.326972in}}{\pgfqpoint{0.888773in}{2.323699in}}{\pgfqpoint{0.882949in}{2.317875in}}%
\pgfpathcurveto{\pgfqpoint{0.877125in}{2.312051in}}{\pgfqpoint{0.873852in}{2.304151in}}{\pgfqpoint{0.873852in}{2.295915in}}%
\pgfpathcurveto{\pgfqpoint{0.873852in}{2.287679in}}{\pgfqpoint{0.877125in}{2.279779in}}{\pgfqpoint{0.882949in}{2.273955in}}%
\pgfpathcurveto{\pgfqpoint{0.888773in}{2.268131in}}{\pgfqpoint{0.896673in}{2.264859in}}{\pgfqpoint{0.904909in}{2.264859in}}%
\pgfpathclose%
\pgfusepath{stroke,fill}%
\end{pgfscope}%
\begin{pgfscope}%
\pgfpathrectangle{\pgfqpoint{0.100000in}{0.212622in}}{\pgfqpoint{3.696000in}{3.696000in}}%
\pgfusepath{clip}%
\pgfsetbuttcap%
\pgfsetroundjoin%
\definecolor{currentfill}{rgb}{0.121569,0.466667,0.705882}%
\pgfsetfillcolor{currentfill}%
\pgfsetfillopacity{0.908878}%
\pgfsetlinewidth{1.003750pt}%
\definecolor{currentstroke}{rgb}{0.121569,0.466667,0.705882}%
\pgfsetstrokecolor{currentstroke}%
\pgfsetstrokeopacity{0.908878}%
\pgfsetdash{}{0pt}%
\pgfpathmoveto{\pgfqpoint{1.766113in}{1.548621in}}%
\pgfpathcurveto{\pgfqpoint{1.774349in}{1.548621in}}{\pgfqpoint{1.782249in}{1.551893in}}{\pgfqpoint{1.788073in}{1.557717in}}%
\pgfpathcurveto{\pgfqpoint{1.793897in}{1.563541in}}{\pgfqpoint{1.797170in}{1.571441in}}{\pgfqpoint{1.797170in}{1.579677in}}%
\pgfpathcurveto{\pgfqpoint{1.797170in}{1.587914in}}{\pgfqpoint{1.793897in}{1.595814in}}{\pgfqpoint{1.788073in}{1.601638in}}%
\pgfpathcurveto{\pgfqpoint{1.782249in}{1.607462in}}{\pgfqpoint{1.774349in}{1.610734in}}{\pgfqpoint{1.766113in}{1.610734in}}%
\pgfpathcurveto{\pgfqpoint{1.757877in}{1.610734in}}{\pgfqpoint{1.749977in}{1.607462in}}{\pgfqpoint{1.744153in}{1.601638in}}%
\pgfpathcurveto{\pgfqpoint{1.738329in}{1.595814in}}{\pgfqpoint{1.735057in}{1.587914in}}{\pgfqpoint{1.735057in}{1.579677in}}%
\pgfpathcurveto{\pgfqpoint{1.735057in}{1.571441in}}{\pgfqpoint{1.738329in}{1.563541in}}{\pgfqpoint{1.744153in}{1.557717in}}%
\pgfpathcurveto{\pgfqpoint{1.749977in}{1.551893in}}{\pgfqpoint{1.757877in}{1.548621in}}{\pgfqpoint{1.766113in}{1.548621in}}%
\pgfpathclose%
\pgfusepath{stroke,fill}%
\end{pgfscope}%
\begin{pgfscope}%
\pgfpathrectangle{\pgfqpoint{0.100000in}{0.212622in}}{\pgfqpoint{3.696000in}{3.696000in}}%
\pgfusepath{clip}%
\pgfsetbuttcap%
\pgfsetroundjoin%
\definecolor{currentfill}{rgb}{0.121569,0.466667,0.705882}%
\pgfsetfillcolor{currentfill}%
\pgfsetfillopacity{0.910510}%
\pgfsetlinewidth{1.003750pt}%
\definecolor{currentstroke}{rgb}{0.121569,0.466667,0.705882}%
\pgfsetstrokecolor{currentstroke}%
\pgfsetstrokeopacity{0.910510}%
\pgfsetdash{}{0pt}%
\pgfpathmoveto{\pgfqpoint{1.767198in}{1.549120in}}%
\pgfpathcurveto{\pgfqpoint{1.775435in}{1.549120in}}{\pgfqpoint{1.783335in}{1.552392in}}{\pgfqpoint{1.789159in}{1.558216in}}%
\pgfpathcurveto{\pgfqpoint{1.794983in}{1.564040in}}{\pgfqpoint{1.798255in}{1.571940in}}{\pgfqpoint{1.798255in}{1.580176in}}%
\pgfpathcurveto{\pgfqpoint{1.798255in}{1.588413in}}{\pgfqpoint{1.794983in}{1.596313in}}{\pgfqpoint{1.789159in}{1.602137in}}%
\pgfpathcurveto{\pgfqpoint{1.783335in}{1.607961in}}{\pgfqpoint{1.775435in}{1.611233in}}{\pgfqpoint{1.767198in}{1.611233in}}%
\pgfpathcurveto{\pgfqpoint{1.758962in}{1.611233in}}{\pgfqpoint{1.751062in}{1.607961in}}{\pgfqpoint{1.745238in}{1.602137in}}%
\pgfpathcurveto{\pgfqpoint{1.739414in}{1.596313in}}{\pgfqpoint{1.736142in}{1.588413in}}{\pgfqpoint{1.736142in}{1.580176in}}%
\pgfpathcurveto{\pgfqpoint{1.736142in}{1.571940in}}{\pgfqpoint{1.739414in}{1.564040in}}{\pgfqpoint{1.745238in}{1.558216in}}%
\pgfpathcurveto{\pgfqpoint{1.751062in}{1.552392in}}{\pgfqpoint{1.758962in}{1.549120in}}{\pgfqpoint{1.767198in}{1.549120in}}%
\pgfpathclose%
\pgfusepath{stroke,fill}%
\end{pgfscope}%
\begin{pgfscope}%
\pgfpathrectangle{\pgfqpoint{0.100000in}{0.212622in}}{\pgfqpoint{3.696000in}{3.696000in}}%
\pgfusepath{clip}%
\pgfsetbuttcap%
\pgfsetroundjoin%
\definecolor{currentfill}{rgb}{0.121569,0.466667,0.705882}%
\pgfsetfillcolor{currentfill}%
\pgfsetfillopacity{0.910828}%
\pgfsetlinewidth{1.003750pt}%
\definecolor{currentstroke}{rgb}{0.121569,0.466667,0.705882}%
\pgfsetstrokecolor{currentstroke}%
\pgfsetstrokeopacity{0.910828}%
\pgfsetdash{}{0pt}%
\pgfpathmoveto{\pgfqpoint{0.941400in}{2.227669in}}%
\pgfpathcurveto{\pgfqpoint{0.949637in}{2.227669in}}{\pgfqpoint{0.957537in}{2.230941in}}{\pgfqpoint{0.963361in}{2.236765in}}%
\pgfpathcurveto{\pgfqpoint{0.969185in}{2.242589in}}{\pgfqpoint{0.972457in}{2.250489in}}{\pgfqpoint{0.972457in}{2.258725in}}%
\pgfpathcurveto{\pgfqpoint{0.972457in}{2.266962in}}{\pgfqpoint{0.969185in}{2.274862in}}{\pgfqpoint{0.963361in}{2.280686in}}%
\pgfpathcurveto{\pgfqpoint{0.957537in}{2.286510in}}{\pgfqpoint{0.949637in}{2.289782in}}{\pgfqpoint{0.941400in}{2.289782in}}%
\pgfpathcurveto{\pgfqpoint{0.933164in}{2.289782in}}{\pgfqpoint{0.925264in}{2.286510in}}{\pgfqpoint{0.919440in}{2.280686in}}%
\pgfpathcurveto{\pgfqpoint{0.913616in}{2.274862in}}{\pgfqpoint{0.910344in}{2.266962in}}{\pgfqpoint{0.910344in}{2.258725in}}%
\pgfpathcurveto{\pgfqpoint{0.910344in}{2.250489in}}{\pgfqpoint{0.913616in}{2.242589in}}{\pgfqpoint{0.919440in}{2.236765in}}%
\pgfpathcurveto{\pgfqpoint{0.925264in}{2.230941in}}{\pgfqpoint{0.933164in}{2.227669in}}{\pgfqpoint{0.941400in}{2.227669in}}%
\pgfpathclose%
\pgfusepath{stroke,fill}%
\end{pgfscope}%
\begin{pgfscope}%
\pgfpathrectangle{\pgfqpoint{0.100000in}{0.212622in}}{\pgfqpoint{3.696000in}{3.696000in}}%
\pgfusepath{clip}%
\pgfsetbuttcap%
\pgfsetroundjoin%
\definecolor{currentfill}{rgb}{0.121569,0.466667,0.705882}%
\pgfsetfillcolor{currentfill}%
\pgfsetfillopacity{0.912119}%
\pgfsetlinewidth{1.003750pt}%
\definecolor{currentstroke}{rgb}{0.121569,0.466667,0.705882}%
\pgfsetstrokecolor{currentstroke}%
\pgfsetstrokeopacity{0.912119}%
\pgfsetdash{}{0pt}%
\pgfpathmoveto{\pgfqpoint{1.768025in}{1.548127in}}%
\pgfpathcurveto{\pgfqpoint{1.776261in}{1.548127in}}{\pgfqpoint{1.784161in}{1.551399in}}{\pgfqpoint{1.789985in}{1.557223in}}%
\pgfpathcurveto{\pgfqpoint{1.795809in}{1.563047in}}{\pgfqpoint{1.799081in}{1.570947in}}{\pgfqpoint{1.799081in}{1.579183in}}%
\pgfpathcurveto{\pgfqpoint{1.799081in}{1.587420in}}{\pgfqpoint{1.795809in}{1.595320in}}{\pgfqpoint{1.789985in}{1.601143in}}%
\pgfpathcurveto{\pgfqpoint{1.784161in}{1.606967in}}{\pgfqpoint{1.776261in}{1.610240in}}{\pgfqpoint{1.768025in}{1.610240in}}%
\pgfpathcurveto{\pgfqpoint{1.759788in}{1.610240in}}{\pgfqpoint{1.751888in}{1.606967in}}{\pgfqpoint{1.746064in}{1.601143in}}%
\pgfpathcurveto{\pgfqpoint{1.740240in}{1.595320in}}{\pgfqpoint{1.736968in}{1.587420in}}{\pgfqpoint{1.736968in}{1.579183in}}%
\pgfpathcurveto{\pgfqpoint{1.736968in}{1.570947in}}{\pgfqpoint{1.740240in}{1.563047in}}{\pgfqpoint{1.746064in}{1.557223in}}%
\pgfpathcurveto{\pgfqpoint{1.751888in}{1.551399in}}{\pgfqpoint{1.759788in}{1.548127in}}{\pgfqpoint{1.768025in}{1.548127in}}%
\pgfpathclose%
\pgfusepath{stroke,fill}%
\end{pgfscope}%
\begin{pgfscope}%
\pgfpathrectangle{\pgfqpoint{0.100000in}{0.212622in}}{\pgfqpoint{3.696000in}{3.696000in}}%
\pgfusepath{clip}%
\pgfsetbuttcap%
\pgfsetroundjoin%
\definecolor{currentfill}{rgb}{0.121569,0.466667,0.705882}%
\pgfsetfillcolor{currentfill}%
\pgfsetfillopacity{0.913753}%
\pgfsetlinewidth{1.003750pt}%
\definecolor{currentstroke}{rgb}{0.121569,0.466667,0.705882}%
\pgfsetstrokecolor{currentstroke}%
\pgfsetstrokeopacity{0.913753}%
\pgfsetdash{}{0pt}%
\pgfpathmoveto{\pgfqpoint{1.769140in}{1.546134in}}%
\pgfpathcurveto{\pgfqpoint{1.777376in}{1.546134in}}{\pgfqpoint{1.785276in}{1.549406in}}{\pgfqpoint{1.791100in}{1.555230in}}%
\pgfpathcurveto{\pgfqpoint{1.796924in}{1.561054in}}{\pgfqpoint{1.800196in}{1.568954in}}{\pgfqpoint{1.800196in}{1.577190in}}%
\pgfpathcurveto{\pgfqpoint{1.800196in}{1.585426in}}{\pgfqpoint{1.796924in}{1.593326in}}{\pgfqpoint{1.791100in}{1.599150in}}%
\pgfpathcurveto{\pgfqpoint{1.785276in}{1.604974in}}{\pgfqpoint{1.777376in}{1.608247in}}{\pgfqpoint{1.769140in}{1.608247in}}%
\pgfpathcurveto{\pgfqpoint{1.760903in}{1.608247in}}{\pgfqpoint{1.753003in}{1.604974in}}{\pgfqpoint{1.747179in}{1.599150in}}%
\pgfpathcurveto{\pgfqpoint{1.741355in}{1.593326in}}{\pgfqpoint{1.738083in}{1.585426in}}{\pgfqpoint{1.738083in}{1.577190in}}%
\pgfpathcurveto{\pgfqpoint{1.738083in}{1.568954in}}{\pgfqpoint{1.741355in}{1.561054in}}{\pgfqpoint{1.747179in}{1.555230in}}%
\pgfpathcurveto{\pgfqpoint{1.753003in}{1.549406in}}{\pgfqpoint{1.760903in}{1.546134in}}{\pgfqpoint{1.769140in}{1.546134in}}%
\pgfpathclose%
\pgfusepath{stroke,fill}%
\end{pgfscope}%
\begin{pgfscope}%
\pgfpathrectangle{\pgfqpoint{0.100000in}{0.212622in}}{\pgfqpoint{3.696000in}{3.696000in}}%
\pgfusepath{clip}%
\pgfsetbuttcap%
\pgfsetroundjoin%
\definecolor{currentfill}{rgb}{0.121569,0.466667,0.705882}%
\pgfsetfillcolor{currentfill}%
\pgfsetfillopacity{0.914707}%
\pgfsetlinewidth{1.003750pt}%
\definecolor{currentstroke}{rgb}{0.121569,0.466667,0.705882}%
\pgfsetstrokecolor{currentstroke}%
\pgfsetstrokeopacity{0.914707}%
\pgfsetdash{}{0pt}%
\pgfpathmoveto{\pgfqpoint{0.975593in}{2.192652in}}%
\pgfpathcurveto{\pgfqpoint{0.983830in}{2.192652in}}{\pgfqpoint{0.991730in}{2.195924in}}{\pgfqpoint{0.997554in}{2.201748in}}%
\pgfpathcurveto{\pgfqpoint{1.003377in}{2.207572in}}{\pgfqpoint{1.006650in}{2.215472in}}{\pgfqpoint{1.006650in}{2.223708in}}%
\pgfpathcurveto{\pgfqpoint{1.006650in}{2.231944in}}{\pgfqpoint{1.003377in}{2.239845in}}{\pgfqpoint{0.997554in}{2.245668in}}%
\pgfpathcurveto{\pgfqpoint{0.991730in}{2.251492in}}{\pgfqpoint{0.983830in}{2.254765in}}{\pgfqpoint{0.975593in}{2.254765in}}%
\pgfpathcurveto{\pgfqpoint{0.967357in}{2.254765in}}{\pgfqpoint{0.959457in}{2.251492in}}{\pgfqpoint{0.953633in}{2.245668in}}%
\pgfpathcurveto{\pgfqpoint{0.947809in}{2.239845in}}{\pgfqpoint{0.944537in}{2.231944in}}{\pgfqpoint{0.944537in}{2.223708in}}%
\pgfpathcurveto{\pgfqpoint{0.944537in}{2.215472in}}{\pgfqpoint{0.947809in}{2.207572in}}{\pgfqpoint{0.953633in}{2.201748in}}%
\pgfpathcurveto{\pgfqpoint{0.959457in}{2.195924in}}{\pgfqpoint{0.967357in}{2.192652in}}{\pgfqpoint{0.975593in}{2.192652in}}%
\pgfpathclose%
\pgfusepath{stroke,fill}%
\end{pgfscope}%
\begin{pgfscope}%
\pgfpathrectangle{\pgfqpoint{0.100000in}{0.212622in}}{\pgfqpoint{3.696000in}{3.696000in}}%
\pgfusepath{clip}%
\pgfsetbuttcap%
\pgfsetroundjoin%
\definecolor{currentfill}{rgb}{0.121569,0.466667,0.705882}%
\pgfsetfillcolor{currentfill}%
\pgfsetfillopacity{0.915095}%
\pgfsetlinewidth{1.003750pt}%
\definecolor{currentstroke}{rgb}{0.121569,0.466667,0.705882}%
\pgfsetstrokecolor{currentstroke}%
\pgfsetstrokeopacity{0.915095}%
\pgfsetdash{}{0pt}%
\pgfpathmoveto{\pgfqpoint{1.770463in}{1.541689in}}%
\pgfpathcurveto{\pgfqpoint{1.778699in}{1.541689in}}{\pgfqpoint{1.786599in}{1.544962in}}{\pgfqpoint{1.792423in}{1.550786in}}%
\pgfpathcurveto{\pgfqpoint{1.798247in}{1.556610in}}{\pgfqpoint{1.801520in}{1.564510in}}{\pgfqpoint{1.801520in}{1.572746in}}%
\pgfpathcurveto{\pgfqpoint{1.801520in}{1.580982in}}{\pgfqpoint{1.798247in}{1.588882in}}{\pgfqpoint{1.792423in}{1.594706in}}%
\pgfpathcurveto{\pgfqpoint{1.786599in}{1.600530in}}{\pgfqpoint{1.778699in}{1.603802in}}{\pgfqpoint{1.770463in}{1.603802in}}%
\pgfpathcurveto{\pgfqpoint{1.762227in}{1.603802in}}{\pgfqpoint{1.754327in}{1.600530in}}{\pgfqpoint{1.748503in}{1.594706in}}%
\pgfpathcurveto{\pgfqpoint{1.742679in}{1.588882in}}{\pgfqpoint{1.739407in}{1.580982in}}{\pgfqpoint{1.739407in}{1.572746in}}%
\pgfpathcurveto{\pgfqpoint{1.739407in}{1.564510in}}{\pgfqpoint{1.742679in}{1.556610in}}{\pgfqpoint{1.748503in}{1.550786in}}%
\pgfpathcurveto{\pgfqpoint{1.754327in}{1.544962in}}{\pgfqpoint{1.762227in}{1.541689in}}{\pgfqpoint{1.770463in}{1.541689in}}%
\pgfpathclose%
\pgfusepath{stroke,fill}%
\end{pgfscope}%
\begin{pgfscope}%
\pgfpathrectangle{\pgfqpoint{0.100000in}{0.212622in}}{\pgfqpoint{3.696000in}{3.696000in}}%
\pgfusepath{clip}%
\pgfsetbuttcap%
\pgfsetroundjoin%
\definecolor{currentfill}{rgb}{0.121569,0.466667,0.705882}%
\pgfsetfillcolor{currentfill}%
\pgfsetfillopacity{0.918133}%
\pgfsetlinewidth{1.003750pt}%
\definecolor{currentstroke}{rgb}{0.121569,0.466667,0.705882}%
\pgfsetstrokecolor{currentstroke}%
\pgfsetstrokeopacity{0.918133}%
\pgfsetdash{}{0pt}%
\pgfpathmoveto{\pgfqpoint{1.771874in}{1.541574in}}%
\pgfpathcurveto{\pgfqpoint{1.780110in}{1.541574in}}{\pgfqpoint{1.788010in}{1.544847in}}{\pgfqpoint{1.793834in}{1.550670in}}%
\pgfpathcurveto{\pgfqpoint{1.799658in}{1.556494in}}{\pgfqpoint{1.802930in}{1.564394in}}{\pgfqpoint{1.802930in}{1.572631in}}%
\pgfpathcurveto{\pgfqpoint{1.802930in}{1.580867in}}{\pgfqpoint{1.799658in}{1.588767in}}{\pgfqpoint{1.793834in}{1.594591in}}%
\pgfpathcurveto{\pgfqpoint{1.788010in}{1.600415in}}{\pgfqpoint{1.780110in}{1.603687in}}{\pgfqpoint{1.771874in}{1.603687in}}%
\pgfpathcurveto{\pgfqpoint{1.763637in}{1.603687in}}{\pgfqpoint{1.755737in}{1.600415in}}{\pgfqpoint{1.749913in}{1.594591in}}%
\pgfpathcurveto{\pgfqpoint{1.744089in}{1.588767in}}{\pgfqpoint{1.740817in}{1.580867in}}{\pgfqpoint{1.740817in}{1.572631in}}%
\pgfpathcurveto{\pgfqpoint{1.740817in}{1.564394in}}{\pgfqpoint{1.744089in}{1.556494in}}{\pgfqpoint{1.749913in}{1.550670in}}%
\pgfpathcurveto{\pgfqpoint{1.755737in}{1.544847in}}{\pgfqpoint{1.763637in}{1.541574in}}{\pgfqpoint{1.771874in}{1.541574in}}%
\pgfpathclose%
\pgfusepath{stroke,fill}%
\end{pgfscope}%
\begin{pgfscope}%
\pgfpathrectangle{\pgfqpoint{0.100000in}{0.212622in}}{\pgfqpoint{3.696000in}{3.696000in}}%
\pgfusepath{clip}%
\pgfsetbuttcap%
\pgfsetroundjoin%
\definecolor{currentfill}{rgb}{0.121569,0.466667,0.705882}%
\pgfsetfillcolor{currentfill}%
\pgfsetfillopacity{0.920077}%
\pgfsetlinewidth{1.003750pt}%
\definecolor{currentstroke}{rgb}{0.121569,0.466667,0.705882}%
\pgfsetstrokecolor{currentstroke}%
\pgfsetstrokeopacity{0.920077}%
\pgfsetdash{}{0pt}%
\pgfpathmoveto{\pgfqpoint{1.009767in}{2.168803in}}%
\pgfpathcurveto{\pgfqpoint{1.018004in}{2.168803in}}{\pgfqpoint{1.025904in}{2.172075in}}{\pgfqpoint{1.031728in}{2.177899in}}%
\pgfpathcurveto{\pgfqpoint{1.037552in}{2.183723in}}{\pgfqpoint{1.040824in}{2.191623in}}{\pgfqpoint{1.040824in}{2.199859in}}%
\pgfpathcurveto{\pgfqpoint{1.040824in}{2.208095in}}{\pgfqpoint{1.037552in}{2.215995in}}{\pgfqpoint{1.031728in}{2.221819in}}%
\pgfpathcurveto{\pgfqpoint{1.025904in}{2.227643in}}{\pgfqpoint{1.018004in}{2.230916in}}{\pgfqpoint{1.009767in}{2.230916in}}%
\pgfpathcurveto{\pgfqpoint{1.001531in}{2.230916in}}{\pgfqpoint{0.993631in}{2.227643in}}{\pgfqpoint{0.987807in}{2.221819in}}%
\pgfpathcurveto{\pgfqpoint{0.981983in}{2.215995in}}{\pgfqpoint{0.978711in}{2.208095in}}{\pgfqpoint{0.978711in}{2.199859in}}%
\pgfpathcurveto{\pgfqpoint{0.978711in}{2.191623in}}{\pgfqpoint{0.981983in}{2.183723in}}{\pgfqpoint{0.987807in}{2.177899in}}%
\pgfpathcurveto{\pgfqpoint{0.993631in}{2.172075in}}{\pgfqpoint{1.001531in}{2.168803in}}{\pgfqpoint{1.009767in}{2.168803in}}%
\pgfpathclose%
\pgfusepath{stroke,fill}%
\end{pgfscope}%
\begin{pgfscope}%
\pgfpathrectangle{\pgfqpoint{0.100000in}{0.212622in}}{\pgfqpoint{3.696000in}{3.696000in}}%
\pgfusepath{clip}%
\pgfsetbuttcap%
\pgfsetroundjoin%
\definecolor{currentfill}{rgb}{0.121569,0.466667,0.705882}%
\pgfsetfillcolor{currentfill}%
\pgfsetfillopacity{0.920690}%
\pgfsetlinewidth{1.003750pt}%
\definecolor{currentstroke}{rgb}{0.121569,0.466667,0.705882}%
\pgfsetstrokecolor{currentstroke}%
\pgfsetstrokeopacity{0.920690}%
\pgfsetdash{}{0pt}%
\pgfpathmoveto{\pgfqpoint{1.045823in}{2.134226in}}%
\pgfpathcurveto{\pgfqpoint{1.054059in}{2.134226in}}{\pgfqpoint{1.061959in}{2.137499in}}{\pgfqpoint{1.067783in}{2.143323in}}%
\pgfpathcurveto{\pgfqpoint{1.073607in}{2.149147in}}{\pgfqpoint{1.076879in}{2.157047in}}{\pgfqpoint{1.076879in}{2.165283in}}%
\pgfpathcurveto{\pgfqpoint{1.076879in}{2.173519in}}{\pgfqpoint{1.073607in}{2.181419in}}{\pgfqpoint{1.067783in}{2.187243in}}%
\pgfpathcurveto{\pgfqpoint{1.061959in}{2.193067in}}{\pgfqpoint{1.054059in}{2.196339in}}{\pgfqpoint{1.045823in}{2.196339in}}%
\pgfpathcurveto{\pgfqpoint{1.037587in}{2.196339in}}{\pgfqpoint{1.029687in}{2.193067in}}{\pgfqpoint{1.023863in}{2.187243in}}%
\pgfpathcurveto{\pgfqpoint{1.018039in}{2.181419in}}{\pgfqpoint{1.014766in}{2.173519in}}{\pgfqpoint{1.014766in}{2.165283in}}%
\pgfpathcurveto{\pgfqpoint{1.014766in}{2.157047in}}{\pgfqpoint{1.018039in}{2.149147in}}{\pgfqpoint{1.023863in}{2.143323in}}%
\pgfpathcurveto{\pgfqpoint{1.029687in}{2.137499in}}{\pgfqpoint{1.037587in}{2.134226in}}{\pgfqpoint{1.045823in}{2.134226in}}%
\pgfpathclose%
\pgfusepath{stroke,fill}%
\end{pgfscope}%
\begin{pgfscope}%
\pgfpathrectangle{\pgfqpoint{0.100000in}{0.212622in}}{\pgfqpoint{3.696000in}{3.696000in}}%
\pgfusepath{clip}%
\pgfsetbuttcap%
\pgfsetroundjoin%
\definecolor{currentfill}{rgb}{0.121569,0.466667,0.705882}%
\pgfsetfillcolor{currentfill}%
\pgfsetfillopacity{0.921778}%
\pgfsetlinewidth{1.003750pt}%
\definecolor{currentstroke}{rgb}{0.121569,0.466667,0.705882}%
\pgfsetstrokecolor{currentstroke}%
\pgfsetstrokeopacity{0.921778}%
\pgfsetdash{}{0pt}%
\pgfpathmoveto{\pgfqpoint{1.774385in}{1.541417in}}%
\pgfpathcurveto{\pgfqpoint{1.782622in}{1.541417in}}{\pgfqpoint{1.790522in}{1.544690in}}{\pgfqpoint{1.796346in}{1.550513in}}%
\pgfpathcurveto{\pgfqpoint{1.802170in}{1.556337in}}{\pgfqpoint{1.805442in}{1.564237in}}{\pgfqpoint{1.805442in}{1.572474in}}%
\pgfpathcurveto{\pgfqpoint{1.805442in}{1.580710in}}{\pgfqpoint{1.802170in}{1.588610in}}{\pgfqpoint{1.796346in}{1.594434in}}%
\pgfpathcurveto{\pgfqpoint{1.790522in}{1.600258in}}{\pgfqpoint{1.782622in}{1.603530in}}{\pgfqpoint{1.774385in}{1.603530in}}%
\pgfpathcurveto{\pgfqpoint{1.766149in}{1.603530in}}{\pgfqpoint{1.758249in}{1.600258in}}{\pgfqpoint{1.752425in}{1.594434in}}%
\pgfpathcurveto{\pgfqpoint{1.746601in}{1.588610in}}{\pgfqpoint{1.743329in}{1.580710in}}{\pgfqpoint{1.743329in}{1.572474in}}%
\pgfpathcurveto{\pgfqpoint{1.743329in}{1.564237in}}{\pgfqpoint{1.746601in}{1.556337in}}{\pgfqpoint{1.752425in}{1.550513in}}%
\pgfpathcurveto{\pgfqpoint{1.758249in}{1.544690in}}{\pgfqpoint{1.766149in}{1.541417in}}{\pgfqpoint{1.774385in}{1.541417in}}%
\pgfpathclose%
\pgfusepath{stroke,fill}%
\end{pgfscope}%
\begin{pgfscope}%
\pgfpathrectangle{\pgfqpoint{0.100000in}{0.212622in}}{\pgfqpoint{3.696000in}{3.696000in}}%
\pgfusepath{clip}%
\pgfsetbuttcap%
\pgfsetroundjoin%
\definecolor{currentfill}{rgb}{0.121569,0.466667,0.705882}%
\pgfsetfillcolor{currentfill}%
\pgfsetfillopacity{0.924888}%
\pgfsetlinewidth{1.003750pt}%
\definecolor{currentstroke}{rgb}{0.121569,0.466667,0.705882}%
\pgfsetstrokecolor{currentstroke}%
\pgfsetstrokeopacity{0.924888}%
\pgfsetdash{}{0pt}%
\pgfpathmoveto{\pgfqpoint{1.076587in}{2.107787in}}%
\pgfpathcurveto{\pgfqpoint{1.084823in}{2.107787in}}{\pgfqpoint{1.092724in}{2.111059in}}{\pgfqpoint{1.098547in}{2.116883in}}%
\pgfpathcurveto{\pgfqpoint{1.104371in}{2.122707in}}{\pgfqpoint{1.107644in}{2.130607in}}{\pgfqpoint{1.107644in}{2.138843in}}%
\pgfpathcurveto{\pgfqpoint{1.107644in}{2.147079in}}{\pgfqpoint{1.104371in}{2.154979in}}{\pgfqpoint{1.098547in}{2.160803in}}%
\pgfpathcurveto{\pgfqpoint{1.092724in}{2.166627in}}{\pgfqpoint{1.084823in}{2.169900in}}{\pgfqpoint{1.076587in}{2.169900in}}%
\pgfpathcurveto{\pgfqpoint{1.068351in}{2.169900in}}{\pgfqpoint{1.060451in}{2.166627in}}{\pgfqpoint{1.054627in}{2.160803in}}%
\pgfpathcurveto{\pgfqpoint{1.048803in}{2.154979in}}{\pgfqpoint{1.045531in}{2.147079in}}{\pgfqpoint{1.045531in}{2.138843in}}%
\pgfpathcurveto{\pgfqpoint{1.045531in}{2.130607in}}{\pgfqpoint{1.048803in}{2.122707in}}{\pgfqpoint{1.054627in}{2.116883in}}%
\pgfpathcurveto{\pgfqpoint{1.060451in}{2.111059in}}{\pgfqpoint{1.068351in}{2.107787in}}{\pgfqpoint{1.076587in}{2.107787in}}%
\pgfpathclose%
\pgfusepath{stroke,fill}%
\end{pgfscope}%
\begin{pgfscope}%
\pgfpathrectangle{\pgfqpoint{0.100000in}{0.212622in}}{\pgfqpoint{3.696000in}{3.696000in}}%
\pgfusepath{clip}%
\pgfsetbuttcap%
\pgfsetroundjoin%
\definecolor{currentfill}{rgb}{0.121569,0.466667,0.705882}%
\pgfsetfillcolor{currentfill}%
\pgfsetfillopacity{0.926016}%
\pgfsetlinewidth{1.003750pt}%
\definecolor{currentstroke}{rgb}{0.121569,0.466667,0.705882}%
\pgfsetstrokecolor{currentstroke}%
\pgfsetstrokeopacity{0.926016}%
\pgfsetdash{}{0pt}%
\pgfpathmoveto{\pgfqpoint{1.776345in}{1.541387in}}%
\pgfpathcurveto{\pgfqpoint{1.784581in}{1.541387in}}{\pgfqpoint{1.792481in}{1.544660in}}{\pgfqpoint{1.798305in}{1.550483in}}%
\pgfpathcurveto{\pgfqpoint{1.804129in}{1.556307in}}{\pgfqpoint{1.807402in}{1.564207in}}{\pgfqpoint{1.807402in}{1.572444in}}%
\pgfpathcurveto{\pgfqpoint{1.807402in}{1.580680in}}{\pgfqpoint{1.804129in}{1.588580in}}{\pgfqpoint{1.798305in}{1.594404in}}%
\pgfpathcurveto{\pgfqpoint{1.792481in}{1.600228in}}{\pgfqpoint{1.784581in}{1.603500in}}{\pgfqpoint{1.776345in}{1.603500in}}%
\pgfpathcurveto{\pgfqpoint{1.768109in}{1.603500in}}{\pgfqpoint{1.760209in}{1.600228in}}{\pgfqpoint{1.754385in}{1.594404in}}%
\pgfpathcurveto{\pgfqpoint{1.748561in}{1.588580in}}{\pgfqpoint{1.745289in}{1.580680in}}{\pgfqpoint{1.745289in}{1.572444in}}%
\pgfpathcurveto{\pgfqpoint{1.745289in}{1.564207in}}{\pgfqpoint{1.748561in}{1.556307in}}{\pgfqpoint{1.754385in}{1.550483in}}%
\pgfpathcurveto{\pgfqpoint{1.760209in}{1.544660in}}{\pgfqpoint{1.768109in}{1.541387in}}{\pgfqpoint{1.776345in}{1.541387in}}%
\pgfpathclose%
\pgfusepath{stroke,fill}%
\end{pgfscope}%
\begin{pgfscope}%
\pgfpathrectangle{\pgfqpoint{0.100000in}{0.212622in}}{\pgfqpoint{3.696000in}{3.696000in}}%
\pgfusepath{clip}%
\pgfsetbuttcap%
\pgfsetroundjoin%
\definecolor{currentfill}{rgb}{0.121569,0.466667,0.705882}%
\pgfsetfillcolor{currentfill}%
\pgfsetfillopacity{0.927221}%
\pgfsetlinewidth{1.003750pt}%
\definecolor{currentstroke}{rgb}{0.121569,0.466667,0.705882}%
\pgfsetstrokecolor{currentstroke}%
\pgfsetstrokeopacity{0.927221}%
\pgfsetdash{}{0pt}%
\pgfpathmoveto{\pgfqpoint{1.107687in}{2.078186in}}%
\pgfpathcurveto{\pgfqpoint{1.115923in}{2.078186in}}{\pgfqpoint{1.123823in}{2.081458in}}{\pgfqpoint{1.129647in}{2.087282in}}%
\pgfpathcurveto{\pgfqpoint{1.135471in}{2.093106in}}{\pgfqpoint{1.138743in}{2.101006in}}{\pgfqpoint{1.138743in}{2.109243in}}%
\pgfpathcurveto{\pgfqpoint{1.138743in}{2.117479in}}{\pgfqpoint{1.135471in}{2.125379in}}{\pgfqpoint{1.129647in}{2.131203in}}%
\pgfpathcurveto{\pgfqpoint{1.123823in}{2.137027in}}{\pgfqpoint{1.115923in}{2.140299in}}{\pgfqpoint{1.107687in}{2.140299in}}%
\pgfpathcurveto{\pgfqpoint{1.099450in}{2.140299in}}{\pgfqpoint{1.091550in}{2.137027in}}{\pgfqpoint{1.085726in}{2.131203in}}%
\pgfpathcurveto{\pgfqpoint{1.079902in}{2.125379in}}{\pgfqpoint{1.076630in}{2.117479in}}{\pgfqpoint{1.076630in}{2.109243in}}%
\pgfpathcurveto{\pgfqpoint{1.076630in}{2.101006in}}{\pgfqpoint{1.079902in}{2.093106in}}{\pgfqpoint{1.085726in}{2.087282in}}%
\pgfpathcurveto{\pgfqpoint{1.091550in}{2.081458in}}{\pgfqpoint{1.099450in}{2.078186in}}{\pgfqpoint{1.107687in}{2.078186in}}%
\pgfpathclose%
\pgfusepath{stroke,fill}%
\end{pgfscope}%
\begin{pgfscope}%
\pgfpathrectangle{\pgfqpoint{0.100000in}{0.212622in}}{\pgfqpoint{3.696000in}{3.696000in}}%
\pgfusepath{clip}%
\pgfsetbuttcap%
\pgfsetroundjoin%
\definecolor{currentfill}{rgb}{0.121569,0.466667,0.705882}%
\pgfsetfillcolor{currentfill}%
\pgfsetfillopacity{0.929853}%
\pgfsetlinewidth{1.003750pt}%
\definecolor{currentstroke}{rgb}{0.121569,0.466667,0.705882}%
\pgfsetstrokecolor{currentstroke}%
\pgfsetstrokeopacity{0.929853}%
\pgfsetdash{}{0pt}%
\pgfpathmoveto{\pgfqpoint{1.778622in}{1.538182in}}%
\pgfpathcurveto{\pgfqpoint{1.786859in}{1.538182in}}{\pgfqpoint{1.794759in}{1.541454in}}{\pgfqpoint{1.800583in}{1.547278in}}%
\pgfpathcurveto{\pgfqpoint{1.806407in}{1.553102in}}{\pgfqpoint{1.809679in}{1.561002in}}{\pgfqpoint{1.809679in}{1.569238in}}%
\pgfpathcurveto{\pgfqpoint{1.809679in}{1.577475in}}{\pgfqpoint{1.806407in}{1.585375in}}{\pgfqpoint{1.800583in}{1.591198in}}%
\pgfpathcurveto{\pgfqpoint{1.794759in}{1.597022in}}{\pgfqpoint{1.786859in}{1.600295in}}{\pgfqpoint{1.778622in}{1.600295in}}%
\pgfpathcurveto{\pgfqpoint{1.770386in}{1.600295in}}{\pgfqpoint{1.762486in}{1.597022in}}{\pgfqpoint{1.756662in}{1.591198in}}%
\pgfpathcurveto{\pgfqpoint{1.750838in}{1.585375in}}{\pgfqpoint{1.747566in}{1.577475in}}{\pgfqpoint{1.747566in}{1.569238in}}%
\pgfpathcurveto{\pgfqpoint{1.747566in}{1.561002in}}{\pgfqpoint{1.750838in}{1.553102in}}{\pgfqpoint{1.756662in}{1.547278in}}%
\pgfpathcurveto{\pgfqpoint{1.762486in}{1.541454in}}{\pgfqpoint{1.770386in}{1.538182in}}{\pgfqpoint{1.778622in}{1.538182in}}%
\pgfpathclose%
\pgfusepath{stroke,fill}%
\end{pgfscope}%
\begin{pgfscope}%
\pgfpathrectangle{\pgfqpoint{0.100000in}{0.212622in}}{\pgfqpoint{3.696000in}{3.696000in}}%
\pgfusepath{clip}%
\pgfsetbuttcap%
\pgfsetroundjoin%
\definecolor{currentfill}{rgb}{0.121569,0.466667,0.705882}%
\pgfsetfillcolor{currentfill}%
\pgfsetfillopacity{0.930153}%
\pgfsetlinewidth{1.003750pt}%
\definecolor{currentstroke}{rgb}{0.121569,0.466667,0.705882}%
\pgfsetstrokecolor{currentstroke}%
\pgfsetstrokeopacity{0.930153}%
\pgfsetdash{}{0pt}%
\pgfpathmoveto{\pgfqpoint{1.134677in}{2.049667in}}%
\pgfpathcurveto{\pgfqpoint{1.142913in}{2.049667in}}{\pgfqpoint{1.150813in}{2.052939in}}{\pgfqpoint{1.156637in}{2.058763in}}%
\pgfpathcurveto{\pgfqpoint{1.162461in}{2.064587in}}{\pgfqpoint{1.165733in}{2.072487in}}{\pgfqpoint{1.165733in}{2.080723in}}%
\pgfpathcurveto{\pgfqpoint{1.165733in}{2.088959in}}{\pgfqpoint{1.162461in}{2.096859in}}{\pgfqpoint{1.156637in}{2.102683in}}%
\pgfpathcurveto{\pgfqpoint{1.150813in}{2.108507in}}{\pgfqpoint{1.142913in}{2.111780in}}{\pgfqpoint{1.134677in}{2.111780in}}%
\pgfpathcurveto{\pgfqpoint{1.126440in}{2.111780in}}{\pgfqpoint{1.118540in}{2.108507in}}{\pgfqpoint{1.112716in}{2.102683in}}%
\pgfpathcurveto{\pgfqpoint{1.106892in}{2.096859in}}{\pgfqpoint{1.103620in}{2.088959in}}{\pgfqpoint{1.103620in}{2.080723in}}%
\pgfpathcurveto{\pgfqpoint{1.103620in}{2.072487in}}{\pgfqpoint{1.106892in}{2.064587in}}{\pgfqpoint{1.112716in}{2.058763in}}%
\pgfpathcurveto{\pgfqpoint{1.118540in}{2.052939in}}{\pgfqpoint{1.126440in}{2.049667in}}{\pgfqpoint{1.134677in}{2.049667in}}%
\pgfpathclose%
\pgfusepath{stroke,fill}%
\end{pgfscope}%
\begin{pgfscope}%
\pgfpathrectangle{\pgfqpoint{0.100000in}{0.212622in}}{\pgfqpoint{3.696000in}{3.696000in}}%
\pgfusepath{clip}%
\pgfsetbuttcap%
\pgfsetroundjoin%
\definecolor{currentfill}{rgb}{0.121569,0.466667,0.705882}%
\pgfsetfillcolor{currentfill}%
\pgfsetfillopacity{0.932851}%
\pgfsetlinewidth{1.003750pt}%
\definecolor{currentstroke}{rgb}{0.121569,0.466667,0.705882}%
\pgfsetstrokecolor{currentstroke}%
\pgfsetstrokeopacity{0.932851}%
\pgfsetdash{}{0pt}%
\pgfpathmoveto{\pgfqpoint{1.780308in}{1.530138in}}%
\pgfpathcurveto{\pgfqpoint{1.788544in}{1.530138in}}{\pgfqpoint{1.796444in}{1.533410in}}{\pgfqpoint{1.802268in}{1.539234in}}%
\pgfpathcurveto{\pgfqpoint{1.808092in}{1.545058in}}{\pgfqpoint{1.811365in}{1.552958in}}{\pgfqpoint{1.811365in}{1.561195in}}%
\pgfpathcurveto{\pgfqpoint{1.811365in}{1.569431in}}{\pgfqpoint{1.808092in}{1.577331in}}{\pgfqpoint{1.802268in}{1.583155in}}%
\pgfpathcurveto{\pgfqpoint{1.796444in}{1.588979in}}{\pgfqpoint{1.788544in}{1.592251in}}{\pgfqpoint{1.780308in}{1.592251in}}%
\pgfpathcurveto{\pgfqpoint{1.772072in}{1.592251in}}{\pgfqpoint{1.764172in}{1.588979in}}{\pgfqpoint{1.758348in}{1.583155in}}%
\pgfpathcurveto{\pgfqpoint{1.752524in}{1.577331in}}{\pgfqpoint{1.749252in}{1.569431in}}{\pgfqpoint{1.749252in}{1.561195in}}%
\pgfpathcurveto{\pgfqpoint{1.749252in}{1.552958in}}{\pgfqpoint{1.752524in}{1.545058in}}{\pgfqpoint{1.758348in}{1.539234in}}%
\pgfpathcurveto{\pgfqpoint{1.764172in}{1.533410in}}{\pgfqpoint{1.772072in}{1.530138in}}{\pgfqpoint{1.780308in}{1.530138in}}%
\pgfpathclose%
\pgfusepath{stroke,fill}%
\end{pgfscope}%
\begin{pgfscope}%
\pgfpathrectangle{\pgfqpoint{0.100000in}{0.212622in}}{\pgfqpoint{3.696000in}{3.696000in}}%
\pgfusepath{clip}%
\pgfsetbuttcap%
\pgfsetroundjoin%
\definecolor{currentfill}{rgb}{0.121569,0.466667,0.705882}%
\pgfsetfillcolor{currentfill}%
\pgfsetfillopacity{0.935366}%
\pgfsetlinewidth{1.003750pt}%
\definecolor{currentstroke}{rgb}{0.121569,0.466667,0.705882}%
\pgfsetstrokecolor{currentstroke}%
\pgfsetstrokeopacity{0.935366}%
\pgfsetdash{}{0pt}%
\pgfpathmoveto{\pgfqpoint{1.161978in}{2.034306in}}%
\pgfpathcurveto{\pgfqpoint{1.170214in}{2.034306in}}{\pgfqpoint{1.178114in}{2.037579in}}{\pgfqpoint{1.183938in}{2.043402in}}%
\pgfpathcurveto{\pgfqpoint{1.189762in}{2.049226in}}{\pgfqpoint{1.193034in}{2.057126in}}{\pgfqpoint{1.193034in}{2.065363in}}%
\pgfpathcurveto{\pgfqpoint{1.193034in}{2.073599in}}{\pgfqpoint{1.189762in}{2.081499in}}{\pgfqpoint{1.183938in}{2.087323in}}%
\pgfpathcurveto{\pgfqpoint{1.178114in}{2.093147in}}{\pgfqpoint{1.170214in}{2.096419in}}{\pgfqpoint{1.161978in}{2.096419in}}%
\pgfpathcurveto{\pgfqpoint{1.153742in}{2.096419in}}{\pgfqpoint{1.145842in}{2.093147in}}{\pgfqpoint{1.140018in}{2.087323in}}%
\pgfpathcurveto{\pgfqpoint{1.134194in}{2.081499in}}{\pgfqpoint{1.130921in}{2.073599in}}{\pgfqpoint{1.130921in}{2.065363in}}%
\pgfpathcurveto{\pgfqpoint{1.130921in}{2.057126in}}{\pgfqpoint{1.134194in}{2.049226in}}{\pgfqpoint{1.140018in}{2.043402in}}%
\pgfpathcurveto{\pgfqpoint{1.145842in}{2.037579in}}{\pgfqpoint{1.153742in}{2.034306in}}{\pgfqpoint{1.161978in}{2.034306in}}%
\pgfpathclose%
\pgfusepath{stroke,fill}%
\end{pgfscope}%
\begin{pgfscope}%
\pgfpathrectangle{\pgfqpoint{0.100000in}{0.212622in}}{\pgfqpoint{3.696000in}{3.696000in}}%
\pgfusepath{clip}%
\pgfsetbuttcap%
\pgfsetroundjoin%
\definecolor{currentfill}{rgb}{0.121569,0.466667,0.705882}%
\pgfsetfillcolor{currentfill}%
\pgfsetfillopacity{0.935932}%
\pgfsetlinewidth{1.003750pt}%
\definecolor{currentstroke}{rgb}{0.121569,0.466667,0.705882}%
\pgfsetstrokecolor{currentstroke}%
\pgfsetstrokeopacity{0.935932}%
\pgfsetdash{}{0pt}%
\pgfpathmoveto{\pgfqpoint{1.188278in}{2.005007in}}%
\pgfpathcurveto{\pgfqpoint{1.196514in}{2.005007in}}{\pgfqpoint{1.204414in}{2.008279in}}{\pgfqpoint{1.210238in}{2.014103in}}%
\pgfpathcurveto{\pgfqpoint{1.216062in}{2.019927in}}{\pgfqpoint{1.219334in}{2.027827in}}{\pgfqpoint{1.219334in}{2.036063in}}%
\pgfpathcurveto{\pgfqpoint{1.219334in}{2.044299in}}{\pgfqpoint{1.216062in}{2.052199in}}{\pgfqpoint{1.210238in}{2.058023in}}%
\pgfpathcurveto{\pgfqpoint{1.204414in}{2.063847in}}{\pgfqpoint{1.196514in}{2.067120in}}{\pgfqpoint{1.188278in}{2.067120in}}%
\pgfpathcurveto{\pgfqpoint{1.180041in}{2.067120in}}{\pgfqpoint{1.172141in}{2.063847in}}{\pgfqpoint{1.166317in}{2.058023in}}%
\pgfpathcurveto{\pgfqpoint{1.160493in}{2.052199in}}{\pgfqpoint{1.157221in}{2.044299in}}{\pgfqpoint{1.157221in}{2.036063in}}%
\pgfpathcurveto{\pgfqpoint{1.157221in}{2.027827in}}{\pgfqpoint{1.160493in}{2.019927in}}{\pgfqpoint{1.166317in}{2.014103in}}%
\pgfpathcurveto{\pgfqpoint{1.172141in}{2.008279in}}{\pgfqpoint{1.180041in}{2.005007in}}{\pgfqpoint{1.188278in}{2.005007in}}%
\pgfpathclose%
\pgfusepath{stroke,fill}%
\end{pgfscope}%
\begin{pgfscope}%
\pgfpathrectangle{\pgfqpoint{0.100000in}{0.212622in}}{\pgfqpoint{3.696000in}{3.696000in}}%
\pgfusepath{clip}%
\pgfsetbuttcap%
\pgfsetroundjoin%
\definecolor{currentfill}{rgb}{0.121569,0.466667,0.705882}%
\pgfsetfillcolor{currentfill}%
\pgfsetfillopacity{0.937445}%
\pgfsetlinewidth{1.003750pt}%
\definecolor{currentstroke}{rgb}{0.121569,0.466667,0.705882}%
\pgfsetstrokecolor{currentstroke}%
\pgfsetstrokeopacity{0.937445}%
\pgfsetdash{}{0pt}%
\pgfpathmoveto{\pgfqpoint{1.783348in}{1.528007in}}%
\pgfpathcurveto{\pgfqpoint{1.791585in}{1.528007in}}{\pgfqpoint{1.799485in}{1.531279in}}{\pgfqpoint{1.805309in}{1.537103in}}%
\pgfpathcurveto{\pgfqpoint{1.811132in}{1.542927in}}{\pgfqpoint{1.814405in}{1.550827in}}{\pgfqpoint{1.814405in}{1.559063in}}%
\pgfpathcurveto{\pgfqpoint{1.814405in}{1.567300in}}{\pgfqpoint{1.811132in}{1.575200in}}{\pgfqpoint{1.805309in}{1.581023in}}%
\pgfpathcurveto{\pgfqpoint{1.799485in}{1.586847in}}{\pgfqpoint{1.791585in}{1.590120in}}{\pgfqpoint{1.783348in}{1.590120in}}%
\pgfpathcurveto{\pgfqpoint{1.775112in}{1.590120in}}{\pgfqpoint{1.767212in}{1.586847in}}{\pgfqpoint{1.761388in}{1.581023in}}%
\pgfpathcurveto{\pgfqpoint{1.755564in}{1.575200in}}{\pgfqpoint{1.752292in}{1.567300in}}{\pgfqpoint{1.752292in}{1.559063in}}%
\pgfpathcurveto{\pgfqpoint{1.752292in}{1.550827in}}{\pgfqpoint{1.755564in}{1.542927in}}{\pgfqpoint{1.761388in}{1.537103in}}%
\pgfpathcurveto{\pgfqpoint{1.767212in}{1.531279in}}{\pgfqpoint{1.775112in}{1.528007in}}{\pgfqpoint{1.783348in}{1.528007in}}%
\pgfpathclose%
\pgfusepath{stroke,fill}%
\end{pgfscope}%
\begin{pgfscope}%
\pgfpathrectangle{\pgfqpoint{0.100000in}{0.212622in}}{\pgfqpoint{3.696000in}{3.696000in}}%
\pgfusepath{clip}%
\pgfsetbuttcap%
\pgfsetroundjoin%
\definecolor{currentfill}{rgb}{0.121569,0.466667,0.705882}%
\pgfsetfillcolor{currentfill}%
\pgfsetfillopacity{0.938596}%
\pgfsetlinewidth{1.003750pt}%
\definecolor{currentstroke}{rgb}{0.121569,0.466667,0.705882}%
\pgfsetstrokecolor{currentstroke}%
\pgfsetstrokeopacity{0.938596}%
\pgfsetdash{}{0pt}%
\pgfpathmoveto{\pgfqpoint{1.213752in}{1.985870in}}%
\pgfpathcurveto{\pgfqpoint{1.221989in}{1.985870in}}{\pgfqpoint{1.229889in}{1.989142in}}{\pgfqpoint{1.235713in}{1.994966in}}%
\pgfpathcurveto{\pgfqpoint{1.241537in}{2.000790in}}{\pgfqpoint{1.244809in}{2.008690in}}{\pgfqpoint{1.244809in}{2.016926in}}%
\pgfpathcurveto{\pgfqpoint{1.244809in}{2.025163in}}{\pgfqpoint{1.241537in}{2.033063in}}{\pgfqpoint{1.235713in}{2.038887in}}%
\pgfpathcurveto{\pgfqpoint{1.229889in}{2.044711in}}{\pgfqpoint{1.221989in}{2.047983in}}{\pgfqpoint{1.213752in}{2.047983in}}%
\pgfpathcurveto{\pgfqpoint{1.205516in}{2.047983in}}{\pgfqpoint{1.197616in}{2.044711in}}{\pgfqpoint{1.191792in}{2.038887in}}%
\pgfpathcurveto{\pgfqpoint{1.185968in}{2.033063in}}{\pgfqpoint{1.182696in}{2.025163in}}{\pgfqpoint{1.182696in}{2.016926in}}%
\pgfpathcurveto{\pgfqpoint{1.182696in}{2.008690in}}{\pgfqpoint{1.185968in}{2.000790in}}{\pgfqpoint{1.191792in}{1.994966in}}%
\pgfpathcurveto{\pgfqpoint{1.197616in}{1.989142in}}{\pgfqpoint{1.205516in}{1.985870in}}{\pgfqpoint{1.213752in}{1.985870in}}%
\pgfpathclose%
\pgfusepath{stroke,fill}%
\end{pgfscope}%
\begin{pgfscope}%
\pgfpathrectangle{\pgfqpoint{0.100000in}{0.212622in}}{\pgfqpoint{3.696000in}{3.696000in}}%
\pgfusepath{clip}%
\pgfsetbuttcap%
\pgfsetroundjoin%
\definecolor{currentfill}{rgb}{0.121569,0.466667,0.705882}%
\pgfsetfillcolor{currentfill}%
\pgfsetfillopacity{0.941281}%
\pgfsetlinewidth{1.003750pt}%
\definecolor{currentstroke}{rgb}{0.121569,0.466667,0.705882}%
\pgfsetstrokecolor{currentstroke}%
\pgfsetstrokeopacity{0.941281}%
\pgfsetdash{}{0pt}%
\pgfpathmoveto{\pgfqpoint{1.237589in}{1.968233in}}%
\pgfpathcurveto{\pgfqpoint{1.245825in}{1.968233in}}{\pgfqpoint{1.253725in}{1.971505in}}{\pgfqpoint{1.259549in}{1.977329in}}%
\pgfpathcurveto{\pgfqpoint{1.265373in}{1.983153in}}{\pgfqpoint{1.268645in}{1.991053in}}{\pgfqpoint{1.268645in}{1.999289in}}%
\pgfpathcurveto{\pgfqpoint{1.268645in}{2.007526in}}{\pgfqpoint{1.265373in}{2.015426in}}{\pgfqpoint{1.259549in}{2.021250in}}%
\pgfpathcurveto{\pgfqpoint{1.253725in}{2.027073in}}{\pgfqpoint{1.245825in}{2.030346in}}{\pgfqpoint{1.237589in}{2.030346in}}%
\pgfpathcurveto{\pgfqpoint{1.229353in}{2.030346in}}{\pgfqpoint{1.221453in}{2.027073in}}{\pgfqpoint{1.215629in}{2.021250in}}%
\pgfpathcurveto{\pgfqpoint{1.209805in}{2.015426in}}{\pgfqpoint{1.206532in}{2.007526in}}{\pgfqpoint{1.206532in}{1.999289in}}%
\pgfpathcurveto{\pgfqpoint{1.206532in}{1.991053in}}{\pgfqpoint{1.209805in}{1.983153in}}{\pgfqpoint{1.215629in}{1.977329in}}%
\pgfpathcurveto{\pgfqpoint{1.221453in}{1.971505in}}{\pgfqpoint{1.229353in}{1.968233in}}{\pgfqpoint{1.237589in}{1.968233in}}%
\pgfpathclose%
\pgfusepath{stroke,fill}%
\end{pgfscope}%
\begin{pgfscope}%
\pgfpathrectangle{\pgfqpoint{0.100000in}{0.212622in}}{\pgfqpoint{3.696000in}{3.696000in}}%
\pgfusepath{clip}%
\pgfsetbuttcap%
\pgfsetroundjoin%
\definecolor{currentfill}{rgb}{0.121569,0.466667,0.705882}%
\pgfsetfillcolor{currentfill}%
\pgfsetfillopacity{0.942475}%
\pgfsetlinewidth{1.003750pt}%
\definecolor{currentstroke}{rgb}{0.121569,0.466667,0.705882}%
\pgfsetstrokecolor{currentstroke}%
\pgfsetstrokeopacity{0.942475}%
\pgfsetdash{}{0pt}%
\pgfpathmoveto{\pgfqpoint{1.786469in}{1.526238in}}%
\pgfpathcurveto{\pgfqpoint{1.794706in}{1.526238in}}{\pgfqpoint{1.802606in}{1.529510in}}{\pgfqpoint{1.808430in}{1.535334in}}%
\pgfpathcurveto{\pgfqpoint{1.814253in}{1.541158in}}{\pgfqpoint{1.817526in}{1.549058in}}{\pgfqpoint{1.817526in}{1.557295in}}%
\pgfpathcurveto{\pgfqpoint{1.817526in}{1.565531in}}{\pgfqpoint{1.814253in}{1.573431in}}{\pgfqpoint{1.808430in}{1.579255in}}%
\pgfpathcurveto{\pgfqpoint{1.802606in}{1.585079in}}{\pgfqpoint{1.794706in}{1.588351in}}{\pgfqpoint{1.786469in}{1.588351in}}%
\pgfpathcurveto{\pgfqpoint{1.778233in}{1.588351in}}{\pgfqpoint{1.770333in}{1.585079in}}{\pgfqpoint{1.764509in}{1.579255in}}%
\pgfpathcurveto{\pgfqpoint{1.758685in}{1.573431in}}{\pgfqpoint{1.755413in}{1.565531in}}{\pgfqpoint{1.755413in}{1.557295in}}%
\pgfpathcurveto{\pgfqpoint{1.755413in}{1.549058in}}{\pgfqpoint{1.758685in}{1.541158in}}{\pgfqpoint{1.764509in}{1.535334in}}%
\pgfpathcurveto{\pgfqpoint{1.770333in}{1.529510in}}{\pgfqpoint{1.778233in}{1.526238in}}{\pgfqpoint{1.786469in}{1.526238in}}%
\pgfpathclose%
\pgfusepath{stroke,fill}%
\end{pgfscope}%
\begin{pgfscope}%
\pgfpathrectangle{\pgfqpoint{0.100000in}{0.212622in}}{\pgfqpoint{3.696000in}{3.696000in}}%
\pgfusepath{clip}%
\pgfsetbuttcap%
\pgfsetroundjoin%
\definecolor{currentfill}{rgb}{0.121569,0.466667,0.705882}%
\pgfsetfillcolor{currentfill}%
\pgfsetfillopacity{0.943266}%
\pgfsetlinewidth{1.003750pt}%
\definecolor{currentstroke}{rgb}{0.121569,0.466667,0.705882}%
\pgfsetstrokecolor{currentstroke}%
\pgfsetstrokeopacity{0.943266}%
\pgfsetdash{}{0pt}%
\pgfpathmoveto{\pgfqpoint{1.281019in}{1.924678in}}%
\pgfpathcurveto{\pgfqpoint{1.289255in}{1.924678in}}{\pgfqpoint{1.297155in}{1.927951in}}{\pgfqpoint{1.302979in}{1.933775in}}%
\pgfpathcurveto{\pgfqpoint{1.308803in}{1.939599in}}{\pgfqpoint{1.312076in}{1.947499in}}{\pgfqpoint{1.312076in}{1.955735in}}%
\pgfpathcurveto{\pgfqpoint{1.312076in}{1.963971in}}{\pgfqpoint{1.308803in}{1.971871in}}{\pgfqpoint{1.302979in}{1.977695in}}%
\pgfpathcurveto{\pgfqpoint{1.297155in}{1.983519in}}{\pgfqpoint{1.289255in}{1.986791in}}{\pgfqpoint{1.281019in}{1.986791in}}%
\pgfpathcurveto{\pgfqpoint{1.272783in}{1.986791in}}{\pgfqpoint{1.264883in}{1.983519in}}{\pgfqpoint{1.259059in}{1.977695in}}%
\pgfpathcurveto{\pgfqpoint{1.253235in}{1.971871in}}{\pgfqpoint{1.249963in}{1.963971in}}{\pgfqpoint{1.249963in}{1.955735in}}%
\pgfpathcurveto{\pgfqpoint{1.249963in}{1.947499in}}{\pgfqpoint{1.253235in}{1.939599in}}{\pgfqpoint{1.259059in}{1.933775in}}%
\pgfpathcurveto{\pgfqpoint{1.264883in}{1.927951in}}{\pgfqpoint{1.272783in}{1.924678in}}{\pgfqpoint{1.281019in}{1.924678in}}%
\pgfpathclose%
\pgfusepath{stroke,fill}%
\end{pgfscope}%
\begin{pgfscope}%
\pgfpathrectangle{\pgfqpoint{0.100000in}{0.212622in}}{\pgfqpoint{3.696000in}{3.696000in}}%
\pgfusepath{clip}%
\pgfsetbuttcap%
\pgfsetroundjoin%
\definecolor{currentfill}{rgb}{0.121569,0.466667,0.705882}%
\pgfsetfillcolor{currentfill}%
\pgfsetfillopacity{0.945222}%
\pgfsetlinewidth{1.003750pt}%
\definecolor{currentstroke}{rgb}{0.121569,0.466667,0.705882}%
\pgfsetstrokecolor{currentstroke}%
\pgfsetstrokeopacity{0.945222}%
\pgfsetdash{}{0pt}%
\pgfpathmoveto{\pgfqpoint{1.258736in}{1.955533in}}%
\pgfpathcurveto{\pgfqpoint{1.266972in}{1.955533in}}{\pgfqpoint{1.274872in}{1.958805in}}{\pgfqpoint{1.280696in}{1.964629in}}%
\pgfpathcurveto{\pgfqpoint{1.286520in}{1.970453in}}{\pgfqpoint{1.289792in}{1.978353in}}{\pgfqpoint{1.289792in}{1.986589in}}%
\pgfpathcurveto{\pgfqpoint{1.289792in}{1.994826in}}{\pgfqpoint{1.286520in}{2.002726in}}{\pgfqpoint{1.280696in}{2.008550in}}%
\pgfpathcurveto{\pgfqpoint{1.274872in}{2.014374in}}{\pgfqpoint{1.266972in}{2.017646in}}{\pgfqpoint{1.258736in}{2.017646in}}%
\pgfpathcurveto{\pgfqpoint{1.250499in}{2.017646in}}{\pgfqpoint{1.242599in}{2.014374in}}{\pgfqpoint{1.236775in}{2.008550in}}%
\pgfpathcurveto{\pgfqpoint{1.230952in}{2.002726in}}{\pgfqpoint{1.227679in}{1.994826in}}{\pgfqpoint{1.227679in}{1.986589in}}%
\pgfpathcurveto{\pgfqpoint{1.227679in}{1.978353in}}{\pgfqpoint{1.230952in}{1.970453in}}{\pgfqpoint{1.236775in}{1.964629in}}%
\pgfpathcurveto{\pgfqpoint{1.242599in}{1.958805in}}{\pgfqpoint{1.250499in}{1.955533in}}{\pgfqpoint{1.258736in}{1.955533in}}%
\pgfpathclose%
\pgfusepath{stroke,fill}%
\end{pgfscope}%
\begin{pgfscope}%
\pgfpathrectangle{\pgfqpoint{0.100000in}{0.212622in}}{\pgfqpoint{3.696000in}{3.696000in}}%
\pgfusepath{clip}%
\pgfsetbuttcap%
\pgfsetroundjoin%
\definecolor{currentfill}{rgb}{0.121569,0.466667,0.705882}%
\pgfsetfillcolor{currentfill}%
\pgfsetfillopacity{0.946443}%
\pgfsetlinewidth{1.003750pt}%
\definecolor{currentstroke}{rgb}{0.121569,0.466667,0.705882}%
\pgfsetstrokecolor{currentstroke}%
\pgfsetstrokeopacity{0.946443}%
\pgfsetdash{}{0pt}%
\pgfpathmoveto{\pgfqpoint{1.298429in}{1.913326in}}%
\pgfpathcurveto{\pgfqpoint{1.306666in}{1.913326in}}{\pgfqpoint{1.314566in}{1.916599in}}{\pgfqpoint{1.320390in}{1.922423in}}%
\pgfpathcurveto{\pgfqpoint{1.326214in}{1.928246in}}{\pgfqpoint{1.329486in}{1.936147in}}{\pgfqpoint{1.329486in}{1.944383in}}%
\pgfpathcurveto{\pgfqpoint{1.329486in}{1.952619in}}{\pgfqpoint{1.326214in}{1.960519in}}{\pgfqpoint{1.320390in}{1.966343in}}%
\pgfpathcurveto{\pgfqpoint{1.314566in}{1.972167in}}{\pgfqpoint{1.306666in}{1.975439in}}{\pgfqpoint{1.298429in}{1.975439in}}%
\pgfpathcurveto{\pgfqpoint{1.290193in}{1.975439in}}{\pgfqpoint{1.282293in}{1.972167in}}{\pgfqpoint{1.276469in}{1.966343in}}%
\pgfpathcurveto{\pgfqpoint{1.270645in}{1.960519in}}{\pgfqpoint{1.267373in}{1.952619in}}{\pgfqpoint{1.267373in}{1.944383in}}%
\pgfpathcurveto{\pgfqpoint{1.267373in}{1.936147in}}{\pgfqpoint{1.270645in}{1.928246in}}{\pgfqpoint{1.276469in}{1.922423in}}%
\pgfpathcurveto{\pgfqpoint{1.282293in}{1.916599in}}{\pgfqpoint{1.290193in}{1.913326in}}{\pgfqpoint{1.298429in}{1.913326in}}%
\pgfpathclose%
\pgfusepath{stroke,fill}%
\end{pgfscope}%
\begin{pgfscope}%
\pgfpathrectangle{\pgfqpoint{0.100000in}{0.212622in}}{\pgfqpoint{3.696000in}{3.696000in}}%
\pgfusepath{clip}%
\pgfsetbuttcap%
\pgfsetroundjoin%
\definecolor{currentfill}{rgb}{0.121569,0.466667,0.705882}%
\pgfsetfillcolor{currentfill}%
\pgfsetfillopacity{0.947471}%
\pgfsetlinewidth{1.003750pt}%
\definecolor{currentstroke}{rgb}{0.121569,0.466667,0.705882}%
\pgfsetstrokecolor{currentstroke}%
\pgfsetstrokeopacity{0.947471}%
\pgfsetdash{}{0pt}%
\pgfpathmoveto{\pgfqpoint{1.315047in}{1.892554in}}%
\pgfpathcurveto{\pgfqpoint{1.323283in}{1.892554in}}{\pgfqpoint{1.331183in}{1.895826in}}{\pgfqpoint{1.337007in}{1.901650in}}%
\pgfpathcurveto{\pgfqpoint{1.342831in}{1.907474in}}{\pgfqpoint{1.346104in}{1.915374in}}{\pgfqpoint{1.346104in}{1.923610in}}%
\pgfpathcurveto{\pgfqpoint{1.346104in}{1.931846in}}{\pgfqpoint{1.342831in}{1.939746in}}{\pgfqpoint{1.337007in}{1.945570in}}%
\pgfpathcurveto{\pgfqpoint{1.331183in}{1.951394in}}{\pgfqpoint{1.323283in}{1.954667in}}{\pgfqpoint{1.315047in}{1.954667in}}%
\pgfpathcurveto{\pgfqpoint{1.306811in}{1.954667in}}{\pgfqpoint{1.298911in}{1.951394in}}{\pgfqpoint{1.293087in}{1.945570in}}%
\pgfpathcurveto{\pgfqpoint{1.287263in}{1.939746in}}{\pgfqpoint{1.283991in}{1.931846in}}{\pgfqpoint{1.283991in}{1.923610in}}%
\pgfpathcurveto{\pgfqpoint{1.283991in}{1.915374in}}{\pgfqpoint{1.287263in}{1.907474in}}{\pgfqpoint{1.293087in}{1.901650in}}%
\pgfpathcurveto{\pgfqpoint{1.298911in}{1.895826in}}{\pgfqpoint{1.306811in}{1.892554in}}{\pgfqpoint{1.315047in}{1.892554in}}%
\pgfpathclose%
\pgfusepath{stroke,fill}%
\end{pgfscope}%
\begin{pgfscope}%
\pgfpathrectangle{\pgfqpoint{0.100000in}{0.212622in}}{\pgfqpoint{3.696000in}{3.696000in}}%
\pgfusepath{clip}%
\pgfsetbuttcap%
\pgfsetroundjoin%
\definecolor{currentfill}{rgb}{0.121569,0.466667,0.705882}%
\pgfsetfillcolor{currentfill}%
\pgfsetfillopacity{0.948646}%
\pgfsetlinewidth{1.003750pt}%
\definecolor{currentstroke}{rgb}{0.121569,0.466667,0.705882}%
\pgfsetstrokecolor{currentstroke}%
\pgfsetstrokeopacity{0.948646}%
\pgfsetdash{}{0pt}%
\pgfpathmoveto{\pgfqpoint{1.789241in}{1.528299in}}%
\pgfpathcurveto{\pgfqpoint{1.797477in}{1.528299in}}{\pgfqpoint{1.805377in}{1.531571in}}{\pgfqpoint{1.811201in}{1.537395in}}%
\pgfpathcurveto{\pgfqpoint{1.817025in}{1.543219in}}{\pgfqpoint{1.820297in}{1.551119in}}{\pgfqpoint{1.820297in}{1.559356in}}%
\pgfpathcurveto{\pgfqpoint{1.820297in}{1.567592in}}{\pgfqpoint{1.817025in}{1.575492in}}{\pgfqpoint{1.811201in}{1.581316in}}%
\pgfpathcurveto{\pgfqpoint{1.805377in}{1.587140in}}{\pgfqpoint{1.797477in}{1.590412in}}{\pgfqpoint{1.789241in}{1.590412in}}%
\pgfpathcurveto{\pgfqpoint{1.781004in}{1.590412in}}{\pgfqpoint{1.773104in}{1.587140in}}{\pgfqpoint{1.767280in}{1.581316in}}%
\pgfpathcurveto{\pgfqpoint{1.761456in}{1.575492in}}{\pgfqpoint{1.758184in}{1.567592in}}{\pgfqpoint{1.758184in}{1.559356in}}%
\pgfpathcurveto{\pgfqpoint{1.758184in}{1.551119in}}{\pgfqpoint{1.761456in}{1.543219in}}{\pgfqpoint{1.767280in}{1.537395in}}%
\pgfpathcurveto{\pgfqpoint{1.773104in}{1.531571in}}{\pgfqpoint{1.781004in}{1.528299in}}{\pgfqpoint{1.789241in}{1.528299in}}%
\pgfpathclose%
\pgfusepath{stroke,fill}%
\end{pgfscope}%
\begin{pgfscope}%
\pgfpathrectangle{\pgfqpoint{0.100000in}{0.212622in}}{\pgfqpoint{3.696000in}{3.696000in}}%
\pgfusepath{clip}%
\pgfsetbuttcap%
\pgfsetroundjoin%
\definecolor{currentfill}{rgb}{0.121569,0.466667,0.705882}%
\pgfsetfillcolor{currentfill}%
\pgfsetfillopacity{0.948827}%
\pgfsetlinewidth{1.003750pt}%
\definecolor{currentstroke}{rgb}{0.121569,0.466667,0.705882}%
\pgfsetstrokecolor{currentstroke}%
\pgfsetstrokeopacity{0.948827}%
\pgfsetdash{}{0pt}%
\pgfpathmoveto{\pgfqpoint{1.329438in}{1.876539in}}%
\pgfpathcurveto{\pgfqpoint{1.337674in}{1.876539in}}{\pgfqpoint{1.345574in}{1.879811in}}{\pgfqpoint{1.351398in}{1.885635in}}%
\pgfpathcurveto{\pgfqpoint{1.357222in}{1.891459in}}{\pgfqpoint{1.360494in}{1.899359in}}{\pgfqpoint{1.360494in}{1.907596in}}%
\pgfpathcurveto{\pgfqpoint{1.360494in}{1.915832in}}{\pgfqpoint{1.357222in}{1.923732in}}{\pgfqpoint{1.351398in}{1.929556in}}%
\pgfpathcurveto{\pgfqpoint{1.345574in}{1.935380in}}{\pgfqpoint{1.337674in}{1.938652in}}{\pgfqpoint{1.329438in}{1.938652in}}%
\pgfpathcurveto{\pgfqpoint{1.321202in}{1.938652in}}{\pgfqpoint{1.313302in}{1.935380in}}{\pgfqpoint{1.307478in}{1.929556in}}%
\pgfpathcurveto{\pgfqpoint{1.301654in}{1.923732in}}{\pgfqpoint{1.298381in}{1.915832in}}{\pgfqpoint{1.298381in}{1.907596in}}%
\pgfpathcurveto{\pgfqpoint{1.298381in}{1.899359in}}{\pgfqpoint{1.301654in}{1.891459in}}{\pgfqpoint{1.307478in}{1.885635in}}%
\pgfpathcurveto{\pgfqpoint{1.313302in}{1.879811in}}{\pgfqpoint{1.321202in}{1.876539in}}{\pgfqpoint{1.329438in}{1.876539in}}%
\pgfpathclose%
\pgfusepath{stroke,fill}%
\end{pgfscope}%
\begin{pgfscope}%
\pgfpathrectangle{\pgfqpoint{0.100000in}{0.212622in}}{\pgfqpoint{3.696000in}{3.696000in}}%
\pgfusepath{clip}%
\pgfsetbuttcap%
\pgfsetroundjoin%
\definecolor{currentfill}{rgb}{0.121569,0.466667,0.705882}%
\pgfsetfillcolor{currentfill}%
\pgfsetfillopacity{0.953311}%
\pgfsetlinewidth{1.003750pt}%
\definecolor{currentstroke}{rgb}{0.121569,0.466667,0.705882}%
\pgfsetstrokecolor{currentstroke}%
\pgfsetstrokeopacity{0.953311}%
\pgfsetdash{}{0pt}%
\pgfpathmoveto{\pgfqpoint{1.791448in}{1.521585in}}%
\pgfpathcurveto{\pgfqpoint{1.799684in}{1.521585in}}{\pgfqpoint{1.807584in}{1.524858in}}{\pgfqpoint{1.813408in}{1.530681in}}%
\pgfpathcurveto{\pgfqpoint{1.819232in}{1.536505in}}{\pgfqpoint{1.822504in}{1.544405in}}{\pgfqpoint{1.822504in}{1.552642in}}%
\pgfpathcurveto{\pgfqpoint{1.822504in}{1.560878in}}{\pgfqpoint{1.819232in}{1.568778in}}{\pgfqpoint{1.813408in}{1.574602in}}%
\pgfpathcurveto{\pgfqpoint{1.807584in}{1.580426in}}{\pgfqpoint{1.799684in}{1.583698in}}{\pgfqpoint{1.791448in}{1.583698in}}%
\pgfpathcurveto{\pgfqpoint{1.783212in}{1.583698in}}{\pgfqpoint{1.775311in}{1.580426in}}{\pgfqpoint{1.769488in}{1.574602in}}%
\pgfpathcurveto{\pgfqpoint{1.763664in}{1.568778in}}{\pgfqpoint{1.760391in}{1.560878in}}{\pgfqpoint{1.760391in}{1.552642in}}%
\pgfpathcurveto{\pgfqpoint{1.760391in}{1.544405in}}{\pgfqpoint{1.763664in}{1.536505in}}{\pgfqpoint{1.769488in}{1.530681in}}%
\pgfpathcurveto{\pgfqpoint{1.775311in}{1.524858in}}{\pgfqpoint{1.783212in}{1.521585in}}{\pgfqpoint{1.791448in}{1.521585in}}%
\pgfpathclose%
\pgfusepath{stroke,fill}%
\end{pgfscope}%
\begin{pgfscope}%
\pgfpathrectangle{\pgfqpoint{0.100000in}{0.212622in}}{\pgfqpoint{3.696000in}{3.696000in}}%
\pgfusepath{clip}%
\pgfsetbuttcap%
\pgfsetroundjoin%
\definecolor{currentfill}{rgb}{0.121569,0.466667,0.705882}%
\pgfsetfillcolor{currentfill}%
\pgfsetfillopacity{0.954208}%
\pgfsetlinewidth{1.003750pt}%
\definecolor{currentstroke}{rgb}{0.121569,0.466667,0.705882}%
\pgfsetstrokecolor{currentstroke}%
\pgfsetstrokeopacity{0.954208}%
\pgfsetdash{}{0pt}%
\pgfpathmoveto{\pgfqpoint{1.355910in}{1.861678in}}%
\pgfpathcurveto{\pgfqpoint{1.364146in}{1.861678in}}{\pgfqpoint{1.372046in}{1.864950in}}{\pgfqpoint{1.377870in}{1.870774in}}%
\pgfpathcurveto{\pgfqpoint{1.383694in}{1.876598in}}{\pgfqpoint{1.386966in}{1.884498in}}{\pgfqpoint{1.386966in}{1.892734in}}%
\pgfpathcurveto{\pgfqpoint{1.386966in}{1.900971in}}{\pgfqpoint{1.383694in}{1.908871in}}{\pgfqpoint{1.377870in}{1.914695in}}%
\pgfpathcurveto{\pgfqpoint{1.372046in}{1.920519in}}{\pgfqpoint{1.364146in}{1.923791in}}{\pgfqpoint{1.355910in}{1.923791in}}%
\pgfpathcurveto{\pgfqpoint{1.347673in}{1.923791in}}{\pgfqpoint{1.339773in}{1.920519in}}{\pgfqpoint{1.333949in}{1.914695in}}%
\pgfpathcurveto{\pgfqpoint{1.328125in}{1.908871in}}{\pgfqpoint{1.324853in}{1.900971in}}{\pgfqpoint{1.324853in}{1.892734in}}%
\pgfpathcurveto{\pgfqpoint{1.324853in}{1.884498in}}{\pgfqpoint{1.328125in}{1.876598in}}{\pgfqpoint{1.333949in}{1.870774in}}%
\pgfpathcurveto{\pgfqpoint{1.339773in}{1.864950in}}{\pgfqpoint{1.347673in}{1.861678in}}{\pgfqpoint{1.355910in}{1.861678in}}%
\pgfpathclose%
\pgfusepath{stroke,fill}%
\end{pgfscope}%
\begin{pgfscope}%
\pgfpathrectangle{\pgfqpoint{0.100000in}{0.212622in}}{\pgfqpoint{3.696000in}{3.696000in}}%
\pgfusepath{clip}%
\pgfsetbuttcap%
\pgfsetroundjoin%
\definecolor{currentfill}{rgb}{0.121569,0.466667,0.705882}%
\pgfsetfillcolor{currentfill}%
\pgfsetfillopacity{0.956791}%
\pgfsetlinewidth{1.003750pt}%
\definecolor{currentstroke}{rgb}{0.121569,0.466667,0.705882}%
\pgfsetstrokecolor{currentstroke}%
\pgfsetstrokeopacity{0.956791}%
\pgfsetdash{}{0pt}%
\pgfpathmoveto{\pgfqpoint{1.382100in}{1.842222in}}%
\pgfpathcurveto{\pgfqpoint{1.390337in}{1.842222in}}{\pgfqpoint{1.398237in}{1.845494in}}{\pgfqpoint{1.404061in}{1.851318in}}%
\pgfpathcurveto{\pgfqpoint{1.409884in}{1.857142in}}{\pgfqpoint{1.413157in}{1.865042in}}{\pgfqpoint{1.413157in}{1.873278in}}%
\pgfpathcurveto{\pgfqpoint{1.413157in}{1.881515in}}{\pgfqpoint{1.409884in}{1.889415in}}{\pgfqpoint{1.404061in}{1.895239in}}%
\pgfpathcurveto{\pgfqpoint{1.398237in}{1.901063in}}{\pgfqpoint{1.390337in}{1.904335in}}{\pgfqpoint{1.382100in}{1.904335in}}%
\pgfpathcurveto{\pgfqpoint{1.373864in}{1.904335in}}{\pgfqpoint{1.365964in}{1.901063in}}{\pgfqpoint{1.360140in}{1.895239in}}%
\pgfpathcurveto{\pgfqpoint{1.354316in}{1.889415in}}{\pgfqpoint{1.351044in}{1.881515in}}{\pgfqpoint{1.351044in}{1.873278in}}%
\pgfpathcurveto{\pgfqpoint{1.351044in}{1.865042in}}{\pgfqpoint{1.354316in}{1.857142in}}{\pgfqpoint{1.360140in}{1.851318in}}%
\pgfpathcurveto{\pgfqpoint{1.365964in}{1.845494in}}{\pgfqpoint{1.373864in}{1.842222in}}{\pgfqpoint{1.382100in}{1.842222in}}%
\pgfpathclose%
\pgfusepath{stroke,fill}%
\end{pgfscope}%
\begin{pgfscope}%
\pgfpathrectangle{\pgfqpoint{0.100000in}{0.212622in}}{\pgfqpoint{3.696000in}{3.696000in}}%
\pgfusepath{clip}%
\pgfsetbuttcap%
\pgfsetroundjoin%
\definecolor{currentfill}{rgb}{0.121569,0.466667,0.705882}%
\pgfsetfillcolor{currentfill}%
\pgfsetfillopacity{0.958524}%
\pgfsetlinewidth{1.003750pt}%
\definecolor{currentstroke}{rgb}{0.121569,0.466667,0.705882}%
\pgfsetstrokecolor{currentstroke}%
\pgfsetstrokeopacity{0.958524}%
\pgfsetdash{}{0pt}%
\pgfpathmoveto{\pgfqpoint{1.794566in}{1.516006in}}%
\pgfpathcurveto{\pgfqpoint{1.802802in}{1.516006in}}{\pgfqpoint{1.810702in}{1.519278in}}{\pgfqpoint{1.816526in}{1.525102in}}%
\pgfpathcurveto{\pgfqpoint{1.822350in}{1.530926in}}{\pgfqpoint{1.825623in}{1.538826in}}{\pgfqpoint{1.825623in}{1.547062in}}%
\pgfpathcurveto{\pgfqpoint{1.825623in}{1.555299in}}{\pgfqpoint{1.822350in}{1.563199in}}{\pgfqpoint{1.816526in}{1.569023in}}%
\pgfpathcurveto{\pgfqpoint{1.810702in}{1.574847in}}{\pgfqpoint{1.802802in}{1.578119in}}{\pgfqpoint{1.794566in}{1.578119in}}%
\pgfpathcurveto{\pgfqpoint{1.786330in}{1.578119in}}{\pgfqpoint{1.778430in}{1.574847in}}{\pgfqpoint{1.772606in}{1.569023in}}%
\pgfpathcurveto{\pgfqpoint{1.766782in}{1.563199in}}{\pgfqpoint{1.763510in}{1.555299in}}{\pgfqpoint{1.763510in}{1.547062in}}%
\pgfpathcurveto{\pgfqpoint{1.763510in}{1.538826in}}{\pgfqpoint{1.766782in}{1.530926in}}{\pgfqpoint{1.772606in}{1.525102in}}%
\pgfpathcurveto{\pgfqpoint{1.778430in}{1.519278in}}{\pgfqpoint{1.786330in}{1.516006in}}{\pgfqpoint{1.794566in}{1.516006in}}%
\pgfpathclose%
\pgfusepath{stroke,fill}%
\end{pgfscope}%
\begin{pgfscope}%
\pgfpathrectangle{\pgfqpoint{0.100000in}{0.212622in}}{\pgfqpoint{3.696000in}{3.696000in}}%
\pgfusepath{clip}%
\pgfsetbuttcap%
\pgfsetroundjoin%
\definecolor{currentfill}{rgb}{0.121569,0.466667,0.705882}%
\pgfsetfillcolor{currentfill}%
\pgfsetfillopacity{0.960083}%
\pgfsetlinewidth{1.003750pt}%
\definecolor{currentstroke}{rgb}{0.121569,0.466667,0.705882}%
\pgfsetstrokecolor{currentstroke}%
\pgfsetstrokeopacity{0.960083}%
\pgfsetdash{}{0pt}%
\pgfpathmoveto{\pgfqpoint{1.407433in}{1.826715in}}%
\pgfpathcurveto{\pgfqpoint{1.415670in}{1.826715in}}{\pgfqpoint{1.423570in}{1.829988in}}{\pgfqpoint{1.429394in}{1.835812in}}%
\pgfpathcurveto{\pgfqpoint{1.435218in}{1.841636in}}{\pgfqpoint{1.438490in}{1.849536in}}{\pgfqpoint{1.438490in}{1.857772in}}%
\pgfpathcurveto{\pgfqpoint{1.438490in}{1.866008in}}{\pgfqpoint{1.435218in}{1.873908in}}{\pgfqpoint{1.429394in}{1.879732in}}%
\pgfpathcurveto{\pgfqpoint{1.423570in}{1.885556in}}{\pgfqpoint{1.415670in}{1.888828in}}{\pgfqpoint{1.407433in}{1.888828in}}%
\pgfpathcurveto{\pgfqpoint{1.399197in}{1.888828in}}{\pgfqpoint{1.391297in}{1.885556in}}{\pgfqpoint{1.385473in}{1.879732in}}%
\pgfpathcurveto{\pgfqpoint{1.379649in}{1.873908in}}{\pgfqpoint{1.376377in}{1.866008in}}{\pgfqpoint{1.376377in}{1.857772in}}%
\pgfpathcurveto{\pgfqpoint{1.376377in}{1.849536in}}{\pgfqpoint{1.379649in}{1.841636in}}{\pgfqpoint{1.385473in}{1.835812in}}%
\pgfpathcurveto{\pgfqpoint{1.391297in}{1.829988in}}{\pgfqpoint{1.399197in}{1.826715in}}{\pgfqpoint{1.407433in}{1.826715in}}%
\pgfpathclose%
\pgfusepath{stroke,fill}%
\end{pgfscope}%
\begin{pgfscope}%
\pgfpathrectangle{\pgfqpoint{0.100000in}{0.212622in}}{\pgfqpoint{3.696000in}{3.696000in}}%
\pgfusepath{clip}%
\pgfsetbuttcap%
\pgfsetroundjoin%
\definecolor{currentfill}{rgb}{0.121569,0.466667,0.705882}%
\pgfsetfillcolor{currentfill}%
\pgfsetfillopacity{0.960640}%
\pgfsetlinewidth{1.003750pt}%
\definecolor{currentstroke}{rgb}{0.121569,0.466667,0.705882}%
\pgfsetstrokecolor{currentstroke}%
\pgfsetstrokeopacity{0.960640}%
\pgfsetdash{}{0pt}%
\pgfpathmoveto{\pgfqpoint{1.430500in}{1.799206in}}%
\pgfpathcurveto{\pgfqpoint{1.438736in}{1.799206in}}{\pgfqpoint{1.446636in}{1.802478in}}{\pgfqpoint{1.452460in}{1.808302in}}%
\pgfpathcurveto{\pgfqpoint{1.458284in}{1.814126in}}{\pgfqpoint{1.461556in}{1.822026in}}{\pgfqpoint{1.461556in}{1.830262in}}%
\pgfpathcurveto{\pgfqpoint{1.461556in}{1.838498in}}{\pgfqpoint{1.458284in}{1.846399in}}{\pgfqpoint{1.452460in}{1.852222in}}%
\pgfpathcurveto{\pgfqpoint{1.446636in}{1.858046in}}{\pgfqpoint{1.438736in}{1.861319in}}{\pgfqpoint{1.430500in}{1.861319in}}%
\pgfpathcurveto{\pgfqpoint{1.422264in}{1.861319in}}{\pgfqpoint{1.414364in}{1.858046in}}{\pgfqpoint{1.408540in}{1.852222in}}%
\pgfpathcurveto{\pgfqpoint{1.402716in}{1.846399in}}{\pgfqpoint{1.399443in}{1.838498in}}{\pgfqpoint{1.399443in}{1.830262in}}%
\pgfpathcurveto{\pgfqpoint{1.399443in}{1.822026in}}{\pgfqpoint{1.402716in}{1.814126in}}{\pgfqpoint{1.408540in}{1.808302in}}%
\pgfpathcurveto{\pgfqpoint{1.414364in}{1.802478in}}{\pgfqpoint{1.422264in}{1.799206in}}{\pgfqpoint{1.430500in}{1.799206in}}%
\pgfpathclose%
\pgfusepath{stroke,fill}%
\end{pgfscope}%
\begin{pgfscope}%
\pgfpathrectangle{\pgfqpoint{0.100000in}{0.212622in}}{\pgfqpoint{3.696000in}{3.696000in}}%
\pgfusepath{clip}%
\pgfsetbuttcap%
\pgfsetroundjoin%
\definecolor{currentfill}{rgb}{0.121569,0.466667,0.705882}%
\pgfsetfillcolor{currentfill}%
\pgfsetfillopacity{0.962102}%
\pgfsetlinewidth{1.003750pt}%
\definecolor{currentstroke}{rgb}{0.121569,0.466667,0.705882}%
\pgfsetstrokecolor{currentstroke}%
\pgfsetstrokeopacity{0.962102}%
\pgfsetdash{}{0pt}%
\pgfpathmoveto{\pgfqpoint{1.452406in}{1.778519in}}%
\pgfpathcurveto{\pgfqpoint{1.460642in}{1.778519in}}{\pgfqpoint{1.468542in}{1.781791in}}{\pgfqpoint{1.474366in}{1.787615in}}%
\pgfpathcurveto{\pgfqpoint{1.480190in}{1.793439in}}{\pgfqpoint{1.483462in}{1.801339in}}{\pgfqpoint{1.483462in}{1.809576in}}%
\pgfpathcurveto{\pgfqpoint{1.483462in}{1.817812in}}{\pgfqpoint{1.480190in}{1.825712in}}{\pgfqpoint{1.474366in}{1.831536in}}%
\pgfpathcurveto{\pgfqpoint{1.468542in}{1.837360in}}{\pgfqpoint{1.460642in}{1.840632in}}{\pgfqpoint{1.452406in}{1.840632in}}%
\pgfpathcurveto{\pgfqpoint{1.444169in}{1.840632in}}{\pgfqpoint{1.436269in}{1.837360in}}{\pgfqpoint{1.430445in}{1.831536in}}%
\pgfpathcurveto{\pgfqpoint{1.424621in}{1.825712in}}{\pgfqpoint{1.421349in}{1.817812in}}{\pgfqpoint{1.421349in}{1.809576in}}%
\pgfpathcurveto{\pgfqpoint{1.421349in}{1.801339in}}{\pgfqpoint{1.424621in}{1.793439in}}{\pgfqpoint{1.430445in}{1.787615in}}%
\pgfpathcurveto{\pgfqpoint{1.436269in}{1.781791in}}{\pgfqpoint{1.444169in}{1.778519in}}{\pgfqpoint{1.452406in}{1.778519in}}%
\pgfpathclose%
\pgfusepath{stroke,fill}%
\end{pgfscope}%
\begin{pgfscope}%
\pgfpathrectangle{\pgfqpoint{0.100000in}{0.212622in}}{\pgfqpoint{3.696000in}{3.696000in}}%
\pgfusepath{clip}%
\pgfsetbuttcap%
\pgfsetroundjoin%
\definecolor{currentfill}{rgb}{0.121569,0.466667,0.705882}%
\pgfsetfillcolor{currentfill}%
\pgfsetfillopacity{0.962972}%
\pgfsetlinewidth{1.003750pt}%
\definecolor{currentstroke}{rgb}{0.121569,0.466667,0.705882}%
\pgfsetstrokecolor{currentstroke}%
\pgfsetstrokeopacity{0.962972}%
\pgfsetdash{}{0pt}%
\pgfpathmoveto{\pgfqpoint{1.798178in}{1.505932in}}%
\pgfpathcurveto{\pgfqpoint{1.806414in}{1.505932in}}{\pgfqpoint{1.814314in}{1.509204in}}{\pgfqpoint{1.820138in}{1.515028in}}%
\pgfpathcurveto{\pgfqpoint{1.825962in}{1.520852in}}{\pgfqpoint{1.829234in}{1.528752in}}{\pgfqpoint{1.829234in}{1.536989in}}%
\pgfpathcurveto{\pgfqpoint{1.829234in}{1.545225in}}{\pgfqpoint{1.825962in}{1.553125in}}{\pgfqpoint{1.820138in}{1.558949in}}%
\pgfpathcurveto{\pgfqpoint{1.814314in}{1.564773in}}{\pgfqpoint{1.806414in}{1.568045in}}{\pgfqpoint{1.798178in}{1.568045in}}%
\pgfpathcurveto{\pgfqpoint{1.789942in}{1.568045in}}{\pgfqpoint{1.782042in}{1.564773in}}{\pgfqpoint{1.776218in}{1.558949in}}%
\pgfpathcurveto{\pgfqpoint{1.770394in}{1.553125in}}{\pgfqpoint{1.767121in}{1.545225in}}{\pgfqpoint{1.767121in}{1.536989in}}%
\pgfpathcurveto{\pgfqpoint{1.767121in}{1.528752in}}{\pgfqpoint{1.770394in}{1.520852in}}{\pgfqpoint{1.776218in}{1.515028in}}%
\pgfpathcurveto{\pgfqpoint{1.782042in}{1.509204in}}{\pgfqpoint{1.789942in}{1.505932in}}{\pgfqpoint{1.798178in}{1.505932in}}%
\pgfpathclose%
\pgfusepath{stroke,fill}%
\end{pgfscope}%
\begin{pgfscope}%
\pgfpathrectangle{\pgfqpoint{0.100000in}{0.212622in}}{\pgfqpoint{3.696000in}{3.696000in}}%
\pgfusepath{clip}%
\pgfsetbuttcap%
\pgfsetroundjoin%
\definecolor{currentfill}{rgb}{0.121569,0.466667,0.705882}%
\pgfsetfillcolor{currentfill}%
\pgfsetfillopacity{0.965620}%
\pgfsetlinewidth{1.003750pt}%
\definecolor{currentstroke}{rgb}{0.121569,0.466667,0.705882}%
\pgfsetstrokecolor{currentstroke}%
\pgfsetstrokeopacity{0.965620}%
\pgfsetdash{}{0pt}%
\pgfpathmoveto{\pgfqpoint{1.471950in}{1.767349in}}%
\pgfpathcurveto{\pgfqpoint{1.480186in}{1.767349in}}{\pgfqpoint{1.488087in}{1.770621in}}{\pgfqpoint{1.493910in}{1.776445in}}%
\pgfpathcurveto{\pgfqpoint{1.499734in}{1.782269in}}{\pgfqpoint{1.503007in}{1.790169in}}{\pgfqpoint{1.503007in}{1.798405in}}%
\pgfpathcurveto{\pgfqpoint{1.503007in}{1.806642in}}{\pgfqpoint{1.499734in}{1.814542in}}{\pgfqpoint{1.493910in}{1.820366in}}%
\pgfpathcurveto{\pgfqpoint{1.488087in}{1.826190in}}{\pgfqpoint{1.480186in}{1.829462in}}{\pgfqpoint{1.471950in}{1.829462in}}%
\pgfpathcurveto{\pgfqpoint{1.463714in}{1.829462in}}{\pgfqpoint{1.455814in}{1.826190in}}{\pgfqpoint{1.449990in}{1.820366in}}%
\pgfpathcurveto{\pgfqpoint{1.444166in}{1.814542in}}{\pgfqpoint{1.440894in}{1.806642in}}{\pgfqpoint{1.440894in}{1.798405in}}%
\pgfpathcurveto{\pgfqpoint{1.440894in}{1.790169in}}{\pgfqpoint{1.444166in}{1.782269in}}{\pgfqpoint{1.449990in}{1.776445in}}%
\pgfpathcurveto{\pgfqpoint{1.455814in}{1.770621in}}{\pgfqpoint{1.463714in}{1.767349in}}{\pgfqpoint{1.471950in}{1.767349in}}%
\pgfpathclose%
\pgfusepath{stroke,fill}%
\end{pgfscope}%
\begin{pgfscope}%
\pgfpathrectangle{\pgfqpoint{0.100000in}{0.212622in}}{\pgfqpoint{3.696000in}{3.696000in}}%
\pgfusepath{clip}%
\pgfsetbuttcap%
\pgfsetroundjoin%
\definecolor{currentfill}{rgb}{0.121569,0.466667,0.705882}%
\pgfsetfillcolor{currentfill}%
\pgfsetfillopacity{0.965882}%
\pgfsetlinewidth{1.003750pt}%
\definecolor{currentstroke}{rgb}{0.121569,0.466667,0.705882}%
\pgfsetstrokecolor{currentstroke}%
\pgfsetstrokeopacity{0.965882}%
\pgfsetdash{}{0pt}%
\pgfpathmoveto{\pgfqpoint{1.799963in}{1.502299in}}%
\pgfpathcurveto{\pgfqpoint{1.808200in}{1.502299in}}{\pgfqpoint{1.816100in}{1.505571in}}{\pgfqpoint{1.821924in}{1.511395in}}%
\pgfpathcurveto{\pgfqpoint{1.827748in}{1.517219in}}{\pgfqpoint{1.831020in}{1.525119in}}{\pgfqpoint{1.831020in}{1.533356in}}%
\pgfpathcurveto{\pgfqpoint{1.831020in}{1.541592in}}{\pgfqpoint{1.827748in}{1.549492in}}{\pgfqpoint{1.821924in}{1.555316in}}%
\pgfpathcurveto{\pgfqpoint{1.816100in}{1.561140in}}{\pgfqpoint{1.808200in}{1.564412in}}{\pgfqpoint{1.799963in}{1.564412in}}%
\pgfpathcurveto{\pgfqpoint{1.791727in}{1.564412in}}{\pgfqpoint{1.783827in}{1.561140in}}{\pgfqpoint{1.778003in}{1.555316in}}%
\pgfpathcurveto{\pgfqpoint{1.772179in}{1.549492in}}{\pgfqpoint{1.768907in}{1.541592in}}{\pgfqpoint{1.768907in}{1.533356in}}%
\pgfpathcurveto{\pgfqpoint{1.768907in}{1.525119in}}{\pgfqpoint{1.772179in}{1.517219in}}{\pgfqpoint{1.778003in}{1.511395in}}%
\pgfpathcurveto{\pgfqpoint{1.783827in}{1.505571in}}{\pgfqpoint{1.791727in}{1.502299in}}{\pgfqpoint{1.799963in}{1.502299in}}%
\pgfpathclose%
\pgfusepath{stroke,fill}%
\end{pgfscope}%
\begin{pgfscope}%
\pgfpathrectangle{\pgfqpoint{0.100000in}{0.212622in}}{\pgfqpoint{3.696000in}{3.696000in}}%
\pgfusepath{clip}%
\pgfsetbuttcap%
\pgfsetroundjoin%
\definecolor{currentfill}{rgb}{0.121569,0.466667,0.705882}%
\pgfsetfillcolor{currentfill}%
\pgfsetfillopacity{0.967620}%
\pgfsetlinewidth{1.003750pt}%
\definecolor{currentstroke}{rgb}{0.121569,0.466667,0.705882}%
\pgfsetstrokecolor{currentstroke}%
\pgfsetstrokeopacity{0.967620}%
\pgfsetdash{}{0pt}%
\pgfpathmoveto{\pgfqpoint{1.506892in}{1.732619in}}%
\pgfpathcurveto{\pgfqpoint{1.515128in}{1.732619in}}{\pgfqpoint{1.523028in}{1.735892in}}{\pgfqpoint{1.528852in}{1.741716in}}%
\pgfpathcurveto{\pgfqpoint{1.534676in}{1.747540in}}{\pgfqpoint{1.537948in}{1.755440in}}{\pgfqpoint{1.537948in}{1.763676in}}%
\pgfpathcurveto{\pgfqpoint{1.537948in}{1.771912in}}{\pgfqpoint{1.534676in}{1.779812in}}{\pgfqpoint{1.528852in}{1.785636in}}%
\pgfpathcurveto{\pgfqpoint{1.523028in}{1.791460in}}{\pgfqpoint{1.515128in}{1.794732in}}{\pgfqpoint{1.506892in}{1.794732in}}%
\pgfpathcurveto{\pgfqpoint{1.498655in}{1.794732in}}{\pgfqpoint{1.490755in}{1.791460in}}{\pgfqpoint{1.484931in}{1.785636in}}%
\pgfpathcurveto{\pgfqpoint{1.479107in}{1.779812in}}{\pgfqpoint{1.475835in}{1.771912in}}{\pgfqpoint{1.475835in}{1.763676in}}%
\pgfpathcurveto{\pgfqpoint{1.475835in}{1.755440in}}{\pgfqpoint{1.479107in}{1.747540in}}{\pgfqpoint{1.484931in}{1.741716in}}%
\pgfpathcurveto{\pgfqpoint{1.490755in}{1.735892in}}{\pgfqpoint{1.498655in}{1.732619in}}{\pgfqpoint{1.506892in}{1.732619in}}%
\pgfpathclose%
\pgfusepath{stroke,fill}%
\end{pgfscope}%
\begin{pgfscope}%
\pgfpathrectangle{\pgfqpoint{0.100000in}{0.212622in}}{\pgfqpoint{3.696000in}{3.696000in}}%
\pgfusepath{clip}%
\pgfsetbuttcap%
\pgfsetroundjoin%
\definecolor{currentfill}{rgb}{0.121569,0.466667,0.705882}%
\pgfsetfillcolor{currentfill}%
\pgfsetfillopacity{0.967842}%
\pgfsetlinewidth{1.003750pt}%
\definecolor{currentstroke}{rgb}{0.121569,0.466667,0.705882}%
\pgfsetstrokecolor{currentstroke}%
\pgfsetstrokeopacity{0.967842}%
\pgfsetdash{}{0pt}%
\pgfpathmoveto{\pgfqpoint{1.490805in}{1.755653in}}%
\pgfpathcurveto{\pgfqpoint{1.499041in}{1.755653in}}{\pgfqpoint{1.506941in}{1.758925in}}{\pgfqpoint{1.512765in}{1.764749in}}%
\pgfpathcurveto{\pgfqpoint{1.518589in}{1.770573in}}{\pgfqpoint{1.521862in}{1.778473in}}{\pgfqpoint{1.521862in}{1.786709in}}%
\pgfpathcurveto{\pgfqpoint{1.521862in}{1.794946in}}{\pgfqpoint{1.518589in}{1.802846in}}{\pgfqpoint{1.512765in}{1.808670in}}%
\pgfpathcurveto{\pgfqpoint{1.506941in}{1.814494in}}{\pgfqpoint{1.499041in}{1.817766in}}{\pgfqpoint{1.490805in}{1.817766in}}%
\pgfpathcurveto{\pgfqpoint{1.482569in}{1.817766in}}{\pgfqpoint{1.474669in}{1.814494in}}{\pgfqpoint{1.468845in}{1.808670in}}%
\pgfpathcurveto{\pgfqpoint{1.463021in}{1.802846in}}{\pgfqpoint{1.459749in}{1.794946in}}{\pgfqpoint{1.459749in}{1.786709in}}%
\pgfpathcurveto{\pgfqpoint{1.459749in}{1.778473in}}{\pgfqpoint{1.463021in}{1.770573in}}{\pgfqpoint{1.468845in}{1.764749in}}%
\pgfpathcurveto{\pgfqpoint{1.474669in}{1.758925in}}{\pgfqpoint{1.482569in}{1.755653in}}{\pgfqpoint{1.490805in}{1.755653in}}%
\pgfpathclose%
\pgfusepath{stroke,fill}%
\end{pgfscope}%
\begin{pgfscope}%
\pgfpathrectangle{\pgfqpoint{0.100000in}{0.212622in}}{\pgfqpoint{3.696000in}{3.696000in}}%
\pgfusepath{clip}%
\pgfsetbuttcap%
\pgfsetroundjoin%
\definecolor{currentfill}{rgb}{0.121569,0.466667,0.705882}%
\pgfsetfillcolor{currentfill}%
\pgfsetfillopacity{0.968806}%
\pgfsetlinewidth{1.003750pt}%
\definecolor{currentstroke}{rgb}{0.121569,0.466667,0.705882}%
\pgfsetstrokecolor{currentstroke}%
\pgfsetstrokeopacity{0.968806}%
\pgfsetdash{}{0pt}%
\pgfpathmoveto{\pgfqpoint{1.520677in}{1.723099in}}%
\pgfpathcurveto{\pgfqpoint{1.528913in}{1.723099in}}{\pgfqpoint{1.536814in}{1.726372in}}{\pgfqpoint{1.542637in}{1.732196in}}%
\pgfpathcurveto{\pgfqpoint{1.548461in}{1.738020in}}{\pgfqpoint{1.551734in}{1.745920in}}{\pgfqpoint{1.551734in}{1.754156in}}%
\pgfpathcurveto{\pgfqpoint{1.551734in}{1.762392in}}{\pgfqpoint{1.548461in}{1.770292in}}{\pgfqpoint{1.542637in}{1.776116in}}%
\pgfpathcurveto{\pgfqpoint{1.536814in}{1.781940in}}{\pgfqpoint{1.528913in}{1.785212in}}{\pgfqpoint{1.520677in}{1.785212in}}%
\pgfpathcurveto{\pgfqpoint{1.512441in}{1.785212in}}{\pgfqpoint{1.504541in}{1.781940in}}{\pgfqpoint{1.498717in}{1.776116in}}%
\pgfpathcurveto{\pgfqpoint{1.492893in}{1.770292in}}{\pgfqpoint{1.489621in}{1.762392in}}{\pgfqpoint{1.489621in}{1.754156in}}%
\pgfpathcurveto{\pgfqpoint{1.489621in}{1.745920in}}{\pgfqpoint{1.492893in}{1.738020in}}{\pgfqpoint{1.498717in}{1.732196in}}%
\pgfpathcurveto{\pgfqpoint{1.504541in}{1.726372in}}{\pgfqpoint{1.512441in}{1.723099in}}{\pgfqpoint{1.520677in}{1.723099in}}%
\pgfpathclose%
\pgfusepath{stroke,fill}%
\end{pgfscope}%
\begin{pgfscope}%
\pgfpathrectangle{\pgfqpoint{0.100000in}{0.212622in}}{\pgfqpoint{3.696000in}{3.696000in}}%
\pgfusepath{clip}%
\pgfsetbuttcap%
\pgfsetroundjoin%
\definecolor{currentfill}{rgb}{0.121569,0.466667,0.705882}%
\pgfsetfillcolor{currentfill}%
\pgfsetfillopacity{0.969102}%
\pgfsetlinewidth{1.003750pt}%
\definecolor{currentstroke}{rgb}{0.121569,0.466667,0.705882}%
\pgfsetstrokecolor{currentstroke}%
\pgfsetstrokeopacity{0.969102}%
\pgfsetdash{}{0pt}%
\pgfpathmoveto{\pgfqpoint{1.802246in}{1.499270in}}%
\pgfpathcurveto{\pgfqpoint{1.810483in}{1.499270in}}{\pgfqpoint{1.818383in}{1.502543in}}{\pgfqpoint{1.824207in}{1.508367in}}%
\pgfpathcurveto{\pgfqpoint{1.830031in}{1.514191in}}{\pgfqpoint{1.833303in}{1.522091in}}{\pgfqpoint{1.833303in}{1.530327in}}%
\pgfpathcurveto{\pgfqpoint{1.833303in}{1.538563in}}{\pgfqpoint{1.830031in}{1.546463in}}{\pgfqpoint{1.824207in}{1.552287in}}%
\pgfpathcurveto{\pgfqpoint{1.818383in}{1.558111in}}{\pgfqpoint{1.810483in}{1.561383in}}{\pgfqpoint{1.802246in}{1.561383in}}%
\pgfpathcurveto{\pgfqpoint{1.794010in}{1.561383in}}{\pgfqpoint{1.786110in}{1.558111in}}{\pgfqpoint{1.780286in}{1.552287in}}%
\pgfpathcurveto{\pgfqpoint{1.774462in}{1.546463in}}{\pgfqpoint{1.771190in}{1.538563in}}{\pgfqpoint{1.771190in}{1.530327in}}%
\pgfpathcurveto{\pgfqpoint{1.771190in}{1.522091in}}{\pgfqpoint{1.774462in}{1.514191in}}{\pgfqpoint{1.780286in}{1.508367in}}%
\pgfpathcurveto{\pgfqpoint{1.786110in}{1.502543in}}{\pgfqpoint{1.794010in}{1.499270in}}{\pgfqpoint{1.802246in}{1.499270in}}%
\pgfpathclose%
\pgfusepath{stroke,fill}%
\end{pgfscope}%
\begin{pgfscope}%
\pgfpathrectangle{\pgfqpoint{0.100000in}{0.212622in}}{\pgfqpoint{3.696000in}{3.696000in}}%
\pgfusepath{clip}%
\pgfsetbuttcap%
\pgfsetroundjoin%
\definecolor{currentfill}{rgb}{0.121569,0.466667,0.705882}%
\pgfsetfillcolor{currentfill}%
\pgfsetfillopacity{0.970383}%
\pgfsetlinewidth{1.003750pt}%
\definecolor{currentstroke}{rgb}{0.121569,0.466667,0.705882}%
\pgfsetstrokecolor{currentstroke}%
\pgfsetstrokeopacity{0.970383}%
\pgfsetdash{}{0pt}%
\pgfpathmoveto{\pgfqpoint{1.532746in}{1.715379in}}%
\pgfpathcurveto{\pgfqpoint{1.540982in}{1.715379in}}{\pgfqpoint{1.548882in}{1.718651in}}{\pgfqpoint{1.554706in}{1.724475in}}%
\pgfpathcurveto{\pgfqpoint{1.560530in}{1.730299in}}{\pgfqpoint{1.563803in}{1.738199in}}{\pgfqpoint{1.563803in}{1.746435in}}%
\pgfpathcurveto{\pgfqpoint{1.563803in}{1.754671in}}{\pgfqpoint{1.560530in}{1.762571in}}{\pgfqpoint{1.554706in}{1.768395in}}%
\pgfpathcurveto{\pgfqpoint{1.548882in}{1.774219in}}{\pgfqpoint{1.540982in}{1.777492in}}{\pgfqpoint{1.532746in}{1.777492in}}%
\pgfpathcurveto{\pgfqpoint{1.524510in}{1.777492in}}{\pgfqpoint{1.516610in}{1.774219in}}{\pgfqpoint{1.510786in}{1.768395in}}%
\pgfpathcurveto{\pgfqpoint{1.504962in}{1.762571in}}{\pgfqpoint{1.501690in}{1.754671in}}{\pgfqpoint{1.501690in}{1.746435in}}%
\pgfpathcurveto{\pgfqpoint{1.501690in}{1.738199in}}{\pgfqpoint{1.504962in}{1.730299in}}{\pgfqpoint{1.510786in}{1.724475in}}%
\pgfpathcurveto{\pgfqpoint{1.516610in}{1.718651in}}{\pgfqpoint{1.524510in}{1.715379in}}{\pgfqpoint{1.532746in}{1.715379in}}%
\pgfpathclose%
\pgfusepath{stroke,fill}%
\end{pgfscope}%
\begin{pgfscope}%
\pgfpathrectangle{\pgfqpoint{0.100000in}{0.212622in}}{\pgfqpoint{3.696000in}{3.696000in}}%
\pgfusepath{clip}%
\pgfsetbuttcap%
\pgfsetroundjoin%
\definecolor{currentfill}{rgb}{0.121569,0.466667,0.705882}%
\pgfsetfillcolor{currentfill}%
\pgfsetfillopacity{0.970902}%
\pgfsetlinewidth{1.003750pt}%
\definecolor{currentstroke}{rgb}{0.121569,0.466667,0.705882}%
\pgfsetstrokecolor{currentstroke}%
\pgfsetstrokeopacity{0.970902}%
\pgfsetdash{}{0pt}%
\pgfpathmoveto{\pgfqpoint{1.543918in}{1.705524in}}%
\pgfpathcurveto{\pgfqpoint{1.552154in}{1.705524in}}{\pgfqpoint{1.560054in}{1.708796in}}{\pgfqpoint{1.565878in}{1.714620in}}%
\pgfpathcurveto{\pgfqpoint{1.571702in}{1.720444in}}{\pgfqpoint{1.574974in}{1.728344in}}{\pgfqpoint{1.574974in}{1.736580in}}%
\pgfpathcurveto{\pgfqpoint{1.574974in}{1.744817in}}{\pgfqpoint{1.571702in}{1.752717in}}{\pgfqpoint{1.565878in}{1.758541in}}%
\pgfpathcurveto{\pgfqpoint{1.560054in}{1.764365in}}{\pgfqpoint{1.552154in}{1.767637in}}{\pgfqpoint{1.543918in}{1.767637in}}%
\pgfpathcurveto{\pgfqpoint{1.535682in}{1.767637in}}{\pgfqpoint{1.527782in}{1.764365in}}{\pgfqpoint{1.521958in}{1.758541in}}%
\pgfpathcurveto{\pgfqpoint{1.516134in}{1.752717in}}{\pgfqpoint{1.512861in}{1.744817in}}{\pgfqpoint{1.512861in}{1.736580in}}%
\pgfpathcurveto{\pgfqpoint{1.512861in}{1.728344in}}{\pgfqpoint{1.516134in}{1.720444in}}{\pgfqpoint{1.521958in}{1.714620in}}%
\pgfpathcurveto{\pgfqpoint{1.527782in}{1.708796in}}{\pgfqpoint{1.535682in}{1.705524in}}{\pgfqpoint{1.543918in}{1.705524in}}%
\pgfpathclose%
\pgfusepath{stroke,fill}%
\end{pgfscope}%
\begin{pgfscope}%
\pgfpathrectangle{\pgfqpoint{0.100000in}{0.212622in}}{\pgfqpoint{3.696000in}{3.696000in}}%
\pgfusepath{clip}%
\pgfsetbuttcap%
\pgfsetroundjoin%
\definecolor{currentfill}{rgb}{0.121569,0.466667,0.705882}%
\pgfsetfillcolor{currentfill}%
\pgfsetfillopacity{0.971556}%
\pgfsetlinewidth{1.003750pt}%
\definecolor{currentstroke}{rgb}{0.121569,0.466667,0.705882}%
\pgfsetstrokecolor{currentstroke}%
\pgfsetstrokeopacity{0.971556}%
\pgfsetdash{}{0pt}%
\pgfpathmoveto{\pgfqpoint{1.553242in}{1.696957in}}%
\pgfpathcurveto{\pgfqpoint{1.561479in}{1.696957in}}{\pgfqpoint{1.569379in}{1.700229in}}{\pgfqpoint{1.575203in}{1.706053in}}%
\pgfpathcurveto{\pgfqpoint{1.581027in}{1.711877in}}{\pgfqpoint{1.584299in}{1.719777in}}{\pgfqpoint{1.584299in}{1.728013in}}%
\pgfpathcurveto{\pgfqpoint{1.584299in}{1.736250in}}{\pgfqpoint{1.581027in}{1.744150in}}{\pgfqpoint{1.575203in}{1.749974in}}%
\pgfpathcurveto{\pgfqpoint{1.569379in}{1.755798in}}{\pgfqpoint{1.561479in}{1.759070in}}{\pgfqpoint{1.553242in}{1.759070in}}%
\pgfpathcurveto{\pgfqpoint{1.545006in}{1.759070in}}{\pgfqpoint{1.537106in}{1.755798in}}{\pgfqpoint{1.531282in}{1.749974in}}%
\pgfpathcurveto{\pgfqpoint{1.525458in}{1.744150in}}{\pgfqpoint{1.522186in}{1.736250in}}{\pgfqpoint{1.522186in}{1.728013in}}%
\pgfpathcurveto{\pgfqpoint{1.522186in}{1.719777in}}{\pgfqpoint{1.525458in}{1.711877in}}{\pgfqpoint{1.531282in}{1.706053in}}%
\pgfpathcurveto{\pgfqpoint{1.537106in}{1.700229in}}{\pgfqpoint{1.545006in}{1.696957in}}{\pgfqpoint{1.553242in}{1.696957in}}%
\pgfpathclose%
\pgfusepath{stroke,fill}%
\end{pgfscope}%
\begin{pgfscope}%
\pgfpathrectangle{\pgfqpoint{0.100000in}{0.212622in}}{\pgfqpoint{3.696000in}{3.696000in}}%
\pgfusepath{clip}%
\pgfsetbuttcap%
\pgfsetroundjoin%
\definecolor{currentfill}{rgb}{0.121569,0.466667,0.705882}%
\pgfsetfillcolor{currentfill}%
\pgfsetfillopacity{0.972640}%
\pgfsetlinewidth{1.003750pt}%
\definecolor{currentstroke}{rgb}{0.121569,0.466667,0.705882}%
\pgfsetstrokecolor{currentstroke}%
\pgfsetstrokeopacity{0.972640}%
\pgfsetdash{}{0pt}%
\pgfpathmoveto{\pgfqpoint{1.561585in}{1.689598in}}%
\pgfpathcurveto{\pgfqpoint{1.569821in}{1.689598in}}{\pgfqpoint{1.577721in}{1.692870in}}{\pgfqpoint{1.583545in}{1.698694in}}%
\pgfpathcurveto{\pgfqpoint{1.589369in}{1.704518in}}{\pgfqpoint{1.592641in}{1.712418in}}{\pgfqpoint{1.592641in}{1.720654in}}%
\pgfpathcurveto{\pgfqpoint{1.592641in}{1.728890in}}{\pgfqpoint{1.589369in}{1.736791in}}{\pgfqpoint{1.583545in}{1.742614in}}%
\pgfpathcurveto{\pgfqpoint{1.577721in}{1.748438in}}{\pgfqpoint{1.569821in}{1.751711in}}{\pgfqpoint{1.561585in}{1.751711in}}%
\pgfpathcurveto{\pgfqpoint{1.553348in}{1.751711in}}{\pgfqpoint{1.545448in}{1.748438in}}{\pgfqpoint{1.539624in}{1.742614in}}%
\pgfpathcurveto{\pgfqpoint{1.533800in}{1.736791in}}{\pgfqpoint{1.530528in}{1.728890in}}{\pgfqpoint{1.530528in}{1.720654in}}%
\pgfpathcurveto{\pgfqpoint{1.530528in}{1.712418in}}{\pgfqpoint{1.533800in}{1.704518in}}{\pgfqpoint{1.539624in}{1.698694in}}%
\pgfpathcurveto{\pgfqpoint{1.545448in}{1.692870in}}{\pgfqpoint{1.553348in}{1.689598in}}{\pgfqpoint{1.561585in}{1.689598in}}%
\pgfpathclose%
\pgfusepath{stroke,fill}%
\end{pgfscope}%
\begin{pgfscope}%
\pgfpathrectangle{\pgfqpoint{0.100000in}{0.212622in}}{\pgfqpoint{3.696000in}{3.696000in}}%
\pgfusepath{clip}%
\pgfsetbuttcap%
\pgfsetroundjoin%
\definecolor{currentfill}{rgb}{0.121569,0.466667,0.705882}%
\pgfsetfillcolor{currentfill}%
\pgfsetfillopacity{0.972954}%
\pgfsetlinewidth{1.003750pt}%
\definecolor{currentstroke}{rgb}{0.121569,0.466667,0.705882}%
\pgfsetstrokecolor{currentstroke}%
\pgfsetstrokeopacity{0.972954}%
\pgfsetdash{}{0pt}%
\pgfpathmoveto{\pgfqpoint{1.568543in}{1.682888in}}%
\pgfpathcurveto{\pgfqpoint{1.576780in}{1.682888in}}{\pgfqpoint{1.584680in}{1.686160in}}{\pgfqpoint{1.590504in}{1.691984in}}%
\pgfpathcurveto{\pgfqpoint{1.596328in}{1.697808in}}{\pgfqpoint{1.599600in}{1.705708in}}{\pgfqpoint{1.599600in}{1.713944in}}%
\pgfpathcurveto{\pgfqpoint{1.599600in}{1.722181in}}{\pgfqpoint{1.596328in}{1.730081in}}{\pgfqpoint{1.590504in}{1.735905in}}%
\pgfpathcurveto{\pgfqpoint{1.584680in}{1.741728in}}{\pgfqpoint{1.576780in}{1.745001in}}{\pgfqpoint{1.568543in}{1.745001in}}%
\pgfpathcurveto{\pgfqpoint{1.560307in}{1.745001in}}{\pgfqpoint{1.552407in}{1.741728in}}{\pgfqpoint{1.546583in}{1.735905in}}%
\pgfpathcurveto{\pgfqpoint{1.540759in}{1.730081in}}{\pgfqpoint{1.537487in}{1.722181in}}{\pgfqpoint{1.537487in}{1.713944in}}%
\pgfpathcurveto{\pgfqpoint{1.537487in}{1.705708in}}{\pgfqpoint{1.540759in}{1.697808in}}{\pgfqpoint{1.546583in}{1.691984in}}%
\pgfpathcurveto{\pgfqpoint{1.552407in}{1.686160in}}{\pgfqpoint{1.560307in}{1.682888in}}{\pgfqpoint{1.568543in}{1.682888in}}%
\pgfpathclose%
\pgfusepath{stroke,fill}%
\end{pgfscope}%
\begin{pgfscope}%
\pgfpathrectangle{\pgfqpoint{0.100000in}{0.212622in}}{\pgfqpoint{3.696000in}{3.696000in}}%
\pgfusepath{clip}%
\pgfsetbuttcap%
\pgfsetroundjoin%
\definecolor{currentfill}{rgb}{0.121569,0.466667,0.705882}%
\pgfsetfillcolor{currentfill}%
\pgfsetfillopacity{0.973203}%
\pgfsetlinewidth{1.003750pt}%
\definecolor{currentstroke}{rgb}{0.121569,0.466667,0.705882}%
\pgfsetstrokecolor{currentstroke}%
\pgfsetstrokeopacity{0.973203}%
\pgfsetdash{}{0pt}%
\pgfpathmoveto{\pgfqpoint{1.804030in}{1.497270in}}%
\pgfpathcurveto{\pgfqpoint{1.812266in}{1.497270in}}{\pgfqpoint{1.820166in}{1.500542in}}{\pgfqpoint{1.825990in}{1.506366in}}%
\pgfpathcurveto{\pgfqpoint{1.831814in}{1.512190in}}{\pgfqpoint{1.835086in}{1.520090in}}{\pgfqpoint{1.835086in}{1.528327in}}%
\pgfpathcurveto{\pgfqpoint{1.835086in}{1.536563in}}{\pgfqpoint{1.831814in}{1.544463in}}{\pgfqpoint{1.825990in}{1.550287in}}%
\pgfpathcurveto{\pgfqpoint{1.820166in}{1.556111in}}{\pgfqpoint{1.812266in}{1.559383in}}{\pgfqpoint{1.804030in}{1.559383in}}%
\pgfpathcurveto{\pgfqpoint{1.795793in}{1.559383in}}{\pgfqpoint{1.787893in}{1.556111in}}{\pgfqpoint{1.782069in}{1.550287in}}%
\pgfpathcurveto{\pgfqpoint{1.776245in}{1.544463in}}{\pgfqpoint{1.772973in}{1.536563in}}{\pgfqpoint{1.772973in}{1.528327in}}%
\pgfpathcurveto{\pgfqpoint{1.772973in}{1.520090in}}{\pgfqpoint{1.776245in}{1.512190in}}{\pgfqpoint{1.782069in}{1.506366in}}%
\pgfpathcurveto{\pgfqpoint{1.787893in}{1.500542in}}{\pgfqpoint{1.795793in}{1.497270in}}{\pgfqpoint{1.804030in}{1.497270in}}%
\pgfpathclose%
\pgfusepath{stroke,fill}%
\end{pgfscope}%
\begin{pgfscope}%
\pgfpathrectangle{\pgfqpoint{0.100000in}{0.212622in}}{\pgfqpoint{3.696000in}{3.696000in}}%
\pgfusepath{clip}%
\pgfsetbuttcap%
\pgfsetroundjoin%
\definecolor{currentfill}{rgb}{0.121569,0.466667,0.705882}%
\pgfsetfillcolor{currentfill}%
\pgfsetfillopacity{0.973824}%
\pgfsetlinewidth{1.003750pt}%
\definecolor{currentstroke}{rgb}{0.121569,0.466667,0.705882}%
\pgfsetstrokecolor{currentstroke}%
\pgfsetstrokeopacity{0.973824}%
\pgfsetdash{}{0pt}%
\pgfpathmoveto{\pgfqpoint{1.574436in}{1.679163in}}%
\pgfpathcurveto{\pgfqpoint{1.582672in}{1.679163in}}{\pgfqpoint{1.590572in}{1.682435in}}{\pgfqpoint{1.596396in}{1.688259in}}%
\pgfpathcurveto{\pgfqpoint{1.602220in}{1.694083in}}{\pgfqpoint{1.605492in}{1.701983in}}{\pgfqpoint{1.605492in}{1.710220in}}%
\pgfpathcurveto{\pgfqpoint{1.605492in}{1.718456in}}{\pgfqpoint{1.602220in}{1.726356in}}{\pgfqpoint{1.596396in}{1.732180in}}%
\pgfpathcurveto{\pgfqpoint{1.590572in}{1.738004in}}{\pgfqpoint{1.582672in}{1.741276in}}{\pgfqpoint{1.574436in}{1.741276in}}%
\pgfpathcurveto{\pgfqpoint{1.566199in}{1.741276in}}{\pgfqpoint{1.558299in}{1.738004in}}{\pgfqpoint{1.552475in}{1.732180in}}%
\pgfpathcurveto{\pgfqpoint{1.546652in}{1.726356in}}{\pgfqpoint{1.543379in}{1.718456in}}{\pgfqpoint{1.543379in}{1.710220in}}%
\pgfpathcurveto{\pgfqpoint{1.543379in}{1.701983in}}{\pgfqpoint{1.546652in}{1.694083in}}{\pgfqpoint{1.552475in}{1.688259in}}%
\pgfpathcurveto{\pgfqpoint{1.558299in}{1.682435in}}{\pgfqpoint{1.566199in}{1.679163in}}{\pgfqpoint{1.574436in}{1.679163in}}%
\pgfpathclose%
\pgfusepath{stroke,fill}%
\end{pgfscope}%
\begin{pgfscope}%
\pgfpathrectangle{\pgfqpoint{0.100000in}{0.212622in}}{\pgfqpoint{3.696000in}{3.696000in}}%
\pgfusepath{clip}%
\pgfsetbuttcap%
\pgfsetroundjoin%
\definecolor{currentfill}{rgb}{0.121569,0.466667,0.705882}%
\pgfsetfillcolor{currentfill}%
\pgfsetfillopacity{0.974015}%
\pgfsetlinewidth{1.003750pt}%
\definecolor{currentstroke}{rgb}{0.121569,0.466667,0.705882}%
\pgfsetstrokecolor{currentstroke}%
\pgfsetstrokeopacity{0.974015}%
\pgfsetdash{}{0pt}%
\pgfpathmoveto{\pgfqpoint{1.578811in}{1.674278in}}%
\pgfpathcurveto{\pgfqpoint{1.587047in}{1.674278in}}{\pgfqpoint{1.594947in}{1.677550in}}{\pgfqpoint{1.600771in}{1.683374in}}%
\pgfpathcurveto{\pgfqpoint{1.606595in}{1.689198in}}{\pgfqpoint{1.609867in}{1.697098in}}{\pgfqpoint{1.609867in}{1.705334in}}%
\pgfpathcurveto{\pgfqpoint{1.609867in}{1.713571in}}{\pgfqpoint{1.606595in}{1.721471in}}{\pgfqpoint{1.600771in}{1.727295in}}%
\pgfpathcurveto{\pgfqpoint{1.594947in}{1.733118in}}{\pgfqpoint{1.587047in}{1.736391in}}{\pgfqpoint{1.578811in}{1.736391in}}%
\pgfpathcurveto{\pgfqpoint{1.570575in}{1.736391in}}{\pgfqpoint{1.562675in}{1.733118in}}{\pgfqpoint{1.556851in}{1.727295in}}%
\pgfpathcurveto{\pgfqpoint{1.551027in}{1.721471in}}{\pgfqpoint{1.547754in}{1.713571in}}{\pgfqpoint{1.547754in}{1.705334in}}%
\pgfpathcurveto{\pgfqpoint{1.547754in}{1.697098in}}{\pgfqpoint{1.551027in}{1.689198in}}{\pgfqpoint{1.556851in}{1.683374in}}%
\pgfpathcurveto{\pgfqpoint{1.562675in}{1.677550in}}{\pgfqpoint{1.570575in}{1.674278in}}{\pgfqpoint{1.578811in}{1.674278in}}%
\pgfpathclose%
\pgfusepath{stroke,fill}%
\end{pgfscope}%
\begin{pgfscope}%
\pgfpathrectangle{\pgfqpoint{0.100000in}{0.212622in}}{\pgfqpoint{3.696000in}{3.696000in}}%
\pgfusepath{clip}%
\pgfsetbuttcap%
\pgfsetroundjoin%
\definecolor{currentfill}{rgb}{0.121569,0.466667,0.705882}%
\pgfsetfillcolor{currentfill}%
\pgfsetfillopacity{0.974837}%
\pgfsetlinewidth{1.003750pt}%
\definecolor{currentstroke}{rgb}{0.121569,0.466667,0.705882}%
\pgfsetstrokecolor{currentstroke}%
\pgfsetstrokeopacity{0.974837}%
\pgfsetdash{}{0pt}%
\pgfpathmoveto{\pgfqpoint{1.586971in}{1.668207in}}%
\pgfpathcurveto{\pgfqpoint{1.595207in}{1.668207in}}{\pgfqpoint{1.603107in}{1.671480in}}{\pgfqpoint{1.608931in}{1.677303in}}%
\pgfpathcurveto{\pgfqpoint{1.614755in}{1.683127in}}{\pgfqpoint{1.618027in}{1.691027in}}{\pgfqpoint{1.618027in}{1.699264in}}%
\pgfpathcurveto{\pgfqpoint{1.618027in}{1.707500in}}{\pgfqpoint{1.614755in}{1.715400in}}{\pgfqpoint{1.608931in}{1.721224in}}%
\pgfpathcurveto{\pgfqpoint{1.603107in}{1.727048in}}{\pgfqpoint{1.595207in}{1.730320in}}{\pgfqpoint{1.586971in}{1.730320in}}%
\pgfpathcurveto{\pgfqpoint{1.578734in}{1.730320in}}{\pgfqpoint{1.570834in}{1.727048in}}{\pgfqpoint{1.565011in}{1.721224in}}%
\pgfpathcurveto{\pgfqpoint{1.559187in}{1.715400in}}{\pgfqpoint{1.555914in}{1.707500in}}{\pgfqpoint{1.555914in}{1.699264in}}%
\pgfpathcurveto{\pgfqpoint{1.555914in}{1.691027in}}{\pgfqpoint{1.559187in}{1.683127in}}{\pgfqpoint{1.565011in}{1.677303in}}%
\pgfpathcurveto{\pgfqpoint{1.570834in}{1.671480in}}{\pgfqpoint{1.578734in}{1.668207in}}{\pgfqpoint{1.586971in}{1.668207in}}%
\pgfpathclose%
\pgfusepath{stroke,fill}%
\end{pgfscope}%
\begin{pgfscope}%
\pgfpathrectangle{\pgfqpoint{0.100000in}{0.212622in}}{\pgfqpoint{3.696000in}{3.696000in}}%
\pgfusepath{clip}%
\pgfsetbuttcap%
\pgfsetroundjoin%
\definecolor{currentfill}{rgb}{0.121569,0.466667,0.705882}%
\pgfsetfillcolor{currentfill}%
\pgfsetfillopacity{0.975754}%
\pgfsetlinewidth{1.003750pt}%
\definecolor{currentstroke}{rgb}{0.121569,0.466667,0.705882}%
\pgfsetstrokecolor{currentstroke}%
\pgfsetstrokeopacity{0.975754}%
\pgfsetdash{}{0pt}%
\pgfpathmoveto{\pgfqpoint{1.592658in}{1.666137in}}%
\pgfpathcurveto{\pgfqpoint{1.600895in}{1.666137in}}{\pgfqpoint{1.608795in}{1.669410in}}{\pgfqpoint{1.614618in}{1.675234in}}%
\pgfpathcurveto{\pgfqpoint{1.620442in}{1.681058in}}{\pgfqpoint{1.623715in}{1.688958in}}{\pgfqpoint{1.623715in}{1.697194in}}%
\pgfpathcurveto{\pgfqpoint{1.623715in}{1.705430in}}{\pgfqpoint{1.620442in}{1.713330in}}{\pgfqpoint{1.614618in}{1.719154in}}%
\pgfpathcurveto{\pgfqpoint{1.608795in}{1.724978in}}{\pgfqpoint{1.600895in}{1.728250in}}{\pgfqpoint{1.592658in}{1.728250in}}%
\pgfpathcurveto{\pgfqpoint{1.584422in}{1.728250in}}{\pgfqpoint{1.576522in}{1.724978in}}{\pgfqpoint{1.570698in}{1.719154in}}%
\pgfpathcurveto{\pgfqpoint{1.564874in}{1.713330in}}{\pgfqpoint{1.561602in}{1.705430in}}{\pgfqpoint{1.561602in}{1.697194in}}%
\pgfpathcurveto{\pgfqpoint{1.561602in}{1.688958in}}{\pgfqpoint{1.564874in}{1.681058in}}{\pgfqpoint{1.570698in}{1.675234in}}%
\pgfpathcurveto{\pgfqpoint{1.576522in}{1.669410in}}{\pgfqpoint{1.584422in}{1.666137in}}{\pgfqpoint{1.592658in}{1.666137in}}%
\pgfpathclose%
\pgfusepath{stroke,fill}%
\end{pgfscope}%
\begin{pgfscope}%
\pgfpathrectangle{\pgfqpoint{0.100000in}{0.212622in}}{\pgfqpoint{3.696000in}{3.696000in}}%
\pgfusepath{clip}%
\pgfsetbuttcap%
\pgfsetroundjoin%
\definecolor{currentfill}{rgb}{0.121569,0.466667,0.705882}%
\pgfsetfillcolor{currentfill}%
\pgfsetfillopacity{0.976362}%
\pgfsetlinewidth{1.003750pt}%
\definecolor{currentstroke}{rgb}{0.121569,0.466667,0.705882}%
\pgfsetstrokecolor{currentstroke}%
\pgfsetstrokeopacity{0.976362}%
\pgfsetdash{}{0pt}%
\pgfpathmoveto{\pgfqpoint{1.602312in}{1.655309in}}%
\pgfpathcurveto{\pgfqpoint{1.610548in}{1.655309in}}{\pgfqpoint{1.618448in}{1.658581in}}{\pgfqpoint{1.624272in}{1.664405in}}%
\pgfpathcurveto{\pgfqpoint{1.630096in}{1.670229in}}{\pgfqpoint{1.633368in}{1.678129in}}{\pgfqpoint{1.633368in}{1.686366in}}%
\pgfpathcurveto{\pgfqpoint{1.633368in}{1.694602in}}{\pgfqpoint{1.630096in}{1.702502in}}{\pgfqpoint{1.624272in}{1.708326in}}%
\pgfpathcurveto{\pgfqpoint{1.618448in}{1.714150in}}{\pgfqpoint{1.610548in}{1.717422in}}{\pgfqpoint{1.602312in}{1.717422in}}%
\pgfpathcurveto{\pgfqpoint{1.594076in}{1.717422in}}{\pgfqpoint{1.586176in}{1.714150in}}{\pgfqpoint{1.580352in}{1.708326in}}%
\pgfpathcurveto{\pgfqpoint{1.574528in}{1.702502in}}{\pgfqpoint{1.571255in}{1.694602in}}{\pgfqpoint{1.571255in}{1.686366in}}%
\pgfpathcurveto{\pgfqpoint{1.571255in}{1.678129in}}{\pgfqpoint{1.574528in}{1.670229in}}{\pgfqpoint{1.580352in}{1.664405in}}%
\pgfpathcurveto{\pgfqpoint{1.586176in}{1.658581in}}{\pgfqpoint{1.594076in}{1.655309in}}{\pgfqpoint{1.602312in}{1.655309in}}%
\pgfpathclose%
\pgfusepath{stroke,fill}%
\end{pgfscope}%
\begin{pgfscope}%
\pgfpathrectangle{\pgfqpoint{0.100000in}{0.212622in}}{\pgfqpoint{3.696000in}{3.696000in}}%
\pgfusepath{clip}%
\pgfsetbuttcap%
\pgfsetroundjoin%
\definecolor{currentfill}{rgb}{0.121569,0.466667,0.705882}%
\pgfsetfillcolor{currentfill}%
\pgfsetfillopacity{0.977296}%
\pgfsetlinewidth{1.003750pt}%
\definecolor{currentstroke}{rgb}{0.121569,0.466667,0.705882}%
\pgfsetstrokecolor{currentstroke}%
\pgfsetstrokeopacity{0.977296}%
\pgfsetdash{}{0pt}%
\pgfpathmoveto{\pgfqpoint{1.806027in}{1.493922in}}%
\pgfpathcurveto{\pgfqpoint{1.814264in}{1.493922in}}{\pgfqpoint{1.822164in}{1.497195in}}{\pgfqpoint{1.827988in}{1.503019in}}%
\pgfpathcurveto{\pgfqpoint{1.833812in}{1.508843in}}{\pgfqpoint{1.837084in}{1.516743in}}{\pgfqpoint{1.837084in}{1.524979in}}%
\pgfpathcurveto{\pgfqpoint{1.837084in}{1.533215in}}{\pgfqpoint{1.833812in}{1.541115in}}{\pgfqpoint{1.827988in}{1.546939in}}%
\pgfpathcurveto{\pgfqpoint{1.822164in}{1.552763in}}{\pgfqpoint{1.814264in}{1.556035in}}{\pgfqpoint{1.806027in}{1.556035in}}%
\pgfpathcurveto{\pgfqpoint{1.797791in}{1.556035in}}{\pgfqpoint{1.789891in}{1.552763in}}{\pgfqpoint{1.784067in}{1.546939in}}%
\pgfpathcurveto{\pgfqpoint{1.778243in}{1.541115in}}{\pgfqpoint{1.774971in}{1.533215in}}{\pgfqpoint{1.774971in}{1.524979in}}%
\pgfpathcurveto{\pgfqpoint{1.774971in}{1.516743in}}{\pgfqpoint{1.778243in}{1.508843in}}{\pgfqpoint{1.784067in}{1.503019in}}%
\pgfpathcurveto{\pgfqpoint{1.789891in}{1.497195in}}{\pgfqpoint{1.797791in}{1.493922in}}{\pgfqpoint{1.806027in}{1.493922in}}%
\pgfpathclose%
\pgfusepath{stroke,fill}%
\end{pgfscope}%
\begin{pgfscope}%
\pgfpathrectangle{\pgfqpoint{0.100000in}{0.212622in}}{\pgfqpoint{3.696000in}{3.696000in}}%
\pgfusepath{clip}%
\pgfsetbuttcap%
\pgfsetroundjoin%
\definecolor{currentfill}{rgb}{0.121569,0.466667,0.705882}%
\pgfsetfillcolor{currentfill}%
\pgfsetfillopacity{0.977652}%
\pgfsetlinewidth{1.003750pt}%
\definecolor{currentstroke}{rgb}{0.121569,0.466667,0.705882}%
\pgfsetstrokecolor{currentstroke}%
\pgfsetstrokeopacity{0.977652}%
\pgfsetdash{}{0pt}%
\pgfpathmoveto{\pgfqpoint{1.610224in}{1.650372in}}%
\pgfpathcurveto{\pgfqpoint{1.618460in}{1.650372in}}{\pgfqpoint{1.626360in}{1.653644in}}{\pgfqpoint{1.632184in}{1.659468in}}%
\pgfpathcurveto{\pgfqpoint{1.638008in}{1.665292in}}{\pgfqpoint{1.641280in}{1.673192in}}{\pgfqpoint{1.641280in}{1.681428in}}%
\pgfpathcurveto{\pgfqpoint{1.641280in}{1.689665in}}{\pgfqpoint{1.638008in}{1.697565in}}{\pgfqpoint{1.632184in}{1.703389in}}%
\pgfpathcurveto{\pgfqpoint{1.626360in}{1.709213in}}{\pgfqpoint{1.618460in}{1.712485in}}{\pgfqpoint{1.610224in}{1.712485in}}%
\pgfpathcurveto{\pgfqpoint{1.601987in}{1.712485in}}{\pgfqpoint{1.594087in}{1.709213in}}{\pgfqpoint{1.588263in}{1.703389in}}%
\pgfpathcurveto{\pgfqpoint{1.582439in}{1.697565in}}{\pgfqpoint{1.579167in}{1.689665in}}{\pgfqpoint{1.579167in}{1.681428in}}%
\pgfpathcurveto{\pgfqpoint{1.579167in}{1.673192in}}{\pgfqpoint{1.582439in}{1.665292in}}{\pgfqpoint{1.588263in}{1.659468in}}%
\pgfpathcurveto{\pgfqpoint{1.594087in}{1.653644in}}{\pgfqpoint{1.601987in}{1.650372in}}{\pgfqpoint{1.610224in}{1.650372in}}%
\pgfpathclose%
\pgfusepath{stroke,fill}%
\end{pgfscope}%
\begin{pgfscope}%
\pgfpathrectangle{\pgfqpoint{0.100000in}{0.212622in}}{\pgfqpoint{3.696000in}{3.696000in}}%
\pgfusepath{clip}%
\pgfsetbuttcap%
\pgfsetroundjoin%
\definecolor{currentfill}{rgb}{0.121569,0.466667,0.705882}%
\pgfsetfillcolor{currentfill}%
\pgfsetfillopacity{0.977773}%
\pgfsetlinewidth{1.003750pt}%
\definecolor{currentstroke}{rgb}{0.121569,0.466667,0.705882}%
\pgfsetstrokecolor{currentstroke}%
\pgfsetstrokeopacity{0.977773}%
\pgfsetdash{}{0pt}%
\pgfpathmoveto{\pgfqpoint{1.624687in}{1.631594in}}%
\pgfpathcurveto{\pgfqpoint{1.632924in}{1.631594in}}{\pgfqpoint{1.640824in}{1.634866in}}{\pgfqpoint{1.646648in}{1.640690in}}%
\pgfpathcurveto{\pgfqpoint{1.652472in}{1.646514in}}{\pgfqpoint{1.655744in}{1.654414in}}{\pgfqpoint{1.655744in}{1.662650in}}%
\pgfpathcurveto{\pgfqpoint{1.655744in}{1.670887in}}{\pgfqpoint{1.652472in}{1.678787in}}{\pgfqpoint{1.646648in}{1.684611in}}%
\pgfpathcurveto{\pgfqpoint{1.640824in}{1.690435in}}{\pgfqpoint{1.632924in}{1.693707in}}{\pgfqpoint{1.624687in}{1.693707in}}%
\pgfpathcurveto{\pgfqpoint{1.616451in}{1.693707in}}{\pgfqpoint{1.608551in}{1.690435in}}{\pgfqpoint{1.602727in}{1.684611in}}%
\pgfpathcurveto{\pgfqpoint{1.596903in}{1.678787in}}{\pgfqpoint{1.593631in}{1.670887in}}{\pgfqpoint{1.593631in}{1.662650in}}%
\pgfpathcurveto{\pgfqpoint{1.593631in}{1.654414in}}{\pgfqpoint{1.596903in}{1.646514in}}{\pgfqpoint{1.602727in}{1.640690in}}%
\pgfpathcurveto{\pgfqpoint{1.608551in}{1.634866in}}{\pgfqpoint{1.616451in}{1.631594in}}{\pgfqpoint{1.624687in}{1.631594in}}%
\pgfpathclose%
\pgfusepath{stroke,fill}%
\end{pgfscope}%
\begin{pgfscope}%
\pgfpathrectangle{\pgfqpoint{0.100000in}{0.212622in}}{\pgfqpoint{3.696000in}{3.696000in}}%
\pgfusepath{clip}%
\pgfsetbuttcap%
\pgfsetroundjoin%
\definecolor{currentfill}{rgb}{0.121569,0.466667,0.705882}%
\pgfsetfillcolor{currentfill}%
\pgfsetfillopacity{0.979010}%
\pgfsetlinewidth{1.003750pt}%
\definecolor{currentstroke}{rgb}{0.121569,0.466667,0.705882}%
\pgfsetstrokecolor{currentstroke}%
\pgfsetstrokeopacity{0.979010}%
\pgfsetdash{}{0pt}%
\pgfpathmoveto{\pgfqpoint{1.636824in}{1.621288in}}%
\pgfpathcurveto{\pgfqpoint{1.645061in}{1.621288in}}{\pgfqpoint{1.652961in}{1.624561in}}{\pgfqpoint{1.658785in}{1.630384in}}%
\pgfpathcurveto{\pgfqpoint{1.664609in}{1.636208in}}{\pgfqpoint{1.667881in}{1.644108in}}{\pgfqpoint{1.667881in}{1.652345in}}%
\pgfpathcurveto{\pgfqpoint{1.667881in}{1.660581in}}{\pgfqpoint{1.664609in}{1.668481in}}{\pgfqpoint{1.658785in}{1.674305in}}%
\pgfpathcurveto{\pgfqpoint{1.652961in}{1.680129in}}{\pgfqpoint{1.645061in}{1.683401in}}{\pgfqpoint{1.636824in}{1.683401in}}%
\pgfpathcurveto{\pgfqpoint{1.628588in}{1.683401in}}{\pgfqpoint{1.620688in}{1.680129in}}{\pgfqpoint{1.614864in}{1.674305in}}%
\pgfpathcurveto{\pgfqpoint{1.609040in}{1.668481in}}{\pgfqpoint{1.605768in}{1.660581in}}{\pgfqpoint{1.605768in}{1.652345in}}%
\pgfpathcurveto{\pgfqpoint{1.605768in}{1.644108in}}{\pgfqpoint{1.609040in}{1.636208in}}{\pgfqpoint{1.614864in}{1.630384in}}%
\pgfpathcurveto{\pgfqpoint{1.620688in}{1.624561in}}{\pgfqpoint{1.628588in}{1.621288in}}{\pgfqpoint{1.636824in}{1.621288in}}%
\pgfpathclose%
\pgfusepath{stroke,fill}%
\end{pgfscope}%
\begin{pgfscope}%
\pgfpathrectangle{\pgfqpoint{0.100000in}{0.212622in}}{\pgfqpoint{3.696000in}{3.696000in}}%
\pgfusepath{clip}%
\pgfsetbuttcap%
\pgfsetroundjoin%
\definecolor{currentfill}{rgb}{0.121569,0.466667,0.705882}%
\pgfsetfillcolor{currentfill}%
\pgfsetfillopacity{0.980290}%
\pgfsetlinewidth{1.003750pt}%
\definecolor{currentstroke}{rgb}{0.121569,0.466667,0.705882}%
\pgfsetstrokecolor{currentstroke}%
\pgfsetstrokeopacity{0.980290}%
\pgfsetdash{}{0pt}%
\pgfpathmoveto{\pgfqpoint{1.647825in}{1.609251in}}%
\pgfpathcurveto{\pgfqpoint{1.656061in}{1.609251in}}{\pgfqpoint{1.663961in}{1.612524in}}{\pgfqpoint{1.669785in}{1.618348in}}%
\pgfpathcurveto{\pgfqpoint{1.675609in}{1.624172in}}{\pgfqpoint{1.678881in}{1.632072in}}{\pgfqpoint{1.678881in}{1.640308in}}%
\pgfpathcurveto{\pgfqpoint{1.678881in}{1.648544in}}{\pgfqpoint{1.675609in}{1.656444in}}{\pgfqpoint{1.669785in}{1.662268in}}%
\pgfpathcurveto{\pgfqpoint{1.663961in}{1.668092in}}{\pgfqpoint{1.656061in}{1.671364in}}{\pgfqpoint{1.647825in}{1.671364in}}%
\pgfpathcurveto{\pgfqpoint{1.639588in}{1.671364in}}{\pgfqpoint{1.631688in}{1.668092in}}{\pgfqpoint{1.625864in}{1.662268in}}%
\pgfpathcurveto{\pgfqpoint{1.620040in}{1.656444in}}{\pgfqpoint{1.616768in}{1.648544in}}{\pgfqpoint{1.616768in}{1.640308in}}%
\pgfpathcurveto{\pgfqpoint{1.616768in}{1.632072in}}{\pgfqpoint{1.620040in}{1.624172in}}{\pgfqpoint{1.625864in}{1.618348in}}%
\pgfpathcurveto{\pgfqpoint{1.631688in}{1.612524in}}{\pgfqpoint{1.639588in}{1.609251in}}{\pgfqpoint{1.647825in}{1.609251in}}%
\pgfpathclose%
\pgfusepath{stroke,fill}%
\end{pgfscope}%
\begin{pgfscope}%
\pgfpathrectangle{\pgfqpoint{0.100000in}{0.212622in}}{\pgfqpoint{3.696000in}{3.696000in}}%
\pgfusepath{clip}%
\pgfsetbuttcap%
\pgfsetroundjoin%
\definecolor{currentfill}{rgb}{0.121569,0.466667,0.705882}%
\pgfsetfillcolor{currentfill}%
\pgfsetfillopacity{0.981386}%
\pgfsetlinewidth{1.003750pt}%
\definecolor{currentstroke}{rgb}{0.121569,0.466667,0.705882}%
\pgfsetstrokecolor{currentstroke}%
\pgfsetstrokeopacity{0.981386}%
\pgfsetdash{}{0pt}%
\pgfpathmoveto{\pgfqpoint{1.658227in}{1.600435in}}%
\pgfpathcurveto{\pgfqpoint{1.666464in}{1.600435in}}{\pgfqpoint{1.674364in}{1.603707in}}{\pgfqpoint{1.680188in}{1.609531in}}%
\pgfpathcurveto{\pgfqpoint{1.686012in}{1.615355in}}{\pgfqpoint{1.689284in}{1.623255in}}{\pgfqpoint{1.689284in}{1.631491in}}%
\pgfpathcurveto{\pgfqpoint{1.689284in}{1.639728in}}{\pgfqpoint{1.686012in}{1.647628in}}{\pgfqpoint{1.680188in}{1.653452in}}%
\pgfpathcurveto{\pgfqpoint{1.674364in}{1.659276in}}{\pgfqpoint{1.666464in}{1.662548in}}{\pgfqpoint{1.658227in}{1.662548in}}%
\pgfpathcurveto{\pgfqpoint{1.649991in}{1.662548in}}{\pgfqpoint{1.642091in}{1.659276in}}{\pgfqpoint{1.636267in}{1.653452in}}%
\pgfpathcurveto{\pgfqpoint{1.630443in}{1.647628in}}{\pgfqpoint{1.627171in}{1.639728in}}{\pgfqpoint{1.627171in}{1.631491in}}%
\pgfpathcurveto{\pgfqpoint{1.627171in}{1.623255in}}{\pgfqpoint{1.630443in}{1.615355in}}{\pgfqpoint{1.636267in}{1.609531in}}%
\pgfpathcurveto{\pgfqpoint{1.642091in}{1.603707in}}{\pgfqpoint{1.649991in}{1.600435in}}{\pgfqpoint{1.658227in}{1.600435in}}%
\pgfpathclose%
\pgfusepath{stroke,fill}%
\end{pgfscope}%
\begin{pgfscope}%
\pgfpathrectangle{\pgfqpoint{0.100000in}{0.212622in}}{\pgfqpoint{3.696000in}{3.696000in}}%
\pgfusepath{clip}%
\pgfsetbuttcap%
\pgfsetroundjoin%
\definecolor{currentfill}{rgb}{0.121569,0.466667,0.705882}%
\pgfsetfillcolor{currentfill}%
\pgfsetfillopacity{0.982210}%
\pgfsetlinewidth{1.003750pt}%
\definecolor{currentstroke}{rgb}{0.121569,0.466667,0.705882}%
\pgfsetstrokecolor{currentstroke}%
\pgfsetstrokeopacity{0.982210}%
\pgfsetdash{}{0pt}%
\pgfpathmoveto{\pgfqpoint{1.808760in}{1.493456in}}%
\pgfpathcurveto{\pgfqpoint{1.816997in}{1.493456in}}{\pgfqpoint{1.824897in}{1.496728in}}{\pgfqpoint{1.830721in}{1.502552in}}%
\pgfpathcurveto{\pgfqpoint{1.836545in}{1.508376in}}{\pgfqpoint{1.839817in}{1.516276in}}{\pgfqpoint{1.839817in}{1.524512in}}%
\pgfpathcurveto{\pgfqpoint{1.839817in}{1.532748in}}{\pgfqpoint{1.836545in}{1.540649in}}{\pgfqpoint{1.830721in}{1.546472in}}%
\pgfpathcurveto{\pgfqpoint{1.824897in}{1.552296in}}{\pgfqpoint{1.816997in}{1.555569in}}{\pgfqpoint{1.808760in}{1.555569in}}%
\pgfpathcurveto{\pgfqpoint{1.800524in}{1.555569in}}{\pgfqpoint{1.792624in}{1.552296in}}{\pgfqpoint{1.786800in}{1.546472in}}%
\pgfpathcurveto{\pgfqpoint{1.780976in}{1.540649in}}{\pgfqpoint{1.777704in}{1.532748in}}{\pgfqpoint{1.777704in}{1.524512in}}%
\pgfpathcurveto{\pgfqpoint{1.777704in}{1.516276in}}{\pgfqpoint{1.780976in}{1.508376in}}{\pgfqpoint{1.786800in}{1.502552in}}%
\pgfpathcurveto{\pgfqpoint{1.792624in}{1.496728in}}{\pgfqpoint{1.800524in}{1.493456in}}{\pgfqpoint{1.808760in}{1.493456in}}%
\pgfpathclose%
\pgfusepath{stroke,fill}%
\end{pgfscope}%
\begin{pgfscope}%
\pgfpathrectangle{\pgfqpoint{0.100000in}{0.212622in}}{\pgfqpoint{3.696000in}{3.696000in}}%
\pgfusepath{clip}%
\pgfsetbuttcap%
\pgfsetroundjoin%
\definecolor{currentfill}{rgb}{0.121569,0.466667,0.705882}%
\pgfsetfillcolor{currentfill}%
\pgfsetfillopacity{0.983463}%
\pgfsetlinewidth{1.003750pt}%
\definecolor{currentstroke}{rgb}{0.121569,0.466667,0.705882}%
\pgfsetstrokecolor{currentstroke}%
\pgfsetstrokeopacity{0.983463}%
\pgfsetdash{}{0pt}%
\pgfpathmoveto{\pgfqpoint{1.667857in}{1.595269in}}%
\pgfpathcurveto{\pgfqpoint{1.676094in}{1.595269in}}{\pgfqpoint{1.683994in}{1.598541in}}{\pgfqpoint{1.689818in}{1.604365in}}%
\pgfpathcurveto{\pgfqpoint{1.695642in}{1.610189in}}{\pgfqpoint{1.698914in}{1.618089in}}{\pgfqpoint{1.698914in}{1.626325in}}%
\pgfpathcurveto{\pgfqpoint{1.698914in}{1.634561in}}{\pgfqpoint{1.695642in}{1.642461in}}{\pgfqpoint{1.689818in}{1.648285in}}%
\pgfpathcurveto{\pgfqpoint{1.683994in}{1.654109in}}{\pgfqpoint{1.676094in}{1.657382in}}{\pgfqpoint{1.667857in}{1.657382in}}%
\pgfpathcurveto{\pgfqpoint{1.659621in}{1.657382in}}{\pgfqpoint{1.651721in}{1.654109in}}{\pgfqpoint{1.645897in}{1.648285in}}%
\pgfpathcurveto{\pgfqpoint{1.640073in}{1.642461in}}{\pgfqpoint{1.636801in}{1.634561in}}{\pgfqpoint{1.636801in}{1.626325in}}%
\pgfpathcurveto{\pgfqpoint{1.636801in}{1.618089in}}{\pgfqpoint{1.640073in}{1.610189in}}{\pgfqpoint{1.645897in}{1.604365in}}%
\pgfpathcurveto{\pgfqpoint{1.651721in}{1.598541in}}{\pgfqpoint{1.659621in}{1.595269in}}{\pgfqpoint{1.667857in}{1.595269in}}%
\pgfpathclose%
\pgfusepath{stroke,fill}%
\end{pgfscope}%
\begin{pgfscope}%
\pgfpathrectangle{\pgfqpoint{0.100000in}{0.212622in}}{\pgfqpoint{3.696000in}{3.696000in}}%
\pgfusepath{clip}%
\pgfsetbuttcap%
\pgfsetroundjoin%
\definecolor{currentfill}{rgb}{0.121569,0.466667,0.705882}%
\pgfsetfillcolor{currentfill}%
\pgfsetfillopacity{0.984684}%
\pgfsetlinewidth{1.003750pt}%
\definecolor{currentstroke}{rgb}{0.121569,0.466667,0.705882}%
\pgfsetstrokecolor{currentstroke}%
\pgfsetstrokeopacity{0.984684}%
\pgfsetdash{}{0pt}%
\pgfpathmoveto{\pgfqpoint{1.675750in}{1.589251in}}%
\pgfpathcurveto{\pgfqpoint{1.683986in}{1.589251in}}{\pgfqpoint{1.691886in}{1.592523in}}{\pgfqpoint{1.697710in}{1.598347in}}%
\pgfpathcurveto{\pgfqpoint{1.703534in}{1.604171in}}{\pgfqpoint{1.706806in}{1.612071in}}{\pgfqpoint{1.706806in}{1.620308in}}%
\pgfpathcurveto{\pgfqpoint{1.706806in}{1.628544in}}{\pgfqpoint{1.703534in}{1.636444in}}{\pgfqpoint{1.697710in}{1.642268in}}%
\pgfpathcurveto{\pgfqpoint{1.691886in}{1.648092in}}{\pgfqpoint{1.683986in}{1.651364in}}{\pgfqpoint{1.675750in}{1.651364in}}%
\pgfpathcurveto{\pgfqpoint{1.667514in}{1.651364in}}{\pgfqpoint{1.659614in}{1.648092in}}{\pgfqpoint{1.653790in}{1.642268in}}%
\pgfpathcurveto{\pgfqpoint{1.647966in}{1.636444in}}{\pgfqpoint{1.644693in}{1.628544in}}{\pgfqpoint{1.644693in}{1.620308in}}%
\pgfpathcurveto{\pgfqpoint{1.644693in}{1.612071in}}{\pgfqpoint{1.647966in}{1.604171in}}{\pgfqpoint{1.653790in}{1.598347in}}%
\pgfpathcurveto{\pgfqpoint{1.659614in}{1.592523in}}{\pgfqpoint{1.667514in}{1.589251in}}{\pgfqpoint{1.675750in}{1.589251in}}%
\pgfpathclose%
\pgfusepath{stroke,fill}%
\end{pgfscope}%
\begin{pgfscope}%
\pgfpathrectangle{\pgfqpoint{0.100000in}{0.212622in}}{\pgfqpoint{3.696000in}{3.696000in}}%
\pgfusepath{clip}%
\pgfsetbuttcap%
\pgfsetroundjoin%
\definecolor{currentfill}{rgb}{0.121569,0.466667,0.705882}%
\pgfsetfillcolor{currentfill}%
\pgfsetfillopacity{0.985984}%
\pgfsetlinewidth{1.003750pt}%
\definecolor{currentstroke}{rgb}{0.121569,0.466667,0.705882}%
\pgfsetstrokecolor{currentstroke}%
\pgfsetstrokeopacity{0.985984}%
\pgfsetdash{}{0pt}%
\pgfpathmoveto{\pgfqpoint{1.682923in}{1.586947in}}%
\pgfpathcurveto{\pgfqpoint{1.691160in}{1.586947in}}{\pgfqpoint{1.699060in}{1.590219in}}{\pgfqpoint{1.704884in}{1.596043in}}%
\pgfpathcurveto{\pgfqpoint{1.710708in}{1.601867in}}{\pgfqpoint{1.713980in}{1.609767in}}{\pgfqpoint{1.713980in}{1.618004in}}%
\pgfpathcurveto{\pgfqpoint{1.713980in}{1.626240in}}{\pgfqpoint{1.710708in}{1.634140in}}{\pgfqpoint{1.704884in}{1.639964in}}%
\pgfpathcurveto{\pgfqpoint{1.699060in}{1.645788in}}{\pgfqpoint{1.691160in}{1.649060in}}{\pgfqpoint{1.682923in}{1.649060in}}%
\pgfpathcurveto{\pgfqpoint{1.674687in}{1.649060in}}{\pgfqpoint{1.666787in}{1.645788in}}{\pgfqpoint{1.660963in}{1.639964in}}%
\pgfpathcurveto{\pgfqpoint{1.655139in}{1.634140in}}{\pgfqpoint{1.651867in}{1.626240in}}{\pgfqpoint{1.651867in}{1.618004in}}%
\pgfpathcurveto{\pgfqpoint{1.651867in}{1.609767in}}{\pgfqpoint{1.655139in}{1.601867in}}{\pgfqpoint{1.660963in}{1.596043in}}%
\pgfpathcurveto{\pgfqpoint{1.666787in}{1.590219in}}{\pgfqpoint{1.674687in}{1.586947in}}{\pgfqpoint{1.682923in}{1.586947in}}%
\pgfpathclose%
\pgfusepath{stroke,fill}%
\end{pgfscope}%
\begin{pgfscope}%
\pgfpathrectangle{\pgfqpoint{0.100000in}{0.212622in}}{\pgfqpoint{3.696000in}{3.696000in}}%
\pgfusepath{clip}%
\pgfsetbuttcap%
\pgfsetroundjoin%
\definecolor{currentfill}{rgb}{0.121569,0.466667,0.705882}%
\pgfsetfillcolor{currentfill}%
\pgfsetfillopacity{0.987105}%
\pgfsetlinewidth{1.003750pt}%
\definecolor{currentstroke}{rgb}{0.121569,0.466667,0.705882}%
\pgfsetstrokecolor{currentstroke}%
\pgfsetstrokeopacity{0.987105}%
\pgfsetdash{}{0pt}%
\pgfpathmoveto{\pgfqpoint{1.694617in}{1.573125in}}%
\pgfpathcurveto{\pgfqpoint{1.702854in}{1.573125in}}{\pgfqpoint{1.710754in}{1.576397in}}{\pgfqpoint{1.716578in}{1.582221in}}%
\pgfpathcurveto{\pgfqpoint{1.722402in}{1.588045in}}{\pgfqpoint{1.725674in}{1.595945in}}{\pgfqpoint{1.725674in}{1.604181in}}%
\pgfpathcurveto{\pgfqpoint{1.725674in}{1.612417in}}{\pgfqpoint{1.722402in}{1.620317in}}{\pgfqpoint{1.716578in}{1.626141in}}%
\pgfpathcurveto{\pgfqpoint{1.710754in}{1.631965in}}{\pgfqpoint{1.702854in}{1.635238in}}{\pgfqpoint{1.694617in}{1.635238in}}%
\pgfpathcurveto{\pgfqpoint{1.686381in}{1.635238in}}{\pgfqpoint{1.678481in}{1.631965in}}{\pgfqpoint{1.672657in}{1.626141in}}%
\pgfpathcurveto{\pgfqpoint{1.666833in}{1.620317in}}{\pgfqpoint{1.663561in}{1.612417in}}{\pgfqpoint{1.663561in}{1.604181in}}%
\pgfpathcurveto{\pgfqpoint{1.663561in}{1.595945in}}{\pgfqpoint{1.666833in}{1.588045in}}{\pgfqpoint{1.672657in}{1.582221in}}%
\pgfpathcurveto{\pgfqpoint{1.678481in}{1.576397in}}{\pgfqpoint{1.686381in}{1.573125in}}{\pgfqpoint{1.694617in}{1.573125in}}%
\pgfpathclose%
\pgfusepath{stroke,fill}%
\end{pgfscope}%
\begin{pgfscope}%
\pgfpathrectangle{\pgfqpoint{0.100000in}{0.212622in}}{\pgfqpoint{3.696000in}{3.696000in}}%
\pgfusepath{clip}%
\pgfsetbuttcap%
\pgfsetroundjoin%
\definecolor{currentfill}{rgb}{0.121569,0.466667,0.705882}%
\pgfsetfillcolor{currentfill}%
\pgfsetfillopacity{0.987144}%
\pgfsetlinewidth{1.003750pt}%
\definecolor{currentstroke}{rgb}{0.121569,0.466667,0.705882}%
\pgfsetstrokecolor{currentstroke}%
\pgfsetstrokeopacity{0.987144}%
\pgfsetdash{}{0pt}%
\pgfpathmoveto{\pgfqpoint{1.811001in}{1.491448in}}%
\pgfpathcurveto{\pgfqpoint{1.819237in}{1.491448in}}{\pgfqpoint{1.827137in}{1.494721in}}{\pgfqpoint{1.832961in}{1.500545in}}%
\pgfpathcurveto{\pgfqpoint{1.838785in}{1.506368in}}{\pgfqpoint{1.842057in}{1.514268in}}{\pgfqpoint{1.842057in}{1.522505in}}%
\pgfpathcurveto{\pgfqpoint{1.842057in}{1.530741in}}{\pgfqpoint{1.838785in}{1.538641in}}{\pgfqpoint{1.832961in}{1.544465in}}%
\pgfpathcurveto{\pgfqpoint{1.827137in}{1.550289in}}{\pgfqpoint{1.819237in}{1.553561in}}{\pgfqpoint{1.811001in}{1.553561in}}%
\pgfpathcurveto{\pgfqpoint{1.802764in}{1.553561in}}{\pgfqpoint{1.794864in}{1.550289in}}{\pgfqpoint{1.789040in}{1.544465in}}%
\pgfpathcurveto{\pgfqpoint{1.783216in}{1.538641in}}{\pgfqpoint{1.779944in}{1.530741in}}{\pgfqpoint{1.779944in}{1.522505in}}%
\pgfpathcurveto{\pgfqpoint{1.779944in}{1.514268in}}{\pgfqpoint{1.783216in}{1.506368in}}{\pgfqpoint{1.789040in}{1.500545in}}%
\pgfpathcurveto{\pgfqpoint{1.794864in}{1.494721in}}{\pgfqpoint{1.802764in}{1.491448in}}{\pgfqpoint{1.811001in}{1.491448in}}%
\pgfpathclose%
\pgfusepath{stroke,fill}%
\end{pgfscope}%
\begin{pgfscope}%
\pgfpathrectangle{\pgfqpoint{0.100000in}{0.212622in}}{\pgfqpoint{3.696000in}{3.696000in}}%
\pgfusepath{clip}%
\pgfsetbuttcap%
\pgfsetroundjoin%
\definecolor{currentfill}{rgb}{0.121569,0.466667,0.705882}%
\pgfsetfillcolor{currentfill}%
\pgfsetfillopacity{0.989001}%
\pgfsetlinewidth{1.003750pt}%
\definecolor{currentstroke}{rgb}{0.121569,0.466667,0.705882}%
\pgfsetstrokecolor{currentstroke}%
\pgfsetstrokeopacity{0.989001}%
\pgfsetdash{}{0pt}%
\pgfpathmoveto{\pgfqpoint{1.716755in}{1.551337in}}%
\pgfpathcurveto{\pgfqpoint{1.724991in}{1.551337in}}{\pgfqpoint{1.732891in}{1.554609in}}{\pgfqpoint{1.738715in}{1.560433in}}%
\pgfpathcurveto{\pgfqpoint{1.744539in}{1.566257in}}{\pgfqpoint{1.747811in}{1.574157in}}{\pgfqpoint{1.747811in}{1.582393in}}%
\pgfpathcurveto{\pgfqpoint{1.747811in}{1.590630in}}{\pgfqpoint{1.744539in}{1.598530in}}{\pgfqpoint{1.738715in}{1.604354in}}%
\pgfpathcurveto{\pgfqpoint{1.732891in}{1.610177in}}{\pgfqpoint{1.724991in}{1.613450in}}{\pgfqpoint{1.716755in}{1.613450in}}%
\pgfpathcurveto{\pgfqpoint{1.708518in}{1.613450in}}{\pgfqpoint{1.700618in}{1.610177in}}{\pgfqpoint{1.694794in}{1.604354in}}%
\pgfpathcurveto{\pgfqpoint{1.688970in}{1.598530in}}{\pgfqpoint{1.685698in}{1.590630in}}{\pgfqpoint{1.685698in}{1.582393in}}%
\pgfpathcurveto{\pgfqpoint{1.685698in}{1.574157in}}{\pgfqpoint{1.688970in}{1.566257in}}{\pgfqpoint{1.694794in}{1.560433in}}%
\pgfpathcurveto{\pgfqpoint{1.700618in}{1.554609in}}{\pgfqpoint{1.708518in}{1.551337in}}{\pgfqpoint{1.716755in}{1.551337in}}%
\pgfpathclose%
\pgfusepath{stroke,fill}%
\end{pgfscope}%
\begin{pgfscope}%
\pgfpathrectangle{\pgfqpoint{0.100000in}{0.212622in}}{\pgfqpoint{3.696000in}{3.696000in}}%
\pgfusepath{clip}%
\pgfsetbuttcap%
\pgfsetroundjoin%
\definecolor{currentfill}{rgb}{0.121569,0.466667,0.705882}%
\pgfsetfillcolor{currentfill}%
\pgfsetfillopacity{0.989220}%
\pgfsetlinewidth{1.003750pt}%
\definecolor{currentstroke}{rgb}{0.121569,0.466667,0.705882}%
\pgfsetstrokecolor{currentstroke}%
\pgfsetstrokeopacity{0.989220}%
\pgfsetdash{}{0pt}%
\pgfpathmoveto{\pgfqpoint{1.705097in}{1.564139in}}%
\pgfpathcurveto{\pgfqpoint{1.713334in}{1.564139in}}{\pgfqpoint{1.721234in}{1.567411in}}{\pgfqpoint{1.727058in}{1.573235in}}%
\pgfpathcurveto{\pgfqpoint{1.732882in}{1.579059in}}{\pgfqpoint{1.736154in}{1.586959in}}{\pgfqpoint{1.736154in}{1.595195in}}%
\pgfpathcurveto{\pgfqpoint{1.736154in}{1.603431in}}{\pgfqpoint{1.732882in}{1.611331in}}{\pgfqpoint{1.727058in}{1.617155in}}%
\pgfpathcurveto{\pgfqpoint{1.721234in}{1.622979in}}{\pgfqpoint{1.713334in}{1.626252in}}{\pgfqpoint{1.705097in}{1.626252in}}%
\pgfpathcurveto{\pgfqpoint{1.696861in}{1.626252in}}{\pgfqpoint{1.688961in}{1.622979in}}{\pgfqpoint{1.683137in}{1.617155in}}%
\pgfpathcurveto{\pgfqpoint{1.677313in}{1.611331in}}{\pgfqpoint{1.674041in}{1.603431in}}{\pgfqpoint{1.674041in}{1.595195in}}%
\pgfpathcurveto{\pgfqpoint{1.674041in}{1.586959in}}{\pgfqpoint{1.677313in}{1.579059in}}{\pgfqpoint{1.683137in}{1.573235in}}%
\pgfpathcurveto{\pgfqpoint{1.688961in}{1.567411in}}{\pgfqpoint{1.696861in}{1.564139in}}{\pgfqpoint{1.705097in}{1.564139in}}%
\pgfpathclose%
\pgfusepath{stroke,fill}%
\end{pgfscope}%
\begin{pgfscope}%
\pgfpathrectangle{\pgfqpoint{0.100000in}{0.212622in}}{\pgfqpoint{3.696000in}{3.696000in}}%
\pgfusepath{clip}%
\pgfsetbuttcap%
\pgfsetroundjoin%
\definecolor{currentfill}{rgb}{0.121569,0.466667,0.705882}%
\pgfsetfillcolor{currentfill}%
\pgfsetfillopacity{0.991095}%
\pgfsetlinewidth{1.003750pt}%
\definecolor{currentstroke}{rgb}{0.121569,0.466667,0.705882}%
\pgfsetstrokecolor{currentstroke}%
\pgfsetstrokeopacity{0.991095}%
\pgfsetdash{}{0pt}%
\pgfpathmoveto{\pgfqpoint{1.725817in}{1.548017in}}%
\pgfpathcurveto{\pgfqpoint{1.734053in}{1.548017in}}{\pgfqpoint{1.741953in}{1.551290in}}{\pgfqpoint{1.747777in}{1.557114in}}%
\pgfpathcurveto{\pgfqpoint{1.753601in}{1.562938in}}{\pgfqpoint{1.756874in}{1.570838in}}{\pgfqpoint{1.756874in}{1.579074in}}%
\pgfpathcurveto{\pgfqpoint{1.756874in}{1.587310in}}{\pgfqpoint{1.753601in}{1.595210in}}{\pgfqpoint{1.747777in}{1.601034in}}%
\pgfpathcurveto{\pgfqpoint{1.741953in}{1.606858in}}{\pgfqpoint{1.734053in}{1.610130in}}{\pgfqpoint{1.725817in}{1.610130in}}%
\pgfpathcurveto{\pgfqpoint{1.717581in}{1.610130in}}{\pgfqpoint{1.709681in}{1.606858in}}{\pgfqpoint{1.703857in}{1.601034in}}%
\pgfpathcurveto{\pgfqpoint{1.698033in}{1.595210in}}{\pgfqpoint{1.694761in}{1.587310in}}{\pgfqpoint{1.694761in}{1.579074in}}%
\pgfpathcurveto{\pgfqpoint{1.694761in}{1.570838in}}{\pgfqpoint{1.698033in}{1.562938in}}{\pgfqpoint{1.703857in}{1.557114in}}%
\pgfpathcurveto{\pgfqpoint{1.709681in}{1.551290in}}{\pgfqpoint{1.717581in}{1.548017in}}{\pgfqpoint{1.725817in}{1.548017in}}%
\pgfpathclose%
\pgfusepath{stroke,fill}%
\end{pgfscope}%
\begin{pgfscope}%
\pgfpathrectangle{\pgfqpoint{0.100000in}{0.212622in}}{\pgfqpoint{3.696000in}{3.696000in}}%
\pgfusepath{clip}%
\pgfsetbuttcap%
\pgfsetroundjoin%
\definecolor{currentfill}{rgb}{0.121569,0.466667,0.705882}%
\pgfsetfillcolor{currentfill}%
\pgfsetfillopacity{0.991555}%
\pgfsetlinewidth{1.003750pt}%
\definecolor{currentstroke}{rgb}{0.121569,0.466667,0.705882}%
\pgfsetstrokecolor{currentstroke}%
\pgfsetstrokeopacity{0.991555}%
\pgfsetdash{}{0pt}%
\pgfpathmoveto{\pgfqpoint{1.813358in}{1.485898in}}%
\pgfpathcurveto{\pgfqpoint{1.821594in}{1.485898in}}{\pgfqpoint{1.829494in}{1.489171in}}{\pgfqpoint{1.835318in}{1.494995in}}%
\pgfpathcurveto{\pgfqpoint{1.841142in}{1.500819in}}{\pgfqpoint{1.844414in}{1.508719in}}{\pgfqpoint{1.844414in}{1.516955in}}%
\pgfpathcurveto{\pgfqpoint{1.844414in}{1.525191in}}{\pgfqpoint{1.841142in}{1.533091in}}{\pgfqpoint{1.835318in}{1.538915in}}%
\pgfpathcurveto{\pgfqpoint{1.829494in}{1.544739in}}{\pgfqpoint{1.821594in}{1.548011in}}{\pgfqpoint{1.813358in}{1.548011in}}%
\pgfpathcurveto{\pgfqpoint{1.805122in}{1.548011in}}{\pgfqpoint{1.797222in}{1.544739in}}{\pgfqpoint{1.791398in}{1.538915in}}%
\pgfpathcurveto{\pgfqpoint{1.785574in}{1.533091in}}{\pgfqpoint{1.782301in}{1.525191in}}{\pgfqpoint{1.782301in}{1.516955in}}%
\pgfpathcurveto{\pgfqpoint{1.782301in}{1.508719in}}{\pgfqpoint{1.785574in}{1.500819in}}{\pgfqpoint{1.791398in}{1.494995in}}%
\pgfpathcurveto{\pgfqpoint{1.797222in}{1.489171in}}{\pgfqpoint{1.805122in}{1.485898in}}{\pgfqpoint{1.813358in}{1.485898in}}%
\pgfpathclose%
\pgfusepath{stroke,fill}%
\end{pgfscope}%
\begin{pgfscope}%
\pgfpathrectangle{\pgfqpoint{0.100000in}{0.212622in}}{\pgfqpoint{3.696000in}{3.696000in}}%
\pgfusepath{clip}%
\pgfsetbuttcap%
\pgfsetroundjoin%
\definecolor{currentfill}{rgb}{0.121569,0.466667,0.705882}%
\pgfsetfillcolor{currentfill}%
\pgfsetfillopacity{0.992689}%
\pgfsetlinewidth{1.003750pt}%
\definecolor{currentstroke}{rgb}{0.121569,0.466667,0.705882}%
\pgfsetstrokecolor{currentstroke}%
\pgfsetstrokeopacity{0.992689}%
\pgfsetdash{}{0pt}%
\pgfpathmoveto{\pgfqpoint{1.734259in}{1.544331in}}%
\pgfpathcurveto{\pgfqpoint{1.742496in}{1.544331in}}{\pgfqpoint{1.750396in}{1.547603in}}{\pgfqpoint{1.756219in}{1.553427in}}%
\pgfpathcurveto{\pgfqpoint{1.762043in}{1.559251in}}{\pgfqpoint{1.765316in}{1.567151in}}{\pgfqpoint{1.765316in}{1.575387in}}%
\pgfpathcurveto{\pgfqpoint{1.765316in}{1.583624in}}{\pgfqpoint{1.762043in}{1.591524in}}{\pgfqpoint{1.756219in}{1.597348in}}%
\pgfpathcurveto{\pgfqpoint{1.750396in}{1.603172in}}{\pgfqpoint{1.742496in}{1.606444in}}{\pgfqpoint{1.734259in}{1.606444in}}%
\pgfpathcurveto{\pgfqpoint{1.726023in}{1.606444in}}{\pgfqpoint{1.718123in}{1.603172in}}{\pgfqpoint{1.712299in}{1.597348in}}%
\pgfpathcurveto{\pgfqpoint{1.706475in}{1.591524in}}{\pgfqpoint{1.703203in}{1.583624in}}{\pgfqpoint{1.703203in}{1.575387in}}%
\pgfpathcurveto{\pgfqpoint{1.703203in}{1.567151in}}{\pgfqpoint{1.706475in}{1.559251in}}{\pgfqpoint{1.712299in}{1.553427in}}%
\pgfpathcurveto{\pgfqpoint{1.718123in}{1.547603in}}{\pgfqpoint{1.726023in}{1.544331in}}{\pgfqpoint{1.734259in}{1.544331in}}%
\pgfpathclose%
\pgfusepath{stroke,fill}%
\end{pgfscope}%
\begin{pgfscope}%
\pgfpathrectangle{\pgfqpoint{0.100000in}{0.212622in}}{\pgfqpoint{3.696000in}{3.696000in}}%
\pgfusepath{clip}%
\pgfsetbuttcap%
\pgfsetroundjoin%
\definecolor{currentfill}{rgb}{0.121569,0.466667,0.705882}%
\pgfsetfillcolor{currentfill}%
\pgfsetfillopacity{0.992816}%
\pgfsetlinewidth{1.003750pt}%
\definecolor{currentstroke}{rgb}{0.121569,0.466667,0.705882}%
\pgfsetstrokecolor{currentstroke}%
\pgfsetstrokeopacity{0.992816}%
\pgfsetdash{}{0pt}%
\pgfpathmoveto{\pgfqpoint{1.741462in}{1.533576in}}%
\pgfpathcurveto{\pgfqpoint{1.749698in}{1.533576in}}{\pgfqpoint{1.757598in}{1.536849in}}{\pgfqpoint{1.763422in}{1.542673in}}%
\pgfpathcurveto{\pgfqpoint{1.769246in}{1.548497in}}{\pgfqpoint{1.772518in}{1.556397in}}{\pgfqpoint{1.772518in}{1.564633in}}%
\pgfpathcurveto{\pgfqpoint{1.772518in}{1.572869in}}{\pgfqpoint{1.769246in}{1.580769in}}{\pgfqpoint{1.763422in}{1.586593in}}%
\pgfpathcurveto{\pgfqpoint{1.757598in}{1.592417in}}{\pgfqpoint{1.749698in}{1.595689in}}{\pgfqpoint{1.741462in}{1.595689in}}%
\pgfpathcurveto{\pgfqpoint{1.733225in}{1.595689in}}{\pgfqpoint{1.725325in}{1.592417in}}{\pgfqpoint{1.719501in}{1.586593in}}%
\pgfpathcurveto{\pgfqpoint{1.713678in}{1.580769in}}{\pgfqpoint{1.710405in}{1.572869in}}{\pgfqpoint{1.710405in}{1.564633in}}%
\pgfpathcurveto{\pgfqpoint{1.710405in}{1.556397in}}{\pgfqpoint{1.713678in}{1.548497in}}{\pgfqpoint{1.719501in}{1.542673in}}%
\pgfpathcurveto{\pgfqpoint{1.725325in}{1.536849in}}{\pgfqpoint{1.733225in}{1.533576in}}{\pgfqpoint{1.741462in}{1.533576in}}%
\pgfpathclose%
\pgfusepath{stroke,fill}%
\end{pgfscope}%
\begin{pgfscope}%
\pgfpathrectangle{\pgfqpoint{0.100000in}{0.212622in}}{\pgfqpoint{3.696000in}{3.696000in}}%
\pgfusepath{clip}%
\pgfsetbuttcap%
\pgfsetroundjoin%
\definecolor{currentfill}{rgb}{0.121569,0.466667,0.705882}%
\pgfsetfillcolor{currentfill}%
\pgfsetfillopacity{0.993750}%
\pgfsetlinewidth{1.003750pt}%
\definecolor{currentstroke}{rgb}{0.121569,0.466667,0.705882}%
\pgfsetstrokecolor{currentstroke}%
\pgfsetstrokeopacity{0.993750}%
\pgfsetdash{}{0pt}%
\pgfpathmoveto{\pgfqpoint{1.748059in}{1.529078in}}%
\pgfpathcurveto{\pgfqpoint{1.756296in}{1.529078in}}{\pgfqpoint{1.764196in}{1.532350in}}{\pgfqpoint{1.770019in}{1.538174in}}%
\pgfpathcurveto{\pgfqpoint{1.775843in}{1.543998in}}{\pgfqpoint{1.779116in}{1.551898in}}{\pgfqpoint{1.779116in}{1.560134in}}%
\pgfpathcurveto{\pgfqpoint{1.779116in}{1.568371in}}{\pgfqpoint{1.775843in}{1.576271in}}{\pgfqpoint{1.770019in}{1.582095in}}%
\pgfpathcurveto{\pgfqpoint{1.764196in}{1.587919in}}{\pgfqpoint{1.756296in}{1.591191in}}{\pgfqpoint{1.748059in}{1.591191in}}%
\pgfpathcurveto{\pgfqpoint{1.739823in}{1.591191in}}{\pgfqpoint{1.731923in}{1.587919in}}{\pgfqpoint{1.726099in}{1.582095in}}%
\pgfpathcurveto{\pgfqpoint{1.720275in}{1.576271in}}{\pgfqpoint{1.717003in}{1.568371in}}{\pgfqpoint{1.717003in}{1.560134in}}%
\pgfpathcurveto{\pgfqpoint{1.717003in}{1.551898in}}{\pgfqpoint{1.720275in}{1.543998in}}{\pgfqpoint{1.726099in}{1.538174in}}%
\pgfpathcurveto{\pgfqpoint{1.731923in}{1.532350in}}{\pgfqpoint{1.739823in}{1.529078in}}{\pgfqpoint{1.748059in}{1.529078in}}%
\pgfpathclose%
\pgfusepath{stroke,fill}%
\end{pgfscope}%
\begin{pgfscope}%
\pgfpathrectangle{\pgfqpoint{0.100000in}{0.212622in}}{\pgfqpoint{3.696000in}{3.696000in}}%
\pgfusepath{clip}%
\pgfsetbuttcap%
\pgfsetroundjoin%
\definecolor{currentfill}{rgb}{0.121569,0.466667,0.705882}%
\pgfsetfillcolor{currentfill}%
\pgfsetfillopacity{0.994288}%
\pgfsetlinewidth{1.003750pt}%
\definecolor{currentstroke}{rgb}{0.121569,0.466667,0.705882}%
\pgfsetstrokecolor{currentstroke}%
\pgfsetstrokeopacity{0.994288}%
\pgfsetdash{}{0pt}%
\pgfpathmoveto{\pgfqpoint{1.814428in}{1.484080in}}%
\pgfpathcurveto{\pgfqpoint{1.822664in}{1.484080in}}{\pgfqpoint{1.830564in}{1.487352in}}{\pgfqpoint{1.836388in}{1.493176in}}%
\pgfpathcurveto{\pgfqpoint{1.842212in}{1.499000in}}{\pgfqpoint{1.845484in}{1.506900in}}{\pgfqpoint{1.845484in}{1.515136in}}%
\pgfpathcurveto{\pgfqpoint{1.845484in}{1.523372in}}{\pgfqpoint{1.842212in}{1.531273in}}{\pgfqpoint{1.836388in}{1.537096in}}%
\pgfpathcurveto{\pgfqpoint{1.830564in}{1.542920in}}{\pgfqpoint{1.822664in}{1.546193in}}{\pgfqpoint{1.814428in}{1.546193in}}%
\pgfpathcurveto{\pgfqpoint{1.806191in}{1.546193in}}{\pgfqpoint{1.798291in}{1.542920in}}{\pgfqpoint{1.792467in}{1.537096in}}%
\pgfpathcurveto{\pgfqpoint{1.786643in}{1.531273in}}{\pgfqpoint{1.783371in}{1.523372in}}{\pgfqpoint{1.783371in}{1.515136in}}%
\pgfpathcurveto{\pgfqpoint{1.783371in}{1.506900in}}{\pgfqpoint{1.786643in}{1.499000in}}{\pgfqpoint{1.792467in}{1.493176in}}%
\pgfpathcurveto{\pgfqpoint{1.798291in}{1.487352in}}{\pgfqpoint{1.806191in}{1.484080in}}{\pgfqpoint{1.814428in}{1.484080in}}%
\pgfpathclose%
\pgfusepath{stroke,fill}%
\end{pgfscope}%
\begin{pgfscope}%
\pgfpathrectangle{\pgfqpoint{0.100000in}{0.212622in}}{\pgfqpoint{3.696000in}{3.696000in}}%
\pgfusepath{clip}%
\pgfsetbuttcap%
\pgfsetroundjoin%
\definecolor{currentfill}{rgb}{0.121569,0.466667,0.705882}%
\pgfsetfillcolor{currentfill}%
\pgfsetfillopacity{0.994403}%
\pgfsetlinewidth{1.003750pt}%
\definecolor{currentstroke}{rgb}{0.121569,0.466667,0.705882}%
\pgfsetstrokecolor{currentstroke}%
\pgfsetstrokeopacity{0.994403}%
\pgfsetdash{}{0pt}%
\pgfpathmoveto{\pgfqpoint{1.752546in}{1.526021in}}%
\pgfpathcurveto{\pgfqpoint{1.760782in}{1.526021in}}{\pgfqpoint{1.768682in}{1.529293in}}{\pgfqpoint{1.774506in}{1.535117in}}%
\pgfpathcurveto{\pgfqpoint{1.780330in}{1.540941in}}{\pgfqpoint{1.783602in}{1.548841in}}{\pgfqpoint{1.783602in}{1.557077in}}%
\pgfpathcurveto{\pgfqpoint{1.783602in}{1.565314in}}{\pgfqpoint{1.780330in}{1.573214in}}{\pgfqpoint{1.774506in}{1.579038in}}%
\pgfpathcurveto{\pgfqpoint{1.768682in}{1.584862in}}{\pgfqpoint{1.760782in}{1.588134in}}{\pgfqpoint{1.752546in}{1.588134in}}%
\pgfpathcurveto{\pgfqpoint{1.744310in}{1.588134in}}{\pgfqpoint{1.736410in}{1.584862in}}{\pgfqpoint{1.730586in}{1.579038in}}%
\pgfpathcurveto{\pgfqpoint{1.724762in}{1.573214in}}{\pgfqpoint{1.721490in}{1.565314in}}{\pgfqpoint{1.721490in}{1.557077in}}%
\pgfpathcurveto{\pgfqpoint{1.721490in}{1.548841in}}{\pgfqpoint{1.724762in}{1.540941in}}{\pgfqpoint{1.730586in}{1.535117in}}%
\pgfpathcurveto{\pgfqpoint{1.736410in}{1.529293in}}{\pgfqpoint{1.744310in}{1.526021in}}{\pgfqpoint{1.752546in}{1.526021in}}%
\pgfpathclose%
\pgfusepath{stroke,fill}%
\end{pgfscope}%
\begin{pgfscope}%
\pgfpathrectangle{\pgfqpoint{0.100000in}{0.212622in}}{\pgfqpoint{3.696000in}{3.696000in}}%
\pgfusepath{clip}%
\pgfsetbuttcap%
\pgfsetroundjoin%
\definecolor{currentfill}{rgb}{0.121569,0.466667,0.705882}%
\pgfsetfillcolor{currentfill}%
\pgfsetfillopacity{0.994888}%
\pgfsetlinewidth{1.003750pt}%
\definecolor{currentstroke}{rgb}{0.121569,0.466667,0.705882}%
\pgfsetstrokecolor{currentstroke}%
\pgfsetstrokeopacity{0.994888}%
\pgfsetdash{}{0pt}%
\pgfpathmoveto{\pgfqpoint{1.756525in}{1.522804in}}%
\pgfpathcurveto{\pgfqpoint{1.764761in}{1.522804in}}{\pgfqpoint{1.772661in}{1.526076in}}{\pgfqpoint{1.778485in}{1.531900in}}%
\pgfpathcurveto{\pgfqpoint{1.784309in}{1.537724in}}{\pgfqpoint{1.787581in}{1.545624in}}{\pgfqpoint{1.787581in}{1.553860in}}%
\pgfpathcurveto{\pgfqpoint{1.787581in}{1.562097in}}{\pgfqpoint{1.784309in}{1.569997in}}{\pgfqpoint{1.778485in}{1.575821in}}%
\pgfpathcurveto{\pgfqpoint{1.772661in}{1.581644in}}{\pgfqpoint{1.764761in}{1.584917in}}{\pgfqpoint{1.756525in}{1.584917in}}%
\pgfpathcurveto{\pgfqpoint{1.748289in}{1.584917in}}{\pgfqpoint{1.740389in}{1.581644in}}{\pgfqpoint{1.734565in}{1.575821in}}%
\pgfpathcurveto{\pgfqpoint{1.728741in}{1.569997in}}{\pgfqpoint{1.725468in}{1.562097in}}{\pgfqpoint{1.725468in}{1.553860in}}%
\pgfpathcurveto{\pgfqpoint{1.725468in}{1.545624in}}{\pgfqpoint{1.728741in}{1.537724in}}{\pgfqpoint{1.734565in}{1.531900in}}%
\pgfpathcurveto{\pgfqpoint{1.740389in}{1.526076in}}{\pgfqpoint{1.748289in}{1.522804in}}{\pgfqpoint{1.756525in}{1.522804in}}%
\pgfpathclose%
\pgfusepath{stroke,fill}%
\end{pgfscope}%
\begin{pgfscope}%
\pgfpathrectangle{\pgfqpoint{0.100000in}{0.212622in}}{\pgfqpoint{3.696000in}{3.696000in}}%
\pgfusepath{clip}%
\pgfsetbuttcap%
\pgfsetroundjoin%
\definecolor{currentfill}{rgb}{0.121569,0.466667,0.705882}%
\pgfsetfillcolor{currentfill}%
\pgfsetfillopacity{0.995199}%
\pgfsetlinewidth{1.003750pt}%
\definecolor{currentstroke}{rgb}{0.121569,0.466667,0.705882}%
\pgfsetstrokecolor{currentstroke}%
\pgfsetstrokeopacity{0.995199}%
\pgfsetdash{}{0pt}%
\pgfpathmoveto{\pgfqpoint{1.759516in}{1.520323in}}%
\pgfpathcurveto{\pgfqpoint{1.767753in}{1.520323in}}{\pgfqpoint{1.775653in}{1.523595in}}{\pgfqpoint{1.781477in}{1.529419in}}%
\pgfpathcurveto{\pgfqpoint{1.787301in}{1.535243in}}{\pgfqpoint{1.790573in}{1.543143in}}{\pgfqpoint{1.790573in}{1.551379in}}%
\pgfpathcurveto{\pgfqpoint{1.790573in}{1.559616in}}{\pgfqpoint{1.787301in}{1.567516in}}{\pgfqpoint{1.781477in}{1.573340in}}%
\pgfpathcurveto{\pgfqpoint{1.775653in}{1.579163in}}{\pgfqpoint{1.767753in}{1.582436in}}{\pgfqpoint{1.759516in}{1.582436in}}%
\pgfpathcurveto{\pgfqpoint{1.751280in}{1.582436in}}{\pgfqpoint{1.743380in}{1.579163in}}{\pgfqpoint{1.737556in}{1.573340in}}%
\pgfpathcurveto{\pgfqpoint{1.731732in}{1.567516in}}{\pgfqpoint{1.728460in}{1.559616in}}{\pgfqpoint{1.728460in}{1.551379in}}%
\pgfpathcurveto{\pgfqpoint{1.728460in}{1.543143in}}{\pgfqpoint{1.731732in}{1.535243in}}{\pgfqpoint{1.737556in}{1.529419in}}%
\pgfpathcurveto{\pgfqpoint{1.743380in}{1.523595in}}{\pgfqpoint{1.751280in}{1.520323in}}{\pgfqpoint{1.759516in}{1.520323in}}%
\pgfpathclose%
\pgfusepath{stroke,fill}%
\end{pgfscope}%
\begin{pgfscope}%
\pgfpathrectangle{\pgfqpoint{0.100000in}{0.212622in}}{\pgfqpoint{3.696000in}{3.696000in}}%
\pgfusepath{clip}%
\pgfsetbuttcap%
\pgfsetroundjoin%
\definecolor{currentfill}{rgb}{0.121569,0.466667,0.705882}%
\pgfsetfillcolor{currentfill}%
\pgfsetfillopacity{0.995627}%
\pgfsetlinewidth{1.003750pt}%
\definecolor{currentstroke}{rgb}{0.121569,0.466667,0.705882}%
\pgfsetstrokecolor{currentstroke}%
\pgfsetstrokeopacity{0.995627}%
\pgfsetdash{}{0pt}%
\pgfpathmoveto{\pgfqpoint{1.765148in}{1.515837in}}%
\pgfpathcurveto{\pgfqpoint{1.773384in}{1.515837in}}{\pgfqpoint{1.781284in}{1.519109in}}{\pgfqpoint{1.787108in}{1.524933in}}%
\pgfpathcurveto{\pgfqpoint{1.792932in}{1.530757in}}{\pgfqpoint{1.796204in}{1.538657in}}{\pgfqpoint{1.796204in}{1.546893in}}%
\pgfpathcurveto{\pgfqpoint{1.796204in}{1.555130in}}{\pgfqpoint{1.792932in}{1.563030in}}{\pgfqpoint{1.787108in}{1.568854in}}%
\pgfpathcurveto{\pgfqpoint{1.781284in}{1.574677in}}{\pgfqpoint{1.773384in}{1.577950in}}{\pgfqpoint{1.765148in}{1.577950in}}%
\pgfpathcurveto{\pgfqpoint{1.756912in}{1.577950in}}{\pgfqpoint{1.749011in}{1.574677in}}{\pgfqpoint{1.743188in}{1.568854in}}%
\pgfpathcurveto{\pgfqpoint{1.737364in}{1.563030in}}{\pgfqpoint{1.734091in}{1.555130in}}{\pgfqpoint{1.734091in}{1.546893in}}%
\pgfpathcurveto{\pgfqpoint{1.734091in}{1.538657in}}{\pgfqpoint{1.737364in}{1.530757in}}{\pgfqpoint{1.743188in}{1.524933in}}%
\pgfpathcurveto{\pgfqpoint{1.749011in}{1.519109in}}{\pgfqpoint{1.756912in}{1.515837in}}{\pgfqpoint{1.765148in}{1.515837in}}%
\pgfpathclose%
\pgfusepath{stroke,fill}%
\end{pgfscope}%
\begin{pgfscope}%
\pgfpathrectangle{\pgfqpoint{0.100000in}{0.212622in}}{\pgfqpoint{3.696000in}{3.696000in}}%
\pgfusepath{clip}%
\pgfsetbuttcap%
\pgfsetroundjoin%
\definecolor{currentfill}{rgb}{0.121569,0.466667,0.705882}%
\pgfsetfillcolor{currentfill}%
\pgfsetfillopacity{0.996255}%
\pgfsetlinewidth{1.003750pt}%
\definecolor{currentstroke}{rgb}{0.121569,0.466667,0.705882}%
\pgfsetstrokecolor{currentstroke}%
\pgfsetstrokeopacity{0.996255}%
\pgfsetdash{}{0pt}%
\pgfpathmoveto{\pgfqpoint{1.768422in}{1.514931in}}%
\pgfpathcurveto{\pgfqpoint{1.776659in}{1.514931in}}{\pgfqpoint{1.784559in}{1.518203in}}{\pgfqpoint{1.790383in}{1.524027in}}%
\pgfpathcurveto{\pgfqpoint{1.796206in}{1.529851in}}{\pgfqpoint{1.799479in}{1.537751in}}{\pgfqpoint{1.799479in}{1.545988in}}%
\pgfpathcurveto{\pgfqpoint{1.799479in}{1.554224in}}{\pgfqpoint{1.796206in}{1.562124in}}{\pgfqpoint{1.790383in}{1.567948in}}%
\pgfpathcurveto{\pgfqpoint{1.784559in}{1.573772in}}{\pgfqpoint{1.776659in}{1.577044in}}{\pgfqpoint{1.768422in}{1.577044in}}%
\pgfpathcurveto{\pgfqpoint{1.760186in}{1.577044in}}{\pgfqpoint{1.752286in}{1.573772in}}{\pgfqpoint{1.746462in}{1.567948in}}%
\pgfpathcurveto{\pgfqpoint{1.740638in}{1.562124in}}{\pgfqpoint{1.737366in}{1.554224in}}{\pgfqpoint{1.737366in}{1.545988in}}%
\pgfpathcurveto{\pgfqpoint{1.737366in}{1.537751in}}{\pgfqpoint{1.740638in}{1.529851in}}{\pgfqpoint{1.746462in}{1.524027in}}%
\pgfpathcurveto{\pgfqpoint{1.752286in}{1.518203in}}{\pgfqpoint{1.760186in}{1.514931in}}{\pgfqpoint{1.768422in}{1.514931in}}%
\pgfpathclose%
\pgfusepath{stroke,fill}%
\end{pgfscope}%
\begin{pgfscope}%
\pgfpathrectangle{\pgfqpoint{0.100000in}{0.212622in}}{\pgfqpoint{3.696000in}{3.696000in}}%
\pgfusepath{clip}%
\pgfsetbuttcap%
\pgfsetroundjoin%
\definecolor{currentfill}{rgb}{0.121569,0.466667,0.705882}%
\pgfsetfillcolor{currentfill}%
\pgfsetfillopacity{0.996338}%
\pgfsetlinewidth{1.003750pt}%
\definecolor{currentstroke}{rgb}{0.121569,0.466667,0.705882}%
\pgfsetstrokecolor{currentstroke}%
\pgfsetstrokeopacity{0.996338}%
\pgfsetdash{}{0pt}%
\pgfpathmoveto{\pgfqpoint{1.771038in}{1.511346in}}%
\pgfpathcurveto{\pgfqpoint{1.779275in}{1.511346in}}{\pgfqpoint{1.787175in}{1.514618in}}{\pgfqpoint{1.792999in}{1.520442in}}%
\pgfpathcurveto{\pgfqpoint{1.798822in}{1.526266in}}{\pgfqpoint{1.802095in}{1.534166in}}{\pgfqpoint{1.802095in}{1.542402in}}%
\pgfpathcurveto{\pgfqpoint{1.802095in}{1.550639in}}{\pgfqpoint{1.798822in}{1.558539in}}{\pgfqpoint{1.792999in}{1.564363in}}%
\pgfpathcurveto{\pgfqpoint{1.787175in}{1.570187in}}{\pgfqpoint{1.779275in}{1.573459in}}{\pgfqpoint{1.771038in}{1.573459in}}%
\pgfpathcurveto{\pgfqpoint{1.762802in}{1.573459in}}{\pgfqpoint{1.754902in}{1.570187in}}{\pgfqpoint{1.749078in}{1.564363in}}%
\pgfpathcurveto{\pgfqpoint{1.743254in}{1.558539in}}{\pgfqpoint{1.739982in}{1.550639in}}{\pgfqpoint{1.739982in}{1.542402in}}%
\pgfpathcurveto{\pgfqpoint{1.739982in}{1.534166in}}{\pgfqpoint{1.743254in}{1.526266in}}{\pgfqpoint{1.749078in}{1.520442in}}%
\pgfpathcurveto{\pgfqpoint{1.754902in}{1.514618in}}{\pgfqpoint{1.762802in}{1.511346in}}{\pgfqpoint{1.771038in}{1.511346in}}%
\pgfpathclose%
\pgfusepath{stroke,fill}%
\end{pgfscope}%
\begin{pgfscope}%
\pgfpathrectangle{\pgfqpoint{0.100000in}{0.212622in}}{\pgfqpoint{3.696000in}{3.696000in}}%
\pgfusepath{clip}%
\pgfsetbuttcap%
\pgfsetroundjoin%
\definecolor{currentfill}{rgb}{0.121569,0.466667,0.705882}%
\pgfsetfillcolor{currentfill}%
\pgfsetfillopacity{0.996393}%
\pgfsetlinewidth{1.003750pt}%
\definecolor{currentstroke}{rgb}{0.121569,0.466667,0.705882}%
\pgfsetstrokecolor{currentstroke}%
\pgfsetstrokeopacity{0.996393}%
\pgfsetdash{}{0pt}%
\pgfpathmoveto{\pgfqpoint{1.772250in}{1.510203in}}%
\pgfpathcurveto{\pgfqpoint{1.780486in}{1.510203in}}{\pgfqpoint{1.788386in}{1.513476in}}{\pgfqpoint{1.794210in}{1.519299in}}%
\pgfpathcurveto{\pgfqpoint{1.800034in}{1.525123in}}{\pgfqpoint{1.803306in}{1.533023in}}{\pgfqpoint{1.803306in}{1.541260in}}%
\pgfpathcurveto{\pgfqpoint{1.803306in}{1.549496in}}{\pgfqpoint{1.800034in}{1.557396in}}{\pgfqpoint{1.794210in}{1.563220in}}%
\pgfpathcurveto{\pgfqpoint{1.788386in}{1.569044in}}{\pgfqpoint{1.780486in}{1.572316in}}{\pgfqpoint{1.772250in}{1.572316in}}%
\pgfpathcurveto{\pgfqpoint{1.764013in}{1.572316in}}{\pgfqpoint{1.756113in}{1.569044in}}{\pgfqpoint{1.750290in}{1.563220in}}%
\pgfpathcurveto{\pgfqpoint{1.744466in}{1.557396in}}{\pgfqpoint{1.741193in}{1.549496in}}{\pgfqpoint{1.741193in}{1.541260in}}%
\pgfpathcurveto{\pgfqpoint{1.741193in}{1.533023in}}{\pgfqpoint{1.744466in}{1.525123in}}{\pgfqpoint{1.750290in}{1.519299in}}%
\pgfpathcurveto{\pgfqpoint{1.756113in}{1.513476in}}{\pgfqpoint{1.764013in}{1.510203in}}{\pgfqpoint{1.772250in}{1.510203in}}%
\pgfpathclose%
\pgfusepath{stroke,fill}%
\end{pgfscope}%
\begin{pgfscope}%
\pgfpathrectangle{\pgfqpoint{0.100000in}{0.212622in}}{\pgfqpoint{3.696000in}{3.696000in}}%
\pgfusepath{clip}%
\pgfsetbuttcap%
\pgfsetroundjoin%
\definecolor{currentfill}{rgb}{0.121569,0.466667,0.705882}%
\pgfsetfillcolor{currentfill}%
\pgfsetfillopacity{0.996520}%
\pgfsetlinewidth{1.003750pt}%
\definecolor{currentstroke}{rgb}{0.121569,0.466667,0.705882}%
\pgfsetstrokecolor{currentstroke}%
\pgfsetstrokeopacity{0.996520}%
\pgfsetdash{}{0pt}%
\pgfpathmoveto{\pgfqpoint{1.814152in}{1.479032in}}%
\pgfpathcurveto{\pgfqpoint{1.822388in}{1.479032in}}{\pgfqpoint{1.830288in}{1.482304in}}{\pgfqpoint{1.836112in}{1.488128in}}%
\pgfpathcurveto{\pgfqpoint{1.841936in}{1.493952in}}{\pgfqpoint{1.845208in}{1.501852in}}{\pgfqpoint{1.845208in}{1.510088in}}%
\pgfpathcurveto{\pgfqpoint{1.845208in}{1.518325in}}{\pgfqpoint{1.841936in}{1.526225in}}{\pgfqpoint{1.836112in}{1.532049in}}%
\pgfpathcurveto{\pgfqpoint{1.830288in}{1.537872in}}{\pgfqpoint{1.822388in}{1.541145in}}{\pgfqpoint{1.814152in}{1.541145in}}%
\pgfpathcurveto{\pgfqpoint{1.805915in}{1.541145in}}{\pgfqpoint{1.798015in}{1.537872in}}{\pgfqpoint{1.792192in}{1.532049in}}%
\pgfpathcurveto{\pgfqpoint{1.786368in}{1.526225in}}{\pgfqpoint{1.783095in}{1.518325in}}{\pgfqpoint{1.783095in}{1.510088in}}%
\pgfpathcurveto{\pgfqpoint{1.783095in}{1.501852in}}{\pgfqpoint{1.786368in}{1.493952in}}{\pgfqpoint{1.792192in}{1.488128in}}%
\pgfpathcurveto{\pgfqpoint{1.798015in}{1.482304in}}{\pgfqpoint{1.805915in}{1.479032in}}{\pgfqpoint{1.814152in}{1.479032in}}%
\pgfpathclose%
\pgfusepath{stroke,fill}%
\end{pgfscope}%
\begin{pgfscope}%
\pgfpathrectangle{\pgfqpoint{0.100000in}{0.212622in}}{\pgfqpoint{3.696000in}{3.696000in}}%
\pgfusepath{clip}%
\pgfsetbuttcap%
\pgfsetroundjoin%
\definecolor{currentfill}{rgb}{0.121569,0.466667,0.705882}%
\pgfsetfillcolor{currentfill}%
\pgfsetfillopacity{0.996675}%
\pgfsetlinewidth{1.003750pt}%
\definecolor{currentstroke}{rgb}{0.121569,0.466667,0.705882}%
\pgfsetstrokecolor{currentstroke}%
\pgfsetstrokeopacity{0.996675}%
\pgfsetdash{}{0pt}%
\pgfpathmoveto{\pgfqpoint{1.774434in}{1.508870in}}%
\pgfpathcurveto{\pgfqpoint{1.782670in}{1.508870in}}{\pgfqpoint{1.790570in}{1.512142in}}{\pgfqpoint{1.796394in}{1.517966in}}%
\pgfpathcurveto{\pgfqpoint{1.802218in}{1.523790in}}{\pgfqpoint{1.805491in}{1.531690in}}{\pgfqpoint{1.805491in}{1.539926in}}%
\pgfpathcurveto{\pgfqpoint{1.805491in}{1.548163in}}{\pgfqpoint{1.802218in}{1.556063in}}{\pgfqpoint{1.796394in}{1.561887in}}%
\pgfpathcurveto{\pgfqpoint{1.790570in}{1.567710in}}{\pgfqpoint{1.782670in}{1.570983in}}{\pgfqpoint{1.774434in}{1.570983in}}%
\pgfpathcurveto{\pgfqpoint{1.766198in}{1.570983in}}{\pgfqpoint{1.758298in}{1.567710in}}{\pgfqpoint{1.752474in}{1.561887in}}%
\pgfpathcurveto{\pgfqpoint{1.746650in}{1.556063in}}{\pgfqpoint{1.743378in}{1.548163in}}{\pgfqpoint{1.743378in}{1.539926in}}%
\pgfpathcurveto{\pgfqpoint{1.743378in}{1.531690in}}{\pgfqpoint{1.746650in}{1.523790in}}{\pgfqpoint{1.752474in}{1.517966in}}%
\pgfpathcurveto{\pgfqpoint{1.758298in}{1.512142in}}{\pgfqpoint{1.766198in}{1.508870in}}{\pgfqpoint{1.774434in}{1.508870in}}%
\pgfpathclose%
\pgfusepath{stroke,fill}%
\end{pgfscope}%
\begin{pgfscope}%
\pgfpathrectangle{\pgfqpoint{0.100000in}{0.212622in}}{\pgfqpoint{3.696000in}{3.696000in}}%
\pgfusepath{clip}%
\pgfsetbuttcap%
\pgfsetroundjoin%
\definecolor{currentfill}{rgb}{0.121569,0.466667,0.705882}%
\pgfsetfillcolor{currentfill}%
\pgfsetfillopacity{0.996776}%
\pgfsetlinewidth{1.003750pt}%
\definecolor{currentstroke}{rgb}{0.121569,0.466667,0.705882}%
\pgfsetstrokecolor{currentstroke}%
\pgfsetstrokeopacity{0.996776}%
\pgfsetdash{}{0pt}%
\pgfpathmoveto{\pgfqpoint{1.775201in}{1.508382in}}%
\pgfpathcurveto{\pgfqpoint{1.783437in}{1.508382in}}{\pgfqpoint{1.791337in}{1.511655in}}{\pgfqpoint{1.797161in}{1.517479in}}%
\pgfpathcurveto{\pgfqpoint{1.802985in}{1.523303in}}{\pgfqpoint{1.806257in}{1.531203in}}{\pgfqpoint{1.806257in}{1.539439in}}%
\pgfpathcurveto{\pgfqpoint{1.806257in}{1.547675in}}{\pgfqpoint{1.802985in}{1.555575in}}{\pgfqpoint{1.797161in}{1.561399in}}%
\pgfpathcurveto{\pgfqpoint{1.791337in}{1.567223in}}{\pgfqpoint{1.783437in}{1.570495in}}{\pgfqpoint{1.775201in}{1.570495in}}%
\pgfpathcurveto{\pgfqpoint{1.766964in}{1.570495in}}{\pgfqpoint{1.759064in}{1.567223in}}{\pgfqpoint{1.753240in}{1.561399in}}%
\pgfpathcurveto{\pgfqpoint{1.747416in}{1.555575in}}{\pgfqpoint{1.744144in}{1.547675in}}{\pgfqpoint{1.744144in}{1.539439in}}%
\pgfpathcurveto{\pgfqpoint{1.744144in}{1.531203in}}{\pgfqpoint{1.747416in}{1.523303in}}{\pgfqpoint{1.753240in}{1.517479in}}%
\pgfpathcurveto{\pgfqpoint{1.759064in}{1.511655in}}{\pgfqpoint{1.766964in}{1.508382in}}{\pgfqpoint{1.775201in}{1.508382in}}%
\pgfpathclose%
\pgfusepath{stroke,fill}%
\end{pgfscope}%
\begin{pgfscope}%
\pgfpathrectangle{\pgfqpoint{0.100000in}{0.212622in}}{\pgfqpoint{3.696000in}{3.696000in}}%
\pgfusepath{clip}%
\pgfsetbuttcap%
\pgfsetroundjoin%
\definecolor{currentfill}{rgb}{0.121569,0.466667,0.705882}%
\pgfsetfillcolor{currentfill}%
\pgfsetfillopacity{0.996791}%
\pgfsetlinewidth{1.003750pt}%
\definecolor{currentstroke}{rgb}{0.121569,0.466667,0.705882}%
\pgfsetstrokecolor{currentstroke}%
\pgfsetstrokeopacity{0.996791}%
\pgfsetdash{}{0pt}%
\pgfpathmoveto{\pgfqpoint{1.775359in}{1.508252in}}%
\pgfpathcurveto{\pgfqpoint{1.783596in}{1.508252in}}{\pgfqpoint{1.791496in}{1.511524in}}{\pgfqpoint{1.797320in}{1.517348in}}%
\pgfpathcurveto{\pgfqpoint{1.803143in}{1.523172in}}{\pgfqpoint{1.806416in}{1.531072in}}{\pgfqpoint{1.806416in}{1.539308in}}%
\pgfpathcurveto{\pgfqpoint{1.806416in}{1.547545in}}{\pgfqpoint{1.803143in}{1.555445in}}{\pgfqpoint{1.797320in}{1.561269in}}%
\pgfpathcurveto{\pgfqpoint{1.791496in}{1.567093in}}{\pgfqpoint{1.783596in}{1.570365in}}{\pgfqpoint{1.775359in}{1.570365in}}%
\pgfpathcurveto{\pgfqpoint{1.767123in}{1.570365in}}{\pgfqpoint{1.759223in}{1.567093in}}{\pgfqpoint{1.753399in}{1.561269in}}%
\pgfpathcurveto{\pgfqpoint{1.747575in}{1.555445in}}{\pgfqpoint{1.744303in}{1.547545in}}{\pgfqpoint{1.744303in}{1.539308in}}%
\pgfpathcurveto{\pgfqpoint{1.744303in}{1.531072in}}{\pgfqpoint{1.747575in}{1.523172in}}{\pgfqpoint{1.753399in}{1.517348in}}%
\pgfpathcurveto{\pgfqpoint{1.759223in}{1.511524in}}{\pgfqpoint{1.767123in}{1.508252in}}{\pgfqpoint{1.775359in}{1.508252in}}%
\pgfpathclose%
\pgfusepath{stroke,fill}%
\end{pgfscope}%
\begin{pgfscope}%
\pgfpathrectangle{\pgfqpoint{0.100000in}{0.212622in}}{\pgfqpoint{3.696000in}{3.696000in}}%
\pgfusepath{clip}%
\pgfsetbuttcap%
\pgfsetroundjoin%
\definecolor{currentfill}{rgb}{0.121569,0.466667,0.705882}%
\pgfsetfillcolor{currentfill}%
\pgfsetfillopacity{0.996828}%
\pgfsetlinewidth{1.003750pt}%
\definecolor{currentstroke}{rgb}{0.121569,0.466667,0.705882}%
\pgfsetstrokecolor{currentstroke}%
\pgfsetstrokeopacity{0.996828}%
\pgfsetdash{}{0pt}%
\pgfpathmoveto{\pgfqpoint{1.775643in}{1.508037in}}%
\pgfpathcurveto{\pgfqpoint{1.783880in}{1.508037in}}{\pgfqpoint{1.791780in}{1.511309in}}{\pgfqpoint{1.797604in}{1.517133in}}%
\pgfpathcurveto{\pgfqpoint{1.803427in}{1.522957in}}{\pgfqpoint{1.806700in}{1.530857in}}{\pgfqpoint{1.806700in}{1.539093in}}%
\pgfpathcurveto{\pgfqpoint{1.806700in}{1.547330in}}{\pgfqpoint{1.803427in}{1.555230in}}{\pgfqpoint{1.797604in}{1.561054in}}%
\pgfpathcurveto{\pgfqpoint{1.791780in}{1.566877in}}{\pgfqpoint{1.783880in}{1.570150in}}{\pgfqpoint{1.775643in}{1.570150in}}%
\pgfpathcurveto{\pgfqpoint{1.767407in}{1.570150in}}{\pgfqpoint{1.759507in}{1.566877in}}{\pgfqpoint{1.753683in}{1.561054in}}%
\pgfpathcurveto{\pgfqpoint{1.747859in}{1.555230in}}{\pgfqpoint{1.744587in}{1.547330in}}{\pgfqpoint{1.744587in}{1.539093in}}%
\pgfpathcurveto{\pgfqpoint{1.744587in}{1.530857in}}{\pgfqpoint{1.747859in}{1.522957in}}{\pgfqpoint{1.753683in}{1.517133in}}%
\pgfpathcurveto{\pgfqpoint{1.759507in}{1.511309in}}{\pgfqpoint{1.767407in}{1.508037in}}{\pgfqpoint{1.775643in}{1.508037in}}%
\pgfpathclose%
\pgfusepath{stroke,fill}%
\end{pgfscope}%
\begin{pgfscope}%
\pgfpathrectangle{\pgfqpoint{0.100000in}{0.212622in}}{\pgfqpoint{3.696000in}{3.696000in}}%
\pgfusepath{clip}%
\pgfsetbuttcap%
\pgfsetroundjoin%
\definecolor{currentfill}{rgb}{0.121569,0.466667,0.705882}%
\pgfsetfillcolor{currentfill}%
\pgfsetfillopacity{0.996906}%
\pgfsetlinewidth{1.003750pt}%
\definecolor{currentstroke}{rgb}{0.121569,0.466667,0.705882}%
\pgfsetstrokecolor{currentstroke}%
\pgfsetstrokeopacity{0.996906}%
\pgfsetdash{}{0pt}%
\pgfpathmoveto{\pgfqpoint{1.776141in}{1.507646in}}%
\pgfpathcurveto{\pgfqpoint{1.784378in}{1.507646in}}{\pgfqpoint{1.792278in}{1.510918in}}{\pgfqpoint{1.798102in}{1.516742in}}%
\pgfpathcurveto{\pgfqpoint{1.803926in}{1.522566in}}{\pgfqpoint{1.807198in}{1.530466in}}{\pgfqpoint{1.807198in}{1.538702in}}%
\pgfpathcurveto{\pgfqpoint{1.807198in}{1.546938in}}{\pgfqpoint{1.803926in}{1.554838in}}{\pgfqpoint{1.798102in}{1.560662in}}%
\pgfpathcurveto{\pgfqpoint{1.792278in}{1.566486in}}{\pgfqpoint{1.784378in}{1.569759in}}{\pgfqpoint{1.776141in}{1.569759in}}%
\pgfpathcurveto{\pgfqpoint{1.767905in}{1.569759in}}{\pgfqpoint{1.760005in}{1.566486in}}{\pgfqpoint{1.754181in}{1.560662in}}%
\pgfpathcurveto{\pgfqpoint{1.748357in}{1.554838in}}{\pgfqpoint{1.745085in}{1.546938in}}{\pgfqpoint{1.745085in}{1.538702in}}%
\pgfpathcurveto{\pgfqpoint{1.745085in}{1.530466in}}{\pgfqpoint{1.748357in}{1.522566in}}{\pgfqpoint{1.754181in}{1.516742in}}%
\pgfpathcurveto{\pgfqpoint{1.760005in}{1.510918in}}{\pgfqpoint{1.767905in}{1.507646in}}{\pgfqpoint{1.776141in}{1.507646in}}%
\pgfpathclose%
\pgfusepath{stroke,fill}%
\end{pgfscope}%
\begin{pgfscope}%
\pgfpathrectangle{\pgfqpoint{0.100000in}{0.212622in}}{\pgfqpoint{3.696000in}{3.696000in}}%
\pgfusepath{clip}%
\pgfsetbuttcap%
\pgfsetroundjoin%
\definecolor{currentfill}{rgb}{0.121569,0.466667,0.705882}%
\pgfsetfillcolor{currentfill}%
\pgfsetfillopacity{0.997032}%
\pgfsetlinewidth{1.003750pt}%
\definecolor{currentstroke}{rgb}{0.121569,0.466667,0.705882}%
\pgfsetstrokecolor{currentstroke}%
\pgfsetstrokeopacity{0.997032}%
\pgfsetdash{}{0pt}%
\pgfpathmoveto{\pgfqpoint{1.777033in}{1.506821in}}%
\pgfpathcurveto{\pgfqpoint{1.785269in}{1.506821in}}{\pgfqpoint{1.793170in}{1.510093in}}{\pgfqpoint{1.798993in}{1.515917in}}%
\pgfpathcurveto{\pgfqpoint{1.804817in}{1.521741in}}{\pgfqpoint{1.808090in}{1.529641in}}{\pgfqpoint{1.808090in}{1.537877in}}%
\pgfpathcurveto{\pgfqpoint{1.808090in}{1.546114in}}{\pgfqpoint{1.804817in}{1.554014in}}{\pgfqpoint{1.798993in}{1.559838in}}%
\pgfpathcurveto{\pgfqpoint{1.793170in}{1.565662in}}{\pgfqpoint{1.785269in}{1.568934in}}{\pgfqpoint{1.777033in}{1.568934in}}%
\pgfpathcurveto{\pgfqpoint{1.768797in}{1.568934in}}{\pgfqpoint{1.760897in}{1.565662in}}{\pgfqpoint{1.755073in}{1.559838in}}%
\pgfpathcurveto{\pgfqpoint{1.749249in}{1.554014in}}{\pgfqpoint{1.745977in}{1.546114in}}{\pgfqpoint{1.745977in}{1.537877in}}%
\pgfpathcurveto{\pgfqpoint{1.745977in}{1.529641in}}{\pgfqpoint{1.749249in}{1.521741in}}{\pgfqpoint{1.755073in}{1.515917in}}%
\pgfpathcurveto{\pgfqpoint{1.760897in}{1.510093in}}{\pgfqpoint{1.768797in}{1.506821in}}{\pgfqpoint{1.777033in}{1.506821in}}%
\pgfpathclose%
\pgfusepath{stroke,fill}%
\end{pgfscope}%
\begin{pgfscope}%
\pgfpathrectangle{\pgfqpoint{0.100000in}{0.212622in}}{\pgfqpoint{3.696000in}{3.696000in}}%
\pgfusepath{clip}%
\pgfsetbuttcap%
\pgfsetroundjoin%
\definecolor{currentfill}{rgb}{0.121569,0.466667,0.705882}%
\pgfsetfillcolor{currentfill}%
\pgfsetfillopacity{0.997205}%
\pgfsetlinewidth{1.003750pt}%
\definecolor{currentstroke}{rgb}{0.121569,0.466667,0.705882}%
\pgfsetstrokecolor{currentstroke}%
\pgfsetstrokeopacity{0.997205}%
\pgfsetdash{}{0pt}%
\pgfpathmoveto{\pgfqpoint{1.778681in}{1.505150in}}%
\pgfpathcurveto{\pgfqpoint{1.786917in}{1.505150in}}{\pgfqpoint{1.794817in}{1.508423in}}{\pgfqpoint{1.800641in}{1.514247in}}%
\pgfpathcurveto{\pgfqpoint{1.806465in}{1.520071in}}{\pgfqpoint{1.809737in}{1.527971in}}{\pgfqpoint{1.809737in}{1.536207in}}%
\pgfpathcurveto{\pgfqpoint{1.809737in}{1.544443in}}{\pgfqpoint{1.806465in}{1.552343in}}{\pgfqpoint{1.800641in}{1.558167in}}%
\pgfpathcurveto{\pgfqpoint{1.794817in}{1.563991in}}{\pgfqpoint{1.786917in}{1.567263in}}{\pgfqpoint{1.778681in}{1.567263in}}%
\pgfpathcurveto{\pgfqpoint{1.770445in}{1.567263in}}{\pgfqpoint{1.762545in}{1.563991in}}{\pgfqpoint{1.756721in}{1.558167in}}%
\pgfpathcurveto{\pgfqpoint{1.750897in}{1.552343in}}{\pgfqpoint{1.747624in}{1.544443in}}{\pgfqpoint{1.747624in}{1.536207in}}%
\pgfpathcurveto{\pgfqpoint{1.747624in}{1.527971in}}{\pgfqpoint{1.750897in}{1.520071in}}{\pgfqpoint{1.756721in}{1.514247in}}%
\pgfpathcurveto{\pgfqpoint{1.762545in}{1.508423in}}{\pgfqpoint{1.770445in}{1.505150in}}{\pgfqpoint{1.778681in}{1.505150in}}%
\pgfpathclose%
\pgfusepath{stroke,fill}%
\end{pgfscope}%
\begin{pgfscope}%
\pgfpathrectangle{\pgfqpoint{0.100000in}{0.212622in}}{\pgfqpoint{3.696000in}{3.696000in}}%
\pgfusepath{clip}%
\pgfsetbuttcap%
\pgfsetroundjoin%
\definecolor{currentfill}{rgb}{0.121569,0.466667,0.705882}%
\pgfsetfillcolor{currentfill}%
\pgfsetfillopacity{0.997586}%
\pgfsetlinewidth{1.003750pt}%
\definecolor{currentstroke}{rgb}{0.121569,0.466667,0.705882}%
\pgfsetstrokecolor{currentstroke}%
\pgfsetstrokeopacity{0.997586}%
\pgfsetdash{}{0pt}%
\pgfpathmoveto{\pgfqpoint{1.781662in}{1.502366in}}%
\pgfpathcurveto{\pgfqpoint{1.789898in}{1.502366in}}{\pgfqpoint{1.797798in}{1.505638in}}{\pgfqpoint{1.803622in}{1.511462in}}%
\pgfpathcurveto{\pgfqpoint{1.809446in}{1.517286in}}{\pgfqpoint{1.812718in}{1.525186in}}{\pgfqpoint{1.812718in}{1.533422in}}%
\pgfpathcurveto{\pgfqpoint{1.812718in}{1.541658in}}{\pgfqpoint{1.809446in}{1.549558in}}{\pgfqpoint{1.803622in}{1.555382in}}%
\pgfpathcurveto{\pgfqpoint{1.797798in}{1.561206in}}{\pgfqpoint{1.789898in}{1.564479in}}{\pgfqpoint{1.781662in}{1.564479in}}%
\pgfpathcurveto{\pgfqpoint{1.773425in}{1.564479in}}{\pgfqpoint{1.765525in}{1.561206in}}{\pgfqpoint{1.759702in}{1.555382in}}%
\pgfpathcurveto{\pgfqpoint{1.753878in}{1.549558in}}{\pgfqpoint{1.750605in}{1.541658in}}{\pgfqpoint{1.750605in}{1.533422in}}%
\pgfpathcurveto{\pgfqpoint{1.750605in}{1.525186in}}{\pgfqpoint{1.753878in}{1.517286in}}{\pgfqpoint{1.759702in}{1.511462in}}%
\pgfpathcurveto{\pgfqpoint{1.765525in}{1.505638in}}{\pgfqpoint{1.773425in}{1.502366in}}{\pgfqpoint{1.781662in}{1.502366in}}%
\pgfpathclose%
\pgfusepath{stroke,fill}%
\end{pgfscope}%
\begin{pgfscope}%
\pgfpathrectangle{\pgfqpoint{0.100000in}{0.212622in}}{\pgfqpoint{3.696000in}{3.696000in}}%
\pgfusepath{clip}%
\pgfsetbuttcap%
\pgfsetroundjoin%
\definecolor{currentfill}{rgb}{0.121569,0.466667,0.705882}%
\pgfsetfillcolor{currentfill}%
\pgfsetfillopacity{0.997973}%
\pgfsetlinewidth{1.003750pt}%
\definecolor{currentstroke}{rgb}{0.121569,0.466667,0.705882}%
\pgfsetstrokecolor{currentstroke}%
\pgfsetstrokeopacity{0.997973}%
\pgfsetdash{}{0pt}%
\pgfpathmoveto{\pgfqpoint{1.787211in}{1.496307in}}%
\pgfpathcurveto{\pgfqpoint{1.795447in}{1.496307in}}{\pgfqpoint{1.803347in}{1.499579in}}{\pgfqpoint{1.809171in}{1.505403in}}%
\pgfpathcurveto{\pgfqpoint{1.814995in}{1.511227in}}{\pgfqpoint{1.818268in}{1.519127in}}{\pgfqpoint{1.818268in}{1.527364in}}%
\pgfpathcurveto{\pgfqpoint{1.818268in}{1.535600in}}{\pgfqpoint{1.814995in}{1.543500in}}{\pgfqpoint{1.809171in}{1.549324in}}%
\pgfpathcurveto{\pgfqpoint{1.803347in}{1.555148in}}{\pgfqpoint{1.795447in}{1.558420in}}{\pgfqpoint{1.787211in}{1.558420in}}%
\pgfpathcurveto{\pgfqpoint{1.778975in}{1.558420in}}{\pgfqpoint{1.771075in}{1.555148in}}{\pgfqpoint{1.765251in}{1.549324in}}%
\pgfpathcurveto{\pgfqpoint{1.759427in}{1.543500in}}{\pgfqpoint{1.756155in}{1.535600in}}{\pgfqpoint{1.756155in}{1.527364in}}%
\pgfpathcurveto{\pgfqpoint{1.756155in}{1.519127in}}{\pgfqpoint{1.759427in}{1.511227in}}{\pgfqpoint{1.765251in}{1.505403in}}%
\pgfpathcurveto{\pgfqpoint{1.771075in}{1.499579in}}{\pgfqpoint{1.778975in}{1.496307in}}{\pgfqpoint{1.787211in}{1.496307in}}%
\pgfpathclose%
\pgfusepath{stroke,fill}%
\end{pgfscope}%
\begin{pgfscope}%
\pgfpathrectangle{\pgfqpoint{0.100000in}{0.212622in}}{\pgfqpoint{3.696000in}{3.696000in}}%
\pgfusepath{clip}%
\pgfsetbuttcap%
\pgfsetroundjoin%
\definecolor{currentfill}{rgb}{0.121569,0.466667,0.705882}%
\pgfsetfillcolor{currentfill}%
\pgfsetfillopacity{0.998320}%
\pgfsetlinewidth{1.003750pt}%
\definecolor{currentstroke}{rgb}{0.121569,0.466667,0.705882}%
\pgfsetstrokecolor{currentstroke}%
\pgfsetstrokeopacity{0.998320}%
\pgfsetdash{}{0pt}%
\pgfpathmoveto{\pgfqpoint{1.813497in}{1.478812in}}%
\pgfpathcurveto{\pgfqpoint{1.821733in}{1.478812in}}{\pgfqpoint{1.829633in}{1.482084in}}{\pgfqpoint{1.835457in}{1.487908in}}%
\pgfpathcurveto{\pgfqpoint{1.841281in}{1.493732in}}{\pgfqpoint{1.844554in}{1.501632in}}{\pgfqpoint{1.844554in}{1.509868in}}%
\pgfpathcurveto{\pgfqpoint{1.844554in}{1.518105in}}{\pgfqpoint{1.841281in}{1.526005in}}{\pgfqpoint{1.835457in}{1.531829in}}%
\pgfpathcurveto{\pgfqpoint{1.829633in}{1.537653in}}{\pgfqpoint{1.821733in}{1.540925in}}{\pgfqpoint{1.813497in}{1.540925in}}%
\pgfpathcurveto{\pgfqpoint{1.805261in}{1.540925in}}{\pgfqpoint{1.797361in}{1.537653in}}{\pgfqpoint{1.791537in}{1.531829in}}%
\pgfpathcurveto{\pgfqpoint{1.785713in}{1.526005in}}{\pgfqpoint{1.782441in}{1.518105in}}{\pgfqpoint{1.782441in}{1.509868in}}%
\pgfpathcurveto{\pgfqpoint{1.782441in}{1.501632in}}{\pgfqpoint{1.785713in}{1.493732in}}{\pgfqpoint{1.791537in}{1.487908in}}%
\pgfpathcurveto{\pgfqpoint{1.797361in}{1.482084in}}{\pgfqpoint{1.805261in}{1.478812in}}{\pgfqpoint{1.813497in}{1.478812in}}%
\pgfpathclose%
\pgfusepath{stroke,fill}%
\end{pgfscope}%
\begin{pgfscope}%
\pgfpathrectangle{\pgfqpoint{0.100000in}{0.212622in}}{\pgfqpoint{3.696000in}{3.696000in}}%
\pgfusepath{clip}%
\pgfsetbuttcap%
\pgfsetroundjoin%
\definecolor{currentfill}{rgb}{0.121569,0.466667,0.705882}%
\pgfsetfillcolor{currentfill}%
\pgfsetfillopacity{0.998365}%
\pgfsetlinewidth{1.003750pt}%
\definecolor{currentstroke}{rgb}{0.121569,0.466667,0.705882}%
\pgfsetstrokecolor{currentstroke}%
\pgfsetstrokeopacity{0.998365}%
\pgfsetdash{}{0pt}%
\pgfpathmoveto{\pgfqpoint{1.797012in}{1.483653in}}%
\pgfpathcurveto{\pgfqpoint{1.805248in}{1.483653in}}{\pgfqpoint{1.813148in}{1.486925in}}{\pgfqpoint{1.818972in}{1.492749in}}%
\pgfpathcurveto{\pgfqpoint{1.824796in}{1.498573in}}{\pgfqpoint{1.828069in}{1.506473in}}{\pgfqpoint{1.828069in}{1.514710in}}%
\pgfpathcurveto{\pgfqpoint{1.828069in}{1.522946in}}{\pgfqpoint{1.824796in}{1.530846in}}{\pgfqpoint{1.818972in}{1.536670in}}%
\pgfpathcurveto{\pgfqpoint{1.813148in}{1.542494in}}{\pgfqpoint{1.805248in}{1.545766in}}{\pgfqpoint{1.797012in}{1.545766in}}%
\pgfpathcurveto{\pgfqpoint{1.788776in}{1.545766in}}{\pgfqpoint{1.780876in}{1.542494in}}{\pgfqpoint{1.775052in}{1.536670in}}%
\pgfpathcurveto{\pgfqpoint{1.769228in}{1.530846in}}{\pgfqpoint{1.765956in}{1.522946in}}{\pgfqpoint{1.765956in}{1.514710in}}%
\pgfpathcurveto{\pgfqpoint{1.765956in}{1.506473in}}{\pgfqpoint{1.769228in}{1.498573in}}{\pgfqpoint{1.775052in}{1.492749in}}%
\pgfpathcurveto{\pgfqpoint{1.780876in}{1.486925in}}{\pgfqpoint{1.788776in}{1.483653in}}{\pgfqpoint{1.797012in}{1.483653in}}%
\pgfpathclose%
\pgfusepath{stroke,fill}%
\end{pgfscope}%
\begin{pgfscope}%
\pgfpathrectangle{\pgfqpoint{0.100000in}{0.212622in}}{\pgfqpoint{3.696000in}{3.696000in}}%
\pgfusepath{clip}%
\pgfsetbuttcap%
\pgfsetroundjoin%
\definecolor{currentfill}{rgb}{0.121569,0.466667,0.705882}%
\pgfsetfillcolor{currentfill}%
\pgfsetfillopacity{0.998777}%
\pgfsetlinewidth{1.003750pt}%
\definecolor{currentstroke}{rgb}{0.121569,0.466667,0.705882}%
\pgfsetstrokecolor{currentstroke}%
\pgfsetstrokeopacity{0.998777}%
\pgfsetdash{}{0pt}%
\pgfpathmoveto{\pgfqpoint{1.792189in}{1.491903in}}%
\pgfpathcurveto{\pgfqpoint{1.800425in}{1.491903in}}{\pgfqpoint{1.808325in}{1.495176in}}{\pgfqpoint{1.814149in}{1.501000in}}%
\pgfpathcurveto{\pgfqpoint{1.819973in}{1.506824in}}{\pgfqpoint{1.823245in}{1.514724in}}{\pgfqpoint{1.823245in}{1.522960in}}%
\pgfpathcurveto{\pgfqpoint{1.823245in}{1.531196in}}{\pgfqpoint{1.819973in}{1.539096in}}{\pgfqpoint{1.814149in}{1.544920in}}%
\pgfpathcurveto{\pgfqpoint{1.808325in}{1.550744in}}{\pgfqpoint{1.800425in}{1.554016in}}{\pgfqpoint{1.792189in}{1.554016in}}%
\pgfpathcurveto{\pgfqpoint{1.783952in}{1.554016in}}{\pgfqpoint{1.776052in}{1.550744in}}{\pgfqpoint{1.770228in}{1.544920in}}%
\pgfpathcurveto{\pgfqpoint{1.764404in}{1.539096in}}{\pgfqpoint{1.761132in}{1.531196in}}{\pgfqpoint{1.761132in}{1.522960in}}%
\pgfpathcurveto{\pgfqpoint{1.761132in}{1.514724in}}{\pgfqpoint{1.764404in}{1.506824in}}{\pgfqpoint{1.770228in}{1.501000in}}%
\pgfpathcurveto{\pgfqpoint{1.776052in}{1.495176in}}{\pgfqpoint{1.783952in}{1.491903in}}{\pgfqpoint{1.792189in}{1.491903in}}%
\pgfpathclose%
\pgfusepath{stroke,fill}%
\end{pgfscope}%
\begin{pgfscope}%
\pgfpathrectangle{\pgfqpoint{0.100000in}{0.212622in}}{\pgfqpoint{3.696000in}{3.696000in}}%
\pgfusepath{clip}%
\pgfsetbuttcap%
\pgfsetroundjoin%
\definecolor{currentfill}{rgb}{0.121569,0.466667,0.705882}%
\pgfsetfillcolor{currentfill}%
\pgfsetfillopacity{0.998948}%
\pgfsetlinewidth{1.003750pt}%
\definecolor{currentstroke}{rgb}{0.121569,0.466667,0.705882}%
\pgfsetstrokecolor{currentstroke}%
\pgfsetstrokeopacity{0.998948}%
\pgfsetdash{}{0pt}%
\pgfpathmoveto{\pgfqpoint{1.812391in}{1.477638in}}%
\pgfpathcurveto{\pgfqpoint{1.820627in}{1.477638in}}{\pgfqpoint{1.828528in}{1.480911in}}{\pgfqpoint{1.834351in}{1.486735in}}%
\pgfpathcurveto{\pgfqpoint{1.840175in}{1.492559in}}{\pgfqpoint{1.843448in}{1.500459in}}{\pgfqpoint{1.843448in}{1.508695in}}%
\pgfpathcurveto{\pgfqpoint{1.843448in}{1.516931in}}{\pgfqpoint{1.840175in}{1.524831in}}{\pgfqpoint{1.834351in}{1.530655in}}%
\pgfpathcurveto{\pgfqpoint{1.828528in}{1.536479in}}{\pgfqpoint{1.820627in}{1.539751in}}{\pgfqpoint{1.812391in}{1.539751in}}%
\pgfpathcurveto{\pgfqpoint{1.804155in}{1.539751in}}{\pgfqpoint{1.796255in}{1.536479in}}{\pgfqpoint{1.790431in}{1.530655in}}%
\pgfpathcurveto{\pgfqpoint{1.784607in}{1.524831in}}{\pgfqpoint{1.781335in}{1.516931in}}{\pgfqpoint{1.781335in}{1.508695in}}%
\pgfpathcurveto{\pgfqpoint{1.781335in}{1.500459in}}{\pgfqpoint{1.784607in}{1.492559in}}{\pgfqpoint{1.790431in}{1.486735in}}%
\pgfpathcurveto{\pgfqpoint{1.796255in}{1.480911in}}{\pgfqpoint{1.804155in}{1.477638in}}{\pgfqpoint{1.812391in}{1.477638in}}%
\pgfpathclose%
\pgfusepath{stroke,fill}%
\end{pgfscope}%
\begin{pgfscope}%
\pgfpathrectangle{\pgfqpoint{0.100000in}{0.212622in}}{\pgfqpoint{3.696000in}{3.696000in}}%
\pgfusepath{clip}%
\pgfsetbuttcap%
\pgfsetroundjoin%
\definecolor{currentfill}{rgb}{0.121569,0.466667,0.705882}%
\pgfsetfillcolor{currentfill}%
\pgfsetfillopacity{0.999449}%
\pgfsetlinewidth{1.003750pt}%
\definecolor{currentstroke}{rgb}{0.121569,0.466667,0.705882}%
\pgfsetstrokecolor{currentstroke}%
\pgfsetstrokeopacity{0.999449}%
\pgfsetdash{}{0pt}%
\pgfpathmoveto{\pgfqpoint{1.807946in}{1.476763in}}%
\pgfpathcurveto{\pgfqpoint{1.816183in}{1.476763in}}{\pgfqpoint{1.824083in}{1.480036in}}{\pgfqpoint{1.829907in}{1.485860in}}%
\pgfpathcurveto{\pgfqpoint{1.835730in}{1.491684in}}{\pgfqpoint{1.839003in}{1.499584in}}{\pgfqpoint{1.839003in}{1.507820in}}%
\pgfpathcurveto{\pgfqpoint{1.839003in}{1.516056in}}{\pgfqpoint{1.835730in}{1.523956in}}{\pgfqpoint{1.829907in}{1.529780in}}%
\pgfpathcurveto{\pgfqpoint{1.824083in}{1.535604in}}{\pgfqpoint{1.816183in}{1.538876in}}{\pgfqpoint{1.807946in}{1.538876in}}%
\pgfpathcurveto{\pgfqpoint{1.799710in}{1.538876in}}{\pgfqpoint{1.791810in}{1.535604in}}{\pgfqpoint{1.785986in}{1.529780in}}%
\pgfpathcurveto{\pgfqpoint{1.780162in}{1.523956in}}{\pgfqpoint{1.776890in}{1.516056in}}{\pgfqpoint{1.776890in}{1.507820in}}%
\pgfpathcurveto{\pgfqpoint{1.776890in}{1.499584in}}{\pgfqpoint{1.780162in}{1.491684in}}{\pgfqpoint{1.785986in}{1.485860in}}%
\pgfpathcurveto{\pgfqpoint{1.791810in}{1.480036in}}{\pgfqpoint{1.799710in}{1.476763in}}{\pgfqpoint{1.807946in}{1.476763in}}%
\pgfpathclose%
\pgfusepath{stroke,fill}%
\end{pgfscope}%
\begin{pgfscope}%
\pgfpathrectangle{\pgfqpoint{0.100000in}{0.212622in}}{\pgfqpoint{3.696000in}{3.696000in}}%
\pgfusepath{clip}%
\pgfsetbuttcap%
\pgfsetroundjoin%
\definecolor{currentfill}{rgb}{0.121569,0.466667,0.705882}%
\pgfsetfillcolor{currentfill}%
\pgfsetfillopacity{0.999485}%
\pgfsetlinewidth{1.003750pt}%
\definecolor{currentstroke}{rgb}{0.121569,0.466667,0.705882}%
\pgfsetstrokecolor{currentstroke}%
\pgfsetstrokeopacity{0.999485}%
\pgfsetdash{}{0pt}%
\pgfpathmoveto{\pgfqpoint{1.811600in}{1.478070in}}%
\pgfpathcurveto{\pgfqpoint{1.819836in}{1.478070in}}{\pgfqpoint{1.827736in}{1.481342in}}{\pgfqpoint{1.833560in}{1.487166in}}%
\pgfpathcurveto{\pgfqpoint{1.839384in}{1.492990in}}{\pgfqpoint{1.842656in}{1.500890in}}{\pgfqpoint{1.842656in}{1.509126in}}%
\pgfpathcurveto{\pgfqpoint{1.842656in}{1.517363in}}{\pgfqpoint{1.839384in}{1.525263in}}{\pgfqpoint{1.833560in}{1.531087in}}%
\pgfpathcurveto{\pgfqpoint{1.827736in}{1.536911in}}{\pgfqpoint{1.819836in}{1.540183in}}{\pgfqpoint{1.811600in}{1.540183in}}%
\pgfpathcurveto{\pgfqpoint{1.803364in}{1.540183in}}{\pgfqpoint{1.795463in}{1.536911in}}{\pgfqpoint{1.789640in}{1.531087in}}%
\pgfpathcurveto{\pgfqpoint{1.783816in}{1.525263in}}{\pgfqpoint{1.780543in}{1.517363in}}{\pgfqpoint{1.780543in}{1.509126in}}%
\pgfpathcurveto{\pgfqpoint{1.780543in}{1.500890in}}{\pgfqpoint{1.783816in}{1.492990in}}{\pgfqpoint{1.789640in}{1.487166in}}%
\pgfpathcurveto{\pgfqpoint{1.795463in}{1.481342in}}{\pgfqpoint{1.803364in}{1.478070in}}{\pgfqpoint{1.811600in}{1.478070in}}%
\pgfpathclose%
\pgfusepath{stroke,fill}%
\end{pgfscope}%
\begin{pgfscope}%
\pgfpathrectangle{\pgfqpoint{0.100000in}{0.212622in}}{\pgfqpoint{3.696000in}{3.696000in}}%
\pgfusepath{clip}%
\pgfsetbuttcap%
\pgfsetroundjoin%
\definecolor{currentfill}{rgb}{0.121569,0.466667,0.705882}%
\pgfsetfillcolor{currentfill}%
\pgfsetfillopacity{0.999666}%
\pgfsetlinewidth{1.003750pt}%
\definecolor{currentstroke}{rgb}{0.121569,0.466667,0.705882}%
\pgfsetstrokecolor{currentstroke}%
\pgfsetstrokeopacity{0.999666}%
\pgfsetdash{}{0pt}%
\pgfpathmoveto{\pgfqpoint{1.801151in}{1.483618in}}%
\pgfpathcurveto{\pgfqpoint{1.809387in}{1.483618in}}{\pgfqpoint{1.817287in}{1.486890in}}{\pgfqpoint{1.823111in}{1.492714in}}%
\pgfpathcurveto{\pgfqpoint{1.828935in}{1.498538in}}{\pgfqpoint{1.832208in}{1.506438in}}{\pgfqpoint{1.832208in}{1.514674in}}%
\pgfpathcurveto{\pgfqpoint{1.832208in}{1.522911in}}{\pgfqpoint{1.828935in}{1.530811in}}{\pgfqpoint{1.823111in}{1.536635in}}%
\pgfpathcurveto{\pgfqpoint{1.817287in}{1.542459in}}{\pgfqpoint{1.809387in}{1.545731in}}{\pgfqpoint{1.801151in}{1.545731in}}%
\pgfpathcurveto{\pgfqpoint{1.792915in}{1.545731in}}{\pgfqpoint{1.785015in}{1.542459in}}{\pgfqpoint{1.779191in}{1.536635in}}%
\pgfpathcurveto{\pgfqpoint{1.773367in}{1.530811in}}{\pgfqpoint{1.770095in}{1.522911in}}{\pgfqpoint{1.770095in}{1.514674in}}%
\pgfpathcurveto{\pgfqpoint{1.770095in}{1.506438in}}{\pgfqpoint{1.773367in}{1.498538in}}{\pgfqpoint{1.779191in}{1.492714in}}%
\pgfpathcurveto{\pgfqpoint{1.785015in}{1.486890in}}{\pgfqpoint{1.792915in}{1.483618in}}{\pgfqpoint{1.801151in}{1.483618in}}%
\pgfpathclose%
\pgfusepath{stroke,fill}%
\end{pgfscope}%
\begin{pgfscope}%
\pgfpathrectangle{\pgfqpoint{0.100000in}{0.212622in}}{\pgfqpoint{3.696000in}{3.696000in}}%
\pgfusepath{clip}%
\pgfsetbuttcap%
\pgfsetroundjoin%
\definecolor{currentfill}{rgb}{0.121569,0.466667,0.705882}%
\pgfsetfillcolor{currentfill}%
\pgfsetfillopacity{0.999805}%
\pgfsetlinewidth{1.003750pt}%
\definecolor{currentstroke}{rgb}{0.121569,0.466667,0.705882}%
\pgfsetstrokecolor{currentstroke}%
\pgfsetstrokeopacity{0.999805}%
\pgfsetdash{}{0pt}%
\pgfpathmoveto{\pgfqpoint{1.803981in}{1.480933in}}%
\pgfpathcurveto{\pgfqpoint{1.812217in}{1.480933in}}{\pgfqpoint{1.820117in}{1.484205in}}{\pgfqpoint{1.825941in}{1.490029in}}%
\pgfpathcurveto{\pgfqpoint{1.831765in}{1.495853in}}{\pgfqpoint{1.835037in}{1.503753in}}{\pgfqpoint{1.835037in}{1.511989in}}%
\pgfpathcurveto{\pgfqpoint{1.835037in}{1.520225in}}{\pgfqpoint{1.831765in}{1.528125in}}{\pgfqpoint{1.825941in}{1.533949in}}%
\pgfpathcurveto{\pgfqpoint{1.820117in}{1.539773in}}{\pgfqpoint{1.812217in}{1.543046in}}{\pgfqpoint{1.803981in}{1.543046in}}%
\pgfpathcurveto{\pgfqpoint{1.795744in}{1.543046in}}{\pgfqpoint{1.787844in}{1.539773in}}{\pgfqpoint{1.782020in}{1.533949in}}%
\pgfpathcurveto{\pgfqpoint{1.776196in}{1.528125in}}{\pgfqpoint{1.772924in}{1.520225in}}{\pgfqpoint{1.772924in}{1.511989in}}%
\pgfpathcurveto{\pgfqpoint{1.772924in}{1.503753in}}{\pgfqpoint{1.776196in}{1.495853in}}{\pgfqpoint{1.782020in}{1.490029in}}%
\pgfpathcurveto{\pgfqpoint{1.787844in}{1.484205in}}{\pgfqpoint{1.795744in}{1.480933in}}{\pgfqpoint{1.803981in}{1.480933in}}%
\pgfpathclose%
\pgfusepath{stroke,fill}%
\end{pgfscope}%
\begin{pgfscope}%
\pgfpathrectangle{\pgfqpoint{0.100000in}{0.212622in}}{\pgfqpoint{3.696000in}{3.696000in}}%
\pgfusepath{clip}%
\pgfsetbuttcap%
\pgfsetroundjoin%
\definecolor{currentfill}{rgb}{0.121569,0.466667,0.705882}%
\pgfsetfillcolor{currentfill}%
\pgfsetfillopacity{0.999816}%
\pgfsetlinewidth{1.003750pt}%
\definecolor{currentstroke}{rgb}{0.121569,0.466667,0.705882}%
\pgfsetstrokecolor{currentstroke}%
\pgfsetstrokeopacity{0.999816}%
\pgfsetdash{}{0pt}%
\pgfpathmoveto{\pgfqpoint{1.810144in}{1.477578in}}%
\pgfpathcurveto{\pgfqpoint{1.818380in}{1.477578in}}{\pgfqpoint{1.826280in}{1.480850in}}{\pgfqpoint{1.832104in}{1.486674in}}%
\pgfpathcurveto{\pgfqpoint{1.837928in}{1.492498in}}{\pgfqpoint{1.841200in}{1.500398in}}{\pgfqpoint{1.841200in}{1.508635in}}%
\pgfpathcurveto{\pgfqpoint{1.841200in}{1.516871in}}{\pgfqpoint{1.837928in}{1.524771in}}{\pgfqpoint{1.832104in}{1.530595in}}%
\pgfpathcurveto{\pgfqpoint{1.826280in}{1.536419in}}{\pgfqpoint{1.818380in}{1.539691in}}{\pgfqpoint{1.810144in}{1.539691in}}%
\pgfpathcurveto{\pgfqpoint{1.801907in}{1.539691in}}{\pgfqpoint{1.794007in}{1.536419in}}{\pgfqpoint{1.788183in}{1.530595in}}%
\pgfpathcurveto{\pgfqpoint{1.782359in}{1.524771in}}{\pgfqpoint{1.779087in}{1.516871in}}{\pgfqpoint{1.779087in}{1.508635in}}%
\pgfpathcurveto{\pgfqpoint{1.779087in}{1.500398in}}{\pgfqpoint{1.782359in}{1.492498in}}{\pgfqpoint{1.788183in}{1.486674in}}%
\pgfpathcurveto{\pgfqpoint{1.794007in}{1.480850in}}{\pgfqpoint{1.801907in}{1.477578in}}{\pgfqpoint{1.810144in}{1.477578in}}%
\pgfpathclose%
\pgfusepath{stroke,fill}%
\end{pgfscope}%
\begin{pgfscope}%
\pgfpathrectangle{\pgfqpoint{0.100000in}{0.212622in}}{\pgfqpoint{3.696000in}{3.696000in}}%
\pgfusepath{clip}%
\pgfsetbuttcap%
\pgfsetroundjoin%
\definecolor{currentfill}{rgb}{0.121569,0.466667,0.705882}%
\pgfsetfillcolor{currentfill}%
\pgfsetlinewidth{1.003750pt}%
\definecolor{currentstroke}{rgb}{0.121569,0.466667,0.705882}%
\pgfsetstrokecolor{currentstroke}%
\pgfsetdash{}{0pt}%
\pgfpathmoveto{\pgfqpoint{1.805262in}{1.480241in}}%
\pgfpathcurveto{\pgfqpoint{1.813499in}{1.480241in}}{\pgfqpoint{1.821399in}{1.483513in}}{\pgfqpoint{1.827223in}{1.489337in}}%
\pgfpathcurveto{\pgfqpoint{1.833047in}{1.495161in}}{\pgfqpoint{1.836319in}{1.503061in}}{\pgfqpoint{1.836319in}{1.511297in}}%
\pgfpathcurveto{\pgfqpoint{1.836319in}{1.519533in}}{\pgfqpoint{1.833047in}{1.527434in}}{\pgfqpoint{1.827223in}{1.533257in}}%
\pgfpathcurveto{\pgfqpoint{1.821399in}{1.539081in}}{\pgfqpoint{1.813499in}{1.542354in}}{\pgfqpoint{1.805262in}{1.542354in}}%
\pgfpathcurveto{\pgfqpoint{1.797026in}{1.542354in}}{\pgfqpoint{1.789126in}{1.539081in}}{\pgfqpoint{1.783302in}{1.533257in}}%
\pgfpathcurveto{\pgfqpoint{1.777478in}{1.527434in}}{\pgfqpoint{1.774206in}{1.519533in}}{\pgfqpoint{1.774206in}{1.511297in}}%
\pgfpathcurveto{\pgfqpoint{1.774206in}{1.503061in}}{\pgfqpoint{1.777478in}{1.495161in}}{\pgfqpoint{1.783302in}{1.489337in}}%
\pgfpathcurveto{\pgfqpoint{1.789126in}{1.483513in}}{\pgfqpoint{1.797026in}{1.480241in}}{\pgfqpoint{1.805262in}{1.480241in}}%
\pgfpathclose%
\pgfusepath{stroke,fill}%
\end{pgfscope}%
\begin{pgfscope}%
\definecolor{textcolor}{rgb}{0.000000,0.000000,0.000000}%
\pgfsetstrokecolor{textcolor}%
\pgfsetfillcolor{textcolor}%
\pgftext[x=1.948000in,y=3.991956in,,base]{\color{textcolor}\rmfamily\fontsize{12.000000}{14.400000}\selectfont Mahony}%
\end{pgfscope}%
\begin{pgfscope}%
\pgfsetbuttcap%
\pgfsetmiterjoin%
\definecolor{currentfill}{rgb}{1.000000,1.000000,1.000000}%
\pgfsetfillcolor{currentfill}%
\pgfsetfillopacity{0.800000}%
\pgfsetlinewidth{1.003750pt}%
\definecolor{currentstroke}{rgb}{0.800000,0.800000,0.800000}%
\pgfsetstrokecolor{currentstroke}%
\pgfsetstrokeopacity{0.800000}%
\pgfsetdash{}{0pt}%
\pgfpathmoveto{\pgfqpoint{2.104889in}{3.410289in}}%
\pgfpathlineto{\pgfqpoint{3.698778in}{3.410289in}}%
\pgfpathquadraticcurveto{\pgfqpoint{3.726556in}{3.410289in}}{\pgfqpoint{3.726556in}{3.438067in}}%
\pgfpathlineto{\pgfqpoint{3.726556in}{3.811400in}}%
\pgfpathquadraticcurveto{\pgfqpoint{3.726556in}{3.839178in}}{\pgfqpoint{3.698778in}{3.839178in}}%
\pgfpathlineto{\pgfqpoint{2.104889in}{3.839178in}}%
\pgfpathquadraticcurveto{\pgfqpoint{2.077111in}{3.839178in}}{\pgfqpoint{2.077111in}{3.811400in}}%
\pgfpathlineto{\pgfqpoint{2.077111in}{3.438067in}}%
\pgfpathquadraticcurveto{\pgfqpoint{2.077111in}{3.410289in}}{\pgfqpoint{2.104889in}{3.410289in}}%
\pgfpathclose%
\pgfusepath{stroke,fill}%
\end{pgfscope}%
\begin{pgfscope}%
\pgfsetrectcap%
\pgfsetroundjoin%
\pgfsetlinewidth{1.505625pt}%
\definecolor{currentstroke}{rgb}{0.121569,0.466667,0.705882}%
\pgfsetstrokecolor{currentstroke}%
\pgfsetdash{}{0pt}%
\pgfpathmoveto{\pgfqpoint{2.132667in}{3.735011in}}%
\pgfpathlineto{\pgfqpoint{2.410444in}{3.735011in}}%
\pgfusepath{stroke}%
\end{pgfscope}%
\begin{pgfscope}%
\definecolor{textcolor}{rgb}{0.000000,0.000000,0.000000}%
\pgfsetstrokecolor{textcolor}%
\pgfsetfillcolor{textcolor}%
\pgftext[x=2.521555in,y=3.686400in,left,base]{\color{textcolor}\rmfamily\fontsize{10.000000}{12.000000}\selectfont Ground truth}%
\end{pgfscope}%
\begin{pgfscope}%
\pgfsetbuttcap%
\pgfsetroundjoin%
\definecolor{currentfill}{rgb}{0.121569,0.466667,0.705882}%
\pgfsetfillcolor{currentfill}%
\pgfsetlinewidth{1.003750pt}%
\definecolor{currentstroke}{rgb}{0.121569,0.466667,0.705882}%
\pgfsetstrokecolor{currentstroke}%
\pgfsetdash{}{0pt}%
\pgfsys@defobject{currentmarker}{\pgfqpoint{-0.031056in}{-0.031056in}}{\pgfqpoint{0.031056in}{0.031056in}}{%
\pgfpathmoveto{\pgfqpoint{0.000000in}{-0.031056in}}%
\pgfpathcurveto{\pgfqpoint{0.008236in}{-0.031056in}}{\pgfqpoint{0.016136in}{-0.027784in}}{\pgfqpoint{0.021960in}{-0.021960in}}%
\pgfpathcurveto{\pgfqpoint{0.027784in}{-0.016136in}}{\pgfqpoint{0.031056in}{-0.008236in}}{\pgfqpoint{0.031056in}{0.000000in}}%
\pgfpathcurveto{\pgfqpoint{0.031056in}{0.008236in}}{\pgfqpoint{0.027784in}{0.016136in}}{\pgfqpoint{0.021960in}{0.021960in}}%
\pgfpathcurveto{\pgfqpoint{0.016136in}{0.027784in}}{\pgfqpoint{0.008236in}{0.031056in}}{\pgfqpoint{0.000000in}{0.031056in}}%
\pgfpathcurveto{\pgfqpoint{-0.008236in}{0.031056in}}{\pgfqpoint{-0.016136in}{0.027784in}}{\pgfqpoint{-0.021960in}{0.021960in}}%
\pgfpathcurveto{\pgfqpoint{-0.027784in}{0.016136in}}{\pgfqpoint{-0.031056in}{0.008236in}}{\pgfqpoint{-0.031056in}{0.000000in}}%
\pgfpathcurveto{\pgfqpoint{-0.031056in}{-0.008236in}}{\pgfqpoint{-0.027784in}{-0.016136in}}{\pgfqpoint{-0.021960in}{-0.021960in}}%
\pgfpathcurveto{\pgfqpoint{-0.016136in}{-0.027784in}}{\pgfqpoint{-0.008236in}{-0.031056in}}{\pgfqpoint{0.000000in}{-0.031056in}}%
\pgfpathclose%
\pgfusepath{stroke,fill}%
}%
\begin{pgfscope}%
\pgfsys@transformshift{2.271555in}{3.529248in}%
\pgfsys@useobject{currentmarker}{}%
\end{pgfscope}%
\end{pgfscope}%
\begin{pgfscope}%
\definecolor{textcolor}{rgb}{0.000000,0.000000,0.000000}%
\pgfsetstrokecolor{textcolor}%
\pgfsetfillcolor{textcolor}%
\pgftext[x=2.521555in,y=3.492789in,left,base]{\color{textcolor}\rmfamily\fontsize{10.000000}{12.000000}\selectfont Estimated position}%
\end{pgfscope}%
\end{pgfpicture}%
\makeatother%
\endgroup%
}
%         \caption{Mahony's 3D position estimation had the lowest displacement error for the 28-meter side triangle experiment.}
%         \label{fig:triangle28_2D}
%     \end{subfigure}
%     \begin{subfigure}{0.49\textwidth}
%         \centering
%         \resizebox{1\linewidth}{!}{%% Creator: Matplotlib, PGF backend
%%
%% To include the figure in your LaTeX document, write
%%   \input{<filename>.pgf}
%%
%% Make sure the required packages are loaded in your preamble
%%   \usepackage{pgf}
%%
%% and, on pdftex
%%   \usepackage[utf8]{inputenc}\DeclareUnicodeCharacter{2212}{-}
%%
%% or, on luatex and xetex
%%   \usepackage{unicode-math}
%%
%% Figures using additional raster images can only be included by \input if
%% they are in the same directory as the main LaTeX file. For loading figures
%% from other directories you can use the `import` package
%%   \usepackage{import}
%%
%% and then include the figures with
%%   \import{<path to file>}{<filename>.pgf}
%%
%% Matplotlib used the following preamble
%%   \usepackage{fontspec}
%%
\begingroup%
\makeatletter%
\begin{pgfpicture}%
\pgfpathrectangle{\pgfpointorigin}{\pgfqpoint{4.342355in}{4.207622in}}%
\pgfusepath{use as bounding box, clip}%
\begin{pgfscope}%
\pgfsetbuttcap%
\pgfsetmiterjoin%
\definecolor{currentfill}{rgb}{1.000000,1.000000,1.000000}%
\pgfsetfillcolor{currentfill}%
\pgfsetlinewidth{0.000000pt}%
\definecolor{currentstroke}{rgb}{1.000000,1.000000,1.000000}%
\pgfsetstrokecolor{currentstroke}%
\pgfsetdash{}{0pt}%
\pgfpathmoveto{\pgfqpoint{0.000000in}{-0.000000in}}%
\pgfpathlineto{\pgfqpoint{4.342355in}{-0.000000in}}%
\pgfpathlineto{\pgfqpoint{4.342355in}{4.207622in}}%
\pgfpathlineto{\pgfqpoint{0.000000in}{4.207622in}}%
\pgfpathclose%
\pgfusepath{fill}%
\end{pgfscope}%
\begin{pgfscope}%
\pgfsetbuttcap%
\pgfsetmiterjoin%
\definecolor{currentfill}{rgb}{1.000000,1.000000,1.000000}%
\pgfsetfillcolor{currentfill}%
\pgfsetlinewidth{0.000000pt}%
\definecolor{currentstroke}{rgb}{0.000000,0.000000,0.000000}%
\pgfsetstrokecolor{currentstroke}%
\pgfsetstrokeopacity{0.000000}%
\pgfsetdash{}{0pt}%
\pgfpathmoveto{\pgfqpoint{0.100000in}{0.212622in}}%
\pgfpathlineto{\pgfqpoint{3.796000in}{0.212622in}}%
\pgfpathlineto{\pgfqpoint{3.796000in}{3.908622in}}%
\pgfpathlineto{\pgfqpoint{0.100000in}{3.908622in}}%
\pgfpathclose%
\pgfusepath{fill}%
\end{pgfscope}%
\begin{pgfscope}%
\pgfsetbuttcap%
\pgfsetmiterjoin%
\definecolor{currentfill}{rgb}{0.950000,0.950000,0.950000}%
\pgfsetfillcolor{currentfill}%
\pgfsetfillopacity{0.500000}%
\pgfsetlinewidth{1.003750pt}%
\definecolor{currentstroke}{rgb}{0.950000,0.950000,0.950000}%
\pgfsetstrokecolor{currentstroke}%
\pgfsetstrokeopacity{0.500000}%
\pgfsetdash{}{0pt}%
\pgfpathmoveto{\pgfqpoint{0.379073in}{1.123938in}}%
\pgfpathlineto{\pgfqpoint{1.599613in}{2.147018in}}%
\pgfpathlineto{\pgfqpoint{1.582647in}{3.622484in}}%
\pgfpathlineto{\pgfqpoint{0.303698in}{2.689165in}}%
\pgfusepath{stroke,fill}%
\end{pgfscope}%
\begin{pgfscope}%
\pgfsetbuttcap%
\pgfsetmiterjoin%
\definecolor{currentfill}{rgb}{0.900000,0.900000,0.900000}%
\pgfsetfillcolor{currentfill}%
\pgfsetfillopacity{0.500000}%
\pgfsetlinewidth{1.003750pt}%
\definecolor{currentstroke}{rgb}{0.900000,0.900000,0.900000}%
\pgfsetstrokecolor{currentstroke}%
\pgfsetstrokeopacity{0.500000}%
\pgfsetdash{}{0pt}%
\pgfpathmoveto{\pgfqpoint{1.599613in}{2.147018in}}%
\pgfpathlineto{\pgfqpoint{3.558144in}{1.577751in}}%
\pgfpathlineto{\pgfqpoint{3.628038in}{3.104037in}}%
\pgfpathlineto{\pgfqpoint{1.582647in}{3.622484in}}%
\pgfusepath{stroke,fill}%
\end{pgfscope}%
\begin{pgfscope}%
\pgfsetbuttcap%
\pgfsetmiterjoin%
\definecolor{currentfill}{rgb}{0.925000,0.925000,0.925000}%
\pgfsetfillcolor{currentfill}%
\pgfsetfillopacity{0.500000}%
\pgfsetlinewidth{1.003750pt}%
\definecolor{currentstroke}{rgb}{0.925000,0.925000,0.925000}%
\pgfsetstrokecolor{currentstroke}%
\pgfsetstrokeopacity{0.500000}%
\pgfsetdash{}{0pt}%
\pgfpathmoveto{\pgfqpoint{0.379073in}{1.123938in}}%
\pgfpathlineto{\pgfqpoint{2.455212in}{0.445871in}}%
\pgfpathlineto{\pgfqpoint{3.558144in}{1.577751in}}%
\pgfpathlineto{\pgfqpoint{1.599613in}{2.147018in}}%
\pgfusepath{stroke,fill}%
\end{pgfscope}%
\begin{pgfscope}%
\pgfsetrectcap%
\pgfsetroundjoin%
\pgfsetlinewidth{0.803000pt}%
\definecolor{currentstroke}{rgb}{0.000000,0.000000,0.000000}%
\pgfsetstrokecolor{currentstroke}%
\pgfsetdash{}{0pt}%
\pgfpathmoveto{\pgfqpoint{0.379073in}{1.123938in}}%
\pgfpathlineto{\pgfqpoint{2.455212in}{0.445871in}}%
\pgfusepath{stroke}%
\end{pgfscope}%
\begin{pgfscope}%
\definecolor{textcolor}{rgb}{0.000000,0.000000,0.000000}%
\pgfsetstrokecolor{textcolor}%
\pgfsetfillcolor{textcolor}%
\pgftext[x=0.730374in, y=0.408886in, left, base,rotate=341.912962]{\color{textcolor}\rmfamily\fontsize{10.000000}{12.000000}\selectfont Position X [\(\displaystyle m\)]}%
\end{pgfscope}%
\begin{pgfscope}%
\pgfsetbuttcap%
\pgfsetroundjoin%
\pgfsetlinewidth{0.803000pt}%
\definecolor{currentstroke}{rgb}{0.690196,0.690196,0.690196}%
\pgfsetstrokecolor{currentstroke}%
\pgfsetdash{}{0pt}%
\pgfpathmoveto{\pgfqpoint{0.638825in}{1.039103in}}%
\pgfpathlineto{\pgfqpoint{1.845599in}{2.075520in}}%
\pgfpathlineto{\pgfqpoint{1.839067in}{3.557488in}}%
\pgfusepath{stroke}%
\end{pgfscope}%
\begin{pgfscope}%
\pgfsetbuttcap%
\pgfsetroundjoin%
\pgfsetlinewidth{0.803000pt}%
\definecolor{currentstroke}{rgb}{0.690196,0.690196,0.690196}%
\pgfsetstrokecolor{currentstroke}%
\pgfsetdash{}{0pt}%
\pgfpathmoveto{\pgfqpoint{1.052229in}{0.904085in}}%
\pgfpathlineto{\pgfqpoint{2.236533in}{1.961891in}}%
\pgfpathlineto{\pgfqpoint{2.246866in}{3.454124in}}%
\pgfusepath{stroke}%
\end{pgfscope}%
\begin{pgfscope}%
\pgfsetbuttcap%
\pgfsetroundjoin%
\pgfsetlinewidth{0.803000pt}%
\definecolor{currentstroke}{rgb}{0.690196,0.690196,0.690196}%
\pgfsetstrokecolor{currentstroke}%
\pgfsetdash{}{0pt}%
\pgfpathmoveto{\pgfqpoint{1.474585in}{0.766144in}}%
\pgfpathlineto{\pgfqpoint{2.635222in}{1.846008in}}%
\pgfpathlineto{\pgfqpoint{2.663108in}{3.348618in}}%
\pgfusepath{stroke}%
\end{pgfscope}%
\begin{pgfscope}%
\pgfsetbuttcap%
\pgfsetroundjoin%
\pgfsetlinewidth{0.803000pt}%
\definecolor{currentstroke}{rgb}{0.690196,0.690196,0.690196}%
\pgfsetstrokecolor{currentstroke}%
\pgfsetdash{}{0pt}%
\pgfpathmoveto{\pgfqpoint{1.906186in}{0.625183in}}%
\pgfpathlineto{\pgfqpoint{3.041899in}{1.727803in}}%
\pgfpathlineto{\pgfqpoint{3.088058in}{3.240906in}}%
\pgfusepath{stroke}%
\end{pgfscope}%
\begin{pgfscope}%
\pgfsetbuttcap%
\pgfsetroundjoin%
\pgfsetlinewidth{0.803000pt}%
\definecolor{currentstroke}{rgb}{0.690196,0.690196,0.690196}%
\pgfsetstrokecolor{currentstroke}%
\pgfsetdash{}{0pt}%
\pgfpathmoveto{\pgfqpoint{2.347339in}{0.481102in}}%
\pgfpathlineto{\pgfqpoint{3.456807in}{1.607205in}}%
\pgfpathlineto{\pgfqpoint{3.521994in}{3.130916in}}%
\pgfusepath{stroke}%
\end{pgfscope}%
\begin{pgfscope}%
\pgfsetrectcap%
\pgfsetroundjoin%
\pgfsetlinewidth{0.803000pt}%
\definecolor{currentstroke}{rgb}{0.000000,0.000000,0.000000}%
\pgfsetstrokecolor{currentstroke}%
\pgfsetdash{}{0pt}%
\pgfpathmoveto{\pgfqpoint{0.649336in}{1.048130in}}%
\pgfpathlineto{\pgfqpoint{0.617757in}{1.021009in}}%
\pgfusepath{stroke}%
\end{pgfscope}%
\begin{pgfscope}%
\definecolor{textcolor}{rgb}{0.000000,0.000000,0.000000}%
\pgfsetstrokecolor{textcolor}%
\pgfsetfillcolor{textcolor}%
\pgftext[x=0.534389in,y=0.819972in,,top]{\color{textcolor}\rmfamily\fontsize{10.000000}{12.000000}\selectfont \(\displaystyle {0}\)}%
\end{pgfscope}%
\begin{pgfscope}%
\pgfsetrectcap%
\pgfsetroundjoin%
\pgfsetlinewidth{0.803000pt}%
\definecolor{currentstroke}{rgb}{0.000000,0.000000,0.000000}%
\pgfsetstrokecolor{currentstroke}%
\pgfsetdash{}{0pt}%
\pgfpathmoveto{\pgfqpoint{1.062554in}{0.913307in}}%
\pgfpathlineto{\pgfqpoint{1.031535in}{0.885601in}}%
\pgfusepath{stroke}%
\end{pgfscope}%
\begin{pgfscope}%
\definecolor{textcolor}{rgb}{0.000000,0.000000,0.000000}%
\pgfsetstrokecolor{textcolor}%
\pgfsetfillcolor{textcolor}%
\pgftext[x=0.948229in,y=0.682089in,,top]{\color{textcolor}\rmfamily\fontsize{10.000000}{12.000000}\selectfont \(\displaystyle {10}\)}%
\end{pgfscope}%
\begin{pgfscope}%
\pgfsetrectcap%
\pgfsetroundjoin%
\pgfsetlinewidth{0.803000pt}%
\definecolor{currentstroke}{rgb}{0.000000,0.000000,0.000000}%
\pgfsetstrokecolor{currentstroke}%
\pgfsetdash{}{0pt}%
\pgfpathmoveto{\pgfqpoint{1.484712in}{0.775566in}}%
\pgfpathlineto{\pgfqpoint{1.454285in}{0.747257in}}%
\pgfusepath{stroke}%
\end{pgfscope}%
\begin{pgfscope}%
\definecolor{textcolor}{rgb}{0.000000,0.000000,0.000000}%
\pgfsetstrokecolor{textcolor}%
\pgfsetfillcolor{textcolor}%
\pgftext[x=1.371064in,y=0.541210in,,top]{\color{textcolor}\rmfamily\fontsize{10.000000}{12.000000}\selectfont \(\displaystyle {20}\)}%
\end{pgfscope}%
\begin{pgfscope}%
\pgfsetrectcap%
\pgfsetroundjoin%
\pgfsetlinewidth{0.803000pt}%
\definecolor{currentstroke}{rgb}{0.000000,0.000000,0.000000}%
\pgfsetstrokecolor{currentstroke}%
\pgfsetdash{}{0pt}%
\pgfpathmoveto{\pgfqpoint{1.916105in}{0.634813in}}%
\pgfpathlineto{\pgfqpoint{1.886304in}{0.605880in}}%
\pgfusepath{stroke}%
\end{pgfscope}%
\begin{pgfscope}%
\definecolor{textcolor}{rgb}{0.000000,0.000000,0.000000}%
\pgfsetstrokecolor{textcolor}%
\pgfsetfillcolor{textcolor}%
\pgftext[x=1.803191in,y=0.397234in,,top]{\color{textcolor}\rmfamily\fontsize{10.000000}{12.000000}\selectfont \(\displaystyle {30}\)}%
\end{pgfscope}%
\begin{pgfscope}%
\pgfsetrectcap%
\pgfsetroundjoin%
\pgfsetlinewidth{0.803000pt}%
\definecolor{currentstroke}{rgb}{0.000000,0.000000,0.000000}%
\pgfsetstrokecolor{currentstroke}%
\pgfsetdash{}{0pt}%
\pgfpathmoveto{\pgfqpoint{2.357038in}{0.490946in}}%
\pgfpathlineto{\pgfqpoint{2.327898in}{0.461369in}}%
\pgfusepath{stroke}%
\end{pgfscope}%
\begin{pgfscope}%
\definecolor{textcolor}{rgb}{0.000000,0.000000,0.000000}%
\pgfsetstrokecolor{textcolor}%
\pgfsetfillcolor{textcolor}%
\pgftext[x=2.244919in,y=0.250060in,,top]{\color{textcolor}\rmfamily\fontsize{10.000000}{12.000000}\selectfont \(\displaystyle {40}\)}%
\end{pgfscope}%
\begin{pgfscope}%
\pgfsetrectcap%
\pgfsetroundjoin%
\pgfsetlinewidth{0.803000pt}%
\definecolor{currentstroke}{rgb}{0.000000,0.000000,0.000000}%
\pgfsetstrokecolor{currentstroke}%
\pgfsetdash{}{0pt}%
\pgfpathmoveto{\pgfqpoint{3.558144in}{1.577751in}}%
\pgfpathlineto{\pgfqpoint{2.455212in}{0.445871in}}%
\pgfusepath{stroke}%
\end{pgfscope}%
\begin{pgfscope}%
\definecolor{textcolor}{rgb}{0.000000,0.000000,0.000000}%
\pgfsetstrokecolor{textcolor}%
\pgfsetfillcolor{textcolor}%
\pgftext[x=3.120747in, y=0.305657in, left, base,rotate=45.742112]{\color{textcolor}\rmfamily\fontsize{10.000000}{12.000000}\selectfont Position Y [\(\displaystyle m\)]}%
\end{pgfscope}%
\begin{pgfscope}%
\pgfsetbuttcap%
\pgfsetroundjoin%
\pgfsetlinewidth{0.803000pt}%
\definecolor{currentstroke}{rgb}{0.690196,0.690196,0.690196}%
\pgfsetstrokecolor{currentstroke}%
\pgfsetdash{}{0pt}%
\pgfpathmoveto{\pgfqpoint{0.375869in}{2.741832in}}%
\pgfpathlineto{\pgfqpoint{0.447702in}{1.181464in}}%
\pgfpathlineto{\pgfqpoint{2.517487in}{0.509780in}}%
\pgfusepath{stroke}%
\end{pgfscope}%
\begin{pgfscope}%
\pgfsetbuttcap%
\pgfsetroundjoin%
\pgfsetlinewidth{0.803000pt}%
\definecolor{currentstroke}{rgb}{0.690196,0.690196,0.690196}%
\pgfsetstrokecolor{currentstroke}%
\pgfsetdash{}{0pt}%
\pgfpathmoveto{\pgfqpoint{0.557754in}{2.874564in}}%
\pgfpathlineto{\pgfqpoint{0.620790in}{1.326549in}}%
\pgfpathlineto{\pgfqpoint{2.674412in}{0.670824in}}%
\pgfusepath{stroke}%
\end{pgfscope}%
\begin{pgfscope}%
\pgfsetbuttcap%
\pgfsetroundjoin%
\pgfsetlinewidth{0.803000pt}%
\definecolor{currentstroke}{rgb}{0.690196,0.690196,0.690196}%
\pgfsetstrokecolor{currentstroke}%
\pgfsetdash{}{0pt}%
\pgfpathmoveto{\pgfqpoint{0.735264in}{3.004102in}}%
\pgfpathlineto{\pgfqpoint{0.789895in}{1.468296in}}%
\pgfpathlineto{\pgfqpoint{2.827536in}{0.827967in}}%
\pgfusepath{stroke}%
\end{pgfscope}%
\begin{pgfscope}%
\pgfsetbuttcap%
\pgfsetroundjoin%
\pgfsetlinewidth{0.803000pt}%
\definecolor{currentstroke}{rgb}{0.690196,0.690196,0.690196}%
\pgfsetstrokecolor{currentstroke}%
\pgfsetdash{}{0pt}%
\pgfpathmoveto{\pgfqpoint{0.908555in}{3.130562in}}%
\pgfpathlineto{\pgfqpoint{0.955152in}{1.606818in}}%
\pgfpathlineto{\pgfqpoint{2.976995in}{0.981349in}}%
\pgfusepath{stroke}%
\end{pgfscope}%
\begin{pgfscope}%
\pgfsetbuttcap%
\pgfsetroundjoin%
\pgfsetlinewidth{0.803000pt}%
\definecolor{currentstroke}{rgb}{0.690196,0.690196,0.690196}%
\pgfsetstrokecolor{currentstroke}%
\pgfsetdash{}{0pt}%
\pgfpathmoveto{\pgfqpoint{1.077775in}{3.254052in}}%
\pgfpathlineto{\pgfqpoint{1.116692in}{1.742224in}}%
\pgfpathlineto{\pgfqpoint{3.122919in}{1.131103in}}%
\pgfusepath{stroke}%
\end{pgfscope}%
\begin{pgfscope}%
\pgfsetbuttcap%
\pgfsetroundjoin%
\pgfsetlinewidth{0.803000pt}%
\definecolor{currentstroke}{rgb}{0.690196,0.690196,0.690196}%
\pgfsetstrokecolor{currentstroke}%
\pgfsetdash{}{0pt}%
\pgfpathmoveto{\pgfqpoint{1.243067in}{3.374674in}}%
\pgfpathlineto{\pgfqpoint{1.274638in}{1.874617in}}%
\pgfpathlineto{\pgfqpoint{3.265433in}{1.277356in}}%
\pgfusepath{stroke}%
\end{pgfscope}%
\begin{pgfscope}%
\pgfsetbuttcap%
\pgfsetroundjoin%
\pgfsetlinewidth{0.803000pt}%
\definecolor{currentstroke}{rgb}{0.690196,0.690196,0.690196}%
\pgfsetstrokecolor{currentstroke}%
\pgfsetdash{}{0pt}%
\pgfpathmoveto{\pgfqpoint{1.404565in}{3.492528in}}%
\pgfpathlineto{\pgfqpoint{1.429109in}{2.004098in}}%
\pgfpathlineto{\pgfqpoint{3.404653in}{1.420231in}}%
\pgfusepath{stroke}%
\end{pgfscope}%
\begin{pgfscope}%
\pgfsetrectcap%
\pgfsetroundjoin%
\pgfsetlinewidth{0.803000pt}%
\definecolor{currentstroke}{rgb}{0.000000,0.000000,0.000000}%
\pgfsetstrokecolor{currentstroke}%
\pgfsetdash{}{0pt}%
\pgfpathmoveto{\pgfqpoint{2.500044in}{0.515441in}}%
\pgfpathlineto{\pgfqpoint{2.552418in}{0.498444in}}%
\pgfusepath{stroke}%
\end{pgfscope}%
\begin{pgfscope}%
\definecolor{textcolor}{rgb}{0.000000,0.000000,0.000000}%
\pgfsetstrokecolor{textcolor}%
\pgfsetfillcolor{textcolor}%
\pgftext[x=2.696573in,y=0.323132in,,top]{\color{textcolor}\rmfamily\fontsize{10.000000}{12.000000}\selectfont \(\displaystyle {-5}\)}%
\end{pgfscope}%
\begin{pgfscope}%
\pgfsetrectcap%
\pgfsetroundjoin%
\pgfsetlinewidth{0.803000pt}%
\definecolor{currentstroke}{rgb}{0.000000,0.000000,0.000000}%
\pgfsetstrokecolor{currentstroke}%
\pgfsetdash{}{0pt}%
\pgfpathmoveto{\pgfqpoint{2.657116in}{0.676347in}}%
\pgfpathlineto{\pgfqpoint{2.709049in}{0.659765in}}%
\pgfusepath{stroke}%
\end{pgfscope}%
\begin{pgfscope}%
\definecolor{textcolor}{rgb}{0.000000,0.000000,0.000000}%
\pgfsetstrokecolor{textcolor}%
\pgfsetfillcolor{textcolor}%
\pgftext[x=2.851394in,y=0.486561in,,top]{\color{textcolor}\rmfamily\fontsize{10.000000}{12.000000}\selectfont \(\displaystyle {0}\)}%
\end{pgfscope}%
\begin{pgfscope}%
\pgfsetrectcap%
\pgfsetroundjoin%
\pgfsetlinewidth{0.803000pt}%
\definecolor{currentstroke}{rgb}{0.000000,0.000000,0.000000}%
\pgfsetstrokecolor{currentstroke}%
\pgfsetdash{}{0pt}%
\pgfpathmoveto{\pgfqpoint{2.810385in}{0.833357in}}%
\pgfpathlineto{\pgfqpoint{2.861882in}{0.817174in}}%
\pgfusepath{stroke}%
\end{pgfscope}%
\begin{pgfscope}%
\definecolor{textcolor}{rgb}{0.000000,0.000000,0.000000}%
\pgfsetstrokecolor{textcolor}%
\pgfsetfillcolor{textcolor}%
\pgftext[x=3.002463in,y=0.646029in,,top]{\color{textcolor}\rmfamily\fontsize{10.000000}{12.000000}\selectfont \(\displaystyle {5}\)}%
\end{pgfscope}%
\begin{pgfscope}%
\pgfsetrectcap%
\pgfsetroundjoin%
\pgfsetlinewidth{0.803000pt}%
\definecolor{currentstroke}{rgb}{0.000000,0.000000,0.000000}%
\pgfsetstrokecolor{currentstroke}%
\pgfsetdash{}{0pt}%
\pgfpathmoveto{\pgfqpoint{2.959987in}{0.986610in}}%
\pgfpathlineto{\pgfqpoint{3.011054in}{0.970812in}}%
\pgfusepath{stroke}%
\end{pgfscope}%
\begin{pgfscope}%
\definecolor{textcolor}{rgb}{0.000000,0.000000,0.000000}%
\pgfsetstrokecolor{textcolor}%
\pgfsetfillcolor{textcolor}%
\pgftext[x=3.149913in,y=0.801678in,,top]{\color{textcolor}\rmfamily\fontsize{10.000000}{12.000000}\selectfont \(\displaystyle {10}\)}%
\end{pgfscope}%
\begin{pgfscope}%
\pgfsetrectcap%
\pgfsetroundjoin%
\pgfsetlinewidth{0.803000pt}%
\definecolor{currentstroke}{rgb}{0.000000,0.000000,0.000000}%
\pgfsetstrokecolor{currentstroke}%
\pgfsetdash{}{0pt}%
\pgfpathmoveto{\pgfqpoint{3.106052in}{1.136241in}}%
\pgfpathlineto{\pgfqpoint{3.156695in}{1.120814in}}%
\pgfusepath{stroke}%
\end{pgfscope}%
\begin{pgfscope}%
\definecolor{textcolor}{rgb}{0.000000,0.000000,0.000000}%
\pgfsetstrokecolor{textcolor}%
\pgfsetfillcolor{textcolor}%
\pgftext[x=3.293875in,y=0.953643in,,top]{\color{textcolor}\rmfamily\fontsize{10.000000}{12.000000}\selectfont \(\displaystyle {15}\)}%
\end{pgfscope}%
\begin{pgfscope}%
\pgfsetrectcap%
\pgfsetroundjoin%
\pgfsetlinewidth{0.803000pt}%
\definecolor{currentstroke}{rgb}{0.000000,0.000000,0.000000}%
\pgfsetstrokecolor{currentstroke}%
\pgfsetdash{}{0pt}%
\pgfpathmoveto{\pgfqpoint{3.248705in}{1.282375in}}%
\pgfpathlineto{\pgfqpoint{3.298929in}{1.267307in}}%
\pgfusepath{stroke}%
\end{pgfscope}%
\begin{pgfscope}%
\definecolor{textcolor}{rgb}{0.000000,0.000000,0.000000}%
\pgfsetstrokecolor{textcolor}%
\pgfsetfillcolor{textcolor}%
\pgftext[x=3.434470in,y=1.102055in,,top]{\color{textcolor}\rmfamily\fontsize{10.000000}{12.000000}\selectfont \(\displaystyle {20}\)}%
\end{pgfscope}%
\begin{pgfscope}%
\pgfsetrectcap%
\pgfsetroundjoin%
\pgfsetlinewidth{0.803000pt}%
\definecolor{currentstroke}{rgb}{0.000000,0.000000,0.000000}%
\pgfsetstrokecolor{currentstroke}%
\pgfsetdash{}{0pt}%
\pgfpathmoveto{\pgfqpoint{3.388063in}{1.425134in}}%
\pgfpathlineto{\pgfqpoint{3.437874in}{1.410413in}}%
\pgfusepath{stroke}%
\end{pgfscope}%
\begin{pgfscope}%
\definecolor{textcolor}{rgb}{0.000000,0.000000,0.000000}%
\pgfsetstrokecolor{textcolor}%
\pgfsetfillcolor{textcolor}%
\pgftext[x=3.571814in,y=1.247035in,,top]{\color{textcolor}\rmfamily\fontsize{10.000000}{12.000000}\selectfont \(\displaystyle {25}\)}%
\end{pgfscope}%
\begin{pgfscope}%
\pgfsetrectcap%
\pgfsetroundjoin%
\pgfsetlinewidth{0.803000pt}%
\definecolor{currentstroke}{rgb}{0.000000,0.000000,0.000000}%
\pgfsetstrokecolor{currentstroke}%
\pgfsetdash{}{0pt}%
\pgfpathmoveto{\pgfqpoint{3.558144in}{1.577751in}}%
\pgfpathlineto{\pgfqpoint{3.628038in}{3.104037in}}%
\pgfusepath{stroke}%
\end{pgfscope}%
\begin{pgfscope}%
\definecolor{textcolor}{rgb}{0.000000,0.000000,0.000000}%
\pgfsetstrokecolor{textcolor}%
\pgfsetfillcolor{textcolor}%
\pgftext[x=4.167903in, y=1.963517in, left, base,rotate=87.378092]{\color{textcolor}\rmfamily\fontsize{10.000000}{12.000000}\selectfont Position Z [\(\displaystyle m\)]}%
\end{pgfscope}%
\begin{pgfscope}%
\pgfsetbuttcap%
\pgfsetroundjoin%
\pgfsetlinewidth{0.803000pt}%
\definecolor{currentstroke}{rgb}{0.690196,0.690196,0.690196}%
\pgfsetstrokecolor{currentstroke}%
\pgfsetdash{}{0pt}%
\pgfpathmoveto{\pgfqpoint{3.563216in}{1.688511in}}%
\pgfpathlineto{\pgfqpoint{1.598380in}{2.254301in}}%
\pgfpathlineto{\pgfqpoint{0.373612in}{1.237346in}}%
\pgfusepath{stroke}%
\end{pgfscope}%
\begin{pgfscope}%
\pgfsetbuttcap%
\pgfsetroundjoin%
\pgfsetlinewidth{0.803000pt}%
\definecolor{currentstroke}{rgb}{0.690196,0.690196,0.690196}%
\pgfsetstrokecolor{currentstroke}%
\pgfsetdash{}{0pt}%
\pgfpathmoveto{\pgfqpoint{3.574656in}{1.938323in}}%
\pgfpathlineto{\pgfqpoint{1.595599in}{2.496147in}}%
\pgfpathlineto{\pgfqpoint{0.361290in}{1.493233in}}%
\pgfusepath{stroke}%
\end{pgfscope}%
\begin{pgfscope}%
\pgfsetbuttcap%
\pgfsetroundjoin%
\pgfsetlinewidth{0.803000pt}%
\definecolor{currentstroke}{rgb}{0.690196,0.690196,0.690196}%
\pgfsetstrokecolor{currentstroke}%
\pgfsetdash{}{0pt}%
\pgfpathmoveto{\pgfqpoint{3.586264in}{2.191812in}}%
\pgfpathlineto{\pgfqpoint{1.592779in}{2.741383in}}%
\pgfpathlineto{\pgfqpoint{0.348779in}{1.753032in}}%
\pgfusepath{stroke}%
\end{pgfscope}%
\begin{pgfscope}%
\pgfsetbuttcap%
\pgfsetroundjoin%
\pgfsetlinewidth{0.803000pt}%
\definecolor{currentstroke}{rgb}{0.690196,0.690196,0.690196}%
\pgfsetstrokecolor{currentstroke}%
\pgfsetdash{}{0pt}%
\pgfpathmoveto{\pgfqpoint{3.598044in}{2.449062in}}%
\pgfpathlineto{\pgfqpoint{1.589919in}{2.990081in}}%
\pgfpathlineto{\pgfqpoint{0.336075in}{2.016833in}}%
\pgfusepath{stroke}%
\end{pgfscope}%
\begin{pgfscope}%
\pgfsetbuttcap%
\pgfsetroundjoin%
\pgfsetlinewidth{0.803000pt}%
\definecolor{currentstroke}{rgb}{0.690196,0.690196,0.690196}%
\pgfsetstrokecolor{currentstroke}%
\pgfsetdash{}{0pt}%
\pgfpathmoveto{\pgfqpoint{3.610001in}{2.710157in}}%
\pgfpathlineto{\pgfqpoint{1.587018in}{3.242315in}}%
\pgfpathlineto{\pgfqpoint{0.323174in}{2.284730in}}%
\pgfusepath{stroke}%
\end{pgfscope}%
\begin{pgfscope}%
\pgfsetbuttcap%
\pgfsetroundjoin%
\pgfsetlinewidth{0.803000pt}%
\definecolor{currentstroke}{rgb}{0.690196,0.690196,0.690196}%
\pgfsetstrokecolor{currentstroke}%
\pgfsetdash{}{0pt}%
\pgfpathmoveto{\pgfqpoint{3.622137in}{2.975183in}}%
\pgfpathlineto{\pgfqpoint{1.584076in}{3.498161in}}%
\pgfpathlineto{\pgfqpoint{0.310071in}{2.556819in}}%
\pgfusepath{stroke}%
\end{pgfscope}%
\begin{pgfscope}%
\pgfsetrectcap%
\pgfsetroundjoin%
\pgfsetlinewidth{0.803000pt}%
\definecolor{currentstroke}{rgb}{0.000000,0.000000,0.000000}%
\pgfsetstrokecolor{currentstroke}%
\pgfsetdash{}{0pt}%
\pgfpathmoveto{\pgfqpoint{3.546724in}{1.693260in}}%
\pgfpathlineto{\pgfqpoint{3.596242in}{1.679001in}}%
\pgfusepath{stroke}%
\end{pgfscope}%
\begin{pgfscope}%
\definecolor{textcolor}{rgb}{0.000000,0.000000,0.000000}%
\pgfsetstrokecolor{textcolor}%
\pgfsetfillcolor{textcolor}%
\pgftext[x=3.817476in,y=1.724476in,,top]{\color{textcolor}\rmfamily\fontsize{10.000000}{12.000000}\selectfont \(\displaystyle {-2}\)}%
\end{pgfscope}%
\begin{pgfscope}%
\pgfsetrectcap%
\pgfsetroundjoin%
\pgfsetlinewidth{0.803000pt}%
\definecolor{currentstroke}{rgb}{0.000000,0.000000,0.000000}%
\pgfsetstrokecolor{currentstroke}%
\pgfsetdash{}{0pt}%
\pgfpathmoveto{\pgfqpoint{3.558038in}{1.943007in}}%
\pgfpathlineto{\pgfqpoint{3.607932in}{1.928943in}}%
\pgfusepath{stroke}%
\end{pgfscope}%
\begin{pgfscope}%
\definecolor{textcolor}{rgb}{0.000000,0.000000,0.000000}%
\pgfsetstrokecolor{textcolor}%
\pgfsetfillcolor{textcolor}%
\pgftext[x=3.830734in,y=1.973792in,,top]{\color{textcolor}\rmfamily\fontsize{10.000000}{12.000000}\selectfont \(\displaystyle {0}\)}%
\end{pgfscope}%
\begin{pgfscope}%
\pgfsetrectcap%
\pgfsetroundjoin%
\pgfsetlinewidth{0.803000pt}%
\definecolor{currentstroke}{rgb}{0.000000,0.000000,0.000000}%
\pgfsetstrokecolor{currentstroke}%
\pgfsetdash{}{0pt}%
\pgfpathmoveto{\pgfqpoint{3.569519in}{2.196429in}}%
\pgfpathlineto{\pgfqpoint{3.619795in}{2.182569in}}%
\pgfusepath{stroke}%
\end{pgfscope}%
\begin{pgfscope}%
\definecolor{textcolor}{rgb}{0.000000,0.000000,0.000000}%
\pgfsetstrokecolor{textcolor}%
\pgfsetfillcolor{textcolor}%
\pgftext[x=3.844186in,y=2.226769in,,top]{\color{textcolor}\rmfamily\fontsize{10.000000}{12.000000}\selectfont \(\displaystyle {2}\)}%
\end{pgfscope}%
\begin{pgfscope}%
\pgfsetrectcap%
\pgfsetroundjoin%
\pgfsetlinewidth{0.803000pt}%
\definecolor{currentstroke}{rgb}{0.000000,0.000000,0.000000}%
\pgfsetstrokecolor{currentstroke}%
\pgfsetdash{}{0pt}%
\pgfpathmoveto{\pgfqpoint{3.581170in}{2.453609in}}%
\pgfpathlineto{\pgfqpoint{3.631833in}{2.439959in}}%
\pgfusepath{stroke}%
\end{pgfscope}%
\begin{pgfscope}%
\definecolor{textcolor}{rgb}{0.000000,0.000000,0.000000}%
\pgfsetstrokecolor{textcolor}%
\pgfsetfillcolor{textcolor}%
\pgftext[x=3.857837in,y=2.483486in,,top]{\color{textcolor}\rmfamily\fontsize{10.000000}{12.000000}\selectfont \(\displaystyle {4}\)}%
\end{pgfscope}%
\begin{pgfscope}%
\pgfsetrectcap%
\pgfsetroundjoin%
\pgfsetlinewidth{0.803000pt}%
\definecolor{currentstroke}{rgb}{0.000000,0.000000,0.000000}%
\pgfsetstrokecolor{currentstroke}%
\pgfsetdash{}{0pt}%
\pgfpathmoveto{\pgfqpoint{3.592996in}{2.714630in}}%
\pgfpathlineto{\pgfqpoint{3.644052in}{2.701199in}}%
\pgfusepath{stroke}%
\end{pgfscope}%
\begin{pgfscope}%
\definecolor{textcolor}{rgb}{0.000000,0.000000,0.000000}%
\pgfsetstrokecolor{textcolor}%
\pgfsetfillcolor{textcolor}%
\pgftext[x=3.871691in,y=2.744029in,,top]{\color{textcolor}\rmfamily\fontsize{10.000000}{12.000000}\selectfont \(\displaystyle {6}\)}%
\end{pgfscope}%
\begin{pgfscope}%
\pgfsetrectcap%
\pgfsetroundjoin%
\pgfsetlinewidth{0.803000pt}%
\definecolor{currentstroke}{rgb}{0.000000,0.000000,0.000000}%
\pgfsetstrokecolor{currentstroke}%
\pgfsetdash{}{0pt}%
\pgfpathmoveto{\pgfqpoint{3.604999in}{2.979580in}}%
\pgfpathlineto{\pgfqpoint{3.656455in}{2.966376in}}%
\pgfusepath{stroke}%
\end{pgfscope}%
\begin{pgfscope}%
\definecolor{textcolor}{rgb}{0.000000,0.000000,0.000000}%
\pgfsetstrokecolor{textcolor}%
\pgfsetfillcolor{textcolor}%
\pgftext[x=3.885754in,y=3.008481in,,top]{\color{textcolor}\rmfamily\fontsize{10.000000}{12.000000}\selectfont \(\displaystyle {8}\)}%
\end{pgfscope}%
\begin{pgfscope}%
\pgfpathrectangle{\pgfqpoint{0.100000in}{0.212622in}}{\pgfqpoint{3.696000in}{3.696000in}}%
\pgfusepath{clip}%
\pgfsetrectcap%
\pgfsetroundjoin%
\pgfsetlinewidth{1.505625pt}%
\definecolor{currentstroke}{rgb}{0.121569,0.466667,0.705882}%
\pgfsetstrokecolor{currentstroke}%
\pgfsetdash{}{0pt}%
\pgfpathmoveto{\pgfqpoint{0.865737in}{1.611460in}}%
\pgfpathlineto{\pgfqpoint{1.764737in}{2.360069in}}%
\pgfpathlineto{\pgfqpoint{2.046127in}{1.245600in}}%
\pgfpathlineto{\pgfqpoint{0.865737in}{1.611460in}}%
\pgfusepath{stroke}%
\end{pgfscope}%
\begin{pgfscope}%
\pgfpathrectangle{\pgfqpoint{0.100000in}{0.212622in}}{\pgfqpoint{3.696000in}{3.696000in}}%
\pgfusepath{clip}%
\pgfsetrectcap%
\pgfsetroundjoin%
\pgfsetlinewidth{1.505625pt}%
\definecolor{currentstroke}{rgb}{1.000000,0.000000,0.000000}%
\pgfsetstrokecolor{currentstroke}%
\pgfsetdash{}{0pt}%
\pgfpathmoveto{\pgfqpoint{0.864999in}{1.612012in}}%
\pgfpathlineto{\pgfqpoint{0.865737in}{1.611460in}}%
\pgfusepath{stroke}%
\end{pgfscope}%
\begin{pgfscope}%
\pgfpathrectangle{\pgfqpoint{0.100000in}{0.212622in}}{\pgfqpoint{3.696000in}{3.696000in}}%
\pgfusepath{clip}%
\pgfsetrectcap%
\pgfsetroundjoin%
\pgfsetlinewidth{1.505625pt}%
\definecolor{currentstroke}{rgb}{1.000000,0.000000,0.000000}%
\pgfsetstrokecolor{currentstroke}%
\pgfsetdash{}{0pt}%
\pgfpathmoveto{\pgfqpoint{0.898778in}{1.558189in}}%
\pgfpathlineto{\pgfqpoint{0.865737in}{1.611460in}}%
\pgfusepath{stroke}%
\end{pgfscope}%
\begin{pgfscope}%
\pgfpathrectangle{\pgfqpoint{0.100000in}{0.212622in}}{\pgfqpoint{3.696000in}{3.696000in}}%
\pgfusepath{clip}%
\pgfsetrectcap%
\pgfsetroundjoin%
\pgfsetlinewidth{1.505625pt}%
\definecolor{currentstroke}{rgb}{1.000000,0.000000,0.000000}%
\pgfsetstrokecolor{currentstroke}%
\pgfsetdash{}{0pt}%
\pgfpathmoveto{\pgfqpoint{0.859925in}{1.523662in}}%
\pgfpathlineto{\pgfqpoint{0.865737in}{1.611460in}}%
\pgfusepath{stroke}%
\end{pgfscope}%
\begin{pgfscope}%
\pgfpathrectangle{\pgfqpoint{0.100000in}{0.212622in}}{\pgfqpoint{3.696000in}{3.696000in}}%
\pgfusepath{clip}%
\pgfsetrectcap%
\pgfsetroundjoin%
\pgfsetlinewidth{1.505625pt}%
\definecolor{currentstroke}{rgb}{1.000000,0.000000,0.000000}%
\pgfsetstrokecolor{currentstroke}%
\pgfsetdash{}{0pt}%
\pgfpathmoveto{\pgfqpoint{0.902127in}{1.609687in}}%
\pgfpathlineto{\pgfqpoint{0.865737in}{1.611460in}}%
\pgfusepath{stroke}%
\end{pgfscope}%
\begin{pgfscope}%
\pgfpathrectangle{\pgfqpoint{0.100000in}{0.212622in}}{\pgfqpoint{3.696000in}{3.696000in}}%
\pgfusepath{clip}%
\pgfsetrectcap%
\pgfsetroundjoin%
\pgfsetlinewidth{1.505625pt}%
\definecolor{currentstroke}{rgb}{1.000000,0.000000,0.000000}%
\pgfsetstrokecolor{currentstroke}%
\pgfsetdash{}{0pt}%
\pgfpathmoveto{\pgfqpoint{1.878396in}{3.370216in}}%
\pgfpathlineto{\pgfqpoint{1.764737in}{2.360069in}}%
\pgfusepath{stroke}%
\end{pgfscope}%
\begin{pgfscope}%
\pgfpathrectangle{\pgfqpoint{0.100000in}{0.212622in}}{\pgfqpoint{3.696000in}{3.696000in}}%
\pgfusepath{clip}%
\pgfsetrectcap%
\pgfsetroundjoin%
\pgfsetlinewidth{1.505625pt}%
\definecolor{currentstroke}{rgb}{1.000000,0.000000,0.000000}%
\pgfsetstrokecolor{currentstroke}%
\pgfsetdash{}{0pt}%
\pgfpathmoveto{\pgfqpoint{1.931998in}{3.220125in}}%
\pgfpathlineto{\pgfqpoint{1.764737in}{2.360069in}}%
\pgfusepath{stroke}%
\end{pgfscope}%
\begin{pgfscope}%
\pgfpathrectangle{\pgfqpoint{0.100000in}{0.212622in}}{\pgfqpoint{3.696000in}{3.696000in}}%
\pgfusepath{clip}%
\pgfsetrectcap%
\pgfsetroundjoin%
\pgfsetlinewidth{1.505625pt}%
\definecolor{currentstroke}{rgb}{1.000000,0.000000,0.000000}%
\pgfsetstrokecolor{currentstroke}%
\pgfsetdash{}{0pt}%
\pgfpathmoveto{\pgfqpoint{2.106736in}{2.101721in}}%
\pgfpathlineto{\pgfqpoint{2.046127in}{1.245600in}}%
\pgfusepath{stroke}%
\end{pgfscope}%
\begin{pgfscope}%
\pgfpathrectangle{\pgfqpoint{0.100000in}{0.212622in}}{\pgfqpoint{3.696000in}{3.696000in}}%
\pgfusepath{clip}%
\pgfsetrectcap%
\pgfsetroundjoin%
\pgfsetlinewidth{1.505625pt}%
\definecolor{currentstroke}{rgb}{1.000000,0.000000,0.000000}%
\pgfsetstrokecolor{currentstroke}%
\pgfsetdash{}{0pt}%
\pgfpathmoveto{\pgfqpoint{2.443691in}{0.787390in}}%
\pgfpathlineto{\pgfqpoint{2.046127in}{1.245600in}}%
\pgfusepath{stroke}%
\end{pgfscope}%
\begin{pgfscope}%
\pgfpathrectangle{\pgfqpoint{0.100000in}{0.212622in}}{\pgfqpoint{3.696000in}{3.696000in}}%
\pgfusepath{clip}%
\pgfsetrectcap%
\pgfsetroundjoin%
\pgfsetlinewidth{1.505625pt}%
\definecolor{currentstroke}{rgb}{1.000000,0.000000,0.000000}%
\pgfsetstrokecolor{currentstroke}%
\pgfsetdash{}{0pt}%
\pgfpathmoveto{\pgfqpoint{2.084811in}{0.993331in}}%
\pgfpathlineto{\pgfqpoint{2.046127in}{1.245600in}}%
\pgfusepath{stroke}%
\end{pgfscope}%
\begin{pgfscope}%
\pgfpathrectangle{\pgfqpoint{0.100000in}{0.212622in}}{\pgfqpoint{3.696000in}{3.696000in}}%
\pgfusepath{clip}%
\pgfsetrectcap%
\pgfsetroundjoin%
\pgfsetlinewidth{1.505625pt}%
\definecolor{currentstroke}{rgb}{1.000000,0.000000,0.000000}%
\pgfsetstrokecolor{currentstroke}%
\pgfsetdash{}{0pt}%
\pgfpathmoveto{\pgfqpoint{1.400390in}{1.215425in}}%
\pgfpathlineto{\pgfqpoint{2.046127in}{1.245600in}}%
\pgfusepath{stroke}%
\end{pgfscope}%
\begin{pgfscope}%
\pgfpathrectangle{\pgfqpoint{0.100000in}{0.212622in}}{\pgfqpoint{3.696000in}{3.696000in}}%
\pgfusepath{clip}%
\pgfsetrectcap%
\pgfsetroundjoin%
\pgfsetlinewidth{1.505625pt}%
\definecolor{currentstroke}{rgb}{1.000000,0.000000,0.000000}%
\pgfsetstrokecolor{currentstroke}%
\pgfsetdash{}{0pt}%
\pgfpathmoveto{\pgfqpoint{0.585743in}{1.251650in}}%
\pgfpathlineto{\pgfqpoint{0.865737in}{1.611460in}}%
\pgfusepath{stroke}%
\end{pgfscope}%
\begin{pgfscope}%
\pgfpathrectangle{\pgfqpoint{0.100000in}{0.212622in}}{\pgfqpoint{3.696000in}{3.696000in}}%
\pgfusepath{clip}%
\pgfsetbuttcap%
\pgfsetroundjoin%
\definecolor{currentfill}{rgb}{1.000000,0.498039,0.054902}%
\pgfsetfillcolor{currentfill}%
\pgfsetfillopacity{0.300000}%
\pgfsetlinewidth{1.003750pt}%
\definecolor{currentstroke}{rgb}{1.000000,0.498039,0.054902}%
\pgfsetstrokecolor{currentstroke}%
\pgfsetstrokeopacity{0.300000}%
\pgfsetdash{}{0pt}%
\pgfpathmoveto{\pgfqpoint{1.878396in}{3.339159in}}%
\pgfpathcurveto{\pgfqpoint{1.886632in}{3.339159in}}{\pgfqpoint{1.894532in}{3.342432in}}{\pgfqpoint{1.900356in}{3.348255in}}%
\pgfpathcurveto{\pgfqpoint{1.906180in}{3.354079in}}{\pgfqpoint{1.909453in}{3.361979in}}{\pgfqpoint{1.909453in}{3.370216in}}%
\pgfpathcurveto{\pgfqpoint{1.909453in}{3.378452in}}{\pgfqpoint{1.906180in}{3.386352in}}{\pgfqpoint{1.900356in}{3.392176in}}%
\pgfpathcurveto{\pgfqpoint{1.894532in}{3.398000in}}{\pgfqpoint{1.886632in}{3.401272in}}{\pgfqpoint{1.878396in}{3.401272in}}%
\pgfpathcurveto{\pgfqpoint{1.870160in}{3.401272in}}{\pgfqpoint{1.862260in}{3.398000in}}{\pgfqpoint{1.856436in}{3.392176in}}%
\pgfpathcurveto{\pgfqpoint{1.850612in}{3.386352in}}{\pgfqpoint{1.847340in}{3.378452in}}{\pgfqpoint{1.847340in}{3.370216in}}%
\pgfpathcurveto{\pgfqpoint{1.847340in}{3.361979in}}{\pgfqpoint{1.850612in}{3.354079in}}{\pgfqpoint{1.856436in}{3.348255in}}%
\pgfpathcurveto{\pgfqpoint{1.862260in}{3.342432in}}{\pgfqpoint{1.870160in}{3.339159in}}{\pgfqpoint{1.878396in}{3.339159in}}%
\pgfpathclose%
\pgfusepath{stroke,fill}%
\end{pgfscope}%
\begin{pgfscope}%
\pgfpathrectangle{\pgfqpoint{0.100000in}{0.212622in}}{\pgfqpoint{3.696000in}{3.696000in}}%
\pgfusepath{clip}%
\pgfsetbuttcap%
\pgfsetroundjoin%
\definecolor{currentfill}{rgb}{1.000000,0.498039,0.054902}%
\pgfsetfillcolor{currentfill}%
\pgfsetfillopacity{0.329481}%
\pgfsetlinewidth{1.003750pt}%
\definecolor{currentstroke}{rgb}{1.000000,0.498039,0.054902}%
\pgfsetstrokecolor{currentstroke}%
\pgfsetstrokeopacity{0.329481}%
\pgfsetdash{}{0pt}%
\pgfpathmoveto{\pgfqpoint{1.931998in}{3.189069in}}%
\pgfpathcurveto{\pgfqpoint{1.940235in}{3.189069in}}{\pgfqpoint{1.948135in}{3.192341in}}{\pgfqpoint{1.953959in}{3.198165in}}%
\pgfpathcurveto{\pgfqpoint{1.959783in}{3.203989in}}{\pgfqpoint{1.963055in}{3.211889in}}{\pgfqpoint{1.963055in}{3.220125in}}%
\pgfpathcurveto{\pgfqpoint{1.963055in}{3.228361in}}{\pgfqpoint{1.959783in}{3.236261in}}{\pgfqpoint{1.953959in}{3.242085in}}%
\pgfpathcurveto{\pgfqpoint{1.948135in}{3.247909in}}{\pgfqpoint{1.940235in}{3.251182in}}{\pgfqpoint{1.931998in}{3.251182in}}%
\pgfpathcurveto{\pgfqpoint{1.923762in}{3.251182in}}{\pgfqpoint{1.915862in}{3.247909in}}{\pgfqpoint{1.910038in}{3.242085in}}%
\pgfpathcurveto{\pgfqpoint{1.904214in}{3.236261in}}{\pgfqpoint{1.900942in}{3.228361in}}{\pgfqpoint{1.900942in}{3.220125in}}%
\pgfpathcurveto{\pgfqpoint{1.900942in}{3.211889in}}{\pgfqpoint{1.904214in}{3.203989in}}{\pgfqpoint{1.910038in}{3.198165in}}%
\pgfpathcurveto{\pgfqpoint{1.915862in}{3.192341in}}{\pgfqpoint{1.923762in}{3.189069in}}{\pgfqpoint{1.931998in}{3.189069in}}%
\pgfpathclose%
\pgfusepath{stroke,fill}%
\end{pgfscope}%
\begin{pgfscope}%
\pgfpathrectangle{\pgfqpoint{0.100000in}{0.212622in}}{\pgfqpoint{3.696000in}{3.696000in}}%
\pgfusepath{clip}%
\pgfsetbuttcap%
\pgfsetroundjoin%
\definecolor{currentfill}{rgb}{1.000000,0.498039,0.054902}%
\pgfsetfillcolor{currentfill}%
\pgfsetfillopacity{0.578496}%
\pgfsetlinewidth{1.003750pt}%
\definecolor{currentstroke}{rgb}{1.000000,0.498039,0.054902}%
\pgfsetstrokecolor{currentstroke}%
\pgfsetstrokeopacity{0.578496}%
\pgfsetdash{}{0pt}%
\pgfpathmoveto{\pgfqpoint{0.898778in}{1.527132in}}%
\pgfpathcurveto{\pgfqpoint{0.907015in}{1.527132in}}{\pgfqpoint{0.914915in}{1.530405in}}{\pgfqpoint{0.920739in}{1.536229in}}%
\pgfpathcurveto{\pgfqpoint{0.926563in}{1.542053in}}{\pgfqpoint{0.929835in}{1.549953in}}{\pgfqpoint{0.929835in}{1.558189in}}%
\pgfpathcurveto{\pgfqpoint{0.929835in}{1.566425in}}{\pgfqpoint{0.926563in}{1.574325in}}{\pgfqpoint{0.920739in}{1.580149in}}%
\pgfpathcurveto{\pgfqpoint{0.914915in}{1.585973in}}{\pgfqpoint{0.907015in}{1.589245in}}{\pgfqpoint{0.898778in}{1.589245in}}%
\pgfpathcurveto{\pgfqpoint{0.890542in}{1.589245in}}{\pgfqpoint{0.882642in}{1.585973in}}{\pgfqpoint{0.876818in}{1.580149in}}%
\pgfpathcurveto{\pgfqpoint{0.870994in}{1.574325in}}{\pgfqpoint{0.867722in}{1.566425in}}{\pgfqpoint{0.867722in}{1.558189in}}%
\pgfpathcurveto{\pgfqpoint{0.867722in}{1.549953in}}{\pgfqpoint{0.870994in}{1.542053in}}{\pgfqpoint{0.876818in}{1.536229in}}%
\pgfpathcurveto{\pgfqpoint{0.882642in}{1.530405in}}{\pgfqpoint{0.890542in}{1.527132in}}{\pgfqpoint{0.898778in}{1.527132in}}%
\pgfpathclose%
\pgfusepath{stroke,fill}%
\end{pgfscope}%
\begin{pgfscope}%
\pgfpathrectangle{\pgfqpoint{0.100000in}{0.212622in}}{\pgfqpoint{3.696000in}{3.696000in}}%
\pgfusepath{clip}%
\pgfsetbuttcap%
\pgfsetroundjoin%
\definecolor{currentfill}{rgb}{1.000000,0.498039,0.054902}%
\pgfsetfillcolor{currentfill}%
\pgfsetfillopacity{0.587159}%
\pgfsetlinewidth{1.003750pt}%
\definecolor{currentstroke}{rgb}{1.000000,0.498039,0.054902}%
\pgfsetstrokecolor{currentstroke}%
\pgfsetstrokeopacity{0.587159}%
\pgfsetdash{}{0pt}%
\pgfpathmoveto{\pgfqpoint{0.585743in}{1.220593in}}%
\pgfpathcurveto{\pgfqpoint{0.593979in}{1.220593in}}{\pgfqpoint{0.601879in}{1.223866in}}{\pgfqpoint{0.607703in}{1.229690in}}%
\pgfpathcurveto{\pgfqpoint{0.613527in}{1.235514in}}{\pgfqpoint{0.616799in}{1.243414in}}{\pgfqpoint{0.616799in}{1.251650in}}%
\pgfpathcurveto{\pgfqpoint{0.616799in}{1.259886in}}{\pgfqpoint{0.613527in}{1.267786in}}{\pgfqpoint{0.607703in}{1.273610in}}%
\pgfpathcurveto{\pgfqpoint{0.601879in}{1.279434in}}{\pgfqpoint{0.593979in}{1.282706in}}{\pgfqpoint{0.585743in}{1.282706in}}%
\pgfpathcurveto{\pgfqpoint{0.577507in}{1.282706in}}{\pgfqpoint{0.569606in}{1.279434in}}{\pgfqpoint{0.563783in}{1.273610in}}%
\pgfpathcurveto{\pgfqpoint{0.557959in}{1.267786in}}{\pgfqpoint{0.554686in}{1.259886in}}{\pgfqpoint{0.554686in}{1.251650in}}%
\pgfpathcurveto{\pgfqpoint{0.554686in}{1.243414in}}{\pgfqpoint{0.557959in}{1.235514in}}{\pgfqpoint{0.563783in}{1.229690in}}%
\pgfpathcurveto{\pgfqpoint{0.569606in}{1.223866in}}{\pgfqpoint{0.577507in}{1.220593in}}{\pgfqpoint{0.585743in}{1.220593in}}%
\pgfpathclose%
\pgfusepath{stroke,fill}%
\end{pgfscope}%
\begin{pgfscope}%
\pgfpathrectangle{\pgfqpoint{0.100000in}{0.212622in}}{\pgfqpoint{3.696000in}{3.696000in}}%
\pgfusepath{clip}%
\pgfsetbuttcap%
\pgfsetroundjoin%
\definecolor{currentfill}{rgb}{1.000000,0.498039,0.054902}%
\pgfsetfillcolor{currentfill}%
\pgfsetfillopacity{0.593933}%
\pgfsetlinewidth{1.003750pt}%
\definecolor{currentstroke}{rgb}{1.000000,0.498039,0.054902}%
\pgfsetstrokecolor{currentstroke}%
\pgfsetstrokeopacity{0.593933}%
\pgfsetdash{}{0pt}%
\pgfpathmoveto{\pgfqpoint{0.864999in}{1.580955in}}%
\pgfpathcurveto{\pgfqpoint{0.873235in}{1.580955in}}{\pgfqpoint{0.881135in}{1.584228in}}{\pgfqpoint{0.886959in}{1.590052in}}%
\pgfpathcurveto{\pgfqpoint{0.892783in}{1.595875in}}{\pgfqpoint{0.896055in}{1.603776in}}{\pgfqpoint{0.896055in}{1.612012in}}%
\pgfpathcurveto{\pgfqpoint{0.896055in}{1.620248in}}{\pgfqpoint{0.892783in}{1.628148in}}{\pgfqpoint{0.886959in}{1.633972in}}%
\pgfpathcurveto{\pgfqpoint{0.881135in}{1.639796in}}{\pgfqpoint{0.873235in}{1.643068in}}{\pgfqpoint{0.864999in}{1.643068in}}%
\pgfpathcurveto{\pgfqpoint{0.856763in}{1.643068in}}{\pgfqpoint{0.848863in}{1.639796in}}{\pgfqpoint{0.843039in}{1.633972in}}%
\pgfpathcurveto{\pgfqpoint{0.837215in}{1.628148in}}{\pgfqpoint{0.833942in}{1.620248in}}{\pgfqpoint{0.833942in}{1.612012in}}%
\pgfpathcurveto{\pgfqpoint{0.833942in}{1.603776in}}{\pgfqpoint{0.837215in}{1.595875in}}{\pgfqpoint{0.843039in}{1.590052in}}%
\pgfpathcurveto{\pgfqpoint{0.848863in}{1.584228in}}{\pgfqpoint{0.856763in}{1.580955in}}{\pgfqpoint{0.864999in}{1.580955in}}%
\pgfpathclose%
\pgfusepath{stroke,fill}%
\end{pgfscope}%
\begin{pgfscope}%
\pgfpathrectangle{\pgfqpoint{0.100000in}{0.212622in}}{\pgfqpoint{3.696000in}{3.696000in}}%
\pgfusepath{clip}%
\pgfsetbuttcap%
\pgfsetroundjoin%
\definecolor{currentfill}{rgb}{1.000000,0.498039,0.054902}%
\pgfsetfillcolor{currentfill}%
\pgfsetfillopacity{0.606580}%
\pgfsetlinewidth{1.003750pt}%
\definecolor{currentstroke}{rgb}{1.000000,0.498039,0.054902}%
\pgfsetstrokecolor{currentstroke}%
\pgfsetstrokeopacity{0.606580}%
\pgfsetdash{}{0pt}%
\pgfpathmoveto{\pgfqpoint{2.106736in}{2.070664in}}%
\pgfpathcurveto{\pgfqpoint{2.114973in}{2.070664in}}{\pgfqpoint{2.122873in}{2.073937in}}{\pgfqpoint{2.128697in}{2.079760in}}%
\pgfpathcurveto{\pgfqpoint{2.134521in}{2.085584in}}{\pgfqpoint{2.137793in}{2.093484in}}{\pgfqpoint{2.137793in}{2.101721in}}%
\pgfpathcurveto{\pgfqpoint{2.137793in}{2.109957in}}{\pgfqpoint{2.134521in}{2.117857in}}{\pgfqpoint{2.128697in}{2.123681in}}%
\pgfpathcurveto{\pgfqpoint{2.122873in}{2.129505in}}{\pgfqpoint{2.114973in}{2.132777in}}{\pgfqpoint{2.106736in}{2.132777in}}%
\pgfpathcurveto{\pgfqpoint{2.098500in}{2.132777in}}{\pgfqpoint{2.090600in}{2.129505in}}{\pgfqpoint{2.084776in}{2.123681in}}%
\pgfpathcurveto{\pgfqpoint{2.078952in}{2.117857in}}{\pgfqpoint{2.075680in}{2.109957in}}{\pgfqpoint{2.075680in}{2.101721in}}%
\pgfpathcurveto{\pgfqpoint{2.075680in}{2.093484in}}{\pgfqpoint{2.078952in}{2.085584in}}{\pgfqpoint{2.084776in}{2.079760in}}%
\pgfpathcurveto{\pgfqpoint{2.090600in}{2.073937in}}{\pgfqpoint{2.098500in}{2.070664in}}{\pgfqpoint{2.106736in}{2.070664in}}%
\pgfpathclose%
\pgfusepath{stroke,fill}%
\end{pgfscope}%
\begin{pgfscope}%
\pgfpathrectangle{\pgfqpoint{0.100000in}{0.212622in}}{\pgfqpoint{3.696000in}{3.696000in}}%
\pgfusepath{clip}%
\pgfsetbuttcap%
\pgfsetroundjoin%
\definecolor{currentfill}{rgb}{1.000000,0.498039,0.054902}%
\pgfsetfillcolor{currentfill}%
\pgfsetfillopacity{0.618828}%
\pgfsetlinewidth{1.003750pt}%
\definecolor{currentstroke}{rgb}{1.000000,0.498039,0.054902}%
\pgfsetstrokecolor{currentstroke}%
\pgfsetstrokeopacity{0.618828}%
\pgfsetdash{}{0pt}%
\pgfpathmoveto{\pgfqpoint{0.902127in}{1.578631in}}%
\pgfpathcurveto{\pgfqpoint{0.910363in}{1.578631in}}{\pgfqpoint{0.918263in}{1.581903in}}{\pgfqpoint{0.924087in}{1.587727in}}%
\pgfpathcurveto{\pgfqpoint{0.929911in}{1.593551in}}{\pgfqpoint{0.933183in}{1.601451in}}{\pgfqpoint{0.933183in}{1.609687in}}%
\pgfpathcurveto{\pgfqpoint{0.933183in}{1.617924in}}{\pgfqpoint{0.929911in}{1.625824in}}{\pgfqpoint{0.924087in}{1.631648in}}%
\pgfpathcurveto{\pgfqpoint{0.918263in}{1.637472in}}{\pgfqpoint{0.910363in}{1.640744in}}{\pgfqpoint{0.902127in}{1.640744in}}%
\pgfpathcurveto{\pgfqpoint{0.893890in}{1.640744in}}{\pgfqpoint{0.885990in}{1.637472in}}{\pgfqpoint{0.880167in}{1.631648in}}%
\pgfpathcurveto{\pgfqpoint{0.874343in}{1.625824in}}{\pgfqpoint{0.871070in}{1.617924in}}{\pgfqpoint{0.871070in}{1.609687in}}%
\pgfpathcurveto{\pgfqpoint{0.871070in}{1.601451in}}{\pgfqpoint{0.874343in}{1.593551in}}{\pgfqpoint{0.880167in}{1.587727in}}%
\pgfpathcurveto{\pgfqpoint{0.885990in}{1.581903in}}{\pgfqpoint{0.893890in}{1.578631in}}{\pgfqpoint{0.902127in}{1.578631in}}%
\pgfpathclose%
\pgfusepath{stroke,fill}%
\end{pgfscope}%
\begin{pgfscope}%
\pgfpathrectangle{\pgfqpoint{0.100000in}{0.212622in}}{\pgfqpoint{3.696000in}{3.696000in}}%
\pgfusepath{clip}%
\pgfsetbuttcap%
\pgfsetroundjoin%
\definecolor{currentfill}{rgb}{1.000000,0.498039,0.054902}%
\pgfsetfillcolor{currentfill}%
\pgfsetfillopacity{0.627170}%
\pgfsetlinewidth{1.003750pt}%
\definecolor{currentstroke}{rgb}{1.000000,0.498039,0.054902}%
\pgfsetstrokecolor{currentstroke}%
\pgfsetstrokeopacity{0.627170}%
\pgfsetdash{}{0pt}%
\pgfpathmoveto{\pgfqpoint{0.859925in}{1.492605in}}%
\pgfpathcurveto{\pgfqpoint{0.868161in}{1.492605in}}{\pgfqpoint{0.876061in}{1.495877in}}{\pgfqpoint{0.881885in}{1.501701in}}%
\pgfpathcurveto{\pgfqpoint{0.887709in}{1.507525in}}{\pgfqpoint{0.890981in}{1.515425in}}{\pgfqpoint{0.890981in}{1.523662in}}%
\pgfpathcurveto{\pgfqpoint{0.890981in}{1.531898in}}{\pgfqpoint{0.887709in}{1.539798in}}{\pgfqpoint{0.881885in}{1.545622in}}%
\pgfpathcurveto{\pgfqpoint{0.876061in}{1.551446in}}{\pgfqpoint{0.868161in}{1.554718in}}{\pgfqpoint{0.859925in}{1.554718in}}%
\pgfpathcurveto{\pgfqpoint{0.851688in}{1.554718in}}{\pgfqpoint{0.843788in}{1.551446in}}{\pgfqpoint{0.837965in}{1.545622in}}%
\pgfpathcurveto{\pgfqpoint{0.832141in}{1.539798in}}{\pgfqpoint{0.828868in}{1.531898in}}{\pgfqpoint{0.828868in}{1.523662in}}%
\pgfpathcurveto{\pgfqpoint{0.828868in}{1.515425in}}{\pgfqpoint{0.832141in}{1.507525in}}{\pgfqpoint{0.837965in}{1.501701in}}%
\pgfpathcurveto{\pgfqpoint{0.843788in}{1.495877in}}{\pgfqpoint{0.851688in}{1.492605in}}{\pgfqpoint{0.859925in}{1.492605in}}%
\pgfpathclose%
\pgfusepath{stroke,fill}%
\end{pgfscope}%
\begin{pgfscope}%
\pgfpathrectangle{\pgfqpoint{0.100000in}{0.212622in}}{\pgfqpoint{3.696000in}{3.696000in}}%
\pgfusepath{clip}%
\pgfsetbuttcap%
\pgfsetroundjoin%
\definecolor{currentfill}{rgb}{1.000000,0.498039,0.054902}%
\pgfsetfillcolor{currentfill}%
\pgfsetfillopacity{0.778128}%
\pgfsetlinewidth{1.003750pt}%
\definecolor{currentstroke}{rgb}{1.000000,0.498039,0.054902}%
\pgfsetstrokecolor{currentstroke}%
\pgfsetstrokeopacity{0.778128}%
\pgfsetdash{}{0pt}%
\pgfpathmoveto{\pgfqpoint{1.400390in}{1.184368in}}%
\pgfpathcurveto{\pgfqpoint{1.408626in}{1.184368in}}{\pgfqpoint{1.416527in}{1.187641in}}{\pgfqpoint{1.422350in}{1.193464in}}%
\pgfpathcurveto{\pgfqpoint{1.428174in}{1.199288in}}{\pgfqpoint{1.431447in}{1.207188in}}{\pgfqpoint{1.431447in}{1.215425in}}%
\pgfpathcurveto{\pgfqpoint{1.431447in}{1.223661in}}{\pgfqpoint{1.428174in}{1.231561in}}{\pgfqpoint{1.422350in}{1.237385in}}%
\pgfpathcurveto{\pgfqpoint{1.416527in}{1.243209in}}{\pgfqpoint{1.408626in}{1.246481in}}{\pgfqpoint{1.400390in}{1.246481in}}%
\pgfpathcurveto{\pgfqpoint{1.392154in}{1.246481in}}{\pgfqpoint{1.384254in}{1.243209in}}{\pgfqpoint{1.378430in}{1.237385in}}%
\pgfpathcurveto{\pgfqpoint{1.372606in}{1.231561in}}{\pgfqpoint{1.369334in}{1.223661in}}{\pgfqpoint{1.369334in}{1.215425in}}%
\pgfpathcurveto{\pgfqpoint{1.369334in}{1.207188in}}{\pgfqpoint{1.372606in}{1.199288in}}{\pgfqpoint{1.378430in}{1.193464in}}%
\pgfpathcurveto{\pgfqpoint{1.384254in}{1.187641in}}{\pgfqpoint{1.392154in}{1.184368in}}{\pgfqpoint{1.400390in}{1.184368in}}%
\pgfpathclose%
\pgfusepath{stroke,fill}%
\end{pgfscope}%
\begin{pgfscope}%
\pgfpathrectangle{\pgfqpoint{0.100000in}{0.212622in}}{\pgfqpoint{3.696000in}{3.696000in}}%
\pgfusepath{clip}%
\pgfsetbuttcap%
\pgfsetroundjoin%
\definecolor{currentfill}{rgb}{1.000000,0.498039,0.054902}%
\pgfsetfillcolor{currentfill}%
\pgfsetfillopacity{0.925805}%
\pgfsetlinewidth{1.003750pt}%
\definecolor{currentstroke}{rgb}{1.000000,0.498039,0.054902}%
\pgfsetstrokecolor{currentstroke}%
\pgfsetstrokeopacity{0.925805}%
\pgfsetdash{}{0pt}%
\pgfpathmoveto{\pgfqpoint{2.084811in}{0.962275in}}%
\pgfpathcurveto{\pgfqpoint{2.093047in}{0.962275in}}{\pgfqpoint{2.100947in}{0.965547in}}{\pgfqpoint{2.106771in}{0.971371in}}%
\pgfpathcurveto{\pgfqpoint{2.112595in}{0.977195in}}{\pgfqpoint{2.115867in}{0.985095in}}{\pgfqpoint{2.115867in}{0.993331in}}%
\pgfpathcurveto{\pgfqpoint{2.115867in}{1.001567in}}{\pgfqpoint{2.112595in}{1.009468in}}{\pgfqpoint{2.106771in}{1.015291in}}%
\pgfpathcurveto{\pgfqpoint{2.100947in}{1.021115in}}{\pgfqpoint{2.093047in}{1.024388in}}{\pgfqpoint{2.084811in}{1.024388in}}%
\pgfpathcurveto{\pgfqpoint{2.076575in}{1.024388in}}{\pgfqpoint{2.068674in}{1.021115in}}{\pgfqpoint{2.062851in}{1.015291in}}%
\pgfpathcurveto{\pgfqpoint{2.057027in}{1.009468in}}{\pgfqpoint{2.053754in}{1.001567in}}{\pgfqpoint{2.053754in}{0.993331in}}%
\pgfpathcurveto{\pgfqpoint{2.053754in}{0.985095in}}{\pgfqpoint{2.057027in}{0.977195in}}{\pgfqpoint{2.062851in}{0.971371in}}%
\pgfpathcurveto{\pgfqpoint{2.068674in}{0.965547in}}{\pgfqpoint{2.076575in}{0.962275in}}{\pgfqpoint{2.084811in}{0.962275in}}%
\pgfpathclose%
\pgfusepath{stroke,fill}%
\end{pgfscope}%
\begin{pgfscope}%
\pgfpathrectangle{\pgfqpoint{0.100000in}{0.212622in}}{\pgfqpoint{3.696000in}{3.696000in}}%
\pgfusepath{clip}%
\pgfsetbuttcap%
\pgfsetroundjoin%
\definecolor{currentfill}{rgb}{1.000000,0.498039,0.054902}%
\pgfsetfillcolor{currentfill}%
\pgfsetlinewidth{1.003750pt}%
\definecolor{currentstroke}{rgb}{1.000000,0.498039,0.054902}%
\pgfsetstrokecolor{currentstroke}%
\pgfsetdash{}{0pt}%
\pgfpathmoveto{\pgfqpoint{2.443691in}{0.756333in}}%
\pgfpathcurveto{\pgfqpoint{2.451927in}{0.756333in}}{\pgfqpoint{2.459827in}{0.759605in}}{\pgfqpoint{2.465651in}{0.765429in}}%
\pgfpathcurveto{\pgfqpoint{2.471475in}{0.771253in}}{\pgfqpoint{2.474747in}{0.779153in}}{\pgfqpoint{2.474747in}{0.787390in}}%
\pgfpathcurveto{\pgfqpoint{2.474747in}{0.795626in}}{\pgfqpoint{2.471475in}{0.803526in}}{\pgfqpoint{2.465651in}{0.809350in}}%
\pgfpathcurveto{\pgfqpoint{2.459827in}{0.815174in}}{\pgfqpoint{2.451927in}{0.818446in}}{\pgfqpoint{2.443691in}{0.818446in}}%
\pgfpathcurveto{\pgfqpoint{2.435454in}{0.818446in}}{\pgfqpoint{2.427554in}{0.815174in}}{\pgfqpoint{2.421730in}{0.809350in}}%
\pgfpathcurveto{\pgfqpoint{2.415906in}{0.803526in}}{\pgfqpoint{2.412634in}{0.795626in}}{\pgfqpoint{2.412634in}{0.787390in}}%
\pgfpathcurveto{\pgfqpoint{2.412634in}{0.779153in}}{\pgfqpoint{2.415906in}{0.771253in}}{\pgfqpoint{2.421730in}{0.765429in}}%
\pgfpathcurveto{\pgfqpoint{2.427554in}{0.759605in}}{\pgfqpoint{2.435454in}{0.756333in}}{\pgfqpoint{2.443691in}{0.756333in}}%
\pgfpathclose%
\pgfusepath{stroke,fill}%
\end{pgfscope}%
\begin{pgfscope}%
\definecolor{textcolor}{rgb}{0.000000,0.000000,0.000000}%
\pgfsetstrokecolor{textcolor}%
\pgfsetfillcolor{textcolor}%
\pgftext[x=1.948000in,y=3.991956in,,base]{\color{textcolor}\rmfamily\fontsize{12.000000}{14.400000}\selectfont EKF}%
\end{pgfscope}%
\begin{pgfscope}%
\pgfpathrectangle{\pgfqpoint{0.100000in}{0.212622in}}{\pgfqpoint{3.696000in}{3.696000in}}%
\pgfusepath{clip}%
\pgfsetbuttcap%
\pgfsetroundjoin%
\definecolor{currentfill}{rgb}{0.121569,0.466667,0.705882}%
\pgfsetfillcolor{currentfill}%
\pgfsetfillopacity{0.300000}%
\pgfsetlinewidth{1.003750pt}%
\definecolor{currentstroke}{rgb}{0.121569,0.466667,0.705882}%
\pgfsetstrokecolor{currentstroke}%
\pgfsetstrokeopacity{0.300000}%
\pgfsetdash{}{0pt}%
\pgfpathmoveto{\pgfqpoint{1.874951in}{3.342068in}}%
\pgfpathcurveto{\pgfqpoint{1.883187in}{3.342068in}}{\pgfqpoint{1.891088in}{3.345341in}}{\pgfqpoint{1.896911in}{3.351164in}}%
\pgfpathcurveto{\pgfqpoint{1.902735in}{3.356988in}}{\pgfqpoint{1.906008in}{3.364888in}}{\pgfqpoint{1.906008in}{3.373125in}}%
\pgfpathcurveto{\pgfqpoint{1.906008in}{3.381361in}}{\pgfqpoint{1.902735in}{3.389261in}}{\pgfqpoint{1.896911in}{3.395085in}}%
\pgfpathcurveto{\pgfqpoint{1.891088in}{3.400909in}}{\pgfqpoint{1.883187in}{3.404181in}}{\pgfqpoint{1.874951in}{3.404181in}}%
\pgfpathcurveto{\pgfqpoint{1.866715in}{3.404181in}}{\pgfqpoint{1.858815in}{3.400909in}}{\pgfqpoint{1.852991in}{3.395085in}}%
\pgfpathcurveto{\pgfqpoint{1.847167in}{3.389261in}}{\pgfqpoint{1.843895in}{3.381361in}}{\pgfqpoint{1.843895in}{3.373125in}}%
\pgfpathcurveto{\pgfqpoint{1.843895in}{3.364888in}}{\pgfqpoint{1.847167in}{3.356988in}}{\pgfqpoint{1.852991in}{3.351164in}}%
\pgfpathcurveto{\pgfqpoint{1.858815in}{3.345341in}}{\pgfqpoint{1.866715in}{3.342068in}}{\pgfqpoint{1.874951in}{3.342068in}}%
\pgfpathclose%
\pgfusepath{stroke,fill}%
\end{pgfscope}%
\begin{pgfscope}%
\pgfpathrectangle{\pgfqpoint{0.100000in}{0.212622in}}{\pgfqpoint{3.696000in}{3.696000in}}%
\pgfusepath{clip}%
\pgfsetbuttcap%
\pgfsetroundjoin%
\definecolor{currentfill}{rgb}{0.121569,0.466667,0.705882}%
\pgfsetfillcolor{currentfill}%
\pgfsetfillopacity{0.300016}%
\pgfsetlinewidth{1.003750pt}%
\definecolor{currentstroke}{rgb}{0.121569,0.466667,0.705882}%
\pgfsetstrokecolor{currentstroke}%
\pgfsetstrokeopacity{0.300016}%
\pgfsetdash{}{0pt}%
\pgfpathmoveto{\pgfqpoint{1.876319in}{3.341152in}}%
\pgfpathcurveto{\pgfqpoint{1.884555in}{3.341152in}}{\pgfqpoint{1.892455in}{3.344424in}}{\pgfqpoint{1.898279in}{3.350248in}}%
\pgfpathcurveto{\pgfqpoint{1.904103in}{3.356072in}}{\pgfqpoint{1.907376in}{3.363972in}}{\pgfqpoint{1.907376in}{3.372208in}}%
\pgfpathcurveto{\pgfqpoint{1.907376in}{3.380445in}}{\pgfqpoint{1.904103in}{3.388345in}}{\pgfqpoint{1.898279in}{3.394169in}}%
\pgfpathcurveto{\pgfqpoint{1.892455in}{3.399992in}}{\pgfqpoint{1.884555in}{3.403265in}}{\pgfqpoint{1.876319in}{3.403265in}}%
\pgfpathcurveto{\pgfqpoint{1.868083in}{3.403265in}}{\pgfqpoint{1.860183in}{3.399992in}}{\pgfqpoint{1.854359in}{3.394169in}}%
\pgfpathcurveto{\pgfqpoint{1.848535in}{3.388345in}}{\pgfqpoint{1.845263in}{3.380445in}}{\pgfqpoint{1.845263in}{3.372208in}}%
\pgfpathcurveto{\pgfqpoint{1.845263in}{3.363972in}}{\pgfqpoint{1.848535in}{3.356072in}}{\pgfqpoint{1.854359in}{3.350248in}}%
\pgfpathcurveto{\pgfqpoint{1.860183in}{3.344424in}}{\pgfqpoint{1.868083in}{3.341152in}}{\pgfqpoint{1.876319in}{3.341152in}}%
\pgfpathclose%
\pgfusepath{stroke,fill}%
\end{pgfscope}%
\begin{pgfscope}%
\pgfpathrectangle{\pgfqpoint{0.100000in}{0.212622in}}{\pgfqpoint{3.696000in}{3.696000in}}%
\pgfusepath{clip}%
\pgfsetbuttcap%
\pgfsetroundjoin%
\definecolor{currentfill}{rgb}{0.121569,0.466667,0.705882}%
\pgfsetfillcolor{currentfill}%
\pgfsetfillopacity{0.300020}%
\pgfsetlinewidth{1.003750pt}%
\definecolor{currentstroke}{rgb}{0.121569,0.466667,0.705882}%
\pgfsetstrokecolor{currentstroke}%
\pgfsetstrokeopacity{0.300020}%
\pgfsetdash{}{0pt}%
\pgfpathmoveto{\pgfqpoint{1.874440in}{3.342430in}}%
\pgfpathcurveto{\pgfqpoint{1.882677in}{3.342430in}}{\pgfqpoint{1.890577in}{3.345702in}}{\pgfqpoint{1.896401in}{3.351526in}}%
\pgfpathcurveto{\pgfqpoint{1.902225in}{3.357350in}}{\pgfqpoint{1.905497in}{3.365250in}}{\pgfqpoint{1.905497in}{3.373486in}}%
\pgfpathcurveto{\pgfqpoint{1.905497in}{3.381722in}}{\pgfqpoint{1.902225in}{3.389623in}}{\pgfqpoint{1.896401in}{3.395446in}}%
\pgfpathcurveto{\pgfqpoint{1.890577in}{3.401270in}}{\pgfqpoint{1.882677in}{3.404543in}}{\pgfqpoint{1.874440in}{3.404543in}}%
\pgfpathcurveto{\pgfqpoint{1.866204in}{3.404543in}}{\pgfqpoint{1.858304in}{3.401270in}}{\pgfqpoint{1.852480in}{3.395446in}}%
\pgfpathcurveto{\pgfqpoint{1.846656in}{3.389623in}}{\pgfqpoint{1.843384in}{3.381722in}}{\pgfqpoint{1.843384in}{3.373486in}}%
\pgfpathcurveto{\pgfqpoint{1.843384in}{3.365250in}}{\pgfqpoint{1.846656in}{3.357350in}}{\pgfqpoint{1.852480in}{3.351526in}}%
\pgfpathcurveto{\pgfqpoint{1.858304in}{3.345702in}}{\pgfqpoint{1.866204in}{3.342430in}}{\pgfqpoint{1.874440in}{3.342430in}}%
\pgfpathclose%
\pgfusepath{stroke,fill}%
\end{pgfscope}%
\begin{pgfscope}%
\pgfpathrectangle{\pgfqpoint{0.100000in}{0.212622in}}{\pgfqpoint{3.696000in}{3.696000in}}%
\pgfusepath{clip}%
\pgfsetbuttcap%
\pgfsetroundjoin%
\definecolor{currentfill}{rgb}{0.121569,0.466667,0.705882}%
\pgfsetfillcolor{currentfill}%
\pgfsetfillopacity{0.300021}%
\pgfsetlinewidth{1.003750pt}%
\definecolor{currentstroke}{rgb}{0.121569,0.466667,0.705882}%
\pgfsetstrokecolor{currentstroke}%
\pgfsetstrokeopacity{0.300021}%
\pgfsetdash{}{0pt}%
\pgfpathmoveto{\pgfqpoint{1.877042in}{3.340509in}}%
\pgfpathcurveto{\pgfqpoint{1.885278in}{3.340509in}}{\pgfqpoint{1.893178in}{3.343781in}}{\pgfqpoint{1.899002in}{3.349605in}}%
\pgfpathcurveto{\pgfqpoint{1.904826in}{3.355429in}}{\pgfqpoint{1.908098in}{3.363329in}}{\pgfqpoint{1.908098in}{3.371565in}}%
\pgfpathcurveto{\pgfqpoint{1.908098in}{3.379802in}}{\pgfqpoint{1.904826in}{3.387702in}}{\pgfqpoint{1.899002in}{3.393526in}}%
\pgfpathcurveto{\pgfqpoint{1.893178in}{3.399350in}}{\pgfqpoint{1.885278in}{3.402622in}}{\pgfqpoint{1.877042in}{3.402622in}}%
\pgfpathcurveto{\pgfqpoint{1.868805in}{3.402622in}}{\pgfqpoint{1.860905in}{3.399350in}}{\pgfqpoint{1.855081in}{3.393526in}}%
\pgfpathcurveto{\pgfqpoint{1.849257in}{3.387702in}}{\pgfqpoint{1.845985in}{3.379802in}}{\pgfqpoint{1.845985in}{3.371565in}}%
\pgfpathcurveto{\pgfqpoint{1.845985in}{3.363329in}}{\pgfqpoint{1.849257in}{3.355429in}}{\pgfqpoint{1.855081in}{3.349605in}}%
\pgfpathcurveto{\pgfqpoint{1.860905in}{3.343781in}}{\pgfqpoint{1.868805in}{3.340509in}}{\pgfqpoint{1.877042in}{3.340509in}}%
\pgfpathclose%
\pgfusepath{stroke,fill}%
\end{pgfscope}%
\begin{pgfscope}%
\pgfpathrectangle{\pgfqpoint{0.100000in}{0.212622in}}{\pgfqpoint{3.696000in}{3.696000in}}%
\pgfusepath{clip}%
\pgfsetbuttcap%
\pgfsetroundjoin%
\definecolor{currentfill}{rgb}{0.121569,0.466667,0.705882}%
\pgfsetfillcolor{currentfill}%
\pgfsetfillopacity{0.300076}%
\pgfsetlinewidth{1.003750pt}%
\definecolor{currentstroke}{rgb}{0.121569,0.466667,0.705882}%
\pgfsetstrokecolor{currentstroke}%
\pgfsetstrokeopacity{0.300076}%
\pgfsetdash{}{0pt}%
\pgfpathmoveto{\pgfqpoint{1.878396in}{3.339159in}}%
\pgfpathcurveto{\pgfqpoint{1.886632in}{3.339159in}}{\pgfqpoint{1.894532in}{3.342432in}}{\pgfqpoint{1.900356in}{3.348255in}}%
\pgfpathcurveto{\pgfqpoint{1.906180in}{3.354079in}}{\pgfqpoint{1.909453in}{3.361979in}}{\pgfqpoint{1.909453in}{3.370216in}}%
\pgfpathcurveto{\pgfqpoint{1.909453in}{3.378452in}}{\pgfqpoint{1.906180in}{3.386352in}}{\pgfqpoint{1.900356in}{3.392176in}}%
\pgfpathcurveto{\pgfqpoint{1.894532in}{3.398000in}}{\pgfqpoint{1.886632in}{3.401272in}}{\pgfqpoint{1.878396in}{3.401272in}}%
\pgfpathcurveto{\pgfqpoint{1.870160in}{3.401272in}}{\pgfqpoint{1.862260in}{3.398000in}}{\pgfqpoint{1.856436in}{3.392176in}}%
\pgfpathcurveto{\pgfqpoint{1.850612in}{3.386352in}}{\pgfqpoint{1.847340in}{3.378452in}}{\pgfqpoint{1.847340in}{3.370216in}}%
\pgfpathcurveto{\pgfqpoint{1.847340in}{3.361979in}}{\pgfqpoint{1.850612in}{3.354079in}}{\pgfqpoint{1.856436in}{3.348255in}}%
\pgfpathcurveto{\pgfqpoint{1.862260in}{3.342432in}}{\pgfqpoint{1.870160in}{3.339159in}}{\pgfqpoint{1.878396in}{3.339159in}}%
\pgfpathclose%
\pgfusepath{stroke,fill}%
\end{pgfscope}%
\begin{pgfscope}%
\pgfpathrectangle{\pgfqpoint{0.100000in}{0.212622in}}{\pgfqpoint{3.696000in}{3.696000in}}%
\pgfusepath{clip}%
\pgfsetbuttcap%
\pgfsetroundjoin%
\definecolor{currentfill}{rgb}{0.121569,0.466667,0.705882}%
\pgfsetfillcolor{currentfill}%
\pgfsetfillopacity{0.300089}%
\pgfsetlinewidth{1.003750pt}%
\definecolor{currentstroke}{rgb}{0.121569,0.466667,0.705882}%
\pgfsetstrokecolor{currentstroke}%
\pgfsetstrokeopacity{0.300089}%
\pgfsetdash{}{0pt}%
\pgfpathmoveto{\pgfqpoint{1.873484in}{3.343063in}}%
\pgfpathcurveto{\pgfqpoint{1.881720in}{3.343063in}}{\pgfqpoint{1.889620in}{3.346335in}}{\pgfqpoint{1.895444in}{3.352159in}}%
\pgfpathcurveto{\pgfqpoint{1.901268in}{3.357983in}}{\pgfqpoint{1.904540in}{3.365883in}}{\pgfqpoint{1.904540in}{3.374119in}}%
\pgfpathcurveto{\pgfqpoint{1.904540in}{3.382356in}}{\pgfqpoint{1.901268in}{3.390256in}}{\pgfqpoint{1.895444in}{3.396080in}}%
\pgfpathcurveto{\pgfqpoint{1.889620in}{3.401904in}}{\pgfqpoint{1.881720in}{3.405176in}}{\pgfqpoint{1.873484in}{3.405176in}}%
\pgfpathcurveto{\pgfqpoint{1.865247in}{3.405176in}}{\pgfqpoint{1.857347in}{3.401904in}}{\pgfqpoint{1.851523in}{3.396080in}}%
\pgfpathcurveto{\pgfqpoint{1.845700in}{3.390256in}}{\pgfqpoint{1.842427in}{3.382356in}}{\pgfqpoint{1.842427in}{3.374119in}}%
\pgfpathcurveto{\pgfqpoint{1.842427in}{3.365883in}}{\pgfqpoint{1.845700in}{3.357983in}}{\pgfqpoint{1.851523in}{3.352159in}}%
\pgfpathcurveto{\pgfqpoint{1.857347in}{3.346335in}}{\pgfqpoint{1.865247in}{3.343063in}}{\pgfqpoint{1.873484in}{3.343063in}}%
\pgfpathclose%
\pgfusepath{stroke,fill}%
\end{pgfscope}%
\begin{pgfscope}%
\pgfpathrectangle{\pgfqpoint{0.100000in}{0.212622in}}{\pgfqpoint{3.696000in}{3.696000in}}%
\pgfusepath{clip}%
\pgfsetbuttcap%
\pgfsetroundjoin%
\definecolor{currentfill}{rgb}{0.121569,0.466667,0.705882}%
\pgfsetfillcolor{currentfill}%
\pgfsetfillopacity{0.300116}%
\pgfsetlinewidth{1.003750pt}%
\definecolor{currentstroke}{rgb}{0.121569,0.466667,0.705882}%
\pgfsetstrokecolor{currentstroke}%
\pgfsetstrokeopacity{0.300116}%
\pgfsetdash{}{0pt}%
\pgfpathmoveto{\pgfqpoint{1.873213in}{3.343217in}}%
\pgfpathcurveto{\pgfqpoint{1.881449in}{3.343217in}}{\pgfqpoint{1.889349in}{3.346489in}}{\pgfqpoint{1.895173in}{3.352313in}}%
\pgfpathcurveto{\pgfqpoint{1.900997in}{3.358137in}}{\pgfqpoint{1.904269in}{3.366037in}}{\pgfqpoint{1.904269in}{3.374273in}}%
\pgfpathcurveto{\pgfqpoint{1.904269in}{3.382510in}}{\pgfqpoint{1.900997in}{3.390410in}}{\pgfqpoint{1.895173in}{3.396234in}}%
\pgfpathcurveto{\pgfqpoint{1.889349in}{3.402058in}}{\pgfqpoint{1.881449in}{3.405330in}}{\pgfqpoint{1.873213in}{3.405330in}}%
\pgfpathcurveto{\pgfqpoint{1.864976in}{3.405330in}}{\pgfqpoint{1.857076in}{3.402058in}}{\pgfqpoint{1.851252in}{3.396234in}}%
\pgfpathcurveto{\pgfqpoint{1.845428in}{3.390410in}}{\pgfqpoint{1.842156in}{3.382510in}}{\pgfqpoint{1.842156in}{3.374273in}}%
\pgfpathcurveto{\pgfqpoint{1.842156in}{3.366037in}}{\pgfqpoint{1.845428in}{3.358137in}}{\pgfqpoint{1.851252in}{3.352313in}}%
\pgfpathcurveto{\pgfqpoint{1.857076in}{3.346489in}}{\pgfqpoint{1.864976in}{3.343217in}}{\pgfqpoint{1.873213in}{3.343217in}}%
\pgfpathclose%
\pgfusepath{stroke,fill}%
\end{pgfscope}%
\begin{pgfscope}%
\pgfpathrectangle{\pgfqpoint{0.100000in}{0.212622in}}{\pgfqpoint{3.696000in}{3.696000in}}%
\pgfusepath{clip}%
\pgfsetbuttcap%
\pgfsetroundjoin%
\definecolor{currentfill}{rgb}{0.121569,0.466667,0.705882}%
\pgfsetfillcolor{currentfill}%
\pgfsetfillopacity{0.300172}%
\pgfsetlinewidth{1.003750pt}%
\definecolor{currentstroke}{rgb}{0.121569,0.466667,0.705882}%
\pgfsetstrokecolor{currentstroke}%
\pgfsetstrokeopacity{0.300172}%
\pgfsetdash{}{0pt}%
\pgfpathmoveto{\pgfqpoint{1.872708in}{3.343418in}}%
\pgfpathcurveto{\pgfqpoint{1.880944in}{3.343418in}}{\pgfqpoint{1.888845in}{3.346690in}}{\pgfqpoint{1.894668in}{3.352514in}}%
\pgfpathcurveto{\pgfqpoint{1.900492in}{3.358338in}}{\pgfqpoint{1.903765in}{3.366238in}}{\pgfqpoint{1.903765in}{3.374474in}}%
\pgfpathcurveto{\pgfqpoint{1.903765in}{3.382710in}}{\pgfqpoint{1.900492in}{3.390611in}}{\pgfqpoint{1.894668in}{3.396434in}}%
\pgfpathcurveto{\pgfqpoint{1.888845in}{3.402258in}}{\pgfqpoint{1.880944in}{3.405531in}}{\pgfqpoint{1.872708in}{3.405531in}}%
\pgfpathcurveto{\pgfqpoint{1.864472in}{3.405531in}}{\pgfqpoint{1.856572in}{3.402258in}}{\pgfqpoint{1.850748in}{3.396434in}}%
\pgfpathcurveto{\pgfqpoint{1.844924in}{3.390611in}}{\pgfqpoint{1.841652in}{3.382710in}}{\pgfqpoint{1.841652in}{3.374474in}}%
\pgfpathcurveto{\pgfqpoint{1.841652in}{3.366238in}}{\pgfqpoint{1.844924in}{3.358338in}}{\pgfqpoint{1.850748in}{3.352514in}}%
\pgfpathcurveto{\pgfqpoint{1.856572in}{3.346690in}}{\pgfqpoint{1.864472in}{3.343418in}}{\pgfqpoint{1.872708in}{3.343418in}}%
\pgfpathclose%
\pgfusepath{stroke,fill}%
\end{pgfscope}%
\begin{pgfscope}%
\pgfpathrectangle{\pgfqpoint{0.100000in}{0.212622in}}{\pgfqpoint{3.696000in}{3.696000in}}%
\pgfusepath{clip}%
\pgfsetbuttcap%
\pgfsetroundjoin%
\definecolor{currentfill}{rgb}{0.121569,0.466667,0.705882}%
\pgfsetfillcolor{currentfill}%
\pgfsetfillopacity{0.300293}%
\pgfsetlinewidth{1.003750pt}%
\definecolor{currentstroke}{rgb}{0.121569,0.466667,0.705882}%
\pgfsetstrokecolor{currentstroke}%
\pgfsetstrokeopacity{0.300293}%
\pgfsetdash{}{0pt}%
\pgfpathmoveto{\pgfqpoint{1.880161in}{3.337594in}}%
\pgfpathcurveto{\pgfqpoint{1.888398in}{3.337594in}}{\pgfqpoint{1.896298in}{3.340866in}}{\pgfqpoint{1.902122in}{3.346690in}}%
\pgfpathcurveto{\pgfqpoint{1.907945in}{3.352514in}}{\pgfqpoint{1.911218in}{3.360414in}}{\pgfqpoint{1.911218in}{3.368650in}}%
\pgfpathcurveto{\pgfqpoint{1.911218in}{3.376886in}}{\pgfqpoint{1.907945in}{3.384786in}}{\pgfqpoint{1.902122in}{3.390610in}}%
\pgfpathcurveto{\pgfqpoint{1.896298in}{3.396434in}}{\pgfqpoint{1.888398in}{3.399707in}}{\pgfqpoint{1.880161in}{3.399707in}}%
\pgfpathcurveto{\pgfqpoint{1.871925in}{3.399707in}}{\pgfqpoint{1.864025in}{3.396434in}}{\pgfqpoint{1.858201in}{3.390610in}}%
\pgfpathcurveto{\pgfqpoint{1.852377in}{3.384786in}}{\pgfqpoint{1.849105in}{3.376886in}}{\pgfqpoint{1.849105in}{3.368650in}}%
\pgfpathcurveto{\pgfqpoint{1.849105in}{3.360414in}}{\pgfqpoint{1.852377in}{3.352514in}}{\pgfqpoint{1.858201in}{3.346690in}}%
\pgfpathcurveto{\pgfqpoint{1.864025in}{3.340866in}}{\pgfqpoint{1.871925in}{3.337594in}}{\pgfqpoint{1.880161in}{3.337594in}}%
\pgfpathclose%
\pgfusepath{stroke,fill}%
\end{pgfscope}%
\begin{pgfscope}%
\pgfpathrectangle{\pgfqpoint{0.100000in}{0.212622in}}{\pgfqpoint{3.696000in}{3.696000in}}%
\pgfusepath{clip}%
\pgfsetbuttcap%
\pgfsetroundjoin%
\definecolor{currentfill}{rgb}{0.121569,0.466667,0.705882}%
\pgfsetfillcolor{currentfill}%
\pgfsetfillopacity{0.300296}%
\pgfsetlinewidth{1.003750pt}%
\definecolor{currentstroke}{rgb}{0.121569,0.466667,0.705882}%
\pgfsetstrokecolor{currentstroke}%
\pgfsetstrokeopacity{0.300296}%
\pgfsetdash{}{0pt}%
\pgfpathmoveto{\pgfqpoint{1.871778in}{3.343683in}}%
\pgfpathcurveto{\pgfqpoint{1.880014in}{3.343683in}}{\pgfqpoint{1.887914in}{3.346956in}}{\pgfqpoint{1.893738in}{3.352780in}}%
\pgfpathcurveto{\pgfqpoint{1.899562in}{3.358604in}}{\pgfqpoint{1.902834in}{3.366504in}}{\pgfqpoint{1.902834in}{3.374740in}}%
\pgfpathcurveto{\pgfqpoint{1.902834in}{3.382976in}}{\pgfqpoint{1.899562in}{3.390876in}}{\pgfqpoint{1.893738in}{3.396700in}}%
\pgfpathcurveto{\pgfqpoint{1.887914in}{3.402524in}}{\pgfqpoint{1.880014in}{3.405796in}}{\pgfqpoint{1.871778in}{3.405796in}}%
\pgfpathcurveto{\pgfqpoint{1.863542in}{3.405796in}}{\pgfqpoint{1.855641in}{3.402524in}}{\pgfqpoint{1.849818in}{3.396700in}}%
\pgfpathcurveto{\pgfqpoint{1.843994in}{3.390876in}}{\pgfqpoint{1.840721in}{3.382976in}}{\pgfqpoint{1.840721in}{3.374740in}}%
\pgfpathcurveto{\pgfqpoint{1.840721in}{3.366504in}}{\pgfqpoint{1.843994in}{3.358604in}}{\pgfqpoint{1.849818in}{3.352780in}}%
\pgfpathcurveto{\pgfqpoint{1.855641in}{3.346956in}}{\pgfqpoint{1.863542in}{3.343683in}}{\pgfqpoint{1.871778in}{3.343683in}}%
\pgfpathclose%
\pgfusepath{stroke,fill}%
\end{pgfscope}%
\begin{pgfscope}%
\pgfpathrectangle{\pgfqpoint{0.100000in}{0.212622in}}{\pgfqpoint{3.696000in}{3.696000in}}%
\pgfusepath{clip}%
\pgfsetbuttcap%
\pgfsetroundjoin%
\definecolor{currentfill}{rgb}{0.121569,0.466667,0.705882}%
\pgfsetfillcolor{currentfill}%
\pgfsetfillopacity{0.300526}%
\pgfsetlinewidth{1.003750pt}%
\definecolor{currentstroke}{rgb}{0.121569,0.466667,0.705882}%
\pgfsetstrokecolor{currentstroke}%
\pgfsetstrokeopacity{0.300526}%
\pgfsetdash{}{0pt}%
\pgfpathmoveto{\pgfqpoint{1.870077in}{3.343918in}}%
\pgfpathcurveto{\pgfqpoint{1.878313in}{3.343918in}}{\pgfqpoint{1.886213in}{3.347190in}}{\pgfqpoint{1.892037in}{3.353014in}}%
\pgfpathcurveto{\pgfqpoint{1.897861in}{3.358838in}}{\pgfqpoint{1.901133in}{3.366738in}}{\pgfqpoint{1.901133in}{3.374974in}}%
\pgfpathcurveto{\pgfqpoint{1.901133in}{3.383210in}}{\pgfqpoint{1.897861in}{3.391110in}}{\pgfqpoint{1.892037in}{3.396934in}}%
\pgfpathcurveto{\pgfqpoint{1.886213in}{3.402758in}}{\pgfqpoint{1.878313in}{3.406031in}}{\pgfqpoint{1.870077in}{3.406031in}}%
\pgfpathcurveto{\pgfqpoint{1.861840in}{3.406031in}}{\pgfqpoint{1.853940in}{3.402758in}}{\pgfqpoint{1.848116in}{3.396934in}}%
\pgfpathcurveto{\pgfqpoint{1.842293in}{3.391110in}}{\pgfqpoint{1.839020in}{3.383210in}}{\pgfqpoint{1.839020in}{3.374974in}}%
\pgfpathcurveto{\pgfqpoint{1.839020in}{3.366738in}}{\pgfqpoint{1.842293in}{3.358838in}}{\pgfqpoint{1.848116in}{3.353014in}}%
\pgfpathcurveto{\pgfqpoint{1.853940in}{3.347190in}}{\pgfqpoint{1.861840in}{3.343918in}}{\pgfqpoint{1.870077in}{3.343918in}}%
\pgfpathclose%
\pgfusepath{stroke,fill}%
\end{pgfscope}%
\begin{pgfscope}%
\pgfpathrectangle{\pgfqpoint{0.100000in}{0.212622in}}{\pgfqpoint{3.696000in}{3.696000in}}%
\pgfusepath{clip}%
\pgfsetbuttcap%
\pgfsetroundjoin%
\definecolor{currentfill}{rgb}{0.121569,0.466667,0.705882}%
\pgfsetfillcolor{currentfill}%
\pgfsetfillopacity{0.300614}%
\pgfsetlinewidth{1.003750pt}%
\definecolor{currentstroke}{rgb}{0.121569,0.466667,0.705882}%
\pgfsetstrokecolor{currentstroke}%
\pgfsetstrokeopacity{0.300614}%
\pgfsetdash{}{0pt}%
\pgfpathmoveto{\pgfqpoint{1.883332in}{3.334262in}}%
\pgfpathcurveto{\pgfqpoint{1.891568in}{3.334262in}}{\pgfqpoint{1.899468in}{3.337534in}}{\pgfqpoint{1.905292in}{3.343358in}}%
\pgfpathcurveto{\pgfqpoint{1.911116in}{3.349182in}}{\pgfqpoint{1.914389in}{3.357082in}}{\pgfqpoint{1.914389in}{3.365318in}}%
\pgfpathcurveto{\pgfqpoint{1.914389in}{3.373555in}}{\pgfqpoint{1.911116in}{3.381455in}}{\pgfqpoint{1.905292in}{3.387279in}}%
\pgfpathcurveto{\pgfqpoint{1.899468in}{3.393103in}}{\pgfqpoint{1.891568in}{3.396375in}}{\pgfqpoint{1.883332in}{3.396375in}}%
\pgfpathcurveto{\pgfqpoint{1.875096in}{3.396375in}}{\pgfqpoint{1.867196in}{3.393103in}}{\pgfqpoint{1.861372in}{3.387279in}}%
\pgfpathcurveto{\pgfqpoint{1.855548in}{3.381455in}}{\pgfqpoint{1.852276in}{3.373555in}}{\pgfqpoint{1.852276in}{3.365318in}}%
\pgfpathcurveto{\pgfqpoint{1.852276in}{3.357082in}}{\pgfqpoint{1.855548in}{3.349182in}}{\pgfqpoint{1.861372in}{3.343358in}}%
\pgfpathcurveto{\pgfqpoint{1.867196in}{3.337534in}}{\pgfqpoint{1.875096in}{3.334262in}}{\pgfqpoint{1.883332in}{3.334262in}}%
\pgfpathclose%
\pgfusepath{stroke,fill}%
\end{pgfscope}%
\begin{pgfscope}%
\pgfpathrectangle{\pgfqpoint{0.100000in}{0.212622in}}{\pgfqpoint{3.696000in}{3.696000in}}%
\pgfusepath{clip}%
\pgfsetbuttcap%
\pgfsetroundjoin%
\definecolor{currentfill}{rgb}{0.121569,0.466667,0.705882}%
\pgfsetfillcolor{currentfill}%
\pgfsetfillopacity{0.300720}%
\pgfsetlinewidth{1.003750pt}%
\definecolor{currentstroke}{rgb}{0.121569,0.466667,0.705882}%
\pgfsetstrokecolor{currentstroke}%
\pgfsetstrokeopacity{0.300720}%
\pgfsetdash{}{0pt}%
\pgfpathmoveto{\pgfqpoint{1.868943in}{3.344025in}}%
\pgfpathcurveto{\pgfqpoint{1.877179in}{3.344025in}}{\pgfqpoint{1.885079in}{3.347297in}}{\pgfqpoint{1.890903in}{3.353121in}}%
\pgfpathcurveto{\pgfqpoint{1.896727in}{3.358945in}}{\pgfqpoint{1.900000in}{3.366845in}}{\pgfqpoint{1.900000in}{3.375081in}}%
\pgfpathcurveto{\pgfqpoint{1.900000in}{3.383318in}}{\pgfqpoint{1.896727in}{3.391218in}}{\pgfqpoint{1.890903in}{3.397042in}}%
\pgfpathcurveto{\pgfqpoint{1.885079in}{3.402866in}}{\pgfqpoint{1.877179in}{3.406138in}}{\pgfqpoint{1.868943in}{3.406138in}}%
\pgfpathcurveto{\pgfqpoint{1.860707in}{3.406138in}}{\pgfqpoint{1.852807in}{3.402866in}}{\pgfqpoint{1.846983in}{3.397042in}}%
\pgfpathcurveto{\pgfqpoint{1.841159in}{3.391218in}}{\pgfqpoint{1.837887in}{3.383318in}}{\pgfqpoint{1.837887in}{3.375081in}}%
\pgfpathcurveto{\pgfqpoint{1.837887in}{3.366845in}}{\pgfqpoint{1.841159in}{3.358945in}}{\pgfqpoint{1.846983in}{3.353121in}}%
\pgfpathcurveto{\pgfqpoint{1.852807in}{3.347297in}}{\pgfqpoint{1.860707in}{3.344025in}}{\pgfqpoint{1.868943in}{3.344025in}}%
\pgfpathclose%
\pgfusepath{stroke,fill}%
\end{pgfscope}%
\begin{pgfscope}%
\pgfpathrectangle{\pgfqpoint{0.100000in}{0.212622in}}{\pgfqpoint{3.696000in}{3.696000in}}%
\pgfusepath{clip}%
\pgfsetbuttcap%
\pgfsetroundjoin%
\definecolor{currentfill}{rgb}{0.121569,0.466667,0.705882}%
\pgfsetfillcolor{currentfill}%
\pgfsetfillopacity{0.300816}%
\pgfsetlinewidth{1.003750pt}%
\definecolor{currentstroke}{rgb}{0.121569,0.466667,0.705882}%
\pgfsetstrokecolor{currentstroke}%
\pgfsetstrokeopacity{0.300816}%
\pgfsetdash{}{0pt}%
\pgfpathmoveto{\pgfqpoint{1.868378in}{3.343992in}}%
\pgfpathcurveto{\pgfqpoint{1.876614in}{3.343992in}}{\pgfqpoint{1.884514in}{3.347264in}}{\pgfqpoint{1.890338in}{3.353088in}}%
\pgfpathcurveto{\pgfqpoint{1.896162in}{3.358912in}}{\pgfqpoint{1.899435in}{3.366812in}}{\pgfqpoint{1.899435in}{3.375048in}}%
\pgfpathcurveto{\pgfqpoint{1.899435in}{3.383285in}}{\pgfqpoint{1.896162in}{3.391185in}}{\pgfqpoint{1.890338in}{3.397009in}}%
\pgfpathcurveto{\pgfqpoint{1.884514in}{3.402832in}}{\pgfqpoint{1.876614in}{3.406105in}}{\pgfqpoint{1.868378in}{3.406105in}}%
\pgfpathcurveto{\pgfqpoint{1.860142in}{3.406105in}}{\pgfqpoint{1.852242in}{3.402832in}}{\pgfqpoint{1.846418in}{3.397009in}}%
\pgfpathcurveto{\pgfqpoint{1.840594in}{3.391185in}}{\pgfqpoint{1.837322in}{3.383285in}}{\pgfqpoint{1.837322in}{3.375048in}}%
\pgfpathcurveto{\pgfqpoint{1.837322in}{3.366812in}}{\pgfqpoint{1.840594in}{3.358912in}}{\pgfqpoint{1.846418in}{3.353088in}}%
\pgfpathcurveto{\pgfqpoint{1.852242in}{3.347264in}}{\pgfqpoint{1.860142in}{3.343992in}}{\pgfqpoint{1.868378in}{3.343992in}}%
\pgfpathclose%
\pgfusepath{stroke,fill}%
\end{pgfscope}%
\begin{pgfscope}%
\pgfpathrectangle{\pgfqpoint{0.100000in}{0.212622in}}{\pgfqpoint{3.696000in}{3.696000in}}%
\pgfusepath{clip}%
\pgfsetbuttcap%
\pgfsetroundjoin%
\definecolor{currentfill}{rgb}{0.121569,0.466667,0.705882}%
\pgfsetfillcolor{currentfill}%
\pgfsetfillopacity{0.300994}%
\pgfsetlinewidth{1.003750pt}%
\definecolor{currentstroke}{rgb}{0.121569,0.466667,0.705882}%
\pgfsetstrokecolor{currentstroke}%
\pgfsetstrokeopacity{0.300994}%
\pgfsetdash{}{0pt}%
\pgfpathmoveto{\pgfqpoint{1.867364in}{3.343756in}}%
\pgfpathcurveto{\pgfqpoint{1.875601in}{3.343756in}}{\pgfqpoint{1.883501in}{3.347028in}}{\pgfqpoint{1.889325in}{3.352852in}}%
\pgfpathcurveto{\pgfqpoint{1.895148in}{3.358676in}}{\pgfqpoint{1.898421in}{3.366576in}}{\pgfqpoint{1.898421in}{3.374812in}}%
\pgfpathcurveto{\pgfqpoint{1.898421in}{3.383048in}}{\pgfqpoint{1.895148in}{3.390948in}}{\pgfqpoint{1.889325in}{3.396772in}}%
\pgfpathcurveto{\pgfqpoint{1.883501in}{3.402596in}}{\pgfqpoint{1.875601in}{3.405869in}}{\pgfqpoint{1.867364in}{3.405869in}}%
\pgfpathcurveto{\pgfqpoint{1.859128in}{3.405869in}}{\pgfqpoint{1.851228in}{3.402596in}}{\pgfqpoint{1.845404in}{3.396772in}}%
\pgfpathcurveto{\pgfqpoint{1.839580in}{3.390948in}}{\pgfqpoint{1.836308in}{3.383048in}}{\pgfqpoint{1.836308in}{3.374812in}}%
\pgfpathcurveto{\pgfqpoint{1.836308in}{3.366576in}}{\pgfqpoint{1.839580in}{3.358676in}}{\pgfqpoint{1.845404in}{3.352852in}}%
\pgfpathcurveto{\pgfqpoint{1.851228in}{3.347028in}}{\pgfqpoint{1.859128in}{3.343756in}}{\pgfqpoint{1.867364in}{3.343756in}}%
\pgfpathclose%
\pgfusepath{stroke,fill}%
\end{pgfscope}%
\begin{pgfscope}%
\pgfpathrectangle{\pgfqpoint{0.100000in}{0.212622in}}{\pgfqpoint{3.696000in}{3.696000in}}%
\pgfusepath{clip}%
\pgfsetbuttcap%
\pgfsetroundjoin%
\definecolor{currentfill}{rgb}{0.121569,0.466667,0.705882}%
\pgfsetfillcolor{currentfill}%
\pgfsetfillopacity{0.301042}%
\pgfsetlinewidth{1.003750pt}%
\definecolor{currentstroke}{rgb}{0.121569,0.466667,0.705882}%
\pgfsetstrokecolor{currentstroke}%
\pgfsetstrokeopacity{0.301042}%
\pgfsetdash{}{0pt}%
\pgfpathmoveto{\pgfqpoint{1.867044in}{3.343604in}}%
\pgfpathcurveto{\pgfqpoint{1.875280in}{3.343604in}}{\pgfqpoint{1.883180in}{3.346876in}}{\pgfqpoint{1.889004in}{3.352700in}}%
\pgfpathcurveto{\pgfqpoint{1.894828in}{3.358524in}}{\pgfqpoint{1.898101in}{3.366424in}}{\pgfqpoint{1.898101in}{3.374661in}}%
\pgfpathcurveto{\pgfqpoint{1.898101in}{3.382897in}}{\pgfqpoint{1.894828in}{3.390797in}}{\pgfqpoint{1.889004in}{3.396621in}}%
\pgfpathcurveto{\pgfqpoint{1.883180in}{3.402445in}}{\pgfqpoint{1.875280in}{3.405717in}}{\pgfqpoint{1.867044in}{3.405717in}}%
\pgfpathcurveto{\pgfqpoint{1.858808in}{3.405717in}}{\pgfqpoint{1.850908in}{3.402445in}}{\pgfqpoint{1.845084in}{3.396621in}}%
\pgfpathcurveto{\pgfqpoint{1.839260in}{3.390797in}}{\pgfqpoint{1.835988in}{3.382897in}}{\pgfqpoint{1.835988in}{3.374661in}}%
\pgfpathcurveto{\pgfqpoint{1.835988in}{3.366424in}}{\pgfqpoint{1.839260in}{3.358524in}}{\pgfqpoint{1.845084in}{3.352700in}}%
\pgfpathcurveto{\pgfqpoint{1.850908in}{3.346876in}}{\pgfqpoint{1.858808in}{3.343604in}}{\pgfqpoint{1.867044in}{3.343604in}}%
\pgfpathclose%
\pgfusepath{stroke,fill}%
\end{pgfscope}%
\begin{pgfscope}%
\pgfpathrectangle{\pgfqpoint{0.100000in}{0.212622in}}{\pgfqpoint{3.696000in}{3.696000in}}%
\pgfusepath{clip}%
\pgfsetbuttcap%
\pgfsetroundjoin%
\definecolor{currentfill}{rgb}{0.121569,0.466667,0.705882}%
\pgfsetfillcolor{currentfill}%
\pgfsetfillopacity{0.301125}%
\pgfsetlinewidth{1.003750pt}%
\definecolor{currentstroke}{rgb}{0.121569,0.466667,0.705882}%
\pgfsetstrokecolor{currentstroke}%
\pgfsetstrokeopacity{0.301125}%
\pgfsetdash{}{0pt}%
\pgfpathmoveto{\pgfqpoint{1.886761in}{3.330115in}}%
\pgfpathcurveto{\pgfqpoint{1.894998in}{3.330115in}}{\pgfqpoint{1.902898in}{3.333387in}}{\pgfqpoint{1.908722in}{3.339211in}}%
\pgfpathcurveto{\pgfqpoint{1.914546in}{3.345035in}}{\pgfqpoint{1.917818in}{3.352935in}}{\pgfqpoint{1.917818in}{3.361171in}}%
\pgfpathcurveto{\pgfqpoint{1.917818in}{3.369408in}}{\pgfqpoint{1.914546in}{3.377308in}}{\pgfqpoint{1.908722in}{3.383132in}}%
\pgfpathcurveto{\pgfqpoint{1.902898in}{3.388956in}}{\pgfqpoint{1.894998in}{3.392228in}}{\pgfqpoint{1.886761in}{3.392228in}}%
\pgfpathcurveto{\pgfqpoint{1.878525in}{3.392228in}}{\pgfqpoint{1.870625in}{3.388956in}}{\pgfqpoint{1.864801in}{3.383132in}}%
\pgfpathcurveto{\pgfqpoint{1.858977in}{3.377308in}}{\pgfqpoint{1.855705in}{3.369408in}}{\pgfqpoint{1.855705in}{3.361171in}}%
\pgfpathcurveto{\pgfqpoint{1.855705in}{3.352935in}}{\pgfqpoint{1.858977in}{3.345035in}}{\pgfqpoint{1.864801in}{3.339211in}}%
\pgfpathcurveto{\pgfqpoint{1.870625in}{3.333387in}}{\pgfqpoint{1.878525in}{3.330115in}}{\pgfqpoint{1.886761in}{3.330115in}}%
\pgfpathclose%
\pgfusepath{stroke,fill}%
\end{pgfscope}%
\begin{pgfscope}%
\pgfpathrectangle{\pgfqpoint{0.100000in}{0.212622in}}{\pgfqpoint{3.696000in}{3.696000in}}%
\pgfusepath{clip}%
\pgfsetbuttcap%
\pgfsetroundjoin%
\definecolor{currentfill}{rgb}{0.121569,0.466667,0.705882}%
\pgfsetfillcolor{currentfill}%
\pgfsetfillopacity{0.301165}%
\pgfsetlinewidth{1.003750pt}%
\definecolor{currentstroke}{rgb}{0.121569,0.466667,0.705882}%
\pgfsetstrokecolor{currentstroke}%
\pgfsetstrokeopacity{0.301165}%
\pgfsetdash{}{0pt}%
\pgfpathmoveto{\pgfqpoint{1.866474in}{3.343393in}}%
\pgfpathcurveto{\pgfqpoint{1.874710in}{3.343393in}}{\pgfqpoint{1.882610in}{3.346665in}}{\pgfqpoint{1.888434in}{3.352489in}}%
\pgfpathcurveto{\pgfqpoint{1.894258in}{3.358313in}}{\pgfqpoint{1.897530in}{3.366213in}}{\pgfqpoint{1.897530in}{3.374449in}}%
\pgfpathcurveto{\pgfqpoint{1.897530in}{3.382686in}}{\pgfqpoint{1.894258in}{3.390586in}}{\pgfqpoint{1.888434in}{3.396410in}}%
\pgfpathcurveto{\pgfqpoint{1.882610in}{3.402233in}}{\pgfqpoint{1.874710in}{3.405506in}}{\pgfqpoint{1.866474in}{3.405506in}}%
\pgfpathcurveto{\pgfqpoint{1.858238in}{3.405506in}}{\pgfqpoint{1.850337in}{3.402233in}}{\pgfqpoint{1.844514in}{3.396410in}}%
\pgfpathcurveto{\pgfqpoint{1.838690in}{3.390586in}}{\pgfqpoint{1.835417in}{3.382686in}}{\pgfqpoint{1.835417in}{3.374449in}}%
\pgfpathcurveto{\pgfqpoint{1.835417in}{3.366213in}}{\pgfqpoint{1.838690in}{3.358313in}}{\pgfqpoint{1.844514in}{3.352489in}}%
\pgfpathcurveto{\pgfqpoint{1.850337in}{3.346665in}}{\pgfqpoint{1.858238in}{3.343393in}}{\pgfqpoint{1.866474in}{3.343393in}}%
\pgfpathclose%
\pgfusepath{stroke,fill}%
\end{pgfscope}%
\begin{pgfscope}%
\pgfpathrectangle{\pgfqpoint{0.100000in}{0.212622in}}{\pgfqpoint{3.696000in}{3.696000in}}%
\pgfusepath{clip}%
\pgfsetbuttcap%
\pgfsetroundjoin%
\definecolor{currentfill}{rgb}{0.121569,0.466667,0.705882}%
\pgfsetfillcolor{currentfill}%
\pgfsetfillopacity{0.301383}%
\pgfsetlinewidth{1.003750pt}%
\definecolor{currentstroke}{rgb}{0.121569,0.466667,0.705882}%
\pgfsetstrokecolor{currentstroke}%
\pgfsetstrokeopacity{0.301383}%
\pgfsetdash{}{0pt}%
\pgfpathmoveto{\pgfqpoint{1.865467in}{3.342823in}}%
\pgfpathcurveto{\pgfqpoint{1.873703in}{3.342823in}}{\pgfqpoint{1.881603in}{3.346096in}}{\pgfqpoint{1.887427in}{3.351920in}}%
\pgfpathcurveto{\pgfqpoint{1.893251in}{3.357744in}}{\pgfqpoint{1.896523in}{3.365644in}}{\pgfqpoint{1.896523in}{3.373880in}}%
\pgfpathcurveto{\pgfqpoint{1.896523in}{3.382116in}}{\pgfqpoint{1.893251in}{3.390016in}}{\pgfqpoint{1.887427in}{3.395840in}}%
\pgfpathcurveto{\pgfqpoint{1.881603in}{3.401664in}}{\pgfqpoint{1.873703in}{3.404936in}}{\pgfqpoint{1.865467in}{3.404936in}}%
\pgfpathcurveto{\pgfqpoint{1.857231in}{3.404936in}}{\pgfqpoint{1.849330in}{3.401664in}}{\pgfqpoint{1.843507in}{3.395840in}}%
\pgfpathcurveto{\pgfqpoint{1.837683in}{3.390016in}}{\pgfqpoint{1.834410in}{3.382116in}}{\pgfqpoint{1.834410in}{3.373880in}}%
\pgfpathcurveto{\pgfqpoint{1.834410in}{3.365644in}}{\pgfqpoint{1.837683in}{3.357744in}}{\pgfqpoint{1.843507in}{3.351920in}}%
\pgfpathcurveto{\pgfqpoint{1.849330in}{3.346096in}}{\pgfqpoint{1.857231in}{3.342823in}}{\pgfqpoint{1.865467in}{3.342823in}}%
\pgfpathclose%
\pgfusepath{stroke,fill}%
\end{pgfscope}%
\begin{pgfscope}%
\pgfpathrectangle{\pgfqpoint{0.100000in}{0.212622in}}{\pgfqpoint{3.696000in}{3.696000in}}%
\pgfusepath{clip}%
\pgfsetbuttcap%
\pgfsetroundjoin%
\definecolor{currentfill}{rgb}{0.121569,0.466667,0.705882}%
\pgfsetfillcolor{currentfill}%
\pgfsetfillopacity{0.301535}%
\pgfsetlinewidth{1.003750pt}%
\definecolor{currentstroke}{rgb}{0.121569,0.466667,0.705882}%
\pgfsetstrokecolor{currentstroke}%
\pgfsetstrokeopacity{0.301535}%
\pgfsetdash{}{0pt}%
\pgfpathmoveto{\pgfqpoint{1.888509in}{3.328017in}}%
\pgfpathcurveto{\pgfqpoint{1.896745in}{3.328017in}}{\pgfqpoint{1.904645in}{3.331289in}}{\pgfqpoint{1.910469in}{3.337113in}}%
\pgfpathcurveto{\pgfqpoint{1.916293in}{3.342937in}}{\pgfqpoint{1.919566in}{3.350837in}}{\pgfqpoint{1.919566in}{3.359073in}}%
\pgfpathcurveto{\pgfqpoint{1.919566in}{3.367310in}}{\pgfqpoint{1.916293in}{3.375210in}}{\pgfqpoint{1.910469in}{3.381034in}}%
\pgfpathcurveto{\pgfqpoint{1.904645in}{3.386857in}}{\pgfqpoint{1.896745in}{3.390130in}}{\pgfqpoint{1.888509in}{3.390130in}}%
\pgfpathcurveto{\pgfqpoint{1.880273in}{3.390130in}}{\pgfqpoint{1.872373in}{3.386857in}}{\pgfqpoint{1.866549in}{3.381034in}}%
\pgfpathcurveto{\pgfqpoint{1.860725in}{3.375210in}}{\pgfqpoint{1.857453in}{3.367310in}}{\pgfqpoint{1.857453in}{3.359073in}}%
\pgfpathcurveto{\pgfqpoint{1.857453in}{3.350837in}}{\pgfqpoint{1.860725in}{3.342937in}}{\pgfqpoint{1.866549in}{3.337113in}}%
\pgfpathcurveto{\pgfqpoint{1.872373in}{3.331289in}}{\pgfqpoint{1.880273in}{3.328017in}}{\pgfqpoint{1.888509in}{3.328017in}}%
\pgfpathclose%
\pgfusepath{stroke,fill}%
\end{pgfscope}%
\begin{pgfscope}%
\pgfpathrectangle{\pgfqpoint{0.100000in}{0.212622in}}{\pgfqpoint{3.696000in}{3.696000in}}%
\pgfusepath{clip}%
\pgfsetbuttcap%
\pgfsetroundjoin%
\definecolor{currentfill}{rgb}{0.121569,0.466667,0.705882}%
\pgfsetfillcolor{currentfill}%
\pgfsetfillopacity{0.301546}%
\pgfsetlinewidth{1.003750pt}%
\definecolor{currentstroke}{rgb}{0.121569,0.466667,0.705882}%
\pgfsetstrokecolor{currentstroke}%
\pgfsetstrokeopacity{0.301546}%
\pgfsetdash{}{0pt}%
\pgfpathmoveto{\pgfqpoint{1.864736in}{3.342249in}}%
\pgfpathcurveto{\pgfqpoint{1.872972in}{3.342249in}}{\pgfqpoint{1.880872in}{3.345521in}}{\pgfqpoint{1.886696in}{3.351345in}}%
\pgfpathcurveto{\pgfqpoint{1.892520in}{3.357169in}}{\pgfqpoint{1.895792in}{3.365069in}}{\pgfqpoint{1.895792in}{3.373305in}}%
\pgfpathcurveto{\pgfqpoint{1.895792in}{3.381541in}}{\pgfqpoint{1.892520in}{3.389442in}}{\pgfqpoint{1.886696in}{3.395265in}}%
\pgfpathcurveto{\pgfqpoint{1.880872in}{3.401089in}}{\pgfqpoint{1.872972in}{3.404362in}}{\pgfqpoint{1.864736in}{3.404362in}}%
\pgfpathcurveto{\pgfqpoint{1.856499in}{3.404362in}}{\pgfqpoint{1.848599in}{3.401089in}}{\pgfqpoint{1.842775in}{3.395265in}}%
\pgfpathcurveto{\pgfqpoint{1.836951in}{3.389442in}}{\pgfqpoint{1.833679in}{3.381541in}}{\pgfqpoint{1.833679in}{3.373305in}}%
\pgfpathcurveto{\pgfqpoint{1.833679in}{3.365069in}}{\pgfqpoint{1.836951in}{3.357169in}}{\pgfqpoint{1.842775in}{3.351345in}}%
\pgfpathcurveto{\pgfqpoint{1.848599in}{3.345521in}}{\pgfqpoint{1.856499in}{3.342249in}}{\pgfqpoint{1.864736in}{3.342249in}}%
\pgfpathclose%
\pgfusepath{stroke,fill}%
\end{pgfscope}%
\begin{pgfscope}%
\pgfpathrectangle{\pgfqpoint{0.100000in}{0.212622in}}{\pgfqpoint{3.696000in}{3.696000in}}%
\pgfusepath{clip}%
\pgfsetbuttcap%
\pgfsetroundjoin%
\definecolor{currentfill}{rgb}{0.121569,0.466667,0.705882}%
\pgfsetfillcolor{currentfill}%
\pgfsetfillopacity{0.301587}%
\pgfsetlinewidth{1.003750pt}%
\definecolor{currentstroke}{rgb}{0.121569,0.466667,0.705882}%
\pgfsetstrokecolor{currentstroke}%
\pgfsetstrokeopacity{0.301587}%
\pgfsetdash{}{0pt}%
\pgfpathmoveto{\pgfqpoint{1.864551in}{3.342075in}}%
\pgfpathcurveto{\pgfqpoint{1.872788in}{3.342075in}}{\pgfqpoint{1.880688in}{3.345347in}}{\pgfqpoint{1.886511in}{3.351171in}}%
\pgfpathcurveto{\pgfqpoint{1.892335in}{3.356995in}}{\pgfqpoint{1.895608in}{3.364895in}}{\pgfqpoint{1.895608in}{3.373131in}}%
\pgfpathcurveto{\pgfqpoint{1.895608in}{3.381368in}}{\pgfqpoint{1.892335in}{3.389268in}}{\pgfqpoint{1.886511in}{3.395092in}}%
\pgfpathcurveto{\pgfqpoint{1.880688in}{3.400915in}}{\pgfqpoint{1.872788in}{3.404188in}}{\pgfqpoint{1.864551in}{3.404188in}}%
\pgfpathcurveto{\pgfqpoint{1.856315in}{3.404188in}}{\pgfqpoint{1.848415in}{3.400915in}}{\pgfqpoint{1.842591in}{3.395092in}}%
\pgfpathcurveto{\pgfqpoint{1.836767in}{3.389268in}}{\pgfqpoint{1.833495in}{3.381368in}}{\pgfqpoint{1.833495in}{3.373131in}}%
\pgfpathcurveto{\pgfqpoint{1.833495in}{3.364895in}}{\pgfqpoint{1.836767in}{3.356995in}}{\pgfqpoint{1.842591in}{3.351171in}}%
\pgfpathcurveto{\pgfqpoint{1.848415in}{3.345347in}}{\pgfqpoint{1.856315in}{3.342075in}}{\pgfqpoint{1.864551in}{3.342075in}}%
\pgfpathclose%
\pgfusepath{stroke,fill}%
\end{pgfscope}%
\begin{pgfscope}%
\pgfpathrectangle{\pgfqpoint{0.100000in}{0.212622in}}{\pgfqpoint{3.696000in}{3.696000in}}%
\pgfusepath{clip}%
\pgfsetbuttcap%
\pgfsetroundjoin%
\definecolor{currentfill}{rgb}{0.121569,0.466667,0.705882}%
\pgfsetfillcolor{currentfill}%
\pgfsetfillopacity{0.301664}%
\pgfsetlinewidth{1.003750pt}%
\definecolor{currentstroke}{rgb}{0.121569,0.466667,0.705882}%
\pgfsetstrokecolor{currentstroke}%
\pgfsetstrokeopacity{0.301664}%
\pgfsetdash{}{0pt}%
\pgfpathmoveto{\pgfqpoint{1.864228in}{3.341731in}}%
\pgfpathcurveto{\pgfqpoint{1.872465in}{3.341731in}}{\pgfqpoint{1.880365in}{3.345003in}}{\pgfqpoint{1.886189in}{3.350827in}}%
\pgfpathcurveto{\pgfqpoint{1.892012in}{3.356651in}}{\pgfqpoint{1.895285in}{3.364551in}}{\pgfqpoint{1.895285in}{3.372787in}}%
\pgfpathcurveto{\pgfqpoint{1.895285in}{3.381024in}}{\pgfqpoint{1.892012in}{3.388924in}}{\pgfqpoint{1.886189in}{3.394748in}}%
\pgfpathcurveto{\pgfqpoint{1.880365in}{3.400572in}}{\pgfqpoint{1.872465in}{3.403844in}}{\pgfqpoint{1.864228in}{3.403844in}}%
\pgfpathcurveto{\pgfqpoint{1.855992in}{3.403844in}}{\pgfqpoint{1.848092in}{3.400572in}}{\pgfqpoint{1.842268in}{3.394748in}}%
\pgfpathcurveto{\pgfqpoint{1.836444in}{3.388924in}}{\pgfqpoint{1.833172in}{3.381024in}}{\pgfqpoint{1.833172in}{3.372787in}}%
\pgfpathcurveto{\pgfqpoint{1.833172in}{3.364551in}}{\pgfqpoint{1.836444in}{3.356651in}}{\pgfqpoint{1.842268in}{3.350827in}}%
\pgfpathcurveto{\pgfqpoint{1.848092in}{3.345003in}}{\pgfqpoint{1.855992in}{3.341731in}}{\pgfqpoint{1.864228in}{3.341731in}}%
\pgfpathclose%
\pgfusepath{stroke,fill}%
\end{pgfscope}%
\begin{pgfscope}%
\pgfpathrectangle{\pgfqpoint{0.100000in}{0.212622in}}{\pgfqpoint{3.696000in}{3.696000in}}%
\pgfusepath{clip}%
\pgfsetbuttcap%
\pgfsetroundjoin%
\definecolor{currentfill}{rgb}{0.121569,0.466667,0.705882}%
\pgfsetfillcolor{currentfill}%
\pgfsetfillopacity{0.301800}%
\pgfsetlinewidth{1.003750pt}%
\definecolor{currentstroke}{rgb}{0.121569,0.466667,0.705882}%
\pgfsetstrokecolor{currentstroke}%
\pgfsetstrokeopacity{0.301800}%
\pgfsetdash{}{0pt}%
\pgfpathmoveto{\pgfqpoint{1.863661in}{3.341041in}}%
\pgfpathcurveto{\pgfqpoint{1.871897in}{3.341041in}}{\pgfqpoint{1.879797in}{3.344314in}}{\pgfqpoint{1.885621in}{3.350138in}}%
\pgfpathcurveto{\pgfqpoint{1.891445in}{3.355962in}}{\pgfqpoint{1.894717in}{3.363862in}}{\pgfqpoint{1.894717in}{3.372098in}}%
\pgfpathcurveto{\pgfqpoint{1.894717in}{3.380334in}}{\pgfqpoint{1.891445in}{3.388234in}}{\pgfqpoint{1.885621in}{3.394058in}}%
\pgfpathcurveto{\pgfqpoint{1.879797in}{3.399882in}}{\pgfqpoint{1.871897in}{3.403154in}}{\pgfqpoint{1.863661in}{3.403154in}}%
\pgfpathcurveto{\pgfqpoint{1.855424in}{3.403154in}}{\pgfqpoint{1.847524in}{3.399882in}}{\pgfqpoint{1.841700in}{3.394058in}}%
\pgfpathcurveto{\pgfqpoint{1.835876in}{3.388234in}}{\pgfqpoint{1.832604in}{3.380334in}}{\pgfqpoint{1.832604in}{3.372098in}}%
\pgfpathcurveto{\pgfqpoint{1.832604in}{3.363862in}}{\pgfqpoint{1.835876in}{3.355962in}}{\pgfqpoint{1.841700in}{3.350138in}}%
\pgfpathcurveto{\pgfqpoint{1.847524in}{3.344314in}}{\pgfqpoint{1.855424in}{3.341041in}}{\pgfqpoint{1.863661in}{3.341041in}}%
\pgfpathclose%
\pgfusepath{stroke,fill}%
\end{pgfscope}%
\begin{pgfscope}%
\pgfpathrectangle{\pgfqpoint{0.100000in}{0.212622in}}{\pgfqpoint{3.696000in}{3.696000in}}%
\pgfusepath{clip}%
\pgfsetbuttcap%
\pgfsetroundjoin%
\definecolor{currentfill}{rgb}{0.121569,0.466667,0.705882}%
\pgfsetfillcolor{currentfill}%
\pgfsetfillopacity{0.301812}%
\pgfsetlinewidth{1.003750pt}%
\definecolor{currentstroke}{rgb}{0.121569,0.466667,0.705882}%
\pgfsetstrokecolor{currentstroke}%
\pgfsetstrokeopacity{0.301812}%
\pgfsetdash{}{0pt}%
\pgfpathmoveto{\pgfqpoint{1.890910in}{3.323936in}}%
\pgfpathcurveto{\pgfqpoint{1.899146in}{3.323936in}}{\pgfqpoint{1.907046in}{3.327208in}}{\pgfqpoint{1.912870in}{3.333032in}}%
\pgfpathcurveto{\pgfqpoint{1.918694in}{3.338856in}}{\pgfqpoint{1.921967in}{3.346756in}}{\pgfqpoint{1.921967in}{3.354992in}}%
\pgfpathcurveto{\pgfqpoint{1.921967in}{3.363228in}}{\pgfqpoint{1.918694in}{3.371129in}}{\pgfqpoint{1.912870in}{3.376952in}}%
\pgfpathcurveto{\pgfqpoint{1.907046in}{3.382776in}}{\pgfqpoint{1.899146in}{3.386049in}}{\pgfqpoint{1.890910in}{3.386049in}}%
\pgfpathcurveto{\pgfqpoint{1.882674in}{3.386049in}}{\pgfqpoint{1.874774in}{3.382776in}}{\pgfqpoint{1.868950in}{3.376952in}}%
\pgfpathcurveto{\pgfqpoint{1.863126in}{3.371129in}}{\pgfqpoint{1.859854in}{3.363228in}}{\pgfqpoint{1.859854in}{3.354992in}}%
\pgfpathcurveto{\pgfqpoint{1.859854in}{3.346756in}}{\pgfqpoint{1.863126in}{3.338856in}}{\pgfqpoint{1.868950in}{3.333032in}}%
\pgfpathcurveto{\pgfqpoint{1.874774in}{3.327208in}}{\pgfqpoint{1.882674in}{3.323936in}}{\pgfqpoint{1.890910in}{3.323936in}}%
\pgfpathclose%
\pgfusepath{stroke,fill}%
\end{pgfscope}%
\begin{pgfscope}%
\pgfpathrectangle{\pgfqpoint{0.100000in}{0.212622in}}{\pgfqpoint{3.696000in}{3.696000in}}%
\pgfusepath{clip}%
\pgfsetbuttcap%
\pgfsetroundjoin%
\definecolor{currentfill}{rgb}{0.121569,0.466667,0.705882}%
\pgfsetfillcolor{currentfill}%
\pgfsetfillopacity{0.302045}%
\pgfsetlinewidth{1.003750pt}%
\definecolor{currentstroke}{rgb}{0.121569,0.466667,0.705882}%
\pgfsetstrokecolor{currentstroke}%
\pgfsetstrokeopacity{0.302045}%
\pgfsetdash{}{0pt}%
\pgfpathmoveto{\pgfqpoint{1.862678in}{3.339663in}}%
\pgfpathcurveto{\pgfqpoint{1.870915in}{3.339663in}}{\pgfqpoint{1.878815in}{3.342935in}}{\pgfqpoint{1.884639in}{3.348759in}}%
\pgfpathcurveto{\pgfqpoint{1.890462in}{3.354583in}}{\pgfqpoint{1.893735in}{3.362483in}}{\pgfqpoint{1.893735in}{3.370719in}}%
\pgfpathcurveto{\pgfqpoint{1.893735in}{3.378956in}}{\pgfqpoint{1.890462in}{3.386856in}}{\pgfqpoint{1.884639in}{3.392680in}}%
\pgfpathcurveto{\pgfqpoint{1.878815in}{3.398504in}}{\pgfqpoint{1.870915in}{3.401776in}}{\pgfqpoint{1.862678in}{3.401776in}}%
\pgfpathcurveto{\pgfqpoint{1.854442in}{3.401776in}}{\pgfqpoint{1.846542in}{3.398504in}}{\pgfqpoint{1.840718in}{3.392680in}}%
\pgfpathcurveto{\pgfqpoint{1.834894in}{3.386856in}}{\pgfqpoint{1.831622in}{3.378956in}}{\pgfqpoint{1.831622in}{3.370719in}}%
\pgfpathcurveto{\pgfqpoint{1.831622in}{3.362483in}}{\pgfqpoint{1.834894in}{3.354583in}}{\pgfqpoint{1.840718in}{3.348759in}}%
\pgfpathcurveto{\pgfqpoint{1.846542in}{3.342935in}}{\pgfqpoint{1.854442in}{3.339663in}}{\pgfqpoint{1.862678in}{3.339663in}}%
\pgfpathclose%
\pgfusepath{stroke,fill}%
\end{pgfscope}%
\begin{pgfscope}%
\pgfpathrectangle{\pgfqpoint{0.100000in}{0.212622in}}{\pgfqpoint{3.696000in}{3.696000in}}%
\pgfusepath{clip}%
\pgfsetbuttcap%
\pgfsetroundjoin%
\definecolor{currentfill}{rgb}{0.121569,0.466667,0.705882}%
\pgfsetfillcolor{currentfill}%
\pgfsetfillopacity{0.302221}%
\pgfsetlinewidth{1.003750pt}%
\definecolor{currentstroke}{rgb}{0.121569,0.466667,0.705882}%
\pgfsetstrokecolor{currentstroke}%
\pgfsetstrokeopacity{0.302221}%
\pgfsetdash{}{0pt}%
\pgfpathmoveto{\pgfqpoint{1.861977in}{3.338588in}}%
\pgfpathcurveto{\pgfqpoint{1.870214in}{3.338588in}}{\pgfqpoint{1.878114in}{3.341860in}}{\pgfqpoint{1.883938in}{3.347684in}}%
\pgfpathcurveto{\pgfqpoint{1.889761in}{3.353508in}}{\pgfqpoint{1.893034in}{3.361408in}}{\pgfqpoint{1.893034in}{3.369644in}}%
\pgfpathcurveto{\pgfqpoint{1.893034in}{3.377880in}}{\pgfqpoint{1.889761in}{3.385780in}}{\pgfqpoint{1.883938in}{3.391604in}}%
\pgfpathcurveto{\pgfqpoint{1.878114in}{3.397428in}}{\pgfqpoint{1.870214in}{3.400701in}}{\pgfqpoint{1.861977in}{3.400701in}}%
\pgfpathcurveto{\pgfqpoint{1.853741in}{3.400701in}}{\pgfqpoint{1.845841in}{3.397428in}}{\pgfqpoint{1.840017in}{3.391604in}}%
\pgfpathcurveto{\pgfqpoint{1.834193in}{3.385780in}}{\pgfqpoint{1.830921in}{3.377880in}}{\pgfqpoint{1.830921in}{3.369644in}}%
\pgfpathcurveto{\pgfqpoint{1.830921in}{3.361408in}}{\pgfqpoint{1.834193in}{3.353508in}}{\pgfqpoint{1.840017in}{3.347684in}}%
\pgfpathcurveto{\pgfqpoint{1.845841in}{3.341860in}}{\pgfqpoint{1.853741in}{3.338588in}}{\pgfqpoint{1.861977in}{3.338588in}}%
\pgfpathclose%
\pgfusepath{stroke,fill}%
\end{pgfscope}%
\begin{pgfscope}%
\pgfpathrectangle{\pgfqpoint{0.100000in}{0.212622in}}{\pgfqpoint{3.696000in}{3.696000in}}%
\pgfusepath{clip}%
\pgfsetbuttcap%
\pgfsetroundjoin%
\definecolor{currentfill}{rgb}{0.121569,0.466667,0.705882}%
\pgfsetfillcolor{currentfill}%
\pgfsetfillopacity{0.302289}%
\pgfsetlinewidth{1.003750pt}%
\definecolor{currentstroke}{rgb}{0.121569,0.466667,0.705882}%
\pgfsetstrokecolor{currentstroke}%
\pgfsetstrokeopacity{0.302289}%
\pgfsetdash{}{0pt}%
\pgfpathmoveto{\pgfqpoint{1.861712in}{3.338151in}}%
\pgfpathcurveto{\pgfqpoint{1.869948in}{3.338151in}}{\pgfqpoint{1.877848in}{3.341423in}}{\pgfqpoint{1.883672in}{3.347247in}}%
\pgfpathcurveto{\pgfqpoint{1.889496in}{3.353071in}}{\pgfqpoint{1.892768in}{3.360971in}}{\pgfqpoint{1.892768in}{3.369207in}}%
\pgfpathcurveto{\pgfqpoint{1.892768in}{3.377443in}}{\pgfqpoint{1.889496in}{3.385343in}}{\pgfqpoint{1.883672in}{3.391167in}}%
\pgfpathcurveto{\pgfqpoint{1.877848in}{3.396991in}}{\pgfqpoint{1.869948in}{3.400264in}}{\pgfqpoint{1.861712in}{3.400264in}}%
\pgfpathcurveto{\pgfqpoint{1.853475in}{3.400264in}}{\pgfqpoint{1.845575in}{3.396991in}}{\pgfqpoint{1.839751in}{3.391167in}}%
\pgfpathcurveto{\pgfqpoint{1.833927in}{3.385343in}}{\pgfqpoint{1.830655in}{3.377443in}}{\pgfqpoint{1.830655in}{3.369207in}}%
\pgfpathcurveto{\pgfqpoint{1.830655in}{3.360971in}}{\pgfqpoint{1.833927in}{3.353071in}}{\pgfqpoint{1.839751in}{3.347247in}}%
\pgfpathcurveto{\pgfqpoint{1.845575in}{3.341423in}}{\pgfqpoint{1.853475in}{3.338151in}}{\pgfqpoint{1.861712in}{3.338151in}}%
\pgfpathclose%
\pgfusepath{stroke,fill}%
\end{pgfscope}%
\begin{pgfscope}%
\pgfpathrectangle{\pgfqpoint{0.100000in}{0.212622in}}{\pgfqpoint{3.696000in}{3.696000in}}%
\pgfusepath{clip}%
\pgfsetbuttcap%
\pgfsetroundjoin%
\definecolor{currentfill}{rgb}{0.121569,0.466667,0.705882}%
\pgfsetfillcolor{currentfill}%
\pgfsetfillopacity{0.302289}%
\pgfsetlinewidth{1.003750pt}%
\definecolor{currentstroke}{rgb}{0.121569,0.466667,0.705882}%
\pgfsetstrokecolor{currentstroke}%
\pgfsetstrokeopacity{0.302289}%
\pgfsetdash{}{0pt}%
\pgfpathmoveto{\pgfqpoint{1.861711in}{3.338150in}}%
\pgfpathcurveto{\pgfqpoint{1.869947in}{3.338150in}}{\pgfqpoint{1.877847in}{3.341422in}}{\pgfqpoint{1.883671in}{3.347246in}}%
\pgfpathcurveto{\pgfqpoint{1.889495in}{3.353070in}}{\pgfqpoint{1.892768in}{3.360970in}}{\pgfqpoint{1.892768in}{3.369206in}}%
\pgfpathcurveto{\pgfqpoint{1.892768in}{3.377443in}}{\pgfqpoint{1.889495in}{3.385343in}}{\pgfqpoint{1.883671in}{3.391167in}}%
\pgfpathcurveto{\pgfqpoint{1.877847in}{3.396990in}}{\pgfqpoint{1.869947in}{3.400263in}}{\pgfqpoint{1.861711in}{3.400263in}}%
\pgfpathcurveto{\pgfqpoint{1.853475in}{3.400263in}}{\pgfqpoint{1.845575in}{3.396990in}}{\pgfqpoint{1.839751in}{3.391167in}}%
\pgfpathcurveto{\pgfqpoint{1.833927in}{3.385343in}}{\pgfqpoint{1.830655in}{3.377443in}}{\pgfqpoint{1.830655in}{3.369206in}}%
\pgfpathcurveto{\pgfqpoint{1.830655in}{3.360970in}}{\pgfqpoint{1.833927in}{3.353070in}}{\pgfqpoint{1.839751in}{3.347246in}}%
\pgfpathcurveto{\pgfqpoint{1.845575in}{3.341422in}}{\pgfqpoint{1.853475in}{3.338150in}}{\pgfqpoint{1.861711in}{3.338150in}}%
\pgfpathclose%
\pgfusepath{stroke,fill}%
\end{pgfscope}%
\begin{pgfscope}%
\pgfpathrectangle{\pgfqpoint{0.100000in}{0.212622in}}{\pgfqpoint{3.696000in}{3.696000in}}%
\pgfusepath{clip}%
\pgfsetbuttcap%
\pgfsetroundjoin%
\definecolor{currentfill}{rgb}{0.121569,0.466667,0.705882}%
\pgfsetfillcolor{currentfill}%
\pgfsetfillopacity{0.302290}%
\pgfsetlinewidth{1.003750pt}%
\definecolor{currentstroke}{rgb}{0.121569,0.466667,0.705882}%
\pgfsetstrokecolor{currentstroke}%
\pgfsetstrokeopacity{0.302290}%
\pgfsetdash{}{0pt}%
\pgfpathmoveto{\pgfqpoint{1.861710in}{3.338148in}}%
\pgfpathcurveto{\pgfqpoint{1.869946in}{3.338148in}}{\pgfqpoint{1.877847in}{3.341421in}}{\pgfqpoint{1.883670in}{3.347245in}}%
\pgfpathcurveto{\pgfqpoint{1.889494in}{3.353068in}}{\pgfqpoint{1.892767in}{3.360968in}}{\pgfqpoint{1.892767in}{3.369205in}}%
\pgfpathcurveto{\pgfqpoint{1.892767in}{3.377441in}}{\pgfqpoint{1.889494in}{3.385341in}}{\pgfqpoint{1.883670in}{3.391165in}}%
\pgfpathcurveto{\pgfqpoint{1.877847in}{3.396989in}}{\pgfqpoint{1.869946in}{3.400261in}}{\pgfqpoint{1.861710in}{3.400261in}}%
\pgfpathcurveto{\pgfqpoint{1.853474in}{3.400261in}}{\pgfqpoint{1.845574in}{3.396989in}}{\pgfqpoint{1.839750in}{3.391165in}}%
\pgfpathcurveto{\pgfqpoint{1.833926in}{3.385341in}}{\pgfqpoint{1.830654in}{3.377441in}}{\pgfqpoint{1.830654in}{3.369205in}}%
\pgfpathcurveto{\pgfqpoint{1.830654in}{3.360968in}}{\pgfqpoint{1.833926in}{3.353068in}}{\pgfqpoint{1.839750in}{3.347245in}}%
\pgfpathcurveto{\pgfqpoint{1.845574in}{3.341421in}}{\pgfqpoint{1.853474in}{3.338148in}}{\pgfqpoint{1.861710in}{3.338148in}}%
\pgfpathclose%
\pgfusepath{stroke,fill}%
\end{pgfscope}%
\begin{pgfscope}%
\pgfpathrectangle{\pgfqpoint{0.100000in}{0.212622in}}{\pgfqpoint{3.696000in}{3.696000in}}%
\pgfusepath{clip}%
\pgfsetbuttcap%
\pgfsetroundjoin%
\definecolor{currentfill}{rgb}{0.121569,0.466667,0.705882}%
\pgfsetfillcolor{currentfill}%
\pgfsetfillopacity{0.302290}%
\pgfsetlinewidth{1.003750pt}%
\definecolor{currentstroke}{rgb}{0.121569,0.466667,0.705882}%
\pgfsetstrokecolor{currentstroke}%
\pgfsetstrokeopacity{0.302290}%
\pgfsetdash{}{0pt}%
\pgfpathmoveto{\pgfqpoint{1.861709in}{3.338145in}}%
\pgfpathcurveto{\pgfqpoint{1.869945in}{3.338145in}}{\pgfqpoint{1.877845in}{3.341418in}}{\pgfqpoint{1.883669in}{3.347242in}}%
\pgfpathcurveto{\pgfqpoint{1.889493in}{3.353066in}}{\pgfqpoint{1.892765in}{3.360966in}}{\pgfqpoint{1.892765in}{3.369202in}}%
\pgfpathcurveto{\pgfqpoint{1.892765in}{3.377438in}}{\pgfqpoint{1.889493in}{3.385338in}}{\pgfqpoint{1.883669in}{3.391162in}}%
\pgfpathcurveto{\pgfqpoint{1.877845in}{3.396986in}}{\pgfqpoint{1.869945in}{3.400258in}}{\pgfqpoint{1.861709in}{3.400258in}}%
\pgfpathcurveto{\pgfqpoint{1.853472in}{3.400258in}}{\pgfqpoint{1.845572in}{3.396986in}}{\pgfqpoint{1.839748in}{3.391162in}}%
\pgfpathcurveto{\pgfqpoint{1.833924in}{3.385338in}}{\pgfqpoint{1.830652in}{3.377438in}}{\pgfqpoint{1.830652in}{3.369202in}}%
\pgfpathcurveto{\pgfqpoint{1.830652in}{3.360966in}}{\pgfqpoint{1.833924in}{3.353066in}}{\pgfqpoint{1.839748in}{3.347242in}}%
\pgfpathcurveto{\pgfqpoint{1.845572in}{3.341418in}}{\pgfqpoint{1.853472in}{3.338145in}}{\pgfqpoint{1.861709in}{3.338145in}}%
\pgfpathclose%
\pgfusepath{stroke,fill}%
\end{pgfscope}%
\begin{pgfscope}%
\pgfpathrectangle{\pgfqpoint{0.100000in}{0.212622in}}{\pgfqpoint{3.696000in}{3.696000in}}%
\pgfusepath{clip}%
\pgfsetbuttcap%
\pgfsetroundjoin%
\definecolor{currentfill}{rgb}{0.121569,0.466667,0.705882}%
\pgfsetfillcolor{currentfill}%
\pgfsetfillopacity{0.302291}%
\pgfsetlinewidth{1.003750pt}%
\definecolor{currentstroke}{rgb}{0.121569,0.466667,0.705882}%
\pgfsetstrokecolor{currentstroke}%
\pgfsetstrokeopacity{0.302291}%
\pgfsetdash{}{0pt}%
\pgfpathmoveto{\pgfqpoint{1.861706in}{3.338140in}}%
\pgfpathcurveto{\pgfqpoint{1.869942in}{3.338140in}}{\pgfqpoint{1.877842in}{3.341413in}}{\pgfqpoint{1.883666in}{3.347237in}}%
\pgfpathcurveto{\pgfqpoint{1.889490in}{3.353061in}}{\pgfqpoint{1.892762in}{3.360961in}}{\pgfqpoint{1.892762in}{3.369197in}}%
\pgfpathcurveto{\pgfqpoint{1.892762in}{3.377433in}}{\pgfqpoint{1.889490in}{3.385333in}}{\pgfqpoint{1.883666in}{3.391157in}}%
\pgfpathcurveto{\pgfqpoint{1.877842in}{3.396981in}}{\pgfqpoint{1.869942in}{3.400253in}}{\pgfqpoint{1.861706in}{3.400253in}}%
\pgfpathcurveto{\pgfqpoint{1.853469in}{3.400253in}}{\pgfqpoint{1.845569in}{3.396981in}}{\pgfqpoint{1.839745in}{3.391157in}}%
\pgfpathcurveto{\pgfqpoint{1.833921in}{3.385333in}}{\pgfqpoint{1.830649in}{3.377433in}}{\pgfqpoint{1.830649in}{3.369197in}}%
\pgfpathcurveto{\pgfqpoint{1.830649in}{3.360961in}}{\pgfqpoint{1.833921in}{3.353061in}}{\pgfqpoint{1.839745in}{3.347237in}}%
\pgfpathcurveto{\pgfqpoint{1.845569in}{3.341413in}}{\pgfqpoint{1.853469in}{3.338140in}}{\pgfqpoint{1.861706in}{3.338140in}}%
\pgfpathclose%
\pgfusepath{stroke,fill}%
\end{pgfscope}%
\begin{pgfscope}%
\pgfpathrectangle{\pgfqpoint{0.100000in}{0.212622in}}{\pgfqpoint{3.696000in}{3.696000in}}%
\pgfusepath{clip}%
\pgfsetbuttcap%
\pgfsetroundjoin%
\definecolor{currentfill}{rgb}{0.121569,0.466667,0.705882}%
\pgfsetfillcolor{currentfill}%
\pgfsetfillopacity{0.302293}%
\pgfsetlinewidth{1.003750pt}%
\definecolor{currentstroke}{rgb}{0.121569,0.466667,0.705882}%
\pgfsetstrokecolor{currentstroke}%
\pgfsetstrokeopacity{0.302293}%
\pgfsetdash{}{0pt}%
\pgfpathmoveto{\pgfqpoint{1.861700in}{3.338131in}}%
\pgfpathcurveto{\pgfqpoint{1.869937in}{3.338131in}}{\pgfqpoint{1.877837in}{3.341404in}}{\pgfqpoint{1.883661in}{3.347228in}}%
\pgfpathcurveto{\pgfqpoint{1.889484in}{3.353051in}}{\pgfqpoint{1.892757in}{3.360951in}}{\pgfqpoint{1.892757in}{3.369188in}}%
\pgfpathcurveto{\pgfqpoint{1.892757in}{3.377424in}}{\pgfqpoint{1.889484in}{3.385324in}}{\pgfqpoint{1.883661in}{3.391148in}}%
\pgfpathcurveto{\pgfqpoint{1.877837in}{3.396972in}}{\pgfqpoint{1.869937in}{3.400244in}}{\pgfqpoint{1.861700in}{3.400244in}}%
\pgfpathcurveto{\pgfqpoint{1.853464in}{3.400244in}}{\pgfqpoint{1.845564in}{3.396972in}}{\pgfqpoint{1.839740in}{3.391148in}}%
\pgfpathcurveto{\pgfqpoint{1.833916in}{3.385324in}}{\pgfqpoint{1.830644in}{3.377424in}}{\pgfqpoint{1.830644in}{3.369188in}}%
\pgfpathcurveto{\pgfqpoint{1.830644in}{3.360951in}}{\pgfqpoint{1.833916in}{3.353051in}}{\pgfqpoint{1.839740in}{3.347228in}}%
\pgfpathcurveto{\pgfqpoint{1.845564in}{3.341404in}}{\pgfqpoint{1.853464in}{3.338131in}}{\pgfqpoint{1.861700in}{3.338131in}}%
\pgfpathclose%
\pgfusepath{stroke,fill}%
\end{pgfscope}%
\begin{pgfscope}%
\pgfpathrectangle{\pgfqpoint{0.100000in}{0.212622in}}{\pgfqpoint{3.696000in}{3.696000in}}%
\pgfusepath{clip}%
\pgfsetbuttcap%
\pgfsetroundjoin%
\definecolor{currentfill}{rgb}{0.121569,0.466667,0.705882}%
\pgfsetfillcolor{currentfill}%
\pgfsetfillopacity{0.302296}%
\pgfsetlinewidth{1.003750pt}%
\definecolor{currentstroke}{rgb}{0.121569,0.466667,0.705882}%
\pgfsetstrokecolor{currentstroke}%
\pgfsetstrokeopacity{0.302296}%
\pgfsetdash{}{0pt}%
\pgfpathmoveto{\pgfqpoint{1.861691in}{3.338115in}}%
\pgfpathcurveto{\pgfqpoint{1.869927in}{3.338115in}}{\pgfqpoint{1.877827in}{3.341387in}}{\pgfqpoint{1.883651in}{3.347211in}}%
\pgfpathcurveto{\pgfqpoint{1.889475in}{3.353035in}}{\pgfqpoint{1.892747in}{3.360935in}}{\pgfqpoint{1.892747in}{3.369171in}}%
\pgfpathcurveto{\pgfqpoint{1.892747in}{3.377408in}}{\pgfqpoint{1.889475in}{3.385308in}}{\pgfqpoint{1.883651in}{3.391132in}}%
\pgfpathcurveto{\pgfqpoint{1.877827in}{3.396956in}}{\pgfqpoint{1.869927in}{3.400228in}}{\pgfqpoint{1.861691in}{3.400228in}}%
\pgfpathcurveto{\pgfqpoint{1.853454in}{3.400228in}}{\pgfqpoint{1.845554in}{3.396956in}}{\pgfqpoint{1.839730in}{3.391132in}}%
\pgfpathcurveto{\pgfqpoint{1.833906in}{3.385308in}}{\pgfqpoint{1.830634in}{3.377408in}}{\pgfqpoint{1.830634in}{3.369171in}}%
\pgfpathcurveto{\pgfqpoint{1.830634in}{3.360935in}}{\pgfqpoint{1.833906in}{3.353035in}}{\pgfqpoint{1.839730in}{3.347211in}}%
\pgfpathcurveto{\pgfqpoint{1.845554in}{3.341387in}}{\pgfqpoint{1.853454in}{3.338115in}}{\pgfqpoint{1.861691in}{3.338115in}}%
\pgfpathclose%
\pgfusepath{stroke,fill}%
\end{pgfscope}%
\begin{pgfscope}%
\pgfpathrectangle{\pgfqpoint{0.100000in}{0.212622in}}{\pgfqpoint{3.696000in}{3.696000in}}%
\pgfusepath{clip}%
\pgfsetbuttcap%
\pgfsetroundjoin%
\definecolor{currentfill}{rgb}{0.121569,0.466667,0.705882}%
\pgfsetfillcolor{currentfill}%
\pgfsetfillopacity{0.302301}%
\pgfsetlinewidth{1.003750pt}%
\definecolor{currentstroke}{rgb}{0.121569,0.466667,0.705882}%
\pgfsetstrokecolor{currentstroke}%
\pgfsetstrokeopacity{0.302301}%
\pgfsetdash{}{0pt}%
\pgfpathmoveto{\pgfqpoint{1.861674in}{3.338085in}}%
\pgfpathcurveto{\pgfqpoint{1.869910in}{3.338085in}}{\pgfqpoint{1.877810in}{3.341357in}}{\pgfqpoint{1.883634in}{3.347181in}}%
\pgfpathcurveto{\pgfqpoint{1.889458in}{3.353005in}}{\pgfqpoint{1.892730in}{3.360905in}}{\pgfqpoint{1.892730in}{3.369141in}}%
\pgfpathcurveto{\pgfqpoint{1.892730in}{3.377378in}}{\pgfqpoint{1.889458in}{3.385278in}}{\pgfqpoint{1.883634in}{3.391102in}}%
\pgfpathcurveto{\pgfqpoint{1.877810in}{3.396926in}}{\pgfqpoint{1.869910in}{3.400198in}}{\pgfqpoint{1.861674in}{3.400198in}}%
\pgfpathcurveto{\pgfqpoint{1.853437in}{3.400198in}}{\pgfqpoint{1.845537in}{3.396926in}}{\pgfqpoint{1.839713in}{3.391102in}}%
\pgfpathcurveto{\pgfqpoint{1.833889in}{3.385278in}}{\pgfqpoint{1.830617in}{3.377378in}}{\pgfqpoint{1.830617in}{3.369141in}}%
\pgfpathcurveto{\pgfqpoint{1.830617in}{3.360905in}}{\pgfqpoint{1.833889in}{3.353005in}}{\pgfqpoint{1.839713in}{3.347181in}}%
\pgfpathcurveto{\pgfqpoint{1.845537in}{3.341357in}}{\pgfqpoint{1.853437in}{3.338085in}}{\pgfqpoint{1.861674in}{3.338085in}}%
\pgfpathclose%
\pgfusepath{stroke,fill}%
\end{pgfscope}%
\begin{pgfscope}%
\pgfpathrectangle{\pgfqpoint{0.100000in}{0.212622in}}{\pgfqpoint{3.696000in}{3.696000in}}%
\pgfusepath{clip}%
\pgfsetbuttcap%
\pgfsetroundjoin%
\definecolor{currentfill}{rgb}{0.121569,0.466667,0.705882}%
\pgfsetfillcolor{currentfill}%
\pgfsetfillopacity{0.302312}%
\pgfsetlinewidth{1.003750pt}%
\definecolor{currentstroke}{rgb}{0.121569,0.466667,0.705882}%
\pgfsetstrokecolor{currentstroke}%
\pgfsetstrokeopacity{0.302312}%
\pgfsetdash{}{0pt}%
\pgfpathmoveto{\pgfqpoint{1.861643in}{3.338031in}}%
\pgfpathcurveto{\pgfqpoint{1.869879in}{3.338031in}}{\pgfqpoint{1.877779in}{3.341303in}}{\pgfqpoint{1.883603in}{3.347127in}}%
\pgfpathcurveto{\pgfqpoint{1.889427in}{3.352951in}}{\pgfqpoint{1.892699in}{3.360851in}}{\pgfqpoint{1.892699in}{3.369087in}}%
\pgfpathcurveto{\pgfqpoint{1.892699in}{3.377324in}}{\pgfqpoint{1.889427in}{3.385224in}}{\pgfqpoint{1.883603in}{3.391048in}}%
\pgfpathcurveto{\pgfqpoint{1.877779in}{3.396872in}}{\pgfqpoint{1.869879in}{3.400144in}}{\pgfqpoint{1.861643in}{3.400144in}}%
\pgfpathcurveto{\pgfqpoint{1.853406in}{3.400144in}}{\pgfqpoint{1.845506in}{3.396872in}}{\pgfqpoint{1.839682in}{3.391048in}}%
\pgfpathcurveto{\pgfqpoint{1.833858in}{3.385224in}}{\pgfqpoint{1.830586in}{3.377324in}}{\pgfqpoint{1.830586in}{3.369087in}}%
\pgfpathcurveto{\pgfqpoint{1.830586in}{3.360851in}}{\pgfqpoint{1.833858in}{3.352951in}}{\pgfqpoint{1.839682in}{3.347127in}}%
\pgfpathcurveto{\pgfqpoint{1.845506in}{3.341303in}}{\pgfqpoint{1.853406in}{3.338031in}}{\pgfqpoint{1.861643in}{3.338031in}}%
\pgfpathclose%
\pgfusepath{stroke,fill}%
\end{pgfscope}%
\begin{pgfscope}%
\pgfpathrectangle{\pgfqpoint{0.100000in}{0.212622in}}{\pgfqpoint{3.696000in}{3.696000in}}%
\pgfusepath{clip}%
\pgfsetbuttcap%
\pgfsetroundjoin%
\definecolor{currentfill}{rgb}{0.121569,0.466667,0.705882}%
\pgfsetfillcolor{currentfill}%
\pgfsetfillopacity{0.302331}%
\pgfsetlinewidth{1.003750pt}%
\definecolor{currentstroke}{rgb}{0.121569,0.466667,0.705882}%
\pgfsetstrokecolor{currentstroke}%
\pgfsetstrokeopacity{0.302331}%
\pgfsetdash{}{0pt}%
\pgfpathmoveto{\pgfqpoint{1.861586in}{3.337933in}}%
\pgfpathcurveto{\pgfqpoint{1.869822in}{3.337933in}}{\pgfqpoint{1.877722in}{3.341205in}}{\pgfqpoint{1.883546in}{3.347029in}}%
\pgfpathcurveto{\pgfqpoint{1.889370in}{3.352853in}}{\pgfqpoint{1.892643in}{3.360753in}}{\pgfqpoint{1.892643in}{3.368989in}}%
\pgfpathcurveto{\pgfqpoint{1.892643in}{3.377226in}}{\pgfqpoint{1.889370in}{3.385126in}}{\pgfqpoint{1.883546in}{3.390950in}}%
\pgfpathcurveto{\pgfqpoint{1.877722in}{3.396774in}}{\pgfqpoint{1.869822in}{3.400046in}}{\pgfqpoint{1.861586in}{3.400046in}}%
\pgfpathcurveto{\pgfqpoint{1.853350in}{3.400046in}}{\pgfqpoint{1.845450in}{3.396774in}}{\pgfqpoint{1.839626in}{3.390950in}}%
\pgfpathcurveto{\pgfqpoint{1.833802in}{3.385126in}}{\pgfqpoint{1.830530in}{3.377226in}}{\pgfqpoint{1.830530in}{3.368989in}}%
\pgfpathcurveto{\pgfqpoint{1.830530in}{3.360753in}}{\pgfqpoint{1.833802in}{3.352853in}}{\pgfqpoint{1.839626in}{3.347029in}}%
\pgfpathcurveto{\pgfqpoint{1.845450in}{3.341205in}}{\pgfqpoint{1.853350in}{3.337933in}}{\pgfqpoint{1.861586in}{3.337933in}}%
\pgfpathclose%
\pgfusepath{stroke,fill}%
\end{pgfscope}%
\begin{pgfscope}%
\pgfpathrectangle{\pgfqpoint{0.100000in}{0.212622in}}{\pgfqpoint{3.696000in}{3.696000in}}%
\pgfusepath{clip}%
\pgfsetbuttcap%
\pgfsetroundjoin%
\definecolor{currentfill}{rgb}{0.121569,0.466667,0.705882}%
\pgfsetfillcolor{currentfill}%
\pgfsetfillopacity{0.302366}%
\pgfsetlinewidth{1.003750pt}%
\definecolor{currentstroke}{rgb}{0.121569,0.466667,0.705882}%
\pgfsetstrokecolor{currentstroke}%
\pgfsetstrokeopacity{0.302366}%
\pgfsetdash{}{0pt}%
\pgfpathmoveto{\pgfqpoint{1.861484in}{3.337755in}}%
\pgfpathcurveto{\pgfqpoint{1.869720in}{3.337755in}}{\pgfqpoint{1.877621in}{3.341028in}}{\pgfqpoint{1.883444in}{3.346852in}}%
\pgfpathcurveto{\pgfqpoint{1.889268in}{3.352676in}}{\pgfqpoint{1.892541in}{3.360576in}}{\pgfqpoint{1.892541in}{3.368812in}}%
\pgfpathcurveto{\pgfqpoint{1.892541in}{3.377048in}}{\pgfqpoint{1.889268in}{3.384948in}}{\pgfqpoint{1.883444in}{3.390772in}}%
\pgfpathcurveto{\pgfqpoint{1.877621in}{3.396596in}}{\pgfqpoint{1.869720in}{3.399868in}}{\pgfqpoint{1.861484in}{3.399868in}}%
\pgfpathcurveto{\pgfqpoint{1.853248in}{3.399868in}}{\pgfqpoint{1.845348in}{3.396596in}}{\pgfqpoint{1.839524in}{3.390772in}}%
\pgfpathcurveto{\pgfqpoint{1.833700in}{3.384948in}}{\pgfqpoint{1.830428in}{3.377048in}}{\pgfqpoint{1.830428in}{3.368812in}}%
\pgfpathcurveto{\pgfqpoint{1.830428in}{3.360576in}}{\pgfqpoint{1.833700in}{3.352676in}}{\pgfqpoint{1.839524in}{3.346852in}}%
\pgfpathcurveto{\pgfqpoint{1.845348in}{3.341028in}}{\pgfqpoint{1.853248in}{3.337755in}}{\pgfqpoint{1.861484in}{3.337755in}}%
\pgfpathclose%
\pgfusepath{stroke,fill}%
\end{pgfscope}%
\begin{pgfscope}%
\pgfpathrectangle{\pgfqpoint{0.100000in}{0.212622in}}{\pgfqpoint{3.696000in}{3.696000in}}%
\pgfusepath{clip}%
\pgfsetbuttcap%
\pgfsetroundjoin%
\definecolor{currentfill}{rgb}{0.121569,0.466667,0.705882}%
\pgfsetfillcolor{currentfill}%
\pgfsetfillopacity{0.302429}%
\pgfsetlinewidth{1.003750pt}%
\definecolor{currentstroke}{rgb}{0.121569,0.466667,0.705882}%
\pgfsetstrokecolor{currentstroke}%
\pgfsetstrokeopacity{0.302429}%
\pgfsetdash{}{0pt}%
\pgfpathmoveto{\pgfqpoint{1.861301in}{3.337430in}}%
\pgfpathcurveto{\pgfqpoint{1.869537in}{3.337430in}}{\pgfqpoint{1.877437in}{3.340702in}}{\pgfqpoint{1.883261in}{3.346526in}}%
\pgfpathcurveto{\pgfqpoint{1.889085in}{3.352350in}}{\pgfqpoint{1.892357in}{3.360250in}}{\pgfqpoint{1.892357in}{3.368487in}}%
\pgfpathcurveto{\pgfqpoint{1.892357in}{3.376723in}}{\pgfqpoint{1.889085in}{3.384623in}}{\pgfqpoint{1.883261in}{3.390447in}}%
\pgfpathcurveto{\pgfqpoint{1.877437in}{3.396271in}}{\pgfqpoint{1.869537in}{3.399543in}}{\pgfqpoint{1.861301in}{3.399543in}}%
\pgfpathcurveto{\pgfqpoint{1.853064in}{3.399543in}}{\pgfqpoint{1.845164in}{3.396271in}}{\pgfqpoint{1.839340in}{3.390447in}}%
\pgfpathcurveto{\pgfqpoint{1.833516in}{3.384623in}}{\pgfqpoint{1.830244in}{3.376723in}}{\pgfqpoint{1.830244in}{3.368487in}}%
\pgfpathcurveto{\pgfqpoint{1.830244in}{3.360250in}}{\pgfqpoint{1.833516in}{3.352350in}}{\pgfqpoint{1.839340in}{3.346526in}}%
\pgfpathcurveto{\pgfqpoint{1.845164in}{3.340702in}}{\pgfqpoint{1.853064in}{3.337430in}}{\pgfqpoint{1.861301in}{3.337430in}}%
\pgfpathclose%
\pgfusepath{stroke,fill}%
\end{pgfscope}%
\begin{pgfscope}%
\pgfpathrectangle{\pgfqpoint{0.100000in}{0.212622in}}{\pgfqpoint{3.696000in}{3.696000in}}%
\pgfusepath{clip}%
\pgfsetbuttcap%
\pgfsetroundjoin%
\definecolor{currentfill}{rgb}{0.121569,0.466667,0.705882}%
\pgfsetfillcolor{currentfill}%
\pgfsetfillopacity{0.302531}%
\pgfsetlinewidth{1.003750pt}%
\definecolor{currentstroke}{rgb}{0.121569,0.466667,0.705882}%
\pgfsetstrokecolor{currentstroke}%
\pgfsetstrokeopacity{0.302531}%
\pgfsetdash{}{0pt}%
\pgfpathmoveto{\pgfqpoint{1.893693in}{3.319069in}}%
\pgfpathcurveto{\pgfqpoint{1.901929in}{3.319069in}}{\pgfqpoint{1.909829in}{3.322341in}}{\pgfqpoint{1.915653in}{3.328165in}}%
\pgfpathcurveto{\pgfqpoint{1.921477in}{3.333989in}}{\pgfqpoint{1.924750in}{3.341889in}}{\pgfqpoint{1.924750in}{3.350125in}}%
\pgfpathcurveto{\pgfqpoint{1.924750in}{3.358361in}}{\pgfqpoint{1.921477in}{3.366261in}}{\pgfqpoint{1.915653in}{3.372085in}}%
\pgfpathcurveto{\pgfqpoint{1.909829in}{3.377909in}}{\pgfqpoint{1.901929in}{3.381182in}}{\pgfqpoint{1.893693in}{3.381182in}}%
\pgfpathcurveto{\pgfqpoint{1.885457in}{3.381182in}}{\pgfqpoint{1.877557in}{3.377909in}}{\pgfqpoint{1.871733in}{3.372085in}}%
\pgfpathcurveto{\pgfqpoint{1.865909in}{3.366261in}}{\pgfqpoint{1.862637in}{3.358361in}}{\pgfqpoint{1.862637in}{3.350125in}}%
\pgfpathcurveto{\pgfqpoint{1.862637in}{3.341889in}}{\pgfqpoint{1.865909in}{3.333989in}}{\pgfqpoint{1.871733in}{3.328165in}}%
\pgfpathcurveto{\pgfqpoint{1.877557in}{3.322341in}}{\pgfqpoint{1.885457in}{3.319069in}}{\pgfqpoint{1.893693in}{3.319069in}}%
\pgfpathclose%
\pgfusepath{stroke,fill}%
\end{pgfscope}%
\begin{pgfscope}%
\pgfpathrectangle{\pgfqpoint{0.100000in}{0.212622in}}{\pgfqpoint{3.696000in}{3.696000in}}%
\pgfusepath{clip}%
\pgfsetbuttcap%
\pgfsetroundjoin%
\definecolor{currentfill}{rgb}{0.121569,0.466667,0.705882}%
\pgfsetfillcolor{currentfill}%
\pgfsetfillopacity{0.302545}%
\pgfsetlinewidth{1.003750pt}%
\definecolor{currentstroke}{rgb}{0.121569,0.466667,0.705882}%
\pgfsetstrokecolor{currentstroke}%
\pgfsetstrokeopacity{0.302545}%
\pgfsetdash{}{0pt}%
\pgfpathmoveto{\pgfqpoint{1.860964in}{3.336843in}}%
\pgfpathcurveto{\pgfqpoint{1.869200in}{3.336843in}}{\pgfqpoint{1.877100in}{3.340115in}}{\pgfqpoint{1.882924in}{3.345939in}}%
\pgfpathcurveto{\pgfqpoint{1.888748in}{3.351763in}}{\pgfqpoint{1.892020in}{3.359663in}}{\pgfqpoint{1.892020in}{3.367899in}}%
\pgfpathcurveto{\pgfqpoint{1.892020in}{3.376136in}}{\pgfqpoint{1.888748in}{3.384036in}}{\pgfqpoint{1.882924in}{3.389860in}}%
\pgfpathcurveto{\pgfqpoint{1.877100in}{3.395683in}}{\pgfqpoint{1.869200in}{3.398956in}}{\pgfqpoint{1.860964in}{3.398956in}}%
\pgfpathcurveto{\pgfqpoint{1.852728in}{3.398956in}}{\pgfqpoint{1.844827in}{3.395683in}}{\pgfqpoint{1.839004in}{3.389860in}}%
\pgfpathcurveto{\pgfqpoint{1.833180in}{3.384036in}}{\pgfqpoint{1.829907in}{3.376136in}}{\pgfqpoint{1.829907in}{3.367899in}}%
\pgfpathcurveto{\pgfqpoint{1.829907in}{3.359663in}}{\pgfqpoint{1.833180in}{3.351763in}}{\pgfqpoint{1.839004in}{3.345939in}}%
\pgfpathcurveto{\pgfqpoint{1.844827in}{3.340115in}}{\pgfqpoint{1.852728in}{3.336843in}}{\pgfqpoint{1.860964in}{3.336843in}}%
\pgfpathclose%
\pgfusepath{stroke,fill}%
\end{pgfscope}%
\begin{pgfscope}%
\pgfpathrectangle{\pgfqpoint{0.100000in}{0.212622in}}{\pgfqpoint{3.696000in}{3.696000in}}%
\pgfusepath{clip}%
\pgfsetbuttcap%
\pgfsetroundjoin%
\definecolor{currentfill}{rgb}{0.121569,0.466667,0.705882}%
\pgfsetfillcolor{currentfill}%
\pgfsetfillopacity{0.302757}%
\pgfsetlinewidth{1.003750pt}%
\definecolor{currentstroke}{rgb}{0.121569,0.466667,0.705882}%
\pgfsetstrokecolor{currentstroke}%
\pgfsetstrokeopacity{0.302757}%
\pgfsetdash{}{0pt}%
\pgfpathmoveto{\pgfqpoint{1.860341in}{3.335793in}}%
\pgfpathcurveto{\pgfqpoint{1.868577in}{3.335793in}}{\pgfqpoint{1.876477in}{3.339065in}}{\pgfqpoint{1.882301in}{3.344889in}}%
\pgfpathcurveto{\pgfqpoint{1.888125in}{3.350713in}}{\pgfqpoint{1.891397in}{3.358613in}}{\pgfqpoint{1.891397in}{3.366850in}}%
\pgfpathcurveto{\pgfqpoint{1.891397in}{3.375086in}}{\pgfqpoint{1.888125in}{3.382986in}}{\pgfqpoint{1.882301in}{3.388810in}}%
\pgfpathcurveto{\pgfqpoint{1.876477in}{3.394634in}}{\pgfqpoint{1.868577in}{3.397906in}}{\pgfqpoint{1.860341in}{3.397906in}}%
\pgfpathcurveto{\pgfqpoint{1.852104in}{3.397906in}}{\pgfqpoint{1.844204in}{3.394634in}}{\pgfqpoint{1.838380in}{3.388810in}}%
\pgfpathcurveto{\pgfqpoint{1.832556in}{3.382986in}}{\pgfqpoint{1.829284in}{3.375086in}}{\pgfqpoint{1.829284in}{3.366850in}}%
\pgfpathcurveto{\pgfqpoint{1.829284in}{3.358613in}}{\pgfqpoint{1.832556in}{3.350713in}}{\pgfqpoint{1.838380in}{3.344889in}}%
\pgfpathcurveto{\pgfqpoint{1.844204in}{3.339065in}}{\pgfqpoint{1.852104in}{3.335793in}}{\pgfqpoint{1.860341in}{3.335793in}}%
\pgfpathclose%
\pgfusepath{stroke,fill}%
\end{pgfscope}%
\begin{pgfscope}%
\pgfpathrectangle{\pgfqpoint{0.100000in}{0.212622in}}{\pgfqpoint{3.696000in}{3.696000in}}%
\pgfusepath{clip}%
\pgfsetbuttcap%
\pgfsetroundjoin%
\definecolor{currentfill}{rgb}{0.121569,0.466667,0.705882}%
\pgfsetfillcolor{currentfill}%
\pgfsetfillopacity{0.302905}%
\pgfsetlinewidth{1.003750pt}%
\definecolor{currentstroke}{rgb}{0.121569,0.466667,0.705882}%
\pgfsetstrokecolor{currentstroke}%
\pgfsetstrokeopacity{0.302905}%
\pgfsetdash{}{0pt}%
\pgfpathmoveto{\pgfqpoint{1.859926in}{3.335093in}}%
\pgfpathcurveto{\pgfqpoint{1.868162in}{3.335093in}}{\pgfqpoint{1.876062in}{3.338365in}}{\pgfqpoint{1.881886in}{3.344189in}}%
\pgfpathcurveto{\pgfqpoint{1.887710in}{3.350013in}}{\pgfqpoint{1.890982in}{3.357913in}}{\pgfqpoint{1.890982in}{3.366149in}}%
\pgfpathcurveto{\pgfqpoint{1.890982in}{3.374386in}}{\pgfqpoint{1.887710in}{3.382286in}}{\pgfqpoint{1.881886in}{3.388110in}}%
\pgfpathcurveto{\pgfqpoint{1.876062in}{3.393934in}}{\pgfqpoint{1.868162in}{3.397206in}}{\pgfqpoint{1.859926in}{3.397206in}}%
\pgfpathcurveto{\pgfqpoint{1.851689in}{3.397206in}}{\pgfqpoint{1.843789in}{3.393934in}}{\pgfqpoint{1.837965in}{3.388110in}}%
\pgfpathcurveto{\pgfqpoint{1.832141in}{3.382286in}}{\pgfqpoint{1.828869in}{3.374386in}}{\pgfqpoint{1.828869in}{3.366149in}}%
\pgfpathcurveto{\pgfqpoint{1.828869in}{3.357913in}}{\pgfqpoint{1.832141in}{3.350013in}}{\pgfqpoint{1.837965in}{3.344189in}}%
\pgfpathcurveto{\pgfqpoint{1.843789in}{3.338365in}}{\pgfqpoint{1.851689in}{3.335093in}}{\pgfqpoint{1.859926in}{3.335093in}}%
\pgfpathclose%
\pgfusepath{stroke,fill}%
\end{pgfscope}%
\begin{pgfscope}%
\pgfpathrectangle{\pgfqpoint{0.100000in}{0.212622in}}{\pgfqpoint{3.696000in}{3.696000in}}%
\pgfusepath{clip}%
\pgfsetbuttcap%
\pgfsetroundjoin%
\definecolor{currentfill}{rgb}{0.121569,0.466667,0.705882}%
\pgfsetfillcolor{currentfill}%
\pgfsetfillopacity{0.302933}%
\pgfsetlinewidth{1.003750pt}%
\definecolor{currentstroke}{rgb}{0.121569,0.466667,0.705882}%
\pgfsetstrokecolor{currentstroke}%
\pgfsetstrokeopacity{0.302933}%
\pgfsetdash{}{0pt}%
\pgfpathmoveto{\pgfqpoint{1.895182in}{3.316337in}}%
\pgfpathcurveto{\pgfqpoint{1.903418in}{3.316337in}}{\pgfqpoint{1.911318in}{3.319609in}}{\pgfqpoint{1.917142in}{3.325433in}}%
\pgfpathcurveto{\pgfqpoint{1.922966in}{3.331257in}}{\pgfqpoint{1.926238in}{3.339157in}}{\pgfqpoint{1.926238in}{3.347394in}}%
\pgfpathcurveto{\pgfqpoint{1.926238in}{3.355630in}}{\pgfqpoint{1.922966in}{3.363530in}}{\pgfqpoint{1.917142in}{3.369354in}}%
\pgfpathcurveto{\pgfqpoint{1.911318in}{3.375178in}}{\pgfqpoint{1.903418in}{3.378450in}}{\pgfqpoint{1.895182in}{3.378450in}}%
\pgfpathcurveto{\pgfqpoint{1.886946in}{3.378450in}}{\pgfqpoint{1.879046in}{3.375178in}}{\pgfqpoint{1.873222in}{3.369354in}}%
\pgfpathcurveto{\pgfqpoint{1.867398in}{3.363530in}}{\pgfqpoint{1.864125in}{3.355630in}}{\pgfqpoint{1.864125in}{3.347394in}}%
\pgfpathcurveto{\pgfqpoint{1.864125in}{3.339157in}}{\pgfqpoint{1.867398in}{3.331257in}}{\pgfqpoint{1.873222in}{3.325433in}}%
\pgfpathcurveto{\pgfqpoint{1.879046in}{3.319609in}}{\pgfqpoint{1.886946in}{3.316337in}}{\pgfqpoint{1.895182in}{3.316337in}}%
\pgfpathclose%
\pgfusepath{stroke,fill}%
\end{pgfscope}%
\begin{pgfscope}%
\pgfpathrectangle{\pgfqpoint{0.100000in}{0.212622in}}{\pgfqpoint{3.696000in}{3.696000in}}%
\pgfusepath{clip}%
\pgfsetbuttcap%
\pgfsetroundjoin%
\definecolor{currentfill}{rgb}{0.121569,0.466667,0.705882}%
\pgfsetfillcolor{currentfill}%
\pgfsetfillopacity{0.303175}%
\pgfsetlinewidth{1.003750pt}%
\definecolor{currentstroke}{rgb}{0.121569,0.466667,0.705882}%
\pgfsetstrokecolor{currentstroke}%
\pgfsetstrokeopacity{0.303175}%
\pgfsetdash{}{0pt}%
\pgfpathmoveto{\pgfqpoint{1.859175in}{3.333811in}}%
\pgfpathcurveto{\pgfqpoint{1.867411in}{3.333811in}}{\pgfqpoint{1.875311in}{3.337083in}}{\pgfqpoint{1.881135in}{3.342907in}}%
\pgfpathcurveto{\pgfqpoint{1.886959in}{3.348731in}}{\pgfqpoint{1.890231in}{3.356631in}}{\pgfqpoint{1.890231in}{3.364867in}}%
\pgfpathcurveto{\pgfqpoint{1.890231in}{3.373103in}}{\pgfqpoint{1.886959in}{3.381003in}}{\pgfqpoint{1.881135in}{3.386827in}}%
\pgfpathcurveto{\pgfqpoint{1.875311in}{3.392651in}}{\pgfqpoint{1.867411in}{3.395924in}}{\pgfqpoint{1.859175in}{3.395924in}}%
\pgfpathcurveto{\pgfqpoint{1.850938in}{3.395924in}}{\pgfqpoint{1.843038in}{3.392651in}}{\pgfqpoint{1.837214in}{3.386827in}}%
\pgfpathcurveto{\pgfqpoint{1.831390in}{3.381003in}}{\pgfqpoint{1.828118in}{3.373103in}}{\pgfqpoint{1.828118in}{3.364867in}}%
\pgfpathcurveto{\pgfqpoint{1.828118in}{3.356631in}}{\pgfqpoint{1.831390in}{3.348731in}}{\pgfqpoint{1.837214in}{3.342907in}}%
\pgfpathcurveto{\pgfqpoint{1.843038in}{3.337083in}}{\pgfqpoint{1.850938in}{3.333811in}}{\pgfqpoint{1.859175in}{3.333811in}}%
\pgfpathclose%
\pgfusepath{stroke,fill}%
\end{pgfscope}%
\begin{pgfscope}%
\pgfpathrectangle{\pgfqpoint{0.100000in}{0.212622in}}{\pgfqpoint{3.696000in}{3.696000in}}%
\pgfusepath{clip}%
\pgfsetbuttcap%
\pgfsetroundjoin%
\definecolor{currentfill}{rgb}{0.121569,0.466667,0.705882}%
\pgfsetfillcolor{currentfill}%
\pgfsetfillopacity{0.303388}%
\pgfsetlinewidth{1.003750pt}%
\definecolor{currentstroke}{rgb}{0.121569,0.466667,0.705882}%
\pgfsetstrokecolor{currentstroke}%
\pgfsetstrokeopacity{0.303388}%
\pgfsetdash{}{0pt}%
\pgfpathmoveto{\pgfqpoint{1.896786in}{3.312867in}}%
\pgfpathcurveto{\pgfqpoint{1.905023in}{3.312867in}}{\pgfqpoint{1.912923in}{3.316139in}}{\pgfqpoint{1.918747in}{3.321963in}}%
\pgfpathcurveto{\pgfqpoint{1.924571in}{3.327787in}}{\pgfqpoint{1.927843in}{3.335687in}}{\pgfqpoint{1.927843in}{3.343923in}}%
\pgfpathcurveto{\pgfqpoint{1.927843in}{3.352160in}}{\pgfqpoint{1.924571in}{3.360060in}}{\pgfqpoint{1.918747in}{3.365884in}}%
\pgfpathcurveto{\pgfqpoint{1.912923in}{3.371708in}}{\pgfqpoint{1.905023in}{3.374980in}}{\pgfqpoint{1.896786in}{3.374980in}}%
\pgfpathcurveto{\pgfqpoint{1.888550in}{3.374980in}}{\pgfqpoint{1.880650in}{3.371708in}}{\pgfqpoint{1.874826in}{3.365884in}}%
\pgfpathcurveto{\pgfqpoint{1.869002in}{3.360060in}}{\pgfqpoint{1.865730in}{3.352160in}}{\pgfqpoint{1.865730in}{3.343923in}}%
\pgfpathcurveto{\pgfqpoint{1.865730in}{3.335687in}}{\pgfqpoint{1.869002in}{3.327787in}}{\pgfqpoint{1.874826in}{3.321963in}}%
\pgfpathcurveto{\pgfqpoint{1.880650in}{3.316139in}}{\pgfqpoint{1.888550in}{3.312867in}}{\pgfqpoint{1.896786in}{3.312867in}}%
\pgfpathclose%
\pgfusepath{stroke,fill}%
\end{pgfscope}%
\begin{pgfscope}%
\pgfpathrectangle{\pgfqpoint{0.100000in}{0.212622in}}{\pgfqpoint{3.696000in}{3.696000in}}%
\pgfusepath{clip}%
\pgfsetbuttcap%
\pgfsetroundjoin%
\definecolor{currentfill}{rgb}{0.121569,0.466667,0.705882}%
\pgfsetfillcolor{currentfill}%
\pgfsetfillopacity{0.303637}%
\pgfsetlinewidth{1.003750pt}%
\definecolor{currentstroke}{rgb}{0.121569,0.466667,0.705882}%
\pgfsetstrokecolor{currentstroke}%
\pgfsetstrokeopacity{0.303637}%
\pgfsetdash{}{0pt}%
\pgfpathmoveto{\pgfqpoint{1.857719in}{3.331495in}}%
\pgfpathcurveto{\pgfqpoint{1.865955in}{3.331495in}}{\pgfqpoint{1.873855in}{3.334767in}}{\pgfqpoint{1.879679in}{3.340591in}}%
\pgfpathcurveto{\pgfqpoint{1.885503in}{3.346415in}}{\pgfqpoint{1.888775in}{3.354315in}}{\pgfqpoint{1.888775in}{3.362552in}}%
\pgfpathcurveto{\pgfqpoint{1.888775in}{3.370788in}}{\pgfqpoint{1.885503in}{3.378688in}}{\pgfqpoint{1.879679in}{3.384512in}}%
\pgfpathcurveto{\pgfqpoint{1.873855in}{3.390336in}}{\pgfqpoint{1.865955in}{3.393608in}}{\pgfqpoint{1.857719in}{3.393608in}}%
\pgfpathcurveto{\pgfqpoint{1.849483in}{3.393608in}}{\pgfqpoint{1.841583in}{3.390336in}}{\pgfqpoint{1.835759in}{3.384512in}}%
\pgfpathcurveto{\pgfqpoint{1.829935in}{3.378688in}}{\pgfqpoint{1.826662in}{3.370788in}}{\pgfqpoint{1.826662in}{3.362552in}}%
\pgfpathcurveto{\pgfqpoint{1.826662in}{3.354315in}}{\pgfqpoint{1.829935in}{3.346415in}}{\pgfqpoint{1.835759in}{3.340591in}}%
\pgfpathcurveto{\pgfqpoint{1.841583in}{3.334767in}}{\pgfqpoint{1.849483in}{3.331495in}}{\pgfqpoint{1.857719in}{3.331495in}}%
\pgfpathclose%
\pgfusepath{stroke,fill}%
\end{pgfscope}%
\begin{pgfscope}%
\pgfpathrectangle{\pgfqpoint{0.100000in}{0.212622in}}{\pgfqpoint{3.696000in}{3.696000in}}%
\pgfusepath{clip}%
\pgfsetbuttcap%
\pgfsetroundjoin%
\definecolor{currentfill}{rgb}{0.121569,0.466667,0.705882}%
\pgfsetfillcolor{currentfill}%
\pgfsetfillopacity{0.304160}%
\pgfsetlinewidth{1.003750pt}%
\definecolor{currentstroke}{rgb}{0.121569,0.466667,0.705882}%
\pgfsetstrokecolor{currentstroke}%
\pgfsetstrokeopacity{0.304160}%
\pgfsetdash{}{0pt}%
\pgfpathmoveto{\pgfqpoint{1.898439in}{3.309433in}}%
\pgfpathcurveto{\pgfqpoint{1.906676in}{3.309433in}}{\pgfqpoint{1.914576in}{3.312705in}}{\pgfqpoint{1.920400in}{3.318529in}}%
\pgfpathcurveto{\pgfqpoint{1.926223in}{3.324353in}}{\pgfqpoint{1.929496in}{3.332253in}}{\pgfqpoint{1.929496in}{3.340490in}}%
\pgfpathcurveto{\pgfqpoint{1.929496in}{3.348726in}}{\pgfqpoint{1.926223in}{3.356626in}}{\pgfqpoint{1.920400in}{3.362450in}}%
\pgfpathcurveto{\pgfqpoint{1.914576in}{3.368274in}}{\pgfqpoint{1.906676in}{3.371546in}}{\pgfqpoint{1.898439in}{3.371546in}}%
\pgfpathcurveto{\pgfqpoint{1.890203in}{3.371546in}}{\pgfqpoint{1.882303in}{3.368274in}}{\pgfqpoint{1.876479in}{3.362450in}}%
\pgfpathcurveto{\pgfqpoint{1.870655in}{3.356626in}}{\pgfqpoint{1.867383in}{3.348726in}}{\pgfqpoint{1.867383in}{3.340490in}}%
\pgfpathcurveto{\pgfqpoint{1.867383in}{3.332253in}}{\pgfqpoint{1.870655in}{3.324353in}}{\pgfqpoint{1.876479in}{3.318529in}}%
\pgfpathcurveto{\pgfqpoint{1.882303in}{3.312705in}}{\pgfqpoint{1.890203in}{3.309433in}}{\pgfqpoint{1.898439in}{3.309433in}}%
\pgfpathclose%
\pgfusepath{stroke,fill}%
\end{pgfscope}%
\begin{pgfscope}%
\pgfpathrectangle{\pgfqpoint{0.100000in}{0.212622in}}{\pgfqpoint{3.696000in}{3.696000in}}%
\pgfusepath{clip}%
\pgfsetbuttcap%
\pgfsetroundjoin%
\definecolor{currentfill}{rgb}{0.121569,0.466667,0.705882}%
\pgfsetfillcolor{currentfill}%
\pgfsetfillopacity{0.304477}%
\pgfsetlinewidth{1.003750pt}%
\definecolor{currentstroke}{rgb}{0.121569,0.466667,0.705882}%
\pgfsetstrokecolor{currentstroke}%
\pgfsetstrokeopacity{0.304477}%
\pgfsetdash{}{0pt}%
\pgfpathmoveto{\pgfqpoint{1.855165in}{3.327145in}}%
\pgfpathcurveto{\pgfqpoint{1.863401in}{3.327145in}}{\pgfqpoint{1.871302in}{3.330417in}}{\pgfqpoint{1.877125in}{3.336241in}}%
\pgfpathcurveto{\pgfqpoint{1.882949in}{3.342065in}}{\pgfqpoint{1.886222in}{3.349965in}}{\pgfqpoint{1.886222in}{3.358201in}}%
\pgfpathcurveto{\pgfqpoint{1.886222in}{3.366437in}}{\pgfqpoint{1.882949in}{3.374337in}}{\pgfqpoint{1.877125in}{3.380161in}}%
\pgfpathcurveto{\pgfqpoint{1.871302in}{3.385985in}}{\pgfqpoint{1.863401in}{3.389258in}}{\pgfqpoint{1.855165in}{3.389258in}}%
\pgfpathcurveto{\pgfqpoint{1.846929in}{3.389258in}}{\pgfqpoint{1.839029in}{3.385985in}}{\pgfqpoint{1.833205in}{3.380161in}}%
\pgfpathcurveto{\pgfqpoint{1.827381in}{3.374337in}}{\pgfqpoint{1.824109in}{3.366437in}}{\pgfqpoint{1.824109in}{3.358201in}}%
\pgfpathcurveto{\pgfqpoint{1.824109in}{3.349965in}}{\pgfqpoint{1.827381in}{3.342065in}}{\pgfqpoint{1.833205in}{3.336241in}}%
\pgfpathcurveto{\pgfqpoint{1.839029in}{3.330417in}}{\pgfqpoint{1.846929in}{3.327145in}}{\pgfqpoint{1.855165in}{3.327145in}}%
\pgfpathclose%
\pgfusepath{stroke,fill}%
\end{pgfscope}%
\begin{pgfscope}%
\pgfpathrectangle{\pgfqpoint{0.100000in}{0.212622in}}{\pgfqpoint{3.696000in}{3.696000in}}%
\pgfusepath{clip}%
\pgfsetbuttcap%
\pgfsetroundjoin%
\definecolor{currentfill}{rgb}{0.121569,0.466667,0.705882}%
\pgfsetfillcolor{currentfill}%
\pgfsetfillopacity{0.304871}%
\pgfsetlinewidth{1.003750pt}%
\definecolor{currentstroke}{rgb}{0.121569,0.466667,0.705882}%
\pgfsetstrokecolor{currentstroke}%
\pgfsetstrokeopacity{0.304871}%
\pgfsetdash{}{0pt}%
\pgfpathmoveto{\pgfqpoint{1.900914in}{3.303623in}}%
\pgfpathcurveto{\pgfqpoint{1.909151in}{3.303623in}}{\pgfqpoint{1.917051in}{3.306896in}}{\pgfqpoint{1.922875in}{3.312720in}}%
\pgfpathcurveto{\pgfqpoint{1.928699in}{3.318544in}}{\pgfqpoint{1.931971in}{3.326444in}}{\pgfqpoint{1.931971in}{3.334680in}}%
\pgfpathcurveto{\pgfqpoint{1.931971in}{3.342916in}}{\pgfqpoint{1.928699in}{3.350816in}}{\pgfqpoint{1.922875in}{3.356640in}}%
\pgfpathcurveto{\pgfqpoint{1.917051in}{3.362464in}}{\pgfqpoint{1.909151in}{3.365736in}}{\pgfqpoint{1.900914in}{3.365736in}}%
\pgfpathcurveto{\pgfqpoint{1.892678in}{3.365736in}}{\pgfqpoint{1.884778in}{3.362464in}}{\pgfqpoint{1.878954in}{3.356640in}}%
\pgfpathcurveto{\pgfqpoint{1.873130in}{3.350816in}}{\pgfqpoint{1.869858in}{3.342916in}}{\pgfqpoint{1.869858in}{3.334680in}}%
\pgfpathcurveto{\pgfqpoint{1.869858in}{3.326444in}}{\pgfqpoint{1.873130in}{3.318544in}}{\pgfqpoint{1.878954in}{3.312720in}}%
\pgfpathcurveto{\pgfqpoint{1.884778in}{3.306896in}}{\pgfqpoint{1.892678in}{3.303623in}}{\pgfqpoint{1.900914in}{3.303623in}}%
\pgfpathclose%
\pgfusepath{stroke,fill}%
\end{pgfscope}%
\begin{pgfscope}%
\pgfpathrectangle{\pgfqpoint{0.100000in}{0.212622in}}{\pgfqpoint{3.696000in}{3.696000in}}%
\pgfusepath{clip}%
\pgfsetbuttcap%
\pgfsetroundjoin%
\definecolor{currentfill}{rgb}{0.121569,0.466667,0.705882}%
\pgfsetfillcolor{currentfill}%
\pgfsetfillopacity{0.305277}%
\pgfsetlinewidth{1.003750pt}%
\definecolor{currentstroke}{rgb}{0.121569,0.466667,0.705882}%
\pgfsetstrokecolor{currentstroke}%
\pgfsetstrokeopacity{0.305277}%
\pgfsetdash{}{0pt}%
\pgfpathmoveto{\pgfqpoint{1.853075in}{3.322973in}}%
\pgfpathcurveto{\pgfqpoint{1.861311in}{3.322973in}}{\pgfqpoint{1.869211in}{3.326245in}}{\pgfqpoint{1.875035in}{3.332069in}}%
\pgfpathcurveto{\pgfqpoint{1.880859in}{3.337893in}}{\pgfqpoint{1.884132in}{3.345793in}}{\pgfqpoint{1.884132in}{3.354030in}}%
\pgfpathcurveto{\pgfqpoint{1.884132in}{3.362266in}}{\pgfqpoint{1.880859in}{3.370166in}}{\pgfqpoint{1.875035in}{3.375990in}}%
\pgfpathcurveto{\pgfqpoint{1.869211in}{3.381814in}}{\pgfqpoint{1.861311in}{3.385086in}}{\pgfqpoint{1.853075in}{3.385086in}}%
\pgfpathcurveto{\pgfqpoint{1.844839in}{3.385086in}}{\pgfqpoint{1.836939in}{3.381814in}}{\pgfqpoint{1.831115in}{3.375990in}}%
\pgfpathcurveto{\pgfqpoint{1.825291in}{3.370166in}}{\pgfqpoint{1.822019in}{3.362266in}}{\pgfqpoint{1.822019in}{3.354030in}}%
\pgfpathcurveto{\pgfqpoint{1.822019in}{3.345793in}}{\pgfqpoint{1.825291in}{3.337893in}}{\pgfqpoint{1.831115in}{3.332069in}}%
\pgfpathcurveto{\pgfqpoint{1.836939in}{3.326245in}}{\pgfqpoint{1.844839in}{3.322973in}}{\pgfqpoint{1.853075in}{3.322973in}}%
\pgfpathclose%
\pgfusepath{stroke,fill}%
\end{pgfscope}%
\begin{pgfscope}%
\pgfpathrectangle{\pgfqpoint{0.100000in}{0.212622in}}{\pgfqpoint{3.696000in}{3.696000in}}%
\pgfusepath{clip}%
\pgfsetbuttcap%
\pgfsetroundjoin%
\definecolor{currentfill}{rgb}{0.121569,0.466667,0.705882}%
\pgfsetfillcolor{currentfill}%
\pgfsetfillopacity{0.305581}%
\pgfsetlinewidth{1.003750pt}%
\definecolor{currentstroke}{rgb}{0.121569,0.466667,0.705882}%
\pgfsetstrokecolor{currentstroke}%
\pgfsetstrokeopacity{0.305581}%
\pgfsetdash{}{0pt}%
\pgfpathmoveto{\pgfqpoint{1.852067in}{3.321370in}}%
\pgfpathcurveto{\pgfqpoint{1.860304in}{3.321370in}}{\pgfqpoint{1.868204in}{3.324643in}}{\pgfqpoint{1.874028in}{3.330467in}}%
\pgfpathcurveto{\pgfqpoint{1.879852in}{3.336291in}}{\pgfqpoint{1.883124in}{3.344191in}}{\pgfqpoint{1.883124in}{3.352427in}}%
\pgfpathcurveto{\pgfqpoint{1.883124in}{3.360663in}}{\pgfqpoint{1.879852in}{3.368563in}}{\pgfqpoint{1.874028in}{3.374387in}}%
\pgfpathcurveto{\pgfqpoint{1.868204in}{3.380211in}}{\pgfqpoint{1.860304in}{3.383483in}}{\pgfqpoint{1.852067in}{3.383483in}}%
\pgfpathcurveto{\pgfqpoint{1.843831in}{3.383483in}}{\pgfqpoint{1.835931in}{3.380211in}}{\pgfqpoint{1.830107in}{3.374387in}}%
\pgfpathcurveto{\pgfqpoint{1.824283in}{3.368563in}}{\pgfqpoint{1.821011in}{3.360663in}}{\pgfqpoint{1.821011in}{3.352427in}}%
\pgfpathcurveto{\pgfqpoint{1.821011in}{3.344191in}}{\pgfqpoint{1.824283in}{3.336291in}}{\pgfqpoint{1.830107in}{3.330467in}}%
\pgfpathcurveto{\pgfqpoint{1.835931in}{3.324643in}}{\pgfqpoint{1.843831in}{3.321370in}}{\pgfqpoint{1.852067in}{3.321370in}}%
\pgfpathclose%
\pgfusepath{stroke,fill}%
\end{pgfscope}%
\begin{pgfscope}%
\pgfpathrectangle{\pgfqpoint{0.100000in}{0.212622in}}{\pgfqpoint{3.696000in}{3.696000in}}%
\pgfusepath{clip}%
\pgfsetbuttcap%
\pgfsetroundjoin%
\definecolor{currentfill}{rgb}{0.121569,0.466667,0.705882}%
\pgfsetfillcolor{currentfill}%
\pgfsetfillopacity{0.305789}%
\pgfsetlinewidth{1.003750pt}%
\definecolor{currentstroke}{rgb}{0.121569,0.466667,0.705882}%
\pgfsetstrokecolor{currentstroke}%
\pgfsetstrokeopacity{0.305789}%
\pgfsetdash{}{0pt}%
\pgfpathmoveto{\pgfqpoint{1.851518in}{3.320312in}}%
\pgfpathcurveto{\pgfqpoint{1.859754in}{3.320312in}}{\pgfqpoint{1.867654in}{3.323585in}}{\pgfqpoint{1.873478in}{3.329409in}}%
\pgfpathcurveto{\pgfqpoint{1.879302in}{3.335233in}}{\pgfqpoint{1.882574in}{3.343133in}}{\pgfqpoint{1.882574in}{3.351369in}}%
\pgfpathcurveto{\pgfqpoint{1.882574in}{3.359605in}}{\pgfqpoint{1.879302in}{3.367505in}}{\pgfqpoint{1.873478in}{3.373329in}}%
\pgfpathcurveto{\pgfqpoint{1.867654in}{3.379153in}}{\pgfqpoint{1.859754in}{3.382425in}}{\pgfqpoint{1.851518in}{3.382425in}}%
\pgfpathcurveto{\pgfqpoint{1.843282in}{3.382425in}}{\pgfqpoint{1.835382in}{3.379153in}}{\pgfqpoint{1.829558in}{3.373329in}}%
\pgfpathcurveto{\pgfqpoint{1.823734in}{3.367505in}}{\pgfqpoint{1.820461in}{3.359605in}}{\pgfqpoint{1.820461in}{3.351369in}}%
\pgfpathcurveto{\pgfqpoint{1.820461in}{3.343133in}}{\pgfqpoint{1.823734in}{3.335233in}}{\pgfqpoint{1.829558in}{3.329409in}}%
\pgfpathcurveto{\pgfqpoint{1.835382in}{3.323585in}}{\pgfqpoint{1.843282in}{3.320312in}}{\pgfqpoint{1.851518in}{3.320312in}}%
\pgfpathclose%
\pgfusepath{stroke,fill}%
\end{pgfscope}%
\begin{pgfscope}%
\pgfpathrectangle{\pgfqpoint{0.100000in}{0.212622in}}{\pgfqpoint{3.696000in}{3.696000in}}%
\pgfusepath{clip}%
\pgfsetbuttcap%
\pgfsetroundjoin%
\definecolor{currentfill}{rgb}{0.121569,0.466667,0.705882}%
\pgfsetfillcolor{currentfill}%
\pgfsetfillopacity{0.305936}%
\pgfsetlinewidth{1.003750pt}%
\definecolor{currentstroke}{rgb}{0.121569,0.466667,0.705882}%
\pgfsetstrokecolor{currentstroke}%
\pgfsetstrokeopacity{0.305936}%
\pgfsetdash{}{0pt}%
\pgfpathmoveto{\pgfqpoint{1.903146in}{3.297662in}}%
\pgfpathcurveto{\pgfqpoint{1.911382in}{3.297662in}}{\pgfqpoint{1.919282in}{3.300934in}}{\pgfqpoint{1.925106in}{3.306758in}}%
\pgfpathcurveto{\pgfqpoint{1.930930in}{3.312582in}}{\pgfqpoint{1.934202in}{3.320482in}}{\pgfqpoint{1.934202in}{3.328718in}}%
\pgfpathcurveto{\pgfqpoint{1.934202in}{3.336954in}}{\pgfqpoint{1.930930in}{3.344854in}}{\pgfqpoint{1.925106in}{3.350678in}}%
\pgfpathcurveto{\pgfqpoint{1.919282in}{3.356502in}}{\pgfqpoint{1.911382in}{3.359775in}}{\pgfqpoint{1.903146in}{3.359775in}}%
\pgfpathcurveto{\pgfqpoint{1.894909in}{3.359775in}}{\pgfqpoint{1.887009in}{3.356502in}}{\pgfqpoint{1.881185in}{3.350678in}}%
\pgfpathcurveto{\pgfqpoint{1.875361in}{3.344854in}}{\pgfqpoint{1.872089in}{3.336954in}}{\pgfqpoint{1.872089in}{3.328718in}}%
\pgfpathcurveto{\pgfqpoint{1.872089in}{3.320482in}}{\pgfqpoint{1.875361in}{3.312582in}}{\pgfqpoint{1.881185in}{3.306758in}}%
\pgfpathcurveto{\pgfqpoint{1.887009in}{3.300934in}}{\pgfqpoint{1.894909in}{3.297662in}}{\pgfqpoint{1.903146in}{3.297662in}}%
\pgfpathclose%
\pgfusepath{stroke,fill}%
\end{pgfscope}%
\begin{pgfscope}%
\pgfpathrectangle{\pgfqpoint{0.100000in}{0.212622in}}{\pgfqpoint{3.696000in}{3.696000in}}%
\pgfusepath{clip}%
\pgfsetbuttcap%
\pgfsetroundjoin%
\definecolor{currentfill}{rgb}{0.121569,0.466667,0.705882}%
\pgfsetfillcolor{currentfill}%
\pgfsetfillopacity{0.306159}%
\pgfsetlinewidth{1.003750pt}%
\definecolor{currentstroke}{rgb}{0.121569,0.466667,0.705882}%
\pgfsetstrokecolor{currentstroke}%
\pgfsetstrokeopacity{0.306159}%
\pgfsetdash{}{0pt}%
\pgfpathmoveto{\pgfqpoint{1.850460in}{3.318434in}}%
\pgfpathcurveto{\pgfqpoint{1.858697in}{3.318434in}}{\pgfqpoint{1.866597in}{3.321706in}}{\pgfqpoint{1.872421in}{3.327530in}}%
\pgfpathcurveto{\pgfqpoint{1.878245in}{3.333354in}}{\pgfqpoint{1.881517in}{3.341254in}}{\pgfqpoint{1.881517in}{3.349491in}}%
\pgfpathcurveto{\pgfqpoint{1.881517in}{3.357727in}}{\pgfqpoint{1.878245in}{3.365627in}}{\pgfqpoint{1.872421in}{3.371451in}}%
\pgfpathcurveto{\pgfqpoint{1.866597in}{3.377275in}}{\pgfqpoint{1.858697in}{3.380547in}}{\pgfqpoint{1.850460in}{3.380547in}}%
\pgfpathcurveto{\pgfqpoint{1.842224in}{3.380547in}}{\pgfqpoint{1.834324in}{3.377275in}}{\pgfqpoint{1.828500in}{3.371451in}}%
\pgfpathcurveto{\pgfqpoint{1.822676in}{3.365627in}}{\pgfqpoint{1.819404in}{3.357727in}}{\pgfqpoint{1.819404in}{3.349491in}}%
\pgfpathcurveto{\pgfqpoint{1.819404in}{3.341254in}}{\pgfqpoint{1.822676in}{3.333354in}}{\pgfqpoint{1.828500in}{3.327530in}}%
\pgfpathcurveto{\pgfqpoint{1.834324in}{3.321706in}}{\pgfqpoint{1.842224in}{3.318434in}}{\pgfqpoint{1.850460in}{3.318434in}}%
\pgfpathclose%
\pgfusepath{stroke,fill}%
\end{pgfscope}%
\begin{pgfscope}%
\pgfpathrectangle{\pgfqpoint{0.100000in}{0.212622in}}{\pgfqpoint{3.696000in}{3.696000in}}%
\pgfusepath{clip}%
\pgfsetbuttcap%
\pgfsetroundjoin%
\definecolor{currentfill}{rgb}{0.121569,0.466667,0.705882}%
\pgfsetfillcolor{currentfill}%
\pgfsetfillopacity{0.306758}%
\pgfsetlinewidth{1.003750pt}%
\definecolor{currentstroke}{rgb}{0.121569,0.466667,0.705882}%
\pgfsetstrokecolor{currentstroke}%
\pgfsetstrokeopacity{0.306758}%
\pgfsetdash{}{0pt}%
\pgfpathmoveto{\pgfqpoint{1.848354in}{3.314986in}}%
\pgfpathcurveto{\pgfqpoint{1.856590in}{3.314986in}}{\pgfqpoint{1.864490in}{3.318258in}}{\pgfqpoint{1.870314in}{3.324082in}}%
\pgfpathcurveto{\pgfqpoint{1.876138in}{3.329906in}}{\pgfqpoint{1.879410in}{3.337806in}}{\pgfqpoint{1.879410in}{3.346042in}}%
\pgfpathcurveto{\pgfqpoint{1.879410in}{3.354279in}}{\pgfqpoint{1.876138in}{3.362179in}}{\pgfqpoint{1.870314in}{3.368002in}}%
\pgfpathcurveto{\pgfqpoint{1.864490in}{3.373826in}}{\pgfqpoint{1.856590in}{3.377099in}}{\pgfqpoint{1.848354in}{3.377099in}}%
\pgfpathcurveto{\pgfqpoint{1.840117in}{3.377099in}}{\pgfqpoint{1.832217in}{3.373826in}}{\pgfqpoint{1.826393in}{3.368002in}}%
\pgfpathcurveto{\pgfqpoint{1.820569in}{3.362179in}}{\pgfqpoint{1.817297in}{3.354279in}}{\pgfqpoint{1.817297in}{3.346042in}}%
\pgfpathcurveto{\pgfqpoint{1.817297in}{3.337806in}}{\pgfqpoint{1.820569in}{3.329906in}}{\pgfqpoint{1.826393in}{3.324082in}}%
\pgfpathcurveto{\pgfqpoint{1.832217in}{3.318258in}}{\pgfqpoint{1.840117in}{3.314986in}}{\pgfqpoint{1.848354in}{3.314986in}}%
\pgfpathclose%
\pgfusepath{stroke,fill}%
\end{pgfscope}%
\begin{pgfscope}%
\pgfpathrectangle{\pgfqpoint{0.100000in}{0.212622in}}{\pgfqpoint{3.696000in}{3.696000in}}%
\pgfusepath{clip}%
\pgfsetbuttcap%
\pgfsetroundjoin%
\definecolor{currentfill}{rgb}{0.121569,0.466667,0.705882}%
\pgfsetfillcolor{currentfill}%
\pgfsetfillopacity{0.307103}%
\pgfsetlinewidth{1.003750pt}%
\definecolor{currentstroke}{rgb}{0.121569,0.466667,0.705882}%
\pgfsetstrokecolor{currentstroke}%
\pgfsetstrokeopacity{0.307103}%
\pgfsetdash{}{0pt}%
\pgfpathmoveto{\pgfqpoint{1.905637in}{3.291170in}}%
\pgfpathcurveto{\pgfqpoint{1.913873in}{3.291170in}}{\pgfqpoint{1.921773in}{3.294442in}}{\pgfqpoint{1.927597in}{3.300266in}}%
\pgfpathcurveto{\pgfqpoint{1.933421in}{3.306090in}}{\pgfqpoint{1.936693in}{3.313990in}}{\pgfqpoint{1.936693in}{3.322227in}}%
\pgfpathcurveto{\pgfqpoint{1.936693in}{3.330463in}}{\pgfqpoint{1.933421in}{3.338363in}}{\pgfqpoint{1.927597in}{3.344187in}}%
\pgfpathcurveto{\pgfqpoint{1.921773in}{3.350011in}}{\pgfqpoint{1.913873in}{3.353283in}}{\pgfqpoint{1.905637in}{3.353283in}}%
\pgfpathcurveto{\pgfqpoint{1.897400in}{3.353283in}}{\pgfqpoint{1.889500in}{3.350011in}}{\pgfqpoint{1.883676in}{3.344187in}}%
\pgfpathcurveto{\pgfqpoint{1.877852in}{3.338363in}}{\pgfqpoint{1.874580in}{3.330463in}}{\pgfqpoint{1.874580in}{3.322227in}}%
\pgfpathcurveto{\pgfqpoint{1.874580in}{3.313990in}}{\pgfqpoint{1.877852in}{3.306090in}}{\pgfqpoint{1.883676in}{3.300266in}}%
\pgfpathcurveto{\pgfqpoint{1.889500in}{3.294442in}}{\pgfqpoint{1.897400in}{3.291170in}}{\pgfqpoint{1.905637in}{3.291170in}}%
\pgfpathclose%
\pgfusepath{stroke,fill}%
\end{pgfscope}%
\begin{pgfscope}%
\pgfpathrectangle{\pgfqpoint{0.100000in}{0.212622in}}{\pgfqpoint{3.696000in}{3.696000in}}%
\pgfusepath{clip}%
\pgfsetbuttcap%
\pgfsetroundjoin%
\definecolor{currentfill}{rgb}{0.121569,0.466667,0.705882}%
\pgfsetfillcolor{currentfill}%
\pgfsetfillopacity{0.307104}%
\pgfsetlinewidth{1.003750pt}%
\definecolor{currentstroke}{rgb}{0.121569,0.466667,0.705882}%
\pgfsetstrokecolor{currentstroke}%
\pgfsetstrokeopacity{0.307104}%
\pgfsetdash{}{0pt}%
\pgfpathmoveto{\pgfqpoint{1.847430in}{3.312997in}}%
\pgfpathcurveto{\pgfqpoint{1.855666in}{3.312997in}}{\pgfqpoint{1.863566in}{3.316269in}}{\pgfqpoint{1.869390in}{3.322093in}}%
\pgfpathcurveto{\pgfqpoint{1.875214in}{3.327917in}}{\pgfqpoint{1.878486in}{3.335817in}}{\pgfqpoint{1.878486in}{3.344053in}}%
\pgfpathcurveto{\pgfqpoint{1.878486in}{3.352290in}}{\pgfqpoint{1.875214in}{3.360190in}}{\pgfqpoint{1.869390in}{3.366014in}}%
\pgfpathcurveto{\pgfqpoint{1.863566in}{3.371838in}}{\pgfqpoint{1.855666in}{3.375110in}}{\pgfqpoint{1.847430in}{3.375110in}}%
\pgfpathcurveto{\pgfqpoint{1.839193in}{3.375110in}}{\pgfqpoint{1.831293in}{3.371838in}}{\pgfqpoint{1.825469in}{3.366014in}}%
\pgfpathcurveto{\pgfqpoint{1.819646in}{3.360190in}}{\pgfqpoint{1.816373in}{3.352290in}}{\pgfqpoint{1.816373in}{3.344053in}}%
\pgfpathcurveto{\pgfqpoint{1.816373in}{3.335817in}}{\pgfqpoint{1.819646in}{3.327917in}}{\pgfqpoint{1.825469in}{3.322093in}}%
\pgfpathcurveto{\pgfqpoint{1.831293in}{3.316269in}}{\pgfqpoint{1.839193in}{3.312997in}}{\pgfqpoint{1.847430in}{3.312997in}}%
\pgfpathclose%
\pgfusepath{stroke,fill}%
\end{pgfscope}%
\begin{pgfscope}%
\pgfpathrectangle{\pgfqpoint{0.100000in}{0.212622in}}{\pgfqpoint{3.696000in}{3.696000in}}%
\pgfusepath{clip}%
\pgfsetbuttcap%
\pgfsetroundjoin%
\definecolor{currentfill}{rgb}{0.121569,0.466667,0.705882}%
\pgfsetfillcolor{currentfill}%
\pgfsetfillopacity{0.307729}%
\pgfsetlinewidth{1.003750pt}%
\definecolor{currentstroke}{rgb}{0.121569,0.466667,0.705882}%
\pgfsetstrokecolor{currentstroke}%
\pgfsetstrokeopacity{0.307729}%
\pgfsetdash{}{0pt}%
\pgfpathmoveto{\pgfqpoint{1.845473in}{3.309705in}}%
\pgfpathcurveto{\pgfqpoint{1.853710in}{3.309705in}}{\pgfqpoint{1.861610in}{3.312978in}}{\pgfqpoint{1.867434in}{3.318802in}}%
\pgfpathcurveto{\pgfqpoint{1.873258in}{3.324626in}}{\pgfqpoint{1.876530in}{3.332526in}}{\pgfqpoint{1.876530in}{3.340762in}}%
\pgfpathcurveto{\pgfqpoint{1.876530in}{3.348998in}}{\pgfqpoint{1.873258in}{3.356898in}}{\pgfqpoint{1.867434in}{3.362722in}}%
\pgfpathcurveto{\pgfqpoint{1.861610in}{3.368546in}}{\pgfqpoint{1.853710in}{3.371818in}}{\pgfqpoint{1.845473in}{3.371818in}}%
\pgfpathcurveto{\pgfqpoint{1.837237in}{3.371818in}}{\pgfqpoint{1.829337in}{3.368546in}}{\pgfqpoint{1.823513in}{3.362722in}}%
\pgfpathcurveto{\pgfqpoint{1.817689in}{3.356898in}}{\pgfqpoint{1.814417in}{3.348998in}}{\pgfqpoint{1.814417in}{3.340762in}}%
\pgfpathcurveto{\pgfqpoint{1.814417in}{3.332526in}}{\pgfqpoint{1.817689in}{3.324626in}}{\pgfqpoint{1.823513in}{3.318802in}}%
\pgfpathcurveto{\pgfqpoint{1.829337in}{3.312978in}}{\pgfqpoint{1.837237in}{3.309705in}}{\pgfqpoint{1.845473in}{3.309705in}}%
\pgfpathclose%
\pgfusepath{stroke,fill}%
\end{pgfscope}%
\begin{pgfscope}%
\pgfpathrectangle{\pgfqpoint{0.100000in}{0.212622in}}{\pgfqpoint{3.696000in}{3.696000in}}%
\pgfusepath{clip}%
\pgfsetbuttcap%
\pgfsetroundjoin%
\definecolor{currentfill}{rgb}{0.121569,0.466667,0.705882}%
\pgfsetfillcolor{currentfill}%
\pgfsetfillopacity{0.308322}%
\pgfsetlinewidth{1.003750pt}%
\definecolor{currentstroke}{rgb}{0.121569,0.466667,0.705882}%
\pgfsetstrokecolor{currentstroke}%
\pgfsetstrokeopacity{0.308322}%
\pgfsetdash{}{0pt}%
\pgfpathmoveto{\pgfqpoint{1.908316in}{3.283640in}}%
\pgfpathcurveto{\pgfqpoint{1.916552in}{3.283640in}}{\pgfqpoint{1.924452in}{3.286913in}}{\pgfqpoint{1.930276in}{3.292736in}}%
\pgfpathcurveto{\pgfqpoint{1.936100in}{3.298560in}}{\pgfqpoint{1.939372in}{3.306460in}}{\pgfqpoint{1.939372in}{3.314697in}}%
\pgfpathcurveto{\pgfqpoint{1.939372in}{3.322933in}}{\pgfqpoint{1.936100in}{3.330833in}}{\pgfqpoint{1.930276in}{3.336657in}}%
\pgfpathcurveto{\pgfqpoint{1.924452in}{3.342481in}}{\pgfqpoint{1.916552in}{3.345753in}}{\pgfqpoint{1.908316in}{3.345753in}}%
\pgfpathcurveto{\pgfqpoint{1.900079in}{3.345753in}}{\pgfqpoint{1.892179in}{3.342481in}}{\pgfqpoint{1.886355in}{3.336657in}}%
\pgfpathcurveto{\pgfqpoint{1.880531in}{3.330833in}}{\pgfqpoint{1.877259in}{3.322933in}}{\pgfqpoint{1.877259in}{3.314697in}}%
\pgfpathcurveto{\pgfqpoint{1.877259in}{3.306460in}}{\pgfqpoint{1.880531in}{3.298560in}}{\pgfqpoint{1.886355in}{3.292736in}}%
\pgfpathcurveto{\pgfqpoint{1.892179in}{3.286913in}}{\pgfqpoint{1.900079in}{3.283640in}}{\pgfqpoint{1.908316in}{3.283640in}}%
\pgfpathclose%
\pgfusepath{stroke,fill}%
\end{pgfscope}%
\begin{pgfscope}%
\pgfpathrectangle{\pgfqpoint{0.100000in}{0.212622in}}{\pgfqpoint{3.696000in}{3.696000in}}%
\pgfusepath{clip}%
\pgfsetbuttcap%
\pgfsetroundjoin%
\definecolor{currentfill}{rgb}{0.121569,0.466667,0.705882}%
\pgfsetfillcolor{currentfill}%
\pgfsetfillopacity{0.308868}%
\pgfsetlinewidth{1.003750pt}%
\definecolor{currentstroke}{rgb}{0.121569,0.466667,0.705882}%
\pgfsetstrokecolor{currentstroke}%
\pgfsetstrokeopacity{0.308868}%
\pgfsetdash{}{0pt}%
\pgfpathmoveto{\pgfqpoint{1.841967in}{3.303660in}}%
\pgfpathcurveto{\pgfqpoint{1.850203in}{3.303660in}}{\pgfqpoint{1.858103in}{3.306933in}}{\pgfqpoint{1.863927in}{3.312757in}}%
\pgfpathcurveto{\pgfqpoint{1.869751in}{3.318581in}}{\pgfqpoint{1.873023in}{3.326481in}}{\pgfqpoint{1.873023in}{3.334717in}}%
\pgfpathcurveto{\pgfqpoint{1.873023in}{3.342953in}}{\pgfqpoint{1.869751in}{3.350853in}}{\pgfqpoint{1.863927in}{3.356677in}}%
\pgfpathcurveto{\pgfqpoint{1.858103in}{3.362501in}}{\pgfqpoint{1.850203in}{3.365773in}}{\pgfqpoint{1.841967in}{3.365773in}}%
\pgfpathcurveto{\pgfqpoint{1.833730in}{3.365773in}}{\pgfqpoint{1.825830in}{3.362501in}}{\pgfqpoint{1.820006in}{3.356677in}}%
\pgfpathcurveto{\pgfqpoint{1.814183in}{3.350853in}}{\pgfqpoint{1.810910in}{3.342953in}}{\pgfqpoint{1.810910in}{3.334717in}}%
\pgfpathcurveto{\pgfqpoint{1.810910in}{3.326481in}}{\pgfqpoint{1.814183in}{3.318581in}}{\pgfqpoint{1.820006in}{3.312757in}}%
\pgfpathcurveto{\pgfqpoint{1.825830in}{3.306933in}}{\pgfqpoint{1.833730in}{3.303660in}}{\pgfqpoint{1.841967in}{3.303660in}}%
\pgfpathclose%
\pgfusepath{stroke,fill}%
\end{pgfscope}%
\begin{pgfscope}%
\pgfpathrectangle{\pgfqpoint{0.100000in}{0.212622in}}{\pgfqpoint{3.696000in}{3.696000in}}%
\pgfusepath{clip}%
\pgfsetbuttcap%
\pgfsetroundjoin%
\definecolor{currentfill}{rgb}{0.121569,0.466667,0.705882}%
\pgfsetfillcolor{currentfill}%
\pgfsetfillopacity{0.309994}%
\pgfsetlinewidth{1.003750pt}%
\definecolor{currentstroke}{rgb}{0.121569,0.466667,0.705882}%
\pgfsetstrokecolor{currentstroke}%
\pgfsetstrokeopacity{0.309994}%
\pgfsetdash{}{0pt}%
\pgfpathmoveto{\pgfqpoint{1.839009in}{3.297816in}}%
\pgfpathcurveto{\pgfqpoint{1.847245in}{3.297816in}}{\pgfqpoint{1.855145in}{3.301089in}}{\pgfqpoint{1.860969in}{3.306913in}}%
\pgfpathcurveto{\pgfqpoint{1.866793in}{3.312736in}}{\pgfqpoint{1.870065in}{3.320636in}}{\pgfqpoint{1.870065in}{3.328873in}}%
\pgfpathcurveto{\pgfqpoint{1.870065in}{3.337109in}}{\pgfqpoint{1.866793in}{3.345009in}}{\pgfqpoint{1.860969in}{3.350833in}}%
\pgfpathcurveto{\pgfqpoint{1.855145in}{3.356657in}}{\pgfqpoint{1.847245in}{3.359929in}}{\pgfqpoint{1.839009in}{3.359929in}}%
\pgfpathcurveto{\pgfqpoint{1.830773in}{3.359929in}}{\pgfqpoint{1.822873in}{3.356657in}}{\pgfqpoint{1.817049in}{3.350833in}}%
\pgfpathcurveto{\pgfqpoint{1.811225in}{3.345009in}}{\pgfqpoint{1.807952in}{3.337109in}}{\pgfqpoint{1.807952in}{3.328873in}}%
\pgfpathcurveto{\pgfqpoint{1.807952in}{3.320636in}}{\pgfqpoint{1.811225in}{3.312736in}}{\pgfqpoint{1.817049in}{3.306913in}}%
\pgfpathcurveto{\pgfqpoint{1.822873in}{3.301089in}}{\pgfqpoint{1.830773in}{3.297816in}}{\pgfqpoint{1.839009in}{3.297816in}}%
\pgfpathclose%
\pgfusepath{stroke,fill}%
\end{pgfscope}%
\begin{pgfscope}%
\pgfpathrectangle{\pgfqpoint{0.100000in}{0.212622in}}{\pgfqpoint{3.696000in}{3.696000in}}%
\pgfusepath{clip}%
\pgfsetbuttcap%
\pgfsetroundjoin%
\definecolor{currentfill}{rgb}{0.121569,0.466667,0.705882}%
\pgfsetfillcolor{currentfill}%
\pgfsetfillopacity{0.310010}%
\pgfsetlinewidth{1.003750pt}%
\definecolor{currentstroke}{rgb}{0.121569,0.466667,0.705882}%
\pgfsetstrokecolor{currentstroke}%
\pgfsetstrokeopacity{0.310010}%
\pgfsetdash{}{0pt}%
\pgfpathmoveto{\pgfqpoint{1.910772in}{3.276190in}}%
\pgfpathcurveto{\pgfqpoint{1.919008in}{3.276190in}}{\pgfqpoint{1.926908in}{3.279463in}}{\pgfqpoint{1.932732in}{3.285287in}}%
\pgfpathcurveto{\pgfqpoint{1.938556in}{3.291111in}}{\pgfqpoint{1.941828in}{3.299011in}}{\pgfqpoint{1.941828in}{3.307247in}}%
\pgfpathcurveto{\pgfqpoint{1.941828in}{3.315483in}}{\pgfqpoint{1.938556in}{3.323383in}}{\pgfqpoint{1.932732in}{3.329207in}}%
\pgfpathcurveto{\pgfqpoint{1.926908in}{3.335031in}}{\pgfqpoint{1.919008in}{3.338303in}}{\pgfqpoint{1.910772in}{3.338303in}}%
\pgfpathcurveto{\pgfqpoint{1.902535in}{3.338303in}}{\pgfqpoint{1.894635in}{3.335031in}}{\pgfqpoint{1.888812in}{3.329207in}}%
\pgfpathcurveto{\pgfqpoint{1.882988in}{3.323383in}}{\pgfqpoint{1.879715in}{3.315483in}}{\pgfqpoint{1.879715in}{3.307247in}}%
\pgfpathcurveto{\pgfqpoint{1.879715in}{3.299011in}}{\pgfqpoint{1.882988in}{3.291111in}}{\pgfqpoint{1.888812in}{3.285287in}}%
\pgfpathcurveto{\pgfqpoint{1.894635in}{3.279463in}}{\pgfqpoint{1.902535in}{3.276190in}}{\pgfqpoint{1.910772in}{3.276190in}}%
\pgfpathclose%
\pgfusepath{stroke,fill}%
\end{pgfscope}%
\begin{pgfscope}%
\pgfpathrectangle{\pgfqpoint{0.100000in}{0.212622in}}{\pgfqpoint{3.696000in}{3.696000in}}%
\pgfusepath{clip}%
\pgfsetbuttcap%
\pgfsetroundjoin%
\definecolor{currentfill}{rgb}{0.121569,0.466667,0.705882}%
\pgfsetfillcolor{currentfill}%
\pgfsetfillopacity{0.310755}%
\pgfsetlinewidth{1.003750pt}%
\definecolor{currentstroke}{rgb}{0.121569,0.466667,0.705882}%
\pgfsetstrokecolor{currentstroke}%
\pgfsetstrokeopacity{0.310755}%
\pgfsetdash{}{0pt}%
\pgfpathmoveto{\pgfqpoint{1.836400in}{3.293651in}}%
\pgfpathcurveto{\pgfqpoint{1.844636in}{3.293651in}}{\pgfqpoint{1.852536in}{3.296924in}}{\pgfqpoint{1.858360in}{3.302747in}}%
\pgfpathcurveto{\pgfqpoint{1.864184in}{3.308571in}}{\pgfqpoint{1.867456in}{3.316471in}}{\pgfqpoint{1.867456in}{3.324708in}}%
\pgfpathcurveto{\pgfqpoint{1.867456in}{3.332944in}}{\pgfqpoint{1.864184in}{3.340844in}}{\pgfqpoint{1.858360in}{3.346668in}}%
\pgfpathcurveto{\pgfqpoint{1.852536in}{3.352492in}}{\pgfqpoint{1.844636in}{3.355764in}}{\pgfqpoint{1.836400in}{3.355764in}}%
\pgfpathcurveto{\pgfqpoint{1.828164in}{3.355764in}}{\pgfqpoint{1.820263in}{3.352492in}}{\pgfqpoint{1.814440in}{3.346668in}}%
\pgfpathcurveto{\pgfqpoint{1.808616in}{3.340844in}}{\pgfqpoint{1.805343in}{3.332944in}}{\pgfqpoint{1.805343in}{3.324708in}}%
\pgfpathcurveto{\pgfqpoint{1.805343in}{3.316471in}}{\pgfqpoint{1.808616in}{3.308571in}}{\pgfqpoint{1.814440in}{3.302747in}}%
\pgfpathcurveto{\pgfqpoint{1.820263in}{3.296924in}}{\pgfqpoint{1.828164in}{3.293651in}}{\pgfqpoint{1.836400in}{3.293651in}}%
\pgfpathclose%
\pgfusepath{stroke,fill}%
\end{pgfscope}%
\begin{pgfscope}%
\pgfpathrectangle{\pgfqpoint{0.100000in}{0.212622in}}{\pgfqpoint{3.696000in}{3.696000in}}%
\pgfusepath{clip}%
\pgfsetbuttcap%
\pgfsetroundjoin%
\definecolor{currentfill}{rgb}{0.121569,0.466667,0.705882}%
\pgfsetfillcolor{currentfill}%
\pgfsetfillopacity{0.311357}%
\pgfsetlinewidth{1.003750pt}%
\definecolor{currentstroke}{rgb}{0.121569,0.466667,0.705882}%
\pgfsetstrokecolor{currentstroke}%
\pgfsetstrokeopacity{0.311357}%
\pgfsetdash{}{0pt}%
\pgfpathmoveto{\pgfqpoint{1.834814in}{3.290463in}}%
\pgfpathcurveto{\pgfqpoint{1.843050in}{3.290463in}}{\pgfqpoint{1.850950in}{3.293735in}}{\pgfqpoint{1.856774in}{3.299559in}}%
\pgfpathcurveto{\pgfqpoint{1.862598in}{3.305383in}}{\pgfqpoint{1.865870in}{3.313283in}}{\pgfqpoint{1.865870in}{3.321519in}}%
\pgfpathcurveto{\pgfqpoint{1.865870in}{3.329756in}}{\pgfqpoint{1.862598in}{3.337656in}}{\pgfqpoint{1.856774in}{3.343480in}}%
\pgfpathcurveto{\pgfqpoint{1.850950in}{3.349303in}}{\pgfqpoint{1.843050in}{3.352576in}}{\pgfqpoint{1.834814in}{3.352576in}}%
\pgfpathcurveto{\pgfqpoint{1.826578in}{3.352576in}}{\pgfqpoint{1.818678in}{3.349303in}}{\pgfqpoint{1.812854in}{3.343480in}}%
\pgfpathcurveto{\pgfqpoint{1.807030in}{3.337656in}}{\pgfqpoint{1.803757in}{3.329756in}}{\pgfqpoint{1.803757in}{3.321519in}}%
\pgfpathcurveto{\pgfqpoint{1.803757in}{3.313283in}}{\pgfqpoint{1.807030in}{3.305383in}}{\pgfqpoint{1.812854in}{3.299559in}}%
\pgfpathcurveto{\pgfqpoint{1.818678in}{3.293735in}}{\pgfqpoint{1.826578in}{3.290463in}}{\pgfqpoint{1.834814in}{3.290463in}}%
\pgfpathclose%
\pgfusepath{stroke,fill}%
\end{pgfscope}%
\begin{pgfscope}%
\pgfpathrectangle{\pgfqpoint{0.100000in}{0.212622in}}{\pgfqpoint{3.696000in}{3.696000in}}%
\pgfusepath{clip}%
\pgfsetbuttcap%
\pgfsetroundjoin%
\definecolor{currentfill}{rgb}{0.121569,0.466667,0.705882}%
\pgfsetfillcolor{currentfill}%
\pgfsetfillopacity{0.311751}%
\pgfsetlinewidth{1.003750pt}%
\definecolor{currentstroke}{rgb}{0.121569,0.466667,0.705882}%
\pgfsetstrokecolor{currentstroke}%
\pgfsetstrokeopacity{0.311751}%
\pgfsetdash{}{0pt}%
\pgfpathmoveto{\pgfqpoint{1.914122in}{3.266545in}}%
\pgfpathcurveto{\pgfqpoint{1.922359in}{3.266545in}}{\pgfqpoint{1.930259in}{3.269818in}}{\pgfqpoint{1.936083in}{3.275641in}}%
\pgfpathcurveto{\pgfqpoint{1.941906in}{3.281465in}}{\pgfqpoint{1.945179in}{3.289365in}}{\pgfqpoint{1.945179in}{3.297602in}}%
\pgfpathcurveto{\pgfqpoint{1.945179in}{3.305838in}}{\pgfqpoint{1.941906in}{3.313738in}}{\pgfqpoint{1.936083in}{3.319562in}}%
\pgfpathcurveto{\pgfqpoint{1.930259in}{3.325386in}}{\pgfqpoint{1.922359in}{3.328658in}}{\pgfqpoint{1.914122in}{3.328658in}}%
\pgfpathcurveto{\pgfqpoint{1.905886in}{3.328658in}}{\pgfqpoint{1.897986in}{3.325386in}}{\pgfqpoint{1.892162in}{3.319562in}}%
\pgfpathcurveto{\pgfqpoint{1.886338in}{3.313738in}}{\pgfqpoint{1.883066in}{3.305838in}}{\pgfqpoint{1.883066in}{3.297602in}}%
\pgfpathcurveto{\pgfqpoint{1.883066in}{3.289365in}}{\pgfqpoint{1.886338in}{3.281465in}}{\pgfqpoint{1.892162in}{3.275641in}}%
\pgfpathcurveto{\pgfqpoint{1.897986in}{3.269818in}}{\pgfqpoint{1.905886in}{3.266545in}}{\pgfqpoint{1.914122in}{3.266545in}}%
\pgfpathclose%
\pgfusepath{stroke,fill}%
\end{pgfscope}%
\begin{pgfscope}%
\pgfpathrectangle{\pgfqpoint{0.100000in}{0.212622in}}{\pgfqpoint{3.696000in}{3.696000in}}%
\pgfusepath{clip}%
\pgfsetbuttcap%
\pgfsetroundjoin%
\definecolor{currentfill}{rgb}{0.121569,0.466667,0.705882}%
\pgfsetfillcolor{currentfill}%
\pgfsetfillopacity{0.311887}%
\pgfsetlinewidth{1.003750pt}%
\definecolor{currentstroke}{rgb}{0.121569,0.466667,0.705882}%
\pgfsetstrokecolor{currentstroke}%
\pgfsetstrokeopacity{0.311887}%
\pgfsetdash{}{0pt}%
\pgfpathmoveto{\pgfqpoint{1.833262in}{3.287794in}}%
\pgfpathcurveto{\pgfqpoint{1.841499in}{3.287794in}}{\pgfqpoint{1.849399in}{3.291066in}}{\pgfqpoint{1.855223in}{3.296890in}}%
\pgfpathcurveto{\pgfqpoint{1.861047in}{3.302714in}}{\pgfqpoint{1.864319in}{3.310614in}}{\pgfqpoint{1.864319in}{3.318850in}}%
\pgfpathcurveto{\pgfqpoint{1.864319in}{3.327086in}}{\pgfqpoint{1.861047in}{3.334986in}}{\pgfqpoint{1.855223in}{3.340810in}}%
\pgfpathcurveto{\pgfqpoint{1.849399in}{3.346634in}}{\pgfqpoint{1.841499in}{3.349907in}}{\pgfqpoint{1.833262in}{3.349907in}}%
\pgfpathcurveto{\pgfqpoint{1.825026in}{3.349907in}}{\pgfqpoint{1.817126in}{3.346634in}}{\pgfqpoint{1.811302in}{3.340810in}}%
\pgfpathcurveto{\pgfqpoint{1.805478in}{3.334986in}}{\pgfqpoint{1.802206in}{3.327086in}}{\pgfqpoint{1.802206in}{3.318850in}}%
\pgfpathcurveto{\pgfqpoint{1.802206in}{3.310614in}}{\pgfqpoint{1.805478in}{3.302714in}}{\pgfqpoint{1.811302in}{3.296890in}}%
\pgfpathcurveto{\pgfqpoint{1.817126in}{3.291066in}}{\pgfqpoint{1.825026in}{3.287794in}}{\pgfqpoint{1.833262in}{3.287794in}}%
\pgfpathclose%
\pgfusepath{stroke,fill}%
\end{pgfscope}%
\begin{pgfscope}%
\pgfpathrectangle{\pgfqpoint{0.100000in}{0.212622in}}{\pgfqpoint{3.696000in}{3.696000in}}%
\pgfusepath{clip}%
\pgfsetbuttcap%
\pgfsetroundjoin%
\definecolor{currentfill}{rgb}{0.121569,0.466667,0.705882}%
\pgfsetfillcolor{currentfill}%
\pgfsetfillopacity{0.312802}%
\pgfsetlinewidth{1.003750pt}%
\definecolor{currentstroke}{rgb}{0.121569,0.466667,0.705882}%
\pgfsetstrokecolor{currentstroke}%
\pgfsetstrokeopacity{0.312802}%
\pgfsetdash{}{0pt}%
\pgfpathmoveto{\pgfqpoint{1.830326in}{3.282902in}}%
\pgfpathcurveto{\pgfqpoint{1.838562in}{3.282902in}}{\pgfqpoint{1.846462in}{3.286175in}}{\pgfqpoint{1.852286in}{3.291999in}}%
\pgfpathcurveto{\pgfqpoint{1.858110in}{3.297822in}}{\pgfqpoint{1.861382in}{3.305723in}}{\pgfqpoint{1.861382in}{3.313959in}}%
\pgfpathcurveto{\pgfqpoint{1.861382in}{3.322195in}}{\pgfqpoint{1.858110in}{3.330095in}}{\pgfqpoint{1.852286in}{3.335919in}}%
\pgfpathcurveto{\pgfqpoint{1.846462in}{3.341743in}}{\pgfqpoint{1.838562in}{3.345015in}}{\pgfqpoint{1.830326in}{3.345015in}}%
\pgfpathcurveto{\pgfqpoint{1.822090in}{3.345015in}}{\pgfqpoint{1.814190in}{3.341743in}}{\pgfqpoint{1.808366in}{3.335919in}}%
\pgfpathcurveto{\pgfqpoint{1.802542in}{3.330095in}}{\pgfqpoint{1.799269in}{3.322195in}}{\pgfqpoint{1.799269in}{3.313959in}}%
\pgfpathcurveto{\pgfqpoint{1.799269in}{3.305723in}}{\pgfqpoint{1.802542in}{3.297822in}}{\pgfqpoint{1.808366in}{3.291999in}}%
\pgfpathcurveto{\pgfqpoint{1.814190in}{3.286175in}}{\pgfqpoint{1.822090in}{3.282902in}}{\pgfqpoint{1.830326in}{3.282902in}}%
\pgfpathclose%
\pgfusepath{stroke,fill}%
\end{pgfscope}%
\begin{pgfscope}%
\pgfpathrectangle{\pgfqpoint{0.100000in}{0.212622in}}{\pgfqpoint{3.696000in}{3.696000in}}%
\pgfusepath{clip}%
\pgfsetbuttcap%
\pgfsetroundjoin%
\definecolor{currentfill}{rgb}{0.121569,0.466667,0.705882}%
\pgfsetfillcolor{currentfill}%
\pgfsetfillopacity{0.312860}%
\pgfsetlinewidth{1.003750pt}%
\definecolor{currentstroke}{rgb}{0.121569,0.466667,0.705882}%
\pgfsetstrokecolor{currentstroke}%
\pgfsetstrokeopacity{0.312860}%
\pgfsetdash{}{0pt}%
\pgfpathmoveto{\pgfqpoint{1.915643in}{3.261431in}}%
\pgfpathcurveto{\pgfqpoint{1.923879in}{3.261431in}}{\pgfqpoint{1.931779in}{3.264704in}}{\pgfqpoint{1.937603in}{3.270528in}}%
\pgfpathcurveto{\pgfqpoint{1.943427in}{3.276352in}}{\pgfqpoint{1.946700in}{3.284252in}}{\pgfqpoint{1.946700in}{3.292488in}}%
\pgfpathcurveto{\pgfqpoint{1.946700in}{3.300724in}}{\pgfqpoint{1.943427in}{3.308624in}}{\pgfqpoint{1.937603in}{3.314448in}}%
\pgfpathcurveto{\pgfqpoint{1.931779in}{3.320272in}}{\pgfqpoint{1.923879in}{3.323544in}}{\pgfqpoint{1.915643in}{3.323544in}}%
\pgfpathcurveto{\pgfqpoint{1.907407in}{3.323544in}}{\pgfqpoint{1.899507in}{3.320272in}}{\pgfqpoint{1.893683in}{3.314448in}}%
\pgfpathcurveto{\pgfqpoint{1.887859in}{3.308624in}}{\pgfqpoint{1.884587in}{3.300724in}}{\pgfqpoint{1.884587in}{3.292488in}}%
\pgfpathcurveto{\pgfqpoint{1.884587in}{3.284252in}}{\pgfqpoint{1.887859in}{3.276352in}}{\pgfqpoint{1.893683in}{3.270528in}}%
\pgfpathcurveto{\pgfqpoint{1.899507in}{3.264704in}}{\pgfqpoint{1.907407in}{3.261431in}}{\pgfqpoint{1.915643in}{3.261431in}}%
\pgfpathclose%
\pgfusepath{stroke,fill}%
\end{pgfscope}%
\begin{pgfscope}%
\pgfpathrectangle{\pgfqpoint{0.100000in}{0.212622in}}{\pgfqpoint{3.696000in}{3.696000in}}%
\pgfusepath{clip}%
\pgfsetbuttcap%
\pgfsetroundjoin%
\definecolor{currentfill}{rgb}{0.121569,0.466667,0.705882}%
\pgfsetfillcolor{currentfill}%
\pgfsetfillopacity{0.313499}%
\pgfsetlinewidth{1.003750pt}%
\definecolor{currentstroke}{rgb}{0.121569,0.466667,0.705882}%
\pgfsetstrokecolor{currentstroke}%
\pgfsetstrokeopacity{0.313499}%
\pgfsetdash{}{0pt}%
\pgfpathmoveto{\pgfqpoint{1.916463in}{3.258715in}}%
\pgfpathcurveto{\pgfqpoint{1.924699in}{3.258715in}}{\pgfqpoint{1.932599in}{3.261987in}}{\pgfqpoint{1.938423in}{3.267811in}}%
\pgfpathcurveto{\pgfqpoint{1.944247in}{3.273635in}}{\pgfqpoint{1.947519in}{3.281535in}}{\pgfqpoint{1.947519in}{3.289771in}}%
\pgfpathcurveto{\pgfqpoint{1.947519in}{3.298007in}}{\pgfqpoint{1.944247in}{3.305908in}}{\pgfqpoint{1.938423in}{3.311731in}}%
\pgfpathcurveto{\pgfqpoint{1.932599in}{3.317555in}}{\pgfqpoint{1.924699in}{3.320828in}}{\pgfqpoint{1.916463in}{3.320828in}}%
\pgfpathcurveto{\pgfqpoint{1.908226in}{3.320828in}}{\pgfqpoint{1.900326in}{3.317555in}}{\pgfqpoint{1.894502in}{3.311731in}}%
\pgfpathcurveto{\pgfqpoint{1.888678in}{3.305908in}}{\pgfqpoint{1.885406in}{3.298007in}}{\pgfqpoint{1.885406in}{3.289771in}}%
\pgfpathcurveto{\pgfqpoint{1.885406in}{3.281535in}}{\pgfqpoint{1.888678in}{3.273635in}}{\pgfqpoint{1.894502in}{3.267811in}}%
\pgfpathcurveto{\pgfqpoint{1.900326in}{3.261987in}}{\pgfqpoint{1.908226in}{3.258715in}}{\pgfqpoint{1.916463in}{3.258715in}}%
\pgfpathclose%
\pgfusepath{stroke,fill}%
\end{pgfscope}%
\begin{pgfscope}%
\pgfpathrectangle{\pgfqpoint{0.100000in}{0.212622in}}{\pgfqpoint{3.696000in}{3.696000in}}%
\pgfusepath{clip}%
\pgfsetbuttcap%
\pgfsetroundjoin%
\definecolor{currentfill}{rgb}{0.121569,0.466667,0.705882}%
\pgfsetfillcolor{currentfill}%
\pgfsetfillopacity{0.313536}%
\pgfsetlinewidth{1.003750pt}%
\definecolor{currentstroke}{rgb}{0.121569,0.466667,0.705882}%
\pgfsetstrokecolor{currentstroke}%
\pgfsetstrokeopacity{0.313536}%
\pgfsetdash{}{0pt}%
\pgfpathmoveto{\pgfqpoint{1.828502in}{3.278851in}}%
\pgfpathcurveto{\pgfqpoint{1.836738in}{3.278851in}}{\pgfqpoint{1.844638in}{3.282123in}}{\pgfqpoint{1.850462in}{3.287947in}}%
\pgfpathcurveto{\pgfqpoint{1.856286in}{3.293771in}}{\pgfqpoint{1.859559in}{3.301671in}}{\pgfqpoint{1.859559in}{3.309907in}}%
\pgfpathcurveto{\pgfqpoint{1.859559in}{3.318143in}}{\pgfqpoint{1.856286in}{3.326044in}}{\pgfqpoint{1.850462in}{3.331867in}}%
\pgfpathcurveto{\pgfqpoint{1.844638in}{3.337691in}}{\pgfqpoint{1.836738in}{3.340964in}}{\pgfqpoint{1.828502in}{3.340964in}}%
\pgfpathcurveto{\pgfqpoint{1.820266in}{3.340964in}}{\pgfqpoint{1.812366in}{3.337691in}}{\pgfqpoint{1.806542in}{3.331867in}}%
\pgfpathcurveto{\pgfqpoint{1.800718in}{3.326044in}}{\pgfqpoint{1.797446in}{3.318143in}}{\pgfqpoint{1.797446in}{3.309907in}}%
\pgfpathcurveto{\pgfqpoint{1.797446in}{3.301671in}}{\pgfqpoint{1.800718in}{3.293771in}}{\pgfqpoint{1.806542in}{3.287947in}}%
\pgfpathcurveto{\pgfqpoint{1.812366in}{3.282123in}}{\pgfqpoint{1.820266in}{3.278851in}}{\pgfqpoint{1.828502in}{3.278851in}}%
\pgfpathclose%
\pgfusepath{stroke,fill}%
\end{pgfscope}%
\begin{pgfscope}%
\pgfpathrectangle{\pgfqpoint{0.100000in}{0.212622in}}{\pgfqpoint{3.696000in}{3.696000in}}%
\pgfusepath{clip}%
\pgfsetbuttcap%
\pgfsetroundjoin%
\definecolor{currentfill}{rgb}{0.121569,0.466667,0.705882}%
\pgfsetfillcolor{currentfill}%
\pgfsetfillopacity{0.313917}%
\pgfsetlinewidth{1.003750pt}%
\definecolor{currentstroke}{rgb}{0.121569,0.466667,0.705882}%
\pgfsetstrokecolor{currentstroke}%
\pgfsetstrokeopacity{0.313917}%
\pgfsetdash{}{0pt}%
\pgfpathmoveto{\pgfqpoint{1.827330in}{3.277025in}}%
\pgfpathcurveto{\pgfqpoint{1.835566in}{3.277025in}}{\pgfqpoint{1.843466in}{3.280297in}}{\pgfqpoint{1.849290in}{3.286121in}}%
\pgfpathcurveto{\pgfqpoint{1.855114in}{3.291945in}}{\pgfqpoint{1.858386in}{3.299845in}}{\pgfqpoint{1.858386in}{3.308081in}}%
\pgfpathcurveto{\pgfqpoint{1.858386in}{3.316318in}}{\pgfqpoint{1.855114in}{3.324218in}}{\pgfqpoint{1.849290in}{3.330042in}}%
\pgfpathcurveto{\pgfqpoint{1.843466in}{3.335866in}}{\pgfqpoint{1.835566in}{3.339138in}}{\pgfqpoint{1.827330in}{3.339138in}}%
\pgfpathcurveto{\pgfqpoint{1.819094in}{3.339138in}}{\pgfqpoint{1.811193in}{3.335866in}}{\pgfqpoint{1.805370in}{3.330042in}}%
\pgfpathcurveto{\pgfqpoint{1.799546in}{3.324218in}}{\pgfqpoint{1.796273in}{3.316318in}}{\pgfqpoint{1.796273in}{3.308081in}}%
\pgfpathcurveto{\pgfqpoint{1.796273in}{3.299845in}}{\pgfqpoint{1.799546in}{3.291945in}}{\pgfqpoint{1.805370in}{3.286121in}}%
\pgfpathcurveto{\pgfqpoint{1.811193in}{3.280297in}}{\pgfqpoint{1.819094in}{3.277025in}}{\pgfqpoint{1.827330in}{3.277025in}}%
\pgfpathclose%
\pgfusepath{stroke,fill}%
\end{pgfscope}%
\begin{pgfscope}%
\pgfpathrectangle{\pgfqpoint{0.100000in}{0.212622in}}{\pgfqpoint{3.696000in}{3.696000in}}%
\pgfusepath{clip}%
\pgfsetbuttcap%
\pgfsetroundjoin%
\definecolor{currentfill}{rgb}{0.121569,0.466667,0.705882}%
\pgfsetfillcolor{currentfill}%
\pgfsetfillopacity{0.314160}%
\pgfsetlinewidth{1.003750pt}%
\definecolor{currentstroke}{rgb}{0.121569,0.466667,0.705882}%
\pgfsetstrokecolor{currentstroke}%
\pgfsetstrokeopacity{0.314160}%
\pgfsetdash{}{0pt}%
\pgfpathmoveto{\pgfqpoint{1.917606in}{3.254818in}}%
\pgfpathcurveto{\pgfqpoint{1.925842in}{3.254818in}}{\pgfqpoint{1.933742in}{3.258091in}}{\pgfqpoint{1.939566in}{3.263915in}}%
\pgfpathcurveto{\pgfqpoint{1.945390in}{3.269739in}}{\pgfqpoint{1.948663in}{3.277639in}}{\pgfqpoint{1.948663in}{3.285875in}}%
\pgfpathcurveto{\pgfqpoint{1.948663in}{3.294111in}}{\pgfqpoint{1.945390in}{3.302011in}}{\pgfqpoint{1.939566in}{3.307835in}}%
\pgfpathcurveto{\pgfqpoint{1.933742in}{3.313659in}}{\pgfqpoint{1.925842in}{3.316931in}}{\pgfqpoint{1.917606in}{3.316931in}}%
\pgfpathcurveto{\pgfqpoint{1.909370in}{3.316931in}}{\pgfqpoint{1.901470in}{3.313659in}}{\pgfqpoint{1.895646in}{3.307835in}}%
\pgfpathcurveto{\pgfqpoint{1.889822in}{3.302011in}}{\pgfqpoint{1.886550in}{3.294111in}}{\pgfqpoint{1.886550in}{3.285875in}}%
\pgfpathcurveto{\pgfqpoint{1.886550in}{3.277639in}}{\pgfqpoint{1.889822in}{3.269739in}}{\pgfqpoint{1.895646in}{3.263915in}}%
\pgfpathcurveto{\pgfqpoint{1.901470in}{3.258091in}}{\pgfqpoint{1.909370in}{3.254818in}}{\pgfqpoint{1.917606in}{3.254818in}}%
\pgfpathclose%
\pgfusepath{stroke,fill}%
\end{pgfscope}%
\begin{pgfscope}%
\pgfpathrectangle{\pgfqpoint{0.100000in}{0.212622in}}{\pgfqpoint{3.696000in}{3.696000in}}%
\pgfusepath{clip}%
\pgfsetbuttcap%
\pgfsetroundjoin%
\definecolor{currentfill}{rgb}{0.121569,0.466667,0.705882}%
\pgfsetfillcolor{currentfill}%
\pgfsetfillopacity{0.314635}%
\pgfsetlinewidth{1.003750pt}%
\definecolor{currentstroke}{rgb}{0.121569,0.466667,0.705882}%
\pgfsetstrokecolor{currentstroke}%
\pgfsetstrokeopacity{0.314635}%
\pgfsetdash{}{0pt}%
\pgfpathmoveto{\pgfqpoint{1.825374in}{3.273554in}}%
\pgfpathcurveto{\pgfqpoint{1.833610in}{3.273554in}}{\pgfqpoint{1.841510in}{3.276827in}}{\pgfqpoint{1.847334in}{3.282651in}}%
\pgfpathcurveto{\pgfqpoint{1.853158in}{3.288474in}}{\pgfqpoint{1.856430in}{3.296375in}}{\pgfqpoint{1.856430in}{3.304611in}}%
\pgfpathcurveto{\pgfqpoint{1.856430in}{3.312847in}}{\pgfqpoint{1.853158in}{3.320747in}}{\pgfqpoint{1.847334in}{3.326571in}}%
\pgfpathcurveto{\pgfqpoint{1.841510in}{3.332395in}}{\pgfqpoint{1.833610in}{3.335667in}}{\pgfqpoint{1.825374in}{3.335667in}}%
\pgfpathcurveto{\pgfqpoint{1.817138in}{3.335667in}}{\pgfqpoint{1.809238in}{3.332395in}}{\pgfqpoint{1.803414in}{3.326571in}}%
\pgfpathcurveto{\pgfqpoint{1.797590in}{3.320747in}}{\pgfqpoint{1.794317in}{3.312847in}}{\pgfqpoint{1.794317in}{3.304611in}}%
\pgfpathcurveto{\pgfqpoint{1.794317in}{3.296375in}}{\pgfqpoint{1.797590in}{3.288474in}}{\pgfqpoint{1.803414in}{3.282651in}}%
\pgfpathcurveto{\pgfqpoint{1.809238in}{3.276827in}}{\pgfqpoint{1.817138in}{3.273554in}}{\pgfqpoint{1.825374in}{3.273554in}}%
\pgfpathclose%
\pgfusepath{stroke,fill}%
\end{pgfscope}%
\begin{pgfscope}%
\pgfpathrectangle{\pgfqpoint{0.100000in}{0.212622in}}{\pgfqpoint{3.696000in}{3.696000in}}%
\pgfusepath{clip}%
\pgfsetbuttcap%
\pgfsetroundjoin%
\definecolor{currentfill}{rgb}{0.121569,0.466667,0.705882}%
\pgfsetfillcolor{currentfill}%
\pgfsetfillopacity{0.315395}%
\pgfsetlinewidth{1.003750pt}%
\definecolor{currentstroke}{rgb}{0.121569,0.466667,0.705882}%
\pgfsetstrokecolor{currentstroke}%
\pgfsetstrokeopacity{0.315395}%
\pgfsetdash{}{0pt}%
\pgfpathmoveto{\pgfqpoint{1.918860in}{3.249757in}}%
\pgfpathcurveto{\pgfqpoint{1.927096in}{3.249757in}}{\pgfqpoint{1.934996in}{3.253029in}}{\pgfqpoint{1.940820in}{3.258853in}}%
\pgfpathcurveto{\pgfqpoint{1.946644in}{3.264677in}}{\pgfqpoint{1.949916in}{3.272577in}}{\pgfqpoint{1.949916in}{3.280813in}}%
\pgfpathcurveto{\pgfqpoint{1.949916in}{3.289050in}}{\pgfqpoint{1.946644in}{3.296950in}}{\pgfqpoint{1.940820in}{3.302774in}}%
\pgfpathcurveto{\pgfqpoint{1.934996in}{3.308598in}}{\pgfqpoint{1.927096in}{3.311870in}}{\pgfqpoint{1.918860in}{3.311870in}}%
\pgfpathcurveto{\pgfqpoint{1.910624in}{3.311870in}}{\pgfqpoint{1.902724in}{3.308598in}}{\pgfqpoint{1.896900in}{3.302774in}}%
\pgfpathcurveto{\pgfqpoint{1.891076in}{3.296950in}}{\pgfqpoint{1.887803in}{3.289050in}}{\pgfqpoint{1.887803in}{3.280813in}}%
\pgfpathcurveto{\pgfqpoint{1.887803in}{3.272577in}}{\pgfqpoint{1.891076in}{3.264677in}}{\pgfqpoint{1.896900in}{3.258853in}}%
\pgfpathcurveto{\pgfqpoint{1.902724in}{3.253029in}}{\pgfqpoint{1.910624in}{3.249757in}}{\pgfqpoint{1.918860in}{3.249757in}}%
\pgfpathclose%
\pgfusepath{stroke,fill}%
\end{pgfscope}%
\begin{pgfscope}%
\pgfpathrectangle{\pgfqpoint{0.100000in}{0.212622in}}{\pgfqpoint{3.696000in}{3.696000in}}%
\pgfusepath{clip}%
\pgfsetbuttcap%
\pgfsetroundjoin%
\definecolor{currentfill}{rgb}{0.121569,0.466667,0.705882}%
\pgfsetfillcolor{currentfill}%
\pgfsetfillopacity{0.315962}%
\pgfsetlinewidth{1.003750pt}%
\definecolor{currentstroke}{rgb}{0.121569,0.466667,0.705882}%
\pgfsetstrokecolor{currentstroke}%
\pgfsetstrokeopacity{0.315962}%
\pgfsetdash{}{0pt}%
\pgfpathmoveto{\pgfqpoint{1.821817in}{3.267333in}}%
\pgfpathcurveto{\pgfqpoint{1.830054in}{3.267333in}}{\pgfqpoint{1.837954in}{3.270606in}}{\pgfqpoint{1.843778in}{3.276430in}}%
\pgfpathcurveto{\pgfqpoint{1.849602in}{3.282254in}}{\pgfqpoint{1.852874in}{3.290154in}}{\pgfqpoint{1.852874in}{3.298390in}}%
\pgfpathcurveto{\pgfqpoint{1.852874in}{3.306626in}}{\pgfqpoint{1.849602in}{3.314526in}}{\pgfqpoint{1.843778in}{3.320350in}}%
\pgfpathcurveto{\pgfqpoint{1.837954in}{3.326174in}}{\pgfqpoint{1.830054in}{3.329446in}}{\pgfqpoint{1.821817in}{3.329446in}}%
\pgfpathcurveto{\pgfqpoint{1.813581in}{3.329446in}}{\pgfqpoint{1.805681in}{3.326174in}}{\pgfqpoint{1.799857in}{3.320350in}}%
\pgfpathcurveto{\pgfqpoint{1.794033in}{3.314526in}}{\pgfqpoint{1.790761in}{3.306626in}}{\pgfqpoint{1.790761in}{3.298390in}}%
\pgfpathcurveto{\pgfqpoint{1.790761in}{3.290154in}}{\pgfqpoint{1.794033in}{3.282254in}}{\pgfqpoint{1.799857in}{3.276430in}}%
\pgfpathcurveto{\pgfqpoint{1.805681in}{3.270606in}}{\pgfqpoint{1.813581in}{3.267333in}}{\pgfqpoint{1.821817in}{3.267333in}}%
\pgfpathclose%
\pgfusepath{stroke,fill}%
\end{pgfscope}%
\begin{pgfscope}%
\pgfpathrectangle{\pgfqpoint{0.100000in}{0.212622in}}{\pgfqpoint{3.696000in}{3.696000in}}%
\pgfusepath{clip}%
\pgfsetbuttcap%
\pgfsetroundjoin%
\definecolor{currentfill}{rgb}{0.121569,0.466667,0.705882}%
\pgfsetfillcolor{currentfill}%
\pgfsetfillopacity{0.316027}%
\pgfsetlinewidth{1.003750pt}%
\definecolor{currentstroke}{rgb}{0.121569,0.466667,0.705882}%
\pgfsetstrokecolor{currentstroke}%
\pgfsetstrokeopacity{0.316027}%
\pgfsetdash{}{0pt}%
\pgfpathmoveto{\pgfqpoint{1.919703in}{3.246958in}}%
\pgfpathcurveto{\pgfqpoint{1.927939in}{3.246958in}}{\pgfqpoint{1.935839in}{3.250230in}}{\pgfqpoint{1.941663in}{3.256054in}}%
\pgfpathcurveto{\pgfqpoint{1.947487in}{3.261878in}}{\pgfqpoint{1.950759in}{3.269778in}}{\pgfqpoint{1.950759in}{3.278015in}}%
\pgfpathcurveto{\pgfqpoint{1.950759in}{3.286251in}}{\pgfqpoint{1.947487in}{3.294151in}}{\pgfqpoint{1.941663in}{3.299975in}}%
\pgfpathcurveto{\pgfqpoint{1.935839in}{3.305799in}}{\pgfqpoint{1.927939in}{3.309071in}}{\pgfqpoint{1.919703in}{3.309071in}}%
\pgfpathcurveto{\pgfqpoint{1.911467in}{3.309071in}}{\pgfqpoint{1.903567in}{3.305799in}}{\pgfqpoint{1.897743in}{3.299975in}}%
\pgfpathcurveto{\pgfqpoint{1.891919in}{3.294151in}}{\pgfqpoint{1.888646in}{3.286251in}}{\pgfqpoint{1.888646in}{3.278015in}}%
\pgfpathcurveto{\pgfqpoint{1.888646in}{3.269778in}}{\pgfqpoint{1.891919in}{3.261878in}}{\pgfqpoint{1.897743in}{3.256054in}}%
\pgfpathcurveto{\pgfqpoint{1.903567in}{3.250230in}}{\pgfqpoint{1.911467in}{3.246958in}}{\pgfqpoint{1.919703in}{3.246958in}}%
\pgfpathclose%
\pgfusepath{stroke,fill}%
\end{pgfscope}%
\begin{pgfscope}%
\pgfpathrectangle{\pgfqpoint{0.100000in}{0.212622in}}{\pgfqpoint{3.696000in}{3.696000in}}%
\pgfusepath{clip}%
\pgfsetbuttcap%
\pgfsetroundjoin%
\definecolor{currentfill}{rgb}{0.121569,0.466667,0.705882}%
\pgfsetfillcolor{currentfill}%
\pgfsetfillopacity{0.316373}%
\pgfsetlinewidth{1.003750pt}%
\definecolor{currentstroke}{rgb}{0.121569,0.466667,0.705882}%
\pgfsetstrokecolor{currentstroke}%
\pgfsetstrokeopacity{0.316373}%
\pgfsetdash{}{0pt}%
\pgfpathmoveto{\pgfqpoint{1.920147in}{3.245385in}}%
\pgfpathcurveto{\pgfqpoint{1.928384in}{3.245385in}}{\pgfqpoint{1.936284in}{3.248658in}}{\pgfqpoint{1.942108in}{3.254482in}}%
\pgfpathcurveto{\pgfqpoint{1.947932in}{3.260306in}}{\pgfqpoint{1.951204in}{3.268206in}}{\pgfqpoint{1.951204in}{3.276442in}}%
\pgfpathcurveto{\pgfqpoint{1.951204in}{3.284678in}}{\pgfqpoint{1.947932in}{3.292578in}}{\pgfqpoint{1.942108in}{3.298402in}}%
\pgfpathcurveto{\pgfqpoint{1.936284in}{3.304226in}}{\pgfqpoint{1.928384in}{3.307498in}}{\pgfqpoint{1.920147in}{3.307498in}}%
\pgfpathcurveto{\pgfqpoint{1.911911in}{3.307498in}}{\pgfqpoint{1.904011in}{3.304226in}}{\pgfqpoint{1.898187in}{3.298402in}}%
\pgfpathcurveto{\pgfqpoint{1.892363in}{3.292578in}}{\pgfqpoint{1.889091in}{3.284678in}}{\pgfqpoint{1.889091in}{3.276442in}}%
\pgfpathcurveto{\pgfqpoint{1.889091in}{3.268206in}}{\pgfqpoint{1.892363in}{3.260306in}}{\pgfqpoint{1.898187in}{3.254482in}}%
\pgfpathcurveto{\pgfqpoint{1.904011in}{3.248658in}}{\pgfqpoint{1.911911in}{3.245385in}}{\pgfqpoint{1.920147in}{3.245385in}}%
\pgfpathclose%
\pgfusepath{stroke,fill}%
\end{pgfscope}%
\begin{pgfscope}%
\pgfpathrectangle{\pgfqpoint{0.100000in}{0.212622in}}{\pgfqpoint{3.696000in}{3.696000in}}%
\pgfusepath{clip}%
\pgfsetbuttcap%
\pgfsetroundjoin%
\definecolor{currentfill}{rgb}{0.121569,0.466667,0.705882}%
\pgfsetfillcolor{currentfill}%
\pgfsetfillopacity{0.316882}%
\pgfsetlinewidth{1.003750pt}%
\definecolor{currentstroke}{rgb}{0.121569,0.466667,0.705882}%
\pgfsetstrokecolor{currentstroke}%
\pgfsetstrokeopacity{0.316882}%
\pgfsetdash{}{0pt}%
\pgfpathmoveto{\pgfqpoint{1.920588in}{3.243583in}}%
\pgfpathcurveto{\pgfqpoint{1.928824in}{3.243583in}}{\pgfqpoint{1.936724in}{3.246855in}}{\pgfqpoint{1.942548in}{3.252679in}}%
\pgfpathcurveto{\pgfqpoint{1.948372in}{3.258503in}}{\pgfqpoint{1.951644in}{3.266403in}}{\pgfqpoint{1.951644in}{3.274639in}}%
\pgfpathcurveto{\pgfqpoint{1.951644in}{3.282876in}}{\pgfqpoint{1.948372in}{3.290776in}}{\pgfqpoint{1.942548in}{3.296600in}}%
\pgfpathcurveto{\pgfqpoint{1.936724in}{3.302424in}}{\pgfqpoint{1.928824in}{3.305696in}}{\pgfqpoint{1.920588in}{3.305696in}}%
\pgfpathcurveto{\pgfqpoint{1.912352in}{3.305696in}}{\pgfqpoint{1.904452in}{3.302424in}}{\pgfqpoint{1.898628in}{3.296600in}}%
\pgfpathcurveto{\pgfqpoint{1.892804in}{3.290776in}}{\pgfqpoint{1.889531in}{3.282876in}}{\pgfqpoint{1.889531in}{3.274639in}}%
\pgfpathcurveto{\pgfqpoint{1.889531in}{3.266403in}}{\pgfqpoint{1.892804in}{3.258503in}}{\pgfqpoint{1.898628in}{3.252679in}}%
\pgfpathcurveto{\pgfqpoint{1.904452in}{3.246855in}}{\pgfqpoint{1.912352in}{3.243583in}}{\pgfqpoint{1.920588in}{3.243583in}}%
\pgfpathclose%
\pgfusepath{stroke,fill}%
\end{pgfscope}%
\begin{pgfscope}%
\pgfpathrectangle{\pgfqpoint{0.100000in}{0.212622in}}{\pgfqpoint{3.696000in}{3.696000in}}%
\pgfusepath{clip}%
\pgfsetbuttcap%
\pgfsetroundjoin%
\definecolor{currentfill}{rgb}{0.121569,0.466667,0.705882}%
\pgfsetfillcolor{currentfill}%
\pgfsetfillopacity{0.317055}%
\pgfsetlinewidth{1.003750pt}%
\definecolor{currentstroke}{rgb}{0.121569,0.466667,0.705882}%
\pgfsetstrokecolor{currentstroke}%
\pgfsetstrokeopacity{0.317055}%
\pgfsetdash{}{0pt}%
\pgfpathmoveto{\pgfqpoint{1.818145in}{3.261519in}}%
\pgfpathcurveto{\pgfqpoint{1.826381in}{3.261519in}}{\pgfqpoint{1.834282in}{3.264791in}}{\pgfqpoint{1.840105in}{3.270615in}}%
\pgfpathcurveto{\pgfqpoint{1.845929in}{3.276439in}}{\pgfqpoint{1.849202in}{3.284339in}}{\pgfqpoint{1.849202in}{3.292575in}}%
\pgfpathcurveto{\pgfqpoint{1.849202in}{3.300811in}}{\pgfqpoint{1.845929in}{3.308711in}}{\pgfqpoint{1.840105in}{3.314535in}}%
\pgfpathcurveto{\pgfqpoint{1.834282in}{3.320359in}}{\pgfqpoint{1.826381in}{3.323632in}}{\pgfqpoint{1.818145in}{3.323632in}}%
\pgfpathcurveto{\pgfqpoint{1.809909in}{3.323632in}}{\pgfqpoint{1.802009in}{3.320359in}}{\pgfqpoint{1.796185in}{3.314535in}}%
\pgfpathcurveto{\pgfqpoint{1.790361in}{3.308711in}}{\pgfqpoint{1.787089in}{3.300811in}}{\pgfqpoint{1.787089in}{3.292575in}}%
\pgfpathcurveto{\pgfqpoint{1.787089in}{3.284339in}}{\pgfqpoint{1.790361in}{3.276439in}}{\pgfqpoint{1.796185in}{3.270615in}}%
\pgfpathcurveto{\pgfqpoint{1.802009in}{3.264791in}}{\pgfqpoint{1.809909in}{3.261519in}}{\pgfqpoint{1.818145in}{3.261519in}}%
\pgfpathclose%
\pgfusepath{stroke,fill}%
\end{pgfscope}%
\begin{pgfscope}%
\pgfpathrectangle{\pgfqpoint{0.100000in}{0.212622in}}{\pgfqpoint{3.696000in}{3.696000in}}%
\pgfusepath{clip}%
\pgfsetbuttcap%
\pgfsetroundjoin%
\definecolor{currentfill}{rgb}{0.121569,0.466667,0.705882}%
\pgfsetfillcolor{currentfill}%
\pgfsetfillopacity{0.317647}%
\pgfsetlinewidth{1.003750pt}%
\definecolor{currentstroke}{rgb}{0.121569,0.466667,0.705882}%
\pgfsetstrokecolor{currentstroke}%
\pgfsetstrokeopacity{0.317647}%
\pgfsetdash{}{0pt}%
\pgfpathmoveto{\pgfqpoint{1.921617in}{3.240062in}}%
\pgfpathcurveto{\pgfqpoint{1.929854in}{3.240062in}}{\pgfqpoint{1.937754in}{3.243335in}}{\pgfqpoint{1.943578in}{3.249159in}}%
\pgfpathcurveto{\pgfqpoint{1.949402in}{3.254983in}}{\pgfqpoint{1.952674in}{3.262883in}}{\pgfqpoint{1.952674in}{3.271119in}}%
\pgfpathcurveto{\pgfqpoint{1.952674in}{3.279355in}}{\pgfqpoint{1.949402in}{3.287255in}}{\pgfqpoint{1.943578in}{3.293079in}}%
\pgfpathcurveto{\pgfqpoint{1.937754in}{3.298903in}}{\pgfqpoint{1.929854in}{3.302175in}}{\pgfqpoint{1.921617in}{3.302175in}}%
\pgfpathcurveto{\pgfqpoint{1.913381in}{3.302175in}}{\pgfqpoint{1.905481in}{3.298903in}}{\pgfqpoint{1.899657in}{3.293079in}}%
\pgfpathcurveto{\pgfqpoint{1.893833in}{3.287255in}}{\pgfqpoint{1.890561in}{3.279355in}}{\pgfqpoint{1.890561in}{3.271119in}}%
\pgfpathcurveto{\pgfqpoint{1.890561in}{3.262883in}}{\pgfqpoint{1.893833in}{3.254983in}}{\pgfqpoint{1.899657in}{3.249159in}}%
\pgfpathcurveto{\pgfqpoint{1.905481in}{3.243335in}}{\pgfqpoint{1.913381in}{3.240062in}}{\pgfqpoint{1.921617in}{3.240062in}}%
\pgfpathclose%
\pgfusepath{stroke,fill}%
\end{pgfscope}%
\begin{pgfscope}%
\pgfpathrectangle{\pgfqpoint{0.100000in}{0.212622in}}{\pgfqpoint{3.696000in}{3.696000in}}%
\pgfusepath{clip}%
\pgfsetbuttcap%
\pgfsetroundjoin%
\definecolor{currentfill}{rgb}{0.121569,0.466667,0.705882}%
\pgfsetfillcolor{currentfill}%
\pgfsetfillopacity{0.317878}%
\pgfsetlinewidth{1.003750pt}%
\definecolor{currentstroke}{rgb}{0.121569,0.466667,0.705882}%
\pgfsetstrokecolor{currentstroke}%
\pgfsetstrokeopacity{0.317878}%
\pgfsetdash{}{0pt}%
\pgfpathmoveto{\pgfqpoint{1.816026in}{3.257211in}}%
\pgfpathcurveto{\pgfqpoint{1.824263in}{3.257211in}}{\pgfqpoint{1.832163in}{3.260483in}}{\pgfqpoint{1.837987in}{3.266307in}}%
\pgfpathcurveto{\pgfqpoint{1.843811in}{3.272131in}}{\pgfqpoint{1.847083in}{3.280031in}}{\pgfqpoint{1.847083in}{3.288267in}}%
\pgfpathcurveto{\pgfqpoint{1.847083in}{3.296504in}}{\pgfqpoint{1.843811in}{3.304404in}}{\pgfqpoint{1.837987in}{3.310228in}}%
\pgfpathcurveto{\pgfqpoint{1.832163in}{3.316052in}}{\pgfqpoint{1.824263in}{3.319324in}}{\pgfqpoint{1.816026in}{3.319324in}}%
\pgfpathcurveto{\pgfqpoint{1.807790in}{3.319324in}}{\pgfqpoint{1.799890in}{3.316052in}}{\pgfqpoint{1.794066in}{3.310228in}}%
\pgfpathcurveto{\pgfqpoint{1.788242in}{3.304404in}}{\pgfqpoint{1.784970in}{3.296504in}}{\pgfqpoint{1.784970in}{3.288267in}}%
\pgfpathcurveto{\pgfqpoint{1.784970in}{3.280031in}}{\pgfqpoint{1.788242in}{3.272131in}}{\pgfqpoint{1.794066in}{3.266307in}}%
\pgfpathcurveto{\pgfqpoint{1.799890in}{3.260483in}}{\pgfqpoint{1.807790in}{3.257211in}}{\pgfqpoint{1.816026in}{3.257211in}}%
\pgfpathclose%
\pgfusepath{stroke,fill}%
\end{pgfscope}%
\begin{pgfscope}%
\pgfpathrectangle{\pgfqpoint{0.100000in}{0.212622in}}{\pgfqpoint{3.696000in}{3.696000in}}%
\pgfusepath{clip}%
\pgfsetbuttcap%
\pgfsetroundjoin%
\definecolor{currentfill}{rgb}{0.121569,0.466667,0.705882}%
\pgfsetfillcolor{currentfill}%
\pgfsetfillopacity{0.318628}%
\pgfsetlinewidth{1.003750pt}%
\definecolor{currentstroke}{rgb}{0.121569,0.466667,0.705882}%
\pgfsetstrokecolor{currentstroke}%
\pgfsetstrokeopacity{0.318628}%
\pgfsetdash{}{0pt}%
\pgfpathmoveto{\pgfqpoint{1.813778in}{3.253540in}}%
\pgfpathcurveto{\pgfqpoint{1.822014in}{3.253540in}}{\pgfqpoint{1.829914in}{3.256812in}}{\pgfqpoint{1.835738in}{3.262636in}}%
\pgfpathcurveto{\pgfqpoint{1.841562in}{3.268460in}}{\pgfqpoint{1.844835in}{3.276360in}}{\pgfqpoint{1.844835in}{3.284597in}}%
\pgfpathcurveto{\pgfqpoint{1.844835in}{3.292833in}}{\pgfqpoint{1.841562in}{3.300733in}}{\pgfqpoint{1.835738in}{3.306557in}}%
\pgfpathcurveto{\pgfqpoint{1.829914in}{3.312381in}}{\pgfqpoint{1.822014in}{3.315653in}}{\pgfqpoint{1.813778in}{3.315653in}}%
\pgfpathcurveto{\pgfqpoint{1.805542in}{3.315653in}}{\pgfqpoint{1.797642in}{3.312381in}}{\pgfqpoint{1.791818in}{3.306557in}}%
\pgfpathcurveto{\pgfqpoint{1.785994in}{3.300733in}}{\pgfqpoint{1.782722in}{3.292833in}}{\pgfqpoint{1.782722in}{3.284597in}}%
\pgfpathcurveto{\pgfqpoint{1.782722in}{3.276360in}}{\pgfqpoint{1.785994in}{3.268460in}}{\pgfqpoint{1.791818in}{3.262636in}}%
\pgfpathcurveto{\pgfqpoint{1.797642in}{3.256812in}}{\pgfqpoint{1.805542in}{3.253540in}}{\pgfqpoint{1.813778in}{3.253540in}}%
\pgfpathclose%
\pgfusepath{stroke,fill}%
\end{pgfscope}%
\begin{pgfscope}%
\pgfpathrectangle{\pgfqpoint{0.100000in}{0.212622in}}{\pgfqpoint{3.696000in}{3.696000in}}%
\pgfusepath{clip}%
\pgfsetbuttcap%
\pgfsetroundjoin%
\definecolor{currentfill}{rgb}{0.121569,0.466667,0.705882}%
\pgfsetfillcolor{currentfill}%
\pgfsetfillopacity{0.318631}%
\pgfsetlinewidth{1.003750pt}%
\definecolor{currentstroke}{rgb}{0.121569,0.466667,0.705882}%
\pgfsetstrokecolor{currentstroke}%
\pgfsetstrokeopacity{0.318631}%
\pgfsetdash{}{0pt}%
\pgfpathmoveto{\pgfqpoint{1.922553in}{3.236064in}}%
\pgfpathcurveto{\pgfqpoint{1.930789in}{3.236064in}}{\pgfqpoint{1.938689in}{3.239337in}}{\pgfqpoint{1.944513in}{3.245161in}}%
\pgfpathcurveto{\pgfqpoint{1.950337in}{3.250985in}}{\pgfqpoint{1.953609in}{3.258885in}}{\pgfqpoint{1.953609in}{3.267121in}}%
\pgfpathcurveto{\pgfqpoint{1.953609in}{3.275357in}}{\pgfqpoint{1.950337in}{3.283257in}}{\pgfqpoint{1.944513in}{3.289081in}}%
\pgfpathcurveto{\pgfqpoint{1.938689in}{3.294905in}}{\pgfqpoint{1.930789in}{3.298177in}}{\pgfqpoint{1.922553in}{3.298177in}}%
\pgfpathcurveto{\pgfqpoint{1.914317in}{3.298177in}}{\pgfqpoint{1.906416in}{3.294905in}}{\pgfqpoint{1.900593in}{3.289081in}}%
\pgfpathcurveto{\pgfqpoint{1.894769in}{3.283257in}}{\pgfqpoint{1.891496in}{3.275357in}}{\pgfqpoint{1.891496in}{3.267121in}}%
\pgfpathcurveto{\pgfqpoint{1.891496in}{3.258885in}}{\pgfqpoint{1.894769in}{3.250985in}}{\pgfqpoint{1.900593in}{3.245161in}}%
\pgfpathcurveto{\pgfqpoint{1.906416in}{3.239337in}}{\pgfqpoint{1.914317in}{3.236064in}}{\pgfqpoint{1.922553in}{3.236064in}}%
\pgfpathclose%
\pgfusepath{stroke,fill}%
\end{pgfscope}%
\begin{pgfscope}%
\pgfpathrectangle{\pgfqpoint{0.100000in}{0.212622in}}{\pgfqpoint{3.696000in}{3.696000in}}%
\pgfusepath{clip}%
\pgfsetbuttcap%
\pgfsetroundjoin%
\definecolor{currentfill}{rgb}{0.121569,0.466667,0.705882}%
\pgfsetfillcolor{currentfill}%
\pgfsetfillopacity{0.319801}%
\pgfsetlinewidth{1.003750pt}%
\definecolor{currentstroke}{rgb}{0.121569,0.466667,0.705882}%
\pgfsetstrokecolor{currentstroke}%
\pgfsetstrokeopacity{0.319801}%
\pgfsetdash{}{0pt}%
\pgfpathmoveto{\pgfqpoint{1.923467in}{3.231709in}}%
\pgfpathcurveto{\pgfqpoint{1.931703in}{3.231709in}}{\pgfqpoint{1.939603in}{3.234981in}}{\pgfqpoint{1.945427in}{3.240805in}}%
\pgfpathcurveto{\pgfqpoint{1.951251in}{3.246629in}}{\pgfqpoint{1.954523in}{3.254529in}}{\pgfqpoint{1.954523in}{3.262765in}}%
\pgfpathcurveto{\pgfqpoint{1.954523in}{3.271002in}}{\pgfqpoint{1.951251in}{3.278902in}}{\pgfqpoint{1.945427in}{3.284726in}}%
\pgfpathcurveto{\pgfqpoint{1.939603in}{3.290550in}}{\pgfqpoint{1.931703in}{3.293822in}}{\pgfqpoint{1.923467in}{3.293822in}}%
\pgfpathcurveto{\pgfqpoint{1.915230in}{3.293822in}}{\pgfqpoint{1.907330in}{3.290550in}}{\pgfqpoint{1.901506in}{3.284726in}}%
\pgfpathcurveto{\pgfqpoint{1.895682in}{3.278902in}}{\pgfqpoint{1.892410in}{3.271002in}}{\pgfqpoint{1.892410in}{3.262765in}}%
\pgfpathcurveto{\pgfqpoint{1.892410in}{3.254529in}}{\pgfqpoint{1.895682in}{3.246629in}}{\pgfqpoint{1.901506in}{3.240805in}}%
\pgfpathcurveto{\pgfqpoint{1.907330in}{3.234981in}}{\pgfqpoint{1.915230in}{3.231709in}}{\pgfqpoint{1.923467in}{3.231709in}}%
\pgfpathclose%
\pgfusepath{stroke,fill}%
\end{pgfscope}%
\begin{pgfscope}%
\pgfpathrectangle{\pgfqpoint{0.100000in}{0.212622in}}{\pgfqpoint{3.696000in}{3.696000in}}%
\pgfusepath{clip}%
\pgfsetbuttcap%
\pgfsetroundjoin%
\definecolor{currentfill}{rgb}{0.121569,0.466667,0.705882}%
\pgfsetfillcolor{currentfill}%
\pgfsetfillopacity{0.320001}%
\pgfsetlinewidth{1.003750pt}%
\definecolor{currentstroke}{rgb}{0.121569,0.466667,0.705882}%
\pgfsetstrokecolor{currentstroke}%
\pgfsetstrokeopacity{0.320001}%
\pgfsetdash{}{0pt}%
\pgfpathmoveto{\pgfqpoint{1.809662in}{3.246942in}}%
\pgfpathcurveto{\pgfqpoint{1.817898in}{3.246942in}}{\pgfqpoint{1.825798in}{3.250215in}}{\pgfqpoint{1.831622in}{3.256039in}}%
\pgfpathcurveto{\pgfqpoint{1.837446in}{3.261863in}}{\pgfqpoint{1.840718in}{3.269763in}}{\pgfqpoint{1.840718in}{3.277999in}}%
\pgfpathcurveto{\pgfqpoint{1.840718in}{3.286235in}}{\pgfqpoint{1.837446in}{3.294135in}}{\pgfqpoint{1.831622in}{3.299959in}}%
\pgfpathcurveto{\pgfqpoint{1.825798in}{3.305783in}}{\pgfqpoint{1.817898in}{3.309055in}}{\pgfqpoint{1.809662in}{3.309055in}}%
\pgfpathcurveto{\pgfqpoint{1.801426in}{3.309055in}}{\pgfqpoint{1.793526in}{3.305783in}}{\pgfqpoint{1.787702in}{3.299959in}}%
\pgfpathcurveto{\pgfqpoint{1.781878in}{3.294135in}}{\pgfqpoint{1.778605in}{3.286235in}}{\pgfqpoint{1.778605in}{3.277999in}}%
\pgfpathcurveto{\pgfqpoint{1.778605in}{3.269763in}}{\pgfqpoint{1.781878in}{3.261863in}}{\pgfqpoint{1.787702in}{3.256039in}}%
\pgfpathcurveto{\pgfqpoint{1.793526in}{3.250215in}}{\pgfqpoint{1.801426in}{3.246942in}}{\pgfqpoint{1.809662in}{3.246942in}}%
\pgfpathclose%
\pgfusepath{stroke,fill}%
\end{pgfscope}%
\begin{pgfscope}%
\pgfpathrectangle{\pgfqpoint{0.100000in}{0.212622in}}{\pgfqpoint{3.696000in}{3.696000in}}%
\pgfusepath{clip}%
\pgfsetbuttcap%
\pgfsetroundjoin%
\definecolor{currentfill}{rgb}{0.121569,0.466667,0.705882}%
\pgfsetfillcolor{currentfill}%
\pgfsetfillopacity{0.320864}%
\pgfsetlinewidth{1.003750pt}%
\definecolor{currentstroke}{rgb}{0.121569,0.466667,0.705882}%
\pgfsetstrokecolor{currentstroke}%
\pgfsetstrokeopacity{0.320864}%
\pgfsetdash{}{0pt}%
\pgfpathmoveto{\pgfqpoint{1.924793in}{3.226573in}}%
\pgfpathcurveto{\pgfqpoint{1.933029in}{3.226573in}}{\pgfqpoint{1.940929in}{3.229845in}}{\pgfqpoint{1.946753in}{3.235669in}}%
\pgfpathcurveto{\pgfqpoint{1.952577in}{3.241493in}}{\pgfqpoint{1.955849in}{3.249393in}}{\pgfqpoint{1.955849in}{3.257629in}}%
\pgfpathcurveto{\pgfqpoint{1.955849in}{3.265866in}}{\pgfqpoint{1.952577in}{3.273766in}}{\pgfqpoint{1.946753in}{3.279590in}}%
\pgfpathcurveto{\pgfqpoint{1.940929in}{3.285414in}}{\pgfqpoint{1.933029in}{3.288686in}}{\pgfqpoint{1.924793in}{3.288686in}}%
\pgfpathcurveto{\pgfqpoint{1.916557in}{3.288686in}}{\pgfqpoint{1.908657in}{3.285414in}}{\pgfqpoint{1.902833in}{3.279590in}}%
\pgfpathcurveto{\pgfqpoint{1.897009in}{3.273766in}}{\pgfqpoint{1.893736in}{3.265866in}}{\pgfqpoint{1.893736in}{3.257629in}}%
\pgfpathcurveto{\pgfqpoint{1.893736in}{3.249393in}}{\pgfqpoint{1.897009in}{3.241493in}}{\pgfqpoint{1.902833in}{3.235669in}}%
\pgfpathcurveto{\pgfqpoint{1.908657in}{3.229845in}}{\pgfqpoint{1.916557in}{3.226573in}}{\pgfqpoint{1.924793in}{3.226573in}}%
\pgfpathclose%
\pgfusepath{stroke,fill}%
\end{pgfscope}%
\begin{pgfscope}%
\pgfpathrectangle{\pgfqpoint{0.100000in}{0.212622in}}{\pgfqpoint{3.696000in}{3.696000in}}%
\pgfusepath{clip}%
\pgfsetbuttcap%
\pgfsetroundjoin%
\definecolor{currentfill}{rgb}{0.121569,0.466667,0.705882}%
\pgfsetfillcolor{currentfill}%
\pgfsetfillopacity{0.321329}%
\pgfsetlinewidth{1.003750pt}%
\definecolor{currentstroke}{rgb}{0.121569,0.466667,0.705882}%
\pgfsetstrokecolor{currentstroke}%
\pgfsetstrokeopacity{0.321329}%
\pgfsetdash{}{0pt}%
\pgfpathmoveto{\pgfqpoint{1.806312in}{3.240144in}}%
\pgfpathcurveto{\pgfqpoint{1.814548in}{3.240144in}}{\pgfqpoint{1.822448in}{3.243416in}}{\pgfqpoint{1.828272in}{3.249240in}}%
\pgfpathcurveto{\pgfqpoint{1.834096in}{3.255064in}}{\pgfqpoint{1.837369in}{3.262964in}}{\pgfqpoint{1.837369in}{3.271200in}}%
\pgfpathcurveto{\pgfqpoint{1.837369in}{3.279436in}}{\pgfqpoint{1.834096in}{3.287336in}}{\pgfqpoint{1.828272in}{3.293160in}}%
\pgfpathcurveto{\pgfqpoint{1.822448in}{3.298984in}}{\pgfqpoint{1.814548in}{3.302257in}}{\pgfqpoint{1.806312in}{3.302257in}}%
\pgfpathcurveto{\pgfqpoint{1.798076in}{3.302257in}}{\pgfqpoint{1.790176in}{3.298984in}}{\pgfqpoint{1.784352in}{3.293160in}}%
\pgfpathcurveto{\pgfqpoint{1.778528in}{3.287336in}}{\pgfqpoint{1.775256in}{3.279436in}}{\pgfqpoint{1.775256in}{3.271200in}}%
\pgfpathcurveto{\pgfqpoint{1.775256in}{3.262964in}}{\pgfqpoint{1.778528in}{3.255064in}}{\pgfqpoint{1.784352in}{3.249240in}}%
\pgfpathcurveto{\pgfqpoint{1.790176in}{3.243416in}}{\pgfqpoint{1.798076in}{3.240144in}}{\pgfqpoint{1.806312in}{3.240144in}}%
\pgfpathclose%
\pgfusepath{stroke,fill}%
\end{pgfscope}%
\begin{pgfscope}%
\pgfpathrectangle{\pgfqpoint{0.100000in}{0.212622in}}{\pgfqpoint{3.696000in}{3.696000in}}%
\pgfusepath{clip}%
\pgfsetbuttcap%
\pgfsetroundjoin%
\definecolor{currentfill}{rgb}{0.121569,0.466667,0.705882}%
\pgfsetfillcolor{currentfill}%
\pgfsetfillopacity{0.322126}%
\pgfsetlinewidth{1.003750pt}%
\definecolor{currentstroke}{rgb}{0.121569,0.466667,0.705882}%
\pgfsetstrokecolor{currentstroke}%
\pgfsetstrokeopacity{0.322126}%
\pgfsetdash{}{0pt}%
\pgfpathmoveto{\pgfqpoint{1.803783in}{3.236266in}}%
\pgfpathcurveto{\pgfqpoint{1.812019in}{3.236266in}}{\pgfqpoint{1.819919in}{3.239539in}}{\pgfqpoint{1.825743in}{3.245362in}}%
\pgfpathcurveto{\pgfqpoint{1.831567in}{3.251186in}}{\pgfqpoint{1.834839in}{3.259086in}}{\pgfqpoint{1.834839in}{3.267323in}}%
\pgfpathcurveto{\pgfqpoint{1.834839in}{3.275559in}}{\pgfqpoint{1.831567in}{3.283459in}}{\pgfqpoint{1.825743in}{3.289283in}}%
\pgfpathcurveto{\pgfqpoint{1.819919in}{3.295107in}}{\pgfqpoint{1.812019in}{3.298379in}}{\pgfqpoint{1.803783in}{3.298379in}}%
\pgfpathcurveto{\pgfqpoint{1.795546in}{3.298379in}}{\pgfqpoint{1.787646in}{3.295107in}}{\pgfqpoint{1.781822in}{3.289283in}}%
\pgfpathcurveto{\pgfqpoint{1.775998in}{3.283459in}}{\pgfqpoint{1.772726in}{3.275559in}}{\pgfqpoint{1.772726in}{3.267323in}}%
\pgfpathcurveto{\pgfqpoint{1.772726in}{3.259086in}}{\pgfqpoint{1.775998in}{3.251186in}}{\pgfqpoint{1.781822in}{3.245362in}}%
\pgfpathcurveto{\pgfqpoint{1.787646in}{3.239539in}}{\pgfqpoint{1.795546in}{3.236266in}}{\pgfqpoint{1.803783in}{3.236266in}}%
\pgfpathclose%
\pgfusepath{stroke,fill}%
\end{pgfscope}%
\begin{pgfscope}%
\pgfpathrectangle{\pgfqpoint{0.100000in}{0.212622in}}{\pgfqpoint{3.696000in}{3.696000in}}%
\pgfusepath{clip}%
\pgfsetbuttcap%
\pgfsetroundjoin%
\definecolor{currentfill}{rgb}{0.121569,0.466667,0.705882}%
\pgfsetfillcolor{currentfill}%
\pgfsetfillopacity{0.322573}%
\pgfsetlinewidth{1.003750pt}%
\definecolor{currentstroke}{rgb}{0.121569,0.466667,0.705882}%
\pgfsetstrokecolor{currentstroke}%
\pgfsetstrokeopacity{0.322573}%
\pgfsetdash{}{0pt}%
\pgfpathmoveto{\pgfqpoint{1.925917in}{3.220118in}}%
\pgfpathcurveto{\pgfqpoint{1.934154in}{3.220118in}}{\pgfqpoint{1.942054in}{3.223391in}}{\pgfqpoint{1.947878in}{3.229215in}}%
\pgfpathcurveto{\pgfqpoint{1.953702in}{3.235039in}}{\pgfqpoint{1.956974in}{3.242939in}}{\pgfqpoint{1.956974in}{3.251175in}}%
\pgfpathcurveto{\pgfqpoint{1.956974in}{3.259411in}}{\pgfqpoint{1.953702in}{3.267311in}}{\pgfqpoint{1.947878in}{3.273135in}}%
\pgfpathcurveto{\pgfqpoint{1.942054in}{3.278959in}}{\pgfqpoint{1.934154in}{3.282231in}}{\pgfqpoint{1.925917in}{3.282231in}}%
\pgfpathcurveto{\pgfqpoint{1.917681in}{3.282231in}}{\pgfqpoint{1.909781in}{3.278959in}}{\pgfqpoint{1.903957in}{3.273135in}}%
\pgfpathcurveto{\pgfqpoint{1.898133in}{3.267311in}}{\pgfqpoint{1.894861in}{3.259411in}}{\pgfqpoint{1.894861in}{3.251175in}}%
\pgfpathcurveto{\pgfqpoint{1.894861in}{3.242939in}}{\pgfqpoint{1.898133in}{3.235039in}}{\pgfqpoint{1.903957in}{3.229215in}}%
\pgfpathcurveto{\pgfqpoint{1.909781in}{3.223391in}}{\pgfqpoint{1.917681in}{3.220118in}}{\pgfqpoint{1.925917in}{3.220118in}}%
\pgfpathclose%
\pgfusepath{stroke,fill}%
\end{pgfscope}%
\begin{pgfscope}%
\pgfpathrectangle{\pgfqpoint{0.100000in}{0.212622in}}{\pgfqpoint{3.696000in}{3.696000in}}%
\pgfusepath{clip}%
\pgfsetbuttcap%
\pgfsetroundjoin%
\definecolor{currentfill}{rgb}{0.121569,0.466667,0.705882}%
\pgfsetfillcolor{currentfill}%
\pgfsetfillopacity{0.323443}%
\pgfsetlinewidth{1.003750pt}%
\definecolor{currentstroke}{rgb}{0.121569,0.466667,0.705882}%
\pgfsetstrokecolor{currentstroke}%
\pgfsetstrokeopacity{0.323443}%
\pgfsetdash{}{0pt}%
\pgfpathmoveto{\pgfqpoint{1.926687in}{3.216425in}}%
\pgfpathcurveto{\pgfqpoint{1.934924in}{3.216425in}}{\pgfqpoint{1.942824in}{3.219697in}}{\pgfqpoint{1.948648in}{3.225521in}}%
\pgfpathcurveto{\pgfqpoint{1.954472in}{3.231345in}}{\pgfqpoint{1.957744in}{3.239245in}}{\pgfqpoint{1.957744in}{3.247481in}}%
\pgfpathcurveto{\pgfqpoint{1.957744in}{3.255718in}}{\pgfqpoint{1.954472in}{3.263618in}}{\pgfqpoint{1.948648in}{3.269441in}}%
\pgfpathcurveto{\pgfqpoint{1.942824in}{3.275265in}}{\pgfqpoint{1.934924in}{3.278538in}}{\pgfqpoint{1.926687in}{3.278538in}}%
\pgfpathcurveto{\pgfqpoint{1.918451in}{3.278538in}}{\pgfqpoint{1.910551in}{3.275265in}}{\pgfqpoint{1.904727in}{3.269441in}}%
\pgfpathcurveto{\pgfqpoint{1.898903in}{3.263618in}}{\pgfqpoint{1.895631in}{3.255718in}}{\pgfqpoint{1.895631in}{3.247481in}}%
\pgfpathcurveto{\pgfqpoint{1.895631in}{3.239245in}}{\pgfqpoint{1.898903in}{3.231345in}}{\pgfqpoint{1.904727in}{3.225521in}}%
\pgfpathcurveto{\pgfqpoint{1.910551in}{3.219697in}}{\pgfqpoint{1.918451in}{3.216425in}}{\pgfqpoint{1.926687in}{3.216425in}}%
\pgfpathclose%
\pgfusepath{stroke,fill}%
\end{pgfscope}%
\begin{pgfscope}%
\pgfpathrectangle{\pgfqpoint{0.100000in}{0.212622in}}{\pgfqpoint{3.696000in}{3.696000in}}%
\pgfusepath{clip}%
\pgfsetbuttcap%
\pgfsetroundjoin%
\definecolor{currentfill}{rgb}{0.121569,0.466667,0.705882}%
\pgfsetfillcolor{currentfill}%
\pgfsetfillopacity{0.323700}%
\pgfsetlinewidth{1.003750pt}%
\definecolor{currentstroke}{rgb}{0.121569,0.466667,0.705882}%
\pgfsetstrokecolor{currentstroke}%
\pgfsetstrokeopacity{0.323700}%
\pgfsetdash{}{0pt}%
\pgfpathmoveto{\pgfqpoint{1.799707in}{3.228978in}}%
\pgfpathcurveto{\pgfqpoint{1.807944in}{3.228978in}}{\pgfqpoint{1.815844in}{3.232251in}}{\pgfqpoint{1.821667in}{3.238075in}}%
\pgfpathcurveto{\pgfqpoint{1.827491in}{3.243899in}}{\pgfqpoint{1.830764in}{3.251799in}}{\pgfqpoint{1.830764in}{3.260035in}}%
\pgfpathcurveto{\pgfqpoint{1.830764in}{3.268271in}}{\pgfqpoint{1.827491in}{3.276171in}}{\pgfqpoint{1.821667in}{3.281995in}}%
\pgfpathcurveto{\pgfqpoint{1.815844in}{3.287819in}}{\pgfqpoint{1.807944in}{3.291091in}}{\pgfqpoint{1.799707in}{3.291091in}}%
\pgfpathcurveto{\pgfqpoint{1.791471in}{3.291091in}}{\pgfqpoint{1.783571in}{3.287819in}}{\pgfqpoint{1.777747in}{3.281995in}}%
\pgfpathcurveto{\pgfqpoint{1.771923in}{3.276171in}}{\pgfqpoint{1.768651in}{3.268271in}}{\pgfqpoint{1.768651in}{3.260035in}}%
\pgfpathcurveto{\pgfqpoint{1.768651in}{3.251799in}}{\pgfqpoint{1.771923in}{3.243899in}}{\pgfqpoint{1.777747in}{3.238075in}}%
\pgfpathcurveto{\pgfqpoint{1.783571in}{3.232251in}}{\pgfqpoint{1.791471in}{3.228978in}}{\pgfqpoint{1.799707in}{3.228978in}}%
\pgfpathclose%
\pgfusepath{stroke,fill}%
\end{pgfscope}%
\begin{pgfscope}%
\pgfpathrectangle{\pgfqpoint{0.100000in}{0.212622in}}{\pgfqpoint{3.696000in}{3.696000in}}%
\pgfusepath{clip}%
\pgfsetbuttcap%
\pgfsetroundjoin%
\definecolor{currentfill}{rgb}{0.121569,0.466667,0.705882}%
\pgfsetfillcolor{currentfill}%
\pgfsetfillopacity{0.323871}%
\pgfsetlinewidth{1.003750pt}%
\definecolor{currentstroke}{rgb}{0.121569,0.466667,0.705882}%
\pgfsetstrokecolor{currentstroke}%
\pgfsetstrokeopacity{0.323871}%
\pgfsetdash{}{0pt}%
\pgfpathmoveto{\pgfqpoint{1.927158in}{3.214237in}}%
\pgfpathcurveto{\pgfqpoint{1.935394in}{3.214237in}}{\pgfqpoint{1.943294in}{3.217510in}}{\pgfqpoint{1.949118in}{3.223334in}}%
\pgfpathcurveto{\pgfqpoint{1.954942in}{3.229157in}}{\pgfqpoint{1.958214in}{3.237058in}}{\pgfqpoint{1.958214in}{3.245294in}}%
\pgfpathcurveto{\pgfqpoint{1.958214in}{3.253530in}}{\pgfqpoint{1.954942in}{3.261430in}}{\pgfqpoint{1.949118in}{3.267254in}}%
\pgfpathcurveto{\pgfqpoint{1.943294in}{3.273078in}}{\pgfqpoint{1.935394in}{3.276350in}}{\pgfqpoint{1.927158in}{3.276350in}}%
\pgfpathcurveto{\pgfqpoint{1.918921in}{3.276350in}}{\pgfqpoint{1.911021in}{3.273078in}}{\pgfqpoint{1.905197in}{3.267254in}}%
\pgfpathcurveto{\pgfqpoint{1.899373in}{3.261430in}}{\pgfqpoint{1.896101in}{3.253530in}}{\pgfqpoint{1.896101in}{3.245294in}}%
\pgfpathcurveto{\pgfqpoint{1.896101in}{3.237058in}}{\pgfqpoint{1.899373in}{3.229157in}}{\pgfqpoint{1.905197in}{3.223334in}}%
\pgfpathcurveto{\pgfqpoint{1.911021in}{3.217510in}}{\pgfqpoint{1.918921in}{3.214237in}}{\pgfqpoint{1.927158in}{3.214237in}}%
\pgfpathclose%
\pgfusepath{stroke,fill}%
\end{pgfscope}%
\begin{pgfscope}%
\pgfpathrectangle{\pgfqpoint{0.100000in}{0.212622in}}{\pgfqpoint{3.696000in}{3.696000in}}%
\pgfusepath{clip}%
\pgfsetbuttcap%
\pgfsetroundjoin%
\definecolor{currentfill}{rgb}{0.121569,0.466667,0.705882}%
\pgfsetfillcolor{currentfill}%
\pgfsetfillopacity{0.324162}%
\pgfsetlinewidth{1.003750pt}%
\definecolor{currentstroke}{rgb}{0.121569,0.466667,0.705882}%
\pgfsetstrokecolor{currentstroke}%
\pgfsetstrokeopacity{0.324162}%
\pgfsetdash{}{0pt}%
\pgfpathmoveto{\pgfqpoint{1.927313in}{3.213161in}}%
\pgfpathcurveto{\pgfqpoint{1.935550in}{3.213161in}}{\pgfqpoint{1.943450in}{3.216433in}}{\pgfqpoint{1.949274in}{3.222257in}}%
\pgfpathcurveto{\pgfqpoint{1.955098in}{3.228081in}}{\pgfqpoint{1.958370in}{3.235981in}}{\pgfqpoint{1.958370in}{3.244217in}}%
\pgfpathcurveto{\pgfqpoint{1.958370in}{3.252453in}}{\pgfqpoint{1.955098in}{3.260353in}}{\pgfqpoint{1.949274in}{3.266177in}}%
\pgfpathcurveto{\pgfqpoint{1.943450in}{3.272001in}}{\pgfqpoint{1.935550in}{3.275274in}}{\pgfqpoint{1.927313in}{3.275274in}}%
\pgfpathcurveto{\pgfqpoint{1.919077in}{3.275274in}}{\pgfqpoint{1.911177in}{3.272001in}}{\pgfqpoint{1.905353in}{3.266177in}}%
\pgfpathcurveto{\pgfqpoint{1.899529in}{3.260353in}}{\pgfqpoint{1.896257in}{3.252453in}}{\pgfqpoint{1.896257in}{3.244217in}}%
\pgfpathcurveto{\pgfqpoint{1.896257in}{3.235981in}}{\pgfqpoint{1.899529in}{3.228081in}}{\pgfqpoint{1.905353in}{3.222257in}}%
\pgfpathcurveto{\pgfqpoint{1.911177in}{3.216433in}}{\pgfqpoint{1.919077in}{3.213161in}}{\pgfqpoint{1.927313in}{3.213161in}}%
\pgfpathclose%
\pgfusepath{stroke,fill}%
\end{pgfscope}%
\begin{pgfscope}%
\pgfpathrectangle{\pgfqpoint{0.100000in}{0.212622in}}{\pgfqpoint{3.696000in}{3.696000in}}%
\pgfusepath{clip}%
\pgfsetbuttcap%
\pgfsetroundjoin%
\definecolor{currentfill}{rgb}{0.121569,0.466667,0.705882}%
\pgfsetfillcolor{currentfill}%
\pgfsetfillopacity{0.324723}%
\pgfsetlinewidth{1.003750pt}%
\definecolor{currentstroke}{rgb}{0.121569,0.466667,0.705882}%
\pgfsetstrokecolor{currentstroke}%
\pgfsetstrokeopacity{0.324723}%
\pgfsetdash{}{0pt}%
\pgfpathmoveto{\pgfqpoint{1.927882in}{3.210388in}}%
\pgfpathcurveto{\pgfqpoint{1.936118in}{3.210388in}}{\pgfqpoint{1.944018in}{3.213661in}}{\pgfqpoint{1.949842in}{3.219485in}}%
\pgfpathcurveto{\pgfqpoint{1.955666in}{3.225308in}}{\pgfqpoint{1.958939in}{3.233209in}}{\pgfqpoint{1.958939in}{3.241445in}}%
\pgfpathcurveto{\pgfqpoint{1.958939in}{3.249681in}}{\pgfqpoint{1.955666in}{3.257581in}}{\pgfqpoint{1.949842in}{3.263405in}}%
\pgfpathcurveto{\pgfqpoint{1.944018in}{3.269229in}}{\pgfqpoint{1.936118in}{3.272501in}}{\pgfqpoint{1.927882in}{3.272501in}}%
\pgfpathcurveto{\pgfqpoint{1.919646in}{3.272501in}}{\pgfqpoint{1.911746in}{3.269229in}}{\pgfqpoint{1.905922in}{3.263405in}}%
\pgfpathcurveto{\pgfqpoint{1.900098in}{3.257581in}}{\pgfqpoint{1.896826in}{3.249681in}}{\pgfqpoint{1.896826in}{3.241445in}}%
\pgfpathcurveto{\pgfqpoint{1.896826in}{3.233209in}}{\pgfqpoint{1.900098in}{3.225308in}}{\pgfqpoint{1.905922in}{3.219485in}}%
\pgfpathcurveto{\pgfqpoint{1.911746in}{3.213661in}}{\pgfqpoint{1.919646in}{3.210388in}}{\pgfqpoint{1.927882in}{3.210388in}}%
\pgfpathclose%
\pgfusepath{stroke,fill}%
\end{pgfscope}%
\begin{pgfscope}%
\pgfpathrectangle{\pgfqpoint{0.100000in}{0.212622in}}{\pgfqpoint{3.696000in}{3.696000in}}%
\pgfusepath{clip}%
\pgfsetbuttcap%
\pgfsetroundjoin%
\definecolor{currentfill}{rgb}{0.121569,0.466667,0.705882}%
\pgfsetfillcolor{currentfill}%
\pgfsetfillopacity{0.325066}%
\pgfsetlinewidth{1.003750pt}%
\definecolor{currentstroke}{rgb}{0.121569,0.466667,0.705882}%
\pgfsetstrokecolor{currentstroke}%
\pgfsetstrokeopacity{0.325066}%
\pgfsetdash{}{0pt}%
\pgfpathmoveto{\pgfqpoint{1.928142in}{3.208948in}}%
\pgfpathcurveto{\pgfqpoint{1.936378in}{3.208948in}}{\pgfqpoint{1.944278in}{3.212220in}}{\pgfqpoint{1.950102in}{3.218044in}}%
\pgfpathcurveto{\pgfqpoint{1.955926in}{3.223868in}}{\pgfqpoint{1.959198in}{3.231768in}}{\pgfqpoint{1.959198in}{3.240004in}}%
\pgfpathcurveto{\pgfqpoint{1.959198in}{3.248240in}}{\pgfqpoint{1.955926in}{3.256140in}}{\pgfqpoint{1.950102in}{3.261964in}}%
\pgfpathcurveto{\pgfqpoint{1.944278in}{3.267788in}}{\pgfqpoint{1.936378in}{3.271061in}}{\pgfqpoint{1.928142in}{3.271061in}}%
\pgfpathcurveto{\pgfqpoint{1.919906in}{3.271061in}}{\pgfqpoint{1.912005in}{3.267788in}}{\pgfqpoint{1.906182in}{3.261964in}}%
\pgfpathcurveto{\pgfqpoint{1.900358in}{3.256140in}}{\pgfqpoint{1.897085in}{3.248240in}}{\pgfqpoint{1.897085in}{3.240004in}}%
\pgfpathcurveto{\pgfqpoint{1.897085in}{3.231768in}}{\pgfqpoint{1.900358in}{3.223868in}}{\pgfqpoint{1.906182in}{3.218044in}}%
\pgfpathcurveto{\pgfqpoint{1.912005in}{3.212220in}}{\pgfqpoint{1.919906in}{3.208948in}}{\pgfqpoint{1.928142in}{3.208948in}}%
\pgfpathclose%
\pgfusepath{stroke,fill}%
\end{pgfscope}%
\begin{pgfscope}%
\pgfpathrectangle{\pgfqpoint{0.100000in}{0.212622in}}{\pgfqpoint{3.696000in}{3.696000in}}%
\pgfusepath{clip}%
\pgfsetbuttcap%
\pgfsetroundjoin%
\definecolor{currentfill}{rgb}{0.121569,0.466667,0.705882}%
\pgfsetfillcolor{currentfill}%
\pgfsetfillopacity{0.325189}%
\pgfsetlinewidth{1.003750pt}%
\definecolor{currentstroke}{rgb}{0.121569,0.466667,0.705882}%
\pgfsetstrokecolor{currentstroke}%
\pgfsetstrokeopacity{0.325189}%
\pgfsetdash{}{0pt}%
\pgfpathmoveto{\pgfqpoint{1.795877in}{3.222016in}}%
\pgfpathcurveto{\pgfqpoint{1.804113in}{3.222016in}}{\pgfqpoint{1.812013in}{3.225289in}}{\pgfqpoint{1.817837in}{3.231113in}}%
\pgfpathcurveto{\pgfqpoint{1.823661in}{3.236937in}}{\pgfqpoint{1.826934in}{3.244837in}}{\pgfqpoint{1.826934in}{3.253073in}}%
\pgfpathcurveto{\pgfqpoint{1.826934in}{3.261309in}}{\pgfqpoint{1.823661in}{3.269209in}}{\pgfqpoint{1.817837in}{3.275033in}}%
\pgfpathcurveto{\pgfqpoint{1.812013in}{3.280857in}}{\pgfqpoint{1.804113in}{3.284129in}}{\pgfqpoint{1.795877in}{3.284129in}}%
\pgfpathcurveto{\pgfqpoint{1.787641in}{3.284129in}}{\pgfqpoint{1.779741in}{3.280857in}}{\pgfqpoint{1.773917in}{3.275033in}}%
\pgfpathcurveto{\pgfqpoint{1.768093in}{3.269209in}}{\pgfqpoint{1.764821in}{3.261309in}}{\pgfqpoint{1.764821in}{3.253073in}}%
\pgfpathcurveto{\pgfqpoint{1.764821in}{3.244837in}}{\pgfqpoint{1.768093in}{3.236937in}}{\pgfqpoint{1.773917in}{3.231113in}}%
\pgfpathcurveto{\pgfqpoint{1.779741in}{3.225289in}}{\pgfqpoint{1.787641in}{3.222016in}}{\pgfqpoint{1.795877in}{3.222016in}}%
\pgfpathclose%
\pgfusepath{stroke,fill}%
\end{pgfscope}%
\begin{pgfscope}%
\pgfpathrectangle{\pgfqpoint{0.100000in}{0.212622in}}{\pgfqpoint{3.696000in}{3.696000in}}%
\pgfusepath{clip}%
\pgfsetbuttcap%
\pgfsetroundjoin%
\definecolor{currentfill}{rgb}{0.121569,0.466667,0.705882}%
\pgfsetfillcolor{currentfill}%
\pgfsetfillopacity{0.325596}%
\pgfsetlinewidth{1.003750pt}%
\definecolor{currentstroke}{rgb}{0.121569,0.466667,0.705882}%
\pgfsetstrokecolor{currentstroke}%
\pgfsetstrokeopacity{0.325596}%
\pgfsetdash{}{0pt}%
\pgfpathmoveto{\pgfqpoint{1.928476in}{3.206945in}}%
\pgfpathcurveto{\pgfqpoint{1.936713in}{3.206945in}}{\pgfqpoint{1.944613in}{3.210217in}}{\pgfqpoint{1.950437in}{3.216041in}}%
\pgfpathcurveto{\pgfqpoint{1.956260in}{3.221865in}}{\pgfqpoint{1.959533in}{3.229765in}}{\pgfqpoint{1.959533in}{3.238001in}}%
\pgfpathcurveto{\pgfqpoint{1.959533in}{3.246238in}}{\pgfqpoint{1.956260in}{3.254138in}}{\pgfqpoint{1.950437in}{3.259962in}}%
\pgfpathcurveto{\pgfqpoint{1.944613in}{3.265786in}}{\pgfqpoint{1.936713in}{3.269058in}}{\pgfqpoint{1.928476in}{3.269058in}}%
\pgfpathcurveto{\pgfqpoint{1.920240in}{3.269058in}}{\pgfqpoint{1.912340in}{3.265786in}}{\pgfqpoint{1.906516in}{3.259962in}}%
\pgfpathcurveto{\pgfqpoint{1.900692in}{3.254138in}}{\pgfqpoint{1.897420in}{3.246238in}}{\pgfqpoint{1.897420in}{3.238001in}}%
\pgfpathcurveto{\pgfqpoint{1.897420in}{3.229765in}}{\pgfqpoint{1.900692in}{3.221865in}}{\pgfqpoint{1.906516in}{3.216041in}}%
\pgfpathcurveto{\pgfqpoint{1.912340in}{3.210217in}}{\pgfqpoint{1.920240in}{3.206945in}}{\pgfqpoint{1.928476in}{3.206945in}}%
\pgfpathclose%
\pgfusepath{stroke,fill}%
\end{pgfscope}%
\begin{pgfscope}%
\pgfpathrectangle{\pgfqpoint{0.100000in}{0.212622in}}{\pgfqpoint{3.696000in}{3.696000in}}%
\pgfusepath{clip}%
\pgfsetbuttcap%
\pgfsetroundjoin%
\definecolor{currentfill}{rgb}{0.121569,0.466667,0.705882}%
\pgfsetfillcolor{currentfill}%
\pgfsetfillopacity{0.326077}%
\pgfsetlinewidth{1.003750pt}%
\definecolor{currentstroke}{rgb}{0.121569,0.466667,0.705882}%
\pgfsetstrokecolor{currentstroke}%
\pgfsetstrokeopacity{0.326077}%
\pgfsetdash{}{0pt}%
\pgfpathmoveto{\pgfqpoint{1.929098in}{3.204421in}}%
\pgfpathcurveto{\pgfqpoint{1.937334in}{3.204421in}}{\pgfqpoint{1.945234in}{3.207693in}}{\pgfqpoint{1.951058in}{3.213517in}}%
\pgfpathcurveto{\pgfqpoint{1.956882in}{3.219341in}}{\pgfqpoint{1.960154in}{3.227241in}}{\pgfqpoint{1.960154in}{3.235477in}}%
\pgfpathcurveto{\pgfqpoint{1.960154in}{3.243713in}}{\pgfqpoint{1.956882in}{3.251614in}}{\pgfqpoint{1.951058in}{3.257437in}}%
\pgfpathcurveto{\pgfqpoint{1.945234in}{3.263261in}}{\pgfqpoint{1.937334in}{3.266534in}}{\pgfqpoint{1.929098in}{3.266534in}}%
\pgfpathcurveto{\pgfqpoint{1.920861in}{3.266534in}}{\pgfqpoint{1.912961in}{3.263261in}}{\pgfqpoint{1.907137in}{3.257437in}}%
\pgfpathcurveto{\pgfqpoint{1.901313in}{3.251614in}}{\pgfqpoint{1.898041in}{3.243713in}}{\pgfqpoint{1.898041in}{3.235477in}}%
\pgfpathcurveto{\pgfqpoint{1.898041in}{3.227241in}}{\pgfqpoint{1.901313in}{3.219341in}}{\pgfqpoint{1.907137in}{3.213517in}}%
\pgfpathcurveto{\pgfqpoint{1.912961in}{3.207693in}}{\pgfqpoint{1.920861in}{3.204421in}}{\pgfqpoint{1.929098in}{3.204421in}}%
\pgfpathclose%
\pgfusepath{stroke,fill}%
\end{pgfscope}%
\begin{pgfscope}%
\pgfpathrectangle{\pgfqpoint{0.100000in}{0.212622in}}{\pgfqpoint{3.696000in}{3.696000in}}%
\pgfusepath{clip}%
\pgfsetbuttcap%
\pgfsetroundjoin%
\definecolor{currentfill}{rgb}{0.121569,0.466667,0.705882}%
\pgfsetfillcolor{currentfill}%
\pgfsetfillopacity{0.326363}%
\pgfsetlinewidth{1.003750pt}%
\definecolor{currentstroke}{rgb}{0.121569,0.466667,0.705882}%
\pgfsetstrokecolor{currentstroke}%
\pgfsetstrokeopacity{0.326363}%
\pgfsetdash{}{0pt}%
\pgfpathmoveto{\pgfqpoint{1.791965in}{3.215838in}}%
\pgfpathcurveto{\pgfqpoint{1.800202in}{3.215838in}}{\pgfqpoint{1.808102in}{3.219110in}}{\pgfqpoint{1.813926in}{3.224934in}}%
\pgfpathcurveto{\pgfqpoint{1.819749in}{3.230758in}}{\pgfqpoint{1.823022in}{3.238658in}}{\pgfqpoint{1.823022in}{3.246894in}}%
\pgfpathcurveto{\pgfqpoint{1.823022in}{3.255130in}}{\pgfqpoint{1.819749in}{3.263030in}}{\pgfqpoint{1.813926in}{3.268854in}}%
\pgfpathcurveto{\pgfqpoint{1.808102in}{3.274678in}}{\pgfqpoint{1.800202in}{3.277951in}}{\pgfqpoint{1.791965in}{3.277951in}}%
\pgfpathcurveto{\pgfqpoint{1.783729in}{3.277951in}}{\pgfqpoint{1.775829in}{3.274678in}}{\pgfqpoint{1.770005in}{3.268854in}}%
\pgfpathcurveto{\pgfqpoint{1.764181in}{3.263030in}}{\pgfqpoint{1.760909in}{3.255130in}}{\pgfqpoint{1.760909in}{3.246894in}}%
\pgfpathcurveto{\pgfqpoint{1.760909in}{3.238658in}}{\pgfqpoint{1.764181in}{3.230758in}}{\pgfqpoint{1.770005in}{3.224934in}}%
\pgfpathcurveto{\pgfqpoint{1.775829in}{3.219110in}}{\pgfqpoint{1.783729in}{3.215838in}}{\pgfqpoint{1.791965in}{3.215838in}}%
\pgfpathclose%
\pgfusepath{stroke,fill}%
\end{pgfscope}%
\begin{pgfscope}%
\pgfpathrectangle{\pgfqpoint{0.100000in}{0.212622in}}{\pgfqpoint{3.696000in}{3.696000in}}%
\pgfusepath{clip}%
\pgfsetbuttcap%
\pgfsetroundjoin%
\definecolor{currentfill}{rgb}{0.121569,0.466667,0.705882}%
\pgfsetfillcolor{currentfill}%
\pgfsetfillopacity{0.327183}%
\pgfsetlinewidth{1.003750pt}%
\definecolor{currentstroke}{rgb}{0.121569,0.466667,0.705882}%
\pgfsetstrokecolor{currentstroke}%
\pgfsetstrokeopacity{0.327183}%
\pgfsetdash{}{0pt}%
\pgfpathmoveto{\pgfqpoint{1.929839in}{3.199893in}}%
\pgfpathcurveto{\pgfqpoint{1.938075in}{3.199893in}}{\pgfqpoint{1.945975in}{3.203165in}}{\pgfqpoint{1.951799in}{3.208989in}}%
\pgfpathcurveto{\pgfqpoint{1.957623in}{3.214813in}}{\pgfqpoint{1.960895in}{3.222713in}}{\pgfqpoint{1.960895in}{3.230950in}}%
\pgfpathcurveto{\pgfqpoint{1.960895in}{3.239186in}}{\pgfqpoint{1.957623in}{3.247086in}}{\pgfqpoint{1.951799in}{3.252910in}}%
\pgfpathcurveto{\pgfqpoint{1.945975in}{3.258734in}}{\pgfqpoint{1.938075in}{3.262006in}}{\pgfqpoint{1.929839in}{3.262006in}}%
\pgfpathcurveto{\pgfqpoint{1.921603in}{3.262006in}}{\pgfqpoint{1.913703in}{3.258734in}}{\pgfqpoint{1.907879in}{3.252910in}}%
\pgfpathcurveto{\pgfqpoint{1.902055in}{3.247086in}}{\pgfqpoint{1.898782in}{3.239186in}}{\pgfqpoint{1.898782in}{3.230950in}}%
\pgfpathcurveto{\pgfqpoint{1.898782in}{3.222713in}}{\pgfqpoint{1.902055in}{3.214813in}}{\pgfqpoint{1.907879in}{3.208989in}}%
\pgfpathcurveto{\pgfqpoint{1.913703in}{3.203165in}}{\pgfqpoint{1.921603in}{3.199893in}}{\pgfqpoint{1.929839in}{3.199893in}}%
\pgfpathclose%
\pgfusepath{stroke,fill}%
\end{pgfscope}%
\begin{pgfscope}%
\pgfpathrectangle{\pgfqpoint{0.100000in}{0.212622in}}{\pgfqpoint{3.696000in}{3.696000in}}%
\pgfusepath{clip}%
\pgfsetbuttcap%
\pgfsetroundjoin%
\definecolor{currentfill}{rgb}{0.121569,0.466667,0.705882}%
\pgfsetfillcolor{currentfill}%
\pgfsetfillopacity{0.327368}%
\pgfsetlinewidth{1.003750pt}%
\definecolor{currentstroke}{rgb}{0.121569,0.466667,0.705882}%
\pgfsetstrokecolor{currentstroke}%
\pgfsetstrokeopacity{0.327368}%
\pgfsetdash{}{0pt}%
\pgfpathmoveto{\pgfqpoint{1.789406in}{3.210406in}}%
\pgfpathcurveto{\pgfqpoint{1.797642in}{3.210406in}}{\pgfqpoint{1.805542in}{3.213679in}}{\pgfqpoint{1.811366in}{3.219503in}}%
\pgfpathcurveto{\pgfqpoint{1.817190in}{3.225327in}}{\pgfqpoint{1.820462in}{3.233227in}}{\pgfqpoint{1.820462in}{3.241463in}}%
\pgfpathcurveto{\pgfqpoint{1.820462in}{3.249699in}}{\pgfqpoint{1.817190in}{3.257599in}}{\pgfqpoint{1.811366in}{3.263423in}}%
\pgfpathcurveto{\pgfqpoint{1.805542in}{3.269247in}}{\pgfqpoint{1.797642in}{3.272519in}}{\pgfqpoint{1.789406in}{3.272519in}}%
\pgfpathcurveto{\pgfqpoint{1.781170in}{3.272519in}}{\pgfqpoint{1.773270in}{3.269247in}}{\pgfqpoint{1.767446in}{3.263423in}}%
\pgfpathcurveto{\pgfqpoint{1.761622in}{3.257599in}}{\pgfqpoint{1.758349in}{3.249699in}}{\pgfqpoint{1.758349in}{3.241463in}}%
\pgfpathcurveto{\pgfqpoint{1.758349in}{3.233227in}}{\pgfqpoint{1.761622in}{3.225327in}}{\pgfqpoint{1.767446in}{3.219503in}}%
\pgfpathcurveto{\pgfqpoint{1.773270in}{3.213679in}}{\pgfqpoint{1.781170in}{3.210406in}}{\pgfqpoint{1.789406in}{3.210406in}}%
\pgfpathclose%
\pgfusepath{stroke,fill}%
\end{pgfscope}%
\begin{pgfscope}%
\pgfpathrectangle{\pgfqpoint{0.100000in}{0.212622in}}{\pgfqpoint{3.696000in}{3.696000in}}%
\pgfusepath{clip}%
\pgfsetbuttcap%
\pgfsetroundjoin%
\definecolor{currentfill}{rgb}{0.121569,0.466667,0.705882}%
\pgfsetfillcolor{currentfill}%
\pgfsetfillopacity{0.328135}%
\pgfsetlinewidth{1.003750pt}%
\definecolor{currentstroke}{rgb}{0.121569,0.466667,0.705882}%
\pgfsetstrokecolor{currentstroke}%
\pgfsetstrokeopacity{0.328135}%
\pgfsetdash{}{0pt}%
\pgfpathmoveto{\pgfqpoint{1.787108in}{3.206793in}}%
\pgfpathcurveto{\pgfqpoint{1.795344in}{3.206793in}}{\pgfqpoint{1.803244in}{3.210066in}}{\pgfqpoint{1.809068in}{3.215889in}}%
\pgfpathcurveto{\pgfqpoint{1.814892in}{3.221713in}}{\pgfqpoint{1.818164in}{3.229613in}}{\pgfqpoint{1.818164in}{3.237850in}}%
\pgfpathcurveto{\pgfqpoint{1.818164in}{3.246086in}}{\pgfqpoint{1.814892in}{3.253986in}}{\pgfqpoint{1.809068in}{3.259810in}}%
\pgfpathcurveto{\pgfqpoint{1.803244in}{3.265634in}}{\pgfqpoint{1.795344in}{3.268906in}}{\pgfqpoint{1.787108in}{3.268906in}}%
\pgfpathcurveto{\pgfqpoint{1.778871in}{3.268906in}}{\pgfqpoint{1.770971in}{3.265634in}}{\pgfqpoint{1.765147in}{3.259810in}}%
\pgfpathcurveto{\pgfqpoint{1.759324in}{3.253986in}}{\pgfqpoint{1.756051in}{3.246086in}}{\pgfqpoint{1.756051in}{3.237850in}}%
\pgfpathcurveto{\pgfqpoint{1.756051in}{3.229613in}}{\pgfqpoint{1.759324in}{3.221713in}}{\pgfqpoint{1.765147in}{3.215889in}}%
\pgfpathcurveto{\pgfqpoint{1.770971in}{3.210066in}}{\pgfqpoint{1.778871in}{3.206793in}}{\pgfqpoint{1.787108in}{3.206793in}}%
\pgfpathclose%
\pgfusepath{stroke,fill}%
\end{pgfscope}%
\begin{pgfscope}%
\pgfpathrectangle{\pgfqpoint{0.100000in}{0.212622in}}{\pgfqpoint{3.696000in}{3.696000in}}%
\pgfusepath{clip}%
\pgfsetbuttcap%
\pgfsetroundjoin%
\definecolor{currentfill}{rgb}{0.121569,0.466667,0.705882}%
\pgfsetfillcolor{currentfill}%
\pgfsetfillopacity{0.328412}%
\pgfsetlinewidth{1.003750pt}%
\definecolor{currentstroke}{rgb}{0.121569,0.466667,0.705882}%
\pgfsetstrokecolor{currentstroke}%
\pgfsetstrokeopacity{0.328412}%
\pgfsetdash{}{0pt}%
\pgfpathmoveto{\pgfqpoint{1.930690in}{3.195076in}}%
\pgfpathcurveto{\pgfqpoint{1.938926in}{3.195076in}}{\pgfqpoint{1.946826in}{3.198348in}}{\pgfqpoint{1.952650in}{3.204172in}}%
\pgfpathcurveto{\pgfqpoint{1.958474in}{3.209996in}}{\pgfqpoint{1.961746in}{3.217896in}}{\pgfqpoint{1.961746in}{3.226133in}}%
\pgfpathcurveto{\pgfqpoint{1.961746in}{3.234369in}}{\pgfqpoint{1.958474in}{3.242269in}}{\pgfqpoint{1.952650in}{3.248093in}}%
\pgfpathcurveto{\pgfqpoint{1.946826in}{3.253917in}}{\pgfqpoint{1.938926in}{3.257189in}}{\pgfqpoint{1.930690in}{3.257189in}}%
\pgfpathcurveto{\pgfqpoint{1.922453in}{3.257189in}}{\pgfqpoint{1.914553in}{3.253917in}}{\pgfqpoint{1.908729in}{3.248093in}}%
\pgfpathcurveto{\pgfqpoint{1.902905in}{3.242269in}}{\pgfqpoint{1.899633in}{3.234369in}}{\pgfqpoint{1.899633in}{3.226133in}}%
\pgfpathcurveto{\pgfqpoint{1.899633in}{3.217896in}}{\pgfqpoint{1.902905in}{3.209996in}}{\pgfqpoint{1.908729in}{3.204172in}}%
\pgfpathcurveto{\pgfqpoint{1.914553in}{3.198348in}}{\pgfqpoint{1.922453in}{3.195076in}}{\pgfqpoint{1.930690in}{3.195076in}}%
\pgfpathclose%
\pgfusepath{stroke,fill}%
\end{pgfscope}%
\begin{pgfscope}%
\pgfpathrectangle{\pgfqpoint{0.100000in}{0.212622in}}{\pgfqpoint{3.696000in}{3.696000in}}%
\pgfusepath{clip}%
\pgfsetbuttcap%
\pgfsetroundjoin%
\definecolor{currentfill}{rgb}{0.121569,0.466667,0.705882}%
\pgfsetfillcolor{currentfill}%
\pgfsetfillopacity{0.329505}%
\pgfsetlinewidth{1.003750pt}%
\definecolor{currentstroke}{rgb}{0.121569,0.466667,0.705882}%
\pgfsetstrokecolor{currentstroke}%
\pgfsetstrokeopacity{0.329505}%
\pgfsetdash{}{0pt}%
\pgfpathmoveto{\pgfqpoint{1.782896in}{3.200171in}}%
\pgfpathcurveto{\pgfqpoint{1.791132in}{3.200171in}}{\pgfqpoint{1.799032in}{3.203443in}}{\pgfqpoint{1.804856in}{3.209267in}}%
\pgfpathcurveto{\pgfqpoint{1.810680in}{3.215091in}}{\pgfqpoint{1.813952in}{3.222991in}}{\pgfqpoint{1.813952in}{3.231227in}}%
\pgfpathcurveto{\pgfqpoint{1.813952in}{3.239464in}}{\pgfqpoint{1.810680in}{3.247364in}}{\pgfqpoint{1.804856in}{3.253188in}}%
\pgfpathcurveto{\pgfqpoint{1.799032in}{3.259012in}}{\pgfqpoint{1.791132in}{3.262284in}}{\pgfqpoint{1.782896in}{3.262284in}}%
\pgfpathcurveto{\pgfqpoint{1.774660in}{3.262284in}}{\pgfqpoint{1.766760in}{3.259012in}}{\pgfqpoint{1.760936in}{3.253188in}}%
\pgfpathcurveto{\pgfqpoint{1.755112in}{3.247364in}}{\pgfqpoint{1.751839in}{3.239464in}}{\pgfqpoint{1.751839in}{3.231227in}}%
\pgfpathcurveto{\pgfqpoint{1.751839in}{3.222991in}}{\pgfqpoint{1.755112in}{3.215091in}}{\pgfqpoint{1.760936in}{3.209267in}}%
\pgfpathcurveto{\pgfqpoint{1.766760in}{3.203443in}}{\pgfqpoint{1.774660in}{3.200171in}}{\pgfqpoint{1.782896in}{3.200171in}}%
\pgfpathclose%
\pgfusepath{stroke,fill}%
\end{pgfscope}%
\begin{pgfscope}%
\pgfpathrectangle{\pgfqpoint{0.100000in}{0.212622in}}{\pgfqpoint{3.696000in}{3.696000in}}%
\pgfusepath{clip}%
\pgfsetbuttcap%
\pgfsetroundjoin%
\definecolor{currentfill}{rgb}{0.121569,0.466667,0.705882}%
\pgfsetfillcolor{currentfill}%
\pgfsetfillopacity{0.329536}%
\pgfsetlinewidth{1.003750pt}%
\definecolor{currentstroke}{rgb}{0.121569,0.466667,0.705882}%
\pgfsetstrokecolor{currentstroke}%
\pgfsetstrokeopacity{0.329536}%
\pgfsetdash{}{0pt}%
\pgfpathmoveto{\pgfqpoint{1.931998in}{3.189069in}}%
\pgfpathcurveto{\pgfqpoint{1.940235in}{3.189069in}}{\pgfqpoint{1.948135in}{3.192341in}}{\pgfqpoint{1.953959in}{3.198165in}}%
\pgfpathcurveto{\pgfqpoint{1.959783in}{3.203989in}}{\pgfqpoint{1.963055in}{3.211889in}}{\pgfqpoint{1.963055in}{3.220125in}}%
\pgfpathcurveto{\pgfqpoint{1.963055in}{3.228361in}}{\pgfqpoint{1.959783in}{3.236261in}}{\pgfqpoint{1.953959in}{3.242085in}}%
\pgfpathcurveto{\pgfqpoint{1.948135in}{3.247909in}}{\pgfqpoint{1.940235in}{3.251182in}}{\pgfqpoint{1.931998in}{3.251182in}}%
\pgfpathcurveto{\pgfqpoint{1.923762in}{3.251182in}}{\pgfqpoint{1.915862in}{3.247909in}}{\pgfqpoint{1.910038in}{3.242085in}}%
\pgfpathcurveto{\pgfqpoint{1.904214in}{3.236261in}}{\pgfqpoint{1.900942in}{3.228361in}}{\pgfqpoint{1.900942in}{3.220125in}}%
\pgfpathcurveto{\pgfqpoint{1.900942in}{3.211889in}}{\pgfqpoint{1.904214in}{3.203989in}}{\pgfqpoint{1.910038in}{3.198165in}}%
\pgfpathcurveto{\pgfqpoint{1.915862in}{3.192341in}}{\pgfqpoint{1.923762in}{3.189069in}}{\pgfqpoint{1.931998in}{3.189069in}}%
\pgfpathclose%
\pgfusepath{stroke,fill}%
\end{pgfscope}%
\begin{pgfscope}%
\pgfpathrectangle{\pgfqpoint{0.100000in}{0.212622in}}{\pgfqpoint{3.696000in}{3.696000in}}%
\pgfusepath{clip}%
\pgfsetbuttcap%
\pgfsetroundjoin%
\definecolor{currentfill}{rgb}{0.121569,0.466667,0.705882}%
\pgfsetfillcolor{currentfill}%
\pgfsetfillopacity{0.330697}%
\pgfsetlinewidth{1.003750pt}%
\definecolor{currentstroke}{rgb}{0.121569,0.466667,0.705882}%
\pgfsetstrokecolor{currentstroke}%
\pgfsetstrokeopacity{0.330697}%
\pgfsetdash{}{0pt}%
\pgfpathmoveto{\pgfqpoint{1.779697in}{3.193919in}}%
\pgfpathcurveto{\pgfqpoint{1.787934in}{3.193919in}}{\pgfqpoint{1.795834in}{3.197192in}}{\pgfqpoint{1.801658in}{3.203016in}}%
\pgfpathcurveto{\pgfqpoint{1.807482in}{3.208840in}}{\pgfqpoint{1.810754in}{3.216740in}}{\pgfqpoint{1.810754in}{3.224976in}}%
\pgfpathcurveto{\pgfqpoint{1.810754in}{3.233212in}}{\pgfqpoint{1.807482in}{3.241112in}}{\pgfqpoint{1.801658in}{3.246936in}}%
\pgfpathcurveto{\pgfqpoint{1.795834in}{3.252760in}}{\pgfqpoint{1.787934in}{3.256032in}}{\pgfqpoint{1.779697in}{3.256032in}}%
\pgfpathcurveto{\pgfqpoint{1.771461in}{3.256032in}}{\pgfqpoint{1.763561in}{3.252760in}}{\pgfqpoint{1.757737in}{3.246936in}}%
\pgfpathcurveto{\pgfqpoint{1.751913in}{3.241112in}}{\pgfqpoint{1.748641in}{3.233212in}}{\pgfqpoint{1.748641in}{3.224976in}}%
\pgfpathcurveto{\pgfqpoint{1.748641in}{3.216740in}}{\pgfqpoint{1.751913in}{3.208840in}}{\pgfqpoint{1.757737in}{3.203016in}}%
\pgfpathcurveto{\pgfqpoint{1.763561in}{3.197192in}}{\pgfqpoint{1.771461in}{3.193919in}}{\pgfqpoint{1.779697in}{3.193919in}}%
\pgfpathclose%
\pgfusepath{stroke,fill}%
\end{pgfscope}%
\begin{pgfscope}%
\pgfpathrectangle{\pgfqpoint{0.100000in}{0.212622in}}{\pgfqpoint{3.696000in}{3.696000in}}%
\pgfusepath{clip}%
\pgfsetbuttcap%
\pgfsetroundjoin%
\definecolor{currentfill}{rgb}{0.121569,0.466667,0.705882}%
\pgfsetfillcolor{currentfill}%
\pgfsetfillopacity{0.331441}%
\pgfsetlinewidth{1.003750pt}%
\definecolor{currentstroke}{rgb}{0.121569,0.466667,0.705882}%
\pgfsetstrokecolor{currentstroke}%
\pgfsetstrokeopacity{0.331441}%
\pgfsetdash{}{0pt}%
\pgfpathmoveto{\pgfqpoint{1.776851in}{3.189316in}}%
\pgfpathcurveto{\pgfqpoint{1.785087in}{3.189316in}}{\pgfqpoint{1.792987in}{3.192588in}}{\pgfqpoint{1.798811in}{3.198412in}}%
\pgfpathcurveto{\pgfqpoint{1.804635in}{3.204236in}}{\pgfqpoint{1.807907in}{3.212136in}}{\pgfqpoint{1.807907in}{3.220373in}}%
\pgfpathcurveto{\pgfqpoint{1.807907in}{3.228609in}}{\pgfqpoint{1.804635in}{3.236509in}}{\pgfqpoint{1.798811in}{3.242333in}}%
\pgfpathcurveto{\pgfqpoint{1.792987in}{3.248157in}}{\pgfqpoint{1.785087in}{3.251429in}}{\pgfqpoint{1.776851in}{3.251429in}}%
\pgfpathcurveto{\pgfqpoint{1.768615in}{3.251429in}}{\pgfqpoint{1.760715in}{3.248157in}}{\pgfqpoint{1.754891in}{3.242333in}}%
\pgfpathcurveto{\pgfqpoint{1.749067in}{3.236509in}}{\pgfqpoint{1.745794in}{3.228609in}}{\pgfqpoint{1.745794in}{3.220373in}}%
\pgfpathcurveto{\pgfqpoint{1.745794in}{3.212136in}}{\pgfqpoint{1.749067in}{3.204236in}}{\pgfqpoint{1.754891in}{3.198412in}}%
\pgfpathcurveto{\pgfqpoint{1.760715in}{3.192588in}}{\pgfqpoint{1.768615in}{3.189316in}}{\pgfqpoint{1.776851in}{3.189316in}}%
\pgfpathclose%
\pgfusepath{stroke,fill}%
\end{pgfscope}%
\begin{pgfscope}%
\pgfpathrectangle{\pgfqpoint{0.100000in}{0.212622in}}{\pgfqpoint{3.696000in}{3.696000in}}%
\pgfusepath{clip}%
\pgfsetbuttcap%
\pgfsetroundjoin%
\definecolor{currentfill}{rgb}{0.121569,0.466667,0.705882}%
\pgfsetfillcolor{currentfill}%
\pgfsetfillopacity{0.331497}%
\pgfsetlinewidth{1.003750pt}%
\definecolor{currentstroke}{rgb}{0.121569,0.466667,0.705882}%
\pgfsetstrokecolor{currentstroke}%
\pgfsetstrokeopacity{0.331497}%
\pgfsetdash{}{0pt}%
\pgfpathmoveto{\pgfqpoint{1.932931in}{3.181460in}}%
\pgfpathcurveto{\pgfqpoint{1.941167in}{3.181460in}}{\pgfqpoint{1.949067in}{3.184732in}}{\pgfqpoint{1.954891in}{3.190556in}}%
\pgfpathcurveto{\pgfqpoint{1.960715in}{3.196380in}}{\pgfqpoint{1.963987in}{3.204280in}}{\pgfqpoint{1.963987in}{3.212517in}}%
\pgfpathcurveto{\pgfqpoint{1.963987in}{3.220753in}}{\pgfqpoint{1.960715in}{3.228653in}}{\pgfqpoint{1.954891in}{3.234477in}}%
\pgfpathcurveto{\pgfqpoint{1.949067in}{3.240301in}}{\pgfqpoint{1.941167in}{3.243573in}}{\pgfqpoint{1.932931in}{3.243573in}}%
\pgfpathcurveto{\pgfqpoint{1.924695in}{3.243573in}}{\pgfqpoint{1.916795in}{3.240301in}}{\pgfqpoint{1.910971in}{3.234477in}}%
\pgfpathcurveto{\pgfqpoint{1.905147in}{3.228653in}}{\pgfqpoint{1.901874in}{3.220753in}}{\pgfqpoint{1.901874in}{3.212517in}}%
\pgfpathcurveto{\pgfqpoint{1.901874in}{3.204280in}}{\pgfqpoint{1.905147in}{3.196380in}}{\pgfqpoint{1.910971in}{3.190556in}}%
\pgfpathcurveto{\pgfqpoint{1.916795in}{3.184732in}}{\pgfqpoint{1.924695in}{3.181460in}}{\pgfqpoint{1.932931in}{3.181460in}}%
\pgfpathclose%
\pgfusepath{stroke,fill}%
\end{pgfscope}%
\begin{pgfscope}%
\pgfpathrectangle{\pgfqpoint{0.100000in}{0.212622in}}{\pgfqpoint{3.696000in}{3.696000in}}%
\pgfusepath{clip}%
\pgfsetbuttcap%
\pgfsetroundjoin%
\definecolor{currentfill}{rgb}{0.121569,0.466667,0.705882}%
\pgfsetfillcolor{currentfill}%
\pgfsetfillopacity{0.331984}%
\pgfsetlinewidth{1.003750pt}%
\definecolor{currentstroke}{rgb}{0.121569,0.466667,0.705882}%
\pgfsetstrokecolor{currentstroke}%
\pgfsetstrokeopacity{0.331984}%
\pgfsetdash{}{0pt}%
\pgfpathmoveto{\pgfqpoint{1.775364in}{3.186138in}}%
\pgfpathcurveto{\pgfqpoint{1.783600in}{3.186138in}}{\pgfqpoint{1.791500in}{3.189410in}}{\pgfqpoint{1.797324in}{3.195234in}}%
\pgfpathcurveto{\pgfqpoint{1.803148in}{3.201058in}}{\pgfqpoint{1.806421in}{3.208958in}}{\pgfqpoint{1.806421in}{3.217194in}}%
\pgfpathcurveto{\pgfqpoint{1.806421in}{3.225430in}}{\pgfqpoint{1.803148in}{3.233331in}}{\pgfqpoint{1.797324in}{3.239154in}}%
\pgfpathcurveto{\pgfqpoint{1.791500in}{3.244978in}}{\pgfqpoint{1.783600in}{3.248251in}}{\pgfqpoint{1.775364in}{3.248251in}}%
\pgfpathcurveto{\pgfqpoint{1.767128in}{3.248251in}}{\pgfqpoint{1.759228in}{3.244978in}}{\pgfqpoint{1.753404in}{3.239154in}}%
\pgfpathcurveto{\pgfqpoint{1.747580in}{3.233331in}}{\pgfqpoint{1.744308in}{3.225430in}}{\pgfqpoint{1.744308in}{3.217194in}}%
\pgfpathcurveto{\pgfqpoint{1.744308in}{3.208958in}}{\pgfqpoint{1.747580in}{3.201058in}}{\pgfqpoint{1.753404in}{3.195234in}}%
\pgfpathcurveto{\pgfqpoint{1.759228in}{3.189410in}}{\pgfqpoint{1.767128in}{3.186138in}}{\pgfqpoint{1.775364in}{3.186138in}}%
\pgfpathclose%
\pgfusepath{stroke,fill}%
\end{pgfscope}%
\begin{pgfscope}%
\pgfpathrectangle{\pgfqpoint{0.100000in}{0.212622in}}{\pgfqpoint{3.696000in}{3.696000in}}%
\pgfusepath{clip}%
\pgfsetbuttcap%
\pgfsetroundjoin%
\definecolor{currentfill}{rgb}{0.121569,0.466667,0.705882}%
\pgfsetfillcolor{currentfill}%
\pgfsetfillopacity{0.332452}%
\pgfsetlinewidth{1.003750pt}%
\definecolor{currentstroke}{rgb}{0.121569,0.466667,0.705882}%
\pgfsetstrokecolor{currentstroke}%
\pgfsetstrokeopacity{0.332452}%
\pgfsetdash{}{0pt}%
\pgfpathmoveto{\pgfqpoint{1.773959in}{3.183677in}}%
\pgfpathcurveto{\pgfqpoint{1.782196in}{3.183677in}}{\pgfqpoint{1.790096in}{3.186950in}}{\pgfqpoint{1.795920in}{3.192773in}}%
\pgfpathcurveto{\pgfqpoint{1.801744in}{3.198597in}}{\pgfqpoint{1.805016in}{3.206497in}}{\pgfqpoint{1.805016in}{3.214734in}}%
\pgfpathcurveto{\pgfqpoint{1.805016in}{3.222970in}}{\pgfqpoint{1.801744in}{3.230870in}}{\pgfqpoint{1.795920in}{3.236694in}}%
\pgfpathcurveto{\pgfqpoint{1.790096in}{3.242518in}}{\pgfqpoint{1.782196in}{3.245790in}}{\pgfqpoint{1.773959in}{3.245790in}}%
\pgfpathcurveto{\pgfqpoint{1.765723in}{3.245790in}}{\pgfqpoint{1.757823in}{3.242518in}}{\pgfqpoint{1.751999in}{3.236694in}}%
\pgfpathcurveto{\pgfqpoint{1.746175in}{3.230870in}}{\pgfqpoint{1.742903in}{3.222970in}}{\pgfqpoint{1.742903in}{3.214734in}}%
\pgfpathcurveto{\pgfqpoint{1.742903in}{3.206497in}}{\pgfqpoint{1.746175in}{3.198597in}}{\pgfqpoint{1.751999in}{3.192773in}}%
\pgfpathcurveto{\pgfqpoint{1.757823in}{3.186950in}}{\pgfqpoint{1.765723in}{3.183677in}}{\pgfqpoint{1.773959in}{3.183677in}}%
\pgfpathclose%
\pgfusepath{stroke,fill}%
\end{pgfscope}%
\begin{pgfscope}%
\pgfpathrectangle{\pgfqpoint{0.100000in}{0.212622in}}{\pgfqpoint{3.696000in}{3.696000in}}%
\pgfusepath{clip}%
\pgfsetbuttcap%
\pgfsetroundjoin%
\definecolor{currentfill}{rgb}{0.121569,0.466667,0.705882}%
\pgfsetfillcolor{currentfill}%
\pgfsetfillopacity{0.332517}%
\pgfsetlinewidth{1.003750pt}%
\definecolor{currentstroke}{rgb}{0.121569,0.466667,0.705882}%
\pgfsetstrokecolor{currentstroke}%
\pgfsetstrokeopacity{0.332517}%
\pgfsetdash{}{0pt}%
\pgfpathmoveto{\pgfqpoint{1.933796in}{3.177354in}}%
\pgfpathcurveto{\pgfqpoint{1.942033in}{3.177354in}}{\pgfqpoint{1.949933in}{3.180627in}}{\pgfqpoint{1.955757in}{3.186450in}}%
\pgfpathcurveto{\pgfqpoint{1.961580in}{3.192274in}}{\pgfqpoint{1.964853in}{3.200174in}}{\pgfqpoint{1.964853in}{3.208411in}}%
\pgfpathcurveto{\pgfqpoint{1.964853in}{3.216647in}}{\pgfqpoint{1.961580in}{3.224547in}}{\pgfqpoint{1.955757in}{3.230371in}}%
\pgfpathcurveto{\pgfqpoint{1.949933in}{3.236195in}}{\pgfqpoint{1.942033in}{3.239467in}}{\pgfqpoint{1.933796in}{3.239467in}}%
\pgfpathcurveto{\pgfqpoint{1.925560in}{3.239467in}}{\pgfqpoint{1.917660in}{3.236195in}}{\pgfqpoint{1.911836in}{3.230371in}}%
\pgfpathcurveto{\pgfqpoint{1.906012in}{3.224547in}}{\pgfqpoint{1.902740in}{3.216647in}}{\pgfqpoint{1.902740in}{3.208411in}}%
\pgfpathcurveto{\pgfqpoint{1.902740in}{3.200174in}}{\pgfqpoint{1.906012in}{3.192274in}}{\pgfqpoint{1.911836in}{3.186450in}}%
\pgfpathcurveto{\pgfqpoint{1.917660in}{3.180627in}}{\pgfqpoint{1.925560in}{3.177354in}}{\pgfqpoint{1.933796in}{3.177354in}}%
\pgfpathclose%
\pgfusepath{stroke,fill}%
\end{pgfscope}%
\begin{pgfscope}%
\pgfpathrectangle{\pgfqpoint{0.100000in}{0.212622in}}{\pgfqpoint{3.696000in}{3.696000in}}%
\pgfusepath{clip}%
\pgfsetbuttcap%
\pgfsetroundjoin%
\definecolor{currentfill}{rgb}{0.121569,0.466667,0.705882}%
\pgfsetfillcolor{currentfill}%
\pgfsetfillopacity{0.333245}%
\pgfsetlinewidth{1.003750pt}%
\definecolor{currentstroke}{rgb}{0.121569,0.466667,0.705882}%
\pgfsetstrokecolor{currentstroke}%
\pgfsetstrokeopacity{0.333245}%
\pgfsetdash{}{0pt}%
\pgfpathmoveto{\pgfqpoint{1.771276in}{3.179158in}}%
\pgfpathcurveto{\pgfqpoint{1.779513in}{3.179158in}}{\pgfqpoint{1.787413in}{3.182430in}}{\pgfqpoint{1.793237in}{3.188254in}}%
\pgfpathcurveto{\pgfqpoint{1.799061in}{3.194078in}}{\pgfqpoint{1.802333in}{3.201978in}}{\pgfqpoint{1.802333in}{3.210215in}}%
\pgfpathcurveto{\pgfqpoint{1.802333in}{3.218451in}}{\pgfqpoint{1.799061in}{3.226351in}}{\pgfqpoint{1.793237in}{3.232175in}}%
\pgfpathcurveto{\pgfqpoint{1.787413in}{3.237999in}}{\pgfqpoint{1.779513in}{3.241271in}}{\pgfqpoint{1.771276in}{3.241271in}}%
\pgfpathcurveto{\pgfqpoint{1.763040in}{3.241271in}}{\pgfqpoint{1.755140in}{3.237999in}}{\pgfqpoint{1.749316in}{3.232175in}}%
\pgfpathcurveto{\pgfqpoint{1.743492in}{3.226351in}}{\pgfqpoint{1.740220in}{3.218451in}}{\pgfqpoint{1.740220in}{3.210215in}}%
\pgfpathcurveto{\pgfqpoint{1.740220in}{3.201978in}}{\pgfqpoint{1.743492in}{3.194078in}}{\pgfqpoint{1.749316in}{3.188254in}}%
\pgfpathcurveto{\pgfqpoint{1.755140in}{3.182430in}}{\pgfqpoint{1.763040in}{3.179158in}}{\pgfqpoint{1.771276in}{3.179158in}}%
\pgfpathclose%
\pgfusepath{stroke,fill}%
\end{pgfscope}%
\begin{pgfscope}%
\pgfpathrectangle{\pgfqpoint{0.100000in}{0.212622in}}{\pgfqpoint{3.696000in}{3.696000in}}%
\pgfusepath{clip}%
\pgfsetbuttcap%
\pgfsetroundjoin%
\definecolor{currentfill}{rgb}{0.121569,0.466667,0.705882}%
\pgfsetfillcolor{currentfill}%
\pgfsetfillopacity{0.333602}%
\pgfsetlinewidth{1.003750pt}%
\definecolor{currentstroke}{rgb}{0.121569,0.466667,0.705882}%
\pgfsetstrokecolor{currentstroke}%
\pgfsetstrokeopacity{0.333602}%
\pgfsetdash{}{0pt}%
\pgfpathmoveto{\pgfqpoint{1.934699in}{3.172908in}}%
\pgfpathcurveto{\pgfqpoint{1.942936in}{3.172908in}}{\pgfqpoint{1.950836in}{3.176180in}}{\pgfqpoint{1.956660in}{3.182004in}}%
\pgfpathcurveto{\pgfqpoint{1.962484in}{3.187828in}}{\pgfqpoint{1.965756in}{3.195728in}}{\pgfqpoint{1.965756in}{3.203965in}}%
\pgfpathcurveto{\pgfqpoint{1.965756in}{3.212201in}}{\pgfqpoint{1.962484in}{3.220101in}}{\pgfqpoint{1.956660in}{3.225925in}}%
\pgfpathcurveto{\pgfqpoint{1.950836in}{3.231749in}}{\pgfqpoint{1.942936in}{3.235021in}}{\pgfqpoint{1.934699in}{3.235021in}}%
\pgfpathcurveto{\pgfqpoint{1.926463in}{3.235021in}}{\pgfqpoint{1.918563in}{3.231749in}}{\pgfqpoint{1.912739in}{3.225925in}}%
\pgfpathcurveto{\pgfqpoint{1.906915in}{3.220101in}}{\pgfqpoint{1.903643in}{3.212201in}}{\pgfqpoint{1.903643in}{3.203965in}}%
\pgfpathcurveto{\pgfqpoint{1.903643in}{3.195728in}}{\pgfqpoint{1.906915in}{3.187828in}}{\pgfqpoint{1.912739in}{3.182004in}}%
\pgfpathcurveto{\pgfqpoint{1.918563in}{3.176180in}}{\pgfqpoint{1.926463in}{3.172908in}}{\pgfqpoint{1.934699in}{3.172908in}}%
\pgfpathclose%
\pgfusepath{stroke,fill}%
\end{pgfscope}%
\begin{pgfscope}%
\pgfpathrectangle{\pgfqpoint{0.100000in}{0.212622in}}{\pgfqpoint{3.696000in}{3.696000in}}%
\pgfusepath{clip}%
\pgfsetbuttcap%
\pgfsetroundjoin%
\definecolor{currentfill}{rgb}{0.121569,0.466667,0.705882}%
\pgfsetfillcolor{currentfill}%
\pgfsetfillopacity{0.333855}%
\pgfsetlinewidth{1.003750pt}%
\definecolor{currentstroke}{rgb}{0.121569,0.466667,0.705882}%
\pgfsetstrokecolor{currentstroke}%
\pgfsetstrokeopacity{0.333855}%
\pgfsetdash{}{0pt}%
\pgfpathmoveto{\pgfqpoint{1.769681in}{3.175704in}}%
\pgfpathcurveto{\pgfqpoint{1.777917in}{3.175704in}}{\pgfqpoint{1.785817in}{3.178976in}}{\pgfqpoint{1.791641in}{3.184800in}}%
\pgfpathcurveto{\pgfqpoint{1.797465in}{3.190624in}}{\pgfqpoint{1.800738in}{3.198524in}}{\pgfqpoint{1.800738in}{3.206761in}}%
\pgfpathcurveto{\pgfqpoint{1.800738in}{3.214997in}}{\pgfqpoint{1.797465in}{3.222897in}}{\pgfqpoint{1.791641in}{3.228721in}}%
\pgfpathcurveto{\pgfqpoint{1.785817in}{3.234545in}}{\pgfqpoint{1.777917in}{3.237817in}}{\pgfqpoint{1.769681in}{3.237817in}}%
\pgfpathcurveto{\pgfqpoint{1.761445in}{3.237817in}}{\pgfqpoint{1.753545in}{3.234545in}}{\pgfqpoint{1.747721in}{3.228721in}}%
\pgfpathcurveto{\pgfqpoint{1.741897in}{3.222897in}}{\pgfqpoint{1.738625in}{3.214997in}}{\pgfqpoint{1.738625in}{3.206761in}}%
\pgfpathcurveto{\pgfqpoint{1.738625in}{3.198524in}}{\pgfqpoint{1.741897in}{3.190624in}}{\pgfqpoint{1.747721in}{3.184800in}}%
\pgfpathcurveto{\pgfqpoint{1.753545in}{3.178976in}}{\pgfqpoint{1.761445in}{3.175704in}}{\pgfqpoint{1.769681in}{3.175704in}}%
\pgfpathclose%
\pgfusepath{stroke,fill}%
\end{pgfscope}%
\begin{pgfscope}%
\pgfpathrectangle{\pgfqpoint{0.100000in}{0.212622in}}{\pgfqpoint{3.696000in}{3.696000in}}%
\pgfusepath{clip}%
\pgfsetbuttcap%
\pgfsetroundjoin%
\definecolor{currentfill}{rgb}{0.121569,0.466667,0.705882}%
\pgfsetfillcolor{currentfill}%
\pgfsetfillopacity{0.334112}%
\pgfsetlinewidth{1.003750pt}%
\definecolor{currentstroke}{rgb}{0.121569,0.466667,0.705882}%
\pgfsetstrokecolor{currentstroke}%
\pgfsetstrokeopacity{0.334112}%
\pgfsetdash{}{0pt}%
\pgfpathmoveto{\pgfqpoint{1.768831in}{3.174311in}}%
\pgfpathcurveto{\pgfqpoint{1.777067in}{3.174311in}}{\pgfqpoint{1.784968in}{3.177583in}}{\pgfqpoint{1.790791in}{3.183407in}}%
\pgfpathcurveto{\pgfqpoint{1.796615in}{3.189231in}}{\pgfqpoint{1.799888in}{3.197131in}}{\pgfqpoint{1.799888in}{3.205368in}}%
\pgfpathcurveto{\pgfqpoint{1.799888in}{3.213604in}}{\pgfqpoint{1.796615in}{3.221504in}}{\pgfqpoint{1.790791in}{3.227328in}}%
\pgfpathcurveto{\pgfqpoint{1.784968in}{3.233152in}}{\pgfqpoint{1.777067in}{3.236424in}}{\pgfqpoint{1.768831in}{3.236424in}}%
\pgfpathcurveto{\pgfqpoint{1.760595in}{3.236424in}}{\pgfqpoint{1.752695in}{3.233152in}}{\pgfqpoint{1.746871in}{3.227328in}}%
\pgfpathcurveto{\pgfqpoint{1.741047in}{3.221504in}}{\pgfqpoint{1.737775in}{3.213604in}}{\pgfqpoint{1.737775in}{3.205368in}}%
\pgfpathcurveto{\pgfqpoint{1.737775in}{3.197131in}}{\pgfqpoint{1.741047in}{3.189231in}}{\pgfqpoint{1.746871in}{3.183407in}}%
\pgfpathcurveto{\pgfqpoint{1.752695in}{3.177583in}}{\pgfqpoint{1.760595in}{3.174311in}}{\pgfqpoint{1.768831in}{3.174311in}}%
\pgfpathclose%
\pgfusepath{stroke,fill}%
\end{pgfscope}%
\begin{pgfscope}%
\pgfpathrectangle{\pgfqpoint{0.100000in}{0.212622in}}{\pgfqpoint{3.696000in}{3.696000in}}%
\pgfusepath{clip}%
\pgfsetbuttcap%
\pgfsetroundjoin%
\definecolor{currentfill}{rgb}{0.121569,0.466667,0.705882}%
\pgfsetfillcolor{currentfill}%
\pgfsetfillopacity{0.334605}%
\pgfsetlinewidth{1.003750pt}%
\definecolor{currentstroke}{rgb}{0.121569,0.466667,0.705882}%
\pgfsetstrokecolor{currentstroke}%
\pgfsetstrokeopacity{0.334605}%
\pgfsetdash{}{0pt}%
\pgfpathmoveto{\pgfqpoint{1.767400in}{3.171715in}}%
\pgfpathcurveto{\pgfqpoint{1.775636in}{3.171715in}}{\pgfqpoint{1.783536in}{3.174988in}}{\pgfqpoint{1.789360in}{3.180812in}}%
\pgfpathcurveto{\pgfqpoint{1.795184in}{3.186636in}}{\pgfqpoint{1.798456in}{3.194536in}}{\pgfqpoint{1.798456in}{3.202772in}}%
\pgfpathcurveto{\pgfqpoint{1.798456in}{3.211008in}}{\pgfqpoint{1.795184in}{3.218908in}}{\pgfqpoint{1.789360in}{3.224732in}}%
\pgfpathcurveto{\pgfqpoint{1.783536in}{3.230556in}}{\pgfqpoint{1.775636in}{3.233828in}}{\pgfqpoint{1.767400in}{3.233828in}}%
\pgfpathcurveto{\pgfqpoint{1.759164in}{3.233828in}}{\pgfqpoint{1.751263in}{3.230556in}}{\pgfqpoint{1.745440in}{3.224732in}}%
\pgfpathcurveto{\pgfqpoint{1.739616in}{3.218908in}}{\pgfqpoint{1.736343in}{3.211008in}}{\pgfqpoint{1.736343in}{3.202772in}}%
\pgfpathcurveto{\pgfqpoint{1.736343in}{3.194536in}}{\pgfqpoint{1.739616in}{3.186636in}}{\pgfqpoint{1.745440in}{3.180812in}}%
\pgfpathcurveto{\pgfqpoint{1.751263in}{3.174988in}}{\pgfqpoint{1.759164in}{3.171715in}}{\pgfqpoint{1.767400in}{3.171715in}}%
\pgfpathclose%
\pgfusepath{stroke,fill}%
\end{pgfscope}%
\begin{pgfscope}%
\pgfpathrectangle{\pgfqpoint{0.100000in}{0.212622in}}{\pgfqpoint{3.696000in}{3.696000in}}%
\pgfusepath{clip}%
\pgfsetbuttcap%
\pgfsetroundjoin%
\definecolor{currentfill}{rgb}{0.121569,0.466667,0.705882}%
\pgfsetfillcolor{currentfill}%
\pgfsetfillopacity{0.334940}%
\pgfsetlinewidth{1.003750pt}%
\definecolor{currentstroke}{rgb}{0.121569,0.466667,0.705882}%
\pgfsetstrokecolor{currentstroke}%
\pgfsetstrokeopacity{0.334940}%
\pgfsetdash{}{0pt}%
\pgfpathmoveto{\pgfqpoint{1.935294in}{3.168422in}}%
\pgfpathcurveto{\pgfqpoint{1.943531in}{3.168422in}}{\pgfqpoint{1.951431in}{3.171694in}}{\pgfqpoint{1.957255in}{3.177518in}}%
\pgfpathcurveto{\pgfqpoint{1.963079in}{3.183342in}}{\pgfqpoint{1.966351in}{3.191242in}}{\pgfqpoint{1.966351in}{3.199479in}}%
\pgfpathcurveto{\pgfqpoint{1.966351in}{3.207715in}}{\pgfqpoint{1.963079in}{3.215615in}}{\pgfqpoint{1.957255in}{3.221439in}}%
\pgfpathcurveto{\pgfqpoint{1.951431in}{3.227263in}}{\pgfqpoint{1.943531in}{3.230535in}}{\pgfqpoint{1.935294in}{3.230535in}}%
\pgfpathcurveto{\pgfqpoint{1.927058in}{3.230535in}}{\pgfqpoint{1.919158in}{3.227263in}}{\pgfqpoint{1.913334in}{3.221439in}}%
\pgfpathcurveto{\pgfqpoint{1.907510in}{3.215615in}}{\pgfqpoint{1.904238in}{3.207715in}}{\pgfqpoint{1.904238in}{3.199479in}}%
\pgfpathcurveto{\pgfqpoint{1.904238in}{3.191242in}}{\pgfqpoint{1.907510in}{3.183342in}}{\pgfqpoint{1.913334in}{3.177518in}}%
\pgfpathcurveto{\pgfqpoint{1.919158in}{3.171694in}}{\pgfqpoint{1.927058in}{3.168422in}}{\pgfqpoint{1.935294in}{3.168422in}}%
\pgfpathclose%
\pgfusepath{stroke,fill}%
\end{pgfscope}%
\begin{pgfscope}%
\pgfpathrectangle{\pgfqpoint{0.100000in}{0.212622in}}{\pgfqpoint{3.696000in}{3.696000in}}%
\pgfusepath{clip}%
\pgfsetbuttcap%
\pgfsetroundjoin%
\definecolor{currentfill}{rgb}{0.121569,0.466667,0.705882}%
\pgfsetfillcolor{currentfill}%
\pgfsetfillopacity{0.335514}%
\pgfsetlinewidth{1.003750pt}%
\definecolor{currentstroke}{rgb}{0.121569,0.466667,0.705882}%
\pgfsetstrokecolor{currentstroke}%
\pgfsetstrokeopacity{0.335514}%
\pgfsetdash{}{0pt}%
\pgfpathmoveto{\pgfqpoint{1.764870in}{3.166949in}}%
\pgfpathcurveto{\pgfqpoint{1.773106in}{3.166949in}}{\pgfqpoint{1.781006in}{3.170221in}}{\pgfqpoint{1.786830in}{3.176045in}}%
\pgfpathcurveto{\pgfqpoint{1.792654in}{3.181869in}}{\pgfqpoint{1.795926in}{3.189769in}}{\pgfqpoint{1.795926in}{3.198006in}}%
\pgfpathcurveto{\pgfqpoint{1.795926in}{3.206242in}}{\pgfqpoint{1.792654in}{3.214142in}}{\pgfqpoint{1.786830in}{3.219966in}}%
\pgfpathcurveto{\pgfqpoint{1.781006in}{3.225790in}}{\pgfqpoint{1.773106in}{3.229062in}}{\pgfqpoint{1.764870in}{3.229062in}}%
\pgfpathcurveto{\pgfqpoint{1.756633in}{3.229062in}}{\pgfqpoint{1.748733in}{3.225790in}}{\pgfqpoint{1.742909in}{3.219966in}}%
\pgfpathcurveto{\pgfqpoint{1.737085in}{3.214142in}}{\pgfqpoint{1.733813in}{3.206242in}}{\pgfqpoint{1.733813in}{3.198006in}}%
\pgfpathcurveto{\pgfqpoint{1.733813in}{3.189769in}}{\pgfqpoint{1.737085in}{3.181869in}}{\pgfqpoint{1.742909in}{3.176045in}}%
\pgfpathcurveto{\pgfqpoint{1.748733in}{3.170221in}}{\pgfqpoint{1.756633in}{3.166949in}}{\pgfqpoint{1.764870in}{3.166949in}}%
\pgfpathclose%
\pgfusepath{stroke,fill}%
\end{pgfscope}%
\begin{pgfscope}%
\pgfpathrectangle{\pgfqpoint{0.100000in}{0.212622in}}{\pgfqpoint{3.696000in}{3.696000in}}%
\pgfusepath{clip}%
\pgfsetbuttcap%
\pgfsetroundjoin%
\definecolor{currentfill}{rgb}{0.121569,0.466667,0.705882}%
\pgfsetfillcolor{currentfill}%
\pgfsetfillopacity{0.336197}%
\pgfsetlinewidth{1.003750pt}%
\definecolor{currentstroke}{rgb}{0.121569,0.466667,0.705882}%
\pgfsetstrokecolor{currentstroke}%
\pgfsetstrokeopacity{0.336197}%
\pgfsetdash{}{0pt}%
\pgfpathmoveto{\pgfqpoint{1.762508in}{3.163085in}}%
\pgfpathcurveto{\pgfqpoint{1.770744in}{3.163085in}}{\pgfqpoint{1.778644in}{3.166358in}}{\pgfqpoint{1.784468in}{3.172181in}}%
\pgfpathcurveto{\pgfqpoint{1.790292in}{3.178005in}}{\pgfqpoint{1.793564in}{3.185905in}}{\pgfqpoint{1.793564in}{3.194142in}}%
\pgfpathcurveto{\pgfqpoint{1.793564in}{3.202378in}}{\pgfqpoint{1.790292in}{3.210278in}}{\pgfqpoint{1.784468in}{3.216102in}}%
\pgfpathcurveto{\pgfqpoint{1.778644in}{3.221926in}}{\pgfqpoint{1.770744in}{3.225198in}}{\pgfqpoint{1.762508in}{3.225198in}}%
\pgfpathcurveto{\pgfqpoint{1.754271in}{3.225198in}}{\pgfqpoint{1.746371in}{3.221926in}}{\pgfqpoint{1.740547in}{3.216102in}}%
\pgfpathcurveto{\pgfqpoint{1.734723in}{3.210278in}}{\pgfqpoint{1.731451in}{3.202378in}}{\pgfqpoint{1.731451in}{3.194142in}}%
\pgfpathcurveto{\pgfqpoint{1.731451in}{3.185905in}}{\pgfqpoint{1.734723in}{3.178005in}}{\pgfqpoint{1.740547in}{3.172181in}}%
\pgfpathcurveto{\pgfqpoint{1.746371in}{3.166358in}}{\pgfqpoint{1.754271in}{3.163085in}}{\pgfqpoint{1.762508in}{3.163085in}}%
\pgfpathclose%
\pgfusepath{stroke,fill}%
\end{pgfscope}%
\begin{pgfscope}%
\pgfpathrectangle{\pgfqpoint{0.100000in}{0.212622in}}{\pgfqpoint{3.696000in}{3.696000in}}%
\pgfusepath{clip}%
\pgfsetbuttcap%
\pgfsetroundjoin%
\definecolor{currentfill}{rgb}{0.121569,0.466667,0.705882}%
\pgfsetfillcolor{currentfill}%
\pgfsetfillopacity{0.336481}%
\pgfsetlinewidth{1.003750pt}%
\definecolor{currentstroke}{rgb}{0.121569,0.466667,0.705882}%
\pgfsetstrokecolor{currentstroke}%
\pgfsetstrokeopacity{0.336481}%
\pgfsetdash{}{0pt}%
\pgfpathmoveto{\pgfqpoint{1.936714in}{3.162383in}}%
\pgfpathcurveto{\pgfqpoint{1.944950in}{3.162383in}}{\pgfqpoint{1.952850in}{3.165656in}}{\pgfqpoint{1.958674in}{3.171480in}}%
\pgfpathcurveto{\pgfqpoint{1.964498in}{3.177304in}}{\pgfqpoint{1.967770in}{3.185204in}}{\pgfqpoint{1.967770in}{3.193440in}}%
\pgfpathcurveto{\pgfqpoint{1.967770in}{3.201676in}}{\pgfqpoint{1.964498in}{3.209576in}}{\pgfqpoint{1.958674in}{3.215400in}}%
\pgfpathcurveto{\pgfqpoint{1.952850in}{3.221224in}}{\pgfqpoint{1.944950in}{3.224496in}}{\pgfqpoint{1.936714in}{3.224496in}}%
\pgfpathcurveto{\pgfqpoint{1.928477in}{3.224496in}}{\pgfqpoint{1.920577in}{3.221224in}}{\pgfqpoint{1.914753in}{3.215400in}}%
\pgfpathcurveto{\pgfqpoint{1.908929in}{3.209576in}}{\pgfqpoint{1.905657in}{3.201676in}}{\pgfqpoint{1.905657in}{3.193440in}}%
\pgfpathcurveto{\pgfqpoint{1.905657in}{3.185204in}}{\pgfqpoint{1.908929in}{3.177304in}}{\pgfqpoint{1.914753in}{3.171480in}}%
\pgfpathcurveto{\pgfqpoint{1.920577in}{3.165656in}}{\pgfqpoint{1.928477in}{3.162383in}}{\pgfqpoint{1.936714in}{3.162383in}}%
\pgfpathclose%
\pgfusepath{stroke,fill}%
\end{pgfscope}%
\begin{pgfscope}%
\pgfpathrectangle{\pgfqpoint{0.100000in}{0.212622in}}{\pgfqpoint{3.696000in}{3.696000in}}%
\pgfusepath{clip}%
\pgfsetbuttcap%
\pgfsetroundjoin%
\definecolor{currentfill}{rgb}{0.121569,0.466667,0.705882}%
\pgfsetfillcolor{currentfill}%
\pgfsetfillopacity{0.336679}%
\pgfsetlinewidth{1.003750pt}%
\definecolor{currentstroke}{rgb}{0.121569,0.466667,0.705882}%
\pgfsetstrokecolor{currentstroke}%
\pgfsetstrokeopacity{0.336679}%
\pgfsetdash{}{0pt}%
\pgfpathmoveto{\pgfqpoint{1.761184in}{3.160405in}}%
\pgfpathcurveto{\pgfqpoint{1.769420in}{3.160405in}}{\pgfqpoint{1.777320in}{3.163677in}}{\pgfqpoint{1.783144in}{3.169501in}}%
\pgfpathcurveto{\pgfqpoint{1.788968in}{3.175325in}}{\pgfqpoint{1.792240in}{3.183225in}}{\pgfqpoint{1.792240in}{3.191461in}}%
\pgfpathcurveto{\pgfqpoint{1.792240in}{3.199697in}}{\pgfqpoint{1.788968in}{3.207597in}}{\pgfqpoint{1.783144in}{3.213421in}}%
\pgfpathcurveto{\pgfqpoint{1.777320in}{3.219245in}}{\pgfqpoint{1.769420in}{3.222518in}}{\pgfqpoint{1.761184in}{3.222518in}}%
\pgfpathcurveto{\pgfqpoint{1.752947in}{3.222518in}}{\pgfqpoint{1.745047in}{3.219245in}}{\pgfqpoint{1.739223in}{3.213421in}}%
\pgfpathcurveto{\pgfqpoint{1.733399in}{3.207597in}}{\pgfqpoint{1.730127in}{3.199697in}}{\pgfqpoint{1.730127in}{3.191461in}}%
\pgfpathcurveto{\pgfqpoint{1.730127in}{3.183225in}}{\pgfqpoint{1.733399in}{3.175325in}}{\pgfqpoint{1.739223in}{3.169501in}}%
\pgfpathcurveto{\pgfqpoint{1.745047in}{3.163677in}}{\pgfqpoint{1.752947in}{3.160405in}}{\pgfqpoint{1.761184in}{3.160405in}}%
\pgfpathclose%
\pgfusepath{stroke,fill}%
\end{pgfscope}%
\begin{pgfscope}%
\pgfpathrectangle{\pgfqpoint{0.100000in}{0.212622in}}{\pgfqpoint{3.696000in}{3.696000in}}%
\pgfusepath{clip}%
\pgfsetbuttcap%
\pgfsetroundjoin%
\definecolor{currentfill}{rgb}{0.121569,0.466667,0.705882}%
\pgfsetfillcolor{currentfill}%
\pgfsetfillopacity{0.337570}%
\pgfsetlinewidth{1.003750pt}%
\definecolor{currentstroke}{rgb}{0.121569,0.466667,0.705882}%
\pgfsetstrokecolor{currentstroke}%
\pgfsetstrokeopacity{0.337570}%
\pgfsetdash{}{0pt}%
\pgfpathmoveto{\pgfqpoint{1.758476in}{3.155978in}}%
\pgfpathcurveto{\pgfqpoint{1.766713in}{3.155978in}}{\pgfqpoint{1.774613in}{3.159250in}}{\pgfqpoint{1.780437in}{3.165074in}}%
\pgfpathcurveto{\pgfqpoint{1.786260in}{3.170898in}}{\pgfqpoint{1.789533in}{3.178798in}}{\pgfqpoint{1.789533in}{3.187035in}}%
\pgfpathcurveto{\pgfqpoint{1.789533in}{3.195271in}}{\pgfqpoint{1.786260in}{3.203171in}}{\pgfqpoint{1.780437in}{3.208995in}}%
\pgfpathcurveto{\pgfqpoint{1.774613in}{3.214819in}}{\pgfqpoint{1.766713in}{3.218091in}}{\pgfqpoint{1.758476in}{3.218091in}}%
\pgfpathcurveto{\pgfqpoint{1.750240in}{3.218091in}}{\pgfqpoint{1.742340in}{3.214819in}}{\pgfqpoint{1.736516in}{3.208995in}}%
\pgfpathcurveto{\pgfqpoint{1.730692in}{3.203171in}}{\pgfqpoint{1.727420in}{3.195271in}}{\pgfqpoint{1.727420in}{3.187035in}}%
\pgfpathcurveto{\pgfqpoint{1.727420in}{3.178798in}}{\pgfqpoint{1.730692in}{3.170898in}}{\pgfqpoint{1.736516in}{3.165074in}}%
\pgfpathcurveto{\pgfqpoint{1.742340in}{3.159250in}}{\pgfqpoint{1.750240in}{3.155978in}}{\pgfqpoint{1.758476in}{3.155978in}}%
\pgfpathclose%
\pgfusepath{stroke,fill}%
\end{pgfscope}%
\begin{pgfscope}%
\pgfpathrectangle{\pgfqpoint{0.100000in}{0.212622in}}{\pgfqpoint{3.696000in}{3.696000in}}%
\pgfusepath{clip}%
\pgfsetbuttcap%
\pgfsetroundjoin%
\definecolor{currentfill}{rgb}{0.121569,0.466667,0.705882}%
\pgfsetfillcolor{currentfill}%
\pgfsetfillopacity{0.338174}%
\pgfsetlinewidth{1.003750pt}%
\definecolor{currentstroke}{rgb}{0.121569,0.466667,0.705882}%
\pgfsetstrokecolor{currentstroke}%
\pgfsetstrokeopacity{0.338174}%
\pgfsetdash{}{0pt}%
\pgfpathmoveto{\pgfqpoint{1.937808in}{3.155929in}}%
\pgfpathcurveto{\pgfqpoint{1.946044in}{3.155929in}}{\pgfqpoint{1.953944in}{3.159201in}}{\pgfqpoint{1.959768in}{3.165025in}}%
\pgfpathcurveto{\pgfqpoint{1.965592in}{3.170849in}}{\pgfqpoint{1.968864in}{3.178749in}}{\pgfqpoint{1.968864in}{3.186985in}}%
\pgfpathcurveto{\pgfqpoint{1.968864in}{3.195221in}}{\pgfqpoint{1.965592in}{3.203121in}}{\pgfqpoint{1.959768in}{3.208945in}}%
\pgfpathcurveto{\pgfqpoint{1.953944in}{3.214769in}}{\pgfqpoint{1.946044in}{3.218042in}}{\pgfqpoint{1.937808in}{3.218042in}}%
\pgfpathcurveto{\pgfqpoint{1.929571in}{3.218042in}}{\pgfqpoint{1.921671in}{3.214769in}}{\pgfqpoint{1.915847in}{3.208945in}}%
\pgfpathcurveto{\pgfqpoint{1.910023in}{3.203121in}}{\pgfqpoint{1.906751in}{3.195221in}}{\pgfqpoint{1.906751in}{3.186985in}}%
\pgfpathcurveto{\pgfqpoint{1.906751in}{3.178749in}}{\pgfqpoint{1.910023in}{3.170849in}}{\pgfqpoint{1.915847in}{3.165025in}}%
\pgfpathcurveto{\pgfqpoint{1.921671in}{3.159201in}}{\pgfqpoint{1.929571in}{3.155929in}}{\pgfqpoint{1.937808in}{3.155929in}}%
\pgfpathclose%
\pgfusepath{stroke,fill}%
\end{pgfscope}%
\begin{pgfscope}%
\pgfpathrectangle{\pgfqpoint{0.100000in}{0.212622in}}{\pgfqpoint{3.696000in}{3.696000in}}%
\pgfusepath{clip}%
\pgfsetbuttcap%
\pgfsetroundjoin%
\definecolor{currentfill}{rgb}{0.121569,0.466667,0.705882}%
\pgfsetfillcolor{currentfill}%
\pgfsetfillopacity{0.339164}%
\pgfsetlinewidth{1.003750pt}%
\definecolor{currentstroke}{rgb}{0.121569,0.466667,0.705882}%
\pgfsetstrokecolor{currentstroke}%
\pgfsetstrokeopacity{0.339164}%
\pgfsetdash{}{0pt}%
\pgfpathmoveto{\pgfqpoint{1.938345in}{3.152561in}}%
\pgfpathcurveto{\pgfqpoint{1.946582in}{3.152561in}}{\pgfqpoint{1.954482in}{3.155834in}}{\pgfqpoint{1.960306in}{3.161658in}}%
\pgfpathcurveto{\pgfqpoint{1.966130in}{3.167482in}}{\pgfqpoint{1.969402in}{3.175382in}}{\pgfqpoint{1.969402in}{3.183618in}}%
\pgfpathcurveto{\pgfqpoint{1.969402in}{3.191854in}}{\pgfqpoint{1.966130in}{3.199754in}}{\pgfqpoint{1.960306in}{3.205578in}}%
\pgfpathcurveto{\pgfqpoint{1.954482in}{3.211402in}}{\pgfqpoint{1.946582in}{3.214674in}}{\pgfqpoint{1.938345in}{3.214674in}}%
\pgfpathcurveto{\pgfqpoint{1.930109in}{3.214674in}}{\pgfqpoint{1.922209in}{3.211402in}}{\pgfqpoint{1.916385in}{3.205578in}}%
\pgfpathcurveto{\pgfqpoint{1.910561in}{3.199754in}}{\pgfqpoint{1.907289in}{3.191854in}}{\pgfqpoint{1.907289in}{3.183618in}}%
\pgfpathcurveto{\pgfqpoint{1.907289in}{3.175382in}}{\pgfqpoint{1.910561in}{3.167482in}}{\pgfqpoint{1.916385in}{3.161658in}}%
\pgfpathcurveto{\pgfqpoint{1.922209in}{3.155834in}}{\pgfqpoint{1.930109in}{3.152561in}}{\pgfqpoint{1.938345in}{3.152561in}}%
\pgfpathclose%
\pgfusepath{stroke,fill}%
\end{pgfscope}%
\begin{pgfscope}%
\pgfpathrectangle{\pgfqpoint{0.100000in}{0.212622in}}{\pgfqpoint{3.696000in}{3.696000in}}%
\pgfusepath{clip}%
\pgfsetbuttcap%
\pgfsetroundjoin%
\definecolor{currentfill}{rgb}{0.121569,0.466667,0.705882}%
\pgfsetfillcolor{currentfill}%
\pgfsetfillopacity{0.339173}%
\pgfsetlinewidth{1.003750pt}%
\definecolor{currentstroke}{rgb}{0.121569,0.466667,0.705882}%
\pgfsetstrokecolor{currentstroke}%
\pgfsetstrokeopacity{0.339173}%
\pgfsetdash{}{0pt}%
\pgfpathmoveto{\pgfqpoint{1.753584in}{3.147815in}}%
\pgfpathcurveto{\pgfqpoint{1.761820in}{3.147815in}}{\pgfqpoint{1.769720in}{3.151087in}}{\pgfqpoint{1.775544in}{3.156911in}}%
\pgfpathcurveto{\pgfqpoint{1.781368in}{3.162735in}}{\pgfqpoint{1.784640in}{3.170635in}}{\pgfqpoint{1.784640in}{3.178871in}}%
\pgfpathcurveto{\pgfqpoint{1.784640in}{3.187107in}}{\pgfqpoint{1.781368in}{3.195007in}}{\pgfqpoint{1.775544in}{3.200831in}}%
\pgfpathcurveto{\pgfqpoint{1.769720in}{3.206655in}}{\pgfqpoint{1.761820in}{3.209928in}}{\pgfqpoint{1.753584in}{3.209928in}}%
\pgfpathcurveto{\pgfqpoint{1.745348in}{3.209928in}}{\pgfqpoint{1.737448in}{3.206655in}}{\pgfqpoint{1.731624in}{3.200831in}}%
\pgfpathcurveto{\pgfqpoint{1.725800in}{3.195007in}}{\pgfqpoint{1.722527in}{3.187107in}}{\pgfqpoint{1.722527in}{3.178871in}}%
\pgfpathcurveto{\pgfqpoint{1.722527in}{3.170635in}}{\pgfqpoint{1.725800in}{3.162735in}}{\pgfqpoint{1.731624in}{3.156911in}}%
\pgfpathcurveto{\pgfqpoint{1.737448in}{3.151087in}}{\pgfqpoint{1.745348in}{3.147815in}}{\pgfqpoint{1.753584in}{3.147815in}}%
\pgfpathclose%
\pgfusepath{stroke,fill}%
\end{pgfscope}%
\begin{pgfscope}%
\pgfpathrectangle{\pgfqpoint{0.100000in}{0.212622in}}{\pgfqpoint{3.696000in}{3.696000in}}%
\pgfusepath{clip}%
\pgfsetbuttcap%
\pgfsetroundjoin%
\definecolor{currentfill}{rgb}{0.121569,0.466667,0.705882}%
\pgfsetfillcolor{currentfill}%
\pgfsetfillopacity{0.340164}%
\pgfsetlinewidth{1.003750pt}%
\definecolor{currentstroke}{rgb}{0.121569,0.466667,0.705882}%
\pgfsetstrokecolor{currentstroke}%
\pgfsetstrokeopacity{0.340164}%
\pgfsetdash{}{0pt}%
\pgfpathmoveto{\pgfqpoint{1.939217in}{3.148503in}}%
\pgfpathcurveto{\pgfqpoint{1.947453in}{3.148503in}}{\pgfqpoint{1.955353in}{3.151775in}}{\pgfqpoint{1.961177in}{3.157599in}}%
\pgfpathcurveto{\pgfqpoint{1.967001in}{3.163423in}}{\pgfqpoint{1.970273in}{3.171323in}}{\pgfqpoint{1.970273in}{3.179559in}}%
\pgfpathcurveto{\pgfqpoint{1.970273in}{3.187796in}}{\pgfqpoint{1.967001in}{3.195696in}}{\pgfqpoint{1.961177in}{3.201520in}}%
\pgfpathcurveto{\pgfqpoint{1.955353in}{3.207343in}}{\pgfqpoint{1.947453in}{3.210616in}}{\pgfqpoint{1.939217in}{3.210616in}}%
\pgfpathcurveto{\pgfqpoint{1.930980in}{3.210616in}}{\pgfqpoint{1.923080in}{3.207343in}}{\pgfqpoint{1.917256in}{3.201520in}}%
\pgfpathcurveto{\pgfqpoint{1.911432in}{3.195696in}}{\pgfqpoint{1.908160in}{3.187796in}}{\pgfqpoint{1.908160in}{3.179559in}}%
\pgfpathcurveto{\pgfqpoint{1.908160in}{3.171323in}}{\pgfqpoint{1.911432in}{3.163423in}}{\pgfqpoint{1.917256in}{3.157599in}}%
\pgfpathcurveto{\pgfqpoint{1.923080in}{3.151775in}}{\pgfqpoint{1.930980in}{3.148503in}}{\pgfqpoint{1.939217in}{3.148503in}}%
\pgfpathclose%
\pgfusepath{stroke,fill}%
\end{pgfscope}%
\begin{pgfscope}%
\pgfpathrectangle{\pgfqpoint{0.100000in}{0.212622in}}{\pgfqpoint{3.696000in}{3.696000in}}%
\pgfusepath{clip}%
\pgfsetbuttcap%
\pgfsetroundjoin%
\definecolor{currentfill}{rgb}{0.121569,0.466667,0.705882}%
\pgfsetfillcolor{currentfill}%
\pgfsetfillopacity{0.340687}%
\pgfsetlinewidth{1.003750pt}%
\definecolor{currentstroke}{rgb}{0.121569,0.466667,0.705882}%
\pgfsetstrokecolor{currentstroke}%
\pgfsetstrokeopacity{0.340687}%
\pgfsetdash{}{0pt}%
\pgfpathmoveto{\pgfqpoint{1.749440in}{3.139982in}}%
\pgfpathcurveto{\pgfqpoint{1.757676in}{3.139982in}}{\pgfqpoint{1.765576in}{3.143254in}}{\pgfqpoint{1.771400in}{3.149078in}}%
\pgfpathcurveto{\pgfqpoint{1.777224in}{3.154902in}}{\pgfqpoint{1.780497in}{3.162802in}}{\pgfqpoint{1.780497in}{3.171039in}}%
\pgfpathcurveto{\pgfqpoint{1.780497in}{3.179275in}}{\pgfqpoint{1.777224in}{3.187175in}}{\pgfqpoint{1.771400in}{3.192999in}}%
\pgfpathcurveto{\pgfqpoint{1.765576in}{3.198823in}}{\pgfqpoint{1.757676in}{3.202095in}}{\pgfqpoint{1.749440in}{3.202095in}}%
\pgfpathcurveto{\pgfqpoint{1.741204in}{3.202095in}}{\pgfqpoint{1.733304in}{3.198823in}}{\pgfqpoint{1.727480in}{3.192999in}}%
\pgfpathcurveto{\pgfqpoint{1.721656in}{3.187175in}}{\pgfqpoint{1.718384in}{3.179275in}}{\pgfqpoint{1.718384in}{3.171039in}}%
\pgfpathcurveto{\pgfqpoint{1.718384in}{3.162802in}}{\pgfqpoint{1.721656in}{3.154902in}}{\pgfqpoint{1.727480in}{3.149078in}}%
\pgfpathcurveto{\pgfqpoint{1.733304in}{3.143254in}}{\pgfqpoint{1.741204in}{3.139982in}}{\pgfqpoint{1.749440in}{3.139982in}}%
\pgfpathclose%
\pgfusepath{stroke,fill}%
\end{pgfscope}%
\begin{pgfscope}%
\pgfpathrectangle{\pgfqpoint{0.100000in}{0.212622in}}{\pgfqpoint{3.696000in}{3.696000in}}%
\pgfusepath{clip}%
\pgfsetbuttcap%
\pgfsetroundjoin%
\definecolor{currentfill}{rgb}{0.121569,0.466667,0.705882}%
\pgfsetfillcolor{currentfill}%
\pgfsetfillopacity{0.341495}%
\pgfsetlinewidth{1.003750pt}%
\definecolor{currentstroke}{rgb}{0.121569,0.466667,0.705882}%
\pgfsetstrokecolor{currentstroke}%
\pgfsetstrokeopacity{0.341495}%
\pgfsetdash{}{0pt}%
\pgfpathmoveto{\pgfqpoint{1.939902in}{3.143621in}}%
\pgfpathcurveto{\pgfqpoint{1.948139in}{3.143621in}}{\pgfqpoint{1.956039in}{3.146893in}}{\pgfqpoint{1.961863in}{3.152717in}}%
\pgfpathcurveto{\pgfqpoint{1.967686in}{3.158541in}}{\pgfqpoint{1.970959in}{3.166441in}}{\pgfqpoint{1.970959in}{3.174677in}}%
\pgfpathcurveto{\pgfqpoint{1.970959in}{3.182913in}}{\pgfqpoint{1.967686in}{3.190813in}}{\pgfqpoint{1.961863in}{3.196637in}}%
\pgfpathcurveto{\pgfqpoint{1.956039in}{3.202461in}}{\pgfqpoint{1.948139in}{3.205734in}}{\pgfqpoint{1.939902in}{3.205734in}}%
\pgfpathcurveto{\pgfqpoint{1.931666in}{3.205734in}}{\pgfqpoint{1.923766in}{3.202461in}}{\pgfqpoint{1.917942in}{3.196637in}}%
\pgfpathcurveto{\pgfqpoint{1.912118in}{3.190813in}}{\pgfqpoint{1.908846in}{3.182913in}}{\pgfqpoint{1.908846in}{3.174677in}}%
\pgfpathcurveto{\pgfqpoint{1.908846in}{3.166441in}}{\pgfqpoint{1.912118in}{3.158541in}}{\pgfqpoint{1.917942in}{3.152717in}}%
\pgfpathcurveto{\pgfqpoint{1.923766in}{3.146893in}}{\pgfqpoint{1.931666in}{3.143621in}}{\pgfqpoint{1.939902in}{3.143621in}}%
\pgfpathclose%
\pgfusepath{stroke,fill}%
\end{pgfscope}%
\begin{pgfscope}%
\pgfpathrectangle{\pgfqpoint{0.100000in}{0.212622in}}{\pgfqpoint{3.696000in}{3.696000in}}%
\pgfusepath{clip}%
\pgfsetbuttcap%
\pgfsetroundjoin%
\definecolor{currentfill}{rgb}{0.121569,0.466667,0.705882}%
\pgfsetfillcolor{currentfill}%
\pgfsetfillopacity{0.341765}%
\pgfsetlinewidth{1.003750pt}%
\definecolor{currentstroke}{rgb}{0.121569,0.466667,0.705882}%
\pgfsetstrokecolor{currentstroke}%
\pgfsetstrokeopacity{0.341765}%
\pgfsetdash{}{0pt}%
\pgfpathmoveto{\pgfqpoint{1.745707in}{3.134075in}}%
\pgfpathcurveto{\pgfqpoint{1.753944in}{3.134075in}}{\pgfqpoint{1.761844in}{3.137347in}}{\pgfqpoint{1.767668in}{3.143171in}}%
\pgfpathcurveto{\pgfqpoint{1.773492in}{3.148995in}}{\pgfqpoint{1.776764in}{3.156895in}}{\pgfqpoint{1.776764in}{3.165131in}}%
\pgfpathcurveto{\pgfqpoint{1.776764in}{3.173368in}}{\pgfqpoint{1.773492in}{3.181268in}}{\pgfqpoint{1.767668in}{3.187092in}}%
\pgfpathcurveto{\pgfqpoint{1.761844in}{3.192915in}}{\pgfqpoint{1.753944in}{3.196188in}}{\pgfqpoint{1.745707in}{3.196188in}}%
\pgfpathcurveto{\pgfqpoint{1.737471in}{3.196188in}}{\pgfqpoint{1.729571in}{3.192915in}}{\pgfqpoint{1.723747in}{3.187092in}}%
\pgfpathcurveto{\pgfqpoint{1.717923in}{3.181268in}}{\pgfqpoint{1.714651in}{3.173368in}}{\pgfqpoint{1.714651in}{3.165131in}}%
\pgfpathcurveto{\pgfqpoint{1.714651in}{3.156895in}}{\pgfqpoint{1.717923in}{3.148995in}}{\pgfqpoint{1.723747in}{3.143171in}}%
\pgfpathcurveto{\pgfqpoint{1.729571in}{3.137347in}}{\pgfqpoint{1.737471in}{3.134075in}}{\pgfqpoint{1.745707in}{3.134075in}}%
\pgfpathclose%
\pgfusepath{stroke,fill}%
\end{pgfscope}%
\begin{pgfscope}%
\pgfpathrectangle{\pgfqpoint{0.100000in}{0.212622in}}{\pgfqpoint{3.696000in}{3.696000in}}%
\pgfusepath{clip}%
\pgfsetbuttcap%
\pgfsetroundjoin%
\definecolor{currentfill}{rgb}{0.121569,0.466667,0.705882}%
\pgfsetfillcolor{currentfill}%
\pgfsetfillopacity{0.342700}%
\pgfsetlinewidth{1.003750pt}%
\definecolor{currentstroke}{rgb}{0.121569,0.466667,0.705882}%
\pgfsetstrokecolor{currentstroke}%
\pgfsetstrokeopacity{0.342700}%
\pgfsetdash{}{0pt}%
\pgfpathmoveto{\pgfqpoint{1.743265in}{3.129223in}}%
\pgfpathcurveto{\pgfqpoint{1.751501in}{3.129223in}}{\pgfqpoint{1.759401in}{3.132495in}}{\pgfqpoint{1.765225in}{3.138319in}}%
\pgfpathcurveto{\pgfqpoint{1.771049in}{3.144143in}}{\pgfqpoint{1.774322in}{3.152043in}}{\pgfqpoint{1.774322in}{3.160279in}}%
\pgfpathcurveto{\pgfqpoint{1.774322in}{3.168515in}}{\pgfqpoint{1.771049in}{3.176415in}}{\pgfqpoint{1.765225in}{3.182239in}}%
\pgfpathcurveto{\pgfqpoint{1.759401in}{3.188063in}}{\pgfqpoint{1.751501in}{3.191336in}}{\pgfqpoint{1.743265in}{3.191336in}}%
\pgfpathcurveto{\pgfqpoint{1.735029in}{3.191336in}}{\pgfqpoint{1.727129in}{3.188063in}}{\pgfqpoint{1.721305in}{3.182239in}}%
\pgfpathcurveto{\pgfqpoint{1.715481in}{3.176415in}}{\pgfqpoint{1.712209in}{3.168515in}}{\pgfqpoint{1.712209in}{3.160279in}}%
\pgfpathcurveto{\pgfqpoint{1.712209in}{3.152043in}}{\pgfqpoint{1.715481in}{3.144143in}}{\pgfqpoint{1.721305in}{3.138319in}}%
\pgfpathcurveto{\pgfqpoint{1.727129in}{3.132495in}}{\pgfqpoint{1.735029in}{3.129223in}}{\pgfqpoint{1.743265in}{3.129223in}}%
\pgfpathclose%
\pgfusepath{stroke,fill}%
\end{pgfscope}%
\begin{pgfscope}%
\pgfpathrectangle{\pgfqpoint{0.100000in}{0.212622in}}{\pgfqpoint{3.696000in}{3.696000in}}%
\pgfusepath{clip}%
\pgfsetbuttcap%
\pgfsetroundjoin%
\definecolor{currentfill}{rgb}{0.121569,0.466667,0.705882}%
\pgfsetfillcolor{currentfill}%
\pgfsetfillopacity{0.343050}%
\pgfsetlinewidth{1.003750pt}%
\definecolor{currentstroke}{rgb}{0.121569,0.466667,0.705882}%
\pgfsetstrokecolor{currentstroke}%
\pgfsetstrokeopacity{0.343050}%
\pgfsetdash{}{0pt}%
\pgfpathmoveto{\pgfqpoint{1.940891in}{3.137992in}}%
\pgfpathcurveto{\pgfqpoint{1.949128in}{3.137992in}}{\pgfqpoint{1.957028in}{3.141264in}}{\pgfqpoint{1.962852in}{3.147088in}}%
\pgfpathcurveto{\pgfqpoint{1.968676in}{3.152912in}}{\pgfqpoint{1.971948in}{3.160812in}}{\pgfqpoint{1.971948in}{3.169048in}}%
\pgfpathcurveto{\pgfqpoint{1.971948in}{3.177284in}}{\pgfqpoint{1.968676in}{3.185184in}}{\pgfqpoint{1.962852in}{3.191008in}}%
\pgfpathcurveto{\pgfqpoint{1.957028in}{3.196832in}}{\pgfqpoint{1.949128in}{3.200105in}}{\pgfqpoint{1.940891in}{3.200105in}}%
\pgfpathcurveto{\pgfqpoint{1.932655in}{3.200105in}}{\pgfqpoint{1.924755in}{3.196832in}}{\pgfqpoint{1.918931in}{3.191008in}}%
\pgfpathcurveto{\pgfqpoint{1.913107in}{3.185184in}}{\pgfqpoint{1.909835in}{3.177284in}}{\pgfqpoint{1.909835in}{3.169048in}}%
\pgfpathcurveto{\pgfqpoint{1.909835in}{3.160812in}}{\pgfqpoint{1.913107in}{3.152912in}}{\pgfqpoint{1.918931in}{3.147088in}}%
\pgfpathcurveto{\pgfqpoint{1.924755in}{3.141264in}}{\pgfqpoint{1.932655in}{3.137992in}}{\pgfqpoint{1.940891in}{3.137992in}}%
\pgfpathclose%
\pgfusepath{stroke,fill}%
\end{pgfscope}%
\begin{pgfscope}%
\pgfpathrectangle{\pgfqpoint{0.100000in}{0.212622in}}{\pgfqpoint{3.696000in}{3.696000in}}%
\pgfusepath{clip}%
\pgfsetbuttcap%
\pgfsetroundjoin%
\definecolor{currentfill}{rgb}{0.121569,0.466667,0.705882}%
\pgfsetfillcolor{currentfill}%
\pgfsetfillopacity{0.343533}%
\pgfsetlinewidth{1.003750pt}%
\definecolor{currentstroke}{rgb}{0.121569,0.466667,0.705882}%
\pgfsetstrokecolor{currentstroke}%
\pgfsetstrokeopacity{0.343533}%
\pgfsetdash{}{0pt}%
\pgfpathmoveto{\pgfqpoint{1.740820in}{3.124912in}}%
\pgfpathcurveto{\pgfqpoint{1.749057in}{3.124912in}}{\pgfqpoint{1.756957in}{3.128184in}}{\pgfqpoint{1.762781in}{3.134008in}}%
\pgfpathcurveto{\pgfqpoint{1.768604in}{3.139832in}}{\pgfqpoint{1.771877in}{3.147732in}}{\pgfqpoint{1.771877in}{3.155968in}}%
\pgfpathcurveto{\pgfqpoint{1.771877in}{3.164205in}}{\pgfqpoint{1.768604in}{3.172105in}}{\pgfqpoint{1.762781in}{3.177929in}}%
\pgfpathcurveto{\pgfqpoint{1.756957in}{3.183753in}}{\pgfqpoint{1.749057in}{3.187025in}}{\pgfqpoint{1.740820in}{3.187025in}}%
\pgfpathcurveto{\pgfqpoint{1.732584in}{3.187025in}}{\pgfqpoint{1.724684in}{3.183753in}}{\pgfqpoint{1.718860in}{3.177929in}}%
\pgfpathcurveto{\pgfqpoint{1.713036in}{3.172105in}}{\pgfqpoint{1.709764in}{3.164205in}}{\pgfqpoint{1.709764in}{3.155968in}}%
\pgfpathcurveto{\pgfqpoint{1.709764in}{3.147732in}}{\pgfqpoint{1.713036in}{3.139832in}}{\pgfqpoint{1.718860in}{3.134008in}}%
\pgfpathcurveto{\pgfqpoint{1.724684in}{3.128184in}}{\pgfqpoint{1.732584in}{3.124912in}}{\pgfqpoint{1.740820in}{3.124912in}}%
\pgfpathclose%
\pgfusepath{stroke,fill}%
\end{pgfscope}%
\begin{pgfscope}%
\pgfpathrectangle{\pgfqpoint{0.100000in}{0.212622in}}{\pgfqpoint{3.696000in}{3.696000in}}%
\pgfusepath{clip}%
\pgfsetbuttcap%
\pgfsetroundjoin%
\definecolor{currentfill}{rgb}{0.121569,0.466667,0.705882}%
\pgfsetfillcolor{currentfill}%
\pgfsetfillopacity{0.344257}%
\pgfsetlinewidth{1.003750pt}%
\definecolor{currentstroke}{rgb}{0.121569,0.466667,0.705882}%
\pgfsetstrokecolor{currentstroke}%
\pgfsetstrokeopacity{0.344257}%
\pgfsetdash{}{0pt}%
\pgfpathmoveto{\pgfqpoint{1.738423in}{3.120936in}}%
\pgfpathcurveto{\pgfqpoint{1.746659in}{3.120936in}}{\pgfqpoint{1.754559in}{3.124209in}}{\pgfqpoint{1.760383in}{3.130033in}}%
\pgfpathcurveto{\pgfqpoint{1.766207in}{3.135856in}}{\pgfqpoint{1.769479in}{3.143757in}}{\pgfqpoint{1.769479in}{3.151993in}}%
\pgfpathcurveto{\pgfqpoint{1.769479in}{3.160229in}}{\pgfqpoint{1.766207in}{3.168129in}}{\pgfqpoint{1.760383in}{3.173953in}}%
\pgfpathcurveto{\pgfqpoint{1.754559in}{3.179777in}}{\pgfqpoint{1.746659in}{3.183049in}}{\pgfqpoint{1.738423in}{3.183049in}}%
\pgfpathcurveto{\pgfqpoint{1.730186in}{3.183049in}}{\pgfqpoint{1.722286in}{3.179777in}}{\pgfqpoint{1.716462in}{3.173953in}}%
\pgfpathcurveto{\pgfqpoint{1.710639in}{3.168129in}}{\pgfqpoint{1.707366in}{3.160229in}}{\pgfqpoint{1.707366in}{3.151993in}}%
\pgfpathcurveto{\pgfqpoint{1.707366in}{3.143757in}}{\pgfqpoint{1.710639in}{3.135856in}}{\pgfqpoint{1.716462in}{3.130033in}}%
\pgfpathcurveto{\pgfqpoint{1.722286in}{3.124209in}}{\pgfqpoint{1.730186in}{3.120936in}}{\pgfqpoint{1.738423in}{3.120936in}}%
\pgfpathclose%
\pgfusepath{stroke,fill}%
\end{pgfscope}%
\begin{pgfscope}%
\pgfpathrectangle{\pgfqpoint{0.100000in}{0.212622in}}{\pgfqpoint{3.696000in}{3.696000in}}%
\pgfusepath{clip}%
\pgfsetbuttcap%
\pgfsetroundjoin%
\definecolor{currentfill}{rgb}{0.121569,0.466667,0.705882}%
\pgfsetfillcolor{currentfill}%
\pgfsetfillopacity{0.344462}%
\pgfsetlinewidth{1.003750pt}%
\definecolor{currentstroke}{rgb}{0.121569,0.466667,0.705882}%
\pgfsetstrokecolor{currentstroke}%
\pgfsetstrokeopacity{0.344462}%
\pgfsetdash{}{0pt}%
\pgfpathmoveto{\pgfqpoint{1.942286in}{3.130855in}}%
\pgfpathcurveto{\pgfqpoint{1.950523in}{3.130855in}}{\pgfqpoint{1.958423in}{3.134127in}}{\pgfqpoint{1.964247in}{3.139951in}}%
\pgfpathcurveto{\pgfqpoint{1.970071in}{3.145775in}}{\pgfqpoint{1.973343in}{3.153675in}}{\pgfqpoint{1.973343in}{3.161911in}}%
\pgfpathcurveto{\pgfqpoint{1.973343in}{3.170147in}}{\pgfqpoint{1.970071in}{3.178047in}}{\pgfqpoint{1.964247in}{3.183871in}}%
\pgfpathcurveto{\pgfqpoint{1.958423in}{3.189695in}}{\pgfqpoint{1.950523in}{3.192968in}}{\pgfqpoint{1.942286in}{3.192968in}}%
\pgfpathcurveto{\pgfqpoint{1.934050in}{3.192968in}}{\pgfqpoint{1.926150in}{3.189695in}}{\pgfqpoint{1.920326in}{3.183871in}}%
\pgfpathcurveto{\pgfqpoint{1.914502in}{3.178047in}}{\pgfqpoint{1.911230in}{3.170147in}}{\pgfqpoint{1.911230in}{3.161911in}}%
\pgfpathcurveto{\pgfqpoint{1.911230in}{3.153675in}}{\pgfqpoint{1.914502in}{3.145775in}}{\pgfqpoint{1.920326in}{3.139951in}}%
\pgfpathcurveto{\pgfqpoint{1.926150in}{3.134127in}}{\pgfqpoint{1.934050in}{3.130855in}}{\pgfqpoint{1.942286in}{3.130855in}}%
\pgfpathclose%
\pgfusepath{stroke,fill}%
\end{pgfscope}%
\begin{pgfscope}%
\pgfpathrectangle{\pgfqpoint{0.100000in}{0.212622in}}{\pgfqpoint{3.696000in}{3.696000in}}%
\pgfusepath{clip}%
\pgfsetbuttcap%
\pgfsetroundjoin%
\definecolor{currentfill}{rgb}{0.121569,0.466667,0.705882}%
\pgfsetfillcolor{currentfill}%
\pgfsetfillopacity{0.345636}%
\pgfsetlinewidth{1.003750pt}%
\definecolor{currentstroke}{rgb}{0.121569,0.466667,0.705882}%
\pgfsetstrokecolor{currentstroke}%
\pgfsetstrokeopacity{0.345636}%
\pgfsetdash{}{0pt}%
\pgfpathmoveto{\pgfqpoint{1.734747in}{3.113079in}}%
\pgfpathcurveto{\pgfqpoint{1.742983in}{3.113079in}}{\pgfqpoint{1.750883in}{3.116351in}}{\pgfqpoint{1.756707in}{3.122175in}}%
\pgfpathcurveto{\pgfqpoint{1.762531in}{3.127999in}}{\pgfqpoint{1.765803in}{3.135899in}}{\pgfqpoint{1.765803in}{3.144135in}}%
\pgfpathcurveto{\pgfqpoint{1.765803in}{3.152372in}}{\pgfqpoint{1.762531in}{3.160272in}}{\pgfqpoint{1.756707in}{3.166096in}}%
\pgfpathcurveto{\pgfqpoint{1.750883in}{3.171920in}}{\pgfqpoint{1.742983in}{3.175192in}}{\pgfqpoint{1.734747in}{3.175192in}}%
\pgfpathcurveto{\pgfqpoint{1.726511in}{3.175192in}}{\pgfqpoint{1.718611in}{3.171920in}}{\pgfqpoint{1.712787in}{3.166096in}}%
\pgfpathcurveto{\pgfqpoint{1.706963in}{3.160272in}}{\pgfqpoint{1.703690in}{3.152372in}}{\pgfqpoint{1.703690in}{3.144135in}}%
\pgfpathcurveto{\pgfqpoint{1.703690in}{3.135899in}}{\pgfqpoint{1.706963in}{3.127999in}}{\pgfqpoint{1.712787in}{3.122175in}}%
\pgfpathcurveto{\pgfqpoint{1.718611in}{3.116351in}}{\pgfqpoint{1.726511in}{3.113079in}}{\pgfqpoint{1.734747in}{3.113079in}}%
\pgfpathclose%
\pgfusepath{stroke,fill}%
\end{pgfscope}%
\begin{pgfscope}%
\pgfpathrectangle{\pgfqpoint{0.100000in}{0.212622in}}{\pgfqpoint{3.696000in}{3.696000in}}%
\pgfusepath{clip}%
\pgfsetbuttcap%
\pgfsetroundjoin%
\definecolor{currentfill}{rgb}{0.121569,0.466667,0.705882}%
\pgfsetfillcolor{currentfill}%
\pgfsetfillopacity{0.346474}%
\pgfsetlinewidth{1.003750pt}%
\definecolor{currentstroke}{rgb}{0.121569,0.466667,0.705882}%
\pgfsetstrokecolor{currentstroke}%
\pgfsetstrokeopacity{0.346474}%
\pgfsetdash{}{0pt}%
\pgfpathmoveto{\pgfqpoint{1.731981in}{3.108816in}}%
\pgfpathcurveto{\pgfqpoint{1.740217in}{3.108816in}}{\pgfqpoint{1.748118in}{3.112089in}}{\pgfqpoint{1.753941in}{3.117913in}}%
\pgfpathcurveto{\pgfqpoint{1.759765in}{3.123736in}}{\pgfqpoint{1.763038in}{3.131637in}}{\pgfqpoint{1.763038in}{3.139873in}}%
\pgfpathcurveto{\pgfqpoint{1.763038in}{3.148109in}}{\pgfqpoint{1.759765in}{3.156009in}}{\pgfqpoint{1.753941in}{3.161833in}}%
\pgfpathcurveto{\pgfqpoint{1.748118in}{3.167657in}}{\pgfqpoint{1.740217in}{3.170929in}}{\pgfqpoint{1.731981in}{3.170929in}}%
\pgfpathcurveto{\pgfqpoint{1.723745in}{3.170929in}}{\pgfqpoint{1.715845in}{3.167657in}}{\pgfqpoint{1.710021in}{3.161833in}}%
\pgfpathcurveto{\pgfqpoint{1.704197in}{3.156009in}}{\pgfqpoint{1.700925in}{3.148109in}}{\pgfqpoint{1.700925in}{3.139873in}}%
\pgfpathcurveto{\pgfqpoint{1.700925in}{3.131637in}}{\pgfqpoint{1.704197in}{3.123736in}}{\pgfqpoint{1.710021in}{3.117913in}}%
\pgfpathcurveto{\pgfqpoint{1.715845in}{3.112089in}}{\pgfqpoint{1.723745in}{3.108816in}}{\pgfqpoint{1.731981in}{3.108816in}}%
\pgfpathclose%
\pgfusepath{stroke,fill}%
\end{pgfscope}%
\begin{pgfscope}%
\pgfpathrectangle{\pgfqpoint{0.100000in}{0.212622in}}{\pgfqpoint{3.696000in}{3.696000in}}%
\pgfusepath{clip}%
\pgfsetbuttcap%
\pgfsetroundjoin%
\definecolor{currentfill}{rgb}{0.121569,0.466667,0.705882}%
\pgfsetfillcolor{currentfill}%
\pgfsetfillopacity{0.346677}%
\pgfsetlinewidth{1.003750pt}%
\definecolor{currentstroke}{rgb}{0.121569,0.466667,0.705882}%
\pgfsetstrokecolor{currentstroke}%
\pgfsetstrokeopacity{0.346677}%
\pgfsetdash{}{0pt}%
\pgfpathmoveto{\pgfqpoint{1.943341in}{3.122761in}}%
\pgfpathcurveto{\pgfqpoint{1.951577in}{3.122761in}}{\pgfqpoint{1.959477in}{3.126033in}}{\pgfqpoint{1.965301in}{3.131857in}}%
\pgfpathcurveto{\pgfqpoint{1.971125in}{3.137681in}}{\pgfqpoint{1.974398in}{3.145581in}}{\pgfqpoint{1.974398in}{3.153818in}}%
\pgfpathcurveto{\pgfqpoint{1.974398in}{3.162054in}}{\pgfqpoint{1.971125in}{3.169954in}}{\pgfqpoint{1.965301in}{3.175778in}}%
\pgfpathcurveto{\pgfqpoint{1.959477in}{3.181602in}}{\pgfqpoint{1.951577in}{3.184874in}}{\pgfqpoint{1.943341in}{3.184874in}}%
\pgfpathcurveto{\pgfqpoint{1.935105in}{3.184874in}}{\pgfqpoint{1.927205in}{3.181602in}}{\pgfqpoint{1.921381in}{3.175778in}}%
\pgfpathcurveto{\pgfqpoint{1.915557in}{3.169954in}}{\pgfqpoint{1.912285in}{3.162054in}}{\pgfqpoint{1.912285in}{3.153818in}}%
\pgfpathcurveto{\pgfqpoint{1.912285in}{3.145581in}}{\pgfqpoint{1.915557in}{3.137681in}}{\pgfqpoint{1.921381in}{3.131857in}}%
\pgfpathcurveto{\pgfqpoint{1.927205in}{3.126033in}}{\pgfqpoint{1.935105in}{3.122761in}}{\pgfqpoint{1.943341in}{3.122761in}}%
\pgfpathclose%
\pgfusepath{stroke,fill}%
\end{pgfscope}%
\begin{pgfscope}%
\pgfpathrectangle{\pgfqpoint{0.100000in}{0.212622in}}{\pgfqpoint{3.696000in}{3.696000in}}%
\pgfusepath{clip}%
\pgfsetbuttcap%
\pgfsetroundjoin%
\definecolor{currentfill}{rgb}{0.121569,0.466667,0.705882}%
\pgfsetfillcolor{currentfill}%
\pgfsetfillopacity{0.347247}%
\pgfsetlinewidth{1.003750pt}%
\definecolor{currentstroke}{rgb}{0.121569,0.466667,0.705882}%
\pgfsetstrokecolor{currentstroke}%
\pgfsetstrokeopacity{0.347247}%
\pgfsetdash{}{0pt}%
\pgfpathmoveto{\pgfqpoint{1.729904in}{3.105021in}}%
\pgfpathcurveto{\pgfqpoint{1.738141in}{3.105021in}}{\pgfqpoint{1.746041in}{3.108293in}}{\pgfqpoint{1.751865in}{3.114117in}}%
\pgfpathcurveto{\pgfqpoint{1.757688in}{3.119941in}}{\pgfqpoint{1.760961in}{3.127841in}}{\pgfqpoint{1.760961in}{3.136077in}}%
\pgfpathcurveto{\pgfqpoint{1.760961in}{3.144314in}}{\pgfqpoint{1.757688in}{3.152214in}}{\pgfqpoint{1.751865in}{3.158038in}}%
\pgfpathcurveto{\pgfqpoint{1.746041in}{3.163862in}}{\pgfqpoint{1.738141in}{3.167134in}}{\pgfqpoint{1.729904in}{3.167134in}}%
\pgfpathcurveto{\pgfqpoint{1.721668in}{3.167134in}}{\pgfqpoint{1.713768in}{3.163862in}}{\pgfqpoint{1.707944in}{3.158038in}}%
\pgfpathcurveto{\pgfqpoint{1.702120in}{3.152214in}}{\pgfqpoint{1.698848in}{3.144314in}}{\pgfqpoint{1.698848in}{3.136077in}}%
\pgfpathcurveto{\pgfqpoint{1.698848in}{3.127841in}}{\pgfqpoint{1.702120in}{3.119941in}}{\pgfqpoint{1.707944in}{3.114117in}}%
\pgfpathcurveto{\pgfqpoint{1.713768in}{3.108293in}}{\pgfqpoint{1.721668in}{3.105021in}}{\pgfqpoint{1.729904in}{3.105021in}}%
\pgfpathclose%
\pgfusepath{stroke,fill}%
\end{pgfscope}%
\begin{pgfscope}%
\pgfpathrectangle{\pgfqpoint{0.100000in}{0.212622in}}{\pgfqpoint{3.696000in}{3.696000in}}%
\pgfusepath{clip}%
\pgfsetbuttcap%
\pgfsetroundjoin%
\definecolor{currentfill}{rgb}{0.121569,0.466667,0.705882}%
\pgfsetfillcolor{currentfill}%
\pgfsetfillopacity{0.348678}%
\pgfsetlinewidth{1.003750pt}%
\definecolor{currentstroke}{rgb}{0.121569,0.466667,0.705882}%
\pgfsetstrokecolor{currentstroke}%
\pgfsetstrokeopacity{0.348678}%
\pgfsetdash{}{0pt}%
\pgfpathmoveto{\pgfqpoint{1.726116in}{3.098233in}}%
\pgfpathcurveto{\pgfqpoint{1.734352in}{3.098233in}}{\pgfqpoint{1.742252in}{3.101505in}}{\pgfqpoint{1.748076in}{3.107329in}}%
\pgfpathcurveto{\pgfqpoint{1.753900in}{3.113153in}}{\pgfqpoint{1.757172in}{3.121053in}}{\pgfqpoint{1.757172in}{3.129289in}}%
\pgfpathcurveto{\pgfqpoint{1.757172in}{3.137525in}}{\pgfqpoint{1.753900in}{3.145425in}}{\pgfqpoint{1.748076in}{3.151249in}}%
\pgfpathcurveto{\pgfqpoint{1.742252in}{3.157073in}}{\pgfqpoint{1.734352in}{3.160346in}}{\pgfqpoint{1.726116in}{3.160346in}}%
\pgfpathcurveto{\pgfqpoint{1.717879in}{3.160346in}}{\pgfqpoint{1.709979in}{3.157073in}}{\pgfqpoint{1.704155in}{3.151249in}}%
\pgfpathcurveto{\pgfqpoint{1.698332in}{3.145425in}}{\pgfqpoint{1.695059in}{3.137525in}}{\pgfqpoint{1.695059in}{3.129289in}}%
\pgfpathcurveto{\pgfqpoint{1.695059in}{3.121053in}}{\pgfqpoint{1.698332in}{3.113153in}}{\pgfqpoint{1.704155in}{3.107329in}}%
\pgfpathcurveto{\pgfqpoint{1.709979in}{3.101505in}}{\pgfqpoint{1.717879in}{3.098233in}}{\pgfqpoint{1.726116in}{3.098233in}}%
\pgfpathclose%
\pgfusepath{stroke,fill}%
\end{pgfscope}%
\begin{pgfscope}%
\pgfpathrectangle{\pgfqpoint{0.100000in}{0.212622in}}{\pgfqpoint{3.696000in}{3.696000in}}%
\pgfusepath{clip}%
\pgfsetbuttcap%
\pgfsetroundjoin%
\definecolor{currentfill}{rgb}{0.121569,0.466667,0.705882}%
\pgfsetfillcolor{currentfill}%
\pgfsetfillopacity{0.348909}%
\pgfsetlinewidth{1.003750pt}%
\definecolor{currentstroke}{rgb}{0.121569,0.466667,0.705882}%
\pgfsetstrokecolor{currentstroke}%
\pgfsetstrokeopacity{0.348909}%
\pgfsetdash{}{0pt}%
\pgfpathmoveto{\pgfqpoint{1.944758in}{3.114297in}}%
\pgfpathcurveto{\pgfqpoint{1.952995in}{3.114297in}}{\pgfqpoint{1.960895in}{3.117569in}}{\pgfqpoint{1.966719in}{3.123393in}}%
\pgfpathcurveto{\pgfqpoint{1.972542in}{3.129217in}}{\pgfqpoint{1.975815in}{3.137117in}}{\pgfqpoint{1.975815in}{3.145353in}}%
\pgfpathcurveto{\pgfqpoint{1.975815in}{3.153590in}}{\pgfqpoint{1.972542in}{3.161490in}}{\pgfqpoint{1.966719in}{3.167314in}}%
\pgfpathcurveto{\pgfqpoint{1.960895in}{3.173138in}}{\pgfqpoint{1.952995in}{3.176410in}}{\pgfqpoint{1.944758in}{3.176410in}}%
\pgfpathcurveto{\pgfqpoint{1.936522in}{3.176410in}}{\pgfqpoint{1.928622in}{3.173138in}}{\pgfqpoint{1.922798in}{3.167314in}}%
\pgfpathcurveto{\pgfqpoint{1.916974in}{3.161490in}}{\pgfqpoint{1.913702in}{3.153590in}}{\pgfqpoint{1.913702in}{3.145353in}}%
\pgfpathcurveto{\pgfqpoint{1.913702in}{3.137117in}}{\pgfqpoint{1.916974in}{3.129217in}}{\pgfqpoint{1.922798in}{3.123393in}}%
\pgfpathcurveto{\pgfqpoint{1.928622in}{3.117569in}}{\pgfqpoint{1.936522in}{3.114297in}}{\pgfqpoint{1.944758in}{3.114297in}}%
\pgfpathclose%
\pgfusepath{stroke,fill}%
\end{pgfscope}%
\begin{pgfscope}%
\pgfpathrectangle{\pgfqpoint{0.100000in}{0.212622in}}{\pgfqpoint{3.696000in}{3.696000in}}%
\pgfusepath{clip}%
\pgfsetbuttcap%
\pgfsetroundjoin%
\definecolor{currentfill}{rgb}{0.121569,0.466667,0.705882}%
\pgfsetfillcolor{currentfill}%
\pgfsetfillopacity{0.349761}%
\pgfsetlinewidth{1.003750pt}%
\definecolor{currentstroke}{rgb}{0.121569,0.466667,0.705882}%
\pgfsetstrokecolor{currentstroke}%
\pgfsetstrokeopacity{0.349761}%
\pgfsetdash{}{0pt}%
\pgfpathmoveto{\pgfqpoint{1.722273in}{3.092026in}}%
\pgfpathcurveto{\pgfqpoint{1.730510in}{3.092026in}}{\pgfqpoint{1.738410in}{3.095298in}}{\pgfqpoint{1.744234in}{3.101122in}}%
\pgfpathcurveto{\pgfqpoint{1.750057in}{3.106946in}}{\pgfqpoint{1.753330in}{3.114846in}}{\pgfqpoint{1.753330in}{3.123083in}}%
\pgfpathcurveto{\pgfqpoint{1.753330in}{3.131319in}}{\pgfqpoint{1.750057in}{3.139219in}}{\pgfqpoint{1.744234in}{3.145043in}}%
\pgfpathcurveto{\pgfqpoint{1.738410in}{3.150867in}}{\pgfqpoint{1.730510in}{3.154139in}}{\pgfqpoint{1.722273in}{3.154139in}}%
\pgfpathcurveto{\pgfqpoint{1.714037in}{3.154139in}}{\pgfqpoint{1.706137in}{3.150867in}}{\pgfqpoint{1.700313in}{3.145043in}}%
\pgfpathcurveto{\pgfqpoint{1.694489in}{3.139219in}}{\pgfqpoint{1.691217in}{3.131319in}}{\pgfqpoint{1.691217in}{3.123083in}}%
\pgfpathcurveto{\pgfqpoint{1.691217in}{3.114846in}}{\pgfqpoint{1.694489in}{3.106946in}}{\pgfqpoint{1.700313in}{3.101122in}}%
\pgfpathcurveto{\pgfqpoint{1.706137in}{3.095298in}}{\pgfqpoint{1.714037in}{3.092026in}}{\pgfqpoint{1.722273in}{3.092026in}}%
\pgfpathclose%
\pgfusepath{stroke,fill}%
\end{pgfscope}%
\begin{pgfscope}%
\pgfpathrectangle{\pgfqpoint{0.100000in}{0.212622in}}{\pgfqpoint{3.696000in}{3.696000in}}%
\pgfusepath{clip}%
\pgfsetbuttcap%
\pgfsetroundjoin%
\definecolor{currentfill}{rgb}{0.121569,0.466667,0.705882}%
\pgfsetfillcolor{currentfill}%
\pgfsetfillopacity{0.350050}%
\pgfsetlinewidth{1.003750pt}%
\definecolor{currentstroke}{rgb}{0.121569,0.466667,0.705882}%
\pgfsetstrokecolor{currentstroke}%
\pgfsetstrokeopacity{0.350050}%
\pgfsetdash{}{0pt}%
\pgfpathmoveto{\pgfqpoint{1.945670in}{3.109412in}}%
\pgfpathcurveto{\pgfqpoint{1.953906in}{3.109412in}}{\pgfqpoint{1.961806in}{3.112684in}}{\pgfqpoint{1.967630in}{3.118508in}}%
\pgfpathcurveto{\pgfqpoint{1.973454in}{3.124332in}}{\pgfqpoint{1.976727in}{3.132232in}}{\pgfqpoint{1.976727in}{3.140468in}}%
\pgfpathcurveto{\pgfqpoint{1.976727in}{3.148705in}}{\pgfqpoint{1.973454in}{3.156605in}}{\pgfqpoint{1.967630in}{3.162429in}}%
\pgfpathcurveto{\pgfqpoint{1.961806in}{3.168253in}}{\pgfqpoint{1.953906in}{3.171525in}}{\pgfqpoint{1.945670in}{3.171525in}}%
\pgfpathcurveto{\pgfqpoint{1.937434in}{3.171525in}}{\pgfqpoint{1.929534in}{3.168253in}}{\pgfqpoint{1.923710in}{3.162429in}}%
\pgfpathcurveto{\pgfqpoint{1.917886in}{3.156605in}}{\pgfqpoint{1.914614in}{3.148705in}}{\pgfqpoint{1.914614in}{3.140468in}}%
\pgfpathcurveto{\pgfqpoint{1.914614in}{3.132232in}}{\pgfqpoint{1.917886in}{3.124332in}}{\pgfqpoint{1.923710in}{3.118508in}}%
\pgfpathcurveto{\pgfqpoint{1.929534in}{3.112684in}}{\pgfqpoint{1.937434in}{3.109412in}}{\pgfqpoint{1.945670in}{3.109412in}}%
\pgfpathclose%
\pgfusepath{stroke,fill}%
\end{pgfscope}%
\begin{pgfscope}%
\pgfpathrectangle{\pgfqpoint{0.100000in}{0.212622in}}{\pgfqpoint{3.696000in}{3.696000in}}%
\pgfusepath{clip}%
\pgfsetbuttcap%
\pgfsetroundjoin%
\definecolor{currentfill}{rgb}{0.121569,0.466667,0.705882}%
\pgfsetfillcolor{currentfill}%
\pgfsetfillopacity{0.350703}%
\pgfsetlinewidth{1.003750pt}%
\definecolor{currentstroke}{rgb}{0.121569,0.466667,0.705882}%
\pgfsetstrokecolor{currentstroke}%
\pgfsetstrokeopacity{0.350703}%
\pgfsetdash{}{0pt}%
\pgfpathmoveto{\pgfqpoint{1.719911in}{3.086967in}}%
\pgfpathcurveto{\pgfqpoint{1.728147in}{3.086967in}}{\pgfqpoint{1.736047in}{3.090239in}}{\pgfqpoint{1.741871in}{3.096063in}}%
\pgfpathcurveto{\pgfqpoint{1.747695in}{3.101887in}}{\pgfqpoint{1.750967in}{3.109787in}}{\pgfqpoint{1.750967in}{3.118023in}}%
\pgfpathcurveto{\pgfqpoint{1.750967in}{3.126260in}}{\pgfqpoint{1.747695in}{3.134160in}}{\pgfqpoint{1.741871in}{3.139984in}}%
\pgfpathcurveto{\pgfqpoint{1.736047in}{3.145808in}}{\pgfqpoint{1.728147in}{3.149080in}}{\pgfqpoint{1.719911in}{3.149080in}}%
\pgfpathcurveto{\pgfqpoint{1.711674in}{3.149080in}}{\pgfqpoint{1.703774in}{3.145808in}}{\pgfqpoint{1.697950in}{3.139984in}}%
\pgfpathcurveto{\pgfqpoint{1.692126in}{3.134160in}}{\pgfqpoint{1.688854in}{3.126260in}}{\pgfqpoint{1.688854in}{3.118023in}}%
\pgfpathcurveto{\pgfqpoint{1.688854in}{3.109787in}}{\pgfqpoint{1.692126in}{3.101887in}}{\pgfqpoint{1.697950in}{3.096063in}}%
\pgfpathcurveto{\pgfqpoint{1.703774in}{3.090239in}}{\pgfqpoint{1.711674in}{3.086967in}}{\pgfqpoint{1.719911in}{3.086967in}}%
\pgfpathclose%
\pgfusepath{stroke,fill}%
\end{pgfscope}%
\begin{pgfscope}%
\pgfpathrectangle{\pgfqpoint{0.100000in}{0.212622in}}{\pgfqpoint{3.696000in}{3.696000in}}%
\pgfusepath{clip}%
\pgfsetbuttcap%
\pgfsetroundjoin%
\definecolor{currentfill}{rgb}{0.121569,0.466667,0.705882}%
\pgfsetfillcolor{currentfill}%
\pgfsetfillopacity{0.350783}%
\pgfsetlinewidth{1.003750pt}%
\definecolor{currentstroke}{rgb}{0.121569,0.466667,0.705882}%
\pgfsetstrokecolor{currentstroke}%
\pgfsetstrokeopacity{0.350783}%
\pgfsetdash{}{0pt}%
\pgfpathmoveto{\pgfqpoint{1.945948in}{3.106959in}}%
\pgfpathcurveto{\pgfqpoint{1.954185in}{3.106959in}}{\pgfqpoint{1.962085in}{3.110232in}}{\pgfqpoint{1.967909in}{3.116055in}}%
\pgfpathcurveto{\pgfqpoint{1.973733in}{3.121879in}}{\pgfqpoint{1.977005in}{3.129779in}}{\pgfqpoint{1.977005in}{3.138016in}}%
\pgfpathcurveto{\pgfqpoint{1.977005in}{3.146252in}}{\pgfqpoint{1.973733in}{3.154152in}}{\pgfqpoint{1.967909in}{3.159976in}}%
\pgfpathcurveto{\pgfqpoint{1.962085in}{3.165800in}}{\pgfqpoint{1.954185in}{3.169072in}}{\pgfqpoint{1.945948in}{3.169072in}}%
\pgfpathcurveto{\pgfqpoint{1.937712in}{3.169072in}}{\pgfqpoint{1.929812in}{3.165800in}}{\pgfqpoint{1.923988in}{3.159976in}}%
\pgfpathcurveto{\pgfqpoint{1.918164in}{3.154152in}}{\pgfqpoint{1.914892in}{3.146252in}}{\pgfqpoint{1.914892in}{3.138016in}}%
\pgfpathcurveto{\pgfqpoint{1.914892in}{3.129779in}}{\pgfqpoint{1.918164in}{3.121879in}}{\pgfqpoint{1.923988in}{3.116055in}}%
\pgfpathcurveto{\pgfqpoint{1.929812in}{3.110232in}}{\pgfqpoint{1.937712in}{3.106959in}}{\pgfqpoint{1.945948in}{3.106959in}}%
\pgfpathclose%
\pgfusepath{stroke,fill}%
\end{pgfscope}%
\begin{pgfscope}%
\pgfpathrectangle{\pgfqpoint{0.100000in}{0.212622in}}{\pgfqpoint{3.696000in}{3.696000in}}%
\pgfusepath{clip}%
\pgfsetbuttcap%
\pgfsetroundjoin%
\definecolor{currentfill}{rgb}{0.121569,0.466667,0.705882}%
\pgfsetfillcolor{currentfill}%
\pgfsetfillopacity{0.351272}%
\pgfsetlinewidth{1.003750pt}%
\definecolor{currentstroke}{rgb}{0.121569,0.466667,0.705882}%
\pgfsetstrokecolor{currentstroke}%
\pgfsetstrokeopacity{0.351272}%
\pgfsetdash{}{0pt}%
\pgfpathmoveto{\pgfqpoint{1.718095in}{3.084160in}}%
\pgfpathcurveto{\pgfqpoint{1.726331in}{3.084160in}}{\pgfqpoint{1.734231in}{3.087433in}}{\pgfqpoint{1.740055in}{3.093257in}}%
\pgfpathcurveto{\pgfqpoint{1.745879in}{3.099080in}}{\pgfqpoint{1.749152in}{3.106981in}}{\pgfqpoint{1.749152in}{3.115217in}}%
\pgfpathcurveto{\pgfqpoint{1.749152in}{3.123453in}}{\pgfqpoint{1.745879in}{3.131353in}}{\pgfqpoint{1.740055in}{3.137177in}}%
\pgfpathcurveto{\pgfqpoint{1.734231in}{3.143001in}}{\pgfqpoint{1.726331in}{3.146273in}}{\pgfqpoint{1.718095in}{3.146273in}}%
\pgfpathcurveto{\pgfqpoint{1.709859in}{3.146273in}}{\pgfqpoint{1.701959in}{3.143001in}}{\pgfqpoint{1.696135in}{3.137177in}}%
\pgfpathcurveto{\pgfqpoint{1.690311in}{3.131353in}}{\pgfqpoint{1.687039in}{3.123453in}}{\pgfqpoint{1.687039in}{3.115217in}}%
\pgfpathcurveto{\pgfqpoint{1.687039in}{3.106981in}}{\pgfqpoint{1.690311in}{3.099080in}}{\pgfqpoint{1.696135in}{3.093257in}}%
\pgfpathcurveto{\pgfqpoint{1.701959in}{3.087433in}}{\pgfqpoint{1.709859in}{3.084160in}}{\pgfqpoint{1.718095in}{3.084160in}}%
\pgfpathclose%
\pgfusepath{stroke,fill}%
\end{pgfscope}%
\begin{pgfscope}%
\pgfpathrectangle{\pgfqpoint{0.100000in}{0.212622in}}{\pgfqpoint{3.696000in}{3.696000in}}%
\pgfusepath{clip}%
\pgfsetbuttcap%
\pgfsetroundjoin%
\definecolor{currentfill}{rgb}{0.121569,0.466667,0.705882}%
\pgfsetfillcolor{currentfill}%
\pgfsetfillopacity{0.351766}%
\pgfsetlinewidth{1.003750pt}%
\definecolor{currentstroke}{rgb}{0.121569,0.466667,0.705882}%
\pgfsetstrokecolor{currentstroke}%
\pgfsetstrokeopacity{0.351766}%
\pgfsetdash{}{0pt}%
\pgfpathmoveto{\pgfqpoint{1.946729in}{3.103163in}}%
\pgfpathcurveto{\pgfqpoint{1.954965in}{3.103163in}}{\pgfqpoint{1.962865in}{3.106436in}}{\pgfqpoint{1.968689in}{3.112260in}}%
\pgfpathcurveto{\pgfqpoint{1.974513in}{3.118084in}}{\pgfqpoint{1.977785in}{3.125984in}}{\pgfqpoint{1.977785in}{3.134220in}}%
\pgfpathcurveto{\pgfqpoint{1.977785in}{3.142456in}}{\pgfqpoint{1.974513in}{3.150356in}}{\pgfqpoint{1.968689in}{3.156180in}}%
\pgfpathcurveto{\pgfqpoint{1.962865in}{3.162004in}}{\pgfqpoint{1.954965in}{3.165276in}}{\pgfqpoint{1.946729in}{3.165276in}}%
\pgfpathcurveto{\pgfqpoint{1.938492in}{3.165276in}}{\pgfqpoint{1.930592in}{3.162004in}}{\pgfqpoint{1.924768in}{3.156180in}}%
\pgfpathcurveto{\pgfqpoint{1.918945in}{3.150356in}}{\pgfqpoint{1.915672in}{3.142456in}}{\pgfqpoint{1.915672in}{3.134220in}}%
\pgfpathcurveto{\pgfqpoint{1.915672in}{3.125984in}}{\pgfqpoint{1.918945in}{3.118084in}}{\pgfqpoint{1.924768in}{3.112260in}}%
\pgfpathcurveto{\pgfqpoint{1.930592in}{3.106436in}}{\pgfqpoint{1.938492in}{3.103163in}}{\pgfqpoint{1.946729in}{3.103163in}}%
\pgfpathclose%
\pgfusepath{stroke,fill}%
\end{pgfscope}%
\begin{pgfscope}%
\pgfpathrectangle{\pgfqpoint{0.100000in}{0.212622in}}{\pgfqpoint{3.696000in}{3.696000in}}%
\pgfusepath{clip}%
\pgfsetbuttcap%
\pgfsetroundjoin%
\definecolor{currentfill}{rgb}{0.121569,0.466667,0.705882}%
\pgfsetfillcolor{currentfill}%
\pgfsetfillopacity{0.352339}%
\pgfsetlinewidth{1.003750pt}%
\definecolor{currentstroke}{rgb}{0.121569,0.466667,0.705882}%
\pgfsetstrokecolor{currentstroke}%
\pgfsetstrokeopacity{0.352339}%
\pgfsetdash{}{0pt}%
\pgfpathmoveto{\pgfqpoint{1.715028in}{3.078846in}}%
\pgfpathcurveto{\pgfqpoint{1.723264in}{3.078846in}}{\pgfqpoint{1.731164in}{3.082118in}}{\pgfqpoint{1.736988in}{3.087942in}}%
\pgfpathcurveto{\pgfqpoint{1.742812in}{3.093766in}}{\pgfqpoint{1.746085in}{3.101666in}}{\pgfqpoint{1.746085in}{3.109903in}}%
\pgfpathcurveto{\pgfqpoint{1.746085in}{3.118139in}}{\pgfqpoint{1.742812in}{3.126039in}}{\pgfqpoint{1.736988in}{3.131863in}}%
\pgfpathcurveto{\pgfqpoint{1.731164in}{3.137687in}}{\pgfqpoint{1.723264in}{3.140959in}}{\pgfqpoint{1.715028in}{3.140959in}}%
\pgfpathcurveto{\pgfqpoint{1.706792in}{3.140959in}}{\pgfqpoint{1.698892in}{3.137687in}}{\pgfqpoint{1.693068in}{3.131863in}}%
\pgfpathcurveto{\pgfqpoint{1.687244in}{3.126039in}}{\pgfqpoint{1.683972in}{3.118139in}}{\pgfqpoint{1.683972in}{3.109903in}}%
\pgfpathcurveto{\pgfqpoint{1.683972in}{3.101666in}}{\pgfqpoint{1.687244in}{3.093766in}}{\pgfqpoint{1.693068in}{3.087942in}}%
\pgfpathcurveto{\pgfqpoint{1.698892in}{3.082118in}}{\pgfqpoint{1.706792in}{3.078846in}}{\pgfqpoint{1.715028in}{3.078846in}}%
\pgfpathclose%
\pgfusepath{stroke,fill}%
\end{pgfscope}%
\begin{pgfscope}%
\pgfpathrectangle{\pgfqpoint{0.100000in}{0.212622in}}{\pgfqpoint{3.696000in}{3.696000in}}%
\pgfusepath{clip}%
\pgfsetbuttcap%
\pgfsetroundjoin%
\definecolor{currentfill}{rgb}{0.121569,0.466667,0.705882}%
\pgfsetfillcolor{currentfill}%
\pgfsetfillopacity{0.352905}%
\pgfsetlinewidth{1.003750pt}%
\definecolor{currentstroke}{rgb}{0.121569,0.466667,0.705882}%
\pgfsetstrokecolor{currentstroke}%
\pgfsetstrokeopacity{0.352905}%
\pgfsetdash{}{0pt}%
\pgfpathmoveto{\pgfqpoint{1.947534in}{3.098568in}}%
\pgfpathcurveto{\pgfqpoint{1.955771in}{3.098568in}}{\pgfqpoint{1.963671in}{3.101840in}}{\pgfqpoint{1.969495in}{3.107664in}}%
\pgfpathcurveto{\pgfqpoint{1.975319in}{3.113488in}}{\pgfqpoint{1.978591in}{3.121388in}}{\pgfqpoint{1.978591in}{3.129624in}}%
\pgfpathcurveto{\pgfqpoint{1.978591in}{3.137861in}}{\pgfqpoint{1.975319in}{3.145761in}}{\pgfqpoint{1.969495in}{3.151585in}}%
\pgfpathcurveto{\pgfqpoint{1.963671in}{3.157409in}}{\pgfqpoint{1.955771in}{3.160681in}}{\pgfqpoint{1.947534in}{3.160681in}}%
\pgfpathcurveto{\pgfqpoint{1.939298in}{3.160681in}}{\pgfqpoint{1.931398in}{3.157409in}}{\pgfqpoint{1.925574in}{3.151585in}}%
\pgfpathcurveto{\pgfqpoint{1.919750in}{3.145761in}}{\pgfqpoint{1.916478in}{3.137861in}}{\pgfqpoint{1.916478in}{3.129624in}}%
\pgfpathcurveto{\pgfqpoint{1.916478in}{3.121388in}}{\pgfqpoint{1.919750in}{3.113488in}}{\pgfqpoint{1.925574in}{3.107664in}}%
\pgfpathcurveto{\pgfqpoint{1.931398in}{3.101840in}}{\pgfqpoint{1.939298in}{3.098568in}}{\pgfqpoint{1.947534in}{3.098568in}}%
\pgfpathclose%
\pgfusepath{stroke,fill}%
\end{pgfscope}%
\begin{pgfscope}%
\pgfpathrectangle{\pgfqpoint{0.100000in}{0.212622in}}{\pgfqpoint{3.696000in}{3.696000in}}%
\pgfusepath{clip}%
\pgfsetbuttcap%
\pgfsetroundjoin%
\definecolor{currentfill}{rgb}{0.121569,0.466667,0.705882}%
\pgfsetfillcolor{currentfill}%
\pgfsetfillopacity{0.353328}%
\pgfsetlinewidth{1.003750pt}%
\definecolor{currentstroke}{rgb}{0.121569,0.466667,0.705882}%
\pgfsetstrokecolor{currentstroke}%
\pgfsetstrokeopacity{0.353328}%
\pgfsetdash{}{0pt}%
\pgfpathmoveto{\pgfqpoint{1.712312in}{3.073849in}}%
\pgfpathcurveto{\pgfqpoint{1.720548in}{3.073849in}}{\pgfqpoint{1.728448in}{3.077121in}}{\pgfqpoint{1.734272in}{3.082945in}}%
\pgfpathcurveto{\pgfqpoint{1.740096in}{3.088769in}}{\pgfqpoint{1.743369in}{3.096669in}}{\pgfqpoint{1.743369in}{3.104905in}}%
\pgfpathcurveto{\pgfqpoint{1.743369in}{3.113142in}}{\pgfqpoint{1.740096in}{3.121042in}}{\pgfqpoint{1.734272in}{3.126866in}}%
\pgfpathcurveto{\pgfqpoint{1.728448in}{3.132690in}}{\pgfqpoint{1.720548in}{3.135962in}}{\pgfqpoint{1.712312in}{3.135962in}}%
\pgfpathcurveto{\pgfqpoint{1.704076in}{3.135962in}}{\pgfqpoint{1.696176in}{3.132690in}}{\pgfqpoint{1.690352in}{3.126866in}}%
\pgfpathcurveto{\pgfqpoint{1.684528in}{3.121042in}}{\pgfqpoint{1.681256in}{3.113142in}}{\pgfqpoint{1.681256in}{3.104905in}}%
\pgfpathcurveto{\pgfqpoint{1.681256in}{3.096669in}}{\pgfqpoint{1.684528in}{3.088769in}}{\pgfqpoint{1.690352in}{3.082945in}}%
\pgfpathcurveto{\pgfqpoint{1.696176in}{3.077121in}}{\pgfqpoint{1.704076in}{3.073849in}}{\pgfqpoint{1.712312in}{3.073849in}}%
\pgfpathclose%
\pgfusepath{stroke,fill}%
\end{pgfscope}%
\begin{pgfscope}%
\pgfpathrectangle{\pgfqpoint{0.100000in}{0.212622in}}{\pgfqpoint{3.696000in}{3.696000in}}%
\pgfusepath{clip}%
\pgfsetbuttcap%
\pgfsetroundjoin%
\definecolor{currentfill}{rgb}{0.121569,0.466667,0.705882}%
\pgfsetfillcolor{currentfill}%
\pgfsetfillopacity{0.354029}%
\pgfsetlinewidth{1.003750pt}%
\definecolor{currentstroke}{rgb}{0.121569,0.466667,0.705882}%
\pgfsetstrokecolor{currentstroke}%
\pgfsetstrokeopacity{0.354029}%
\pgfsetdash{}{0pt}%
\pgfpathmoveto{\pgfqpoint{1.709788in}{3.069952in}}%
\pgfpathcurveto{\pgfqpoint{1.718024in}{3.069952in}}{\pgfqpoint{1.725925in}{3.073225in}}{\pgfqpoint{1.731748in}{3.079049in}}%
\pgfpathcurveto{\pgfqpoint{1.737572in}{3.084873in}}{\pgfqpoint{1.740845in}{3.092773in}}{\pgfqpoint{1.740845in}{3.101009in}}%
\pgfpathcurveto{\pgfqpoint{1.740845in}{3.109245in}}{\pgfqpoint{1.737572in}{3.117145in}}{\pgfqpoint{1.731748in}{3.122969in}}%
\pgfpathcurveto{\pgfqpoint{1.725925in}{3.128793in}}{\pgfqpoint{1.718024in}{3.132065in}}{\pgfqpoint{1.709788in}{3.132065in}}%
\pgfpathcurveto{\pgfqpoint{1.701552in}{3.132065in}}{\pgfqpoint{1.693652in}{3.128793in}}{\pgfqpoint{1.687828in}{3.122969in}}%
\pgfpathcurveto{\pgfqpoint{1.682004in}{3.117145in}}{\pgfqpoint{1.678732in}{3.109245in}}{\pgfqpoint{1.678732in}{3.101009in}}%
\pgfpathcurveto{\pgfqpoint{1.678732in}{3.092773in}}{\pgfqpoint{1.682004in}{3.084873in}}{\pgfqpoint{1.687828in}{3.079049in}}%
\pgfpathcurveto{\pgfqpoint{1.693652in}{3.073225in}}{\pgfqpoint{1.701552in}{3.069952in}}{\pgfqpoint{1.709788in}{3.069952in}}%
\pgfpathclose%
\pgfusepath{stroke,fill}%
\end{pgfscope}%
\begin{pgfscope}%
\pgfpathrectangle{\pgfqpoint{0.100000in}{0.212622in}}{\pgfqpoint{3.696000in}{3.696000in}}%
\pgfusepath{clip}%
\pgfsetbuttcap%
\pgfsetroundjoin%
\definecolor{currentfill}{rgb}{0.121569,0.466667,0.705882}%
\pgfsetfillcolor{currentfill}%
\pgfsetfillopacity{0.354338}%
\pgfsetlinewidth{1.003750pt}%
\definecolor{currentstroke}{rgb}{0.121569,0.466667,0.705882}%
\pgfsetstrokecolor{currentstroke}%
\pgfsetstrokeopacity{0.354338}%
\pgfsetdash{}{0pt}%
\pgfpathmoveto{\pgfqpoint{1.948215in}{3.093557in}}%
\pgfpathcurveto{\pgfqpoint{1.956451in}{3.093557in}}{\pgfqpoint{1.964351in}{3.096830in}}{\pgfqpoint{1.970175in}{3.102654in}}%
\pgfpathcurveto{\pgfqpoint{1.975999in}{3.108478in}}{\pgfqpoint{1.979271in}{3.116378in}}{\pgfqpoint{1.979271in}{3.124614in}}%
\pgfpathcurveto{\pgfqpoint{1.979271in}{3.132850in}}{\pgfqpoint{1.975999in}{3.140750in}}{\pgfqpoint{1.970175in}{3.146574in}}%
\pgfpathcurveto{\pgfqpoint{1.964351in}{3.152398in}}{\pgfqpoint{1.956451in}{3.155670in}}{\pgfqpoint{1.948215in}{3.155670in}}%
\pgfpathcurveto{\pgfqpoint{1.939978in}{3.155670in}}{\pgfqpoint{1.932078in}{3.152398in}}{\pgfqpoint{1.926254in}{3.146574in}}%
\pgfpathcurveto{\pgfqpoint{1.920430in}{3.140750in}}{\pgfqpoint{1.917158in}{3.132850in}}{\pgfqpoint{1.917158in}{3.124614in}}%
\pgfpathcurveto{\pgfqpoint{1.917158in}{3.116378in}}{\pgfqpoint{1.920430in}{3.108478in}}{\pgfqpoint{1.926254in}{3.102654in}}%
\pgfpathcurveto{\pgfqpoint{1.932078in}{3.096830in}}{\pgfqpoint{1.939978in}{3.093557in}}{\pgfqpoint{1.948215in}{3.093557in}}%
\pgfpathclose%
\pgfusepath{stroke,fill}%
\end{pgfscope}%
\begin{pgfscope}%
\pgfpathrectangle{\pgfqpoint{0.100000in}{0.212622in}}{\pgfqpoint{3.696000in}{3.696000in}}%
\pgfusepath{clip}%
\pgfsetbuttcap%
\pgfsetroundjoin%
\definecolor{currentfill}{rgb}{0.121569,0.466667,0.705882}%
\pgfsetfillcolor{currentfill}%
\pgfsetfillopacity{0.354526}%
\pgfsetlinewidth{1.003750pt}%
\definecolor{currentstroke}{rgb}{0.121569,0.466667,0.705882}%
\pgfsetstrokecolor{currentstroke}%
\pgfsetstrokeopacity{0.354526}%
\pgfsetdash{}{0pt}%
\pgfpathmoveto{\pgfqpoint{1.708518in}{3.067389in}}%
\pgfpathcurveto{\pgfqpoint{1.716755in}{3.067389in}}{\pgfqpoint{1.724655in}{3.070661in}}{\pgfqpoint{1.730479in}{3.076485in}}%
\pgfpathcurveto{\pgfqpoint{1.736302in}{3.082309in}}{\pgfqpoint{1.739575in}{3.090209in}}{\pgfqpoint{1.739575in}{3.098445in}}%
\pgfpathcurveto{\pgfqpoint{1.739575in}{3.106681in}}{\pgfqpoint{1.736302in}{3.114581in}}{\pgfqpoint{1.730479in}{3.120405in}}%
\pgfpathcurveto{\pgfqpoint{1.724655in}{3.126229in}}{\pgfqpoint{1.716755in}{3.129502in}}{\pgfqpoint{1.708518in}{3.129502in}}%
\pgfpathcurveto{\pgfqpoint{1.700282in}{3.129502in}}{\pgfqpoint{1.692382in}{3.126229in}}{\pgfqpoint{1.686558in}{3.120405in}}%
\pgfpathcurveto{\pgfqpoint{1.680734in}{3.114581in}}{\pgfqpoint{1.677462in}{3.106681in}}{\pgfqpoint{1.677462in}{3.098445in}}%
\pgfpathcurveto{\pgfqpoint{1.677462in}{3.090209in}}{\pgfqpoint{1.680734in}{3.082309in}}{\pgfqpoint{1.686558in}{3.076485in}}%
\pgfpathcurveto{\pgfqpoint{1.692382in}{3.070661in}}{\pgfqpoint{1.700282in}{3.067389in}}{\pgfqpoint{1.708518in}{3.067389in}}%
\pgfpathclose%
\pgfusepath{stroke,fill}%
\end{pgfscope}%
\begin{pgfscope}%
\pgfpathrectangle{\pgfqpoint{0.100000in}{0.212622in}}{\pgfqpoint{3.696000in}{3.696000in}}%
\pgfusepath{clip}%
\pgfsetbuttcap%
\pgfsetroundjoin%
\definecolor{currentfill}{rgb}{0.121569,0.466667,0.705882}%
\pgfsetfillcolor{currentfill}%
\pgfsetfillopacity{0.354939}%
\pgfsetlinewidth{1.003750pt}%
\definecolor{currentstroke}{rgb}{0.121569,0.466667,0.705882}%
\pgfsetstrokecolor{currentstroke}%
\pgfsetstrokeopacity{0.354939}%
\pgfsetdash{}{0pt}%
\pgfpathmoveto{\pgfqpoint{1.707321in}{3.065403in}}%
\pgfpathcurveto{\pgfqpoint{1.715557in}{3.065403in}}{\pgfqpoint{1.723457in}{3.068675in}}{\pgfqpoint{1.729281in}{3.074499in}}%
\pgfpathcurveto{\pgfqpoint{1.735105in}{3.080323in}}{\pgfqpoint{1.738377in}{3.088223in}}{\pgfqpoint{1.738377in}{3.096459in}}%
\pgfpathcurveto{\pgfqpoint{1.738377in}{3.104696in}}{\pgfqpoint{1.735105in}{3.112596in}}{\pgfqpoint{1.729281in}{3.118420in}}%
\pgfpathcurveto{\pgfqpoint{1.723457in}{3.124244in}}{\pgfqpoint{1.715557in}{3.127516in}}{\pgfqpoint{1.707321in}{3.127516in}}%
\pgfpathcurveto{\pgfqpoint{1.699084in}{3.127516in}}{\pgfqpoint{1.691184in}{3.124244in}}{\pgfqpoint{1.685360in}{3.118420in}}%
\pgfpathcurveto{\pgfqpoint{1.679536in}{3.112596in}}{\pgfqpoint{1.676264in}{3.104696in}}{\pgfqpoint{1.676264in}{3.096459in}}%
\pgfpathcurveto{\pgfqpoint{1.676264in}{3.088223in}}{\pgfqpoint{1.679536in}{3.080323in}}{\pgfqpoint{1.685360in}{3.074499in}}%
\pgfpathcurveto{\pgfqpoint{1.691184in}{3.068675in}}{\pgfqpoint{1.699084in}{3.065403in}}{\pgfqpoint{1.707321in}{3.065403in}}%
\pgfpathclose%
\pgfusepath{stroke,fill}%
\end{pgfscope}%
\begin{pgfscope}%
\pgfpathrectangle{\pgfqpoint{0.100000in}{0.212622in}}{\pgfqpoint{3.696000in}{3.696000in}}%
\pgfusepath{clip}%
\pgfsetbuttcap%
\pgfsetroundjoin%
\definecolor{currentfill}{rgb}{0.121569,0.466667,0.705882}%
\pgfsetfillcolor{currentfill}%
\pgfsetfillopacity{0.355637}%
\pgfsetlinewidth{1.003750pt}%
\definecolor{currentstroke}{rgb}{0.121569,0.466667,0.705882}%
\pgfsetstrokecolor{currentstroke}%
\pgfsetstrokeopacity{0.355637}%
\pgfsetdash{}{0pt}%
\pgfpathmoveto{\pgfqpoint{1.705037in}{3.061743in}}%
\pgfpathcurveto{\pgfqpoint{1.713274in}{3.061743in}}{\pgfqpoint{1.721174in}{3.065016in}}{\pgfqpoint{1.726998in}{3.070839in}}%
\pgfpathcurveto{\pgfqpoint{1.732822in}{3.076663in}}{\pgfqpoint{1.736094in}{3.084563in}}{\pgfqpoint{1.736094in}{3.092800in}}%
\pgfpathcurveto{\pgfqpoint{1.736094in}{3.101036in}}{\pgfqpoint{1.732822in}{3.108936in}}{\pgfqpoint{1.726998in}{3.114760in}}%
\pgfpathcurveto{\pgfqpoint{1.721174in}{3.120584in}}{\pgfqpoint{1.713274in}{3.123856in}}{\pgfqpoint{1.705037in}{3.123856in}}%
\pgfpathcurveto{\pgfqpoint{1.696801in}{3.123856in}}{\pgfqpoint{1.688901in}{3.120584in}}{\pgfqpoint{1.683077in}{3.114760in}}%
\pgfpathcurveto{\pgfqpoint{1.677253in}{3.108936in}}{\pgfqpoint{1.673981in}{3.101036in}}{\pgfqpoint{1.673981in}{3.092800in}}%
\pgfpathcurveto{\pgfqpoint{1.673981in}{3.084563in}}{\pgfqpoint{1.677253in}{3.076663in}}{\pgfqpoint{1.683077in}{3.070839in}}%
\pgfpathcurveto{\pgfqpoint{1.688901in}{3.065016in}}{\pgfqpoint{1.696801in}{3.061743in}}{\pgfqpoint{1.705037in}{3.061743in}}%
\pgfpathclose%
\pgfusepath{stroke,fill}%
\end{pgfscope}%
\begin{pgfscope}%
\pgfpathrectangle{\pgfqpoint{0.100000in}{0.212622in}}{\pgfqpoint{3.696000in}{3.696000in}}%
\pgfusepath{clip}%
\pgfsetbuttcap%
\pgfsetroundjoin%
\definecolor{currentfill}{rgb}{0.121569,0.466667,0.705882}%
\pgfsetfillcolor{currentfill}%
\pgfsetfillopacity{0.355714}%
\pgfsetlinewidth{1.003750pt}%
\definecolor{currentstroke}{rgb}{0.121569,0.466667,0.705882}%
\pgfsetstrokecolor{currentstroke}%
\pgfsetstrokeopacity{0.355714}%
\pgfsetdash{}{0pt}%
\pgfpathmoveto{\pgfqpoint{1.949352in}{3.087718in}}%
\pgfpathcurveto{\pgfqpoint{1.957588in}{3.087718in}}{\pgfqpoint{1.965488in}{3.090990in}}{\pgfqpoint{1.971312in}{3.096814in}}%
\pgfpathcurveto{\pgfqpoint{1.977136in}{3.102638in}}{\pgfqpoint{1.980409in}{3.110538in}}{\pgfqpoint{1.980409in}{3.118774in}}%
\pgfpathcurveto{\pgfqpoint{1.980409in}{3.127010in}}{\pgfqpoint{1.977136in}{3.134911in}}{\pgfqpoint{1.971312in}{3.140734in}}%
\pgfpathcurveto{\pgfqpoint{1.965488in}{3.146558in}}{\pgfqpoint{1.957588in}{3.149831in}}{\pgfqpoint{1.949352in}{3.149831in}}%
\pgfpathcurveto{\pgfqpoint{1.941116in}{3.149831in}}{\pgfqpoint{1.933216in}{3.146558in}}{\pgfqpoint{1.927392in}{3.140734in}}%
\pgfpathcurveto{\pgfqpoint{1.921568in}{3.134911in}}{\pgfqpoint{1.918296in}{3.127010in}}{\pgfqpoint{1.918296in}{3.118774in}}%
\pgfpathcurveto{\pgfqpoint{1.918296in}{3.110538in}}{\pgfqpoint{1.921568in}{3.102638in}}{\pgfqpoint{1.927392in}{3.096814in}}%
\pgfpathcurveto{\pgfqpoint{1.933216in}{3.090990in}}{\pgfqpoint{1.941116in}{3.087718in}}{\pgfqpoint{1.949352in}{3.087718in}}%
\pgfpathclose%
\pgfusepath{stroke,fill}%
\end{pgfscope}%
\begin{pgfscope}%
\pgfpathrectangle{\pgfqpoint{0.100000in}{0.212622in}}{\pgfqpoint{3.696000in}{3.696000in}}%
\pgfusepath{clip}%
\pgfsetbuttcap%
\pgfsetroundjoin%
\definecolor{currentfill}{rgb}{0.121569,0.466667,0.705882}%
\pgfsetfillcolor{currentfill}%
\pgfsetfillopacity{0.356291}%
\pgfsetlinewidth{1.003750pt}%
\definecolor{currentstroke}{rgb}{0.121569,0.466667,0.705882}%
\pgfsetstrokecolor{currentstroke}%
\pgfsetstrokeopacity{0.356291}%
\pgfsetdash{}{0pt}%
\pgfpathmoveto{\pgfqpoint{1.703391in}{3.058330in}}%
\pgfpathcurveto{\pgfqpoint{1.711627in}{3.058330in}}{\pgfqpoint{1.719528in}{3.061602in}}{\pgfqpoint{1.725351in}{3.067426in}}%
\pgfpathcurveto{\pgfqpoint{1.731175in}{3.073250in}}{\pgfqpoint{1.734448in}{3.081150in}}{\pgfqpoint{1.734448in}{3.089387in}}%
\pgfpathcurveto{\pgfqpoint{1.734448in}{3.097623in}}{\pgfqpoint{1.731175in}{3.105523in}}{\pgfqpoint{1.725351in}{3.111347in}}%
\pgfpathcurveto{\pgfqpoint{1.719528in}{3.117171in}}{\pgfqpoint{1.711627in}{3.120443in}}{\pgfqpoint{1.703391in}{3.120443in}}%
\pgfpathcurveto{\pgfqpoint{1.695155in}{3.120443in}}{\pgfqpoint{1.687255in}{3.117171in}}{\pgfqpoint{1.681431in}{3.111347in}}%
\pgfpathcurveto{\pgfqpoint{1.675607in}{3.105523in}}{\pgfqpoint{1.672335in}{3.097623in}}{\pgfqpoint{1.672335in}{3.089387in}}%
\pgfpathcurveto{\pgfqpoint{1.672335in}{3.081150in}}{\pgfqpoint{1.675607in}{3.073250in}}{\pgfqpoint{1.681431in}{3.067426in}}%
\pgfpathcurveto{\pgfqpoint{1.687255in}{3.061602in}}{\pgfqpoint{1.695155in}{3.058330in}}{\pgfqpoint{1.703391in}{3.058330in}}%
\pgfpathclose%
\pgfusepath{stroke,fill}%
\end{pgfscope}%
\begin{pgfscope}%
\pgfpathrectangle{\pgfqpoint{0.100000in}{0.212622in}}{\pgfqpoint{3.696000in}{3.696000in}}%
\pgfusepath{clip}%
\pgfsetbuttcap%
\pgfsetroundjoin%
\definecolor{currentfill}{rgb}{0.121569,0.466667,0.705882}%
\pgfsetfillcolor{currentfill}%
\pgfsetfillopacity{0.356428}%
\pgfsetlinewidth{1.003750pt}%
\definecolor{currentstroke}{rgb}{0.121569,0.466667,0.705882}%
\pgfsetstrokecolor{currentstroke}%
\pgfsetstrokeopacity{0.356428}%
\pgfsetdash{}{0pt}%
\pgfpathmoveto{\pgfqpoint{1.702959in}{3.057675in}}%
\pgfpathcurveto{\pgfqpoint{1.711195in}{3.057675in}}{\pgfqpoint{1.719095in}{3.060947in}}{\pgfqpoint{1.724919in}{3.066771in}}%
\pgfpathcurveto{\pgfqpoint{1.730743in}{3.072595in}}{\pgfqpoint{1.734016in}{3.080495in}}{\pgfqpoint{1.734016in}{3.088732in}}%
\pgfpathcurveto{\pgfqpoint{1.734016in}{3.096968in}}{\pgfqpoint{1.730743in}{3.104868in}}{\pgfqpoint{1.724919in}{3.110692in}}%
\pgfpathcurveto{\pgfqpoint{1.719095in}{3.116516in}}{\pgfqpoint{1.711195in}{3.119788in}}{\pgfqpoint{1.702959in}{3.119788in}}%
\pgfpathcurveto{\pgfqpoint{1.694723in}{3.119788in}}{\pgfqpoint{1.686823in}{3.116516in}}{\pgfqpoint{1.680999in}{3.110692in}}%
\pgfpathcurveto{\pgfqpoint{1.675175in}{3.104868in}}{\pgfqpoint{1.671903in}{3.096968in}}{\pgfqpoint{1.671903in}{3.088732in}}%
\pgfpathcurveto{\pgfqpoint{1.671903in}{3.080495in}}{\pgfqpoint{1.675175in}{3.072595in}}{\pgfqpoint{1.680999in}{3.066771in}}%
\pgfpathcurveto{\pgfqpoint{1.686823in}{3.060947in}}{\pgfqpoint{1.694723in}{3.057675in}}{\pgfqpoint{1.702959in}{3.057675in}}%
\pgfpathclose%
\pgfusepath{stroke,fill}%
\end{pgfscope}%
\begin{pgfscope}%
\pgfpathrectangle{\pgfqpoint{0.100000in}{0.212622in}}{\pgfqpoint{3.696000in}{3.696000in}}%
\pgfusepath{clip}%
\pgfsetbuttcap%
\pgfsetroundjoin%
\definecolor{currentfill}{rgb}{0.121569,0.466667,0.705882}%
\pgfsetfillcolor{currentfill}%
\pgfsetfillopacity{0.356697}%
\pgfsetlinewidth{1.003750pt}%
\definecolor{currentstroke}{rgb}{0.121569,0.466667,0.705882}%
\pgfsetstrokecolor{currentstroke}%
\pgfsetstrokeopacity{0.356697}%
\pgfsetdash{}{0pt}%
\pgfpathmoveto{\pgfqpoint{1.702249in}{3.056455in}}%
\pgfpathcurveto{\pgfqpoint{1.710485in}{3.056455in}}{\pgfqpoint{1.718385in}{3.059727in}}{\pgfqpoint{1.724209in}{3.065551in}}%
\pgfpathcurveto{\pgfqpoint{1.730033in}{3.071375in}}{\pgfqpoint{1.733306in}{3.079275in}}{\pgfqpoint{1.733306in}{3.087512in}}%
\pgfpathcurveto{\pgfqpoint{1.733306in}{3.095748in}}{\pgfqpoint{1.730033in}{3.103648in}}{\pgfqpoint{1.724209in}{3.109472in}}%
\pgfpathcurveto{\pgfqpoint{1.718385in}{3.115296in}}{\pgfqpoint{1.710485in}{3.118568in}}{\pgfqpoint{1.702249in}{3.118568in}}%
\pgfpathcurveto{\pgfqpoint{1.694013in}{3.118568in}}{\pgfqpoint{1.686113in}{3.115296in}}{\pgfqpoint{1.680289in}{3.109472in}}%
\pgfpathcurveto{\pgfqpoint{1.674465in}{3.103648in}}{\pgfqpoint{1.671193in}{3.095748in}}{\pgfqpoint{1.671193in}{3.087512in}}%
\pgfpathcurveto{\pgfqpoint{1.671193in}{3.079275in}}{\pgfqpoint{1.674465in}{3.071375in}}{\pgfqpoint{1.680289in}{3.065551in}}%
\pgfpathcurveto{\pgfqpoint{1.686113in}{3.059727in}}{\pgfqpoint{1.694013in}{3.056455in}}{\pgfqpoint{1.702249in}{3.056455in}}%
\pgfpathclose%
\pgfusepath{stroke,fill}%
\end{pgfscope}%
\begin{pgfscope}%
\pgfpathrectangle{\pgfqpoint{0.100000in}{0.212622in}}{\pgfqpoint{3.696000in}{3.696000in}}%
\pgfusepath{clip}%
\pgfsetbuttcap%
\pgfsetroundjoin%
\definecolor{currentfill}{rgb}{0.121569,0.466667,0.705882}%
\pgfsetfillcolor{currentfill}%
\pgfsetfillopacity{0.357173}%
\pgfsetlinewidth{1.003750pt}%
\definecolor{currentstroke}{rgb}{0.121569,0.466667,0.705882}%
\pgfsetstrokecolor{currentstroke}%
\pgfsetstrokeopacity{0.357173}%
\pgfsetdash{}{0pt}%
\pgfpathmoveto{\pgfqpoint{1.700950in}{3.054196in}}%
\pgfpathcurveto{\pgfqpoint{1.709186in}{3.054196in}}{\pgfqpoint{1.717086in}{3.057469in}}{\pgfqpoint{1.722910in}{3.063293in}}%
\pgfpathcurveto{\pgfqpoint{1.728734in}{3.069116in}}{\pgfqpoint{1.732006in}{3.077017in}}{\pgfqpoint{1.732006in}{3.085253in}}%
\pgfpathcurveto{\pgfqpoint{1.732006in}{3.093489in}}{\pgfqpoint{1.728734in}{3.101389in}}{\pgfqpoint{1.722910in}{3.107213in}}%
\pgfpathcurveto{\pgfqpoint{1.717086in}{3.113037in}}{\pgfqpoint{1.709186in}{3.116309in}}{\pgfqpoint{1.700950in}{3.116309in}}%
\pgfpathcurveto{\pgfqpoint{1.692713in}{3.116309in}}{\pgfqpoint{1.684813in}{3.113037in}}{\pgfqpoint{1.678989in}{3.107213in}}%
\pgfpathcurveto{\pgfqpoint{1.673165in}{3.101389in}}{\pgfqpoint{1.669893in}{3.093489in}}{\pgfqpoint{1.669893in}{3.085253in}}%
\pgfpathcurveto{\pgfqpoint{1.669893in}{3.077017in}}{\pgfqpoint{1.673165in}{3.069116in}}{\pgfqpoint{1.678989in}{3.063293in}}%
\pgfpathcurveto{\pgfqpoint{1.684813in}{3.057469in}}{\pgfqpoint{1.692713in}{3.054196in}}{\pgfqpoint{1.700950in}{3.054196in}}%
\pgfpathclose%
\pgfusepath{stroke,fill}%
\end{pgfscope}%
\begin{pgfscope}%
\pgfpathrectangle{\pgfqpoint{0.100000in}{0.212622in}}{\pgfqpoint{3.696000in}{3.696000in}}%
\pgfusepath{clip}%
\pgfsetbuttcap%
\pgfsetroundjoin%
\definecolor{currentfill}{rgb}{0.121569,0.466667,0.705882}%
\pgfsetfillcolor{currentfill}%
\pgfsetfillopacity{0.357622}%
\pgfsetlinewidth{1.003750pt}%
\definecolor{currentstroke}{rgb}{0.121569,0.466667,0.705882}%
\pgfsetstrokecolor{currentstroke}%
\pgfsetstrokeopacity{0.357622}%
\pgfsetdash{}{0pt}%
\pgfpathmoveto{\pgfqpoint{1.950360in}{3.080402in}}%
\pgfpathcurveto{\pgfqpoint{1.958596in}{3.080402in}}{\pgfqpoint{1.966497in}{3.083675in}}{\pgfqpoint{1.972320in}{3.089498in}}%
\pgfpathcurveto{\pgfqpoint{1.978144in}{3.095322in}}{\pgfqpoint{1.981417in}{3.103222in}}{\pgfqpoint{1.981417in}{3.111459in}}%
\pgfpathcurveto{\pgfqpoint{1.981417in}{3.119695in}}{\pgfqpoint{1.978144in}{3.127595in}}{\pgfqpoint{1.972320in}{3.133419in}}%
\pgfpathcurveto{\pgfqpoint{1.966497in}{3.139243in}}{\pgfqpoint{1.958596in}{3.142515in}}{\pgfqpoint{1.950360in}{3.142515in}}%
\pgfpathcurveto{\pgfqpoint{1.942124in}{3.142515in}}{\pgfqpoint{1.934224in}{3.139243in}}{\pgfqpoint{1.928400in}{3.133419in}}%
\pgfpathcurveto{\pgfqpoint{1.922576in}{3.127595in}}{\pgfqpoint{1.919304in}{3.119695in}}{\pgfqpoint{1.919304in}{3.111459in}}%
\pgfpathcurveto{\pgfqpoint{1.919304in}{3.103222in}}{\pgfqpoint{1.922576in}{3.095322in}}{\pgfqpoint{1.928400in}{3.089498in}}%
\pgfpathcurveto{\pgfqpoint{1.934224in}{3.083675in}}{\pgfqpoint{1.942124in}{3.080402in}}{\pgfqpoint{1.950360in}{3.080402in}}%
\pgfpathclose%
\pgfusepath{stroke,fill}%
\end{pgfscope}%
\begin{pgfscope}%
\pgfpathrectangle{\pgfqpoint{0.100000in}{0.212622in}}{\pgfqpoint{3.696000in}{3.696000in}}%
\pgfusepath{clip}%
\pgfsetbuttcap%
\pgfsetroundjoin%
\definecolor{currentfill}{rgb}{0.121569,0.466667,0.705882}%
\pgfsetfillcolor{currentfill}%
\pgfsetfillopacity{0.357925}%
\pgfsetlinewidth{1.003750pt}%
\definecolor{currentstroke}{rgb}{0.121569,0.466667,0.705882}%
\pgfsetstrokecolor{currentstroke}%
\pgfsetstrokeopacity{0.357925}%
\pgfsetdash{}{0pt}%
\pgfpathmoveto{\pgfqpoint{1.698374in}{3.049943in}}%
\pgfpathcurveto{\pgfqpoint{1.706610in}{3.049943in}}{\pgfqpoint{1.714511in}{3.053215in}}{\pgfqpoint{1.720334in}{3.059039in}}%
\pgfpathcurveto{\pgfqpoint{1.726158in}{3.064863in}}{\pgfqpoint{1.729431in}{3.072763in}}{\pgfqpoint{1.729431in}{3.081000in}}%
\pgfpathcurveto{\pgfqpoint{1.729431in}{3.089236in}}{\pgfqpoint{1.726158in}{3.097136in}}{\pgfqpoint{1.720334in}{3.102960in}}%
\pgfpathcurveto{\pgfqpoint{1.714511in}{3.108784in}}{\pgfqpoint{1.706610in}{3.112056in}}{\pgfqpoint{1.698374in}{3.112056in}}%
\pgfpathcurveto{\pgfqpoint{1.690138in}{3.112056in}}{\pgfqpoint{1.682238in}{3.108784in}}{\pgfqpoint{1.676414in}{3.102960in}}%
\pgfpathcurveto{\pgfqpoint{1.670590in}{3.097136in}}{\pgfqpoint{1.667318in}{3.089236in}}{\pgfqpoint{1.667318in}{3.081000in}}%
\pgfpathcurveto{\pgfqpoint{1.667318in}{3.072763in}}{\pgfqpoint{1.670590in}{3.064863in}}{\pgfqpoint{1.676414in}{3.059039in}}%
\pgfpathcurveto{\pgfqpoint{1.682238in}{3.053215in}}{\pgfqpoint{1.690138in}{3.049943in}}{\pgfqpoint{1.698374in}{3.049943in}}%
\pgfpathclose%
\pgfusepath{stroke,fill}%
\end{pgfscope}%
\begin{pgfscope}%
\pgfpathrectangle{\pgfqpoint{0.100000in}{0.212622in}}{\pgfqpoint{3.696000in}{3.696000in}}%
\pgfusepath{clip}%
\pgfsetbuttcap%
\pgfsetroundjoin%
\definecolor{currentfill}{rgb}{0.121569,0.466667,0.705882}%
\pgfsetfillcolor{currentfill}%
\pgfsetfillopacity{0.358306}%
\pgfsetlinewidth{1.003750pt}%
\definecolor{currentstroke}{rgb}{0.121569,0.466667,0.705882}%
\pgfsetstrokecolor{currentstroke}%
\pgfsetstrokeopacity{0.358306}%
\pgfsetdash{}{0pt}%
\pgfpathmoveto{\pgfqpoint{1.697347in}{3.047792in}}%
\pgfpathcurveto{\pgfqpoint{1.705584in}{3.047792in}}{\pgfqpoint{1.713484in}{3.051065in}}{\pgfqpoint{1.719308in}{3.056889in}}%
\pgfpathcurveto{\pgfqpoint{1.725132in}{3.062713in}}{\pgfqpoint{1.728404in}{3.070613in}}{\pgfqpoint{1.728404in}{3.078849in}}%
\pgfpathcurveto{\pgfqpoint{1.728404in}{3.087085in}}{\pgfqpoint{1.725132in}{3.094985in}}{\pgfqpoint{1.719308in}{3.100809in}}%
\pgfpathcurveto{\pgfqpoint{1.713484in}{3.106633in}}{\pgfqpoint{1.705584in}{3.109905in}}{\pgfqpoint{1.697347in}{3.109905in}}%
\pgfpathcurveto{\pgfqpoint{1.689111in}{3.109905in}}{\pgfqpoint{1.681211in}{3.106633in}}{\pgfqpoint{1.675387in}{3.100809in}}%
\pgfpathcurveto{\pgfqpoint{1.669563in}{3.094985in}}{\pgfqpoint{1.666291in}{3.087085in}}{\pgfqpoint{1.666291in}{3.078849in}}%
\pgfpathcurveto{\pgfqpoint{1.666291in}{3.070613in}}{\pgfqpoint{1.669563in}{3.062713in}}{\pgfqpoint{1.675387in}{3.056889in}}%
\pgfpathcurveto{\pgfqpoint{1.681211in}{3.051065in}}{\pgfqpoint{1.689111in}{3.047792in}}{\pgfqpoint{1.697347in}{3.047792in}}%
\pgfpathclose%
\pgfusepath{stroke,fill}%
\end{pgfscope}%
\begin{pgfscope}%
\pgfpathrectangle{\pgfqpoint{0.100000in}{0.212622in}}{\pgfqpoint{3.696000in}{3.696000in}}%
\pgfusepath{clip}%
\pgfsetbuttcap%
\pgfsetroundjoin%
\definecolor{currentfill}{rgb}{0.121569,0.466667,0.705882}%
\pgfsetfillcolor{currentfill}%
\pgfsetfillopacity{0.358688}%
\pgfsetlinewidth{1.003750pt}%
\definecolor{currentstroke}{rgb}{0.121569,0.466667,0.705882}%
\pgfsetstrokecolor{currentstroke}%
\pgfsetstrokeopacity{0.358688}%
\pgfsetdash{}{0pt}%
\pgfpathmoveto{\pgfqpoint{1.951007in}{3.076527in}}%
\pgfpathcurveto{\pgfqpoint{1.959243in}{3.076527in}}{\pgfqpoint{1.967143in}{3.079799in}}{\pgfqpoint{1.972967in}{3.085623in}}%
\pgfpathcurveto{\pgfqpoint{1.978791in}{3.091447in}}{\pgfqpoint{1.982063in}{3.099347in}}{\pgfqpoint{1.982063in}{3.107583in}}%
\pgfpathcurveto{\pgfqpoint{1.982063in}{3.115820in}}{\pgfqpoint{1.978791in}{3.123720in}}{\pgfqpoint{1.972967in}{3.129544in}}%
\pgfpathcurveto{\pgfqpoint{1.967143in}{3.135367in}}{\pgfqpoint{1.959243in}{3.138640in}}{\pgfqpoint{1.951007in}{3.138640in}}%
\pgfpathcurveto{\pgfqpoint{1.942771in}{3.138640in}}{\pgfqpoint{1.934871in}{3.135367in}}{\pgfqpoint{1.929047in}{3.129544in}}%
\pgfpathcurveto{\pgfqpoint{1.923223in}{3.123720in}}{\pgfqpoint{1.919950in}{3.115820in}}{\pgfqpoint{1.919950in}{3.107583in}}%
\pgfpathcurveto{\pgfqpoint{1.919950in}{3.099347in}}{\pgfqpoint{1.923223in}{3.091447in}}{\pgfqpoint{1.929047in}{3.085623in}}%
\pgfpathcurveto{\pgfqpoint{1.934871in}{3.079799in}}{\pgfqpoint{1.942771in}{3.076527in}}{\pgfqpoint{1.951007in}{3.076527in}}%
\pgfpathclose%
\pgfusepath{stroke,fill}%
\end{pgfscope}%
\begin{pgfscope}%
\pgfpathrectangle{\pgfqpoint{0.100000in}{0.212622in}}{\pgfqpoint{3.696000in}{3.696000in}}%
\pgfusepath{clip}%
\pgfsetbuttcap%
\pgfsetroundjoin%
\definecolor{currentfill}{rgb}{0.121569,0.466667,0.705882}%
\pgfsetfillcolor{currentfill}%
\pgfsetfillopacity{0.359011}%
\pgfsetlinewidth{1.003750pt}%
\definecolor{currentstroke}{rgb}{0.121569,0.466667,0.705882}%
\pgfsetstrokecolor{currentstroke}%
\pgfsetstrokeopacity{0.359011}%
\pgfsetdash{}{0pt}%
\pgfpathmoveto{\pgfqpoint{1.695212in}{3.044265in}}%
\pgfpathcurveto{\pgfqpoint{1.703449in}{3.044265in}}{\pgfqpoint{1.711349in}{3.047538in}}{\pgfqpoint{1.717173in}{3.053361in}}%
\pgfpathcurveto{\pgfqpoint{1.722996in}{3.059185in}}{\pgfqpoint{1.726269in}{3.067085in}}{\pgfqpoint{1.726269in}{3.075322in}}%
\pgfpathcurveto{\pgfqpoint{1.726269in}{3.083558in}}{\pgfqpoint{1.722996in}{3.091458in}}{\pgfqpoint{1.717173in}{3.097282in}}%
\pgfpathcurveto{\pgfqpoint{1.711349in}{3.103106in}}{\pgfqpoint{1.703449in}{3.106378in}}{\pgfqpoint{1.695212in}{3.106378in}}%
\pgfpathcurveto{\pgfqpoint{1.686976in}{3.106378in}}{\pgfqpoint{1.679076in}{3.103106in}}{\pgfqpoint{1.673252in}{3.097282in}}%
\pgfpathcurveto{\pgfqpoint{1.667428in}{3.091458in}}{\pgfqpoint{1.664156in}{3.083558in}}{\pgfqpoint{1.664156in}{3.075322in}}%
\pgfpathcurveto{\pgfqpoint{1.664156in}{3.067085in}}{\pgfqpoint{1.667428in}{3.059185in}}{\pgfqpoint{1.673252in}{3.053361in}}%
\pgfpathcurveto{\pgfqpoint{1.679076in}{3.047538in}}{\pgfqpoint{1.686976in}{3.044265in}}{\pgfqpoint{1.695212in}{3.044265in}}%
\pgfpathclose%
\pgfusepath{stroke,fill}%
\end{pgfscope}%
\begin{pgfscope}%
\pgfpathrectangle{\pgfqpoint{0.100000in}{0.212622in}}{\pgfqpoint{3.696000in}{3.696000in}}%
\pgfusepath{clip}%
\pgfsetbuttcap%
\pgfsetroundjoin%
\definecolor{currentfill}{rgb}{0.121569,0.466667,0.705882}%
\pgfsetfillcolor{currentfill}%
\pgfsetfillopacity{0.359725}%
\pgfsetlinewidth{1.003750pt}%
\definecolor{currentstroke}{rgb}{0.121569,0.466667,0.705882}%
\pgfsetstrokecolor{currentstroke}%
\pgfsetstrokeopacity{0.359725}%
\pgfsetdash{}{0pt}%
\pgfpathmoveto{\pgfqpoint{1.951952in}{3.071825in}}%
\pgfpathcurveto{\pgfqpoint{1.960188in}{3.071825in}}{\pgfqpoint{1.968088in}{3.075097in}}{\pgfqpoint{1.973912in}{3.080921in}}%
\pgfpathcurveto{\pgfqpoint{1.979736in}{3.086745in}}{\pgfqpoint{1.983008in}{3.094645in}}{\pgfqpoint{1.983008in}{3.102881in}}%
\pgfpathcurveto{\pgfqpoint{1.983008in}{3.111117in}}{\pgfqpoint{1.979736in}{3.119017in}}{\pgfqpoint{1.973912in}{3.124841in}}%
\pgfpathcurveto{\pgfqpoint{1.968088in}{3.130665in}}{\pgfqpoint{1.960188in}{3.133938in}}{\pgfqpoint{1.951952in}{3.133938in}}%
\pgfpathcurveto{\pgfqpoint{1.943715in}{3.133938in}}{\pgfqpoint{1.935815in}{3.130665in}}{\pgfqpoint{1.929991in}{3.124841in}}%
\pgfpathcurveto{\pgfqpoint{1.924167in}{3.119017in}}{\pgfqpoint{1.920895in}{3.111117in}}{\pgfqpoint{1.920895in}{3.102881in}}%
\pgfpathcurveto{\pgfqpoint{1.920895in}{3.094645in}}{\pgfqpoint{1.924167in}{3.086745in}}{\pgfqpoint{1.929991in}{3.080921in}}%
\pgfpathcurveto{\pgfqpoint{1.935815in}{3.075097in}}{\pgfqpoint{1.943715in}{3.071825in}}{\pgfqpoint{1.951952in}{3.071825in}}%
\pgfpathclose%
\pgfusepath{stroke,fill}%
\end{pgfscope}%
\begin{pgfscope}%
\pgfpathrectangle{\pgfqpoint{0.100000in}{0.212622in}}{\pgfqpoint{3.696000in}{3.696000in}}%
\pgfusepath{clip}%
\pgfsetbuttcap%
\pgfsetroundjoin%
\definecolor{currentfill}{rgb}{0.121569,0.466667,0.705882}%
\pgfsetfillcolor{currentfill}%
\pgfsetfillopacity{0.360221}%
\pgfsetlinewidth{1.003750pt}%
\definecolor{currentstroke}{rgb}{0.121569,0.466667,0.705882}%
\pgfsetstrokecolor{currentstroke}%
\pgfsetstrokeopacity{0.360221}%
\pgfsetdash{}{0pt}%
\pgfpathmoveto{\pgfqpoint{1.691232in}{3.037713in}}%
\pgfpathcurveto{\pgfqpoint{1.699469in}{3.037713in}}{\pgfqpoint{1.707369in}{3.040985in}}{\pgfqpoint{1.713193in}{3.046809in}}%
\pgfpathcurveto{\pgfqpoint{1.719017in}{3.052633in}}{\pgfqpoint{1.722289in}{3.060533in}}{\pgfqpoint{1.722289in}{3.068769in}}%
\pgfpathcurveto{\pgfqpoint{1.722289in}{3.077006in}}{\pgfqpoint{1.719017in}{3.084906in}}{\pgfqpoint{1.713193in}{3.090730in}}%
\pgfpathcurveto{\pgfqpoint{1.707369in}{3.096554in}}{\pgfqpoint{1.699469in}{3.099826in}}{\pgfqpoint{1.691232in}{3.099826in}}%
\pgfpathcurveto{\pgfqpoint{1.682996in}{3.099826in}}{\pgfqpoint{1.675096in}{3.096554in}}{\pgfqpoint{1.669272in}{3.090730in}}%
\pgfpathcurveto{\pgfqpoint{1.663448in}{3.084906in}}{\pgfqpoint{1.660176in}{3.077006in}}{\pgfqpoint{1.660176in}{3.068769in}}%
\pgfpathcurveto{\pgfqpoint{1.660176in}{3.060533in}}{\pgfqpoint{1.663448in}{3.052633in}}{\pgfqpoint{1.669272in}{3.046809in}}%
\pgfpathcurveto{\pgfqpoint{1.675096in}{3.040985in}}{\pgfqpoint{1.682996in}{3.037713in}}{\pgfqpoint{1.691232in}{3.037713in}}%
\pgfpathclose%
\pgfusepath{stroke,fill}%
\end{pgfscope}%
\begin{pgfscope}%
\pgfpathrectangle{\pgfqpoint{0.100000in}{0.212622in}}{\pgfqpoint{3.696000in}{3.696000in}}%
\pgfusepath{clip}%
\pgfsetbuttcap%
\pgfsetroundjoin%
\definecolor{currentfill}{rgb}{0.121569,0.466667,0.705882}%
\pgfsetfillcolor{currentfill}%
\pgfsetfillopacity{0.361275}%
\pgfsetlinewidth{1.003750pt}%
\definecolor{currentstroke}{rgb}{0.121569,0.466667,0.705882}%
\pgfsetstrokecolor{currentstroke}%
\pgfsetstrokeopacity{0.361275}%
\pgfsetdash{}{0pt}%
\pgfpathmoveto{\pgfqpoint{1.952655in}{3.065991in}}%
\pgfpathcurveto{\pgfqpoint{1.960891in}{3.065991in}}{\pgfqpoint{1.968791in}{3.069264in}}{\pgfqpoint{1.974615in}{3.075088in}}%
\pgfpathcurveto{\pgfqpoint{1.980439in}{3.080911in}}{\pgfqpoint{1.983712in}{3.088812in}}{\pgfqpoint{1.983712in}{3.097048in}}%
\pgfpathcurveto{\pgfqpoint{1.983712in}{3.105284in}}{\pgfqpoint{1.980439in}{3.113184in}}{\pgfqpoint{1.974615in}{3.119008in}}%
\pgfpathcurveto{\pgfqpoint{1.968791in}{3.124832in}}{\pgfqpoint{1.960891in}{3.128104in}}{\pgfqpoint{1.952655in}{3.128104in}}%
\pgfpathcurveto{\pgfqpoint{1.944419in}{3.128104in}}{\pgfqpoint{1.936519in}{3.124832in}}{\pgfqpoint{1.930695in}{3.119008in}}%
\pgfpathcurveto{\pgfqpoint{1.924871in}{3.113184in}}{\pgfqpoint{1.921599in}{3.105284in}}{\pgfqpoint{1.921599in}{3.097048in}}%
\pgfpathcurveto{\pgfqpoint{1.921599in}{3.088812in}}{\pgfqpoint{1.924871in}{3.080911in}}{\pgfqpoint{1.930695in}{3.075088in}}%
\pgfpathcurveto{\pgfqpoint{1.936519in}{3.069264in}}{\pgfqpoint{1.944419in}{3.065991in}}{\pgfqpoint{1.952655in}{3.065991in}}%
\pgfpathclose%
\pgfusepath{stroke,fill}%
\end{pgfscope}%
\begin{pgfscope}%
\pgfpathrectangle{\pgfqpoint{0.100000in}{0.212622in}}{\pgfqpoint{3.696000in}{3.696000in}}%
\pgfusepath{clip}%
\pgfsetbuttcap%
\pgfsetroundjoin%
\definecolor{currentfill}{rgb}{0.121569,0.466667,0.705882}%
\pgfsetfillcolor{currentfill}%
\pgfsetfillopacity{0.361356}%
\pgfsetlinewidth{1.003750pt}%
\definecolor{currentstroke}{rgb}{0.121569,0.466667,0.705882}%
\pgfsetstrokecolor{currentstroke}%
\pgfsetstrokeopacity{0.361356}%
\pgfsetdash{}{0pt}%
\pgfpathmoveto{\pgfqpoint{1.688235in}{3.031233in}}%
\pgfpathcurveto{\pgfqpoint{1.696472in}{3.031233in}}{\pgfqpoint{1.704372in}{3.034505in}}{\pgfqpoint{1.710196in}{3.040329in}}%
\pgfpathcurveto{\pgfqpoint{1.716020in}{3.046153in}}{\pgfqpoint{1.719292in}{3.054053in}}{\pgfqpoint{1.719292in}{3.062289in}}%
\pgfpathcurveto{\pgfqpoint{1.719292in}{3.070526in}}{\pgfqpoint{1.716020in}{3.078426in}}{\pgfqpoint{1.710196in}{3.084250in}}%
\pgfpathcurveto{\pgfqpoint{1.704372in}{3.090074in}}{\pgfqpoint{1.696472in}{3.093346in}}{\pgfqpoint{1.688235in}{3.093346in}}%
\pgfpathcurveto{\pgfqpoint{1.679999in}{3.093346in}}{\pgfqpoint{1.672099in}{3.090074in}}{\pgfqpoint{1.666275in}{3.084250in}}%
\pgfpathcurveto{\pgfqpoint{1.660451in}{3.078426in}}{\pgfqpoint{1.657179in}{3.070526in}}{\pgfqpoint{1.657179in}{3.062289in}}%
\pgfpathcurveto{\pgfqpoint{1.657179in}{3.054053in}}{\pgfqpoint{1.660451in}{3.046153in}}{\pgfqpoint{1.666275in}{3.040329in}}%
\pgfpathcurveto{\pgfqpoint{1.672099in}{3.034505in}}{\pgfqpoint{1.679999in}{3.031233in}}{\pgfqpoint{1.688235in}{3.031233in}}%
\pgfpathclose%
\pgfusepath{stroke,fill}%
\end{pgfscope}%
\begin{pgfscope}%
\pgfpathrectangle{\pgfqpoint{0.100000in}{0.212622in}}{\pgfqpoint{3.696000in}{3.696000in}}%
\pgfusepath{clip}%
\pgfsetbuttcap%
\pgfsetroundjoin%
\definecolor{currentfill}{rgb}{0.121569,0.466667,0.705882}%
\pgfsetfillcolor{currentfill}%
\pgfsetfillopacity{0.362019}%
\pgfsetlinewidth{1.003750pt}%
\definecolor{currentstroke}{rgb}{0.121569,0.466667,0.705882}%
\pgfsetstrokecolor{currentstroke}%
\pgfsetstrokeopacity{0.362019}%
\pgfsetdash{}{0pt}%
\pgfpathmoveto{\pgfqpoint{1.686046in}{3.027668in}}%
\pgfpathcurveto{\pgfqpoint{1.694282in}{3.027668in}}{\pgfqpoint{1.702182in}{3.030940in}}{\pgfqpoint{1.708006in}{3.036764in}}%
\pgfpathcurveto{\pgfqpoint{1.713830in}{3.042588in}}{\pgfqpoint{1.717102in}{3.050488in}}{\pgfqpoint{1.717102in}{3.058724in}}%
\pgfpathcurveto{\pgfqpoint{1.717102in}{3.066960in}}{\pgfqpoint{1.713830in}{3.074861in}}{\pgfqpoint{1.708006in}{3.080684in}}%
\pgfpathcurveto{\pgfqpoint{1.702182in}{3.086508in}}{\pgfqpoint{1.694282in}{3.089781in}}{\pgfqpoint{1.686046in}{3.089781in}}%
\pgfpathcurveto{\pgfqpoint{1.677809in}{3.089781in}}{\pgfqpoint{1.669909in}{3.086508in}}{\pgfqpoint{1.664086in}{3.080684in}}%
\pgfpathcurveto{\pgfqpoint{1.658262in}{3.074861in}}{\pgfqpoint{1.654989in}{3.066960in}}{\pgfqpoint{1.654989in}{3.058724in}}%
\pgfpathcurveto{\pgfqpoint{1.654989in}{3.050488in}}{\pgfqpoint{1.658262in}{3.042588in}}{\pgfqpoint{1.664086in}{3.036764in}}%
\pgfpathcurveto{\pgfqpoint{1.669909in}{3.030940in}}{\pgfqpoint{1.677809in}{3.027668in}}{\pgfqpoint{1.686046in}{3.027668in}}%
\pgfpathclose%
\pgfusepath{stroke,fill}%
\end{pgfscope}%
\begin{pgfscope}%
\pgfpathrectangle{\pgfqpoint{0.100000in}{0.212622in}}{\pgfqpoint{3.696000in}{3.696000in}}%
\pgfusepath{clip}%
\pgfsetbuttcap%
\pgfsetroundjoin%
\definecolor{currentfill}{rgb}{0.121569,0.466667,0.705882}%
\pgfsetfillcolor{currentfill}%
\pgfsetfillopacity{0.362652}%
\pgfsetlinewidth{1.003750pt}%
\definecolor{currentstroke}{rgb}{0.121569,0.466667,0.705882}%
\pgfsetstrokecolor{currentstroke}%
\pgfsetstrokeopacity{0.362652}%
\pgfsetdash{}{0pt}%
\pgfpathmoveto{\pgfqpoint{1.684165in}{3.024303in}}%
\pgfpathcurveto{\pgfqpoint{1.692401in}{3.024303in}}{\pgfqpoint{1.700301in}{3.027575in}}{\pgfqpoint{1.706125in}{3.033399in}}%
\pgfpathcurveto{\pgfqpoint{1.711949in}{3.039223in}}{\pgfqpoint{1.715221in}{3.047123in}}{\pgfqpoint{1.715221in}{3.055359in}}%
\pgfpathcurveto{\pgfqpoint{1.715221in}{3.063595in}}{\pgfqpoint{1.711949in}{3.071495in}}{\pgfqpoint{1.706125in}{3.077319in}}%
\pgfpathcurveto{\pgfqpoint{1.700301in}{3.083143in}}{\pgfqpoint{1.692401in}{3.086416in}}{\pgfqpoint{1.684165in}{3.086416in}}%
\pgfpathcurveto{\pgfqpoint{1.675928in}{3.086416in}}{\pgfqpoint{1.668028in}{3.083143in}}{\pgfqpoint{1.662204in}{3.077319in}}%
\pgfpathcurveto{\pgfqpoint{1.656380in}{3.071495in}}{\pgfqpoint{1.653108in}{3.063595in}}{\pgfqpoint{1.653108in}{3.055359in}}%
\pgfpathcurveto{\pgfqpoint{1.653108in}{3.047123in}}{\pgfqpoint{1.656380in}{3.039223in}}{\pgfqpoint{1.662204in}{3.033399in}}%
\pgfpathcurveto{\pgfqpoint{1.668028in}{3.027575in}}{\pgfqpoint{1.675928in}{3.024303in}}{\pgfqpoint{1.684165in}{3.024303in}}%
\pgfpathclose%
\pgfusepath{stroke,fill}%
\end{pgfscope}%
\begin{pgfscope}%
\pgfpathrectangle{\pgfqpoint{0.100000in}{0.212622in}}{\pgfqpoint{3.696000in}{3.696000in}}%
\pgfusepath{clip}%
\pgfsetbuttcap%
\pgfsetroundjoin%
\definecolor{currentfill}{rgb}{0.121569,0.466667,0.705882}%
\pgfsetfillcolor{currentfill}%
\pgfsetfillopacity{0.362974}%
\pgfsetlinewidth{1.003750pt}%
\definecolor{currentstroke}{rgb}{0.121569,0.466667,0.705882}%
\pgfsetstrokecolor{currentstroke}%
\pgfsetstrokeopacity{0.362974}%
\pgfsetdash{}{0pt}%
\pgfpathmoveto{\pgfqpoint{1.953655in}{3.059652in}}%
\pgfpathcurveto{\pgfqpoint{1.961891in}{3.059652in}}{\pgfqpoint{1.969791in}{3.062924in}}{\pgfqpoint{1.975615in}{3.068748in}}%
\pgfpathcurveto{\pgfqpoint{1.981439in}{3.074572in}}{\pgfqpoint{1.984711in}{3.082472in}}{\pgfqpoint{1.984711in}{3.090709in}}%
\pgfpathcurveto{\pgfqpoint{1.984711in}{3.098945in}}{\pgfqpoint{1.981439in}{3.106845in}}{\pgfqpoint{1.975615in}{3.112669in}}%
\pgfpathcurveto{\pgfqpoint{1.969791in}{3.118493in}}{\pgfqpoint{1.961891in}{3.121765in}}{\pgfqpoint{1.953655in}{3.121765in}}%
\pgfpathcurveto{\pgfqpoint{1.945419in}{3.121765in}}{\pgfqpoint{1.937519in}{3.118493in}}{\pgfqpoint{1.931695in}{3.112669in}}%
\pgfpathcurveto{\pgfqpoint{1.925871in}{3.106845in}}{\pgfqpoint{1.922599in}{3.098945in}}{\pgfqpoint{1.922599in}{3.090709in}}%
\pgfpathcurveto{\pgfqpoint{1.922599in}{3.082472in}}{\pgfqpoint{1.925871in}{3.074572in}}{\pgfqpoint{1.931695in}{3.068748in}}%
\pgfpathcurveto{\pgfqpoint{1.937519in}{3.062924in}}{\pgfqpoint{1.945419in}{3.059652in}}{\pgfqpoint{1.953655in}{3.059652in}}%
\pgfpathclose%
\pgfusepath{stroke,fill}%
\end{pgfscope}%
\begin{pgfscope}%
\pgfpathrectangle{\pgfqpoint{0.100000in}{0.212622in}}{\pgfqpoint{3.696000in}{3.696000in}}%
\pgfusepath{clip}%
\pgfsetbuttcap%
\pgfsetroundjoin%
\definecolor{currentfill}{rgb}{0.121569,0.466667,0.705882}%
\pgfsetfillcolor{currentfill}%
\pgfsetfillopacity{0.363826}%
\pgfsetlinewidth{1.003750pt}%
\definecolor{currentstroke}{rgb}{0.121569,0.466667,0.705882}%
\pgfsetstrokecolor{currentstroke}%
\pgfsetstrokeopacity{0.363826}%
\pgfsetdash{}{0pt}%
\pgfpathmoveto{\pgfqpoint{1.680962in}{3.018000in}}%
\pgfpathcurveto{\pgfqpoint{1.689198in}{3.018000in}}{\pgfqpoint{1.697098in}{3.021272in}}{\pgfqpoint{1.702922in}{3.027096in}}%
\pgfpathcurveto{\pgfqpoint{1.708746in}{3.032920in}}{\pgfqpoint{1.712018in}{3.040820in}}{\pgfqpoint{1.712018in}{3.049056in}}%
\pgfpathcurveto{\pgfqpoint{1.712018in}{3.057292in}}{\pgfqpoint{1.708746in}{3.065193in}}{\pgfqpoint{1.702922in}{3.071016in}}%
\pgfpathcurveto{\pgfqpoint{1.697098in}{3.076840in}}{\pgfqpoint{1.689198in}{3.080113in}}{\pgfqpoint{1.680962in}{3.080113in}}%
\pgfpathcurveto{\pgfqpoint{1.672725in}{3.080113in}}{\pgfqpoint{1.664825in}{3.076840in}}{\pgfqpoint{1.659001in}{3.071016in}}%
\pgfpathcurveto{\pgfqpoint{1.653177in}{3.065193in}}{\pgfqpoint{1.649905in}{3.057292in}}{\pgfqpoint{1.649905in}{3.049056in}}%
\pgfpathcurveto{\pgfqpoint{1.649905in}{3.040820in}}{\pgfqpoint{1.653177in}{3.032920in}}{\pgfqpoint{1.659001in}{3.027096in}}%
\pgfpathcurveto{\pgfqpoint{1.664825in}{3.021272in}}{\pgfqpoint{1.672725in}{3.018000in}}{\pgfqpoint{1.680962in}{3.018000in}}%
\pgfpathclose%
\pgfusepath{stroke,fill}%
\end{pgfscope}%
\begin{pgfscope}%
\pgfpathrectangle{\pgfqpoint{0.100000in}{0.212622in}}{\pgfqpoint{3.696000in}{3.696000in}}%
\pgfusepath{clip}%
\pgfsetbuttcap%
\pgfsetroundjoin%
\definecolor{currentfill}{rgb}{0.121569,0.466667,0.705882}%
\pgfsetfillcolor{currentfill}%
\pgfsetfillopacity{0.364605}%
\pgfsetlinewidth{1.003750pt}%
\definecolor{currentstroke}{rgb}{0.121569,0.466667,0.705882}%
\pgfsetstrokecolor{currentstroke}%
\pgfsetstrokeopacity{0.364605}%
\pgfsetdash{}{0pt}%
\pgfpathmoveto{\pgfqpoint{1.955060in}{3.052315in}}%
\pgfpathcurveto{\pgfqpoint{1.963296in}{3.052315in}}{\pgfqpoint{1.971196in}{3.055588in}}{\pgfqpoint{1.977020in}{3.061412in}}%
\pgfpathcurveto{\pgfqpoint{1.982844in}{3.067236in}}{\pgfqpoint{1.986116in}{3.075136in}}{\pgfqpoint{1.986116in}{3.083372in}}%
\pgfpathcurveto{\pgfqpoint{1.986116in}{3.091608in}}{\pgfqpoint{1.982844in}{3.099508in}}{\pgfqpoint{1.977020in}{3.105332in}}%
\pgfpathcurveto{\pgfqpoint{1.971196in}{3.111156in}}{\pgfqpoint{1.963296in}{3.114428in}}{\pgfqpoint{1.955060in}{3.114428in}}%
\pgfpathcurveto{\pgfqpoint{1.946823in}{3.114428in}}{\pgfqpoint{1.938923in}{3.111156in}}{\pgfqpoint{1.933099in}{3.105332in}}%
\pgfpathcurveto{\pgfqpoint{1.927275in}{3.099508in}}{\pgfqpoint{1.924003in}{3.091608in}}{\pgfqpoint{1.924003in}{3.083372in}}%
\pgfpathcurveto{\pgfqpoint{1.924003in}{3.075136in}}{\pgfqpoint{1.927275in}{3.067236in}}{\pgfqpoint{1.933099in}{3.061412in}}%
\pgfpathcurveto{\pgfqpoint{1.938923in}{3.055588in}}{\pgfqpoint{1.946823in}{3.052315in}}{\pgfqpoint{1.955060in}{3.052315in}}%
\pgfpathclose%
\pgfusepath{stroke,fill}%
\end{pgfscope}%
\begin{pgfscope}%
\pgfpathrectangle{\pgfqpoint{0.100000in}{0.212622in}}{\pgfqpoint{3.696000in}{3.696000in}}%
\pgfusepath{clip}%
\pgfsetbuttcap%
\pgfsetroundjoin%
\definecolor{currentfill}{rgb}{0.121569,0.466667,0.705882}%
\pgfsetfillcolor{currentfill}%
\pgfsetfillopacity{0.364778}%
\pgfsetlinewidth{1.003750pt}%
\definecolor{currentstroke}{rgb}{0.121569,0.466667,0.705882}%
\pgfsetstrokecolor{currentstroke}%
\pgfsetstrokeopacity{0.364778}%
\pgfsetdash{}{0pt}%
\pgfpathmoveto{\pgfqpoint{1.677685in}{3.012600in}}%
\pgfpathcurveto{\pgfqpoint{1.685921in}{3.012600in}}{\pgfqpoint{1.693821in}{3.015873in}}{\pgfqpoint{1.699645in}{3.021697in}}%
\pgfpathcurveto{\pgfqpoint{1.705469in}{3.027521in}}{\pgfqpoint{1.708741in}{3.035421in}}{\pgfqpoint{1.708741in}{3.043657in}}%
\pgfpathcurveto{\pgfqpoint{1.708741in}{3.051893in}}{\pgfqpoint{1.705469in}{3.059793in}}{\pgfqpoint{1.699645in}{3.065617in}}%
\pgfpathcurveto{\pgfqpoint{1.693821in}{3.071441in}}{\pgfqpoint{1.685921in}{3.074713in}}{\pgfqpoint{1.677685in}{3.074713in}}%
\pgfpathcurveto{\pgfqpoint{1.669449in}{3.074713in}}{\pgfqpoint{1.661549in}{3.071441in}}{\pgfqpoint{1.655725in}{3.065617in}}%
\pgfpathcurveto{\pgfqpoint{1.649901in}{3.059793in}}{\pgfqpoint{1.646628in}{3.051893in}}{\pgfqpoint{1.646628in}{3.043657in}}%
\pgfpathcurveto{\pgfqpoint{1.646628in}{3.035421in}}{\pgfqpoint{1.649901in}{3.027521in}}{\pgfqpoint{1.655725in}{3.021697in}}%
\pgfpathcurveto{\pgfqpoint{1.661549in}{3.015873in}}{\pgfqpoint{1.669449in}{3.012600in}}{\pgfqpoint{1.677685in}{3.012600in}}%
\pgfpathclose%
\pgfusepath{stroke,fill}%
\end{pgfscope}%
\begin{pgfscope}%
\pgfpathrectangle{\pgfqpoint{0.100000in}{0.212622in}}{\pgfqpoint{3.696000in}{3.696000in}}%
\pgfusepath{clip}%
\pgfsetbuttcap%
\pgfsetroundjoin%
\definecolor{currentfill}{rgb}{0.121569,0.466667,0.705882}%
\pgfsetfillcolor{currentfill}%
\pgfsetfillopacity{0.365540}%
\pgfsetlinewidth{1.003750pt}%
\definecolor{currentstroke}{rgb}{0.121569,0.466667,0.705882}%
\pgfsetstrokecolor{currentstroke}%
\pgfsetstrokeopacity{0.365540}%
\pgfsetdash{}{0pt}%
\pgfpathmoveto{\pgfqpoint{1.675661in}{3.008291in}}%
\pgfpathcurveto{\pgfqpoint{1.683897in}{3.008291in}}{\pgfqpoint{1.691797in}{3.011564in}}{\pgfqpoint{1.697621in}{3.017388in}}%
\pgfpathcurveto{\pgfqpoint{1.703445in}{3.023212in}}{\pgfqpoint{1.706717in}{3.031112in}}{\pgfqpoint{1.706717in}{3.039348in}}%
\pgfpathcurveto{\pgfqpoint{1.706717in}{3.047584in}}{\pgfqpoint{1.703445in}{3.055484in}}{\pgfqpoint{1.697621in}{3.061308in}}%
\pgfpathcurveto{\pgfqpoint{1.691797in}{3.067132in}}{\pgfqpoint{1.683897in}{3.070404in}}{\pgfqpoint{1.675661in}{3.070404in}}%
\pgfpathcurveto{\pgfqpoint{1.667425in}{3.070404in}}{\pgfqpoint{1.659525in}{3.067132in}}{\pgfqpoint{1.653701in}{3.061308in}}%
\pgfpathcurveto{\pgfqpoint{1.647877in}{3.055484in}}{\pgfqpoint{1.644604in}{3.047584in}}{\pgfqpoint{1.644604in}{3.039348in}}%
\pgfpathcurveto{\pgfqpoint{1.644604in}{3.031112in}}{\pgfqpoint{1.647877in}{3.023212in}}{\pgfqpoint{1.653701in}{3.017388in}}%
\pgfpathcurveto{\pgfqpoint{1.659525in}{3.011564in}}{\pgfqpoint{1.667425in}{3.008291in}}{\pgfqpoint{1.675661in}{3.008291in}}%
\pgfpathclose%
\pgfusepath{stroke,fill}%
\end{pgfscope}%
\begin{pgfscope}%
\pgfpathrectangle{\pgfqpoint{0.100000in}{0.212622in}}{\pgfqpoint{3.696000in}{3.696000in}}%
\pgfusepath{clip}%
\pgfsetbuttcap%
\pgfsetroundjoin%
\definecolor{currentfill}{rgb}{0.121569,0.466667,0.705882}%
\pgfsetfillcolor{currentfill}%
\pgfsetfillopacity{0.365673}%
\pgfsetlinewidth{1.003750pt}%
\definecolor{currentstroke}{rgb}{0.121569,0.466667,0.705882}%
\pgfsetstrokecolor{currentstroke}%
\pgfsetstrokeopacity{0.365673}%
\pgfsetdash{}{0pt}%
\pgfpathmoveto{\pgfqpoint{1.955497in}{3.048666in}}%
\pgfpathcurveto{\pgfqpoint{1.963733in}{3.048666in}}{\pgfqpoint{1.971633in}{3.051939in}}{\pgfqpoint{1.977457in}{3.057763in}}%
\pgfpathcurveto{\pgfqpoint{1.983281in}{3.063587in}}{\pgfqpoint{1.986554in}{3.071487in}}{\pgfqpoint{1.986554in}{3.079723in}}%
\pgfpathcurveto{\pgfqpoint{1.986554in}{3.087959in}}{\pgfqpoint{1.983281in}{3.095859in}}{\pgfqpoint{1.977457in}{3.101683in}}%
\pgfpathcurveto{\pgfqpoint{1.971633in}{3.107507in}}{\pgfqpoint{1.963733in}{3.110779in}}{\pgfqpoint{1.955497in}{3.110779in}}%
\pgfpathcurveto{\pgfqpoint{1.947261in}{3.110779in}}{\pgfqpoint{1.939361in}{3.107507in}}{\pgfqpoint{1.933537in}{3.101683in}}%
\pgfpathcurveto{\pgfqpoint{1.927713in}{3.095859in}}{\pgfqpoint{1.924441in}{3.087959in}}{\pgfqpoint{1.924441in}{3.079723in}}%
\pgfpathcurveto{\pgfqpoint{1.924441in}{3.071487in}}{\pgfqpoint{1.927713in}{3.063587in}}{\pgfqpoint{1.933537in}{3.057763in}}%
\pgfpathcurveto{\pgfqpoint{1.939361in}{3.051939in}}{\pgfqpoint{1.947261in}{3.048666in}}{\pgfqpoint{1.955497in}{3.048666in}}%
\pgfpathclose%
\pgfusepath{stroke,fill}%
\end{pgfscope}%
\begin{pgfscope}%
\pgfpathrectangle{\pgfqpoint{0.100000in}{0.212622in}}{\pgfqpoint{3.696000in}{3.696000in}}%
\pgfusepath{clip}%
\pgfsetbuttcap%
\pgfsetroundjoin%
\definecolor{currentfill}{rgb}{0.121569,0.466667,0.705882}%
\pgfsetfillcolor{currentfill}%
\pgfsetfillopacity{0.366225}%
\pgfsetlinewidth{1.003750pt}%
\definecolor{currentstroke}{rgb}{0.121569,0.466667,0.705882}%
\pgfsetstrokecolor{currentstroke}%
\pgfsetstrokeopacity{0.366225}%
\pgfsetdash{}{0pt}%
\pgfpathmoveto{\pgfqpoint{1.673559in}{3.004910in}}%
\pgfpathcurveto{\pgfqpoint{1.681795in}{3.004910in}}{\pgfqpoint{1.689695in}{3.008182in}}{\pgfqpoint{1.695519in}{3.014006in}}%
\pgfpathcurveto{\pgfqpoint{1.701343in}{3.019830in}}{\pgfqpoint{1.704615in}{3.027730in}}{\pgfqpoint{1.704615in}{3.035966in}}%
\pgfpathcurveto{\pgfqpoint{1.704615in}{3.044203in}}{\pgfqpoint{1.701343in}{3.052103in}}{\pgfqpoint{1.695519in}{3.057927in}}%
\pgfpathcurveto{\pgfqpoint{1.689695in}{3.063751in}}{\pgfqpoint{1.681795in}{3.067023in}}{\pgfqpoint{1.673559in}{3.067023in}}%
\pgfpathcurveto{\pgfqpoint{1.665323in}{3.067023in}}{\pgfqpoint{1.657423in}{3.063751in}}{\pgfqpoint{1.651599in}{3.057927in}}%
\pgfpathcurveto{\pgfqpoint{1.645775in}{3.052103in}}{\pgfqpoint{1.642502in}{3.044203in}}{\pgfqpoint{1.642502in}{3.035966in}}%
\pgfpathcurveto{\pgfqpoint{1.642502in}{3.027730in}}{\pgfqpoint{1.645775in}{3.019830in}}{\pgfqpoint{1.651599in}{3.014006in}}%
\pgfpathcurveto{\pgfqpoint{1.657423in}{3.008182in}}{\pgfqpoint{1.665323in}{3.004910in}}{\pgfqpoint{1.673559in}{3.004910in}}%
\pgfpathclose%
\pgfusepath{stroke,fill}%
\end{pgfscope}%
\begin{pgfscope}%
\pgfpathrectangle{\pgfqpoint{0.100000in}{0.212622in}}{\pgfqpoint{3.696000in}{3.696000in}}%
\pgfusepath{clip}%
\pgfsetbuttcap%
\pgfsetroundjoin%
\definecolor{currentfill}{rgb}{0.121569,0.466667,0.705882}%
\pgfsetfillcolor{currentfill}%
\pgfsetfillopacity{0.367005}%
\pgfsetlinewidth{1.003750pt}%
\definecolor{currentstroke}{rgb}{0.121569,0.466667,0.705882}%
\pgfsetstrokecolor{currentstroke}%
\pgfsetstrokeopacity{0.367005}%
\pgfsetdash{}{0pt}%
\pgfpathmoveto{\pgfqpoint{1.956435in}{3.043999in}}%
\pgfpathcurveto{\pgfqpoint{1.964672in}{3.043999in}}{\pgfqpoint{1.972572in}{3.047272in}}{\pgfqpoint{1.978396in}{3.053096in}}%
\pgfpathcurveto{\pgfqpoint{1.984220in}{3.058920in}}{\pgfqpoint{1.987492in}{3.066820in}}{\pgfqpoint{1.987492in}{3.075056in}}%
\pgfpathcurveto{\pgfqpoint{1.987492in}{3.083292in}}{\pgfqpoint{1.984220in}{3.091192in}}{\pgfqpoint{1.978396in}{3.097016in}}%
\pgfpathcurveto{\pgfqpoint{1.972572in}{3.102840in}}{\pgfqpoint{1.964672in}{3.106112in}}{\pgfqpoint{1.956435in}{3.106112in}}%
\pgfpathcurveto{\pgfqpoint{1.948199in}{3.106112in}}{\pgfqpoint{1.940299in}{3.102840in}}{\pgfqpoint{1.934475in}{3.097016in}}%
\pgfpathcurveto{\pgfqpoint{1.928651in}{3.091192in}}{\pgfqpoint{1.925379in}{3.083292in}}{\pgfqpoint{1.925379in}{3.075056in}}%
\pgfpathcurveto{\pgfqpoint{1.925379in}{3.066820in}}{\pgfqpoint{1.928651in}{3.058920in}}{\pgfqpoint{1.934475in}{3.053096in}}%
\pgfpathcurveto{\pgfqpoint{1.940299in}{3.047272in}}{\pgfqpoint{1.948199in}{3.043999in}}{\pgfqpoint{1.956435in}{3.043999in}}%
\pgfpathclose%
\pgfusepath{stroke,fill}%
\end{pgfscope}%
\begin{pgfscope}%
\pgfpathrectangle{\pgfqpoint{0.100000in}{0.212622in}}{\pgfqpoint{3.696000in}{3.696000in}}%
\pgfusepath{clip}%
\pgfsetbuttcap%
\pgfsetroundjoin%
\definecolor{currentfill}{rgb}{0.121569,0.466667,0.705882}%
\pgfsetfillcolor{currentfill}%
\pgfsetfillopacity{0.367468}%
\pgfsetlinewidth{1.003750pt}%
\definecolor{currentstroke}{rgb}{0.121569,0.466667,0.705882}%
\pgfsetstrokecolor{currentstroke}%
\pgfsetstrokeopacity{0.367468}%
\pgfsetdash{}{0pt}%
\pgfpathmoveto{\pgfqpoint{1.669857in}{2.998572in}}%
\pgfpathcurveto{\pgfqpoint{1.678093in}{2.998572in}}{\pgfqpoint{1.685993in}{3.001845in}}{\pgfqpoint{1.691817in}{3.007668in}}%
\pgfpathcurveto{\pgfqpoint{1.697641in}{3.013492in}}{\pgfqpoint{1.700913in}{3.021392in}}{\pgfqpoint{1.700913in}{3.029629in}}%
\pgfpathcurveto{\pgfqpoint{1.700913in}{3.037865in}}{\pgfqpoint{1.697641in}{3.045765in}}{\pgfqpoint{1.691817in}{3.051589in}}%
\pgfpathcurveto{\pgfqpoint{1.685993in}{3.057413in}}{\pgfqpoint{1.678093in}{3.060685in}}{\pgfqpoint{1.669857in}{3.060685in}}%
\pgfpathcurveto{\pgfqpoint{1.661620in}{3.060685in}}{\pgfqpoint{1.653720in}{3.057413in}}{\pgfqpoint{1.647897in}{3.051589in}}%
\pgfpathcurveto{\pgfqpoint{1.642073in}{3.045765in}}{\pgfqpoint{1.638800in}{3.037865in}}{\pgfqpoint{1.638800in}{3.029629in}}%
\pgfpathcurveto{\pgfqpoint{1.638800in}{3.021392in}}{\pgfqpoint{1.642073in}{3.013492in}}{\pgfqpoint{1.647897in}{3.007668in}}%
\pgfpathcurveto{\pgfqpoint{1.653720in}{3.001845in}}{\pgfqpoint{1.661620in}{2.998572in}}{\pgfqpoint{1.669857in}{2.998572in}}%
\pgfpathclose%
\pgfusepath{stroke,fill}%
\end{pgfscope}%
\begin{pgfscope}%
\pgfpathrectangle{\pgfqpoint{0.100000in}{0.212622in}}{\pgfqpoint{3.696000in}{3.696000in}}%
\pgfusepath{clip}%
\pgfsetbuttcap%
\pgfsetroundjoin%
\definecolor{currentfill}{rgb}{0.121569,0.466667,0.705882}%
\pgfsetfillcolor{currentfill}%
\pgfsetfillopacity{0.368398}%
\pgfsetlinewidth{1.003750pt}%
\definecolor{currentstroke}{rgb}{0.121569,0.466667,0.705882}%
\pgfsetstrokecolor{currentstroke}%
\pgfsetstrokeopacity{0.368398}%
\pgfsetdash{}{0pt}%
\pgfpathmoveto{\pgfqpoint{1.957549in}{3.038854in}}%
\pgfpathcurveto{\pgfqpoint{1.965785in}{3.038854in}}{\pgfqpoint{1.973685in}{3.042127in}}{\pgfqpoint{1.979509in}{3.047950in}}%
\pgfpathcurveto{\pgfqpoint{1.985333in}{3.053774in}}{\pgfqpoint{1.988605in}{3.061674in}}{\pgfqpoint{1.988605in}{3.069911in}}%
\pgfpathcurveto{\pgfqpoint{1.988605in}{3.078147in}}{\pgfqpoint{1.985333in}{3.086047in}}{\pgfqpoint{1.979509in}{3.091871in}}%
\pgfpathcurveto{\pgfqpoint{1.973685in}{3.097695in}}{\pgfqpoint{1.965785in}{3.100967in}}{\pgfqpoint{1.957549in}{3.100967in}}%
\pgfpathcurveto{\pgfqpoint{1.949312in}{3.100967in}}{\pgfqpoint{1.941412in}{3.097695in}}{\pgfqpoint{1.935588in}{3.091871in}}%
\pgfpathcurveto{\pgfqpoint{1.929764in}{3.086047in}}{\pgfqpoint{1.926492in}{3.078147in}}{\pgfqpoint{1.926492in}{3.069911in}}%
\pgfpathcurveto{\pgfqpoint{1.926492in}{3.061674in}}{\pgfqpoint{1.929764in}{3.053774in}}{\pgfqpoint{1.935588in}{3.047950in}}%
\pgfpathcurveto{\pgfqpoint{1.941412in}{3.042127in}}{\pgfqpoint{1.949312in}{3.038854in}}{\pgfqpoint{1.957549in}{3.038854in}}%
\pgfpathclose%
\pgfusepath{stroke,fill}%
\end{pgfscope}%
\begin{pgfscope}%
\pgfpathrectangle{\pgfqpoint{0.100000in}{0.212622in}}{\pgfqpoint{3.696000in}{3.696000in}}%
\pgfusepath{clip}%
\pgfsetbuttcap%
\pgfsetroundjoin%
\definecolor{currentfill}{rgb}{0.121569,0.466667,0.705882}%
\pgfsetfillcolor{currentfill}%
\pgfsetfillopacity{0.368532}%
\pgfsetlinewidth{1.003750pt}%
\definecolor{currentstroke}{rgb}{0.121569,0.466667,0.705882}%
\pgfsetstrokecolor{currentstroke}%
\pgfsetstrokeopacity{0.368532}%
\pgfsetdash{}{0pt}%
\pgfpathmoveto{\pgfqpoint{1.667024in}{2.993225in}}%
\pgfpathcurveto{\pgfqpoint{1.675261in}{2.993225in}}{\pgfqpoint{1.683161in}{2.996498in}}{\pgfqpoint{1.688985in}{3.002322in}}%
\pgfpathcurveto{\pgfqpoint{1.694808in}{3.008145in}}{\pgfqpoint{1.698081in}{3.016045in}}{\pgfqpoint{1.698081in}{3.024282in}}%
\pgfpathcurveto{\pgfqpoint{1.698081in}{3.032518in}}{\pgfqpoint{1.694808in}{3.040418in}}{\pgfqpoint{1.688985in}{3.046242in}}%
\pgfpathcurveto{\pgfqpoint{1.683161in}{3.052066in}}{\pgfqpoint{1.675261in}{3.055338in}}{\pgfqpoint{1.667024in}{3.055338in}}%
\pgfpathcurveto{\pgfqpoint{1.658788in}{3.055338in}}{\pgfqpoint{1.650888in}{3.052066in}}{\pgfqpoint{1.645064in}{3.046242in}}%
\pgfpathcurveto{\pgfqpoint{1.639240in}{3.040418in}}{\pgfqpoint{1.635968in}{3.032518in}}{\pgfqpoint{1.635968in}{3.024282in}}%
\pgfpathcurveto{\pgfqpoint{1.635968in}{3.016045in}}{\pgfqpoint{1.639240in}{3.008145in}}{\pgfqpoint{1.645064in}{3.002322in}}%
\pgfpathcurveto{\pgfqpoint{1.650888in}{2.996498in}}{\pgfqpoint{1.658788in}{2.993225in}}{\pgfqpoint{1.667024in}{2.993225in}}%
\pgfpathclose%
\pgfusepath{stroke,fill}%
\end{pgfscope}%
\begin{pgfscope}%
\pgfpathrectangle{\pgfqpoint{0.100000in}{0.212622in}}{\pgfqpoint{3.696000in}{3.696000in}}%
\pgfusepath{clip}%
\pgfsetbuttcap%
\pgfsetroundjoin%
\definecolor{currentfill}{rgb}{0.121569,0.466667,0.705882}%
\pgfsetfillcolor{currentfill}%
\pgfsetfillopacity{0.369308}%
\pgfsetlinewidth{1.003750pt}%
\definecolor{currentstroke}{rgb}{0.121569,0.466667,0.705882}%
\pgfsetstrokecolor{currentstroke}%
\pgfsetstrokeopacity{0.369308}%
\pgfsetdash{}{0pt}%
\pgfpathmoveto{\pgfqpoint{1.664343in}{2.988964in}}%
\pgfpathcurveto{\pgfqpoint{1.672579in}{2.988964in}}{\pgfqpoint{1.680479in}{2.992236in}}{\pgfqpoint{1.686303in}{2.998060in}}%
\pgfpathcurveto{\pgfqpoint{1.692127in}{3.003884in}}{\pgfqpoint{1.695399in}{3.011784in}}{\pgfqpoint{1.695399in}{3.020021in}}%
\pgfpathcurveto{\pgfqpoint{1.695399in}{3.028257in}}{\pgfqpoint{1.692127in}{3.036157in}}{\pgfqpoint{1.686303in}{3.041981in}}%
\pgfpathcurveto{\pgfqpoint{1.680479in}{3.047805in}}{\pgfqpoint{1.672579in}{3.051077in}}{\pgfqpoint{1.664343in}{3.051077in}}%
\pgfpathcurveto{\pgfqpoint{1.656106in}{3.051077in}}{\pgfqpoint{1.648206in}{3.047805in}}{\pgfqpoint{1.642382in}{3.041981in}}%
\pgfpathcurveto{\pgfqpoint{1.636558in}{3.036157in}}{\pgfqpoint{1.633286in}{3.028257in}}{\pgfqpoint{1.633286in}{3.020021in}}%
\pgfpathcurveto{\pgfqpoint{1.633286in}{3.011784in}}{\pgfqpoint{1.636558in}{3.003884in}}{\pgfqpoint{1.642382in}{2.998060in}}%
\pgfpathcurveto{\pgfqpoint{1.648206in}{2.992236in}}{\pgfqpoint{1.656106in}{2.988964in}}{\pgfqpoint{1.664343in}{2.988964in}}%
\pgfpathclose%
\pgfusepath{stroke,fill}%
\end{pgfscope}%
\begin{pgfscope}%
\pgfpathrectangle{\pgfqpoint{0.100000in}{0.212622in}}{\pgfqpoint{3.696000in}{3.696000in}}%
\pgfusepath{clip}%
\pgfsetbuttcap%
\pgfsetroundjoin%
\definecolor{currentfill}{rgb}{0.121569,0.466667,0.705882}%
\pgfsetfillcolor{currentfill}%
\pgfsetfillopacity{0.369779}%
\pgfsetlinewidth{1.003750pt}%
\definecolor{currentstroke}{rgb}{0.121569,0.466667,0.705882}%
\pgfsetstrokecolor{currentstroke}%
\pgfsetstrokeopacity{0.369779}%
\pgfsetdash{}{0pt}%
\pgfpathmoveto{\pgfqpoint{1.663081in}{2.986437in}}%
\pgfpathcurveto{\pgfqpoint{1.671318in}{2.986437in}}{\pgfqpoint{1.679218in}{2.989709in}}{\pgfqpoint{1.685041in}{2.995533in}}%
\pgfpathcurveto{\pgfqpoint{1.690865in}{3.001357in}}{\pgfqpoint{1.694138in}{3.009257in}}{\pgfqpoint{1.694138in}{3.017493in}}%
\pgfpathcurveto{\pgfqpoint{1.694138in}{3.025730in}}{\pgfqpoint{1.690865in}{3.033630in}}{\pgfqpoint{1.685041in}{3.039454in}}%
\pgfpathcurveto{\pgfqpoint{1.679218in}{3.045277in}}{\pgfqpoint{1.671318in}{3.048550in}}{\pgfqpoint{1.663081in}{3.048550in}}%
\pgfpathcurveto{\pgfqpoint{1.654845in}{3.048550in}}{\pgfqpoint{1.646945in}{3.045277in}}{\pgfqpoint{1.641121in}{3.039454in}}%
\pgfpathcurveto{\pgfqpoint{1.635297in}{3.033630in}}{\pgfqpoint{1.632025in}{3.025730in}}{\pgfqpoint{1.632025in}{3.017493in}}%
\pgfpathcurveto{\pgfqpoint{1.632025in}{3.009257in}}{\pgfqpoint{1.635297in}{3.001357in}}{\pgfqpoint{1.641121in}{2.995533in}}%
\pgfpathcurveto{\pgfqpoint{1.646945in}{2.989709in}}{\pgfqpoint{1.654845in}{2.986437in}}{\pgfqpoint{1.663081in}{2.986437in}}%
\pgfpathclose%
\pgfusepath{stroke,fill}%
\end{pgfscope}%
\begin{pgfscope}%
\pgfpathrectangle{\pgfqpoint{0.100000in}{0.212622in}}{\pgfqpoint{3.696000in}{3.696000in}}%
\pgfusepath{clip}%
\pgfsetbuttcap%
\pgfsetroundjoin%
\definecolor{currentfill}{rgb}{0.121569,0.466667,0.705882}%
\pgfsetfillcolor{currentfill}%
\pgfsetfillopacity{0.370173}%
\pgfsetlinewidth{1.003750pt}%
\definecolor{currentstroke}{rgb}{0.121569,0.466667,0.705882}%
\pgfsetstrokecolor{currentstroke}%
\pgfsetstrokeopacity{0.370173}%
\pgfsetdash{}{0pt}%
\pgfpathmoveto{\pgfqpoint{1.958284in}{3.033425in}}%
\pgfpathcurveto{\pgfqpoint{1.966520in}{3.033425in}}{\pgfqpoint{1.974420in}{3.036698in}}{\pgfqpoint{1.980244in}{3.042521in}}%
\pgfpathcurveto{\pgfqpoint{1.986068in}{3.048345in}}{\pgfqpoint{1.989340in}{3.056245in}}{\pgfqpoint{1.989340in}{3.064482in}}%
\pgfpathcurveto{\pgfqpoint{1.989340in}{3.072718in}}{\pgfqpoint{1.986068in}{3.080618in}}{\pgfqpoint{1.980244in}{3.086442in}}%
\pgfpathcurveto{\pgfqpoint{1.974420in}{3.092266in}}{\pgfqpoint{1.966520in}{3.095538in}}{\pgfqpoint{1.958284in}{3.095538in}}%
\pgfpathcurveto{\pgfqpoint{1.950048in}{3.095538in}}{\pgfqpoint{1.942148in}{3.092266in}}{\pgfqpoint{1.936324in}{3.086442in}}%
\pgfpathcurveto{\pgfqpoint{1.930500in}{3.080618in}}{\pgfqpoint{1.927227in}{3.072718in}}{\pgfqpoint{1.927227in}{3.064482in}}%
\pgfpathcurveto{\pgfqpoint{1.927227in}{3.056245in}}{\pgfqpoint{1.930500in}{3.048345in}}{\pgfqpoint{1.936324in}{3.042521in}}%
\pgfpathcurveto{\pgfqpoint{1.942148in}{3.036698in}}{\pgfqpoint{1.950048in}{3.033425in}}{\pgfqpoint{1.958284in}{3.033425in}}%
\pgfpathclose%
\pgfusepath{stroke,fill}%
\end{pgfscope}%
\begin{pgfscope}%
\pgfpathrectangle{\pgfqpoint{0.100000in}{0.212622in}}{\pgfqpoint{3.696000in}{3.696000in}}%
\pgfusepath{clip}%
\pgfsetbuttcap%
\pgfsetroundjoin%
\definecolor{currentfill}{rgb}{0.121569,0.466667,0.705882}%
\pgfsetfillcolor{currentfill}%
\pgfsetfillopacity{0.370636}%
\pgfsetlinewidth{1.003750pt}%
\definecolor{currentstroke}{rgb}{0.121569,0.466667,0.705882}%
\pgfsetstrokecolor{currentstroke}%
\pgfsetstrokeopacity{0.370636}%
\pgfsetdash{}{0pt}%
\pgfpathmoveto{\pgfqpoint{1.660572in}{2.982114in}}%
\pgfpathcurveto{\pgfqpoint{1.668809in}{2.982114in}}{\pgfqpoint{1.676709in}{2.985386in}}{\pgfqpoint{1.682533in}{2.991210in}}%
\pgfpathcurveto{\pgfqpoint{1.688357in}{2.997034in}}{\pgfqpoint{1.691629in}{3.004934in}}{\pgfqpoint{1.691629in}{3.013170in}}%
\pgfpathcurveto{\pgfqpoint{1.691629in}{3.021407in}}{\pgfqpoint{1.688357in}{3.029307in}}{\pgfqpoint{1.682533in}{3.035131in}}%
\pgfpathcurveto{\pgfqpoint{1.676709in}{3.040955in}}{\pgfqpoint{1.668809in}{3.044227in}}{\pgfqpoint{1.660572in}{3.044227in}}%
\pgfpathcurveto{\pgfqpoint{1.652336in}{3.044227in}}{\pgfqpoint{1.644436in}{3.040955in}}{\pgfqpoint{1.638612in}{3.035131in}}%
\pgfpathcurveto{\pgfqpoint{1.632788in}{3.029307in}}{\pgfqpoint{1.629516in}{3.021407in}}{\pgfqpoint{1.629516in}{3.013170in}}%
\pgfpathcurveto{\pgfqpoint{1.629516in}{3.004934in}}{\pgfqpoint{1.632788in}{2.997034in}}{\pgfqpoint{1.638612in}{2.991210in}}%
\pgfpathcurveto{\pgfqpoint{1.644436in}{2.985386in}}{\pgfqpoint{1.652336in}{2.982114in}}{\pgfqpoint{1.660572in}{2.982114in}}%
\pgfpathclose%
\pgfusepath{stroke,fill}%
\end{pgfscope}%
\begin{pgfscope}%
\pgfpathrectangle{\pgfqpoint{0.100000in}{0.212622in}}{\pgfqpoint{3.696000in}{3.696000in}}%
\pgfusepath{clip}%
\pgfsetbuttcap%
\pgfsetroundjoin%
\definecolor{currentfill}{rgb}{0.121569,0.466667,0.705882}%
\pgfsetfillcolor{currentfill}%
\pgfsetfillopacity{0.372110}%
\pgfsetlinewidth{1.003750pt}%
\definecolor{currentstroke}{rgb}{0.121569,0.466667,0.705882}%
\pgfsetstrokecolor{currentstroke}%
\pgfsetstrokeopacity{0.372110}%
\pgfsetdash{}{0pt}%
\pgfpathmoveto{\pgfqpoint{1.655817in}{2.974196in}}%
\pgfpathcurveto{\pgfqpoint{1.664053in}{2.974196in}}{\pgfqpoint{1.671953in}{2.977468in}}{\pgfqpoint{1.677777in}{2.983292in}}%
\pgfpathcurveto{\pgfqpoint{1.683601in}{2.989116in}}{\pgfqpoint{1.686874in}{2.997016in}}{\pgfqpoint{1.686874in}{3.005253in}}%
\pgfpathcurveto{\pgfqpoint{1.686874in}{3.013489in}}{\pgfqpoint{1.683601in}{3.021389in}}{\pgfqpoint{1.677777in}{3.027213in}}%
\pgfpathcurveto{\pgfqpoint{1.671953in}{3.033037in}}{\pgfqpoint{1.664053in}{3.036309in}}{\pgfqpoint{1.655817in}{3.036309in}}%
\pgfpathcurveto{\pgfqpoint{1.647581in}{3.036309in}}{\pgfqpoint{1.639681in}{3.033037in}}{\pgfqpoint{1.633857in}{3.027213in}}%
\pgfpathcurveto{\pgfqpoint{1.628033in}{3.021389in}}{\pgfqpoint{1.624761in}{3.013489in}}{\pgfqpoint{1.624761in}{3.005253in}}%
\pgfpathcurveto{\pgfqpoint{1.624761in}{2.997016in}}{\pgfqpoint{1.628033in}{2.989116in}}{\pgfqpoint{1.633857in}{2.983292in}}%
\pgfpathcurveto{\pgfqpoint{1.639681in}{2.977468in}}{\pgfqpoint{1.647581in}{2.974196in}}{\pgfqpoint{1.655817in}{2.974196in}}%
\pgfpathclose%
\pgfusepath{stroke,fill}%
\end{pgfscope}%
\begin{pgfscope}%
\pgfpathrectangle{\pgfqpoint{0.100000in}{0.212622in}}{\pgfqpoint{3.696000in}{3.696000in}}%
\pgfusepath{clip}%
\pgfsetbuttcap%
\pgfsetroundjoin%
\definecolor{currentfill}{rgb}{0.121569,0.466667,0.705882}%
\pgfsetfillcolor{currentfill}%
\pgfsetfillopacity{0.372111}%
\pgfsetlinewidth{1.003750pt}%
\definecolor{currentstroke}{rgb}{0.121569,0.466667,0.705882}%
\pgfsetstrokecolor{currentstroke}%
\pgfsetstrokeopacity{0.372111}%
\pgfsetdash{}{0pt}%
\pgfpathmoveto{\pgfqpoint{1.959716in}{3.027465in}}%
\pgfpathcurveto{\pgfqpoint{1.967952in}{3.027465in}}{\pgfqpoint{1.975852in}{3.030737in}}{\pgfqpoint{1.981676in}{3.036561in}}%
\pgfpathcurveto{\pgfqpoint{1.987500in}{3.042385in}}{\pgfqpoint{1.990772in}{3.050285in}}{\pgfqpoint{1.990772in}{3.058522in}}%
\pgfpathcurveto{\pgfqpoint{1.990772in}{3.066758in}}{\pgfqpoint{1.987500in}{3.074658in}}{\pgfqpoint{1.981676in}{3.080482in}}%
\pgfpathcurveto{\pgfqpoint{1.975852in}{3.086306in}}{\pgfqpoint{1.967952in}{3.089578in}}{\pgfqpoint{1.959716in}{3.089578in}}%
\pgfpathcurveto{\pgfqpoint{1.951480in}{3.089578in}}{\pgfqpoint{1.943579in}{3.086306in}}{\pgfqpoint{1.937756in}{3.080482in}}%
\pgfpathcurveto{\pgfqpoint{1.931932in}{3.074658in}}{\pgfqpoint{1.928659in}{3.066758in}}{\pgfqpoint{1.928659in}{3.058522in}}%
\pgfpathcurveto{\pgfqpoint{1.928659in}{3.050285in}}{\pgfqpoint{1.931932in}{3.042385in}}{\pgfqpoint{1.937756in}{3.036561in}}%
\pgfpathcurveto{\pgfqpoint{1.943579in}{3.030737in}}{\pgfqpoint{1.951480in}{3.027465in}}{\pgfqpoint{1.959716in}{3.027465in}}%
\pgfpathclose%
\pgfusepath{stroke,fill}%
\end{pgfscope}%
\begin{pgfscope}%
\pgfpathrectangle{\pgfqpoint{0.100000in}{0.212622in}}{\pgfqpoint{3.696000in}{3.696000in}}%
\pgfusepath{clip}%
\pgfsetbuttcap%
\pgfsetroundjoin%
\definecolor{currentfill}{rgb}{0.121569,0.466667,0.705882}%
\pgfsetfillcolor{currentfill}%
\pgfsetfillopacity{0.373533}%
\pgfsetlinewidth{1.003750pt}%
\definecolor{currentstroke}{rgb}{0.121569,0.466667,0.705882}%
\pgfsetstrokecolor{currentstroke}%
\pgfsetstrokeopacity{0.373533}%
\pgfsetdash{}{0pt}%
\pgfpathmoveto{\pgfqpoint{1.652316in}{2.966527in}}%
\pgfpathcurveto{\pgfqpoint{1.660552in}{2.966527in}}{\pgfqpoint{1.668452in}{2.969799in}}{\pgfqpoint{1.674276in}{2.975623in}}%
\pgfpathcurveto{\pgfqpoint{1.680100in}{2.981447in}}{\pgfqpoint{1.683372in}{2.989347in}}{\pgfqpoint{1.683372in}{2.997583in}}%
\pgfpathcurveto{\pgfqpoint{1.683372in}{3.005819in}}{\pgfqpoint{1.680100in}{3.013719in}}{\pgfqpoint{1.674276in}{3.019543in}}%
\pgfpathcurveto{\pgfqpoint{1.668452in}{3.025367in}}{\pgfqpoint{1.660552in}{3.028640in}}{\pgfqpoint{1.652316in}{3.028640in}}%
\pgfpathcurveto{\pgfqpoint{1.644079in}{3.028640in}}{\pgfqpoint{1.636179in}{3.025367in}}{\pgfqpoint{1.630355in}{3.019543in}}%
\pgfpathcurveto{\pgfqpoint{1.624531in}{3.013719in}}{\pgfqpoint{1.621259in}{3.005819in}}{\pgfqpoint{1.621259in}{2.997583in}}%
\pgfpathcurveto{\pgfqpoint{1.621259in}{2.989347in}}{\pgfqpoint{1.624531in}{2.981447in}}{\pgfqpoint{1.630355in}{2.975623in}}%
\pgfpathcurveto{\pgfqpoint{1.636179in}{2.969799in}}{\pgfqpoint{1.644079in}{2.966527in}}{\pgfqpoint{1.652316in}{2.966527in}}%
\pgfpathclose%
\pgfusepath{stroke,fill}%
\end{pgfscope}%
\begin{pgfscope}%
\pgfpathrectangle{\pgfqpoint{0.100000in}{0.212622in}}{\pgfqpoint{3.696000in}{3.696000in}}%
\pgfusepath{clip}%
\pgfsetbuttcap%
\pgfsetroundjoin%
\definecolor{currentfill}{rgb}{0.121569,0.466667,0.705882}%
\pgfsetfillcolor{currentfill}%
\pgfsetfillopacity{0.374109}%
\pgfsetlinewidth{1.003750pt}%
\definecolor{currentstroke}{rgb}{0.121569,0.466667,0.705882}%
\pgfsetstrokecolor{currentstroke}%
\pgfsetstrokeopacity{0.374109}%
\pgfsetdash{}{0pt}%
\pgfpathmoveto{\pgfqpoint{1.961141in}{3.020427in}}%
\pgfpathcurveto{\pgfqpoint{1.969378in}{3.020427in}}{\pgfqpoint{1.977278in}{3.023700in}}{\pgfqpoint{1.983102in}{3.029524in}}%
\pgfpathcurveto{\pgfqpoint{1.988925in}{3.035348in}}{\pgfqpoint{1.992198in}{3.043248in}}{\pgfqpoint{1.992198in}{3.051484in}}%
\pgfpathcurveto{\pgfqpoint{1.992198in}{3.059720in}}{\pgfqpoint{1.988925in}{3.067620in}}{\pgfqpoint{1.983102in}{3.073444in}}%
\pgfpathcurveto{\pgfqpoint{1.977278in}{3.079268in}}{\pgfqpoint{1.969378in}{3.082540in}}{\pgfqpoint{1.961141in}{3.082540in}}%
\pgfpathcurveto{\pgfqpoint{1.952905in}{3.082540in}}{\pgfqpoint{1.945005in}{3.079268in}}{\pgfqpoint{1.939181in}{3.073444in}}%
\pgfpathcurveto{\pgfqpoint{1.933357in}{3.067620in}}{\pgfqpoint{1.930085in}{3.059720in}}{\pgfqpoint{1.930085in}{3.051484in}}%
\pgfpathcurveto{\pgfqpoint{1.930085in}{3.043248in}}{\pgfqpoint{1.933357in}{3.035348in}}{\pgfqpoint{1.939181in}{3.029524in}}%
\pgfpathcurveto{\pgfqpoint{1.945005in}{3.023700in}}{\pgfqpoint{1.952905in}{3.020427in}}{\pgfqpoint{1.961141in}{3.020427in}}%
\pgfpathclose%
\pgfusepath{stroke,fill}%
\end{pgfscope}%
\begin{pgfscope}%
\pgfpathrectangle{\pgfqpoint{0.100000in}{0.212622in}}{\pgfqpoint{3.696000in}{3.696000in}}%
\pgfusepath{clip}%
\pgfsetbuttcap%
\pgfsetroundjoin%
\definecolor{currentfill}{rgb}{0.121569,0.466667,0.705882}%
\pgfsetfillcolor{currentfill}%
\pgfsetfillopacity{0.374451}%
\pgfsetlinewidth{1.003750pt}%
\definecolor{currentstroke}{rgb}{0.121569,0.466667,0.705882}%
\pgfsetstrokecolor{currentstroke}%
\pgfsetstrokeopacity{0.374451}%
\pgfsetdash{}{0pt}%
\pgfpathmoveto{\pgfqpoint{1.649441in}{2.961850in}}%
\pgfpathcurveto{\pgfqpoint{1.657678in}{2.961850in}}{\pgfqpoint{1.665578in}{2.965123in}}{\pgfqpoint{1.671402in}{2.970947in}}%
\pgfpathcurveto{\pgfqpoint{1.677226in}{2.976771in}}{\pgfqpoint{1.680498in}{2.984671in}}{\pgfqpoint{1.680498in}{2.992907in}}%
\pgfpathcurveto{\pgfqpoint{1.680498in}{3.001143in}}{\pgfqpoint{1.677226in}{3.009043in}}{\pgfqpoint{1.671402in}{3.014867in}}%
\pgfpathcurveto{\pgfqpoint{1.665578in}{3.020691in}}{\pgfqpoint{1.657678in}{3.023963in}}{\pgfqpoint{1.649441in}{3.023963in}}%
\pgfpathcurveto{\pgfqpoint{1.641205in}{3.023963in}}{\pgfqpoint{1.633305in}{3.020691in}}{\pgfqpoint{1.627481in}{3.014867in}}%
\pgfpathcurveto{\pgfqpoint{1.621657in}{3.009043in}}{\pgfqpoint{1.618385in}{3.001143in}}{\pgfqpoint{1.618385in}{2.992907in}}%
\pgfpathcurveto{\pgfqpoint{1.618385in}{2.984671in}}{\pgfqpoint{1.621657in}{2.976771in}}{\pgfqpoint{1.627481in}{2.970947in}}%
\pgfpathcurveto{\pgfqpoint{1.633305in}{2.965123in}}{\pgfqpoint{1.641205in}{2.961850in}}{\pgfqpoint{1.649441in}{2.961850in}}%
\pgfpathclose%
\pgfusepath{stroke,fill}%
\end{pgfscope}%
\begin{pgfscope}%
\pgfpathrectangle{\pgfqpoint{0.100000in}{0.212622in}}{\pgfqpoint{3.696000in}{3.696000in}}%
\pgfusepath{clip}%
\pgfsetbuttcap%
\pgfsetroundjoin%
\definecolor{currentfill}{rgb}{0.121569,0.466667,0.705882}%
\pgfsetfillcolor{currentfill}%
\pgfsetfillopacity{0.375357}%
\pgfsetlinewidth{1.003750pt}%
\definecolor{currentstroke}{rgb}{0.121569,0.466667,0.705882}%
\pgfsetstrokecolor{currentstroke}%
\pgfsetstrokeopacity{0.375357}%
\pgfsetdash{}{0pt}%
\pgfpathmoveto{\pgfqpoint{1.646992in}{2.957300in}}%
\pgfpathcurveto{\pgfqpoint{1.655228in}{2.957300in}}{\pgfqpoint{1.663128in}{2.960572in}}{\pgfqpoint{1.668952in}{2.966396in}}%
\pgfpathcurveto{\pgfqpoint{1.674776in}{2.972220in}}{\pgfqpoint{1.678048in}{2.980120in}}{\pgfqpoint{1.678048in}{2.988356in}}%
\pgfpathcurveto{\pgfqpoint{1.678048in}{2.996593in}}{\pgfqpoint{1.674776in}{3.004493in}}{\pgfqpoint{1.668952in}{3.010317in}}%
\pgfpathcurveto{\pgfqpoint{1.663128in}{3.016141in}}{\pgfqpoint{1.655228in}{3.019413in}}{\pgfqpoint{1.646992in}{3.019413in}}%
\pgfpathcurveto{\pgfqpoint{1.638756in}{3.019413in}}{\pgfqpoint{1.630856in}{3.016141in}}{\pgfqpoint{1.625032in}{3.010317in}}%
\pgfpathcurveto{\pgfqpoint{1.619208in}{3.004493in}}{\pgfqpoint{1.615935in}{2.996593in}}{\pgfqpoint{1.615935in}{2.988356in}}%
\pgfpathcurveto{\pgfqpoint{1.615935in}{2.980120in}}{\pgfqpoint{1.619208in}{2.972220in}}{\pgfqpoint{1.625032in}{2.966396in}}%
\pgfpathcurveto{\pgfqpoint{1.630856in}{2.960572in}}{\pgfqpoint{1.638756in}{2.957300in}}{\pgfqpoint{1.646992in}{2.957300in}}%
\pgfpathclose%
\pgfusepath{stroke,fill}%
\end{pgfscope}%
\begin{pgfscope}%
\pgfpathrectangle{\pgfqpoint{0.100000in}{0.212622in}}{\pgfqpoint{3.696000in}{3.696000in}}%
\pgfusepath{clip}%
\pgfsetbuttcap%
\pgfsetroundjoin%
\definecolor{currentfill}{rgb}{0.121569,0.466667,0.705882}%
\pgfsetfillcolor{currentfill}%
\pgfsetfillopacity{0.376177}%
\pgfsetlinewidth{1.003750pt}%
\definecolor{currentstroke}{rgb}{0.121569,0.466667,0.705882}%
\pgfsetstrokecolor{currentstroke}%
\pgfsetstrokeopacity{0.376177}%
\pgfsetdash{}{0pt}%
\pgfpathmoveto{\pgfqpoint{1.644800in}{2.953297in}}%
\pgfpathcurveto{\pgfqpoint{1.653036in}{2.953297in}}{\pgfqpoint{1.660936in}{2.956569in}}{\pgfqpoint{1.666760in}{2.962393in}}%
\pgfpathcurveto{\pgfqpoint{1.672584in}{2.968217in}}{\pgfqpoint{1.675857in}{2.976117in}}{\pgfqpoint{1.675857in}{2.984353in}}%
\pgfpathcurveto{\pgfqpoint{1.675857in}{2.992590in}}{\pgfqpoint{1.672584in}{3.000490in}}{\pgfqpoint{1.666760in}{3.006314in}}%
\pgfpathcurveto{\pgfqpoint{1.660936in}{3.012138in}}{\pgfqpoint{1.653036in}{3.015410in}}{\pgfqpoint{1.644800in}{3.015410in}}%
\pgfpathcurveto{\pgfqpoint{1.636564in}{3.015410in}}{\pgfqpoint{1.628664in}{3.012138in}}{\pgfqpoint{1.622840in}{3.006314in}}%
\pgfpathcurveto{\pgfqpoint{1.617016in}{3.000490in}}{\pgfqpoint{1.613744in}{2.992590in}}{\pgfqpoint{1.613744in}{2.984353in}}%
\pgfpathcurveto{\pgfqpoint{1.613744in}{2.976117in}}{\pgfqpoint{1.617016in}{2.968217in}}{\pgfqpoint{1.622840in}{2.962393in}}%
\pgfpathcurveto{\pgfqpoint{1.628664in}{2.956569in}}{\pgfqpoint{1.636564in}{2.953297in}}{\pgfqpoint{1.644800in}{2.953297in}}%
\pgfpathclose%
\pgfusepath{stroke,fill}%
\end{pgfscope}%
\begin{pgfscope}%
\pgfpathrectangle{\pgfqpoint{0.100000in}{0.212622in}}{\pgfqpoint{3.696000in}{3.696000in}}%
\pgfusepath{clip}%
\pgfsetbuttcap%
\pgfsetroundjoin%
\definecolor{currentfill}{rgb}{0.121569,0.466667,0.705882}%
\pgfsetfillcolor{currentfill}%
\pgfsetfillopacity{0.376551}%
\pgfsetlinewidth{1.003750pt}%
\definecolor{currentstroke}{rgb}{0.121569,0.466667,0.705882}%
\pgfsetstrokecolor{currentstroke}%
\pgfsetstrokeopacity{0.376551}%
\pgfsetdash{}{0pt}%
\pgfpathmoveto{\pgfqpoint{1.962212in}{3.013436in}}%
\pgfpathcurveto{\pgfqpoint{1.970448in}{3.013436in}}{\pgfqpoint{1.978348in}{3.016708in}}{\pgfqpoint{1.984172in}{3.022532in}}%
\pgfpathcurveto{\pgfqpoint{1.989996in}{3.028356in}}{\pgfqpoint{1.993268in}{3.036256in}}{\pgfqpoint{1.993268in}{3.044493in}}%
\pgfpathcurveto{\pgfqpoint{1.993268in}{3.052729in}}{\pgfqpoint{1.989996in}{3.060629in}}{\pgfqpoint{1.984172in}{3.066453in}}%
\pgfpathcurveto{\pgfqpoint{1.978348in}{3.072277in}}{\pgfqpoint{1.970448in}{3.075549in}}{\pgfqpoint{1.962212in}{3.075549in}}%
\pgfpathcurveto{\pgfqpoint{1.953976in}{3.075549in}}{\pgfqpoint{1.946075in}{3.072277in}}{\pgfqpoint{1.940252in}{3.066453in}}%
\pgfpathcurveto{\pgfqpoint{1.934428in}{3.060629in}}{\pgfqpoint{1.931155in}{3.052729in}}{\pgfqpoint{1.931155in}{3.044493in}}%
\pgfpathcurveto{\pgfqpoint{1.931155in}{3.036256in}}{\pgfqpoint{1.934428in}{3.028356in}}{\pgfqpoint{1.940252in}{3.022532in}}%
\pgfpathcurveto{\pgfqpoint{1.946075in}{3.016708in}}{\pgfqpoint{1.953976in}{3.013436in}}{\pgfqpoint{1.962212in}{3.013436in}}%
\pgfpathclose%
\pgfusepath{stroke,fill}%
\end{pgfscope}%
\begin{pgfscope}%
\pgfpathrectangle{\pgfqpoint{0.100000in}{0.212622in}}{\pgfqpoint{3.696000in}{3.696000in}}%
\pgfusepath{clip}%
\pgfsetbuttcap%
\pgfsetroundjoin%
\definecolor{currentfill}{rgb}{0.121569,0.466667,0.705882}%
\pgfsetfillcolor{currentfill}%
\pgfsetfillopacity{0.376840}%
\pgfsetlinewidth{1.003750pt}%
\definecolor{currentstroke}{rgb}{0.121569,0.466667,0.705882}%
\pgfsetstrokecolor{currentstroke}%
\pgfsetstrokeopacity{0.376840}%
\pgfsetdash{}{0pt}%
\pgfpathmoveto{\pgfqpoint{1.642520in}{2.949696in}}%
\pgfpathcurveto{\pgfqpoint{1.650757in}{2.949696in}}{\pgfqpoint{1.658657in}{2.952968in}}{\pgfqpoint{1.664481in}{2.958792in}}%
\pgfpathcurveto{\pgfqpoint{1.670305in}{2.964616in}}{\pgfqpoint{1.673577in}{2.972516in}}{\pgfqpoint{1.673577in}{2.980753in}}%
\pgfpathcurveto{\pgfqpoint{1.673577in}{2.988989in}}{\pgfqpoint{1.670305in}{2.996889in}}{\pgfqpoint{1.664481in}{3.002713in}}%
\pgfpathcurveto{\pgfqpoint{1.658657in}{3.008537in}}{\pgfqpoint{1.650757in}{3.011809in}}{\pgfqpoint{1.642520in}{3.011809in}}%
\pgfpathcurveto{\pgfqpoint{1.634284in}{3.011809in}}{\pgfqpoint{1.626384in}{3.008537in}}{\pgfqpoint{1.620560in}{3.002713in}}%
\pgfpathcurveto{\pgfqpoint{1.614736in}{2.996889in}}{\pgfqpoint{1.611464in}{2.988989in}}{\pgfqpoint{1.611464in}{2.980753in}}%
\pgfpathcurveto{\pgfqpoint{1.611464in}{2.972516in}}{\pgfqpoint{1.614736in}{2.964616in}}{\pgfqpoint{1.620560in}{2.958792in}}%
\pgfpathcurveto{\pgfqpoint{1.626384in}{2.952968in}}{\pgfqpoint{1.634284in}{2.949696in}}{\pgfqpoint{1.642520in}{2.949696in}}%
\pgfpathclose%
\pgfusepath{stroke,fill}%
\end{pgfscope}%
\begin{pgfscope}%
\pgfpathrectangle{\pgfqpoint{0.100000in}{0.212622in}}{\pgfqpoint{3.696000in}{3.696000in}}%
\pgfusepath{clip}%
\pgfsetbuttcap%
\pgfsetroundjoin%
\definecolor{currentfill}{rgb}{0.121569,0.466667,0.705882}%
\pgfsetfillcolor{currentfill}%
\pgfsetfillopacity{0.377235}%
\pgfsetlinewidth{1.003750pt}%
\definecolor{currentstroke}{rgb}{0.121569,0.466667,0.705882}%
\pgfsetstrokecolor{currentstroke}%
\pgfsetstrokeopacity{0.377235}%
\pgfsetdash{}{0pt}%
\pgfpathmoveto{\pgfqpoint{1.641500in}{2.947577in}}%
\pgfpathcurveto{\pgfqpoint{1.649736in}{2.947577in}}{\pgfqpoint{1.657636in}{2.950849in}}{\pgfqpoint{1.663460in}{2.956673in}}%
\pgfpathcurveto{\pgfqpoint{1.669284in}{2.962497in}}{\pgfqpoint{1.672556in}{2.970397in}}{\pgfqpoint{1.672556in}{2.978634in}}%
\pgfpathcurveto{\pgfqpoint{1.672556in}{2.986870in}}{\pgfqpoint{1.669284in}{2.994770in}}{\pgfqpoint{1.663460in}{3.000594in}}%
\pgfpathcurveto{\pgfqpoint{1.657636in}{3.006418in}}{\pgfqpoint{1.649736in}{3.009690in}}{\pgfqpoint{1.641500in}{3.009690in}}%
\pgfpathcurveto{\pgfqpoint{1.633263in}{3.009690in}}{\pgfqpoint{1.625363in}{3.006418in}}{\pgfqpoint{1.619539in}{3.000594in}}%
\pgfpathcurveto{\pgfqpoint{1.613716in}{2.994770in}}{\pgfqpoint{1.610443in}{2.986870in}}{\pgfqpoint{1.610443in}{2.978634in}}%
\pgfpathcurveto{\pgfqpoint{1.610443in}{2.970397in}}{\pgfqpoint{1.613716in}{2.962497in}}{\pgfqpoint{1.619539in}{2.956673in}}%
\pgfpathcurveto{\pgfqpoint{1.625363in}{2.950849in}}{\pgfqpoint{1.633263in}{2.947577in}}{\pgfqpoint{1.641500in}{2.947577in}}%
\pgfpathclose%
\pgfusepath{stroke,fill}%
\end{pgfscope}%
\begin{pgfscope}%
\pgfpathrectangle{\pgfqpoint{0.100000in}{0.212622in}}{\pgfqpoint{3.696000in}{3.696000in}}%
\pgfusepath{clip}%
\pgfsetbuttcap%
\pgfsetroundjoin%
\definecolor{currentfill}{rgb}{0.121569,0.466667,0.705882}%
\pgfsetfillcolor{currentfill}%
\pgfsetfillopacity{0.377947}%
\pgfsetlinewidth{1.003750pt}%
\definecolor{currentstroke}{rgb}{0.121569,0.466667,0.705882}%
\pgfsetstrokecolor{currentstroke}%
\pgfsetstrokeopacity{0.377947}%
\pgfsetdash{}{0pt}%
\pgfpathmoveto{\pgfqpoint{1.639383in}{2.944025in}}%
\pgfpathcurveto{\pgfqpoint{1.647619in}{2.944025in}}{\pgfqpoint{1.655519in}{2.947297in}}{\pgfqpoint{1.661343in}{2.953121in}}%
\pgfpathcurveto{\pgfqpoint{1.667167in}{2.958945in}}{\pgfqpoint{1.670439in}{2.966845in}}{\pgfqpoint{1.670439in}{2.975081in}}%
\pgfpathcurveto{\pgfqpoint{1.670439in}{2.983318in}}{\pgfqpoint{1.667167in}{2.991218in}}{\pgfqpoint{1.661343in}{2.997042in}}%
\pgfpathcurveto{\pgfqpoint{1.655519in}{3.002865in}}{\pgfqpoint{1.647619in}{3.006138in}}{\pgfqpoint{1.639383in}{3.006138in}}%
\pgfpathcurveto{\pgfqpoint{1.631146in}{3.006138in}}{\pgfqpoint{1.623246in}{3.002865in}}{\pgfqpoint{1.617422in}{2.997042in}}%
\pgfpathcurveto{\pgfqpoint{1.611598in}{2.991218in}}{\pgfqpoint{1.608326in}{2.983318in}}{\pgfqpoint{1.608326in}{2.975081in}}%
\pgfpathcurveto{\pgfqpoint{1.608326in}{2.966845in}}{\pgfqpoint{1.611598in}{2.958945in}}{\pgfqpoint{1.617422in}{2.953121in}}%
\pgfpathcurveto{\pgfqpoint{1.623246in}{2.947297in}}{\pgfqpoint{1.631146in}{2.944025in}}{\pgfqpoint{1.639383in}{2.944025in}}%
\pgfpathclose%
\pgfusepath{stroke,fill}%
\end{pgfscope}%
\begin{pgfscope}%
\pgfpathrectangle{\pgfqpoint{0.100000in}{0.212622in}}{\pgfqpoint{3.696000in}{3.696000in}}%
\pgfusepath{clip}%
\pgfsetbuttcap%
\pgfsetroundjoin%
\definecolor{currentfill}{rgb}{0.121569,0.466667,0.705882}%
\pgfsetfillcolor{currentfill}%
\pgfsetfillopacity{0.379201}%
\pgfsetlinewidth{1.003750pt}%
\definecolor{currentstroke}{rgb}{0.121569,0.466667,0.705882}%
\pgfsetstrokecolor{currentstroke}%
\pgfsetstrokeopacity{0.379201}%
\pgfsetdash{}{0pt}%
\pgfpathmoveto{\pgfqpoint{1.635444in}{2.937532in}}%
\pgfpathcurveto{\pgfqpoint{1.643680in}{2.937532in}}{\pgfqpoint{1.651580in}{2.940804in}}{\pgfqpoint{1.657404in}{2.946628in}}%
\pgfpathcurveto{\pgfqpoint{1.663228in}{2.952452in}}{\pgfqpoint{1.666501in}{2.960352in}}{\pgfqpoint{1.666501in}{2.968588in}}%
\pgfpathcurveto{\pgfqpoint{1.666501in}{2.976825in}}{\pgfqpoint{1.663228in}{2.984725in}}{\pgfqpoint{1.657404in}{2.990549in}}%
\pgfpathcurveto{\pgfqpoint{1.651580in}{2.996373in}}{\pgfqpoint{1.643680in}{2.999645in}}{\pgfqpoint{1.635444in}{2.999645in}}%
\pgfpathcurveto{\pgfqpoint{1.627208in}{2.999645in}}{\pgfqpoint{1.619308in}{2.996373in}}{\pgfqpoint{1.613484in}{2.990549in}}%
\pgfpathcurveto{\pgfqpoint{1.607660in}{2.984725in}}{\pgfqpoint{1.604388in}{2.976825in}}{\pgfqpoint{1.604388in}{2.968588in}}%
\pgfpathcurveto{\pgfqpoint{1.604388in}{2.960352in}}{\pgfqpoint{1.607660in}{2.952452in}}{\pgfqpoint{1.613484in}{2.946628in}}%
\pgfpathcurveto{\pgfqpoint{1.619308in}{2.940804in}}{\pgfqpoint{1.627208in}{2.937532in}}{\pgfqpoint{1.635444in}{2.937532in}}%
\pgfpathclose%
\pgfusepath{stroke,fill}%
\end{pgfscope}%
\begin{pgfscope}%
\pgfpathrectangle{\pgfqpoint{0.100000in}{0.212622in}}{\pgfqpoint{3.696000in}{3.696000in}}%
\pgfusepath{clip}%
\pgfsetbuttcap%
\pgfsetroundjoin%
\definecolor{currentfill}{rgb}{0.121569,0.466667,0.705882}%
\pgfsetfillcolor{currentfill}%
\pgfsetfillopacity{0.379252}%
\pgfsetlinewidth{1.003750pt}%
\definecolor{currentstroke}{rgb}{0.121569,0.466667,0.705882}%
\pgfsetstrokecolor{currentstroke}%
\pgfsetstrokeopacity{0.379252}%
\pgfsetdash{}{0pt}%
\pgfpathmoveto{\pgfqpoint{1.964517in}{3.004198in}}%
\pgfpathcurveto{\pgfqpoint{1.972753in}{3.004198in}}{\pgfqpoint{1.980653in}{3.007470in}}{\pgfqpoint{1.986477in}{3.013294in}}%
\pgfpathcurveto{\pgfqpoint{1.992301in}{3.019118in}}{\pgfqpoint{1.995573in}{3.027018in}}{\pgfqpoint{1.995573in}{3.035254in}}%
\pgfpathcurveto{\pgfqpoint{1.995573in}{3.043490in}}{\pgfqpoint{1.992301in}{3.051390in}}{\pgfqpoint{1.986477in}{3.057214in}}%
\pgfpathcurveto{\pgfqpoint{1.980653in}{3.063038in}}{\pgfqpoint{1.972753in}{3.066311in}}{\pgfqpoint{1.964517in}{3.066311in}}%
\pgfpathcurveto{\pgfqpoint{1.956281in}{3.066311in}}{\pgfqpoint{1.948381in}{3.063038in}}{\pgfqpoint{1.942557in}{3.057214in}}%
\pgfpathcurveto{\pgfqpoint{1.936733in}{3.051390in}}{\pgfqpoint{1.933460in}{3.043490in}}{\pgfqpoint{1.933460in}{3.035254in}}%
\pgfpathcurveto{\pgfqpoint{1.933460in}{3.027018in}}{\pgfqpoint{1.936733in}{3.019118in}}{\pgfqpoint{1.942557in}{3.013294in}}%
\pgfpathcurveto{\pgfqpoint{1.948381in}{3.007470in}}{\pgfqpoint{1.956281in}{3.004198in}}{\pgfqpoint{1.964517in}{3.004198in}}%
\pgfpathclose%
\pgfusepath{stroke,fill}%
\end{pgfscope}%
\begin{pgfscope}%
\pgfpathrectangle{\pgfqpoint{0.100000in}{0.212622in}}{\pgfqpoint{3.696000in}{3.696000in}}%
\pgfusepath{clip}%
\pgfsetbuttcap%
\pgfsetroundjoin%
\definecolor{currentfill}{rgb}{0.121569,0.466667,0.705882}%
\pgfsetfillcolor{currentfill}%
\pgfsetfillopacity{0.380377}%
\pgfsetlinewidth{1.003750pt}%
\definecolor{currentstroke}{rgb}{0.121569,0.466667,0.705882}%
\pgfsetstrokecolor{currentstroke}%
\pgfsetstrokeopacity{0.380377}%
\pgfsetdash{}{0pt}%
\pgfpathmoveto{\pgfqpoint{1.632464in}{2.931495in}}%
\pgfpathcurveto{\pgfqpoint{1.640700in}{2.931495in}}{\pgfqpoint{1.648600in}{2.934767in}}{\pgfqpoint{1.654424in}{2.940591in}}%
\pgfpathcurveto{\pgfqpoint{1.660248in}{2.946415in}}{\pgfqpoint{1.663520in}{2.954315in}}{\pgfqpoint{1.663520in}{2.962552in}}%
\pgfpathcurveto{\pgfqpoint{1.663520in}{2.970788in}}{\pgfqpoint{1.660248in}{2.978688in}}{\pgfqpoint{1.654424in}{2.984512in}}%
\pgfpathcurveto{\pgfqpoint{1.648600in}{2.990336in}}{\pgfqpoint{1.640700in}{2.993608in}}{\pgfqpoint{1.632464in}{2.993608in}}%
\pgfpathcurveto{\pgfqpoint{1.624227in}{2.993608in}}{\pgfqpoint{1.616327in}{2.990336in}}{\pgfqpoint{1.610503in}{2.984512in}}%
\pgfpathcurveto{\pgfqpoint{1.604679in}{2.978688in}}{\pgfqpoint{1.601407in}{2.970788in}}{\pgfqpoint{1.601407in}{2.962552in}}%
\pgfpathcurveto{\pgfqpoint{1.601407in}{2.954315in}}{\pgfqpoint{1.604679in}{2.946415in}}{\pgfqpoint{1.610503in}{2.940591in}}%
\pgfpathcurveto{\pgfqpoint{1.616327in}{2.934767in}}{\pgfqpoint{1.624227in}{2.931495in}}{\pgfqpoint{1.632464in}{2.931495in}}%
\pgfpathclose%
\pgfusepath{stroke,fill}%
\end{pgfscope}%
\begin{pgfscope}%
\pgfpathrectangle{\pgfqpoint{0.100000in}{0.212622in}}{\pgfqpoint{3.696000in}{3.696000in}}%
\pgfusepath{clip}%
\pgfsetbuttcap%
\pgfsetroundjoin%
\definecolor{currentfill}{rgb}{0.121569,0.466667,0.705882}%
\pgfsetfillcolor{currentfill}%
\pgfsetfillopacity{0.380695}%
\pgfsetlinewidth{1.003750pt}%
\definecolor{currentstroke}{rgb}{0.121569,0.466667,0.705882}%
\pgfsetstrokecolor{currentstroke}%
\pgfsetstrokeopacity{0.380695}%
\pgfsetdash{}{0pt}%
\pgfpathmoveto{\pgfqpoint{1.965630in}{2.998785in}}%
\pgfpathcurveto{\pgfqpoint{1.973866in}{2.998785in}}{\pgfqpoint{1.981766in}{3.002057in}}{\pgfqpoint{1.987590in}{3.007881in}}%
\pgfpathcurveto{\pgfqpoint{1.993414in}{3.013705in}}{\pgfqpoint{1.996686in}{3.021605in}}{\pgfqpoint{1.996686in}{3.029841in}}%
\pgfpathcurveto{\pgfqpoint{1.996686in}{3.038078in}}{\pgfqpoint{1.993414in}{3.045978in}}{\pgfqpoint{1.987590in}{3.051802in}}%
\pgfpathcurveto{\pgfqpoint{1.981766in}{3.057626in}}{\pgfqpoint{1.973866in}{3.060898in}}{\pgfqpoint{1.965630in}{3.060898in}}%
\pgfpathcurveto{\pgfqpoint{1.957394in}{3.060898in}}{\pgfqpoint{1.949494in}{3.057626in}}{\pgfqpoint{1.943670in}{3.051802in}}%
\pgfpathcurveto{\pgfqpoint{1.937846in}{3.045978in}}{\pgfqpoint{1.934573in}{3.038078in}}{\pgfqpoint{1.934573in}{3.029841in}}%
\pgfpathcurveto{\pgfqpoint{1.934573in}{3.021605in}}{\pgfqpoint{1.937846in}{3.013705in}}{\pgfqpoint{1.943670in}{3.007881in}}%
\pgfpathcurveto{\pgfqpoint{1.949494in}{3.002057in}}{\pgfqpoint{1.957394in}{2.998785in}}{\pgfqpoint{1.965630in}{2.998785in}}%
\pgfpathclose%
\pgfusepath{stroke,fill}%
\end{pgfscope}%
\begin{pgfscope}%
\pgfpathrectangle{\pgfqpoint{0.100000in}{0.212622in}}{\pgfqpoint{3.696000in}{3.696000in}}%
\pgfusepath{clip}%
\pgfsetbuttcap%
\pgfsetroundjoin%
\definecolor{currentfill}{rgb}{0.121569,0.466667,0.705882}%
\pgfsetfillcolor{currentfill}%
\pgfsetfillopacity{0.380929}%
\pgfsetlinewidth{1.003750pt}%
\definecolor{currentstroke}{rgb}{0.121569,0.466667,0.705882}%
\pgfsetstrokecolor{currentstroke}%
\pgfsetstrokeopacity{0.380929}%
\pgfsetdash{}{0pt}%
\pgfpathmoveto{\pgfqpoint{1.630707in}{2.928735in}}%
\pgfpathcurveto{\pgfqpoint{1.638943in}{2.928735in}}{\pgfqpoint{1.646843in}{2.932007in}}{\pgfqpoint{1.652667in}{2.937831in}}%
\pgfpathcurveto{\pgfqpoint{1.658491in}{2.943655in}}{\pgfqpoint{1.661763in}{2.951555in}}{\pgfqpoint{1.661763in}{2.959792in}}%
\pgfpathcurveto{\pgfqpoint{1.661763in}{2.968028in}}{\pgfqpoint{1.658491in}{2.975928in}}{\pgfqpoint{1.652667in}{2.981752in}}%
\pgfpathcurveto{\pgfqpoint{1.646843in}{2.987576in}}{\pgfqpoint{1.638943in}{2.990848in}}{\pgfqpoint{1.630707in}{2.990848in}}%
\pgfpathcurveto{\pgfqpoint{1.622470in}{2.990848in}}{\pgfqpoint{1.614570in}{2.987576in}}{\pgfqpoint{1.608746in}{2.981752in}}%
\pgfpathcurveto{\pgfqpoint{1.602922in}{2.975928in}}{\pgfqpoint{1.599650in}{2.968028in}}{\pgfqpoint{1.599650in}{2.959792in}}%
\pgfpathcurveto{\pgfqpoint{1.599650in}{2.951555in}}{\pgfqpoint{1.602922in}{2.943655in}}{\pgfqpoint{1.608746in}{2.937831in}}%
\pgfpathcurveto{\pgfqpoint{1.614570in}{2.932007in}}{\pgfqpoint{1.622470in}{2.928735in}}{\pgfqpoint{1.630707in}{2.928735in}}%
\pgfpathclose%
\pgfusepath{stroke,fill}%
\end{pgfscope}%
\begin{pgfscope}%
\pgfpathrectangle{\pgfqpoint{0.100000in}{0.212622in}}{\pgfqpoint{3.696000in}{3.696000in}}%
\pgfusepath{clip}%
\pgfsetbuttcap%
\pgfsetroundjoin%
\definecolor{currentfill}{rgb}{0.121569,0.466667,0.705882}%
\pgfsetfillcolor{currentfill}%
\pgfsetfillopacity{0.381438}%
\pgfsetlinewidth{1.003750pt}%
\definecolor{currentstroke}{rgb}{0.121569,0.466667,0.705882}%
\pgfsetstrokecolor{currentstroke}%
\pgfsetstrokeopacity{0.381438}%
\pgfsetdash{}{0pt}%
\pgfpathmoveto{\pgfqpoint{1.629373in}{2.926254in}}%
\pgfpathcurveto{\pgfqpoint{1.637610in}{2.926254in}}{\pgfqpoint{1.645510in}{2.929526in}}{\pgfqpoint{1.651334in}{2.935350in}}%
\pgfpathcurveto{\pgfqpoint{1.657157in}{2.941174in}}{\pgfqpoint{1.660430in}{2.949074in}}{\pgfqpoint{1.660430in}{2.957310in}}%
\pgfpathcurveto{\pgfqpoint{1.660430in}{2.965547in}}{\pgfqpoint{1.657157in}{2.973447in}}{\pgfqpoint{1.651334in}{2.979271in}}%
\pgfpathcurveto{\pgfqpoint{1.645510in}{2.985094in}}{\pgfqpoint{1.637610in}{2.988367in}}{\pgfqpoint{1.629373in}{2.988367in}}%
\pgfpathcurveto{\pgfqpoint{1.621137in}{2.988367in}}{\pgfqpoint{1.613237in}{2.985094in}}{\pgfqpoint{1.607413in}{2.979271in}}%
\pgfpathcurveto{\pgfqpoint{1.601589in}{2.973447in}}{\pgfqpoint{1.598317in}{2.965547in}}{\pgfqpoint{1.598317in}{2.957310in}}%
\pgfpathcurveto{\pgfqpoint{1.598317in}{2.949074in}}{\pgfqpoint{1.601589in}{2.941174in}}{\pgfqpoint{1.607413in}{2.935350in}}%
\pgfpathcurveto{\pgfqpoint{1.613237in}{2.929526in}}{\pgfqpoint{1.621137in}{2.926254in}}{\pgfqpoint{1.629373in}{2.926254in}}%
\pgfpathclose%
\pgfusepath{stroke,fill}%
\end{pgfscope}%
\begin{pgfscope}%
\pgfpathrectangle{\pgfqpoint{0.100000in}{0.212622in}}{\pgfqpoint{3.696000in}{3.696000in}}%
\pgfusepath{clip}%
\pgfsetbuttcap%
\pgfsetroundjoin%
\definecolor{currentfill}{rgb}{0.121569,0.466667,0.705882}%
\pgfsetfillcolor{currentfill}%
\pgfsetfillopacity{0.381574}%
\pgfsetlinewidth{1.003750pt}%
\definecolor{currentstroke}{rgb}{0.121569,0.466667,0.705882}%
\pgfsetstrokecolor{currentstroke}%
\pgfsetstrokeopacity{0.381574}%
\pgfsetdash{}{0pt}%
\pgfpathmoveto{\pgfqpoint{1.966025in}{2.995957in}}%
\pgfpathcurveto{\pgfqpoint{1.974261in}{2.995957in}}{\pgfqpoint{1.982161in}{2.999230in}}{\pgfqpoint{1.987985in}{3.005054in}}%
\pgfpathcurveto{\pgfqpoint{1.993809in}{3.010878in}}{\pgfqpoint{1.997081in}{3.018778in}}{\pgfqpoint{1.997081in}{3.027014in}}%
\pgfpathcurveto{\pgfqpoint{1.997081in}{3.035250in}}{\pgfqpoint{1.993809in}{3.043150in}}{\pgfqpoint{1.987985in}{3.048974in}}%
\pgfpathcurveto{\pgfqpoint{1.982161in}{3.054798in}}{\pgfqpoint{1.974261in}{3.058070in}}{\pgfqpoint{1.966025in}{3.058070in}}%
\pgfpathcurveto{\pgfqpoint{1.957788in}{3.058070in}}{\pgfqpoint{1.949888in}{3.054798in}}{\pgfqpoint{1.944064in}{3.048974in}}%
\pgfpathcurveto{\pgfqpoint{1.938241in}{3.043150in}}{\pgfqpoint{1.934968in}{3.035250in}}{\pgfqpoint{1.934968in}{3.027014in}}%
\pgfpathcurveto{\pgfqpoint{1.934968in}{3.018778in}}{\pgfqpoint{1.938241in}{3.010878in}}{\pgfqpoint{1.944064in}{3.005054in}}%
\pgfpathcurveto{\pgfqpoint{1.949888in}{2.999230in}}{\pgfqpoint{1.957788in}{2.995957in}}{\pgfqpoint{1.966025in}{2.995957in}}%
\pgfpathclose%
\pgfusepath{stroke,fill}%
\end{pgfscope}%
\begin{pgfscope}%
\pgfpathrectangle{\pgfqpoint{0.100000in}{0.212622in}}{\pgfqpoint{3.696000in}{3.696000in}}%
\pgfusepath{clip}%
\pgfsetbuttcap%
\pgfsetroundjoin%
\definecolor{currentfill}{rgb}{0.121569,0.466667,0.705882}%
\pgfsetfillcolor{currentfill}%
\pgfsetfillopacity{0.382379}%
\pgfsetlinewidth{1.003750pt}%
\definecolor{currentstroke}{rgb}{0.121569,0.466667,0.705882}%
\pgfsetstrokecolor{currentstroke}%
\pgfsetstrokeopacity{0.382379}%
\pgfsetdash{}{0pt}%
\pgfpathmoveto{\pgfqpoint{1.626987in}{2.921756in}}%
\pgfpathcurveto{\pgfqpoint{1.635224in}{2.921756in}}{\pgfqpoint{1.643124in}{2.925028in}}{\pgfqpoint{1.648948in}{2.930852in}}%
\pgfpathcurveto{\pgfqpoint{1.654771in}{2.936676in}}{\pgfqpoint{1.658044in}{2.944576in}}{\pgfqpoint{1.658044in}{2.952812in}}%
\pgfpathcurveto{\pgfqpoint{1.658044in}{2.961049in}}{\pgfqpoint{1.654771in}{2.968949in}}{\pgfqpoint{1.648948in}{2.974773in}}%
\pgfpathcurveto{\pgfqpoint{1.643124in}{2.980596in}}{\pgfqpoint{1.635224in}{2.983869in}}{\pgfqpoint{1.626987in}{2.983869in}}%
\pgfpathcurveto{\pgfqpoint{1.618751in}{2.983869in}}{\pgfqpoint{1.610851in}{2.980596in}}{\pgfqpoint{1.605027in}{2.974773in}}%
\pgfpathcurveto{\pgfqpoint{1.599203in}{2.968949in}}{\pgfqpoint{1.595931in}{2.961049in}}{\pgfqpoint{1.595931in}{2.952812in}}%
\pgfpathcurveto{\pgfqpoint{1.595931in}{2.944576in}}{\pgfqpoint{1.599203in}{2.936676in}}{\pgfqpoint{1.605027in}{2.930852in}}%
\pgfpathcurveto{\pgfqpoint{1.610851in}{2.925028in}}{\pgfqpoint{1.618751in}{2.921756in}}{\pgfqpoint{1.626987in}{2.921756in}}%
\pgfpathclose%
\pgfusepath{stroke,fill}%
\end{pgfscope}%
\begin{pgfscope}%
\pgfpathrectangle{\pgfqpoint{0.100000in}{0.212622in}}{\pgfqpoint{3.696000in}{3.696000in}}%
\pgfusepath{clip}%
\pgfsetbuttcap%
\pgfsetroundjoin%
\definecolor{currentfill}{rgb}{0.121569,0.466667,0.705882}%
\pgfsetfillcolor{currentfill}%
\pgfsetfillopacity{0.382827}%
\pgfsetlinewidth{1.003750pt}%
\definecolor{currentstroke}{rgb}{0.121569,0.466667,0.705882}%
\pgfsetstrokecolor{currentstroke}%
\pgfsetstrokeopacity{0.382827}%
\pgfsetdash{}{0pt}%
\pgfpathmoveto{\pgfqpoint{1.966944in}{2.991470in}}%
\pgfpathcurveto{\pgfqpoint{1.975180in}{2.991470in}}{\pgfqpoint{1.983080in}{2.994742in}}{\pgfqpoint{1.988904in}{3.000566in}}%
\pgfpathcurveto{\pgfqpoint{1.994728in}{3.006390in}}{\pgfqpoint{1.998001in}{3.014290in}}{\pgfqpoint{1.998001in}{3.022526in}}%
\pgfpathcurveto{\pgfqpoint{1.998001in}{3.030763in}}{\pgfqpoint{1.994728in}{3.038663in}}{\pgfqpoint{1.988904in}{3.044487in}}%
\pgfpathcurveto{\pgfqpoint{1.983080in}{3.050311in}}{\pgfqpoint{1.975180in}{3.053583in}}{\pgfqpoint{1.966944in}{3.053583in}}%
\pgfpathcurveto{\pgfqpoint{1.958708in}{3.053583in}}{\pgfqpoint{1.950808in}{3.050311in}}{\pgfqpoint{1.944984in}{3.044487in}}%
\pgfpathcurveto{\pgfqpoint{1.939160in}{3.038663in}}{\pgfqpoint{1.935888in}{3.030763in}}{\pgfqpoint{1.935888in}{3.022526in}}%
\pgfpathcurveto{\pgfqpoint{1.935888in}{3.014290in}}{\pgfqpoint{1.939160in}{3.006390in}}{\pgfqpoint{1.944984in}{3.000566in}}%
\pgfpathcurveto{\pgfqpoint{1.950808in}{2.994742in}}{\pgfqpoint{1.958708in}{2.991470in}}{\pgfqpoint{1.966944in}{2.991470in}}%
\pgfpathclose%
\pgfusepath{stroke,fill}%
\end{pgfscope}%
\begin{pgfscope}%
\pgfpathrectangle{\pgfqpoint{0.100000in}{0.212622in}}{\pgfqpoint{3.696000in}{3.696000in}}%
\pgfusepath{clip}%
\pgfsetbuttcap%
\pgfsetroundjoin%
\definecolor{currentfill}{rgb}{0.121569,0.466667,0.705882}%
\pgfsetfillcolor{currentfill}%
\pgfsetfillopacity{0.383030}%
\pgfsetlinewidth{1.003750pt}%
\definecolor{currentstroke}{rgb}{0.121569,0.466667,0.705882}%
\pgfsetstrokecolor{currentstroke}%
\pgfsetstrokeopacity{0.383030}%
\pgfsetdash{}{0pt}%
\pgfpathmoveto{\pgfqpoint{1.624798in}{2.918357in}}%
\pgfpathcurveto{\pgfqpoint{1.633034in}{2.918357in}}{\pgfqpoint{1.640934in}{2.921629in}}{\pgfqpoint{1.646758in}{2.927453in}}%
\pgfpathcurveto{\pgfqpoint{1.652582in}{2.933277in}}{\pgfqpoint{1.655854in}{2.941177in}}{\pgfqpoint{1.655854in}{2.949414in}}%
\pgfpathcurveto{\pgfqpoint{1.655854in}{2.957650in}}{\pgfqpoint{1.652582in}{2.965550in}}{\pgfqpoint{1.646758in}{2.971374in}}%
\pgfpathcurveto{\pgfqpoint{1.640934in}{2.977198in}}{\pgfqpoint{1.633034in}{2.980470in}}{\pgfqpoint{1.624798in}{2.980470in}}%
\pgfpathcurveto{\pgfqpoint{1.616561in}{2.980470in}}{\pgfqpoint{1.608661in}{2.977198in}}{\pgfqpoint{1.602837in}{2.971374in}}%
\pgfpathcurveto{\pgfqpoint{1.597013in}{2.965550in}}{\pgfqpoint{1.593741in}{2.957650in}}{\pgfqpoint{1.593741in}{2.949414in}}%
\pgfpathcurveto{\pgfqpoint{1.593741in}{2.941177in}}{\pgfqpoint{1.597013in}{2.933277in}}{\pgfqpoint{1.602837in}{2.927453in}}%
\pgfpathcurveto{\pgfqpoint{1.608661in}{2.921629in}}{\pgfqpoint{1.616561in}{2.918357in}}{\pgfqpoint{1.624798in}{2.918357in}}%
\pgfpathclose%
\pgfusepath{stroke,fill}%
\end{pgfscope}%
\begin{pgfscope}%
\pgfpathrectangle{\pgfqpoint{0.100000in}{0.212622in}}{\pgfqpoint{3.696000in}{3.696000in}}%
\pgfusepath{clip}%
\pgfsetbuttcap%
\pgfsetroundjoin%
\definecolor{currentfill}{rgb}{0.121569,0.466667,0.705882}%
\pgfsetfillcolor{currentfill}%
\pgfsetfillopacity{0.383343}%
\pgfsetlinewidth{1.003750pt}%
\definecolor{currentstroke}{rgb}{0.121569,0.466667,0.705882}%
\pgfsetstrokecolor{currentstroke}%
\pgfsetstrokeopacity{0.383343}%
\pgfsetdash{}{0pt}%
\pgfpathmoveto{\pgfqpoint{1.623986in}{2.916782in}}%
\pgfpathcurveto{\pgfqpoint{1.632222in}{2.916782in}}{\pgfqpoint{1.640122in}{2.920054in}}{\pgfqpoint{1.645946in}{2.925878in}}%
\pgfpathcurveto{\pgfqpoint{1.651770in}{2.931702in}}{\pgfqpoint{1.655042in}{2.939602in}}{\pgfqpoint{1.655042in}{2.947839in}}%
\pgfpathcurveto{\pgfqpoint{1.655042in}{2.956075in}}{\pgfqpoint{1.651770in}{2.963975in}}{\pgfqpoint{1.645946in}{2.969799in}}%
\pgfpathcurveto{\pgfqpoint{1.640122in}{2.975623in}}{\pgfqpoint{1.632222in}{2.978895in}}{\pgfqpoint{1.623986in}{2.978895in}}%
\pgfpathcurveto{\pgfqpoint{1.615749in}{2.978895in}}{\pgfqpoint{1.607849in}{2.975623in}}{\pgfqpoint{1.602025in}{2.969799in}}%
\pgfpathcurveto{\pgfqpoint{1.596201in}{2.963975in}}{\pgfqpoint{1.592929in}{2.956075in}}{\pgfqpoint{1.592929in}{2.947839in}}%
\pgfpathcurveto{\pgfqpoint{1.592929in}{2.939602in}}{\pgfqpoint{1.596201in}{2.931702in}}{\pgfqpoint{1.602025in}{2.925878in}}%
\pgfpathcurveto{\pgfqpoint{1.607849in}{2.920054in}}{\pgfqpoint{1.615749in}{2.916782in}}{\pgfqpoint{1.623986in}{2.916782in}}%
\pgfpathclose%
\pgfusepath{stroke,fill}%
\end{pgfscope}%
\begin{pgfscope}%
\pgfpathrectangle{\pgfqpoint{0.100000in}{0.212622in}}{\pgfqpoint{3.696000in}{3.696000in}}%
\pgfusepath{clip}%
\pgfsetbuttcap%
\pgfsetroundjoin%
\definecolor{currentfill}{rgb}{0.121569,0.466667,0.705882}%
\pgfsetfillcolor{currentfill}%
\pgfsetfillopacity{0.383587}%
\pgfsetlinewidth{1.003750pt}%
\definecolor{currentstroke}{rgb}{0.121569,0.466667,0.705882}%
\pgfsetstrokecolor{currentstroke}%
\pgfsetstrokeopacity{0.383587}%
\pgfsetdash{}{0pt}%
\pgfpathmoveto{\pgfqpoint{1.623297in}{2.915589in}}%
\pgfpathcurveto{\pgfqpoint{1.631533in}{2.915589in}}{\pgfqpoint{1.639433in}{2.918861in}}{\pgfqpoint{1.645257in}{2.924685in}}%
\pgfpathcurveto{\pgfqpoint{1.651081in}{2.930509in}}{\pgfqpoint{1.654354in}{2.938409in}}{\pgfqpoint{1.654354in}{2.946645in}}%
\pgfpathcurveto{\pgfqpoint{1.654354in}{2.954881in}}{\pgfqpoint{1.651081in}{2.962781in}}{\pgfqpoint{1.645257in}{2.968605in}}%
\pgfpathcurveto{\pgfqpoint{1.639433in}{2.974429in}}{\pgfqpoint{1.631533in}{2.977702in}}{\pgfqpoint{1.623297in}{2.977702in}}%
\pgfpathcurveto{\pgfqpoint{1.615061in}{2.977702in}}{\pgfqpoint{1.607161in}{2.974429in}}{\pgfqpoint{1.601337in}{2.968605in}}%
\pgfpathcurveto{\pgfqpoint{1.595513in}{2.962781in}}{\pgfqpoint{1.592241in}{2.954881in}}{\pgfqpoint{1.592241in}{2.946645in}}%
\pgfpathcurveto{\pgfqpoint{1.592241in}{2.938409in}}{\pgfqpoint{1.595513in}{2.930509in}}{\pgfqpoint{1.601337in}{2.924685in}}%
\pgfpathcurveto{\pgfqpoint{1.607161in}{2.918861in}}{\pgfqpoint{1.615061in}{2.915589in}}{\pgfqpoint{1.623297in}{2.915589in}}%
\pgfpathclose%
\pgfusepath{stroke,fill}%
\end{pgfscope}%
\begin{pgfscope}%
\pgfpathrectangle{\pgfqpoint{0.100000in}{0.212622in}}{\pgfqpoint{3.696000in}{3.696000in}}%
\pgfusepath{clip}%
\pgfsetbuttcap%
\pgfsetroundjoin%
\definecolor{currentfill}{rgb}{0.121569,0.466667,0.705882}%
\pgfsetfillcolor{currentfill}%
\pgfsetfillopacity{0.384002}%
\pgfsetlinewidth{1.003750pt}%
\definecolor{currentstroke}{rgb}{0.121569,0.466667,0.705882}%
\pgfsetstrokecolor{currentstroke}%
\pgfsetstrokeopacity{0.384002}%
\pgfsetdash{}{0pt}%
\pgfpathmoveto{\pgfqpoint{1.621968in}{2.913409in}}%
\pgfpathcurveto{\pgfqpoint{1.630204in}{2.913409in}}{\pgfqpoint{1.638104in}{2.916681in}}{\pgfqpoint{1.643928in}{2.922505in}}%
\pgfpathcurveto{\pgfqpoint{1.649752in}{2.928329in}}{\pgfqpoint{1.653024in}{2.936229in}}{\pgfqpoint{1.653024in}{2.944465in}}%
\pgfpathcurveto{\pgfqpoint{1.653024in}{2.952702in}}{\pgfqpoint{1.649752in}{2.960602in}}{\pgfqpoint{1.643928in}{2.966426in}}%
\pgfpathcurveto{\pgfqpoint{1.638104in}{2.972249in}}{\pgfqpoint{1.630204in}{2.975522in}}{\pgfqpoint{1.621968in}{2.975522in}}%
\pgfpathcurveto{\pgfqpoint{1.613731in}{2.975522in}}{\pgfqpoint{1.605831in}{2.972249in}}{\pgfqpoint{1.600007in}{2.966426in}}%
\pgfpathcurveto{\pgfqpoint{1.594183in}{2.960602in}}{\pgfqpoint{1.590911in}{2.952702in}}{\pgfqpoint{1.590911in}{2.944465in}}%
\pgfpathcurveto{\pgfqpoint{1.590911in}{2.936229in}}{\pgfqpoint{1.594183in}{2.928329in}}{\pgfqpoint{1.600007in}{2.922505in}}%
\pgfpathcurveto{\pgfqpoint{1.605831in}{2.916681in}}{\pgfqpoint{1.613731in}{2.913409in}}{\pgfqpoint{1.621968in}{2.913409in}}%
\pgfpathclose%
\pgfusepath{stroke,fill}%
\end{pgfscope}%
\begin{pgfscope}%
\pgfpathrectangle{\pgfqpoint{0.100000in}{0.212622in}}{\pgfqpoint{3.696000in}{3.696000in}}%
\pgfusepath{clip}%
\pgfsetbuttcap%
\pgfsetroundjoin%
\definecolor{currentfill}{rgb}{0.121569,0.466667,0.705882}%
\pgfsetfillcolor{currentfill}%
\pgfsetfillopacity{0.384197}%
\pgfsetlinewidth{1.003750pt}%
\definecolor{currentstroke}{rgb}{0.121569,0.466667,0.705882}%
\pgfsetstrokecolor{currentstroke}%
\pgfsetstrokeopacity{0.384197}%
\pgfsetdash{}{0pt}%
\pgfpathmoveto{\pgfqpoint{1.968024in}{2.986323in}}%
\pgfpathcurveto{\pgfqpoint{1.976260in}{2.986323in}}{\pgfqpoint{1.984160in}{2.989596in}}{\pgfqpoint{1.989984in}{2.995420in}}%
\pgfpathcurveto{\pgfqpoint{1.995808in}{3.001244in}}{\pgfqpoint{1.999080in}{3.009144in}}{\pgfqpoint{1.999080in}{3.017380in}}%
\pgfpathcurveto{\pgfqpoint{1.999080in}{3.025616in}}{\pgfqpoint{1.995808in}{3.033516in}}{\pgfqpoint{1.989984in}{3.039340in}}%
\pgfpathcurveto{\pgfqpoint{1.984160in}{3.045164in}}{\pgfqpoint{1.976260in}{3.048436in}}{\pgfqpoint{1.968024in}{3.048436in}}%
\pgfpathcurveto{\pgfqpoint{1.959788in}{3.048436in}}{\pgfqpoint{1.951888in}{3.045164in}}{\pgfqpoint{1.946064in}{3.039340in}}%
\pgfpathcurveto{\pgfqpoint{1.940240in}{3.033516in}}{\pgfqpoint{1.936967in}{3.025616in}}{\pgfqpoint{1.936967in}{3.017380in}}%
\pgfpathcurveto{\pgfqpoint{1.936967in}{3.009144in}}{\pgfqpoint{1.940240in}{3.001244in}}{\pgfqpoint{1.946064in}{2.995420in}}%
\pgfpathcurveto{\pgfqpoint{1.951888in}{2.989596in}}{\pgfqpoint{1.959788in}{2.986323in}}{\pgfqpoint{1.968024in}{2.986323in}}%
\pgfpathclose%
\pgfusepath{stroke,fill}%
\end{pgfscope}%
\begin{pgfscope}%
\pgfpathrectangle{\pgfqpoint{0.100000in}{0.212622in}}{\pgfqpoint{3.696000in}{3.696000in}}%
\pgfusepath{clip}%
\pgfsetbuttcap%
\pgfsetroundjoin%
\definecolor{currentfill}{rgb}{0.121569,0.466667,0.705882}%
\pgfsetfillcolor{currentfill}%
\pgfsetfillopacity{0.384785}%
\pgfsetlinewidth{1.003750pt}%
\definecolor{currentstroke}{rgb}{0.121569,0.466667,0.705882}%
\pgfsetstrokecolor{currentstroke}%
\pgfsetstrokeopacity{0.384785}%
\pgfsetdash{}{0pt}%
\pgfpathmoveto{\pgfqpoint{1.619959in}{2.909051in}}%
\pgfpathcurveto{\pgfqpoint{1.628195in}{2.909051in}}{\pgfqpoint{1.636095in}{2.912323in}}{\pgfqpoint{1.641919in}{2.918147in}}%
\pgfpathcurveto{\pgfqpoint{1.647743in}{2.923971in}}{\pgfqpoint{1.651015in}{2.931871in}}{\pgfqpoint{1.651015in}{2.940107in}}%
\pgfpathcurveto{\pgfqpoint{1.651015in}{2.948344in}}{\pgfqpoint{1.647743in}{2.956244in}}{\pgfqpoint{1.641919in}{2.962068in}}%
\pgfpathcurveto{\pgfqpoint{1.636095in}{2.967891in}}{\pgfqpoint{1.628195in}{2.971164in}}{\pgfqpoint{1.619959in}{2.971164in}}%
\pgfpathcurveto{\pgfqpoint{1.611723in}{2.971164in}}{\pgfqpoint{1.603823in}{2.967891in}}{\pgfqpoint{1.597999in}{2.962068in}}%
\pgfpathcurveto{\pgfqpoint{1.592175in}{2.956244in}}{\pgfqpoint{1.588902in}{2.948344in}}{\pgfqpoint{1.588902in}{2.940107in}}%
\pgfpathcurveto{\pgfqpoint{1.588902in}{2.931871in}}{\pgfqpoint{1.592175in}{2.923971in}}{\pgfqpoint{1.597999in}{2.918147in}}%
\pgfpathcurveto{\pgfqpoint{1.603823in}{2.912323in}}{\pgfqpoint{1.611723in}{2.909051in}}{\pgfqpoint{1.619959in}{2.909051in}}%
\pgfpathclose%
\pgfusepath{stroke,fill}%
\end{pgfscope}%
\begin{pgfscope}%
\pgfpathrectangle{\pgfqpoint{0.100000in}{0.212622in}}{\pgfqpoint{3.696000in}{3.696000in}}%
\pgfusepath{clip}%
\pgfsetbuttcap%
\pgfsetroundjoin%
\definecolor{currentfill}{rgb}{0.121569,0.466667,0.705882}%
\pgfsetfillcolor{currentfill}%
\pgfsetfillopacity{0.384940}%
\pgfsetlinewidth{1.003750pt}%
\definecolor{currentstroke}{rgb}{0.121569,0.466667,0.705882}%
\pgfsetstrokecolor{currentstroke}%
\pgfsetstrokeopacity{0.384940}%
\pgfsetdash{}{0pt}%
\pgfpathmoveto{\pgfqpoint{1.619428in}{2.908204in}}%
\pgfpathcurveto{\pgfqpoint{1.627664in}{2.908204in}}{\pgfqpoint{1.635565in}{2.911476in}}{\pgfqpoint{1.641388in}{2.917300in}}%
\pgfpathcurveto{\pgfqpoint{1.647212in}{2.923124in}}{\pgfqpoint{1.650485in}{2.931024in}}{\pgfqpoint{1.650485in}{2.939260in}}%
\pgfpathcurveto{\pgfqpoint{1.650485in}{2.947496in}}{\pgfqpoint{1.647212in}{2.955397in}}{\pgfqpoint{1.641388in}{2.961220in}}%
\pgfpathcurveto{\pgfqpoint{1.635565in}{2.967044in}}{\pgfqpoint{1.627664in}{2.970317in}}{\pgfqpoint{1.619428in}{2.970317in}}%
\pgfpathcurveto{\pgfqpoint{1.611192in}{2.970317in}}{\pgfqpoint{1.603292in}{2.967044in}}{\pgfqpoint{1.597468in}{2.961220in}}%
\pgfpathcurveto{\pgfqpoint{1.591644in}{2.955397in}}{\pgfqpoint{1.588372in}{2.947496in}}{\pgfqpoint{1.588372in}{2.939260in}}%
\pgfpathcurveto{\pgfqpoint{1.588372in}{2.931024in}}{\pgfqpoint{1.591644in}{2.923124in}}{\pgfqpoint{1.597468in}{2.917300in}}%
\pgfpathcurveto{\pgfqpoint{1.603292in}{2.911476in}}{\pgfqpoint{1.611192in}{2.908204in}}{\pgfqpoint{1.619428in}{2.908204in}}%
\pgfpathclose%
\pgfusepath{stroke,fill}%
\end{pgfscope}%
\begin{pgfscope}%
\pgfpathrectangle{\pgfqpoint{0.100000in}{0.212622in}}{\pgfqpoint{3.696000in}{3.696000in}}%
\pgfusepath{clip}%
\pgfsetbuttcap%
\pgfsetroundjoin%
\definecolor{currentfill}{rgb}{0.121569,0.466667,0.705882}%
\pgfsetfillcolor{currentfill}%
\pgfsetfillopacity{0.385035}%
\pgfsetlinewidth{1.003750pt}%
\definecolor{currentstroke}{rgb}{0.121569,0.466667,0.705882}%
\pgfsetstrokecolor{currentstroke}%
\pgfsetstrokeopacity{0.385035}%
\pgfsetdash{}{0pt}%
\pgfpathmoveto{\pgfqpoint{1.968340in}{2.983587in}}%
\pgfpathcurveto{\pgfqpoint{1.976576in}{2.983587in}}{\pgfqpoint{1.984476in}{2.986860in}}{\pgfqpoint{1.990300in}{2.992684in}}%
\pgfpathcurveto{\pgfqpoint{1.996124in}{2.998508in}}{\pgfqpoint{1.999396in}{3.006408in}}{\pgfqpoint{1.999396in}{3.014644in}}%
\pgfpathcurveto{\pgfqpoint{1.999396in}{3.022880in}}{\pgfqpoint{1.996124in}{3.030780in}}{\pgfqpoint{1.990300in}{3.036604in}}%
\pgfpathcurveto{\pgfqpoint{1.984476in}{3.042428in}}{\pgfqpoint{1.976576in}{3.045700in}}{\pgfqpoint{1.968340in}{3.045700in}}%
\pgfpathcurveto{\pgfqpoint{1.960103in}{3.045700in}}{\pgfqpoint{1.952203in}{3.042428in}}{\pgfqpoint{1.946379in}{3.036604in}}%
\pgfpathcurveto{\pgfqpoint{1.940555in}{3.030780in}}{\pgfqpoint{1.937283in}{3.022880in}}{\pgfqpoint{1.937283in}{3.014644in}}%
\pgfpathcurveto{\pgfqpoint{1.937283in}{3.006408in}}{\pgfqpoint{1.940555in}{2.998508in}}{\pgfqpoint{1.946379in}{2.992684in}}%
\pgfpathcurveto{\pgfqpoint{1.952203in}{2.986860in}}{\pgfqpoint{1.960103in}{2.983587in}}{\pgfqpoint{1.968340in}{2.983587in}}%
\pgfpathclose%
\pgfusepath{stroke,fill}%
\end{pgfscope}%
\begin{pgfscope}%
\pgfpathrectangle{\pgfqpoint{0.100000in}{0.212622in}}{\pgfqpoint{3.696000in}{3.696000in}}%
\pgfusepath{clip}%
\pgfsetbuttcap%
\pgfsetroundjoin%
\definecolor{currentfill}{rgb}{0.121569,0.466667,0.705882}%
\pgfsetfillcolor{currentfill}%
\pgfsetfillopacity{0.385250}%
\pgfsetlinewidth{1.003750pt}%
\definecolor{currentstroke}{rgb}{0.121569,0.466667,0.705882}%
\pgfsetstrokecolor{currentstroke}%
\pgfsetstrokeopacity{0.385250}%
\pgfsetdash{}{0pt}%
\pgfpathmoveto{\pgfqpoint{1.618591in}{2.906599in}}%
\pgfpathcurveto{\pgfqpoint{1.626828in}{2.906599in}}{\pgfqpoint{1.634728in}{2.909872in}}{\pgfqpoint{1.640552in}{2.915696in}}%
\pgfpathcurveto{\pgfqpoint{1.646376in}{2.921520in}}{\pgfqpoint{1.649648in}{2.929420in}}{\pgfqpoint{1.649648in}{2.937656in}}%
\pgfpathcurveto{\pgfqpoint{1.649648in}{2.945892in}}{\pgfqpoint{1.646376in}{2.953792in}}{\pgfqpoint{1.640552in}{2.959616in}}%
\pgfpathcurveto{\pgfqpoint{1.634728in}{2.965440in}}{\pgfqpoint{1.626828in}{2.968712in}}{\pgfqpoint{1.618591in}{2.968712in}}%
\pgfpathcurveto{\pgfqpoint{1.610355in}{2.968712in}}{\pgfqpoint{1.602455in}{2.965440in}}{\pgfqpoint{1.596631in}{2.959616in}}%
\pgfpathcurveto{\pgfqpoint{1.590807in}{2.953792in}}{\pgfqpoint{1.587535in}{2.945892in}}{\pgfqpoint{1.587535in}{2.937656in}}%
\pgfpathcurveto{\pgfqpoint{1.587535in}{2.929420in}}{\pgfqpoint{1.590807in}{2.921520in}}{\pgfqpoint{1.596631in}{2.915696in}}%
\pgfpathcurveto{\pgfqpoint{1.602455in}{2.909872in}}{\pgfqpoint{1.610355in}{2.906599in}}{\pgfqpoint{1.618591in}{2.906599in}}%
\pgfpathclose%
\pgfusepath{stroke,fill}%
\end{pgfscope}%
\begin{pgfscope}%
\pgfpathrectangle{\pgfqpoint{0.100000in}{0.212622in}}{\pgfqpoint{3.696000in}{3.696000in}}%
\pgfusepath{clip}%
\pgfsetbuttcap%
\pgfsetroundjoin%
\definecolor{currentfill}{rgb}{0.121569,0.466667,0.705882}%
\pgfsetfillcolor{currentfill}%
\pgfsetfillopacity{0.385806}%
\pgfsetlinewidth{1.003750pt}%
\definecolor{currentstroke}{rgb}{0.121569,0.466667,0.705882}%
\pgfsetstrokecolor{currentstroke}%
\pgfsetstrokeopacity{0.385806}%
\pgfsetdash{}{0pt}%
\pgfpathmoveto{\pgfqpoint{1.616982in}{2.903752in}}%
\pgfpathcurveto{\pgfqpoint{1.625219in}{2.903752in}}{\pgfqpoint{1.633119in}{2.907025in}}{\pgfqpoint{1.638943in}{2.912849in}}%
\pgfpathcurveto{\pgfqpoint{1.644767in}{2.918673in}}{\pgfqpoint{1.648039in}{2.926573in}}{\pgfqpoint{1.648039in}{2.934809in}}%
\pgfpathcurveto{\pgfqpoint{1.648039in}{2.943045in}}{\pgfqpoint{1.644767in}{2.950945in}}{\pgfqpoint{1.638943in}{2.956769in}}%
\pgfpathcurveto{\pgfqpoint{1.633119in}{2.962593in}}{\pgfqpoint{1.625219in}{2.965865in}}{\pgfqpoint{1.616982in}{2.965865in}}%
\pgfpathcurveto{\pgfqpoint{1.608746in}{2.965865in}}{\pgfqpoint{1.600846in}{2.962593in}}{\pgfqpoint{1.595022in}{2.956769in}}%
\pgfpathcurveto{\pgfqpoint{1.589198in}{2.950945in}}{\pgfqpoint{1.585926in}{2.943045in}}{\pgfqpoint{1.585926in}{2.934809in}}%
\pgfpathcurveto{\pgfqpoint{1.585926in}{2.926573in}}{\pgfqpoint{1.589198in}{2.918673in}}{\pgfqpoint{1.595022in}{2.912849in}}%
\pgfpathcurveto{\pgfqpoint{1.600846in}{2.907025in}}{\pgfqpoint{1.608746in}{2.903752in}}{\pgfqpoint{1.616982in}{2.903752in}}%
\pgfpathclose%
\pgfusepath{stroke,fill}%
\end{pgfscope}%
\begin{pgfscope}%
\pgfpathrectangle{\pgfqpoint{0.100000in}{0.212622in}}{\pgfqpoint{3.696000in}{3.696000in}}%
\pgfusepath{clip}%
\pgfsetbuttcap%
\pgfsetroundjoin%
\definecolor{currentfill}{rgb}{0.121569,0.466667,0.705882}%
\pgfsetfillcolor{currentfill}%
\pgfsetfillopacity{0.386250}%
\pgfsetlinewidth{1.003750pt}%
\definecolor{currentstroke}{rgb}{0.121569,0.466667,0.705882}%
\pgfsetstrokecolor{currentstroke}%
\pgfsetstrokeopacity{0.386250}%
\pgfsetdash{}{0pt}%
\pgfpathmoveto{\pgfqpoint{1.969201in}{2.979029in}}%
\pgfpathcurveto{\pgfqpoint{1.977437in}{2.979029in}}{\pgfqpoint{1.985337in}{2.982301in}}{\pgfqpoint{1.991161in}{2.988125in}}%
\pgfpathcurveto{\pgfqpoint{1.996985in}{2.993949in}}{\pgfqpoint{2.000258in}{3.001849in}}{\pgfqpoint{2.000258in}{3.010085in}}%
\pgfpathcurveto{\pgfqpoint{2.000258in}{3.018321in}}{\pgfqpoint{1.996985in}{3.026222in}}{\pgfqpoint{1.991161in}{3.032045in}}%
\pgfpathcurveto{\pgfqpoint{1.985337in}{3.037869in}}{\pgfqpoint{1.977437in}{3.041142in}}{\pgfqpoint{1.969201in}{3.041142in}}%
\pgfpathcurveto{\pgfqpoint{1.960965in}{3.041142in}}{\pgfqpoint{1.953065in}{3.037869in}}{\pgfqpoint{1.947241in}{3.032045in}}%
\pgfpathcurveto{\pgfqpoint{1.941417in}{3.026222in}}{\pgfqpoint{1.938145in}{3.018321in}}{\pgfqpoint{1.938145in}{3.010085in}}%
\pgfpathcurveto{\pgfqpoint{1.938145in}{3.001849in}}{\pgfqpoint{1.941417in}{2.993949in}}{\pgfqpoint{1.947241in}{2.988125in}}%
\pgfpathcurveto{\pgfqpoint{1.953065in}{2.982301in}}{\pgfqpoint{1.960965in}{2.979029in}}{\pgfqpoint{1.969201in}{2.979029in}}%
\pgfpathclose%
\pgfusepath{stroke,fill}%
\end{pgfscope}%
\begin{pgfscope}%
\pgfpathrectangle{\pgfqpoint{0.100000in}{0.212622in}}{\pgfqpoint{3.696000in}{3.696000in}}%
\pgfusepath{clip}%
\pgfsetbuttcap%
\pgfsetroundjoin%
\definecolor{currentfill}{rgb}{0.121569,0.466667,0.705882}%
\pgfsetfillcolor{currentfill}%
\pgfsetfillopacity{0.386712}%
\pgfsetlinewidth{1.003750pt}%
\definecolor{currentstroke}{rgb}{0.121569,0.466667,0.705882}%
\pgfsetstrokecolor{currentstroke}%
\pgfsetstrokeopacity{0.386712}%
\pgfsetdash{}{0pt}%
\pgfpathmoveto{\pgfqpoint{1.613800in}{2.898545in}}%
\pgfpathcurveto{\pgfqpoint{1.622036in}{2.898545in}}{\pgfqpoint{1.629936in}{2.901818in}}{\pgfqpoint{1.635760in}{2.907642in}}%
\pgfpathcurveto{\pgfqpoint{1.641584in}{2.913465in}}{\pgfqpoint{1.644856in}{2.921366in}}{\pgfqpoint{1.644856in}{2.929602in}}%
\pgfpathcurveto{\pgfqpoint{1.644856in}{2.937838in}}{\pgfqpoint{1.641584in}{2.945738in}}{\pgfqpoint{1.635760in}{2.951562in}}%
\pgfpathcurveto{\pgfqpoint{1.629936in}{2.957386in}}{\pgfqpoint{1.622036in}{2.960658in}}{\pgfqpoint{1.613800in}{2.960658in}}%
\pgfpathcurveto{\pgfqpoint{1.605563in}{2.960658in}}{\pgfqpoint{1.597663in}{2.957386in}}{\pgfqpoint{1.591839in}{2.951562in}}%
\pgfpathcurveto{\pgfqpoint{1.586015in}{2.945738in}}{\pgfqpoint{1.582743in}{2.937838in}}{\pgfqpoint{1.582743in}{2.929602in}}%
\pgfpathcurveto{\pgfqpoint{1.582743in}{2.921366in}}{\pgfqpoint{1.586015in}{2.913465in}}{\pgfqpoint{1.591839in}{2.907642in}}%
\pgfpathcurveto{\pgfqpoint{1.597663in}{2.901818in}}{\pgfqpoint{1.605563in}{2.898545in}}{\pgfqpoint{1.613800in}{2.898545in}}%
\pgfpathclose%
\pgfusepath{stroke,fill}%
\end{pgfscope}%
\begin{pgfscope}%
\pgfpathrectangle{\pgfqpoint{0.100000in}{0.212622in}}{\pgfqpoint{3.696000in}{3.696000in}}%
\pgfusepath{clip}%
\pgfsetbuttcap%
\pgfsetroundjoin%
\definecolor{currentfill}{rgb}{0.121569,0.466667,0.705882}%
\pgfsetfillcolor{currentfill}%
\pgfsetfillopacity{0.387427}%
\pgfsetlinewidth{1.003750pt}%
\definecolor{currentstroke}{rgb}{0.121569,0.466667,0.705882}%
\pgfsetstrokecolor{currentstroke}%
\pgfsetstrokeopacity{0.387427}%
\pgfsetdash{}{0pt}%
\pgfpathmoveto{\pgfqpoint{1.611900in}{2.894431in}}%
\pgfpathcurveto{\pgfqpoint{1.620136in}{2.894431in}}{\pgfqpoint{1.628036in}{2.897703in}}{\pgfqpoint{1.633860in}{2.903527in}}%
\pgfpathcurveto{\pgfqpoint{1.639684in}{2.909351in}}{\pgfqpoint{1.642956in}{2.917251in}}{\pgfqpoint{1.642956in}{2.925487in}}%
\pgfpathcurveto{\pgfqpoint{1.642956in}{2.933723in}}{\pgfqpoint{1.639684in}{2.941623in}}{\pgfqpoint{1.633860in}{2.947447in}}%
\pgfpathcurveto{\pgfqpoint{1.628036in}{2.953271in}}{\pgfqpoint{1.620136in}{2.956544in}}{\pgfqpoint{1.611900in}{2.956544in}}%
\pgfpathcurveto{\pgfqpoint{1.603664in}{2.956544in}}{\pgfqpoint{1.595764in}{2.953271in}}{\pgfqpoint{1.589940in}{2.947447in}}%
\pgfpathcurveto{\pgfqpoint{1.584116in}{2.941623in}}{\pgfqpoint{1.580843in}{2.933723in}}{\pgfqpoint{1.580843in}{2.925487in}}%
\pgfpathcurveto{\pgfqpoint{1.580843in}{2.917251in}}{\pgfqpoint{1.584116in}{2.909351in}}{\pgfqpoint{1.589940in}{2.903527in}}%
\pgfpathcurveto{\pgfqpoint{1.595764in}{2.897703in}}{\pgfqpoint{1.603664in}{2.894431in}}{\pgfqpoint{1.611900in}{2.894431in}}%
\pgfpathclose%
\pgfusepath{stroke,fill}%
\end{pgfscope}%
\begin{pgfscope}%
\pgfpathrectangle{\pgfqpoint{0.100000in}{0.212622in}}{\pgfqpoint{3.696000in}{3.696000in}}%
\pgfusepath{clip}%
\pgfsetbuttcap%
\pgfsetroundjoin%
\definecolor{currentfill}{rgb}{0.121569,0.466667,0.705882}%
\pgfsetfillcolor{currentfill}%
\pgfsetfillopacity{0.387673}%
\pgfsetlinewidth{1.003750pt}%
\definecolor{currentstroke}{rgb}{0.121569,0.466667,0.705882}%
\pgfsetstrokecolor{currentstroke}%
\pgfsetstrokeopacity{0.387673}%
\pgfsetdash{}{0pt}%
\pgfpathmoveto{\pgfqpoint{1.970195in}{2.973399in}}%
\pgfpathcurveto{\pgfqpoint{1.978431in}{2.973399in}}{\pgfqpoint{1.986331in}{2.976671in}}{\pgfqpoint{1.992155in}{2.982495in}}%
\pgfpathcurveto{\pgfqpoint{1.997979in}{2.988319in}}{\pgfqpoint{2.001251in}{2.996219in}}{\pgfqpoint{2.001251in}{3.004455in}}%
\pgfpathcurveto{\pgfqpoint{2.001251in}{3.012692in}}{\pgfqpoint{1.997979in}{3.020592in}}{\pgfqpoint{1.992155in}{3.026416in}}%
\pgfpathcurveto{\pgfqpoint{1.986331in}{3.032240in}}{\pgfqpoint{1.978431in}{3.035512in}}{\pgfqpoint{1.970195in}{3.035512in}}%
\pgfpathcurveto{\pgfqpoint{1.961958in}{3.035512in}}{\pgfqpoint{1.954058in}{3.032240in}}{\pgfqpoint{1.948234in}{3.026416in}}%
\pgfpathcurveto{\pgfqpoint{1.942410in}{3.020592in}}{\pgfqpoint{1.939138in}{3.012692in}}{\pgfqpoint{1.939138in}{3.004455in}}%
\pgfpathcurveto{\pgfqpoint{1.939138in}{2.996219in}}{\pgfqpoint{1.942410in}{2.988319in}}{\pgfqpoint{1.948234in}{2.982495in}}%
\pgfpathcurveto{\pgfqpoint{1.954058in}{2.976671in}}{\pgfqpoint{1.961958in}{2.973399in}}{\pgfqpoint{1.970195in}{2.973399in}}%
\pgfpathclose%
\pgfusepath{stroke,fill}%
\end{pgfscope}%
\begin{pgfscope}%
\pgfpathrectangle{\pgfqpoint{0.100000in}{0.212622in}}{\pgfqpoint{3.696000in}{3.696000in}}%
\pgfusepath{clip}%
\pgfsetbuttcap%
\pgfsetroundjoin%
\definecolor{currentfill}{rgb}{0.121569,0.466667,0.705882}%
\pgfsetfillcolor{currentfill}%
\pgfsetfillopacity{0.387962}%
\pgfsetlinewidth{1.003750pt}%
\definecolor{currentstroke}{rgb}{0.121569,0.466667,0.705882}%
\pgfsetstrokecolor{currentstroke}%
\pgfsetstrokeopacity{0.387962}%
\pgfsetdash{}{0pt}%
\pgfpathmoveto{\pgfqpoint{1.610181in}{2.891684in}}%
\pgfpathcurveto{\pgfqpoint{1.618417in}{2.891684in}}{\pgfqpoint{1.626317in}{2.894956in}}{\pgfqpoint{1.632141in}{2.900780in}}%
\pgfpathcurveto{\pgfqpoint{1.637965in}{2.906604in}}{\pgfqpoint{1.641237in}{2.914504in}}{\pgfqpoint{1.641237in}{2.922741in}}%
\pgfpathcurveto{\pgfqpoint{1.641237in}{2.930977in}}{\pgfqpoint{1.637965in}{2.938877in}}{\pgfqpoint{1.632141in}{2.944701in}}%
\pgfpathcurveto{\pgfqpoint{1.626317in}{2.950525in}}{\pgfqpoint{1.618417in}{2.953797in}}{\pgfqpoint{1.610181in}{2.953797in}}%
\pgfpathcurveto{\pgfqpoint{1.601944in}{2.953797in}}{\pgfqpoint{1.594044in}{2.950525in}}{\pgfqpoint{1.588220in}{2.944701in}}%
\pgfpathcurveto{\pgfqpoint{1.582396in}{2.938877in}}{\pgfqpoint{1.579124in}{2.930977in}}{\pgfqpoint{1.579124in}{2.922741in}}%
\pgfpathcurveto{\pgfqpoint{1.579124in}{2.914504in}}{\pgfqpoint{1.582396in}{2.906604in}}{\pgfqpoint{1.588220in}{2.900780in}}%
\pgfpathcurveto{\pgfqpoint{1.594044in}{2.894956in}}{\pgfqpoint{1.601944in}{2.891684in}}{\pgfqpoint{1.610181in}{2.891684in}}%
\pgfpathclose%
\pgfusepath{stroke,fill}%
\end{pgfscope}%
\begin{pgfscope}%
\pgfpathrectangle{\pgfqpoint{0.100000in}{0.212622in}}{\pgfqpoint{3.696000in}{3.696000in}}%
\pgfusepath{clip}%
\pgfsetbuttcap%
\pgfsetroundjoin%
\definecolor{currentfill}{rgb}{0.121569,0.466667,0.705882}%
\pgfsetfillcolor{currentfill}%
\pgfsetfillopacity{0.388952}%
\pgfsetlinewidth{1.003750pt}%
\definecolor{currentstroke}{rgb}{0.121569,0.466667,0.705882}%
\pgfsetstrokecolor{currentstroke}%
\pgfsetstrokeopacity{0.388952}%
\pgfsetdash{}{0pt}%
\pgfpathmoveto{\pgfqpoint{1.607179in}{2.886570in}}%
\pgfpathcurveto{\pgfqpoint{1.615415in}{2.886570in}}{\pgfqpoint{1.623315in}{2.889842in}}{\pgfqpoint{1.629139in}{2.895666in}}%
\pgfpathcurveto{\pgfqpoint{1.634963in}{2.901490in}}{\pgfqpoint{1.638235in}{2.909390in}}{\pgfqpoint{1.638235in}{2.917626in}}%
\pgfpathcurveto{\pgfqpoint{1.638235in}{2.925863in}}{\pgfqpoint{1.634963in}{2.933763in}}{\pgfqpoint{1.629139in}{2.939586in}}%
\pgfpathcurveto{\pgfqpoint{1.623315in}{2.945410in}}{\pgfqpoint{1.615415in}{2.948683in}}{\pgfqpoint{1.607179in}{2.948683in}}%
\pgfpathcurveto{\pgfqpoint{1.598943in}{2.948683in}}{\pgfqpoint{1.591043in}{2.945410in}}{\pgfqpoint{1.585219in}{2.939586in}}%
\pgfpathcurveto{\pgfqpoint{1.579395in}{2.933763in}}{\pgfqpoint{1.576122in}{2.925863in}}{\pgfqpoint{1.576122in}{2.917626in}}%
\pgfpathcurveto{\pgfqpoint{1.576122in}{2.909390in}}{\pgfqpoint{1.579395in}{2.901490in}}{\pgfqpoint{1.585219in}{2.895666in}}%
\pgfpathcurveto{\pgfqpoint{1.591043in}{2.889842in}}{\pgfqpoint{1.598943in}{2.886570in}}{\pgfqpoint{1.607179in}{2.886570in}}%
\pgfpathclose%
\pgfusepath{stroke,fill}%
\end{pgfscope}%
\begin{pgfscope}%
\pgfpathrectangle{\pgfqpoint{0.100000in}{0.212622in}}{\pgfqpoint{3.696000in}{3.696000in}}%
\pgfusepath{clip}%
\pgfsetbuttcap%
\pgfsetroundjoin%
\definecolor{currentfill}{rgb}{0.121569,0.466667,0.705882}%
\pgfsetfillcolor{currentfill}%
\pgfsetfillopacity{0.389362}%
\pgfsetlinewidth{1.003750pt}%
\definecolor{currentstroke}{rgb}{0.121569,0.466667,0.705882}%
\pgfsetstrokecolor{currentstroke}%
\pgfsetstrokeopacity{0.389362}%
\pgfsetdash{}{0pt}%
\pgfpathmoveto{\pgfqpoint{1.970828in}{2.967104in}}%
\pgfpathcurveto{\pgfqpoint{1.979065in}{2.967104in}}{\pgfqpoint{1.986965in}{2.970376in}}{\pgfqpoint{1.992789in}{2.976200in}}%
\pgfpathcurveto{\pgfqpoint{1.998613in}{2.982024in}}{\pgfqpoint{2.001885in}{2.989924in}}{\pgfqpoint{2.001885in}{2.998160in}}%
\pgfpathcurveto{\pgfqpoint{2.001885in}{3.006396in}}{\pgfqpoint{1.998613in}{3.014296in}}{\pgfqpoint{1.992789in}{3.020120in}}%
\pgfpathcurveto{\pgfqpoint{1.986965in}{3.025944in}}{\pgfqpoint{1.979065in}{3.029217in}}{\pgfqpoint{1.970828in}{3.029217in}}%
\pgfpathcurveto{\pgfqpoint{1.962592in}{3.029217in}}{\pgfqpoint{1.954692in}{3.025944in}}{\pgfqpoint{1.948868in}{3.020120in}}%
\pgfpathcurveto{\pgfqpoint{1.943044in}{3.014296in}}{\pgfqpoint{1.939772in}{3.006396in}}{\pgfqpoint{1.939772in}{2.998160in}}%
\pgfpathcurveto{\pgfqpoint{1.939772in}{2.989924in}}{\pgfqpoint{1.943044in}{2.982024in}}{\pgfqpoint{1.948868in}{2.976200in}}%
\pgfpathcurveto{\pgfqpoint{1.954692in}{2.970376in}}{\pgfqpoint{1.962592in}{2.967104in}}{\pgfqpoint{1.970828in}{2.967104in}}%
\pgfpathclose%
\pgfusepath{stroke,fill}%
\end{pgfscope}%
\begin{pgfscope}%
\pgfpathrectangle{\pgfqpoint{0.100000in}{0.212622in}}{\pgfqpoint{3.696000in}{3.696000in}}%
\pgfusepath{clip}%
\pgfsetbuttcap%
\pgfsetroundjoin%
\definecolor{currentfill}{rgb}{0.121569,0.466667,0.705882}%
\pgfsetfillcolor{currentfill}%
\pgfsetfillopacity{0.389814}%
\pgfsetlinewidth{1.003750pt}%
\definecolor{currentstroke}{rgb}{0.121569,0.466667,0.705882}%
\pgfsetstrokecolor{currentstroke}%
\pgfsetstrokeopacity{0.389814}%
\pgfsetdash{}{0pt}%
\pgfpathmoveto{\pgfqpoint{1.604890in}{2.882107in}}%
\pgfpathcurveto{\pgfqpoint{1.613126in}{2.882107in}}{\pgfqpoint{1.621026in}{2.885379in}}{\pgfqpoint{1.626850in}{2.891203in}}%
\pgfpathcurveto{\pgfqpoint{1.632674in}{2.897027in}}{\pgfqpoint{1.635947in}{2.904927in}}{\pgfqpoint{1.635947in}{2.913163in}}%
\pgfpathcurveto{\pgfqpoint{1.635947in}{2.921400in}}{\pgfqpoint{1.632674in}{2.929300in}}{\pgfqpoint{1.626850in}{2.935124in}}%
\pgfpathcurveto{\pgfqpoint{1.621026in}{2.940948in}}{\pgfqpoint{1.613126in}{2.944220in}}{\pgfqpoint{1.604890in}{2.944220in}}%
\pgfpathcurveto{\pgfqpoint{1.596654in}{2.944220in}}{\pgfqpoint{1.588754in}{2.940948in}}{\pgfqpoint{1.582930in}{2.935124in}}%
\pgfpathcurveto{\pgfqpoint{1.577106in}{2.929300in}}{\pgfqpoint{1.573834in}{2.921400in}}{\pgfqpoint{1.573834in}{2.913163in}}%
\pgfpathcurveto{\pgfqpoint{1.573834in}{2.904927in}}{\pgfqpoint{1.577106in}{2.897027in}}{\pgfqpoint{1.582930in}{2.891203in}}%
\pgfpathcurveto{\pgfqpoint{1.588754in}{2.885379in}}{\pgfqpoint{1.596654in}{2.882107in}}{\pgfqpoint{1.604890in}{2.882107in}}%
\pgfpathclose%
\pgfusepath{stroke,fill}%
\end{pgfscope}%
\begin{pgfscope}%
\pgfpathrectangle{\pgfqpoint{0.100000in}{0.212622in}}{\pgfqpoint{3.696000in}{3.696000in}}%
\pgfusepath{clip}%
\pgfsetbuttcap%
\pgfsetroundjoin%
\definecolor{currentfill}{rgb}{0.121569,0.466667,0.705882}%
\pgfsetfillcolor{currentfill}%
\pgfsetfillopacity{0.390247}%
\pgfsetlinewidth{1.003750pt}%
\definecolor{currentstroke}{rgb}{0.121569,0.466667,0.705882}%
\pgfsetstrokecolor{currentstroke}%
\pgfsetstrokeopacity{0.390247}%
\pgfsetdash{}{0pt}%
\pgfpathmoveto{\pgfqpoint{1.603340in}{2.879734in}}%
\pgfpathcurveto{\pgfqpoint{1.611576in}{2.879734in}}{\pgfqpoint{1.619476in}{2.883006in}}{\pgfqpoint{1.625300in}{2.888830in}}%
\pgfpathcurveto{\pgfqpoint{1.631124in}{2.894654in}}{\pgfqpoint{1.634397in}{2.902554in}}{\pgfqpoint{1.634397in}{2.910790in}}%
\pgfpathcurveto{\pgfqpoint{1.634397in}{2.919027in}}{\pgfqpoint{1.631124in}{2.926927in}}{\pgfqpoint{1.625300in}{2.932751in}}%
\pgfpathcurveto{\pgfqpoint{1.619476in}{2.938575in}}{\pgfqpoint{1.611576in}{2.941847in}}{\pgfqpoint{1.603340in}{2.941847in}}%
\pgfpathcurveto{\pgfqpoint{1.595104in}{2.941847in}}{\pgfqpoint{1.587204in}{2.938575in}}{\pgfqpoint{1.581380in}{2.932751in}}%
\pgfpathcurveto{\pgfqpoint{1.575556in}{2.926927in}}{\pgfqpoint{1.572284in}{2.919027in}}{\pgfqpoint{1.572284in}{2.910790in}}%
\pgfpathcurveto{\pgfqpoint{1.572284in}{2.902554in}}{\pgfqpoint{1.575556in}{2.894654in}}{\pgfqpoint{1.581380in}{2.888830in}}%
\pgfpathcurveto{\pgfqpoint{1.587204in}{2.883006in}}{\pgfqpoint{1.595104in}{2.879734in}}{\pgfqpoint{1.603340in}{2.879734in}}%
\pgfpathclose%
\pgfusepath{stroke,fill}%
\end{pgfscope}%
\begin{pgfscope}%
\pgfpathrectangle{\pgfqpoint{0.100000in}{0.212622in}}{\pgfqpoint{3.696000in}{3.696000in}}%
\pgfusepath{clip}%
\pgfsetbuttcap%
\pgfsetroundjoin%
\definecolor{currentfill}{rgb}{0.121569,0.466667,0.705882}%
\pgfsetfillcolor{currentfill}%
\pgfsetfillopacity{0.390541}%
\pgfsetlinewidth{1.003750pt}%
\definecolor{currentstroke}{rgb}{0.121569,0.466667,0.705882}%
\pgfsetstrokecolor{currentstroke}%
\pgfsetstrokeopacity{0.390541}%
\pgfsetdash{}{0pt}%
\pgfpathmoveto{\pgfqpoint{1.602556in}{2.878273in}}%
\pgfpathcurveto{\pgfqpoint{1.610792in}{2.878273in}}{\pgfqpoint{1.618692in}{2.881546in}}{\pgfqpoint{1.624516in}{2.887370in}}%
\pgfpathcurveto{\pgfqpoint{1.630340in}{2.893194in}}{\pgfqpoint{1.633612in}{2.901094in}}{\pgfqpoint{1.633612in}{2.909330in}}%
\pgfpathcurveto{\pgfqpoint{1.633612in}{2.917566in}}{\pgfqpoint{1.630340in}{2.925466in}}{\pgfqpoint{1.624516in}{2.931290in}}%
\pgfpathcurveto{\pgfqpoint{1.618692in}{2.937114in}}{\pgfqpoint{1.610792in}{2.940386in}}{\pgfqpoint{1.602556in}{2.940386in}}%
\pgfpathcurveto{\pgfqpoint{1.594320in}{2.940386in}}{\pgfqpoint{1.586420in}{2.937114in}}{\pgfqpoint{1.580596in}{2.931290in}}%
\pgfpathcurveto{\pgfqpoint{1.574772in}{2.925466in}}{\pgfqpoint{1.571499in}{2.917566in}}{\pgfqpoint{1.571499in}{2.909330in}}%
\pgfpathcurveto{\pgfqpoint{1.571499in}{2.901094in}}{\pgfqpoint{1.574772in}{2.893194in}}{\pgfqpoint{1.580596in}{2.887370in}}%
\pgfpathcurveto{\pgfqpoint{1.586420in}{2.881546in}}{\pgfqpoint{1.594320in}{2.878273in}}{\pgfqpoint{1.602556in}{2.878273in}}%
\pgfpathclose%
\pgfusepath{stroke,fill}%
\end{pgfscope}%
\begin{pgfscope}%
\pgfpathrectangle{\pgfqpoint{0.100000in}{0.212622in}}{\pgfqpoint{3.696000in}{3.696000in}}%
\pgfusepath{clip}%
\pgfsetbuttcap%
\pgfsetroundjoin%
\definecolor{currentfill}{rgb}{0.121569,0.466667,0.705882}%
\pgfsetfillcolor{currentfill}%
\pgfsetfillopacity{0.391058}%
\pgfsetlinewidth{1.003750pt}%
\definecolor{currentstroke}{rgb}{0.121569,0.466667,0.705882}%
\pgfsetstrokecolor{currentstroke}%
\pgfsetstrokeopacity{0.391058}%
\pgfsetdash{}{0pt}%
\pgfpathmoveto{\pgfqpoint{1.601060in}{2.875641in}}%
\pgfpathcurveto{\pgfqpoint{1.609297in}{2.875641in}}{\pgfqpoint{1.617197in}{2.878913in}}{\pgfqpoint{1.623021in}{2.884737in}}%
\pgfpathcurveto{\pgfqpoint{1.628845in}{2.890561in}}{\pgfqpoint{1.632117in}{2.898461in}}{\pgfqpoint{1.632117in}{2.906697in}}%
\pgfpathcurveto{\pgfqpoint{1.632117in}{2.914933in}}{\pgfqpoint{1.628845in}{2.922834in}}{\pgfqpoint{1.623021in}{2.928657in}}%
\pgfpathcurveto{\pgfqpoint{1.617197in}{2.934481in}}{\pgfqpoint{1.609297in}{2.937754in}}{\pgfqpoint{1.601060in}{2.937754in}}%
\pgfpathcurveto{\pgfqpoint{1.592824in}{2.937754in}}{\pgfqpoint{1.584924in}{2.934481in}}{\pgfqpoint{1.579100in}{2.928657in}}%
\pgfpathcurveto{\pgfqpoint{1.573276in}{2.922834in}}{\pgfqpoint{1.570004in}{2.914933in}}{\pgfqpoint{1.570004in}{2.906697in}}%
\pgfpathcurveto{\pgfqpoint{1.570004in}{2.898461in}}{\pgfqpoint{1.573276in}{2.890561in}}{\pgfqpoint{1.579100in}{2.884737in}}%
\pgfpathcurveto{\pgfqpoint{1.584924in}{2.878913in}}{\pgfqpoint{1.592824in}{2.875641in}}{\pgfqpoint{1.601060in}{2.875641in}}%
\pgfpathclose%
\pgfusepath{stroke,fill}%
\end{pgfscope}%
\begin{pgfscope}%
\pgfpathrectangle{\pgfqpoint{0.100000in}{0.212622in}}{\pgfqpoint{3.696000in}{3.696000in}}%
\pgfusepath{clip}%
\pgfsetbuttcap%
\pgfsetroundjoin%
\definecolor{currentfill}{rgb}{0.121569,0.466667,0.705882}%
\pgfsetfillcolor{currentfill}%
\pgfsetfillopacity{0.391269}%
\pgfsetlinewidth{1.003750pt}%
\definecolor{currentstroke}{rgb}{0.121569,0.466667,0.705882}%
\pgfsetstrokecolor{currentstroke}%
\pgfsetstrokeopacity{0.391269}%
\pgfsetdash{}{0pt}%
\pgfpathmoveto{\pgfqpoint{1.972391in}{2.958868in}}%
\pgfpathcurveto{\pgfqpoint{1.980627in}{2.958868in}}{\pgfqpoint{1.988527in}{2.962141in}}{\pgfqpoint{1.994351in}{2.967964in}}%
\pgfpathcurveto{\pgfqpoint{2.000175in}{2.973788in}}{\pgfqpoint{2.003447in}{2.981688in}}{\pgfqpoint{2.003447in}{2.989925in}}%
\pgfpathcurveto{\pgfqpoint{2.003447in}{2.998161in}}{\pgfqpoint{2.000175in}{3.006061in}}{\pgfqpoint{1.994351in}{3.011885in}}%
\pgfpathcurveto{\pgfqpoint{1.988527in}{3.017709in}}{\pgfqpoint{1.980627in}{3.020981in}}{\pgfqpoint{1.972391in}{3.020981in}}%
\pgfpathcurveto{\pgfqpoint{1.964155in}{3.020981in}}{\pgfqpoint{1.956255in}{3.017709in}}{\pgfqpoint{1.950431in}{3.011885in}}%
\pgfpathcurveto{\pgfqpoint{1.944607in}{3.006061in}}{\pgfqpoint{1.941334in}{2.998161in}}{\pgfqpoint{1.941334in}{2.989925in}}%
\pgfpathcurveto{\pgfqpoint{1.941334in}{2.981688in}}{\pgfqpoint{1.944607in}{2.973788in}}{\pgfqpoint{1.950431in}{2.967964in}}%
\pgfpathcurveto{\pgfqpoint{1.956255in}{2.962141in}}{\pgfqpoint{1.964155in}{2.958868in}}{\pgfqpoint{1.972391in}{2.958868in}}%
\pgfpathclose%
\pgfusepath{stroke,fill}%
\end{pgfscope}%
\begin{pgfscope}%
\pgfpathrectangle{\pgfqpoint{0.100000in}{0.212622in}}{\pgfqpoint{3.696000in}{3.696000in}}%
\pgfusepath{clip}%
\pgfsetbuttcap%
\pgfsetroundjoin%
\definecolor{currentfill}{rgb}{0.121569,0.466667,0.705882}%
\pgfsetfillcolor{currentfill}%
\pgfsetfillopacity{0.391902}%
\pgfsetlinewidth{1.003750pt}%
\definecolor{currentstroke}{rgb}{0.121569,0.466667,0.705882}%
\pgfsetstrokecolor{currentstroke}%
\pgfsetstrokeopacity{0.391902}%
\pgfsetdash{}{0pt}%
\pgfpathmoveto{\pgfqpoint{1.598152in}{2.870749in}}%
\pgfpathcurveto{\pgfqpoint{1.606388in}{2.870749in}}{\pgfqpoint{1.614288in}{2.874022in}}{\pgfqpoint{1.620112in}{2.879846in}}%
\pgfpathcurveto{\pgfqpoint{1.625936in}{2.885670in}}{\pgfqpoint{1.629208in}{2.893570in}}{\pgfqpoint{1.629208in}{2.901806in}}%
\pgfpathcurveto{\pgfqpoint{1.629208in}{2.910042in}}{\pgfqpoint{1.625936in}{2.917942in}}{\pgfqpoint{1.620112in}{2.923766in}}%
\pgfpathcurveto{\pgfqpoint{1.614288in}{2.929590in}}{\pgfqpoint{1.606388in}{2.932862in}}{\pgfqpoint{1.598152in}{2.932862in}}%
\pgfpathcurveto{\pgfqpoint{1.589915in}{2.932862in}}{\pgfqpoint{1.582015in}{2.929590in}}{\pgfqpoint{1.576191in}{2.923766in}}%
\pgfpathcurveto{\pgfqpoint{1.570368in}{2.917942in}}{\pgfqpoint{1.567095in}{2.910042in}}{\pgfqpoint{1.567095in}{2.901806in}}%
\pgfpathcurveto{\pgfqpoint{1.567095in}{2.893570in}}{\pgfqpoint{1.570368in}{2.885670in}}{\pgfqpoint{1.576191in}{2.879846in}}%
\pgfpathcurveto{\pgfqpoint{1.582015in}{2.874022in}}{\pgfqpoint{1.589915in}{2.870749in}}{\pgfqpoint{1.598152in}{2.870749in}}%
\pgfpathclose%
\pgfusepath{stroke,fill}%
\end{pgfscope}%
\begin{pgfscope}%
\pgfpathrectangle{\pgfqpoint{0.100000in}{0.212622in}}{\pgfqpoint{3.696000in}{3.696000in}}%
\pgfusepath{clip}%
\pgfsetbuttcap%
\pgfsetroundjoin%
\definecolor{currentfill}{rgb}{0.121569,0.466667,0.705882}%
\pgfsetfillcolor{currentfill}%
\pgfsetfillopacity{0.392608}%
\pgfsetlinewidth{1.003750pt}%
\definecolor{currentstroke}{rgb}{0.121569,0.466667,0.705882}%
\pgfsetstrokecolor{currentstroke}%
\pgfsetstrokeopacity{0.392608}%
\pgfsetdash{}{0pt}%
\pgfpathmoveto{\pgfqpoint{1.596337in}{2.866786in}}%
\pgfpathcurveto{\pgfqpoint{1.604573in}{2.866786in}}{\pgfqpoint{1.612473in}{2.870058in}}{\pgfqpoint{1.618297in}{2.875882in}}%
\pgfpathcurveto{\pgfqpoint{1.624121in}{2.881706in}}{\pgfqpoint{1.627393in}{2.889606in}}{\pgfqpoint{1.627393in}{2.897842in}}%
\pgfpathcurveto{\pgfqpoint{1.627393in}{2.906079in}}{\pgfqpoint{1.624121in}{2.913979in}}{\pgfqpoint{1.618297in}{2.919803in}}%
\pgfpathcurveto{\pgfqpoint{1.612473in}{2.925627in}}{\pgfqpoint{1.604573in}{2.928899in}}{\pgfqpoint{1.596337in}{2.928899in}}%
\pgfpathcurveto{\pgfqpoint{1.588100in}{2.928899in}}{\pgfqpoint{1.580200in}{2.925627in}}{\pgfqpoint{1.574376in}{2.919803in}}%
\pgfpathcurveto{\pgfqpoint{1.568552in}{2.913979in}}{\pgfqpoint{1.565280in}{2.906079in}}{\pgfqpoint{1.565280in}{2.897842in}}%
\pgfpathcurveto{\pgfqpoint{1.565280in}{2.889606in}}{\pgfqpoint{1.568552in}{2.881706in}}{\pgfqpoint{1.574376in}{2.875882in}}%
\pgfpathcurveto{\pgfqpoint{1.580200in}{2.870058in}}{\pgfqpoint{1.588100in}{2.866786in}}{\pgfqpoint{1.596337in}{2.866786in}}%
\pgfpathclose%
\pgfusepath{stroke,fill}%
\end{pgfscope}%
\begin{pgfscope}%
\pgfpathrectangle{\pgfqpoint{0.100000in}{0.212622in}}{\pgfqpoint{3.696000in}{3.696000in}}%
\pgfusepath{clip}%
\pgfsetbuttcap%
\pgfsetroundjoin%
\definecolor{currentfill}{rgb}{0.121569,0.466667,0.705882}%
\pgfsetfillcolor{currentfill}%
\pgfsetfillopacity{0.392865}%
\pgfsetlinewidth{1.003750pt}%
\definecolor{currentstroke}{rgb}{0.121569,0.466667,0.705882}%
\pgfsetstrokecolor{currentstroke}%
\pgfsetstrokeopacity{0.392865}%
\pgfsetdash{}{0pt}%
\pgfpathmoveto{\pgfqpoint{1.595490in}{2.865413in}}%
\pgfpathcurveto{\pgfqpoint{1.603727in}{2.865413in}}{\pgfqpoint{1.611627in}{2.868685in}}{\pgfqpoint{1.617451in}{2.874509in}}%
\pgfpathcurveto{\pgfqpoint{1.623275in}{2.880333in}}{\pgfqpoint{1.626547in}{2.888233in}}{\pgfqpoint{1.626547in}{2.896470in}}%
\pgfpathcurveto{\pgfqpoint{1.626547in}{2.904706in}}{\pgfqpoint{1.623275in}{2.912606in}}{\pgfqpoint{1.617451in}{2.918430in}}%
\pgfpathcurveto{\pgfqpoint{1.611627in}{2.924254in}}{\pgfqpoint{1.603727in}{2.927526in}}{\pgfqpoint{1.595490in}{2.927526in}}%
\pgfpathcurveto{\pgfqpoint{1.587254in}{2.927526in}}{\pgfqpoint{1.579354in}{2.924254in}}{\pgfqpoint{1.573530in}{2.918430in}}%
\pgfpathcurveto{\pgfqpoint{1.567706in}{2.912606in}}{\pgfqpoint{1.564434in}{2.904706in}}{\pgfqpoint{1.564434in}{2.896470in}}%
\pgfpathcurveto{\pgfqpoint{1.564434in}{2.888233in}}{\pgfqpoint{1.567706in}{2.880333in}}{\pgfqpoint{1.573530in}{2.874509in}}%
\pgfpathcurveto{\pgfqpoint{1.579354in}{2.868685in}}{\pgfqpoint{1.587254in}{2.865413in}}{\pgfqpoint{1.595490in}{2.865413in}}%
\pgfpathclose%
\pgfusepath{stroke,fill}%
\end{pgfscope}%
\begin{pgfscope}%
\pgfpathrectangle{\pgfqpoint{0.100000in}{0.212622in}}{\pgfqpoint{3.696000in}{3.696000in}}%
\pgfusepath{clip}%
\pgfsetbuttcap%
\pgfsetroundjoin%
\definecolor{currentfill}{rgb}{0.121569,0.466667,0.705882}%
\pgfsetfillcolor{currentfill}%
\pgfsetfillopacity{0.393360}%
\pgfsetlinewidth{1.003750pt}%
\definecolor{currentstroke}{rgb}{0.121569,0.466667,0.705882}%
\pgfsetstrokecolor{currentstroke}%
\pgfsetstrokeopacity{0.393360}%
\pgfsetdash{}{0pt}%
\pgfpathmoveto{\pgfqpoint{1.594081in}{2.862844in}}%
\pgfpathcurveto{\pgfqpoint{1.602317in}{2.862844in}}{\pgfqpoint{1.610217in}{2.866116in}}{\pgfqpoint{1.616041in}{2.871940in}}%
\pgfpathcurveto{\pgfqpoint{1.621865in}{2.877764in}}{\pgfqpoint{1.625137in}{2.885664in}}{\pgfqpoint{1.625137in}{2.893900in}}%
\pgfpathcurveto{\pgfqpoint{1.625137in}{2.902136in}}{\pgfqpoint{1.621865in}{2.910037in}}{\pgfqpoint{1.616041in}{2.915860in}}%
\pgfpathcurveto{\pgfqpoint{1.610217in}{2.921684in}}{\pgfqpoint{1.602317in}{2.924957in}}{\pgfqpoint{1.594081in}{2.924957in}}%
\pgfpathcurveto{\pgfqpoint{1.585844in}{2.924957in}}{\pgfqpoint{1.577944in}{2.921684in}}{\pgfqpoint{1.572120in}{2.915860in}}%
\pgfpathcurveto{\pgfqpoint{1.566296in}{2.910037in}}{\pgfqpoint{1.563024in}{2.902136in}}{\pgfqpoint{1.563024in}{2.893900in}}%
\pgfpathcurveto{\pgfqpoint{1.563024in}{2.885664in}}{\pgfqpoint{1.566296in}{2.877764in}}{\pgfqpoint{1.572120in}{2.871940in}}%
\pgfpathcurveto{\pgfqpoint{1.577944in}{2.866116in}}{\pgfqpoint{1.585844in}{2.862844in}}{\pgfqpoint{1.594081in}{2.862844in}}%
\pgfpathclose%
\pgfusepath{stroke,fill}%
\end{pgfscope}%
\begin{pgfscope}%
\pgfpathrectangle{\pgfqpoint{0.100000in}{0.212622in}}{\pgfqpoint{3.696000in}{3.696000in}}%
\pgfusepath{clip}%
\pgfsetbuttcap%
\pgfsetroundjoin%
\definecolor{currentfill}{rgb}{0.121569,0.466667,0.705882}%
\pgfsetfillcolor{currentfill}%
\pgfsetfillopacity{0.393474}%
\pgfsetlinewidth{1.003750pt}%
\definecolor{currentstroke}{rgb}{0.121569,0.466667,0.705882}%
\pgfsetstrokecolor{currentstroke}%
\pgfsetstrokeopacity{0.393474}%
\pgfsetdash{}{0pt}%
\pgfpathmoveto{\pgfqpoint{1.973733in}{2.950452in}}%
\pgfpathcurveto{\pgfqpoint{1.981969in}{2.950452in}}{\pgfqpoint{1.989869in}{2.953725in}}{\pgfqpoint{1.995693in}{2.959548in}}%
\pgfpathcurveto{\pgfqpoint{2.001517in}{2.965372in}}{\pgfqpoint{2.004789in}{2.973272in}}{\pgfqpoint{2.004789in}{2.981509in}}%
\pgfpathcurveto{\pgfqpoint{2.004789in}{2.989745in}}{\pgfqpoint{2.001517in}{2.997645in}}{\pgfqpoint{1.995693in}{3.003469in}}%
\pgfpathcurveto{\pgfqpoint{1.989869in}{3.009293in}}{\pgfqpoint{1.981969in}{3.012565in}}{\pgfqpoint{1.973733in}{3.012565in}}%
\pgfpathcurveto{\pgfqpoint{1.965497in}{3.012565in}}{\pgfqpoint{1.957597in}{3.009293in}}{\pgfqpoint{1.951773in}{3.003469in}}%
\pgfpathcurveto{\pgfqpoint{1.945949in}{2.997645in}}{\pgfqpoint{1.942676in}{2.989745in}}{\pgfqpoint{1.942676in}{2.981509in}}%
\pgfpathcurveto{\pgfqpoint{1.942676in}{2.973272in}}{\pgfqpoint{1.945949in}{2.965372in}}{\pgfqpoint{1.951773in}{2.959548in}}%
\pgfpathcurveto{\pgfqpoint{1.957597in}{2.953725in}}{\pgfqpoint{1.965497in}{2.950452in}}{\pgfqpoint{1.973733in}{2.950452in}}%
\pgfpathclose%
\pgfusepath{stroke,fill}%
\end{pgfscope}%
\begin{pgfscope}%
\pgfpathrectangle{\pgfqpoint{0.100000in}{0.212622in}}{\pgfqpoint{3.696000in}{3.696000in}}%
\pgfusepath{clip}%
\pgfsetbuttcap%
\pgfsetroundjoin%
\definecolor{currentfill}{rgb}{0.121569,0.466667,0.705882}%
\pgfsetfillcolor{currentfill}%
\pgfsetfillopacity{0.394272}%
\pgfsetlinewidth{1.003750pt}%
\definecolor{currentstroke}{rgb}{0.121569,0.466667,0.705882}%
\pgfsetstrokecolor{currentstroke}%
\pgfsetstrokeopacity{0.394272}%
\pgfsetdash{}{0pt}%
\pgfpathmoveto{\pgfqpoint{1.591556in}{2.858166in}}%
\pgfpathcurveto{\pgfqpoint{1.599792in}{2.858166in}}{\pgfqpoint{1.607692in}{2.861438in}}{\pgfqpoint{1.613516in}{2.867262in}}%
\pgfpathcurveto{\pgfqpoint{1.619340in}{2.873086in}}{\pgfqpoint{1.622612in}{2.880986in}}{\pgfqpoint{1.622612in}{2.889222in}}%
\pgfpathcurveto{\pgfqpoint{1.622612in}{2.897458in}}{\pgfqpoint{1.619340in}{2.905359in}}{\pgfqpoint{1.613516in}{2.911182in}}%
\pgfpathcurveto{\pgfqpoint{1.607692in}{2.917006in}}{\pgfqpoint{1.599792in}{2.920279in}}{\pgfqpoint{1.591556in}{2.920279in}}%
\pgfpathcurveto{\pgfqpoint{1.583319in}{2.920279in}}{\pgfqpoint{1.575419in}{2.917006in}}{\pgfqpoint{1.569595in}{2.911182in}}%
\pgfpathcurveto{\pgfqpoint{1.563772in}{2.905359in}}{\pgfqpoint{1.560499in}{2.897458in}}{\pgfqpoint{1.560499in}{2.889222in}}%
\pgfpathcurveto{\pgfqpoint{1.560499in}{2.880986in}}{\pgfqpoint{1.563772in}{2.873086in}}{\pgfqpoint{1.569595in}{2.867262in}}%
\pgfpathcurveto{\pgfqpoint{1.575419in}{2.861438in}}{\pgfqpoint{1.583319in}{2.858166in}}{\pgfqpoint{1.591556in}{2.858166in}}%
\pgfpathclose%
\pgfusepath{stroke,fill}%
\end{pgfscope}%
\begin{pgfscope}%
\pgfpathrectangle{\pgfqpoint{0.100000in}{0.212622in}}{\pgfqpoint{3.696000in}{3.696000in}}%
\pgfusepath{clip}%
\pgfsetbuttcap%
\pgfsetroundjoin%
\definecolor{currentfill}{rgb}{0.121569,0.466667,0.705882}%
\pgfsetfillcolor{currentfill}%
\pgfsetfillopacity{0.394883}%
\pgfsetlinewidth{1.003750pt}%
\definecolor{currentstroke}{rgb}{0.121569,0.466667,0.705882}%
\pgfsetstrokecolor{currentstroke}%
\pgfsetstrokeopacity{0.394883}%
\pgfsetdash{}{0pt}%
\pgfpathmoveto{\pgfqpoint{1.589336in}{2.854601in}}%
\pgfpathcurveto{\pgfqpoint{1.597572in}{2.854601in}}{\pgfqpoint{1.605472in}{2.857873in}}{\pgfqpoint{1.611296in}{2.863697in}}%
\pgfpathcurveto{\pgfqpoint{1.617120in}{2.869521in}}{\pgfqpoint{1.620392in}{2.877421in}}{\pgfqpoint{1.620392in}{2.885657in}}%
\pgfpathcurveto{\pgfqpoint{1.620392in}{2.893894in}}{\pgfqpoint{1.617120in}{2.901794in}}{\pgfqpoint{1.611296in}{2.907618in}}%
\pgfpathcurveto{\pgfqpoint{1.605472in}{2.913442in}}{\pgfqpoint{1.597572in}{2.916714in}}{\pgfqpoint{1.589336in}{2.916714in}}%
\pgfpathcurveto{\pgfqpoint{1.581100in}{2.916714in}}{\pgfqpoint{1.573200in}{2.913442in}}{\pgfqpoint{1.567376in}{2.907618in}}%
\pgfpathcurveto{\pgfqpoint{1.561552in}{2.901794in}}{\pgfqpoint{1.558279in}{2.893894in}}{\pgfqpoint{1.558279in}{2.885657in}}%
\pgfpathcurveto{\pgfqpoint{1.558279in}{2.877421in}}{\pgfqpoint{1.561552in}{2.869521in}}{\pgfqpoint{1.567376in}{2.863697in}}%
\pgfpathcurveto{\pgfqpoint{1.573200in}{2.857873in}}{\pgfqpoint{1.581100in}{2.854601in}}{\pgfqpoint{1.589336in}{2.854601in}}%
\pgfpathclose%
\pgfusepath{stroke,fill}%
\end{pgfscope}%
\begin{pgfscope}%
\pgfpathrectangle{\pgfqpoint{0.100000in}{0.212622in}}{\pgfqpoint{3.696000in}{3.696000in}}%
\pgfusepath{clip}%
\pgfsetbuttcap%
\pgfsetroundjoin%
\definecolor{currentfill}{rgb}{0.121569,0.466667,0.705882}%
\pgfsetfillcolor{currentfill}%
\pgfsetfillopacity{0.395271}%
\pgfsetlinewidth{1.003750pt}%
\definecolor{currentstroke}{rgb}{0.121569,0.466667,0.705882}%
\pgfsetstrokecolor{currentstroke}%
\pgfsetstrokeopacity{0.395271}%
\pgfsetdash{}{0pt}%
\pgfpathmoveto{\pgfqpoint{1.588263in}{2.852455in}}%
\pgfpathcurveto{\pgfqpoint{1.596499in}{2.852455in}}{\pgfqpoint{1.604399in}{2.855728in}}{\pgfqpoint{1.610223in}{2.861552in}}%
\pgfpathcurveto{\pgfqpoint{1.616047in}{2.867376in}}{\pgfqpoint{1.619319in}{2.875276in}}{\pgfqpoint{1.619319in}{2.883512in}}%
\pgfpathcurveto{\pgfqpoint{1.619319in}{2.891748in}}{\pgfqpoint{1.616047in}{2.899648in}}{\pgfqpoint{1.610223in}{2.905472in}}%
\pgfpathcurveto{\pgfqpoint{1.604399in}{2.911296in}}{\pgfqpoint{1.596499in}{2.914568in}}{\pgfqpoint{1.588263in}{2.914568in}}%
\pgfpathcurveto{\pgfqpoint{1.580026in}{2.914568in}}{\pgfqpoint{1.572126in}{2.911296in}}{\pgfqpoint{1.566302in}{2.905472in}}%
\pgfpathcurveto{\pgfqpoint{1.560478in}{2.899648in}}{\pgfqpoint{1.557206in}{2.891748in}}{\pgfqpoint{1.557206in}{2.883512in}}%
\pgfpathcurveto{\pgfqpoint{1.557206in}{2.875276in}}{\pgfqpoint{1.560478in}{2.867376in}}{\pgfqpoint{1.566302in}{2.861552in}}%
\pgfpathcurveto{\pgfqpoint{1.572126in}{2.855728in}}{\pgfqpoint{1.580026in}{2.852455in}}{\pgfqpoint{1.588263in}{2.852455in}}%
\pgfpathclose%
\pgfusepath{stroke,fill}%
\end{pgfscope}%
\begin{pgfscope}%
\pgfpathrectangle{\pgfqpoint{0.100000in}{0.212622in}}{\pgfqpoint{3.696000in}{3.696000in}}%
\pgfusepath{clip}%
\pgfsetbuttcap%
\pgfsetroundjoin%
\definecolor{currentfill}{rgb}{0.121569,0.466667,0.705882}%
\pgfsetfillcolor{currentfill}%
\pgfsetfillopacity{0.395959}%
\pgfsetlinewidth{1.003750pt}%
\definecolor{currentstroke}{rgb}{0.121569,0.466667,0.705882}%
\pgfsetstrokecolor{currentstroke}%
\pgfsetstrokeopacity{0.395959}%
\pgfsetdash{}{0pt}%
\pgfpathmoveto{\pgfqpoint{1.586118in}{2.848725in}}%
\pgfpathcurveto{\pgfqpoint{1.594355in}{2.848725in}}{\pgfqpoint{1.602255in}{2.851997in}}{\pgfqpoint{1.608079in}{2.857821in}}%
\pgfpathcurveto{\pgfqpoint{1.613902in}{2.863645in}}{\pgfqpoint{1.617175in}{2.871545in}}{\pgfqpoint{1.617175in}{2.879781in}}%
\pgfpathcurveto{\pgfqpoint{1.617175in}{2.888018in}}{\pgfqpoint{1.613902in}{2.895918in}}{\pgfqpoint{1.608079in}{2.901741in}}%
\pgfpathcurveto{\pgfqpoint{1.602255in}{2.907565in}}{\pgfqpoint{1.594355in}{2.910838in}}{\pgfqpoint{1.586118in}{2.910838in}}%
\pgfpathcurveto{\pgfqpoint{1.577882in}{2.910838in}}{\pgfqpoint{1.569982in}{2.907565in}}{\pgfqpoint{1.564158in}{2.901741in}}%
\pgfpathcurveto{\pgfqpoint{1.558334in}{2.895918in}}{\pgfqpoint{1.555062in}{2.888018in}}{\pgfqpoint{1.555062in}{2.879781in}}%
\pgfpathcurveto{\pgfqpoint{1.555062in}{2.871545in}}{\pgfqpoint{1.558334in}{2.863645in}}{\pgfqpoint{1.564158in}{2.857821in}}%
\pgfpathcurveto{\pgfqpoint{1.569982in}{2.851997in}}{\pgfqpoint{1.577882in}{2.848725in}}{\pgfqpoint{1.586118in}{2.848725in}}%
\pgfpathclose%
\pgfusepath{stroke,fill}%
\end{pgfscope}%
\begin{pgfscope}%
\pgfpathrectangle{\pgfqpoint{0.100000in}{0.212622in}}{\pgfqpoint{3.696000in}{3.696000in}}%
\pgfusepath{clip}%
\pgfsetbuttcap%
\pgfsetroundjoin%
\definecolor{currentfill}{rgb}{0.121569,0.466667,0.705882}%
\pgfsetfillcolor{currentfill}%
\pgfsetfillopacity{0.395988}%
\pgfsetlinewidth{1.003750pt}%
\definecolor{currentstroke}{rgb}{0.121569,0.466667,0.705882}%
\pgfsetstrokecolor{currentstroke}%
\pgfsetstrokeopacity{0.395988}%
\pgfsetdash{}{0pt}%
\pgfpathmoveto{\pgfqpoint{1.974747in}{2.941476in}}%
\pgfpathcurveto{\pgfqpoint{1.982983in}{2.941476in}}{\pgfqpoint{1.990883in}{2.944748in}}{\pgfqpoint{1.996707in}{2.950572in}}%
\pgfpathcurveto{\pgfqpoint{2.002531in}{2.956396in}}{\pgfqpoint{2.005803in}{2.964296in}}{\pgfqpoint{2.005803in}{2.972532in}}%
\pgfpathcurveto{\pgfqpoint{2.005803in}{2.980769in}}{\pgfqpoint{2.002531in}{2.988669in}}{\pgfqpoint{1.996707in}{2.994493in}}%
\pgfpathcurveto{\pgfqpoint{1.990883in}{3.000316in}}{\pgfqpoint{1.982983in}{3.003589in}}{\pgfqpoint{1.974747in}{3.003589in}}%
\pgfpathcurveto{\pgfqpoint{1.966511in}{3.003589in}}{\pgfqpoint{1.958611in}{3.000316in}}{\pgfqpoint{1.952787in}{2.994493in}}%
\pgfpathcurveto{\pgfqpoint{1.946963in}{2.988669in}}{\pgfqpoint{1.943690in}{2.980769in}}{\pgfqpoint{1.943690in}{2.972532in}}%
\pgfpathcurveto{\pgfqpoint{1.943690in}{2.964296in}}{\pgfqpoint{1.946963in}{2.956396in}}{\pgfqpoint{1.952787in}{2.950572in}}%
\pgfpathcurveto{\pgfqpoint{1.958611in}{2.944748in}}{\pgfqpoint{1.966511in}{2.941476in}}{\pgfqpoint{1.974747in}{2.941476in}}%
\pgfpathclose%
\pgfusepath{stroke,fill}%
\end{pgfscope}%
\begin{pgfscope}%
\pgfpathrectangle{\pgfqpoint{0.100000in}{0.212622in}}{\pgfqpoint{3.696000in}{3.696000in}}%
\pgfusepath{clip}%
\pgfsetbuttcap%
\pgfsetroundjoin%
\definecolor{currentfill}{rgb}{0.121569,0.466667,0.705882}%
\pgfsetfillcolor{currentfill}%
\pgfsetfillopacity{0.397118}%
\pgfsetlinewidth{1.003750pt}%
\definecolor{currentstroke}{rgb}{0.121569,0.466667,0.705882}%
\pgfsetstrokecolor{currentstroke}%
\pgfsetstrokeopacity{0.397118}%
\pgfsetdash{}{0pt}%
\pgfpathmoveto{\pgfqpoint{1.582053in}{2.841834in}}%
\pgfpathcurveto{\pgfqpoint{1.590289in}{2.841834in}}{\pgfqpoint{1.598189in}{2.845106in}}{\pgfqpoint{1.604013in}{2.850930in}}%
\pgfpathcurveto{\pgfqpoint{1.609837in}{2.856754in}}{\pgfqpoint{1.613109in}{2.864654in}}{\pgfqpoint{1.613109in}{2.872891in}}%
\pgfpathcurveto{\pgfqpoint{1.613109in}{2.881127in}}{\pgfqpoint{1.609837in}{2.889027in}}{\pgfqpoint{1.604013in}{2.894851in}}%
\pgfpathcurveto{\pgfqpoint{1.598189in}{2.900675in}}{\pgfqpoint{1.590289in}{2.903947in}}{\pgfqpoint{1.582053in}{2.903947in}}%
\pgfpathcurveto{\pgfqpoint{1.573816in}{2.903947in}}{\pgfqpoint{1.565916in}{2.900675in}}{\pgfqpoint{1.560092in}{2.894851in}}%
\pgfpathcurveto{\pgfqpoint{1.554268in}{2.889027in}}{\pgfqpoint{1.550996in}{2.881127in}}{\pgfqpoint{1.550996in}{2.872891in}}%
\pgfpathcurveto{\pgfqpoint{1.550996in}{2.864654in}}{\pgfqpoint{1.554268in}{2.856754in}}{\pgfqpoint{1.560092in}{2.850930in}}%
\pgfpathcurveto{\pgfqpoint{1.565916in}{2.845106in}}{\pgfqpoint{1.573816in}{2.841834in}}{\pgfqpoint{1.582053in}{2.841834in}}%
\pgfpathclose%
\pgfusepath{stroke,fill}%
\end{pgfscope}%
\begin{pgfscope}%
\pgfpathrectangle{\pgfqpoint{0.100000in}{0.212622in}}{\pgfqpoint{3.696000in}{3.696000in}}%
\pgfusepath{clip}%
\pgfsetbuttcap%
\pgfsetroundjoin%
\definecolor{currentfill}{rgb}{0.121569,0.466667,0.705882}%
\pgfsetfillcolor{currentfill}%
\pgfsetfillopacity{0.398199}%
\pgfsetlinewidth{1.003750pt}%
\definecolor{currentstroke}{rgb}{0.121569,0.466667,0.705882}%
\pgfsetstrokecolor{currentstroke}%
\pgfsetstrokeopacity{0.398199}%
\pgfsetdash{}{0pt}%
\pgfpathmoveto{\pgfqpoint{1.579108in}{2.835371in}}%
\pgfpathcurveto{\pgfqpoint{1.587345in}{2.835371in}}{\pgfqpoint{1.595245in}{2.838643in}}{\pgfqpoint{1.601069in}{2.844467in}}%
\pgfpathcurveto{\pgfqpoint{1.606893in}{2.850291in}}{\pgfqpoint{1.610165in}{2.858191in}}{\pgfqpoint{1.610165in}{2.866427in}}%
\pgfpathcurveto{\pgfqpoint{1.610165in}{2.874664in}}{\pgfqpoint{1.606893in}{2.882564in}}{\pgfqpoint{1.601069in}{2.888388in}}%
\pgfpathcurveto{\pgfqpoint{1.595245in}{2.894211in}}{\pgfqpoint{1.587345in}{2.897484in}}{\pgfqpoint{1.579108in}{2.897484in}}%
\pgfpathcurveto{\pgfqpoint{1.570872in}{2.897484in}}{\pgfqpoint{1.562972in}{2.894211in}}{\pgfqpoint{1.557148in}{2.888388in}}%
\pgfpathcurveto{\pgfqpoint{1.551324in}{2.882564in}}{\pgfqpoint{1.548052in}{2.874664in}}{\pgfqpoint{1.548052in}{2.866427in}}%
\pgfpathcurveto{\pgfqpoint{1.548052in}{2.858191in}}{\pgfqpoint{1.551324in}{2.850291in}}{\pgfqpoint{1.557148in}{2.844467in}}%
\pgfpathcurveto{\pgfqpoint{1.562972in}{2.838643in}}{\pgfqpoint{1.570872in}{2.835371in}}{\pgfqpoint{1.579108in}{2.835371in}}%
\pgfpathclose%
\pgfusepath{stroke,fill}%
\end{pgfscope}%
\begin{pgfscope}%
\pgfpathrectangle{\pgfqpoint{0.100000in}{0.212622in}}{\pgfqpoint{3.696000in}{3.696000in}}%
\pgfusepath{clip}%
\pgfsetbuttcap%
\pgfsetroundjoin%
\definecolor{currentfill}{rgb}{0.121569,0.466667,0.705882}%
\pgfsetfillcolor{currentfill}%
\pgfsetfillopacity{0.398280}%
\pgfsetlinewidth{1.003750pt}%
\definecolor{currentstroke}{rgb}{0.121569,0.466667,0.705882}%
\pgfsetstrokecolor{currentstroke}%
\pgfsetstrokeopacity{0.398280}%
\pgfsetdash{}{0pt}%
\pgfpathmoveto{\pgfqpoint{1.976630in}{2.931614in}}%
\pgfpathcurveto{\pgfqpoint{1.984866in}{2.931614in}}{\pgfqpoint{1.992766in}{2.934886in}}{\pgfqpoint{1.998590in}{2.940710in}}%
\pgfpathcurveto{\pgfqpoint{2.004414in}{2.946534in}}{\pgfqpoint{2.007686in}{2.954434in}}{\pgfqpoint{2.007686in}{2.962670in}}%
\pgfpathcurveto{\pgfqpoint{2.007686in}{2.970907in}}{\pgfqpoint{2.004414in}{2.978807in}}{\pgfqpoint{1.998590in}{2.984631in}}%
\pgfpathcurveto{\pgfqpoint{1.992766in}{2.990454in}}{\pgfqpoint{1.984866in}{2.993727in}}{\pgfqpoint{1.976630in}{2.993727in}}%
\pgfpathcurveto{\pgfqpoint{1.968393in}{2.993727in}}{\pgfqpoint{1.960493in}{2.990454in}}{\pgfqpoint{1.954669in}{2.984631in}}%
\pgfpathcurveto{\pgfqpoint{1.948846in}{2.978807in}}{\pgfqpoint{1.945573in}{2.970907in}}{\pgfqpoint{1.945573in}{2.962670in}}%
\pgfpathcurveto{\pgfqpoint{1.945573in}{2.954434in}}{\pgfqpoint{1.948846in}{2.946534in}}{\pgfqpoint{1.954669in}{2.940710in}}%
\pgfpathcurveto{\pgfqpoint{1.960493in}{2.934886in}}{\pgfqpoint{1.968393in}{2.931614in}}{\pgfqpoint{1.976630in}{2.931614in}}%
\pgfpathclose%
\pgfusepath{stroke,fill}%
\end{pgfscope}%
\begin{pgfscope}%
\pgfpathrectangle{\pgfqpoint{0.100000in}{0.212622in}}{\pgfqpoint{3.696000in}{3.696000in}}%
\pgfusepath{clip}%
\pgfsetbuttcap%
\pgfsetroundjoin%
\definecolor{currentfill}{rgb}{0.121569,0.466667,0.705882}%
\pgfsetfillcolor{currentfill}%
\pgfsetfillopacity{0.398875}%
\pgfsetlinewidth{1.003750pt}%
\definecolor{currentstroke}{rgb}{0.121569,0.466667,0.705882}%
\pgfsetstrokecolor{currentstroke}%
\pgfsetstrokeopacity{0.398875}%
\pgfsetdash{}{0pt}%
\pgfpathmoveto{\pgfqpoint{1.576858in}{2.831779in}}%
\pgfpathcurveto{\pgfqpoint{1.585094in}{2.831779in}}{\pgfqpoint{1.592994in}{2.835052in}}{\pgfqpoint{1.598818in}{2.840875in}}%
\pgfpathcurveto{\pgfqpoint{1.604642in}{2.846699in}}{\pgfqpoint{1.607914in}{2.854599in}}{\pgfqpoint{1.607914in}{2.862836in}}%
\pgfpathcurveto{\pgfqpoint{1.607914in}{2.871072in}}{\pgfqpoint{1.604642in}{2.878972in}}{\pgfqpoint{1.598818in}{2.884796in}}%
\pgfpathcurveto{\pgfqpoint{1.592994in}{2.890620in}}{\pgfqpoint{1.585094in}{2.893892in}}{\pgfqpoint{1.576858in}{2.893892in}}%
\pgfpathcurveto{\pgfqpoint{1.568621in}{2.893892in}}{\pgfqpoint{1.560721in}{2.890620in}}{\pgfqpoint{1.554897in}{2.884796in}}%
\pgfpathcurveto{\pgfqpoint{1.549073in}{2.878972in}}{\pgfqpoint{1.545801in}{2.871072in}}{\pgfqpoint{1.545801in}{2.862836in}}%
\pgfpathcurveto{\pgfqpoint{1.545801in}{2.854599in}}{\pgfqpoint{1.549073in}{2.846699in}}{\pgfqpoint{1.554897in}{2.840875in}}%
\pgfpathcurveto{\pgfqpoint{1.560721in}{2.835052in}}{\pgfqpoint{1.568621in}{2.831779in}}{\pgfqpoint{1.576858in}{2.831779in}}%
\pgfpathclose%
\pgfusepath{stroke,fill}%
\end{pgfscope}%
\begin{pgfscope}%
\pgfpathrectangle{\pgfqpoint{0.100000in}{0.212622in}}{\pgfqpoint{3.696000in}{3.696000in}}%
\pgfusepath{clip}%
\pgfsetbuttcap%
\pgfsetroundjoin%
\definecolor{currentfill}{rgb}{0.121569,0.466667,0.705882}%
\pgfsetfillcolor{currentfill}%
\pgfsetfillopacity{0.399465}%
\pgfsetlinewidth{1.003750pt}%
\definecolor{currentstroke}{rgb}{0.121569,0.466667,0.705882}%
\pgfsetstrokecolor{currentstroke}%
\pgfsetstrokeopacity{0.399465}%
\pgfsetdash{}{0pt}%
\pgfpathmoveto{\pgfqpoint{1.575165in}{2.828641in}}%
\pgfpathcurveto{\pgfqpoint{1.583402in}{2.828641in}}{\pgfqpoint{1.591302in}{2.831913in}}{\pgfqpoint{1.597126in}{2.837737in}}%
\pgfpathcurveto{\pgfqpoint{1.602949in}{2.843561in}}{\pgfqpoint{1.606222in}{2.851461in}}{\pgfqpoint{1.606222in}{2.859697in}}%
\pgfpathcurveto{\pgfqpoint{1.606222in}{2.867934in}}{\pgfqpoint{1.602949in}{2.875834in}}{\pgfqpoint{1.597126in}{2.881658in}}%
\pgfpathcurveto{\pgfqpoint{1.591302in}{2.887482in}}{\pgfqpoint{1.583402in}{2.890754in}}{\pgfqpoint{1.575165in}{2.890754in}}%
\pgfpathcurveto{\pgfqpoint{1.566929in}{2.890754in}}{\pgfqpoint{1.559029in}{2.887482in}}{\pgfqpoint{1.553205in}{2.881658in}}%
\pgfpathcurveto{\pgfqpoint{1.547381in}{2.875834in}}{\pgfqpoint{1.544109in}{2.867934in}}{\pgfqpoint{1.544109in}{2.859697in}}%
\pgfpathcurveto{\pgfqpoint{1.544109in}{2.851461in}}{\pgfqpoint{1.547381in}{2.843561in}}{\pgfqpoint{1.553205in}{2.837737in}}%
\pgfpathcurveto{\pgfqpoint{1.559029in}{2.831913in}}{\pgfqpoint{1.566929in}{2.828641in}}{\pgfqpoint{1.575165in}{2.828641in}}%
\pgfpathclose%
\pgfusepath{stroke,fill}%
\end{pgfscope}%
\begin{pgfscope}%
\pgfpathrectangle{\pgfqpoint{0.100000in}{0.212622in}}{\pgfqpoint{3.696000in}{3.696000in}}%
\pgfusepath{clip}%
\pgfsetbuttcap%
\pgfsetroundjoin%
\definecolor{currentfill}{rgb}{0.121569,0.466667,0.705882}%
\pgfsetfillcolor{currentfill}%
\pgfsetfillopacity{0.400570}%
\pgfsetlinewidth{1.003750pt}%
\definecolor{currentstroke}{rgb}{0.121569,0.466667,0.705882}%
\pgfsetstrokecolor{currentstroke}%
\pgfsetstrokeopacity{0.400570}%
\pgfsetdash{}{0pt}%
\pgfpathmoveto{\pgfqpoint{1.571986in}{2.823193in}}%
\pgfpathcurveto{\pgfqpoint{1.580222in}{2.823193in}}{\pgfqpoint{1.588122in}{2.826465in}}{\pgfqpoint{1.593946in}{2.832289in}}%
\pgfpathcurveto{\pgfqpoint{1.599770in}{2.838113in}}{\pgfqpoint{1.603042in}{2.846013in}}{\pgfqpoint{1.603042in}{2.854249in}}%
\pgfpathcurveto{\pgfqpoint{1.603042in}{2.862485in}}{\pgfqpoint{1.599770in}{2.870385in}}{\pgfqpoint{1.593946in}{2.876209in}}%
\pgfpathcurveto{\pgfqpoint{1.588122in}{2.882033in}}{\pgfqpoint{1.580222in}{2.885306in}}{\pgfqpoint{1.571986in}{2.885306in}}%
\pgfpathcurveto{\pgfqpoint{1.563749in}{2.885306in}}{\pgfqpoint{1.555849in}{2.882033in}}{\pgfqpoint{1.550025in}{2.876209in}}%
\pgfpathcurveto{\pgfqpoint{1.544201in}{2.870385in}}{\pgfqpoint{1.540929in}{2.862485in}}{\pgfqpoint{1.540929in}{2.854249in}}%
\pgfpathcurveto{\pgfqpoint{1.540929in}{2.846013in}}{\pgfqpoint{1.544201in}{2.838113in}}{\pgfqpoint{1.550025in}{2.832289in}}%
\pgfpathcurveto{\pgfqpoint{1.555849in}{2.826465in}}{\pgfqpoint{1.563749in}{2.823193in}}{\pgfqpoint{1.571986in}{2.823193in}}%
\pgfpathclose%
\pgfusepath{stroke,fill}%
\end{pgfscope}%
\begin{pgfscope}%
\pgfpathrectangle{\pgfqpoint{0.100000in}{0.212622in}}{\pgfqpoint{3.696000in}{3.696000in}}%
\pgfusepath{clip}%
\pgfsetbuttcap%
\pgfsetroundjoin%
\definecolor{currentfill}{rgb}{0.121569,0.466667,0.705882}%
\pgfsetfillcolor{currentfill}%
\pgfsetfillopacity{0.400833}%
\pgfsetlinewidth{1.003750pt}%
\definecolor{currentstroke}{rgb}{0.121569,0.466667,0.705882}%
\pgfsetstrokecolor{currentstroke}%
\pgfsetstrokeopacity{0.400833}%
\pgfsetdash{}{0pt}%
\pgfpathmoveto{\pgfqpoint{1.978139in}{2.920853in}}%
\pgfpathcurveto{\pgfqpoint{1.986375in}{2.920853in}}{\pgfqpoint{1.994275in}{2.924125in}}{\pgfqpoint{2.000099in}{2.929949in}}%
\pgfpathcurveto{\pgfqpoint{2.005923in}{2.935773in}}{\pgfqpoint{2.009195in}{2.943673in}}{\pgfqpoint{2.009195in}{2.951909in}}%
\pgfpathcurveto{\pgfqpoint{2.009195in}{2.960146in}}{\pgfqpoint{2.005923in}{2.968046in}}{\pgfqpoint{2.000099in}{2.973870in}}%
\pgfpathcurveto{\pgfqpoint{1.994275in}{2.979694in}}{\pgfqpoint{1.986375in}{2.982966in}}{\pgfqpoint{1.978139in}{2.982966in}}%
\pgfpathcurveto{\pgfqpoint{1.969903in}{2.982966in}}{\pgfqpoint{1.962003in}{2.979694in}}{\pgfqpoint{1.956179in}{2.973870in}}%
\pgfpathcurveto{\pgfqpoint{1.950355in}{2.968046in}}{\pgfqpoint{1.947082in}{2.960146in}}{\pgfqpoint{1.947082in}{2.951909in}}%
\pgfpathcurveto{\pgfqpoint{1.947082in}{2.943673in}}{\pgfqpoint{1.950355in}{2.935773in}}{\pgfqpoint{1.956179in}{2.929949in}}%
\pgfpathcurveto{\pgfqpoint{1.962003in}{2.924125in}}{\pgfqpoint{1.969903in}{2.920853in}}{\pgfqpoint{1.978139in}{2.920853in}}%
\pgfpathclose%
\pgfusepath{stroke,fill}%
\end{pgfscope}%
\begin{pgfscope}%
\pgfpathrectangle{\pgfqpoint{0.100000in}{0.212622in}}{\pgfqpoint{3.696000in}{3.696000in}}%
\pgfusepath{clip}%
\pgfsetbuttcap%
\pgfsetroundjoin%
\definecolor{currentfill}{rgb}{0.121569,0.466667,0.705882}%
\pgfsetfillcolor{currentfill}%
\pgfsetfillopacity{0.401464}%
\pgfsetlinewidth{1.003750pt}%
\definecolor{currentstroke}{rgb}{0.121569,0.466667,0.705882}%
\pgfsetstrokecolor{currentstroke}%
\pgfsetstrokeopacity{0.401464}%
\pgfsetdash{}{0pt}%
\pgfpathmoveto{\pgfqpoint{1.568686in}{2.817970in}}%
\pgfpathcurveto{\pgfqpoint{1.576922in}{2.817970in}}{\pgfqpoint{1.584822in}{2.821242in}}{\pgfqpoint{1.590646in}{2.827066in}}%
\pgfpathcurveto{\pgfqpoint{1.596470in}{2.832890in}}{\pgfqpoint{1.599742in}{2.840790in}}{\pgfqpoint{1.599742in}{2.849026in}}%
\pgfpathcurveto{\pgfqpoint{1.599742in}{2.857262in}}{\pgfqpoint{1.596470in}{2.865162in}}{\pgfqpoint{1.590646in}{2.870986in}}%
\pgfpathcurveto{\pgfqpoint{1.584822in}{2.876810in}}{\pgfqpoint{1.576922in}{2.880083in}}{\pgfqpoint{1.568686in}{2.880083in}}%
\pgfpathcurveto{\pgfqpoint{1.560449in}{2.880083in}}{\pgfqpoint{1.552549in}{2.876810in}}{\pgfqpoint{1.546725in}{2.870986in}}%
\pgfpathcurveto{\pgfqpoint{1.540902in}{2.865162in}}{\pgfqpoint{1.537629in}{2.857262in}}{\pgfqpoint{1.537629in}{2.849026in}}%
\pgfpathcurveto{\pgfqpoint{1.537629in}{2.840790in}}{\pgfqpoint{1.540902in}{2.832890in}}{\pgfqpoint{1.546725in}{2.827066in}}%
\pgfpathcurveto{\pgfqpoint{1.552549in}{2.821242in}}{\pgfqpoint{1.560449in}{2.817970in}}{\pgfqpoint{1.568686in}{2.817970in}}%
\pgfpathclose%
\pgfusepath{stroke,fill}%
\end{pgfscope}%
\begin{pgfscope}%
\pgfpathrectangle{\pgfqpoint{0.100000in}{0.212622in}}{\pgfqpoint{3.696000in}{3.696000in}}%
\pgfusepath{clip}%
\pgfsetbuttcap%
\pgfsetroundjoin%
\definecolor{currentfill}{rgb}{0.121569,0.466667,0.705882}%
\pgfsetfillcolor{currentfill}%
\pgfsetfillopacity{0.402243}%
\pgfsetlinewidth{1.003750pt}%
\definecolor{currentstroke}{rgb}{0.121569,0.466667,0.705882}%
\pgfsetstrokecolor{currentstroke}%
\pgfsetstrokeopacity{0.402243}%
\pgfsetdash{}{0pt}%
\pgfpathmoveto{\pgfqpoint{1.566640in}{2.813496in}}%
\pgfpathcurveto{\pgfqpoint{1.574876in}{2.813496in}}{\pgfqpoint{1.582776in}{2.816768in}}{\pgfqpoint{1.588600in}{2.822592in}}%
\pgfpathcurveto{\pgfqpoint{1.594424in}{2.828416in}}{\pgfqpoint{1.597697in}{2.836316in}}{\pgfqpoint{1.597697in}{2.844553in}}%
\pgfpathcurveto{\pgfqpoint{1.597697in}{2.852789in}}{\pgfqpoint{1.594424in}{2.860689in}}{\pgfqpoint{1.588600in}{2.866513in}}%
\pgfpathcurveto{\pgfqpoint{1.582776in}{2.872337in}}{\pgfqpoint{1.574876in}{2.875609in}}{\pgfqpoint{1.566640in}{2.875609in}}%
\pgfpathcurveto{\pgfqpoint{1.558404in}{2.875609in}}{\pgfqpoint{1.550504in}{2.872337in}}{\pgfqpoint{1.544680in}{2.866513in}}%
\pgfpathcurveto{\pgfqpoint{1.538856in}{2.860689in}}{\pgfqpoint{1.535584in}{2.852789in}}{\pgfqpoint{1.535584in}{2.844553in}}%
\pgfpathcurveto{\pgfqpoint{1.535584in}{2.836316in}}{\pgfqpoint{1.538856in}{2.828416in}}{\pgfqpoint{1.544680in}{2.822592in}}%
\pgfpathcurveto{\pgfqpoint{1.550504in}{2.816768in}}{\pgfqpoint{1.558404in}{2.813496in}}{\pgfqpoint{1.566640in}{2.813496in}}%
\pgfpathclose%
\pgfusepath{stroke,fill}%
\end{pgfscope}%
\begin{pgfscope}%
\pgfpathrectangle{\pgfqpoint{0.100000in}{0.212622in}}{\pgfqpoint{3.696000in}{3.696000in}}%
\pgfusepath{clip}%
\pgfsetbuttcap%
\pgfsetroundjoin%
\definecolor{currentfill}{rgb}{0.121569,0.466667,0.705882}%
\pgfsetfillcolor{currentfill}%
\pgfsetfillopacity{0.402492}%
\pgfsetlinewidth{1.003750pt}%
\definecolor{currentstroke}{rgb}{0.121569,0.466667,0.705882}%
\pgfsetstrokecolor{currentstroke}%
\pgfsetstrokeopacity{0.402492}%
\pgfsetdash{}{0pt}%
\pgfpathmoveto{\pgfqpoint{1.565837in}{2.812205in}}%
\pgfpathcurveto{\pgfqpoint{1.574073in}{2.812205in}}{\pgfqpoint{1.581973in}{2.815478in}}{\pgfqpoint{1.587797in}{2.821302in}}%
\pgfpathcurveto{\pgfqpoint{1.593621in}{2.827126in}}{\pgfqpoint{1.596893in}{2.835026in}}{\pgfqpoint{1.596893in}{2.843262in}}%
\pgfpathcurveto{\pgfqpoint{1.596893in}{2.851498in}}{\pgfqpoint{1.593621in}{2.859398in}}{\pgfqpoint{1.587797in}{2.865222in}}%
\pgfpathcurveto{\pgfqpoint{1.581973in}{2.871046in}}{\pgfqpoint{1.574073in}{2.874318in}}{\pgfqpoint{1.565837in}{2.874318in}}%
\pgfpathcurveto{\pgfqpoint{1.557601in}{2.874318in}}{\pgfqpoint{1.549701in}{2.871046in}}{\pgfqpoint{1.543877in}{2.865222in}}%
\pgfpathcurveto{\pgfqpoint{1.538053in}{2.859398in}}{\pgfqpoint{1.534780in}{2.851498in}}{\pgfqpoint{1.534780in}{2.843262in}}%
\pgfpathcurveto{\pgfqpoint{1.534780in}{2.835026in}}{\pgfqpoint{1.538053in}{2.827126in}}{\pgfqpoint{1.543877in}{2.821302in}}%
\pgfpathcurveto{\pgfqpoint{1.549701in}{2.815478in}}{\pgfqpoint{1.557601in}{2.812205in}}{\pgfqpoint{1.565837in}{2.812205in}}%
\pgfpathclose%
\pgfusepath{stroke,fill}%
\end{pgfscope}%
\begin{pgfscope}%
\pgfpathrectangle{\pgfqpoint{0.100000in}{0.212622in}}{\pgfqpoint{3.696000in}{3.696000in}}%
\pgfusepath{clip}%
\pgfsetbuttcap%
\pgfsetroundjoin%
\definecolor{currentfill}{rgb}{0.121569,0.466667,0.705882}%
\pgfsetfillcolor{currentfill}%
\pgfsetfillopacity{0.402678}%
\pgfsetlinewidth{1.003750pt}%
\definecolor{currentstroke}{rgb}{0.121569,0.466667,0.705882}%
\pgfsetstrokecolor{currentstroke}%
\pgfsetstrokeopacity{0.402678}%
\pgfsetdash{}{0pt}%
\pgfpathmoveto{\pgfqpoint{1.565290in}{2.811234in}}%
\pgfpathcurveto{\pgfqpoint{1.573526in}{2.811234in}}{\pgfqpoint{1.581426in}{2.814507in}}{\pgfqpoint{1.587250in}{2.820330in}}%
\pgfpathcurveto{\pgfqpoint{1.593074in}{2.826154in}}{\pgfqpoint{1.596347in}{2.834054in}}{\pgfqpoint{1.596347in}{2.842291in}}%
\pgfpathcurveto{\pgfqpoint{1.596347in}{2.850527in}}{\pgfqpoint{1.593074in}{2.858427in}}{\pgfqpoint{1.587250in}{2.864251in}}%
\pgfpathcurveto{\pgfqpoint{1.581426in}{2.870075in}}{\pgfqpoint{1.573526in}{2.873347in}}{\pgfqpoint{1.565290in}{2.873347in}}%
\pgfpathcurveto{\pgfqpoint{1.557054in}{2.873347in}}{\pgfqpoint{1.549154in}{2.870075in}}{\pgfqpoint{1.543330in}{2.864251in}}%
\pgfpathcurveto{\pgfqpoint{1.537506in}{2.858427in}}{\pgfqpoint{1.534234in}{2.850527in}}{\pgfqpoint{1.534234in}{2.842291in}}%
\pgfpathcurveto{\pgfqpoint{1.534234in}{2.834054in}}{\pgfqpoint{1.537506in}{2.826154in}}{\pgfqpoint{1.543330in}{2.820330in}}%
\pgfpathcurveto{\pgfqpoint{1.549154in}{2.814507in}}{\pgfqpoint{1.557054in}{2.811234in}}{\pgfqpoint{1.565290in}{2.811234in}}%
\pgfpathclose%
\pgfusepath{stroke,fill}%
\end{pgfscope}%
\begin{pgfscope}%
\pgfpathrectangle{\pgfqpoint{0.100000in}{0.212622in}}{\pgfqpoint{3.696000in}{3.696000in}}%
\pgfusepath{clip}%
\pgfsetbuttcap%
\pgfsetroundjoin%
\definecolor{currentfill}{rgb}{0.121569,0.466667,0.705882}%
\pgfsetfillcolor{currentfill}%
\pgfsetfillopacity{0.403007}%
\pgfsetlinewidth{1.003750pt}%
\definecolor{currentstroke}{rgb}{0.121569,0.466667,0.705882}%
\pgfsetstrokecolor{currentstroke}%
\pgfsetstrokeopacity{0.403007}%
\pgfsetdash{}{0pt}%
\pgfpathmoveto{\pgfqpoint{1.564310in}{2.809415in}}%
\pgfpathcurveto{\pgfqpoint{1.572547in}{2.809415in}}{\pgfqpoint{1.580447in}{2.812688in}}{\pgfqpoint{1.586271in}{2.818512in}}%
\pgfpathcurveto{\pgfqpoint{1.592095in}{2.824336in}}{\pgfqpoint{1.595367in}{2.832236in}}{\pgfqpoint{1.595367in}{2.840472in}}%
\pgfpathcurveto{\pgfqpoint{1.595367in}{2.848708in}}{\pgfqpoint{1.592095in}{2.856608in}}{\pgfqpoint{1.586271in}{2.862432in}}%
\pgfpathcurveto{\pgfqpoint{1.580447in}{2.868256in}}{\pgfqpoint{1.572547in}{2.871528in}}{\pgfqpoint{1.564310in}{2.871528in}}%
\pgfpathcurveto{\pgfqpoint{1.556074in}{2.871528in}}{\pgfqpoint{1.548174in}{2.868256in}}{\pgfqpoint{1.542350in}{2.862432in}}%
\pgfpathcurveto{\pgfqpoint{1.536526in}{2.856608in}}{\pgfqpoint{1.533254in}{2.848708in}}{\pgfqpoint{1.533254in}{2.840472in}}%
\pgfpathcurveto{\pgfqpoint{1.533254in}{2.832236in}}{\pgfqpoint{1.536526in}{2.824336in}}{\pgfqpoint{1.542350in}{2.818512in}}%
\pgfpathcurveto{\pgfqpoint{1.548174in}{2.812688in}}{\pgfqpoint{1.556074in}{2.809415in}}{\pgfqpoint{1.564310in}{2.809415in}}%
\pgfpathclose%
\pgfusepath{stroke,fill}%
\end{pgfscope}%
\begin{pgfscope}%
\pgfpathrectangle{\pgfqpoint{0.100000in}{0.212622in}}{\pgfqpoint{3.696000in}{3.696000in}}%
\pgfusepath{clip}%
\pgfsetbuttcap%
\pgfsetroundjoin%
\definecolor{currentfill}{rgb}{0.121569,0.466667,0.705882}%
\pgfsetfillcolor{currentfill}%
\pgfsetfillopacity{0.403531}%
\pgfsetlinewidth{1.003750pt}%
\definecolor{currentstroke}{rgb}{0.121569,0.466667,0.705882}%
\pgfsetstrokecolor{currentstroke}%
\pgfsetstrokeopacity{0.403531}%
\pgfsetdash{}{0pt}%
\pgfpathmoveto{\pgfqpoint{1.562313in}{2.806139in}}%
\pgfpathcurveto{\pgfqpoint{1.570550in}{2.806139in}}{\pgfqpoint{1.578450in}{2.809411in}}{\pgfqpoint{1.584274in}{2.815235in}}%
\pgfpathcurveto{\pgfqpoint{1.590098in}{2.821059in}}{\pgfqpoint{1.593370in}{2.828959in}}{\pgfqpoint{1.593370in}{2.837195in}}%
\pgfpathcurveto{\pgfqpoint{1.593370in}{2.845432in}}{\pgfqpoint{1.590098in}{2.853332in}}{\pgfqpoint{1.584274in}{2.859156in}}%
\pgfpathcurveto{\pgfqpoint{1.578450in}{2.864980in}}{\pgfqpoint{1.570550in}{2.868252in}}{\pgfqpoint{1.562313in}{2.868252in}}%
\pgfpathcurveto{\pgfqpoint{1.554077in}{2.868252in}}{\pgfqpoint{1.546177in}{2.864980in}}{\pgfqpoint{1.540353in}{2.859156in}}%
\pgfpathcurveto{\pgfqpoint{1.534529in}{2.853332in}}{\pgfqpoint{1.531257in}{2.845432in}}{\pgfqpoint{1.531257in}{2.837195in}}%
\pgfpathcurveto{\pgfqpoint{1.531257in}{2.828959in}}{\pgfqpoint{1.534529in}{2.821059in}}{\pgfqpoint{1.540353in}{2.815235in}}%
\pgfpathcurveto{\pgfqpoint{1.546177in}{2.809411in}}{\pgfqpoint{1.554077in}{2.806139in}}{\pgfqpoint{1.562313in}{2.806139in}}%
\pgfpathclose%
\pgfusepath{stroke,fill}%
\end{pgfscope}%
\begin{pgfscope}%
\pgfpathrectangle{\pgfqpoint{0.100000in}{0.212622in}}{\pgfqpoint{3.696000in}{3.696000in}}%
\pgfusepath{clip}%
\pgfsetbuttcap%
\pgfsetroundjoin%
\definecolor{currentfill}{rgb}{0.121569,0.466667,0.705882}%
\pgfsetfillcolor{currentfill}%
\pgfsetfillopacity{0.403795}%
\pgfsetlinewidth{1.003750pt}%
\definecolor{currentstroke}{rgb}{0.121569,0.466667,0.705882}%
\pgfsetstrokecolor{currentstroke}%
\pgfsetstrokeopacity{0.403795}%
\pgfsetdash{}{0pt}%
\pgfpathmoveto{\pgfqpoint{1.561578in}{2.804505in}}%
\pgfpathcurveto{\pgfqpoint{1.569814in}{2.804505in}}{\pgfqpoint{1.577715in}{2.807777in}}{\pgfqpoint{1.583538in}{2.813601in}}%
\pgfpathcurveto{\pgfqpoint{1.589362in}{2.819425in}}{\pgfqpoint{1.592635in}{2.827325in}}{\pgfqpoint{1.592635in}{2.835561in}}%
\pgfpathcurveto{\pgfqpoint{1.592635in}{2.843797in}}{\pgfqpoint{1.589362in}{2.851697in}}{\pgfqpoint{1.583538in}{2.857521in}}%
\pgfpathcurveto{\pgfqpoint{1.577715in}{2.863345in}}{\pgfqpoint{1.569814in}{2.866618in}}{\pgfqpoint{1.561578in}{2.866618in}}%
\pgfpathcurveto{\pgfqpoint{1.553342in}{2.866618in}}{\pgfqpoint{1.545442in}{2.863345in}}{\pgfqpoint{1.539618in}{2.857521in}}%
\pgfpathcurveto{\pgfqpoint{1.533794in}{2.851697in}}{\pgfqpoint{1.530522in}{2.843797in}}{\pgfqpoint{1.530522in}{2.835561in}}%
\pgfpathcurveto{\pgfqpoint{1.530522in}{2.827325in}}{\pgfqpoint{1.533794in}{2.819425in}}{\pgfqpoint{1.539618in}{2.813601in}}%
\pgfpathcurveto{\pgfqpoint{1.545442in}{2.807777in}}{\pgfqpoint{1.553342in}{2.804505in}}{\pgfqpoint{1.561578in}{2.804505in}}%
\pgfpathclose%
\pgfusepath{stroke,fill}%
\end{pgfscope}%
\begin{pgfscope}%
\pgfpathrectangle{\pgfqpoint{0.100000in}{0.212622in}}{\pgfqpoint{3.696000in}{3.696000in}}%
\pgfusepath{clip}%
\pgfsetbuttcap%
\pgfsetroundjoin%
\definecolor{currentfill}{rgb}{0.121569,0.466667,0.705882}%
\pgfsetfillcolor{currentfill}%
\pgfsetfillopacity{0.403799}%
\pgfsetlinewidth{1.003750pt}%
\definecolor{currentstroke}{rgb}{0.121569,0.466667,0.705882}%
\pgfsetstrokecolor{currentstroke}%
\pgfsetstrokeopacity{0.403799}%
\pgfsetdash{}{0pt}%
\pgfpathmoveto{\pgfqpoint{1.979299in}{2.909624in}}%
\pgfpathcurveto{\pgfqpoint{1.987536in}{2.909624in}}{\pgfqpoint{1.995436in}{2.912896in}}{\pgfqpoint{2.001260in}{2.918720in}}%
\pgfpathcurveto{\pgfqpoint{2.007083in}{2.924544in}}{\pgfqpoint{2.010356in}{2.932444in}}{\pgfqpoint{2.010356in}{2.940680in}}%
\pgfpathcurveto{\pgfqpoint{2.010356in}{2.948916in}}{\pgfqpoint{2.007083in}{2.956816in}}{\pgfqpoint{2.001260in}{2.962640in}}%
\pgfpathcurveto{\pgfqpoint{1.995436in}{2.968464in}}{\pgfqpoint{1.987536in}{2.971737in}}{\pgfqpoint{1.979299in}{2.971737in}}%
\pgfpathcurveto{\pgfqpoint{1.971063in}{2.971737in}}{\pgfqpoint{1.963163in}{2.968464in}}{\pgfqpoint{1.957339in}{2.962640in}}%
\pgfpathcurveto{\pgfqpoint{1.951515in}{2.956816in}}{\pgfqpoint{1.948243in}{2.948916in}}{\pgfqpoint{1.948243in}{2.940680in}}%
\pgfpathcurveto{\pgfqpoint{1.948243in}{2.932444in}}{\pgfqpoint{1.951515in}{2.924544in}}{\pgfqpoint{1.957339in}{2.918720in}}%
\pgfpathcurveto{\pgfqpoint{1.963163in}{2.912896in}}{\pgfqpoint{1.971063in}{2.909624in}}{\pgfqpoint{1.979299in}{2.909624in}}%
\pgfpathclose%
\pgfusepath{stroke,fill}%
\end{pgfscope}%
\begin{pgfscope}%
\pgfpathrectangle{\pgfqpoint{0.100000in}{0.212622in}}{\pgfqpoint{3.696000in}{3.696000in}}%
\pgfusepath{clip}%
\pgfsetbuttcap%
\pgfsetroundjoin%
\definecolor{currentfill}{rgb}{0.121569,0.466667,0.705882}%
\pgfsetfillcolor{currentfill}%
\pgfsetfillopacity{0.404312}%
\pgfsetlinewidth{1.003750pt}%
\definecolor{currentstroke}{rgb}{0.121569,0.466667,0.705882}%
\pgfsetstrokecolor{currentstroke}%
\pgfsetstrokeopacity{0.404312}%
\pgfsetdash{}{0pt}%
\pgfpathmoveto{\pgfqpoint{1.560023in}{2.801948in}}%
\pgfpathcurveto{\pgfqpoint{1.568260in}{2.801948in}}{\pgfqpoint{1.576160in}{2.805220in}}{\pgfqpoint{1.581984in}{2.811044in}}%
\pgfpathcurveto{\pgfqpoint{1.587807in}{2.816868in}}{\pgfqpoint{1.591080in}{2.824768in}}{\pgfqpoint{1.591080in}{2.833005in}}%
\pgfpathcurveto{\pgfqpoint{1.591080in}{2.841241in}}{\pgfqpoint{1.587807in}{2.849141in}}{\pgfqpoint{1.581984in}{2.854965in}}%
\pgfpathcurveto{\pgfqpoint{1.576160in}{2.860789in}}{\pgfqpoint{1.568260in}{2.864061in}}{\pgfqpoint{1.560023in}{2.864061in}}%
\pgfpathcurveto{\pgfqpoint{1.551787in}{2.864061in}}{\pgfqpoint{1.543887in}{2.860789in}}{\pgfqpoint{1.538063in}{2.854965in}}%
\pgfpathcurveto{\pgfqpoint{1.532239in}{2.849141in}}{\pgfqpoint{1.528967in}{2.841241in}}{\pgfqpoint{1.528967in}{2.833005in}}%
\pgfpathcurveto{\pgfqpoint{1.528967in}{2.824768in}}{\pgfqpoint{1.532239in}{2.816868in}}{\pgfqpoint{1.538063in}{2.811044in}}%
\pgfpathcurveto{\pgfqpoint{1.543887in}{2.805220in}}{\pgfqpoint{1.551787in}{2.801948in}}{\pgfqpoint{1.560023in}{2.801948in}}%
\pgfpathclose%
\pgfusepath{stroke,fill}%
\end{pgfscope}%
\begin{pgfscope}%
\pgfpathrectangle{\pgfqpoint{0.100000in}{0.212622in}}{\pgfqpoint{3.696000in}{3.696000in}}%
\pgfusepath{clip}%
\pgfsetbuttcap%
\pgfsetroundjoin%
\definecolor{currentfill}{rgb}{0.121569,0.466667,0.705882}%
\pgfsetfillcolor{currentfill}%
\pgfsetfillopacity{0.405240}%
\pgfsetlinewidth{1.003750pt}%
\definecolor{currentstroke}{rgb}{0.121569,0.466667,0.705882}%
\pgfsetstrokecolor{currentstroke}%
\pgfsetstrokeopacity{0.405240}%
\pgfsetdash{}{0pt}%
\pgfpathmoveto{\pgfqpoint{1.557205in}{2.797240in}}%
\pgfpathcurveto{\pgfqpoint{1.565442in}{2.797240in}}{\pgfqpoint{1.573342in}{2.800512in}}{\pgfqpoint{1.579166in}{2.806336in}}%
\pgfpathcurveto{\pgfqpoint{1.584990in}{2.812160in}}{\pgfqpoint{1.588262in}{2.820060in}}{\pgfqpoint{1.588262in}{2.828296in}}%
\pgfpathcurveto{\pgfqpoint{1.588262in}{2.836533in}}{\pgfqpoint{1.584990in}{2.844433in}}{\pgfqpoint{1.579166in}{2.850257in}}%
\pgfpathcurveto{\pgfqpoint{1.573342in}{2.856081in}}{\pgfqpoint{1.565442in}{2.859353in}}{\pgfqpoint{1.557205in}{2.859353in}}%
\pgfpathcurveto{\pgfqpoint{1.548969in}{2.859353in}}{\pgfqpoint{1.541069in}{2.856081in}}{\pgfqpoint{1.535245in}{2.850257in}}%
\pgfpathcurveto{\pgfqpoint{1.529421in}{2.844433in}}{\pgfqpoint{1.526149in}{2.836533in}}{\pgfqpoint{1.526149in}{2.828296in}}%
\pgfpathcurveto{\pgfqpoint{1.526149in}{2.820060in}}{\pgfqpoint{1.529421in}{2.812160in}}{\pgfqpoint{1.535245in}{2.806336in}}%
\pgfpathcurveto{\pgfqpoint{1.541069in}{2.800512in}}{\pgfqpoint{1.548969in}{2.797240in}}{\pgfqpoint{1.557205in}{2.797240in}}%
\pgfpathclose%
\pgfusepath{stroke,fill}%
\end{pgfscope}%
\begin{pgfscope}%
\pgfpathrectangle{\pgfqpoint{0.100000in}{0.212622in}}{\pgfqpoint{3.696000in}{3.696000in}}%
\pgfusepath{clip}%
\pgfsetbuttcap%
\pgfsetroundjoin%
\definecolor{currentfill}{rgb}{0.121569,0.466667,0.705882}%
\pgfsetfillcolor{currentfill}%
\pgfsetfillopacity{0.406100}%
\pgfsetlinewidth{1.003750pt}%
\definecolor{currentstroke}{rgb}{0.121569,0.466667,0.705882}%
\pgfsetstrokecolor{currentstroke}%
\pgfsetstrokeopacity{0.406100}%
\pgfsetdash{}{0pt}%
\pgfpathmoveto{\pgfqpoint{1.554894in}{2.792949in}}%
\pgfpathcurveto{\pgfqpoint{1.563131in}{2.792949in}}{\pgfqpoint{1.571031in}{2.796221in}}{\pgfqpoint{1.576855in}{2.802045in}}%
\pgfpathcurveto{\pgfqpoint{1.582679in}{2.807869in}}{\pgfqpoint{1.585951in}{2.815769in}}{\pgfqpoint{1.585951in}{2.824005in}}%
\pgfpathcurveto{\pgfqpoint{1.585951in}{2.832241in}}{\pgfqpoint{1.582679in}{2.840142in}}{\pgfqpoint{1.576855in}{2.845965in}}%
\pgfpathcurveto{\pgfqpoint{1.571031in}{2.851789in}}{\pgfqpoint{1.563131in}{2.855062in}}{\pgfqpoint{1.554894in}{2.855062in}}%
\pgfpathcurveto{\pgfqpoint{1.546658in}{2.855062in}}{\pgfqpoint{1.538758in}{2.851789in}}{\pgfqpoint{1.532934in}{2.845965in}}%
\pgfpathcurveto{\pgfqpoint{1.527110in}{2.840142in}}{\pgfqpoint{1.523838in}{2.832241in}}{\pgfqpoint{1.523838in}{2.824005in}}%
\pgfpathcurveto{\pgfqpoint{1.523838in}{2.815769in}}{\pgfqpoint{1.527110in}{2.807869in}}{\pgfqpoint{1.532934in}{2.802045in}}%
\pgfpathcurveto{\pgfqpoint{1.538758in}{2.796221in}}{\pgfqpoint{1.546658in}{2.792949in}}{\pgfqpoint{1.554894in}{2.792949in}}%
\pgfpathclose%
\pgfusepath{stroke,fill}%
\end{pgfscope}%
\begin{pgfscope}%
\pgfpathrectangle{\pgfqpoint{0.100000in}{0.212622in}}{\pgfqpoint{3.696000in}{3.696000in}}%
\pgfusepath{clip}%
\pgfsetbuttcap%
\pgfsetroundjoin%
\definecolor{currentfill}{rgb}{0.121569,0.466667,0.705882}%
\pgfsetfillcolor{currentfill}%
\pgfsetfillopacity{0.406552}%
\pgfsetlinewidth{1.003750pt}%
\definecolor{currentstroke}{rgb}{0.121569,0.466667,0.705882}%
\pgfsetstrokecolor{currentstroke}%
\pgfsetstrokeopacity{0.406552}%
\pgfsetdash{}{0pt}%
\pgfpathmoveto{\pgfqpoint{1.553327in}{2.790540in}}%
\pgfpathcurveto{\pgfqpoint{1.561563in}{2.790540in}}{\pgfqpoint{1.569463in}{2.793813in}}{\pgfqpoint{1.575287in}{2.799637in}}%
\pgfpathcurveto{\pgfqpoint{1.581111in}{2.805460in}}{\pgfqpoint{1.584384in}{2.813361in}}{\pgfqpoint{1.584384in}{2.821597in}}%
\pgfpathcurveto{\pgfqpoint{1.584384in}{2.829833in}}{\pgfqpoint{1.581111in}{2.837733in}}{\pgfqpoint{1.575287in}{2.843557in}}%
\pgfpathcurveto{\pgfqpoint{1.569463in}{2.849381in}}{\pgfqpoint{1.561563in}{2.852653in}}{\pgfqpoint{1.553327in}{2.852653in}}%
\pgfpathcurveto{\pgfqpoint{1.545091in}{2.852653in}}{\pgfqpoint{1.537191in}{2.849381in}}{\pgfqpoint{1.531367in}{2.843557in}}%
\pgfpathcurveto{\pgfqpoint{1.525543in}{2.837733in}}{\pgfqpoint{1.522271in}{2.829833in}}{\pgfqpoint{1.522271in}{2.821597in}}%
\pgfpathcurveto{\pgfqpoint{1.522271in}{2.813361in}}{\pgfqpoint{1.525543in}{2.805460in}}{\pgfqpoint{1.531367in}{2.799637in}}%
\pgfpathcurveto{\pgfqpoint{1.537191in}{2.793813in}}{\pgfqpoint{1.545091in}{2.790540in}}{\pgfqpoint{1.553327in}{2.790540in}}%
\pgfpathclose%
\pgfusepath{stroke,fill}%
\end{pgfscope}%
\begin{pgfscope}%
\pgfpathrectangle{\pgfqpoint{0.100000in}{0.212622in}}{\pgfqpoint{3.696000in}{3.696000in}}%
\pgfusepath{clip}%
\pgfsetbuttcap%
\pgfsetroundjoin%
\definecolor{currentfill}{rgb}{0.121569,0.466667,0.705882}%
\pgfsetfillcolor{currentfill}%
\pgfsetfillopacity{0.406883}%
\pgfsetlinewidth{1.003750pt}%
\definecolor{currentstroke}{rgb}{0.121569,0.466667,0.705882}%
\pgfsetstrokecolor{currentstroke}%
\pgfsetstrokeopacity{0.406883}%
\pgfsetdash{}{0pt}%
\pgfpathmoveto{\pgfqpoint{1.981628in}{2.896166in}}%
\pgfpathcurveto{\pgfqpoint{1.989864in}{2.896166in}}{\pgfqpoint{1.997764in}{2.899438in}}{\pgfqpoint{2.003588in}{2.905262in}}%
\pgfpathcurveto{\pgfqpoint{2.009412in}{2.911086in}}{\pgfqpoint{2.012684in}{2.918986in}}{\pgfqpoint{2.012684in}{2.927222in}}%
\pgfpathcurveto{\pgfqpoint{2.012684in}{2.935459in}}{\pgfqpoint{2.009412in}{2.943359in}}{\pgfqpoint{2.003588in}{2.949183in}}%
\pgfpathcurveto{\pgfqpoint{1.997764in}{2.955007in}}{\pgfqpoint{1.989864in}{2.958279in}}{\pgfqpoint{1.981628in}{2.958279in}}%
\pgfpathcurveto{\pgfqpoint{1.973391in}{2.958279in}}{\pgfqpoint{1.965491in}{2.955007in}}{\pgfqpoint{1.959667in}{2.949183in}}%
\pgfpathcurveto{\pgfqpoint{1.953843in}{2.943359in}}{\pgfqpoint{1.950571in}{2.935459in}}{\pgfqpoint{1.950571in}{2.927222in}}%
\pgfpathcurveto{\pgfqpoint{1.950571in}{2.918986in}}{\pgfqpoint{1.953843in}{2.911086in}}{\pgfqpoint{1.959667in}{2.905262in}}%
\pgfpathcurveto{\pgfqpoint{1.965491in}{2.899438in}}{\pgfqpoint{1.973391in}{2.896166in}}{\pgfqpoint{1.981628in}{2.896166in}}%
\pgfpathclose%
\pgfusepath{stroke,fill}%
\end{pgfscope}%
\begin{pgfscope}%
\pgfpathrectangle{\pgfqpoint{0.100000in}{0.212622in}}{\pgfqpoint{3.696000in}{3.696000in}}%
\pgfusepath{clip}%
\pgfsetbuttcap%
\pgfsetroundjoin%
\definecolor{currentfill}{rgb}{0.121569,0.466667,0.705882}%
\pgfsetfillcolor{currentfill}%
\pgfsetfillopacity{0.406889}%
\pgfsetlinewidth{1.003750pt}%
\definecolor{currentstroke}{rgb}{0.121569,0.466667,0.705882}%
\pgfsetstrokecolor{currentstroke}%
\pgfsetstrokeopacity{0.406889}%
\pgfsetdash{}{0pt}%
\pgfpathmoveto{\pgfqpoint{1.552395in}{2.788750in}}%
\pgfpathcurveto{\pgfqpoint{1.560631in}{2.788750in}}{\pgfqpoint{1.568531in}{2.792023in}}{\pgfqpoint{1.574355in}{2.797847in}}%
\pgfpathcurveto{\pgfqpoint{1.580179in}{2.803671in}}{\pgfqpoint{1.583451in}{2.811571in}}{\pgfqpoint{1.583451in}{2.819807in}}%
\pgfpathcurveto{\pgfqpoint{1.583451in}{2.828043in}}{\pgfqpoint{1.580179in}{2.835943in}}{\pgfqpoint{1.574355in}{2.841767in}}%
\pgfpathcurveto{\pgfqpoint{1.568531in}{2.847591in}}{\pgfqpoint{1.560631in}{2.850863in}}{\pgfqpoint{1.552395in}{2.850863in}}%
\pgfpathcurveto{\pgfqpoint{1.544159in}{2.850863in}}{\pgfqpoint{1.536259in}{2.847591in}}{\pgfqpoint{1.530435in}{2.841767in}}%
\pgfpathcurveto{\pgfqpoint{1.524611in}{2.835943in}}{\pgfqpoint{1.521338in}{2.828043in}}{\pgfqpoint{1.521338in}{2.819807in}}%
\pgfpathcurveto{\pgfqpoint{1.521338in}{2.811571in}}{\pgfqpoint{1.524611in}{2.803671in}}{\pgfqpoint{1.530435in}{2.797847in}}%
\pgfpathcurveto{\pgfqpoint{1.536259in}{2.792023in}}{\pgfqpoint{1.544159in}{2.788750in}}{\pgfqpoint{1.552395in}{2.788750in}}%
\pgfpathclose%
\pgfusepath{stroke,fill}%
\end{pgfscope}%
\begin{pgfscope}%
\pgfpathrectangle{\pgfqpoint{0.100000in}{0.212622in}}{\pgfqpoint{3.696000in}{3.696000in}}%
\pgfusepath{clip}%
\pgfsetbuttcap%
\pgfsetroundjoin%
\definecolor{currentfill}{rgb}{0.121569,0.466667,0.705882}%
\pgfsetfillcolor{currentfill}%
\pgfsetfillopacity{0.407472}%
\pgfsetlinewidth{1.003750pt}%
\definecolor{currentstroke}{rgb}{0.121569,0.466667,0.705882}%
\pgfsetstrokecolor{currentstroke}%
\pgfsetstrokeopacity{0.407472}%
\pgfsetdash{}{0pt}%
\pgfpathmoveto{\pgfqpoint{1.550584in}{2.785534in}}%
\pgfpathcurveto{\pgfqpoint{1.558820in}{2.785534in}}{\pgfqpoint{1.566720in}{2.788806in}}{\pgfqpoint{1.572544in}{2.794630in}}%
\pgfpathcurveto{\pgfqpoint{1.578368in}{2.800454in}}{\pgfqpoint{1.581640in}{2.808354in}}{\pgfqpoint{1.581640in}{2.816591in}}%
\pgfpathcurveto{\pgfqpoint{1.581640in}{2.824827in}}{\pgfqpoint{1.578368in}{2.832727in}}{\pgfqpoint{1.572544in}{2.838551in}}%
\pgfpathcurveto{\pgfqpoint{1.566720in}{2.844375in}}{\pgfqpoint{1.558820in}{2.847647in}}{\pgfqpoint{1.550584in}{2.847647in}}%
\pgfpathcurveto{\pgfqpoint{1.542348in}{2.847647in}}{\pgfqpoint{1.534448in}{2.844375in}}{\pgfqpoint{1.528624in}{2.838551in}}%
\pgfpathcurveto{\pgfqpoint{1.522800in}{2.832727in}}{\pgfqpoint{1.519527in}{2.824827in}}{\pgfqpoint{1.519527in}{2.816591in}}%
\pgfpathcurveto{\pgfqpoint{1.519527in}{2.808354in}}{\pgfqpoint{1.522800in}{2.800454in}}{\pgfqpoint{1.528624in}{2.794630in}}%
\pgfpathcurveto{\pgfqpoint{1.534448in}{2.788806in}}{\pgfqpoint{1.542348in}{2.785534in}}{\pgfqpoint{1.550584in}{2.785534in}}%
\pgfpathclose%
\pgfusepath{stroke,fill}%
\end{pgfscope}%
\begin{pgfscope}%
\pgfpathrectangle{\pgfqpoint{0.100000in}{0.212622in}}{\pgfqpoint{3.696000in}{3.696000in}}%
\pgfusepath{clip}%
\pgfsetbuttcap%
\pgfsetroundjoin%
\definecolor{currentfill}{rgb}{0.121569,0.466667,0.705882}%
\pgfsetfillcolor{currentfill}%
\pgfsetfillopacity{0.408390}%
\pgfsetlinewidth{1.003750pt}%
\definecolor{currentstroke}{rgb}{0.121569,0.466667,0.705882}%
\pgfsetstrokecolor{currentstroke}%
\pgfsetstrokeopacity{0.408390}%
\pgfsetdash{}{0pt}%
\pgfpathmoveto{\pgfqpoint{1.546995in}{2.779595in}}%
\pgfpathcurveto{\pgfqpoint{1.555231in}{2.779595in}}{\pgfqpoint{1.563131in}{2.782867in}}{\pgfqpoint{1.568955in}{2.788691in}}%
\pgfpathcurveto{\pgfqpoint{1.574779in}{2.794515in}}{\pgfqpoint{1.578051in}{2.802415in}}{\pgfqpoint{1.578051in}{2.810651in}}%
\pgfpathcurveto{\pgfqpoint{1.578051in}{2.818888in}}{\pgfqpoint{1.574779in}{2.826788in}}{\pgfqpoint{1.568955in}{2.832612in}}%
\pgfpathcurveto{\pgfqpoint{1.563131in}{2.838436in}}{\pgfqpoint{1.555231in}{2.841708in}}{\pgfqpoint{1.546995in}{2.841708in}}%
\pgfpathcurveto{\pgfqpoint{1.538758in}{2.841708in}}{\pgfqpoint{1.530858in}{2.838436in}}{\pgfqpoint{1.525034in}{2.832612in}}%
\pgfpathcurveto{\pgfqpoint{1.519210in}{2.826788in}}{\pgfqpoint{1.515938in}{2.818888in}}{\pgfqpoint{1.515938in}{2.810651in}}%
\pgfpathcurveto{\pgfqpoint{1.515938in}{2.802415in}}{\pgfqpoint{1.519210in}{2.794515in}}{\pgfqpoint{1.525034in}{2.788691in}}%
\pgfpathcurveto{\pgfqpoint{1.530858in}{2.782867in}}{\pgfqpoint{1.538758in}{2.779595in}}{\pgfqpoint{1.546995in}{2.779595in}}%
\pgfpathclose%
\pgfusepath{stroke,fill}%
\end{pgfscope}%
\begin{pgfscope}%
\pgfpathrectangle{\pgfqpoint{0.100000in}{0.212622in}}{\pgfqpoint{3.696000in}{3.696000in}}%
\pgfusepath{clip}%
\pgfsetbuttcap%
\pgfsetroundjoin%
\definecolor{currentfill}{rgb}{0.121569,0.466667,0.705882}%
\pgfsetfillcolor{currentfill}%
\pgfsetfillopacity{0.409042}%
\pgfsetlinewidth{1.003750pt}%
\definecolor{currentstroke}{rgb}{0.121569,0.466667,0.705882}%
\pgfsetstrokecolor{currentstroke}%
\pgfsetstrokeopacity{0.409042}%
\pgfsetdash{}{0pt}%
\pgfpathmoveto{\pgfqpoint{1.545139in}{2.775338in}}%
\pgfpathcurveto{\pgfqpoint{1.553375in}{2.775338in}}{\pgfqpoint{1.561275in}{2.778610in}}{\pgfqpoint{1.567099in}{2.784434in}}%
\pgfpathcurveto{\pgfqpoint{1.572923in}{2.790258in}}{\pgfqpoint{1.576196in}{2.798158in}}{\pgfqpoint{1.576196in}{2.806394in}}%
\pgfpathcurveto{\pgfqpoint{1.576196in}{2.814630in}}{\pgfqpoint{1.572923in}{2.822530in}}{\pgfqpoint{1.567099in}{2.828354in}}%
\pgfpathcurveto{\pgfqpoint{1.561275in}{2.834178in}}{\pgfqpoint{1.553375in}{2.837451in}}{\pgfqpoint{1.545139in}{2.837451in}}%
\pgfpathcurveto{\pgfqpoint{1.536903in}{2.837451in}}{\pgfqpoint{1.529003in}{2.834178in}}{\pgfqpoint{1.523179in}{2.828354in}}%
\pgfpathcurveto{\pgfqpoint{1.517355in}{2.822530in}}{\pgfqpoint{1.514083in}{2.814630in}}{\pgfqpoint{1.514083in}{2.806394in}}%
\pgfpathcurveto{\pgfqpoint{1.514083in}{2.798158in}}{\pgfqpoint{1.517355in}{2.790258in}}{\pgfqpoint{1.523179in}{2.784434in}}%
\pgfpathcurveto{\pgfqpoint{1.529003in}{2.778610in}}{\pgfqpoint{1.536903in}{2.775338in}}{\pgfqpoint{1.545139in}{2.775338in}}%
\pgfpathclose%
\pgfusepath{stroke,fill}%
\end{pgfscope}%
\begin{pgfscope}%
\pgfpathrectangle{\pgfqpoint{0.100000in}{0.212622in}}{\pgfqpoint{3.696000in}{3.696000in}}%
\pgfusepath{clip}%
\pgfsetbuttcap%
\pgfsetroundjoin%
\definecolor{currentfill}{rgb}{0.121569,0.466667,0.705882}%
\pgfsetfillcolor{currentfill}%
\pgfsetfillopacity{0.409625}%
\pgfsetlinewidth{1.003750pt}%
\definecolor{currentstroke}{rgb}{0.121569,0.466667,0.705882}%
\pgfsetstrokecolor{currentstroke}%
\pgfsetstrokeopacity{0.409625}%
\pgfsetdash{}{0pt}%
\pgfpathmoveto{\pgfqpoint{1.543324in}{2.772276in}}%
\pgfpathcurveto{\pgfqpoint{1.551560in}{2.772276in}}{\pgfqpoint{1.559460in}{2.775548in}}{\pgfqpoint{1.565284in}{2.781372in}}%
\pgfpathcurveto{\pgfqpoint{1.571108in}{2.787196in}}{\pgfqpoint{1.574381in}{2.795096in}}{\pgfqpoint{1.574381in}{2.803332in}}%
\pgfpathcurveto{\pgfqpoint{1.574381in}{2.811569in}}{\pgfqpoint{1.571108in}{2.819469in}}{\pgfqpoint{1.565284in}{2.825293in}}%
\pgfpathcurveto{\pgfqpoint{1.559460in}{2.831117in}}{\pgfqpoint{1.551560in}{2.834389in}}{\pgfqpoint{1.543324in}{2.834389in}}%
\pgfpathcurveto{\pgfqpoint{1.535088in}{2.834389in}}{\pgfqpoint{1.527188in}{2.831117in}}{\pgfqpoint{1.521364in}{2.825293in}}%
\pgfpathcurveto{\pgfqpoint{1.515540in}{2.819469in}}{\pgfqpoint{1.512268in}{2.811569in}}{\pgfqpoint{1.512268in}{2.803332in}}%
\pgfpathcurveto{\pgfqpoint{1.512268in}{2.795096in}}{\pgfqpoint{1.515540in}{2.787196in}}{\pgfqpoint{1.521364in}{2.781372in}}%
\pgfpathcurveto{\pgfqpoint{1.527188in}{2.775548in}}{\pgfqpoint{1.535088in}{2.772276in}}{\pgfqpoint{1.543324in}{2.772276in}}%
\pgfpathclose%
\pgfusepath{stroke,fill}%
\end{pgfscope}%
\begin{pgfscope}%
\pgfpathrectangle{\pgfqpoint{0.100000in}{0.212622in}}{\pgfqpoint{3.696000in}{3.696000in}}%
\pgfusepath{clip}%
\pgfsetbuttcap%
\pgfsetroundjoin%
\definecolor{currentfill}{rgb}{0.121569,0.466667,0.705882}%
\pgfsetfillcolor{currentfill}%
\pgfsetfillopacity{0.410176}%
\pgfsetlinewidth{1.003750pt}%
\definecolor{currentstroke}{rgb}{0.121569,0.466667,0.705882}%
\pgfsetstrokecolor{currentstroke}%
\pgfsetstrokeopacity{0.410176}%
\pgfsetdash{}{0pt}%
\pgfpathmoveto{\pgfqpoint{1.983693in}{2.882269in}}%
\pgfpathcurveto{\pgfqpoint{1.991930in}{2.882269in}}{\pgfqpoint{1.999830in}{2.885541in}}{\pgfqpoint{2.005654in}{2.891365in}}%
\pgfpathcurveto{\pgfqpoint{2.011478in}{2.897189in}}{\pgfqpoint{2.014750in}{2.905089in}}{\pgfqpoint{2.014750in}{2.913325in}}%
\pgfpathcurveto{\pgfqpoint{2.014750in}{2.921561in}}{\pgfqpoint{2.011478in}{2.929462in}}{\pgfqpoint{2.005654in}{2.935285in}}%
\pgfpathcurveto{\pgfqpoint{1.999830in}{2.941109in}}{\pgfqpoint{1.991930in}{2.944382in}}{\pgfqpoint{1.983693in}{2.944382in}}%
\pgfpathcurveto{\pgfqpoint{1.975457in}{2.944382in}}{\pgfqpoint{1.967557in}{2.941109in}}{\pgfqpoint{1.961733in}{2.935285in}}%
\pgfpathcurveto{\pgfqpoint{1.955909in}{2.929462in}}{\pgfqpoint{1.952637in}{2.921561in}}{\pgfqpoint{1.952637in}{2.913325in}}%
\pgfpathcurveto{\pgfqpoint{1.952637in}{2.905089in}}{\pgfqpoint{1.955909in}{2.897189in}}{\pgfqpoint{1.961733in}{2.891365in}}%
\pgfpathcurveto{\pgfqpoint{1.967557in}{2.885541in}}{\pgfqpoint{1.975457in}{2.882269in}}{\pgfqpoint{1.983693in}{2.882269in}}%
\pgfpathclose%
\pgfusepath{stroke,fill}%
\end{pgfscope}%
\begin{pgfscope}%
\pgfpathrectangle{\pgfqpoint{0.100000in}{0.212622in}}{\pgfqpoint{3.696000in}{3.696000in}}%
\pgfusepath{clip}%
\pgfsetbuttcap%
\pgfsetroundjoin%
\definecolor{currentfill}{rgb}{0.121569,0.466667,0.705882}%
\pgfsetfillcolor{currentfill}%
\pgfsetfillopacity{0.410675}%
\pgfsetlinewidth{1.003750pt}%
\definecolor{currentstroke}{rgb}{0.121569,0.466667,0.705882}%
\pgfsetstrokecolor{currentstroke}%
\pgfsetstrokeopacity{0.410675}%
\pgfsetdash{}{0pt}%
\pgfpathmoveto{\pgfqpoint{1.540037in}{2.766650in}}%
\pgfpathcurveto{\pgfqpoint{1.548273in}{2.766650in}}{\pgfqpoint{1.556173in}{2.769922in}}{\pgfqpoint{1.561997in}{2.775746in}}%
\pgfpathcurveto{\pgfqpoint{1.567821in}{2.781570in}}{\pgfqpoint{1.571093in}{2.789470in}}{\pgfqpoint{1.571093in}{2.797706in}}%
\pgfpathcurveto{\pgfqpoint{1.571093in}{2.805942in}}{\pgfqpoint{1.567821in}{2.813842in}}{\pgfqpoint{1.561997in}{2.819666in}}%
\pgfpathcurveto{\pgfqpoint{1.556173in}{2.825490in}}{\pgfqpoint{1.548273in}{2.828763in}}{\pgfqpoint{1.540037in}{2.828763in}}%
\pgfpathcurveto{\pgfqpoint{1.531800in}{2.828763in}}{\pgfqpoint{1.523900in}{2.825490in}}{\pgfqpoint{1.518077in}{2.819666in}}%
\pgfpathcurveto{\pgfqpoint{1.512253in}{2.813842in}}{\pgfqpoint{1.508980in}{2.805942in}}{\pgfqpoint{1.508980in}{2.797706in}}%
\pgfpathcurveto{\pgfqpoint{1.508980in}{2.789470in}}{\pgfqpoint{1.512253in}{2.781570in}}{\pgfqpoint{1.518077in}{2.775746in}}%
\pgfpathcurveto{\pgfqpoint{1.523900in}{2.769922in}}{\pgfqpoint{1.531800in}{2.766650in}}{\pgfqpoint{1.540037in}{2.766650in}}%
\pgfpathclose%
\pgfusepath{stroke,fill}%
\end{pgfscope}%
\begin{pgfscope}%
\pgfpathrectangle{\pgfqpoint{0.100000in}{0.212622in}}{\pgfqpoint{3.696000in}{3.696000in}}%
\pgfusepath{clip}%
\pgfsetbuttcap%
\pgfsetroundjoin%
\definecolor{currentfill}{rgb}{0.121569,0.466667,0.705882}%
\pgfsetfillcolor{currentfill}%
\pgfsetfillopacity{0.411709}%
\pgfsetlinewidth{1.003750pt}%
\definecolor{currentstroke}{rgb}{0.121569,0.466667,0.705882}%
\pgfsetstrokecolor{currentstroke}%
\pgfsetstrokeopacity{0.411709}%
\pgfsetdash{}{0pt}%
\pgfpathmoveto{\pgfqpoint{1.537295in}{2.761092in}}%
\pgfpathcurveto{\pgfqpoint{1.545531in}{2.761092in}}{\pgfqpoint{1.553431in}{2.764365in}}{\pgfqpoint{1.559255in}{2.770188in}}%
\pgfpathcurveto{\pgfqpoint{1.565079in}{2.776012in}}{\pgfqpoint{1.568351in}{2.783912in}}{\pgfqpoint{1.568351in}{2.792149in}}%
\pgfpathcurveto{\pgfqpoint{1.568351in}{2.800385in}}{\pgfqpoint{1.565079in}{2.808285in}}{\pgfqpoint{1.559255in}{2.814109in}}%
\pgfpathcurveto{\pgfqpoint{1.553431in}{2.819933in}}{\pgfqpoint{1.545531in}{2.823205in}}{\pgfqpoint{1.537295in}{2.823205in}}%
\pgfpathcurveto{\pgfqpoint{1.529059in}{2.823205in}}{\pgfqpoint{1.521159in}{2.819933in}}{\pgfqpoint{1.515335in}{2.814109in}}%
\pgfpathcurveto{\pgfqpoint{1.509511in}{2.808285in}}{\pgfqpoint{1.506238in}{2.800385in}}{\pgfqpoint{1.506238in}{2.792149in}}%
\pgfpathcurveto{\pgfqpoint{1.506238in}{2.783912in}}{\pgfqpoint{1.509511in}{2.776012in}}{\pgfqpoint{1.515335in}{2.770188in}}%
\pgfpathcurveto{\pgfqpoint{1.521159in}{2.764365in}}{\pgfqpoint{1.529059in}{2.761092in}}{\pgfqpoint{1.537295in}{2.761092in}}%
\pgfpathclose%
\pgfusepath{stroke,fill}%
\end{pgfscope}%
\begin{pgfscope}%
\pgfpathrectangle{\pgfqpoint{0.100000in}{0.212622in}}{\pgfqpoint{3.696000in}{3.696000in}}%
\pgfusepath{clip}%
\pgfsetbuttcap%
\pgfsetroundjoin%
\definecolor{currentfill}{rgb}{0.121569,0.466667,0.705882}%
\pgfsetfillcolor{currentfill}%
\pgfsetfillopacity{0.412156}%
\pgfsetlinewidth{1.003750pt}%
\definecolor{currentstroke}{rgb}{0.121569,0.466667,0.705882}%
\pgfsetstrokecolor{currentstroke}%
\pgfsetstrokeopacity{0.412156}%
\pgfsetdash{}{0pt}%
\pgfpathmoveto{\pgfqpoint{1.535772in}{2.758757in}}%
\pgfpathcurveto{\pgfqpoint{1.544008in}{2.758757in}}{\pgfqpoint{1.551908in}{2.762029in}}{\pgfqpoint{1.557732in}{2.767853in}}%
\pgfpathcurveto{\pgfqpoint{1.563556in}{2.773677in}}{\pgfqpoint{1.566828in}{2.781577in}}{\pgfqpoint{1.566828in}{2.789814in}}%
\pgfpathcurveto{\pgfqpoint{1.566828in}{2.798050in}}{\pgfqpoint{1.563556in}{2.805950in}}{\pgfqpoint{1.557732in}{2.811774in}}%
\pgfpathcurveto{\pgfqpoint{1.551908in}{2.817598in}}{\pgfqpoint{1.544008in}{2.820870in}}{\pgfqpoint{1.535772in}{2.820870in}}%
\pgfpathcurveto{\pgfqpoint{1.527535in}{2.820870in}}{\pgfqpoint{1.519635in}{2.817598in}}{\pgfqpoint{1.513811in}{2.811774in}}%
\pgfpathcurveto{\pgfqpoint{1.507987in}{2.805950in}}{\pgfqpoint{1.504715in}{2.798050in}}{\pgfqpoint{1.504715in}{2.789814in}}%
\pgfpathcurveto{\pgfqpoint{1.504715in}{2.781577in}}{\pgfqpoint{1.507987in}{2.773677in}}{\pgfqpoint{1.513811in}{2.767853in}}%
\pgfpathcurveto{\pgfqpoint{1.519635in}{2.762029in}}{\pgfqpoint{1.527535in}{2.758757in}}{\pgfqpoint{1.535772in}{2.758757in}}%
\pgfpathclose%
\pgfusepath{stroke,fill}%
\end{pgfscope}%
\begin{pgfscope}%
\pgfpathrectangle{\pgfqpoint{0.100000in}{0.212622in}}{\pgfqpoint{3.696000in}{3.696000in}}%
\pgfusepath{clip}%
\pgfsetbuttcap%
\pgfsetroundjoin%
\definecolor{currentfill}{rgb}{0.121569,0.466667,0.705882}%
\pgfsetfillcolor{currentfill}%
\pgfsetfillopacity{0.413023}%
\pgfsetlinewidth{1.003750pt}%
\definecolor{currentstroke}{rgb}{0.121569,0.466667,0.705882}%
\pgfsetstrokecolor{currentstroke}%
\pgfsetstrokeopacity{0.413023}%
\pgfsetdash{}{0pt}%
\pgfpathmoveto{\pgfqpoint{1.533369in}{2.754191in}}%
\pgfpathcurveto{\pgfqpoint{1.541605in}{2.754191in}}{\pgfqpoint{1.549505in}{2.757464in}}{\pgfqpoint{1.555329in}{2.763287in}}%
\pgfpathcurveto{\pgfqpoint{1.561153in}{2.769111in}}{\pgfqpoint{1.564425in}{2.777011in}}{\pgfqpoint{1.564425in}{2.785248in}}%
\pgfpathcurveto{\pgfqpoint{1.564425in}{2.793484in}}{\pgfqpoint{1.561153in}{2.801384in}}{\pgfqpoint{1.555329in}{2.807208in}}%
\pgfpathcurveto{\pgfqpoint{1.549505in}{2.813032in}}{\pgfqpoint{1.541605in}{2.816304in}}{\pgfqpoint{1.533369in}{2.816304in}}%
\pgfpathcurveto{\pgfqpoint{1.525133in}{2.816304in}}{\pgfqpoint{1.517233in}{2.813032in}}{\pgfqpoint{1.511409in}{2.807208in}}%
\pgfpathcurveto{\pgfqpoint{1.505585in}{2.801384in}}{\pgfqpoint{1.502312in}{2.793484in}}{\pgfqpoint{1.502312in}{2.785248in}}%
\pgfpathcurveto{\pgfqpoint{1.502312in}{2.777011in}}{\pgfqpoint{1.505585in}{2.769111in}}{\pgfqpoint{1.511409in}{2.763287in}}%
\pgfpathcurveto{\pgfqpoint{1.517233in}{2.757464in}}{\pgfqpoint{1.525133in}{2.754191in}}{\pgfqpoint{1.533369in}{2.754191in}}%
\pgfpathclose%
\pgfusepath{stroke,fill}%
\end{pgfscope}%
\begin{pgfscope}%
\pgfpathrectangle{\pgfqpoint{0.100000in}{0.212622in}}{\pgfqpoint{3.696000in}{3.696000in}}%
\pgfusepath{clip}%
\pgfsetbuttcap%
\pgfsetroundjoin%
\definecolor{currentfill}{rgb}{0.121569,0.466667,0.705882}%
\pgfsetfillcolor{currentfill}%
\pgfsetfillopacity{0.413816}%
\pgfsetlinewidth{1.003750pt}%
\definecolor{currentstroke}{rgb}{0.121569,0.466667,0.705882}%
\pgfsetstrokecolor{currentstroke}%
\pgfsetstrokeopacity{0.413816}%
\pgfsetdash{}{0pt}%
\pgfpathmoveto{\pgfqpoint{1.531009in}{2.750051in}}%
\pgfpathcurveto{\pgfqpoint{1.539245in}{2.750051in}}{\pgfqpoint{1.547145in}{2.753323in}}{\pgfqpoint{1.552969in}{2.759147in}}%
\pgfpathcurveto{\pgfqpoint{1.558793in}{2.764971in}}{\pgfqpoint{1.562066in}{2.772871in}}{\pgfqpoint{1.562066in}{2.781107in}}%
\pgfpathcurveto{\pgfqpoint{1.562066in}{2.789344in}}{\pgfqpoint{1.558793in}{2.797244in}}{\pgfqpoint{1.552969in}{2.803068in}}%
\pgfpathcurveto{\pgfqpoint{1.547145in}{2.808892in}}{\pgfqpoint{1.539245in}{2.812164in}}{\pgfqpoint{1.531009in}{2.812164in}}%
\pgfpathcurveto{\pgfqpoint{1.522773in}{2.812164in}}{\pgfqpoint{1.514873in}{2.808892in}}{\pgfqpoint{1.509049in}{2.803068in}}%
\pgfpathcurveto{\pgfqpoint{1.503225in}{2.797244in}}{\pgfqpoint{1.499953in}{2.789344in}}{\pgfqpoint{1.499953in}{2.781107in}}%
\pgfpathcurveto{\pgfqpoint{1.499953in}{2.772871in}}{\pgfqpoint{1.503225in}{2.764971in}}{\pgfqpoint{1.509049in}{2.759147in}}%
\pgfpathcurveto{\pgfqpoint{1.514873in}{2.753323in}}{\pgfqpoint{1.522773in}{2.750051in}}{\pgfqpoint{1.531009in}{2.750051in}}%
\pgfpathclose%
\pgfusepath{stroke,fill}%
\end{pgfscope}%
\begin{pgfscope}%
\pgfpathrectangle{\pgfqpoint{0.100000in}{0.212622in}}{\pgfqpoint{3.696000in}{3.696000in}}%
\pgfusepath{clip}%
\pgfsetbuttcap%
\pgfsetroundjoin%
\definecolor{currentfill}{rgb}{0.121569,0.466667,0.705882}%
\pgfsetfillcolor{currentfill}%
\pgfsetfillopacity{0.413906}%
\pgfsetlinewidth{1.003750pt}%
\definecolor{currentstroke}{rgb}{0.121569,0.466667,0.705882}%
\pgfsetstrokecolor{currentstroke}%
\pgfsetstrokeopacity{0.413906}%
\pgfsetdash{}{0pt}%
\pgfpathmoveto{\pgfqpoint{1.985144in}{2.868660in}}%
\pgfpathcurveto{\pgfqpoint{1.993380in}{2.868660in}}{\pgfqpoint{2.001280in}{2.871932in}}{\pgfqpoint{2.007104in}{2.877756in}}%
\pgfpathcurveto{\pgfqpoint{2.012928in}{2.883580in}}{\pgfqpoint{2.016201in}{2.891480in}}{\pgfqpoint{2.016201in}{2.899716in}}%
\pgfpathcurveto{\pgfqpoint{2.016201in}{2.907952in}}{\pgfqpoint{2.012928in}{2.915853in}}{\pgfqpoint{2.007104in}{2.921676in}}%
\pgfpathcurveto{\pgfqpoint{2.001280in}{2.927500in}}{\pgfqpoint{1.993380in}{2.930773in}}{\pgfqpoint{1.985144in}{2.930773in}}%
\pgfpathcurveto{\pgfqpoint{1.976908in}{2.930773in}}{\pgfqpoint{1.969008in}{2.927500in}}{\pgfqpoint{1.963184in}{2.921676in}}%
\pgfpathcurveto{\pgfqpoint{1.957360in}{2.915853in}}{\pgfqpoint{1.954088in}{2.907952in}}{\pgfqpoint{1.954088in}{2.899716in}}%
\pgfpathcurveto{\pgfqpoint{1.954088in}{2.891480in}}{\pgfqpoint{1.957360in}{2.883580in}}{\pgfqpoint{1.963184in}{2.877756in}}%
\pgfpathcurveto{\pgfqpoint{1.969008in}{2.871932in}}{\pgfqpoint{1.976908in}{2.868660in}}{\pgfqpoint{1.985144in}{2.868660in}}%
\pgfpathclose%
\pgfusepath{stroke,fill}%
\end{pgfscope}%
\begin{pgfscope}%
\pgfpathrectangle{\pgfqpoint{0.100000in}{0.212622in}}{\pgfqpoint{3.696000in}{3.696000in}}%
\pgfusepath{clip}%
\pgfsetbuttcap%
\pgfsetroundjoin%
\definecolor{currentfill}{rgb}{0.121569,0.466667,0.705882}%
\pgfsetfillcolor{currentfill}%
\pgfsetfillopacity{0.415074}%
\pgfsetlinewidth{1.003750pt}%
\definecolor{currentstroke}{rgb}{0.121569,0.466667,0.705882}%
\pgfsetstrokecolor{currentstroke}%
\pgfsetstrokeopacity{0.415074}%
\pgfsetdash{}{0pt}%
\pgfpathmoveto{\pgfqpoint{1.526438in}{2.742221in}}%
\pgfpathcurveto{\pgfqpoint{1.534675in}{2.742221in}}{\pgfqpoint{1.542575in}{2.745494in}}{\pgfqpoint{1.548398in}{2.751318in}}%
\pgfpathcurveto{\pgfqpoint{1.554222in}{2.757142in}}{\pgfqpoint{1.557495in}{2.765042in}}{\pgfqpoint{1.557495in}{2.773278in}}%
\pgfpathcurveto{\pgfqpoint{1.557495in}{2.781514in}}{\pgfqpoint{1.554222in}{2.789414in}}{\pgfqpoint{1.548398in}{2.795238in}}%
\pgfpathcurveto{\pgfqpoint{1.542575in}{2.801062in}}{\pgfqpoint{1.534675in}{2.804334in}}{\pgfqpoint{1.526438in}{2.804334in}}%
\pgfpathcurveto{\pgfqpoint{1.518202in}{2.804334in}}{\pgfqpoint{1.510302in}{2.801062in}}{\pgfqpoint{1.504478in}{2.795238in}}%
\pgfpathcurveto{\pgfqpoint{1.498654in}{2.789414in}}{\pgfqpoint{1.495382in}{2.781514in}}{\pgfqpoint{1.495382in}{2.773278in}}%
\pgfpathcurveto{\pgfqpoint{1.495382in}{2.765042in}}{\pgfqpoint{1.498654in}{2.757142in}}{\pgfqpoint{1.504478in}{2.751318in}}%
\pgfpathcurveto{\pgfqpoint{1.510302in}{2.745494in}}{\pgfqpoint{1.518202in}{2.742221in}}{\pgfqpoint{1.526438in}{2.742221in}}%
\pgfpathclose%
\pgfusepath{stroke,fill}%
\end{pgfscope}%
\begin{pgfscope}%
\pgfpathrectangle{\pgfqpoint{0.100000in}{0.212622in}}{\pgfqpoint{3.696000in}{3.696000in}}%
\pgfusepath{clip}%
\pgfsetbuttcap%
\pgfsetroundjoin%
\definecolor{currentfill}{rgb}{0.121569,0.466667,0.705882}%
\pgfsetfillcolor{currentfill}%
\pgfsetfillopacity{0.416284}%
\pgfsetlinewidth{1.003750pt}%
\definecolor{currentstroke}{rgb}{0.121569,0.466667,0.705882}%
\pgfsetstrokecolor{currentstroke}%
\pgfsetstrokeopacity{0.416284}%
\pgfsetdash{}{0pt}%
\pgfpathmoveto{\pgfqpoint{1.523091in}{2.734781in}}%
\pgfpathcurveto{\pgfqpoint{1.531328in}{2.734781in}}{\pgfqpoint{1.539228in}{2.738053in}}{\pgfqpoint{1.545052in}{2.743877in}}%
\pgfpathcurveto{\pgfqpoint{1.550875in}{2.749701in}}{\pgfqpoint{1.554148in}{2.757601in}}{\pgfqpoint{1.554148in}{2.765837in}}%
\pgfpathcurveto{\pgfqpoint{1.554148in}{2.774074in}}{\pgfqpoint{1.550875in}{2.781974in}}{\pgfqpoint{1.545052in}{2.787798in}}%
\pgfpathcurveto{\pgfqpoint{1.539228in}{2.793622in}}{\pgfqpoint{1.531328in}{2.796894in}}{\pgfqpoint{1.523091in}{2.796894in}}%
\pgfpathcurveto{\pgfqpoint{1.514855in}{2.796894in}}{\pgfqpoint{1.506955in}{2.793622in}}{\pgfqpoint{1.501131in}{2.787798in}}%
\pgfpathcurveto{\pgfqpoint{1.495307in}{2.781974in}}{\pgfqpoint{1.492035in}{2.774074in}}{\pgfqpoint{1.492035in}{2.765837in}}%
\pgfpathcurveto{\pgfqpoint{1.492035in}{2.757601in}}{\pgfqpoint{1.495307in}{2.749701in}}{\pgfqpoint{1.501131in}{2.743877in}}%
\pgfpathcurveto{\pgfqpoint{1.506955in}{2.738053in}}{\pgfqpoint{1.514855in}{2.734781in}}{\pgfqpoint{1.523091in}{2.734781in}}%
\pgfpathclose%
\pgfusepath{stroke,fill}%
\end{pgfscope}%
\begin{pgfscope}%
\pgfpathrectangle{\pgfqpoint{0.100000in}{0.212622in}}{\pgfqpoint{3.696000in}{3.696000in}}%
\pgfusepath{clip}%
\pgfsetbuttcap%
\pgfsetroundjoin%
\definecolor{currentfill}{rgb}{0.121569,0.466667,0.705882}%
\pgfsetfillcolor{currentfill}%
\pgfsetfillopacity{0.416911}%
\pgfsetlinewidth{1.003750pt}%
\definecolor{currentstroke}{rgb}{0.121569,0.466667,0.705882}%
\pgfsetstrokecolor{currentstroke}%
\pgfsetstrokeopacity{0.416911}%
\pgfsetdash{}{0pt}%
\pgfpathmoveto{\pgfqpoint{1.520821in}{2.731024in}}%
\pgfpathcurveto{\pgfqpoint{1.529058in}{2.731024in}}{\pgfqpoint{1.536958in}{2.734296in}}{\pgfqpoint{1.542782in}{2.740120in}}%
\pgfpathcurveto{\pgfqpoint{1.548606in}{2.745944in}}{\pgfqpoint{1.551878in}{2.753844in}}{\pgfqpoint{1.551878in}{2.762080in}}%
\pgfpathcurveto{\pgfqpoint{1.551878in}{2.770317in}}{\pgfqpoint{1.548606in}{2.778217in}}{\pgfqpoint{1.542782in}{2.784041in}}%
\pgfpathcurveto{\pgfqpoint{1.536958in}{2.789864in}}{\pgfqpoint{1.529058in}{2.793137in}}{\pgfqpoint{1.520821in}{2.793137in}}%
\pgfpathcurveto{\pgfqpoint{1.512585in}{2.793137in}}{\pgfqpoint{1.504685in}{2.789864in}}{\pgfqpoint{1.498861in}{2.784041in}}%
\pgfpathcurveto{\pgfqpoint{1.493037in}{2.778217in}}{\pgfqpoint{1.489765in}{2.770317in}}{\pgfqpoint{1.489765in}{2.762080in}}%
\pgfpathcurveto{\pgfqpoint{1.489765in}{2.753844in}}{\pgfqpoint{1.493037in}{2.745944in}}{\pgfqpoint{1.498861in}{2.740120in}}%
\pgfpathcurveto{\pgfqpoint{1.504685in}{2.734296in}}{\pgfqpoint{1.512585in}{2.731024in}}{\pgfqpoint{1.520821in}{2.731024in}}%
\pgfpathclose%
\pgfusepath{stroke,fill}%
\end{pgfscope}%
\begin{pgfscope}%
\pgfpathrectangle{\pgfqpoint{0.100000in}{0.212622in}}{\pgfqpoint{3.696000in}{3.696000in}}%
\pgfusepath{clip}%
\pgfsetbuttcap%
\pgfsetroundjoin%
\definecolor{currentfill}{rgb}{0.121569,0.466667,0.705882}%
\pgfsetfillcolor{currentfill}%
\pgfsetfillopacity{0.417160}%
\pgfsetlinewidth{1.003750pt}%
\definecolor{currentstroke}{rgb}{0.121569,0.466667,0.705882}%
\pgfsetstrokecolor{currentstroke}%
\pgfsetstrokeopacity{0.417160}%
\pgfsetdash{}{0pt}%
\pgfpathmoveto{\pgfqpoint{1.988076in}{2.853776in}}%
\pgfpathcurveto{\pgfqpoint{1.996312in}{2.853776in}}{\pgfqpoint{2.004212in}{2.857049in}}{\pgfqpoint{2.010036in}{2.862872in}}%
\pgfpathcurveto{\pgfqpoint{2.015860in}{2.868696in}}{\pgfqpoint{2.019132in}{2.876596in}}{\pgfqpoint{2.019132in}{2.884833in}}%
\pgfpathcurveto{\pgfqpoint{2.019132in}{2.893069in}}{\pgfqpoint{2.015860in}{2.900969in}}{\pgfqpoint{2.010036in}{2.906793in}}%
\pgfpathcurveto{\pgfqpoint{2.004212in}{2.912617in}}{\pgfqpoint{1.996312in}{2.915889in}}{\pgfqpoint{1.988076in}{2.915889in}}%
\pgfpathcurveto{\pgfqpoint{1.979840in}{2.915889in}}{\pgfqpoint{1.971940in}{2.912617in}}{\pgfqpoint{1.966116in}{2.906793in}}%
\pgfpathcurveto{\pgfqpoint{1.960292in}{2.900969in}}{\pgfqpoint{1.957019in}{2.893069in}}{\pgfqpoint{1.957019in}{2.884833in}}%
\pgfpathcurveto{\pgfqpoint{1.957019in}{2.876596in}}{\pgfqpoint{1.960292in}{2.868696in}}{\pgfqpoint{1.966116in}{2.862872in}}%
\pgfpathcurveto{\pgfqpoint{1.971940in}{2.857049in}}{\pgfqpoint{1.979840in}{2.853776in}}{\pgfqpoint{1.988076in}{2.853776in}}%
\pgfpathclose%
\pgfusepath{stroke,fill}%
\end{pgfscope}%
\begin{pgfscope}%
\pgfpathrectangle{\pgfqpoint{0.100000in}{0.212622in}}{\pgfqpoint{3.696000in}{3.696000in}}%
\pgfusepath{clip}%
\pgfsetbuttcap%
\pgfsetroundjoin%
\definecolor{currentfill}{rgb}{0.121569,0.466667,0.705882}%
\pgfsetfillcolor{currentfill}%
\pgfsetfillopacity{0.417438}%
\pgfsetlinewidth{1.003750pt}%
\definecolor{currentstroke}{rgb}{0.121569,0.466667,0.705882}%
\pgfsetstrokecolor{currentstroke}%
\pgfsetstrokeopacity{0.417438}%
\pgfsetdash{}{0pt}%
\pgfpathmoveto{\pgfqpoint{1.519317in}{2.727915in}}%
\pgfpathcurveto{\pgfqpoint{1.527554in}{2.727915in}}{\pgfqpoint{1.535454in}{2.731187in}}{\pgfqpoint{1.541278in}{2.737011in}}%
\pgfpathcurveto{\pgfqpoint{1.547102in}{2.742835in}}{\pgfqpoint{1.550374in}{2.750735in}}{\pgfqpoint{1.550374in}{2.758971in}}%
\pgfpathcurveto{\pgfqpoint{1.550374in}{2.767208in}}{\pgfqpoint{1.547102in}{2.775108in}}{\pgfqpoint{1.541278in}{2.780932in}}%
\pgfpathcurveto{\pgfqpoint{1.535454in}{2.786756in}}{\pgfqpoint{1.527554in}{2.790028in}}{\pgfqpoint{1.519317in}{2.790028in}}%
\pgfpathcurveto{\pgfqpoint{1.511081in}{2.790028in}}{\pgfqpoint{1.503181in}{2.786756in}}{\pgfqpoint{1.497357in}{2.780932in}}%
\pgfpathcurveto{\pgfqpoint{1.491533in}{2.775108in}}{\pgfqpoint{1.488261in}{2.767208in}}{\pgfqpoint{1.488261in}{2.758971in}}%
\pgfpathcurveto{\pgfqpoint{1.488261in}{2.750735in}}{\pgfqpoint{1.491533in}{2.742835in}}{\pgfqpoint{1.497357in}{2.737011in}}%
\pgfpathcurveto{\pgfqpoint{1.503181in}{2.731187in}}{\pgfqpoint{1.511081in}{2.727915in}}{\pgfqpoint{1.519317in}{2.727915in}}%
\pgfpathclose%
\pgfusepath{stroke,fill}%
\end{pgfscope}%
\begin{pgfscope}%
\pgfpathrectangle{\pgfqpoint{0.100000in}{0.212622in}}{\pgfqpoint{3.696000in}{3.696000in}}%
\pgfusepath{clip}%
\pgfsetbuttcap%
\pgfsetroundjoin%
\definecolor{currentfill}{rgb}{0.121569,0.466667,0.705882}%
\pgfsetfillcolor{currentfill}%
\pgfsetfillopacity{0.418405}%
\pgfsetlinewidth{1.003750pt}%
\definecolor{currentstroke}{rgb}{0.121569,0.466667,0.705882}%
\pgfsetstrokecolor{currentstroke}%
\pgfsetstrokeopacity{0.418405}%
\pgfsetdash{}{0pt}%
\pgfpathmoveto{\pgfqpoint{1.516343in}{2.722590in}}%
\pgfpathcurveto{\pgfqpoint{1.524580in}{2.722590in}}{\pgfqpoint{1.532480in}{2.725862in}}{\pgfqpoint{1.538304in}{2.731686in}}%
\pgfpathcurveto{\pgfqpoint{1.544128in}{2.737510in}}{\pgfqpoint{1.547400in}{2.745410in}}{\pgfqpoint{1.547400in}{2.753646in}}%
\pgfpathcurveto{\pgfqpoint{1.547400in}{2.761883in}}{\pgfqpoint{1.544128in}{2.769783in}}{\pgfqpoint{1.538304in}{2.775607in}}%
\pgfpathcurveto{\pgfqpoint{1.532480in}{2.781431in}}{\pgfqpoint{1.524580in}{2.784703in}}{\pgfqpoint{1.516343in}{2.784703in}}%
\pgfpathcurveto{\pgfqpoint{1.508107in}{2.784703in}}{\pgfqpoint{1.500207in}{2.781431in}}{\pgfqpoint{1.494383in}{2.775607in}}%
\pgfpathcurveto{\pgfqpoint{1.488559in}{2.769783in}}{\pgfqpoint{1.485287in}{2.761883in}}{\pgfqpoint{1.485287in}{2.753646in}}%
\pgfpathcurveto{\pgfqpoint{1.485287in}{2.745410in}}{\pgfqpoint{1.488559in}{2.737510in}}{\pgfqpoint{1.494383in}{2.731686in}}%
\pgfpathcurveto{\pgfqpoint{1.500207in}{2.725862in}}{\pgfqpoint{1.508107in}{2.722590in}}{\pgfqpoint{1.516343in}{2.722590in}}%
\pgfpathclose%
\pgfusepath{stroke,fill}%
\end{pgfscope}%
\begin{pgfscope}%
\pgfpathrectangle{\pgfqpoint{0.100000in}{0.212622in}}{\pgfqpoint{3.696000in}{3.696000in}}%
\pgfusepath{clip}%
\pgfsetbuttcap%
\pgfsetroundjoin%
\definecolor{currentfill}{rgb}{0.121569,0.466667,0.705882}%
\pgfsetfillcolor{currentfill}%
\pgfsetfillopacity{0.420013}%
\pgfsetlinewidth{1.003750pt}%
\definecolor{currentstroke}{rgb}{0.121569,0.466667,0.705882}%
\pgfsetstrokecolor{currentstroke}%
\pgfsetstrokeopacity{0.420013}%
\pgfsetdash{}{0pt}%
\pgfpathmoveto{\pgfqpoint{1.510662in}{2.712724in}}%
\pgfpathcurveto{\pgfqpoint{1.518898in}{2.712724in}}{\pgfqpoint{1.526798in}{2.715996in}}{\pgfqpoint{1.532622in}{2.721820in}}%
\pgfpathcurveto{\pgfqpoint{1.538446in}{2.727644in}}{\pgfqpoint{1.541718in}{2.735544in}}{\pgfqpoint{1.541718in}{2.743780in}}%
\pgfpathcurveto{\pgfqpoint{1.541718in}{2.752016in}}{\pgfqpoint{1.538446in}{2.759916in}}{\pgfqpoint{1.532622in}{2.765740in}}%
\pgfpathcurveto{\pgfqpoint{1.526798in}{2.771564in}}{\pgfqpoint{1.518898in}{2.774837in}}{\pgfqpoint{1.510662in}{2.774837in}}%
\pgfpathcurveto{\pgfqpoint{1.502426in}{2.774837in}}{\pgfqpoint{1.494525in}{2.771564in}}{\pgfqpoint{1.488702in}{2.765740in}}%
\pgfpathcurveto{\pgfqpoint{1.482878in}{2.759916in}}{\pgfqpoint{1.479605in}{2.752016in}}{\pgfqpoint{1.479605in}{2.743780in}}%
\pgfpathcurveto{\pgfqpoint{1.479605in}{2.735544in}}{\pgfqpoint{1.482878in}{2.727644in}}{\pgfqpoint{1.488702in}{2.721820in}}%
\pgfpathcurveto{\pgfqpoint{1.494525in}{2.715996in}}{\pgfqpoint{1.502426in}{2.712724in}}{\pgfqpoint{1.510662in}{2.712724in}}%
\pgfpathclose%
\pgfusepath{stroke,fill}%
\end{pgfscope}%
\begin{pgfscope}%
\pgfpathrectangle{\pgfqpoint{0.100000in}{0.212622in}}{\pgfqpoint{3.696000in}{3.696000in}}%
\pgfusepath{clip}%
\pgfsetbuttcap%
\pgfsetroundjoin%
\definecolor{currentfill}{rgb}{0.121569,0.466667,0.705882}%
\pgfsetfillcolor{currentfill}%
\pgfsetfillopacity{0.420493}%
\pgfsetlinewidth{1.003750pt}%
\definecolor{currentstroke}{rgb}{0.121569,0.466667,0.705882}%
\pgfsetstrokecolor{currentstroke}%
\pgfsetstrokeopacity{0.420493}%
\pgfsetdash{}{0pt}%
\pgfpathmoveto{\pgfqpoint{1.990323in}{2.837026in}}%
\pgfpathcurveto{\pgfqpoint{1.998559in}{2.837026in}}{\pgfqpoint{2.006459in}{2.840299in}}{\pgfqpoint{2.012283in}{2.846122in}}%
\pgfpathcurveto{\pgfqpoint{2.018107in}{2.851946in}}{\pgfqpoint{2.021380in}{2.859846in}}{\pgfqpoint{2.021380in}{2.868083in}}%
\pgfpathcurveto{\pgfqpoint{2.021380in}{2.876319in}}{\pgfqpoint{2.018107in}{2.884219in}}{\pgfqpoint{2.012283in}{2.890043in}}%
\pgfpathcurveto{\pgfqpoint{2.006459in}{2.895867in}}{\pgfqpoint{1.998559in}{2.899139in}}{\pgfqpoint{1.990323in}{2.899139in}}%
\pgfpathcurveto{\pgfqpoint{1.982087in}{2.899139in}}{\pgfqpoint{1.974187in}{2.895867in}}{\pgfqpoint{1.968363in}{2.890043in}}%
\pgfpathcurveto{\pgfqpoint{1.962539in}{2.884219in}}{\pgfqpoint{1.959267in}{2.876319in}}{\pgfqpoint{1.959267in}{2.868083in}}%
\pgfpathcurveto{\pgfqpoint{1.959267in}{2.859846in}}{\pgfqpoint{1.962539in}{2.851946in}}{\pgfqpoint{1.968363in}{2.846122in}}%
\pgfpathcurveto{\pgfqpoint{1.974187in}{2.840299in}}{\pgfqpoint{1.982087in}{2.837026in}}{\pgfqpoint{1.990323in}{2.837026in}}%
\pgfpathclose%
\pgfusepath{stroke,fill}%
\end{pgfscope}%
\begin{pgfscope}%
\pgfpathrectangle{\pgfqpoint{0.100000in}{0.212622in}}{\pgfqpoint{3.696000in}{3.696000in}}%
\pgfusepath{clip}%
\pgfsetbuttcap%
\pgfsetroundjoin%
\definecolor{currentfill}{rgb}{0.121569,0.466667,0.705882}%
\pgfsetfillcolor{currentfill}%
\pgfsetfillopacity{0.421648}%
\pgfsetlinewidth{1.003750pt}%
\definecolor{currentstroke}{rgb}{0.121569,0.466667,0.705882}%
\pgfsetstrokecolor{currentstroke}%
\pgfsetstrokeopacity{0.421648}%
\pgfsetdash{}{0pt}%
\pgfpathmoveto{\pgfqpoint{1.506082in}{2.702830in}}%
\pgfpathcurveto{\pgfqpoint{1.514318in}{2.702830in}}{\pgfqpoint{1.522218in}{2.706103in}}{\pgfqpoint{1.528042in}{2.711927in}}%
\pgfpathcurveto{\pgfqpoint{1.533866in}{2.717750in}}{\pgfqpoint{1.537138in}{2.725651in}}{\pgfqpoint{1.537138in}{2.733887in}}%
\pgfpathcurveto{\pgfqpoint{1.537138in}{2.742123in}}{\pgfqpoint{1.533866in}{2.750023in}}{\pgfqpoint{1.528042in}{2.755847in}}%
\pgfpathcurveto{\pgfqpoint{1.522218in}{2.761671in}}{\pgfqpoint{1.514318in}{2.764943in}}{\pgfqpoint{1.506082in}{2.764943in}}%
\pgfpathcurveto{\pgfqpoint{1.497845in}{2.764943in}}{\pgfqpoint{1.489945in}{2.761671in}}{\pgfqpoint{1.484122in}{2.755847in}}%
\pgfpathcurveto{\pgfqpoint{1.478298in}{2.750023in}}{\pgfqpoint{1.475025in}{2.742123in}}{\pgfqpoint{1.475025in}{2.733887in}}%
\pgfpathcurveto{\pgfqpoint{1.475025in}{2.725651in}}{\pgfqpoint{1.478298in}{2.717750in}}{\pgfqpoint{1.484122in}{2.711927in}}%
\pgfpathcurveto{\pgfqpoint{1.489945in}{2.706103in}}{\pgfqpoint{1.497845in}{2.702830in}}{\pgfqpoint{1.506082in}{2.702830in}}%
\pgfpathclose%
\pgfusepath{stroke,fill}%
\end{pgfscope}%
\begin{pgfscope}%
\pgfpathrectangle{\pgfqpoint{0.100000in}{0.212622in}}{\pgfqpoint{3.696000in}{3.696000in}}%
\pgfusepath{clip}%
\pgfsetbuttcap%
\pgfsetroundjoin%
\definecolor{currentfill}{rgb}{0.121569,0.466667,0.705882}%
\pgfsetfillcolor{currentfill}%
\pgfsetfillopacity{0.422628}%
\pgfsetlinewidth{1.003750pt}%
\definecolor{currentstroke}{rgb}{0.121569,0.466667,0.705882}%
\pgfsetstrokecolor{currentstroke}%
\pgfsetstrokeopacity{0.422628}%
\pgfsetdash{}{0pt}%
\pgfpathmoveto{\pgfqpoint{1.502349in}{2.696633in}}%
\pgfpathcurveto{\pgfqpoint{1.510585in}{2.696633in}}{\pgfqpoint{1.518485in}{2.699905in}}{\pgfqpoint{1.524309in}{2.705729in}}%
\pgfpathcurveto{\pgfqpoint{1.530133in}{2.711553in}}{\pgfqpoint{1.533405in}{2.719453in}}{\pgfqpoint{1.533405in}{2.727689in}}%
\pgfpathcurveto{\pgfqpoint{1.533405in}{2.735925in}}{\pgfqpoint{1.530133in}{2.743825in}}{\pgfqpoint{1.524309in}{2.749649in}}%
\pgfpathcurveto{\pgfqpoint{1.518485in}{2.755473in}}{\pgfqpoint{1.510585in}{2.758746in}}{\pgfqpoint{1.502349in}{2.758746in}}%
\pgfpathcurveto{\pgfqpoint{1.494112in}{2.758746in}}{\pgfqpoint{1.486212in}{2.755473in}}{\pgfqpoint{1.480388in}{2.749649in}}%
\pgfpathcurveto{\pgfqpoint{1.474564in}{2.743825in}}{\pgfqpoint{1.471292in}{2.735925in}}{\pgfqpoint{1.471292in}{2.727689in}}%
\pgfpathcurveto{\pgfqpoint{1.471292in}{2.719453in}}{\pgfqpoint{1.474564in}{2.711553in}}{\pgfqpoint{1.480388in}{2.705729in}}%
\pgfpathcurveto{\pgfqpoint{1.486212in}{2.699905in}}{\pgfqpoint{1.494112in}{2.696633in}}{\pgfqpoint{1.502349in}{2.696633in}}%
\pgfpathclose%
\pgfusepath{stroke,fill}%
\end{pgfscope}%
\begin{pgfscope}%
\pgfpathrectangle{\pgfqpoint{0.100000in}{0.212622in}}{\pgfqpoint{3.696000in}{3.696000in}}%
\pgfusepath{clip}%
\pgfsetbuttcap%
\pgfsetroundjoin%
\definecolor{currentfill}{rgb}{0.121569,0.466667,0.705882}%
\pgfsetfillcolor{currentfill}%
\pgfsetfillopacity{0.424411}%
\pgfsetlinewidth{1.003750pt}%
\definecolor{currentstroke}{rgb}{0.121569,0.466667,0.705882}%
\pgfsetstrokecolor{currentstroke}%
\pgfsetstrokeopacity{0.424411}%
\pgfsetdash{}{0pt}%
\pgfpathmoveto{\pgfqpoint{1.992210in}{2.820184in}}%
\pgfpathcurveto{\pgfqpoint{2.000446in}{2.820184in}}{\pgfqpoint{2.008347in}{2.823457in}}{\pgfqpoint{2.014170in}{2.829280in}}%
\pgfpathcurveto{\pgfqpoint{2.019994in}{2.835104in}}{\pgfqpoint{2.023267in}{2.843004in}}{\pgfqpoint{2.023267in}{2.851241in}}%
\pgfpathcurveto{\pgfqpoint{2.023267in}{2.859477in}}{\pgfqpoint{2.019994in}{2.867377in}}{\pgfqpoint{2.014170in}{2.873201in}}%
\pgfpathcurveto{\pgfqpoint{2.008347in}{2.879025in}}{\pgfqpoint{2.000446in}{2.882297in}}{\pgfqpoint{1.992210in}{2.882297in}}%
\pgfpathcurveto{\pgfqpoint{1.983974in}{2.882297in}}{\pgfqpoint{1.976074in}{2.879025in}}{\pgfqpoint{1.970250in}{2.873201in}}%
\pgfpathcurveto{\pgfqpoint{1.964426in}{2.867377in}}{\pgfqpoint{1.961154in}{2.859477in}}{\pgfqpoint{1.961154in}{2.851241in}}%
\pgfpathcurveto{\pgfqpoint{1.961154in}{2.843004in}}{\pgfqpoint{1.964426in}{2.835104in}}{\pgfqpoint{1.970250in}{2.829280in}}%
\pgfpathcurveto{\pgfqpoint{1.976074in}{2.823457in}}{\pgfqpoint{1.983974in}{2.820184in}}{\pgfqpoint{1.992210in}{2.820184in}}%
\pgfpathclose%
\pgfusepath{stroke,fill}%
\end{pgfscope}%
\begin{pgfscope}%
\pgfpathrectangle{\pgfqpoint{0.100000in}{0.212622in}}{\pgfqpoint{3.696000in}{3.696000in}}%
\pgfusepath{clip}%
\pgfsetbuttcap%
\pgfsetroundjoin%
\definecolor{currentfill}{rgb}{0.121569,0.466667,0.705882}%
\pgfsetfillcolor{currentfill}%
\pgfsetfillopacity{0.424678}%
\pgfsetlinewidth{1.003750pt}%
\definecolor{currentstroke}{rgb}{0.121569,0.466667,0.705882}%
\pgfsetstrokecolor{currentstroke}%
\pgfsetstrokeopacity{0.424678}%
\pgfsetdash{}{0pt}%
\pgfpathmoveto{\pgfqpoint{1.496463in}{2.685102in}}%
\pgfpathcurveto{\pgfqpoint{1.504700in}{2.685102in}}{\pgfqpoint{1.512600in}{2.688374in}}{\pgfqpoint{1.518424in}{2.694198in}}%
\pgfpathcurveto{\pgfqpoint{1.524248in}{2.700022in}}{\pgfqpoint{1.527520in}{2.707922in}}{\pgfqpoint{1.527520in}{2.716158in}}%
\pgfpathcurveto{\pgfqpoint{1.527520in}{2.724394in}}{\pgfqpoint{1.524248in}{2.732294in}}{\pgfqpoint{1.518424in}{2.738118in}}%
\pgfpathcurveto{\pgfqpoint{1.512600in}{2.743942in}}{\pgfqpoint{1.504700in}{2.747215in}}{\pgfqpoint{1.496463in}{2.747215in}}%
\pgfpathcurveto{\pgfqpoint{1.488227in}{2.747215in}}{\pgfqpoint{1.480327in}{2.743942in}}{\pgfqpoint{1.474503in}{2.738118in}}%
\pgfpathcurveto{\pgfqpoint{1.468679in}{2.732294in}}{\pgfqpoint{1.465407in}{2.724394in}}{\pgfqpoint{1.465407in}{2.716158in}}%
\pgfpathcurveto{\pgfqpoint{1.465407in}{2.707922in}}{\pgfqpoint{1.468679in}{2.700022in}}{\pgfqpoint{1.474503in}{2.694198in}}%
\pgfpathcurveto{\pgfqpoint{1.480327in}{2.688374in}}{\pgfqpoint{1.488227in}{2.685102in}}{\pgfqpoint{1.496463in}{2.685102in}}%
\pgfpathclose%
\pgfusepath{stroke,fill}%
\end{pgfscope}%
\begin{pgfscope}%
\pgfpathrectangle{\pgfqpoint{0.100000in}{0.212622in}}{\pgfqpoint{3.696000in}{3.696000in}}%
\pgfusepath{clip}%
\pgfsetbuttcap%
\pgfsetroundjoin%
\definecolor{currentfill}{rgb}{0.121569,0.466667,0.705882}%
\pgfsetfillcolor{currentfill}%
\pgfsetfillopacity{0.426614}%
\pgfsetlinewidth{1.003750pt}%
\definecolor{currentstroke}{rgb}{0.121569,0.466667,0.705882}%
\pgfsetstrokecolor{currentstroke}%
\pgfsetstrokeopacity{0.426614}%
\pgfsetdash{}{0pt}%
\pgfpathmoveto{\pgfqpoint{1.490574in}{2.674275in}}%
\pgfpathcurveto{\pgfqpoint{1.498810in}{2.674275in}}{\pgfqpoint{1.506710in}{2.677547in}}{\pgfqpoint{1.512534in}{2.683371in}}%
\pgfpathcurveto{\pgfqpoint{1.518358in}{2.689195in}}{\pgfqpoint{1.521630in}{2.697095in}}{\pgfqpoint{1.521630in}{2.705331in}}%
\pgfpathcurveto{\pgfqpoint{1.521630in}{2.713567in}}{\pgfqpoint{1.518358in}{2.721467in}}{\pgfqpoint{1.512534in}{2.727291in}}%
\pgfpathcurveto{\pgfqpoint{1.506710in}{2.733115in}}{\pgfqpoint{1.498810in}{2.736388in}}{\pgfqpoint{1.490574in}{2.736388in}}%
\pgfpathcurveto{\pgfqpoint{1.482338in}{2.736388in}}{\pgfqpoint{1.474438in}{2.733115in}}{\pgfqpoint{1.468614in}{2.727291in}}%
\pgfpathcurveto{\pgfqpoint{1.462790in}{2.721467in}}{\pgfqpoint{1.459517in}{2.713567in}}{\pgfqpoint{1.459517in}{2.705331in}}%
\pgfpathcurveto{\pgfqpoint{1.459517in}{2.697095in}}{\pgfqpoint{1.462790in}{2.689195in}}{\pgfqpoint{1.468614in}{2.683371in}}%
\pgfpathcurveto{\pgfqpoint{1.474438in}{2.677547in}}{\pgfqpoint{1.482338in}{2.674275in}}{\pgfqpoint{1.490574in}{2.674275in}}%
\pgfpathclose%
\pgfusepath{stroke,fill}%
\end{pgfscope}%
\begin{pgfscope}%
\pgfpathrectangle{\pgfqpoint{0.100000in}{0.212622in}}{\pgfqpoint{3.696000in}{3.696000in}}%
\pgfusepath{clip}%
\pgfsetbuttcap%
\pgfsetroundjoin%
\definecolor{currentfill}{rgb}{0.121569,0.466667,0.705882}%
\pgfsetfillcolor{currentfill}%
\pgfsetfillopacity{0.428122}%
\pgfsetlinewidth{1.003750pt}%
\definecolor{currentstroke}{rgb}{0.121569,0.466667,0.705882}%
\pgfsetstrokecolor{currentstroke}%
\pgfsetstrokeopacity{0.428122}%
\pgfsetdash{}{0pt}%
\pgfpathmoveto{\pgfqpoint{1.484598in}{2.663675in}}%
\pgfpathcurveto{\pgfqpoint{1.492835in}{2.663675in}}{\pgfqpoint{1.500735in}{2.666947in}}{\pgfqpoint{1.506559in}{2.672771in}}%
\pgfpathcurveto{\pgfqpoint{1.512382in}{2.678595in}}{\pgfqpoint{1.515655in}{2.686495in}}{\pgfqpoint{1.515655in}{2.694731in}}%
\pgfpathcurveto{\pgfqpoint{1.515655in}{2.702967in}}{\pgfqpoint{1.512382in}{2.710868in}}{\pgfqpoint{1.506559in}{2.716691in}}%
\pgfpathcurveto{\pgfqpoint{1.500735in}{2.722515in}}{\pgfqpoint{1.492835in}{2.725788in}}{\pgfqpoint{1.484598in}{2.725788in}}%
\pgfpathcurveto{\pgfqpoint{1.476362in}{2.725788in}}{\pgfqpoint{1.468462in}{2.722515in}}{\pgfqpoint{1.462638in}{2.716691in}}%
\pgfpathcurveto{\pgfqpoint{1.456814in}{2.710868in}}{\pgfqpoint{1.453542in}{2.702967in}}{\pgfqpoint{1.453542in}{2.694731in}}%
\pgfpathcurveto{\pgfqpoint{1.453542in}{2.686495in}}{\pgfqpoint{1.456814in}{2.678595in}}{\pgfqpoint{1.462638in}{2.672771in}}%
\pgfpathcurveto{\pgfqpoint{1.468462in}{2.666947in}}{\pgfqpoint{1.476362in}{2.663675in}}{\pgfqpoint{1.484598in}{2.663675in}}%
\pgfpathclose%
\pgfusepath{stroke,fill}%
\end{pgfscope}%
\begin{pgfscope}%
\pgfpathrectangle{\pgfqpoint{0.100000in}{0.212622in}}{\pgfqpoint{3.696000in}{3.696000in}}%
\pgfusepath{clip}%
\pgfsetbuttcap%
\pgfsetroundjoin%
\definecolor{currentfill}{rgb}{0.121569,0.466667,0.705882}%
\pgfsetfillcolor{currentfill}%
\pgfsetfillopacity{0.428300}%
\pgfsetlinewidth{1.003750pt}%
\definecolor{currentstroke}{rgb}{0.121569,0.466667,0.705882}%
\pgfsetstrokecolor{currentstroke}%
\pgfsetstrokeopacity{0.428300}%
\pgfsetdash{}{0pt}%
\pgfpathmoveto{\pgfqpoint{1.995548in}{2.801457in}}%
\pgfpathcurveto{\pgfqpoint{2.003784in}{2.801457in}}{\pgfqpoint{2.011684in}{2.804729in}}{\pgfqpoint{2.017508in}{2.810553in}}%
\pgfpathcurveto{\pgfqpoint{2.023332in}{2.816377in}}{\pgfqpoint{2.026605in}{2.824277in}}{\pgfqpoint{2.026605in}{2.832513in}}%
\pgfpathcurveto{\pgfqpoint{2.026605in}{2.840750in}}{\pgfqpoint{2.023332in}{2.848650in}}{\pgfqpoint{2.017508in}{2.854474in}}%
\pgfpathcurveto{\pgfqpoint{2.011684in}{2.860298in}}{\pgfqpoint{2.003784in}{2.863570in}}{\pgfqpoint{1.995548in}{2.863570in}}%
\pgfpathcurveto{\pgfqpoint{1.987312in}{2.863570in}}{\pgfqpoint{1.979412in}{2.860298in}}{\pgfqpoint{1.973588in}{2.854474in}}%
\pgfpathcurveto{\pgfqpoint{1.967764in}{2.848650in}}{\pgfqpoint{1.964492in}{2.840750in}}{\pgfqpoint{1.964492in}{2.832513in}}%
\pgfpathcurveto{\pgfqpoint{1.964492in}{2.824277in}}{\pgfqpoint{1.967764in}{2.816377in}}{\pgfqpoint{1.973588in}{2.810553in}}%
\pgfpathcurveto{\pgfqpoint{1.979412in}{2.804729in}}{\pgfqpoint{1.987312in}{2.801457in}}{\pgfqpoint{1.995548in}{2.801457in}}%
\pgfpathclose%
\pgfusepath{stroke,fill}%
\end{pgfscope}%
\begin{pgfscope}%
\pgfpathrectangle{\pgfqpoint{0.100000in}{0.212622in}}{\pgfqpoint{3.696000in}{3.696000in}}%
\pgfusepath{clip}%
\pgfsetbuttcap%
\pgfsetroundjoin%
\definecolor{currentfill}{rgb}{0.121569,0.466667,0.705882}%
\pgfsetfillcolor{currentfill}%
\pgfsetfillopacity{0.429585}%
\pgfsetlinewidth{1.003750pt}%
\definecolor{currentstroke}{rgb}{0.121569,0.466667,0.705882}%
\pgfsetstrokecolor{currentstroke}%
\pgfsetstrokeopacity{0.429585}%
\pgfsetdash{}{0pt}%
\pgfpathmoveto{\pgfqpoint{1.480252in}{2.653510in}}%
\pgfpathcurveto{\pgfqpoint{1.488489in}{2.653510in}}{\pgfqpoint{1.496389in}{2.656782in}}{\pgfqpoint{1.502213in}{2.662606in}}%
\pgfpathcurveto{\pgfqpoint{1.508037in}{2.668430in}}{\pgfqpoint{1.511309in}{2.676330in}}{\pgfqpoint{1.511309in}{2.684566in}}%
\pgfpathcurveto{\pgfqpoint{1.511309in}{2.692802in}}{\pgfqpoint{1.508037in}{2.700702in}}{\pgfqpoint{1.502213in}{2.706526in}}%
\pgfpathcurveto{\pgfqpoint{1.496389in}{2.712350in}}{\pgfqpoint{1.488489in}{2.715623in}}{\pgfqpoint{1.480252in}{2.715623in}}%
\pgfpathcurveto{\pgfqpoint{1.472016in}{2.715623in}}{\pgfqpoint{1.464116in}{2.712350in}}{\pgfqpoint{1.458292in}{2.706526in}}%
\pgfpathcurveto{\pgfqpoint{1.452468in}{2.700702in}}{\pgfqpoint{1.449196in}{2.692802in}}{\pgfqpoint{1.449196in}{2.684566in}}%
\pgfpathcurveto{\pgfqpoint{1.449196in}{2.676330in}}{\pgfqpoint{1.452468in}{2.668430in}}{\pgfqpoint{1.458292in}{2.662606in}}%
\pgfpathcurveto{\pgfqpoint{1.464116in}{2.656782in}}{\pgfqpoint{1.472016in}{2.653510in}}{\pgfqpoint{1.480252in}{2.653510in}}%
\pgfpathclose%
\pgfusepath{stroke,fill}%
\end{pgfscope}%
\begin{pgfscope}%
\pgfpathrectangle{\pgfqpoint{0.100000in}{0.212622in}}{\pgfqpoint{3.696000in}{3.696000in}}%
\pgfusepath{clip}%
\pgfsetbuttcap%
\pgfsetroundjoin%
\definecolor{currentfill}{rgb}{0.121569,0.466667,0.705882}%
\pgfsetfillcolor{currentfill}%
\pgfsetfillopacity{0.430602}%
\pgfsetlinewidth{1.003750pt}%
\definecolor{currentstroke}{rgb}{0.121569,0.466667,0.705882}%
\pgfsetstrokecolor{currentstroke}%
\pgfsetstrokeopacity{0.430602}%
\pgfsetdash{}{0pt}%
\pgfpathmoveto{\pgfqpoint{1.476316in}{2.647057in}}%
\pgfpathcurveto{\pgfqpoint{1.484552in}{2.647057in}}{\pgfqpoint{1.492452in}{2.650329in}}{\pgfqpoint{1.498276in}{2.656153in}}%
\pgfpathcurveto{\pgfqpoint{1.504100in}{2.661977in}}{\pgfqpoint{1.507372in}{2.669877in}}{\pgfqpoint{1.507372in}{2.678113in}}%
\pgfpathcurveto{\pgfqpoint{1.507372in}{2.686349in}}{\pgfqpoint{1.504100in}{2.694249in}}{\pgfqpoint{1.498276in}{2.700073in}}%
\pgfpathcurveto{\pgfqpoint{1.492452in}{2.705897in}}{\pgfqpoint{1.484552in}{2.709170in}}{\pgfqpoint{1.476316in}{2.709170in}}%
\pgfpathcurveto{\pgfqpoint{1.468080in}{2.709170in}}{\pgfqpoint{1.460179in}{2.705897in}}{\pgfqpoint{1.454356in}{2.700073in}}%
\pgfpathcurveto{\pgfqpoint{1.448532in}{2.694249in}}{\pgfqpoint{1.445259in}{2.686349in}}{\pgfqpoint{1.445259in}{2.678113in}}%
\pgfpathcurveto{\pgfqpoint{1.445259in}{2.669877in}}{\pgfqpoint{1.448532in}{2.661977in}}{\pgfqpoint{1.454356in}{2.656153in}}%
\pgfpathcurveto{\pgfqpoint{1.460179in}{2.650329in}}{\pgfqpoint{1.468080in}{2.647057in}}{\pgfqpoint{1.476316in}{2.647057in}}%
\pgfpathclose%
\pgfusepath{stroke,fill}%
\end{pgfscope}%
\begin{pgfscope}%
\pgfpathrectangle{\pgfqpoint{0.100000in}{0.212622in}}{\pgfqpoint{3.696000in}{3.696000in}}%
\pgfusepath{clip}%
\pgfsetbuttcap%
\pgfsetroundjoin%
\definecolor{currentfill}{rgb}{0.121569,0.466667,0.705882}%
\pgfsetfillcolor{currentfill}%
\pgfsetfillopacity{0.432565}%
\pgfsetlinewidth{1.003750pt}%
\definecolor{currentstroke}{rgb}{0.121569,0.466667,0.705882}%
\pgfsetstrokecolor{currentstroke}%
\pgfsetstrokeopacity{0.432565}%
\pgfsetdash{}{0pt}%
\pgfpathmoveto{\pgfqpoint{1.998039in}{2.782645in}}%
\pgfpathcurveto{\pgfqpoint{2.006275in}{2.782645in}}{\pgfqpoint{2.014176in}{2.785917in}}{\pgfqpoint{2.019999in}{2.791741in}}%
\pgfpathcurveto{\pgfqpoint{2.025823in}{2.797565in}}{\pgfqpoint{2.029096in}{2.805465in}}{\pgfqpoint{2.029096in}{2.813701in}}%
\pgfpathcurveto{\pgfqpoint{2.029096in}{2.821938in}}{\pgfqpoint{2.025823in}{2.829838in}}{\pgfqpoint{2.019999in}{2.835662in}}%
\pgfpathcurveto{\pgfqpoint{2.014176in}{2.841486in}}{\pgfqpoint{2.006275in}{2.844758in}}{\pgfqpoint{1.998039in}{2.844758in}}%
\pgfpathcurveto{\pgfqpoint{1.989803in}{2.844758in}}{\pgfqpoint{1.981903in}{2.841486in}}{\pgfqpoint{1.976079in}{2.835662in}}%
\pgfpathcurveto{\pgfqpoint{1.970255in}{2.829838in}}{\pgfqpoint{1.966983in}{2.821938in}}{\pgfqpoint{1.966983in}{2.813701in}}%
\pgfpathcurveto{\pgfqpoint{1.966983in}{2.805465in}}{\pgfqpoint{1.970255in}{2.797565in}}{\pgfqpoint{1.976079in}{2.791741in}}%
\pgfpathcurveto{\pgfqpoint{1.981903in}{2.785917in}}{\pgfqpoint{1.989803in}{2.782645in}}{\pgfqpoint{1.998039in}{2.782645in}}%
\pgfpathclose%
\pgfusepath{stroke,fill}%
\end{pgfscope}%
\begin{pgfscope}%
\pgfpathrectangle{\pgfqpoint{0.100000in}{0.212622in}}{\pgfqpoint{3.696000in}{3.696000in}}%
\pgfusepath{clip}%
\pgfsetbuttcap%
\pgfsetroundjoin%
\definecolor{currentfill}{rgb}{0.121569,0.466667,0.705882}%
\pgfsetfillcolor{currentfill}%
\pgfsetfillopacity{0.432598}%
\pgfsetlinewidth{1.003750pt}%
\definecolor{currentstroke}{rgb}{0.121569,0.466667,0.705882}%
\pgfsetstrokecolor{currentstroke}%
\pgfsetstrokeopacity{0.432598}%
\pgfsetdash{}{0pt}%
\pgfpathmoveto{\pgfqpoint{1.469984in}{2.634650in}}%
\pgfpathcurveto{\pgfqpoint{1.478220in}{2.634650in}}{\pgfqpoint{1.486120in}{2.637922in}}{\pgfqpoint{1.491944in}{2.643746in}}%
\pgfpathcurveto{\pgfqpoint{1.497768in}{2.649570in}}{\pgfqpoint{1.501040in}{2.657470in}}{\pgfqpoint{1.501040in}{2.665706in}}%
\pgfpathcurveto{\pgfqpoint{1.501040in}{2.673942in}}{\pgfqpoint{1.497768in}{2.681842in}}{\pgfqpoint{1.491944in}{2.687666in}}%
\pgfpathcurveto{\pgfqpoint{1.486120in}{2.693490in}}{\pgfqpoint{1.478220in}{2.696763in}}{\pgfqpoint{1.469984in}{2.696763in}}%
\pgfpathcurveto{\pgfqpoint{1.461748in}{2.696763in}}{\pgfqpoint{1.453847in}{2.693490in}}{\pgfqpoint{1.448024in}{2.687666in}}%
\pgfpathcurveto{\pgfqpoint{1.442200in}{2.681842in}}{\pgfqpoint{1.438927in}{2.673942in}}{\pgfqpoint{1.438927in}{2.665706in}}%
\pgfpathcurveto{\pgfqpoint{1.438927in}{2.657470in}}{\pgfqpoint{1.442200in}{2.649570in}}{\pgfqpoint{1.448024in}{2.643746in}}%
\pgfpathcurveto{\pgfqpoint{1.453847in}{2.637922in}}{\pgfqpoint{1.461748in}{2.634650in}}{\pgfqpoint{1.469984in}{2.634650in}}%
\pgfpathclose%
\pgfusepath{stroke,fill}%
\end{pgfscope}%
\begin{pgfscope}%
\pgfpathrectangle{\pgfqpoint{0.100000in}{0.212622in}}{\pgfqpoint{3.696000in}{3.696000in}}%
\pgfusepath{clip}%
\pgfsetbuttcap%
\pgfsetroundjoin%
\definecolor{currentfill}{rgb}{0.121569,0.466667,0.705882}%
\pgfsetfillcolor{currentfill}%
\pgfsetfillopacity{0.434456}%
\pgfsetlinewidth{1.003750pt}%
\definecolor{currentstroke}{rgb}{0.121569,0.466667,0.705882}%
\pgfsetstrokecolor{currentstroke}%
\pgfsetstrokeopacity{0.434456}%
\pgfsetdash{}{0pt}%
\pgfpathmoveto{\pgfqpoint{1.463695in}{2.622727in}}%
\pgfpathcurveto{\pgfqpoint{1.471931in}{2.622727in}}{\pgfqpoint{1.479831in}{2.625999in}}{\pgfqpoint{1.485655in}{2.631823in}}%
\pgfpathcurveto{\pgfqpoint{1.491479in}{2.637647in}}{\pgfqpoint{1.494752in}{2.645547in}}{\pgfqpoint{1.494752in}{2.653784in}}%
\pgfpathcurveto{\pgfqpoint{1.494752in}{2.662020in}}{\pgfqpoint{1.491479in}{2.669920in}}{\pgfqpoint{1.485655in}{2.675744in}}%
\pgfpathcurveto{\pgfqpoint{1.479831in}{2.681568in}}{\pgfqpoint{1.471931in}{2.684840in}}{\pgfqpoint{1.463695in}{2.684840in}}%
\pgfpathcurveto{\pgfqpoint{1.455459in}{2.684840in}}{\pgfqpoint{1.447559in}{2.681568in}}{\pgfqpoint{1.441735in}{2.675744in}}%
\pgfpathcurveto{\pgfqpoint{1.435911in}{2.669920in}}{\pgfqpoint{1.432639in}{2.662020in}}{\pgfqpoint{1.432639in}{2.653784in}}%
\pgfpathcurveto{\pgfqpoint{1.432639in}{2.645547in}}{\pgfqpoint{1.435911in}{2.637647in}}{\pgfqpoint{1.441735in}{2.631823in}}%
\pgfpathcurveto{\pgfqpoint{1.447559in}{2.625999in}}{\pgfqpoint{1.455459in}{2.622727in}}{\pgfqpoint{1.463695in}{2.622727in}}%
\pgfpathclose%
\pgfusepath{stroke,fill}%
\end{pgfscope}%
\begin{pgfscope}%
\pgfpathrectangle{\pgfqpoint{0.100000in}{0.212622in}}{\pgfqpoint{3.696000in}{3.696000in}}%
\pgfusepath{clip}%
\pgfsetbuttcap%
\pgfsetroundjoin%
\definecolor{currentfill}{rgb}{0.121569,0.466667,0.705882}%
\pgfsetfillcolor{currentfill}%
\pgfsetfillopacity{0.435929}%
\pgfsetlinewidth{1.003750pt}%
\definecolor{currentstroke}{rgb}{0.121569,0.466667,0.705882}%
\pgfsetstrokecolor{currentstroke}%
\pgfsetstrokeopacity{0.435929}%
\pgfsetdash{}{0pt}%
\pgfpathmoveto{\pgfqpoint{1.457365in}{2.611635in}}%
\pgfpathcurveto{\pgfqpoint{1.465602in}{2.611635in}}{\pgfqpoint{1.473502in}{2.614908in}}{\pgfqpoint{1.479326in}{2.620732in}}%
\pgfpathcurveto{\pgfqpoint{1.485150in}{2.626556in}}{\pgfqpoint{1.488422in}{2.634456in}}{\pgfqpoint{1.488422in}{2.642692in}}%
\pgfpathcurveto{\pgfqpoint{1.488422in}{2.650928in}}{\pgfqpoint{1.485150in}{2.658828in}}{\pgfqpoint{1.479326in}{2.664652in}}%
\pgfpathcurveto{\pgfqpoint{1.473502in}{2.670476in}}{\pgfqpoint{1.465602in}{2.673748in}}{\pgfqpoint{1.457365in}{2.673748in}}%
\pgfpathcurveto{\pgfqpoint{1.449129in}{2.673748in}}{\pgfqpoint{1.441229in}{2.670476in}}{\pgfqpoint{1.435405in}{2.664652in}}%
\pgfpathcurveto{\pgfqpoint{1.429581in}{2.658828in}}{\pgfqpoint{1.426309in}{2.650928in}}{\pgfqpoint{1.426309in}{2.642692in}}%
\pgfpathcurveto{\pgfqpoint{1.426309in}{2.634456in}}{\pgfqpoint{1.429581in}{2.626556in}}{\pgfqpoint{1.435405in}{2.620732in}}%
\pgfpathcurveto{\pgfqpoint{1.441229in}{2.614908in}}{\pgfqpoint{1.449129in}{2.611635in}}{\pgfqpoint{1.457365in}{2.611635in}}%
\pgfpathclose%
\pgfusepath{stroke,fill}%
\end{pgfscope}%
\begin{pgfscope}%
\pgfpathrectangle{\pgfqpoint{0.100000in}{0.212622in}}{\pgfqpoint{3.696000in}{3.696000in}}%
\pgfusepath{clip}%
\pgfsetbuttcap%
\pgfsetroundjoin%
\definecolor{currentfill}{rgb}{0.121569,0.466667,0.705882}%
\pgfsetfillcolor{currentfill}%
\pgfsetfillopacity{0.437302}%
\pgfsetlinewidth{1.003750pt}%
\definecolor{currentstroke}{rgb}{0.121569,0.466667,0.705882}%
\pgfsetstrokecolor{currentstroke}%
\pgfsetstrokeopacity{0.437302}%
\pgfsetdash{}{0pt}%
\pgfpathmoveto{\pgfqpoint{2.000243in}{2.764706in}}%
\pgfpathcurveto{\pgfqpoint{2.008479in}{2.764706in}}{\pgfqpoint{2.016379in}{2.767978in}}{\pgfqpoint{2.022203in}{2.773802in}}%
\pgfpathcurveto{\pgfqpoint{2.028027in}{2.779626in}}{\pgfqpoint{2.031299in}{2.787526in}}{\pgfqpoint{2.031299in}{2.795763in}}%
\pgfpathcurveto{\pgfqpoint{2.031299in}{2.803999in}}{\pgfqpoint{2.028027in}{2.811899in}}{\pgfqpoint{2.022203in}{2.817723in}}%
\pgfpathcurveto{\pgfqpoint{2.016379in}{2.823547in}}{\pgfqpoint{2.008479in}{2.826819in}}{\pgfqpoint{2.000243in}{2.826819in}}%
\pgfpathcurveto{\pgfqpoint{1.992006in}{2.826819in}}{\pgfqpoint{1.984106in}{2.823547in}}{\pgfqpoint{1.978282in}{2.817723in}}%
\pgfpathcurveto{\pgfqpoint{1.972459in}{2.811899in}}{\pgfqpoint{1.969186in}{2.803999in}}{\pgfqpoint{1.969186in}{2.795763in}}%
\pgfpathcurveto{\pgfqpoint{1.969186in}{2.787526in}}{\pgfqpoint{1.972459in}{2.779626in}}{\pgfqpoint{1.978282in}{2.773802in}}%
\pgfpathcurveto{\pgfqpoint{1.984106in}{2.767978in}}{\pgfqpoint{1.992006in}{2.764706in}}{\pgfqpoint{2.000243in}{2.764706in}}%
\pgfpathclose%
\pgfusepath{stroke,fill}%
\end{pgfscope}%
\begin{pgfscope}%
\pgfpathrectangle{\pgfqpoint{0.100000in}{0.212622in}}{\pgfqpoint{3.696000in}{3.696000in}}%
\pgfusepath{clip}%
\pgfsetbuttcap%
\pgfsetroundjoin%
\definecolor{currentfill}{rgb}{0.121569,0.466667,0.705882}%
\pgfsetfillcolor{currentfill}%
\pgfsetfillopacity{0.437549}%
\pgfsetlinewidth{1.003750pt}%
\definecolor{currentstroke}{rgb}{0.121569,0.466667,0.705882}%
\pgfsetstrokecolor{currentstroke}%
\pgfsetstrokeopacity{0.437549}%
\pgfsetdash{}{0pt}%
\pgfpathmoveto{\pgfqpoint{1.452638in}{2.601524in}}%
\pgfpathcurveto{\pgfqpoint{1.460874in}{2.601524in}}{\pgfqpoint{1.468774in}{2.604796in}}{\pgfqpoint{1.474598in}{2.610620in}}%
\pgfpathcurveto{\pgfqpoint{1.480422in}{2.616444in}}{\pgfqpoint{1.483694in}{2.624344in}}{\pgfqpoint{1.483694in}{2.632580in}}%
\pgfpathcurveto{\pgfqpoint{1.483694in}{2.640817in}}{\pgfqpoint{1.480422in}{2.648717in}}{\pgfqpoint{1.474598in}{2.654541in}}%
\pgfpathcurveto{\pgfqpoint{1.468774in}{2.660365in}}{\pgfqpoint{1.460874in}{2.663637in}}{\pgfqpoint{1.452638in}{2.663637in}}%
\pgfpathcurveto{\pgfqpoint{1.444402in}{2.663637in}}{\pgfqpoint{1.436502in}{2.660365in}}{\pgfqpoint{1.430678in}{2.654541in}}%
\pgfpathcurveto{\pgfqpoint{1.424854in}{2.648717in}}{\pgfqpoint{1.421581in}{2.640817in}}{\pgfqpoint{1.421581in}{2.632580in}}%
\pgfpathcurveto{\pgfqpoint{1.421581in}{2.624344in}}{\pgfqpoint{1.424854in}{2.616444in}}{\pgfqpoint{1.430678in}{2.610620in}}%
\pgfpathcurveto{\pgfqpoint{1.436502in}{2.604796in}}{\pgfqpoint{1.444402in}{2.601524in}}{\pgfqpoint{1.452638in}{2.601524in}}%
\pgfpathclose%
\pgfusepath{stroke,fill}%
\end{pgfscope}%
\begin{pgfscope}%
\pgfpathrectangle{\pgfqpoint{0.100000in}{0.212622in}}{\pgfqpoint{3.696000in}{3.696000in}}%
\pgfusepath{clip}%
\pgfsetbuttcap%
\pgfsetroundjoin%
\definecolor{currentfill}{rgb}{0.121569,0.466667,0.705882}%
\pgfsetfillcolor{currentfill}%
\pgfsetfillopacity{0.438724}%
\pgfsetlinewidth{1.003750pt}%
\definecolor{currentstroke}{rgb}{0.121569,0.466667,0.705882}%
\pgfsetstrokecolor{currentstroke}%
\pgfsetstrokeopacity{0.438724}%
\pgfsetdash{}{0pt}%
\pgfpathmoveto{\pgfqpoint{1.448111in}{2.594001in}}%
\pgfpathcurveto{\pgfqpoint{1.456347in}{2.594001in}}{\pgfqpoint{1.464247in}{2.597273in}}{\pgfqpoint{1.470071in}{2.603097in}}%
\pgfpathcurveto{\pgfqpoint{1.475895in}{2.608921in}}{\pgfqpoint{1.479167in}{2.616821in}}{\pgfqpoint{1.479167in}{2.625058in}}%
\pgfpathcurveto{\pgfqpoint{1.479167in}{2.633294in}}{\pgfqpoint{1.475895in}{2.641194in}}{\pgfqpoint{1.470071in}{2.647018in}}%
\pgfpathcurveto{\pgfqpoint{1.464247in}{2.652842in}}{\pgfqpoint{1.456347in}{2.656114in}}{\pgfqpoint{1.448111in}{2.656114in}}%
\pgfpathcurveto{\pgfqpoint{1.439874in}{2.656114in}}{\pgfqpoint{1.431974in}{2.652842in}}{\pgfqpoint{1.426150in}{2.647018in}}%
\pgfpathcurveto{\pgfqpoint{1.420326in}{2.641194in}}{\pgfqpoint{1.417054in}{2.633294in}}{\pgfqpoint{1.417054in}{2.625058in}}%
\pgfpathcurveto{\pgfqpoint{1.417054in}{2.616821in}}{\pgfqpoint{1.420326in}{2.608921in}}{\pgfqpoint{1.426150in}{2.603097in}}%
\pgfpathcurveto{\pgfqpoint{1.431974in}{2.597273in}}{\pgfqpoint{1.439874in}{2.594001in}}{\pgfqpoint{1.448111in}{2.594001in}}%
\pgfpathclose%
\pgfusepath{stroke,fill}%
\end{pgfscope}%
\begin{pgfscope}%
\pgfpathrectangle{\pgfqpoint{0.100000in}{0.212622in}}{\pgfqpoint{3.696000in}{3.696000in}}%
\pgfusepath{clip}%
\pgfsetbuttcap%
\pgfsetroundjoin%
\definecolor{currentfill}{rgb}{0.121569,0.466667,0.705882}%
\pgfsetfillcolor{currentfill}%
\pgfsetfillopacity{0.439891}%
\pgfsetlinewidth{1.003750pt}%
\definecolor{currentstroke}{rgb}{0.121569,0.466667,0.705882}%
\pgfsetstrokecolor{currentstroke}%
\pgfsetstrokeopacity{0.439891}%
\pgfsetdash{}{0pt}%
\pgfpathmoveto{\pgfqpoint{1.444169in}{2.586509in}}%
\pgfpathcurveto{\pgfqpoint{1.452405in}{2.586509in}}{\pgfqpoint{1.460305in}{2.589781in}}{\pgfqpoint{1.466129in}{2.595605in}}%
\pgfpathcurveto{\pgfqpoint{1.471953in}{2.601429in}}{\pgfqpoint{1.475225in}{2.609329in}}{\pgfqpoint{1.475225in}{2.617566in}}%
\pgfpathcurveto{\pgfqpoint{1.475225in}{2.625802in}}{\pgfqpoint{1.471953in}{2.633702in}}{\pgfqpoint{1.466129in}{2.639526in}}%
\pgfpathcurveto{\pgfqpoint{1.460305in}{2.645350in}}{\pgfqpoint{1.452405in}{2.648622in}}{\pgfqpoint{1.444169in}{2.648622in}}%
\pgfpathcurveto{\pgfqpoint{1.435933in}{2.648622in}}{\pgfqpoint{1.428033in}{2.645350in}}{\pgfqpoint{1.422209in}{2.639526in}}%
\pgfpathcurveto{\pgfqpoint{1.416385in}{2.633702in}}{\pgfqpoint{1.413112in}{2.625802in}}{\pgfqpoint{1.413112in}{2.617566in}}%
\pgfpathcurveto{\pgfqpoint{1.413112in}{2.609329in}}{\pgfqpoint{1.416385in}{2.601429in}}{\pgfqpoint{1.422209in}{2.595605in}}%
\pgfpathcurveto{\pgfqpoint{1.428033in}{2.589781in}}{\pgfqpoint{1.435933in}{2.586509in}}{\pgfqpoint{1.444169in}{2.586509in}}%
\pgfpathclose%
\pgfusepath{stroke,fill}%
\end{pgfscope}%
\begin{pgfscope}%
\pgfpathrectangle{\pgfqpoint{0.100000in}{0.212622in}}{\pgfqpoint{3.696000in}{3.696000in}}%
\pgfusepath{clip}%
\pgfsetbuttcap%
\pgfsetroundjoin%
\definecolor{currentfill}{rgb}{0.121569,0.466667,0.705882}%
\pgfsetfillcolor{currentfill}%
\pgfsetfillopacity{0.440976}%
\pgfsetlinewidth{1.003750pt}%
\definecolor{currentstroke}{rgb}{0.121569,0.466667,0.705882}%
\pgfsetstrokecolor{currentstroke}%
\pgfsetstrokeopacity{0.440976}%
\pgfsetdash{}{0pt}%
\pgfpathmoveto{\pgfqpoint{1.440415in}{2.579302in}}%
\pgfpathcurveto{\pgfqpoint{1.448651in}{2.579302in}}{\pgfqpoint{1.456551in}{2.582575in}}{\pgfqpoint{1.462375in}{2.588399in}}%
\pgfpathcurveto{\pgfqpoint{1.468199in}{2.594223in}}{\pgfqpoint{1.471471in}{2.602123in}}{\pgfqpoint{1.471471in}{2.610359in}}%
\pgfpathcurveto{\pgfqpoint{1.471471in}{2.618595in}}{\pgfqpoint{1.468199in}{2.626495in}}{\pgfqpoint{1.462375in}{2.632319in}}%
\pgfpathcurveto{\pgfqpoint{1.456551in}{2.638143in}}{\pgfqpoint{1.448651in}{2.641415in}}{\pgfqpoint{1.440415in}{2.641415in}}%
\pgfpathcurveto{\pgfqpoint{1.432179in}{2.641415in}}{\pgfqpoint{1.424279in}{2.638143in}}{\pgfqpoint{1.418455in}{2.632319in}}%
\pgfpathcurveto{\pgfqpoint{1.412631in}{2.626495in}}{\pgfqpoint{1.409358in}{2.618595in}}{\pgfqpoint{1.409358in}{2.610359in}}%
\pgfpathcurveto{\pgfqpoint{1.409358in}{2.602123in}}{\pgfqpoint{1.412631in}{2.594223in}}{\pgfqpoint{1.418455in}{2.588399in}}%
\pgfpathcurveto{\pgfqpoint{1.424279in}{2.582575in}}{\pgfqpoint{1.432179in}{2.579302in}}{\pgfqpoint{1.440415in}{2.579302in}}%
\pgfpathclose%
\pgfusepath{stroke,fill}%
\end{pgfscope}%
\begin{pgfscope}%
\pgfpathrectangle{\pgfqpoint{0.100000in}{0.212622in}}{\pgfqpoint{3.696000in}{3.696000in}}%
\pgfusepath{clip}%
\pgfsetbuttcap%
\pgfsetroundjoin%
\definecolor{currentfill}{rgb}{0.121569,0.466667,0.705882}%
\pgfsetfillcolor{currentfill}%
\pgfsetfillopacity{0.441822}%
\pgfsetlinewidth{1.003750pt}%
\definecolor{currentstroke}{rgb}{0.121569,0.466667,0.705882}%
\pgfsetstrokecolor{currentstroke}%
\pgfsetstrokeopacity{0.441822}%
\pgfsetdash{}{0pt}%
\pgfpathmoveto{\pgfqpoint{1.436769in}{2.572712in}}%
\pgfpathcurveto{\pgfqpoint{1.445006in}{2.572712in}}{\pgfqpoint{1.452906in}{2.575984in}}{\pgfqpoint{1.458730in}{2.581808in}}%
\pgfpathcurveto{\pgfqpoint{1.464554in}{2.587632in}}{\pgfqpoint{1.467826in}{2.595532in}}{\pgfqpoint{1.467826in}{2.603768in}}%
\pgfpathcurveto{\pgfqpoint{1.467826in}{2.612005in}}{\pgfqpoint{1.464554in}{2.619905in}}{\pgfqpoint{1.458730in}{2.625729in}}%
\pgfpathcurveto{\pgfqpoint{1.452906in}{2.631553in}}{\pgfqpoint{1.445006in}{2.634825in}}{\pgfqpoint{1.436769in}{2.634825in}}%
\pgfpathcurveto{\pgfqpoint{1.428533in}{2.634825in}}{\pgfqpoint{1.420633in}{2.631553in}}{\pgfqpoint{1.414809in}{2.625729in}}%
\pgfpathcurveto{\pgfqpoint{1.408985in}{2.619905in}}{\pgfqpoint{1.405713in}{2.612005in}}{\pgfqpoint{1.405713in}{2.603768in}}%
\pgfpathcurveto{\pgfqpoint{1.405713in}{2.595532in}}{\pgfqpoint{1.408985in}{2.587632in}}{\pgfqpoint{1.414809in}{2.581808in}}%
\pgfpathcurveto{\pgfqpoint{1.420633in}{2.575984in}}{\pgfqpoint{1.428533in}{2.572712in}}{\pgfqpoint{1.436769in}{2.572712in}}%
\pgfpathclose%
\pgfusepath{stroke,fill}%
\end{pgfscope}%
\begin{pgfscope}%
\pgfpathrectangle{\pgfqpoint{0.100000in}{0.212622in}}{\pgfqpoint{3.696000in}{3.696000in}}%
\pgfusepath{clip}%
\pgfsetbuttcap%
\pgfsetroundjoin%
\definecolor{currentfill}{rgb}{0.121569,0.466667,0.705882}%
\pgfsetfillcolor{currentfill}%
\pgfsetfillopacity{0.441871}%
\pgfsetlinewidth{1.003750pt}%
\definecolor{currentstroke}{rgb}{0.121569,0.466667,0.705882}%
\pgfsetstrokecolor{currentstroke}%
\pgfsetstrokeopacity{0.441871}%
\pgfsetdash{}{0pt}%
\pgfpathmoveto{\pgfqpoint{2.004057in}{2.745205in}}%
\pgfpathcurveto{\pgfqpoint{2.012293in}{2.745205in}}{\pgfqpoint{2.020193in}{2.748477in}}{\pgfqpoint{2.026017in}{2.754301in}}%
\pgfpathcurveto{\pgfqpoint{2.031841in}{2.760125in}}{\pgfqpoint{2.035114in}{2.768025in}}{\pgfqpoint{2.035114in}{2.776262in}}%
\pgfpathcurveto{\pgfqpoint{2.035114in}{2.784498in}}{\pgfqpoint{2.031841in}{2.792398in}}{\pgfqpoint{2.026017in}{2.798222in}}%
\pgfpathcurveto{\pgfqpoint{2.020193in}{2.804046in}}{\pgfqpoint{2.012293in}{2.807318in}}{\pgfqpoint{2.004057in}{2.807318in}}%
\pgfpathcurveto{\pgfqpoint{1.995821in}{2.807318in}}{\pgfqpoint{1.987921in}{2.804046in}}{\pgfqpoint{1.982097in}{2.798222in}}%
\pgfpathcurveto{\pgfqpoint{1.976273in}{2.792398in}}{\pgfqpoint{1.973001in}{2.784498in}}{\pgfqpoint{1.973001in}{2.776262in}}%
\pgfpathcurveto{\pgfqpoint{1.973001in}{2.768025in}}{\pgfqpoint{1.976273in}{2.760125in}}{\pgfqpoint{1.982097in}{2.754301in}}%
\pgfpathcurveto{\pgfqpoint{1.987921in}{2.748477in}}{\pgfqpoint{1.995821in}{2.745205in}}{\pgfqpoint{2.004057in}{2.745205in}}%
\pgfpathclose%
\pgfusepath{stroke,fill}%
\end{pgfscope}%
\begin{pgfscope}%
\pgfpathrectangle{\pgfqpoint{0.100000in}{0.212622in}}{\pgfqpoint{3.696000in}{3.696000in}}%
\pgfusepath{clip}%
\pgfsetbuttcap%
\pgfsetroundjoin%
\definecolor{currentfill}{rgb}{0.121569,0.466667,0.705882}%
\pgfsetfillcolor{currentfill}%
\pgfsetfillopacity{0.442480}%
\pgfsetlinewidth{1.003750pt}%
\definecolor{currentstroke}{rgb}{0.121569,0.466667,0.705882}%
\pgfsetstrokecolor{currentstroke}%
\pgfsetstrokeopacity{0.442480}%
\pgfsetdash{}{0pt}%
\pgfpathmoveto{\pgfqpoint{1.434504in}{2.567454in}}%
\pgfpathcurveto{\pgfqpoint{1.442740in}{2.567454in}}{\pgfqpoint{1.450640in}{2.570726in}}{\pgfqpoint{1.456464in}{2.576550in}}%
\pgfpathcurveto{\pgfqpoint{1.462288in}{2.582374in}}{\pgfqpoint{1.465560in}{2.590274in}}{\pgfqpoint{1.465560in}{2.598511in}}%
\pgfpathcurveto{\pgfqpoint{1.465560in}{2.606747in}}{\pgfqpoint{1.462288in}{2.614647in}}{\pgfqpoint{1.456464in}{2.620471in}}%
\pgfpathcurveto{\pgfqpoint{1.450640in}{2.626295in}}{\pgfqpoint{1.442740in}{2.629567in}}{\pgfqpoint{1.434504in}{2.629567in}}%
\pgfpathcurveto{\pgfqpoint{1.426268in}{2.629567in}}{\pgfqpoint{1.418368in}{2.626295in}}{\pgfqpoint{1.412544in}{2.620471in}}%
\pgfpathcurveto{\pgfqpoint{1.406720in}{2.614647in}}{\pgfqpoint{1.403447in}{2.606747in}}{\pgfqpoint{1.403447in}{2.598511in}}%
\pgfpathcurveto{\pgfqpoint{1.403447in}{2.590274in}}{\pgfqpoint{1.406720in}{2.582374in}}{\pgfqpoint{1.412544in}{2.576550in}}%
\pgfpathcurveto{\pgfqpoint{1.418368in}{2.570726in}}{\pgfqpoint{1.426268in}{2.567454in}}{\pgfqpoint{1.434504in}{2.567454in}}%
\pgfpathclose%
\pgfusepath{stroke,fill}%
\end{pgfscope}%
\begin{pgfscope}%
\pgfpathrectangle{\pgfqpoint{0.100000in}{0.212622in}}{\pgfqpoint{3.696000in}{3.696000in}}%
\pgfusepath{clip}%
\pgfsetbuttcap%
\pgfsetroundjoin%
\definecolor{currentfill}{rgb}{0.121569,0.466667,0.705882}%
\pgfsetfillcolor{currentfill}%
\pgfsetfillopacity{0.443032}%
\pgfsetlinewidth{1.003750pt}%
\definecolor{currentstroke}{rgb}{0.121569,0.466667,0.705882}%
\pgfsetstrokecolor{currentstroke}%
\pgfsetstrokeopacity{0.443032}%
\pgfsetdash{}{0pt}%
\pgfpathmoveto{\pgfqpoint{1.432392in}{2.563727in}}%
\pgfpathcurveto{\pgfqpoint{1.440628in}{2.563727in}}{\pgfqpoint{1.448528in}{2.566999in}}{\pgfqpoint{1.454352in}{2.572823in}}%
\pgfpathcurveto{\pgfqpoint{1.460176in}{2.578647in}}{\pgfqpoint{1.463448in}{2.586547in}}{\pgfqpoint{1.463448in}{2.594784in}}%
\pgfpathcurveto{\pgfqpoint{1.463448in}{2.603020in}}{\pgfqpoint{1.460176in}{2.610920in}}{\pgfqpoint{1.454352in}{2.616744in}}%
\pgfpathcurveto{\pgfqpoint{1.448528in}{2.622568in}}{\pgfqpoint{1.440628in}{2.625840in}}{\pgfqpoint{1.432392in}{2.625840in}}%
\pgfpathcurveto{\pgfqpoint{1.424156in}{2.625840in}}{\pgfqpoint{1.416256in}{2.622568in}}{\pgfqpoint{1.410432in}{2.616744in}}%
\pgfpathcurveto{\pgfqpoint{1.404608in}{2.610920in}}{\pgfqpoint{1.401335in}{2.603020in}}{\pgfqpoint{1.401335in}{2.594784in}}%
\pgfpathcurveto{\pgfqpoint{1.401335in}{2.586547in}}{\pgfqpoint{1.404608in}{2.578647in}}{\pgfqpoint{1.410432in}{2.572823in}}%
\pgfpathcurveto{\pgfqpoint{1.416256in}{2.566999in}}{\pgfqpoint{1.424156in}{2.563727in}}{\pgfqpoint{1.432392in}{2.563727in}}%
\pgfpathclose%
\pgfusepath{stroke,fill}%
\end{pgfscope}%
\begin{pgfscope}%
\pgfpathrectangle{\pgfqpoint{0.100000in}{0.212622in}}{\pgfqpoint{3.696000in}{3.696000in}}%
\pgfusepath{clip}%
\pgfsetbuttcap%
\pgfsetroundjoin%
\definecolor{currentfill}{rgb}{0.121569,0.466667,0.705882}%
\pgfsetfillcolor{currentfill}%
\pgfsetfillopacity{0.444004}%
\pgfsetlinewidth{1.003750pt}%
\definecolor{currentstroke}{rgb}{0.121569,0.466667,0.705882}%
\pgfsetstrokecolor{currentstroke}%
\pgfsetstrokeopacity{0.444004}%
\pgfsetdash{}{0pt}%
\pgfpathmoveto{\pgfqpoint{1.428822in}{2.556429in}}%
\pgfpathcurveto{\pgfqpoint{1.437059in}{2.556429in}}{\pgfqpoint{1.444959in}{2.559701in}}{\pgfqpoint{1.450783in}{2.565525in}}%
\pgfpathcurveto{\pgfqpoint{1.456606in}{2.571349in}}{\pgfqpoint{1.459879in}{2.579249in}}{\pgfqpoint{1.459879in}{2.587486in}}%
\pgfpathcurveto{\pgfqpoint{1.459879in}{2.595722in}}{\pgfqpoint{1.456606in}{2.603622in}}{\pgfqpoint{1.450783in}{2.609446in}}%
\pgfpathcurveto{\pgfqpoint{1.444959in}{2.615270in}}{\pgfqpoint{1.437059in}{2.618542in}}{\pgfqpoint{1.428822in}{2.618542in}}%
\pgfpathcurveto{\pgfqpoint{1.420586in}{2.618542in}}{\pgfqpoint{1.412686in}{2.615270in}}{\pgfqpoint{1.406862in}{2.609446in}}%
\pgfpathcurveto{\pgfqpoint{1.401038in}{2.603622in}}{\pgfqpoint{1.397766in}{2.595722in}}{\pgfqpoint{1.397766in}{2.587486in}}%
\pgfpathcurveto{\pgfqpoint{1.397766in}{2.579249in}}{\pgfqpoint{1.401038in}{2.571349in}}{\pgfqpoint{1.406862in}{2.565525in}}%
\pgfpathcurveto{\pgfqpoint{1.412686in}{2.559701in}}{\pgfqpoint{1.420586in}{2.556429in}}{\pgfqpoint{1.428822in}{2.556429in}}%
\pgfpathclose%
\pgfusepath{stroke,fill}%
\end{pgfscope}%
\begin{pgfscope}%
\pgfpathrectangle{\pgfqpoint{0.100000in}{0.212622in}}{\pgfqpoint{3.696000in}{3.696000in}}%
\pgfusepath{clip}%
\pgfsetbuttcap%
\pgfsetroundjoin%
\definecolor{currentfill}{rgb}{0.121569,0.466667,0.705882}%
\pgfsetfillcolor{currentfill}%
\pgfsetfillopacity{0.444450}%
\pgfsetlinewidth{1.003750pt}%
\definecolor{currentstroke}{rgb}{0.121569,0.466667,0.705882}%
\pgfsetstrokecolor{currentstroke}%
\pgfsetstrokeopacity{0.444450}%
\pgfsetdash{}{0pt}%
\pgfpathmoveto{\pgfqpoint{2.005979in}{2.734551in}}%
\pgfpathcurveto{\pgfqpoint{2.014215in}{2.734551in}}{\pgfqpoint{2.022115in}{2.737824in}}{\pgfqpoint{2.027939in}{2.743648in}}%
\pgfpathcurveto{\pgfqpoint{2.033763in}{2.749472in}}{\pgfqpoint{2.037035in}{2.757372in}}{\pgfqpoint{2.037035in}{2.765608in}}%
\pgfpathcurveto{\pgfqpoint{2.037035in}{2.773844in}}{\pgfqpoint{2.033763in}{2.781744in}}{\pgfqpoint{2.027939in}{2.787568in}}%
\pgfpathcurveto{\pgfqpoint{2.022115in}{2.793392in}}{\pgfqpoint{2.014215in}{2.796664in}}{\pgfqpoint{2.005979in}{2.796664in}}%
\pgfpathcurveto{\pgfqpoint{1.997742in}{2.796664in}}{\pgfqpoint{1.989842in}{2.793392in}}{\pgfqpoint{1.984018in}{2.787568in}}%
\pgfpathcurveto{\pgfqpoint{1.978195in}{2.781744in}}{\pgfqpoint{1.974922in}{2.773844in}}{\pgfqpoint{1.974922in}{2.765608in}}%
\pgfpathcurveto{\pgfqpoint{1.974922in}{2.757372in}}{\pgfqpoint{1.978195in}{2.749472in}}{\pgfqpoint{1.984018in}{2.743648in}}%
\pgfpathcurveto{\pgfqpoint{1.989842in}{2.737824in}}{\pgfqpoint{1.997742in}{2.734551in}}{\pgfqpoint{2.005979in}{2.734551in}}%
\pgfpathclose%
\pgfusepath{stroke,fill}%
\end{pgfscope}%
\begin{pgfscope}%
\pgfpathrectangle{\pgfqpoint{0.100000in}{0.212622in}}{\pgfqpoint{3.696000in}{3.696000in}}%
\pgfusepath{clip}%
\pgfsetbuttcap%
\pgfsetroundjoin%
\definecolor{currentfill}{rgb}{0.121569,0.466667,0.705882}%
\pgfsetfillcolor{currentfill}%
\pgfsetfillopacity{0.444954}%
\pgfsetlinewidth{1.003750pt}%
\definecolor{currentstroke}{rgb}{0.121569,0.466667,0.705882}%
\pgfsetstrokecolor{currentstroke}%
\pgfsetstrokeopacity{0.444954}%
\pgfsetdash{}{0pt}%
\pgfpathmoveto{\pgfqpoint{1.425601in}{2.549988in}}%
\pgfpathcurveto{\pgfqpoint{1.433837in}{2.549988in}}{\pgfqpoint{1.441737in}{2.553260in}}{\pgfqpoint{1.447561in}{2.559084in}}%
\pgfpathcurveto{\pgfqpoint{1.453385in}{2.564908in}}{\pgfqpoint{1.456657in}{2.572808in}}{\pgfqpoint{1.456657in}{2.581044in}}%
\pgfpathcurveto{\pgfqpoint{1.456657in}{2.589280in}}{\pgfqpoint{1.453385in}{2.597180in}}{\pgfqpoint{1.447561in}{2.603004in}}%
\pgfpathcurveto{\pgfqpoint{1.441737in}{2.608828in}}{\pgfqpoint{1.433837in}{2.612101in}}{\pgfqpoint{1.425601in}{2.612101in}}%
\pgfpathcurveto{\pgfqpoint{1.417364in}{2.612101in}}{\pgfqpoint{1.409464in}{2.608828in}}{\pgfqpoint{1.403640in}{2.603004in}}%
\pgfpathcurveto{\pgfqpoint{1.397816in}{2.597180in}}{\pgfqpoint{1.394544in}{2.589280in}}{\pgfqpoint{1.394544in}{2.581044in}}%
\pgfpathcurveto{\pgfqpoint{1.394544in}{2.572808in}}{\pgfqpoint{1.397816in}{2.564908in}}{\pgfqpoint{1.403640in}{2.559084in}}%
\pgfpathcurveto{\pgfqpoint{1.409464in}{2.553260in}}{\pgfqpoint{1.417364in}{2.549988in}}{\pgfqpoint{1.425601in}{2.549988in}}%
\pgfpathclose%
\pgfusepath{stroke,fill}%
\end{pgfscope}%
\begin{pgfscope}%
\pgfpathrectangle{\pgfqpoint{0.100000in}{0.212622in}}{\pgfqpoint{3.696000in}{3.696000in}}%
\pgfusepath{clip}%
\pgfsetbuttcap%
\pgfsetroundjoin%
\definecolor{currentfill}{rgb}{0.121569,0.466667,0.705882}%
\pgfsetfillcolor{currentfill}%
\pgfsetfillopacity{0.445693}%
\pgfsetlinewidth{1.003750pt}%
\definecolor{currentstroke}{rgb}{0.121569,0.466667,0.705882}%
\pgfsetstrokecolor{currentstroke}%
\pgfsetstrokeopacity{0.445693}%
\pgfsetdash{}{0pt}%
\pgfpathmoveto{\pgfqpoint{1.422283in}{2.544187in}}%
\pgfpathcurveto{\pgfqpoint{1.430519in}{2.544187in}}{\pgfqpoint{1.438419in}{2.547459in}}{\pgfqpoint{1.444243in}{2.553283in}}%
\pgfpathcurveto{\pgfqpoint{1.450067in}{2.559107in}}{\pgfqpoint{1.453339in}{2.567007in}}{\pgfqpoint{1.453339in}{2.575244in}}%
\pgfpathcurveto{\pgfqpoint{1.453339in}{2.583480in}}{\pgfqpoint{1.450067in}{2.591380in}}{\pgfqpoint{1.444243in}{2.597204in}}%
\pgfpathcurveto{\pgfqpoint{1.438419in}{2.603028in}}{\pgfqpoint{1.430519in}{2.606300in}}{\pgfqpoint{1.422283in}{2.606300in}}%
\pgfpathcurveto{\pgfqpoint{1.414046in}{2.606300in}}{\pgfqpoint{1.406146in}{2.603028in}}{\pgfqpoint{1.400323in}{2.597204in}}%
\pgfpathcurveto{\pgfqpoint{1.394499in}{2.591380in}}{\pgfqpoint{1.391226in}{2.583480in}}{\pgfqpoint{1.391226in}{2.575244in}}%
\pgfpathcurveto{\pgfqpoint{1.391226in}{2.567007in}}{\pgfqpoint{1.394499in}{2.559107in}}{\pgfqpoint{1.400323in}{2.553283in}}%
\pgfpathcurveto{\pgfqpoint{1.406146in}{2.547459in}}{\pgfqpoint{1.414046in}{2.544187in}}{\pgfqpoint{1.422283in}{2.544187in}}%
\pgfpathclose%
\pgfusepath{stroke,fill}%
\end{pgfscope}%
\begin{pgfscope}%
\pgfpathrectangle{\pgfqpoint{0.100000in}{0.212622in}}{\pgfqpoint{3.696000in}{3.696000in}}%
\pgfusepath{clip}%
\pgfsetbuttcap%
\pgfsetroundjoin%
\definecolor{currentfill}{rgb}{0.121569,0.466667,0.705882}%
\pgfsetfillcolor{currentfill}%
\pgfsetfillopacity{0.446154}%
\pgfsetlinewidth{1.003750pt}%
\definecolor{currentstroke}{rgb}{0.121569,0.466667,0.705882}%
\pgfsetstrokecolor{currentstroke}%
\pgfsetstrokeopacity{0.446154}%
\pgfsetdash{}{0pt}%
\pgfpathmoveto{\pgfqpoint{1.420570in}{2.540205in}}%
\pgfpathcurveto{\pgfqpoint{1.428806in}{2.540205in}}{\pgfqpoint{1.436706in}{2.543477in}}{\pgfqpoint{1.442530in}{2.549301in}}%
\pgfpathcurveto{\pgfqpoint{1.448354in}{2.555125in}}{\pgfqpoint{1.451626in}{2.563025in}}{\pgfqpoint{1.451626in}{2.571261in}}%
\pgfpathcurveto{\pgfqpoint{1.451626in}{2.579498in}}{\pgfqpoint{1.448354in}{2.587398in}}{\pgfqpoint{1.442530in}{2.593222in}}%
\pgfpathcurveto{\pgfqpoint{1.436706in}{2.599046in}}{\pgfqpoint{1.428806in}{2.602318in}}{\pgfqpoint{1.420570in}{2.602318in}}%
\pgfpathcurveto{\pgfqpoint{1.412334in}{2.602318in}}{\pgfqpoint{1.404434in}{2.599046in}}{\pgfqpoint{1.398610in}{2.593222in}}%
\pgfpathcurveto{\pgfqpoint{1.392786in}{2.587398in}}{\pgfqpoint{1.389513in}{2.579498in}}{\pgfqpoint{1.389513in}{2.571261in}}%
\pgfpathcurveto{\pgfqpoint{1.389513in}{2.563025in}}{\pgfqpoint{1.392786in}{2.555125in}}{\pgfqpoint{1.398610in}{2.549301in}}%
\pgfpathcurveto{\pgfqpoint{1.404434in}{2.543477in}}{\pgfqpoint{1.412334in}{2.540205in}}{\pgfqpoint{1.420570in}{2.540205in}}%
\pgfpathclose%
\pgfusepath{stroke,fill}%
\end{pgfscope}%
\begin{pgfscope}%
\pgfpathrectangle{\pgfqpoint{0.100000in}{0.212622in}}{\pgfqpoint{3.696000in}{3.696000in}}%
\pgfusepath{clip}%
\pgfsetbuttcap%
\pgfsetroundjoin%
\definecolor{currentfill}{rgb}{0.121569,0.466667,0.705882}%
\pgfsetfillcolor{currentfill}%
\pgfsetfillopacity{0.446576}%
\pgfsetlinewidth{1.003750pt}%
\definecolor{currentstroke}{rgb}{0.121569,0.466667,0.705882}%
\pgfsetstrokecolor{currentstroke}%
\pgfsetstrokeopacity{0.446576}%
\pgfsetdash{}{0pt}%
\pgfpathmoveto{\pgfqpoint{1.418940in}{2.537166in}}%
\pgfpathcurveto{\pgfqpoint{1.427176in}{2.537166in}}{\pgfqpoint{1.435076in}{2.540439in}}{\pgfqpoint{1.440900in}{2.546263in}}%
\pgfpathcurveto{\pgfqpoint{1.446724in}{2.552086in}}{\pgfqpoint{1.449996in}{2.559987in}}{\pgfqpoint{1.449996in}{2.568223in}}%
\pgfpathcurveto{\pgfqpoint{1.449996in}{2.576459in}}{\pgfqpoint{1.446724in}{2.584359in}}{\pgfqpoint{1.440900in}{2.590183in}}%
\pgfpathcurveto{\pgfqpoint{1.435076in}{2.596007in}}{\pgfqpoint{1.427176in}{2.599279in}}{\pgfqpoint{1.418940in}{2.599279in}}%
\pgfpathcurveto{\pgfqpoint{1.410703in}{2.599279in}}{\pgfqpoint{1.402803in}{2.596007in}}{\pgfqpoint{1.396979in}{2.590183in}}%
\pgfpathcurveto{\pgfqpoint{1.391156in}{2.584359in}}{\pgfqpoint{1.387883in}{2.576459in}}{\pgfqpoint{1.387883in}{2.568223in}}%
\pgfpathcurveto{\pgfqpoint{1.387883in}{2.559987in}}{\pgfqpoint{1.391156in}{2.552086in}}{\pgfqpoint{1.396979in}{2.546263in}}%
\pgfpathcurveto{\pgfqpoint{1.402803in}{2.540439in}}{\pgfqpoint{1.410703in}{2.537166in}}{\pgfqpoint{1.418940in}{2.537166in}}%
\pgfpathclose%
\pgfusepath{stroke,fill}%
\end{pgfscope}%
\begin{pgfscope}%
\pgfpathrectangle{\pgfqpoint{0.100000in}{0.212622in}}{\pgfqpoint{3.696000in}{3.696000in}}%
\pgfusepath{clip}%
\pgfsetbuttcap%
\pgfsetroundjoin%
\definecolor{currentfill}{rgb}{0.121569,0.466667,0.705882}%
\pgfsetfillcolor{currentfill}%
\pgfsetfillopacity{0.447282}%
\pgfsetlinewidth{1.003750pt}%
\definecolor{currentstroke}{rgb}{0.121569,0.466667,0.705882}%
\pgfsetstrokecolor{currentstroke}%
\pgfsetstrokeopacity{0.447282}%
\pgfsetdash{}{0pt}%
\pgfpathmoveto{\pgfqpoint{1.415874in}{2.531577in}}%
\pgfpathcurveto{\pgfqpoint{1.424110in}{2.531577in}}{\pgfqpoint{1.432010in}{2.534849in}}{\pgfqpoint{1.437834in}{2.540673in}}%
\pgfpathcurveto{\pgfqpoint{1.443658in}{2.546497in}}{\pgfqpoint{1.446930in}{2.554397in}}{\pgfqpoint{1.446930in}{2.562633in}}%
\pgfpathcurveto{\pgfqpoint{1.446930in}{2.570870in}}{\pgfqpoint{1.443658in}{2.578770in}}{\pgfqpoint{1.437834in}{2.584594in}}%
\pgfpathcurveto{\pgfqpoint{1.432010in}{2.590418in}}{\pgfqpoint{1.424110in}{2.593690in}}{\pgfqpoint{1.415874in}{2.593690in}}%
\pgfpathcurveto{\pgfqpoint{1.407637in}{2.593690in}}{\pgfqpoint{1.399737in}{2.590418in}}{\pgfqpoint{1.393913in}{2.584594in}}%
\pgfpathcurveto{\pgfqpoint{1.388089in}{2.578770in}}{\pgfqpoint{1.384817in}{2.570870in}}{\pgfqpoint{1.384817in}{2.562633in}}%
\pgfpathcurveto{\pgfqpoint{1.384817in}{2.554397in}}{\pgfqpoint{1.388089in}{2.546497in}}{\pgfqpoint{1.393913in}{2.540673in}}%
\pgfpathcurveto{\pgfqpoint{1.399737in}{2.534849in}}{\pgfqpoint{1.407637in}{2.531577in}}{\pgfqpoint{1.415874in}{2.531577in}}%
\pgfpathclose%
\pgfusepath{stroke,fill}%
\end{pgfscope}%
\begin{pgfscope}%
\pgfpathrectangle{\pgfqpoint{0.100000in}{0.212622in}}{\pgfqpoint{3.696000in}{3.696000in}}%
\pgfusepath{clip}%
\pgfsetbuttcap%
\pgfsetroundjoin%
\definecolor{currentfill}{rgb}{0.121569,0.466667,0.705882}%
\pgfsetfillcolor{currentfill}%
\pgfsetfillopacity{0.447608}%
\pgfsetlinewidth{1.003750pt}%
\definecolor{currentstroke}{rgb}{0.121569,0.466667,0.705882}%
\pgfsetstrokecolor{currentstroke}%
\pgfsetstrokeopacity{0.447608}%
\pgfsetdash{}{0pt}%
\pgfpathmoveto{\pgfqpoint{2.007390in}{2.723331in}}%
\pgfpathcurveto{\pgfqpoint{2.015626in}{2.723331in}}{\pgfqpoint{2.023526in}{2.726603in}}{\pgfqpoint{2.029350in}{2.732427in}}%
\pgfpathcurveto{\pgfqpoint{2.035174in}{2.738251in}}{\pgfqpoint{2.038447in}{2.746151in}}{\pgfqpoint{2.038447in}{2.754388in}}%
\pgfpathcurveto{\pgfqpoint{2.038447in}{2.762624in}}{\pgfqpoint{2.035174in}{2.770524in}}{\pgfqpoint{2.029350in}{2.776348in}}%
\pgfpathcurveto{\pgfqpoint{2.023526in}{2.782172in}}{\pgfqpoint{2.015626in}{2.785444in}}{\pgfqpoint{2.007390in}{2.785444in}}%
\pgfpathcurveto{\pgfqpoint{1.999154in}{2.785444in}}{\pgfqpoint{1.991254in}{2.782172in}}{\pgfqpoint{1.985430in}{2.776348in}}%
\pgfpathcurveto{\pgfqpoint{1.979606in}{2.770524in}}{\pgfqpoint{1.976334in}{2.762624in}}{\pgfqpoint{1.976334in}{2.754388in}}%
\pgfpathcurveto{\pgfqpoint{1.976334in}{2.746151in}}{\pgfqpoint{1.979606in}{2.738251in}}{\pgfqpoint{1.985430in}{2.732427in}}%
\pgfpathcurveto{\pgfqpoint{1.991254in}{2.726603in}}{\pgfqpoint{1.999154in}{2.723331in}}{\pgfqpoint{2.007390in}{2.723331in}}%
\pgfpathclose%
\pgfusepath{stroke,fill}%
\end{pgfscope}%
\begin{pgfscope}%
\pgfpathrectangle{\pgfqpoint{0.100000in}{0.212622in}}{\pgfqpoint{3.696000in}{3.696000in}}%
\pgfusepath{clip}%
\pgfsetbuttcap%
\pgfsetroundjoin%
\definecolor{currentfill}{rgb}{0.121569,0.466667,0.705882}%
\pgfsetfillcolor{currentfill}%
\pgfsetfillopacity{0.447766}%
\pgfsetlinewidth{1.003750pt}%
\definecolor{currentstroke}{rgb}{0.121569,0.466667,0.705882}%
\pgfsetstrokecolor{currentstroke}%
\pgfsetstrokeopacity{0.447766}%
\pgfsetdash{}{0pt}%
\pgfpathmoveto{\pgfqpoint{1.414197in}{2.527528in}}%
\pgfpathcurveto{\pgfqpoint{1.422433in}{2.527528in}}{\pgfqpoint{1.430333in}{2.530801in}}{\pgfqpoint{1.436157in}{2.536625in}}%
\pgfpathcurveto{\pgfqpoint{1.441981in}{2.542449in}}{\pgfqpoint{1.445253in}{2.550349in}}{\pgfqpoint{1.445253in}{2.558585in}}%
\pgfpathcurveto{\pgfqpoint{1.445253in}{2.566821in}}{\pgfqpoint{1.441981in}{2.574721in}}{\pgfqpoint{1.436157in}{2.580545in}}%
\pgfpathcurveto{\pgfqpoint{1.430333in}{2.586369in}}{\pgfqpoint{1.422433in}{2.589641in}}{\pgfqpoint{1.414197in}{2.589641in}}%
\pgfpathcurveto{\pgfqpoint{1.405960in}{2.589641in}}{\pgfqpoint{1.398060in}{2.586369in}}{\pgfqpoint{1.392236in}{2.580545in}}%
\pgfpathcurveto{\pgfqpoint{1.386412in}{2.574721in}}{\pgfqpoint{1.383140in}{2.566821in}}{\pgfqpoint{1.383140in}{2.558585in}}%
\pgfpathcurveto{\pgfqpoint{1.383140in}{2.550349in}}{\pgfqpoint{1.386412in}{2.542449in}}{\pgfqpoint{1.392236in}{2.536625in}}%
\pgfpathcurveto{\pgfqpoint{1.398060in}{2.530801in}}{\pgfqpoint{1.405960in}{2.527528in}}{\pgfqpoint{1.414197in}{2.527528in}}%
\pgfpathclose%
\pgfusepath{stroke,fill}%
\end{pgfscope}%
\begin{pgfscope}%
\pgfpathrectangle{\pgfqpoint{0.100000in}{0.212622in}}{\pgfqpoint{3.696000in}{3.696000in}}%
\pgfusepath{clip}%
\pgfsetbuttcap%
\pgfsetroundjoin%
\definecolor{currentfill}{rgb}{0.121569,0.466667,0.705882}%
\pgfsetfillcolor{currentfill}%
\pgfsetfillopacity{0.448047}%
\pgfsetlinewidth{1.003750pt}%
\definecolor{currentstroke}{rgb}{0.121569,0.466667,0.705882}%
\pgfsetstrokecolor{currentstroke}%
\pgfsetstrokeopacity{0.448047}%
\pgfsetdash{}{0pt}%
\pgfpathmoveto{\pgfqpoint{1.413058in}{2.525519in}}%
\pgfpathcurveto{\pgfqpoint{1.421295in}{2.525519in}}{\pgfqpoint{1.429195in}{2.528791in}}{\pgfqpoint{1.435019in}{2.534615in}}%
\pgfpathcurveto{\pgfqpoint{1.440843in}{2.540439in}}{\pgfqpoint{1.444115in}{2.548339in}}{\pgfqpoint{1.444115in}{2.556575in}}%
\pgfpathcurveto{\pgfqpoint{1.444115in}{2.564812in}}{\pgfqpoint{1.440843in}{2.572712in}}{\pgfqpoint{1.435019in}{2.578536in}}%
\pgfpathcurveto{\pgfqpoint{1.429195in}{2.584360in}}{\pgfqpoint{1.421295in}{2.587632in}}{\pgfqpoint{1.413058in}{2.587632in}}%
\pgfpathcurveto{\pgfqpoint{1.404822in}{2.587632in}}{\pgfqpoint{1.396922in}{2.584360in}}{\pgfqpoint{1.391098in}{2.578536in}}%
\pgfpathcurveto{\pgfqpoint{1.385274in}{2.572712in}}{\pgfqpoint{1.382002in}{2.564812in}}{\pgfqpoint{1.382002in}{2.556575in}}%
\pgfpathcurveto{\pgfqpoint{1.382002in}{2.548339in}}{\pgfqpoint{1.385274in}{2.540439in}}{\pgfqpoint{1.391098in}{2.534615in}}%
\pgfpathcurveto{\pgfqpoint{1.396922in}{2.528791in}}{\pgfqpoint{1.404822in}{2.525519in}}{\pgfqpoint{1.413058in}{2.525519in}}%
\pgfpathclose%
\pgfusepath{stroke,fill}%
\end{pgfscope}%
\begin{pgfscope}%
\pgfpathrectangle{\pgfqpoint{0.100000in}{0.212622in}}{\pgfqpoint{3.696000in}{3.696000in}}%
\pgfusepath{clip}%
\pgfsetbuttcap%
\pgfsetroundjoin%
\definecolor{currentfill}{rgb}{0.121569,0.466667,0.705882}%
\pgfsetfillcolor{currentfill}%
\pgfsetfillopacity{0.448544}%
\pgfsetlinewidth{1.003750pt}%
\definecolor{currentstroke}{rgb}{0.121569,0.466667,0.705882}%
\pgfsetstrokecolor{currentstroke}%
\pgfsetstrokeopacity{0.448544}%
\pgfsetdash{}{0pt}%
\pgfpathmoveto{\pgfqpoint{1.411011in}{2.521777in}}%
\pgfpathcurveto{\pgfqpoint{1.419247in}{2.521777in}}{\pgfqpoint{1.427147in}{2.525049in}}{\pgfqpoint{1.432971in}{2.530873in}}%
\pgfpathcurveto{\pgfqpoint{1.438795in}{2.536697in}}{\pgfqpoint{1.442067in}{2.544597in}}{\pgfqpoint{1.442067in}{2.552833in}}%
\pgfpathcurveto{\pgfqpoint{1.442067in}{2.561070in}}{\pgfqpoint{1.438795in}{2.568970in}}{\pgfqpoint{1.432971in}{2.574794in}}%
\pgfpathcurveto{\pgfqpoint{1.427147in}{2.580618in}}{\pgfqpoint{1.419247in}{2.583890in}}{\pgfqpoint{1.411011in}{2.583890in}}%
\pgfpathcurveto{\pgfqpoint{1.402775in}{2.583890in}}{\pgfqpoint{1.394874in}{2.580618in}}{\pgfqpoint{1.389051in}{2.574794in}}%
\pgfpathcurveto{\pgfqpoint{1.383227in}{2.568970in}}{\pgfqpoint{1.379954in}{2.561070in}}{\pgfqpoint{1.379954in}{2.552833in}}%
\pgfpathcurveto{\pgfqpoint{1.379954in}{2.544597in}}{\pgfqpoint{1.383227in}{2.536697in}}{\pgfqpoint{1.389051in}{2.530873in}}%
\pgfpathcurveto{\pgfqpoint{1.394874in}{2.525049in}}{\pgfqpoint{1.402775in}{2.521777in}}{\pgfqpoint{1.411011in}{2.521777in}}%
\pgfpathclose%
\pgfusepath{stroke,fill}%
\end{pgfscope}%
\begin{pgfscope}%
\pgfpathrectangle{\pgfqpoint{0.100000in}{0.212622in}}{\pgfqpoint{3.696000in}{3.696000in}}%
\pgfusepath{clip}%
\pgfsetbuttcap%
\pgfsetroundjoin%
\definecolor{currentfill}{rgb}{0.121569,0.466667,0.705882}%
\pgfsetfillcolor{currentfill}%
\pgfsetfillopacity{0.449467}%
\pgfsetlinewidth{1.003750pt}%
\definecolor{currentstroke}{rgb}{0.121569,0.466667,0.705882}%
\pgfsetstrokecolor{currentstroke}%
\pgfsetstrokeopacity{0.449467}%
\pgfsetdash{}{0pt}%
\pgfpathmoveto{\pgfqpoint{1.407957in}{2.514185in}}%
\pgfpathcurveto{\pgfqpoint{1.416193in}{2.514185in}}{\pgfqpoint{1.424093in}{2.517457in}}{\pgfqpoint{1.429917in}{2.523281in}}%
\pgfpathcurveto{\pgfqpoint{1.435741in}{2.529105in}}{\pgfqpoint{1.439013in}{2.537005in}}{\pgfqpoint{1.439013in}{2.545241in}}%
\pgfpathcurveto{\pgfqpoint{1.439013in}{2.553477in}}{\pgfqpoint{1.435741in}{2.561378in}}{\pgfqpoint{1.429917in}{2.567201in}}%
\pgfpathcurveto{\pgfqpoint{1.424093in}{2.573025in}}{\pgfqpoint{1.416193in}{2.576298in}}{\pgfqpoint{1.407957in}{2.576298in}}%
\pgfpathcurveto{\pgfqpoint{1.399721in}{2.576298in}}{\pgfqpoint{1.391820in}{2.573025in}}{\pgfqpoint{1.385997in}{2.567201in}}%
\pgfpathcurveto{\pgfqpoint{1.380173in}{2.561378in}}{\pgfqpoint{1.376900in}{2.553477in}}{\pgfqpoint{1.376900in}{2.545241in}}%
\pgfpathcurveto{\pgfqpoint{1.376900in}{2.537005in}}{\pgfqpoint{1.380173in}{2.529105in}}{\pgfqpoint{1.385997in}{2.523281in}}%
\pgfpathcurveto{\pgfqpoint{1.391820in}{2.517457in}}{\pgfqpoint{1.399721in}{2.514185in}}{\pgfqpoint{1.407957in}{2.514185in}}%
\pgfpathclose%
\pgfusepath{stroke,fill}%
\end{pgfscope}%
\begin{pgfscope}%
\pgfpathrectangle{\pgfqpoint{0.100000in}{0.212622in}}{\pgfqpoint{3.696000in}{3.696000in}}%
\pgfusepath{clip}%
\pgfsetbuttcap%
\pgfsetroundjoin%
\definecolor{currentfill}{rgb}{0.121569,0.466667,0.705882}%
\pgfsetfillcolor{currentfill}%
\pgfsetfillopacity{0.450952}%
\pgfsetlinewidth{1.003750pt}%
\definecolor{currentstroke}{rgb}{0.121569,0.466667,0.705882}%
\pgfsetstrokecolor{currentstroke}%
\pgfsetstrokeopacity{0.450952}%
\pgfsetdash{}{0pt}%
\pgfpathmoveto{\pgfqpoint{2.010208in}{2.710057in}}%
\pgfpathcurveto{\pgfqpoint{2.018444in}{2.710057in}}{\pgfqpoint{2.026344in}{2.713330in}}{\pgfqpoint{2.032168in}{2.719154in}}%
\pgfpathcurveto{\pgfqpoint{2.037992in}{2.724977in}}{\pgfqpoint{2.041265in}{2.732878in}}{\pgfqpoint{2.041265in}{2.741114in}}%
\pgfpathcurveto{\pgfqpoint{2.041265in}{2.749350in}}{\pgfqpoint{2.037992in}{2.757250in}}{\pgfqpoint{2.032168in}{2.763074in}}%
\pgfpathcurveto{\pgfqpoint{2.026344in}{2.768898in}}{\pgfqpoint{2.018444in}{2.772170in}}{\pgfqpoint{2.010208in}{2.772170in}}%
\pgfpathcurveto{\pgfqpoint{2.001972in}{2.772170in}}{\pgfqpoint{1.994072in}{2.768898in}}{\pgfqpoint{1.988248in}{2.763074in}}%
\pgfpathcurveto{\pgfqpoint{1.982424in}{2.757250in}}{\pgfqpoint{1.979152in}{2.749350in}}{\pgfqpoint{1.979152in}{2.741114in}}%
\pgfpathcurveto{\pgfqpoint{1.979152in}{2.732878in}}{\pgfqpoint{1.982424in}{2.724977in}}{\pgfqpoint{1.988248in}{2.719154in}}%
\pgfpathcurveto{\pgfqpoint{1.994072in}{2.713330in}}{\pgfqpoint{2.001972in}{2.710057in}}{\pgfqpoint{2.010208in}{2.710057in}}%
\pgfpathclose%
\pgfusepath{stroke,fill}%
\end{pgfscope}%
\begin{pgfscope}%
\pgfpathrectangle{\pgfqpoint{0.100000in}{0.212622in}}{\pgfqpoint{3.696000in}{3.696000in}}%
\pgfusepath{clip}%
\pgfsetbuttcap%
\pgfsetroundjoin%
\definecolor{currentfill}{rgb}{0.121569,0.466667,0.705882}%
\pgfsetfillcolor{currentfill}%
\pgfsetfillopacity{0.451372}%
\pgfsetlinewidth{1.003750pt}%
\definecolor{currentstroke}{rgb}{0.121569,0.466667,0.705882}%
\pgfsetstrokecolor{currentstroke}%
\pgfsetstrokeopacity{0.451372}%
\pgfsetdash{}{0pt}%
\pgfpathmoveto{\pgfqpoint{1.401538in}{2.502277in}}%
\pgfpathcurveto{\pgfqpoint{1.409775in}{2.502277in}}{\pgfqpoint{1.417675in}{2.505549in}}{\pgfqpoint{1.423499in}{2.511373in}}%
\pgfpathcurveto{\pgfqpoint{1.429323in}{2.517197in}}{\pgfqpoint{1.432595in}{2.525097in}}{\pgfqpoint{1.432595in}{2.533334in}}%
\pgfpathcurveto{\pgfqpoint{1.432595in}{2.541570in}}{\pgfqpoint{1.429323in}{2.549470in}}{\pgfqpoint{1.423499in}{2.555294in}}%
\pgfpathcurveto{\pgfqpoint{1.417675in}{2.561118in}}{\pgfqpoint{1.409775in}{2.564390in}}{\pgfqpoint{1.401538in}{2.564390in}}%
\pgfpathcurveto{\pgfqpoint{1.393302in}{2.564390in}}{\pgfqpoint{1.385402in}{2.561118in}}{\pgfqpoint{1.379578in}{2.555294in}}%
\pgfpathcurveto{\pgfqpoint{1.373754in}{2.549470in}}{\pgfqpoint{1.370482in}{2.541570in}}{\pgfqpoint{1.370482in}{2.533334in}}%
\pgfpathcurveto{\pgfqpoint{1.370482in}{2.525097in}}{\pgfqpoint{1.373754in}{2.517197in}}{\pgfqpoint{1.379578in}{2.511373in}}%
\pgfpathcurveto{\pgfqpoint{1.385402in}{2.505549in}}{\pgfqpoint{1.393302in}{2.502277in}}{\pgfqpoint{1.401538in}{2.502277in}}%
\pgfpathclose%
\pgfusepath{stroke,fill}%
\end{pgfscope}%
\begin{pgfscope}%
\pgfpathrectangle{\pgfqpoint{0.100000in}{0.212622in}}{\pgfqpoint{3.696000in}{3.696000in}}%
\pgfusepath{clip}%
\pgfsetbuttcap%
\pgfsetroundjoin%
\definecolor{currentfill}{rgb}{0.121569,0.466667,0.705882}%
\pgfsetfillcolor{currentfill}%
\pgfsetfillopacity{0.453132}%
\pgfsetlinewidth{1.003750pt}%
\definecolor{currentstroke}{rgb}{0.121569,0.466667,0.705882}%
\pgfsetstrokecolor{currentstroke}%
\pgfsetstrokeopacity{0.453132}%
\pgfsetdash{}{0pt}%
\pgfpathmoveto{\pgfqpoint{1.395062in}{2.490955in}}%
\pgfpathcurveto{\pgfqpoint{1.403298in}{2.490955in}}{\pgfqpoint{1.411198in}{2.494228in}}{\pgfqpoint{1.417022in}{2.500052in}}%
\pgfpathcurveto{\pgfqpoint{1.422846in}{2.505876in}}{\pgfqpoint{1.426118in}{2.513776in}}{\pgfqpoint{1.426118in}{2.522012in}}%
\pgfpathcurveto{\pgfqpoint{1.426118in}{2.530248in}}{\pgfqpoint{1.422846in}{2.538148in}}{\pgfqpoint{1.417022in}{2.543972in}}%
\pgfpathcurveto{\pgfqpoint{1.411198in}{2.549796in}}{\pgfqpoint{1.403298in}{2.553068in}}{\pgfqpoint{1.395062in}{2.553068in}}%
\pgfpathcurveto{\pgfqpoint{1.386826in}{2.553068in}}{\pgfqpoint{1.378926in}{2.549796in}}{\pgfqpoint{1.373102in}{2.543972in}}%
\pgfpathcurveto{\pgfqpoint{1.367278in}{2.538148in}}{\pgfqpoint{1.364005in}{2.530248in}}{\pgfqpoint{1.364005in}{2.522012in}}%
\pgfpathcurveto{\pgfqpoint{1.364005in}{2.513776in}}{\pgfqpoint{1.367278in}{2.505876in}}{\pgfqpoint{1.373102in}{2.500052in}}%
\pgfpathcurveto{\pgfqpoint{1.378926in}{2.494228in}}{\pgfqpoint{1.386826in}{2.490955in}}{\pgfqpoint{1.395062in}{2.490955in}}%
\pgfpathclose%
\pgfusepath{stroke,fill}%
\end{pgfscope}%
\begin{pgfscope}%
\pgfpathrectangle{\pgfqpoint{0.100000in}{0.212622in}}{\pgfqpoint{3.696000in}{3.696000in}}%
\pgfusepath{clip}%
\pgfsetbuttcap%
\pgfsetroundjoin%
\definecolor{currentfill}{rgb}{0.121569,0.466667,0.705882}%
\pgfsetfillcolor{currentfill}%
\pgfsetfillopacity{0.454883}%
\pgfsetlinewidth{1.003750pt}%
\definecolor{currentstroke}{rgb}{0.121569,0.466667,0.705882}%
\pgfsetstrokecolor{currentstroke}%
\pgfsetstrokeopacity{0.454883}%
\pgfsetdash{}{0pt}%
\pgfpathmoveto{\pgfqpoint{1.389662in}{2.479909in}}%
\pgfpathcurveto{\pgfqpoint{1.397898in}{2.479909in}}{\pgfqpoint{1.405798in}{2.483181in}}{\pgfqpoint{1.411622in}{2.489005in}}%
\pgfpathcurveto{\pgfqpoint{1.417446in}{2.494829in}}{\pgfqpoint{1.420718in}{2.502729in}}{\pgfqpoint{1.420718in}{2.510965in}}%
\pgfpathcurveto{\pgfqpoint{1.420718in}{2.519202in}}{\pgfqpoint{1.417446in}{2.527102in}}{\pgfqpoint{1.411622in}{2.532926in}}%
\pgfpathcurveto{\pgfqpoint{1.405798in}{2.538749in}}{\pgfqpoint{1.397898in}{2.542022in}}{\pgfqpoint{1.389662in}{2.542022in}}%
\pgfpathcurveto{\pgfqpoint{1.381425in}{2.542022in}}{\pgfqpoint{1.373525in}{2.538749in}}{\pgfqpoint{1.367701in}{2.532926in}}%
\pgfpathcurveto{\pgfqpoint{1.361877in}{2.527102in}}{\pgfqpoint{1.358605in}{2.519202in}}{\pgfqpoint{1.358605in}{2.510965in}}%
\pgfpathcurveto{\pgfqpoint{1.358605in}{2.502729in}}{\pgfqpoint{1.361877in}{2.494829in}}{\pgfqpoint{1.367701in}{2.489005in}}%
\pgfpathcurveto{\pgfqpoint{1.373525in}{2.483181in}}{\pgfqpoint{1.381425in}{2.479909in}}{\pgfqpoint{1.389662in}{2.479909in}}%
\pgfpathclose%
\pgfusepath{stroke,fill}%
\end{pgfscope}%
\begin{pgfscope}%
\pgfpathrectangle{\pgfqpoint{0.100000in}{0.212622in}}{\pgfqpoint{3.696000in}{3.696000in}}%
\pgfusepath{clip}%
\pgfsetbuttcap%
\pgfsetroundjoin%
\definecolor{currentfill}{rgb}{0.121569,0.466667,0.705882}%
\pgfsetfillcolor{currentfill}%
\pgfsetfillopacity{0.454981}%
\pgfsetlinewidth{1.003750pt}%
\definecolor{currentstroke}{rgb}{0.121569,0.466667,0.705882}%
\pgfsetstrokecolor{currentstroke}%
\pgfsetstrokeopacity{0.454981}%
\pgfsetdash{}{0pt}%
\pgfpathmoveto{\pgfqpoint{2.012613in}{2.697344in}}%
\pgfpathcurveto{\pgfqpoint{2.020849in}{2.697344in}}{\pgfqpoint{2.028749in}{2.700616in}}{\pgfqpoint{2.034573in}{2.706440in}}%
\pgfpathcurveto{\pgfqpoint{2.040397in}{2.712264in}}{\pgfqpoint{2.043669in}{2.720164in}}{\pgfqpoint{2.043669in}{2.728400in}}%
\pgfpathcurveto{\pgfqpoint{2.043669in}{2.736637in}}{\pgfqpoint{2.040397in}{2.744537in}}{\pgfqpoint{2.034573in}{2.750361in}}%
\pgfpathcurveto{\pgfqpoint{2.028749in}{2.756184in}}{\pgfqpoint{2.020849in}{2.759457in}}{\pgfqpoint{2.012613in}{2.759457in}}%
\pgfpathcurveto{\pgfqpoint{2.004376in}{2.759457in}}{\pgfqpoint{1.996476in}{2.756184in}}{\pgfqpoint{1.990652in}{2.750361in}}%
\pgfpathcurveto{\pgfqpoint{1.984828in}{2.744537in}}{\pgfqpoint{1.981556in}{2.736637in}}{\pgfqpoint{1.981556in}{2.728400in}}%
\pgfpathcurveto{\pgfqpoint{1.981556in}{2.720164in}}{\pgfqpoint{1.984828in}{2.712264in}}{\pgfqpoint{1.990652in}{2.706440in}}%
\pgfpathcurveto{\pgfqpoint{1.996476in}{2.700616in}}{\pgfqpoint{2.004376in}{2.697344in}}{\pgfqpoint{2.012613in}{2.697344in}}%
\pgfpathclose%
\pgfusepath{stroke,fill}%
\end{pgfscope}%
\begin{pgfscope}%
\pgfpathrectangle{\pgfqpoint{0.100000in}{0.212622in}}{\pgfqpoint{3.696000in}{3.696000in}}%
\pgfusepath{clip}%
\pgfsetbuttcap%
\pgfsetroundjoin%
\definecolor{currentfill}{rgb}{0.121569,0.466667,0.705882}%
\pgfsetfillcolor{currentfill}%
\pgfsetfillopacity{0.456134}%
\pgfsetlinewidth{1.003750pt}%
\definecolor{currentstroke}{rgb}{0.121569,0.466667,0.705882}%
\pgfsetstrokecolor{currentstroke}%
\pgfsetstrokeopacity{0.456134}%
\pgfsetdash{}{0pt}%
\pgfpathmoveto{\pgfqpoint{1.384397in}{2.470827in}}%
\pgfpathcurveto{\pgfqpoint{1.392634in}{2.470827in}}{\pgfqpoint{1.400534in}{2.474099in}}{\pgfqpoint{1.406358in}{2.479923in}}%
\pgfpathcurveto{\pgfqpoint{1.412182in}{2.485747in}}{\pgfqpoint{1.415454in}{2.493647in}}{\pgfqpoint{1.415454in}{2.501884in}}%
\pgfpathcurveto{\pgfqpoint{1.415454in}{2.510120in}}{\pgfqpoint{1.412182in}{2.518020in}}{\pgfqpoint{1.406358in}{2.523844in}}%
\pgfpathcurveto{\pgfqpoint{1.400534in}{2.529668in}}{\pgfqpoint{1.392634in}{2.532940in}}{\pgfqpoint{1.384397in}{2.532940in}}%
\pgfpathcurveto{\pgfqpoint{1.376161in}{2.532940in}}{\pgfqpoint{1.368261in}{2.529668in}}{\pgfqpoint{1.362437in}{2.523844in}}%
\pgfpathcurveto{\pgfqpoint{1.356613in}{2.518020in}}{\pgfqpoint{1.353341in}{2.510120in}}{\pgfqpoint{1.353341in}{2.501884in}}%
\pgfpathcurveto{\pgfqpoint{1.353341in}{2.493647in}}{\pgfqpoint{1.356613in}{2.485747in}}{\pgfqpoint{1.362437in}{2.479923in}}%
\pgfpathcurveto{\pgfqpoint{1.368261in}{2.474099in}}{\pgfqpoint{1.376161in}{2.470827in}}{\pgfqpoint{1.384397in}{2.470827in}}%
\pgfpathclose%
\pgfusepath{stroke,fill}%
\end{pgfscope}%
\begin{pgfscope}%
\pgfpathrectangle{\pgfqpoint{0.100000in}{0.212622in}}{\pgfqpoint{3.696000in}{3.696000in}}%
\pgfusepath{clip}%
\pgfsetbuttcap%
\pgfsetroundjoin%
\definecolor{currentfill}{rgb}{0.121569,0.466667,0.705882}%
\pgfsetfillcolor{currentfill}%
\pgfsetfillopacity{0.457237}%
\pgfsetlinewidth{1.003750pt}%
\definecolor{currentstroke}{rgb}{0.121569,0.466667,0.705882}%
\pgfsetstrokecolor{currentstroke}%
\pgfsetstrokeopacity{0.457237}%
\pgfsetdash{}{0pt}%
\pgfpathmoveto{\pgfqpoint{1.380868in}{2.462880in}}%
\pgfpathcurveto{\pgfqpoint{1.389105in}{2.462880in}}{\pgfqpoint{1.397005in}{2.466152in}}{\pgfqpoint{1.402829in}{2.471976in}}%
\pgfpathcurveto{\pgfqpoint{1.408653in}{2.477800in}}{\pgfqpoint{1.411925in}{2.485700in}}{\pgfqpoint{1.411925in}{2.493936in}}%
\pgfpathcurveto{\pgfqpoint{1.411925in}{2.502173in}}{\pgfqpoint{1.408653in}{2.510073in}}{\pgfqpoint{1.402829in}{2.515897in}}%
\pgfpathcurveto{\pgfqpoint{1.397005in}{2.521721in}}{\pgfqpoint{1.389105in}{2.524993in}}{\pgfqpoint{1.380868in}{2.524993in}}%
\pgfpathcurveto{\pgfqpoint{1.372632in}{2.524993in}}{\pgfqpoint{1.364732in}{2.521721in}}{\pgfqpoint{1.358908in}{2.515897in}}%
\pgfpathcurveto{\pgfqpoint{1.353084in}{2.510073in}}{\pgfqpoint{1.349812in}{2.502173in}}{\pgfqpoint{1.349812in}{2.493936in}}%
\pgfpathcurveto{\pgfqpoint{1.349812in}{2.485700in}}{\pgfqpoint{1.353084in}{2.477800in}}{\pgfqpoint{1.358908in}{2.471976in}}%
\pgfpathcurveto{\pgfqpoint{1.364732in}{2.466152in}}{\pgfqpoint{1.372632in}{2.462880in}}{\pgfqpoint{1.380868in}{2.462880in}}%
\pgfpathclose%
\pgfusepath{stroke,fill}%
\end{pgfscope}%
\begin{pgfscope}%
\pgfpathrectangle{\pgfqpoint{0.100000in}{0.212622in}}{\pgfqpoint{3.696000in}{3.696000in}}%
\pgfusepath{clip}%
\pgfsetbuttcap%
\pgfsetroundjoin%
\definecolor{currentfill}{rgb}{0.121569,0.466667,0.705882}%
\pgfsetfillcolor{currentfill}%
\pgfsetfillopacity{0.457993}%
\pgfsetlinewidth{1.003750pt}%
\definecolor{currentstroke}{rgb}{0.121569,0.466667,0.705882}%
\pgfsetstrokecolor{currentstroke}%
\pgfsetstrokeopacity{0.457993}%
\pgfsetdash{}{0pt}%
\pgfpathmoveto{\pgfqpoint{1.378041in}{2.458099in}}%
\pgfpathcurveto{\pgfqpoint{1.386278in}{2.458099in}}{\pgfqpoint{1.394178in}{2.461372in}}{\pgfqpoint{1.400002in}{2.467196in}}%
\pgfpathcurveto{\pgfqpoint{1.405826in}{2.473020in}}{\pgfqpoint{1.409098in}{2.480920in}}{\pgfqpoint{1.409098in}{2.489156in}}%
\pgfpathcurveto{\pgfqpoint{1.409098in}{2.497392in}}{\pgfqpoint{1.405826in}{2.505292in}}{\pgfqpoint{1.400002in}{2.511116in}}%
\pgfpathcurveto{\pgfqpoint{1.394178in}{2.516940in}}{\pgfqpoint{1.386278in}{2.520212in}}{\pgfqpoint{1.378041in}{2.520212in}}%
\pgfpathcurveto{\pgfqpoint{1.369805in}{2.520212in}}{\pgfqpoint{1.361905in}{2.516940in}}{\pgfqpoint{1.356081in}{2.511116in}}%
\pgfpathcurveto{\pgfqpoint{1.350257in}{2.505292in}}{\pgfqpoint{1.346985in}{2.497392in}}{\pgfqpoint{1.346985in}{2.489156in}}%
\pgfpathcurveto{\pgfqpoint{1.346985in}{2.480920in}}{\pgfqpoint{1.350257in}{2.473020in}}{\pgfqpoint{1.356081in}{2.467196in}}%
\pgfpathcurveto{\pgfqpoint{1.361905in}{2.461372in}}{\pgfqpoint{1.369805in}{2.458099in}}{\pgfqpoint{1.378041in}{2.458099in}}%
\pgfpathclose%
\pgfusepath{stroke,fill}%
\end{pgfscope}%
\begin{pgfscope}%
\pgfpathrectangle{\pgfqpoint{0.100000in}{0.212622in}}{\pgfqpoint{3.696000in}{3.696000in}}%
\pgfusepath{clip}%
\pgfsetbuttcap%
\pgfsetroundjoin%
\definecolor{currentfill}{rgb}{0.121569,0.466667,0.705882}%
\pgfsetfillcolor{currentfill}%
\pgfsetfillopacity{0.458729}%
\pgfsetlinewidth{1.003750pt}%
\definecolor{currentstroke}{rgb}{0.121569,0.466667,0.705882}%
\pgfsetstrokecolor{currentstroke}%
\pgfsetstrokeopacity{0.458729}%
\pgfsetdash{}{0pt}%
\pgfpathmoveto{\pgfqpoint{1.375728in}{2.453469in}}%
\pgfpathcurveto{\pgfqpoint{1.383964in}{2.453469in}}{\pgfqpoint{1.391864in}{2.456741in}}{\pgfqpoint{1.397688in}{2.462565in}}%
\pgfpathcurveto{\pgfqpoint{1.403512in}{2.468389in}}{\pgfqpoint{1.406784in}{2.476289in}}{\pgfqpoint{1.406784in}{2.484525in}}%
\pgfpathcurveto{\pgfqpoint{1.406784in}{2.492761in}}{\pgfqpoint{1.403512in}{2.500661in}}{\pgfqpoint{1.397688in}{2.506485in}}%
\pgfpathcurveto{\pgfqpoint{1.391864in}{2.512309in}}{\pgfqpoint{1.383964in}{2.515582in}}{\pgfqpoint{1.375728in}{2.515582in}}%
\pgfpathcurveto{\pgfqpoint{1.367492in}{2.515582in}}{\pgfqpoint{1.359592in}{2.512309in}}{\pgfqpoint{1.353768in}{2.506485in}}%
\pgfpathcurveto{\pgfqpoint{1.347944in}{2.500661in}}{\pgfqpoint{1.344671in}{2.492761in}}{\pgfqpoint{1.344671in}{2.484525in}}%
\pgfpathcurveto{\pgfqpoint{1.344671in}{2.476289in}}{\pgfqpoint{1.347944in}{2.468389in}}{\pgfqpoint{1.353768in}{2.462565in}}%
\pgfpathcurveto{\pgfqpoint{1.359592in}{2.456741in}}{\pgfqpoint{1.367492in}{2.453469in}}{\pgfqpoint{1.375728in}{2.453469in}}%
\pgfpathclose%
\pgfusepath{stroke,fill}%
\end{pgfscope}%
\begin{pgfscope}%
\pgfpathrectangle{\pgfqpoint{0.100000in}{0.212622in}}{\pgfqpoint{3.696000in}{3.696000in}}%
\pgfusepath{clip}%
\pgfsetbuttcap%
\pgfsetroundjoin%
\definecolor{currentfill}{rgb}{0.121569,0.466667,0.705882}%
\pgfsetfillcolor{currentfill}%
\pgfsetfillopacity{0.459326}%
\pgfsetlinewidth{1.003750pt}%
\definecolor{currentstroke}{rgb}{0.121569,0.466667,0.705882}%
\pgfsetstrokecolor{currentstroke}%
\pgfsetstrokeopacity{0.459326}%
\pgfsetdash{}{0pt}%
\pgfpathmoveto{\pgfqpoint{2.014448in}{2.684268in}}%
\pgfpathcurveto{\pgfqpoint{2.022684in}{2.684268in}}{\pgfqpoint{2.030584in}{2.687540in}}{\pgfqpoint{2.036408in}{2.693364in}}%
\pgfpathcurveto{\pgfqpoint{2.042232in}{2.699188in}}{\pgfqpoint{2.045504in}{2.707088in}}{\pgfqpoint{2.045504in}{2.715324in}}%
\pgfpathcurveto{\pgfqpoint{2.045504in}{2.723560in}}{\pgfqpoint{2.042232in}{2.731461in}}{\pgfqpoint{2.036408in}{2.737284in}}%
\pgfpathcurveto{\pgfqpoint{2.030584in}{2.743108in}}{\pgfqpoint{2.022684in}{2.746381in}}{\pgfqpoint{2.014448in}{2.746381in}}%
\pgfpathcurveto{\pgfqpoint{2.006211in}{2.746381in}}{\pgfqpoint{1.998311in}{2.743108in}}{\pgfqpoint{1.992487in}{2.737284in}}%
\pgfpathcurveto{\pgfqpoint{1.986664in}{2.731461in}}{\pgfqpoint{1.983391in}{2.723560in}}{\pgfqpoint{1.983391in}{2.715324in}}%
\pgfpathcurveto{\pgfqpoint{1.983391in}{2.707088in}}{\pgfqpoint{1.986664in}{2.699188in}}{\pgfqpoint{1.992487in}{2.693364in}}%
\pgfpathcurveto{\pgfqpoint{1.998311in}{2.687540in}}{\pgfqpoint{2.006211in}{2.684268in}}{\pgfqpoint{2.014448in}{2.684268in}}%
\pgfpathclose%
\pgfusepath{stroke,fill}%
\end{pgfscope}%
\begin{pgfscope}%
\pgfpathrectangle{\pgfqpoint{0.100000in}{0.212622in}}{\pgfqpoint{3.696000in}{3.696000in}}%
\pgfusepath{clip}%
\pgfsetbuttcap%
\pgfsetroundjoin%
\definecolor{currentfill}{rgb}{0.121569,0.466667,0.705882}%
\pgfsetfillcolor{currentfill}%
\pgfsetfillopacity{0.459425}%
\pgfsetlinewidth{1.003750pt}%
\definecolor{currentstroke}{rgb}{0.121569,0.466667,0.705882}%
\pgfsetstrokecolor{currentstroke}%
\pgfsetstrokeopacity{0.459425}%
\pgfsetdash{}{0pt}%
\pgfpathmoveto{\pgfqpoint{1.373447in}{2.449554in}}%
\pgfpathcurveto{\pgfqpoint{1.381683in}{2.449554in}}{\pgfqpoint{1.389584in}{2.452826in}}{\pgfqpoint{1.395407in}{2.458650in}}%
\pgfpathcurveto{\pgfqpoint{1.401231in}{2.464474in}}{\pgfqpoint{1.404504in}{2.472374in}}{\pgfqpoint{1.404504in}{2.480611in}}%
\pgfpathcurveto{\pgfqpoint{1.404504in}{2.488847in}}{\pgfqpoint{1.401231in}{2.496747in}}{\pgfqpoint{1.395407in}{2.502571in}}%
\pgfpathcurveto{\pgfqpoint{1.389584in}{2.508395in}}{\pgfqpoint{1.381683in}{2.511667in}}{\pgfqpoint{1.373447in}{2.511667in}}%
\pgfpathcurveto{\pgfqpoint{1.365211in}{2.511667in}}{\pgfqpoint{1.357311in}{2.508395in}}{\pgfqpoint{1.351487in}{2.502571in}}%
\pgfpathcurveto{\pgfqpoint{1.345663in}{2.496747in}}{\pgfqpoint{1.342391in}{2.488847in}}{\pgfqpoint{1.342391in}{2.480611in}}%
\pgfpathcurveto{\pgfqpoint{1.342391in}{2.472374in}}{\pgfqpoint{1.345663in}{2.464474in}}{\pgfqpoint{1.351487in}{2.458650in}}%
\pgfpathcurveto{\pgfqpoint{1.357311in}{2.452826in}}{\pgfqpoint{1.365211in}{2.449554in}}{\pgfqpoint{1.373447in}{2.449554in}}%
\pgfpathclose%
\pgfusepath{stroke,fill}%
\end{pgfscope}%
\begin{pgfscope}%
\pgfpathrectangle{\pgfqpoint{0.100000in}{0.212622in}}{\pgfqpoint{3.696000in}{3.696000in}}%
\pgfusepath{clip}%
\pgfsetbuttcap%
\pgfsetroundjoin%
\definecolor{currentfill}{rgb}{0.121569,0.466667,0.705882}%
\pgfsetfillcolor{currentfill}%
\pgfsetfillopacity{0.460607}%
\pgfsetlinewidth{1.003750pt}%
\definecolor{currentstroke}{rgb}{0.121569,0.466667,0.705882}%
\pgfsetstrokecolor{currentstroke}%
\pgfsetstrokeopacity{0.460607}%
\pgfsetdash{}{0pt}%
\pgfpathmoveto{\pgfqpoint{1.369259in}{2.442175in}}%
\pgfpathcurveto{\pgfqpoint{1.377495in}{2.442175in}}{\pgfqpoint{1.385395in}{2.445447in}}{\pgfqpoint{1.391219in}{2.451271in}}%
\pgfpathcurveto{\pgfqpoint{1.397043in}{2.457095in}}{\pgfqpoint{1.400315in}{2.464995in}}{\pgfqpoint{1.400315in}{2.473231in}}%
\pgfpathcurveto{\pgfqpoint{1.400315in}{2.481468in}}{\pgfqpoint{1.397043in}{2.489368in}}{\pgfqpoint{1.391219in}{2.495192in}}%
\pgfpathcurveto{\pgfqpoint{1.385395in}{2.501015in}}{\pgfqpoint{1.377495in}{2.504288in}}{\pgfqpoint{1.369259in}{2.504288in}}%
\pgfpathcurveto{\pgfqpoint{1.361023in}{2.504288in}}{\pgfqpoint{1.353123in}{2.501015in}}{\pgfqpoint{1.347299in}{2.495192in}}%
\pgfpathcurveto{\pgfqpoint{1.341475in}{2.489368in}}{\pgfqpoint{1.338202in}{2.481468in}}{\pgfqpoint{1.338202in}{2.473231in}}%
\pgfpathcurveto{\pgfqpoint{1.338202in}{2.464995in}}{\pgfqpoint{1.341475in}{2.457095in}}{\pgfqpoint{1.347299in}{2.451271in}}%
\pgfpathcurveto{\pgfqpoint{1.353123in}{2.445447in}}{\pgfqpoint{1.361023in}{2.442175in}}{\pgfqpoint{1.369259in}{2.442175in}}%
\pgfpathclose%
\pgfusepath{stroke,fill}%
\end{pgfscope}%
\begin{pgfscope}%
\pgfpathrectangle{\pgfqpoint{0.100000in}{0.212622in}}{\pgfqpoint{3.696000in}{3.696000in}}%
\pgfusepath{clip}%
\pgfsetbuttcap%
\pgfsetroundjoin%
\definecolor{currentfill}{rgb}{0.121569,0.466667,0.705882}%
\pgfsetfillcolor{currentfill}%
\pgfsetfillopacity{0.461643}%
\pgfsetlinewidth{1.003750pt}%
\definecolor{currentstroke}{rgb}{0.121569,0.466667,0.705882}%
\pgfsetstrokecolor{currentstroke}%
\pgfsetstrokeopacity{0.461643}%
\pgfsetdash{}{0pt}%
\pgfpathmoveto{\pgfqpoint{1.366097in}{2.435351in}}%
\pgfpathcurveto{\pgfqpoint{1.374333in}{2.435351in}}{\pgfqpoint{1.382233in}{2.438623in}}{\pgfqpoint{1.388057in}{2.444447in}}%
\pgfpathcurveto{\pgfqpoint{1.393881in}{2.450271in}}{\pgfqpoint{1.397153in}{2.458171in}}{\pgfqpoint{1.397153in}{2.466407in}}%
\pgfpathcurveto{\pgfqpoint{1.397153in}{2.474643in}}{\pgfqpoint{1.393881in}{2.482544in}}{\pgfqpoint{1.388057in}{2.488367in}}%
\pgfpathcurveto{\pgfqpoint{1.382233in}{2.494191in}}{\pgfqpoint{1.374333in}{2.497464in}}{\pgfqpoint{1.366097in}{2.497464in}}%
\pgfpathcurveto{\pgfqpoint{1.357860in}{2.497464in}}{\pgfqpoint{1.349960in}{2.494191in}}{\pgfqpoint{1.344136in}{2.488367in}}%
\pgfpathcurveto{\pgfqpoint{1.338312in}{2.482544in}}{\pgfqpoint{1.335040in}{2.474643in}}{\pgfqpoint{1.335040in}{2.466407in}}%
\pgfpathcurveto{\pgfqpoint{1.335040in}{2.458171in}}{\pgfqpoint{1.338312in}{2.450271in}}{\pgfqpoint{1.344136in}{2.444447in}}%
\pgfpathcurveto{\pgfqpoint{1.349960in}{2.438623in}}{\pgfqpoint{1.357860in}{2.435351in}}{\pgfqpoint{1.366097in}{2.435351in}}%
\pgfpathclose%
\pgfusepath{stroke,fill}%
\end{pgfscope}%
\begin{pgfscope}%
\pgfpathrectangle{\pgfqpoint{0.100000in}{0.212622in}}{\pgfqpoint{3.696000in}{3.696000in}}%
\pgfusepath{clip}%
\pgfsetbuttcap%
\pgfsetroundjoin%
\definecolor{currentfill}{rgb}{0.121569,0.466667,0.705882}%
\pgfsetfillcolor{currentfill}%
\pgfsetfillopacity{0.462373}%
\pgfsetlinewidth{1.003750pt}%
\definecolor{currentstroke}{rgb}{0.121569,0.466667,0.705882}%
\pgfsetstrokecolor{currentstroke}%
\pgfsetstrokeopacity{0.462373}%
\pgfsetdash{}{0pt}%
\pgfpathmoveto{\pgfqpoint{1.363173in}{2.430006in}}%
\pgfpathcurveto{\pgfqpoint{1.371410in}{2.430006in}}{\pgfqpoint{1.379310in}{2.433278in}}{\pgfqpoint{1.385134in}{2.439102in}}%
\pgfpathcurveto{\pgfqpoint{1.390958in}{2.444926in}}{\pgfqpoint{1.394230in}{2.452826in}}{\pgfqpoint{1.394230in}{2.461062in}}%
\pgfpathcurveto{\pgfqpoint{1.394230in}{2.469299in}}{\pgfqpoint{1.390958in}{2.477199in}}{\pgfqpoint{1.385134in}{2.483023in}}%
\pgfpathcurveto{\pgfqpoint{1.379310in}{2.488847in}}{\pgfqpoint{1.371410in}{2.492119in}}{\pgfqpoint{1.363173in}{2.492119in}}%
\pgfpathcurveto{\pgfqpoint{1.354937in}{2.492119in}}{\pgfqpoint{1.347037in}{2.488847in}}{\pgfqpoint{1.341213in}{2.483023in}}%
\pgfpathcurveto{\pgfqpoint{1.335389in}{2.477199in}}{\pgfqpoint{1.332117in}{2.469299in}}{\pgfqpoint{1.332117in}{2.461062in}}%
\pgfpathcurveto{\pgfqpoint{1.332117in}{2.452826in}}{\pgfqpoint{1.335389in}{2.444926in}}{\pgfqpoint{1.341213in}{2.439102in}}%
\pgfpathcurveto{\pgfqpoint{1.347037in}{2.433278in}}{\pgfqpoint{1.354937in}{2.430006in}}{\pgfqpoint{1.363173in}{2.430006in}}%
\pgfpathclose%
\pgfusepath{stroke,fill}%
\end{pgfscope}%
\begin{pgfscope}%
\pgfpathrectangle{\pgfqpoint{0.100000in}{0.212622in}}{\pgfqpoint{3.696000in}{3.696000in}}%
\pgfusepath{clip}%
\pgfsetbuttcap%
\pgfsetroundjoin%
\definecolor{currentfill}{rgb}{0.121569,0.466667,0.705882}%
\pgfsetfillcolor{currentfill}%
\pgfsetfillopacity{0.462829}%
\pgfsetlinewidth{1.003750pt}%
\definecolor{currentstroke}{rgb}{0.121569,0.466667,0.705882}%
\pgfsetstrokecolor{currentstroke}%
\pgfsetstrokeopacity{0.462829}%
\pgfsetdash{}{0pt}%
\pgfpathmoveto{\pgfqpoint{1.361882in}{2.426768in}}%
\pgfpathcurveto{\pgfqpoint{1.370119in}{2.426768in}}{\pgfqpoint{1.378019in}{2.430040in}}{\pgfqpoint{1.383843in}{2.435864in}}%
\pgfpathcurveto{\pgfqpoint{1.389667in}{2.441688in}}{\pgfqpoint{1.392939in}{2.449588in}}{\pgfqpoint{1.392939in}{2.457824in}}%
\pgfpathcurveto{\pgfqpoint{1.392939in}{2.466060in}}{\pgfqpoint{1.389667in}{2.473960in}}{\pgfqpoint{1.383843in}{2.479784in}}%
\pgfpathcurveto{\pgfqpoint{1.378019in}{2.485608in}}{\pgfqpoint{1.370119in}{2.488881in}}{\pgfqpoint{1.361882in}{2.488881in}}%
\pgfpathcurveto{\pgfqpoint{1.353646in}{2.488881in}}{\pgfqpoint{1.345746in}{2.485608in}}{\pgfqpoint{1.339922in}{2.479784in}}%
\pgfpathcurveto{\pgfqpoint{1.334098in}{2.473960in}}{\pgfqpoint{1.330826in}{2.466060in}}{\pgfqpoint{1.330826in}{2.457824in}}%
\pgfpathcurveto{\pgfqpoint{1.330826in}{2.449588in}}{\pgfqpoint{1.334098in}{2.441688in}}{\pgfqpoint{1.339922in}{2.435864in}}%
\pgfpathcurveto{\pgfqpoint{1.345746in}{2.430040in}}{\pgfqpoint{1.353646in}{2.426768in}}{\pgfqpoint{1.361882in}{2.426768in}}%
\pgfpathclose%
\pgfusepath{stroke,fill}%
\end{pgfscope}%
\begin{pgfscope}%
\pgfpathrectangle{\pgfqpoint{0.100000in}{0.212622in}}{\pgfqpoint{3.696000in}{3.696000in}}%
\pgfusepath{clip}%
\pgfsetbuttcap%
\pgfsetroundjoin%
\definecolor{currentfill}{rgb}{0.121569,0.466667,0.705882}%
\pgfsetfillcolor{currentfill}%
\pgfsetfillopacity{0.463077}%
\pgfsetlinewidth{1.003750pt}%
\definecolor{currentstroke}{rgb}{0.121569,0.466667,0.705882}%
\pgfsetstrokecolor{currentstroke}%
\pgfsetstrokeopacity{0.463077}%
\pgfsetdash{}{0pt}%
\pgfpathmoveto{\pgfqpoint{2.017840in}{2.668458in}}%
\pgfpathcurveto{\pgfqpoint{2.026076in}{2.668458in}}{\pgfqpoint{2.033976in}{2.671730in}}{\pgfqpoint{2.039800in}{2.677554in}}%
\pgfpathcurveto{\pgfqpoint{2.045624in}{2.683378in}}{\pgfqpoint{2.048897in}{2.691278in}}{\pgfqpoint{2.048897in}{2.699515in}}%
\pgfpathcurveto{\pgfqpoint{2.048897in}{2.707751in}}{\pgfqpoint{2.045624in}{2.715651in}}{\pgfqpoint{2.039800in}{2.721475in}}%
\pgfpathcurveto{\pgfqpoint{2.033976in}{2.727299in}}{\pgfqpoint{2.026076in}{2.730571in}}{\pgfqpoint{2.017840in}{2.730571in}}%
\pgfpathcurveto{\pgfqpoint{2.009604in}{2.730571in}}{\pgfqpoint{2.001704in}{2.727299in}}{\pgfqpoint{1.995880in}{2.721475in}}%
\pgfpathcurveto{\pgfqpoint{1.990056in}{2.715651in}}{\pgfqpoint{1.986784in}{2.707751in}}{\pgfqpoint{1.986784in}{2.699515in}}%
\pgfpathcurveto{\pgfqpoint{1.986784in}{2.691278in}}{\pgfqpoint{1.990056in}{2.683378in}}{\pgfqpoint{1.995880in}{2.677554in}}%
\pgfpathcurveto{\pgfqpoint{2.001704in}{2.671730in}}{\pgfqpoint{2.009604in}{2.668458in}}{\pgfqpoint{2.017840in}{2.668458in}}%
\pgfpathclose%
\pgfusepath{stroke,fill}%
\end{pgfscope}%
\begin{pgfscope}%
\pgfpathrectangle{\pgfqpoint{0.100000in}{0.212622in}}{\pgfqpoint{3.696000in}{3.696000in}}%
\pgfusepath{clip}%
\pgfsetbuttcap%
\pgfsetroundjoin%
\definecolor{currentfill}{rgb}{0.121569,0.466667,0.705882}%
\pgfsetfillcolor{currentfill}%
\pgfsetfillopacity{0.463200}%
\pgfsetlinewidth{1.003750pt}%
\definecolor{currentstroke}{rgb}{0.121569,0.466667,0.705882}%
\pgfsetstrokecolor{currentstroke}%
\pgfsetstrokeopacity{0.463200}%
\pgfsetdash{}{0pt}%
\pgfpathmoveto{\pgfqpoint{1.360744in}{2.424864in}}%
\pgfpathcurveto{\pgfqpoint{1.368980in}{2.424864in}}{\pgfqpoint{1.376880in}{2.428136in}}{\pgfqpoint{1.382704in}{2.433960in}}%
\pgfpathcurveto{\pgfqpoint{1.388528in}{2.439784in}}{\pgfqpoint{1.391800in}{2.447684in}}{\pgfqpoint{1.391800in}{2.455920in}}%
\pgfpathcurveto{\pgfqpoint{1.391800in}{2.464156in}}{\pgfqpoint{1.388528in}{2.472057in}}{\pgfqpoint{1.382704in}{2.477880in}}%
\pgfpathcurveto{\pgfqpoint{1.376880in}{2.483704in}}{\pgfqpoint{1.368980in}{2.486977in}}{\pgfqpoint{1.360744in}{2.486977in}}%
\pgfpathcurveto{\pgfqpoint{1.352508in}{2.486977in}}{\pgfqpoint{1.344608in}{2.483704in}}{\pgfqpoint{1.338784in}{2.477880in}}%
\pgfpathcurveto{\pgfqpoint{1.332960in}{2.472057in}}{\pgfqpoint{1.329687in}{2.464156in}}{\pgfqpoint{1.329687in}{2.455920in}}%
\pgfpathcurveto{\pgfqpoint{1.329687in}{2.447684in}}{\pgfqpoint{1.332960in}{2.439784in}}{\pgfqpoint{1.338784in}{2.433960in}}%
\pgfpathcurveto{\pgfqpoint{1.344608in}{2.428136in}}{\pgfqpoint{1.352508in}{2.424864in}}{\pgfqpoint{1.360744in}{2.424864in}}%
\pgfpathclose%
\pgfusepath{stroke,fill}%
\end{pgfscope}%
\begin{pgfscope}%
\pgfpathrectangle{\pgfqpoint{0.100000in}{0.212622in}}{\pgfqpoint{3.696000in}{3.696000in}}%
\pgfusepath{clip}%
\pgfsetbuttcap%
\pgfsetroundjoin%
\definecolor{currentfill}{rgb}{0.121569,0.466667,0.705882}%
\pgfsetfillcolor{currentfill}%
\pgfsetfillopacity{0.463913}%
\pgfsetlinewidth{1.003750pt}%
\definecolor{currentstroke}{rgb}{0.121569,0.466667,0.705882}%
\pgfsetstrokecolor{currentstroke}%
\pgfsetstrokeopacity{0.463913}%
\pgfsetdash{}{0pt}%
\pgfpathmoveto{\pgfqpoint{1.358777in}{2.421410in}}%
\pgfpathcurveto{\pgfqpoint{1.367013in}{2.421410in}}{\pgfqpoint{1.374914in}{2.424683in}}{\pgfqpoint{1.380737in}{2.430507in}}%
\pgfpathcurveto{\pgfqpoint{1.386561in}{2.436331in}}{\pgfqpoint{1.389834in}{2.444231in}}{\pgfqpoint{1.389834in}{2.452467in}}%
\pgfpathcurveto{\pgfqpoint{1.389834in}{2.460703in}}{\pgfqpoint{1.386561in}{2.468603in}}{\pgfqpoint{1.380737in}{2.474427in}}%
\pgfpathcurveto{\pgfqpoint{1.374914in}{2.480251in}}{\pgfqpoint{1.367013in}{2.483523in}}{\pgfqpoint{1.358777in}{2.483523in}}%
\pgfpathcurveto{\pgfqpoint{1.350541in}{2.483523in}}{\pgfqpoint{1.342641in}{2.480251in}}{\pgfqpoint{1.336817in}{2.474427in}}%
\pgfpathcurveto{\pgfqpoint{1.330993in}{2.468603in}}{\pgfqpoint{1.327721in}{2.460703in}}{\pgfqpoint{1.327721in}{2.452467in}}%
\pgfpathcurveto{\pgfqpoint{1.327721in}{2.444231in}}{\pgfqpoint{1.330993in}{2.436331in}}{\pgfqpoint{1.336817in}{2.430507in}}%
\pgfpathcurveto{\pgfqpoint{1.342641in}{2.424683in}}{\pgfqpoint{1.350541in}{2.421410in}}{\pgfqpoint{1.358777in}{2.421410in}}%
\pgfpathclose%
\pgfusepath{stroke,fill}%
\end{pgfscope}%
\begin{pgfscope}%
\pgfpathrectangle{\pgfqpoint{0.100000in}{0.212622in}}{\pgfqpoint{3.696000in}{3.696000in}}%
\pgfusepath{clip}%
\pgfsetbuttcap%
\pgfsetroundjoin%
\definecolor{currentfill}{rgb}{0.121569,0.466667,0.705882}%
\pgfsetfillcolor{currentfill}%
\pgfsetfillopacity{0.464529}%
\pgfsetlinewidth{1.003750pt}%
\definecolor{currentstroke}{rgb}{0.121569,0.466667,0.705882}%
\pgfsetstrokecolor{currentstroke}%
\pgfsetstrokeopacity{0.464529}%
\pgfsetdash{}{0pt}%
\pgfpathmoveto{\pgfqpoint{1.357086in}{2.418483in}}%
\pgfpathcurveto{\pgfqpoint{1.365323in}{2.418483in}}{\pgfqpoint{1.373223in}{2.421755in}}{\pgfqpoint{1.379047in}{2.427579in}}%
\pgfpathcurveto{\pgfqpoint{1.384870in}{2.433403in}}{\pgfqpoint{1.388143in}{2.441303in}}{\pgfqpoint{1.388143in}{2.449540in}}%
\pgfpathcurveto{\pgfqpoint{1.388143in}{2.457776in}}{\pgfqpoint{1.384870in}{2.465676in}}{\pgfqpoint{1.379047in}{2.471500in}}%
\pgfpathcurveto{\pgfqpoint{1.373223in}{2.477324in}}{\pgfqpoint{1.365323in}{2.480596in}}{\pgfqpoint{1.357086in}{2.480596in}}%
\pgfpathcurveto{\pgfqpoint{1.348850in}{2.480596in}}{\pgfqpoint{1.340950in}{2.477324in}}{\pgfqpoint{1.335126in}{2.471500in}}%
\pgfpathcurveto{\pgfqpoint{1.329302in}{2.465676in}}{\pgfqpoint{1.326030in}{2.457776in}}{\pgfqpoint{1.326030in}{2.449540in}}%
\pgfpathcurveto{\pgfqpoint{1.326030in}{2.441303in}}{\pgfqpoint{1.329302in}{2.433403in}}{\pgfqpoint{1.335126in}{2.427579in}}%
\pgfpathcurveto{\pgfqpoint{1.340950in}{2.421755in}}{\pgfqpoint{1.348850in}{2.418483in}}{\pgfqpoint{1.357086in}{2.418483in}}%
\pgfpathclose%
\pgfusepath{stroke,fill}%
\end{pgfscope}%
\begin{pgfscope}%
\pgfpathrectangle{\pgfqpoint{0.100000in}{0.212622in}}{\pgfqpoint{3.696000in}{3.696000in}}%
\pgfusepath{clip}%
\pgfsetbuttcap%
\pgfsetroundjoin%
\definecolor{currentfill}{rgb}{0.121569,0.466667,0.705882}%
\pgfsetfillcolor{currentfill}%
\pgfsetfillopacity{0.465035}%
\pgfsetlinewidth{1.003750pt}%
\definecolor{currentstroke}{rgb}{0.121569,0.466667,0.705882}%
\pgfsetstrokecolor{currentstroke}%
\pgfsetstrokeopacity{0.465035}%
\pgfsetdash{}{0pt}%
\pgfpathmoveto{\pgfqpoint{1.355443in}{2.415833in}}%
\pgfpathcurveto{\pgfqpoint{1.363679in}{2.415833in}}{\pgfqpoint{1.371579in}{2.419106in}}{\pgfqpoint{1.377403in}{2.424930in}}%
\pgfpathcurveto{\pgfqpoint{1.383227in}{2.430754in}}{\pgfqpoint{1.386500in}{2.438654in}}{\pgfqpoint{1.386500in}{2.446890in}}%
\pgfpathcurveto{\pgfqpoint{1.386500in}{2.455126in}}{\pgfqpoint{1.383227in}{2.463026in}}{\pgfqpoint{1.377403in}{2.468850in}}%
\pgfpathcurveto{\pgfqpoint{1.371579in}{2.474674in}}{\pgfqpoint{1.363679in}{2.477946in}}{\pgfqpoint{1.355443in}{2.477946in}}%
\pgfpathcurveto{\pgfqpoint{1.347207in}{2.477946in}}{\pgfqpoint{1.339307in}{2.474674in}}{\pgfqpoint{1.333483in}{2.468850in}}%
\pgfpathcurveto{\pgfqpoint{1.327659in}{2.463026in}}{\pgfqpoint{1.324387in}{2.455126in}}{\pgfqpoint{1.324387in}{2.446890in}}%
\pgfpathcurveto{\pgfqpoint{1.324387in}{2.438654in}}{\pgfqpoint{1.327659in}{2.430754in}}{\pgfqpoint{1.333483in}{2.424930in}}%
\pgfpathcurveto{\pgfqpoint{1.339307in}{2.419106in}}{\pgfqpoint{1.347207in}{2.415833in}}{\pgfqpoint{1.355443in}{2.415833in}}%
\pgfpathclose%
\pgfusepath{stroke,fill}%
\end{pgfscope}%
\begin{pgfscope}%
\pgfpathrectangle{\pgfqpoint{0.100000in}{0.212622in}}{\pgfqpoint{3.696000in}{3.696000in}}%
\pgfusepath{clip}%
\pgfsetbuttcap%
\pgfsetroundjoin%
\definecolor{currentfill}{rgb}{0.121569,0.466667,0.705882}%
\pgfsetfillcolor{currentfill}%
\pgfsetfillopacity{0.465212}%
\pgfsetlinewidth{1.003750pt}%
\definecolor{currentstroke}{rgb}{0.121569,0.466667,0.705882}%
\pgfsetstrokecolor{currentstroke}%
\pgfsetstrokeopacity{0.465212}%
\pgfsetdash{}{0pt}%
\pgfpathmoveto{\pgfqpoint{2.019240in}{2.659588in}}%
\pgfpathcurveto{\pgfqpoint{2.027476in}{2.659588in}}{\pgfqpoint{2.035376in}{2.662860in}}{\pgfqpoint{2.041200in}{2.668684in}}%
\pgfpathcurveto{\pgfqpoint{2.047024in}{2.674508in}}{\pgfqpoint{2.050296in}{2.682408in}}{\pgfqpoint{2.050296in}{2.690644in}}%
\pgfpathcurveto{\pgfqpoint{2.050296in}{2.698880in}}{\pgfqpoint{2.047024in}{2.706781in}}{\pgfqpoint{2.041200in}{2.712604in}}%
\pgfpathcurveto{\pgfqpoint{2.035376in}{2.718428in}}{\pgfqpoint{2.027476in}{2.721701in}}{\pgfqpoint{2.019240in}{2.721701in}}%
\pgfpathcurveto{\pgfqpoint{2.011004in}{2.721701in}}{\pgfqpoint{2.003104in}{2.718428in}}{\pgfqpoint{1.997280in}{2.712604in}}%
\pgfpathcurveto{\pgfqpoint{1.991456in}{2.706781in}}{\pgfqpoint{1.988183in}{2.698880in}}{\pgfqpoint{1.988183in}{2.690644in}}%
\pgfpathcurveto{\pgfqpoint{1.988183in}{2.682408in}}{\pgfqpoint{1.991456in}{2.674508in}}{\pgfqpoint{1.997280in}{2.668684in}}%
\pgfpathcurveto{\pgfqpoint{2.003104in}{2.662860in}}{\pgfqpoint{2.011004in}{2.659588in}}{\pgfqpoint{2.019240in}{2.659588in}}%
\pgfpathclose%
\pgfusepath{stroke,fill}%
\end{pgfscope}%
\begin{pgfscope}%
\pgfpathrectangle{\pgfqpoint{0.100000in}{0.212622in}}{\pgfqpoint{3.696000in}{3.696000in}}%
\pgfusepath{clip}%
\pgfsetbuttcap%
\pgfsetroundjoin%
\definecolor{currentfill}{rgb}{0.121569,0.466667,0.705882}%
\pgfsetfillcolor{currentfill}%
\pgfsetfillopacity{0.466006}%
\pgfsetlinewidth{1.003750pt}%
\definecolor{currentstroke}{rgb}{0.121569,0.466667,0.705882}%
\pgfsetstrokecolor{currentstroke}%
\pgfsetstrokeopacity{0.466006}%
\pgfsetdash{}{0pt}%
\pgfpathmoveto{\pgfqpoint{1.352902in}{2.410625in}}%
\pgfpathcurveto{\pgfqpoint{1.361139in}{2.410625in}}{\pgfqpoint{1.369039in}{2.413897in}}{\pgfqpoint{1.374863in}{2.419721in}}%
\pgfpathcurveto{\pgfqpoint{1.380686in}{2.425545in}}{\pgfqpoint{1.383959in}{2.433445in}}{\pgfqpoint{1.383959in}{2.441681in}}%
\pgfpathcurveto{\pgfqpoint{1.383959in}{2.449917in}}{\pgfqpoint{1.380686in}{2.457817in}}{\pgfqpoint{1.374863in}{2.463641in}}%
\pgfpathcurveto{\pgfqpoint{1.369039in}{2.469465in}}{\pgfqpoint{1.361139in}{2.472738in}}{\pgfqpoint{1.352902in}{2.472738in}}%
\pgfpathcurveto{\pgfqpoint{1.344666in}{2.472738in}}{\pgfqpoint{1.336766in}{2.469465in}}{\pgfqpoint{1.330942in}{2.463641in}}%
\pgfpathcurveto{\pgfqpoint{1.325118in}{2.457817in}}{\pgfqpoint{1.321846in}{2.449917in}}{\pgfqpoint{1.321846in}{2.441681in}}%
\pgfpathcurveto{\pgfqpoint{1.321846in}{2.433445in}}{\pgfqpoint{1.325118in}{2.425545in}}{\pgfqpoint{1.330942in}{2.419721in}}%
\pgfpathcurveto{\pgfqpoint{1.336766in}{2.413897in}}{\pgfqpoint{1.344666in}{2.410625in}}{\pgfqpoint{1.352902in}{2.410625in}}%
\pgfpathclose%
\pgfusepath{stroke,fill}%
\end{pgfscope}%
\begin{pgfscope}%
\pgfpathrectangle{\pgfqpoint{0.100000in}{0.212622in}}{\pgfqpoint{3.696000in}{3.696000in}}%
\pgfusepath{clip}%
\pgfsetbuttcap%
\pgfsetroundjoin%
\definecolor{currentfill}{rgb}{0.121569,0.466667,0.705882}%
\pgfsetfillcolor{currentfill}%
\pgfsetfillopacity{0.466453}%
\pgfsetlinewidth{1.003750pt}%
\definecolor{currentstroke}{rgb}{0.121569,0.466667,0.705882}%
\pgfsetstrokecolor{currentstroke}%
\pgfsetstrokeopacity{0.466453}%
\pgfsetdash{}{0pt}%
\pgfpathmoveto{\pgfqpoint{1.351482in}{2.408270in}}%
\pgfpathcurveto{\pgfqpoint{1.359718in}{2.408270in}}{\pgfqpoint{1.367618in}{2.411542in}}{\pgfqpoint{1.373442in}{2.417366in}}%
\pgfpathcurveto{\pgfqpoint{1.379266in}{2.423190in}}{\pgfqpoint{1.382538in}{2.431090in}}{\pgfqpoint{1.382538in}{2.439326in}}%
\pgfpathcurveto{\pgfqpoint{1.382538in}{2.447562in}}{\pgfqpoint{1.379266in}{2.455462in}}{\pgfqpoint{1.373442in}{2.461286in}}%
\pgfpathcurveto{\pgfqpoint{1.367618in}{2.467110in}}{\pgfqpoint{1.359718in}{2.470383in}}{\pgfqpoint{1.351482in}{2.470383in}}%
\pgfpathcurveto{\pgfqpoint{1.343246in}{2.470383in}}{\pgfqpoint{1.335346in}{2.467110in}}{\pgfqpoint{1.329522in}{2.461286in}}%
\pgfpathcurveto{\pgfqpoint{1.323698in}{2.455462in}}{\pgfqpoint{1.320425in}{2.447562in}}{\pgfqpoint{1.320425in}{2.439326in}}%
\pgfpathcurveto{\pgfqpoint{1.320425in}{2.431090in}}{\pgfqpoint{1.323698in}{2.423190in}}{\pgfqpoint{1.329522in}{2.417366in}}%
\pgfpathcurveto{\pgfqpoint{1.335346in}{2.411542in}}{\pgfqpoint{1.343246in}{2.408270in}}{\pgfqpoint{1.351482in}{2.408270in}}%
\pgfpathclose%
\pgfusepath{stroke,fill}%
\end{pgfscope}%
\begin{pgfscope}%
\pgfpathrectangle{\pgfqpoint{0.100000in}{0.212622in}}{\pgfqpoint{3.696000in}{3.696000in}}%
\pgfusepath{clip}%
\pgfsetbuttcap%
\pgfsetroundjoin%
\definecolor{currentfill}{rgb}{0.121569,0.466667,0.705882}%
\pgfsetfillcolor{currentfill}%
\pgfsetfillopacity{0.466817}%
\pgfsetlinewidth{1.003750pt}%
\definecolor{currentstroke}{rgb}{0.121569,0.466667,0.705882}%
\pgfsetstrokecolor{currentstroke}%
\pgfsetstrokeopacity{0.466817}%
\pgfsetdash{}{0pt}%
\pgfpathmoveto{\pgfqpoint{1.350489in}{2.406279in}}%
\pgfpathcurveto{\pgfqpoint{1.358726in}{2.406279in}}{\pgfqpoint{1.366626in}{2.409552in}}{\pgfqpoint{1.372450in}{2.415376in}}%
\pgfpathcurveto{\pgfqpoint{1.378273in}{2.421200in}}{\pgfqpoint{1.381546in}{2.429100in}}{\pgfqpoint{1.381546in}{2.437336in}}%
\pgfpathcurveto{\pgfqpoint{1.381546in}{2.445572in}}{\pgfqpoint{1.378273in}{2.453472in}}{\pgfqpoint{1.372450in}{2.459296in}}%
\pgfpathcurveto{\pgfqpoint{1.366626in}{2.465120in}}{\pgfqpoint{1.358726in}{2.468392in}}{\pgfqpoint{1.350489in}{2.468392in}}%
\pgfpathcurveto{\pgfqpoint{1.342253in}{2.468392in}}{\pgfqpoint{1.334353in}{2.465120in}}{\pgfqpoint{1.328529in}{2.459296in}}%
\pgfpathcurveto{\pgfqpoint{1.322705in}{2.453472in}}{\pgfqpoint{1.319433in}{2.445572in}}{\pgfqpoint{1.319433in}{2.437336in}}%
\pgfpathcurveto{\pgfqpoint{1.319433in}{2.429100in}}{\pgfqpoint{1.322705in}{2.421200in}}{\pgfqpoint{1.328529in}{2.415376in}}%
\pgfpathcurveto{\pgfqpoint{1.334353in}{2.409552in}}{\pgfqpoint{1.342253in}{2.406279in}}{\pgfqpoint{1.350489in}{2.406279in}}%
\pgfpathclose%
\pgfusepath{stroke,fill}%
\end{pgfscope}%
\begin{pgfscope}%
\pgfpathrectangle{\pgfqpoint{0.100000in}{0.212622in}}{\pgfqpoint{3.696000in}{3.696000in}}%
\pgfusepath{clip}%
\pgfsetbuttcap%
\pgfsetroundjoin%
\definecolor{currentfill}{rgb}{0.121569,0.466667,0.705882}%
\pgfsetfillcolor{currentfill}%
\pgfsetfillopacity{0.467500}%
\pgfsetlinewidth{1.003750pt}%
\definecolor{currentstroke}{rgb}{0.121569,0.466667,0.705882}%
\pgfsetstrokecolor{currentstroke}%
\pgfsetstrokeopacity{0.467500}%
\pgfsetdash{}{0pt}%
\pgfpathmoveto{\pgfqpoint{1.348450in}{2.403044in}}%
\pgfpathcurveto{\pgfqpoint{1.356686in}{2.403044in}}{\pgfqpoint{1.364586in}{2.406316in}}{\pgfqpoint{1.370410in}{2.412140in}}%
\pgfpathcurveto{\pgfqpoint{1.376234in}{2.417964in}}{\pgfqpoint{1.379506in}{2.425864in}}{\pgfqpoint{1.379506in}{2.434101in}}%
\pgfpathcurveto{\pgfqpoint{1.379506in}{2.442337in}}{\pgfqpoint{1.376234in}{2.450237in}}{\pgfqpoint{1.370410in}{2.456061in}}%
\pgfpathcurveto{\pgfqpoint{1.364586in}{2.461885in}}{\pgfqpoint{1.356686in}{2.465157in}}{\pgfqpoint{1.348450in}{2.465157in}}%
\pgfpathcurveto{\pgfqpoint{1.340213in}{2.465157in}}{\pgfqpoint{1.332313in}{2.461885in}}{\pgfqpoint{1.326489in}{2.456061in}}%
\pgfpathcurveto{\pgfqpoint{1.320665in}{2.450237in}}{\pgfqpoint{1.317393in}{2.442337in}}{\pgfqpoint{1.317393in}{2.434101in}}%
\pgfpathcurveto{\pgfqpoint{1.317393in}{2.425864in}}{\pgfqpoint{1.320665in}{2.417964in}}{\pgfqpoint{1.326489in}{2.412140in}}%
\pgfpathcurveto{\pgfqpoint{1.332313in}{2.406316in}}{\pgfqpoint{1.340213in}{2.403044in}}{\pgfqpoint{1.348450in}{2.403044in}}%
\pgfpathclose%
\pgfusepath{stroke,fill}%
\end{pgfscope}%
\begin{pgfscope}%
\pgfpathrectangle{\pgfqpoint{0.100000in}{0.212622in}}{\pgfqpoint{3.696000in}{3.696000in}}%
\pgfusepath{clip}%
\pgfsetbuttcap%
\pgfsetroundjoin%
\definecolor{currentfill}{rgb}{0.121569,0.466667,0.705882}%
\pgfsetfillcolor{currentfill}%
\pgfsetfillopacity{0.467751}%
\pgfsetlinewidth{1.003750pt}%
\definecolor{currentstroke}{rgb}{0.121569,0.466667,0.705882}%
\pgfsetstrokecolor{currentstroke}%
\pgfsetstrokeopacity{0.467751}%
\pgfsetdash{}{0pt}%
\pgfpathmoveto{\pgfqpoint{2.020497in}{2.650151in}}%
\pgfpathcurveto{\pgfqpoint{2.028733in}{2.650151in}}{\pgfqpoint{2.036633in}{2.653423in}}{\pgfqpoint{2.042457in}{2.659247in}}%
\pgfpathcurveto{\pgfqpoint{2.048281in}{2.665071in}}{\pgfqpoint{2.051553in}{2.672971in}}{\pgfqpoint{2.051553in}{2.681207in}}%
\pgfpathcurveto{\pgfqpoint{2.051553in}{2.689444in}}{\pgfqpoint{2.048281in}{2.697344in}}{\pgfqpoint{2.042457in}{2.703168in}}%
\pgfpathcurveto{\pgfqpoint{2.036633in}{2.708991in}}{\pgfqpoint{2.028733in}{2.712264in}}{\pgfqpoint{2.020497in}{2.712264in}}%
\pgfpathcurveto{\pgfqpoint{2.012261in}{2.712264in}}{\pgfqpoint{2.004361in}{2.708991in}}{\pgfqpoint{1.998537in}{2.703168in}}%
\pgfpathcurveto{\pgfqpoint{1.992713in}{2.697344in}}{\pgfqpoint{1.989440in}{2.689444in}}{\pgfqpoint{1.989440in}{2.681207in}}%
\pgfpathcurveto{\pgfqpoint{1.989440in}{2.672971in}}{\pgfqpoint{1.992713in}{2.665071in}}{\pgfqpoint{1.998537in}{2.659247in}}%
\pgfpathcurveto{\pgfqpoint{2.004361in}{2.653423in}}{\pgfqpoint{2.012261in}{2.650151in}}{\pgfqpoint{2.020497in}{2.650151in}}%
\pgfpathclose%
\pgfusepath{stroke,fill}%
\end{pgfscope}%
\begin{pgfscope}%
\pgfpathrectangle{\pgfqpoint{0.100000in}{0.212622in}}{\pgfqpoint{3.696000in}{3.696000in}}%
\pgfusepath{clip}%
\pgfsetbuttcap%
\pgfsetroundjoin%
\definecolor{currentfill}{rgb}{0.121569,0.466667,0.705882}%
\pgfsetfillcolor{currentfill}%
\pgfsetfillopacity{0.467880}%
\pgfsetlinewidth{1.003750pt}%
\definecolor{currentstroke}{rgb}{0.121569,0.466667,0.705882}%
\pgfsetstrokecolor{currentstroke}%
\pgfsetstrokeopacity{0.467880}%
\pgfsetdash{}{0pt}%
\pgfpathmoveto{\pgfqpoint{1.347759in}{2.401353in}}%
\pgfpathcurveto{\pgfqpoint{1.355996in}{2.401353in}}{\pgfqpoint{1.363896in}{2.404625in}}{\pgfqpoint{1.369720in}{2.410449in}}%
\pgfpathcurveto{\pgfqpoint{1.375544in}{2.416273in}}{\pgfqpoint{1.378816in}{2.424173in}}{\pgfqpoint{1.378816in}{2.432409in}}%
\pgfpathcurveto{\pgfqpoint{1.378816in}{2.440646in}}{\pgfqpoint{1.375544in}{2.448546in}}{\pgfqpoint{1.369720in}{2.454370in}}%
\pgfpathcurveto{\pgfqpoint{1.363896in}{2.460193in}}{\pgfqpoint{1.355996in}{2.463466in}}{\pgfqpoint{1.347759in}{2.463466in}}%
\pgfpathcurveto{\pgfqpoint{1.339523in}{2.463466in}}{\pgfqpoint{1.331623in}{2.460193in}}{\pgfqpoint{1.325799in}{2.454370in}}%
\pgfpathcurveto{\pgfqpoint{1.319975in}{2.448546in}}{\pgfqpoint{1.316703in}{2.440646in}}{\pgfqpoint{1.316703in}{2.432409in}}%
\pgfpathcurveto{\pgfqpoint{1.316703in}{2.424173in}}{\pgfqpoint{1.319975in}{2.416273in}}{\pgfqpoint{1.325799in}{2.410449in}}%
\pgfpathcurveto{\pgfqpoint{1.331623in}{2.404625in}}{\pgfqpoint{1.339523in}{2.401353in}}{\pgfqpoint{1.347759in}{2.401353in}}%
\pgfpathclose%
\pgfusepath{stroke,fill}%
\end{pgfscope}%
\begin{pgfscope}%
\pgfpathrectangle{\pgfqpoint{0.100000in}{0.212622in}}{\pgfqpoint{3.696000in}{3.696000in}}%
\pgfusepath{clip}%
\pgfsetbuttcap%
\pgfsetroundjoin%
\definecolor{currentfill}{rgb}{0.121569,0.466667,0.705882}%
\pgfsetfillcolor{currentfill}%
\pgfsetfillopacity{0.468486}%
\pgfsetlinewidth{1.003750pt}%
\definecolor{currentstroke}{rgb}{0.121569,0.466667,0.705882}%
\pgfsetstrokecolor{currentstroke}%
\pgfsetstrokeopacity{0.468486}%
\pgfsetdash{}{0pt}%
\pgfpathmoveto{\pgfqpoint{1.346039in}{2.398397in}}%
\pgfpathcurveto{\pgfqpoint{1.354275in}{2.398397in}}{\pgfqpoint{1.362175in}{2.401670in}}{\pgfqpoint{1.367999in}{2.407494in}}%
\pgfpathcurveto{\pgfqpoint{1.373823in}{2.413318in}}{\pgfqpoint{1.377096in}{2.421218in}}{\pgfqpoint{1.377096in}{2.429454in}}%
\pgfpathcurveto{\pgfqpoint{1.377096in}{2.437690in}}{\pgfqpoint{1.373823in}{2.445590in}}{\pgfqpoint{1.367999in}{2.451414in}}%
\pgfpathcurveto{\pgfqpoint{1.362175in}{2.457238in}}{\pgfqpoint{1.354275in}{2.460510in}}{\pgfqpoint{1.346039in}{2.460510in}}%
\pgfpathcurveto{\pgfqpoint{1.337803in}{2.460510in}}{\pgfqpoint{1.329903in}{2.457238in}}{\pgfqpoint{1.324079in}{2.451414in}}%
\pgfpathcurveto{\pgfqpoint{1.318255in}{2.445590in}}{\pgfqpoint{1.314983in}{2.437690in}}{\pgfqpoint{1.314983in}{2.429454in}}%
\pgfpathcurveto{\pgfqpoint{1.314983in}{2.421218in}}{\pgfqpoint{1.318255in}{2.413318in}}{\pgfqpoint{1.324079in}{2.407494in}}%
\pgfpathcurveto{\pgfqpoint{1.329903in}{2.401670in}}{\pgfqpoint{1.337803in}{2.398397in}}{\pgfqpoint{1.346039in}{2.398397in}}%
\pgfpathclose%
\pgfusepath{stroke,fill}%
\end{pgfscope}%
\begin{pgfscope}%
\pgfpathrectangle{\pgfqpoint{0.100000in}{0.212622in}}{\pgfqpoint{3.696000in}{3.696000in}}%
\pgfusepath{clip}%
\pgfsetbuttcap%
\pgfsetroundjoin%
\definecolor{currentfill}{rgb}{0.121569,0.466667,0.705882}%
\pgfsetfillcolor{currentfill}%
\pgfsetfillopacity{0.469640}%
\pgfsetlinewidth{1.003750pt}%
\definecolor{currentstroke}{rgb}{0.121569,0.466667,0.705882}%
\pgfsetstrokecolor{currentstroke}%
\pgfsetstrokeopacity{0.469640}%
\pgfsetdash{}{0pt}%
\pgfpathmoveto{\pgfqpoint{1.343180in}{2.392892in}}%
\pgfpathcurveto{\pgfqpoint{1.351416in}{2.392892in}}{\pgfqpoint{1.359316in}{2.396164in}}{\pgfqpoint{1.365140in}{2.401988in}}%
\pgfpathcurveto{\pgfqpoint{1.370964in}{2.407812in}}{\pgfqpoint{1.374236in}{2.415712in}}{\pgfqpoint{1.374236in}{2.423948in}}%
\pgfpathcurveto{\pgfqpoint{1.374236in}{2.432185in}}{\pgfqpoint{1.370964in}{2.440085in}}{\pgfqpoint{1.365140in}{2.445909in}}%
\pgfpathcurveto{\pgfqpoint{1.359316in}{2.451733in}}{\pgfqpoint{1.351416in}{2.455005in}}{\pgfqpoint{1.343180in}{2.455005in}}%
\pgfpathcurveto{\pgfqpoint{1.334944in}{2.455005in}}{\pgfqpoint{1.327043in}{2.451733in}}{\pgfqpoint{1.321220in}{2.445909in}}%
\pgfpathcurveto{\pgfqpoint{1.315396in}{2.440085in}}{\pgfqpoint{1.312123in}{2.432185in}}{\pgfqpoint{1.312123in}{2.423948in}}%
\pgfpathcurveto{\pgfqpoint{1.312123in}{2.415712in}}{\pgfqpoint{1.315396in}{2.407812in}}{\pgfqpoint{1.321220in}{2.401988in}}%
\pgfpathcurveto{\pgfqpoint{1.327043in}{2.396164in}}{\pgfqpoint{1.334944in}{2.392892in}}{\pgfqpoint{1.343180in}{2.392892in}}%
\pgfpathclose%
\pgfusepath{stroke,fill}%
\end{pgfscope}%
\begin{pgfscope}%
\pgfpathrectangle{\pgfqpoint{0.100000in}{0.212622in}}{\pgfqpoint{3.696000in}{3.696000in}}%
\pgfusepath{clip}%
\pgfsetbuttcap%
\pgfsetroundjoin%
\definecolor{currentfill}{rgb}{0.121569,0.466667,0.705882}%
\pgfsetfillcolor{currentfill}%
\pgfsetfillopacity{0.470376}%
\pgfsetlinewidth{1.003750pt}%
\definecolor{currentstroke}{rgb}{0.121569,0.466667,0.705882}%
\pgfsetstrokecolor{currentstroke}%
\pgfsetstrokeopacity{0.470376}%
\pgfsetdash{}{0pt}%
\pgfpathmoveto{\pgfqpoint{2.022652in}{2.638684in}}%
\pgfpathcurveto{\pgfqpoint{2.030888in}{2.638684in}}{\pgfqpoint{2.038788in}{2.641957in}}{\pgfqpoint{2.044612in}{2.647781in}}%
\pgfpathcurveto{\pgfqpoint{2.050436in}{2.653605in}}{\pgfqpoint{2.053708in}{2.661505in}}{\pgfqpoint{2.053708in}{2.669741in}}%
\pgfpathcurveto{\pgfqpoint{2.053708in}{2.677977in}}{\pgfqpoint{2.050436in}{2.685877in}}{\pgfqpoint{2.044612in}{2.691701in}}%
\pgfpathcurveto{\pgfqpoint{2.038788in}{2.697525in}}{\pgfqpoint{2.030888in}{2.700797in}}{\pgfqpoint{2.022652in}{2.700797in}}%
\pgfpathcurveto{\pgfqpoint{2.014416in}{2.700797in}}{\pgfqpoint{2.006516in}{2.697525in}}{\pgfqpoint{2.000692in}{2.691701in}}%
\pgfpathcurveto{\pgfqpoint{1.994868in}{2.685877in}}{\pgfqpoint{1.991595in}{2.677977in}}{\pgfqpoint{1.991595in}{2.669741in}}%
\pgfpathcurveto{\pgfqpoint{1.991595in}{2.661505in}}{\pgfqpoint{1.994868in}{2.653605in}}{\pgfqpoint{2.000692in}{2.647781in}}%
\pgfpathcurveto{\pgfqpoint{2.006516in}{2.641957in}}{\pgfqpoint{2.014416in}{2.638684in}}{\pgfqpoint{2.022652in}{2.638684in}}%
\pgfpathclose%
\pgfusepath{stroke,fill}%
\end{pgfscope}%
\begin{pgfscope}%
\pgfpathrectangle{\pgfqpoint{0.100000in}{0.212622in}}{\pgfqpoint{3.696000in}{3.696000in}}%
\pgfusepath{clip}%
\pgfsetbuttcap%
\pgfsetroundjoin%
\definecolor{currentfill}{rgb}{0.121569,0.466667,0.705882}%
\pgfsetfillcolor{currentfill}%
\pgfsetfillopacity{0.470709}%
\pgfsetlinewidth{1.003750pt}%
\definecolor{currentstroke}{rgb}{0.121569,0.466667,0.705882}%
\pgfsetstrokecolor{currentstroke}%
\pgfsetstrokeopacity{0.470709}%
\pgfsetdash{}{0pt}%
\pgfpathmoveto{\pgfqpoint{1.340494in}{2.388200in}}%
\pgfpathcurveto{\pgfqpoint{1.348731in}{2.388200in}}{\pgfqpoint{1.356631in}{2.391472in}}{\pgfqpoint{1.362455in}{2.397296in}}%
\pgfpathcurveto{\pgfqpoint{1.368279in}{2.403120in}}{\pgfqpoint{1.371551in}{2.411020in}}{\pgfqpoint{1.371551in}{2.419257in}}%
\pgfpathcurveto{\pgfqpoint{1.371551in}{2.427493in}}{\pgfqpoint{1.368279in}{2.435393in}}{\pgfqpoint{1.362455in}{2.441217in}}%
\pgfpathcurveto{\pgfqpoint{1.356631in}{2.447041in}}{\pgfqpoint{1.348731in}{2.450313in}}{\pgfqpoint{1.340494in}{2.450313in}}%
\pgfpathcurveto{\pgfqpoint{1.332258in}{2.450313in}}{\pgfqpoint{1.324358in}{2.447041in}}{\pgfqpoint{1.318534in}{2.441217in}}%
\pgfpathcurveto{\pgfqpoint{1.312710in}{2.435393in}}{\pgfqpoint{1.309438in}{2.427493in}}{\pgfqpoint{1.309438in}{2.419257in}}%
\pgfpathcurveto{\pgfqpoint{1.309438in}{2.411020in}}{\pgfqpoint{1.312710in}{2.403120in}}{\pgfqpoint{1.318534in}{2.397296in}}%
\pgfpathcurveto{\pgfqpoint{1.324358in}{2.391472in}}{\pgfqpoint{1.332258in}{2.388200in}}{\pgfqpoint{1.340494in}{2.388200in}}%
\pgfpathclose%
\pgfusepath{stroke,fill}%
\end{pgfscope}%
\begin{pgfscope}%
\pgfpathrectangle{\pgfqpoint{0.100000in}{0.212622in}}{\pgfqpoint{3.696000in}{3.696000in}}%
\pgfusepath{clip}%
\pgfsetbuttcap%
\pgfsetroundjoin%
\definecolor{currentfill}{rgb}{0.121569,0.466667,0.705882}%
\pgfsetfillcolor{currentfill}%
\pgfsetfillopacity{0.472653}%
\pgfsetlinewidth{1.003750pt}%
\definecolor{currentstroke}{rgb}{0.121569,0.466667,0.705882}%
\pgfsetstrokecolor{currentstroke}%
\pgfsetstrokeopacity{0.472653}%
\pgfsetdash{}{0pt}%
\pgfpathmoveto{\pgfqpoint{1.335262in}{2.380115in}}%
\pgfpathcurveto{\pgfqpoint{1.343498in}{2.380115in}}{\pgfqpoint{1.351399in}{2.383387in}}{\pgfqpoint{1.357222in}{2.389211in}}%
\pgfpathcurveto{\pgfqpoint{1.363046in}{2.395035in}}{\pgfqpoint{1.366319in}{2.402935in}}{\pgfqpoint{1.366319in}{2.411171in}}%
\pgfpathcurveto{\pgfqpoint{1.366319in}{2.419408in}}{\pgfqpoint{1.363046in}{2.427308in}}{\pgfqpoint{1.357222in}{2.433131in}}%
\pgfpathcurveto{\pgfqpoint{1.351399in}{2.438955in}}{\pgfqpoint{1.343498in}{2.442228in}}{\pgfqpoint{1.335262in}{2.442228in}}%
\pgfpathcurveto{\pgfqpoint{1.327026in}{2.442228in}}{\pgfqpoint{1.319126in}{2.438955in}}{\pgfqpoint{1.313302in}{2.433131in}}%
\pgfpathcurveto{\pgfqpoint{1.307478in}{2.427308in}}{\pgfqpoint{1.304206in}{2.419408in}}{\pgfqpoint{1.304206in}{2.411171in}}%
\pgfpathcurveto{\pgfqpoint{1.304206in}{2.402935in}}{\pgfqpoint{1.307478in}{2.395035in}}{\pgfqpoint{1.313302in}{2.389211in}}%
\pgfpathcurveto{\pgfqpoint{1.319126in}{2.383387in}}{\pgfqpoint{1.327026in}{2.380115in}}{\pgfqpoint{1.335262in}{2.380115in}}%
\pgfpathclose%
\pgfusepath{stroke,fill}%
\end{pgfscope}%
\begin{pgfscope}%
\pgfpathrectangle{\pgfqpoint{0.100000in}{0.212622in}}{\pgfqpoint{3.696000in}{3.696000in}}%
\pgfusepath{clip}%
\pgfsetbuttcap%
\pgfsetroundjoin%
\definecolor{currentfill}{rgb}{0.121569,0.466667,0.705882}%
\pgfsetfillcolor{currentfill}%
\pgfsetfillopacity{0.473325}%
\pgfsetlinewidth{1.003750pt}%
\definecolor{currentstroke}{rgb}{0.121569,0.466667,0.705882}%
\pgfsetstrokecolor{currentstroke}%
\pgfsetstrokeopacity{0.473325}%
\pgfsetdash{}{0pt}%
\pgfpathmoveto{\pgfqpoint{2.023966in}{2.626849in}}%
\pgfpathcurveto{\pgfqpoint{2.032202in}{2.626849in}}{\pgfqpoint{2.040103in}{2.630121in}}{\pgfqpoint{2.045926in}{2.635945in}}%
\pgfpathcurveto{\pgfqpoint{2.051750in}{2.641769in}}{\pgfqpoint{2.055023in}{2.649669in}}{\pgfqpoint{2.055023in}{2.657905in}}%
\pgfpathcurveto{\pgfqpoint{2.055023in}{2.666141in}}{\pgfqpoint{2.051750in}{2.674042in}}{\pgfqpoint{2.045926in}{2.679865in}}%
\pgfpathcurveto{\pgfqpoint{2.040103in}{2.685689in}}{\pgfqpoint{2.032202in}{2.688962in}}{\pgfqpoint{2.023966in}{2.688962in}}%
\pgfpathcurveto{\pgfqpoint{2.015730in}{2.688962in}}{\pgfqpoint{2.007830in}{2.685689in}}{\pgfqpoint{2.002006in}{2.679865in}}%
\pgfpathcurveto{\pgfqpoint{1.996182in}{2.674042in}}{\pgfqpoint{1.992910in}{2.666141in}}{\pgfqpoint{1.992910in}{2.657905in}}%
\pgfpathcurveto{\pgfqpoint{1.992910in}{2.649669in}}{\pgfqpoint{1.996182in}{2.641769in}}{\pgfqpoint{2.002006in}{2.635945in}}%
\pgfpathcurveto{\pgfqpoint{2.007830in}{2.630121in}}{\pgfqpoint{2.015730in}{2.626849in}}{\pgfqpoint{2.023966in}{2.626849in}}%
\pgfpathclose%
\pgfusepath{stroke,fill}%
\end{pgfscope}%
\begin{pgfscope}%
\pgfpathrectangle{\pgfqpoint{0.100000in}{0.212622in}}{\pgfqpoint{3.696000in}{3.696000in}}%
\pgfusepath{clip}%
\pgfsetbuttcap%
\pgfsetroundjoin%
\definecolor{currentfill}{rgb}{0.121569,0.466667,0.705882}%
\pgfsetfillcolor{currentfill}%
\pgfsetfillopacity{0.474483}%
\pgfsetlinewidth{1.003750pt}%
\definecolor{currentstroke}{rgb}{0.121569,0.466667,0.705882}%
\pgfsetstrokecolor{currentstroke}%
\pgfsetstrokeopacity{0.474483}%
\pgfsetdash{}{0pt}%
\pgfpathmoveto{\pgfqpoint{1.331203in}{2.372501in}}%
\pgfpathcurveto{\pgfqpoint{1.339439in}{2.372501in}}{\pgfqpoint{1.347339in}{2.375773in}}{\pgfqpoint{1.353163in}{2.381597in}}%
\pgfpathcurveto{\pgfqpoint{1.358987in}{2.387421in}}{\pgfqpoint{1.362260in}{2.395321in}}{\pgfqpoint{1.362260in}{2.403557in}}%
\pgfpathcurveto{\pgfqpoint{1.362260in}{2.411794in}}{\pgfqpoint{1.358987in}{2.419694in}}{\pgfqpoint{1.353163in}{2.425518in}}%
\pgfpathcurveto{\pgfqpoint{1.347339in}{2.431341in}}{\pgfqpoint{1.339439in}{2.434614in}}{\pgfqpoint{1.331203in}{2.434614in}}%
\pgfpathcurveto{\pgfqpoint{1.322967in}{2.434614in}}{\pgfqpoint{1.315067in}{2.431341in}}{\pgfqpoint{1.309243in}{2.425518in}}%
\pgfpathcurveto{\pgfqpoint{1.303419in}{2.419694in}}{\pgfqpoint{1.300147in}{2.411794in}}{\pgfqpoint{1.300147in}{2.403557in}}%
\pgfpathcurveto{\pgfqpoint{1.300147in}{2.395321in}}{\pgfqpoint{1.303419in}{2.387421in}}{\pgfqpoint{1.309243in}{2.381597in}}%
\pgfpathcurveto{\pgfqpoint{1.315067in}{2.375773in}}{\pgfqpoint{1.322967in}{2.372501in}}{\pgfqpoint{1.331203in}{2.372501in}}%
\pgfpathclose%
\pgfusepath{stroke,fill}%
\end{pgfscope}%
\begin{pgfscope}%
\pgfpathrectangle{\pgfqpoint{0.100000in}{0.212622in}}{\pgfqpoint{3.696000in}{3.696000in}}%
\pgfusepath{clip}%
\pgfsetbuttcap%
\pgfsetroundjoin%
\definecolor{currentfill}{rgb}{0.121569,0.466667,0.705882}%
\pgfsetfillcolor{currentfill}%
\pgfsetfillopacity{0.476056}%
\pgfsetlinewidth{1.003750pt}%
\definecolor{currentstroke}{rgb}{0.121569,0.466667,0.705882}%
\pgfsetstrokecolor{currentstroke}%
\pgfsetstrokeopacity{0.476056}%
\pgfsetdash{}{0pt}%
\pgfpathmoveto{\pgfqpoint{1.327053in}{2.366375in}}%
\pgfpathcurveto{\pgfqpoint{1.335289in}{2.366375in}}{\pgfqpoint{1.343189in}{2.369647in}}{\pgfqpoint{1.349013in}{2.375471in}}%
\pgfpathcurveto{\pgfqpoint{1.354837in}{2.381295in}}{\pgfqpoint{1.358109in}{2.389195in}}{\pgfqpoint{1.358109in}{2.397431in}}%
\pgfpathcurveto{\pgfqpoint{1.358109in}{2.405668in}}{\pgfqpoint{1.354837in}{2.413568in}}{\pgfqpoint{1.349013in}{2.419391in}}%
\pgfpathcurveto{\pgfqpoint{1.343189in}{2.425215in}}{\pgfqpoint{1.335289in}{2.428488in}}{\pgfqpoint{1.327053in}{2.428488in}}%
\pgfpathcurveto{\pgfqpoint{1.318816in}{2.428488in}}{\pgfqpoint{1.310916in}{2.425215in}}{\pgfqpoint{1.305092in}{2.419391in}}%
\pgfpathcurveto{\pgfqpoint{1.299268in}{2.413568in}}{\pgfqpoint{1.295996in}{2.405668in}}{\pgfqpoint{1.295996in}{2.397431in}}%
\pgfpathcurveto{\pgfqpoint{1.295996in}{2.389195in}}{\pgfqpoint{1.299268in}{2.381295in}}{\pgfqpoint{1.305092in}{2.375471in}}%
\pgfpathcurveto{\pgfqpoint{1.310916in}{2.369647in}}{\pgfqpoint{1.318816in}{2.366375in}}{\pgfqpoint{1.327053in}{2.366375in}}%
\pgfpathclose%
\pgfusepath{stroke,fill}%
\end{pgfscope}%
\begin{pgfscope}%
\pgfpathrectangle{\pgfqpoint{0.100000in}{0.212622in}}{\pgfqpoint{3.696000in}{3.696000in}}%
\pgfusepath{clip}%
\pgfsetbuttcap%
\pgfsetroundjoin%
\definecolor{currentfill}{rgb}{0.121569,0.466667,0.705882}%
\pgfsetfillcolor{currentfill}%
\pgfsetfillopacity{0.476483}%
\pgfsetlinewidth{1.003750pt}%
\definecolor{currentstroke}{rgb}{0.121569,0.466667,0.705882}%
\pgfsetstrokecolor{currentstroke}%
\pgfsetstrokeopacity{0.476483}%
\pgfsetdash{}{0pt}%
\pgfpathmoveto{\pgfqpoint{2.025284in}{2.614517in}}%
\pgfpathcurveto{\pgfqpoint{2.033521in}{2.614517in}}{\pgfqpoint{2.041421in}{2.617790in}}{\pgfqpoint{2.047245in}{2.623614in}}%
\pgfpathcurveto{\pgfqpoint{2.053068in}{2.629438in}}{\pgfqpoint{2.056341in}{2.637338in}}{\pgfqpoint{2.056341in}{2.645574in}}%
\pgfpathcurveto{\pgfqpoint{2.056341in}{2.653810in}}{\pgfqpoint{2.053068in}{2.661710in}}{\pgfqpoint{2.047245in}{2.667534in}}%
\pgfpathcurveto{\pgfqpoint{2.041421in}{2.673358in}}{\pgfqpoint{2.033521in}{2.676630in}}{\pgfqpoint{2.025284in}{2.676630in}}%
\pgfpathcurveto{\pgfqpoint{2.017048in}{2.676630in}}{\pgfqpoint{2.009148in}{2.673358in}}{\pgfqpoint{2.003324in}{2.667534in}}%
\pgfpathcurveto{\pgfqpoint{1.997500in}{2.661710in}}{\pgfqpoint{1.994228in}{2.653810in}}{\pgfqpoint{1.994228in}{2.645574in}}%
\pgfpathcurveto{\pgfqpoint{1.994228in}{2.637338in}}{\pgfqpoint{1.997500in}{2.629438in}}{\pgfqpoint{2.003324in}{2.623614in}}%
\pgfpathcurveto{\pgfqpoint{2.009148in}{2.617790in}}{\pgfqpoint{2.017048in}{2.614517in}}{\pgfqpoint{2.025284in}{2.614517in}}%
\pgfpathclose%
\pgfusepath{stroke,fill}%
\end{pgfscope}%
\begin{pgfscope}%
\pgfpathrectangle{\pgfqpoint{0.100000in}{0.212622in}}{\pgfqpoint{3.696000in}{3.696000in}}%
\pgfusepath{clip}%
\pgfsetbuttcap%
\pgfsetroundjoin%
\definecolor{currentfill}{rgb}{0.121569,0.466667,0.705882}%
\pgfsetfillcolor{currentfill}%
\pgfsetfillopacity{0.477300}%
\pgfsetlinewidth{1.003750pt}%
\definecolor{currentstroke}{rgb}{0.121569,0.466667,0.705882}%
\pgfsetstrokecolor{currentstroke}%
\pgfsetstrokeopacity{0.477300}%
\pgfsetdash{}{0pt}%
\pgfpathmoveto{\pgfqpoint{1.323895in}{2.361662in}}%
\pgfpathcurveto{\pgfqpoint{1.332131in}{2.361662in}}{\pgfqpoint{1.340031in}{2.364934in}}{\pgfqpoint{1.345855in}{2.370758in}}%
\pgfpathcurveto{\pgfqpoint{1.351679in}{2.376582in}}{\pgfqpoint{1.354951in}{2.384482in}}{\pgfqpoint{1.354951in}{2.392719in}}%
\pgfpathcurveto{\pgfqpoint{1.354951in}{2.400955in}}{\pgfqpoint{1.351679in}{2.408855in}}{\pgfqpoint{1.345855in}{2.414679in}}%
\pgfpathcurveto{\pgfqpoint{1.340031in}{2.420503in}}{\pgfqpoint{1.332131in}{2.423775in}}{\pgfqpoint{1.323895in}{2.423775in}}%
\pgfpathcurveto{\pgfqpoint{1.315658in}{2.423775in}}{\pgfqpoint{1.307758in}{2.420503in}}{\pgfqpoint{1.301934in}{2.414679in}}%
\pgfpathcurveto{\pgfqpoint{1.296110in}{2.408855in}}{\pgfqpoint{1.292838in}{2.400955in}}{\pgfqpoint{1.292838in}{2.392719in}}%
\pgfpathcurveto{\pgfqpoint{1.292838in}{2.384482in}}{\pgfqpoint{1.296110in}{2.376582in}}{\pgfqpoint{1.301934in}{2.370758in}}%
\pgfpathcurveto{\pgfqpoint{1.307758in}{2.364934in}}{\pgfqpoint{1.315658in}{2.361662in}}{\pgfqpoint{1.323895in}{2.361662in}}%
\pgfpathclose%
\pgfusepath{stroke,fill}%
\end{pgfscope}%
\begin{pgfscope}%
\pgfpathrectangle{\pgfqpoint{0.100000in}{0.212622in}}{\pgfqpoint{3.696000in}{3.696000in}}%
\pgfusepath{clip}%
\pgfsetbuttcap%
\pgfsetroundjoin%
\definecolor{currentfill}{rgb}{0.121569,0.466667,0.705882}%
\pgfsetfillcolor{currentfill}%
\pgfsetfillopacity{0.477808}%
\pgfsetlinewidth{1.003750pt}%
\definecolor{currentstroke}{rgb}{0.121569,0.466667,0.705882}%
\pgfsetstrokecolor{currentstroke}%
\pgfsetstrokeopacity{0.477808}%
\pgfsetdash{}{0pt}%
\pgfpathmoveto{\pgfqpoint{1.322412in}{2.359211in}}%
\pgfpathcurveto{\pgfqpoint{1.330648in}{2.359211in}}{\pgfqpoint{1.338548in}{2.362483in}}{\pgfqpoint{1.344372in}{2.368307in}}%
\pgfpathcurveto{\pgfqpoint{1.350196in}{2.374131in}}{\pgfqpoint{1.353469in}{2.382031in}}{\pgfqpoint{1.353469in}{2.390267in}}%
\pgfpathcurveto{\pgfqpoint{1.353469in}{2.398504in}}{\pgfqpoint{1.350196in}{2.406404in}}{\pgfqpoint{1.344372in}{2.412228in}}%
\pgfpathcurveto{\pgfqpoint{1.338548in}{2.418052in}}{\pgfqpoint{1.330648in}{2.421324in}}{\pgfqpoint{1.322412in}{2.421324in}}%
\pgfpathcurveto{\pgfqpoint{1.314176in}{2.421324in}}{\pgfqpoint{1.306276in}{2.418052in}}{\pgfqpoint{1.300452in}{2.412228in}}%
\pgfpathcurveto{\pgfqpoint{1.294628in}{2.406404in}}{\pgfqpoint{1.291356in}{2.398504in}}{\pgfqpoint{1.291356in}{2.390267in}}%
\pgfpathcurveto{\pgfqpoint{1.291356in}{2.382031in}}{\pgfqpoint{1.294628in}{2.374131in}}{\pgfqpoint{1.300452in}{2.368307in}}%
\pgfpathcurveto{\pgfqpoint{1.306276in}{2.362483in}}{\pgfqpoint{1.314176in}{2.359211in}}{\pgfqpoint{1.322412in}{2.359211in}}%
\pgfpathclose%
\pgfusepath{stroke,fill}%
\end{pgfscope}%
\begin{pgfscope}%
\pgfpathrectangle{\pgfqpoint{0.100000in}{0.212622in}}{\pgfqpoint{3.696000in}{3.696000in}}%
\pgfusepath{clip}%
\pgfsetbuttcap%
\pgfsetroundjoin%
\definecolor{currentfill}{rgb}{0.121569,0.466667,0.705882}%
\pgfsetfillcolor{currentfill}%
\pgfsetfillopacity{0.477901}%
\pgfsetlinewidth{1.003750pt}%
\definecolor{currentstroke}{rgb}{0.121569,0.466667,0.705882}%
\pgfsetstrokecolor{currentstroke}%
\pgfsetstrokeopacity{0.477901}%
\pgfsetdash{}{0pt}%
\pgfpathmoveto{\pgfqpoint{2.026513in}{2.606916in}}%
\pgfpathcurveto{\pgfqpoint{2.034750in}{2.606916in}}{\pgfqpoint{2.042650in}{2.610188in}}{\pgfqpoint{2.048474in}{2.616012in}}%
\pgfpathcurveto{\pgfqpoint{2.054298in}{2.621836in}}{\pgfqpoint{2.057570in}{2.629736in}}{\pgfqpoint{2.057570in}{2.637972in}}%
\pgfpathcurveto{\pgfqpoint{2.057570in}{2.646209in}}{\pgfqpoint{2.054298in}{2.654109in}}{\pgfqpoint{2.048474in}{2.659933in}}%
\pgfpathcurveto{\pgfqpoint{2.042650in}{2.665756in}}{\pgfqpoint{2.034750in}{2.669029in}}{\pgfqpoint{2.026513in}{2.669029in}}%
\pgfpathcurveto{\pgfqpoint{2.018277in}{2.669029in}}{\pgfqpoint{2.010377in}{2.665756in}}{\pgfqpoint{2.004553in}{2.659933in}}%
\pgfpathcurveto{\pgfqpoint{1.998729in}{2.654109in}}{\pgfqpoint{1.995457in}{2.646209in}}{\pgfqpoint{1.995457in}{2.637972in}}%
\pgfpathcurveto{\pgfqpoint{1.995457in}{2.629736in}}{\pgfqpoint{1.998729in}{2.621836in}}{\pgfqpoint{2.004553in}{2.616012in}}%
\pgfpathcurveto{\pgfqpoint{2.010377in}{2.610188in}}{\pgfqpoint{2.018277in}{2.606916in}}{\pgfqpoint{2.026513in}{2.606916in}}%
\pgfpathclose%
\pgfusepath{stroke,fill}%
\end{pgfscope}%
\begin{pgfscope}%
\pgfpathrectangle{\pgfqpoint{0.100000in}{0.212622in}}{\pgfqpoint{3.696000in}{3.696000in}}%
\pgfusepath{clip}%
\pgfsetbuttcap%
\pgfsetroundjoin%
\definecolor{currentfill}{rgb}{0.121569,0.466667,0.705882}%
\pgfsetfillcolor{currentfill}%
\pgfsetfillopacity{0.478770}%
\pgfsetlinewidth{1.003750pt}%
\definecolor{currentstroke}{rgb}{0.121569,0.466667,0.705882}%
\pgfsetstrokecolor{currentstroke}%
\pgfsetstrokeopacity{0.478770}%
\pgfsetdash{}{0pt}%
\pgfpathmoveto{\pgfqpoint{1.319814in}{2.354765in}}%
\pgfpathcurveto{\pgfqpoint{1.328051in}{2.354765in}}{\pgfqpoint{1.335951in}{2.358037in}}{\pgfqpoint{1.341775in}{2.363861in}}%
\pgfpathcurveto{\pgfqpoint{1.347599in}{2.369685in}}{\pgfqpoint{1.350871in}{2.377585in}}{\pgfqpoint{1.350871in}{2.385822in}}%
\pgfpathcurveto{\pgfqpoint{1.350871in}{2.394058in}}{\pgfqpoint{1.347599in}{2.401958in}}{\pgfqpoint{1.341775in}{2.407782in}}%
\pgfpathcurveto{\pgfqpoint{1.335951in}{2.413606in}}{\pgfqpoint{1.328051in}{2.416878in}}{\pgfqpoint{1.319814in}{2.416878in}}%
\pgfpathcurveto{\pgfqpoint{1.311578in}{2.416878in}}{\pgfqpoint{1.303678in}{2.413606in}}{\pgfqpoint{1.297854in}{2.407782in}}%
\pgfpathcurveto{\pgfqpoint{1.292030in}{2.401958in}}{\pgfqpoint{1.288758in}{2.394058in}}{\pgfqpoint{1.288758in}{2.385822in}}%
\pgfpathcurveto{\pgfqpoint{1.288758in}{2.377585in}}{\pgfqpoint{1.292030in}{2.369685in}}{\pgfqpoint{1.297854in}{2.363861in}}%
\pgfpathcurveto{\pgfqpoint{1.303678in}{2.358037in}}{\pgfqpoint{1.311578in}{2.354765in}}{\pgfqpoint{1.319814in}{2.354765in}}%
\pgfpathclose%
\pgfusepath{stroke,fill}%
\end{pgfscope}%
\begin{pgfscope}%
\pgfpathrectangle{\pgfqpoint{0.100000in}{0.212622in}}{\pgfqpoint{3.696000in}{3.696000in}}%
\pgfusepath{clip}%
\pgfsetbuttcap%
\pgfsetroundjoin%
\definecolor{currentfill}{rgb}{0.121569,0.466667,0.705882}%
\pgfsetfillcolor{currentfill}%
\pgfsetfillopacity{0.479588}%
\pgfsetlinewidth{1.003750pt}%
\definecolor{currentstroke}{rgb}{0.121569,0.466667,0.705882}%
\pgfsetstrokecolor{currentstroke}%
\pgfsetstrokeopacity{0.479588}%
\pgfsetdash{}{0pt}%
\pgfpathmoveto{\pgfqpoint{1.317490in}{2.350663in}}%
\pgfpathcurveto{\pgfqpoint{1.325727in}{2.350663in}}{\pgfqpoint{1.333627in}{2.353936in}}{\pgfqpoint{1.339450in}{2.359760in}}%
\pgfpathcurveto{\pgfqpoint{1.345274in}{2.365583in}}{\pgfqpoint{1.348547in}{2.373483in}}{\pgfqpoint{1.348547in}{2.381720in}}%
\pgfpathcurveto{\pgfqpoint{1.348547in}{2.389956in}}{\pgfqpoint{1.345274in}{2.397856in}}{\pgfqpoint{1.339450in}{2.403680in}}%
\pgfpathcurveto{\pgfqpoint{1.333627in}{2.409504in}}{\pgfqpoint{1.325727in}{2.412776in}}{\pgfqpoint{1.317490in}{2.412776in}}%
\pgfpathcurveto{\pgfqpoint{1.309254in}{2.412776in}}{\pgfqpoint{1.301354in}{2.409504in}}{\pgfqpoint{1.295530in}{2.403680in}}%
\pgfpathcurveto{\pgfqpoint{1.289706in}{2.397856in}}{\pgfqpoint{1.286434in}{2.389956in}}{\pgfqpoint{1.286434in}{2.381720in}}%
\pgfpathcurveto{\pgfqpoint{1.286434in}{2.373483in}}{\pgfqpoint{1.289706in}{2.365583in}}{\pgfqpoint{1.295530in}{2.359760in}}%
\pgfpathcurveto{\pgfqpoint{1.301354in}{2.353936in}}{\pgfqpoint{1.309254in}{2.350663in}}{\pgfqpoint{1.317490in}{2.350663in}}%
\pgfpathclose%
\pgfusepath{stroke,fill}%
\end{pgfscope}%
\begin{pgfscope}%
\pgfpathrectangle{\pgfqpoint{0.100000in}{0.212622in}}{\pgfqpoint{3.696000in}{3.696000in}}%
\pgfusepath{clip}%
\pgfsetbuttcap%
\pgfsetroundjoin%
\definecolor{currentfill}{rgb}{0.121569,0.466667,0.705882}%
\pgfsetfillcolor{currentfill}%
\pgfsetfillopacity{0.480000}%
\pgfsetlinewidth{1.003750pt}%
\definecolor{currentstroke}{rgb}{0.121569,0.466667,0.705882}%
\pgfsetstrokecolor{currentstroke}%
\pgfsetstrokeopacity{0.480000}%
\pgfsetdash{}{0pt}%
\pgfpathmoveto{\pgfqpoint{2.027330in}{2.597881in}}%
\pgfpathcurveto{\pgfqpoint{2.035566in}{2.597881in}}{\pgfqpoint{2.043466in}{2.601153in}}{\pgfqpoint{2.049290in}{2.606977in}}%
\pgfpathcurveto{\pgfqpoint{2.055114in}{2.612801in}}{\pgfqpoint{2.058386in}{2.620701in}}{\pgfqpoint{2.058386in}{2.628937in}}%
\pgfpathcurveto{\pgfqpoint{2.058386in}{2.637174in}}{\pgfqpoint{2.055114in}{2.645074in}}{\pgfqpoint{2.049290in}{2.650898in}}%
\pgfpathcurveto{\pgfqpoint{2.043466in}{2.656722in}}{\pgfqpoint{2.035566in}{2.659994in}}{\pgfqpoint{2.027330in}{2.659994in}}%
\pgfpathcurveto{\pgfqpoint{2.019094in}{2.659994in}}{\pgfqpoint{2.011194in}{2.656722in}}{\pgfqpoint{2.005370in}{2.650898in}}%
\pgfpathcurveto{\pgfqpoint{1.999546in}{2.645074in}}{\pgfqpoint{1.996273in}{2.637174in}}{\pgfqpoint{1.996273in}{2.628937in}}%
\pgfpathcurveto{\pgfqpoint{1.996273in}{2.620701in}}{\pgfqpoint{1.999546in}{2.612801in}}{\pgfqpoint{2.005370in}{2.606977in}}%
\pgfpathcurveto{\pgfqpoint{2.011194in}{2.601153in}}{\pgfqpoint{2.019094in}{2.597881in}}{\pgfqpoint{2.027330in}{2.597881in}}%
\pgfpathclose%
\pgfusepath{stroke,fill}%
\end{pgfscope}%
\begin{pgfscope}%
\pgfpathrectangle{\pgfqpoint{0.100000in}{0.212622in}}{\pgfqpoint{3.696000in}{3.696000in}}%
\pgfusepath{clip}%
\pgfsetbuttcap%
\pgfsetroundjoin%
\definecolor{currentfill}{rgb}{0.121569,0.466667,0.705882}%
\pgfsetfillcolor{currentfill}%
\pgfsetfillopacity{0.480217}%
\pgfsetlinewidth{1.003750pt}%
\definecolor{currentstroke}{rgb}{0.121569,0.466667,0.705882}%
\pgfsetstrokecolor{currentstroke}%
\pgfsetstrokeopacity{0.480217}%
\pgfsetdash{}{0pt}%
\pgfpathmoveto{\pgfqpoint{1.315404in}{2.347182in}}%
\pgfpathcurveto{\pgfqpoint{1.323641in}{2.347182in}}{\pgfqpoint{1.331541in}{2.350454in}}{\pgfqpoint{1.337365in}{2.356278in}}%
\pgfpathcurveto{\pgfqpoint{1.343188in}{2.362102in}}{\pgfqpoint{1.346461in}{2.370002in}}{\pgfqpoint{1.346461in}{2.378238in}}%
\pgfpathcurveto{\pgfqpoint{1.346461in}{2.386475in}}{\pgfqpoint{1.343188in}{2.394375in}}{\pgfqpoint{1.337365in}{2.400199in}}%
\pgfpathcurveto{\pgfqpoint{1.331541in}{2.406022in}}{\pgfqpoint{1.323641in}{2.409295in}}{\pgfqpoint{1.315404in}{2.409295in}}%
\pgfpathcurveto{\pgfqpoint{1.307168in}{2.409295in}}{\pgfqpoint{1.299268in}{2.406022in}}{\pgfqpoint{1.293444in}{2.400199in}}%
\pgfpathcurveto{\pgfqpoint{1.287620in}{2.394375in}}{\pgfqpoint{1.284348in}{2.386475in}}{\pgfqpoint{1.284348in}{2.378238in}}%
\pgfpathcurveto{\pgfqpoint{1.284348in}{2.370002in}}{\pgfqpoint{1.287620in}{2.362102in}}{\pgfqpoint{1.293444in}{2.356278in}}%
\pgfpathcurveto{\pgfqpoint{1.299268in}{2.350454in}}{\pgfqpoint{1.307168in}{2.347182in}}{\pgfqpoint{1.315404in}{2.347182in}}%
\pgfpathclose%
\pgfusepath{stroke,fill}%
\end{pgfscope}%
\begin{pgfscope}%
\pgfpathrectangle{\pgfqpoint{0.100000in}{0.212622in}}{\pgfqpoint{3.696000in}{3.696000in}}%
\pgfusepath{clip}%
\pgfsetbuttcap%
\pgfsetroundjoin%
\definecolor{currentfill}{rgb}{0.121569,0.466667,0.705882}%
\pgfsetfillcolor{currentfill}%
\pgfsetfillopacity{0.481166}%
\pgfsetlinewidth{1.003750pt}%
\definecolor{currentstroke}{rgb}{0.121569,0.466667,0.705882}%
\pgfsetstrokecolor{currentstroke}%
\pgfsetstrokeopacity{0.481166}%
\pgfsetdash{}{0pt}%
\pgfpathmoveto{\pgfqpoint{2.027822in}{2.592983in}}%
\pgfpathcurveto{\pgfqpoint{2.036058in}{2.592983in}}{\pgfqpoint{2.043958in}{2.596255in}}{\pgfqpoint{2.049782in}{2.602079in}}%
\pgfpathcurveto{\pgfqpoint{2.055606in}{2.607903in}}{\pgfqpoint{2.058878in}{2.615803in}}{\pgfqpoint{2.058878in}{2.624040in}}%
\pgfpathcurveto{\pgfqpoint{2.058878in}{2.632276in}}{\pgfqpoint{2.055606in}{2.640176in}}{\pgfqpoint{2.049782in}{2.646000in}}%
\pgfpathcurveto{\pgfqpoint{2.043958in}{2.651824in}}{\pgfqpoint{2.036058in}{2.655096in}}{\pgfqpoint{2.027822in}{2.655096in}}%
\pgfpathcurveto{\pgfqpoint{2.019586in}{2.655096in}}{\pgfqpoint{2.011686in}{2.651824in}}{\pgfqpoint{2.005862in}{2.646000in}}%
\pgfpathcurveto{\pgfqpoint{2.000038in}{2.640176in}}{\pgfqpoint{1.996765in}{2.632276in}}{\pgfqpoint{1.996765in}{2.624040in}}%
\pgfpathcurveto{\pgfqpoint{1.996765in}{2.615803in}}{\pgfqpoint{2.000038in}{2.607903in}}{\pgfqpoint{2.005862in}{2.602079in}}%
\pgfpathcurveto{\pgfqpoint{2.011686in}{2.596255in}}{\pgfqpoint{2.019586in}{2.592983in}}{\pgfqpoint{2.027822in}{2.592983in}}%
\pgfpathclose%
\pgfusepath{stroke,fill}%
\end{pgfscope}%
\begin{pgfscope}%
\pgfpathrectangle{\pgfqpoint{0.100000in}{0.212622in}}{\pgfqpoint{3.696000in}{3.696000in}}%
\pgfusepath{clip}%
\pgfsetbuttcap%
\pgfsetroundjoin%
\definecolor{currentfill}{rgb}{0.121569,0.466667,0.705882}%
\pgfsetfillcolor{currentfill}%
\pgfsetfillopacity{0.481422}%
\pgfsetlinewidth{1.003750pt}%
\definecolor{currentstroke}{rgb}{0.121569,0.466667,0.705882}%
\pgfsetstrokecolor{currentstroke}%
\pgfsetstrokeopacity{0.481422}%
\pgfsetdash{}{0pt}%
\pgfpathmoveto{\pgfqpoint{1.312117in}{2.340417in}}%
\pgfpathcurveto{\pgfqpoint{1.320353in}{2.340417in}}{\pgfqpoint{1.328253in}{2.343690in}}{\pgfqpoint{1.334077in}{2.349514in}}%
\pgfpathcurveto{\pgfqpoint{1.339901in}{2.355337in}}{\pgfqpoint{1.343173in}{2.363238in}}{\pgfqpoint{1.343173in}{2.371474in}}%
\pgfpathcurveto{\pgfqpoint{1.343173in}{2.379710in}}{\pgfqpoint{1.339901in}{2.387610in}}{\pgfqpoint{1.334077in}{2.393434in}}%
\pgfpathcurveto{\pgfqpoint{1.328253in}{2.399258in}}{\pgfqpoint{1.320353in}{2.402530in}}{\pgfqpoint{1.312117in}{2.402530in}}%
\pgfpathcurveto{\pgfqpoint{1.303880in}{2.402530in}}{\pgfqpoint{1.295980in}{2.399258in}}{\pgfqpoint{1.290156in}{2.393434in}}%
\pgfpathcurveto{\pgfqpoint{1.284332in}{2.387610in}}{\pgfqpoint{1.281060in}{2.379710in}}{\pgfqpoint{1.281060in}{2.371474in}}%
\pgfpathcurveto{\pgfqpoint{1.281060in}{2.363238in}}{\pgfqpoint{1.284332in}{2.355337in}}{\pgfqpoint{1.290156in}{2.349514in}}%
\pgfpathcurveto{\pgfqpoint{1.295980in}{2.343690in}}{\pgfqpoint{1.303880in}{2.340417in}}{\pgfqpoint{1.312117in}{2.340417in}}%
\pgfpathclose%
\pgfusepath{stroke,fill}%
\end{pgfscope}%
\begin{pgfscope}%
\pgfpathrectangle{\pgfqpoint{0.100000in}{0.212622in}}{\pgfqpoint{3.696000in}{3.696000in}}%
\pgfusepath{clip}%
\pgfsetbuttcap%
\pgfsetroundjoin%
\definecolor{currentfill}{rgb}{0.121569,0.466667,0.705882}%
\pgfsetfillcolor{currentfill}%
\pgfsetfillopacity{0.481691}%
\pgfsetlinewidth{1.003750pt}%
\definecolor{currentstroke}{rgb}{0.121569,0.466667,0.705882}%
\pgfsetstrokecolor{currentstroke}%
\pgfsetstrokeopacity{0.481691}%
\pgfsetdash{}{0pt}%
\pgfpathmoveto{\pgfqpoint{2.028285in}{2.590000in}}%
\pgfpathcurveto{\pgfqpoint{2.036521in}{2.590000in}}{\pgfqpoint{2.044421in}{2.593272in}}{\pgfqpoint{2.050245in}{2.599096in}}%
\pgfpathcurveto{\pgfqpoint{2.056069in}{2.604920in}}{\pgfqpoint{2.059342in}{2.612820in}}{\pgfqpoint{2.059342in}{2.621057in}}%
\pgfpathcurveto{\pgfqpoint{2.059342in}{2.629293in}}{\pgfqpoint{2.056069in}{2.637193in}}{\pgfqpoint{2.050245in}{2.643017in}}%
\pgfpathcurveto{\pgfqpoint{2.044421in}{2.648841in}}{\pgfqpoint{2.036521in}{2.652113in}}{\pgfqpoint{2.028285in}{2.652113in}}%
\pgfpathcurveto{\pgfqpoint{2.020049in}{2.652113in}}{\pgfqpoint{2.012149in}{2.648841in}}{\pgfqpoint{2.006325in}{2.643017in}}%
\pgfpathcurveto{\pgfqpoint{2.000501in}{2.637193in}}{\pgfqpoint{1.997229in}{2.629293in}}{\pgfqpoint{1.997229in}{2.621057in}}%
\pgfpathcurveto{\pgfqpoint{1.997229in}{2.612820in}}{\pgfqpoint{2.000501in}{2.604920in}}{\pgfqpoint{2.006325in}{2.599096in}}%
\pgfpathcurveto{\pgfqpoint{2.012149in}{2.593272in}}{\pgfqpoint{2.020049in}{2.590000in}}{\pgfqpoint{2.028285in}{2.590000in}}%
\pgfpathclose%
\pgfusepath{stroke,fill}%
\end{pgfscope}%
\begin{pgfscope}%
\pgfpathrectangle{\pgfqpoint{0.100000in}{0.212622in}}{\pgfqpoint{3.696000in}{3.696000in}}%
\pgfusepath{clip}%
\pgfsetbuttcap%
\pgfsetroundjoin%
\definecolor{currentfill}{rgb}{0.121569,0.466667,0.705882}%
\pgfsetfillcolor{currentfill}%
\pgfsetfillopacity{0.481953}%
\pgfsetlinewidth{1.003750pt}%
\definecolor{currentstroke}{rgb}{0.121569,0.466667,0.705882}%
\pgfsetstrokecolor{currentstroke}%
\pgfsetstrokeopacity{0.481953}%
\pgfsetdash{}{0pt}%
\pgfpathmoveto{\pgfqpoint{1.310048in}{2.336803in}}%
\pgfpathcurveto{\pgfqpoint{1.318285in}{2.336803in}}{\pgfqpoint{1.326185in}{2.340076in}}{\pgfqpoint{1.332009in}{2.345900in}}%
\pgfpathcurveto{\pgfqpoint{1.337833in}{2.351724in}}{\pgfqpoint{1.341105in}{2.359624in}}{\pgfqpoint{1.341105in}{2.367860in}}%
\pgfpathcurveto{\pgfqpoint{1.341105in}{2.376096in}}{\pgfqpoint{1.337833in}{2.383996in}}{\pgfqpoint{1.332009in}{2.389820in}}%
\pgfpathcurveto{\pgfqpoint{1.326185in}{2.395644in}}{\pgfqpoint{1.318285in}{2.398916in}}{\pgfqpoint{1.310048in}{2.398916in}}%
\pgfpathcurveto{\pgfqpoint{1.301812in}{2.398916in}}{\pgfqpoint{1.293912in}{2.395644in}}{\pgfqpoint{1.288088in}{2.389820in}}%
\pgfpathcurveto{\pgfqpoint{1.282264in}{2.383996in}}{\pgfqpoint{1.278992in}{2.376096in}}{\pgfqpoint{1.278992in}{2.367860in}}%
\pgfpathcurveto{\pgfqpoint{1.278992in}{2.359624in}}{\pgfqpoint{1.282264in}{2.351724in}}{\pgfqpoint{1.288088in}{2.345900in}}%
\pgfpathcurveto{\pgfqpoint{1.293912in}{2.340076in}}{\pgfqpoint{1.301812in}{2.336803in}}{\pgfqpoint{1.310048in}{2.336803in}}%
\pgfpathclose%
\pgfusepath{stroke,fill}%
\end{pgfscope}%
\begin{pgfscope}%
\pgfpathrectangle{\pgfqpoint{0.100000in}{0.212622in}}{\pgfqpoint{3.696000in}{3.696000in}}%
\pgfusepath{clip}%
\pgfsetbuttcap%
\pgfsetroundjoin%
\definecolor{currentfill}{rgb}{0.121569,0.466667,0.705882}%
\pgfsetfillcolor{currentfill}%
\pgfsetfillopacity{0.482426}%
\pgfsetlinewidth{1.003750pt}%
\definecolor{currentstroke}{rgb}{0.121569,0.466667,0.705882}%
\pgfsetstrokecolor{currentstroke}%
\pgfsetstrokeopacity{0.482426}%
\pgfsetdash{}{0pt}%
\pgfpathmoveto{\pgfqpoint{1.308448in}{2.333199in}}%
\pgfpathcurveto{\pgfqpoint{1.316684in}{2.333199in}}{\pgfqpoint{1.324584in}{2.336472in}}{\pgfqpoint{1.330408in}{2.342296in}}%
\pgfpathcurveto{\pgfqpoint{1.336232in}{2.348119in}}{\pgfqpoint{1.339504in}{2.356020in}}{\pgfqpoint{1.339504in}{2.364256in}}%
\pgfpathcurveto{\pgfqpoint{1.339504in}{2.372492in}}{\pgfqpoint{1.336232in}{2.380392in}}{\pgfqpoint{1.330408in}{2.386216in}}%
\pgfpathcurveto{\pgfqpoint{1.324584in}{2.392040in}}{\pgfqpoint{1.316684in}{2.395312in}}{\pgfqpoint{1.308448in}{2.395312in}}%
\pgfpathcurveto{\pgfqpoint{1.300211in}{2.395312in}}{\pgfqpoint{1.292311in}{2.392040in}}{\pgfqpoint{1.286487in}{2.386216in}}%
\pgfpathcurveto{\pgfqpoint{1.280663in}{2.380392in}}{\pgfqpoint{1.277391in}{2.372492in}}{\pgfqpoint{1.277391in}{2.364256in}}%
\pgfpathcurveto{\pgfqpoint{1.277391in}{2.356020in}}{\pgfqpoint{1.280663in}{2.348119in}}{\pgfqpoint{1.286487in}{2.342296in}}%
\pgfpathcurveto{\pgfqpoint{1.292311in}{2.336472in}}{\pgfqpoint{1.300211in}{2.333199in}}{\pgfqpoint{1.308448in}{2.333199in}}%
\pgfpathclose%
\pgfusepath{stroke,fill}%
\end{pgfscope}%
\begin{pgfscope}%
\pgfpathrectangle{\pgfqpoint{0.100000in}{0.212622in}}{\pgfqpoint{3.696000in}{3.696000in}}%
\pgfusepath{clip}%
\pgfsetbuttcap%
\pgfsetroundjoin%
\definecolor{currentfill}{rgb}{0.121569,0.466667,0.705882}%
\pgfsetfillcolor{currentfill}%
\pgfsetfillopacity{0.482601}%
\pgfsetlinewidth{1.003750pt}%
\definecolor{currentstroke}{rgb}{0.121569,0.466667,0.705882}%
\pgfsetstrokecolor{currentstroke}%
\pgfsetstrokeopacity{0.482601}%
\pgfsetdash{}{0pt}%
\pgfpathmoveto{\pgfqpoint{2.028599in}{2.585546in}}%
\pgfpathcurveto{\pgfqpoint{2.036835in}{2.585546in}}{\pgfqpoint{2.044735in}{2.588818in}}{\pgfqpoint{2.050559in}{2.594642in}}%
\pgfpathcurveto{\pgfqpoint{2.056383in}{2.600466in}}{\pgfqpoint{2.059655in}{2.608366in}}{\pgfqpoint{2.059655in}{2.616603in}}%
\pgfpathcurveto{\pgfqpoint{2.059655in}{2.624839in}}{\pgfqpoint{2.056383in}{2.632739in}}{\pgfqpoint{2.050559in}{2.638563in}}%
\pgfpathcurveto{\pgfqpoint{2.044735in}{2.644387in}}{\pgfqpoint{2.036835in}{2.647659in}}{\pgfqpoint{2.028599in}{2.647659in}}%
\pgfpathcurveto{\pgfqpoint{2.020363in}{2.647659in}}{\pgfqpoint{2.012463in}{2.644387in}}{\pgfqpoint{2.006639in}{2.638563in}}%
\pgfpathcurveto{\pgfqpoint{2.000815in}{2.632739in}}{\pgfqpoint{1.997542in}{2.624839in}}{\pgfqpoint{1.997542in}{2.616603in}}%
\pgfpathcurveto{\pgfqpoint{1.997542in}{2.608366in}}{\pgfqpoint{2.000815in}{2.600466in}}{\pgfqpoint{2.006639in}{2.594642in}}%
\pgfpathcurveto{\pgfqpoint{2.012463in}{2.588818in}}{\pgfqpoint{2.020363in}{2.585546in}}{\pgfqpoint{2.028599in}{2.585546in}}%
\pgfpathclose%
\pgfusepath{stroke,fill}%
\end{pgfscope}%
\begin{pgfscope}%
\pgfpathrectangle{\pgfqpoint{0.100000in}{0.212622in}}{\pgfqpoint{3.696000in}{3.696000in}}%
\pgfusepath{clip}%
\pgfsetbuttcap%
\pgfsetroundjoin%
\definecolor{currentfill}{rgb}{0.121569,0.466667,0.705882}%
\pgfsetfillcolor{currentfill}%
\pgfsetfillopacity{0.482761}%
\pgfsetlinewidth{1.003750pt}%
\definecolor{currentstroke}{rgb}{0.121569,0.466667,0.705882}%
\pgfsetstrokecolor{currentstroke}%
\pgfsetstrokeopacity{0.482761}%
\pgfsetdash{}{0pt}%
\pgfpathmoveto{\pgfqpoint{1.307104in}{2.330559in}}%
\pgfpathcurveto{\pgfqpoint{1.315340in}{2.330559in}}{\pgfqpoint{1.323240in}{2.333831in}}{\pgfqpoint{1.329064in}{2.339655in}}%
\pgfpathcurveto{\pgfqpoint{1.334888in}{2.345479in}}{\pgfqpoint{1.338160in}{2.353379in}}{\pgfqpoint{1.338160in}{2.361615in}}%
\pgfpathcurveto{\pgfqpoint{1.338160in}{2.369851in}}{\pgfqpoint{1.334888in}{2.377751in}}{\pgfqpoint{1.329064in}{2.383575in}}%
\pgfpathcurveto{\pgfqpoint{1.323240in}{2.389399in}}{\pgfqpoint{1.315340in}{2.392672in}}{\pgfqpoint{1.307104in}{2.392672in}}%
\pgfpathcurveto{\pgfqpoint{1.298867in}{2.392672in}}{\pgfqpoint{1.290967in}{2.389399in}}{\pgfqpoint{1.285143in}{2.383575in}}%
\pgfpathcurveto{\pgfqpoint{1.279319in}{2.377751in}}{\pgfqpoint{1.276047in}{2.369851in}}{\pgfqpoint{1.276047in}{2.361615in}}%
\pgfpathcurveto{\pgfqpoint{1.276047in}{2.353379in}}{\pgfqpoint{1.279319in}{2.345479in}}{\pgfqpoint{1.285143in}{2.339655in}}%
\pgfpathcurveto{\pgfqpoint{1.290967in}{2.333831in}}{\pgfqpoint{1.298867in}{2.330559in}}{\pgfqpoint{1.307104in}{2.330559in}}%
\pgfpathclose%
\pgfusepath{stroke,fill}%
\end{pgfscope}%
\begin{pgfscope}%
\pgfpathrectangle{\pgfqpoint{0.100000in}{0.212622in}}{\pgfqpoint{3.696000in}{3.696000in}}%
\pgfusepath{clip}%
\pgfsetbuttcap%
\pgfsetroundjoin%
\definecolor{currentfill}{rgb}{0.121569,0.466667,0.705882}%
\pgfsetfillcolor{currentfill}%
\pgfsetfillopacity{0.482943}%
\pgfsetlinewidth{1.003750pt}%
\definecolor{currentstroke}{rgb}{0.121569,0.466667,0.705882}%
\pgfsetstrokecolor{currentstroke}%
\pgfsetstrokeopacity{0.482943}%
\pgfsetdash{}{0pt}%
\pgfpathmoveto{\pgfqpoint{1.306567in}{2.328901in}}%
\pgfpathcurveto{\pgfqpoint{1.314804in}{2.328901in}}{\pgfqpoint{1.322704in}{2.332174in}}{\pgfqpoint{1.328528in}{2.337997in}}%
\pgfpathcurveto{\pgfqpoint{1.334352in}{2.343821in}}{\pgfqpoint{1.337624in}{2.351721in}}{\pgfqpoint{1.337624in}{2.359958in}}%
\pgfpathcurveto{\pgfqpoint{1.337624in}{2.368194in}}{\pgfqpoint{1.334352in}{2.376094in}}{\pgfqpoint{1.328528in}{2.381918in}}%
\pgfpathcurveto{\pgfqpoint{1.322704in}{2.387742in}}{\pgfqpoint{1.314804in}{2.391014in}}{\pgfqpoint{1.306567in}{2.391014in}}%
\pgfpathcurveto{\pgfqpoint{1.298331in}{2.391014in}}{\pgfqpoint{1.290431in}{2.387742in}}{\pgfqpoint{1.284607in}{2.381918in}}%
\pgfpathcurveto{\pgfqpoint{1.278783in}{2.376094in}}{\pgfqpoint{1.275511in}{2.368194in}}{\pgfqpoint{1.275511in}{2.359958in}}%
\pgfpathcurveto{\pgfqpoint{1.275511in}{2.351721in}}{\pgfqpoint{1.278783in}{2.343821in}}{\pgfqpoint{1.284607in}{2.337997in}}%
\pgfpathcurveto{\pgfqpoint{1.290431in}{2.332174in}}{\pgfqpoint{1.298331in}{2.328901in}}{\pgfqpoint{1.306567in}{2.328901in}}%
\pgfpathclose%
\pgfusepath{stroke,fill}%
\end{pgfscope}%
\begin{pgfscope}%
\pgfpathrectangle{\pgfqpoint{0.100000in}{0.212622in}}{\pgfqpoint{3.696000in}{3.696000in}}%
\pgfusepath{clip}%
\pgfsetbuttcap%
\pgfsetroundjoin%
\definecolor{currentfill}{rgb}{0.121569,0.466667,0.705882}%
\pgfsetfillcolor{currentfill}%
\pgfsetfillopacity{0.483130}%
\pgfsetlinewidth{1.003750pt}%
\definecolor{currentstroke}{rgb}{0.121569,0.466667,0.705882}%
\pgfsetstrokecolor{currentstroke}%
\pgfsetstrokeopacity{0.483130}%
\pgfsetdash{}{0pt}%
\pgfpathmoveto{\pgfqpoint{1.306051in}{2.328019in}}%
\pgfpathcurveto{\pgfqpoint{1.314288in}{2.328019in}}{\pgfqpoint{1.322188in}{2.331291in}}{\pgfqpoint{1.328012in}{2.337115in}}%
\pgfpathcurveto{\pgfqpoint{1.333835in}{2.342939in}}{\pgfqpoint{1.337108in}{2.350839in}}{\pgfqpoint{1.337108in}{2.359076in}}%
\pgfpathcurveto{\pgfqpoint{1.337108in}{2.367312in}}{\pgfqpoint{1.333835in}{2.375212in}}{\pgfqpoint{1.328012in}{2.381036in}}%
\pgfpathcurveto{\pgfqpoint{1.322188in}{2.386860in}}{\pgfqpoint{1.314288in}{2.390132in}}{\pgfqpoint{1.306051in}{2.390132in}}%
\pgfpathcurveto{\pgfqpoint{1.297815in}{2.390132in}}{\pgfqpoint{1.289915in}{2.386860in}}{\pgfqpoint{1.284091in}{2.381036in}}%
\pgfpathcurveto{\pgfqpoint{1.278267in}{2.375212in}}{\pgfqpoint{1.274995in}{2.367312in}}{\pgfqpoint{1.274995in}{2.359076in}}%
\pgfpathcurveto{\pgfqpoint{1.274995in}{2.350839in}}{\pgfqpoint{1.278267in}{2.342939in}}{\pgfqpoint{1.284091in}{2.337115in}}%
\pgfpathcurveto{\pgfqpoint{1.289915in}{2.331291in}}{\pgfqpoint{1.297815in}{2.328019in}}{\pgfqpoint{1.306051in}{2.328019in}}%
\pgfpathclose%
\pgfusepath{stroke,fill}%
\end{pgfscope}%
\begin{pgfscope}%
\pgfpathrectangle{\pgfqpoint{0.100000in}{0.212622in}}{\pgfqpoint{3.696000in}{3.696000in}}%
\pgfusepath{clip}%
\pgfsetbuttcap%
\pgfsetroundjoin%
\definecolor{currentfill}{rgb}{0.121569,0.466667,0.705882}%
\pgfsetfillcolor{currentfill}%
\pgfsetfillopacity{0.483466}%
\pgfsetlinewidth{1.003750pt}%
\definecolor{currentstroke}{rgb}{0.121569,0.466667,0.705882}%
\pgfsetstrokecolor{currentstroke}%
\pgfsetstrokeopacity{0.483466}%
\pgfsetdash{}{0pt}%
\pgfpathmoveto{\pgfqpoint{1.305077in}{2.326444in}}%
\pgfpathcurveto{\pgfqpoint{1.313314in}{2.326444in}}{\pgfqpoint{1.321214in}{2.329717in}}{\pgfqpoint{1.327038in}{2.335540in}}%
\pgfpathcurveto{\pgfqpoint{1.332862in}{2.341364in}}{\pgfqpoint{1.336134in}{2.349264in}}{\pgfqpoint{1.336134in}{2.357501in}}%
\pgfpathcurveto{\pgfqpoint{1.336134in}{2.365737in}}{\pgfqpoint{1.332862in}{2.373637in}}{\pgfqpoint{1.327038in}{2.379461in}}%
\pgfpathcurveto{\pgfqpoint{1.321214in}{2.385285in}}{\pgfqpoint{1.313314in}{2.388557in}}{\pgfqpoint{1.305077in}{2.388557in}}%
\pgfpathcurveto{\pgfqpoint{1.296841in}{2.388557in}}{\pgfqpoint{1.288941in}{2.385285in}}{\pgfqpoint{1.283117in}{2.379461in}}%
\pgfpathcurveto{\pgfqpoint{1.277293in}{2.373637in}}{\pgfqpoint{1.274021in}{2.365737in}}{\pgfqpoint{1.274021in}{2.357501in}}%
\pgfpathcurveto{\pgfqpoint{1.274021in}{2.349264in}}{\pgfqpoint{1.277293in}{2.341364in}}{\pgfqpoint{1.283117in}{2.335540in}}%
\pgfpathcurveto{\pgfqpoint{1.288941in}{2.329717in}}{\pgfqpoint{1.296841in}{2.326444in}}{\pgfqpoint{1.305077in}{2.326444in}}%
\pgfpathclose%
\pgfusepath{stroke,fill}%
\end{pgfscope}%
\begin{pgfscope}%
\pgfpathrectangle{\pgfqpoint{0.100000in}{0.212622in}}{\pgfqpoint{3.696000in}{3.696000in}}%
\pgfusepath{clip}%
\pgfsetbuttcap%
\pgfsetroundjoin%
\definecolor{currentfill}{rgb}{0.121569,0.466667,0.705882}%
\pgfsetfillcolor{currentfill}%
\pgfsetfillopacity{0.483611}%
\pgfsetlinewidth{1.003750pt}%
\definecolor{currentstroke}{rgb}{0.121569,0.466667,0.705882}%
\pgfsetstrokecolor{currentstroke}%
\pgfsetstrokeopacity{0.483611}%
\pgfsetdash{}{0pt}%
\pgfpathmoveto{\pgfqpoint{2.029146in}{2.580755in}}%
\pgfpathcurveto{\pgfqpoint{2.037383in}{2.580755in}}{\pgfqpoint{2.045283in}{2.584028in}}{\pgfqpoint{2.051106in}{2.589851in}}%
\pgfpathcurveto{\pgfqpoint{2.056930in}{2.595675in}}{\pgfqpoint{2.060203in}{2.603575in}}{\pgfqpoint{2.060203in}{2.611812in}}%
\pgfpathcurveto{\pgfqpoint{2.060203in}{2.620048in}}{\pgfqpoint{2.056930in}{2.627948in}}{\pgfqpoint{2.051106in}{2.633772in}}%
\pgfpathcurveto{\pgfqpoint{2.045283in}{2.639596in}}{\pgfqpoint{2.037383in}{2.642868in}}{\pgfqpoint{2.029146in}{2.642868in}}%
\pgfpathcurveto{\pgfqpoint{2.020910in}{2.642868in}}{\pgfqpoint{2.013010in}{2.639596in}}{\pgfqpoint{2.007186in}{2.633772in}}%
\pgfpathcurveto{\pgfqpoint{2.001362in}{2.627948in}}{\pgfqpoint{1.998090in}{2.620048in}}{\pgfqpoint{1.998090in}{2.611812in}}%
\pgfpathcurveto{\pgfqpoint{1.998090in}{2.603575in}}{\pgfqpoint{2.001362in}{2.595675in}}{\pgfqpoint{2.007186in}{2.589851in}}%
\pgfpathcurveto{\pgfqpoint{2.013010in}{2.584028in}}{\pgfqpoint{2.020910in}{2.580755in}}{\pgfqpoint{2.029146in}{2.580755in}}%
\pgfpathclose%
\pgfusepath{stroke,fill}%
\end{pgfscope}%
\begin{pgfscope}%
\pgfpathrectangle{\pgfqpoint{0.100000in}{0.212622in}}{\pgfqpoint{3.696000in}{3.696000in}}%
\pgfusepath{clip}%
\pgfsetbuttcap%
\pgfsetroundjoin%
\definecolor{currentfill}{rgb}{0.121569,0.466667,0.705882}%
\pgfsetfillcolor{currentfill}%
\pgfsetfillopacity{0.484081}%
\pgfsetlinewidth{1.003750pt}%
\definecolor{currentstroke}{rgb}{0.121569,0.466667,0.705882}%
\pgfsetstrokecolor{currentstroke}%
\pgfsetstrokeopacity{0.484081}%
\pgfsetdash{}{0pt}%
\pgfpathmoveto{\pgfqpoint{1.303429in}{2.323437in}}%
\pgfpathcurveto{\pgfqpoint{1.311665in}{2.323437in}}{\pgfqpoint{1.319565in}{2.326710in}}{\pgfqpoint{1.325389in}{2.332534in}}%
\pgfpathcurveto{\pgfqpoint{1.331213in}{2.338357in}}{\pgfqpoint{1.334485in}{2.346258in}}{\pgfqpoint{1.334485in}{2.354494in}}%
\pgfpathcurveto{\pgfqpoint{1.334485in}{2.362730in}}{\pgfqpoint{1.331213in}{2.370630in}}{\pgfqpoint{1.325389in}{2.376454in}}%
\pgfpathcurveto{\pgfqpoint{1.319565in}{2.382278in}}{\pgfqpoint{1.311665in}{2.385550in}}{\pgfqpoint{1.303429in}{2.385550in}}%
\pgfpathcurveto{\pgfqpoint{1.295192in}{2.385550in}}{\pgfqpoint{1.287292in}{2.382278in}}{\pgfqpoint{1.281468in}{2.376454in}}%
\pgfpathcurveto{\pgfqpoint{1.275645in}{2.370630in}}{\pgfqpoint{1.272372in}{2.362730in}}{\pgfqpoint{1.272372in}{2.354494in}}%
\pgfpathcurveto{\pgfqpoint{1.272372in}{2.346258in}}{\pgfqpoint{1.275645in}{2.338357in}}{\pgfqpoint{1.281468in}{2.332534in}}%
\pgfpathcurveto{\pgfqpoint{1.287292in}{2.326710in}}{\pgfqpoint{1.295192in}{2.323437in}}{\pgfqpoint{1.303429in}{2.323437in}}%
\pgfpathclose%
\pgfusepath{stroke,fill}%
\end{pgfscope}%
\begin{pgfscope}%
\pgfpathrectangle{\pgfqpoint{0.100000in}{0.212622in}}{\pgfqpoint{3.696000in}{3.696000in}}%
\pgfusepath{clip}%
\pgfsetbuttcap%
\pgfsetroundjoin%
\definecolor{currentfill}{rgb}{0.121569,0.466667,0.705882}%
\pgfsetfillcolor{currentfill}%
\pgfsetfillopacity{0.484615}%
\pgfsetlinewidth{1.003750pt}%
\definecolor{currentstroke}{rgb}{0.121569,0.466667,0.705882}%
\pgfsetstrokecolor{currentstroke}%
\pgfsetstrokeopacity{0.484615}%
\pgfsetdash{}{0pt}%
\pgfpathmoveto{\pgfqpoint{2.029934in}{2.574772in}}%
\pgfpathcurveto{\pgfqpoint{2.038171in}{2.574772in}}{\pgfqpoint{2.046071in}{2.578044in}}{\pgfqpoint{2.051895in}{2.583868in}}%
\pgfpathcurveto{\pgfqpoint{2.057719in}{2.589692in}}{\pgfqpoint{2.060991in}{2.597592in}}{\pgfqpoint{2.060991in}{2.605828in}}%
\pgfpathcurveto{\pgfqpoint{2.060991in}{2.614065in}}{\pgfqpoint{2.057719in}{2.621965in}}{\pgfqpoint{2.051895in}{2.627788in}}%
\pgfpathcurveto{\pgfqpoint{2.046071in}{2.633612in}}{\pgfqpoint{2.038171in}{2.636885in}}{\pgfqpoint{2.029934in}{2.636885in}}%
\pgfpathcurveto{\pgfqpoint{2.021698in}{2.636885in}}{\pgfqpoint{2.013798in}{2.633612in}}{\pgfqpoint{2.007974in}{2.627788in}}%
\pgfpathcurveto{\pgfqpoint{2.002150in}{2.621965in}}{\pgfqpoint{1.998878in}{2.614065in}}{\pgfqpoint{1.998878in}{2.605828in}}%
\pgfpathcurveto{\pgfqpoint{1.998878in}{2.597592in}}{\pgfqpoint{2.002150in}{2.589692in}}{\pgfqpoint{2.007974in}{2.583868in}}%
\pgfpathcurveto{\pgfqpoint{2.013798in}{2.578044in}}{\pgfqpoint{2.021698in}{2.574772in}}{\pgfqpoint{2.029934in}{2.574772in}}%
\pgfpathclose%
\pgfusepath{stroke,fill}%
\end{pgfscope}%
\begin{pgfscope}%
\pgfpathrectangle{\pgfqpoint{0.100000in}{0.212622in}}{\pgfqpoint{3.696000in}{3.696000in}}%
\pgfusepath{clip}%
\pgfsetbuttcap%
\pgfsetroundjoin%
\definecolor{currentfill}{rgb}{0.121569,0.466667,0.705882}%
\pgfsetfillcolor{currentfill}%
\pgfsetfillopacity{0.485139}%
\pgfsetlinewidth{1.003750pt}%
\definecolor{currentstroke}{rgb}{0.121569,0.466667,0.705882}%
\pgfsetstrokecolor{currentstroke}%
\pgfsetstrokeopacity{0.485139}%
\pgfsetdash{}{0pt}%
\pgfpathmoveto{\pgfqpoint{1.300169in}{2.318096in}}%
\pgfpathcurveto{\pgfqpoint{1.308405in}{2.318096in}}{\pgfqpoint{1.316306in}{2.321369in}}{\pgfqpoint{1.322129in}{2.327193in}}%
\pgfpathcurveto{\pgfqpoint{1.327953in}{2.333017in}}{\pgfqpoint{1.331226in}{2.340917in}}{\pgfqpoint{1.331226in}{2.349153in}}%
\pgfpathcurveto{\pgfqpoint{1.331226in}{2.357389in}}{\pgfqpoint{1.327953in}{2.365289in}}{\pgfqpoint{1.322129in}{2.371113in}}%
\pgfpathcurveto{\pgfqpoint{1.316306in}{2.376937in}}{\pgfqpoint{1.308405in}{2.380209in}}{\pgfqpoint{1.300169in}{2.380209in}}%
\pgfpathcurveto{\pgfqpoint{1.291933in}{2.380209in}}{\pgfqpoint{1.284033in}{2.376937in}}{\pgfqpoint{1.278209in}{2.371113in}}%
\pgfpathcurveto{\pgfqpoint{1.272385in}{2.365289in}}{\pgfqpoint{1.269113in}{2.357389in}}{\pgfqpoint{1.269113in}{2.349153in}}%
\pgfpathcurveto{\pgfqpoint{1.269113in}{2.340917in}}{\pgfqpoint{1.272385in}{2.333017in}}{\pgfqpoint{1.278209in}{2.327193in}}%
\pgfpathcurveto{\pgfqpoint{1.284033in}{2.321369in}}{\pgfqpoint{1.291933in}{2.318096in}}{\pgfqpoint{1.300169in}{2.318096in}}%
\pgfpathclose%
\pgfusepath{stroke,fill}%
\end{pgfscope}%
\begin{pgfscope}%
\pgfpathrectangle{\pgfqpoint{0.100000in}{0.212622in}}{\pgfqpoint{3.696000in}{3.696000in}}%
\pgfusepath{clip}%
\pgfsetbuttcap%
\pgfsetroundjoin%
\definecolor{currentfill}{rgb}{0.121569,0.466667,0.705882}%
\pgfsetfillcolor{currentfill}%
\pgfsetfillopacity{0.485933}%
\pgfsetlinewidth{1.003750pt}%
\definecolor{currentstroke}{rgb}{0.121569,0.466667,0.705882}%
\pgfsetstrokecolor{currentstroke}%
\pgfsetstrokeopacity{0.485933}%
\pgfsetdash{}{0pt}%
\pgfpathmoveto{\pgfqpoint{2.030353in}{2.568699in}}%
\pgfpathcurveto{\pgfqpoint{2.038590in}{2.568699in}}{\pgfqpoint{2.046490in}{2.571971in}}{\pgfqpoint{2.052314in}{2.577795in}}%
\pgfpathcurveto{\pgfqpoint{2.058137in}{2.583619in}}{\pgfqpoint{2.061410in}{2.591519in}}{\pgfqpoint{2.061410in}{2.599755in}}%
\pgfpathcurveto{\pgfqpoint{2.061410in}{2.607992in}}{\pgfqpoint{2.058137in}{2.615892in}}{\pgfqpoint{2.052314in}{2.621716in}}%
\pgfpathcurveto{\pgfqpoint{2.046490in}{2.627539in}}{\pgfqpoint{2.038590in}{2.630812in}}{\pgfqpoint{2.030353in}{2.630812in}}%
\pgfpathcurveto{\pgfqpoint{2.022117in}{2.630812in}}{\pgfqpoint{2.014217in}{2.627539in}}{\pgfqpoint{2.008393in}{2.621716in}}%
\pgfpathcurveto{\pgfqpoint{2.002569in}{2.615892in}}{\pgfqpoint{1.999297in}{2.607992in}}{\pgfqpoint{1.999297in}{2.599755in}}%
\pgfpathcurveto{\pgfqpoint{1.999297in}{2.591519in}}{\pgfqpoint{2.002569in}{2.583619in}}{\pgfqpoint{2.008393in}{2.577795in}}%
\pgfpathcurveto{\pgfqpoint{2.014217in}{2.571971in}}{\pgfqpoint{2.022117in}{2.568699in}}{\pgfqpoint{2.030353in}{2.568699in}}%
\pgfpathclose%
\pgfusepath{stroke,fill}%
\end{pgfscope}%
\begin{pgfscope}%
\pgfpathrectangle{\pgfqpoint{0.100000in}{0.212622in}}{\pgfqpoint{3.696000in}{3.696000in}}%
\pgfusepath{clip}%
\pgfsetbuttcap%
\pgfsetroundjoin%
\definecolor{currentfill}{rgb}{0.121569,0.466667,0.705882}%
\pgfsetfillcolor{currentfill}%
\pgfsetfillopacity{0.486088}%
\pgfsetlinewidth{1.003750pt}%
\definecolor{currentstroke}{rgb}{0.121569,0.466667,0.705882}%
\pgfsetstrokecolor{currentstroke}%
\pgfsetstrokeopacity{0.486088}%
\pgfsetdash{}{0pt}%
\pgfpathmoveto{\pgfqpoint{1.297769in}{2.313359in}}%
\pgfpathcurveto{\pgfqpoint{1.306006in}{2.313359in}}{\pgfqpoint{1.313906in}{2.316631in}}{\pgfqpoint{1.319730in}{2.322455in}}%
\pgfpathcurveto{\pgfqpoint{1.325554in}{2.328279in}}{\pgfqpoint{1.328826in}{2.336179in}}{\pgfqpoint{1.328826in}{2.344415in}}%
\pgfpathcurveto{\pgfqpoint{1.328826in}{2.352652in}}{\pgfqpoint{1.325554in}{2.360552in}}{\pgfqpoint{1.319730in}{2.366375in}}%
\pgfpathcurveto{\pgfqpoint{1.313906in}{2.372199in}}{\pgfqpoint{1.306006in}{2.375472in}}{\pgfqpoint{1.297769in}{2.375472in}}%
\pgfpathcurveto{\pgfqpoint{1.289533in}{2.375472in}}{\pgfqpoint{1.281633in}{2.372199in}}{\pgfqpoint{1.275809in}{2.366375in}}%
\pgfpathcurveto{\pgfqpoint{1.269985in}{2.360552in}}{\pgfqpoint{1.266713in}{2.352652in}}{\pgfqpoint{1.266713in}{2.344415in}}%
\pgfpathcurveto{\pgfqpoint{1.266713in}{2.336179in}}{\pgfqpoint{1.269985in}{2.328279in}}{\pgfqpoint{1.275809in}{2.322455in}}%
\pgfpathcurveto{\pgfqpoint{1.281633in}{2.316631in}}{\pgfqpoint{1.289533in}{2.313359in}}{\pgfqpoint{1.297769in}{2.313359in}}%
\pgfpathclose%
\pgfusepath{stroke,fill}%
\end{pgfscope}%
\begin{pgfscope}%
\pgfpathrectangle{\pgfqpoint{0.100000in}{0.212622in}}{\pgfqpoint{3.696000in}{3.696000in}}%
\pgfusepath{clip}%
\pgfsetbuttcap%
\pgfsetroundjoin%
\definecolor{currentfill}{rgb}{0.121569,0.466667,0.705882}%
\pgfsetfillcolor{currentfill}%
\pgfsetfillopacity{0.486692}%
\pgfsetlinewidth{1.003750pt}%
\definecolor{currentstroke}{rgb}{0.121569,0.466667,0.705882}%
\pgfsetstrokecolor{currentstroke}%
\pgfsetstrokeopacity{0.486692}%
\pgfsetdash{}{0pt}%
\pgfpathmoveto{\pgfqpoint{1.295710in}{2.309977in}}%
\pgfpathcurveto{\pgfqpoint{1.303946in}{2.309977in}}{\pgfqpoint{1.311846in}{2.313249in}}{\pgfqpoint{1.317670in}{2.319073in}}%
\pgfpathcurveto{\pgfqpoint{1.323494in}{2.324897in}}{\pgfqpoint{1.326766in}{2.332797in}}{\pgfqpoint{1.326766in}{2.341034in}}%
\pgfpathcurveto{\pgfqpoint{1.326766in}{2.349270in}}{\pgfqpoint{1.323494in}{2.357170in}}{\pgfqpoint{1.317670in}{2.362994in}}%
\pgfpathcurveto{\pgfqpoint{1.311846in}{2.368818in}}{\pgfqpoint{1.303946in}{2.372090in}}{\pgfqpoint{1.295710in}{2.372090in}}%
\pgfpathcurveto{\pgfqpoint{1.287474in}{2.372090in}}{\pgfqpoint{1.279574in}{2.368818in}}{\pgfqpoint{1.273750in}{2.362994in}}%
\pgfpathcurveto{\pgfqpoint{1.267926in}{2.357170in}}{\pgfqpoint{1.264653in}{2.349270in}}{\pgfqpoint{1.264653in}{2.341034in}}%
\pgfpathcurveto{\pgfqpoint{1.264653in}{2.332797in}}{\pgfqpoint{1.267926in}{2.324897in}}{\pgfqpoint{1.273750in}{2.319073in}}%
\pgfpathcurveto{\pgfqpoint{1.279574in}{2.313249in}}{\pgfqpoint{1.287474in}{2.309977in}}{\pgfqpoint{1.295710in}{2.309977in}}%
\pgfpathclose%
\pgfusepath{stroke,fill}%
\end{pgfscope}%
\begin{pgfscope}%
\pgfpathrectangle{\pgfqpoint{0.100000in}{0.212622in}}{\pgfqpoint{3.696000in}{3.696000in}}%
\pgfusepath{clip}%
\pgfsetbuttcap%
\pgfsetroundjoin%
\definecolor{currentfill}{rgb}{0.121569,0.466667,0.705882}%
\pgfsetfillcolor{currentfill}%
\pgfsetfillopacity{0.487105}%
\pgfsetlinewidth{1.003750pt}%
\definecolor{currentstroke}{rgb}{0.121569,0.466667,0.705882}%
\pgfsetstrokecolor{currentstroke}%
\pgfsetstrokeopacity{0.487105}%
\pgfsetdash{}{0pt}%
\pgfpathmoveto{\pgfqpoint{1.294531in}{2.307564in}}%
\pgfpathcurveto{\pgfqpoint{1.302768in}{2.307564in}}{\pgfqpoint{1.310668in}{2.310836in}}{\pgfqpoint{1.316492in}{2.316660in}}%
\pgfpathcurveto{\pgfqpoint{1.322316in}{2.322484in}}{\pgfqpoint{1.325588in}{2.330384in}}{\pgfqpoint{1.325588in}{2.338621in}}%
\pgfpathcurveto{\pgfqpoint{1.325588in}{2.346857in}}{\pgfqpoint{1.322316in}{2.354757in}}{\pgfqpoint{1.316492in}{2.360581in}}%
\pgfpathcurveto{\pgfqpoint{1.310668in}{2.366405in}}{\pgfqpoint{1.302768in}{2.369677in}}{\pgfqpoint{1.294531in}{2.369677in}}%
\pgfpathcurveto{\pgfqpoint{1.286295in}{2.369677in}}{\pgfqpoint{1.278395in}{2.366405in}}{\pgfqpoint{1.272571in}{2.360581in}}%
\pgfpathcurveto{\pgfqpoint{1.266747in}{2.354757in}}{\pgfqpoint{1.263475in}{2.346857in}}{\pgfqpoint{1.263475in}{2.338621in}}%
\pgfpathcurveto{\pgfqpoint{1.263475in}{2.330384in}}{\pgfqpoint{1.266747in}{2.322484in}}{\pgfqpoint{1.272571in}{2.316660in}}%
\pgfpathcurveto{\pgfqpoint{1.278395in}{2.310836in}}{\pgfqpoint{1.286295in}{2.307564in}}{\pgfqpoint{1.294531in}{2.307564in}}%
\pgfpathclose%
\pgfusepath{stroke,fill}%
\end{pgfscope}%
\begin{pgfscope}%
\pgfpathrectangle{\pgfqpoint{0.100000in}{0.212622in}}{\pgfqpoint{3.696000in}{3.696000in}}%
\pgfusepath{clip}%
\pgfsetbuttcap%
\pgfsetroundjoin%
\definecolor{currentfill}{rgb}{0.121569,0.466667,0.705882}%
\pgfsetfillcolor{currentfill}%
\pgfsetfillopacity{0.487127}%
\pgfsetlinewidth{1.003750pt}%
\definecolor{currentstroke}{rgb}{0.121569,0.466667,0.705882}%
\pgfsetstrokecolor{currentstroke}%
\pgfsetstrokeopacity{0.487127}%
\pgfsetdash{}{0pt}%
\pgfpathmoveto{\pgfqpoint{1.294458in}{2.307429in}}%
\pgfpathcurveto{\pgfqpoint{1.302694in}{2.307429in}}{\pgfqpoint{1.310594in}{2.310701in}}{\pgfqpoint{1.316418in}{2.316525in}}%
\pgfpathcurveto{\pgfqpoint{1.322242in}{2.322349in}}{\pgfqpoint{1.325514in}{2.330249in}}{\pgfqpoint{1.325514in}{2.338485in}}%
\pgfpathcurveto{\pgfqpoint{1.325514in}{2.346722in}}{\pgfqpoint{1.322242in}{2.354622in}}{\pgfqpoint{1.316418in}{2.360446in}}%
\pgfpathcurveto{\pgfqpoint{1.310594in}{2.366270in}}{\pgfqpoint{1.302694in}{2.369542in}}{\pgfqpoint{1.294458in}{2.369542in}}%
\pgfpathcurveto{\pgfqpoint{1.286221in}{2.369542in}}{\pgfqpoint{1.278321in}{2.366270in}}{\pgfqpoint{1.272497in}{2.360446in}}%
\pgfpathcurveto{\pgfqpoint{1.266673in}{2.354622in}}{\pgfqpoint{1.263401in}{2.346722in}}{\pgfqpoint{1.263401in}{2.338485in}}%
\pgfpathcurveto{\pgfqpoint{1.263401in}{2.330249in}}{\pgfqpoint{1.266673in}{2.322349in}}{\pgfqpoint{1.272497in}{2.316525in}}%
\pgfpathcurveto{\pgfqpoint{1.278321in}{2.310701in}}{\pgfqpoint{1.286221in}{2.307429in}}{\pgfqpoint{1.294458in}{2.307429in}}%
\pgfpathclose%
\pgfusepath{stroke,fill}%
\end{pgfscope}%
\begin{pgfscope}%
\pgfpathrectangle{\pgfqpoint{0.100000in}{0.212622in}}{\pgfqpoint{3.696000in}{3.696000in}}%
\pgfusepath{clip}%
\pgfsetbuttcap%
\pgfsetroundjoin%
\definecolor{currentfill}{rgb}{0.121569,0.466667,0.705882}%
\pgfsetfillcolor{currentfill}%
\pgfsetfillopacity{0.487170}%
\pgfsetlinewidth{1.003750pt}%
\definecolor{currentstroke}{rgb}{0.121569,0.466667,0.705882}%
\pgfsetstrokecolor{currentstroke}%
\pgfsetstrokeopacity{0.487170}%
\pgfsetdash{}{0pt}%
\pgfpathmoveto{\pgfqpoint{1.294341in}{2.307169in}}%
\pgfpathcurveto{\pgfqpoint{1.302577in}{2.307169in}}{\pgfqpoint{1.310477in}{2.310442in}}{\pgfqpoint{1.316301in}{2.316266in}}%
\pgfpathcurveto{\pgfqpoint{1.322125in}{2.322089in}}{\pgfqpoint{1.325397in}{2.329990in}}{\pgfqpoint{1.325397in}{2.338226in}}%
\pgfpathcurveto{\pgfqpoint{1.325397in}{2.346462in}}{\pgfqpoint{1.322125in}{2.354362in}}{\pgfqpoint{1.316301in}{2.360186in}}%
\pgfpathcurveto{\pgfqpoint{1.310477in}{2.366010in}}{\pgfqpoint{1.302577in}{2.369282in}}{\pgfqpoint{1.294341in}{2.369282in}}%
\pgfpathcurveto{\pgfqpoint{1.286104in}{2.369282in}}{\pgfqpoint{1.278204in}{2.366010in}}{\pgfqpoint{1.272380in}{2.360186in}}%
\pgfpathcurveto{\pgfqpoint{1.266556in}{2.354362in}}{\pgfqpoint{1.263284in}{2.346462in}}{\pgfqpoint{1.263284in}{2.338226in}}%
\pgfpathcurveto{\pgfqpoint{1.263284in}{2.329990in}}{\pgfqpoint{1.266556in}{2.322089in}}{\pgfqpoint{1.272380in}{2.316266in}}%
\pgfpathcurveto{\pgfqpoint{1.278204in}{2.310442in}}{\pgfqpoint{1.286104in}{2.307169in}}{\pgfqpoint{1.294341in}{2.307169in}}%
\pgfpathclose%
\pgfusepath{stroke,fill}%
\end{pgfscope}%
\begin{pgfscope}%
\pgfpathrectangle{\pgfqpoint{0.100000in}{0.212622in}}{\pgfqpoint{3.696000in}{3.696000in}}%
\pgfusepath{clip}%
\pgfsetbuttcap%
\pgfsetroundjoin%
\definecolor{currentfill}{rgb}{0.121569,0.466667,0.705882}%
\pgfsetfillcolor{currentfill}%
\pgfsetfillopacity{0.487254}%
\pgfsetlinewidth{1.003750pt}%
\definecolor{currentstroke}{rgb}{0.121569,0.466667,0.705882}%
\pgfsetstrokecolor{currentstroke}%
\pgfsetstrokeopacity{0.487254}%
\pgfsetdash{}{0pt}%
\pgfpathmoveto{\pgfqpoint{1.294097in}{2.306757in}}%
\pgfpathcurveto{\pgfqpoint{1.302333in}{2.306757in}}{\pgfqpoint{1.310233in}{2.310029in}}{\pgfqpoint{1.316057in}{2.315853in}}%
\pgfpathcurveto{\pgfqpoint{1.321881in}{2.321677in}}{\pgfqpoint{1.325153in}{2.329577in}}{\pgfqpoint{1.325153in}{2.337813in}}%
\pgfpathcurveto{\pgfqpoint{1.325153in}{2.346050in}}{\pgfqpoint{1.321881in}{2.353950in}}{\pgfqpoint{1.316057in}{2.359774in}}%
\pgfpathcurveto{\pgfqpoint{1.310233in}{2.365598in}}{\pgfqpoint{1.302333in}{2.368870in}}{\pgfqpoint{1.294097in}{2.368870in}}%
\pgfpathcurveto{\pgfqpoint{1.285861in}{2.368870in}}{\pgfqpoint{1.277961in}{2.365598in}}{\pgfqpoint{1.272137in}{2.359774in}}%
\pgfpathcurveto{\pgfqpoint{1.266313in}{2.353950in}}{\pgfqpoint{1.263040in}{2.346050in}}{\pgfqpoint{1.263040in}{2.337813in}}%
\pgfpathcurveto{\pgfqpoint{1.263040in}{2.329577in}}{\pgfqpoint{1.266313in}{2.321677in}}{\pgfqpoint{1.272137in}{2.315853in}}%
\pgfpathcurveto{\pgfqpoint{1.277961in}{2.310029in}}{\pgfqpoint{1.285861in}{2.306757in}}{\pgfqpoint{1.294097in}{2.306757in}}%
\pgfpathclose%
\pgfusepath{stroke,fill}%
\end{pgfscope}%
\begin{pgfscope}%
\pgfpathrectangle{\pgfqpoint{0.100000in}{0.212622in}}{\pgfqpoint{3.696000in}{3.696000in}}%
\pgfusepath{clip}%
\pgfsetbuttcap%
\pgfsetroundjoin%
\definecolor{currentfill}{rgb}{0.121569,0.466667,0.705882}%
\pgfsetfillcolor{currentfill}%
\pgfsetfillopacity{0.487369}%
\pgfsetlinewidth{1.003750pt}%
\definecolor{currentstroke}{rgb}{0.121569,0.466667,0.705882}%
\pgfsetstrokecolor{currentstroke}%
\pgfsetstrokeopacity{0.487369}%
\pgfsetdash{}{0pt}%
\pgfpathmoveto{\pgfqpoint{2.031208in}{2.561755in}}%
\pgfpathcurveto{\pgfqpoint{2.039444in}{2.561755in}}{\pgfqpoint{2.047344in}{2.565027in}}{\pgfqpoint{2.053168in}{2.570851in}}%
\pgfpathcurveto{\pgfqpoint{2.058992in}{2.576675in}}{\pgfqpoint{2.062264in}{2.584575in}}{\pgfqpoint{2.062264in}{2.592811in}}%
\pgfpathcurveto{\pgfqpoint{2.062264in}{2.601047in}}{\pgfqpoint{2.058992in}{2.608947in}}{\pgfqpoint{2.053168in}{2.614771in}}%
\pgfpathcurveto{\pgfqpoint{2.047344in}{2.620595in}}{\pgfqpoint{2.039444in}{2.623868in}}{\pgfqpoint{2.031208in}{2.623868in}}%
\pgfpathcurveto{\pgfqpoint{2.022971in}{2.623868in}}{\pgfqpoint{2.015071in}{2.620595in}}{\pgfqpoint{2.009247in}{2.614771in}}%
\pgfpathcurveto{\pgfqpoint{2.003423in}{2.608947in}}{\pgfqpoint{2.000151in}{2.601047in}}{\pgfqpoint{2.000151in}{2.592811in}}%
\pgfpathcurveto{\pgfqpoint{2.000151in}{2.584575in}}{\pgfqpoint{2.003423in}{2.576675in}}{\pgfqpoint{2.009247in}{2.570851in}}%
\pgfpathcurveto{\pgfqpoint{2.015071in}{2.565027in}}{\pgfqpoint{2.022971in}{2.561755in}}{\pgfqpoint{2.031208in}{2.561755in}}%
\pgfpathclose%
\pgfusepath{stroke,fill}%
\end{pgfscope}%
\begin{pgfscope}%
\pgfpathrectangle{\pgfqpoint{0.100000in}{0.212622in}}{\pgfqpoint{3.696000in}{3.696000in}}%
\pgfusepath{clip}%
\pgfsetbuttcap%
\pgfsetroundjoin%
\definecolor{currentfill}{rgb}{0.121569,0.466667,0.705882}%
\pgfsetfillcolor{currentfill}%
\pgfsetfillopacity{0.487419}%
\pgfsetlinewidth{1.003750pt}%
\definecolor{currentstroke}{rgb}{0.121569,0.466667,0.705882}%
\pgfsetstrokecolor{currentstroke}%
\pgfsetstrokeopacity{0.487419}%
\pgfsetdash{}{0pt}%
\pgfpathmoveto{\pgfqpoint{1.293685in}{2.306015in}}%
\pgfpathcurveto{\pgfqpoint{1.301921in}{2.306015in}}{\pgfqpoint{1.309821in}{2.309287in}}{\pgfqpoint{1.315645in}{2.315111in}}%
\pgfpathcurveto{\pgfqpoint{1.321469in}{2.320935in}}{\pgfqpoint{1.324741in}{2.328835in}}{\pgfqpoint{1.324741in}{2.337071in}}%
\pgfpathcurveto{\pgfqpoint{1.324741in}{2.345307in}}{\pgfqpoint{1.321469in}{2.353208in}}{\pgfqpoint{1.315645in}{2.359031in}}%
\pgfpathcurveto{\pgfqpoint{1.309821in}{2.364855in}}{\pgfqpoint{1.301921in}{2.368128in}}{\pgfqpoint{1.293685in}{2.368128in}}%
\pgfpathcurveto{\pgfqpoint{1.285448in}{2.368128in}}{\pgfqpoint{1.277548in}{2.364855in}}{\pgfqpoint{1.271724in}{2.359031in}}%
\pgfpathcurveto{\pgfqpoint{1.265901in}{2.353208in}}{\pgfqpoint{1.262628in}{2.345307in}}{\pgfqpoint{1.262628in}{2.337071in}}%
\pgfpathcurveto{\pgfqpoint{1.262628in}{2.328835in}}{\pgfqpoint{1.265901in}{2.320935in}}{\pgfqpoint{1.271724in}{2.315111in}}%
\pgfpathcurveto{\pgfqpoint{1.277548in}{2.309287in}}{\pgfqpoint{1.285448in}{2.306015in}}{\pgfqpoint{1.293685in}{2.306015in}}%
\pgfpathclose%
\pgfusepath{stroke,fill}%
\end{pgfscope}%
\begin{pgfscope}%
\pgfpathrectangle{\pgfqpoint{0.100000in}{0.212622in}}{\pgfqpoint{3.696000in}{3.696000in}}%
\pgfusepath{clip}%
\pgfsetbuttcap%
\pgfsetroundjoin%
\definecolor{currentfill}{rgb}{0.121569,0.466667,0.705882}%
\pgfsetfillcolor{currentfill}%
\pgfsetfillopacity{0.487697}%
\pgfsetlinewidth{1.003750pt}%
\definecolor{currentstroke}{rgb}{0.121569,0.466667,0.705882}%
\pgfsetstrokecolor{currentstroke}%
\pgfsetstrokeopacity{0.487697}%
\pgfsetdash{}{0pt}%
\pgfpathmoveto{\pgfqpoint{1.292865in}{2.304673in}}%
\pgfpathcurveto{\pgfqpoint{1.301101in}{2.304673in}}{\pgfqpoint{1.309001in}{2.307946in}}{\pgfqpoint{1.314825in}{2.313770in}}%
\pgfpathcurveto{\pgfqpoint{1.320649in}{2.319594in}}{\pgfqpoint{1.323921in}{2.327494in}}{\pgfqpoint{1.323921in}{2.335730in}}%
\pgfpathcurveto{\pgfqpoint{1.323921in}{2.343966in}}{\pgfqpoint{1.320649in}{2.351866in}}{\pgfqpoint{1.314825in}{2.357690in}}%
\pgfpathcurveto{\pgfqpoint{1.309001in}{2.363514in}}{\pgfqpoint{1.301101in}{2.366786in}}{\pgfqpoint{1.292865in}{2.366786in}}%
\pgfpathcurveto{\pgfqpoint{1.284628in}{2.366786in}}{\pgfqpoint{1.276728in}{2.363514in}}{\pgfqpoint{1.270904in}{2.357690in}}%
\pgfpathcurveto{\pgfqpoint{1.265080in}{2.351866in}}{\pgfqpoint{1.261808in}{2.343966in}}{\pgfqpoint{1.261808in}{2.335730in}}%
\pgfpathcurveto{\pgfqpoint{1.261808in}{2.327494in}}{\pgfqpoint{1.265080in}{2.319594in}}{\pgfqpoint{1.270904in}{2.313770in}}%
\pgfpathcurveto{\pgfqpoint{1.276728in}{2.307946in}}{\pgfqpoint{1.284628in}{2.304673in}}{\pgfqpoint{1.292865in}{2.304673in}}%
\pgfpathclose%
\pgfusepath{stroke,fill}%
\end{pgfscope}%
\begin{pgfscope}%
\pgfpathrectangle{\pgfqpoint{0.100000in}{0.212622in}}{\pgfqpoint{3.696000in}{3.696000in}}%
\pgfusepath{clip}%
\pgfsetbuttcap%
\pgfsetroundjoin%
\definecolor{currentfill}{rgb}{0.121569,0.466667,0.705882}%
\pgfsetfillcolor{currentfill}%
\pgfsetfillopacity{0.488212}%
\pgfsetlinewidth{1.003750pt}%
\definecolor{currentstroke}{rgb}{0.121569,0.466667,0.705882}%
\pgfsetstrokecolor{currentstroke}%
\pgfsetstrokeopacity{0.488212}%
\pgfsetdash{}{0pt}%
\pgfpathmoveto{\pgfqpoint{1.291474in}{2.302132in}}%
\pgfpathcurveto{\pgfqpoint{1.299710in}{2.302132in}}{\pgfqpoint{1.307610in}{2.305404in}}{\pgfqpoint{1.313434in}{2.311228in}}%
\pgfpathcurveto{\pgfqpoint{1.319258in}{2.317052in}}{\pgfqpoint{1.322530in}{2.324952in}}{\pgfqpoint{1.322530in}{2.333188in}}%
\pgfpathcurveto{\pgfqpoint{1.322530in}{2.341425in}}{\pgfqpoint{1.319258in}{2.349325in}}{\pgfqpoint{1.313434in}{2.355149in}}%
\pgfpathcurveto{\pgfqpoint{1.307610in}{2.360973in}}{\pgfqpoint{1.299710in}{2.364245in}}{\pgfqpoint{1.291474in}{2.364245in}}%
\pgfpathcurveto{\pgfqpoint{1.283238in}{2.364245in}}{\pgfqpoint{1.275338in}{2.360973in}}{\pgfqpoint{1.269514in}{2.355149in}}%
\pgfpathcurveto{\pgfqpoint{1.263690in}{2.349325in}}{\pgfqpoint{1.260417in}{2.341425in}}{\pgfqpoint{1.260417in}{2.333188in}}%
\pgfpathcurveto{\pgfqpoint{1.260417in}{2.324952in}}{\pgfqpoint{1.263690in}{2.317052in}}{\pgfqpoint{1.269514in}{2.311228in}}%
\pgfpathcurveto{\pgfqpoint{1.275338in}{2.305404in}}{\pgfqpoint{1.283238in}{2.302132in}}{\pgfqpoint{1.291474in}{2.302132in}}%
\pgfpathclose%
\pgfusepath{stroke,fill}%
\end{pgfscope}%
\begin{pgfscope}%
\pgfpathrectangle{\pgfqpoint{0.100000in}{0.212622in}}{\pgfqpoint{3.696000in}{3.696000in}}%
\pgfusepath{clip}%
\pgfsetbuttcap%
\pgfsetroundjoin%
\definecolor{currentfill}{rgb}{0.121569,0.466667,0.705882}%
\pgfsetfillcolor{currentfill}%
\pgfsetfillopacity{0.488597}%
\pgfsetlinewidth{1.003750pt}%
\definecolor{currentstroke}{rgb}{0.121569,0.466667,0.705882}%
\pgfsetstrokecolor{currentstroke}%
\pgfsetstrokeopacity{0.488597}%
\pgfsetdash{}{0pt}%
\pgfpathmoveto{\pgfqpoint{1.290440in}{2.300188in}}%
\pgfpathcurveto{\pgfqpoint{1.298676in}{2.300188in}}{\pgfqpoint{1.306576in}{2.303461in}}{\pgfqpoint{1.312400in}{2.309285in}}%
\pgfpathcurveto{\pgfqpoint{1.318224in}{2.315109in}}{\pgfqpoint{1.321496in}{2.323009in}}{\pgfqpoint{1.321496in}{2.331245in}}%
\pgfpathcurveto{\pgfqpoint{1.321496in}{2.339481in}}{\pgfqpoint{1.318224in}{2.347381in}}{\pgfqpoint{1.312400in}{2.353205in}}%
\pgfpathcurveto{\pgfqpoint{1.306576in}{2.359029in}}{\pgfqpoint{1.298676in}{2.362301in}}{\pgfqpoint{1.290440in}{2.362301in}}%
\pgfpathcurveto{\pgfqpoint{1.282204in}{2.362301in}}{\pgfqpoint{1.274304in}{2.359029in}}{\pgfqpoint{1.268480in}{2.353205in}}%
\pgfpathcurveto{\pgfqpoint{1.262656in}{2.347381in}}{\pgfqpoint{1.259383in}{2.339481in}}{\pgfqpoint{1.259383in}{2.331245in}}%
\pgfpathcurveto{\pgfqpoint{1.259383in}{2.323009in}}{\pgfqpoint{1.262656in}{2.315109in}}{\pgfqpoint{1.268480in}{2.309285in}}%
\pgfpathcurveto{\pgfqpoint{1.274304in}{2.303461in}}{\pgfqpoint{1.282204in}{2.300188in}}{\pgfqpoint{1.290440in}{2.300188in}}%
\pgfpathclose%
\pgfusepath{stroke,fill}%
\end{pgfscope}%
\begin{pgfscope}%
\pgfpathrectangle{\pgfqpoint{0.100000in}{0.212622in}}{\pgfqpoint{3.696000in}{3.696000in}}%
\pgfusepath{clip}%
\pgfsetbuttcap%
\pgfsetroundjoin%
\definecolor{currentfill}{rgb}{0.121569,0.466667,0.705882}%
\pgfsetfillcolor{currentfill}%
\pgfsetfillopacity{0.488841}%
\pgfsetlinewidth{1.003750pt}%
\definecolor{currentstroke}{rgb}{0.121569,0.466667,0.705882}%
\pgfsetstrokecolor{currentstroke}%
\pgfsetstrokeopacity{0.488841}%
\pgfsetdash{}{0pt}%
\pgfpathmoveto{\pgfqpoint{1.289673in}{2.298893in}}%
\pgfpathcurveto{\pgfqpoint{1.297909in}{2.298893in}}{\pgfqpoint{1.305809in}{2.302166in}}{\pgfqpoint{1.311633in}{2.307990in}}%
\pgfpathcurveto{\pgfqpoint{1.317457in}{2.313814in}}{\pgfqpoint{1.320729in}{2.321714in}}{\pgfqpoint{1.320729in}{2.329950in}}%
\pgfpathcurveto{\pgfqpoint{1.320729in}{2.338186in}}{\pgfqpoint{1.317457in}{2.346086in}}{\pgfqpoint{1.311633in}{2.351910in}}%
\pgfpathcurveto{\pgfqpoint{1.305809in}{2.357734in}}{\pgfqpoint{1.297909in}{2.361006in}}{\pgfqpoint{1.289673in}{2.361006in}}%
\pgfpathcurveto{\pgfqpoint{1.281436in}{2.361006in}}{\pgfqpoint{1.273536in}{2.357734in}}{\pgfqpoint{1.267712in}{2.351910in}}%
\pgfpathcurveto{\pgfqpoint{1.261888in}{2.346086in}}{\pgfqpoint{1.258616in}{2.338186in}}{\pgfqpoint{1.258616in}{2.329950in}}%
\pgfpathcurveto{\pgfqpoint{1.258616in}{2.321714in}}{\pgfqpoint{1.261888in}{2.313814in}}{\pgfqpoint{1.267712in}{2.307990in}}%
\pgfpathcurveto{\pgfqpoint{1.273536in}{2.302166in}}{\pgfqpoint{1.281436in}{2.298893in}}{\pgfqpoint{1.289673in}{2.298893in}}%
\pgfpathclose%
\pgfusepath{stroke,fill}%
\end{pgfscope}%
\begin{pgfscope}%
\pgfpathrectangle{\pgfqpoint{0.100000in}{0.212622in}}{\pgfqpoint{3.696000in}{3.696000in}}%
\pgfusepath{clip}%
\pgfsetbuttcap%
\pgfsetroundjoin%
\definecolor{currentfill}{rgb}{0.121569,0.466667,0.705882}%
\pgfsetfillcolor{currentfill}%
\pgfsetfillopacity{0.488841}%
\pgfsetlinewidth{1.003750pt}%
\definecolor{currentstroke}{rgb}{0.121569,0.466667,0.705882}%
\pgfsetstrokecolor{currentstroke}%
\pgfsetstrokeopacity{0.488841}%
\pgfsetdash{}{0pt}%
\pgfpathmoveto{\pgfqpoint{2.032398in}{2.553712in}}%
\pgfpathcurveto{\pgfqpoint{2.040635in}{2.553712in}}{\pgfqpoint{2.048535in}{2.556984in}}{\pgfqpoint{2.054359in}{2.562808in}}%
\pgfpathcurveto{\pgfqpoint{2.060183in}{2.568632in}}{\pgfqpoint{2.063455in}{2.576532in}}{\pgfqpoint{2.063455in}{2.584768in}}%
\pgfpathcurveto{\pgfqpoint{2.063455in}{2.593005in}}{\pgfqpoint{2.060183in}{2.600905in}}{\pgfqpoint{2.054359in}{2.606729in}}%
\pgfpathcurveto{\pgfqpoint{2.048535in}{2.612552in}}{\pgfqpoint{2.040635in}{2.615825in}}{\pgfqpoint{2.032398in}{2.615825in}}%
\pgfpathcurveto{\pgfqpoint{2.024162in}{2.615825in}}{\pgfqpoint{2.016262in}{2.612552in}}{\pgfqpoint{2.010438in}{2.606729in}}%
\pgfpathcurveto{\pgfqpoint{2.004614in}{2.600905in}}{\pgfqpoint{2.001342in}{2.593005in}}{\pgfqpoint{2.001342in}{2.584768in}}%
\pgfpathcurveto{\pgfqpoint{2.001342in}{2.576532in}}{\pgfqpoint{2.004614in}{2.568632in}}{\pgfqpoint{2.010438in}{2.562808in}}%
\pgfpathcurveto{\pgfqpoint{2.016262in}{2.556984in}}{\pgfqpoint{2.024162in}{2.553712in}}{\pgfqpoint{2.032398in}{2.553712in}}%
\pgfpathclose%
\pgfusepath{stroke,fill}%
\end{pgfscope}%
\begin{pgfscope}%
\pgfpathrectangle{\pgfqpoint{0.100000in}{0.212622in}}{\pgfqpoint{3.696000in}{3.696000in}}%
\pgfusepath{clip}%
\pgfsetbuttcap%
\pgfsetroundjoin%
\definecolor{currentfill}{rgb}{0.121569,0.466667,0.705882}%
\pgfsetfillcolor{currentfill}%
\pgfsetfillopacity{0.489287}%
\pgfsetlinewidth{1.003750pt}%
\definecolor{currentstroke}{rgb}{0.121569,0.466667,0.705882}%
\pgfsetstrokecolor{currentstroke}%
\pgfsetstrokeopacity{0.489287}%
\pgfsetdash{}{0pt}%
\pgfpathmoveto{\pgfqpoint{1.288423in}{2.296362in}}%
\pgfpathcurveto{\pgfqpoint{1.296660in}{2.296362in}}{\pgfqpoint{1.304560in}{2.299635in}}{\pgfqpoint{1.310384in}{2.305458in}}%
\pgfpathcurveto{\pgfqpoint{1.316207in}{2.311282in}}{\pgfqpoint{1.319480in}{2.319182in}}{\pgfqpoint{1.319480in}{2.327419in}}%
\pgfpathcurveto{\pgfqpoint{1.319480in}{2.335655in}}{\pgfqpoint{1.316207in}{2.343555in}}{\pgfqpoint{1.310384in}{2.349379in}}%
\pgfpathcurveto{\pgfqpoint{1.304560in}{2.355203in}}{\pgfqpoint{1.296660in}{2.358475in}}{\pgfqpoint{1.288423in}{2.358475in}}%
\pgfpathcurveto{\pgfqpoint{1.280187in}{2.358475in}}{\pgfqpoint{1.272287in}{2.355203in}}{\pgfqpoint{1.266463in}{2.349379in}}%
\pgfpathcurveto{\pgfqpoint{1.260639in}{2.343555in}}{\pgfqpoint{1.257367in}{2.335655in}}{\pgfqpoint{1.257367in}{2.327419in}}%
\pgfpathcurveto{\pgfqpoint{1.257367in}{2.319182in}}{\pgfqpoint{1.260639in}{2.311282in}}{\pgfqpoint{1.266463in}{2.305458in}}%
\pgfpathcurveto{\pgfqpoint{1.272287in}{2.299635in}}{\pgfqpoint{1.280187in}{2.296362in}}{\pgfqpoint{1.288423in}{2.296362in}}%
\pgfpathclose%
\pgfusepath{stroke,fill}%
\end{pgfscope}%
\begin{pgfscope}%
\pgfpathrectangle{\pgfqpoint{0.100000in}{0.212622in}}{\pgfqpoint{3.696000in}{3.696000in}}%
\pgfusepath{clip}%
\pgfsetbuttcap%
\pgfsetroundjoin%
\definecolor{currentfill}{rgb}{0.121569,0.466667,0.705882}%
\pgfsetfillcolor{currentfill}%
\pgfsetfillopacity{0.490051}%
\pgfsetlinewidth{1.003750pt}%
\definecolor{currentstroke}{rgb}{0.121569,0.466667,0.705882}%
\pgfsetstrokecolor{currentstroke}%
\pgfsetstrokeopacity{0.490051}%
\pgfsetdash{}{0pt}%
\pgfpathmoveto{\pgfqpoint{1.285853in}{2.291980in}}%
\pgfpathcurveto{\pgfqpoint{1.294089in}{2.291980in}}{\pgfqpoint{1.301989in}{2.295252in}}{\pgfqpoint{1.307813in}{2.301076in}}%
\pgfpathcurveto{\pgfqpoint{1.313637in}{2.306900in}}{\pgfqpoint{1.316909in}{2.314800in}}{\pgfqpoint{1.316909in}{2.323036in}}%
\pgfpathcurveto{\pgfqpoint{1.316909in}{2.331273in}}{\pgfqpoint{1.313637in}{2.339173in}}{\pgfqpoint{1.307813in}{2.344996in}}%
\pgfpathcurveto{\pgfqpoint{1.301989in}{2.350820in}}{\pgfqpoint{1.294089in}{2.354093in}}{\pgfqpoint{1.285853in}{2.354093in}}%
\pgfpathcurveto{\pgfqpoint{1.277617in}{2.354093in}}{\pgfqpoint{1.269716in}{2.350820in}}{\pgfqpoint{1.263893in}{2.344996in}}%
\pgfpathcurveto{\pgfqpoint{1.258069in}{2.339173in}}{\pgfqpoint{1.254796in}{2.331273in}}{\pgfqpoint{1.254796in}{2.323036in}}%
\pgfpathcurveto{\pgfqpoint{1.254796in}{2.314800in}}{\pgfqpoint{1.258069in}{2.306900in}}{\pgfqpoint{1.263893in}{2.301076in}}%
\pgfpathcurveto{\pgfqpoint{1.269716in}{2.295252in}}{\pgfqpoint{1.277617in}{2.291980in}}{\pgfqpoint{1.285853in}{2.291980in}}%
\pgfpathclose%
\pgfusepath{stroke,fill}%
\end{pgfscope}%
\begin{pgfscope}%
\pgfpathrectangle{\pgfqpoint{0.100000in}{0.212622in}}{\pgfqpoint{3.696000in}{3.696000in}}%
\pgfusepath{clip}%
\pgfsetbuttcap%
\pgfsetroundjoin%
\definecolor{currentfill}{rgb}{0.121569,0.466667,0.705882}%
\pgfsetfillcolor{currentfill}%
\pgfsetfillopacity{0.490768}%
\pgfsetlinewidth{1.003750pt}%
\definecolor{currentstroke}{rgb}{0.121569,0.466667,0.705882}%
\pgfsetstrokecolor{currentstroke}%
\pgfsetstrokeopacity{0.490768}%
\pgfsetdash{}{0pt}%
\pgfpathmoveto{\pgfqpoint{1.283798in}{2.287881in}}%
\pgfpathcurveto{\pgfqpoint{1.292034in}{2.287881in}}{\pgfqpoint{1.299934in}{2.291153in}}{\pgfqpoint{1.305758in}{2.296977in}}%
\pgfpathcurveto{\pgfqpoint{1.311582in}{2.302801in}}{\pgfqpoint{1.314854in}{2.310701in}}{\pgfqpoint{1.314854in}{2.318937in}}%
\pgfpathcurveto{\pgfqpoint{1.314854in}{2.327174in}}{\pgfqpoint{1.311582in}{2.335074in}}{\pgfqpoint{1.305758in}{2.340898in}}%
\pgfpathcurveto{\pgfqpoint{1.299934in}{2.346722in}}{\pgfqpoint{1.292034in}{2.349994in}}{\pgfqpoint{1.283798in}{2.349994in}}%
\pgfpathcurveto{\pgfqpoint{1.275561in}{2.349994in}}{\pgfqpoint{1.267661in}{2.346722in}}{\pgfqpoint{1.261838in}{2.340898in}}%
\pgfpathcurveto{\pgfqpoint{1.256014in}{2.335074in}}{\pgfqpoint{1.252741in}{2.327174in}}{\pgfqpoint{1.252741in}{2.318937in}}%
\pgfpathcurveto{\pgfqpoint{1.252741in}{2.310701in}}{\pgfqpoint{1.256014in}{2.302801in}}{\pgfqpoint{1.261838in}{2.296977in}}%
\pgfpathcurveto{\pgfqpoint{1.267661in}{2.291153in}}{\pgfqpoint{1.275561in}{2.287881in}}{\pgfqpoint{1.283798in}{2.287881in}}%
\pgfpathclose%
\pgfusepath{stroke,fill}%
\end{pgfscope}%
\begin{pgfscope}%
\pgfpathrectangle{\pgfqpoint{0.100000in}{0.212622in}}{\pgfqpoint{3.696000in}{3.696000in}}%
\pgfusepath{clip}%
\pgfsetbuttcap%
\pgfsetroundjoin%
\definecolor{currentfill}{rgb}{0.121569,0.466667,0.705882}%
\pgfsetfillcolor{currentfill}%
\pgfsetfillopacity{0.490856}%
\pgfsetlinewidth{1.003750pt}%
\definecolor{currentstroke}{rgb}{0.121569,0.466667,0.705882}%
\pgfsetstrokecolor{currentstroke}%
\pgfsetstrokeopacity{0.490856}%
\pgfsetdash{}{0pt}%
\pgfpathmoveto{\pgfqpoint{2.033121in}{2.544503in}}%
\pgfpathcurveto{\pgfqpoint{2.041358in}{2.544503in}}{\pgfqpoint{2.049258in}{2.547776in}}{\pgfqpoint{2.055082in}{2.553600in}}%
\pgfpathcurveto{\pgfqpoint{2.060906in}{2.559424in}}{\pgfqpoint{2.064178in}{2.567324in}}{\pgfqpoint{2.064178in}{2.575560in}}%
\pgfpathcurveto{\pgfqpoint{2.064178in}{2.583796in}}{\pgfqpoint{2.060906in}{2.591696in}}{\pgfqpoint{2.055082in}{2.597520in}}%
\pgfpathcurveto{\pgfqpoint{2.049258in}{2.603344in}}{\pgfqpoint{2.041358in}{2.606616in}}{\pgfqpoint{2.033121in}{2.606616in}}%
\pgfpathcurveto{\pgfqpoint{2.024885in}{2.606616in}}{\pgfqpoint{2.016985in}{2.603344in}}{\pgfqpoint{2.011161in}{2.597520in}}%
\pgfpathcurveto{\pgfqpoint{2.005337in}{2.591696in}}{\pgfqpoint{2.002065in}{2.583796in}}{\pgfqpoint{2.002065in}{2.575560in}}%
\pgfpathcurveto{\pgfqpoint{2.002065in}{2.567324in}}{\pgfqpoint{2.005337in}{2.559424in}}{\pgfqpoint{2.011161in}{2.553600in}}%
\pgfpathcurveto{\pgfqpoint{2.016985in}{2.547776in}}{\pgfqpoint{2.024885in}{2.544503in}}{\pgfqpoint{2.033121in}{2.544503in}}%
\pgfpathclose%
\pgfusepath{stroke,fill}%
\end{pgfscope}%
\begin{pgfscope}%
\pgfpathrectangle{\pgfqpoint{0.100000in}{0.212622in}}{\pgfqpoint{3.696000in}{3.696000in}}%
\pgfusepath{clip}%
\pgfsetbuttcap%
\pgfsetroundjoin%
\definecolor{currentfill}{rgb}{0.121569,0.466667,0.705882}%
\pgfsetfillcolor{currentfill}%
\pgfsetfillopacity{0.491047}%
\pgfsetlinewidth{1.003750pt}%
\definecolor{currentstroke}{rgb}{0.121569,0.466667,0.705882}%
\pgfsetstrokecolor{currentstroke}%
\pgfsetstrokeopacity{0.491047}%
\pgfsetdash{}{0pt}%
\pgfpathmoveto{\pgfqpoint{1.282790in}{2.286002in}}%
\pgfpathcurveto{\pgfqpoint{1.291026in}{2.286002in}}{\pgfqpoint{1.298926in}{2.289274in}}{\pgfqpoint{1.304750in}{2.295098in}}%
\pgfpathcurveto{\pgfqpoint{1.310574in}{2.300922in}}{\pgfqpoint{1.313846in}{2.308822in}}{\pgfqpoint{1.313846in}{2.317058in}}%
\pgfpathcurveto{\pgfqpoint{1.313846in}{2.325295in}}{\pgfqpoint{1.310574in}{2.333195in}}{\pgfqpoint{1.304750in}{2.339019in}}%
\pgfpathcurveto{\pgfqpoint{1.298926in}{2.344842in}}{\pgfqpoint{1.291026in}{2.348115in}}{\pgfqpoint{1.282790in}{2.348115in}}%
\pgfpathcurveto{\pgfqpoint{1.274553in}{2.348115in}}{\pgfqpoint{1.266653in}{2.344842in}}{\pgfqpoint{1.260829in}{2.339019in}}%
\pgfpathcurveto{\pgfqpoint{1.255005in}{2.333195in}}{\pgfqpoint{1.251733in}{2.325295in}}{\pgfqpoint{1.251733in}{2.317058in}}%
\pgfpathcurveto{\pgfqpoint{1.251733in}{2.308822in}}{\pgfqpoint{1.255005in}{2.300922in}}{\pgfqpoint{1.260829in}{2.295098in}}%
\pgfpathcurveto{\pgfqpoint{1.266653in}{2.289274in}}{\pgfqpoint{1.274553in}{2.286002in}}{\pgfqpoint{1.282790in}{2.286002in}}%
\pgfpathclose%
\pgfusepath{stroke,fill}%
\end{pgfscope}%
\begin{pgfscope}%
\pgfpathrectangle{\pgfqpoint{0.100000in}{0.212622in}}{\pgfqpoint{3.696000in}{3.696000in}}%
\pgfusepath{clip}%
\pgfsetbuttcap%
\pgfsetroundjoin%
\definecolor{currentfill}{rgb}{0.121569,0.466667,0.705882}%
\pgfsetfillcolor{currentfill}%
\pgfsetfillopacity{0.491553}%
\pgfsetlinewidth{1.003750pt}%
\definecolor{currentstroke}{rgb}{0.121569,0.466667,0.705882}%
\pgfsetstrokecolor{currentstroke}%
\pgfsetstrokeopacity{0.491553}%
\pgfsetdash{}{0pt}%
\pgfpathmoveto{\pgfqpoint{1.281218in}{2.282254in}}%
\pgfpathcurveto{\pgfqpoint{1.289454in}{2.282254in}}{\pgfqpoint{1.297354in}{2.285527in}}{\pgfqpoint{1.303178in}{2.291351in}}%
\pgfpathcurveto{\pgfqpoint{1.309002in}{2.297175in}}{\pgfqpoint{1.312275in}{2.305075in}}{\pgfqpoint{1.312275in}{2.313311in}}%
\pgfpathcurveto{\pgfqpoint{1.312275in}{2.321547in}}{\pgfqpoint{1.309002in}{2.329447in}}{\pgfqpoint{1.303178in}{2.335271in}}%
\pgfpathcurveto{\pgfqpoint{1.297354in}{2.341095in}}{\pgfqpoint{1.289454in}{2.344367in}}{\pgfqpoint{1.281218in}{2.344367in}}%
\pgfpathcurveto{\pgfqpoint{1.272982in}{2.344367in}}{\pgfqpoint{1.265082in}{2.341095in}}{\pgfqpoint{1.259258in}{2.335271in}}%
\pgfpathcurveto{\pgfqpoint{1.253434in}{2.329447in}}{\pgfqpoint{1.250162in}{2.321547in}}{\pgfqpoint{1.250162in}{2.313311in}}%
\pgfpathcurveto{\pgfqpoint{1.250162in}{2.305075in}}{\pgfqpoint{1.253434in}{2.297175in}}{\pgfqpoint{1.259258in}{2.291351in}}%
\pgfpathcurveto{\pgfqpoint{1.265082in}{2.285527in}}{\pgfqpoint{1.272982in}{2.282254in}}{\pgfqpoint{1.281218in}{2.282254in}}%
\pgfpathclose%
\pgfusepath{stroke,fill}%
\end{pgfscope}%
\begin{pgfscope}%
\pgfpathrectangle{\pgfqpoint{0.100000in}{0.212622in}}{\pgfqpoint{3.696000in}{3.696000in}}%
\pgfusepath{clip}%
\pgfsetbuttcap%
\pgfsetroundjoin%
\definecolor{currentfill}{rgb}{0.121569,0.466667,0.705882}%
\pgfsetfillcolor{currentfill}%
\pgfsetfillopacity{0.492486}%
\pgfsetlinewidth{1.003750pt}%
\definecolor{currentstroke}{rgb}{0.121569,0.466667,0.705882}%
\pgfsetstrokecolor{currentstroke}%
\pgfsetstrokeopacity{0.492486}%
\pgfsetdash{}{0pt}%
\pgfpathmoveto{\pgfqpoint{1.277863in}{2.276107in}}%
\pgfpathcurveto{\pgfqpoint{1.286099in}{2.276107in}}{\pgfqpoint{1.293999in}{2.279379in}}{\pgfqpoint{1.299823in}{2.285203in}}%
\pgfpathcurveto{\pgfqpoint{1.305647in}{2.291027in}}{\pgfqpoint{1.308919in}{2.298927in}}{\pgfqpoint{1.308919in}{2.307163in}}%
\pgfpathcurveto{\pgfqpoint{1.308919in}{2.315400in}}{\pgfqpoint{1.305647in}{2.323300in}}{\pgfqpoint{1.299823in}{2.329124in}}%
\pgfpathcurveto{\pgfqpoint{1.293999in}{2.334948in}}{\pgfqpoint{1.286099in}{2.338220in}}{\pgfqpoint{1.277863in}{2.338220in}}%
\pgfpathcurveto{\pgfqpoint{1.269627in}{2.338220in}}{\pgfqpoint{1.261726in}{2.334948in}}{\pgfqpoint{1.255903in}{2.329124in}}%
\pgfpathcurveto{\pgfqpoint{1.250079in}{2.323300in}}{\pgfqpoint{1.246806in}{2.315400in}}{\pgfqpoint{1.246806in}{2.307163in}}%
\pgfpathcurveto{\pgfqpoint{1.246806in}{2.298927in}}{\pgfqpoint{1.250079in}{2.291027in}}{\pgfqpoint{1.255903in}{2.285203in}}%
\pgfpathcurveto{\pgfqpoint{1.261726in}{2.279379in}}{\pgfqpoint{1.269627in}{2.276107in}}{\pgfqpoint{1.277863in}{2.276107in}}%
\pgfpathclose%
\pgfusepath{stroke,fill}%
\end{pgfscope}%
\begin{pgfscope}%
\pgfpathrectangle{\pgfqpoint{0.100000in}{0.212622in}}{\pgfqpoint{3.696000in}{3.696000in}}%
\pgfusepath{clip}%
\pgfsetbuttcap%
\pgfsetroundjoin%
\definecolor{currentfill}{rgb}{0.121569,0.466667,0.705882}%
\pgfsetfillcolor{currentfill}%
\pgfsetfillopacity{0.493074}%
\pgfsetlinewidth{1.003750pt}%
\definecolor{currentstroke}{rgb}{0.121569,0.466667,0.705882}%
\pgfsetstrokecolor{currentstroke}%
\pgfsetstrokeopacity{0.493074}%
\pgfsetdash{}{0pt}%
\pgfpathmoveto{\pgfqpoint{2.034277in}{2.534780in}}%
\pgfpathcurveto{\pgfqpoint{2.042513in}{2.534780in}}{\pgfqpoint{2.050413in}{2.538053in}}{\pgfqpoint{2.056237in}{2.543877in}}%
\pgfpathcurveto{\pgfqpoint{2.062061in}{2.549701in}}{\pgfqpoint{2.065333in}{2.557601in}}{\pgfqpoint{2.065333in}{2.565837in}}%
\pgfpathcurveto{\pgfqpoint{2.065333in}{2.574073in}}{\pgfqpoint{2.062061in}{2.581973in}}{\pgfqpoint{2.056237in}{2.587797in}}%
\pgfpathcurveto{\pgfqpoint{2.050413in}{2.593621in}}{\pgfqpoint{2.042513in}{2.596893in}}{\pgfqpoint{2.034277in}{2.596893in}}%
\pgfpathcurveto{\pgfqpoint{2.026041in}{2.596893in}}{\pgfqpoint{2.018141in}{2.593621in}}{\pgfqpoint{2.012317in}{2.587797in}}%
\pgfpathcurveto{\pgfqpoint{2.006493in}{2.581973in}}{\pgfqpoint{2.003220in}{2.574073in}}{\pgfqpoint{2.003220in}{2.565837in}}%
\pgfpathcurveto{\pgfqpoint{2.003220in}{2.557601in}}{\pgfqpoint{2.006493in}{2.549701in}}{\pgfqpoint{2.012317in}{2.543877in}}%
\pgfpathcurveto{\pgfqpoint{2.018141in}{2.538053in}}{\pgfqpoint{2.026041in}{2.534780in}}{\pgfqpoint{2.034277in}{2.534780in}}%
\pgfpathclose%
\pgfusepath{stroke,fill}%
\end{pgfscope}%
\begin{pgfscope}%
\pgfpathrectangle{\pgfqpoint{0.100000in}{0.212622in}}{\pgfqpoint{3.696000in}{3.696000in}}%
\pgfusepath{clip}%
\pgfsetbuttcap%
\pgfsetroundjoin%
\definecolor{currentfill}{rgb}{0.121569,0.466667,0.705882}%
\pgfsetfillcolor{currentfill}%
\pgfsetfillopacity{0.493185}%
\pgfsetlinewidth{1.003750pt}%
\definecolor{currentstroke}{rgb}{0.121569,0.466667,0.705882}%
\pgfsetstrokecolor{currentstroke}%
\pgfsetstrokeopacity{0.493185}%
\pgfsetdash{}{0pt}%
\pgfpathmoveto{\pgfqpoint{1.275211in}{2.270695in}}%
\pgfpathcurveto{\pgfqpoint{1.283448in}{2.270695in}}{\pgfqpoint{1.291348in}{2.273967in}}{\pgfqpoint{1.297172in}{2.279791in}}%
\pgfpathcurveto{\pgfqpoint{1.302995in}{2.285615in}}{\pgfqpoint{1.306268in}{2.293515in}}{\pgfqpoint{1.306268in}{2.301752in}}%
\pgfpathcurveto{\pgfqpoint{1.306268in}{2.309988in}}{\pgfqpoint{1.302995in}{2.317888in}}{\pgfqpoint{1.297172in}{2.323712in}}%
\pgfpathcurveto{\pgfqpoint{1.291348in}{2.329536in}}{\pgfqpoint{1.283448in}{2.332808in}}{\pgfqpoint{1.275211in}{2.332808in}}%
\pgfpathcurveto{\pgfqpoint{1.266975in}{2.332808in}}{\pgfqpoint{1.259075in}{2.329536in}}{\pgfqpoint{1.253251in}{2.323712in}}%
\pgfpathcurveto{\pgfqpoint{1.247427in}{2.317888in}}{\pgfqpoint{1.244155in}{2.309988in}}{\pgfqpoint{1.244155in}{2.301752in}}%
\pgfpathcurveto{\pgfqpoint{1.244155in}{2.293515in}}{\pgfqpoint{1.247427in}{2.285615in}}{\pgfqpoint{1.253251in}{2.279791in}}%
\pgfpathcurveto{\pgfqpoint{1.259075in}{2.273967in}}{\pgfqpoint{1.266975in}{2.270695in}}{\pgfqpoint{1.275211in}{2.270695in}}%
\pgfpathclose%
\pgfusepath{stroke,fill}%
\end{pgfscope}%
\begin{pgfscope}%
\pgfpathrectangle{\pgfqpoint{0.100000in}{0.212622in}}{\pgfqpoint{3.696000in}{3.696000in}}%
\pgfusepath{clip}%
\pgfsetbuttcap%
\pgfsetroundjoin%
\definecolor{currentfill}{rgb}{0.121569,0.466667,0.705882}%
\pgfsetfillcolor{currentfill}%
\pgfsetfillopacity{0.493225}%
\pgfsetlinewidth{1.003750pt}%
\definecolor{currentstroke}{rgb}{0.121569,0.466667,0.705882}%
\pgfsetstrokecolor{currentstroke}%
\pgfsetstrokeopacity{0.493225}%
\pgfsetdash{}{0pt}%
\pgfpathmoveto{\pgfqpoint{1.275015in}{2.270274in}}%
\pgfpathcurveto{\pgfqpoint{1.283251in}{2.270274in}}{\pgfqpoint{1.291151in}{2.273546in}}{\pgfqpoint{1.296975in}{2.279370in}}%
\pgfpathcurveto{\pgfqpoint{1.302799in}{2.285194in}}{\pgfqpoint{1.306071in}{2.293094in}}{\pgfqpoint{1.306071in}{2.301330in}}%
\pgfpathcurveto{\pgfqpoint{1.306071in}{2.309567in}}{\pgfqpoint{1.302799in}{2.317467in}}{\pgfqpoint{1.296975in}{2.323291in}}%
\pgfpathcurveto{\pgfqpoint{1.291151in}{2.329115in}}{\pgfqpoint{1.283251in}{2.332387in}}{\pgfqpoint{1.275015in}{2.332387in}}%
\pgfpathcurveto{\pgfqpoint{1.266779in}{2.332387in}}{\pgfqpoint{1.258879in}{2.329115in}}{\pgfqpoint{1.253055in}{2.323291in}}%
\pgfpathcurveto{\pgfqpoint{1.247231in}{2.317467in}}{\pgfqpoint{1.243958in}{2.309567in}}{\pgfqpoint{1.243958in}{2.301330in}}%
\pgfpathcurveto{\pgfqpoint{1.243958in}{2.293094in}}{\pgfqpoint{1.247231in}{2.285194in}}{\pgfqpoint{1.253055in}{2.279370in}}%
\pgfpathcurveto{\pgfqpoint{1.258879in}{2.273546in}}{\pgfqpoint{1.266779in}{2.270274in}}{\pgfqpoint{1.275015in}{2.270274in}}%
\pgfpathclose%
\pgfusepath{stroke,fill}%
\end{pgfscope}%
\begin{pgfscope}%
\pgfpathrectangle{\pgfqpoint{0.100000in}{0.212622in}}{\pgfqpoint{3.696000in}{3.696000in}}%
\pgfusepath{clip}%
\pgfsetbuttcap%
\pgfsetroundjoin%
\definecolor{currentfill}{rgb}{0.121569,0.466667,0.705882}%
\pgfsetfillcolor{currentfill}%
\pgfsetfillopacity{0.493277}%
\pgfsetlinewidth{1.003750pt}%
\definecolor{currentstroke}{rgb}{0.121569,0.466667,0.705882}%
\pgfsetstrokecolor{currentstroke}%
\pgfsetstrokeopacity{0.493277}%
\pgfsetdash{}{0pt}%
\pgfpathmoveto{\pgfqpoint{1.274687in}{2.269395in}}%
\pgfpathcurveto{\pgfqpoint{1.282923in}{2.269395in}}{\pgfqpoint{1.290823in}{2.272667in}}{\pgfqpoint{1.296647in}{2.278491in}}%
\pgfpathcurveto{\pgfqpoint{1.302471in}{2.284315in}}{\pgfqpoint{1.305743in}{2.292215in}}{\pgfqpoint{1.305743in}{2.300451in}}%
\pgfpathcurveto{\pgfqpoint{1.305743in}{2.308687in}}{\pgfqpoint{1.302471in}{2.316587in}}{\pgfqpoint{1.296647in}{2.322411in}}%
\pgfpathcurveto{\pgfqpoint{1.290823in}{2.328235in}}{\pgfqpoint{1.282923in}{2.331508in}}{\pgfqpoint{1.274687in}{2.331508in}}%
\pgfpathcurveto{\pgfqpoint{1.266451in}{2.331508in}}{\pgfqpoint{1.258551in}{2.328235in}}{\pgfqpoint{1.252727in}{2.322411in}}%
\pgfpathcurveto{\pgfqpoint{1.246903in}{2.316587in}}{\pgfqpoint{1.243630in}{2.308687in}}{\pgfqpoint{1.243630in}{2.300451in}}%
\pgfpathcurveto{\pgfqpoint{1.243630in}{2.292215in}}{\pgfqpoint{1.246903in}{2.284315in}}{\pgfqpoint{1.252727in}{2.278491in}}%
\pgfpathcurveto{\pgfqpoint{1.258551in}{2.272667in}}{\pgfqpoint{1.266451in}{2.269395in}}{\pgfqpoint{1.274687in}{2.269395in}}%
\pgfpathclose%
\pgfusepath{stroke,fill}%
\end{pgfscope}%
\begin{pgfscope}%
\pgfpathrectangle{\pgfqpoint{0.100000in}{0.212622in}}{\pgfqpoint{3.696000in}{3.696000in}}%
\pgfusepath{clip}%
\pgfsetbuttcap%
\pgfsetroundjoin%
\definecolor{currentfill}{rgb}{0.121569,0.466667,0.705882}%
\pgfsetfillcolor{currentfill}%
\pgfsetfillopacity{0.493488}%
\pgfsetlinewidth{1.003750pt}%
\definecolor{currentstroke}{rgb}{0.121569,0.466667,0.705882}%
\pgfsetstrokecolor{currentstroke}%
\pgfsetstrokeopacity{0.493488}%
\pgfsetdash{}{0pt}%
\pgfpathmoveto{\pgfqpoint{1.274021in}{2.268320in}}%
\pgfpathcurveto{\pgfqpoint{1.282258in}{2.268320in}}{\pgfqpoint{1.290158in}{2.271592in}}{\pgfqpoint{1.295982in}{2.277416in}}%
\pgfpathcurveto{\pgfqpoint{1.301806in}{2.283240in}}{\pgfqpoint{1.305078in}{2.291140in}}{\pgfqpoint{1.305078in}{2.299376in}}%
\pgfpathcurveto{\pgfqpoint{1.305078in}{2.307613in}}{\pgfqpoint{1.301806in}{2.315513in}}{\pgfqpoint{1.295982in}{2.321337in}}%
\pgfpathcurveto{\pgfqpoint{1.290158in}{2.327161in}}{\pgfqpoint{1.282258in}{2.330433in}}{\pgfqpoint{1.274021in}{2.330433in}}%
\pgfpathcurveto{\pgfqpoint{1.265785in}{2.330433in}}{\pgfqpoint{1.257885in}{2.327161in}}{\pgfqpoint{1.252061in}{2.321337in}}%
\pgfpathcurveto{\pgfqpoint{1.246237in}{2.315513in}}{\pgfqpoint{1.242965in}{2.307613in}}{\pgfqpoint{1.242965in}{2.299376in}}%
\pgfpathcurveto{\pgfqpoint{1.242965in}{2.291140in}}{\pgfqpoint{1.246237in}{2.283240in}}{\pgfqpoint{1.252061in}{2.277416in}}%
\pgfpathcurveto{\pgfqpoint{1.257885in}{2.271592in}}{\pgfqpoint{1.265785in}{2.268320in}}{\pgfqpoint{1.274021in}{2.268320in}}%
\pgfpathclose%
\pgfusepath{stroke,fill}%
\end{pgfscope}%
\begin{pgfscope}%
\pgfpathrectangle{\pgfqpoint{0.100000in}{0.212622in}}{\pgfqpoint{3.696000in}{3.696000in}}%
\pgfusepath{clip}%
\pgfsetbuttcap%
\pgfsetroundjoin%
\definecolor{currentfill}{rgb}{0.121569,0.466667,0.705882}%
\pgfsetfillcolor{currentfill}%
\pgfsetfillopacity{0.493903}%
\pgfsetlinewidth{1.003750pt}%
\definecolor{currentstroke}{rgb}{0.121569,0.466667,0.705882}%
\pgfsetstrokecolor{currentstroke}%
\pgfsetstrokeopacity{0.493903}%
\pgfsetdash{}{0pt}%
\pgfpathmoveto{\pgfqpoint{1.272949in}{2.266303in}}%
\pgfpathcurveto{\pgfqpoint{1.281185in}{2.266303in}}{\pgfqpoint{1.289085in}{2.269575in}}{\pgfqpoint{1.294909in}{2.275399in}}%
\pgfpathcurveto{\pgfqpoint{1.300733in}{2.281223in}}{\pgfqpoint{1.304005in}{2.289123in}}{\pgfqpoint{1.304005in}{2.297359in}}%
\pgfpathcurveto{\pgfqpoint{1.304005in}{2.305595in}}{\pgfqpoint{1.300733in}{2.313495in}}{\pgfqpoint{1.294909in}{2.319319in}}%
\pgfpathcurveto{\pgfqpoint{1.289085in}{2.325143in}}{\pgfqpoint{1.281185in}{2.328416in}}{\pgfqpoint{1.272949in}{2.328416in}}%
\pgfpathcurveto{\pgfqpoint{1.264713in}{2.328416in}}{\pgfqpoint{1.256813in}{2.325143in}}{\pgfqpoint{1.250989in}{2.319319in}}%
\pgfpathcurveto{\pgfqpoint{1.245165in}{2.313495in}}{\pgfqpoint{1.241892in}{2.305595in}}{\pgfqpoint{1.241892in}{2.297359in}}%
\pgfpathcurveto{\pgfqpoint{1.241892in}{2.289123in}}{\pgfqpoint{1.245165in}{2.281223in}}{\pgfqpoint{1.250989in}{2.275399in}}%
\pgfpathcurveto{\pgfqpoint{1.256813in}{2.269575in}}{\pgfqpoint{1.264713in}{2.266303in}}{\pgfqpoint{1.272949in}{2.266303in}}%
\pgfpathclose%
\pgfusepath{stroke,fill}%
\end{pgfscope}%
\begin{pgfscope}%
\pgfpathrectangle{\pgfqpoint{0.100000in}{0.212622in}}{\pgfqpoint{3.696000in}{3.696000in}}%
\pgfusepath{clip}%
\pgfsetbuttcap%
\pgfsetroundjoin%
\definecolor{currentfill}{rgb}{0.121569,0.466667,0.705882}%
\pgfsetfillcolor{currentfill}%
\pgfsetfillopacity{0.494252}%
\pgfsetlinewidth{1.003750pt}%
\definecolor{currentstroke}{rgb}{0.121569,0.466667,0.705882}%
\pgfsetstrokecolor{currentstroke}%
\pgfsetstrokeopacity{0.494252}%
\pgfsetdash{}{0pt}%
\pgfpathmoveto{\pgfqpoint{1.272027in}{2.264744in}}%
\pgfpathcurveto{\pgfqpoint{1.280264in}{2.264744in}}{\pgfqpoint{1.288164in}{2.268016in}}{\pgfqpoint{1.293988in}{2.273840in}}%
\pgfpathcurveto{\pgfqpoint{1.299812in}{2.279664in}}{\pgfqpoint{1.303084in}{2.287564in}}{\pgfqpoint{1.303084in}{2.295800in}}%
\pgfpathcurveto{\pgfqpoint{1.303084in}{2.304037in}}{\pgfqpoint{1.299812in}{2.311937in}}{\pgfqpoint{1.293988in}{2.317760in}}%
\pgfpathcurveto{\pgfqpoint{1.288164in}{2.323584in}}{\pgfqpoint{1.280264in}{2.326857in}}{\pgfqpoint{1.272027in}{2.326857in}}%
\pgfpathcurveto{\pgfqpoint{1.263791in}{2.326857in}}{\pgfqpoint{1.255891in}{2.323584in}}{\pgfqpoint{1.250067in}{2.317760in}}%
\pgfpathcurveto{\pgfqpoint{1.244243in}{2.311937in}}{\pgfqpoint{1.240971in}{2.304037in}}{\pgfqpoint{1.240971in}{2.295800in}}%
\pgfpathcurveto{\pgfqpoint{1.240971in}{2.287564in}}{\pgfqpoint{1.244243in}{2.279664in}}{\pgfqpoint{1.250067in}{2.273840in}}%
\pgfpathcurveto{\pgfqpoint{1.255891in}{2.268016in}}{\pgfqpoint{1.263791in}{2.264744in}}{\pgfqpoint{1.272027in}{2.264744in}}%
\pgfpathclose%
\pgfusepath{stroke,fill}%
\end{pgfscope}%
\begin{pgfscope}%
\pgfpathrectangle{\pgfqpoint{0.100000in}{0.212622in}}{\pgfqpoint{3.696000in}{3.696000in}}%
\pgfusepath{clip}%
\pgfsetbuttcap%
\pgfsetroundjoin%
\definecolor{currentfill}{rgb}{0.121569,0.466667,0.705882}%
\pgfsetfillcolor{currentfill}%
\pgfsetfillopacity{0.494887}%
\pgfsetlinewidth{1.003750pt}%
\definecolor{currentstroke}{rgb}{0.121569,0.466667,0.705882}%
\pgfsetstrokecolor{currentstroke}%
\pgfsetstrokeopacity{0.494887}%
\pgfsetdash{}{0pt}%
\pgfpathmoveto{\pgfqpoint{1.270379in}{2.261876in}}%
\pgfpathcurveto{\pgfqpoint{1.278616in}{2.261876in}}{\pgfqpoint{1.286516in}{2.265149in}}{\pgfqpoint{1.292340in}{2.270973in}}%
\pgfpathcurveto{\pgfqpoint{1.298164in}{2.276796in}}{\pgfqpoint{1.301436in}{2.284696in}}{\pgfqpoint{1.301436in}{2.292933in}}%
\pgfpathcurveto{\pgfqpoint{1.301436in}{2.301169in}}{\pgfqpoint{1.298164in}{2.309069in}}{\pgfqpoint{1.292340in}{2.314893in}}%
\pgfpathcurveto{\pgfqpoint{1.286516in}{2.320717in}}{\pgfqpoint{1.278616in}{2.323989in}}{\pgfqpoint{1.270379in}{2.323989in}}%
\pgfpathcurveto{\pgfqpoint{1.262143in}{2.323989in}}{\pgfqpoint{1.254243in}{2.320717in}}{\pgfqpoint{1.248419in}{2.314893in}}%
\pgfpathcurveto{\pgfqpoint{1.242595in}{2.309069in}}{\pgfqpoint{1.239323in}{2.301169in}}{\pgfqpoint{1.239323in}{2.292933in}}%
\pgfpathcurveto{\pgfqpoint{1.239323in}{2.284696in}}{\pgfqpoint{1.242595in}{2.276796in}}{\pgfqpoint{1.248419in}{2.270973in}}%
\pgfpathcurveto{\pgfqpoint{1.254243in}{2.265149in}}{\pgfqpoint{1.262143in}{2.261876in}}{\pgfqpoint{1.270379in}{2.261876in}}%
\pgfpathclose%
\pgfusepath{stroke,fill}%
\end{pgfscope}%
\begin{pgfscope}%
\pgfpathrectangle{\pgfqpoint{0.100000in}{0.212622in}}{\pgfqpoint{3.696000in}{3.696000in}}%
\pgfusepath{clip}%
\pgfsetbuttcap%
\pgfsetroundjoin%
\definecolor{currentfill}{rgb}{0.121569,0.466667,0.705882}%
\pgfsetfillcolor{currentfill}%
\pgfsetfillopacity{0.495079}%
\pgfsetlinewidth{1.003750pt}%
\definecolor{currentstroke}{rgb}{0.121569,0.466667,0.705882}%
\pgfsetstrokecolor{currentstroke}%
\pgfsetstrokeopacity{0.495079}%
\pgfsetdash{}{0pt}%
\pgfpathmoveto{\pgfqpoint{2.036056in}{2.523138in}}%
\pgfpathcurveto{\pgfqpoint{2.044292in}{2.523138in}}{\pgfqpoint{2.052192in}{2.526410in}}{\pgfqpoint{2.058016in}{2.532234in}}%
\pgfpathcurveto{\pgfqpoint{2.063840in}{2.538058in}}{\pgfqpoint{2.067112in}{2.545958in}}{\pgfqpoint{2.067112in}{2.554194in}}%
\pgfpathcurveto{\pgfqpoint{2.067112in}{2.562431in}}{\pgfqpoint{2.063840in}{2.570331in}}{\pgfqpoint{2.058016in}{2.576154in}}%
\pgfpathcurveto{\pgfqpoint{2.052192in}{2.581978in}}{\pgfqpoint{2.044292in}{2.585251in}}{\pgfqpoint{2.036056in}{2.585251in}}%
\pgfpathcurveto{\pgfqpoint{2.027819in}{2.585251in}}{\pgfqpoint{2.019919in}{2.581978in}}{\pgfqpoint{2.014095in}{2.576154in}}%
\pgfpathcurveto{\pgfqpoint{2.008272in}{2.570331in}}{\pgfqpoint{2.004999in}{2.562431in}}{\pgfqpoint{2.004999in}{2.554194in}}%
\pgfpathcurveto{\pgfqpoint{2.004999in}{2.545958in}}{\pgfqpoint{2.008272in}{2.538058in}}{\pgfqpoint{2.014095in}{2.532234in}}%
\pgfpathcurveto{\pgfqpoint{2.019919in}{2.526410in}}{\pgfqpoint{2.027819in}{2.523138in}}{\pgfqpoint{2.036056in}{2.523138in}}%
\pgfpathclose%
\pgfusepath{stroke,fill}%
\end{pgfscope}%
\begin{pgfscope}%
\pgfpathrectangle{\pgfqpoint{0.100000in}{0.212622in}}{\pgfqpoint{3.696000in}{3.696000in}}%
\pgfusepath{clip}%
\pgfsetbuttcap%
\pgfsetroundjoin%
\definecolor{currentfill}{rgb}{0.121569,0.466667,0.705882}%
\pgfsetfillcolor{currentfill}%
\pgfsetfillopacity{0.496016}%
\pgfsetlinewidth{1.003750pt}%
\definecolor{currentstroke}{rgb}{0.121569,0.466667,0.705882}%
\pgfsetstrokecolor{currentstroke}%
\pgfsetstrokeopacity{0.496016}%
\pgfsetdash{}{0pt}%
\pgfpathmoveto{\pgfqpoint{1.267430in}{2.256504in}}%
\pgfpathcurveto{\pgfqpoint{1.275666in}{2.256504in}}{\pgfqpoint{1.283566in}{2.259776in}}{\pgfqpoint{1.289390in}{2.265600in}}%
\pgfpathcurveto{\pgfqpoint{1.295214in}{2.271424in}}{\pgfqpoint{1.298486in}{2.279324in}}{\pgfqpoint{1.298486in}{2.287561in}}%
\pgfpathcurveto{\pgfqpoint{1.298486in}{2.295797in}}{\pgfqpoint{1.295214in}{2.303697in}}{\pgfqpoint{1.289390in}{2.309521in}}%
\pgfpathcurveto{\pgfqpoint{1.283566in}{2.315345in}}{\pgfqpoint{1.275666in}{2.318617in}}{\pgfqpoint{1.267430in}{2.318617in}}%
\pgfpathcurveto{\pgfqpoint{1.259194in}{2.318617in}}{\pgfqpoint{1.251293in}{2.315345in}}{\pgfqpoint{1.245470in}{2.309521in}}%
\pgfpathcurveto{\pgfqpoint{1.239646in}{2.303697in}}{\pgfqpoint{1.236373in}{2.295797in}}{\pgfqpoint{1.236373in}{2.287561in}}%
\pgfpathcurveto{\pgfqpoint{1.236373in}{2.279324in}}{\pgfqpoint{1.239646in}{2.271424in}}{\pgfqpoint{1.245470in}{2.265600in}}%
\pgfpathcurveto{\pgfqpoint{1.251293in}{2.259776in}}{\pgfqpoint{1.259194in}{2.256504in}}{\pgfqpoint{1.267430in}{2.256504in}}%
\pgfpathclose%
\pgfusepath{stroke,fill}%
\end{pgfscope}%
\begin{pgfscope}%
\pgfpathrectangle{\pgfqpoint{0.100000in}{0.212622in}}{\pgfqpoint{3.696000in}{3.696000in}}%
\pgfusepath{clip}%
\pgfsetbuttcap%
\pgfsetroundjoin%
\definecolor{currentfill}{rgb}{0.121569,0.466667,0.705882}%
\pgfsetfillcolor{currentfill}%
\pgfsetfillopacity{0.497844}%
\pgfsetlinewidth{1.003750pt}%
\definecolor{currentstroke}{rgb}{0.121569,0.466667,0.705882}%
\pgfsetstrokecolor{currentstroke}%
\pgfsetstrokeopacity{0.497844}%
\pgfsetdash{}{0pt}%
\pgfpathmoveto{\pgfqpoint{2.037083in}{2.509851in}}%
\pgfpathcurveto{\pgfqpoint{2.045319in}{2.509851in}}{\pgfqpoint{2.053219in}{2.513123in}}{\pgfqpoint{2.059043in}{2.518947in}}%
\pgfpathcurveto{\pgfqpoint{2.064867in}{2.524771in}}{\pgfqpoint{2.068139in}{2.532671in}}{\pgfqpoint{2.068139in}{2.540907in}}%
\pgfpathcurveto{\pgfqpoint{2.068139in}{2.549143in}}{\pgfqpoint{2.064867in}{2.557043in}}{\pgfqpoint{2.059043in}{2.562867in}}%
\pgfpathcurveto{\pgfqpoint{2.053219in}{2.568691in}}{\pgfqpoint{2.045319in}{2.571964in}}{\pgfqpoint{2.037083in}{2.571964in}}%
\pgfpathcurveto{\pgfqpoint{2.028846in}{2.571964in}}{\pgfqpoint{2.020946in}{2.568691in}}{\pgfqpoint{2.015122in}{2.562867in}}%
\pgfpathcurveto{\pgfqpoint{2.009298in}{2.557043in}}{\pgfqpoint{2.006026in}{2.549143in}}{\pgfqpoint{2.006026in}{2.540907in}}%
\pgfpathcurveto{\pgfqpoint{2.006026in}{2.532671in}}{\pgfqpoint{2.009298in}{2.524771in}}{\pgfqpoint{2.015122in}{2.518947in}}%
\pgfpathcurveto{\pgfqpoint{2.020946in}{2.513123in}}{\pgfqpoint{2.028846in}{2.509851in}}{\pgfqpoint{2.037083in}{2.509851in}}%
\pgfpathclose%
\pgfusepath{stroke,fill}%
\end{pgfscope}%
\begin{pgfscope}%
\pgfpathrectangle{\pgfqpoint{0.100000in}{0.212622in}}{\pgfqpoint{3.696000in}{3.696000in}}%
\pgfusepath{clip}%
\pgfsetbuttcap%
\pgfsetroundjoin%
\definecolor{currentfill}{rgb}{0.121569,0.466667,0.705882}%
\pgfsetfillcolor{currentfill}%
\pgfsetfillopacity{0.498013}%
\pgfsetlinewidth{1.003750pt}%
\definecolor{currentstroke}{rgb}{0.121569,0.466667,0.705882}%
\pgfsetstrokecolor{currentstroke}%
\pgfsetstrokeopacity{0.498013}%
\pgfsetdash{}{0pt}%
\pgfpathmoveto{\pgfqpoint{1.261685in}{2.247014in}}%
\pgfpathcurveto{\pgfqpoint{1.269921in}{2.247014in}}{\pgfqpoint{1.277821in}{2.250287in}}{\pgfqpoint{1.283645in}{2.256111in}}%
\pgfpathcurveto{\pgfqpoint{1.289469in}{2.261934in}}{\pgfqpoint{1.292742in}{2.269835in}}{\pgfqpoint{1.292742in}{2.278071in}}%
\pgfpathcurveto{\pgfqpoint{1.292742in}{2.286307in}}{\pgfqpoint{1.289469in}{2.294207in}}{\pgfqpoint{1.283645in}{2.300031in}}%
\pgfpathcurveto{\pgfqpoint{1.277821in}{2.305855in}}{\pgfqpoint{1.269921in}{2.309127in}}{\pgfqpoint{1.261685in}{2.309127in}}%
\pgfpathcurveto{\pgfqpoint{1.253449in}{2.309127in}}{\pgfqpoint{1.245549in}{2.305855in}}{\pgfqpoint{1.239725in}{2.300031in}}%
\pgfpathcurveto{\pgfqpoint{1.233901in}{2.294207in}}{\pgfqpoint{1.230629in}{2.286307in}}{\pgfqpoint{1.230629in}{2.278071in}}%
\pgfpathcurveto{\pgfqpoint{1.230629in}{2.269835in}}{\pgfqpoint{1.233901in}{2.261934in}}{\pgfqpoint{1.239725in}{2.256111in}}%
\pgfpathcurveto{\pgfqpoint{1.245549in}{2.250287in}}{\pgfqpoint{1.253449in}{2.247014in}}{\pgfqpoint{1.261685in}{2.247014in}}%
\pgfpathclose%
\pgfusepath{stroke,fill}%
\end{pgfscope}%
\begin{pgfscope}%
\pgfpathrectangle{\pgfqpoint{0.100000in}{0.212622in}}{\pgfqpoint{3.696000in}{3.696000in}}%
\pgfusepath{clip}%
\pgfsetbuttcap%
\pgfsetroundjoin%
\definecolor{currentfill}{rgb}{0.121569,0.466667,0.705882}%
\pgfsetfillcolor{currentfill}%
\pgfsetfillopacity{0.499860}%
\pgfsetlinewidth{1.003750pt}%
\definecolor{currentstroke}{rgb}{0.121569,0.466667,0.705882}%
\pgfsetstrokecolor{currentstroke}%
\pgfsetstrokeopacity{0.499860}%
\pgfsetdash{}{0pt}%
\pgfpathmoveto{\pgfqpoint{1.256477in}{2.237463in}}%
\pgfpathcurveto{\pgfqpoint{1.264714in}{2.237463in}}{\pgfqpoint{1.272614in}{2.240736in}}{\pgfqpoint{1.278438in}{2.246559in}}%
\pgfpathcurveto{\pgfqpoint{1.284262in}{2.252383in}}{\pgfqpoint{1.287534in}{2.260283in}}{\pgfqpoint{1.287534in}{2.268520in}}%
\pgfpathcurveto{\pgfqpoint{1.287534in}{2.276756in}}{\pgfqpoint{1.284262in}{2.284656in}}{\pgfqpoint{1.278438in}{2.290480in}}%
\pgfpathcurveto{\pgfqpoint{1.272614in}{2.296304in}}{\pgfqpoint{1.264714in}{2.299576in}}{\pgfqpoint{1.256477in}{2.299576in}}%
\pgfpathcurveto{\pgfqpoint{1.248241in}{2.299576in}}{\pgfqpoint{1.240341in}{2.296304in}}{\pgfqpoint{1.234517in}{2.290480in}}%
\pgfpathcurveto{\pgfqpoint{1.228693in}{2.284656in}}{\pgfqpoint{1.225421in}{2.276756in}}{\pgfqpoint{1.225421in}{2.268520in}}%
\pgfpathcurveto{\pgfqpoint{1.225421in}{2.260283in}}{\pgfqpoint{1.228693in}{2.252383in}}{\pgfqpoint{1.234517in}{2.246559in}}%
\pgfpathcurveto{\pgfqpoint{1.240341in}{2.240736in}}{\pgfqpoint{1.248241in}{2.237463in}}{\pgfqpoint{1.256477in}{2.237463in}}%
\pgfpathclose%
\pgfusepath{stroke,fill}%
\end{pgfscope}%
\begin{pgfscope}%
\pgfpathrectangle{\pgfqpoint{0.100000in}{0.212622in}}{\pgfqpoint{3.696000in}{3.696000in}}%
\pgfusepath{clip}%
\pgfsetbuttcap%
\pgfsetroundjoin%
\definecolor{currentfill}{rgb}{0.121569,0.466667,0.705882}%
\pgfsetfillcolor{currentfill}%
\pgfsetfillopacity{0.500787}%
\pgfsetlinewidth{1.003750pt}%
\definecolor{currentstroke}{rgb}{0.121569,0.466667,0.705882}%
\pgfsetstrokecolor{currentstroke}%
\pgfsetstrokeopacity{0.500787}%
\pgfsetdash{}{0pt}%
\pgfpathmoveto{\pgfqpoint{2.038287in}{2.496419in}}%
\pgfpathcurveto{\pgfqpoint{2.046523in}{2.496419in}}{\pgfqpoint{2.054423in}{2.499691in}}{\pgfqpoint{2.060247in}{2.505515in}}%
\pgfpathcurveto{\pgfqpoint{2.066071in}{2.511339in}}{\pgfqpoint{2.069343in}{2.519239in}}{\pgfqpoint{2.069343in}{2.527476in}}%
\pgfpathcurveto{\pgfqpoint{2.069343in}{2.535712in}}{\pgfqpoint{2.066071in}{2.543612in}}{\pgfqpoint{2.060247in}{2.549436in}}%
\pgfpathcurveto{\pgfqpoint{2.054423in}{2.555260in}}{\pgfqpoint{2.046523in}{2.558532in}}{\pgfqpoint{2.038287in}{2.558532in}}%
\pgfpathcurveto{\pgfqpoint{2.030051in}{2.558532in}}{\pgfqpoint{2.022151in}{2.555260in}}{\pgfqpoint{2.016327in}{2.549436in}}%
\pgfpathcurveto{\pgfqpoint{2.010503in}{2.543612in}}{\pgfqpoint{2.007230in}{2.535712in}}{\pgfqpoint{2.007230in}{2.527476in}}%
\pgfpathcurveto{\pgfqpoint{2.007230in}{2.519239in}}{\pgfqpoint{2.010503in}{2.511339in}}{\pgfqpoint{2.016327in}{2.505515in}}%
\pgfpathcurveto{\pgfqpoint{2.022151in}{2.499691in}}{\pgfqpoint{2.030051in}{2.496419in}}{\pgfqpoint{2.038287in}{2.496419in}}%
\pgfpathclose%
\pgfusepath{stroke,fill}%
\end{pgfscope}%
\begin{pgfscope}%
\pgfpathrectangle{\pgfqpoint{0.100000in}{0.212622in}}{\pgfqpoint{3.696000in}{3.696000in}}%
\pgfusepath{clip}%
\pgfsetbuttcap%
\pgfsetroundjoin%
\definecolor{currentfill}{rgb}{0.121569,0.466667,0.705882}%
\pgfsetfillcolor{currentfill}%
\pgfsetfillopacity{0.501385}%
\pgfsetlinewidth{1.003750pt}%
\definecolor{currentstroke}{rgb}{0.121569,0.466667,0.705882}%
\pgfsetstrokecolor{currentstroke}%
\pgfsetstrokeopacity{0.501385}%
\pgfsetdash{}{0pt}%
\pgfpathmoveto{\pgfqpoint{1.251284in}{2.228493in}}%
\pgfpathcurveto{\pgfqpoint{1.259520in}{2.228493in}}{\pgfqpoint{1.267420in}{2.231765in}}{\pgfqpoint{1.273244in}{2.237589in}}%
\pgfpathcurveto{\pgfqpoint{1.279068in}{2.243413in}}{\pgfqpoint{1.282341in}{2.251313in}}{\pgfqpoint{1.282341in}{2.259550in}}%
\pgfpathcurveto{\pgfqpoint{1.282341in}{2.267786in}}{\pgfqpoint{1.279068in}{2.275686in}}{\pgfqpoint{1.273244in}{2.281510in}}%
\pgfpathcurveto{\pgfqpoint{1.267420in}{2.287334in}}{\pgfqpoint{1.259520in}{2.290606in}}{\pgfqpoint{1.251284in}{2.290606in}}%
\pgfpathcurveto{\pgfqpoint{1.243048in}{2.290606in}}{\pgfqpoint{1.235148in}{2.287334in}}{\pgfqpoint{1.229324in}{2.281510in}}%
\pgfpathcurveto{\pgfqpoint{1.223500in}{2.275686in}}{\pgfqpoint{1.220228in}{2.267786in}}{\pgfqpoint{1.220228in}{2.259550in}}%
\pgfpathcurveto{\pgfqpoint{1.220228in}{2.251313in}}{\pgfqpoint{1.223500in}{2.243413in}}{\pgfqpoint{1.229324in}{2.237589in}}%
\pgfpathcurveto{\pgfqpoint{1.235148in}{2.231765in}}{\pgfqpoint{1.243048in}{2.228493in}}{\pgfqpoint{1.251284in}{2.228493in}}%
\pgfpathclose%
\pgfusepath{stroke,fill}%
\end{pgfscope}%
\begin{pgfscope}%
\pgfpathrectangle{\pgfqpoint{0.100000in}{0.212622in}}{\pgfqpoint{3.696000in}{3.696000in}}%
\pgfusepath{clip}%
\pgfsetbuttcap%
\pgfsetroundjoin%
\definecolor{currentfill}{rgb}{0.121569,0.466667,0.705882}%
\pgfsetfillcolor{currentfill}%
\pgfsetfillopacity{0.502174}%
\pgfsetlinewidth{1.003750pt}%
\definecolor{currentstroke}{rgb}{0.121569,0.466667,0.705882}%
\pgfsetstrokecolor{currentstroke}%
\pgfsetstrokeopacity{0.502174}%
\pgfsetdash{}{0pt}%
\pgfpathmoveto{\pgfqpoint{2.039487in}{2.488574in}}%
\pgfpathcurveto{\pgfqpoint{2.047723in}{2.488574in}}{\pgfqpoint{2.055623in}{2.491846in}}{\pgfqpoint{2.061447in}{2.497670in}}%
\pgfpathcurveto{\pgfqpoint{2.067271in}{2.503494in}}{\pgfqpoint{2.070543in}{2.511394in}}{\pgfqpoint{2.070543in}{2.519630in}}%
\pgfpathcurveto{\pgfqpoint{2.070543in}{2.527866in}}{\pgfqpoint{2.067271in}{2.535767in}}{\pgfqpoint{2.061447in}{2.541590in}}%
\pgfpathcurveto{\pgfqpoint{2.055623in}{2.547414in}}{\pgfqpoint{2.047723in}{2.550687in}}{\pgfqpoint{2.039487in}{2.550687in}}%
\pgfpathcurveto{\pgfqpoint{2.031250in}{2.550687in}}{\pgfqpoint{2.023350in}{2.547414in}}{\pgfqpoint{2.017526in}{2.541590in}}%
\pgfpathcurveto{\pgfqpoint{2.011702in}{2.535767in}}{\pgfqpoint{2.008430in}{2.527866in}}{\pgfqpoint{2.008430in}{2.519630in}}%
\pgfpathcurveto{\pgfqpoint{2.008430in}{2.511394in}}{\pgfqpoint{2.011702in}{2.503494in}}{\pgfqpoint{2.017526in}{2.497670in}}%
\pgfpathcurveto{\pgfqpoint{2.023350in}{2.491846in}}{\pgfqpoint{2.031250in}{2.488574in}}{\pgfqpoint{2.039487in}{2.488574in}}%
\pgfpathclose%
\pgfusepath{stroke,fill}%
\end{pgfscope}%
\begin{pgfscope}%
\pgfpathrectangle{\pgfqpoint{0.100000in}{0.212622in}}{\pgfqpoint{3.696000in}{3.696000in}}%
\pgfusepath{clip}%
\pgfsetbuttcap%
\pgfsetroundjoin%
\definecolor{currentfill}{rgb}{0.121569,0.466667,0.705882}%
\pgfsetfillcolor{currentfill}%
\pgfsetfillopacity{0.502817}%
\pgfsetlinewidth{1.003750pt}%
\definecolor{currentstroke}{rgb}{0.121569,0.466667,0.705882}%
\pgfsetstrokecolor{currentstroke}%
\pgfsetstrokeopacity{0.502817}%
\pgfsetdash{}{0pt}%
\pgfpathmoveto{\pgfqpoint{1.247241in}{2.219081in}}%
\pgfpathcurveto{\pgfqpoint{1.255477in}{2.219081in}}{\pgfqpoint{1.263377in}{2.222353in}}{\pgfqpoint{1.269201in}{2.228177in}}%
\pgfpathcurveto{\pgfqpoint{1.275025in}{2.234001in}}{\pgfqpoint{1.278297in}{2.241901in}}{\pgfqpoint{1.278297in}{2.250137in}}%
\pgfpathcurveto{\pgfqpoint{1.278297in}{2.258373in}}{\pgfqpoint{1.275025in}{2.266273in}}{\pgfqpoint{1.269201in}{2.272097in}}%
\pgfpathcurveto{\pgfqpoint{1.263377in}{2.277921in}}{\pgfqpoint{1.255477in}{2.281194in}}{\pgfqpoint{1.247241in}{2.281194in}}%
\pgfpathcurveto{\pgfqpoint{1.239004in}{2.281194in}}{\pgfqpoint{1.231104in}{2.277921in}}{\pgfqpoint{1.225281in}{2.272097in}}%
\pgfpathcurveto{\pgfqpoint{1.219457in}{2.266273in}}{\pgfqpoint{1.216184in}{2.258373in}}{\pgfqpoint{1.216184in}{2.250137in}}%
\pgfpathcurveto{\pgfqpoint{1.216184in}{2.241901in}}{\pgfqpoint{1.219457in}{2.234001in}}{\pgfqpoint{1.225281in}{2.228177in}}%
\pgfpathcurveto{\pgfqpoint{1.231104in}{2.222353in}}{\pgfqpoint{1.239004in}{2.219081in}}{\pgfqpoint{1.247241in}{2.219081in}}%
\pgfpathclose%
\pgfusepath{stroke,fill}%
\end{pgfscope}%
\begin{pgfscope}%
\pgfpathrectangle{\pgfqpoint{0.100000in}{0.212622in}}{\pgfqpoint{3.696000in}{3.696000in}}%
\pgfusepath{clip}%
\pgfsetbuttcap%
\pgfsetroundjoin%
\definecolor{currentfill}{rgb}{0.121569,0.466667,0.705882}%
\pgfsetfillcolor{currentfill}%
\pgfsetfillopacity{0.504047}%
\pgfsetlinewidth{1.003750pt}%
\definecolor{currentstroke}{rgb}{0.121569,0.466667,0.705882}%
\pgfsetstrokecolor{currentstroke}%
\pgfsetstrokeopacity{0.504047}%
\pgfsetdash{}{0pt}%
\pgfpathmoveto{\pgfqpoint{2.040333in}{2.479730in}}%
\pgfpathcurveto{\pgfqpoint{2.048570in}{2.479730in}}{\pgfqpoint{2.056470in}{2.483002in}}{\pgfqpoint{2.062294in}{2.488826in}}%
\pgfpathcurveto{\pgfqpoint{2.068118in}{2.494650in}}{\pgfqpoint{2.071390in}{2.502550in}}{\pgfqpoint{2.071390in}{2.510786in}}%
\pgfpathcurveto{\pgfqpoint{2.071390in}{2.519022in}}{\pgfqpoint{2.068118in}{2.526923in}}{\pgfqpoint{2.062294in}{2.532746in}}%
\pgfpathcurveto{\pgfqpoint{2.056470in}{2.538570in}}{\pgfqpoint{2.048570in}{2.541843in}}{\pgfqpoint{2.040333in}{2.541843in}}%
\pgfpathcurveto{\pgfqpoint{2.032097in}{2.541843in}}{\pgfqpoint{2.024197in}{2.538570in}}{\pgfqpoint{2.018373in}{2.532746in}}%
\pgfpathcurveto{\pgfqpoint{2.012549in}{2.526923in}}{\pgfqpoint{2.009277in}{2.519022in}}{\pgfqpoint{2.009277in}{2.510786in}}%
\pgfpathcurveto{\pgfqpoint{2.009277in}{2.502550in}}{\pgfqpoint{2.012549in}{2.494650in}}{\pgfqpoint{2.018373in}{2.488826in}}%
\pgfpathcurveto{\pgfqpoint{2.024197in}{2.483002in}}{\pgfqpoint{2.032097in}{2.479730in}}{\pgfqpoint{2.040333in}{2.479730in}}%
\pgfpathclose%
\pgfusepath{stroke,fill}%
\end{pgfscope}%
\begin{pgfscope}%
\pgfpathrectangle{\pgfqpoint{0.100000in}{0.212622in}}{\pgfqpoint{3.696000in}{3.696000in}}%
\pgfusepath{clip}%
\pgfsetbuttcap%
\pgfsetroundjoin%
\definecolor{currentfill}{rgb}{0.121569,0.466667,0.705882}%
\pgfsetfillcolor{currentfill}%
\pgfsetfillopacity{0.504312}%
\pgfsetlinewidth{1.003750pt}%
\definecolor{currentstroke}{rgb}{0.121569,0.466667,0.705882}%
\pgfsetstrokecolor{currentstroke}%
\pgfsetstrokeopacity{0.504312}%
\pgfsetdash{}{0pt}%
\pgfpathmoveto{\pgfqpoint{1.242735in}{2.211571in}}%
\pgfpathcurveto{\pgfqpoint{1.250971in}{2.211571in}}{\pgfqpoint{1.258871in}{2.214843in}}{\pgfqpoint{1.264695in}{2.220667in}}%
\pgfpathcurveto{\pgfqpoint{1.270519in}{2.226491in}}{\pgfqpoint{1.273791in}{2.234391in}}{\pgfqpoint{1.273791in}{2.242627in}}%
\pgfpathcurveto{\pgfqpoint{1.273791in}{2.250864in}}{\pgfqpoint{1.270519in}{2.258764in}}{\pgfqpoint{1.264695in}{2.264588in}}%
\pgfpathcurveto{\pgfqpoint{1.258871in}{2.270412in}}{\pgfqpoint{1.250971in}{2.273684in}}{\pgfqpoint{1.242735in}{2.273684in}}%
\pgfpathcurveto{\pgfqpoint{1.234499in}{2.273684in}}{\pgfqpoint{1.226598in}{2.270412in}}{\pgfqpoint{1.220775in}{2.264588in}}%
\pgfpathcurveto{\pgfqpoint{1.214951in}{2.258764in}}{\pgfqpoint{1.211678in}{2.250864in}}{\pgfqpoint{1.211678in}{2.242627in}}%
\pgfpathcurveto{\pgfqpoint{1.211678in}{2.234391in}}{\pgfqpoint{1.214951in}{2.226491in}}{\pgfqpoint{1.220775in}{2.220667in}}%
\pgfpathcurveto{\pgfqpoint{1.226598in}{2.214843in}}{\pgfqpoint{1.234499in}{2.211571in}}{\pgfqpoint{1.242735in}{2.211571in}}%
\pgfpathclose%
\pgfusepath{stroke,fill}%
\end{pgfscope}%
\begin{pgfscope}%
\pgfpathrectangle{\pgfqpoint{0.100000in}{0.212622in}}{\pgfqpoint{3.696000in}{3.696000in}}%
\pgfusepath{clip}%
\pgfsetbuttcap%
\pgfsetroundjoin%
\definecolor{currentfill}{rgb}{0.121569,0.466667,0.705882}%
\pgfsetfillcolor{currentfill}%
\pgfsetfillopacity{0.505098}%
\pgfsetlinewidth{1.003750pt}%
\definecolor{currentstroke}{rgb}{0.121569,0.466667,0.705882}%
\pgfsetstrokecolor{currentstroke}%
\pgfsetstrokeopacity{0.505098}%
\pgfsetdash{}{0pt}%
\pgfpathmoveto{\pgfqpoint{2.040739in}{2.474898in}}%
\pgfpathcurveto{\pgfqpoint{2.048975in}{2.474898in}}{\pgfqpoint{2.056875in}{2.478171in}}{\pgfqpoint{2.062699in}{2.483995in}}%
\pgfpathcurveto{\pgfqpoint{2.068523in}{2.489819in}}{\pgfqpoint{2.071796in}{2.497719in}}{\pgfqpoint{2.071796in}{2.505955in}}%
\pgfpathcurveto{\pgfqpoint{2.071796in}{2.514191in}}{\pgfqpoint{2.068523in}{2.522091in}}{\pgfqpoint{2.062699in}{2.527915in}}%
\pgfpathcurveto{\pgfqpoint{2.056875in}{2.533739in}}{\pgfqpoint{2.048975in}{2.537011in}}{\pgfqpoint{2.040739in}{2.537011in}}%
\pgfpathcurveto{\pgfqpoint{2.032503in}{2.537011in}}{\pgfqpoint{2.024603in}{2.533739in}}{\pgfqpoint{2.018779in}{2.527915in}}%
\pgfpathcurveto{\pgfqpoint{2.012955in}{2.522091in}}{\pgfqpoint{2.009683in}{2.514191in}}{\pgfqpoint{2.009683in}{2.505955in}}%
\pgfpathcurveto{\pgfqpoint{2.009683in}{2.497719in}}{\pgfqpoint{2.012955in}{2.489819in}}{\pgfqpoint{2.018779in}{2.483995in}}%
\pgfpathcurveto{\pgfqpoint{2.024603in}{2.478171in}}{\pgfqpoint{2.032503in}{2.474898in}}{\pgfqpoint{2.040739in}{2.474898in}}%
\pgfpathclose%
\pgfusepath{stroke,fill}%
\end{pgfscope}%
\begin{pgfscope}%
\pgfpathrectangle{\pgfqpoint{0.100000in}{0.212622in}}{\pgfqpoint{3.696000in}{3.696000in}}%
\pgfusepath{clip}%
\pgfsetbuttcap%
\pgfsetroundjoin%
\definecolor{currentfill}{rgb}{0.121569,0.466667,0.705882}%
\pgfsetfillcolor{currentfill}%
\pgfsetfillopacity{0.505590}%
\pgfsetlinewidth{1.003750pt}%
\definecolor{currentstroke}{rgb}{0.121569,0.466667,0.705882}%
\pgfsetstrokecolor{currentstroke}%
\pgfsetstrokeopacity{0.505590}%
\pgfsetdash{}{0pt}%
\pgfpathmoveto{\pgfqpoint{1.238859in}{2.204367in}}%
\pgfpathcurveto{\pgfqpoint{1.247095in}{2.204367in}}{\pgfqpoint{1.254995in}{2.207639in}}{\pgfqpoint{1.260819in}{2.213463in}}%
\pgfpathcurveto{\pgfqpoint{1.266643in}{2.219287in}}{\pgfqpoint{1.269915in}{2.227187in}}{\pgfqpoint{1.269915in}{2.235424in}}%
\pgfpathcurveto{\pgfqpoint{1.269915in}{2.243660in}}{\pgfqpoint{1.266643in}{2.251560in}}{\pgfqpoint{1.260819in}{2.257384in}}%
\pgfpathcurveto{\pgfqpoint{1.254995in}{2.263208in}}{\pgfqpoint{1.247095in}{2.266480in}}{\pgfqpoint{1.238859in}{2.266480in}}%
\pgfpathcurveto{\pgfqpoint{1.230623in}{2.266480in}}{\pgfqpoint{1.222723in}{2.263208in}}{\pgfqpoint{1.216899in}{2.257384in}}%
\pgfpathcurveto{\pgfqpoint{1.211075in}{2.251560in}}{\pgfqpoint{1.207802in}{2.243660in}}{\pgfqpoint{1.207802in}{2.235424in}}%
\pgfpathcurveto{\pgfqpoint{1.207802in}{2.227187in}}{\pgfqpoint{1.211075in}{2.219287in}}{\pgfqpoint{1.216899in}{2.213463in}}%
\pgfpathcurveto{\pgfqpoint{1.222723in}{2.207639in}}{\pgfqpoint{1.230623in}{2.204367in}}{\pgfqpoint{1.238859in}{2.204367in}}%
\pgfpathclose%
\pgfusepath{stroke,fill}%
\end{pgfscope}%
\begin{pgfscope}%
\pgfpathrectangle{\pgfqpoint{0.100000in}{0.212622in}}{\pgfqpoint{3.696000in}{3.696000in}}%
\pgfusepath{clip}%
\pgfsetbuttcap%
\pgfsetroundjoin%
\definecolor{currentfill}{rgb}{0.121569,0.466667,0.705882}%
\pgfsetfillcolor{currentfill}%
\pgfsetfillopacity{0.505990}%
\pgfsetlinewidth{1.003750pt}%
\definecolor{currentstroke}{rgb}{0.121569,0.466667,0.705882}%
\pgfsetstrokecolor{currentstroke}%
\pgfsetstrokeopacity{0.505990}%
\pgfsetdash{}{0pt}%
\pgfpathmoveto{\pgfqpoint{1.237431in}{2.201752in}}%
\pgfpathcurveto{\pgfqpoint{1.245667in}{2.201752in}}{\pgfqpoint{1.253567in}{2.205025in}}{\pgfqpoint{1.259391in}{2.210848in}}%
\pgfpathcurveto{\pgfqpoint{1.265215in}{2.216672in}}{\pgfqpoint{1.268487in}{2.224572in}}{\pgfqpoint{1.268487in}{2.232809in}}%
\pgfpathcurveto{\pgfqpoint{1.268487in}{2.241045in}}{\pgfqpoint{1.265215in}{2.248945in}}{\pgfqpoint{1.259391in}{2.254769in}}%
\pgfpathcurveto{\pgfqpoint{1.253567in}{2.260593in}}{\pgfqpoint{1.245667in}{2.263865in}}{\pgfqpoint{1.237431in}{2.263865in}}%
\pgfpathcurveto{\pgfqpoint{1.229195in}{2.263865in}}{\pgfqpoint{1.221295in}{2.260593in}}{\pgfqpoint{1.215471in}{2.254769in}}%
\pgfpathcurveto{\pgfqpoint{1.209647in}{2.248945in}}{\pgfqpoint{1.206374in}{2.241045in}}{\pgfqpoint{1.206374in}{2.232809in}}%
\pgfpathcurveto{\pgfqpoint{1.206374in}{2.224572in}}{\pgfqpoint{1.209647in}{2.216672in}}{\pgfqpoint{1.215471in}{2.210848in}}%
\pgfpathcurveto{\pgfqpoint{1.221295in}{2.205025in}}{\pgfqpoint{1.229195in}{2.201752in}}{\pgfqpoint{1.237431in}{2.201752in}}%
\pgfpathclose%
\pgfusepath{stroke,fill}%
\end{pgfscope}%
\begin{pgfscope}%
\pgfpathrectangle{\pgfqpoint{0.100000in}{0.212622in}}{\pgfqpoint{3.696000in}{3.696000in}}%
\pgfusepath{clip}%
\pgfsetbuttcap%
\pgfsetroundjoin%
\definecolor{currentfill}{rgb}{0.121569,0.466667,0.705882}%
\pgfsetfillcolor{currentfill}%
\pgfsetfillopacity{0.506089}%
\pgfsetlinewidth{1.003750pt}%
\definecolor{currentstroke}{rgb}{0.121569,0.466667,0.705882}%
\pgfsetstrokecolor{currentstroke}%
\pgfsetstrokeopacity{0.506089}%
\pgfsetdash{}{0pt}%
\pgfpathmoveto{\pgfqpoint{2.041580in}{2.469239in}}%
\pgfpathcurveto{\pgfqpoint{2.049816in}{2.469239in}}{\pgfqpoint{2.057716in}{2.472512in}}{\pgfqpoint{2.063540in}{2.478336in}}%
\pgfpathcurveto{\pgfqpoint{2.069364in}{2.484160in}}{\pgfqpoint{2.072636in}{2.492060in}}{\pgfqpoint{2.072636in}{2.500296in}}%
\pgfpathcurveto{\pgfqpoint{2.072636in}{2.508532in}}{\pgfqpoint{2.069364in}{2.516432in}}{\pgfqpoint{2.063540in}{2.522256in}}%
\pgfpathcurveto{\pgfqpoint{2.057716in}{2.528080in}}{\pgfqpoint{2.049816in}{2.531352in}}{\pgfqpoint{2.041580in}{2.531352in}}%
\pgfpathcurveto{\pgfqpoint{2.033343in}{2.531352in}}{\pgfqpoint{2.025443in}{2.528080in}}{\pgfqpoint{2.019619in}{2.522256in}}%
\pgfpathcurveto{\pgfqpoint{2.013795in}{2.516432in}}{\pgfqpoint{2.010523in}{2.508532in}}{\pgfqpoint{2.010523in}{2.500296in}}%
\pgfpathcurveto{\pgfqpoint{2.010523in}{2.492060in}}{\pgfqpoint{2.013795in}{2.484160in}}{\pgfqpoint{2.019619in}{2.478336in}}%
\pgfpathcurveto{\pgfqpoint{2.025443in}{2.472512in}}{\pgfqpoint{2.033343in}{2.469239in}}{\pgfqpoint{2.041580in}{2.469239in}}%
\pgfpathclose%
\pgfusepath{stroke,fill}%
\end{pgfscope}%
\begin{pgfscope}%
\pgfpathrectangle{\pgfqpoint{0.100000in}{0.212622in}}{\pgfqpoint{3.696000in}{3.696000in}}%
\pgfusepath{clip}%
\pgfsetbuttcap%
\pgfsetroundjoin%
\definecolor{currentfill}{rgb}{0.121569,0.466667,0.705882}%
\pgfsetfillcolor{currentfill}%
\pgfsetfillopacity{0.506278}%
\pgfsetlinewidth{1.003750pt}%
\definecolor{currentstroke}{rgb}{0.121569,0.466667,0.705882}%
\pgfsetstrokecolor{currentstroke}%
\pgfsetstrokeopacity{0.506278}%
\pgfsetdash{}{0pt}%
\pgfpathmoveto{\pgfqpoint{1.236251in}{2.199264in}}%
\pgfpathcurveto{\pgfqpoint{1.244488in}{2.199264in}}{\pgfqpoint{1.252388in}{2.202537in}}{\pgfqpoint{1.258212in}{2.208361in}}%
\pgfpathcurveto{\pgfqpoint{1.264036in}{2.214184in}}{\pgfqpoint{1.267308in}{2.222085in}}{\pgfqpoint{1.267308in}{2.230321in}}%
\pgfpathcurveto{\pgfqpoint{1.267308in}{2.238557in}}{\pgfqpoint{1.264036in}{2.246457in}}{\pgfqpoint{1.258212in}{2.252281in}}%
\pgfpathcurveto{\pgfqpoint{1.252388in}{2.258105in}}{\pgfqpoint{1.244488in}{2.261377in}}{\pgfqpoint{1.236251in}{2.261377in}}%
\pgfpathcurveto{\pgfqpoint{1.228015in}{2.261377in}}{\pgfqpoint{1.220115in}{2.258105in}}{\pgfqpoint{1.214291in}{2.252281in}}%
\pgfpathcurveto{\pgfqpoint{1.208467in}{2.246457in}}{\pgfqpoint{1.205195in}{2.238557in}}{\pgfqpoint{1.205195in}{2.230321in}}%
\pgfpathcurveto{\pgfqpoint{1.205195in}{2.222085in}}{\pgfqpoint{1.208467in}{2.214184in}}{\pgfqpoint{1.214291in}{2.208361in}}%
\pgfpathcurveto{\pgfqpoint{1.220115in}{2.202537in}}{\pgfqpoint{1.228015in}{2.199264in}}{\pgfqpoint{1.236251in}{2.199264in}}%
\pgfpathclose%
\pgfusepath{stroke,fill}%
\end{pgfscope}%
\begin{pgfscope}%
\pgfpathrectangle{\pgfqpoint{0.100000in}{0.212622in}}{\pgfqpoint{3.696000in}{3.696000in}}%
\pgfusepath{clip}%
\pgfsetbuttcap%
\pgfsetroundjoin%
\definecolor{currentfill}{rgb}{0.121569,0.466667,0.705882}%
\pgfsetfillcolor{currentfill}%
\pgfsetfillopacity{0.506629}%
\pgfsetlinewidth{1.003750pt}%
\definecolor{currentstroke}{rgb}{0.121569,0.466667,0.705882}%
\pgfsetstrokecolor{currentstroke}%
\pgfsetstrokeopacity{0.506629}%
\pgfsetdash{}{0pt}%
\pgfpathmoveto{\pgfqpoint{1.235256in}{2.197725in}}%
\pgfpathcurveto{\pgfqpoint{1.243492in}{2.197725in}}{\pgfqpoint{1.251392in}{2.200997in}}{\pgfqpoint{1.257216in}{2.206821in}}%
\pgfpathcurveto{\pgfqpoint{1.263040in}{2.212645in}}{\pgfqpoint{1.266312in}{2.220545in}}{\pgfqpoint{1.266312in}{2.228781in}}%
\pgfpathcurveto{\pgfqpoint{1.266312in}{2.237018in}}{\pgfqpoint{1.263040in}{2.244918in}}{\pgfqpoint{1.257216in}{2.250742in}}%
\pgfpathcurveto{\pgfqpoint{1.251392in}{2.256565in}}{\pgfqpoint{1.243492in}{2.259838in}}{\pgfqpoint{1.235256in}{2.259838in}}%
\pgfpathcurveto{\pgfqpoint{1.227019in}{2.259838in}}{\pgfqpoint{1.219119in}{2.256565in}}{\pgfqpoint{1.213295in}{2.250742in}}%
\pgfpathcurveto{\pgfqpoint{1.207472in}{2.244918in}}{\pgfqpoint{1.204199in}{2.237018in}}{\pgfqpoint{1.204199in}{2.228781in}}%
\pgfpathcurveto{\pgfqpoint{1.204199in}{2.220545in}}{\pgfqpoint{1.207472in}{2.212645in}}{\pgfqpoint{1.213295in}{2.206821in}}%
\pgfpathcurveto{\pgfqpoint{1.219119in}{2.200997in}}{\pgfqpoint{1.227019in}{2.197725in}}{\pgfqpoint{1.235256in}{2.197725in}}%
\pgfpathclose%
\pgfusepath{stroke,fill}%
\end{pgfscope}%
\begin{pgfscope}%
\pgfpathrectangle{\pgfqpoint{0.100000in}{0.212622in}}{\pgfqpoint{3.696000in}{3.696000in}}%
\pgfusepath{clip}%
\pgfsetbuttcap%
\pgfsetroundjoin%
\definecolor{currentfill}{rgb}{0.121569,0.466667,0.705882}%
\pgfsetfillcolor{currentfill}%
\pgfsetfillopacity{0.507284}%
\pgfsetlinewidth{1.003750pt}%
\definecolor{currentstroke}{rgb}{0.121569,0.466667,0.705882}%
\pgfsetstrokecolor{currentstroke}%
\pgfsetstrokeopacity{0.507284}%
\pgfsetdash{}{0pt}%
\pgfpathmoveto{\pgfqpoint{1.233482in}{2.194925in}}%
\pgfpathcurveto{\pgfqpoint{1.241718in}{2.194925in}}{\pgfqpoint{1.249618in}{2.198198in}}{\pgfqpoint{1.255442in}{2.204022in}}%
\pgfpathcurveto{\pgfqpoint{1.261266in}{2.209846in}}{\pgfqpoint{1.264539in}{2.217746in}}{\pgfqpoint{1.264539in}{2.225982in}}%
\pgfpathcurveto{\pgfqpoint{1.264539in}{2.234218in}}{\pgfqpoint{1.261266in}{2.242118in}}{\pgfqpoint{1.255442in}{2.247942in}}%
\pgfpathcurveto{\pgfqpoint{1.249618in}{2.253766in}}{\pgfqpoint{1.241718in}{2.257038in}}{\pgfqpoint{1.233482in}{2.257038in}}%
\pgfpathcurveto{\pgfqpoint{1.225246in}{2.257038in}}{\pgfqpoint{1.217346in}{2.253766in}}{\pgfqpoint{1.211522in}{2.247942in}}%
\pgfpathcurveto{\pgfqpoint{1.205698in}{2.242118in}}{\pgfqpoint{1.202426in}{2.234218in}}{\pgfqpoint{1.202426in}{2.225982in}}%
\pgfpathcurveto{\pgfqpoint{1.202426in}{2.217746in}}{\pgfqpoint{1.205698in}{2.209846in}}{\pgfqpoint{1.211522in}{2.204022in}}%
\pgfpathcurveto{\pgfqpoint{1.217346in}{2.198198in}}{\pgfqpoint{1.225246in}{2.194925in}}{\pgfqpoint{1.233482in}{2.194925in}}%
\pgfpathclose%
\pgfusepath{stroke,fill}%
\end{pgfscope}%
\begin{pgfscope}%
\pgfpathrectangle{\pgfqpoint{0.100000in}{0.212622in}}{\pgfqpoint{3.696000in}{3.696000in}}%
\pgfusepath{clip}%
\pgfsetbuttcap%
\pgfsetroundjoin%
\definecolor{currentfill}{rgb}{0.121569,0.466667,0.705882}%
\pgfsetfillcolor{currentfill}%
\pgfsetfillopacity{0.507387}%
\pgfsetlinewidth{1.003750pt}%
\definecolor{currentstroke}{rgb}{0.121569,0.466667,0.705882}%
\pgfsetstrokecolor{currentstroke}%
\pgfsetstrokeopacity{0.507387}%
\pgfsetdash{}{0pt}%
\pgfpathmoveto{\pgfqpoint{2.042131in}{2.462482in}}%
\pgfpathcurveto{\pgfqpoint{2.050367in}{2.462482in}}{\pgfqpoint{2.058267in}{2.465754in}}{\pgfqpoint{2.064091in}{2.471578in}}%
\pgfpathcurveto{\pgfqpoint{2.069915in}{2.477402in}}{\pgfqpoint{2.073187in}{2.485302in}}{\pgfqpoint{2.073187in}{2.493538in}}%
\pgfpathcurveto{\pgfqpoint{2.073187in}{2.501775in}}{\pgfqpoint{2.069915in}{2.509675in}}{\pgfqpoint{2.064091in}{2.515499in}}%
\pgfpathcurveto{\pgfqpoint{2.058267in}{2.521322in}}{\pgfqpoint{2.050367in}{2.524595in}}{\pgfqpoint{2.042131in}{2.524595in}}%
\pgfpathcurveto{\pgfqpoint{2.033895in}{2.524595in}}{\pgfqpoint{2.025995in}{2.521322in}}{\pgfqpoint{2.020171in}{2.515499in}}%
\pgfpathcurveto{\pgfqpoint{2.014347in}{2.509675in}}{\pgfqpoint{2.011074in}{2.501775in}}{\pgfqpoint{2.011074in}{2.493538in}}%
\pgfpathcurveto{\pgfqpoint{2.011074in}{2.485302in}}{\pgfqpoint{2.014347in}{2.477402in}}{\pgfqpoint{2.020171in}{2.471578in}}%
\pgfpathcurveto{\pgfqpoint{2.025995in}{2.465754in}}{\pgfqpoint{2.033895in}{2.462482in}}{\pgfqpoint{2.042131in}{2.462482in}}%
\pgfpathclose%
\pgfusepath{stroke,fill}%
\end{pgfscope}%
\begin{pgfscope}%
\pgfpathrectangle{\pgfqpoint{0.100000in}{0.212622in}}{\pgfqpoint{3.696000in}{3.696000in}}%
\pgfusepath{clip}%
\pgfsetbuttcap%
\pgfsetroundjoin%
\definecolor{currentfill}{rgb}{0.121569,0.466667,0.705882}%
\pgfsetfillcolor{currentfill}%
\pgfsetfillopacity{0.507855}%
\pgfsetlinewidth{1.003750pt}%
\definecolor{currentstroke}{rgb}{0.121569,0.466667,0.705882}%
\pgfsetstrokecolor{currentstroke}%
\pgfsetstrokeopacity{0.507855}%
\pgfsetdash{}{0pt}%
\pgfpathmoveto{\pgfqpoint{1.232011in}{2.192313in}}%
\pgfpathcurveto{\pgfqpoint{1.240247in}{2.192313in}}{\pgfqpoint{1.248147in}{2.195586in}}{\pgfqpoint{1.253971in}{2.201410in}}%
\pgfpathcurveto{\pgfqpoint{1.259795in}{2.207234in}}{\pgfqpoint{1.263068in}{2.215134in}}{\pgfqpoint{1.263068in}{2.223370in}}%
\pgfpathcurveto{\pgfqpoint{1.263068in}{2.231606in}}{\pgfqpoint{1.259795in}{2.239506in}}{\pgfqpoint{1.253971in}{2.245330in}}%
\pgfpathcurveto{\pgfqpoint{1.248147in}{2.251154in}}{\pgfqpoint{1.240247in}{2.254426in}}{\pgfqpoint{1.232011in}{2.254426in}}%
\pgfpathcurveto{\pgfqpoint{1.223775in}{2.254426in}}{\pgfqpoint{1.215875in}{2.251154in}}{\pgfqpoint{1.210051in}{2.245330in}}%
\pgfpathcurveto{\pgfqpoint{1.204227in}{2.239506in}}{\pgfqpoint{1.200955in}{2.231606in}}{\pgfqpoint{1.200955in}{2.223370in}}%
\pgfpathcurveto{\pgfqpoint{1.200955in}{2.215134in}}{\pgfqpoint{1.204227in}{2.207234in}}{\pgfqpoint{1.210051in}{2.201410in}}%
\pgfpathcurveto{\pgfqpoint{1.215875in}{2.195586in}}{\pgfqpoint{1.223775in}{2.192313in}}{\pgfqpoint{1.232011in}{2.192313in}}%
\pgfpathclose%
\pgfusepath{stroke,fill}%
\end{pgfscope}%
\begin{pgfscope}%
\pgfpathrectangle{\pgfqpoint{0.100000in}{0.212622in}}{\pgfqpoint{3.696000in}{3.696000in}}%
\pgfusepath{clip}%
\pgfsetbuttcap%
\pgfsetroundjoin%
\definecolor{currentfill}{rgb}{0.121569,0.466667,0.705882}%
\pgfsetfillcolor{currentfill}%
\pgfsetfillopacity{0.508830}%
\pgfsetlinewidth{1.003750pt}%
\definecolor{currentstroke}{rgb}{0.121569,0.466667,0.705882}%
\pgfsetstrokecolor{currentstroke}%
\pgfsetstrokeopacity{0.508830}%
\pgfsetdash{}{0pt}%
\pgfpathmoveto{\pgfqpoint{1.229107in}{2.187626in}}%
\pgfpathcurveto{\pgfqpoint{1.237343in}{2.187626in}}{\pgfqpoint{1.245243in}{2.190898in}}{\pgfqpoint{1.251067in}{2.196722in}}%
\pgfpathcurveto{\pgfqpoint{1.256891in}{2.202546in}}{\pgfqpoint{1.260163in}{2.210446in}}{\pgfqpoint{1.260163in}{2.218682in}}%
\pgfpathcurveto{\pgfqpoint{1.260163in}{2.226918in}}{\pgfqpoint{1.256891in}{2.234818in}}{\pgfqpoint{1.251067in}{2.240642in}}%
\pgfpathcurveto{\pgfqpoint{1.245243in}{2.246466in}}{\pgfqpoint{1.237343in}{2.249739in}}{\pgfqpoint{1.229107in}{2.249739in}}%
\pgfpathcurveto{\pgfqpoint{1.220870in}{2.249739in}}{\pgfqpoint{1.212970in}{2.246466in}}{\pgfqpoint{1.207146in}{2.240642in}}%
\pgfpathcurveto{\pgfqpoint{1.201322in}{2.234818in}}{\pgfqpoint{1.198050in}{2.226918in}}{\pgfqpoint{1.198050in}{2.218682in}}%
\pgfpathcurveto{\pgfqpoint{1.198050in}{2.210446in}}{\pgfqpoint{1.201322in}{2.202546in}}{\pgfqpoint{1.207146in}{2.196722in}}%
\pgfpathcurveto{\pgfqpoint{1.212970in}{2.190898in}}{\pgfqpoint{1.220870in}{2.187626in}}{\pgfqpoint{1.229107in}{2.187626in}}%
\pgfpathclose%
\pgfusepath{stroke,fill}%
\end{pgfscope}%
\begin{pgfscope}%
\pgfpathrectangle{\pgfqpoint{0.100000in}{0.212622in}}{\pgfqpoint{3.696000in}{3.696000in}}%
\pgfusepath{clip}%
\pgfsetbuttcap%
\pgfsetroundjoin%
\definecolor{currentfill}{rgb}{0.121569,0.466667,0.705882}%
\pgfsetfillcolor{currentfill}%
\pgfsetfillopacity{0.508915}%
\pgfsetlinewidth{1.003750pt}%
\definecolor{currentstroke}{rgb}{0.121569,0.466667,0.705882}%
\pgfsetstrokecolor{currentstroke}%
\pgfsetstrokeopacity{0.508915}%
\pgfsetdash{}{0pt}%
\pgfpathmoveto{\pgfqpoint{2.042686in}{2.455221in}}%
\pgfpathcurveto{\pgfqpoint{2.050922in}{2.455221in}}{\pgfqpoint{2.058822in}{2.458494in}}{\pgfqpoint{2.064646in}{2.464318in}}%
\pgfpathcurveto{\pgfqpoint{2.070470in}{2.470142in}}{\pgfqpoint{2.073742in}{2.478042in}}{\pgfqpoint{2.073742in}{2.486278in}}%
\pgfpathcurveto{\pgfqpoint{2.073742in}{2.494514in}}{\pgfqpoint{2.070470in}{2.502414in}}{\pgfqpoint{2.064646in}{2.508238in}}%
\pgfpathcurveto{\pgfqpoint{2.058822in}{2.514062in}}{\pgfqpoint{2.050922in}{2.517334in}}{\pgfqpoint{2.042686in}{2.517334in}}%
\pgfpathcurveto{\pgfqpoint{2.034449in}{2.517334in}}{\pgfqpoint{2.026549in}{2.514062in}}{\pgfqpoint{2.020725in}{2.508238in}}%
\pgfpathcurveto{\pgfqpoint{2.014902in}{2.502414in}}{\pgfqpoint{2.011629in}{2.494514in}}{\pgfqpoint{2.011629in}{2.486278in}}%
\pgfpathcurveto{\pgfqpoint{2.011629in}{2.478042in}}{\pgfqpoint{2.014902in}{2.470142in}}{\pgfqpoint{2.020725in}{2.464318in}}%
\pgfpathcurveto{\pgfqpoint{2.026549in}{2.458494in}}{\pgfqpoint{2.034449in}{2.455221in}}{\pgfqpoint{2.042686in}{2.455221in}}%
\pgfpathclose%
\pgfusepath{stroke,fill}%
\end{pgfscope}%
\begin{pgfscope}%
\pgfpathrectangle{\pgfqpoint{0.100000in}{0.212622in}}{\pgfqpoint{3.696000in}{3.696000in}}%
\pgfusepath{clip}%
\pgfsetbuttcap%
\pgfsetroundjoin%
\definecolor{currentfill}{rgb}{0.121569,0.466667,0.705882}%
\pgfsetfillcolor{currentfill}%
\pgfsetfillopacity{0.509730}%
\pgfsetlinewidth{1.003750pt}%
\definecolor{currentstroke}{rgb}{0.121569,0.466667,0.705882}%
\pgfsetstrokecolor{currentstroke}%
\pgfsetstrokeopacity{0.509730}%
\pgfsetdash{}{0pt}%
\pgfpathmoveto{\pgfqpoint{1.226591in}{2.183177in}}%
\pgfpathcurveto{\pgfqpoint{1.234827in}{2.183177in}}{\pgfqpoint{1.242727in}{2.186449in}}{\pgfqpoint{1.248551in}{2.192273in}}%
\pgfpathcurveto{\pgfqpoint{1.254375in}{2.198097in}}{\pgfqpoint{1.257648in}{2.205997in}}{\pgfqpoint{1.257648in}{2.214233in}}%
\pgfpathcurveto{\pgfqpoint{1.257648in}{2.222470in}}{\pgfqpoint{1.254375in}{2.230370in}}{\pgfqpoint{1.248551in}{2.236194in}}%
\pgfpathcurveto{\pgfqpoint{1.242727in}{2.242018in}}{\pgfqpoint{1.234827in}{2.245290in}}{\pgfqpoint{1.226591in}{2.245290in}}%
\pgfpathcurveto{\pgfqpoint{1.218355in}{2.245290in}}{\pgfqpoint{1.210455in}{2.242018in}}{\pgfqpoint{1.204631in}{2.236194in}}%
\pgfpathcurveto{\pgfqpoint{1.198807in}{2.230370in}}{\pgfqpoint{1.195535in}{2.222470in}}{\pgfqpoint{1.195535in}{2.214233in}}%
\pgfpathcurveto{\pgfqpoint{1.195535in}{2.205997in}}{\pgfqpoint{1.198807in}{2.198097in}}{\pgfqpoint{1.204631in}{2.192273in}}%
\pgfpathcurveto{\pgfqpoint{1.210455in}{2.186449in}}{\pgfqpoint{1.218355in}{2.183177in}}{\pgfqpoint{1.226591in}{2.183177in}}%
\pgfpathclose%
\pgfusepath{stroke,fill}%
\end{pgfscope}%
\begin{pgfscope}%
\pgfpathrectangle{\pgfqpoint{0.100000in}{0.212622in}}{\pgfqpoint{3.696000in}{3.696000in}}%
\pgfusepath{clip}%
\pgfsetbuttcap%
\pgfsetroundjoin%
\definecolor{currentfill}{rgb}{0.121569,0.466667,0.705882}%
\pgfsetfillcolor{currentfill}%
\pgfsetfillopacity{0.510386}%
\pgfsetlinewidth{1.003750pt}%
\definecolor{currentstroke}{rgb}{0.121569,0.466667,0.705882}%
\pgfsetstrokecolor{currentstroke}%
\pgfsetstrokeopacity{0.510386}%
\pgfsetdash{}{0pt}%
\pgfpathmoveto{\pgfqpoint{2.044060in}{2.446159in}}%
\pgfpathcurveto{\pgfqpoint{2.052296in}{2.446159in}}{\pgfqpoint{2.060196in}{2.449431in}}{\pgfqpoint{2.066020in}{2.455255in}}%
\pgfpathcurveto{\pgfqpoint{2.071844in}{2.461079in}}{\pgfqpoint{2.075116in}{2.468979in}}{\pgfqpoint{2.075116in}{2.477216in}}%
\pgfpathcurveto{\pgfqpoint{2.075116in}{2.485452in}}{\pgfqpoint{2.071844in}{2.493352in}}{\pgfqpoint{2.066020in}{2.499176in}}%
\pgfpathcurveto{\pgfqpoint{2.060196in}{2.505000in}}{\pgfqpoint{2.052296in}{2.508272in}}{\pgfqpoint{2.044060in}{2.508272in}}%
\pgfpathcurveto{\pgfqpoint{2.035824in}{2.508272in}}{\pgfqpoint{2.027924in}{2.505000in}}{\pgfqpoint{2.022100in}{2.499176in}}%
\pgfpathcurveto{\pgfqpoint{2.016276in}{2.493352in}}{\pgfqpoint{2.013003in}{2.485452in}}{\pgfqpoint{2.013003in}{2.477216in}}%
\pgfpathcurveto{\pgfqpoint{2.013003in}{2.468979in}}{\pgfqpoint{2.016276in}{2.461079in}}{\pgfqpoint{2.022100in}{2.455255in}}%
\pgfpathcurveto{\pgfqpoint{2.027924in}{2.449431in}}{\pgfqpoint{2.035824in}{2.446159in}}{\pgfqpoint{2.044060in}{2.446159in}}%
\pgfpathclose%
\pgfusepath{stroke,fill}%
\end{pgfscope}%
\begin{pgfscope}%
\pgfpathrectangle{\pgfqpoint{0.100000in}{0.212622in}}{\pgfqpoint{3.696000in}{3.696000in}}%
\pgfusepath{clip}%
\pgfsetbuttcap%
\pgfsetroundjoin%
\definecolor{currentfill}{rgb}{0.121569,0.466667,0.705882}%
\pgfsetfillcolor{currentfill}%
\pgfsetfillopacity{0.511235}%
\pgfsetlinewidth{1.003750pt}%
\definecolor{currentstroke}{rgb}{0.121569,0.466667,0.705882}%
\pgfsetstrokecolor{currentstroke}%
\pgfsetstrokeopacity{0.511235}%
\pgfsetdash{}{0pt}%
\pgfpathmoveto{\pgfqpoint{1.221684in}{2.175071in}}%
\pgfpathcurveto{\pgfqpoint{1.229920in}{2.175071in}}{\pgfqpoint{1.237820in}{2.178343in}}{\pgfqpoint{1.243644in}{2.184167in}}%
\pgfpathcurveto{\pgfqpoint{1.249468in}{2.189991in}}{\pgfqpoint{1.252740in}{2.197891in}}{\pgfqpoint{1.252740in}{2.206127in}}%
\pgfpathcurveto{\pgfqpoint{1.252740in}{2.214364in}}{\pgfqpoint{1.249468in}{2.222264in}}{\pgfqpoint{1.243644in}{2.228088in}}%
\pgfpathcurveto{\pgfqpoint{1.237820in}{2.233911in}}{\pgfqpoint{1.229920in}{2.237184in}}{\pgfqpoint{1.221684in}{2.237184in}}%
\pgfpathcurveto{\pgfqpoint{1.213447in}{2.237184in}}{\pgfqpoint{1.205547in}{2.233911in}}{\pgfqpoint{1.199723in}{2.228088in}}%
\pgfpathcurveto{\pgfqpoint{1.193900in}{2.222264in}}{\pgfqpoint{1.190627in}{2.214364in}}{\pgfqpoint{1.190627in}{2.206127in}}%
\pgfpathcurveto{\pgfqpoint{1.190627in}{2.197891in}}{\pgfqpoint{1.193900in}{2.189991in}}{\pgfqpoint{1.199723in}{2.184167in}}%
\pgfpathcurveto{\pgfqpoint{1.205547in}{2.178343in}}{\pgfqpoint{1.213447in}{2.175071in}}{\pgfqpoint{1.221684in}{2.175071in}}%
\pgfpathclose%
\pgfusepath{stroke,fill}%
\end{pgfscope}%
\begin{pgfscope}%
\pgfpathrectangle{\pgfqpoint{0.100000in}{0.212622in}}{\pgfqpoint{3.696000in}{3.696000in}}%
\pgfusepath{clip}%
\pgfsetbuttcap%
\pgfsetroundjoin%
\definecolor{currentfill}{rgb}{0.121569,0.466667,0.705882}%
\pgfsetfillcolor{currentfill}%
\pgfsetfillopacity{0.512194}%
\pgfsetlinewidth{1.003750pt}%
\definecolor{currentstroke}{rgb}{0.121569,0.466667,0.705882}%
\pgfsetstrokecolor{currentstroke}%
\pgfsetstrokeopacity{0.512194}%
\pgfsetdash{}{0pt}%
\pgfpathmoveto{\pgfqpoint{2.045079in}{2.436501in}}%
\pgfpathcurveto{\pgfqpoint{2.053315in}{2.436501in}}{\pgfqpoint{2.061215in}{2.439774in}}{\pgfqpoint{2.067039in}{2.445597in}}%
\pgfpathcurveto{\pgfqpoint{2.072863in}{2.451421in}}{\pgfqpoint{2.076135in}{2.459321in}}{\pgfqpoint{2.076135in}{2.467558in}}%
\pgfpathcurveto{\pgfqpoint{2.076135in}{2.475794in}}{\pgfqpoint{2.072863in}{2.483694in}}{\pgfqpoint{2.067039in}{2.489518in}}%
\pgfpathcurveto{\pgfqpoint{2.061215in}{2.495342in}}{\pgfqpoint{2.053315in}{2.498614in}}{\pgfqpoint{2.045079in}{2.498614in}}%
\pgfpathcurveto{\pgfqpoint{2.036843in}{2.498614in}}{\pgfqpoint{2.028943in}{2.495342in}}{\pgfqpoint{2.023119in}{2.489518in}}%
\pgfpathcurveto{\pgfqpoint{2.017295in}{2.483694in}}{\pgfqpoint{2.014022in}{2.475794in}}{\pgfqpoint{2.014022in}{2.467558in}}%
\pgfpathcurveto{\pgfqpoint{2.014022in}{2.459321in}}{\pgfqpoint{2.017295in}{2.451421in}}{\pgfqpoint{2.023119in}{2.445597in}}%
\pgfpathcurveto{\pgfqpoint{2.028943in}{2.439774in}}{\pgfqpoint{2.036843in}{2.436501in}}{\pgfqpoint{2.045079in}{2.436501in}}%
\pgfpathclose%
\pgfusepath{stroke,fill}%
\end{pgfscope}%
\begin{pgfscope}%
\pgfpathrectangle{\pgfqpoint{0.100000in}{0.212622in}}{\pgfqpoint{3.696000in}{3.696000in}}%
\pgfusepath{clip}%
\pgfsetbuttcap%
\pgfsetroundjoin%
\definecolor{currentfill}{rgb}{0.121569,0.466667,0.705882}%
\pgfsetfillcolor{currentfill}%
\pgfsetfillopacity{0.512651}%
\pgfsetlinewidth{1.003750pt}%
\definecolor{currentstroke}{rgb}{0.121569,0.466667,0.705882}%
\pgfsetstrokecolor{currentstroke}%
\pgfsetstrokeopacity{0.512651}%
\pgfsetdash{}{0pt}%
\pgfpathmoveto{\pgfqpoint{1.218162in}{2.167202in}}%
\pgfpathcurveto{\pgfqpoint{1.226399in}{2.167202in}}{\pgfqpoint{1.234299in}{2.170474in}}{\pgfqpoint{1.240123in}{2.176298in}}%
\pgfpathcurveto{\pgfqpoint{1.245947in}{2.182122in}}{\pgfqpoint{1.249219in}{2.190022in}}{\pgfqpoint{1.249219in}{2.198259in}}%
\pgfpathcurveto{\pgfqpoint{1.249219in}{2.206495in}}{\pgfqpoint{1.245947in}{2.214395in}}{\pgfqpoint{1.240123in}{2.220219in}}%
\pgfpathcurveto{\pgfqpoint{1.234299in}{2.226043in}}{\pgfqpoint{1.226399in}{2.229315in}}{\pgfqpoint{1.218162in}{2.229315in}}%
\pgfpathcurveto{\pgfqpoint{1.209926in}{2.229315in}}{\pgfqpoint{1.202026in}{2.226043in}}{\pgfqpoint{1.196202in}{2.220219in}}%
\pgfpathcurveto{\pgfqpoint{1.190378in}{2.214395in}}{\pgfqpoint{1.187106in}{2.206495in}}{\pgfqpoint{1.187106in}{2.198259in}}%
\pgfpathcurveto{\pgfqpoint{1.187106in}{2.190022in}}{\pgfqpoint{1.190378in}{2.182122in}}{\pgfqpoint{1.196202in}{2.176298in}}%
\pgfpathcurveto{\pgfqpoint{1.202026in}{2.170474in}}{\pgfqpoint{1.209926in}{2.167202in}}{\pgfqpoint{1.218162in}{2.167202in}}%
\pgfpathclose%
\pgfusepath{stroke,fill}%
\end{pgfscope}%
\begin{pgfscope}%
\pgfpathrectangle{\pgfqpoint{0.100000in}{0.212622in}}{\pgfqpoint{3.696000in}{3.696000in}}%
\pgfusepath{clip}%
\pgfsetbuttcap%
\pgfsetroundjoin%
\definecolor{currentfill}{rgb}{0.121569,0.466667,0.705882}%
\pgfsetfillcolor{currentfill}%
\pgfsetfillopacity{0.513903}%
\pgfsetlinewidth{1.003750pt}%
\definecolor{currentstroke}{rgb}{0.121569,0.466667,0.705882}%
\pgfsetstrokecolor{currentstroke}%
\pgfsetstrokeopacity{0.513903}%
\pgfsetdash{}{0pt}%
\pgfpathmoveto{\pgfqpoint{1.214291in}{2.161049in}}%
\pgfpathcurveto{\pgfqpoint{1.222527in}{2.161049in}}{\pgfqpoint{1.230427in}{2.164321in}}{\pgfqpoint{1.236251in}{2.170145in}}%
\pgfpathcurveto{\pgfqpoint{1.242075in}{2.175969in}}{\pgfqpoint{1.245348in}{2.183869in}}{\pgfqpoint{1.245348in}{2.192105in}}%
\pgfpathcurveto{\pgfqpoint{1.245348in}{2.200341in}}{\pgfqpoint{1.242075in}{2.208241in}}{\pgfqpoint{1.236251in}{2.214065in}}%
\pgfpathcurveto{\pgfqpoint{1.230427in}{2.219889in}}{\pgfqpoint{1.222527in}{2.223162in}}{\pgfqpoint{1.214291in}{2.223162in}}%
\pgfpathcurveto{\pgfqpoint{1.206055in}{2.223162in}}{\pgfqpoint{1.198155in}{2.219889in}}{\pgfqpoint{1.192331in}{2.214065in}}%
\pgfpathcurveto{\pgfqpoint{1.186507in}{2.208241in}}{\pgfqpoint{1.183235in}{2.200341in}}{\pgfqpoint{1.183235in}{2.192105in}}%
\pgfpathcurveto{\pgfqpoint{1.183235in}{2.183869in}}{\pgfqpoint{1.186507in}{2.175969in}}{\pgfqpoint{1.192331in}{2.170145in}}%
\pgfpathcurveto{\pgfqpoint{1.198155in}{2.164321in}}{\pgfqpoint{1.206055in}{2.161049in}}{\pgfqpoint{1.214291in}{2.161049in}}%
\pgfpathclose%
\pgfusepath{stroke,fill}%
\end{pgfscope}%
\begin{pgfscope}%
\pgfpathrectangle{\pgfqpoint{0.100000in}{0.212622in}}{\pgfqpoint{3.696000in}{3.696000in}}%
\pgfusepath{clip}%
\pgfsetbuttcap%
\pgfsetroundjoin%
\definecolor{currentfill}{rgb}{0.121569,0.466667,0.705882}%
\pgfsetfillcolor{currentfill}%
\pgfsetfillopacity{0.514426}%
\pgfsetlinewidth{1.003750pt}%
\definecolor{currentstroke}{rgb}{0.121569,0.466667,0.705882}%
\pgfsetstrokecolor{currentstroke}%
\pgfsetstrokeopacity{0.514426}%
\pgfsetdash{}{0pt}%
\pgfpathmoveto{\pgfqpoint{2.045966in}{2.427063in}}%
\pgfpathcurveto{\pgfqpoint{2.054202in}{2.427063in}}{\pgfqpoint{2.062102in}{2.430335in}}{\pgfqpoint{2.067926in}{2.436159in}}%
\pgfpathcurveto{\pgfqpoint{2.073750in}{2.441983in}}{\pgfqpoint{2.077023in}{2.449883in}}{\pgfqpoint{2.077023in}{2.458120in}}%
\pgfpathcurveto{\pgfqpoint{2.077023in}{2.466356in}}{\pgfqpoint{2.073750in}{2.474256in}}{\pgfqpoint{2.067926in}{2.480080in}}%
\pgfpathcurveto{\pgfqpoint{2.062102in}{2.485904in}}{\pgfqpoint{2.054202in}{2.489176in}}{\pgfqpoint{2.045966in}{2.489176in}}%
\pgfpathcurveto{\pgfqpoint{2.037730in}{2.489176in}}{\pgfqpoint{2.029830in}{2.485904in}}{\pgfqpoint{2.024006in}{2.480080in}}%
\pgfpathcurveto{\pgfqpoint{2.018182in}{2.474256in}}{\pgfqpoint{2.014910in}{2.466356in}}{\pgfqpoint{2.014910in}{2.458120in}}%
\pgfpathcurveto{\pgfqpoint{2.014910in}{2.449883in}}{\pgfqpoint{2.018182in}{2.441983in}}{\pgfqpoint{2.024006in}{2.436159in}}%
\pgfpathcurveto{\pgfqpoint{2.029830in}{2.430335in}}{\pgfqpoint{2.037730in}{2.427063in}}{\pgfqpoint{2.045966in}{2.427063in}}%
\pgfpathclose%
\pgfusepath{stroke,fill}%
\end{pgfscope}%
\begin{pgfscope}%
\pgfpathrectangle{\pgfqpoint{0.100000in}{0.212622in}}{\pgfqpoint{3.696000in}{3.696000in}}%
\pgfusepath{clip}%
\pgfsetbuttcap%
\pgfsetroundjoin%
\definecolor{currentfill}{rgb}{0.121569,0.466667,0.705882}%
\pgfsetfillcolor{currentfill}%
\pgfsetfillopacity{0.514914}%
\pgfsetlinewidth{1.003750pt}%
\definecolor{currentstroke}{rgb}{0.121569,0.466667,0.705882}%
\pgfsetstrokecolor{currentstroke}%
\pgfsetstrokeopacity{0.514914}%
\pgfsetdash{}{0pt}%
\pgfpathmoveto{\pgfqpoint{1.211561in}{2.155933in}}%
\pgfpathcurveto{\pgfqpoint{1.219797in}{2.155933in}}{\pgfqpoint{1.227697in}{2.159205in}}{\pgfqpoint{1.233521in}{2.165029in}}%
\pgfpathcurveto{\pgfqpoint{1.239345in}{2.170853in}}{\pgfqpoint{1.242618in}{2.178753in}}{\pgfqpoint{1.242618in}{2.186989in}}%
\pgfpathcurveto{\pgfqpoint{1.242618in}{2.195225in}}{\pgfqpoint{1.239345in}{2.203125in}}{\pgfqpoint{1.233521in}{2.208949in}}%
\pgfpathcurveto{\pgfqpoint{1.227697in}{2.214773in}}{\pgfqpoint{1.219797in}{2.218046in}}{\pgfqpoint{1.211561in}{2.218046in}}%
\pgfpathcurveto{\pgfqpoint{1.203325in}{2.218046in}}{\pgfqpoint{1.195425in}{2.214773in}}{\pgfqpoint{1.189601in}{2.208949in}}%
\pgfpathcurveto{\pgfqpoint{1.183777in}{2.203125in}}{\pgfqpoint{1.180505in}{2.195225in}}{\pgfqpoint{1.180505in}{2.186989in}}%
\pgfpathcurveto{\pgfqpoint{1.180505in}{2.178753in}}{\pgfqpoint{1.183777in}{2.170853in}}{\pgfqpoint{1.189601in}{2.165029in}}%
\pgfpathcurveto{\pgfqpoint{1.195425in}{2.159205in}}{\pgfqpoint{1.203325in}{2.155933in}}{\pgfqpoint{1.211561in}{2.155933in}}%
\pgfpathclose%
\pgfusepath{stroke,fill}%
\end{pgfscope}%
\begin{pgfscope}%
\pgfpathrectangle{\pgfqpoint{0.100000in}{0.212622in}}{\pgfqpoint{3.696000in}{3.696000in}}%
\pgfusepath{clip}%
\pgfsetbuttcap%
\pgfsetroundjoin%
\definecolor{currentfill}{rgb}{0.121569,0.466667,0.705882}%
\pgfsetfillcolor{currentfill}%
\pgfsetfillopacity{0.515213}%
\pgfsetlinewidth{1.003750pt}%
\definecolor{currentstroke}{rgb}{0.121569,0.466667,0.705882}%
\pgfsetstrokecolor{currentstroke}%
\pgfsetstrokeopacity{0.515213}%
\pgfsetdash{}{0pt}%
\pgfpathmoveto{\pgfqpoint{1.210594in}{2.154363in}}%
\pgfpathcurveto{\pgfqpoint{1.218830in}{2.154363in}}{\pgfqpoint{1.226730in}{2.157635in}}{\pgfqpoint{1.232554in}{2.163459in}}%
\pgfpathcurveto{\pgfqpoint{1.238378in}{2.169283in}}{\pgfqpoint{1.241651in}{2.177183in}}{\pgfqpoint{1.241651in}{2.185420in}}%
\pgfpathcurveto{\pgfqpoint{1.241651in}{2.193656in}}{\pgfqpoint{1.238378in}{2.201556in}}{\pgfqpoint{1.232554in}{2.207380in}}%
\pgfpathcurveto{\pgfqpoint{1.226730in}{2.213204in}}{\pgfqpoint{1.218830in}{2.216476in}}{\pgfqpoint{1.210594in}{2.216476in}}%
\pgfpathcurveto{\pgfqpoint{1.202358in}{2.216476in}}{\pgfqpoint{1.194458in}{2.213204in}}{\pgfqpoint{1.188634in}{2.207380in}}%
\pgfpathcurveto{\pgfqpoint{1.182810in}{2.201556in}}{\pgfqpoint{1.179538in}{2.193656in}}{\pgfqpoint{1.179538in}{2.185420in}}%
\pgfpathcurveto{\pgfqpoint{1.179538in}{2.177183in}}{\pgfqpoint{1.182810in}{2.169283in}}{\pgfqpoint{1.188634in}{2.163459in}}%
\pgfpathcurveto{\pgfqpoint{1.194458in}{2.157635in}}{\pgfqpoint{1.202358in}{2.154363in}}{\pgfqpoint{1.210594in}{2.154363in}}%
\pgfpathclose%
\pgfusepath{stroke,fill}%
\end{pgfscope}%
\begin{pgfscope}%
\pgfpathrectangle{\pgfqpoint{0.100000in}{0.212622in}}{\pgfqpoint{3.696000in}{3.696000in}}%
\pgfusepath{clip}%
\pgfsetbuttcap%
\pgfsetroundjoin%
\definecolor{currentfill}{rgb}{0.121569,0.466667,0.705882}%
\pgfsetfillcolor{currentfill}%
\pgfsetfillopacity{0.515343}%
\pgfsetlinewidth{1.003750pt}%
\definecolor{currentstroke}{rgb}{0.121569,0.466667,0.705882}%
\pgfsetstrokecolor{currentstroke}%
\pgfsetstrokeopacity{0.515343}%
\pgfsetdash{}{0pt}%
\pgfpathmoveto{\pgfqpoint{1.210322in}{2.153679in}}%
\pgfpathcurveto{\pgfqpoint{1.218559in}{2.153679in}}{\pgfqpoint{1.226459in}{2.156951in}}{\pgfqpoint{1.232283in}{2.162775in}}%
\pgfpathcurveto{\pgfqpoint{1.238107in}{2.168599in}}{\pgfqpoint{1.241379in}{2.176499in}}{\pgfqpoint{1.241379in}{2.184735in}}%
\pgfpathcurveto{\pgfqpoint{1.241379in}{2.192971in}}{\pgfqpoint{1.238107in}{2.200871in}}{\pgfqpoint{1.232283in}{2.206695in}}%
\pgfpathcurveto{\pgfqpoint{1.226459in}{2.212519in}}{\pgfqpoint{1.218559in}{2.215792in}}{\pgfqpoint{1.210322in}{2.215792in}}%
\pgfpathcurveto{\pgfqpoint{1.202086in}{2.215792in}}{\pgfqpoint{1.194186in}{2.212519in}}{\pgfqpoint{1.188362in}{2.206695in}}%
\pgfpathcurveto{\pgfqpoint{1.182538in}{2.200871in}}{\pgfqpoint{1.179266in}{2.192971in}}{\pgfqpoint{1.179266in}{2.184735in}}%
\pgfpathcurveto{\pgfqpoint{1.179266in}{2.176499in}}{\pgfqpoint{1.182538in}{2.168599in}}{\pgfqpoint{1.188362in}{2.162775in}}%
\pgfpathcurveto{\pgfqpoint{1.194186in}{2.156951in}}{\pgfqpoint{1.202086in}{2.153679in}}{\pgfqpoint{1.210322in}{2.153679in}}%
\pgfpathclose%
\pgfusepath{stroke,fill}%
\end{pgfscope}%
\begin{pgfscope}%
\pgfpathrectangle{\pgfqpoint{0.100000in}{0.212622in}}{\pgfqpoint{3.696000in}{3.696000in}}%
\pgfusepath{clip}%
\pgfsetbuttcap%
\pgfsetroundjoin%
\definecolor{currentfill}{rgb}{0.121569,0.466667,0.705882}%
\pgfsetfillcolor{currentfill}%
\pgfsetfillopacity{0.515502}%
\pgfsetlinewidth{1.003750pt}%
\definecolor{currentstroke}{rgb}{0.121569,0.466667,0.705882}%
\pgfsetstrokecolor{currentstroke}%
\pgfsetstrokeopacity{0.515502}%
\pgfsetdash{}{0pt}%
\pgfpathmoveto{\pgfqpoint{2.046920in}{2.421678in}}%
\pgfpathcurveto{\pgfqpoint{2.055156in}{2.421678in}}{\pgfqpoint{2.063056in}{2.424951in}}{\pgfqpoint{2.068880in}{2.430774in}}%
\pgfpathcurveto{\pgfqpoint{2.074704in}{2.436598in}}{\pgfqpoint{2.077976in}{2.444498in}}{\pgfqpoint{2.077976in}{2.452735in}}%
\pgfpathcurveto{\pgfqpoint{2.077976in}{2.460971in}}{\pgfqpoint{2.074704in}{2.468871in}}{\pgfqpoint{2.068880in}{2.474695in}}%
\pgfpathcurveto{\pgfqpoint{2.063056in}{2.480519in}}{\pgfqpoint{2.055156in}{2.483791in}}{\pgfqpoint{2.046920in}{2.483791in}}%
\pgfpathcurveto{\pgfqpoint{2.038684in}{2.483791in}}{\pgfqpoint{2.030784in}{2.480519in}}{\pgfqpoint{2.024960in}{2.474695in}}%
\pgfpathcurveto{\pgfqpoint{2.019136in}{2.468871in}}{\pgfqpoint{2.015863in}{2.460971in}}{\pgfqpoint{2.015863in}{2.452735in}}%
\pgfpathcurveto{\pgfqpoint{2.015863in}{2.444498in}}{\pgfqpoint{2.019136in}{2.436598in}}{\pgfqpoint{2.024960in}{2.430774in}}%
\pgfpathcurveto{\pgfqpoint{2.030784in}{2.424951in}}{\pgfqpoint{2.038684in}{2.421678in}}{\pgfqpoint{2.046920in}{2.421678in}}%
\pgfpathclose%
\pgfusepath{stroke,fill}%
\end{pgfscope}%
\begin{pgfscope}%
\pgfpathrectangle{\pgfqpoint{0.100000in}{0.212622in}}{\pgfqpoint{3.696000in}{3.696000in}}%
\pgfusepath{clip}%
\pgfsetbuttcap%
\pgfsetroundjoin%
\definecolor{currentfill}{rgb}{0.121569,0.466667,0.705882}%
\pgfsetfillcolor{currentfill}%
\pgfsetfillopacity{0.515579}%
\pgfsetlinewidth{1.003750pt}%
\definecolor{currentstroke}{rgb}{0.121569,0.466667,0.705882}%
\pgfsetstrokecolor{currentstroke}%
\pgfsetstrokeopacity{0.515579}%
\pgfsetdash{}{0pt}%
\pgfpathmoveto{\pgfqpoint{1.209645in}{2.152625in}}%
\pgfpathcurveto{\pgfqpoint{1.217881in}{2.152625in}}{\pgfqpoint{1.225781in}{2.155897in}}{\pgfqpoint{1.231605in}{2.161721in}}%
\pgfpathcurveto{\pgfqpoint{1.237429in}{2.167545in}}{\pgfqpoint{1.240701in}{2.175445in}}{\pgfqpoint{1.240701in}{2.183682in}}%
\pgfpathcurveto{\pgfqpoint{1.240701in}{2.191918in}}{\pgfqpoint{1.237429in}{2.199818in}}{\pgfqpoint{1.231605in}{2.205642in}}%
\pgfpathcurveto{\pgfqpoint{1.225781in}{2.211466in}}{\pgfqpoint{1.217881in}{2.214738in}}{\pgfqpoint{1.209645in}{2.214738in}}%
\pgfpathcurveto{\pgfqpoint{1.201409in}{2.214738in}}{\pgfqpoint{1.193509in}{2.211466in}}{\pgfqpoint{1.187685in}{2.205642in}}%
\pgfpathcurveto{\pgfqpoint{1.181861in}{2.199818in}}{\pgfqpoint{1.178588in}{2.191918in}}{\pgfqpoint{1.178588in}{2.183682in}}%
\pgfpathcurveto{\pgfqpoint{1.178588in}{2.175445in}}{\pgfqpoint{1.181861in}{2.167545in}}{\pgfqpoint{1.187685in}{2.161721in}}%
\pgfpathcurveto{\pgfqpoint{1.193509in}{2.155897in}}{\pgfqpoint{1.201409in}{2.152625in}}{\pgfqpoint{1.209645in}{2.152625in}}%
\pgfpathclose%
\pgfusepath{stroke,fill}%
\end{pgfscope}%
\begin{pgfscope}%
\pgfpathrectangle{\pgfqpoint{0.100000in}{0.212622in}}{\pgfqpoint{3.696000in}{3.696000in}}%
\pgfusepath{clip}%
\pgfsetbuttcap%
\pgfsetroundjoin%
\definecolor{currentfill}{rgb}{0.121569,0.466667,0.705882}%
\pgfsetfillcolor{currentfill}%
\pgfsetfillopacity{0.516033}%
\pgfsetlinewidth{1.003750pt}%
\definecolor{currentstroke}{rgb}{0.121569,0.466667,0.705882}%
\pgfsetstrokecolor{currentstroke}%
\pgfsetstrokeopacity{0.516033}%
\pgfsetdash{}{0pt}%
\pgfpathmoveto{\pgfqpoint{1.208522in}{2.150659in}}%
\pgfpathcurveto{\pgfqpoint{1.216759in}{2.150659in}}{\pgfqpoint{1.224659in}{2.153932in}}{\pgfqpoint{1.230483in}{2.159756in}}%
\pgfpathcurveto{\pgfqpoint{1.236307in}{2.165580in}}{\pgfqpoint{1.239579in}{2.173480in}}{\pgfqpoint{1.239579in}{2.181716in}}%
\pgfpathcurveto{\pgfqpoint{1.239579in}{2.189952in}}{\pgfqpoint{1.236307in}{2.197852in}}{\pgfqpoint{1.230483in}{2.203676in}}%
\pgfpathcurveto{\pgfqpoint{1.224659in}{2.209500in}}{\pgfqpoint{1.216759in}{2.212772in}}{\pgfqpoint{1.208522in}{2.212772in}}%
\pgfpathcurveto{\pgfqpoint{1.200286in}{2.212772in}}{\pgfqpoint{1.192386in}{2.209500in}}{\pgfqpoint{1.186562in}{2.203676in}}%
\pgfpathcurveto{\pgfqpoint{1.180738in}{2.197852in}}{\pgfqpoint{1.177466in}{2.189952in}}{\pgfqpoint{1.177466in}{2.181716in}}%
\pgfpathcurveto{\pgfqpoint{1.177466in}{2.173480in}}{\pgfqpoint{1.180738in}{2.165580in}}{\pgfqpoint{1.186562in}{2.159756in}}%
\pgfpathcurveto{\pgfqpoint{1.192386in}{2.153932in}}{\pgfqpoint{1.200286in}{2.150659in}}{\pgfqpoint{1.208522in}{2.150659in}}%
\pgfpathclose%
\pgfusepath{stroke,fill}%
\end{pgfscope}%
\begin{pgfscope}%
\pgfpathrectangle{\pgfqpoint{0.100000in}{0.212622in}}{\pgfqpoint{3.696000in}{3.696000in}}%
\pgfusepath{clip}%
\pgfsetbuttcap%
\pgfsetroundjoin%
\definecolor{currentfill}{rgb}{0.121569,0.466667,0.705882}%
\pgfsetfillcolor{currentfill}%
\pgfsetfillopacity{0.516390}%
\pgfsetlinewidth{1.003750pt}%
\definecolor{currentstroke}{rgb}{0.121569,0.466667,0.705882}%
\pgfsetstrokecolor{currentstroke}%
\pgfsetstrokeopacity{0.516390}%
\pgfsetdash{}{0pt}%
\pgfpathmoveto{\pgfqpoint{1.207558in}{2.149043in}}%
\pgfpathcurveto{\pgfqpoint{1.215794in}{2.149043in}}{\pgfqpoint{1.223694in}{2.152315in}}{\pgfqpoint{1.229518in}{2.158139in}}%
\pgfpathcurveto{\pgfqpoint{1.235342in}{2.163963in}}{\pgfqpoint{1.238614in}{2.171863in}}{\pgfqpoint{1.238614in}{2.180100in}}%
\pgfpathcurveto{\pgfqpoint{1.238614in}{2.188336in}}{\pgfqpoint{1.235342in}{2.196236in}}{\pgfqpoint{1.229518in}{2.202060in}}%
\pgfpathcurveto{\pgfqpoint{1.223694in}{2.207884in}}{\pgfqpoint{1.215794in}{2.211156in}}{\pgfqpoint{1.207558in}{2.211156in}}%
\pgfpathcurveto{\pgfqpoint{1.199321in}{2.211156in}}{\pgfqpoint{1.191421in}{2.207884in}}{\pgfqpoint{1.185597in}{2.202060in}}%
\pgfpathcurveto{\pgfqpoint{1.179774in}{2.196236in}}{\pgfqpoint{1.176501in}{2.188336in}}{\pgfqpoint{1.176501in}{2.180100in}}%
\pgfpathcurveto{\pgfqpoint{1.176501in}{2.171863in}}{\pgfqpoint{1.179774in}{2.163963in}}{\pgfqpoint{1.185597in}{2.158139in}}%
\pgfpathcurveto{\pgfqpoint{1.191421in}{2.152315in}}{\pgfqpoint{1.199321in}{2.149043in}}{\pgfqpoint{1.207558in}{2.149043in}}%
\pgfpathclose%
\pgfusepath{stroke,fill}%
\end{pgfscope}%
\begin{pgfscope}%
\pgfpathrectangle{\pgfqpoint{0.100000in}{0.212622in}}{\pgfqpoint{3.696000in}{3.696000in}}%
\pgfusepath{clip}%
\pgfsetbuttcap%
\pgfsetroundjoin%
\definecolor{currentfill}{rgb}{0.121569,0.466667,0.705882}%
\pgfsetfillcolor{currentfill}%
\pgfsetfillopacity{0.516916}%
\pgfsetlinewidth{1.003750pt}%
\definecolor{currentstroke}{rgb}{0.121569,0.466667,0.705882}%
\pgfsetstrokecolor{currentstroke}%
\pgfsetstrokeopacity{0.516916}%
\pgfsetdash{}{0pt}%
\pgfpathmoveto{\pgfqpoint{2.047774in}{2.414751in}}%
\pgfpathcurveto{\pgfqpoint{2.056010in}{2.414751in}}{\pgfqpoint{2.063910in}{2.418024in}}{\pgfqpoint{2.069734in}{2.423848in}}%
\pgfpathcurveto{\pgfqpoint{2.075558in}{2.429672in}}{\pgfqpoint{2.078830in}{2.437572in}}{\pgfqpoint{2.078830in}{2.445808in}}%
\pgfpathcurveto{\pgfqpoint{2.078830in}{2.454044in}}{\pgfqpoint{2.075558in}{2.461944in}}{\pgfqpoint{2.069734in}{2.467768in}}%
\pgfpathcurveto{\pgfqpoint{2.063910in}{2.473592in}}{\pgfqpoint{2.056010in}{2.476864in}}{\pgfqpoint{2.047774in}{2.476864in}}%
\pgfpathcurveto{\pgfqpoint{2.039537in}{2.476864in}}{\pgfqpoint{2.031637in}{2.473592in}}{\pgfqpoint{2.025813in}{2.467768in}}%
\pgfpathcurveto{\pgfqpoint{2.019989in}{2.461944in}}{\pgfqpoint{2.016717in}{2.454044in}}{\pgfqpoint{2.016717in}{2.445808in}}%
\pgfpathcurveto{\pgfqpoint{2.016717in}{2.437572in}}{\pgfqpoint{2.019989in}{2.429672in}}{\pgfqpoint{2.025813in}{2.423848in}}%
\pgfpathcurveto{\pgfqpoint{2.031637in}{2.418024in}}{\pgfqpoint{2.039537in}{2.414751in}}{\pgfqpoint{2.047774in}{2.414751in}}%
\pgfpathclose%
\pgfusepath{stroke,fill}%
\end{pgfscope}%
\begin{pgfscope}%
\pgfpathrectangle{\pgfqpoint{0.100000in}{0.212622in}}{\pgfqpoint{3.696000in}{3.696000in}}%
\pgfusepath{clip}%
\pgfsetbuttcap%
\pgfsetroundjoin%
\definecolor{currentfill}{rgb}{0.121569,0.466667,0.705882}%
\pgfsetfillcolor{currentfill}%
\pgfsetfillopacity{0.517034}%
\pgfsetlinewidth{1.003750pt}%
\definecolor{currentstroke}{rgb}{0.121569,0.466667,0.705882}%
\pgfsetstrokecolor{currentstroke}%
\pgfsetstrokeopacity{0.517034}%
\pgfsetdash{}{0pt}%
\pgfpathmoveto{\pgfqpoint{1.205782in}{2.146109in}}%
\pgfpathcurveto{\pgfqpoint{1.214019in}{2.146109in}}{\pgfqpoint{1.221919in}{2.149381in}}{\pgfqpoint{1.227742in}{2.155205in}}%
\pgfpathcurveto{\pgfqpoint{1.233566in}{2.161029in}}{\pgfqpoint{1.236839in}{2.168929in}}{\pgfqpoint{1.236839in}{2.177166in}}%
\pgfpathcurveto{\pgfqpoint{1.236839in}{2.185402in}}{\pgfqpoint{1.233566in}{2.193302in}}{\pgfqpoint{1.227742in}{2.199126in}}%
\pgfpathcurveto{\pgfqpoint{1.221919in}{2.204950in}}{\pgfqpoint{1.214019in}{2.208222in}}{\pgfqpoint{1.205782in}{2.208222in}}%
\pgfpathcurveto{\pgfqpoint{1.197546in}{2.208222in}}{\pgfqpoint{1.189646in}{2.204950in}}{\pgfqpoint{1.183822in}{2.199126in}}%
\pgfpathcurveto{\pgfqpoint{1.177998in}{2.193302in}}{\pgfqpoint{1.174726in}{2.185402in}}{\pgfqpoint{1.174726in}{2.177166in}}%
\pgfpathcurveto{\pgfqpoint{1.174726in}{2.168929in}}{\pgfqpoint{1.177998in}{2.161029in}}{\pgfqpoint{1.183822in}{2.155205in}}%
\pgfpathcurveto{\pgfqpoint{1.189646in}{2.149381in}}{\pgfqpoint{1.197546in}{2.146109in}}{\pgfqpoint{1.205782in}{2.146109in}}%
\pgfpathclose%
\pgfusepath{stroke,fill}%
\end{pgfscope}%
\begin{pgfscope}%
\pgfpathrectangle{\pgfqpoint{0.100000in}{0.212622in}}{\pgfqpoint{3.696000in}{3.696000in}}%
\pgfusepath{clip}%
\pgfsetbuttcap%
\pgfsetroundjoin%
\definecolor{currentfill}{rgb}{0.121569,0.466667,0.705882}%
\pgfsetfillcolor{currentfill}%
\pgfsetfillopacity{0.517575}%
\pgfsetlinewidth{1.003750pt}%
\definecolor{currentstroke}{rgb}{0.121569,0.466667,0.705882}%
\pgfsetstrokecolor{currentstroke}%
\pgfsetstrokeopacity{0.517575}%
\pgfsetdash{}{0pt}%
\pgfpathmoveto{\pgfqpoint{1.204321in}{2.143519in}}%
\pgfpathcurveto{\pgfqpoint{1.212557in}{2.143519in}}{\pgfqpoint{1.220457in}{2.146791in}}{\pgfqpoint{1.226281in}{2.152615in}}%
\pgfpathcurveto{\pgfqpoint{1.232105in}{2.158439in}}{\pgfqpoint{1.235378in}{2.166339in}}{\pgfqpoint{1.235378in}{2.174576in}}%
\pgfpathcurveto{\pgfqpoint{1.235378in}{2.182812in}}{\pgfqpoint{1.232105in}{2.190712in}}{\pgfqpoint{1.226281in}{2.196536in}}%
\pgfpathcurveto{\pgfqpoint{1.220457in}{2.202360in}}{\pgfqpoint{1.212557in}{2.205632in}}{\pgfqpoint{1.204321in}{2.205632in}}%
\pgfpathcurveto{\pgfqpoint{1.196085in}{2.205632in}}{\pgfqpoint{1.188185in}{2.202360in}}{\pgfqpoint{1.182361in}{2.196536in}}%
\pgfpathcurveto{\pgfqpoint{1.176537in}{2.190712in}}{\pgfqpoint{1.173265in}{2.182812in}}{\pgfqpoint{1.173265in}{2.174576in}}%
\pgfpathcurveto{\pgfqpoint{1.173265in}{2.166339in}}{\pgfqpoint{1.176537in}{2.158439in}}{\pgfqpoint{1.182361in}{2.152615in}}%
\pgfpathcurveto{\pgfqpoint{1.188185in}{2.146791in}}{\pgfqpoint{1.196085in}{2.143519in}}{\pgfqpoint{1.204321in}{2.143519in}}%
\pgfpathclose%
\pgfusepath{stroke,fill}%
\end{pgfscope}%
\begin{pgfscope}%
\pgfpathrectangle{\pgfqpoint{0.100000in}{0.212622in}}{\pgfqpoint{3.696000in}{3.696000in}}%
\pgfusepath{clip}%
\pgfsetbuttcap%
\pgfsetroundjoin%
\definecolor{currentfill}{rgb}{0.121569,0.466667,0.705882}%
\pgfsetfillcolor{currentfill}%
\pgfsetfillopacity{0.518559}%
\pgfsetlinewidth{1.003750pt}%
\definecolor{currentstroke}{rgb}{0.121569,0.466667,0.705882}%
\pgfsetstrokecolor{currentstroke}%
\pgfsetstrokeopacity{0.518559}%
\pgfsetdash{}{0pt}%
\pgfpathmoveto{\pgfqpoint{1.201556in}{2.138954in}}%
\pgfpathcurveto{\pgfqpoint{1.209793in}{2.138954in}}{\pgfqpoint{1.217693in}{2.142226in}}{\pgfqpoint{1.223517in}{2.148050in}}%
\pgfpathcurveto{\pgfqpoint{1.229341in}{2.153874in}}{\pgfqpoint{1.232613in}{2.161774in}}{\pgfqpoint{1.232613in}{2.170010in}}%
\pgfpathcurveto{\pgfqpoint{1.232613in}{2.178246in}}{\pgfqpoint{1.229341in}{2.186147in}}{\pgfqpoint{1.223517in}{2.191970in}}%
\pgfpathcurveto{\pgfqpoint{1.217693in}{2.197794in}}{\pgfqpoint{1.209793in}{2.201067in}}{\pgfqpoint{1.201556in}{2.201067in}}%
\pgfpathcurveto{\pgfqpoint{1.193320in}{2.201067in}}{\pgfqpoint{1.185420in}{2.197794in}}{\pgfqpoint{1.179596in}{2.191970in}}%
\pgfpathcurveto{\pgfqpoint{1.173772in}{2.186147in}}{\pgfqpoint{1.170500in}{2.178246in}}{\pgfqpoint{1.170500in}{2.170010in}}%
\pgfpathcurveto{\pgfqpoint{1.170500in}{2.161774in}}{\pgfqpoint{1.173772in}{2.153874in}}{\pgfqpoint{1.179596in}{2.148050in}}%
\pgfpathcurveto{\pgfqpoint{1.185420in}{2.142226in}}{\pgfqpoint{1.193320in}{2.138954in}}{\pgfqpoint{1.201556in}{2.138954in}}%
\pgfpathclose%
\pgfusepath{stroke,fill}%
\end{pgfscope}%
\begin{pgfscope}%
\pgfpathrectangle{\pgfqpoint{0.100000in}{0.212622in}}{\pgfqpoint{3.696000in}{3.696000in}}%
\pgfusepath{clip}%
\pgfsetbuttcap%
\pgfsetroundjoin%
\definecolor{currentfill}{rgb}{0.121569,0.466667,0.705882}%
\pgfsetfillcolor{currentfill}%
\pgfsetfillopacity{0.518646}%
\pgfsetlinewidth{1.003750pt}%
\definecolor{currentstroke}{rgb}{0.121569,0.466667,0.705882}%
\pgfsetstrokecolor{currentstroke}%
\pgfsetstrokeopacity{0.518646}%
\pgfsetdash{}{0pt}%
\pgfpathmoveto{\pgfqpoint{2.048356in}{2.407794in}}%
\pgfpathcurveto{\pgfqpoint{2.056592in}{2.407794in}}{\pgfqpoint{2.064492in}{2.411066in}}{\pgfqpoint{2.070316in}{2.416890in}}%
\pgfpathcurveto{\pgfqpoint{2.076140in}{2.422714in}}{\pgfqpoint{2.079412in}{2.430614in}}{\pgfqpoint{2.079412in}{2.438851in}}%
\pgfpathcurveto{\pgfqpoint{2.079412in}{2.447087in}}{\pgfqpoint{2.076140in}{2.454987in}}{\pgfqpoint{2.070316in}{2.460811in}}%
\pgfpathcurveto{\pgfqpoint{2.064492in}{2.466635in}}{\pgfqpoint{2.056592in}{2.469907in}}{\pgfqpoint{2.048356in}{2.469907in}}%
\pgfpathcurveto{\pgfqpoint{2.040119in}{2.469907in}}{\pgfqpoint{2.032219in}{2.466635in}}{\pgfqpoint{2.026395in}{2.460811in}}%
\pgfpathcurveto{\pgfqpoint{2.020572in}{2.454987in}}{\pgfqpoint{2.017299in}{2.447087in}}{\pgfqpoint{2.017299in}{2.438851in}}%
\pgfpathcurveto{\pgfqpoint{2.017299in}{2.430614in}}{\pgfqpoint{2.020572in}{2.422714in}}{\pgfqpoint{2.026395in}{2.416890in}}%
\pgfpathcurveto{\pgfqpoint{2.032219in}{2.411066in}}{\pgfqpoint{2.040119in}{2.407794in}}{\pgfqpoint{2.048356in}{2.407794in}}%
\pgfpathclose%
\pgfusepath{stroke,fill}%
\end{pgfscope}%
\begin{pgfscope}%
\pgfpathrectangle{\pgfqpoint{0.100000in}{0.212622in}}{\pgfqpoint{3.696000in}{3.696000in}}%
\pgfusepath{clip}%
\pgfsetbuttcap%
\pgfsetroundjoin%
\definecolor{currentfill}{rgb}{0.121569,0.466667,0.705882}%
\pgfsetfillcolor{currentfill}%
\pgfsetfillopacity{0.519426}%
\pgfsetlinewidth{1.003750pt}%
\definecolor{currentstroke}{rgb}{0.121569,0.466667,0.705882}%
\pgfsetstrokecolor{currentstroke}%
\pgfsetstrokeopacity{0.519426}%
\pgfsetdash{}{0pt}%
\pgfpathmoveto{\pgfqpoint{1.199200in}{2.134733in}}%
\pgfpathcurveto{\pgfqpoint{1.207437in}{2.134733in}}{\pgfqpoint{1.215337in}{2.138006in}}{\pgfqpoint{1.221161in}{2.143830in}}%
\pgfpathcurveto{\pgfqpoint{1.226985in}{2.149653in}}{\pgfqpoint{1.230257in}{2.157554in}}{\pgfqpoint{1.230257in}{2.165790in}}%
\pgfpathcurveto{\pgfqpoint{1.230257in}{2.174026in}}{\pgfqpoint{1.226985in}{2.181926in}}{\pgfqpoint{1.221161in}{2.187750in}}%
\pgfpathcurveto{\pgfqpoint{1.215337in}{2.193574in}}{\pgfqpoint{1.207437in}{2.196846in}}{\pgfqpoint{1.199200in}{2.196846in}}%
\pgfpathcurveto{\pgfqpoint{1.190964in}{2.196846in}}{\pgfqpoint{1.183064in}{2.193574in}}{\pgfqpoint{1.177240in}{2.187750in}}%
\pgfpathcurveto{\pgfqpoint{1.171416in}{2.181926in}}{\pgfqpoint{1.168144in}{2.174026in}}{\pgfqpoint{1.168144in}{2.165790in}}%
\pgfpathcurveto{\pgfqpoint{1.168144in}{2.157554in}}{\pgfqpoint{1.171416in}{2.149653in}}{\pgfqpoint{1.177240in}{2.143830in}}%
\pgfpathcurveto{\pgfqpoint{1.183064in}{2.138006in}}{\pgfqpoint{1.190964in}{2.134733in}}{\pgfqpoint{1.199200in}{2.134733in}}%
\pgfpathclose%
\pgfusepath{stroke,fill}%
\end{pgfscope}%
\begin{pgfscope}%
\pgfpathrectangle{\pgfqpoint{0.100000in}{0.212622in}}{\pgfqpoint{3.696000in}{3.696000in}}%
\pgfusepath{clip}%
\pgfsetbuttcap%
\pgfsetroundjoin%
\definecolor{currentfill}{rgb}{0.121569,0.466667,0.705882}%
\pgfsetfillcolor{currentfill}%
\pgfsetfillopacity{0.520548}%
\pgfsetlinewidth{1.003750pt}%
\definecolor{currentstroke}{rgb}{0.121569,0.466667,0.705882}%
\pgfsetstrokecolor{currentstroke}%
\pgfsetstrokeopacity{0.520548}%
\pgfsetdash{}{0pt}%
\pgfpathmoveto{\pgfqpoint{2.049933in}{2.399161in}}%
\pgfpathcurveto{\pgfqpoint{2.058169in}{2.399161in}}{\pgfqpoint{2.066069in}{2.402433in}}{\pgfqpoint{2.071893in}{2.408257in}}%
\pgfpathcurveto{\pgfqpoint{2.077717in}{2.414081in}}{\pgfqpoint{2.080989in}{2.421981in}}{\pgfqpoint{2.080989in}{2.430217in}}%
\pgfpathcurveto{\pgfqpoint{2.080989in}{2.438453in}}{\pgfqpoint{2.077717in}{2.446353in}}{\pgfqpoint{2.071893in}{2.452177in}}%
\pgfpathcurveto{\pgfqpoint{2.066069in}{2.458001in}}{\pgfqpoint{2.058169in}{2.461274in}}{\pgfqpoint{2.049933in}{2.461274in}}%
\pgfpathcurveto{\pgfqpoint{2.041696in}{2.461274in}}{\pgfqpoint{2.033796in}{2.458001in}}{\pgfqpoint{2.027972in}{2.452177in}}%
\pgfpathcurveto{\pgfqpoint{2.022149in}{2.446353in}}{\pgfqpoint{2.018876in}{2.438453in}}{\pgfqpoint{2.018876in}{2.430217in}}%
\pgfpathcurveto{\pgfqpoint{2.018876in}{2.421981in}}{\pgfqpoint{2.022149in}{2.414081in}}{\pgfqpoint{2.027972in}{2.408257in}}%
\pgfpathcurveto{\pgfqpoint{2.033796in}{2.402433in}}{\pgfqpoint{2.041696in}{2.399161in}}{\pgfqpoint{2.049933in}{2.399161in}}%
\pgfpathclose%
\pgfusepath{stroke,fill}%
\end{pgfscope}%
\begin{pgfscope}%
\pgfpathrectangle{\pgfqpoint{0.100000in}{0.212622in}}{\pgfqpoint{3.696000in}{3.696000in}}%
\pgfusepath{clip}%
\pgfsetbuttcap%
\pgfsetroundjoin%
\definecolor{currentfill}{rgb}{0.121569,0.466667,0.705882}%
\pgfsetfillcolor{currentfill}%
\pgfsetfillopacity{0.520930}%
\pgfsetlinewidth{1.003750pt}%
\definecolor{currentstroke}{rgb}{0.121569,0.466667,0.705882}%
\pgfsetstrokecolor{currentstroke}%
\pgfsetstrokeopacity{0.520930}%
\pgfsetdash{}{0pt}%
\pgfpathmoveto{\pgfqpoint{1.194611in}{2.127198in}}%
\pgfpathcurveto{\pgfqpoint{1.202848in}{2.127198in}}{\pgfqpoint{1.210748in}{2.130471in}}{\pgfqpoint{1.216572in}{2.136295in}}%
\pgfpathcurveto{\pgfqpoint{1.222395in}{2.142119in}}{\pgfqpoint{1.225668in}{2.150019in}}{\pgfqpoint{1.225668in}{2.158255in}}%
\pgfpathcurveto{\pgfqpoint{1.225668in}{2.166491in}}{\pgfqpoint{1.222395in}{2.174391in}}{\pgfqpoint{1.216572in}{2.180215in}}%
\pgfpathcurveto{\pgfqpoint{1.210748in}{2.186039in}}{\pgfqpoint{1.202848in}{2.189311in}}{\pgfqpoint{1.194611in}{2.189311in}}%
\pgfpathcurveto{\pgfqpoint{1.186375in}{2.189311in}}{\pgfqpoint{1.178475in}{2.186039in}}{\pgfqpoint{1.172651in}{2.180215in}}%
\pgfpathcurveto{\pgfqpoint{1.166827in}{2.174391in}}{\pgfqpoint{1.163555in}{2.166491in}}{\pgfqpoint{1.163555in}{2.158255in}}%
\pgfpathcurveto{\pgfqpoint{1.163555in}{2.150019in}}{\pgfqpoint{1.166827in}{2.142119in}}{\pgfqpoint{1.172651in}{2.136295in}}%
\pgfpathcurveto{\pgfqpoint{1.178475in}{2.130471in}}{\pgfqpoint{1.186375in}{2.127198in}}{\pgfqpoint{1.194611in}{2.127198in}}%
\pgfpathclose%
\pgfusepath{stroke,fill}%
\end{pgfscope}%
\begin{pgfscope}%
\pgfpathrectangle{\pgfqpoint{0.100000in}{0.212622in}}{\pgfqpoint{3.696000in}{3.696000in}}%
\pgfusepath{clip}%
\pgfsetbuttcap%
\pgfsetroundjoin%
\definecolor{currentfill}{rgb}{0.121569,0.466667,0.705882}%
\pgfsetfillcolor{currentfill}%
\pgfsetfillopacity{0.522760}%
\pgfsetlinewidth{1.003750pt}%
\definecolor{currentstroke}{rgb}{0.121569,0.466667,0.705882}%
\pgfsetstrokecolor{currentstroke}%
\pgfsetstrokeopacity{0.522760}%
\pgfsetdash{}{0pt}%
\pgfpathmoveto{\pgfqpoint{2.051526in}{2.389675in}}%
\pgfpathcurveto{\pgfqpoint{2.059762in}{2.389675in}}{\pgfqpoint{2.067662in}{2.392947in}}{\pgfqpoint{2.073486in}{2.398771in}}%
\pgfpathcurveto{\pgfqpoint{2.079310in}{2.404595in}}{\pgfqpoint{2.082582in}{2.412495in}}{\pgfqpoint{2.082582in}{2.420732in}}%
\pgfpathcurveto{\pgfqpoint{2.082582in}{2.428968in}}{\pgfqpoint{2.079310in}{2.436868in}}{\pgfqpoint{2.073486in}{2.442692in}}%
\pgfpathcurveto{\pgfqpoint{2.067662in}{2.448516in}}{\pgfqpoint{2.059762in}{2.451788in}}{\pgfqpoint{2.051526in}{2.451788in}}%
\pgfpathcurveto{\pgfqpoint{2.043289in}{2.451788in}}{\pgfqpoint{2.035389in}{2.448516in}}{\pgfqpoint{2.029565in}{2.442692in}}%
\pgfpathcurveto{\pgfqpoint{2.023741in}{2.436868in}}{\pgfqpoint{2.020469in}{2.428968in}}{\pgfqpoint{2.020469in}{2.420732in}}%
\pgfpathcurveto{\pgfqpoint{2.020469in}{2.412495in}}{\pgfqpoint{2.023741in}{2.404595in}}{\pgfqpoint{2.029565in}{2.398771in}}%
\pgfpathcurveto{\pgfqpoint{2.035389in}{2.392947in}}{\pgfqpoint{2.043289in}{2.389675in}}{\pgfqpoint{2.051526in}{2.389675in}}%
\pgfpathclose%
\pgfusepath{stroke,fill}%
\end{pgfscope}%
\begin{pgfscope}%
\pgfpathrectangle{\pgfqpoint{0.100000in}{0.212622in}}{\pgfqpoint{3.696000in}{3.696000in}}%
\pgfusepath{clip}%
\pgfsetbuttcap%
\pgfsetroundjoin%
\definecolor{currentfill}{rgb}{0.121569,0.466667,0.705882}%
\pgfsetfillcolor{currentfill}%
\pgfsetfillopacity{0.523687}%
\pgfsetlinewidth{1.003750pt}%
\definecolor{currentstroke}{rgb}{0.121569,0.466667,0.705882}%
\pgfsetstrokecolor{currentstroke}%
\pgfsetstrokeopacity{0.523687}%
\pgfsetdash{}{0pt}%
\pgfpathmoveto{\pgfqpoint{1.186767in}{2.112878in}}%
\pgfpathcurveto{\pgfqpoint{1.195004in}{2.112878in}}{\pgfqpoint{1.202904in}{2.116150in}}{\pgfqpoint{1.208727in}{2.121974in}}%
\pgfpathcurveto{\pgfqpoint{1.214551in}{2.127798in}}{\pgfqpoint{1.217824in}{2.135698in}}{\pgfqpoint{1.217824in}{2.143935in}}%
\pgfpathcurveto{\pgfqpoint{1.217824in}{2.152171in}}{\pgfqpoint{1.214551in}{2.160071in}}{\pgfqpoint{1.208727in}{2.165895in}}%
\pgfpathcurveto{\pgfqpoint{1.202904in}{2.171719in}}{\pgfqpoint{1.195004in}{2.174991in}}{\pgfqpoint{1.186767in}{2.174991in}}%
\pgfpathcurveto{\pgfqpoint{1.178531in}{2.174991in}}{\pgfqpoint{1.170631in}{2.171719in}}{\pgfqpoint{1.164807in}{2.165895in}}%
\pgfpathcurveto{\pgfqpoint{1.158983in}{2.160071in}}{\pgfqpoint{1.155711in}{2.152171in}}{\pgfqpoint{1.155711in}{2.143935in}}%
\pgfpathcurveto{\pgfqpoint{1.155711in}{2.135698in}}{\pgfqpoint{1.158983in}{2.127798in}}{\pgfqpoint{1.164807in}{2.121974in}}%
\pgfpathcurveto{\pgfqpoint{1.170631in}{2.116150in}}{\pgfqpoint{1.178531in}{2.112878in}}{\pgfqpoint{1.186767in}{2.112878in}}%
\pgfpathclose%
\pgfusepath{stroke,fill}%
\end{pgfscope}%
\begin{pgfscope}%
\pgfpathrectangle{\pgfqpoint{0.100000in}{0.212622in}}{\pgfqpoint{3.696000in}{3.696000in}}%
\pgfusepath{clip}%
\pgfsetbuttcap%
\pgfsetroundjoin%
\definecolor{currentfill}{rgb}{0.121569,0.466667,0.705882}%
\pgfsetfillcolor{currentfill}%
\pgfsetfillopacity{0.525535}%
\pgfsetlinewidth{1.003750pt}%
\definecolor{currentstroke}{rgb}{0.121569,0.466667,0.705882}%
\pgfsetstrokecolor{currentstroke}%
\pgfsetstrokeopacity{0.525535}%
\pgfsetdash{}{0pt}%
\pgfpathmoveto{\pgfqpoint{2.052639in}{2.379655in}}%
\pgfpathcurveto{\pgfqpoint{2.060875in}{2.379655in}}{\pgfqpoint{2.068775in}{2.382927in}}{\pgfqpoint{2.074599in}{2.388751in}}%
\pgfpathcurveto{\pgfqpoint{2.080423in}{2.394575in}}{\pgfqpoint{2.083695in}{2.402475in}}{\pgfqpoint{2.083695in}{2.410711in}}%
\pgfpathcurveto{\pgfqpoint{2.083695in}{2.418948in}}{\pgfqpoint{2.080423in}{2.426848in}}{\pgfqpoint{2.074599in}{2.432672in}}%
\pgfpathcurveto{\pgfqpoint{2.068775in}{2.438495in}}{\pgfqpoint{2.060875in}{2.441768in}}{\pgfqpoint{2.052639in}{2.441768in}}%
\pgfpathcurveto{\pgfqpoint{2.044402in}{2.441768in}}{\pgfqpoint{2.036502in}{2.438495in}}{\pgfqpoint{2.030678in}{2.432672in}}%
\pgfpathcurveto{\pgfqpoint{2.024854in}{2.426848in}}{\pgfqpoint{2.021582in}{2.418948in}}{\pgfqpoint{2.021582in}{2.410711in}}%
\pgfpathcurveto{\pgfqpoint{2.021582in}{2.402475in}}{\pgfqpoint{2.024854in}{2.394575in}}{\pgfqpoint{2.030678in}{2.388751in}}%
\pgfpathcurveto{\pgfqpoint{2.036502in}{2.382927in}}{\pgfqpoint{2.044402in}{2.379655in}}{\pgfqpoint{2.052639in}{2.379655in}}%
\pgfpathclose%
\pgfusepath{stroke,fill}%
\end{pgfscope}%
\begin{pgfscope}%
\pgfpathrectangle{\pgfqpoint{0.100000in}{0.212622in}}{\pgfqpoint{3.696000in}{3.696000in}}%
\pgfusepath{clip}%
\pgfsetbuttcap%
\pgfsetroundjoin%
\definecolor{currentfill}{rgb}{0.121569,0.466667,0.705882}%
\pgfsetfillcolor{currentfill}%
\pgfsetfillopacity{0.526028}%
\pgfsetlinewidth{1.003750pt}%
\definecolor{currentstroke}{rgb}{0.121569,0.466667,0.705882}%
\pgfsetstrokecolor{currentstroke}%
\pgfsetstrokeopacity{0.526028}%
\pgfsetdash{}{0pt}%
\pgfpathmoveto{\pgfqpoint{1.178638in}{2.099946in}}%
\pgfpathcurveto{\pgfqpoint{1.186874in}{2.099946in}}{\pgfqpoint{1.194774in}{2.103219in}}{\pgfqpoint{1.200598in}{2.109043in}}%
\pgfpathcurveto{\pgfqpoint{1.206422in}{2.114866in}}{\pgfqpoint{1.209695in}{2.122767in}}{\pgfqpoint{1.209695in}{2.131003in}}%
\pgfpathcurveto{\pgfqpoint{1.209695in}{2.139239in}}{\pgfqpoint{1.206422in}{2.147139in}}{\pgfqpoint{1.200598in}{2.152963in}}%
\pgfpathcurveto{\pgfqpoint{1.194774in}{2.158787in}}{\pgfqpoint{1.186874in}{2.162059in}}{\pgfqpoint{1.178638in}{2.162059in}}%
\pgfpathcurveto{\pgfqpoint{1.170402in}{2.162059in}}{\pgfqpoint{1.162502in}{2.158787in}}{\pgfqpoint{1.156678in}{2.152963in}}%
\pgfpathcurveto{\pgfqpoint{1.150854in}{2.147139in}}{\pgfqpoint{1.147582in}{2.139239in}}{\pgfqpoint{1.147582in}{2.131003in}}%
\pgfpathcurveto{\pgfqpoint{1.147582in}{2.122767in}}{\pgfqpoint{1.150854in}{2.114866in}}{\pgfqpoint{1.156678in}{2.109043in}}%
\pgfpathcurveto{\pgfqpoint{1.162502in}{2.103219in}}{\pgfqpoint{1.170402in}{2.099946in}}{\pgfqpoint{1.178638in}{2.099946in}}%
\pgfpathclose%
\pgfusepath{stroke,fill}%
\end{pgfscope}%
\begin{pgfscope}%
\pgfpathrectangle{\pgfqpoint{0.100000in}{0.212622in}}{\pgfqpoint{3.696000in}{3.696000in}}%
\pgfusepath{clip}%
\pgfsetbuttcap%
\pgfsetroundjoin%
\definecolor{currentfill}{rgb}{0.121569,0.466667,0.705882}%
\pgfsetfillcolor{currentfill}%
\pgfsetfillopacity{0.528365}%
\pgfsetlinewidth{1.003750pt}%
\definecolor{currentstroke}{rgb}{0.121569,0.466667,0.705882}%
\pgfsetstrokecolor{currentstroke}%
\pgfsetstrokeopacity{0.528365}%
\pgfsetdash{}{0pt}%
\pgfpathmoveto{\pgfqpoint{1.172526in}{2.087249in}}%
\pgfpathcurveto{\pgfqpoint{1.180762in}{2.087249in}}{\pgfqpoint{1.188662in}{2.090521in}}{\pgfqpoint{1.194486in}{2.096345in}}%
\pgfpathcurveto{\pgfqpoint{1.200310in}{2.102169in}}{\pgfqpoint{1.203582in}{2.110069in}}{\pgfqpoint{1.203582in}{2.118305in}}%
\pgfpathcurveto{\pgfqpoint{1.203582in}{2.126541in}}{\pgfqpoint{1.200310in}{2.134441in}}{\pgfqpoint{1.194486in}{2.140265in}}%
\pgfpathcurveto{\pgfqpoint{1.188662in}{2.146089in}}{\pgfqpoint{1.180762in}{2.149362in}}{\pgfqpoint{1.172526in}{2.149362in}}%
\pgfpathcurveto{\pgfqpoint{1.164289in}{2.149362in}}{\pgfqpoint{1.156389in}{2.146089in}}{\pgfqpoint{1.150565in}{2.140265in}}%
\pgfpathcurveto{\pgfqpoint{1.144742in}{2.134441in}}{\pgfqpoint{1.141469in}{2.126541in}}{\pgfqpoint{1.141469in}{2.118305in}}%
\pgfpathcurveto{\pgfqpoint{1.141469in}{2.110069in}}{\pgfqpoint{1.144742in}{2.102169in}}{\pgfqpoint{1.150565in}{2.096345in}}%
\pgfpathcurveto{\pgfqpoint{1.156389in}{2.090521in}}{\pgfqpoint{1.164289in}{2.087249in}}{\pgfqpoint{1.172526in}{2.087249in}}%
\pgfpathclose%
\pgfusepath{stroke,fill}%
\end{pgfscope}%
\begin{pgfscope}%
\pgfpathrectangle{\pgfqpoint{0.100000in}{0.212622in}}{\pgfqpoint{3.696000in}{3.696000in}}%
\pgfusepath{clip}%
\pgfsetbuttcap%
\pgfsetroundjoin%
\definecolor{currentfill}{rgb}{0.121569,0.466667,0.705882}%
\pgfsetfillcolor{currentfill}%
\pgfsetfillopacity{0.528613}%
\pgfsetlinewidth{1.003750pt}%
\definecolor{currentstroke}{rgb}{0.121569,0.466667,0.705882}%
\pgfsetstrokecolor{currentstroke}%
\pgfsetstrokeopacity{0.528613}%
\pgfsetdash{}{0pt}%
\pgfpathmoveto{\pgfqpoint{2.055035in}{2.367393in}}%
\pgfpathcurveto{\pgfqpoint{2.063271in}{2.367393in}}{\pgfqpoint{2.071171in}{2.370665in}}{\pgfqpoint{2.076995in}{2.376489in}}%
\pgfpathcurveto{\pgfqpoint{2.082819in}{2.382313in}}{\pgfqpoint{2.086091in}{2.390213in}}{\pgfqpoint{2.086091in}{2.398449in}}%
\pgfpathcurveto{\pgfqpoint{2.086091in}{2.406685in}}{\pgfqpoint{2.082819in}{2.414585in}}{\pgfqpoint{2.076995in}{2.420409in}}%
\pgfpathcurveto{\pgfqpoint{2.071171in}{2.426233in}}{\pgfqpoint{2.063271in}{2.429506in}}{\pgfqpoint{2.055035in}{2.429506in}}%
\pgfpathcurveto{\pgfqpoint{2.046798in}{2.429506in}}{\pgfqpoint{2.038898in}{2.426233in}}{\pgfqpoint{2.033074in}{2.420409in}}%
\pgfpathcurveto{\pgfqpoint{2.027251in}{2.414585in}}{\pgfqpoint{2.023978in}{2.406685in}}{\pgfqpoint{2.023978in}{2.398449in}}%
\pgfpathcurveto{\pgfqpoint{2.023978in}{2.390213in}}{\pgfqpoint{2.027251in}{2.382313in}}{\pgfqpoint{2.033074in}{2.376489in}}%
\pgfpathcurveto{\pgfqpoint{2.038898in}{2.370665in}}{\pgfqpoint{2.046798in}{2.367393in}}{\pgfqpoint{2.055035in}{2.367393in}}%
\pgfpathclose%
\pgfusepath{stroke,fill}%
\end{pgfscope}%
\begin{pgfscope}%
\pgfpathrectangle{\pgfqpoint{0.100000in}{0.212622in}}{\pgfqpoint{3.696000in}{3.696000in}}%
\pgfusepath{clip}%
\pgfsetbuttcap%
\pgfsetroundjoin%
\definecolor{currentfill}{rgb}{0.121569,0.466667,0.705882}%
\pgfsetfillcolor{currentfill}%
\pgfsetfillopacity{0.530130}%
\pgfsetlinewidth{1.003750pt}%
\definecolor{currentstroke}{rgb}{0.121569,0.466667,0.705882}%
\pgfsetstrokecolor{currentstroke}%
\pgfsetstrokeopacity{0.530130}%
\pgfsetdash{}{0pt}%
\pgfpathmoveto{\pgfqpoint{1.166246in}{2.076342in}}%
\pgfpathcurveto{\pgfqpoint{1.174482in}{2.076342in}}{\pgfqpoint{1.182382in}{2.079614in}}{\pgfqpoint{1.188206in}{2.085438in}}%
\pgfpathcurveto{\pgfqpoint{1.194030in}{2.091262in}}{\pgfqpoint{1.197303in}{2.099162in}}{\pgfqpoint{1.197303in}{2.107398in}}%
\pgfpathcurveto{\pgfqpoint{1.197303in}{2.115634in}}{\pgfqpoint{1.194030in}{2.123534in}}{\pgfqpoint{1.188206in}{2.129358in}}%
\pgfpathcurveto{\pgfqpoint{1.182382in}{2.135182in}}{\pgfqpoint{1.174482in}{2.138455in}}{\pgfqpoint{1.166246in}{2.138455in}}%
\pgfpathcurveto{\pgfqpoint{1.158010in}{2.138455in}}{\pgfqpoint{1.150110in}{2.135182in}}{\pgfqpoint{1.144286in}{2.129358in}}%
\pgfpathcurveto{\pgfqpoint{1.138462in}{2.123534in}}{\pgfqpoint{1.135190in}{2.115634in}}{\pgfqpoint{1.135190in}{2.107398in}}%
\pgfpathcurveto{\pgfqpoint{1.135190in}{2.099162in}}{\pgfqpoint{1.138462in}{2.091262in}}{\pgfqpoint{1.144286in}{2.085438in}}%
\pgfpathcurveto{\pgfqpoint{1.150110in}{2.079614in}}{\pgfqpoint{1.158010in}{2.076342in}}{\pgfqpoint{1.166246in}{2.076342in}}%
\pgfpathclose%
\pgfusepath{stroke,fill}%
\end{pgfscope}%
\begin{pgfscope}%
\pgfpathrectangle{\pgfqpoint{0.100000in}{0.212622in}}{\pgfqpoint{3.696000in}{3.696000in}}%
\pgfusepath{clip}%
\pgfsetbuttcap%
\pgfsetroundjoin%
\definecolor{currentfill}{rgb}{0.121569,0.466667,0.705882}%
\pgfsetfillcolor{currentfill}%
\pgfsetfillopacity{0.531711}%
\pgfsetlinewidth{1.003750pt}%
\definecolor{currentstroke}{rgb}{0.121569,0.466667,0.705882}%
\pgfsetstrokecolor{currentstroke}%
\pgfsetstrokeopacity{0.531711}%
\pgfsetdash{}{0pt}%
\pgfpathmoveto{\pgfqpoint{1.161249in}{2.065556in}}%
\pgfpathcurveto{\pgfqpoint{1.169485in}{2.065556in}}{\pgfqpoint{1.177385in}{2.068828in}}{\pgfqpoint{1.183209in}{2.074652in}}%
\pgfpathcurveto{\pgfqpoint{1.189033in}{2.080476in}}{\pgfqpoint{1.192305in}{2.088376in}}{\pgfqpoint{1.192305in}{2.096612in}}%
\pgfpathcurveto{\pgfqpoint{1.192305in}{2.104848in}}{\pgfqpoint{1.189033in}{2.112748in}}{\pgfqpoint{1.183209in}{2.118572in}}%
\pgfpathcurveto{\pgfqpoint{1.177385in}{2.124396in}}{\pgfqpoint{1.169485in}{2.127669in}}{\pgfqpoint{1.161249in}{2.127669in}}%
\pgfpathcurveto{\pgfqpoint{1.153012in}{2.127669in}}{\pgfqpoint{1.145112in}{2.124396in}}{\pgfqpoint{1.139289in}{2.118572in}}%
\pgfpathcurveto{\pgfqpoint{1.133465in}{2.112748in}}{\pgfqpoint{1.130192in}{2.104848in}}{\pgfqpoint{1.130192in}{2.096612in}}%
\pgfpathcurveto{\pgfqpoint{1.130192in}{2.088376in}}{\pgfqpoint{1.133465in}{2.080476in}}{\pgfqpoint{1.139289in}{2.074652in}}%
\pgfpathcurveto{\pgfqpoint{1.145112in}{2.068828in}}{\pgfqpoint{1.153012in}{2.065556in}}{\pgfqpoint{1.161249in}{2.065556in}}%
\pgfpathclose%
\pgfusepath{stroke,fill}%
\end{pgfscope}%
\begin{pgfscope}%
\pgfpathrectangle{\pgfqpoint{0.100000in}{0.212622in}}{\pgfqpoint{3.696000in}{3.696000in}}%
\pgfusepath{clip}%
\pgfsetbuttcap%
\pgfsetroundjoin%
\definecolor{currentfill}{rgb}{0.121569,0.466667,0.705882}%
\pgfsetfillcolor{currentfill}%
\pgfsetfillopacity{0.531719}%
\pgfsetlinewidth{1.003750pt}%
\definecolor{currentstroke}{rgb}{0.121569,0.466667,0.705882}%
\pgfsetstrokecolor{currentstroke}%
\pgfsetstrokeopacity{0.531719}%
\pgfsetdash{}{0pt}%
\pgfpathmoveto{\pgfqpoint{2.057478in}{2.353735in}}%
\pgfpathcurveto{\pgfqpoint{2.065714in}{2.353735in}}{\pgfqpoint{2.073614in}{2.357008in}}{\pgfqpoint{2.079438in}{2.362831in}}%
\pgfpathcurveto{\pgfqpoint{2.085262in}{2.368655in}}{\pgfqpoint{2.088534in}{2.376555in}}{\pgfqpoint{2.088534in}{2.384792in}}%
\pgfpathcurveto{\pgfqpoint{2.088534in}{2.393028in}}{\pgfqpoint{2.085262in}{2.400928in}}{\pgfqpoint{2.079438in}{2.406752in}}%
\pgfpathcurveto{\pgfqpoint{2.073614in}{2.412576in}}{\pgfqpoint{2.065714in}{2.415848in}}{\pgfqpoint{2.057478in}{2.415848in}}%
\pgfpathcurveto{\pgfqpoint{2.049242in}{2.415848in}}{\pgfqpoint{2.041342in}{2.412576in}}{\pgfqpoint{2.035518in}{2.406752in}}%
\pgfpathcurveto{\pgfqpoint{2.029694in}{2.400928in}}{\pgfqpoint{2.026421in}{2.393028in}}{\pgfqpoint{2.026421in}{2.384792in}}%
\pgfpathcurveto{\pgfqpoint{2.026421in}{2.376555in}}{\pgfqpoint{2.029694in}{2.368655in}}{\pgfqpoint{2.035518in}{2.362831in}}%
\pgfpathcurveto{\pgfqpoint{2.041342in}{2.357008in}}{\pgfqpoint{2.049242in}{2.353735in}}{\pgfqpoint{2.057478in}{2.353735in}}%
\pgfpathclose%
\pgfusepath{stroke,fill}%
\end{pgfscope}%
\begin{pgfscope}%
\pgfpathrectangle{\pgfqpoint{0.100000in}{0.212622in}}{\pgfqpoint{3.696000in}{3.696000in}}%
\pgfusepath{clip}%
\pgfsetbuttcap%
\pgfsetroundjoin%
\definecolor{currentfill}{rgb}{0.121569,0.466667,0.705882}%
\pgfsetfillcolor{currentfill}%
\pgfsetfillopacity{0.533252}%
\pgfsetlinewidth{1.003750pt}%
\definecolor{currentstroke}{rgb}{0.121569,0.466667,0.705882}%
\pgfsetstrokecolor{currentstroke}%
\pgfsetstrokeopacity{0.533252}%
\pgfsetdash{}{0pt}%
\pgfpathmoveto{\pgfqpoint{1.156158in}{2.057111in}}%
\pgfpathcurveto{\pgfqpoint{1.164394in}{2.057111in}}{\pgfqpoint{1.172294in}{2.060383in}}{\pgfqpoint{1.178118in}{2.066207in}}%
\pgfpathcurveto{\pgfqpoint{1.183942in}{2.072031in}}{\pgfqpoint{1.187214in}{2.079931in}}{\pgfqpoint{1.187214in}{2.088168in}}%
\pgfpathcurveto{\pgfqpoint{1.187214in}{2.096404in}}{\pgfqpoint{1.183942in}{2.104304in}}{\pgfqpoint{1.178118in}{2.110128in}}%
\pgfpathcurveto{\pgfqpoint{1.172294in}{2.115952in}}{\pgfqpoint{1.164394in}{2.119224in}}{\pgfqpoint{1.156158in}{2.119224in}}%
\pgfpathcurveto{\pgfqpoint{1.147922in}{2.119224in}}{\pgfqpoint{1.140021in}{2.115952in}}{\pgfqpoint{1.134198in}{2.110128in}}%
\pgfpathcurveto{\pgfqpoint{1.128374in}{2.104304in}}{\pgfqpoint{1.125101in}{2.096404in}}{\pgfqpoint{1.125101in}{2.088168in}}%
\pgfpathcurveto{\pgfqpoint{1.125101in}{2.079931in}}{\pgfqpoint{1.128374in}{2.072031in}}{\pgfqpoint{1.134198in}{2.066207in}}%
\pgfpathcurveto{\pgfqpoint{1.140021in}{2.060383in}}{\pgfqpoint{1.147922in}{2.057111in}}{\pgfqpoint{1.156158in}{2.057111in}}%
\pgfpathclose%
\pgfusepath{stroke,fill}%
\end{pgfscope}%
\begin{pgfscope}%
\pgfpathrectangle{\pgfqpoint{0.100000in}{0.212622in}}{\pgfqpoint{3.696000in}{3.696000in}}%
\pgfusepath{clip}%
\pgfsetbuttcap%
\pgfsetroundjoin%
\definecolor{currentfill}{rgb}{0.121569,0.466667,0.705882}%
\pgfsetfillcolor{currentfill}%
\pgfsetfillopacity{0.534641}%
\pgfsetlinewidth{1.003750pt}%
\definecolor{currentstroke}{rgb}{0.121569,0.466667,0.705882}%
\pgfsetstrokecolor{currentstroke}%
\pgfsetstrokeopacity{0.534641}%
\pgfsetdash{}{0pt}%
\pgfpathmoveto{\pgfqpoint{1.151579in}{2.048589in}}%
\pgfpathcurveto{\pgfqpoint{1.159815in}{2.048589in}}{\pgfqpoint{1.167715in}{2.051861in}}{\pgfqpoint{1.173539in}{2.057685in}}%
\pgfpathcurveto{\pgfqpoint{1.179363in}{2.063509in}}{\pgfqpoint{1.182636in}{2.071409in}}{\pgfqpoint{1.182636in}{2.079645in}}%
\pgfpathcurveto{\pgfqpoint{1.182636in}{2.087882in}}{\pgfqpoint{1.179363in}{2.095782in}}{\pgfqpoint{1.173539in}{2.101606in}}%
\pgfpathcurveto{\pgfqpoint{1.167715in}{2.107430in}}{\pgfqpoint{1.159815in}{2.110702in}}{\pgfqpoint{1.151579in}{2.110702in}}%
\pgfpathcurveto{\pgfqpoint{1.143343in}{2.110702in}}{\pgfqpoint{1.135443in}{2.107430in}}{\pgfqpoint{1.129619in}{2.101606in}}%
\pgfpathcurveto{\pgfqpoint{1.123795in}{2.095782in}}{\pgfqpoint{1.120523in}{2.087882in}}{\pgfqpoint{1.120523in}{2.079645in}}%
\pgfpathcurveto{\pgfqpoint{1.120523in}{2.071409in}}{\pgfqpoint{1.123795in}{2.063509in}}{\pgfqpoint{1.129619in}{2.057685in}}%
\pgfpathcurveto{\pgfqpoint{1.135443in}{2.051861in}}{\pgfqpoint{1.143343in}{2.048589in}}{\pgfqpoint{1.151579in}{2.048589in}}%
\pgfpathclose%
\pgfusepath{stroke,fill}%
\end{pgfscope}%
\begin{pgfscope}%
\pgfpathrectangle{\pgfqpoint{0.100000in}{0.212622in}}{\pgfqpoint{3.696000in}{3.696000in}}%
\pgfusepath{clip}%
\pgfsetbuttcap%
\pgfsetroundjoin%
\definecolor{currentfill}{rgb}{0.121569,0.466667,0.705882}%
\pgfsetfillcolor{currentfill}%
\pgfsetfillopacity{0.535146}%
\pgfsetlinewidth{1.003750pt}%
\definecolor{currentstroke}{rgb}{0.121569,0.466667,0.705882}%
\pgfsetstrokecolor{currentstroke}%
\pgfsetstrokeopacity{0.535146}%
\pgfsetdash{}{0pt}%
\pgfpathmoveto{\pgfqpoint{1.149654in}{2.045189in}}%
\pgfpathcurveto{\pgfqpoint{1.157890in}{2.045189in}}{\pgfqpoint{1.165790in}{2.048461in}}{\pgfqpoint{1.171614in}{2.054285in}}%
\pgfpathcurveto{\pgfqpoint{1.177438in}{2.060109in}}{\pgfqpoint{1.180711in}{2.068009in}}{\pgfqpoint{1.180711in}{2.076245in}}%
\pgfpathcurveto{\pgfqpoint{1.180711in}{2.084482in}}{\pgfqpoint{1.177438in}{2.092382in}}{\pgfqpoint{1.171614in}{2.098206in}}%
\pgfpathcurveto{\pgfqpoint{1.165790in}{2.104030in}}{\pgfqpoint{1.157890in}{2.107302in}}{\pgfqpoint{1.149654in}{2.107302in}}%
\pgfpathcurveto{\pgfqpoint{1.141418in}{2.107302in}}{\pgfqpoint{1.133518in}{2.104030in}}{\pgfqpoint{1.127694in}{2.098206in}}%
\pgfpathcurveto{\pgfqpoint{1.121870in}{2.092382in}}{\pgfqpoint{1.118598in}{2.084482in}}{\pgfqpoint{1.118598in}{2.076245in}}%
\pgfpathcurveto{\pgfqpoint{1.118598in}{2.068009in}}{\pgfqpoint{1.121870in}{2.060109in}}{\pgfqpoint{1.127694in}{2.054285in}}%
\pgfpathcurveto{\pgfqpoint{1.133518in}{2.048461in}}{\pgfqpoint{1.141418in}{2.045189in}}{\pgfqpoint{1.149654in}{2.045189in}}%
\pgfpathclose%
\pgfusepath{stroke,fill}%
\end{pgfscope}%
\begin{pgfscope}%
\pgfpathrectangle{\pgfqpoint{0.100000in}{0.212622in}}{\pgfqpoint{3.696000in}{3.696000in}}%
\pgfusepath{clip}%
\pgfsetbuttcap%
\pgfsetroundjoin%
\definecolor{currentfill}{rgb}{0.121569,0.466667,0.705882}%
\pgfsetfillcolor{currentfill}%
\pgfsetfillopacity{0.535499}%
\pgfsetlinewidth{1.003750pt}%
\definecolor{currentstroke}{rgb}{0.121569,0.466667,0.705882}%
\pgfsetstrokecolor{currentstroke}%
\pgfsetstrokeopacity{0.535499}%
\pgfsetdash{}{0pt}%
\pgfpathmoveto{\pgfqpoint{2.058959in}{2.340050in}}%
\pgfpathcurveto{\pgfqpoint{2.067195in}{2.340050in}}{\pgfqpoint{2.075095in}{2.343322in}}{\pgfqpoint{2.080919in}{2.349146in}}%
\pgfpathcurveto{\pgfqpoint{2.086743in}{2.354970in}}{\pgfqpoint{2.090015in}{2.362870in}}{\pgfqpoint{2.090015in}{2.371106in}}%
\pgfpathcurveto{\pgfqpoint{2.090015in}{2.379342in}}{\pgfqpoint{2.086743in}{2.387242in}}{\pgfqpoint{2.080919in}{2.393066in}}%
\pgfpathcurveto{\pgfqpoint{2.075095in}{2.398890in}}{\pgfqpoint{2.067195in}{2.402163in}}{\pgfqpoint{2.058959in}{2.402163in}}%
\pgfpathcurveto{\pgfqpoint{2.050723in}{2.402163in}}{\pgfqpoint{2.042823in}{2.398890in}}{\pgfqpoint{2.036999in}{2.393066in}}%
\pgfpathcurveto{\pgfqpoint{2.031175in}{2.387242in}}{\pgfqpoint{2.027902in}{2.379342in}}{\pgfqpoint{2.027902in}{2.371106in}}%
\pgfpathcurveto{\pgfqpoint{2.027902in}{2.362870in}}{\pgfqpoint{2.031175in}{2.354970in}}{\pgfqpoint{2.036999in}{2.349146in}}%
\pgfpathcurveto{\pgfqpoint{2.042823in}{2.343322in}}{\pgfqpoint{2.050723in}{2.340050in}}{\pgfqpoint{2.058959in}{2.340050in}}%
\pgfpathclose%
\pgfusepath{stroke,fill}%
\end{pgfscope}%
\begin{pgfscope}%
\pgfpathrectangle{\pgfqpoint{0.100000in}{0.212622in}}{\pgfqpoint{3.696000in}{3.696000in}}%
\pgfusepath{clip}%
\pgfsetbuttcap%
\pgfsetroundjoin%
\definecolor{currentfill}{rgb}{0.121569,0.466667,0.705882}%
\pgfsetfillcolor{currentfill}%
\pgfsetfillopacity{0.535637}%
\pgfsetlinewidth{1.003750pt}%
\definecolor{currentstroke}{rgb}{0.121569,0.466667,0.705882}%
\pgfsetstrokecolor{currentstroke}%
\pgfsetstrokeopacity{0.535637}%
\pgfsetdash{}{0pt}%
\pgfpathmoveto{\pgfqpoint{1.148433in}{2.041709in}}%
\pgfpathcurveto{\pgfqpoint{1.156670in}{2.041709in}}{\pgfqpoint{1.164570in}{2.044981in}}{\pgfqpoint{1.170394in}{2.050805in}}%
\pgfpathcurveto{\pgfqpoint{1.176217in}{2.056629in}}{\pgfqpoint{1.179490in}{2.064529in}}{\pgfqpoint{1.179490in}{2.072766in}}%
\pgfpathcurveto{\pgfqpoint{1.179490in}{2.081002in}}{\pgfqpoint{1.176217in}{2.088902in}}{\pgfqpoint{1.170394in}{2.094726in}}%
\pgfpathcurveto{\pgfqpoint{1.164570in}{2.100550in}}{\pgfqpoint{1.156670in}{2.103822in}}{\pgfqpoint{1.148433in}{2.103822in}}%
\pgfpathcurveto{\pgfqpoint{1.140197in}{2.103822in}}{\pgfqpoint{1.132297in}{2.100550in}}{\pgfqpoint{1.126473in}{2.094726in}}%
\pgfpathcurveto{\pgfqpoint{1.120649in}{2.088902in}}{\pgfqpoint{1.117377in}{2.081002in}}{\pgfqpoint{1.117377in}{2.072766in}}%
\pgfpathcurveto{\pgfqpoint{1.117377in}{2.064529in}}{\pgfqpoint{1.120649in}{2.056629in}}{\pgfqpoint{1.126473in}{2.050805in}}%
\pgfpathcurveto{\pgfqpoint{1.132297in}{2.044981in}}{\pgfqpoint{1.140197in}{2.041709in}}{\pgfqpoint{1.148433in}{2.041709in}}%
\pgfpathclose%
\pgfusepath{stroke,fill}%
\end{pgfscope}%
\begin{pgfscope}%
\pgfpathrectangle{\pgfqpoint{0.100000in}{0.212622in}}{\pgfqpoint{3.696000in}{3.696000in}}%
\pgfusepath{clip}%
\pgfsetbuttcap%
\pgfsetroundjoin%
\definecolor{currentfill}{rgb}{0.121569,0.466667,0.705882}%
\pgfsetfillcolor{currentfill}%
\pgfsetfillopacity{0.535899}%
\pgfsetlinewidth{1.003750pt}%
\definecolor{currentstroke}{rgb}{0.121569,0.466667,0.705882}%
\pgfsetstrokecolor{currentstroke}%
\pgfsetstrokeopacity{0.535899}%
\pgfsetdash{}{0pt}%
\pgfpathmoveto{\pgfqpoint{1.147590in}{2.040414in}}%
\pgfpathcurveto{\pgfqpoint{1.155827in}{2.040414in}}{\pgfqpoint{1.163727in}{2.043686in}}{\pgfqpoint{1.169551in}{2.049510in}}%
\pgfpathcurveto{\pgfqpoint{1.175375in}{2.055334in}}{\pgfqpoint{1.178647in}{2.063234in}}{\pgfqpoint{1.178647in}{2.071471in}}%
\pgfpathcurveto{\pgfqpoint{1.178647in}{2.079707in}}{\pgfqpoint{1.175375in}{2.087607in}}{\pgfqpoint{1.169551in}{2.093431in}}%
\pgfpathcurveto{\pgfqpoint{1.163727in}{2.099255in}}{\pgfqpoint{1.155827in}{2.102527in}}{\pgfqpoint{1.147590in}{2.102527in}}%
\pgfpathcurveto{\pgfqpoint{1.139354in}{2.102527in}}{\pgfqpoint{1.131454in}{2.099255in}}{\pgfqpoint{1.125630in}{2.093431in}}%
\pgfpathcurveto{\pgfqpoint{1.119806in}{2.087607in}}{\pgfqpoint{1.116534in}{2.079707in}}{\pgfqpoint{1.116534in}{2.071471in}}%
\pgfpathcurveto{\pgfqpoint{1.116534in}{2.063234in}}{\pgfqpoint{1.119806in}{2.055334in}}{\pgfqpoint{1.125630in}{2.049510in}}%
\pgfpathcurveto{\pgfqpoint{1.131454in}{2.043686in}}{\pgfqpoint{1.139354in}{2.040414in}}{\pgfqpoint{1.147590in}{2.040414in}}%
\pgfpathclose%
\pgfusepath{stroke,fill}%
\end{pgfscope}%
\begin{pgfscope}%
\pgfpathrectangle{\pgfqpoint{0.100000in}{0.212622in}}{\pgfqpoint{3.696000in}{3.696000in}}%
\pgfusepath{clip}%
\pgfsetbuttcap%
\pgfsetroundjoin%
\definecolor{currentfill}{rgb}{0.121569,0.466667,0.705882}%
\pgfsetfillcolor{currentfill}%
\pgfsetfillopacity{0.536042}%
\pgfsetlinewidth{1.003750pt}%
\definecolor{currentstroke}{rgb}{0.121569,0.466667,0.705882}%
\pgfsetstrokecolor{currentstroke}%
\pgfsetstrokeopacity{0.536042}%
\pgfsetdash{}{0pt}%
\pgfpathmoveto{\pgfqpoint{1.147194in}{2.039717in}}%
\pgfpathcurveto{\pgfqpoint{1.155431in}{2.039717in}}{\pgfqpoint{1.163331in}{2.042989in}}{\pgfqpoint{1.169155in}{2.048813in}}%
\pgfpathcurveto{\pgfqpoint{1.174978in}{2.054637in}}{\pgfqpoint{1.178251in}{2.062537in}}{\pgfqpoint{1.178251in}{2.070774in}}%
\pgfpathcurveto{\pgfqpoint{1.178251in}{2.079010in}}{\pgfqpoint{1.174978in}{2.086910in}}{\pgfqpoint{1.169155in}{2.092734in}}%
\pgfpathcurveto{\pgfqpoint{1.163331in}{2.098558in}}{\pgfqpoint{1.155431in}{2.101830in}}{\pgfqpoint{1.147194in}{2.101830in}}%
\pgfpathcurveto{\pgfqpoint{1.138958in}{2.101830in}}{\pgfqpoint{1.131058in}{2.098558in}}{\pgfqpoint{1.125234in}{2.092734in}}%
\pgfpathcurveto{\pgfqpoint{1.119410in}{2.086910in}}{\pgfqpoint{1.116138in}{2.079010in}}{\pgfqpoint{1.116138in}{2.070774in}}%
\pgfpathcurveto{\pgfqpoint{1.116138in}{2.062537in}}{\pgfqpoint{1.119410in}{2.054637in}}{\pgfqpoint{1.125234in}{2.048813in}}%
\pgfpathcurveto{\pgfqpoint{1.131058in}{2.042989in}}{\pgfqpoint{1.138958in}{2.039717in}}{\pgfqpoint{1.147194in}{2.039717in}}%
\pgfpathclose%
\pgfusepath{stroke,fill}%
\end{pgfscope}%
\begin{pgfscope}%
\pgfpathrectangle{\pgfqpoint{0.100000in}{0.212622in}}{\pgfqpoint{3.696000in}{3.696000in}}%
\pgfusepath{clip}%
\pgfsetbuttcap%
\pgfsetroundjoin%
\definecolor{currentfill}{rgb}{0.121569,0.466667,0.705882}%
\pgfsetfillcolor{currentfill}%
\pgfsetfillopacity{0.536051}%
\pgfsetlinewidth{1.003750pt}%
\definecolor{currentstroke}{rgb}{0.121569,0.466667,0.705882}%
\pgfsetstrokecolor{currentstroke}%
\pgfsetstrokeopacity{0.536051}%
\pgfsetdash{}{0pt}%
\pgfpathmoveto{\pgfqpoint{1.147171in}{2.039683in}}%
\pgfpathcurveto{\pgfqpoint{1.155407in}{2.039683in}}{\pgfqpoint{1.163307in}{2.042956in}}{\pgfqpoint{1.169131in}{2.048780in}}%
\pgfpathcurveto{\pgfqpoint{1.174955in}{2.054604in}}{\pgfqpoint{1.178228in}{2.062504in}}{\pgfqpoint{1.178228in}{2.070740in}}%
\pgfpathcurveto{\pgfqpoint{1.178228in}{2.078976in}}{\pgfqpoint{1.174955in}{2.086876in}}{\pgfqpoint{1.169131in}{2.092700in}}%
\pgfpathcurveto{\pgfqpoint{1.163307in}{2.098524in}}{\pgfqpoint{1.155407in}{2.101796in}}{\pgfqpoint{1.147171in}{2.101796in}}%
\pgfpathcurveto{\pgfqpoint{1.138935in}{2.101796in}}{\pgfqpoint{1.131035in}{2.098524in}}{\pgfqpoint{1.125211in}{2.092700in}}%
\pgfpathcurveto{\pgfqpoint{1.119387in}{2.086876in}}{\pgfqpoint{1.116115in}{2.078976in}}{\pgfqpoint{1.116115in}{2.070740in}}%
\pgfpathcurveto{\pgfqpoint{1.116115in}{2.062504in}}{\pgfqpoint{1.119387in}{2.054604in}}{\pgfqpoint{1.125211in}{2.048780in}}%
\pgfpathcurveto{\pgfqpoint{1.131035in}{2.042956in}}{\pgfqpoint{1.138935in}{2.039683in}}{\pgfqpoint{1.147171in}{2.039683in}}%
\pgfpathclose%
\pgfusepath{stroke,fill}%
\end{pgfscope}%
\begin{pgfscope}%
\pgfpathrectangle{\pgfqpoint{0.100000in}{0.212622in}}{\pgfqpoint{3.696000in}{3.696000in}}%
\pgfusepath{clip}%
\pgfsetbuttcap%
\pgfsetroundjoin%
\definecolor{currentfill}{rgb}{0.121569,0.466667,0.705882}%
\pgfsetfillcolor{currentfill}%
\pgfsetfillopacity{0.536066}%
\pgfsetlinewidth{1.003750pt}%
\definecolor{currentstroke}{rgb}{0.121569,0.466667,0.705882}%
\pgfsetstrokecolor{currentstroke}%
\pgfsetstrokeopacity{0.536066}%
\pgfsetdash{}{0pt}%
\pgfpathmoveto{\pgfqpoint{1.147130in}{2.039620in}}%
\pgfpathcurveto{\pgfqpoint{1.155366in}{2.039620in}}{\pgfqpoint{1.163266in}{2.042892in}}{\pgfqpoint{1.169090in}{2.048716in}}%
\pgfpathcurveto{\pgfqpoint{1.174914in}{2.054540in}}{\pgfqpoint{1.178187in}{2.062440in}}{\pgfqpoint{1.178187in}{2.070676in}}%
\pgfpathcurveto{\pgfqpoint{1.178187in}{2.078913in}}{\pgfqpoint{1.174914in}{2.086813in}}{\pgfqpoint{1.169090in}{2.092637in}}%
\pgfpathcurveto{\pgfqpoint{1.163266in}{2.098461in}}{\pgfqpoint{1.155366in}{2.101733in}}{\pgfqpoint{1.147130in}{2.101733in}}%
\pgfpathcurveto{\pgfqpoint{1.138894in}{2.101733in}}{\pgfqpoint{1.130994in}{2.098461in}}{\pgfqpoint{1.125170in}{2.092637in}}%
\pgfpathcurveto{\pgfqpoint{1.119346in}{2.086813in}}{\pgfqpoint{1.116074in}{2.078913in}}{\pgfqpoint{1.116074in}{2.070676in}}%
\pgfpathcurveto{\pgfqpoint{1.116074in}{2.062440in}}{\pgfqpoint{1.119346in}{2.054540in}}{\pgfqpoint{1.125170in}{2.048716in}}%
\pgfpathcurveto{\pgfqpoint{1.130994in}{2.042892in}}{\pgfqpoint{1.138894in}{2.039620in}}{\pgfqpoint{1.147130in}{2.039620in}}%
\pgfpathclose%
\pgfusepath{stroke,fill}%
\end{pgfscope}%
\begin{pgfscope}%
\pgfpathrectangle{\pgfqpoint{0.100000in}{0.212622in}}{\pgfqpoint{3.696000in}{3.696000in}}%
\pgfusepath{clip}%
\pgfsetbuttcap%
\pgfsetroundjoin%
\definecolor{currentfill}{rgb}{0.121569,0.466667,0.705882}%
\pgfsetfillcolor{currentfill}%
\pgfsetfillopacity{0.536092}%
\pgfsetlinewidth{1.003750pt}%
\definecolor{currentstroke}{rgb}{0.121569,0.466667,0.705882}%
\pgfsetstrokecolor{currentstroke}%
\pgfsetstrokeopacity{0.536092}%
\pgfsetdash{}{0pt}%
\pgfpathmoveto{\pgfqpoint{1.147052in}{2.039505in}}%
\pgfpathcurveto{\pgfqpoint{1.155288in}{2.039505in}}{\pgfqpoint{1.163189in}{2.042777in}}{\pgfqpoint{1.169012in}{2.048601in}}%
\pgfpathcurveto{\pgfqpoint{1.174836in}{2.054425in}}{\pgfqpoint{1.178109in}{2.062325in}}{\pgfqpoint{1.178109in}{2.070562in}}%
\pgfpathcurveto{\pgfqpoint{1.178109in}{2.078798in}}{\pgfqpoint{1.174836in}{2.086698in}}{\pgfqpoint{1.169012in}{2.092522in}}%
\pgfpathcurveto{\pgfqpoint{1.163189in}{2.098346in}}{\pgfqpoint{1.155288in}{2.101618in}}{\pgfqpoint{1.147052in}{2.101618in}}%
\pgfpathcurveto{\pgfqpoint{1.138816in}{2.101618in}}{\pgfqpoint{1.130916in}{2.098346in}}{\pgfqpoint{1.125092in}{2.092522in}}%
\pgfpathcurveto{\pgfqpoint{1.119268in}{2.086698in}}{\pgfqpoint{1.115996in}{2.078798in}}{\pgfqpoint{1.115996in}{2.070562in}}%
\pgfpathcurveto{\pgfqpoint{1.115996in}{2.062325in}}{\pgfqpoint{1.119268in}{2.054425in}}{\pgfqpoint{1.125092in}{2.048601in}}%
\pgfpathcurveto{\pgfqpoint{1.130916in}{2.042777in}}{\pgfqpoint{1.138816in}{2.039505in}}{\pgfqpoint{1.147052in}{2.039505in}}%
\pgfpathclose%
\pgfusepath{stroke,fill}%
\end{pgfscope}%
\begin{pgfscope}%
\pgfpathrectangle{\pgfqpoint{0.100000in}{0.212622in}}{\pgfqpoint{3.696000in}{3.696000in}}%
\pgfusepath{clip}%
\pgfsetbuttcap%
\pgfsetroundjoin%
\definecolor{currentfill}{rgb}{0.121569,0.466667,0.705882}%
\pgfsetfillcolor{currentfill}%
\pgfsetfillopacity{0.536139}%
\pgfsetlinewidth{1.003750pt}%
\definecolor{currentstroke}{rgb}{0.121569,0.466667,0.705882}%
\pgfsetstrokecolor{currentstroke}%
\pgfsetstrokeopacity{0.536139}%
\pgfsetdash{}{0pt}%
\pgfpathmoveto{\pgfqpoint{1.146911in}{2.039294in}}%
\pgfpathcurveto{\pgfqpoint{1.155148in}{2.039294in}}{\pgfqpoint{1.163048in}{2.042566in}}{\pgfqpoint{1.168871in}{2.048390in}}%
\pgfpathcurveto{\pgfqpoint{1.174695in}{2.054214in}}{\pgfqpoint{1.177968in}{2.062114in}}{\pgfqpoint{1.177968in}{2.070351in}}%
\pgfpathcurveto{\pgfqpoint{1.177968in}{2.078587in}}{\pgfqpoint{1.174695in}{2.086487in}}{\pgfqpoint{1.168871in}{2.092311in}}%
\pgfpathcurveto{\pgfqpoint{1.163048in}{2.098135in}}{\pgfqpoint{1.155148in}{2.101407in}}{\pgfqpoint{1.146911in}{2.101407in}}%
\pgfpathcurveto{\pgfqpoint{1.138675in}{2.101407in}}{\pgfqpoint{1.130775in}{2.098135in}}{\pgfqpoint{1.124951in}{2.092311in}}%
\pgfpathcurveto{\pgfqpoint{1.119127in}{2.086487in}}{\pgfqpoint{1.115855in}{2.078587in}}{\pgfqpoint{1.115855in}{2.070351in}}%
\pgfpathcurveto{\pgfqpoint{1.115855in}{2.062114in}}{\pgfqpoint{1.119127in}{2.054214in}}{\pgfqpoint{1.124951in}{2.048390in}}%
\pgfpathcurveto{\pgfqpoint{1.130775in}{2.042566in}}{\pgfqpoint{1.138675in}{2.039294in}}{\pgfqpoint{1.146911in}{2.039294in}}%
\pgfpathclose%
\pgfusepath{stroke,fill}%
\end{pgfscope}%
\begin{pgfscope}%
\pgfpathrectangle{\pgfqpoint{0.100000in}{0.212622in}}{\pgfqpoint{3.696000in}{3.696000in}}%
\pgfusepath{clip}%
\pgfsetbuttcap%
\pgfsetroundjoin%
\definecolor{currentfill}{rgb}{0.121569,0.466667,0.705882}%
\pgfsetfillcolor{currentfill}%
\pgfsetfillopacity{0.536234}%
\pgfsetlinewidth{1.003750pt}%
\definecolor{currentstroke}{rgb}{0.121569,0.466667,0.705882}%
\pgfsetstrokecolor{currentstroke}%
\pgfsetstrokeopacity{0.536234}%
\pgfsetdash{}{0pt}%
\pgfpathmoveto{\pgfqpoint{1.146674in}{2.038914in}}%
\pgfpathcurveto{\pgfqpoint{1.154910in}{2.038914in}}{\pgfqpoint{1.162810in}{2.042187in}}{\pgfqpoint{1.168634in}{2.048011in}}%
\pgfpathcurveto{\pgfqpoint{1.174458in}{2.053834in}}{\pgfqpoint{1.177730in}{2.061735in}}{\pgfqpoint{1.177730in}{2.069971in}}%
\pgfpathcurveto{\pgfqpoint{1.177730in}{2.078207in}}{\pgfqpoint{1.174458in}{2.086107in}}{\pgfqpoint{1.168634in}{2.091931in}}%
\pgfpathcurveto{\pgfqpoint{1.162810in}{2.097755in}}{\pgfqpoint{1.154910in}{2.101027in}}{\pgfqpoint{1.146674in}{2.101027in}}%
\pgfpathcurveto{\pgfqpoint{1.138438in}{2.101027in}}{\pgfqpoint{1.130538in}{2.097755in}}{\pgfqpoint{1.124714in}{2.091931in}}%
\pgfpathcurveto{\pgfqpoint{1.118890in}{2.086107in}}{\pgfqpoint{1.115617in}{2.078207in}}{\pgfqpoint{1.115617in}{2.069971in}}%
\pgfpathcurveto{\pgfqpoint{1.115617in}{2.061735in}}{\pgfqpoint{1.118890in}{2.053834in}}{\pgfqpoint{1.124714in}{2.048011in}}%
\pgfpathcurveto{\pgfqpoint{1.130538in}{2.042187in}}{\pgfqpoint{1.138438in}{2.038914in}}{\pgfqpoint{1.146674in}{2.038914in}}%
\pgfpathclose%
\pgfusepath{stroke,fill}%
\end{pgfscope}%
\begin{pgfscope}%
\pgfpathrectangle{\pgfqpoint{0.100000in}{0.212622in}}{\pgfqpoint{3.696000in}{3.696000in}}%
\pgfusepath{clip}%
\pgfsetbuttcap%
\pgfsetroundjoin%
\definecolor{currentfill}{rgb}{0.121569,0.466667,0.705882}%
\pgfsetfillcolor{currentfill}%
\pgfsetfillopacity{0.536398}%
\pgfsetlinewidth{1.003750pt}%
\definecolor{currentstroke}{rgb}{0.121569,0.466667,0.705882}%
\pgfsetstrokecolor{currentstroke}%
\pgfsetstrokeopacity{0.536398}%
\pgfsetdash{}{0pt}%
\pgfpathmoveto{\pgfqpoint{1.146227in}{2.038217in}}%
\pgfpathcurveto{\pgfqpoint{1.154463in}{2.038217in}}{\pgfqpoint{1.162363in}{2.041489in}}{\pgfqpoint{1.168187in}{2.047313in}}%
\pgfpathcurveto{\pgfqpoint{1.174011in}{2.053137in}}{\pgfqpoint{1.177283in}{2.061037in}}{\pgfqpoint{1.177283in}{2.069273in}}%
\pgfpathcurveto{\pgfqpoint{1.177283in}{2.077510in}}{\pgfqpoint{1.174011in}{2.085410in}}{\pgfqpoint{1.168187in}{2.091234in}}%
\pgfpathcurveto{\pgfqpoint{1.162363in}{2.097057in}}{\pgfqpoint{1.154463in}{2.100330in}}{\pgfqpoint{1.146227in}{2.100330in}}%
\pgfpathcurveto{\pgfqpoint{1.137991in}{2.100330in}}{\pgfqpoint{1.130090in}{2.097057in}}{\pgfqpoint{1.124267in}{2.091234in}}%
\pgfpathcurveto{\pgfqpoint{1.118443in}{2.085410in}}{\pgfqpoint{1.115170in}{2.077510in}}{\pgfqpoint{1.115170in}{2.069273in}}%
\pgfpathcurveto{\pgfqpoint{1.115170in}{2.061037in}}{\pgfqpoint{1.118443in}{2.053137in}}{\pgfqpoint{1.124267in}{2.047313in}}%
\pgfpathcurveto{\pgfqpoint{1.130090in}{2.041489in}}{\pgfqpoint{1.137991in}{2.038217in}}{\pgfqpoint{1.146227in}{2.038217in}}%
\pgfpathclose%
\pgfusepath{stroke,fill}%
\end{pgfscope}%
\begin{pgfscope}%
\pgfpathrectangle{\pgfqpoint{0.100000in}{0.212622in}}{\pgfqpoint{3.696000in}{3.696000in}}%
\pgfusepath{clip}%
\pgfsetbuttcap%
\pgfsetroundjoin%
\definecolor{currentfill}{rgb}{0.121569,0.466667,0.705882}%
\pgfsetfillcolor{currentfill}%
\pgfsetfillopacity{0.536689}%
\pgfsetlinewidth{1.003750pt}%
\definecolor{currentstroke}{rgb}{0.121569,0.466667,0.705882}%
\pgfsetstrokecolor{currentstroke}%
\pgfsetstrokeopacity{0.536689}%
\pgfsetdash{}{0pt}%
\pgfpathmoveto{\pgfqpoint{1.145437in}{2.036889in}}%
\pgfpathcurveto{\pgfqpoint{1.153674in}{2.036889in}}{\pgfqpoint{1.161574in}{2.040161in}}{\pgfqpoint{1.167398in}{2.045985in}}%
\pgfpathcurveto{\pgfqpoint{1.173222in}{2.051809in}}{\pgfqpoint{1.176494in}{2.059709in}}{\pgfqpoint{1.176494in}{2.067945in}}%
\pgfpathcurveto{\pgfqpoint{1.176494in}{2.076182in}}{\pgfqpoint{1.173222in}{2.084082in}}{\pgfqpoint{1.167398in}{2.089906in}}%
\pgfpathcurveto{\pgfqpoint{1.161574in}{2.095729in}}{\pgfqpoint{1.153674in}{2.099002in}}{\pgfqpoint{1.145437in}{2.099002in}}%
\pgfpathcurveto{\pgfqpoint{1.137201in}{2.099002in}}{\pgfqpoint{1.129301in}{2.095729in}}{\pgfqpoint{1.123477in}{2.089906in}}%
\pgfpathcurveto{\pgfqpoint{1.117653in}{2.084082in}}{\pgfqpoint{1.114381in}{2.076182in}}{\pgfqpoint{1.114381in}{2.067945in}}%
\pgfpathcurveto{\pgfqpoint{1.114381in}{2.059709in}}{\pgfqpoint{1.117653in}{2.051809in}}{\pgfqpoint{1.123477in}{2.045985in}}%
\pgfpathcurveto{\pgfqpoint{1.129301in}{2.040161in}}{\pgfqpoint{1.137201in}{2.036889in}}{\pgfqpoint{1.145437in}{2.036889in}}%
\pgfpathclose%
\pgfusepath{stroke,fill}%
\end{pgfscope}%
\begin{pgfscope}%
\pgfpathrectangle{\pgfqpoint{0.100000in}{0.212622in}}{\pgfqpoint{3.696000in}{3.696000in}}%
\pgfusepath{clip}%
\pgfsetbuttcap%
\pgfsetroundjoin%
\definecolor{currentfill}{rgb}{0.121569,0.466667,0.705882}%
\pgfsetfillcolor{currentfill}%
\pgfsetfillopacity{0.537223}%
\pgfsetlinewidth{1.003750pt}%
\definecolor{currentstroke}{rgb}{0.121569,0.466667,0.705882}%
\pgfsetstrokecolor{currentstroke}%
\pgfsetstrokeopacity{0.537223}%
\pgfsetdash{}{0pt}%
\pgfpathmoveto{\pgfqpoint{1.143961in}{2.034550in}}%
\pgfpathcurveto{\pgfqpoint{1.152197in}{2.034550in}}{\pgfqpoint{1.160097in}{2.037823in}}{\pgfqpoint{1.165921in}{2.043647in}}%
\pgfpathcurveto{\pgfqpoint{1.171745in}{2.049471in}}{\pgfqpoint{1.175017in}{2.057371in}}{\pgfqpoint{1.175017in}{2.065607in}}%
\pgfpathcurveto{\pgfqpoint{1.175017in}{2.073843in}}{\pgfqpoint{1.171745in}{2.081743in}}{\pgfqpoint{1.165921in}{2.087567in}}%
\pgfpathcurveto{\pgfqpoint{1.160097in}{2.093391in}}{\pgfqpoint{1.152197in}{2.096663in}}{\pgfqpoint{1.143961in}{2.096663in}}%
\pgfpathcurveto{\pgfqpoint{1.135724in}{2.096663in}}{\pgfqpoint{1.127824in}{2.093391in}}{\pgfqpoint{1.122000in}{2.087567in}}%
\pgfpathcurveto{\pgfqpoint{1.116176in}{2.081743in}}{\pgfqpoint{1.112904in}{2.073843in}}{\pgfqpoint{1.112904in}{2.065607in}}%
\pgfpathcurveto{\pgfqpoint{1.112904in}{2.057371in}}{\pgfqpoint{1.116176in}{2.049471in}}{\pgfqpoint{1.122000in}{2.043647in}}%
\pgfpathcurveto{\pgfqpoint{1.127824in}{2.037823in}}{\pgfqpoint{1.135724in}{2.034550in}}{\pgfqpoint{1.143961in}{2.034550in}}%
\pgfpathclose%
\pgfusepath{stroke,fill}%
\end{pgfscope}%
\begin{pgfscope}%
\pgfpathrectangle{\pgfqpoint{0.100000in}{0.212622in}}{\pgfqpoint{3.696000in}{3.696000in}}%
\pgfusepath{clip}%
\pgfsetbuttcap%
\pgfsetroundjoin%
\definecolor{currentfill}{rgb}{0.121569,0.466667,0.705882}%
\pgfsetfillcolor{currentfill}%
\pgfsetfillopacity{0.537619}%
\pgfsetlinewidth{1.003750pt}%
\definecolor{currentstroke}{rgb}{0.121569,0.466667,0.705882}%
\pgfsetstrokecolor{currentstroke}%
\pgfsetstrokeopacity{0.537619}%
\pgfsetdash{}{0pt}%
\pgfpathmoveto{\pgfqpoint{1.142899in}{2.032721in}}%
\pgfpathcurveto{\pgfqpoint{1.151135in}{2.032721in}}{\pgfqpoint{1.159035in}{2.035994in}}{\pgfqpoint{1.164859in}{2.041818in}}%
\pgfpathcurveto{\pgfqpoint{1.170683in}{2.047641in}}{\pgfqpoint{1.173955in}{2.055541in}}{\pgfqpoint{1.173955in}{2.063778in}}%
\pgfpathcurveto{\pgfqpoint{1.173955in}{2.072014in}}{\pgfqpoint{1.170683in}{2.079914in}}{\pgfqpoint{1.164859in}{2.085738in}}%
\pgfpathcurveto{\pgfqpoint{1.159035in}{2.091562in}}{\pgfqpoint{1.151135in}{2.094834in}}{\pgfqpoint{1.142899in}{2.094834in}}%
\pgfpathcurveto{\pgfqpoint{1.134662in}{2.094834in}}{\pgfqpoint{1.126762in}{2.091562in}}{\pgfqpoint{1.120938in}{2.085738in}}%
\pgfpathcurveto{\pgfqpoint{1.115114in}{2.079914in}}{\pgfqpoint{1.111842in}{2.072014in}}{\pgfqpoint{1.111842in}{2.063778in}}%
\pgfpathcurveto{\pgfqpoint{1.111842in}{2.055541in}}{\pgfqpoint{1.115114in}{2.047641in}}{\pgfqpoint{1.120938in}{2.041818in}}%
\pgfpathcurveto{\pgfqpoint{1.126762in}{2.035994in}}{\pgfqpoint{1.134662in}{2.032721in}}{\pgfqpoint{1.142899in}{2.032721in}}%
\pgfpathclose%
\pgfusepath{stroke,fill}%
\end{pgfscope}%
\begin{pgfscope}%
\pgfpathrectangle{\pgfqpoint{0.100000in}{0.212622in}}{\pgfqpoint{3.696000in}{3.696000in}}%
\pgfusepath{clip}%
\pgfsetbuttcap%
\pgfsetroundjoin%
\definecolor{currentfill}{rgb}{0.121569,0.466667,0.705882}%
\pgfsetfillcolor{currentfill}%
\pgfsetfillopacity{0.538285}%
\pgfsetlinewidth{1.003750pt}%
\definecolor{currentstroke}{rgb}{0.121569,0.466667,0.705882}%
\pgfsetstrokecolor{currentstroke}%
\pgfsetstrokeopacity{0.538285}%
\pgfsetdash{}{0pt}%
\pgfpathmoveto{\pgfqpoint{1.140858in}{2.029352in}}%
\pgfpathcurveto{\pgfqpoint{1.149094in}{2.029352in}}{\pgfqpoint{1.156994in}{2.032624in}}{\pgfqpoint{1.162818in}{2.038448in}}%
\pgfpathcurveto{\pgfqpoint{1.168642in}{2.044272in}}{\pgfqpoint{1.171914in}{2.052172in}}{\pgfqpoint{1.171914in}{2.060408in}}%
\pgfpathcurveto{\pgfqpoint{1.171914in}{2.068645in}}{\pgfqpoint{1.168642in}{2.076545in}}{\pgfqpoint{1.162818in}{2.082369in}}%
\pgfpathcurveto{\pgfqpoint{1.156994in}{2.088193in}}{\pgfqpoint{1.149094in}{2.091465in}}{\pgfqpoint{1.140858in}{2.091465in}}%
\pgfpathcurveto{\pgfqpoint{1.132621in}{2.091465in}}{\pgfqpoint{1.124721in}{2.088193in}}{\pgfqpoint{1.118897in}{2.082369in}}%
\pgfpathcurveto{\pgfqpoint{1.113073in}{2.076545in}}{\pgfqpoint{1.109801in}{2.068645in}}{\pgfqpoint{1.109801in}{2.060408in}}%
\pgfpathcurveto{\pgfqpoint{1.109801in}{2.052172in}}{\pgfqpoint{1.113073in}{2.044272in}}{\pgfqpoint{1.118897in}{2.038448in}}%
\pgfpathcurveto{\pgfqpoint{1.124721in}{2.032624in}}{\pgfqpoint{1.132621in}{2.029352in}}{\pgfqpoint{1.140858in}{2.029352in}}%
\pgfpathclose%
\pgfusepath{stroke,fill}%
\end{pgfscope}%
\begin{pgfscope}%
\pgfpathrectangle{\pgfqpoint{0.100000in}{0.212622in}}{\pgfqpoint{3.696000in}{3.696000in}}%
\pgfusepath{clip}%
\pgfsetbuttcap%
\pgfsetroundjoin%
\definecolor{currentfill}{rgb}{0.121569,0.466667,0.705882}%
\pgfsetfillcolor{currentfill}%
\pgfsetfillopacity{0.539529}%
\pgfsetlinewidth{1.003750pt}%
\definecolor{currentstroke}{rgb}{0.121569,0.466667,0.705882}%
\pgfsetstrokecolor{currentstroke}%
\pgfsetstrokeopacity{0.539529}%
\pgfsetdash{}{0pt}%
\pgfpathmoveto{\pgfqpoint{1.137437in}{2.022958in}}%
\pgfpathcurveto{\pgfqpoint{1.145673in}{2.022958in}}{\pgfqpoint{1.153574in}{2.026230in}}{\pgfqpoint{1.159397in}{2.032054in}}%
\pgfpathcurveto{\pgfqpoint{1.165221in}{2.037878in}}{\pgfqpoint{1.168494in}{2.045778in}}{\pgfqpoint{1.168494in}{2.054015in}}%
\pgfpathcurveto{\pgfqpoint{1.168494in}{2.062251in}}{\pgfqpoint{1.165221in}{2.070151in}}{\pgfqpoint{1.159397in}{2.075975in}}%
\pgfpathcurveto{\pgfqpoint{1.153574in}{2.081799in}}{\pgfqpoint{1.145673in}{2.085071in}}{\pgfqpoint{1.137437in}{2.085071in}}%
\pgfpathcurveto{\pgfqpoint{1.129201in}{2.085071in}}{\pgfqpoint{1.121301in}{2.081799in}}{\pgfqpoint{1.115477in}{2.075975in}}%
\pgfpathcurveto{\pgfqpoint{1.109653in}{2.070151in}}{\pgfqpoint{1.106381in}{2.062251in}}{\pgfqpoint{1.106381in}{2.054015in}}%
\pgfpathcurveto{\pgfqpoint{1.106381in}{2.045778in}}{\pgfqpoint{1.109653in}{2.037878in}}{\pgfqpoint{1.115477in}{2.032054in}}%
\pgfpathcurveto{\pgfqpoint{1.121301in}{2.026230in}}{\pgfqpoint{1.129201in}{2.022958in}}{\pgfqpoint{1.137437in}{2.022958in}}%
\pgfpathclose%
\pgfusepath{stroke,fill}%
\end{pgfscope}%
\begin{pgfscope}%
\pgfpathrectangle{\pgfqpoint{0.100000in}{0.212622in}}{\pgfqpoint{3.696000in}{3.696000in}}%
\pgfusepath{clip}%
\pgfsetbuttcap%
\pgfsetroundjoin%
\definecolor{currentfill}{rgb}{0.121569,0.466667,0.705882}%
\pgfsetfillcolor{currentfill}%
\pgfsetfillopacity{0.539605}%
\pgfsetlinewidth{1.003750pt}%
\definecolor{currentstroke}{rgb}{0.121569,0.466667,0.705882}%
\pgfsetstrokecolor{currentstroke}%
\pgfsetstrokeopacity{0.539605}%
\pgfsetdash{}{0pt}%
\pgfpathmoveto{\pgfqpoint{2.060432in}{2.326475in}}%
\pgfpathcurveto{\pgfqpoint{2.068669in}{2.326475in}}{\pgfqpoint{2.076569in}{2.329748in}}{\pgfqpoint{2.082393in}{2.335572in}}%
\pgfpathcurveto{\pgfqpoint{2.088217in}{2.341396in}}{\pgfqpoint{2.091489in}{2.349296in}}{\pgfqpoint{2.091489in}{2.357532in}}%
\pgfpathcurveto{\pgfqpoint{2.091489in}{2.365768in}}{\pgfqpoint{2.088217in}{2.373668in}}{\pgfqpoint{2.082393in}{2.379492in}}%
\pgfpathcurveto{\pgfqpoint{2.076569in}{2.385316in}}{\pgfqpoint{2.068669in}{2.388588in}}{\pgfqpoint{2.060432in}{2.388588in}}%
\pgfpathcurveto{\pgfqpoint{2.052196in}{2.388588in}}{\pgfqpoint{2.044296in}{2.385316in}}{\pgfqpoint{2.038472in}{2.379492in}}%
\pgfpathcurveto{\pgfqpoint{2.032648in}{2.373668in}}{\pgfqpoint{2.029376in}{2.365768in}}{\pgfqpoint{2.029376in}{2.357532in}}%
\pgfpathcurveto{\pgfqpoint{2.029376in}{2.349296in}}{\pgfqpoint{2.032648in}{2.341396in}}{\pgfqpoint{2.038472in}{2.335572in}}%
\pgfpathcurveto{\pgfqpoint{2.044296in}{2.329748in}}{\pgfqpoint{2.052196in}{2.326475in}}{\pgfqpoint{2.060432in}{2.326475in}}%
\pgfpathclose%
\pgfusepath{stroke,fill}%
\end{pgfscope}%
\begin{pgfscope}%
\pgfpathrectangle{\pgfqpoint{0.100000in}{0.212622in}}{\pgfqpoint{3.696000in}{3.696000in}}%
\pgfusepath{clip}%
\pgfsetbuttcap%
\pgfsetroundjoin%
\definecolor{currentfill}{rgb}{0.121569,0.466667,0.705882}%
\pgfsetfillcolor{currentfill}%
\pgfsetfillopacity{0.541720}%
\pgfsetlinewidth{1.003750pt}%
\definecolor{currentstroke}{rgb}{0.121569,0.466667,0.705882}%
\pgfsetstrokecolor{currentstroke}%
\pgfsetstrokeopacity{0.541720}%
\pgfsetdash{}{0pt}%
\pgfpathmoveto{\pgfqpoint{1.130635in}{2.011847in}}%
\pgfpathcurveto{\pgfqpoint{1.138871in}{2.011847in}}{\pgfqpoint{1.146771in}{2.015119in}}{\pgfqpoint{1.152595in}{2.020943in}}%
\pgfpathcurveto{\pgfqpoint{1.158419in}{2.026767in}}{\pgfqpoint{1.161691in}{2.034667in}}{\pgfqpoint{1.161691in}{2.042903in}}%
\pgfpathcurveto{\pgfqpoint{1.161691in}{2.051140in}}{\pgfqpoint{1.158419in}{2.059040in}}{\pgfqpoint{1.152595in}{2.064864in}}%
\pgfpathcurveto{\pgfqpoint{1.146771in}{2.070688in}}{\pgfqpoint{1.138871in}{2.073960in}}{\pgfqpoint{1.130635in}{2.073960in}}%
\pgfpathcurveto{\pgfqpoint{1.122398in}{2.073960in}}{\pgfqpoint{1.114498in}{2.070688in}}{\pgfqpoint{1.108674in}{2.064864in}}%
\pgfpathcurveto{\pgfqpoint{1.102851in}{2.059040in}}{\pgfqpoint{1.099578in}{2.051140in}}{\pgfqpoint{1.099578in}{2.042903in}}%
\pgfpathcurveto{\pgfqpoint{1.099578in}{2.034667in}}{\pgfqpoint{1.102851in}{2.026767in}}{\pgfqpoint{1.108674in}{2.020943in}}%
\pgfpathcurveto{\pgfqpoint{1.114498in}{2.015119in}}{\pgfqpoint{1.122398in}{2.011847in}}{\pgfqpoint{1.130635in}{2.011847in}}%
\pgfpathclose%
\pgfusepath{stroke,fill}%
\end{pgfscope}%
\begin{pgfscope}%
\pgfpathrectangle{\pgfqpoint{0.100000in}{0.212622in}}{\pgfqpoint{3.696000in}{3.696000in}}%
\pgfusepath{clip}%
\pgfsetbuttcap%
\pgfsetroundjoin%
\definecolor{currentfill}{rgb}{0.121569,0.466667,0.705882}%
\pgfsetfillcolor{currentfill}%
\pgfsetfillopacity{0.543548}%
\pgfsetlinewidth{1.003750pt}%
\definecolor{currentstroke}{rgb}{0.121569,0.466667,0.705882}%
\pgfsetstrokecolor{currentstroke}%
\pgfsetstrokeopacity{0.543548}%
\pgfsetdash{}{0pt}%
\pgfpathmoveto{\pgfqpoint{2.063555in}{2.311696in}}%
\pgfpathcurveto{\pgfqpoint{2.071791in}{2.311696in}}{\pgfqpoint{2.079691in}{2.314968in}}{\pgfqpoint{2.085515in}{2.320792in}}%
\pgfpathcurveto{\pgfqpoint{2.091339in}{2.326616in}}{\pgfqpoint{2.094612in}{2.334516in}}{\pgfqpoint{2.094612in}{2.342752in}}%
\pgfpathcurveto{\pgfqpoint{2.094612in}{2.350989in}}{\pgfqpoint{2.091339in}{2.358889in}}{\pgfqpoint{2.085515in}{2.364713in}}%
\pgfpathcurveto{\pgfqpoint{2.079691in}{2.370537in}}{\pgfqpoint{2.071791in}{2.373809in}}{\pgfqpoint{2.063555in}{2.373809in}}%
\pgfpathcurveto{\pgfqpoint{2.055319in}{2.373809in}}{\pgfqpoint{2.047419in}{2.370537in}}{\pgfqpoint{2.041595in}{2.364713in}}%
\pgfpathcurveto{\pgfqpoint{2.035771in}{2.358889in}}{\pgfqpoint{2.032499in}{2.350989in}}{\pgfqpoint{2.032499in}{2.342752in}}%
\pgfpathcurveto{\pgfqpoint{2.032499in}{2.334516in}}{\pgfqpoint{2.035771in}{2.326616in}}{\pgfqpoint{2.041595in}{2.320792in}}%
\pgfpathcurveto{\pgfqpoint{2.047419in}{2.314968in}}{\pgfqpoint{2.055319in}{2.311696in}}{\pgfqpoint{2.063555in}{2.311696in}}%
\pgfpathclose%
\pgfusepath{stroke,fill}%
\end{pgfscope}%
\begin{pgfscope}%
\pgfpathrectangle{\pgfqpoint{0.100000in}{0.212622in}}{\pgfqpoint{3.696000in}{3.696000in}}%
\pgfusepath{clip}%
\pgfsetbuttcap%
\pgfsetroundjoin%
\definecolor{currentfill}{rgb}{0.121569,0.466667,0.705882}%
\pgfsetfillcolor{currentfill}%
\pgfsetfillopacity{0.543923}%
\pgfsetlinewidth{1.003750pt}%
\definecolor{currentstroke}{rgb}{0.121569,0.466667,0.705882}%
\pgfsetstrokecolor{currentstroke}%
\pgfsetstrokeopacity{0.543923}%
\pgfsetdash{}{0pt}%
\pgfpathmoveto{\pgfqpoint{1.124665in}{2.000884in}}%
\pgfpathcurveto{\pgfqpoint{1.132901in}{2.000884in}}{\pgfqpoint{1.140801in}{2.004156in}}{\pgfqpoint{1.146625in}{2.009980in}}%
\pgfpathcurveto{\pgfqpoint{1.152449in}{2.015804in}}{\pgfqpoint{1.155721in}{2.023704in}}{\pgfqpoint{1.155721in}{2.031940in}}%
\pgfpathcurveto{\pgfqpoint{1.155721in}{2.040177in}}{\pgfqpoint{1.152449in}{2.048077in}}{\pgfqpoint{1.146625in}{2.053901in}}%
\pgfpathcurveto{\pgfqpoint{1.140801in}{2.059725in}}{\pgfqpoint{1.132901in}{2.062997in}}{\pgfqpoint{1.124665in}{2.062997in}}%
\pgfpathcurveto{\pgfqpoint{1.116428in}{2.062997in}}{\pgfqpoint{1.108528in}{2.059725in}}{\pgfqpoint{1.102704in}{2.053901in}}%
\pgfpathcurveto{\pgfqpoint{1.096880in}{2.048077in}}{\pgfqpoint{1.093608in}{2.040177in}}{\pgfqpoint{1.093608in}{2.031940in}}%
\pgfpathcurveto{\pgfqpoint{1.093608in}{2.023704in}}{\pgfqpoint{1.096880in}{2.015804in}}{\pgfqpoint{1.102704in}{2.009980in}}%
\pgfpathcurveto{\pgfqpoint{1.108528in}{2.004156in}}{\pgfqpoint{1.116428in}{2.000884in}}{\pgfqpoint{1.124665in}{2.000884in}}%
\pgfpathclose%
\pgfusepath{stroke,fill}%
\end{pgfscope}%
\begin{pgfscope}%
\pgfpathrectangle{\pgfqpoint{0.100000in}{0.212622in}}{\pgfqpoint{3.696000in}{3.696000in}}%
\pgfusepath{clip}%
\pgfsetbuttcap%
\pgfsetroundjoin%
\definecolor{currentfill}{rgb}{0.121569,0.466667,0.705882}%
\pgfsetfillcolor{currentfill}%
\pgfsetfillopacity{0.545289}%
\pgfsetlinewidth{1.003750pt}%
\definecolor{currentstroke}{rgb}{0.121569,0.466667,0.705882}%
\pgfsetstrokecolor{currentstroke}%
\pgfsetstrokeopacity{0.545289}%
\pgfsetdash{}{0pt}%
\pgfpathmoveto{\pgfqpoint{1.119788in}{1.992829in}}%
\pgfpathcurveto{\pgfqpoint{1.128024in}{1.992829in}}{\pgfqpoint{1.135924in}{1.996102in}}{\pgfqpoint{1.141748in}{2.001926in}}%
\pgfpathcurveto{\pgfqpoint{1.147572in}{2.007750in}}{\pgfqpoint{1.150844in}{2.015650in}}{\pgfqpoint{1.150844in}{2.023886in}}%
\pgfpathcurveto{\pgfqpoint{1.150844in}{2.032122in}}{\pgfqpoint{1.147572in}{2.040022in}}{\pgfqpoint{1.141748in}{2.045846in}}%
\pgfpathcurveto{\pgfqpoint{1.135924in}{2.051670in}}{\pgfqpoint{1.128024in}{2.054942in}}{\pgfqpoint{1.119788in}{2.054942in}}%
\pgfpathcurveto{\pgfqpoint{1.111551in}{2.054942in}}{\pgfqpoint{1.103651in}{2.051670in}}{\pgfqpoint{1.097827in}{2.045846in}}%
\pgfpathcurveto{\pgfqpoint{1.092003in}{2.040022in}}{\pgfqpoint{1.088731in}{2.032122in}}{\pgfqpoint{1.088731in}{2.023886in}}%
\pgfpathcurveto{\pgfqpoint{1.088731in}{2.015650in}}{\pgfqpoint{1.092003in}{2.007750in}}{\pgfqpoint{1.097827in}{2.001926in}}%
\pgfpathcurveto{\pgfqpoint{1.103651in}{1.996102in}}{\pgfqpoint{1.111551in}{1.992829in}}{\pgfqpoint{1.119788in}{1.992829in}}%
\pgfpathclose%
\pgfusepath{stroke,fill}%
\end{pgfscope}%
\begin{pgfscope}%
\pgfpathrectangle{\pgfqpoint{0.100000in}{0.212622in}}{\pgfqpoint{3.696000in}{3.696000in}}%
\pgfusepath{clip}%
\pgfsetbuttcap%
\pgfsetroundjoin%
\definecolor{currentfill}{rgb}{0.121569,0.466667,0.705882}%
\pgfsetfillcolor{currentfill}%
\pgfsetfillopacity{0.546697}%
\pgfsetlinewidth{1.003750pt}%
\definecolor{currentstroke}{rgb}{0.121569,0.466667,0.705882}%
\pgfsetstrokecolor{currentstroke}%
\pgfsetstrokeopacity{0.546697}%
\pgfsetdash{}{0pt}%
\pgfpathmoveto{\pgfqpoint{1.116133in}{1.985025in}}%
\pgfpathcurveto{\pgfqpoint{1.124370in}{1.985025in}}{\pgfqpoint{1.132270in}{1.988297in}}{\pgfqpoint{1.138094in}{1.994121in}}%
\pgfpathcurveto{\pgfqpoint{1.143918in}{1.999945in}}{\pgfqpoint{1.147190in}{2.007845in}}{\pgfqpoint{1.147190in}{2.016081in}}%
\pgfpathcurveto{\pgfqpoint{1.147190in}{2.024317in}}{\pgfqpoint{1.143918in}{2.032217in}}{\pgfqpoint{1.138094in}{2.038041in}}%
\pgfpathcurveto{\pgfqpoint{1.132270in}{2.043865in}}{\pgfqpoint{1.124370in}{2.047138in}}{\pgfqpoint{1.116133in}{2.047138in}}%
\pgfpathcurveto{\pgfqpoint{1.107897in}{2.047138in}}{\pgfqpoint{1.099997in}{2.043865in}}{\pgfqpoint{1.094173in}{2.038041in}}%
\pgfpathcurveto{\pgfqpoint{1.088349in}{2.032217in}}{\pgfqpoint{1.085077in}{2.024317in}}{\pgfqpoint{1.085077in}{2.016081in}}%
\pgfpathcurveto{\pgfqpoint{1.085077in}{2.007845in}}{\pgfqpoint{1.088349in}{1.999945in}}{\pgfqpoint{1.094173in}{1.994121in}}%
\pgfpathcurveto{\pgfqpoint{1.099997in}{1.988297in}}{\pgfqpoint{1.107897in}{1.985025in}}{\pgfqpoint{1.116133in}{1.985025in}}%
\pgfpathclose%
\pgfusepath{stroke,fill}%
\end{pgfscope}%
\begin{pgfscope}%
\pgfpathrectangle{\pgfqpoint{0.100000in}{0.212622in}}{\pgfqpoint{3.696000in}{3.696000in}}%
\pgfusepath{clip}%
\pgfsetbuttcap%
\pgfsetroundjoin%
\definecolor{currentfill}{rgb}{0.121569,0.466667,0.705882}%
\pgfsetfillcolor{currentfill}%
\pgfsetfillopacity{0.547550}%
\pgfsetlinewidth{1.003750pt}%
\definecolor{currentstroke}{rgb}{0.121569,0.466667,0.705882}%
\pgfsetstrokecolor{currentstroke}%
\pgfsetstrokeopacity{0.547550}%
\pgfsetdash{}{0pt}%
\pgfpathmoveto{\pgfqpoint{1.113048in}{1.979849in}}%
\pgfpathcurveto{\pgfqpoint{1.121285in}{1.979849in}}{\pgfqpoint{1.129185in}{1.983122in}}{\pgfqpoint{1.135009in}{1.988946in}}%
\pgfpathcurveto{\pgfqpoint{1.140833in}{1.994769in}}{\pgfqpoint{1.144105in}{2.002670in}}{\pgfqpoint{1.144105in}{2.010906in}}%
\pgfpathcurveto{\pgfqpoint{1.144105in}{2.019142in}}{\pgfqpoint{1.140833in}{2.027042in}}{\pgfqpoint{1.135009in}{2.032866in}}%
\pgfpathcurveto{\pgfqpoint{1.129185in}{2.038690in}}{\pgfqpoint{1.121285in}{2.041962in}}{\pgfqpoint{1.113048in}{2.041962in}}%
\pgfpathcurveto{\pgfqpoint{1.104812in}{2.041962in}}{\pgfqpoint{1.096912in}{2.038690in}}{\pgfqpoint{1.091088in}{2.032866in}}%
\pgfpathcurveto{\pgfqpoint{1.085264in}{2.027042in}}{\pgfqpoint{1.081992in}{2.019142in}}{\pgfqpoint{1.081992in}{2.010906in}}%
\pgfpathcurveto{\pgfqpoint{1.081992in}{2.002670in}}{\pgfqpoint{1.085264in}{1.994769in}}{\pgfqpoint{1.091088in}{1.988946in}}%
\pgfpathcurveto{\pgfqpoint{1.096912in}{1.983122in}}{\pgfqpoint{1.104812in}{1.979849in}}{\pgfqpoint{1.113048in}{1.979849in}}%
\pgfpathclose%
\pgfusepath{stroke,fill}%
\end{pgfscope}%
\begin{pgfscope}%
\pgfpathrectangle{\pgfqpoint{0.100000in}{0.212622in}}{\pgfqpoint{3.696000in}{3.696000in}}%
\pgfusepath{clip}%
\pgfsetbuttcap%
\pgfsetroundjoin%
\definecolor{currentfill}{rgb}{0.121569,0.466667,0.705882}%
\pgfsetfillcolor{currentfill}%
\pgfsetfillopacity{0.547634}%
\pgfsetlinewidth{1.003750pt}%
\definecolor{currentstroke}{rgb}{0.121569,0.466667,0.705882}%
\pgfsetstrokecolor{currentstroke}%
\pgfsetstrokeopacity{0.547634}%
\pgfsetdash{}{0pt}%
\pgfpathmoveto{\pgfqpoint{2.066594in}{2.295172in}}%
\pgfpathcurveto{\pgfqpoint{2.074830in}{2.295172in}}{\pgfqpoint{2.082730in}{2.298445in}}{\pgfqpoint{2.088554in}{2.304269in}}%
\pgfpathcurveto{\pgfqpoint{2.094378in}{2.310093in}}{\pgfqpoint{2.097651in}{2.317993in}}{\pgfqpoint{2.097651in}{2.326229in}}%
\pgfpathcurveto{\pgfqpoint{2.097651in}{2.334465in}}{\pgfqpoint{2.094378in}{2.342365in}}{\pgfqpoint{2.088554in}{2.348189in}}%
\pgfpathcurveto{\pgfqpoint{2.082730in}{2.354013in}}{\pgfqpoint{2.074830in}{2.357285in}}{\pgfqpoint{2.066594in}{2.357285in}}%
\pgfpathcurveto{\pgfqpoint{2.058358in}{2.357285in}}{\pgfqpoint{2.050458in}{2.354013in}}{\pgfqpoint{2.044634in}{2.348189in}}%
\pgfpathcurveto{\pgfqpoint{2.038810in}{2.342365in}}{\pgfqpoint{2.035538in}{2.334465in}}{\pgfqpoint{2.035538in}{2.326229in}}%
\pgfpathcurveto{\pgfqpoint{2.035538in}{2.317993in}}{\pgfqpoint{2.038810in}{2.310093in}}{\pgfqpoint{2.044634in}{2.304269in}}%
\pgfpathcurveto{\pgfqpoint{2.050458in}{2.298445in}}{\pgfqpoint{2.058358in}{2.295172in}}{\pgfqpoint{2.066594in}{2.295172in}}%
\pgfpathclose%
\pgfusepath{stroke,fill}%
\end{pgfscope}%
\begin{pgfscope}%
\pgfpathrectangle{\pgfqpoint{0.100000in}{0.212622in}}{\pgfqpoint{3.696000in}{3.696000in}}%
\pgfusepath{clip}%
\pgfsetbuttcap%
\pgfsetroundjoin%
\definecolor{currentfill}{rgb}{0.121569,0.466667,0.705882}%
\pgfsetfillcolor{currentfill}%
\pgfsetfillopacity{0.548394}%
\pgfsetlinewidth{1.003750pt}%
\definecolor{currentstroke}{rgb}{0.121569,0.466667,0.705882}%
\pgfsetstrokecolor{currentstroke}%
\pgfsetstrokeopacity{0.548394}%
\pgfsetdash{}{0pt}%
\pgfpathmoveto{\pgfqpoint{1.110696in}{1.974611in}}%
\pgfpathcurveto{\pgfqpoint{1.118932in}{1.974611in}}{\pgfqpoint{1.126832in}{1.977884in}}{\pgfqpoint{1.132656in}{1.983708in}}%
\pgfpathcurveto{\pgfqpoint{1.138480in}{1.989531in}}{\pgfqpoint{1.141752in}{1.997432in}}{\pgfqpoint{1.141752in}{2.005668in}}%
\pgfpathcurveto{\pgfqpoint{1.141752in}{2.013904in}}{\pgfqpoint{1.138480in}{2.021804in}}{\pgfqpoint{1.132656in}{2.027628in}}%
\pgfpathcurveto{\pgfqpoint{1.126832in}{2.033452in}}{\pgfqpoint{1.118932in}{2.036724in}}{\pgfqpoint{1.110696in}{2.036724in}}%
\pgfpathcurveto{\pgfqpoint{1.102459in}{2.036724in}}{\pgfqpoint{1.094559in}{2.033452in}}{\pgfqpoint{1.088735in}{2.027628in}}%
\pgfpathcurveto{\pgfqpoint{1.082911in}{2.021804in}}{\pgfqpoint{1.079639in}{2.013904in}}{\pgfqpoint{1.079639in}{2.005668in}}%
\pgfpathcurveto{\pgfqpoint{1.079639in}{1.997432in}}{\pgfqpoint{1.082911in}{1.989531in}}{\pgfqpoint{1.088735in}{1.983708in}}%
\pgfpathcurveto{\pgfqpoint{1.094559in}{1.977884in}}{\pgfqpoint{1.102459in}{1.974611in}}{\pgfqpoint{1.110696in}{1.974611in}}%
\pgfpathclose%
\pgfusepath{stroke,fill}%
\end{pgfscope}%
\begin{pgfscope}%
\pgfpathrectangle{\pgfqpoint{0.100000in}{0.212622in}}{\pgfqpoint{3.696000in}{3.696000in}}%
\pgfusepath{clip}%
\pgfsetbuttcap%
\pgfsetroundjoin%
\definecolor{currentfill}{rgb}{0.121569,0.466667,0.705882}%
\pgfsetfillcolor{currentfill}%
\pgfsetfillopacity{0.548837}%
\pgfsetlinewidth{1.003750pt}%
\definecolor{currentstroke}{rgb}{0.121569,0.466667,0.705882}%
\pgfsetstrokecolor{currentstroke}%
\pgfsetstrokeopacity{0.548837}%
\pgfsetdash{}{0pt}%
\pgfpathmoveto{\pgfqpoint{1.109097in}{1.971765in}}%
\pgfpathcurveto{\pgfqpoint{1.117333in}{1.971765in}}{\pgfqpoint{1.125233in}{1.975037in}}{\pgfqpoint{1.131057in}{1.980861in}}%
\pgfpathcurveto{\pgfqpoint{1.136881in}{1.986685in}}{\pgfqpoint{1.140153in}{1.994585in}}{\pgfqpoint{1.140153in}{2.002822in}}%
\pgfpathcurveto{\pgfqpoint{1.140153in}{2.011058in}}{\pgfqpoint{1.136881in}{2.018958in}}{\pgfqpoint{1.131057in}{2.024782in}}%
\pgfpathcurveto{\pgfqpoint{1.125233in}{2.030606in}}{\pgfqpoint{1.117333in}{2.033878in}}{\pgfqpoint{1.109097in}{2.033878in}}%
\pgfpathcurveto{\pgfqpoint{1.100861in}{2.033878in}}{\pgfqpoint{1.092961in}{2.030606in}}{\pgfqpoint{1.087137in}{2.024782in}}%
\pgfpathcurveto{\pgfqpoint{1.081313in}{2.018958in}}{\pgfqpoint{1.078040in}{2.011058in}}{\pgfqpoint{1.078040in}{2.002822in}}%
\pgfpathcurveto{\pgfqpoint{1.078040in}{1.994585in}}{\pgfqpoint{1.081313in}{1.986685in}}{\pgfqpoint{1.087137in}{1.980861in}}%
\pgfpathcurveto{\pgfqpoint{1.092961in}{1.975037in}}{\pgfqpoint{1.100861in}{1.971765in}}{\pgfqpoint{1.109097in}{1.971765in}}%
\pgfpathclose%
\pgfusepath{stroke,fill}%
\end{pgfscope}%
\begin{pgfscope}%
\pgfpathrectangle{\pgfqpoint{0.100000in}{0.212622in}}{\pgfqpoint{3.696000in}{3.696000in}}%
\pgfusepath{clip}%
\pgfsetbuttcap%
\pgfsetroundjoin%
\definecolor{currentfill}{rgb}{0.121569,0.466667,0.705882}%
\pgfsetfillcolor{currentfill}%
\pgfsetfillopacity{0.549711}%
\pgfsetlinewidth{1.003750pt}%
\definecolor{currentstroke}{rgb}{0.121569,0.466667,0.705882}%
\pgfsetstrokecolor{currentstroke}%
\pgfsetstrokeopacity{0.549711}%
\pgfsetdash{}{0pt}%
\pgfpathmoveto{\pgfqpoint{1.106685in}{1.966223in}}%
\pgfpathcurveto{\pgfqpoint{1.114921in}{1.966223in}}{\pgfqpoint{1.122821in}{1.969495in}}{\pgfqpoint{1.128645in}{1.975319in}}%
\pgfpathcurveto{\pgfqpoint{1.134469in}{1.981143in}}{\pgfqpoint{1.137741in}{1.989043in}}{\pgfqpoint{1.137741in}{1.997279in}}%
\pgfpathcurveto{\pgfqpoint{1.137741in}{2.005516in}}{\pgfqpoint{1.134469in}{2.013416in}}{\pgfqpoint{1.128645in}{2.019240in}}%
\pgfpathcurveto{\pgfqpoint{1.122821in}{2.025064in}}{\pgfqpoint{1.114921in}{2.028336in}}{\pgfqpoint{1.106685in}{2.028336in}}%
\pgfpathcurveto{\pgfqpoint{1.098448in}{2.028336in}}{\pgfqpoint{1.090548in}{2.025064in}}{\pgfqpoint{1.084724in}{2.019240in}}%
\pgfpathcurveto{\pgfqpoint{1.078900in}{2.013416in}}{\pgfqpoint{1.075628in}{2.005516in}}{\pgfqpoint{1.075628in}{1.997279in}}%
\pgfpathcurveto{\pgfqpoint{1.075628in}{1.989043in}}{\pgfqpoint{1.078900in}{1.981143in}}{\pgfqpoint{1.084724in}{1.975319in}}%
\pgfpathcurveto{\pgfqpoint{1.090548in}{1.969495in}}{\pgfqpoint{1.098448in}{1.966223in}}{\pgfqpoint{1.106685in}{1.966223in}}%
\pgfpathclose%
\pgfusepath{stroke,fill}%
\end{pgfscope}%
\begin{pgfscope}%
\pgfpathrectangle{\pgfqpoint{0.100000in}{0.212622in}}{\pgfqpoint{3.696000in}{3.696000in}}%
\pgfusepath{clip}%
\pgfsetbuttcap%
\pgfsetroundjoin%
\definecolor{currentfill}{rgb}{0.121569,0.466667,0.705882}%
\pgfsetfillcolor{currentfill}%
\pgfsetfillopacity{0.550141}%
\pgfsetlinewidth{1.003750pt}%
\definecolor{currentstroke}{rgb}{0.121569,0.466667,0.705882}%
\pgfsetstrokecolor{currentstroke}%
\pgfsetstrokeopacity{0.550141}%
\pgfsetdash{}{0pt}%
\pgfpathmoveto{\pgfqpoint{1.105188in}{1.963590in}}%
\pgfpathcurveto{\pgfqpoint{1.113425in}{1.963590in}}{\pgfqpoint{1.121325in}{1.966863in}}{\pgfqpoint{1.127149in}{1.972687in}}%
\pgfpathcurveto{\pgfqpoint{1.132972in}{1.978511in}}{\pgfqpoint{1.136245in}{1.986411in}}{\pgfqpoint{1.136245in}{1.994647in}}%
\pgfpathcurveto{\pgfqpoint{1.136245in}{2.002883in}}{\pgfqpoint{1.132972in}{2.010783in}}{\pgfqpoint{1.127149in}{2.016607in}}%
\pgfpathcurveto{\pgfqpoint{1.121325in}{2.022431in}}{\pgfqpoint{1.113425in}{2.025703in}}{\pgfqpoint{1.105188in}{2.025703in}}%
\pgfpathcurveto{\pgfqpoint{1.096952in}{2.025703in}}{\pgfqpoint{1.089052in}{2.022431in}}{\pgfqpoint{1.083228in}{2.016607in}}%
\pgfpathcurveto{\pgfqpoint{1.077404in}{2.010783in}}{\pgfqpoint{1.074132in}{2.002883in}}{\pgfqpoint{1.074132in}{1.994647in}}%
\pgfpathcurveto{\pgfqpoint{1.074132in}{1.986411in}}{\pgfqpoint{1.077404in}{1.978511in}}{\pgfqpoint{1.083228in}{1.972687in}}%
\pgfpathcurveto{\pgfqpoint{1.089052in}{1.966863in}}{\pgfqpoint{1.096952in}{1.963590in}}{\pgfqpoint{1.105188in}{1.963590in}}%
\pgfpathclose%
\pgfusepath{stroke,fill}%
\end{pgfscope}%
\begin{pgfscope}%
\pgfpathrectangle{\pgfqpoint{0.100000in}{0.212622in}}{\pgfqpoint{3.696000in}{3.696000in}}%
\pgfusepath{clip}%
\pgfsetbuttcap%
\pgfsetroundjoin%
\definecolor{currentfill}{rgb}{0.121569,0.466667,0.705882}%
\pgfsetfillcolor{currentfill}%
\pgfsetfillopacity{0.550954}%
\pgfsetlinewidth{1.003750pt}%
\definecolor{currentstroke}{rgb}{0.121569,0.466667,0.705882}%
\pgfsetstrokecolor{currentstroke}%
\pgfsetstrokeopacity{0.550954}%
\pgfsetdash{}{0pt}%
\pgfpathmoveto{\pgfqpoint{1.102927in}{1.958353in}}%
\pgfpathcurveto{\pgfqpoint{1.111163in}{1.958353in}}{\pgfqpoint{1.119063in}{1.961625in}}{\pgfqpoint{1.124887in}{1.967449in}}%
\pgfpathcurveto{\pgfqpoint{1.130711in}{1.973273in}}{\pgfqpoint{1.133983in}{1.981173in}}{\pgfqpoint{1.133983in}{1.989409in}}%
\pgfpathcurveto{\pgfqpoint{1.133983in}{1.997646in}}{\pgfqpoint{1.130711in}{2.005546in}}{\pgfqpoint{1.124887in}{2.011370in}}%
\pgfpathcurveto{\pgfqpoint{1.119063in}{2.017194in}}{\pgfqpoint{1.111163in}{2.020466in}}{\pgfqpoint{1.102927in}{2.020466in}}%
\pgfpathcurveto{\pgfqpoint{1.094691in}{2.020466in}}{\pgfqpoint{1.086790in}{2.017194in}}{\pgfqpoint{1.080967in}{2.011370in}}%
\pgfpathcurveto{\pgfqpoint{1.075143in}{2.005546in}}{\pgfqpoint{1.071870in}{1.997646in}}{\pgfqpoint{1.071870in}{1.989409in}}%
\pgfpathcurveto{\pgfqpoint{1.071870in}{1.981173in}}{\pgfqpoint{1.075143in}{1.973273in}}{\pgfqpoint{1.080967in}{1.967449in}}%
\pgfpathcurveto{\pgfqpoint{1.086790in}{1.961625in}}{\pgfqpoint{1.094691in}{1.958353in}}{\pgfqpoint{1.102927in}{1.958353in}}%
\pgfpathclose%
\pgfusepath{stroke,fill}%
\end{pgfscope}%
\begin{pgfscope}%
\pgfpathrectangle{\pgfqpoint{0.100000in}{0.212622in}}{\pgfqpoint{3.696000in}{3.696000in}}%
\pgfusepath{clip}%
\pgfsetbuttcap%
\pgfsetroundjoin%
\definecolor{currentfill}{rgb}{0.121569,0.466667,0.705882}%
\pgfsetfillcolor{currentfill}%
\pgfsetfillopacity{0.551187}%
\pgfsetlinewidth{1.003750pt}%
\definecolor{currentstroke}{rgb}{0.121569,0.466667,0.705882}%
\pgfsetstrokecolor{currentstroke}%
\pgfsetstrokeopacity{0.551187}%
\pgfsetdash{}{0pt}%
\pgfpathmoveto{\pgfqpoint{1.102055in}{1.956691in}}%
\pgfpathcurveto{\pgfqpoint{1.110291in}{1.956691in}}{\pgfqpoint{1.118191in}{1.959963in}}{\pgfqpoint{1.124015in}{1.965787in}}%
\pgfpathcurveto{\pgfqpoint{1.129839in}{1.971611in}}{\pgfqpoint{1.133111in}{1.979511in}}{\pgfqpoint{1.133111in}{1.987748in}}%
\pgfpathcurveto{\pgfqpoint{1.133111in}{1.995984in}}{\pgfqpoint{1.129839in}{2.003884in}}{\pgfqpoint{1.124015in}{2.009708in}}%
\pgfpathcurveto{\pgfqpoint{1.118191in}{2.015532in}}{\pgfqpoint{1.110291in}{2.018804in}}{\pgfqpoint{1.102055in}{2.018804in}}%
\pgfpathcurveto{\pgfqpoint{1.093818in}{2.018804in}}{\pgfqpoint{1.085918in}{2.015532in}}{\pgfqpoint{1.080094in}{2.009708in}}%
\pgfpathcurveto{\pgfqpoint{1.074270in}{2.003884in}}{\pgfqpoint{1.070998in}{1.995984in}}{\pgfqpoint{1.070998in}{1.987748in}}%
\pgfpathcurveto{\pgfqpoint{1.070998in}{1.979511in}}{\pgfqpoint{1.074270in}{1.971611in}}{\pgfqpoint{1.080094in}{1.965787in}}%
\pgfpathcurveto{\pgfqpoint{1.085918in}{1.959963in}}{\pgfqpoint{1.093818in}{1.956691in}}{\pgfqpoint{1.102055in}{1.956691in}}%
\pgfpathclose%
\pgfusepath{stroke,fill}%
\end{pgfscope}%
\begin{pgfscope}%
\pgfpathrectangle{\pgfqpoint{0.100000in}{0.212622in}}{\pgfqpoint{3.696000in}{3.696000in}}%
\pgfusepath{clip}%
\pgfsetbuttcap%
\pgfsetroundjoin%
\definecolor{currentfill}{rgb}{0.121569,0.466667,0.705882}%
\pgfsetfillcolor{currentfill}%
\pgfsetfillopacity{0.551305}%
\pgfsetlinewidth{1.003750pt}%
\definecolor{currentstroke}{rgb}{0.121569,0.466667,0.705882}%
\pgfsetstrokecolor{currentstroke}%
\pgfsetstrokeopacity{0.551305}%
\pgfsetdash{}{0pt}%
\pgfpathmoveto{\pgfqpoint{1.101664in}{1.955727in}}%
\pgfpathcurveto{\pgfqpoint{1.109900in}{1.955727in}}{\pgfqpoint{1.117800in}{1.959000in}}{\pgfqpoint{1.123624in}{1.964824in}}%
\pgfpathcurveto{\pgfqpoint{1.129448in}{1.970648in}}{\pgfqpoint{1.132720in}{1.978548in}}{\pgfqpoint{1.132720in}{1.986784in}}%
\pgfpathcurveto{\pgfqpoint{1.132720in}{1.995020in}}{\pgfqpoint{1.129448in}{2.002920in}}{\pgfqpoint{1.123624in}{2.008744in}}%
\pgfpathcurveto{\pgfqpoint{1.117800in}{2.014568in}}{\pgfqpoint{1.109900in}{2.017840in}}{\pgfqpoint{1.101664in}{2.017840in}}%
\pgfpathcurveto{\pgfqpoint{1.093428in}{2.017840in}}{\pgfqpoint{1.085528in}{2.014568in}}{\pgfqpoint{1.079704in}{2.008744in}}%
\pgfpathcurveto{\pgfqpoint{1.073880in}{2.002920in}}{\pgfqpoint{1.070607in}{1.995020in}}{\pgfqpoint{1.070607in}{1.986784in}}%
\pgfpathcurveto{\pgfqpoint{1.070607in}{1.978548in}}{\pgfqpoint{1.073880in}{1.970648in}}{\pgfqpoint{1.079704in}{1.964824in}}%
\pgfpathcurveto{\pgfqpoint{1.085528in}{1.959000in}}{\pgfqpoint{1.093428in}{1.955727in}}{\pgfqpoint{1.101664in}{1.955727in}}%
\pgfpathclose%
\pgfusepath{stroke,fill}%
\end{pgfscope}%
\begin{pgfscope}%
\pgfpathrectangle{\pgfqpoint{0.100000in}{0.212622in}}{\pgfqpoint{3.696000in}{3.696000in}}%
\pgfusepath{clip}%
\pgfsetbuttcap%
\pgfsetroundjoin%
\definecolor{currentfill}{rgb}{0.121569,0.466667,0.705882}%
\pgfsetfillcolor{currentfill}%
\pgfsetfillopacity{0.551545}%
\pgfsetlinewidth{1.003750pt}%
\definecolor{currentstroke}{rgb}{0.121569,0.466667,0.705882}%
\pgfsetstrokecolor{currentstroke}%
\pgfsetstrokeopacity{0.551545}%
\pgfsetdash{}{0pt}%
\pgfpathmoveto{\pgfqpoint{1.100851in}{1.954191in}}%
\pgfpathcurveto{\pgfqpoint{1.109088in}{1.954191in}}{\pgfqpoint{1.116988in}{1.957463in}}{\pgfqpoint{1.122812in}{1.963287in}}%
\pgfpathcurveto{\pgfqpoint{1.128636in}{1.969111in}}{\pgfqpoint{1.131908in}{1.977011in}}{\pgfqpoint{1.131908in}{1.985247in}}%
\pgfpathcurveto{\pgfqpoint{1.131908in}{1.993483in}}{\pgfqpoint{1.128636in}{2.001383in}}{\pgfqpoint{1.122812in}{2.007207in}}%
\pgfpathcurveto{\pgfqpoint{1.116988in}{2.013031in}}{\pgfqpoint{1.109088in}{2.016304in}}{\pgfqpoint{1.100851in}{2.016304in}}%
\pgfpathcurveto{\pgfqpoint{1.092615in}{2.016304in}}{\pgfqpoint{1.084715in}{2.013031in}}{\pgfqpoint{1.078891in}{2.007207in}}%
\pgfpathcurveto{\pgfqpoint{1.073067in}{2.001383in}}{\pgfqpoint{1.069795in}{1.993483in}}{\pgfqpoint{1.069795in}{1.985247in}}%
\pgfpathcurveto{\pgfqpoint{1.069795in}{1.977011in}}{\pgfqpoint{1.073067in}{1.969111in}}{\pgfqpoint{1.078891in}{1.963287in}}%
\pgfpathcurveto{\pgfqpoint{1.084715in}{1.957463in}}{\pgfqpoint{1.092615in}{1.954191in}}{\pgfqpoint{1.100851in}{1.954191in}}%
\pgfpathclose%
\pgfusepath{stroke,fill}%
\end{pgfscope}%
\begin{pgfscope}%
\pgfpathrectangle{\pgfqpoint{0.100000in}{0.212622in}}{\pgfqpoint{3.696000in}{3.696000in}}%
\pgfusepath{clip}%
\pgfsetbuttcap%
\pgfsetroundjoin%
\definecolor{currentfill}{rgb}{0.121569,0.466667,0.705882}%
\pgfsetfillcolor{currentfill}%
\pgfsetfillopacity{0.551731}%
\pgfsetlinewidth{1.003750pt}%
\definecolor{currentstroke}{rgb}{0.121569,0.466667,0.705882}%
\pgfsetstrokecolor{currentstroke}%
\pgfsetstrokeopacity{0.551731}%
\pgfsetdash{}{0pt}%
\pgfpathmoveto{\pgfqpoint{1.100463in}{1.952896in}}%
\pgfpathcurveto{\pgfqpoint{1.108700in}{1.952896in}}{\pgfqpoint{1.116600in}{1.956168in}}{\pgfqpoint{1.122424in}{1.961992in}}%
\pgfpathcurveto{\pgfqpoint{1.128248in}{1.967816in}}{\pgfqpoint{1.131520in}{1.975716in}}{\pgfqpoint{1.131520in}{1.983952in}}%
\pgfpathcurveto{\pgfqpoint{1.131520in}{1.992188in}}{\pgfqpoint{1.128248in}{2.000088in}}{\pgfqpoint{1.122424in}{2.005912in}}%
\pgfpathcurveto{\pgfqpoint{1.116600in}{2.011736in}}{\pgfqpoint{1.108700in}{2.015009in}}{\pgfqpoint{1.100463in}{2.015009in}}%
\pgfpathcurveto{\pgfqpoint{1.092227in}{2.015009in}}{\pgfqpoint{1.084327in}{2.011736in}}{\pgfqpoint{1.078503in}{2.005912in}}%
\pgfpathcurveto{\pgfqpoint{1.072679in}{2.000088in}}{\pgfqpoint{1.069407in}{1.992188in}}{\pgfqpoint{1.069407in}{1.983952in}}%
\pgfpathcurveto{\pgfqpoint{1.069407in}{1.975716in}}{\pgfqpoint{1.072679in}{1.967816in}}{\pgfqpoint{1.078503in}{1.961992in}}%
\pgfpathcurveto{\pgfqpoint{1.084327in}{1.956168in}}{\pgfqpoint{1.092227in}{1.952896in}}{\pgfqpoint{1.100463in}{1.952896in}}%
\pgfpathclose%
\pgfusepath{stroke,fill}%
\end{pgfscope}%
\begin{pgfscope}%
\pgfpathrectangle{\pgfqpoint{0.100000in}{0.212622in}}{\pgfqpoint{3.696000in}{3.696000in}}%
\pgfusepath{clip}%
\pgfsetbuttcap%
\pgfsetroundjoin%
\definecolor{currentfill}{rgb}{0.121569,0.466667,0.705882}%
\pgfsetfillcolor{currentfill}%
\pgfsetfillopacity{0.552125}%
\pgfsetlinewidth{1.003750pt}%
\definecolor{currentstroke}{rgb}{0.121569,0.466667,0.705882}%
\pgfsetstrokecolor{currentstroke}%
\pgfsetstrokeopacity{0.552125}%
\pgfsetdash{}{0pt}%
\pgfpathmoveto{\pgfqpoint{1.099368in}{1.951091in}}%
\pgfpathcurveto{\pgfqpoint{1.107604in}{1.951091in}}{\pgfqpoint{1.115505in}{1.954364in}}{\pgfqpoint{1.121328in}{1.960188in}}%
\pgfpathcurveto{\pgfqpoint{1.127152in}{1.966011in}}{\pgfqpoint{1.130425in}{1.973912in}}{\pgfqpoint{1.130425in}{1.982148in}}%
\pgfpathcurveto{\pgfqpoint{1.130425in}{1.990384in}}{\pgfqpoint{1.127152in}{1.998284in}}{\pgfqpoint{1.121328in}{2.004108in}}%
\pgfpathcurveto{\pgfqpoint{1.115505in}{2.009932in}}{\pgfqpoint{1.107604in}{2.013204in}}{\pgfqpoint{1.099368in}{2.013204in}}%
\pgfpathcurveto{\pgfqpoint{1.091132in}{2.013204in}}{\pgfqpoint{1.083232in}{2.009932in}}{\pgfqpoint{1.077408in}{2.004108in}}%
\pgfpathcurveto{\pgfqpoint{1.071584in}{1.998284in}}{\pgfqpoint{1.068312in}{1.990384in}}{\pgfqpoint{1.068312in}{1.982148in}}%
\pgfpathcurveto{\pgfqpoint{1.068312in}{1.973912in}}{\pgfqpoint{1.071584in}{1.966011in}}{\pgfqpoint{1.077408in}{1.960188in}}%
\pgfpathcurveto{\pgfqpoint{1.083232in}{1.954364in}}{\pgfqpoint{1.091132in}{1.951091in}}{\pgfqpoint{1.099368in}{1.951091in}}%
\pgfpathclose%
\pgfusepath{stroke,fill}%
\end{pgfscope}%
\begin{pgfscope}%
\pgfpathrectangle{\pgfqpoint{0.100000in}{0.212622in}}{\pgfqpoint{3.696000in}{3.696000in}}%
\pgfusepath{clip}%
\pgfsetbuttcap%
\pgfsetroundjoin%
\definecolor{currentfill}{rgb}{0.121569,0.466667,0.705882}%
\pgfsetfillcolor{currentfill}%
\pgfsetfillopacity{0.552565}%
\pgfsetlinewidth{1.003750pt}%
\definecolor{currentstroke}{rgb}{0.121569,0.466667,0.705882}%
\pgfsetstrokecolor{currentstroke}%
\pgfsetstrokeopacity{0.552565}%
\pgfsetdash{}{0pt}%
\pgfpathmoveto{\pgfqpoint{2.068941in}{2.279255in}}%
\pgfpathcurveto{\pgfqpoint{2.077177in}{2.279255in}}{\pgfqpoint{2.085077in}{2.282527in}}{\pgfqpoint{2.090901in}{2.288351in}}%
\pgfpathcurveto{\pgfqpoint{2.096725in}{2.294175in}}{\pgfqpoint{2.099997in}{2.302075in}}{\pgfqpoint{2.099997in}{2.310311in}}%
\pgfpathcurveto{\pgfqpoint{2.099997in}{2.318548in}}{\pgfqpoint{2.096725in}{2.326448in}}{\pgfqpoint{2.090901in}{2.332272in}}%
\pgfpathcurveto{\pgfqpoint{2.085077in}{2.338096in}}{\pgfqpoint{2.077177in}{2.341368in}}{\pgfqpoint{2.068941in}{2.341368in}}%
\pgfpathcurveto{\pgfqpoint{2.060705in}{2.341368in}}{\pgfqpoint{2.052804in}{2.338096in}}{\pgfqpoint{2.046981in}{2.332272in}}%
\pgfpathcurveto{\pgfqpoint{2.041157in}{2.326448in}}{\pgfqpoint{2.037884in}{2.318548in}}{\pgfqpoint{2.037884in}{2.310311in}}%
\pgfpathcurveto{\pgfqpoint{2.037884in}{2.302075in}}{\pgfqpoint{2.041157in}{2.294175in}}{\pgfqpoint{2.046981in}{2.288351in}}%
\pgfpathcurveto{\pgfqpoint{2.052804in}{2.282527in}}{\pgfqpoint{2.060705in}{2.279255in}}{\pgfqpoint{2.068941in}{2.279255in}}%
\pgfpathclose%
\pgfusepath{stroke,fill}%
\end{pgfscope}%
\begin{pgfscope}%
\pgfpathrectangle{\pgfqpoint{0.100000in}{0.212622in}}{\pgfqpoint{3.696000in}{3.696000in}}%
\pgfusepath{clip}%
\pgfsetbuttcap%
\pgfsetroundjoin%
\definecolor{currentfill}{rgb}{0.121569,0.466667,0.705882}%
\pgfsetfillcolor{currentfill}%
\pgfsetfillopacity{0.552860}%
\pgfsetlinewidth{1.003750pt}%
\definecolor{currentstroke}{rgb}{0.121569,0.466667,0.705882}%
\pgfsetstrokecolor{currentstroke}%
\pgfsetstrokeopacity{0.552860}%
\pgfsetdash{}{0pt}%
\pgfpathmoveto{\pgfqpoint{1.097531in}{1.947691in}}%
\pgfpathcurveto{\pgfqpoint{1.105767in}{1.947691in}}{\pgfqpoint{1.113667in}{1.950963in}}{\pgfqpoint{1.119491in}{1.956787in}}%
\pgfpathcurveto{\pgfqpoint{1.125315in}{1.962611in}}{\pgfqpoint{1.128587in}{1.970511in}}{\pgfqpoint{1.128587in}{1.978747in}}%
\pgfpathcurveto{\pgfqpoint{1.128587in}{1.986983in}}{\pgfqpoint{1.125315in}{1.994883in}}{\pgfqpoint{1.119491in}{2.000707in}}%
\pgfpathcurveto{\pgfqpoint{1.113667in}{2.006531in}}{\pgfqpoint{1.105767in}{2.009804in}}{\pgfqpoint{1.097531in}{2.009804in}}%
\pgfpathcurveto{\pgfqpoint{1.089295in}{2.009804in}}{\pgfqpoint{1.081395in}{2.006531in}}{\pgfqpoint{1.075571in}{2.000707in}}%
\pgfpathcurveto{\pgfqpoint{1.069747in}{1.994883in}}{\pgfqpoint{1.066474in}{1.986983in}}{\pgfqpoint{1.066474in}{1.978747in}}%
\pgfpathcurveto{\pgfqpoint{1.066474in}{1.970511in}}{\pgfqpoint{1.069747in}{1.962611in}}{\pgfqpoint{1.075571in}{1.956787in}}%
\pgfpathcurveto{\pgfqpoint{1.081395in}{1.950963in}}{\pgfqpoint{1.089295in}{1.947691in}}{\pgfqpoint{1.097531in}{1.947691in}}%
\pgfpathclose%
\pgfusepath{stroke,fill}%
\end{pgfscope}%
\begin{pgfscope}%
\pgfpathrectangle{\pgfqpoint{0.100000in}{0.212622in}}{\pgfqpoint{3.696000in}{3.696000in}}%
\pgfusepath{clip}%
\pgfsetbuttcap%
\pgfsetroundjoin%
\definecolor{currentfill}{rgb}{0.121569,0.466667,0.705882}%
\pgfsetfillcolor{currentfill}%
\pgfsetfillopacity{0.553464}%
\pgfsetlinewidth{1.003750pt}%
\definecolor{currentstroke}{rgb}{0.121569,0.466667,0.705882}%
\pgfsetstrokecolor{currentstroke}%
\pgfsetstrokeopacity{0.553464}%
\pgfsetdash{}{0pt}%
\pgfpathmoveto{\pgfqpoint{1.095924in}{1.944861in}}%
\pgfpathcurveto{\pgfqpoint{1.104160in}{1.944861in}}{\pgfqpoint{1.112060in}{1.948133in}}{\pgfqpoint{1.117884in}{1.953957in}}%
\pgfpathcurveto{\pgfqpoint{1.123708in}{1.959781in}}{\pgfqpoint{1.126981in}{1.967681in}}{\pgfqpoint{1.126981in}{1.975917in}}%
\pgfpathcurveto{\pgfqpoint{1.126981in}{1.984153in}}{\pgfqpoint{1.123708in}{1.992054in}}{\pgfqpoint{1.117884in}{1.997877in}}%
\pgfpathcurveto{\pgfqpoint{1.112060in}{2.003701in}}{\pgfqpoint{1.104160in}{2.006974in}}{\pgfqpoint{1.095924in}{2.006974in}}%
\pgfpathcurveto{\pgfqpoint{1.087688in}{2.006974in}}{\pgfqpoint{1.079788in}{2.003701in}}{\pgfqpoint{1.073964in}{1.997877in}}%
\pgfpathcurveto{\pgfqpoint{1.068140in}{1.992054in}}{\pgfqpoint{1.064868in}{1.984153in}}{\pgfqpoint{1.064868in}{1.975917in}}%
\pgfpathcurveto{\pgfqpoint{1.064868in}{1.967681in}}{\pgfqpoint{1.068140in}{1.959781in}}{\pgfqpoint{1.073964in}{1.953957in}}%
\pgfpathcurveto{\pgfqpoint{1.079788in}{1.948133in}}{\pgfqpoint{1.087688in}{1.944861in}}{\pgfqpoint{1.095924in}{1.944861in}}%
\pgfpathclose%
\pgfusepath{stroke,fill}%
\end{pgfscope}%
\begin{pgfscope}%
\pgfpathrectangle{\pgfqpoint{0.100000in}{0.212622in}}{\pgfqpoint{3.696000in}{3.696000in}}%
\pgfusepath{clip}%
\pgfsetbuttcap%
\pgfsetroundjoin%
\definecolor{currentfill}{rgb}{0.121569,0.466667,0.705882}%
\pgfsetfillcolor{currentfill}%
\pgfsetfillopacity{0.554584}%
\pgfsetlinewidth{1.003750pt}%
\definecolor{currentstroke}{rgb}{0.121569,0.466667,0.705882}%
\pgfsetstrokecolor{currentstroke}%
\pgfsetstrokeopacity{0.554584}%
\pgfsetdash{}{0pt}%
\pgfpathmoveto{\pgfqpoint{1.093040in}{1.939745in}}%
\pgfpathcurveto{\pgfqpoint{1.101276in}{1.939745in}}{\pgfqpoint{1.109176in}{1.943017in}}{\pgfqpoint{1.115000in}{1.948841in}}%
\pgfpathcurveto{\pgfqpoint{1.120824in}{1.954665in}}{\pgfqpoint{1.124097in}{1.962565in}}{\pgfqpoint{1.124097in}{1.970801in}}%
\pgfpathcurveto{\pgfqpoint{1.124097in}{1.979037in}}{\pgfqpoint{1.120824in}{1.986937in}}{\pgfqpoint{1.115000in}{1.992761in}}%
\pgfpathcurveto{\pgfqpoint{1.109176in}{1.998585in}}{\pgfqpoint{1.101276in}{2.001857in}}{\pgfqpoint{1.093040in}{2.001857in}}%
\pgfpathcurveto{\pgfqpoint{1.084804in}{2.001857in}}{\pgfqpoint{1.076904in}{1.998585in}}{\pgfqpoint{1.071080in}{1.992761in}}%
\pgfpathcurveto{\pgfqpoint{1.065256in}{1.986937in}}{\pgfqpoint{1.061984in}{1.979037in}}{\pgfqpoint{1.061984in}{1.970801in}}%
\pgfpathcurveto{\pgfqpoint{1.061984in}{1.962565in}}{\pgfqpoint{1.065256in}{1.954665in}}{\pgfqpoint{1.071080in}{1.948841in}}%
\pgfpathcurveto{\pgfqpoint{1.076904in}{1.943017in}}{\pgfqpoint{1.084804in}{1.939745in}}{\pgfqpoint{1.093040in}{1.939745in}}%
\pgfpathclose%
\pgfusepath{stroke,fill}%
\end{pgfscope}%
\begin{pgfscope}%
\pgfpathrectangle{\pgfqpoint{0.100000in}{0.212622in}}{\pgfqpoint{3.696000in}{3.696000in}}%
\pgfusepath{clip}%
\pgfsetbuttcap%
\pgfsetroundjoin%
\definecolor{currentfill}{rgb}{0.121569,0.466667,0.705882}%
\pgfsetfillcolor{currentfill}%
\pgfsetfillopacity{0.555545}%
\pgfsetlinewidth{1.003750pt}%
\definecolor{currentstroke}{rgb}{0.121569,0.466667,0.705882}%
\pgfsetstrokecolor{currentstroke}%
\pgfsetstrokeopacity{0.555545}%
\pgfsetdash{}{0pt}%
\pgfpathmoveto{\pgfqpoint{1.090521in}{1.934920in}}%
\pgfpathcurveto{\pgfqpoint{1.098758in}{1.934920in}}{\pgfqpoint{1.106658in}{1.938192in}}{\pgfqpoint{1.112482in}{1.944016in}}%
\pgfpathcurveto{\pgfqpoint{1.118306in}{1.949840in}}{\pgfqpoint{1.121578in}{1.957740in}}{\pgfqpoint{1.121578in}{1.965977in}}%
\pgfpathcurveto{\pgfqpoint{1.121578in}{1.974213in}}{\pgfqpoint{1.118306in}{1.982113in}}{\pgfqpoint{1.112482in}{1.987937in}}%
\pgfpathcurveto{\pgfqpoint{1.106658in}{1.993761in}}{\pgfqpoint{1.098758in}{1.997033in}}{\pgfqpoint{1.090521in}{1.997033in}}%
\pgfpathcurveto{\pgfqpoint{1.082285in}{1.997033in}}{\pgfqpoint{1.074385in}{1.993761in}}{\pgfqpoint{1.068561in}{1.987937in}}%
\pgfpathcurveto{\pgfqpoint{1.062737in}{1.982113in}}{\pgfqpoint{1.059465in}{1.974213in}}{\pgfqpoint{1.059465in}{1.965977in}}%
\pgfpathcurveto{\pgfqpoint{1.059465in}{1.957740in}}{\pgfqpoint{1.062737in}{1.949840in}}{\pgfqpoint{1.068561in}{1.944016in}}%
\pgfpathcurveto{\pgfqpoint{1.074385in}{1.938192in}}{\pgfqpoint{1.082285in}{1.934920in}}{\pgfqpoint{1.090521in}{1.934920in}}%
\pgfpathclose%
\pgfusepath{stroke,fill}%
\end{pgfscope}%
\begin{pgfscope}%
\pgfpathrectangle{\pgfqpoint{0.100000in}{0.212622in}}{\pgfqpoint{3.696000in}{3.696000in}}%
\pgfusepath{clip}%
\pgfsetbuttcap%
\pgfsetroundjoin%
\definecolor{currentfill}{rgb}{0.121569,0.466667,0.705882}%
\pgfsetfillcolor{currentfill}%
\pgfsetfillopacity{0.557240}%
\pgfsetlinewidth{1.003750pt}%
\definecolor{currentstroke}{rgb}{0.121569,0.466667,0.705882}%
\pgfsetstrokecolor{currentstroke}%
\pgfsetstrokeopacity{0.557240}%
\pgfsetdash{}{0pt}%
\pgfpathmoveto{\pgfqpoint{1.085465in}{1.926558in}}%
\pgfpathcurveto{\pgfqpoint{1.093701in}{1.926558in}}{\pgfqpoint{1.101601in}{1.929831in}}{\pgfqpoint{1.107425in}{1.935655in}}%
\pgfpathcurveto{\pgfqpoint{1.113249in}{1.941479in}}{\pgfqpoint{1.116521in}{1.949379in}}{\pgfqpoint{1.116521in}{1.957615in}}%
\pgfpathcurveto{\pgfqpoint{1.116521in}{1.965851in}}{\pgfqpoint{1.113249in}{1.973751in}}{\pgfqpoint{1.107425in}{1.979575in}}%
\pgfpathcurveto{\pgfqpoint{1.101601in}{1.985399in}}{\pgfqpoint{1.093701in}{1.988671in}}{\pgfqpoint{1.085465in}{1.988671in}}%
\pgfpathcurveto{\pgfqpoint{1.077229in}{1.988671in}}{\pgfqpoint{1.069329in}{1.985399in}}{\pgfqpoint{1.063505in}{1.979575in}}%
\pgfpathcurveto{\pgfqpoint{1.057681in}{1.973751in}}{\pgfqpoint{1.054408in}{1.965851in}}{\pgfqpoint{1.054408in}{1.957615in}}%
\pgfpathcurveto{\pgfqpoint{1.054408in}{1.949379in}}{\pgfqpoint{1.057681in}{1.941479in}}{\pgfqpoint{1.063505in}{1.935655in}}%
\pgfpathcurveto{\pgfqpoint{1.069329in}{1.929831in}}{\pgfqpoint{1.077229in}{1.926558in}}{\pgfqpoint{1.085465in}{1.926558in}}%
\pgfpathclose%
\pgfusepath{stroke,fill}%
\end{pgfscope}%
\begin{pgfscope}%
\pgfpathrectangle{\pgfqpoint{0.100000in}{0.212622in}}{\pgfqpoint{3.696000in}{3.696000in}}%
\pgfusepath{clip}%
\pgfsetbuttcap%
\pgfsetroundjoin%
\definecolor{currentfill}{rgb}{0.121569,0.466667,0.705882}%
\pgfsetfillcolor{currentfill}%
\pgfsetfillopacity{0.557442}%
\pgfsetlinewidth{1.003750pt}%
\definecolor{currentstroke}{rgb}{0.121569,0.466667,0.705882}%
\pgfsetstrokecolor{currentstroke}%
\pgfsetstrokeopacity{0.557442}%
\pgfsetdash{}{0pt}%
\pgfpathmoveto{\pgfqpoint{2.072334in}{2.262802in}}%
\pgfpathcurveto{\pgfqpoint{2.080570in}{2.262802in}}{\pgfqpoint{2.088470in}{2.266075in}}{\pgfqpoint{2.094294in}{2.271899in}}%
\pgfpathcurveto{\pgfqpoint{2.100118in}{2.277722in}}{\pgfqpoint{2.103390in}{2.285623in}}{\pgfqpoint{2.103390in}{2.293859in}}%
\pgfpathcurveto{\pgfqpoint{2.103390in}{2.302095in}}{\pgfqpoint{2.100118in}{2.309995in}}{\pgfqpoint{2.094294in}{2.315819in}}%
\pgfpathcurveto{\pgfqpoint{2.088470in}{2.321643in}}{\pgfqpoint{2.080570in}{2.324915in}}{\pgfqpoint{2.072334in}{2.324915in}}%
\pgfpathcurveto{\pgfqpoint{2.064098in}{2.324915in}}{\pgfqpoint{2.056198in}{2.321643in}}{\pgfqpoint{2.050374in}{2.315819in}}%
\pgfpathcurveto{\pgfqpoint{2.044550in}{2.309995in}}{\pgfqpoint{2.041277in}{2.302095in}}{\pgfqpoint{2.041277in}{2.293859in}}%
\pgfpathcurveto{\pgfqpoint{2.041277in}{2.285623in}}{\pgfqpoint{2.044550in}{2.277722in}}{\pgfqpoint{2.050374in}{2.271899in}}%
\pgfpathcurveto{\pgfqpoint{2.056198in}{2.266075in}}{\pgfqpoint{2.064098in}{2.262802in}}{\pgfqpoint{2.072334in}{2.262802in}}%
\pgfpathclose%
\pgfusepath{stroke,fill}%
\end{pgfscope}%
\begin{pgfscope}%
\pgfpathrectangle{\pgfqpoint{0.100000in}{0.212622in}}{\pgfqpoint{3.696000in}{3.696000in}}%
\pgfusepath{clip}%
\pgfsetbuttcap%
\pgfsetroundjoin%
\definecolor{currentfill}{rgb}{0.121569,0.466667,0.705882}%
\pgfsetfillcolor{currentfill}%
\pgfsetfillopacity{0.558826}%
\pgfsetlinewidth{1.003750pt}%
\definecolor{currentstroke}{rgb}{0.121569,0.466667,0.705882}%
\pgfsetstrokecolor{currentstroke}%
\pgfsetstrokeopacity{0.558826}%
\pgfsetdash{}{0pt}%
\pgfpathmoveto{\pgfqpoint{1.081035in}{1.918458in}}%
\pgfpathcurveto{\pgfqpoint{1.089271in}{1.918458in}}{\pgfqpoint{1.097171in}{1.921730in}}{\pgfqpoint{1.102995in}{1.927554in}}%
\pgfpathcurveto{\pgfqpoint{1.108819in}{1.933378in}}{\pgfqpoint{1.112092in}{1.941278in}}{\pgfqpoint{1.112092in}{1.949514in}}%
\pgfpathcurveto{\pgfqpoint{1.112092in}{1.957750in}}{\pgfqpoint{1.108819in}{1.965650in}}{\pgfqpoint{1.102995in}{1.971474in}}%
\pgfpathcurveto{\pgfqpoint{1.097171in}{1.977298in}}{\pgfqpoint{1.089271in}{1.980571in}}{\pgfqpoint{1.081035in}{1.980571in}}%
\pgfpathcurveto{\pgfqpoint{1.072799in}{1.980571in}}{\pgfqpoint{1.064899in}{1.977298in}}{\pgfqpoint{1.059075in}{1.971474in}}%
\pgfpathcurveto{\pgfqpoint{1.053251in}{1.965650in}}{\pgfqpoint{1.049979in}{1.957750in}}{\pgfqpoint{1.049979in}{1.949514in}}%
\pgfpathcurveto{\pgfqpoint{1.049979in}{1.941278in}}{\pgfqpoint{1.053251in}{1.933378in}}{\pgfqpoint{1.059075in}{1.927554in}}%
\pgfpathcurveto{\pgfqpoint{1.064899in}{1.921730in}}{\pgfqpoint{1.072799in}{1.918458in}}{\pgfqpoint{1.081035in}{1.918458in}}%
\pgfpathclose%
\pgfusepath{stroke,fill}%
\end{pgfscope}%
\begin{pgfscope}%
\pgfpathrectangle{\pgfqpoint{0.100000in}{0.212622in}}{\pgfqpoint{3.696000in}{3.696000in}}%
\pgfusepath{clip}%
\pgfsetbuttcap%
\pgfsetroundjoin%
\definecolor{currentfill}{rgb}{0.121569,0.466667,0.705882}%
\pgfsetfillcolor{currentfill}%
\pgfsetfillopacity{0.560093}%
\pgfsetlinewidth{1.003750pt}%
\definecolor{currentstroke}{rgb}{0.121569,0.466667,0.705882}%
\pgfsetstrokecolor{currentstroke}%
\pgfsetstrokeopacity{0.560093}%
\pgfsetdash{}{0pt}%
\pgfpathmoveto{\pgfqpoint{1.076908in}{1.911397in}}%
\pgfpathcurveto{\pgfqpoint{1.085144in}{1.911397in}}{\pgfqpoint{1.093044in}{1.914670in}}{\pgfqpoint{1.098868in}{1.920494in}}%
\pgfpathcurveto{\pgfqpoint{1.104692in}{1.926318in}}{\pgfqpoint{1.107964in}{1.934218in}}{\pgfqpoint{1.107964in}{1.942454in}}%
\pgfpathcurveto{\pgfqpoint{1.107964in}{1.950690in}}{\pgfqpoint{1.104692in}{1.958590in}}{\pgfqpoint{1.098868in}{1.964414in}}%
\pgfpathcurveto{\pgfqpoint{1.093044in}{1.970238in}}{\pgfqpoint{1.085144in}{1.973510in}}{\pgfqpoint{1.076908in}{1.973510in}}%
\pgfpathcurveto{\pgfqpoint{1.068671in}{1.973510in}}{\pgfqpoint{1.060771in}{1.970238in}}{\pgfqpoint{1.054947in}{1.964414in}}%
\pgfpathcurveto{\pgfqpoint{1.049124in}{1.958590in}}{\pgfqpoint{1.045851in}{1.950690in}}{\pgfqpoint{1.045851in}{1.942454in}}%
\pgfpathcurveto{\pgfqpoint{1.045851in}{1.934218in}}{\pgfqpoint{1.049124in}{1.926318in}}{\pgfqpoint{1.054947in}{1.920494in}}%
\pgfpathcurveto{\pgfqpoint{1.060771in}{1.914670in}}{\pgfqpoint{1.068671in}{1.911397in}}{\pgfqpoint{1.076908in}{1.911397in}}%
\pgfpathclose%
\pgfusepath{stroke,fill}%
\end{pgfscope}%
\begin{pgfscope}%
\pgfpathrectangle{\pgfqpoint{0.100000in}{0.212622in}}{\pgfqpoint{3.696000in}{3.696000in}}%
\pgfusepath{clip}%
\pgfsetbuttcap%
\pgfsetroundjoin%
\definecolor{currentfill}{rgb}{0.121569,0.466667,0.705882}%
\pgfsetfillcolor{currentfill}%
\pgfsetfillopacity{0.561259}%
\pgfsetlinewidth{1.003750pt}%
\definecolor{currentstroke}{rgb}{0.121569,0.466667,0.705882}%
\pgfsetstrokecolor{currentstroke}%
\pgfsetstrokeopacity{0.561259}%
\pgfsetdash{}{0pt}%
\pgfpathmoveto{\pgfqpoint{1.073681in}{1.904844in}}%
\pgfpathcurveto{\pgfqpoint{1.081917in}{1.904844in}}{\pgfqpoint{1.089817in}{1.908116in}}{\pgfqpoint{1.095641in}{1.913940in}}%
\pgfpathcurveto{\pgfqpoint{1.101465in}{1.919764in}}{\pgfqpoint{1.104737in}{1.927664in}}{\pgfqpoint{1.104737in}{1.935900in}}%
\pgfpathcurveto{\pgfqpoint{1.104737in}{1.944136in}}{\pgfqpoint{1.101465in}{1.952036in}}{\pgfqpoint{1.095641in}{1.957860in}}%
\pgfpathcurveto{\pgfqpoint{1.089817in}{1.963684in}}{\pgfqpoint{1.081917in}{1.966957in}}{\pgfqpoint{1.073681in}{1.966957in}}%
\pgfpathcurveto{\pgfqpoint{1.065444in}{1.966957in}}{\pgfqpoint{1.057544in}{1.963684in}}{\pgfqpoint{1.051721in}{1.957860in}}%
\pgfpathcurveto{\pgfqpoint{1.045897in}{1.952036in}}{\pgfqpoint{1.042624in}{1.944136in}}{\pgfqpoint{1.042624in}{1.935900in}}%
\pgfpathcurveto{\pgfqpoint{1.042624in}{1.927664in}}{\pgfqpoint{1.045897in}{1.919764in}}{\pgfqpoint{1.051721in}{1.913940in}}%
\pgfpathcurveto{\pgfqpoint{1.057544in}{1.908116in}}{\pgfqpoint{1.065444in}{1.904844in}}{\pgfqpoint{1.073681in}{1.904844in}}%
\pgfpathclose%
\pgfusepath{stroke,fill}%
\end{pgfscope}%
\begin{pgfscope}%
\pgfpathrectangle{\pgfqpoint{0.100000in}{0.212622in}}{\pgfqpoint{3.696000in}{3.696000in}}%
\pgfusepath{clip}%
\pgfsetbuttcap%
\pgfsetroundjoin%
\definecolor{currentfill}{rgb}{0.121569,0.466667,0.705882}%
\pgfsetfillcolor{currentfill}%
\pgfsetfillopacity{0.561774}%
\pgfsetlinewidth{1.003750pt}%
\definecolor{currentstroke}{rgb}{0.121569,0.466667,0.705882}%
\pgfsetstrokecolor{currentstroke}%
\pgfsetstrokeopacity{0.561774}%
\pgfsetdash{}{0pt}%
\pgfpathmoveto{\pgfqpoint{1.071781in}{1.901337in}}%
\pgfpathcurveto{\pgfqpoint{1.080017in}{1.901337in}}{\pgfqpoint{1.087917in}{1.904609in}}{\pgfqpoint{1.093741in}{1.910433in}}%
\pgfpathcurveto{\pgfqpoint{1.099565in}{1.916257in}}{\pgfqpoint{1.102837in}{1.924157in}}{\pgfqpoint{1.102837in}{1.932393in}}%
\pgfpathcurveto{\pgfqpoint{1.102837in}{1.940629in}}{\pgfqpoint{1.099565in}{1.948529in}}{\pgfqpoint{1.093741in}{1.954353in}}%
\pgfpathcurveto{\pgfqpoint{1.087917in}{1.960177in}}{\pgfqpoint{1.080017in}{1.963450in}}{\pgfqpoint{1.071781in}{1.963450in}}%
\pgfpathcurveto{\pgfqpoint{1.063544in}{1.963450in}}{\pgfqpoint{1.055644in}{1.960177in}}{\pgfqpoint{1.049820in}{1.954353in}}%
\pgfpathcurveto{\pgfqpoint{1.043997in}{1.948529in}}{\pgfqpoint{1.040724in}{1.940629in}}{\pgfqpoint{1.040724in}{1.932393in}}%
\pgfpathcurveto{\pgfqpoint{1.040724in}{1.924157in}}{\pgfqpoint{1.043997in}{1.916257in}}{\pgfqpoint{1.049820in}{1.910433in}}%
\pgfpathcurveto{\pgfqpoint{1.055644in}{1.904609in}}{\pgfqpoint{1.063544in}{1.901337in}}{\pgfqpoint{1.071781in}{1.901337in}}%
\pgfpathclose%
\pgfusepath{stroke,fill}%
\end{pgfscope}%
\begin{pgfscope}%
\pgfpathrectangle{\pgfqpoint{0.100000in}{0.212622in}}{\pgfqpoint{3.696000in}{3.696000in}}%
\pgfusepath{clip}%
\pgfsetbuttcap%
\pgfsetroundjoin%
\definecolor{currentfill}{rgb}{0.121569,0.466667,0.705882}%
\pgfsetfillcolor{currentfill}%
\pgfsetfillopacity{0.562122}%
\pgfsetlinewidth{1.003750pt}%
\definecolor{currentstroke}{rgb}{0.121569,0.466667,0.705882}%
\pgfsetstrokecolor{currentstroke}%
\pgfsetstrokeopacity{0.562122}%
\pgfsetdash{}{0pt}%
\pgfpathmoveto{\pgfqpoint{2.076365in}{2.244236in}}%
\pgfpathcurveto{\pgfqpoint{2.084601in}{2.244236in}}{\pgfqpoint{2.092501in}{2.247508in}}{\pgfqpoint{2.098325in}{2.253332in}}%
\pgfpathcurveto{\pgfqpoint{2.104149in}{2.259156in}}{\pgfqpoint{2.107421in}{2.267056in}}{\pgfqpoint{2.107421in}{2.275292in}}%
\pgfpathcurveto{\pgfqpoint{2.107421in}{2.283528in}}{\pgfqpoint{2.104149in}{2.291428in}}{\pgfqpoint{2.098325in}{2.297252in}}%
\pgfpathcurveto{\pgfqpoint{2.092501in}{2.303076in}}{\pgfqpoint{2.084601in}{2.306349in}}{\pgfqpoint{2.076365in}{2.306349in}}%
\pgfpathcurveto{\pgfqpoint{2.068128in}{2.306349in}}{\pgfqpoint{2.060228in}{2.303076in}}{\pgfqpoint{2.054404in}{2.297252in}}%
\pgfpathcurveto{\pgfqpoint{2.048581in}{2.291428in}}{\pgfqpoint{2.045308in}{2.283528in}}{\pgfqpoint{2.045308in}{2.275292in}}%
\pgfpathcurveto{\pgfqpoint{2.045308in}{2.267056in}}{\pgfqpoint{2.048581in}{2.259156in}}{\pgfqpoint{2.054404in}{2.253332in}}%
\pgfpathcurveto{\pgfqpoint{2.060228in}{2.247508in}}{\pgfqpoint{2.068128in}{2.244236in}}{\pgfqpoint{2.076365in}{2.244236in}}%
\pgfpathclose%
\pgfusepath{stroke,fill}%
\end{pgfscope}%
\begin{pgfscope}%
\pgfpathrectangle{\pgfqpoint{0.100000in}{0.212622in}}{\pgfqpoint{3.696000in}{3.696000in}}%
\pgfusepath{clip}%
\pgfsetbuttcap%
\pgfsetroundjoin%
\definecolor{currentfill}{rgb}{0.121569,0.466667,0.705882}%
\pgfsetfillcolor{currentfill}%
\pgfsetfillopacity{0.562737}%
\pgfsetlinewidth{1.003750pt}%
\definecolor{currentstroke}{rgb}{0.121569,0.466667,0.705882}%
\pgfsetstrokecolor{currentstroke}%
\pgfsetstrokeopacity{0.562737}%
\pgfsetdash{}{0pt}%
\pgfpathmoveto{\pgfqpoint{1.069010in}{1.894249in}}%
\pgfpathcurveto{\pgfqpoint{1.077246in}{1.894249in}}{\pgfqpoint{1.085146in}{1.897521in}}{\pgfqpoint{1.090970in}{1.903345in}}%
\pgfpathcurveto{\pgfqpoint{1.096794in}{1.909169in}}{\pgfqpoint{1.100066in}{1.917069in}}{\pgfqpoint{1.100066in}{1.925305in}}%
\pgfpathcurveto{\pgfqpoint{1.100066in}{1.933542in}}{\pgfqpoint{1.096794in}{1.941442in}}{\pgfqpoint{1.090970in}{1.947266in}}%
\pgfpathcurveto{\pgfqpoint{1.085146in}{1.953090in}}{\pgfqpoint{1.077246in}{1.956362in}}{\pgfqpoint{1.069010in}{1.956362in}}%
\pgfpathcurveto{\pgfqpoint{1.060773in}{1.956362in}}{\pgfqpoint{1.052873in}{1.953090in}}{\pgfqpoint{1.047049in}{1.947266in}}%
\pgfpathcurveto{\pgfqpoint{1.041225in}{1.941442in}}{\pgfqpoint{1.037953in}{1.933542in}}{\pgfqpoint{1.037953in}{1.925305in}}%
\pgfpathcurveto{\pgfqpoint{1.037953in}{1.917069in}}{\pgfqpoint{1.041225in}{1.909169in}}{\pgfqpoint{1.047049in}{1.903345in}}%
\pgfpathcurveto{\pgfqpoint{1.052873in}{1.897521in}}{\pgfqpoint{1.060773in}{1.894249in}}{\pgfqpoint{1.069010in}{1.894249in}}%
\pgfpathclose%
\pgfusepath{stroke,fill}%
\end{pgfscope}%
\begin{pgfscope}%
\pgfpathrectangle{\pgfqpoint{0.100000in}{0.212622in}}{\pgfqpoint{3.696000in}{3.696000in}}%
\pgfusepath{clip}%
\pgfsetbuttcap%
\pgfsetroundjoin%
\definecolor{currentfill}{rgb}{0.121569,0.466667,0.705882}%
\pgfsetfillcolor{currentfill}%
\pgfsetfillopacity{0.563359}%
\pgfsetlinewidth{1.003750pt}%
\definecolor{currentstroke}{rgb}{0.121569,0.466667,0.705882}%
\pgfsetstrokecolor{currentstroke}%
\pgfsetstrokeopacity{0.563359}%
\pgfsetdash{}{0pt}%
\pgfpathmoveto{\pgfqpoint{1.066874in}{1.890352in}}%
\pgfpathcurveto{\pgfqpoint{1.075110in}{1.890352in}}{\pgfqpoint{1.083010in}{1.893624in}}{\pgfqpoint{1.088834in}{1.899448in}}%
\pgfpathcurveto{\pgfqpoint{1.094658in}{1.905272in}}{\pgfqpoint{1.097930in}{1.913172in}}{\pgfqpoint{1.097930in}{1.921408in}}%
\pgfpathcurveto{\pgfqpoint{1.097930in}{1.929645in}}{\pgfqpoint{1.094658in}{1.937545in}}{\pgfqpoint{1.088834in}{1.943369in}}%
\pgfpathcurveto{\pgfqpoint{1.083010in}{1.949192in}}{\pgfqpoint{1.075110in}{1.952465in}}{\pgfqpoint{1.066874in}{1.952465in}}%
\pgfpathcurveto{\pgfqpoint{1.058637in}{1.952465in}}{\pgfqpoint{1.050737in}{1.949192in}}{\pgfqpoint{1.044913in}{1.943369in}}%
\pgfpathcurveto{\pgfqpoint{1.039089in}{1.937545in}}{\pgfqpoint{1.035817in}{1.929645in}}{\pgfqpoint{1.035817in}{1.921408in}}%
\pgfpathcurveto{\pgfqpoint{1.035817in}{1.913172in}}{\pgfqpoint{1.039089in}{1.905272in}}{\pgfqpoint{1.044913in}{1.899448in}}%
\pgfpathcurveto{\pgfqpoint{1.050737in}{1.893624in}}{\pgfqpoint{1.058637in}{1.890352in}}{\pgfqpoint{1.066874in}{1.890352in}}%
\pgfpathclose%
\pgfusepath{stroke,fill}%
\end{pgfscope}%
\begin{pgfscope}%
\pgfpathrectangle{\pgfqpoint{0.100000in}{0.212622in}}{\pgfqpoint{3.696000in}{3.696000in}}%
\pgfusepath{clip}%
\pgfsetbuttcap%
\pgfsetroundjoin%
\definecolor{currentfill}{rgb}{0.121569,0.466667,0.705882}%
\pgfsetfillcolor{currentfill}%
\pgfsetfillopacity{0.564500}%
\pgfsetlinewidth{1.003750pt}%
\definecolor{currentstroke}{rgb}{0.121569,0.466667,0.705882}%
\pgfsetstrokecolor{currentstroke}%
\pgfsetstrokeopacity{0.564500}%
\pgfsetdash{}{0pt}%
\pgfpathmoveto{\pgfqpoint{1.063641in}{1.882533in}}%
\pgfpathcurveto{\pgfqpoint{1.071877in}{1.882533in}}{\pgfqpoint{1.079777in}{1.885805in}}{\pgfqpoint{1.085601in}{1.891629in}}%
\pgfpathcurveto{\pgfqpoint{1.091425in}{1.897453in}}{\pgfqpoint{1.094697in}{1.905353in}}{\pgfqpoint{1.094697in}{1.913590in}}%
\pgfpathcurveto{\pgfqpoint{1.094697in}{1.921826in}}{\pgfqpoint{1.091425in}{1.929726in}}{\pgfqpoint{1.085601in}{1.935550in}}%
\pgfpathcurveto{\pgfqpoint{1.079777in}{1.941374in}}{\pgfqpoint{1.071877in}{1.944646in}}{\pgfqpoint{1.063641in}{1.944646in}}%
\pgfpathcurveto{\pgfqpoint{1.055405in}{1.944646in}}{\pgfqpoint{1.047505in}{1.941374in}}{\pgfqpoint{1.041681in}{1.935550in}}%
\pgfpathcurveto{\pgfqpoint{1.035857in}{1.929726in}}{\pgfqpoint{1.032584in}{1.921826in}}{\pgfqpoint{1.032584in}{1.913590in}}%
\pgfpathcurveto{\pgfqpoint{1.032584in}{1.905353in}}{\pgfqpoint{1.035857in}{1.897453in}}{\pgfqpoint{1.041681in}{1.891629in}}%
\pgfpathcurveto{\pgfqpoint{1.047505in}{1.885805in}}{\pgfqpoint{1.055405in}{1.882533in}}{\pgfqpoint{1.063641in}{1.882533in}}%
\pgfpathclose%
\pgfusepath{stroke,fill}%
\end{pgfscope}%
\begin{pgfscope}%
\pgfpathrectangle{\pgfqpoint{0.100000in}{0.212622in}}{\pgfqpoint{3.696000in}{3.696000in}}%
\pgfusepath{clip}%
\pgfsetbuttcap%
\pgfsetroundjoin%
\definecolor{currentfill}{rgb}{0.121569,0.466667,0.705882}%
\pgfsetfillcolor{currentfill}%
\pgfsetfillopacity{0.564986}%
\pgfsetlinewidth{1.003750pt}%
\definecolor{currentstroke}{rgb}{0.121569,0.466667,0.705882}%
\pgfsetstrokecolor{currentstroke}%
\pgfsetstrokeopacity{0.564986}%
\pgfsetdash{}{0pt}%
\pgfpathmoveto{\pgfqpoint{2.077789in}{2.234384in}}%
\pgfpathcurveto{\pgfqpoint{2.086025in}{2.234384in}}{\pgfqpoint{2.093925in}{2.237656in}}{\pgfqpoint{2.099749in}{2.243480in}}%
\pgfpathcurveto{\pgfqpoint{2.105573in}{2.249304in}}{\pgfqpoint{2.108846in}{2.257204in}}{\pgfqpoint{2.108846in}{2.265440in}}%
\pgfpathcurveto{\pgfqpoint{2.108846in}{2.273676in}}{\pgfqpoint{2.105573in}{2.281577in}}{\pgfqpoint{2.099749in}{2.287400in}}%
\pgfpathcurveto{\pgfqpoint{2.093925in}{2.293224in}}{\pgfqpoint{2.086025in}{2.296497in}}{\pgfqpoint{2.077789in}{2.296497in}}%
\pgfpathcurveto{\pgfqpoint{2.069553in}{2.296497in}}{\pgfqpoint{2.061653in}{2.293224in}}{\pgfqpoint{2.055829in}{2.287400in}}%
\pgfpathcurveto{\pgfqpoint{2.050005in}{2.281577in}}{\pgfqpoint{2.046733in}{2.273676in}}{\pgfqpoint{2.046733in}{2.265440in}}%
\pgfpathcurveto{\pgfqpoint{2.046733in}{2.257204in}}{\pgfqpoint{2.050005in}{2.249304in}}{\pgfqpoint{2.055829in}{2.243480in}}%
\pgfpathcurveto{\pgfqpoint{2.061653in}{2.237656in}}{\pgfqpoint{2.069553in}{2.234384in}}{\pgfqpoint{2.077789in}{2.234384in}}%
\pgfpathclose%
\pgfusepath{stroke,fill}%
\end{pgfscope}%
\begin{pgfscope}%
\pgfpathrectangle{\pgfqpoint{0.100000in}{0.212622in}}{\pgfqpoint{3.696000in}{3.696000in}}%
\pgfusepath{clip}%
\pgfsetbuttcap%
\pgfsetroundjoin%
\definecolor{currentfill}{rgb}{0.121569,0.466667,0.705882}%
\pgfsetfillcolor{currentfill}%
\pgfsetfillopacity{0.565479}%
\pgfsetlinewidth{1.003750pt}%
\definecolor{currentstroke}{rgb}{0.121569,0.466667,0.705882}%
\pgfsetstrokecolor{currentstroke}%
\pgfsetstrokeopacity{0.565479}%
\pgfsetdash{}{0pt}%
\pgfpathmoveto{\pgfqpoint{1.060113in}{1.875184in}}%
\pgfpathcurveto{\pgfqpoint{1.068349in}{1.875184in}}{\pgfqpoint{1.076249in}{1.878457in}}{\pgfqpoint{1.082073in}{1.884281in}}%
\pgfpathcurveto{\pgfqpoint{1.087897in}{1.890105in}}{\pgfqpoint{1.091169in}{1.898005in}}{\pgfqpoint{1.091169in}{1.906241in}}%
\pgfpathcurveto{\pgfqpoint{1.091169in}{1.914477in}}{\pgfqpoint{1.087897in}{1.922377in}}{\pgfqpoint{1.082073in}{1.928201in}}%
\pgfpathcurveto{\pgfqpoint{1.076249in}{1.934025in}}{\pgfqpoint{1.068349in}{1.937297in}}{\pgfqpoint{1.060113in}{1.937297in}}%
\pgfpathcurveto{\pgfqpoint{1.051876in}{1.937297in}}{\pgfqpoint{1.043976in}{1.934025in}}{\pgfqpoint{1.038152in}{1.928201in}}%
\pgfpathcurveto{\pgfqpoint{1.032329in}{1.922377in}}{\pgfqpoint{1.029056in}{1.914477in}}{\pgfqpoint{1.029056in}{1.906241in}}%
\pgfpathcurveto{\pgfqpoint{1.029056in}{1.898005in}}{\pgfqpoint{1.032329in}{1.890105in}}{\pgfqpoint{1.038152in}{1.884281in}}%
\pgfpathcurveto{\pgfqpoint{1.043976in}{1.878457in}}{\pgfqpoint{1.051876in}{1.875184in}}{\pgfqpoint{1.060113in}{1.875184in}}%
\pgfpathclose%
\pgfusepath{stroke,fill}%
\end{pgfscope}%
\begin{pgfscope}%
\pgfpathrectangle{\pgfqpoint{0.100000in}{0.212622in}}{\pgfqpoint{3.696000in}{3.696000in}}%
\pgfusepath{clip}%
\pgfsetbuttcap%
\pgfsetroundjoin%
\definecolor{currentfill}{rgb}{0.121569,0.466667,0.705882}%
\pgfsetfillcolor{currentfill}%
\pgfsetfillopacity{0.566151}%
\pgfsetlinewidth{1.003750pt}%
\definecolor{currentstroke}{rgb}{0.121569,0.466667,0.705882}%
\pgfsetstrokecolor{currentstroke}%
\pgfsetstrokeopacity{0.566151}%
\pgfsetdash{}{0pt}%
\pgfpathmoveto{\pgfqpoint{1.056586in}{1.867875in}}%
\pgfpathcurveto{\pgfqpoint{1.064822in}{1.867875in}}{\pgfqpoint{1.072722in}{1.871147in}}{\pgfqpoint{1.078546in}{1.876971in}}%
\pgfpathcurveto{\pgfqpoint{1.084370in}{1.882795in}}{\pgfqpoint{1.087642in}{1.890695in}}{\pgfqpoint{1.087642in}{1.898931in}}%
\pgfpathcurveto{\pgfqpoint{1.087642in}{1.907167in}}{\pgfqpoint{1.084370in}{1.915067in}}{\pgfqpoint{1.078546in}{1.920891in}}%
\pgfpathcurveto{\pgfqpoint{1.072722in}{1.926715in}}{\pgfqpoint{1.064822in}{1.929988in}}{\pgfqpoint{1.056586in}{1.929988in}}%
\pgfpathcurveto{\pgfqpoint{1.048349in}{1.929988in}}{\pgfqpoint{1.040449in}{1.926715in}}{\pgfqpoint{1.034625in}{1.920891in}}%
\pgfpathcurveto{\pgfqpoint{1.028801in}{1.915067in}}{\pgfqpoint{1.025529in}{1.907167in}}{\pgfqpoint{1.025529in}{1.898931in}}%
\pgfpathcurveto{\pgfqpoint{1.025529in}{1.890695in}}{\pgfqpoint{1.028801in}{1.882795in}}{\pgfqpoint{1.034625in}{1.876971in}}%
\pgfpathcurveto{\pgfqpoint{1.040449in}{1.871147in}}{\pgfqpoint{1.048349in}{1.867875in}}{\pgfqpoint{1.056586in}{1.867875in}}%
\pgfpathclose%
\pgfusepath{stroke,fill}%
\end{pgfscope}%
\begin{pgfscope}%
\pgfpathrectangle{\pgfqpoint{0.100000in}{0.212622in}}{\pgfqpoint{3.696000in}{3.696000in}}%
\pgfusepath{clip}%
\pgfsetbuttcap%
\pgfsetroundjoin%
\definecolor{currentfill}{rgb}{0.121569,0.466667,0.705882}%
\pgfsetfillcolor{currentfill}%
\pgfsetfillopacity{0.566305}%
\pgfsetlinewidth{1.003750pt}%
\definecolor{currentstroke}{rgb}{0.121569,0.466667,0.705882}%
\pgfsetstrokecolor{currentstroke}%
\pgfsetstrokeopacity{0.566305}%
\pgfsetdash{}{0pt}%
\pgfpathmoveto{\pgfqpoint{1.055746in}{1.865910in}}%
\pgfpathcurveto{\pgfqpoint{1.063983in}{1.865910in}}{\pgfqpoint{1.071883in}{1.869182in}}{\pgfqpoint{1.077707in}{1.875006in}}%
\pgfpathcurveto{\pgfqpoint{1.083530in}{1.880830in}}{\pgfqpoint{1.086803in}{1.888730in}}{\pgfqpoint{1.086803in}{1.896966in}}%
\pgfpathcurveto{\pgfqpoint{1.086803in}{1.905202in}}{\pgfqpoint{1.083530in}{1.913103in}}{\pgfqpoint{1.077707in}{1.918926in}}%
\pgfpathcurveto{\pgfqpoint{1.071883in}{1.924750in}}{\pgfqpoint{1.063983in}{1.928023in}}{\pgfqpoint{1.055746in}{1.928023in}}%
\pgfpathcurveto{\pgfqpoint{1.047510in}{1.928023in}}{\pgfqpoint{1.039610in}{1.924750in}}{\pgfqpoint{1.033786in}{1.918926in}}%
\pgfpathcurveto{\pgfqpoint{1.027962in}{1.913103in}}{\pgfqpoint{1.024690in}{1.905202in}}{\pgfqpoint{1.024690in}{1.896966in}}%
\pgfpathcurveto{\pgfqpoint{1.024690in}{1.888730in}}{\pgfqpoint{1.027962in}{1.880830in}}{\pgfqpoint{1.033786in}{1.875006in}}%
\pgfpathcurveto{\pgfqpoint{1.039610in}{1.869182in}}{\pgfqpoint{1.047510in}{1.865910in}}{\pgfqpoint{1.055746in}{1.865910in}}%
\pgfpathclose%
\pgfusepath{stroke,fill}%
\end{pgfscope}%
\begin{pgfscope}%
\pgfpathrectangle{\pgfqpoint{0.100000in}{0.212622in}}{\pgfqpoint{3.696000in}{3.696000in}}%
\pgfusepath{clip}%
\pgfsetbuttcap%
\pgfsetroundjoin%
\definecolor{currentfill}{rgb}{0.121569,0.466667,0.705882}%
\pgfsetfillcolor{currentfill}%
\pgfsetfillopacity{0.566353}%
\pgfsetlinewidth{1.003750pt}%
\definecolor{currentstroke}{rgb}{0.121569,0.466667,0.705882}%
\pgfsetstrokecolor{currentstroke}%
\pgfsetstrokeopacity{0.566353}%
\pgfsetdash{}{0pt}%
\pgfpathmoveto{\pgfqpoint{1.055208in}{1.864481in}}%
\pgfpathcurveto{\pgfqpoint{1.063444in}{1.864481in}}{\pgfqpoint{1.071344in}{1.867754in}}{\pgfqpoint{1.077168in}{1.873578in}}%
\pgfpathcurveto{\pgfqpoint{1.082992in}{1.879402in}}{\pgfqpoint{1.086264in}{1.887302in}}{\pgfqpoint{1.086264in}{1.895538in}}%
\pgfpathcurveto{\pgfqpoint{1.086264in}{1.903774in}}{\pgfqpoint{1.082992in}{1.911674in}}{\pgfqpoint{1.077168in}{1.917498in}}%
\pgfpathcurveto{\pgfqpoint{1.071344in}{1.923322in}}{\pgfqpoint{1.063444in}{1.926594in}}{\pgfqpoint{1.055208in}{1.926594in}}%
\pgfpathcurveto{\pgfqpoint{1.046971in}{1.926594in}}{\pgfqpoint{1.039071in}{1.923322in}}{\pgfqpoint{1.033247in}{1.917498in}}%
\pgfpathcurveto{\pgfqpoint{1.027423in}{1.911674in}}{\pgfqpoint{1.024151in}{1.903774in}}{\pgfqpoint{1.024151in}{1.895538in}}%
\pgfpathcurveto{\pgfqpoint{1.024151in}{1.887302in}}{\pgfqpoint{1.027423in}{1.879402in}}{\pgfqpoint{1.033247in}{1.873578in}}%
\pgfpathcurveto{\pgfqpoint{1.039071in}{1.867754in}}{\pgfqpoint{1.046971in}{1.864481in}}{\pgfqpoint{1.055208in}{1.864481in}}%
\pgfpathclose%
\pgfusepath{stroke,fill}%
\end{pgfscope}%
\begin{pgfscope}%
\pgfpathrectangle{\pgfqpoint{0.100000in}{0.212622in}}{\pgfqpoint{3.696000in}{3.696000in}}%
\pgfusepath{clip}%
\pgfsetbuttcap%
\pgfsetroundjoin%
\definecolor{currentfill}{rgb}{0.121569,0.466667,0.705882}%
\pgfsetfillcolor{currentfill}%
\pgfsetfillopacity{0.566463}%
\pgfsetlinewidth{1.003750pt}%
\definecolor{currentstroke}{rgb}{0.121569,0.466667,0.705882}%
\pgfsetstrokecolor{currentstroke}%
\pgfsetstrokeopacity{0.566463}%
\pgfsetdash{}{0pt}%
\pgfpathmoveto{\pgfqpoint{1.054874in}{1.863907in}}%
\pgfpathcurveto{\pgfqpoint{1.063111in}{1.863907in}}{\pgfqpoint{1.071011in}{1.867179in}}{\pgfqpoint{1.076834in}{1.873003in}}%
\pgfpathcurveto{\pgfqpoint{1.082658in}{1.878827in}}{\pgfqpoint{1.085931in}{1.886727in}}{\pgfqpoint{1.085931in}{1.894963in}}%
\pgfpathcurveto{\pgfqpoint{1.085931in}{1.903199in}}{\pgfqpoint{1.082658in}{1.911099in}}{\pgfqpoint{1.076834in}{1.916923in}}%
\pgfpathcurveto{\pgfqpoint{1.071011in}{1.922747in}}{\pgfqpoint{1.063111in}{1.926020in}}{\pgfqpoint{1.054874in}{1.926020in}}%
\pgfpathcurveto{\pgfqpoint{1.046638in}{1.926020in}}{\pgfqpoint{1.038738in}{1.922747in}}{\pgfqpoint{1.032914in}{1.916923in}}%
\pgfpathcurveto{\pgfqpoint{1.027090in}{1.911099in}}{\pgfqpoint{1.023818in}{1.903199in}}{\pgfqpoint{1.023818in}{1.894963in}}%
\pgfpathcurveto{\pgfqpoint{1.023818in}{1.886727in}}{\pgfqpoint{1.027090in}{1.878827in}}{\pgfqpoint{1.032914in}{1.873003in}}%
\pgfpathcurveto{\pgfqpoint{1.038738in}{1.867179in}}{\pgfqpoint{1.046638in}{1.863907in}}{\pgfqpoint{1.054874in}{1.863907in}}%
\pgfpathclose%
\pgfusepath{stroke,fill}%
\end{pgfscope}%
\begin{pgfscope}%
\pgfpathrectangle{\pgfqpoint{0.100000in}{0.212622in}}{\pgfqpoint{3.696000in}{3.696000in}}%
\pgfusepath{clip}%
\pgfsetbuttcap%
\pgfsetroundjoin%
\definecolor{currentfill}{rgb}{0.121569,0.466667,0.705882}%
\pgfsetfillcolor{currentfill}%
\pgfsetfillopacity{0.566670}%
\pgfsetlinewidth{1.003750pt}%
\definecolor{currentstroke}{rgb}{0.121569,0.466667,0.705882}%
\pgfsetstrokecolor{currentstroke}%
\pgfsetstrokeopacity{0.566670}%
\pgfsetdash{}{0pt}%
\pgfpathmoveto{\pgfqpoint{1.054272in}{1.862878in}}%
\pgfpathcurveto{\pgfqpoint{1.062508in}{1.862878in}}{\pgfqpoint{1.070409in}{1.866150in}}{\pgfqpoint{1.076232in}{1.871974in}}%
\pgfpathcurveto{\pgfqpoint{1.082056in}{1.877798in}}{\pgfqpoint{1.085329in}{1.885698in}}{\pgfqpoint{1.085329in}{1.893934in}}%
\pgfpathcurveto{\pgfqpoint{1.085329in}{1.902171in}}{\pgfqpoint{1.082056in}{1.910071in}}{\pgfqpoint{1.076232in}{1.915895in}}%
\pgfpathcurveto{\pgfqpoint{1.070409in}{1.921719in}}{\pgfqpoint{1.062508in}{1.924991in}}{\pgfqpoint{1.054272in}{1.924991in}}%
\pgfpathcurveto{\pgfqpoint{1.046036in}{1.924991in}}{\pgfqpoint{1.038136in}{1.921719in}}{\pgfqpoint{1.032312in}{1.915895in}}%
\pgfpathcurveto{\pgfqpoint{1.026488in}{1.910071in}}{\pgfqpoint{1.023216in}{1.902171in}}{\pgfqpoint{1.023216in}{1.893934in}}%
\pgfpathcurveto{\pgfqpoint{1.023216in}{1.885698in}}{\pgfqpoint{1.026488in}{1.877798in}}{\pgfqpoint{1.032312in}{1.871974in}}%
\pgfpathcurveto{\pgfqpoint{1.038136in}{1.866150in}}{\pgfqpoint{1.046036in}{1.862878in}}{\pgfqpoint{1.054272in}{1.862878in}}%
\pgfpathclose%
\pgfusepath{stroke,fill}%
\end{pgfscope}%
\begin{pgfscope}%
\pgfpathrectangle{\pgfqpoint{0.100000in}{0.212622in}}{\pgfqpoint{3.696000in}{3.696000in}}%
\pgfusepath{clip}%
\pgfsetbuttcap%
\pgfsetroundjoin%
\definecolor{currentfill}{rgb}{0.121569,0.466667,0.705882}%
\pgfsetfillcolor{currentfill}%
\pgfsetfillopacity{0.567034}%
\pgfsetlinewidth{1.003750pt}%
\definecolor{currentstroke}{rgb}{0.121569,0.466667,0.705882}%
\pgfsetstrokecolor{currentstroke}%
\pgfsetstrokeopacity{0.567034}%
\pgfsetdash{}{0pt}%
\pgfpathmoveto{\pgfqpoint{1.053217in}{1.860908in}}%
\pgfpathcurveto{\pgfqpoint{1.061453in}{1.860908in}}{\pgfqpoint{1.069353in}{1.864181in}}{\pgfqpoint{1.075177in}{1.870005in}}%
\pgfpathcurveto{\pgfqpoint{1.081001in}{1.875829in}}{\pgfqpoint{1.084273in}{1.883729in}}{\pgfqpoint{1.084273in}{1.891965in}}%
\pgfpathcurveto{\pgfqpoint{1.084273in}{1.900201in}}{\pgfqpoint{1.081001in}{1.908101in}}{\pgfqpoint{1.075177in}{1.913925in}}%
\pgfpathcurveto{\pgfqpoint{1.069353in}{1.919749in}}{\pgfqpoint{1.061453in}{1.923021in}}{\pgfqpoint{1.053217in}{1.923021in}}%
\pgfpathcurveto{\pgfqpoint{1.044981in}{1.923021in}}{\pgfqpoint{1.037080in}{1.919749in}}{\pgfqpoint{1.031257in}{1.913925in}}%
\pgfpathcurveto{\pgfqpoint{1.025433in}{1.908101in}}{\pgfqpoint{1.022160in}{1.900201in}}{\pgfqpoint{1.022160in}{1.891965in}}%
\pgfpathcurveto{\pgfqpoint{1.022160in}{1.883729in}}{\pgfqpoint{1.025433in}{1.875829in}}{\pgfqpoint{1.031257in}{1.870005in}}%
\pgfpathcurveto{\pgfqpoint{1.037080in}{1.864181in}}{\pgfqpoint{1.044981in}{1.860908in}}{\pgfqpoint{1.053217in}{1.860908in}}%
\pgfpathclose%
\pgfusepath{stroke,fill}%
\end{pgfscope}%
\begin{pgfscope}%
\pgfpathrectangle{\pgfqpoint{0.100000in}{0.212622in}}{\pgfqpoint{3.696000in}{3.696000in}}%
\pgfusepath{clip}%
\pgfsetbuttcap%
\pgfsetroundjoin%
\definecolor{currentfill}{rgb}{0.121569,0.466667,0.705882}%
\pgfsetfillcolor{currentfill}%
\pgfsetfillopacity{0.567644}%
\pgfsetlinewidth{1.003750pt}%
\definecolor{currentstroke}{rgb}{0.121569,0.466667,0.705882}%
\pgfsetstrokecolor{currentstroke}%
\pgfsetstrokeopacity{0.567644}%
\pgfsetdash{}{0pt}%
\pgfpathmoveto{\pgfqpoint{1.051219in}{1.857246in}}%
\pgfpathcurveto{\pgfqpoint{1.059456in}{1.857246in}}{\pgfqpoint{1.067356in}{1.860519in}}{\pgfqpoint{1.073180in}{1.866343in}}%
\pgfpathcurveto{\pgfqpoint{1.079004in}{1.872166in}}{\pgfqpoint{1.082276in}{1.880067in}}{\pgfqpoint{1.082276in}{1.888303in}}%
\pgfpathcurveto{\pgfqpoint{1.082276in}{1.896539in}}{\pgfqpoint{1.079004in}{1.904439in}}{\pgfqpoint{1.073180in}{1.910263in}}%
\pgfpathcurveto{\pgfqpoint{1.067356in}{1.916087in}}{\pgfqpoint{1.059456in}{1.919359in}}{\pgfqpoint{1.051219in}{1.919359in}}%
\pgfpathcurveto{\pgfqpoint{1.042983in}{1.919359in}}{\pgfqpoint{1.035083in}{1.916087in}}{\pgfqpoint{1.029259in}{1.910263in}}%
\pgfpathcurveto{\pgfqpoint{1.023435in}{1.904439in}}{\pgfqpoint{1.020163in}{1.896539in}}{\pgfqpoint{1.020163in}{1.888303in}}%
\pgfpathcurveto{\pgfqpoint{1.020163in}{1.880067in}}{\pgfqpoint{1.023435in}{1.872166in}}{\pgfqpoint{1.029259in}{1.866343in}}%
\pgfpathcurveto{\pgfqpoint{1.035083in}{1.860519in}}{\pgfqpoint{1.042983in}{1.857246in}}{\pgfqpoint{1.051219in}{1.857246in}}%
\pgfpathclose%
\pgfusepath{stroke,fill}%
\end{pgfscope}%
\begin{pgfscope}%
\pgfpathrectangle{\pgfqpoint{0.100000in}{0.212622in}}{\pgfqpoint{3.696000in}{3.696000in}}%
\pgfusepath{clip}%
\pgfsetbuttcap%
\pgfsetroundjoin%
\definecolor{currentfill}{rgb}{0.121569,0.466667,0.705882}%
\pgfsetfillcolor{currentfill}%
\pgfsetfillopacity{0.568014}%
\pgfsetlinewidth{1.003750pt}%
\definecolor{currentstroke}{rgb}{0.121569,0.466667,0.705882}%
\pgfsetstrokecolor{currentstroke}%
\pgfsetstrokeopacity{0.568014}%
\pgfsetdash{}{0pt}%
\pgfpathmoveto{\pgfqpoint{2.080047in}{2.222979in}}%
\pgfpathcurveto{\pgfqpoint{2.088283in}{2.222979in}}{\pgfqpoint{2.096183in}{2.226251in}}{\pgfqpoint{2.102007in}{2.232075in}}%
\pgfpathcurveto{\pgfqpoint{2.107831in}{2.237899in}}{\pgfqpoint{2.111104in}{2.245799in}}{\pgfqpoint{2.111104in}{2.254035in}}%
\pgfpathcurveto{\pgfqpoint{2.111104in}{2.262272in}}{\pgfqpoint{2.107831in}{2.270172in}}{\pgfqpoint{2.102007in}{2.275996in}}%
\pgfpathcurveto{\pgfqpoint{2.096183in}{2.281820in}}{\pgfqpoint{2.088283in}{2.285092in}}{\pgfqpoint{2.080047in}{2.285092in}}%
\pgfpathcurveto{\pgfqpoint{2.071811in}{2.285092in}}{\pgfqpoint{2.063911in}{2.281820in}}{\pgfqpoint{2.058087in}{2.275996in}}%
\pgfpathcurveto{\pgfqpoint{2.052263in}{2.270172in}}{\pgfqpoint{2.048991in}{2.262272in}}{\pgfqpoint{2.048991in}{2.254035in}}%
\pgfpathcurveto{\pgfqpoint{2.048991in}{2.245799in}}{\pgfqpoint{2.052263in}{2.237899in}}{\pgfqpoint{2.058087in}{2.232075in}}%
\pgfpathcurveto{\pgfqpoint{2.063911in}{2.226251in}}{\pgfqpoint{2.071811in}{2.222979in}}{\pgfqpoint{2.080047in}{2.222979in}}%
\pgfpathclose%
\pgfusepath{stroke,fill}%
\end{pgfscope}%
\begin{pgfscope}%
\pgfpathrectangle{\pgfqpoint{0.100000in}{0.212622in}}{\pgfqpoint{3.696000in}{3.696000in}}%
\pgfusepath{clip}%
\pgfsetbuttcap%
\pgfsetroundjoin%
\definecolor{currentfill}{rgb}{0.121569,0.466667,0.705882}%
\pgfsetfillcolor{currentfill}%
\pgfsetfillopacity{0.568788}%
\pgfsetlinewidth{1.003750pt}%
\definecolor{currentstroke}{rgb}{0.121569,0.466667,0.705882}%
\pgfsetstrokecolor{currentstroke}%
\pgfsetstrokeopacity{0.568788}%
\pgfsetdash{}{0pt}%
\pgfpathmoveto{\pgfqpoint{1.047848in}{1.850376in}}%
\pgfpathcurveto{\pgfqpoint{1.056084in}{1.850376in}}{\pgfqpoint{1.063984in}{1.853648in}}{\pgfqpoint{1.069808in}{1.859472in}}%
\pgfpathcurveto{\pgfqpoint{1.075632in}{1.865296in}}{\pgfqpoint{1.078905in}{1.873196in}}{\pgfqpoint{1.078905in}{1.881432in}}%
\pgfpathcurveto{\pgfqpoint{1.078905in}{1.889669in}}{\pgfqpoint{1.075632in}{1.897569in}}{\pgfqpoint{1.069808in}{1.903393in}}%
\pgfpathcurveto{\pgfqpoint{1.063984in}{1.909217in}}{\pgfqpoint{1.056084in}{1.912489in}}{\pgfqpoint{1.047848in}{1.912489in}}%
\pgfpathcurveto{\pgfqpoint{1.039612in}{1.912489in}}{\pgfqpoint{1.031712in}{1.909217in}}{\pgfqpoint{1.025888in}{1.903393in}}%
\pgfpathcurveto{\pgfqpoint{1.020064in}{1.897569in}}{\pgfqpoint{1.016792in}{1.889669in}}{\pgfqpoint{1.016792in}{1.881432in}}%
\pgfpathcurveto{\pgfqpoint{1.016792in}{1.873196in}}{\pgfqpoint{1.020064in}{1.865296in}}{\pgfqpoint{1.025888in}{1.859472in}}%
\pgfpathcurveto{\pgfqpoint{1.031712in}{1.853648in}}{\pgfqpoint{1.039612in}{1.850376in}}{\pgfqpoint{1.047848in}{1.850376in}}%
\pgfpathclose%
\pgfusepath{stroke,fill}%
\end{pgfscope}%
\begin{pgfscope}%
\pgfpathrectangle{\pgfqpoint{0.100000in}{0.212622in}}{\pgfqpoint{3.696000in}{3.696000in}}%
\pgfusepath{clip}%
\pgfsetbuttcap%
\pgfsetroundjoin%
\definecolor{currentfill}{rgb}{0.121569,0.466667,0.705882}%
\pgfsetfillcolor{currentfill}%
\pgfsetfillopacity{0.570688}%
\pgfsetlinewidth{1.003750pt}%
\definecolor{currentstroke}{rgb}{0.121569,0.466667,0.705882}%
\pgfsetstrokecolor{currentstroke}%
\pgfsetstrokeopacity{0.570688}%
\pgfsetdash{}{0pt}%
\pgfpathmoveto{\pgfqpoint{1.041006in}{1.838177in}}%
\pgfpathcurveto{\pgfqpoint{1.049242in}{1.838177in}}{\pgfqpoint{1.057143in}{1.841449in}}{\pgfqpoint{1.062966in}{1.847273in}}%
\pgfpathcurveto{\pgfqpoint{1.068790in}{1.853097in}}{\pgfqpoint{1.072063in}{1.860997in}}{\pgfqpoint{1.072063in}{1.869233in}}%
\pgfpathcurveto{\pgfqpoint{1.072063in}{1.877470in}}{\pgfqpoint{1.068790in}{1.885370in}}{\pgfqpoint{1.062966in}{1.891194in}}%
\pgfpathcurveto{\pgfqpoint{1.057143in}{1.897017in}}{\pgfqpoint{1.049242in}{1.900290in}}{\pgfqpoint{1.041006in}{1.900290in}}%
\pgfpathcurveto{\pgfqpoint{1.032770in}{1.900290in}}{\pgfqpoint{1.024870in}{1.897017in}}{\pgfqpoint{1.019046in}{1.891194in}}%
\pgfpathcurveto{\pgfqpoint{1.013222in}{1.885370in}}{\pgfqpoint{1.009950in}{1.877470in}}{\pgfqpoint{1.009950in}{1.869233in}}%
\pgfpathcurveto{\pgfqpoint{1.009950in}{1.860997in}}{\pgfqpoint{1.013222in}{1.853097in}}{\pgfqpoint{1.019046in}{1.847273in}}%
\pgfpathcurveto{\pgfqpoint{1.024870in}{1.841449in}}{\pgfqpoint{1.032770in}{1.838177in}}{\pgfqpoint{1.041006in}{1.838177in}}%
\pgfpathclose%
\pgfusepath{stroke,fill}%
\end{pgfscope}%
\begin{pgfscope}%
\pgfpathrectangle{\pgfqpoint{0.100000in}{0.212622in}}{\pgfqpoint{3.696000in}{3.696000in}}%
\pgfusepath{clip}%
\pgfsetbuttcap%
\pgfsetroundjoin%
\definecolor{currentfill}{rgb}{0.121569,0.466667,0.705882}%
\pgfsetfillcolor{currentfill}%
\pgfsetfillopacity{0.570951}%
\pgfsetlinewidth{1.003750pt}%
\definecolor{currentstroke}{rgb}{0.121569,0.466667,0.705882}%
\pgfsetstrokecolor{currentstroke}%
\pgfsetstrokeopacity{0.570951}%
\pgfsetdash{}{0pt}%
\pgfpathmoveto{\pgfqpoint{2.082822in}{2.209838in}}%
\pgfpathcurveto{\pgfqpoint{2.091058in}{2.209838in}}{\pgfqpoint{2.098958in}{2.213111in}}{\pgfqpoint{2.104782in}{2.218935in}}%
\pgfpathcurveto{\pgfqpoint{2.110606in}{2.224759in}}{\pgfqpoint{2.113878in}{2.232659in}}{\pgfqpoint{2.113878in}{2.240895in}}%
\pgfpathcurveto{\pgfqpoint{2.113878in}{2.249131in}}{\pgfqpoint{2.110606in}{2.257031in}}{\pgfqpoint{2.104782in}{2.262855in}}%
\pgfpathcurveto{\pgfqpoint{2.098958in}{2.268679in}}{\pgfqpoint{2.091058in}{2.271951in}}{\pgfqpoint{2.082822in}{2.271951in}}%
\pgfpathcurveto{\pgfqpoint{2.074585in}{2.271951in}}{\pgfqpoint{2.066685in}{2.268679in}}{\pgfqpoint{2.060861in}{2.262855in}}%
\pgfpathcurveto{\pgfqpoint{2.055037in}{2.257031in}}{\pgfqpoint{2.051765in}{2.249131in}}{\pgfqpoint{2.051765in}{2.240895in}}%
\pgfpathcurveto{\pgfqpoint{2.051765in}{2.232659in}}{\pgfqpoint{2.055037in}{2.224759in}}{\pgfqpoint{2.060861in}{2.218935in}}%
\pgfpathcurveto{\pgfqpoint{2.066685in}{2.213111in}}{\pgfqpoint{2.074585in}{2.209838in}}{\pgfqpoint{2.082822in}{2.209838in}}%
\pgfpathclose%
\pgfusepath{stroke,fill}%
\end{pgfscope}%
\begin{pgfscope}%
\pgfpathrectangle{\pgfqpoint{0.100000in}{0.212622in}}{\pgfqpoint{3.696000in}{3.696000in}}%
\pgfusepath{clip}%
\pgfsetbuttcap%
\pgfsetroundjoin%
\definecolor{currentfill}{rgb}{0.121569,0.466667,0.705882}%
\pgfsetfillcolor{currentfill}%
\pgfsetfillopacity{0.572490}%
\pgfsetlinewidth{1.003750pt}%
\definecolor{currentstroke}{rgb}{0.121569,0.466667,0.705882}%
\pgfsetstrokecolor{currentstroke}%
\pgfsetstrokeopacity{0.572490}%
\pgfsetdash{}{0pt}%
\pgfpathmoveto{\pgfqpoint{1.036295in}{1.826737in}}%
\pgfpathcurveto{\pgfqpoint{1.044531in}{1.826737in}}{\pgfqpoint{1.052431in}{1.830009in}}{\pgfqpoint{1.058255in}{1.835833in}}%
\pgfpathcurveto{\pgfqpoint{1.064079in}{1.841657in}}{\pgfqpoint{1.067352in}{1.849557in}}{\pgfqpoint{1.067352in}{1.857794in}}%
\pgfpathcurveto{\pgfqpoint{1.067352in}{1.866030in}}{\pgfqpoint{1.064079in}{1.873930in}}{\pgfqpoint{1.058255in}{1.879754in}}%
\pgfpathcurveto{\pgfqpoint{1.052431in}{1.885578in}}{\pgfqpoint{1.044531in}{1.888850in}}{\pgfqpoint{1.036295in}{1.888850in}}%
\pgfpathcurveto{\pgfqpoint{1.028059in}{1.888850in}}{\pgfqpoint{1.020159in}{1.885578in}}{\pgfqpoint{1.014335in}{1.879754in}}%
\pgfpathcurveto{\pgfqpoint{1.008511in}{1.873930in}}{\pgfqpoint{1.005239in}{1.866030in}}{\pgfqpoint{1.005239in}{1.857794in}}%
\pgfpathcurveto{\pgfqpoint{1.005239in}{1.849557in}}{\pgfqpoint{1.008511in}{1.841657in}}{\pgfqpoint{1.014335in}{1.835833in}}%
\pgfpathcurveto{\pgfqpoint{1.020159in}{1.830009in}}{\pgfqpoint{1.028059in}{1.826737in}}{\pgfqpoint{1.036295in}{1.826737in}}%
\pgfpathclose%
\pgfusepath{stroke,fill}%
\end{pgfscope}%
\begin{pgfscope}%
\pgfpathrectangle{\pgfqpoint{0.100000in}{0.212622in}}{\pgfqpoint{3.696000in}{3.696000in}}%
\pgfusepath{clip}%
\pgfsetbuttcap%
\pgfsetroundjoin%
\definecolor{currentfill}{rgb}{0.121569,0.466667,0.705882}%
\pgfsetfillcolor{currentfill}%
\pgfsetfillopacity{0.573791}%
\pgfsetlinewidth{1.003750pt}%
\definecolor{currentstroke}{rgb}{0.121569,0.466667,0.705882}%
\pgfsetstrokecolor{currentstroke}%
\pgfsetstrokeopacity{0.573791}%
\pgfsetdash{}{0pt}%
\pgfpathmoveto{\pgfqpoint{1.031642in}{1.817959in}}%
\pgfpathcurveto{\pgfqpoint{1.039878in}{1.817959in}}{\pgfqpoint{1.047778in}{1.821232in}}{\pgfqpoint{1.053602in}{1.827056in}}%
\pgfpathcurveto{\pgfqpoint{1.059426in}{1.832880in}}{\pgfqpoint{1.062699in}{1.840780in}}{\pgfqpoint{1.062699in}{1.849016in}}%
\pgfpathcurveto{\pgfqpoint{1.062699in}{1.857252in}}{\pgfqpoint{1.059426in}{1.865152in}}{\pgfqpoint{1.053602in}{1.870976in}}%
\pgfpathcurveto{\pgfqpoint{1.047778in}{1.876800in}}{\pgfqpoint{1.039878in}{1.880072in}}{\pgfqpoint{1.031642in}{1.880072in}}%
\pgfpathcurveto{\pgfqpoint{1.023406in}{1.880072in}}{\pgfqpoint{1.015506in}{1.876800in}}{\pgfqpoint{1.009682in}{1.870976in}}%
\pgfpathcurveto{\pgfqpoint{1.003858in}{1.865152in}}{\pgfqpoint{1.000586in}{1.857252in}}{\pgfqpoint{1.000586in}{1.849016in}}%
\pgfpathcurveto{\pgfqpoint{1.000586in}{1.840780in}}{\pgfqpoint{1.003858in}{1.832880in}}{\pgfqpoint{1.009682in}{1.827056in}}%
\pgfpathcurveto{\pgfqpoint{1.015506in}{1.821232in}}{\pgfqpoint{1.023406in}{1.817959in}}{\pgfqpoint{1.031642in}{1.817959in}}%
\pgfpathclose%
\pgfusepath{stroke,fill}%
\end{pgfscope}%
\begin{pgfscope}%
\pgfpathrectangle{\pgfqpoint{0.100000in}{0.212622in}}{\pgfqpoint{3.696000in}{3.696000in}}%
\pgfusepath{clip}%
\pgfsetbuttcap%
\pgfsetroundjoin%
\definecolor{currentfill}{rgb}{0.121569,0.466667,0.705882}%
\pgfsetfillcolor{currentfill}%
\pgfsetfillopacity{0.574779}%
\pgfsetlinewidth{1.003750pt}%
\definecolor{currentstroke}{rgb}{0.121569,0.466667,0.705882}%
\pgfsetstrokecolor{currentstroke}%
\pgfsetstrokeopacity{0.574779}%
\pgfsetdash{}{0pt}%
\pgfpathmoveto{\pgfqpoint{2.084657in}{2.196051in}}%
\pgfpathcurveto{\pgfqpoint{2.092894in}{2.196051in}}{\pgfqpoint{2.100794in}{2.199324in}}{\pgfqpoint{2.106618in}{2.205148in}}%
\pgfpathcurveto{\pgfqpoint{2.112442in}{2.210972in}}{\pgfqpoint{2.115714in}{2.218872in}}{\pgfqpoint{2.115714in}{2.227108in}}%
\pgfpathcurveto{\pgfqpoint{2.115714in}{2.235344in}}{\pgfqpoint{2.112442in}{2.243244in}}{\pgfqpoint{2.106618in}{2.249068in}}%
\pgfpathcurveto{\pgfqpoint{2.100794in}{2.254892in}}{\pgfqpoint{2.092894in}{2.258164in}}{\pgfqpoint{2.084657in}{2.258164in}}%
\pgfpathcurveto{\pgfqpoint{2.076421in}{2.258164in}}{\pgfqpoint{2.068521in}{2.254892in}}{\pgfqpoint{2.062697in}{2.249068in}}%
\pgfpathcurveto{\pgfqpoint{2.056873in}{2.243244in}}{\pgfqpoint{2.053601in}{2.235344in}}{\pgfqpoint{2.053601in}{2.227108in}}%
\pgfpathcurveto{\pgfqpoint{2.053601in}{2.218872in}}{\pgfqpoint{2.056873in}{2.210972in}}{\pgfqpoint{2.062697in}{2.205148in}}%
\pgfpathcurveto{\pgfqpoint{2.068521in}{2.199324in}}{\pgfqpoint{2.076421in}{2.196051in}}{\pgfqpoint{2.084657in}{2.196051in}}%
\pgfpathclose%
\pgfusepath{stroke,fill}%
\end{pgfscope}%
\begin{pgfscope}%
\pgfpathrectangle{\pgfqpoint{0.100000in}{0.212622in}}{\pgfqpoint{3.696000in}{3.696000in}}%
\pgfusepath{clip}%
\pgfsetbuttcap%
\pgfsetroundjoin%
\definecolor{currentfill}{rgb}{0.121569,0.466667,0.705882}%
\pgfsetfillcolor{currentfill}%
\pgfsetfillopacity{0.575007}%
\pgfsetlinewidth{1.003750pt}%
\definecolor{currentstroke}{rgb}{0.121569,0.466667,0.705882}%
\pgfsetstrokecolor{currentstroke}%
\pgfsetstrokeopacity{0.575007}%
\pgfsetdash{}{0pt}%
\pgfpathmoveto{\pgfqpoint{1.028340in}{1.809047in}}%
\pgfpathcurveto{\pgfqpoint{1.036576in}{1.809047in}}{\pgfqpoint{1.044476in}{1.812319in}}{\pgfqpoint{1.050300in}{1.818143in}}%
\pgfpathcurveto{\pgfqpoint{1.056124in}{1.823967in}}{\pgfqpoint{1.059397in}{1.831867in}}{\pgfqpoint{1.059397in}{1.840104in}}%
\pgfpathcurveto{\pgfqpoint{1.059397in}{1.848340in}}{\pgfqpoint{1.056124in}{1.856240in}}{\pgfqpoint{1.050300in}{1.862064in}}%
\pgfpathcurveto{\pgfqpoint{1.044476in}{1.867888in}}{\pgfqpoint{1.036576in}{1.871160in}}{\pgfqpoint{1.028340in}{1.871160in}}%
\pgfpathcurveto{\pgfqpoint{1.020104in}{1.871160in}}{\pgfqpoint{1.012204in}{1.867888in}}{\pgfqpoint{1.006380in}{1.862064in}}%
\pgfpathcurveto{\pgfqpoint{1.000556in}{1.856240in}}{\pgfqpoint{0.997284in}{1.848340in}}{\pgfqpoint{0.997284in}{1.840104in}}%
\pgfpathcurveto{\pgfqpoint{0.997284in}{1.831867in}}{\pgfqpoint{1.000556in}{1.823967in}}{\pgfqpoint{1.006380in}{1.818143in}}%
\pgfpathcurveto{\pgfqpoint{1.012204in}{1.812319in}}{\pgfqpoint{1.020104in}{1.809047in}}{\pgfqpoint{1.028340in}{1.809047in}}%
\pgfpathclose%
\pgfusepath{stroke,fill}%
\end{pgfscope}%
\begin{pgfscope}%
\pgfpathrectangle{\pgfqpoint{0.100000in}{0.212622in}}{\pgfqpoint{3.696000in}{3.696000in}}%
\pgfusepath{clip}%
\pgfsetbuttcap%
\pgfsetroundjoin%
\definecolor{currentfill}{rgb}{0.121569,0.466667,0.705882}%
\pgfsetfillcolor{currentfill}%
\pgfsetfillopacity{0.576210}%
\pgfsetlinewidth{1.003750pt}%
\definecolor{currentstroke}{rgb}{0.121569,0.466667,0.705882}%
\pgfsetstrokecolor{currentstroke}%
\pgfsetstrokeopacity{0.576210}%
\pgfsetdash{}{0pt}%
\pgfpathmoveto{\pgfqpoint{1.024712in}{1.802136in}}%
\pgfpathcurveto{\pgfqpoint{1.032948in}{1.802136in}}{\pgfqpoint{1.040848in}{1.805408in}}{\pgfqpoint{1.046672in}{1.811232in}}%
\pgfpathcurveto{\pgfqpoint{1.052496in}{1.817056in}}{\pgfqpoint{1.055769in}{1.824956in}}{\pgfqpoint{1.055769in}{1.833193in}}%
\pgfpathcurveto{\pgfqpoint{1.055769in}{1.841429in}}{\pgfqpoint{1.052496in}{1.849329in}}{\pgfqpoint{1.046672in}{1.855153in}}%
\pgfpathcurveto{\pgfqpoint{1.040848in}{1.860977in}}{\pgfqpoint{1.032948in}{1.864249in}}{\pgfqpoint{1.024712in}{1.864249in}}%
\pgfpathcurveto{\pgfqpoint{1.016476in}{1.864249in}}{\pgfqpoint{1.008576in}{1.860977in}}{\pgfqpoint{1.002752in}{1.855153in}}%
\pgfpathcurveto{\pgfqpoint{0.996928in}{1.849329in}}{\pgfqpoint{0.993656in}{1.841429in}}{\pgfqpoint{0.993656in}{1.833193in}}%
\pgfpathcurveto{\pgfqpoint{0.993656in}{1.824956in}}{\pgfqpoint{0.996928in}{1.817056in}}{\pgfqpoint{1.002752in}{1.811232in}}%
\pgfpathcurveto{\pgfqpoint{1.008576in}{1.805408in}}{\pgfqpoint{1.016476in}{1.802136in}}{\pgfqpoint{1.024712in}{1.802136in}}%
\pgfpathclose%
\pgfusepath{stroke,fill}%
\end{pgfscope}%
\begin{pgfscope}%
\pgfpathrectangle{\pgfqpoint{0.100000in}{0.212622in}}{\pgfqpoint{3.696000in}{3.696000in}}%
\pgfusepath{clip}%
\pgfsetbuttcap%
\pgfsetroundjoin%
\definecolor{currentfill}{rgb}{0.121569,0.466667,0.705882}%
\pgfsetfillcolor{currentfill}%
\pgfsetfillopacity{0.576865}%
\pgfsetlinewidth{1.003750pt}%
\definecolor{currentstroke}{rgb}{0.121569,0.466667,0.705882}%
\pgfsetstrokecolor{currentstroke}%
\pgfsetstrokeopacity{0.576865}%
\pgfsetdash{}{0pt}%
\pgfpathmoveto{\pgfqpoint{0.897365in}{1.537425in}}%
\pgfpathcurveto{\pgfqpoint{0.905601in}{1.537425in}}{\pgfqpoint{0.913501in}{1.540697in}}{\pgfqpoint{0.919325in}{1.546521in}}%
\pgfpathcurveto{\pgfqpoint{0.925149in}{1.552345in}}{\pgfqpoint{0.928421in}{1.560245in}}{\pgfqpoint{0.928421in}{1.568481in}}%
\pgfpathcurveto{\pgfqpoint{0.928421in}{1.576717in}}{\pgfqpoint{0.925149in}{1.584617in}}{\pgfqpoint{0.919325in}{1.590441in}}%
\pgfpathcurveto{\pgfqpoint{0.913501in}{1.596265in}}{\pgfqpoint{0.905601in}{1.599538in}}{\pgfqpoint{0.897365in}{1.599538in}}%
\pgfpathcurveto{\pgfqpoint{0.889128in}{1.599538in}}{\pgfqpoint{0.881228in}{1.596265in}}{\pgfqpoint{0.875404in}{1.590441in}}%
\pgfpathcurveto{\pgfqpoint{0.869581in}{1.584617in}}{\pgfqpoint{0.866308in}{1.576717in}}{\pgfqpoint{0.866308in}{1.568481in}}%
\pgfpathcurveto{\pgfqpoint{0.866308in}{1.560245in}}{\pgfqpoint{0.869581in}{1.552345in}}{\pgfqpoint{0.875404in}{1.546521in}}%
\pgfpathcurveto{\pgfqpoint{0.881228in}{1.540697in}}{\pgfqpoint{0.889128in}{1.537425in}}{\pgfqpoint{0.897365in}{1.537425in}}%
\pgfpathclose%
\pgfusepath{stroke,fill}%
\end{pgfscope}%
\begin{pgfscope}%
\pgfpathrectangle{\pgfqpoint{0.100000in}{0.212622in}}{\pgfqpoint{3.696000in}{3.696000in}}%
\pgfusepath{clip}%
\pgfsetbuttcap%
\pgfsetroundjoin%
\definecolor{currentfill}{rgb}{0.121569,0.466667,0.705882}%
\pgfsetfillcolor{currentfill}%
\pgfsetfillopacity{0.576875}%
\pgfsetlinewidth{1.003750pt}%
\definecolor{currentstroke}{rgb}{0.121569,0.466667,0.705882}%
\pgfsetstrokecolor{currentstroke}%
\pgfsetstrokeopacity{0.576875}%
\pgfsetdash{}{0pt}%
\pgfpathmoveto{\pgfqpoint{0.896486in}{1.539137in}}%
\pgfpathcurveto{\pgfqpoint{0.904723in}{1.539137in}}{\pgfqpoint{0.912623in}{1.542409in}}{\pgfqpoint{0.918447in}{1.548233in}}%
\pgfpathcurveto{\pgfqpoint{0.924271in}{1.554057in}}{\pgfqpoint{0.927543in}{1.561957in}}{\pgfqpoint{0.927543in}{1.570193in}}%
\pgfpathcurveto{\pgfqpoint{0.927543in}{1.578430in}}{\pgfqpoint{0.924271in}{1.586330in}}{\pgfqpoint{0.918447in}{1.592154in}}%
\pgfpathcurveto{\pgfqpoint{0.912623in}{1.597978in}}{\pgfqpoint{0.904723in}{1.601250in}}{\pgfqpoint{0.896486in}{1.601250in}}%
\pgfpathcurveto{\pgfqpoint{0.888250in}{1.601250in}}{\pgfqpoint{0.880350in}{1.597978in}}{\pgfqpoint{0.874526in}{1.592154in}}%
\pgfpathcurveto{\pgfqpoint{0.868702in}{1.586330in}}{\pgfqpoint{0.865430in}{1.578430in}}{\pgfqpoint{0.865430in}{1.570193in}}%
\pgfpathcurveto{\pgfqpoint{0.865430in}{1.561957in}}{\pgfqpoint{0.868702in}{1.554057in}}{\pgfqpoint{0.874526in}{1.548233in}}%
\pgfpathcurveto{\pgfqpoint{0.880350in}{1.542409in}}{\pgfqpoint{0.888250in}{1.539137in}}{\pgfqpoint{0.896486in}{1.539137in}}%
\pgfpathclose%
\pgfusepath{stroke,fill}%
\end{pgfscope}%
\begin{pgfscope}%
\pgfpathrectangle{\pgfqpoint{0.100000in}{0.212622in}}{\pgfqpoint{3.696000in}{3.696000in}}%
\pgfusepath{clip}%
\pgfsetbuttcap%
\pgfsetroundjoin%
\definecolor{currentfill}{rgb}{0.121569,0.466667,0.705882}%
\pgfsetfillcolor{currentfill}%
\pgfsetfillopacity{0.576892}%
\pgfsetlinewidth{1.003750pt}%
\definecolor{currentstroke}{rgb}{0.121569,0.466667,0.705882}%
\pgfsetstrokecolor{currentstroke}%
\pgfsetstrokeopacity{0.576892}%
\pgfsetdash{}{0pt}%
\pgfpathmoveto{\pgfqpoint{0.897794in}{1.536490in}}%
\pgfpathcurveto{\pgfqpoint{0.906031in}{1.536490in}}{\pgfqpoint{0.913931in}{1.539762in}}{\pgfqpoint{0.919755in}{1.545586in}}%
\pgfpathcurveto{\pgfqpoint{0.925579in}{1.551410in}}{\pgfqpoint{0.928851in}{1.559310in}}{\pgfqpoint{0.928851in}{1.567546in}}%
\pgfpathcurveto{\pgfqpoint{0.928851in}{1.575782in}}{\pgfqpoint{0.925579in}{1.583682in}}{\pgfqpoint{0.919755in}{1.589506in}}%
\pgfpathcurveto{\pgfqpoint{0.913931in}{1.595330in}}{\pgfqpoint{0.906031in}{1.598603in}}{\pgfqpoint{0.897794in}{1.598603in}}%
\pgfpathcurveto{\pgfqpoint{0.889558in}{1.598603in}}{\pgfqpoint{0.881658in}{1.595330in}}{\pgfqpoint{0.875834in}{1.589506in}}%
\pgfpathcurveto{\pgfqpoint{0.870010in}{1.583682in}}{\pgfqpoint{0.866738in}{1.575782in}}{\pgfqpoint{0.866738in}{1.567546in}}%
\pgfpathcurveto{\pgfqpoint{0.866738in}{1.559310in}}{\pgfqpoint{0.870010in}{1.551410in}}{\pgfqpoint{0.875834in}{1.545586in}}%
\pgfpathcurveto{\pgfqpoint{0.881658in}{1.539762in}}{\pgfqpoint{0.889558in}{1.536490in}}{\pgfqpoint{0.897794in}{1.536490in}}%
\pgfpathclose%
\pgfusepath{stroke,fill}%
\end{pgfscope}%
\begin{pgfscope}%
\pgfpathrectangle{\pgfqpoint{0.100000in}{0.212622in}}{\pgfqpoint{3.696000in}{3.696000in}}%
\pgfusepath{clip}%
\pgfsetbuttcap%
\pgfsetroundjoin%
\definecolor{currentfill}{rgb}{0.121569,0.466667,0.705882}%
\pgfsetfillcolor{currentfill}%
\pgfsetfillopacity{0.576933}%
\pgfsetlinewidth{1.003750pt}%
\definecolor{currentstroke}{rgb}{0.121569,0.466667,0.705882}%
\pgfsetstrokecolor{currentstroke}%
\pgfsetstrokeopacity{0.576933}%
\pgfsetdash{}{0pt}%
\pgfpathmoveto{\pgfqpoint{0.895742in}{1.540432in}}%
\pgfpathcurveto{\pgfqpoint{0.903978in}{1.540432in}}{\pgfqpoint{0.911878in}{1.543705in}}{\pgfqpoint{0.917702in}{1.549529in}}%
\pgfpathcurveto{\pgfqpoint{0.923526in}{1.555353in}}{\pgfqpoint{0.926798in}{1.563253in}}{\pgfqpoint{0.926798in}{1.571489in}}%
\pgfpathcurveto{\pgfqpoint{0.926798in}{1.579725in}}{\pgfqpoint{0.923526in}{1.587625in}}{\pgfqpoint{0.917702in}{1.593449in}}%
\pgfpathcurveto{\pgfqpoint{0.911878in}{1.599273in}}{\pgfqpoint{0.903978in}{1.602545in}}{\pgfqpoint{0.895742in}{1.602545in}}%
\pgfpathcurveto{\pgfqpoint{0.887506in}{1.602545in}}{\pgfqpoint{0.879606in}{1.599273in}}{\pgfqpoint{0.873782in}{1.593449in}}%
\pgfpathcurveto{\pgfqpoint{0.867958in}{1.587625in}}{\pgfqpoint{0.864685in}{1.579725in}}{\pgfqpoint{0.864685in}{1.571489in}}%
\pgfpathcurveto{\pgfqpoint{0.864685in}{1.563253in}}{\pgfqpoint{0.867958in}{1.555353in}}{\pgfqpoint{0.873782in}{1.549529in}}%
\pgfpathcurveto{\pgfqpoint{0.879606in}{1.543705in}}{\pgfqpoint{0.887506in}{1.540432in}}{\pgfqpoint{0.895742in}{1.540432in}}%
\pgfpathclose%
\pgfusepath{stroke,fill}%
\end{pgfscope}%
\begin{pgfscope}%
\pgfpathrectangle{\pgfqpoint{0.100000in}{0.212622in}}{\pgfqpoint{3.696000in}{3.696000in}}%
\pgfusepath{clip}%
\pgfsetbuttcap%
\pgfsetroundjoin%
\definecolor{currentfill}{rgb}{0.121569,0.466667,0.705882}%
\pgfsetfillcolor{currentfill}%
\pgfsetfillopacity{0.576995}%
\pgfsetlinewidth{1.003750pt}%
\definecolor{currentstroke}{rgb}{0.121569,0.466667,0.705882}%
\pgfsetstrokecolor{currentstroke}%
\pgfsetstrokeopacity{0.576995}%
\pgfsetdash{}{0pt}%
\pgfpathmoveto{\pgfqpoint{0.898463in}{1.534883in}}%
\pgfpathcurveto{\pgfqpoint{0.906700in}{1.534883in}}{\pgfqpoint{0.914600in}{1.538156in}}{\pgfqpoint{0.920424in}{1.543979in}}%
\pgfpathcurveto{\pgfqpoint{0.926248in}{1.549803in}}{\pgfqpoint{0.929520in}{1.557703in}}{\pgfqpoint{0.929520in}{1.565940in}}%
\pgfpathcurveto{\pgfqpoint{0.929520in}{1.574176in}}{\pgfqpoint{0.926248in}{1.582076in}}{\pgfqpoint{0.920424in}{1.587900in}}%
\pgfpathcurveto{\pgfqpoint{0.914600in}{1.593724in}}{\pgfqpoint{0.906700in}{1.596996in}}{\pgfqpoint{0.898463in}{1.596996in}}%
\pgfpathcurveto{\pgfqpoint{0.890227in}{1.596996in}}{\pgfqpoint{0.882327in}{1.593724in}}{\pgfqpoint{0.876503in}{1.587900in}}%
\pgfpathcurveto{\pgfqpoint{0.870679in}{1.582076in}}{\pgfqpoint{0.867407in}{1.574176in}}{\pgfqpoint{0.867407in}{1.565940in}}%
\pgfpathcurveto{\pgfqpoint{0.867407in}{1.557703in}}{\pgfqpoint{0.870679in}{1.549803in}}{\pgfqpoint{0.876503in}{1.543979in}}%
\pgfpathcurveto{\pgfqpoint{0.882327in}{1.538156in}}{\pgfqpoint{0.890227in}{1.534883in}}{\pgfqpoint{0.898463in}{1.534883in}}%
\pgfpathclose%
\pgfusepath{stroke,fill}%
\end{pgfscope}%
\begin{pgfscope}%
\pgfpathrectangle{\pgfqpoint{0.100000in}{0.212622in}}{\pgfqpoint{3.696000in}{3.696000in}}%
\pgfusepath{clip}%
\pgfsetbuttcap%
\pgfsetroundjoin%
\definecolor{currentfill}{rgb}{0.121569,0.466667,0.705882}%
\pgfsetfillcolor{currentfill}%
\pgfsetfillopacity{0.577078}%
\pgfsetlinewidth{1.003750pt}%
\definecolor{currentstroke}{rgb}{0.121569,0.466667,0.705882}%
\pgfsetstrokecolor{currentstroke}%
\pgfsetstrokeopacity{0.577078}%
\pgfsetdash{}{0pt}%
\pgfpathmoveto{\pgfqpoint{0.898783in}{1.534018in}}%
\pgfpathcurveto{\pgfqpoint{0.907020in}{1.534018in}}{\pgfqpoint{0.914920in}{1.537290in}}{\pgfqpoint{0.920744in}{1.543114in}}%
\pgfpathcurveto{\pgfqpoint{0.926568in}{1.548938in}}{\pgfqpoint{0.929840in}{1.556838in}}{\pgfqpoint{0.929840in}{1.565074in}}%
\pgfpathcurveto{\pgfqpoint{0.929840in}{1.573310in}}{\pgfqpoint{0.926568in}{1.581210in}}{\pgfqpoint{0.920744in}{1.587034in}}%
\pgfpathcurveto{\pgfqpoint{0.914920in}{1.592858in}}{\pgfqpoint{0.907020in}{1.596131in}}{\pgfqpoint{0.898783in}{1.596131in}}%
\pgfpathcurveto{\pgfqpoint{0.890547in}{1.596131in}}{\pgfqpoint{0.882647in}{1.592858in}}{\pgfqpoint{0.876823in}{1.587034in}}%
\pgfpathcurveto{\pgfqpoint{0.870999in}{1.581210in}}{\pgfqpoint{0.867727in}{1.573310in}}{\pgfqpoint{0.867727in}{1.565074in}}%
\pgfpathcurveto{\pgfqpoint{0.867727in}{1.556838in}}{\pgfqpoint{0.870999in}{1.548938in}}{\pgfqpoint{0.876823in}{1.543114in}}%
\pgfpathcurveto{\pgfqpoint{0.882647in}{1.537290in}}{\pgfqpoint{0.890547in}{1.534018in}}{\pgfqpoint{0.898783in}{1.534018in}}%
\pgfpathclose%
\pgfusepath{stroke,fill}%
\end{pgfscope}%
\begin{pgfscope}%
\pgfpathrectangle{\pgfqpoint{0.100000in}{0.212622in}}{\pgfqpoint{3.696000in}{3.696000in}}%
\pgfusepath{clip}%
\pgfsetbuttcap%
\pgfsetroundjoin%
\definecolor{currentfill}{rgb}{0.121569,0.466667,0.705882}%
\pgfsetfillcolor{currentfill}%
\pgfsetfillopacity{0.577132}%
\pgfsetlinewidth{1.003750pt}%
\definecolor{currentstroke}{rgb}{0.121569,0.466667,0.705882}%
\pgfsetstrokecolor{currentstroke}%
\pgfsetstrokeopacity{0.577132}%
\pgfsetdash{}{0pt}%
\pgfpathmoveto{\pgfqpoint{0.898936in}{1.533538in}}%
\pgfpathcurveto{\pgfqpoint{0.907173in}{1.533538in}}{\pgfqpoint{0.915073in}{1.536811in}}{\pgfqpoint{0.920897in}{1.542635in}}%
\pgfpathcurveto{\pgfqpoint{0.926720in}{1.548459in}}{\pgfqpoint{0.929993in}{1.556359in}}{\pgfqpoint{0.929993in}{1.564595in}}%
\pgfpathcurveto{\pgfqpoint{0.929993in}{1.572831in}}{\pgfqpoint{0.926720in}{1.580731in}}{\pgfqpoint{0.920897in}{1.586555in}}%
\pgfpathcurveto{\pgfqpoint{0.915073in}{1.592379in}}{\pgfqpoint{0.907173in}{1.595651in}}{\pgfqpoint{0.898936in}{1.595651in}}%
\pgfpathcurveto{\pgfqpoint{0.890700in}{1.595651in}}{\pgfqpoint{0.882800in}{1.592379in}}{\pgfqpoint{0.876976in}{1.586555in}}%
\pgfpathcurveto{\pgfqpoint{0.871152in}{1.580731in}}{\pgfqpoint{0.867880in}{1.572831in}}{\pgfqpoint{0.867880in}{1.564595in}}%
\pgfpathcurveto{\pgfqpoint{0.867880in}{1.556359in}}{\pgfqpoint{0.871152in}{1.548459in}}{\pgfqpoint{0.876976in}{1.542635in}}%
\pgfpathcurveto{\pgfqpoint{0.882800in}{1.536811in}}{\pgfqpoint{0.890700in}{1.533538in}}{\pgfqpoint{0.898936in}{1.533538in}}%
\pgfpathclose%
\pgfusepath{stroke,fill}%
\end{pgfscope}%
\begin{pgfscope}%
\pgfpathrectangle{\pgfqpoint{0.100000in}{0.212622in}}{\pgfqpoint{3.696000in}{3.696000in}}%
\pgfusepath{clip}%
\pgfsetbuttcap%
\pgfsetroundjoin%
\definecolor{currentfill}{rgb}{0.121569,0.466667,0.705882}%
\pgfsetfillcolor{currentfill}%
\pgfsetfillopacity{0.577143}%
\pgfsetlinewidth{1.003750pt}%
\definecolor{currentstroke}{rgb}{0.121569,0.466667,0.705882}%
\pgfsetstrokecolor{currentstroke}%
\pgfsetstrokeopacity{0.577143}%
\pgfsetdash{}{0pt}%
\pgfpathmoveto{\pgfqpoint{0.894254in}{1.542788in}}%
\pgfpathcurveto{\pgfqpoint{0.902491in}{1.542788in}}{\pgfqpoint{0.910391in}{1.546060in}}{\pgfqpoint{0.916215in}{1.551884in}}%
\pgfpathcurveto{\pgfqpoint{0.922039in}{1.557708in}}{\pgfqpoint{0.925311in}{1.565608in}}{\pgfqpoint{0.925311in}{1.573845in}}%
\pgfpathcurveto{\pgfqpoint{0.925311in}{1.582081in}}{\pgfqpoint{0.922039in}{1.589981in}}{\pgfqpoint{0.916215in}{1.595805in}}%
\pgfpathcurveto{\pgfqpoint{0.910391in}{1.601629in}}{\pgfqpoint{0.902491in}{1.604901in}}{\pgfqpoint{0.894254in}{1.604901in}}%
\pgfpathcurveto{\pgfqpoint{0.886018in}{1.604901in}}{\pgfqpoint{0.878118in}{1.601629in}}{\pgfqpoint{0.872294in}{1.595805in}}%
\pgfpathcurveto{\pgfqpoint{0.866470in}{1.589981in}}{\pgfqpoint{0.863198in}{1.582081in}}{\pgfqpoint{0.863198in}{1.573845in}}%
\pgfpathcurveto{\pgfqpoint{0.863198in}{1.565608in}}{\pgfqpoint{0.866470in}{1.557708in}}{\pgfqpoint{0.872294in}{1.551884in}}%
\pgfpathcurveto{\pgfqpoint{0.878118in}{1.546060in}}{\pgfqpoint{0.886018in}{1.542788in}}{\pgfqpoint{0.894254in}{1.542788in}}%
\pgfpathclose%
\pgfusepath{stroke,fill}%
\end{pgfscope}%
\begin{pgfscope}%
\pgfpathrectangle{\pgfqpoint{0.100000in}{0.212622in}}{\pgfqpoint{3.696000in}{3.696000in}}%
\pgfusepath{clip}%
\pgfsetbuttcap%
\pgfsetroundjoin%
\definecolor{currentfill}{rgb}{0.121569,0.466667,0.705882}%
\pgfsetfillcolor{currentfill}%
\pgfsetfillopacity{0.577168}%
\pgfsetlinewidth{1.003750pt}%
\definecolor{currentstroke}{rgb}{0.121569,0.466667,0.705882}%
\pgfsetstrokecolor{currentstroke}%
\pgfsetstrokeopacity{0.577168}%
\pgfsetdash{}{0pt}%
\pgfpathmoveto{\pgfqpoint{0.899008in}{1.533278in}}%
\pgfpathcurveto{\pgfqpoint{0.907245in}{1.533278in}}{\pgfqpoint{0.915145in}{1.536550in}}{\pgfqpoint{0.920969in}{1.542374in}}%
\pgfpathcurveto{\pgfqpoint{0.926792in}{1.548198in}}{\pgfqpoint{0.930065in}{1.556098in}}{\pgfqpoint{0.930065in}{1.564334in}}%
\pgfpathcurveto{\pgfqpoint{0.930065in}{1.572571in}}{\pgfqpoint{0.926792in}{1.580471in}}{\pgfqpoint{0.920969in}{1.586295in}}%
\pgfpathcurveto{\pgfqpoint{0.915145in}{1.592119in}}{\pgfqpoint{0.907245in}{1.595391in}}{\pgfqpoint{0.899008in}{1.595391in}}%
\pgfpathcurveto{\pgfqpoint{0.890772in}{1.595391in}}{\pgfqpoint{0.882872in}{1.592119in}}{\pgfqpoint{0.877048in}{1.586295in}}%
\pgfpathcurveto{\pgfqpoint{0.871224in}{1.580471in}}{\pgfqpoint{0.867952in}{1.572571in}}{\pgfqpoint{0.867952in}{1.564334in}}%
\pgfpathcurveto{\pgfqpoint{0.867952in}{1.556098in}}{\pgfqpoint{0.871224in}{1.548198in}}{\pgfqpoint{0.877048in}{1.542374in}}%
\pgfpathcurveto{\pgfqpoint{0.882872in}{1.536550in}}{\pgfqpoint{0.890772in}{1.533278in}}{\pgfqpoint{0.899008in}{1.533278in}}%
\pgfpathclose%
\pgfusepath{stroke,fill}%
\end{pgfscope}%
\begin{pgfscope}%
\pgfpathrectangle{\pgfqpoint{0.100000in}{0.212622in}}{\pgfqpoint{3.696000in}{3.696000in}}%
\pgfusepath{clip}%
\pgfsetbuttcap%
\pgfsetroundjoin%
\definecolor{currentfill}{rgb}{0.121569,0.466667,0.705882}%
\pgfsetfillcolor{currentfill}%
\pgfsetfillopacity{0.577189}%
\pgfsetlinewidth{1.003750pt}%
\definecolor{currentstroke}{rgb}{0.121569,0.466667,0.705882}%
\pgfsetstrokecolor{currentstroke}%
\pgfsetstrokeopacity{0.577189}%
\pgfsetdash{}{0pt}%
\pgfpathmoveto{\pgfqpoint{0.899041in}{1.533134in}}%
\pgfpathcurveto{\pgfqpoint{0.907277in}{1.533134in}}{\pgfqpoint{0.915177in}{1.536406in}}{\pgfqpoint{0.921001in}{1.542230in}}%
\pgfpathcurveto{\pgfqpoint{0.926825in}{1.548054in}}{\pgfqpoint{0.930097in}{1.555954in}}{\pgfqpoint{0.930097in}{1.564191in}}%
\pgfpathcurveto{\pgfqpoint{0.930097in}{1.572427in}}{\pgfqpoint{0.926825in}{1.580327in}}{\pgfqpoint{0.921001in}{1.586151in}}%
\pgfpathcurveto{\pgfqpoint{0.915177in}{1.591975in}}{\pgfqpoint{0.907277in}{1.595247in}}{\pgfqpoint{0.899041in}{1.595247in}}%
\pgfpathcurveto{\pgfqpoint{0.890804in}{1.595247in}}{\pgfqpoint{0.882904in}{1.591975in}}{\pgfqpoint{0.877080in}{1.586151in}}%
\pgfpathcurveto{\pgfqpoint{0.871257in}{1.580327in}}{\pgfqpoint{0.867984in}{1.572427in}}{\pgfqpoint{0.867984in}{1.564191in}}%
\pgfpathcurveto{\pgfqpoint{0.867984in}{1.555954in}}{\pgfqpoint{0.871257in}{1.548054in}}{\pgfqpoint{0.877080in}{1.542230in}}%
\pgfpathcurveto{\pgfqpoint{0.882904in}{1.536406in}}{\pgfqpoint{0.890804in}{1.533134in}}{\pgfqpoint{0.899041in}{1.533134in}}%
\pgfpathclose%
\pgfusepath{stroke,fill}%
\end{pgfscope}%
\begin{pgfscope}%
\pgfpathrectangle{\pgfqpoint{0.100000in}{0.212622in}}{\pgfqpoint{3.696000in}{3.696000in}}%
\pgfusepath{clip}%
\pgfsetbuttcap%
\pgfsetroundjoin%
\definecolor{currentfill}{rgb}{0.121569,0.466667,0.705882}%
\pgfsetfillcolor{currentfill}%
\pgfsetfillopacity{0.577202}%
\pgfsetlinewidth{1.003750pt}%
\definecolor{currentstroke}{rgb}{0.121569,0.466667,0.705882}%
\pgfsetstrokecolor{currentstroke}%
\pgfsetstrokeopacity{0.577202}%
\pgfsetdash{}{0pt}%
\pgfpathmoveto{\pgfqpoint{0.899055in}{1.533055in}}%
\pgfpathcurveto{\pgfqpoint{0.907291in}{1.533055in}}{\pgfqpoint{0.915191in}{1.536328in}}{\pgfqpoint{0.921015in}{1.542152in}}%
\pgfpathcurveto{\pgfqpoint{0.926839in}{1.547976in}}{\pgfqpoint{0.930112in}{1.555876in}}{\pgfqpoint{0.930112in}{1.564112in}}%
\pgfpathcurveto{\pgfqpoint{0.930112in}{1.572348in}}{\pgfqpoint{0.926839in}{1.580248in}}{\pgfqpoint{0.921015in}{1.586072in}}%
\pgfpathcurveto{\pgfqpoint{0.915191in}{1.591896in}}{\pgfqpoint{0.907291in}{1.595168in}}{\pgfqpoint{0.899055in}{1.595168in}}%
\pgfpathcurveto{\pgfqpoint{0.890819in}{1.595168in}}{\pgfqpoint{0.882919in}{1.591896in}}{\pgfqpoint{0.877095in}{1.586072in}}%
\pgfpathcurveto{\pgfqpoint{0.871271in}{1.580248in}}{\pgfqpoint{0.867999in}{1.572348in}}{\pgfqpoint{0.867999in}{1.564112in}}%
\pgfpathcurveto{\pgfqpoint{0.867999in}{1.555876in}}{\pgfqpoint{0.871271in}{1.547976in}}{\pgfqpoint{0.877095in}{1.542152in}}%
\pgfpathcurveto{\pgfqpoint{0.882919in}{1.536328in}}{\pgfqpoint{0.890819in}{1.533055in}}{\pgfqpoint{0.899055in}{1.533055in}}%
\pgfpathclose%
\pgfusepath{stroke,fill}%
\end{pgfscope}%
\begin{pgfscope}%
\pgfpathrectangle{\pgfqpoint{0.100000in}{0.212622in}}{\pgfqpoint{3.696000in}{3.696000in}}%
\pgfusepath{clip}%
\pgfsetbuttcap%
\pgfsetroundjoin%
\definecolor{currentfill}{rgb}{0.121569,0.466667,0.705882}%
\pgfsetfillcolor{currentfill}%
\pgfsetfillopacity{0.577205}%
\pgfsetlinewidth{1.003750pt}%
\definecolor{currentstroke}{rgb}{0.121569,0.466667,0.705882}%
\pgfsetstrokecolor{currentstroke}%
\pgfsetstrokeopacity{0.577205}%
\pgfsetdash{}{0pt}%
\pgfpathmoveto{\pgfqpoint{1.021393in}{1.795848in}}%
\pgfpathcurveto{\pgfqpoint{1.029629in}{1.795848in}}{\pgfqpoint{1.037529in}{1.799120in}}{\pgfqpoint{1.043353in}{1.804944in}}%
\pgfpathcurveto{\pgfqpoint{1.049177in}{1.810768in}}{\pgfqpoint{1.052449in}{1.818668in}}{\pgfqpoint{1.052449in}{1.826904in}}%
\pgfpathcurveto{\pgfqpoint{1.052449in}{1.835140in}}{\pgfqpoint{1.049177in}{1.843041in}}{\pgfqpoint{1.043353in}{1.848864in}}%
\pgfpathcurveto{\pgfqpoint{1.037529in}{1.854688in}}{\pgfqpoint{1.029629in}{1.857961in}}{\pgfqpoint{1.021393in}{1.857961in}}%
\pgfpathcurveto{\pgfqpoint{1.013156in}{1.857961in}}{\pgfqpoint{1.005256in}{1.854688in}}{\pgfqpoint{0.999432in}{1.848864in}}%
\pgfpathcurveto{\pgfqpoint{0.993608in}{1.843041in}}{\pgfqpoint{0.990336in}{1.835140in}}{\pgfqpoint{0.990336in}{1.826904in}}%
\pgfpathcurveto{\pgfqpoint{0.990336in}{1.818668in}}{\pgfqpoint{0.993608in}{1.810768in}}{\pgfqpoint{0.999432in}{1.804944in}}%
\pgfpathcurveto{\pgfqpoint{1.005256in}{1.799120in}}{\pgfqpoint{1.013156in}{1.795848in}}{\pgfqpoint{1.021393in}{1.795848in}}%
\pgfpathclose%
\pgfusepath{stroke,fill}%
\end{pgfscope}%
\begin{pgfscope}%
\pgfpathrectangle{\pgfqpoint{0.100000in}{0.212622in}}{\pgfqpoint{3.696000in}{3.696000in}}%
\pgfusepath{clip}%
\pgfsetbuttcap%
\pgfsetroundjoin%
\definecolor{currentfill}{rgb}{0.121569,0.466667,0.705882}%
\pgfsetfillcolor{currentfill}%
\pgfsetfillopacity{0.577210}%
\pgfsetlinewidth{1.003750pt}%
\definecolor{currentstroke}{rgb}{0.121569,0.466667,0.705882}%
\pgfsetstrokecolor{currentstroke}%
\pgfsetstrokeopacity{0.577210}%
\pgfsetdash{}{0pt}%
\pgfpathmoveto{\pgfqpoint{0.899061in}{1.533013in}}%
\pgfpathcurveto{\pgfqpoint{0.907297in}{1.533013in}}{\pgfqpoint{0.915198in}{1.536285in}}{\pgfqpoint{0.921021in}{1.542109in}}%
\pgfpathcurveto{\pgfqpoint{0.926845in}{1.547933in}}{\pgfqpoint{0.930118in}{1.555833in}}{\pgfqpoint{0.930118in}{1.564069in}}%
\pgfpathcurveto{\pgfqpoint{0.930118in}{1.572306in}}{\pgfqpoint{0.926845in}{1.580206in}}{\pgfqpoint{0.921021in}{1.586030in}}%
\pgfpathcurveto{\pgfqpoint{0.915198in}{1.591854in}}{\pgfqpoint{0.907297in}{1.595126in}}{\pgfqpoint{0.899061in}{1.595126in}}%
\pgfpathcurveto{\pgfqpoint{0.890825in}{1.595126in}}{\pgfqpoint{0.882925in}{1.591854in}}{\pgfqpoint{0.877101in}{1.586030in}}%
\pgfpathcurveto{\pgfqpoint{0.871277in}{1.580206in}}{\pgfqpoint{0.868005in}{1.572306in}}{\pgfqpoint{0.868005in}{1.564069in}}%
\pgfpathcurveto{\pgfqpoint{0.868005in}{1.555833in}}{\pgfqpoint{0.871277in}{1.547933in}}{\pgfqpoint{0.877101in}{1.542109in}}%
\pgfpathcurveto{\pgfqpoint{0.882925in}{1.536285in}}{\pgfqpoint{0.890825in}{1.533013in}}{\pgfqpoint{0.899061in}{1.533013in}}%
\pgfpathclose%
\pgfusepath{stroke,fill}%
\end{pgfscope}%
\begin{pgfscope}%
\pgfpathrectangle{\pgfqpoint{0.100000in}{0.212622in}}{\pgfqpoint{3.696000in}{3.696000in}}%
\pgfusepath{clip}%
\pgfsetbuttcap%
\pgfsetroundjoin%
\definecolor{currentfill}{rgb}{0.121569,0.466667,0.705882}%
\pgfsetfillcolor{currentfill}%
\pgfsetfillopacity{0.577215}%
\pgfsetlinewidth{1.003750pt}%
\definecolor{currentstroke}{rgb}{0.121569,0.466667,0.705882}%
\pgfsetstrokecolor{currentstroke}%
\pgfsetstrokeopacity{0.577215}%
\pgfsetdash{}{0pt}%
\pgfpathmoveto{\pgfqpoint{0.899064in}{1.532990in}}%
\pgfpathcurveto{\pgfqpoint{0.907300in}{1.532990in}}{\pgfqpoint{0.915200in}{1.536262in}}{\pgfqpoint{0.921024in}{1.542086in}}%
\pgfpathcurveto{\pgfqpoint{0.926848in}{1.547910in}}{\pgfqpoint{0.930120in}{1.555810in}}{\pgfqpoint{0.930120in}{1.564046in}}%
\pgfpathcurveto{\pgfqpoint{0.930120in}{1.572282in}}{\pgfqpoint{0.926848in}{1.580182in}}{\pgfqpoint{0.921024in}{1.586006in}}%
\pgfpathcurveto{\pgfqpoint{0.915200in}{1.591830in}}{\pgfqpoint{0.907300in}{1.595103in}}{\pgfqpoint{0.899064in}{1.595103in}}%
\pgfpathcurveto{\pgfqpoint{0.890827in}{1.595103in}}{\pgfqpoint{0.882927in}{1.591830in}}{\pgfqpoint{0.877103in}{1.586006in}}%
\pgfpathcurveto{\pgfqpoint{0.871279in}{1.580182in}}{\pgfqpoint{0.868007in}{1.572282in}}{\pgfqpoint{0.868007in}{1.564046in}}%
\pgfpathcurveto{\pgfqpoint{0.868007in}{1.555810in}}{\pgfqpoint{0.871279in}{1.547910in}}{\pgfqpoint{0.877103in}{1.542086in}}%
\pgfpathcurveto{\pgfqpoint{0.882927in}{1.536262in}}{\pgfqpoint{0.890827in}{1.532990in}}{\pgfqpoint{0.899064in}{1.532990in}}%
\pgfpathclose%
\pgfusepath{stroke,fill}%
\end{pgfscope}%
\begin{pgfscope}%
\pgfpathrectangle{\pgfqpoint{0.100000in}{0.212622in}}{\pgfqpoint{3.696000in}{3.696000in}}%
\pgfusepath{clip}%
\pgfsetbuttcap%
\pgfsetroundjoin%
\definecolor{currentfill}{rgb}{0.121569,0.466667,0.705882}%
\pgfsetfillcolor{currentfill}%
\pgfsetfillopacity{0.577217}%
\pgfsetlinewidth{1.003750pt}%
\definecolor{currentstroke}{rgb}{0.121569,0.466667,0.705882}%
\pgfsetstrokecolor{currentstroke}%
\pgfsetstrokeopacity{0.577217}%
\pgfsetdash{}{0pt}%
\pgfpathmoveto{\pgfqpoint{0.899065in}{1.532976in}}%
\pgfpathcurveto{\pgfqpoint{0.907301in}{1.532976in}}{\pgfqpoint{0.915201in}{1.536249in}}{\pgfqpoint{0.921025in}{1.542073in}}%
\pgfpathcurveto{\pgfqpoint{0.926849in}{1.547897in}}{\pgfqpoint{0.930121in}{1.555797in}}{\pgfqpoint{0.930121in}{1.564033in}}%
\pgfpathcurveto{\pgfqpoint{0.930121in}{1.572269in}}{\pgfqpoint{0.926849in}{1.580169in}}{\pgfqpoint{0.921025in}{1.585993in}}%
\pgfpathcurveto{\pgfqpoint{0.915201in}{1.591817in}}{\pgfqpoint{0.907301in}{1.595089in}}{\pgfqpoint{0.899065in}{1.595089in}}%
\pgfpathcurveto{\pgfqpoint{0.890828in}{1.595089in}}{\pgfqpoint{0.882928in}{1.591817in}}{\pgfqpoint{0.877104in}{1.585993in}}%
\pgfpathcurveto{\pgfqpoint{0.871280in}{1.580169in}}{\pgfqpoint{0.868008in}{1.572269in}}{\pgfqpoint{0.868008in}{1.564033in}}%
\pgfpathcurveto{\pgfqpoint{0.868008in}{1.555797in}}{\pgfqpoint{0.871280in}{1.547897in}}{\pgfqpoint{0.877104in}{1.542073in}}%
\pgfpathcurveto{\pgfqpoint{0.882928in}{1.536249in}}{\pgfqpoint{0.890828in}{1.532976in}}{\pgfqpoint{0.899065in}{1.532976in}}%
\pgfpathclose%
\pgfusepath{stroke,fill}%
\end{pgfscope}%
\begin{pgfscope}%
\pgfpathrectangle{\pgfqpoint{0.100000in}{0.212622in}}{\pgfqpoint{3.696000in}{3.696000in}}%
\pgfusepath{clip}%
\pgfsetbuttcap%
\pgfsetroundjoin%
\definecolor{currentfill}{rgb}{0.121569,0.466667,0.705882}%
\pgfsetfillcolor{currentfill}%
\pgfsetfillopacity{0.577339}%
\pgfsetlinewidth{1.003750pt}%
\definecolor{currentstroke}{rgb}{0.121569,0.466667,0.705882}%
\pgfsetstrokecolor{currentstroke}%
\pgfsetstrokeopacity{0.577339}%
\pgfsetdash{}{0pt}%
\pgfpathmoveto{\pgfqpoint{0.899089in}{1.532341in}}%
\pgfpathcurveto{\pgfqpoint{0.907325in}{1.532341in}}{\pgfqpoint{0.915225in}{1.535614in}}{\pgfqpoint{0.921049in}{1.541438in}}%
\pgfpathcurveto{\pgfqpoint{0.926873in}{1.547262in}}{\pgfqpoint{0.930146in}{1.555162in}}{\pgfqpoint{0.930146in}{1.563398in}}%
\pgfpathcurveto{\pgfqpoint{0.930146in}{1.571634in}}{\pgfqpoint{0.926873in}{1.579534in}}{\pgfqpoint{0.921049in}{1.585358in}}%
\pgfpathcurveto{\pgfqpoint{0.915225in}{1.591182in}}{\pgfqpoint{0.907325in}{1.594454in}}{\pgfqpoint{0.899089in}{1.594454in}}%
\pgfpathcurveto{\pgfqpoint{0.890853in}{1.594454in}}{\pgfqpoint{0.882953in}{1.591182in}}{\pgfqpoint{0.877129in}{1.585358in}}%
\pgfpathcurveto{\pgfqpoint{0.871305in}{1.579534in}}{\pgfqpoint{0.868033in}{1.571634in}}{\pgfqpoint{0.868033in}{1.563398in}}%
\pgfpathcurveto{\pgfqpoint{0.868033in}{1.555162in}}{\pgfqpoint{0.871305in}{1.547262in}}{\pgfqpoint{0.877129in}{1.541438in}}%
\pgfpathcurveto{\pgfqpoint{0.882953in}{1.535614in}}{\pgfqpoint{0.890853in}{1.532341in}}{\pgfqpoint{0.899089in}{1.532341in}}%
\pgfpathclose%
\pgfusepath{stroke,fill}%
\end{pgfscope}%
\begin{pgfscope}%
\pgfpathrectangle{\pgfqpoint{0.100000in}{0.212622in}}{\pgfqpoint{3.696000in}{3.696000in}}%
\pgfusepath{clip}%
\pgfsetbuttcap%
\pgfsetroundjoin%
\definecolor{currentfill}{rgb}{0.121569,0.466667,0.705882}%
\pgfsetfillcolor{currentfill}%
\pgfsetfillopacity{0.577379}%
\pgfsetlinewidth{1.003750pt}%
\definecolor{currentstroke}{rgb}{0.121569,0.466667,0.705882}%
\pgfsetstrokecolor{currentstroke}%
\pgfsetstrokeopacity{0.577379}%
\pgfsetdash{}{0pt}%
\pgfpathmoveto{\pgfqpoint{1.020772in}{1.794675in}}%
\pgfpathcurveto{\pgfqpoint{1.029008in}{1.794675in}}{\pgfqpoint{1.036908in}{1.797948in}}{\pgfqpoint{1.042732in}{1.803772in}}%
\pgfpathcurveto{\pgfqpoint{1.048556in}{1.809596in}}{\pgfqpoint{1.051829in}{1.817496in}}{\pgfqpoint{1.051829in}{1.825732in}}%
\pgfpathcurveto{\pgfqpoint{1.051829in}{1.833968in}}{\pgfqpoint{1.048556in}{1.841868in}}{\pgfqpoint{1.042732in}{1.847692in}}%
\pgfpathcurveto{\pgfqpoint{1.036908in}{1.853516in}}{\pgfqpoint{1.029008in}{1.856788in}}{\pgfqpoint{1.020772in}{1.856788in}}%
\pgfpathcurveto{\pgfqpoint{1.012536in}{1.856788in}}{\pgfqpoint{1.004636in}{1.853516in}}{\pgfqpoint{0.998812in}{1.847692in}}%
\pgfpathcurveto{\pgfqpoint{0.992988in}{1.841868in}}{\pgfqpoint{0.989716in}{1.833968in}}{\pgfqpoint{0.989716in}{1.825732in}}%
\pgfpathcurveto{\pgfqpoint{0.989716in}{1.817496in}}{\pgfqpoint{0.992988in}{1.809596in}}{\pgfqpoint{0.998812in}{1.803772in}}%
\pgfpathcurveto{\pgfqpoint{1.004636in}{1.797948in}}{\pgfqpoint{1.012536in}{1.794675in}}{\pgfqpoint{1.020772in}{1.794675in}}%
\pgfpathclose%
\pgfusepath{stroke,fill}%
\end{pgfscope}%
\begin{pgfscope}%
\pgfpathrectangle{\pgfqpoint{0.100000in}{0.212622in}}{\pgfqpoint{3.696000in}{3.696000in}}%
\pgfusepath{clip}%
\pgfsetbuttcap%
\pgfsetroundjoin%
\definecolor{currentfill}{rgb}{0.121569,0.466667,0.705882}%
\pgfsetfillcolor{currentfill}%
\pgfsetfillopacity{0.577408}%
\pgfsetlinewidth{1.003750pt}%
\definecolor{currentstroke}{rgb}{0.121569,0.466667,0.705882}%
\pgfsetstrokecolor{currentstroke}%
\pgfsetstrokeopacity{0.577408}%
\pgfsetdash{}{0pt}%
\pgfpathmoveto{\pgfqpoint{0.899091in}{1.531993in}}%
\pgfpathcurveto{\pgfqpoint{0.907327in}{1.531993in}}{\pgfqpoint{0.915227in}{1.535265in}}{\pgfqpoint{0.921051in}{1.541089in}}%
\pgfpathcurveto{\pgfqpoint{0.926875in}{1.546913in}}{\pgfqpoint{0.930147in}{1.554813in}}{\pgfqpoint{0.930147in}{1.563049in}}%
\pgfpathcurveto{\pgfqpoint{0.930147in}{1.571285in}}{\pgfqpoint{0.926875in}{1.579185in}}{\pgfqpoint{0.921051in}{1.585009in}}%
\pgfpathcurveto{\pgfqpoint{0.915227in}{1.590833in}}{\pgfqpoint{0.907327in}{1.594106in}}{\pgfqpoint{0.899091in}{1.594106in}}%
\pgfpathcurveto{\pgfqpoint{0.890855in}{1.594106in}}{\pgfqpoint{0.882955in}{1.590833in}}{\pgfqpoint{0.877131in}{1.585009in}}%
\pgfpathcurveto{\pgfqpoint{0.871307in}{1.579185in}}{\pgfqpoint{0.868034in}{1.571285in}}{\pgfqpoint{0.868034in}{1.563049in}}%
\pgfpathcurveto{\pgfqpoint{0.868034in}{1.554813in}}{\pgfqpoint{0.871307in}{1.546913in}}{\pgfqpoint{0.877131in}{1.541089in}}%
\pgfpathcurveto{\pgfqpoint{0.882955in}{1.535265in}}{\pgfqpoint{0.890855in}{1.531993in}}{\pgfqpoint{0.899091in}{1.531993in}}%
\pgfpathclose%
\pgfusepath{stroke,fill}%
\end{pgfscope}%
\begin{pgfscope}%
\pgfpathrectangle{\pgfqpoint{0.100000in}{0.212622in}}{\pgfqpoint{3.696000in}{3.696000in}}%
\pgfusepath{clip}%
\pgfsetbuttcap%
\pgfsetroundjoin%
\definecolor{currentfill}{rgb}{0.121569,0.466667,0.705882}%
\pgfsetfillcolor{currentfill}%
\pgfsetfillopacity{0.577458}%
\pgfsetlinewidth{1.003750pt}%
\definecolor{currentstroke}{rgb}{0.121569,0.466667,0.705882}%
\pgfsetstrokecolor{currentstroke}%
\pgfsetstrokeopacity{0.577458}%
\pgfsetdash{}{0pt}%
\pgfpathmoveto{\pgfqpoint{1.020617in}{1.794122in}}%
\pgfpathcurveto{\pgfqpoint{1.028853in}{1.794122in}}{\pgfqpoint{1.036753in}{1.797394in}}{\pgfqpoint{1.042577in}{1.803218in}}%
\pgfpathcurveto{\pgfqpoint{1.048401in}{1.809042in}}{\pgfqpoint{1.051673in}{1.816942in}}{\pgfqpoint{1.051673in}{1.825179in}}%
\pgfpathcurveto{\pgfqpoint{1.051673in}{1.833415in}}{\pgfqpoint{1.048401in}{1.841315in}}{\pgfqpoint{1.042577in}{1.847139in}}%
\pgfpathcurveto{\pgfqpoint{1.036753in}{1.852963in}}{\pgfqpoint{1.028853in}{1.856235in}}{\pgfqpoint{1.020617in}{1.856235in}}%
\pgfpathcurveto{\pgfqpoint{1.012380in}{1.856235in}}{\pgfqpoint{1.004480in}{1.852963in}}{\pgfqpoint{0.998656in}{1.847139in}}%
\pgfpathcurveto{\pgfqpoint{0.992832in}{1.841315in}}{\pgfqpoint{0.989560in}{1.833415in}}{\pgfqpoint{0.989560in}{1.825179in}}%
\pgfpathcurveto{\pgfqpoint{0.989560in}{1.816942in}}{\pgfqpoint{0.992832in}{1.809042in}}{\pgfqpoint{0.998656in}{1.803218in}}%
\pgfpathcurveto{\pgfqpoint{1.004480in}{1.797394in}}{\pgfqpoint{1.012380in}{1.794122in}}{\pgfqpoint{1.020617in}{1.794122in}}%
\pgfpathclose%
\pgfusepath{stroke,fill}%
\end{pgfscope}%
\begin{pgfscope}%
\pgfpathrectangle{\pgfqpoint{0.100000in}{0.212622in}}{\pgfqpoint{3.696000in}{3.696000in}}%
\pgfusepath{clip}%
\pgfsetbuttcap%
\pgfsetroundjoin%
\definecolor{currentfill}{rgb}{0.121569,0.466667,0.705882}%
\pgfsetfillcolor{currentfill}%
\pgfsetfillopacity{0.577563}%
\pgfsetlinewidth{1.003750pt}%
\definecolor{currentstroke}{rgb}{0.121569,0.466667,0.705882}%
\pgfsetstrokecolor{currentstroke}%
\pgfsetstrokeopacity{0.577563}%
\pgfsetdash{}{0pt}%
\pgfpathmoveto{\pgfqpoint{0.899076in}{1.531211in}}%
\pgfpathcurveto{\pgfqpoint{0.907312in}{1.531211in}}{\pgfqpoint{0.915212in}{1.534484in}}{\pgfqpoint{0.921036in}{1.540308in}}%
\pgfpathcurveto{\pgfqpoint{0.926860in}{1.546132in}}{\pgfqpoint{0.930133in}{1.554032in}}{\pgfqpoint{0.930133in}{1.562268in}}%
\pgfpathcurveto{\pgfqpoint{0.930133in}{1.570504in}}{\pgfqpoint{0.926860in}{1.578404in}}{\pgfqpoint{0.921036in}{1.584228in}}%
\pgfpathcurveto{\pgfqpoint{0.915212in}{1.590052in}}{\pgfqpoint{0.907312in}{1.593324in}}{\pgfqpoint{0.899076in}{1.593324in}}%
\pgfpathcurveto{\pgfqpoint{0.890840in}{1.593324in}}{\pgfqpoint{0.882940in}{1.590052in}}{\pgfqpoint{0.877116in}{1.584228in}}%
\pgfpathcurveto{\pgfqpoint{0.871292in}{1.578404in}}{\pgfqpoint{0.868020in}{1.570504in}}{\pgfqpoint{0.868020in}{1.562268in}}%
\pgfpathcurveto{\pgfqpoint{0.868020in}{1.554032in}}{\pgfqpoint{0.871292in}{1.546132in}}{\pgfqpoint{0.877116in}{1.540308in}}%
\pgfpathcurveto{\pgfqpoint{0.882940in}{1.534484in}}{\pgfqpoint{0.890840in}{1.531211in}}{\pgfqpoint{0.899076in}{1.531211in}}%
\pgfpathclose%
\pgfusepath{stroke,fill}%
\end{pgfscope}%
\begin{pgfscope}%
\pgfpathrectangle{\pgfqpoint{0.100000in}{0.212622in}}{\pgfqpoint{3.696000in}{3.696000in}}%
\pgfusepath{clip}%
\pgfsetbuttcap%
\pgfsetroundjoin%
\definecolor{currentfill}{rgb}{0.121569,0.466667,0.705882}%
\pgfsetfillcolor{currentfill}%
\pgfsetfillopacity{0.577618}%
\pgfsetlinewidth{1.003750pt}%
\definecolor{currentstroke}{rgb}{0.121569,0.466667,0.705882}%
\pgfsetstrokecolor{currentstroke}%
\pgfsetstrokeopacity{0.577618}%
\pgfsetdash{}{0pt}%
\pgfpathmoveto{\pgfqpoint{1.020149in}{1.793331in}}%
\pgfpathcurveto{\pgfqpoint{1.028385in}{1.793331in}}{\pgfqpoint{1.036285in}{1.796604in}}{\pgfqpoint{1.042109in}{1.802428in}}%
\pgfpathcurveto{\pgfqpoint{1.047933in}{1.808251in}}{\pgfqpoint{1.051205in}{1.816152in}}{\pgfqpoint{1.051205in}{1.824388in}}%
\pgfpathcurveto{\pgfqpoint{1.051205in}{1.832624in}}{\pgfqpoint{1.047933in}{1.840524in}}{\pgfqpoint{1.042109in}{1.846348in}}%
\pgfpathcurveto{\pgfqpoint{1.036285in}{1.852172in}}{\pgfqpoint{1.028385in}{1.855444in}}{\pgfqpoint{1.020149in}{1.855444in}}%
\pgfpathcurveto{\pgfqpoint{1.011912in}{1.855444in}}{\pgfqpoint{1.004012in}{1.852172in}}{\pgfqpoint{0.998188in}{1.846348in}}%
\pgfpathcurveto{\pgfqpoint{0.992364in}{1.840524in}}{\pgfqpoint{0.989092in}{1.832624in}}{\pgfqpoint{0.989092in}{1.824388in}}%
\pgfpathcurveto{\pgfqpoint{0.989092in}{1.816152in}}{\pgfqpoint{0.992364in}{1.808251in}}{\pgfqpoint{0.998188in}{1.802428in}}%
\pgfpathcurveto{\pgfqpoint{1.004012in}{1.796604in}}{\pgfqpoint{1.011912in}{1.793331in}}{\pgfqpoint{1.020149in}{1.793331in}}%
\pgfpathclose%
\pgfusepath{stroke,fill}%
\end{pgfscope}%
\begin{pgfscope}%
\pgfpathrectangle{\pgfqpoint{0.100000in}{0.212622in}}{\pgfqpoint{3.696000in}{3.696000in}}%
\pgfusepath{clip}%
\pgfsetbuttcap%
\pgfsetroundjoin%
\definecolor{currentfill}{rgb}{0.121569,0.466667,0.705882}%
\pgfsetfillcolor{currentfill}%
\pgfsetfillopacity{0.577730}%
\pgfsetlinewidth{1.003750pt}%
\definecolor{currentstroke}{rgb}{0.121569,0.466667,0.705882}%
\pgfsetstrokecolor{currentstroke}%
\pgfsetstrokeopacity{0.577730}%
\pgfsetdash{}{0pt}%
\pgfpathmoveto{\pgfqpoint{0.891326in}{1.547034in}}%
\pgfpathcurveto{\pgfqpoint{0.899563in}{1.547034in}}{\pgfqpoint{0.907463in}{1.550306in}}{\pgfqpoint{0.913287in}{1.556130in}}%
\pgfpathcurveto{\pgfqpoint{0.919111in}{1.561954in}}{\pgfqpoint{0.922383in}{1.569854in}}{\pgfqpoint{0.922383in}{1.578091in}}%
\pgfpathcurveto{\pgfqpoint{0.922383in}{1.586327in}}{\pgfqpoint{0.919111in}{1.594227in}}{\pgfqpoint{0.913287in}{1.600051in}}%
\pgfpathcurveto{\pgfqpoint{0.907463in}{1.605875in}}{\pgfqpoint{0.899563in}{1.609147in}}{\pgfqpoint{0.891326in}{1.609147in}}%
\pgfpathcurveto{\pgfqpoint{0.883090in}{1.609147in}}{\pgfqpoint{0.875190in}{1.605875in}}{\pgfqpoint{0.869366in}{1.600051in}}%
\pgfpathcurveto{\pgfqpoint{0.863542in}{1.594227in}}{\pgfqpoint{0.860270in}{1.586327in}}{\pgfqpoint{0.860270in}{1.578091in}}%
\pgfpathcurveto{\pgfqpoint{0.860270in}{1.569854in}}{\pgfqpoint{0.863542in}{1.561954in}}{\pgfqpoint{0.869366in}{1.556130in}}%
\pgfpathcurveto{\pgfqpoint{0.875190in}{1.550306in}}{\pgfqpoint{0.883090in}{1.547034in}}{\pgfqpoint{0.891326in}{1.547034in}}%
\pgfpathclose%
\pgfusepath{stroke,fill}%
\end{pgfscope}%
\begin{pgfscope}%
\pgfpathrectangle{\pgfqpoint{0.100000in}{0.212622in}}{\pgfqpoint{3.696000in}{3.696000in}}%
\pgfusepath{clip}%
\pgfsetbuttcap%
\pgfsetroundjoin%
\definecolor{currentfill}{rgb}{0.121569,0.466667,0.705882}%
\pgfsetfillcolor{currentfill}%
\pgfsetfillopacity{0.577812}%
\pgfsetlinewidth{1.003750pt}%
\definecolor{currentstroke}{rgb}{0.121569,0.466667,0.705882}%
\pgfsetstrokecolor{currentstroke}%
\pgfsetstrokeopacity{0.577812}%
\pgfsetdash{}{0pt}%
\pgfpathmoveto{\pgfqpoint{0.899016in}{1.529938in}}%
\pgfpathcurveto{\pgfqpoint{0.907253in}{1.529938in}}{\pgfqpoint{0.915153in}{1.533211in}}{\pgfqpoint{0.920977in}{1.539034in}}%
\pgfpathcurveto{\pgfqpoint{0.926800in}{1.544858in}}{\pgfqpoint{0.930073in}{1.552758in}}{\pgfqpoint{0.930073in}{1.560995in}}%
\pgfpathcurveto{\pgfqpoint{0.930073in}{1.569231in}}{\pgfqpoint{0.926800in}{1.577131in}}{\pgfqpoint{0.920977in}{1.582955in}}%
\pgfpathcurveto{\pgfqpoint{0.915153in}{1.588779in}}{\pgfqpoint{0.907253in}{1.592051in}}{\pgfqpoint{0.899016in}{1.592051in}}%
\pgfpathcurveto{\pgfqpoint{0.890780in}{1.592051in}}{\pgfqpoint{0.882880in}{1.588779in}}{\pgfqpoint{0.877056in}{1.582955in}}%
\pgfpathcurveto{\pgfqpoint{0.871232in}{1.577131in}}{\pgfqpoint{0.867960in}{1.569231in}}{\pgfqpoint{0.867960in}{1.560995in}}%
\pgfpathcurveto{\pgfqpoint{0.867960in}{1.552758in}}{\pgfqpoint{0.871232in}{1.544858in}}{\pgfqpoint{0.877056in}{1.539034in}}%
\pgfpathcurveto{\pgfqpoint{0.882880in}{1.533211in}}{\pgfqpoint{0.890780in}{1.529938in}}{\pgfqpoint{0.899016in}{1.529938in}}%
\pgfpathclose%
\pgfusepath{stroke,fill}%
\end{pgfscope}%
\begin{pgfscope}%
\pgfpathrectangle{\pgfqpoint{0.100000in}{0.212622in}}{\pgfqpoint{3.696000in}{3.696000in}}%
\pgfusepath{clip}%
\pgfsetbuttcap%
\pgfsetroundjoin%
\definecolor{currentfill}{rgb}{0.121569,0.466667,0.705882}%
\pgfsetfillcolor{currentfill}%
\pgfsetfillopacity{0.577927}%
\pgfsetlinewidth{1.003750pt}%
\definecolor{currentstroke}{rgb}{0.121569,0.466667,0.705882}%
\pgfsetstrokecolor{currentstroke}%
\pgfsetstrokeopacity{0.577927}%
\pgfsetdash{}{0pt}%
\pgfpathmoveto{\pgfqpoint{1.019384in}{1.791861in}}%
\pgfpathcurveto{\pgfqpoint{1.027620in}{1.791861in}}{\pgfqpoint{1.035520in}{1.795133in}}{\pgfqpoint{1.041344in}{1.800957in}}%
\pgfpathcurveto{\pgfqpoint{1.047168in}{1.806781in}}{\pgfqpoint{1.050441in}{1.814681in}}{\pgfqpoint{1.050441in}{1.822917in}}%
\pgfpathcurveto{\pgfqpoint{1.050441in}{1.831154in}}{\pgfqpoint{1.047168in}{1.839054in}}{\pgfqpoint{1.041344in}{1.844878in}}%
\pgfpathcurveto{\pgfqpoint{1.035520in}{1.850701in}}{\pgfqpoint{1.027620in}{1.853974in}}{\pgfqpoint{1.019384in}{1.853974in}}%
\pgfpathcurveto{\pgfqpoint{1.011148in}{1.853974in}}{\pgfqpoint{1.003248in}{1.850701in}}{\pgfqpoint{0.997424in}{1.844878in}}%
\pgfpathcurveto{\pgfqpoint{0.991600in}{1.839054in}}{\pgfqpoint{0.988328in}{1.831154in}}{\pgfqpoint{0.988328in}{1.822917in}}%
\pgfpathcurveto{\pgfqpoint{0.988328in}{1.814681in}}{\pgfqpoint{0.991600in}{1.806781in}}{\pgfqpoint{0.997424in}{1.800957in}}%
\pgfpathcurveto{\pgfqpoint{1.003248in}{1.795133in}}{\pgfqpoint{1.011148in}{1.791861in}}{\pgfqpoint{1.019384in}{1.791861in}}%
\pgfpathclose%
\pgfusepath{stroke,fill}%
\end{pgfscope}%
\begin{pgfscope}%
\pgfpathrectangle{\pgfqpoint{0.100000in}{0.212622in}}{\pgfqpoint{3.696000in}{3.696000in}}%
\pgfusepath{clip}%
\pgfsetbuttcap%
\pgfsetroundjoin%
\definecolor{currentfill}{rgb}{0.121569,0.466667,0.705882}%
\pgfsetfillcolor{currentfill}%
\pgfsetfillopacity{0.577950}%
\pgfsetlinewidth{1.003750pt}%
\definecolor{currentstroke}{rgb}{0.121569,0.466667,0.705882}%
\pgfsetstrokecolor{currentstroke}%
\pgfsetstrokeopacity{0.577950}%
\pgfsetdash{}{0pt}%
\pgfpathmoveto{\pgfqpoint{0.898968in}{1.529235in}}%
\pgfpathcurveto{\pgfqpoint{0.907204in}{1.529235in}}{\pgfqpoint{0.915104in}{1.532507in}}{\pgfqpoint{0.920928in}{1.538331in}}%
\pgfpathcurveto{\pgfqpoint{0.926752in}{1.544155in}}{\pgfqpoint{0.930024in}{1.552055in}}{\pgfqpoint{0.930024in}{1.560291in}}%
\pgfpathcurveto{\pgfqpoint{0.930024in}{1.568528in}}{\pgfqpoint{0.926752in}{1.576428in}}{\pgfqpoint{0.920928in}{1.582252in}}%
\pgfpathcurveto{\pgfqpoint{0.915104in}{1.588076in}}{\pgfqpoint{0.907204in}{1.591348in}}{\pgfqpoint{0.898968in}{1.591348in}}%
\pgfpathcurveto{\pgfqpoint{0.890731in}{1.591348in}}{\pgfqpoint{0.882831in}{1.588076in}}{\pgfqpoint{0.877007in}{1.582252in}}%
\pgfpathcurveto{\pgfqpoint{0.871184in}{1.576428in}}{\pgfqpoint{0.867911in}{1.568528in}}{\pgfqpoint{0.867911in}{1.560291in}}%
\pgfpathcurveto{\pgfqpoint{0.867911in}{1.552055in}}{\pgfqpoint{0.871184in}{1.544155in}}{\pgfqpoint{0.877007in}{1.538331in}}%
\pgfpathcurveto{\pgfqpoint{0.882831in}{1.532507in}}{\pgfqpoint{0.890731in}{1.529235in}}{\pgfqpoint{0.898968in}{1.529235in}}%
\pgfpathclose%
\pgfusepath{stroke,fill}%
\end{pgfscope}%
\begin{pgfscope}%
\pgfpathrectangle{\pgfqpoint{0.100000in}{0.212622in}}{\pgfqpoint{3.696000in}{3.696000in}}%
\pgfusepath{clip}%
\pgfsetbuttcap%
\pgfsetroundjoin%
\definecolor{currentfill}{rgb}{0.121569,0.466667,0.705882}%
\pgfsetfillcolor{currentfill}%
\pgfsetfillopacity{0.578176}%
\pgfsetlinewidth{1.003750pt}%
\definecolor{currentstroke}{rgb}{0.121569,0.466667,0.705882}%
\pgfsetstrokecolor{currentstroke}%
\pgfsetstrokeopacity{0.578176}%
\pgfsetdash{}{0pt}%
\pgfpathmoveto{\pgfqpoint{0.898875in}{1.528099in}}%
\pgfpathcurveto{\pgfqpoint{0.907112in}{1.528099in}}{\pgfqpoint{0.915012in}{1.531372in}}{\pgfqpoint{0.920836in}{1.537196in}}%
\pgfpathcurveto{\pgfqpoint{0.926660in}{1.543020in}}{\pgfqpoint{0.929932in}{1.550920in}}{\pgfqpoint{0.929932in}{1.559156in}}%
\pgfpathcurveto{\pgfqpoint{0.929932in}{1.567392in}}{\pgfqpoint{0.926660in}{1.575292in}}{\pgfqpoint{0.920836in}{1.581116in}}%
\pgfpathcurveto{\pgfqpoint{0.915012in}{1.586940in}}{\pgfqpoint{0.907112in}{1.590212in}}{\pgfqpoint{0.898875in}{1.590212in}}%
\pgfpathcurveto{\pgfqpoint{0.890639in}{1.590212in}}{\pgfqpoint{0.882739in}{1.586940in}}{\pgfqpoint{0.876915in}{1.581116in}}%
\pgfpathcurveto{\pgfqpoint{0.871091in}{1.575292in}}{\pgfqpoint{0.867819in}{1.567392in}}{\pgfqpoint{0.867819in}{1.559156in}}%
\pgfpathcurveto{\pgfqpoint{0.867819in}{1.550920in}}{\pgfqpoint{0.871091in}{1.543020in}}{\pgfqpoint{0.876915in}{1.537196in}}%
\pgfpathcurveto{\pgfqpoint{0.882739in}{1.531372in}}{\pgfqpoint{0.890639in}{1.528099in}}{\pgfqpoint{0.898875in}{1.528099in}}%
\pgfpathclose%
\pgfusepath{stroke,fill}%
\end{pgfscope}%
\begin{pgfscope}%
\pgfpathrectangle{\pgfqpoint{0.100000in}{0.212622in}}{\pgfqpoint{3.696000in}{3.696000in}}%
\pgfusepath{clip}%
\pgfsetbuttcap%
\pgfsetroundjoin%
\definecolor{currentfill}{rgb}{0.121569,0.466667,0.705882}%
\pgfsetfillcolor{currentfill}%
\pgfsetfillopacity{0.578301}%
\pgfsetlinewidth{1.003750pt}%
\definecolor{currentstroke}{rgb}{0.121569,0.466667,0.705882}%
\pgfsetstrokecolor{currentstroke}%
\pgfsetstrokeopacity{0.578301}%
\pgfsetdash{}{0pt}%
\pgfpathmoveto{\pgfqpoint{0.898816in}{1.527477in}}%
\pgfpathcurveto{\pgfqpoint{0.907052in}{1.527477in}}{\pgfqpoint{0.914952in}{1.530750in}}{\pgfqpoint{0.920776in}{1.536574in}}%
\pgfpathcurveto{\pgfqpoint{0.926600in}{1.542398in}}{\pgfqpoint{0.929872in}{1.550298in}}{\pgfqpoint{0.929872in}{1.558534in}}%
\pgfpathcurveto{\pgfqpoint{0.929872in}{1.566770in}}{\pgfqpoint{0.926600in}{1.574670in}}{\pgfqpoint{0.920776in}{1.580494in}}%
\pgfpathcurveto{\pgfqpoint{0.914952in}{1.586318in}}{\pgfqpoint{0.907052in}{1.589590in}}{\pgfqpoint{0.898816in}{1.589590in}}%
\pgfpathcurveto{\pgfqpoint{0.890579in}{1.589590in}}{\pgfqpoint{0.882679in}{1.586318in}}{\pgfqpoint{0.876855in}{1.580494in}}%
\pgfpathcurveto{\pgfqpoint{0.871031in}{1.574670in}}{\pgfqpoint{0.867759in}{1.566770in}}{\pgfqpoint{0.867759in}{1.558534in}}%
\pgfpathcurveto{\pgfqpoint{0.867759in}{1.550298in}}{\pgfqpoint{0.871031in}{1.542398in}}{\pgfqpoint{0.876855in}{1.536574in}}%
\pgfpathcurveto{\pgfqpoint{0.882679in}{1.530750in}}{\pgfqpoint{0.890579in}{1.527477in}}{\pgfqpoint{0.898816in}{1.527477in}}%
\pgfpathclose%
\pgfusepath{stroke,fill}%
\end{pgfscope}%
\begin{pgfscope}%
\pgfpathrectangle{\pgfqpoint{0.100000in}{0.212622in}}{\pgfqpoint{3.696000in}{3.696000in}}%
\pgfusepath{clip}%
\pgfsetbuttcap%
\pgfsetroundjoin%
\definecolor{currentfill}{rgb}{0.121569,0.466667,0.705882}%
\pgfsetfillcolor{currentfill}%
\pgfsetfillopacity{0.578369}%
\pgfsetlinewidth{1.003750pt}%
\definecolor{currentstroke}{rgb}{0.121569,0.466667,0.705882}%
\pgfsetstrokecolor{currentstroke}%
\pgfsetstrokeopacity{0.578369}%
\pgfsetdash{}{0pt}%
\pgfpathmoveto{\pgfqpoint{0.898778in}{1.527132in}}%
\pgfpathcurveto{\pgfqpoint{0.907015in}{1.527132in}}{\pgfqpoint{0.914915in}{1.530405in}}{\pgfqpoint{0.920739in}{1.536229in}}%
\pgfpathcurveto{\pgfqpoint{0.926563in}{1.542053in}}{\pgfqpoint{0.929835in}{1.549953in}}{\pgfqpoint{0.929835in}{1.558189in}}%
\pgfpathcurveto{\pgfqpoint{0.929835in}{1.566425in}}{\pgfqpoint{0.926563in}{1.574325in}}{\pgfqpoint{0.920739in}{1.580149in}}%
\pgfpathcurveto{\pgfqpoint{0.914915in}{1.585973in}}{\pgfqpoint{0.907015in}{1.589245in}}{\pgfqpoint{0.898778in}{1.589245in}}%
\pgfpathcurveto{\pgfqpoint{0.890542in}{1.589245in}}{\pgfqpoint{0.882642in}{1.585973in}}{\pgfqpoint{0.876818in}{1.580149in}}%
\pgfpathcurveto{\pgfqpoint{0.870994in}{1.574325in}}{\pgfqpoint{0.867722in}{1.566425in}}{\pgfqpoint{0.867722in}{1.558189in}}%
\pgfpathcurveto{\pgfqpoint{0.867722in}{1.549953in}}{\pgfqpoint{0.870994in}{1.542053in}}{\pgfqpoint{0.876818in}{1.536229in}}%
\pgfpathcurveto{\pgfqpoint{0.882642in}{1.530405in}}{\pgfqpoint{0.890542in}{1.527132in}}{\pgfqpoint{0.898778in}{1.527132in}}%
\pgfpathclose%
\pgfusepath{stroke,fill}%
\end{pgfscope}%
\begin{pgfscope}%
\pgfpathrectangle{\pgfqpoint{0.100000in}{0.212622in}}{\pgfqpoint{3.696000in}{3.696000in}}%
\pgfusepath{clip}%
\pgfsetbuttcap%
\pgfsetroundjoin%
\definecolor{currentfill}{rgb}{0.121569,0.466667,0.705882}%
\pgfsetfillcolor{currentfill}%
\pgfsetfillopacity{0.578407}%
\pgfsetlinewidth{1.003750pt}%
\definecolor{currentstroke}{rgb}{0.121569,0.466667,0.705882}%
\pgfsetstrokecolor{currentstroke}%
\pgfsetstrokeopacity{0.578407}%
\pgfsetdash{}{0pt}%
\pgfpathmoveto{\pgfqpoint{0.898756in}{1.526942in}}%
\pgfpathcurveto{\pgfqpoint{0.906992in}{1.526942in}}{\pgfqpoint{0.914892in}{1.530214in}}{\pgfqpoint{0.920716in}{1.536038in}}%
\pgfpathcurveto{\pgfqpoint{0.926540in}{1.541862in}}{\pgfqpoint{0.929812in}{1.549762in}}{\pgfqpoint{0.929812in}{1.557998in}}%
\pgfpathcurveto{\pgfqpoint{0.929812in}{1.566234in}}{\pgfqpoint{0.926540in}{1.574134in}}{\pgfqpoint{0.920716in}{1.579958in}}%
\pgfpathcurveto{\pgfqpoint{0.914892in}{1.585782in}}{\pgfqpoint{0.906992in}{1.589055in}}{\pgfqpoint{0.898756in}{1.589055in}}%
\pgfpathcurveto{\pgfqpoint{0.890519in}{1.589055in}}{\pgfqpoint{0.882619in}{1.585782in}}{\pgfqpoint{0.876795in}{1.579958in}}%
\pgfpathcurveto{\pgfqpoint{0.870971in}{1.574134in}}{\pgfqpoint{0.867699in}{1.566234in}}{\pgfqpoint{0.867699in}{1.557998in}}%
\pgfpathcurveto{\pgfqpoint{0.867699in}{1.549762in}}{\pgfqpoint{0.870971in}{1.541862in}}{\pgfqpoint{0.876795in}{1.536038in}}%
\pgfpathcurveto{\pgfqpoint{0.882619in}{1.530214in}}{\pgfqpoint{0.890519in}{1.526942in}}{\pgfqpoint{0.898756in}{1.526942in}}%
\pgfpathclose%
\pgfusepath{stroke,fill}%
\end{pgfscope}%
\begin{pgfscope}%
\pgfpathrectangle{\pgfqpoint{0.100000in}{0.212622in}}{\pgfqpoint{3.696000in}{3.696000in}}%
\pgfusepath{clip}%
\pgfsetbuttcap%
\pgfsetroundjoin%
\definecolor{currentfill}{rgb}{0.121569,0.466667,0.705882}%
\pgfsetfillcolor{currentfill}%
\pgfsetfillopacity{0.578498}%
\pgfsetlinewidth{1.003750pt}%
\definecolor{currentstroke}{rgb}{0.121569,0.466667,0.705882}%
\pgfsetstrokecolor{currentstroke}%
\pgfsetstrokeopacity{0.578498}%
\pgfsetdash{}{0pt}%
\pgfpathmoveto{\pgfqpoint{1.017872in}{1.789359in}}%
\pgfpathcurveto{\pgfqpoint{1.026108in}{1.789359in}}{\pgfqpoint{1.034008in}{1.792631in}}{\pgfqpoint{1.039832in}{1.798455in}}%
\pgfpathcurveto{\pgfqpoint{1.045656in}{1.804279in}}{\pgfqpoint{1.048928in}{1.812179in}}{\pgfqpoint{1.048928in}{1.820415in}}%
\pgfpathcurveto{\pgfqpoint{1.048928in}{1.828652in}}{\pgfqpoint{1.045656in}{1.836552in}}{\pgfqpoint{1.039832in}{1.842376in}}%
\pgfpathcurveto{\pgfqpoint{1.034008in}{1.848200in}}{\pgfqpoint{1.026108in}{1.851472in}}{\pgfqpoint{1.017872in}{1.851472in}}%
\pgfpathcurveto{\pgfqpoint{1.009636in}{1.851472in}}{\pgfqpoint{1.001736in}{1.848200in}}{\pgfqpoint{0.995912in}{1.842376in}}%
\pgfpathcurveto{\pgfqpoint{0.990088in}{1.836552in}}{\pgfqpoint{0.986815in}{1.828652in}}{\pgfqpoint{0.986815in}{1.820415in}}%
\pgfpathcurveto{\pgfqpoint{0.986815in}{1.812179in}}{\pgfqpoint{0.990088in}{1.804279in}}{\pgfqpoint{0.995912in}{1.798455in}}%
\pgfpathcurveto{\pgfqpoint{1.001736in}{1.792631in}}{\pgfqpoint{1.009636in}{1.789359in}}{\pgfqpoint{1.017872in}{1.789359in}}%
\pgfpathclose%
\pgfusepath{stroke,fill}%
\end{pgfscope}%
\begin{pgfscope}%
\pgfpathrectangle{\pgfqpoint{0.100000in}{0.212622in}}{\pgfqpoint{3.696000in}{3.696000in}}%
\pgfusepath{clip}%
\pgfsetbuttcap%
\pgfsetroundjoin%
\definecolor{currentfill}{rgb}{0.121569,0.466667,0.705882}%
\pgfsetfillcolor{currentfill}%
\pgfsetfillopacity{0.578501}%
\pgfsetlinewidth{1.003750pt}%
\definecolor{currentstroke}{rgb}{0.121569,0.466667,0.705882}%
\pgfsetstrokecolor{currentstroke}%
\pgfsetstrokeopacity{0.578501}%
\pgfsetdash{}{0pt}%
\pgfpathmoveto{\pgfqpoint{0.888512in}{1.550962in}}%
\pgfpathcurveto{\pgfqpoint{0.896748in}{1.550962in}}{\pgfqpoint{0.904648in}{1.554234in}}{\pgfqpoint{0.910472in}{1.560058in}}%
\pgfpathcurveto{\pgfqpoint{0.916296in}{1.565882in}}{\pgfqpoint{0.919568in}{1.573782in}}{\pgfqpoint{0.919568in}{1.582018in}}%
\pgfpathcurveto{\pgfqpoint{0.919568in}{1.590254in}}{\pgfqpoint{0.916296in}{1.598154in}}{\pgfqpoint{0.910472in}{1.603978in}}%
\pgfpathcurveto{\pgfqpoint{0.904648in}{1.609802in}}{\pgfqpoint{0.896748in}{1.613075in}}{\pgfqpoint{0.888512in}{1.613075in}}%
\pgfpathcurveto{\pgfqpoint{0.880275in}{1.613075in}}{\pgfqpoint{0.872375in}{1.609802in}}{\pgfqpoint{0.866551in}{1.603978in}}%
\pgfpathcurveto{\pgfqpoint{0.860727in}{1.598154in}}{\pgfqpoint{0.857455in}{1.590254in}}{\pgfqpoint{0.857455in}{1.582018in}}%
\pgfpathcurveto{\pgfqpoint{0.857455in}{1.573782in}}{\pgfqpoint{0.860727in}{1.565882in}}{\pgfqpoint{0.866551in}{1.560058in}}%
\pgfpathcurveto{\pgfqpoint{0.872375in}{1.554234in}}{\pgfqpoint{0.880275in}{1.550962in}}{\pgfqpoint{0.888512in}{1.550962in}}%
\pgfpathclose%
\pgfusepath{stroke,fill}%
\end{pgfscope}%
\begin{pgfscope}%
\pgfpathrectangle{\pgfqpoint{0.100000in}{0.212622in}}{\pgfqpoint{3.696000in}{3.696000in}}%
\pgfusepath{clip}%
\pgfsetbuttcap%
\pgfsetroundjoin%
\definecolor{currentfill}{rgb}{0.121569,0.466667,0.705882}%
\pgfsetfillcolor{currentfill}%
\pgfsetfillopacity{0.578534}%
\pgfsetlinewidth{1.003750pt}%
\definecolor{currentstroke}{rgb}{0.121569,0.466667,0.705882}%
\pgfsetstrokecolor{currentstroke}%
\pgfsetstrokeopacity{0.578534}%
\pgfsetdash{}{0pt}%
\pgfpathmoveto{\pgfqpoint{0.898672in}{1.526308in}}%
\pgfpathcurveto{\pgfqpoint{0.906908in}{1.526308in}}{\pgfqpoint{0.914808in}{1.529580in}}{\pgfqpoint{0.920632in}{1.535404in}}%
\pgfpathcurveto{\pgfqpoint{0.926456in}{1.541228in}}{\pgfqpoint{0.929728in}{1.549128in}}{\pgfqpoint{0.929728in}{1.557365in}}%
\pgfpathcurveto{\pgfqpoint{0.929728in}{1.565601in}}{\pgfqpoint{0.926456in}{1.573501in}}{\pgfqpoint{0.920632in}{1.579325in}}%
\pgfpathcurveto{\pgfqpoint{0.914808in}{1.585149in}}{\pgfqpoint{0.906908in}{1.588421in}}{\pgfqpoint{0.898672in}{1.588421in}}%
\pgfpathcurveto{\pgfqpoint{0.890436in}{1.588421in}}{\pgfqpoint{0.882535in}{1.585149in}}{\pgfqpoint{0.876712in}{1.579325in}}%
\pgfpathcurveto{\pgfqpoint{0.870888in}{1.573501in}}{\pgfqpoint{0.867615in}{1.565601in}}{\pgfqpoint{0.867615in}{1.557365in}}%
\pgfpathcurveto{\pgfqpoint{0.867615in}{1.549128in}}{\pgfqpoint{0.870888in}{1.541228in}}{\pgfqpoint{0.876712in}{1.535404in}}%
\pgfpathcurveto{\pgfqpoint{0.882535in}{1.529580in}}{\pgfqpoint{0.890436in}{1.526308in}}{\pgfqpoint{0.898672in}{1.526308in}}%
\pgfpathclose%
\pgfusepath{stroke,fill}%
\end{pgfscope}%
\begin{pgfscope}%
\pgfpathrectangle{\pgfqpoint{0.100000in}{0.212622in}}{\pgfqpoint{3.696000in}{3.696000in}}%
\pgfusepath{clip}%
\pgfsetbuttcap%
\pgfsetroundjoin%
\definecolor{currentfill}{rgb}{0.121569,0.466667,0.705882}%
\pgfsetfillcolor{currentfill}%
\pgfsetfillopacity{0.578680}%
\pgfsetlinewidth{1.003750pt}%
\definecolor{currentstroke}{rgb}{0.121569,0.466667,0.705882}%
\pgfsetstrokecolor{currentstroke}%
\pgfsetstrokeopacity{0.578680}%
\pgfsetdash{}{0pt}%
\pgfpathmoveto{\pgfqpoint{2.087003in}{2.181532in}}%
\pgfpathcurveto{\pgfqpoint{2.095239in}{2.181532in}}{\pgfqpoint{2.103139in}{2.184805in}}{\pgfqpoint{2.108963in}{2.190629in}}%
\pgfpathcurveto{\pgfqpoint{2.114787in}{2.196453in}}{\pgfqpoint{2.118059in}{2.204353in}}{\pgfqpoint{2.118059in}{2.212589in}}%
\pgfpathcurveto{\pgfqpoint{2.118059in}{2.220825in}}{\pgfqpoint{2.114787in}{2.228725in}}{\pgfqpoint{2.108963in}{2.234549in}}%
\pgfpathcurveto{\pgfqpoint{2.103139in}{2.240373in}}{\pgfqpoint{2.095239in}{2.243645in}}{\pgfqpoint{2.087003in}{2.243645in}}%
\pgfpathcurveto{\pgfqpoint{2.078766in}{2.243645in}}{\pgfqpoint{2.070866in}{2.240373in}}{\pgfqpoint{2.065042in}{2.234549in}}%
\pgfpathcurveto{\pgfqpoint{2.059219in}{2.228725in}}{\pgfqpoint{2.055946in}{2.220825in}}{\pgfqpoint{2.055946in}{2.212589in}}%
\pgfpathcurveto{\pgfqpoint{2.055946in}{2.204353in}}{\pgfqpoint{2.059219in}{2.196453in}}{\pgfqpoint{2.065042in}{2.190629in}}%
\pgfpathcurveto{\pgfqpoint{2.070866in}{2.184805in}}{\pgfqpoint{2.078766in}{2.181532in}}{\pgfqpoint{2.087003in}{2.181532in}}%
\pgfpathclose%
\pgfusepath{stroke,fill}%
\end{pgfscope}%
\begin{pgfscope}%
\pgfpathrectangle{\pgfqpoint{0.100000in}{0.212622in}}{\pgfqpoint{3.696000in}{3.696000in}}%
\pgfusepath{clip}%
\pgfsetbuttcap%
\pgfsetroundjoin%
\definecolor{currentfill}{rgb}{0.121569,0.466667,0.705882}%
\pgfsetfillcolor{currentfill}%
\pgfsetfillopacity{0.578765}%
\pgfsetlinewidth{1.003750pt}%
\definecolor{currentstroke}{rgb}{0.121569,0.466667,0.705882}%
\pgfsetstrokecolor{currentstroke}%
\pgfsetstrokeopacity{0.578765}%
\pgfsetdash{}{0pt}%
\pgfpathmoveto{\pgfqpoint{0.898513in}{1.525120in}}%
\pgfpathcurveto{\pgfqpoint{0.906749in}{1.525120in}}{\pgfqpoint{0.914649in}{1.528393in}}{\pgfqpoint{0.920473in}{1.534217in}}%
\pgfpathcurveto{\pgfqpoint{0.926297in}{1.540041in}}{\pgfqpoint{0.929569in}{1.547941in}}{\pgfqpoint{0.929569in}{1.556177in}}%
\pgfpathcurveto{\pgfqpoint{0.929569in}{1.564413in}}{\pgfqpoint{0.926297in}{1.572313in}}{\pgfqpoint{0.920473in}{1.578137in}}%
\pgfpathcurveto{\pgfqpoint{0.914649in}{1.583961in}}{\pgfqpoint{0.906749in}{1.587233in}}{\pgfqpoint{0.898513in}{1.587233in}}%
\pgfpathcurveto{\pgfqpoint{0.890276in}{1.587233in}}{\pgfqpoint{0.882376in}{1.583961in}}{\pgfqpoint{0.876552in}{1.578137in}}%
\pgfpathcurveto{\pgfqpoint{0.870728in}{1.572313in}}{\pgfqpoint{0.867456in}{1.564413in}}{\pgfqpoint{0.867456in}{1.556177in}}%
\pgfpathcurveto{\pgfqpoint{0.867456in}{1.547941in}}{\pgfqpoint{0.870728in}{1.540041in}}{\pgfqpoint{0.876552in}{1.534217in}}%
\pgfpathcurveto{\pgfqpoint{0.882376in}{1.528393in}}{\pgfqpoint{0.890276in}{1.525120in}}{\pgfqpoint{0.898513in}{1.525120in}}%
\pgfpathclose%
\pgfusepath{stroke,fill}%
\end{pgfscope}%
\begin{pgfscope}%
\pgfpathrectangle{\pgfqpoint{0.100000in}{0.212622in}}{\pgfqpoint{3.696000in}{3.696000in}}%
\pgfusepath{clip}%
\pgfsetbuttcap%
\pgfsetroundjoin%
\definecolor{currentfill}{rgb}{0.121569,0.466667,0.705882}%
\pgfsetfillcolor{currentfill}%
\pgfsetfillopacity{0.579051}%
\pgfsetlinewidth{1.003750pt}%
\definecolor{currentstroke}{rgb}{0.121569,0.466667,0.705882}%
\pgfsetstrokecolor{currentstroke}%
\pgfsetstrokeopacity{0.579051}%
\pgfsetdash{}{0pt}%
\pgfpathmoveto{\pgfqpoint{0.898288in}{1.523537in}}%
\pgfpathcurveto{\pgfqpoint{0.906525in}{1.523537in}}{\pgfqpoint{0.914425in}{1.526810in}}{\pgfqpoint{0.920249in}{1.532634in}}%
\pgfpathcurveto{\pgfqpoint{0.926073in}{1.538458in}}{\pgfqpoint{0.929345in}{1.546358in}}{\pgfqpoint{0.929345in}{1.554594in}}%
\pgfpathcurveto{\pgfqpoint{0.929345in}{1.562830in}}{\pgfqpoint{0.926073in}{1.570730in}}{\pgfqpoint{0.920249in}{1.576554in}}%
\pgfpathcurveto{\pgfqpoint{0.914425in}{1.582378in}}{\pgfqpoint{0.906525in}{1.585650in}}{\pgfqpoint{0.898288in}{1.585650in}}%
\pgfpathcurveto{\pgfqpoint{0.890052in}{1.585650in}}{\pgfqpoint{0.882152in}{1.582378in}}{\pgfqpoint{0.876328in}{1.576554in}}%
\pgfpathcurveto{\pgfqpoint{0.870504in}{1.570730in}}{\pgfqpoint{0.867232in}{1.562830in}}{\pgfqpoint{0.867232in}{1.554594in}}%
\pgfpathcurveto{\pgfqpoint{0.867232in}{1.546358in}}{\pgfqpoint{0.870504in}{1.538458in}}{\pgfqpoint{0.876328in}{1.532634in}}%
\pgfpathcurveto{\pgfqpoint{0.882152in}{1.526810in}}{\pgfqpoint{0.890052in}{1.523537in}}{\pgfqpoint{0.898288in}{1.523537in}}%
\pgfpathclose%
\pgfusepath{stroke,fill}%
\end{pgfscope}%
\begin{pgfscope}%
\pgfpathrectangle{\pgfqpoint{0.100000in}{0.212622in}}{\pgfqpoint{3.696000in}{3.696000in}}%
\pgfusepath{clip}%
\pgfsetbuttcap%
\pgfsetroundjoin%
\definecolor{currentfill}{rgb}{0.121569,0.466667,0.705882}%
\pgfsetfillcolor{currentfill}%
\pgfsetfillopacity{0.579209}%
\pgfsetlinewidth{1.003750pt}%
\definecolor{currentstroke}{rgb}{0.121569,0.466667,0.705882}%
\pgfsetstrokecolor{currentstroke}%
\pgfsetstrokeopacity{0.579209}%
\pgfsetdash{}{0pt}%
\pgfpathmoveto{\pgfqpoint{0.898154in}{1.522668in}}%
\pgfpathcurveto{\pgfqpoint{0.906390in}{1.522668in}}{\pgfqpoint{0.914290in}{1.525940in}}{\pgfqpoint{0.920114in}{1.531764in}}%
\pgfpathcurveto{\pgfqpoint{0.925938in}{1.537588in}}{\pgfqpoint{0.929210in}{1.545488in}}{\pgfqpoint{0.929210in}{1.553724in}}%
\pgfpathcurveto{\pgfqpoint{0.929210in}{1.561960in}}{\pgfqpoint{0.925938in}{1.569860in}}{\pgfqpoint{0.920114in}{1.575684in}}%
\pgfpathcurveto{\pgfqpoint{0.914290in}{1.581508in}}{\pgfqpoint{0.906390in}{1.584781in}}{\pgfqpoint{0.898154in}{1.584781in}}%
\pgfpathcurveto{\pgfqpoint{0.889917in}{1.584781in}}{\pgfqpoint{0.882017in}{1.581508in}}{\pgfqpoint{0.876193in}{1.575684in}}%
\pgfpathcurveto{\pgfqpoint{0.870370in}{1.569860in}}{\pgfqpoint{0.867097in}{1.561960in}}{\pgfqpoint{0.867097in}{1.553724in}}%
\pgfpathcurveto{\pgfqpoint{0.867097in}{1.545488in}}{\pgfqpoint{0.870370in}{1.537588in}}{\pgfqpoint{0.876193in}{1.531764in}}%
\pgfpathcurveto{\pgfqpoint{0.882017in}{1.525940in}}{\pgfqpoint{0.889917in}{1.522668in}}{\pgfqpoint{0.898154in}{1.522668in}}%
\pgfpathclose%
\pgfusepath{stroke,fill}%
\end{pgfscope}%
\begin{pgfscope}%
\pgfpathrectangle{\pgfqpoint{0.100000in}{0.212622in}}{\pgfqpoint{3.696000in}{3.696000in}}%
\pgfusepath{clip}%
\pgfsetbuttcap%
\pgfsetroundjoin%
\definecolor{currentfill}{rgb}{0.121569,0.466667,0.705882}%
\pgfsetfillcolor{currentfill}%
\pgfsetfillopacity{0.579398}%
\pgfsetlinewidth{1.003750pt}%
\definecolor{currentstroke}{rgb}{0.121569,0.466667,0.705882}%
\pgfsetstrokecolor{currentstroke}%
\pgfsetstrokeopacity{0.579398}%
\pgfsetdash{}{0pt}%
\pgfpathmoveto{\pgfqpoint{0.885949in}{1.554477in}}%
\pgfpathcurveto{\pgfqpoint{0.894186in}{1.554477in}}{\pgfqpoint{0.902086in}{1.557750in}}{\pgfqpoint{0.907910in}{1.563573in}}%
\pgfpathcurveto{\pgfqpoint{0.913733in}{1.569397in}}{\pgfqpoint{0.917006in}{1.577297in}}{\pgfqpoint{0.917006in}{1.585534in}}%
\pgfpathcurveto{\pgfqpoint{0.917006in}{1.593770in}}{\pgfqpoint{0.913733in}{1.601670in}}{\pgfqpoint{0.907910in}{1.607494in}}%
\pgfpathcurveto{\pgfqpoint{0.902086in}{1.613318in}}{\pgfqpoint{0.894186in}{1.616590in}}{\pgfqpoint{0.885949in}{1.616590in}}%
\pgfpathcurveto{\pgfqpoint{0.877713in}{1.616590in}}{\pgfqpoint{0.869813in}{1.613318in}}{\pgfqpoint{0.863989in}{1.607494in}}%
\pgfpathcurveto{\pgfqpoint{0.858165in}{1.601670in}}{\pgfqpoint{0.854893in}{1.593770in}}{\pgfqpoint{0.854893in}{1.585534in}}%
\pgfpathcurveto{\pgfqpoint{0.854893in}{1.577297in}}{\pgfqpoint{0.858165in}{1.569397in}}{\pgfqpoint{0.863989in}{1.563573in}}%
\pgfpathcurveto{\pgfqpoint{0.869813in}{1.557750in}}{\pgfqpoint{0.877713in}{1.554477in}}{\pgfqpoint{0.885949in}{1.554477in}}%
\pgfpathclose%
\pgfusepath{stroke,fill}%
\end{pgfscope}%
\begin{pgfscope}%
\pgfpathrectangle{\pgfqpoint{0.100000in}{0.212622in}}{\pgfqpoint{3.696000in}{3.696000in}}%
\pgfusepath{clip}%
\pgfsetbuttcap%
\pgfsetroundjoin%
\definecolor{currentfill}{rgb}{0.121569,0.466667,0.705882}%
\pgfsetfillcolor{currentfill}%
\pgfsetfillopacity{0.579479}%
\pgfsetlinewidth{1.003750pt}%
\definecolor{currentstroke}{rgb}{0.121569,0.466667,0.705882}%
\pgfsetstrokecolor{currentstroke}%
\pgfsetstrokeopacity{0.579479}%
\pgfsetdash{}{0pt}%
\pgfpathmoveto{\pgfqpoint{0.897895in}{1.521141in}}%
\pgfpathcurveto{\pgfqpoint{0.906131in}{1.521141in}}{\pgfqpoint{0.914031in}{1.524414in}}{\pgfqpoint{0.919855in}{1.530238in}}%
\pgfpathcurveto{\pgfqpoint{0.925679in}{1.536061in}}{\pgfqpoint{0.928951in}{1.543962in}}{\pgfqpoint{0.928951in}{1.552198in}}%
\pgfpathcurveto{\pgfqpoint{0.928951in}{1.560434in}}{\pgfqpoint{0.925679in}{1.568334in}}{\pgfqpoint{0.919855in}{1.574158in}}%
\pgfpathcurveto{\pgfqpoint{0.914031in}{1.579982in}}{\pgfqpoint{0.906131in}{1.583254in}}{\pgfqpoint{0.897895in}{1.583254in}}%
\pgfpathcurveto{\pgfqpoint{0.889658in}{1.583254in}}{\pgfqpoint{0.881758in}{1.579982in}}{\pgfqpoint{0.875934in}{1.574158in}}%
\pgfpathcurveto{\pgfqpoint{0.870110in}{1.568334in}}{\pgfqpoint{0.866838in}{1.560434in}}{\pgfqpoint{0.866838in}{1.552198in}}%
\pgfpathcurveto{\pgfqpoint{0.866838in}{1.543962in}}{\pgfqpoint{0.870110in}{1.536061in}}{\pgfqpoint{0.875934in}{1.530238in}}%
\pgfpathcurveto{\pgfqpoint{0.881758in}{1.524414in}}{\pgfqpoint{0.889658in}{1.521141in}}{\pgfqpoint{0.897895in}{1.521141in}}%
\pgfpathclose%
\pgfusepath{stroke,fill}%
\end{pgfscope}%
\begin{pgfscope}%
\pgfpathrectangle{\pgfqpoint{0.100000in}{0.212622in}}{\pgfqpoint{3.696000in}{3.696000in}}%
\pgfusepath{clip}%
\pgfsetbuttcap%
\pgfsetroundjoin%
\definecolor{currentfill}{rgb}{0.121569,0.466667,0.705882}%
\pgfsetfillcolor{currentfill}%
\pgfsetfillopacity{0.579573}%
\pgfsetlinewidth{1.003750pt}%
\definecolor{currentstroke}{rgb}{0.121569,0.466667,0.705882}%
\pgfsetstrokecolor{currentstroke}%
\pgfsetstrokeopacity{0.579573}%
\pgfsetdash{}{0pt}%
\pgfpathmoveto{\pgfqpoint{1.015268in}{1.784769in}}%
\pgfpathcurveto{\pgfqpoint{1.023504in}{1.784769in}}{\pgfqpoint{1.031404in}{1.788041in}}{\pgfqpoint{1.037228in}{1.793865in}}%
\pgfpathcurveto{\pgfqpoint{1.043052in}{1.799689in}}{\pgfqpoint{1.046324in}{1.807589in}}{\pgfqpoint{1.046324in}{1.815826in}}%
\pgfpathcurveto{\pgfqpoint{1.046324in}{1.824062in}}{\pgfqpoint{1.043052in}{1.831962in}}{\pgfqpoint{1.037228in}{1.837786in}}%
\pgfpathcurveto{\pgfqpoint{1.031404in}{1.843610in}}{\pgfqpoint{1.023504in}{1.846882in}}{\pgfqpoint{1.015268in}{1.846882in}}%
\pgfpathcurveto{\pgfqpoint{1.007031in}{1.846882in}}{\pgfqpoint{0.999131in}{1.843610in}}{\pgfqpoint{0.993307in}{1.837786in}}%
\pgfpathcurveto{\pgfqpoint{0.987483in}{1.831962in}}{\pgfqpoint{0.984211in}{1.824062in}}{\pgfqpoint{0.984211in}{1.815826in}}%
\pgfpathcurveto{\pgfqpoint{0.984211in}{1.807589in}}{\pgfqpoint{0.987483in}{1.799689in}}{\pgfqpoint{0.993307in}{1.793865in}}%
\pgfpathcurveto{\pgfqpoint{0.999131in}{1.788041in}}{\pgfqpoint{1.007031in}{1.784769in}}{\pgfqpoint{1.015268in}{1.784769in}}%
\pgfpathclose%
\pgfusepath{stroke,fill}%
\end{pgfscope}%
\begin{pgfscope}%
\pgfpathrectangle{\pgfqpoint{0.100000in}{0.212622in}}{\pgfqpoint{3.696000in}{3.696000in}}%
\pgfusepath{clip}%
\pgfsetbuttcap%
\pgfsetroundjoin%
\definecolor{currentfill}{rgb}{0.121569,0.466667,0.705882}%
\pgfsetfillcolor{currentfill}%
\pgfsetfillopacity{0.579627}%
\pgfsetlinewidth{1.003750pt}%
\definecolor{currentstroke}{rgb}{0.121569,0.466667,0.705882}%
\pgfsetstrokecolor{currentstroke}%
\pgfsetstrokeopacity{0.579627}%
\pgfsetdash{}{0pt}%
\pgfpathmoveto{\pgfqpoint{0.897749in}{1.520299in}}%
\pgfpathcurveto{\pgfqpoint{0.905985in}{1.520299in}}{\pgfqpoint{0.913885in}{1.523571in}}{\pgfqpoint{0.919709in}{1.529395in}}%
\pgfpathcurveto{\pgfqpoint{0.925533in}{1.535219in}}{\pgfqpoint{0.928805in}{1.543119in}}{\pgfqpoint{0.928805in}{1.551356in}}%
\pgfpathcurveto{\pgfqpoint{0.928805in}{1.559592in}}{\pgfqpoint{0.925533in}{1.567492in}}{\pgfqpoint{0.919709in}{1.573316in}}%
\pgfpathcurveto{\pgfqpoint{0.913885in}{1.579140in}}{\pgfqpoint{0.905985in}{1.582412in}}{\pgfqpoint{0.897749in}{1.582412in}}%
\pgfpathcurveto{\pgfqpoint{0.889512in}{1.582412in}}{\pgfqpoint{0.881612in}{1.579140in}}{\pgfqpoint{0.875788in}{1.573316in}}%
\pgfpathcurveto{\pgfqpoint{0.869964in}{1.567492in}}{\pgfqpoint{0.866692in}{1.559592in}}{\pgfqpoint{0.866692in}{1.551356in}}%
\pgfpathcurveto{\pgfqpoint{0.866692in}{1.543119in}}{\pgfqpoint{0.869964in}{1.535219in}}{\pgfqpoint{0.875788in}{1.529395in}}%
\pgfpathcurveto{\pgfqpoint{0.881612in}{1.523571in}}{\pgfqpoint{0.889512in}{1.520299in}}{\pgfqpoint{0.897749in}{1.520299in}}%
\pgfpathclose%
\pgfusepath{stroke,fill}%
\end{pgfscope}%
\begin{pgfscope}%
\pgfpathrectangle{\pgfqpoint{0.100000in}{0.212622in}}{\pgfqpoint{3.696000in}{3.696000in}}%
\pgfusepath{clip}%
\pgfsetbuttcap%
\pgfsetroundjoin%
\definecolor{currentfill}{rgb}{0.121569,0.466667,0.705882}%
\pgfsetfillcolor{currentfill}%
\pgfsetfillopacity{0.579706}%
\pgfsetlinewidth{1.003750pt}%
\definecolor{currentstroke}{rgb}{0.121569,0.466667,0.705882}%
\pgfsetstrokecolor{currentstroke}%
\pgfsetstrokeopacity{0.579706}%
\pgfsetdash{}{0pt}%
\pgfpathmoveto{\pgfqpoint{0.897665in}{1.519829in}}%
\pgfpathcurveto{\pgfqpoint{0.905902in}{1.519829in}}{\pgfqpoint{0.913802in}{1.523101in}}{\pgfqpoint{0.919626in}{1.528925in}}%
\pgfpathcurveto{\pgfqpoint{0.925450in}{1.534749in}}{\pgfqpoint{0.928722in}{1.542649in}}{\pgfqpoint{0.928722in}{1.550886in}}%
\pgfpathcurveto{\pgfqpoint{0.928722in}{1.559122in}}{\pgfqpoint{0.925450in}{1.567022in}}{\pgfqpoint{0.919626in}{1.572846in}}%
\pgfpathcurveto{\pgfqpoint{0.913802in}{1.578670in}}{\pgfqpoint{0.905902in}{1.581942in}}{\pgfqpoint{0.897665in}{1.581942in}}%
\pgfpathcurveto{\pgfqpoint{0.889429in}{1.581942in}}{\pgfqpoint{0.881529in}{1.578670in}}{\pgfqpoint{0.875705in}{1.572846in}}%
\pgfpathcurveto{\pgfqpoint{0.869881in}{1.567022in}}{\pgfqpoint{0.866609in}{1.559122in}}{\pgfqpoint{0.866609in}{1.550886in}}%
\pgfpathcurveto{\pgfqpoint{0.866609in}{1.542649in}}{\pgfqpoint{0.869881in}{1.534749in}}{\pgfqpoint{0.875705in}{1.528925in}}%
\pgfpathcurveto{\pgfqpoint{0.881529in}{1.523101in}}{\pgfqpoint{0.889429in}{1.519829in}}{\pgfqpoint{0.897665in}{1.519829in}}%
\pgfpathclose%
\pgfusepath{stroke,fill}%
\end{pgfscope}%
\begin{pgfscope}%
\pgfpathrectangle{\pgfqpoint{0.100000in}{0.212622in}}{\pgfqpoint{3.696000in}{3.696000in}}%
\pgfusepath{clip}%
\pgfsetbuttcap%
\pgfsetroundjoin%
\definecolor{currentfill}{rgb}{0.121569,0.466667,0.705882}%
\pgfsetfillcolor{currentfill}%
\pgfsetfillopacity{0.579911}%
\pgfsetlinewidth{1.003750pt}%
\definecolor{currentstroke}{rgb}{0.121569,0.466667,0.705882}%
\pgfsetstrokecolor{currentstroke}%
\pgfsetstrokeopacity{0.579911}%
\pgfsetdash{}{0pt}%
\pgfpathmoveto{\pgfqpoint{0.897460in}{1.518594in}}%
\pgfpathcurveto{\pgfqpoint{0.905697in}{1.518594in}}{\pgfqpoint{0.913597in}{1.521866in}}{\pgfqpoint{0.919421in}{1.527690in}}%
\pgfpathcurveto{\pgfqpoint{0.925245in}{1.533514in}}{\pgfqpoint{0.928517in}{1.541414in}}{\pgfqpoint{0.928517in}{1.549650in}}%
\pgfpathcurveto{\pgfqpoint{0.928517in}{1.557886in}}{\pgfqpoint{0.925245in}{1.565786in}}{\pgfqpoint{0.919421in}{1.571610in}}%
\pgfpathcurveto{\pgfqpoint{0.913597in}{1.577434in}}{\pgfqpoint{0.905697in}{1.580707in}}{\pgfqpoint{0.897460in}{1.580707in}}%
\pgfpathcurveto{\pgfqpoint{0.889224in}{1.580707in}}{\pgfqpoint{0.881324in}{1.577434in}}{\pgfqpoint{0.875500in}{1.571610in}}%
\pgfpathcurveto{\pgfqpoint{0.869676in}{1.565786in}}{\pgfqpoint{0.866404in}{1.557886in}}{\pgfqpoint{0.866404in}{1.549650in}}%
\pgfpathcurveto{\pgfqpoint{0.866404in}{1.541414in}}{\pgfqpoint{0.869676in}{1.533514in}}{\pgfqpoint{0.875500in}{1.527690in}}%
\pgfpathcurveto{\pgfqpoint{0.881324in}{1.521866in}}{\pgfqpoint{0.889224in}{1.518594in}}{\pgfqpoint{0.897460in}{1.518594in}}%
\pgfpathclose%
\pgfusepath{stroke,fill}%
\end{pgfscope}%
\begin{pgfscope}%
\pgfpathrectangle{\pgfqpoint{0.100000in}{0.212622in}}{\pgfqpoint{3.696000in}{3.696000in}}%
\pgfusepath{clip}%
\pgfsetbuttcap%
\pgfsetroundjoin%
\definecolor{currentfill}{rgb}{0.121569,0.466667,0.705882}%
\pgfsetfillcolor{currentfill}%
\pgfsetfillopacity{0.580023}%
\pgfsetlinewidth{1.003750pt}%
\definecolor{currentstroke}{rgb}{0.121569,0.466667,0.705882}%
\pgfsetstrokecolor{currentstroke}%
\pgfsetstrokeopacity{0.580023}%
\pgfsetdash{}{0pt}%
\pgfpathmoveto{\pgfqpoint{0.897347in}{1.517908in}}%
\pgfpathcurveto{\pgfqpoint{0.905583in}{1.517908in}}{\pgfqpoint{0.913483in}{1.521181in}}{\pgfqpoint{0.919307in}{1.527005in}}%
\pgfpathcurveto{\pgfqpoint{0.925131in}{1.532828in}}{\pgfqpoint{0.928403in}{1.540729in}}{\pgfqpoint{0.928403in}{1.548965in}}%
\pgfpathcurveto{\pgfqpoint{0.928403in}{1.557201in}}{\pgfqpoint{0.925131in}{1.565101in}}{\pgfqpoint{0.919307in}{1.570925in}}%
\pgfpathcurveto{\pgfqpoint{0.913483in}{1.576749in}}{\pgfqpoint{0.905583in}{1.580021in}}{\pgfqpoint{0.897347in}{1.580021in}}%
\pgfpathcurveto{\pgfqpoint{0.889111in}{1.580021in}}{\pgfqpoint{0.881211in}{1.576749in}}{\pgfqpoint{0.875387in}{1.570925in}}%
\pgfpathcurveto{\pgfqpoint{0.869563in}{1.565101in}}{\pgfqpoint{0.866290in}{1.557201in}}{\pgfqpoint{0.866290in}{1.548965in}}%
\pgfpathcurveto{\pgfqpoint{0.866290in}{1.540729in}}{\pgfqpoint{0.869563in}{1.532828in}}{\pgfqpoint{0.875387in}{1.527005in}}%
\pgfpathcurveto{\pgfqpoint{0.881211in}{1.521181in}}{\pgfqpoint{0.889111in}{1.517908in}}{\pgfqpoint{0.897347in}{1.517908in}}%
\pgfpathclose%
\pgfusepath{stroke,fill}%
\end{pgfscope}%
\begin{pgfscope}%
\pgfpathrectangle{\pgfqpoint{0.100000in}{0.212622in}}{\pgfqpoint{3.696000in}{3.696000in}}%
\pgfusepath{clip}%
\pgfsetbuttcap%
\pgfsetroundjoin%
\definecolor{currentfill}{rgb}{0.121569,0.466667,0.705882}%
\pgfsetfillcolor{currentfill}%
\pgfsetfillopacity{0.580204}%
\pgfsetlinewidth{1.003750pt}%
\definecolor{currentstroke}{rgb}{0.121569,0.466667,0.705882}%
\pgfsetstrokecolor{currentstroke}%
\pgfsetstrokeopacity{0.580204}%
\pgfsetdash{}{0pt}%
\pgfpathmoveto{\pgfqpoint{0.897140in}{1.516724in}}%
\pgfpathcurveto{\pgfqpoint{0.905376in}{1.516724in}}{\pgfqpoint{0.913276in}{1.519996in}}{\pgfqpoint{0.919100in}{1.525820in}}%
\pgfpathcurveto{\pgfqpoint{0.924924in}{1.531644in}}{\pgfqpoint{0.928196in}{1.539544in}}{\pgfqpoint{0.928196in}{1.547781in}}%
\pgfpathcurveto{\pgfqpoint{0.928196in}{1.556017in}}{\pgfqpoint{0.924924in}{1.563917in}}{\pgfqpoint{0.919100in}{1.569741in}}%
\pgfpathcurveto{\pgfqpoint{0.913276in}{1.575565in}}{\pgfqpoint{0.905376in}{1.578837in}}{\pgfqpoint{0.897140in}{1.578837in}}%
\pgfpathcurveto{\pgfqpoint{0.888904in}{1.578837in}}{\pgfqpoint{0.881004in}{1.575565in}}{\pgfqpoint{0.875180in}{1.569741in}}%
\pgfpathcurveto{\pgfqpoint{0.869356in}{1.563917in}}{\pgfqpoint{0.866083in}{1.556017in}}{\pgfqpoint{0.866083in}{1.547781in}}%
\pgfpathcurveto{\pgfqpoint{0.866083in}{1.539544in}}{\pgfqpoint{0.869356in}{1.531644in}}{\pgfqpoint{0.875180in}{1.525820in}}%
\pgfpathcurveto{\pgfqpoint{0.881004in}{1.519996in}}{\pgfqpoint{0.888904in}{1.516724in}}{\pgfqpoint{0.897140in}{1.516724in}}%
\pgfpathclose%
\pgfusepath{stroke,fill}%
\end{pgfscope}%
\begin{pgfscope}%
\pgfpathrectangle{\pgfqpoint{0.100000in}{0.212622in}}{\pgfqpoint{3.696000in}{3.696000in}}%
\pgfusepath{clip}%
\pgfsetbuttcap%
\pgfsetroundjoin%
\definecolor{currentfill}{rgb}{0.121569,0.466667,0.705882}%
\pgfsetfillcolor{currentfill}%
\pgfsetfillopacity{0.580309}%
\pgfsetlinewidth{1.003750pt}%
\definecolor{currentstroke}{rgb}{0.121569,0.466667,0.705882}%
\pgfsetstrokecolor{currentstroke}%
\pgfsetstrokeopacity{0.580309}%
\pgfsetdash{}{0pt}%
\pgfpathmoveto{\pgfqpoint{0.897028in}{1.516092in}}%
\pgfpathcurveto{\pgfqpoint{0.905264in}{1.516092in}}{\pgfqpoint{0.913164in}{1.519364in}}{\pgfqpoint{0.918988in}{1.525188in}}%
\pgfpathcurveto{\pgfqpoint{0.924812in}{1.531012in}}{\pgfqpoint{0.928084in}{1.538912in}}{\pgfqpoint{0.928084in}{1.547149in}}%
\pgfpathcurveto{\pgfqpoint{0.928084in}{1.555385in}}{\pgfqpoint{0.924812in}{1.563285in}}{\pgfqpoint{0.918988in}{1.569109in}}%
\pgfpathcurveto{\pgfqpoint{0.913164in}{1.574933in}}{\pgfqpoint{0.905264in}{1.578205in}}{\pgfqpoint{0.897028in}{1.578205in}}%
\pgfpathcurveto{\pgfqpoint{0.888791in}{1.578205in}}{\pgfqpoint{0.880891in}{1.574933in}}{\pgfqpoint{0.875067in}{1.569109in}}%
\pgfpathcurveto{\pgfqpoint{0.869243in}{1.563285in}}{\pgfqpoint{0.865971in}{1.555385in}}{\pgfqpoint{0.865971in}{1.547149in}}%
\pgfpathcurveto{\pgfqpoint{0.865971in}{1.538912in}}{\pgfqpoint{0.869243in}{1.531012in}}{\pgfqpoint{0.875067in}{1.525188in}}%
\pgfpathcurveto{\pgfqpoint{0.880891in}{1.519364in}}{\pgfqpoint{0.888791in}{1.516092in}}{\pgfqpoint{0.897028in}{1.516092in}}%
\pgfpathclose%
\pgfusepath{stroke,fill}%
\end{pgfscope}%
\begin{pgfscope}%
\pgfpathrectangle{\pgfqpoint{0.100000in}{0.212622in}}{\pgfqpoint{3.696000in}{3.696000in}}%
\pgfusepath{clip}%
\pgfsetbuttcap%
\pgfsetroundjoin%
\definecolor{currentfill}{rgb}{0.121569,0.466667,0.705882}%
\pgfsetfillcolor{currentfill}%
\pgfsetfillopacity{0.580364}%
\pgfsetlinewidth{1.003750pt}%
\definecolor{currentstroke}{rgb}{0.121569,0.466667,0.705882}%
\pgfsetstrokecolor{currentstroke}%
\pgfsetstrokeopacity{0.580364}%
\pgfsetdash{}{0pt}%
\pgfpathmoveto{\pgfqpoint{0.896960in}{1.515735in}}%
\pgfpathcurveto{\pgfqpoint{0.905196in}{1.515735in}}{\pgfqpoint{0.913096in}{1.519007in}}{\pgfqpoint{0.918920in}{1.524831in}}%
\pgfpathcurveto{\pgfqpoint{0.924744in}{1.530655in}}{\pgfqpoint{0.928016in}{1.538555in}}{\pgfqpoint{0.928016in}{1.546792in}}%
\pgfpathcurveto{\pgfqpoint{0.928016in}{1.555028in}}{\pgfqpoint{0.924744in}{1.562928in}}{\pgfqpoint{0.918920in}{1.568752in}}%
\pgfpathcurveto{\pgfqpoint{0.913096in}{1.574576in}}{\pgfqpoint{0.905196in}{1.577848in}}{\pgfqpoint{0.896960in}{1.577848in}}%
\pgfpathcurveto{\pgfqpoint{0.888724in}{1.577848in}}{\pgfqpoint{0.880824in}{1.574576in}}{\pgfqpoint{0.875000in}{1.568752in}}%
\pgfpathcurveto{\pgfqpoint{0.869176in}{1.562928in}}{\pgfqpoint{0.865903in}{1.555028in}}{\pgfqpoint{0.865903in}{1.546792in}}%
\pgfpathcurveto{\pgfqpoint{0.865903in}{1.538555in}}{\pgfqpoint{0.869176in}{1.530655in}}{\pgfqpoint{0.875000in}{1.524831in}}%
\pgfpathcurveto{\pgfqpoint{0.880824in}{1.519007in}}{\pgfqpoint{0.888724in}{1.515735in}}{\pgfqpoint{0.896960in}{1.515735in}}%
\pgfpathclose%
\pgfusepath{stroke,fill}%
\end{pgfscope}%
\begin{pgfscope}%
\pgfpathrectangle{\pgfqpoint{0.100000in}{0.212622in}}{\pgfqpoint{3.696000in}{3.696000in}}%
\pgfusepath{clip}%
\pgfsetbuttcap%
\pgfsetroundjoin%
\definecolor{currentfill}{rgb}{0.121569,0.466667,0.705882}%
\pgfsetfillcolor{currentfill}%
\pgfsetfillopacity{0.580412}%
\pgfsetlinewidth{1.003750pt}%
\definecolor{currentstroke}{rgb}{0.121569,0.466667,0.705882}%
\pgfsetstrokecolor{currentstroke}%
\pgfsetstrokeopacity{0.580412}%
\pgfsetdash{}{0pt}%
\pgfpathmoveto{\pgfqpoint{2.088878in}{2.172530in}}%
\pgfpathcurveto{\pgfqpoint{2.097114in}{2.172530in}}{\pgfqpoint{2.105014in}{2.175802in}}{\pgfqpoint{2.110838in}{2.181626in}}%
\pgfpathcurveto{\pgfqpoint{2.116662in}{2.187450in}}{\pgfqpoint{2.119934in}{2.195350in}}{\pgfqpoint{2.119934in}{2.203587in}}%
\pgfpathcurveto{\pgfqpoint{2.119934in}{2.211823in}}{\pgfqpoint{2.116662in}{2.219723in}}{\pgfqpoint{2.110838in}{2.225547in}}%
\pgfpathcurveto{\pgfqpoint{2.105014in}{2.231371in}}{\pgfqpoint{2.097114in}{2.234643in}}{\pgfqpoint{2.088878in}{2.234643in}}%
\pgfpathcurveto{\pgfqpoint{2.080642in}{2.234643in}}{\pgfqpoint{2.072742in}{2.231371in}}{\pgfqpoint{2.066918in}{2.225547in}}%
\pgfpathcurveto{\pgfqpoint{2.061094in}{2.219723in}}{\pgfqpoint{2.057821in}{2.211823in}}{\pgfqpoint{2.057821in}{2.203587in}}%
\pgfpathcurveto{\pgfqpoint{2.057821in}{2.195350in}}{\pgfqpoint{2.061094in}{2.187450in}}{\pgfqpoint{2.066918in}{2.181626in}}%
\pgfpathcurveto{\pgfqpoint{2.072742in}{2.175802in}}{\pgfqpoint{2.080642in}{2.172530in}}{\pgfqpoint{2.088878in}{2.172530in}}%
\pgfpathclose%
\pgfusepath{stroke,fill}%
\end{pgfscope}%
\begin{pgfscope}%
\pgfpathrectangle{\pgfqpoint{0.100000in}{0.212622in}}{\pgfqpoint{3.696000in}{3.696000in}}%
\pgfusepath{clip}%
\pgfsetbuttcap%
\pgfsetroundjoin%
\definecolor{currentfill}{rgb}{0.121569,0.466667,0.705882}%
\pgfsetfillcolor{currentfill}%
\pgfsetfillopacity{0.580454}%
\pgfsetlinewidth{1.003750pt}%
\definecolor{currentstroke}{rgb}{0.121569,0.466667,0.705882}%
\pgfsetstrokecolor{currentstroke}%
\pgfsetstrokeopacity{0.580454}%
\pgfsetdash{}{0pt}%
\pgfpathmoveto{\pgfqpoint{1.012759in}{1.780331in}}%
\pgfpathcurveto{\pgfqpoint{1.020995in}{1.780331in}}{\pgfqpoint{1.028895in}{1.783604in}}{\pgfqpoint{1.034719in}{1.789428in}}%
\pgfpathcurveto{\pgfqpoint{1.040543in}{1.795252in}}{\pgfqpoint{1.043815in}{1.803152in}}{\pgfqpoint{1.043815in}{1.811388in}}%
\pgfpathcurveto{\pgfqpoint{1.043815in}{1.819624in}}{\pgfqpoint{1.040543in}{1.827524in}}{\pgfqpoint{1.034719in}{1.833348in}}%
\pgfpathcurveto{\pgfqpoint{1.028895in}{1.839172in}}{\pgfqpoint{1.020995in}{1.842444in}}{\pgfqpoint{1.012759in}{1.842444in}}%
\pgfpathcurveto{\pgfqpoint{1.004522in}{1.842444in}}{\pgfqpoint{0.996622in}{1.839172in}}{\pgfqpoint{0.990798in}{1.833348in}}%
\pgfpathcurveto{\pgfqpoint{0.984975in}{1.827524in}}{\pgfqpoint{0.981702in}{1.819624in}}{\pgfqpoint{0.981702in}{1.811388in}}%
\pgfpathcurveto{\pgfqpoint{0.981702in}{1.803152in}}{\pgfqpoint{0.984975in}{1.795252in}}{\pgfqpoint{0.990798in}{1.789428in}}%
\pgfpathcurveto{\pgfqpoint{0.996622in}{1.783604in}}{\pgfqpoint{1.004522in}{1.780331in}}{\pgfqpoint{1.012759in}{1.780331in}}%
\pgfpathclose%
\pgfusepath{stroke,fill}%
\end{pgfscope}%
\begin{pgfscope}%
\pgfpathrectangle{\pgfqpoint{0.100000in}{0.212622in}}{\pgfqpoint{3.696000in}{3.696000in}}%
\pgfusepath{clip}%
\pgfsetbuttcap%
\pgfsetroundjoin%
\definecolor{currentfill}{rgb}{0.121569,0.466667,0.705882}%
\pgfsetfillcolor{currentfill}%
\pgfsetfillopacity{0.580479}%
\pgfsetlinewidth{1.003750pt}%
\definecolor{currentstroke}{rgb}{0.121569,0.466667,0.705882}%
\pgfsetstrokecolor{currentstroke}%
\pgfsetstrokeopacity{0.580479}%
\pgfsetdash{}{0pt}%
\pgfpathmoveto{\pgfqpoint{0.896820in}{1.514948in}}%
\pgfpathcurveto{\pgfqpoint{0.905056in}{1.514948in}}{\pgfqpoint{0.912956in}{1.518220in}}{\pgfqpoint{0.918780in}{1.524044in}}%
\pgfpathcurveto{\pgfqpoint{0.924604in}{1.529868in}}{\pgfqpoint{0.927876in}{1.537768in}}{\pgfqpoint{0.927876in}{1.546005in}}%
\pgfpathcurveto{\pgfqpoint{0.927876in}{1.554241in}}{\pgfqpoint{0.924604in}{1.562141in}}{\pgfqpoint{0.918780in}{1.567965in}}%
\pgfpathcurveto{\pgfqpoint{0.912956in}{1.573789in}}{\pgfqpoint{0.905056in}{1.577061in}}{\pgfqpoint{0.896820in}{1.577061in}}%
\pgfpathcurveto{\pgfqpoint{0.888584in}{1.577061in}}{\pgfqpoint{0.880684in}{1.573789in}}{\pgfqpoint{0.874860in}{1.567965in}}%
\pgfpathcurveto{\pgfqpoint{0.869036in}{1.562141in}}{\pgfqpoint{0.865763in}{1.554241in}}{\pgfqpoint{0.865763in}{1.546005in}}%
\pgfpathcurveto{\pgfqpoint{0.865763in}{1.537768in}}{\pgfqpoint{0.869036in}{1.529868in}}{\pgfqpoint{0.874860in}{1.524044in}}%
\pgfpathcurveto{\pgfqpoint{0.880684in}{1.518220in}}{\pgfqpoint{0.888584in}{1.514948in}}{\pgfqpoint{0.896820in}{1.514948in}}%
\pgfpathclose%
\pgfusepath{stroke,fill}%
\end{pgfscope}%
\begin{pgfscope}%
\pgfpathrectangle{\pgfqpoint{0.100000in}{0.212622in}}{\pgfqpoint{3.696000in}{3.696000in}}%
\pgfusepath{clip}%
\pgfsetbuttcap%
\pgfsetroundjoin%
\definecolor{currentfill}{rgb}{0.121569,0.466667,0.705882}%
\pgfsetfillcolor{currentfill}%
\pgfsetfillopacity{0.580543}%
\pgfsetlinewidth{1.003750pt}%
\definecolor{currentstroke}{rgb}{0.121569,0.466667,0.705882}%
\pgfsetstrokecolor{currentstroke}%
\pgfsetstrokeopacity{0.580543}%
\pgfsetdash{}{0pt}%
\pgfpathmoveto{\pgfqpoint{0.896742in}{1.514518in}}%
\pgfpathcurveto{\pgfqpoint{0.904979in}{1.514518in}}{\pgfqpoint{0.912879in}{1.517790in}}{\pgfqpoint{0.918703in}{1.523614in}}%
\pgfpathcurveto{\pgfqpoint{0.924527in}{1.529438in}}{\pgfqpoint{0.927799in}{1.537338in}}{\pgfqpoint{0.927799in}{1.545574in}}%
\pgfpathcurveto{\pgfqpoint{0.927799in}{1.553811in}}{\pgfqpoint{0.924527in}{1.561711in}}{\pgfqpoint{0.918703in}{1.567535in}}%
\pgfpathcurveto{\pgfqpoint{0.912879in}{1.573358in}}{\pgfqpoint{0.904979in}{1.576631in}}{\pgfqpoint{0.896742in}{1.576631in}}%
\pgfpathcurveto{\pgfqpoint{0.888506in}{1.576631in}}{\pgfqpoint{0.880606in}{1.573358in}}{\pgfqpoint{0.874782in}{1.567535in}}%
\pgfpathcurveto{\pgfqpoint{0.868958in}{1.561711in}}{\pgfqpoint{0.865686in}{1.553811in}}{\pgfqpoint{0.865686in}{1.545574in}}%
\pgfpathcurveto{\pgfqpoint{0.865686in}{1.537338in}}{\pgfqpoint{0.868958in}{1.529438in}}{\pgfqpoint{0.874782in}{1.523614in}}%
\pgfpathcurveto{\pgfqpoint{0.880606in}{1.517790in}}{\pgfqpoint{0.888506in}{1.514518in}}{\pgfqpoint{0.896742in}{1.514518in}}%
\pgfpathclose%
\pgfusepath{stroke,fill}%
\end{pgfscope}%
\begin{pgfscope}%
\pgfpathrectangle{\pgfqpoint{0.100000in}{0.212622in}}{\pgfqpoint{3.696000in}{3.696000in}}%
\pgfusepath{clip}%
\pgfsetbuttcap%
\pgfsetroundjoin%
\definecolor{currentfill}{rgb}{0.121569,0.466667,0.705882}%
\pgfsetfillcolor{currentfill}%
\pgfsetfillopacity{0.580672}%
\pgfsetlinewidth{1.003750pt}%
\definecolor{currentstroke}{rgb}{0.121569,0.466667,0.705882}%
\pgfsetstrokecolor{currentstroke}%
\pgfsetstrokeopacity{0.580672}%
\pgfsetdash{}{0pt}%
\pgfpathmoveto{\pgfqpoint{0.896573in}{1.513638in}}%
\pgfpathcurveto{\pgfqpoint{0.904810in}{1.513638in}}{\pgfqpoint{0.912710in}{1.516910in}}{\pgfqpoint{0.918534in}{1.522734in}}%
\pgfpathcurveto{\pgfqpoint{0.924358in}{1.528558in}}{\pgfqpoint{0.927630in}{1.536458in}}{\pgfqpoint{0.927630in}{1.544695in}}%
\pgfpathcurveto{\pgfqpoint{0.927630in}{1.552931in}}{\pgfqpoint{0.924358in}{1.560831in}}{\pgfqpoint{0.918534in}{1.566655in}}%
\pgfpathcurveto{\pgfqpoint{0.912710in}{1.572479in}}{\pgfqpoint{0.904810in}{1.575751in}}{\pgfqpoint{0.896573in}{1.575751in}}%
\pgfpathcurveto{\pgfqpoint{0.888337in}{1.575751in}}{\pgfqpoint{0.880437in}{1.572479in}}{\pgfqpoint{0.874613in}{1.566655in}}%
\pgfpathcurveto{\pgfqpoint{0.868789in}{1.560831in}}{\pgfqpoint{0.865517in}{1.552931in}}{\pgfqpoint{0.865517in}{1.544695in}}%
\pgfpathcurveto{\pgfqpoint{0.865517in}{1.536458in}}{\pgfqpoint{0.868789in}{1.528558in}}{\pgfqpoint{0.874613in}{1.522734in}}%
\pgfpathcurveto{\pgfqpoint{0.880437in}{1.516910in}}{\pgfqpoint{0.888337in}{1.513638in}}{\pgfqpoint{0.896573in}{1.513638in}}%
\pgfpathclose%
\pgfusepath{stroke,fill}%
\end{pgfscope}%
\begin{pgfscope}%
\pgfpathrectangle{\pgfqpoint{0.100000in}{0.212622in}}{\pgfqpoint{3.696000in}{3.696000in}}%
\pgfusepath{clip}%
\pgfsetbuttcap%
\pgfsetroundjoin%
\definecolor{currentfill}{rgb}{0.121569,0.466667,0.705882}%
\pgfsetfillcolor{currentfill}%
\pgfsetfillopacity{0.580745}%
\pgfsetlinewidth{1.003750pt}%
\definecolor{currentstroke}{rgb}{0.121569,0.466667,0.705882}%
\pgfsetstrokecolor{currentstroke}%
\pgfsetstrokeopacity{0.580745}%
\pgfsetdash{}{0pt}%
\pgfpathmoveto{\pgfqpoint{0.896484in}{1.513162in}}%
\pgfpathcurveto{\pgfqpoint{0.904720in}{1.513162in}}{\pgfqpoint{0.912620in}{1.516434in}}{\pgfqpoint{0.918444in}{1.522258in}}%
\pgfpathcurveto{\pgfqpoint{0.924268in}{1.528082in}}{\pgfqpoint{0.927540in}{1.535982in}}{\pgfqpoint{0.927540in}{1.544219in}}%
\pgfpathcurveto{\pgfqpoint{0.927540in}{1.552455in}}{\pgfqpoint{0.924268in}{1.560355in}}{\pgfqpoint{0.918444in}{1.566179in}}%
\pgfpathcurveto{\pgfqpoint{0.912620in}{1.572003in}}{\pgfqpoint{0.904720in}{1.575275in}}{\pgfqpoint{0.896484in}{1.575275in}}%
\pgfpathcurveto{\pgfqpoint{0.888248in}{1.575275in}}{\pgfqpoint{0.880348in}{1.572003in}}{\pgfqpoint{0.874524in}{1.566179in}}%
\pgfpathcurveto{\pgfqpoint{0.868700in}{1.560355in}}{\pgfqpoint{0.865427in}{1.552455in}}{\pgfqpoint{0.865427in}{1.544219in}}%
\pgfpathcurveto{\pgfqpoint{0.865427in}{1.535982in}}{\pgfqpoint{0.868700in}{1.528082in}}{\pgfqpoint{0.874524in}{1.522258in}}%
\pgfpathcurveto{\pgfqpoint{0.880348in}{1.516434in}}{\pgfqpoint{0.888248in}{1.513162in}}{\pgfqpoint{0.896484in}{1.513162in}}%
\pgfpathclose%
\pgfusepath{stroke,fill}%
\end{pgfscope}%
\begin{pgfscope}%
\pgfpathrectangle{\pgfqpoint{0.100000in}{0.212622in}}{\pgfqpoint{3.696000in}{3.696000in}}%
\pgfusepath{clip}%
\pgfsetbuttcap%
\pgfsetroundjoin%
\definecolor{currentfill}{rgb}{0.121569,0.466667,0.705882}%
\pgfsetfillcolor{currentfill}%
\pgfsetfillopacity{0.580878}%
\pgfsetlinewidth{1.003750pt}%
\definecolor{currentstroke}{rgb}{0.121569,0.466667,0.705882}%
\pgfsetstrokecolor{currentstroke}%
\pgfsetstrokeopacity{0.580878}%
\pgfsetdash{}{0pt}%
\pgfpathmoveto{\pgfqpoint{0.896322in}{1.512306in}}%
\pgfpathcurveto{\pgfqpoint{0.904558in}{1.512306in}}{\pgfqpoint{0.912458in}{1.515578in}}{\pgfqpoint{0.918282in}{1.521402in}}%
\pgfpathcurveto{\pgfqpoint{0.924106in}{1.527226in}}{\pgfqpoint{0.927378in}{1.535126in}}{\pgfqpoint{0.927378in}{1.543363in}}%
\pgfpathcurveto{\pgfqpoint{0.927378in}{1.551599in}}{\pgfqpoint{0.924106in}{1.559499in}}{\pgfqpoint{0.918282in}{1.565323in}}%
\pgfpathcurveto{\pgfqpoint{0.912458in}{1.571147in}}{\pgfqpoint{0.904558in}{1.574419in}}{\pgfqpoint{0.896322in}{1.574419in}}%
\pgfpathcurveto{\pgfqpoint{0.888086in}{1.574419in}}{\pgfqpoint{0.880186in}{1.571147in}}{\pgfqpoint{0.874362in}{1.565323in}}%
\pgfpathcurveto{\pgfqpoint{0.868538in}{1.559499in}}{\pgfqpoint{0.865265in}{1.551599in}}{\pgfqpoint{0.865265in}{1.543363in}}%
\pgfpathcurveto{\pgfqpoint{0.865265in}{1.535126in}}{\pgfqpoint{0.868538in}{1.527226in}}{\pgfqpoint{0.874362in}{1.521402in}}%
\pgfpathcurveto{\pgfqpoint{0.880186in}{1.515578in}}{\pgfqpoint{0.888086in}{1.512306in}}{\pgfqpoint{0.896322in}{1.512306in}}%
\pgfpathclose%
\pgfusepath{stroke,fill}%
\end{pgfscope}%
\begin{pgfscope}%
\pgfpathrectangle{\pgfqpoint{0.100000in}{0.212622in}}{\pgfqpoint{3.696000in}{3.696000in}}%
\pgfusepath{clip}%
\pgfsetbuttcap%
\pgfsetroundjoin%
\definecolor{currentfill}{rgb}{0.121569,0.466667,0.705882}%
\pgfsetfillcolor{currentfill}%
\pgfsetfillopacity{0.581072}%
\pgfsetlinewidth{1.003750pt}%
\definecolor{currentstroke}{rgb}{0.121569,0.466667,0.705882}%
\pgfsetstrokecolor{currentstroke}%
\pgfsetstrokeopacity{0.581072}%
\pgfsetdash{}{0pt}%
\pgfpathmoveto{\pgfqpoint{0.896068in}{1.511051in}}%
\pgfpathcurveto{\pgfqpoint{0.904304in}{1.511051in}}{\pgfqpoint{0.912204in}{1.514323in}}{\pgfqpoint{0.918028in}{1.520147in}}%
\pgfpathcurveto{\pgfqpoint{0.923852in}{1.525971in}}{\pgfqpoint{0.927125in}{1.533871in}}{\pgfqpoint{0.927125in}{1.542107in}}%
\pgfpathcurveto{\pgfqpoint{0.927125in}{1.550344in}}{\pgfqpoint{0.923852in}{1.558244in}}{\pgfqpoint{0.918028in}{1.564068in}}%
\pgfpathcurveto{\pgfqpoint{0.912204in}{1.569892in}}{\pgfqpoint{0.904304in}{1.573164in}}{\pgfqpoint{0.896068in}{1.573164in}}%
\pgfpathcurveto{\pgfqpoint{0.887832in}{1.573164in}}{\pgfqpoint{0.879932in}{1.569892in}}{\pgfqpoint{0.874108in}{1.564068in}}%
\pgfpathcurveto{\pgfqpoint{0.868284in}{1.558244in}}{\pgfqpoint{0.865012in}{1.550344in}}{\pgfqpoint{0.865012in}{1.542107in}}%
\pgfpathcurveto{\pgfqpoint{0.865012in}{1.533871in}}{\pgfqpoint{0.868284in}{1.525971in}}{\pgfqpoint{0.874108in}{1.520147in}}%
\pgfpathcurveto{\pgfqpoint{0.879932in}{1.514323in}}{\pgfqpoint{0.887832in}{1.511051in}}{\pgfqpoint{0.896068in}{1.511051in}}%
\pgfpathclose%
\pgfusepath{stroke,fill}%
\end{pgfscope}%
\begin{pgfscope}%
\pgfpathrectangle{\pgfqpoint{0.100000in}{0.212622in}}{\pgfqpoint{3.696000in}{3.696000in}}%
\pgfusepath{clip}%
\pgfsetbuttcap%
\pgfsetroundjoin%
\definecolor{currentfill}{rgb}{0.121569,0.466667,0.705882}%
\pgfsetfillcolor{currentfill}%
\pgfsetfillopacity{0.581179}%
\pgfsetlinewidth{1.003750pt}%
\definecolor{currentstroke}{rgb}{0.121569,0.466667,0.705882}%
\pgfsetstrokecolor{currentstroke}%
\pgfsetstrokeopacity{0.581179}%
\pgfsetdash{}{0pt}%
\pgfpathmoveto{\pgfqpoint{0.895928in}{1.510360in}}%
\pgfpathcurveto{\pgfqpoint{0.904165in}{1.510360in}}{\pgfqpoint{0.912065in}{1.513632in}}{\pgfqpoint{0.917889in}{1.519456in}}%
\pgfpathcurveto{\pgfqpoint{0.923713in}{1.525280in}}{\pgfqpoint{0.926985in}{1.533180in}}{\pgfqpoint{0.926985in}{1.541416in}}%
\pgfpathcurveto{\pgfqpoint{0.926985in}{1.549652in}}{\pgfqpoint{0.923713in}{1.557552in}}{\pgfqpoint{0.917889in}{1.563376in}}%
\pgfpathcurveto{\pgfqpoint{0.912065in}{1.569200in}}{\pgfqpoint{0.904165in}{1.572473in}}{\pgfqpoint{0.895928in}{1.572473in}}%
\pgfpathcurveto{\pgfqpoint{0.887692in}{1.572473in}}{\pgfqpoint{0.879792in}{1.569200in}}{\pgfqpoint{0.873968in}{1.563376in}}%
\pgfpathcurveto{\pgfqpoint{0.868144in}{1.557552in}}{\pgfqpoint{0.864872in}{1.549652in}}{\pgfqpoint{0.864872in}{1.541416in}}%
\pgfpathcurveto{\pgfqpoint{0.864872in}{1.533180in}}{\pgfqpoint{0.868144in}{1.525280in}}{\pgfqpoint{0.873968in}{1.519456in}}%
\pgfpathcurveto{\pgfqpoint{0.879792in}{1.513632in}}{\pgfqpoint{0.887692in}{1.510360in}}{\pgfqpoint{0.895928in}{1.510360in}}%
\pgfpathclose%
\pgfusepath{stroke,fill}%
\end{pgfscope}%
\begin{pgfscope}%
\pgfpathrectangle{\pgfqpoint{0.100000in}{0.212622in}}{\pgfqpoint{3.696000in}{3.696000in}}%
\pgfusepath{clip}%
\pgfsetbuttcap%
\pgfsetroundjoin%
\definecolor{currentfill}{rgb}{0.121569,0.466667,0.705882}%
\pgfsetfillcolor{currentfill}%
\pgfsetfillopacity{0.581346}%
\pgfsetlinewidth{1.003750pt}%
\definecolor{currentstroke}{rgb}{0.121569,0.466667,0.705882}%
\pgfsetstrokecolor{currentstroke}%
\pgfsetstrokeopacity{0.581346}%
\pgfsetdash{}{0pt}%
\pgfpathmoveto{\pgfqpoint{0.895706in}{1.509289in}}%
\pgfpathcurveto{\pgfqpoint{0.903942in}{1.509289in}}{\pgfqpoint{0.911842in}{1.512561in}}{\pgfqpoint{0.917666in}{1.518385in}}%
\pgfpathcurveto{\pgfqpoint{0.923490in}{1.524209in}}{\pgfqpoint{0.926763in}{1.532109in}}{\pgfqpoint{0.926763in}{1.540345in}}%
\pgfpathcurveto{\pgfqpoint{0.926763in}{1.548581in}}{\pgfqpoint{0.923490in}{1.556482in}}{\pgfqpoint{0.917666in}{1.562305in}}%
\pgfpathcurveto{\pgfqpoint{0.911842in}{1.568129in}}{\pgfqpoint{0.903942in}{1.571402in}}{\pgfqpoint{0.895706in}{1.571402in}}%
\pgfpathcurveto{\pgfqpoint{0.887470in}{1.571402in}}{\pgfqpoint{0.879570in}{1.568129in}}{\pgfqpoint{0.873746in}{1.562305in}}%
\pgfpathcurveto{\pgfqpoint{0.867922in}{1.556482in}}{\pgfqpoint{0.864650in}{1.548581in}}{\pgfqpoint{0.864650in}{1.540345in}}%
\pgfpathcurveto{\pgfqpoint{0.864650in}{1.532109in}}{\pgfqpoint{0.867922in}{1.524209in}}{\pgfqpoint{0.873746in}{1.518385in}}%
\pgfpathcurveto{\pgfqpoint{0.879570in}{1.512561in}}{\pgfqpoint{0.887470in}{1.509289in}}{\pgfqpoint{0.895706in}{1.509289in}}%
\pgfpathclose%
\pgfusepath{stroke,fill}%
\end{pgfscope}%
\begin{pgfscope}%
\pgfpathrectangle{\pgfqpoint{0.100000in}{0.212622in}}{\pgfqpoint{3.696000in}{3.696000in}}%
\pgfusepath{clip}%
\pgfsetbuttcap%
\pgfsetroundjoin%
\definecolor{currentfill}{rgb}{0.121569,0.466667,0.705882}%
\pgfsetfillcolor{currentfill}%
\pgfsetfillopacity{0.581581}%
\pgfsetlinewidth{1.003750pt}%
\definecolor{currentstroke}{rgb}{0.121569,0.466667,0.705882}%
\pgfsetstrokecolor{currentstroke}%
\pgfsetstrokeopacity{0.581581}%
\pgfsetdash{}{0pt}%
\pgfpathmoveto{\pgfqpoint{0.895404in}{1.507825in}}%
\pgfpathcurveto{\pgfqpoint{0.903641in}{1.507825in}}{\pgfqpoint{0.911541in}{1.511097in}}{\pgfqpoint{0.917365in}{1.516921in}}%
\pgfpathcurveto{\pgfqpoint{0.923189in}{1.522745in}}{\pgfqpoint{0.926461in}{1.530645in}}{\pgfqpoint{0.926461in}{1.538881in}}%
\pgfpathcurveto{\pgfqpoint{0.926461in}{1.547117in}}{\pgfqpoint{0.923189in}{1.555017in}}{\pgfqpoint{0.917365in}{1.560841in}}%
\pgfpathcurveto{\pgfqpoint{0.911541in}{1.566665in}}{\pgfqpoint{0.903641in}{1.569938in}}{\pgfqpoint{0.895404in}{1.569938in}}%
\pgfpathcurveto{\pgfqpoint{0.887168in}{1.569938in}}{\pgfqpoint{0.879268in}{1.566665in}}{\pgfqpoint{0.873444in}{1.560841in}}%
\pgfpathcurveto{\pgfqpoint{0.867620in}{1.555017in}}{\pgfqpoint{0.864348in}{1.547117in}}{\pgfqpoint{0.864348in}{1.538881in}}%
\pgfpathcurveto{\pgfqpoint{0.864348in}{1.530645in}}{\pgfqpoint{0.867620in}{1.522745in}}{\pgfqpoint{0.873444in}{1.516921in}}%
\pgfpathcurveto{\pgfqpoint{0.879268in}{1.511097in}}{\pgfqpoint{0.887168in}{1.507825in}}{\pgfqpoint{0.895404in}{1.507825in}}%
\pgfpathclose%
\pgfusepath{stroke,fill}%
\end{pgfscope}%
\begin{pgfscope}%
\pgfpathrectangle{\pgfqpoint{0.100000in}{0.212622in}}{\pgfqpoint{3.696000in}{3.696000in}}%
\pgfusepath{clip}%
\pgfsetbuttcap%
\pgfsetroundjoin%
\definecolor{currentfill}{rgb}{0.121569,0.466667,0.705882}%
\pgfsetfillcolor{currentfill}%
\pgfsetfillopacity{0.581594}%
\pgfsetlinewidth{1.003750pt}%
\definecolor{currentstroke}{rgb}{0.121569,0.466667,0.705882}%
\pgfsetstrokecolor{currentstroke}%
\pgfsetstrokeopacity{0.581594}%
\pgfsetdash{}{0pt}%
\pgfpathmoveto{\pgfqpoint{0.881102in}{1.561504in}}%
\pgfpathcurveto{\pgfqpoint{0.889338in}{1.561504in}}{\pgfqpoint{0.897238in}{1.564777in}}{\pgfqpoint{0.903062in}{1.570601in}}%
\pgfpathcurveto{\pgfqpoint{0.908886in}{1.576424in}}{\pgfqpoint{0.912159in}{1.584325in}}{\pgfqpoint{0.912159in}{1.592561in}}%
\pgfpathcurveto{\pgfqpoint{0.912159in}{1.600797in}}{\pgfqpoint{0.908886in}{1.608697in}}{\pgfqpoint{0.903062in}{1.614521in}}%
\pgfpathcurveto{\pgfqpoint{0.897238in}{1.620345in}}{\pgfqpoint{0.889338in}{1.623617in}}{\pgfqpoint{0.881102in}{1.623617in}}%
\pgfpathcurveto{\pgfqpoint{0.872866in}{1.623617in}}{\pgfqpoint{0.864966in}{1.620345in}}{\pgfqpoint{0.859142in}{1.614521in}}%
\pgfpathcurveto{\pgfqpoint{0.853318in}{1.608697in}}{\pgfqpoint{0.850046in}{1.600797in}}{\pgfqpoint{0.850046in}{1.592561in}}%
\pgfpathcurveto{\pgfqpoint{0.850046in}{1.584325in}}{\pgfqpoint{0.853318in}{1.576424in}}{\pgfqpoint{0.859142in}{1.570601in}}%
\pgfpathcurveto{\pgfqpoint{0.864966in}{1.564777in}}{\pgfqpoint{0.872866in}{1.561504in}}{\pgfqpoint{0.881102in}{1.561504in}}%
\pgfpathclose%
\pgfusepath{stroke,fill}%
\end{pgfscope}%
\begin{pgfscope}%
\pgfpathrectangle{\pgfqpoint{0.100000in}{0.212622in}}{\pgfqpoint{3.696000in}{3.696000in}}%
\pgfusepath{clip}%
\pgfsetbuttcap%
\pgfsetroundjoin%
\definecolor{currentfill}{rgb}{0.121569,0.466667,0.705882}%
\pgfsetfillcolor{currentfill}%
\pgfsetfillopacity{0.581717}%
\pgfsetlinewidth{1.003750pt}%
\definecolor{currentstroke}{rgb}{0.121569,0.466667,0.705882}%
\pgfsetstrokecolor{currentstroke}%
\pgfsetstrokeopacity{0.581717}%
\pgfsetdash{}{0pt}%
\pgfpathmoveto{\pgfqpoint{0.895242in}{1.507044in}}%
\pgfpathcurveto{\pgfqpoint{0.903478in}{1.507044in}}{\pgfqpoint{0.911378in}{1.510316in}}{\pgfqpoint{0.917202in}{1.516140in}}%
\pgfpathcurveto{\pgfqpoint{0.923026in}{1.521964in}}{\pgfqpoint{0.926299in}{1.529864in}}{\pgfqpoint{0.926299in}{1.538100in}}%
\pgfpathcurveto{\pgfqpoint{0.926299in}{1.546336in}}{\pgfqpoint{0.923026in}{1.554236in}}{\pgfqpoint{0.917202in}{1.560060in}}%
\pgfpathcurveto{\pgfqpoint{0.911378in}{1.565884in}}{\pgfqpoint{0.903478in}{1.569157in}}{\pgfqpoint{0.895242in}{1.569157in}}%
\pgfpathcurveto{\pgfqpoint{0.887006in}{1.569157in}}{\pgfqpoint{0.879106in}{1.565884in}}{\pgfqpoint{0.873282in}{1.560060in}}%
\pgfpathcurveto{\pgfqpoint{0.867458in}{1.554236in}}{\pgfqpoint{0.864186in}{1.546336in}}{\pgfqpoint{0.864186in}{1.538100in}}%
\pgfpathcurveto{\pgfqpoint{0.864186in}{1.529864in}}{\pgfqpoint{0.867458in}{1.521964in}}{\pgfqpoint{0.873282in}{1.516140in}}%
\pgfpathcurveto{\pgfqpoint{0.879106in}{1.510316in}}{\pgfqpoint{0.887006in}{1.507044in}}{\pgfqpoint{0.895242in}{1.507044in}}%
\pgfpathclose%
\pgfusepath{stroke,fill}%
\end{pgfscope}%
\begin{pgfscope}%
\pgfpathrectangle{\pgfqpoint{0.100000in}{0.212622in}}{\pgfqpoint{3.696000in}{3.696000in}}%
\pgfusepath{clip}%
\pgfsetbuttcap%
\pgfsetroundjoin%
\definecolor{currentfill}{rgb}{0.121569,0.466667,0.705882}%
\pgfsetfillcolor{currentfill}%
\pgfsetfillopacity{0.581796}%
\pgfsetlinewidth{1.003750pt}%
\definecolor{currentstroke}{rgb}{0.121569,0.466667,0.705882}%
\pgfsetstrokecolor{currentstroke}%
\pgfsetstrokeopacity{0.581796}%
\pgfsetdash{}{0pt}%
\pgfpathmoveto{\pgfqpoint{0.895153in}{1.506629in}}%
\pgfpathcurveto{\pgfqpoint{0.903389in}{1.506629in}}{\pgfqpoint{0.911289in}{1.509901in}}{\pgfqpoint{0.917113in}{1.515725in}}%
\pgfpathcurveto{\pgfqpoint{0.922937in}{1.521549in}}{\pgfqpoint{0.926209in}{1.529449in}}{\pgfqpoint{0.926209in}{1.537685in}}%
\pgfpathcurveto{\pgfqpoint{0.926209in}{1.545921in}}{\pgfqpoint{0.922937in}{1.553821in}}{\pgfqpoint{0.917113in}{1.559645in}}%
\pgfpathcurveto{\pgfqpoint{0.911289in}{1.565469in}}{\pgfqpoint{0.903389in}{1.568742in}}{\pgfqpoint{0.895153in}{1.568742in}}%
\pgfpathcurveto{\pgfqpoint{0.886916in}{1.568742in}}{\pgfqpoint{0.879016in}{1.565469in}}{\pgfqpoint{0.873192in}{1.559645in}}%
\pgfpathcurveto{\pgfqpoint{0.867368in}{1.553821in}}{\pgfqpoint{0.864096in}{1.545921in}}{\pgfqpoint{0.864096in}{1.537685in}}%
\pgfpathcurveto{\pgfqpoint{0.864096in}{1.529449in}}{\pgfqpoint{0.867368in}{1.521549in}}{\pgfqpoint{0.873192in}{1.515725in}}%
\pgfpathcurveto{\pgfqpoint{0.879016in}{1.509901in}}{\pgfqpoint{0.886916in}{1.506629in}}{\pgfqpoint{0.895153in}{1.506629in}}%
\pgfpathclose%
\pgfusepath{stroke,fill}%
\end{pgfscope}%
\begin{pgfscope}%
\pgfpathrectangle{\pgfqpoint{0.100000in}{0.212622in}}{\pgfqpoint{3.696000in}{3.696000in}}%
\pgfusepath{clip}%
\pgfsetbuttcap%
\pgfsetroundjoin%
\definecolor{currentfill}{rgb}{0.121569,0.466667,0.705882}%
\pgfsetfillcolor{currentfill}%
\pgfsetfillopacity{0.581840}%
\pgfsetlinewidth{1.003750pt}%
\definecolor{currentstroke}{rgb}{0.121569,0.466667,0.705882}%
\pgfsetstrokecolor{currentstroke}%
\pgfsetstrokeopacity{0.581840}%
\pgfsetdash{}{0pt}%
\pgfpathmoveto{\pgfqpoint{0.895102in}{1.506404in}}%
\pgfpathcurveto{\pgfqpoint{0.903338in}{1.506404in}}{\pgfqpoint{0.911238in}{1.509676in}}{\pgfqpoint{0.917062in}{1.515500in}}%
\pgfpathcurveto{\pgfqpoint{0.922886in}{1.521324in}}{\pgfqpoint{0.926159in}{1.529224in}}{\pgfqpoint{0.926159in}{1.537460in}}%
\pgfpathcurveto{\pgfqpoint{0.926159in}{1.545697in}}{\pgfqpoint{0.922886in}{1.553597in}}{\pgfqpoint{0.917062in}{1.559421in}}%
\pgfpathcurveto{\pgfqpoint{0.911238in}{1.565245in}}{\pgfqpoint{0.903338in}{1.568517in}}{\pgfqpoint{0.895102in}{1.568517in}}%
\pgfpathcurveto{\pgfqpoint{0.886866in}{1.568517in}}{\pgfqpoint{0.878966in}{1.565245in}}{\pgfqpoint{0.873142in}{1.559421in}}%
\pgfpathcurveto{\pgfqpoint{0.867318in}{1.553597in}}{\pgfqpoint{0.864046in}{1.545697in}}{\pgfqpoint{0.864046in}{1.537460in}}%
\pgfpathcurveto{\pgfqpoint{0.864046in}{1.529224in}}{\pgfqpoint{0.867318in}{1.521324in}}{\pgfqpoint{0.873142in}{1.515500in}}%
\pgfpathcurveto{\pgfqpoint{0.878966in}{1.509676in}}{\pgfqpoint{0.886866in}{1.506404in}}{\pgfqpoint{0.895102in}{1.506404in}}%
\pgfpathclose%
\pgfusepath{stroke,fill}%
\end{pgfscope}%
\begin{pgfscope}%
\pgfpathrectangle{\pgfqpoint{0.100000in}{0.212622in}}{\pgfqpoint{3.696000in}{3.696000in}}%
\pgfusepath{clip}%
\pgfsetbuttcap%
\pgfsetroundjoin%
\definecolor{currentfill}{rgb}{0.121569,0.466667,0.705882}%
\pgfsetfillcolor{currentfill}%
\pgfsetfillopacity{0.581963}%
\pgfsetlinewidth{1.003750pt}%
\definecolor{currentstroke}{rgb}{0.121569,0.466667,0.705882}%
\pgfsetstrokecolor{currentstroke}%
\pgfsetstrokeopacity{0.581963}%
\pgfsetdash{}{0pt}%
\pgfpathmoveto{\pgfqpoint{0.894965in}{1.505805in}}%
\pgfpathcurveto{\pgfqpoint{0.903201in}{1.505805in}}{\pgfqpoint{0.911101in}{1.509077in}}{\pgfqpoint{0.916925in}{1.514901in}}%
\pgfpathcurveto{\pgfqpoint{0.922749in}{1.520725in}}{\pgfqpoint{0.926021in}{1.528625in}}{\pgfqpoint{0.926021in}{1.536862in}}%
\pgfpathcurveto{\pgfqpoint{0.926021in}{1.545098in}}{\pgfqpoint{0.922749in}{1.552998in}}{\pgfqpoint{0.916925in}{1.558822in}}%
\pgfpathcurveto{\pgfqpoint{0.911101in}{1.564646in}}{\pgfqpoint{0.903201in}{1.567918in}}{\pgfqpoint{0.894965in}{1.567918in}}%
\pgfpathcurveto{\pgfqpoint{0.886729in}{1.567918in}}{\pgfqpoint{0.878829in}{1.564646in}}{\pgfqpoint{0.873005in}{1.558822in}}%
\pgfpathcurveto{\pgfqpoint{0.867181in}{1.552998in}}{\pgfqpoint{0.863908in}{1.545098in}}{\pgfqpoint{0.863908in}{1.536862in}}%
\pgfpathcurveto{\pgfqpoint{0.863908in}{1.528625in}}{\pgfqpoint{0.867181in}{1.520725in}}{\pgfqpoint{0.873005in}{1.514901in}}%
\pgfpathcurveto{\pgfqpoint{0.878829in}{1.509077in}}{\pgfqpoint{0.886729in}{1.505805in}}{\pgfqpoint{0.894965in}{1.505805in}}%
\pgfpathclose%
\pgfusepath{stroke,fill}%
\end{pgfscope}%
\begin{pgfscope}%
\pgfpathrectangle{\pgfqpoint{0.100000in}{0.212622in}}{\pgfqpoint{3.696000in}{3.696000in}}%
\pgfusepath{clip}%
\pgfsetbuttcap%
\pgfsetroundjoin%
\definecolor{currentfill}{rgb}{0.121569,0.466667,0.705882}%
\pgfsetfillcolor{currentfill}%
\pgfsetfillopacity{0.582033}%
\pgfsetlinewidth{1.003750pt}%
\definecolor{currentstroke}{rgb}{0.121569,0.466667,0.705882}%
\pgfsetstrokecolor{currentstroke}%
\pgfsetstrokeopacity{0.582033}%
\pgfsetdash{}{0pt}%
\pgfpathmoveto{\pgfqpoint{0.894893in}{1.505480in}}%
\pgfpathcurveto{\pgfqpoint{0.903129in}{1.505480in}}{\pgfqpoint{0.911029in}{1.508753in}}{\pgfqpoint{0.916853in}{1.514576in}}%
\pgfpathcurveto{\pgfqpoint{0.922677in}{1.520400in}}{\pgfqpoint{0.925949in}{1.528300in}}{\pgfqpoint{0.925949in}{1.536537in}}%
\pgfpathcurveto{\pgfqpoint{0.925949in}{1.544773in}}{\pgfqpoint{0.922677in}{1.552673in}}{\pgfqpoint{0.916853in}{1.558497in}}%
\pgfpathcurveto{\pgfqpoint{0.911029in}{1.564321in}}{\pgfqpoint{0.903129in}{1.567593in}}{\pgfqpoint{0.894893in}{1.567593in}}%
\pgfpathcurveto{\pgfqpoint{0.886657in}{1.567593in}}{\pgfqpoint{0.878757in}{1.564321in}}{\pgfqpoint{0.872933in}{1.558497in}}%
\pgfpathcurveto{\pgfqpoint{0.867109in}{1.552673in}}{\pgfqpoint{0.863836in}{1.544773in}}{\pgfqpoint{0.863836in}{1.536537in}}%
\pgfpathcurveto{\pgfqpoint{0.863836in}{1.528300in}}{\pgfqpoint{0.867109in}{1.520400in}}{\pgfqpoint{0.872933in}{1.514576in}}%
\pgfpathcurveto{\pgfqpoint{0.878757in}{1.508753in}}{\pgfqpoint{0.886657in}{1.505480in}}{\pgfqpoint{0.894893in}{1.505480in}}%
\pgfpathclose%
\pgfusepath{stroke,fill}%
\end{pgfscope}%
\begin{pgfscope}%
\pgfpathrectangle{\pgfqpoint{0.100000in}{0.212622in}}{\pgfqpoint{3.696000in}{3.696000in}}%
\pgfusepath{clip}%
\pgfsetbuttcap%
\pgfsetroundjoin%
\definecolor{currentfill}{rgb}{0.121569,0.466667,0.705882}%
\pgfsetfillcolor{currentfill}%
\pgfsetfillopacity{0.582080}%
\pgfsetlinewidth{1.003750pt}%
\definecolor{currentstroke}{rgb}{0.121569,0.466667,0.705882}%
\pgfsetstrokecolor{currentstroke}%
\pgfsetstrokeopacity{0.582080}%
\pgfsetdash{}{0pt}%
\pgfpathmoveto{\pgfqpoint{1.008194in}{1.772355in}}%
\pgfpathcurveto{\pgfqpoint{1.016430in}{1.772355in}}{\pgfqpoint{1.024330in}{1.775628in}}{\pgfqpoint{1.030154in}{1.781452in}}%
\pgfpathcurveto{\pgfqpoint{1.035978in}{1.787276in}}{\pgfqpoint{1.039250in}{1.795176in}}{\pgfqpoint{1.039250in}{1.803412in}}%
\pgfpathcurveto{\pgfqpoint{1.039250in}{1.811648in}}{\pgfqpoint{1.035978in}{1.819548in}}{\pgfqpoint{1.030154in}{1.825372in}}%
\pgfpathcurveto{\pgfqpoint{1.024330in}{1.831196in}}{\pgfqpoint{1.016430in}{1.834468in}}{\pgfqpoint{1.008194in}{1.834468in}}%
\pgfpathcurveto{\pgfqpoint{0.999958in}{1.834468in}}{\pgfqpoint{0.992058in}{1.831196in}}{\pgfqpoint{0.986234in}{1.825372in}}%
\pgfpathcurveto{\pgfqpoint{0.980410in}{1.819548in}}{\pgfqpoint{0.977137in}{1.811648in}}{\pgfqpoint{0.977137in}{1.803412in}}%
\pgfpathcurveto{\pgfqpoint{0.977137in}{1.795176in}}{\pgfqpoint{0.980410in}{1.787276in}}{\pgfqpoint{0.986234in}{1.781452in}}%
\pgfpathcurveto{\pgfqpoint{0.992058in}{1.775628in}}{\pgfqpoint{0.999958in}{1.772355in}}{\pgfqpoint{1.008194in}{1.772355in}}%
\pgfpathclose%
\pgfusepath{stroke,fill}%
\end{pgfscope}%
\begin{pgfscope}%
\pgfpathrectangle{\pgfqpoint{0.100000in}{0.212622in}}{\pgfqpoint{3.696000in}{3.696000in}}%
\pgfusepath{clip}%
\pgfsetbuttcap%
\pgfsetroundjoin%
\definecolor{currentfill}{rgb}{0.121569,0.466667,0.705882}%
\pgfsetfillcolor{currentfill}%
\pgfsetfillopacity{0.582212}%
\pgfsetlinewidth{1.003750pt}%
\definecolor{currentstroke}{rgb}{0.121569,0.466667,0.705882}%
\pgfsetstrokecolor{currentstroke}%
\pgfsetstrokeopacity{0.582212}%
\pgfsetdash{}{0pt}%
\pgfpathmoveto{\pgfqpoint{0.894706in}{1.504659in}}%
\pgfpathcurveto{\pgfqpoint{0.902942in}{1.504659in}}{\pgfqpoint{0.910842in}{1.507931in}}{\pgfqpoint{0.916666in}{1.513755in}}%
\pgfpathcurveto{\pgfqpoint{0.922490in}{1.519579in}}{\pgfqpoint{0.925762in}{1.527479in}}{\pgfqpoint{0.925762in}{1.535715in}}%
\pgfpathcurveto{\pgfqpoint{0.925762in}{1.543951in}}{\pgfqpoint{0.922490in}{1.551851in}}{\pgfqpoint{0.916666in}{1.557675in}}%
\pgfpathcurveto{\pgfqpoint{0.910842in}{1.563499in}}{\pgfqpoint{0.902942in}{1.566772in}}{\pgfqpoint{0.894706in}{1.566772in}}%
\pgfpathcurveto{\pgfqpoint{0.886469in}{1.566772in}}{\pgfqpoint{0.878569in}{1.563499in}}{\pgfqpoint{0.872745in}{1.557675in}}%
\pgfpathcurveto{\pgfqpoint{0.866921in}{1.551851in}}{\pgfqpoint{0.863649in}{1.543951in}}{\pgfqpoint{0.863649in}{1.535715in}}%
\pgfpathcurveto{\pgfqpoint{0.863649in}{1.527479in}}{\pgfqpoint{0.866921in}{1.519579in}}{\pgfqpoint{0.872745in}{1.513755in}}%
\pgfpathcurveto{\pgfqpoint{0.878569in}{1.507931in}}{\pgfqpoint{0.886469in}{1.504659in}}{\pgfqpoint{0.894706in}{1.504659in}}%
\pgfpathclose%
\pgfusepath{stroke,fill}%
\end{pgfscope}%
\begin{pgfscope}%
\pgfpathrectangle{\pgfqpoint{0.100000in}{0.212622in}}{\pgfqpoint{3.696000in}{3.696000in}}%
\pgfusepath{clip}%
\pgfsetbuttcap%
\pgfsetroundjoin%
\definecolor{currentfill}{rgb}{0.121569,0.466667,0.705882}%
\pgfsetfillcolor{currentfill}%
\pgfsetfillopacity{0.582310}%
\pgfsetlinewidth{1.003750pt}%
\definecolor{currentstroke}{rgb}{0.121569,0.466667,0.705882}%
\pgfsetstrokecolor{currentstroke}%
\pgfsetstrokeopacity{0.582310}%
\pgfsetdash{}{0pt}%
\pgfpathmoveto{\pgfqpoint{0.894595in}{1.504205in}}%
\pgfpathcurveto{\pgfqpoint{0.902831in}{1.504205in}}{\pgfqpoint{0.910731in}{1.507477in}}{\pgfqpoint{0.916555in}{1.513301in}}%
\pgfpathcurveto{\pgfqpoint{0.922379in}{1.519125in}}{\pgfqpoint{0.925651in}{1.527025in}}{\pgfqpoint{0.925651in}{1.535261in}}%
\pgfpathcurveto{\pgfqpoint{0.925651in}{1.543497in}}{\pgfqpoint{0.922379in}{1.551397in}}{\pgfqpoint{0.916555in}{1.557221in}}%
\pgfpathcurveto{\pgfqpoint{0.910731in}{1.563045in}}{\pgfqpoint{0.902831in}{1.566318in}}{\pgfqpoint{0.894595in}{1.566318in}}%
\pgfpathcurveto{\pgfqpoint{0.886359in}{1.566318in}}{\pgfqpoint{0.878459in}{1.563045in}}{\pgfqpoint{0.872635in}{1.557221in}}%
\pgfpathcurveto{\pgfqpoint{0.866811in}{1.551397in}}{\pgfqpoint{0.863538in}{1.543497in}}{\pgfqpoint{0.863538in}{1.535261in}}%
\pgfpathcurveto{\pgfqpoint{0.863538in}{1.527025in}}{\pgfqpoint{0.866811in}{1.519125in}}{\pgfqpoint{0.872635in}{1.513301in}}%
\pgfpathcurveto{\pgfqpoint{0.878459in}{1.507477in}}{\pgfqpoint{0.886359in}{1.504205in}}{\pgfqpoint{0.894595in}{1.504205in}}%
\pgfpathclose%
\pgfusepath{stroke,fill}%
\end{pgfscope}%
\begin{pgfscope}%
\pgfpathrectangle{\pgfqpoint{0.100000in}{0.212622in}}{\pgfqpoint{3.696000in}{3.696000in}}%
\pgfusepath{clip}%
\pgfsetbuttcap%
\pgfsetroundjoin%
\definecolor{currentfill}{rgb}{0.121569,0.466667,0.705882}%
\pgfsetfillcolor{currentfill}%
\pgfsetfillopacity{0.582502}%
\pgfsetlinewidth{1.003750pt}%
\definecolor{currentstroke}{rgb}{0.121569,0.466667,0.705882}%
\pgfsetstrokecolor{currentstroke}%
\pgfsetstrokeopacity{0.582502}%
\pgfsetdash{}{0pt}%
\pgfpathmoveto{\pgfqpoint{0.894390in}{1.503340in}}%
\pgfpathcurveto{\pgfqpoint{0.902627in}{1.503340in}}{\pgfqpoint{0.910527in}{1.506612in}}{\pgfqpoint{0.916351in}{1.512436in}}%
\pgfpathcurveto{\pgfqpoint{0.922174in}{1.518260in}}{\pgfqpoint{0.925447in}{1.526160in}}{\pgfqpoint{0.925447in}{1.534396in}}%
\pgfpathcurveto{\pgfqpoint{0.925447in}{1.542632in}}{\pgfqpoint{0.922174in}{1.550532in}}{\pgfqpoint{0.916351in}{1.556356in}}%
\pgfpathcurveto{\pgfqpoint{0.910527in}{1.562180in}}{\pgfqpoint{0.902627in}{1.565453in}}{\pgfqpoint{0.894390in}{1.565453in}}%
\pgfpathcurveto{\pgfqpoint{0.886154in}{1.565453in}}{\pgfqpoint{0.878254in}{1.562180in}}{\pgfqpoint{0.872430in}{1.556356in}}%
\pgfpathcurveto{\pgfqpoint{0.866606in}{1.550532in}}{\pgfqpoint{0.863334in}{1.542632in}}{\pgfqpoint{0.863334in}{1.534396in}}%
\pgfpathcurveto{\pgfqpoint{0.863334in}{1.526160in}}{\pgfqpoint{0.866606in}{1.518260in}}{\pgfqpoint{0.872430in}{1.512436in}}%
\pgfpathcurveto{\pgfqpoint{0.878254in}{1.506612in}}{\pgfqpoint{0.886154in}{1.503340in}}{\pgfqpoint{0.894390in}{1.503340in}}%
\pgfpathclose%
\pgfusepath{stroke,fill}%
\end{pgfscope}%
\begin{pgfscope}%
\pgfpathrectangle{\pgfqpoint{0.100000in}{0.212622in}}{\pgfqpoint{3.696000in}{3.696000in}}%
\pgfusepath{clip}%
\pgfsetbuttcap%
\pgfsetroundjoin%
\definecolor{currentfill}{rgb}{0.121569,0.466667,0.705882}%
\pgfsetfillcolor{currentfill}%
\pgfsetfillopacity{0.582809}%
\pgfsetlinewidth{1.003750pt}%
\definecolor{currentstroke}{rgb}{0.121569,0.466667,0.705882}%
\pgfsetstrokecolor{currentstroke}%
\pgfsetstrokeopacity{0.582809}%
\pgfsetdash{}{0pt}%
\pgfpathmoveto{\pgfqpoint{0.894090in}{1.502049in}}%
\pgfpathcurveto{\pgfqpoint{0.902326in}{1.502049in}}{\pgfqpoint{0.910226in}{1.505321in}}{\pgfqpoint{0.916050in}{1.511145in}}%
\pgfpathcurveto{\pgfqpoint{0.921874in}{1.516969in}}{\pgfqpoint{0.925146in}{1.524869in}}{\pgfqpoint{0.925146in}{1.533105in}}%
\pgfpathcurveto{\pgfqpoint{0.925146in}{1.541342in}}{\pgfqpoint{0.921874in}{1.549242in}}{\pgfqpoint{0.916050in}{1.555066in}}%
\pgfpathcurveto{\pgfqpoint{0.910226in}{1.560889in}}{\pgfqpoint{0.902326in}{1.564162in}}{\pgfqpoint{0.894090in}{1.564162in}}%
\pgfpathcurveto{\pgfqpoint{0.885854in}{1.564162in}}{\pgfqpoint{0.877954in}{1.560889in}}{\pgfqpoint{0.872130in}{1.555066in}}%
\pgfpathcurveto{\pgfqpoint{0.866306in}{1.549242in}}{\pgfqpoint{0.863033in}{1.541342in}}{\pgfqpoint{0.863033in}{1.533105in}}%
\pgfpathcurveto{\pgfqpoint{0.863033in}{1.524869in}}{\pgfqpoint{0.866306in}{1.516969in}}{\pgfqpoint{0.872130in}{1.511145in}}%
\pgfpathcurveto{\pgfqpoint{0.877954in}{1.505321in}}{\pgfqpoint{0.885854in}{1.502049in}}{\pgfqpoint{0.894090in}{1.502049in}}%
\pgfpathclose%
\pgfusepath{stroke,fill}%
\end{pgfscope}%
\begin{pgfscope}%
\pgfpathrectangle{\pgfqpoint{0.100000in}{0.212622in}}{\pgfqpoint{3.696000in}{3.696000in}}%
\pgfusepath{clip}%
\pgfsetbuttcap%
\pgfsetroundjoin%
\definecolor{currentfill}{rgb}{0.121569,0.466667,0.705882}%
\pgfsetfillcolor{currentfill}%
\pgfsetfillopacity{0.582863}%
\pgfsetlinewidth{1.003750pt}%
\definecolor{currentstroke}{rgb}{0.121569,0.466667,0.705882}%
\pgfsetstrokecolor{currentstroke}%
\pgfsetstrokeopacity{0.582863}%
\pgfsetdash{}{0pt}%
\pgfpathmoveto{\pgfqpoint{2.090247in}{2.163047in}}%
\pgfpathcurveto{\pgfqpoint{2.098484in}{2.163047in}}{\pgfqpoint{2.106384in}{2.166319in}}{\pgfqpoint{2.112208in}{2.172143in}}%
\pgfpathcurveto{\pgfqpoint{2.118032in}{2.177967in}}{\pgfqpoint{2.121304in}{2.185867in}}{\pgfqpoint{2.121304in}{2.194103in}}%
\pgfpathcurveto{\pgfqpoint{2.121304in}{2.202340in}}{\pgfqpoint{2.118032in}{2.210240in}}{\pgfqpoint{2.112208in}{2.216064in}}%
\pgfpathcurveto{\pgfqpoint{2.106384in}{2.221888in}}{\pgfqpoint{2.098484in}{2.225160in}}{\pgfqpoint{2.090247in}{2.225160in}}%
\pgfpathcurveto{\pgfqpoint{2.082011in}{2.225160in}}{\pgfqpoint{2.074111in}{2.221888in}}{\pgfqpoint{2.068287in}{2.216064in}}%
\pgfpathcurveto{\pgfqpoint{2.062463in}{2.210240in}}{\pgfqpoint{2.059191in}{2.202340in}}{\pgfqpoint{2.059191in}{2.194103in}}%
\pgfpathcurveto{\pgfqpoint{2.059191in}{2.185867in}}{\pgfqpoint{2.062463in}{2.177967in}}{\pgfqpoint{2.068287in}{2.172143in}}%
\pgfpathcurveto{\pgfqpoint{2.074111in}{2.166319in}}{\pgfqpoint{2.082011in}{2.163047in}}{\pgfqpoint{2.090247in}{2.163047in}}%
\pgfpathclose%
\pgfusepath{stroke,fill}%
\end{pgfscope}%
\begin{pgfscope}%
\pgfpathrectangle{\pgfqpoint{0.100000in}{0.212622in}}{\pgfqpoint{3.696000in}{3.696000in}}%
\pgfusepath{clip}%
\pgfsetbuttcap%
\pgfsetroundjoin%
\definecolor{currentfill}{rgb}{0.121569,0.466667,0.705882}%
\pgfsetfillcolor{currentfill}%
\pgfsetfillopacity{0.583261}%
\pgfsetlinewidth{1.003750pt}%
\definecolor{currentstroke}{rgb}{0.121569,0.466667,0.705882}%
\pgfsetstrokecolor{currentstroke}%
\pgfsetstrokeopacity{0.583261}%
\pgfsetdash{}{0pt}%
\pgfpathmoveto{\pgfqpoint{0.893656in}{1.500188in}}%
\pgfpathcurveto{\pgfqpoint{0.901893in}{1.500188in}}{\pgfqpoint{0.909793in}{1.503461in}}{\pgfqpoint{0.915617in}{1.509284in}}%
\pgfpathcurveto{\pgfqpoint{0.921440in}{1.515108in}}{\pgfqpoint{0.924713in}{1.523008in}}{\pgfqpoint{0.924713in}{1.531245in}}%
\pgfpathcurveto{\pgfqpoint{0.924713in}{1.539481in}}{\pgfqpoint{0.921440in}{1.547381in}}{\pgfqpoint{0.915617in}{1.553205in}}%
\pgfpathcurveto{\pgfqpoint{0.909793in}{1.559029in}}{\pgfqpoint{0.901893in}{1.562301in}}{\pgfqpoint{0.893656in}{1.562301in}}%
\pgfpathcurveto{\pgfqpoint{0.885420in}{1.562301in}}{\pgfqpoint{0.877520in}{1.559029in}}{\pgfqpoint{0.871696in}{1.553205in}}%
\pgfpathcurveto{\pgfqpoint{0.865872in}{1.547381in}}{\pgfqpoint{0.862600in}{1.539481in}}{\pgfqpoint{0.862600in}{1.531245in}}%
\pgfpathcurveto{\pgfqpoint{0.862600in}{1.523008in}}{\pgfqpoint{0.865872in}{1.515108in}}{\pgfqpoint{0.871696in}{1.509284in}}%
\pgfpathcurveto{\pgfqpoint{0.877520in}{1.503461in}}{\pgfqpoint{0.885420in}{1.500188in}}{\pgfqpoint{0.893656in}{1.500188in}}%
\pgfpathclose%
\pgfusepath{stroke,fill}%
\end{pgfscope}%
\begin{pgfscope}%
\pgfpathrectangle{\pgfqpoint{0.100000in}{0.212622in}}{\pgfqpoint{3.696000in}{3.696000in}}%
\pgfusepath{clip}%
\pgfsetbuttcap%
\pgfsetroundjoin%
\definecolor{currentfill}{rgb}{0.121569,0.466667,0.705882}%
\pgfsetfillcolor{currentfill}%
\pgfsetfillopacity{0.583874}%
\pgfsetlinewidth{1.003750pt}%
\definecolor{currentstroke}{rgb}{0.121569,0.466667,0.705882}%
\pgfsetstrokecolor{currentstroke}%
\pgfsetstrokeopacity{0.583874}%
\pgfsetdash{}{0pt}%
\pgfpathmoveto{\pgfqpoint{0.893073in}{1.497753in}}%
\pgfpathcurveto{\pgfqpoint{0.901310in}{1.497753in}}{\pgfqpoint{0.909210in}{1.501025in}}{\pgfqpoint{0.915033in}{1.506849in}}%
\pgfpathcurveto{\pgfqpoint{0.920857in}{1.512673in}}{\pgfqpoint{0.924130in}{1.520573in}}{\pgfqpoint{0.924130in}{1.528810in}}%
\pgfpathcurveto{\pgfqpoint{0.924130in}{1.537046in}}{\pgfqpoint{0.920857in}{1.544946in}}{\pgfqpoint{0.915033in}{1.550770in}}%
\pgfpathcurveto{\pgfqpoint{0.909210in}{1.556594in}}{\pgfqpoint{0.901310in}{1.559866in}}{\pgfqpoint{0.893073in}{1.559866in}}%
\pgfpathcurveto{\pgfqpoint{0.884837in}{1.559866in}}{\pgfqpoint{0.876937in}{1.556594in}}{\pgfqpoint{0.871113in}{1.550770in}}%
\pgfpathcurveto{\pgfqpoint{0.865289in}{1.544946in}}{\pgfqpoint{0.862017in}{1.537046in}}{\pgfqpoint{0.862017in}{1.528810in}}%
\pgfpathcurveto{\pgfqpoint{0.862017in}{1.520573in}}{\pgfqpoint{0.865289in}{1.512673in}}{\pgfqpoint{0.871113in}{1.506849in}}%
\pgfpathcurveto{\pgfqpoint{0.876937in}{1.501025in}}{\pgfqpoint{0.884837in}{1.497753in}}{\pgfqpoint{0.893073in}{1.497753in}}%
\pgfpathclose%
\pgfusepath{stroke,fill}%
\end{pgfscope}%
\begin{pgfscope}%
\pgfpathrectangle{\pgfqpoint{0.100000in}{0.212622in}}{\pgfqpoint{3.696000in}{3.696000in}}%
\pgfusepath{clip}%
\pgfsetbuttcap%
\pgfsetroundjoin%
\definecolor{currentfill}{rgb}{0.121569,0.466667,0.705882}%
\pgfsetfillcolor{currentfill}%
\pgfsetfillopacity{0.584188}%
\pgfsetlinewidth{1.003750pt}%
\definecolor{currentstroke}{rgb}{0.121569,0.466667,0.705882}%
\pgfsetstrokecolor{currentstroke}%
\pgfsetstrokeopacity{0.584188}%
\pgfsetdash{}{0pt}%
\pgfpathmoveto{\pgfqpoint{0.892740in}{1.496332in}}%
\pgfpathcurveto{\pgfqpoint{0.900977in}{1.496332in}}{\pgfqpoint{0.908877in}{1.499604in}}{\pgfqpoint{0.914701in}{1.505428in}}%
\pgfpathcurveto{\pgfqpoint{0.920525in}{1.511252in}}{\pgfqpoint{0.923797in}{1.519152in}}{\pgfqpoint{0.923797in}{1.527388in}}%
\pgfpathcurveto{\pgfqpoint{0.923797in}{1.535625in}}{\pgfqpoint{0.920525in}{1.543525in}}{\pgfqpoint{0.914701in}{1.549349in}}%
\pgfpathcurveto{\pgfqpoint{0.908877in}{1.555173in}}{\pgfqpoint{0.900977in}{1.558445in}}{\pgfqpoint{0.892740in}{1.558445in}}%
\pgfpathcurveto{\pgfqpoint{0.884504in}{1.558445in}}{\pgfqpoint{0.876604in}{1.555173in}}{\pgfqpoint{0.870780in}{1.549349in}}%
\pgfpathcurveto{\pgfqpoint{0.864956in}{1.543525in}}{\pgfqpoint{0.861684in}{1.535625in}}{\pgfqpoint{0.861684in}{1.527388in}}%
\pgfpathcurveto{\pgfqpoint{0.861684in}{1.519152in}}{\pgfqpoint{0.864956in}{1.511252in}}{\pgfqpoint{0.870780in}{1.505428in}}%
\pgfpathcurveto{\pgfqpoint{0.876604in}{1.499604in}}{\pgfqpoint{0.884504in}{1.496332in}}{\pgfqpoint{0.892740in}{1.496332in}}%
\pgfpathclose%
\pgfusepath{stroke,fill}%
\end{pgfscope}%
\begin{pgfscope}%
\pgfpathrectangle{\pgfqpoint{0.100000in}{0.212622in}}{\pgfqpoint{3.696000in}{3.696000in}}%
\pgfusepath{clip}%
\pgfsetbuttcap%
\pgfsetroundjoin%
\definecolor{currentfill}{rgb}{0.121569,0.466667,0.705882}%
\pgfsetfillcolor{currentfill}%
\pgfsetfillopacity{0.584296}%
\pgfsetlinewidth{1.003750pt}%
\definecolor{currentstroke}{rgb}{0.121569,0.466667,0.705882}%
\pgfsetstrokecolor{currentstroke}%
\pgfsetstrokeopacity{0.584296}%
\pgfsetdash{}{0pt}%
\pgfpathmoveto{\pgfqpoint{0.876605in}{1.568903in}}%
\pgfpathcurveto{\pgfqpoint{0.884841in}{1.568903in}}{\pgfqpoint{0.892741in}{1.572175in}}{\pgfqpoint{0.898565in}{1.577999in}}%
\pgfpathcurveto{\pgfqpoint{0.904389in}{1.583823in}}{\pgfqpoint{0.907661in}{1.591723in}}{\pgfqpoint{0.907661in}{1.599959in}}%
\pgfpathcurveto{\pgfqpoint{0.907661in}{1.608196in}}{\pgfqpoint{0.904389in}{1.616096in}}{\pgfqpoint{0.898565in}{1.621920in}}%
\pgfpathcurveto{\pgfqpoint{0.892741in}{1.627744in}}{\pgfqpoint{0.884841in}{1.631016in}}{\pgfqpoint{0.876605in}{1.631016in}}%
\pgfpathcurveto{\pgfqpoint{0.868368in}{1.631016in}}{\pgfqpoint{0.860468in}{1.627744in}}{\pgfqpoint{0.854645in}{1.621920in}}%
\pgfpathcurveto{\pgfqpoint{0.848821in}{1.616096in}}{\pgfqpoint{0.845548in}{1.608196in}}{\pgfqpoint{0.845548in}{1.599959in}}%
\pgfpathcurveto{\pgfqpoint{0.845548in}{1.591723in}}{\pgfqpoint{0.848821in}{1.583823in}}{\pgfqpoint{0.854645in}{1.577999in}}%
\pgfpathcurveto{\pgfqpoint{0.860468in}{1.572175in}}{\pgfqpoint{0.868368in}{1.568903in}}{\pgfqpoint{0.876605in}{1.568903in}}%
\pgfpathclose%
\pgfusepath{stroke,fill}%
\end{pgfscope}%
\begin{pgfscope}%
\pgfpathrectangle{\pgfqpoint{0.100000in}{0.212622in}}{\pgfqpoint{3.696000in}{3.696000in}}%
\pgfusepath{clip}%
\pgfsetbuttcap%
\pgfsetroundjoin%
\definecolor{currentfill}{rgb}{0.121569,0.466667,0.705882}%
\pgfsetfillcolor{currentfill}%
\pgfsetfillopacity{0.584354}%
\pgfsetlinewidth{1.003750pt}%
\definecolor{currentstroke}{rgb}{0.121569,0.466667,0.705882}%
\pgfsetstrokecolor{currentstroke}%
\pgfsetstrokeopacity{0.584354}%
\pgfsetdash{}{0pt}%
\pgfpathmoveto{\pgfqpoint{0.892550in}{1.495522in}}%
\pgfpathcurveto{\pgfqpoint{0.900787in}{1.495522in}}{\pgfqpoint{0.908687in}{1.498794in}}{\pgfqpoint{0.914511in}{1.504618in}}%
\pgfpathcurveto{\pgfqpoint{0.920334in}{1.510442in}}{\pgfqpoint{0.923607in}{1.518342in}}{\pgfqpoint{0.923607in}{1.526578in}}%
\pgfpathcurveto{\pgfqpoint{0.923607in}{1.534814in}}{\pgfqpoint{0.920334in}{1.542715in}}{\pgfqpoint{0.914511in}{1.548538in}}%
\pgfpathcurveto{\pgfqpoint{0.908687in}{1.554362in}}{\pgfqpoint{0.900787in}{1.557635in}}{\pgfqpoint{0.892550in}{1.557635in}}%
\pgfpathcurveto{\pgfqpoint{0.884314in}{1.557635in}}{\pgfqpoint{0.876414in}{1.554362in}}{\pgfqpoint{0.870590in}{1.548538in}}%
\pgfpathcurveto{\pgfqpoint{0.864766in}{1.542715in}}{\pgfqpoint{0.861494in}{1.534814in}}{\pgfqpoint{0.861494in}{1.526578in}}%
\pgfpathcurveto{\pgfqpoint{0.861494in}{1.518342in}}{\pgfqpoint{0.864766in}{1.510442in}}{\pgfqpoint{0.870590in}{1.504618in}}%
\pgfpathcurveto{\pgfqpoint{0.876414in}{1.498794in}}{\pgfqpoint{0.884314in}{1.495522in}}{\pgfqpoint{0.892550in}{1.495522in}}%
\pgfpathclose%
\pgfusepath{stroke,fill}%
\end{pgfscope}%
\begin{pgfscope}%
\pgfpathrectangle{\pgfqpoint{0.100000in}{0.212622in}}{\pgfqpoint{3.696000in}{3.696000in}}%
\pgfusepath{clip}%
\pgfsetbuttcap%
\pgfsetroundjoin%
\definecolor{currentfill}{rgb}{0.121569,0.466667,0.705882}%
\pgfsetfillcolor{currentfill}%
\pgfsetfillopacity{0.584441}%
\pgfsetlinewidth{1.003750pt}%
\definecolor{currentstroke}{rgb}{0.121569,0.466667,0.705882}%
\pgfsetstrokecolor{currentstroke}%
\pgfsetstrokeopacity{0.584441}%
\pgfsetdash{}{0pt}%
\pgfpathmoveto{\pgfqpoint{0.892445in}{1.495063in}}%
\pgfpathcurveto{\pgfqpoint{0.900681in}{1.495063in}}{\pgfqpoint{0.908581in}{1.498335in}}{\pgfqpoint{0.914405in}{1.504159in}}%
\pgfpathcurveto{\pgfqpoint{0.920229in}{1.509983in}}{\pgfqpoint{0.923501in}{1.517883in}}{\pgfqpoint{0.923501in}{1.526119in}}%
\pgfpathcurveto{\pgfqpoint{0.923501in}{1.534356in}}{\pgfqpoint{0.920229in}{1.542256in}}{\pgfqpoint{0.914405in}{1.548080in}}%
\pgfpathcurveto{\pgfqpoint{0.908581in}{1.553904in}}{\pgfqpoint{0.900681in}{1.557176in}}{\pgfqpoint{0.892445in}{1.557176in}}%
\pgfpathcurveto{\pgfqpoint{0.884208in}{1.557176in}}{\pgfqpoint{0.876308in}{1.553904in}}{\pgfqpoint{0.870484in}{1.548080in}}%
\pgfpathcurveto{\pgfqpoint{0.864660in}{1.542256in}}{\pgfqpoint{0.861388in}{1.534356in}}{\pgfqpoint{0.861388in}{1.526119in}}%
\pgfpathcurveto{\pgfqpoint{0.861388in}{1.517883in}}{\pgfqpoint{0.864660in}{1.509983in}}{\pgfqpoint{0.870484in}{1.504159in}}%
\pgfpathcurveto{\pgfqpoint{0.876308in}{1.498335in}}{\pgfqpoint{0.884208in}{1.495063in}}{\pgfqpoint{0.892445in}{1.495063in}}%
\pgfpathclose%
\pgfusepath{stroke,fill}%
\end{pgfscope}%
\begin{pgfscope}%
\pgfpathrectangle{\pgfqpoint{0.100000in}{0.212622in}}{\pgfqpoint{3.696000in}{3.696000in}}%
\pgfusepath{clip}%
\pgfsetbuttcap%
\pgfsetroundjoin%
\definecolor{currentfill}{rgb}{0.121569,0.466667,0.705882}%
\pgfsetfillcolor{currentfill}%
\pgfsetfillopacity{0.584488}%
\pgfsetlinewidth{1.003750pt}%
\definecolor{currentstroke}{rgb}{0.121569,0.466667,0.705882}%
\pgfsetstrokecolor{currentstroke}%
\pgfsetstrokeopacity{0.584488}%
\pgfsetdash{}{0pt}%
\pgfpathmoveto{\pgfqpoint{0.892389in}{1.494805in}}%
\pgfpathcurveto{\pgfqpoint{0.900625in}{1.494805in}}{\pgfqpoint{0.908525in}{1.498078in}}{\pgfqpoint{0.914349in}{1.503902in}}%
\pgfpathcurveto{\pgfqpoint{0.920173in}{1.509726in}}{\pgfqpoint{0.923445in}{1.517626in}}{\pgfqpoint{0.923445in}{1.525862in}}%
\pgfpathcurveto{\pgfqpoint{0.923445in}{1.534098in}}{\pgfqpoint{0.920173in}{1.541998in}}{\pgfqpoint{0.914349in}{1.547822in}}%
\pgfpathcurveto{\pgfqpoint{0.908525in}{1.553646in}}{\pgfqpoint{0.900625in}{1.556918in}}{\pgfqpoint{0.892389in}{1.556918in}}%
\pgfpathcurveto{\pgfqpoint{0.884153in}{1.556918in}}{\pgfqpoint{0.876253in}{1.553646in}}{\pgfqpoint{0.870429in}{1.547822in}}%
\pgfpathcurveto{\pgfqpoint{0.864605in}{1.541998in}}{\pgfqpoint{0.861332in}{1.534098in}}{\pgfqpoint{0.861332in}{1.525862in}}%
\pgfpathcurveto{\pgfqpoint{0.861332in}{1.517626in}}{\pgfqpoint{0.864605in}{1.509726in}}{\pgfqpoint{0.870429in}{1.503902in}}%
\pgfpathcurveto{\pgfqpoint{0.876253in}{1.498078in}}{\pgfqpoint{0.884153in}{1.494805in}}{\pgfqpoint{0.892389in}{1.494805in}}%
\pgfpathclose%
\pgfusepath{stroke,fill}%
\end{pgfscope}%
\begin{pgfscope}%
\pgfpathrectangle{\pgfqpoint{0.100000in}{0.212622in}}{\pgfqpoint{3.696000in}{3.696000in}}%
\pgfusepath{clip}%
\pgfsetbuttcap%
\pgfsetroundjoin%
\definecolor{currentfill}{rgb}{0.121569,0.466667,0.705882}%
\pgfsetfillcolor{currentfill}%
\pgfsetfillopacity{0.584513}%
\pgfsetlinewidth{1.003750pt}%
\definecolor{currentstroke}{rgb}{0.121569,0.466667,0.705882}%
\pgfsetstrokecolor{currentstroke}%
\pgfsetstrokeopacity{0.584513}%
\pgfsetdash{}{0pt}%
\pgfpathmoveto{\pgfqpoint{0.892359in}{1.494662in}}%
\pgfpathcurveto{\pgfqpoint{0.900595in}{1.494662in}}{\pgfqpoint{0.908495in}{1.497935in}}{\pgfqpoint{0.914319in}{1.503759in}}%
\pgfpathcurveto{\pgfqpoint{0.920143in}{1.509583in}}{\pgfqpoint{0.923415in}{1.517483in}}{\pgfqpoint{0.923415in}{1.525719in}}%
\pgfpathcurveto{\pgfqpoint{0.923415in}{1.533955in}}{\pgfqpoint{0.920143in}{1.541855in}}{\pgfqpoint{0.914319in}{1.547679in}}%
\pgfpathcurveto{\pgfqpoint{0.908495in}{1.553503in}}{\pgfqpoint{0.900595in}{1.556775in}}{\pgfqpoint{0.892359in}{1.556775in}}%
\pgfpathcurveto{\pgfqpoint{0.884122in}{1.556775in}}{\pgfqpoint{0.876222in}{1.553503in}}{\pgfqpoint{0.870398in}{1.547679in}}%
\pgfpathcurveto{\pgfqpoint{0.864575in}{1.541855in}}{\pgfqpoint{0.861302in}{1.533955in}}{\pgfqpoint{0.861302in}{1.525719in}}%
\pgfpathcurveto{\pgfqpoint{0.861302in}{1.517483in}}{\pgfqpoint{0.864575in}{1.509583in}}{\pgfqpoint{0.870398in}{1.503759in}}%
\pgfpathcurveto{\pgfqpoint{0.876222in}{1.497935in}}{\pgfqpoint{0.884122in}{1.494662in}}{\pgfqpoint{0.892359in}{1.494662in}}%
\pgfpathclose%
\pgfusepath{stroke,fill}%
\end{pgfscope}%
\begin{pgfscope}%
\pgfpathrectangle{\pgfqpoint{0.100000in}{0.212622in}}{\pgfqpoint{3.696000in}{3.696000in}}%
\pgfusepath{clip}%
\pgfsetbuttcap%
\pgfsetroundjoin%
\definecolor{currentfill}{rgb}{0.121569,0.466667,0.705882}%
\pgfsetfillcolor{currentfill}%
\pgfsetfillopacity{0.584527}%
\pgfsetlinewidth{1.003750pt}%
\definecolor{currentstroke}{rgb}{0.121569,0.466667,0.705882}%
\pgfsetstrokecolor{currentstroke}%
\pgfsetstrokeopacity{0.584527}%
\pgfsetdash{}{0pt}%
\pgfpathmoveto{\pgfqpoint{0.892342in}{1.494584in}}%
\pgfpathcurveto{\pgfqpoint{0.900578in}{1.494584in}}{\pgfqpoint{0.908478in}{1.497856in}}{\pgfqpoint{0.914302in}{1.503680in}}%
\pgfpathcurveto{\pgfqpoint{0.920126in}{1.509504in}}{\pgfqpoint{0.923399in}{1.517404in}}{\pgfqpoint{0.923399in}{1.525640in}}%
\pgfpathcurveto{\pgfqpoint{0.923399in}{1.533876in}}{\pgfqpoint{0.920126in}{1.541777in}}{\pgfqpoint{0.914302in}{1.547600in}}%
\pgfpathcurveto{\pgfqpoint{0.908478in}{1.553424in}}{\pgfqpoint{0.900578in}{1.556697in}}{\pgfqpoint{0.892342in}{1.556697in}}%
\pgfpathcurveto{\pgfqpoint{0.884106in}{1.556697in}}{\pgfqpoint{0.876206in}{1.553424in}}{\pgfqpoint{0.870382in}{1.547600in}}%
\pgfpathcurveto{\pgfqpoint{0.864558in}{1.541777in}}{\pgfqpoint{0.861286in}{1.533876in}}{\pgfqpoint{0.861286in}{1.525640in}}%
\pgfpathcurveto{\pgfqpoint{0.861286in}{1.517404in}}{\pgfqpoint{0.864558in}{1.509504in}}{\pgfqpoint{0.870382in}{1.503680in}}%
\pgfpathcurveto{\pgfqpoint{0.876206in}{1.497856in}}{\pgfqpoint{0.884106in}{1.494584in}}{\pgfqpoint{0.892342in}{1.494584in}}%
\pgfpathclose%
\pgfusepath{stroke,fill}%
\end{pgfscope}%
\begin{pgfscope}%
\pgfpathrectangle{\pgfqpoint{0.100000in}{0.212622in}}{\pgfqpoint{3.696000in}{3.696000in}}%
\pgfusepath{clip}%
\pgfsetbuttcap%
\pgfsetroundjoin%
\definecolor{currentfill}{rgb}{0.121569,0.466667,0.705882}%
\pgfsetfillcolor{currentfill}%
\pgfsetfillopacity{0.584535}%
\pgfsetlinewidth{1.003750pt}%
\definecolor{currentstroke}{rgb}{0.121569,0.466667,0.705882}%
\pgfsetstrokecolor{currentstroke}%
\pgfsetstrokeopacity{0.584535}%
\pgfsetdash{}{0pt}%
\pgfpathmoveto{\pgfqpoint{0.892332in}{1.494540in}}%
\pgfpathcurveto{\pgfqpoint{0.900568in}{1.494540in}}{\pgfqpoint{0.908468in}{1.497812in}}{\pgfqpoint{0.914292in}{1.503636in}}%
\pgfpathcurveto{\pgfqpoint{0.920116in}{1.509460in}}{\pgfqpoint{0.923389in}{1.517360in}}{\pgfqpoint{0.923389in}{1.525596in}}%
\pgfpathcurveto{\pgfqpoint{0.923389in}{1.533833in}}{\pgfqpoint{0.920116in}{1.541733in}}{\pgfqpoint{0.914292in}{1.547557in}}%
\pgfpathcurveto{\pgfqpoint{0.908468in}{1.553381in}}{\pgfqpoint{0.900568in}{1.556653in}}{\pgfqpoint{0.892332in}{1.556653in}}%
\pgfpathcurveto{\pgfqpoint{0.884096in}{1.556653in}}{\pgfqpoint{0.876196in}{1.553381in}}{\pgfqpoint{0.870372in}{1.547557in}}%
\pgfpathcurveto{\pgfqpoint{0.864548in}{1.541733in}}{\pgfqpoint{0.861276in}{1.533833in}}{\pgfqpoint{0.861276in}{1.525596in}}%
\pgfpathcurveto{\pgfqpoint{0.861276in}{1.517360in}}{\pgfqpoint{0.864548in}{1.509460in}}{\pgfqpoint{0.870372in}{1.503636in}}%
\pgfpathcurveto{\pgfqpoint{0.876196in}{1.497812in}}{\pgfqpoint{0.884096in}{1.494540in}}{\pgfqpoint{0.892332in}{1.494540in}}%
\pgfpathclose%
\pgfusepath{stroke,fill}%
\end{pgfscope}%
\begin{pgfscope}%
\pgfpathrectangle{\pgfqpoint{0.100000in}{0.212622in}}{\pgfqpoint{3.696000in}{3.696000in}}%
\pgfusepath{clip}%
\pgfsetbuttcap%
\pgfsetroundjoin%
\definecolor{currentfill}{rgb}{0.121569,0.466667,0.705882}%
\pgfsetfillcolor{currentfill}%
\pgfsetfillopacity{0.584539}%
\pgfsetlinewidth{1.003750pt}%
\definecolor{currentstroke}{rgb}{0.121569,0.466667,0.705882}%
\pgfsetstrokecolor{currentstroke}%
\pgfsetstrokeopacity{0.584539}%
\pgfsetdash{}{0pt}%
\pgfpathmoveto{\pgfqpoint{0.892327in}{1.494516in}}%
\pgfpathcurveto{\pgfqpoint{0.900563in}{1.494516in}}{\pgfqpoint{0.908463in}{1.497788in}}{\pgfqpoint{0.914287in}{1.503612in}}%
\pgfpathcurveto{\pgfqpoint{0.920111in}{1.509436in}}{\pgfqpoint{0.923383in}{1.517336in}}{\pgfqpoint{0.923383in}{1.525572in}}%
\pgfpathcurveto{\pgfqpoint{0.923383in}{1.533809in}}{\pgfqpoint{0.920111in}{1.541709in}}{\pgfqpoint{0.914287in}{1.547533in}}%
\pgfpathcurveto{\pgfqpoint{0.908463in}{1.553356in}}{\pgfqpoint{0.900563in}{1.556629in}}{\pgfqpoint{0.892327in}{1.556629in}}%
\pgfpathcurveto{\pgfqpoint{0.884091in}{1.556629in}}{\pgfqpoint{0.876191in}{1.553356in}}{\pgfqpoint{0.870367in}{1.547533in}}%
\pgfpathcurveto{\pgfqpoint{0.864543in}{1.541709in}}{\pgfqpoint{0.861270in}{1.533809in}}{\pgfqpoint{0.861270in}{1.525572in}}%
\pgfpathcurveto{\pgfqpoint{0.861270in}{1.517336in}}{\pgfqpoint{0.864543in}{1.509436in}}{\pgfqpoint{0.870367in}{1.503612in}}%
\pgfpathcurveto{\pgfqpoint{0.876191in}{1.497788in}}{\pgfqpoint{0.884091in}{1.494516in}}{\pgfqpoint{0.892327in}{1.494516in}}%
\pgfpathclose%
\pgfusepath{stroke,fill}%
\end{pgfscope}%
\begin{pgfscope}%
\pgfpathrectangle{\pgfqpoint{0.100000in}{0.212622in}}{\pgfqpoint{3.696000in}{3.696000in}}%
\pgfusepath{clip}%
\pgfsetbuttcap%
\pgfsetroundjoin%
\definecolor{currentfill}{rgb}{0.121569,0.466667,0.705882}%
\pgfsetfillcolor{currentfill}%
\pgfsetfillopacity{0.584541}%
\pgfsetlinewidth{1.003750pt}%
\definecolor{currentstroke}{rgb}{0.121569,0.466667,0.705882}%
\pgfsetstrokecolor{currentstroke}%
\pgfsetstrokeopacity{0.584541}%
\pgfsetdash{}{0pt}%
\pgfpathmoveto{\pgfqpoint{0.892324in}{1.494503in}}%
\pgfpathcurveto{\pgfqpoint{0.900561in}{1.494503in}}{\pgfqpoint{0.908461in}{1.497775in}}{\pgfqpoint{0.914285in}{1.503599in}}%
\pgfpathcurveto{\pgfqpoint{0.920108in}{1.509423in}}{\pgfqpoint{0.923381in}{1.517323in}}{\pgfqpoint{0.923381in}{1.525559in}}%
\pgfpathcurveto{\pgfqpoint{0.923381in}{1.533795in}}{\pgfqpoint{0.920108in}{1.541695in}}{\pgfqpoint{0.914285in}{1.547519in}}%
\pgfpathcurveto{\pgfqpoint{0.908461in}{1.553343in}}{\pgfqpoint{0.900561in}{1.556616in}}{\pgfqpoint{0.892324in}{1.556616in}}%
\pgfpathcurveto{\pgfqpoint{0.884088in}{1.556616in}}{\pgfqpoint{0.876188in}{1.553343in}}{\pgfqpoint{0.870364in}{1.547519in}}%
\pgfpathcurveto{\pgfqpoint{0.864540in}{1.541695in}}{\pgfqpoint{0.861268in}{1.533795in}}{\pgfqpoint{0.861268in}{1.525559in}}%
\pgfpathcurveto{\pgfqpoint{0.861268in}{1.517323in}}{\pgfqpoint{0.864540in}{1.509423in}}{\pgfqpoint{0.870364in}{1.503599in}}%
\pgfpathcurveto{\pgfqpoint{0.876188in}{1.497775in}}{\pgfqpoint{0.884088in}{1.494503in}}{\pgfqpoint{0.892324in}{1.494503in}}%
\pgfpathclose%
\pgfusepath{stroke,fill}%
\end{pgfscope}%
\begin{pgfscope}%
\pgfpathrectangle{\pgfqpoint{0.100000in}{0.212622in}}{\pgfqpoint{3.696000in}{3.696000in}}%
\pgfusepath{clip}%
\pgfsetbuttcap%
\pgfsetroundjoin%
\definecolor{currentfill}{rgb}{0.121569,0.466667,0.705882}%
\pgfsetfillcolor{currentfill}%
\pgfsetfillopacity{0.584542}%
\pgfsetlinewidth{1.003750pt}%
\definecolor{currentstroke}{rgb}{0.121569,0.466667,0.705882}%
\pgfsetstrokecolor{currentstroke}%
\pgfsetstrokeopacity{0.584542}%
\pgfsetdash{}{0pt}%
\pgfpathmoveto{\pgfqpoint{0.892323in}{1.494495in}}%
\pgfpathcurveto{\pgfqpoint{0.900559in}{1.494495in}}{\pgfqpoint{0.908459in}{1.497768in}}{\pgfqpoint{0.914283in}{1.503592in}}%
\pgfpathcurveto{\pgfqpoint{0.920107in}{1.509416in}}{\pgfqpoint{0.923379in}{1.517316in}}{\pgfqpoint{0.923379in}{1.525552in}}%
\pgfpathcurveto{\pgfqpoint{0.923379in}{1.533788in}}{\pgfqpoint{0.920107in}{1.541688in}}{\pgfqpoint{0.914283in}{1.547512in}}%
\pgfpathcurveto{\pgfqpoint{0.908459in}{1.553336in}}{\pgfqpoint{0.900559in}{1.556608in}}{\pgfqpoint{0.892323in}{1.556608in}}%
\pgfpathcurveto{\pgfqpoint{0.884087in}{1.556608in}}{\pgfqpoint{0.876186in}{1.553336in}}{\pgfqpoint{0.870363in}{1.547512in}}%
\pgfpathcurveto{\pgfqpoint{0.864539in}{1.541688in}}{\pgfqpoint{0.861266in}{1.533788in}}{\pgfqpoint{0.861266in}{1.525552in}}%
\pgfpathcurveto{\pgfqpoint{0.861266in}{1.517316in}}{\pgfqpoint{0.864539in}{1.509416in}}{\pgfqpoint{0.870363in}{1.503592in}}%
\pgfpathcurveto{\pgfqpoint{0.876186in}{1.497768in}}{\pgfqpoint{0.884087in}{1.494495in}}{\pgfqpoint{0.892323in}{1.494495in}}%
\pgfpathclose%
\pgfusepath{stroke,fill}%
\end{pgfscope}%
\begin{pgfscope}%
\pgfpathrectangle{\pgfqpoint{0.100000in}{0.212622in}}{\pgfqpoint{3.696000in}{3.696000in}}%
\pgfusepath{clip}%
\pgfsetbuttcap%
\pgfsetroundjoin%
\definecolor{currentfill}{rgb}{0.121569,0.466667,0.705882}%
\pgfsetfillcolor{currentfill}%
\pgfsetfillopacity{0.584543}%
\pgfsetlinewidth{1.003750pt}%
\definecolor{currentstroke}{rgb}{0.121569,0.466667,0.705882}%
\pgfsetstrokecolor{currentstroke}%
\pgfsetstrokeopacity{0.584543}%
\pgfsetdash{}{0pt}%
\pgfpathmoveto{\pgfqpoint{0.892322in}{1.494491in}}%
\pgfpathcurveto{\pgfqpoint{0.900558in}{1.494491in}}{\pgfqpoint{0.908458in}{1.497764in}}{\pgfqpoint{0.914282in}{1.503588in}}%
\pgfpathcurveto{\pgfqpoint{0.920106in}{1.509412in}}{\pgfqpoint{0.923379in}{1.517312in}}{\pgfqpoint{0.923379in}{1.525548in}}%
\pgfpathcurveto{\pgfqpoint{0.923379in}{1.533784in}}{\pgfqpoint{0.920106in}{1.541684in}}{\pgfqpoint{0.914282in}{1.547508in}}%
\pgfpathcurveto{\pgfqpoint{0.908458in}{1.553332in}}{\pgfqpoint{0.900558in}{1.556604in}}{\pgfqpoint{0.892322in}{1.556604in}}%
\pgfpathcurveto{\pgfqpoint{0.884086in}{1.556604in}}{\pgfqpoint{0.876186in}{1.553332in}}{\pgfqpoint{0.870362in}{1.547508in}}%
\pgfpathcurveto{\pgfqpoint{0.864538in}{1.541684in}}{\pgfqpoint{0.861266in}{1.533784in}}{\pgfqpoint{0.861266in}{1.525548in}}%
\pgfpathcurveto{\pgfqpoint{0.861266in}{1.517312in}}{\pgfqpoint{0.864538in}{1.509412in}}{\pgfqpoint{0.870362in}{1.503588in}}%
\pgfpathcurveto{\pgfqpoint{0.876186in}{1.497764in}}{\pgfqpoint{0.884086in}{1.494491in}}{\pgfqpoint{0.892322in}{1.494491in}}%
\pgfpathclose%
\pgfusepath{stroke,fill}%
\end{pgfscope}%
\begin{pgfscope}%
\pgfpathrectangle{\pgfqpoint{0.100000in}{0.212622in}}{\pgfqpoint{3.696000in}{3.696000in}}%
\pgfusepath{clip}%
\pgfsetbuttcap%
\pgfsetroundjoin%
\definecolor{currentfill}{rgb}{0.121569,0.466667,0.705882}%
\pgfsetfillcolor{currentfill}%
\pgfsetfillopacity{0.584543}%
\pgfsetlinewidth{1.003750pt}%
\definecolor{currentstroke}{rgb}{0.121569,0.466667,0.705882}%
\pgfsetstrokecolor{currentstroke}%
\pgfsetstrokeopacity{0.584543}%
\pgfsetdash{}{0pt}%
\pgfpathmoveto{\pgfqpoint{0.892322in}{1.494489in}}%
\pgfpathcurveto{\pgfqpoint{0.900558in}{1.494489in}}{\pgfqpoint{0.908458in}{1.497762in}}{\pgfqpoint{0.914282in}{1.503585in}}%
\pgfpathcurveto{\pgfqpoint{0.920106in}{1.509409in}}{\pgfqpoint{0.923378in}{1.517309in}}{\pgfqpoint{0.923378in}{1.525546in}}%
\pgfpathcurveto{\pgfqpoint{0.923378in}{1.533782in}}{\pgfqpoint{0.920106in}{1.541682in}}{\pgfqpoint{0.914282in}{1.547506in}}%
\pgfpathcurveto{\pgfqpoint{0.908458in}{1.553330in}}{\pgfqpoint{0.900558in}{1.556602in}}{\pgfqpoint{0.892322in}{1.556602in}}%
\pgfpathcurveto{\pgfqpoint{0.884085in}{1.556602in}}{\pgfqpoint{0.876185in}{1.553330in}}{\pgfqpoint{0.870361in}{1.547506in}}%
\pgfpathcurveto{\pgfqpoint{0.864537in}{1.541682in}}{\pgfqpoint{0.861265in}{1.533782in}}{\pgfqpoint{0.861265in}{1.525546in}}%
\pgfpathcurveto{\pgfqpoint{0.861265in}{1.517309in}}{\pgfqpoint{0.864537in}{1.509409in}}{\pgfqpoint{0.870361in}{1.503585in}}%
\pgfpathcurveto{\pgfqpoint{0.876185in}{1.497762in}}{\pgfqpoint{0.884085in}{1.494489in}}{\pgfqpoint{0.892322in}{1.494489in}}%
\pgfpathclose%
\pgfusepath{stroke,fill}%
\end{pgfscope}%
\begin{pgfscope}%
\pgfpathrectangle{\pgfqpoint{0.100000in}{0.212622in}}{\pgfqpoint{3.696000in}{3.696000in}}%
\pgfusepath{clip}%
\pgfsetbuttcap%
\pgfsetroundjoin%
\definecolor{currentfill}{rgb}{0.121569,0.466667,0.705882}%
\pgfsetfillcolor{currentfill}%
\pgfsetfillopacity{0.584619}%
\pgfsetlinewidth{1.003750pt}%
\definecolor{currentstroke}{rgb}{0.121569,0.466667,0.705882}%
\pgfsetstrokecolor{currentstroke}%
\pgfsetstrokeopacity{0.584619}%
\pgfsetdash{}{0pt}%
\pgfpathmoveto{\pgfqpoint{0.892238in}{1.494078in}}%
\pgfpathcurveto{\pgfqpoint{0.900474in}{1.494078in}}{\pgfqpoint{0.908374in}{1.497350in}}{\pgfqpoint{0.914198in}{1.503174in}}%
\pgfpathcurveto{\pgfqpoint{0.920022in}{1.508998in}}{\pgfqpoint{0.923295in}{1.516898in}}{\pgfqpoint{0.923295in}{1.525134in}}%
\pgfpathcurveto{\pgfqpoint{0.923295in}{1.533371in}}{\pgfqpoint{0.920022in}{1.541271in}}{\pgfqpoint{0.914198in}{1.547095in}}%
\pgfpathcurveto{\pgfqpoint{0.908374in}{1.552919in}}{\pgfqpoint{0.900474in}{1.556191in}}{\pgfqpoint{0.892238in}{1.556191in}}%
\pgfpathcurveto{\pgfqpoint{0.884002in}{1.556191in}}{\pgfqpoint{0.876102in}{1.552919in}}{\pgfqpoint{0.870278in}{1.547095in}}%
\pgfpathcurveto{\pgfqpoint{0.864454in}{1.541271in}}{\pgfqpoint{0.861182in}{1.533371in}}{\pgfqpoint{0.861182in}{1.525134in}}%
\pgfpathcurveto{\pgfqpoint{0.861182in}{1.516898in}}{\pgfqpoint{0.864454in}{1.508998in}}{\pgfqpoint{0.870278in}{1.503174in}}%
\pgfpathcurveto{\pgfqpoint{0.876102in}{1.497350in}}{\pgfqpoint{0.884002in}{1.494078in}}{\pgfqpoint{0.892238in}{1.494078in}}%
\pgfpathclose%
\pgfusepath{stroke,fill}%
\end{pgfscope}%
\begin{pgfscope}%
\pgfpathrectangle{\pgfqpoint{0.100000in}{0.212622in}}{\pgfqpoint{3.696000in}{3.696000in}}%
\pgfusepath{clip}%
\pgfsetbuttcap%
\pgfsetroundjoin%
\definecolor{currentfill}{rgb}{0.121569,0.466667,0.705882}%
\pgfsetfillcolor{currentfill}%
\pgfsetfillopacity{0.584661}%
\pgfsetlinewidth{1.003750pt}%
\definecolor{currentstroke}{rgb}{0.121569,0.466667,0.705882}%
\pgfsetstrokecolor{currentstroke}%
\pgfsetstrokeopacity{0.584661}%
\pgfsetdash{}{0pt}%
\pgfpathmoveto{\pgfqpoint{0.892187in}{1.493853in}}%
\pgfpathcurveto{\pgfqpoint{0.900424in}{1.493853in}}{\pgfqpoint{0.908324in}{1.497125in}}{\pgfqpoint{0.914148in}{1.502949in}}%
\pgfpathcurveto{\pgfqpoint{0.919972in}{1.508773in}}{\pgfqpoint{0.923244in}{1.516673in}}{\pgfqpoint{0.923244in}{1.524909in}}%
\pgfpathcurveto{\pgfqpoint{0.923244in}{1.533146in}}{\pgfqpoint{0.919972in}{1.541046in}}{\pgfqpoint{0.914148in}{1.546870in}}%
\pgfpathcurveto{\pgfqpoint{0.908324in}{1.552694in}}{\pgfqpoint{0.900424in}{1.555966in}}{\pgfqpoint{0.892187in}{1.555966in}}%
\pgfpathcurveto{\pgfqpoint{0.883951in}{1.555966in}}{\pgfqpoint{0.876051in}{1.552694in}}{\pgfqpoint{0.870227in}{1.546870in}}%
\pgfpathcurveto{\pgfqpoint{0.864403in}{1.541046in}}{\pgfqpoint{0.861131in}{1.533146in}}{\pgfqpoint{0.861131in}{1.524909in}}%
\pgfpathcurveto{\pgfqpoint{0.861131in}{1.516673in}}{\pgfqpoint{0.864403in}{1.508773in}}{\pgfqpoint{0.870227in}{1.502949in}}%
\pgfpathcurveto{\pgfqpoint{0.876051in}{1.497125in}}{\pgfqpoint{0.883951in}{1.493853in}}{\pgfqpoint{0.892187in}{1.493853in}}%
\pgfpathclose%
\pgfusepath{stroke,fill}%
\end{pgfscope}%
\begin{pgfscope}%
\pgfpathrectangle{\pgfqpoint{0.100000in}{0.212622in}}{\pgfqpoint{3.696000in}{3.696000in}}%
\pgfusepath{clip}%
\pgfsetbuttcap%
\pgfsetroundjoin%
\definecolor{currentfill}{rgb}{0.121569,0.466667,0.705882}%
\pgfsetfillcolor{currentfill}%
\pgfsetfillopacity{0.584777}%
\pgfsetlinewidth{1.003750pt}%
\definecolor{currentstroke}{rgb}{0.121569,0.466667,0.705882}%
\pgfsetstrokecolor{currentstroke}%
\pgfsetstrokeopacity{0.584777}%
\pgfsetdash{}{0pt}%
\pgfpathmoveto{\pgfqpoint{0.892040in}{1.493237in}}%
\pgfpathcurveto{\pgfqpoint{0.900276in}{1.493237in}}{\pgfqpoint{0.908176in}{1.496509in}}{\pgfqpoint{0.914000in}{1.502333in}}%
\pgfpathcurveto{\pgfqpoint{0.919824in}{1.508157in}}{\pgfqpoint{0.923096in}{1.516057in}}{\pgfqpoint{0.923096in}{1.524294in}}%
\pgfpathcurveto{\pgfqpoint{0.923096in}{1.532530in}}{\pgfqpoint{0.919824in}{1.540430in}}{\pgfqpoint{0.914000in}{1.546254in}}%
\pgfpathcurveto{\pgfqpoint{0.908176in}{1.552078in}}{\pgfqpoint{0.900276in}{1.555350in}}{\pgfqpoint{0.892040in}{1.555350in}}%
\pgfpathcurveto{\pgfqpoint{0.883803in}{1.555350in}}{\pgfqpoint{0.875903in}{1.552078in}}{\pgfqpoint{0.870079in}{1.546254in}}%
\pgfpathcurveto{\pgfqpoint{0.864255in}{1.540430in}}{\pgfqpoint{0.860983in}{1.532530in}}{\pgfqpoint{0.860983in}{1.524294in}}%
\pgfpathcurveto{\pgfqpoint{0.860983in}{1.516057in}}{\pgfqpoint{0.864255in}{1.508157in}}{\pgfqpoint{0.870079in}{1.502333in}}%
\pgfpathcurveto{\pgfqpoint{0.875903in}{1.496509in}}{\pgfqpoint{0.883803in}{1.493237in}}{\pgfqpoint{0.892040in}{1.493237in}}%
\pgfpathclose%
\pgfusepath{stroke,fill}%
\end{pgfscope}%
\begin{pgfscope}%
\pgfpathrectangle{\pgfqpoint{0.100000in}{0.212622in}}{\pgfqpoint{3.696000in}{3.696000in}}%
\pgfusepath{clip}%
\pgfsetbuttcap%
\pgfsetroundjoin%
\definecolor{currentfill}{rgb}{0.121569,0.466667,0.705882}%
\pgfsetfillcolor{currentfill}%
\pgfsetfillopacity{0.584842}%
\pgfsetlinewidth{1.003750pt}%
\definecolor{currentstroke}{rgb}{0.121569,0.466667,0.705882}%
\pgfsetstrokecolor{currentstroke}%
\pgfsetstrokeopacity{0.584842}%
\pgfsetdash{}{0pt}%
\pgfpathmoveto{\pgfqpoint{0.891960in}{1.492905in}}%
\pgfpathcurveto{\pgfqpoint{0.900196in}{1.492905in}}{\pgfqpoint{0.908096in}{1.496177in}}{\pgfqpoint{0.913920in}{1.502001in}}%
\pgfpathcurveto{\pgfqpoint{0.919744in}{1.507825in}}{\pgfqpoint{0.923016in}{1.515725in}}{\pgfqpoint{0.923016in}{1.523961in}}%
\pgfpathcurveto{\pgfqpoint{0.923016in}{1.532197in}}{\pgfqpoint{0.919744in}{1.540097in}}{\pgfqpoint{0.913920in}{1.545921in}}%
\pgfpathcurveto{\pgfqpoint{0.908096in}{1.551745in}}{\pgfqpoint{0.900196in}{1.555018in}}{\pgfqpoint{0.891960in}{1.555018in}}%
\pgfpathcurveto{\pgfqpoint{0.883723in}{1.555018in}}{\pgfqpoint{0.875823in}{1.551745in}}{\pgfqpoint{0.869999in}{1.545921in}}%
\pgfpathcurveto{\pgfqpoint{0.864175in}{1.540097in}}{\pgfqpoint{0.860903in}{1.532197in}}{\pgfqpoint{0.860903in}{1.523961in}}%
\pgfpathcurveto{\pgfqpoint{0.860903in}{1.515725in}}{\pgfqpoint{0.864175in}{1.507825in}}{\pgfqpoint{0.869999in}{1.502001in}}%
\pgfpathcurveto{\pgfqpoint{0.875823in}{1.496177in}}{\pgfqpoint{0.883723in}{1.492905in}}{\pgfqpoint{0.891960in}{1.492905in}}%
\pgfpathclose%
\pgfusepath{stroke,fill}%
\end{pgfscope}%
\begin{pgfscope}%
\pgfpathrectangle{\pgfqpoint{0.100000in}{0.212622in}}{\pgfqpoint{3.696000in}{3.696000in}}%
\pgfusepath{clip}%
\pgfsetbuttcap%
\pgfsetroundjoin%
\definecolor{currentfill}{rgb}{0.121569,0.466667,0.705882}%
\pgfsetfillcolor{currentfill}%
\pgfsetfillopacity{0.584877}%
\pgfsetlinewidth{1.003750pt}%
\definecolor{currentstroke}{rgb}{0.121569,0.466667,0.705882}%
\pgfsetstrokecolor{currentstroke}%
\pgfsetstrokeopacity{0.584877}%
\pgfsetdash{}{0pt}%
\pgfpathmoveto{\pgfqpoint{0.891913in}{1.492723in}}%
\pgfpathcurveto{\pgfqpoint{0.900149in}{1.492723in}}{\pgfqpoint{0.908049in}{1.495995in}}{\pgfqpoint{0.913873in}{1.501819in}}%
\pgfpathcurveto{\pgfqpoint{0.919697in}{1.507643in}}{\pgfqpoint{0.922969in}{1.515543in}}{\pgfqpoint{0.922969in}{1.523779in}}%
\pgfpathcurveto{\pgfqpoint{0.922969in}{1.532015in}}{\pgfqpoint{0.919697in}{1.539915in}}{\pgfqpoint{0.913873in}{1.545739in}}%
\pgfpathcurveto{\pgfqpoint{0.908049in}{1.551563in}}{\pgfqpoint{0.900149in}{1.554836in}}{\pgfqpoint{0.891913in}{1.554836in}}%
\pgfpathcurveto{\pgfqpoint{0.883677in}{1.554836in}}{\pgfqpoint{0.875777in}{1.551563in}}{\pgfqpoint{0.869953in}{1.545739in}}%
\pgfpathcurveto{\pgfqpoint{0.864129in}{1.539915in}}{\pgfqpoint{0.860856in}{1.532015in}}{\pgfqpoint{0.860856in}{1.523779in}}%
\pgfpathcurveto{\pgfqpoint{0.860856in}{1.515543in}}{\pgfqpoint{0.864129in}{1.507643in}}{\pgfqpoint{0.869953in}{1.501819in}}%
\pgfpathcurveto{\pgfqpoint{0.875777in}{1.495995in}}{\pgfqpoint{0.883677in}{1.492723in}}{\pgfqpoint{0.891913in}{1.492723in}}%
\pgfpathclose%
\pgfusepath{stroke,fill}%
\end{pgfscope}%
\begin{pgfscope}%
\pgfpathrectangle{\pgfqpoint{0.100000in}{0.212622in}}{\pgfqpoint{3.696000in}{3.696000in}}%
\pgfusepath{clip}%
\pgfsetbuttcap%
\pgfsetroundjoin%
\definecolor{currentfill}{rgb}{0.121569,0.466667,0.705882}%
\pgfsetfillcolor{currentfill}%
\pgfsetfillopacity{0.584897}%
\pgfsetlinewidth{1.003750pt}%
\definecolor{currentstroke}{rgb}{0.121569,0.466667,0.705882}%
\pgfsetstrokecolor{currentstroke}%
\pgfsetstrokeopacity{0.584897}%
\pgfsetdash{}{0pt}%
\pgfpathmoveto{\pgfqpoint{0.891886in}{1.492623in}}%
\pgfpathcurveto{\pgfqpoint{0.900123in}{1.492623in}}{\pgfqpoint{0.908023in}{1.495895in}}{\pgfqpoint{0.913847in}{1.501719in}}%
\pgfpathcurveto{\pgfqpoint{0.919670in}{1.507543in}}{\pgfqpoint{0.922943in}{1.515443in}}{\pgfqpoint{0.922943in}{1.523679in}}%
\pgfpathcurveto{\pgfqpoint{0.922943in}{1.531916in}}{\pgfqpoint{0.919670in}{1.539816in}}{\pgfqpoint{0.913847in}{1.545640in}}%
\pgfpathcurveto{\pgfqpoint{0.908023in}{1.551463in}}{\pgfqpoint{0.900123in}{1.554736in}}{\pgfqpoint{0.891886in}{1.554736in}}%
\pgfpathcurveto{\pgfqpoint{0.883650in}{1.554736in}}{\pgfqpoint{0.875750in}{1.551463in}}{\pgfqpoint{0.869926in}{1.545640in}}%
\pgfpathcurveto{\pgfqpoint{0.864102in}{1.539816in}}{\pgfqpoint{0.860830in}{1.531916in}}{\pgfqpoint{0.860830in}{1.523679in}}%
\pgfpathcurveto{\pgfqpoint{0.860830in}{1.515443in}}{\pgfqpoint{0.864102in}{1.507543in}}{\pgfqpoint{0.869926in}{1.501719in}}%
\pgfpathcurveto{\pgfqpoint{0.875750in}{1.495895in}}{\pgfqpoint{0.883650in}{1.492623in}}{\pgfqpoint{0.891886in}{1.492623in}}%
\pgfpathclose%
\pgfusepath{stroke,fill}%
\end{pgfscope}%
\begin{pgfscope}%
\pgfpathrectangle{\pgfqpoint{0.100000in}{0.212622in}}{\pgfqpoint{3.696000in}{3.696000in}}%
\pgfusepath{clip}%
\pgfsetbuttcap%
\pgfsetroundjoin%
\definecolor{currentfill}{rgb}{0.121569,0.466667,0.705882}%
\pgfsetfillcolor{currentfill}%
\pgfsetfillopacity{0.584908}%
\pgfsetlinewidth{1.003750pt}%
\definecolor{currentstroke}{rgb}{0.121569,0.466667,0.705882}%
\pgfsetstrokecolor{currentstroke}%
\pgfsetstrokeopacity{0.584908}%
\pgfsetdash{}{0pt}%
\pgfpathmoveto{\pgfqpoint{0.891871in}{1.492568in}}%
\pgfpathcurveto{\pgfqpoint{0.900107in}{1.492568in}}{\pgfqpoint{0.908007in}{1.495840in}}{\pgfqpoint{0.913831in}{1.501664in}}%
\pgfpathcurveto{\pgfqpoint{0.919655in}{1.507488in}}{\pgfqpoint{0.922927in}{1.515388in}}{\pgfqpoint{0.922927in}{1.523624in}}%
\pgfpathcurveto{\pgfqpoint{0.922927in}{1.531860in}}{\pgfqpoint{0.919655in}{1.539760in}}{\pgfqpoint{0.913831in}{1.545584in}}%
\pgfpathcurveto{\pgfqpoint{0.908007in}{1.551408in}}{\pgfqpoint{0.900107in}{1.554681in}}{\pgfqpoint{0.891871in}{1.554681in}}%
\pgfpathcurveto{\pgfqpoint{0.883635in}{1.554681in}}{\pgfqpoint{0.875735in}{1.551408in}}{\pgfqpoint{0.869911in}{1.545584in}}%
\pgfpathcurveto{\pgfqpoint{0.864087in}{1.539760in}}{\pgfqpoint{0.860814in}{1.531860in}}{\pgfqpoint{0.860814in}{1.523624in}}%
\pgfpathcurveto{\pgfqpoint{0.860814in}{1.515388in}}{\pgfqpoint{0.864087in}{1.507488in}}{\pgfqpoint{0.869911in}{1.501664in}}%
\pgfpathcurveto{\pgfqpoint{0.875735in}{1.495840in}}{\pgfqpoint{0.883635in}{1.492568in}}{\pgfqpoint{0.891871in}{1.492568in}}%
\pgfpathclose%
\pgfusepath{stroke,fill}%
\end{pgfscope}%
\begin{pgfscope}%
\pgfpathrectangle{\pgfqpoint{0.100000in}{0.212622in}}{\pgfqpoint{3.696000in}{3.696000in}}%
\pgfusepath{clip}%
\pgfsetbuttcap%
\pgfsetroundjoin%
\definecolor{currentfill}{rgb}{0.121569,0.466667,0.705882}%
\pgfsetfillcolor{currentfill}%
\pgfsetfillopacity{0.584914}%
\pgfsetlinewidth{1.003750pt}%
\definecolor{currentstroke}{rgb}{0.121569,0.466667,0.705882}%
\pgfsetstrokecolor{currentstroke}%
\pgfsetstrokeopacity{0.584914}%
\pgfsetdash{}{0pt}%
\pgfpathmoveto{\pgfqpoint{0.891863in}{1.492537in}}%
\pgfpathcurveto{\pgfqpoint{0.900099in}{1.492537in}}{\pgfqpoint{0.907999in}{1.495809in}}{\pgfqpoint{0.913823in}{1.501633in}}%
\pgfpathcurveto{\pgfqpoint{0.919647in}{1.507457in}}{\pgfqpoint{0.922919in}{1.515357in}}{\pgfqpoint{0.922919in}{1.523593in}}%
\pgfpathcurveto{\pgfqpoint{0.922919in}{1.531830in}}{\pgfqpoint{0.919647in}{1.539730in}}{\pgfqpoint{0.913823in}{1.545554in}}%
\pgfpathcurveto{\pgfqpoint{0.907999in}{1.551378in}}{\pgfqpoint{0.900099in}{1.554650in}}{\pgfqpoint{0.891863in}{1.554650in}}%
\pgfpathcurveto{\pgfqpoint{0.883626in}{1.554650in}}{\pgfqpoint{0.875726in}{1.551378in}}{\pgfqpoint{0.869902in}{1.545554in}}%
\pgfpathcurveto{\pgfqpoint{0.864078in}{1.539730in}}{\pgfqpoint{0.860806in}{1.531830in}}{\pgfqpoint{0.860806in}{1.523593in}}%
\pgfpathcurveto{\pgfqpoint{0.860806in}{1.515357in}}{\pgfqpoint{0.864078in}{1.507457in}}{\pgfqpoint{0.869902in}{1.501633in}}%
\pgfpathcurveto{\pgfqpoint{0.875726in}{1.495809in}}{\pgfqpoint{0.883626in}{1.492537in}}{\pgfqpoint{0.891863in}{1.492537in}}%
\pgfpathclose%
\pgfusepath{stroke,fill}%
\end{pgfscope}%
\begin{pgfscope}%
\pgfpathrectangle{\pgfqpoint{0.100000in}{0.212622in}}{\pgfqpoint{3.696000in}{3.696000in}}%
\pgfusepath{clip}%
\pgfsetbuttcap%
\pgfsetroundjoin%
\definecolor{currentfill}{rgb}{0.121569,0.466667,0.705882}%
\pgfsetfillcolor{currentfill}%
\pgfsetfillopacity{0.584917}%
\pgfsetlinewidth{1.003750pt}%
\definecolor{currentstroke}{rgb}{0.121569,0.466667,0.705882}%
\pgfsetstrokecolor{currentstroke}%
\pgfsetstrokeopacity{0.584917}%
\pgfsetdash{}{0pt}%
\pgfpathmoveto{\pgfqpoint{0.891858in}{1.492520in}}%
\pgfpathcurveto{\pgfqpoint{0.900094in}{1.492520in}}{\pgfqpoint{0.907994in}{1.495793in}}{\pgfqpoint{0.913818in}{1.501616in}}%
\pgfpathcurveto{\pgfqpoint{0.919642in}{1.507440in}}{\pgfqpoint{0.922914in}{1.515340in}}{\pgfqpoint{0.922914in}{1.523577in}}%
\pgfpathcurveto{\pgfqpoint{0.922914in}{1.531813in}}{\pgfqpoint{0.919642in}{1.539713in}}{\pgfqpoint{0.913818in}{1.545537in}}%
\pgfpathcurveto{\pgfqpoint{0.907994in}{1.551361in}}{\pgfqpoint{0.900094in}{1.554633in}}{\pgfqpoint{0.891858in}{1.554633in}}%
\pgfpathcurveto{\pgfqpoint{0.883622in}{1.554633in}}{\pgfqpoint{0.875721in}{1.551361in}}{\pgfqpoint{0.869898in}{1.545537in}}%
\pgfpathcurveto{\pgfqpoint{0.864074in}{1.539713in}}{\pgfqpoint{0.860801in}{1.531813in}}{\pgfqpoint{0.860801in}{1.523577in}}%
\pgfpathcurveto{\pgfqpoint{0.860801in}{1.515340in}}{\pgfqpoint{0.864074in}{1.507440in}}{\pgfqpoint{0.869898in}{1.501616in}}%
\pgfpathcurveto{\pgfqpoint{0.875721in}{1.495793in}}{\pgfqpoint{0.883622in}{1.492520in}}{\pgfqpoint{0.891858in}{1.492520in}}%
\pgfpathclose%
\pgfusepath{stroke,fill}%
\end{pgfscope}%
\begin{pgfscope}%
\pgfpathrectangle{\pgfqpoint{0.100000in}{0.212622in}}{\pgfqpoint{3.696000in}{3.696000in}}%
\pgfusepath{clip}%
\pgfsetbuttcap%
\pgfsetroundjoin%
\definecolor{currentfill}{rgb}{0.121569,0.466667,0.705882}%
\pgfsetfillcolor{currentfill}%
\pgfsetfillopacity{0.584919}%
\pgfsetlinewidth{1.003750pt}%
\definecolor{currentstroke}{rgb}{0.121569,0.466667,0.705882}%
\pgfsetstrokecolor{currentstroke}%
\pgfsetstrokeopacity{0.584919}%
\pgfsetdash{}{0pt}%
\pgfpathmoveto{\pgfqpoint{0.891855in}{1.492511in}}%
\pgfpathcurveto{\pgfqpoint{0.900092in}{1.492511in}}{\pgfqpoint{0.907992in}{1.495784in}}{\pgfqpoint{0.913815in}{1.501608in}}%
\pgfpathcurveto{\pgfqpoint{0.919639in}{1.507431in}}{\pgfqpoint{0.922912in}{1.515332in}}{\pgfqpoint{0.922912in}{1.523568in}}%
\pgfpathcurveto{\pgfqpoint{0.922912in}{1.531804in}}{\pgfqpoint{0.919639in}{1.539704in}}{\pgfqpoint{0.913815in}{1.545528in}}%
\pgfpathcurveto{\pgfqpoint{0.907992in}{1.551352in}}{\pgfqpoint{0.900092in}{1.554624in}}{\pgfqpoint{0.891855in}{1.554624in}}%
\pgfpathcurveto{\pgfqpoint{0.883619in}{1.554624in}}{\pgfqpoint{0.875719in}{1.551352in}}{\pgfqpoint{0.869895in}{1.545528in}}%
\pgfpathcurveto{\pgfqpoint{0.864071in}{1.539704in}}{\pgfqpoint{0.860799in}{1.531804in}}{\pgfqpoint{0.860799in}{1.523568in}}%
\pgfpathcurveto{\pgfqpoint{0.860799in}{1.515332in}}{\pgfqpoint{0.864071in}{1.507431in}}{\pgfqpoint{0.869895in}{1.501608in}}%
\pgfpathcurveto{\pgfqpoint{0.875719in}{1.495784in}}{\pgfqpoint{0.883619in}{1.492511in}}{\pgfqpoint{0.891855in}{1.492511in}}%
\pgfpathclose%
\pgfusepath{stroke,fill}%
\end{pgfscope}%
\begin{pgfscope}%
\pgfpathrectangle{\pgfqpoint{0.100000in}{0.212622in}}{\pgfqpoint{3.696000in}{3.696000in}}%
\pgfusepath{clip}%
\pgfsetbuttcap%
\pgfsetroundjoin%
\definecolor{currentfill}{rgb}{0.121569,0.466667,0.705882}%
\pgfsetfillcolor{currentfill}%
\pgfsetfillopacity{0.584920}%
\pgfsetlinewidth{1.003750pt}%
\definecolor{currentstroke}{rgb}{0.121569,0.466667,0.705882}%
\pgfsetstrokecolor{currentstroke}%
\pgfsetstrokeopacity{0.584920}%
\pgfsetdash{}{0pt}%
\pgfpathmoveto{\pgfqpoint{0.891854in}{1.492506in}}%
\pgfpathcurveto{\pgfqpoint{0.900090in}{1.492506in}}{\pgfqpoint{0.907990in}{1.495779in}}{\pgfqpoint{0.913814in}{1.501603in}}%
\pgfpathcurveto{\pgfqpoint{0.919638in}{1.507427in}}{\pgfqpoint{0.922910in}{1.515327in}}{\pgfqpoint{0.922910in}{1.523563in}}%
\pgfpathcurveto{\pgfqpoint{0.922910in}{1.531799in}}{\pgfqpoint{0.919638in}{1.539699in}}{\pgfqpoint{0.913814in}{1.545523in}}%
\pgfpathcurveto{\pgfqpoint{0.907990in}{1.551347in}}{\pgfqpoint{0.900090in}{1.554619in}}{\pgfqpoint{0.891854in}{1.554619in}}%
\pgfpathcurveto{\pgfqpoint{0.883618in}{1.554619in}}{\pgfqpoint{0.875717in}{1.551347in}}{\pgfqpoint{0.869894in}{1.545523in}}%
\pgfpathcurveto{\pgfqpoint{0.864070in}{1.539699in}}{\pgfqpoint{0.860797in}{1.531799in}}{\pgfqpoint{0.860797in}{1.523563in}}%
\pgfpathcurveto{\pgfqpoint{0.860797in}{1.515327in}}{\pgfqpoint{0.864070in}{1.507427in}}{\pgfqpoint{0.869894in}{1.501603in}}%
\pgfpathcurveto{\pgfqpoint{0.875717in}{1.495779in}}{\pgfqpoint{0.883618in}{1.492506in}}{\pgfqpoint{0.891854in}{1.492506in}}%
\pgfpathclose%
\pgfusepath{stroke,fill}%
\end{pgfscope}%
\begin{pgfscope}%
\pgfpathrectangle{\pgfqpoint{0.100000in}{0.212622in}}{\pgfqpoint{3.696000in}{3.696000in}}%
\pgfusepath{clip}%
\pgfsetbuttcap%
\pgfsetroundjoin%
\definecolor{currentfill}{rgb}{0.121569,0.466667,0.705882}%
\pgfsetfillcolor{currentfill}%
\pgfsetfillopacity{0.584996}%
\pgfsetlinewidth{1.003750pt}%
\definecolor{currentstroke}{rgb}{0.121569,0.466667,0.705882}%
\pgfsetstrokecolor{currentstroke}%
\pgfsetstrokeopacity{0.584996}%
\pgfsetdash{}{0pt}%
\pgfpathmoveto{\pgfqpoint{0.891754in}{1.492164in}}%
\pgfpathcurveto{\pgfqpoint{0.899990in}{1.492164in}}{\pgfqpoint{0.907890in}{1.495437in}}{\pgfqpoint{0.913714in}{1.501261in}}%
\pgfpathcurveto{\pgfqpoint{0.919538in}{1.507085in}}{\pgfqpoint{0.922810in}{1.514985in}}{\pgfqpoint{0.922810in}{1.523221in}}%
\pgfpathcurveto{\pgfqpoint{0.922810in}{1.531457in}}{\pgfqpoint{0.919538in}{1.539357in}}{\pgfqpoint{0.913714in}{1.545181in}}%
\pgfpathcurveto{\pgfqpoint{0.907890in}{1.551005in}}{\pgfqpoint{0.899990in}{1.554277in}}{\pgfqpoint{0.891754in}{1.554277in}}%
\pgfpathcurveto{\pgfqpoint{0.883518in}{1.554277in}}{\pgfqpoint{0.875618in}{1.551005in}}{\pgfqpoint{0.869794in}{1.545181in}}%
\pgfpathcurveto{\pgfqpoint{0.863970in}{1.539357in}}{\pgfqpoint{0.860697in}{1.531457in}}{\pgfqpoint{0.860697in}{1.523221in}}%
\pgfpathcurveto{\pgfqpoint{0.860697in}{1.514985in}}{\pgfqpoint{0.863970in}{1.507085in}}{\pgfqpoint{0.869794in}{1.501261in}}%
\pgfpathcurveto{\pgfqpoint{0.875618in}{1.495437in}}{\pgfqpoint{0.883518in}{1.492164in}}{\pgfqpoint{0.891754in}{1.492164in}}%
\pgfpathclose%
\pgfusepath{stroke,fill}%
\end{pgfscope}%
\begin{pgfscope}%
\pgfpathrectangle{\pgfqpoint{0.100000in}{0.212622in}}{\pgfqpoint{3.696000in}{3.696000in}}%
\pgfusepath{clip}%
\pgfsetbuttcap%
\pgfsetroundjoin%
\definecolor{currentfill}{rgb}{0.121569,0.466667,0.705882}%
\pgfsetfillcolor{currentfill}%
\pgfsetfillopacity{0.585037}%
\pgfsetlinewidth{1.003750pt}%
\definecolor{currentstroke}{rgb}{0.121569,0.466667,0.705882}%
\pgfsetstrokecolor{currentstroke}%
\pgfsetstrokeopacity{0.585037}%
\pgfsetdash{}{0pt}%
\pgfpathmoveto{\pgfqpoint{0.891698in}{1.491976in}}%
\pgfpathcurveto{\pgfqpoint{0.899934in}{1.491976in}}{\pgfqpoint{0.907834in}{1.495248in}}{\pgfqpoint{0.913658in}{1.501072in}}%
\pgfpathcurveto{\pgfqpoint{0.919482in}{1.506896in}}{\pgfqpoint{0.922754in}{1.514796in}}{\pgfqpoint{0.922754in}{1.523033in}}%
\pgfpathcurveto{\pgfqpoint{0.922754in}{1.531269in}}{\pgfqpoint{0.919482in}{1.539169in}}{\pgfqpoint{0.913658in}{1.544993in}}%
\pgfpathcurveto{\pgfqpoint{0.907834in}{1.550817in}}{\pgfqpoint{0.899934in}{1.554089in}}{\pgfqpoint{0.891698in}{1.554089in}}%
\pgfpathcurveto{\pgfqpoint{0.883462in}{1.554089in}}{\pgfqpoint{0.875562in}{1.550817in}}{\pgfqpoint{0.869738in}{1.544993in}}%
\pgfpathcurveto{\pgfqpoint{0.863914in}{1.539169in}}{\pgfqpoint{0.860641in}{1.531269in}}{\pgfqpoint{0.860641in}{1.523033in}}%
\pgfpathcurveto{\pgfqpoint{0.860641in}{1.514796in}}{\pgfqpoint{0.863914in}{1.506896in}}{\pgfqpoint{0.869738in}{1.501072in}}%
\pgfpathcurveto{\pgfqpoint{0.875562in}{1.495248in}}{\pgfqpoint{0.883462in}{1.491976in}}{\pgfqpoint{0.891698in}{1.491976in}}%
\pgfpathclose%
\pgfusepath{stroke,fill}%
\end{pgfscope}%
\begin{pgfscope}%
\pgfpathrectangle{\pgfqpoint{0.100000in}{0.212622in}}{\pgfqpoint{3.696000in}{3.696000in}}%
\pgfusepath{clip}%
\pgfsetbuttcap%
\pgfsetroundjoin%
\definecolor{currentfill}{rgb}{0.121569,0.466667,0.705882}%
\pgfsetfillcolor{currentfill}%
\pgfsetfillopacity{0.585059}%
\pgfsetlinewidth{1.003750pt}%
\definecolor{currentstroke}{rgb}{0.121569,0.466667,0.705882}%
\pgfsetstrokecolor{currentstroke}%
\pgfsetstrokeopacity{0.585059}%
\pgfsetdash{}{0pt}%
\pgfpathmoveto{\pgfqpoint{0.891666in}{1.491868in}}%
\pgfpathcurveto{\pgfqpoint{0.899902in}{1.491868in}}{\pgfqpoint{0.907802in}{1.495141in}}{\pgfqpoint{0.913626in}{1.500965in}}%
\pgfpathcurveto{\pgfqpoint{0.919450in}{1.506789in}}{\pgfqpoint{0.922723in}{1.514689in}}{\pgfqpoint{0.922723in}{1.522925in}}%
\pgfpathcurveto{\pgfqpoint{0.922723in}{1.531161in}}{\pgfqpoint{0.919450in}{1.539061in}}{\pgfqpoint{0.913626in}{1.544885in}}%
\pgfpathcurveto{\pgfqpoint{0.907802in}{1.550709in}}{\pgfqpoint{0.899902in}{1.553981in}}{\pgfqpoint{0.891666in}{1.553981in}}%
\pgfpathcurveto{\pgfqpoint{0.883430in}{1.553981in}}{\pgfqpoint{0.875530in}{1.550709in}}{\pgfqpoint{0.869706in}{1.544885in}}%
\pgfpathcurveto{\pgfqpoint{0.863882in}{1.539061in}}{\pgfqpoint{0.860610in}{1.531161in}}{\pgfqpoint{0.860610in}{1.522925in}}%
\pgfpathcurveto{\pgfqpoint{0.860610in}{1.514689in}}{\pgfqpoint{0.863882in}{1.506789in}}{\pgfqpoint{0.869706in}{1.500965in}}%
\pgfpathcurveto{\pgfqpoint{0.875530in}{1.495141in}}{\pgfqpoint{0.883430in}{1.491868in}}{\pgfqpoint{0.891666in}{1.491868in}}%
\pgfpathclose%
\pgfusepath{stroke,fill}%
\end{pgfscope}%
\begin{pgfscope}%
\pgfpathrectangle{\pgfqpoint{0.100000in}{0.212622in}}{\pgfqpoint{3.696000in}{3.696000in}}%
\pgfusepath{clip}%
\pgfsetbuttcap%
\pgfsetroundjoin%
\definecolor{currentfill}{rgb}{0.121569,0.466667,0.705882}%
\pgfsetfillcolor{currentfill}%
\pgfsetfillopacity{0.585071}%
\pgfsetlinewidth{1.003750pt}%
\definecolor{currentstroke}{rgb}{0.121569,0.466667,0.705882}%
\pgfsetstrokecolor{currentstroke}%
\pgfsetstrokeopacity{0.585071}%
\pgfsetdash{}{0pt}%
\pgfpathmoveto{\pgfqpoint{0.891649in}{1.491807in}}%
\pgfpathcurveto{\pgfqpoint{0.899885in}{1.491807in}}{\pgfqpoint{0.907786in}{1.495079in}}{\pgfqpoint{0.913609in}{1.500903in}}%
\pgfpathcurveto{\pgfqpoint{0.919433in}{1.506727in}}{\pgfqpoint{0.922706in}{1.514627in}}{\pgfqpoint{0.922706in}{1.522863in}}%
\pgfpathcurveto{\pgfqpoint{0.922706in}{1.531100in}}{\pgfqpoint{0.919433in}{1.539000in}}{\pgfqpoint{0.913609in}{1.544824in}}%
\pgfpathcurveto{\pgfqpoint{0.907786in}{1.550647in}}{\pgfqpoint{0.899885in}{1.553920in}}{\pgfqpoint{0.891649in}{1.553920in}}%
\pgfpathcurveto{\pgfqpoint{0.883413in}{1.553920in}}{\pgfqpoint{0.875513in}{1.550647in}}{\pgfqpoint{0.869689in}{1.544824in}}%
\pgfpathcurveto{\pgfqpoint{0.863865in}{1.539000in}}{\pgfqpoint{0.860593in}{1.531100in}}{\pgfqpoint{0.860593in}{1.522863in}}%
\pgfpathcurveto{\pgfqpoint{0.860593in}{1.514627in}}{\pgfqpoint{0.863865in}{1.506727in}}{\pgfqpoint{0.869689in}{1.500903in}}%
\pgfpathcurveto{\pgfqpoint{0.875513in}{1.495079in}}{\pgfqpoint{0.883413in}{1.491807in}}{\pgfqpoint{0.891649in}{1.491807in}}%
\pgfpathclose%
\pgfusepath{stroke,fill}%
\end{pgfscope}%
\begin{pgfscope}%
\pgfpathrectangle{\pgfqpoint{0.100000in}{0.212622in}}{\pgfqpoint{3.696000in}{3.696000in}}%
\pgfusepath{clip}%
\pgfsetbuttcap%
\pgfsetroundjoin%
\definecolor{currentfill}{rgb}{0.121569,0.466667,0.705882}%
\pgfsetfillcolor{currentfill}%
\pgfsetfillopacity{0.585076}%
\pgfsetlinewidth{1.003750pt}%
\definecolor{currentstroke}{rgb}{0.121569,0.466667,0.705882}%
\pgfsetstrokecolor{currentstroke}%
\pgfsetstrokeopacity{0.585076}%
\pgfsetdash{}{0pt}%
\pgfpathmoveto{\pgfqpoint{1.000159in}{1.757641in}}%
\pgfpathcurveto{\pgfqpoint{1.008395in}{1.757641in}}{\pgfqpoint{1.016295in}{1.760913in}}{\pgfqpoint{1.022119in}{1.766737in}}%
\pgfpathcurveto{\pgfqpoint{1.027943in}{1.772561in}}{\pgfqpoint{1.031215in}{1.780461in}}{\pgfqpoint{1.031215in}{1.788697in}}%
\pgfpathcurveto{\pgfqpoint{1.031215in}{1.796934in}}{\pgfqpoint{1.027943in}{1.804834in}}{\pgfqpoint{1.022119in}{1.810658in}}%
\pgfpathcurveto{\pgfqpoint{1.016295in}{1.816482in}}{\pgfqpoint{1.008395in}{1.819754in}}{\pgfqpoint{1.000159in}{1.819754in}}%
\pgfpathcurveto{\pgfqpoint{0.991922in}{1.819754in}}{\pgfqpoint{0.984022in}{1.816482in}}{\pgfqpoint{0.978198in}{1.810658in}}%
\pgfpathcurveto{\pgfqpoint{0.972374in}{1.804834in}}{\pgfqpoint{0.969102in}{1.796934in}}{\pgfqpoint{0.969102in}{1.788697in}}%
\pgfpathcurveto{\pgfqpoint{0.969102in}{1.780461in}}{\pgfqpoint{0.972374in}{1.772561in}}{\pgfqpoint{0.978198in}{1.766737in}}%
\pgfpathcurveto{\pgfqpoint{0.984022in}{1.760913in}}{\pgfqpoint{0.991922in}{1.757641in}}{\pgfqpoint{1.000159in}{1.757641in}}%
\pgfpathclose%
\pgfusepath{stroke,fill}%
\end{pgfscope}%
\begin{pgfscope}%
\pgfpathrectangle{\pgfqpoint{0.100000in}{0.212622in}}{\pgfqpoint{3.696000in}{3.696000in}}%
\pgfusepath{clip}%
\pgfsetbuttcap%
\pgfsetroundjoin%
\definecolor{currentfill}{rgb}{0.121569,0.466667,0.705882}%
\pgfsetfillcolor{currentfill}%
\pgfsetfillopacity{0.585077}%
\pgfsetlinewidth{1.003750pt}%
\definecolor{currentstroke}{rgb}{0.121569,0.466667,0.705882}%
\pgfsetstrokecolor{currentstroke}%
\pgfsetstrokeopacity{0.585077}%
\pgfsetdash{}{0pt}%
\pgfpathmoveto{\pgfqpoint{0.891639in}{1.491772in}}%
\pgfpathcurveto{\pgfqpoint{0.899876in}{1.491772in}}{\pgfqpoint{0.907776in}{1.495044in}}{\pgfqpoint{0.913600in}{1.500868in}}%
\pgfpathcurveto{\pgfqpoint{0.919424in}{1.506692in}}{\pgfqpoint{0.922696in}{1.514592in}}{\pgfqpoint{0.922696in}{1.522829in}}%
\pgfpathcurveto{\pgfqpoint{0.922696in}{1.531065in}}{\pgfqpoint{0.919424in}{1.538965in}}{\pgfqpoint{0.913600in}{1.544789in}}%
\pgfpathcurveto{\pgfqpoint{0.907776in}{1.550613in}}{\pgfqpoint{0.899876in}{1.553885in}}{\pgfqpoint{0.891639in}{1.553885in}}%
\pgfpathcurveto{\pgfqpoint{0.883403in}{1.553885in}}{\pgfqpoint{0.875503in}{1.550613in}}{\pgfqpoint{0.869679in}{1.544789in}}%
\pgfpathcurveto{\pgfqpoint{0.863855in}{1.538965in}}{\pgfqpoint{0.860583in}{1.531065in}}{\pgfqpoint{0.860583in}{1.522829in}}%
\pgfpathcurveto{\pgfqpoint{0.860583in}{1.514592in}}{\pgfqpoint{0.863855in}{1.506692in}}{\pgfqpoint{0.869679in}{1.500868in}}%
\pgfpathcurveto{\pgfqpoint{0.875503in}{1.495044in}}{\pgfqpoint{0.883403in}{1.491772in}}{\pgfqpoint{0.891639in}{1.491772in}}%
\pgfpathclose%
\pgfusepath{stroke,fill}%
\end{pgfscope}%
\begin{pgfscope}%
\pgfpathrectangle{\pgfqpoint{0.100000in}{0.212622in}}{\pgfqpoint{3.696000in}{3.696000in}}%
\pgfusepath{clip}%
\pgfsetbuttcap%
\pgfsetroundjoin%
\definecolor{currentfill}{rgb}{0.121569,0.466667,0.705882}%
\pgfsetfillcolor{currentfill}%
\pgfsetfillopacity{0.585080}%
\pgfsetlinewidth{1.003750pt}%
\definecolor{currentstroke}{rgb}{0.121569,0.466667,0.705882}%
\pgfsetstrokecolor{currentstroke}%
\pgfsetstrokeopacity{0.585080}%
\pgfsetdash{}{0pt}%
\pgfpathmoveto{\pgfqpoint{0.891634in}{1.491753in}}%
\pgfpathcurveto{\pgfqpoint{0.899870in}{1.491753in}}{\pgfqpoint{0.907770in}{1.495025in}}{\pgfqpoint{0.913594in}{1.500849in}}%
\pgfpathcurveto{\pgfqpoint{0.919418in}{1.506673in}}{\pgfqpoint{0.922691in}{1.514573in}}{\pgfqpoint{0.922691in}{1.522809in}}%
\pgfpathcurveto{\pgfqpoint{0.922691in}{1.531045in}}{\pgfqpoint{0.919418in}{1.538946in}}{\pgfqpoint{0.913594in}{1.544769in}}%
\pgfpathcurveto{\pgfqpoint{0.907770in}{1.550593in}}{\pgfqpoint{0.899870in}{1.553866in}}{\pgfqpoint{0.891634in}{1.553866in}}%
\pgfpathcurveto{\pgfqpoint{0.883398in}{1.553866in}}{\pgfqpoint{0.875498in}{1.550593in}}{\pgfqpoint{0.869674in}{1.544769in}}%
\pgfpathcurveto{\pgfqpoint{0.863850in}{1.538946in}}{\pgfqpoint{0.860578in}{1.531045in}}{\pgfqpoint{0.860578in}{1.522809in}}%
\pgfpathcurveto{\pgfqpoint{0.860578in}{1.514573in}}{\pgfqpoint{0.863850in}{1.506673in}}{\pgfqpoint{0.869674in}{1.500849in}}%
\pgfpathcurveto{\pgfqpoint{0.875498in}{1.495025in}}{\pgfqpoint{0.883398in}{1.491753in}}{\pgfqpoint{0.891634in}{1.491753in}}%
\pgfpathclose%
\pgfusepath{stroke,fill}%
\end{pgfscope}%
\begin{pgfscope}%
\pgfpathrectangle{\pgfqpoint{0.100000in}{0.212622in}}{\pgfqpoint{3.696000in}{3.696000in}}%
\pgfusepath{clip}%
\pgfsetbuttcap%
\pgfsetroundjoin%
\definecolor{currentfill}{rgb}{0.121569,0.466667,0.705882}%
\pgfsetfillcolor{currentfill}%
\pgfsetfillopacity{0.585081}%
\pgfsetlinewidth{1.003750pt}%
\definecolor{currentstroke}{rgb}{0.121569,0.466667,0.705882}%
\pgfsetstrokecolor{currentstroke}%
\pgfsetstrokeopacity{0.585081}%
\pgfsetdash{}{0pt}%
\pgfpathmoveto{\pgfqpoint{0.891631in}{1.491742in}}%
\pgfpathcurveto{\pgfqpoint{0.899867in}{1.491742in}}{\pgfqpoint{0.907767in}{1.495014in}}{\pgfqpoint{0.913591in}{1.500838in}}%
\pgfpathcurveto{\pgfqpoint{0.919415in}{1.506662in}}{\pgfqpoint{0.922688in}{1.514562in}}{\pgfqpoint{0.922688in}{1.522798in}}%
\pgfpathcurveto{\pgfqpoint{0.922688in}{1.531035in}}{\pgfqpoint{0.919415in}{1.538935in}}{\pgfqpoint{0.913591in}{1.544759in}}%
\pgfpathcurveto{\pgfqpoint{0.907767in}{1.550583in}}{\pgfqpoint{0.899867in}{1.553855in}}{\pgfqpoint{0.891631in}{1.553855in}}%
\pgfpathcurveto{\pgfqpoint{0.883395in}{1.553855in}}{\pgfqpoint{0.875495in}{1.550583in}}{\pgfqpoint{0.869671in}{1.544759in}}%
\pgfpathcurveto{\pgfqpoint{0.863847in}{1.538935in}}{\pgfqpoint{0.860575in}{1.531035in}}{\pgfqpoint{0.860575in}{1.522798in}}%
\pgfpathcurveto{\pgfqpoint{0.860575in}{1.514562in}}{\pgfqpoint{0.863847in}{1.506662in}}{\pgfqpoint{0.869671in}{1.500838in}}%
\pgfpathcurveto{\pgfqpoint{0.875495in}{1.495014in}}{\pgfqpoint{0.883395in}{1.491742in}}{\pgfqpoint{0.891631in}{1.491742in}}%
\pgfpathclose%
\pgfusepath{stroke,fill}%
\end{pgfscope}%
\begin{pgfscope}%
\pgfpathrectangle{\pgfqpoint{0.100000in}{0.212622in}}{\pgfqpoint{3.696000in}{3.696000in}}%
\pgfusepath{clip}%
\pgfsetbuttcap%
\pgfsetroundjoin%
\definecolor{currentfill}{rgb}{0.121569,0.466667,0.705882}%
\pgfsetfillcolor{currentfill}%
\pgfsetfillopacity{0.585174}%
\pgfsetlinewidth{1.003750pt}%
\definecolor{currentstroke}{rgb}{0.121569,0.466667,0.705882}%
\pgfsetstrokecolor{currentstroke}%
\pgfsetstrokeopacity{0.585174}%
\pgfsetdash{}{0pt}%
\pgfpathmoveto{\pgfqpoint{0.891477in}{1.491165in}}%
\pgfpathcurveto{\pgfqpoint{0.899714in}{1.491165in}}{\pgfqpoint{0.907614in}{1.494437in}}{\pgfqpoint{0.913438in}{1.500261in}}%
\pgfpathcurveto{\pgfqpoint{0.919262in}{1.506085in}}{\pgfqpoint{0.922534in}{1.513985in}}{\pgfqpoint{0.922534in}{1.522222in}}%
\pgfpathcurveto{\pgfqpoint{0.922534in}{1.530458in}}{\pgfqpoint{0.919262in}{1.538358in}}{\pgfqpoint{0.913438in}{1.544182in}}%
\pgfpathcurveto{\pgfqpoint{0.907614in}{1.550006in}}{\pgfqpoint{0.899714in}{1.553278in}}{\pgfqpoint{0.891477in}{1.553278in}}%
\pgfpathcurveto{\pgfqpoint{0.883241in}{1.553278in}}{\pgfqpoint{0.875341in}{1.550006in}}{\pgfqpoint{0.869517in}{1.544182in}}%
\pgfpathcurveto{\pgfqpoint{0.863693in}{1.538358in}}{\pgfqpoint{0.860421in}{1.530458in}}{\pgfqpoint{0.860421in}{1.522222in}}%
\pgfpathcurveto{\pgfqpoint{0.860421in}{1.513985in}}{\pgfqpoint{0.863693in}{1.506085in}}{\pgfqpoint{0.869517in}{1.500261in}}%
\pgfpathcurveto{\pgfqpoint{0.875341in}{1.494437in}}{\pgfqpoint{0.883241in}{1.491165in}}{\pgfqpoint{0.891477in}{1.491165in}}%
\pgfpathclose%
\pgfusepath{stroke,fill}%
\end{pgfscope}%
\begin{pgfscope}%
\pgfpathrectangle{\pgfqpoint{0.100000in}{0.212622in}}{\pgfqpoint{3.696000in}{3.696000in}}%
\pgfusepath{clip}%
\pgfsetbuttcap%
\pgfsetroundjoin%
\definecolor{currentfill}{rgb}{0.121569,0.466667,0.705882}%
\pgfsetfillcolor{currentfill}%
\pgfsetfillopacity{0.585225}%
\pgfsetlinewidth{1.003750pt}%
\definecolor{currentstroke}{rgb}{0.121569,0.466667,0.705882}%
\pgfsetstrokecolor{currentstroke}%
\pgfsetstrokeopacity{0.585225}%
\pgfsetdash{}{0pt}%
\pgfpathmoveto{\pgfqpoint{0.891391in}{1.490851in}}%
\pgfpathcurveto{\pgfqpoint{0.899627in}{1.490851in}}{\pgfqpoint{0.907527in}{1.494123in}}{\pgfqpoint{0.913351in}{1.499947in}}%
\pgfpathcurveto{\pgfqpoint{0.919175in}{1.505771in}}{\pgfqpoint{0.922447in}{1.513671in}}{\pgfqpoint{0.922447in}{1.521907in}}%
\pgfpathcurveto{\pgfqpoint{0.922447in}{1.530144in}}{\pgfqpoint{0.919175in}{1.538044in}}{\pgfqpoint{0.913351in}{1.543868in}}%
\pgfpathcurveto{\pgfqpoint{0.907527in}{1.549691in}}{\pgfqpoint{0.899627in}{1.552964in}}{\pgfqpoint{0.891391in}{1.552964in}}%
\pgfpathcurveto{\pgfqpoint{0.883154in}{1.552964in}}{\pgfqpoint{0.875254in}{1.549691in}}{\pgfqpoint{0.869431in}{1.543868in}}%
\pgfpathcurveto{\pgfqpoint{0.863607in}{1.538044in}}{\pgfqpoint{0.860334in}{1.530144in}}{\pgfqpoint{0.860334in}{1.521907in}}%
\pgfpathcurveto{\pgfqpoint{0.860334in}{1.513671in}}{\pgfqpoint{0.863607in}{1.505771in}}{\pgfqpoint{0.869431in}{1.499947in}}%
\pgfpathcurveto{\pgfqpoint{0.875254in}{1.494123in}}{\pgfqpoint{0.883154in}{1.490851in}}{\pgfqpoint{0.891391in}{1.490851in}}%
\pgfpathclose%
\pgfusepath{stroke,fill}%
\end{pgfscope}%
\begin{pgfscope}%
\pgfpathrectangle{\pgfqpoint{0.100000in}{0.212622in}}{\pgfqpoint{3.696000in}{3.696000in}}%
\pgfusepath{clip}%
\pgfsetbuttcap%
\pgfsetroundjoin%
\definecolor{currentfill}{rgb}{0.121569,0.466667,0.705882}%
\pgfsetfillcolor{currentfill}%
\pgfsetfillopacity{0.585349}%
\pgfsetlinewidth{1.003750pt}%
\definecolor{currentstroke}{rgb}{0.121569,0.466667,0.705882}%
\pgfsetstrokecolor{currentstroke}%
\pgfsetstrokeopacity{0.585349}%
\pgfsetdash{}{0pt}%
\pgfpathmoveto{\pgfqpoint{0.891179in}{1.490091in}}%
\pgfpathcurveto{\pgfqpoint{0.899415in}{1.490091in}}{\pgfqpoint{0.907315in}{1.493364in}}{\pgfqpoint{0.913139in}{1.499188in}}%
\pgfpathcurveto{\pgfqpoint{0.918963in}{1.505012in}}{\pgfqpoint{0.922235in}{1.512912in}}{\pgfqpoint{0.922235in}{1.521148in}}%
\pgfpathcurveto{\pgfqpoint{0.922235in}{1.529384in}}{\pgfqpoint{0.918963in}{1.537284in}}{\pgfqpoint{0.913139in}{1.543108in}}%
\pgfpathcurveto{\pgfqpoint{0.907315in}{1.548932in}}{\pgfqpoint{0.899415in}{1.552204in}}{\pgfqpoint{0.891179in}{1.552204in}}%
\pgfpathcurveto{\pgfqpoint{0.882942in}{1.552204in}}{\pgfqpoint{0.875042in}{1.548932in}}{\pgfqpoint{0.869218in}{1.543108in}}%
\pgfpathcurveto{\pgfqpoint{0.863394in}{1.537284in}}{\pgfqpoint{0.860122in}{1.529384in}}{\pgfqpoint{0.860122in}{1.521148in}}%
\pgfpathcurveto{\pgfqpoint{0.860122in}{1.512912in}}{\pgfqpoint{0.863394in}{1.505012in}}{\pgfqpoint{0.869218in}{1.499188in}}%
\pgfpathcurveto{\pgfqpoint{0.875042in}{1.493364in}}{\pgfqpoint{0.882942in}{1.490091in}}{\pgfqpoint{0.891179in}{1.490091in}}%
\pgfpathclose%
\pgfusepath{stroke,fill}%
\end{pgfscope}%
\begin{pgfscope}%
\pgfpathrectangle{\pgfqpoint{0.100000in}{0.212622in}}{\pgfqpoint{3.696000in}{3.696000in}}%
\pgfusepath{clip}%
\pgfsetbuttcap%
\pgfsetroundjoin%
\definecolor{currentfill}{rgb}{0.121569,0.466667,0.705882}%
\pgfsetfillcolor{currentfill}%
\pgfsetfillopacity{0.585417}%
\pgfsetlinewidth{1.003750pt}%
\definecolor{currentstroke}{rgb}{0.121569,0.466667,0.705882}%
\pgfsetstrokecolor{currentstroke}%
\pgfsetstrokeopacity{0.585417}%
\pgfsetdash{}{0pt}%
\pgfpathmoveto{\pgfqpoint{0.891058in}{1.489674in}}%
\pgfpathcurveto{\pgfqpoint{0.899295in}{1.489674in}}{\pgfqpoint{0.907195in}{1.492946in}}{\pgfqpoint{0.913019in}{1.498770in}}%
\pgfpathcurveto{\pgfqpoint{0.918843in}{1.504594in}}{\pgfqpoint{0.922115in}{1.512494in}}{\pgfqpoint{0.922115in}{1.520731in}}%
\pgfpathcurveto{\pgfqpoint{0.922115in}{1.528967in}}{\pgfqpoint{0.918843in}{1.536867in}}{\pgfqpoint{0.913019in}{1.542691in}}%
\pgfpathcurveto{\pgfqpoint{0.907195in}{1.548515in}}{\pgfqpoint{0.899295in}{1.551787in}}{\pgfqpoint{0.891058in}{1.551787in}}%
\pgfpathcurveto{\pgfqpoint{0.882822in}{1.551787in}}{\pgfqpoint{0.874922in}{1.548515in}}{\pgfqpoint{0.869098in}{1.542691in}}%
\pgfpathcurveto{\pgfqpoint{0.863274in}{1.536867in}}{\pgfqpoint{0.860002in}{1.528967in}}{\pgfqpoint{0.860002in}{1.520731in}}%
\pgfpathcurveto{\pgfqpoint{0.860002in}{1.512494in}}{\pgfqpoint{0.863274in}{1.504594in}}{\pgfqpoint{0.869098in}{1.498770in}}%
\pgfpathcurveto{\pgfqpoint{0.874922in}{1.492946in}}{\pgfqpoint{0.882822in}{1.489674in}}{\pgfqpoint{0.891058in}{1.489674in}}%
\pgfpathclose%
\pgfusepath{stroke,fill}%
\end{pgfscope}%
\begin{pgfscope}%
\pgfpathrectangle{\pgfqpoint{0.100000in}{0.212622in}}{\pgfqpoint{3.696000in}{3.696000in}}%
\pgfusepath{clip}%
\pgfsetbuttcap%
\pgfsetroundjoin%
\definecolor{currentfill}{rgb}{0.121569,0.466667,0.705882}%
\pgfsetfillcolor{currentfill}%
\pgfsetfillopacity{0.585454}%
\pgfsetlinewidth{1.003750pt}%
\definecolor{currentstroke}{rgb}{0.121569,0.466667,0.705882}%
\pgfsetstrokecolor{currentstroke}%
\pgfsetstrokeopacity{0.585454}%
\pgfsetdash{}{0pt}%
\pgfpathmoveto{\pgfqpoint{0.890994in}{1.489444in}}%
\pgfpathcurveto{\pgfqpoint{0.899230in}{1.489444in}}{\pgfqpoint{0.907130in}{1.492717in}}{\pgfqpoint{0.912954in}{1.498541in}}%
\pgfpathcurveto{\pgfqpoint{0.918778in}{1.504365in}}{\pgfqpoint{0.922050in}{1.512265in}}{\pgfqpoint{0.922050in}{1.520501in}}%
\pgfpathcurveto{\pgfqpoint{0.922050in}{1.528737in}}{\pgfqpoint{0.918778in}{1.536637in}}{\pgfqpoint{0.912954in}{1.542461in}}%
\pgfpathcurveto{\pgfqpoint{0.907130in}{1.548285in}}{\pgfqpoint{0.899230in}{1.551557in}}{\pgfqpoint{0.890994in}{1.551557in}}%
\pgfpathcurveto{\pgfqpoint{0.882757in}{1.551557in}}{\pgfqpoint{0.874857in}{1.548285in}}{\pgfqpoint{0.869033in}{1.542461in}}%
\pgfpathcurveto{\pgfqpoint{0.863210in}{1.536637in}}{\pgfqpoint{0.859937in}{1.528737in}}{\pgfqpoint{0.859937in}{1.520501in}}%
\pgfpathcurveto{\pgfqpoint{0.859937in}{1.512265in}}{\pgfqpoint{0.863210in}{1.504365in}}{\pgfqpoint{0.869033in}{1.498541in}}%
\pgfpathcurveto{\pgfqpoint{0.874857in}{1.492717in}}{\pgfqpoint{0.882757in}{1.489444in}}{\pgfqpoint{0.890994in}{1.489444in}}%
\pgfpathclose%
\pgfusepath{stroke,fill}%
\end{pgfscope}%
\begin{pgfscope}%
\pgfpathrectangle{\pgfqpoint{0.100000in}{0.212622in}}{\pgfqpoint{3.696000in}{3.696000in}}%
\pgfusepath{clip}%
\pgfsetbuttcap%
\pgfsetroundjoin%
\definecolor{currentfill}{rgb}{0.121569,0.466667,0.705882}%
\pgfsetfillcolor{currentfill}%
\pgfsetfillopacity{0.585474}%
\pgfsetlinewidth{1.003750pt}%
\definecolor{currentstroke}{rgb}{0.121569,0.466667,0.705882}%
\pgfsetstrokecolor{currentstroke}%
\pgfsetstrokeopacity{0.585474}%
\pgfsetdash{}{0pt}%
\pgfpathmoveto{\pgfqpoint{0.890959in}{1.489317in}}%
\pgfpathcurveto{\pgfqpoint{0.899195in}{1.489317in}}{\pgfqpoint{0.907095in}{1.492589in}}{\pgfqpoint{0.912919in}{1.498413in}}%
\pgfpathcurveto{\pgfqpoint{0.918743in}{1.504237in}}{\pgfqpoint{0.922015in}{1.512137in}}{\pgfqpoint{0.922015in}{1.520373in}}%
\pgfpathcurveto{\pgfqpoint{0.922015in}{1.528609in}}{\pgfqpoint{0.918743in}{1.536509in}}{\pgfqpoint{0.912919in}{1.542333in}}%
\pgfpathcurveto{\pgfqpoint{0.907095in}{1.548157in}}{\pgfqpoint{0.899195in}{1.551430in}}{\pgfqpoint{0.890959in}{1.551430in}}%
\pgfpathcurveto{\pgfqpoint{0.882722in}{1.551430in}}{\pgfqpoint{0.874822in}{1.548157in}}{\pgfqpoint{0.868998in}{1.542333in}}%
\pgfpathcurveto{\pgfqpoint{0.863174in}{1.536509in}}{\pgfqpoint{0.859902in}{1.528609in}}{\pgfqpoint{0.859902in}{1.520373in}}%
\pgfpathcurveto{\pgfqpoint{0.859902in}{1.512137in}}{\pgfqpoint{0.863174in}{1.504237in}}{\pgfqpoint{0.868998in}{1.498413in}}%
\pgfpathcurveto{\pgfqpoint{0.874822in}{1.492589in}}{\pgfqpoint{0.882722in}{1.489317in}}{\pgfqpoint{0.890959in}{1.489317in}}%
\pgfpathclose%
\pgfusepath{stroke,fill}%
\end{pgfscope}%
\begin{pgfscope}%
\pgfpathrectangle{\pgfqpoint{0.100000in}{0.212622in}}{\pgfqpoint{3.696000in}{3.696000in}}%
\pgfusepath{clip}%
\pgfsetbuttcap%
\pgfsetroundjoin%
\definecolor{currentfill}{rgb}{0.121569,0.466667,0.705882}%
\pgfsetfillcolor{currentfill}%
\pgfsetfillopacity{0.585486}%
\pgfsetlinewidth{1.003750pt}%
\definecolor{currentstroke}{rgb}{0.121569,0.466667,0.705882}%
\pgfsetstrokecolor{currentstroke}%
\pgfsetstrokeopacity{0.585486}%
\pgfsetdash{}{0pt}%
\pgfpathmoveto{\pgfqpoint{0.890939in}{1.489246in}}%
\pgfpathcurveto{\pgfqpoint{0.899175in}{1.489246in}}{\pgfqpoint{0.907075in}{1.492519in}}{\pgfqpoint{0.912899in}{1.498343in}}%
\pgfpathcurveto{\pgfqpoint{0.918723in}{1.504166in}}{\pgfqpoint{0.921996in}{1.512067in}}{\pgfqpoint{0.921996in}{1.520303in}}%
\pgfpathcurveto{\pgfqpoint{0.921996in}{1.528539in}}{\pgfqpoint{0.918723in}{1.536439in}}{\pgfqpoint{0.912899in}{1.542263in}}%
\pgfpathcurveto{\pgfqpoint{0.907075in}{1.548087in}}{\pgfqpoint{0.899175in}{1.551359in}}{\pgfqpoint{0.890939in}{1.551359in}}%
\pgfpathcurveto{\pgfqpoint{0.882703in}{1.551359in}}{\pgfqpoint{0.874803in}{1.548087in}}{\pgfqpoint{0.868979in}{1.542263in}}%
\pgfpathcurveto{\pgfqpoint{0.863155in}{1.536439in}}{\pgfqpoint{0.859883in}{1.528539in}}{\pgfqpoint{0.859883in}{1.520303in}}%
\pgfpathcurveto{\pgfqpoint{0.859883in}{1.512067in}}{\pgfqpoint{0.863155in}{1.504166in}}{\pgfqpoint{0.868979in}{1.498343in}}%
\pgfpathcurveto{\pgfqpoint{0.874803in}{1.492519in}}{\pgfqpoint{0.882703in}{1.489246in}}{\pgfqpoint{0.890939in}{1.489246in}}%
\pgfpathclose%
\pgfusepath{stroke,fill}%
\end{pgfscope}%
\begin{pgfscope}%
\pgfpathrectangle{\pgfqpoint{0.100000in}{0.212622in}}{\pgfqpoint{3.696000in}{3.696000in}}%
\pgfusepath{clip}%
\pgfsetbuttcap%
\pgfsetroundjoin%
\definecolor{currentfill}{rgb}{0.121569,0.466667,0.705882}%
\pgfsetfillcolor{currentfill}%
\pgfsetfillopacity{0.585492}%
\pgfsetlinewidth{1.003750pt}%
\definecolor{currentstroke}{rgb}{0.121569,0.466667,0.705882}%
\pgfsetstrokecolor{currentstroke}%
\pgfsetstrokeopacity{0.585492}%
\pgfsetdash{}{0pt}%
\pgfpathmoveto{\pgfqpoint{0.890929in}{1.489208in}}%
\pgfpathcurveto{\pgfqpoint{0.899165in}{1.489208in}}{\pgfqpoint{0.907065in}{1.492480in}}{\pgfqpoint{0.912889in}{1.498304in}}%
\pgfpathcurveto{\pgfqpoint{0.918713in}{1.504128in}}{\pgfqpoint{0.921985in}{1.512028in}}{\pgfqpoint{0.921985in}{1.520264in}}%
\pgfpathcurveto{\pgfqpoint{0.921985in}{1.528500in}}{\pgfqpoint{0.918713in}{1.536400in}}{\pgfqpoint{0.912889in}{1.542224in}}%
\pgfpathcurveto{\pgfqpoint{0.907065in}{1.548048in}}{\pgfqpoint{0.899165in}{1.551321in}}{\pgfqpoint{0.890929in}{1.551321in}}%
\pgfpathcurveto{\pgfqpoint{0.882693in}{1.551321in}}{\pgfqpoint{0.874793in}{1.548048in}}{\pgfqpoint{0.868969in}{1.542224in}}%
\pgfpathcurveto{\pgfqpoint{0.863145in}{1.536400in}}{\pgfqpoint{0.859872in}{1.528500in}}{\pgfqpoint{0.859872in}{1.520264in}}%
\pgfpathcurveto{\pgfqpoint{0.859872in}{1.512028in}}{\pgfqpoint{0.863145in}{1.504128in}}{\pgfqpoint{0.868969in}{1.498304in}}%
\pgfpathcurveto{\pgfqpoint{0.874793in}{1.492480in}}{\pgfqpoint{0.882693in}{1.489208in}}{\pgfqpoint{0.890929in}{1.489208in}}%
\pgfpathclose%
\pgfusepath{stroke,fill}%
\end{pgfscope}%
\begin{pgfscope}%
\pgfpathrectangle{\pgfqpoint{0.100000in}{0.212622in}}{\pgfqpoint{3.696000in}{3.696000in}}%
\pgfusepath{clip}%
\pgfsetbuttcap%
\pgfsetroundjoin%
\definecolor{currentfill}{rgb}{0.121569,0.466667,0.705882}%
\pgfsetfillcolor{currentfill}%
\pgfsetfillopacity{0.585495}%
\pgfsetlinewidth{1.003750pt}%
\definecolor{currentstroke}{rgb}{0.121569,0.466667,0.705882}%
\pgfsetstrokecolor{currentstroke}%
\pgfsetstrokeopacity{0.585495}%
\pgfsetdash{}{0pt}%
\pgfpathmoveto{\pgfqpoint{0.890923in}{1.489186in}}%
\pgfpathcurveto{\pgfqpoint{0.899159in}{1.489186in}}{\pgfqpoint{0.907059in}{1.492458in}}{\pgfqpoint{0.912883in}{1.498282in}}%
\pgfpathcurveto{\pgfqpoint{0.918707in}{1.504106in}}{\pgfqpoint{0.921980in}{1.512006in}}{\pgfqpoint{0.921980in}{1.520243in}}%
\pgfpathcurveto{\pgfqpoint{0.921980in}{1.528479in}}{\pgfqpoint{0.918707in}{1.536379in}}{\pgfqpoint{0.912883in}{1.542203in}}%
\pgfpathcurveto{\pgfqpoint{0.907059in}{1.548027in}}{\pgfqpoint{0.899159in}{1.551299in}}{\pgfqpoint{0.890923in}{1.551299in}}%
\pgfpathcurveto{\pgfqpoint{0.882687in}{1.551299in}}{\pgfqpoint{0.874787in}{1.548027in}}{\pgfqpoint{0.868963in}{1.542203in}}%
\pgfpathcurveto{\pgfqpoint{0.863139in}{1.536379in}}{\pgfqpoint{0.859867in}{1.528479in}}{\pgfqpoint{0.859867in}{1.520243in}}%
\pgfpathcurveto{\pgfqpoint{0.859867in}{1.512006in}}{\pgfqpoint{0.863139in}{1.504106in}}{\pgfqpoint{0.868963in}{1.498282in}}%
\pgfpathcurveto{\pgfqpoint{0.874787in}{1.492458in}}{\pgfqpoint{0.882687in}{1.489186in}}{\pgfqpoint{0.890923in}{1.489186in}}%
\pgfpathclose%
\pgfusepath{stroke,fill}%
\end{pgfscope}%
\begin{pgfscope}%
\pgfpathrectangle{\pgfqpoint{0.100000in}{0.212622in}}{\pgfqpoint{3.696000in}{3.696000in}}%
\pgfusepath{clip}%
\pgfsetbuttcap%
\pgfsetroundjoin%
\definecolor{currentfill}{rgb}{0.121569,0.466667,0.705882}%
\pgfsetfillcolor{currentfill}%
\pgfsetfillopacity{0.585497}%
\pgfsetlinewidth{1.003750pt}%
\definecolor{currentstroke}{rgb}{0.121569,0.466667,0.705882}%
\pgfsetstrokecolor{currentstroke}%
\pgfsetstrokeopacity{0.585497}%
\pgfsetdash{}{0pt}%
\pgfpathmoveto{\pgfqpoint{0.890920in}{1.489174in}}%
\pgfpathcurveto{\pgfqpoint{0.899156in}{1.489174in}}{\pgfqpoint{0.907056in}{1.492447in}}{\pgfqpoint{0.912880in}{1.498271in}}%
\pgfpathcurveto{\pgfqpoint{0.918704in}{1.504095in}}{\pgfqpoint{0.921976in}{1.511995in}}{\pgfqpoint{0.921976in}{1.520231in}}%
\pgfpathcurveto{\pgfqpoint{0.921976in}{1.528467in}}{\pgfqpoint{0.918704in}{1.536367in}}{\pgfqpoint{0.912880in}{1.542191in}}%
\pgfpathcurveto{\pgfqpoint{0.907056in}{1.548015in}}{\pgfqpoint{0.899156in}{1.551287in}}{\pgfqpoint{0.890920in}{1.551287in}}%
\pgfpathcurveto{\pgfqpoint{0.882684in}{1.551287in}}{\pgfqpoint{0.874784in}{1.548015in}}{\pgfqpoint{0.868960in}{1.542191in}}%
\pgfpathcurveto{\pgfqpoint{0.863136in}{1.536367in}}{\pgfqpoint{0.859863in}{1.528467in}}{\pgfqpoint{0.859863in}{1.520231in}}%
\pgfpathcurveto{\pgfqpoint{0.859863in}{1.511995in}}{\pgfqpoint{0.863136in}{1.504095in}}{\pgfqpoint{0.868960in}{1.498271in}}%
\pgfpathcurveto{\pgfqpoint{0.874784in}{1.492447in}}{\pgfqpoint{0.882684in}{1.489174in}}{\pgfqpoint{0.890920in}{1.489174in}}%
\pgfpathclose%
\pgfusepath{stroke,fill}%
\end{pgfscope}%
\begin{pgfscope}%
\pgfpathrectangle{\pgfqpoint{0.100000in}{0.212622in}}{\pgfqpoint{3.696000in}{3.696000in}}%
\pgfusepath{clip}%
\pgfsetbuttcap%
\pgfsetroundjoin%
\definecolor{currentfill}{rgb}{0.121569,0.466667,0.705882}%
\pgfsetfillcolor{currentfill}%
\pgfsetfillopacity{0.585498}%
\pgfsetlinewidth{1.003750pt}%
\definecolor{currentstroke}{rgb}{0.121569,0.466667,0.705882}%
\pgfsetstrokecolor{currentstroke}%
\pgfsetstrokeopacity{0.585498}%
\pgfsetdash{}{0pt}%
\pgfpathmoveto{\pgfqpoint{0.890918in}{1.489168in}}%
\pgfpathcurveto{\pgfqpoint{0.899155in}{1.489168in}}{\pgfqpoint{0.907055in}{1.492440in}}{\pgfqpoint{0.912878in}{1.498264in}}%
\pgfpathcurveto{\pgfqpoint{0.918702in}{1.504088in}}{\pgfqpoint{0.921975in}{1.511988in}}{\pgfqpoint{0.921975in}{1.520225in}}%
\pgfpathcurveto{\pgfqpoint{0.921975in}{1.528461in}}{\pgfqpoint{0.918702in}{1.536361in}}{\pgfqpoint{0.912878in}{1.542185in}}%
\pgfpathcurveto{\pgfqpoint{0.907055in}{1.548009in}}{\pgfqpoint{0.899155in}{1.551281in}}{\pgfqpoint{0.890918in}{1.551281in}}%
\pgfpathcurveto{\pgfqpoint{0.882682in}{1.551281in}}{\pgfqpoint{0.874782in}{1.548009in}}{\pgfqpoint{0.868958in}{1.542185in}}%
\pgfpathcurveto{\pgfqpoint{0.863134in}{1.536361in}}{\pgfqpoint{0.859862in}{1.528461in}}{\pgfqpoint{0.859862in}{1.520225in}}%
\pgfpathcurveto{\pgfqpoint{0.859862in}{1.511988in}}{\pgfqpoint{0.863134in}{1.504088in}}{\pgfqpoint{0.868958in}{1.498264in}}%
\pgfpathcurveto{\pgfqpoint{0.874782in}{1.492440in}}{\pgfqpoint{0.882682in}{1.489168in}}{\pgfqpoint{0.890918in}{1.489168in}}%
\pgfpathclose%
\pgfusepath{stroke,fill}%
\end{pgfscope}%
\begin{pgfscope}%
\pgfpathrectangle{\pgfqpoint{0.100000in}{0.212622in}}{\pgfqpoint{3.696000in}{3.696000in}}%
\pgfusepath{clip}%
\pgfsetbuttcap%
\pgfsetroundjoin%
\definecolor{currentfill}{rgb}{0.121569,0.466667,0.705882}%
\pgfsetfillcolor{currentfill}%
\pgfsetfillopacity{0.585498}%
\pgfsetlinewidth{1.003750pt}%
\definecolor{currentstroke}{rgb}{0.121569,0.466667,0.705882}%
\pgfsetstrokecolor{currentstroke}%
\pgfsetstrokeopacity{0.585498}%
\pgfsetdash{}{0pt}%
\pgfpathmoveto{\pgfqpoint{0.890917in}{1.489165in}}%
\pgfpathcurveto{\pgfqpoint{0.899154in}{1.489165in}}{\pgfqpoint{0.907054in}{1.492437in}}{\pgfqpoint{0.912878in}{1.498261in}}%
\pgfpathcurveto{\pgfqpoint{0.918701in}{1.504085in}}{\pgfqpoint{0.921974in}{1.511985in}}{\pgfqpoint{0.921974in}{1.520221in}}%
\pgfpathcurveto{\pgfqpoint{0.921974in}{1.528457in}}{\pgfqpoint{0.918701in}{1.536357in}}{\pgfqpoint{0.912878in}{1.542181in}}%
\pgfpathcurveto{\pgfqpoint{0.907054in}{1.548005in}}{\pgfqpoint{0.899154in}{1.551278in}}{\pgfqpoint{0.890917in}{1.551278in}}%
\pgfpathcurveto{\pgfqpoint{0.882681in}{1.551278in}}{\pgfqpoint{0.874781in}{1.548005in}}{\pgfqpoint{0.868957in}{1.542181in}}%
\pgfpathcurveto{\pgfqpoint{0.863133in}{1.536357in}}{\pgfqpoint{0.859861in}{1.528457in}}{\pgfqpoint{0.859861in}{1.520221in}}%
\pgfpathcurveto{\pgfqpoint{0.859861in}{1.511985in}}{\pgfqpoint{0.863133in}{1.504085in}}{\pgfqpoint{0.868957in}{1.498261in}}%
\pgfpathcurveto{\pgfqpoint{0.874781in}{1.492437in}}{\pgfqpoint{0.882681in}{1.489165in}}{\pgfqpoint{0.890917in}{1.489165in}}%
\pgfpathclose%
\pgfusepath{stroke,fill}%
\end{pgfscope}%
\begin{pgfscope}%
\pgfpathrectangle{\pgfqpoint{0.100000in}{0.212622in}}{\pgfqpoint{3.696000in}{3.696000in}}%
\pgfusepath{clip}%
\pgfsetbuttcap%
\pgfsetroundjoin%
\definecolor{currentfill}{rgb}{0.121569,0.466667,0.705882}%
\pgfsetfillcolor{currentfill}%
\pgfsetfillopacity{0.585499}%
\pgfsetlinewidth{1.003750pt}%
\definecolor{currentstroke}{rgb}{0.121569,0.466667,0.705882}%
\pgfsetstrokecolor{currentstroke}%
\pgfsetstrokeopacity{0.585499}%
\pgfsetdash{}{0pt}%
\pgfpathmoveto{\pgfqpoint{0.890917in}{1.489163in}}%
\pgfpathcurveto{\pgfqpoint{0.899153in}{1.489163in}}{\pgfqpoint{0.907053in}{1.492435in}}{\pgfqpoint{0.912877in}{1.498259in}}%
\pgfpathcurveto{\pgfqpoint{0.918701in}{1.504083in}}{\pgfqpoint{0.921973in}{1.511983in}}{\pgfqpoint{0.921973in}{1.520219in}}%
\pgfpathcurveto{\pgfqpoint{0.921973in}{1.528455in}}{\pgfqpoint{0.918701in}{1.536355in}}{\pgfqpoint{0.912877in}{1.542179in}}%
\pgfpathcurveto{\pgfqpoint{0.907053in}{1.548003in}}{\pgfqpoint{0.899153in}{1.551276in}}{\pgfqpoint{0.890917in}{1.551276in}}%
\pgfpathcurveto{\pgfqpoint{0.882680in}{1.551276in}}{\pgfqpoint{0.874780in}{1.548003in}}{\pgfqpoint{0.868956in}{1.542179in}}%
\pgfpathcurveto{\pgfqpoint{0.863133in}{1.536355in}}{\pgfqpoint{0.859860in}{1.528455in}}{\pgfqpoint{0.859860in}{1.520219in}}%
\pgfpathcurveto{\pgfqpoint{0.859860in}{1.511983in}}{\pgfqpoint{0.863133in}{1.504083in}}{\pgfqpoint{0.868956in}{1.498259in}}%
\pgfpathcurveto{\pgfqpoint{0.874780in}{1.492435in}}{\pgfqpoint{0.882680in}{1.489163in}}{\pgfqpoint{0.890917in}{1.489163in}}%
\pgfpathclose%
\pgfusepath{stroke,fill}%
\end{pgfscope}%
\begin{pgfscope}%
\pgfpathrectangle{\pgfqpoint{0.100000in}{0.212622in}}{\pgfqpoint{3.696000in}{3.696000in}}%
\pgfusepath{clip}%
\pgfsetbuttcap%
\pgfsetroundjoin%
\definecolor{currentfill}{rgb}{0.121569,0.466667,0.705882}%
\pgfsetfillcolor{currentfill}%
\pgfsetfillopacity{0.585499}%
\pgfsetlinewidth{1.003750pt}%
\definecolor{currentstroke}{rgb}{0.121569,0.466667,0.705882}%
\pgfsetstrokecolor{currentstroke}%
\pgfsetstrokeopacity{0.585499}%
\pgfsetdash{}{0pt}%
\pgfpathmoveto{\pgfqpoint{0.890916in}{1.489162in}}%
\pgfpathcurveto{\pgfqpoint{0.899153in}{1.489162in}}{\pgfqpoint{0.907053in}{1.492434in}}{\pgfqpoint{0.912877in}{1.498258in}}%
\pgfpathcurveto{\pgfqpoint{0.918701in}{1.504082in}}{\pgfqpoint{0.921973in}{1.511982in}}{\pgfqpoint{0.921973in}{1.520218in}}%
\pgfpathcurveto{\pgfqpoint{0.921973in}{1.528454in}}{\pgfqpoint{0.918701in}{1.536354in}}{\pgfqpoint{0.912877in}{1.542178in}}%
\pgfpathcurveto{\pgfqpoint{0.907053in}{1.548002in}}{\pgfqpoint{0.899153in}{1.551275in}}{\pgfqpoint{0.890916in}{1.551275in}}%
\pgfpathcurveto{\pgfqpoint{0.882680in}{1.551275in}}{\pgfqpoint{0.874780in}{1.548002in}}{\pgfqpoint{0.868956in}{1.542178in}}%
\pgfpathcurveto{\pgfqpoint{0.863132in}{1.536354in}}{\pgfqpoint{0.859860in}{1.528454in}}{\pgfqpoint{0.859860in}{1.520218in}}%
\pgfpathcurveto{\pgfqpoint{0.859860in}{1.511982in}}{\pgfqpoint{0.863132in}{1.504082in}}{\pgfqpoint{0.868956in}{1.498258in}}%
\pgfpathcurveto{\pgfqpoint{0.874780in}{1.492434in}}{\pgfqpoint{0.882680in}{1.489162in}}{\pgfqpoint{0.890916in}{1.489162in}}%
\pgfpathclose%
\pgfusepath{stroke,fill}%
\end{pgfscope}%
\begin{pgfscope}%
\pgfpathrectangle{\pgfqpoint{0.100000in}{0.212622in}}{\pgfqpoint{3.696000in}{3.696000in}}%
\pgfusepath{clip}%
\pgfsetbuttcap%
\pgfsetroundjoin%
\definecolor{currentfill}{rgb}{0.121569,0.466667,0.705882}%
\pgfsetfillcolor{currentfill}%
\pgfsetfillopacity{0.585499}%
\pgfsetlinewidth{1.003750pt}%
\definecolor{currentstroke}{rgb}{0.121569,0.466667,0.705882}%
\pgfsetstrokecolor{currentstroke}%
\pgfsetstrokeopacity{0.585499}%
\pgfsetdash{}{0pt}%
\pgfpathmoveto{\pgfqpoint{0.890916in}{1.489161in}}%
\pgfpathcurveto{\pgfqpoint{0.899153in}{1.489161in}}{\pgfqpoint{0.907053in}{1.492433in}}{\pgfqpoint{0.912877in}{1.498257in}}%
\pgfpathcurveto{\pgfqpoint{0.918700in}{1.504081in}}{\pgfqpoint{0.921973in}{1.511981in}}{\pgfqpoint{0.921973in}{1.520217in}}%
\pgfpathcurveto{\pgfqpoint{0.921973in}{1.528454in}}{\pgfqpoint{0.918700in}{1.536354in}}{\pgfqpoint{0.912877in}{1.542178in}}%
\pgfpathcurveto{\pgfqpoint{0.907053in}{1.548002in}}{\pgfqpoint{0.899153in}{1.551274in}}{\pgfqpoint{0.890916in}{1.551274in}}%
\pgfpathcurveto{\pgfqpoint{0.882680in}{1.551274in}}{\pgfqpoint{0.874780in}{1.548002in}}{\pgfqpoint{0.868956in}{1.542178in}}%
\pgfpathcurveto{\pgfqpoint{0.863132in}{1.536354in}}{\pgfqpoint{0.859860in}{1.528454in}}{\pgfqpoint{0.859860in}{1.520217in}}%
\pgfpathcurveto{\pgfqpoint{0.859860in}{1.511981in}}{\pgfqpoint{0.863132in}{1.504081in}}{\pgfqpoint{0.868956in}{1.498257in}}%
\pgfpathcurveto{\pgfqpoint{0.874780in}{1.492433in}}{\pgfqpoint{0.882680in}{1.489161in}}{\pgfqpoint{0.890916in}{1.489161in}}%
\pgfpathclose%
\pgfusepath{stroke,fill}%
\end{pgfscope}%
\begin{pgfscope}%
\pgfpathrectangle{\pgfqpoint{0.100000in}{0.212622in}}{\pgfqpoint{3.696000in}{3.696000in}}%
\pgfusepath{clip}%
\pgfsetbuttcap%
\pgfsetroundjoin%
\definecolor{currentfill}{rgb}{0.121569,0.466667,0.705882}%
\pgfsetfillcolor{currentfill}%
\pgfsetfillopacity{0.585499}%
\pgfsetlinewidth{1.003750pt}%
\definecolor{currentstroke}{rgb}{0.121569,0.466667,0.705882}%
\pgfsetstrokecolor{currentstroke}%
\pgfsetstrokeopacity{0.585499}%
\pgfsetdash{}{0pt}%
\pgfpathmoveto{\pgfqpoint{0.890916in}{1.489161in}}%
\pgfpathcurveto{\pgfqpoint{0.899152in}{1.489161in}}{\pgfqpoint{0.907053in}{1.492433in}}{\pgfqpoint{0.912876in}{1.498257in}}%
\pgfpathcurveto{\pgfqpoint{0.918700in}{1.504081in}}{\pgfqpoint{0.921973in}{1.511981in}}{\pgfqpoint{0.921973in}{1.520217in}}%
\pgfpathcurveto{\pgfqpoint{0.921973in}{1.528453in}}{\pgfqpoint{0.918700in}{1.536353in}}{\pgfqpoint{0.912876in}{1.542177in}}%
\pgfpathcurveto{\pgfqpoint{0.907053in}{1.548001in}}{\pgfqpoint{0.899152in}{1.551274in}}{\pgfqpoint{0.890916in}{1.551274in}}%
\pgfpathcurveto{\pgfqpoint{0.882680in}{1.551274in}}{\pgfqpoint{0.874780in}{1.548001in}}{\pgfqpoint{0.868956in}{1.542177in}}%
\pgfpathcurveto{\pgfqpoint{0.863132in}{1.536353in}}{\pgfqpoint{0.859860in}{1.528453in}}{\pgfqpoint{0.859860in}{1.520217in}}%
\pgfpathcurveto{\pgfqpoint{0.859860in}{1.511981in}}{\pgfqpoint{0.863132in}{1.504081in}}{\pgfqpoint{0.868956in}{1.498257in}}%
\pgfpathcurveto{\pgfqpoint{0.874780in}{1.492433in}}{\pgfqpoint{0.882680in}{1.489161in}}{\pgfqpoint{0.890916in}{1.489161in}}%
\pgfpathclose%
\pgfusepath{stroke,fill}%
\end{pgfscope}%
\begin{pgfscope}%
\pgfpathrectangle{\pgfqpoint{0.100000in}{0.212622in}}{\pgfqpoint{3.696000in}{3.696000in}}%
\pgfusepath{clip}%
\pgfsetbuttcap%
\pgfsetroundjoin%
\definecolor{currentfill}{rgb}{0.121569,0.466667,0.705882}%
\pgfsetfillcolor{currentfill}%
\pgfsetfillopacity{0.585499}%
\pgfsetlinewidth{1.003750pt}%
\definecolor{currentstroke}{rgb}{0.121569,0.466667,0.705882}%
\pgfsetstrokecolor{currentstroke}%
\pgfsetstrokeopacity{0.585499}%
\pgfsetdash{}{0pt}%
\pgfpathmoveto{\pgfqpoint{0.890916in}{1.489160in}}%
\pgfpathcurveto{\pgfqpoint{0.899152in}{1.489160in}}{\pgfqpoint{0.907052in}{1.492433in}}{\pgfqpoint{0.912876in}{1.498257in}}%
\pgfpathcurveto{\pgfqpoint{0.918700in}{1.504081in}}{\pgfqpoint{0.921973in}{1.511981in}}{\pgfqpoint{0.921973in}{1.520217in}}%
\pgfpathcurveto{\pgfqpoint{0.921973in}{1.528453in}}{\pgfqpoint{0.918700in}{1.536353in}}{\pgfqpoint{0.912876in}{1.542177in}}%
\pgfpathcurveto{\pgfqpoint{0.907052in}{1.548001in}}{\pgfqpoint{0.899152in}{1.551273in}}{\pgfqpoint{0.890916in}{1.551273in}}%
\pgfpathcurveto{\pgfqpoint{0.882680in}{1.551273in}}{\pgfqpoint{0.874780in}{1.548001in}}{\pgfqpoint{0.868956in}{1.542177in}}%
\pgfpathcurveto{\pgfqpoint{0.863132in}{1.536353in}}{\pgfqpoint{0.859860in}{1.528453in}}{\pgfqpoint{0.859860in}{1.520217in}}%
\pgfpathcurveto{\pgfqpoint{0.859860in}{1.511981in}}{\pgfqpoint{0.863132in}{1.504081in}}{\pgfqpoint{0.868956in}{1.498257in}}%
\pgfpathcurveto{\pgfqpoint{0.874780in}{1.492433in}}{\pgfqpoint{0.882680in}{1.489160in}}{\pgfqpoint{0.890916in}{1.489160in}}%
\pgfpathclose%
\pgfusepath{stroke,fill}%
\end{pgfscope}%
\begin{pgfscope}%
\pgfpathrectangle{\pgfqpoint{0.100000in}{0.212622in}}{\pgfqpoint{3.696000in}{3.696000in}}%
\pgfusepath{clip}%
\pgfsetbuttcap%
\pgfsetroundjoin%
\definecolor{currentfill}{rgb}{0.121569,0.466667,0.705882}%
\pgfsetfillcolor{currentfill}%
\pgfsetfillopacity{0.585499}%
\pgfsetlinewidth{1.003750pt}%
\definecolor{currentstroke}{rgb}{0.121569,0.466667,0.705882}%
\pgfsetstrokecolor{currentstroke}%
\pgfsetstrokeopacity{0.585499}%
\pgfsetdash{}{0pt}%
\pgfpathmoveto{\pgfqpoint{0.890916in}{1.489160in}}%
\pgfpathcurveto{\pgfqpoint{0.899152in}{1.489160in}}{\pgfqpoint{0.907052in}{1.492433in}}{\pgfqpoint{0.912876in}{1.498257in}}%
\pgfpathcurveto{\pgfqpoint{0.918700in}{1.504081in}}{\pgfqpoint{0.921973in}{1.511981in}}{\pgfqpoint{0.921973in}{1.520217in}}%
\pgfpathcurveto{\pgfqpoint{0.921973in}{1.528453in}}{\pgfqpoint{0.918700in}{1.536353in}}{\pgfqpoint{0.912876in}{1.542177in}}%
\pgfpathcurveto{\pgfqpoint{0.907052in}{1.548001in}}{\pgfqpoint{0.899152in}{1.551273in}}{\pgfqpoint{0.890916in}{1.551273in}}%
\pgfpathcurveto{\pgfqpoint{0.882680in}{1.551273in}}{\pgfqpoint{0.874780in}{1.548001in}}{\pgfqpoint{0.868956in}{1.542177in}}%
\pgfpathcurveto{\pgfqpoint{0.863132in}{1.536353in}}{\pgfqpoint{0.859860in}{1.528453in}}{\pgfqpoint{0.859860in}{1.520217in}}%
\pgfpathcurveto{\pgfqpoint{0.859860in}{1.511981in}}{\pgfqpoint{0.863132in}{1.504081in}}{\pgfqpoint{0.868956in}{1.498257in}}%
\pgfpathcurveto{\pgfqpoint{0.874780in}{1.492433in}}{\pgfqpoint{0.882680in}{1.489160in}}{\pgfqpoint{0.890916in}{1.489160in}}%
\pgfpathclose%
\pgfusepath{stroke,fill}%
\end{pgfscope}%
\begin{pgfscope}%
\pgfpathrectangle{\pgfqpoint{0.100000in}{0.212622in}}{\pgfqpoint{3.696000in}{3.696000in}}%
\pgfusepath{clip}%
\pgfsetbuttcap%
\pgfsetroundjoin%
\definecolor{currentfill}{rgb}{0.121569,0.466667,0.705882}%
\pgfsetfillcolor{currentfill}%
\pgfsetfillopacity{0.585499}%
\pgfsetlinewidth{1.003750pt}%
\definecolor{currentstroke}{rgb}{0.121569,0.466667,0.705882}%
\pgfsetstrokecolor{currentstroke}%
\pgfsetstrokeopacity{0.585499}%
\pgfsetdash{}{0pt}%
\pgfpathmoveto{\pgfqpoint{0.890916in}{1.489160in}}%
\pgfpathcurveto{\pgfqpoint{0.899152in}{1.489160in}}{\pgfqpoint{0.907052in}{1.492433in}}{\pgfqpoint{0.912876in}{1.498257in}}%
\pgfpathcurveto{\pgfqpoint{0.918700in}{1.504080in}}{\pgfqpoint{0.921973in}{1.511981in}}{\pgfqpoint{0.921973in}{1.520217in}}%
\pgfpathcurveto{\pgfqpoint{0.921973in}{1.528453in}}{\pgfqpoint{0.918700in}{1.536353in}}{\pgfqpoint{0.912876in}{1.542177in}}%
\pgfpathcurveto{\pgfqpoint{0.907052in}{1.548001in}}{\pgfqpoint{0.899152in}{1.551273in}}{\pgfqpoint{0.890916in}{1.551273in}}%
\pgfpathcurveto{\pgfqpoint{0.882680in}{1.551273in}}{\pgfqpoint{0.874780in}{1.548001in}}{\pgfqpoint{0.868956in}{1.542177in}}%
\pgfpathcurveto{\pgfqpoint{0.863132in}{1.536353in}}{\pgfqpoint{0.859860in}{1.528453in}}{\pgfqpoint{0.859860in}{1.520217in}}%
\pgfpathcurveto{\pgfqpoint{0.859860in}{1.511981in}}{\pgfqpoint{0.863132in}{1.504080in}}{\pgfqpoint{0.868956in}{1.498257in}}%
\pgfpathcurveto{\pgfqpoint{0.874780in}{1.492433in}}{\pgfqpoint{0.882680in}{1.489160in}}{\pgfqpoint{0.890916in}{1.489160in}}%
\pgfpathclose%
\pgfusepath{stroke,fill}%
\end{pgfscope}%
\begin{pgfscope}%
\pgfpathrectangle{\pgfqpoint{0.100000in}{0.212622in}}{\pgfqpoint{3.696000in}{3.696000in}}%
\pgfusepath{clip}%
\pgfsetbuttcap%
\pgfsetroundjoin%
\definecolor{currentfill}{rgb}{0.121569,0.466667,0.705882}%
\pgfsetfillcolor{currentfill}%
\pgfsetfillopacity{0.585499}%
\pgfsetlinewidth{1.003750pt}%
\definecolor{currentstroke}{rgb}{0.121569,0.466667,0.705882}%
\pgfsetstrokecolor{currentstroke}%
\pgfsetstrokeopacity{0.585499}%
\pgfsetdash{}{0pt}%
\pgfpathmoveto{\pgfqpoint{0.890916in}{1.489160in}}%
\pgfpathcurveto{\pgfqpoint{0.899152in}{1.489160in}}{\pgfqpoint{0.907052in}{1.492433in}}{\pgfqpoint{0.912876in}{1.498256in}}%
\pgfpathcurveto{\pgfqpoint{0.918700in}{1.504080in}}{\pgfqpoint{0.921973in}{1.511980in}}{\pgfqpoint{0.921973in}{1.520217in}}%
\pgfpathcurveto{\pgfqpoint{0.921973in}{1.528453in}}{\pgfqpoint{0.918700in}{1.536353in}}{\pgfqpoint{0.912876in}{1.542177in}}%
\pgfpathcurveto{\pgfqpoint{0.907052in}{1.548001in}}{\pgfqpoint{0.899152in}{1.551273in}}{\pgfqpoint{0.890916in}{1.551273in}}%
\pgfpathcurveto{\pgfqpoint{0.882680in}{1.551273in}}{\pgfqpoint{0.874780in}{1.548001in}}{\pgfqpoint{0.868956in}{1.542177in}}%
\pgfpathcurveto{\pgfqpoint{0.863132in}{1.536353in}}{\pgfqpoint{0.859860in}{1.528453in}}{\pgfqpoint{0.859860in}{1.520217in}}%
\pgfpathcurveto{\pgfqpoint{0.859860in}{1.511980in}}{\pgfqpoint{0.863132in}{1.504080in}}{\pgfqpoint{0.868956in}{1.498256in}}%
\pgfpathcurveto{\pgfqpoint{0.874780in}{1.492433in}}{\pgfqpoint{0.882680in}{1.489160in}}{\pgfqpoint{0.890916in}{1.489160in}}%
\pgfpathclose%
\pgfusepath{stroke,fill}%
\end{pgfscope}%
\begin{pgfscope}%
\pgfpathrectangle{\pgfqpoint{0.100000in}{0.212622in}}{\pgfqpoint{3.696000in}{3.696000in}}%
\pgfusepath{clip}%
\pgfsetbuttcap%
\pgfsetroundjoin%
\definecolor{currentfill}{rgb}{0.121569,0.466667,0.705882}%
\pgfsetfillcolor{currentfill}%
\pgfsetfillopacity{0.585499}%
\pgfsetlinewidth{1.003750pt}%
\definecolor{currentstroke}{rgb}{0.121569,0.466667,0.705882}%
\pgfsetstrokecolor{currentstroke}%
\pgfsetstrokeopacity{0.585499}%
\pgfsetdash{}{0pt}%
\pgfpathmoveto{\pgfqpoint{0.890916in}{1.489160in}}%
\pgfpathcurveto{\pgfqpoint{0.899152in}{1.489160in}}{\pgfqpoint{0.907052in}{1.492433in}}{\pgfqpoint{0.912876in}{1.498256in}}%
\pgfpathcurveto{\pgfqpoint{0.918700in}{1.504080in}}{\pgfqpoint{0.921973in}{1.511980in}}{\pgfqpoint{0.921973in}{1.520217in}}%
\pgfpathcurveto{\pgfqpoint{0.921973in}{1.528453in}}{\pgfqpoint{0.918700in}{1.536353in}}{\pgfqpoint{0.912876in}{1.542177in}}%
\pgfpathcurveto{\pgfqpoint{0.907052in}{1.548001in}}{\pgfqpoint{0.899152in}{1.551273in}}{\pgfqpoint{0.890916in}{1.551273in}}%
\pgfpathcurveto{\pgfqpoint{0.882680in}{1.551273in}}{\pgfqpoint{0.874780in}{1.548001in}}{\pgfqpoint{0.868956in}{1.542177in}}%
\pgfpathcurveto{\pgfqpoint{0.863132in}{1.536353in}}{\pgfqpoint{0.859860in}{1.528453in}}{\pgfqpoint{0.859860in}{1.520217in}}%
\pgfpathcurveto{\pgfqpoint{0.859860in}{1.511980in}}{\pgfqpoint{0.863132in}{1.504080in}}{\pgfqpoint{0.868956in}{1.498256in}}%
\pgfpathcurveto{\pgfqpoint{0.874780in}{1.492433in}}{\pgfqpoint{0.882680in}{1.489160in}}{\pgfqpoint{0.890916in}{1.489160in}}%
\pgfpathclose%
\pgfusepath{stroke,fill}%
\end{pgfscope}%
\begin{pgfscope}%
\pgfpathrectangle{\pgfqpoint{0.100000in}{0.212622in}}{\pgfqpoint{3.696000in}{3.696000in}}%
\pgfusepath{clip}%
\pgfsetbuttcap%
\pgfsetroundjoin%
\definecolor{currentfill}{rgb}{0.121569,0.466667,0.705882}%
\pgfsetfillcolor{currentfill}%
\pgfsetfillopacity{0.585499}%
\pgfsetlinewidth{1.003750pt}%
\definecolor{currentstroke}{rgb}{0.121569,0.466667,0.705882}%
\pgfsetstrokecolor{currentstroke}%
\pgfsetstrokeopacity{0.585499}%
\pgfsetdash{}{0pt}%
\pgfpathmoveto{\pgfqpoint{0.890916in}{1.489160in}}%
\pgfpathcurveto{\pgfqpoint{0.899152in}{1.489160in}}{\pgfqpoint{0.907052in}{1.492433in}}{\pgfqpoint{0.912876in}{1.498256in}}%
\pgfpathcurveto{\pgfqpoint{0.918700in}{1.504080in}}{\pgfqpoint{0.921973in}{1.511980in}}{\pgfqpoint{0.921973in}{1.520217in}}%
\pgfpathcurveto{\pgfqpoint{0.921973in}{1.528453in}}{\pgfqpoint{0.918700in}{1.536353in}}{\pgfqpoint{0.912876in}{1.542177in}}%
\pgfpathcurveto{\pgfqpoint{0.907052in}{1.548001in}}{\pgfqpoint{0.899152in}{1.551273in}}{\pgfqpoint{0.890916in}{1.551273in}}%
\pgfpathcurveto{\pgfqpoint{0.882680in}{1.551273in}}{\pgfqpoint{0.874780in}{1.548001in}}{\pgfqpoint{0.868956in}{1.542177in}}%
\pgfpathcurveto{\pgfqpoint{0.863132in}{1.536353in}}{\pgfqpoint{0.859860in}{1.528453in}}{\pgfqpoint{0.859860in}{1.520217in}}%
\pgfpathcurveto{\pgfqpoint{0.859860in}{1.511980in}}{\pgfqpoint{0.863132in}{1.504080in}}{\pgfqpoint{0.868956in}{1.498256in}}%
\pgfpathcurveto{\pgfqpoint{0.874780in}{1.492433in}}{\pgfqpoint{0.882680in}{1.489160in}}{\pgfqpoint{0.890916in}{1.489160in}}%
\pgfpathclose%
\pgfusepath{stroke,fill}%
\end{pgfscope}%
\begin{pgfscope}%
\pgfpathrectangle{\pgfqpoint{0.100000in}{0.212622in}}{\pgfqpoint{3.696000in}{3.696000in}}%
\pgfusepath{clip}%
\pgfsetbuttcap%
\pgfsetroundjoin%
\definecolor{currentfill}{rgb}{0.121569,0.466667,0.705882}%
\pgfsetfillcolor{currentfill}%
\pgfsetfillopacity{0.585499}%
\pgfsetlinewidth{1.003750pt}%
\definecolor{currentstroke}{rgb}{0.121569,0.466667,0.705882}%
\pgfsetstrokecolor{currentstroke}%
\pgfsetstrokeopacity{0.585499}%
\pgfsetdash{}{0pt}%
\pgfpathmoveto{\pgfqpoint{0.890916in}{1.489160in}}%
\pgfpathcurveto{\pgfqpoint{0.899152in}{1.489160in}}{\pgfqpoint{0.907052in}{1.492433in}}{\pgfqpoint{0.912876in}{1.498256in}}%
\pgfpathcurveto{\pgfqpoint{0.918700in}{1.504080in}}{\pgfqpoint{0.921973in}{1.511980in}}{\pgfqpoint{0.921973in}{1.520217in}}%
\pgfpathcurveto{\pgfqpoint{0.921973in}{1.528453in}}{\pgfqpoint{0.918700in}{1.536353in}}{\pgfqpoint{0.912876in}{1.542177in}}%
\pgfpathcurveto{\pgfqpoint{0.907052in}{1.548001in}}{\pgfqpoint{0.899152in}{1.551273in}}{\pgfqpoint{0.890916in}{1.551273in}}%
\pgfpathcurveto{\pgfqpoint{0.882680in}{1.551273in}}{\pgfqpoint{0.874780in}{1.548001in}}{\pgfqpoint{0.868956in}{1.542177in}}%
\pgfpathcurveto{\pgfqpoint{0.863132in}{1.536353in}}{\pgfqpoint{0.859860in}{1.528453in}}{\pgfqpoint{0.859860in}{1.520217in}}%
\pgfpathcurveto{\pgfqpoint{0.859860in}{1.511980in}}{\pgfqpoint{0.863132in}{1.504080in}}{\pgfqpoint{0.868956in}{1.498256in}}%
\pgfpathcurveto{\pgfqpoint{0.874780in}{1.492433in}}{\pgfqpoint{0.882680in}{1.489160in}}{\pgfqpoint{0.890916in}{1.489160in}}%
\pgfpathclose%
\pgfusepath{stroke,fill}%
\end{pgfscope}%
\begin{pgfscope}%
\pgfpathrectangle{\pgfqpoint{0.100000in}{0.212622in}}{\pgfqpoint{3.696000in}{3.696000in}}%
\pgfusepath{clip}%
\pgfsetbuttcap%
\pgfsetroundjoin%
\definecolor{currentfill}{rgb}{0.121569,0.466667,0.705882}%
\pgfsetfillcolor{currentfill}%
\pgfsetfillopacity{0.585499}%
\pgfsetlinewidth{1.003750pt}%
\definecolor{currentstroke}{rgb}{0.121569,0.466667,0.705882}%
\pgfsetstrokecolor{currentstroke}%
\pgfsetstrokeopacity{0.585499}%
\pgfsetdash{}{0pt}%
\pgfpathmoveto{\pgfqpoint{0.890916in}{1.489160in}}%
\pgfpathcurveto{\pgfqpoint{0.899152in}{1.489160in}}{\pgfqpoint{0.907052in}{1.492433in}}{\pgfqpoint{0.912876in}{1.498256in}}%
\pgfpathcurveto{\pgfqpoint{0.918700in}{1.504080in}}{\pgfqpoint{0.921973in}{1.511980in}}{\pgfqpoint{0.921973in}{1.520217in}}%
\pgfpathcurveto{\pgfqpoint{0.921973in}{1.528453in}}{\pgfqpoint{0.918700in}{1.536353in}}{\pgfqpoint{0.912876in}{1.542177in}}%
\pgfpathcurveto{\pgfqpoint{0.907052in}{1.548001in}}{\pgfqpoint{0.899152in}{1.551273in}}{\pgfqpoint{0.890916in}{1.551273in}}%
\pgfpathcurveto{\pgfqpoint{0.882680in}{1.551273in}}{\pgfqpoint{0.874780in}{1.548001in}}{\pgfqpoint{0.868956in}{1.542177in}}%
\pgfpathcurveto{\pgfqpoint{0.863132in}{1.536353in}}{\pgfqpoint{0.859860in}{1.528453in}}{\pgfqpoint{0.859860in}{1.520217in}}%
\pgfpathcurveto{\pgfqpoint{0.859860in}{1.511980in}}{\pgfqpoint{0.863132in}{1.504080in}}{\pgfqpoint{0.868956in}{1.498256in}}%
\pgfpathcurveto{\pgfqpoint{0.874780in}{1.492433in}}{\pgfqpoint{0.882680in}{1.489160in}}{\pgfqpoint{0.890916in}{1.489160in}}%
\pgfpathclose%
\pgfusepath{stroke,fill}%
\end{pgfscope}%
\begin{pgfscope}%
\pgfpathrectangle{\pgfqpoint{0.100000in}{0.212622in}}{\pgfqpoint{3.696000in}{3.696000in}}%
\pgfusepath{clip}%
\pgfsetbuttcap%
\pgfsetroundjoin%
\definecolor{currentfill}{rgb}{0.121569,0.466667,0.705882}%
\pgfsetfillcolor{currentfill}%
\pgfsetfillopacity{0.585499}%
\pgfsetlinewidth{1.003750pt}%
\definecolor{currentstroke}{rgb}{0.121569,0.466667,0.705882}%
\pgfsetstrokecolor{currentstroke}%
\pgfsetstrokeopacity{0.585499}%
\pgfsetdash{}{0pt}%
\pgfpathmoveto{\pgfqpoint{0.890916in}{1.489160in}}%
\pgfpathcurveto{\pgfqpoint{0.899152in}{1.489160in}}{\pgfqpoint{0.907052in}{1.492433in}}{\pgfqpoint{0.912876in}{1.498256in}}%
\pgfpathcurveto{\pgfqpoint{0.918700in}{1.504080in}}{\pgfqpoint{0.921973in}{1.511980in}}{\pgfqpoint{0.921973in}{1.520217in}}%
\pgfpathcurveto{\pgfqpoint{0.921973in}{1.528453in}}{\pgfqpoint{0.918700in}{1.536353in}}{\pgfqpoint{0.912876in}{1.542177in}}%
\pgfpathcurveto{\pgfqpoint{0.907052in}{1.548001in}}{\pgfqpoint{0.899152in}{1.551273in}}{\pgfqpoint{0.890916in}{1.551273in}}%
\pgfpathcurveto{\pgfqpoint{0.882680in}{1.551273in}}{\pgfqpoint{0.874780in}{1.548001in}}{\pgfqpoint{0.868956in}{1.542177in}}%
\pgfpathcurveto{\pgfqpoint{0.863132in}{1.536353in}}{\pgfqpoint{0.859860in}{1.528453in}}{\pgfqpoint{0.859860in}{1.520217in}}%
\pgfpathcurveto{\pgfqpoint{0.859860in}{1.511980in}}{\pgfqpoint{0.863132in}{1.504080in}}{\pgfqpoint{0.868956in}{1.498256in}}%
\pgfpathcurveto{\pgfqpoint{0.874780in}{1.492433in}}{\pgfqpoint{0.882680in}{1.489160in}}{\pgfqpoint{0.890916in}{1.489160in}}%
\pgfpathclose%
\pgfusepath{stroke,fill}%
\end{pgfscope}%
\begin{pgfscope}%
\pgfpathrectangle{\pgfqpoint{0.100000in}{0.212622in}}{\pgfqpoint{3.696000in}{3.696000in}}%
\pgfusepath{clip}%
\pgfsetbuttcap%
\pgfsetroundjoin%
\definecolor{currentfill}{rgb}{0.121569,0.466667,0.705882}%
\pgfsetfillcolor{currentfill}%
\pgfsetfillopacity{0.585499}%
\pgfsetlinewidth{1.003750pt}%
\definecolor{currentstroke}{rgb}{0.121569,0.466667,0.705882}%
\pgfsetstrokecolor{currentstroke}%
\pgfsetstrokeopacity{0.585499}%
\pgfsetdash{}{0pt}%
\pgfpathmoveto{\pgfqpoint{0.890916in}{1.489160in}}%
\pgfpathcurveto{\pgfqpoint{0.899152in}{1.489160in}}{\pgfqpoint{0.907052in}{1.492433in}}{\pgfqpoint{0.912876in}{1.498256in}}%
\pgfpathcurveto{\pgfqpoint{0.918700in}{1.504080in}}{\pgfqpoint{0.921973in}{1.511980in}}{\pgfqpoint{0.921973in}{1.520217in}}%
\pgfpathcurveto{\pgfqpoint{0.921973in}{1.528453in}}{\pgfqpoint{0.918700in}{1.536353in}}{\pgfqpoint{0.912876in}{1.542177in}}%
\pgfpathcurveto{\pgfqpoint{0.907052in}{1.548001in}}{\pgfqpoint{0.899152in}{1.551273in}}{\pgfqpoint{0.890916in}{1.551273in}}%
\pgfpathcurveto{\pgfqpoint{0.882680in}{1.551273in}}{\pgfqpoint{0.874780in}{1.548001in}}{\pgfqpoint{0.868956in}{1.542177in}}%
\pgfpathcurveto{\pgfqpoint{0.863132in}{1.536353in}}{\pgfqpoint{0.859860in}{1.528453in}}{\pgfqpoint{0.859860in}{1.520217in}}%
\pgfpathcurveto{\pgfqpoint{0.859860in}{1.511980in}}{\pgfqpoint{0.863132in}{1.504080in}}{\pgfqpoint{0.868956in}{1.498256in}}%
\pgfpathcurveto{\pgfqpoint{0.874780in}{1.492433in}}{\pgfqpoint{0.882680in}{1.489160in}}{\pgfqpoint{0.890916in}{1.489160in}}%
\pgfpathclose%
\pgfusepath{stroke,fill}%
\end{pgfscope}%
\begin{pgfscope}%
\pgfpathrectangle{\pgfqpoint{0.100000in}{0.212622in}}{\pgfqpoint{3.696000in}{3.696000in}}%
\pgfusepath{clip}%
\pgfsetbuttcap%
\pgfsetroundjoin%
\definecolor{currentfill}{rgb}{0.121569,0.466667,0.705882}%
\pgfsetfillcolor{currentfill}%
\pgfsetfillopacity{0.585499}%
\pgfsetlinewidth{1.003750pt}%
\definecolor{currentstroke}{rgb}{0.121569,0.466667,0.705882}%
\pgfsetstrokecolor{currentstroke}%
\pgfsetstrokeopacity{0.585499}%
\pgfsetdash{}{0pt}%
\pgfpathmoveto{\pgfqpoint{0.890916in}{1.489160in}}%
\pgfpathcurveto{\pgfqpoint{0.899152in}{1.489160in}}{\pgfqpoint{0.907052in}{1.492433in}}{\pgfqpoint{0.912876in}{1.498256in}}%
\pgfpathcurveto{\pgfqpoint{0.918700in}{1.504080in}}{\pgfqpoint{0.921973in}{1.511980in}}{\pgfqpoint{0.921973in}{1.520217in}}%
\pgfpathcurveto{\pgfqpoint{0.921973in}{1.528453in}}{\pgfqpoint{0.918700in}{1.536353in}}{\pgfqpoint{0.912876in}{1.542177in}}%
\pgfpathcurveto{\pgfqpoint{0.907052in}{1.548001in}}{\pgfqpoint{0.899152in}{1.551273in}}{\pgfqpoint{0.890916in}{1.551273in}}%
\pgfpathcurveto{\pgfqpoint{0.882680in}{1.551273in}}{\pgfqpoint{0.874780in}{1.548001in}}{\pgfqpoint{0.868956in}{1.542177in}}%
\pgfpathcurveto{\pgfqpoint{0.863132in}{1.536353in}}{\pgfqpoint{0.859860in}{1.528453in}}{\pgfqpoint{0.859860in}{1.520217in}}%
\pgfpathcurveto{\pgfqpoint{0.859860in}{1.511980in}}{\pgfqpoint{0.863132in}{1.504080in}}{\pgfqpoint{0.868956in}{1.498256in}}%
\pgfpathcurveto{\pgfqpoint{0.874780in}{1.492433in}}{\pgfqpoint{0.882680in}{1.489160in}}{\pgfqpoint{0.890916in}{1.489160in}}%
\pgfpathclose%
\pgfusepath{stroke,fill}%
\end{pgfscope}%
\begin{pgfscope}%
\pgfpathrectangle{\pgfqpoint{0.100000in}{0.212622in}}{\pgfqpoint{3.696000in}{3.696000in}}%
\pgfusepath{clip}%
\pgfsetbuttcap%
\pgfsetroundjoin%
\definecolor{currentfill}{rgb}{0.121569,0.466667,0.705882}%
\pgfsetfillcolor{currentfill}%
\pgfsetfillopacity{0.585499}%
\pgfsetlinewidth{1.003750pt}%
\definecolor{currentstroke}{rgb}{0.121569,0.466667,0.705882}%
\pgfsetstrokecolor{currentstroke}%
\pgfsetstrokeopacity{0.585499}%
\pgfsetdash{}{0pt}%
\pgfpathmoveto{\pgfqpoint{0.890916in}{1.489160in}}%
\pgfpathcurveto{\pgfqpoint{0.899152in}{1.489160in}}{\pgfqpoint{0.907052in}{1.492433in}}{\pgfqpoint{0.912876in}{1.498256in}}%
\pgfpathcurveto{\pgfqpoint{0.918700in}{1.504080in}}{\pgfqpoint{0.921973in}{1.511980in}}{\pgfqpoint{0.921973in}{1.520217in}}%
\pgfpathcurveto{\pgfqpoint{0.921973in}{1.528453in}}{\pgfqpoint{0.918700in}{1.536353in}}{\pgfqpoint{0.912876in}{1.542177in}}%
\pgfpathcurveto{\pgfqpoint{0.907052in}{1.548001in}}{\pgfqpoint{0.899152in}{1.551273in}}{\pgfqpoint{0.890916in}{1.551273in}}%
\pgfpathcurveto{\pgfqpoint{0.882680in}{1.551273in}}{\pgfqpoint{0.874780in}{1.548001in}}{\pgfqpoint{0.868956in}{1.542177in}}%
\pgfpathcurveto{\pgfqpoint{0.863132in}{1.536353in}}{\pgfqpoint{0.859860in}{1.528453in}}{\pgfqpoint{0.859860in}{1.520217in}}%
\pgfpathcurveto{\pgfqpoint{0.859860in}{1.511980in}}{\pgfqpoint{0.863132in}{1.504080in}}{\pgfqpoint{0.868956in}{1.498256in}}%
\pgfpathcurveto{\pgfqpoint{0.874780in}{1.492433in}}{\pgfqpoint{0.882680in}{1.489160in}}{\pgfqpoint{0.890916in}{1.489160in}}%
\pgfpathclose%
\pgfusepath{stroke,fill}%
\end{pgfscope}%
\begin{pgfscope}%
\pgfpathrectangle{\pgfqpoint{0.100000in}{0.212622in}}{\pgfqpoint{3.696000in}{3.696000in}}%
\pgfusepath{clip}%
\pgfsetbuttcap%
\pgfsetroundjoin%
\definecolor{currentfill}{rgb}{0.121569,0.466667,0.705882}%
\pgfsetfillcolor{currentfill}%
\pgfsetfillopacity{0.585499}%
\pgfsetlinewidth{1.003750pt}%
\definecolor{currentstroke}{rgb}{0.121569,0.466667,0.705882}%
\pgfsetstrokecolor{currentstroke}%
\pgfsetstrokeopacity{0.585499}%
\pgfsetdash{}{0pt}%
\pgfpathmoveto{\pgfqpoint{0.890916in}{1.489160in}}%
\pgfpathcurveto{\pgfqpoint{0.899152in}{1.489160in}}{\pgfqpoint{0.907052in}{1.492433in}}{\pgfqpoint{0.912876in}{1.498256in}}%
\pgfpathcurveto{\pgfqpoint{0.918700in}{1.504080in}}{\pgfqpoint{0.921973in}{1.511980in}}{\pgfqpoint{0.921973in}{1.520217in}}%
\pgfpathcurveto{\pgfqpoint{0.921973in}{1.528453in}}{\pgfqpoint{0.918700in}{1.536353in}}{\pgfqpoint{0.912876in}{1.542177in}}%
\pgfpathcurveto{\pgfqpoint{0.907052in}{1.548001in}}{\pgfqpoint{0.899152in}{1.551273in}}{\pgfqpoint{0.890916in}{1.551273in}}%
\pgfpathcurveto{\pgfqpoint{0.882680in}{1.551273in}}{\pgfqpoint{0.874780in}{1.548001in}}{\pgfqpoint{0.868956in}{1.542177in}}%
\pgfpathcurveto{\pgfqpoint{0.863132in}{1.536353in}}{\pgfqpoint{0.859860in}{1.528453in}}{\pgfqpoint{0.859860in}{1.520217in}}%
\pgfpathcurveto{\pgfqpoint{0.859860in}{1.511980in}}{\pgfqpoint{0.863132in}{1.504080in}}{\pgfqpoint{0.868956in}{1.498256in}}%
\pgfpathcurveto{\pgfqpoint{0.874780in}{1.492433in}}{\pgfqpoint{0.882680in}{1.489160in}}{\pgfqpoint{0.890916in}{1.489160in}}%
\pgfpathclose%
\pgfusepath{stroke,fill}%
\end{pgfscope}%
\begin{pgfscope}%
\pgfpathrectangle{\pgfqpoint{0.100000in}{0.212622in}}{\pgfqpoint{3.696000in}{3.696000in}}%
\pgfusepath{clip}%
\pgfsetbuttcap%
\pgfsetroundjoin%
\definecolor{currentfill}{rgb}{0.121569,0.466667,0.705882}%
\pgfsetfillcolor{currentfill}%
\pgfsetfillopacity{0.585499}%
\pgfsetlinewidth{1.003750pt}%
\definecolor{currentstroke}{rgb}{0.121569,0.466667,0.705882}%
\pgfsetstrokecolor{currentstroke}%
\pgfsetstrokeopacity{0.585499}%
\pgfsetdash{}{0pt}%
\pgfpathmoveto{\pgfqpoint{0.890916in}{1.489160in}}%
\pgfpathcurveto{\pgfqpoint{0.899152in}{1.489160in}}{\pgfqpoint{0.907052in}{1.492433in}}{\pgfqpoint{0.912876in}{1.498256in}}%
\pgfpathcurveto{\pgfqpoint{0.918700in}{1.504080in}}{\pgfqpoint{0.921973in}{1.511980in}}{\pgfqpoint{0.921973in}{1.520217in}}%
\pgfpathcurveto{\pgfqpoint{0.921973in}{1.528453in}}{\pgfqpoint{0.918700in}{1.536353in}}{\pgfqpoint{0.912876in}{1.542177in}}%
\pgfpathcurveto{\pgfqpoint{0.907052in}{1.548001in}}{\pgfqpoint{0.899152in}{1.551273in}}{\pgfqpoint{0.890916in}{1.551273in}}%
\pgfpathcurveto{\pgfqpoint{0.882680in}{1.551273in}}{\pgfqpoint{0.874780in}{1.548001in}}{\pgfqpoint{0.868956in}{1.542177in}}%
\pgfpathcurveto{\pgfqpoint{0.863132in}{1.536353in}}{\pgfqpoint{0.859860in}{1.528453in}}{\pgfqpoint{0.859860in}{1.520217in}}%
\pgfpathcurveto{\pgfqpoint{0.859860in}{1.511980in}}{\pgfqpoint{0.863132in}{1.504080in}}{\pgfqpoint{0.868956in}{1.498256in}}%
\pgfpathcurveto{\pgfqpoint{0.874780in}{1.492433in}}{\pgfqpoint{0.882680in}{1.489160in}}{\pgfqpoint{0.890916in}{1.489160in}}%
\pgfpathclose%
\pgfusepath{stroke,fill}%
\end{pgfscope}%
\begin{pgfscope}%
\pgfpathrectangle{\pgfqpoint{0.100000in}{0.212622in}}{\pgfqpoint{3.696000in}{3.696000in}}%
\pgfusepath{clip}%
\pgfsetbuttcap%
\pgfsetroundjoin%
\definecolor{currentfill}{rgb}{0.121569,0.466667,0.705882}%
\pgfsetfillcolor{currentfill}%
\pgfsetfillopacity{0.585499}%
\pgfsetlinewidth{1.003750pt}%
\definecolor{currentstroke}{rgb}{0.121569,0.466667,0.705882}%
\pgfsetstrokecolor{currentstroke}%
\pgfsetstrokeopacity{0.585499}%
\pgfsetdash{}{0pt}%
\pgfpathmoveto{\pgfqpoint{0.890916in}{1.489160in}}%
\pgfpathcurveto{\pgfqpoint{0.899152in}{1.489160in}}{\pgfqpoint{0.907052in}{1.492433in}}{\pgfqpoint{0.912876in}{1.498256in}}%
\pgfpathcurveto{\pgfqpoint{0.918700in}{1.504080in}}{\pgfqpoint{0.921973in}{1.511980in}}{\pgfqpoint{0.921973in}{1.520217in}}%
\pgfpathcurveto{\pgfqpoint{0.921973in}{1.528453in}}{\pgfqpoint{0.918700in}{1.536353in}}{\pgfqpoint{0.912876in}{1.542177in}}%
\pgfpathcurveto{\pgfqpoint{0.907052in}{1.548001in}}{\pgfqpoint{0.899152in}{1.551273in}}{\pgfqpoint{0.890916in}{1.551273in}}%
\pgfpathcurveto{\pgfqpoint{0.882680in}{1.551273in}}{\pgfqpoint{0.874780in}{1.548001in}}{\pgfqpoint{0.868956in}{1.542177in}}%
\pgfpathcurveto{\pgfqpoint{0.863132in}{1.536353in}}{\pgfqpoint{0.859860in}{1.528453in}}{\pgfqpoint{0.859860in}{1.520217in}}%
\pgfpathcurveto{\pgfqpoint{0.859860in}{1.511980in}}{\pgfqpoint{0.863132in}{1.504080in}}{\pgfqpoint{0.868956in}{1.498256in}}%
\pgfpathcurveto{\pgfqpoint{0.874780in}{1.492433in}}{\pgfqpoint{0.882680in}{1.489160in}}{\pgfqpoint{0.890916in}{1.489160in}}%
\pgfpathclose%
\pgfusepath{stroke,fill}%
\end{pgfscope}%
\begin{pgfscope}%
\pgfpathrectangle{\pgfqpoint{0.100000in}{0.212622in}}{\pgfqpoint{3.696000in}{3.696000in}}%
\pgfusepath{clip}%
\pgfsetbuttcap%
\pgfsetroundjoin%
\definecolor{currentfill}{rgb}{0.121569,0.466667,0.705882}%
\pgfsetfillcolor{currentfill}%
\pgfsetfillopacity{0.585499}%
\pgfsetlinewidth{1.003750pt}%
\definecolor{currentstroke}{rgb}{0.121569,0.466667,0.705882}%
\pgfsetstrokecolor{currentstroke}%
\pgfsetstrokeopacity{0.585499}%
\pgfsetdash{}{0pt}%
\pgfpathmoveto{\pgfqpoint{0.890916in}{1.489160in}}%
\pgfpathcurveto{\pgfqpoint{0.899152in}{1.489160in}}{\pgfqpoint{0.907052in}{1.492433in}}{\pgfqpoint{0.912876in}{1.498256in}}%
\pgfpathcurveto{\pgfqpoint{0.918700in}{1.504080in}}{\pgfqpoint{0.921973in}{1.511980in}}{\pgfqpoint{0.921973in}{1.520217in}}%
\pgfpathcurveto{\pgfqpoint{0.921973in}{1.528453in}}{\pgfqpoint{0.918700in}{1.536353in}}{\pgfqpoint{0.912876in}{1.542177in}}%
\pgfpathcurveto{\pgfqpoint{0.907052in}{1.548001in}}{\pgfqpoint{0.899152in}{1.551273in}}{\pgfqpoint{0.890916in}{1.551273in}}%
\pgfpathcurveto{\pgfqpoint{0.882680in}{1.551273in}}{\pgfqpoint{0.874780in}{1.548001in}}{\pgfqpoint{0.868956in}{1.542177in}}%
\pgfpathcurveto{\pgfqpoint{0.863132in}{1.536353in}}{\pgfqpoint{0.859860in}{1.528453in}}{\pgfqpoint{0.859860in}{1.520217in}}%
\pgfpathcurveto{\pgfqpoint{0.859860in}{1.511980in}}{\pgfqpoint{0.863132in}{1.504080in}}{\pgfqpoint{0.868956in}{1.498256in}}%
\pgfpathcurveto{\pgfqpoint{0.874780in}{1.492433in}}{\pgfqpoint{0.882680in}{1.489160in}}{\pgfqpoint{0.890916in}{1.489160in}}%
\pgfpathclose%
\pgfusepath{stroke,fill}%
\end{pgfscope}%
\begin{pgfscope}%
\pgfpathrectangle{\pgfqpoint{0.100000in}{0.212622in}}{\pgfqpoint{3.696000in}{3.696000in}}%
\pgfusepath{clip}%
\pgfsetbuttcap%
\pgfsetroundjoin%
\definecolor{currentfill}{rgb}{0.121569,0.466667,0.705882}%
\pgfsetfillcolor{currentfill}%
\pgfsetfillopacity{0.585499}%
\pgfsetlinewidth{1.003750pt}%
\definecolor{currentstroke}{rgb}{0.121569,0.466667,0.705882}%
\pgfsetstrokecolor{currentstroke}%
\pgfsetstrokeopacity{0.585499}%
\pgfsetdash{}{0pt}%
\pgfpathmoveto{\pgfqpoint{0.890916in}{1.489160in}}%
\pgfpathcurveto{\pgfqpoint{0.899152in}{1.489160in}}{\pgfqpoint{0.907052in}{1.492433in}}{\pgfqpoint{0.912876in}{1.498256in}}%
\pgfpathcurveto{\pgfqpoint{0.918700in}{1.504080in}}{\pgfqpoint{0.921973in}{1.511980in}}{\pgfqpoint{0.921973in}{1.520217in}}%
\pgfpathcurveto{\pgfqpoint{0.921973in}{1.528453in}}{\pgfqpoint{0.918700in}{1.536353in}}{\pgfqpoint{0.912876in}{1.542177in}}%
\pgfpathcurveto{\pgfqpoint{0.907052in}{1.548001in}}{\pgfqpoint{0.899152in}{1.551273in}}{\pgfqpoint{0.890916in}{1.551273in}}%
\pgfpathcurveto{\pgfqpoint{0.882680in}{1.551273in}}{\pgfqpoint{0.874780in}{1.548001in}}{\pgfqpoint{0.868956in}{1.542177in}}%
\pgfpathcurveto{\pgfqpoint{0.863132in}{1.536353in}}{\pgfqpoint{0.859860in}{1.528453in}}{\pgfqpoint{0.859860in}{1.520217in}}%
\pgfpathcurveto{\pgfqpoint{0.859860in}{1.511980in}}{\pgfqpoint{0.863132in}{1.504080in}}{\pgfqpoint{0.868956in}{1.498256in}}%
\pgfpathcurveto{\pgfqpoint{0.874780in}{1.492433in}}{\pgfqpoint{0.882680in}{1.489160in}}{\pgfqpoint{0.890916in}{1.489160in}}%
\pgfpathclose%
\pgfusepath{stroke,fill}%
\end{pgfscope}%
\begin{pgfscope}%
\pgfpathrectangle{\pgfqpoint{0.100000in}{0.212622in}}{\pgfqpoint{3.696000in}{3.696000in}}%
\pgfusepath{clip}%
\pgfsetbuttcap%
\pgfsetroundjoin%
\definecolor{currentfill}{rgb}{0.121569,0.466667,0.705882}%
\pgfsetfillcolor{currentfill}%
\pgfsetfillopacity{0.585499}%
\pgfsetlinewidth{1.003750pt}%
\definecolor{currentstroke}{rgb}{0.121569,0.466667,0.705882}%
\pgfsetstrokecolor{currentstroke}%
\pgfsetstrokeopacity{0.585499}%
\pgfsetdash{}{0pt}%
\pgfpathmoveto{\pgfqpoint{0.890916in}{1.489160in}}%
\pgfpathcurveto{\pgfqpoint{0.899152in}{1.489160in}}{\pgfqpoint{0.907052in}{1.492433in}}{\pgfqpoint{0.912876in}{1.498256in}}%
\pgfpathcurveto{\pgfqpoint{0.918700in}{1.504080in}}{\pgfqpoint{0.921973in}{1.511980in}}{\pgfqpoint{0.921973in}{1.520217in}}%
\pgfpathcurveto{\pgfqpoint{0.921973in}{1.528453in}}{\pgfqpoint{0.918700in}{1.536353in}}{\pgfqpoint{0.912876in}{1.542177in}}%
\pgfpathcurveto{\pgfqpoint{0.907052in}{1.548001in}}{\pgfqpoint{0.899152in}{1.551273in}}{\pgfqpoint{0.890916in}{1.551273in}}%
\pgfpathcurveto{\pgfqpoint{0.882680in}{1.551273in}}{\pgfqpoint{0.874780in}{1.548001in}}{\pgfqpoint{0.868956in}{1.542177in}}%
\pgfpathcurveto{\pgfqpoint{0.863132in}{1.536353in}}{\pgfqpoint{0.859860in}{1.528453in}}{\pgfqpoint{0.859860in}{1.520217in}}%
\pgfpathcurveto{\pgfqpoint{0.859860in}{1.511980in}}{\pgfqpoint{0.863132in}{1.504080in}}{\pgfqpoint{0.868956in}{1.498256in}}%
\pgfpathcurveto{\pgfqpoint{0.874780in}{1.492433in}}{\pgfqpoint{0.882680in}{1.489160in}}{\pgfqpoint{0.890916in}{1.489160in}}%
\pgfpathclose%
\pgfusepath{stroke,fill}%
\end{pgfscope}%
\begin{pgfscope}%
\pgfpathrectangle{\pgfqpoint{0.100000in}{0.212622in}}{\pgfqpoint{3.696000in}{3.696000in}}%
\pgfusepath{clip}%
\pgfsetbuttcap%
\pgfsetroundjoin%
\definecolor{currentfill}{rgb}{0.121569,0.466667,0.705882}%
\pgfsetfillcolor{currentfill}%
\pgfsetfillopacity{0.585499}%
\pgfsetlinewidth{1.003750pt}%
\definecolor{currentstroke}{rgb}{0.121569,0.466667,0.705882}%
\pgfsetstrokecolor{currentstroke}%
\pgfsetstrokeopacity{0.585499}%
\pgfsetdash{}{0pt}%
\pgfpathmoveto{\pgfqpoint{0.890916in}{1.489160in}}%
\pgfpathcurveto{\pgfqpoint{0.899152in}{1.489160in}}{\pgfqpoint{0.907052in}{1.492433in}}{\pgfqpoint{0.912876in}{1.498256in}}%
\pgfpathcurveto{\pgfqpoint{0.918700in}{1.504080in}}{\pgfqpoint{0.921973in}{1.511980in}}{\pgfqpoint{0.921973in}{1.520217in}}%
\pgfpathcurveto{\pgfqpoint{0.921973in}{1.528453in}}{\pgfqpoint{0.918700in}{1.536353in}}{\pgfqpoint{0.912876in}{1.542177in}}%
\pgfpathcurveto{\pgfqpoint{0.907052in}{1.548001in}}{\pgfqpoint{0.899152in}{1.551273in}}{\pgfqpoint{0.890916in}{1.551273in}}%
\pgfpathcurveto{\pgfqpoint{0.882680in}{1.551273in}}{\pgfqpoint{0.874780in}{1.548001in}}{\pgfqpoint{0.868956in}{1.542177in}}%
\pgfpathcurveto{\pgfqpoint{0.863132in}{1.536353in}}{\pgfqpoint{0.859860in}{1.528453in}}{\pgfqpoint{0.859860in}{1.520217in}}%
\pgfpathcurveto{\pgfqpoint{0.859860in}{1.511980in}}{\pgfqpoint{0.863132in}{1.504080in}}{\pgfqpoint{0.868956in}{1.498256in}}%
\pgfpathcurveto{\pgfqpoint{0.874780in}{1.492433in}}{\pgfqpoint{0.882680in}{1.489160in}}{\pgfqpoint{0.890916in}{1.489160in}}%
\pgfpathclose%
\pgfusepath{stroke,fill}%
\end{pgfscope}%
\begin{pgfscope}%
\pgfpathrectangle{\pgfqpoint{0.100000in}{0.212622in}}{\pgfqpoint{3.696000in}{3.696000in}}%
\pgfusepath{clip}%
\pgfsetbuttcap%
\pgfsetroundjoin%
\definecolor{currentfill}{rgb}{0.121569,0.466667,0.705882}%
\pgfsetfillcolor{currentfill}%
\pgfsetfillopacity{0.585499}%
\pgfsetlinewidth{1.003750pt}%
\definecolor{currentstroke}{rgb}{0.121569,0.466667,0.705882}%
\pgfsetstrokecolor{currentstroke}%
\pgfsetstrokeopacity{0.585499}%
\pgfsetdash{}{0pt}%
\pgfpathmoveto{\pgfqpoint{0.890916in}{1.489160in}}%
\pgfpathcurveto{\pgfqpoint{0.899152in}{1.489160in}}{\pgfqpoint{0.907052in}{1.492433in}}{\pgfqpoint{0.912876in}{1.498256in}}%
\pgfpathcurveto{\pgfqpoint{0.918700in}{1.504080in}}{\pgfqpoint{0.921973in}{1.511980in}}{\pgfqpoint{0.921973in}{1.520217in}}%
\pgfpathcurveto{\pgfqpoint{0.921973in}{1.528453in}}{\pgfqpoint{0.918700in}{1.536353in}}{\pgfqpoint{0.912876in}{1.542177in}}%
\pgfpathcurveto{\pgfqpoint{0.907052in}{1.548001in}}{\pgfqpoint{0.899152in}{1.551273in}}{\pgfqpoint{0.890916in}{1.551273in}}%
\pgfpathcurveto{\pgfqpoint{0.882680in}{1.551273in}}{\pgfqpoint{0.874780in}{1.548001in}}{\pgfqpoint{0.868956in}{1.542177in}}%
\pgfpathcurveto{\pgfqpoint{0.863132in}{1.536353in}}{\pgfqpoint{0.859860in}{1.528453in}}{\pgfqpoint{0.859860in}{1.520217in}}%
\pgfpathcurveto{\pgfqpoint{0.859860in}{1.511980in}}{\pgfqpoint{0.863132in}{1.504080in}}{\pgfqpoint{0.868956in}{1.498256in}}%
\pgfpathcurveto{\pgfqpoint{0.874780in}{1.492433in}}{\pgfqpoint{0.882680in}{1.489160in}}{\pgfqpoint{0.890916in}{1.489160in}}%
\pgfpathclose%
\pgfusepath{stroke,fill}%
\end{pgfscope}%
\begin{pgfscope}%
\pgfpathrectangle{\pgfqpoint{0.100000in}{0.212622in}}{\pgfqpoint{3.696000in}{3.696000in}}%
\pgfusepath{clip}%
\pgfsetbuttcap%
\pgfsetroundjoin%
\definecolor{currentfill}{rgb}{0.121569,0.466667,0.705882}%
\pgfsetfillcolor{currentfill}%
\pgfsetfillopacity{0.585499}%
\pgfsetlinewidth{1.003750pt}%
\definecolor{currentstroke}{rgb}{0.121569,0.466667,0.705882}%
\pgfsetstrokecolor{currentstroke}%
\pgfsetstrokeopacity{0.585499}%
\pgfsetdash{}{0pt}%
\pgfpathmoveto{\pgfqpoint{0.890916in}{1.489160in}}%
\pgfpathcurveto{\pgfqpoint{0.899152in}{1.489160in}}{\pgfqpoint{0.907052in}{1.492433in}}{\pgfqpoint{0.912876in}{1.498256in}}%
\pgfpathcurveto{\pgfqpoint{0.918700in}{1.504080in}}{\pgfqpoint{0.921973in}{1.511980in}}{\pgfqpoint{0.921973in}{1.520217in}}%
\pgfpathcurveto{\pgfqpoint{0.921973in}{1.528453in}}{\pgfqpoint{0.918700in}{1.536353in}}{\pgfqpoint{0.912876in}{1.542177in}}%
\pgfpathcurveto{\pgfqpoint{0.907052in}{1.548001in}}{\pgfqpoint{0.899152in}{1.551273in}}{\pgfqpoint{0.890916in}{1.551273in}}%
\pgfpathcurveto{\pgfqpoint{0.882680in}{1.551273in}}{\pgfqpoint{0.874780in}{1.548001in}}{\pgfqpoint{0.868956in}{1.542177in}}%
\pgfpathcurveto{\pgfqpoint{0.863132in}{1.536353in}}{\pgfqpoint{0.859860in}{1.528453in}}{\pgfqpoint{0.859860in}{1.520217in}}%
\pgfpathcurveto{\pgfqpoint{0.859860in}{1.511980in}}{\pgfqpoint{0.863132in}{1.504080in}}{\pgfqpoint{0.868956in}{1.498256in}}%
\pgfpathcurveto{\pgfqpoint{0.874780in}{1.492433in}}{\pgfqpoint{0.882680in}{1.489160in}}{\pgfqpoint{0.890916in}{1.489160in}}%
\pgfpathclose%
\pgfusepath{stroke,fill}%
\end{pgfscope}%
\begin{pgfscope}%
\pgfpathrectangle{\pgfqpoint{0.100000in}{0.212622in}}{\pgfqpoint{3.696000in}{3.696000in}}%
\pgfusepath{clip}%
\pgfsetbuttcap%
\pgfsetroundjoin%
\definecolor{currentfill}{rgb}{0.121569,0.466667,0.705882}%
\pgfsetfillcolor{currentfill}%
\pgfsetfillopacity{0.585499}%
\pgfsetlinewidth{1.003750pt}%
\definecolor{currentstroke}{rgb}{0.121569,0.466667,0.705882}%
\pgfsetstrokecolor{currentstroke}%
\pgfsetstrokeopacity{0.585499}%
\pgfsetdash{}{0pt}%
\pgfpathmoveto{\pgfqpoint{0.890916in}{1.489160in}}%
\pgfpathcurveto{\pgfqpoint{0.899152in}{1.489160in}}{\pgfqpoint{0.907052in}{1.492433in}}{\pgfqpoint{0.912876in}{1.498256in}}%
\pgfpathcurveto{\pgfqpoint{0.918700in}{1.504080in}}{\pgfqpoint{0.921973in}{1.511980in}}{\pgfqpoint{0.921973in}{1.520217in}}%
\pgfpathcurveto{\pgfqpoint{0.921973in}{1.528453in}}{\pgfqpoint{0.918700in}{1.536353in}}{\pgfqpoint{0.912876in}{1.542177in}}%
\pgfpathcurveto{\pgfqpoint{0.907052in}{1.548001in}}{\pgfqpoint{0.899152in}{1.551273in}}{\pgfqpoint{0.890916in}{1.551273in}}%
\pgfpathcurveto{\pgfqpoint{0.882680in}{1.551273in}}{\pgfqpoint{0.874780in}{1.548001in}}{\pgfqpoint{0.868956in}{1.542177in}}%
\pgfpathcurveto{\pgfqpoint{0.863132in}{1.536353in}}{\pgfqpoint{0.859860in}{1.528453in}}{\pgfqpoint{0.859860in}{1.520217in}}%
\pgfpathcurveto{\pgfqpoint{0.859860in}{1.511980in}}{\pgfqpoint{0.863132in}{1.504080in}}{\pgfqpoint{0.868956in}{1.498256in}}%
\pgfpathcurveto{\pgfqpoint{0.874780in}{1.492433in}}{\pgfqpoint{0.882680in}{1.489160in}}{\pgfqpoint{0.890916in}{1.489160in}}%
\pgfpathclose%
\pgfusepath{stroke,fill}%
\end{pgfscope}%
\begin{pgfscope}%
\pgfpathrectangle{\pgfqpoint{0.100000in}{0.212622in}}{\pgfqpoint{3.696000in}{3.696000in}}%
\pgfusepath{clip}%
\pgfsetbuttcap%
\pgfsetroundjoin%
\definecolor{currentfill}{rgb}{0.121569,0.466667,0.705882}%
\pgfsetfillcolor{currentfill}%
\pgfsetfillopacity{0.585499}%
\pgfsetlinewidth{1.003750pt}%
\definecolor{currentstroke}{rgb}{0.121569,0.466667,0.705882}%
\pgfsetstrokecolor{currentstroke}%
\pgfsetstrokeopacity{0.585499}%
\pgfsetdash{}{0pt}%
\pgfpathmoveto{\pgfqpoint{0.890916in}{1.489160in}}%
\pgfpathcurveto{\pgfqpoint{0.899152in}{1.489160in}}{\pgfqpoint{0.907052in}{1.492433in}}{\pgfqpoint{0.912876in}{1.498256in}}%
\pgfpathcurveto{\pgfqpoint{0.918700in}{1.504080in}}{\pgfqpoint{0.921973in}{1.511980in}}{\pgfqpoint{0.921973in}{1.520217in}}%
\pgfpathcurveto{\pgfqpoint{0.921973in}{1.528453in}}{\pgfqpoint{0.918700in}{1.536353in}}{\pgfqpoint{0.912876in}{1.542177in}}%
\pgfpathcurveto{\pgfqpoint{0.907052in}{1.548001in}}{\pgfqpoint{0.899152in}{1.551273in}}{\pgfqpoint{0.890916in}{1.551273in}}%
\pgfpathcurveto{\pgfqpoint{0.882680in}{1.551273in}}{\pgfqpoint{0.874780in}{1.548001in}}{\pgfqpoint{0.868956in}{1.542177in}}%
\pgfpathcurveto{\pgfqpoint{0.863132in}{1.536353in}}{\pgfqpoint{0.859860in}{1.528453in}}{\pgfqpoint{0.859860in}{1.520217in}}%
\pgfpathcurveto{\pgfqpoint{0.859860in}{1.511980in}}{\pgfqpoint{0.863132in}{1.504080in}}{\pgfqpoint{0.868956in}{1.498256in}}%
\pgfpathcurveto{\pgfqpoint{0.874780in}{1.492433in}}{\pgfqpoint{0.882680in}{1.489160in}}{\pgfqpoint{0.890916in}{1.489160in}}%
\pgfpathclose%
\pgfusepath{stroke,fill}%
\end{pgfscope}%
\begin{pgfscope}%
\pgfpathrectangle{\pgfqpoint{0.100000in}{0.212622in}}{\pgfqpoint{3.696000in}{3.696000in}}%
\pgfusepath{clip}%
\pgfsetbuttcap%
\pgfsetroundjoin%
\definecolor{currentfill}{rgb}{0.121569,0.466667,0.705882}%
\pgfsetfillcolor{currentfill}%
\pgfsetfillopacity{0.585499}%
\pgfsetlinewidth{1.003750pt}%
\definecolor{currentstroke}{rgb}{0.121569,0.466667,0.705882}%
\pgfsetstrokecolor{currentstroke}%
\pgfsetstrokeopacity{0.585499}%
\pgfsetdash{}{0pt}%
\pgfpathmoveto{\pgfqpoint{0.890916in}{1.489160in}}%
\pgfpathcurveto{\pgfqpoint{0.899152in}{1.489160in}}{\pgfqpoint{0.907052in}{1.492433in}}{\pgfqpoint{0.912876in}{1.498256in}}%
\pgfpathcurveto{\pgfqpoint{0.918700in}{1.504080in}}{\pgfqpoint{0.921973in}{1.511980in}}{\pgfqpoint{0.921973in}{1.520217in}}%
\pgfpathcurveto{\pgfqpoint{0.921973in}{1.528453in}}{\pgfqpoint{0.918700in}{1.536353in}}{\pgfqpoint{0.912876in}{1.542177in}}%
\pgfpathcurveto{\pgfqpoint{0.907052in}{1.548001in}}{\pgfqpoint{0.899152in}{1.551273in}}{\pgfqpoint{0.890916in}{1.551273in}}%
\pgfpathcurveto{\pgfqpoint{0.882680in}{1.551273in}}{\pgfqpoint{0.874780in}{1.548001in}}{\pgfqpoint{0.868956in}{1.542177in}}%
\pgfpathcurveto{\pgfqpoint{0.863132in}{1.536353in}}{\pgfqpoint{0.859860in}{1.528453in}}{\pgfqpoint{0.859860in}{1.520217in}}%
\pgfpathcurveto{\pgfqpoint{0.859860in}{1.511980in}}{\pgfqpoint{0.863132in}{1.504080in}}{\pgfqpoint{0.868956in}{1.498256in}}%
\pgfpathcurveto{\pgfqpoint{0.874780in}{1.492433in}}{\pgfqpoint{0.882680in}{1.489160in}}{\pgfqpoint{0.890916in}{1.489160in}}%
\pgfpathclose%
\pgfusepath{stroke,fill}%
\end{pgfscope}%
\begin{pgfscope}%
\pgfpathrectangle{\pgfqpoint{0.100000in}{0.212622in}}{\pgfqpoint{3.696000in}{3.696000in}}%
\pgfusepath{clip}%
\pgfsetbuttcap%
\pgfsetroundjoin%
\definecolor{currentfill}{rgb}{0.121569,0.466667,0.705882}%
\pgfsetfillcolor{currentfill}%
\pgfsetfillopacity{0.585499}%
\pgfsetlinewidth{1.003750pt}%
\definecolor{currentstroke}{rgb}{0.121569,0.466667,0.705882}%
\pgfsetstrokecolor{currentstroke}%
\pgfsetstrokeopacity{0.585499}%
\pgfsetdash{}{0pt}%
\pgfpathmoveto{\pgfqpoint{0.890916in}{1.489160in}}%
\pgfpathcurveto{\pgfqpoint{0.899152in}{1.489160in}}{\pgfqpoint{0.907052in}{1.492433in}}{\pgfqpoint{0.912876in}{1.498256in}}%
\pgfpathcurveto{\pgfqpoint{0.918700in}{1.504080in}}{\pgfqpoint{0.921973in}{1.511980in}}{\pgfqpoint{0.921973in}{1.520217in}}%
\pgfpathcurveto{\pgfqpoint{0.921973in}{1.528453in}}{\pgfqpoint{0.918700in}{1.536353in}}{\pgfqpoint{0.912876in}{1.542177in}}%
\pgfpathcurveto{\pgfqpoint{0.907052in}{1.548001in}}{\pgfqpoint{0.899152in}{1.551273in}}{\pgfqpoint{0.890916in}{1.551273in}}%
\pgfpathcurveto{\pgfqpoint{0.882680in}{1.551273in}}{\pgfqpoint{0.874780in}{1.548001in}}{\pgfqpoint{0.868956in}{1.542177in}}%
\pgfpathcurveto{\pgfqpoint{0.863132in}{1.536353in}}{\pgfqpoint{0.859860in}{1.528453in}}{\pgfqpoint{0.859860in}{1.520217in}}%
\pgfpathcurveto{\pgfqpoint{0.859860in}{1.511980in}}{\pgfqpoint{0.863132in}{1.504080in}}{\pgfqpoint{0.868956in}{1.498256in}}%
\pgfpathcurveto{\pgfqpoint{0.874780in}{1.492433in}}{\pgfqpoint{0.882680in}{1.489160in}}{\pgfqpoint{0.890916in}{1.489160in}}%
\pgfpathclose%
\pgfusepath{stroke,fill}%
\end{pgfscope}%
\begin{pgfscope}%
\pgfpathrectangle{\pgfqpoint{0.100000in}{0.212622in}}{\pgfqpoint{3.696000in}{3.696000in}}%
\pgfusepath{clip}%
\pgfsetbuttcap%
\pgfsetroundjoin%
\definecolor{currentfill}{rgb}{0.121569,0.466667,0.705882}%
\pgfsetfillcolor{currentfill}%
\pgfsetfillopacity{0.585499}%
\pgfsetlinewidth{1.003750pt}%
\definecolor{currentstroke}{rgb}{0.121569,0.466667,0.705882}%
\pgfsetstrokecolor{currentstroke}%
\pgfsetstrokeopacity{0.585499}%
\pgfsetdash{}{0pt}%
\pgfpathmoveto{\pgfqpoint{0.890916in}{1.489160in}}%
\pgfpathcurveto{\pgfqpoint{0.899152in}{1.489160in}}{\pgfqpoint{0.907052in}{1.492433in}}{\pgfqpoint{0.912876in}{1.498256in}}%
\pgfpathcurveto{\pgfqpoint{0.918700in}{1.504080in}}{\pgfqpoint{0.921973in}{1.511980in}}{\pgfqpoint{0.921973in}{1.520217in}}%
\pgfpathcurveto{\pgfqpoint{0.921973in}{1.528453in}}{\pgfqpoint{0.918700in}{1.536353in}}{\pgfqpoint{0.912876in}{1.542177in}}%
\pgfpathcurveto{\pgfqpoint{0.907052in}{1.548001in}}{\pgfqpoint{0.899152in}{1.551273in}}{\pgfqpoint{0.890916in}{1.551273in}}%
\pgfpathcurveto{\pgfqpoint{0.882680in}{1.551273in}}{\pgfqpoint{0.874780in}{1.548001in}}{\pgfqpoint{0.868956in}{1.542177in}}%
\pgfpathcurveto{\pgfqpoint{0.863132in}{1.536353in}}{\pgfqpoint{0.859860in}{1.528453in}}{\pgfqpoint{0.859860in}{1.520217in}}%
\pgfpathcurveto{\pgfqpoint{0.859860in}{1.511980in}}{\pgfqpoint{0.863132in}{1.504080in}}{\pgfqpoint{0.868956in}{1.498256in}}%
\pgfpathcurveto{\pgfqpoint{0.874780in}{1.492433in}}{\pgfqpoint{0.882680in}{1.489160in}}{\pgfqpoint{0.890916in}{1.489160in}}%
\pgfpathclose%
\pgfusepath{stroke,fill}%
\end{pgfscope}%
\begin{pgfscope}%
\pgfpathrectangle{\pgfqpoint{0.100000in}{0.212622in}}{\pgfqpoint{3.696000in}{3.696000in}}%
\pgfusepath{clip}%
\pgfsetbuttcap%
\pgfsetroundjoin%
\definecolor{currentfill}{rgb}{0.121569,0.466667,0.705882}%
\pgfsetfillcolor{currentfill}%
\pgfsetfillopacity{0.585499}%
\pgfsetlinewidth{1.003750pt}%
\definecolor{currentstroke}{rgb}{0.121569,0.466667,0.705882}%
\pgfsetstrokecolor{currentstroke}%
\pgfsetstrokeopacity{0.585499}%
\pgfsetdash{}{0pt}%
\pgfpathmoveto{\pgfqpoint{0.890916in}{1.489160in}}%
\pgfpathcurveto{\pgfqpoint{0.899152in}{1.489160in}}{\pgfqpoint{0.907052in}{1.492433in}}{\pgfqpoint{0.912876in}{1.498256in}}%
\pgfpathcurveto{\pgfqpoint{0.918700in}{1.504080in}}{\pgfqpoint{0.921973in}{1.511980in}}{\pgfqpoint{0.921973in}{1.520217in}}%
\pgfpathcurveto{\pgfqpoint{0.921973in}{1.528453in}}{\pgfqpoint{0.918700in}{1.536353in}}{\pgfqpoint{0.912876in}{1.542177in}}%
\pgfpathcurveto{\pgfqpoint{0.907052in}{1.548001in}}{\pgfqpoint{0.899152in}{1.551273in}}{\pgfqpoint{0.890916in}{1.551273in}}%
\pgfpathcurveto{\pgfqpoint{0.882680in}{1.551273in}}{\pgfqpoint{0.874780in}{1.548001in}}{\pgfqpoint{0.868956in}{1.542177in}}%
\pgfpathcurveto{\pgfqpoint{0.863132in}{1.536353in}}{\pgfqpoint{0.859860in}{1.528453in}}{\pgfqpoint{0.859860in}{1.520217in}}%
\pgfpathcurveto{\pgfqpoint{0.859860in}{1.511980in}}{\pgfqpoint{0.863132in}{1.504080in}}{\pgfqpoint{0.868956in}{1.498256in}}%
\pgfpathcurveto{\pgfqpoint{0.874780in}{1.492433in}}{\pgfqpoint{0.882680in}{1.489160in}}{\pgfqpoint{0.890916in}{1.489160in}}%
\pgfpathclose%
\pgfusepath{stroke,fill}%
\end{pgfscope}%
\begin{pgfscope}%
\pgfpathrectangle{\pgfqpoint{0.100000in}{0.212622in}}{\pgfqpoint{3.696000in}{3.696000in}}%
\pgfusepath{clip}%
\pgfsetbuttcap%
\pgfsetroundjoin%
\definecolor{currentfill}{rgb}{0.121569,0.466667,0.705882}%
\pgfsetfillcolor{currentfill}%
\pgfsetfillopacity{0.585499}%
\pgfsetlinewidth{1.003750pt}%
\definecolor{currentstroke}{rgb}{0.121569,0.466667,0.705882}%
\pgfsetstrokecolor{currentstroke}%
\pgfsetstrokeopacity{0.585499}%
\pgfsetdash{}{0pt}%
\pgfpathmoveto{\pgfqpoint{0.890916in}{1.489160in}}%
\pgfpathcurveto{\pgfqpoint{0.899152in}{1.489160in}}{\pgfqpoint{0.907052in}{1.492433in}}{\pgfqpoint{0.912876in}{1.498256in}}%
\pgfpathcurveto{\pgfqpoint{0.918700in}{1.504080in}}{\pgfqpoint{0.921973in}{1.511980in}}{\pgfqpoint{0.921973in}{1.520217in}}%
\pgfpathcurveto{\pgfqpoint{0.921973in}{1.528453in}}{\pgfqpoint{0.918700in}{1.536353in}}{\pgfqpoint{0.912876in}{1.542177in}}%
\pgfpathcurveto{\pgfqpoint{0.907052in}{1.548001in}}{\pgfqpoint{0.899152in}{1.551273in}}{\pgfqpoint{0.890916in}{1.551273in}}%
\pgfpathcurveto{\pgfqpoint{0.882680in}{1.551273in}}{\pgfqpoint{0.874780in}{1.548001in}}{\pgfqpoint{0.868956in}{1.542177in}}%
\pgfpathcurveto{\pgfqpoint{0.863132in}{1.536353in}}{\pgfqpoint{0.859860in}{1.528453in}}{\pgfqpoint{0.859860in}{1.520217in}}%
\pgfpathcurveto{\pgfqpoint{0.859860in}{1.511980in}}{\pgfqpoint{0.863132in}{1.504080in}}{\pgfqpoint{0.868956in}{1.498256in}}%
\pgfpathcurveto{\pgfqpoint{0.874780in}{1.492433in}}{\pgfqpoint{0.882680in}{1.489160in}}{\pgfqpoint{0.890916in}{1.489160in}}%
\pgfpathclose%
\pgfusepath{stroke,fill}%
\end{pgfscope}%
\begin{pgfscope}%
\pgfpathrectangle{\pgfqpoint{0.100000in}{0.212622in}}{\pgfqpoint{3.696000in}{3.696000in}}%
\pgfusepath{clip}%
\pgfsetbuttcap%
\pgfsetroundjoin%
\definecolor{currentfill}{rgb}{0.121569,0.466667,0.705882}%
\pgfsetfillcolor{currentfill}%
\pgfsetfillopacity{0.585499}%
\pgfsetlinewidth{1.003750pt}%
\definecolor{currentstroke}{rgb}{0.121569,0.466667,0.705882}%
\pgfsetstrokecolor{currentstroke}%
\pgfsetstrokeopacity{0.585499}%
\pgfsetdash{}{0pt}%
\pgfpathmoveto{\pgfqpoint{0.890916in}{1.489160in}}%
\pgfpathcurveto{\pgfqpoint{0.899152in}{1.489160in}}{\pgfqpoint{0.907052in}{1.492433in}}{\pgfqpoint{0.912876in}{1.498256in}}%
\pgfpathcurveto{\pgfqpoint{0.918700in}{1.504080in}}{\pgfqpoint{0.921973in}{1.511980in}}{\pgfqpoint{0.921973in}{1.520217in}}%
\pgfpathcurveto{\pgfqpoint{0.921973in}{1.528453in}}{\pgfqpoint{0.918700in}{1.536353in}}{\pgfqpoint{0.912876in}{1.542177in}}%
\pgfpathcurveto{\pgfqpoint{0.907052in}{1.548001in}}{\pgfqpoint{0.899152in}{1.551273in}}{\pgfqpoint{0.890916in}{1.551273in}}%
\pgfpathcurveto{\pgfqpoint{0.882680in}{1.551273in}}{\pgfqpoint{0.874780in}{1.548001in}}{\pgfqpoint{0.868956in}{1.542177in}}%
\pgfpathcurveto{\pgfqpoint{0.863132in}{1.536353in}}{\pgfqpoint{0.859860in}{1.528453in}}{\pgfqpoint{0.859860in}{1.520217in}}%
\pgfpathcurveto{\pgfqpoint{0.859860in}{1.511980in}}{\pgfqpoint{0.863132in}{1.504080in}}{\pgfqpoint{0.868956in}{1.498256in}}%
\pgfpathcurveto{\pgfqpoint{0.874780in}{1.492433in}}{\pgfqpoint{0.882680in}{1.489160in}}{\pgfqpoint{0.890916in}{1.489160in}}%
\pgfpathclose%
\pgfusepath{stroke,fill}%
\end{pgfscope}%
\begin{pgfscope}%
\pgfpathrectangle{\pgfqpoint{0.100000in}{0.212622in}}{\pgfqpoint{3.696000in}{3.696000in}}%
\pgfusepath{clip}%
\pgfsetbuttcap%
\pgfsetroundjoin%
\definecolor{currentfill}{rgb}{0.121569,0.466667,0.705882}%
\pgfsetfillcolor{currentfill}%
\pgfsetfillopacity{0.585499}%
\pgfsetlinewidth{1.003750pt}%
\definecolor{currentstroke}{rgb}{0.121569,0.466667,0.705882}%
\pgfsetstrokecolor{currentstroke}%
\pgfsetstrokeopacity{0.585499}%
\pgfsetdash{}{0pt}%
\pgfpathmoveto{\pgfqpoint{0.890916in}{1.489160in}}%
\pgfpathcurveto{\pgfqpoint{0.899152in}{1.489160in}}{\pgfqpoint{0.907052in}{1.492433in}}{\pgfqpoint{0.912876in}{1.498256in}}%
\pgfpathcurveto{\pgfqpoint{0.918700in}{1.504080in}}{\pgfqpoint{0.921973in}{1.511980in}}{\pgfqpoint{0.921973in}{1.520217in}}%
\pgfpathcurveto{\pgfqpoint{0.921973in}{1.528453in}}{\pgfqpoint{0.918700in}{1.536353in}}{\pgfqpoint{0.912876in}{1.542177in}}%
\pgfpathcurveto{\pgfqpoint{0.907052in}{1.548001in}}{\pgfqpoint{0.899152in}{1.551273in}}{\pgfqpoint{0.890916in}{1.551273in}}%
\pgfpathcurveto{\pgfqpoint{0.882680in}{1.551273in}}{\pgfqpoint{0.874780in}{1.548001in}}{\pgfqpoint{0.868956in}{1.542177in}}%
\pgfpathcurveto{\pgfqpoint{0.863132in}{1.536353in}}{\pgfqpoint{0.859860in}{1.528453in}}{\pgfqpoint{0.859860in}{1.520217in}}%
\pgfpathcurveto{\pgfqpoint{0.859860in}{1.511980in}}{\pgfqpoint{0.863132in}{1.504080in}}{\pgfqpoint{0.868956in}{1.498256in}}%
\pgfpathcurveto{\pgfqpoint{0.874780in}{1.492433in}}{\pgfqpoint{0.882680in}{1.489160in}}{\pgfqpoint{0.890916in}{1.489160in}}%
\pgfpathclose%
\pgfusepath{stroke,fill}%
\end{pgfscope}%
\begin{pgfscope}%
\pgfpathrectangle{\pgfqpoint{0.100000in}{0.212622in}}{\pgfqpoint{3.696000in}{3.696000in}}%
\pgfusepath{clip}%
\pgfsetbuttcap%
\pgfsetroundjoin%
\definecolor{currentfill}{rgb}{0.121569,0.466667,0.705882}%
\pgfsetfillcolor{currentfill}%
\pgfsetfillopacity{0.585499}%
\pgfsetlinewidth{1.003750pt}%
\definecolor{currentstroke}{rgb}{0.121569,0.466667,0.705882}%
\pgfsetstrokecolor{currentstroke}%
\pgfsetstrokeopacity{0.585499}%
\pgfsetdash{}{0pt}%
\pgfpathmoveto{\pgfqpoint{0.890916in}{1.489160in}}%
\pgfpathcurveto{\pgfqpoint{0.899152in}{1.489160in}}{\pgfqpoint{0.907052in}{1.492433in}}{\pgfqpoint{0.912876in}{1.498256in}}%
\pgfpathcurveto{\pgfqpoint{0.918700in}{1.504080in}}{\pgfqpoint{0.921973in}{1.511980in}}{\pgfqpoint{0.921973in}{1.520217in}}%
\pgfpathcurveto{\pgfqpoint{0.921973in}{1.528453in}}{\pgfqpoint{0.918700in}{1.536353in}}{\pgfqpoint{0.912876in}{1.542177in}}%
\pgfpathcurveto{\pgfqpoint{0.907052in}{1.548001in}}{\pgfqpoint{0.899152in}{1.551273in}}{\pgfqpoint{0.890916in}{1.551273in}}%
\pgfpathcurveto{\pgfqpoint{0.882680in}{1.551273in}}{\pgfqpoint{0.874780in}{1.548001in}}{\pgfqpoint{0.868956in}{1.542177in}}%
\pgfpathcurveto{\pgfqpoint{0.863132in}{1.536353in}}{\pgfqpoint{0.859860in}{1.528453in}}{\pgfqpoint{0.859860in}{1.520217in}}%
\pgfpathcurveto{\pgfqpoint{0.859860in}{1.511980in}}{\pgfqpoint{0.863132in}{1.504080in}}{\pgfqpoint{0.868956in}{1.498256in}}%
\pgfpathcurveto{\pgfqpoint{0.874780in}{1.492433in}}{\pgfqpoint{0.882680in}{1.489160in}}{\pgfqpoint{0.890916in}{1.489160in}}%
\pgfpathclose%
\pgfusepath{stroke,fill}%
\end{pgfscope}%
\begin{pgfscope}%
\pgfpathrectangle{\pgfqpoint{0.100000in}{0.212622in}}{\pgfqpoint{3.696000in}{3.696000in}}%
\pgfusepath{clip}%
\pgfsetbuttcap%
\pgfsetroundjoin%
\definecolor{currentfill}{rgb}{0.121569,0.466667,0.705882}%
\pgfsetfillcolor{currentfill}%
\pgfsetfillopacity{0.585549}%
\pgfsetlinewidth{1.003750pt}%
\definecolor{currentstroke}{rgb}{0.121569,0.466667,0.705882}%
\pgfsetstrokecolor{currentstroke}%
\pgfsetstrokeopacity{0.585549}%
\pgfsetdash{}{0pt}%
\pgfpathmoveto{\pgfqpoint{2.091715in}{2.152751in}}%
\pgfpathcurveto{\pgfqpoint{2.099951in}{2.152751in}}{\pgfqpoint{2.107851in}{2.156023in}}{\pgfqpoint{2.113675in}{2.161847in}}%
\pgfpathcurveto{\pgfqpoint{2.119499in}{2.167671in}}{\pgfqpoint{2.122771in}{2.175571in}}{\pgfqpoint{2.122771in}{2.183807in}}%
\pgfpathcurveto{\pgfqpoint{2.122771in}{2.192044in}}{\pgfqpoint{2.119499in}{2.199944in}}{\pgfqpoint{2.113675in}{2.205768in}}%
\pgfpathcurveto{\pgfqpoint{2.107851in}{2.211592in}}{\pgfqpoint{2.099951in}{2.214864in}}{\pgfqpoint{2.091715in}{2.214864in}}%
\pgfpathcurveto{\pgfqpoint{2.083478in}{2.214864in}}{\pgfqpoint{2.075578in}{2.211592in}}{\pgfqpoint{2.069754in}{2.205768in}}%
\pgfpathcurveto{\pgfqpoint{2.063931in}{2.199944in}}{\pgfqpoint{2.060658in}{2.192044in}}{\pgfqpoint{2.060658in}{2.183807in}}%
\pgfpathcurveto{\pgfqpoint{2.060658in}{2.175571in}}{\pgfqpoint{2.063931in}{2.167671in}}{\pgfqpoint{2.069754in}{2.161847in}}%
\pgfpathcurveto{\pgfqpoint{2.075578in}{2.156023in}}{\pgfqpoint{2.083478in}{2.152751in}}{\pgfqpoint{2.091715in}{2.152751in}}%
\pgfpathclose%
\pgfusepath{stroke,fill}%
\end{pgfscope}%
\begin{pgfscope}%
\pgfpathrectangle{\pgfqpoint{0.100000in}{0.212622in}}{\pgfqpoint{3.696000in}{3.696000in}}%
\pgfusepath{clip}%
\pgfsetbuttcap%
\pgfsetroundjoin%
\definecolor{currentfill}{rgb}{0.121569,0.466667,0.705882}%
\pgfsetfillcolor{currentfill}%
\pgfsetfillopacity{0.585550}%
\pgfsetlinewidth{1.003750pt}%
\definecolor{currentstroke}{rgb}{0.121569,0.466667,0.705882}%
\pgfsetstrokecolor{currentstroke}%
\pgfsetstrokeopacity{0.585550}%
\pgfsetdash{}{0pt}%
\pgfpathmoveto{\pgfqpoint{0.890769in}{1.488751in}}%
\pgfpathcurveto{\pgfqpoint{0.899005in}{1.488751in}}{\pgfqpoint{0.906906in}{1.492024in}}{\pgfqpoint{0.912729in}{1.497848in}}%
\pgfpathcurveto{\pgfqpoint{0.918553in}{1.503672in}}{\pgfqpoint{0.921826in}{1.511572in}}{\pgfqpoint{0.921826in}{1.519808in}}%
\pgfpathcurveto{\pgfqpoint{0.921826in}{1.528044in}}{\pgfqpoint{0.918553in}{1.535944in}}{\pgfqpoint{0.912729in}{1.541768in}}%
\pgfpathcurveto{\pgfqpoint{0.906906in}{1.547592in}}{\pgfqpoint{0.899005in}{1.550864in}}{\pgfqpoint{0.890769in}{1.550864in}}%
\pgfpathcurveto{\pgfqpoint{0.882533in}{1.550864in}}{\pgfqpoint{0.874633in}{1.547592in}}{\pgfqpoint{0.868809in}{1.541768in}}%
\pgfpathcurveto{\pgfqpoint{0.862985in}{1.535944in}}{\pgfqpoint{0.859713in}{1.528044in}}{\pgfqpoint{0.859713in}{1.519808in}}%
\pgfpathcurveto{\pgfqpoint{0.859713in}{1.511572in}}{\pgfqpoint{0.862985in}{1.503672in}}{\pgfqpoint{0.868809in}{1.497848in}}%
\pgfpathcurveto{\pgfqpoint{0.874633in}{1.492024in}}{\pgfqpoint{0.882533in}{1.488751in}}{\pgfqpoint{0.890769in}{1.488751in}}%
\pgfpathclose%
\pgfusepath{stroke,fill}%
\end{pgfscope}%
\begin{pgfscope}%
\pgfpathrectangle{\pgfqpoint{0.100000in}{0.212622in}}{\pgfqpoint{3.696000in}{3.696000in}}%
\pgfusepath{clip}%
\pgfsetbuttcap%
\pgfsetroundjoin%
\definecolor{currentfill}{rgb}{0.121569,0.466667,0.705882}%
\pgfsetfillcolor{currentfill}%
\pgfsetfillopacity{0.585577}%
\pgfsetlinewidth{1.003750pt}%
\definecolor{currentstroke}{rgb}{0.121569,0.466667,0.705882}%
\pgfsetstrokecolor{currentstroke}%
\pgfsetstrokeopacity{0.585577}%
\pgfsetdash{}{0pt}%
\pgfpathmoveto{\pgfqpoint{0.890685in}{1.488527in}}%
\pgfpathcurveto{\pgfqpoint{0.898921in}{1.488527in}}{\pgfqpoint{0.906821in}{1.491800in}}{\pgfqpoint{0.912645in}{1.497624in}}%
\pgfpathcurveto{\pgfqpoint{0.918469in}{1.503447in}}{\pgfqpoint{0.921742in}{1.511347in}}{\pgfqpoint{0.921742in}{1.519584in}}%
\pgfpathcurveto{\pgfqpoint{0.921742in}{1.527820in}}{\pgfqpoint{0.918469in}{1.535720in}}{\pgfqpoint{0.912645in}{1.541544in}}%
\pgfpathcurveto{\pgfqpoint{0.906821in}{1.547368in}}{\pgfqpoint{0.898921in}{1.550640in}}{\pgfqpoint{0.890685in}{1.550640in}}%
\pgfpathcurveto{\pgfqpoint{0.882449in}{1.550640in}}{\pgfqpoint{0.874549in}{1.547368in}}{\pgfqpoint{0.868725in}{1.541544in}}%
\pgfpathcurveto{\pgfqpoint{0.862901in}{1.535720in}}{\pgfqpoint{0.859629in}{1.527820in}}{\pgfqpoint{0.859629in}{1.519584in}}%
\pgfpathcurveto{\pgfqpoint{0.859629in}{1.511347in}}{\pgfqpoint{0.862901in}{1.503447in}}{\pgfqpoint{0.868725in}{1.497624in}}%
\pgfpathcurveto{\pgfqpoint{0.874549in}{1.491800in}}{\pgfqpoint{0.882449in}{1.488527in}}{\pgfqpoint{0.890685in}{1.488527in}}%
\pgfpathclose%
\pgfusepath{stroke,fill}%
\end{pgfscope}%
\begin{pgfscope}%
\pgfpathrectangle{\pgfqpoint{0.100000in}{0.212622in}}{\pgfqpoint{3.696000in}{3.696000in}}%
\pgfusepath{clip}%
\pgfsetbuttcap%
\pgfsetroundjoin%
\definecolor{currentfill}{rgb}{0.121569,0.466667,0.705882}%
\pgfsetfillcolor{currentfill}%
\pgfsetfillopacity{0.585850}%
\pgfsetlinewidth{1.003750pt}%
\definecolor{currentstroke}{rgb}{0.121569,0.466667,0.705882}%
\pgfsetstrokecolor{currentstroke}%
\pgfsetstrokeopacity{0.585850}%
\pgfsetdash{}{0pt}%
\pgfpathmoveto{\pgfqpoint{0.889838in}{1.486929in}}%
\pgfpathcurveto{\pgfqpoint{0.898074in}{1.486929in}}{\pgfqpoint{0.905974in}{1.490201in}}{\pgfqpoint{0.911798in}{1.496025in}}%
\pgfpathcurveto{\pgfqpoint{0.917622in}{1.501849in}}{\pgfqpoint{0.920895in}{1.509749in}}{\pgfqpoint{0.920895in}{1.517985in}}%
\pgfpathcurveto{\pgfqpoint{0.920895in}{1.526222in}}{\pgfqpoint{0.917622in}{1.534122in}}{\pgfqpoint{0.911798in}{1.539946in}}%
\pgfpathcurveto{\pgfqpoint{0.905974in}{1.545770in}}{\pgfqpoint{0.898074in}{1.549042in}}{\pgfqpoint{0.889838in}{1.549042in}}%
\pgfpathcurveto{\pgfqpoint{0.881602in}{1.549042in}}{\pgfqpoint{0.873702in}{1.545770in}}{\pgfqpoint{0.867878in}{1.539946in}}%
\pgfpathcurveto{\pgfqpoint{0.862054in}{1.534122in}}{\pgfqpoint{0.858782in}{1.526222in}}{\pgfqpoint{0.858782in}{1.517985in}}%
\pgfpathcurveto{\pgfqpoint{0.858782in}{1.509749in}}{\pgfqpoint{0.862054in}{1.501849in}}{\pgfqpoint{0.867878in}{1.496025in}}%
\pgfpathcurveto{\pgfqpoint{0.873702in}{1.490201in}}{\pgfqpoint{0.881602in}{1.486929in}}{\pgfqpoint{0.889838in}{1.486929in}}%
\pgfpathclose%
\pgfusepath{stroke,fill}%
\end{pgfscope}%
\begin{pgfscope}%
\pgfpathrectangle{\pgfqpoint{0.100000in}{0.212622in}}{\pgfqpoint{3.696000in}{3.696000in}}%
\pgfusepath{clip}%
\pgfsetbuttcap%
\pgfsetroundjoin%
\definecolor{currentfill}{rgb}{0.121569,0.466667,0.705882}%
\pgfsetfillcolor{currentfill}%
\pgfsetfillopacity{0.586013}%
\pgfsetlinewidth{1.003750pt}%
\definecolor{currentstroke}{rgb}{0.121569,0.466667,0.705882}%
\pgfsetstrokecolor{currentstroke}%
\pgfsetstrokeopacity{0.586013}%
\pgfsetdash{}{0pt}%
\pgfpathmoveto{\pgfqpoint{0.889350in}{1.486125in}}%
\pgfpathcurveto{\pgfqpoint{0.897587in}{1.486125in}}{\pgfqpoint{0.905487in}{1.489398in}}{\pgfqpoint{0.911311in}{1.495222in}}%
\pgfpathcurveto{\pgfqpoint{0.917135in}{1.501046in}}{\pgfqpoint{0.920407in}{1.508946in}}{\pgfqpoint{0.920407in}{1.517182in}}%
\pgfpathcurveto{\pgfqpoint{0.920407in}{1.525418in}}{\pgfqpoint{0.917135in}{1.533318in}}{\pgfqpoint{0.911311in}{1.539142in}}%
\pgfpathcurveto{\pgfqpoint{0.905487in}{1.544966in}}{\pgfqpoint{0.897587in}{1.548238in}}{\pgfqpoint{0.889350in}{1.548238in}}%
\pgfpathcurveto{\pgfqpoint{0.881114in}{1.548238in}}{\pgfqpoint{0.873214in}{1.544966in}}{\pgfqpoint{0.867390in}{1.539142in}}%
\pgfpathcurveto{\pgfqpoint{0.861566in}{1.533318in}}{\pgfqpoint{0.858294in}{1.525418in}}{\pgfqpoint{0.858294in}{1.517182in}}%
\pgfpathcurveto{\pgfqpoint{0.858294in}{1.508946in}}{\pgfqpoint{0.861566in}{1.501046in}}{\pgfqpoint{0.867390in}{1.495222in}}%
\pgfpathcurveto{\pgfqpoint{0.873214in}{1.489398in}}{\pgfqpoint{0.881114in}{1.486125in}}{\pgfqpoint{0.889350in}{1.486125in}}%
\pgfpathclose%
\pgfusepath{stroke,fill}%
\end{pgfscope}%
\begin{pgfscope}%
\pgfpathrectangle{\pgfqpoint{0.100000in}{0.212622in}}{\pgfqpoint{3.696000in}{3.696000in}}%
\pgfusepath{clip}%
\pgfsetbuttcap%
\pgfsetroundjoin%
\definecolor{currentfill}{rgb}{0.121569,0.466667,0.705882}%
\pgfsetfillcolor{currentfill}%
\pgfsetfillopacity{0.586119}%
\pgfsetlinewidth{1.003750pt}%
\definecolor{currentstroke}{rgb}{0.121569,0.466667,0.705882}%
\pgfsetstrokecolor{currentstroke}%
\pgfsetstrokeopacity{0.586119}%
\pgfsetdash{}{0pt}%
\pgfpathmoveto{\pgfqpoint{0.889051in}{1.485793in}}%
\pgfpathcurveto{\pgfqpoint{0.897287in}{1.485793in}}{\pgfqpoint{0.905187in}{1.489065in}}{\pgfqpoint{0.911011in}{1.494889in}}%
\pgfpathcurveto{\pgfqpoint{0.916835in}{1.500713in}}{\pgfqpoint{0.920107in}{1.508613in}}{\pgfqpoint{0.920107in}{1.516849in}}%
\pgfpathcurveto{\pgfqpoint{0.920107in}{1.525085in}}{\pgfqpoint{0.916835in}{1.532985in}}{\pgfqpoint{0.911011in}{1.538809in}}%
\pgfpathcurveto{\pgfqpoint{0.905187in}{1.544633in}}{\pgfqpoint{0.897287in}{1.547906in}}{\pgfqpoint{0.889051in}{1.547906in}}%
\pgfpathcurveto{\pgfqpoint{0.880815in}{1.547906in}}{\pgfqpoint{0.872914in}{1.544633in}}{\pgfqpoint{0.867091in}{1.538809in}}%
\pgfpathcurveto{\pgfqpoint{0.861267in}{1.532985in}}{\pgfqpoint{0.857994in}{1.525085in}}{\pgfqpoint{0.857994in}{1.516849in}}%
\pgfpathcurveto{\pgfqpoint{0.857994in}{1.508613in}}{\pgfqpoint{0.861267in}{1.500713in}}{\pgfqpoint{0.867091in}{1.494889in}}%
\pgfpathcurveto{\pgfqpoint{0.872914in}{1.489065in}}{\pgfqpoint{0.880815in}{1.485793in}}{\pgfqpoint{0.889051in}{1.485793in}}%
\pgfpathclose%
\pgfusepath{stroke,fill}%
\end{pgfscope}%
\begin{pgfscope}%
\pgfpathrectangle{\pgfqpoint{0.100000in}{0.212622in}}{\pgfqpoint{3.696000in}{3.696000in}}%
\pgfusepath{clip}%
\pgfsetbuttcap%
\pgfsetroundjoin%
\definecolor{currentfill}{rgb}{0.121569,0.466667,0.705882}%
\pgfsetfillcolor{currentfill}%
\pgfsetfillopacity{0.586196}%
\pgfsetlinewidth{1.003750pt}%
\definecolor{currentstroke}{rgb}{0.121569,0.466667,0.705882}%
\pgfsetstrokecolor{currentstroke}%
\pgfsetstrokeopacity{0.586196}%
\pgfsetdash{}{0pt}%
\pgfpathmoveto{\pgfqpoint{0.888901in}{1.485649in}}%
\pgfpathcurveto{\pgfqpoint{0.897137in}{1.485649in}}{\pgfqpoint{0.905038in}{1.488922in}}{\pgfqpoint{0.910861in}{1.494746in}}%
\pgfpathcurveto{\pgfqpoint{0.916685in}{1.500570in}}{\pgfqpoint{0.919958in}{1.508470in}}{\pgfqpoint{0.919958in}{1.516706in}}%
\pgfpathcurveto{\pgfqpoint{0.919958in}{1.524942in}}{\pgfqpoint{0.916685in}{1.532842in}}{\pgfqpoint{0.910861in}{1.538666in}}%
\pgfpathcurveto{\pgfqpoint{0.905038in}{1.544490in}}{\pgfqpoint{0.897137in}{1.547762in}}{\pgfqpoint{0.888901in}{1.547762in}}%
\pgfpathcurveto{\pgfqpoint{0.880665in}{1.547762in}}{\pgfqpoint{0.872765in}{1.544490in}}{\pgfqpoint{0.866941in}{1.538666in}}%
\pgfpathcurveto{\pgfqpoint{0.861117in}{1.532842in}}{\pgfqpoint{0.857845in}{1.524942in}}{\pgfqpoint{0.857845in}{1.516706in}}%
\pgfpathcurveto{\pgfqpoint{0.857845in}{1.508470in}}{\pgfqpoint{0.861117in}{1.500570in}}{\pgfqpoint{0.866941in}{1.494746in}}%
\pgfpathcurveto{\pgfqpoint{0.872765in}{1.488922in}}{\pgfqpoint{0.880665in}{1.485649in}}{\pgfqpoint{0.888901in}{1.485649in}}%
\pgfpathclose%
\pgfusepath{stroke,fill}%
\end{pgfscope}%
\begin{pgfscope}%
\pgfpathrectangle{\pgfqpoint{0.100000in}{0.212622in}}{\pgfqpoint{3.696000in}{3.696000in}}%
\pgfusepath{clip}%
\pgfsetbuttcap%
\pgfsetroundjoin%
\definecolor{currentfill}{rgb}{0.121569,0.466667,0.705882}%
\pgfsetfillcolor{currentfill}%
\pgfsetfillopacity{0.586462}%
\pgfsetlinewidth{1.003750pt}%
\definecolor{currentstroke}{rgb}{0.121569,0.466667,0.705882}%
\pgfsetstrokecolor{currentstroke}%
\pgfsetstrokeopacity{0.586462}%
\pgfsetdash{}{0pt}%
\pgfpathmoveto{\pgfqpoint{0.888437in}{1.485303in}}%
\pgfpathcurveto{\pgfqpoint{0.896673in}{1.485303in}}{\pgfqpoint{0.904573in}{1.488575in}}{\pgfqpoint{0.910397in}{1.494399in}}%
\pgfpathcurveto{\pgfqpoint{0.916221in}{1.500223in}}{\pgfqpoint{0.919494in}{1.508123in}}{\pgfqpoint{0.919494in}{1.516359in}}%
\pgfpathcurveto{\pgfqpoint{0.919494in}{1.524596in}}{\pgfqpoint{0.916221in}{1.532496in}}{\pgfqpoint{0.910397in}{1.538320in}}%
\pgfpathcurveto{\pgfqpoint{0.904573in}{1.544144in}}{\pgfqpoint{0.896673in}{1.547416in}}{\pgfqpoint{0.888437in}{1.547416in}}%
\pgfpathcurveto{\pgfqpoint{0.880201in}{1.547416in}}{\pgfqpoint{0.872301in}{1.544144in}}{\pgfqpoint{0.866477in}{1.538320in}}%
\pgfpathcurveto{\pgfqpoint{0.860653in}{1.532496in}}{\pgfqpoint{0.857381in}{1.524596in}}{\pgfqpoint{0.857381in}{1.516359in}}%
\pgfpathcurveto{\pgfqpoint{0.857381in}{1.508123in}}{\pgfqpoint{0.860653in}{1.500223in}}{\pgfqpoint{0.866477in}{1.494399in}}%
\pgfpathcurveto{\pgfqpoint{0.872301in}{1.488575in}}{\pgfqpoint{0.880201in}{1.485303in}}{\pgfqpoint{0.888437in}{1.485303in}}%
\pgfpathclose%
\pgfusepath{stroke,fill}%
\end{pgfscope}%
\begin{pgfscope}%
\pgfpathrectangle{\pgfqpoint{0.100000in}{0.212622in}}{\pgfqpoint{3.696000in}{3.696000in}}%
\pgfusepath{clip}%
\pgfsetbuttcap%
\pgfsetroundjoin%
\definecolor{currentfill}{rgb}{0.121569,0.466667,0.705882}%
\pgfsetfillcolor{currentfill}%
\pgfsetfillopacity{0.586653}%
\pgfsetlinewidth{1.003750pt}%
\definecolor{currentstroke}{rgb}{0.121569,0.466667,0.705882}%
\pgfsetstrokecolor{currentstroke}%
\pgfsetstrokeopacity{0.586653}%
\pgfsetdash{}{0pt}%
\pgfpathmoveto{\pgfqpoint{0.888199in}{1.485247in}}%
\pgfpathcurveto{\pgfqpoint{0.896436in}{1.485247in}}{\pgfqpoint{0.904336in}{1.488520in}}{\pgfqpoint{0.910160in}{1.494344in}}%
\pgfpathcurveto{\pgfqpoint{0.915984in}{1.500168in}}{\pgfqpoint{0.919256in}{1.508068in}}{\pgfqpoint{0.919256in}{1.516304in}}%
\pgfpathcurveto{\pgfqpoint{0.919256in}{1.524540in}}{\pgfqpoint{0.915984in}{1.532440in}}{\pgfqpoint{0.910160in}{1.538264in}}%
\pgfpathcurveto{\pgfqpoint{0.904336in}{1.544088in}}{\pgfqpoint{0.896436in}{1.547360in}}{\pgfqpoint{0.888199in}{1.547360in}}%
\pgfpathcurveto{\pgfqpoint{0.879963in}{1.547360in}}{\pgfqpoint{0.872063in}{1.544088in}}{\pgfqpoint{0.866239in}{1.538264in}}%
\pgfpathcurveto{\pgfqpoint{0.860415in}{1.532440in}}{\pgfqpoint{0.857143in}{1.524540in}}{\pgfqpoint{0.857143in}{1.516304in}}%
\pgfpathcurveto{\pgfqpoint{0.857143in}{1.508068in}}{\pgfqpoint{0.860415in}{1.500168in}}{\pgfqpoint{0.866239in}{1.494344in}}%
\pgfpathcurveto{\pgfqpoint{0.872063in}{1.488520in}}{\pgfqpoint{0.879963in}{1.485247in}}{\pgfqpoint{0.888199in}{1.485247in}}%
\pgfpathclose%
\pgfusepath{stroke,fill}%
\end{pgfscope}%
\begin{pgfscope}%
\pgfpathrectangle{\pgfqpoint{0.100000in}{0.212622in}}{\pgfqpoint{3.696000in}{3.696000in}}%
\pgfusepath{clip}%
\pgfsetbuttcap%
\pgfsetroundjoin%
\definecolor{currentfill}{rgb}{0.121569,0.466667,0.705882}%
\pgfsetfillcolor{currentfill}%
\pgfsetfillopacity{0.586797}%
\pgfsetlinewidth{1.003750pt}%
\definecolor{currentstroke}{rgb}{0.121569,0.466667,0.705882}%
\pgfsetstrokecolor{currentstroke}%
\pgfsetstrokeopacity{0.586797}%
\pgfsetdash{}{0pt}%
\pgfpathmoveto{\pgfqpoint{2.092908in}{2.146570in}}%
\pgfpathcurveto{\pgfqpoint{2.101144in}{2.146570in}}{\pgfqpoint{2.109044in}{2.149843in}}{\pgfqpoint{2.114868in}{2.155667in}}%
\pgfpathcurveto{\pgfqpoint{2.120692in}{2.161491in}}{\pgfqpoint{2.123964in}{2.169391in}}{\pgfqpoint{2.123964in}{2.177627in}}%
\pgfpathcurveto{\pgfqpoint{2.123964in}{2.185863in}}{\pgfqpoint{2.120692in}{2.193763in}}{\pgfqpoint{2.114868in}{2.199587in}}%
\pgfpathcurveto{\pgfqpoint{2.109044in}{2.205411in}}{\pgfqpoint{2.101144in}{2.208683in}}{\pgfqpoint{2.092908in}{2.208683in}}%
\pgfpathcurveto{\pgfqpoint{2.084671in}{2.208683in}}{\pgfqpoint{2.076771in}{2.205411in}}{\pgfqpoint{2.070947in}{2.199587in}}%
\pgfpathcurveto{\pgfqpoint{2.065123in}{2.193763in}}{\pgfqpoint{2.061851in}{2.185863in}}{\pgfqpoint{2.061851in}{2.177627in}}%
\pgfpathcurveto{\pgfqpoint{2.061851in}{2.169391in}}{\pgfqpoint{2.065123in}{2.161491in}}{\pgfqpoint{2.070947in}{2.155667in}}%
\pgfpathcurveto{\pgfqpoint{2.076771in}{2.149843in}}{\pgfqpoint{2.084671in}{2.146570in}}{\pgfqpoint{2.092908in}{2.146570in}}%
\pgfpathclose%
\pgfusepath{stroke,fill}%
\end{pgfscope}%
\begin{pgfscope}%
\pgfpathrectangle{\pgfqpoint{0.100000in}{0.212622in}}{\pgfqpoint{3.696000in}{3.696000in}}%
\pgfusepath{clip}%
\pgfsetbuttcap%
\pgfsetroundjoin%
\definecolor{currentfill}{rgb}{0.121569,0.466667,0.705882}%
\pgfsetfillcolor{currentfill}%
\pgfsetfillopacity{0.587027}%
\pgfsetlinewidth{1.003750pt}%
\definecolor{currentstroke}{rgb}{0.121569,0.466667,0.705882}%
\pgfsetstrokecolor{currentstroke}%
\pgfsetstrokeopacity{0.587027}%
\pgfsetdash{}{0pt}%
\pgfpathmoveto{\pgfqpoint{0.585743in}{1.220593in}}%
\pgfpathcurveto{\pgfqpoint{0.593979in}{1.220593in}}{\pgfqpoint{0.601879in}{1.223866in}}{\pgfqpoint{0.607703in}{1.229690in}}%
\pgfpathcurveto{\pgfqpoint{0.613527in}{1.235514in}}{\pgfqpoint{0.616799in}{1.243414in}}{\pgfqpoint{0.616799in}{1.251650in}}%
\pgfpathcurveto{\pgfqpoint{0.616799in}{1.259886in}}{\pgfqpoint{0.613527in}{1.267786in}}{\pgfqpoint{0.607703in}{1.273610in}}%
\pgfpathcurveto{\pgfqpoint{0.601879in}{1.279434in}}{\pgfqpoint{0.593979in}{1.282706in}}{\pgfqpoint{0.585743in}{1.282706in}}%
\pgfpathcurveto{\pgfqpoint{0.577507in}{1.282706in}}{\pgfqpoint{0.569606in}{1.279434in}}{\pgfqpoint{0.563783in}{1.273610in}}%
\pgfpathcurveto{\pgfqpoint{0.557959in}{1.267786in}}{\pgfqpoint{0.554686in}{1.259886in}}{\pgfqpoint{0.554686in}{1.251650in}}%
\pgfpathcurveto{\pgfqpoint{0.554686in}{1.243414in}}{\pgfqpoint{0.557959in}{1.235514in}}{\pgfqpoint{0.563783in}{1.229690in}}%
\pgfpathcurveto{\pgfqpoint{0.569606in}{1.223866in}}{\pgfqpoint{0.577507in}{1.220593in}}{\pgfqpoint{0.585743in}{1.220593in}}%
\pgfpathclose%
\pgfusepath{stroke,fill}%
\end{pgfscope}%
\begin{pgfscope}%
\pgfpathrectangle{\pgfqpoint{0.100000in}{0.212622in}}{\pgfqpoint{3.696000in}{3.696000in}}%
\pgfusepath{clip}%
\pgfsetbuttcap%
\pgfsetroundjoin%
\definecolor{currentfill}{rgb}{0.121569,0.466667,0.705882}%
\pgfsetfillcolor{currentfill}%
\pgfsetfillopacity{0.587440}%
\pgfsetlinewidth{1.003750pt}%
\definecolor{currentstroke}{rgb}{0.121569,0.466667,0.705882}%
\pgfsetstrokecolor{currentstroke}%
\pgfsetstrokeopacity{0.587440}%
\pgfsetdash{}{0pt}%
\pgfpathmoveto{\pgfqpoint{0.887224in}{1.485086in}}%
\pgfpathcurveto{\pgfqpoint{0.895460in}{1.485086in}}{\pgfqpoint{0.903360in}{1.488358in}}{\pgfqpoint{0.909184in}{1.494182in}}%
\pgfpathcurveto{\pgfqpoint{0.915008in}{1.500006in}}{\pgfqpoint{0.918280in}{1.507906in}}{\pgfqpoint{0.918280in}{1.516143in}}%
\pgfpathcurveto{\pgfqpoint{0.918280in}{1.524379in}}{\pgfqpoint{0.915008in}{1.532279in}}{\pgfqpoint{0.909184in}{1.538103in}}%
\pgfpathcurveto{\pgfqpoint{0.903360in}{1.543927in}}{\pgfqpoint{0.895460in}{1.547199in}}{\pgfqpoint{0.887224in}{1.547199in}}%
\pgfpathcurveto{\pgfqpoint{0.878987in}{1.547199in}}{\pgfqpoint{0.871087in}{1.543927in}}{\pgfqpoint{0.865263in}{1.538103in}}%
\pgfpathcurveto{\pgfqpoint{0.859439in}{1.532279in}}{\pgfqpoint{0.856167in}{1.524379in}}{\pgfqpoint{0.856167in}{1.516143in}}%
\pgfpathcurveto{\pgfqpoint{0.856167in}{1.507906in}}{\pgfqpoint{0.859439in}{1.500006in}}{\pgfqpoint{0.865263in}{1.494182in}}%
\pgfpathcurveto{\pgfqpoint{0.871087in}{1.488358in}}{\pgfqpoint{0.878987in}{1.485086in}}{\pgfqpoint{0.887224in}{1.485086in}}%
\pgfpathclose%
\pgfusepath{stroke,fill}%
\end{pgfscope}%
\begin{pgfscope}%
\pgfpathrectangle{\pgfqpoint{0.100000in}{0.212622in}}{\pgfqpoint{3.696000in}{3.696000in}}%
\pgfusepath{clip}%
\pgfsetbuttcap%
\pgfsetroundjoin%
\definecolor{currentfill}{rgb}{0.121569,0.466667,0.705882}%
\pgfsetfillcolor{currentfill}%
\pgfsetfillopacity{0.587569}%
\pgfsetlinewidth{1.003750pt}%
\definecolor{currentstroke}{rgb}{0.121569,0.466667,0.705882}%
\pgfsetstrokecolor{currentstroke}%
\pgfsetstrokeopacity{0.587569}%
\pgfsetdash{}{0pt}%
\pgfpathmoveto{\pgfqpoint{0.588738in}{1.218323in}}%
\pgfpathcurveto{\pgfqpoint{0.596974in}{1.218323in}}{\pgfqpoint{0.604874in}{1.221595in}}{\pgfqpoint{0.610698in}{1.227419in}}%
\pgfpathcurveto{\pgfqpoint{0.616522in}{1.233243in}}{\pgfqpoint{0.619795in}{1.241143in}}{\pgfqpoint{0.619795in}{1.249380in}}%
\pgfpathcurveto{\pgfqpoint{0.619795in}{1.257616in}}{\pgfqpoint{0.616522in}{1.265516in}}{\pgfqpoint{0.610698in}{1.271340in}}%
\pgfpathcurveto{\pgfqpoint{0.604874in}{1.277164in}}{\pgfqpoint{0.596974in}{1.280436in}}{\pgfqpoint{0.588738in}{1.280436in}}%
\pgfpathcurveto{\pgfqpoint{0.580502in}{1.280436in}}{\pgfqpoint{0.572602in}{1.277164in}}{\pgfqpoint{0.566778in}{1.271340in}}%
\pgfpathcurveto{\pgfqpoint{0.560954in}{1.265516in}}{\pgfqpoint{0.557682in}{1.257616in}}{\pgfqpoint{0.557682in}{1.249380in}}%
\pgfpathcurveto{\pgfqpoint{0.557682in}{1.241143in}}{\pgfqpoint{0.560954in}{1.233243in}}{\pgfqpoint{0.566778in}{1.227419in}}%
\pgfpathcurveto{\pgfqpoint{0.572602in}{1.221595in}}{\pgfqpoint{0.580502in}{1.218323in}}{\pgfqpoint{0.588738in}{1.218323in}}%
\pgfpathclose%
\pgfusepath{stroke,fill}%
\end{pgfscope}%
\begin{pgfscope}%
\pgfpathrectangle{\pgfqpoint{0.100000in}{0.212622in}}{\pgfqpoint{3.696000in}{3.696000in}}%
\pgfusepath{clip}%
\pgfsetbuttcap%
\pgfsetroundjoin%
\definecolor{currentfill}{rgb}{0.121569,0.466667,0.705882}%
\pgfsetfillcolor{currentfill}%
\pgfsetfillopacity{0.587774}%
\pgfsetlinewidth{1.003750pt}%
\definecolor{currentstroke}{rgb}{0.121569,0.466667,0.705882}%
\pgfsetstrokecolor{currentstroke}%
\pgfsetstrokeopacity{0.587774}%
\pgfsetdash{}{0pt}%
\pgfpathmoveto{\pgfqpoint{0.991587in}{1.743450in}}%
\pgfpathcurveto{\pgfqpoint{0.999823in}{1.743450in}}{\pgfqpoint{1.007723in}{1.746723in}}{\pgfqpoint{1.013547in}{1.752546in}}%
\pgfpathcurveto{\pgfqpoint{1.019371in}{1.758370in}}{\pgfqpoint{1.022644in}{1.766270in}}{\pgfqpoint{1.022644in}{1.774507in}}%
\pgfpathcurveto{\pgfqpoint{1.022644in}{1.782743in}}{\pgfqpoint{1.019371in}{1.790643in}}{\pgfqpoint{1.013547in}{1.796467in}}%
\pgfpathcurveto{\pgfqpoint{1.007723in}{1.802291in}}{\pgfqpoint{0.999823in}{1.805563in}}{\pgfqpoint{0.991587in}{1.805563in}}%
\pgfpathcurveto{\pgfqpoint{0.983351in}{1.805563in}}{\pgfqpoint{0.975451in}{1.802291in}}{\pgfqpoint{0.969627in}{1.796467in}}%
\pgfpathcurveto{\pgfqpoint{0.963803in}{1.790643in}}{\pgfqpoint{0.960531in}{1.782743in}}{\pgfqpoint{0.960531in}{1.774507in}}%
\pgfpathcurveto{\pgfqpoint{0.960531in}{1.766270in}}{\pgfqpoint{0.963803in}{1.758370in}}{\pgfqpoint{0.969627in}{1.752546in}}%
\pgfpathcurveto{\pgfqpoint{0.975451in}{1.746723in}}{\pgfqpoint{0.983351in}{1.743450in}}{\pgfqpoint{0.991587in}{1.743450in}}%
\pgfpathclose%
\pgfusepath{stroke,fill}%
\end{pgfscope}%
\begin{pgfscope}%
\pgfpathrectangle{\pgfqpoint{0.100000in}{0.212622in}}{\pgfqpoint{3.696000in}{3.696000in}}%
\pgfusepath{clip}%
\pgfsetbuttcap%
\pgfsetroundjoin%
\definecolor{currentfill}{rgb}{0.121569,0.466667,0.705882}%
\pgfsetfillcolor{currentfill}%
\pgfsetfillopacity{0.587952}%
\pgfsetlinewidth{1.003750pt}%
\definecolor{currentstroke}{rgb}{0.121569,0.466667,0.705882}%
\pgfsetstrokecolor{currentstroke}%
\pgfsetstrokeopacity{0.587952}%
\pgfsetdash{}{0pt}%
\pgfpathmoveto{\pgfqpoint{0.873148in}{1.576271in}}%
\pgfpathcurveto{\pgfqpoint{0.881384in}{1.576271in}}{\pgfqpoint{0.889284in}{1.579543in}}{\pgfqpoint{0.895108in}{1.585367in}}%
\pgfpathcurveto{\pgfqpoint{0.900932in}{1.591191in}}{\pgfqpoint{0.904205in}{1.599091in}}{\pgfqpoint{0.904205in}{1.607327in}}%
\pgfpathcurveto{\pgfqpoint{0.904205in}{1.615564in}}{\pgfqpoint{0.900932in}{1.623464in}}{\pgfqpoint{0.895108in}{1.629288in}}%
\pgfpathcurveto{\pgfqpoint{0.889284in}{1.635112in}}{\pgfqpoint{0.881384in}{1.638384in}}{\pgfqpoint{0.873148in}{1.638384in}}%
\pgfpathcurveto{\pgfqpoint{0.864912in}{1.638384in}}{\pgfqpoint{0.857012in}{1.635112in}}{\pgfqpoint{0.851188in}{1.629288in}}%
\pgfpathcurveto{\pgfqpoint{0.845364in}{1.623464in}}{\pgfqpoint{0.842092in}{1.615564in}}{\pgfqpoint{0.842092in}{1.607327in}}%
\pgfpathcurveto{\pgfqpoint{0.842092in}{1.599091in}}{\pgfqpoint{0.845364in}{1.591191in}}{\pgfqpoint{0.851188in}{1.585367in}}%
\pgfpathcurveto{\pgfqpoint{0.857012in}{1.579543in}}{\pgfqpoint{0.864912in}{1.576271in}}{\pgfqpoint{0.873148in}{1.576271in}}%
\pgfpathclose%
\pgfusepath{stroke,fill}%
\end{pgfscope}%
\begin{pgfscope}%
\pgfpathrectangle{\pgfqpoint{0.100000in}{0.212622in}}{\pgfqpoint{3.696000in}{3.696000in}}%
\pgfusepath{clip}%
\pgfsetbuttcap%
\pgfsetroundjoin%
\definecolor{currentfill}{rgb}{0.121569,0.466667,0.705882}%
\pgfsetfillcolor{currentfill}%
\pgfsetfillopacity{0.588243}%
\pgfsetlinewidth{1.003750pt}%
\definecolor{currentstroke}{rgb}{0.121569,0.466667,0.705882}%
\pgfsetstrokecolor{currentstroke}%
\pgfsetstrokeopacity{0.588243}%
\pgfsetdash{}{0pt}%
\pgfpathmoveto{\pgfqpoint{0.591323in}{1.217001in}}%
\pgfpathcurveto{\pgfqpoint{0.599559in}{1.217001in}}{\pgfqpoint{0.607459in}{1.220274in}}{\pgfqpoint{0.613283in}{1.226097in}}%
\pgfpathcurveto{\pgfqpoint{0.619107in}{1.231921in}}{\pgfqpoint{0.622379in}{1.239821in}}{\pgfqpoint{0.622379in}{1.248058in}}%
\pgfpathcurveto{\pgfqpoint{0.622379in}{1.256294in}}{\pgfqpoint{0.619107in}{1.264194in}}{\pgfqpoint{0.613283in}{1.270018in}}%
\pgfpathcurveto{\pgfqpoint{0.607459in}{1.275842in}}{\pgfqpoint{0.599559in}{1.279114in}}{\pgfqpoint{0.591323in}{1.279114in}}%
\pgfpathcurveto{\pgfqpoint{0.583086in}{1.279114in}}{\pgfqpoint{0.575186in}{1.275842in}}{\pgfqpoint{0.569362in}{1.270018in}}%
\pgfpathcurveto{\pgfqpoint{0.563538in}{1.264194in}}{\pgfqpoint{0.560266in}{1.256294in}}{\pgfqpoint{0.560266in}{1.248058in}}%
\pgfpathcurveto{\pgfqpoint{0.560266in}{1.239821in}}{\pgfqpoint{0.563538in}{1.231921in}}{\pgfqpoint{0.569362in}{1.226097in}}%
\pgfpathcurveto{\pgfqpoint{0.575186in}{1.220274in}}{\pgfqpoint{0.583086in}{1.217001in}}{\pgfqpoint{0.591323in}{1.217001in}}%
\pgfpathclose%
\pgfusepath{stroke,fill}%
\end{pgfscope}%
\begin{pgfscope}%
\pgfpathrectangle{\pgfqpoint{0.100000in}{0.212622in}}{\pgfqpoint{3.696000in}{3.696000in}}%
\pgfusepath{clip}%
\pgfsetbuttcap%
\pgfsetroundjoin%
\definecolor{currentfill}{rgb}{0.121569,0.466667,0.705882}%
\pgfsetfillcolor{currentfill}%
\pgfsetfillopacity{0.588661}%
\pgfsetlinewidth{1.003750pt}%
\definecolor{currentstroke}{rgb}{0.121569,0.466667,0.705882}%
\pgfsetstrokecolor{currentstroke}%
\pgfsetstrokeopacity{0.588661}%
\pgfsetdash{}{0pt}%
\pgfpathmoveto{\pgfqpoint{2.094011in}{2.139026in}}%
\pgfpathcurveto{\pgfqpoint{2.102247in}{2.139026in}}{\pgfqpoint{2.110147in}{2.142299in}}{\pgfqpoint{2.115971in}{2.148123in}}%
\pgfpathcurveto{\pgfqpoint{2.121795in}{2.153946in}}{\pgfqpoint{2.125067in}{2.161847in}}{\pgfqpoint{2.125067in}{2.170083in}}%
\pgfpathcurveto{\pgfqpoint{2.125067in}{2.178319in}}{\pgfqpoint{2.121795in}{2.186219in}}{\pgfqpoint{2.115971in}{2.192043in}}%
\pgfpathcurveto{\pgfqpoint{2.110147in}{2.197867in}}{\pgfqpoint{2.102247in}{2.201139in}}{\pgfqpoint{2.094011in}{2.201139in}}%
\pgfpathcurveto{\pgfqpoint{2.085775in}{2.201139in}}{\pgfqpoint{2.077875in}{2.197867in}}{\pgfqpoint{2.072051in}{2.192043in}}%
\pgfpathcurveto{\pgfqpoint{2.066227in}{2.186219in}}{\pgfqpoint{2.062954in}{2.178319in}}{\pgfqpoint{2.062954in}{2.170083in}}%
\pgfpathcurveto{\pgfqpoint{2.062954in}{2.161847in}}{\pgfqpoint{2.066227in}{2.153946in}}{\pgfqpoint{2.072051in}{2.148123in}}%
\pgfpathcurveto{\pgfqpoint{2.077875in}{2.142299in}}{\pgfqpoint{2.085775in}{2.139026in}}{\pgfqpoint{2.094011in}{2.139026in}}%
\pgfpathclose%
\pgfusepath{stroke,fill}%
\end{pgfscope}%
\begin{pgfscope}%
\pgfpathrectangle{\pgfqpoint{0.100000in}{0.212622in}}{\pgfqpoint{3.696000in}{3.696000in}}%
\pgfusepath{clip}%
\pgfsetbuttcap%
\pgfsetroundjoin%
\definecolor{currentfill}{rgb}{0.121569,0.466667,0.705882}%
\pgfsetfillcolor{currentfill}%
\pgfsetfillopacity{0.588849}%
\pgfsetlinewidth{1.003750pt}%
\definecolor{currentstroke}{rgb}{0.121569,0.466667,0.705882}%
\pgfsetstrokecolor{currentstroke}%
\pgfsetstrokeopacity{0.588849}%
\pgfsetdash{}{0pt}%
\pgfpathmoveto{\pgfqpoint{0.885606in}{1.485021in}}%
\pgfpathcurveto{\pgfqpoint{0.893842in}{1.485021in}}{\pgfqpoint{0.901743in}{1.488293in}}{\pgfqpoint{0.907566in}{1.494117in}}%
\pgfpathcurveto{\pgfqpoint{0.913390in}{1.499941in}}{\pgfqpoint{0.916663in}{1.507841in}}{\pgfqpoint{0.916663in}{1.516078in}}%
\pgfpathcurveto{\pgfqpoint{0.916663in}{1.524314in}}{\pgfqpoint{0.913390in}{1.532214in}}{\pgfqpoint{0.907566in}{1.538038in}}%
\pgfpathcurveto{\pgfqpoint{0.901743in}{1.543862in}}{\pgfqpoint{0.893842in}{1.547134in}}{\pgfqpoint{0.885606in}{1.547134in}}%
\pgfpathcurveto{\pgfqpoint{0.877370in}{1.547134in}}{\pgfqpoint{0.869470in}{1.543862in}}{\pgfqpoint{0.863646in}{1.538038in}}%
\pgfpathcurveto{\pgfqpoint{0.857822in}{1.532214in}}{\pgfqpoint{0.854550in}{1.524314in}}{\pgfqpoint{0.854550in}{1.516078in}}%
\pgfpathcurveto{\pgfqpoint{0.854550in}{1.507841in}}{\pgfqpoint{0.857822in}{1.499941in}}{\pgfqpoint{0.863646in}{1.494117in}}%
\pgfpathcurveto{\pgfqpoint{0.869470in}{1.488293in}}{\pgfqpoint{0.877370in}{1.485021in}}{\pgfqpoint{0.885606in}{1.485021in}}%
\pgfpathclose%
\pgfusepath{stroke,fill}%
\end{pgfscope}%
\begin{pgfscope}%
\pgfpathrectangle{\pgfqpoint{0.100000in}{0.212622in}}{\pgfqpoint{3.696000in}{3.696000in}}%
\pgfusepath{clip}%
\pgfsetbuttcap%
\pgfsetroundjoin%
\definecolor{currentfill}{rgb}{0.121569,0.466667,0.705882}%
\pgfsetfillcolor{currentfill}%
\pgfsetfillopacity{0.589434}%
\pgfsetlinewidth{1.003750pt}%
\definecolor{currentstroke}{rgb}{0.121569,0.466667,0.705882}%
\pgfsetstrokecolor{currentstroke}%
\pgfsetstrokeopacity{0.589434}%
\pgfsetdash{}{0pt}%
\pgfpathmoveto{\pgfqpoint{0.596309in}{1.215382in}}%
\pgfpathcurveto{\pgfqpoint{0.604545in}{1.215382in}}{\pgfqpoint{0.612445in}{1.218654in}}{\pgfqpoint{0.618269in}{1.224478in}}%
\pgfpathcurveto{\pgfqpoint{0.624093in}{1.230302in}}{\pgfqpoint{0.627365in}{1.238202in}}{\pgfqpoint{0.627365in}{1.246439in}}%
\pgfpathcurveto{\pgfqpoint{0.627365in}{1.254675in}}{\pgfqpoint{0.624093in}{1.262575in}}{\pgfqpoint{0.618269in}{1.268399in}}%
\pgfpathcurveto{\pgfqpoint{0.612445in}{1.274223in}}{\pgfqpoint{0.604545in}{1.277495in}}{\pgfqpoint{0.596309in}{1.277495in}}%
\pgfpathcurveto{\pgfqpoint{0.588073in}{1.277495in}}{\pgfqpoint{0.580172in}{1.274223in}}{\pgfqpoint{0.574349in}{1.268399in}}%
\pgfpathcurveto{\pgfqpoint{0.568525in}{1.262575in}}{\pgfqpoint{0.565252in}{1.254675in}}{\pgfqpoint{0.565252in}{1.246439in}}%
\pgfpathcurveto{\pgfqpoint{0.565252in}{1.238202in}}{\pgfqpoint{0.568525in}{1.230302in}}{\pgfqpoint{0.574349in}{1.224478in}}%
\pgfpathcurveto{\pgfqpoint{0.580172in}{1.218654in}}{\pgfqpoint{0.588073in}{1.215382in}}{\pgfqpoint{0.596309in}{1.215382in}}%
\pgfpathclose%
\pgfusepath{stroke,fill}%
\end{pgfscope}%
\begin{pgfscope}%
\pgfpathrectangle{\pgfqpoint{0.100000in}{0.212622in}}{\pgfqpoint{3.696000in}{3.696000in}}%
\pgfusepath{clip}%
\pgfsetbuttcap%
\pgfsetroundjoin%
\definecolor{currentfill}{rgb}{0.121569,0.466667,0.705882}%
\pgfsetfillcolor{currentfill}%
\pgfsetfillopacity{0.590431}%
\pgfsetlinewidth{1.003750pt}%
\definecolor{currentstroke}{rgb}{0.121569,0.466667,0.705882}%
\pgfsetstrokecolor{currentstroke}%
\pgfsetstrokeopacity{0.590431}%
\pgfsetdash{}{0pt}%
\pgfpathmoveto{\pgfqpoint{0.984221in}{1.729435in}}%
\pgfpathcurveto{\pgfqpoint{0.992458in}{1.729435in}}{\pgfqpoint{1.000358in}{1.732708in}}{\pgfqpoint{1.006182in}{1.738532in}}%
\pgfpathcurveto{\pgfqpoint{1.012005in}{1.744356in}}{\pgfqpoint{1.015278in}{1.752256in}}{\pgfqpoint{1.015278in}{1.760492in}}%
\pgfpathcurveto{\pgfqpoint{1.015278in}{1.768728in}}{\pgfqpoint{1.012005in}{1.776628in}}{\pgfqpoint{1.006182in}{1.782452in}}%
\pgfpathcurveto{\pgfqpoint{1.000358in}{1.788276in}}{\pgfqpoint{0.992458in}{1.791548in}}{\pgfqpoint{0.984221in}{1.791548in}}%
\pgfpathcurveto{\pgfqpoint{0.975985in}{1.791548in}}{\pgfqpoint{0.968085in}{1.788276in}}{\pgfqpoint{0.962261in}{1.782452in}}%
\pgfpathcurveto{\pgfqpoint{0.956437in}{1.776628in}}{\pgfqpoint{0.953165in}{1.768728in}}{\pgfqpoint{0.953165in}{1.760492in}}%
\pgfpathcurveto{\pgfqpoint{0.953165in}{1.752256in}}{\pgfqpoint{0.956437in}{1.744356in}}{\pgfqpoint{0.962261in}{1.738532in}}%
\pgfpathcurveto{\pgfqpoint{0.968085in}{1.732708in}}{\pgfqpoint{0.975985in}{1.729435in}}{\pgfqpoint{0.984221in}{1.729435in}}%
\pgfpathclose%
\pgfusepath{stroke,fill}%
\end{pgfscope}%
\begin{pgfscope}%
\pgfpathrectangle{\pgfqpoint{0.100000in}{0.212622in}}{\pgfqpoint{3.696000in}{3.696000in}}%
\pgfusepath{clip}%
\pgfsetbuttcap%
\pgfsetroundjoin%
\definecolor{currentfill}{rgb}{0.121569,0.466667,0.705882}%
\pgfsetfillcolor{currentfill}%
\pgfsetfillopacity{0.590564}%
\pgfsetlinewidth{1.003750pt}%
\definecolor{currentstroke}{rgb}{0.121569,0.466667,0.705882}%
\pgfsetstrokecolor{currentstroke}%
\pgfsetstrokeopacity{0.590564}%
\pgfsetdash{}{0pt}%
\pgfpathmoveto{\pgfqpoint{0.870639in}{1.579868in}}%
\pgfpathcurveto{\pgfqpoint{0.878875in}{1.579868in}}{\pgfqpoint{0.886775in}{1.583140in}}{\pgfqpoint{0.892599in}{1.588964in}}%
\pgfpathcurveto{\pgfqpoint{0.898423in}{1.594788in}}{\pgfqpoint{0.901696in}{1.602688in}}{\pgfqpoint{0.901696in}{1.610924in}}%
\pgfpathcurveto{\pgfqpoint{0.901696in}{1.619161in}}{\pgfqpoint{0.898423in}{1.627061in}}{\pgfqpoint{0.892599in}{1.632885in}}%
\pgfpathcurveto{\pgfqpoint{0.886775in}{1.638709in}}{\pgfqpoint{0.878875in}{1.641981in}}{\pgfqpoint{0.870639in}{1.641981in}}%
\pgfpathcurveto{\pgfqpoint{0.862403in}{1.641981in}}{\pgfqpoint{0.854503in}{1.638709in}}{\pgfqpoint{0.848679in}{1.632885in}}%
\pgfpathcurveto{\pgfqpoint{0.842855in}{1.627061in}}{\pgfqpoint{0.839583in}{1.619161in}}{\pgfqpoint{0.839583in}{1.610924in}}%
\pgfpathcurveto{\pgfqpoint{0.839583in}{1.602688in}}{\pgfqpoint{0.842855in}{1.594788in}}{\pgfqpoint{0.848679in}{1.588964in}}%
\pgfpathcurveto{\pgfqpoint{0.854503in}{1.583140in}}{\pgfqpoint{0.862403in}{1.579868in}}{\pgfqpoint{0.870639in}{1.579868in}}%
\pgfpathclose%
\pgfusepath{stroke,fill}%
\end{pgfscope}%
\begin{pgfscope}%
\pgfpathrectangle{\pgfqpoint{0.100000in}{0.212622in}}{\pgfqpoint{3.696000in}{3.696000in}}%
\pgfusepath{clip}%
\pgfsetbuttcap%
\pgfsetroundjoin%
\definecolor{currentfill}{rgb}{0.121569,0.466667,0.705882}%
\pgfsetfillcolor{currentfill}%
\pgfsetfillopacity{0.590821}%
\pgfsetlinewidth{1.003750pt}%
\definecolor{currentstroke}{rgb}{0.121569,0.466667,0.705882}%
\pgfsetstrokecolor{currentstroke}%
\pgfsetstrokeopacity{0.590821}%
\pgfsetdash{}{0pt}%
\pgfpathmoveto{\pgfqpoint{2.095052in}{2.130807in}}%
\pgfpathcurveto{\pgfqpoint{2.103289in}{2.130807in}}{\pgfqpoint{2.111189in}{2.134080in}}{\pgfqpoint{2.117013in}{2.139904in}}%
\pgfpathcurveto{\pgfqpoint{2.122837in}{2.145728in}}{\pgfqpoint{2.126109in}{2.153628in}}{\pgfqpoint{2.126109in}{2.161864in}}%
\pgfpathcurveto{\pgfqpoint{2.126109in}{2.170100in}}{\pgfqpoint{2.122837in}{2.178000in}}{\pgfqpoint{2.117013in}{2.183824in}}%
\pgfpathcurveto{\pgfqpoint{2.111189in}{2.189648in}}{\pgfqpoint{2.103289in}{2.192920in}}{\pgfqpoint{2.095052in}{2.192920in}}%
\pgfpathcurveto{\pgfqpoint{2.086816in}{2.192920in}}{\pgfqpoint{2.078916in}{2.189648in}}{\pgfqpoint{2.073092in}{2.183824in}}%
\pgfpathcurveto{\pgfqpoint{2.067268in}{2.178000in}}{\pgfqpoint{2.063996in}{2.170100in}}{\pgfqpoint{2.063996in}{2.161864in}}%
\pgfpathcurveto{\pgfqpoint{2.063996in}{2.153628in}}{\pgfqpoint{2.067268in}{2.145728in}}{\pgfqpoint{2.073092in}{2.139904in}}%
\pgfpathcurveto{\pgfqpoint{2.078916in}{2.134080in}}{\pgfqpoint{2.086816in}{2.130807in}}{\pgfqpoint{2.095052in}{2.130807in}}%
\pgfpathclose%
\pgfusepath{stroke,fill}%
\end{pgfscope}%
\begin{pgfscope}%
\pgfpathrectangle{\pgfqpoint{0.100000in}{0.212622in}}{\pgfqpoint{3.696000in}{3.696000in}}%
\pgfusepath{clip}%
\pgfsetbuttcap%
\pgfsetroundjoin%
\definecolor{currentfill}{rgb}{0.121569,0.466667,0.705882}%
\pgfsetfillcolor{currentfill}%
\pgfsetfillopacity{0.591000}%
\pgfsetlinewidth{1.003750pt}%
\definecolor{currentstroke}{rgb}{0.121569,0.466667,0.705882}%
\pgfsetstrokecolor{currentstroke}%
\pgfsetstrokeopacity{0.591000}%
\pgfsetdash{}{0pt}%
\pgfpathmoveto{\pgfqpoint{0.883340in}{1.485127in}}%
\pgfpathcurveto{\pgfqpoint{0.891576in}{1.485127in}}{\pgfqpoint{0.899476in}{1.488400in}}{\pgfqpoint{0.905300in}{1.494224in}}%
\pgfpathcurveto{\pgfqpoint{0.911124in}{1.500048in}}{\pgfqpoint{0.914396in}{1.507948in}}{\pgfqpoint{0.914396in}{1.516184in}}%
\pgfpathcurveto{\pgfqpoint{0.914396in}{1.524420in}}{\pgfqpoint{0.911124in}{1.532320in}}{\pgfqpoint{0.905300in}{1.538144in}}%
\pgfpathcurveto{\pgfqpoint{0.899476in}{1.543968in}}{\pgfqpoint{0.891576in}{1.547240in}}{\pgfqpoint{0.883340in}{1.547240in}}%
\pgfpathcurveto{\pgfqpoint{0.875103in}{1.547240in}}{\pgfqpoint{0.867203in}{1.543968in}}{\pgfqpoint{0.861380in}{1.538144in}}%
\pgfpathcurveto{\pgfqpoint{0.855556in}{1.532320in}}{\pgfqpoint{0.852283in}{1.524420in}}{\pgfqpoint{0.852283in}{1.516184in}}%
\pgfpathcurveto{\pgfqpoint{0.852283in}{1.507948in}}{\pgfqpoint{0.855556in}{1.500048in}}{\pgfqpoint{0.861380in}{1.494224in}}%
\pgfpathcurveto{\pgfqpoint{0.867203in}{1.488400in}}{\pgfqpoint{0.875103in}{1.485127in}}{\pgfqpoint{0.883340in}{1.485127in}}%
\pgfpathclose%
\pgfusepath{stroke,fill}%
\end{pgfscope}%
\begin{pgfscope}%
\pgfpathrectangle{\pgfqpoint{0.100000in}{0.212622in}}{\pgfqpoint{3.696000in}{3.696000in}}%
\pgfusepath{clip}%
\pgfsetbuttcap%
\pgfsetroundjoin%
\definecolor{currentfill}{rgb}{0.121569,0.466667,0.705882}%
\pgfsetfillcolor{currentfill}%
\pgfsetfillopacity{0.592236}%
\pgfsetlinewidth{1.003750pt}%
\definecolor{currentstroke}{rgb}{0.121569,0.466667,0.705882}%
\pgfsetstrokecolor{currentstroke}%
\pgfsetstrokeopacity{0.592236}%
\pgfsetdash{}{0pt}%
\pgfpathmoveto{\pgfqpoint{0.868177in}{1.581284in}}%
\pgfpathcurveto{\pgfqpoint{0.876413in}{1.581284in}}{\pgfqpoint{0.884313in}{1.584556in}}{\pgfqpoint{0.890137in}{1.590380in}}%
\pgfpathcurveto{\pgfqpoint{0.895961in}{1.596204in}}{\pgfqpoint{0.899233in}{1.604104in}}{\pgfqpoint{0.899233in}{1.612341in}}%
\pgfpathcurveto{\pgfqpoint{0.899233in}{1.620577in}}{\pgfqpoint{0.895961in}{1.628477in}}{\pgfqpoint{0.890137in}{1.634301in}}%
\pgfpathcurveto{\pgfqpoint{0.884313in}{1.640125in}}{\pgfqpoint{0.876413in}{1.643397in}}{\pgfqpoint{0.868177in}{1.643397in}}%
\pgfpathcurveto{\pgfqpoint{0.859940in}{1.643397in}}{\pgfqpoint{0.852040in}{1.640125in}}{\pgfqpoint{0.846216in}{1.634301in}}%
\pgfpathcurveto{\pgfqpoint{0.840392in}{1.628477in}}{\pgfqpoint{0.837120in}{1.620577in}}{\pgfqpoint{0.837120in}{1.612341in}}%
\pgfpathcurveto{\pgfqpoint{0.837120in}{1.604104in}}{\pgfqpoint{0.840392in}{1.596204in}}{\pgfqpoint{0.846216in}{1.590380in}}%
\pgfpathcurveto{\pgfqpoint{0.852040in}{1.584556in}}{\pgfqpoint{0.859940in}{1.581284in}}{\pgfqpoint{0.868177in}{1.581284in}}%
\pgfpathclose%
\pgfusepath{stroke,fill}%
\end{pgfscope}%
\begin{pgfscope}%
\pgfpathrectangle{\pgfqpoint{0.100000in}{0.212622in}}{\pgfqpoint{3.696000in}{3.696000in}}%
\pgfusepath{clip}%
\pgfsetbuttcap%
\pgfsetroundjoin%
\definecolor{currentfill}{rgb}{0.121569,0.466667,0.705882}%
\pgfsetfillcolor{currentfill}%
\pgfsetfillopacity{0.592607}%
\pgfsetlinewidth{1.003750pt}%
\definecolor{currentstroke}{rgb}{0.121569,0.466667,0.705882}%
\pgfsetstrokecolor{currentstroke}%
\pgfsetstrokeopacity{0.592607}%
\pgfsetdash{}{0pt}%
\pgfpathmoveto{\pgfqpoint{0.976654in}{1.716762in}}%
\pgfpathcurveto{\pgfqpoint{0.984890in}{1.716762in}}{\pgfqpoint{0.992790in}{1.720034in}}{\pgfqpoint{0.998614in}{1.725858in}}%
\pgfpathcurveto{\pgfqpoint{1.004438in}{1.731682in}}{\pgfqpoint{1.007710in}{1.739582in}}{\pgfqpoint{1.007710in}{1.747819in}}%
\pgfpathcurveto{\pgfqpoint{1.007710in}{1.756055in}}{\pgfqpoint{1.004438in}{1.763955in}}{\pgfqpoint{0.998614in}{1.769779in}}%
\pgfpathcurveto{\pgfqpoint{0.992790in}{1.775603in}}{\pgfqpoint{0.984890in}{1.778875in}}{\pgfqpoint{0.976654in}{1.778875in}}%
\pgfpathcurveto{\pgfqpoint{0.968417in}{1.778875in}}{\pgfqpoint{0.960517in}{1.775603in}}{\pgfqpoint{0.954693in}{1.769779in}}%
\pgfpathcurveto{\pgfqpoint{0.948870in}{1.763955in}}{\pgfqpoint{0.945597in}{1.756055in}}{\pgfqpoint{0.945597in}{1.747819in}}%
\pgfpathcurveto{\pgfqpoint{0.945597in}{1.739582in}}{\pgfqpoint{0.948870in}{1.731682in}}{\pgfqpoint{0.954693in}{1.725858in}}%
\pgfpathcurveto{\pgfqpoint{0.960517in}{1.720034in}}{\pgfqpoint{0.968417in}{1.716762in}}{\pgfqpoint{0.976654in}{1.716762in}}%
\pgfpathclose%
\pgfusepath{stroke,fill}%
\end{pgfscope}%
\begin{pgfscope}%
\pgfpathrectangle{\pgfqpoint{0.100000in}{0.212622in}}{\pgfqpoint{3.696000in}{3.696000in}}%
\pgfusepath{clip}%
\pgfsetbuttcap%
\pgfsetroundjoin%
\definecolor{currentfill}{rgb}{0.121569,0.466667,0.705882}%
\pgfsetfillcolor{currentfill}%
\pgfsetfillopacity{0.592953}%
\pgfsetlinewidth{1.003750pt}%
\definecolor{currentstroke}{rgb}{0.121569,0.466667,0.705882}%
\pgfsetstrokecolor{currentstroke}%
\pgfsetstrokeopacity{0.592953}%
\pgfsetdash{}{0pt}%
\pgfpathmoveto{\pgfqpoint{0.604651in}{1.215754in}}%
\pgfpathcurveto{\pgfqpoint{0.612887in}{1.215754in}}{\pgfqpoint{0.620787in}{1.219026in}}{\pgfqpoint{0.626611in}{1.224850in}}%
\pgfpathcurveto{\pgfqpoint{0.632435in}{1.230674in}}{\pgfqpoint{0.635707in}{1.238574in}}{\pgfqpoint{0.635707in}{1.246810in}}%
\pgfpathcurveto{\pgfqpoint{0.635707in}{1.255047in}}{\pgfqpoint{0.632435in}{1.262947in}}{\pgfqpoint{0.626611in}{1.268771in}}%
\pgfpathcurveto{\pgfqpoint{0.620787in}{1.274595in}}{\pgfqpoint{0.612887in}{1.277867in}}{\pgfqpoint{0.604651in}{1.277867in}}%
\pgfpathcurveto{\pgfqpoint{0.596414in}{1.277867in}}{\pgfqpoint{0.588514in}{1.274595in}}{\pgfqpoint{0.582690in}{1.268771in}}%
\pgfpathcurveto{\pgfqpoint{0.576866in}{1.262947in}}{\pgfqpoint{0.573594in}{1.255047in}}{\pgfqpoint{0.573594in}{1.246810in}}%
\pgfpathcurveto{\pgfqpoint{0.573594in}{1.238574in}}{\pgfqpoint{0.576866in}{1.230674in}}{\pgfqpoint{0.582690in}{1.224850in}}%
\pgfpathcurveto{\pgfqpoint{0.588514in}{1.219026in}}{\pgfqpoint{0.596414in}{1.215754in}}{\pgfqpoint{0.604651in}{1.215754in}}%
\pgfpathclose%
\pgfusepath{stroke,fill}%
\end{pgfscope}%
\begin{pgfscope}%
\pgfpathrectangle{\pgfqpoint{0.100000in}{0.212622in}}{\pgfqpoint{3.696000in}{3.696000in}}%
\pgfusepath{clip}%
\pgfsetbuttcap%
\pgfsetroundjoin%
\definecolor{currentfill}{rgb}{0.121569,0.466667,0.705882}%
\pgfsetfillcolor{currentfill}%
\pgfsetfillopacity{0.593162}%
\pgfsetlinewidth{1.003750pt}%
\definecolor{currentstroke}{rgb}{0.121569,0.466667,0.705882}%
\pgfsetstrokecolor{currentstroke}%
\pgfsetstrokeopacity{0.593162}%
\pgfsetdash{}{0pt}%
\pgfpathmoveto{\pgfqpoint{0.866388in}{1.581450in}}%
\pgfpathcurveto{\pgfqpoint{0.874624in}{1.581450in}}{\pgfqpoint{0.882524in}{1.584723in}}{\pgfqpoint{0.888348in}{1.590547in}}%
\pgfpathcurveto{\pgfqpoint{0.894172in}{1.596371in}}{\pgfqpoint{0.897444in}{1.604271in}}{\pgfqpoint{0.897444in}{1.612507in}}%
\pgfpathcurveto{\pgfqpoint{0.897444in}{1.620743in}}{\pgfqpoint{0.894172in}{1.628643in}}{\pgfqpoint{0.888348in}{1.634467in}}%
\pgfpathcurveto{\pgfqpoint{0.882524in}{1.640291in}}{\pgfqpoint{0.874624in}{1.643563in}}{\pgfqpoint{0.866388in}{1.643563in}}%
\pgfpathcurveto{\pgfqpoint{0.858152in}{1.643563in}}{\pgfqpoint{0.850252in}{1.640291in}}{\pgfqpoint{0.844428in}{1.634467in}}%
\pgfpathcurveto{\pgfqpoint{0.838604in}{1.628643in}}{\pgfqpoint{0.835331in}{1.620743in}}{\pgfqpoint{0.835331in}{1.612507in}}%
\pgfpathcurveto{\pgfqpoint{0.835331in}{1.604271in}}{\pgfqpoint{0.838604in}{1.596371in}}{\pgfqpoint{0.844428in}{1.590547in}}%
\pgfpathcurveto{\pgfqpoint{0.850252in}{1.584723in}}{\pgfqpoint{0.858152in}{1.581450in}}{\pgfqpoint{0.866388in}{1.581450in}}%
\pgfpathclose%
\pgfusepath{stroke,fill}%
\end{pgfscope}%
\begin{pgfscope}%
\pgfpathrectangle{\pgfqpoint{0.100000in}{0.212622in}}{\pgfqpoint{3.696000in}{3.696000in}}%
\pgfusepath{clip}%
\pgfsetbuttcap%
\pgfsetroundjoin%
\definecolor{currentfill}{rgb}{0.121569,0.466667,0.705882}%
\pgfsetfillcolor{currentfill}%
\pgfsetfillopacity{0.593166}%
\pgfsetlinewidth{1.003750pt}%
\definecolor{currentstroke}{rgb}{0.121569,0.466667,0.705882}%
\pgfsetstrokecolor{currentstroke}%
\pgfsetstrokeopacity{0.593166}%
\pgfsetdash{}{0pt}%
\pgfpathmoveto{\pgfqpoint{2.097278in}{2.120543in}}%
\pgfpathcurveto{\pgfqpoint{2.105514in}{2.120543in}}{\pgfqpoint{2.113414in}{2.123816in}}{\pgfqpoint{2.119238in}{2.129640in}}%
\pgfpathcurveto{\pgfqpoint{2.125062in}{2.135463in}}{\pgfqpoint{2.128335in}{2.143364in}}{\pgfqpoint{2.128335in}{2.151600in}}%
\pgfpathcurveto{\pgfqpoint{2.128335in}{2.159836in}}{\pgfqpoint{2.125062in}{2.167736in}}{\pgfqpoint{2.119238in}{2.173560in}}%
\pgfpathcurveto{\pgfqpoint{2.113414in}{2.179384in}}{\pgfqpoint{2.105514in}{2.182656in}}{\pgfqpoint{2.097278in}{2.182656in}}%
\pgfpathcurveto{\pgfqpoint{2.089042in}{2.182656in}}{\pgfqpoint{2.081142in}{2.179384in}}{\pgfqpoint{2.075318in}{2.173560in}}%
\pgfpathcurveto{\pgfqpoint{2.069494in}{2.167736in}}{\pgfqpoint{2.066222in}{2.159836in}}{\pgfqpoint{2.066222in}{2.151600in}}%
\pgfpathcurveto{\pgfqpoint{2.066222in}{2.143364in}}{\pgfqpoint{2.069494in}{2.135463in}}{\pgfqpoint{2.075318in}{2.129640in}}%
\pgfpathcurveto{\pgfqpoint{2.081142in}{2.123816in}}{\pgfqpoint{2.089042in}{2.120543in}}{\pgfqpoint{2.097278in}{2.120543in}}%
\pgfpathclose%
\pgfusepath{stroke,fill}%
\end{pgfscope}%
\begin{pgfscope}%
\pgfpathrectangle{\pgfqpoint{0.100000in}{0.212622in}}{\pgfqpoint{3.696000in}{3.696000in}}%
\pgfusepath{clip}%
\pgfsetbuttcap%
\pgfsetroundjoin%
\definecolor{currentfill}{rgb}{0.121569,0.466667,0.705882}%
\pgfsetfillcolor{currentfill}%
\pgfsetfillopacity{0.593795}%
\pgfsetlinewidth{1.003750pt}%
\definecolor{currentstroke}{rgb}{0.121569,0.466667,0.705882}%
\pgfsetstrokecolor{currentstroke}%
\pgfsetstrokeopacity{0.593795}%
\pgfsetdash{}{0pt}%
\pgfpathmoveto{\pgfqpoint{0.864999in}{1.580955in}}%
\pgfpathcurveto{\pgfqpoint{0.873235in}{1.580955in}}{\pgfqpoint{0.881135in}{1.584228in}}{\pgfqpoint{0.886959in}{1.590052in}}%
\pgfpathcurveto{\pgfqpoint{0.892783in}{1.595875in}}{\pgfqpoint{0.896055in}{1.603776in}}{\pgfqpoint{0.896055in}{1.612012in}}%
\pgfpathcurveto{\pgfqpoint{0.896055in}{1.620248in}}{\pgfqpoint{0.892783in}{1.628148in}}{\pgfqpoint{0.886959in}{1.633972in}}%
\pgfpathcurveto{\pgfqpoint{0.881135in}{1.639796in}}{\pgfqpoint{0.873235in}{1.643068in}}{\pgfqpoint{0.864999in}{1.643068in}}%
\pgfpathcurveto{\pgfqpoint{0.856763in}{1.643068in}}{\pgfqpoint{0.848863in}{1.639796in}}{\pgfqpoint{0.843039in}{1.633972in}}%
\pgfpathcurveto{\pgfqpoint{0.837215in}{1.628148in}}{\pgfqpoint{0.833942in}{1.620248in}}{\pgfqpoint{0.833942in}{1.612012in}}%
\pgfpathcurveto{\pgfqpoint{0.833942in}{1.603776in}}{\pgfqpoint{0.837215in}{1.595875in}}{\pgfqpoint{0.843039in}{1.590052in}}%
\pgfpathcurveto{\pgfqpoint{0.848863in}{1.584228in}}{\pgfqpoint{0.856763in}{1.580955in}}{\pgfqpoint{0.864999in}{1.580955in}}%
\pgfpathclose%
\pgfusepath{stroke,fill}%
\end{pgfscope}%
\begin{pgfscope}%
\pgfpathrectangle{\pgfqpoint{0.100000in}{0.212622in}}{\pgfqpoint{3.696000in}{3.696000in}}%
\pgfusepath{clip}%
\pgfsetbuttcap%
\pgfsetroundjoin%
\definecolor{currentfill}{rgb}{0.121569,0.466667,0.705882}%
\pgfsetfillcolor{currentfill}%
\pgfsetfillopacity{0.593807}%
\pgfsetlinewidth{1.003750pt}%
\definecolor{currentstroke}{rgb}{0.121569,0.466667,0.705882}%
\pgfsetstrokecolor{currentstroke}%
\pgfsetstrokeopacity{0.593807}%
\pgfsetdash{}{0pt}%
\pgfpathmoveto{\pgfqpoint{0.880557in}{1.485399in}}%
\pgfpathcurveto{\pgfqpoint{0.888793in}{1.485399in}}{\pgfqpoint{0.896693in}{1.488672in}}{\pgfqpoint{0.902517in}{1.494496in}}%
\pgfpathcurveto{\pgfqpoint{0.908341in}{1.500319in}}{\pgfqpoint{0.911614in}{1.508220in}}{\pgfqpoint{0.911614in}{1.516456in}}%
\pgfpathcurveto{\pgfqpoint{0.911614in}{1.524692in}}{\pgfqpoint{0.908341in}{1.532592in}}{\pgfqpoint{0.902517in}{1.538416in}}%
\pgfpathcurveto{\pgfqpoint{0.896693in}{1.544240in}}{\pgfqpoint{0.888793in}{1.547512in}}{\pgfqpoint{0.880557in}{1.547512in}}%
\pgfpathcurveto{\pgfqpoint{0.872321in}{1.547512in}}{\pgfqpoint{0.864421in}{1.544240in}}{\pgfqpoint{0.858597in}{1.538416in}}%
\pgfpathcurveto{\pgfqpoint{0.852773in}{1.532592in}}{\pgfqpoint{0.849501in}{1.524692in}}{\pgfqpoint{0.849501in}{1.516456in}}%
\pgfpathcurveto{\pgfqpoint{0.849501in}{1.508220in}}{\pgfqpoint{0.852773in}{1.500319in}}{\pgfqpoint{0.858597in}{1.494496in}}%
\pgfpathcurveto{\pgfqpoint{0.864421in}{1.488672in}}{\pgfqpoint{0.872321in}{1.485399in}}{\pgfqpoint{0.880557in}{1.485399in}}%
\pgfpathclose%
\pgfusepath{stroke,fill}%
\end{pgfscope}%
\begin{pgfscope}%
\pgfpathrectangle{\pgfqpoint{0.100000in}{0.212622in}}{\pgfqpoint{3.696000in}{3.696000in}}%
\pgfusepath{clip}%
\pgfsetbuttcap%
\pgfsetroundjoin%
\definecolor{currentfill}{rgb}{0.121569,0.466667,0.705882}%
\pgfsetfillcolor{currentfill}%
\pgfsetfillopacity{0.594461}%
\pgfsetlinewidth{1.003750pt}%
\definecolor{currentstroke}{rgb}{0.121569,0.466667,0.705882}%
\pgfsetstrokecolor{currentstroke}%
\pgfsetstrokeopacity{0.594461}%
\pgfsetdash{}{0pt}%
\pgfpathmoveto{\pgfqpoint{2.098335in}{2.114744in}}%
\pgfpathcurveto{\pgfqpoint{2.106572in}{2.114744in}}{\pgfqpoint{2.114472in}{2.118016in}}{\pgfqpoint{2.120296in}{2.123840in}}%
\pgfpathcurveto{\pgfqpoint{2.126120in}{2.129664in}}{\pgfqpoint{2.129392in}{2.137564in}}{\pgfqpoint{2.129392in}{2.145800in}}%
\pgfpathcurveto{\pgfqpoint{2.129392in}{2.154036in}}{\pgfqpoint{2.126120in}{2.161936in}}{\pgfqpoint{2.120296in}{2.167760in}}%
\pgfpathcurveto{\pgfqpoint{2.114472in}{2.173584in}}{\pgfqpoint{2.106572in}{2.176857in}}{\pgfqpoint{2.098335in}{2.176857in}}%
\pgfpathcurveto{\pgfqpoint{2.090099in}{2.176857in}}{\pgfqpoint{2.082199in}{2.173584in}}{\pgfqpoint{2.076375in}{2.167760in}}%
\pgfpathcurveto{\pgfqpoint{2.070551in}{2.161936in}}{\pgfqpoint{2.067279in}{2.154036in}}{\pgfqpoint{2.067279in}{2.145800in}}%
\pgfpathcurveto{\pgfqpoint{2.067279in}{2.137564in}}{\pgfqpoint{2.070551in}{2.129664in}}{\pgfqpoint{2.076375in}{2.123840in}}%
\pgfpathcurveto{\pgfqpoint{2.082199in}{2.118016in}}{\pgfqpoint{2.090099in}{2.114744in}}{\pgfqpoint{2.098335in}{2.114744in}}%
\pgfpathclose%
\pgfusepath{stroke,fill}%
\end{pgfscope}%
\begin{pgfscope}%
\pgfpathrectangle{\pgfqpoint{0.100000in}{0.212622in}}{\pgfqpoint{3.696000in}{3.696000in}}%
\pgfusepath{clip}%
\pgfsetbuttcap%
\pgfsetroundjoin%
\definecolor{currentfill}{rgb}{0.121569,0.466667,0.705882}%
\pgfsetfillcolor{currentfill}%
\pgfsetfillopacity{0.594764}%
\pgfsetlinewidth{1.003750pt}%
\definecolor{currentstroke}{rgb}{0.121569,0.466667,0.705882}%
\pgfsetstrokecolor{currentstroke}%
\pgfsetstrokeopacity{0.594764}%
\pgfsetdash{}{0pt}%
\pgfpathmoveto{\pgfqpoint{0.970980in}{1.704764in}}%
\pgfpathcurveto{\pgfqpoint{0.979216in}{1.704764in}}{\pgfqpoint{0.987116in}{1.708036in}}{\pgfqpoint{0.992940in}{1.713860in}}%
\pgfpathcurveto{\pgfqpoint{0.998764in}{1.719684in}}{\pgfqpoint{1.002036in}{1.727584in}}{\pgfqpoint{1.002036in}{1.735821in}}%
\pgfpathcurveto{\pgfqpoint{1.002036in}{1.744057in}}{\pgfqpoint{0.998764in}{1.751957in}}{\pgfqpoint{0.992940in}{1.757781in}}%
\pgfpathcurveto{\pgfqpoint{0.987116in}{1.763605in}}{\pgfqpoint{0.979216in}{1.766877in}}{\pgfqpoint{0.970980in}{1.766877in}}%
\pgfpathcurveto{\pgfqpoint{0.962744in}{1.766877in}}{\pgfqpoint{0.954844in}{1.763605in}}{\pgfqpoint{0.949020in}{1.757781in}}%
\pgfpathcurveto{\pgfqpoint{0.943196in}{1.751957in}}{\pgfqpoint{0.939923in}{1.744057in}}{\pgfqpoint{0.939923in}{1.735821in}}%
\pgfpathcurveto{\pgfqpoint{0.939923in}{1.727584in}}{\pgfqpoint{0.943196in}{1.719684in}}{\pgfqpoint{0.949020in}{1.713860in}}%
\pgfpathcurveto{\pgfqpoint{0.954844in}{1.708036in}}{\pgfqpoint{0.962744in}{1.704764in}}{\pgfqpoint{0.970980in}{1.704764in}}%
\pgfpathclose%
\pgfusepath{stroke,fill}%
\end{pgfscope}%
\begin{pgfscope}%
\pgfpathrectangle{\pgfqpoint{0.100000in}{0.212622in}}{\pgfqpoint{3.696000in}{3.696000in}}%
\pgfusepath{clip}%
\pgfsetbuttcap%
\pgfsetroundjoin%
\definecolor{currentfill}{rgb}{0.121569,0.466667,0.705882}%
\pgfsetfillcolor{currentfill}%
\pgfsetfillopacity{0.596202}%
\pgfsetlinewidth{1.003750pt}%
\definecolor{currentstroke}{rgb}{0.121569,0.466667,0.705882}%
\pgfsetstrokecolor{currentstroke}%
\pgfsetstrokeopacity{0.596202}%
\pgfsetdash{}{0pt}%
\pgfpathmoveto{\pgfqpoint{2.099084in}{2.108186in}}%
\pgfpathcurveto{\pgfqpoint{2.107321in}{2.108186in}}{\pgfqpoint{2.115221in}{2.111458in}}{\pgfqpoint{2.121044in}{2.117282in}}%
\pgfpathcurveto{\pgfqpoint{2.126868in}{2.123106in}}{\pgfqpoint{2.130141in}{2.131006in}}{\pgfqpoint{2.130141in}{2.139242in}}%
\pgfpathcurveto{\pgfqpoint{2.130141in}{2.147478in}}{\pgfqpoint{2.126868in}{2.155378in}}{\pgfqpoint{2.121044in}{2.161202in}}%
\pgfpathcurveto{\pgfqpoint{2.115221in}{2.167026in}}{\pgfqpoint{2.107321in}{2.170299in}}{\pgfqpoint{2.099084in}{2.170299in}}%
\pgfpathcurveto{\pgfqpoint{2.090848in}{2.170299in}}{\pgfqpoint{2.082948in}{2.167026in}}{\pgfqpoint{2.077124in}{2.161202in}}%
\pgfpathcurveto{\pgfqpoint{2.071300in}{2.155378in}}{\pgfqpoint{2.068028in}{2.147478in}}{\pgfqpoint{2.068028in}{2.139242in}}%
\pgfpathcurveto{\pgfqpoint{2.068028in}{2.131006in}}{\pgfqpoint{2.071300in}{2.123106in}}{\pgfqpoint{2.077124in}{2.117282in}}%
\pgfpathcurveto{\pgfqpoint{2.082948in}{2.111458in}}{\pgfqpoint{2.090848in}{2.108186in}}{\pgfqpoint{2.099084in}{2.108186in}}%
\pgfpathclose%
\pgfusepath{stroke,fill}%
\end{pgfscope}%
\begin{pgfscope}%
\pgfpathrectangle{\pgfqpoint{0.100000in}{0.212622in}}{\pgfqpoint{3.696000in}{3.696000in}}%
\pgfusepath{clip}%
\pgfsetbuttcap%
\pgfsetroundjoin%
\definecolor{currentfill}{rgb}{0.121569,0.466667,0.705882}%
\pgfsetfillcolor{currentfill}%
\pgfsetfillopacity{0.596318}%
\pgfsetlinewidth{1.003750pt}%
\definecolor{currentstroke}{rgb}{0.121569,0.466667,0.705882}%
\pgfsetstrokecolor{currentstroke}%
\pgfsetstrokeopacity{0.596318}%
\pgfsetdash{}{0pt}%
\pgfpathmoveto{\pgfqpoint{0.965404in}{1.695082in}}%
\pgfpathcurveto{\pgfqpoint{0.973641in}{1.695082in}}{\pgfqpoint{0.981541in}{1.698354in}}{\pgfqpoint{0.987365in}{1.704178in}}%
\pgfpathcurveto{\pgfqpoint{0.993189in}{1.710002in}}{\pgfqpoint{0.996461in}{1.717902in}}{\pgfqpoint{0.996461in}{1.726138in}}%
\pgfpathcurveto{\pgfqpoint{0.996461in}{1.734375in}}{\pgfqpoint{0.993189in}{1.742275in}}{\pgfqpoint{0.987365in}{1.748099in}}%
\pgfpathcurveto{\pgfqpoint{0.981541in}{1.753922in}}{\pgfqpoint{0.973641in}{1.757195in}}{\pgfqpoint{0.965404in}{1.757195in}}%
\pgfpathcurveto{\pgfqpoint{0.957168in}{1.757195in}}{\pgfqpoint{0.949268in}{1.753922in}}{\pgfqpoint{0.943444in}{1.748099in}}%
\pgfpathcurveto{\pgfqpoint{0.937620in}{1.742275in}}{\pgfqpoint{0.934348in}{1.734375in}}{\pgfqpoint{0.934348in}{1.726138in}}%
\pgfpathcurveto{\pgfqpoint{0.934348in}{1.717902in}}{\pgfqpoint{0.937620in}{1.710002in}}{\pgfqpoint{0.943444in}{1.704178in}}%
\pgfpathcurveto{\pgfqpoint{0.949268in}{1.698354in}}{\pgfqpoint{0.957168in}{1.695082in}}{\pgfqpoint{0.965404in}{1.695082in}}%
\pgfpathclose%
\pgfusepath{stroke,fill}%
\end{pgfscope}%
\begin{pgfscope}%
\pgfpathrectangle{\pgfqpoint{0.100000in}{0.212622in}}{\pgfqpoint{3.696000in}{3.696000in}}%
\pgfusepath{clip}%
\pgfsetbuttcap%
\pgfsetroundjoin%
\definecolor{currentfill}{rgb}{0.121569,0.466667,0.705882}%
\pgfsetfillcolor{currentfill}%
\pgfsetfillopacity{0.597441}%
\pgfsetlinewidth{1.003750pt}%
\definecolor{currentstroke}{rgb}{0.121569,0.466667,0.705882}%
\pgfsetstrokecolor{currentstroke}%
\pgfsetstrokeopacity{0.597441}%
\pgfsetdash{}{0pt}%
\pgfpathmoveto{\pgfqpoint{0.620771in}{1.211398in}}%
\pgfpathcurveto{\pgfqpoint{0.629007in}{1.211398in}}{\pgfqpoint{0.636907in}{1.214670in}}{\pgfqpoint{0.642731in}{1.220494in}}%
\pgfpathcurveto{\pgfqpoint{0.648555in}{1.226318in}}{\pgfqpoint{0.651827in}{1.234218in}}{\pgfqpoint{0.651827in}{1.242454in}}%
\pgfpathcurveto{\pgfqpoint{0.651827in}{1.250691in}}{\pgfqpoint{0.648555in}{1.258591in}}{\pgfqpoint{0.642731in}{1.264415in}}%
\pgfpathcurveto{\pgfqpoint{0.636907in}{1.270239in}}{\pgfqpoint{0.629007in}{1.273511in}}{\pgfqpoint{0.620771in}{1.273511in}}%
\pgfpathcurveto{\pgfqpoint{0.612535in}{1.273511in}}{\pgfqpoint{0.604635in}{1.270239in}}{\pgfqpoint{0.598811in}{1.264415in}}%
\pgfpathcurveto{\pgfqpoint{0.592987in}{1.258591in}}{\pgfqpoint{0.589714in}{1.250691in}}{\pgfqpoint{0.589714in}{1.242454in}}%
\pgfpathcurveto{\pgfqpoint{0.589714in}{1.234218in}}{\pgfqpoint{0.592987in}{1.226318in}}{\pgfqpoint{0.598811in}{1.220494in}}%
\pgfpathcurveto{\pgfqpoint{0.604635in}{1.214670in}}{\pgfqpoint{0.612535in}{1.211398in}}{\pgfqpoint{0.620771in}{1.211398in}}%
\pgfpathclose%
\pgfusepath{stroke,fill}%
\end{pgfscope}%
\begin{pgfscope}%
\pgfpathrectangle{\pgfqpoint{0.100000in}{0.212622in}}{\pgfqpoint{3.696000in}{3.696000in}}%
\pgfusepath{clip}%
\pgfsetbuttcap%
\pgfsetroundjoin%
\definecolor{currentfill}{rgb}{0.121569,0.466667,0.705882}%
\pgfsetfillcolor{currentfill}%
\pgfsetfillopacity{0.597478}%
\pgfsetlinewidth{1.003750pt}%
\definecolor{currentstroke}{rgb}{0.121569,0.466667,0.705882}%
\pgfsetstrokecolor{currentstroke}%
\pgfsetstrokeopacity{0.597478}%
\pgfsetdash{}{0pt}%
\pgfpathmoveto{\pgfqpoint{0.877299in}{1.485991in}}%
\pgfpathcurveto{\pgfqpoint{0.885536in}{1.485991in}}{\pgfqpoint{0.893436in}{1.489263in}}{\pgfqpoint{0.899260in}{1.495087in}}%
\pgfpathcurveto{\pgfqpoint{0.905084in}{1.500911in}}{\pgfqpoint{0.908356in}{1.508811in}}{\pgfqpoint{0.908356in}{1.517047in}}%
\pgfpathcurveto{\pgfqpoint{0.908356in}{1.525284in}}{\pgfqpoint{0.905084in}{1.533184in}}{\pgfqpoint{0.899260in}{1.539008in}}%
\pgfpathcurveto{\pgfqpoint{0.893436in}{1.544832in}}{\pgfqpoint{0.885536in}{1.548104in}}{\pgfqpoint{0.877299in}{1.548104in}}%
\pgfpathcurveto{\pgfqpoint{0.869063in}{1.548104in}}{\pgfqpoint{0.861163in}{1.544832in}}{\pgfqpoint{0.855339in}{1.539008in}}%
\pgfpathcurveto{\pgfqpoint{0.849515in}{1.533184in}}{\pgfqpoint{0.846243in}{1.525284in}}{\pgfqpoint{0.846243in}{1.517047in}}%
\pgfpathcurveto{\pgfqpoint{0.846243in}{1.508811in}}{\pgfqpoint{0.849515in}{1.500911in}}{\pgfqpoint{0.855339in}{1.495087in}}%
\pgfpathcurveto{\pgfqpoint{0.861163in}{1.489263in}}{\pgfqpoint{0.869063in}{1.485991in}}{\pgfqpoint{0.877299in}{1.485991in}}%
\pgfpathclose%
\pgfusepath{stroke,fill}%
\end{pgfscope}%
\begin{pgfscope}%
\pgfpathrectangle{\pgfqpoint{0.100000in}{0.212622in}}{\pgfqpoint{3.696000in}{3.696000in}}%
\pgfusepath{clip}%
\pgfsetbuttcap%
\pgfsetroundjoin%
\definecolor{currentfill}{rgb}{0.121569,0.466667,0.705882}%
\pgfsetfillcolor{currentfill}%
\pgfsetfillopacity{0.597742}%
\pgfsetlinewidth{1.003750pt}%
\definecolor{currentstroke}{rgb}{0.121569,0.466667,0.705882}%
\pgfsetstrokecolor{currentstroke}%
\pgfsetstrokeopacity{0.597742}%
\pgfsetdash{}{0pt}%
\pgfpathmoveto{\pgfqpoint{0.961284in}{1.686511in}}%
\pgfpathcurveto{\pgfqpoint{0.969520in}{1.686511in}}{\pgfqpoint{0.977420in}{1.689784in}}{\pgfqpoint{0.983244in}{1.695607in}}%
\pgfpathcurveto{\pgfqpoint{0.989068in}{1.701431in}}{\pgfqpoint{0.992340in}{1.709331in}}{\pgfqpoint{0.992340in}{1.717568in}}%
\pgfpathcurveto{\pgfqpoint{0.992340in}{1.725804in}}{\pgfqpoint{0.989068in}{1.733704in}}{\pgfqpoint{0.983244in}{1.739528in}}%
\pgfpathcurveto{\pgfqpoint{0.977420in}{1.745352in}}{\pgfqpoint{0.969520in}{1.748624in}}{\pgfqpoint{0.961284in}{1.748624in}}%
\pgfpathcurveto{\pgfqpoint{0.953047in}{1.748624in}}{\pgfqpoint{0.945147in}{1.745352in}}{\pgfqpoint{0.939323in}{1.739528in}}%
\pgfpathcurveto{\pgfqpoint{0.933499in}{1.733704in}}{\pgfqpoint{0.930227in}{1.725804in}}{\pgfqpoint{0.930227in}{1.717568in}}%
\pgfpathcurveto{\pgfqpoint{0.930227in}{1.709331in}}{\pgfqpoint{0.933499in}{1.701431in}}{\pgfqpoint{0.939323in}{1.695607in}}%
\pgfpathcurveto{\pgfqpoint{0.945147in}{1.689784in}}{\pgfqpoint{0.953047in}{1.686511in}}{\pgfqpoint{0.961284in}{1.686511in}}%
\pgfpathclose%
\pgfusepath{stroke,fill}%
\end{pgfscope}%
\begin{pgfscope}%
\pgfpathrectangle{\pgfqpoint{0.100000in}{0.212622in}}{\pgfqpoint{3.696000in}{3.696000in}}%
\pgfusepath{clip}%
\pgfsetbuttcap%
\pgfsetroundjoin%
\definecolor{currentfill}{rgb}{0.121569,0.466667,0.705882}%
\pgfsetfillcolor{currentfill}%
\pgfsetfillopacity{0.598465}%
\pgfsetlinewidth{1.003750pt}%
\definecolor{currentstroke}{rgb}{0.121569,0.466667,0.705882}%
\pgfsetstrokecolor{currentstroke}%
\pgfsetstrokeopacity{0.598465}%
\pgfsetdash{}{0pt}%
\pgfpathmoveto{\pgfqpoint{2.101109in}{2.098783in}}%
\pgfpathcurveto{\pgfqpoint{2.109345in}{2.098783in}}{\pgfqpoint{2.117245in}{2.102055in}}{\pgfqpoint{2.123069in}{2.107879in}}%
\pgfpathcurveto{\pgfqpoint{2.128893in}{2.113703in}}{\pgfqpoint{2.132165in}{2.121603in}}{\pgfqpoint{2.132165in}{2.129839in}}%
\pgfpathcurveto{\pgfqpoint{2.132165in}{2.138075in}}{\pgfqpoint{2.128893in}{2.145976in}}{\pgfqpoint{2.123069in}{2.151799in}}%
\pgfpathcurveto{\pgfqpoint{2.117245in}{2.157623in}}{\pgfqpoint{2.109345in}{2.160896in}}{\pgfqpoint{2.101109in}{2.160896in}}%
\pgfpathcurveto{\pgfqpoint{2.092873in}{2.160896in}}{\pgfqpoint{2.084973in}{2.157623in}}{\pgfqpoint{2.079149in}{2.151799in}}%
\pgfpathcurveto{\pgfqpoint{2.073325in}{2.145976in}}{\pgfqpoint{2.070052in}{2.138075in}}{\pgfqpoint{2.070052in}{2.129839in}}%
\pgfpathcurveto{\pgfqpoint{2.070052in}{2.121603in}}{\pgfqpoint{2.073325in}{2.113703in}}{\pgfqpoint{2.079149in}{2.107879in}}%
\pgfpathcurveto{\pgfqpoint{2.084973in}{2.102055in}}{\pgfqpoint{2.092873in}{2.098783in}}{\pgfqpoint{2.101109in}{2.098783in}}%
\pgfpathclose%
\pgfusepath{stroke,fill}%
\end{pgfscope}%
\begin{pgfscope}%
\pgfpathrectangle{\pgfqpoint{0.100000in}{0.212622in}}{\pgfqpoint{3.696000in}{3.696000in}}%
\pgfusepath{clip}%
\pgfsetbuttcap%
\pgfsetroundjoin%
\definecolor{currentfill}{rgb}{0.121569,0.466667,0.705882}%
\pgfsetfillcolor{currentfill}%
\pgfsetfillopacity{0.598632}%
\pgfsetlinewidth{1.003750pt}%
\definecolor{currentstroke}{rgb}{0.121569,0.466667,0.705882}%
\pgfsetstrokecolor{currentstroke}%
\pgfsetstrokeopacity{0.598632}%
\pgfsetdash{}{0pt}%
\pgfpathmoveto{\pgfqpoint{0.957958in}{1.680535in}}%
\pgfpathcurveto{\pgfqpoint{0.966195in}{1.680535in}}{\pgfqpoint{0.974095in}{1.683807in}}{\pgfqpoint{0.979919in}{1.689631in}}%
\pgfpathcurveto{\pgfqpoint{0.985742in}{1.695455in}}{\pgfqpoint{0.989015in}{1.703355in}}{\pgfqpoint{0.989015in}{1.711592in}}%
\pgfpathcurveto{\pgfqpoint{0.989015in}{1.719828in}}{\pgfqpoint{0.985742in}{1.727728in}}{\pgfqpoint{0.979919in}{1.733552in}}%
\pgfpathcurveto{\pgfqpoint{0.974095in}{1.739376in}}{\pgfqpoint{0.966195in}{1.742648in}}{\pgfqpoint{0.957958in}{1.742648in}}%
\pgfpathcurveto{\pgfqpoint{0.949722in}{1.742648in}}{\pgfqpoint{0.941822in}{1.739376in}}{\pgfqpoint{0.935998in}{1.733552in}}%
\pgfpathcurveto{\pgfqpoint{0.930174in}{1.727728in}}{\pgfqpoint{0.926902in}{1.719828in}}{\pgfqpoint{0.926902in}{1.711592in}}%
\pgfpathcurveto{\pgfqpoint{0.926902in}{1.703355in}}{\pgfqpoint{0.930174in}{1.695455in}}{\pgfqpoint{0.935998in}{1.689631in}}%
\pgfpathcurveto{\pgfqpoint{0.941822in}{1.683807in}}{\pgfqpoint{0.949722in}{1.680535in}}{\pgfqpoint{0.957958in}{1.680535in}}%
\pgfpathclose%
\pgfusepath{stroke,fill}%
\end{pgfscope}%
\begin{pgfscope}%
\pgfpathrectangle{\pgfqpoint{0.100000in}{0.212622in}}{\pgfqpoint{3.696000in}{3.696000in}}%
\pgfusepath{clip}%
\pgfsetbuttcap%
\pgfsetroundjoin%
\definecolor{currentfill}{rgb}{0.121569,0.466667,0.705882}%
\pgfsetfillcolor{currentfill}%
\pgfsetfillopacity{0.599384}%
\pgfsetlinewidth{1.003750pt}%
\definecolor{currentstroke}{rgb}{0.121569,0.466667,0.705882}%
\pgfsetstrokecolor{currentstroke}%
\pgfsetstrokeopacity{0.599384}%
\pgfsetdash{}{0pt}%
\pgfpathmoveto{\pgfqpoint{0.955971in}{1.675791in}}%
\pgfpathcurveto{\pgfqpoint{0.964207in}{1.675791in}}{\pgfqpoint{0.972107in}{1.679063in}}{\pgfqpoint{0.977931in}{1.684887in}}%
\pgfpathcurveto{\pgfqpoint{0.983755in}{1.690711in}}{\pgfqpoint{0.987028in}{1.698611in}}{\pgfqpoint{0.987028in}{1.706848in}}%
\pgfpathcurveto{\pgfqpoint{0.987028in}{1.715084in}}{\pgfqpoint{0.983755in}{1.722984in}}{\pgfqpoint{0.977931in}{1.728808in}}%
\pgfpathcurveto{\pgfqpoint{0.972107in}{1.734632in}}{\pgfqpoint{0.964207in}{1.737904in}}{\pgfqpoint{0.955971in}{1.737904in}}%
\pgfpathcurveto{\pgfqpoint{0.947735in}{1.737904in}}{\pgfqpoint{0.939835in}{1.734632in}}{\pgfqpoint{0.934011in}{1.728808in}}%
\pgfpathcurveto{\pgfqpoint{0.928187in}{1.722984in}}{\pgfqpoint{0.924915in}{1.715084in}}{\pgfqpoint{0.924915in}{1.706848in}}%
\pgfpathcurveto{\pgfqpoint{0.924915in}{1.698611in}}{\pgfqpoint{0.928187in}{1.690711in}}{\pgfqpoint{0.934011in}{1.684887in}}%
\pgfpathcurveto{\pgfqpoint{0.939835in}{1.679063in}}{\pgfqpoint{0.947735in}{1.675791in}}{\pgfqpoint{0.955971in}{1.675791in}}%
\pgfpathclose%
\pgfusepath{stroke,fill}%
\end{pgfscope}%
\begin{pgfscope}%
\pgfpathrectangle{\pgfqpoint{0.100000in}{0.212622in}}{\pgfqpoint{3.696000in}{3.696000in}}%
\pgfusepath{clip}%
\pgfsetbuttcap%
\pgfsetroundjoin%
\definecolor{currentfill}{rgb}{0.121569,0.466667,0.705882}%
\pgfsetfillcolor{currentfill}%
\pgfsetfillopacity{0.599699}%
\pgfsetlinewidth{1.003750pt}%
\definecolor{currentstroke}{rgb}{0.121569,0.466667,0.705882}%
\pgfsetstrokecolor{currentstroke}%
\pgfsetstrokeopacity{0.599699}%
\pgfsetdash{}{0pt}%
\pgfpathmoveto{\pgfqpoint{0.954842in}{1.673818in}}%
\pgfpathcurveto{\pgfqpoint{0.963078in}{1.673818in}}{\pgfqpoint{0.970978in}{1.677090in}}{\pgfqpoint{0.976802in}{1.682914in}}%
\pgfpathcurveto{\pgfqpoint{0.982626in}{1.688738in}}{\pgfqpoint{0.985899in}{1.696638in}}{\pgfqpoint{0.985899in}{1.704874in}}%
\pgfpathcurveto{\pgfqpoint{0.985899in}{1.713111in}}{\pgfqpoint{0.982626in}{1.721011in}}{\pgfqpoint{0.976802in}{1.726835in}}%
\pgfpathcurveto{\pgfqpoint{0.970978in}{1.732659in}}{\pgfqpoint{0.963078in}{1.735931in}}{\pgfqpoint{0.954842in}{1.735931in}}%
\pgfpathcurveto{\pgfqpoint{0.946606in}{1.735931in}}{\pgfqpoint{0.938706in}{1.732659in}}{\pgfqpoint{0.932882in}{1.726835in}}%
\pgfpathcurveto{\pgfqpoint{0.927058in}{1.721011in}}{\pgfqpoint{0.923786in}{1.713111in}}{\pgfqpoint{0.923786in}{1.704874in}}%
\pgfpathcurveto{\pgfqpoint{0.923786in}{1.696638in}}{\pgfqpoint{0.927058in}{1.688738in}}{\pgfqpoint{0.932882in}{1.682914in}}%
\pgfpathcurveto{\pgfqpoint{0.938706in}{1.677090in}}{\pgfqpoint{0.946606in}{1.673818in}}{\pgfqpoint{0.954842in}{1.673818in}}%
\pgfpathclose%
\pgfusepath{stroke,fill}%
\end{pgfscope}%
\begin{pgfscope}%
\pgfpathrectangle{\pgfqpoint{0.100000in}{0.212622in}}{\pgfqpoint{3.696000in}{3.696000in}}%
\pgfusepath{clip}%
\pgfsetbuttcap%
\pgfsetroundjoin%
\definecolor{currentfill}{rgb}{0.121569,0.466667,0.705882}%
\pgfsetfillcolor{currentfill}%
\pgfsetfillopacity{0.599768}%
\pgfsetlinewidth{1.003750pt}%
\definecolor{currentstroke}{rgb}{0.121569,0.466667,0.705882}%
\pgfsetstrokecolor{currentstroke}%
\pgfsetstrokeopacity{0.599768}%
\pgfsetdash{}{0pt}%
\pgfpathmoveto{\pgfqpoint{2.102087in}{2.093692in}}%
\pgfpathcurveto{\pgfqpoint{2.110323in}{2.093692in}}{\pgfqpoint{2.118223in}{2.096964in}}{\pgfqpoint{2.124047in}{2.102788in}}%
\pgfpathcurveto{\pgfqpoint{2.129871in}{2.108612in}}{\pgfqpoint{2.133144in}{2.116512in}}{\pgfqpoint{2.133144in}{2.124748in}}%
\pgfpathcurveto{\pgfqpoint{2.133144in}{2.132985in}}{\pgfqpoint{2.129871in}{2.140885in}}{\pgfqpoint{2.124047in}{2.146709in}}%
\pgfpathcurveto{\pgfqpoint{2.118223in}{2.152532in}}{\pgfqpoint{2.110323in}{2.155805in}}{\pgfqpoint{2.102087in}{2.155805in}}%
\pgfpathcurveto{\pgfqpoint{2.093851in}{2.155805in}}{\pgfqpoint{2.085951in}{2.152532in}}{\pgfqpoint{2.080127in}{2.146709in}}%
\pgfpathcurveto{\pgfqpoint{2.074303in}{2.140885in}}{\pgfqpoint{2.071031in}{2.132985in}}{\pgfqpoint{2.071031in}{2.124748in}}%
\pgfpathcurveto{\pgfqpoint{2.071031in}{2.116512in}}{\pgfqpoint{2.074303in}{2.108612in}}{\pgfqpoint{2.080127in}{2.102788in}}%
\pgfpathcurveto{\pgfqpoint{2.085951in}{2.096964in}}{\pgfqpoint{2.093851in}{2.093692in}}{\pgfqpoint{2.102087in}{2.093692in}}%
\pgfpathclose%
\pgfusepath{stroke,fill}%
\end{pgfscope}%
\begin{pgfscope}%
\pgfpathrectangle{\pgfqpoint{0.100000in}{0.212622in}}{\pgfqpoint{3.696000in}{3.696000in}}%
\pgfusepath{clip}%
\pgfsetbuttcap%
\pgfsetroundjoin%
\definecolor{currentfill}{rgb}{0.121569,0.466667,0.705882}%
\pgfsetfillcolor{currentfill}%
\pgfsetfillopacity{0.599831}%
\pgfsetlinewidth{1.003750pt}%
\definecolor{currentstroke}{rgb}{0.121569,0.466667,0.705882}%
\pgfsetstrokecolor{currentstroke}%
\pgfsetstrokeopacity{0.599831}%
\pgfsetdash{}{0pt}%
\pgfpathmoveto{\pgfqpoint{0.954443in}{1.672953in}}%
\pgfpathcurveto{\pgfqpoint{0.962680in}{1.672953in}}{\pgfqpoint{0.970580in}{1.676226in}}{\pgfqpoint{0.976404in}{1.682050in}}%
\pgfpathcurveto{\pgfqpoint{0.982228in}{1.687874in}}{\pgfqpoint{0.985500in}{1.695774in}}{\pgfqpoint{0.985500in}{1.704010in}}%
\pgfpathcurveto{\pgfqpoint{0.985500in}{1.712246in}}{\pgfqpoint{0.982228in}{1.720146in}}{\pgfqpoint{0.976404in}{1.725970in}}%
\pgfpathcurveto{\pgfqpoint{0.970580in}{1.731794in}}{\pgfqpoint{0.962680in}{1.735066in}}{\pgfqpoint{0.954443in}{1.735066in}}%
\pgfpathcurveto{\pgfqpoint{0.946207in}{1.735066in}}{\pgfqpoint{0.938307in}{1.731794in}}{\pgfqpoint{0.932483in}{1.725970in}}%
\pgfpathcurveto{\pgfqpoint{0.926659in}{1.720146in}}{\pgfqpoint{0.923387in}{1.712246in}}{\pgfqpoint{0.923387in}{1.704010in}}%
\pgfpathcurveto{\pgfqpoint{0.923387in}{1.695774in}}{\pgfqpoint{0.926659in}{1.687874in}}{\pgfqpoint{0.932483in}{1.682050in}}%
\pgfpathcurveto{\pgfqpoint{0.938307in}{1.676226in}}{\pgfqpoint{0.946207in}{1.672953in}}{\pgfqpoint{0.954443in}{1.672953in}}%
\pgfpathclose%
\pgfusepath{stroke,fill}%
\end{pgfscope}%
\begin{pgfscope}%
\pgfpathrectangle{\pgfqpoint{0.100000in}{0.212622in}}{\pgfqpoint{3.696000in}{3.696000in}}%
\pgfusepath{clip}%
\pgfsetbuttcap%
\pgfsetroundjoin%
\definecolor{currentfill}{rgb}{0.121569,0.466667,0.705882}%
\pgfsetfillcolor{currentfill}%
\pgfsetfillopacity{0.600050}%
\pgfsetlinewidth{1.003750pt}%
\definecolor{currentstroke}{rgb}{0.121569,0.466667,0.705882}%
\pgfsetstrokecolor{currentstroke}%
\pgfsetstrokeopacity{0.600050}%
\pgfsetdash{}{0pt}%
\pgfpathmoveto{\pgfqpoint{0.953629in}{1.671422in}}%
\pgfpathcurveto{\pgfqpoint{0.961865in}{1.671422in}}{\pgfqpoint{0.969765in}{1.674695in}}{\pgfqpoint{0.975589in}{1.680519in}}%
\pgfpathcurveto{\pgfqpoint{0.981413in}{1.686343in}}{\pgfqpoint{0.984686in}{1.694243in}}{\pgfqpoint{0.984686in}{1.702479in}}%
\pgfpathcurveto{\pgfqpoint{0.984686in}{1.710715in}}{\pgfqpoint{0.981413in}{1.718615in}}{\pgfqpoint{0.975589in}{1.724439in}}%
\pgfpathcurveto{\pgfqpoint{0.969765in}{1.730263in}}{\pgfqpoint{0.961865in}{1.733535in}}{\pgfqpoint{0.953629in}{1.733535in}}%
\pgfpathcurveto{\pgfqpoint{0.945393in}{1.733535in}}{\pgfqpoint{0.937493in}{1.730263in}}{\pgfqpoint{0.931669in}{1.724439in}}%
\pgfpathcurveto{\pgfqpoint{0.925845in}{1.718615in}}{\pgfqpoint{0.922573in}{1.710715in}}{\pgfqpoint{0.922573in}{1.702479in}}%
\pgfpathcurveto{\pgfqpoint{0.922573in}{1.694243in}}{\pgfqpoint{0.925845in}{1.686343in}}{\pgfqpoint{0.931669in}{1.680519in}}%
\pgfpathcurveto{\pgfqpoint{0.937493in}{1.674695in}}{\pgfqpoint{0.945393in}{1.671422in}}{\pgfqpoint{0.953629in}{1.671422in}}%
\pgfpathclose%
\pgfusepath{stroke,fill}%
\end{pgfscope}%
\begin{pgfscope}%
\pgfpathrectangle{\pgfqpoint{0.100000in}{0.212622in}}{\pgfqpoint{3.696000in}{3.696000in}}%
\pgfusepath{clip}%
\pgfsetbuttcap%
\pgfsetroundjoin%
\definecolor{currentfill}{rgb}{0.121569,0.466667,0.705882}%
\pgfsetfillcolor{currentfill}%
\pgfsetfillopacity{0.600204}%
\pgfsetlinewidth{1.003750pt}%
\definecolor{currentstroke}{rgb}{0.121569,0.466667,0.705882}%
\pgfsetstrokecolor{currentstroke}%
\pgfsetstrokeopacity{0.600204}%
\pgfsetdash{}{0pt}%
\pgfpathmoveto{\pgfqpoint{0.953275in}{1.670317in}}%
\pgfpathcurveto{\pgfqpoint{0.961512in}{1.670317in}}{\pgfqpoint{0.969412in}{1.673589in}}{\pgfqpoint{0.975236in}{1.679413in}}%
\pgfpathcurveto{\pgfqpoint{0.981060in}{1.685237in}}{\pgfqpoint{0.984332in}{1.693137in}}{\pgfqpoint{0.984332in}{1.701373in}}%
\pgfpathcurveto{\pgfqpoint{0.984332in}{1.709610in}}{\pgfqpoint{0.981060in}{1.717510in}}{\pgfqpoint{0.975236in}{1.723334in}}%
\pgfpathcurveto{\pgfqpoint{0.969412in}{1.729157in}}{\pgfqpoint{0.961512in}{1.732430in}}{\pgfqpoint{0.953275in}{1.732430in}}%
\pgfpathcurveto{\pgfqpoint{0.945039in}{1.732430in}}{\pgfqpoint{0.937139in}{1.729157in}}{\pgfqpoint{0.931315in}{1.723334in}}%
\pgfpathcurveto{\pgfqpoint{0.925491in}{1.717510in}}{\pgfqpoint{0.922219in}{1.709610in}}{\pgfqpoint{0.922219in}{1.701373in}}%
\pgfpathcurveto{\pgfqpoint{0.922219in}{1.693137in}}{\pgfqpoint{0.925491in}{1.685237in}}{\pgfqpoint{0.931315in}{1.679413in}}%
\pgfpathcurveto{\pgfqpoint{0.937139in}{1.673589in}}{\pgfqpoint{0.945039in}{1.670317in}}{\pgfqpoint{0.953275in}{1.670317in}}%
\pgfpathclose%
\pgfusepath{stroke,fill}%
\end{pgfscope}%
\begin{pgfscope}%
\pgfpathrectangle{\pgfqpoint{0.100000in}{0.212622in}}{\pgfqpoint{3.696000in}{3.696000in}}%
\pgfusepath{clip}%
\pgfsetbuttcap%
\pgfsetroundjoin%
\definecolor{currentfill}{rgb}{0.121569,0.466667,0.705882}%
\pgfsetfillcolor{currentfill}%
\pgfsetfillopacity{0.600541}%
\pgfsetlinewidth{1.003750pt}%
\definecolor{currentstroke}{rgb}{0.121569,0.466667,0.705882}%
\pgfsetstrokecolor{currentstroke}%
\pgfsetstrokeopacity{0.600541}%
\pgfsetdash{}{0pt}%
\pgfpathmoveto{\pgfqpoint{2.102511in}{2.091002in}}%
\pgfpathcurveto{\pgfqpoint{2.110747in}{2.091002in}}{\pgfqpoint{2.118647in}{2.094275in}}{\pgfqpoint{2.124471in}{2.100099in}}%
\pgfpathcurveto{\pgfqpoint{2.130295in}{2.105923in}}{\pgfqpoint{2.133567in}{2.113823in}}{\pgfqpoint{2.133567in}{2.122059in}}%
\pgfpathcurveto{\pgfqpoint{2.133567in}{2.130295in}}{\pgfqpoint{2.130295in}{2.138195in}}{\pgfqpoint{2.124471in}{2.144019in}}%
\pgfpathcurveto{\pgfqpoint{2.118647in}{2.149843in}}{\pgfqpoint{2.110747in}{2.153115in}}{\pgfqpoint{2.102511in}{2.153115in}}%
\pgfpathcurveto{\pgfqpoint{2.094274in}{2.153115in}}{\pgfqpoint{2.086374in}{2.149843in}}{\pgfqpoint{2.080550in}{2.144019in}}%
\pgfpathcurveto{\pgfqpoint{2.074726in}{2.138195in}}{\pgfqpoint{2.071454in}{2.130295in}}{\pgfqpoint{2.071454in}{2.122059in}}%
\pgfpathcurveto{\pgfqpoint{2.071454in}{2.113823in}}{\pgfqpoint{2.074726in}{2.105923in}}{\pgfqpoint{2.080550in}{2.100099in}}%
\pgfpathcurveto{\pgfqpoint{2.086374in}{2.094275in}}{\pgfqpoint{2.094274in}{2.091002in}}{\pgfqpoint{2.102511in}{2.091002in}}%
\pgfpathclose%
\pgfusepath{stroke,fill}%
\end{pgfscope}%
\begin{pgfscope}%
\pgfpathrectangle{\pgfqpoint{0.100000in}{0.212622in}}{\pgfqpoint{3.696000in}{3.696000in}}%
\pgfusepath{clip}%
\pgfsetbuttcap%
\pgfsetroundjoin%
\definecolor{currentfill}{rgb}{0.121569,0.466667,0.705882}%
\pgfsetfillcolor{currentfill}%
\pgfsetfillopacity{0.600547}%
\pgfsetlinewidth{1.003750pt}%
\definecolor{currentstroke}{rgb}{0.121569,0.466667,0.705882}%
\pgfsetstrokecolor{currentstroke}%
\pgfsetstrokeopacity{0.600547}%
\pgfsetdash{}{0pt}%
\pgfpathmoveto{\pgfqpoint{0.952332in}{1.668824in}}%
\pgfpathcurveto{\pgfqpoint{0.960569in}{1.668824in}}{\pgfqpoint{0.968469in}{1.672096in}}{\pgfqpoint{0.974293in}{1.677920in}}%
\pgfpathcurveto{\pgfqpoint{0.980116in}{1.683744in}}{\pgfqpoint{0.983389in}{1.691644in}}{\pgfqpoint{0.983389in}{1.699880in}}%
\pgfpathcurveto{\pgfqpoint{0.983389in}{1.708117in}}{\pgfqpoint{0.980116in}{1.716017in}}{\pgfqpoint{0.974293in}{1.721841in}}%
\pgfpathcurveto{\pgfqpoint{0.968469in}{1.727664in}}{\pgfqpoint{0.960569in}{1.730937in}}{\pgfqpoint{0.952332in}{1.730937in}}%
\pgfpathcurveto{\pgfqpoint{0.944096in}{1.730937in}}{\pgfqpoint{0.936196in}{1.727664in}}{\pgfqpoint{0.930372in}{1.721841in}}%
\pgfpathcurveto{\pgfqpoint{0.924548in}{1.716017in}}{\pgfqpoint{0.921276in}{1.708117in}}{\pgfqpoint{0.921276in}{1.699880in}}%
\pgfpathcurveto{\pgfqpoint{0.921276in}{1.691644in}}{\pgfqpoint{0.924548in}{1.683744in}}{\pgfqpoint{0.930372in}{1.677920in}}%
\pgfpathcurveto{\pgfqpoint{0.936196in}{1.672096in}}{\pgfqpoint{0.944096in}{1.668824in}}{\pgfqpoint{0.952332in}{1.668824in}}%
\pgfpathclose%
\pgfusepath{stroke,fill}%
\end{pgfscope}%
\begin{pgfscope}%
\pgfpathrectangle{\pgfqpoint{0.100000in}{0.212622in}}{\pgfqpoint{3.696000in}{3.696000in}}%
\pgfusepath{clip}%
\pgfsetbuttcap%
\pgfsetroundjoin%
\definecolor{currentfill}{rgb}{0.121569,0.466667,0.705882}%
\pgfsetfillcolor{currentfill}%
\pgfsetfillopacity{0.601214}%
\pgfsetlinewidth{1.003750pt}%
\definecolor{currentstroke}{rgb}{0.121569,0.466667,0.705882}%
\pgfsetstrokecolor{currentstroke}%
\pgfsetstrokeopacity{0.601214}%
\pgfsetdash{}{0pt}%
\pgfpathmoveto{\pgfqpoint{0.950727in}{1.666126in}}%
\pgfpathcurveto{\pgfqpoint{0.958963in}{1.666126in}}{\pgfqpoint{0.966863in}{1.669399in}}{\pgfqpoint{0.972687in}{1.675222in}}%
\pgfpathcurveto{\pgfqpoint{0.978511in}{1.681046in}}{\pgfqpoint{0.981783in}{1.688946in}}{\pgfqpoint{0.981783in}{1.697183in}}%
\pgfpathcurveto{\pgfqpoint{0.981783in}{1.705419in}}{\pgfqpoint{0.978511in}{1.713319in}}{\pgfqpoint{0.972687in}{1.719143in}}%
\pgfpathcurveto{\pgfqpoint{0.966863in}{1.724967in}}{\pgfqpoint{0.958963in}{1.728239in}}{\pgfqpoint{0.950727in}{1.728239in}}%
\pgfpathcurveto{\pgfqpoint{0.942490in}{1.728239in}}{\pgfqpoint{0.934590in}{1.724967in}}{\pgfqpoint{0.928766in}{1.719143in}}%
\pgfpathcurveto{\pgfqpoint{0.922943in}{1.713319in}}{\pgfqpoint{0.919670in}{1.705419in}}{\pgfqpoint{0.919670in}{1.697183in}}%
\pgfpathcurveto{\pgfqpoint{0.919670in}{1.688946in}}{\pgfqpoint{0.922943in}{1.681046in}}{\pgfqpoint{0.928766in}{1.675222in}}%
\pgfpathcurveto{\pgfqpoint{0.934590in}{1.669399in}}{\pgfqpoint{0.942490in}{1.666126in}}{\pgfqpoint{0.950727in}{1.666126in}}%
\pgfpathclose%
\pgfusepath{stroke,fill}%
\end{pgfscope}%
\begin{pgfscope}%
\pgfpathrectangle{\pgfqpoint{0.100000in}{0.212622in}}{\pgfqpoint{3.696000in}{3.696000in}}%
\pgfusepath{clip}%
\pgfsetbuttcap%
\pgfsetroundjoin%
\definecolor{currentfill}{rgb}{0.121569,0.466667,0.705882}%
\pgfsetfillcolor{currentfill}%
\pgfsetfillopacity{0.601660}%
\pgfsetlinewidth{1.003750pt}%
\definecolor{currentstroke}{rgb}{0.121569,0.466667,0.705882}%
\pgfsetstrokecolor{currentstroke}%
\pgfsetstrokeopacity{0.601660}%
\pgfsetdash{}{0pt}%
\pgfpathmoveto{\pgfqpoint{0.949555in}{1.664201in}}%
\pgfpathcurveto{\pgfqpoint{0.957791in}{1.664201in}}{\pgfqpoint{0.965691in}{1.667473in}}{\pgfqpoint{0.971515in}{1.673297in}}%
\pgfpathcurveto{\pgfqpoint{0.977339in}{1.679121in}}{\pgfqpoint{0.980611in}{1.687021in}}{\pgfqpoint{0.980611in}{1.695258in}}%
\pgfpathcurveto{\pgfqpoint{0.980611in}{1.703494in}}{\pgfqpoint{0.977339in}{1.711394in}}{\pgfqpoint{0.971515in}{1.717218in}}%
\pgfpathcurveto{\pgfqpoint{0.965691in}{1.723042in}}{\pgfqpoint{0.957791in}{1.726314in}}{\pgfqpoint{0.949555in}{1.726314in}}%
\pgfpathcurveto{\pgfqpoint{0.941319in}{1.726314in}}{\pgfqpoint{0.933419in}{1.723042in}}{\pgfqpoint{0.927595in}{1.717218in}}%
\pgfpathcurveto{\pgfqpoint{0.921771in}{1.711394in}}{\pgfqpoint{0.918498in}{1.703494in}}{\pgfqpoint{0.918498in}{1.695258in}}%
\pgfpathcurveto{\pgfqpoint{0.918498in}{1.687021in}}{\pgfqpoint{0.921771in}{1.679121in}}{\pgfqpoint{0.927595in}{1.673297in}}%
\pgfpathcurveto{\pgfqpoint{0.933419in}{1.667473in}}{\pgfqpoint{0.941319in}{1.664201in}}{\pgfqpoint{0.949555in}{1.664201in}}%
\pgfpathclose%
\pgfusepath{stroke,fill}%
\end{pgfscope}%
\begin{pgfscope}%
\pgfpathrectangle{\pgfqpoint{0.100000in}{0.212622in}}{\pgfqpoint{3.696000in}{3.696000in}}%
\pgfusepath{clip}%
\pgfsetbuttcap%
\pgfsetroundjoin%
\definecolor{currentfill}{rgb}{0.121569,0.466667,0.705882}%
\pgfsetfillcolor{currentfill}%
\pgfsetfillopacity{0.601691}%
\pgfsetlinewidth{1.003750pt}%
\definecolor{currentstroke}{rgb}{0.121569,0.466667,0.705882}%
\pgfsetstrokecolor{currentstroke}%
\pgfsetstrokeopacity{0.601691}%
\pgfsetdash{}{0pt}%
\pgfpathmoveto{\pgfqpoint{2.103637in}{2.086515in}}%
\pgfpathcurveto{\pgfqpoint{2.111874in}{2.086515in}}{\pgfqpoint{2.119774in}{2.089787in}}{\pgfqpoint{2.125598in}{2.095611in}}%
\pgfpathcurveto{\pgfqpoint{2.131422in}{2.101435in}}{\pgfqpoint{2.134694in}{2.109335in}}{\pgfqpoint{2.134694in}{2.117571in}}%
\pgfpathcurveto{\pgfqpoint{2.134694in}{2.125808in}}{\pgfqpoint{2.131422in}{2.133708in}}{\pgfqpoint{2.125598in}{2.139532in}}%
\pgfpathcurveto{\pgfqpoint{2.119774in}{2.145355in}}{\pgfqpoint{2.111874in}{2.148628in}}{\pgfqpoint{2.103637in}{2.148628in}}%
\pgfpathcurveto{\pgfqpoint{2.095401in}{2.148628in}}{\pgfqpoint{2.087501in}{2.145355in}}{\pgfqpoint{2.081677in}{2.139532in}}%
\pgfpathcurveto{\pgfqpoint{2.075853in}{2.133708in}}{\pgfqpoint{2.072581in}{2.125808in}}{\pgfqpoint{2.072581in}{2.117571in}}%
\pgfpathcurveto{\pgfqpoint{2.072581in}{2.109335in}}{\pgfqpoint{2.075853in}{2.101435in}}{\pgfqpoint{2.081677in}{2.095611in}}%
\pgfpathcurveto{\pgfqpoint{2.087501in}{2.089787in}}{\pgfqpoint{2.095401in}{2.086515in}}{\pgfqpoint{2.103637in}{2.086515in}}%
\pgfpathclose%
\pgfusepath{stroke,fill}%
\end{pgfscope}%
\begin{pgfscope}%
\pgfpathrectangle{\pgfqpoint{0.100000in}{0.212622in}}{\pgfqpoint{3.696000in}{3.696000in}}%
\pgfusepath{clip}%
\pgfsetbuttcap%
\pgfsetroundjoin%
\definecolor{currentfill}{rgb}{0.121569,0.466667,0.705882}%
\pgfsetfillcolor{currentfill}%
\pgfsetfillopacity{0.601872}%
\pgfsetlinewidth{1.003750pt}%
\definecolor{currentstroke}{rgb}{0.121569,0.466667,0.705882}%
\pgfsetstrokecolor{currentstroke}%
\pgfsetstrokeopacity{0.601872}%
\pgfsetdash{}{0pt}%
\pgfpathmoveto{\pgfqpoint{0.637190in}{1.208762in}}%
\pgfpathcurveto{\pgfqpoint{0.645426in}{1.208762in}}{\pgfqpoint{0.653326in}{1.212035in}}{\pgfqpoint{0.659150in}{1.217859in}}%
\pgfpathcurveto{\pgfqpoint{0.664974in}{1.223683in}}{\pgfqpoint{0.668246in}{1.231583in}}{\pgfqpoint{0.668246in}{1.239819in}}%
\pgfpathcurveto{\pgfqpoint{0.668246in}{1.248055in}}{\pgfqpoint{0.664974in}{1.255955in}}{\pgfqpoint{0.659150in}{1.261779in}}%
\pgfpathcurveto{\pgfqpoint{0.653326in}{1.267603in}}{\pgfqpoint{0.645426in}{1.270875in}}{\pgfqpoint{0.637190in}{1.270875in}}%
\pgfpathcurveto{\pgfqpoint{0.628953in}{1.270875in}}{\pgfqpoint{0.621053in}{1.267603in}}{\pgfqpoint{0.615230in}{1.261779in}}%
\pgfpathcurveto{\pgfqpoint{0.609406in}{1.255955in}}{\pgfqpoint{0.606133in}{1.248055in}}{\pgfqpoint{0.606133in}{1.239819in}}%
\pgfpathcurveto{\pgfqpoint{0.606133in}{1.231583in}}{\pgfqpoint{0.609406in}{1.223683in}}{\pgfqpoint{0.615230in}{1.217859in}}%
\pgfpathcurveto{\pgfqpoint{0.621053in}{1.212035in}}{\pgfqpoint{0.628953in}{1.208762in}}{\pgfqpoint{0.637190in}{1.208762in}}%
\pgfpathclose%
\pgfusepath{stroke,fill}%
\end{pgfscope}%
\begin{pgfscope}%
\pgfpathrectangle{\pgfqpoint{0.100000in}{0.212622in}}{\pgfqpoint{3.696000in}{3.696000in}}%
\pgfusepath{clip}%
\pgfsetbuttcap%
\pgfsetroundjoin%
\definecolor{currentfill}{rgb}{0.121569,0.466667,0.705882}%
\pgfsetfillcolor{currentfill}%
\pgfsetfillopacity{0.601910}%
\pgfsetlinewidth{1.003750pt}%
\definecolor{currentstroke}{rgb}{0.121569,0.466667,0.705882}%
\pgfsetstrokecolor{currentstroke}%
\pgfsetstrokeopacity{0.601910}%
\pgfsetdash{}{0pt}%
\pgfpathmoveto{\pgfqpoint{0.873744in}{1.486842in}}%
\pgfpathcurveto{\pgfqpoint{0.881980in}{1.486842in}}{\pgfqpoint{0.889880in}{1.490114in}}{\pgfqpoint{0.895704in}{1.495938in}}%
\pgfpathcurveto{\pgfqpoint{0.901528in}{1.501762in}}{\pgfqpoint{0.904801in}{1.509662in}}{\pgfqpoint{0.904801in}{1.517899in}}%
\pgfpathcurveto{\pgfqpoint{0.904801in}{1.526135in}}{\pgfqpoint{0.901528in}{1.534035in}}{\pgfqpoint{0.895704in}{1.539859in}}%
\pgfpathcurveto{\pgfqpoint{0.889880in}{1.545683in}}{\pgfqpoint{0.881980in}{1.548955in}}{\pgfqpoint{0.873744in}{1.548955in}}%
\pgfpathcurveto{\pgfqpoint{0.865508in}{1.548955in}}{\pgfqpoint{0.857608in}{1.545683in}}{\pgfqpoint{0.851784in}{1.539859in}}%
\pgfpathcurveto{\pgfqpoint{0.845960in}{1.534035in}}{\pgfqpoint{0.842688in}{1.526135in}}{\pgfqpoint{0.842688in}{1.517899in}}%
\pgfpathcurveto{\pgfqpoint{0.842688in}{1.509662in}}{\pgfqpoint{0.845960in}{1.501762in}}{\pgfqpoint{0.851784in}{1.495938in}}%
\pgfpathcurveto{\pgfqpoint{0.857608in}{1.490114in}}{\pgfqpoint{0.865508in}{1.486842in}}{\pgfqpoint{0.873744in}{1.486842in}}%
\pgfpathclose%
\pgfusepath{stroke,fill}%
\end{pgfscope}%
\begin{pgfscope}%
\pgfpathrectangle{\pgfqpoint{0.100000in}{0.212622in}}{\pgfqpoint{3.696000in}{3.696000in}}%
\pgfusepath{clip}%
\pgfsetbuttcap%
\pgfsetroundjoin%
\definecolor{currentfill}{rgb}{0.121569,0.466667,0.705882}%
\pgfsetfillcolor{currentfill}%
\pgfsetfillopacity{0.602474}%
\pgfsetlinewidth{1.003750pt}%
\definecolor{currentstroke}{rgb}{0.121569,0.466667,0.705882}%
\pgfsetstrokecolor{currentstroke}%
\pgfsetstrokeopacity{0.602474}%
\pgfsetdash{}{0pt}%
\pgfpathmoveto{\pgfqpoint{0.947367in}{1.660792in}}%
\pgfpathcurveto{\pgfqpoint{0.955604in}{1.660792in}}{\pgfqpoint{0.963504in}{1.664064in}}{\pgfqpoint{0.969328in}{1.669888in}}%
\pgfpathcurveto{\pgfqpoint{0.975151in}{1.675712in}}{\pgfqpoint{0.978424in}{1.683612in}}{\pgfqpoint{0.978424in}{1.691848in}}%
\pgfpathcurveto{\pgfqpoint{0.978424in}{1.700085in}}{\pgfqpoint{0.975151in}{1.707985in}}{\pgfqpoint{0.969328in}{1.713809in}}%
\pgfpathcurveto{\pgfqpoint{0.963504in}{1.719633in}}{\pgfqpoint{0.955604in}{1.722905in}}{\pgfqpoint{0.947367in}{1.722905in}}%
\pgfpathcurveto{\pgfqpoint{0.939131in}{1.722905in}}{\pgfqpoint{0.931231in}{1.719633in}}{\pgfqpoint{0.925407in}{1.713809in}}%
\pgfpathcurveto{\pgfqpoint{0.919583in}{1.707985in}}{\pgfqpoint{0.916311in}{1.700085in}}{\pgfqpoint{0.916311in}{1.691848in}}%
\pgfpathcurveto{\pgfqpoint{0.916311in}{1.683612in}}{\pgfqpoint{0.919583in}{1.675712in}}{\pgfqpoint{0.925407in}{1.669888in}}%
\pgfpathcurveto{\pgfqpoint{0.931231in}{1.664064in}}{\pgfqpoint{0.939131in}{1.660792in}}{\pgfqpoint{0.947367in}{1.660792in}}%
\pgfpathclose%
\pgfusepath{stroke,fill}%
\end{pgfscope}%
\begin{pgfscope}%
\pgfpathrectangle{\pgfqpoint{0.100000in}{0.212622in}}{\pgfqpoint{3.696000in}{3.696000in}}%
\pgfusepath{clip}%
\pgfsetbuttcap%
\pgfsetroundjoin%
\definecolor{currentfill}{rgb}{0.121569,0.466667,0.705882}%
\pgfsetfillcolor{currentfill}%
\pgfsetfillopacity{0.603005}%
\pgfsetlinewidth{1.003750pt}%
\definecolor{currentstroke}{rgb}{0.121569,0.466667,0.705882}%
\pgfsetstrokecolor{currentstroke}%
\pgfsetstrokeopacity{0.603005}%
\pgfsetdash{}{0pt}%
\pgfpathmoveto{\pgfqpoint{0.945969in}{1.658485in}}%
\pgfpathcurveto{\pgfqpoint{0.954205in}{1.658485in}}{\pgfqpoint{0.962105in}{1.661757in}}{\pgfqpoint{0.967929in}{1.667581in}}%
\pgfpathcurveto{\pgfqpoint{0.973753in}{1.673405in}}{\pgfqpoint{0.977025in}{1.681305in}}{\pgfqpoint{0.977025in}{1.689541in}}%
\pgfpathcurveto{\pgfqpoint{0.977025in}{1.697778in}}{\pgfqpoint{0.973753in}{1.705678in}}{\pgfqpoint{0.967929in}{1.711502in}}%
\pgfpathcurveto{\pgfqpoint{0.962105in}{1.717325in}}{\pgfqpoint{0.954205in}{1.720598in}}{\pgfqpoint{0.945969in}{1.720598in}}%
\pgfpathcurveto{\pgfqpoint{0.937733in}{1.720598in}}{\pgfqpoint{0.929833in}{1.717325in}}{\pgfqpoint{0.924009in}{1.711502in}}%
\pgfpathcurveto{\pgfqpoint{0.918185in}{1.705678in}}{\pgfqpoint{0.914912in}{1.697778in}}{\pgfqpoint{0.914912in}{1.689541in}}%
\pgfpathcurveto{\pgfqpoint{0.914912in}{1.681305in}}{\pgfqpoint{0.918185in}{1.673405in}}{\pgfqpoint{0.924009in}{1.667581in}}%
\pgfpathcurveto{\pgfqpoint{0.929833in}{1.661757in}}{\pgfqpoint{0.937733in}{1.658485in}}{\pgfqpoint{0.945969in}{1.658485in}}%
\pgfpathclose%
\pgfusepath{stroke,fill}%
\end{pgfscope}%
\begin{pgfscope}%
\pgfpathrectangle{\pgfqpoint{0.100000in}{0.212622in}}{\pgfqpoint{3.696000in}{3.696000in}}%
\pgfusepath{clip}%
\pgfsetbuttcap%
\pgfsetroundjoin%
\definecolor{currentfill}{rgb}{0.121569,0.466667,0.705882}%
\pgfsetfillcolor{currentfill}%
\pgfsetfillopacity{0.603075}%
\pgfsetlinewidth{1.003750pt}%
\definecolor{currentstroke}{rgb}{0.121569,0.466667,0.705882}%
\pgfsetstrokecolor{currentstroke}%
\pgfsetstrokeopacity{0.603075}%
\pgfsetdash{}{0pt}%
\pgfpathmoveto{\pgfqpoint{2.104588in}{2.081938in}}%
\pgfpathcurveto{\pgfqpoint{2.112824in}{2.081938in}}{\pgfqpoint{2.120724in}{2.085210in}}{\pgfqpoint{2.126548in}{2.091034in}}%
\pgfpathcurveto{\pgfqpoint{2.132372in}{2.096858in}}{\pgfqpoint{2.135644in}{2.104758in}}{\pgfqpoint{2.135644in}{2.112994in}}%
\pgfpathcurveto{\pgfqpoint{2.135644in}{2.121230in}}{\pgfqpoint{2.132372in}{2.129131in}}{\pgfqpoint{2.126548in}{2.134954in}}%
\pgfpathcurveto{\pgfqpoint{2.120724in}{2.140778in}}{\pgfqpoint{2.112824in}{2.144051in}}{\pgfqpoint{2.104588in}{2.144051in}}%
\pgfpathcurveto{\pgfqpoint{2.096351in}{2.144051in}}{\pgfqpoint{2.088451in}{2.140778in}}{\pgfqpoint{2.082627in}{2.134954in}}%
\pgfpathcurveto{\pgfqpoint{2.076803in}{2.129131in}}{\pgfqpoint{2.073531in}{2.121230in}}{\pgfqpoint{2.073531in}{2.112994in}}%
\pgfpathcurveto{\pgfqpoint{2.073531in}{2.104758in}}{\pgfqpoint{2.076803in}{2.096858in}}{\pgfqpoint{2.082627in}{2.091034in}}%
\pgfpathcurveto{\pgfqpoint{2.088451in}{2.085210in}}{\pgfqpoint{2.096351in}{2.081938in}}{\pgfqpoint{2.104588in}{2.081938in}}%
\pgfpathclose%
\pgfusepath{stroke,fill}%
\end{pgfscope}%
\begin{pgfscope}%
\pgfpathrectangle{\pgfqpoint{0.100000in}{0.212622in}}{\pgfqpoint{3.696000in}{3.696000in}}%
\pgfusepath{clip}%
\pgfsetbuttcap%
\pgfsetroundjoin%
\definecolor{currentfill}{rgb}{0.121569,0.466667,0.705882}%
\pgfsetfillcolor{currentfill}%
\pgfsetfillopacity{0.603861}%
\pgfsetlinewidth{1.003750pt}%
\definecolor{currentstroke}{rgb}{0.121569,0.466667,0.705882}%
\pgfsetstrokecolor{currentstroke}%
\pgfsetstrokeopacity{0.603861}%
\pgfsetdash{}{0pt}%
\pgfpathmoveto{\pgfqpoint{2.104998in}{2.079411in}}%
\pgfpathcurveto{\pgfqpoint{2.113235in}{2.079411in}}{\pgfqpoint{2.121135in}{2.082684in}}{\pgfqpoint{2.126959in}{2.088507in}}%
\pgfpathcurveto{\pgfqpoint{2.132783in}{2.094331in}}{\pgfqpoint{2.136055in}{2.102231in}}{\pgfqpoint{2.136055in}{2.110468in}}%
\pgfpathcurveto{\pgfqpoint{2.136055in}{2.118704in}}{\pgfqpoint{2.132783in}{2.126604in}}{\pgfqpoint{2.126959in}{2.132428in}}%
\pgfpathcurveto{\pgfqpoint{2.121135in}{2.138252in}}{\pgfqpoint{2.113235in}{2.141524in}}{\pgfqpoint{2.104998in}{2.141524in}}%
\pgfpathcurveto{\pgfqpoint{2.096762in}{2.141524in}}{\pgfqpoint{2.088862in}{2.138252in}}{\pgfqpoint{2.083038in}{2.132428in}}%
\pgfpathcurveto{\pgfqpoint{2.077214in}{2.126604in}}{\pgfqpoint{2.073942in}{2.118704in}}{\pgfqpoint{2.073942in}{2.110468in}}%
\pgfpathcurveto{\pgfqpoint{2.073942in}{2.102231in}}{\pgfqpoint{2.077214in}{2.094331in}}{\pgfqpoint{2.083038in}{2.088507in}}%
\pgfpathcurveto{\pgfqpoint{2.088862in}{2.082684in}}{\pgfqpoint{2.096762in}{2.079411in}}{\pgfqpoint{2.104998in}{2.079411in}}%
\pgfpathclose%
\pgfusepath{stroke,fill}%
\end{pgfscope}%
\begin{pgfscope}%
\pgfpathrectangle{\pgfqpoint{0.100000in}{0.212622in}}{\pgfqpoint{3.696000in}{3.696000in}}%
\pgfusepath{clip}%
\pgfsetbuttcap%
\pgfsetroundjoin%
\definecolor{currentfill}{rgb}{0.121569,0.466667,0.705882}%
\pgfsetfillcolor{currentfill}%
\pgfsetfillopacity{0.604011}%
\pgfsetlinewidth{1.003750pt}%
\definecolor{currentstroke}{rgb}{0.121569,0.466667,0.705882}%
\pgfsetstrokecolor{currentstroke}%
\pgfsetstrokeopacity{0.604011}%
\pgfsetdash{}{0pt}%
\pgfpathmoveto{\pgfqpoint{0.943307in}{1.654597in}}%
\pgfpathcurveto{\pgfqpoint{0.951543in}{1.654597in}}{\pgfqpoint{0.959443in}{1.657869in}}{\pgfqpoint{0.965267in}{1.663693in}}%
\pgfpathcurveto{\pgfqpoint{0.971091in}{1.669517in}}{\pgfqpoint{0.974363in}{1.677417in}}{\pgfqpoint{0.974363in}{1.685653in}}%
\pgfpathcurveto{\pgfqpoint{0.974363in}{1.693890in}}{\pgfqpoint{0.971091in}{1.701790in}}{\pgfqpoint{0.965267in}{1.707614in}}%
\pgfpathcurveto{\pgfqpoint{0.959443in}{1.713438in}}{\pgfqpoint{0.951543in}{1.716710in}}{\pgfqpoint{0.943307in}{1.716710in}}%
\pgfpathcurveto{\pgfqpoint{0.935070in}{1.716710in}}{\pgfqpoint{0.927170in}{1.713438in}}{\pgfqpoint{0.921346in}{1.707614in}}%
\pgfpathcurveto{\pgfqpoint{0.915522in}{1.701790in}}{\pgfqpoint{0.912250in}{1.693890in}}{\pgfqpoint{0.912250in}{1.685653in}}%
\pgfpathcurveto{\pgfqpoint{0.912250in}{1.677417in}}{\pgfqpoint{0.915522in}{1.669517in}}{\pgfqpoint{0.921346in}{1.663693in}}%
\pgfpathcurveto{\pgfqpoint{0.927170in}{1.657869in}}{\pgfqpoint{0.935070in}{1.654597in}}{\pgfqpoint{0.943307in}{1.654597in}}%
\pgfpathclose%
\pgfusepath{stroke,fill}%
\end{pgfscope}%
\begin{pgfscope}%
\pgfpathrectangle{\pgfqpoint{0.100000in}{0.212622in}}{\pgfqpoint{3.696000in}{3.696000in}}%
\pgfusepath{clip}%
\pgfsetbuttcap%
\pgfsetroundjoin%
\definecolor{currentfill}{rgb}{0.121569,0.466667,0.705882}%
\pgfsetfillcolor{currentfill}%
\pgfsetfillopacity{0.604828}%
\pgfsetlinewidth{1.003750pt}%
\definecolor{currentstroke}{rgb}{0.121569,0.466667,0.705882}%
\pgfsetstrokecolor{currentstroke}%
\pgfsetstrokeopacity{0.604828}%
\pgfsetdash{}{0pt}%
\pgfpathmoveto{\pgfqpoint{0.941390in}{1.651524in}}%
\pgfpathcurveto{\pgfqpoint{0.949626in}{1.651524in}}{\pgfqpoint{0.957526in}{1.654796in}}{\pgfqpoint{0.963350in}{1.660620in}}%
\pgfpathcurveto{\pgfqpoint{0.969174in}{1.666444in}}{\pgfqpoint{0.972447in}{1.674344in}}{\pgfqpoint{0.972447in}{1.682581in}}%
\pgfpathcurveto{\pgfqpoint{0.972447in}{1.690817in}}{\pgfqpoint{0.969174in}{1.698717in}}{\pgfqpoint{0.963350in}{1.704541in}}%
\pgfpathcurveto{\pgfqpoint{0.957526in}{1.710365in}}{\pgfqpoint{0.949626in}{1.713637in}}{\pgfqpoint{0.941390in}{1.713637in}}%
\pgfpathcurveto{\pgfqpoint{0.933154in}{1.713637in}}{\pgfqpoint{0.925254in}{1.710365in}}{\pgfqpoint{0.919430in}{1.704541in}}%
\pgfpathcurveto{\pgfqpoint{0.913606in}{1.698717in}}{\pgfqpoint{0.910334in}{1.690817in}}{\pgfqpoint{0.910334in}{1.682581in}}%
\pgfpathcurveto{\pgfqpoint{0.910334in}{1.674344in}}{\pgfqpoint{0.913606in}{1.666444in}}{\pgfqpoint{0.919430in}{1.660620in}}%
\pgfpathcurveto{\pgfqpoint{0.925254in}{1.654796in}}{\pgfqpoint{0.933154in}{1.651524in}}{\pgfqpoint{0.941390in}{1.651524in}}%
\pgfpathclose%
\pgfusepath{stroke,fill}%
\end{pgfscope}%
\begin{pgfscope}%
\pgfpathrectangle{\pgfqpoint{0.100000in}{0.212622in}}{\pgfqpoint{3.696000in}{3.696000in}}%
\pgfusepath{clip}%
\pgfsetbuttcap%
\pgfsetroundjoin%
\definecolor{currentfill}{rgb}{0.121569,0.466667,0.705882}%
\pgfsetfillcolor{currentfill}%
\pgfsetfillopacity{0.604909}%
\pgfsetlinewidth{1.003750pt}%
\definecolor{currentstroke}{rgb}{0.121569,0.466667,0.705882}%
\pgfsetstrokecolor{currentstroke}%
\pgfsetstrokeopacity{0.604909}%
\pgfsetdash{}{0pt}%
\pgfpathmoveto{\pgfqpoint{2.105862in}{2.075552in}}%
\pgfpathcurveto{\pgfqpoint{2.114098in}{2.075552in}}{\pgfqpoint{2.121998in}{2.078824in}}{\pgfqpoint{2.127822in}{2.084648in}}%
\pgfpathcurveto{\pgfqpoint{2.133646in}{2.090472in}}{\pgfqpoint{2.136918in}{2.098372in}}{\pgfqpoint{2.136918in}{2.106608in}}%
\pgfpathcurveto{\pgfqpoint{2.136918in}{2.114844in}}{\pgfqpoint{2.133646in}{2.122745in}}{\pgfqpoint{2.127822in}{2.128568in}}%
\pgfpathcurveto{\pgfqpoint{2.121998in}{2.134392in}}{\pgfqpoint{2.114098in}{2.137665in}}{\pgfqpoint{2.105862in}{2.137665in}}%
\pgfpathcurveto{\pgfqpoint{2.097626in}{2.137665in}}{\pgfqpoint{2.089725in}{2.134392in}}{\pgfqpoint{2.083902in}{2.128568in}}%
\pgfpathcurveto{\pgfqpoint{2.078078in}{2.122745in}}{\pgfqpoint{2.074805in}{2.114844in}}{\pgfqpoint{2.074805in}{2.106608in}}%
\pgfpathcurveto{\pgfqpoint{2.074805in}{2.098372in}}{\pgfqpoint{2.078078in}{2.090472in}}{\pgfqpoint{2.083902in}{2.084648in}}%
\pgfpathcurveto{\pgfqpoint{2.089725in}{2.078824in}}{\pgfqpoint{2.097626in}{2.075552in}}{\pgfqpoint{2.105862in}{2.075552in}}%
\pgfpathclose%
\pgfusepath{stroke,fill}%
\end{pgfscope}%
\begin{pgfscope}%
\pgfpathrectangle{\pgfqpoint{0.100000in}{0.212622in}}{\pgfqpoint{3.696000in}{3.696000in}}%
\pgfusepath{clip}%
\pgfsetbuttcap%
\pgfsetroundjoin%
\definecolor{currentfill}{rgb}{0.121569,0.466667,0.705882}%
\pgfsetfillcolor{currentfill}%
\pgfsetfillopacity{0.605502}%
\pgfsetlinewidth{1.003750pt}%
\definecolor{currentstroke}{rgb}{0.121569,0.466667,0.705882}%
\pgfsetstrokecolor{currentstroke}%
\pgfsetstrokeopacity{0.605502}%
\pgfsetdash{}{0pt}%
\pgfpathmoveto{\pgfqpoint{2.106294in}{2.073445in}}%
\pgfpathcurveto{\pgfqpoint{2.114530in}{2.073445in}}{\pgfqpoint{2.122430in}{2.076717in}}{\pgfqpoint{2.128254in}{2.082541in}}%
\pgfpathcurveto{\pgfqpoint{2.134078in}{2.088365in}}{\pgfqpoint{2.137351in}{2.096265in}}{\pgfqpoint{2.137351in}{2.104502in}}%
\pgfpathcurveto{\pgfqpoint{2.137351in}{2.112738in}}{\pgfqpoint{2.134078in}{2.120638in}}{\pgfqpoint{2.128254in}{2.126462in}}%
\pgfpathcurveto{\pgfqpoint{2.122430in}{2.132286in}}{\pgfqpoint{2.114530in}{2.135558in}}{\pgfqpoint{2.106294in}{2.135558in}}%
\pgfpathcurveto{\pgfqpoint{2.098058in}{2.135558in}}{\pgfqpoint{2.090158in}{2.132286in}}{\pgfqpoint{2.084334in}{2.126462in}}%
\pgfpathcurveto{\pgfqpoint{2.078510in}{2.120638in}}{\pgfqpoint{2.075238in}{2.112738in}}{\pgfqpoint{2.075238in}{2.104502in}}%
\pgfpathcurveto{\pgfqpoint{2.075238in}{2.096265in}}{\pgfqpoint{2.078510in}{2.088365in}}{\pgfqpoint{2.084334in}{2.082541in}}%
\pgfpathcurveto{\pgfqpoint{2.090158in}{2.076717in}}{\pgfqpoint{2.098058in}{2.073445in}}{\pgfqpoint{2.106294in}{2.073445in}}%
\pgfpathclose%
\pgfusepath{stroke,fill}%
\end{pgfscope}%
\begin{pgfscope}%
\pgfpathrectangle{\pgfqpoint{0.100000in}{0.212622in}}{\pgfqpoint{3.696000in}{3.696000in}}%
\pgfusepath{clip}%
\pgfsetbuttcap%
\pgfsetroundjoin%
\definecolor{currentfill}{rgb}{0.121569,0.466667,0.705882}%
\pgfsetfillcolor{currentfill}%
\pgfsetfillopacity{0.605523}%
\pgfsetlinewidth{1.003750pt}%
\definecolor{currentstroke}{rgb}{0.121569,0.466667,0.705882}%
\pgfsetstrokecolor{currentstroke}%
\pgfsetstrokeopacity{0.605523}%
\pgfsetdash{}{0pt}%
\pgfpathmoveto{\pgfqpoint{0.939611in}{1.648738in}}%
\pgfpathcurveto{\pgfqpoint{0.947847in}{1.648738in}}{\pgfqpoint{0.955747in}{1.652011in}}{\pgfqpoint{0.961571in}{1.657835in}}%
\pgfpathcurveto{\pgfqpoint{0.967395in}{1.663659in}}{\pgfqpoint{0.970667in}{1.671559in}}{\pgfqpoint{0.970667in}{1.679795in}}%
\pgfpathcurveto{\pgfqpoint{0.970667in}{1.688031in}}{\pgfqpoint{0.967395in}{1.695931in}}{\pgfqpoint{0.961571in}{1.701755in}}%
\pgfpathcurveto{\pgfqpoint{0.955747in}{1.707579in}}{\pgfqpoint{0.947847in}{1.710851in}}{\pgfqpoint{0.939611in}{1.710851in}}%
\pgfpathcurveto{\pgfqpoint{0.931374in}{1.710851in}}{\pgfqpoint{0.923474in}{1.707579in}}{\pgfqpoint{0.917650in}{1.701755in}}%
\pgfpathcurveto{\pgfqpoint{0.911826in}{1.695931in}}{\pgfqpoint{0.908554in}{1.688031in}}{\pgfqpoint{0.908554in}{1.679795in}}%
\pgfpathcurveto{\pgfqpoint{0.908554in}{1.671559in}}{\pgfqpoint{0.911826in}{1.663659in}}{\pgfqpoint{0.917650in}{1.657835in}}%
\pgfpathcurveto{\pgfqpoint{0.923474in}{1.652011in}}{\pgfqpoint{0.931374in}{1.648738in}}{\pgfqpoint{0.939611in}{1.648738in}}%
\pgfpathclose%
\pgfusepath{stroke,fill}%
\end{pgfscope}%
\begin{pgfscope}%
\pgfpathrectangle{\pgfqpoint{0.100000in}{0.212622in}}{\pgfqpoint{3.696000in}{3.696000in}}%
\pgfusepath{clip}%
\pgfsetbuttcap%
\pgfsetroundjoin%
\definecolor{currentfill}{rgb}{0.121569,0.466667,0.705882}%
\pgfsetfillcolor{currentfill}%
\pgfsetfillopacity{0.605616}%
\pgfsetlinewidth{1.003750pt}%
\definecolor{currentstroke}{rgb}{0.121569,0.466667,0.705882}%
\pgfsetstrokecolor{currentstroke}%
\pgfsetstrokeopacity{0.605616}%
\pgfsetdash{}{0pt}%
\pgfpathmoveto{\pgfqpoint{0.653848in}{1.205825in}}%
\pgfpathcurveto{\pgfqpoint{0.662084in}{1.205825in}}{\pgfqpoint{0.669984in}{1.209097in}}{\pgfqpoint{0.675808in}{1.214921in}}%
\pgfpathcurveto{\pgfqpoint{0.681632in}{1.220745in}}{\pgfqpoint{0.684904in}{1.228645in}}{\pgfqpoint{0.684904in}{1.236881in}}%
\pgfpathcurveto{\pgfqpoint{0.684904in}{1.245117in}}{\pgfqpoint{0.681632in}{1.253017in}}{\pgfqpoint{0.675808in}{1.258841in}}%
\pgfpathcurveto{\pgfqpoint{0.669984in}{1.264665in}}{\pgfqpoint{0.662084in}{1.267938in}}{\pgfqpoint{0.653848in}{1.267938in}}%
\pgfpathcurveto{\pgfqpoint{0.645612in}{1.267938in}}{\pgfqpoint{0.637712in}{1.264665in}}{\pgfqpoint{0.631888in}{1.258841in}}%
\pgfpathcurveto{\pgfqpoint{0.626064in}{1.253017in}}{\pgfqpoint{0.622791in}{1.245117in}}{\pgfqpoint{0.622791in}{1.236881in}}%
\pgfpathcurveto{\pgfqpoint{0.622791in}{1.228645in}}{\pgfqpoint{0.626064in}{1.220745in}}{\pgfqpoint{0.631888in}{1.214921in}}%
\pgfpathcurveto{\pgfqpoint{0.637712in}{1.209097in}}{\pgfqpoint{0.645612in}{1.205825in}}{\pgfqpoint{0.653848in}{1.205825in}}%
\pgfpathclose%
\pgfusepath{stroke,fill}%
\end{pgfscope}%
\begin{pgfscope}%
\pgfpathrectangle{\pgfqpoint{0.100000in}{0.212622in}}{\pgfqpoint{3.696000in}{3.696000in}}%
\pgfusepath{clip}%
\pgfsetbuttcap%
\pgfsetroundjoin%
\definecolor{currentfill}{rgb}{0.121569,0.466667,0.705882}%
\pgfsetfillcolor{currentfill}%
\pgfsetfillopacity{0.606148}%
\pgfsetlinewidth{1.003750pt}%
\definecolor{currentstroke}{rgb}{0.121569,0.466667,0.705882}%
\pgfsetstrokecolor{currentstroke}%
\pgfsetstrokeopacity{0.606148}%
\pgfsetdash{}{0pt}%
\pgfpathmoveto{\pgfqpoint{0.938208in}{1.646250in}}%
\pgfpathcurveto{\pgfqpoint{0.946444in}{1.646250in}}{\pgfqpoint{0.954344in}{1.649523in}}{\pgfqpoint{0.960168in}{1.655346in}}%
\pgfpathcurveto{\pgfqpoint{0.965992in}{1.661170in}}{\pgfqpoint{0.969264in}{1.669070in}}{\pgfqpoint{0.969264in}{1.677307in}}%
\pgfpathcurveto{\pgfqpoint{0.969264in}{1.685543in}}{\pgfqpoint{0.965992in}{1.693443in}}{\pgfqpoint{0.960168in}{1.699267in}}%
\pgfpathcurveto{\pgfqpoint{0.954344in}{1.705091in}}{\pgfqpoint{0.946444in}{1.708363in}}{\pgfqpoint{0.938208in}{1.708363in}}%
\pgfpathcurveto{\pgfqpoint{0.929971in}{1.708363in}}{\pgfqpoint{0.922071in}{1.705091in}}{\pgfqpoint{0.916247in}{1.699267in}}%
\pgfpathcurveto{\pgfqpoint{0.910423in}{1.693443in}}{\pgfqpoint{0.907151in}{1.685543in}}{\pgfqpoint{0.907151in}{1.677307in}}%
\pgfpathcurveto{\pgfqpoint{0.907151in}{1.669070in}}{\pgfqpoint{0.910423in}{1.661170in}}{\pgfqpoint{0.916247in}{1.655346in}}%
\pgfpathcurveto{\pgfqpoint{0.922071in}{1.649523in}}{\pgfqpoint{0.929971in}{1.646250in}}{\pgfqpoint{0.938208in}{1.646250in}}%
\pgfpathclose%
\pgfusepath{stroke,fill}%
\end{pgfscope}%
\begin{pgfscope}%
\pgfpathrectangle{\pgfqpoint{0.100000in}{0.212622in}}{\pgfqpoint{3.696000in}{3.696000in}}%
\pgfusepath{clip}%
\pgfsetbuttcap%
\pgfsetroundjoin%
\definecolor{currentfill}{rgb}{0.121569,0.466667,0.705882}%
\pgfsetfillcolor{currentfill}%
\pgfsetfillopacity{0.606433}%
\pgfsetlinewidth{1.003750pt}%
\definecolor{currentstroke}{rgb}{0.121569,0.466667,0.705882}%
\pgfsetstrokecolor{currentstroke}%
\pgfsetstrokeopacity{0.606433}%
\pgfsetdash{}{0pt}%
\pgfpathmoveto{\pgfqpoint{2.106736in}{2.070664in}}%
\pgfpathcurveto{\pgfqpoint{2.114973in}{2.070664in}}{\pgfqpoint{2.122873in}{2.073937in}}{\pgfqpoint{2.128697in}{2.079760in}}%
\pgfpathcurveto{\pgfqpoint{2.134521in}{2.085584in}}{\pgfqpoint{2.137793in}{2.093484in}}{\pgfqpoint{2.137793in}{2.101721in}}%
\pgfpathcurveto{\pgfqpoint{2.137793in}{2.109957in}}{\pgfqpoint{2.134521in}{2.117857in}}{\pgfqpoint{2.128697in}{2.123681in}}%
\pgfpathcurveto{\pgfqpoint{2.122873in}{2.129505in}}{\pgfqpoint{2.114973in}{2.132777in}}{\pgfqpoint{2.106736in}{2.132777in}}%
\pgfpathcurveto{\pgfqpoint{2.098500in}{2.132777in}}{\pgfqpoint{2.090600in}{2.129505in}}{\pgfqpoint{2.084776in}{2.123681in}}%
\pgfpathcurveto{\pgfqpoint{2.078952in}{2.117857in}}{\pgfqpoint{2.075680in}{2.109957in}}{\pgfqpoint{2.075680in}{2.101721in}}%
\pgfpathcurveto{\pgfqpoint{2.075680in}{2.093484in}}{\pgfqpoint{2.078952in}{2.085584in}}{\pgfqpoint{2.084776in}{2.079760in}}%
\pgfpathcurveto{\pgfqpoint{2.090600in}{2.073937in}}{\pgfqpoint{2.098500in}{2.070664in}}{\pgfqpoint{2.106736in}{2.070664in}}%
\pgfpathclose%
\pgfusepath{stroke,fill}%
\end{pgfscope}%
\begin{pgfscope}%
\pgfpathrectangle{\pgfqpoint{0.100000in}{0.212622in}}{\pgfqpoint{3.696000in}{3.696000in}}%
\pgfusepath{clip}%
\pgfsetbuttcap%
\pgfsetroundjoin%
\definecolor{currentfill}{rgb}{0.121569,0.466667,0.705882}%
\pgfsetfillcolor{currentfill}%
\pgfsetfillopacity{0.606460}%
\pgfsetlinewidth{1.003750pt}%
\definecolor{currentstroke}{rgb}{0.121569,0.466667,0.705882}%
\pgfsetstrokecolor{currentstroke}%
\pgfsetstrokeopacity{0.606460}%
\pgfsetdash{}{0pt}%
\pgfpathmoveto{\pgfqpoint{0.937322in}{1.644827in}}%
\pgfpathcurveto{\pgfqpoint{0.945558in}{1.644827in}}{\pgfqpoint{0.953458in}{1.648099in}}{\pgfqpoint{0.959282in}{1.653923in}}%
\pgfpathcurveto{\pgfqpoint{0.965106in}{1.659747in}}{\pgfqpoint{0.968378in}{1.667647in}}{\pgfqpoint{0.968378in}{1.675883in}}%
\pgfpathcurveto{\pgfqpoint{0.968378in}{1.684119in}}{\pgfqpoint{0.965106in}{1.692019in}}{\pgfqpoint{0.959282in}{1.697843in}}%
\pgfpathcurveto{\pgfqpoint{0.953458in}{1.703667in}}{\pgfqpoint{0.945558in}{1.706940in}}{\pgfqpoint{0.937322in}{1.706940in}}%
\pgfpathcurveto{\pgfqpoint{0.929086in}{1.706940in}}{\pgfqpoint{0.921186in}{1.703667in}}{\pgfqpoint{0.915362in}{1.697843in}}%
\pgfpathcurveto{\pgfqpoint{0.909538in}{1.692019in}}{\pgfqpoint{0.906265in}{1.684119in}}{\pgfqpoint{0.906265in}{1.675883in}}%
\pgfpathcurveto{\pgfqpoint{0.906265in}{1.667647in}}{\pgfqpoint{0.909538in}{1.659747in}}{\pgfqpoint{0.915362in}{1.653923in}}%
\pgfpathcurveto{\pgfqpoint{0.921186in}{1.648099in}}{\pgfqpoint{0.929086in}{1.644827in}}{\pgfqpoint{0.937322in}{1.644827in}}%
\pgfpathclose%
\pgfusepath{stroke,fill}%
\end{pgfscope}%
\begin{pgfscope}%
\pgfpathrectangle{\pgfqpoint{0.100000in}{0.212622in}}{\pgfqpoint{3.696000in}{3.696000in}}%
\pgfusepath{clip}%
\pgfsetbuttcap%
\pgfsetroundjoin%
\definecolor{currentfill}{rgb}{0.121569,0.466667,0.705882}%
\pgfsetfillcolor{currentfill}%
\pgfsetfillopacity{0.606888}%
\pgfsetlinewidth{1.003750pt}%
\definecolor{currentstroke}{rgb}{0.121569,0.466667,0.705882}%
\pgfsetstrokecolor{currentstroke}%
\pgfsetstrokeopacity{0.606888}%
\pgfsetdash{}{0pt}%
\pgfpathmoveto{\pgfqpoint{0.870062in}{1.487791in}}%
\pgfpathcurveto{\pgfqpoint{0.878298in}{1.487791in}}{\pgfqpoint{0.886198in}{1.491063in}}{\pgfqpoint{0.892022in}{1.496887in}}%
\pgfpathcurveto{\pgfqpoint{0.897846in}{1.502711in}}{\pgfqpoint{0.901118in}{1.510611in}}{\pgfqpoint{0.901118in}{1.518848in}}%
\pgfpathcurveto{\pgfqpoint{0.901118in}{1.527084in}}{\pgfqpoint{0.897846in}{1.534984in}}{\pgfqpoint{0.892022in}{1.540808in}}%
\pgfpathcurveto{\pgfqpoint{0.886198in}{1.546632in}}{\pgfqpoint{0.878298in}{1.549904in}}{\pgfqpoint{0.870062in}{1.549904in}}%
\pgfpathcurveto{\pgfqpoint{0.861826in}{1.549904in}}{\pgfqpoint{0.853926in}{1.546632in}}{\pgfqpoint{0.848102in}{1.540808in}}%
\pgfpathcurveto{\pgfqpoint{0.842278in}{1.534984in}}{\pgfqpoint{0.839005in}{1.527084in}}{\pgfqpoint{0.839005in}{1.518848in}}%
\pgfpathcurveto{\pgfqpoint{0.839005in}{1.510611in}}{\pgfqpoint{0.842278in}{1.502711in}}{\pgfqpoint{0.848102in}{1.496887in}}%
\pgfpathcurveto{\pgfqpoint{0.853926in}{1.491063in}}{\pgfqpoint{0.861826in}{1.487791in}}{\pgfqpoint{0.870062in}{1.487791in}}%
\pgfpathclose%
\pgfusepath{stroke,fill}%
\end{pgfscope}%
\begin{pgfscope}%
\pgfpathrectangle{\pgfqpoint{0.100000in}{0.212622in}}{\pgfqpoint{3.696000in}{3.696000in}}%
\pgfusepath{clip}%
\pgfsetbuttcap%
\pgfsetroundjoin%
\definecolor{currentfill}{rgb}{0.121569,0.466667,0.705882}%
\pgfsetfillcolor{currentfill}%
\pgfsetfillopacity{0.607034}%
\pgfsetlinewidth{1.003750pt}%
\definecolor{currentstroke}{rgb}{0.121569,0.466667,0.705882}%
\pgfsetstrokecolor{currentstroke}%
\pgfsetstrokeopacity{0.607034}%
\pgfsetdash{}{0pt}%
\pgfpathmoveto{\pgfqpoint{0.935833in}{1.642104in}}%
\pgfpathcurveto{\pgfqpoint{0.944069in}{1.642104in}}{\pgfqpoint{0.951969in}{1.645377in}}{\pgfqpoint{0.957793in}{1.651201in}}%
\pgfpathcurveto{\pgfqpoint{0.963617in}{1.657025in}}{\pgfqpoint{0.966889in}{1.664925in}}{\pgfqpoint{0.966889in}{1.673161in}}%
\pgfpathcurveto{\pgfqpoint{0.966889in}{1.681397in}}{\pgfqpoint{0.963617in}{1.689297in}}{\pgfqpoint{0.957793in}{1.695121in}}%
\pgfpathcurveto{\pgfqpoint{0.951969in}{1.700945in}}{\pgfqpoint{0.944069in}{1.704217in}}{\pgfqpoint{0.935833in}{1.704217in}}%
\pgfpathcurveto{\pgfqpoint{0.927597in}{1.704217in}}{\pgfqpoint{0.919696in}{1.700945in}}{\pgfqpoint{0.913873in}{1.695121in}}%
\pgfpathcurveto{\pgfqpoint{0.908049in}{1.689297in}}{\pgfqpoint{0.904776in}{1.681397in}}{\pgfqpoint{0.904776in}{1.673161in}}%
\pgfpathcurveto{\pgfqpoint{0.904776in}{1.664925in}}{\pgfqpoint{0.908049in}{1.657025in}}{\pgfqpoint{0.913873in}{1.651201in}}%
\pgfpathcurveto{\pgfqpoint{0.919696in}{1.645377in}}{\pgfqpoint{0.927597in}{1.642104in}}{\pgfqpoint{0.935833in}{1.642104in}}%
\pgfpathclose%
\pgfusepath{stroke,fill}%
\end{pgfscope}%
\begin{pgfscope}%
\pgfpathrectangle{\pgfqpoint{0.100000in}{0.212622in}}{\pgfqpoint{3.696000in}{3.696000in}}%
\pgfusepath{clip}%
\pgfsetbuttcap%
\pgfsetroundjoin%
\definecolor{currentfill}{rgb}{0.121569,0.466667,0.705882}%
\pgfsetfillcolor{currentfill}%
\pgfsetfillopacity{0.607422}%
\pgfsetlinewidth{1.003750pt}%
\definecolor{currentstroke}{rgb}{0.121569,0.466667,0.705882}%
\pgfsetstrokecolor{currentstroke}%
\pgfsetstrokeopacity{0.607422}%
\pgfsetdash{}{0pt}%
\pgfpathmoveto{\pgfqpoint{0.934465in}{1.639705in}}%
\pgfpathcurveto{\pgfqpoint{0.942701in}{1.639705in}}{\pgfqpoint{0.950601in}{1.642977in}}{\pgfqpoint{0.956425in}{1.648801in}}%
\pgfpathcurveto{\pgfqpoint{0.962249in}{1.654625in}}{\pgfqpoint{0.965521in}{1.662525in}}{\pgfqpoint{0.965521in}{1.670761in}}%
\pgfpathcurveto{\pgfqpoint{0.965521in}{1.678997in}}{\pgfqpoint{0.962249in}{1.686897in}}{\pgfqpoint{0.956425in}{1.692721in}}%
\pgfpathcurveto{\pgfqpoint{0.950601in}{1.698545in}}{\pgfqpoint{0.942701in}{1.701818in}}{\pgfqpoint{0.934465in}{1.701818in}}%
\pgfpathcurveto{\pgfqpoint{0.926228in}{1.701818in}}{\pgfqpoint{0.918328in}{1.698545in}}{\pgfqpoint{0.912504in}{1.692721in}}%
\pgfpathcurveto{\pgfqpoint{0.906680in}{1.686897in}}{\pgfqpoint{0.903408in}{1.678997in}}{\pgfqpoint{0.903408in}{1.670761in}}%
\pgfpathcurveto{\pgfqpoint{0.903408in}{1.662525in}}{\pgfqpoint{0.906680in}{1.654625in}}{\pgfqpoint{0.912504in}{1.648801in}}%
\pgfpathcurveto{\pgfqpoint{0.918328in}{1.642977in}}{\pgfqpoint{0.926228in}{1.639705in}}{\pgfqpoint{0.934465in}{1.639705in}}%
\pgfpathclose%
\pgfusepath{stroke,fill}%
\end{pgfscope}%
\begin{pgfscope}%
\pgfpathrectangle{\pgfqpoint{0.100000in}{0.212622in}}{\pgfqpoint{3.696000in}{3.696000in}}%
\pgfusepath{clip}%
\pgfsetbuttcap%
\pgfsetroundjoin%
\definecolor{currentfill}{rgb}{0.121569,0.466667,0.705882}%
\pgfsetfillcolor{currentfill}%
\pgfsetfillopacity{0.607776}%
\pgfsetlinewidth{1.003750pt}%
\definecolor{currentstroke}{rgb}{0.121569,0.466667,0.705882}%
\pgfsetstrokecolor{currentstroke}%
\pgfsetstrokeopacity{0.607776}%
\pgfsetdash{}{0pt}%
\pgfpathmoveto{\pgfqpoint{0.933657in}{1.637440in}}%
\pgfpathcurveto{\pgfqpoint{0.941893in}{1.637440in}}{\pgfqpoint{0.949793in}{1.640712in}}{\pgfqpoint{0.955617in}{1.646536in}}%
\pgfpathcurveto{\pgfqpoint{0.961441in}{1.652360in}}{\pgfqpoint{0.964713in}{1.660260in}}{\pgfqpoint{0.964713in}{1.668496in}}%
\pgfpathcurveto{\pgfqpoint{0.964713in}{1.676733in}}{\pgfqpoint{0.961441in}{1.684633in}}{\pgfqpoint{0.955617in}{1.690457in}}%
\pgfpathcurveto{\pgfqpoint{0.949793in}{1.696280in}}{\pgfqpoint{0.941893in}{1.699553in}}{\pgfqpoint{0.933657in}{1.699553in}}%
\pgfpathcurveto{\pgfqpoint{0.925421in}{1.699553in}}{\pgfqpoint{0.917521in}{1.696280in}}{\pgfqpoint{0.911697in}{1.690457in}}%
\pgfpathcurveto{\pgfqpoint{0.905873in}{1.684633in}}{\pgfqpoint{0.902600in}{1.676733in}}{\pgfqpoint{0.902600in}{1.668496in}}%
\pgfpathcurveto{\pgfqpoint{0.902600in}{1.660260in}}{\pgfqpoint{0.905873in}{1.652360in}}{\pgfqpoint{0.911697in}{1.646536in}}%
\pgfpathcurveto{\pgfqpoint{0.917521in}{1.640712in}}{\pgfqpoint{0.925421in}{1.637440in}}{\pgfqpoint{0.933657in}{1.637440in}}%
\pgfpathclose%
\pgfusepath{stroke,fill}%
\end{pgfscope}%
\begin{pgfscope}%
\pgfpathrectangle{\pgfqpoint{0.100000in}{0.212622in}}{\pgfqpoint{3.696000in}{3.696000in}}%
\pgfusepath{clip}%
\pgfsetbuttcap%
\pgfsetroundjoin%
\definecolor{currentfill}{rgb}{0.121569,0.466667,0.705882}%
\pgfsetfillcolor{currentfill}%
\pgfsetfillopacity{0.607889}%
\pgfsetlinewidth{1.003750pt}%
\definecolor{currentstroke}{rgb}{0.121569,0.466667,0.705882}%
\pgfsetstrokecolor{currentstroke}%
\pgfsetstrokeopacity{0.607889}%
\pgfsetdash{}{0pt}%
\pgfpathmoveto{\pgfqpoint{2.107971in}{2.065821in}}%
\pgfpathcurveto{\pgfqpoint{2.116208in}{2.065821in}}{\pgfqpoint{2.124108in}{2.069093in}}{\pgfqpoint{2.129932in}{2.074917in}}%
\pgfpathcurveto{\pgfqpoint{2.135756in}{2.080741in}}{\pgfqpoint{2.139028in}{2.088641in}}{\pgfqpoint{2.139028in}{2.096877in}}%
\pgfpathcurveto{\pgfqpoint{2.139028in}{2.105114in}}{\pgfqpoint{2.135756in}{2.113014in}}{\pgfqpoint{2.129932in}{2.118838in}}%
\pgfpathcurveto{\pgfqpoint{2.124108in}{2.124661in}}{\pgfqpoint{2.116208in}{2.127934in}}{\pgfqpoint{2.107971in}{2.127934in}}%
\pgfpathcurveto{\pgfqpoint{2.099735in}{2.127934in}}{\pgfqpoint{2.091835in}{2.124661in}}{\pgfqpoint{2.086011in}{2.118838in}}%
\pgfpathcurveto{\pgfqpoint{2.080187in}{2.113014in}}{\pgfqpoint{2.076915in}{2.105114in}}{\pgfqpoint{2.076915in}{2.096877in}}%
\pgfpathcurveto{\pgfqpoint{2.076915in}{2.088641in}}{\pgfqpoint{2.080187in}{2.080741in}}{\pgfqpoint{2.086011in}{2.074917in}}%
\pgfpathcurveto{\pgfqpoint{2.091835in}{2.069093in}}{\pgfqpoint{2.099735in}{2.065821in}}{\pgfqpoint{2.107971in}{2.065821in}}%
\pgfpathclose%
\pgfusepath{stroke,fill}%
\end{pgfscope}%
\begin{pgfscope}%
\pgfpathrectangle{\pgfqpoint{0.100000in}{0.212622in}}{\pgfqpoint{3.696000in}{3.696000in}}%
\pgfusepath{clip}%
\pgfsetbuttcap%
\pgfsetroundjoin%
\definecolor{currentfill}{rgb}{0.121569,0.466667,0.705882}%
\pgfsetfillcolor{currentfill}%
\pgfsetfillopacity{0.608418}%
\pgfsetlinewidth{1.003750pt}%
\definecolor{currentstroke}{rgb}{0.121569,0.466667,0.705882}%
\pgfsetstrokecolor{currentstroke}%
\pgfsetstrokeopacity{0.608418}%
\pgfsetdash{}{0pt}%
\pgfpathmoveto{\pgfqpoint{0.931672in}{1.633826in}}%
\pgfpathcurveto{\pgfqpoint{0.939908in}{1.633826in}}{\pgfqpoint{0.947808in}{1.637099in}}{\pgfqpoint{0.953632in}{1.642923in}}%
\pgfpathcurveto{\pgfqpoint{0.959456in}{1.648747in}}{\pgfqpoint{0.962728in}{1.656647in}}{\pgfqpoint{0.962728in}{1.664883in}}%
\pgfpathcurveto{\pgfqpoint{0.962728in}{1.673119in}}{\pgfqpoint{0.959456in}{1.681019in}}{\pgfqpoint{0.953632in}{1.686843in}}%
\pgfpathcurveto{\pgfqpoint{0.947808in}{1.692667in}}{\pgfqpoint{0.939908in}{1.695939in}}{\pgfqpoint{0.931672in}{1.695939in}}%
\pgfpathcurveto{\pgfqpoint{0.923435in}{1.695939in}}{\pgfqpoint{0.915535in}{1.692667in}}{\pgfqpoint{0.909711in}{1.686843in}}%
\pgfpathcurveto{\pgfqpoint{0.903887in}{1.681019in}}{\pgfqpoint{0.900615in}{1.673119in}}{\pgfqpoint{0.900615in}{1.664883in}}%
\pgfpathcurveto{\pgfqpoint{0.900615in}{1.656647in}}{\pgfqpoint{0.903887in}{1.648747in}}{\pgfqpoint{0.909711in}{1.642923in}}%
\pgfpathcurveto{\pgfqpoint{0.915535in}{1.637099in}}{\pgfqpoint{0.923435in}{1.633826in}}{\pgfqpoint{0.931672in}{1.633826in}}%
\pgfpathclose%
\pgfusepath{stroke,fill}%
\end{pgfscope}%
\begin{pgfscope}%
\pgfpathrectangle{\pgfqpoint{0.100000in}{0.212622in}}{\pgfqpoint{3.696000in}{3.696000in}}%
\pgfusepath{clip}%
\pgfsetbuttcap%
\pgfsetroundjoin%
\definecolor{currentfill}{rgb}{0.121569,0.466667,0.705882}%
\pgfsetfillcolor{currentfill}%
\pgfsetfillopacity{0.608834}%
\pgfsetlinewidth{1.003750pt}%
\definecolor{currentstroke}{rgb}{0.121569,0.466667,0.705882}%
\pgfsetstrokecolor{currentstroke}%
\pgfsetstrokeopacity{0.608834}%
\pgfsetdash{}{0pt}%
\pgfpathmoveto{\pgfqpoint{0.669070in}{1.204320in}}%
\pgfpathcurveto{\pgfqpoint{0.677306in}{1.204320in}}{\pgfqpoint{0.685206in}{1.207592in}}{\pgfqpoint{0.691030in}{1.213416in}}%
\pgfpathcurveto{\pgfqpoint{0.696854in}{1.219240in}}{\pgfqpoint{0.700127in}{1.227140in}}{\pgfqpoint{0.700127in}{1.235377in}}%
\pgfpathcurveto{\pgfqpoint{0.700127in}{1.243613in}}{\pgfqpoint{0.696854in}{1.251513in}}{\pgfqpoint{0.691030in}{1.257337in}}%
\pgfpathcurveto{\pgfqpoint{0.685206in}{1.263161in}}{\pgfqpoint{0.677306in}{1.266433in}}{\pgfqpoint{0.669070in}{1.266433in}}%
\pgfpathcurveto{\pgfqpoint{0.660834in}{1.266433in}}{\pgfqpoint{0.652934in}{1.263161in}}{\pgfqpoint{0.647110in}{1.257337in}}%
\pgfpathcurveto{\pgfqpoint{0.641286in}{1.251513in}}{\pgfqpoint{0.638014in}{1.243613in}}{\pgfqpoint{0.638014in}{1.235377in}}%
\pgfpathcurveto{\pgfqpoint{0.638014in}{1.227140in}}{\pgfqpoint{0.641286in}{1.219240in}}{\pgfqpoint{0.647110in}{1.213416in}}%
\pgfpathcurveto{\pgfqpoint{0.652934in}{1.207592in}}{\pgfqpoint{0.660834in}{1.204320in}}{\pgfqpoint{0.669070in}{1.204320in}}%
\pgfpathclose%
\pgfusepath{stroke,fill}%
\end{pgfscope}%
\begin{pgfscope}%
\pgfpathrectangle{\pgfqpoint{0.100000in}{0.212622in}}{\pgfqpoint{3.696000in}{3.696000in}}%
\pgfusepath{clip}%
\pgfsetbuttcap%
\pgfsetroundjoin%
\definecolor{currentfill}{rgb}{0.121569,0.466667,0.705882}%
\pgfsetfillcolor{currentfill}%
\pgfsetfillopacity{0.609524}%
\pgfsetlinewidth{1.003750pt}%
\definecolor{currentstroke}{rgb}{0.121569,0.466667,0.705882}%
\pgfsetstrokecolor{currentstroke}%
\pgfsetstrokeopacity{0.609524}%
\pgfsetdash{}{0pt}%
\pgfpathmoveto{\pgfqpoint{0.928019in}{1.627090in}}%
\pgfpathcurveto{\pgfqpoint{0.936255in}{1.627090in}}{\pgfqpoint{0.944155in}{1.630362in}}{\pgfqpoint{0.949979in}{1.636186in}}%
\pgfpathcurveto{\pgfqpoint{0.955803in}{1.642010in}}{\pgfqpoint{0.959075in}{1.649910in}}{\pgfqpoint{0.959075in}{1.658146in}}%
\pgfpathcurveto{\pgfqpoint{0.959075in}{1.666383in}}{\pgfqpoint{0.955803in}{1.674283in}}{\pgfqpoint{0.949979in}{1.680107in}}%
\pgfpathcurveto{\pgfqpoint{0.944155in}{1.685931in}}{\pgfqpoint{0.936255in}{1.689203in}}{\pgfqpoint{0.928019in}{1.689203in}}%
\pgfpathcurveto{\pgfqpoint{0.919783in}{1.689203in}}{\pgfqpoint{0.911883in}{1.685931in}}{\pgfqpoint{0.906059in}{1.680107in}}%
\pgfpathcurveto{\pgfqpoint{0.900235in}{1.674283in}}{\pgfqpoint{0.896962in}{1.666383in}}{\pgfqpoint{0.896962in}{1.658146in}}%
\pgfpathcurveto{\pgfqpoint{0.896962in}{1.649910in}}{\pgfqpoint{0.900235in}{1.642010in}}{\pgfqpoint{0.906059in}{1.636186in}}%
\pgfpathcurveto{\pgfqpoint{0.911883in}{1.630362in}}{\pgfqpoint{0.919783in}{1.627090in}}{\pgfqpoint{0.928019in}{1.627090in}}%
\pgfpathclose%
\pgfusepath{stroke,fill}%
\end{pgfscope}%
\begin{pgfscope}%
\pgfpathrectangle{\pgfqpoint{0.100000in}{0.212622in}}{\pgfqpoint{3.696000in}{3.696000in}}%
\pgfusepath{clip}%
\pgfsetbuttcap%
\pgfsetroundjoin%
\definecolor{currentfill}{rgb}{0.121569,0.466667,0.705882}%
\pgfsetfillcolor{currentfill}%
\pgfsetfillopacity{0.609793}%
\pgfsetlinewidth{1.003750pt}%
\definecolor{currentstroke}{rgb}{0.121569,0.466667,0.705882}%
\pgfsetstrokecolor{currentstroke}%
\pgfsetstrokeopacity{0.609793}%
\pgfsetdash{}{0pt}%
\pgfpathmoveto{\pgfqpoint{2.109164in}{2.060881in}}%
\pgfpathcurveto{\pgfqpoint{2.117400in}{2.060881in}}{\pgfqpoint{2.125300in}{2.064154in}}{\pgfqpoint{2.131124in}{2.069978in}}%
\pgfpathcurveto{\pgfqpoint{2.136948in}{2.075802in}}{\pgfqpoint{2.140220in}{2.083702in}}{\pgfqpoint{2.140220in}{2.091938in}}%
\pgfpathcurveto{\pgfqpoint{2.140220in}{2.100174in}}{\pgfqpoint{2.136948in}{2.108074in}}{\pgfqpoint{2.131124in}{2.113898in}}%
\pgfpathcurveto{\pgfqpoint{2.125300in}{2.119722in}}{\pgfqpoint{2.117400in}{2.122994in}}{\pgfqpoint{2.109164in}{2.122994in}}%
\pgfpathcurveto{\pgfqpoint{2.100928in}{2.122994in}}{\pgfqpoint{2.093028in}{2.119722in}}{\pgfqpoint{2.087204in}{2.113898in}}%
\pgfpathcurveto{\pgfqpoint{2.081380in}{2.108074in}}{\pgfqpoint{2.078107in}{2.100174in}}{\pgfqpoint{2.078107in}{2.091938in}}%
\pgfpathcurveto{\pgfqpoint{2.078107in}{2.083702in}}{\pgfqpoint{2.081380in}{2.075802in}}{\pgfqpoint{2.087204in}{2.069978in}}%
\pgfpathcurveto{\pgfqpoint{2.093028in}{2.064154in}}{\pgfqpoint{2.100928in}{2.060881in}}{\pgfqpoint{2.109164in}{2.060881in}}%
\pgfpathclose%
\pgfusepath{stroke,fill}%
\end{pgfscope}%
\begin{pgfscope}%
\pgfpathrectangle{\pgfqpoint{0.100000in}{0.212622in}}{\pgfqpoint{3.696000in}{3.696000in}}%
\pgfusepath{clip}%
\pgfsetbuttcap%
\pgfsetroundjoin%
\definecolor{currentfill}{rgb}{0.121569,0.466667,0.705882}%
\pgfsetfillcolor{currentfill}%
\pgfsetfillopacity{0.610455}%
\pgfsetlinewidth{1.003750pt}%
\definecolor{currentstroke}{rgb}{0.121569,0.466667,0.705882}%
\pgfsetstrokecolor{currentstroke}%
\pgfsetstrokeopacity{0.610455}%
\pgfsetdash{}{0pt}%
\pgfpathmoveto{\pgfqpoint{0.924846in}{1.620738in}}%
\pgfpathcurveto{\pgfqpoint{0.933083in}{1.620738in}}{\pgfqpoint{0.940983in}{1.624011in}}{\pgfqpoint{0.946807in}{1.629835in}}%
\pgfpathcurveto{\pgfqpoint{0.952631in}{1.635659in}}{\pgfqpoint{0.955903in}{1.643559in}}{\pgfqpoint{0.955903in}{1.651795in}}%
\pgfpathcurveto{\pgfqpoint{0.955903in}{1.660031in}}{\pgfqpoint{0.952631in}{1.667931in}}{\pgfqpoint{0.946807in}{1.673755in}}%
\pgfpathcurveto{\pgfqpoint{0.940983in}{1.679579in}}{\pgfqpoint{0.933083in}{1.682851in}}{\pgfqpoint{0.924846in}{1.682851in}}%
\pgfpathcurveto{\pgfqpoint{0.916610in}{1.682851in}}{\pgfqpoint{0.908710in}{1.679579in}}{\pgfqpoint{0.902886in}{1.673755in}}%
\pgfpathcurveto{\pgfqpoint{0.897062in}{1.667931in}}{\pgfqpoint{0.893790in}{1.660031in}}{\pgfqpoint{0.893790in}{1.651795in}}%
\pgfpathcurveto{\pgfqpoint{0.893790in}{1.643559in}}{\pgfqpoint{0.897062in}{1.635659in}}{\pgfqpoint{0.902886in}{1.629835in}}%
\pgfpathcurveto{\pgfqpoint{0.908710in}{1.624011in}}{\pgfqpoint{0.916610in}{1.620738in}}{\pgfqpoint{0.924846in}{1.620738in}}%
\pgfpathclose%
\pgfusepath{stroke,fill}%
\end{pgfscope}%
\begin{pgfscope}%
\pgfpathrectangle{\pgfqpoint{0.100000in}{0.212622in}}{\pgfqpoint{3.696000in}{3.696000in}}%
\pgfusepath{clip}%
\pgfsetbuttcap%
\pgfsetroundjoin%
\definecolor{currentfill}{rgb}{0.121569,0.466667,0.705882}%
\pgfsetfillcolor{currentfill}%
\pgfsetfillopacity{0.611159}%
\pgfsetlinewidth{1.003750pt}%
\definecolor{currentstroke}{rgb}{0.121569,0.466667,0.705882}%
\pgfsetstrokecolor{currentstroke}%
\pgfsetstrokeopacity{0.611159}%
\pgfsetdash{}{0pt}%
\pgfpathmoveto{\pgfqpoint{0.922142in}{1.615834in}}%
\pgfpathcurveto{\pgfqpoint{0.930378in}{1.615834in}}{\pgfqpoint{0.938278in}{1.619106in}}{\pgfqpoint{0.944102in}{1.624930in}}%
\pgfpathcurveto{\pgfqpoint{0.949926in}{1.630754in}}{\pgfqpoint{0.953198in}{1.638654in}}{\pgfqpoint{0.953198in}{1.646890in}}%
\pgfpathcurveto{\pgfqpoint{0.953198in}{1.655127in}}{\pgfqpoint{0.949926in}{1.663027in}}{\pgfqpoint{0.944102in}{1.668851in}}%
\pgfpathcurveto{\pgfqpoint{0.938278in}{1.674675in}}{\pgfqpoint{0.930378in}{1.677947in}}{\pgfqpoint{0.922142in}{1.677947in}}%
\pgfpathcurveto{\pgfqpoint{0.913906in}{1.677947in}}{\pgfqpoint{0.906006in}{1.674675in}}{\pgfqpoint{0.900182in}{1.668851in}}%
\pgfpathcurveto{\pgfqpoint{0.894358in}{1.663027in}}{\pgfqpoint{0.891085in}{1.655127in}}{\pgfqpoint{0.891085in}{1.646890in}}%
\pgfpathcurveto{\pgfqpoint{0.891085in}{1.638654in}}{\pgfqpoint{0.894358in}{1.630754in}}{\pgfqpoint{0.900182in}{1.624930in}}%
\pgfpathcurveto{\pgfqpoint{0.906006in}{1.619106in}}{\pgfqpoint{0.913906in}{1.615834in}}{\pgfqpoint{0.922142in}{1.615834in}}%
\pgfpathclose%
\pgfusepath{stroke,fill}%
\end{pgfscope}%
\begin{pgfscope}%
\pgfpathrectangle{\pgfqpoint{0.100000in}{0.212622in}}{\pgfqpoint{3.696000in}{3.696000in}}%
\pgfusepath{clip}%
\pgfsetbuttcap%
\pgfsetroundjoin%
\definecolor{currentfill}{rgb}{0.121569,0.466667,0.705882}%
\pgfsetfillcolor{currentfill}%
\pgfsetfillopacity{0.611848}%
\pgfsetlinewidth{1.003750pt}%
\definecolor{currentstroke}{rgb}{0.121569,0.466667,0.705882}%
\pgfsetstrokecolor{currentstroke}%
\pgfsetstrokeopacity{0.611848}%
\pgfsetdash{}{0pt}%
\pgfpathmoveto{\pgfqpoint{0.919931in}{1.611514in}}%
\pgfpathcurveto{\pgfqpoint{0.928167in}{1.611514in}}{\pgfqpoint{0.936067in}{1.614787in}}{\pgfqpoint{0.941891in}{1.620611in}}%
\pgfpathcurveto{\pgfqpoint{0.947715in}{1.626435in}}{\pgfqpoint{0.950987in}{1.634335in}}{\pgfqpoint{0.950987in}{1.642571in}}%
\pgfpathcurveto{\pgfqpoint{0.950987in}{1.650807in}}{\pgfqpoint{0.947715in}{1.658707in}}{\pgfqpoint{0.941891in}{1.664531in}}%
\pgfpathcurveto{\pgfqpoint{0.936067in}{1.670355in}}{\pgfqpoint{0.928167in}{1.673627in}}{\pgfqpoint{0.919931in}{1.673627in}}%
\pgfpathcurveto{\pgfqpoint{0.911694in}{1.673627in}}{\pgfqpoint{0.903794in}{1.670355in}}{\pgfqpoint{0.897970in}{1.664531in}}%
\pgfpathcurveto{\pgfqpoint{0.892146in}{1.658707in}}{\pgfqpoint{0.888874in}{1.650807in}}{\pgfqpoint{0.888874in}{1.642571in}}%
\pgfpathcurveto{\pgfqpoint{0.888874in}{1.634335in}}{\pgfqpoint{0.892146in}{1.626435in}}{\pgfqpoint{0.897970in}{1.620611in}}%
\pgfpathcurveto{\pgfqpoint{0.903794in}{1.614787in}}{\pgfqpoint{0.911694in}{1.611514in}}{\pgfqpoint{0.919931in}{1.611514in}}%
\pgfpathclose%
\pgfusepath{stroke,fill}%
\end{pgfscope}%
\begin{pgfscope}%
\pgfpathrectangle{\pgfqpoint{0.100000in}{0.212622in}}{\pgfqpoint{3.696000in}{3.696000in}}%
\pgfusepath{clip}%
\pgfsetbuttcap%
\pgfsetroundjoin%
\definecolor{currentfill}{rgb}{0.121569,0.466667,0.705882}%
\pgfsetfillcolor{currentfill}%
\pgfsetfillopacity{0.611899}%
\pgfsetlinewidth{1.003750pt}%
\definecolor{currentstroke}{rgb}{0.121569,0.466667,0.705882}%
\pgfsetstrokecolor{currentstroke}%
\pgfsetstrokeopacity{0.611899}%
\pgfsetdash{}{0pt}%
\pgfpathmoveto{\pgfqpoint{2.110008in}{2.055721in}}%
\pgfpathcurveto{\pgfqpoint{2.118245in}{2.055721in}}{\pgfqpoint{2.126145in}{2.058993in}}{\pgfqpoint{2.131969in}{2.064817in}}%
\pgfpathcurveto{\pgfqpoint{2.137793in}{2.070641in}}{\pgfqpoint{2.141065in}{2.078541in}}{\pgfqpoint{2.141065in}{2.086777in}}%
\pgfpathcurveto{\pgfqpoint{2.141065in}{2.095014in}}{\pgfqpoint{2.137793in}{2.102914in}}{\pgfqpoint{2.131969in}{2.108738in}}%
\pgfpathcurveto{\pgfqpoint{2.126145in}{2.114562in}}{\pgfqpoint{2.118245in}{2.117834in}}{\pgfqpoint{2.110008in}{2.117834in}}%
\pgfpathcurveto{\pgfqpoint{2.101772in}{2.117834in}}{\pgfqpoint{2.093872in}{2.114562in}}{\pgfqpoint{2.088048in}{2.108738in}}%
\pgfpathcurveto{\pgfqpoint{2.082224in}{2.102914in}}{\pgfqpoint{2.078952in}{2.095014in}}{\pgfqpoint{2.078952in}{2.086777in}}%
\pgfpathcurveto{\pgfqpoint{2.078952in}{2.078541in}}{\pgfqpoint{2.082224in}{2.070641in}}{\pgfqpoint{2.088048in}{2.064817in}}%
\pgfpathcurveto{\pgfqpoint{2.093872in}{2.058993in}}{\pgfqpoint{2.101772in}{2.055721in}}{\pgfqpoint{2.110008in}{2.055721in}}%
\pgfpathclose%
\pgfusepath{stroke,fill}%
\end{pgfscope}%
\begin{pgfscope}%
\pgfpathrectangle{\pgfqpoint{0.100000in}{0.212622in}}{\pgfqpoint{3.696000in}{3.696000in}}%
\pgfusepath{clip}%
\pgfsetbuttcap%
\pgfsetroundjoin%
\definecolor{currentfill}{rgb}{0.121569,0.466667,0.705882}%
\pgfsetfillcolor{currentfill}%
\pgfsetfillopacity{0.612086}%
\pgfsetlinewidth{1.003750pt}%
\definecolor{currentstroke}{rgb}{0.121569,0.466667,0.705882}%
\pgfsetstrokecolor{currentstroke}%
\pgfsetstrokeopacity{0.612086}%
\pgfsetdash{}{0pt}%
\pgfpathmoveto{\pgfqpoint{0.681636in}{1.204953in}}%
\pgfpathcurveto{\pgfqpoint{0.689872in}{1.204953in}}{\pgfqpoint{0.697772in}{1.208225in}}{\pgfqpoint{0.703596in}{1.214049in}}%
\pgfpathcurveto{\pgfqpoint{0.709420in}{1.219873in}}{\pgfqpoint{0.712692in}{1.227773in}}{\pgfqpoint{0.712692in}{1.236009in}}%
\pgfpathcurveto{\pgfqpoint{0.712692in}{1.244246in}}{\pgfqpoint{0.709420in}{1.252146in}}{\pgfqpoint{0.703596in}{1.257970in}}%
\pgfpathcurveto{\pgfqpoint{0.697772in}{1.263793in}}{\pgfqpoint{0.689872in}{1.267066in}}{\pgfqpoint{0.681636in}{1.267066in}}%
\pgfpathcurveto{\pgfqpoint{0.673399in}{1.267066in}}{\pgfqpoint{0.665499in}{1.263793in}}{\pgfqpoint{0.659675in}{1.257970in}}%
\pgfpathcurveto{\pgfqpoint{0.653851in}{1.252146in}}{\pgfqpoint{0.650579in}{1.244246in}}{\pgfqpoint{0.650579in}{1.236009in}}%
\pgfpathcurveto{\pgfqpoint{0.650579in}{1.227773in}}{\pgfqpoint{0.653851in}{1.219873in}}{\pgfqpoint{0.659675in}{1.214049in}}%
\pgfpathcurveto{\pgfqpoint{0.665499in}{1.208225in}}{\pgfqpoint{0.673399in}{1.204953in}}{\pgfqpoint{0.681636in}{1.204953in}}%
\pgfpathclose%
\pgfusepath{stroke,fill}%
\end{pgfscope}%
\begin{pgfscope}%
\pgfpathrectangle{\pgfqpoint{0.100000in}{0.212622in}}{\pgfqpoint{3.696000in}{3.696000in}}%
\pgfusepath{clip}%
\pgfsetbuttcap%
\pgfsetroundjoin%
\definecolor{currentfill}{rgb}{0.121569,0.466667,0.705882}%
\pgfsetfillcolor{currentfill}%
\pgfsetfillopacity{0.612297}%
\pgfsetlinewidth{1.003750pt}%
\definecolor{currentstroke}{rgb}{0.121569,0.466667,0.705882}%
\pgfsetstrokecolor{currentstroke}%
\pgfsetstrokeopacity{0.612297}%
\pgfsetdash{}{0pt}%
\pgfpathmoveto{\pgfqpoint{0.866428in}{1.489010in}}%
\pgfpathcurveto{\pgfqpoint{0.874665in}{1.489010in}}{\pgfqpoint{0.882565in}{1.492283in}}{\pgfqpoint{0.888389in}{1.498107in}}%
\pgfpathcurveto{\pgfqpoint{0.894212in}{1.503931in}}{\pgfqpoint{0.897485in}{1.511831in}}{\pgfqpoint{0.897485in}{1.520067in}}%
\pgfpathcurveto{\pgfqpoint{0.897485in}{1.528303in}}{\pgfqpoint{0.894212in}{1.536203in}}{\pgfqpoint{0.888389in}{1.542027in}}%
\pgfpathcurveto{\pgfqpoint{0.882565in}{1.547851in}}{\pgfqpoint{0.874665in}{1.551123in}}{\pgfqpoint{0.866428in}{1.551123in}}%
\pgfpathcurveto{\pgfqpoint{0.858192in}{1.551123in}}{\pgfqpoint{0.850292in}{1.547851in}}{\pgfqpoint{0.844468in}{1.542027in}}%
\pgfpathcurveto{\pgfqpoint{0.838644in}{1.536203in}}{\pgfqpoint{0.835372in}{1.528303in}}{\pgfqpoint{0.835372in}{1.520067in}}%
\pgfpathcurveto{\pgfqpoint{0.835372in}{1.511831in}}{\pgfqpoint{0.838644in}{1.503931in}}{\pgfqpoint{0.844468in}{1.498107in}}%
\pgfpathcurveto{\pgfqpoint{0.850292in}{1.492283in}}{\pgfqpoint{0.858192in}{1.489010in}}{\pgfqpoint{0.866428in}{1.489010in}}%
\pgfpathclose%
\pgfusepath{stroke,fill}%
\end{pgfscope}%
\begin{pgfscope}%
\pgfpathrectangle{\pgfqpoint{0.100000in}{0.212622in}}{\pgfqpoint{3.696000in}{3.696000in}}%
\pgfusepath{clip}%
\pgfsetbuttcap%
\pgfsetroundjoin%
\definecolor{currentfill}{rgb}{0.121569,0.466667,0.705882}%
\pgfsetfillcolor{currentfill}%
\pgfsetfillopacity{0.612389}%
\pgfsetlinewidth{1.003750pt}%
\definecolor{currentstroke}{rgb}{0.121569,0.466667,0.705882}%
\pgfsetstrokecolor{currentstroke}%
\pgfsetstrokeopacity{0.612389}%
\pgfsetdash{}{0pt}%
\pgfpathmoveto{\pgfqpoint{0.918224in}{1.608576in}}%
\pgfpathcurveto{\pgfqpoint{0.926460in}{1.608576in}}{\pgfqpoint{0.934360in}{1.611848in}}{\pgfqpoint{0.940184in}{1.617672in}}%
\pgfpathcurveto{\pgfqpoint{0.946008in}{1.623496in}}{\pgfqpoint{0.949281in}{1.631396in}}{\pgfqpoint{0.949281in}{1.639632in}}%
\pgfpathcurveto{\pgfqpoint{0.949281in}{1.647869in}}{\pgfqpoint{0.946008in}{1.655769in}}{\pgfqpoint{0.940184in}{1.661593in}}%
\pgfpathcurveto{\pgfqpoint{0.934360in}{1.667416in}}{\pgfqpoint{0.926460in}{1.670689in}}{\pgfqpoint{0.918224in}{1.670689in}}%
\pgfpathcurveto{\pgfqpoint{0.909988in}{1.670689in}}{\pgfqpoint{0.902088in}{1.667416in}}{\pgfqpoint{0.896264in}{1.661593in}}%
\pgfpathcurveto{\pgfqpoint{0.890440in}{1.655769in}}{\pgfqpoint{0.887168in}{1.647869in}}{\pgfqpoint{0.887168in}{1.639632in}}%
\pgfpathcurveto{\pgfqpoint{0.887168in}{1.631396in}}{\pgfqpoint{0.890440in}{1.623496in}}{\pgfqpoint{0.896264in}{1.617672in}}%
\pgfpathcurveto{\pgfqpoint{0.902088in}{1.611848in}}{\pgfqpoint{0.909988in}{1.608576in}}{\pgfqpoint{0.918224in}{1.608576in}}%
\pgfpathclose%
\pgfusepath{stroke,fill}%
\end{pgfscope}%
\begin{pgfscope}%
\pgfpathrectangle{\pgfqpoint{0.100000in}{0.212622in}}{\pgfqpoint{3.696000in}{3.696000in}}%
\pgfusepath{clip}%
\pgfsetbuttcap%
\pgfsetroundjoin%
\definecolor{currentfill}{rgb}{0.121569,0.466667,0.705882}%
\pgfsetfillcolor{currentfill}%
\pgfsetfillopacity{0.613529}%
\pgfsetlinewidth{1.003750pt}%
\definecolor{currentstroke}{rgb}{0.121569,0.466667,0.705882}%
\pgfsetstrokecolor{currentstroke}%
\pgfsetstrokeopacity{0.613529}%
\pgfsetdash{}{0pt}%
\pgfpathmoveto{\pgfqpoint{0.915934in}{1.602945in}}%
\pgfpathcurveto{\pgfqpoint{0.924170in}{1.602945in}}{\pgfqpoint{0.932070in}{1.606217in}}{\pgfqpoint{0.937894in}{1.612041in}}%
\pgfpathcurveto{\pgfqpoint{0.943718in}{1.617865in}}{\pgfqpoint{0.946991in}{1.625765in}}{\pgfqpoint{0.946991in}{1.634001in}}%
\pgfpathcurveto{\pgfqpoint{0.946991in}{1.642237in}}{\pgfqpoint{0.943718in}{1.650138in}}{\pgfqpoint{0.937894in}{1.655961in}}%
\pgfpathcurveto{\pgfqpoint{0.932070in}{1.661785in}}{\pgfqpoint{0.924170in}{1.665058in}}{\pgfqpoint{0.915934in}{1.665058in}}%
\pgfpathcurveto{\pgfqpoint{0.907698in}{1.665058in}}{\pgfqpoint{0.899798in}{1.661785in}}{\pgfqpoint{0.893974in}{1.655961in}}%
\pgfpathcurveto{\pgfqpoint{0.888150in}{1.650138in}}{\pgfqpoint{0.884878in}{1.642237in}}{\pgfqpoint{0.884878in}{1.634001in}}%
\pgfpathcurveto{\pgfqpoint{0.884878in}{1.625765in}}{\pgfqpoint{0.888150in}{1.617865in}}{\pgfqpoint{0.893974in}{1.612041in}}%
\pgfpathcurveto{\pgfqpoint{0.899798in}{1.606217in}}{\pgfqpoint{0.907698in}{1.602945in}}{\pgfqpoint{0.915934in}{1.602945in}}%
\pgfpathclose%
\pgfusepath{stroke,fill}%
\end{pgfscope}%
\begin{pgfscope}%
\pgfpathrectangle{\pgfqpoint{0.100000in}{0.212622in}}{\pgfqpoint{3.696000in}{3.696000in}}%
\pgfusepath{clip}%
\pgfsetbuttcap%
\pgfsetroundjoin%
\definecolor{currentfill}{rgb}{0.121569,0.466667,0.705882}%
\pgfsetfillcolor{currentfill}%
\pgfsetfillopacity{0.614350}%
\pgfsetlinewidth{1.003750pt}%
\definecolor{currentstroke}{rgb}{0.121569,0.466667,0.705882}%
\pgfsetstrokecolor{currentstroke}%
\pgfsetstrokeopacity{0.614350}%
\pgfsetdash{}{0pt}%
\pgfpathmoveto{\pgfqpoint{2.111929in}{2.048638in}}%
\pgfpathcurveto{\pgfqpoint{2.120165in}{2.048638in}}{\pgfqpoint{2.128065in}{2.051910in}}{\pgfqpoint{2.133889in}{2.057734in}}%
\pgfpathcurveto{\pgfqpoint{2.139713in}{2.063558in}}{\pgfqpoint{2.142985in}{2.071458in}}{\pgfqpoint{2.142985in}{2.079694in}}%
\pgfpathcurveto{\pgfqpoint{2.142985in}{2.087931in}}{\pgfqpoint{2.139713in}{2.095831in}}{\pgfqpoint{2.133889in}{2.101655in}}%
\pgfpathcurveto{\pgfqpoint{2.128065in}{2.107479in}}{\pgfqpoint{2.120165in}{2.110751in}}{\pgfqpoint{2.111929in}{2.110751in}}%
\pgfpathcurveto{\pgfqpoint{2.103692in}{2.110751in}}{\pgfqpoint{2.095792in}{2.107479in}}{\pgfqpoint{2.089968in}{2.101655in}}%
\pgfpathcurveto{\pgfqpoint{2.084144in}{2.095831in}}{\pgfqpoint{2.080872in}{2.087931in}}{\pgfqpoint{2.080872in}{2.079694in}}%
\pgfpathcurveto{\pgfqpoint{2.080872in}{2.071458in}}{\pgfqpoint{2.084144in}{2.063558in}}{\pgfqpoint{2.089968in}{2.057734in}}%
\pgfpathcurveto{\pgfqpoint{2.095792in}{2.051910in}}{\pgfqpoint{2.103692in}{2.048638in}}{\pgfqpoint{2.111929in}{2.048638in}}%
\pgfpathclose%
\pgfusepath{stroke,fill}%
\end{pgfscope}%
\begin{pgfscope}%
\pgfpathrectangle{\pgfqpoint{0.100000in}{0.212622in}}{\pgfqpoint{3.696000in}{3.696000in}}%
\pgfusepath{clip}%
\pgfsetbuttcap%
\pgfsetroundjoin%
\definecolor{currentfill}{rgb}{0.121569,0.466667,0.705882}%
\pgfsetfillcolor{currentfill}%
\pgfsetfillopacity{0.614709}%
\pgfsetlinewidth{1.003750pt}%
\definecolor{currentstroke}{rgb}{0.121569,0.466667,0.705882}%
\pgfsetstrokecolor{currentstroke}%
\pgfsetstrokeopacity{0.614709}%
\pgfsetdash{}{0pt}%
\pgfpathmoveto{\pgfqpoint{0.913621in}{1.599486in}}%
\pgfpathcurveto{\pgfqpoint{0.921857in}{1.599486in}}{\pgfqpoint{0.929757in}{1.602758in}}{\pgfqpoint{0.935581in}{1.608582in}}%
\pgfpathcurveto{\pgfqpoint{0.941405in}{1.614406in}}{\pgfqpoint{0.944677in}{1.622306in}}{\pgfqpoint{0.944677in}{1.630542in}}%
\pgfpathcurveto{\pgfqpoint{0.944677in}{1.638778in}}{\pgfqpoint{0.941405in}{1.646678in}}{\pgfqpoint{0.935581in}{1.652502in}}%
\pgfpathcurveto{\pgfqpoint{0.929757in}{1.658326in}}{\pgfqpoint{0.921857in}{1.661599in}}{\pgfqpoint{0.913621in}{1.661599in}}%
\pgfpathcurveto{\pgfqpoint{0.905385in}{1.661599in}}{\pgfqpoint{0.897485in}{1.658326in}}{\pgfqpoint{0.891661in}{1.652502in}}%
\pgfpathcurveto{\pgfqpoint{0.885837in}{1.646678in}}{\pgfqpoint{0.882564in}{1.638778in}}{\pgfqpoint{0.882564in}{1.630542in}}%
\pgfpathcurveto{\pgfqpoint{0.882564in}{1.622306in}}{\pgfqpoint{0.885837in}{1.614406in}}{\pgfqpoint{0.891661in}{1.608582in}}%
\pgfpathcurveto{\pgfqpoint{0.897485in}{1.602758in}}{\pgfqpoint{0.905385in}{1.599486in}}{\pgfqpoint{0.913621in}{1.599486in}}%
\pgfpathclose%
\pgfusepath{stroke,fill}%
\end{pgfscope}%
\begin{pgfscope}%
\pgfpathrectangle{\pgfqpoint{0.100000in}{0.212622in}}{\pgfqpoint{3.696000in}{3.696000in}}%
\pgfusepath{clip}%
\pgfsetbuttcap%
\pgfsetroundjoin%
\definecolor{currentfill}{rgb}{0.121569,0.466667,0.705882}%
\pgfsetfillcolor{currentfill}%
\pgfsetfillopacity{0.615615}%
\pgfsetlinewidth{1.003750pt}%
\definecolor{currentstroke}{rgb}{0.121569,0.466667,0.705882}%
\pgfsetstrokecolor{currentstroke}%
\pgfsetstrokeopacity{0.615615}%
\pgfsetdash{}{0pt}%
\pgfpathmoveto{\pgfqpoint{2.113107in}{2.044557in}}%
\pgfpathcurveto{\pgfqpoint{2.121343in}{2.044557in}}{\pgfqpoint{2.129243in}{2.047830in}}{\pgfqpoint{2.135067in}{2.053654in}}%
\pgfpathcurveto{\pgfqpoint{2.140891in}{2.059478in}}{\pgfqpoint{2.144164in}{2.067378in}}{\pgfqpoint{2.144164in}{2.075614in}}%
\pgfpathcurveto{\pgfqpoint{2.144164in}{2.083850in}}{\pgfqpoint{2.140891in}{2.091750in}}{\pgfqpoint{2.135067in}{2.097574in}}%
\pgfpathcurveto{\pgfqpoint{2.129243in}{2.103398in}}{\pgfqpoint{2.121343in}{2.106670in}}{\pgfqpoint{2.113107in}{2.106670in}}%
\pgfpathcurveto{\pgfqpoint{2.104871in}{2.106670in}}{\pgfqpoint{2.096971in}{2.103398in}}{\pgfqpoint{2.091147in}{2.097574in}}%
\pgfpathcurveto{\pgfqpoint{2.085323in}{2.091750in}}{\pgfqpoint{2.082051in}{2.083850in}}{\pgfqpoint{2.082051in}{2.075614in}}%
\pgfpathcurveto{\pgfqpoint{2.082051in}{2.067378in}}{\pgfqpoint{2.085323in}{2.059478in}}{\pgfqpoint{2.091147in}{2.053654in}}%
\pgfpathcurveto{\pgfqpoint{2.096971in}{2.047830in}}{\pgfqpoint{2.104871in}{2.044557in}}{\pgfqpoint{2.113107in}{2.044557in}}%
\pgfpathclose%
\pgfusepath{stroke,fill}%
\end{pgfscope}%
\begin{pgfscope}%
\pgfpathrectangle{\pgfqpoint{0.100000in}{0.212622in}}{\pgfqpoint{3.696000in}{3.696000in}}%
\pgfusepath{clip}%
\pgfsetbuttcap%
\pgfsetroundjoin%
\definecolor{currentfill}{rgb}{0.121569,0.466667,0.705882}%
\pgfsetfillcolor{currentfill}%
\pgfsetfillopacity{0.615629}%
\pgfsetlinewidth{1.003750pt}%
\definecolor{currentstroke}{rgb}{0.121569,0.466667,0.705882}%
\pgfsetstrokecolor{currentstroke}%
\pgfsetstrokeopacity{0.615629}%
\pgfsetdash{}{0pt}%
\pgfpathmoveto{\pgfqpoint{0.911369in}{1.596392in}}%
\pgfpathcurveto{\pgfqpoint{0.919605in}{1.596392in}}{\pgfqpoint{0.927505in}{1.599664in}}{\pgfqpoint{0.933329in}{1.605488in}}%
\pgfpathcurveto{\pgfqpoint{0.939153in}{1.611312in}}{\pgfqpoint{0.942426in}{1.619212in}}{\pgfqpoint{0.942426in}{1.627449in}}%
\pgfpathcurveto{\pgfqpoint{0.942426in}{1.635685in}}{\pgfqpoint{0.939153in}{1.643585in}}{\pgfqpoint{0.933329in}{1.649409in}}%
\pgfpathcurveto{\pgfqpoint{0.927505in}{1.655233in}}{\pgfqpoint{0.919605in}{1.658505in}}{\pgfqpoint{0.911369in}{1.658505in}}%
\pgfpathcurveto{\pgfqpoint{0.903133in}{1.658505in}}{\pgfqpoint{0.895233in}{1.655233in}}{\pgfqpoint{0.889409in}{1.649409in}}%
\pgfpathcurveto{\pgfqpoint{0.883585in}{1.643585in}}{\pgfqpoint{0.880313in}{1.635685in}}{\pgfqpoint{0.880313in}{1.627449in}}%
\pgfpathcurveto{\pgfqpoint{0.880313in}{1.619212in}}{\pgfqpoint{0.883585in}{1.611312in}}{\pgfqpoint{0.889409in}{1.605488in}}%
\pgfpathcurveto{\pgfqpoint{0.895233in}{1.599664in}}{\pgfqpoint{0.903133in}{1.596392in}}{\pgfqpoint{0.911369in}{1.596392in}}%
\pgfpathclose%
\pgfusepath{stroke,fill}%
\end{pgfscope}%
\begin{pgfscope}%
\pgfpathrectangle{\pgfqpoint{0.100000in}{0.212622in}}{\pgfqpoint{3.696000in}{3.696000in}}%
\pgfusepath{clip}%
\pgfsetbuttcap%
\pgfsetroundjoin%
\definecolor{currentfill}{rgb}{0.121569,0.466667,0.705882}%
\pgfsetfillcolor{currentfill}%
\pgfsetfillopacity{0.615875}%
\pgfsetlinewidth{1.003750pt}%
\definecolor{currentstroke}{rgb}{0.121569,0.466667,0.705882}%
\pgfsetstrokecolor{currentstroke}%
\pgfsetstrokeopacity{0.615875}%
\pgfsetdash{}{0pt}%
\pgfpathmoveto{\pgfqpoint{0.693410in}{1.207581in}}%
\pgfpathcurveto{\pgfqpoint{0.701646in}{1.207581in}}{\pgfqpoint{0.709547in}{1.210853in}}{\pgfqpoint{0.715370in}{1.216677in}}%
\pgfpathcurveto{\pgfqpoint{0.721194in}{1.222501in}}{\pgfqpoint{0.724467in}{1.230401in}}{\pgfqpoint{0.724467in}{1.238638in}}%
\pgfpathcurveto{\pgfqpoint{0.724467in}{1.246874in}}{\pgfqpoint{0.721194in}{1.254774in}}{\pgfqpoint{0.715370in}{1.260598in}}%
\pgfpathcurveto{\pgfqpoint{0.709547in}{1.266422in}}{\pgfqpoint{0.701646in}{1.269694in}}{\pgfqpoint{0.693410in}{1.269694in}}%
\pgfpathcurveto{\pgfqpoint{0.685174in}{1.269694in}}{\pgfqpoint{0.677274in}{1.266422in}}{\pgfqpoint{0.671450in}{1.260598in}}%
\pgfpathcurveto{\pgfqpoint{0.665626in}{1.254774in}}{\pgfqpoint{0.662354in}{1.246874in}}{\pgfqpoint{0.662354in}{1.238638in}}%
\pgfpathcurveto{\pgfqpoint{0.662354in}{1.230401in}}{\pgfqpoint{0.665626in}{1.222501in}}{\pgfqpoint{0.671450in}{1.216677in}}%
\pgfpathcurveto{\pgfqpoint{0.677274in}{1.210853in}}{\pgfqpoint{0.685174in}{1.207581in}}{\pgfqpoint{0.693410in}{1.207581in}}%
\pgfpathclose%
\pgfusepath{stroke,fill}%
\end{pgfscope}%
\begin{pgfscope}%
\pgfpathrectangle{\pgfqpoint{0.100000in}{0.212622in}}{\pgfqpoint{3.696000in}{3.696000in}}%
\pgfusepath{clip}%
\pgfsetbuttcap%
\pgfsetroundjoin%
\definecolor{currentfill}{rgb}{0.121569,0.466667,0.705882}%
\pgfsetfillcolor{currentfill}%
\pgfsetfillopacity{0.616363}%
\pgfsetlinewidth{1.003750pt}%
\definecolor{currentstroke}{rgb}{0.121569,0.466667,0.705882}%
\pgfsetstrokecolor{currentstroke}%
\pgfsetstrokeopacity{0.616363}%
\pgfsetdash{}{0pt}%
\pgfpathmoveto{\pgfqpoint{0.909941in}{1.593836in}}%
\pgfpathcurveto{\pgfqpoint{0.918177in}{1.593836in}}{\pgfqpoint{0.926077in}{1.597109in}}{\pgfqpoint{0.931901in}{1.602933in}}%
\pgfpathcurveto{\pgfqpoint{0.937725in}{1.608756in}}{\pgfqpoint{0.940997in}{1.616657in}}{\pgfqpoint{0.940997in}{1.624893in}}%
\pgfpathcurveto{\pgfqpoint{0.940997in}{1.633129in}}{\pgfqpoint{0.937725in}{1.641029in}}{\pgfqpoint{0.931901in}{1.646853in}}%
\pgfpathcurveto{\pgfqpoint{0.926077in}{1.652677in}}{\pgfqpoint{0.918177in}{1.655949in}}{\pgfqpoint{0.909941in}{1.655949in}}%
\pgfpathcurveto{\pgfqpoint{0.901704in}{1.655949in}}{\pgfqpoint{0.893804in}{1.652677in}}{\pgfqpoint{0.887980in}{1.646853in}}%
\pgfpathcurveto{\pgfqpoint{0.882156in}{1.641029in}}{\pgfqpoint{0.878884in}{1.633129in}}{\pgfqpoint{0.878884in}{1.624893in}}%
\pgfpathcurveto{\pgfqpoint{0.878884in}{1.616657in}}{\pgfqpoint{0.882156in}{1.608756in}}{\pgfqpoint{0.887980in}{1.602933in}}%
\pgfpathcurveto{\pgfqpoint{0.893804in}{1.597109in}}{\pgfqpoint{0.901704in}{1.593836in}}{\pgfqpoint{0.909941in}{1.593836in}}%
\pgfpathclose%
\pgfusepath{stroke,fill}%
\end{pgfscope}%
\begin{pgfscope}%
\pgfpathrectangle{\pgfqpoint{0.100000in}{0.212622in}}{\pgfqpoint{3.696000in}{3.696000in}}%
\pgfusepath{clip}%
\pgfsetbuttcap%
\pgfsetroundjoin%
\definecolor{currentfill}{rgb}{0.121569,0.466667,0.705882}%
\pgfsetfillcolor{currentfill}%
\pgfsetfillopacity{0.616725}%
\pgfsetlinewidth{1.003750pt}%
\definecolor{currentstroke}{rgb}{0.121569,0.466667,0.705882}%
\pgfsetstrokecolor{currentstroke}%
\pgfsetstrokeopacity{0.616725}%
\pgfsetdash{}{0pt}%
\pgfpathmoveto{\pgfqpoint{0.908989in}{1.592460in}}%
\pgfpathcurveto{\pgfqpoint{0.917225in}{1.592460in}}{\pgfqpoint{0.925125in}{1.595733in}}{\pgfqpoint{0.930949in}{1.601557in}}%
\pgfpathcurveto{\pgfqpoint{0.936773in}{1.607381in}}{\pgfqpoint{0.940045in}{1.615281in}}{\pgfqpoint{0.940045in}{1.623517in}}%
\pgfpathcurveto{\pgfqpoint{0.940045in}{1.631753in}}{\pgfqpoint{0.936773in}{1.639653in}}{\pgfqpoint{0.930949in}{1.645477in}}%
\pgfpathcurveto{\pgfqpoint{0.925125in}{1.651301in}}{\pgfqpoint{0.917225in}{1.654573in}}{\pgfqpoint{0.908989in}{1.654573in}}%
\pgfpathcurveto{\pgfqpoint{0.900752in}{1.654573in}}{\pgfqpoint{0.892852in}{1.651301in}}{\pgfqpoint{0.887028in}{1.645477in}}%
\pgfpathcurveto{\pgfqpoint{0.881204in}{1.639653in}}{\pgfqpoint{0.877932in}{1.631753in}}{\pgfqpoint{0.877932in}{1.623517in}}%
\pgfpathcurveto{\pgfqpoint{0.877932in}{1.615281in}}{\pgfqpoint{0.881204in}{1.607381in}}{\pgfqpoint{0.887028in}{1.601557in}}%
\pgfpathcurveto{\pgfqpoint{0.892852in}{1.595733in}}{\pgfqpoint{0.900752in}{1.592460in}}{\pgfqpoint{0.908989in}{1.592460in}}%
\pgfpathclose%
\pgfusepath{stroke,fill}%
\end{pgfscope}%
\begin{pgfscope}%
\pgfpathrectangle{\pgfqpoint{0.100000in}{0.212622in}}{\pgfqpoint{3.696000in}{3.696000in}}%
\pgfusepath{clip}%
\pgfsetbuttcap%
\pgfsetroundjoin%
\definecolor{currentfill}{rgb}{0.121569,0.466667,0.705882}%
\pgfsetfillcolor{currentfill}%
\pgfsetfillopacity{0.616853}%
\pgfsetlinewidth{1.003750pt}%
\definecolor{currentstroke}{rgb}{0.121569,0.466667,0.705882}%
\pgfsetstrokecolor{currentstroke}%
\pgfsetstrokeopacity{0.616853}%
\pgfsetdash{}{0pt}%
\pgfpathmoveto{\pgfqpoint{0.908738in}{1.591994in}}%
\pgfpathcurveto{\pgfqpoint{0.916975in}{1.591994in}}{\pgfqpoint{0.924875in}{1.595266in}}{\pgfqpoint{0.930699in}{1.601090in}}%
\pgfpathcurveto{\pgfqpoint{0.936523in}{1.606914in}}{\pgfqpoint{0.939795in}{1.614814in}}{\pgfqpoint{0.939795in}{1.623051in}}%
\pgfpathcurveto{\pgfqpoint{0.939795in}{1.631287in}}{\pgfqpoint{0.936523in}{1.639187in}}{\pgfqpoint{0.930699in}{1.645011in}}%
\pgfpathcurveto{\pgfqpoint{0.924875in}{1.650835in}}{\pgfqpoint{0.916975in}{1.654107in}}{\pgfqpoint{0.908738in}{1.654107in}}%
\pgfpathcurveto{\pgfqpoint{0.900502in}{1.654107in}}{\pgfqpoint{0.892602in}{1.650835in}}{\pgfqpoint{0.886778in}{1.645011in}}%
\pgfpathcurveto{\pgfqpoint{0.880954in}{1.639187in}}{\pgfqpoint{0.877682in}{1.631287in}}{\pgfqpoint{0.877682in}{1.623051in}}%
\pgfpathcurveto{\pgfqpoint{0.877682in}{1.614814in}}{\pgfqpoint{0.880954in}{1.606914in}}{\pgfqpoint{0.886778in}{1.601090in}}%
\pgfpathcurveto{\pgfqpoint{0.892602in}{1.595266in}}{\pgfqpoint{0.900502in}{1.591994in}}{\pgfqpoint{0.908738in}{1.591994in}}%
\pgfpathclose%
\pgfusepath{stroke,fill}%
\end{pgfscope}%
\begin{pgfscope}%
\pgfpathrectangle{\pgfqpoint{0.100000in}{0.212622in}}{\pgfqpoint{3.696000in}{3.696000in}}%
\pgfusepath{clip}%
\pgfsetbuttcap%
\pgfsetroundjoin%
\definecolor{currentfill}{rgb}{0.121569,0.466667,0.705882}%
\pgfsetfillcolor{currentfill}%
\pgfsetfillopacity{0.617045}%
\pgfsetlinewidth{1.003750pt}%
\definecolor{currentstroke}{rgb}{0.121569,0.466667,0.705882}%
\pgfsetstrokecolor{currentstroke}%
\pgfsetstrokeopacity{0.617045}%
\pgfsetdash{}{0pt}%
\pgfpathmoveto{\pgfqpoint{0.908196in}{1.591103in}}%
\pgfpathcurveto{\pgfqpoint{0.916432in}{1.591103in}}{\pgfqpoint{0.924332in}{1.594376in}}{\pgfqpoint{0.930156in}{1.600200in}}%
\pgfpathcurveto{\pgfqpoint{0.935980in}{1.606023in}}{\pgfqpoint{0.939252in}{1.613924in}}{\pgfqpoint{0.939252in}{1.622160in}}%
\pgfpathcurveto{\pgfqpoint{0.939252in}{1.630396in}}{\pgfqpoint{0.935980in}{1.638296in}}{\pgfqpoint{0.930156in}{1.644120in}}%
\pgfpathcurveto{\pgfqpoint{0.924332in}{1.649944in}}{\pgfqpoint{0.916432in}{1.653216in}}{\pgfqpoint{0.908196in}{1.653216in}}%
\pgfpathcurveto{\pgfqpoint{0.899960in}{1.653216in}}{\pgfqpoint{0.892059in}{1.649944in}}{\pgfqpoint{0.886236in}{1.644120in}}%
\pgfpathcurveto{\pgfqpoint{0.880412in}{1.638296in}}{\pgfqpoint{0.877139in}{1.630396in}}{\pgfqpoint{0.877139in}{1.622160in}}%
\pgfpathcurveto{\pgfqpoint{0.877139in}{1.613924in}}{\pgfqpoint{0.880412in}{1.606023in}}{\pgfqpoint{0.886236in}{1.600200in}}%
\pgfpathcurveto{\pgfqpoint{0.892059in}{1.594376in}}{\pgfqpoint{0.899960in}{1.591103in}}{\pgfqpoint{0.908196in}{1.591103in}}%
\pgfpathclose%
\pgfusepath{stroke,fill}%
\end{pgfscope}%
\begin{pgfscope}%
\pgfpathrectangle{\pgfqpoint{0.100000in}{0.212622in}}{\pgfqpoint{3.696000in}{3.696000in}}%
\pgfusepath{clip}%
\pgfsetbuttcap%
\pgfsetroundjoin%
\definecolor{currentfill}{rgb}{0.121569,0.466667,0.705882}%
\pgfsetfillcolor{currentfill}%
\pgfsetfillopacity{0.617275}%
\pgfsetlinewidth{1.003750pt}%
\definecolor{currentstroke}{rgb}{0.121569,0.466667,0.705882}%
\pgfsetstrokecolor{currentstroke}%
\pgfsetstrokeopacity{0.617275}%
\pgfsetdash{}{0pt}%
\pgfpathmoveto{\pgfqpoint{2.113950in}{2.040129in}}%
\pgfpathcurveto{\pgfqpoint{2.122186in}{2.040129in}}{\pgfqpoint{2.130087in}{2.043401in}}{\pgfqpoint{2.135910in}{2.049225in}}%
\pgfpathcurveto{\pgfqpoint{2.141734in}{2.055049in}}{\pgfqpoint{2.145007in}{2.062949in}}{\pgfqpoint{2.145007in}{2.071185in}}%
\pgfpathcurveto{\pgfqpoint{2.145007in}{2.079422in}}{\pgfqpoint{2.141734in}{2.087322in}}{\pgfqpoint{2.135910in}{2.093146in}}%
\pgfpathcurveto{\pgfqpoint{2.130087in}{2.098970in}}{\pgfqpoint{2.122186in}{2.102242in}}{\pgfqpoint{2.113950in}{2.102242in}}%
\pgfpathcurveto{\pgfqpoint{2.105714in}{2.102242in}}{\pgfqpoint{2.097814in}{2.098970in}}{\pgfqpoint{2.091990in}{2.093146in}}%
\pgfpathcurveto{\pgfqpoint{2.086166in}{2.087322in}}{\pgfqpoint{2.082894in}{2.079422in}}{\pgfqpoint{2.082894in}{2.071185in}}%
\pgfpathcurveto{\pgfqpoint{2.082894in}{2.062949in}}{\pgfqpoint{2.086166in}{2.055049in}}{\pgfqpoint{2.091990in}{2.049225in}}%
\pgfpathcurveto{\pgfqpoint{2.097814in}{2.043401in}}{\pgfqpoint{2.105714in}{2.040129in}}{\pgfqpoint{2.113950in}{2.040129in}}%
\pgfpathclose%
\pgfusepath{stroke,fill}%
\end{pgfscope}%
\begin{pgfscope}%
\pgfpathrectangle{\pgfqpoint{0.100000in}{0.212622in}}{\pgfqpoint{3.696000in}{3.696000in}}%
\pgfusepath{clip}%
\pgfsetbuttcap%
\pgfsetroundjoin%
\definecolor{currentfill}{rgb}{0.121569,0.466667,0.705882}%
\pgfsetfillcolor{currentfill}%
\pgfsetfillopacity{0.617389}%
\pgfsetlinewidth{1.003750pt}%
\definecolor{currentstroke}{rgb}{0.121569,0.466667,0.705882}%
\pgfsetstrokecolor{currentstroke}%
\pgfsetstrokeopacity{0.617389}%
\pgfsetdash{}{0pt}%
\pgfpathmoveto{\pgfqpoint{0.907250in}{1.589408in}}%
\pgfpathcurveto{\pgfqpoint{0.915486in}{1.589408in}}{\pgfqpoint{0.923386in}{1.592680in}}{\pgfqpoint{0.929210in}{1.598504in}}%
\pgfpathcurveto{\pgfqpoint{0.935034in}{1.604328in}}{\pgfqpoint{0.938306in}{1.612228in}}{\pgfqpoint{0.938306in}{1.620464in}}%
\pgfpathcurveto{\pgfqpoint{0.938306in}{1.628700in}}{\pgfqpoint{0.935034in}{1.636600in}}{\pgfqpoint{0.929210in}{1.642424in}}%
\pgfpathcurveto{\pgfqpoint{0.923386in}{1.648248in}}{\pgfqpoint{0.915486in}{1.651521in}}{\pgfqpoint{0.907250in}{1.651521in}}%
\pgfpathcurveto{\pgfqpoint{0.899014in}{1.651521in}}{\pgfqpoint{0.891114in}{1.648248in}}{\pgfqpoint{0.885290in}{1.642424in}}%
\pgfpathcurveto{\pgfqpoint{0.879466in}{1.636600in}}{\pgfqpoint{0.876193in}{1.628700in}}{\pgfqpoint{0.876193in}{1.620464in}}%
\pgfpathcurveto{\pgfqpoint{0.876193in}{1.612228in}}{\pgfqpoint{0.879466in}{1.604328in}}{\pgfqpoint{0.885290in}{1.598504in}}%
\pgfpathcurveto{\pgfqpoint{0.891114in}{1.592680in}}{\pgfqpoint{0.899014in}{1.589408in}}{\pgfqpoint{0.907250in}{1.589408in}}%
\pgfpathclose%
\pgfusepath{stroke,fill}%
\end{pgfscope}%
\begin{pgfscope}%
\pgfpathrectangle{\pgfqpoint{0.100000in}{0.212622in}}{\pgfqpoint{3.696000in}{3.696000in}}%
\pgfusepath{clip}%
\pgfsetbuttcap%
\pgfsetroundjoin%
\definecolor{currentfill}{rgb}{0.121569,0.466667,0.705882}%
\pgfsetfillcolor{currentfill}%
\pgfsetfillopacity{0.617933}%
\pgfsetlinewidth{1.003750pt}%
\definecolor{currentstroke}{rgb}{0.121569,0.466667,0.705882}%
\pgfsetstrokecolor{currentstroke}%
\pgfsetstrokeopacity{0.617933}%
\pgfsetdash{}{0pt}%
\pgfpathmoveto{\pgfqpoint{0.905543in}{1.586014in}}%
\pgfpathcurveto{\pgfqpoint{0.913780in}{1.586014in}}{\pgfqpoint{0.921680in}{1.589287in}}{\pgfqpoint{0.927504in}{1.595111in}}%
\pgfpathcurveto{\pgfqpoint{0.933328in}{1.600935in}}{\pgfqpoint{0.936600in}{1.608835in}}{\pgfqpoint{0.936600in}{1.617071in}}%
\pgfpathcurveto{\pgfqpoint{0.936600in}{1.625307in}}{\pgfqpoint{0.933328in}{1.633207in}}{\pgfqpoint{0.927504in}{1.639031in}}%
\pgfpathcurveto{\pgfqpoint{0.921680in}{1.644855in}}{\pgfqpoint{0.913780in}{1.648127in}}{\pgfqpoint{0.905543in}{1.648127in}}%
\pgfpathcurveto{\pgfqpoint{0.897307in}{1.648127in}}{\pgfqpoint{0.889407in}{1.644855in}}{\pgfqpoint{0.883583in}{1.639031in}}%
\pgfpathcurveto{\pgfqpoint{0.877759in}{1.633207in}}{\pgfqpoint{0.874487in}{1.625307in}}{\pgfqpoint{0.874487in}{1.617071in}}%
\pgfpathcurveto{\pgfqpoint{0.874487in}{1.608835in}}{\pgfqpoint{0.877759in}{1.600935in}}{\pgfqpoint{0.883583in}{1.595111in}}%
\pgfpathcurveto{\pgfqpoint{0.889407in}{1.589287in}}{\pgfqpoint{0.897307in}{1.586014in}}{\pgfqpoint{0.905543in}{1.586014in}}%
\pgfpathclose%
\pgfusepath{stroke,fill}%
\end{pgfscope}%
\begin{pgfscope}%
\pgfpathrectangle{\pgfqpoint{0.100000in}{0.212622in}}{\pgfqpoint{3.696000in}{3.696000in}}%
\pgfusepath{clip}%
\pgfsetbuttcap%
\pgfsetroundjoin%
\definecolor{currentfill}{rgb}{0.121569,0.466667,0.705882}%
\pgfsetfillcolor{currentfill}%
\pgfsetfillopacity{0.617944}%
\pgfsetlinewidth{1.003750pt}%
\definecolor{currentstroke}{rgb}{0.121569,0.466667,0.705882}%
\pgfsetstrokecolor{currentstroke}%
\pgfsetstrokeopacity{0.617944}%
\pgfsetdash{}{0pt}%
\pgfpathmoveto{\pgfqpoint{0.862968in}{1.490380in}}%
\pgfpathcurveto{\pgfqpoint{0.871204in}{1.490380in}}{\pgfqpoint{0.879105in}{1.493652in}}{\pgfqpoint{0.884928in}{1.499476in}}%
\pgfpathcurveto{\pgfqpoint{0.890752in}{1.505300in}}{\pgfqpoint{0.894025in}{1.513200in}}{\pgfqpoint{0.894025in}{1.521436in}}%
\pgfpathcurveto{\pgfqpoint{0.894025in}{1.529672in}}{\pgfqpoint{0.890752in}{1.537573in}}{\pgfqpoint{0.884928in}{1.543396in}}%
\pgfpathcurveto{\pgfqpoint{0.879105in}{1.549220in}}{\pgfqpoint{0.871204in}{1.552493in}}{\pgfqpoint{0.862968in}{1.552493in}}%
\pgfpathcurveto{\pgfqpoint{0.854732in}{1.552493in}}{\pgfqpoint{0.846832in}{1.549220in}}{\pgfqpoint{0.841008in}{1.543396in}}%
\pgfpathcurveto{\pgfqpoint{0.835184in}{1.537573in}}{\pgfqpoint{0.831912in}{1.529672in}}{\pgfqpoint{0.831912in}{1.521436in}}%
\pgfpathcurveto{\pgfqpoint{0.831912in}{1.513200in}}{\pgfqpoint{0.835184in}{1.505300in}}{\pgfqpoint{0.841008in}{1.499476in}}%
\pgfpathcurveto{\pgfqpoint{0.846832in}{1.493652in}}{\pgfqpoint{0.854732in}{1.490380in}}{\pgfqpoint{0.862968in}{1.490380in}}%
\pgfpathclose%
\pgfusepath{stroke,fill}%
\end{pgfscope}%
\begin{pgfscope}%
\pgfpathrectangle{\pgfqpoint{0.100000in}{0.212622in}}{\pgfqpoint{3.696000in}{3.696000in}}%
\pgfusepath{clip}%
\pgfsetbuttcap%
\pgfsetroundjoin%
\definecolor{currentfill}{rgb}{0.121569,0.466667,0.705882}%
\pgfsetfillcolor{currentfill}%
\pgfsetfillopacity{0.618203}%
\pgfsetlinewidth{1.003750pt}%
\definecolor{currentstroke}{rgb}{0.121569,0.466667,0.705882}%
\pgfsetstrokecolor{currentstroke}%
\pgfsetstrokeopacity{0.618203}%
\pgfsetdash{}{0pt}%
\pgfpathmoveto{\pgfqpoint{0.904094in}{1.583298in}}%
\pgfpathcurveto{\pgfqpoint{0.912330in}{1.583298in}}{\pgfqpoint{0.920230in}{1.586570in}}{\pgfqpoint{0.926054in}{1.592394in}}%
\pgfpathcurveto{\pgfqpoint{0.931878in}{1.598218in}}{\pgfqpoint{0.935151in}{1.606118in}}{\pgfqpoint{0.935151in}{1.614355in}}%
\pgfpathcurveto{\pgfqpoint{0.935151in}{1.622591in}}{\pgfqpoint{0.931878in}{1.630491in}}{\pgfqpoint{0.926054in}{1.636315in}}%
\pgfpathcurveto{\pgfqpoint{0.920230in}{1.642139in}}{\pgfqpoint{0.912330in}{1.645411in}}{\pgfqpoint{0.904094in}{1.645411in}}%
\pgfpathcurveto{\pgfqpoint{0.895858in}{1.645411in}}{\pgfqpoint{0.887958in}{1.642139in}}{\pgfqpoint{0.882134in}{1.636315in}}%
\pgfpathcurveto{\pgfqpoint{0.876310in}{1.630491in}}{\pgfqpoint{0.873038in}{1.622591in}}{\pgfqpoint{0.873038in}{1.614355in}}%
\pgfpathcurveto{\pgfqpoint{0.873038in}{1.606118in}}{\pgfqpoint{0.876310in}{1.598218in}}{\pgfqpoint{0.882134in}{1.592394in}}%
\pgfpathcurveto{\pgfqpoint{0.887958in}{1.586570in}}{\pgfqpoint{0.895858in}{1.583298in}}{\pgfqpoint{0.904094in}{1.583298in}}%
\pgfpathclose%
\pgfusepath{stroke,fill}%
\end{pgfscope}%
\begin{pgfscope}%
\pgfpathrectangle{\pgfqpoint{0.100000in}{0.212622in}}{\pgfqpoint{3.696000in}{3.696000in}}%
\pgfusepath{clip}%
\pgfsetbuttcap%
\pgfsetroundjoin%
\definecolor{currentfill}{rgb}{0.121569,0.466667,0.705882}%
\pgfsetfillcolor{currentfill}%
\pgfsetfillopacity{0.618439}%
\pgfsetlinewidth{1.003750pt}%
\definecolor{currentstroke}{rgb}{0.121569,0.466667,0.705882}%
\pgfsetstrokecolor{currentstroke}%
\pgfsetstrokeopacity{0.618439}%
\pgfsetdash{}{0pt}%
\pgfpathmoveto{\pgfqpoint{0.903244in}{1.581149in}}%
\pgfpathcurveto{\pgfqpoint{0.911480in}{1.581149in}}{\pgfqpoint{0.919380in}{1.584422in}}{\pgfqpoint{0.925204in}{1.590246in}}%
\pgfpathcurveto{\pgfqpoint{0.931028in}{1.596069in}}{\pgfqpoint{0.934300in}{1.603970in}}{\pgfqpoint{0.934300in}{1.612206in}}%
\pgfpathcurveto{\pgfqpoint{0.934300in}{1.620442in}}{\pgfqpoint{0.931028in}{1.628342in}}{\pgfqpoint{0.925204in}{1.634166in}}%
\pgfpathcurveto{\pgfqpoint{0.919380in}{1.639990in}}{\pgfqpoint{0.911480in}{1.643262in}}{\pgfqpoint{0.903244in}{1.643262in}}%
\pgfpathcurveto{\pgfqpoint{0.895007in}{1.643262in}}{\pgfqpoint{0.887107in}{1.639990in}}{\pgfqpoint{0.881284in}{1.634166in}}%
\pgfpathcurveto{\pgfqpoint{0.875460in}{1.628342in}}{\pgfqpoint{0.872187in}{1.620442in}}{\pgfqpoint{0.872187in}{1.612206in}}%
\pgfpathcurveto{\pgfqpoint{0.872187in}{1.603970in}}{\pgfqpoint{0.875460in}{1.596069in}}{\pgfqpoint{0.881284in}{1.590246in}}%
\pgfpathcurveto{\pgfqpoint{0.887107in}{1.584422in}}{\pgfqpoint{0.895007in}{1.581149in}}{\pgfqpoint{0.903244in}{1.581149in}}%
\pgfpathclose%
\pgfusepath{stroke,fill}%
\end{pgfscope}%
\begin{pgfscope}%
\pgfpathrectangle{\pgfqpoint{0.100000in}{0.212622in}}{\pgfqpoint{3.696000in}{3.696000in}}%
\pgfusepath{clip}%
\pgfsetbuttcap%
\pgfsetroundjoin%
\definecolor{currentfill}{rgb}{0.121569,0.466667,0.705882}%
\pgfsetfillcolor{currentfill}%
\pgfsetfillopacity{0.618519}%
\pgfsetlinewidth{1.003750pt}%
\definecolor{currentstroke}{rgb}{0.121569,0.466667,0.705882}%
\pgfsetstrokecolor{currentstroke}%
\pgfsetstrokeopacity{0.618519}%
\pgfsetdash{}{0pt}%
\pgfpathmoveto{\pgfqpoint{0.902816in}{1.580289in}}%
\pgfpathcurveto{\pgfqpoint{0.911052in}{1.580289in}}{\pgfqpoint{0.918952in}{1.583561in}}{\pgfqpoint{0.924776in}{1.589385in}}%
\pgfpathcurveto{\pgfqpoint{0.930600in}{1.595209in}}{\pgfqpoint{0.933873in}{1.603109in}}{\pgfqpoint{0.933873in}{1.611346in}}%
\pgfpathcurveto{\pgfqpoint{0.933873in}{1.619582in}}{\pgfqpoint{0.930600in}{1.627482in}}{\pgfqpoint{0.924776in}{1.633306in}}%
\pgfpathcurveto{\pgfqpoint{0.918952in}{1.639130in}}{\pgfqpoint{0.911052in}{1.642402in}}{\pgfqpoint{0.902816in}{1.642402in}}%
\pgfpathcurveto{\pgfqpoint{0.894580in}{1.642402in}}{\pgfqpoint{0.886680in}{1.639130in}}{\pgfqpoint{0.880856in}{1.633306in}}%
\pgfpathcurveto{\pgfqpoint{0.875032in}{1.627482in}}{\pgfqpoint{0.871760in}{1.619582in}}{\pgfqpoint{0.871760in}{1.611346in}}%
\pgfpathcurveto{\pgfqpoint{0.871760in}{1.603109in}}{\pgfqpoint{0.875032in}{1.595209in}}{\pgfqpoint{0.880856in}{1.589385in}}%
\pgfpathcurveto{\pgfqpoint{0.886680in}{1.583561in}}{\pgfqpoint{0.894580in}{1.580289in}}{\pgfqpoint{0.902816in}{1.580289in}}%
\pgfpathclose%
\pgfusepath{stroke,fill}%
\end{pgfscope}%
\begin{pgfscope}%
\pgfpathrectangle{\pgfqpoint{0.100000in}{0.212622in}}{\pgfqpoint{3.696000in}{3.696000in}}%
\pgfusepath{clip}%
\pgfsetbuttcap%
\pgfsetroundjoin%
\definecolor{currentfill}{rgb}{0.121569,0.466667,0.705882}%
\pgfsetfillcolor{currentfill}%
\pgfsetfillopacity{0.618672}%
\pgfsetlinewidth{1.003750pt}%
\definecolor{currentstroke}{rgb}{0.121569,0.466667,0.705882}%
\pgfsetstrokecolor{currentstroke}%
\pgfsetstrokeopacity{0.618672}%
\pgfsetdash{}{0pt}%
\pgfpathmoveto{\pgfqpoint{0.902127in}{1.578631in}}%
\pgfpathcurveto{\pgfqpoint{0.910363in}{1.578631in}}{\pgfqpoint{0.918263in}{1.581903in}}{\pgfqpoint{0.924087in}{1.587727in}}%
\pgfpathcurveto{\pgfqpoint{0.929911in}{1.593551in}}{\pgfqpoint{0.933183in}{1.601451in}}{\pgfqpoint{0.933183in}{1.609687in}}%
\pgfpathcurveto{\pgfqpoint{0.933183in}{1.617924in}}{\pgfqpoint{0.929911in}{1.625824in}}{\pgfqpoint{0.924087in}{1.631648in}}%
\pgfpathcurveto{\pgfqpoint{0.918263in}{1.637472in}}{\pgfqpoint{0.910363in}{1.640744in}}{\pgfqpoint{0.902127in}{1.640744in}}%
\pgfpathcurveto{\pgfqpoint{0.893890in}{1.640744in}}{\pgfqpoint{0.885990in}{1.637472in}}{\pgfqpoint{0.880167in}{1.631648in}}%
\pgfpathcurveto{\pgfqpoint{0.874343in}{1.625824in}}{\pgfqpoint{0.871070in}{1.617924in}}{\pgfqpoint{0.871070in}{1.609687in}}%
\pgfpathcurveto{\pgfqpoint{0.871070in}{1.601451in}}{\pgfqpoint{0.874343in}{1.593551in}}{\pgfqpoint{0.880167in}{1.587727in}}%
\pgfpathcurveto{\pgfqpoint{0.885990in}{1.581903in}}{\pgfqpoint{0.893890in}{1.578631in}}{\pgfqpoint{0.902127in}{1.578631in}}%
\pgfpathclose%
\pgfusepath{stroke,fill}%
\end{pgfscope}%
\begin{pgfscope}%
\pgfpathrectangle{\pgfqpoint{0.100000in}{0.212622in}}{\pgfqpoint{3.696000in}{3.696000in}}%
\pgfusepath{clip}%
\pgfsetbuttcap%
\pgfsetroundjoin%
\definecolor{currentfill}{rgb}{0.121569,0.466667,0.705882}%
\pgfsetfillcolor{currentfill}%
\pgfsetfillopacity{0.618817}%
\pgfsetlinewidth{1.003750pt}%
\definecolor{currentstroke}{rgb}{0.121569,0.466667,0.705882}%
\pgfsetstrokecolor{currentstroke}%
\pgfsetstrokeopacity{0.618817}%
\pgfsetdash{}{0pt}%
\pgfpathmoveto{\pgfqpoint{0.900747in}{1.575313in}}%
\pgfpathcurveto{\pgfqpoint{0.908983in}{1.575313in}}{\pgfqpoint{0.916883in}{1.578585in}}{\pgfqpoint{0.922707in}{1.584409in}}%
\pgfpathcurveto{\pgfqpoint{0.928531in}{1.590233in}}{\pgfqpoint{0.931803in}{1.598133in}}{\pgfqpoint{0.931803in}{1.606370in}}%
\pgfpathcurveto{\pgfqpoint{0.931803in}{1.614606in}}{\pgfqpoint{0.928531in}{1.622506in}}{\pgfqpoint{0.922707in}{1.628330in}}%
\pgfpathcurveto{\pgfqpoint{0.916883in}{1.634154in}}{\pgfqpoint{0.908983in}{1.637426in}}{\pgfqpoint{0.900747in}{1.637426in}}%
\pgfpathcurveto{\pgfqpoint{0.892511in}{1.637426in}}{\pgfqpoint{0.884611in}{1.634154in}}{\pgfqpoint{0.878787in}{1.628330in}}%
\pgfpathcurveto{\pgfqpoint{0.872963in}{1.622506in}}{\pgfqpoint{0.869690in}{1.614606in}}{\pgfqpoint{0.869690in}{1.606370in}}%
\pgfpathcurveto{\pgfqpoint{0.869690in}{1.598133in}}{\pgfqpoint{0.872963in}{1.590233in}}{\pgfqpoint{0.878787in}{1.584409in}}%
\pgfpathcurveto{\pgfqpoint{0.884611in}{1.578585in}}{\pgfqpoint{0.892511in}{1.575313in}}{\pgfqpoint{0.900747in}{1.575313in}}%
\pgfpathclose%
\pgfusepath{stroke,fill}%
\end{pgfscope}%
\begin{pgfscope}%
\pgfpathrectangle{\pgfqpoint{0.100000in}{0.212622in}}{\pgfqpoint{3.696000in}{3.696000in}}%
\pgfusepath{clip}%
\pgfsetbuttcap%
\pgfsetroundjoin%
\definecolor{currentfill}{rgb}{0.121569,0.466667,0.705882}%
\pgfsetfillcolor{currentfill}%
\pgfsetfillopacity{0.618871}%
\pgfsetlinewidth{1.003750pt}%
\definecolor{currentstroke}{rgb}{0.121569,0.466667,0.705882}%
\pgfsetstrokecolor{currentstroke}%
\pgfsetstrokeopacity{0.618871}%
\pgfsetdash{}{0pt}%
\pgfpathmoveto{\pgfqpoint{0.899622in}{1.572364in}}%
\pgfpathcurveto{\pgfqpoint{0.907859in}{1.572364in}}{\pgfqpoint{0.915759in}{1.575636in}}{\pgfqpoint{0.921583in}{1.581460in}}%
\pgfpathcurveto{\pgfqpoint{0.927406in}{1.587284in}}{\pgfqpoint{0.930679in}{1.595184in}}{\pgfqpoint{0.930679in}{1.603420in}}%
\pgfpathcurveto{\pgfqpoint{0.930679in}{1.611657in}}{\pgfqpoint{0.927406in}{1.619557in}}{\pgfqpoint{0.921583in}{1.625381in}}%
\pgfpathcurveto{\pgfqpoint{0.915759in}{1.631205in}}{\pgfqpoint{0.907859in}{1.634477in}}{\pgfqpoint{0.899622in}{1.634477in}}%
\pgfpathcurveto{\pgfqpoint{0.891386in}{1.634477in}}{\pgfqpoint{0.883486in}{1.631205in}}{\pgfqpoint{0.877662in}{1.625381in}}%
\pgfpathcurveto{\pgfqpoint{0.871838in}{1.619557in}}{\pgfqpoint{0.868566in}{1.611657in}}{\pgfqpoint{0.868566in}{1.603420in}}%
\pgfpathcurveto{\pgfqpoint{0.868566in}{1.595184in}}{\pgfqpoint{0.871838in}{1.587284in}}{\pgfqpoint{0.877662in}{1.581460in}}%
\pgfpathcurveto{\pgfqpoint{0.883486in}{1.575636in}}{\pgfqpoint{0.891386in}{1.572364in}}{\pgfqpoint{0.899622in}{1.572364in}}%
\pgfpathclose%
\pgfusepath{stroke,fill}%
\end{pgfscope}%
\begin{pgfscope}%
\pgfpathrectangle{\pgfqpoint{0.100000in}{0.212622in}}{\pgfqpoint{3.696000in}{3.696000in}}%
\pgfusepath{clip}%
\pgfsetbuttcap%
\pgfsetroundjoin%
\definecolor{currentfill}{rgb}{0.121569,0.466667,0.705882}%
\pgfsetfillcolor{currentfill}%
\pgfsetfillopacity{0.619157}%
\pgfsetlinewidth{1.003750pt}%
\definecolor{currentstroke}{rgb}{0.121569,0.466667,0.705882}%
\pgfsetstrokecolor{currentstroke}%
\pgfsetstrokeopacity{0.619157}%
\pgfsetdash{}{0pt}%
\pgfpathmoveto{\pgfqpoint{2.115372in}{2.034854in}}%
\pgfpathcurveto{\pgfqpoint{2.123608in}{2.034854in}}{\pgfqpoint{2.131508in}{2.038126in}}{\pgfqpoint{2.137332in}{2.043950in}}%
\pgfpathcurveto{\pgfqpoint{2.143156in}{2.049774in}}{\pgfqpoint{2.146429in}{2.057674in}}{\pgfqpoint{2.146429in}{2.065910in}}%
\pgfpathcurveto{\pgfqpoint{2.146429in}{2.074146in}}{\pgfqpoint{2.143156in}{2.082046in}}{\pgfqpoint{2.137332in}{2.087870in}}%
\pgfpathcurveto{\pgfqpoint{2.131508in}{2.093694in}}{\pgfqpoint{2.123608in}{2.096967in}}{\pgfqpoint{2.115372in}{2.096967in}}%
\pgfpathcurveto{\pgfqpoint{2.107136in}{2.096967in}}{\pgfqpoint{2.099236in}{2.093694in}}{\pgfqpoint{2.093412in}{2.087870in}}%
\pgfpathcurveto{\pgfqpoint{2.087588in}{2.082046in}}{\pgfqpoint{2.084316in}{2.074146in}}{\pgfqpoint{2.084316in}{2.065910in}}%
\pgfpathcurveto{\pgfqpoint{2.084316in}{2.057674in}}{\pgfqpoint{2.087588in}{2.049774in}}{\pgfqpoint{2.093412in}{2.043950in}}%
\pgfpathcurveto{\pgfqpoint{2.099236in}{2.038126in}}{\pgfqpoint{2.107136in}{2.034854in}}{\pgfqpoint{2.115372in}{2.034854in}}%
\pgfpathclose%
\pgfusepath{stroke,fill}%
\end{pgfscope}%
\begin{pgfscope}%
\pgfpathrectangle{\pgfqpoint{0.100000in}{0.212622in}}{\pgfqpoint{3.696000in}{3.696000in}}%
\pgfusepath{clip}%
\pgfsetbuttcap%
\pgfsetroundjoin%
\definecolor{currentfill}{rgb}{0.121569,0.466667,0.705882}%
\pgfsetfillcolor{currentfill}%
\pgfsetfillopacity{0.619272}%
\pgfsetlinewidth{1.003750pt}%
\definecolor{currentstroke}{rgb}{0.121569,0.466667,0.705882}%
\pgfsetstrokecolor{currentstroke}%
\pgfsetstrokeopacity{0.619272}%
\pgfsetdash{}{0pt}%
\pgfpathmoveto{\pgfqpoint{0.897515in}{1.568175in}}%
\pgfpathcurveto{\pgfqpoint{0.905751in}{1.568175in}}{\pgfqpoint{0.913651in}{1.571448in}}{\pgfqpoint{0.919475in}{1.577272in}}%
\pgfpathcurveto{\pgfqpoint{0.925299in}{1.583095in}}{\pgfqpoint{0.928572in}{1.590996in}}{\pgfqpoint{0.928572in}{1.599232in}}%
\pgfpathcurveto{\pgfqpoint{0.928572in}{1.607468in}}{\pgfqpoint{0.925299in}{1.615368in}}{\pgfqpoint{0.919475in}{1.621192in}}%
\pgfpathcurveto{\pgfqpoint{0.913651in}{1.627016in}}{\pgfqpoint{0.905751in}{1.630288in}}{\pgfqpoint{0.897515in}{1.630288in}}%
\pgfpathcurveto{\pgfqpoint{0.889279in}{1.630288in}}{\pgfqpoint{0.881379in}{1.627016in}}{\pgfqpoint{0.875555in}{1.621192in}}%
\pgfpathcurveto{\pgfqpoint{0.869731in}{1.615368in}}{\pgfqpoint{0.866459in}{1.607468in}}{\pgfqpoint{0.866459in}{1.599232in}}%
\pgfpathcurveto{\pgfqpoint{0.866459in}{1.590996in}}{\pgfqpoint{0.869731in}{1.583095in}}{\pgfqpoint{0.875555in}{1.577272in}}%
\pgfpathcurveto{\pgfqpoint{0.881379in}{1.571448in}}{\pgfqpoint{0.889279in}{1.568175in}}{\pgfqpoint{0.897515in}{1.568175in}}%
\pgfpathclose%
\pgfusepath{stroke,fill}%
\end{pgfscope}%
\begin{pgfscope}%
\pgfpathrectangle{\pgfqpoint{0.100000in}{0.212622in}}{\pgfqpoint{3.696000in}{3.696000in}}%
\pgfusepath{clip}%
\pgfsetbuttcap%
\pgfsetroundjoin%
\definecolor{currentfill}{rgb}{0.121569,0.466667,0.705882}%
\pgfsetfillcolor{currentfill}%
\pgfsetfillopacity{0.619629}%
\pgfsetlinewidth{1.003750pt}%
\definecolor{currentstroke}{rgb}{0.121569,0.466667,0.705882}%
\pgfsetstrokecolor{currentstroke}%
\pgfsetstrokeopacity{0.619629}%
\pgfsetdash{}{0pt}%
\pgfpathmoveto{\pgfqpoint{0.895533in}{1.564590in}}%
\pgfpathcurveto{\pgfqpoint{0.903770in}{1.564590in}}{\pgfqpoint{0.911670in}{1.567863in}}{\pgfqpoint{0.917494in}{1.573687in}}%
\pgfpathcurveto{\pgfqpoint{0.923317in}{1.579510in}}{\pgfqpoint{0.926590in}{1.587411in}}{\pgfqpoint{0.926590in}{1.595647in}}%
\pgfpathcurveto{\pgfqpoint{0.926590in}{1.603883in}}{\pgfqpoint{0.923317in}{1.611783in}}{\pgfqpoint{0.917494in}{1.617607in}}%
\pgfpathcurveto{\pgfqpoint{0.911670in}{1.623431in}}{\pgfqpoint{0.903770in}{1.626703in}}{\pgfqpoint{0.895533in}{1.626703in}}%
\pgfpathcurveto{\pgfqpoint{0.887297in}{1.626703in}}{\pgfqpoint{0.879397in}{1.623431in}}{\pgfqpoint{0.873573in}{1.617607in}}%
\pgfpathcurveto{\pgfqpoint{0.867749in}{1.611783in}}{\pgfqpoint{0.864477in}{1.603883in}}{\pgfqpoint{0.864477in}{1.595647in}}%
\pgfpathcurveto{\pgfqpoint{0.864477in}{1.587411in}}{\pgfqpoint{0.867749in}{1.579510in}}{\pgfqpoint{0.873573in}{1.573687in}}%
\pgfpathcurveto{\pgfqpoint{0.879397in}{1.567863in}}{\pgfqpoint{0.887297in}{1.564590in}}{\pgfqpoint{0.895533in}{1.564590in}}%
\pgfpathclose%
\pgfusepath{stroke,fill}%
\end{pgfscope}%
\begin{pgfscope}%
\pgfpathrectangle{\pgfqpoint{0.100000in}{0.212622in}}{\pgfqpoint{3.696000in}{3.696000in}}%
\pgfusepath{clip}%
\pgfsetbuttcap%
\pgfsetroundjoin%
\definecolor{currentfill}{rgb}{0.121569,0.466667,0.705882}%
\pgfsetfillcolor{currentfill}%
\pgfsetfillopacity{0.619897}%
\pgfsetlinewidth{1.003750pt}%
\definecolor{currentstroke}{rgb}{0.121569,0.466667,0.705882}%
\pgfsetstrokecolor{currentstroke}%
\pgfsetstrokeopacity{0.619897}%
\pgfsetdash{}{0pt}%
\pgfpathmoveto{\pgfqpoint{0.894404in}{1.561947in}}%
\pgfpathcurveto{\pgfqpoint{0.902641in}{1.561947in}}{\pgfqpoint{0.910541in}{1.565220in}}{\pgfqpoint{0.916365in}{1.571044in}}%
\pgfpathcurveto{\pgfqpoint{0.922189in}{1.576868in}}{\pgfqpoint{0.925461in}{1.584768in}}{\pgfqpoint{0.925461in}{1.593004in}}%
\pgfpathcurveto{\pgfqpoint{0.925461in}{1.601240in}}{\pgfqpoint{0.922189in}{1.609140in}}{\pgfqpoint{0.916365in}{1.614964in}}%
\pgfpathcurveto{\pgfqpoint{0.910541in}{1.620788in}}{\pgfqpoint{0.902641in}{1.624060in}}{\pgfqpoint{0.894404in}{1.624060in}}%
\pgfpathcurveto{\pgfqpoint{0.886168in}{1.624060in}}{\pgfqpoint{0.878268in}{1.620788in}}{\pgfqpoint{0.872444in}{1.614964in}}%
\pgfpathcurveto{\pgfqpoint{0.866620in}{1.609140in}}{\pgfqpoint{0.863348in}{1.601240in}}{\pgfqpoint{0.863348in}{1.593004in}}%
\pgfpathcurveto{\pgfqpoint{0.863348in}{1.584768in}}{\pgfqpoint{0.866620in}{1.576868in}}{\pgfqpoint{0.872444in}{1.571044in}}%
\pgfpathcurveto{\pgfqpoint{0.878268in}{1.565220in}}{\pgfqpoint{0.886168in}{1.561947in}}{\pgfqpoint{0.894404in}{1.561947in}}%
\pgfpathclose%
\pgfusepath{stroke,fill}%
\end{pgfscope}%
\begin{pgfscope}%
\pgfpathrectangle{\pgfqpoint{0.100000in}{0.212622in}}{\pgfqpoint{3.696000in}{3.696000in}}%
\pgfusepath{clip}%
\pgfsetbuttcap%
\pgfsetroundjoin%
\definecolor{currentfill}{rgb}{0.121569,0.466667,0.705882}%
\pgfsetfillcolor{currentfill}%
\pgfsetfillopacity{0.620310}%
\pgfsetlinewidth{1.003750pt}%
\definecolor{currentstroke}{rgb}{0.121569,0.466667,0.705882}%
\pgfsetstrokecolor{currentstroke}%
\pgfsetstrokeopacity{0.620310}%
\pgfsetdash{}{0pt}%
\pgfpathmoveto{\pgfqpoint{0.892046in}{1.557333in}}%
\pgfpathcurveto{\pgfqpoint{0.900282in}{1.557333in}}{\pgfqpoint{0.908182in}{1.560605in}}{\pgfqpoint{0.914006in}{1.566429in}}%
\pgfpathcurveto{\pgfqpoint{0.919830in}{1.572253in}}{\pgfqpoint{0.923102in}{1.580153in}}{\pgfqpoint{0.923102in}{1.588389in}}%
\pgfpathcurveto{\pgfqpoint{0.923102in}{1.596626in}}{\pgfqpoint{0.919830in}{1.604526in}}{\pgfqpoint{0.914006in}{1.610350in}}%
\pgfpathcurveto{\pgfqpoint{0.908182in}{1.616174in}}{\pgfqpoint{0.900282in}{1.619446in}}{\pgfqpoint{0.892046in}{1.619446in}}%
\pgfpathcurveto{\pgfqpoint{0.883809in}{1.619446in}}{\pgfqpoint{0.875909in}{1.616174in}}{\pgfqpoint{0.870085in}{1.610350in}}%
\pgfpathcurveto{\pgfqpoint{0.864261in}{1.604526in}}{\pgfqpoint{0.860989in}{1.596626in}}{\pgfqpoint{0.860989in}{1.588389in}}%
\pgfpathcurveto{\pgfqpoint{0.860989in}{1.580153in}}{\pgfqpoint{0.864261in}{1.572253in}}{\pgfqpoint{0.870085in}{1.566429in}}%
\pgfpathcurveto{\pgfqpoint{0.875909in}{1.560605in}}{\pgfqpoint{0.883809in}{1.557333in}}{\pgfqpoint{0.892046in}{1.557333in}}%
\pgfpathclose%
\pgfusepath{stroke,fill}%
\end{pgfscope}%
\begin{pgfscope}%
\pgfpathrectangle{\pgfqpoint{0.100000in}{0.212622in}}{\pgfqpoint{3.696000in}{3.696000in}}%
\pgfusepath{clip}%
\pgfsetbuttcap%
\pgfsetroundjoin%
\definecolor{currentfill}{rgb}{0.121569,0.466667,0.705882}%
\pgfsetfillcolor{currentfill}%
\pgfsetfillopacity{0.620587}%
\pgfsetlinewidth{1.003750pt}%
\definecolor{currentstroke}{rgb}{0.121569,0.466667,0.705882}%
\pgfsetstrokecolor{currentstroke}%
\pgfsetstrokeopacity{0.620587}%
\pgfsetdash{}{0pt}%
\pgfpathmoveto{\pgfqpoint{0.891074in}{1.554597in}}%
\pgfpathcurveto{\pgfqpoint{0.899310in}{1.554597in}}{\pgfqpoint{0.907210in}{1.557870in}}{\pgfqpoint{0.913034in}{1.563694in}}%
\pgfpathcurveto{\pgfqpoint{0.918858in}{1.569517in}}{\pgfqpoint{0.922131in}{1.577417in}}{\pgfqpoint{0.922131in}{1.585654in}}%
\pgfpathcurveto{\pgfqpoint{0.922131in}{1.593890in}}{\pgfqpoint{0.918858in}{1.601790in}}{\pgfqpoint{0.913034in}{1.607614in}}%
\pgfpathcurveto{\pgfqpoint{0.907210in}{1.613438in}}{\pgfqpoint{0.899310in}{1.616710in}}{\pgfqpoint{0.891074in}{1.616710in}}%
\pgfpathcurveto{\pgfqpoint{0.882838in}{1.616710in}}{\pgfqpoint{0.874938in}{1.613438in}}{\pgfqpoint{0.869114in}{1.607614in}}%
\pgfpathcurveto{\pgfqpoint{0.863290in}{1.601790in}}{\pgfqpoint{0.860018in}{1.593890in}}{\pgfqpoint{0.860018in}{1.585654in}}%
\pgfpathcurveto{\pgfqpoint{0.860018in}{1.577417in}}{\pgfqpoint{0.863290in}{1.569517in}}{\pgfqpoint{0.869114in}{1.563694in}}%
\pgfpathcurveto{\pgfqpoint{0.874938in}{1.557870in}}{\pgfqpoint{0.882838in}{1.554597in}}{\pgfqpoint{0.891074in}{1.554597in}}%
\pgfpathclose%
\pgfusepath{stroke,fill}%
\end{pgfscope}%
\begin{pgfscope}%
\pgfpathrectangle{\pgfqpoint{0.100000in}{0.212622in}}{\pgfqpoint{3.696000in}{3.696000in}}%
\pgfusepath{clip}%
\pgfsetbuttcap%
\pgfsetroundjoin%
\definecolor{currentfill}{rgb}{0.121569,0.466667,0.705882}%
\pgfsetfillcolor{currentfill}%
\pgfsetfillopacity{0.620732}%
\pgfsetlinewidth{1.003750pt}%
\definecolor{currentstroke}{rgb}{0.121569,0.466667,0.705882}%
\pgfsetstrokecolor{currentstroke}%
\pgfsetstrokeopacity{0.620732}%
\pgfsetdash{}{0pt}%
\pgfpathmoveto{\pgfqpoint{0.715394in}{1.205895in}}%
\pgfpathcurveto{\pgfqpoint{0.723630in}{1.205895in}}{\pgfqpoint{0.731530in}{1.209168in}}{\pgfqpoint{0.737354in}{1.214992in}}%
\pgfpathcurveto{\pgfqpoint{0.743178in}{1.220816in}}{\pgfqpoint{0.746451in}{1.228716in}}{\pgfqpoint{0.746451in}{1.236952in}}%
\pgfpathcurveto{\pgfqpoint{0.746451in}{1.245188in}}{\pgfqpoint{0.743178in}{1.253088in}}{\pgfqpoint{0.737354in}{1.258912in}}%
\pgfpathcurveto{\pgfqpoint{0.731530in}{1.264736in}}{\pgfqpoint{0.723630in}{1.268008in}}{\pgfqpoint{0.715394in}{1.268008in}}%
\pgfpathcurveto{\pgfqpoint{0.707158in}{1.268008in}}{\pgfqpoint{0.699258in}{1.264736in}}{\pgfqpoint{0.693434in}{1.258912in}}%
\pgfpathcurveto{\pgfqpoint{0.687610in}{1.253088in}}{\pgfqpoint{0.684338in}{1.245188in}}{\pgfqpoint{0.684338in}{1.236952in}}%
\pgfpathcurveto{\pgfqpoint{0.684338in}{1.228716in}}{\pgfqpoint{0.687610in}{1.220816in}}{\pgfqpoint{0.693434in}{1.214992in}}%
\pgfpathcurveto{\pgfqpoint{0.699258in}{1.209168in}}{\pgfqpoint{0.707158in}{1.205895in}}{\pgfqpoint{0.715394in}{1.205895in}}%
\pgfpathclose%
\pgfusepath{stroke,fill}%
\end{pgfscope}%
\begin{pgfscope}%
\pgfpathrectangle{\pgfqpoint{0.100000in}{0.212622in}}{\pgfqpoint{3.696000in}{3.696000in}}%
\pgfusepath{clip}%
\pgfsetbuttcap%
\pgfsetroundjoin%
\definecolor{currentfill}{rgb}{0.121569,0.466667,0.705882}%
\pgfsetfillcolor{currentfill}%
\pgfsetfillopacity{0.620927}%
\pgfsetlinewidth{1.003750pt}%
\definecolor{currentstroke}{rgb}{0.121569,0.466667,0.705882}%
\pgfsetstrokecolor{currentstroke}%
\pgfsetstrokeopacity{0.620927}%
\pgfsetdash{}{0pt}%
\pgfpathmoveto{\pgfqpoint{2.117143in}{2.028511in}}%
\pgfpathcurveto{\pgfqpoint{2.125379in}{2.028511in}}{\pgfqpoint{2.133279in}{2.031783in}}{\pgfqpoint{2.139103in}{2.037607in}}%
\pgfpathcurveto{\pgfqpoint{2.144927in}{2.043431in}}{\pgfqpoint{2.148199in}{2.051331in}}{\pgfqpoint{2.148199in}{2.059567in}}%
\pgfpathcurveto{\pgfqpoint{2.148199in}{2.067803in}}{\pgfqpoint{2.144927in}{2.075703in}}{\pgfqpoint{2.139103in}{2.081527in}}%
\pgfpathcurveto{\pgfqpoint{2.133279in}{2.087351in}}{\pgfqpoint{2.125379in}{2.090624in}}{\pgfqpoint{2.117143in}{2.090624in}}%
\pgfpathcurveto{\pgfqpoint{2.108906in}{2.090624in}}{\pgfqpoint{2.101006in}{2.087351in}}{\pgfqpoint{2.095182in}{2.081527in}}%
\pgfpathcurveto{\pgfqpoint{2.089358in}{2.075703in}}{\pgfqpoint{2.086086in}{2.067803in}}{\pgfqpoint{2.086086in}{2.059567in}}%
\pgfpathcurveto{\pgfqpoint{2.086086in}{2.051331in}}{\pgfqpoint{2.089358in}{2.043431in}}{\pgfqpoint{2.095182in}{2.037607in}}%
\pgfpathcurveto{\pgfqpoint{2.101006in}{2.031783in}}{\pgfqpoint{2.108906in}{2.028511in}}{\pgfqpoint{2.117143in}{2.028511in}}%
\pgfpathclose%
\pgfusepath{stroke,fill}%
\end{pgfscope}%
\begin{pgfscope}%
\pgfpathrectangle{\pgfqpoint{0.100000in}{0.212622in}}{\pgfqpoint{3.696000in}{3.696000in}}%
\pgfusepath{clip}%
\pgfsetbuttcap%
\pgfsetroundjoin%
\definecolor{currentfill}{rgb}{0.121569,0.466667,0.705882}%
\pgfsetfillcolor{currentfill}%
\pgfsetfillopacity{0.621053}%
\pgfsetlinewidth{1.003750pt}%
\definecolor{currentstroke}{rgb}{0.121569,0.466667,0.705882}%
\pgfsetstrokecolor{currentstroke}%
\pgfsetstrokeopacity{0.621053}%
\pgfsetdash{}{0pt}%
\pgfpathmoveto{\pgfqpoint{0.861282in}{1.491095in}}%
\pgfpathcurveto{\pgfqpoint{0.869518in}{1.491095in}}{\pgfqpoint{0.877418in}{1.494367in}}{\pgfqpoint{0.883242in}{1.500191in}}%
\pgfpathcurveto{\pgfqpoint{0.889066in}{1.506015in}}{\pgfqpoint{0.892338in}{1.513915in}}{\pgfqpoint{0.892338in}{1.522152in}}%
\pgfpathcurveto{\pgfqpoint{0.892338in}{1.530388in}}{\pgfqpoint{0.889066in}{1.538288in}}{\pgfqpoint{0.883242in}{1.544112in}}%
\pgfpathcurveto{\pgfqpoint{0.877418in}{1.549936in}}{\pgfqpoint{0.869518in}{1.553208in}}{\pgfqpoint{0.861282in}{1.553208in}}%
\pgfpathcurveto{\pgfqpoint{0.853045in}{1.553208in}}{\pgfqpoint{0.845145in}{1.549936in}}{\pgfqpoint{0.839321in}{1.544112in}}%
\pgfpathcurveto{\pgfqpoint{0.833497in}{1.538288in}}{\pgfqpoint{0.830225in}{1.530388in}}{\pgfqpoint{0.830225in}{1.522152in}}%
\pgfpathcurveto{\pgfqpoint{0.830225in}{1.513915in}}{\pgfqpoint{0.833497in}{1.506015in}}{\pgfqpoint{0.839321in}{1.500191in}}%
\pgfpathcurveto{\pgfqpoint{0.845145in}{1.494367in}}{\pgfqpoint{0.853045in}{1.491095in}}{\pgfqpoint{0.861282in}{1.491095in}}%
\pgfpathclose%
\pgfusepath{stroke,fill}%
\end{pgfscope}%
\begin{pgfscope}%
\pgfpathrectangle{\pgfqpoint{0.100000in}{0.212622in}}{\pgfqpoint{3.696000in}{3.696000in}}%
\pgfusepath{clip}%
\pgfsetbuttcap%
\pgfsetroundjoin%
\definecolor{currentfill}{rgb}{0.121569,0.466667,0.705882}%
\pgfsetfillcolor{currentfill}%
\pgfsetfillopacity{0.621057}%
\pgfsetlinewidth{1.003750pt}%
\definecolor{currentstroke}{rgb}{0.121569,0.466667,0.705882}%
\pgfsetstrokecolor{currentstroke}%
\pgfsetstrokeopacity{0.621057}%
\pgfsetdash{}{0pt}%
\pgfpathmoveto{\pgfqpoint{0.888728in}{1.550254in}}%
\pgfpathcurveto{\pgfqpoint{0.896964in}{1.550254in}}{\pgfqpoint{0.904864in}{1.553526in}}{\pgfqpoint{0.910688in}{1.559350in}}%
\pgfpathcurveto{\pgfqpoint{0.916512in}{1.565174in}}{\pgfqpoint{0.919784in}{1.573074in}}{\pgfqpoint{0.919784in}{1.581310in}}%
\pgfpathcurveto{\pgfqpoint{0.919784in}{1.589547in}}{\pgfqpoint{0.916512in}{1.597447in}}{\pgfqpoint{0.910688in}{1.603271in}}%
\pgfpathcurveto{\pgfqpoint{0.904864in}{1.609094in}}{\pgfqpoint{0.896964in}{1.612367in}}{\pgfqpoint{0.888728in}{1.612367in}}%
\pgfpathcurveto{\pgfqpoint{0.880492in}{1.612367in}}{\pgfqpoint{0.872592in}{1.609094in}}{\pgfqpoint{0.866768in}{1.603271in}}%
\pgfpathcurveto{\pgfqpoint{0.860944in}{1.597447in}}{\pgfqpoint{0.857671in}{1.589547in}}{\pgfqpoint{0.857671in}{1.581310in}}%
\pgfpathcurveto{\pgfqpoint{0.857671in}{1.573074in}}{\pgfqpoint{0.860944in}{1.565174in}}{\pgfqpoint{0.866768in}{1.559350in}}%
\pgfpathcurveto{\pgfqpoint{0.872592in}{1.553526in}}{\pgfqpoint{0.880492in}{1.550254in}}{\pgfqpoint{0.888728in}{1.550254in}}%
\pgfpathclose%
\pgfusepath{stroke,fill}%
\end{pgfscope}%
\begin{pgfscope}%
\pgfpathrectangle{\pgfqpoint{0.100000in}{0.212622in}}{\pgfqpoint{3.696000in}{3.696000in}}%
\pgfusepath{clip}%
\pgfsetbuttcap%
\pgfsetroundjoin%
\definecolor{currentfill}{rgb}{0.121569,0.466667,0.705882}%
\pgfsetfillcolor{currentfill}%
\pgfsetfillopacity{0.621984}%
\pgfsetlinewidth{1.003750pt}%
\definecolor{currentstroke}{rgb}{0.121569,0.466667,0.705882}%
\pgfsetstrokecolor{currentstroke}%
\pgfsetstrokeopacity{0.621984}%
\pgfsetdash{}{0pt}%
\pgfpathmoveto{\pgfqpoint{0.884578in}{1.542406in}}%
\pgfpathcurveto{\pgfqpoint{0.892815in}{1.542406in}}{\pgfqpoint{0.900715in}{1.545678in}}{\pgfqpoint{0.906539in}{1.551502in}}%
\pgfpathcurveto{\pgfqpoint{0.912363in}{1.557326in}}{\pgfqpoint{0.915635in}{1.565226in}}{\pgfqpoint{0.915635in}{1.573462in}}%
\pgfpathcurveto{\pgfqpoint{0.915635in}{1.581698in}}{\pgfqpoint{0.912363in}{1.589598in}}{\pgfqpoint{0.906539in}{1.595422in}}%
\pgfpathcurveto{\pgfqpoint{0.900715in}{1.601246in}}{\pgfqpoint{0.892815in}{1.604519in}}{\pgfqpoint{0.884578in}{1.604519in}}%
\pgfpathcurveto{\pgfqpoint{0.876342in}{1.604519in}}{\pgfqpoint{0.868442in}{1.601246in}}{\pgfqpoint{0.862618in}{1.595422in}}%
\pgfpathcurveto{\pgfqpoint{0.856794in}{1.589598in}}{\pgfqpoint{0.853522in}{1.581698in}}{\pgfqpoint{0.853522in}{1.573462in}}%
\pgfpathcurveto{\pgfqpoint{0.853522in}{1.565226in}}{\pgfqpoint{0.856794in}{1.557326in}}{\pgfqpoint{0.862618in}{1.551502in}}%
\pgfpathcurveto{\pgfqpoint{0.868442in}{1.545678in}}{\pgfqpoint{0.876342in}{1.542406in}}{\pgfqpoint{0.884578in}{1.542406in}}%
\pgfpathclose%
\pgfusepath{stroke,fill}%
\end{pgfscope}%
\begin{pgfscope}%
\pgfpathrectangle{\pgfqpoint{0.100000in}{0.212622in}}{\pgfqpoint{3.696000in}{3.696000in}}%
\pgfusepath{clip}%
\pgfsetbuttcap%
\pgfsetroundjoin%
\definecolor{currentfill}{rgb}{0.121569,0.466667,0.705882}%
\pgfsetfillcolor{currentfill}%
\pgfsetfillopacity{0.622759}%
\pgfsetlinewidth{1.003750pt}%
\definecolor{currentstroke}{rgb}{0.121569,0.466667,0.705882}%
\pgfsetstrokecolor{currentstroke}%
\pgfsetstrokeopacity{0.622759}%
\pgfsetdash{}{0pt}%
\pgfpathmoveto{\pgfqpoint{0.860483in}{1.491464in}}%
\pgfpathcurveto{\pgfqpoint{0.868720in}{1.491464in}}{\pgfqpoint{0.876620in}{1.494736in}}{\pgfqpoint{0.882444in}{1.500560in}}%
\pgfpathcurveto{\pgfqpoint{0.888268in}{1.506384in}}{\pgfqpoint{0.891540in}{1.514284in}}{\pgfqpoint{0.891540in}{1.522520in}}%
\pgfpathcurveto{\pgfqpoint{0.891540in}{1.530757in}}{\pgfqpoint{0.888268in}{1.538657in}}{\pgfqpoint{0.882444in}{1.544481in}}%
\pgfpathcurveto{\pgfqpoint{0.876620in}{1.550305in}}{\pgfqpoint{0.868720in}{1.553577in}}{\pgfqpoint{0.860483in}{1.553577in}}%
\pgfpathcurveto{\pgfqpoint{0.852247in}{1.553577in}}{\pgfqpoint{0.844347in}{1.550305in}}{\pgfqpoint{0.838523in}{1.544481in}}%
\pgfpathcurveto{\pgfqpoint{0.832699in}{1.538657in}}{\pgfqpoint{0.829427in}{1.530757in}}{\pgfqpoint{0.829427in}{1.522520in}}%
\pgfpathcurveto{\pgfqpoint{0.829427in}{1.514284in}}{\pgfqpoint{0.832699in}{1.506384in}}{\pgfqpoint{0.838523in}{1.500560in}}%
\pgfpathcurveto{\pgfqpoint{0.844347in}{1.494736in}}{\pgfqpoint{0.852247in}{1.491464in}}{\pgfqpoint{0.860483in}{1.491464in}}%
\pgfpathclose%
\pgfusepath{stroke,fill}%
\end{pgfscope}%
\begin{pgfscope}%
\pgfpathrectangle{\pgfqpoint{0.100000in}{0.212622in}}{\pgfqpoint{3.696000in}{3.696000in}}%
\pgfusepath{clip}%
\pgfsetbuttcap%
\pgfsetroundjoin%
\definecolor{currentfill}{rgb}{0.121569,0.466667,0.705882}%
\pgfsetfillcolor{currentfill}%
\pgfsetfillopacity{0.622832}%
\pgfsetlinewidth{1.003750pt}%
\definecolor{currentstroke}{rgb}{0.121569,0.466667,0.705882}%
\pgfsetstrokecolor{currentstroke}%
\pgfsetstrokeopacity{0.622832}%
\pgfsetdash{}{0pt}%
\pgfpathmoveto{\pgfqpoint{0.881239in}{1.535428in}}%
\pgfpathcurveto{\pgfqpoint{0.889475in}{1.535428in}}{\pgfqpoint{0.897375in}{1.538700in}}{\pgfqpoint{0.903199in}{1.544524in}}%
\pgfpathcurveto{\pgfqpoint{0.909023in}{1.550348in}}{\pgfqpoint{0.912295in}{1.558248in}}{\pgfqpoint{0.912295in}{1.566484in}}%
\pgfpathcurveto{\pgfqpoint{0.912295in}{1.574721in}}{\pgfqpoint{0.909023in}{1.582621in}}{\pgfqpoint{0.903199in}{1.588445in}}%
\pgfpathcurveto{\pgfqpoint{0.897375in}{1.594269in}}{\pgfqpoint{0.889475in}{1.597541in}}{\pgfqpoint{0.881239in}{1.597541in}}%
\pgfpathcurveto{\pgfqpoint{0.873002in}{1.597541in}}{\pgfqpoint{0.865102in}{1.594269in}}{\pgfqpoint{0.859278in}{1.588445in}}%
\pgfpathcurveto{\pgfqpoint{0.853454in}{1.582621in}}{\pgfqpoint{0.850182in}{1.574721in}}{\pgfqpoint{0.850182in}{1.566484in}}%
\pgfpathcurveto{\pgfqpoint{0.850182in}{1.558248in}}{\pgfqpoint{0.853454in}{1.550348in}}{\pgfqpoint{0.859278in}{1.544524in}}%
\pgfpathcurveto{\pgfqpoint{0.865102in}{1.538700in}}{\pgfqpoint{0.873002in}{1.535428in}}{\pgfqpoint{0.881239in}{1.535428in}}%
\pgfpathclose%
\pgfusepath{stroke,fill}%
\end{pgfscope}%
\begin{pgfscope}%
\pgfpathrectangle{\pgfqpoint{0.100000in}{0.212622in}}{\pgfqpoint{3.696000in}{3.696000in}}%
\pgfusepath{clip}%
\pgfsetbuttcap%
\pgfsetroundjoin%
\definecolor{currentfill}{rgb}{0.121569,0.466667,0.705882}%
\pgfsetfillcolor{currentfill}%
\pgfsetfillopacity{0.623294}%
\pgfsetlinewidth{1.003750pt}%
\definecolor{currentstroke}{rgb}{0.121569,0.466667,0.705882}%
\pgfsetstrokecolor{currentstroke}%
\pgfsetstrokeopacity{0.623294}%
\pgfsetdash{}{0pt}%
\pgfpathmoveto{\pgfqpoint{2.118541in}{2.021335in}}%
\pgfpathcurveto{\pgfqpoint{2.126778in}{2.021335in}}{\pgfqpoint{2.134678in}{2.024607in}}{\pgfqpoint{2.140502in}{2.030431in}}%
\pgfpathcurveto{\pgfqpoint{2.146326in}{2.036255in}}{\pgfqpoint{2.149598in}{2.044155in}}{\pgfqpoint{2.149598in}{2.052391in}}%
\pgfpathcurveto{\pgfqpoint{2.149598in}{2.060628in}}{\pgfqpoint{2.146326in}{2.068528in}}{\pgfqpoint{2.140502in}{2.074352in}}%
\pgfpathcurveto{\pgfqpoint{2.134678in}{2.080175in}}{\pgfqpoint{2.126778in}{2.083448in}}{\pgfqpoint{2.118541in}{2.083448in}}%
\pgfpathcurveto{\pgfqpoint{2.110305in}{2.083448in}}{\pgfqpoint{2.102405in}{2.080175in}}{\pgfqpoint{2.096581in}{2.074352in}}%
\pgfpathcurveto{\pgfqpoint{2.090757in}{2.068528in}}{\pgfqpoint{2.087485in}{2.060628in}}{\pgfqpoint{2.087485in}{2.052391in}}%
\pgfpathcurveto{\pgfqpoint{2.087485in}{2.044155in}}{\pgfqpoint{2.090757in}{2.036255in}}{\pgfqpoint{2.096581in}{2.030431in}}%
\pgfpathcurveto{\pgfqpoint{2.102405in}{2.024607in}}{\pgfqpoint{2.110305in}{2.021335in}}{\pgfqpoint{2.118541in}{2.021335in}}%
\pgfpathclose%
\pgfusepath{stroke,fill}%
\end{pgfscope}%
\begin{pgfscope}%
\pgfpathrectangle{\pgfqpoint{0.100000in}{0.212622in}}{\pgfqpoint{3.696000in}{3.696000in}}%
\pgfusepath{clip}%
\pgfsetbuttcap%
\pgfsetroundjoin%
\definecolor{currentfill}{rgb}{0.121569,0.466667,0.705882}%
\pgfsetfillcolor{currentfill}%
\pgfsetfillopacity{0.623696}%
\pgfsetlinewidth{1.003750pt}%
\definecolor{currentstroke}{rgb}{0.121569,0.466667,0.705882}%
\pgfsetstrokecolor{currentstroke}%
\pgfsetstrokeopacity{0.623696}%
\pgfsetdash{}{0pt}%
\pgfpathmoveto{\pgfqpoint{0.860116in}{1.491665in}}%
\pgfpathcurveto{\pgfqpoint{0.868352in}{1.491665in}}{\pgfqpoint{0.876252in}{1.494938in}}{\pgfqpoint{0.882076in}{1.500762in}}%
\pgfpathcurveto{\pgfqpoint{0.887900in}{1.506586in}}{\pgfqpoint{0.891172in}{1.514486in}}{\pgfqpoint{0.891172in}{1.522722in}}%
\pgfpathcurveto{\pgfqpoint{0.891172in}{1.530958in}}{\pgfqpoint{0.887900in}{1.538858in}}{\pgfqpoint{0.882076in}{1.544682in}}%
\pgfpathcurveto{\pgfqpoint{0.876252in}{1.550506in}}{\pgfqpoint{0.868352in}{1.553778in}}{\pgfqpoint{0.860116in}{1.553778in}}%
\pgfpathcurveto{\pgfqpoint{0.851879in}{1.553778in}}{\pgfqpoint{0.843979in}{1.550506in}}{\pgfqpoint{0.838155in}{1.544682in}}%
\pgfpathcurveto{\pgfqpoint{0.832331in}{1.538858in}}{\pgfqpoint{0.829059in}{1.530958in}}{\pgfqpoint{0.829059in}{1.522722in}}%
\pgfpathcurveto{\pgfqpoint{0.829059in}{1.514486in}}{\pgfqpoint{0.832331in}{1.506586in}}{\pgfqpoint{0.838155in}{1.500762in}}%
\pgfpathcurveto{\pgfqpoint{0.843979in}{1.494938in}}{\pgfqpoint{0.851879in}{1.491665in}}{\pgfqpoint{0.860116in}{1.491665in}}%
\pgfpathclose%
\pgfusepath{stroke,fill}%
\end{pgfscope}%
\begin{pgfscope}%
\pgfpathrectangle{\pgfqpoint{0.100000in}{0.212622in}}{\pgfqpoint{3.696000in}{3.696000in}}%
\pgfusepath{clip}%
\pgfsetbuttcap%
\pgfsetroundjoin%
\definecolor{currentfill}{rgb}{0.121569,0.466667,0.705882}%
\pgfsetfillcolor{currentfill}%
\pgfsetfillopacity{0.624198}%
\pgfsetlinewidth{1.003750pt}%
\definecolor{currentstroke}{rgb}{0.121569,0.466667,0.705882}%
\pgfsetstrokecolor{currentstroke}%
\pgfsetstrokeopacity{0.624198}%
\pgfsetdash{}{0pt}%
\pgfpathmoveto{\pgfqpoint{0.874576in}{1.523018in}}%
\pgfpathcurveto{\pgfqpoint{0.882812in}{1.523018in}}{\pgfqpoint{0.890713in}{1.526291in}}{\pgfqpoint{0.896536in}{1.532115in}}%
\pgfpathcurveto{\pgfqpoint{0.902360in}{1.537939in}}{\pgfqpoint{0.905633in}{1.545839in}}{\pgfqpoint{0.905633in}{1.554075in}}%
\pgfpathcurveto{\pgfqpoint{0.905633in}{1.562311in}}{\pgfqpoint{0.902360in}{1.570211in}}{\pgfqpoint{0.896536in}{1.576035in}}%
\pgfpathcurveto{\pgfqpoint{0.890713in}{1.581859in}}{\pgfqpoint{0.882812in}{1.585131in}}{\pgfqpoint{0.874576in}{1.585131in}}%
\pgfpathcurveto{\pgfqpoint{0.866340in}{1.585131in}}{\pgfqpoint{0.858440in}{1.581859in}}{\pgfqpoint{0.852616in}{1.576035in}}%
\pgfpathcurveto{\pgfqpoint{0.846792in}{1.570211in}}{\pgfqpoint{0.843520in}{1.562311in}}{\pgfqpoint{0.843520in}{1.554075in}}%
\pgfpathcurveto{\pgfqpoint{0.843520in}{1.545839in}}{\pgfqpoint{0.846792in}{1.537939in}}{\pgfqpoint{0.852616in}{1.532115in}}%
\pgfpathcurveto{\pgfqpoint{0.858440in}{1.526291in}}{\pgfqpoint{0.866340in}{1.523018in}}{\pgfqpoint{0.874576in}{1.523018in}}%
\pgfpathclose%
\pgfusepath{stroke,fill}%
\end{pgfscope}%
\begin{pgfscope}%
\pgfpathrectangle{\pgfqpoint{0.100000in}{0.212622in}}{\pgfqpoint{3.696000in}{3.696000in}}%
\pgfusepath{clip}%
\pgfsetbuttcap%
\pgfsetroundjoin%
\definecolor{currentfill}{rgb}{0.121569,0.466667,0.705882}%
\pgfsetfillcolor{currentfill}%
\pgfsetfillopacity{0.624209}%
\pgfsetlinewidth{1.003750pt}%
\definecolor{currentstroke}{rgb}{0.121569,0.466667,0.705882}%
\pgfsetstrokecolor{currentstroke}%
\pgfsetstrokeopacity{0.624209}%
\pgfsetdash{}{0pt}%
\pgfpathmoveto{\pgfqpoint{0.859945in}{1.491771in}}%
\pgfpathcurveto{\pgfqpoint{0.868181in}{1.491771in}}{\pgfqpoint{0.876082in}{1.495044in}}{\pgfqpoint{0.881905in}{1.500868in}}%
\pgfpathcurveto{\pgfqpoint{0.887729in}{1.506692in}}{\pgfqpoint{0.891002in}{1.514592in}}{\pgfqpoint{0.891002in}{1.522828in}}%
\pgfpathcurveto{\pgfqpoint{0.891002in}{1.531064in}}{\pgfqpoint{0.887729in}{1.538964in}}{\pgfqpoint{0.881905in}{1.544788in}}%
\pgfpathcurveto{\pgfqpoint{0.876082in}{1.550612in}}{\pgfqpoint{0.868181in}{1.553884in}}{\pgfqpoint{0.859945in}{1.553884in}}%
\pgfpathcurveto{\pgfqpoint{0.851709in}{1.553884in}}{\pgfqpoint{0.843809in}{1.550612in}}{\pgfqpoint{0.837985in}{1.544788in}}%
\pgfpathcurveto{\pgfqpoint{0.832161in}{1.538964in}}{\pgfqpoint{0.828889in}{1.531064in}}{\pgfqpoint{0.828889in}{1.522828in}}%
\pgfpathcurveto{\pgfqpoint{0.828889in}{1.514592in}}{\pgfqpoint{0.832161in}{1.506692in}}{\pgfqpoint{0.837985in}{1.500868in}}%
\pgfpathcurveto{\pgfqpoint{0.843809in}{1.495044in}}{\pgfqpoint{0.851709in}{1.491771in}}{\pgfqpoint{0.859945in}{1.491771in}}%
\pgfpathclose%
\pgfusepath{stroke,fill}%
\end{pgfscope}%
\begin{pgfscope}%
\pgfpathrectangle{\pgfqpoint{0.100000in}{0.212622in}}{\pgfqpoint{3.696000in}{3.696000in}}%
\pgfusepath{clip}%
\pgfsetbuttcap%
\pgfsetroundjoin%
\definecolor{currentfill}{rgb}{0.121569,0.466667,0.705882}%
\pgfsetfillcolor{currentfill}%
\pgfsetfillopacity{0.624556}%
\pgfsetlinewidth{1.003750pt}%
\definecolor{currentstroke}{rgb}{0.121569,0.466667,0.705882}%
\pgfsetstrokecolor{currentstroke}%
\pgfsetstrokeopacity{0.624556}%
\pgfsetdash{}{0pt}%
\pgfpathmoveto{\pgfqpoint{2.119465in}{2.017383in}}%
\pgfpathcurveto{\pgfqpoint{2.127701in}{2.017383in}}{\pgfqpoint{2.135601in}{2.020655in}}{\pgfqpoint{2.141425in}{2.026479in}}%
\pgfpathcurveto{\pgfqpoint{2.147249in}{2.032303in}}{\pgfqpoint{2.150521in}{2.040203in}}{\pgfqpoint{2.150521in}{2.048439in}}%
\pgfpathcurveto{\pgfqpoint{2.150521in}{2.056675in}}{\pgfqpoint{2.147249in}{2.064575in}}{\pgfqpoint{2.141425in}{2.070399in}}%
\pgfpathcurveto{\pgfqpoint{2.135601in}{2.076223in}}{\pgfqpoint{2.127701in}{2.079496in}}{\pgfqpoint{2.119465in}{2.079496in}}%
\pgfpathcurveto{\pgfqpoint{2.111229in}{2.079496in}}{\pgfqpoint{2.103329in}{2.076223in}}{\pgfqpoint{2.097505in}{2.070399in}}%
\pgfpathcurveto{\pgfqpoint{2.091681in}{2.064575in}}{\pgfqpoint{2.088408in}{2.056675in}}{\pgfqpoint{2.088408in}{2.048439in}}%
\pgfpathcurveto{\pgfqpoint{2.088408in}{2.040203in}}{\pgfqpoint{2.091681in}{2.032303in}}{\pgfqpoint{2.097505in}{2.026479in}}%
\pgfpathcurveto{\pgfqpoint{2.103329in}{2.020655in}}{\pgfqpoint{2.111229in}{2.017383in}}{\pgfqpoint{2.119465in}{2.017383in}}%
\pgfpathclose%
\pgfusepath{stroke,fill}%
\end{pgfscope}%
\begin{pgfscope}%
\pgfpathrectangle{\pgfqpoint{0.100000in}{0.212622in}}{\pgfqpoint{3.696000in}{3.696000in}}%
\pgfusepath{clip}%
\pgfsetbuttcap%
\pgfsetroundjoin%
\definecolor{currentfill}{rgb}{0.121569,0.466667,0.705882}%
\pgfsetfillcolor{currentfill}%
\pgfsetfillopacity{0.624911}%
\pgfsetlinewidth{1.003750pt}%
\definecolor{currentstroke}{rgb}{0.121569,0.466667,0.705882}%
\pgfsetstrokecolor{currentstroke}%
\pgfsetstrokeopacity{0.624911}%
\pgfsetdash{}{0pt}%
\pgfpathmoveto{\pgfqpoint{0.859771in}{1.491931in}}%
\pgfpathcurveto{\pgfqpoint{0.868007in}{1.491931in}}{\pgfqpoint{0.875907in}{1.495204in}}{\pgfqpoint{0.881731in}{1.501028in}}%
\pgfpathcurveto{\pgfqpoint{0.887555in}{1.506851in}}{\pgfqpoint{0.890828in}{1.514752in}}{\pgfqpoint{0.890828in}{1.522988in}}%
\pgfpathcurveto{\pgfqpoint{0.890828in}{1.531224in}}{\pgfqpoint{0.887555in}{1.539124in}}{\pgfqpoint{0.881731in}{1.544948in}}%
\pgfpathcurveto{\pgfqpoint{0.875907in}{1.550772in}}{\pgfqpoint{0.868007in}{1.554044in}}{\pgfqpoint{0.859771in}{1.554044in}}%
\pgfpathcurveto{\pgfqpoint{0.851535in}{1.554044in}}{\pgfqpoint{0.843635in}{1.550772in}}{\pgfqpoint{0.837811in}{1.544948in}}%
\pgfpathcurveto{\pgfqpoint{0.831987in}{1.539124in}}{\pgfqpoint{0.828715in}{1.531224in}}{\pgfqpoint{0.828715in}{1.522988in}}%
\pgfpathcurveto{\pgfqpoint{0.828715in}{1.514752in}}{\pgfqpoint{0.831987in}{1.506851in}}{\pgfqpoint{0.837811in}{1.501028in}}%
\pgfpathcurveto{\pgfqpoint{0.843635in}{1.495204in}}{\pgfqpoint{0.851535in}{1.491931in}}{\pgfqpoint{0.859771in}{1.491931in}}%
\pgfpathclose%
\pgfusepath{stroke,fill}%
\end{pgfscope}%
\begin{pgfscope}%
\pgfpathrectangle{\pgfqpoint{0.100000in}{0.212622in}}{\pgfqpoint{3.696000in}{3.696000in}}%
\pgfusepath{clip}%
\pgfsetbuttcap%
\pgfsetroundjoin%
\definecolor{currentfill}{rgb}{0.121569,0.466667,0.705882}%
\pgfsetfillcolor{currentfill}%
\pgfsetfillopacity{0.625256}%
\pgfsetlinewidth{1.003750pt}%
\definecolor{currentstroke}{rgb}{0.121569,0.466667,0.705882}%
\pgfsetstrokecolor{currentstroke}%
\pgfsetstrokeopacity{0.625256}%
\pgfsetdash{}{0pt}%
\pgfpathmoveto{\pgfqpoint{0.737458in}{1.204942in}}%
\pgfpathcurveto{\pgfqpoint{0.745694in}{1.204942in}}{\pgfqpoint{0.753594in}{1.208214in}}{\pgfqpoint{0.759418in}{1.214038in}}%
\pgfpathcurveto{\pgfqpoint{0.765242in}{1.219862in}}{\pgfqpoint{0.768515in}{1.227762in}}{\pgfqpoint{0.768515in}{1.235999in}}%
\pgfpathcurveto{\pgfqpoint{0.768515in}{1.244235in}}{\pgfqpoint{0.765242in}{1.252135in}}{\pgfqpoint{0.759418in}{1.257959in}}%
\pgfpathcurveto{\pgfqpoint{0.753594in}{1.263783in}}{\pgfqpoint{0.745694in}{1.267055in}}{\pgfqpoint{0.737458in}{1.267055in}}%
\pgfpathcurveto{\pgfqpoint{0.729222in}{1.267055in}}{\pgfqpoint{0.721322in}{1.263783in}}{\pgfqpoint{0.715498in}{1.257959in}}%
\pgfpathcurveto{\pgfqpoint{0.709674in}{1.252135in}}{\pgfqpoint{0.706402in}{1.244235in}}{\pgfqpoint{0.706402in}{1.235999in}}%
\pgfpathcurveto{\pgfqpoint{0.706402in}{1.227762in}}{\pgfqpoint{0.709674in}{1.219862in}}{\pgfqpoint{0.715498in}{1.214038in}}%
\pgfpathcurveto{\pgfqpoint{0.721322in}{1.208214in}}{\pgfqpoint{0.729222in}{1.204942in}}{\pgfqpoint{0.737458in}{1.204942in}}%
\pgfpathclose%
\pgfusepath{stroke,fill}%
\end{pgfscope}%
\begin{pgfscope}%
\pgfpathrectangle{\pgfqpoint{0.100000in}{0.212622in}}{\pgfqpoint{3.696000in}{3.696000in}}%
\pgfusepath{clip}%
\pgfsetbuttcap%
\pgfsetroundjoin%
\definecolor{currentfill}{rgb}{0.121569,0.466667,0.705882}%
\pgfsetfillcolor{currentfill}%
\pgfsetfillopacity{0.625290}%
\pgfsetlinewidth{1.003750pt}%
\definecolor{currentstroke}{rgb}{0.121569,0.466667,0.705882}%
\pgfsetstrokecolor{currentstroke}%
\pgfsetstrokeopacity{0.625290}%
\pgfsetdash{}{0pt}%
\pgfpathmoveto{\pgfqpoint{0.859701in}{1.492000in}}%
\pgfpathcurveto{\pgfqpoint{0.867937in}{1.492000in}}{\pgfqpoint{0.875837in}{1.495273in}}{\pgfqpoint{0.881661in}{1.501097in}}%
\pgfpathcurveto{\pgfqpoint{0.887485in}{1.506921in}}{\pgfqpoint{0.890757in}{1.514821in}}{\pgfqpoint{0.890757in}{1.523057in}}%
\pgfpathcurveto{\pgfqpoint{0.890757in}{1.531293in}}{\pgfqpoint{0.887485in}{1.539193in}}{\pgfqpoint{0.881661in}{1.545017in}}%
\pgfpathcurveto{\pgfqpoint{0.875837in}{1.550841in}}{\pgfqpoint{0.867937in}{1.554113in}}{\pgfqpoint{0.859701in}{1.554113in}}%
\pgfpathcurveto{\pgfqpoint{0.851464in}{1.554113in}}{\pgfqpoint{0.843564in}{1.550841in}}{\pgfqpoint{0.837740in}{1.545017in}}%
\pgfpathcurveto{\pgfqpoint{0.831916in}{1.539193in}}{\pgfqpoint{0.828644in}{1.531293in}}{\pgfqpoint{0.828644in}{1.523057in}}%
\pgfpathcurveto{\pgfqpoint{0.828644in}{1.514821in}}{\pgfqpoint{0.831916in}{1.506921in}}{\pgfqpoint{0.837740in}{1.501097in}}%
\pgfpathcurveto{\pgfqpoint{0.843564in}{1.495273in}}{\pgfqpoint{0.851464in}{1.492000in}}{\pgfqpoint{0.859701in}{1.492000in}}%
\pgfpathclose%
\pgfusepath{stroke,fill}%
\end{pgfscope}%
\begin{pgfscope}%
\pgfpathrectangle{\pgfqpoint{0.100000in}{0.212622in}}{\pgfqpoint{3.696000in}{3.696000in}}%
\pgfusepath{clip}%
\pgfsetbuttcap%
\pgfsetroundjoin%
\definecolor{currentfill}{rgb}{0.121569,0.466667,0.705882}%
\pgfsetfillcolor{currentfill}%
\pgfsetfillopacity{0.625601}%
\pgfsetlinewidth{1.003750pt}%
\definecolor{currentstroke}{rgb}{0.121569,0.466667,0.705882}%
\pgfsetstrokecolor{currentstroke}%
\pgfsetstrokeopacity{0.625601}%
\pgfsetdash{}{0pt}%
\pgfpathmoveto{\pgfqpoint{0.869649in}{1.511870in}}%
\pgfpathcurveto{\pgfqpoint{0.877885in}{1.511870in}}{\pgfqpoint{0.885785in}{1.515142in}}{\pgfqpoint{0.891609in}{1.520966in}}%
\pgfpathcurveto{\pgfqpoint{0.897433in}{1.526790in}}{\pgfqpoint{0.900706in}{1.534690in}}{\pgfqpoint{0.900706in}{1.542926in}}%
\pgfpathcurveto{\pgfqpoint{0.900706in}{1.551163in}}{\pgfqpoint{0.897433in}{1.559063in}}{\pgfqpoint{0.891609in}{1.564887in}}%
\pgfpathcurveto{\pgfqpoint{0.885785in}{1.570711in}}{\pgfqpoint{0.877885in}{1.573983in}}{\pgfqpoint{0.869649in}{1.573983in}}%
\pgfpathcurveto{\pgfqpoint{0.861413in}{1.573983in}}{\pgfqpoint{0.853513in}{1.570711in}}{\pgfqpoint{0.847689in}{1.564887in}}%
\pgfpathcurveto{\pgfqpoint{0.841865in}{1.559063in}}{\pgfqpoint{0.838593in}{1.551163in}}{\pgfqpoint{0.838593in}{1.542926in}}%
\pgfpathcurveto{\pgfqpoint{0.838593in}{1.534690in}}{\pgfqpoint{0.841865in}{1.526790in}}{\pgfqpoint{0.847689in}{1.520966in}}%
\pgfpathcurveto{\pgfqpoint{0.853513in}{1.515142in}}{\pgfqpoint{0.861413in}{1.511870in}}{\pgfqpoint{0.869649in}{1.511870in}}%
\pgfpathclose%
\pgfusepath{stroke,fill}%
\end{pgfscope}%
\begin{pgfscope}%
\pgfpathrectangle{\pgfqpoint{0.100000in}{0.212622in}}{\pgfqpoint{3.696000in}{3.696000in}}%
\pgfusepath{clip}%
\pgfsetbuttcap%
\pgfsetroundjoin%
\definecolor{currentfill}{rgb}{0.121569,0.466667,0.705882}%
\pgfsetfillcolor{currentfill}%
\pgfsetfillopacity{0.625912}%
\pgfsetlinewidth{1.003750pt}%
\definecolor{currentstroke}{rgb}{0.121569,0.466667,0.705882}%
\pgfsetstrokecolor{currentstroke}%
\pgfsetstrokeopacity{0.625912}%
\pgfsetdash{}{0pt}%
\pgfpathmoveto{\pgfqpoint{0.859654in}{1.492123in}}%
\pgfpathcurveto{\pgfqpoint{0.867890in}{1.492123in}}{\pgfqpoint{0.875790in}{1.495395in}}{\pgfqpoint{0.881614in}{1.501219in}}%
\pgfpathcurveto{\pgfqpoint{0.887438in}{1.507043in}}{\pgfqpoint{0.890711in}{1.514943in}}{\pgfqpoint{0.890711in}{1.523180in}}%
\pgfpathcurveto{\pgfqpoint{0.890711in}{1.531416in}}{\pgfqpoint{0.887438in}{1.539316in}}{\pgfqpoint{0.881614in}{1.545140in}}%
\pgfpathcurveto{\pgfqpoint{0.875790in}{1.550964in}}{\pgfqpoint{0.867890in}{1.554236in}}{\pgfqpoint{0.859654in}{1.554236in}}%
\pgfpathcurveto{\pgfqpoint{0.851418in}{1.554236in}}{\pgfqpoint{0.843518in}{1.550964in}}{\pgfqpoint{0.837694in}{1.545140in}}%
\pgfpathcurveto{\pgfqpoint{0.831870in}{1.539316in}}{\pgfqpoint{0.828598in}{1.531416in}}{\pgfqpoint{0.828598in}{1.523180in}}%
\pgfpathcurveto{\pgfqpoint{0.828598in}{1.514943in}}{\pgfqpoint{0.831870in}{1.507043in}}{\pgfqpoint{0.837694in}{1.501219in}}%
\pgfpathcurveto{\pgfqpoint{0.843518in}{1.495395in}}{\pgfqpoint{0.851418in}{1.492123in}}{\pgfqpoint{0.859654in}{1.492123in}}%
\pgfpathclose%
\pgfusepath{stroke,fill}%
\end{pgfscope}%
\begin{pgfscope}%
\pgfpathrectangle{\pgfqpoint{0.100000in}{0.212622in}}{\pgfqpoint{3.696000in}{3.696000in}}%
\pgfusepath{clip}%
\pgfsetbuttcap%
\pgfsetroundjoin%
\definecolor{currentfill}{rgb}{0.121569,0.466667,0.705882}%
\pgfsetfillcolor{currentfill}%
\pgfsetfillopacity{0.625923}%
\pgfsetlinewidth{1.003750pt}%
\definecolor{currentstroke}{rgb}{0.121569,0.466667,0.705882}%
\pgfsetstrokecolor{currentstroke}%
\pgfsetstrokeopacity{0.625923}%
\pgfsetdash{}{0pt}%
\pgfpathmoveto{\pgfqpoint{2.120837in}{2.011986in}}%
\pgfpathcurveto{\pgfqpoint{2.129073in}{2.011986in}}{\pgfqpoint{2.136973in}{2.015259in}}{\pgfqpoint{2.142797in}{2.021083in}}%
\pgfpathcurveto{\pgfqpoint{2.148621in}{2.026906in}}{\pgfqpoint{2.151894in}{2.034807in}}{\pgfqpoint{2.151894in}{2.043043in}}%
\pgfpathcurveto{\pgfqpoint{2.151894in}{2.051279in}}{\pgfqpoint{2.148621in}{2.059179in}}{\pgfqpoint{2.142797in}{2.065003in}}%
\pgfpathcurveto{\pgfqpoint{2.136973in}{2.070827in}}{\pgfqpoint{2.129073in}{2.074099in}}{\pgfqpoint{2.120837in}{2.074099in}}%
\pgfpathcurveto{\pgfqpoint{2.112601in}{2.074099in}}{\pgfqpoint{2.104701in}{2.070827in}}{\pgfqpoint{2.098877in}{2.065003in}}%
\pgfpathcurveto{\pgfqpoint{2.093053in}{2.059179in}}{\pgfqpoint{2.089781in}{2.051279in}}{\pgfqpoint{2.089781in}{2.043043in}}%
\pgfpathcurveto{\pgfqpoint{2.089781in}{2.034807in}}{\pgfqpoint{2.093053in}{2.026906in}}{\pgfqpoint{2.098877in}{2.021083in}}%
\pgfpathcurveto{\pgfqpoint{2.104701in}{2.015259in}}{\pgfqpoint{2.112601in}{2.011986in}}{\pgfqpoint{2.120837in}{2.011986in}}%
\pgfpathclose%
\pgfusepath{stroke,fill}%
\end{pgfscope}%
\begin{pgfscope}%
\pgfpathrectangle{\pgfqpoint{0.100000in}{0.212622in}}{\pgfqpoint{3.696000in}{3.696000in}}%
\pgfusepath{clip}%
\pgfsetbuttcap%
\pgfsetroundjoin%
\definecolor{currentfill}{rgb}{0.121569,0.466667,0.705882}%
\pgfsetfillcolor{currentfill}%
\pgfsetfillopacity{0.626252}%
\pgfsetlinewidth{1.003750pt}%
\definecolor{currentstroke}{rgb}{0.121569,0.466667,0.705882}%
\pgfsetstrokecolor{currentstroke}%
\pgfsetstrokeopacity{0.626252}%
\pgfsetdash{}{0pt}%
\pgfpathmoveto{\pgfqpoint{0.859660in}{1.492204in}}%
\pgfpathcurveto{\pgfqpoint{0.867897in}{1.492204in}}{\pgfqpoint{0.875797in}{1.495477in}}{\pgfqpoint{0.881621in}{1.501301in}}%
\pgfpathcurveto{\pgfqpoint{0.887445in}{1.507125in}}{\pgfqpoint{0.890717in}{1.515025in}}{\pgfqpoint{0.890717in}{1.523261in}}%
\pgfpathcurveto{\pgfqpoint{0.890717in}{1.531497in}}{\pgfqpoint{0.887445in}{1.539397in}}{\pgfqpoint{0.881621in}{1.545221in}}%
\pgfpathcurveto{\pgfqpoint{0.875797in}{1.551045in}}{\pgfqpoint{0.867897in}{1.554317in}}{\pgfqpoint{0.859660in}{1.554317in}}%
\pgfpathcurveto{\pgfqpoint{0.851424in}{1.554317in}}{\pgfqpoint{0.843524in}{1.551045in}}{\pgfqpoint{0.837700in}{1.545221in}}%
\pgfpathcurveto{\pgfqpoint{0.831876in}{1.539397in}}{\pgfqpoint{0.828604in}{1.531497in}}{\pgfqpoint{0.828604in}{1.523261in}}%
\pgfpathcurveto{\pgfqpoint{0.828604in}{1.515025in}}{\pgfqpoint{0.831876in}{1.507125in}}{\pgfqpoint{0.837700in}{1.501301in}}%
\pgfpathcurveto{\pgfqpoint{0.843524in}{1.495477in}}{\pgfqpoint{0.851424in}{1.492204in}}{\pgfqpoint{0.859660in}{1.492204in}}%
\pgfpathclose%
\pgfusepath{stroke,fill}%
\end{pgfscope}%
\begin{pgfscope}%
\pgfpathrectangle{\pgfqpoint{0.100000in}{0.212622in}}{\pgfqpoint{3.696000in}{3.696000in}}%
\pgfusepath{clip}%
\pgfsetbuttcap%
\pgfsetroundjoin%
\definecolor{currentfill}{rgb}{0.121569,0.466667,0.705882}%
\pgfsetfillcolor{currentfill}%
\pgfsetfillopacity{0.626443}%
\pgfsetlinewidth{1.003750pt}%
\definecolor{currentstroke}{rgb}{0.121569,0.466667,0.705882}%
\pgfsetstrokecolor{currentstroke}%
\pgfsetstrokeopacity{0.626443}%
\pgfsetdash{}{0pt}%
\pgfpathmoveto{\pgfqpoint{0.859680in}{1.492275in}}%
\pgfpathcurveto{\pgfqpoint{0.867916in}{1.492275in}}{\pgfqpoint{0.875816in}{1.495547in}}{\pgfqpoint{0.881640in}{1.501371in}}%
\pgfpathcurveto{\pgfqpoint{0.887464in}{1.507195in}}{\pgfqpoint{0.890736in}{1.515095in}}{\pgfqpoint{0.890736in}{1.523331in}}%
\pgfpathcurveto{\pgfqpoint{0.890736in}{1.531567in}}{\pgfqpoint{0.887464in}{1.539468in}}{\pgfqpoint{0.881640in}{1.545291in}}%
\pgfpathcurveto{\pgfqpoint{0.875816in}{1.551115in}}{\pgfqpoint{0.867916in}{1.554388in}}{\pgfqpoint{0.859680in}{1.554388in}}%
\pgfpathcurveto{\pgfqpoint{0.851444in}{1.554388in}}{\pgfqpoint{0.843544in}{1.551115in}}{\pgfqpoint{0.837720in}{1.545291in}}%
\pgfpathcurveto{\pgfqpoint{0.831896in}{1.539468in}}{\pgfqpoint{0.828623in}{1.531567in}}{\pgfqpoint{0.828623in}{1.523331in}}%
\pgfpathcurveto{\pgfqpoint{0.828623in}{1.515095in}}{\pgfqpoint{0.831896in}{1.507195in}}{\pgfqpoint{0.837720in}{1.501371in}}%
\pgfpathcurveto{\pgfqpoint{0.843544in}{1.495547in}}{\pgfqpoint{0.851444in}{1.492275in}}{\pgfqpoint{0.859680in}{1.492275in}}%
\pgfpathclose%
\pgfusepath{stroke,fill}%
\end{pgfscope}%
\begin{pgfscope}%
\pgfpathrectangle{\pgfqpoint{0.100000in}{0.212622in}}{\pgfqpoint{3.696000in}{3.696000in}}%
\pgfusepath{clip}%
\pgfsetbuttcap%
\pgfsetroundjoin%
\definecolor{currentfill}{rgb}{0.121569,0.466667,0.705882}%
\pgfsetfillcolor{currentfill}%
\pgfsetfillopacity{0.626494}%
\pgfsetlinewidth{1.003750pt}%
\definecolor{currentstroke}{rgb}{0.121569,0.466667,0.705882}%
\pgfsetstrokecolor{currentstroke}%
\pgfsetstrokeopacity{0.626494}%
\pgfsetdash{}{0pt}%
\pgfpathmoveto{\pgfqpoint{0.864918in}{1.502680in}}%
\pgfpathcurveto{\pgfqpoint{0.873155in}{1.502680in}}{\pgfqpoint{0.881055in}{1.505952in}}{\pgfqpoint{0.886879in}{1.511776in}}%
\pgfpathcurveto{\pgfqpoint{0.892703in}{1.517600in}}{\pgfqpoint{0.895975in}{1.525500in}}{\pgfqpoint{0.895975in}{1.533737in}}%
\pgfpathcurveto{\pgfqpoint{0.895975in}{1.541973in}}{\pgfqpoint{0.892703in}{1.549873in}}{\pgfqpoint{0.886879in}{1.555697in}}%
\pgfpathcurveto{\pgfqpoint{0.881055in}{1.561521in}}{\pgfqpoint{0.873155in}{1.564793in}}{\pgfqpoint{0.864918in}{1.564793in}}%
\pgfpathcurveto{\pgfqpoint{0.856682in}{1.564793in}}{\pgfqpoint{0.848782in}{1.561521in}}{\pgfqpoint{0.842958in}{1.555697in}}%
\pgfpathcurveto{\pgfqpoint{0.837134in}{1.549873in}}{\pgfqpoint{0.833862in}{1.541973in}}{\pgfqpoint{0.833862in}{1.533737in}}%
\pgfpathcurveto{\pgfqpoint{0.833862in}{1.525500in}}{\pgfqpoint{0.837134in}{1.517600in}}{\pgfqpoint{0.842958in}{1.511776in}}%
\pgfpathcurveto{\pgfqpoint{0.848782in}{1.505952in}}{\pgfqpoint{0.856682in}{1.502680in}}{\pgfqpoint{0.864918in}{1.502680in}}%
\pgfpathclose%
\pgfusepath{stroke,fill}%
\end{pgfscope}%
\begin{pgfscope}%
\pgfpathrectangle{\pgfqpoint{0.100000in}{0.212622in}}{\pgfqpoint{3.696000in}{3.696000in}}%
\pgfusepath{clip}%
\pgfsetbuttcap%
\pgfsetroundjoin%
\definecolor{currentfill}{rgb}{0.121569,0.466667,0.705882}%
\pgfsetfillcolor{currentfill}%
\pgfsetfillopacity{0.626546}%
\pgfsetlinewidth{1.003750pt}%
\definecolor{currentstroke}{rgb}{0.121569,0.466667,0.705882}%
\pgfsetstrokecolor{currentstroke}%
\pgfsetstrokeopacity{0.626546}%
\pgfsetdash{}{0pt}%
\pgfpathmoveto{\pgfqpoint{0.859702in}{1.492317in}}%
\pgfpathcurveto{\pgfqpoint{0.867938in}{1.492317in}}{\pgfqpoint{0.875838in}{1.495589in}}{\pgfqpoint{0.881662in}{1.501413in}}%
\pgfpathcurveto{\pgfqpoint{0.887486in}{1.507237in}}{\pgfqpoint{0.890758in}{1.515137in}}{\pgfqpoint{0.890758in}{1.523373in}}%
\pgfpathcurveto{\pgfqpoint{0.890758in}{1.531610in}}{\pgfqpoint{0.887486in}{1.539510in}}{\pgfqpoint{0.881662in}{1.545334in}}%
\pgfpathcurveto{\pgfqpoint{0.875838in}{1.551158in}}{\pgfqpoint{0.867938in}{1.554430in}}{\pgfqpoint{0.859702in}{1.554430in}}%
\pgfpathcurveto{\pgfqpoint{0.851465in}{1.554430in}}{\pgfqpoint{0.843565in}{1.551158in}}{\pgfqpoint{0.837741in}{1.545334in}}%
\pgfpathcurveto{\pgfqpoint{0.831917in}{1.539510in}}{\pgfqpoint{0.828645in}{1.531610in}}{\pgfqpoint{0.828645in}{1.523373in}}%
\pgfpathcurveto{\pgfqpoint{0.828645in}{1.515137in}}{\pgfqpoint{0.831917in}{1.507237in}}{\pgfqpoint{0.837741in}{1.501413in}}%
\pgfpathcurveto{\pgfqpoint{0.843565in}{1.495589in}}{\pgfqpoint{0.851465in}{1.492317in}}{\pgfqpoint{0.859702in}{1.492317in}}%
\pgfpathclose%
\pgfusepath{stroke,fill}%
\end{pgfscope}%
\begin{pgfscope}%
\pgfpathrectangle{\pgfqpoint{0.100000in}{0.212622in}}{\pgfqpoint{3.696000in}{3.696000in}}%
\pgfusepath{clip}%
\pgfsetbuttcap%
\pgfsetroundjoin%
\definecolor{currentfill}{rgb}{0.121569,0.466667,0.705882}%
\pgfsetfillcolor{currentfill}%
\pgfsetfillopacity{0.626601}%
\pgfsetlinewidth{1.003750pt}%
\definecolor{currentstroke}{rgb}{0.121569,0.466667,0.705882}%
\pgfsetstrokecolor{currentstroke}%
\pgfsetstrokeopacity{0.626601}%
\pgfsetdash{}{0pt}%
\pgfpathmoveto{\pgfqpoint{0.859721in}{1.492341in}}%
\pgfpathcurveto{\pgfqpoint{0.867957in}{1.492341in}}{\pgfqpoint{0.875857in}{1.495614in}}{\pgfqpoint{0.881681in}{1.501437in}}%
\pgfpathcurveto{\pgfqpoint{0.887505in}{1.507261in}}{\pgfqpoint{0.890777in}{1.515161in}}{\pgfqpoint{0.890777in}{1.523398in}}%
\pgfpathcurveto{\pgfqpoint{0.890777in}{1.531634in}}{\pgfqpoint{0.887505in}{1.539534in}}{\pgfqpoint{0.881681in}{1.545358in}}%
\pgfpathcurveto{\pgfqpoint{0.875857in}{1.551182in}}{\pgfqpoint{0.867957in}{1.554454in}}{\pgfqpoint{0.859721in}{1.554454in}}%
\pgfpathcurveto{\pgfqpoint{0.851484in}{1.554454in}}{\pgfqpoint{0.843584in}{1.551182in}}{\pgfqpoint{0.837760in}{1.545358in}}%
\pgfpathcurveto{\pgfqpoint{0.831936in}{1.539534in}}{\pgfqpoint{0.828664in}{1.531634in}}{\pgfqpoint{0.828664in}{1.523398in}}%
\pgfpathcurveto{\pgfqpoint{0.828664in}{1.515161in}}{\pgfqpoint{0.831936in}{1.507261in}}{\pgfqpoint{0.837760in}{1.501437in}}%
\pgfpathcurveto{\pgfqpoint{0.843584in}{1.495614in}}{\pgfqpoint{0.851484in}{1.492341in}}{\pgfqpoint{0.859721in}{1.492341in}}%
\pgfpathclose%
\pgfusepath{stroke,fill}%
\end{pgfscope}%
\begin{pgfscope}%
\pgfpathrectangle{\pgfqpoint{0.100000in}{0.212622in}}{\pgfqpoint{3.696000in}{3.696000in}}%
\pgfusepath{clip}%
\pgfsetbuttcap%
\pgfsetroundjoin%
\definecolor{currentfill}{rgb}{0.121569,0.466667,0.705882}%
\pgfsetfillcolor{currentfill}%
\pgfsetfillopacity{0.626821}%
\pgfsetlinewidth{1.003750pt}%
\definecolor{currentstroke}{rgb}{0.121569,0.466667,0.705882}%
\pgfsetstrokecolor{currentstroke}%
\pgfsetstrokeopacity{0.626821}%
\pgfsetdash{}{0pt}%
\pgfpathmoveto{\pgfqpoint{0.859819in}{1.492465in}}%
\pgfpathcurveto{\pgfqpoint{0.868055in}{1.492465in}}{\pgfqpoint{0.875955in}{1.495737in}}{\pgfqpoint{0.881779in}{1.501561in}}%
\pgfpathcurveto{\pgfqpoint{0.887603in}{1.507385in}}{\pgfqpoint{0.890875in}{1.515285in}}{\pgfqpoint{0.890875in}{1.523521in}}%
\pgfpathcurveto{\pgfqpoint{0.890875in}{1.531757in}}{\pgfqpoint{0.887603in}{1.539657in}}{\pgfqpoint{0.881779in}{1.545481in}}%
\pgfpathcurveto{\pgfqpoint{0.875955in}{1.551305in}}{\pgfqpoint{0.868055in}{1.554578in}}{\pgfqpoint{0.859819in}{1.554578in}}%
\pgfpathcurveto{\pgfqpoint{0.851582in}{1.554578in}}{\pgfqpoint{0.843682in}{1.551305in}}{\pgfqpoint{0.837858in}{1.545481in}}%
\pgfpathcurveto{\pgfqpoint{0.832034in}{1.539657in}}{\pgfqpoint{0.828762in}{1.531757in}}{\pgfqpoint{0.828762in}{1.523521in}}%
\pgfpathcurveto{\pgfqpoint{0.828762in}{1.515285in}}{\pgfqpoint{0.832034in}{1.507385in}}{\pgfqpoint{0.837858in}{1.501561in}}%
\pgfpathcurveto{\pgfqpoint{0.843682in}{1.495737in}}{\pgfqpoint{0.851582in}{1.492465in}}{\pgfqpoint{0.859819in}{1.492465in}}%
\pgfpathclose%
\pgfusepath{stroke,fill}%
\end{pgfscope}%
\begin{pgfscope}%
\pgfpathrectangle{\pgfqpoint{0.100000in}{0.212622in}}{\pgfqpoint{3.696000in}{3.696000in}}%
\pgfusepath{clip}%
\pgfsetbuttcap%
\pgfsetroundjoin%
\definecolor{currentfill}{rgb}{0.121569,0.466667,0.705882}%
\pgfsetfillcolor{currentfill}%
\pgfsetfillopacity{0.626942}%
\pgfsetlinewidth{1.003750pt}%
\definecolor{currentstroke}{rgb}{0.121569,0.466667,0.705882}%
\pgfsetstrokecolor{currentstroke}%
\pgfsetstrokeopacity{0.626942}%
\pgfsetdash{}{0pt}%
\pgfpathmoveto{\pgfqpoint{0.859883in}{1.492547in}}%
\pgfpathcurveto{\pgfqpoint{0.868119in}{1.492547in}}{\pgfqpoint{0.876019in}{1.495820in}}{\pgfqpoint{0.881843in}{1.501644in}}%
\pgfpathcurveto{\pgfqpoint{0.887667in}{1.507468in}}{\pgfqpoint{0.890939in}{1.515368in}}{\pgfqpoint{0.890939in}{1.523604in}}%
\pgfpathcurveto{\pgfqpoint{0.890939in}{1.531840in}}{\pgfqpoint{0.887667in}{1.539740in}}{\pgfqpoint{0.881843in}{1.545564in}}%
\pgfpathcurveto{\pgfqpoint{0.876019in}{1.551388in}}{\pgfqpoint{0.868119in}{1.554660in}}{\pgfqpoint{0.859883in}{1.554660in}}%
\pgfpathcurveto{\pgfqpoint{0.851647in}{1.554660in}}{\pgfqpoint{0.843747in}{1.551388in}}{\pgfqpoint{0.837923in}{1.545564in}}%
\pgfpathcurveto{\pgfqpoint{0.832099in}{1.539740in}}{\pgfqpoint{0.828826in}{1.531840in}}{\pgfqpoint{0.828826in}{1.523604in}}%
\pgfpathcurveto{\pgfqpoint{0.828826in}{1.515368in}}{\pgfqpoint{0.832099in}{1.507468in}}{\pgfqpoint{0.837923in}{1.501644in}}%
\pgfpathcurveto{\pgfqpoint{0.843747in}{1.495820in}}{\pgfqpoint{0.851647in}{1.492547in}}{\pgfqpoint{0.859883in}{1.492547in}}%
\pgfpathclose%
\pgfusepath{stroke,fill}%
\end{pgfscope}%
\begin{pgfscope}%
\pgfpathrectangle{\pgfqpoint{0.100000in}{0.212622in}}{\pgfqpoint{3.696000in}{3.696000in}}%
\pgfusepath{clip}%
\pgfsetbuttcap%
\pgfsetroundjoin%
\definecolor{currentfill}{rgb}{0.121569,0.466667,0.705882}%
\pgfsetfillcolor{currentfill}%
\pgfsetfillopacity{0.627009}%
\pgfsetlinewidth{1.003750pt}%
\definecolor{currentstroke}{rgb}{0.121569,0.466667,0.705882}%
\pgfsetstrokecolor{currentstroke}%
\pgfsetstrokeopacity{0.627009}%
\pgfsetdash{}{0pt}%
\pgfpathmoveto{\pgfqpoint{0.859925in}{1.492605in}}%
\pgfpathcurveto{\pgfqpoint{0.868161in}{1.492605in}}{\pgfqpoint{0.876061in}{1.495877in}}{\pgfqpoint{0.881885in}{1.501701in}}%
\pgfpathcurveto{\pgfqpoint{0.887709in}{1.507525in}}{\pgfqpoint{0.890981in}{1.515425in}}{\pgfqpoint{0.890981in}{1.523662in}}%
\pgfpathcurveto{\pgfqpoint{0.890981in}{1.531898in}}{\pgfqpoint{0.887709in}{1.539798in}}{\pgfqpoint{0.881885in}{1.545622in}}%
\pgfpathcurveto{\pgfqpoint{0.876061in}{1.551446in}}{\pgfqpoint{0.868161in}{1.554718in}}{\pgfqpoint{0.859925in}{1.554718in}}%
\pgfpathcurveto{\pgfqpoint{0.851688in}{1.554718in}}{\pgfqpoint{0.843788in}{1.551446in}}{\pgfqpoint{0.837965in}{1.545622in}}%
\pgfpathcurveto{\pgfqpoint{0.832141in}{1.539798in}}{\pgfqpoint{0.828868in}{1.531898in}}{\pgfqpoint{0.828868in}{1.523662in}}%
\pgfpathcurveto{\pgfqpoint{0.828868in}{1.515425in}}{\pgfqpoint{0.832141in}{1.507525in}}{\pgfqpoint{0.837965in}{1.501701in}}%
\pgfpathcurveto{\pgfqpoint{0.843788in}{1.495877in}}{\pgfqpoint{0.851688in}{1.492605in}}{\pgfqpoint{0.859925in}{1.492605in}}%
\pgfpathclose%
\pgfusepath{stroke,fill}%
\end{pgfscope}%
\begin{pgfscope}%
\pgfpathrectangle{\pgfqpoint{0.100000in}{0.212622in}}{\pgfqpoint{3.696000in}{3.696000in}}%
\pgfusepath{clip}%
\pgfsetbuttcap%
\pgfsetroundjoin%
\definecolor{currentfill}{rgb}{0.121569,0.466667,0.705882}%
\pgfsetfillcolor{currentfill}%
\pgfsetfillopacity{0.627045}%
\pgfsetlinewidth{1.003750pt}%
\definecolor{currentstroke}{rgb}{0.121569,0.466667,0.705882}%
\pgfsetstrokecolor{currentstroke}%
\pgfsetstrokeopacity{0.627045}%
\pgfsetdash{}{0pt}%
\pgfpathmoveto{\pgfqpoint{0.859952in}{1.492641in}}%
\pgfpathcurveto{\pgfqpoint{0.868189in}{1.492641in}}{\pgfqpoint{0.876089in}{1.495914in}}{\pgfqpoint{0.881913in}{1.501738in}}%
\pgfpathcurveto{\pgfqpoint{0.887736in}{1.507562in}}{\pgfqpoint{0.891009in}{1.515462in}}{\pgfqpoint{0.891009in}{1.523698in}}%
\pgfpathcurveto{\pgfqpoint{0.891009in}{1.531934in}}{\pgfqpoint{0.887736in}{1.539834in}}{\pgfqpoint{0.881913in}{1.545658in}}%
\pgfpathcurveto{\pgfqpoint{0.876089in}{1.551482in}}{\pgfqpoint{0.868189in}{1.554754in}}{\pgfqpoint{0.859952in}{1.554754in}}%
\pgfpathcurveto{\pgfqpoint{0.851716in}{1.554754in}}{\pgfqpoint{0.843816in}{1.551482in}}{\pgfqpoint{0.837992in}{1.545658in}}%
\pgfpathcurveto{\pgfqpoint{0.832168in}{1.539834in}}{\pgfqpoint{0.828896in}{1.531934in}}{\pgfqpoint{0.828896in}{1.523698in}}%
\pgfpathcurveto{\pgfqpoint{0.828896in}{1.515462in}}{\pgfqpoint{0.832168in}{1.507562in}}{\pgfqpoint{0.837992in}{1.501738in}}%
\pgfpathcurveto{\pgfqpoint{0.843816in}{1.495914in}}{\pgfqpoint{0.851716in}{1.492641in}}{\pgfqpoint{0.859952in}{1.492641in}}%
\pgfpathclose%
\pgfusepath{stroke,fill}%
\end{pgfscope}%
\begin{pgfscope}%
\pgfpathrectangle{\pgfqpoint{0.100000in}{0.212622in}}{\pgfqpoint{3.696000in}{3.696000in}}%
\pgfusepath{clip}%
\pgfsetbuttcap%
\pgfsetroundjoin%
\definecolor{currentfill}{rgb}{0.121569,0.466667,0.705882}%
\pgfsetfillcolor{currentfill}%
\pgfsetfillopacity{0.627144}%
\pgfsetlinewidth{1.003750pt}%
\definecolor{currentstroke}{rgb}{0.121569,0.466667,0.705882}%
\pgfsetstrokecolor{currentstroke}%
\pgfsetstrokeopacity{0.627144}%
\pgfsetdash{}{0pt}%
\pgfpathmoveto{\pgfqpoint{0.861917in}{1.496639in}}%
\pgfpathcurveto{\pgfqpoint{0.870154in}{1.496639in}}{\pgfqpoint{0.878054in}{1.499912in}}{\pgfqpoint{0.883878in}{1.505736in}}%
\pgfpathcurveto{\pgfqpoint{0.889702in}{1.511560in}}{\pgfqpoint{0.892974in}{1.519460in}}{\pgfqpoint{0.892974in}{1.527696in}}%
\pgfpathcurveto{\pgfqpoint{0.892974in}{1.535932in}}{\pgfqpoint{0.889702in}{1.543832in}}{\pgfqpoint{0.883878in}{1.549656in}}%
\pgfpathcurveto{\pgfqpoint{0.878054in}{1.555480in}}{\pgfqpoint{0.870154in}{1.558752in}}{\pgfqpoint{0.861917in}{1.558752in}}%
\pgfpathcurveto{\pgfqpoint{0.853681in}{1.558752in}}{\pgfqpoint{0.845781in}{1.555480in}}{\pgfqpoint{0.839957in}{1.549656in}}%
\pgfpathcurveto{\pgfqpoint{0.834133in}{1.543832in}}{\pgfqpoint{0.830861in}{1.535932in}}{\pgfqpoint{0.830861in}{1.527696in}}%
\pgfpathcurveto{\pgfqpoint{0.830861in}{1.519460in}}{\pgfqpoint{0.834133in}{1.511560in}}{\pgfqpoint{0.839957in}{1.505736in}}%
\pgfpathcurveto{\pgfqpoint{0.845781in}{1.499912in}}{\pgfqpoint{0.853681in}{1.496639in}}{\pgfqpoint{0.861917in}{1.496639in}}%
\pgfpathclose%
\pgfusepath{stroke,fill}%
\end{pgfscope}%
\begin{pgfscope}%
\pgfpathrectangle{\pgfqpoint{0.100000in}{0.212622in}}{\pgfqpoint{3.696000in}{3.696000in}}%
\pgfusepath{clip}%
\pgfsetbuttcap%
\pgfsetroundjoin%
\definecolor{currentfill}{rgb}{0.121569,0.466667,0.705882}%
\pgfsetfillcolor{currentfill}%
\pgfsetfillopacity{0.627238}%
\pgfsetlinewidth{1.003750pt}%
\definecolor{currentstroke}{rgb}{0.121569,0.466667,0.705882}%
\pgfsetstrokecolor{currentstroke}%
\pgfsetstrokeopacity{0.627238}%
\pgfsetdash{}{0pt}%
\pgfpathmoveto{\pgfqpoint{0.860133in}{1.492854in}}%
\pgfpathcurveto{\pgfqpoint{0.868369in}{1.492854in}}{\pgfqpoint{0.876269in}{1.496126in}}{\pgfqpoint{0.882093in}{1.501950in}}%
\pgfpathcurveto{\pgfqpoint{0.887917in}{1.507774in}}{\pgfqpoint{0.891189in}{1.515674in}}{\pgfqpoint{0.891189in}{1.523911in}}%
\pgfpathcurveto{\pgfqpoint{0.891189in}{1.532147in}}{\pgfqpoint{0.887917in}{1.540047in}}{\pgfqpoint{0.882093in}{1.545871in}}%
\pgfpathcurveto{\pgfqpoint{0.876269in}{1.551695in}}{\pgfqpoint{0.868369in}{1.554967in}}{\pgfqpoint{0.860133in}{1.554967in}}%
\pgfpathcurveto{\pgfqpoint{0.851896in}{1.554967in}}{\pgfqpoint{0.843996in}{1.551695in}}{\pgfqpoint{0.838172in}{1.545871in}}%
\pgfpathcurveto{\pgfqpoint{0.832348in}{1.540047in}}{\pgfqpoint{0.829076in}{1.532147in}}{\pgfqpoint{0.829076in}{1.523911in}}%
\pgfpathcurveto{\pgfqpoint{0.829076in}{1.515674in}}{\pgfqpoint{0.832348in}{1.507774in}}{\pgfqpoint{0.838172in}{1.501950in}}%
\pgfpathcurveto{\pgfqpoint{0.843996in}{1.496126in}}{\pgfqpoint{0.851896in}{1.492854in}}{\pgfqpoint{0.860133in}{1.492854in}}%
\pgfpathclose%
\pgfusepath{stroke,fill}%
\end{pgfscope}%
\begin{pgfscope}%
\pgfpathrectangle{\pgfqpoint{0.100000in}{0.212622in}}{\pgfqpoint{3.696000in}{3.696000in}}%
\pgfusepath{clip}%
\pgfsetbuttcap%
\pgfsetroundjoin%
\definecolor{currentfill}{rgb}{0.121569,0.466667,0.705882}%
\pgfsetfillcolor{currentfill}%
\pgfsetfillopacity{0.627333}%
\pgfsetlinewidth{1.003750pt}%
\definecolor{currentstroke}{rgb}{0.121569,0.466667,0.705882}%
\pgfsetstrokecolor{currentstroke}%
\pgfsetstrokeopacity{0.627333}%
\pgfsetdash{}{0pt}%
\pgfpathmoveto{\pgfqpoint{0.860807in}{1.494116in}}%
\pgfpathcurveto{\pgfqpoint{0.869043in}{1.494116in}}{\pgfqpoint{0.876943in}{1.497388in}}{\pgfqpoint{0.882767in}{1.503212in}}%
\pgfpathcurveto{\pgfqpoint{0.888591in}{1.509036in}}{\pgfqpoint{0.891863in}{1.516936in}}{\pgfqpoint{0.891863in}{1.525172in}}%
\pgfpathcurveto{\pgfqpoint{0.891863in}{1.533408in}}{\pgfqpoint{0.888591in}{1.541308in}}{\pgfqpoint{0.882767in}{1.547132in}}%
\pgfpathcurveto{\pgfqpoint{0.876943in}{1.552956in}}{\pgfqpoint{0.869043in}{1.556229in}}{\pgfqpoint{0.860807in}{1.556229in}}%
\pgfpathcurveto{\pgfqpoint{0.852571in}{1.556229in}}{\pgfqpoint{0.844670in}{1.552956in}}{\pgfqpoint{0.838847in}{1.547132in}}%
\pgfpathcurveto{\pgfqpoint{0.833023in}{1.541308in}}{\pgfqpoint{0.829750in}{1.533408in}}{\pgfqpoint{0.829750in}{1.525172in}}%
\pgfpathcurveto{\pgfqpoint{0.829750in}{1.516936in}}{\pgfqpoint{0.833023in}{1.509036in}}{\pgfqpoint{0.838847in}{1.503212in}}%
\pgfpathcurveto{\pgfqpoint{0.844670in}{1.497388in}}{\pgfqpoint{0.852571in}{1.494116in}}{\pgfqpoint{0.860807in}{1.494116in}}%
\pgfpathclose%
\pgfusepath{stroke,fill}%
\end{pgfscope}%
\begin{pgfscope}%
\pgfpathrectangle{\pgfqpoint{0.100000in}{0.212622in}}{\pgfqpoint{3.696000in}{3.696000in}}%
\pgfusepath{clip}%
\pgfsetbuttcap%
\pgfsetroundjoin%
\definecolor{currentfill}{rgb}{0.121569,0.466667,0.705882}%
\pgfsetfillcolor{currentfill}%
\pgfsetfillopacity{0.627335}%
\pgfsetlinewidth{1.003750pt}%
\definecolor{currentstroke}{rgb}{0.121569,0.466667,0.705882}%
\pgfsetstrokecolor{currentstroke}%
\pgfsetstrokeopacity{0.627335}%
\pgfsetdash{}{0pt}%
\pgfpathmoveto{\pgfqpoint{0.860246in}{1.492965in}}%
\pgfpathcurveto{\pgfqpoint{0.868482in}{1.492965in}}{\pgfqpoint{0.876382in}{1.496238in}}{\pgfqpoint{0.882206in}{1.502062in}}%
\pgfpathcurveto{\pgfqpoint{0.888030in}{1.507886in}}{\pgfqpoint{0.891302in}{1.515786in}}{\pgfqpoint{0.891302in}{1.524022in}}%
\pgfpathcurveto{\pgfqpoint{0.891302in}{1.532258in}}{\pgfqpoint{0.888030in}{1.540158in}}{\pgfqpoint{0.882206in}{1.545982in}}%
\pgfpathcurveto{\pgfqpoint{0.876382in}{1.551806in}}{\pgfqpoint{0.868482in}{1.555078in}}{\pgfqpoint{0.860246in}{1.555078in}}%
\pgfpathcurveto{\pgfqpoint{0.852010in}{1.555078in}}{\pgfqpoint{0.844110in}{1.551806in}}{\pgfqpoint{0.838286in}{1.545982in}}%
\pgfpathcurveto{\pgfqpoint{0.832462in}{1.540158in}}{\pgfqpoint{0.829189in}{1.532258in}}{\pgfqpoint{0.829189in}{1.524022in}}%
\pgfpathcurveto{\pgfqpoint{0.829189in}{1.515786in}}{\pgfqpoint{0.832462in}{1.507886in}}{\pgfqpoint{0.838286in}{1.502062in}}%
\pgfpathcurveto{\pgfqpoint{0.844110in}{1.496238in}}{\pgfqpoint{0.852010in}{1.492965in}}{\pgfqpoint{0.860246in}{1.492965in}}%
\pgfpathclose%
\pgfusepath{stroke,fill}%
\end{pgfscope}%
\begin{pgfscope}%
\pgfpathrectangle{\pgfqpoint{0.100000in}{0.212622in}}{\pgfqpoint{3.696000in}{3.696000in}}%
\pgfusepath{clip}%
\pgfsetbuttcap%
\pgfsetroundjoin%
\definecolor{currentfill}{rgb}{0.121569,0.466667,0.705882}%
\pgfsetfillcolor{currentfill}%
\pgfsetfillopacity{0.627348}%
\pgfsetlinewidth{1.003750pt}%
\definecolor{currentstroke}{rgb}{0.121569,0.466667,0.705882}%
\pgfsetstrokecolor{currentstroke}%
\pgfsetstrokeopacity{0.627348}%
\pgfsetdash{}{0pt}%
\pgfpathmoveto{\pgfqpoint{0.860756in}{1.493935in}}%
\pgfpathcurveto{\pgfqpoint{0.868993in}{1.493935in}}{\pgfqpoint{0.876893in}{1.497207in}}{\pgfqpoint{0.882717in}{1.503031in}}%
\pgfpathcurveto{\pgfqpoint{0.888541in}{1.508855in}}{\pgfqpoint{0.891813in}{1.516755in}}{\pgfqpoint{0.891813in}{1.524992in}}%
\pgfpathcurveto{\pgfqpoint{0.891813in}{1.533228in}}{\pgfqpoint{0.888541in}{1.541128in}}{\pgfqpoint{0.882717in}{1.546952in}}%
\pgfpathcurveto{\pgfqpoint{0.876893in}{1.552776in}}{\pgfqpoint{0.868993in}{1.556048in}}{\pgfqpoint{0.860756in}{1.556048in}}%
\pgfpathcurveto{\pgfqpoint{0.852520in}{1.556048in}}{\pgfqpoint{0.844620in}{1.552776in}}{\pgfqpoint{0.838796in}{1.546952in}}%
\pgfpathcurveto{\pgfqpoint{0.832972in}{1.541128in}}{\pgfqpoint{0.829700in}{1.533228in}}{\pgfqpoint{0.829700in}{1.524992in}}%
\pgfpathcurveto{\pgfqpoint{0.829700in}{1.516755in}}{\pgfqpoint{0.832972in}{1.508855in}}{\pgfqpoint{0.838796in}{1.503031in}}%
\pgfpathcurveto{\pgfqpoint{0.844620in}{1.497207in}}{\pgfqpoint{0.852520in}{1.493935in}}{\pgfqpoint{0.860756in}{1.493935in}}%
\pgfpathclose%
\pgfusepath{stroke,fill}%
\end{pgfscope}%
\begin{pgfscope}%
\pgfpathrectangle{\pgfqpoint{0.100000in}{0.212622in}}{\pgfqpoint{3.696000in}{3.696000in}}%
\pgfusepath{clip}%
\pgfsetbuttcap%
\pgfsetroundjoin%
\definecolor{currentfill}{rgb}{0.121569,0.466667,0.705882}%
\pgfsetfillcolor{currentfill}%
\pgfsetfillopacity{0.627375}%
\pgfsetlinewidth{1.003750pt}%
\definecolor{currentstroke}{rgb}{0.121569,0.466667,0.705882}%
\pgfsetstrokecolor{currentstroke}%
\pgfsetstrokeopacity{0.627375}%
\pgfsetdash{}{0pt}%
\pgfpathmoveto{\pgfqpoint{0.860622in}{1.493650in}}%
\pgfpathcurveto{\pgfqpoint{0.868858in}{1.493650in}}{\pgfqpoint{0.876758in}{1.496922in}}{\pgfqpoint{0.882582in}{1.502746in}}%
\pgfpathcurveto{\pgfqpoint{0.888406in}{1.508570in}}{\pgfqpoint{0.891678in}{1.516470in}}{\pgfqpoint{0.891678in}{1.524706in}}%
\pgfpathcurveto{\pgfqpoint{0.891678in}{1.532943in}}{\pgfqpoint{0.888406in}{1.540843in}}{\pgfqpoint{0.882582in}{1.546667in}}%
\pgfpathcurveto{\pgfqpoint{0.876758in}{1.552491in}}{\pgfqpoint{0.868858in}{1.555763in}}{\pgfqpoint{0.860622in}{1.555763in}}%
\pgfpathcurveto{\pgfqpoint{0.852385in}{1.555763in}}{\pgfqpoint{0.844485in}{1.552491in}}{\pgfqpoint{0.838661in}{1.546667in}}%
\pgfpathcurveto{\pgfqpoint{0.832837in}{1.540843in}}{\pgfqpoint{0.829565in}{1.532943in}}{\pgfqpoint{0.829565in}{1.524706in}}%
\pgfpathcurveto{\pgfqpoint{0.829565in}{1.516470in}}{\pgfqpoint{0.832837in}{1.508570in}}{\pgfqpoint{0.838661in}{1.502746in}}%
\pgfpathcurveto{\pgfqpoint{0.844485in}{1.496922in}}{\pgfqpoint{0.852385in}{1.493650in}}{\pgfqpoint{0.860622in}{1.493650in}}%
\pgfpathclose%
\pgfusepath{stroke,fill}%
\end{pgfscope}%
\begin{pgfscope}%
\pgfpathrectangle{\pgfqpoint{0.100000in}{0.212622in}}{\pgfqpoint{3.696000in}{3.696000in}}%
\pgfusepath{clip}%
\pgfsetbuttcap%
\pgfsetroundjoin%
\definecolor{currentfill}{rgb}{0.121569,0.466667,0.705882}%
\pgfsetfillcolor{currentfill}%
\pgfsetfillopacity{0.627383}%
\pgfsetlinewidth{1.003750pt}%
\definecolor{currentstroke}{rgb}{0.121569,0.466667,0.705882}%
\pgfsetstrokecolor{currentstroke}%
\pgfsetstrokeopacity{0.627383}%
\pgfsetdash{}{0pt}%
\pgfpathmoveto{\pgfqpoint{0.860315in}{1.493027in}}%
\pgfpathcurveto{\pgfqpoint{0.868551in}{1.493027in}}{\pgfqpoint{0.876451in}{1.496299in}}{\pgfqpoint{0.882275in}{1.502123in}}%
\pgfpathcurveto{\pgfqpoint{0.888099in}{1.507947in}}{\pgfqpoint{0.891372in}{1.515847in}}{\pgfqpoint{0.891372in}{1.524084in}}%
\pgfpathcurveto{\pgfqpoint{0.891372in}{1.532320in}}{\pgfqpoint{0.888099in}{1.540220in}}{\pgfqpoint{0.882275in}{1.546044in}}%
\pgfpathcurveto{\pgfqpoint{0.876451in}{1.551868in}}{\pgfqpoint{0.868551in}{1.555140in}}{\pgfqpoint{0.860315in}{1.555140in}}%
\pgfpathcurveto{\pgfqpoint{0.852079in}{1.555140in}}{\pgfqpoint{0.844179in}{1.551868in}}{\pgfqpoint{0.838355in}{1.546044in}}%
\pgfpathcurveto{\pgfqpoint{0.832531in}{1.540220in}}{\pgfqpoint{0.829259in}{1.532320in}}{\pgfqpoint{0.829259in}{1.524084in}}%
\pgfpathcurveto{\pgfqpoint{0.829259in}{1.515847in}}{\pgfqpoint{0.832531in}{1.507947in}}{\pgfqpoint{0.838355in}{1.502123in}}%
\pgfpathcurveto{\pgfqpoint{0.844179in}{1.496299in}}{\pgfqpoint{0.852079in}{1.493027in}}{\pgfqpoint{0.860315in}{1.493027in}}%
\pgfpathclose%
\pgfusepath{stroke,fill}%
\end{pgfscope}%
\begin{pgfscope}%
\pgfpathrectangle{\pgfqpoint{0.100000in}{0.212622in}}{\pgfqpoint{3.696000in}{3.696000in}}%
\pgfusepath{clip}%
\pgfsetbuttcap%
\pgfsetroundjoin%
\definecolor{currentfill}{rgb}{0.121569,0.466667,0.705882}%
\pgfsetfillcolor{currentfill}%
\pgfsetfillopacity{0.627409}%
\pgfsetlinewidth{1.003750pt}%
\definecolor{currentstroke}{rgb}{0.121569,0.466667,0.705882}%
\pgfsetstrokecolor{currentstroke}%
\pgfsetstrokeopacity{0.627409}%
\pgfsetdash{}{0pt}%
\pgfpathmoveto{\pgfqpoint{0.860356in}{1.493065in}}%
\pgfpathcurveto{\pgfqpoint{0.868593in}{1.493065in}}{\pgfqpoint{0.876493in}{1.496337in}}{\pgfqpoint{0.882317in}{1.502161in}}%
\pgfpathcurveto{\pgfqpoint{0.888141in}{1.507985in}}{\pgfqpoint{0.891413in}{1.515885in}}{\pgfqpoint{0.891413in}{1.524122in}}%
\pgfpathcurveto{\pgfqpoint{0.891413in}{1.532358in}}{\pgfqpoint{0.888141in}{1.540258in}}{\pgfqpoint{0.882317in}{1.546082in}}%
\pgfpathcurveto{\pgfqpoint{0.876493in}{1.551906in}}{\pgfqpoint{0.868593in}{1.555178in}}{\pgfqpoint{0.860356in}{1.555178in}}%
\pgfpathcurveto{\pgfqpoint{0.852120in}{1.555178in}}{\pgfqpoint{0.844220in}{1.551906in}}{\pgfqpoint{0.838396in}{1.546082in}}%
\pgfpathcurveto{\pgfqpoint{0.832572in}{1.540258in}}{\pgfqpoint{0.829300in}{1.532358in}}{\pgfqpoint{0.829300in}{1.524122in}}%
\pgfpathcurveto{\pgfqpoint{0.829300in}{1.515885in}}{\pgfqpoint{0.832572in}{1.507985in}}{\pgfqpoint{0.838396in}{1.502161in}}%
\pgfpathcurveto{\pgfqpoint{0.844220in}{1.496337in}}{\pgfqpoint{0.852120in}{1.493065in}}{\pgfqpoint{0.860356in}{1.493065in}}%
\pgfpathclose%
\pgfusepath{stroke,fill}%
\end{pgfscope}%
\begin{pgfscope}%
\pgfpathrectangle{\pgfqpoint{0.100000in}{0.212622in}}{\pgfqpoint{3.696000in}{3.696000in}}%
\pgfusepath{clip}%
\pgfsetbuttcap%
\pgfsetroundjoin%
\definecolor{currentfill}{rgb}{0.121569,0.466667,0.705882}%
\pgfsetfillcolor{currentfill}%
\pgfsetfillopacity{0.627421}%
\pgfsetlinewidth{1.003750pt}%
\definecolor{currentstroke}{rgb}{0.121569,0.466667,0.705882}%
\pgfsetstrokecolor{currentstroke}%
\pgfsetstrokeopacity{0.627421}%
\pgfsetdash{}{0pt}%
\pgfpathmoveto{\pgfqpoint{0.860381in}{1.493087in}}%
\pgfpathcurveto{\pgfqpoint{0.868617in}{1.493087in}}{\pgfqpoint{0.876517in}{1.496359in}}{\pgfqpoint{0.882341in}{1.502183in}}%
\pgfpathcurveto{\pgfqpoint{0.888165in}{1.508007in}}{\pgfqpoint{0.891438in}{1.515907in}}{\pgfqpoint{0.891438in}{1.524143in}}%
\pgfpathcurveto{\pgfqpoint{0.891438in}{1.532379in}}{\pgfqpoint{0.888165in}{1.540279in}}{\pgfqpoint{0.882341in}{1.546103in}}%
\pgfpathcurveto{\pgfqpoint{0.876517in}{1.551927in}}{\pgfqpoint{0.868617in}{1.555200in}}{\pgfqpoint{0.860381in}{1.555200in}}%
\pgfpathcurveto{\pgfqpoint{0.852145in}{1.555200in}}{\pgfqpoint{0.844245in}{1.551927in}}{\pgfqpoint{0.838421in}{1.546103in}}%
\pgfpathcurveto{\pgfqpoint{0.832597in}{1.540279in}}{\pgfqpoint{0.829325in}{1.532379in}}{\pgfqpoint{0.829325in}{1.524143in}}%
\pgfpathcurveto{\pgfqpoint{0.829325in}{1.515907in}}{\pgfqpoint{0.832597in}{1.508007in}}{\pgfqpoint{0.838421in}{1.502183in}}%
\pgfpathcurveto{\pgfqpoint{0.844245in}{1.496359in}}{\pgfqpoint{0.852145in}{1.493087in}}{\pgfqpoint{0.860381in}{1.493087in}}%
\pgfpathclose%
\pgfusepath{stroke,fill}%
\end{pgfscope}%
\begin{pgfscope}%
\pgfpathrectangle{\pgfqpoint{0.100000in}{0.212622in}}{\pgfqpoint{3.696000in}{3.696000in}}%
\pgfusepath{clip}%
\pgfsetbuttcap%
\pgfsetroundjoin%
\definecolor{currentfill}{rgb}{0.121569,0.466667,0.705882}%
\pgfsetfillcolor{currentfill}%
\pgfsetfillopacity{0.627427}%
\pgfsetlinewidth{1.003750pt}%
\definecolor{currentstroke}{rgb}{0.121569,0.466667,0.705882}%
\pgfsetstrokecolor{currentstroke}%
\pgfsetstrokeopacity{0.627427}%
\pgfsetdash{}{0pt}%
\pgfpathmoveto{\pgfqpoint{0.860395in}{1.493099in}}%
\pgfpathcurveto{\pgfqpoint{0.868632in}{1.493099in}}{\pgfqpoint{0.876532in}{1.496372in}}{\pgfqpoint{0.882356in}{1.502196in}}%
\pgfpathcurveto{\pgfqpoint{0.888180in}{1.508020in}}{\pgfqpoint{0.891452in}{1.515920in}}{\pgfqpoint{0.891452in}{1.524156in}}%
\pgfpathcurveto{\pgfqpoint{0.891452in}{1.532392in}}{\pgfqpoint{0.888180in}{1.540292in}}{\pgfqpoint{0.882356in}{1.546116in}}%
\pgfpathcurveto{\pgfqpoint{0.876532in}{1.551940in}}{\pgfqpoint{0.868632in}{1.555212in}}{\pgfqpoint{0.860395in}{1.555212in}}%
\pgfpathcurveto{\pgfqpoint{0.852159in}{1.555212in}}{\pgfqpoint{0.844259in}{1.551940in}}{\pgfqpoint{0.838435in}{1.546116in}}%
\pgfpathcurveto{\pgfqpoint{0.832611in}{1.540292in}}{\pgfqpoint{0.829339in}{1.532392in}}{\pgfqpoint{0.829339in}{1.524156in}}%
\pgfpathcurveto{\pgfqpoint{0.829339in}{1.515920in}}{\pgfqpoint{0.832611in}{1.508020in}}{\pgfqpoint{0.838435in}{1.502196in}}%
\pgfpathcurveto{\pgfqpoint{0.844259in}{1.496372in}}{\pgfqpoint{0.852159in}{1.493099in}}{\pgfqpoint{0.860395in}{1.493099in}}%
\pgfpathclose%
\pgfusepath{stroke,fill}%
\end{pgfscope}%
\begin{pgfscope}%
\pgfpathrectangle{\pgfqpoint{0.100000in}{0.212622in}}{\pgfqpoint{3.696000in}{3.696000in}}%
\pgfusepath{clip}%
\pgfsetbuttcap%
\pgfsetroundjoin%
\definecolor{currentfill}{rgb}{0.121569,0.466667,0.705882}%
\pgfsetfillcolor{currentfill}%
\pgfsetfillopacity{0.627430}%
\pgfsetlinewidth{1.003750pt}%
\definecolor{currentstroke}{rgb}{0.121569,0.466667,0.705882}%
\pgfsetstrokecolor{currentstroke}%
\pgfsetstrokeopacity{0.627430}%
\pgfsetdash{}{0pt}%
\pgfpathmoveto{\pgfqpoint{0.860404in}{1.493107in}}%
\pgfpathcurveto{\pgfqpoint{0.868640in}{1.493107in}}{\pgfqpoint{0.876540in}{1.496379in}}{\pgfqpoint{0.882364in}{1.502203in}}%
\pgfpathcurveto{\pgfqpoint{0.888188in}{1.508027in}}{\pgfqpoint{0.891460in}{1.515927in}}{\pgfqpoint{0.891460in}{1.524163in}}%
\pgfpathcurveto{\pgfqpoint{0.891460in}{1.532399in}}{\pgfqpoint{0.888188in}{1.540299in}}{\pgfqpoint{0.882364in}{1.546123in}}%
\pgfpathcurveto{\pgfqpoint{0.876540in}{1.551947in}}{\pgfqpoint{0.868640in}{1.555220in}}{\pgfqpoint{0.860404in}{1.555220in}}%
\pgfpathcurveto{\pgfqpoint{0.852168in}{1.555220in}}{\pgfqpoint{0.844268in}{1.551947in}}{\pgfqpoint{0.838444in}{1.546123in}}%
\pgfpathcurveto{\pgfqpoint{0.832620in}{1.540299in}}{\pgfqpoint{0.829347in}{1.532399in}}{\pgfqpoint{0.829347in}{1.524163in}}%
\pgfpathcurveto{\pgfqpoint{0.829347in}{1.515927in}}{\pgfqpoint{0.832620in}{1.508027in}}{\pgfqpoint{0.838444in}{1.502203in}}%
\pgfpathcurveto{\pgfqpoint{0.844268in}{1.496379in}}{\pgfqpoint{0.852168in}{1.493107in}}{\pgfqpoint{0.860404in}{1.493107in}}%
\pgfpathclose%
\pgfusepath{stroke,fill}%
\end{pgfscope}%
\begin{pgfscope}%
\pgfpathrectangle{\pgfqpoint{0.100000in}{0.212622in}}{\pgfqpoint{3.696000in}{3.696000in}}%
\pgfusepath{clip}%
\pgfsetbuttcap%
\pgfsetroundjoin%
\definecolor{currentfill}{rgb}{0.121569,0.466667,0.705882}%
\pgfsetfillcolor{currentfill}%
\pgfsetfillopacity{0.627432}%
\pgfsetlinewidth{1.003750pt}%
\definecolor{currentstroke}{rgb}{0.121569,0.466667,0.705882}%
\pgfsetstrokecolor{currentstroke}%
\pgfsetstrokeopacity{0.627432}%
\pgfsetdash{}{0pt}%
\pgfpathmoveto{\pgfqpoint{0.860409in}{1.493111in}}%
\pgfpathcurveto{\pgfqpoint{0.868645in}{1.493111in}}{\pgfqpoint{0.876545in}{1.496383in}}{\pgfqpoint{0.882369in}{1.502207in}}%
\pgfpathcurveto{\pgfqpoint{0.888193in}{1.508031in}}{\pgfqpoint{0.891465in}{1.515931in}}{\pgfqpoint{0.891465in}{1.524168in}}%
\pgfpathcurveto{\pgfqpoint{0.891465in}{1.532404in}}{\pgfqpoint{0.888193in}{1.540304in}}{\pgfqpoint{0.882369in}{1.546128in}}%
\pgfpathcurveto{\pgfqpoint{0.876545in}{1.551952in}}{\pgfqpoint{0.868645in}{1.555224in}}{\pgfqpoint{0.860409in}{1.555224in}}%
\pgfpathcurveto{\pgfqpoint{0.852172in}{1.555224in}}{\pgfqpoint{0.844272in}{1.551952in}}{\pgfqpoint{0.838448in}{1.546128in}}%
\pgfpathcurveto{\pgfqpoint{0.832624in}{1.540304in}}{\pgfqpoint{0.829352in}{1.532404in}}{\pgfqpoint{0.829352in}{1.524168in}}%
\pgfpathcurveto{\pgfqpoint{0.829352in}{1.515931in}}{\pgfqpoint{0.832624in}{1.508031in}}{\pgfqpoint{0.838448in}{1.502207in}}%
\pgfpathcurveto{\pgfqpoint{0.844272in}{1.496383in}}{\pgfqpoint{0.852172in}{1.493111in}}{\pgfqpoint{0.860409in}{1.493111in}}%
\pgfpathclose%
\pgfusepath{stroke,fill}%
\end{pgfscope}%
\begin{pgfscope}%
\pgfpathrectangle{\pgfqpoint{0.100000in}{0.212622in}}{\pgfqpoint{3.696000in}{3.696000in}}%
\pgfusepath{clip}%
\pgfsetbuttcap%
\pgfsetroundjoin%
\definecolor{currentfill}{rgb}{0.121569,0.466667,0.705882}%
\pgfsetfillcolor{currentfill}%
\pgfsetfillopacity{0.627433}%
\pgfsetlinewidth{1.003750pt}%
\definecolor{currentstroke}{rgb}{0.121569,0.466667,0.705882}%
\pgfsetstrokecolor{currentstroke}%
\pgfsetstrokeopacity{0.627433}%
\pgfsetdash{}{0pt}%
\pgfpathmoveto{\pgfqpoint{0.860411in}{1.493114in}}%
\pgfpathcurveto{\pgfqpoint{0.868648in}{1.493114in}}{\pgfqpoint{0.876548in}{1.496386in}}{\pgfqpoint{0.882372in}{1.502210in}}%
\pgfpathcurveto{\pgfqpoint{0.888196in}{1.508034in}}{\pgfqpoint{0.891468in}{1.515934in}}{\pgfqpoint{0.891468in}{1.524170in}}%
\pgfpathcurveto{\pgfqpoint{0.891468in}{1.532406in}}{\pgfqpoint{0.888196in}{1.540306in}}{\pgfqpoint{0.882372in}{1.546130in}}%
\pgfpathcurveto{\pgfqpoint{0.876548in}{1.551954in}}{\pgfqpoint{0.868648in}{1.555227in}}{\pgfqpoint{0.860411in}{1.555227in}}%
\pgfpathcurveto{\pgfqpoint{0.852175in}{1.555227in}}{\pgfqpoint{0.844275in}{1.551954in}}{\pgfqpoint{0.838451in}{1.546130in}}%
\pgfpathcurveto{\pgfqpoint{0.832627in}{1.540306in}}{\pgfqpoint{0.829355in}{1.532406in}}{\pgfqpoint{0.829355in}{1.524170in}}%
\pgfpathcurveto{\pgfqpoint{0.829355in}{1.515934in}}{\pgfqpoint{0.832627in}{1.508034in}}{\pgfqpoint{0.838451in}{1.502210in}}%
\pgfpathcurveto{\pgfqpoint{0.844275in}{1.496386in}}{\pgfqpoint{0.852175in}{1.493114in}}{\pgfqpoint{0.860411in}{1.493114in}}%
\pgfpathclose%
\pgfusepath{stroke,fill}%
\end{pgfscope}%
\begin{pgfscope}%
\pgfpathrectangle{\pgfqpoint{0.100000in}{0.212622in}}{\pgfqpoint{3.696000in}{3.696000in}}%
\pgfusepath{clip}%
\pgfsetbuttcap%
\pgfsetroundjoin%
\definecolor{currentfill}{rgb}{0.121569,0.466667,0.705882}%
\pgfsetfillcolor{currentfill}%
\pgfsetfillopacity{0.627433}%
\pgfsetlinewidth{1.003750pt}%
\definecolor{currentstroke}{rgb}{0.121569,0.466667,0.705882}%
\pgfsetstrokecolor{currentstroke}%
\pgfsetstrokeopacity{0.627433}%
\pgfsetdash{}{0pt}%
\pgfpathmoveto{\pgfqpoint{0.860413in}{1.493115in}}%
\pgfpathcurveto{\pgfqpoint{0.868649in}{1.493115in}}{\pgfqpoint{0.876549in}{1.496387in}}{\pgfqpoint{0.882373in}{1.502211in}}%
\pgfpathcurveto{\pgfqpoint{0.888197in}{1.508035in}}{\pgfqpoint{0.891469in}{1.515935in}}{\pgfqpoint{0.891469in}{1.524172in}}%
\pgfpathcurveto{\pgfqpoint{0.891469in}{1.532408in}}{\pgfqpoint{0.888197in}{1.540308in}}{\pgfqpoint{0.882373in}{1.546132in}}%
\pgfpathcurveto{\pgfqpoint{0.876549in}{1.551956in}}{\pgfqpoint{0.868649in}{1.555228in}}{\pgfqpoint{0.860413in}{1.555228in}}%
\pgfpathcurveto{\pgfqpoint{0.852177in}{1.555228in}}{\pgfqpoint{0.844277in}{1.551956in}}{\pgfqpoint{0.838453in}{1.546132in}}%
\pgfpathcurveto{\pgfqpoint{0.832629in}{1.540308in}}{\pgfqpoint{0.829356in}{1.532408in}}{\pgfqpoint{0.829356in}{1.524172in}}%
\pgfpathcurveto{\pgfqpoint{0.829356in}{1.515935in}}{\pgfqpoint{0.832629in}{1.508035in}}{\pgfqpoint{0.838453in}{1.502211in}}%
\pgfpathcurveto{\pgfqpoint{0.844277in}{1.496387in}}{\pgfqpoint{0.852177in}{1.493115in}}{\pgfqpoint{0.860413in}{1.493115in}}%
\pgfpathclose%
\pgfusepath{stroke,fill}%
\end{pgfscope}%
\begin{pgfscope}%
\pgfpathrectangle{\pgfqpoint{0.100000in}{0.212622in}}{\pgfqpoint{3.696000in}{3.696000in}}%
\pgfusepath{clip}%
\pgfsetbuttcap%
\pgfsetroundjoin%
\definecolor{currentfill}{rgb}{0.121569,0.466667,0.705882}%
\pgfsetfillcolor{currentfill}%
\pgfsetfillopacity{0.627433}%
\pgfsetlinewidth{1.003750pt}%
\definecolor{currentstroke}{rgb}{0.121569,0.466667,0.705882}%
\pgfsetstrokecolor{currentstroke}%
\pgfsetstrokeopacity{0.627433}%
\pgfsetdash{}{0pt}%
\pgfpathmoveto{\pgfqpoint{0.860414in}{1.493116in}}%
\pgfpathcurveto{\pgfqpoint{0.868650in}{1.493116in}}{\pgfqpoint{0.876550in}{1.496388in}}{\pgfqpoint{0.882374in}{1.502212in}}%
\pgfpathcurveto{\pgfqpoint{0.888198in}{1.508036in}}{\pgfqpoint{0.891470in}{1.515936in}}{\pgfqpoint{0.891470in}{1.524172in}}%
\pgfpathcurveto{\pgfqpoint{0.891470in}{1.532409in}}{\pgfqpoint{0.888198in}{1.540309in}}{\pgfqpoint{0.882374in}{1.546133in}}%
\pgfpathcurveto{\pgfqpoint{0.876550in}{1.551957in}}{\pgfqpoint{0.868650in}{1.555229in}}{\pgfqpoint{0.860414in}{1.555229in}}%
\pgfpathcurveto{\pgfqpoint{0.852177in}{1.555229in}}{\pgfqpoint{0.844277in}{1.551957in}}{\pgfqpoint{0.838453in}{1.546133in}}%
\pgfpathcurveto{\pgfqpoint{0.832630in}{1.540309in}}{\pgfqpoint{0.829357in}{1.532409in}}{\pgfqpoint{0.829357in}{1.524172in}}%
\pgfpathcurveto{\pgfqpoint{0.829357in}{1.515936in}}{\pgfqpoint{0.832630in}{1.508036in}}{\pgfqpoint{0.838453in}{1.502212in}}%
\pgfpathcurveto{\pgfqpoint{0.844277in}{1.496388in}}{\pgfqpoint{0.852177in}{1.493116in}}{\pgfqpoint{0.860414in}{1.493116in}}%
\pgfpathclose%
\pgfusepath{stroke,fill}%
\end{pgfscope}%
\begin{pgfscope}%
\pgfpathrectangle{\pgfqpoint{0.100000in}{0.212622in}}{\pgfqpoint{3.696000in}{3.696000in}}%
\pgfusepath{clip}%
\pgfsetbuttcap%
\pgfsetroundjoin%
\definecolor{currentfill}{rgb}{0.121569,0.466667,0.705882}%
\pgfsetfillcolor{currentfill}%
\pgfsetfillopacity{0.627433}%
\pgfsetlinewidth{1.003750pt}%
\definecolor{currentstroke}{rgb}{0.121569,0.466667,0.705882}%
\pgfsetstrokecolor{currentstroke}%
\pgfsetstrokeopacity{0.627433}%
\pgfsetdash{}{0pt}%
\pgfpathmoveto{\pgfqpoint{0.860414in}{1.493116in}}%
\pgfpathcurveto{\pgfqpoint{0.868650in}{1.493116in}}{\pgfqpoint{0.876551in}{1.496389in}}{\pgfqpoint{0.882374in}{1.502213in}}%
\pgfpathcurveto{\pgfqpoint{0.888198in}{1.508036in}}{\pgfqpoint{0.891471in}{1.515937in}}{\pgfqpoint{0.891471in}{1.524173in}}%
\pgfpathcurveto{\pgfqpoint{0.891471in}{1.532409in}}{\pgfqpoint{0.888198in}{1.540309in}}{\pgfqpoint{0.882374in}{1.546133in}}%
\pgfpathcurveto{\pgfqpoint{0.876551in}{1.551957in}}{\pgfqpoint{0.868650in}{1.555229in}}{\pgfqpoint{0.860414in}{1.555229in}}%
\pgfpathcurveto{\pgfqpoint{0.852178in}{1.555229in}}{\pgfqpoint{0.844278in}{1.551957in}}{\pgfqpoint{0.838454in}{1.546133in}}%
\pgfpathcurveto{\pgfqpoint{0.832630in}{1.540309in}}{\pgfqpoint{0.829358in}{1.532409in}}{\pgfqpoint{0.829358in}{1.524173in}}%
\pgfpathcurveto{\pgfqpoint{0.829358in}{1.515937in}}{\pgfqpoint{0.832630in}{1.508036in}}{\pgfqpoint{0.838454in}{1.502213in}}%
\pgfpathcurveto{\pgfqpoint{0.844278in}{1.496389in}}{\pgfqpoint{0.852178in}{1.493116in}}{\pgfqpoint{0.860414in}{1.493116in}}%
\pgfpathclose%
\pgfusepath{stroke,fill}%
\end{pgfscope}%
\begin{pgfscope}%
\pgfpathrectangle{\pgfqpoint{0.100000in}{0.212622in}}{\pgfqpoint{3.696000in}{3.696000in}}%
\pgfusepath{clip}%
\pgfsetbuttcap%
\pgfsetroundjoin%
\definecolor{currentfill}{rgb}{0.121569,0.466667,0.705882}%
\pgfsetfillcolor{currentfill}%
\pgfsetfillopacity{0.627433}%
\pgfsetlinewidth{1.003750pt}%
\definecolor{currentstroke}{rgb}{0.121569,0.466667,0.705882}%
\pgfsetstrokecolor{currentstroke}%
\pgfsetstrokeopacity{0.627433}%
\pgfsetdash{}{0pt}%
\pgfpathmoveto{\pgfqpoint{0.860414in}{1.493117in}}%
\pgfpathcurveto{\pgfqpoint{0.868651in}{1.493117in}}{\pgfqpoint{0.876551in}{1.496389in}}{\pgfqpoint{0.882375in}{1.502213in}}%
\pgfpathcurveto{\pgfqpoint{0.888199in}{1.508037in}}{\pgfqpoint{0.891471in}{1.515937in}}{\pgfqpoint{0.891471in}{1.524173in}}%
\pgfpathcurveto{\pgfqpoint{0.891471in}{1.532409in}}{\pgfqpoint{0.888199in}{1.540309in}}{\pgfqpoint{0.882375in}{1.546133in}}%
\pgfpathcurveto{\pgfqpoint{0.876551in}{1.551957in}}{\pgfqpoint{0.868651in}{1.555230in}}{\pgfqpoint{0.860414in}{1.555230in}}%
\pgfpathcurveto{\pgfqpoint{0.852178in}{1.555230in}}{\pgfqpoint{0.844278in}{1.551957in}}{\pgfqpoint{0.838454in}{1.546133in}}%
\pgfpathcurveto{\pgfqpoint{0.832630in}{1.540309in}}{\pgfqpoint{0.829358in}{1.532409in}}{\pgfqpoint{0.829358in}{1.524173in}}%
\pgfpathcurveto{\pgfqpoint{0.829358in}{1.515937in}}{\pgfqpoint{0.832630in}{1.508037in}}{\pgfqpoint{0.838454in}{1.502213in}}%
\pgfpathcurveto{\pgfqpoint{0.844278in}{1.496389in}}{\pgfqpoint{0.852178in}{1.493117in}}{\pgfqpoint{0.860414in}{1.493117in}}%
\pgfpathclose%
\pgfusepath{stroke,fill}%
\end{pgfscope}%
\begin{pgfscope}%
\pgfpathrectangle{\pgfqpoint{0.100000in}{0.212622in}}{\pgfqpoint{3.696000in}{3.696000in}}%
\pgfusepath{clip}%
\pgfsetbuttcap%
\pgfsetroundjoin%
\definecolor{currentfill}{rgb}{0.121569,0.466667,0.705882}%
\pgfsetfillcolor{currentfill}%
\pgfsetfillopacity{0.627433}%
\pgfsetlinewidth{1.003750pt}%
\definecolor{currentstroke}{rgb}{0.121569,0.466667,0.705882}%
\pgfsetstrokecolor{currentstroke}%
\pgfsetstrokeopacity{0.627433}%
\pgfsetdash{}{0pt}%
\pgfpathmoveto{\pgfqpoint{0.860415in}{1.493117in}}%
\pgfpathcurveto{\pgfqpoint{0.868651in}{1.493117in}}{\pgfqpoint{0.876551in}{1.496389in}}{\pgfqpoint{0.882375in}{1.502213in}}%
\pgfpathcurveto{\pgfqpoint{0.888199in}{1.508037in}}{\pgfqpoint{0.891471in}{1.515937in}}{\pgfqpoint{0.891471in}{1.524173in}}%
\pgfpathcurveto{\pgfqpoint{0.891471in}{1.532410in}}{\pgfqpoint{0.888199in}{1.540310in}}{\pgfqpoint{0.882375in}{1.546134in}}%
\pgfpathcurveto{\pgfqpoint{0.876551in}{1.551957in}}{\pgfqpoint{0.868651in}{1.555230in}}{\pgfqpoint{0.860415in}{1.555230in}}%
\pgfpathcurveto{\pgfqpoint{0.852178in}{1.555230in}}{\pgfqpoint{0.844278in}{1.551957in}}{\pgfqpoint{0.838454in}{1.546134in}}%
\pgfpathcurveto{\pgfqpoint{0.832630in}{1.540310in}}{\pgfqpoint{0.829358in}{1.532410in}}{\pgfqpoint{0.829358in}{1.524173in}}%
\pgfpathcurveto{\pgfqpoint{0.829358in}{1.515937in}}{\pgfqpoint{0.832630in}{1.508037in}}{\pgfqpoint{0.838454in}{1.502213in}}%
\pgfpathcurveto{\pgfqpoint{0.844278in}{1.496389in}}{\pgfqpoint{0.852178in}{1.493117in}}{\pgfqpoint{0.860415in}{1.493117in}}%
\pgfpathclose%
\pgfusepath{stroke,fill}%
\end{pgfscope}%
\begin{pgfscope}%
\pgfpathrectangle{\pgfqpoint{0.100000in}{0.212622in}}{\pgfqpoint{3.696000in}{3.696000in}}%
\pgfusepath{clip}%
\pgfsetbuttcap%
\pgfsetroundjoin%
\definecolor{currentfill}{rgb}{0.121569,0.466667,0.705882}%
\pgfsetfillcolor{currentfill}%
\pgfsetfillopacity{0.627433}%
\pgfsetlinewidth{1.003750pt}%
\definecolor{currentstroke}{rgb}{0.121569,0.466667,0.705882}%
\pgfsetstrokecolor{currentstroke}%
\pgfsetstrokeopacity{0.627433}%
\pgfsetdash{}{0pt}%
\pgfpathmoveto{\pgfqpoint{0.860415in}{1.493117in}}%
\pgfpathcurveto{\pgfqpoint{0.868651in}{1.493117in}}{\pgfqpoint{0.876551in}{1.496389in}}{\pgfqpoint{0.882375in}{1.502213in}}%
\pgfpathcurveto{\pgfqpoint{0.888199in}{1.508037in}}{\pgfqpoint{0.891471in}{1.515937in}}{\pgfqpoint{0.891471in}{1.524173in}}%
\pgfpathcurveto{\pgfqpoint{0.891471in}{1.532410in}}{\pgfqpoint{0.888199in}{1.540310in}}{\pgfqpoint{0.882375in}{1.546134in}}%
\pgfpathcurveto{\pgfqpoint{0.876551in}{1.551958in}}{\pgfqpoint{0.868651in}{1.555230in}}{\pgfqpoint{0.860415in}{1.555230in}}%
\pgfpathcurveto{\pgfqpoint{0.852178in}{1.555230in}}{\pgfqpoint{0.844278in}{1.551958in}}{\pgfqpoint{0.838454in}{1.546134in}}%
\pgfpathcurveto{\pgfqpoint{0.832631in}{1.540310in}}{\pgfqpoint{0.829358in}{1.532410in}}{\pgfqpoint{0.829358in}{1.524173in}}%
\pgfpathcurveto{\pgfqpoint{0.829358in}{1.515937in}}{\pgfqpoint{0.832631in}{1.508037in}}{\pgfqpoint{0.838454in}{1.502213in}}%
\pgfpathcurveto{\pgfqpoint{0.844278in}{1.496389in}}{\pgfqpoint{0.852178in}{1.493117in}}{\pgfqpoint{0.860415in}{1.493117in}}%
\pgfpathclose%
\pgfusepath{stroke,fill}%
\end{pgfscope}%
\begin{pgfscope}%
\pgfpathrectangle{\pgfqpoint{0.100000in}{0.212622in}}{\pgfqpoint{3.696000in}{3.696000in}}%
\pgfusepath{clip}%
\pgfsetbuttcap%
\pgfsetroundjoin%
\definecolor{currentfill}{rgb}{0.121569,0.466667,0.705882}%
\pgfsetfillcolor{currentfill}%
\pgfsetfillopacity{0.627433}%
\pgfsetlinewidth{1.003750pt}%
\definecolor{currentstroke}{rgb}{0.121569,0.466667,0.705882}%
\pgfsetstrokecolor{currentstroke}%
\pgfsetstrokeopacity{0.627433}%
\pgfsetdash{}{0pt}%
\pgfpathmoveto{\pgfqpoint{0.860415in}{1.493117in}}%
\pgfpathcurveto{\pgfqpoint{0.868651in}{1.493117in}}{\pgfqpoint{0.876551in}{1.496389in}}{\pgfqpoint{0.882375in}{1.502213in}}%
\pgfpathcurveto{\pgfqpoint{0.888199in}{1.508037in}}{\pgfqpoint{0.891471in}{1.515937in}}{\pgfqpoint{0.891471in}{1.524173in}}%
\pgfpathcurveto{\pgfqpoint{0.891471in}{1.532410in}}{\pgfqpoint{0.888199in}{1.540310in}}{\pgfqpoint{0.882375in}{1.546134in}}%
\pgfpathcurveto{\pgfqpoint{0.876551in}{1.551958in}}{\pgfqpoint{0.868651in}{1.555230in}}{\pgfqpoint{0.860415in}{1.555230in}}%
\pgfpathcurveto{\pgfqpoint{0.852178in}{1.555230in}}{\pgfqpoint{0.844278in}{1.551958in}}{\pgfqpoint{0.838454in}{1.546134in}}%
\pgfpathcurveto{\pgfqpoint{0.832631in}{1.540310in}}{\pgfqpoint{0.829358in}{1.532410in}}{\pgfqpoint{0.829358in}{1.524173in}}%
\pgfpathcurveto{\pgfqpoint{0.829358in}{1.515937in}}{\pgfqpoint{0.832631in}{1.508037in}}{\pgfqpoint{0.838454in}{1.502213in}}%
\pgfpathcurveto{\pgfqpoint{0.844278in}{1.496389in}}{\pgfqpoint{0.852178in}{1.493117in}}{\pgfqpoint{0.860415in}{1.493117in}}%
\pgfpathclose%
\pgfusepath{stroke,fill}%
\end{pgfscope}%
\begin{pgfscope}%
\pgfpathrectangle{\pgfqpoint{0.100000in}{0.212622in}}{\pgfqpoint{3.696000in}{3.696000in}}%
\pgfusepath{clip}%
\pgfsetbuttcap%
\pgfsetroundjoin%
\definecolor{currentfill}{rgb}{0.121569,0.466667,0.705882}%
\pgfsetfillcolor{currentfill}%
\pgfsetfillopacity{0.627433}%
\pgfsetlinewidth{1.003750pt}%
\definecolor{currentstroke}{rgb}{0.121569,0.466667,0.705882}%
\pgfsetstrokecolor{currentstroke}%
\pgfsetstrokeopacity{0.627433}%
\pgfsetdash{}{0pt}%
\pgfpathmoveto{\pgfqpoint{0.860415in}{1.493117in}}%
\pgfpathcurveto{\pgfqpoint{0.868651in}{1.493117in}}{\pgfqpoint{0.876551in}{1.496389in}}{\pgfqpoint{0.882375in}{1.502213in}}%
\pgfpathcurveto{\pgfqpoint{0.888199in}{1.508037in}}{\pgfqpoint{0.891471in}{1.515937in}}{\pgfqpoint{0.891471in}{1.524173in}}%
\pgfpathcurveto{\pgfqpoint{0.891471in}{1.532410in}}{\pgfqpoint{0.888199in}{1.540310in}}{\pgfqpoint{0.882375in}{1.546134in}}%
\pgfpathcurveto{\pgfqpoint{0.876551in}{1.551958in}}{\pgfqpoint{0.868651in}{1.555230in}}{\pgfqpoint{0.860415in}{1.555230in}}%
\pgfpathcurveto{\pgfqpoint{0.852178in}{1.555230in}}{\pgfqpoint{0.844278in}{1.551958in}}{\pgfqpoint{0.838454in}{1.546134in}}%
\pgfpathcurveto{\pgfqpoint{0.832631in}{1.540310in}}{\pgfqpoint{0.829358in}{1.532410in}}{\pgfqpoint{0.829358in}{1.524173in}}%
\pgfpathcurveto{\pgfqpoint{0.829358in}{1.515937in}}{\pgfqpoint{0.832631in}{1.508037in}}{\pgfqpoint{0.838454in}{1.502213in}}%
\pgfpathcurveto{\pgfqpoint{0.844278in}{1.496389in}}{\pgfqpoint{0.852178in}{1.493117in}}{\pgfqpoint{0.860415in}{1.493117in}}%
\pgfpathclose%
\pgfusepath{stroke,fill}%
\end{pgfscope}%
\begin{pgfscope}%
\pgfpathrectangle{\pgfqpoint{0.100000in}{0.212622in}}{\pgfqpoint{3.696000in}{3.696000in}}%
\pgfusepath{clip}%
\pgfsetbuttcap%
\pgfsetroundjoin%
\definecolor{currentfill}{rgb}{0.121569,0.466667,0.705882}%
\pgfsetfillcolor{currentfill}%
\pgfsetfillopacity{0.627433}%
\pgfsetlinewidth{1.003750pt}%
\definecolor{currentstroke}{rgb}{0.121569,0.466667,0.705882}%
\pgfsetstrokecolor{currentstroke}%
\pgfsetstrokeopacity{0.627433}%
\pgfsetdash{}{0pt}%
\pgfpathmoveto{\pgfqpoint{0.860415in}{1.493117in}}%
\pgfpathcurveto{\pgfqpoint{0.868651in}{1.493117in}}{\pgfqpoint{0.876551in}{1.496389in}}{\pgfqpoint{0.882375in}{1.502213in}}%
\pgfpathcurveto{\pgfqpoint{0.888199in}{1.508037in}}{\pgfqpoint{0.891471in}{1.515937in}}{\pgfqpoint{0.891471in}{1.524173in}}%
\pgfpathcurveto{\pgfqpoint{0.891471in}{1.532410in}}{\pgfqpoint{0.888199in}{1.540310in}}{\pgfqpoint{0.882375in}{1.546134in}}%
\pgfpathcurveto{\pgfqpoint{0.876551in}{1.551958in}}{\pgfqpoint{0.868651in}{1.555230in}}{\pgfqpoint{0.860415in}{1.555230in}}%
\pgfpathcurveto{\pgfqpoint{0.852178in}{1.555230in}}{\pgfqpoint{0.844278in}{1.551958in}}{\pgfqpoint{0.838454in}{1.546134in}}%
\pgfpathcurveto{\pgfqpoint{0.832631in}{1.540310in}}{\pgfqpoint{0.829358in}{1.532410in}}{\pgfqpoint{0.829358in}{1.524173in}}%
\pgfpathcurveto{\pgfqpoint{0.829358in}{1.515937in}}{\pgfqpoint{0.832631in}{1.508037in}}{\pgfqpoint{0.838454in}{1.502213in}}%
\pgfpathcurveto{\pgfqpoint{0.844278in}{1.496389in}}{\pgfqpoint{0.852178in}{1.493117in}}{\pgfqpoint{0.860415in}{1.493117in}}%
\pgfpathclose%
\pgfusepath{stroke,fill}%
\end{pgfscope}%
\begin{pgfscope}%
\pgfpathrectangle{\pgfqpoint{0.100000in}{0.212622in}}{\pgfqpoint{3.696000in}{3.696000in}}%
\pgfusepath{clip}%
\pgfsetbuttcap%
\pgfsetroundjoin%
\definecolor{currentfill}{rgb}{0.121569,0.466667,0.705882}%
\pgfsetfillcolor{currentfill}%
\pgfsetfillopacity{0.627433}%
\pgfsetlinewidth{1.003750pt}%
\definecolor{currentstroke}{rgb}{0.121569,0.466667,0.705882}%
\pgfsetstrokecolor{currentstroke}%
\pgfsetstrokeopacity{0.627433}%
\pgfsetdash{}{0pt}%
\pgfpathmoveto{\pgfqpoint{0.860415in}{1.493117in}}%
\pgfpathcurveto{\pgfqpoint{0.868651in}{1.493117in}}{\pgfqpoint{0.876551in}{1.496389in}}{\pgfqpoint{0.882375in}{1.502213in}}%
\pgfpathcurveto{\pgfqpoint{0.888199in}{1.508037in}}{\pgfqpoint{0.891471in}{1.515937in}}{\pgfqpoint{0.891471in}{1.524173in}}%
\pgfpathcurveto{\pgfqpoint{0.891471in}{1.532410in}}{\pgfqpoint{0.888199in}{1.540310in}}{\pgfqpoint{0.882375in}{1.546134in}}%
\pgfpathcurveto{\pgfqpoint{0.876551in}{1.551958in}}{\pgfqpoint{0.868651in}{1.555230in}}{\pgfqpoint{0.860415in}{1.555230in}}%
\pgfpathcurveto{\pgfqpoint{0.852178in}{1.555230in}}{\pgfqpoint{0.844278in}{1.551958in}}{\pgfqpoint{0.838454in}{1.546134in}}%
\pgfpathcurveto{\pgfqpoint{0.832631in}{1.540310in}}{\pgfqpoint{0.829358in}{1.532410in}}{\pgfqpoint{0.829358in}{1.524173in}}%
\pgfpathcurveto{\pgfqpoint{0.829358in}{1.515937in}}{\pgfqpoint{0.832631in}{1.508037in}}{\pgfqpoint{0.838454in}{1.502213in}}%
\pgfpathcurveto{\pgfqpoint{0.844278in}{1.496389in}}{\pgfqpoint{0.852178in}{1.493117in}}{\pgfqpoint{0.860415in}{1.493117in}}%
\pgfpathclose%
\pgfusepath{stroke,fill}%
\end{pgfscope}%
\begin{pgfscope}%
\pgfpathrectangle{\pgfqpoint{0.100000in}{0.212622in}}{\pgfqpoint{3.696000in}{3.696000in}}%
\pgfusepath{clip}%
\pgfsetbuttcap%
\pgfsetroundjoin%
\definecolor{currentfill}{rgb}{0.121569,0.466667,0.705882}%
\pgfsetfillcolor{currentfill}%
\pgfsetfillopacity{0.627433}%
\pgfsetlinewidth{1.003750pt}%
\definecolor{currentstroke}{rgb}{0.121569,0.466667,0.705882}%
\pgfsetstrokecolor{currentstroke}%
\pgfsetstrokeopacity{0.627433}%
\pgfsetdash{}{0pt}%
\pgfpathmoveto{\pgfqpoint{0.860415in}{1.493117in}}%
\pgfpathcurveto{\pgfqpoint{0.868651in}{1.493117in}}{\pgfqpoint{0.876551in}{1.496389in}}{\pgfqpoint{0.882375in}{1.502213in}}%
\pgfpathcurveto{\pgfqpoint{0.888199in}{1.508037in}}{\pgfqpoint{0.891471in}{1.515937in}}{\pgfqpoint{0.891471in}{1.524173in}}%
\pgfpathcurveto{\pgfqpoint{0.891471in}{1.532410in}}{\pgfqpoint{0.888199in}{1.540310in}}{\pgfqpoint{0.882375in}{1.546134in}}%
\pgfpathcurveto{\pgfqpoint{0.876551in}{1.551958in}}{\pgfqpoint{0.868651in}{1.555230in}}{\pgfqpoint{0.860415in}{1.555230in}}%
\pgfpathcurveto{\pgfqpoint{0.852178in}{1.555230in}}{\pgfqpoint{0.844278in}{1.551958in}}{\pgfqpoint{0.838454in}{1.546134in}}%
\pgfpathcurveto{\pgfqpoint{0.832631in}{1.540310in}}{\pgfqpoint{0.829358in}{1.532410in}}{\pgfqpoint{0.829358in}{1.524173in}}%
\pgfpathcurveto{\pgfqpoint{0.829358in}{1.515937in}}{\pgfqpoint{0.832631in}{1.508037in}}{\pgfqpoint{0.838454in}{1.502213in}}%
\pgfpathcurveto{\pgfqpoint{0.844278in}{1.496389in}}{\pgfqpoint{0.852178in}{1.493117in}}{\pgfqpoint{0.860415in}{1.493117in}}%
\pgfpathclose%
\pgfusepath{stroke,fill}%
\end{pgfscope}%
\begin{pgfscope}%
\pgfpathrectangle{\pgfqpoint{0.100000in}{0.212622in}}{\pgfqpoint{3.696000in}{3.696000in}}%
\pgfusepath{clip}%
\pgfsetbuttcap%
\pgfsetroundjoin%
\definecolor{currentfill}{rgb}{0.121569,0.466667,0.705882}%
\pgfsetfillcolor{currentfill}%
\pgfsetfillopacity{0.627433}%
\pgfsetlinewidth{1.003750pt}%
\definecolor{currentstroke}{rgb}{0.121569,0.466667,0.705882}%
\pgfsetstrokecolor{currentstroke}%
\pgfsetstrokeopacity{0.627433}%
\pgfsetdash{}{0pt}%
\pgfpathmoveto{\pgfqpoint{0.860415in}{1.493117in}}%
\pgfpathcurveto{\pgfqpoint{0.868651in}{1.493117in}}{\pgfqpoint{0.876551in}{1.496389in}}{\pgfqpoint{0.882375in}{1.502213in}}%
\pgfpathcurveto{\pgfqpoint{0.888199in}{1.508037in}}{\pgfqpoint{0.891471in}{1.515937in}}{\pgfqpoint{0.891471in}{1.524173in}}%
\pgfpathcurveto{\pgfqpoint{0.891471in}{1.532410in}}{\pgfqpoint{0.888199in}{1.540310in}}{\pgfqpoint{0.882375in}{1.546134in}}%
\pgfpathcurveto{\pgfqpoint{0.876551in}{1.551958in}}{\pgfqpoint{0.868651in}{1.555230in}}{\pgfqpoint{0.860415in}{1.555230in}}%
\pgfpathcurveto{\pgfqpoint{0.852178in}{1.555230in}}{\pgfqpoint{0.844278in}{1.551958in}}{\pgfqpoint{0.838454in}{1.546134in}}%
\pgfpathcurveto{\pgfqpoint{0.832631in}{1.540310in}}{\pgfqpoint{0.829358in}{1.532410in}}{\pgfqpoint{0.829358in}{1.524173in}}%
\pgfpathcurveto{\pgfqpoint{0.829358in}{1.515937in}}{\pgfqpoint{0.832631in}{1.508037in}}{\pgfqpoint{0.838454in}{1.502213in}}%
\pgfpathcurveto{\pgfqpoint{0.844278in}{1.496389in}}{\pgfqpoint{0.852178in}{1.493117in}}{\pgfqpoint{0.860415in}{1.493117in}}%
\pgfpathclose%
\pgfusepath{stroke,fill}%
\end{pgfscope}%
\begin{pgfscope}%
\pgfpathrectangle{\pgfqpoint{0.100000in}{0.212622in}}{\pgfqpoint{3.696000in}{3.696000in}}%
\pgfusepath{clip}%
\pgfsetbuttcap%
\pgfsetroundjoin%
\definecolor{currentfill}{rgb}{0.121569,0.466667,0.705882}%
\pgfsetfillcolor{currentfill}%
\pgfsetfillopacity{0.627433}%
\pgfsetlinewidth{1.003750pt}%
\definecolor{currentstroke}{rgb}{0.121569,0.466667,0.705882}%
\pgfsetstrokecolor{currentstroke}%
\pgfsetstrokeopacity{0.627433}%
\pgfsetdash{}{0pt}%
\pgfpathmoveto{\pgfqpoint{0.860415in}{1.493117in}}%
\pgfpathcurveto{\pgfqpoint{0.868651in}{1.493117in}}{\pgfqpoint{0.876551in}{1.496389in}}{\pgfqpoint{0.882375in}{1.502213in}}%
\pgfpathcurveto{\pgfqpoint{0.888199in}{1.508037in}}{\pgfqpoint{0.891471in}{1.515937in}}{\pgfqpoint{0.891471in}{1.524173in}}%
\pgfpathcurveto{\pgfqpoint{0.891471in}{1.532410in}}{\pgfqpoint{0.888199in}{1.540310in}}{\pgfqpoint{0.882375in}{1.546134in}}%
\pgfpathcurveto{\pgfqpoint{0.876551in}{1.551958in}}{\pgfqpoint{0.868651in}{1.555230in}}{\pgfqpoint{0.860415in}{1.555230in}}%
\pgfpathcurveto{\pgfqpoint{0.852178in}{1.555230in}}{\pgfqpoint{0.844278in}{1.551958in}}{\pgfqpoint{0.838454in}{1.546134in}}%
\pgfpathcurveto{\pgfqpoint{0.832631in}{1.540310in}}{\pgfqpoint{0.829358in}{1.532410in}}{\pgfqpoint{0.829358in}{1.524173in}}%
\pgfpathcurveto{\pgfqpoint{0.829358in}{1.515937in}}{\pgfqpoint{0.832631in}{1.508037in}}{\pgfqpoint{0.838454in}{1.502213in}}%
\pgfpathcurveto{\pgfqpoint{0.844278in}{1.496389in}}{\pgfqpoint{0.852178in}{1.493117in}}{\pgfqpoint{0.860415in}{1.493117in}}%
\pgfpathclose%
\pgfusepath{stroke,fill}%
\end{pgfscope}%
\begin{pgfscope}%
\pgfpathrectangle{\pgfqpoint{0.100000in}{0.212622in}}{\pgfqpoint{3.696000in}{3.696000in}}%
\pgfusepath{clip}%
\pgfsetbuttcap%
\pgfsetroundjoin%
\definecolor{currentfill}{rgb}{0.121569,0.466667,0.705882}%
\pgfsetfillcolor{currentfill}%
\pgfsetfillopacity{0.627433}%
\pgfsetlinewidth{1.003750pt}%
\definecolor{currentstroke}{rgb}{0.121569,0.466667,0.705882}%
\pgfsetstrokecolor{currentstroke}%
\pgfsetstrokeopacity{0.627433}%
\pgfsetdash{}{0pt}%
\pgfpathmoveto{\pgfqpoint{0.860415in}{1.493117in}}%
\pgfpathcurveto{\pgfqpoint{0.868651in}{1.493117in}}{\pgfqpoint{0.876551in}{1.496389in}}{\pgfqpoint{0.882375in}{1.502213in}}%
\pgfpathcurveto{\pgfqpoint{0.888199in}{1.508037in}}{\pgfqpoint{0.891471in}{1.515937in}}{\pgfqpoint{0.891471in}{1.524173in}}%
\pgfpathcurveto{\pgfqpoint{0.891471in}{1.532410in}}{\pgfqpoint{0.888199in}{1.540310in}}{\pgfqpoint{0.882375in}{1.546134in}}%
\pgfpathcurveto{\pgfqpoint{0.876551in}{1.551958in}}{\pgfqpoint{0.868651in}{1.555230in}}{\pgfqpoint{0.860415in}{1.555230in}}%
\pgfpathcurveto{\pgfqpoint{0.852178in}{1.555230in}}{\pgfqpoint{0.844278in}{1.551958in}}{\pgfqpoint{0.838454in}{1.546134in}}%
\pgfpathcurveto{\pgfqpoint{0.832631in}{1.540310in}}{\pgfqpoint{0.829358in}{1.532410in}}{\pgfqpoint{0.829358in}{1.524173in}}%
\pgfpathcurveto{\pgfqpoint{0.829358in}{1.515937in}}{\pgfqpoint{0.832631in}{1.508037in}}{\pgfqpoint{0.838454in}{1.502213in}}%
\pgfpathcurveto{\pgfqpoint{0.844278in}{1.496389in}}{\pgfqpoint{0.852178in}{1.493117in}}{\pgfqpoint{0.860415in}{1.493117in}}%
\pgfpathclose%
\pgfusepath{stroke,fill}%
\end{pgfscope}%
\begin{pgfscope}%
\pgfpathrectangle{\pgfqpoint{0.100000in}{0.212622in}}{\pgfqpoint{3.696000in}{3.696000in}}%
\pgfusepath{clip}%
\pgfsetbuttcap%
\pgfsetroundjoin%
\definecolor{currentfill}{rgb}{0.121569,0.466667,0.705882}%
\pgfsetfillcolor{currentfill}%
\pgfsetfillopacity{0.627433}%
\pgfsetlinewidth{1.003750pt}%
\definecolor{currentstroke}{rgb}{0.121569,0.466667,0.705882}%
\pgfsetstrokecolor{currentstroke}%
\pgfsetstrokeopacity{0.627433}%
\pgfsetdash{}{0pt}%
\pgfpathmoveto{\pgfqpoint{0.860415in}{1.493117in}}%
\pgfpathcurveto{\pgfqpoint{0.868651in}{1.493117in}}{\pgfqpoint{0.876551in}{1.496389in}}{\pgfqpoint{0.882375in}{1.502213in}}%
\pgfpathcurveto{\pgfqpoint{0.888199in}{1.508037in}}{\pgfqpoint{0.891471in}{1.515937in}}{\pgfqpoint{0.891471in}{1.524173in}}%
\pgfpathcurveto{\pgfqpoint{0.891471in}{1.532410in}}{\pgfqpoint{0.888199in}{1.540310in}}{\pgfqpoint{0.882375in}{1.546134in}}%
\pgfpathcurveto{\pgfqpoint{0.876551in}{1.551958in}}{\pgfqpoint{0.868651in}{1.555230in}}{\pgfqpoint{0.860415in}{1.555230in}}%
\pgfpathcurveto{\pgfqpoint{0.852178in}{1.555230in}}{\pgfqpoint{0.844278in}{1.551958in}}{\pgfqpoint{0.838454in}{1.546134in}}%
\pgfpathcurveto{\pgfqpoint{0.832631in}{1.540310in}}{\pgfqpoint{0.829358in}{1.532410in}}{\pgfqpoint{0.829358in}{1.524173in}}%
\pgfpathcurveto{\pgfqpoint{0.829358in}{1.515937in}}{\pgfqpoint{0.832631in}{1.508037in}}{\pgfqpoint{0.838454in}{1.502213in}}%
\pgfpathcurveto{\pgfqpoint{0.844278in}{1.496389in}}{\pgfqpoint{0.852178in}{1.493117in}}{\pgfqpoint{0.860415in}{1.493117in}}%
\pgfpathclose%
\pgfusepath{stroke,fill}%
\end{pgfscope}%
\begin{pgfscope}%
\pgfpathrectangle{\pgfqpoint{0.100000in}{0.212622in}}{\pgfqpoint{3.696000in}{3.696000in}}%
\pgfusepath{clip}%
\pgfsetbuttcap%
\pgfsetroundjoin%
\definecolor{currentfill}{rgb}{0.121569,0.466667,0.705882}%
\pgfsetfillcolor{currentfill}%
\pgfsetfillopacity{0.627433}%
\pgfsetlinewidth{1.003750pt}%
\definecolor{currentstroke}{rgb}{0.121569,0.466667,0.705882}%
\pgfsetstrokecolor{currentstroke}%
\pgfsetstrokeopacity{0.627433}%
\pgfsetdash{}{0pt}%
\pgfpathmoveto{\pgfqpoint{0.860415in}{1.493117in}}%
\pgfpathcurveto{\pgfqpoint{0.868651in}{1.493117in}}{\pgfqpoint{0.876551in}{1.496389in}}{\pgfqpoint{0.882375in}{1.502213in}}%
\pgfpathcurveto{\pgfqpoint{0.888199in}{1.508037in}}{\pgfqpoint{0.891471in}{1.515937in}}{\pgfqpoint{0.891471in}{1.524173in}}%
\pgfpathcurveto{\pgfqpoint{0.891471in}{1.532410in}}{\pgfqpoint{0.888199in}{1.540310in}}{\pgfqpoint{0.882375in}{1.546134in}}%
\pgfpathcurveto{\pgfqpoint{0.876551in}{1.551958in}}{\pgfqpoint{0.868651in}{1.555230in}}{\pgfqpoint{0.860415in}{1.555230in}}%
\pgfpathcurveto{\pgfqpoint{0.852178in}{1.555230in}}{\pgfqpoint{0.844278in}{1.551958in}}{\pgfqpoint{0.838454in}{1.546134in}}%
\pgfpathcurveto{\pgfqpoint{0.832631in}{1.540310in}}{\pgfqpoint{0.829358in}{1.532410in}}{\pgfqpoint{0.829358in}{1.524173in}}%
\pgfpathcurveto{\pgfqpoint{0.829358in}{1.515937in}}{\pgfqpoint{0.832631in}{1.508037in}}{\pgfqpoint{0.838454in}{1.502213in}}%
\pgfpathcurveto{\pgfqpoint{0.844278in}{1.496389in}}{\pgfqpoint{0.852178in}{1.493117in}}{\pgfqpoint{0.860415in}{1.493117in}}%
\pgfpathclose%
\pgfusepath{stroke,fill}%
\end{pgfscope}%
\begin{pgfscope}%
\pgfpathrectangle{\pgfqpoint{0.100000in}{0.212622in}}{\pgfqpoint{3.696000in}{3.696000in}}%
\pgfusepath{clip}%
\pgfsetbuttcap%
\pgfsetroundjoin%
\definecolor{currentfill}{rgb}{0.121569,0.466667,0.705882}%
\pgfsetfillcolor{currentfill}%
\pgfsetfillopacity{0.627433}%
\pgfsetlinewidth{1.003750pt}%
\definecolor{currentstroke}{rgb}{0.121569,0.466667,0.705882}%
\pgfsetstrokecolor{currentstroke}%
\pgfsetstrokeopacity{0.627433}%
\pgfsetdash{}{0pt}%
\pgfpathmoveto{\pgfqpoint{0.860415in}{1.493117in}}%
\pgfpathcurveto{\pgfqpoint{0.868651in}{1.493117in}}{\pgfqpoint{0.876551in}{1.496389in}}{\pgfqpoint{0.882375in}{1.502213in}}%
\pgfpathcurveto{\pgfqpoint{0.888199in}{1.508037in}}{\pgfqpoint{0.891471in}{1.515937in}}{\pgfqpoint{0.891471in}{1.524173in}}%
\pgfpathcurveto{\pgfqpoint{0.891471in}{1.532410in}}{\pgfqpoint{0.888199in}{1.540310in}}{\pgfqpoint{0.882375in}{1.546134in}}%
\pgfpathcurveto{\pgfqpoint{0.876551in}{1.551958in}}{\pgfqpoint{0.868651in}{1.555230in}}{\pgfqpoint{0.860415in}{1.555230in}}%
\pgfpathcurveto{\pgfqpoint{0.852178in}{1.555230in}}{\pgfqpoint{0.844278in}{1.551958in}}{\pgfqpoint{0.838454in}{1.546134in}}%
\pgfpathcurveto{\pgfqpoint{0.832631in}{1.540310in}}{\pgfqpoint{0.829358in}{1.532410in}}{\pgfqpoint{0.829358in}{1.524173in}}%
\pgfpathcurveto{\pgfqpoint{0.829358in}{1.515937in}}{\pgfqpoint{0.832631in}{1.508037in}}{\pgfqpoint{0.838454in}{1.502213in}}%
\pgfpathcurveto{\pgfqpoint{0.844278in}{1.496389in}}{\pgfqpoint{0.852178in}{1.493117in}}{\pgfqpoint{0.860415in}{1.493117in}}%
\pgfpathclose%
\pgfusepath{stroke,fill}%
\end{pgfscope}%
\begin{pgfscope}%
\pgfpathrectangle{\pgfqpoint{0.100000in}{0.212622in}}{\pgfqpoint{3.696000in}{3.696000in}}%
\pgfusepath{clip}%
\pgfsetbuttcap%
\pgfsetroundjoin%
\definecolor{currentfill}{rgb}{0.121569,0.466667,0.705882}%
\pgfsetfillcolor{currentfill}%
\pgfsetfillopacity{0.627433}%
\pgfsetlinewidth{1.003750pt}%
\definecolor{currentstroke}{rgb}{0.121569,0.466667,0.705882}%
\pgfsetstrokecolor{currentstroke}%
\pgfsetstrokeopacity{0.627433}%
\pgfsetdash{}{0pt}%
\pgfpathmoveto{\pgfqpoint{0.860415in}{1.493117in}}%
\pgfpathcurveto{\pgfqpoint{0.868651in}{1.493117in}}{\pgfqpoint{0.876551in}{1.496389in}}{\pgfqpoint{0.882375in}{1.502213in}}%
\pgfpathcurveto{\pgfqpoint{0.888199in}{1.508037in}}{\pgfqpoint{0.891471in}{1.515937in}}{\pgfqpoint{0.891471in}{1.524173in}}%
\pgfpathcurveto{\pgfqpoint{0.891471in}{1.532410in}}{\pgfqpoint{0.888199in}{1.540310in}}{\pgfqpoint{0.882375in}{1.546134in}}%
\pgfpathcurveto{\pgfqpoint{0.876551in}{1.551958in}}{\pgfqpoint{0.868651in}{1.555230in}}{\pgfqpoint{0.860415in}{1.555230in}}%
\pgfpathcurveto{\pgfqpoint{0.852178in}{1.555230in}}{\pgfqpoint{0.844278in}{1.551958in}}{\pgfqpoint{0.838454in}{1.546134in}}%
\pgfpathcurveto{\pgfqpoint{0.832631in}{1.540310in}}{\pgfqpoint{0.829358in}{1.532410in}}{\pgfqpoint{0.829358in}{1.524173in}}%
\pgfpathcurveto{\pgfqpoint{0.829358in}{1.515937in}}{\pgfqpoint{0.832631in}{1.508037in}}{\pgfqpoint{0.838454in}{1.502213in}}%
\pgfpathcurveto{\pgfqpoint{0.844278in}{1.496389in}}{\pgfqpoint{0.852178in}{1.493117in}}{\pgfqpoint{0.860415in}{1.493117in}}%
\pgfpathclose%
\pgfusepath{stroke,fill}%
\end{pgfscope}%
\begin{pgfscope}%
\pgfpathrectangle{\pgfqpoint{0.100000in}{0.212622in}}{\pgfqpoint{3.696000in}{3.696000in}}%
\pgfusepath{clip}%
\pgfsetbuttcap%
\pgfsetroundjoin%
\definecolor{currentfill}{rgb}{0.121569,0.466667,0.705882}%
\pgfsetfillcolor{currentfill}%
\pgfsetfillopacity{0.627433}%
\pgfsetlinewidth{1.003750pt}%
\definecolor{currentstroke}{rgb}{0.121569,0.466667,0.705882}%
\pgfsetstrokecolor{currentstroke}%
\pgfsetstrokeopacity{0.627433}%
\pgfsetdash{}{0pt}%
\pgfpathmoveto{\pgfqpoint{0.860415in}{1.493117in}}%
\pgfpathcurveto{\pgfqpoint{0.868651in}{1.493117in}}{\pgfqpoint{0.876551in}{1.496389in}}{\pgfqpoint{0.882375in}{1.502213in}}%
\pgfpathcurveto{\pgfqpoint{0.888199in}{1.508037in}}{\pgfqpoint{0.891471in}{1.515937in}}{\pgfqpoint{0.891471in}{1.524173in}}%
\pgfpathcurveto{\pgfqpoint{0.891471in}{1.532410in}}{\pgfqpoint{0.888199in}{1.540310in}}{\pgfqpoint{0.882375in}{1.546134in}}%
\pgfpathcurveto{\pgfqpoint{0.876551in}{1.551958in}}{\pgfqpoint{0.868651in}{1.555230in}}{\pgfqpoint{0.860415in}{1.555230in}}%
\pgfpathcurveto{\pgfqpoint{0.852178in}{1.555230in}}{\pgfqpoint{0.844278in}{1.551958in}}{\pgfqpoint{0.838454in}{1.546134in}}%
\pgfpathcurveto{\pgfqpoint{0.832631in}{1.540310in}}{\pgfqpoint{0.829358in}{1.532410in}}{\pgfqpoint{0.829358in}{1.524173in}}%
\pgfpathcurveto{\pgfqpoint{0.829358in}{1.515937in}}{\pgfqpoint{0.832631in}{1.508037in}}{\pgfqpoint{0.838454in}{1.502213in}}%
\pgfpathcurveto{\pgfqpoint{0.844278in}{1.496389in}}{\pgfqpoint{0.852178in}{1.493117in}}{\pgfqpoint{0.860415in}{1.493117in}}%
\pgfpathclose%
\pgfusepath{stroke,fill}%
\end{pgfscope}%
\begin{pgfscope}%
\pgfpathrectangle{\pgfqpoint{0.100000in}{0.212622in}}{\pgfqpoint{3.696000in}{3.696000in}}%
\pgfusepath{clip}%
\pgfsetbuttcap%
\pgfsetroundjoin%
\definecolor{currentfill}{rgb}{0.121569,0.466667,0.705882}%
\pgfsetfillcolor{currentfill}%
\pgfsetfillopacity{0.627433}%
\pgfsetlinewidth{1.003750pt}%
\definecolor{currentstroke}{rgb}{0.121569,0.466667,0.705882}%
\pgfsetstrokecolor{currentstroke}%
\pgfsetstrokeopacity{0.627433}%
\pgfsetdash{}{0pt}%
\pgfpathmoveto{\pgfqpoint{0.860415in}{1.493117in}}%
\pgfpathcurveto{\pgfqpoint{0.868651in}{1.493117in}}{\pgfqpoint{0.876551in}{1.496389in}}{\pgfqpoint{0.882375in}{1.502213in}}%
\pgfpathcurveto{\pgfqpoint{0.888199in}{1.508037in}}{\pgfqpoint{0.891471in}{1.515937in}}{\pgfqpoint{0.891471in}{1.524173in}}%
\pgfpathcurveto{\pgfqpoint{0.891471in}{1.532410in}}{\pgfqpoint{0.888199in}{1.540310in}}{\pgfqpoint{0.882375in}{1.546134in}}%
\pgfpathcurveto{\pgfqpoint{0.876551in}{1.551958in}}{\pgfqpoint{0.868651in}{1.555230in}}{\pgfqpoint{0.860415in}{1.555230in}}%
\pgfpathcurveto{\pgfqpoint{0.852178in}{1.555230in}}{\pgfqpoint{0.844278in}{1.551958in}}{\pgfqpoint{0.838454in}{1.546134in}}%
\pgfpathcurveto{\pgfqpoint{0.832631in}{1.540310in}}{\pgfqpoint{0.829358in}{1.532410in}}{\pgfqpoint{0.829358in}{1.524173in}}%
\pgfpathcurveto{\pgfqpoint{0.829358in}{1.515937in}}{\pgfqpoint{0.832631in}{1.508037in}}{\pgfqpoint{0.838454in}{1.502213in}}%
\pgfpathcurveto{\pgfqpoint{0.844278in}{1.496389in}}{\pgfqpoint{0.852178in}{1.493117in}}{\pgfqpoint{0.860415in}{1.493117in}}%
\pgfpathclose%
\pgfusepath{stroke,fill}%
\end{pgfscope}%
\begin{pgfscope}%
\pgfpathrectangle{\pgfqpoint{0.100000in}{0.212622in}}{\pgfqpoint{3.696000in}{3.696000in}}%
\pgfusepath{clip}%
\pgfsetbuttcap%
\pgfsetroundjoin%
\definecolor{currentfill}{rgb}{0.121569,0.466667,0.705882}%
\pgfsetfillcolor{currentfill}%
\pgfsetfillopacity{0.627433}%
\pgfsetlinewidth{1.003750pt}%
\definecolor{currentstroke}{rgb}{0.121569,0.466667,0.705882}%
\pgfsetstrokecolor{currentstroke}%
\pgfsetstrokeopacity{0.627433}%
\pgfsetdash{}{0pt}%
\pgfpathmoveto{\pgfqpoint{0.860415in}{1.493117in}}%
\pgfpathcurveto{\pgfqpoint{0.868651in}{1.493117in}}{\pgfqpoint{0.876551in}{1.496389in}}{\pgfqpoint{0.882375in}{1.502213in}}%
\pgfpathcurveto{\pgfqpoint{0.888199in}{1.508037in}}{\pgfqpoint{0.891471in}{1.515937in}}{\pgfqpoint{0.891471in}{1.524173in}}%
\pgfpathcurveto{\pgfqpoint{0.891471in}{1.532410in}}{\pgfqpoint{0.888199in}{1.540310in}}{\pgfqpoint{0.882375in}{1.546134in}}%
\pgfpathcurveto{\pgfqpoint{0.876551in}{1.551958in}}{\pgfqpoint{0.868651in}{1.555230in}}{\pgfqpoint{0.860415in}{1.555230in}}%
\pgfpathcurveto{\pgfqpoint{0.852178in}{1.555230in}}{\pgfqpoint{0.844278in}{1.551958in}}{\pgfqpoint{0.838454in}{1.546134in}}%
\pgfpathcurveto{\pgfqpoint{0.832631in}{1.540310in}}{\pgfqpoint{0.829358in}{1.532410in}}{\pgfqpoint{0.829358in}{1.524173in}}%
\pgfpathcurveto{\pgfqpoint{0.829358in}{1.515937in}}{\pgfqpoint{0.832631in}{1.508037in}}{\pgfqpoint{0.838454in}{1.502213in}}%
\pgfpathcurveto{\pgfqpoint{0.844278in}{1.496389in}}{\pgfqpoint{0.852178in}{1.493117in}}{\pgfqpoint{0.860415in}{1.493117in}}%
\pgfpathclose%
\pgfusepath{stroke,fill}%
\end{pgfscope}%
\begin{pgfscope}%
\pgfpathrectangle{\pgfqpoint{0.100000in}{0.212622in}}{\pgfqpoint{3.696000in}{3.696000in}}%
\pgfusepath{clip}%
\pgfsetbuttcap%
\pgfsetroundjoin%
\definecolor{currentfill}{rgb}{0.121569,0.466667,0.705882}%
\pgfsetfillcolor{currentfill}%
\pgfsetfillopacity{0.627433}%
\pgfsetlinewidth{1.003750pt}%
\definecolor{currentstroke}{rgb}{0.121569,0.466667,0.705882}%
\pgfsetstrokecolor{currentstroke}%
\pgfsetstrokeopacity{0.627433}%
\pgfsetdash{}{0pt}%
\pgfpathmoveto{\pgfqpoint{0.860415in}{1.493117in}}%
\pgfpathcurveto{\pgfqpoint{0.868651in}{1.493117in}}{\pgfqpoint{0.876551in}{1.496389in}}{\pgfqpoint{0.882375in}{1.502213in}}%
\pgfpathcurveto{\pgfqpoint{0.888199in}{1.508037in}}{\pgfqpoint{0.891471in}{1.515937in}}{\pgfqpoint{0.891471in}{1.524173in}}%
\pgfpathcurveto{\pgfqpoint{0.891471in}{1.532410in}}{\pgfqpoint{0.888199in}{1.540310in}}{\pgfqpoint{0.882375in}{1.546134in}}%
\pgfpathcurveto{\pgfqpoint{0.876551in}{1.551958in}}{\pgfqpoint{0.868651in}{1.555230in}}{\pgfqpoint{0.860415in}{1.555230in}}%
\pgfpathcurveto{\pgfqpoint{0.852178in}{1.555230in}}{\pgfqpoint{0.844278in}{1.551958in}}{\pgfqpoint{0.838454in}{1.546134in}}%
\pgfpathcurveto{\pgfqpoint{0.832631in}{1.540310in}}{\pgfqpoint{0.829358in}{1.532410in}}{\pgfqpoint{0.829358in}{1.524173in}}%
\pgfpathcurveto{\pgfqpoint{0.829358in}{1.515937in}}{\pgfqpoint{0.832631in}{1.508037in}}{\pgfqpoint{0.838454in}{1.502213in}}%
\pgfpathcurveto{\pgfqpoint{0.844278in}{1.496389in}}{\pgfqpoint{0.852178in}{1.493117in}}{\pgfqpoint{0.860415in}{1.493117in}}%
\pgfpathclose%
\pgfusepath{stroke,fill}%
\end{pgfscope}%
\begin{pgfscope}%
\pgfpathrectangle{\pgfqpoint{0.100000in}{0.212622in}}{\pgfqpoint{3.696000in}{3.696000in}}%
\pgfusepath{clip}%
\pgfsetbuttcap%
\pgfsetroundjoin%
\definecolor{currentfill}{rgb}{0.121569,0.466667,0.705882}%
\pgfsetfillcolor{currentfill}%
\pgfsetfillopacity{0.627433}%
\pgfsetlinewidth{1.003750pt}%
\definecolor{currentstroke}{rgb}{0.121569,0.466667,0.705882}%
\pgfsetstrokecolor{currentstroke}%
\pgfsetstrokeopacity{0.627433}%
\pgfsetdash{}{0pt}%
\pgfpathmoveto{\pgfqpoint{0.860415in}{1.493117in}}%
\pgfpathcurveto{\pgfqpoint{0.868651in}{1.493117in}}{\pgfqpoint{0.876551in}{1.496389in}}{\pgfqpoint{0.882375in}{1.502213in}}%
\pgfpathcurveto{\pgfqpoint{0.888199in}{1.508037in}}{\pgfqpoint{0.891471in}{1.515937in}}{\pgfqpoint{0.891471in}{1.524173in}}%
\pgfpathcurveto{\pgfqpoint{0.891471in}{1.532410in}}{\pgfqpoint{0.888199in}{1.540310in}}{\pgfqpoint{0.882375in}{1.546134in}}%
\pgfpathcurveto{\pgfqpoint{0.876551in}{1.551958in}}{\pgfqpoint{0.868651in}{1.555230in}}{\pgfqpoint{0.860415in}{1.555230in}}%
\pgfpathcurveto{\pgfqpoint{0.852178in}{1.555230in}}{\pgfqpoint{0.844278in}{1.551958in}}{\pgfqpoint{0.838454in}{1.546134in}}%
\pgfpathcurveto{\pgfqpoint{0.832631in}{1.540310in}}{\pgfqpoint{0.829358in}{1.532410in}}{\pgfqpoint{0.829358in}{1.524173in}}%
\pgfpathcurveto{\pgfqpoint{0.829358in}{1.515937in}}{\pgfqpoint{0.832631in}{1.508037in}}{\pgfqpoint{0.838454in}{1.502213in}}%
\pgfpathcurveto{\pgfqpoint{0.844278in}{1.496389in}}{\pgfqpoint{0.852178in}{1.493117in}}{\pgfqpoint{0.860415in}{1.493117in}}%
\pgfpathclose%
\pgfusepath{stroke,fill}%
\end{pgfscope}%
\begin{pgfscope}%
\pgfpathrectangle{\pgfqpoint{0.100000in}{0.212622in}}{\pgfqpoint{3.696000in}{3.696000in}}%
\pgfusepath{clip}%
\pgfsetbuttcap%
\pgfsetroundjoin%
\definecolor{currentfill}{rgb}{0.121569,0.466667,0.705882}%
\pgfsetfillcolor{currentfill}%
\pgfsetfillopacity{0.627433}%
\pgfsetlinewidth{1.003750pt}%
\definecolor{currentstroke}{rgb}{0.121569,0.466667,0.705882}%
\pgfsetstrokecolor{currentstroke}%
\pgfsetstrokeopacity{0.627433}%
\pgfsetdash{}{0pt}%
\pgfpathmoveto{\pgfqpoint{0.860415in}{1.493117in}}%
\pgfpathcurveto{\pgfqpoint{0.868651in}{1.493117in}}{\pgfqpoint{0.876551in}{1.496389in}}{\pgfqpoint{0.882375in}{1.502213in}}%
\pgfpathcurveto{\pgfqpoint{0.888199in}{1.508037in}}{\pgfqpoint{0.891471in}{1.515937in}}{\pgfqpoint{0.891471in}{1.524173in}}%
\pgfpathcurveto{\pgfqpoint{0.891471in}{1.532410in}}{\pgfqpoint{0.888199in}{1.540310in}}{\pgfqpoint{0.882375in}{1.546134in}}%
\pgfpathcurveto{\pgfqpoint{0.876551in}{1.551958in}}{\pgfqpoint{0.868651in}{1.555230in}}{\pgfqpoint{0.860415in}{1.555230in}}%
\pgfpathcurveto{\pgfqpoint{0.852178in}{1.555230in}}{\pgfqpoint{0.844278in}{1.551958in}}{\pgfqpoint{0.838454in}{1.546134in}}%
\pgfpathcurveto{\pgfqpoint{0.832631in}{1.540310in}}{\pgfqpoint{0.829358in}{1.532410in}}{\pgfqpoint{0.829358in}{1.524173in}}%
\pgfpathcurveto{\pgfqpoint{0.829358in}{1.515937in}}{\pgfqpoint{0.832631in}{1.508037in}}{\pgfqpoint{0.838454in}{1.502213in}}%
\pgfpathcurveto{\pgfqpoint{0.844278in}{1.496389in}}{\pgfqpoint{0.852178in}{1.493117in}}{\pgfqpoint{0.860415in}{1.493117in}}%
\pgfpathclose%
\pgfusepath{stroke,fill}%
\end{pgfscope}%
\begin{pgfscope}%
\pgfpathrectangle{\pgfqpoint{0.100000in}{0.212622in}}{\pgfqpoint{3.696000in}{3.696000in}}%
\pgfusepath{clip}%
\pgfsetbuttcap%
\pgfsetroundjoin%
\definecolor{currentfill}{rgb}{0.121569,0.466667,0.705882}%
\pgfsetfillcolor{currentfill}%
\pgfsetfillopacity{0.627433}%
\pgfsetlinewidth{1.003750pt}%
\definecolor{currentstroke}{rgb}{0.121569,0.466667,0.705882}%
\pgfsetstrokecolor{currentstroke}%
\pgfsetstrokeopacity{0.627433}%
\pgfsetdash{}{0pt}%
\pgfpathmoveto{\pgfqpoint{0.860415in}{1.493117in}}%
\pgfpathcurveto{\pgfqpoint{0.868651in}{1.493117in}}{\pgfqpoint{0.876551in}{1.496389in}}{\pgfqpoint{0.882375in}{1.502213in}}%
\pgfpathcurveto{\pgfqpoint{0.888199in}{1.508037in}}{\pgfqpoint{0.891471in}{1.515937in}}{\pgfqpoint{0.891471in}{1.524173in}}%
\pgfpathcurveto{\pgfqpoint{0.891471in}{1.532410in}}{\pgfqpoint{0.888199in}{1.540310in}}{\pgfqpoint{0.882375in}{1.546134in}}%
\pgfpathcurveto{\pgfqpoint{0.876551in}{1.551958in}}{\pgfqpoint{0.868651in}{1.555230in}}{\pgfqpoint{0.860415in}{1.555230in}}%
\pgfpathcurveto{\pgfqpoint{0.852178in}{1.555230in}}{\pgfqpoint{0.844278in}{1.551958in}}{\pgfqpoint{0.838454in}{1.546134in}}%
\pgfpathcurveto{\pgfqpoint{0.832631in}{1.540310in}}{\pgfqpoint{0.829358in}{1.532410in}}{\pgfqpoint{0.829358in}{1.524173in}}%
\pgfpathcurveto{\pgfqpoint{0.829358in}{1.515937in}}{\pgfqpoint{0.832631in}{1.508037in}}{\pgfqpoint{0.838454in}{1.502213in}}%
\pgfpathcurveto{\pgfqpoint{0.844278in}{1.496389in}}{\pgfqpoint{0.852178in}{1.493117in}}{\pgfqpoint{0.860415in}{1.493117in}}%
\pgfpathclose%
\pgfusepath{stroke,fill}%
\end{pgfscope}%
\begin{pgfscope}%
\pgfpathrectangle{\pgfqpoint{0.100000in}{0.212622in}}{\pgfqpoint{3.696000in}{3.696000in}}%
\pgfusepath{clip}%
\pgfsetbuttcap%
\pgfsetroundjoin%
\definecolor{currentfill}{rgb}{0.121569,0.466667,0.705882}%
\pgfsetfillcolor{currentfill}%
\pgfsetfillopacity{0.627433}%
\pgfsetlinewidth{1.003750pt}%
\definecolor{currentstroke}{rgb}{0.121569,0.466667,0.705882}%
\pgfsetstrokecolor{currentstroke}%
\pgfsetstrokeopacity{0.627433}%
\pgfsetdash{}{0pt}%
\pgfpathmoveto{\pgfqpoint{0.860415in}{1.493117in}}%
\pgfpathcurveto{\pgfqpoint{0.868651in}{1.493117in}}{\pgfqpoint{0.876551in}{1.496389in}}{\pgfqpoint{0.882375in}{1.502213in}}%
\pgfpathcurveto{\pgfqpoint{0.888199in}{1.508037in}}{\pgfqpoint{0.891471in}{1.515937in}}{\pgfqpoint{0.891471in}{1.524173in}}%
\pgfpathcurveto{\pgfqpoint{0.891471in}{1.532410in}}{\pgfqpoint{0.888199in}{1.540310in}}{\pgfqpoint{0.882375in}{1.546134in}}%
\pgfpathcurveto{\pgfqpoint{0.876551in}{1.551958in}}{\pgfqpoint{0.868651in}{1.555230in}}{\pgfqpoint{0.860415in}{1.555230in}}%
\pgfpathcurveto{\pgfqpoint{0.852178in}{1.555230in}}{\pgfqpoint{0.844278in}{1.551958in}}{\pgfqpoint{0.838454in}{1.546134in}}%
\pgfpathcurveto{\pgfqpoint{0.832631in}{1.540310in}}{\pgfqpoint{0.829358in}{1.532410in}}{\pgfqpoint{0.829358in}{1.524173in}}%
\pgfpathcurveto{\pgfqpoint{0.829358in}{1.515937in}}{\pgfqpoint{0.832631in}{1.508037in}}{\pgfqpoint{0.838454in}{1.502213in}}%
\pgfpathcurveto{\pgfqpoint{0.844278in}{1.496389in}}{\pgfqpoint{0.852178in}{1.493117in}}{\pgfqpoint{0.860415in}{1.493117in}}%
\pgfpathclose%
\pgfusepath{stroke,fill}%
\end{pgfscope}%
\begin{pgfscope}%
\pgfpathrectangle{\pgfqpoint{0.100000in}{0.212622in}}{\pgfqpoint{3.696000in}{3.696000in}}%
\pgfusepath{clip}%
\pgfsetbuttcap%
\pgfsetroundjoin%
\definecolor{currentfill}{rgb}{0.121569,0.466667,0.705882}%
\pgfsetfillcolor{currentfill}%
\pgfsetfillopacity{0.627433}%
\pgfsetlinewidth{1.003750pt}%
\definecolor{currentstroke}{rgb}{0.121569,0.466667,0.705882}%
\pgfsetstrokecolor{currentstroke}%
\pgfsetstrokeopacity{0.627433}%
\pgfsetdash{}{0pt}%
\pgfpathmoveto{\pgfqpoint{0.860415in}{1.493117in}}%
\pgfpathcurveto{\pgfqpoint{0.868651in}{1.493117in}}{\pgfqpoint{0.876551in}{1.496389in}}{\pgfqpoint{0.882375in}{1.502213in}}%
\pgfpathcurveto{\pgfqpoint{0.888199in}{1.508037in}}{\pgfqpoint{0.891471in}{1.515937in}}{\pgfqpoint{0.891471in}{1.524173in}}%
\pgfpathcurveto{\pgfqpoint{0.891471in}{1.532410in}}{\pgfqpoint{0.888199in}{1.540310in}}{\pgfqpoint{0.882375in}{1.546134in}}%
\pgfpathcurveto{\pgfqpoint{0.876551in}{1.551958in}}{\pgfqpoint{0.868651in}{1.555230in}}{\pgfqpoint{0.860415in}{1.555230in}}%
\pgfpathcurveto{\pgfqpoint{0.852178in}{1.555230in}}{\pgfqpoint{0.844278in}{1.551958in}}{\pgfqpoint{0.838454in}{1.546134in}}%
\pgfpathcurveto{\pgfqpoint{0.832631in}{1.540310in}}{\pgfqpoint{0.829358in}{1.532410in}}{\pgfqpoint{0.829358in}{1.524173in}}%
\pgfpathcurveto{\pgfqpoint{0.829358in}{1.515937in}}{\pgfqpoint{0.832631in}{1.508037in}}{\pgfqpoint{0.838454in}{1.502213in}}%
\pgfpathcurveto{\pgfqpoint{0.844278in}{1.496389in}}{\pgfqpoint{0.852178in}{1.493117in}}{\pgfqpoint{0.860415in}{1.493117in}}%
\pgfpathclose%
\pgfusepath{stroke,fill}%
\end{pgfscope}%
\begin{pgfscope}%
\pgfpathrectangle{\pgfqpoint{0.100000in}{0.212622in}}{\pgfqpoint{3.696000in}{3.696000in}}%
\pgfusepath{clip}%
\pgfsetbuttcap%
\pgfsetroundjoin%
\definecolor{currentfill}{rgb}{0.121569,0.466667,0.705882}%
\pgfsetfillcolor{currentfill}%
\pgfsetfillopacity{0.627433}%
\pgfsetlinewidth{1.003750pt}%
\definecolor{currentstroke}{rgb}{0.121569,0.466667,0.705882}%
\pgfsetstrokecolor{currentstroke}%
\pgfsetstrokeopacity{0.627433}%
\pgfsetdash{}{0pt}%
\pgfpathmoveto{\pgfqpoint{0.860415in}{1.493117in}}%
\pgfpathcurveto{\pgfqpoint{0.868651in}{1.493117in}}{\pgfqpoint{0.876551in}{1.496389in}}{\pgfqpoint{0.882375in}{1.502213in}}%
\pgfpathcurveto{\pgfqpoint{0.888199in}{1.508037in}}{\pgfqpoint{0.891471in}{1.515937in}}{\pgfqpoint{0.891471in}{1.524173in}}%
\pgfpathcurveto{\pgfqpoint{0.891471in}{1.532410in}}{\pgfqpoint{0.888199in}{1.540310in}}{\pgfqpoint{0.882375in}{1.546134in}}%
\pgfpathcurveto{\pgfqpoint{0.876551in}{1.551958in}}{\pgfqpoint{0.868651in}{1.555230in}}{\pgfqpoint{0.860415in}{1.555230in}}%
\pgfpathcurveto{\pgfqpoint{0.852178in}{1.555230in}}{\pgfqpoint{0.844278in}{1.551958in}}{\pgfqpoint{0.838454in}{1.546134in}}%
\pgfpathcurveto{\pgfqpoint{0.832631in}{1.540310in}}{\pgfqpoint{0.829358in}{1.532410in}}{\pgfqpoint{0.829358in}{1.524173in}}%
\pgfpathcurveto{\pgfqpoint{0.829358in}{1.515937in}}{\pgfqpoint{0.832631in}{1.508037in}}{\pgfqpoint{0.838454in}{1.502213in}}%
\pgfpathcurveto{\pgfqpoint{0.844278in}{1.496389in}}{\pgfqpoint{0.852178in}{1.493117in}}{\pgfqpoint{0.860415in}{1.493117in}}%
\pgfpathclose%
\pgfusepath{stroke,fill}%
\end{pgfscope}%
\begin{pgfscope}%
\pgfpathrectangle{\pgfqpoint{0.100000in}{0.212622in}}{\pgfqpoint{3.696000in}{3.696000in}}%
\pgfusepath{clip}%
\pgfsetbuttcap%
\pgfsetroundjoin%
\definecolor{currentfill}{rgb}{0.121569,0.466667,0.705882}%
\pgfsetfillcolor{currentfill}%
\pgfsetfillopacity{0.627433}%
\pgfsetlinewidth{1.003750pt}%
\definecolor{currentstroke}{rgb}{0.121569,0.466667,0.705882}%
\pgfsetstrokecolor{currentstroke}%
\pgfsetstrokeopacity{0.627433}%
\pgfsetdash{}{0pt}%
\pgfpathmoveto{\pgfqpoint{0.860415in}{1.493117in}}%
\pgfpathcurveto{\pgfqpoint{0.868651in}{1.493117in}}{\pgfqpoint{0.876551in}{1.496389in}}{\pgfqpoint{0.882375in}{1.502213in}}%
\pgfpathcurveto{\pgfqpoint{0.888199in}{1.508037in}}{\pgfqpoint{0.891471in}{1.515937in}}{\pgfqpoint{0.891471in}{1.524173in}}%
\pgfpathcurveto{\pgfqpoint{0.891471in}{1.532410in}}{\pgfqpoint{0.888199in}{1.540310in}}{\pgfqpoint{0.882375in}{1.546134in}}%
\pgfpathcurveto{\pgfqpoint{0.876551in}{1.551958in}}{\pgfqpoint{0.868651in}{1.555230in}}{\pgfqpoint{0.860415in}{1.555230in}}%
\pgfpathcurveto{\pgfqpoint{0.852178in}{1.555230in}}{\pgfqpoint{0.844278in}{1.551958in}}{\pgfqpoint{0.838454in}{1.546134in}}%
\pgfpathcurveto{\pgfqpoint{0.832631in}{1.540310in}}{\pgfqpoint{0.829358in}{1.532410in}}{\pgfqpoint{0.829358in}{1.524173in}}%
\pgfpathcurveto{\pgfqpoint{0.829358in}{1.515937in}}{\pgfqpoint{0.832631in}{1.508037in}}{\pgfqpoint{0.838454in}{1.502213in}}%
\pgfpathcurveto{\pgfqpoint{0.844278in}{1.496389in}}{\pgfqpoint{0.852178in}{1.493117in}}{\pgfqpoint{0.860415in}{1.493117in}}%
\pgfpathclose%
\pgfusepath{stroke,fill}%
\end{pgfscope}%
\begin{pgfscope}%
\pgfpathrectangle{\pgfqpoint{0.100000in}{0.212622in}}{\pgfqpoint{3.696000in}{3.696000in}}%
\pgfusepath{clip}%
\pgfsetbuttcap%
\pgfsetroundjoin%
\definecolor{currentfill}{rgb}{0.121569,0.466667,0.705882}%
\pgfsetfillcolor{currentfill}%
\pgfsetfillopacity{0.627433}%
\pgfsetlinewidth{1.003750pt}%
\definecolor{currentstroke}{rgb}{0.121569,0.466667,0.705882}%
\pgfsetstrokecolor{currentstroke}%
\pgfsetstrokeopacity{0.627433}%
\pgfsetdash{}{0pt}%
\pgfpathmoveto{\pgfqpoint{0.860415in}{1.493117in}}%
\pgfpathcurveto{\pgfqpoint{0.868651in}{1.493117in}}{\pgfqpoint{0.876551in}{1.496389in}}{\pgfqpoint{0.882375in}{1.502213in}}%
\pgfpathcurveto{\pgfqpoint{0.888199in}{1.508037in}}{\pgfqpoint{0.891471in}{1.515937in}}{\pgfqpoint{0.891471in}{1.524173in}}%
\pgfpathcurveto{\pgfqpoint{0.891471in}{1.532410in}}{\pgfqpoint{0.888199in}{1.540310in}}{\pgfqpoint{0.882375in}{1.546134in}}%
\pgfpathcurveto{\pgfqpoint{0.876551in}{1.551958in}}{\pgfqpoint{0.868651in}{1.555230in}}{\pgfqpoint{0.860415in}{1.555230in}}%
\pgfpathcurveto{\pgfqpoint{0.852178in}{1.555230in}}{\pgfqpoint{0.844278in}{1.551958in}}{\pgfqpoint{0.838454in}{1.546134in}}%
\pgfpathcurveto{\pgfqpoint{0.832631in}{1.540310in}}{\pgfqpoint{0.829358in}{1.532410in}}{\pgfqpoint{0.829358in}{1.524173in}}%
\pgfpathcurveto{\pgfqpoint{0.829358in}{1.515937in}}{\pgfqpoint{0.832631in}{1.508037in}}{\pgfqpoint{0.838454in}{1.502213in}}%
\pgfpathcurveto{\pgfqpoint{0.844278in}{1.496389in}}{\pgfqpoint{0.852178in}{1.493117in}}{\pgfqpoint{0.860415in}{1.493117in}}%
\pgfpathclose%
\pgfusepath{stroke,fill}%
\end{pgfscope}%
\begin{pgfscope}%
\pgfpathrectangle{\pgfqpoint{0.100000in}{0.212622in}}{\pgfqpoint{3.696000in}{3.696000in}}%
\pgfusepath{clip}%
\pgfsetbuttcap%
\pgfsetroundjoin%
\definecolor{currentfill}{rgb}{0.121569,0.466667,0.705882}%
\pgfsetfillcolor{currentfill}%
\pgfsetfillopacity{0.627433}%
\pgfsetlinewidth{1.003750pt}%
\definecolor{currentstroke}{rgb}{0.121569,0.466667,0.705882}%
\pgfsetstrokecolor{currentstroke}%
\pgfsetstrokeopacity{0.627433}%
\pgfsetdash{}{0pt}%
\pgfpathmoveto{\pgfqpoint{0.860415in}{1.493117in}}%
\pgfpathcurveto{\pgfqpoint{0.868651in}{1.493117in}}{\pgfqpoint{0.876551in}{1.496389in}}{\pgfqpoint{0.882375in}{1.502213in}}%
\pgfpathcurveto{\pgfqpoint{0.888199in}{1.508037in}}{\pgfqpoint{0.891471in}{1.515937in}}{\pgfqpoint{0.891471in}{1.524173in}}%
\pgfpathcurveto{\pgfqpoint{0.891471in}{1.532410in}}{\pgfqpoint{0.888199in}{1.540310in}}{\pgfqpoint{0.882375in}{1.546134in}}%
\pgfpathcurveto{\pgfqpoint{0.876551in}{1.551958in}}{\pgfqpoint{0.868651in}{1.555230in}}{\pgfqpoint{0.860415in}{1.555230in}}%
\pgfpathcurveto{\pgfqpoint{0.852178in}{1.555230in}}{\pgfqpoint{0.844278in}{1.551958in}}{\pgfqpoint{0.838454in}{1.546134in}}%
\pgfpathcurveto{\pgfqpoint{0.832631in}{1.540310in}}{\pgfqpoint{0.829358in}{1.532410in}}{\pgfqpoint{0.829358in}{1.524173in}}%
\pgfpathcurveto{\pgfqpoint{0.829358in}{1.515937in}}{\pgfqpoint{0.832631in}{1.508037in}}{\pgfqpoint{0.838454in}{1.502213in}}%
\pgfpathcurveto{\pgfqpoint{0.844278in}{1.496389in}}{\pgfqpoint{0.852178in}{1.493117in}}{\pgfqpoint{0.860415in}{1.493117in}}%
\pgfpathclose%
\pgfusepath{stroke,fill}%
\end{pgfscope}%
\begin{pgfscope}%
\pgfpathrectangle{\pgfqpoint{0.100000in}{0.212622in}}{\pgfqpoint{3.696000in}{3.696000in}}%
\pgfusepath{clip}%
\pgfsetbuttcap%
\pgfsetroundjoin%
\definecolor{currentfill}{rgb}{0.121569,0.466667,0.705882}%
\pgfsetfillcolor{currentfill}%
\pgfsetfillopacity{0.627433}%
\pgfsetlinewidth{1.003750pt}%
\definecolor{currentstroke}{rgb}{0.121569,0.466667,0.705882}%
\pgfsetstrokecolor{currentstroke}%
\pgfsetstrokeopacity{0.627433}%
\pgfsetdash{}{0pt}%
\pgfpathmoveto{\pgfqpoint{0.860415in}{1.493117in}}%
\pgfpathcurveto{\pgfqpoint{0.868651in}{1.493117in}}{\pgfqpoint{0.876551in}{1.496389in}}{\pgfqpoint{0.882375in}{1.502213in}}%
\pgfpathcurveto{\pgfqpoint{0.888199in}{1.508037in}}{\pgfqpoint{0.891471in}{1.515937in}}{\pgfqpoint{0.891471in}{1.524173in}}%
\pgfpathcurveto{\pgfqpoint{0.891471in}{1.532410in}}{\pgfqpoint{0.888199in}{1.540310in}}{\pgfqpoint{0.882375in}{1.546134in}}%
\pgfpathcurveto{\pgfqpoint{0.876551in}{1.551958in}}{\pgfqpoint{0.868651in}{1.555230in}}{\pgfqpoint{0.860415in}{1.555230in}}%
\pgfpathcurveto{\pgfqpoint{0.852178in}{1.555230in}}{\pgfqpoint{0.844278in}{1.551958in}}{\pgfqpoint{0.838454in}{1.546134in}}%
\pgfpathcurveto{\pgfqpoint{0.832631in}{1.540310in}}{\pgfqpoint{0.829358in}{1.532410in}}{\pgfqpoint{0.829358in}{1.524173in}}%
\pgfpathcurveto{\pgfqpoint{0.829358in}{1.515937in}}{\pgfqpoint{0.832631in}{1.508037in}}{\pgfqpoint{0.838454in}{1.502213in}}%
\pgfpathcurveto{\pgfqpoint{0.844278in}{1.496389in}}{\pgfqpoint{0.852178in}{1.493117in}}{\pgfqpoint{0.860415in}{1.493117in}}%
\pgfpathclose%
\pgfusepath{stroke,fill}%
\end{pgfscope}%
\begin{pgfscope}%
\pgfpathrectangle{\pgfqpoint{0.100000in}{0.212622in}}{\pgfqpoint{3.696000in}{3.696000in}}%
\pgfusepath{clip}%
\pgfsetbuttcap%
\pgfsetroundjoin%
\definecolor{currentfill}{rgb}{0.121569,0.466667,0.705882}%
\pgfsetfillcolor{currentfill}%
\pgfsetfillopacity{0.627433}%
\pgfsetlinewidth{1.003750pt}%
\definecolor{currentstroke}{rgb}{0.121569,0.466667,0.705882}%
\pgfsetstrokecolor{currentstroke}%
\pgfsetstrokeopacity{0.627433}%
\pgfsetdash{}{0pt}%
\pgfpathmoveto{\pgfqpoint{0.860415in}{1.493117in}}%
\pgfpathcurveto{\pgfqpoint{0.868651in}{1.493117in}}{\pgfqpoint{0.876551in}{1.496389in}}{\pgfqpoint{0.882375in}{1.502213in}}%
\pgfpathcurveto{\pgfqpoint{0.888199in}{1.508037in}}{\pgfqpoint{0.891471in}{1.515937in}}{\pgfqpoint{0.891471in}{1.524173in}}%
\pgfpathcurveto{\pgfqpoint{0.891471in}{1.532410in}}{\pgfqpoint{0.888199in}{1.540310in}}{\pgfqpoint{0.882375in}{1.546134in}}%
\pgfpathcurveto{\pgfqpoint{0.876551in}{1.551958in}}{\pgfqpoint{0.868651in}{1.555230in}}{\pgfqpoint{0.860415in}{1.555230in}}%
\pgfpathcurveto{\pgfqpoint{0.852178in}{1.555230in}}{\pgfqpoint{0.844278in}{1.551958in}}{\pgfqpoint{0.838454in}{1.546134in}}%
\pgfpathcurveto{\pgfqpoint{0.832631in}{1.540310in}}{\pgfqpoint{0.829358in}{1.532410in}}{\pgfqpoint{0.829358in}{1.524173in}}%
\pgfpathcurveto{\pgfqpoint{0.829358in}{1.515937in}}{\pgfqpoint{0.832631in}{1.508037in}}{\pgfqpoint{0.838454in}{1.502213in}}%
\pgfpathcurveto{\pgfqpoint{0.844278in}{1.496389in}}{\pgfqpoint{0.852178in}{1.493117in}}{\pgfqpoint{0.860415in}{1.493117in}}%
\pgfpathclose%
\pgfusepath{stroke,fill}%
\end{pgfscope}%
\begin{pgfscope}%
\pgfpathrectangle{\pgfqpoint{0.100000in}{0.212622in}}{\pgfqpoint{3.696000in}{3.696000in}}%
\pgfusepath{clip}%
\pgfsetbuttcap%
\pgfsetroundjoin%
\definecolor{currentfill}{rgb}{0.121569,0.466667,0.705882}%
\pgfsetfillcolor{currentfill}%
\pgfsetfillopacity{0.627433}%
\pgfsetlinewidth{1.003750pt}%
\definecolor{currentstroke}{rgb}{0.121569,0.466667,0.705882}%
\pgfsetstrokecolor{currentstroke}%
\pgfsetstrokeopacity{0.627433}%
\pgfsetdash{}{0pt}%
\pgfpathmoveto{\pgfqpoint{0.860415in}{1.493117in}}%
\pgfpathcurveto{\pgfqpoint{0.868651in}{1.493117in}}{\pgfqpoint{0.876551in}{1.496389in}}{\pgfqpoint{0.882375in}{1.502213in}}%
\pgfpathcurveto{\pgfqpoint{0.888199in}{1.508037in}}{\pgfqpoint{0.891471in}{1.515937in}}{\pgfqpoint{0.891471in}{1.524173in}}%
\pgfpathcurveto{\pgfqpoint{0.891471in}{1.532410in}}{\pgfqpoint{0.888199in}{1.540310in}}{\pgfqpoint{0.882375in}{1.546134in}}%
\pgfpathcurveto{\pgfqpoint{0.876551in}{1.551958in}}{\pgfqpoint{0.868651in}{1.555230in}}{\pgfqpoint{0.860415in}{1.555230in}}%
\pgfpathcurveto{\pgfqpoint{0.852178in}{1.555230in}}{\pgfqpoint{0.844278in}{1.551958in}}{\pgfqpoint{0.838454in}{1.546134in}}%
\pgfpathcurveto{\pgfqpoint{0.832631in}{1.540310in}}{\pgfqpoint{0.829358in}{1.532410in}}{\pgfqpoint{0.829358in}{1.524173in}}%
\pgfpathcurveto{\pgfqpoint{0.829358in}{1.515937in}}{\pgfqpoint{0.832631in}{1.508037in}}{\pgfqpoint{0.838454in}{1.502213in}}%
\pgfpathcurveto{\pgfqpoint{0.844278in}{1.496389in}}{\pgfqpoint{0.852178in}{1.493117in}}{\pgfqpoint{0.860415in}{1.493117in}}%
\pgfpathclose%
\pgfusepath{stroke,fill}%
\end{pgfscope}%
\begin{pgfscope}%
\pgfpathrectangle{\pgfqpoint{0.100000in}{0.212622in}}{\pgfqpoint{3.696000in}{3.696000in}}%
\pgfusepath{clip}%
\pgfsetbuttcap%
\pgfsetroundjoin%
\definecolor{currentfill}{rgb}{0.121569,0.466667,0.705882}%
\pgfsetfillcolor{currentfill}%
\pgfsetfillopacity{0.627433}%
\pgfsetlinewidth{1.003750pt}%
\definecolor{currentstroke}{rgb}{0.121569,0.466667,0.705882}%
\pgfsetstrokecolor{currentstroke}%
\pgfsetstrokeopacity{0.627433}%
\pgfsetdash{}{0pt}%
\pgfpathmoveto{\pgfqpoint{0.860415in}{1.493117in}}%
\pgfpathcurveto{\pgfqpoint{0.868651in}{1.493117in}}{\pgfqpoint{0.876551in}{1.496389in}}{\pgfqpoint{0.882375in}{1.502213in}}%
\pgfpathcurveto{\pgfqpoint{0.888199in}{1.508037in}}{\pgfqpoint{0.891471in}{1.515937in}}{\pgfqpoint{0.891471in}{1.524173in}}%
\pgfpathcurveto{\pgfqpoint{0.891471in}{1.532410in}}{\pgfqpoint{0.888199in}{1.540310in}}{\pgfqpoint{0.882375in}{1.546134in}}%
\pgfpathcurveto{\pgfqpoint{0.876551in}{1.551958in}}{\pgfqpoint{0.868651in}{1.555230in}}{\pgfqpoint{0.860415in}{1.555230in}}%
\pgfpathcurveto{\pgfqpoint{0.852178in}{1.555230in}}{\pgfqpoint{0.844278in}{1.551958in}}{\pgfqpoint{0.838454in}{1.546134in}}%
\pgfpathcurveto{\pgfqpoint{0.832631in}{1.540310in}}{\pgfqpoint{0.829358in}{1.532410in}}{\pgfqpoint{0.829358in}{1.524173in}}%
\pgfpathcurveto{\pgfqpoint{0.829358in}{1.515937in}}{\pgfqpoint{0.832631in}{1.508037in}}{\pgfqpoint{0.838454in}{1.502213in}}%
\pgfpathcurveto{\pgfqpoint{0.844278in}{1.496389in}}{\pgfqpoint{0.852178in}{1.493117in}}{\pgfqpoint{0.860415in}{1.493117in}}%
\pgfpathclose%
\pgfusepath{stroke,fill}%
\end{pgfscope}%
\begin{pgfscope}%
\pgfpathrectangle{\pgfqpoint{0.100000in}{0.212622in}}{\pgfqpoint{3.696000in}{3.696000in}}%
\pgfusepath{clip}%
\pgfsetbuttcap%
\pgfsetroundjoin%
\definecolor{currentfill}{rgb}{0.121569,0.466667,0.705882}%
\pgfsetfillcolor{currentfill}%
\pgfsetfillopacity{0.627433}%
\pgfsetlinewidth{1.003750pt}%
\definecolor{currentstroke}{rgb}{0.121569,0.466667,0.705882}%
\pgfsetstrokecolor{currentstroke}%
\pgfsetstrokeopacity{0.627433}%
\pgfsetdash{}{0pt}%
\pgfpathmoveto{\pgfqpoint{0.860415in}{1.493117in}}%
\pgfpathcurveto{\pgfqpoint{0.868651in}{1.493117in}}{\pgfqpoint{0.876551in}{1.496389in}}{\pgfqpoint{0.882375in}{1.502213in}}%
\pgfpathcurveto{\pgfqpoint{0.888199in}{1.508037in}}{\pgfqpoint{0.891471in}{1.515937in}}{\pgfqpoint{0.891471in}{1.524173in}}%
\pgfpathcurveto{\pgfqpoint{0.891471in}{1.532410in}}{\pgfqpoint{0.888199in}{1.540310in}}{\pgfqpoint{0.882375in}{1.546134in}}%
\pgfpathcurveto{\pgfqpoint{0.876551in}{1.551958in}}{\pgfqpoint{0.868651in}{1.555230in}}{\pgfqpoint{0.860415in}{1.555230in}}%
\pgfpathcurveto{\pgfqpoint{0.852178in}{1.555230in}}{\pgfqpoint{0.844278in}{1.551958in}}{\pgfqpoint{0.838454in}{1.546134in}}%
\pgfpathcurveto{\pgfqpoint{0.832631in}{1.540310in}}{\pgfqpoint{0.829358in}{1.532410in}}{\pgfqpoint{0.829358in}{1.524173in}}%
\pgfpathcurveto{\pgfqpoint{0.829358in}{1.515937in}}{\pgfqpoint{0.832631in}{1.508037in}}{\pgfqpoint{0.838454in}{1.502213in}}%
\pgfpathcurveto{\pgfqpoint{0.844278in}{1.496389in}}{\pgfqpoint{0.852178in}{1.493117in}}{\pgfqpoint{0.860415in}{1.493117in}}%
\pgfpathclose%
\pgfusepath{stroke,fill}%
\end{pgfscope}%
\begin{pgfscope}%
\pgfpathrectangle{\pgfqpoint{0.100000in}{0.212622in}}{\pgfqpoint{3.696000in}{3.696000in}}%
\pgfusepath{clip}%
\pgfsetbuttcap%
\pgfsetroundjoin%
\definecolor{currentfill}{rgb}{0.121569,0.466667,0.705882}%
\pgfsetfillcolor{currentfill}%
\pgfsetfillopacity{0.627433}%
\pgfsetlinewidth{1.003750pt}%
\definecolor{currentstroke}{rgb}{0.121569,0.466667,0.705882}%
\pgfsetstrokecolor{currentstroke}%
\pgfsetstrokeopacity{0.627433}%
\pgfsetdash{}{0pt}%
\pgfpathmoveto{\pgfqpoint{0.860415in}{1.493117in}}%
\pgfpathcurveto{\pgfqpoint{0.868651in}{1.493117in}}{\pgfqpoint{0.876551in}{1.496389in}}{\pgfqpoint{0.882375in}{1.502213in}}%
\pgfpathcurveto{\pgfqpoint{0.888199in}{1.508037in}}{\pgfqpoint{0.891471in}{1.515937in}}{\pgfqpoint{0.891471in}{1.524173in}}%
\pgfpathcurveto{\pgfqpoint{0.891471in}{1.532410in}}{\pgfqpoint{0.888199in}{1.540310in}}{\pgfqpoint{0.882375in}{1.546134in}}%
\pgfpathcurveto{\pgfqpoint{0.876551in}{1.551958in}}{\pgfqpoint{0.868651in}{1.555230in}}{\pgfqpoint{0.860415in}{1.555230in}}%
\pgfpathcurveto{\pgfqpoint{0.852178in}{1.555230in}}{\pgfqpoint{0.844278in}{1.551958in}}{\pgfqpoint{0.838454in}{1.546134in}}%
\pgfpathcurveto{\pgfqpoint{0.832631in}{1.540310in}}{\pgfqpoint{0.829358in}{1.532410in}}{\pgfqpoint{0.829358in}{1.524173in}}%
\pgfpathcurveto{\pgfqpoint{0.829358in}{1.515937in}}{\pgfqpoint{0.832631in}{1.508037in}}{\pgfqpoint{0.838454in}{1.502213in}}%
\pgfpathcurveto{\pgfqpoint{0.844278in}{1.496389in}}{\pgfqpoint{0.852178in}{1.493117in}}{\pgfqpoint{0.860415in}{1.493117in}}%
\pgfpathclose%
\pgfusepath{stroke,fill}%
\end{pgfscope}%
\begin{pgfscope}%
\pgfpathrectangle{\pgfqpoint{0.100000in}{0.212622in}}{\pgfqpoint{3.696000in}{3.696000in}}%
\pgfusepath{clip}%
\pgfsetbuttcap%
\pgfsetroundjoin%
\definecolor{currentfill}{rgb}{0.121569,0.466667,0.705882}%
\pgfsetfillcolor{currentfill}%
\pgfsetfillopacity{0.627433}%
\pgfsetlinewidth{1.003750pt}%
\definecolor{currentstroke}{rgb}{0.121569,0.466667,0.705882}%
\pgfsetstrokecolor{currentstroke}%
\pgfsetstrokeopacity{0.627433}%
\pgfsetdash{}{0pt}%
\pgfpathmoveto{\pgfqpoint{0.860415in}{1.493117in}}%
\pgfpathcurveto{\pgfqpoint{0.868651in}{1.493117in}}{\pgfqpoint{0.876551in}{1.496389in}}{\pgfqpoint{0.882375in}{1.502213in}}%
\pgfpathcurveto{\pgfqpoint{0.888199in}{1.508037in}}{\pgfqpoint{0.891471in}{1.515937in}}{\pgfqpoint{0.891471in}{1.524173in}}%
\pgfpathcurveto{\pgfqpoint{0.891471in}{1.532410in}}{\pgfqpoint{0.888199in}{1.540310in}}{\pgfqpoint{0.882375in}{1.546134in}}%
\pgfpathcurveto{\pgfqpoint{0.876551in}{1.551958in}}{\pgfqpoint{0.868651in}{1.555230in}}{\pgfqpoint{0.860415in}{1.555230in}}%
\pgfpathcurveto{\pgfqpoint{0.852178in}{1.555230in}}{\pgfqpoint{0.844278in}{1.551958in}}{\pgfqpoint{0.838454in}{1.546134in}}%
\pgfpathcurveto{\pgfqpoint{0.832631in}{1.540310in}}{\pgfqpoint{0.829358in}{1.532410in}}{\pgfqpoint{0.829358in}{1.524173in}}%
\pgfpathcurveto{\pgfqpoint{0.829358in}{1.515937in}}{\pgfqpoint{0.832631in}{1.508037in}}{\pgfqpoint{0.838454in}{1.502213in}}%
\pgfpathcurveto{\pgfqpoint{0.844278in}{1.496389in}}{\pgfqpoint{0.852178in}{1.493117in}}{\pgfqpoint{0.860415in}{1.493117in}}%
\pgfpathclose%
\pgfusepath{stroke,fill}%
\end{pgfscope}%
\begin{pgfscope}%
\pgfpathrectangle{\pgfqpoint{0.100000in}{0.212622in}}{\pgfqpoint{3.696000in}{3.696000in}}%
\pgfusepath{clip}%
\pgfsetbuttcap%
\pgfsetroundjoin%
\definecolor{currentfill}{rgb}{0.121569,0.466667,0.705882}%
\pgfsetfillcolor{currentfill}%
\pgfsetfillopacity{0.627433}%
\pgfsetlinewidth{1.003750pt}%
\definecolor{currentstroke}{rgb}{0.121569,0.466667,0.705882}%
\pgfsetstrokecolor{currentstroke}%
\pgfsetstrokeopacity{0.627433}%
\pgfsetdash{}{0pt}%
\pgfpathmoveto{\pgfqpoint{0.860415in}{1.493117in}}%
\pgfpathcurveto{\pgfqpoint{0.868651in}{1.493117in}}{\pgfqpoint{0.876551in}{1.496389in}}{\pgfqpoint{0.882375in}{1.502213in}}%
\pgfpathcurveto{\pgfqpoint{0.888199in}{1.508037in}}{\pgfqpoint{0.891471in}{1.515937in}}{\pgfqpoint{0.891471in}{1.524173in}}%
\pgfpathcurveto{\pgfqpoint{0.891471in}{1.532410in}}{\pgfqpoint{0.888199in}{1.540310in}}{\pgfqpoint{0.882375in}{1.546134in}}%
\pgfpathcurveto{\pgfqpoint{0.876551in}{1.551958in}}{\pgfqpoint{0.868651in}{1.555230in}}{\pgfqpoint{0.860415in}{1.555230in}}%
\pgfpathcurveto{\pgfqpoint{0.852178in}{1.555230in}}{\pgfqpoint{0.844278in}{1.551958in}}{\pgfqpoint{0.838454in}{1.546134in}}%
\pgfpathcurveto{\pgfqpoint{0.832631in}{1.540310in}}{\pgfqpoint{0.829358in}{1.532410in}}{\pgfqpoint{0.829358in}{1.524173in}}%
\pgfpathcurveto{\pgfqpoint{0.829358in}{1.515937in}}{\pgfqpoint{0.832631in}{1.508037in}}{\pgfqpoint{0.838454in}{1.502213in}}%
\pgfpathcurveto{\pgfqpoint{0.844278in}{1.496389in}}{\pgfqpoint{0.852178in}{1.493117in}}{\pgfqpoint{0.860415in}{1.493117in}}%
\pgfpathclose%
\pgfusepath{stroke,fill}%
\end{pgfscope}%
\begin{pgfscope}%
\pgfpathrectangle{\pgfqpoint{0.100000in}{0.212622in}}{\pgfqpoint{3.696000in}{3.696000in}}%
\pgfusepath{clip}%
\pgfsetbuttcap%
\pgfsetroundjoin%
\definecolor{currentfill}{rgb}{0.121569,0.466667,0.705882}%
\pgfsetfillcolor{currentfill}%
\pgfsetfillopacity{0.627433}%
\pgfsetlinewidth{1.003750pt}%
\definecolor{currentstroke}{rgb}{0.121569,0.466667,0.705882}%
\pgfsetstrokecolor{currentstroke}%
\pgfsetstrokeopacity{0.627433}%
\pgfsetdash{}{0pt}%
\pgfpathmoveto{\pgfqpoint{0.860415in}{1.493117in}}%
\pgfpathcurveto{\pgfqpoint{0.868651in}{1.493117in}}{\pgfqpoint{0.876551in}{1.496389in}}{\pgfqpoint{0.882375in}{1.502213in}}%
\pgfpathcurveto{\pgfqpoint{0.888199in}{1.508037in}}{\pgfqpoint{0.891471in}{1.515937in}}{\pgfqpoint{0.891471in}{1.524173in}}%
\pgfpathcurveto{\pgfqpoint{0.891471in}{1.532410in}}{\pgfqpoint{0.888199in}{1.540310in}}{\pgfqpoint{0.882375in}{1.546134in}}%
\pgfpathcurveto{\pgfqpoint{0.876551in}{1.551958in}}{\pgfqpoint{0.868651in}{1.555230in}}{\pgfqpoint{0.860415in}{1.555230in}}%
\pgfpathcurveto{\pgfqpoint{0.852178in}{1.555230in}}{\pgfqpoint{0.844278in}{1.551958in}}{\pgfqpoint{0.838454in}{1.546134in}}%
\pgfpathcurveto{\pgfqpoint{0.832631in}{1.540310in}}{\pgfqpoint{0.829358in}{1.532410in}}{\pgfqpoint{0.829358in}{1.524173in}}%
\pgfpathcurveto{\pgfqpoint{0.829358in}{1.515937in}}{\pgfqpoint{0.832631in}{1.508037in}}{\pgfqpoint{0.838454in}{1.502213in}}%
\pgfpathcurveto{\pgfqpoint{0.844278in}{1.496389in}}{\pgfqpoint{0.852178in}{1.493117in}}{\pgfqpoint{0.860415in}{1.493117in}}%
\pgfpathclose%
\pgfusepath{stroke,fill}%
\end{pgfscope}%
\begin{pgfscope}%
\pgfpathrectangle{\pgfqpoint{0.100000in}{0.212622in}}{\pgfqpoint{3.696000in}{3.696000in}}%
\pgfusepath{clip}%
\pgfsetbuttcap%
\pgfsetroundjoin%
\definecolor{currentfill}{rgb}{0.121569,0.466667,0.705882}%
\pgfsetfillcolor{currentfill}%
\pgfsetfillopacity{0.627433}%
\pgfsetlinewidth{1.003750pt}%
\definecolor{currentstroke}{rgb}{0.121569,0.466667,0.705882}%
\pgfsetstrokecolor{currentstroke}%
\pgfsetstrokeopacity{0.627433}%
\pgfsetdash{}{0pt}%
\pgfpathmoveto{\pgfqpoint{0.860415in}{1.493117in}}%
\pgfpathcurveto{\pgfqpoint{0.868651in}{1.493117in}}{\pgfqpoint{0.876551in}{1.496389in}}{\pgfqpoint{0.882375in}{1.502213in}}%
\pgfpathcurveto{\pgfqpoint{0.888199in}{1.508037in}}{\pgfqpoint{0.891471in}{1.515937in}}{\pgfqpoint{0.891471in}{1.524173in}}%
\pgfpathcurveto{\pgfqpoint{0.891471in}{1.532410in}}{\pgfqpoint{0.888199in}{1.540310in}}{\pgfqpoint{0.882375in}{1.546134in}}%
\pgfpathcurveto{\pgfqpoint{0.876551in}{1.551958in}}{\pgfqpoint{0.868651in}{1.555230in}}{\pgfqpoint{0.860415in}{1.555230in}}%
\pgfpathcurveto{\pgfqpoint{0.852178in}{1.555230in}}{\pgfqpoint{0.844278in}{1.551958in}}{\pgfqpoint{0.838454in}{1.546134in}}%
\pgfpathcurveto{\pgfqpoint{0.832631in}{1.540310in}}{\pgfqpoint{0.829358in}{1.532410in}}{\pgfqpoint{0.829358in}{1.524173in}}%
\pgfpathcurveto{\pgfqpoint{0.829358in}{1.515937in}}{\pgfqpoint{0.832631in}{1.508037in}}{\pgfqpoint{0.838454in}{1.502213in}}%
\pgfpathcurveto{\pgfqpoint{0.844278in}{1.496389in}}{\pgfqpoint{0.852178in}{1.493117in}}{\pgfqpoint{0.860415in}{1.493117in}}%
\pgfpathclose%
\pgfusepath{stroke,fill}%
\end{pgfscope}%
\begin{pgfscope}%
\pgfpathrectangle{\pgfqpoint{0.100000in}{0.212622in}}{\pgfqpoint{3.696000in}{3.696000in}}%
\pgfusepath{clip}%
\pgfsetbuttcap%
\pgfsetroundjoin%
\definecolor{currentfill}{rgb}{0.121569,0.466667,0.705882}%
\pgfsetfillcolor{currentfill}%
\pgfsetfillopacity{0.627433}%
\pgfsetlinewidth{1.003750pt}%
\definecolor{currentstroke}{rgb}{0.121569,0.466667,0.705882}%
\pgfsetstrokecolor{currentstroke}%
\pgfsetstrokeopacity{0.627433}%
\pgfsetdash{}{0pt}%
\pgfpathmoveto{\pgfqpoint{0.860415in}{1.493117in}}%
\pgfpathcurveto{\pgfqpoint{0.868651in}{1.493117in}}{\pgfqpoint{0.876551in}{1.496389in}}{\pgfqpoint{0.882375in}{1.502213in}}%
\pgfpathcurveto{\pgfqpoint{0.888199in}{1.508037in}}{\pgfqpoint{0.891471in}{1.515937in}}{\pgfqpoint{0.891471in}{1.524173in}}%
\pgfpathcurveto{\pgfqpoint{0.891471in}{1.532410in}}{\pgfqpoint{0.888199in}{1.540310in}}{\pgfqpoint{0.882375in}{1.546134in}}%
\pgfpathcurveto{\pgfqpoint{0.876551in}{1.551958in}}{\pgfqpoint{0.868651in}{1.555230in}}{\pgfqpoint{0.860415in}{1.555230in}}%
\pgfpathcurveto{\pgfqpoint{0.852178in}{1.555230in}}{\pgfqpoint{0.844278in}{1.551958in}}{\pgfqpoint{0.838454in}{1.546134in}}%
\pgfpathcurveto{\pgfqpoint{0.832631in}{1.540310in}}{\pgfqpoint{0.829358in}{1.532410in}}{\pgfqpoint{0.829358in}{1.524173in}}%
\pgfpathcurveto{\pgfqpoint{0.829358in}{1.515937in}}{\pgfqpoint{0.832631in}{1.508037in}}{\pgfqpoint{0.838454in}{1.502213in}}%
\pgfpathcurveto{\pgfqpoint{0.844278in}{1.496389in}}{\pgfqpoint{0.852178in}{1.493117in}}{\pgfqpoint{0.860415in}{1.493117in}}%
\pgfpathclose%
\pgfusepath{stroke,fill}%
\end{pgfscope}%
\begin{pgfscope}%
\pgfpathrectangle{\pgfqpoint{0.100000in}{0.212622in}}{\pgfqpoint{3.696000in}{3.696000in}}%
\pgfusepath{clip}%
\pgfsetbuttcap%
\pgfsetroundjoin%
\definecolor{currentfill}{rgb}{0.121569,0.466667,0.705882}%
\pgfsetfillcolor{currentfill}%
\pgfsetfillopacity{0.627433}%
\pgfsetlinewidth{1.003750pt}%
\definecolor{currentstroke}{rgb}{0.121569,0.466667,0.705882}%
\pgfsetstrokecolor{currentstroke}%
\pgfsetstrokeopacity{0.627433}%
\pgfsetdash{}{0pt}%
\pgfpathmoveto{\pgfqpoint{0.860415in}{1.493117in}}%
\pgfpathcurveto{\pgfqpoint{0.868651in}{1.493117in}}{\pgfqpoint{0.876551in}{1.496389in}}{\pgfqpoint{0.882375in}{1.502213in}}%
\pgfpathcurveto{\pgfqpoint{0.888199in}{1.508037in}}{\pgfqpoint{0.891471in}{1.515937in}}{\pgfqpoint{0.891471in}{1.524173in}}%
\pgfpathcurveto{\pgfqpoint{0.891471in}{1.532410in}}{\pgfqpoint{0.888199in}{1.540310in}}{\pgfqpoint{0.882375in}{1.546134in}}%
\pgfpathcurveto{\pgfqpoint{0.876551in}{1.551958in}}{\pgfqpoint{0.868651in}{1.555230in}}{\pgfqpoint{0.860415in}{1.555230in}}%
\pgfpathcurveto{\pgfqpoint{0.852178in}{1.555230in}}{\pgfqpoint{0.844278in}{1.551958in}}{\pgfqpoint{0.838454in}{1.546134in}}%
\pgfpathcurveto{\pgfqpoint{0.832631in}{1.540310in}}{\pgfqpoint{0.829358in}{1.532410in}}{\pgfqpoint{0.829358in}{1.524173in}}%
\pgfpathcurveto{\pgfqpoint{0.829358in}{1.515937in}}{\pgfqpoint{0.832631in}{1.508037in}}{\pgfqpoint{0.838454in}{1.502213in}}%
\pgfpathcurveto{\pgfqpoint{0.844278in}{1.496389in}}{\pgfqpoint{0.852178in}{1.493117in}}{\pgfqpoint{0.860415in}{1.493117in}}%
\pgfpathclose%
\pgfusepath{stroke,fill}%
\end{pgfscope}%
\begin{pgfscope}%
\pgfpathrectangle{\pgfqpoint{0.100000in}{0.212622in}}{\pgfqpoint{3.696000in}{3.696000in}}%
\pgfusepath{clip}%
\pgfsetbuttcap%
\pgfsetroundjoin%
\definecolor{currentfill}{rgb}{0.121569,0.466667,0.705882}%
\pgfsetfillcolor{currentfill}%
\pgfsetfillopacity{0.627433}%
\pgfsetlinewidth{1.003750pt}%
\definecolor{currentstroke}{rgb}{0.121569,0.466667,0.705882}%
\pgfsetstrokecolor{currentstroke}%
\pgfsetstrokeopacity{0.627433}%
\pgfsetdash{}{0pt}%
\pgfpathmoveto{\pgfqpoint{0.860415in}{1.493117in}}%
\pgfpathcurveto{\pgfqpoint{0.868651in}{1.493117in}}{\pgfqpoint{0.876551in}{1.496389in}}{\pgfqpoint{0.882375in}{1.502213in}}%
\pgfpathcurveto{\pgfqpoint{0.888199in}{1.508037in}}{\pgfqpoint{0.891471in}{1.515937in}}{\pgfqpoint{0.891471in}{1.524173in}}%
\pgfpathcurveto{\pgfqpoint{0.891471in}{1.532410in}}{\pgfqpoint{0.888199in}{1.540310in}}{\pgfqpoint{0.882375in}{1.546134in}}%
\pgfpathcurveto{\pgfqpoint{0.876551in}{1.551958in}}{\pgfqpoint{0.868651in}{1.555230in}}{\pgfqpoint{0.860415in}{1.555230in}}%
\pgfpathcurveto{\pgfqpoint{0.852178in}{1.555230in}}{\pgfqpoint{0.844278in}{1.551958in}}{\pgfqpoint{0.838454in}{1.546134in}}%
\pgfpathcurveto{\pgfqpoint{0.832631in}{1.540310in}}{\pgfqpoint{0.829358in}{1.532410in}}{\pgfqpoint{0.829358in}{1.524173in}}%
\pgfpathcurveto{\pgfqpoint{0.829358in}{1.515937in}}{\pgfqpoint{0.832631in}{1.508037in}}{\pgfqpoint{0.838454in}{1.502213in}}%
\pgfpathcurveto{\pgfqpoint{0.844278in}{1.496389in}}{\pgfqpoint{0.852178in}{1.493117in}}{\pgfqpoint{0.860415in}{1.493117in}}%
\pgfpathclose%
\pgfusepath{stroke,fill}%
\end{pgfscope}%
\begin{pgfscope}%
\pgfpathrectangle{\pgfqpoint{0.100000in}{0.212622in}}{\pgfqpoint{3.696000in}{3.696000in}}%
\pgfusepath{clip}%
\pgfsetbuttcap%
\pgfsetroundjoin%
\definecolor{currentfill}{rgb}{0.121569,0.466667,0.705882}%
\pgfsetfillcolor{currentfill}%
\pgfsetfillopacity{0.627433}%
\pgfsetlinewidth{1.003750pt}%
\definecolor{currentstroke}{rgb}{0.121569,0.466667,0.705882}%
\pgfsetstrokecolor{currentstroke}%
\pgfsetstrokeopacity{0.627433}%
\pgfsetdash{}{0pt}%
\pgfpathmoveto{\pgfqpoint{0.860415in}{1.493117in}}%
\pgfpathcurveto{\pgfqpoint{0.868651in}{1.493117in}}{\pgfqpoint{0.876551in}{1.496389in}}{\pgfqpoint{0.882375in}{1.502213in}}%
\pgfpathcurveto{\pgfqpoint{0.888199in}{1.508037in}}{\pgfqpoint{0.891471in}{1.515937in}}{\pgfqpoint{0.891471in}{1.524173in}}%
\pgfpathcurveto{\pgfqpoint{0.891471in}{1.532410in}}{\pgfqpoint{0.888199in}{1.540310in}}{\pgfqpoint{0.882375in}{1.546134in}}%
\pgfpathcurveto{\pgfqpoint{0.876551in}{1.551958in}}{\pgfqpoint{0.868651in}{1.555230in}}{\pgfqpoint{0.860415in}{1.555230in}}%
\pgfpathcurveto{\pgfqpoint{0.852178in}{1.555230in}}{\pgfqpoint{0.844278in}{1.551958in}}{\pgfqpoint{0.838454in}{1.546134in}}%
\pgfpathcurveto{\pgfqpoint{0.832631in}{1.540310in}}{\pgfqpoint{0.829358in}{1.532410in}}{\pgfqpoint{0.829358in}{1.524173in}}%
\pgfpathcurveto{\pgfqpoint{0.829358in}{1.515937in}}{\pgfqpoint{0.832631in}{1.508037in}}{\pgfqpoint{0.838454in}{1.502213in}}%
\pgfpathcurveto{\pgfqpoint{0.844278in}{1.496389in}}{\pgfqpoint{0.852178in}{1.493117in}}{\pgfqpoint{0.860415in}{1.493117in}}%
\pgfpathclose%
\pgfusepath{stroke,fill}%
\end{pgfscope}%
\begin{pgfscope}%
\pgfpathrectangle{\pgfqpoint{0.100000in}{0.212622in}}{\pgfqpoint{3.696000in}{3.696000in}}%
\pgfusepath{clip}%
\pgfsetbuttcap%
\pgfsetroundjoin%
\definecolor{currentfill}{rgb}{0.121569,0.466667,0.705882}%
\pgfsetfillcolor{currentfill}%
\pgfsetfillopacity{0.627433}%
\pgfsetlinewidth{1.003750pt}%
\definecolor{currentstroke}{rgb}{0.121569,0.466667,0.705882}%
\pgfsetstrokecolor{currentstroke}%
\pgfsetstrokeopacity{0.627433}%
\pgfsetdash{}{0pt}%
\pgfpathmoveto{\pgfqpoint{0.860415in}{1.493117in}}%
\pgfpathcurveto{\pgfqpoint{0.868651in}{1.493117in}}{\pgfqpoint{0.876551in}{1.496389in}}{\pgfqpoint{0.882375in}{1.502213in}}%
\pgfpathcurveto{\pgfqpoint{0.888199in}{1.508037in}}{\pgfqpoint{0.891471in}{1.515937in}}{\pgfqpoint{0.891471in}{1.524173in}}%
\pgfpathcurveto{\pgfqpoint{0.891471in}{1.532410in}}{\pgfqpoint{0.888199in}{1.540310in}}{\pgfqpoint{0.882375in}{1.546134in}}%
\pgfpathcurveto{\pgfqpoint{0.876551in}{1.551958in}}{\pgfqpoint{0.868651in}{1.555230in}}{\pgfqpoint{0.860415in}{1.555230in}}%
\pgfpathcurveto{\pgfqpoint{0.852178in}{1.555230in}}{\pgfqpoint{0.844278in}{1.551958in}}{\pgfqpoint{0.838454in}{1.546134in}}%
\pgfpathcurveto{\pgfqpoint{0.832631in}{1.540310in}}{\pgfqpoint{0.829358in}{1.532410in}}{\pgfqpoint{0.829358in}{1.524173in}}%
\pgfpathcurveto{\pgfqpoint{0.829358in}{1.515937in}}{\pgfqpoint{0.832631in}{1.508037in}}{\pgfqpoint{0.838454in}{1.502213in}}%
\pgfpathcurveto{\pgfqpoint{0.844278in}{1.496389in}}{\pgfqpoint{0.852178in}{1.493117in}}{\pgfqpoint{0.860415in}{1.493117in}}%
\pgfpathclose%
\pgfusepath{stroke,fill}%
\end{pgfscope}%
\begin{pgfscope}%
\pgfpathrectangle{\pgfqpoint{0.100000in}{0.212622in}}{\pgfqpoint{3.696000in}{3.696000in}}%
\pgfusepath{clip}%
\pgfsetbuttcap%
\pgfsetroundjoin%
\definecolor{currentfill}{rgb}{0.121569,0.466667,0.705882}%
\pgfsetfillcolor{currentfill}%
\pgfsetfillopacity{0.627433}%
\pgfsetlinewidth{1.003750pt}%
\definecolor{currentstroke}{rgb}{0.121569,0.466667,0.705882}%
\pgfsetstrokecolor{currentstroke}%
\pgfsetstrokeopacity{0.627433}%
\pgfsetdash{}{0pt}%
\pgfpathmoveto{\pgfqpoint{0.860415in}{1.493117in}}%
\pgfpathcurveto{\pgfqpoint{0.868651in}{1.493117in}}{\pgfqpoint{0.876551in}{1.496389in}}{\pgfqpoint{0.882375in}{1.502213in}}%
\pgfpathcurveto{\pgfqpoint{0.888199in}{1.508037in}}{\pgfqpoint{0.891471in}{1.515937in}}{\pgfqpoint{0.891471in}{1.524173in}}%
\pgfpathcurveto{\pgfqpoint{0.891471in}{1.532410in}}{\pgfqpoint{0.888199in}{1.540310in}}{\pgfqpoint{0.882375in}{1.546134in}}%
\pgfpathcurveto{\pgfqpoint{0.876551in}{1.551958in}}{\pgfqpoint{0.868651in}{1.555230in}}{\pgfqpoint{0.860415in}{1.555230in}}%
\pgfpathcurveto{\pgfqpoint{0.852178in}{1.555230in}}{\pgfqpoint{0.844278in}{1.551958in}}{\pgfqpoint{0.838454in}{1.546134in}}%
\pgfpathcurveto{\pgfqpoint{0.832631in}{1.540310in}}{\pgfqpoint{0.829358in}{1.532410in}}{\pgfqpoint{0.829358in}{1.524173in}}%
\pgfpathcurveto{\pgfqpoint{0.829358in}{1.515937in}}{\pgfqpoint{0.832631in}{1.508037in}}{\pgfqpoint{0.838454in}{1.502213in}}%
\pgfpathcurveto{\pgfqpoint{0.844278in}{1.496389in}}{\pgfqpoint{0.852178in}{1.493117in}}{\pgfqpoint{0.860415in}{1.493117in}}%
\pgfpathclose%
\pgfusepath{stroke,fill}%
\end{pgfscope}%
\begin{pgfscope}%
\pgfpathrectangle{\pgfqpoint{0.100000in}{0.212622in}}{\pgfqpoint{3.696000in}{3.696000in}}%
\pgfusepath{clip}%
\pgfsetbuttcap%
\pgfsetroundjoin%
\definecolor{currentfill}{rgb}{0.121569,0.466667,0.705882}%
\pgfsetfillcolor{currentfill}%
\pgfsetfillopacity{0.627433}%
\pgfsetlinewidth{1.003750pt}%
\definecolor{currentstroke}{rgb}{0.121569,0.466667,0.705882}%
\pgfsetstrokecolor{currentstroke}%
\pgfsetstrokeopacity{0.627433}%
\pgfsetdash{}{0pt}%
\pgfpathmoveto{\pgfqpoint{0.860415in}{1.493117in}}%
\pgfpathcurveto{\pgfqpoint{0.868651in}{1.493117in}}{\pgfqpoint{0.876551in}{1.496389in}}{\pgfqpoint{0.882375in}{1.502213in}}%
\pgfpathcurveto{\pgfqpoint{0.888199in}{1.508037in}}{\pgfqpoint{0.891471in}{1.515937in}}{\pgfqpoint{0.891471in}{1.524173in}}%
\pgfpathcurveto{\pgfqpoint{0.891471in}{1.532410in}}{\pgfqpoint{0.888199in}{1.540310in}}{\pgfqpoint{0.882375in}{1.546134in}}%
\pgfpathcurveto{\pgfqpoint{0.876551in}{1.551958in}}{\pgfqpoint{0.868651in}{1.555230in}}{\pgfqpoint{0.860415in}{1.555230in}}%
\pgfpathcurveto{\pgfqpoint{0.852178in}{1.555230in}}{\pgfqpoint{0.844278in}{1.551958in}}{\pgfqpoint{0.838454in}{1.546134in}}%
\pgfpathcurveto{\pgfqpoint{0.832631in}{1.540310in}}{\pgfqpoint{0.829358in}{1.532410in}}{\pgfqpoint{0.829358in}{1.524173in}}%
\pgfpathcurveto{\pgfqpoint{0.829358in}{1.515937in}}{\pgfqpoint{0.832631in}{1.508037in}}{\pgfqpoint{0.838454in}{1.502213in}}%
\pgfpathcurveto{\pgfqpoint{0.844278in}{1.496389in}}{\pgfqpoint{0.852178in}{1.493117in}}{\pgfqpoint{0.860415in}{1.493117in}}%
\pgfpathclose%
\pgfusepath{stroke,fill}%
\end{pgfscope}%
\begin{pgfscope}%
\pgfpathrectangle{\pgfqpoint{0.100000in}{0.212622in}}{\pgfqpoint{3.696000in}{3.696000in}}%
\pgfusepath{clip}%
\pgfsetbuttcap%
\pgfsetroundjoin%
\definecolor{currentfill}{rgb}{0.121569,0.466667,0.705882}%
\pgfsetfillcolor{currentfill}%
\pgfsetfillopacity{0.627433}%
\pgfsetlinewidth{1.003750pt}%
\definecolor{currentstroke}{rgb}{0.121569,0.466667,0.705882}%
\pgfsetstrokecolor{currentstroke}%
\pgfsetstrokeopacity{0.627433}%
\pgfsetdash{}{0pt}%
\pgfpathmoveto{\pgfqpoint{0.860415in}{1.493117in}}%
\pgfpathcurveto{\pgfqpoint{0.868651in}{1.493117in}}{\pgfqpoint{0.876551in}{1.496389in}}{\pgfqpoint{0.882375in}{1.502213in}}%
\pgfpathcurveto{\pgfqpoint{0.888199in}{1.508037in}}{\pgfqpoint{0.891471in}{1.515937in}}{\pgfqpoint{0.891471in}{1.524173in}}%
\pgfpathcurveto{\pgfqpoint{0.891471in}{1.532410in}}{\pgfqpoint{0.888199in}{1.540310in}}{\pgfqpoint{0.882375in}{1.546134in}}%
\pgfpathcurveto{\pgfqpoint{0.876551in}{1.551958in}}{\pgfqpoint{0.868651in}{1.555230in}}{\pgfqpoint{0.860415in}{1.555230in}}%
\pgfpathcurveto{\pgfqpoint{0.852178in}{1.555230in}}{\pgfqpoint{0.844278in}{1.551958in}}{\pgfqpoint{0.838454in}{1.546134in}}%
\pgfpathcurveto{\pgfqpoint{0.832631in}{1.540310in}}{\pgfqpoint{0.829358in}{1.532410in}}{\pgfqpoint{0.829358in}{1.524173in}}%
\pgfpathcurveto{\pgfqpoint{0.829358in}{1.515937in}}{\pgfqpoint{0.832631in}{1.508037in}}{\pgfqpoint{0.838454in}{1.502213in}}%
\pgfpathcurveto{\pgfqpoint{0.844278in}{1.496389in}}{\pgfqpoint{0.852178in}{1.493117in}}{\pgfqpoint{0.860415in}{1.493117in}}%
\pgfpathclose%
\pgfusepath{stroke,fill}%
\end{pgfscope}%
\begin{pgfscope}%
\pgfpathrectangle{\pgfqpoint{0.100000in}{0.212622in}}{\pgfqpoint{3.696000in}{3.696000in}}%
\pgfusepath{clip}%
\pgfsetbuttcap%
\pgfsetroundjoin%
\definecolor{currentfill}{rgb}{0.121569,0.466667,0.705882}%
\pgfsetfillcolor{currentfill}%
\pgfsetfillopacity{0.627433}%
\pgfsetlinewidth{1.003750pt}%
\definecolor{currentstroke}{rgb}{0.121569,0.466667,0.705882}%
\pgfsetstrokecolor{currentstroke}%
\pgfsetstrokeopacity{0.627433}%
\pgfsetdash{}{0pt}%
\pgfpathmoveto{\pgfqpoint{0.860415in}{1.493117in}}%
\pgfpathcurveto{\pgfqpoint{0.868651in}{1.493117in}}{\pgfqpoint{0.876551in}{1.496389in}}{\pgfqpoint{0.882375in}{1.502213in}}%
\pgfpathcurveto{\pgfqpoint{0.888199in}{1.508037in}}{\pgfqpoint{0.891471in}{1.515937in}}{\pgfqpoint{0.891471in}{1.524173in}}%
\pgfpathcurveto{\pgfqpoint{0.891471in}{1.532410in}}{\pgfqpoint{0.888199in}{1.540310in}}{\pgfqpoint{0.882375in}{1.546134in}}%
\pgfpathcurveto{\pgfqpoint{0.876551in}{1.551958in}}{\pgfqpoint{0.868651in}{1.555230in}}{\pgfqpoint{0.860415in}{1.555230in}}%
\pgfpathcurveto{\pgfqpoint{0.852178in}{1.555230in}}{\pgfqpoint{0.844278in}{1.551958in}}{\pgfqpoint{0.838454in}{1.546134in}}%
\pgfpathcurveto{\pgfqpoint{0.832631in}{1.540310in}}{\pgfqpoint{0.829358in}{1.532410in}}{\pgfqpoint{0.829358in}{1.524173in}}%
\pgfpathcurveto{\pgfqpoint{0.829358in}{1.515937in}}{\pgfqpoint{0.832631in}{1.508037in}}{\pgfqpoint{0.838454in}{1.502213in}}%
\pgfpathcurveto{\pgfqpoint{0.844278in}{1.496389in}}{\pgfqpoint{0.852178in}{1.493117in}}{\pgfqpoint{0.860415in}{1.493117in}}%
\pgfpathclose%
\pgfusepath{stroke,fill}%
\end{pgfscope}%
\begin{pgfscope}%
\pgfpathrectangle{\pgfqpoint{0.100000in}{0.212622in}}{\pgfqpoint{3.696000in}{3.696000in}}%
\pgfusepath{clip}%
\pgfsetbuttcap%
\pgfsetroundjoin%
\definecolor{currentfill}{rgb}{0.121569,0.466667,0.705882}%
\pgfsetfillcolor{currentfill}%
\pgfsetfillopacity{0.627433}%
\pgfsetlinewidth{1.003750pt}%
\definecolor{currentstroke}{rgb}{0.121569,0.466667,0.705882}%
\pgfsetstrokecolor{currentstroke}%
\pgfsetstrokeopacity{0.627433}%
\pgfsetdash{}{0pt}%
\pgfpathmoveto{\pgfqpoint{0.860415in}{1.493117in}}%
\pgfpathcurveto{\pgfqpoint{0.868651in}{1.493117in}}{\pgfqpoint{0.876551in}{1.496389in}}{\pgfqpoint{0.882375in}{1.502213in}}%
\pgfpathcurveto{\pgfqpoint{0.888199in}{1.508037in}}{\pgfqpoint{0.891471in}{1.515937in}}{\pgfqpoint{0.891471in}{1.524173in}}%
\pgfpathcurveto{\pgfqpoint{0.891471in}{1.532410in}}{\pgfqpoint{0.888199in}{1.540310in}}{\pgfqpoint{0.882375in}{1.546134in}}%
\pgfpathcurveto{\pgfqpoint{0.876551in}{1.551958in}}{\pgfqpoint{0.868651in}{1.555230in}}{\pgfqpoint{0.860415in}{1.555230in}}%
\pgfpathcurveto{\pgfqpoint{0.852178in}{1.555230in}}{\pgfqpoint{0.844278in}{1.551958in}}{\pgfqpoint{0.838454in}{1.546134in}}%
\pgfpathcurveto{\pgfqpoint{0.832631in}{1.540310in}}{\pgfqpoint{0.829358in}{1.532410in}}{\pgfqpoint{0.829358in}{1.524173in}}%
\pgfpathcurveto{\pgfqpoint{0.829358in}{1.515937in}}{\pgfqpoint{0.832631in}{1.508037in}}{\pgfqpoint{0.838454in}{1.502213in}}%
\pgfpathcurveto{\pgfqpoint{0.844278in}{1.496389in}}{\pgfqpoint{0.852178in}{1.493117in}}{\pgfqpoint{0.860415in}{1.493117in}}%
\pgfpathclose%
\pgfusepath{stroke,fill}%
\end{pgfscope}%
\begin{pgfscope}%
\pgfpathrectangle{\pgfqpoint{0.100000in}{0.212622in}}{\pgfqpoint{3.696000in}{3.696000in}}%
\pgfusepath{clip}%
\pgfsetbuttcap%
\pgfsetroundjoin%
\definecolor{currentfill}{rgb}{0.121569,0.466667,0.705882}%
\pgfsetfillcolor{currentfill}%
\pgfsetfillopacity{0.627433}%
\pgfsetlinewidth{1.003750pt}%
\definecolor{currentstroke}{rgb}{0.121569,0.466667,0.705882}%
\pgfsetstrokecolor{currentstroke}%
\pgfsetstrokeopacity{0.627433}%
\pgfsetdash{}{0pt}%
\pgfpathmoveto{\pgfqpoint{0.860415in}{1.493117in}}%
\pgfpathcurveto{\pgfqpoint{0.868651in}{1.493117in}}{\pgfqpoint{0.876551in}{1.496389in}}{\pgfqpoint{0.882375in}{1.502213in}}%
\pgfpathcurveto{\pgfqpoint{0.888199in}{1.508037in}}{\pgfqpoint{0.891471in}{1.515937in}}{\pgfqpoint{0.891471in}{1.524173in}}%
\pgfpathcurveto{\pgfqpoint{0.891471in}{1.532410in}}{\pgfqpoint{0.888199in}{1.540310in}}{\pgfqpoint{0.882375in}{1.546134in}}%
\pgfpathcurveto{\pgfqpoint{0.876551in}{1.551958in}}{\pgfqpoint{0.868651in}{1.555230in}}{\pgfqpoint{0.860415in}{1.555230in}}%
\pgfpathcurveto{\pgfqpoint{0.852178in}{1.555230in}}{\pgfqpoint{0.844278in}{1.551958in}}{\pgfqpoint{0.838454in}{1.546134in}}%
\pgfpathcurveto{\pgfqpoint{0.832631in}{1.540310in}}{\pgfqpoint{0.829358in}{1.532410in}}{\pgfqpoint{0.829358in}{1.524173in}}%
\pgfpathcurveto{\pgfqpoint{0.829358in}{1.515937in}}{\pgfqpoint{0.832631in}{1.508037in}}{\pgfqpoint{0.838454in}{1.502213in}}%
\pgfpathcurveto{\pgfqpoint{0.844278in}{1.496389in}}{\pgfqpoint{0.852178in}{1.493117in}}{\pgfqpoint{0.860415in}{1.493117in}}%
\pgfpathclose%
\pgfusepath{stroke,fill}%
\end{pgfscope}%
\begin{pgfscope}%
\pgfpathrectangle{\pgfqpoint{0.100000in}{0.212622in}}{\pgfqpoint{3.696000in}{3.696000in}}%
\pgfusepath{clip}%
\pgfsetbuttcap%
\pgfsetroundjoin%
\definecolor{currentfill}{rgb}{0.121569,0.466667,0.705882}%
\pgfsetfillcolor{currentfill}%
\pgfsetfillopacity{0.627433}%
\pgfsetlinewidth{1.003750pt}%
\definecolor{currentstroke}{rgb}{0.121569,0.466667,0.705882}%
\pgfsetstrokecolor{currentstroke}%
\pgfsetstrokeopacity{0.627433}%
\pgfsetdash{}{0pt}%
\pgfpathmoveto{\pgfqpoint{0.860415in}{1.493117in}}%
\pgfpathcurveto{\pgfqpoint{0.868651in}{1.493117in}}{\pgfqpoint{0.876551in}{1.496389in}}{\pgfqpoint{0.882375in}{1.502213in}}%
\pgfpathcurveto{\pgfqpoint{0.888199in}{1.508037in}}{\pgfqpoint{0.891471in}{1.515937in}}{\pgfqpoint{0.891471in}{1.524173in}}%
\pgfpathcurveto{\pgfqpoint{0.891471in}{1.532410in}}{\pgfqpoint{0.888199in}{1.540310in}}{\pgfqpoint{0.882375in}{1.546134in}}%
\pgfpathcurveto{\pgfqpoint{0.876551in}{1.551958in}}{\pgfqpoint{0.868651in}{1.555230in}}{\pgfqpoint{0.860415in}{1.555230in}}%
\pgfpathcurveto{\pgfqpoint{0.852178in}{1.555230in}}{\pgfqpoint{0.844278in}{1.551958in}}{\pgfqpoint{0.838454in}{1.546134in}}%
\pgfpathcurveto{\pgfqpoint{0.832631in}{1.540310in}}{\pgfqpoint{0.829358in}{1.532410in}}{\pgfqpoint{0.829358in}{1.524173in}}%
\pgfpathcurveto{\pgfqpoint{0.829358in}{1.515937in}}{\pgfqpoint{0.832631in}{1.508037in}}{\pgfqpoint{0.838454in}{1.502213in}}%
\pgfpathcurveto{\pgfqpoint{0.844278in}{1.496389in}}{\pgfqpoint{0.852178in}{1.493117in}}{\pgfqpoint{0.860415in}{1.493117in}}%
\pgfpathclose%
\pgfusepath{stroke,fill}%
\end{pgfscope}%
\begin{pgfscope}%
\pgfpathrectangle{\pgfqpoint{0.100000in}{0.212622in}}{\pgfqpoint{3.696000in}{3.696000in}}%
\pgfusepath{clip}%
\pgfsetbuttcap%
\pgfsetroundjoin%
\definecolor{currentfill}{rgb}{0.121569,0.466667,0.705882}%
\pgfsetfillcolor{currentfill}%
\pgfsetfillopacity{0.627433}%
\pgfsetlinewidth{1.003750pt}%
\definecolor{currentstroke}{rgb}{0.121569,0.466667,0.705882}%
\pgfsetstrokecolor{currentstroke}%
\pgfsetstrokeopacity{0.627433}%
\pgfsetdash{}{0pt}%
\pgfpathmoveto{\pgfqpoint{0.860415in}{1.493117in}}%
\pgfpathcurveto{\pgfqpoint{0.868651in}{1.493117in}}{\pgfqpoint{0.876551in}{1.496389in}}{\pgfqpoint{0.882375in}{1.502213in}}%
\pgfpathcurveto{\pgfqpoint{0.888199in}{1.508037in}}{\pgfqpoint{0.891471in}{1.515937in}}{\pgfqpoint{0.891471in}{1.524173in}}%
\pgfpathcurveto{\pgfqpoint{0.891471in}{1.532410in}}{\pgfqpoint{0.888199in}{1.540310in}}{\pgfqpoint{0.882375in}{1.546134in}}%
\pgfpathcurveto{\pgfqpoint{0.876551in}{1.551958in}}{\pgfqpoint{0.868651in}{1.555230in}}{\pgfqpoint{0.860415in}{1.555230in}}%
\pgfpathcurveto{\pgfqpoint{0.852178in}{1.555230in}}{\pgfqpoint{0.844278in}{1.551958in}}{\pgfqpoint{0.838454in}{1.546134in}}%
\pgfpathcurveto{\pgfqpoint{0.832631in}{1.540310in}}{\pgfqpoint{0.829358in}{1.532410in}}{\pgfqpoint{0.829358in}{1.524173in}}%
\pgfpathcurveto{\pgfqpoint{0.829358in}{1.515937in}}{\pgfqpoint{0.832631in}{1.508037in}}{\pgfqpoint{0.838454in}{1.502213in}}%
\pgfpathcurveto{\pgfqpoint{0.844278in}{1.496389in}}{\pgfqpoint{0.852178in}{1.493117in}}{\pgfqpoint{0.860415in}{1.493117in}}%
\pgfpathclose%
\pgfusepath{stroke,fill}%
\end{pgfscope}%
\begin{pgfscope}%
\pgfpathrectangle{\pgfqpoint{0.100000in}{0.212622in}}{\pgfqpoint{3.696000in}{3.696000in}}%
\pgfusepath{clip}%
\pgfsetbuttcap%
\pgfsetroundjoin%
\definecolor{currentfill}{rgb}{0.121569,0.466667,0.705882}%
\pgfsetfillcolor{currentfill}%
\pgfsetfillopacity{0.627433}%
\pgfsetlinewidth{1.003750pt}%
\definecolor{currentstroke}{rgb}{0.121569,0.466667,0.705882}%
\pgfsetstrokecolor{currentstroke}%
\pgfsetstrokeopacity{0.627433}%
\pgfsetdash{}{0pt}%
\pgfpathmoveto{\pgfqpoint{0.860415in}{1.493117in}}%
\pgfpathcurveto{\pgfqpoint{0.868651in}{1.493117in}}{\pgfqpoint{0.876551in}{1.496389in}}{\pgfqpoint{0.882375in}{1.502213in}}%
\pgfpathcurveto{\pgfqpoint{0.888199in}{1.508037in}}{\pgfqpoint{0.891471in}{1.515937in}}{\pgfqpoint{0.891471in}{1.524173in}}%
\pgfpathcurveto{\pgfqpoint{0.891471in}{1.532410in}}{\pgfqpoint{0.888199in}{1.540310in}}{\pgfqpoint{0.882375in}{1.546134in}}%
\pgfpathcurveto{\pgfqpoint{0.876551in}{1.551958in}}{\pgfqpoint{0.868651in}{1.555230in}}{\pgfqpoint{0.860415in}{1.555230in}}%
\pgfpathcurveto{\pgfqpoint{0.852178in}{1.555230in}}{\pgfqpoint{0.844278in}{1.551958in}}{\pgfqpoint{0.838454in}{1.546134in}}%
\pgfpathcurveto{\pgfqpoint{0.832631in}{1.540310in}}{\pgfqpoint{0.829358in}{1.532410in}}{\pgfqpoint{0.829358in}{1.524173in}}%
\pgfpathcurveto{\pgfqpoint{0.829358in}{1.515937in}}{\pgfqpoint{0.832631in}{1.508037in}}{\pgfqpoint{0.838454in}{1.502213in}}%
\pgfpathcurveto{\pgfqpoint{0.844278in}{1.496389in}}{\pgfqpoint{0.852178in}{1.493117in}}{\pgfqpoint{0.860415in}{1.493117in}}%
\pgfpathclose%
\pgfusepath{stroke,fill}%
\end{pgfscope}%
\begin{pgfscope}%
\pgfpathrectangle{\pgfqpoint{0.100000in}{0.212622in}}{\pgfqpoint{3.696000in}{3.696000in}}%
\pgfusepath{clip}%
\pgfsetbuttcap%
\pgfsetroundjoin%
\definecolor{currentfill}{rgb}{0.121569,0.466667,0.705882}%
\pgfsetfillcolor{currentfill}%
\pgfsetfillopacity{0.627433}%
\pgfsetlinewidth{1.003750pt}%
\definecolor{currentstroke}{rgb}{0.121569,0.466667,0.705882}%
\pgfsetstrokecolor{currentstroke}%
\pgfsetstrokeopacity{0.627433}%
\pgfsetdash{}{0pt}%
\pgfpathmoveto{\pgfqpoint{0.860415in}{1.493117in}}%
\pgfpathcurveto{\pgfqpoint{0.868651in}{1.493117in}}{\pgfqpoint{0.876551in}{1.496389in}}{\pgfqpoint{0.882375in}{1.502213in}}%
\pgfpathcurveto{\pgfqpoint{0.888199in}{1.508037in}}{\pgfqpoint{0.891471in}{1.515937in}}{\pgfqpoint{0.891471in}{1.524173in}}%
\pgfpathcurveto{\pgfqpoint{0.891471in}{1.532410in}}{\pgfqpoint{0.888199in}{1.540310in}}{\pgfqpoint{0.882375in}{1.546134in}}%
\pgfpathcurveto{\pgfqpoint{0.876551in}{1.551958in}}{\pgfqpoint{0.868651in}{1.555230in}}{\pgfqpoint{0.860415in}{1.555230in}}%
\pgfpathcurveto{\pgfqpoint{0.852178in}{1.555230in}}{\pgfqpoint{0.844278in}{1.551958in}}{\pgfqpoint{0.838454in}{1.546134in}}%
\pgfpathcurveto{\pgfqpoint{0.832631in}{1.540310in}}{\pgfqpoint{0.829358in}{1.532410in}}{\pgfqpoint{0.829358in}{1.524173in}}%
\pgfpathcurveto{\pgfqpoint{0.829358in}{1.515937in}}{\pgfqpoint{0.832631in}{1.508037in}}{\pgfqpoint{0.838454in}{1.502213in}}%
\pgfpathcurveto{\pgfqpoint{0.844278in}{1.496389in}}{\pgfqpoint{0.852178in}{1.493117in}}{\pgfqpoint{0.860415in}{1.493117in}}%
\pgfpathclose%
\pgfusepath{stroke,fill}%
\end{pgfscope}%
\begin{pgfscope}%
\pgfpathrectangle{\pgfqpoint{0.100000in}{0.212622in}}{\pgfqpoint{3.696000in}{3.696000in}}%
\pgfusepath{clip}%
\pgfsetbuttcap%
\pgfsetroundjoin%
\definecolor{currentfill}{rgb}{0.121569,0.466667,0.705882}%
\pgfsetfillcolor{currentfill}%
\pgfsetfillopacity{0.627433}%
\pgfsetlinewidth{1.003750pt}%
\definecolor{currentstroke}{rgb}{0.121569,0.466667,0.705882}%
\pgfsetstrokecolor{currentstroke}%
\pgfsetstrokeopacity{0.627433}%
\pgfsetdash{}{0pt}%
\pgfpathmoveto{\pgfqpoint{0.860415in}{1.493117in}}%
\pgfpathcurveto{\pgfqpoint{0.868651in}{1.493117in}}{\pgfqpoint{0.876551in}{1.496389in}}{\pgfqpoint{0.882375in}{1.502213in}}%
\pgfpathcurveto{\pgfqpoint{0.888199in}{1.508037in}}{\pgfqpoint{0.891471in}{1.515937in}}{\pgfqpoint{0.891471in}{1.524173in}}%
\pgfpathcurveto{\pgfqpoint{0.891471in}{1.532410in}}{\pgfqpoint{0.888199in}{1.540310in}}{\pgfqpoint{0.882375in}{1.546134in}}%
\pgfpathcurveto{\pgfqpoint{0.876551in}{1.551958in}}{\pgfqpoint{0.868651in}{1.555230in}}{\pgfqpoint{0.860415in}{1.555230in}}%
\pgfpathcurveto{\pgfqpoint{0.852178in}{1.555230in}}{\pgfqpoint{0.844278in}{1.551958in}}{\pgfqpoint{0.838454in}{1.546134in}}%
\pgfpathcurveto{\pgfqpoint{0.832631in}{1.540310in}}{\pgfqpoint{0.829358in}{1.532410in}}{\pgfqpoint{0.829358in}{1.524173in}}%
\pgfpathcurveto{\pgfqpoint{0.829358in}{1.515937in}}{\pgfqpoint{0.832631in}{1.508037in}}{\pgfqpoint{0.838454in}{1.502213in}}%
\pgfpathcurveto{\pgfqpoint{0.844278in}{1.496389in}}{\pgfqpoint{0.852178in}{1.493117in}}{\pgfqpoint{0.860415in}{1.493117in}}%
\pgfpathclose%
\pgfusepath{stroke,fill}%
\end{pgfscope}%
\begin{pgfscope}%
\pgfpathrectangle{\pgfqpoint{0.100000in}{0.212622in}}{\pgfqpoint{3.696000in}{3.696000in}}%
\pgfusepath{clip}%
\pgfsetbuttcap%
\pgfsetroundjoin%
\definecolor{currentfill}{rgb}{0.121569,0.466667,0.705882}%
\pgfsetfillcolor{currentfill}%
\pgfsetfillopacity{0.627433}%
\pgfsetlinewidth{1.003750pt}%
\definecolor{currentstroke}{rgb}{0.121569,0.466667,0.705882}%
\pgfsetstrokecolor{currentstroke}%
\pgfsetstrokeopacity{0.627433}%
\pgfsetdash{}{0pt}%
\pgfpathmoveto{\pgfqpoint{0.860415in}{1.493117in}}%
\pgfpathcurveto{\pgfqpoint{0.868651in}{1.493117in}}{\pgfqpoint{0.876551in}{1.496389in}}{\pgfqpoint{0.882375in}{1.502213in}}%
\pgfpathcurveto{\pgfqpoint{0.888199in}{1.508037in}}{\pgfqpoint{0.891471in}{1.515937in}}{\pgfqpoint{0.891471in}{1.524173in}}%
\pgfpathcurveto{\pgfqpoint{0.891471in}{1.532410in}}{\pgfqpoint{0.888199in}{1.540310in}}{\pgfqpoint{0.882375in}{1.546134in}}%
\pgfpathcurveto{\pgfqpoint{0.876551in}{1.551958in}}{\pgfqpoint{0.868651in}{1.555230in}}{\pgfqpoint{0.860415in}{1.555230in}}%
\pgfpathcurveto{\pgfqpoint{0.852178in}{1.555230in}}{\pgfqpoint{0.844278in}{1.551958in}}{\pgfqpoint{0.838454in}{1.546134in}}%
\pgfpathcurveto{\pgfqpoint{0.832631in}{1.540310in}}{\pgfqpoint{0.829358in}{1.532410in}}{\pgfqpoint{0.829358in}{1.524173in}}%
\pgfpathcurveto{\pgfqpoint{0.829358in}{1.515937in}}{\pgfqpoint{0.832631in}{1.508037in}}{\pgfqpoint{0.838454in}{1.502213in}}%
\pgfpathcurveto{\pgfqpoint{0.844278in}{1.496389in}}{\pgfqpoint{0.852178in}{1.493117in}}{\pgfqpoint{0.860415in}{1.493117in}}%
\pgfpathclose%
\pgfusepath{stroke,fill}%
\end{pgfscope}%
\begin{pgfscope}%
\pgfpathrectangle{\pgfqpoint{0.100000in}{0.212622in}}{\pgfqpoint{3.696000in}{3.696000in}}%
\pgfusepath{clip}%
\pgfsetbuttcap%
\pgfsetroundjoin%
\definecolor{currentfill}{rgb}{0.121569,0.466667,0.705882}%
\pgfsetfillcolor{currentfill}%
\pgfsetfillopacity{0.627433}%
\pgfsetlinewidth{1.003750pt}%
\definecolor{currentstroke}{rgb}{0.121569,0.466667,0.705882}%
\pgfsetstrokecolor{currentstroke}%
\pgfsetstrokeopacity{0.627433}%
\pgfsetdash{}{0pt}%
\pgfpathmoveto{\pgfqpoint{0.860415in}{1.493117in}}%
\pgfpathcurveto{\pgfqpoint{0.868651in}{1.493117in}}{\pgfqpoint{0.876551in}{1.496389in}}{\pgfqpoint{0.882375in}{1.502213in}}%
\pgfpathcurveto{\pgfqpoint{0.888199in}{1.508037in}}{\pgfqpoint{0.891471in}{1.515937in}}{\pgfqpoint{0.891471in}{1.524173in}}%
\pgfpathcurveto{\pgfqpoint{0.891471in}{1.532410in}}{\pgfqpoint{0.888199in}{1.540310in}}{\pgfqpoint{0.882375in}{1.546134in}}%
\pgfpathcurveto{\pgfqpoint{0.876551in}{1.551958in}}{\pgfqpoint{0.868651in}{1.555230in}}{\pgfqpoint{0.860415in}{1.555230in}}%
\pgfpathcurveto{\pgfqpoint{0.852178in}{1.555230in}}{\pgfqpoint{0.844278in}{1.551958in}}{\pgfqpoint{0.838454in}{1.546134in}}%
\pgfpathcurveto{\pgfqpoint{0.832631in}{1.540310in}}{\pgfqpoint{0.829358in}{1.532410in}}{\pgfqpoint{0.829358in}{1.524173in}}%
\pgfpathcurveto{\pgfqpoint{0.829358in}{1.515937in}}{\pgfqpoint{0.832631in}{1.508037in}}{\pgfqpoint{0.838454in}{1.502213in}}%
\pgfpathcurveto{\pgfqpoint{0.844278in}{1.496389in}}{\pgfqpoint{0.852178in}{1.493117in}}{\pgfqpoint{0.860415in}{1.493117in}}%
\pgfpathclose%
\pgfusepath{stroke,fill}%
\end{pgfscope}%
\begin{pgfscope}%
\pgfpathrectangle{\pgfqpoint{0.100000in}{0.212622in}}{\pgfqpoint{3.696000in}{3.696000in}}%
\pgfusepath{clip}%
\pgfsetbuttcap%
\pgfsetroundjoin%
\definecolor{currentfill}{rgb}{0.121569,0.466667,0.705882}%
\pgfsetfillcolor{currentfill}%
\pgfsetfillopacity{0.627433}%
\pgfsetlinewidth{1.003750pt}%
\definecolor{currentstroke}{rgb}{0.121569,0.466667,0.705882}%
\pgfsetstrokecolor{currentstroke}%
\pgfsetstrokeopacity{0.627433}%
\pgfsetdash{}{0pt}%
\pgfpathmoveto{\pgfqpoint{0.860415in}{1.493117in}}%
\pgfpathcurveto{\pgfqpoint{0.868651in}{1.493117in}}{\pgfqpoint{0.876551in}{1.496389in}}{\pgfqpoint{0.882375in}{1.502213in}}%
\pgfpathcurveto{\pgfqpoint{0.888199in}{1.508037in}}{\pgfqpoint{0.891471in}{1.515937in}}{\pgfqpoint{0.891471in}{1.524173in}}%
\pgfpathcurveto{\pgfqpoint{0.891471in}{1.532410in}}{\pgfqpoint{0.888199in}{1.540310in}}{\pgfqpoint{0.882375in}{1.546134in}}%
\pgfpathcurveto{\pgfqpoint{0.876551in}{1.551958in}}{\pgfqpoint{0.868651in}{1.555230in}}{\pgfqpoint{0.860415in}{1.555230in}}%
\pgfpathcurveto{\pgfqpoint{0.852178in}{1.555230in}}{\pgfqpoint{0.844278in}{1.551958in}}{\pgfqpoint{0.838454in}{1.546134in}}%
\pgfpathcurveto{\pgfqpoint{0.832631in}{1.540310in}}{\pgfqpoint{0.829358in}{1.532410in}}{\pgfqpoint{0.829358in}{1.524173in}}%
\pgfpathcurveto{\pgfqpoint{0.829358in}{1.515937in}}{\pgfqpoint{0.832631in}{1.508037in}}{\pgfqpoint{0.838454in}{1.502213in}}%
\pgfpathcurveto{\pgfqpoint{0.844278in}{1.496389in}}{\pgfqpoint{0.852178in}{1.493117in}}{\pgfqpoint{0.860415in}{1.493117in}}%
\pgfpathclose%
\pgfusepath{stroke,fill}%
\end{pgfscope}%
\begin{pgfscope}%
\pgfpathrectangle{\pgfqpoint{0.100000in}{0.212622in}}{\pgfqpoint{3.696000in}{3.696000in}}%
\pgfusepath{clip}%
\pgfsetbuttcap%
\pgfsetroundjoin%
\definecolor{currentfill}{rgb}{0.121569,0.466667,0.705882}%
\pgfsetfillcolor{currentfill}%
\pgfsetfillopacity{0.627433}%
\pgfsetlinewidth{1.003750pt}%
\definecolor{currentstroke}{rgb}{0.121569,0.466667,0.705882}%
\pgfsetstrokecolor{currentstroke}%
\pgfsetstrokeopacity{0.627433}%
\pgfsetdash{}{0pt}%
\pgfpathmoveto{\pgfqpoint{0.860415in}{1.493117in}}%
\pgfpathcurveto{\pgfqpoint{0.868651in}{1.493117in}}{\pgfqpoint{0.876551in}{1.496389in}}{\pgfqpoint{0.882375in}{1.502213in}}%
\pgfpathcurveto{\pgfqpoint{0.888199in}{1.508037in}}{\pgfqpoint{0.891471in}{1.515937in}}{\pgfqpoint{0.891471in}{1.524173in}}%
\pgfpathcurveto{\pgfqpoint{0.891471in}{1.532410in}}{\pgfqpoint{0.888199in}{1.540310in}}{\pgfqpoint{0.882375in}{1.546134in}}%
\pgfpathcurveto{\pgfqpoint{0.876551in}{1.551958in}}{\pgfqpoint{0.868651in}{1.555230in}}{\pgfqpoint{0.860415in}{1.555230in}}%
\pgfpathcurveto{\pgfqpoint{0.852178in}{1.555230in}}{\pgfqpoint{0.844278in}{1.551958in}}{\pgfqpoint{0.838454in}{1.546134in}}%
\pgfpathcurveto{\pgfqpoint{0.832631in}{1.540310in}}{\pgfqpoint{0.829358in}{1.532410in}}{\pgfqpoint{0.829358in}{1.524173in}}%
\pgfpathcurveto{\pgfqpoint{0.829358in}{1.515937in}}{\pgfqpoint{0.832631in}{1.508037in}}{\pgfqpoint{0.838454in}{1.502213in}}%
\pgfpathcurveto{\pgfqpoint{0.844278in}{1.496389in}}{\pgfqpoint{0.852178in}{1.493117in}}{\pgfqpoint{0.860415in}{1.493117in}}%
\pgfpathclose%
\pgfusepath{stroke,fill}%
\end{pgfscope}%
\begin{pgfscope}%
\pgfpathrectangle{\pgfqpoint{0.100000in}{0.212622in}}{\pgfqpoint{3.696000in}{3.696000in}}%
\pgfusepath{clip}%
\pgfsetbuttcap%
\pgfsetroundjoin%
\definecolor{currentfill}{rgb}{0.121569,0.466667,0.705882}%
\pgfsetfillcolor{currentfill}%
\pgfsetfillopacity{0.627433}%
\pgfsetlinewidth{1.003750pt}%
\definecolor{currentstroke}{rgb}{0.121569,0.466667,0.705882}%
\pgfsetstrokecolor{currentstroke}%
\pgfsetstrokeopacity{0.627433}%
\pgfsetdash{}{0pt}%
\pgfpathmoveto{\pgfqpoint{0.860415in}{1.493117in}}%
\pgfpathcurveto{\pgfqpoint{0.868651in}{1.493117in}}{\pgfqpoint{0.876551in}{1.496389in}}{\pgfqpoint{0.882375in}{1.502213in}}%
\pgfpathcurveto{\pgfqpoint{0.888199in}{1.508037in}}{\pgfqpoint{0.891471in}{1.515937in}}{\pgfqpoint{0.891471in}{1.524173in}}%
\pgfpathcurveto{\pgfqpoint{0.891471in}{1.532410in}}{\pgfqpoint{0.888199in}{1.540310in}}{\pgfqpoint{0.882375in}{1.546134in}}%
\pgfpathcurveto{\pgfqpoint{0.876551in}{1.551958in}}{\pgfqpoint{0.868651in}{1.555230in}}{\pgfqpoint{0.860415in}{1.555230in}}%
\pgfpathcurveto{\pgfqpoint{0.852178in}{1.555230in}}{\pgfqpoint{0.844278in}{1.551958in}}{\pgfqpoint{0.838454in}{1.546134in}}%
\pgfpathcurveto{\pgfqpoint{0.832631in}{1.540310in}}{\pgfqpoint{0.829358in}{1.532410in}}{\pgfqpoint{0.829358in}{1.524173in}}%
\pgfpathcurveto{\pgfqpoint{0.829358in}{1.515937in}}{\pgfqpoint{0.832631in}{1.508037in}}{\pgfqpoint{0.838454in}{1.502213in}}%
\pgfpathcurveto{\pgfqpoint{0.844278in}{1.496389in}}{\pgfqpoint{0.852178in}{1.493117in}}{\pgfqpoint{0.860415in}{1.493117in}}%
\pgfpathclose%
\pgfusepath{stroke,fill}%
\end{pgfscope}%
\begin{pgfscope}%
\pgfpathrectangle{\pgfqpoint{0.100000in}{0.212622in}}{\pgfqpoint{3.696000in}{3.696000in}}%
\pgfusepath{clip}%
\pgfsetbuttcap%
\pgfsetroundjoin%
\definecolor{currentfill}{rgb}{0.121569,0.466667,0.705882}%
\pgfsetfillcolor{currentfill}%
\pgfsetfillopacity{0.627433}%
\pgfsetlinewidth{1.003750pt}%
\definecolor{currentstroke}{rgb}{0.121569,0.466667,0.705882}%
\pgfsetstrokecolor{currentstroke}%
\pgfsetstrokeopacity{0.627433}%
\pgfsetdash{}{0pt}%
\pgfpathmoveto{\pgfqpoint{0.860415in}{1.493117in}}%
\pgfpathcurveto{\pgfqpoint{0.868651in}{1.493117in}}{\pgfqpoint{0.876551in}{1.496389in}}{\pgfqpoint{0.882375in}{1.502213in}}%
\pgfpathcurveto{\pgfqpoint{0.888199in}{1.508037in}}{\pgfqpoint{0.891471in}{1.515937in}}{\pgfqpoint{0.891471in}{1.524173in}}%
\pgfpathcurveto{\pgfqpoint{0.891471in}{1.532410in}}{\pgfqpoint{0.888199in}{1.540310in}}{\pgfqpoint{0.882375in}{1.546134in}}%
\pgfpathcurveto{\pgfqpoint{0.876551in}{1.551958in}}{\pgfqpoint{0.868651in}{1.555230in}}{\pgfqpoint{0.860415in}{1.555230in}}%
\pgfpathcurveto{\pgfqpoint{0.852178in}{1.555230in}}{\pgfqpoint{0.844278in}{1.551958in}}{\pgfqpoint{0.838454in}{1.546134in}}%
\pgfpathcurveto{\pgfqpoint{0.832631in}{1.540310in}}{\pgfqpoint{0.829358in}{1.532410in}}{\pgfqpoint{0.829358in}{1.524173in}}%
\pgfpathcurveto{\pgfqpoint{0.829358in}{1.515937in}}{\pgfqpoint{0.832631in}{1.508037in}}{\pgfqpoint{0.838454in}{1.502213in}}%
\pgfpathcurveto{\pgfqpoint{0.844278in}{1.496389in}}{\pgfqpoint{0.852178in}{1.493117in}}{\pgfqpoint{0.860415in}{1.493117in}}%
\pgfpathclose%
\pgfusepath{stroke,fill}%
\end{pgfscope}%
\begin{pgfscope}%
\pgfpathrectangle{\pgfqpoint{0.100000in}{0.212622in}}{\pgfqpoint{3.696000in}{3.696000in}}%
\pgfusepath{clip}%
\pgfsetbuttcap%
\pgfsetroundjoin%
\definecolor{currentfill}{rgb}{0.121569,0.466667,0.705882}%
\pgfsetfillcolor{currentfill}%
\pgfsetfillopacity{0.627433}%
\pgfsetlinewidth{1.003750pt}%
\definecolor{currentstroke}{rgb}{0.121569,0.466667,0.705882}%
\pgfsetstrokecolor{currentstroke}%
\pgfsetstrokeopacity{0.627433}%
\pgfsetdash{}{0pt}%
\pgfpathmoveto{\pgfqpoint{0.860415in}{1.493117in}}%
\pgfpathcurveto{\pgfqpoint{0.868651in}{1.493117in}}{\pgfqpoint{0.876551in}{1.496389in}}{\pgfqpoint{0.882375in}{1.502213in}}%
\pgfpathcurveto{\pgfqpoint{0.888199in}{1.508037in}}{\pgfqpoint{0.891471in}{1.515937in}}{\pgfqpoint{0.891471in}{1.524173in}}%
\pgfpathcurveto{\pgfqpoint{0.891471in}{1.532410in}}{\pgfqpoint{0.888199in}{1.540310in}}{\pgfqpoint{0.882375in}{1.546134in}}%
\pgfpathcurveto{\pgfqpoint{0.876551in}{1.551958in}}{\pgfqpoint{0.868651in}{1.555230in}}{\pgfqpoint{0.860415in}{1.555230in}}%
\pgfpathcurveto{\pgfqpoint{0.852178in}{1.555230in}}{\pgfqpoint{0.844278in}{1.551958in}}{\pgfqpoint{0.838454in}{1.546134in}}%
\pgfpathcurveto{\pgfqpoint{0.832631in}{1.540310in}}{\pgfqpoint{0.829358in}{1.532410in}}{\pgfqpoint{0.829358in}{1.524173in}}%
\pgfpathcurveto{\pgfqpoint{0.829358in}{1.515937in}}{\pgfqpoint{0.832631in}{1.508037in}}{\pgfqpoint{0.838454in}{1.502213in}}%
\pgfpathcurveto{\pgfqpoint{0.844278in}{1.496389in}}{\pgfqpoint{0.852178in}{1.493117in}}{\pgfqpoint{0.860415in}{1.493117in}}%
\pgfpathclose%
\pgfusepath{stroke,fill}%
\end{pgfscope}%
\begin{pgfscope}%
\pgfpathrectangle{\pgfqpoint{0.100000in}{0.212622in}}{\pgfqpoint{3.696000in}{3.696000in}}%
\pgfusepath{clip}%
\pgfsetbuttcap%
\pgfsetroundjoin%
\definecolor{currentfill}{rgb}{0.121569,0.466667,0.705882}%
\pgfsetfillcolor{currentfill}%
\pgfsetfillopacity{0.627433}%
\pgfsetlinewidth{1.003750pt}%
\definecolor{currentstroke}{rgb}{0.121569,0.466667,0.705882}%
\pgfsetstrokecolor{currentstroke}%
\pgfsetstrokeopacity{0.627433}%
\pgfsetdash{}{0pt}%
\pgfpathmoveto{\pgfqpoint{0.860415in}{1.493117in}}%
\pgfpathcurveto{\pgfqpoint{0.868651in}{1.493117in}}{\pgfqpoint{0.876551in}{1.496389in}}{\pgfqpoint{0.882375in}{1.502213in}}%
\pgfpathcurveto{\pgfqpoint{0.888199in}{1.508037in}}{\pgfqpoint{0.891471in}{1.515937in}}{\pgfqpoint{0.891471in}{1.524173in}}%
\pgfpathcurveto{\pgfqpoint{0.891471in}{1.532410in}}{\pgfqpoint{0.888199in}{1.540310in}}{\pgfqpoint{0.882375in}{1.546134in}}%
\pgfpathcurveto{\pgfqpoint{0.876551in}{1.551958in}}{\pgfqpoint{0.868651in}{1.555230in}}{\pgfqpoint{0.860415in}{1.555230in}}%
\pgfpathcurveto{\pgfqpoint{0.852178in}{1.555230in}}{\pgfqpoint{0.844278in}{1.551958in}}{\pgfqpoint{0.838454in}{1.546134in}}%
\pgfpathcurveto{\pgfqpoint{0.832631in}{1.540310in}}{\pgfqpoint{0.829358in}{1.532410in}}{\pgfqpoint{0.829358in}{1.524173in}}%
\pgfpathcurveto{\pgfqpoint{0.829358in}{1.515937in}}{\pgfqpoint{0.832631in}{1.508037in}}{\pgfqpoint{0.838454in}{1.502213in}}%
\pgfpathcurveto{\pgfqpoint{0.844278in}{1.496389in}}{\pgfqpoint{0.852178in}{1.493117in}}{\pgfqpoint{0.860415in}{1.493117in}}%
\pgfpathclose%
\pgfusepath{stroke,fill}%
\end{pgfscope}%
\begin{pgfscope}%
\pgfpathrectangle{\pgfqpoint{0.100000in}{0.212622in}}{\pgfqpoint{3.696000in}{3.696000in}}%
\pgfusepath{clip}%
\pgfsetbuttcap%
\pgfsetroundjoin%
\definecolor{currentfill}{rgb}{0.121569,0.466667,0.705882}%
\pgfsetfillcolor{currentfill}%
\pgfsetfillopacity{0.627433}%
\pgfsetlinewidth{1.003750pt}%
\definecolor{currentstroke}{rgb}{0.121569,0.466667,0.705882}%
\pgfsetstrokecolor{currentstroke}%
\pgfsetstrokeopacity{0.627433}%
\pgfsetdash{}{0pt}%
\pgfpathmoveto{\pgfqpoint{0.860415in}{1.493117in}}%
\pgfpathcurveto{\pgfqpoint{0.868651in}{1.493117in}}{\pgfqpoint{0.876551in}{1.496389in}}{\pgfqpoint{0.882375in}{1.502213in}}%
\pgfpathcurveto{\pgfqpoint{0.888199in}{1.508037in}}{\pgfqpoint{0.891471in}{1.515937in}}{\pgfqpoint{0.891471in}{1.524173in}}%
\pgfpathcurveto{\pgfqpoint{0.891471in}{1.532410in}}{\pgfqpoint{0.888199in}{1.540310in}}{\pgfqpoint{0.882375in}{1.546134in}}%
\pgfpathcurveto{\pgfqpoint{0.876551in}{1.551958in}}{\pgfqpoint{0.868651in}{1.555230in}}{\pgfqpoint{0.860415in}{1.555230in}}%
\pgfpathcurveto{\pgfqpoint{0.852178in}{1.555230in}}{\pgfqpoint{0.844278in}{1.551958in}}{\pgfqpoint{0.838454in}{1.546134in}}%
\pgfpathcurveto{\pgfqpoint{0.832631in}{1.540310in}}{\pgfqpoint{0.829358in}{1.532410in}}{\pgfqpoint{0.829358in}{1.524173in}}%
\pgfpathcurveto{\pgfqpoint{0.829358in}{1.515937in}}{\pgfqpoint{0.832631in}{1.508037in}}{\pgfqpoint{0.838454in}{1.502213in}}%
\pgfpathcurveto{\pgfqpoint{0.844278in}{1.496389in}}{\pgfqpoint{0.852178in}{1.493117in}}{\pgfqpoint{0.860415in}{1.493117in}}%
\pgfpathclose%
\pgfusepath{stroke,fill}%
\end{pgfscope}%
\begin{pgfscope}%
\pgfpathrectangle{\pgfqpoint{0.100000in}{0.212622in}}{\pgfqpoint{3.696000in}{3.696000in}}%
\pgfusepath{clip}%
\pgfsetbuttcap%
\pgfsetroundjoin%
\definecolor{currentfill}{rgb}{0.121569,0.466667,0.705882}%
\pgfsetfillcolor{currentfill}%
\pgfsetfillopacity{0.627433}%
\pgfsetlinewidth{1.003750pt}%
\definecolor{currentstroke}{rgb}{0.121569,0.466667,0.705882}%
\pgfsetstrokecolor{currentstroke}%
\pgfsetstrokeopacity{0.627433}%
\pgfsetdash{}{0pt}%
\pgfpathmoveto{\pgfqpoint{0.860415in}{1.493117in}}%
\pgfpathcurveto{\pgfqpoint{0.868651in}{1.493117in}}{\pgfqpoint{0.876551in}{1.496389in}}{\pgfqpoint{0.882375in}{1.502213in}}%
\pgfpathcurveto{\pgfqpoint{0.888199in}{1.508037in}}{\pgfqpoint{0.891471in}{1.515937in}}{\pgfqpoint{0.891471in}{1.524173in}}%
\pgfpathcurveto{\pgfqpoint{0.891471in}{1.532410in}}{\pgfqpoint{0.888199in}{1.540310in}}{\pgfqpoint{0.882375in}{1.546134in}}%
\pgfpathcurveto{\pgfqpoint{0.876551in}{1.551958in}}{\pgfqpoint{0.868651in}{1.555230in}}{\pgfqpoint{0.860415in}{1.555230in}}%
\pgfpathcurveto{\pgfqpoint{0.852178in}{1.555230in}}{\pgfqpoint{0.844278in}{1.551958in}}{\pgfqpoint{0.838454in}{1.546134in}}%
\pgfpathcurveto{\pgfqpoint{0.832631in}{1.540310in}}{\pgfqpoint{0.829358in}{1.532410in}}{\pgfqpoint{0.829358in}{1.524173in}}%
\pgfpathcurveto{\pgfqpoint{0.829358in}{1.515937in}}{\pgfqpoint{0.832631in}{1.508037in}}{\pgfqpoint{0.838454in}{1.502213in}}%
\pgfpathcurveto{\pgfqpoint{0.844278in}{1.496389in}}{\pgfqpoint{0.852178in}{1.493117in}}{\pgfqpoint{0.860415in}{1.493117in}}%
\pgfpathclose%
\pgfusepath{stroke,fill}%
\end{pgfscope}%
\begin{pgfscope}%
\pgfpathrectangle{\pgfqpoint{0.100000in}{0.212622in}}{\pgfqpoint{3.696000in}{3.696000in}}%
\pgfusepath{clip}%
\pgfsetbuttcap%
\pgfsetroundjoin%
\definecolor{currentfill}{rgb}{0.121569,0.466667,0.705882}%
\pgfsetfillcolor{currentfill}%
\pgfsetfillopacity{0.627433}%
\pgfsetlinewidth{1.003750pt}%
\definecolor{currentstroke}{rgb}{0.121569,0.466667,0.705882}%
\pgfsetstrokecolor{currentstroke}%
\pgfsetstrokeopacity{0.627433}%
\pgfsetdash{}{0pt}%
\pgfpathmoveto{\pgfqpoint{0.860415in}{1.493117in}}%
\pgfpathcurveto{\pgfqpoint{0.868651in}{1.493117in}}{\pgfqpoint{0.876551in}{1.496389in}}{\pgfqpoint{0.882375in}{1.502213in}}%
\pgfpathcurveto{\pgfqpoint{0.888199in}{1.508037in}}{\pgfqpoint{0.891471in}{1.515937in}}{\pgfqpoint{0.891471in}{1.524173in}}%
\pgfpathcurveto{\pgfqpoint{0.891471in}{1.532410in}}{\pgfqpoint{0.888199in}{1.540310in}}{\pgfqpoint{0.882375in}{1.546134in}}%
\pgfpathcurveto{\pgfqpoint{0.876551in}{1.551958in}}{\pgfqpoint{0.868651in}{1.555230in}}{\pgfqpoint{0.860415in}{1.555230in}}%
\pgfpathcurveto{\pgfqpoint{0.852178in}{1.555230in}}{\pgfqpoint{0.844278in}{1.551958in}}{\pgfqpoint{0.838454in}{1.546134in}}%
\pgfpathcurveto{\pgfqpoint{0.832631in}{1.540310in}}{\pgfqpoint{0.829358in}{1.532410in}}{\pgfqpoint{0.829358in}{1.524173in}}%
\pgfpathcurveto{\pgfqpoint{0.829358in}{1.515937in}}{\pgfqpoint{0.832631in}{1.508037in}}{\pgfqpoint{0.838454in}{1.502213in}}%
\pgfpathcurveto{\pgfqpoint{0.844278in}{1.496389in}}{\pgfqpoint{0.852178in}{1.493117in}}{\pgfqpoint{0.860415in}{1.493117in}}%
\pgfpathclose%
\pgfusepath{stroke,fill}%
\end{pgfscope}%
\begin{pgfscope}%
\pgfpathrectangle{\pgfqpoint{0.100000in}{0.212622in}}{\pgfqpoint{3.696000in}{3.696000in}}%
\pgfusepath{clip}%
\pgfsetbuttcap%
\pgfsetroundjoin%
\definecolor{currentfill}{rgb}{0.121569,0.466667,0.705882}%
\pgfsetfillcolor{currentfill}%
\pgfsetfillopacity{0.627433}%
\pgfsetlinewidth{1.003750pt}%
\definecolor{currentstroke}{rgb}{0.121569,0.466667,0.705882}%
\pgfsetstrokecolor{currentstroke}%
\pgfsetstrokeopacity{0.627433}%
\pgfsetdash{}{0pt}%
\pgfpathmoveto{\pgfqpoint{0.860415in}{1.493117in}}%
\pgfpathcurveto{\pgfqpoint{0.868651in}{1.493117in}}{\pgfqpoint{0.876551in}{1.496389in}}{\pgfqpoint{0.882375in}{1.502213in}}%
\pgfpathcurveto{\pgfqpoint{0.888199in}{1.508037in}}{\pgfqpoint{0.891471in}{1.515937in}}{\pgfqpoint{0.891471in}{1.524173in}}%
\pgfpathcurveto{\pgfqpoint{0.891471in}{1.532410in}}{\pgfqpoint{0.888199in}{1.540310in}}{\pgfqpoint{0.882375in}{1.546134in}}%
\pgfpathcurveto{\pgfqpoint{0.876551in}{1.551958in}}{\pgfqpoint{0.868651in}{1.555230in}}{\pgfqpoint{0.860415in}{1.555230in}}%
\pgfpathcurveto{\pgfqpoint{0.852178in}{1.555230in}}{\pgfqpoint{0.844278in}{1.551958in}}{\pgfqpoint{0.838454in}{1.546134in}}%
\pgfpathcurveto{\pgfqpoint{0.832631in}{1.540310in}}{\pgfqpoint{0.829358in}{1.532410in}}{\pgfqpoint{0.829358in}{1.524173in}}%
\pgfpathcurveto{\pgfqpoint{0.829358in}{1.515937in}}{\pgfqpoint{0.832631in}{1.508037in}}{\pgfqpoint{0.838454in}{1.502213in}}%
\pgfpathcurveto{\pgfqpoint{0.844278in}{1.496389in}}{\pgfqpoint{0.852178in}{1.493117in}}{\pgfqpoint{0.860415in}{1.493117in}}%
\pgfpathclose%
\pgfusepath{stroke,fill}%
\end{pgfscope}%
\begin{pgfscope}%
\pgfpathrectangle{\pgfqpoint{0.100000in}{0.212622in}}{\pgfqpoint{3.696000in}{3.696000in}}%
\pgfusepath{clip}%
\pgfsetbuttcap%
\pgfsetroundjoin%
\definecolor{currentfill}{rgb}{0.121569,0.466667,0.705882}%
\pgfsetfillcolor{currentfill}%
\pgfsetfillopacity{0.627433}%
\pgfsetlinewidth{1.003750pt}%
\definecolor{currentstroke}{rgb}{0.121569,0.466667,0.705882}%
\pgfsetstrokecolor{currentstroke}%
\pgfsetstrokeopacity{0.627433}%
\pgfsetdash{}{0pt}%
\pgfpathmoveto{\pgfqpoint{0.860415in}{1.493117in}}%
\pgfpathcurveto{\pgfqpoint{0.868651in}{1.493117in}}{\pgfqpoint{0.876551in}{1.496389in}}{\pgfqpoint{0.882375in}{1.502213in}}%
\pgfpathcurveto{\pgfqpoint{0.888199in}{1.508037in}}{\pgfqpoint{0.891471in}{1.515937in}}{\pgfqpoint{0.891471in}{1.524173in}}%
\pgfpathcurveto{\pgfqpoint{0.891471in}{1.532410in}}{\pgfqpoint{0.888199in}{1.540310in}}{\pgfqpoint{0.882375in}{1.546134in}}%
\pgfpathcurveto{\pgfqpoint{0.876551in}{1.551958in}}{\pgfqpoint{0.868651in}{1.555230in}}{\pgfqpoint{0.860415in}{1.555230in}}%
\pgfpathcurveto{\pgfqpoint{0.852178in}{1.555230in}}{\pgfqpoint{0.844278in}{1.551958in}}{\pgfqpoint{0.838454in}{1.546134in}}%
\pgfpathcurveto{\pgfqpoint{0.832631in}{1.540310in}}{\pgfqpoint{0.829358in}{1.532410in}}{\pgfqpoint{0.829358in}{1.524173in}}%
\pgfpathcurveto{\pgfqpoint{0.829358in}{1.515937in}}{\pgfqpoint{0.832631in}{1.508037in}}{\pgfqpoint{0.838454in}{1.502213in}}%
\pgfpathcurveto{\pgfqpoint{0.844278in}{1.496389in}}{\pgfqpoint{0.852178in}{1.493117in}}{\pgfqpoint{0.860415in}{1.493117in}}%
\pgfpathclose%
\pgfusepath{stroke,fill}%
\end{pgfscope}%
\begin{pgfscope}%
\pgfpathrectangle{\pgfqpoint{0.100000in}{0.212622in}}{\pgfqpoint{3.696000in}{3.696000in}}%
\pgfusepath{clip}%
\pgfsetbuttcap%
\pgfsetroundjoin%
\definecolor{currentfill}{rgb}{0.121569,0.466667,0.705882}%
\pgfsetfillcolor{currentfill}%
\pgfsetfillopacity{0.627433}%
\pgfsetlinewidth{1.003750pt}%
\definecolor{currentstroke}{rgb}{0.121569,0.466667,0.705882}%
\pgfsetstrokecolor{currentstroke}%
\pgfsetstrokeopacity{0.627433}%
\pgfsetdash{}{0pt}%
\pgfpathmoveto{\pgfqpoint{0.860415in}{1.493117in}}%
\pgfpathcurveto{\pgfqpoint{0.868651in}{1.493117in}}{\pgfqpoint{0.876551in}{1.496389in}}{\pgfqpoint{0.882375in}{1.502213in}}%
\pgfpathcurveto{\pgfqpoint{0.888199in}{1.508037in}}{\pgfqpoint{0.891471in}{1.515937in}}{\pgfqpoint{0.891471in}{1.524173in}}%
\pgfpathcurveto{\pgfqpoint{0.891471in}{1.532410in}}{\pgfqpoint{0.888199in}{1.540310in}}{\pgfqpoint{0.882375in}{1.546134in}}%
\pgfpathcurveto{\pgfqpoint{0.876551in}{1.551958in}}{\pgfqpoint{0.868651in}{1.555230in}}{\pgfqpoint{0.860415in}{1.555230in}}%
\pgfpathcurveto{\pgfqpoint{0.852178in}{1.555230in}}{\pgfqpoint{0.844278in}{1.551958in}}{\pgfqpoint{0.838454in}{1.546134in}}%
\pgfpathcurveto{\pgfqpoint{0.832631in}{1.540310in}}{\pgfqpoint{0.829358in}{1.532410in}}{\pgfqpoint{0.829358in}{1.524173in}}%
\pgfpathcurveto{\pgfqpoint{0.829358in}{1.515937in}}{\pgfqpoint{0.832631in}{1.508037in}}{\pgfqpoint{0.838454in}{1.502213in}}%
\pgfpathcurveto{\pgfqpoint{0.844278in}{1.496389in}}{\pgfqpoint{0.852178in}{1.493117in}}{\pgfqpoint{0.860415in}{1.493117in}}%
\pgfpathclose%
\pgfusepath{stroke,fill}%
\end{pgfscope}%
\begin{pgfscope}%
\pgfpathrectangle{\pgfqpoint{0.100000in}{0.212622in}}{\pgfqpoint{3.696000in}{3.696000in}}%
\pgfusepath{clip}%
\pgfsetbuttcap%
\pgfsetroundjoin%
\definecolor{currentfill}{rgb}{0.121569,0.466667,0.705882}%
\pgfsetfillcolor{currentfill}%
\pgfsetfillopacity{0.627433}%
\pgfsetlinewidth{1.003750pt}%
\definecolor{currentstroke}{rgb}{0.121569,0.466667,0.705882}%
\pgfsetstrokecolor{currentstroke}%
\pgfsetstrokeopacity{0.627433}%
\pgfsetdash{}{0pt}%
\pgfpathmoveto{\pgfqpoint{0.860415in}{1.493117in}}%
\pgfpathcurveto{\pgfqpoint{0.868651in}{1.493117in}}{\pgfqpoint{0.876551in}{1.496389in}}{\pgfqpoint{0.882375in}{1.502213in}}%
\pgfpathcurveto{\pgfqpoint{0.888199in}{1.508037in}}{\pgfqpoint{0.891471in}{1.515937in}}{\pgfqpoint{0.891471in}{1.524173in}}%
\pgfpathcurveto{\pgfqpoint{0.891471in}{1.532410in}}{\pgfqpoint{0.888199in}{1.540310in}}{\pgfqpoint{0.882375in}{1.546134in}}%
\pgfpathcurveto{\pgfqpoint{0.876551in}{1.551958in}}{\pgfqpoint{0.868651in}{1.555230in}}{\pgfqpoint{0.860415in}{1.555230in}}%
\pgfpathcurveto{\pgfqpoint{0.852178in}{1.555230in}}{\pgfqpoint{0.844278in}{1.551958in}}{\pgfqpoint{0.838454in}{1.546134in}}%
\pgfpathcurveto{\pgfqpoint{0.832631in}{1.540310in}}{\pgfqpoint{0.829358in}{1.532410in}}{\pgfqpoint{0.829358in}{1.524173in}}%
\pgfpathcurveto{\pgfqpoint{0.829358in}{1.515937in}}{\pgfqpoint{0.832631in}{1.508037in}}{\pgfqpoint{0.838454in}{1.502213in}}%
\pgfpathcurveto{\pgfqpoint{0.844278in}{1.496389in}}{\pgfqpoint{0.852178in}{1.493117in}}{\pgfqpoint{0.860415in}{1.493117in}}%
\pgfpathclose%
\pgfusepath{stroke,fill}%
\end{pgfscope}%
\begin{pgfscope}%
\pgfpathrectangle{\pgfqpoint{0.100000in}{0.212622in}}{\pgfqpoint{3.696000in}{3.696000in}}%
\pgfusepath{clip}%
\pgfsetbuttcap%
\pgfsetroundjoin%
\definecolor{currentfill}{rgb}{0.121569,0.466667,0.705882}%
\pgfsetfillcolor{currentfill}%
\pgfsetfillopacity{0.627433}%
\pgfsetlinewidth{1.003750pt}%
\definecolor{currentstroke}{rgb}{0.121569,0.466667,0.705882}%
\pgfsetstrokecolor{currentstroke}%
\pgfsetstrokeopacity{0.627433}%
\pgfsetdash{}{0pt}%
\pgfpathmoveto{\pgfqpoint{0.860415in}{1.493117in}}%
\pgfpathcurveto{\pgfqpoint{0.868651in}{1.493117in}}{\pgfqpoint{0.876551in}{1.496389in}}{\pgfqpoint{0.882375in}{1.502213in}}%
\pgfpathcurveto{\pgfqpoint{0.888199in}{1.508037in}}{\pgfqpoint{0.891471in}{1.515937in}}{\pgfqpoint{0.891471in}{1.524173in}}%
\pgfpathcurveto{\pgfqpoint{0.891471in}{1.532410in}}{\pgfqpoint{0.888199in}{1.540310in}}{\pgfqpoint{0.882375in}{1.546134in}}%
\pgfpathcurveto{\pgfqpoint{0.876551in}{1.551958in}}{\pgfqpoint{0.868651in}{1.555230in}}{\pgfqpoint{0.860415in}{1.555230in}}%
\pgfpathcurveto{\pgfqpoint{0.852178in}{1.555230in}}{\pgfqpoint{0.844278in}{1.551958in}}{\pgfqpoint{0.838454in}{1.546134in}}%
\pgfpathcurveto{\pgfqpoint{0.832631in}{1.540310in}}{\pgfqpoint{0.829358in}{1.532410in}}{\pgfqpoint{0.829358in}{1.524173in}}%
\pgfpathcurveto{\pgfqpoint{0.829358in}{1.515937in}}{\pgfqpoint{0.832631in}{1.508037in}}{\pgfqpoint{0.838454in}{1.502213in}}%
\pgfpathcurveto{\pgfqpoint{0.844278in}{1.496389in}}{\pgfqpoint{0.852178in}{1.493117in}}{\pgfqpoint{0.860415in}{1.493117in}}%
\pgfpathclose%
\pgfusepath{stroke,fill}%
\end{pgfscope}%
\begin{pgfscope}%
\pgfpathrectangle{\pgfqpoint{0.100000in}{0.212622in}}{\pgfqpoint{3.696000in}{3.696000in}}%
\pgfusepath{clip}%
\pgfsetbuttcap%
\pgfsetroundjoin%
\definecolor{currentfill}{rgb}{0.121569,0.466667,0.705882}%
\pgfsetfillcolor{currentfill}%
\pgfsetfillopacity{0.627433}%
\pgfsetlinewidth{1.003750pt}%
\definecolor{currentstroke}{rgb}{0.121569,0.466667,0.705882}%
\pgfsetstrokecolor{currentstroke}%
\pgfsetstrokeopacity{0.627433}%
\pgfsetdash{}{0pt}%
\pgfpathmoveto{\pgfqpoint{0.860415in}{1.493117in}}%
\pgfpathcurveto{\pgfqpoint{0.868651in}{1.493117in}}{\pgfqpoint{0.876551in}{1.496389in}}{\pgfqpoint{0.882375in}{1.502213in}}%
\pgfpathcurveto{\pgfqpoint{0.888199in}{1.508037in}}{\pgfqpoint{0.891471in}{1.515937in}}{\pgfqpoint{0.891471in}{1.524173in}}%
\pgfpathcurveto{\pgfqpoint{0.891471in}{1.532410in}}{\pgfqpoint{0.888199in}{1.540310in}}{\pgfqpoint{0.882375in}{1.546134in}}%
\pgfpathcurveto{\pgfqpoint{0.876551in}{1.551958in}}{\pgfqpoint{0.868651in}{1.555230in}}{\pgfqpoint{0.860415in}{1.555230in}}%
\pgfpathcurveto{\pgfqpoint{0.852178in}{1.555230in}}{\pgfqpoint{0.844278in}{1.551958in}}{\pgfqpoint{0.838454in}{1.546134in}}%
\pgfpathcurveto{\pgfqpoint{0.832631in}{1.540310in}}{\pgfqpoint{0.829358in}{1.532410in}}{\pgfqpoint{0.829358in}{1.524173in}}%
\pgfpathcurveto{\pgfqpoint{0.829358in}{1.515937in}}{\pgfqpoint{0.832631in}{1.508037in}}{\pgfqpoint{0.838454in}{1.502213in}}%
\pgfpathcurveto{\pgfqpoint{0.844278in}{1.496389in}}{\pgfqpoint{0.852178in}{1.493117in}}{\pgfqpoint{0.860415in}{1.493117in}}%
\pgfpathclose%
\pgfusepath{stroke,fill}%
\end{pgfscope}%
\begin{pgfscope}%
\pgfpathrectangle{\pgfqpoint{0.100000in}{0.212622in}}{\pgfqpoint{3.696000in}{3.696000in}}%
\pgfusepath{clip}%
\pgfsetbuttcap%
\pgfsetroundjoin%
\definecolor{currentfill}{rgb}{0.121569,0.466667,0.705882}%
\pgfsetfillcolor{currentfill}%
\pgfsetfillopacity{0.627433}%
\pgfsetlinewidth{1.003750pt}%
\definecolor{currentstroke}{rgb}{0.121569,0.466667,0.705882}%
\pgfsetstrokecolor{currentstroke}%
\pgfsetstrokeopacity{0.627433}%
\pgfsetdash{}{0pt}%
\pgfpathmoveto{\pgfqpoint{0.860415in}{1.493117in}}%
\pgfpathcurveto{\pgfqpoint{0.868651in}{1.493117in}}{\pgfqpoint{0.876551in}{1.496389in}}{\pgfqpoint{0.882375in}{1.502213in}}%
\pgfpathcurveto{\pgfqpoint{0.888199in}{1.508037in}}{\pgfqpoint{0.891471in}{1.515937in}}{\pgfqpoint{0.891471in}{1.524173in}}%
\pgfpathcurveto{\pgfqpoint{0.891471in}{1.532410in}}{\pgfqpoint{0.888199in}{1.540310in}}{\pgfqpoint{0.882375in}{1.546134in}}%
\pgfpathcurveto{\pgfqpoint{0.876551in}{1.551958in}}{\pgfqpoint{0.868651in}{1.555230in}}{\pgfqpoint{0.860415in}{1.555230in}}%
\pgfpathcurveto{\pgfqpoint{0.852178in}{1.555230in}}{\pgfqpoint{0.844278in}{1.551958in}}{\pgfqpoint{0.838454in}{1.546134in}}%
\pgfpathcurveto{\pgfqpoint{0.832631in}{1.540310in}}{\pgfqpoint{0.829358in}{1.532410in}}{\pgfqpoint{0.829358in}{1.524173in}}%
\pgfpathcurveto{\pgfqpoint{0.829358in}{1.515937in}}{\pgfqpoint{0.832631in}{1.508037in}}{\pgfqpoint{0.838454in}{1.502213in}}%
\pgfpathcurveto{\pgfqpoint{0.844278in}{1.496389in}}{\pgfqpoint{0.852178in}{1.493117in}}{\pgfqpoint{0.860415in}{1.493117in}}%
\pgfpathclose%
\pgfusepath{stroke,fill}%
\end{pgfscope}%
\begin{pgfscope}%
\pgfpathrectangle{\pgfqpoint{0.100000in}{0.212622in}}{\pgfqpoint{3.696000in}{3.696000in}}%
\pgfusepath{clip}%
\pgfsetbuttcap%
\pgfsetroundjoin%
\definecolor{currentfill}{rgb}{0.121569,0.466667,0.705882}%
\pgfsetfillcolor{currentfill}%
\pgfsetfillopacity{0.627433}%
\pgfsetlinewidth{1.003750pt}%
\definecolor{currentstroke}{rgb}{0.121569,0.466667,0.705882}%
\pgfsetstrokecolor{currentstroke}%
\pgfsetstrokeopacity{0.627433}%
\pgfsetdash{}{0pt}%
\pgfpathmoveto{\pgfqpoint{0.860415in}{1.493117in}}%
\pgfpathcurveto{\pgfqpoint{0.868651in}{1.493117in}}{\pgfqpoint{0.876551in}{1.496389in}}{\pgfqpoint{0.882375in}{1.502213in}}%
\pgfpathcurveto{\pgfqpoint{0.888199in}{1.508037in}}{\pgfqpoint{0.891471in}{1.515937in}}{\pgfqpoint{0.891471in}{1.524173in}}%
\pgfpathcurveto{\pgfqpoint{0.891471in}{1.532410in}}{\pgfqpoint{0.888199in}{1.540310in}}{\pgfqpoint{0.882375in}{1.546134in}}%
\pgfpathcurveto{\pgfqpoint{0.876551in}{1.551958in}}{\pgfqpoint{0.868651in}{1.555230in}}{\pgfqpoint{0.860415in}{1.555230in}}%
\pgfpathcurveto{\pgfqpoint{0.852178in}{1.555230in}}{\pgfqpoint{0.844278in}{1.551958in}}{\pgfqpoint{0.838454in}{1.546134in}}%
\pgfpathcurveto{\pgfqpoint{0.832631in}{1.540310in}}{\pgfqpoint{0.829358in}{1.532410in}}{\pgfqpoint{0.829358in}{1.524173in}}%
\pgfpathcurveto{\pgfqpoint{0.829358in}{1.515937in}}{\pgfqpoint{0.832631in}{1.508037in}}{\pgfqpoint{0.838454in}{1.502213in}}%
\pgfpathcurveto{\pgfqpoint{0.844278in}{1.496389in}}{\pgfqpoint{0.852178in}{1.493117in}}{\pgfqpoint{0.860415in}{1.493117in}}%
\pgfpathclose%
\pgfusepath{stroke,fill}%
\end{pgfscope}%
\begin{pgfscope}%
\pgfpathrectangle{\pgfqpoint{0.100000in}{0.212622in}}{\pgfqpoint{3.696000in}{3.696000in}}%
\pgfusepath{clip}%
\pgfsetbuttcap%
\pgfsetroundjoin%
\definecolor{currentfill}{rgb}{0.121569,0.466667,0.705882}%
\pgfsetfillcolor{currentfill}%
\pgfsetfillopacity{0.627433}%
\pgfsetlinewidth{1.003750pt}%
\definecolor{currentstroke}{rgb}{0.121569,0.466667,0.705882}%
\pgfsetstrokecolor{currentstroke}%
\pgfsetstrokeopacity{0.627433}%
\pgfsetdash{}{0pt}%
\pgfpathmoveto{\pgfqpoint{0.860415in}{1.493117in}}%
\pgfpathcurveto{\pgfqpoint{0.868651in}{1.493117in}}{\pgfqpoint{0.876551in}{1.496389in}}{\pgfqpoint{0.882375in}{1.502213in}}%
\pgfpathcurveto{\pgfqpoint{0.888199in}{1.508037in}}{\pgfqpoint{0.891471in}{1.515937in}}{\pgfqpoint{0.891471in}{1.524173in}}%
\pgfpathcurveto{\pgfqpoint{0.891471in}{1.532410in}}{\pgfqpoint{0.888199in}{1.540310in}}{\pgfqpoint{0.882375in}{1.546134in}}%
\pgfpathcurveto{\pgfqpoint{0.876551in}{1.551958in}}{\pgfqpoint{0.868651in}{1.555230in}}{\pgfqpoint{0.860415in}{1.555230in}}%
\pgfpathcurveto{\pgfqpoint{0.852178in}{1.555230in}}{\pgfqpoint{0.844278in}{1.551958in}}{\pgfqpoint{0.838454in}{1.546134in}}%
\pgfpathcurveto{\pgfqpoint{0.832631in}{1.540310in}}{\pgfqpoint{0.829358in}{1.532410in}}{\pgfqpoint{0.829358in}{1.524173in}}%
\pgfpathcurveto{\pgfqpoint{0.829358in}{1.515937in}}{\pgfqpoint{0.832631in}{1.508037in}}{\pgfqpoint{0.838454in}{1.502213in}}%
\pgfpathcurveto{\pgfqpoint{0.844278in}{1.496389in}}{\pgfqpoint{0.852178in}{1.493117in}}{\pgfqpoint{0.860415in}{1.493117in}}%
\pgfpathclose%
\pgfusepath{stroke,fill}%
\end{pgfscope}%
\begin{pgfscope}%
\pgfpathrectangle{\pgfqpoint{0.100000in}{0.212622in}}{\pgfqpoint{3.696000in}{3.696000in}}%
\pgfusepath{clip}%
\pgfsetbuttcap%
\pgfsetroundjoin%
\definecolor{currentfill}{rgb}{0.121569,0.466667,0.705882}%
\pgfsetfillcolor{currentfill}%
\pgfsetfillopacity{0.627433}%
\pgfsetlinewidth{1.003750pt}%
\definecolor{currentstroke}{rgb}{0.121569,0.466667,0.705882}%
\pgfsetstrokecolor{currentstroke}%
\pgfsetstrokeopacity{0.627433}%
\pgfsetdash{}{0pt}%
\pgfpathmoveto{\pgfqpoint{0.860415in}{1.493117in}}%
\pgfpathcurveto{\pgfqpoint{0.868651in}{1.493117in}}{\pgfqpoint{0.876551in}{1.496389in}}{\pgfqpoint{0.882375in}{1.502213in}}%
\pgfpathcurveto{\pgfqpoint{0.888199in}{1.508037in}}{\pgfqpoint{0.891471in}{1.515937in}}{\pgfqpoint{0.891471in}{1.524173in}}%
\pgfpathcurveto{\pgfqpoint{0.891471in}{1.532410in}}{\pgfqpoint{0.888199in}{1.540310in}}{\pgfqpoint{0.882375in}{1.546134in}}%
\pgfpathcurveto{\pgfqpoint{0.876551in}{1.551958in}}{\pgfqpoint{0.868651in}{1.555230in}}{\pgfqpoint{0.860415in}{1.555230in}}%
\pgfpathcurveto{\pgfqpoint{0.852178in}{1.555230in}}{\pgfqpoint{0.844278in}{1.551958in}}{\pgfqpoint{0.838454in}{1.546134in}}%
\pgfpathcurveto{\pgfqpoint{0.832631in}{1.540310in}}{\pgfqpoint{0.829358in}{1.532410in}}{\pgfqpoint{0.829358in}{1.524173in}}%
\pgfpathcurveto{\pgfqpoint{0.829358in}{1.515937in}}{\pgfqpoint{0.832631in}{1.508037in}}{\pgfqpoint{0.838454in}{1.502213in}}%
\pgfpathcurveto{\pgfqpoint{0.844278in}{1.496389in}}{\pgfqpoint{0.852178in}{1.493117in}}{\pgfqpoint{0.860415in}{1.493117in}}%
\pgfpathclose%
\pgfusepath{stroke,fill}%
\end{pgfscope}%
\begin{pgfscope}%
\pgfpathrectangle{\pgfqpoint{0.100000in}{0.212622in}}{\pgfqpoint{3.696000in}{3.696000in}}%
\pgfusepath{clip}%
\pgfsetbuttcap%
\pgfsetroundjoin%
\definecolor{currentfill}{rgb}{0.121569,0.466667,0.705882}%
\pgfsetfillcolor{currentfill}%
\pgfsetfillopacity{0.627433}%
\pgfsetlinewidth{1.003750pt}%
\definecolor{currentstroke}{rgb}{0.121569,0.466667,0.705882}%
\pgfsetstrokecolor{currentstroke}%
\pgfsetstrokeopacity{0.627433}%
\pgfsetdash{}{0pt}%
\pgfpathmoveto{\pgfqpoint{0.860415in}{1.493117in}}%
\pgfpathcurveto{\pgfqpoint{0.868651in}{1.493117in}}{\pgfqpoint{0.876551in}{1.496389in}}{\pgfqpoint{0.882375in}{1.502213in}}%
\pgfpathcurveto{\pgfqpoint{0.888199in}{1.508037in}}{\pgfqpoint{0.891471in}{1.515937in}}{\pgfqpoint{0.891471in}{1.524173in}}%
\pgfpathcurveto{\pgfqpoint{0.891471in}{1.532410in}}{\pgfqpoint{0.888199in}{1.540310in}}{\pgfqpoint{0.882375in}{1.546134in}}%
\pgfpathcurveto{\pgfqpoint{0.876551in}{1.551958in}}{\pgfqpoint{0.868651in}{1.555230in}}{\pgfqpoint{0.860415in}{1.555230in}}%
\pgfpathcurveto{\pgfqpoint{0.852178in}{1.555230in}}{\pgfqpoint{0.844278in}{1.551958in}}{\pgfqpoint{0.838454in}{1.546134in}}%
\pgfpathcurveto{\pgfqpoint{0.832631in}{1.540310in}}{\pgfqpoint{0.829358in}{1.532410in}}{\pgfqpoint{0.829358in}{1.524173in}}%
\pgfpathcurveto{\pgfqpoint{0.829358in}{1.515937in}}{\pgfqpoint{0.832631in}{1.508037in}}{\pgfqpoint{0.838454in}{1.502213in}}%
\pgfpathcurveto{\pgfqpoint{0.844278in}{1.496389in}}{\pgfqpoint{0.852178in}{1.493117in}}{\pgfqpoint{0.860415in}{1.493117in}}%
\pgfpathclose%
\pgfusepath{stroke,fill}%
\end{pgfscope}%
\begin{pgfscope}%
\pgfpathrectangle{\pgfqpoint{0.100000in}{0.212622in}}{\pgfqpoint{3.696000in}{3.696000in}}%
\pgfusepath{clip}%
\pgfsetbuttcap%
\pgfsetroundjoin%
\definecolor{currentfill}{rgb}{0.121569,0.466667,0.705882}%
\pgfsetfillcolor{currentfill}%
\pgfsetfillopacity{0.627433}%
\pgfsetlinewidth{1.003750pt}%
\definecolor{currentstroke}{rgb}{0.121569,0.466667,0.705882}%
\pgfsetstrokecolor{currentstroke}%
\pgfsetstrokeopacity{0.627433}%
\pgfsetdash{}{0pt}%
\pgfpathmoveto{\pgfqpoint{0.860415in}{1.493117in}}%
\pgfpathcurveto{\pgfqpoint{0.868651in}{1.493117in}}{\pgfqpoint{0.876551in}{1.496389in}}{\pgfqpoint{0.882375in}{1.502213in}}%
\pgfpathcurveto{\pgfqpoint{0.888199in}{1.508037in}}{\pgfqpoint{0.891471in}{1.515937in}}{\pgfqpoint{0.891471in}{1.524173in}}%
\pgfpathcurveto{\pgfqpoint{0.891471in}{1.532410in}}{\pgfqpoint{0.888199in}{1.540310in}}{\pgfqpoint{0.882375in}{1.546134in}}%
\pgfpathcurveto{\pgfqpoint{0.876551in}{1.551958in}}{\pgfqpoint{0.868651in}{1.555230in}}{\pgfqpoint{0.860415in}{1.555230in}}%
\pgfpathcurveto{\pgfqpoint{0.852178in}{1.555230in}}{\pgfqpoint{0.844278in}{1.551958in}}{\pgfqpoint{0.838454in}{1.546134in}}%
\pgfpathcurveto{\pgfqpoint{0.832631in}{1.540310in}}{\pgfqpoint{0.829358in}{1.532410in}}{\pgfqpoint{0.829358in}{1.524173in}}%
\pgfpathcurveto{\pgfqpoint{0.829358in}{1.515937in}}{\pgfqpoint{0.832631in}{1.508037in}}{\pgfqpoint{0.838454in}{1.502213in}}%
\pgfpathcurveto{\pgfqpoint{0.844278in}{1.496389in}}{\pgfqpoint{0.852178in}{1.493117in}}{\pgfqpoint{0.860415in}{1.493117in}}%
\pgfpathclose%
\pgfusepath{stroke,fill}%
\end{pgfscope}%
\begin{pgfscope}%
\pgfpathrectangle{\pgfqpoint{0.100000in}{0.212622in}}{\pgfqpoint{3.696000in}{3.696000in}}%
\pgfusepath{clip}%
\pgfsetbuttcap%
\pgfsetroundjoin%
\definecolor{currentfill}{rgb}{0.121569,0.466667,0.705882}%
\pgfsetfillcolor{currentfill}%
\pgfsetfillopacity{0.627433}%
\pgfsetlinewidth{1.003750pt}%
\definecolor{currentstroke}{rgb}{0.121569,0.466667,0.705882}%
\pgfsetstrokecolor{currentstroke}%
\pgfsetstrokeopacity{0.627433}%
\pgfsetdash{}{0pt}%
\pgfpathmoveto{\pgfqpoint{0.860415in}{1.493117in}}%
\pgfpathcurveto{\pgfqpoint{0.868651in}{1.493117in}}{\pgfqpoint{0.876551in}{1.496389in}}{\pgfqpoint{0.882375in}{1.502213in}}%
\pgfpathcurveto{\pgfqpoint{0.888199in}{1.508037in}}{\pgfqpoint{0.891471in}{1.515937in}}{\pgfqpoint{0.891471in}{1.524173in}}%
\pgfpathcurveto{\pgfqpoint{0.891471in}{1.532410in}}{\pgfqpoint{0.888199in}{1.540310in}}{\pgfqpoint{0.882375in}{1.546134in}}%
\pgfpathcurveto{\pgfqpoint{0.876551in}{1.551958in}}{\pgfqpoint{0.868651in}{1.555230in}}{\pgfqpoint{0.860415in}{1.555230in}}%
\pgfpathcurveto{\pgfqpoint{0.852178in}{1.555230in}}{\pgfqpoint{0.844278in}{1.551958in}}{\pgfqpoint{0.838454in}{1.546134in}}%
\pgfpathcurveto{\pgfqpoint{0.832631in}{1.540310in}}{\pgfqpoint{0.829358in}{1.532410in}}{\pgfqpoint{0.829358in}{1.524173in}}%
\pgfpathcurveto{\pgfqpoint{0.829358in}{1.515937in}}{\pgfqpoint{0.832631in}{1.508037in}}{\pgfqpoint{0.838454in}{1.502213in}}%
\pgfpathcurveto{\pgfqpoint{0.844278in}{1.496389in}}{\pgfqpoint{0.852178in}{1.493117in}}{\pgfqpoint{0.860415in}{1.493117in}}%
\pgfpathclose%
\pgfusepath{stroke,fill}%
\end{pgfscope}%
\begin{pgfscope}%
\pgfpathrectangle{\pgfqpoint{0.100000in}{0.212622in}}{\pgfqpoint{3.696000in}{3.696000in}}%
\pgfusepath{clip}%
\pgfsetbuttcap%
\pgfsetroundjoin%
\definecolor{currentfill}{rgb}{0.121569,0.466667,0.705882}%
\pgfsetfillcolor{currentfill}%
\pgfsetfillopacity{0.627433}%
\pgfsetlinewidth{1.003750pt}%
\definecolor{currentstroke}{rgb}{0.121569,0.466667,0.705882}%
\pgfsetstrokecolor{currentstroke}%
\pgfsetstrokeopacity{0.627433}%
\pgfsetdash{}{0pt}%
\pgfpathmoveto{\pgfqpoint{0.860415in}{1.493117in}}%
\pgfpathcurveto{\pgfqpoint{0.868651in}{1.493117in}}{\pgfqpoint{0.876551in}{1.496389in}}{\pgfqpoint{0.882375in}{1.502213in}}%
\pgfpathcurveto{\pgfqpoint{0.888199in}{1.508037in}}{\pgfqpoint{0.891471in}{1.515937in}}{\pgfqpoint{0.891471in}{1.524173in}}%
\pgfpathcurveto{\pgfqpoint{0.891471in}{1.532410in}}{\pgfqpoint{0.888199in}{1.540310in}}{\pgfqpoint{0.882375in}{1.546134in}}%
\pgfpathcurveto{\pgfqpoint{0.876551in}{1.551958in}}{\pgfqpoint{0.868651in}{1.555230in}}{\pgfqpoint{0.860415in}{1.555230in}}%
\pgfpathcurveto{\pgfqpoint{0.852178in}{1.555230in}}{\pgfqpoint{0.844278in}{1.551958in}}{\pgfqpoint{0.838454in}{1.546134in}}%
\pgfpathcurveto{\pgfqpoint{0.832631in}{1.540310in}}{\pgfqpoint{0.829358in}{1.532410in}}{\pgfqpoint{0.829358in}{1.524173in}}%
\pgfpathcurveto{\pgfqpoint{0.829358in}{1.515937in}}{\pgfqpoint{0.832631in}{1.508037in}}{\pgfqpoint{0.838454in}{1.502213in}}%
\pgfpathcurveto{\pgfqpoint{0.844278in}{1.496389in}}{\pgfqpoint{0.852178in}{1.493117in}}{\pgfqpoint{0.860415in}{1.493117in}}%
\pgfpathclose%
\pgfusepath{stroke,fill}%
\end{pgfscope}%
\begin{pgfscope}%
\pgfpathrectangle{\pgfqpoint{0.100000in}{0.212622in}}{\pgfqpoint{3.696000in}{3.696000in}}%
\pgfusepath{clip}%
\pgfsetbuttcap%
\pgfsetroundjoin%
\definecolor{currentfill}{rgb}{0.121569,0.466667,0.705882}%
\pgfsetfillcolor{currentfill}%
\pgfsetfillopacity{0.627433}%
\pgfsetlinewidth{1.003750pt}%
\definecolor{currentstroke}{rgb}{0.121569,0.466667,0.705882}%
\pgfsetstrokecolor{currentstroke}%
\pgfsetstrokeopacity{0.627433}%
\pgfsetdash{}{0pt}%
\pgfpathmoveto{\pgfqpoint{0.860415in}{1.493117in}}%
\pgfpathcurveto{\pgfqpoint{0.868651in}{1.493117in}}{\pgfqpoint{0.876551in}{1.496389in}}{\pgfqpoint{0.882375in}{1.502213in}}%
\pgfpathcurveto{\pgfqpoint{0.888199in}{1.508037in}}{\pgfqpoint{0.891471in}{1.515937in}}{\pgfqpoint{0.891471in}{1.524173in}}%
\pgfpathcurveto{\pgfqpoint{0.891471in}{1.532410in}}{\pgfqpoint{0.888199in}{1.540310in}}{\pgfqpoint{0.882375in}{1.546134in}}%
\pgfpathcurveto{\pgfqpoint{0.876551in}{1.551958in}}{\pgfqpoint{0.868651in}{1.555230in}}{\pgfqpoint{0.860415in}{1.555230in}}%
\pgfpathcurveto{\pgfqpoint{0.852178in}{1.555230in}}{\pgfqpoint{0.844278in}{1.551958in}}{\pgfqpoint{0.838454in}{1.546134in}}%
\pgfpathcurveto{\pgfqpoint{0.832631in}{1.540310in}}{\pgfqpoint{0.829358in}{1.532410in}}{\pgfqpoint{0.829358in}{1.524173in}}%
\pgfpathcurveto{\pgfqpoint{0.829358in}{1.515937in}}{\pgfqpoint{0.832631in}{1.508037in}}{\pgfqpoint{0.838454in}{1.502213in}}%
\pgfpathcurveto{\pgfqpoint{0.844278in}{1.496389in}}{\pgfqpoint{0.852178in}{1.493117in}}{\pgfqpoint{0.860415in}{1.493117in}}%
\pgfpathclose%
\pgfusepath{stroke,fill}%
\end{pgfscope}%
\begin{pgfscope}%
\pgfpathrectangle{\pgfqpoint{0.100000in}{0.212622in}}{\pgfqpoint{3.696000in}{3.696000in}}%
\pgfusepath{clip}%
\pgfsetbuttcap%
\pgfsetroundjoin%
\definecolor{currentfill}{rgb}{0.121569,0.466667,0.705882}%
\pgfsetfillcolor{currentfill}%
\pgfsetfillopacity{0.627433}%
\pgfsetlinewidth{1.003750pt}%
\definecolor{currentstroke}{rgb}{0.121569,0.466667,0.705882}%
\pgfsetstrokecolor{currentstroke}%
\pgfsetstrokeopacity{0.627433}%
\pgfsetdash{}{0pt}%
\pgfpathmoveto{\pgfqpoint{0.860415in}{1.493117in}}%
\pgfpathcurveto{\pgfqpoint{0.868651in}{1.493117in}}{\pgfqpoint{0.876551in}{1.496389in}}{\pgfqpoint{0.882375in}{1.502213in}}%
\pgfpathcurveto{\pgfqpoint{0.888199in}{1.508037in}}{\pgfqpoint{0.891471in}{1.515937in}}{\pgfqpoint{0.891471in}{1.524173in}}%
\pgfpathcurveto{\pgfqpoint{0.891471in}{1.532410in}}{\pgfqpoint{0.888199in}{1.540310in}}{\pgfqpoint{0.882375in}{1.546134in}}%
\pgfpathcurveto{\pgfqpoint{0.876551in}{1.551958in}}{\pgfqpoint{0.868651in}{1.555230in}}{\pgfqpoint{0.860415in}{1.555230in}}%
\pgfpathcurveto{\pgfqpoint{0.852178in}{1.555230in}}{\pgfqpoint{0.844278in}{1.551958in}}{\pgfqpoint{0.838454in}{1.546134in}}%
\pgfpathcurveto{\pgfqpoint{0.832631in}{1.540310in}}{\pgfqpoint{0.829358in}{1.532410in}}{\pgfqpoint{0.829358in}{1.524173in}}%
\pgfpathcurveto{\pgfqpoint{0.829358in}{1.515937in}}{\pgfqpoint{0.832631in}{1.508037in}}{\pgfqpoint{0.838454in}{1.502213in}}%
\pgfpathcurveto{\pgfqpoint{0.844278in}{1.496389in}}{\pgfqpoint{0.852178in}{1.493117in}}{\pgfqpoint{0.860415in}{1.493117in}}%
\pgfpathclose%
\pgfusepath{stroke,fill}%
\end{pgfscope}%
\begin{pgfscope}%
\pgfpathrectangle{\pgfqpoint{0.100000in}{0.212622in}}{\pgfqpoint{3.696000in}{3.696000in}}%
\pgfusepath{clip}%
\pgfsetbuttcap%
\pgfsetroundjoin%
\definecolor{currentfill}{rgb}{0.121569,0.466667,0.705882}%
\pgfsetfillcolor{currentfill}%
\pgfsetfillopacity{0.627433}%
\pgfsetlinewidth{1.003750pt}%
\definecolor{currentstroke}{rgb}{0.121569,0.466667,0.705882}%
\pgfsetstrokecolor{currentstroke}%
\pgfsetstrokeopacity{0.627433}%
\pgfsetdash{}{0pt}%
\pgfpathmoveto{\pgfqpoint{0.860415in}{1.493117in}}%
\pgfpathcurveto{\pgfqpoint{0.868651in}{1.493117in}}{\pgfqpoint{0.876551in}{1.496389in}}{\pgfqpoint{0.882375in}{1.502213in}}%
\pgfpathcurveto{\pgfqpoint{0.888199in}{1.508037in}}{\pgfqpoint{0.891471in}{1.515937in}}{\pgfqpoint{0.891471in}{1.524173in}}%
\pgfpathcurveto{\pgfqpoint{0.891471in}{1.532410in}}{\pgfqpoint{0.888199in}{1.540310in}}{\pgfqpoint{0.882375in}{1.546134in}}%
\pgfpathcurveto{\pgfqpoint{0.876551in}{1.551958in}}{\pgfqpoint{0.868651in}{1.555230in}}{\pgfqpoint{0.860415in}{1.555230in}}%
\pgfpathcurveto{\pgfqpoint{0.852178in}{1.555230in}}{\pgfqpoint{0.844278in}{1.551958in}}{\pgfqpoint{0.838454in}{1.546134in}}%
\pgfpathcurveto{\pgfqpoint{0.832631in}{1.540310in}}{\pgfqpoint{0.829358in}{1.532410in}}{\pgfqpoint{0.829358in}{1.524173in}}%
\pgfpathcurveto{\pgfqpoint{0.829358in}{1.515937in}}{\pgfqpoint{0.832631in}{1.508037in}}{\pgfqpoint{0.838454in}{1.502213in}}%
\pgfpathcurveto{\pgfqpoint{0.844278in}{1.496389in}}{\pgfqpoint{0.852178in}{1.493117in}}{\pgfqpoint{0.860415in}{1.493117in}}%
\pgfpathclose%
\pgfusepath{stroke,fill}%
\end{pgfscope}%
\begin{pgfscope}%
\pgfpathrectangle{\pgfqpoint{0.100000in}{0.212622in}}{\pgfqpoint{3.696000in}{3.696000in}}%
\pgfusepath{clip}%
\pgfsetbuttcap%
\pgfsetroundjoin%
\definecolor{currentfill}{rgb}{0.121569,0.466667,0.705882}%
\pgfsetfillcolor{currentfill}%
\pgfsetfillopacity{0.627433}%
\pgfsetlinewidth{1.003750pt}%
\definecolor{currentstroke}{rgb}{0.121569,0.466667,0.705882}%
\pgfsetstrokecolor{currentstroke}%
\pgfsetstrokeopacity{0.627433}%
\pgfsetdash{}{0pt}%
\pgfpathmoveto{\pgfqpoint{0.860415in}{1.493117in}}%
\pgfpathcurveto{\pgfqpoint{0.868651in}{1.493117in}}{\pgfqpoint{0.876551in}{1.496389in}}{\pgfqpoint{0.882375in}{1.502213in}}%
\pgfpathcurveto{\pgfqpoint{0.888199in}{1.508037in}}{\pgfqpoint{0.891471in}{1.515937in}}{\pgfqpoint{0.891471in}{1.524173in}}%
\pgfpathcurveto{\pgfqpoint{0.891471in}{1.532410in}}{\pgfqpoint{0.888199in}{1.540310in}}{\pgfqpoint{0.882375in}{1.546134in}}%
\pgfpathcurveto{\pgfqpoint{0.876551in}{1.551958in}}{\pgfqpoint{0.868651in}{1.555230in}}{\pgfqpoint{0.860415in}{1.555230in}}%
\pgfpathcurveto{\pgfqpoint{0.852178in}{1.555230in}}{\pgfqpoint{0.844278in}{1.551958in}}{\pgfqpoint{0.838454in}{1.546134in}}%
\pgfpathcurveto{\pgfqpoint{0.832631in}{1.540310in}}{\pgfqpoint{0.829358in}{1.532410in}}{\pgfqpoint{0.829358in}{1.524173in}}%
\pgfpathcurveto{\pgfqpoint{0.829358in}{1.515937in}}{\pgfqpoint{0.832631in}{1.508037in}}{\pgfqpoint{0.838454in}{1.502213in}}%
\pgfpathcurveto{\pgfqpoint{0.844278in}{1.496389in}}{\pgfqpoint{0.852178in}{1.493117in}}{\pgfqpoint{0.860415in}{1.493117in}}%
\pgfpathclose%
\pgfusepath{stroke,fill}%
\end{pgfscope}%
\begin{pgfscope}%
\pgfpathrectangle{\pgfqpoint{0.100000in}{0.212622in}}{\pgfqpoint{3.696000in}{3.696000in}}%
\pgfusepath{clip}%
\pgfsetbuttcap%
\pgfsetroundjoin%
\definecolor{currentfill}{rgb}{0.121569,0.466667,0.705882}%
\pgfsetfillcolor{currentfill}%
\pgfsetfillopacity{0.627433}%
\pgfsetlinewidth{1.003750pt}%
\definecolor{currentstroke}{rgb}{0.121569,0.466667,0.705882}%
\pgfsetstrokecolor{currentstroke}%
\pgfsetstrokeopacity{0.627433}%
\pgfsetdash{}{0pt}%
\pgfpathmoveto{\pgfqpoint{0.860415in}{1.493117in}}%
\pgfpathcurveto{\pgfqpoint{0.868651in}{1.493117in}}{\pgfqpoint{0.876551in}{1.496389in}}{\pgfqpoint{0.882375in}{1.502213in}}%
\pgfpathcurveto{\pgfqpoint{0.888199in}{1.508037in}}{\pgfqpoint{0.891471in}{1.515937in}}{\pgfqpoint{0.891471in}{1.524173in}}%
\pgfpathcurveto{\pgfqpoint{0.891471in}{1.532410in}}{\pgfqpoint{0.888199in}{1.540310in}}{\pgfqpoint{0.882375in}{1.546134in}}%
\pgfpathcurveto{\pgfqpoint{0.876551in}{1.551958in}}{\pgfqpoint{0.868651in}{1.555230in}}{\pgfqpoint{0.860415in}{1.555230in}}%
\pgfpathcurveto{\pgfqpoint{0.852178in}{1.555230in}}{\pgfqpoint{0.844278in}{1.551958in}}{\pgfqpoint{0.838454in}{1.546134in}}%
\pgfpathcurveto{\pgfqpoint{0.832631in}{1.540310in}}{\pgfqpoint{0.829358in}{1.532410in}}{\pgfqpoint{0.829358in}{1.524173in}}%
\pgfpathcurveto{\pgfqpoint{0.829358in}{1.515937in}}{\pgfqpoint{0.832631in}{1.508037in}}{\pgfqpoint{0.838454in}{1.502213in}}%
\pgfpathcurveto{\pgfqpoint{0.844278in}{1.496389in}}{\pgfqpoint{0.852178in}{1.493117in}}{\pgfqpoint{0.860415in}{1.493117in}}%
\pgfpathclose%
\pgfusepath{stroke,fill}%
\end{pgfscope}%
\begin{pgfscope}%
\pgfpathrectangle{\pgfqpoint{0.100000in}{0.212622in}}{\pgfqpoint{3.696000in}{3.696000in}}%
\pgfusepath{clip}%
\pgfsetbuttcap%
\pgfsetroundjoin%
\definecolor{currentfill}{rgb}{0.121569,0.466667,0.705882}%
\pgfsetfillcolor{currentfill}%
\pgfsetfillopacity{0.627433}%
\pgfsetlinewidth{1.003750pt}%
\definecolor{currentstroke}{rgb}{0.121569,0.466667,0.705882}%
\pgfsetstrokecolor{currentstroke}%
\pgfsetstrokeopacity{0.627433}%
\pgfsetdash{}{0pt}%
\pgfpathmoveto{\pgfqpoint{0.860415in}{1.493117in}}%
\pgfpathcurveto{\pgfqpoint{0.868651in}{1.493117in}}{\pgfqpoint{0.876551in}{1.496389in}}{\pgfqpoint{0.882375in}{1.502213in}}%
\pgfpathcurveto{\pgfqpoint{0.888199in}{1.508037in}}{\pgfqpoint{0.891471in}{1.515937in}}{\pgfqpoint{0.891471in}{1.524173in}}%
\pgfpathcurveto{\pgfqpoint{0.891471in}{1.532410in}}{\pgfqpoint{0.888199in}{1.540310in}}{\pgfqpoint{0.882375in}{1.546134in}}%
\pgfpathcurveto{\pgfqpoint{0.876551in}{1.551958in}}{\pgfqpoint{0.868651in}{1.555230in}}{\pgfqpoint{0.860415in}{1.555230in}}%
\pgfpathcurveto{\pgfqpoint{0.852178in}{1.555230in}}{\pgfqpoint{0.844278in}{1.551958in}}{\pgfqpoint{0.838454in}{1.546134in}}%
\pgfpathcurveto{\pgfqpoint{0.832631in}{1.540310in}}{\pgfqpoint{0.829358in}{1.532410in}}{\pgfqpoint{0.829358in}{1.524173in}}%
\pgfpathcurveto{\pgfqpoint{0.829358in}{1.515937in}}{\pgfqpoint{0.832631in}{1.508037in}}{\pgfqpoint{0.838454in}{1.502213in}}%
\pgfpathcurveto{\pgfqpoint{0.844278in}{1.496389in}}{\pgfqpoint{0.852178in}{1.493117in}}{\pgfqpoint{0.860415in}{1.493117in}}%
\pgfpathclose%
\pgfusepath{stroke,fill}%
\end{pgfscope}%
\begin{pgfscope}%
\pgfpathrectangle{\pgfqpoint{0.100000in}{0.212622in}}{\pgfqpoint{3.696000in}{3.696000in}}%
\pgfusepath{clip}%
\pgfsetbuttcap%
\pgfsetroundjoin%
\definecolor{currentfill}{rgb}{0.121569,0.466667,0.705882}%
\pgfsetfillcolor{currentfill}%
\pgfsetfillopacity{0.627433}%
\pgfsetlinewidth{1.003750pt}%
\definecolor{currentstroke}{rgb}{0.121569,0.466667,0.705882}%
\pgfsetstrokecolor{currentstroke}%
\pgfsetstrokeopacity{0.627433}%
\pgfsetdash{}{0pt}%
\pgfpathmoveto{\pgfqpoint{0.860415in}{1.493117in}}%
\pgfpathcurveto{\pgfqpoint{0.868651in}{1.493117in}}{\pgfqpoint{0.876551in}{1.496389in}}{\pgfqpoint{0.882375in}{1.502213in}}%
\pgfpathcurveto{\pgfqpoint{0.888199in}{1.508037in}}{\pgfqpoint{0.891471in}{1.515937in}}{\pgfqpoint{0.891471in}{1.524173in}}%
\pgfpathcurveto{\pgfqpoint{0.891471in}{1.532410in}}{\pgfqpoint{0.888199in}{1.540310in}}{\pgfqpoint{0.882375in}{1.546134in}}%
\pgfpathcurveto{\pgfqpoint{0.876551in}{1.551958in}}{\pgfqpoint{0.868651in}{1.555230in}}{\pgfqpoint{0.860415in}{1.555230in}}%
\pgfpathcurveto{\pgfqpoint{0.852178in}{1.555230in}}{\pgfqpoint{0.844278in}{1.551958in}}{\pgfqpoint{0.838454in}{1.546134in}}%
\pgfpathcurveto{\pgfqpoint{0.832631in}{1.540310in}}{\pgfqpoint{0.829358in}{1.532410in}}{\pgfqpoint{0.829358in}{1.524173in}}%
\pgfpathcurveto{\pgfqpoint{0.829358in}{1.515937in}}{\pgfqpoint{0.832631in}{1.508037in}}{\pgfqpoint{0.838454in}{1.502213in}}%
\pgfpathcurveto{\pgfqpoint{0.844278in}{1.496389in}}{\pgfqpoint{0.852178in}{1.493117in}}{\pgfqpoint{0.860415in}{1.493117in}}%
\pgfpathclose%
\pgfusepath{stroke,fill}%
\end{pgfscope}%
\begin{pgfscope}%
\pgfpathrectangle{\pgfqpoint{0.100000in}{0.212622in}}{\pgfqpoint{3.696000in}{3.696000in}}%
\pgfusepath{clip}%
\pgfsetbuttcap%
\pgfsetroundjoin%
\definecolor{currentfill}{rgb}{0.121569,0.466667,0.705882}%
\pgfsetfillcolor{currentfill}%
\pgfsetfillopacity{0.627433}%
\pgfsetlinewidth{1.003750pt}%
\definecolor{currentstroke}{rgb}{0.121569,0.466667,0.705882}%
\pgfsetstrokecolor{currentstroke}%
\pgfsetstrokeopacity{0.627433}%
\pgfsetdash{}{0pt}%
\pgfpathmoveto{\pgfqpoint{0.860415in}{1.493117in}}%
\pgfpathcurveto{\pgfqpoint{0.868651in}{1.493117in}}{\pgfqpoint{0.876551in}{1.496389in}}{\pgfqpoint{0.882375in}{1.502213in}}%
\pgfpathcurveto{\pgfqpoint{0.888199in}{1.508037in}}{\pgfqpoint{0.891471in}{1.515937in}}{\pgfqpoint{0.891471in}{1.524173in}}%
\pgfpathcurveto{\pgfqpoint{0.891471in}{1.532410in}}{\pgfqpoint{0.888199in}{1.540310in}}{\pgfqpoint{0.882375in}{1.546134in}}%
\pgfpathcurveto{\pgfqpoint{0.876551in}{1.551958in}}{\pgfqpoint{0.868651in}{1.555230in}}{\pgfqpoint{0.860415in}{1.555230in}}%
\pgfpathcurveto{\pgfqpoint{0.852178in}{1.555230in}}{\pgfqpoint{0.844278in}{1.551958in}}{\pgfqpoint{0.838454in}{1.546134in}}%
\pgfpathcurveto{\pgfqpoint{0.832631in}{1.540310in}}{\pgfqpoint{0.829358in}{1.532410in}}{\pgfqpoint{0.829358in}{1.524173in}}%
\pgfpathcurveto{\pgfqpoint{0.829358in}{1.515937in}}{\pgfqpoint{0.832631in}{1.508037in}}{\pgfqpoint{0.838454in}{1.502213in}}%
\pgfpathcurveto{\pgfqpoint{0.844278in}{1.496389in}}{\pgfqpoint{0.852178in}{1.493117in}}{\pgfqpoint{0.860415in}{1.493117in}}%
\pgfpathclose%
\pgfusepath{stroke,fill}%
\end{pgfscope}%
\begin{pgfscope}%
\pgfpathrectangle{\pgfqpoint{0.100000in}{0.212622in}}{\pgfqpoint{3.696000in}{3.696000in}}%
\pgfusepath{clip}%
\pgfsetbuttcap%
\pgfsetroundjoin%
\definecolor{currentfill}{rgb}{0.121569,0.466667,0.705882}%
\pgfsetfillcolor{currentfill}%
\pgfsetfillopacity{0.627433}%
\pgfsetlinewidth{1.003750pt}%
\definecolor{currentstroke}{rgb}{0.121569,0.466667,0.705882}%
\pgfsetstrokecolor{currentstroke}%
\pgfsetstrokeopacity{0.627433}%
\pgfsetdash{}{0pt}%
\pgfpathmoveto{\pgfqpoint{0.860415in}{1.493117in}}%
\pgfpathcurveto{\pgfqpoint{0.868651in}{1.493117in}}{\pgfqpoint{0.876551in}{1.496389in}}{\pgfqpoint{0.882375in}{1.502213in}}%
\pgfpathcurveto{\pgfqpoint{0.888199in}{1.508037in}}{\pgfqpoint{0.891471in}{1.515937in}}{\pgfqpoint{0.891471in}{1.524173in}}%
\pgfpathcurveto{\pgfqpoint{0.891471in}{1.532410in}}{\pgfqpoint{0.888199in}{1.540310in}}{\pgfqpoint{0.882375in}{1.546134in}}%
\pgfpathcurveto{\pgfqpoint{0.876551in}{1.551958in}}{\pgfqpoint{0.868651in}{1.555230in}}{\pgfqpoint{0.860415in}{1.555230in}}%
\pgfpathcurveto{\pgfqpoint{0.852178in}{1.555230in}}{\pgfqpoint{0.844278in}{1.551958in}}{\pgfqpoint{0.838454in}{1.546134in}}%
\pgfpathcurveto{\pgfqpoint{0.832631in}{1.540310in}}{\pgfqpoint{0.829358in}{1.532410in}}{\pgfqpoint{0.829358in}{1.524173in}}%
\pgfpathcurveto{\pgfqpoint{0.829358in}{1.515937in}}{\pgfqpoint{0.832631in}{1.508037in}}{\pgfqpoint{0.838454in}{1.502213in}}%
\pgfpathcurveto{\pgfqpoint{0.844278in}{1.496389in}}{\pgfqpoint{0.852178in}{1.493117in}}{\pgfqpoint{0.860415in}{1.493117in}}%
\pgfpathclose%
\pgfusepath{stroke,fill}%
\end{pgfscope}%
\begin{pgfscope}%
\pgfpathrectangle{\pgfqpoint{0.100000in}{0.212622in}}{\pgfqpoint{3.696000in}{3.696000in}}%
\pgfusepath{clip}%
\pgfsetbuttcap%
\pgfsetroundjoin%
\definecolor{currentfill}{rgb}{0.121569,0.466667,0.705882}%
\pgfsetfillcolor{currentfill}%
\pgfsetfillopacity{0.627433}%
\pgfsetlinewidth{1.003750pt}%
\definecolor{currentstroke}{rgb}{0.121569,0.466667,0.705882}%
\pgfsetstrokecolor{currentstroke}%
\pgfsetstrokeopacity{0.627433}%
\pgfsetdash{}{0pt}%
\pgfpathmoveto{\pgfqpoint{0.860415in}{1.493117in}}%
\pgfpathcurveto{\pgfqpoint{0.868651in}{1.493117in}}{\pgfqpoint{0.876551in}{1.496389in}}{\pgfqpoint{0.882375in}{1.502213in}}%
\pgfpathcurveto{\pgfqpoint{0.888199in}{1.508037in}}{\pgfqpoint{0.891471in}{1.515937in}}{\pgfqpoint{0.891471in}{1.524173in}}%
\pgfpathcurveto{\pgfqpoint{0.891471in}{1.532410in}}{\pgfqpoint{0.888199in}{1.540310in}}{\pgfqpoint{0.882375in}{1.546134in}}%
\pgfpathcurveto{\pgfqpoint{0.876551in}{1.551958in}}{\pgfqpoint{0.868651in}{1.555230in}}{\pgfqpoint{0.860415in}{1.555230in}}%
\pgfpathcurveto{\pgfqpoint{0.852178in}{1.555230in}}{\pgfqpoint{0.844278in}{1.551958in}}{\pgfqpoint{0.838454in}{1.546134in}}%
\pgfpathcurveto{\pgfqpoint{0.832631in}{1.540310in}}{\pgfqpoint{0.829358in}{1.532410in}}{\pgfqpoint{0.829358in}{1.524173in}}%
\pgfpathcurveto{\pgfqpoint{0.829358in}{1.515937in}}{\pgfqpoint{0.832631in}{1.508037in}}{\pgfqpoint{0.838454in}{1.502213in}}%
\pgfpathcurveto{\pgfqpoint{0.844278in}{1.496389in}}{\pgfqpoint{0.852178in}{1.493117in}}{\pgfqpoint{0.860415in}{1.493117in}}%
\pgfpathclose%
\pgfusepath{stroke,fill}%
\end{pgfscope}%
\begin{pgfscope}%
\pgfpathrectangle{\pgfqpoint{0.100000in}{0.212622in}}{\pgfqpoint{3.696000in}{3.696000in}}%
\pgfusepath{clip}%
\pgfsetbuttcap%
\pgfsetroundjoin%
\definecolor{currentfill}{rgb}{0.121569,0.466667,0.705882}%
\pgfsetfillcolor{currentfill}%
\pgfsetfillopacity{0.627433}%
\pgfsetlinewidth{1.003750pt}%
\definecolor{currentstroke}{rgb}{0.121569,0.466667,0.705882}%
\pgfsetstrokecolor{currentstroke}%
\pgfsetstrokeopacity{0.627433}%
\pgfsetdash{}{0pt}%
\pgfpathmoveto{\pgfqpoint{0.860415in}{1.493117in}}%
\pgfpathcurveto{\pgfqpoint{0.868651in}{1.493117in}}{\pgfqpoint{0.876551in}{1.496389in}}{\pgfqpoint{0.882375in}{1.502213in}}%
\pgfpathcurveto{\pgfqpoint{0.888199in}{1.508037in}}{\pgfqpoint{0.891471in}{1.515937in}}{\pgfqpoint{0.891471in}{1.524173in}}%
\pgfpathcurveto{\pgfqpoint{0.891471in}{1.532410in}}{\pgfqpoint{0.888199in}{1.540310in}}{\pgfqpoint{0.882375in}{1.546134in}}%
\pgfpathcurveto{\pgfqpoint{0.876551in}{1.551958in}}{\pgfqpoint{0.868651in}{1.555230in}}{\pgfqpoint{0.860415in}{1.555230in}}%
\pgfpathcurveto{\pgfqpoint{0.852178in}{1.555230in}}{\pgfqpoint{0.844278in}{1.551958in}}{\pgfqpoint{0.838454in}{1.546134in}}%
\pgfpathcurveto{\pgfqpoint{0.832631in}{1.540310in}}{\pgfqpoint{0.829358in}{1.532410in}}{\pgfqpoint{0.829358in}{1.524173in}}%
\pgfpathcurveto{\pgfqpoint{0.829358in}{1.515937in}}{\pgfqpoint{0.832631in}{1.508037in}}{\pgfqpoint{0.838454in}{1.502213in}}%
\pgfpathcurveto{\pgfqpoint{0.844278in}{1.496389in}}{\pgfqpoint{0.852178in}{1.493117in}}{\pgfqpoint{0.860415in}{1.493117in}}%
\pgfpathclose%
\pgfusepath{stroke,fill}%
\end{pgfscope}%
\begin{pgfscope}%
\pgfpathrectangle{\pgfqpoint{0.100000in}{0.212622in}}{\pgfqpoint{3.696000in}{3.696000in}}%
\pgfusepath{clip}%
\pgfsetbuttcap%
\pgfsetroundjoin%
\definecolor{currentfill}{rgb}{0.121569,0.466667,0.705882}%
\pgfsetfillcolor{currentfill}%
\pgfsetfillopacity{0.627433}%
\pgfsetlinewidth{1.003750pt}%
\definecolor{currentstroke}{rgb}{0.121569,0.466667,0.705882}%
\pgfsetstrokecolor{currentstroke}%
\pgfsetstrokeopacity{0.627433}%
\pgfsetdash{}{0pt}%
\pgfpathmoveto{\pgfqpoint{0.860415in}{1.493117in}}%
\pgfpathcurveto{\pgfqpoint{0.868651in}{1.493117in}}{\pgfqpoint{0.876551in}{1.496389in}}{\pgfqpoint{0.882375in}{1.502213in}}%
\pgfpathcurveto{\pgfqpoint{0.888199in}{1.508037in}}{\pgfqpoint{0.891471in}{1.515937in}}{\pgfqpoint{0.891471in}{1.524173in}}%
\pgfpathcurveto{\pgfqpoint{0.891471in}{1.532410in}}{\pgfqpoint{0.888199in}{1.540310in}}{\pgfqpoint{0.882375in}{1.546134in}}%
\pgfpathcurveto{\pgfqpoint{0.876551in}{1.551958in}}{\pgfqpoint{0.868651in}{1.555230in}}{\pgfqpoint{0.860415in}{1.555230in}}%
\pgfpathcurveto{\pgfqpoint{0.852178in}{1.555230in}}{\pgfqpoint{0.844278in}{1.551958in}}{\pgfqpoint{0.838454in}{1.546134in}}%
\pgfpathcurveto{\pgfqpoint{0.832631in}{1.540310in}}{\pgfqpoint{0.829358in}{1.532410in}}{\pgfqpoint{0.829358in}{1.524173in}}%
\pgfpathcurveto{\pgfqpoint{0.829358in}{1.515937in}}{\pgfqpoint{0.832631in}{1.508037in}}{\pgfqpoint{0.838454in}{1.502213in}}%
\pgfpathcurveto{\pgfqpoint{0.844278in}{1.496389in}}{\pgfqpoint{0.852178in}{1.493117in}}{\pgfqpoint{0.860415in}{1.493117in}}%
\pgfpathclose%
\pgfusepath{stroke,fill}%
\end{pgfscope}%
\begin{pgfscope}%
\pgfpathrectangle{\pgfqpoint{0.100000in}{0.212622in}}{\pgfqpoint{3.696000in}{3.696000in}}%
\pgfusepath{clip}%
\pgfsetbuttcap%
\pgfsetroundjoin%
\definecolor{currentfill}{rgb}{0.121569,0.466667,0.705882}%
\pgfsetfillcolor{currentfill}%
\pgfsetfillopacity{0.627433}%
\pgfsetlinewidth{1.003750pt}%
\definecolor{currentstroke}{rgb}{0.121569,0.466667,0.705882}%
\pgfsetstrokecolor{currentstroke}%
\pgfsetstrokeopacity{0.627433}%
\pgfsetdash{}{0pt}%
\pgfpathmoveto{\pgfqpoint{0.860415in}{1.493117in}}%
\pgfpathcurveto{\pgfqpoint{0.868651in}{1.493117in}}{\pgfqpoint{0.876551in}{1.496389in}}{\pgfqpoint{0.882375in}{1.502213in}}%
\pgfpathcurveto{\pgfqpoint{0.888199in}{1.508037in}}{\pgfqpoint{0.891471in}{1.515937in}}{\pgfqpoint{0.891471in}{1.524173in}}%
\pgfpathcurveto{\pgfqpoint{0.891471in}{1.532410in}}{\pgfqpoint{0.888199in}{1.540310in}}{\pgfqpoint{0.882375in}{1.546134in}}%
\pgfpathcurveto{\pgfqpoint{0.876551in}{1.551958in}}{\pgfqpoint{0.868651in}{1.555230in}}{\pgfqpoint{0.860415in}{1.555230in}}%
\pgfpathcurveto{\pgfqpoint{0.852178in}{1.555230in}}{\pgfqpoint{0.844278in}{1.551958in}}{\pgfqpoint{0.838454in}{1.546134in}}%
\pgfpathcurveto{\pgfqpoint{0.832630in}{1.540310in}}{\pgfqpoint{0.829358in}{1.532410in}}{\pgfqpoint{0.829358in}{1.524173in}}%
\pgfpathcurveto{\pgfqpoint{0.829358in}{1.515937in}}{\pgfqpoint{0.832630in}{1.508037in}}{\pgfqpoint{0.838454in}{1.502213in}}%
\pgfpathcurveto{\pgfqpoint{0.844278in}{1.496389in}}{\pgfqpoint{0.852178in}{1.493117in}}{\pgfqpoint{0.860415in}{1.493117in}}%
\pgfpathclose%
\pgfusepath{stroke,fill}%
\end{pgfscope}%
\begin{pgfscope}%
\pgfpathrectangle{\pgfqpoint{0.100000in}{0.212622in}}{\pgfqpoint{3.696000in}{3.696000in}}%
\pgfusepath{clip}%
\pgfsetbuttcap%
\pgfsetroundjoin%
\definecolor{currentfill}{rgb}{0.121569,0.466667,0.705882}%
\pgfsetfillcolor{currentfill}%
\pgfsetfillopacity{0.627433}%
\pgfsetlinewidth{1.003750pt}%
\definecolor{currentstroke}{rgb}{0.121569,0.466667,0.705882}%
\pgfsetstrokecolor{currentstroke}%
\pgfsetstrokeopacity{0.627433}%
\pgfsetdash{}{0pt}%
\pgfpathmoveto{\pgfqpoint{0.860415in}{1.493117in}}%
\pgfpathcurveto{\pgfqpoint{0.868651in}{1.493117in}}{\pgfqpoint{0.876551in}{1.496389in}}{\pgfqpoint{0.882375in}{1.502213in}}%
\pgfpathcurveto{\pgfqpoint{0.888199in}{1.508037in}}{\pgfqpoint{0.891471in}{1.515937in}}{\pgfqpoint{0.891471in}{1.524173in}}%
\pgfpathcurveto{\pgfqpoint{0.891471in}{1.532410in}}{\pgfqpoint{0.888199in}{1.540310in}}{\pgfqpoint{0.882375in}{1.546134in}}%
\pgfpathcurveto{\pgfqpoint{0.876551in}{1.551958in}}{\pgfqpoint{0.868651in}{1.555230in}}{\pgfqpoint{0.860415in}{1.555230in}}%
\pgfpathcurveto{\pgfqpoint{0.852178in}{1.555230in}}{\pgfqpoint{0.844278in}{1.551958in}}{\pgfqpoint{0.838454in}{1.546134in}}%
\pgfpathcurveto{\pgfqpoint{0.832631in}{1.540310in}}{\pgfqpoint{0.829358in}{1.532410in}}{\pgfqpoint{0.829358in}{1.524173in}}%
\pgfpathcurveto{\pgfqpoint{0.829358in}{1.515937in}}{\pgfqpoint{0.832631in}{1.508037in}}{\pgfqpoint{0.838454in}{1.502213in}}%
\pgfpathcurveto{\pgfqpoint{0.844278in}{1.496389in}}{\pgfqpoint{0.852178in}{1.493117in}}{\pgfqpoint{0.860415in}{1.493117in}}%
\pgfpathclose%
\pgfusepath{stroke,fill}%
\end{pgfscope}%
\begin{pgfscope}%
\pgfpathrectangle{\pgfqpoint{0.100000in}{0.212622in}}{\pgfqpoint{3.696000in}{3.696000in}}%
\pgfusepath{clip}%
\pgfsetbuttcap%
\pgfsetroundjoin%
\definecolor{currentfill}{rgb}{0.121569,0.466667,0.705882}%
\pgfsetfillcolor{currentfill}%
\pgfsetfillopacity{0.627433}%
\pgfsetlinewidth{1.003750pt}%
\definecolor{currentstroke}{rgb}{0.121569,0.466667,0.705882}%
\pgfsetstrokecolor{currentstroke}%
\pgfsetstrokeopacity{0.627433}%
\pgfsetdash{}{0pt}%
\pgfpathmoveto{\pgfqpoint{0.860415in}{1.493117in}}%
\pgfpathcurveto{\pgfqpoint{0.868651in}{1.493117in}}{\pgfqpoint{0.876551in}{1.496389in}}{\pgfqpoint{0.882375in}{1.502213in}}%
\pgfpathcurveto{\pgfqpoint{0.888199in}{1.508037in}}{\pgfqpoint{0.891471in}{1.515937in}}{\pgfqpoint{0.891471in}{1.524173in}}%
\pgfpathcurveto{\pgfqpoint{0.891471in}{1.532410in}}{\pgfqpoint{0.888199in}{1.540310in}}{\pgfqpoint{0.882375in}{1.546134in}}%
\pgfpathcurveto{\pgfqpoint{0.876551in}{1.551958in}}{\pgfqpoint{0.868651in}{1.555230in}}{\pgfqpoint{0.860415in}{1.555230in}}%
\pgfpathcurveto{\pgfqpoint{0.852178in}{1.555230in}}{\pgfqpoint{0.844278in}{1.551958in}}{\pgfqpoint{0.838454in}{1.546134in}}%
\pgfpathcurveto{\pgfqpoint{0.832631in}{1.540310in}}{\pgfqpoint{0.829358in}{1.532410in}}{\pgfqpoint{0.829358in}{1.524173in}}%
\pgfpathcurveto{\pgfqpoint{0.829358in}{1.515937in}}{\pgfqpoint{0.832631in}{1.508037in}}{\pgfqpoint{0.838454in}{1.502213in}}%
\pgfpathcurveto{\pgfqpoint{0.844278in}{1.496389in}}{\pgfqpoint{0.852178in}{1.493117in}}{\pgfqpoint{0.860415in}{1.493117in}}%
\pgfpathclose%
\pgfusepath{stroke,fill}%
\end{pgfscope}%
\begin{pgfscope}%
\pgfpathrectangle{\pgfqpoint{0.100000in}{0.212622in}}{\pgfqpoint{3.696000in}{3.696000in}}%
\pgfusepath{clip}%
\pgfsetbuttcap%
\pgfsetroundjoin%
\definecolor{currentfill}{rgb}{0.121569,0.466667,0.705882}%
\pgfsetfillcolor{currentfill}%
\pgfsetfillopacity{0.628136}%
\pgfsetlinewidth{1.003750pt}%
\definecolor{currentstroke}{rgb}{0.121569,0.466667,0.705882}%
\pgfsetstrokecolor{currentstroke}%
\pgfsetstrokeopacity{0.628136}%
\pgfsetdash{}{0pt}%
\pgfpathmoveto{\pgfqpoint{2.122061in}{2.004903in}}%
\pgfpathcurveto{\pgfqpoint{2.130297in}{2.004903in}}{\pgfqpoint{2.138197in}{2.008176in}}{\pgfqpoint{2.144021in}{2.014000in}}%
\pgfpathcurveto{\pgfqpoint{2.149845in}{2.019824in}}{\pgfqpoint{2.153117in}{2.027724in}}{\pgfqpoint{2.153117in}{2.035960in}}%
\pgfpathcurveto{\pgfqpoint{2.153117in}{2.044196in}}{\pgfqpoint{2.149845in}{2.052096in}}{\pgfqpoint{2.144021in}{2.057920in}}%
\pgfpathcurveto{\pgfqpoint{2.138197in}{2.063744in}}{\pgfqpoint{2.130297in}{2.067016in}}{\pgfqpoint{2.122061in}{2.067016in}}%
\pgfpathcurveto{\pgfqpoint{2.113824in}{2.067016in}}{\pgfqpoint{2.105924in}{2.063744in}}{\pgfqpoint{2.100100in}{2.057920in}}%
\pgfpathcurveto{\pgfqpoint{2.094276in}{2.052096in}}{\pgfqpoint{2.091004in}{2.044196in}}{\pgfqpoint{2.091004in}{2.035960in}}%
\pgfpathcurveto{\pgfqpoint{2.091004in}{2.027724in}}{\pgfqpoint{2.094276in}{2.019824in}}{\pgfqpoint{2.100100in}{2.014000in}}%
\pgfpathcurveto{\pgfqpoint{2.105924in}{2.008176in}}{\pgfqpoint{2.113824in}{2.004903in}}{\pgfqpoint{2.122061in}{2.004903in}}%
\pgfpathclose%
\pgfusepath{stroke,fill}%
\end{pgfscope}%
\begin{pgfscope}%
\pgfpathrectangle{\pgfqpoint{0.100000in}{0.212622in}}{\pgfqpoint{3.696000in}{3.696000in}}%
\pgfusepath{clip}%
\pgfsetbuttcap%
\pgfsetroundjoin%
\definecolor{currentfill}{rgb}{0.121569,0.466667,0.705882}%
\pgfsetfillcolor{currentfill}%
\pgfsetfillopacity{0.630318}%
\pgfsetlinewidth{1.003750pt}%
\definecolor{currentstroke}{rgb}{0.121569,0.466667,0.705882}%
\pgfsetstrokecolor{currentstroke}%
\pgfsetstrokeopacity{0.630318}%
\pgfsetdash{}{0pt}%
\pgfpathmoveto{\pgfqpoint{0.757388in}{1.205105in}}%
\pgfpathcurveto{\pgfqpoint{0.765625in}{1.205105in}}{\pgfqpoint{0.773525in}{1.208377in}}{\pgfqpoint{0.779349in}{1.214201in}}%
\pgfpathcurveto{\pgfqpoint{0.785173in}{1.220025in}}{\pgfqpoint{0.788445in}{1.227925in}}{\pgfqpoint{0.788445in}{1.236162in}}%
\pgfpathcurveto{\pgfqpoint{0.788445in}{1.244398in}}{\pgfqpoint{0.785173in}{1.252298in}}{\pgfqpoint{0.779349in}{1.258122in}}%
\pgfpathcurveto{\pgfqpoint{0.773525in}{1.263946in}}{\pgfqpoint{0.765625in}{1.267218in}}{\pgfqpoint{0.757388in}{1.267218in}}%
\pgfpathcurveto{\pgfqpoint{0.749152in}{1.267218in}}{\pgfqpoint{0.741252in}{1.263946in}}{\pgfqpoint{0.735428in}{1.258122in}}%
\pgfpathcurveto{\pgfqpoint{0.729604in}{1.252298in}}{\pgfqpoint{0.726332in}{1.244398in}}{\pgfqpoint{0.726332in}{1.236162in}}%
\pgfpathcurveto{\pgfqpoint{0.726332in}{1.227925in}}{\pgfqpoint{0.729604in}{1.220025in}}{\pgfqpoint{0.735428in}{1.214201in}}%
\pgfpathcurveto{\pgfqpoint{0.741252in}{1.208377in}}{\pgfqpoint{0.749152in}{1.205105in}}{\pgfqpoint{0.757388in}{1.205105in}}%
\pgfpathclose%
\pgfusepath{stroke,fill}%
\end{pgfscope}%
\begin{pgfscope}%
\pgfpathrectangle{\pgfqpoint{0.100000in}{0.212622in}}{\pgfqpoint{3.696000in}{3.696000in}}%
\pgfusepath{clip}%
\pgfsetbuttcap%
\pgfsetroundjoin%
\definecolor{currentfill}{rgb}{0.121569,0.466667,0.705882}%
\pgfsetfillcolor{currentfill}%
\pgfsetfillopacity{0.630468}%
\pgfsetlinewidth{1.003750pt}%
\definecolor{currentstroke}{rgb}{0.121569,0.466667,0.705882}%
\pgfsetstrokecolor{currentstroke}%
\pgfsetstrokeopacity{0.630468}%
\pgfsetdash{}{0pt}%
\pgfpathmoveto{\pgfqpoint{2.123735in}{1.997077in}}%
\pgfpathcurveto{\pgfqpoint{2.131971in}{1.997077in}}{\pgfqpoint{2.139871in}{2.000349in}}{\pgfqpoint{2.145695in}{2.006173in}}%
\pgfpathcurveto{\pgfqpoint{2.151519in}{2.011997in}}{\pgfqpoint{2.154791in}{2.019897in}}{\pgfqpoint{2.154791in}{2.028133in}}%
\pgfpathcurveto{\pgfqpoint{2.154791in}{2.036369in}}{\pgfqpoint{2.151519in}{2.044269in}}{\pgfqpoint{2.145695in}{2.050093in}}%
\pgfpathcurveto{\pgfqpoint{2.139871in}{2.055917in}}{\pgfqpoint{2.131971in}{2.059190in}}{\pgfqpoint{2.123735in}{2.059190in}}%
\pgfpathcurveto{\pgfqpoint{2.115499in}{2.059190in}}{\pgfqpoint{2.107598in}{2.055917in}}{\pgfqpoint{2.101775in}{2.050093in}}%
\pgfpathcurveto{\pgfqpoint{2.095951in}{2.044269in}}{\pgfqpoint{2.092678in}{2.036369in}}{\pgfqpoint{2.092678in}{2.028133in}}%
\pgfpathcurveto{\pgfqpoint{2.092678in}{2.019897in}}{\pgfqpoint{2.095951in}{2.011997in}}{\pgfqpoint{2.101775in}{2.006173in}}%
\pgfpathcurveto{\pgfqpoint{2.107598in}{2.000349in}}{\pgfqpoint{2.115499in}{1.997077in}}{\pgfqpoint{2.123735in}{1.997077in}}%
\pgfpathclose%
\pgfusepath{stroke,fill}%
\end{pgfscope}%
\begin{pgfscope}%
\pgfpathrectangle{\pgfqpoint{0.100000in}{0.212622in}}{\pgfqpoint{3.696000in}{3.696000in}}%
\pgfusepath{clip}%
\pgfsetbuttcap%
\pgfsetroundjoin%
\definecolor{currentfill}{rgb}{0.121569,0.466667,0.705882}%
\pgfsetfillcolor{currentfill}%
\pgfsetfillopacity{0.632619}%
\pgfsetlinewidth{1.003750pt}%
\definecolor{currentstroke}{rgb}{0.121569,0.466667,0.705882}%
\pgfsetstrokecolor{currentstroke}%
\pgfsetstrokeopacity{0.632619}%
\pgfsetdash{}{0pt}%
\pgfpathmoveto{\pgfqpoint{2.126009in}{1.988580in}}%
\pgfpathcurveto{\pgfqpoint{2.134246in}{1.988580in}}{\pgfqpoint{2.142146in}{1.991852in}}{\pgfqpoint{2.147970in}{1.997676in}}%
\pgfpathcurveto{\pgfqpoint{2.153794in}{2.003500in}}{\pgfqpoint{2.157066in}{2.011400in}}{\pgfqpoint{2.157066in}{2.019636in}}%
\pgfpathcurveto{\pgfqpoint{2.157066in}{2.027873in}}{\pgfqpoint{2.153794in}{2.035773in}}{\pgfqpoint{2.147970in}{2.041597in}}%
\pgfpathcurveto{\pgfqpoint{2.142146in}{2.047421in}}{\pgfqpoint{2.134246in}{2.050693in}}{\pgfqpoint{2.126009in}{2.050693in}}%
\pgfpathcurveto{\pgfqpoint{2.117773in}{2.050693in}}{\pgfqpoint{2.109873in}{2.047421in}}{\pgfqpoint{2.104049in}{2.041597in}}%
\pgfpathcurveto{\pgfqpoint{2.098225in}{2.035773in}}{\pgfqpoint{2.094953in}{2.027873in}}{\pgfqpoint{2.094953in}{2.019636in}}%
\pgfpathcurveto{\pgfqpoint{2.094953in}{2.011400in}}{\pgfqpoint{2.098225in}{2.003500in}}{\pgfqpoint{2.104049in}{1.997676in}}%
\pgfpathcurveto{\pgfqpoint{2.109873in}{1.991852in}}{\pgfqpoint{2.117773in}{1.988580in}}{\pgfqpoint{2.126009in}{1.988580in}}%
\pgfpathclose%
\pgfusepath{stroke,fill}%
\end{pgfscope}%
\begin{pgfscope}%
\pgfpathrectangle{\pgfqpoint{0.100000in}{0.212622in}}{\pgfqpoint{3.696000in}{3.696000in}}%
\pgfusepath{clip}%
\pgfsetbuttcap%
\pgfsetroundjoin%
\definecolor{currentfill}{rgb}{0.121569,0.466667,0.705882}%
\pgfsetfillcolor{currentfill}%
\pgfsetfillopacity{0.635072}%
\pgfsetlinewidth{1.003750pt}%
\definecolor{currentstroke}{rgb}{0.121569,0.466667,0.705882}%
\pgfsetstrokecolor{currentstroke}%
\pgfsetstrokeopacity{0.635072}%
\pgfsetdash{}{0pt}%
\pgfpathmoveto{\pgfqpoint{0.776862in}{1.206321in}}%
\pgfpathcurveto{\pgfqpoint{0.785099in}{1.206321in}}{\pgfqpoint{0.792999in}{1.209594in}}{\pgfqpoint{0.798823in}{1.215418in}}%
\pgfpathcurveto{\pgfqpoint{0.804647in}{1.221241in}}{\pgfqpoint{0.807919in}{1.229142in}}{\pgfqpoint{0.807919in}{1.237378in}}%
\pgfpathcurveto{\pgfqpoint{0.807919in}{1.245614in}}{\pgfqpoint{0.804647in}{1.253514in}}{\pgfqpoint{0.798823in}{1.259338in}}%
\pgfpathcurveto{\pgfqpoint{0.792999in}{1.265162in}}{\pgfqpoint{0.785099in}{1.268434in}}{\pgfqpoint{0.776862in}{1.268434in}}%
\pgfpathcurveto{\pgfqpoint{0.768626in}{1.268434in}}{\pgfqpoint{0.760726in}{1.265162in}}{\pgfqpoint{0.754902in}{1.259338in}}%
\pgfpathcurveto{\pgfqpoint{0.749078in}{1.253514in}}{\pgfqpoint{0.745806in}{1.245614in}}{\pgfqpoint{0.745806in}{1.237378in}}%
\pgfpathcurveto{\pgfqpoint{0.745806in}{1.229142in}}{\pgfqpoint{0.749078in}{1.221241in}}{\pgfqpoint{0.754902in}{1.215418in}}%
\pgfpathcurveto{\pgfqpoint{0.760726in}{1.209594in}}{\pgfqpoint{0.768626in}{1.206321in}}{\pgfqpoint{0.776862in}{1.206321in}}%
\pgfpathclose%
\pgfusepath{stroke,fill}%
\end{pgfscope}%
\begin{pgfscope}%
\pgfpathrectangle{\pgfqpoint{0.100000in}{0.212622in}}{\pgfqpoint{3.696000in}{3.696000in}}%
\pgfusepath{clip}%
\pgfsetbuttcap%
\pgfsetroundjoin%
\definecolor{currentfill}{rgb}{0.121569,0.466667,0.705882}%
\pgfsetfillcolor{currentfill}%
\pgfsetfillopacity{0.635330}%
\pgfsetlinewidth{1.003750pt}%
\definecolor{currentstroke}{rgb}{0.121569,0.466667,0.705882}%
\pgfsetstrokecolor{currentstroke}%
\pgfsetstrokeopacity{0.635330}%
\pgfsetdash{}{0pt}%
\pgfpathmoveto{\pgfqpoint{2.127634in}{1.979274in}}%
\pgfpathcurveto{\pgfqpoint{2.135870in}{1.979274in}}{\pgfqpoint{2.143770in}{1.982546in}}{\pgfqpoint{2.149594in}{1.988370in}}%
\pgfpathcurveto{\pgfqpoint{2.155418in}{1.994194in}}{\pgfqpoint{2.158691in}{2.002094in}}{\pgfqpoint{2.158691in}{2.010330in}}%
\pgfpathcurveto{\pgfqpoint{2.158691in}{2.018567in}}{\pgfqpoint{2.155418in}{2.026467in}}{\pgfqpoint{2.149594in}{2.032291in}}%
\pgfpathcurveto{\pgfqpoint{2.143770in}{2.038115in}}{\pgfqpoint{2.135870in}{2.041387in}}{\pgfqpoint{2.127634in}{2.041387in}}%
\pgfpathcurveto{\pgfqpoint{2.119398in}{2.041387in}}{\pgfqpoint{2.111498in}{2.038115in}}{\pgfqpoint{2.105674in}{2.032291in}}%
\pgfpathcurveto{\pgfqpoint{2.099850in}{2.026467in}}{\pgfqpoint{2.096578in}{2.018567in}}{\pgfqpoint{2.096578in}{2.010330in}}%
\pgfpathcurveto{\pgfqpoint{2.096578in}{2.002094in}}{\pgfqpoint{2.099850in}{1.994194in}}{\pgfqpoint{2.105674in}{1.988370in}}%
\pgfpathcurveto{\pgfqpoint{2.111498in}{1.982546in}}{\pgfqpoint{2.119398in}{1.979274in}}{\pgfqpoint{2.127634in}{1.979274in}}%
\pgfpathclose%
\pgfusepath{stroke,fill}%
\end{pgfscope}%
\begin{pgfscope}%
\pgfpathrectangle{\pgfqpoint{0.100000in}{0.212622in}}{\pgfqpoint{3.696000in}{3.696000in}}%
\pgfusepath{clip}%
\pgfsetbuttcap%
\pgfsetroundjoin%
\definecolor{currentfill}{rgb}{0.121569,0.466667,0.705882}%
\pgfsetfillcolor{currentfill}%
\pgfsetfillopacity{0.638215}%
\pgfsetlinewidth{1.003750pt}%
\definecolor{currentstroke}{rgb}{0.121569,0.466667,0.705882}%
\pgfsetstrokecolor{currentstroke}%
\pgfsetstrokeopacity{0.638215}%
\pgfsetdash{}{0pt}%
\pgfpathmoveto{\pgfqpoint{2.129619in}{1.969697in}}%
\pgfpathcurveto{\pgfqpoint{2.137855in}{1.969697in}}{\pgfqpoint{2.145755in}{1.972969in}}{\pgfqpoint{2.151579in}{1.978793in}}%
\pgfpathcurveto{\pgfqpoint{2.157403in}{1.984617in}}{\pgfqpoint{2.160675in}{1.992517in}}{\pgfqpoint{2.160675in}{2.000753in}}%
\pgfpathcurveto{\pgfqpoint{2.160675in}{2.008989in}}{\pgfqpoint{2.157403in}{2.016889in}}{\pgfqpoint{2.151579in}{2.022713in}}%
\pgfpathcurveto{\pgfqpoint{2.145755in}{2.028537in}}{\pgfqpoint{2.137855in}{2.031810in}}{\pgfqpoint{2.129619in}{2.031810in}}%
\pgfpathcurveto{\pgfqpoint{2.121382in}{2.031810in}}{\pgfqpoint{2.113482in}{2.028537in}}{\pgfqpoint{2.107658in}{2.022713in}}%
\pgfpathcurveto{\pgfqpoint{2.101834in}{2.016889in}}{\pgfqpoint{2.098562in}{2.008989in}}{\pgfqpoint{2.098562in}{2.000753in}}%
\pgfpathcurveto{\pgfqpoint{2.098562in}{1.992517in}}{\pgfqpoint{2.101834in}{1.984617in}}{\pgfqpoint{2.107658in}{1.978793in}}%
\pgfpathcurveto{\pgfqpoint{2.113482in}{1.972969in}}{\pgfqpoint{2.121382in}{1.969697in}}{\pgfqpoint{2.129619in}{1.969697in}}%
\pgfpathclose%
\pgfusepath{stroke,fill}%
\end{pgfscope}%
\begin{pgfscope}%
\pgfpathrectangle{\pgfqpoint{0.100000in}{0.212622in}}{\pgfqpoint{3.696000in}{3.696000in}}%
\pgfusepath{clip}%
\pgfsetbuttcap%
\pgfsetroundjoin%
\definecolor{currentfill}{rgb}{0.121569,0.466667,0.705882}%
\pgfsetfillcolor{currentfill}%
\pgfsetfillopacity{0.639512}%
\pgfsetlinewidth{1.003750pt}%
\definecolor{currentstroke}{rgb}{0.121569,0.466667,0.705882}%
\pgfsetstrokecolor{currentstroke}%
\pgfsetstrokeopacity{0.639512}%
\pgfsetdash{}{0pt}%
\pgfpathmoveto{\pgfqpoint{0.795991in}{1.207054in}}%
\pgfpathcurveto{\pgfqpoint{0.804227in}{1.207054in}}{\pgfqpoint{0.812127in}{1.210327in}}{\pgfqpoint{0.817951in}{1.216151in}}%
\pgfpathcurveto{\pgfqpoint{0.823775in}{1.221974in}}{\pgfqpoint{0.827048in}{1.229875in}}{\pgfqpoint{0.827048in}{1.238111in}}%
\pgfpathcurveto{\pgfqpoint{0.827048in}{1.246347in}}{\pgfqpoint{0.823775in}{1.254247in}}{\pgfqpoint{0.817951in}{1.260071in}}%
\pgfpathcurveto{\pgfqpoint{0.812127in}{1.265895in}}{\pgfqpoint{0.804227in}{1.269167in}}{\pgfqpoint{0.795991in}{1.269167in}}%
\pgfpathcurveto{\pgfqpoint{0.787755in}{1.269167in}}{\pgfqpoint{0.779855in}{1.265895in}}{\pgfqpoint{0.774031in}{1.260071in}}%
\pgfpathcurveto{\pgfqpoint{0.768207in}{1.254247in}}{\pgfqpoint{0.764935in}{1.246347in}}{\pgfqpoint{0.764935in}{1.238111in}}%
\pgfpathcurveto{\pgfqpoint{0.764935in}{1.229875in}}{\pgfqpoint{0.768207in}{1.221974in}}{\pgfqpoint{0.774031in}{1.216151in}}%
\pgfpathcurveto{\pgfqpoint{0.779855in}{1.210327in}}{\pgfqpoint{0.787755in}{1.207054in}}{\pgfqpoint{0.795991in}{1.207054in}}%
\pgfpathclose%
\pgfusepath{stroke,fill}%
\end{pgfscope}%
\begin{pgfscope}%
\pgfpathrectangle{\pgfqpoint{0.100000in}{0.212622in}}{\pgfqpoint{3.696000in}{3.696000in}}%
\pgfusepath{clip}%
\pgfsetbuttcap%
\pgfsetroundjoin%
\definecolor{currentfill}{rgb}{0.121569,0.466667,0.705882}%
\pgfsetfillcolor{currentfill}%
\pgfsetfillopacity{0.640813}%
\pgfsetlinewidth{1.003750pt}%
\definecolor{currentstroke}{rgb}{0.121569,0.466667,0.705882}%
\pgfsetstrokecolor{currentstroke}%
\pgfsetstrokeopacity{0.640813}%
\pgfsetdash{}{0pt}%
\pgfpathmoveto{\pgfqpoint{2.132438in}{1.958888in}}%
\pgfpathcurveto{\pgfqpoint{2.140674in}{1.958888in}}{\pgfqpoint{2.148574in}{1.962160in}}{\pgfqpoint{2.154398in}{1.967984in}}%
\pgfpathcurveto{\pgfqpoint{2.160222in}{1.973808in}}{\pgfqpoint{2.163495in}{1.981708in}}{\pgfqpoint{2.163495in}{1.989944in}}%
\pgfpathcurveto{\pgfqpoint{2.163495in}{1.998181in}}{\pgfqpoint{2.160222in}{2.006081in}}{\pgfqpoint{2.154398in}{2.011905in}}%
\pgfpathcurveto{\pgfqpoint{2.148574in}{2.017729in}}{\pgfqpoint{2.140674in}{2.021001in}}{\pgfqpoint{2.132438in}{2.021001in}}%
\pgfpathcurveto{\pgfqpoint{2.124202in}{2.021001in}}{\pgfqpoint{2.116302in}{2.017729in}}{\pgfqpoint{2.110478in}{2.011905in}}%
\pgfpathcurveto{\pgfqpoint{2.104654in}{2.006081in}}{\pgfqpoint{2.101382in}{1.998181in}}{\pgfqpoint{2.101382in}{1.989944in}}%
\pgfpathcurveto{\pgfqpoint{2.101382in}{1.981708in}}{\pgfqpoint{2.104654in}{1.973808in}}{\pgfqpoint{2.110478in}{1.967984in}}%
\pgfpathcurveto{\pgfqpoint{2.116302in}{1.962160in}}{\pgfqpoint{2.124202in}{1.958888in}}{\pgfqpoint{2.132438in}{1.958888in}}%
\pgfpathclose%
\pgfusepath{stroke,fill}%
\end{pgfscope}%
\begin{pgfscope}%
\pgfpathrectangle{\pgfqpoint{0.100000in}{0.212622in}}{\pgfqpoint{3.696000in}{3.696000in}}%
\pgfusepath{clip}%
\pgfsetbuttcap%
\pgfsetroundjoin%
\definecolor{currentfill}{rgb}{0.121569,0.466667,0.705882}%
\pgfsetfillcolor{currentfill}%
\pgfsetfillopacity{0.643654}%
\pgfsetlinewidth{1.003750pt}%
\definecolor{currentstroke}{rgb}{0.121569,0.466667,0.705882}%
\pgfsetstrokecolor{currentstroke}%
\pgfsetstrokeopacity{0.643654}%
\pgfsetdash{}{0pt}%
\pgfpathmoveto{\pgfqpoint{0.813414in}{1.207609in}}%
\pgfpathcurveto{\pgfqpoint{0.821650in}{1.207609in}}{\pgfqpoint{0.829550in}{1.210881in}}{\pgfqpoint{0.835374in}{1.216705in}}%
\pgfpathcurveto{\pgfqpoint{0.841198in}{1.222529in}}{\pgfqpoint{0.844470in}{1.230429in}}{\pgfqpoint{0.844470in}{1.238666in}}%
\pgfpathcurveto{\pgfqpoint{0.844470in}{1.246902in}}{\pgfqpoint{0.841198in}{1.254802in}}{\pgfqpoint{0.835374in}{1.260626in}}%
\pgfpathcurveto{\pgfqpoint{0.829550in}{1.266450in}}{\pgfqpoint{0.821650in}{1.269722in}}{\pgfqpoint{0.813414in}{1.269722in}}%
\pgfpathcurveto{\pgfqpoint{0.805177in}{1.269722in}}{\pgfqpoint{0.797277in}{1.266450in}}{\pgfqpoint{0.791453in}{1.260626in}}%
\pgfpathcurveto{\pgfqpoint{0.785630in}{1.254802in}}{\pgfqpoint{0.782357in}{1.246902in}}{\pgfqpoint{0.782357in}{1.238666in}}%
\pgfpathcurveto{\pgfqpoint{0.782357in}{1.230429in}}{\pgfqpoint{0.785630in}{1.222529in}}{\pgfqpoint{0.791453in}{1.216705in}}%
\pgfpathcurveto{\pgfqpoint{0.797277in}{1.210881in}}{\pgfqpoint{0.805177in}{1.207609in}}{\pgfqpoint{0.813414in}{1.207609in}}%
\pgfpathclose%
\pgfusepath{stroke,fill}%
\end{pgfscope}%
\begin{pgfscope}%
\pgfpathrectangle{\pgfqpoint{0.100000in}{0.212622in}}{\pgfqpoint{3.696000in}{3.696000in}}%
\pgfusepath{clip}%
\pgfsetbuttcap%
\pgfsetroundjoin%
\definecolor{currentfill}{rgb}{0.121569,0.466667,0.705882}%
\pgfsetfillcolor{currentfill}%
\pgfsetfillopacity{0.644312}%
\pgfsetlinewidth{1.003750pt}%
\definecolor{currentstroke}{rgb}{0.121569,0.466667,0.705882}%
\pgfsetstrokecolor{currentstroke}%
\pgfsetstrokeopacity{0.644312}%
\pgfsetdash{}{0pt}%
\pgfpathmoveto{\pgfqpoint{2.134846in}{1.946473in}}%
\pgfpathcurveto{\pgfqpoint{2.143082in}{1.946473in}}{\pgfqpoint{2.150982in}{1.949745in}}{\pgfqpoint{2.156806in}{1.955569in}}%
\pgfpathcurveto{\pgfqpoint{2.162630in}{1.961393in}}{\pgfqpoint{2.165902in}{1.969293in}}{\pgfqpoint{2.165902in}{1.977530in}}%
\pgfpathcurveto{\pgfqpoint{2.165902in}{1.985766in}}{\pgfqpoint{2.162630in}{1.993666in}}{\pgfqpoint{2.156806in}{1.999490in}}%
\pgfpathcurveto{\pgfqpoint{2.150982in}{2.005314in}}{\pgfqpoint{2.143082in}{2.008586in}}{\pgfqpoint{2.134846in}{2.008586in}}%
\pgfpathcurveto{\pgfqpoint{2.126610in}{2.008586in}}{\pgfqpoint{2.118710in}{2.005314in}}{\pgfqpoint{2.112886in}{1.999490in}}%
\pgfpathcurveto{\pgfqpoint{2.107062in}{1.993666in}}{\pgfqpoint{2.103789in}{1.985766in}}{\pgfqpoint{2.103789in}{1.977530in}}%
\pgfpathcurveto{\pgfqpoint{2.103789in}{1.969293in}}{\pgfqpoint{2.107062in}{1.961393in}}{\pgfqpoint{2.112886in}{1.955569in}}%
\pgfpathcurveto{\pgfqpoint{2.118710in}{1.949745in}}{\pgfqpoint{2.126610in}{1.946473in}}{\pgfqpoint{2.134846in}{1.946473in}}%
\pgfpathclose%
\pgfusepath{stroke,fill}%
\end{pgfscope}%
\begin{pgfscope}%
\pgfpathrectangle{\pgfqpoint{0.100000in}{0.212622in}}{\pgfqpoint{3.696000in}{3.696000in}}%
\pgfusepath{clip}%
\pgfsetbuttcap%
\pgfsetroundjoin%
\definecolor{currentfill}{rgb}{0.121569,0.466667,0.705882}%
\pgfsetfillcolor{currentfill}%
\pgfsetfillopacity{0.647952}%
\pgfsetlinewidth{1.003750pt}%
\definecolor{currentstroke}{rgb}{0.121569,0.466667,0.705882}%
\pgfsetstrokecolor{currentstroke}%
\pgfsetstrokeopacity{0.647952}%
\pgfsetdash{}{0pt}%
\pgfpathmoveto{\pgfqpoint{0.830271in}{1.207872in}}%
\pgfpathcurveto{\pgfqpoint{0.838507in}{1.207872in}}{\pgfqpoint{0.846407in}{1.211144in}}{\pgfqpoint{0.852231in}{1.216968in}}%
\pgfpathcurveto{\pgfqpoint{0.858055in}{1.222792in}}{\pgfqpoint{0.861327in}{1.230692in}}{\pgfqpoint{0.861327in}{1.238929in}}%
\pgfpathcurveto{\pgfqpoint{0.861327in}{1.247165in}}{\pgfqpoint{0.858055in}{1.255065in}}{\pgfqpoint{0.852231in}{1.260889in}}%
\pgfpathcurveto{\pgfqpoint{0.846407in}{1.266713in}}{\pgfqpoint{0.838507in}{1.269985in}}{\pgfqpoint{0.830271in}{1.269985in}}%
\pgfpathcurveto{\pgfqpoint{0.822035in}{1.269985in}}{\pgfqpoint{0.814135in}{1.266713in}}{\pgfqpoint{0.808311in}{1.260889in}}%
\pgfpathcurveto{\pgfqpoint{0.802487in}{1.255065in}}{\pgfqpoint{0.799214in}{1.247165in}}{\pgfqpoint{0.799214in}{1.238929in}}%
\pgfpathcurveto{\pgfqpoint{0.799214in}{1.230692in}}{\pgfqpoint{0.802487in}{1.222792in}}{\pgfqpoint{0.808311in}{1.216968in}}%
\pgfpathcurveto{\pgfqpoint{0.814135in}{1.211144in}}{\pgfqpoint{0.822035in}{1.207872in}}{\pgfqpoint{0.830271in}{1.207872in}}%
\pgfpathclose%
\pgfusepath{stroke,fill}%
\end{pgfscope}%
\begin{pgfscope}%
\pgfpathrectangle{\pgfqpoint{0.100000in}{0.212622in}}{\pgfqpoint{3.696000in}{3.696000in}}%
\pgfusepath{clip}%
\pgfsetbuttcap%
\pgfsetroundjoin%
\definecolor{currentfill}{rgb}{0.121569,0.466667,0.705882}%
\pgfsetfillcolor{currentfill}%
\pgfsetfillopacity{0.648222}%
\pgfsetlinewidth{1.003750pt}%
\definecolor{currentstroke}{rgb}{0.121569,0.466667,0.705882}%
\pgfsetstrokecolor{currentstroke}%
\pgfsetstrokeopacity{0.648222}%
\pgfsetdash{}{0pt}%
\pgfpathmoveto{\pgfqpoint{2.137133in}{1.934212in}}%
\pgfpathcurveto{\pgfqpoint{2.145369in}{1.934212in}}{\pgfqpoint{2.153269in}{1.937484in}}{\pgfqpoint{2.159093in}{1.943308in}}%
\pgfpathcurveto{\pgfqpoint{2.164917in}{1.949132in}}{\pgfqpoint{2.168189in}{1.957032in}}{\pgfqpoint{2.168189in}{1.965268in}}%
\pgfpathcurveto{\pgfqpoint{2.168189in}{1.973504in}}{\pgfqpoint{2.164917in}{1.981405in}}{\pgfqpoint{2.159093in}{1.987228in}}%
\pgfpathcurveto{\pgfqpoint{2.153269in}{1.993052in}}{\pgfqpoint{2.145369in}{1.996325in}}{\pgfqpoint{2.137133in}{1.996325in}}%
\pgfpathcurveto{\pgfqpoint{2.128897in}{1.996325in}}{\pgfqpoint{2.120996in}{1.993052in}}{\pgfqpoint{2.115173in}{1.987228in}}%
\pgfpathcurveto{\pgfqpoint{2.109349in}{1.981405in}}{\pgfqpoint{2.106076in}{1.973504in}}{\pgfqpoint{2.106076in}{1.965268in}}%
\pgfpathcurveto{\pgfqpoint{2.106076in}{1.957032in}}{\pgfqpoint{2.109349in}{1.949132in}}{\pgfqpoint{2.115173in}{1.943308in}}%
\pgfpathcurveto{\pgfqpoint{2.120996in}{1.937484in}}{\pgfqpoint{2.128897in}{1.934212in}}{\pgfqpoint{2.137133in}{1.934212in}}%
\pgfpathclose%
\pgfusepath{stroke,fill}%
\end{pgfscope}%
\begin{pgfscope}%
\pgfpathrectangle{\pgfqpoint{0.100000in}{0.212622in}}{\pgfqpoint{3.696000in}{3.696000in}}%
\pgfusepath{clip}%
\pgfsetbuttcap%
\pgfsetroundjoin%
\definecolor{currentfill}{rgb}{0.121569,0.466667,0.705882}%
\pgfsetfillcolor{currentfill}%
\pgfsetfillopacity{0.651280}%
\pgfsetlinewidth{1.003750pt}%
\definecolor{currentstroke}{rgb}{0.121569,0.466667,0.705882}%
\pgfsetstrokecolor{currentstroke}%
\pgfsetstrokeopacity{0.651280}%
\pgfsetdash{}{0pt}%
\pgfpathmoveto{\pgfqpoint{0.845324in}{1.207064in}}%
\pgfpathcurveto{\pgfqpoint{0.853560in}{1.207064in}}{\pgfqpoint{0.861460in}{1.210337in}}{\pgfqpoint{0.867284in}{1.216161in}}%
\pgfpathcurveto{\pgfqpoint{0.873108in}{1.221985in}}{\pgfqpoint{0.876380in}{1.229885in}}{\pgfqpoint{0.876380in}{1.238121in}}%
\pgfpathcurveto{\pgfqpoint{0.876380in}{1.246357in}}{\pgfqpoint{0.873108in}{1.254257in}}{\pgfqpoint{0.867284in}{1.260081in}}%
\pgfpathcurveto{\pgfqpoint{0.861460in}{1.265905in}}{\pgfqpoint{0.853560in}{1.269177in}}{\pgfqpoint{0.845324in}{1.269177in}}%
\pgfpathcurveto{\pgfqpoint{0.837087in}{1.269177in}}{\pgfqpoint{0.829187in}{1.265905in}}{\pgfqpoint{0.823363in}{1.260081in}}%
\pgfpathcurveto{\pgfqpoint{0.817540in}{1.254257in}}{\pgfqpoint{0.814267in}{1.246357in}}{\pgfqpoint{0.814267in}{1.238121in}}%
\pgfpathcurveto{\pgfqpoint{0.814267in}{1.229885in}}{\pgfqpoint{0.817540in}{1.221985in}}{\pgfqpoint{0.823363in}{1.216161in}}%
\pgfpathcurveto{\pgfqpoint{0.829187in}{1.210337in}}{\pgfqpoint{0.837087in}{1.207064in}}{\pgfqpoint{0.845324in}{1.207064in}}%
\pgfpathclose%
\pgfusepath{stroke,fill}%
\end{pgfscope}%
\begin{pgfscope}%
\pgfpathrectangle{\pgfqpoint{0.100000in}{0.212622in}}{\pgfqpoint{3.696000in}{3.696000in}}%
\pgfusepath{clip}%
\pgfsetbuttcap%
\pgfsetroundjoin%
\definecolor{currentfill}{rgb}{0.121569,0.466667,0.705882}%
\pgfsetfillcolor{currentfill}%
\pgfsetfillopacity{0.651975}%
\pgfsetlinewidth{1.003750pt}%
\definecolor{currentstroke}{rgb}{0.121569,0.466667,0.705882}%
\pgfsetstrokecolor{currentstroke}%
\pgfsetstrokeopacity{0.651975}%
\pgfsetdash{}{0pt}%
\pgfpathmoveto{\pgfqpoint{2.141042in}{1.920082in}}%
\pgfpathcurveto{\pgfqpoint{2.149279in}{1.920082in}}{\pgfqpoint{2.157179in}{1.923354in}}{\pgfqpoint{2.163003in}{1.929178in}}%
\pgfpathcurveto{\pgfqpoint{2.168827in}{1.935002in}}{\pgfqpoint{2.172099in}{1.942902in}}{\pgfqpoint{2.172099in}{1.951138in}}%
\pgfpathcurveto{\pgfqpoint{2.172099in}{1.959375in}}{\pgfqpoint{2.168827in}{1.967275in}}{\pgfqpoint{2.163003in}{1.973098in}}%
\pgfpathcurveto{\pgfqpoint{2.157179in}{1.978922in}}{\pgfqpoint{2.149279in}{1.982195in}}{\pgfqpoint{2.141042in}{1.982195in}}%
\pgfpathcurveto{\pgfqpoint{2.132806in}{1.982195in}}{\pgfqpoint{2.124906in}{1.978922in}}{\pgfqpoint{2.119082in}{1.973098in}}%
\pgfpathcurveto{\pgfqpoint{2.113258in}{1.967275in}}{\pgfqpoint{2.109986in}{1.959375in}}{\pgfqpoint{2.109986in}{1.951138in}}%
\pgfpathcurveto{\pgfqpoint{2.109986in}{1.942902in}}{\pgfqpoint{2.113258in}{1.935002in}}{\pgfqpoint{2.119082in}{1.929178in}}%
\pgfpathcurveto{\pgfqpoint{2.124906in}{1.923354in}}{\pgfqpoint{2.132806in}{1.920082in}}{\pgfqpoint{2.141042in}{1.920082in}}%
\pgfpathclose%
\pgfusepath{stroke,fill}%
\end{pgfscope}%
\begin{pgfscope}%
\pgfpathrectangle{\pgfqpoint{0.100000in}{0.212622in}}{\pgfqpoint{3.696000in}{3.696000in}}%
\pgfusepath{clip}%
\pgfsetbuttcap%
\pgfsetroundjoin%
\definecolor{currentfill}{rgb}{0.121569,0.466667,0.705882}%
\pgfsetfillcolor{currentfill}%
\pgfsetfillopacity{0.654651}%
\pgfsetlinewidth{1.003750pt}%
\definecolor{currentstroke}{rgb}{0.121569,0.466667,0.705882}%
\pgfsetstrokecolor{currentstroke}%
\pgfsetstrokeopacity{0.654651}%
\pgfsetdash{}{0pt}%
\pgfpathmoveto{\pgfqpoint{0.859395in}{1.206069in}}%
\pgfpathcurveto{\pgfqpoint{0.867631in}{1.206069in}}{\pgfqpoint{0.875531in}{1.209341in}}{\pgfqpoint{0.881355in}{1.215165in}}%
\pgfpathcurveto{\pgfqpoint{0.887179in}{1.220989in}}{\pgfqpoint{0.890451in}{1.228889in}}{\pgfqpoint{0.890451in}{1.237125in}}%
\pgfpathcurveto{\pgfqpoint{0.890451in}{1.245361in}}{\pgfqpoint{0.887179in}{1.253261in}}{\pgfqpoint{0.881355in}{1.259085in}}%
\pgfpathcurveto{\pgfqpoint{0.875531in}{1.264909in}}{\pgfqpoint{0.867631in}{1.268182in}}{\pgfqpoint{0.859395in}{1.268182in}}%
\pgfpathcurveto{\pgfqpoint{0.851158in}{1.268182in}}{\pgfqpoint{0.843258in}{1.264909in}}{\pgfqpoint{0.837434in}{1.259085in}}%
\pgfpathcurveto{\pgfqpoint{0.831610in}{1.253261in}}{\pgfqpoint{0.828338in}{1.245361in}}{\pgfqpoint{0.828338in}{1.237125in}}%
\pgfpathcurveto{\pgfqpoint{0.828338in}{1.228889in}}{\pgfqpoint{0.831610in}{1.220989in}}{\pgfqpoint{0.837434in}{1.215165in}}%
\pgfpathcurveto{\pgfqpoint{0.843258in}{1.209341in}}{\pgfqpoint{0.851158in}{1.206069in}}{\pgfqpoint{0.859395in}{1.206069in}}%
\pgfpathclose%
\pgfusepath{stroke,fill}%
\end{pgfscope}%
\begin{pgfscope}%
\pgfpathrectangle{\pgfqpoint{0.100000in}{0.212622in}}{\pgfqpoint{3.696000in}{3.696000in}}%
\pgfusepath{clip}%
\pgfsetbuttcap%
\pgfsetroundjoin%
\definecolor{currentfill}{rgb}{0.121569,0.466667,0.705882}%
\pgfsetfillcolor{currentfill}%
\pgfsetfillopacity{0.656320}%
\pgfsetlinewidth{1.003750pt}%
\definecolor{currentstroke}{rgb}{0.121569,0.466667,0.705882}%
\pgfsetstrokecolor{currentstroke}%
\pgfsetstrokeopacity{0.656320}%
\pgfsetdash{}{0pt}%
\pgfpathmoveto{\pgfqpoint{2.144133in}{1.906538in}}%
\pgfpathcurveto{\pgfqpoint{2.152370in}{1.906538in}}{\pgfqpoint{2.160270in}{1.909811in}}{\pgfqpoint{2.166094in}{1.915635in}}%
\pgfpathcurveto{\pgfqpoint{2.171918in}{1.921459in}}{\pgfqpoint{2.175190in}{1.929359in}}{\pgfqpoint{2.175190in}{1.937595in}}%
\pgfpathcurveto{\pgfqpoint{2.175190in}{1.945831in}}{\pgfqpoint{2.171918in}{1.953731in}}{\pgfqpoint{2.166094in}{1.959555in}}%
\pgfpathcurveto{\pgfqpoint{2.160270in}{1.965379in}}{\pgfqpoint{2.152370in}{1.968651in}}{\pgfqpoint{2.144133in}{1.968651in}}%
\pgfpathcurveto{\pgfqpoint{2.135897in}{1.968651in}}{\pgfqpoint{2.127997in}{1.965379in}}{\pgfqpoint{2.122173in}{1.959555in}}%
\pgfpathcurveto{\pgfqpoint{2.116349in}{1.953731in}}{\pgfqpoint{2.113077in}{1.945831in}}{\pgfqpoint{2.113077in}{1.937595in}}%
\pgfpathcurveto{\pgfqpoint{2.113077in}{1.929359in}}{\pgfqpoint{2.116349in}{1.921459in}}{\pgfqpoint{2.122173in}{1.915635in}}%
\pgfpathcurveto{\pgfqpoint{2.127997in}{1.909811in}}{\pgfqpoint{2.135897in}{1.906538in}}{\pgfqpoint{2.144133in}{1.906538in}}%
\pgfpathclose%
\pgfusepath{stroke,fill}%
\end{pgfscope}%
\begin{pgfscope}%
\pgfpathrectangle{\pgfqpoint{0.100000in}{0.212622in}}{\pgfqpoint{3.696000in}{3.696000in}}%
\pgfusepath{clip}%
\pgfsetbuttcap%
\pgfsetroundjoin%
\definecolor{currentfill}{rgb}{0.121569,0.466667,0.705882}%
\pgfsetfillcolor{currentfill}%
\pgfsetfillopacity{0.657979}%
\pgfsetlinewidth{1.003750pt}%
\definecolor{currentstroke}{rgb}{0.121569,0.466667,0.705882}%
\pgfsetstrokecolor{currentstroke}%
\pgfsetstrokeopacity{0.657979}%
\pgfsetdash{}{0pt}%
\pgfpathmoveto{\pgfqpoint{0.873297in}{1.205977in}}%
\pgfpathcurveto{\pgfqpoint{0.881533in}{1.205977in}}{\pgfqpoint{0.889433in}{1.209249in}}{\pgfqpoint{0.895257in}{1.215073in}}%
\pgfpathcurveto{\pgfqpoint{0.901081in}{1.220897in}}{\pgfqpoint{0.904353in}{1.228797in}}{\pgfqpoint{0.904353in}{1.237033in}}%
\pgfpathcurveto{\pgfqpoint{0.904353in}{1.245269in}}{\pgfqpoint{0.901081in}{1.253169in}}{\pgfqpoint{0.895257in}{1.258993in}}%
\pgfpathcurveto{\pgfqpoint{0.889433in}{1.264817in}}{\pgfqpoint{0.881533in}{1.268090in}}{\pgfqpoint{0.873297in}{1.268090in}}%
\pgfpathcurveto{\pgfqpoint{0.865060in}{1.268090in}}{\pgfqpoint{0.857160in}{1.264817in}}{\pgfqpoint{0.851336in}{1.258993in}}%
\pgfpathcurveto{\pgfqpoint{0.845512in}{1.253169in}}{\pgfqpoint{0.842240in}{1.245269in}}{\pgfqpoint{0.842240in}{1.237033in}}%
\pgfpathcurveto{\pgfqpoint{0.842240in}{1.228797in}}{\pgfqpoint{0.845512in}{1.220897in}}{\pgfqpoint{0.851336in}{1.215073in}}%
\pgfpathcurveto{\pgfqpoint{0.857160in}{1.209249in}}{\pgfqpoint{0.865060in}{1.205977in}}{\pgfqpoint{0.873297in}{1.205977in}}%
\pgfpathclose%
\pgfusepath{stroke,fill}%
\end{pgfscope}%
\begin{pgfscope}%
\pgfpathrectangle{\pgfqpoint{0.100000in}{0.212622in}}{\pgfqpoint{3.696000in}{3.696000in}}%
\pgfusepath{clip}%
\pgfsetbuttcap%
\pgfsetroundjoin%
\definecolor{currentfill}{rgb}{0.121569,0.466667,0.705882}%
\pgfsetfillcolor{currentfill}%
\pgfsetfillopacity{0.660970}%
\pgfsetlinewidth{1.003750pt}%
\definecolor{currentstroke}{rgb}{0.121569,0.466667,0.705882}%
\pgfsetstrokecolor{currentstroke}%
\pgfsetstrokeopacity{0.660970}%
\pgfsetdash{}{0pt}%
\pgfpathmoveto{\pgfqpoint{0.886813in}{1.205599in}}%
\pgfpathcurveto{\pgfqpoint{0.895049in}{1.205599in}}{\pgfqpoint{0.902949in}{1.208871in}}{\pgfqpoint{0.908773in}{1.214695in}}%
\pgfpathcurveto{\pgfqpoint{0.914597in}{1.220519in}}{\pgfqpoint{0.917869in}{1.228419in}}{\pgfqpoint{0.917869in}{1.236655in}}%
\pgfpathcurveto{\pgfqpoint{0.917869in}{1.244891in}}{\pgfqpoint{0.914597in}{1.252791in}}{\pgfqpoint{0.908773in}{1.258615in}}%
\pgfpathcurveto{\pgfqpoint{0.902949in}{1.264439in}}{\pgfqpoint{0.895049in}{1.267712in}}{\pgfqpoint{0.886813in}{1.267712in}}%
\pgfpathcurveto{\pgfqpoint{0.878577in}{1.267712in}}{\pgfqpoint{0.870676in}{1.264439in}}{\pgfqpoint{0.864853in}{1.258615in}}%
\pgfpathcurveto{\pgfqpoint{0.859029in}{1.252791in}}{\pgfqpoint{0.855756in}{1.244891in}}{\pgfqpoint{0.855756in}{1.236655in}}%
\pgfpathcurveto{\pgfqpoint{0.855756in}{1.228419in}}{\pgfqpoint{0.859029in}{1.220519in}}{\pgfqpoint{0.864853in}{1.214695in}}%
\pgfpathcurveto{\pgfqpoint{0.870676in}{1.208871in}}{\pgfqpoint{0.878577in}{1.205599in}}{\pgfqpoint{0.886813in}{1.205599in}}%
\pgfpathclose%
\pgfusepath{stroke,fill}%
\end{pgfscope}%
\begin{pgfscope}%
\pgfpathrectangle{\pgfqpoint{0.100000in}{0.212622in}}{\pgfqpoint{3.696000in}{3.696000in}}%
\pgfusepath{clip}%
\pgfsetbuttcap%
\pgfsetroundjoin%
\definecolor{currentfill}{rgb}{0.121569,0.466667,0.705882}%
\pgfsetfillcolor{currentfill}%
\pgfsetfillopacity{0.661320}%
\pgfsetlinewidth{1.003750pt}%
\definecolor{currentstroke}{rgb}{0.121569,0.466667,0.705882}%
\pgfsetstrokecolor{currentstroke}%
\pgfsetstrokeopacity{0.661320}%
\pgfsetdash{}{0pt}%
\pgfpathmoveto{\pgfqpoint{2.146972in}{1.892810in}}%
\pgfpathcurveto{\pgfqpoint{2.155208in}{1.892810in}}{\pgfqpoint{2.163109in}{1.896082in}}{\pgfqpoint{2.168932in}{1.901906in}}%
\pgfpathcurveto{\pgfqpoint{2.174756in}{1.907730in}}{\pgfqpoint{2.178029in}{1.915630in}}{\pgfqpoint{2.178029in}{1.923866in}}%
\pgfpathcurveto{\pgfqpoint{2.178029in}{1.932102in}}{\pgfqpoint{2.174756in}{1.940002in}}{\pgfqpoint{2.168932in}{1.945826in}}%
\pgfpathcurveto{\pgfqpoint{2.163109in}{1.951650in}}{\pgfqpoint{2.155208in}{1.954923in}}{\pgfqpoint{2.146972in}{1.954923in}}%
\pgfpathcurveto{\pgfqpoint{2.138736in}{1.954923in}}{\pgfqpoint{2.130836in}{1.951650in}}{\pgfqpoint{2.125012in}{1.945826in}}%
\pgfpathcurveto{\pgfqpoint{2.119188in}{1.940002in}}{\pgfqpoint{2.115916in}{1.932102in}}{\pgfqpoint{2.115916in}{1.923866in}}%
\pgfpathcurveto{\pgfqpoint{2.115916in}{1.915630in}}{\pgfqpoint{2.119188in}{1.907730in}}{\pgfqpoint{2.125012in}{1.901906in}}%
\pgfpathcurveto{\pgfqpoint{2.130836in}{1.896082in}}{\pgfqpoint{2.138736in}{1.892810in}}{\pgfqpoint{2.146972in}{1.892810in}}%
\pgfpathclose%
\pgfusepath{stroke,fill}%
\end{pgfscope}%
\begin{pgfscope}%
\pgfpathrectangle{\pgfqpoint{0.100000in}{0.212622in}}{\pgfqpoint{3.696000in}{3.696000in}}%
\pgfusepath{clip}%
\pgfsetbuttcap%
\pgfsetroundjoin%
\definecolor{currentfill}{rgb}{0.121569,0.466667,0.705882}%
\pgfsetfillcolor{currentfill}%
\pgfsetfillopacity{0.663711}%
\pgfsetlinewidth{1.003750pt}%
\definecolor{currentstroke}{rgb}{0.121569,0.466667,0.705882}%
\pgfsetstrokecolor{currentstroke}%
\pgfsetstrokeopacity{0.663711}%
\pgfsetdash{}{0pt}%
\pgfpathmoveto{\pgfqpoint{0.899048in}{1.205639in}}%
\pgfpathcurveto{\pgfqpoint{0.907284in}{1.205639in}}{\pgfqpoint{0.915184in}{1.208912in}}{\pgfqpoint{0.921008in}{1.214736in}}%
\pgfpathcurveto{\pgfqpoint{0.926832in}{1.220560in}}{\pgfqpoint{0.930105in}{1.228460in}}{\pgfqpoint{0.930105in}{1.236696in}}%
\pgfpathcurveto{\pgfqpoint{0.930105in}{1.244932in}}{\pgfqpoint{0.926832in}{1.252832in}}{\pgfqpoint{0.921008in}{1.258656in}}%
\pgfpathcurveto{\pgfqpoint{0.915184in}{1.264480in}}{\pgfqpoint{0.907284in}{1.267752in}}{\pgfqpoint{0.899048in}{1.267752in}}%
\pgfpathcurveto{\pgfqpoint{0.890812in}{1.267752in}}{\pgfqpoint{0.882912in}{1.264480in}}{\pgfqpoint{0.877088in}{1.258656in}}%
\pgfpathcurveto{\pgfqpoint{0.871264in}{1.252832in}}{\pgfqpoint{0.867992in}{1.244932in}}{\pgfqpoint{0.867992in}{1.236696in}}%
\pgfpathcurveto{\pgfqpoint{0.867992in}{1.228460in}}{\pgfqpoint{0.871264in}{1.220560in}}{\pgfqpoint{0.877088in}{1.214736in}}%
\pgfpathcurveto{\pgfqpoint{0.882912in}{1.208912in}}{\pgfqpoint{0.890812in}{1.205639in}}{\pgfqpoint{0.899048in}{1.205639in}}%
\pgfpathclose%
\pgfusepath{stroke,fill}%
\end{pgfscope}%
\begin{pgfscope}%
\pgfpathrectangle{\pgfqpoint{0.100000in}{0.212622in}}{\pgfqpoint{3.696000in}{3.696000in}}%
\pgfusepath{clip}%
\pgfsetbuttcap%
\pgfsetroundjoin%
\definecolor{currentfill}{rgb}{0.121569,0.466667,0.705882}%
\pgfsetfillcolor{currentfill}%
\pgfsetfillopacity{0.666375}%
\pgfsetlinewidth{1.003750pt}%
\definecolor{currentstroke}{rgb}{0.121569,0.466667,0.705882}%
\pgfsetstrokecolor{currentstroke}%
\pgfsetstrokeopacity{0.666375}%
\pgfsetdash{}{0pt}%
\pgfpathmoveto{\pgfqpoint{2.151875in}{1.876932in}}%
\pgfpathcurveto{\pgfqpoint{2.160111in}{1.876932in}}{\pgfqpoint{2.168011in}{1.880205in}}{\pgfqpoint{2.173835in}{1.886028in}}%
\pgfpathcurveto{\pgfqpoint{2.179659in}{1.891852in}}{\pgfqpoint{2.182931in}{1.899752in}}{\pgfqpoint{2.182931in}{1.907989in}}%
\pgfpathcurveto{\pgfqpoint{2.182931in}{1.916225in}}{\pgfqpoint{2.179659in}{1.924125in}}{\pgfqpoint{2.173835in}{1.929949in}}%
\pgfpathcurveto{\pgfqpoint{2.168011in}{1.935773in}}{\pgfqpoint{2.160111in}{1.939045in}}{\pgfqpoint{2.151875in}{1.939045in}}%
\pgfpathcurveto{\pgfqpoint{2.143638in}{1.939045in}}{\pgfqpoint{2.135738in}{1.935773in}}{\pgfqpoint{2.129914in}{1.929949in}}%
\pgfpathcurveto{\pgfqpoint{2.124090in}{1.924125in}}{\pgfqpoint{2.120818in}{1.916225in}}{\pgfqpoint{2.120818in}{1.907989in}}%
\pgfpathcurveto{\pgfqpoint{2.120818in}{1.899752in}}{\pgfqpoint{2.124090in}{1.891852in}}{\pgfqpoint{2.129914in}{1.886028in}}%
\pgfpathcurveto{\pgfqpoint{2.135738in}{1.880205in}}{\pgfqpoint{2.143638in}{1.876932in}}{\pgfqpoint{2.151875in}{1.876932in}}%
\pgfpathclose%
\pgfusepath{stroke,fill}%
\end{pgfscope}%
\begin{pgfscope}%
\pgfpathrectangle{\pgfqpoint{0.100000in}{0.212622in}}{\pgfqpoint{3.696000in}{3.696000in}}%
\pgfusepath{clip}%
\pgfsetbuttcap%
\pgfsetroundjoin%
\definecolor{currentfill}{rgb}{0.121569,0.466667,0.705882}%
\pgfsetfillcolor{currentfill}%
\pgfsetfillopacity{0.666494}%
\pgfsetlinewidth{1.003750pt}%
\definecolor{currentstroke}{rgb}{0.121569,0.466667,0.705882}%
\pgfsetstrokecolor{currentstroke}%
\pgfsetstrokeopacity{0.666494}%
\pgfsetdash{}{0pt}%
\pgfpathmoveto{\pgfqpoint{0.910968in}{1.206153in}}%
\pgfpathcurveto{\pgfqpoint{0.919205in}{1.206153in}}{\pgfqpoint{0.927105in}{1.209425in}}{\pgfqpoint{0.932928in}{1.215249in}}%
\pgfpathcurveto{\pgfqpoint{0.938752in}{1.221073in}}{\pgfqpoint{0.942025in}{1.228973in}}{\pgfqpoint{0.942025in}{1.237209in}}%
\pgfpathcurveto{\pgfqpoint{0.942025in}{1.245445in}}{\pgfqpoint{0.938752in}{1.253345in}}{\pgfqpoint{0.932928in}{1.259169in}}%
\pgfpathcurveto{\pgfqpoint{0.927105in}{1.264993in}}{\pgfqpoint{0.919205in}{1.268266in}}{\pgfqpoint{0.910968in}{1.268266in}}%
\pgfpathcurveto{\pgfqpoint{0.902732in}{1.268266in}}{\pgfqpoint{0.894832in}{1.264993in}}{\pgfqpoint{0.889008in}{1.259169in}}%
\pgfpathcurveto{\pgfqpoint{0.883184in}{1.253345in}}{\pgfqpoint{0.879912in}{1.245445in}}{\pgfqpoint{0.879912in}{1.237209in}}%
\pgfpathcurveto{\pgfqpoint{0.879912in}{1.228973in}}{\pgfqpoint{0.883184in}{1.221073in}}{\pgfqpoint{0.889008in}{1.215249in}}%
\pgfpathcurveto{\pgfqpoint{0.894832in}{1.209425in}}{\pgfqpoint{0.902732in}{1.206153in}}{\pgfqpoint{0.910968in}{1.206153in}}%
\pgfpathclose%
\pgfusepath{stroke,fill}%
\end{pgfscope}%
\begin{pgfscope}%
\pgfpathrectangle{\pgfqpoint{0.100000in}{0.212622in}}{\pgfqpoint{3.696000in}{3.696000in}}%
\pgfusepath{clip}%
\pgfsetbuttcap%
\pgfsetroundjoin%
\definecolor{currentfill}{rgb}{0.121569,0.466667,0.705882}%
\pgfsetfillcolor{currentfill}%
\pgfsetfillopacity{0.668438}%
\pgfsetlinewidth{1.003750pt}%
\definecolor{currentstroke}{rgb}{0.121569,0.466667,0.705882}%
\pgfsetstrokecolor{currentstroke}%
\pgfsetstrokeopacity{0.668438}%
\pgfsetdash{}{0pt}%
\pgfpathmoveto{\pgfqpoint{0.922012in}{1.205249in}}%
\pgfpathcurveto{\pgfqpoint{0.930248in}{1.205249in}}{\pgfqpoint{0.938148in}{1.208521in}}{\pgfqpoint{0.943972in}{1.214345in}}%
\pgfpathcurveto{\pgfqpoint{0.949796in}{1.220169in}}{\pgfqpoint{0.953068in}{1.228069in}}{\pgfqpoint{0.953068in}{1.236305in}}%
\pgfpathcurveto{\pgfqpoint{0.953068in}{1.244541in}}{\pgfqpoint{0.949796in}{1.252441in}}{\pgfqpoint{0.943972in}{1.258265in}}%
\pgfpathcurveto{\pgfqpoint{0.938148in}{1.264089in}}{\pgfqpoint{0.930248in}{1.267362in}}{\pgfqpoint{0.922012in}{1.267362in}}%
\pgfpathcurveto{\pgfqpoint{0.913776in}{1.267362in}}{\pgfqpoint{0.905876in}{1.264089in}}{\pgfqpoint{0.900052in}{1.258265in}}%
\pgfpathcurveto{\pgfqpoint{0.894228in}{1.252441in}}{\pgfqpoint{0.890955in}{1.244541in}}{\pgfqpoint{0.890955in}{1.236305in}}%
\pgfpathcurveto{\pgfqpoint{0.890955in}{1.228069in}}{\pgfqpoint{0.894228in}{1.220169in}}{\pgfqpoint{0.900052in}{1.214345in}}%
\pgfpathcurveto{\pgfqpoint{0.905876in}{1.208521in}}{\pgfqpoint{0.913776in}{1.205249in}}{\pgfqpoint{0.922012in}{1.205249in}}%
\pgfpathclose%
\pgfusepath{stroke,fill}%
\end{pgfscope}%
\begin{pgfscope}%
\pgfpathrectangle{\pgfqpoint{0.100000in}{0.212622in}}{\pgfqpoint{3.696000in}{3.696000in}}%
\pgfusepath{clip}%
\pgfsetbuttcap%
\pgfsetroundjoin%
\definecolor{currentfill}{rgb}{0.121569,0.466667,0.705882}%
\pgfsetfillcolor{currentfill}%
\pgfsetfillopacity{0.670380}%
\pgfsetlinewidth{1.003750pt}%
\definecolor{currentstroke}{rgb}{0.121569,0.466667,0.705882}%
\pgfsetstrokecolor{currentstroke}%
\pgfsetstrokeopacity{0.670380}%
\pgfsetdash{}{0pt}%
\pgfpathmoveto{\pgfqpoint{0.931783in}{1.204344in}}%
\pgfpathcurveto{\pgfqpoint{0.940019in}{1.204344in}}{\pgfqpoint{0.947919in}{1.207616in}}{\pgfqpoint{0.953743in}{1.213440in}}%
\pgfpathcurveto{\pgfqpoint{0.959567in}{1.219264in}}{\pgfqpoint{0.962840in}{1.227164in}}{\pgfqpoint{0.962840in}{1.235400in}}%
\pgfpathcurveto{\pgfqpoint{0.962840in}{1.243637in}}{\pgfqpoint{0.959567in}{1.251537in}}{\pgfqpoint{0.953743in}{1.257361in}}%
\pgfpathcurveto{\pgfqpoint{0.947919in}{1.263184in}}{\pgfqpoint{0.940019in}{1.266457in}}{\pgfqpoint{0.931783in}{1.266457in}}%
\pgfpathcurveto{\pgfqpoint{0.923547in}{1.266457in}}{\pgfqpoint{0.915647in}{1.263184in}}{\pgfqpoint{0.909823in}{1.257361in}}%
\pgfpathcurveto{\pgfqpoint{0.903999in}{1.251537in}}{\pgfqpoint{0.900727in}{1.243637in}}{\pgfqpoint{0.900727in}{1.235400in}}%
\pgfpathcurveto{\pgfqpoint{0.900727in}{1.227164in}}{\pgfqpoint{0.903999in}{1.219264in}}{\pgfqpoint{0.909823in}{1.213440in}}%
\pgfpathcurveto{\pgfqpoint{0.915647in}{1.207616in}}{\pgfqpoint{0.923547in}{1.204344in}}{\pgfqpoint{0.931783in}{1.204344in}}%
\pgfpathclose%
\pgfusepath{stroke,fill}%
\end{pgfscope}%
\begin{pgfscope}%
\pgfpathrectangle{\pgfqpoint{0.100000in}{0.212622in}}{\pgfqpoint{3.696000in}{3.696000in}}%
\pgfusepath{clip}%
\pgfsetbuttcap%
\pgfsetroundjoin%
\definecolor{currentfill}{rgb}{0.121569,0.466667,0.705882}%
\pgfsetfillcolor{currentfill}%
\pgfsetfillopacity{0.672141}%
\pgfsetlinewidth{1.003750pt}%
\definecolor{currentstroke}{rgb}{0.121569,0.466667,0.705882}%
\pgfsetstrokecolor{currentstroke}%
\pgfsetstrokeopacity{0.672141}%
\pgfsetdash{}{0pt}%
\pgfpathmoveto{\pgfqpoint{2.156477in}{1.860995in}}%
\pgfpathcurveto{\pgfqpoint{2.164713in}{1.860995in}}{\pgfqpoint{2.172613in}{1.864267in}}{\pgfqpoint{2.178437in}{1.870091in}}%
\pgfpathcurveto{\pgfqpoint{2.184261in}{1.875915in}}{\pgfqpoint{2.187534in}{1.883815in}}{\pgfqpoint{2.187534in}{1.892051in}}%
\pgfpathcurveto{\pgfqpoint{2.187534in}{1.900288in}}{\pgfqpoint{2.184261in}{1.908188in}}{\pgfqpoint{2.178437in}{1.914012in}}%
\pgfpathcurveto{\pgfqpoint{2.172613in}{1.919835in}}{\pgfqpoint{2.164713in}{1.923108in}}{\pgfqpoint{2.156477in}{1.923108in}}%
\pgfpathcurveto{\pgfqpoint{2.148241in}{1.923108in}}{\pgfqpoint{2.140341in}{1.919835in}}{\pgfqpoint{2.134517in}{1.914012in}}%
\pgfpathcurveto{\pgfqpoint{2.128693in}{1.908188in}}{\pgfqpoint{2.125421in}{1.900288in}}{\pgfqpoint{2.125421in}{1.892051in}}%
\pgfpathcurveto{\pgfqpoint{2.125421in}{1.883815in}}{\pgfqpoint{2.128693in}{1.875915in}}{\pgfqpoint{2.134517in}{1.870091in}}%
\pgfpathcurveto{\pgfqpoint{2.140341in}{1.864267in}}{\pgfqpoint{2.148241in}{1.860995in}}{\pgfqpoint{2.156477in}{1.860995in}}%
\pgfpathclose%
\pgfusepath{stroke,fill}%
\end{pgfscope}%
\begin{pgfscope}%
\pgfpathrectangle{\pgfqpoint{0.100000in}{0.212622in}}{\pgfqpoint{3.696000in}{3.696000in}}%
\pgfusepath{clip}%
\pgfsetbuttcap%
\pgfsetroundjoin%
\definecolor{currentfill}{rgb}{0.121569,0.466667,0.705882}%
\pgfsetfillcolor{currentfill}%
\pgfsetfillopacity{0.672376}%
\pgfsetlinewidth{1.003750pt}%
\definecolor{currentstroke}{rgb}{0.121569,0.466667,0.705882}%
\pgfsetstrokecolor{currentstroke}%
\pgfsetstrokeopacity{0.672376}%
\pgfsetdash{}{0pt}%
\pgfpathmoveto{\pgfqpoint{0.941319in}{1.203863in}}%
\pgfpathcurveto{\pgfqpoint{0.949555in}{1.203863in}}{\pgfqpoint{0.957455in}{1.207136in}}{\pgfqpoint{0.963279in}{1.212960in}}%
\pgfpathcurveto{\pgfqpoint{0.969103in}{1.218784in}}{\pgfqpoint{0.972375in}{1.226684in}}{\pgfqpoint{0.972375in}{1.234920in}}%
\pgfpathcurveto{\pgfqpoint{0.972375in}{1.243156in}}{\pgfqpoint{0.969103in}{1.251056in}}{\pgfqpoint{0.963279in}{1.256880in}}%
\pgfpathcurveto{\pgfqpoint{0.957455in}{1.262704in}}{\pgfqpoint{0.949555in}{1.265976in}}{\pgfqpoint{0.941319in}{1.265976in}}%
\pgfpathcurveto{\pgfqpoint{0.933082in}{1.265976in}}{\pgfqpoint{0.925182in}{1.262704in}}{\pgfqpoint{0.919358in}{1.256880in}}%
\pgfpathcurveto{\pgfqpoint{0.913534in}{1.251056in}}{\pgfqpoint{0.910262in}{1.243156in}}{\pgfqpoint{0.910262in}{1.234920in}}%
\pgfpathcurveto{\pgfqpoint{0.910262in}{1.226684in}}{\pgfqpoint{0.913534in}{1.218784in}}{\pgfqpoint{0.919358in}{1.212960in}}%
\pgfpathcurveto{\pgfqpoint{0.925182in}{1.207136in}}{\pgfqpoint{0.933082in}{1.203863in}}{\pgfqpoint{0.941319in}{1.203863in}}%
\pgfpathclose%
\pgfusepath{stroke,fill}%
\end{pgfscope}%
\begin{pgfscope}%
\pgfpathrectangle{\pgfqpoint{0.100000in}{0.212622in}}{\pgfqpoint{3.696000in}{3.696000in}}%
\pgfusepath{clip}%
\pgfsetbuttcap%
\pgfsetroundjoin%
\definecolor{currentfill}{rgb}{0.121569,0.466667,0.705882}%
\pgfsetfillcolor{currentfill}%
\pgfsetfillopacity{0.674140}%
\pgfsetlinewidth{1.003750pt}%
\definecolor{currentstroke}{rgb}{0.121569,0.466667,0.705882}%
\pgfsetstrokecolor{currentstroke}%
\pgfsetstrokeopacity{0.674140}%
\pgfsetdash{}{0pt}%
\pgfpathmoveto{\pgfqpoint{0.950460in}{1.203272in}}%
\pgfpathcurveto{\pgfqpoint{0.958697in}{1.203272in}}{\pgfqpoint{0.966597in}{1.206544in}}{\pgfqpoint{0.972421in}{1.212368in}}%
\pgfpathcurveto{\pgfqpoint{0.978245in}{1.218192in}}{\pgfqpoint{0.981517in}{1.226092in}}{\pgfqpoint{0.981517in}{1.234329in}}%
\pgfpathcurveto{\pgfqpoint{0.981517in}{1.242565in}}{\pgfqpoint{0.978245in}{1.250465in}}{\pgfqpoint{0.972421in}{1.256289in}}%
\pgfpathcurveto{\pgfqpoint{0.966597in}{1.262113in}}{\pgfqpoint{0.958697in}{1.265385in}}{\pgfqpoint{0.950460in}{1.265385in}}%
\pgfpathcurveto{\pgfqpoint{0.942224in}{1.265385in}}{\pgfqpoint{0.934324in}{1.262113in}}{\pgfqpoint{0.928500in}{1.256289in}}%
\pgfpathcurveto{\pgfqpoint{0.922676in}{1.250465in}}{\pgfqpoint{0.919404in}{1.242565in}}{\pgfqpoint{0.919404in}{1.234329in}}%
\pgfpathcurveto{\pgfqpoint{0.919404in}{1.226092in}}{\pgfqpoint{0.922676in}{1.218192in}}{\pgfqpoint{0.928500in}{1.212368in}}%
\pgfpathcurveto{\pgfqpoint{0.934324in}{1.206544in}}{\pgfqpoint{0.942224in}{1.203272in}}{\pgfqpoint{0.950460in}{1.203272in}}%
\pgfpathclose%
\pgfusepath{stroke,fill}%
\end{pgfscope}%
\begin{pgfscope}%
\pgfpathrectangle{\pgfqpoint{0.100000in}{0.212622in}}{\pgfqpoint{3.696000in}{3.696000in}}%
\pgfusepath{clip}%
\pgfsetbuttcap%
\pgfsetroundjoin%
\definecolor{currentfill}{rgb}{0.121569,0.466667,0.705882}%
\pgfsetfillcolor{currentfill}%
\pgfsetfillopacity{0.675792}%
\pgfsetlinewidth{1.003750pt}%
\definecolor{currentstroke}{rgb}{0.121569,0.466667,0.705882}%
\pgfsetstrokecolor{currentstroke}%
\pgfsetstrokeopacity{0.675792}%
\pgfsetdash{}{0pt}%
\pgfpathmoveto{\pgfqpoint{0.959279in}{1.202750in}}%
\pgfpathcurveto{\pgfqpoint{0.967515in}{1.202750in}}{\pgfqpoint{0.975415in}{1.206023in}}{\pgfqpoint{0.981239in}{1.211847in}}%
\pgfpathcurveto{\pgfqpoint{0.987063in}{1.217670in}}{\pgfqpoint{0.990336in}{1.225570in}}{\pgfqpoint{0.990336in}{1.233807in}}%
\pgfpathcurveto{\pgfqpoint{0.990336in}{1.242043in}}{\pgfqpoint{0.987063in}{1.249943in}}{\pgfqpoint{0.981239in}{1.255767in}}%
\pgfpathcurveto{\pgfqpoint{0.975415in}{1.261591in}}{\pgfqpoint{0.967515in}{1.264863in}}{\pgfqpoint{0.959279in}{1.264863in}}%
\pgfpathcurveto{\pgfqpoint{0.951043in}{1.264863in}}{\pgfqpoint{0.943143in}{1.261591in}}{\pgfqpoint{0.937319in}{1.255767in}}%
\pgfpathcurveto{\pgfqpoint{0.931495in}{1.249943in}}{\pgfqpoint{0.928223in}{1.242043in}}{\pgfqpoint{0.928223in}{1.233807in}}%
\pgfpathcurveto{\pgfqpoint{0.928223in}{1.225570in}}{\pgfqpoint{0.931495in}{1.217670in}}{\pgfqpoint{0.937319in}{1.211847in}}%
\pgfpathcurveto{\pgfqpoint{0.943143in}{1.206023in}}{\pgfqpoint{0.951043in}{1.202750in}}{\pgfqpoint{0.959279in}{1.202750in}}%
\pgfpathclose%
\pgfusepath{stroke,fill}%
\end{pgfscope}%
\begin{pgfscope}%
\pgfpathrectangle{\pgfqpoint{0.100000in}{0.212622in}}{\pgfqpoint{3.696000in}{3.696000in}}%
\pgfusepath{clip}%
\pgfsetbuttcap%
\pgfsetroundjoin%
\definecolor{currentfill}{rgb}{0.121569,0.466667,0.705882}%
\pgfsetfillcolor{currentfill}%
\pgfsetfillopacity{0.677342}%
\pgfsetlinewidth{1.003750pt}%
\definecolor{currentstroke}{rgb}{0.121569,0.466667,0.705882}%
\pgfsetstrokecolor{currentstroke}%
\pgfsetstrokeopacity{0.677342}%
\pgfsetdash{}{0pt}%
\pgfpathmoveto{\pgfqpoint{0.966501in}{1.202492in}}%
\pgfpathcurveto{\pgfqpoint{0.974737in}{1.202492in}}{\pgfqpoint{0.982638in}{1.205765in}}{\pgfqpoint{0.988461in}{1.211589in}}%
\pgfpathcurveto{\pgfqpoint{0.994285in}{1.217413in}}{\pgfqpoint{0.997558in}{1.225313in}}{\pgfqpoint{0.997558in}{1.233549in}}%
\pgfpathcurveto{\pgfqpoint{0.997558in}{1.241785in}}{\pgfqpoint{0.994285in}{1.249685in}}{\pgfqpoint{0.988461in}{1.255509in}}%
\pgfpathcurveto{\pgfqpoint{0.982638in}{1.261333in}}{\pgfqpoint{0.974737in}{1.264605in}}{\pgfqpoint{0.966501in}{1.264605in}}%
\pgfpathcurveto{\pgfqpoint{0.958265in}{1.264605in}}{\pgfqpoint{0.950365in}{1.261333in}}{\pgfqpoint{0.944541in}{1.255509in}}%
\pgfpathcurveto{\pgfqpoint{0.938717in}{1.249685in}}{\pgfqpoint{0.935445in}{1.241785in}}{\pgfqpoint{0.935445in}{1.233549in}}%
\pgfpathcurveto{\pgfqpoint{0.935445in}{1.225313in}}{\pgfqpoint{0.938717in}{1.217413in}}{\pgfqpoint{0.944541in}{1.211589in}}%
\pgfpathcurveto{\pgfqpoint{0.950365in}{1.205765in}}{\pgfqpoint{0.958265in}{1.202492in}}{\pgfqpoint{0.966501in}{1.202492in}}%
\pgfpathclose%
\pgfusepath{stroke,fill}%
\end{pgfscope}%
\begin{pgfscope}%
\pgfpathrectangle{\pgfqpoint{0.100000in}{0.212622in}}{\pgfqpoint{3.696000in}{3.696000in}}%
\pgfusepath{clip}%
\pgfsetbuttcap%
\pgfsetroundjoin%
\definecolor{currentfill}{rgb}{0.121569,0.466667,0.705882}%
\pgfsetfillcolor{currentfill}%
\pgfsetfillopacity{0.678430}%
\pgfsetlinewidth{1.003750pt}%
\definecolor{currentstroke}{rgb}{0.121569,0.466667,0.705882}%
\pgfsetstrokecolor{currentstroke}%
\pgfsetstrokeopacity{0.678430}%
\pgfsetdash{}{0pt}%
\pgfpathmoveto{\pgfqpoint{2.160048in}{1.844477in}}%
\pgfpathcurveto{\pgfqpoint{2.168284in}{1.844477in}}{\pgfqpoint{2.176184in}{1.847750in}}{\pgfqpoint{2.182008in}{1.853574in}}%
\pgfpathcurveto{\pgfqpoint{2.187832in}{1.859398in}}{\pgfqpoint{2.191105in}{1.867298in}}{\pgfqpoint{2.191105in}{1.875534in}}%
\pgfpathcurveto{\pgfqpoint{2.191105in}{1.883770in}}{\pgfqpoint{2.187832in}{1.891670in}}{\pgfqpoint{2.182008in}{1.897494in}}%
\pgfpathcurveto{\pgfqpoint{2.176184in}{1.903318in}}{\pgfqpoint{2.168284in}{1.906590in}}{\pgfqpoint{2.160048in}{1.906590in}}%
\pgfpathcurveto{\pgfqpoint{2.151812in}{1.906590in}}{\pgfqpoint{2.143912in}{1.903318in}}{\pgfqpoint{2.138088in}{1.897494in}}%
\pgfpathcurveto{\pgfqpoint{2.132264in}{1.891670in}}{\pgfqpoint{2.128992in}{1.883770in}}{\pgfqpoint{2.128992in}{1.875534in}}%
\pgfpathcurveto{\pgfqpoint{2.128992in}{1.867298in}}{\pgfqpoint{2.132264in}{1.859398in}}{\pgfqpoint{2.138088in}{1.853574in}}%
\pgfpathcurveto{\pgfqpoint{2.143912in}{1.847750in}}{\pgfqpoint{2.151812in}{1.844477in}}{\pgfqpoint{2.160048in}{1.844477in}}%
\pgfpathclose%
\pgfusepath{stroke,fill}%
\end{pgfscope}%
\begin{pgfscope}%
\pgfpathrectangle{\pgfqpoint{0.100000in}{0.212622in}}{\pgfqpoint{3.696000in}{3.696000in}}%
\pgfusepath{clip}%
\pgfsetbuttcap%
\pgfsetroundjoin%
\definecolor{currentfill}{rgb}{0.121569,0.466667,0.705882}%
\pgfsetfillcolor{currentfill}%
\pgfsetfillopacity{0.678762}%
\pgfsetlinewidth{1.003750pt}%
\definecolor{currentstroke}{rgb}{0.121569,0.466667,0.705882}%
\pgfsetstrokecolor{currentstroke}%
\pgfsetstrokeopacity{0.678762}%
\pgfsetdash{}{0pt}%
\pgfpathmoveto{\pgfqpoint{0.973617in}{1.202407in}}%
\pgfpathcurveto{\pgfqpoint{0.981853in}{1.202407in}}{\pgfqpoint{0.989753in}{1.205680in}}{\pgfqpoint{0.995577in}{1.211504in}}%
\pgfpathcurveto{\pgfqpoint{1.001401in}{1.217328in}}{\pgfqpoint{1.004673in}{1.225228in}}{\pgfqpoint{1.004673in}{1.233464in}}%
\pgfpathcurveto{\pgfqpoint{1.004673in}{1.241700in}}{\pgfqpoint{1.001401in}{1.249600in}}{\pgfqpoint{0.995577in}{1.255424in}}%
\pgfpathcurveto{\pgfqpoint{0.989753in}{1.261248in}}{\pgfqpoint{0.981853in}{1.264520in}}{\pgfqpoint{0.973617in}{1.264520in}}%
\pgfpathcurveto{\pgfqpoint{0.965380in}{1.264520in}}{\pgfqpoint{0.957480in}{1.261248in}}{\pgfqpoint{0.951656in}{1.255424in}}%
\pgfpathcurveto{\pgfqpoint{0.945833in}{1.249600in}}{\pgfqpoint{0.942560in}{1.241700in}}{\pgfqpoint{0.942560in}{1.233464in}}%
\pgfpathcurveto{\pgfqpoint{0.942560in}{1.225228in}}{\pgfqpoint{0.945833in}{1.217328in}}{\pgfqpoint{0.951656in}{1.211504in}}%
\pgfpathcurveto{\pgfqpoint{0.957480in}{1.205680in}}{\pgfqpoint{0.965380in}{1.202407in}}{\pgfqpoint{0.973617in}{1.202407in}}%
\pgfpathclose%
\pgfusepath{stroke,fill}%
\end{pgfscope}%
\begin{pgfscope}%
\pgfpathrectangle{\pgfqpoint{0.100000in}{0.212622in}}{\pgfqpoint{3.696000in}{3.696000in}}%
\pgfusepath{clip}%
\pgfsetbuttcap%
\pgfsetroundjoin%
\definecolor{currentfill}{rgb}{0.121569,0.466667,0.705882}%
\pgfsetfillcolor{currentfill}%
\pgfsetfillopacity{0.679957}%
\pgfsetlinewidth{1.003750pt}%
\definecolor{currentstroke}{rgb}{0.121569,0.466667,0.705882}%
\pgfsetstrokecolor{currentstroke}%
\pgfsetstrokeopacity{0.679957}%
\pgfsetdash{}{0pt}%
\pgfpathmoveto{\pgfqpoint{0.980056in}{1.202276in}}%
\pgfpathcurveto{\pgfqpoint{0.988292in}{1.202276in}}{\pgfqpoint{0.996192in}{1.205548in}}{\pgfqpoint{1.002016in}{1.211372in}}%
\pgfpathcurveto{\pgfqpoint{1.007840in}{1.217196in}}{\pgfqpoint{1.011112in}{1.225096in}}{\pgfqpoint{1.011112in}{1.233332in}}%
\pgfpathcurveto{\pgfqpoint{1.011112in}{1.241568in}}{\pgfqpoint{1.007840in}{1.249468in}}{\pgfqpoint{1.002016in}{1.255292in}}%
\pgfpathcurveto{\pgfqpoint{0.996192in}{1.261116in}}{\pgfqpoint{0.988292in}{1.264389in}}{\pgfqpoint{0.980056in}{1.264389in}}%
\pgfpathcurveto{\pgfqpoint{0.971820in}{1.264389in}}{\pgfqpoint{0.963920in}{1.261116in}}{\pgfqpoint{0.958096in}{1.255292in}}%
\pgfpathcurveto{\pgfqpoint{0.952272in}{1.249468in}}{\pgfqpoint{0.948999in}{1.241568in}}{\pgfqpoint{0.948999in}{1.233332in}}%
\pgfpathcurveto{\pgfqpoint{0.948999in}{1.225096in}}{\pgfqpoint{0.952272in}{1.217196in}}{\pgfqpoint{0.958096in}{1.211372in}}%
\pgfpathcurveto{\pgfqpoint{0.963920in}{1.205548in}}{\pgfqpoint{0.971820in}{1.202276in}}{\pgfqpoint{0.980056in}{1.202276in}}%
\pgfpathclose%
\pgfusepath{stroke,fill}%
\end{pgfscope}%
\begin{pgfscope}%
\pgfpathrectangle{\pgfqpoint{0.100000in}{0.212622in}}{\pgfqpoint{3.696000in}{3.696000in}}%
\pgfusepath{clip}%
\pgfsetbuttcap%
\pgfsetroundjoin%
\definecolor{currentfill}{rgb}{0.121569,0.466667,0.705882}%
\pgfsetfillcolor{currentfill}%
\pgfsetfillopacity{0.680975}%
\pgfsetlinewidth{1.003750pt}%
\definecolor{currentstroke}{rgb}{0.121569,0.466667,0.705882}%
\pgfsetstrokecolor{currentstroke}%
\pgfsetstrokeopacity{0.680975}%
\pgfsetdash{}{0pt}%
\pgfpathmoveto{\pgfqpoint{0.984913in}{1.202154in}}%
\pgfpathcurveto{\pgfqpoint{0.993149in}{1.202154in}}{\pgfqpoint{1.001049in}{1.205427in}}{\pgfqpoint{1.006873in}{1.211251in}}%
\pgfpathcurveto{\pgfqpoint{1.012697in}{1.217075in}}{\pgfqpoint{1.015969in}{1.224975in}}{\pgfqpoint{1.015969in}{1.233211in}}%
\pgfpathcurveto{\pgfqpoint{1.015969in}{1.241447in}}{\pgfqpoint{1.012697in}{1.249347in}}{\pgfqpoint{1.006873in}{1.255171in}}%
\pgfpathcurveto{\pgfqpoint{1.001049in}{1.260995in}}{\pgfqpoint{0.993149in}{1.264267in}}{\pgfqpoint{0.984913in}{1.264267in}}%
\pgfpathcurveto{\pgfqpoint{0.976677in}{1.264267in}}{\pgfqpoint{0.968777in}{1.260995in}}{\pgfqpoint{0.962953in}{1.255171in}}%
\pgfpathcurveto{\pgfqpoint{0.957129in}{1.249347in}}{\pgfqpoint{0.953856in}{1.241447in}}{\pgfqpoint{0.953856in}{1.233211in}}%
\pgfpathcurveto{\pgfqpoint{0.953856in}{1.224975in}}{\pgfqpoint{0.957129in}{1.217075in}}{\pgfqpoint{0.962953in}{1.211251in}}%
\pgfpathcurveto{\pgfqpoint{0.968777in}{1.205427in}}{\pgfqpoint{0.976677in}{1.202154in}}{\pgfqpoint{0.984913in}{1.202154in}}%
\pgfpathclose%
\pgfusepath{stroke,fill}%
\end{pgfscope}%
\begin{pgfscope}%
\pgfpathrectangle{\pgfqpoint{0.100000in}{0.212622in}}{\pgfqpoint{3.696000in}{3.696000in}}%
\pgfusepath{clip}%
\pgfsetbuttcap%
\pgfsetroundjoin%
\definecolor{currentfill}{rgb}{0.121569,0.466667,0.705882}%
\pgfsetfillcolor{currentfill}%
\pgfsetfillopacity{0.682990}%
\pgfsetlinewidth{1.003750pt}%
\definecolor{currentstroke}{rgb}{0.121569,0.466667,0.705882}%
\pgfsetstrokecolor{currentstroke}%
\pgfsetstrokeopacity{0.682990}%
\pgfsetdash{}{0pt}%
\pgfpathmoveto{\pgfqpoint{0.993716in}{1.202478in}}%
\pgfpathcurveto{\pgfqpoint{1.001952in}{1.202478in}}{\pgfqpoint{1.009852in}{1.205750in}}{\pgfqpoint{1.015676in}{1.211574in}}%
\pgfpathcurveto{\pgfqpoint{1.021500in}{1.217398in}}{\pgfqpoint{1.024773in}{1.225298in}}{\pgfqpoint{1.024773in}{1.233535in}}%
\pgfpathcurveto{\pgfqpoint{1.024773in}{1.241771in}}{\pgfqpoint{1.021500in}{1.249671in}}{\pgfqpoint{1.015676in}{1.255495in}}%
\pgfpathcurveto{\pgfqpoint{1.009852in}{1.261319in}}{\pgfqpoint{1.001952in}{1.264591in}}{\pgfqpoint{0.993716in}{1.264591in}}%
\pgfpathcurveto{\pgfqpoint{0.985480in}{1.264591in}}{\pgfqpoint{0.977580in}{1.261319in}}{\pgfqpoint{0.971756in}{1.255495in}}%
\pgfpathcurveto{\pgfqpoint{0.965932in}{1.249671in}}{\pgfqpoint{0.962660in}{1.241771in}}{\pgfqpoint{0.962660in}{1.233535in}}%
\pgfpathcurveto{\pgfqpoint{0.962660in}{1.225298in}}{\pgfqpoint{0.965932in}{1.217398in}}{\pgfqpoint{0.971756in}{1.211574in}}%
\pgfpathcurveto{\pgfqpoint{0.977580in}{1.205750in}}{\pgfqpoint{0.985480in}{1.202478in}}{\pgfqpoint{0.993716in}{1.202478in}}%
\pgfpathclose%
\pgfusepath{stroke,fill}%
\end{pgfscope}%
\begin{pgfscope}%
\pgfpathrectangle{\pgfqpoint{0.100000in}{0.212622in}}{\pgfqpoint{3.696000in}{3.696000in}}%
\pgfusepath{clip}%
\pgfsetbuttcap%
\pgfsetroundjoin%
\definecolor{currentfill}{rgb}{0.121569,0.466667,0.705882}%
\pgfsetfillcolor{currentfill}%
\pgfsetfillopacity{0.684424}%
\pgfsetlinewidth{1.003750pt}%
\definecolor{currentstroke}{rgb}{0.121569,0.466667,0.705882}%
\pgfsetstrokecolor{currentstroke}%
\pgfsetstrokeopacity{0.684424}%
\pgfsetdash{}{0pt}%
\pgfpathmoveto{\pgfqpoint{1.001094in}{1.202867in}}%
\pgfpathcurveto{\pgfqpoint{1.009330in}{1.202867in}}{\pgfqpoint{1.017230in}{1.206140in}}{\pgfqpoint{1.023054in}{1.211964in}}%
\pgfpathcurveto{\pgfqpoint{1.028878in}{1.217788in}}{\pgfqpoint{1.032151in}{1.225688in}}{\pgfqpoint{1.032151in}{1.233924in}}%
\pgfpathcurveto{\pgfqpoint{1.032151in}{1.242160in}}{\pgfqpoint{1.028878in}{1.250060in}}{\pgfqpoint{1.023054in}{1.255884in}}%
\pgfpathcurveto{\pgfqpoint{1.017230in}{1.261708in}}{\pgfqpoint{1.009330in}{1.264980in}}{\pgfqpoint{1.001094in}{1.264980in}}%
\pgfpathcurveto{\pgfqpoint{0.992858in}{1.264980in}}{\pgfqpoint{0.984958in}{1.261708in}}{\pgfqpoint{0.979134in}{1.255884in}}%
\pgfpathcurveto{\pgfqpoint{0.973310in}{1.250060in}}{\pgfqpoint{0.970038in}{1.242160in}}{\pgfqpoint{0.970038in}{1.233924in}}%
\pgfpathcurveto{\pgfqpoint{0.970038in}{1.225688in}}{\pgfqpoint{0.973310in}{1.217788in}}{\pgfqpoint{0.979134in}{1.211964in}}%
\pgfpathcurveto{\pgfqpoint{0.984958in}{1.206140in}}{\pgfqpoint{0.992858in}{1.202867in}}{\pgfqpoint{1.001094in}{1.202867in}}%
\pgfpathclose%
\pgfusepath{stroke,fill}%
\end{pgfscope}%
\begin{pgfscope}%
\pgfpathrectangle{\pgfqpoint{0.100000in}{0.212622in}}{\pgfqpoint{3.696000in}{3.696000in}}%
\pgfusepath{clip}%
\pgfsetbuttcap%
\pgfsetroundjoin%
\definecolor{currentfill}{rgb}{0.121569,0.466667,0.705882}%
\pgfsetfillcolor{currentfill}%
\pgfsetfillopacity{0.684915}%
\pgfsetlinewidth{1.003750pt}%
\definecolor{currentstroke}{rgb}{0.121569,0.466667,0.705882}%
\pgfsetstrokecolor{currentstroke}%
\pgfsetstrokeopacity{0.684915}%
\pgfsetdash{}{0pt}%
\pgfpathmoveto{\pgfqpoint{2.166055in}{1.826284in}}%
\pgfpathcurveto{\pgfqpoint{2.174291in}{1.826284in}}{\pgfqpoint{2.182191in}{1.829556in}}{\pgfqpoint{2.188015in}{1.835380in}}%
\pgfpathcurveto{\pgfqpoint{2.193839in}{1.841204in}}{\pgfqpoint{2.197111in}{1.849104in}}{\pgfqpoint{2.197111in}{1.857341in}}%
\pgfpathcurveto{\pgfqpoint{2.197111in}{1.865577in}}{\pgfqpoint{2.193839in}{1.873477in}}{\pgfqpoint{2.188015in}{1.879301in}}%
\pgfpathcurveto{\pgfqpoint{2.182191in}{1.885125in}}{\pgfqpoint{2.174291in}{1.888397in}}{\pgfqpoint{2.166055in}{1.888397in}}%
\pgfpathcurveto{\pgfqpoint{2.157819in}{1.888397in}}{\pgfqpoint{2.149919in}{1.885125in}}{\pgfqpoint{2.144095in}{1.879301in}}%
\pgfpathcurveto{\pgfqpoint{2.138271in}{1.873477in}}{\pgfqpoint{2.134998in}{1.865577in}}{\pgfqpoint{2.134998in}{1.857341in}}%
\pgfpathcurveto{\pgfqpoint{2.134998in}{1.849104in}}{\pgfqpoint{2.138271in}{1.841204in}}{\pgfqpoint{2.144095in}{1.835380in}}%
\pgfpathcurveto{\pgfqpoint{2.149919in}{1.829556in}}{\pgfqpoint{2.157819in}{1.826284in}}{\pgfqpoint{2.166055in}{1.826284in}}%
\pgfpathclose%
\pgfusepath{stroke,fill}%
\end{pgfscope}%
\begin{pgfscope}%
\pgfpathrectangle{\pgfqpoint{0.100000in}{0.212622in}}{\pgfqpoint{3.696000in}{3.696000in}}%
\pgfusepath{clip}%
\pgfsetbuttcap%
\pgfsetroundjoin%
\definecolor{currentfill}{rgb}{0.121569,0.466667,0.705882}%
\pgfsetfillcolor{currentfill}%
\pgfsetfillopacity{0.686006}%
\pgfsetlinewidth{1.003750pt}%
\definecolor{currentstroke}{rgb}{0.121569,0.466667,0.705882}%
\pgfsetstrokecolor{currentstroke}%
\pgfsetstrokeopacity{0.686006}%
\pgfsetdash{}{0pt}%
\pgfpathmoveto{\pgfqpoint{1.007963in}{1.203567in}}%
\pgfpathcurveto{\pgfqpoint{1.016200in}{1.203567in}}{\pgfqpoint{1.024100in}{1.206840in}}{\pgfqpoint{1.029924in}{1.212664in}}%
\pgfpathcurveto{\pgfqpoint{1.035748in}{1.218488in}}{\pgfqpoint{1.039020in}{1.226388in}}{\pgfqpoint{1.039020in}{1.234624in}}%
\pgfpathcurveto{\pgfqpoint{1.039020in}{1.242860in}}{\pgfqpoint{1.035748in}{1.250760in}}{\pgfqpoint{1.029924in}{1.256584in}}%
\pgfpathcurveto{\pgfqpoint{1.024100in}{1.262408in}}{\pgfqpoint{1.016200in}{1.265680in}}{\pgfqpoint{1.007963in}{1.265680in}}%
\pgfpathcurveto{\pgfqpoint{0.999727in}{1.265680in}}{\pgfqpoint{0.991827in}{1.262408in}}{\pgfqpoint{0.986003in}{1.256584in}}%
\pgfpathcurveto{\pgfqpoint{0.980179in}{1.250760in}}{\pgfqpoint{0.976907in}{1.242860in}}{\pgfqpoint{0.976907in}{1.234624in}}%
\pgfpathcurveto{\pgfqpoint{0.976907in}{1.226388in}}{\pgfqpoint{0.980179in}{1.218488in}}{\pgfqpoint{0.986003in}{1.212664in}}%
\pgfpathcurveto{\pgfqpoint{0.991827in}{1.206840in}}{\pgfqpoint{0.999727in}{1.203567in}}{\pgfqpoint{1.007963in}{1.203567in}}%
\pgfpathclose%
\pgfusepath{stroke,fill}%
\end{pgfscope}%
\begin{pgfscope}%
\pgfpathrectangle{\pgfqpoint{0.100000in}{0.212622in}}{\pgfqpoint{3.696000in}{3.696000in}}%
\pgfusepath{clip}%
\pgfsetbuttcap%
\pgfsetroundjoin%
\definecolor{currentfill}{rgb}{0.121569,0.466667,0.705882}%
\pgfsetfillcolor{currentfill}%
\pgfsetfillopacity{0.687404}%
\pgfsetlinewidth{1.003750pt}%
\definecolor{currentstroke}{rgb}{0.121569,0.466667,0.705882}%
\pgfsetstrokecolor{currentstroke}%
\pgfsetstrokeopacity{0.687404}%
\pgfsetdash{}{0pt}%
\pgfpathmoveto{\pgfqpoint{1.014000in}{1.203795in}}%
\pgfpathcurveto{\pgfqpoint{1.022236in}{1.203795in}}{\pgfqpoint{1.030136in}{1.207068in}}{\pgfqpoint{1.035960in}{1.212892in}}%
\pgfpathcurveto{\pgfqpoint{1.041784in}{1.218716in}}{\pgfqpoint{1.045057in}{1.226616in}}{\pgfqpoint{1.045057in}{1.234852in}}%
\pgfpathcurveto{\pgfqpoint{1.045057in}{1.243088in}}{\pgfqpoint{1.041784in}{1.250988in}}{\pgfqpoint{1.035960in}{1.256812in}}%
\pgfpathcurveto{\pgfqpoint{1.030136in}{1.262636in}}{\pgfqpoint{1.022236in}{1.265908in}}{\pgfqpoint{1.014000in}{1.265908in}}%
\pgfpathcurveto{\pgfqpoint{1.005764in}{1.265908in}}{\pgfqpoint{0.997864in}{1.262636in}}{\pgfqpoint{0.992040in}{1.256812in}}%
\pgfpathcurveto{\pgfqpoint{0.986216in}{1.250988in}}{\pgfqpoint{0.982944in}{1.243088in}}{\pgfqpoint{0.982944in}{1.234852in}}%
\pgfpathcurveto{\pgfqpoint{0.982944in}{1.226616in}}{\pgfqpoint{0.986216in}{1.218716in}}{\pgfqpoint{0.992040in}{1.212892in}}%
\pgfpathcurveto{\pgfqpoint{0.997864in}{1.207068in}}{\pgfqpoint{1.005764in}{1.203795in}}{\pgfqpoint{1.014000in}{1.203795in}}%
\pgfpathclose%
\pgfusepath{stroke,fill}%
\end{pgfscope}%
\begin{pgfscope}%
\pgfpathrectangle{\pgfqpoint{0.100000in}{0.212622in}}{\pgfqpoint{3.696000in}{3.696000in}}%
\pgfusepath{clip}%
\pgfsetbuttcap%
\pgfsetroundjoin%
\definecolor{currentfill}{rgb}{0.121569,0.466667,0.705882}%
\pgfsetfillcolor{currentfill}%
\pgfsetfillopacity{0.688388}%
\pgfsetlinewidth{1.003750pt}%
\definecolor{currentstroke}{rgb}{0.121569,0.466667,0.705882}%
\pgfsetstrokecolor{currentstroke}%
\pgfsetstrokeopacity{0.688388}%
\pgfsetdash{}{0pt}%
\pgfpathmoveto{\pgfqpoint{2.169239in}{1.815762in}}%
\pgfpathcurveto{\pgfqpoint{2.177475in}{1.815762in}}{\pgfqpoint{2.185375in}{1.819034in}}{\pgfqpoint{2.191199in}{1.824858in}}%
\pgfpathcurveto{\pgfqpoint{2.197023in}{1.830682in}}{\pgfqpoint{2.200296in}{1.838582in}}{\pgfqpoint{2.200296in}{1.846818in}}%
\pgfpathcurveto{\pgfqpoint{2.200296in}{1.855054in}}{\pgfqpoint{2.197023in}{1.862955in}}{\pgfqpoint{2.191199in}{1.868778in}}%
\pgfpathcurveto{\pgfqpoint{2.185375in}{1.874602in}}{\pgfqpoint{2.177475in}{1.877875in}}{\pgfqpoint{2.169239in}{1.877875in}}%
\pgfpathcurveto{\pgfqpoint{2.161003in}{1.877875in}}{\pgfqpoint{2.153103in}{1.874602in}}{\pgfqpoint{2.147279in}{1.868778in}}%
\pgfpathcurveto{\pgfqpoint{2.141455in}{1.862955in}}{\pgfqpoint{2.138183in}{1.855054in}}{\pgfqpoint{2.138183in}{1.846818in}}%
\pgfpathcurveto{\pgfqpoint{2.138183in}{1.838582in}}{\pgfqpoint{2.141455in}{1.830682in}}{\pgfqpoint{2.147279in}{1.824858in}}%
\pgfpathcurveto{\pgfqpoint{2.153103in}{1.819034in}}{\pgfqpoint{2.161003in}{1.815762in}}{\pgfqpoint{2.169239in}{1.815762in}}%
\pgfpathclose%
\pgfusepath{stroke,fill}%
\end{pgfscope}%
\begin{pgfscope}%
\pgfpathrectangle{\pgfqpoint{0.100000in}{0.212622in}}{\pgfqpoint{3.696000in}{3.696000in}}%
\pgfusepath{clip}%
\pgfsetbuttcap%
\pgfsetroundjoin%
\definecolor{currentfill}{rgb}{0.121569,0.466667,0.705882}%
\pgfsetfillcolor{currentfill}%
\pgfsetfillopacity{0.688440}%
\pgfsetlinewidth{1.003750pt}%
\definecolor{currentstroke}{rgb}{0.121569,0.466667,0.705882}%
\pgfsetstrokecolor{currentstroke}%
\pgfsetstrokeopacity{0.688440}%
\pgfsetdash{}{0pt}%
\pgfpathmoveto{\pgfqpoint{1.018762in}{1.204042in}}%
\pgfpathcurveto{\pgfqpoint{1.026998in}{1.204042in}}{\pgfqpoint{1.034898in}{1.207314in}}{\pgfqpoint{1.040722in}{1.213138in}}%
\pgfpathcurveto{\pgfqpoint{1.046546in}{1.218962in}}{\pgfqpoint{1.049819in}{1.226862in}}{\pgfqpoint{1.049819in}{1.235098in}}%
\pgfpathcurveto{\pgfqpoint{1.049819in}{1.243334in}}{\pgfqpoint{1.046546in}{1.251235in}}{\pgfqpoint{1.040722in}{1.257058in}}%
\pgfpathcurveto{\pgfqpoint{1.034898in}{1.262882in}}{\pgfqpoint{1.026998in}{1.266155in}}{\pgfqpoint{1.018762in}{1.266155in}}%
\pgfpathcurveto{\pgfqpoint{1.010526in}{1.266155in}}{\pgfqpoint{1.002626in}{1.262882in}}{\pgfqpoint{0.996802in}{1.257058in}}%
\pgfpathcurveto{\pgfqpoint{0.990978in}{1.251235in}}{\pgfqpoint{0.987706in}{1.243334in}}{\pgfqpoint{0.987706in}{1.235098in}}%
\pgfpathcurveto{\pgfqpoint{0.987706in}{1.226862in}}{\pgfqpoint{0.990978in}{1.218962in}}{\pgfqpoint{0.996802in}{1.213138in}}%
\pgfpathcurveto{\pgfqpoint{1.002626in}{1.207314in}}{\pgfqpoint{1.010526in}{1.204042in}}{\pgfqpoint{1.018762in}{1.204042in}}%
\pgfpathclose%
\pgfusepath{stroke,fill}%
\end{pgfscope}%
\begin{pgfscope}%
\pgfpathrectangle{\pgfqpoint{0.100000in}{0.212622in}}{\pgfqpoint{3.696000in}{3.696000in}}%
\pgfusepath{clip}%
\pgfsetbuttcap%
\pgfsetroundjoin%
\definecolor{currentfill}{rgb}{0.121569,0.466667,0.705882}%
\pgfsetfillcolor{currentfill}%
\pgfsetfillopacity{0.689306}%
\pgfsetlinewidth{1.003750pt}%
\definecolor{currentstroke}{rgb}{0.121569,0.466667,0.705882}%
\pgfsetstrokecolor{currentstroke}%
\pgfsetstrokeopacity{0.689306}%
\pgfsetdash{}{0pt}%
\pgfpathmoveto{\pgfqpoint{1.023125in}{1.204047in}}%
\pgfpathcurveto{\pgfqpoint{1.031361in}{1.204047in}}{\pgfqpoint{1.039261in}{1.207319in}}{\pgfqpoint{1.045085in}{1.213143in}}%
\pgfpathcurveto{\pgfqpoint{1.050909in}{1.218967in}}{\pgfqpoint{1.054181in}{1.226867in}}{\pgfqpoint{1.054181in}{1.235103in}}%
\pgfpathcurveto{\pgfqpoint{1.054181in}{1.243340in}}{\pgfqpoint{1.050909in}{1.251240in}}{\pgfqpoint{1.045085in}{1.257064in}}%
\pgfpathcurveto{\pgfqpoint{1.039261in}{1.262888in}}{\pgfqpoint{1.031361in}{1.266160in}}{\pgfqpoint{1.023125in}{1.266160in}}%
\pgfpathcurveto{\pgfqpoint{1.014888in}{1.266160in}}{\pgfqpoint{1.006988in}{1.262888in}}{\pgfqpoint{1.001164in}{1.257064in}}%
\pgfpathcurveto{\pgfqpoint{0.995341in}{1.251240in}}{\pgfqpoint{0.992068in}{1.243340in}}{\pgfqpoint{0.992068in}{1.235103in}}%
\pgfpathcurveto{\pgfqpoint{0.992068in}{1.226867in}}{\pgfqpoint{0.995341in}{1.218967in}}{\pgfqpoint{1.001164in}{1.213143in}}%
\pgfpathcurveto{\pgfqpoint{1.006988in}{1.207319in}}{\pgfqpoint{1.014888in}{1.204047in}}{\pgfqpoint{1.023125in}{1.204047in}}%
\pgfpathclose%
\pgfusepath{stroke,fill}%
\end{pgfscope}%
\begin{pgfscope}%
\pgfpathrectangle{\pgfqpoint{0.100000in}{0.212622in}}{\pgfqpoint{3.696000in}{3.696000in}}%
\pgfusepath{clip}%
\pgfsetbuttcap%
\pgfsetroundjoin%
\definecolor{currentfill}{rgb}{0.121569,0.466667,0.705882}%
\pgfsetfillcolor{currentfill}%
\pgfsetfillopacity{0.689898}%
\pgfsetlinewidth{1.003750pt}%
\definecolor{currentstroke}{rgb}{0.121569,0.466667,0.705882}%
\pgfsetstrokecolor{currentstroke}%
\pgfsetstrokeopacity{0.689898}%
\pgfsetdash{}{0pt}%
\pgfpathmoveto{\pgfqpoint{1.025845in}{1.203971in}}%
\pgfpathcurveto{\pgfqpoint{1.034081in}{1.203971in}}{\pgfqpoint{1.041981in}{1.207243in}}{\pgfqpoint{1.047805in}{1.213067in}}%
\pgfpathcurveto{\pgfqpoint{1.053629in}{1.218891in}}{\pgfqpoint{1.056901in}{1.226791in}}{\pgfqpoint{1.056901in}{1.235027in}}%
\pgfpathcurveto{\pgfqpoint{1.056901in}{1.243264in}}{\pgfqpoint{1.053629in}{1.251164in}}{\pgfqpoint{1.047805in}{1.256988in}}%
\pgfpathcurveto{\pgfqpoint{1.041981in}{1.262812in}}{\pgfqpoint{1.034081in}{1.266084in}}{\pgfqpoint{1.025845in}{1.266084in}}%
\pgfpathcurveto{\pgfqpoint{1.017608in}{1.266084in}}{\pgfqpoint{1.009708in}{1.262812in}}{\pgfqpoint{1.003884in}{1.256988in}}%
\pgfpathcurveto{\pgfqpoint{0.998061in}{1.251164in}}{\pgfqpoint{0.994788in}{1.243264in}}{\pgfqpoint{0.994788in}{1.235027in}}%
\pgfpathcurveto{\pgfqpoint{0.994788in}{1.226791in}}{\pgfqpoint{0.998061in}{1.218891in}}{\pgfqpoint{1.003884in}{1.213067in}}%
\pgfpathcurveto{\pgfqpoint{1.009708in}{1.207243in}}{\pgfqpoint{1.017608in}{1.203971in}}{\pgfqpoint{1.025845in}{1.203971in}}%
\pgfpathclose%
\pgfusepath{stroke,fill}%
\end{pgfscope}%
\begin{pgfscope}%
\pgfpathrectangle{\pgfqpoint{0.100000in}{0.212622in}}{\pgfqpoint{3.696000in}{3.696000in}}%
\pgfusepath{clip}%
\pgfsetbuttcap%
\pgfsetroundjoin%
\definecolor{currentfill}{rgb}{0.121569,0.466667,0.705882}%
\pgfsetfillcolor{currentfill}%
\pgfsetfillopacity{0.691035}%
\pgfsetlinewidth{1.003750pt}%
\definecolor{currentstroke}{rgb}{0.121569,0.466667,0.705882}%
\pgfsetstrokecolor{currentstroke}%
\pgfsetstrokeopacity{0.691035}%
\pgfsetdash{}{0pt}%
\pgfpathmoveto{\pgfqpoint{1.030744in}{1.203886in}}%
\pgfpathcurveto{\pgfqpoint{1.038981in}{1.203886in}}{\pgfqpoint{1.046881in}{1.207159in}}{\pgfqpoint{1.052705in}{1.212983in}}%
\pgfpathcurveto{\pgfqpoint{1.058528in}{1.218807in}}{\pgfqpoint{1.061801in}{1.226707in}}{\pgfqpoint{1.061801in}{1.234943in}}%
\pgfpathcurveto{\pgfqpoint{1.061801in}{1.243179in}}{\pgfqpoint{1.058528in}{1.251079in}}{\pgfqpoint{1.052705in}{1.256903in}}%
\pgfpathcurveto{\pgfqpoint{1.046881in}{1.262727in}}{\pgfqpoint{1.038981in}{1.265999in}}{\pgfqpoint{1.030744in}{1.265999in}}%
\pgfpathcurveto{\pgfqpoint{1.022508in}{1.265999in}}{\pgfqpoint{1.014608in}{1.262727in}}{\pgfqpoint{1.008784in}{1.256903in}}%
\pgfpathcurveto{\pgfqpoint{1.002960in}{1.251079in}}{\pgfqpoint{0.999688in}{1.243179in}}{\pgfqpoint{0.999688in}{1.234943in}}%
\pgfpathcurveto{\pgfqpoint{0.999688in}{1.226707in}}{\pgfqpoint{1.002960in}{1.218807in}}{\pgfqpoint{1.008784in}{1.212983in}}%
\pgfpathcurveto{\pgfqpoint{1.014608in}{1.207159in}}{\pgfqpoint{1.022508in}{1.203886in}}{\pgfqpoint{1.030744in}{1.203886in}}%
\pgfpathclose%
\pgfusepath{stroke,fill}%
\end{pgfscope}%
\begin{pgfscope}%
\pgfpathrectangle{\pgfqpoint{0.100000in}{0.212622in}}{\pgfqpoint{3.696000in}{3.696000in}}%
\pgfusepath{clip}%
\pgfsetbuttcap%
\pgfsetroundjoin%
\definecolor{currentfill}{rgb}{0.121569,0.466667,0.705882}%
\pgfsetfillcolor{currentfill}%
\pgfsetfillopacity{0.691720}%
\pgfsetlinewidth{1.003750pt}%
\definecolor{currentstroke}{rgb}{0.121569,0.466667,0.705882}%
\pgfsetstrokecolor{currentstroke}%
\pgfsetstrokeopacity{0.691720}%
\pgfsetdash{}{0pt}%
\pgfpathmoveto{\pgfqpoint{1.033812in}{1.203978in}}%
\pgfpathcurveto{\pgfqpoint{1.042049in}{1.203978in}}{\pgfqpoint{1.049949in}{1.207251in}}{\pgfqpoint{1.055773in}{1.213074in}}%
\pgfpathcurveto{\pgfqpoint{1.061597in}{1.218898in}}{\pgfqpoint{1.064869in}{1.226798in}}{\pgfqpoint{1.064869in}{1.235035in}}%
\pgfpathcurveto{\pgfqpoint{1.064869in}{1.243271in}}{\pgfqpoint{1.061597in}{1.251171in}}{\pgfqpoint{1.055773in}{1.256995in}}%
\pgfpathcurveto{\pgfqpoint{1.049949in}{1.262819in}}{\pgfqpoint{1.042049in}{1.266091in}}{\pgfqpoint{1.033812in}{1.266091in}}%
\pgfpathcurveto{\pgfqpoint{1.025576in}{1.266091in}}{\pgfqpoint{1.017676in}{1.262819in}}{\pgfqpoint{1.011852in}{1.256995in}}%
\pgfpathcurveto{\pgfqpoint{1.006028in}{1.251171in}}{\pgfqpoint{1.002756in}{1.243271in}}{\pgfqpoint{1.002756in}{1.235035in}}%
\pgfpathcurveto{\pgfqpoint{1.002756in}{1.226798in}}{\pgfqpoint{1.006028in}{1.218898in}}{\pgfqpoint{1.011852in}{1.213074in}}%
\pgfpathcurveto{\pgfqpoint{1.017676in}{1.207251in}}{\pgfqpoint{1.025576in}{1.203978in}}{\pgfqpoint{1.033812in}{1.203978in}}%
\pgfpathclose%
\pgfusepath{stroke,fill}%
\end{pgfscope}%
\begin{pgfscope}%
\pgfpathrectangle{\pgfqpoint{0.100000in}{0.212622in}}{\pgfqpoint{3.696000in}{3.696000in}}%
\pgfusepath{clip}%
\pgfsetbuttcap%
\pgfsetroundjoin%
\definecolor{currentfill}{rgb}{0.121569,0.466667,0.705882}%
\pgfsetfillcolor{currentfill}%
\pgfsetfillopacity{0.692632}%
\pgfsetlinewidth{1.003750pt}%
\definecolor{currentstroke}{rgb}{0.121569,0.466667,0.705882}%
\pgfsetstrokecolor{currentstroke}%
\pgfsetstrokeopacity{0.692632}%
\pgfsetdash{}{0pt}%
\pgfpathmoveto{\pgfqpoint{2.171665in}{1.804568in}}%
\pgfpathcurveto{\pgfqpoint{2.179901in}{1.804568in}}{\pgfqpoint{2.187801in}{1.807841in}}{\pgfqpoint{2.193625in}{1.813665in}}%
\pgfpathcurveto{\pgfqpoint{2.199449in}{1.819489in}}{\pgfqpoint{2.202721in}{1.827389in}}{\pgfqpoint{2.202721in}{1.835625in}}%
\pgfpathcurveto{\pgfqpoint{2.202721in}{1.843861in}}{\pgfqpoint{2.199449in}{1.851761in}}{\pgfqpoint{2.193625in}{1.857585in}}%
\pgfpathcurveto{\pgfqpoint{2.187801in}{1.863409in}}{\pgfqpoint{2.179901in}{1.866681in}}{\pgfqpoint{2.171665in}{1.866681in}}%
\pgfpathcurveto{\pgfqpoint{2.163428in}{1.866681in}}{\pgfqpoint{2.155528in}{1.863409in}}{\pgfqpoint{2.149704in}{1.857585in}}%
\pgfpathcurveto{\pgfqpoint{2.143880in}{1.851761in}}{\pgfqpoint{2.140608in}{1.843861in}}{\pgfqpoint{2.140608in}{1.835625in}}%
\pgfpathcurveto{\pgfqpoint{2.140608in}{1.827389in}}{\pgfqpoint{2.143880in}{1.819489in}}{\pgfqpoint{2.149704in}{1.813665in}}%
\pgfpathcurveto{\pgfqpoint{2.155528in}{1.807841in}}{\pgfqpoint{2.163428in}{1.804568in}}{\pgfqpoint{2.171665in}{1.804568in}}%
\pgfpathclose%
\pgfusepath{stroke,fill}%
\end{pgfscope}%
\begin{pgfscope}%
\pgfpathrectangle{\pgfqpoint{0.100000in}{0.212622in}}{\pgfqpoint{3.696000in}{3.696000in}}%
\pgfusepath{clip}%
\pgfsetbuttcap%
\pgfsetroundjoin%
\definecolor{currentfill}{rgb}{0.121569,0.466667,0.705882}%
\pgfsetfillcolor{currentfill}%
\pgfsetfillopacity{0.693146}%
\pgfsetlinewidth{1.003750pt}%
\definecolor{currentstroke}{rgb}{0.121569,0.466667,0.705882}%
\pgfsetstrokecolor{currentstroke}%
\pgfsetstrokeopacity{0.693146}%
\pgfsetdash{}{0pt}%
\pgfpathmoveto{\pgfqpoint{1.039299in}{1.204495in}}%
\pgfpathcurveto{\pgfqpoint{1.047536in}{1.204495in}}{\pgfqpoint{1.055436in}{1.207767in}}{\pgfqpoint{1.061260in}{1.213591in}}%
\pgfpathcurveto{\pgfqpoint{1.067083in}{1.219415in}}{\pgfqpoint{1.070356in}{1.227315in}}{\pgfqpoint{1.070356in}{1.235551in}}%
\pgfpathcurveto{\pgfqpoint{1.070356in}{1.243787in}}{\pgfqpoint{1.067083in}{1.251687in}}{\pgfqpoint{1.061260in}{1.257511in}}%
\pgfpathcurveto{\pgfqpoint{1.055436in}{1.263335in}}{\pgfqpoint{1.047536in}{1.266608in}}{\pgfqpoint{1.039299in}{1.266608in}}%
\pgfpathcurveto{\pgfqpoint{1.031063in}{1.266608in}}{\pgfqpoint{1.023163in}{1.263335in}}{\pgfqpoint{1.017339in}{1.257511in}}%
\pgfpathcurveto{\pgfqpoint{1.011515in}{1.251687in}}{\pgfqpoint{1.008243in}{1.243787in}}{\pgfqpoint{1.008243in}{1.235551in}}%
\pgfpathcurveto{\pgfqpoint{1.008243in}{1.227315in}}{\pgfqpoint{1.011515in}{1.219415in}}{\pgfqpoint{1.017339in}{1.213591in}}%
\pgfpathcurveto{\pgfqpoint{1.023163in}{1.207767in}}{\pgfqpoint{1.031063in}{1.204495in}}{\pgfqpoint{1.039299in}{1.204495in}}%
\pgfpathclose%
\pgfusepath{stroke,fill}%
\end{pgfscope}%
\begin{pgfscope}%
\pgfpathrectangle{\pgfqpoint{0.100000in}{0.212622in}}{\pgfqpoint{3.696000in}{3.696000in}}%
\pgfusepath{clip}%
\pgfsetbuttcap%
\pgfsetroundjoin%
\definecolor{currentfill}{rgb}{0.121569,0.466667,0.705882}%
\pgfsetfillcolor{currentfill}%
\pgfsetfillopacity{0.694289}%
\pgfsetlinewidth{1.003750pt}%
\definecolor{currentstroke}{rgb}{0.121569,0.466667,0.705882}%
\pgfsetstrokecolor{currentstroke}%
\pgfsetstrokeopacity{0.694289}%
\pgfsetdash{}{0pt}%
\pgfpathmoveto{\pgfqpoint{1.044050in}{1.204514in}}%
\pgfpathcurveto{\pgfqpoint{1.052286in}{1.204514in}}{\pgfqpoint{1.060186in}{1.207787in}}{\pgfqpoint{1.066010in}{1.213611in}}%
\pgfpathcurveto{\pgfqpoint{1.071834in}{1.219435in}}{\pgfqpoint{1.075106in}{1.227335in}}{\pgfqpoint{1.075106in}{1.235571in}}%
\pgfpathcurveto{\pgfqpoint{1.075106in}{1.243807in}}{\pgfqpoint{1.071834in}{1.251707in}}{\pgfqpoint{1.066010in}{1.257531in}}%
\pgfpathcurveto{\pgfqpoint{1.060186in}{1.263355in}}{\pgfqpoint{1.052286in}{1.266627in}}{\pgfqpoint{1.044050in}{1.266627in}}%
\pgfpathcurveto{\pgfqpoint{1.035813in}{1.266627in}}{\pgfqpoint{1.027913in}{1.263355in}}{\pgfqpoint{1.022089in}{1.257531in}}%
\pgfpathcurveto{\pgfqpoint{1.016265in}{1.251707in}}{\pgfqpoint{1.012993in}{1.243807in}}{\pgfqpoint{1.012993in}{1.235571in}}%
\pgfpathcurveto{\pgfqpoint{1.012993in}{1.227335in}}{\pgfqpoint{1.016265in}{1.219435in}}{\pgfqpoint{1.022089in}{1.213611in}}%
\pgfpathcurveto{\pgfqpoint{1.027913in}{1.207787in}}{\pgfqpoint{1.035813in}{1.204514in}}{\pgfqpoint{1.044050in}{1.204514in}}%
\pgfpathclose%
\pgfusepath{stroke,fill}%
\end{pgfscope}%
\begin{pgfscope}%
\pgfpathrectangle{\pgfqpoint{0.100000in}{0.212622in}}{\pgfqpoint{3.696000in}{3.696000in}}%
\pgfusepath{clip}%
\pgfsetbuttcap%
\pgfsetroundjoin%
\definecolor{currentfill}{rgb}{0.121569,0.466667,0.705882}%
\pgfsetfillcolor{currentfill}%
\pgfsetfillopacity{0.695081}%
\pgfsetlinewidth{1.003750pt}%
\definecolor{currentstroke}{rgb}{0.121569,0.466667,0.705882}%
\pgfsetstrokecolor{currentstroke}%
\pgfsetstrokeopacity{0.695081}%
\pgfsetdash{}{0pt}%
\pgfpathmoveto{\pgfqpoint{1.047909in}{1.204498in}}%
\pgfpathcurveto{\pgfqpoint{1.056145in}{1.204498in}}{\pgfqpoint{1.064045in}{1.207771in}}{\pgfqpoint{1.069869in}{1.213595in}}%
\pgfpathcurveto{\pgfqpoint{1.075693in}{1.219419in}}{\pgfqpoint{1.078965in}{1.227319in}}{\pgfqpoint{1.078965in}{1.235555in}}%
\pgfpathcurveto{\pgfqpoint{1.078965in}{1.243791in}}{\pgfqpoint{1.075693in}{1.251691in}}{\pgfqpoint{1.069869in}{1.257515in}}%
\pgfpathcurveto{\pgfqpoint{1.064045in}{1.263339in}}{\pgfqpoint{1.056145in}{1.266611in}}{\pgfqpoint{1.047909in}{1.266611in}}%
\pgfpathcurveto{\pgfqpoint{1.039673in}{1.266611in}}{\pgfqpoint{1.031773in}{1.263339in}}{\pgfqpoint{1.025949in}{1.257515in}}%
\pgfpathcurveto{\pgfqpoint{1.020125in}{1.251691in}}{\pgfqpoint{1.016852in}{1.243791in}}{\pgfqpoint{1.016852in}{1.235555in}}%
\pgfpathcurveto{\pgfqpoint{1.016852in}{1.227319in}}{\pgfqpoint{1.020125in}{1.219419in}}{\pgfqpoint{1.025949in}{1.213595in}}%
\pgfpathcurveto{\pgfqpoint{1.031773in}{1.207771in}}{\pgfqpoint{1.039673in}{1.204498in}}{\pgfqpoint{1.047909in}{1.204498in}}%
\pgfpathclose%
\pgfusepath{stroke,fill}%
\end{pgfscope}%
\begin{pgfscope}%
\pgfpathrectangle{\pgfqpoint{0.100000in}{0.212622in}}{\pgfqpoint{3.696000in}{3.696000in}}%
\pgfusepath{clip}%
\pgfsetbuttcap%
\pgfsetroundjoin%
\definecolor{currentfill}{rgb}{0.121569,0.466667,0.705882}%
\pgfsetfillcolor{currentfill}%
\pgfsetfillopacity{0.696694}%
\pgfsetlinewidth{1.003750pt}%
\definecolor{currentstroke}{rgb}{0.121569,0.466667,0.705882}%
\pgfsetstrokecolor{currentstroke}%
\pgfsetstrokeopacity{0.696694}%
\pgfsetdash{}{0pt}%
\pgfpathmoveto{\pgfqpoint{1.054902in}{1.205069in}}%
\pgfpathcurveto{\pgfqpoint{1.063139in}{1.205069in}}{\pgfqpoint{1.071039in}{1.208341in}}{\pgfqpoint{1.076863in}{1.214165in}}%
\pgfpathcurveto{\pgfqpoint{1.082687in}{1.219989in}}{\pgfqpoint{1.085959in}{1.227889in}}{\pgfqpoint{1.085959in}{1.236125in}}%
\pgfpathcurveto{\pgfqpoint{1.085959in}{1.244362in}}{\pgfqpoint{1.082687in}{1.252262in}}{\pgfqpoint{1.076863in}{1.258086in}}%
\pgfpathcurveto{\pgfqpoint{1.071039in}{1.263909in}}{\pgfqpoint{1.063139in}{1.267182in}}{\pgfqpoint{1.054902in}{1.267182in}}%
\pgfpathcurveto{\pgfqpoint{1.046666in}{1.267182in}}{\pgfqpoint{1.038766in}{1.263909in}}{\pgfqpoint{1.032942in}{1.258086in}}%
\pgfpathcurveto{\pgfqpoint{1.027118in}{1.252262in}}{\pgfqpoint{1.023846in}{1.244362in}}{\pgfqpoint{1.023846in}{1.236125in}}%
\pgfpathcurveto{\pgfqpoint{1.023846in}{1.227889in}}{\pgfqpoint{1.027118in}{1.219989in}}{\pgfqpoint{1.032942in}{1.214165in}}%
\pgfpathcurveto{\pgfqpoint{1.038766in}{1.208341in}}{\pgfqpoint{1.046666in}{1.205069in}}{\pgfqpoint{1.054902in}{1.205069in}}%
\pgfpathclose%
\pgfusepath{stroke,fill}%
\end{pgfscope}%
\begin{pgfscope}%
\pgfpathrectangle{\pgfqpoint{0.100000in}{0.212622in}}{\pgfqpoint{3.696000in}{3.696000in}}%
\pgfusepath{clip}%
\pgfsetbuttcap%
\pgfsetroundjoin%
\definecolor{currentfill}{rgb}{0.121569,0.466667,0.705882}%
\pgfsetfillcolor{currentfill}%
\pgfsetfillopacity{0.696812}%
\pgfsetlinewidth{1.003750pt}%
\definecolor{currentstroke}{rgb}{0.121569,0.466667,0.705882}%
\pgfsetstrokecolor{currentstroke}%
\pgfsetstrokeopacity{0.696812}%
\pgfsetdash{}{0pt}%
\pgfpathmoveto{\pgfqpoint{2.175034in}{1.792341in}}%
\pgfpathcurveto{\pgfqpoint{2.183270in}{1.792341in}}{\pgfqpoint{2.191170in}{1.795614in}}{\pgfqpoint{2.196994in}{1.801438in}}%
\pgfpathcurveto{\pgfqpoint{2.202818in}{1.807261in}}{\pgfqpoint{2.206090in}{1.815161in}}{\pgfqpoint{2.206090in}{1.823398in}}%
\pgfpathcurveto{\pgfqpoint{2.206090in}{1.831634in}}{\pgfqpoint{2.202818in}{1.839534in}}{\pgfqpoint{2.196994in}{1.845358in}}%
\pgfpathcurveto{\pgfqpoint{2.191170in}{1.851182in}}{\pgfqpoint{2.183270in}{1.854454in}}{\pgfqpoint{2.175034in}{1.854454in}}%
\pgfpathcurveto{\pgfqpoint{2.166797in}{1.854454in}}{\pgfqpoint{2.158897in}{1.851182in}}{\pgfqpoint{2.153073in}{1.845358in}}%
\pgfpathcurveto{\pgfqpoint{2.147249in}{1.839534in}}{\pgfqpoint{2.143977in}{1.831634in}}{\pgfqpoint{2.143977in}{1.823398in}}%
\pgfpathcurveto{\pgfqpoint{2.143977in}{1.815161in}}{\pgfqpoint{2.147249in}{1.807261in}}{\pgfqpoint{2.153073in}{1.801438in}}%
\pgfpathcurveto{\pgfqpoint{2.158897in}{1.795614in}}{\pgfqpoint{2.166797in}{1.792341in}}{\pgfqpoint{2.175034in}{1.792341in}}%
\pgfpathclose%
\pgfusepath{stroke,fill}%
\end{pgfscope}%
\begin{pgfscope}%
\pgfpathrectangle{\pgfqpoint{0.100000in}{0.212622in}}{\pgfqpoint{3.696000in}{3.696000in}}%
\pgfusepath{clip}%
\pgfsetbuttcap%
\pgfsetroundjoin%
\definecolor{currentfill}{rgb}{0.121569,0.466667,0.705882}%
\pgfsetfillcolor{currentfill}%
\pgfsetfillopacity{0.697968}%
\pgfsetlinewidth{1.003750pt}%
\definecolor{currentstroke}{rgb}{0.121569,0.466667,0.705882}%
\pgfsetstrokecolor{currentstroke}%
\pgfsetstrokeopacity{0.697968}%
\pgfsetdash{}{0pt}%
\pgfpathmoveto{\pgfqpoint{1.060464in}{1.205058in}}%
\pgfpathcurveto{\pgfqpoint{1.068700in}{1.205058in}}{\pgfqpoint{1.076600in}{1.208330in}}{\pgfqpoint{1.082424in}{1.214154in}}%
\pgfpathcurveto{\pgfqpoint{1.088248in}{1.219978in}}{\pgfqpoint{1.091520in}{1.227878in}}{\pgfqpoint{1.091520in}{1.236114in}}%
\pgfpathcurveto{\pgfqpoint{1.091520in}{1.244351in}}{\pgfqpoint{1.088248in}{1.252251in}}{\pgfqpoint{1.082424in}{1.258075in}}%
\pgfpathcurveto{\pgfqpoint{1.076600in}{1.263899in}}{\pgfqpoint{1.068700in}{1.267171in}}{\pgfqpoint{1.060464in}{1.267171in}}%
\pgfpathcurveto{\pgfqpoint{1.052228in}{1.267171in}}{\pgfqpoint{1.044328in}{1.263899in}}{\pgfqpoint{1.038504in}{1.258075in}}%
\pgfpathcurveto{\pgfqpoint{1.032680in}{1.252251in}}{\pgfqpoint{1.029407in}{1.244351in}}{\pgfqpoint{1.029407in}{1.236114in}}%
\pgfpathcurveto{\pgfqpoint{1.029407in}{1.227878in}}{\pgfqpoint{1.032680in}{1.219978in}}{\pgfqpoint{1.038504in}{1.214154in}}%
\pgfpathcurveto{\pgfqpoint{1.044328in}{1.208330in}}{\pgfqpoint{1.052228in}{1.205058in}}{\pgfqpoint{1.060464in}{1.205058in}}%
\pgfpathclose%
\pgfusepath{stroke,fill}%
\end{pgfscope}%
\begin{pgfscope}%
\pgfpathrectangle{\pgfqpoint{0.100000in}{0.212622in}}{\pgfqpoint{3.696000in}{3.696000in}}%
\pgfusepath{clip}%
\pgfsetbuttcap%
\pgfsetroundjoin%
\definecolor{currentfill}{rgb}{0.121569,0.466667,0.705882}%
\pgfsetfillcolor{currentfill}%
\pgfsetfillopacity{0.699031}%
\pgfsetlinewidth{1.003750pt}%
\definecolor{currentstroke}{rgb}{0.121569,0.466667,0.705882}%
\pgfsetstrokecolor{currentstroke}%
\pgfsetstrokeopacity{0.699031}%
\pgfsetdash{}{0pt}%
\pgfpathmoveto{\pgfqpoint{1.065510in}{1.204993in}}%
\pgfpathcurveto{\pgfqpoint{1.073746in}{1.204993in}}{\pgfqpoint{1.081646in}{1.208265in}}{\pgfqpoint{1.087470in}{1.214089in}}%
\pgfpathcurveto{\pgfqpoint{1.093294in}{1.219913in}}{\pgfqpoint{1.096566in}{1.227813in}}{\pgfqpoint{1.096566in}{1.236049in}}%
\pgfpathcurveto{\pgfqpoint{1.096566in}{1.244286in}}{\pgfqpoint{1.093294in}{1.252186in}}{\pgfqpoint{1.087470in}{1.258010in}}%
\pgfpathcurveto{\pgfqpoint{1.081646in}{1.263834in}}{\pgfqpoint{1.073746in}{1.267106in}}{\pgfqpoint{1.065510in}{1.267106in}}%
\pgfpathcurveto{\pgfqpoint{1.057274in}{1.267106in}}{\pgfqpoint{1.049374in}{1.263834in}}{\pgfqpoint{1.043550in}{1.258010in}}%
\pgfpathcurveto{\pgfqpoint{1.037726in}{1.252186in}}{\pgfqpoint{1.034453in}{1.244286in}}{\pgfqpoint{1.034453in}{1.236049in}}%
\pgfpathcurveto{\pgfqpoint{1.034453in}{1.227813in}}{\pgfqpoint{1.037726in}{1.219913in}}{\pgfqpoint{1.043550in}{1.214089in}}%
\pgfpathcurveto{\pgfqpoint{1.049374in}{1.208265in}}{\pgfqpoint{1.057274in}{1.204993in}}{\pgfqpoint{1.065510in}{1.204993in}}%
\pgfpathclose%
\pgfusepath{stroke,fill}%
\end{pgfscope}%
\begin{pgfscope}%
\pgfpathrectangle{\pgfqpoint{0.100000in}{0.212622in}}{\pgfqpoint{3.696000in}{3.696000in}}%
\pgfusepath{clip}%
\pgfsetbuttcap%
\pgfsetroundjoin%
\definecolor{currentfill}{rgb}{0.121569,0.466667,0.705882}%
\pgfsetfillcolor{currentfill}%
\pgfsetfillopacity{0.699825}%
\pgfsetlinewidth{1.003750pt}%
\definecolor{currentstroke}{rgb}{0.121569,0.466667,0.705882}%
\pgfsetstrokecolor{currentstroke}%
\pgfsetstrokeopacity{0.699825}%
\pgfsetdash{}{0pt}%
\pgfpathmoveto{\pgfqpoint{1.069551in}{1.204762in}}%
\pgfpathcurveto{\pgfqpoint{1.077788in}{1.204762in}}{\pgfqpoint{1.085688in}{1.208034in}}{\pgfqpoint{1.091512in}{1.213858in}}%
\pgfpathcurveto{\pgfqpoint{1.097336in}{1.219682in}}{\pgfqpoint{1.100608in}{1.227582in}}{\pgfqpoint{1.100608in}{1.235818in}}%
\pgfpathcurveto{\pgfqpoint{1.100608in}{1.244054in}}{\pgfqpoint{1.097336in}{1.251955in}}{\pgfqpoint{1.091512in}{1.257778in}}%
\pgfpathcurveto{\pgfqpoint{1.085688in}{1.263602in}}{\pgfqpoint{1.077788in}{1.266875in}}{\pgfqpoint{1.069551in}{1.266875in}}%
\pgfpathcurveto{\pgfqpoint{1.061315in}{1.266875in}}{\pgfqpoint{1.053415in}{1.263602in}}{\pgfqpoint{1.047591in}{1.257778in}}%
\pgfpathcurveto{\pgfqpoint{1.041767in}{1.251955in}}{\pgfqpoint{1.038495in}{1.244054in}}{\pgfqpoint{1.038495in}{1.235818in}}%
\pgfpathcurveto{\pgfqpoint{1.038495in}{1.227582in}}{\pgfqpoint{1.041767in}{1.219682in}}{\pgfqpoint{1.047591in}{1.213858in}}%
\pgfpathcurveto{\pgfqpoint{1.053415in}{1.208034in}}{\pgfqpoint{1.061315in}{1.204762in}}{\pgfqpoint{1.069551in}{1.204762in}}%
\pgfpathclose%
\pgfusepath{stroke,fill}%
\end{pgfscope}%
\begin{pgfscope}%
\pgfpathrectangle{\pgfqpoint{0.100000in}{0.212622in}}{\pgfqpoint{3.696000in}{3.696000in}}%
\pgfusepath{clip}%
\pgfsetbuttcap%
\pgfsetroundjoin%
\definecolor{currentfill}{rgb}{0.121569,0.466667,0.705882}%
\pgfsetfillcolor{currentfill}%
\pgfsetfillopacity{0.700495}%
\pgfsetlinewidth{1.003750pt}%
\definecolor{currentstroke}{rgb}{0.121569,0.466667,0.705882}%
\pgfsetstrokecolor{currentstroke}%
\pgfsetstrokeopacity{0.700495}%
\pgfsetdash{}{0pt}%
\pgfpathmoveto{\pgfqpoint{1.072593in}{1.204503in}}%
\pgfpathcurveto{\pgfqpoint{1.080829in}{1.204503in}}{\pgfqpoint{1.088729in}{1.207775in}}{\pgfqpoint{1.094553in}{1.213599in}}%
\pgfpathcurveto{\pgfqpoint{1.100377in}{1.219423in}}{\pgfqpoint{1.103649in}{1.227323in}}{\pgfqpoint{1.103649in}{1.235559in}}%
\pgfpathcurveto{\pgfqpoint{1.103649in}{1.243796in}}{\pgfqpoint{1.100377in}{1.251696in}}{\pgfqpoint{1.094553in}{1.257520in}}%
\pgfpathcurveto{\pgfqpoint{1.088729in}{1.263344in}}{\pgfqpoint{1.080829in}{1.266616in}}{\pgfqpoint{1.072593in}{1.266616in}}%
\pgfpathcurveto{\pgfqpoint{1.064356in}{1.266616in}}{\pgfqpoint{1.056456in}{1.263344in}}{\pgfqpoint{1.050632in}{1.257520in}}%
\pgfpathcurveto{\pgfqpoint{1.044808in}{1.251696in}}{\pgfqpoint{1.041536in}{1.243796in}}{\pgfqpoint{1.041536in}{1.235559in}}%
\pgfpathcurveto{\pgfqpoint{1.041536in}{1.227323in}}{\pgfqpoint{1.044808in}{1.219423in}}{\pgfqpoint{1.050632in}{1.213599in}}%
\pgfpathcurveto{\pgfqpoint{1.056456in}{1.207775in}}{\pgfqpoint{1.064356in}{1.204503in}}{\pgfqpoint{1.072593in}{1.204503in}}%
\pgfpathclose%
\pgfusepath{stroke,fill}%
\end{pgfscope}%
\begin{pgfscope}%
\pgfpathrectangle{\pgfqpoint{0.100000in}{0.212622in}}{\pgfqpoint{3.696000in}{3.696000in}}%
\pgfusepath{clip}%
\pgfsetbuttcap%
\pgfsetroundjoin%
\definecolor{currentfill}{rgb}{0.121569,0.466667,0.705882}%
\pgfsetfillcolor{currentfill}%
\pgfsetfillopacity{0.700842}%
\pgfsetlinewidth{1.003750pt}%
\definecolor{currentstroke}{rgb}{0.121569,0.466667,0.705882}%
\pgfsetstrokecolor{currentstroke}%
\pgfsetstrokeopacity{0.700842}%
\pgfsetdash{}{0pt}%
\pgfpathmoveto{\pgfqpoint{2.178990in}{1.778971in}}%
\pgfpathcurveto{\pgfqpoint{2.187226in}{1.778971in}}{\pgfqpoint{2.195126in}{1.782244in}}{\pgfqpoint{2.200950in}{1.788068in}}%
\pgfpathcurveto{\pgfqpoint{2.206774in}{1.793892in}}{\pgfqpoint{2.210047in}{1.801792in}}{\pgfqpoint{2.210047in}{1.810028in}}%
\pgfpathcurveto{\pgfqpoint{2.210047in}{1.818264in}}{\pgfqpoint{2.206774in}{1.826164in}}{\pgfqpoint{2.200950in}{1.831988in}}%
\pgfpathcurveto{\pgfqpoint{2.195126in}{1.837812in}}{\pgfqpoint{2.187226in}{1.841084in}}{\pgfqpoint{2.178990in}{1.841084in}}%
\pgfpathcurveto{\pgfqpoint{2.170754in}{1.841084in}}{\pgfqpoint{2.162854in}{1.837812in}}{\pgfqpoint{2.157030in}{1.831988in}}%
\pgfpathcurveto{\pgfqpoint{2.151206in}{1.826164in}}{\pgfqpoint{2.147934in}{1.818264in}}{\pgfqpoint{2.147934in}{1.810028in}}%
\pgfpathcurveto{\pgfqpoint{2.147934in}{1.801792in}}{\pgfqpoint{2.151206in}{1.793892in}}{\pgfqpoint{2.157030in}{1.788068in}}%
\pgfpathcurveto{\pgfqpoint{2.162854in}{1.782244in}}{\pgfqpoint{2.170754in}{1.778971in}}{\pgfqpoint{2.178990in}{1.778971in}}%
\pgfpathclose%
\pgfusepath{stroke,fill}%
\end{pgfscope}%
\begin{pgfscope}%
\pgfpathrectangle{\pgfqpoint{0.100000in}{0.212622in}}{\pgfqpoint{3.696000in}{3.696000in}}%
\pgfusepath{clip}%
\pgfsetbuttcap%
\pgfsetroundjoin%
\definecolor{currentfill}{rgb}{0.121569,0.466667,0.705882}%
\pgfsetfillcolor{currentfill}%
\pgfsetfillopacity{0.701707}%
\pgfsetlinewidth{1.003750pt}%
\definecolor{currentstroke}{rgb}{0.121569,0.466667,0.705882}%
\pgfsetstrokecolor{currentstroke}%
\pgfsetstrokeopacity{0.701707}%
\pgfsetdash{}{0pt}%
\pgfpathmoveto{\pgfqpoint{1.078139in}{1.204076in}}%
\pgfpathcurveto{\pgfqpoint{1.086375in}{1.204076in}}{\pgfqpoint{1.094275in}{1.207348in}}{\pgfqpoint{1.100099in}{1.213172in}}%
\pgfpathcurveto{\pgfqpoint{1.105923in}{1.218996in}}{\pgfqpoint{1.109195in}{1.226896in}}{\pgfqpoint{1.109195in}{1.235133in}}%
\pgfpathcurveto{\pgfqpoint{1.109195in}{1.243369in}}{\pgfqpoint{1.105923in}{1.251269in}}{\pgfqpoint{1.100099in}{1.257093in}}%
\pgfpathcurveto{\pgfqpoint{1.094275in}{1.262917in}}{\pgfqpoint{1.086375in}{1.266189in}}{\pgfqpoint{1.078139in}{1.266189in}}%
\pgfpathcurveto{\pgfqpoint{1.069902in}{1.266189in}}{\pgfqpoint{1.062002in}{1.262917in}}{\pgfqpoint{1.056178in}{1.257093in}}%
\pgfpathcurveto{\pgfqpoint{1.050355in}{1.251269in}}{\pgfqpoint{1.047082in}{1.243369in}}{\pgfqpoint{1.047082in}{1.235133in}}%
\pgfpathcurveto{\pgfqpoint{1.047082in}{1.226896in}}{\pgfqpoint{1.050355in}{1.218996in}}{\pgfqpoint{1.056178in}{1.213172in}}%
\pgfpathcurveto{\pgfqpoint{1.062002in}{1.207348in}}{\pgfqpoint{1.069902in}{1.204076in}}{\pgfqpoint{1.078139in}{1.204076in}}%
\pgfpathclose%
\pgfusepath{stroke,fill}%
\end{pgfscope}%
\begin{pgfscope}%
\pgfpathrectangle{\pgfqpoint{0.100000in}{0.212622in}}{\pgfqpoint{3.696000in}{3.696000in}}%
\pgfusepath{clip}%
\pgfsetbuttcap%
\pgfsetroundjoin%
\definecolor{currentfill}{rgb}{0.121569,0.466667,0.705882}%
\pgfsetfillcolor{currentfill}%
\pgfsetfillopacity{0.702730}%
\pgfsetlinewidth{1.003750pt}%
\definecolor{currentstroke}{rgb}{0.121569,0.466667,0.705882}%
\pgfsetstrokecolor{currentstroke}%
\pgfsetstrokeopacity{0.702730}%
\pgfsetdash{}{0pt}%
\pgfpathmoveto{\pgfqpoint{1.083003in}{1.203991in}}%
\pgfpathcurveto{\pgfqpoint{1.091239in}{1.203991in}}{\pgfqpoint{1.099139in}{1.207264in}}{\pgfqpoint{1.104963in}{1.213087in}}%
\pgfpathcurveto{\pgfqpoint{1.110787in}{1.218911in}}{\pgfqpoint{1.114060in}{1.226811in}}{\pgfqpoint{1.114060in}{1.235048in}}%
\pgfpathcurveto{\pgfqpoint{1.114060in}{1.243284in}}{\pgfqpoint{1.110787in}{1.251184in}}{\pgfqpoint{1.104963in}{1.257008in}}%
\pgfpathcurveto{\pgfqpoint{1.099139in}{1.262832in}}{\pgfqpoint{1.091239in}{1.266104in}}{\pgfqpoint{1.083003in}{1.266104in}}%
\pgfpathcurveto{\pgfqpoint{1.074767in}{1.266104in}}{\pgfqpoint{1.066867in}{1.262832in}}{\pgfqpoint{1.061043in}{1.257008in}}%
\pgfpathcurveto{\pgfqpoint{1.055219in}{1.251184in}}{\pgfqpoint{1.051947in}{1.243284in}}{\pgfqpoint{1.051947in}{1.235048in}}%
\pgfpathcurveto{\pgfqpoint{1.051947in}{1.226811in}}{\pgfqpoint{1.055219in}{1.218911in}}{\pgfqpoint{1.061043in}{1.213087in}}%
\pgfpathcurveto{\pgfqpoint{1.066867in}{1.207264in}}{\pgfqpoint{1.074767in}{1.203991in}}{\pgfqpoint{1.083003in}{1.203991in}}%
\pgfpathclose%
\pgfusepath{stroke,fill}%
\end{pgfscope}%
\begin{pgfscope}%
\pgfpathrectangle{\pgfqpoint{0.100000in}{0.212622in}}{\pgfqpoint{3.696000in}{3.696000in}}%
\pgfusepath{clip}%
\pgfsetbuttcap%
\pgfsetroundjoin%
\definecolor{currentfill}{rgb}{0.121569,0.466667,0.705882}%
\pgfsetfillcolor{currentfill}%
\pgfsetfillopacity{0.703296}%
\pgfsetlinewidth{1.003750pt}%
\definecolor{currentstroke}{rgb}{0.121569,0.466667,0.705882}%
\pgfsetstrokecolor{currentstroke}%
\pgfsetstrokeopacity{0.703296}%
\pgfsetdash{}{0pt}%
\pgfpathmoveto{\pgfqpoint{2.180519in}{1.771818in}}%
\pgfpathcurveto{\pgfqpoint{2.188756in}{1.771818in}}{\pgfqpoint{2.196656in}{1.775090in}}{\pgfqpoint{2.202480in}{1.780914in}}%
\pgfpathcurveto{\pgfqpoint{2.208304in}{1.786738in}}{\pgfqpoint{2.211576in}{1.794638in}}{\pgfqpoint{2.211576in}{1.802874in}}%
\pgfpathcurveto{\pgfqpoint{2.211576in}{1.811110in}}{\pgfqpoint{2.208304in}{1.819010in}}{\pgfqpoint{2.202480in}{1.824834in}}%
\pgfpathcurveto{\pgfqpoint{2.196656in}{1.830658in}}{\pgfqpoint{2.188756in}{1.833931in}}{\pgfqpoint{2.180519in}{1.833931in}}%
\pgfpathcurveto{\pgfqpoint{2.172283in}{1.833931in}}{\pgfqpoint{2.164383in}{1.830658in}}{\pgfqpoint{2.158559in}{1.824834in}}%
\pgfpathcurveto{\pgfqpoint{2.152735in}{1.819010in}}{\pgfqpoint{2.149463in}{1.811110in}}{\pgfqpoint{2.149463in}{1.802874in}}%
\pgfpathcurveto{\pgfqpoint{2.149463in}{1.794638in}}{\pgfqpoint{2.152735in}{1.786738in}}{\pgfqpoint{2.158559in}{1.780914in}}%
\pgfpathcurveto{\pgfqpoint{2.164383in}{1.775090in}}{\pgfqpoint{2.172283in}{1.771818in}}{\pgfqpoint{2.180519in}{1.771818in}}%
\pgfpathclose%
\pgfusepath{stroke,fill}%
\end{pgfscope}%
\begin{pgfscope}%
\pgfpathrectangle{\pgfqpoint{0.100000in}{0.212622in}}{\pgfqpoint{3.696000in}{3.696000in}}%
\pgfusepath{clip}%
\pgfsetbuttcap%
\pgfsetroundjoin%
\definecolor{currentfill}{rgb}{0.121569,0.466667,0.705882}%
\pgfsetfillcolor{currentfill}%
\pgfsetfillopacity{0.704853}%
\pgfsetlinewidth{1.003750pt}%
\definecolor{currentstroke}{rgb}{0.121569,0.466667,0.705882}%
\pgfsetstrokecolor{currentstroke}%
\pgfsetstrokeopacity{0.704853}%
\pgfsetdash{}{0pt}%
\pgfpathmoveto{\pgfqpoint{1.091706in}{1.204288in}}%
\pgfpathcurveto{\pgfqpoint{1.099942in}{1.204288in}}{\pgfqpoint{1.107842in}{1.207561in}}{\pgfqpoint{1.113666in}{1.213385in}}%
\pgfpathcurveto{\pgfqpoint{1.119490in}{1.219209in}}{\pgfqpoint{1.122763in}{1.227109in}}{\pgfqpoint{1.122763in}{1.235345in}}%
\pgfpathcurveto{\pgfqpoint{1.122763in}{1.243581in}}{\pgfqpoint{1.119490in}{1.251481in}}{\pgfqpoint{1.113666in}{1.257305in}}%
\pgfpathcurveto{\pgfqpoint{1.107842in}{1.263129in}}{\pgfqpoint{1.099942in}{1.266401in}}{\pgfqpoint{1.091706in}{1.266401in}}%
\pgfpathcurveto{\pgfqpoint{1.083470in}{1.266401in}}{\pgfqpoint{1.075570in}{1.263129in}}{\pgfqpoint{1.069746in}{1.257305in}}%
\pgfpathcurveto{\pgfqpoint{1.063922in}{1.251481in}}{\pgfqpoint{1.060650in}{1.243581in}}{\pgfqpoint{1.060650in}{1.235345in}}%
\pgfpathcurveto{\pgfqpoint{1.060650in}{1.227109in}}{\pgfqpoint{1.063922in}{1.219209in}}{\pgfqpoint{1.069746in}{1.213385in}}%
\pgfpathcurveto{\pgfqpoint{1.075570in}{1.207561in}}{\pgfqpoint{1.083470in}{1.204288in}}{\pgfqpoint{1.091706in}{1.204288in}}%
\pgfpathclose%
\pgfusepath{stroke,fill}%
\end{pgfscope}%
\begin{pgfscope}%
\pgfpathrectangle{\pgfqpoint{0.100000in}{0.212622in}}{\pgfqpoint{3.696000in}{3.696000in}}%
\pgfusepath{clip}%
\pgfsetbuttcap%
\pgfsetroundjoin%
\definecolor{currentfill}{rgb}{0.121569,0.466667,0.705882}%
\pgfsetfillcolor{currentfill}%
\pgfsetfillopacity{0.705885}%
\pgfsetlinewidth{1.003750pt}%
\definecolor{currentstroke}{rgb}{0.121569,0.466667,0.705882}%
\pgfsetstrokecolor{currentstroke}%
\pgfsetstrokeopacity{0.705885}%
\pgfsetdash{}{0pt}%
\pgfpathmoveto{\pgfqpoint{2.182433in}{1.764053in}}%
\pgfpathcurveto{\pgfqpoint{2.190670in}{1.764053in}}{\pgfqpoint{2.198570in}{1.767326in}}{\pgfqpoint{2.204394in}{1.773149in}}%
\pgfpathcurveto{\pgfqpoint{2.210218in}{1.778973in}}{\pgfqpoint{2.213490in}{1.786873in}}{\pgfqpoint{2.213490in}{1.795110in}}%
\pgfpathcurveto{\pgfqpoint{2.213490in}{1.803346in}}{\pgfqpoint{2.210218in}{1.811246in}}{\pgfqpoint{2.204394in}{1.817070in}}%
\pgfpathcurveto{\pgfqpoint{2.198570in}{1.822894in}}{\pgfqpoint{2.190670in}{1.826166in}}{\pgfqpoint{2.182433in}{1.826166in}}%
\pgfpathcurveto{\pgfqpoint{2.174197in}{1.826166in}}{\pgfqpoint{2.166297in}{1.822894in}}{\pgfqpoint{2.160473in}{1.817070in}}%
\pgfpathcurveto{\pgfqpoint{2.154649in}{1.811246in}}{\pgfqpoint{2.151377in}{1.803346in}}{\pgfqpoint{2.151377in}{1.795110in}}%
\pgfpathcurveto{\pgfqpoint{2.151377in}{1.786873in}}{\pgfqpoint{2.154649in}{1.778973in}}{\pgfqpoint{2.160473in}{1.773149in}}%
\pgfpathcurveto{\pgfqpoint{2.166297in}{1.767326in}}{\pgfqpoint{2.174197in}{1.764053in}}{\pgfqpoint{2.182433in}{1.764053in}}%
\pgfpathclose%
\pgfusepath{stroke,fill}%
\end{pgfscope}%
\begin{pgfscope}%
\pgfpathrectangle{\pgfqpoint{0.100000in}{0.212622in}}{\pgfqpoint{3.696000in}{3.696000in}}%
\pgfusepath{clip}%
\pgfsetbuttcap%
\pgfsetroundjoin%
\definecolor{currentfill}{rgb}{0.121569,0.466667,0.705882}%
\pgfsetfillcolor{currentfill}%
\pgfsetfillopacity{0.706595}%
\pgfsetlinewidth{1.003750pt}%
\definecolor{currentstroke}{rgb}{0.121569,0.466667,0.705882}%
\pgfsetstrokecolor{currentstroke}%
\pgfsetstrokeopacity{0.706595}%
\pgfsetdash{}{0pt}%
\pgfpathmoveto{\pgfqpoint{1.098991in}{1.204049in}}%
\pgfpathcurveto{\pgfqpoint{1.107227in}{1.204049in}}{\pgfqpoint{1.115127in}{1.207322in}}{\pgfqpoint{1.120951in}{1.213146in}}%
\pgfpathcurveto{\pgfqpoint{1.126775in}{1.218970in}}{\pgfqpoint{1.130047in}{1.226870in}}{\pgfqpoint{1.130047in}{1.235106in}}%
\pgfpathcurveto{\pgfqpoint{1.130047in}{1.243342in}}{\pgfqpoint{1.126775in}{1.251242in}}{\pgfqpoint{1.120951in}{1.257066in}}%
\pgfpathcurveto{\pgfqpoint{1.115127in}{1.262890in}}{\pgfqpoint{1.107227in}{1.266162in}}{\pgfqpoint{1.098991in}{1.266162in}}%
\pgfpathcurveto{\pgfqpoint{1.090755in}{1.266162in}}{\pgfqpoint{1.082855in}{1.262890in}}{\pgfqpoint{1.077031in}{1.257066in}}%
\pgfpathcurveto{\pgfqpoint{1.071207in}{1.251242in}}{\pgfqpoint{1.067934in}{1.243342in}}{\pgfqpoint{1.067934in}{1.235106in}}%
\pgfpathcurveto{\pgfqpoint{1.067934in}{1.226870in}}{\pgfqpoint{1.071207in}{1.218970in}}{\pgfqpoint{1.077031in}{1.213146in}}%
\pgfpathcurveto{\pgfqpoint{1.082855in}{1.207322in}}{\pgfqpoint{1.090755in}{1.204049in}}{\pgfqpoint{1.098991in}{1.204049in}}%
\pgfpathclose%
\pgfusepath{stroke,fill}%
\end{pgfscope}%
\begin{pgfscope}%
\pgfpathrectangle{\pgfqpoint{0.100000in}{0.212622in}}{\pgfqpoint{3.696000in}{3.696000in}}%
\pgfusepath{clip}%
\pgfsetbuttcap%
\pgfsetroundjoin%
\definecolor{currentfill}{rgb}{0.121569,0.466667,0.705882}%
\pgfsetfillcolor{currentfill}%
\pgfsetfillopacity{0.708125}%
\pgfsetlinewidth{1.003750pt}%
\definecolor{currentstroke}{rgb}{0.121569,0.466667,0.705882}%
\pgfsetstrokecolor{currentstroke}%
\pgfsetstrokeopacity{0.708125}%
\pgfsetdash{}{0pt}%
\pgfpathmoveto{\pgfqpoint{2.184752in}{1.754713in}}%
\pgfpathcurveto{\pgfqpoint{2.192988in}{1.754713in}}{\pgfqpoint{2.200888in}{1.757985in}}{\pgfqpoint{2.206712in}{1.763809in}}%
\pgfpathcurveto{\pgfqpoint{2.212536in}{1.769633in}}{\pgfqpoint{2.215809in}{1.777533in}}{\pgfqpoint{2.215809in}{1.785769in}}%
\pgfpathcurveto{\pgfqpoint{2.215809in}{1.794005in}}{\pgfqpoint{2.212536in}{1.801906in}}{\pgfqpoint{2.206712in}{1.807729in}}%
\pgfpathcurveto{\pgfqpoint{2.200888in}{1.813553in}}{\pgfqpoint{2.192988in}{1.816826in}}{\pgfqpoint{2.184752in}{1.816826in}}%
\pgfpathcurveto{\pgfqpoint{2.176516in}{1.816826in}}{\pgfqpoint{2.168616in}{1.813553in}}{\pgfqpoint{2.162792in}{1.807729in}}%
\pgfpathcurveto{\pgfqpoint{2.156968in}{1.801906in}}{\pgfqpoint{2.153696in}{1.794005in}}{\pgfqpoint{2.153696in}{1.785769in}}%
\pgfpathcurveto{\pgfqpoint{2.153696in}{1.777533in}}{\pgfqpoint{2.156968in}{1.769633in}}{\pgfqpoint{2.162792in}{1.763809in}}%
\pgfpathcurveto{\pgfqpoint{2.168616in}{1.757985in}}{\pgfqpoint{2.176516in}{1.754713in}}{\pgfqpoint{2.184752in}{1.754713in}}%
\pgfpathclose%
\pgfusepath{stroke,fill}%
\end{pgfscope}%
\begin{pgfscope}%
\pgfpathrectangle{\pgfqpoint{0.100000in}{0.212622in}}{\pgfqpoint{3.696000in}{3.696000in}}%
\pgfusepath{clip}%
\pgfsetbuttcap%
\pgfsetroundjoin%
\definecolor{currentfill}{rgb}{0.121569,0.466667,0.705882}%
\pgfsetfillcolor{currentfill}%
\pgfsetfillopacity{0.708133}%
\pgfsetlinewidth{1.003750pt}%
\definecolor{currentstroke}{rgb}{0.121569,0.466667,0.705882}%
\pgfsetstrokecolor{currentstroke}%
\pgfsetstrokeopacity{0.708133}%
\pgfsetdash{}{0pt}%
\pgfpathmoveto{\pgfqpoint{1.106009in}{1.203757in}}%
\pgfpathcurveto{\pgfqpoint{1.114246in}{1.203757in}}{\pgfqpoint{1.122146in}{1.207029in}}{\pgfqpoint{1.127970in}{1.212853in}}%
\pgfpathcurveto{\pgfqpoint{1.133794in}{1.218677in}}{\pgfqpoint{1.137066in}{1.226577in}}{\pgfqpoint{1.137066in}{1.234813in}}%
\pgfpathcurveto{\pgfqpoint{1.137066in}{1.243049in}}{\pgfqpoint{1.133794in}{1.250949in}}{\pgfqpoint{1.127970in}{1.256773in}}%
\pgfpathcurveto{\pgfqpoint{1.122146in}{1.262597in}}{\pgfqpoint{1.114246in}{1.265870in}}{\pgfqpoint{1.106009in}{1.265870in}}%
\pgfpathcurveto{\pgfqpoint{1.097773in}{1.265870in}}{\pgfqpoint{1.089873in}{1.262597in}}{\pgfqpoint{1.084049in}{1.256773in}}%
\pgfpathcurveto{\pgfqpoint{1.078225in}{1.250949in}}{\pgfqpoint{1.074953in}{1.243049in}}{\pgfqpoint{1.074953in}{1.234813in}}%
\pgfpathcurveto{\pgfqpoint{1.074953in}{1.226577in}}{\pgfqpoint{1.078225in}{1.218677in}}{\pgfqpoint{1.084049in}{1.212853in}}%
\pgfpathcurveto{\pgfqpoint{1.089873in}{1.207029in}}{\pgfqpoint{1.097773in}{1.203757in}}{\pgfqpoint{1.106009in}{1.203757in}}%
\pgfpathclose%
\pgfusepath{stroke,fill}%
\end{pgfscope}%
\begin{pgfscope}%
\pgfpathrectangle{\pgfqpoint{0.100000in}{0.212622in}}{\pgfqpoint{3.696000in}{3.696000in}}%
\pgfusepath{clip}%
\pgfsetbuttcap%
\pgfsetroundjoin%
\definecolor{currentfill}{rgb}{0.121569,0.466667,0.705882}%
\pgfsetfillcolor{currentfill}%
\pgfsetfillopacity{0.709384}%
\pgfsetlinewidth{1.003750pt}%
\definecolor{currentstroke}{rgb}{0.121569,0.466667,0.705882}%
\pgfsetstrokecolor{currentstroke}%
\pgfsetstrokeopacity{0.709384}%
\pgfsetdash{}{0pt}%
\pgfpathmoveto{\pgfqpoint{1.112290in}{1.203272in}}%
\pgfpathcurveto{\pgfqpoint{1.120526in}{1.203272in}}{\pgfqpoint{1.128426in}{1.206544in}}{\pgfqpoint{1.134250in}{1.212368in}}%
\pgfpathcurveto{\pgfqpoint{1.140074in}{1.218192in}}{\pgfqpoint{1.143346in}{1.226092in}}{\pgfqpoint{1.143346in}{1.234328in}}%
\pgfpathcurveto{\pgfqpoint{1.143346in}{1.242565in}}{\pgfqpoint{1.140074in}{1.250465in}}{\pgfqpoint{1.134250in}{1.256289in}}%
\pgfpathcurveto{\pgfqpoint{1.128426in}{1.262112in}}{\pgfqpoint{1.120526in}{1.265385in}}{\pgfqpoint{1.112290in}{1.265385in}}%
\pgfpathcurveto{\pgfqpoint{1.104054in}{1.265385in}}{\pgfqpoint{1.096153in}{1.262112in}}{\pgfqpoint{1.090330in}{1.256289in}}%
\pgfpathcurveto{\pgfqpoint{1.084506in}{1.250465in}}{\pgfqpoint{1.081233in}{1.242565in}}{\pgfqpoint{1.081233in}{1.234328in}}%
\pgfpathcurveto{\pgfqpoint{1.081233in}{1.226092in}}{\pgfqpoint{1.084506in}{1.218192in}}{\pgfqpoint{1.090330in}{1.212368in}}%
\pgfpathcurveto{\pgfqpoint{1.096153in}{1.206544in}}{\pgfqpoint{1.104054in}{1.203272in}}{\pgfqpoint{1.112290in}{1.203272in}}%
\pgfpathclose%
\pgfusepath{stroke,fill}%
\end{pgfscope}%
\begin{pgfscope}%
\pgfpathrectangle{\pgfqpoint{0.100000in}{0.212622in}}{\pgfqpoint{3.696000in}{3.696000in}}%
\pgfusepath{clip}%
\pgfsetbuttcap%
\pgfsetroundjoin%
\definecolor{currentfill}{rgb}{0.121569,0.466667,0.705882}%
\pgfsetfillcolor{currentfill}%
\pgfsetfillopacity{0.710377}%
\pgfsetlinewidth{1.003750pt}%
\definecolor{currentstroke}{rgb}{0.121569,0.466667,0.705882}%
\pgfsetstrokecolor{currentstroke}%
\pgfsetstrokeopacity{0.710377}%
\pgfsetdash{}{0pt}%
\pgfpathmoveto{\pgfqpoint{1.117003in}{1.202598in}}%
\pgfpathcurveto{\pgfqpoint{1.125239in}{1.202598in}}{\pgfqpoint{1.133139in}{1.205870in}}{\pgfqpoint{1.138963in}{1.211694in}}%
\pgfpathcurveto{\pgfqpoint{1.144787in}{1.217518in}}{\pgfqpoint{1.148059in}{1.225418in}}{\pgfqpoint{1.148059in}{1.233654in}}%
\pgfpathcurveto{\pgfqpoint{1.148059in}{1.241891in}}{\pgfqpoint{1.144787in}{1.249791in}}{\pgfqpoint{1.138963in}{1.255615in}}%
\pgfpathcurveto{\pgfqpoint{1.133139in}{1.261439in}}{\pgfqpoint{1.125239in}{1.264711in}}{\pgfqpoint{1.117003in}{1.264711in}}%
\pgfpathcurveto{\pgfqpoint{1.108767in}{1.264711in}}{\pgfqpoint{1.100867in}{1.261439in}}{\pgfqpoint{1.095043in}{1.255615in}}%
\pgfpathcurveto{\pgfqpoint{1.089219in}{1.249791in}}{\pgfqpoint{1.085946in}{1.241891in}}{\pgfqpoint{1.085946in}{1.233654in}}%
\pgfpathcurveto{\pgfqpoint{1.085946in}{1.225418in}}{\pgfqpoint{1.089219in}{1.217518in}}{\pgfqpoint{1.095043in}{1.211694in}}%
\pgfpathcurveto{\pgfqpoint{1.100867in}{1.205870in}}{\pgfqpoint{1.108767in}{1.202598in}}{\pgfqpoint{1.117003in}{1.202598in}}%
\pgfpathclose%
\pgfusepath{stroke,fill}%
\end{pgfscope}%
\begin{pgfscope}%
\pgfpathrectangle{\pgfqpoint{0.100000in}{0.212622in}}{\pgfqpoint{3.696000in}{3.696000in}}%
\pgfusepath{clip}%
\pgfsetbuttcap%
\pgfsetroundjoin%
\definecolor{currentfill}{rgb}{0.121569,0.466667,0.705882}%
\pgfsetfillcolor{currentfill}%
\pgfsetfillopacity{0.711015}%
\pgfsetlinewidth{1.003750pt}%
\definecolor{currentstroke}{rgb}{0.121569,0.466667,0.705882}%
\pgfsetstrokecolor{currentstroke}%
\pgfsetstrokeopacity{0.711015}%
\pgfsetdash{}{0pt}%
\pgfpathmoveto{\pgfqpoint{2.186685in}{1.744333in}}%
\pgfpathcurveto{\pgfqpoint{2.194921in}{1.744333in}}{\pgfqpoint{2.202821in}{1.747606in}}{\pgfqpoint{2.208645in}{1.753429in}}%
\pgfpathcurveto{\pgfqpoint{2.214469in}{1.759253in}}{\pgfqpoint{2.217741in}{1.767153in}}{\pgfqpoint{2.217741in}{1.775390in}}%
\pgfpathcurveto{\pgfqpoint{2.217741in}{1.783626in}}{\pgfqpoint{2.214469in}{1.791526in}}{\pgfqpoint{2.208645in}{1.797350in}}%
\pgfpathcurveto{\pgfqpoint{2.202821in}{1.803174in}}{\pgfqpoint{2.194921in}{1.806446in}}{\pgfqpoint{2.186685in}{1.806446in}}%
\pgfpathcurveto{\pgfqpoint{2.178448in}{1.806446in}}{\pgfqpoint{2.170548in}{1.803174in}}{\pgfqpoint{2.164724in}{1.797350in}}%
\pgfpathcurveto{\pgfqpoint{2.158901in}{1.791526in}}{\pgfqpoint{2.155628in}{1.783626in}}{\pgfqpoint{2.155628in}{1.775390in}}%
\pgfpathcurveto{\pgfqpoint{2.155628in}{1.767153in}}{\pgfqpoint{2.158901in}{1.759253in}}{\pgfqpoint{2.164724in}{1.753429in}}%
\pgfpathcurveto{\pgfqpoint{2.170548in}{1.747606in}}{\pgfqpoint{2.178448in}{1.744333in}}{\pgfqpoint{2.186685in}{1.744333in}}%
\pgfpathclose%
\pgfusepath{stroke,fill}%
\end{pgfscope}%
\begin{pgfscope}%
\pgfpathrectangle{\pgfqpoint{0.100000in}{0.212622in}}{\pgfqpoint{3.696000in}{3.696000in}}%
\pgfusepath{clip}%
\pgfsetbuttcap%
\pgfsetroundjoin%
\definecolor{currentfill}{rgb}{0.121569,0.466667,0.705882}%
\pgfsetfillcolor{currentfill}%
\pgfsetfillopacity{0.712270}%
\pgfsetlinewidth{1.003750pt}%
\definecolor{currentstroke}{rgb}{0.121569,0.466667,0.705882}%
\pgfsetstrokecolor{currentstroke}%
\pgfsetstrokeopacity{0.712270}%
\pgfsetdash{}{0pt}%
\pgfpathmoveto{\pgfqpoint{1.125644in}{1.202008in}}%
\pgfpathcurveto{\pgfqpoint{1.133880in}{1.202008in}}{\pgfqpoint{1.141780in}{1.205280in}}{\pgfqpoint{1.147604in}{1.211104in}}%
\pgfpathcurveto{\pgfqpoint{1.153428in}{1.216928in}}{\pgfqpoint{1.156701in}{1.224828in}}{\pgfqpoint{1.156701in}{1.233064in}}%
\pgfpathcurveto{\pgfqpoint{1.156701in}{1.241300in}}{\pgfqpoint{1.153428in}{1.249200in}}{\pgfqpoint{1.147604in}{1.255024in}}%
\pgfpathcurveto{\pgfqpoint{1.141780in}{1.260848in}}{\pgfqpoint{1.133880in}{1.264121in}}{\pgfqpoint{1.125644in}{1.264121in}}%
\pgfpathcurveto{\pgfqpoint{1.117408in}{1.264121in}}{\pgfqpoint{1.109508in}{1.260848in}}{\pgfqpoint{1.103684in}{1.255024in}}%
\pgfpathcurveto{\pgfqpoint{1.097860in}{1.249200in}}{\pgfqpoint{1.094588in}{1.241300in}}{\pgfqpoint{1.094588in}{1.233064in}}%
\pgfpathcurveto{\pgfqpoint{1.094588in}{1.224828in}}{\pgfqpoint{1.097860in}{1.216928in}}{\pgfqpoint{1.103684in}{1.211104in}}%
\pgfpathcurveto{\pgfqpoint{1.109508in}{1.205280in}}{\pgfqpoint{1.117408in}{1.202008in}}{\pgfqpoint{1.125644in}{1.202008in}}%
\pgfpathclose%
\pgfusepath{stroke,fill}%
\end{pgfscope}%
\begin{pgfscope}%
\pgfpathrectangle{\pgfqpoint{0.100000in}{0.212622in}}{\pgfqpoint{3.696000in}{3.696000in}}%
\pgfusepath{clip}%
\pgfsetbuttcap%
\pgfsetroundjoin%
\definecolor{currentfill}{rgb}{0.121569,0.466667,0.705882}%
\pgfsetfillcolor{currentfill}%
\pgfsetfillopacity{0.713939}%
\pgfsetlinewidth{1.003750pt}%
\definecolor{currentstroke}{rgb}{0.121569,0.466667,0.705882}%
\pgfsetstrokecolor{currentstroke}%
\pgfsetstrokeopacity{0.713939}%
\pgfsetdash{}{0pt}%
\pgfpathmoveto{\pgfqpoint{1.133590in}{1.201565in}}%
\pgfpathcurveto{\pgfqpoint{1.141826in}{1.201565in}}{\pgfqpoint{1.149726in}{1.204837in}}{\pgfqpoint{1.155550in}{1.210661in}}%
\pgfpathcurveto{\pgfqpoint{1.161374in}{1.216485in}}{\pgfqpoint{1.164646in}{1.224385in}}{\pgfqpoint{1.164646in}{1.232621in}}%
\pgfpathcurveto{\pgfqpoint{1.164646in}{1.240858in}}{\pgfqpoint{1.161374in}{1.248758in}}{\pgfqpoint{1.155550in}{1.254582in}}%
\pgfpathcurveto{\pgfqpoint{1.149726in}{1.260406in}}{\pgfqpoint{1.141826in}{1.263678in}}{\pgfqpoint{1.133590in}{1.263678in}}%
\pgfpathcurveto{\pgfqpoint{1.125354in}{1.263678in}}{\pgfqpoint{1.117454in}{1.260406in}}{\pgfqpoint{1.111630in}{1.254582in}}%
\pgfpathcurveto{\pgfqpoint{1.105806in}{1.248758in}}{\pgfqpoint{1.102533in}{1.240858in}}{\pgfqpoint{1.102533in}{1.232621in}}%
\pgfpathcurveto{\pgfqpoint{1.102533in}{1.224385in}}{\pgfqpoint{1.105806in}{1.216485in}}{\pgfqpoint{1.111630in}{1.210661in}}%
\pgfpathcurveto{\pgfqpoint{1.117454in}{1.204837in}}{\pgfqpoint{1.125354in}{1.201565in}}{\pgfqpoint{1.133590in}{1.201565in}}%
\pgfpathclose%
\pgfusepath{stroke,fill}%
\end{pgfscope}%
\begin{pgfscope}%
\pgfpathrectangle{\pgfqpoint{0.100000in}{0.212622in}}{\pgfqpoint{3.696000in}{3.696000in}}%
\pgfusepath{clip}%
\pgfsetbuttcap%
\pgfsetroundjoin%
\definecolor{currentfill}{rgb}{0.121569,0.466667,0.705882}%
\pgfsetfillcolor{currentfill}%
\pgfsetfillopacity{0.714129}%
\pgfsetlinewidth{1.003750pt}%
\definecolor{currentstroke}{rgb}{0.121569,0.466667,0.705882}%
\pgfsetstrokecolor{currentstroke}%
\pgfsetstrokeopacity{0.714129}%
\pgfsetdash{}{0pt}%
\pgfpathmoveto{\pgfqpoint{2.188902in}{1.733664in}}%
\pgfpathcurveto{\pgfqpoint{2.197138in}{1.733664in}}{\pgfqpoint{2.205038in}{1.736937in}}{\pgfqpoint{2.210862in}{1.742761in}}%
\pgfpathcurveto{\pgfqpoint{2.216686in}{1.748584in}}{\pgfqpoint{2.219958in}{1.756485in}}{\pgfqpoint{2.219958in}{1.764721in}}%
\pgfpathcurveto{\pgfqpoint{2.219958in}{1.772957in}}{\pgfqpoint{2.216686in}{1.780857in}}{\pgfqpoint{2.210862in}{1.786681in}}%
\pgfpathcurveto{\pgfqpoint{2.205038in}{1.792505in}}{\pgfqpoint{2.197138in}{1.795777in}}{\pgfqpoint{2.188902in}{1.795777in}}%
\pgfpathcurveto{\pgfqpoint{2.180665in}{1.795777in}}{\pgfqpoint{2.172765in}{1.792505in}}{\pgfqpoint{2.166941in}{1.786681in}}%
\pgfpathcurveto{\pgfqpoint{2.161117in}{1.780857in}}{\pgfqpoint{2.157845in}{1.772957in}}{\pgfqpoint{2.157845in}{1.764721in}}%
\pgfpathcurveto{\pgfqpoint{2.157845in}{1.756485in}}{\pgfqpoint{2.161117in}{1.748584in}}{\pgfqpoint{2.166941in}{1.742761in}}%
\pgfpathcurveto{\pgfqpoint{2.172765in}{1.736937in}}{\pgfqpoint{2.180665in}{1.733664in}}{\pgfqpoint{2.188902in}{1.733664in}}%
\pgfpathclose%
\pgfusepath{stroke,fill}%
\end{pgfscope}%
\begin{pgfscope}%
\pgfpathrectangle{\pgfqpoint{0.100000in}{0.212622in}}{\pgfqpoint{3.696000in}{3.696000in}}%
\pgfusepath{clip}%
\pgfsetbuttcap%
\pgfsetroundjoin%
\definecolor{currentfill}{rgb}{0.121569,0.466667,0.705882}%
\pgfsetfillcolor{currentfill}%
\pgfsetfillopacity{0.717117}%
\pgfsetlinewidth{1.003750pt}%
\definecolor{currentstroke}{rgb}{0.121569,0.466667,0.705882}%
\pgfsetstrokecolor{currentstroke}%
\pgfsetstrokeopacity{0.717117}%
\pgfsetdash{}{0pt}%
\pgfpathmoveto{\pgfqpoint{2.192041in}{1.722248in}}%
\pgfpathcurveto{\pgfqpoint{2.200277in}{1.722248in}}{\pgfqpoint{2.208177in}{1.725520in}}{\pgfqpoint{2.214001in}{1.731344in}}%
\pgfpathcurveto{\pgfqpoint{2.219825in}{1.737168in}}{\pgfqpoint{2.223097in}{1.745068in}}{\pgfqpoint{2.223097in}{1.753304in}}%
\pgfpathcurveto{\pgfqpoint{2.223097in}{1.761541in}}{\pgfqpoint{2.219825in}{1.769441in}}{\pgfqpoint{2.214001in}{1.775265in}}%
\pgfpathcurveto{\pgfqpoint{2.208177in}{1.781089in}}{\pgfqpoint{2.200277in}{1.784361in}}{\pgfqpoint{2.192041in}{1.784361in}}%
\pgfpathcurveto{\pgfqpoint{2.183805in}{1.784361in}}{\pgfqpoint{2.175905in}{1.781089in}}{\pgfqpoint{2.170081in}{1.775265in}}%
\pgfpathcurveto{\pgfqpoint{2.164257in}{1.769441in}}{\pgfqpoint{2.160984in}{1.761541in}}{\pgfqpoint{2.160984in}{1.753304in}}%
\pgfpathcurveto{\pgfqpoint{2.160984in}{1.745068in}}{\pgfqpoint{2.164257in}{1.737168in}}{\pgfqpoint{2.170081in}{1.731344in}}%
\pgfpathcurveto{\pgfqpoint{2.175905in}{1.725520in}}{\pgfqpoint{2.183805in}{1.722248in}}{\pgfqpoint{2.192041in}{1.722248in}}%
\pgfpathclose%
\pgfusepath{stroke,fill}%
\end{pgfscope}%
\begin{pgfscope}%
\pgfpathrectangle{\pgfqpoint{0.100000in}{0.212622in}}{\pgfqpoint{3.696000in}{3.696000in}}%
\pgfusepath{clip}%
\pgfsetbuttcap%
\pgfsetroundjoin%
\definecolor{currentfill}{rgb}{0.121569,0.466667,0.705882}%
\pgfsetfillcolor{currentfill}%
\pgfsetfillopacity{0.717188}%
\pgfsetlinewidth{1.003750pt}%
\definecolor{currentstroke}{rgb}{0.121569,0.466667,0.705882}%
\pgfsetstrokecolor{currentstroke}%
\pgfsetstrokeopacity{0.717188}%
\pgfsetdash{}{0pt}%
\pgfpathmoveto{\pgfqpoint{1.148020in}{1.201512in}}%
\pgfpathcurveto{\pgfqpoint{1.156256in}{1.201512in}}{\pgfqpoint{1.164156in}{1.204785in}}{\pgfqpoint{1.169980in}{1.210608in}}%
\pgfpathcurveto{\pgfqpoint{1.175804in}{1.216432in}}{\pgfqpoint{1.179076in}{1.224332in}}{\pgfqpoint{1.179076in}{1.232569in}}%
\pgfpathcurveto{\pgfqpoint{1.179076in}{1.240805in}}{\pgfqpoint{1.175804in}{1.248705in}}{\pgfqpoint{1.169980in}{1.254529in}}%
\pgfpathcurveto{\pgfqpoint{1.164156in}{1.260353in}}{\pgfqpoint{1.156256in}{1.263625in}}{\pgfqpoint{1.148020in}{1.263625in}}%
\pgfpathcurveto{\pgfqpoint{1.139783in}{1.263625in}}{\pgfqpoint{1.131883in}{1.260353in}}{\pgfqpoint{1.126059in}{1.254529in}}%
\pgfpathcurveto{\pgfqpoint{1.120235in}{1.248705in}}{\pgfqpoint{1.116963in}{1.240805in}}{\pgfqpoint{1.116963in}{1.232569in}}%
\pgfpathcurveto{\pgfqpoint{1.116963in}{1.224332in}}{\pgfqpoint{1.120235in}{1.216432in}}{\pgfqpoint{1.126059in}{1.210608in}}%
\pgfpathcurveto{\pgfqpoint{1.131883in}{1.204785in}}{\pgfqpoint{1.139783in}{1.201512in}}{\pgfqpoint{1.148020in}{1.201512in}}%
\pgfpathclose%
\pgfusepath{stroke,fill}%
\end{pgfscope}%
\begin{pgfscope}%
\pgfpathrectangle{\pgfqpoint{0.100000in}{0.212622in}}{\pgfqpoint{3.696000in}{3.696000in}}%
\pgfusepath{clip}%
\pgfsetbuttcap%
\pgfsetroundjoin%
\definecolor{currentfill}{rgb}{0.121569,0.466667,0.705882}%
\pgfsetfillcolor{currentfill}%
\pgfsetfillopacity{0.720302}%
\pgfsetlinewidth{1.003750pt}%
\definecolor{currentstroke}{rgb}{0.121569,0.466667,0.705882}%
\pgfsetstrokecolor{currentstroke}%
\pgfsetstrokeopacity{0.720302}%
\pgfsetdash{}{0pt}%
\pgfpathmoveto{\pgfqpoint{1.161315in}{1.200777in}}%
\pgfpathcurveto{\pgfqpoint{1.169551in}{1.200777in}}{\pgfqpoint{1.177451in}{1.204049in}}{\pgfqpoint{1.183275in}{1.209873in}}%
\pgfpathcurveto{\pgfqpoint{1.189099in}{1.215697in}}{\pgfqpoint{1.192371in}{1.223597in}}{\pgfqpoint{1.192371in}{1.231833in}}%
\pgfpathcurveto{\pgfqpoint{1.192371in}{1.240070in}}{\pgfqpoint{1.189099in}{1.247970in}}{\pgfqpoint{1.183275in}{1.253794in}}%
\pgfpathcurveto{\pgfqpoint{1.177451in}{1.259618in}}{\pgfqpoint{1.169551in}{1.262890in}}{\pgfqpoint{1.161315in}{1.262890in}}%
\pgfpathcurveto{\pgfqpoint{1.153079in}{1.262890in}}{\pgfqpoint{1.145179in}{1.259618in}}{\pgfqpoint{1.139355in}{1.253794in}}%
\pgfpathcurveto{\pgfqpoint{1.133531in}{1.247970in}}{\pgfqpoint{1.130258in}{1.240070in}}{\pgfqpoint{1.130258in}{1.231833in}}%
\pgfpathcurveto{\pgfqpoint{1.130258in}{1.223597in}}{\pgfqpoint{1.133531in}{1.215697in}}{\pgfqpoint{1.139355in}{1.209873in}}%
\pgfpathcurveto{\pgfqpoint{1.145179in}{1.204049in}}{\pgfqpoint{1.153079in}{1.200777in}}{\pgfqpoint{1.161315in}{1.200777in}}%
\pgfpathclose%
\pgfusepath{stroke,fill}%
\end{pgfscope}%
\begin{pgfscope}%
\pgfpathrectangle{\pgfqpoint{0.100000in}{0.212622in}}{\pgfqpoint{3.696000in}{3.696000in}}%
\pgfusepath{clip}%
\pgfsetbuttcap%
\pgfsetroundjoin%
\definecolor{currentfill}{rgb}{0.121569,0.466667,0.705882}%
\pgfsetfillcolor{currentfill}%
\pgfsetfillopacity{0.720664}%
\pgfsetlinewidth{1.003750pt}%
\definecolor{currentstroke}{rgb}{0.121569,0.466667,0.705882}%
\pgfsetstrokecolor{currentstroke}%
\pgfsetstrokeopacity{0.720664}%
\pgfsetdash{}{0pt}%
\pgfpathmoveto{\pgfqpoint{2.194796in}{1.710660in}}%
\pgfpathcurveto{\pgfqpoint{2.203032in}{1.710660in}}{\pgfqpoint{2.210932in}{1.713933in}}{\pgfqpoint{2.216756in}{1.719757in}}%
\pgfpathcurveto{\pgfqpoint{2.222580in}{1.725580in}}{\pgfqpoint{2.225852in}{1.733481in}}{\pgfqpoint{2.225852in}{1.741717in}}%
\pgfpathcurveto{\pgfqpoint{2.225852in}{1.749953in}}{\pgfqpoint{2.222580in}{1.757853in}}{\pgfqpoint{2.216756in}{1.763677in}}%
\pgfpathcurveto{\pgfqpoint{2.210932in}{1.769501in}}{\pgfqpoint{2.203032in}{1.772773in}}{\pgfqpoint{2.194796in}{1.772773in}}%
\pgfpathcurveto{\pgfqpoint{2.186559in}{1.772773in}}{\pgfqpoint{2.178659in}{1.769501in}}{\pgfqpoint{2.172835in}{1.763677in}}%
\pgfpathcurveto{\pgfqpoint{2.167011in}{1.757853in}}{\pgfqpoint{2.163739in}{1.749953in}}{\pgfqpoint{2.163739in}{1.741717in}}%
\pgfpathcurveto{\pgfqpoint{2.163739in}{1.733481in}}{\pgfqpoint{2.167011in}{1.725580in}}{\pgfqpoint{2.172835in}{1.719757in}}%
\pgfpathcurveto{\pgfqpoint{2.178659in}{1.713933in}}{\pgfqpoint{2.186559in}{1.710660in}}{\pgfqpoint{2.194796in}{1.710660in}}%
\pgfpathclose%
\pgfusepath{stroke,fill}%
\end{pgfscope}%
\begin{pgfscope}%
\pgfpathrectangle{\pgfqpoint{0.100000in}{0.212622in}}{\pgfqpoint{3.696000in}{3.696000in}}%
\pgfusepath{clip}%
\pgfsetbuttcap%
\pgfsetroundjoin%
\definecolor{currentfill}{rgb}{0.121569,0.466667,0.705882}%
\pgfsetfillcolor{currentfill}%
\pgfsetfillopacity{0.722903}%
\pgfsetlinewidth{1.003750pt}%
\definecolor{currentstroke}{rgb}{0.121569,0.466667,0.705882}%
\pgfsetstrokecolor{currentstroke}%
\pgfsetstrokeopacity{0.722903}%
\pgfsetdash{}{0pt}%
\pgfpathmoveto{\pgfqpoint{1.173821in}{1.199850in}}%
\pgfpathcurveto{\pgfqpoint{1.182057in}{1.199850in}}{\pgfqpoint{1.189957in}{1.203122in}}{\pgfqpoint{1.195781in}{1.208946in}}%
\pgfpathcurveto{\pgfqpoint{1.201605in}{1.214770in}}{\pgfqpoint{1.204878in}{1.222670in}}{\pgfqpoint{1.204878in}{1.230906in}}%
\pgfpathcurveto{\pgfqpoint{1.204878in}{1.239143in}}{\pgfqpoint{1.201605in}{1.247043in}}{\pgfqpoint{1.195781in}{1.252867in}}%
\pgfpathcurveto{\pgfqpoint{1.189957in}{1.258691in}}{\pgfqpoint{1.182057in}{1.261963in}}{\pgfqpoint{1.173821in}{1.261963in}}%
\pgfpathcurveto{\pgfqpoint{1.165585in}{1.261963in}}{\pgfqpoint{1.157685in}{1.258691in}}{\pgfqpoint{1.151861in}{1.252867in}}%
\pgfpathcurveto{\pgfqpoint{1.146037in}{1.247043in}}{\pgfqpoint{1.142765in}{1.239143in}}{\pgfqpoint{1.142765in}{1.230906in}}%
\pgfpathcurveto{\pgfqpoint{1.142765in}{1.222670in}}{\pgfqpoint{1.146037in}{1.214770in}}{\pgfqpoint{1.151861in}{1.208946in}}%
\pgfpathcurveto{\pgfqpoint{1.157685in}{1.203122in}}{\pgfqpoint{1.165585in}{1.199850in}}{\pgfqpoint{1.173821in}{1.199850in}}%
\pgfpathclose%
\pgfusepath{stroke,fill}%
\end{pgfscope}%
\begin{pgfscope}%
\pgfpathrectangle{\pgfqpoint{0.100000in}{0.212622in}}{\pgfqpoint{3.696000in}{3.696000in}}%
\pgfusepath{clip}%
\pgfsetbuttcap%
\pgfsetroundjoin%
\definecolor{currentfill}{rgb}{0.121569,0.466667,0.705882}%
\pgfsetfillcolor{currentfill}%
\pgfsetfillopacity{0.724756}%
\pgfsetlinewidth{1.003750pt}%
\definecolor{currentstroke}{rgb}{0.121569,0.466667,0.705882}%
\pgfsetstrokecolor{currentstroke}%
\pgfsetstrokeopacity{0.724756}%
\pgfsetdash{}{0pt}%
\pgfpathmoveto{\pgfqpoint{2.197416in}{1.698429in}}%
\pgfpathcurveto{\pgfqpoint{2.205652in}{1.698429in}}{\pgfqpoint{2.213552in}{1.701701in}}{\pgfqpoint{2.219376in}{1.707525in}}%
\pgfpathcurveto{\pgfqpoint{2.225200in}{1.713349in}}{\pgfqpoint{2.228472in}{1.721249in}}{\pgfqpoint{2.228472in}{1.729485in}}%
\pgfpathcurveto{\pgfqpoint{2.228472in}{1.737721in}}{\pgfqpoint{2.225200in}{1.745622in}}{\pgfqpoint{2.219376in}{1.751445in}}%
\pgfpathcurveto{\pgfqpoint{2.213552in}{1.757269in}}{\pgfqpoint{2.205652in}{1.760542in}}{\pgfqpoint{2.197416in}{1.760542in}}%
\pgfpathcurveto{\pgfqpoint{2.189179in}{1.760542in}}{\pgfqpoint{2.181279in}{1.757269in}}{\pgfqpoint{2.175455in}{1.751445in}}%
\pgfpathcurveto{\pgfqpoint{2.169632in}{1.745622in}}{\pgfqpoint{2.166359in}{1.737721in}}{\pgfqpoint{2.166359in}{1.729485in}}%
\pgfpathcurveto{\pgfqpoint{2.166359in}{1.721249in}}{\pgfqpoint{2.169632in}{1.713349in}}{\pgfqpoint{2.175455in}{1.707525in}}%
\pgfpathcurveto{\pgfqpoint{2.181279in}{1.701701in}}{\pgfqpoint{2.189179in}{1.698429in}}{\pgfqpoint{2.197416in}{1.698429in}}%
\pgfpathclose%
\pgfusepath{stroke,fill}%
\end{pgfscope}%
\begin{pgfscope}%
\pgfpathrectangle{\pgfqpoint{0.100000in}{0.212622in}}{\pgfqpoint{3.696000in}{3.696000in}}%
\pgfusepath{clip}%
\pgfsetbuttcap%
\pgfsetroundjoin%
\definecolor{currentfill}{rgb}{0.121569,0.466667,0.705882}%
\pgfsetfillcolor{currentfill}%
\pgfsetfillopacity{0.725474}%
\pgfsetlinewidth{1.003750pt}%
\definecolor{currentstroke}{rgb}{0.121569,0.466667,0.705882}%
\pgfsetstrokecolor{currentstroke}%
\pgfsetstrokeopacity{0.725474}%
\pgfsetdash{}{0pt}%
\pgfpathmoveto{\pgfqpoint{1.186008in}{1.198832in}}%
\pgfpathcurveto{\pgfqpoint{1.194244in}{1.198832in}}{\pgfqpoint{1.202144in}{1.202104in}}{\pgfqpoint{1.207968in}{1.207928in}}%
\pgfpathcurveto{\pgfqpoint{1.213792in}{1.213752in}}{\pgfqpoint{1.217064in}{1.221652in}}{\pgfqpoint{1.217064in}{1.229888in}}%
\pgfpathcurveto{\pgfqpoint{1.217064in}{1.238124in}}{\pgfqpoint{1.213792in}{1.246024in}}{\pgfqpoint{1.207968in}{1.251848in}}%
\pgfpathcurveto{\pgfqpoint{1.202144in}{1.257672in}}{\pgfqpoint{1.194244in}{1.260945in}}{\pgfqpoint{1.186008in}{1.260945in}}%
\pgfpathcurveto{\pgfqpoint{1.177771in}{1.260945in}}{\pgfqpoint{1.169871in}{1.257672in}}{\pgfqpoint{1.164047in}{1.251848in}}%
\pgfpathcurveto{\pgfqpoint{1.158223in}{1.246024in}}{\pgfqpoint{1.154951in}{1.238124in}}{\pgfqpoint{1.154951in}{1.229888in}}%
\pgfpathcurveto{\pgfqpoint{1.154951in}{1.221652in}}{\pgfqpoint{1.158223in}{1.213752in}}{\pgfqpoint{1.164047in}{1.207928in}}%
\pgfpathcurveto{\pgfqpoint{1.169871in}{1.202104in}}{\pgfqpoint{1.177771in}{1.198832in}}{\pgfqpoint{1.186008in}{1.198832in}}%
\pgfpathclose%
\pgfusepath{stroke,fill}%
\end{pgfscope}%
\begin{pgfscope}%
\pgfpathrectangle{\pgfqpoint{0.100000in}{0.212622in}}{\pgfqpoint{3.696000in}{3.696000in}}%
\pgfusepath{clip}%
\pgfsetbuttcap%
\pgfsetroundjoin%
\definecolor{currentfill}{rgb}{0.121569,0.466667,0.705882}%
\pgfsetfillcolor{currentfill}%
\pgfsetfillopacity{0.727857}%
\pgfsetlinewidth{1.003750pt}%
\definecolor{currentstroke}{rgb}{0.121569,0.466667,0.705882}%
\pgfsetstrokecolor{currentstroke}%
\pgfsetstrokeopacity{0.727857}%
\pgfsetdash{}{0pt}%
\pgfpathmoveto{\pgfqpoint{1.196114in}{1.197771in}}%
\pgfpathcurveto{\pgfqpoint{1.204350in}{1.197771in}}{\pgfqpoint{1.212250in}{1.201043in}}{\pgfqpoint{1.218074in}{1.206867in}}%
\pgfpathcurveto{\pgfqpoint{1.223898in}{1.212691in}}{\pgfqpoint{1.227170in}{1.220591in}}{\pgfqpoint{1.227170in}{1.228828in}}%
\pgfpathcurveto{\pgfqpoint{1.227170in}{1.237064in}}{\pgfqpoint{1.223898in}{1.244964in}}{\pgfqpoint{1.218074in}{1.250788in}}%
\pgfpathcurveto{\pgfqpoint{1.212250in}{1.256612in}}{\pgfqpoint{1.204350in}{1.259884in}}{\pgfqpoint{1.196114in}{1.259884in}}%
\pgfpathcurveto{\pgfqpoint{1.187878in}{1.259884in}}{\pgfqpoint{1.179977in}{1.256612in}}{\pgfqpoint{1.174154in}{1.250788in}}%
\pgfpathcurveto{\pgfqpoint{1.168330in}{1.244964in}}{\pgfqpoint{1.165057in}{1.237064in}}{\pgfqpoint{1.165057in}{1.228828in}}%
\pgfpathcurveto{\pgfqpoint{1.165057in}{1.220591in}}{\pgfqpoint{1.168330in}{1.212691in}}{\pgfqpoint{1.174154in}{1.206867in}}%
\pgfpathcurveto{\pgfqpoint{1.179977in}{1.201043in}}{\pgfqpoint{1.187878in}{1.197771in}}{\pgfqpoint{1.196114in}{1.197771in}}%
\pgfpathclose%
\pgfusepath{stroke,fill}%
\end{pgfscope}%
\begin{pgfscope}%
\pgfpathrectangle{\pgfqpoint{0.100000in}{0.212622in}}{\pgfqpoint{3.696000in}{3.696000in}}%
\pgfusepath{clip}%
\pgfsetbuttcap%
\pgfsetroundjoin%
\definecolor{currentfill}{rgb}{0.121569,0.466667,0.705882}%
\pgfsetfillcolor{currentfill}%
\pgfsetfillopacity{0.728950}%
\pgfsetlinewidth{1.003750pt}%
\definecolor{currentstroke}{rgb}{0.121569,0.466667,0.705882}%
\pgfsetstrokecolor{currentstroke}%
\pgfsetstrokeopacity{0.728950}%
\pgfsetdash{}{0pt}%
\pgfpathmoveto{\pgfqpoint{2.201585in}{1.684440in}}%
\pgfpathcurveto{\pgfqpoint{2.209822in}{1.684440in}}{\pgfqpoint{2.217722in}{1.687712in}}{\pgfqpoint{2.223546in}{1.693536in}}%
\pgfpathcurveto{\pgfqpoint{2.229370in}{1.699360in}}{\pgfqpoint{2.232642in}{1.707260in}}{\pgfqpoint{2.232642in}{1.715497in}}%
\pgfpathcurveto{\pgfqpoint{2.232642in}{1.723733in}}{\pgfqpoint{2.229370in}{1.731633in}}{\pgfqpoint{2.223546in}{1.737457in}}%
\pgfpathcurveto{\pgfqpoint{2.217722in}{1.743281in}}{\pgfqpoint{2.209822in}{1.746553in}}{\pgfqpoint{2.201585in}{1.746553in}}%
\pgfpathcurveto{\pgfqpoint{2.193349in}{1.746553in}}{\pgfqpoint{2.185449in}{1.743281in}}{\pgfqpoint{2.179625in}{1.737457in}}%
\pgfpathcurveto{\pgfqpoint{2.173801in}{1.731633in}}{\pgfqpoint{2.170529in}{1.723733in}}{\pgfqpoint{2.170529in}{1.715497in}}%
\pgfpathcurveto{\pgfqpoint{2.170529in}{1.707260in}}{\pgfqpoint{2.173801in}{1.699360in}}{\pgfqpoint{2.179625in}{1.693536in}}%
\pgfpathcurveto{\pgfqpoint{2.185449in}{1.687712in}}{\pgfqpoint{2.193349in}{1.684440in}}{\pgfqpoint{2.201585in}{1.684440in}}%
\pgfpathclose%
\pgfusepath{stroke,fill}%
\end{pgfscope}%
\begin{pgfscope}%
\pgfpathrectangle{\pgfqpoint{0.100000in}{0.212622in}}{\pgfqpoint{3.696000in}{3.696000in}}%
\pgfusepath{clip}%
\pgfsetbuttcap%
\pgfsetroundjoin%
\definecolor{currentfill}{rgb}{0.121569,0.466667,0.705882}%
\pgfsetfillcolor{currentfill}%
\pgfsetfillopacity{0.730092}%
\pgfsetlinewidth{1.003750pt}%
\definecolor{currentstroke}{rgb}{0.121569,0.466667,0.705882}%
\pgfsetstrokecolor{currentstroke}%
\pgfsetstrokeopacity{0.730092}%
\pgfsetdash{}{0pt}%
\pgfpathmoveto{\pgfqpoint{1.205610in}{1.196871in}}%
\pgfpathcurveto{\pgfqpoint{1.213846in}{1.196871in}}{\pgfqpoint{1.221747in}{1.200143in}}{\pgfqpoint{1.227570in}{1.205967in}}%
\pgfpathcurveto{\pgfqpoint{1.233394in}{1.211791in}}{\pgfqpoint{1.236667in}{1.219691in}}{\pgfqpoint{1.236667in}{1.227927in}}%
\pgfpathcurveto{\pgfqpoint{1.236667in}{1.236163in}}{\pgfqpoint{1.233394in}{1.244064in}}{\pgfqpoint{1.227570in}{1.249887in}}%
\pgfpathcurveto{\pgfqpoint{1.221747in}{1.255711in}}{\pgfqpoint{1.213846in}{1.258984in}}{\pgfqpoint{1.205610in}{1.258984in}}%
\pgfpathcurveto{\pgfqpoint{1.197374in}{1.258984in}}{\pgfqpoint{1.189474in}{1.255711in}}{\pgfqpoint{1.183650in}{1.249887in}}%
\pgfpathcurveto{\pgfqpoint{1.177826in}{1.244064in}}{\pgfqpoint{1.174554in}{1.236163in}}{\pgfqpoint{1.174554in}{1.227927in}}%
\pgfpathcurveto{\pgfqpoint{1.174554in}{1.219691in}}{\pgfqpoint{1.177826in}{1.211791in}}{\pgfqpoint{1.183650in}{1.205967in}}%
\pgfpathcurveto{\pgfqpoint{1.189474in}{1.200143in}}{\pgfqpoint{1.197374in}{1.196871in}}{\pgfqpoint{1.205610in}{1.196871in}}%
\pgfpathclose%
\pgfusepath{stroke,fill}%
\end{pgfscope}%
\begin{pgfscope}%
\pgfpathrectangle{\pgfqpoint{0.100000in}{0.212622in}}{\pgfqpoint{3.696000in}{3.696000in}}%
\pgfusepath{clip}%
\pgfsetbuttcap%
\pgfsetroundjoin%
\definecolor{currentfill}{rgb}{0.121569,0.466667,0.705882}%
\pgfsetfillcolor{currentfill}%
\pgfsetfillopacity{0.732085}%
\pgfsetlinewidth{1.003750pt}%
\definecolor{currentstroke}{rgb}{0.121569,0.466667,0.705882}%
\pgfsetstrokecolor{currentstroke}%
\pgfsetstrokeopacity{0.732085}%
\pgfsetdash{}{0pt}%
\pgfpathmoveto{\pgfqpoint{1.214166in}{1.196379in}}%
\pgfpathcurveto{\pgfqpoint{1.222402in}{1.196379in}}{\pgfqpoint{1.230302in}{1.199652in}}{\pgfqpoint{1.236126in}{1.205476in}}%
\pgfpathcurveto{\pgfqpoint{1.241950in}{1.211300in}}{\pgfqpoint{1.245222in}{1.219200in}}{\pgfqpoint{1.245222in}{1.227436in}}%
\pgfpathcurveto{\pgfqpoint{1.245222in}{1.235672in}}{\pgfqpoint{1.241950in}{1.243572in}}{\pgfqpoint{1.236126in}{1.249396in}}%
\pgfpathcurveto{\pgfqpoint{1.230302in}{1.255220in}}{\pgfqpoint{1.222402in}{1.258492in}}{\pgfqpoint{1.214166in}{1.258492in}}%
\pgfpathcurveto{\pgfqpoint{1.205929in}{1.258492in}}{\pgfqpoint{1.198029in}{1.255220in}}{\pgfqpoint{1.192206in}{1.249396in}}%
\pgfpathcurveto{\pgfqpoint{1.186382in}{1.243572in}}{\pgfqpoint{1.183109in}{1.235672in}}{\pgfqpoint{1.183109in}{1.227436in}}%
\pgfpathcurveto{\pgfqpoint{1.183109in}{1.219200in}}{\pgfqpoint{1.186382in}{1.211300in}}{\pgfqpoint{1.192206in}{1.205476in}}%
\pgfpathcurveto{\pgfqpoint{1.198029in}{1.199652in}}{\pgfqpoint{1.205929in}{1.196379in}}{\pgfqpoint{1.214166in}{1.196379in}}%
\pgfpathclose%
\pgfusepath{stroke,fill}%
\end{pgfscope}%
\begin{pgfscope}%
\pgfpathrectangle{\pgfqpoint{0.100000in}{0.212622in}}{\pgfqpoint{3.696000in}{3.696000in}}%
\pgfusepath{clip}%
\pgfsetbuttcap%
\pgfsetroundjoin%
\definecolor{currentfill}{rgb}{0.121569,0.466667,0.705882}%
\pgfsetfillcolor{currentfill}%
\pgfsetfillopacity{0.733325}%
\pgfsetlinewidth{1.003750pt}%
\definecolor{currentstroke}{rgb}{0.121569,0.466667,0.705882}%
\pgfsetstrokecolor{currentstroke}%
\pgfsetstrokeopacity{0.733325}%
\pgfsetdash{}{0pt}%
\pgfpathmoveto{\pgfqpoint{2.205439in}{1.669924in}}%
\pgfpathcurveto{\pgfqpoint{2.213675in}{1.669924in}}{\pgfqpoint{2.221575in}{1.673196in}}{\pgfqpoint{2.227399in}{1.679020in}}%
\pgfpathcurveto{\pgfqpoint{2.233223in}{1.684844in}}{\pgfqpoint{2.236495in}{1.692744in}}{\pgfqpoint{2.236495in}{1.700980in}}%
\pgfpathcurveto{\pgfqpoint{2.236495in}{1.709217in}}{\pgfqpoint{2.233223in}{1.717117in}}{\pgfqpoint{2.227399in}{1.722941in}}%
\pgfpathcurveto{\pgfqpoint{2.221575in}{1.728765in}}{\pgfqpoint{2.213675in}{1.732037in}}{\pgfqpoint{2.205439in}{1.732037in}}%
\pgfpathcurveto{\pgfqpoint{2.197202in}{1.732037in}}{\pgfqpoint{2.189302in}{1.728765in}}{\pgfqpoint{2.183478in}{1.722941in}}%
\pgfpathcurveto{\pgfqpoint{2.177654in}{1.717117in}}{\pgfqpoint{2.174382in}{1.709217in}}{\pgfqpoint{2.174382in}{1.700980in}}%
\pgfpathcurveto{\pgfqpoint{2.174382in}{1.692744in}}{\pgfqpoint{2.177654in}{1.684844in}}{\pgfqpoint{2.183478in}{1.679020in}}%
\pgfpathcurveto{\pgfqpoint{2.189302in}{1.673196in}}{\pgfqpoint{2.197202in}{1.669924in}}{\pgfqpoint{2.205439in}{1.669924in}}%
\pgfpathclose%
\pgfusepath{stroke,fill}%
\end{pgfscope}%
\begin{pgfscope}%
\pgfpathrectangle{\pgfqpoint{0.100000in}{0.212622in}}{\pgfqpoint{3.696000in}{3.696000in}}%
\pgfusepath{clip}%
\pgfsetbuttcap%
\pgfsetroundjoin%
\definecolor{currentfill}{rgb}{0.121569,0.466667,0.705882}%
\pgfsetfillcolor{currentfill}%
\pgfsetfillopacity{0.733901}%
\pgfsetlinewidth{1.003750pt}%
\definecolor{currentstroke}{rgb}{0.121569,0.466667,0.705882}%
\pgfsetstrokecolor{currentstroke}%
\pgfsetstrokeopacity{0.733901}%
\pgfsetdash{}{0pt}%
\pgfpathmoveto{\pgfqpoint{1.221409in}{1.195551in}}%
\pgfpathcurveto{\pgfqpoint{1.229645in}{1.195551in}}{\pgfqpoint{1.237545in}{1.198823in}}{\pgfqpoint{1.243369in}{1.204647in}}%
\pgfpathcurveto{\pgfqpoint{1.249193in}{1.210471in}}{\pgfqpoint{1.252466in}{1.218371in}}{\pgfqpoint{1.252466in}{1.226608in}}%
\pgfpathcurveto{\pgfqpoint{1.252466in}{1.234844in}}{\pgfqpoint{1.249193in}{1.242744in}}{\pgfqpoint{1.243369in}{1.248568in}}%
\pgfpathcurveto{\pgfqpoint{1.237545in}{1.254392in}}{\pgfqpoint{1.229645in}{1.257664in}}{\pgfqpoint{1.221409in}{1.257664in}}%
\pgfpathcurveto{\pgfqpoint{1.213173in}{1.257664in}}{\pgfqpoint{1.205273in}{1.254392in}}{\pgfqpoint{1.199449in}{1.248568in}}%
\pgfpathcurveto{\pgfqpoint{1.193625in}{1.242744in}}{\pgfqpoint{1.190353in}{1.234844in}}{\pgfqpoint{1.190353in}{1.226608in}}%
\pgfpathcurveto{\pgfqpoint{1.190353in}{1.218371in}}{\pgfqpoint{1.193625in}{1.210471in}}{\pgfqpoint{1.199449in}{1.204647in}}%
\pgfpathcurveto{\pgfqpoint{1.205273in}{1.198823in}}{\pgfqpoint{1.213173in}{1.195551in}}{\pgfqpoint{1.221409in}{1.195551in}}%
\pgfpathclose%
\pgfusepath{stroke,fill}%
\end{pgfscope}%
\begin{pgfscope}%
\pgfpathrectangle{\pgfqpoint{0.100000in}{0.212622in}}{\pgfqpoint{3.696000in}{3.696000in}}%
\pgfusepath{clip}%
\pgfsetbuttcap%
\pgfsetroundjoin%
\definecolor{currentfill}{rgb}{0.121569,0.466667,0.705882}%
\pgfsetfillcolor{currentfill}%
\pgfsetfillopacity{0.737562}%
\pgfsetlinewidth{1.003750pt}%
\definecolor{currentstroke}{rgb}{0.121569,0.466667,0.705882}%
\pgfsetstrokecolor{currentstroke}%
\pgfsetstrokeopacity{0.737562}%
\pgfsetdash{}{0pt}%
\pgfpathmoveto{\pgfqpoint{1.234508in}{1.195163in}}%
\pgfpathcurveto{\pgfqpoint{1.242744in}{1.195163in}}{\pgfqpoint{1.250644in}{1.198435in}}{\pgfqpoint{1.256468in}{1.204259in}}%
\pgfpathcurveto{\pgfqpoint{1.262292in}{1.210083in}}{\pgfqpoint{1.265564in}{1.217983in}}{\pgfqpoint{1.265564in}{1.226220in}}%
\pgfpathcurveto{\pgfqpoint{1.265564in}{1.234456in}}{\pgfqpoint{1.262292in}{1.242356in}}{\pgfqpoint{1.256468in}{1.248180in}}%
\pgfpathcurveto{\pgfqpoint{1.250644in}{1.254004in}}{\pgfqpoint{1.242744in}{1.257276in}}{\pgfqpoint{1.234508in}{1.257276in}}%
\pgfpathcurveto{\pgfqpoint{1.226272in}{1.257276in}}{\pgfqpoint{1.218371in}{1.254004in}}{\pgfqpoint{1.212548in}{1.248180in}}%
\pgfpathcurveto{\pgfqpoint{1.206724in}{1.242356in}}{\pgfqpoint{1.203451in}{1.234456in}}{\pgfqpoint{1.203451in}{1.226220in}}%
\pgfpathcurveto{\pgfqpoint{1.203451in}{1.217983in}}{\pgfqpoint{1.206724in}{1.210083in}}{\pgfqpoint{1.212548in}{1.204259in}}%
\pgfpathcurveto{\pgfqpoint{1.218371in}{1.198435in}}{\pgfqpoint{1.226272in}{1.195163in}}{\pgfqpoint{1.234508in}{1.195163in}}%
\pgfpathclose%
\pgfusepath{stroke,fill}%
\end{pgfscope}%
\begin{pgfscope}%
\pgfpathrectangle{\pgfqpoint{0.100000in}{0.212622in}}{\pgfqpoint{3.696000in}{3.696000in}}%
\pgfusepath{clip}%
\pgfsetbuttcap%
\pgfsetroundjoin%
\definecolor{currentfill}{rgb}{0.121569,0.466667,0.705882}%
\pgfsetfillcolor{currentfill}%
\pgfsetfillopacity{0.738216}%
\pgfsetlinewidth{1.003750pt}%
\definecolor{currentstroke}{rgb}{0.121569,0.466667,0.705882}%
\pgfsetstrokecolor{currentstroke}%
\pgfsetstrokeopacity{0.738216}%
\pgfsetdash{}{0pt}%
\pgfpathmoveto{\pgfqpoint{2.208686in}{1.655630in}}%
\pgfpathcurveto{\pgfqpoint{2.216922in}{1.655630in}}{\pgfqpoint{2.224822in}{1.658902in}}{\pgfqpoint{2.230646in}{1.664726in}}%
\pgfpathcurveto{\pgfqpoint{2.236470in}{1.670550in}}{\pgfqpoint{2.239743in}{1.678450in}}{\pgfqpoint{2.239743in}{1.686686in}}%
\pgfpathcurveto{\pgfqpoint{2.239743in}{1.694922in}}{\pgfqpoint{2.236470in}{1.702822in}}{\pgfqpoint{2.230646in}{1.708646in}}%
\pgfpathcurveto{\pgfqpoint{2.224822in}{1.714470in}}{\pgfqpoint{2.216922in}{1.717743in}}{\pgfqpoint{2.208686in}{1.717743in}}%
\pgfpathcurveto{\pgfqpoint{2.200450in}{1.717743in}}{\pgfqpoint{2.192550in}{1.714470in}}{\pgfqpoint{2.186726in}{1.708646in}}%
\pgfpathcurveto{\pgfqpoint{2.180902in}{1.702822in}}{\pgfqpoint{2.177630in}{1.694922in}}{\pgfqpoint{2.177630in}{1.686686in}}%
\pgfpathcurveto{\pgfqpoint{2.177630in}{1.678450in}}{\pgfqpoint{2.180902in}{1.670550in}}{\pgfqpoint{2.186726in}{1.664726in}}%
\pgfpathcurveto{\pgfqpoint{2.192550in}{1.658902in}}{\pgfqpoint{2.200450in}{1.655630in}}{\pgfqpoint{2.208686in}{1.655630in}}%
\pgfpathclose%
\pgfusepath{stroke,fill}%
\end{pgfscope}%
\begin{pgfscope}%
\pgfpathrectangle{\pgfqpoint{0.100000in}{0.212622in}}{\pgfqpoint{3.696000in}{3.696000in}}%
\pgfusepath{clip}%
\pgfsetbuttcap%
\pgfsetroundjoin%
\definecolor{currentfill}{rgb}{0.121569,0.466667,0.705882}%
\pgfsetfillcolor{currentfill}%
\pgfsetfillopacity{0.740343}%
\pgfsetlinewidth{1.003750pt}%
\definecolor{currentstroke}{rgb}{0.121569,0.466667,0.705882}%
\pgfsetstrokecolor{currentstroke}%
\pgfsetstrokeopacity{0.740343}%
\pgfsetdash{}{0pt}%
\pgfpathmoveto{\pgfqpoint{1.245449in}{1.195404in}}%
\pgfpathcurveto{\pgfqpoint{1.253685in}{1.195404in}}{\pgfqpoint{1.261585in}{1.198676in}}{\pgfqpoint{1.267409in}{1.204500in}}%
\pgfpathcurveto{\pgfqpoint{1.273233in}{1.210324in}}{\pgfqpoint{1.276505in}{1.218224in}}{\pgfqpoint{1.276505in}{1.226460in}}%
\pgfpathcurveto{\pgfqpoint{1.276505in}{1.234697in}}{\pgfqpoint{1.273233in}{1.242597in}}{\pgfqpoint{1.267409in}{1.248421in}}%
\pgfpathcurveto{\pgfqpoint{1.261585in}{1.254245in}}{\pgfqpoint{1.253685in}{1.257517in}}{\pgfqpoint{1.245449in}{1.257517in}}%
\pgfpathcurveto{\pgfqpoint{1.237212in}{1.257517in}}{\pgfqpoint{1.229312in}{1.254245in}}{\pgfqpoint{1.223488in}{1.248421in}}%
\pgfpathcurveto{\pgfqpoint{1.217665in}{1.242597in}}{\pgfqpoint{1.214392in}{1.234697in}}{\pgfqpoint{1.214392in}{1.226460in}}%
\pgfpathcurveto{\pgfqpoint{1.214392in}{1.218224in}}{\pgfqpoint{1.217665in}{1.210324in}}{\pgfqpoint{1.223488in}{1.204500in}}%
\pgfpathcurveto{\pgfqpoint{1.229312in}{1.198676in}}{\pgfqpoint{1.237212in}{1.195404in}}{\pgfqpoint{1.245449in}{1.195404in}}%
\pgfpathclose%
\pgfusepath{stroke,fill}%
\end{pgfscope}%
\begin{pgfscope}%
\pgfpathrectangle{\pgfqpoint{0.100000in}{0.212622in}}{\pgfqpoint{3.696000in}{3.696000in}}%
\pgfusepath{clip}%
\pgfsetbuttcap%
\pgfsetroundjoin%
\definecolor{currentfill}{rgb}{0.121569,0.466667,0.705882}%
\pgfsetfillcolor{currentfill}%
\pgfsetfillopacity{0.742980}%
\pgfsetlinewidth{1.003750pt}%
\definecolor{currentstroke}{rgb}{0.121569,0.466667,0.705882}%
\pgfsetstrokecolor{currentstroke}%
\pgfsetstrokeopacity{0.742980}%
\pgfsetdash{}{0pt}%
\pgfpathmoveto{\pgfqpoint{1.255780in}{1.195869in}}%
\pgfpathcurveto{\pgfqpoint{1.264016in}{1.195869in}}{\pgfqpoint{1.271917in}{1.199141in}}{\pgfqpoint{1.277740in}{1.204965in}}%
\pgfpathcurveto{\pgfqpoint{1.283564in}{1.210789in}}{\pgfqpoint{1.286837in}{1.218689in}}{\pgfqpoint{1.286837in}{1.226925in}}%
\pgfpathcurveto{\pgfqpoint{1.286837in}{1.235162in}}{\pgfqpoint{1.283564in}{1.243062in}}{\pgfqpoint{1.277740in}{1.248886in}}%
\pgfpathcurveto{\pgfqpoint{1.271917in}{1.254710in}}{\pgfqpoint{1.264016in}{1.257982in}}{\pgfqpoint{1.255780in}{1.257982in}}%
\pgfpathcurveto{\pgfqpoint{1.247544in}{1.257982in}}{\pgfqpoint{1.239644in}{1.254710in}}{\pgfqpoint{1.233820in}{1.248886in}}%
\pgfpathcurveto{\pgfqpoint{1.227996in}{1.243062in}}{\pgfqpoint{1.224724in}{1.235162in}}{\pgfqpoint{1.224724in}{1.226925in}}%
\pgfpathcurveto{\pgfqpoint{1.224724in}{1.218689in}}{\pgfqpoint{1.227996in}{1.210789in}}{\pgfqpoint{1.233820in}{1.204965in}}%
\pgfpathcurveto{\pgfqpoint{1.239644in}{1.199141in}}{\pgfqpoint{1.247544in}{1.195869in}}{\pgfqpoint{1.255780in}{1.195869in}}%
\pgfpathclose%
\pgfusepath{stroke,fill}%
\end{pgfscope}%
\begin{pgfscope}%
\pgfpathrectangle{\pgfqpoint{0.100000in}{0.212622in}}{\pgfqpoint{3.696000in}{3.696000in}}%
\pgfusepath{clip}%
\pgfsetbuttcap%
\pgfsetroundjoin%
\definecolor{currentfill}{rgb}{0.121569,0.466667,0.705882}%
\pgfsetfillcolor{currentfill}%
\pgfsetfillopacity{0.743222}%
\pgfsetlinewidth{1.003750pt}%
\definecolor{currentstroke}{rgb}{0.121569,0.466667,0.705882}%
\pgfsetstrokecolor{currentstroke}%
\pgfsetstrokeopacity{0.743222}%
\pgfsetdash{}{0pt}%
\pgfpathmoveto{\pgfqpoint{2.213273in}{1.639627in}}%
\pgfpathcurveto{\pgfqpoint{2.221510in}{1.639627in}}{\pgfqpoint{2.229410in}{1.642899in}}{\pgfqpoint{2.235234in}{1.648723in}}%
\pgfpathcurveto{\pgfqpoint{2.241058in}{1.654547in}}{\pgfqpoint{2.244330in}{1.662447in}}{\pgfqpoint{2.244330in}{1.670683in}}%
\pgfpathcurveto{\pgfqpoint{2.244330in}{1.678920in}}{\pgfqpoint{2.241058in}{1.686820in}}{\pgfqpoint{2.235234in}{1.692644in}}%
\pgfpathcurveto{\pgfqpoint{2.229410in}{1.698468in}}{\pgfqpoint{2.221510in}{1.701740in}}{\pgfqpoint{2.213273in}{1.701740in}}%
\pgfpathcurveto{\pgfqpoint{2.205037in}{1.701740in}}{\pgfqpoint{2.197137in}{1.698468in}}{\pgfqpoint{2.191313in}{1.692644in}}%
\pgfpathcurveto{\pgfqpoint{2.185489in}{1.686820in}}{\pgfqpoint{2.182217in}{1.678920in}}{\pgfqpoint{2.182217in}{1.670683in}}%
\pgfpathcurveto{\pgfqpoint{2.182217in}{1.662447in}}{\pgfqpoint{2.185489in}{1.654547in}}{\pgfqpoint{2.191313in}{1.648723in}}%
\pgfpathcurveto{\pgfqpoint{2.197137in}{1.642899in}}{\pgfqpoint{2.205037in}{1.639627in}}{\pgfqpoint{2.213273in}{1.639627in}}%
\pgfpathclose%
\pgfusepath{stroke,fill}%
\end{pgfscope}%
\begin{pgfscope}%
\pgfpathrectangle{\pgfqpoint{0.100000in}{0.212622in}}{\pgfqpoint{3.696000in}{3.696000in}}%
\pgfusepath{clip}%
\pgfsetbuttcap%
\pgfsetroundjoin%
\definecolor{currentfill}{rgb}{0.121569,0.466667,0.705882}%
\pgfsetfillcolor{currentfill}%
\pgfsetfillopacity{0.745351}%
\pgfsetlinewidth{1.003750pt}%
\definecolor{currentstroke}{rgb}{0.121569,0.466667,0.705882}%
\pgfsetstrokecolor{currentstroke}%
\pgfsetstrokeopacity{0.745351}%
\pgfsetdash{}{0pt}%
\pgfpathmoveto{\pgfqpoint{1.264987in}{1.196609in}}%
\pgfpathcurveto{\pgfqpoint{1.273223in}{1.196609in}}{\pgfqpoint{1.281123in}{1.199881in}}{\pgfqpoint{1.286947in}{1.205705in}}%
\pgfpathcurveto{\pgfqpoint{1.292771in}{1.211529in}}{\pgfqpoint{1.296044in}{1.219429in}}{\pgfqpoint{1.296044in}{1.227666in}}%
\pgfpathcurveto{\pgfqpoint{1.296044in}{1.235902in}}{\pgfqpoint{1.292771in}{1.243802in}}{\pgfqpoint{1.286947in}{1.249626in}}%
\pgfpathcurveto{\pgfqpoint{1.281123in}{1.255450in}}{\pgfqpoint{1.273223in}{1.258722in}}{\pgfqpoint{1.264987in}{1.258722in}}%
\pgfpathcurveto{\pgfqpoint{1.256751in}{1.258722in}}{\pgfqpoint{1.248851in}{1.255450in}}{\pgfqpoint{1.243027in}{1.249626in}}%
\pgfpathcurveto{\pgfqpoint{1.237203in}{1.243802in}}{\pgfqpoint{1.233931in}{1.235902in}}{\pgfqpoint{1.233931in}{1.227666in}}%
\pgfpathcurveto{\pgfqpoint{1.233931in}{1.219429in}}{\pgfqpoint{1.237203in}{1.211529in}}{\pgfqpoint{1.243027in}{1.205705in}}%
\pgfpathcurveto{\pgfqpoint{1.248851in}{1.199881in}}{\pgfqpoint{1.256751in}{1.196609in}}{\pgfqpoint{1.264987in}{1.196609in}}%
\pgfpathclose%
\pgfusepath{stroke,fill}%
\end{pgfscope}%
\begin{pgfscope}%
\pgfpathrectangle{\pgfqpoint{0.100000in}{0.212622in}}{\pgfqpoint{3.696000in}{3.696000in}}%
\pgfusepath{clip}%
\pgfsetbuttcap%
\pgfsetroundjoin%
\definecolor{currentfill}{rgb}{0.121569,0.466667,0.705882}%
\pgfsetfillcolor{currentfill}%
\pgfsetfillopacity{0.747345}%
\pgfsetlinewidth{1.003750pt}%
\definecolor{currentstroke}{rgb}{0.121569,0.466667,0.705882}%
\pgfsetstrokecolor{currentstroke}%
\pgfsetstrokeopacity{0.747345}%
\pgfsetdash{}{0pt}%
\pgfpathmoveto{\pgfqpoint{1.272103in}{1.198150in}}%
\pgfpathcurveto{\pgfqpoint{1.280340in}{1.198150in}}{\pgfqpoint{1.288240in}{1.201423in}}{\pgfqpoint{1.294064in}{1.207247in}}%
\pgfpathcurveto{\pgfqpoint{1.299888in}{1.213071in}}{\pgfqpoint{1.303160in}{1.220971in}}{\pgfqpoint{1.303160in}{1.229207in}}%
\pgfpathcurveto{\pgfqpoint{1.303160in}{1.237443in}}{\pgfqpoint{1.299888in}{1.245343in}}{\pgfqpoint{1.294064in}{1.251167in}}%
\pgfpathcurveto{\pgfqpoint{1.288240in}{1.256991in}}{\pgfqpoint{1.280340in}{1.260263in}}{\pgfqpoint{1.272103in}{1.260263in}}%
\pgfpathcurveto{\pgfqpoint{1.263867in}{1.260263in}}{\pgfqpoint{1.255967in}{1.256991in}}{\pgfqpoint{1.250143in}{1.251167in}}%
\pgfpathcurveto{\pgfqpoint{1.244319in}{1.245343in}}{\pgfqpoint{1.241047in}{1.237443in}}{\pgfqpoint{1.241047in}{1.229207in}}%
\pgfpathcurveto{\pgfqpoint{1.241047in}{1.220971in}}{\pgfqpoint{1.244319in}{1.213071in}}{\pgfqpoint{1.250143in}{1.207247in}}%
\pgfpathcurveto{\pgfqpoint{1.255967in}{1.201423in}}{\pgfqpoint{1.263867in}{1.198150in}}{\pgfqpoint{1.272103in}{1.198150in}}%
\pgfpathclose%
\pgfusepath{stroke,fill}%
\end{pgfscope}%
\begin{pgfscope}%
\pgfpathrectangle{\pgfqpoint{0.100000in}{0.212622in}}{\pgfqpoint{3.696000in}{3.696000in}}%
\pgfusepath{clip}%
\pgfsetbuttcap%
\pgfsetroundjoin%
\definecolor{currentfill}{rgb}{0.121569,0.466667,0.705882}%
\pgfsetfillcolor{currentfill}%
\pgfsetfillopacity{0.748034}%
\pgfsetlinewidth{1.003750pt}%
\definecolor{currentstroke}{rgb}{0.121569,0.466667,0.705882}%
\pgfsetstrokecolor{currentstroke}%
\pgfsetstrokeopacity{0.748034}%
\pgfsetdash{}{0pt}%
\pgfpathmoveto{\pgfqpoint{2.218363in}{1.621909in}}%
\pgfpathcurveto{\pgfqpoint{2.226599in}{1.621909in}}{\pgfqpoint{2.234499in}{1.625182in}}{\pgfqpoint{2.240323in}{1.631005in}}%
\pgfpathcurveto{\pgfqpoint{2.246147in}{1.636829in}}{\pgfqpoint{2.249419in}{1.644729in}}{\pgfqpoint{2.249419in}{1.652966in}}%
\pgfpathcurveto{\pgfqpoint{2.249419in}{1.661202in}}{\pgfqpoint{2.246147in}{1.669102in}}{\pgfqpoint{2.240323in}{1.674926in}}%
\pgfpathcurveto{\pgfqpoint{2.234499in}{1.680750in}}{\pgfqpoint{2.226599in}{1.684022in}}{\pgfqpoint{2.218363in}{1.684022in}}%
\pgfpathcurveto{\pgfqpoint{2.210127in}{1.684022in}}{\pgfqpoint{2.202227in}{1.680750in}}{\pgfqpoint{2.196403in}{1.674926in}}%
\pgfpathcurveto{\pgfqpoint{2.190579in}{1.669102in}}{\pgfqpoint{2.187306in}{1.661202in}}{\pgfqpoint{2.187306in}{1.652966in}}%
\pgfpathcurveto{\pgfqpoint{2.187306in}{1.644729in}}{\pgfqpoint{2.190579in}{1.636829in}}{\pgfqpoint{2.196403in}{1.631005in}}%
\pgfpathcurveto{\pgfqpoint{2.202227in}{1.625182in}}{\pgfqpoint{2.210127in}{1.621909in}}{\pgfqpoint{2.218363in}{1.621909in}}%
\pgfpathclose%
\pgfusepath{stroke,fill}%
\end{pgfscope}%
\begin{pgfscope}%
\pgfpathrectangle{\pgfqpoint{0.100000in}{0.212622in}}{\pgfqpoint{3.696000in}{3.696000in}}%
\pgfusepath{clip}%
\pgfsetbuttcap%
\pgfsetroundjoin%
\definecolor{currentfill}{rgb}{0.121569,0.466667,0.705882}%
\pgfsetfillcolor{currentfill}%
\pgfsetfillopacity{0.749243}%
\pgfsetlinewidth{1.003750pt}%
\definecolor{currentstroke}{rgb}{0.121569,0.466667,0.705882}%
\pgfsetstrokecolor{currentstroke}%
\pgfsetstrokeopacity{0.749243}%
\pgfsetdash{}{0pt}%
\pgfpathmoveto{\pgfqpoint{1.278849in}{1.199806in}}%
\pgfpathcurveto{\pgfqpoint{1.287085in}{1.199806in}}{\pgfqpoint{1.294985in}{1.203078in}}{\pgfqpoint{1.300809in}{1.208902in}}%
\pgfpathcurveto{\pgfqpoint{1.306633in}{1.214726in}}{\pgfqpoint{1.309906in}{1.222626in}}{\pgfqpoint{1.309906in}{1.230862in}}%
\pgfpathcurveto{\pgfqpoint{1.309906in}{1.239098in}}{\pgfqpoint{1.306633in}{1.246998in}}{\pgfqpoint{1.300809in}{1.252822in}}%
\pgfpathcurveto{\pgfqpoint{1.294985in}{1.258646in}}{\pgfqpoint{1.287085in}{1.261919in}}{\pgfqpoint{1.278849in}{1.261919in}}%
\pgfpathcurveto{\pgfqpoint{1.270613in}{1.261919in}}{\pgfqpoint{1.262713in}{1.258646in}}{\pgfqpoint{1.256889in}{1.252822in}}%
\pgfpathcurveto{\pgfqpoint{1.251065in}{1.246998in}}{\pgfqpoint{1.247793in}{1.239098in}}{\pgfqpoint{1.247793in}{1.230862in}}%
\pgfpathcurveto{\pgfqpoint{1.247793in}{1.222626in}}{\pgfqpoint{1.251065in}{1.214726in}}{\pgfqpoint{1.256889in}{1.208902in}}%
\pgfpathcurveto{\pgfqpoint{1.262713in}{1.203078in}}{\pgfqpoint{1.270613in}{1.199806in}}{\pgfqpoint{1.278849in}{1.199806in}}%
\pgfpathclose%
\pgfusepath{stroke,fill}%
\end{pgfscope}%
\begin{pgfscope}%
\pgfpathrectangle{\pgfqpoint{0.100000in}{0.212622in}}{\pgfqpoint{3.696000in}{3.696000in}}%
\pgfusepath{clip}%
\pgfsetbuttcap%
\pgfsetroundjoin%
\definecolor{currentfill}{rgb}{0.121569,0.466667,0.705882}%
\pgfsetfillcolor{currentfill}%
\pgfsetfillopacity{0.751162}%
\pgfsetlinewidth{1.003750pt}%
\definecolor{currentstroke}{rgb}{0.121569,0.466667,0.705882}%
\pgfsetstrokecolor{currentstroke}%
\pgfsetstrokeopacity{0.751162}%
\pgfsetdash{}{0pt}%
\pgfpathmoveto{\pgfqpoint{1.284629in}{1.201758in}}%
\pgfpathcurveto{\pgfqpoint{1.292866in}{1.201758in}}{\pgfqpoint{1.300766in}{1.205031in}}{\pgfqpoint{1.306590in}{1.210855in}}%
\pgfpathcurveto{\pgfqpoint{1.312414in}{1.216679in}}{\pgfqpoint{1.315686in}{1.224579in}}{\pgfqpoint{1.315686in}{1.232815in}}%
\pgfpathcurveto{\pgfqpoint{1.315686in}{1.241051in}}{\pgfqpoint{1.312414in}{1.248951in}}{\pgfqpoint{1.306590in}{1.254775in}}%
\pgfpathcurveto{\pgfqpoint{1.300766in}{1.260599in}}{\pgfqpoint{1.292866in}{1.263871in}}{\pgfqpoint{1.284629in}{1.263871in}}%
\pgfpathcurveto{\pgfqpoint{1.276393in}{1.263871in}}{\pgfqpoint{1.268493in}{1.260599in}}{\pgfqpoint{1.262669in}{1.254775in}}%
\pgfpathcurveto{\pgfqpoint{1.256845in}{1.248951in}}{\pgfqpoint{1.253573in}{1.241051in}}{\pgfqpoint{1.253573in}{1.232815in}}%
\pgfpathcurveto{\pgfqpoint{1.253573in}{1.224579in}}{\pgfqpoint{1.256845in}{1.216679in}}{\pgfqpoint{1.262669in}{1.210855in}}%
\pgfpathcurveto{\pgfqpoint{1.268493in}{1.205031in}}{\pgfqpoint{1.276393in}{1.201758in}}{\pgfqpoint{1.284629in}{1.201758in}}%
\pgfpathclose%
\pgfusepath{stroke,fill}%
\end{pgfscope}%
\begin{pgfscope}%
\pgfpathrectangle{\pgfqpoint{0.100000in}{0.212622in}}{\pgfqpoint{3.696000in}{3.696000in}}%
\pgfusepath{clip}%
\pgfsetbuttcap%
\pgfsetroundjoin%
\definecolor{currentfill}{rgb}{0.121569,0.466667,0.705882}%
\pgfsetfillcolor{currentfill}%
\pgfsetfillopacity{0.753393}%
\pgfsetlinewidth{1.003750pt}%
\definecolor{currentstroke}{rgb}{0.121569,0.466667,0.705882}%
\pgfsetstrokecolor{currentstroke}%
\pgfsetstrokeopacity{0.753393}%
\pgfsetdash{}{0pt}%
\pgfpathmoveto{\pgfqpoint{1.294975in}{1.199668in}}%
\pgfpathcurveto{\pgfqpoint{1.303211in}{1.199668in}}{\pgfqpoint{1.311111in}{1.202940in}}{\pgfqpoint{1.316935in}{1.208764in}}%
\pgfpathcurveto{\pgfqpoint{1.322759in}{1.214588in}}{\pgfqpoint{1.326031in}{1.222488in}}{\pgfqpoint{1.326031in}{1.230725in}}%
\pgfpathcurveto{\pgfqpoint{1.326031in}{1.238961in}}{\pgfqpoint{1.322759in}{1.246861in}}{\pgfqpoint{1.316935in}{1.252685in}}%
\pgfpathcurveto{\pgfqpoint{1.311111in}{1.258509in}}{\pgfqpoint{1.303211in}{1.261781in}}{\pgfqpoint{1.294975in}{1.261781in}}%
\pgfpathcurveto{\pgfqpoint{1.286739in}{1.261781in}}{\pgfqpoint{1.278839in}{1.258509in}}{\pgfqpoint{1.273015in}{1.252685in}}%
\pgfpathcurveto{\pgfqpoint{1.267191in}{1.246861in}}{\pgfqpoint{1.263918in}{1.238961in}}{\pgfqpoint{1.263918in}{1.230725in}}%
\pgfpathcurveto{\pgfqpoint{1.263918in}{1.222488in}}{\pgfqpoint{1.267191in}{1.214588in}}{\pgfqpoint{1.273015in}{1.208764in}}%
\pgfpathcurveto{\pgfqpoint{1.278839in}{1.202940in}}{\pgfqpoint{1.286739in}{1.199668in}}{\pgfqpoint{1.294975in}{1.199668in}}%
\pgfpathclose%
\pgfusepath{stroke,fill}%
\end{pgfscope}%
\begin{pgfscope}%
\pgfpathrectangle{\pgfqpoint{0.100000in}{0.212622in}}{\pgfqpoint{3.696000in}{3.696000in}}%
\pgfusepath{clip}%
\pgfsetbuttcap%
\pgfsetroundjoin%
\definecolor{currentfill}{rgb}{0.121569,0.466667,0.705882}%
\pgfsetfillcolor{currentfill}%
\pgfsetfillopacity{0.753738}%
\pgfsetlinewidth{1.003750pt}%
\definecolor{currentstroke}{rgb}{0.121569,0.466667,0.705882}%
\pgfsetstrokecolor{currentstroke}%
\pgfsetstrokeopacity{0.753738}%
\pgfsetdash{}{0pt}%
\pgfpathmoveto{\pgfqpoint{2.222352in}{1.603953in}}%
\pgfpathcurveto{\pgfqpoint{2.230588in}{1.603953in}}{\pgfqpoint{2.238488in}{1.607225in}}{\pgfqpoint{2.244312in}{1.613049in}}%
\pgfpathcurveto{\pgfqpoint{2.250136in}{1.618873in}}{\pgfqpoint{2.253409in}{1.626773in}}{\pgfqpoint{2.253409in}{1.635009in}}%
\pgfpathcurveto{\pgfqpoint{2.253409in}{1.643245in}}{\pgfqpoint{2.250136in}{1.651145in}}{\pgfqpoint{2.244312in}{1.656969in}}%
\pgfpathcurveto{\pgfqpoint{2.238488in}{1.662793in}}{\pgfqpoint{2.230588in}{1.666066in}}{\pgfqpoint{2.222352in}{1.666066in}}%
\pgfpathcurveto{\pgfqpoint{2.214116in}{1.666066in}}{\pgfqpoint{2.206216in}{1.662793in}}{\pgfqpoint{2.200392in}{1.656969in}}%
\pgfpathcurveto{\pgfqpoint{2.194568in}{1.651145in}}{\pgfqpoint{2.191296in}{1.643245in}}{\pgfqpoint{2.191296in}{1.635009in}}%
\pgfpathcurveto{\pgfqpoint{2.191296in}{1.626773in}}{\pgfqpoint{2.194568in}{1.618873in}}{\pgfqpoint{2.200392in}{1.613049in}}%
\pgfpathcurveto{\pgfqpoint{2.206216in}{1.607225in}}{\pgfqpoint{2.214116in}{1.603953in}}{\pgfqpoint{2.222352in}{1.603953in}}%
\pgfpathclose%
\pgfusepath{stroke,fill}%
\end{pgfscope}%
\begin{pgfscope}%
\pgfpathrectangle{\pgfqpoint{0.100000in}{0.212622in}}{\pgfqpoint{3.696000in}{3.696000in}}%
\pgfusepath{clip}%
\pgfsetbuttcap%
\pgfsetroundjoin%
\definecolor{currentfill}{rgb}{0.121569,0.466667,0.705882}%
\pgfsetfillcolor{currentfill}%
\pgfsetfillopacity{0.755465}%
\pgfsetlinewidth{1.003750pt}%
\definecolor{currentstroke}{rgb}{0.121569,0.466667,0.705882}%
\pgfsetstrokecolor{currentstroke}%
\pgfsetstrokeopacity{0.755465}%
\pgfsetdash{}{0pt}%
\pgfpathmoveto{\pgfqpoint{1.304471in}{1.198132in}}%
\pgfpathcurveto{\pgfqpoint{1.312707in}{1.198132in}}{\pgfqpoint{1.320607in}{1.201404in}}{\pgfqpoint{1.326431in}{1.207228in}}%
\pgfpathcurveto{\pgfqpoint{1.332255in}{1.213052in}}{\pgfqpoint{1.335527in}{1.220952in}}{\pgfqpoint{1.335527in}{1.229189in}}%
\pgfpathcurveto{\pgfqpoint{1.335527in}{1.237425in}}{\pgfqpoint{1.332255in}{1.245325in}}{\pgfqpoint{1.326431in}{1.251149in}}%
\pgfpathcurveto{\pgfqpoint{1.320607in}{1.256973in}}{\pgfqpoint{1.312707in}{1.260245in}}{\pgfqpoint{1.304471in}{1.260245in}}%
\pgfpathcurveto{\pgfqpoint{1.296235in}{1.260245in}}{\pgfqpoint{1.288335in}{1.256973in}}{\pgfqpoint{1.282511in}{1.251149in}}%
\pgfpathcurveto{\pgfqpoint{1.276687in}{1.245325in}}{\pgfqpoint{1.273414in}{1.237425in}}{\pgfqpoint{1.273414in}{1.229189in}}%
\pgfpathcurveto{\pgfqpoint{1.273414in}{1.220952in}}{\pgfqpoint{1.276687in}{1.213052in}}{\pgfqpoint{1.282511in}{1.207228in}}%
\pgfpathcurveto{\pgfqpoint{1.288335in}{1.201404in}}{\pgfqpoint{1.296235in}{1.198132in}}{\pgfqpoint{1.304471in}{1.198132in}}%
\pgfpathclose%
\pgfusepath{stroke,fill}%
\end{pgfscope}%
\begin{pgfscope}%
\pgfpathrectangle{\pgfqpoint{0.100000in}{0.212622in}}{\pgfqpoint{3.696000in}{3.696000in}}%
\pgfusepath{clip}%
\pgfsetbuttcap%
\pgfsetroundjoin%
\definecolor{currentfill}{rgb}{0.121569,0.466667,0.705882}%
\pgfsetfillcolor{currentfill}%
\pgfsetfillopacity{0.757153}%
\pgfsetlinewidth{1.003750pt}%
\definecolor{currentstroke}{rgb}{0.121569,0.466667,0.705882}%
\pgfsetstrokecolor{currentstroke}%
\pgfsetstrokeopacity{0.757153}%
\pgfsetdash{}{0pt}%
\pgfpathmoveto{\pgfqpoint{1.312124in}{1.196532in}}%
\pgfpathcurveto{\pgfqpoint{1.320360in}{1.196532in}}{\pgfqpoint{1.328260in}{1.199805in}}{\pgfqpoint{1.334084in}{1.205628in}}%
\pgfpathcurveto{\pgfqpoint{1.339908in}{1.211452in}}{\pgfqpoint{1.343181in}{1.219352in}}{\pgfqpoint{1.343181in}{1.227589in}}%
\pgfpathcurveto{\pgfqpoint{1.343181in}{1.235825in}}{\pgfqpoint{1.339908in}{1.243725in}}{\pgfqpoint{1.334084in}{1.249549in}}%
\pgfpathcurveto{\pgfqpoint{1.328260in}{1.255373in}}{\pgfqpoint{1.320360in}{1.258645in}}{\pgfqpoint{1.312124in}{1.258645in}}%
\pgfpathcurveto{\pgfqpoint{1.303888in}{1.258645in}}{\pgfqpoint{1.295988in}{1.255373in}}{\pgfqpoint{1.290164in}{1.249549in}}%
\pgfpathcurveto{\pgfqpoint{1.284340in}{1.243725in}}{\pgfqpoint{1.281068in}{1.235825in}}{\pgfqpoint{1.281068in}{1.227589in}}%
\pgfpathcurveto{\pgfqpoint{1.281068in}{1.219352in}}{\pgfqpoint{1.284340in}{1.211452in}}{\pgfqpoint{1.290164in}{1.205628in}}%
\pgfpathcurveto{\pgfqpoint{1.295988in}{1.199805in}}{\pgfqpoint{1.303888in}{1.196532in}}{\pgfqpoint{1.312124in}{1.196532in}}%
\pgfpathclose%
\pgfusepath{stroke,fill}%
\end{pgfscope}%
\begin{pgfscope}%
\pgfpathrectangle{\pgfqpoint{0.100000in}{0.212622in}}{\pgfqpoint{3.696000in}{3.696000in}}%
\pgfusepath{clip}%
\pgfsetbuttcap%
\pgfsetroundjoin%
\definecolor{currentfill}{rgb}{0.121569,0.466667,0.705882}%
\pgfsetfillcolor{currentfill}%
\pgfsetfillopacity{0.759372}%
\pgfsetlinewidth{1.003750pt}%
\definecolor{currentstroke}{rgb}{0.121569,0.466667,0.705882}%
\pgfsetstrokecolor{currentstroke}%
\pgfsetstrokeopacity{0.759372}%
\pgfsetdash{}{0pt}%
\pgfpathmoveto{\pgfqpoint{2.227212in}{1.585693in}}%
\pgfpathcurveto{\pgfqpoint{2.235449in}{1.585693in}}{\pgfqpoint{2.243349in}{1.588965in}}{\pgfqpoint{2.249173in}{1.594789in}}%
\pgfpathcurveto{\pgfqpoint{2.254997in}{1.600613in}}{\pgfqpoint{2.258269in}{1.608513in}}{\pgfqpoint{2.258269in}{1.616750in}}%
\pgfpathcurveto{\pgfqpoint{2.258269in}{1.624986in}}{\pgfqpoint{2.254997in}{1.632886in}}{\pgfqpoint{2.249173in}{1.638710in}}%
\pgfpathcurveto{\pgfqpoint{2.243349in}{1.644534in}}{\pgfqpoint{2.235449in}{1.647806in}}{\pgfqpoint{2.227212in}{1.647806in}}%
\pgfpathcurveto{\pgfqpoint{2.218976in}{1.647806in}}{\pgfqpoint{2.211076in}{1.644534in}}{\pgfqpoint{2.205252in}{1.638710in}}%
\pgfpathcurveto{\pgfqpoint{2.199428in}{1.632886in}}{\pgfqpoint{2.196156in}{1.624986in}}{\pgfqpoint{2.196156in}{1.616750in}}%
\pgfpathcurveto{\pgfqpoint{2.196156in}{1.608513in}}{\pgfqpoint{2.199428in}{1.600613in}}{\pgfqpoint{2.205252in}{1.594789in}}%
\pgfpathcurveto{\pgfqpoint{2.211076in}{1.588965in}}{\pgfqpoint{2.218976in}{1.585693in}}{\pgfqpoint{2.227212in}{1.585693in}}%
\pgfpathclose%
\pgfusepath{stroke,fill}%
\end{pgfscope}%
\begin{pgfscope}%
\pgfpathrectangle{\pgfqpoint{0.100000in}{0.212622in}}{\pgfqpoint{3.696000in}{3.696000in}}%
\pgfusepath{clip}%
\pgfsetbuttcap%
\pgfsetroundjoin%
\definecolor{currentfill}{rgb}{0.121569,0.466667,0.705882}%
\pgfsetfillcolor{currentfill}%
\pgfsetfillopacity{0.760396}%
\pgfsetlinewidth{1.003750pt}%
\definecolor{currentstroke}{rgb}{0.121569,0.466667,0.705882}%
\pgfsetstrokecolor{currentstroke}%
\pgfsetstrokeopacity{0.760396}%
\pgfsetdash{}{0pt}%
\pgfpathmoveto{\pgfqpoint{1.325965in}{1.193996in}}%
\pgfpathcurveto{\pgfqpoint{1.334201in}{1.193996in}}{\pgfqpoint{1.342101in}{1.197268in}}{\pgfqpoint{1.347925in}{1.203092in}}%
\pgfpathcurveto{\pgfqpoint{1.353749in}{1.208916in}}{\pgfqpoint{1.357021in}{1.216816in}}{\pgfqpoint{1.357021in}{1.225052in}}%
\pgfpathcurveto{\pgfqpoint{1.357021in}{1.233289in}}{\pgfqpoint{1.353749in}{1.241189in}}{\pgfqpoint{1.347925in}{1.247013in}}%
\pgfpathcurveto{\pgfqpoint{1.342101in}{1.252837in}}{\pgfqpoint{1.334201in}{1.256109in}}{\pgfqpoint{1.325965in}{1.256109in}}%
\pgfpathcurveto{\pgfqpoint{1.317728in}{1.256109in}}{\pgfqpoint{1.309828in}{1.252837in}}{\pgfqpoint{1.304004in}{1.247013in}}%
\pgfpathcurveto{\pgfqpoint{1.298181in}{1.241189in}}{\pgfqpoint{1.294908in}{1.233289in}}{\pgfqpoint{1.294908in}{1.225052in}}%
\pgfpathcurveto{\pgfqpoint{1.294908in}{1.216816in}}{\pgfqpoint{1.298181in}{1.208916in}}{\pgfqpoint{1.304004in}{1.203092in}}%
\pgfpathcurveto{\pgfqpoint{1.309828in}{1.197268in}}{\pgfqpoint{1.317728in}{1.193996in}}{\pgfqpoint{1.325965in}{1.193996in}}%
\pgfpathclose%
\pgfusepath{stroke,fill}%
\end{pgfscope}%
\begin{pgfscope}%
\pgfpathrectangle{\pgfqpoint{0.100000in}{0.212622in}}{\pgfqpoint{3.696000in}{3.696000in}}%
\pgfusepath{clip}%
\pgfsetbuttcap%
\pgfsetroundjoin%
\definecolor{currentfill}{rgb}{0.121569,0.466667,0.705882}%
\pgfsetfillcolor{currentfill}%
\pgfsetfillopacity{0.762208}%
\pgfsetlinewidth{1.003750pt}%
\definecolor{currentstroke}{rgb}{0.121569,0.466667,0.705882}%
\pgfsetstrokecolor{currentstroke}%
\pgfsetstrokeopacity{0.762208}%
\pgfsetdash{}{0pt}%
\pgfpathmoveto{\pgfqpoint{2.230278in}{1.575107in}}%
\pgfpathcurveto{\pgfqpoint{2.238514in}{1.575107in}}{\pgfqpoint{2.246414in}{1.578379in}}{\pgfqpoint{2.252238in}{1.584203in}}%
\pgfpathcurveto{\pgfqpoint{2.258062in}{1.590027in}}{\pgfqpoint{2.261334in}{1.597927in}}{\pgfqpoint{2.261334in}{1.606163in}}%
\pgfpathcurveto{\pgfqpoint{2.261334in}{1.614400in}}{\pgfqpoint{2.258062in}{1.622300in}}{\pgfqpoint{2.252238in}{1.628124in}}%
\pgfpathcurveto{\pgfqpoint{2.246414in}{1.633948in}}{\pgfqpoint{2.238514in}{1.637220in}}{\pgfqpoint{2.230278in}{1.637220in}}%
\pgfpathcurveto{\pgfqpoint{2.222041in}{1.637220in}}{\pgfqpoint{2.214141in}{1.633948in}}{\pgfqpoint{2.208317in}{1.628124in}}%
\pgfpathcurveto{\pgfqpoint{2.202494in}{1.622300in}}{\pgfqpoint{2.199221in}{1.614400in}}{\pgfqpoint{2.199221in}{1.606163in}}%
\pgfpathcurveto{\pgfqpoint{2.199221in}{1.597927in}}{\pgfqpoint{2.202494in}{1.590027in}}{\pgfqpoint{2.208317in}{1.584203in}}%
\pgfpathcurveto{\pgfqpoint{2.214141in}{1.578379in}}{\pgfqpoint{2.222041in}{1.575107in}}{\pgfqpoint{2.230278in}{1.575107in}}%
\pgfpathclose%
\pgfusepath{stroke,fill}%
\end{pgfscope}%
\begin{pgfscope}%
\pgfpathrectangle{\pgfqpoint{0.100000in}{0.212622in}}{\pgfqpoint{3.696000in}{3.696000in}}%
\pgfusepath{clip}%
\pgfsetbuttcap%
\pgfsetroundjoin%
\definecolor{currentfill}{rgb}{0.121569,0.466667,0.705882}%
\pgfsetfillcolor{currentfill}%
\pgfsetfillopacity{0.763030}%
\pgfsetlinewidth{1.003750pt}%
\definecolor{currentstroke}{rgb}{0.121569,0.466667,0.705882}%
\pgfsetstrokecolor{currentstroke}%
\pgfsetstrokeopacity{0.763030}%
\pgfsetdash{}{0pt}%
\pgfpathmoveto{\pgfqpoint{1.338157in}{1.192310in}}%
\pgfpathcurveto{\pgfqpoint{1.346393in}{1.192310in}}{\pgfqpoint{1.354293in}{1.195582in}}{\pgfqpoint{1.360117in}{1.201406in}}%
\pgfpathcurveto{\pgfqpoint{1.365941in}{1.207230in}}{\pgfqpoint{1.369214in}{1.215130in}}{\pgfqpoint{1.369214in}{1.223367in}}%
\pgfpathcurveto{\pgfqpoint{1.369214in}{1.231603in}}{\pgfqpoint{1.365941in}{1.239503in}}{\pgfqpoint{1.360117in}{1.245327in}}%
\pgfpathcurveto{\pgfqpoint{1.354293in}{1.251151in}}{\pgfqpoint{1.346393in}{1.254423in}}{\pgfqpoint{1.338157in}{1.254423in}}%
\pgfpathcurveto{\pgfqpoint{1.329921in}{1.254423in}}{\pgfqpoint{1.322021in}{1.251151in}}{\pgfqpoint{1.316197in}{1.245327in}}%
\pgfpathcurveto{\pgfqpoint{1.310373in}{1.239503in}}{\pgfqpoint{1.307101in}{1.231603in}}{\pgfqpoint{1.307101in}{1.223367in}}%
\pgfpathcurveto{\pgfqpoint{1.307101in}{1.215130in}}{\pgfqpoint{1.310373in}{1.207230in}}{\pgfqpoint{1.316197in}{1.201406in}}%
\pgfpathcurveto{\pgfqpoint{1.322021in}{1.195582in}}{\pgfqpoint{1.329921in}{1.192310in}}{\pgfqpoint{1.338157in}{1.192310in}}%
\pgfpathclose%
\pgfusepath{stroke,fill}%
\end{pgfscope}%
\begin{pgfscope}%
\pgfpathrectangle{\pgfqpoint{0.100000in}{0.212622in}}{\pgfqpoint{3.696000in}{3.696000in}}%
\pgfusepath{clip}%
\pgfsetbuttcap%
\pgfsetroundjoin%
\definecolor{currentfill}{rgb}{0.121569,0.466667,0.705882}%
\pgfsetfillcolor{currentfill}%
\pgfsetfillopacity{0.765566}%
\pgfsetlinewidth{1.003750pt}%
\definecolor{currentstroke}{rgb}{0.121569,0.466667,0.705882}%
\pgfsetstrokecolor{currentstroke}%
\pgfsetstrokeopacity{0.765566}%
\pgfsetdash{}{0pt}%
\pgfpathmoveto{\pgfqpoint{2.232659in}{1.563944in}}%
\pgfpathcurveto{\pgfqpoint{2.240895in}{1.563944in}}{\pgfqpoint{2.248795in}{1.567216in}}{\pgfqpoint{2.254619in}{1.573040in}}%
\pgfpathcurveto{\pgfqpoint{2.260443in}{1.578864in}}{\pgfqpoint{2.263716in}{1.586764in}}{\pgfqpoint{2.263716in}{1.595000in}}%
\pgfpathcurveto{\pgfqpoint{2.263716in}{1.603236in}}{\pgfqpoint{2.260443in}{1.611137in}}{\pgfqpoint{2.254619in}{1.616960in}}%
\pgfpathcurveto{\pgfqpoint{2.248795in}{1.622784in}}{\pgfqpoint{2.240895in}{1.626057in}}{\pgfqpoint{2.232659in}{1.626057in}}%
\pgfpathcurveto{\pgfqpoint{2.224423in}{1.626057in}}{\pgfqpoint{2.216523in}{1.622784in}}{\pgfqpoint{2.210699in}{1.616960in}}%
\pgfpathcurveto{\pgfqpoint{2.204875in}{1.611137in}}{\pgfqpoint{2.201603in}{1.603236in}}{\pgfqpoint{2.201603in}{1.595000in}}%
\pgfpathcurveto{\pgfqpoint{2.201603in}{1.586764in}}{\pgfqpoint{2.204875in}{1.578864in}}{\pgfqpoint{2.210699in}{1.573040in}}%
\pgfpathcurveto{\pgfqpoint{2.216523in}{1.567216in}}{\pgfqpoint{2.224423in}{1.563944in}}{\pgfqpoint{2.232659in}{1.563944in}}%
\pgfpathclose%
\pgfusepath{stroke,fill}%
\end{pgfscope}%
\begin{pgfscope}%
\pgfpathrectangle{\pgfqpoint{0.100000in}{0.212622in}}{\pgfqpoint{3.696000in}{3.696000in}}%
\pgfusepath{clip}%
\pgfsetbuttcap%
\pgfsetroundjoin%
\definecolor{currentfill}{rgb}{0.121569,0.466667,0.705882}%
\pgfsetfillcolor{currentfill}%
\pgfsetfillopacity{0.765733}%
\pgfsetlinewidth{1.003750pt}%
\definecolor{currentstroke}{rgb}{0.121569,0.466667,0.705882}%
\pgfsetstrokecolor{currentstroke}%
\pgfsetstrokeopacity{0.765733}%
\pgfsetdash{}{0pt}%
\pgfpathmoveto{\pgfqpoint{1.349216in}{1.191105in}}%
\pgfpathcurveto{\pgfqpoint{1.357452in}{1.191105in}}{\pgfqpoint{1.365352in}{1.194377in}}{\pgfqpoint{1.371176in}{1.200201in}}%
\pgfpathcurveto{\pgfqpoint{1.377000in}{1.206025in}}{\pgfqpoint{1.380272in}{1.213925in}}{\pgfqpoint{1.380272in}{1.222162in}}%
\pgfpathcurveto{\pgfqpoint{1.380272in}{1.230398in}}{\pgfqpoint{1.377000in}{1.238298in}}{\pgfqpoint{1.371176in}{1.244122in}}%
\pgfpathcurveto{\pgfqpoint{1.365352in}{1.249946in}}{\pgfqpoint{1.357452in}{1.253218in}}{\pgfqpoint{1.349216in}{1.253218in}}%
\pgfpathcurveto{\pgfqpoint{1.340980in}{1.253218in}}{\pgfqpoint{1.333080in}{1.249946in}}{\pgfqpoint{1.327256in}{1.244122in}}%
\pgfpathcurveto{\pgfqpoint{1.321432in}{1.238298in}}{\pgfqpoint{1.318159in}{1.230398in}}{\pgfqpoint{1.318159in}{1.222162in}}%
\pgfpathcurveto{\pgfqpoint{1.318159in}{1.213925in}}{\pgfqpoint{1.321432in}{1.206025in}}{\pgfqpoint{1.327256in}{1.200201in}}%
\pgfpathcurveto{\pgfqpoint{1.333080in}{1.194377in}}{\pgfqpoint{1.340980in}{1.191105in}}{\pgfqpoint{1.349216in}{1.191105in}}%
\pgfpathclose%
\pgfusepath{stroke,fill}%
\end{pgfscope}%
\begin{pgfscope}%
\pgfpathrectangle{\pgfqpoint{0.100000in}{0.212622in}}{\pgfqpoint{3.696000in}{3.696000in}}%
\pgfusepath{clip}%
\pgfsetbuttcap%
\pgfsetroundjoin%
\definecolor{currentfill}{rgb}{0.121569,0.466667,0.705882}%
\pgfsetfillcolor{currentfill}%
\pgfsetfillopacity{0.768264}%
\pgfsetlinewidth{1.003750pt}%
\definecolor{currentstroke}{rgb}{0.121569,0.466667,0.705882}%
\pgfsetstrokecolor{currentstroke}%
\pgfsetstrokeopacity{0.768264}%
\pgfsetdash{}{0pt}%
\pgfpathmoveto{\pgfqpoint{1.359549in}{1.189977in}}%
\pgfpathcurveto{\pgfqpoint{1.367785in}{1.189977in}}{\pgfqpoint{1.375685in}{1.193249in}}{\pgfqpoint{1.381509in}{1.199073in}}%
\pgfpathcurveto{\pgfqpoint{1.387333in}{1.204897in}}{\pgfqpoint{1.390605in}{1.212797in}}{\pgfqpoint{1.390605in}{1.221033in}}%
\pgfpathcurveto{\pgfqpoint{1.390605in}{1.229269in}}{\pgfqpoint{1.387333in}{1.237169in}}{\pgfqpoint{1.381509in}{1.242993in}}%
\pgfpathcurveto{\pgfqpoint{1.375685in}{1.248817in}}{\pgfqpoint{1.367785in}{1.252090in}}{\pgfqpoint{1.359549in}{1.252090in}}%
\pgfpathcurveto{\pgfqpoint{1.351313in}{1.252090in}}{\pgfqpoint{1.343413in}{1.248817in}}{\pgfqpoint{1.337589in}{1.242993in}}%
\pgfpathcurveto{\pgfqpoint{1.331765in}{1.237169in}}{\pgfqpoint{1.328492in}{1.229269in}}{\pgfqpoint{1.328492in}{1.221033in}}%
\pgfpathcurveto{\pgfqpoint{1.328492in}{1.212797in}}{\pgfqpoint{1.331765in}{1.204897in}}{\pgfqpoint{1.337589in}{1.199073in}}%
\pgfpathcurveto{\pgfqpoint{1.343413in}{1.193249in}}{\pgfqpoint{1.351313in}{1.189977in}}{\pgfqpoint{1.359549in}{1.189977in}}%
\pgfpathclose%
\pgfusepath{stroke,fill}%
\end{pgfscope}%
\begin{pgfscope}%
\pgfpathrectangle{\pgfqpoint{0.100000in}{0.212622in}}{\pgfqpoint{3.696000in}{3.696000in}}%
\pgfusepath{clip}%
\pgfsetbuttcap%
\pgfsetroundjoin%
\definecolor{currentfill}{rgb}{0.121569,0.466667,0.705882}%
\pgfsetfillcolor{currentfill}%
\pgfsetfillopacity{0.769079}%
\pgfsetlinewidth{1.003750pt}%
\definecolor{currentstroke}{rgb}{0.121569,0.466667,0.705882}%
\pgfsetstrokecolor{currentstroke}%
\pgfsetstrokeopacity{0.769079}%
\pgfsetdash{}{0pt}%
\pgfpathmoveto{\pgfqpoint{2.235480in}{1.551935in}}%
\pgfpathcurveto{\pgfqpoint{2.243716in}{1.551935in}}{\pgfqpoint{2.251616in}{1.555208in}}{\pgfqpoint{2.257440in}{1.561032in}}%
\pgfpathcurveto{\pgfqpoint{2.263264in}{1.566855in}}{\pgfqpoint{2.266536in}{1.574756in}}{\pgfqpoint{2.266536in}{1.582992in}}%
\pgfpathcurveto{\pgfqpoint{2.266536in}{1.591228in}}{\pgfqpoint{2.263264in}{1.599128in}}{\pgfqpoint{2.257440in}{1.604952in}}%
\pgfpathcurveto{\pgfqpoint{2.251616in}{1.610776in}}{\pgfqpoint{2.243716in}{1.614048in}}{\pgfqpoint{2.235480in}{1.614048in}}%
\pgfpathcurveto{\pgfqpoint{2.227243in}{1.614048in}}{\pgfqpoint{2.219343in}{1.610776in}}{\pgfqpoint{2.213519in}{1.604952in}}%
\pgfpathcurveto{\pgfqpoint{2.207696in}{1.599128in}}{\pgfqpoint{2.204423in}{1.591228in}}{\pgfqpoint{2.204423in}{1.582992in}}%
\pgfpathcurveto{\pgfqpoint{2.204423in}{1.574756in}}{\pgfqpoint{2.207696in}{1.566855in}}{\pgfqpoint{2.213519in}{1.561032in}}%
\pgfpathcurveto{\pgfqpoint{2.219343in}{1.555208in}}{\pgfqpoint{2.227243in}{1.551935in}}{\pgfqpoint{2.235480in}{1.551935in}}%
\pgfpathclose%
\pgfusepath{stroke,fill}%
\end{pgfscope}%
\begin{pgfscope}%
\pgfpathrectangle{\pgfqpoint{0.100000in}{0.212622in}}{\pgfqpoint{3.696000in}{3.696000in}}%
\pgfusepath{clip}%
\pgfsetbuttcap%
\pgfsetroundjoin%
\definecolor{currentfill}{rgb}{0.121569,0.466667,0.705882}%
\pgfsetfillcolor{currentfill}%
\pgfsetfillopacity{0.770503}%
\pgfsetlinewidth{1.003750pt}%
\definecolor{currentstroke}{rgb}{0.121569,0.466667,0.705882}%
\pgfsetstrokecolor{currentstroke}%
\pgfsetstrokeopacity{0.770503}%
\pgfsetdash{}{0pt}%
\pgfpathmoveto{\pgfqpoint{1.369478in}{1.188519in}}%
\pgfpathcurveto{\pgfqpoint{1.377714in}{1.188519in}}{\pgfqpoint{1.385614in}{1.191791in}}{\pgfqpoint{1.391438in}{1.197615in}}%
\pgfpathcurveto{\pgfqpoint{1.397262in}{1.203439in}}{\pgfqpoint{1.400534in}{1.211339in}}{\pgfqpoint{1.400534in}{1.219575in}}%
\pgfpathcurveto{\pgfqpoint{1.400534in}{1.227812in}}{\pgfqpoint{1.397262in}{1.235712in}}{\pgfqpoint{1.391438in}{1.241535in}}%
\pgfpathcurveto{\pgfqpoint{1.385614in}{1.247359in}}{\pgfqpoint{1.377714in}{1.250632in}}{\pgfqpoint{1.369478in}{1.250632in}}%
\pgfpathcurveto{\pgfqpoint{1.361242in}{1.250632in}}{\pgfqpoint{1.353342in}{1.247359in}}{\pgfqpoint{1.347518in}{1.241535in}}%
\pgfpathcurveto{\pgfqpoint{1.341694in}{1.235712in}}{\pgfqpoint{1.338421in}{1.227812in}}{\pgfqpoint{1.338421in}{1.219575in}}%
\pgfpathcurveto{\pgfqpoint{1.338421in}{1.211339in}}{\pgfqpoint{1.341694in}{1.203439in}}{\pgfqpoint{1.347518in}{1.197615in}}%
\pgfpathcurveto{\pgfqpoint{1.353342in}{1.191791in}}{\pgfqpoint{1.361242in}{1.188519in}}{\pgfqpoint{1.369478in}{1.188519in}}%
\pgfpathclose%
\pgfusepath{stroke,fill}%
\end{pgfscope}%
\begin{pgfscope}%
\pgfpathrectangle{\pgfqpoint{0.100000in}{0.212622in}}{\pgfqpoint{3.696000in}{3.696000in}}%
\pgfusepath{clip}%
\pgfsetbuttcap%
\pgfsetroundjoin%
\definecolor{currentfill}{rgb}{0.121569,0.466667,0.705882}%
\pgfsetfillcolor{currentfill}%
\pgfsetfillopacity{0.772466}%
\pgfsetlinewidth{1.003750pt}%
\definecolor{currentstroke}{rgb}{0.121569,0.466667,0.705882}%
\pgfsetstrokecolor{currentstroke}%
\pgfsetstrokeopacity{0.772466}%
\pgfsetdash{}{0pt}%
\pgfpathmoveto{\pgfqpoint{2.239239in}{1.538818in}}%
\pgfpathcurveto{\pgfqpoint{2.247475in}{1.538818in}}{\pgfqpoint{2.255375in}{1.542090in}}{\pgfqpoint{2.261199in}{1.547914in}}%
\pgfpathcurveto{\pgfqpoint{2.267023in}{1.553738in}}{\pgfqpoint{2.270296in}{1.561638in}}{\pgfqpoint{2.270296in}{1.569874in}}%
\pgfpathcurveto{\pgfqpoint{2.270296in}{1.578111in}}{\pgfqpoint{2.267023in}{1.586011in}}{\pgfqpoint{2.261199in}{1.591835in}}%
\pgfpathcurveto{\pgfqpoint{2.255375in}{1.597659in}}{\pgfqpoint{2.247475in}{1.600931in}}{\pgfqpoint{2.239239in}{1.600931in}}%
\pgfpathcurveto{\pgfqpoint{2.231003in}{1.600931in}}{\pgfqpoint{2.223103in}{1.597659in}}{\pgfqpoint{2.217279in}{1.591835in}}%
\pgfpathcurveto{\pgfqpoint{2.211455in}{1.586011in}}{\pgfqpoint{2.208183in}{1.578111in}}{\pgfqpoint{2.208183in}{1.569874in}}%
\pgfpathcurveto{\pgfqpoint{2.208183in}{1.561638in}}{\pgfqpoint{2.211455in}{1.553738in}}{\pgfqpoint{2.217279in}{1.547914in}}%
\pgfpathcurveto{\pgfqpoint{2.223103in}{1.542090in}}{\pgfqpoint{2.231003in}{1.538818in}}{\pgfqpoint{2.239239in}{1.538818in}}%
\pgfpathclose%
\pgfusepath{stroke,fill}%
\end{pgfscope}%
\begin{pgfscope}%
\pgfpathrectangle{\pgfqpoint{0.100000in}{0.212622in}}{\pgfqpoint{3.696000in}{3.696000in}}%
\pgfusepath{clip}%
\pgfsetbuttcap%
\pgfsetroundjoin%
\definecolor{currentfill}{rgb}{0.121569,0.466667,0.705882}%
\pgfsetfillcolor{currentfill}%
\pgfsetfillopacity{0.772783}%
\pgfsetlinewidth{1.003750pt}%
\definecolor{currentstroke}{rgb}{0.121569,0.466667,0.705882}%
\pgfsetstrokecolor{currentstroke}%
\pgfsetstrokeopacity{0.772783}%
\pgfsetdash{}{0pt}%
\pgfpathmoveto{\pgfqpoint{1.378761in}{1.187786in}}%
\pgfpathcurveto{\pgfqpoint{1.386998in}{1.187786in}}{\pgfqpoint{1.394898in}{1.191058in}}{\pgfqpoint{1.400722in}{1.196882in}}%
\pgfpathcurveto{\pgfqpoint{1.406545in}{1.202706in}}{\pgfqpoint{1.409818in}{1.210606in}}{\pgfqpoint{1.409818in}{1.218843in}}%
\pgfpathcurveto{\pgfqpoint{1.409818in}{1.227079in}}{\pgfqpoint{1.406545in}{1.234979in}}{\pgfqpoint{1.400722in}{1.240803in}}%
\pgfpathcurveto{\pgfqpoint{1.394898in}{1.246627in}}{\pgfqpoint{1.386998in}{1.249899in}}{\pgfqpoint{1.378761in}{1.249899in}}%
\pgfpathcurveto{\pgfqpoint{1.370525in}{1.249899in}}{\pgfqpoint{1.362625in}{1.246627in}}{\pgfqpoint{1.356801in}{1.240803in}}%
\pgfpathcurveto{\pgfqpoint{1.350977in}{1.234979in}}{\pgfqpoint{1.347705in}{1.227079in}}{\pgfqpoint{1.347705in}{1.218843in}}%
\pgfpathcurveto{\pgfqpoint{1.347705in}{1.210606in}}{\pgfqpoint{1.350977in}{1.202706in}}{\pgfqpoint{1.356801in}{1.196882in}}%
\pgfpathcurveto{\pgfqpoint{1.362625in}{1.191058in}}{\pgfqpoint{1.370525in}{1.187786in}}{\pgfqpoint{1.378761in}{1.187786in}}%
\pgfpathclose%
\pgfusepath{stroke,fill}%
\end{pgfscope}%
\begin{pgfscope}%
\pgfpathrectangle{\pgfqpoint{0.100000in}{0.212622in}}{\pgfqpoint{3.696000in}{3.696000in}}%
\pgfusepath{clip}%
\pgfsetbuttcap%
\pgfsetroundjoin%
\definecolor{currentfill}{rgb}{0.121569,0.466667,0.705882}%
\pgfsetfillcolor{currentfill}%
\pgfsetfillopacity{0.774762}%
\pgfsetlinewidth{1.003750pt}%
\definecolor{currentstroke}{rgb}{0.121569,0.466667,0.705882}%
\pgfsetstrokecolor{currentstroke}%
\pgfsetstrokeopacity{0.774762}%
\pgfsetdash{}{0pt}%
\pgfpathmoveto{\pgfqpoint{1.386417in}{1.187104in}}%
\pgfpathcurveto{\pgfqpoint{1.394653in}{1.187104in}}{\pgfqpoint{1.402553in}{1.190376in}}{\pgfqpoint{1.408377in}{1.196200in}}%
\pgfpathcurveto{\pgfqpoint{1.414201in}{1.202024in}}{\pgfqpoint{1.417473in}{1.209924in}}{\pgfqpoint{1.417473in}{1.218161in}}%
\pgfpathcurveto{\pgfqpoint{1.417473in}{1.226397in}}{\pgfqpoint{1.414201in}{1.234297in}}{\pgfqpoint{1.408377in}{1.240121in}}%
\pgfpathcurveto{\pgfqpoint{1.402553in}{1.245945in}}{\pgfqpoint{1.394653in}{1.249217in}}{\pgfqpoint{1.386417in}{1.249217in}}%
\pgfpathcurveto{\pgfqpoint{1.378180in}{1.249217in}}{\pgfqpoint{1.370280in}{1.245945in}}{\pgfqpoint{1.364456in}{1.240121in}}%
\pgfpathcurveto{\pgfqpoint{1.358633in}{1.234297in}}{\pgfqpoint{1.355360in}{1.226397in}}{\pgfqpoint{1.355360in}{1.218161in}}%
\pgfpathcurveto{\pgfqpoint{1.355360in}{1.209924in}}{\pgfqpoint{1.358633in}{1.202024in}}{\pgfqpoint{1.364456in}{1.196200in}}%
\pgfpathcurveto{\pgfqpoint{1.370280in}{1.190376in}}{\pgfqpoint{1.378180in}{1.187104in}}{\pgfqpoint{1.386417in}{1.187104in}}%
\pgfpathclose%
\pgfusepath{stroke,fill}%
\end{pgfscope}%
\begin{pgfscope}%
\pgfpathrectangle{\pgfqpoint{0.100000in}{0.212622in}}{\pgfqpoint{3.696000in}{3.696000in}}%
\pgfusepath{clip}%
\pgfsetbuttcap%
\pgfsetroundjoin%
\definecolor{currentfill}{rgb}{0.121569,0.466667,0.705882}%
\pgfsetfillcolor{currentfill}%
\pgfsetfillopacity{0.776307}%
\pgfsetlinewidth{1.003750pt}%
\definecolor{currentstroke}{rgb}{0.121569,0.466667,0.705882}%
\pgfsetstrokecolor{currentstroke}%
\pgfsetstrokeopacity{0.776307}%
\pgfsetdash{}{0pt}%
\pgfpathmoveto{\pgfqpoint{1.393621in}{1.185848in}}%
\pgfpathcurveto{\pgfqpoint{1.401858in}{1.185848in}}{\pgfqpoint{1.409758in}{1.189120in}}{\pgfqpoint{1.415582in}{1.194944in}}%
\pgfpathcurveto{\pgfqpoint{1.421405in}{1.200768in}}{\pgfqpoint{1.424678in}{1.208668in}}{\pgfqpoint{1.424678in}{1.216904in}}%
\pgfpathcurveto{\pgfqpoint{1.424678in}{1.225141in}}{\pgfqpoint{1.421405in}{1.233041in}}{\pgfqpoint{1.415582in}{1.238865in}}%
\pgfpathcurveto{\pgfqpoint{1.409758in}{1.244688in}}{\pgfqpoint{1.401858in}{1.247961in}}{\pgfqpoint{1.393621in}{1.247961in}}%
\pgfpathcurveto{\pgfqpoint{1.385385in}{1.247961in}}{\pgfqpoint{1.377485in}{1.244688in}}{\pgfqpoint{1.371661in}{1.238865in}}%
\pgfpathcurveto{\pgfqpoint{1.365837in}{1.233041in}}{\pgfqpoint{1.362565in}{1.225141in}}{\pgfqpoint{1.362565in}{1.216904in}}%
\pgfpathcurveto{\pgfqpoint{1.362565in}{1.208668in}}{\pgfqpoint{1.365837in}{1.200768in}}{\pgfqpoint{1.371661in}{1.194944in}}%
\pgfpathcurveto{\pgfqpoint{1.377485in}{1.189120in}}{\pgfqpoint{1.385385in}{1.185848in}}{\pgfqpoint{1.393621in}{1.185848in}}%
\pgfpathclose%
\pgfusepath{stroke,fill}%
\end{pgfscope}%
\begin{pgfscope}%
\pgfpathrectangle{\pgfqpoint{0.100000in}{0.212622in}}{\pgfqpoint{3.696000in}{3.696000in}}%
\pgfusepath{clip}%
\pgfsetbuttcap%
\pgfsetroundjoin%
\definecolor{currentfill}{rgb}{0.121569,0.466667,0.705882}%
\pgfsetfillcolor{currentfill}%
\pgfsetfillopacity{0.776647}%
\pgfsetlinewidth{1.003750pt}%
\definecolor{currentstroke}{rgb}{0.121569,0.466667,0.705882}%
\pgfsetstrokecolor{currentstroke}%
\pgfsetstrokeopacity{0.776647}%
\pgfsetdash{}{0pt}%
\pgfpathmoveto{\pgfqpoint{2.242705in}{1.524434in}}%
\pgfpathcurveto{\pgfqpoint{2.250942in}{1.524434in}}{\pgfqpoint{2.258842in}{1.527707in}}{\pgfqpoint{2.264666in}{1.533530in}}%
\pgfpathcurveto{\pgfqpoint{2.270490in}{1.539354in}}{\pgfqpoint{2.273762in}{1.547254in}}{\pgfqpoint{2.273762in}{1.555491in}}%
\pgfpathcurveto{\pgfqpoint{2.273762in}{1.563727in}}{\pgfqpoint{2.270490in}{1.571627in}}{\pgfqpoint{2.264666in}{1.577451in}}%
\pgfpathcurveto{\pgfqpoint{2.258842in}{1.583275in}}{\pgfqpoint{2.250942in}{1.586547in}}{\pgfqpoint{2.242705in}{1.586547in}}%
\pgfpathcurveto{\pgfqpoint{2.234469in}{1.586547in}}{\pgfqpoint{2.226569in}{1.583275in}}{\pgfqpoint{2.220745in}{1.577451in}}%
\pgfpathcurveto{\pgfqpoint{2.214921in}{1.571627in}}{\pgfqpoint{2.211649in}{1.563727in}}{\pgfqpoint{2.211649in}{1.555491in}}%
\pgfpathcurveto{\pgfqpoint{2.211649in}{1.547254in}}{\pgfqpoint{2.214921in}{1.539354in}}{\pgfqpoint{2.220745in}{1.533530in}}%
\pgfpathcurveto{\pgfqpoint{2.226569in}{1.527707in}}{\pgfqpoint{2.234469in}{1.524434in}}{\pgfqpoint{2.242705in}{1.524434in}}%
\pgfpathclose%
\pgfusepath{stroke,fill}%
\end{pgfscope}%
\begin{pgfscope}%
\pgfpathrectangle{\pgfqpoint{0.100000in}{0.212622in}}{\pgfqpoint{3.696000in}{3.696000in}}%
\pgfusepath{clip}%
\pgfsetbuttcap%
\pgfsetroundjoin%
\definecolor{currentfill}{rgb}{0.121569,0.466667,0.705882}%
\pgfsetfillcolor{currentfill}%
\pgfsetfillopacity{0.777856}%
\pgfsetlinewidth{1.003750pt}%
\definecolor{currentstroke}{rgb}{0.121569,0.466667,0.705882}%
\pgfsetstrokecolor{currentstroke}%
\pgfsetstrokeopacity{0.777856}%
\pgfsetdash{}{0pt}%
\pgfpathmoveto{\pgfqpoint{1.400390in}{1.184368in}}%
\pgfpathcurveto{\pgfqpoint{1.408626in}{1.184368in}}{\pgfqpoint{1.416527in}{1.187641in}}{\pgfqpoint{1.422350in}{1.193464in}}%
\pgfpathcurveto{\pgfqpoint{1.428174in}{1.199288in}}{\pgfqpoint{1.431447in}{1.207188in}}{\pgfqpoint{1.431447in}{1.215425in}}%
\pgfpathcurveto{\pgfqpoint{1.431447in}{1.223661in}}{\pgfqpoint{1.428174in}{1.231561in}}{\pgfqpoint{1.422350in}{1.237385in}}%
\pgfpathcurveto{\pgfqpoint{1.416527in}{1.243209in}}{\pgfqpoint{1.408626in}{1.246481in}}{\pgfqpoint{1.400390in}{1.246481in}}%
\pgfpathcurveto{\pgfqpoint{1.392154in}{1.246481in}}{\pgfqpoint{1.384254in}{1.243209in}}{\pgfqpoint{1.378430in}{1.237385in}}%
\pgfpathcurveto{\pgfqpoint{1.372606in}{1.231561in}}{\pgfqpoint{1.369334in}{1.223661in}}{\pgfqpoint{1.369334in}{1.215425in}}%
\pgfpathcurveto{\pgfqpoint{1.369334in}{1.207188in}}{\pgfqpoint{1.372606in}{1.199288in}}{\pgfqpoint{1.378430in}{1.193464in}}%
\pgfpathcurveto{\pgfqpoint{1.384254in}{1.187641in}}{\pgfqpoint{1.392154in}{1.184368in}}{\pgfqpoint{1.400390in}{1.184368in}}%
\pgfpathclose%
\pgfusepath{stroke,fill}%
\end{pgfscope}%
\begin{pgfscope}%
\pgfpathrectangle{\pgfqpoint{0.100000in}{0.212622in}}{\pgfqpoint{3.696000in}{3.696000in}}%
\pgfusepath{clip}%
\pgfsetbuttcap%
\pgfsetroundjoin%
\definecolor{currentfill}{rgb}{0.121569,0.466667,0.705882}%
\pgfsetfillcolor{currentfill}%
\pgfsetfillopacity{0.779018}%
\pgfsetlinewidth{1.003750pt}%
\definecolor{currentstroke}{rgb}{0.121569,0.466667,0.705882}%
\pgfsetstrokecolor{currentstroke}%
\pgfsetstrokeopacity{0.779018}%
\pgfsetdash{}{0pt}%
\pgfpathmoveto{\pgfqpoint{1.405509in}{1.183113in}}%
\pgfpathcurveto{\pgfqpoint{1.413745in}{1.183113in}}{\pgfqpoint{1.421645in}{1.186385in}}{\pgfqpoint{1.427469in}{1.192209in}}%
\pgfpathcurveto{\pgfqpoint{1.433293in}{1.198033in}}{\pgfqpoint{1.436565in}{1.205933in}}{\pgfqpoint{1.436565in}{1.214170in}}%
\pgfpathcurveto{\pgfqpoint{1.436565in}{1.222406in}}{\pgfqpoint{1.433293in}{1.230306in}}{\pgfqpoint{1.427469in}{1.236130in}}%
\pgfpathcurveto{\pgfqpoint{1.421645in}{1.241954in}}{\pgfqpoint{1.413745in}{1.245226in}}{\pgfqpoint{1.405509in}{1.245226in}}%
\pgfpathcurveto{\pgfqpoint{1.397273in}{1.245226in}}{\pgfqpoint{1.389373in}{1.241954in}}{\pgfqpoint{1.383549in}{1.236130in}}%
\pgfpathcurveto{\pgfqpoint{1.377725in}{1.230306in}}{\pgfqpoint{1.374452in}{1.222406in}}{\pgfqpoint{1.374452in}{1.214170in}}%
\pgfpathcurveto{\pgfqpoint{1.374452in}{1.205933in}}{\pgfqpoint{1.377725in}{1.198033in}}{\pgfqpoint{1.383549in}{1.192209in}}%
\pgfpathcurveto{\pgfqpoint{1.389373in}{1.186385in}}{\pgfqpoint{1.397273in}{1.183113in}}{\pgfqpoint{1.405509in}{1.183113in}}%
\pgfpathclose%
\pgfusepath{stroke,fill}%
\end{pgfscope}%
\begin{pgfscope}%
\pgfpathrectangle{\pgfqpoint{0.100000in}{0.212622in}}{\pgfqpoint{3.696000in}{3.696000in}}%
\pgfusepath{clip}%
\pgfsetbuttcap%
\pgfsetroundjoin%
\definecolor{currentfill}{rgb}{0.121569,0.466667,0.705882}%
\pgfsetfillcolor{currentfill}%
\pgfsetfillopacity{0.781001}%
\pgfsetlinewidth{1.003750pt}%
\definecolor{currentstroke}{rgb}{0.121569,0.466667,0.705882}%
\pgfsetstrokecolor{currentstroke}%
\pgfsetstrokeopacity{0.781001}%
\pgfsetdash{}{0pt}%
\pgfpathmoveto{\pgfqpoint{1.414744in}{1.180050in}}%
\pgfpathcurveto{\pgfqpoint{1.422980in}{1.180050in}}{\pgfqpoint{1.430880in}{1.183322in}}{\pgfqpoint{1.436704in}{1.189146in}}%
\pgfpathcurveto{\pgfqpoint{1.442528in}{1.194970in}}{\pgfqpoint{1.445801in}{1.202870in}}{\pgfqpoint{1.445801in}{1.211106in}}%
\pgfpathcurveto{\pgfqpoint{1.445801in}{1.219342in}}{\pgfqpoint{1.442528in}{1.227242in}}{\pgfqpoint{1.436704in}{1.233066in}}%
\pgfpathcurveto{\pgfqpoint{1.430880in}{1.238890in}}{\pgfqpoint{1.422980in}{1.242163in}}{\pgfqpoint{1.414744in}{1.242163in}}%
\pgfpathcurveto{\pgfqpoint{1.406508in}{1.242163in}}{\pgfqpoint{1.398608in}{1.238890in}}{\pgfqpoint{1.392784in}{1.233066in}}%
\pgfpathcurveto{\pgfqpoint{1.386960in}{1.227242in}}{\pgfqpoint{1.383688in}{1.219342in}}{\pgfqpoint{1.383688in}{1.211106in}}%
\pgfpathcurveto{\pgfqpoint{1.383688in}{1.202870in}}{\pgfqpoint{1.386960in}{1.194970in}}{\pgfqpoint{1.392784in}{1.189146in}}%
\pgfpathcurveto{\pgfqpoint{1.398608in}{1.183322in}}{\pgfqpoint{1.406508in}{1.180050in}}{\pgfqpoint{1.414744in}{1.180050in}}%
\pgfpathclose%
\pgfusepath{stroke,fill}%
\end{pgfscope}%
\begin{pgfscope}%
\pgfpathrectangle{\pgfqpoint{0.100000in}{0.212622in}}{\pgfqpoint{3.696000in}{3.696000in}}%
\pgfusepath{clip}%
\pgfsetbuttcap%
\pgfsetroundjoin%
\definecolor{currentfill}{rgb}{0.121569,0.466667,0.705882}%
\pgfsetfillcolor{currentfill}%
\pgfsetfillopacity{0.781144}%
\pgfsetlinewidth{1.003750pt}%
\definecolor{currentstroke}{rgb}{0.121569,0.466667,0.705882}%
\pgfsetstrokecolor{currentstroke}%
\pgfsetstrokeopacity{0.781144}%
\pgfsetdash{}{0pt}%
\pgfpathmoveto{\pgfqpoint{2.246151in}{1.509296in}}%
\pgfpathcurveto{\pgfqpoint{2.254387in}{1.509296in}}{\pgfqpoint{2.262287in}{1.512569in}}{\pgfqpoint{2.268111in}{1.518392in}}%
\pgfpathcurveto{\pgfqpoint{2.273935in}{1.524216in}}{\pgfqpoint{2.277207in}{1.532116in}}{\pgfqpoint{2.277207in}{1.540353in}}%
\pgfpathcurveto{\pgfqpoint{2.277207in}{1.548589in}}{\pgfqpoint{2.273935in}{1.556489in}}{\pgfqpoint{2.268111in}{1.562313in}}%
\pgfpathcurveto{\pgfqpoint{2.262287in}{1.568137in}}{\pgfqpoint{2.254387in}{1.571409in}}{\pgfqpoint{2.246151in}{1.571409in}}%
\pgfpathcurveto{\pgfqpoint{2.237915in}{1.571409in}}{\pgfqpoint{2.230015in}{1.568137in}}{\pgfqpoint{2.224191in}{1.562313in}}%
\pgfpathcurveto{\pgfqpoint{2.218367in}{1.556489in}}{\pgfqpoint{2.215094in}{1.548589in}}{\pgfqpoint{2.215094in}{1.540353in}}%
\pgfpathcurveto{\pgfqpoint{2.215094in}{1.532116in}}{\pgfqpoint{2.218367in}{1.524216in}}{\pgfqpoint{2.224191in}{1.518392in}}%
\pgfpathcurveto{\pgfqpoint{2.230015in}{1.512569in}}{\pgfqpoint{2.237915in}{1.509296in}}{\pgfqpoint{2.246151in}{1.509296in}}%
\pgfpathclose%
\pgfusepath{stroke,fill}%
\end{pgfscope}%
\begin{pgfscope}%
\pgfpathrectangle{\pgfqpoint{0.100000in}{0.212622in}}{\pgfqpoint{3.696000in}{3.696000in}}%
\pgfusepath{clip}%
\pgfsetbuttcap%
\pgfsetroundjoin%
\definecolor{currentfill}{rgb}{0.121569,0.466667,0.705882}%
\pgfsetfillcolor{currentfill}%
\pgfsetfillopacity{0.782688}%
\pgfsetlinewidth{1.003750pt}%
\definecolor{currentstroke}{rgb}{0.121569,0.466667,0.705882}%
\pgfsetstrokecolor{currentstroke}%
\pgfsetstrokeopacity{0.782688}%
\pgfsetdash{}{0pt}%
\pgfpathmoveto{\pgfqpoint{1.422330in}{1.177236in}}%
\pgfpathcurveto{\pgfqpoint{1.430566in}{1.177236in}}{\pgfqpoint{1.438466in}{1.180508in}}{\pgfqpoint{1.444290in}{1.186332in}}%
\pgfpathcurveto{\pgfqpoint{1.450114in}{1.192156in}}{\pgfqpoint{1.453386in}{1.200056in}}{\pgfqpoint{1.453386in}{1.208293in}}%
\pgfpathcurveto{\pgfqpoint{1.453386in}{1.216529in}}{\pgfqpoint{1.450114in}{1.224429in}}{\pgfqpoint{1.444290in}{1.230253in}}%
\pgfpathcurveto{\pgfqpoint{1.438466in}{1.236077in}}{\pgfqpoint{1.430566in}{1.239349in}}{\pgfqpoint{1.422330in}{1.239349in}}%
\pgfpathcurveto{\pgfqpoint{1.414093in}{1.239349in}}{\pgfqpoint{1.406193in}{1.236077in}}{\pgfqpoint{1.400369in}{1.230253in}}%
\pgfpathcurveto{\pgfqpoint{1.394545in}{1.224429in}}{\pgfqpoint{1.391273in}{1.216529in}}{\pgfqpoint{1.391273in}{1.208293in}}%
\pgfpathcurveto{\pgfqpoint{1.391273in}{1.200056in}}{\pgfqpoint{1.394545in}{1.192156in}}{\pgfqpoint{1.400369in}{1.186332in}}%
\pgfpathcurveto{\pgfqpoint{1.406193in}{1.180508in}}{\pgfqpoint{1.414093in}{1.177236in}}{\pgfqpoint{1.422330in}{1.177236in}}%
\pgfpathclose%
\pgfusepath{stroke,fill}%
\end{pgfscope}%
\begin{pgfscope}%
\pgfpathrectangle{\pgfqpoint{0.100000in}{0.212622in}}{\pgfqpoint{3.696000in}{3.696000in}}%
\pgfusepath{clip}%
\pgfsetbuttcap%
\pgfsetroundjoin%
\definecolor{currentfill}{rgb}{0.121569,0.466667,0.705882}%
\pgfsetfillcolor{currentfill}%
\pgfsetfillopacity{0.784025}%
\pgfsetlinewidth{1.003750pt}%
\definecolor{currentstroke}{rgb}{0.121569,0.466667,0.705882}%
\pgfsetstrokecolor{currentstroke}%
\pgfsetstrokeopacity{0.784025}%
\pgfsetdash{}{0pt}%
\pgfpathmoveto{\pgfqpoint{1.428580in}{1.174533in}}%
\pgfpathcurveto{\pgfqpoint{1.436817in}{1.174533in}}{\pgfqpoint{1.444717in}{1.177806in}}{\pgfqpoint{1.450541in}{1.183630in}}%
\pgfpathcurveto{\pgfqpoint{1.456364in}{1.189454in}}{\pgfqpoint{1.459637in}{1.197354in}}{\pgfqpoint{1.459637in}{1.205590in}}%
\pgfpathcurveto{\pgfqpoint{1.459637in}{1.213826in}}{\pgfqpoint{1.456364in}{1.221726in}}{\pgfqpoint{1.450541in}{1.227550in}}%
\pgfpathcurveto{\pgfqpoint{1.444717in}{1.233374in}}{\pgfqpoint{1.436817in}{1.236646in}}{\pgfqpoint{1.428580in}{1.236646in}}%
\pgfpathcurveto{\pgfqpoint{1.420344in}{1.236646in}}{\pgfqpoint{1.412444in}{1.233374in}}{\pgfqpoint{1.406620in}{1.227550in}}%
\pgfpathcurveto{\pgfqpoint{1.400796in}{1.221726in}}{\pgfqpoint{1.397524in}{1.213826in}}{\pgfqpoint{1.397524in}{1.205590in}}%
\pgfpathcurveto{\pgfqpoint{1.397524in}{1.197354in}}{\pgfqpoint{1.400796in}{1.189454in}}{\pgfqpoint{1.406620in}{1.183630in}}%
\pgfpathcurveto{\pgfqpoint{1.412444in}{1.177806in}}{\pgfqpoint{1.420344in}{1.174533in}}{\pgfqpoint{1.428580in}{1.174533in}}%
\pgfpathclose%
\pgfusepath{stroke,fill}%
\end{pgfscope}%
\begin{pgfscope}%
\pgfpathrectangle{\pgfqpoint{0.100000in}{0.212622in}}{\pgfqpoint{3.696000in}{3.696000in}}%
\pgfusepath{clip}%
\pgfsetbuttcap%
\pgfsetroundjoin%
\definecolor{currentfill}{rgb}{0.121569,0.466667,0.705882}%
\pgfsetfillcolor{currentfill}%
\pgfsetfillopacity{0.785686}%
\pgfsetlinewidth{1.003750pt}%
\definecolor{currentstroke}{rgb}{0.121569,0.466667,0.705882}%
\pgfsetstrokecolor{currentstroke}%
\pgfsetstrokeopacity{0.785686}%
\pgfsetdash{}{0pt}%
\pgfpathmoveto{\pgfqpoint{2.251202in}{1.493562in}}%
\pgfpathcurveto{\pgfqpoint{2.259439in}{1.493562in}}{\pgfqpoint{2.267339in}{1.496834in}}{\pgfqpoint{2.273163in}{1.502658in}}%
\pgfpathcurveto{\pgfqpoint{2.278987in}{1.508482in}}{\pgfqpoint{2.282259in}{1.516382in}}{\pgfqpoint{2.282259in}{1.524618in}}%
\pgfpathcurveto{\pgfqpoint{2.282259in}{1.532854in}}{\pgfqpoint{2.278987in}{1.540754in}}{\pgfqpoint{2.273163in}{1.546578in}}%
\pgfpathcurveto{\pgfqpoint{2.267339in}{1.552402in}}{\pgfqpoint{2.259439in}{1.555675in}}{\pgfqpoint{2.251202in}{1.555675in}}%
\pgfpathcurveto{\pgfqpoint{2.242966in}{1.555675in}}{\pgfqpoint{2.235066in}{1.552402in}}{\pgfqpoint{2.229242in}{1.546578in}}%
\pgfpathcurveto{\pgfqpoint{2.223418in}{1.540754in}}{\pgfqpoint{2.220146in}{1.532854in}}{\pgfqpoint{2.220146in}{1.524618in}}%
\pgfpathcurveto{\pgfqpoint{2.220146in}{1.516382in}}{\pgfqpoint{2.223418in}{1.508482in}}{\pgfqpoint{2.229242in}{1.502658in}}%
\pgfpathcurveto{\pgfqpoint{2.235066in}{1.496834in}}{\pgfqpoint{2.242966in}{1.493562in}}{\pgfqpoint{2.251202in}{1.493562in}}%
\pgfpathclose%
\pgfusepath{stroke,fill}%
\end{pgfscope}%
\begin{pgfscope}%
\pgfpathrectangle{\pgfqpoint{0.100000in}{0.212622in}}{\pgfqpoint{3.696000in}{3.696000in}}%
\pgfusepath{clip}%
\pgfsetbuttcap%
\pgfsetroundjoin%
\definecolor{currentfill}{rgb}{0.121569,0.466667,0.705882}%
\pgfsetfillcolor{currentfill}%
\pgfsetfillopacity{0.786509}%
\pgfsetlinewidth{1.003750pt}%
\definecolor{currentstroke}{rgb}{0.121569,0.466667,0.705882}%
\pgfsetstrokecolor{currentstroke}%
\pgfsetstrokeopacity{0.786509}%
\pgfsetdash{}{0pt}%
\pgfpathmoveto{\pgfqpoint{1.439972in}{1.169892in}}%
\pgfpathcurveto{\pgfqpoint{1.448208in}{1.169892in}}{\pgfqpoint{1.456108in}{1.173165in}}{\pgfqpoint{1.461932in}{1.178988in}}%
\pgfpathcurveto{\pgfqpoint{1.467756in}{1.184812in}}{\pgfqpoint{1.471028in}{1.192712in}}{\pgfqpoint{1.471028in}{1.200949in}}%
\pgfpathcurveto{\pgfqpoint{1.471028in}{1.209185in}}{\pgfqpoint{1.467756in}{1.217085in}}{\pgfqpoint{1.461932in}{1.222909in}}%
\pgfpathcurveto{\pgfqpoint{1.456108in}{1.228733in}}{\pgfqpoint{1.448208in}{1.232005in}}{\pgfqpoint{1.439972in}{1.232005in}}%
\pgfpathcurveto{\pgfqpoint{1.431736in}{1.232005in}}{\pgfqpoint{1.423835in}{1.228733in}}{\pgfqpoint{1.418012in}{1.222909in}}%
\pgfpathcurveto{\pgfqpoint{1.412188in}{1.217085in}}{\pgfqpoint{1.408915in}{1.209185in}}{\pgfqpoint{1.408915in}{1.200949in}}%
\pgfpathcurveto{\pgfqpoint{1.408915in}{1.192712in}}{\pgfqpoint{1.412188in}{1.184812in}}{\pgfqpoint{1.418012in}{1.178988in}}%
\pgfpathcurveto{\pgfqpoint{1.423835in}{1.173165in}}{\pgfqpoint{1.431736in}{1.169892in}}{\pgfqpoint{1.439972in}{1.169892in}}%
\pgfpathclose%
\pgfusepath{stroke,fill}%
\end{pgfscope}%
\begin{pgfscope}%
\pgfpathrectangle{\pgfqpoint{0.100000in}{0.212622in}}{\pgfqpoint{3.696000in}{3.696000in}}%
\pgfusepath{clip}%
\pgfsetbuttcap%
\pgfsetroundjoin%
\definecolor{currentfill}{rgb}{0.121569,0.466667,0.705882}%
\pgfsetfillcolor{currentfill}%
\pgfsetfillopacity{0.788806}%
\pgfsetlinewidth{1.003750pt}%
\definecolor{currentstroke}{rgb}{0.121569,0.466667,0.705882}%
\pgfsetstrokecolor{currentstroke}%
\pgfsetstrokeopacity{0.788806}%
\pgfsetdash{}{0pt}%
\pgfpathmoveto{\pgfqpoint{1.449815in}{1.166000in}}%
\pgfpathcurveto{\pgfqpoint{1.458051in}{1.166000in}}{\pgfqpoint{1.465951in}{1.169272in}}{\pgfqpoint{1.471775in}{1.175096in}}%
\pgfpathcurveto{\pgfqpoint{1.477599in}{1.180920in}}{\pgfqpoint{1.480871in}{1.188820in}}{\pgfqpoint{1.480871in}{1.197056in}}%
\pgfpathcurveto{\pgfqpoint{1.480871in}{1.205292in}}{\pgfqpoint{1.477599in}{1.213192in}}{\pgfqpoint{1.471775in}{1.219016in}}%
\pgfpathcurveto{\pgfqpoint{1.465951in}{1.224840in}}{\pgfqpoint{1.458051in}{1.228113in}}{\pgfqpoint{1.449815in}{1.228113in}}%
\pgfpathcurveto{\pgfqpoint{1.441578in}{1.228113in}}{\pgfqpoint{1.433678in}{1.224840in}}{\pgfqpoint{1.427854in}{1.219016in}}%
\pgfpathcurveto{\pgfqpoint{1.422030in}{1.213192in}}{\pgfqpoint{1.418758in}{1.205292in}}{\pgfqpoint{1.418758in}{1.197056in}}%
\pgfpathcurveto{\pgfqpoint{1.418758in}{1.188820in}}{\pgfqpoint{1.422030in}{1.180920in}}{\pgfqpoint{1.427854in}{1.175096in}}%
\pgfpathcurveto{\pgfqpoint{1.433678in}{1.169272in}}{\pgfqpoint{1.441578in}{1.166000in}}{\pgfqpoint{1.449815in}{1.166000in}}%
\pgfpathclose%
\pgfusepath{stroke,fill}%
\end{pgfscope}%
\begin{pgfscope}%
\pgfpathrectangle{\pgfqpoint{0.100000in}{0.212622in}}{\pgfqpoint{3.696000in}{3.696000in}}%
\pgfusepath{clip}%
\pgfsetbuttcap%
\pgfsetroundjoin%
\definecolor{currentfill}{rgb}{0.121569,0.466667,0.705882}%
\pgfsetfillcolor{currentfill}%
\pgfsetfillopacity{0.790582}%
\pgfsetlinewidth{1.003750pt}%
\definecolor{currentstroke}{rgb}{0.121569,0.466667,0.705882}%
\pgfsetstrokecolor{currentstroke}%
\pgfsetstrokeopacity{0.790582}%
\pgfsetdash{}{0pt}%
\pgfpathmoveto{\pgfqpoint{2.255697in}{1.477758in}}%
\pgfpathcurveto{\pgfqpoint{2.263933in}{1.477758in}}{\pgfqpoint{2.271833in}{1.481031in}}{\pgfqpoint{2.277657in}{1.486855in}}%
\pgfpathcurveto{\pgfqpoint{2.283481in}{1.492678in}}{\pgfqpoint{2.286753in}{1.500578in}}{\pgfqpoint{2.286753in}{1.508815in}}%
\pgfpathcurveto{\pgfqpoint{2.286753in}{1.517051in}}{\pgfqpoint{2.283481in}{1.524951in}}{\pgfqpoint{2.277657in}{1.530775in}}%
\pgfpathcurveto{\pgfqpoint{2.271833in}{1.536599in}}{\pgfqpoint{2.263933in}{1.539871in}}{\pgfqpoint{2.255697in}{1.539871in}}%
\pgfpathcurveto{\pgfqpoint{2.247461in}{1.539871in}}{\pgfqpoint{2.239561in}{1.536599in}}{\pgfqpoint{2.233737in}{1.530775in}}%
\pgfpathcurveto{\pgfqpoint{2.227913in}{1.524951in}}{\pgfqpoint{2.224640in}{1.517051in}}{\pgfqpoint{2.224640in}{1.508815in}}%
\pgfpathcurveto{\pgfqpoint{2.224640in}{1.500578in}}{\pgfqpoint{2.227913in}{1.492678in}}{\pgfqpoint{2.233737in}{1.486855in}}%
\pgfpathcurveto{\pgfqpoint{2.239561in}{1.481031in}}{\pgfqpoint{2.247461in}{1.477758in}}{\pgfqpoint{2.255697in}{1.477758in}}%
\pgfpathclose%
\pgfusepath{stroke,fill}%
\end{pgfscope}%
\begin{pgfscope}%
\pgfpathrectangle{\pgfqpoint{0.100000in}{0.212622in}}{\pgfqpoint{3.696000in}{3.696000in}}%
\pgfusepath{clip}%
\pgfsetbuttcap%
\pgfsetroundjoin%
\definecolor{currentfill}{rgb}{0.121569,0.466667,0.705882}%
\pgfsetfillcolor{currentfill}%
\pgfsetfillopacity{0.790957}%
\pgfsetlinewidth{1.003750pt}%
\definecolor{currentstroke}{rgb}{0.121569,0.466667,0.705882}%
\pgfsetstrokecolor{currentstroke}%
\pgfsetstrokeopacity{0.790957}%
\pgfsetdash{}{0pt}%
\pgfpathmoveto{\pgfqpoint{1.459176in}{1.161882in}}%
\pgfpathcurveto{\pgfqpoint{1.467412in}{1.161882in}}{\pgfqpoint{1.475312in}{1.165154in}}{\pgfqpoint{1.481136in}{1.170978in}}%
\pgfpathcurveto{\pgfqpoint{1.486960in}{1.176802in}}{\pgfqpoint{1.490232in}{1.184702in}}{\pgfqpoint{1.490232in}{1.192938in}}%
\pgfpathcurveto{\pgfqpoint{1.490232in}{1.201175in}}{\pgfqpoint{1.486960in}{1.209075in}}{\pgfqpoint{1.481136in}{1.214899in}}%
\pgfpathcurveto{\pgfqpoint{1.475312in}{1.220723in}}{\pgfqpoint{1.467412in}{1.223995in}}{\pgfqpoint{1.459176in}{1.223995in}}%
\pgfpathcurveto{\pgfqpoint{1.450939in}{1.223995in}}{\pgfqpoint{1.443039in}{1.220723in}}{\pgfqpoint{1.437215in}{1.214899in}}%
\pgfpathcurveto{\pgfqpoint{1.431391in}{1.209075in}}{\pgfqpoint{1.428119in}{1.201175in}}{\pgfqpoint{1.428119in}{1.192938in}}%
\pgfpathcurveto{\pgfqpoint{1.428119in}{1.184702in}}{\pgfqpoint{1.431391in}{1.176802in}}{\pgfqpoint{1.437215in}{1.170978in}}%
\pgfpathcurveto{\pgfqpoint{1.443039in}{1.165154in}}{\pgfqpoint{1.450939in}{1.161882in}}{\pgfqpoint{1.459176in}{1.161882in}}%
\pgfpathclose%
\pgfusepath{stroke,fill}%
\end{pgfscope}%
\begin{pgfscope}%
\pgfpathrectangle{\pgfqpoint{0.100000in}{0.212622in}}{\pgfqpoint{3.696000in}{3.696000in}}%
\pgfusepath{clip}%
\pgfsetbuttcap%
\pgfsetroundjoin%
\definecolor{currentfill}{rgb}{0.121569,0.466667,0.705882}%
\pgfsetfillcolor{currentfill}%
\pgfsetfillopacity{0.792936}%
\pgfsetlinewidth{1.003750pt}%
\definecolor{currentstroke}{rgb}{0.121569,0.466667,0.705882}%
\pgfsetstrokecolor{currentstroke}%
\pgfsetstrokeopacity{0.792936}%
\pgfsetdash{}{0pt}%
\pgfpathmoveto{\pgfqpoint{1.467916in}{1.158940in}}%
\pgfpathcurveto{\pgfqpoint{1.476152in}{1.158940in}}{\pgfqpoint{1.484052in}{1.162213in}}{\pgfqpoint{1.489876in}{1.168037in}}%
\pgfpathcurveto{\pgfqpoint{1.495700in}{1.173861in}}{\pgfqpoint{1.498972in}{1.181761in}}{\pgfqpoint{1.498972in}{1.189997in}}%
\pgfpathcurveto{\pgfqpoint{1.498972in}{1.198233in}}{\pgfqpoint{1.495700in}{1.206133in}}{\pgfqpoint{1.489876in}{1.211957in}}%
\pgfpathcurveto{\pgfqpoint{1.484052in}{1.217781in}}{\pgfqpoint{1.476152in}{1.221053in}}{\pgfqpoint{1.467916in}{1.221053in}}%
\pgfpathcurveto{\pgfqpoint{1.459680in}{1.221053in}}{\pgfqpoint{1.451780in}{1.217781in}}{\pgfqpoint{1.445956in}{1.211957in}}%
\pgfpathcurveto{\pgfqpoint{1.440132in}{1.206133in}}{\pgfqpoint{1.436859in}{1.198233in}}{\pgfqpoint{1.436859in}{1.189997in}}%
\pgfpathcurveto{\pgfqpoint{1.436859in}{1.181761in}}{\pgfqpoint{1.440132in}{1.173861in}}{\pgfqpoint{1.445956in}{1.168037in}}%
\pgfpathcurveto{\pgfqpoint{1.451780in}{1.162213in}}{\pgfqpoint{1.459680in}{1.158940in}}{\pgfqpoint{1.467916in}{1.158940in}}%
\pgfpathclose%
\pgfusepath{stroke,fill}%
\end{pgfscope}%
\begin{pgfscope}%
\pgfpathrectangle{\pgfqpoint{0.100000in}{0.212622in}}{\pgfqpoint{3.696000in}{3.696000in}}%
\pgfusepath{clip}%
\pgfsetbuttcap%
\pgfsetroundjoin%
\definecolor{currentfill}{rgb}{0.121569,0.466667,0.705882}%
\pgfsetfillcolor{currentfill}%
\pgfsetfillopacity{0.794948}%
\pgfsetlinewidth{1.003750pt}%
\definecolor{currentstroke}{rgb}{0.121569,0.466667,0.705882}%
\pgfsetstrokecolor{currentstroke}%
\pgfsetstrokeopacity{0.794948}%
\pgfsetdash{}{0pt}%
\pgfpathmoveto{\pgfqpoint{1.475634in}{1.156593in}}%
\pgfpathcurveto{\pgfqpoint{1.483870in}{1.156593in}}{\pgfqpoint{1.491770in}{1.159865in}}{\pgfqpoint{1.497594in}{1.165689in}}%
\pgfpathcurveto{\pgfqpoint{1.503418in}{1.171513in}}{\pgfqpoint{1.506690in}{1.179413in}}{\pgfqpoint{1.506690in}{1.187650in}}%
\pgfpathcurveto{\pgfqpoint{1.506690in}{1.195886in}}{\pgfqpoint{1.503418in}{1.203786in}}{\pgfqpoint{1.497594in}{1.209610in}}%
\pgfpathcurveto{\pgfqpoint{1.491770in}{1.215434in}}{\pgfqpoint{1.483870in}{1.218706in}}{\pgfqpoint{1.475634in}{1.218706in}}%
\pgfpathcurveto{\pgfqpoint{1.467397in}{1.218706in}}{\pgfqpoint{1.459497in}{1.215434in}}{\pgfqpoint{1.453673in}{1.209610in}}%
\pgfpathcurveto{\pgfqpoint{1.447850in}{1.203786in}}{\pgfqpoint{1.444577in}{1.195886in}}{\pgfqpoint{1.444577in}{1.187650in}}%
\pgfpathcurveto{\pgfqpoint{1.444577in}{1.179413in}}{\pgfqpoint{1.447850in}{1.171513in}}{\pgfqpoint{1.453673in}{1.165689in}}%
\pgfpathcurveto{\pgfqpoint{1.459497in}{1.159865in}}{\pgfqpoint{1.467397in}{1.156593in}}{\pgfqpoint{1.475634in}{1.156593in}}%
\pgfpathclose%
\pgfusepath{stroke,fill}%
\end{pgfscope}%
\begin{pgfscope}%
\pgfpathrectangle{\pgfqpoint{0.100000in}{0.212622in}}{\pgfqpoint{3.696000in}{3.696000in}}%
\pgfusepath{clip}%
\pgfsetbuttcap%
\pgfsetroundjoin%
\definecolor{currentfill}{rgb}{0.121569,0.466667,0.705882}%
\pgfsetfillcolor{currentfill}%
\pgfsetfillopacity{0.795912}%
\pgfsetlinewidth{1.003750pt}%
\definecolor{currentstroke}{rgb}{0.121569,0.466667,0.705882}%
\pgfsetstrokecolor{currentstroke}%
\pgfsetstrokeopacity{0.795912}%
\pgfsetdash{}{0pt}%
\pgfpathmoveto{\pgfqpoint{2.259505in}{1.461166in}}%
\pgfpathcurveto{\pgfqpoint{2.267742in}{1.461166in}}{\pgfqpoint{2.275642in}{1.464439in}}{\pgfqpoint{2.281466in}{1.470263in}}%
\pgfpathcurveto{\pgfqpoint{2.287290in}{1.476086in}}{\pgfqpoint{2.290562in}{1.483987in}}{\pgfqpoint{2.290562in}{1.492223in}}%
\pgfpathcurveto{\pgfqpoint{2.290562in}{1.500459in}}{\pgfqpoint{2.287290in}{1.508359in}}{\pgfqpoint{2.281466in}{1.514183in}}%
\pgfpathcurveto{\pgfqpoint{2.275642in}{1.520007in}}{\pgfqpoint{2.267742in}{1.523279in}}{\pgfqpoint{2.259505in}{1.523279in}}%
\pgfpathcurveto{\pgfqpoint{2.251269in}{1.523279in}}{\pgfqpoint{2.243369in}{1.520007in}}{\pgfqpoint{2.237545in}{1.514183in}}%
\pgfpathcurveto{\pgfqpoint{2.231721in}{1.508359in}}{\pgfqpoint{2.228449in}{1.500459in}}{\pgfqpoint{2.228449in}{1.492223in}}%
\pgfpathcurveto{\pgfqpoint{2.228449in}{1.483987in}}{\pgfqpoint{2.231721in}{1.476086in}}{\pgfqpoint{2.237545in}{1.470263in}}%
\pgfpathcurveto{\pgfqpoint{2.243369in}{1.464439in}}{\pgfqpoint{2.251269in}{1.461166in}}{\pgfqpoint{2.259505in}{1.461166in}}%
\pgfpathclose%
\pgfusepath{stroke,fill}%
\end{pgfscope}%
\begin{pgfscope}%
\pgfpathrectangle{\pgfqpoint{0.100000in}{0.212622in}}{\pgfqpoint{3.696000in}{3.696000in}}%
\pgfusepath{clip}%
\pgfsetbuttcap%
\pgfsetroundjoin%
\definecolor{currentfill}{rgb}{0.121569,0.466667,0.705882}%
\pgfsetfillcolor{currentfill}%
\pgfsetfillopacity{0.796724}%
\pgfsetlinewidth{1.003750pt}%
\definecolor{currentstroke}{rgb}{0.121569,0.466667,0.705882}%
\pgfsetstrokecolor{currentstroke}%
\pgfsetstrokeopacity{0.796724}%
\pgfsetdash{}{0pt}%
\pgfpathmoveto{\pgfqpoint{1.482981in}{1.154354in}}%
\pgfpathcurveto{\pgfqpoint{1.491218in}{1.154354in}}{\pgfqpoint{1.499118in}{1.157626in}}{\pgfqpoint{1.504942in}{1.163450in}}%
\pgfpathcurveto{\pgfqpoint{1.510766in}{1.169274in}}{\pgfqpoint{1.514038in}{1.177174in}}{\pgfqpoint{1.514038in}{1.185410in}}%
\pgfpathcurveto{\pgfqpoint{1.514038in}{1.193646in}}{\pgfqpoint{1.510766in}{1.201546in}}{\pgfqpoint{1.504942in}{1.207370in}}%
\pgfpathcurveto{\pgfqpoint{1.499118in}{1.213194in}}{\pgfqpoint{1.491218in}{1.216467in}}{\pgfqpoint{1.482981in}{1.216467in}}%
\pgfpathcurveto{\pgfqpoint{1.474745in}{1.216467in}}{\pgfqpoint{1.466845in}{1.213194in}}{\pgfqpoint{1.461021in}{1.207370in}}%
\pgfpathcurveto{\pgfqpoint{1.455197in}{1.201546in}}{\pgfqpoint{1.451925in}{1.193646in}}{\pgfqpoint{1.451925in}{1.185410in}}%
\pgfpathcurveto{\pgfqpoint{1.451925in}{1.177174in}}{\pgfqpoint{1.455197in}{1.169274in}}{\pgfqpoint{1.461021in}{1.163450in}}%
\pgfpathcurveto{\pgfqpoint{1.466845in}{1.157626in}}{\pgfqpoint{1.474745in}{1.154354in}}{\pgfqpoint{1.482981in}{1.154354in}}%
\pgfpathclose%
\pgfusepath{stroke,fill}%
\end{pgfscope}%
\begin{pgfscope}%
\pgfpathrectangle{\pgfqpoint{0.100000in}{0.212622in}}{\pgfqpoint{3.696000in}{3.696000in}}%
\pgfusepath{clip}%
\pgfsetbuttcap%
\pgfsetroundjoin%
\definecolor{currentfill}{rgb}{0.121569,0.466667,0.705882}%
\pgfsetfillcolor{currentfill}%
\pgfsetfillopacity{0.797944}%
\pgfsetlinewidth{1.003750pt}%
\definecolor{currentstroke}{rgb}{0.121569,0.466667,0.705882}%
\pgfsetstrokecolor{currentstroke}%
\pgfsetstrokeopacity{0.797944}%
\pgfsetdash{}{0pt}%
\pgfpathmoveto{\pgfqpoint{1.488515in}{1.152157in}}%
\pgfpathcurveto{\pgfqpoint{1.496751in}{1.152157in}}{\pgfqpoint{1.504651in}{1.155430in}}{\pgfqpoint{1.510475in}{1.161254in}}%
\pgfpathcurveto{\pgfqpoint{1.516299in}{1.167077in}}{\pgfqpoint{1.519571in}{1.174978in}}{\pgfqpoint{1.519571in}{1.183214in}}%
\pgfpathcurveto{\pgfqpoint{1.519571in}{1.191450in}}{\pgfqpoint{1.516299in}{1.199350in}}{\pgfqpoint{1.510475in}{1.205174in}}%
\pgfpathcurveto{\pgfqpoint{1.504651in}{1.210998in}}{\pgfqpoint{1.496751in}{1.214270in}}{\pgfqpoint{1.488515in}{1.214270in}}%
\pgfpathcurveto{\pgfqpoint{1.480278in}{1.214270in}}{\pgfqpoint{1.472378in}{1.210998in}}{\pgfqpoint{1.466554in}{1.205174in}}%
\pgfpathcurveto{\pgfqpoint{1.460730in}{1.199350in}}{\pgfqpoint{1.457458in}{1.191450in}}{\pgfqpoint{1.457458in}{1.183214in}}%
\pgfpathcurveto{\pgfqpoint{1.457458in}{1.174978in}}{\pgfqpoint{1.460730in}{1.167077in}}{\pgfqpoint{1.466554in}{1.161254in}}%
\pgfpathcurveto{\pgfqpoint{1.472378in}{1.155430in}}{\pgfqpoint{1.480278in}{1.152157in}}{\pgfqpoint{1.488515in}{1.152157in}}%
\pgfpathclose%
\pgfusepath{stroke,fill}%
\end{pgfscope}%
\begin{pgfscope}%
\pgfpathrectangle{\pgfqpoint{0.100000in}{0.212622in}}{\pgfqpoint{3.696000in}{3.696000in}}%
\pgfusepath{clip}%
\pgfsetbuttcap%
\pgfsetroundjoin%
\definecolor{currentfill}{rgb}{0.121569,0.466667,0.705882}%
\pgfsetfillcolor{currentfill}%
\pgfsetfillopacity{0.800146}%
\pgfsetlinewidth{1.003750pt}%
\definecolor{currentstroke}{rgb}{0.121569,0.466667,0.705882}%
\pgfsetstrokecolor{currentstroke}%
\pgfsetstrokeopacity{0.800146}%
\pgfsetdash{}{0pt}%
\pgfpathmoveto{\pgfqpoint{1.498765in}{1.148750in}}%
\pgfpathcurveto{\pgfqpoint{1.507001in}{1.148750in}}{\pgfqpoint{1.514901in}{1.152022in}}{\pgfqpoint{1.520725in}{1.157846in}}%
\pgfpathcurveto{\pgfqpoint{1.526549in}{1.163670in}}{\pgfqpoint{1.529822in}{1.171570in}}{\pgfqpoint{1.529822in}{1.179806in}}%
\pgfpathcurveto{\pgfqpoint{1.529822in}{1.188043in}}{\pgfqpoint{1.526549in}{1.195943in}}{\pgfqpoint{1.520725in}{1.201767in}}%
\pgfpathcurveto{\pgfqpoint{1.514901in}{1.207590in}}{\pgfqpoint{1.507001in}{1.210863in}}{\pgfqpoint{1.498765in}{1.210863in}}%
\pgfpathcurveto{\pgfqpoint{1.490529in}{1.210863in}}{\pgfqpoint{1.482629in}{1.207590in}}{\pgfqpoint{1.476805in}{1.201767in}}%
\pgfpathcurveto{\pgfqpoint{1.470981in}{1.195943in}}{\pgfqpoint{1.467709in}{1.188043in}}{\pgfqpoint{1.467709in}{1.179806in}}%
\pgfpathcurveto{\pgfqpoint{1.467709in}{1.171570in}}{\pgfqpoint{1.470981in}{1.163670in}}{\pgfqpoint{1.476805in}{1.157846in}}%
\pgfpathcurveto{\pgfqpoint{1.482629in}{1.152022in}}{\pgfqpoint{1.490529in}{1.148750in}}{\pgfqpoint{1.498765in}{1.148750in}}%
\pgfpathclose%
\pgfusepath{stroke,fill}%
\end{pgfscope}%
\begin{pgfscope}%
\pgfpathrectangle{\pgfqpoint{0.100000in}{0.212622in}}{\pgfqpoint{3.696000in}{3.696000in}}%
\pgfusepath{clip}%
\pgfsetbuttcap%
\pgfsetroundjoin%
\definecolor{currentfill}{rgb}{0.121569,0.466667,0.705882}%
\pgfsetfillcolor{currentfill}%
\pgfsetfillopacity{0.801293}%
\pgfsetlinewidth{1.003750pt}%
\definecolor{currentstroke}{rgb}{0.121569,0.466667,0.705882}%
\pgfsetstrokecolor{currentstroke}%
\pgfsetstrokeopacity{0.801293}%
\pgfsetdash{}{0pt}%
\pgfpathmoveto{\pgfqpoint{2.265310in}{1.443202in}}%
\pgfpathcurveto{\pgfqpoint{2.273546in}{1.443202in}}{\pgfqpoint{2.281447in}{1.446474in}}{\pgfqpoint{2.287270in}{1.452298in}}%
\pgfpathcurveto{\pgfqpoint{2.293094in}{1.458122in}}{\pgfqpoint{2.296367in}{1.466022in}}{\pgfqpoint{2.296367in}{1.474258in}}%
\pgfpathcurveto{\pgfqpoint{2.296367in}{1.482495in}}{\pgfqpoint{2.293094in}{1.490395in}}{\pgfqpoint{2.287270in}{1.496219in}}%
\pgfpathcurveto{\pgfqpoint{2.281447in}{1.502043in}}{\pgfqpoint{2.273546in}{1.505315in}}{\pgfqpoint{2.265310in}{1.505315in}}%
\pgfpathcurveto{\pgfqpoint{2.257074in}{1.505315in}}{\pgfqpoint{2.249174in}{1.502043in}}{\pgfqpoint{2.243350in}{1.496219in}}%
\pgfpathcurveto{\pgfqpoint{2.237526in}{1.490395in}}{\pgfqpoint{2.234254in}{1.482495in}}{\pgfqpoint{2.234254in}{1.474258in}}%
\pgfpathcurveto{\pgfqpoint{2.234254in}{1.466022in}}{\pgfqpoint{2.237526in}{1.458122in}}{\pgfqpoint{2.243350in}{1.452298in}}%
\pgfpathcurveto{\pgfqpoint{2.249174in}{1.446474in}}{\pgfqpoint{2.257074in}{1.443202in}}{\pgfqpoint{2.265310in}{1.443202in}}%
\pgfpathclose%
\pgfusepath{stroke,fill}%
\end{pgfscope}%
\begin{pgfscope}%
\pgfpathrectangle{\pgfqpoint{0.100000in}{0.212622in}}{\pgfqpoint{3.696000in}{3.696000in}}%
\pgfusepath{clip}%
\pgfsetbuttcap%
\pgfsetroundjoin%
\definecolor{currentfill}{rgb}{0.121569,0.466667,0.705882}%
\pgfsetfillcolor{currentfill}%
\pgfsetfillopacity{0.802010}%
\pgfsetlinewidth{1.003750pt}%
\definecolor{currentstroke}{rgb}{0.121569,0.466667,0.705882}%
\pgfsetstrokecolor{currentstroke}%
\pgfsetstrokeopacity{0.802010}%
\pgfsetdash{}{0pt}%
\pgfpathmoveto{\pgfqpoint{1.506835in}{1.146057in}}%
\pgfpathcurveto{\pgfqpoint{1.515071in}{1.146057in}}{\pgfqpoint{1.522971in}{1.149330in}}{\pgfqpoint{1.528795in}{1.155153in}}%
\pgfpathcurveto{\pgfqpoint{1.534619in}{1.160977in}}{\pgfqpoint{1.537891in}{1.168877in}}{\pgfqpoint{1.537891in}{1.177114in}}%
\pgfpathcurveto{\pgfqpoint{1.537891in}{1.185350in}}{\pgfqpoint{1.534619in}{1.193250in}}{\pgfqpoint{1.528795in}{1.199074in}}%
\pgfpathcurveto{\pgfqpoint{1.522971in}{1.204898in}}{\pgfqpoint{1.515071in}{1.208170in}}{\pgfqpoint{1.506835in}{1.208170in}}%
\pgfpathcurveto{\pgfqpoint{1.498599in}{1.208170in}}{\pgfqpoint{1.490699in}{1.204898in}}{\pgfqpoint{1.484875in}{1.199074in}}%
\pgfpathcurveto{\pgfqpoint{1.479051in}{1.193250in}}{\pgfqpoint{1.475778in}{1.185350in}}{\pgfqpoint{1.475778in}{1.177114in}}%
\pgfpathcurveto{\pgfqpoint{1.475778in}{1.168877in}}{\pgfqpoint{1.479051in}{1.160977in}}{\pgfqpoint{1.484875in}{1.155153in}}%
\pgfpathcurveto{\pgfqpoint{1.490699in}{1.149330in}}{\pgfqpoint{1.498599in}{1.146057in}}{\pgfqpoint{1.506835in}{1.146057in}}%
\pgfpathclose%
\pgfusepath{stroke,fill}%
\end{pgfscope}%
\begin{pgfscope}%
\pgfpathrectangle{\pgfqpoint{0.100000in}{0.212622in}}{\pgfqpoint{3.696000in}{3.696000in}}%
\pgfusepath{clip}%
\pgfsetbuttcap%
\pgfsetroundjoin%
\definecolor{currentfill}{rgb}{0.121569,0.466667,0.705882}%
\pgfsetfillcolor{currentfill}%
\pgfsetfillopacity{0.803694}%
\pgfsetlinewidth{1.003750pt}%
\definecolor{currentstroke}{rgb}{0.121569,0.466667,0.705882}%
\pgfsetstrokecolor{currentstroke}%
\pgfsetstrokeopacity{0.803694}%
\pgfsetdash{}{0pt}%
\pgfpathmoveto{\pgfqpoint{1.514281in}{1.143287in}}%
\pgfpathcurveto{\pgfqpoint{1.522517in}{1.143287in}}{\pgfqpoint{1.530417in}{1.146559in}}{\pgfqpoint{1.536241in}{1.152383in}}%
\pgfpathcurveto{\pgfqpoint{1.542065in}{1.158207in}}{\pgfqpoint{1.545337in}{1.166107in}}{\pgfqpoint{1.545337in}{1.174344in}}%
\pgfpathcurveto{\pgfqpoint{1.545337in}{1.182580in}}{\pgfqpoint{1.542065in}{1.190480in}}{\pgfqpoint{1.536241in}{1.196304in}}%
\pgfpathcurveto{\pgfqpoint{1.530417in}{1.202128in}}{\pgfqpoint{1.522517in}{1.205400in}}{\pgfqpoint{1.514281in}{1.205400in}}%
\pgfpathcurveto{\pgfqpoint{1.506045in}{1.205400in}}{\pgfqpoint{1.498144in}{1.202128in}}{\pgfqpoint{1.492321in}{1.196304in}}%
\pgfpathcurveto{\pgfqpoint{1.486497in}{1.190480in}}{\pgfqpoint{1.483224in}{1.182580in}}{\pgfqpoint{1.483224in}{1.174344in}}%
\pgfpathcurveto{\pgfqpoint{1.483224in}{1.166107in}}{\pgfqpoint{1.486497in}{1.158207in}}{\pgfqpoint{1.492321in}{1.152383in}}%
\pgfpathcurveto{\pgfqpoint{1.498144in}{1.146559in}}{\pgfqpoint{1.506045in}{1.143287in}}{\pgfqpoint{1.514281in}{1.143287in}}%
\pgfpathclose%
\pgfusepath{stroke,fill}%
\end{pgfscope}%
\begin{pgfscope}%
\pgfpathrectangle{\pgfqpoint{0.100000in}{0.212622in}}{\pgfqpoint{3.696000in}{3.696000in}}%
\pgfusepath{clip}%
\pgfsetbuttcap%
\pgfsetroundjoin%
\definecolor{currentfill}{rgb}{0.121569,0.466667,0.705882}%
\pgfsetfillcolor{currentfill}%
\pgfsetfillopacity{0.805138}%
\pgfsetlinewidth{1.003750pt}%
\definecolor{currentstroke}{rgb}{0.121569,0.466667,0.705882}%
\pgfsetstrokecolor{currentstroke}%
\pgfsetstrokeopacity{0.805138}%
\pgfsetdash{}{0pt}%
\pgfpathmoveto{\pgfqpoint{1.521114in}{1.140742in}}%
\pgfpathcurveto{\pgfqpoint{1.529350in}{1.140742in}}{\pgfqpoint{1.537250in}{1.144014in}}{\pgfqpoint{1.543074in}{1.149838in}}%
\pgfpathcurveto{\pgfqpoint{1.548898in}{1.155662in}}{\pgfqpoint{1.552170in}{1.163562in}}{\pgfqpoint{1.552170in}{1.171798in}}%
\pgfpathcurveto{\pgfqpoint{1.552170in}{1.180035in}}{\pgfqpoint{1.548898in}{1.187935in}}{\pgfqpoint{1.543074in}{1.193759in}}%
\pgfpathcurveto{\pgfqpoint{1.537250in}{1.199583in}}{\pgfqpoint{1.529350in}{1.202855in}}{\pgfqpoint{1.521114in}{1.202855in}}%
\pgfpathcurveto{\pgfqpoint{1.512877in}{1.202855in}}{\pgfqpoint{1.504977in}{1.199583in}}{\pgfqpoint{1.499153in}{1.193759in}}%
\pgfpathcurveto{\pgfqpoint{1.493329in}{1.187935in}}{\pgfqpoint{1.490057in}{1.180035in}}{\pgfqpoint{1.490057in}{1.171798in}}%
\pgfpathcurveto{\pgfqpoint{1.490057in}{1.163562in}}{\pgfqpoint{1.493329in}{1.155662in}}{\pgfqpoint{1.499153in}{1.149838in}}%
\pgfpathcurveto{\pgfqpoint{1.504977in}{1.144014in}}{\pgfqpoint{1.512877in}{1.140742in}}{\pgfqpoint{1.521114in}{1.140742in}}%
\pgfpathclose%
\pgfusepath{stroke,fill}%
\end{pgfscope}%
\begin{pgfscope}%
\pgfpathrectangle{\pgfqpoint{0.100000in}{0.212622in}}{\pgfqpoint{3.696000in}{3.696000in}}%
\pgfusepath{clip}%
\pgfsetbuttcap%
\pgfsetroundjoin%
\definecolor{currentfill}{rgb}{0.121569,0.466667,0.705882}%
\pgfsetfillcolor{currentfill}%
\pgfsetfillopacity{0.806374}%
\pgfsetlinewidth{1.003750pt}%
\definecolor{currentstroke}{rgb}{0.121569,0.466667,0.705882}%
\pgfsetstrokecolor{currentstroke}%
\pgfsetstrokeopacity{0.806374}%
\pgfsetdash{}{0pt}%
\pgfpathmoveto{\pgfqpoint{1.526682in}{1.138574in}}%
\pgfpathcurveto{\pgfqpoint{1.534918in}{1.138574in}}{\pgfqpoint{1.542818in}{1.141847in}}{\pgfqpoint{1.548642in}{1.147671in}}%
\pgfpathcurveto{\pgfqpoint{1.554466in}{1.153495in}}{\pgfqpoint{1.557738in}{1.161395in}}{\pgfqpoint{1.557738in}{1.169631in}}%
\pgfpathcurveto{\pgfqpoint{1.557738in}{1.177867in}}{\pgfqpoint{1.554466in}{1.185767in}}{\pgfqpoint{1.548642in}{1.191591in}}%
\pgfpathcurveto{\pgfqpoint{1.542818in}{1.197415in}}{\pgfqpoint{1.534918in}{1.200687in}}{\pgfqpoint{1.526682in}{1.200687in}}%
\pgfpathcurveto{\pgfqpoint{1.518446in}{1.200687in}}{\pgfqpoint{1.510546in}{1.197415in}}{\pgfqpoint{1.504722in}{1.191591in}}%
\pgfpathcurveto{\pgfqpoint{1.498898in}{1.185767in}}{\pgfqpoint{1.495625in}{1.177867in}}{\pgfqpoint{1.495625in}{1.169631in}}%
\pgfpathcurveto{\pgfqpoint{1.495625in}{1.161395in}}{\pgfqpoint{1.498898in}{1.153495in}}{\pgfqpoint{1.504722in}{1.147671in}}%
\pgfpathcurveto{\pgfqpoint{1.510546in}{1.141847in}}{\pgfqpoint{1.518446in}{1.138574in}}{\pgfqpoint{1.526682in}{1.138574in}}%
\pgfpathclose%
\pgfusepath{stroke,fill}%
\end{pgfscope}%
\begin{pgfscope}%
\pgfpathrectangle{\pgfqpoint{0.100000in}{0.212622in}}{\pgfqpoint{3.696000in}{3.696000in}}%
\pgfusepath{clip}%
\pgfsetbuttcap%
\pgfsetroundjoin%
\definecolor{currentfill}{rgb}{0.121569,0.466667,0.705882}%
\pgfsetfillcolor{currentfill}%
\pgfsetfillopacity{0.806756}%
\pgfsetlinewidth{1.003750pt}%
\definecolor{currentstroke}{rgb}{0.121569,0.466667,0.705882}%
\pgfsetstrokecolor{currentstroke}%
\pgfsetstrokeopacity{0.806756}%
\pgfsetdash{}{0pt}%
\pgfpathmoveto{\pgfqpoint{2.270881in}{1.424518in}}%
\pgfpathcurveto{\pgfqpoint{2.279117in}{1.424518in}}{\pgfqpoint{2.287018in}{1.427790in}}{\pgfqpoint{2.292841in}{1.433614in}}%
\pgfpathcurveto{\pgfqpoint{2.298665in}{1.439438in}}{\pgfqpoint{2.301938in}{1.447338in}}{\pgfqpoint{2.301938in}{1.455575in}}%
\pgfpathcurveto{\pgfqpoint{2.301938in}{1.463811in}}{\pgfqpoint{2.298665in}{1.471711in}}{\pgfqpoint{2.292841in}{1.477535in}}%
\pgfpathcurveto{\pgfqpoint{2.287018in}{1.483359in}}{\pgfqpoint{2.279117in}{1.486631in}}{\pgfqpoint{2.270881in}{1.486631in}}%
\pgfpathcurveto{\pgfqpoint{2.262645in}{1.486631in}}{\pgfqpoint{2.254745in}{1.483359in}}{\pgfqpoint{2.248921in}{1.477535in}}%
\pgfpathcurveto{\pgfqpoint{2.243097in}{1.471711in}}{\pgfqpoint{2.239825in}{1.463811in}}{\pgfqpoint{2.239825in}{1.455575in}}%
\pgfpathcurveto{\pgfqpoint{2.239825in}{1.447338in}}{\pgfqpoint{2.243097in}{1.439438in}}{\pgfqpoint{2.248921in}{1.433614in}}%
\pgfpathcurveto{\pgfqpoint{2.254745in}{1.427790in}}{\pgfqpoint{2.262645in}{1.424518in}}{\pgfqpoint{2.270881in}{1.424518in}}%
\pgfpathclose%
\pgfusepath{stroke,fill}%
\end{pgfscope}%
\begin{pgfscope}%
\pgfpathrectangle{\pgfqpoint{0.100000in}{0.212622in}}{\pgfqpoint{3.696000in}{3.696000in}}%
\pgfusepath{clip}%
\pgfsetbuttcap%
\pgfsetroundjoin%
\definecolor{currentfill}{rgb}{0.121569,0.466667,0.705882}%
\pgfsetfillcolor{currentfill}%
\pgfsetfillopacity{0.808518}%
\pgfsetlinewidth{1.003750pt}%
\definecolor{currentstroke}{rgb}{0.121569,0.466667,0.705882}%
\pgfsetstrokecolor{currentstroke}%
\pgfsetstrokeopacity{0.808518}%
\pgfsetdash{}{0pt}%
\pgfpathmoveto{\pgfqpoint{1.536929in}{1.134648in}}%
\pgfpathcurveto{\pgfqpoint{1.545165in}{1.134648in}}{\pgfqpoint{1.553065in}{1.137920in}}{\pgfqpoint{1.558889in}{1.143744in}}%
\pgfpathcurveto{\pgfqpoint{1.564713in}{1.149568in}}{\pgfqpoint{1.567986in}{1.157468in}}{\pgfqpoint{1.567986in}{1.165705in}}%
\pgfpathcurveto{\pgfqpoint{1.567986in}{1.173941in}}{\pgfqpoint{1.564713in}{1.181841in}}{\pgfqpoint{1.558889in}{1.187665in}}%
\pgfpathcurveto{\pgfqpoint{1.553065in}{1.193489in}}{\pgfqpoint{1.545165in}{1.196761in}}{\pgfqpoint{1.536929in}{1.196761in}}%
\pgfpathcurveto{\pgfqpoint{1.528693in}{1.196761in}}{\pgfqpoint{1.520793in}{1.193489in}}{\pgfqpoint{1.514969in}{1.187665in}}%
\pgfpathcurveto{\pgfqpoint{1.509145in}{1.181841in}}{\pgfqpoint{1.505873in}{1.173941in}}{\pgfqpoint{1.505873in}{1.165705in}}%
\pgfpathcurveto{\pgfqpoint{1.505873in}{1.157468in}}{\pgfqpoint{1.509145in}{1.149568in}}{\pgfqpoint{1.514969in}{1.143744in}}%
\pgfpathcurveto{\pgfqpoint{1.520793in}{1.137920in}}{\pgfqpoint{1.528693in}{1.134648in}}{\pgfqpoint{1.536929in}{1.134648in}}%
\pgfpathclose%
\pgfusepath{stroke,fill}%
\end{pgfscope}%
\begin{pgfscope}%
\pgfpathrectangle{\pgfqpoint{0.100000in}{0.212622in}}{\pgfqpoint{3.696000in}{3.696000in}}%
\pgfusepath{clip}%
\pgfsetbuttcap%
\pgfsetroundjoin%
\definecolor{currentfill}{rgb}{0.121569,0.466667,0.705882}%
\pgfsetfillcolor{currentfill}%
\pgfsetfillopacity{0.810125}%
\pgfsetlinewidth{1.003750pt}%
\definecolor{currentstroke}{rgb}{0.121569,0.466667,0.705882}%
\pgfsetstrokecolor{currentstroke}%
\pgfsetstrokeopacity{0.810125}%
\pgfsetdash{}{0pt}%
\pgfpathmoveto{\pgfqpoint{1.545246in}{1.131291in}}%
\pgfpathcurveto{\pgfqpoint{1.553482in}{1.131291in}}{\pgfqpoint{1.561382in}{1.134563in}}{\pgfqpoint{1.567206in}{1.140387in}}%
\pgfpathcurveto{\pgfqpoint{1.573030in}{1.146211in}}{\pgfqpoint{1.576302in}{1.154111in}}{\pgfqpoint{1.576302in}{1.162347in}}%
\pgfpathcurveto{\pgfqpoint{1.576302in}{1.170584in}}{\pgfqpoint{1.573030in}{1.178484in}}{\pgfqpoint{1.567206in}{1.184308in}}%
\pgfpathcurveto{\pgfqpoint{1.561382in}{1.190132in}}{\pgfqpoint{1.553482in}{1.193404in}}{\pgfqpoint{1.545246in}{1.193404in}}%
\pgfpathcurveto{\pgfqpoint{1.537009in}{1.193404in}}{\pgfqpoint{1.529109in}{1.190132in}}{\pgfqpoint{1.523285in}{1.184308in}}%
\pgfpathcurveto{\pgfqpoint{1.517462in}{1.178484in}}{\pgfqpoint{1.514189in}{1.170584in}}{\pgfqpoint{1.514189in}{1.162347in}}%
\pgfpathcurveto{\pgfqpoint{1.514189in}{1.154111in}}{\pgfqpoint{1.517462in}{1.146211in}}{\pgfqpoint{1.523285in}{1.140387in}}%
\pgfpathcurveto{\pgfqpoint{1.529109in}{1.134563in}}{\pgfqpoint{1.537009in}{1.131291in}}{\pgfqpoint{1.545246in}{1.131291in}}%
\pgfpathclose%
\pgfusepath{stroke,fill}%
\end{pgfscope}%
\begin{pgfscope}%
\pgfpathrectangle{\pgfqpoint{0.100000in}{0.212622in}}{\pgfqpoint{3.696000in}{3.696000in}}%
\pgfusepath{clip}%
\pgfsetbuttcap%
\pgfsetroundjoin%
\definecolor{currentfill}{rgb}{0.121569,0.466667,0.705882}%
\pgfsetfillcolor{currentfill}%
\pgfsetfillopacity{0.811884}%
\pgfsetlinewidth{1.003750pt}%
\definecolor{currentstroke}{rgb}{0.121569,0.466667,0.705882}%
\pgfsetstrokecolor{currentstroke}%
\pgfsetstrokeopacity{0.811884}%
\pgfsetdash{}{0pt}%
\pgfpathmoveto{\pgfqpoint{1.553020in}{1.128941in}}%
\pgfpathcurveto{\pgfqpoint{1.561256in}{1.128941in}}{\pgfqpoint{1.569156in}{1.132213in}}{\pgfqpoint{1.574980in}{1.138037in}}%
\pgfpathcurveto{\pgfqpoint{1.580804in}{1.143861in}}{\pgfqpoint{1.584076in}{1.151761in}}{\pgfqpoint{1.584076in}{1.159998in}}%
\pgfpathcurveto{\pgfqpoint{1.584076in}{1.168234in}}{\pgfqpoint{1.580804in}{1.176134in}}{\pgfqpoint{1.574980in}{1.181958in}}%
\pgfpathcurveto{\pgfqpoint{1.569156in}{1.187782in}}{\pgfqpoint{1.561256in}{1.191054in}}{\pgfqpoint{1.553020in}{1.191054in}}%
\pgfpathcurveto{\pgfqpoint{1.544783in}{1.191054in}}{\pgfqpoint{1.536883in}{1.187782in}}{\pgfqpoint{1.531059in}{1.181958in}}%
\pgfpathcurveto{\pgfqpoint{1.525235in}{1.176134in}}{\pgfqpoint{1.521963in}{1.168234in}}{\pgfqpoint{1.521963in}{1.159998in}}%
\pgfpathcurveto{\pgfqpoint{1.521963in}{1.151761in}}{\pgfqpoint{1.525235in}{1.143861in}}{\pgfqpoint{1.531059in}{1.138037in}}%
\pgfpathcurveto{\pgfqpoint{1.536883in}{1.132213in}}{\pgfqpoint{1.544783in}{1.128941in}}{\pgfqpoint{1.553020in}{1.128941in}}%
\pgfpathclose%
\pgfusepath{stroke,fill}%
\end{pgfscope}%
\begin{pgfscope}%
\pgfpathrectangle{\pgfqpoint{0.100000in}{0.212622in}}{\pgfqpoint{3.696000in}{3.696000in}}%
\pgfusepath{clip}%
\pgfsetbuttcap%
\pgfsetroundjoin%
\definecolor{currentfill}{rgb}{0.121569,0.466667,0.705882}%
\pgfsetfillcolor{currentfill}%
\pgfsetfillopacity{0.812933}%
\pgfsetlinewidth{1.003750pt}%
\definecolor{currentstroke}{rgb}{0.121569,0.466667,0.705882}%
\pgfsetstrokecolor{currentstroke}%
\pgfsetstrokeopacity{0.812933}%
\pgfsetdash{}{0pt}%
\pgfpathmoveto{\pgfqpoint{2.275297in}{1.405400in}}%
\pgfpathcurveto{\pgfqpoint{2.283533in}{1.405400in}}{\pgfqpoint{2.291433in}{1.408673in}}{\pgfqpoint{2.297257in}{1.414496in}}%
\pgfpathcurveto{\pgfqpoint{2.303081in}{1.420320in}}{\pgfqpoint{2.306353in}{1.428220in}}{\pgfqpoint{2.306353in}{1.436457in}}%
\pgfpathcurveto{\pgfqpoint{2.306353in}{1.444693in}}{\pgfqpoint{2.303081in}{1.452593in}}{\pgfqpoint{2.297257in}{1.458417in}}%
\pgfpathcurveto{\pgfqpoint{2.291433in}{1.464241in}}{\pgfqpoint{2.283533in}{1.467513in}}{\pgfqpoint{2.275297in}{1.467513in}}%
\pgfpathcurveto{\pgfqpoint{2.267061in}{1.467513in}}{\pgfqpoint{2.259161in}{1.464241in}}{\pgfqpoint{2.253337in}{1.458417in}}%
\pgfpathcurveto{\pgfqpoint{2.247513in}{1.452593in}}{\pgfqpoint{2.244240in}{1.444693in}}{\pgfqpoint{2.244240in}{1.436457in}}%
\pgfpathcurveto{\pgfqpoint{2.244240in}{1.428220in}}{\pgfqpoint{2.247513in}{1.420320in}}{\pgfqpoint{2.253337in}{1.414496in}}%
\pgfpathcurveto{\pgfqpoint{2.259161in}{1.408673in}}{\pgfqpoint{2.267061in}{1.405400in}}{\pgfqpoint{2.275297in}{1.405400in}}%
\pgfpathclose%
\pgfusepath{stroke,fill}%
\end{pgfscope}%
\begin{pgfscope}%
\pgfpathrectangle{\pgfqpoint{0.100000in}{0.212622in}}{\pgfqpoint{3.696000in}{3.696000in}}%
\pgfusepath{clip}%
\pgfsetbuttcap%
\pgfsetroundjoin%
\definecolor{currentfill}{rgb}{0.121569,0.466667,0.705882}%
\pgfsetfillcolor{currentfill}%
\pgfsetfillopacity{0.813201}%
\pgfsetlinewidth{1.003750pt}%
\definecolor{currentstroke}{rgb}{0.121569,0.466667,0.705882}%
\pgfsetstrokecolor{currentstroke}%
\pgfsetstrokeopacity{0.813201}%
\pgfsetdash{}{0pt}%
\pgfpathmoveto{\pgfqpoint{1.558712in}{1.127154in}}%
\pgfpathcurveto{\pgfqpoint{1.566948in}{1.127154in}}{\pgfqpoint{1.574848in}{1.130427in}}{\pgfqpoint{1.580672in}{1.136251in}}%
\pgfpathcurveto{\pgfqpoint{1.586496in}{1.142074in}}{\pgfqpoint{1.589768in}{1.149975in}}{\pgfqpoint{1.589768in}{1.158211in}}%
\pgfpathcurveto{\pgfqpoint{1.589768in}{1.166447in}}{\pgfqpoint{1.586496in}{1.174347in}}{\pgfqpoint{1.580672in}{1.180171in}}%
\pgfpathcurveto{\pgfqpoint{1.574848in}{1.185995in}}{\pgfqpoint{1.566948in}{1.189267in}}{\pgfqpoint{1.558712in}{1.189267in}}%
\pgfpathcurveto{\pgfqpoint{1.550476in}{1.189267in}}{\pgfqpoint{1.542575in}{1.185995in}}{\pgfqpoint{1.536752in}{1.180171in}}%
\pgfpathcurveto{\pgfqpoint{1.530928in}{1.174347in}}{\pgfqpoint{1.527655in}{1.166447in}}{\pgfqpoint{1.527655in}{1.158211in}}%
\pgfpathcurveto{\pgfqpoint{1.527655in}{1.149975in}}{\pgfqpoint{1.530928in}{1.142074in}}{\pgfqpoint{1.536752in}{1.136251in}}%
\pgfpathcurveto{\pgfqpoint{1.542575in}{1.130427in}}{\pgfqpoint{1.550476in}{1.127154in}}{\pgfqpoint{1.558712in}{1.127154in}}%
\pgfpathclose%
\pgfusepath{stroke,fill}%
\end{pgfscope}%
\begin{pgfscope}%
\pgfpathrectangle{\pgfqpoint{0.100000in}{0.212622in}}{\pgfqpoint{3.696000in}{3.696000in}}%
\pgfusepath{clip}%
\pgfsetbuttcap%
\pgfsetroundjoin%
\definecolor{currentfill}{rgb}{0.121569,0.466667,0.705882}%
\pgfsetfillcolor{currentfill}%
\pgfsetfillopacity{0.814324}%
\pgfsetlinewidth{1.003750pt}%
\definecolor{currentstroke}{rgb}{0.121569,0.466667,0.705882}%
\pgfsetstrokecolor{currentstroke}%
\pgfsetstrokeopacity{0.814324}%
\pgfsetdash{}{0pt}%
\pgfpathmoveto{\pgfqpoint{1.563732in}{1.125489in}}%
\pgfpathcurveto{\pgfqpoint{1.571968in}{1.125489in}}{\pgfqpoint{1.579868in}{1.128761in}}{\pgfqpoint{1.585692in}{1.134585in}}%
\pgfpathcurveto{\pgfqpoint{1.591516in}{1.140409in}}{\pgfqpoint{1.594789in}{1.148309in}}{\pgfqpoint{1.594789in}{1.156546in}}%
\pgfpathcurveto{\pgfqpoint{1.594789in}{1.164782in}}{\pgfqpoint{1.591516in}{1.172682in}}{\pgfqpoint{1.585692in}{1.178506in}}%
\pgfpathcurveto{\pgfqpoint{1.579868in}{1.184330in}}{\pgfqpoint{1.571968in}{1.187602in}}{\pgfqpoint{1.563732in}{1.187602in}}%
\pgfpathcurveto{\pgfqpoint{1.555496in}{1.187602in}}{\pgfqpoint{1.547596in}{1.184330in}}{\pgfqpoint{1.541772in}{1.178506in}}%
\pgfpathcurveto{\pgfqpoint{1.535948in}{1.172682in}}{\pgfqpoint{1.532676in}{1.164782in}}{\pgfqpoint{1.532676in}{1.156546in}}%
\pgfpathcurveto{\pgfqpoint{1.532676in}{1.148309in}}{\pgfqpoint{1.535948in}{1.140409in}}{\pgfqpoint{1.541772in}{1.134585in}}%
\pgfpathcurveto{\pgfqpoint{1.547596in}{1.128761in}}{\pgfqpoint{1.555496in}{1.125489in}}{\pgfqpoint{1.563732in}{1.125489in}}%
\pgfpathclose%
\pgfusepath{stroke,fill}%
\end{pgfscope}%
\begin{pgfscope}%
\pgfpathrectangle{\pgfqpoint{0.100000in}{0.212622in}}{\pgfqpoint{3.696000in}{3.696000in}}%
\pgfusepath{clip}%
\pgfsetbuttcap%
\pgfsetroundjoin%
\definecolor{currentfill}{rgb}{0.121569,0.466667,0.705882}%
\pgfsetfillcolor{currentfill}%
\pgfsetfillopacity{0.816374}%
\pgfsetlinewidth{1.003750pt}%
\definecolor{currentstroke}{rgb}{0.121569,0.466667,0.705882}%
\pgfsetstrokecolor{currentstroke}%
\pgfsetstrokeopacity{0.816374}%
\pgfsetdash{}{0pt}%
\pgfpathmoveto{\pgfqpoint{1.572929in}{1.122727in}}%
\pgfpathcurveto{\pgfqpoint{1.581165in}{1.122727in}}{\pgfqpoint{1.589065in}{1.126000in}}{\pgfqpoint{1.594889in}{1.131824in}}%
\pgfpathcurveto{\pgfqpoint{1.600713in}{1.137647in}}{\pgfqpoint{1.603986in}{1.145548in}}{\pgfqpoint{1.603986in}{1.153784in}}%
\pgfpathcurveto{\pgfqpoint{1.603986in}{1.162020in}}{\pgfqpoint{1.600713in}{1.169920in}}{\pgfqpoint{1.594889in}{1.175744in}}%
\pgfpathcurveto{\pgfqpoint{1.589065in}{1.181568in}}{\pgfqpoint{1.581165in}{1.184840in}}{\pgfqpoint{1.572929in}{1.184840in}}%
\pgfpathcurveto{\pgfqpoint{1.564693in}{1.184840in}}{\pgfqpoint{1.556793in}{1.181568in}}{\pgfqpoint{1.550969in}{1.175744in}}%
\pgfpathcurveto{\pgfqpoint{1.545145in}{1.169920in}}{\pgfqpoint{1.541873in}{1.162020in}}{\pgfqpoint{1.541873in}{1.153784in}}%
\pgfpathcurveto{\pgfqpoint{1.541873in}{1.145548in}}{\pgfqpoint{1.545145in}{1.137647in}}{\pgfqpoint{1.550969in}{1.131824in}}%
\pgfpathcurveto{\pgfqpoint{1.556793in}{1.126000in}}{\pgfqpoint{1.564693in}{1.122727in}}{\pgfqpoint{1.572929in}{1.122727in}}%
\pgfpathclose%
\pgfusepath{stroke,fill}%
\end{pgfscope}%
\begin{pgfscope}%
\pgfpathrectangle{\pgfqpoint{0.100000in}{0.212622in}}{\pgfqpoint{3.696000in}{3.696000in}}%
\pgfusepath{clip}%
\pgfsetbuttcap%
\pgfsetroundjoin%
\definecolor{currentfill}{rgb}{0.121569,0.466667,0.705882}%
\pgfsetfillcolor{currentfill}%
\pgfsetfillopacity{0.818239}%
\pgfsetlinewidth{1.003750pt}%
\definecolor{currentstroke}{rgb}{0.121569,0.466667,0.705882}%
\pgfsetstrokecolor{currentstroke}%
\pgfsetstrokeopacity{0.818239}%
\pgfsetdash{}{0pt}%
\pgfpathmoveto{\pgfqpoint{1.580579in}{1.120478in}}%
\pgfpathcurveto{\pgfqpoint{1.588815in}{1.120478in}}{\pgfqpoint{1.596715in}{1.123751in}}{\pgfqpoint{1.602539in}{1.129574in}}%
\pgfpathcurveto{\pgfqpoint{1.608363in}{1.135398in}}{\pgfqpoint{1.611636in}{1.143298in}}{\pgfqpoint{1.611636in}{1.151535in}}%
\pgfpathcurveto{\pgfqpoint{1.611636in}{1.159771in}}{\pgfqpoint{1.608363in}{1.167671in}}{\pgfqpoint{1.602539in}{1.173495in}}%
\pgfpathcurveto{\pgfqpoint{1.596715in}{1.179319in}}{\pgfqpoint{1.588815in}{1.182591in}}{\pgfqpoint{1.580579in}{1.182591in}}%
\pgfpathcurveto{\pgfqpoint{1.572343in}{1.182591in}}{\pgfqpoint{1.564443in}{1.179319in}}{\pgfqpoint{1.558619in}{1.173495in}}%
\pgfpathcurveto{\pgfqpoint{1.552795in}{1.167671in}}{\pgfqpoint{1.549523in}{1.159771in}}{\pgfqpoint{1.549523in}{1.151535in}}%
\pgfpathcurveto{\pgfqpoint{1.549523in}{1.143298in}}{\pgfqpoint{1.552795in}{1.135398in}}{\pgfqpoint{1.558619in}{1.129574in}}%
\pgfpathcurveto{\pgfqpoint{1.564443in}{1.123751in}}{\pgfqpoint{1.572343in}{1.120478in}}{\pgfqpoint{1.580579in}{1.120478in}}%
\pgfpathclose%
\pgfusepath{stroke,fill}%
\end{pgfscope}%
\begin{pgfscope}%
\pgfpathrectangle{\pgfqpoint{0.100000in}{0.212622in}}{\pgfqpoint{3.696000in}{3.696000in}}%
\pgfusepath{clip}%
\pgfsetbuttcap%
\pgfsetroundjoin%
\definecolor{currentfill}{rgb}{0.121569,0.466667,0.705882}%
\pgfsetfillcolor{currentfill}%
\pgfsetfillopacity{0.819282}%
\pgfsetlinewidth{1.003750pt}%
\definecolor{currentstroke}{rgb}{0.121569,0.466667,0.705882}%
\pgfsetstrokecolor{currentstroke}%
\pgfsetstrokeopacity{0.819282}%
\pgfsetdash{}{0pt}%
\pgfpathmoveto{\pgfqpoint{2.281812in}{1.384887in}}%
\pgfpathcurveto{\pgfqpoint{2.290048in}{1.384887in}}{\pgfqpoint{2.297948in}{1.388159in}}{\pgfqpoint{2.303772in}{1.393983in}}%
\pgfpathcurveto{\pgfqpoint{2.309596in}{1.399807in}}{\pgfqpoint{2.312868in}{1.407707in}}{\pgfqpoint{2.312868in}{1.415943in}}%
\pgfpathcurveto{\pgfqpoint{2.312868in}{1.424180in}}{\pgfqpoint{2.309596in}{1.432080in}}{\pgfqpoint{2.303772in}{1.437904in}}%
\pgfpathcurveto{\pgfqpoint{2.297948in}{1.443727in}}{\pgfqpoint{2.290048in}{1.447000in}}{\pgfqpoint{2.281812in}{1.447000in}}%
\pgfpathcurveto{\pgfqpoint{2.273575in}{1.447000in}}{\pgfqpoint{2.265675in}{1.443727in}}{\pgfqpoint{2.259852in}{1.437904in}}%
\pgfpathcurveto{\pgfqpoint{2.254028in}{1.432080in}}{\pgfqpoint{2.250755in}{1.424180in}}{\pgfqpoint{2.250755in}{1.415943in}}%
\pgfpathcurveto{\pgfqpoint{2.250755in}{1.407707in}}{\pgfqpoint{2.254028in}{1.399807in}}{\pgfqpoint{2.259852in}{1.393983in}}%
\pgfpathcurveto{\pgfqpoint{2.265675in}{1.388159in}}{\pgfqpoint{2.273575in}{1.384887in}}{\pgfqpoint{2.281812in}{1.384887in}}%
\pgfpathclose%
\pgfusepath{stroke,fill}%
\end{pgfscope}%
\begin{pgfscope}%
\pgfpathrectangle{\pgfqpoint{0.100000in}{0.212622in}}{\pgfqpoint{3.696000in}{3.696000in}}%
\pgfusepath{clip}%
\pgfsetbuttcap%
\pgfsetroundjoin%
\definecolor{currentfill}{rgb}{0.121569,0.466667,0.705882}%
\pgfsetfillcolor{currentfill}%
\pgfsetfillopacity{0.821476}%
\pgfsetlinewidth{1.003750pt}%
\definecolor{currentstroke}{rgb}{0.121569,0.466667,0.705882}%
\pgfsetstrokecolor{currentstroke}%
\pgfsetstrokeopacity{0.821476}%
\pgfsetdash{}{0pt}%
\pgfpathmoveto{\pgfqpoint{1.594505in}{1.115831in}}%
\pgfpathcurveto{\pgfqpoint{1.602741in}{1.115831in}}{\pgfqpoint{1.610641in}{1.119104in}}{\pgfqpoint{1.616465in}{1.124928in}}%
\pgfpathcurveto{\pgfqpoint{1.622289in}{1.130751in}}{\pgfqpoint{1.625561in}{1.138652in}}{\pgfqpoint{1.625561in}{1.146888in}}%
\pgfpathcurveto{\pgfqpoint{1.625561in}{1.155124in}}{\pgfqpoint{1.622289in}{1.163024in}}{\pgfqpoint{1.616465in}{1.168848in}}%
\pgfpathcurveto{\pgfqpoint{1.610641in}{1.174672in}}{\pgfqpoint{1.602741in}{1.177944in}}{\pgfqpoint{1.594505in}{1.177944in}}%
\pgfpathcurveto{\pgfqpoint{1.586268in}{1.177944in}}{\pgfqpoint{1.578368in}{1.174672in}}{\pgfqpoint{1.572544in}{1.168848in}}%
\pgfpathcurveto{\pgfqpoint{1.566720in}{1.163024in}}{\pgfqpoint{1.563448in}{1.155124in}}{\pgfqpoint{1.563448in}{1.146888in}}%
\pgfpathcurveto{\pgfqpoint{1.563448in}{1.138652in}}{\pgfqpoint{1.566720in}{1.130751in}}{\pgfqpoint{1.572544in}{1.124928in}}%
\pgfpathcurveto{\pgfqpoint{1.578368in}{1.119104in}}{\pgfqpoint{1.586268in}{1.115831in}}{\pgfqpoint{1.594505in}{1.115831in}}%
\pgfpathclose%
\pgfusepath{stroke,fill}%
\end{pgfscope}%
\begin{pgfscope}%
\pgfpathrectangle{\pgfqpoint{0.100000in}{0.212622in}}{\pgfqpoint{3.696000in}{3.696000in}}%
\pgfusepath{clip}%
\pgfsetbuttcap%
\pgfsetroundjoin%
\definecolor{currentfill}{rgb}{0.121569,0.466667,0.705882}%
\pgfsetfillcolor{currentfill}%
\pgfsetfillopacity{0.824625}%
\pgfsetlinewidth{1.003750pt}%
\definecolor{currentstroke}{rgb}{0.121569,0.466667,0.705882}%
\pgfsetstrokecolor{currentstroke}%
\pgfsetstrokeopacity{0.824625}%
\pgfsetdash{}{0pt}%
\pgfpathmoveto{\pgfqpoint{1.607691in}{1.110362in}}%
\pgfpathcurveto{\pgfqpoint{1.615927in}{1.110362in}}{\pgfqpoint{1.623827in}{1.113635in}}{\pgfqpoint{1.629651in}{1.119459in}}%
\pgfpathcurveto{\pgfqpoint{1.635475in}{1.125283in}}{\pgfqpoint{1.638747in}{1.133183in}}{\pgfqpoint{1.638747in}{1.141419in}}%
\pgfpathcurveto{\pgfqpoint{1.638747in}{1.149655in}}{\pgfqpoint{1.635475in}{1.157555in}}{\pgfqpoint{1.629651in}{1.163379in}}%
\pgfpathcurveto{\pgfqpoint{1.623827in}{1.169203in}}{\pgfqpoint{1.615927in}{1.172475in}}{\pgfqpoint{1.607691in}{1.172475in}}%
\pgfpathcurveto{\pgfqpoint{1.599454in}{1.172475in}}{\pgfqpoint{1.591554in}{1.169203in}}{\pgfqpoint{1.585730in}{1.163379in}}%
\pgfpathcurveto{\pgfqpoint{1.579906in}{1.157555in}}{\pgfqpoint{1.576634in}{1.149655in}}{\pgfqpoint{1.576634in}{1.141419in}}%
\pgfpathcurveto{\pgfqpoint{1.576634in}{1.133183in}}{\pgfqpoint{1.579906in}{1.125283in}}{\pgfqpoint{1.585730in}{1.119459in}}%
\pgfpathcurveto{\pgfqpoint{1.591554in}{1.113635in}}{\pgfqpoint{1.599454in}{1.110362in}}{\pgfqpoint{1.607691in}{1.110362in}}%
\pgfpathclose%
\pgfusepath{stroke,fill}%
\end{pgfscope}%
\begin{pgfscope}%
\pgfpathrectangle{\pgfqpoint{0.100000in}{0.212622in}}{\pgfqpoint{3.696000in}{3.696000in}}%
\pgfusepath{clip}%
\pgfsetbuttcap%
\pgfsetroundjoin%
\definecolor{currentfill}{rgb}{0.121569,0.466667,0.705882}%
\pgfsetfillcolor{currentfill}%
\pgfsetfillopacity{0.825320}%
\pgfsetlinewidth{1.003750pt}%
\definecolor{currentstroke}{rgb}{0.121569,0.466667,0.705882}%
\pgfsetstrokecolor{currentstroke}%
\pgfsetstrokeopacity{0.825320}%
\pgfsetdash{}{0pt}%
\pgfpathmoveto{\pgfqpoint{2.288869in}{1.362235in}}%
\pgfpathcurveto{\pgfqpoint{2.297105in}{1.362235in}}{\pgfqpoint{2.305005in}{1.365507in}}{\pgfqpoint{2.310829in}{1.371331in}}%
\pgfpathcurveto{\pgfqpoint{2.316653in}{1.377155in}}{\pgfqpoint{2.319925in}{1.385055in}}{\pgfqpoint{2.319925in}{1.393292in}}%
\pgfpathcurveto{\pgfqpoint{2.319925in}{1.401528in}}{\pgfqpoint{2.316653in}{1.409428in}}{\pgfqpoint{2.310829in}{1.415252in}}%
\pgfpathcurveto{\pgfqpoint{2.305005in}{1.421076in}}{\pgfqpoint{2.297105in}{1.424348in}}{\pgfqpoint{2.288869in}{1.424348in}}%
\pgfpathcurveto{\pgfqpoint{2.280632in}{1.424348in}}{\pgfqpoint{2.272732in}{1.421076in}}{\pgfqpoint{2.266908in}{1.415252in}}%
\pgfpathcurveto{\pgfqpoint{2.261084in}{1.409428in}}{\pgfqpoint{2.257812in}{1.401528in}}{\pgfqpoint{2.257812in}{1.393292in}}%
\pgfpathcurveto{\pgfqpoint{2.257812in}{1.385055in}}{\pgfqpoint{2.261084in}{1.377155in}}{\pgfqpoint{2.266908in}{1.371331in}}%
\pgfpathcurveto{\pgfqpoint{2.272732in}{1.365507in}}{\pgfqpoint{2.280632in}{1.362235in}}{\pgfqpoint{2.288869in}{1.362235in}}%
\pgfpathclose%
\pgfusepath{stroke,fill}%
\end{pgfscope}%
\begin{pgfscope}%
\pgfpathrectangle{\pgfqpoint{0.100000in}{0.212622in}}{\pgfqpoint{3.696000in}{3.696000in}}%
\pgfusepath{clip}%
\pgfsetbuttcap%
\pgfsetroundjoin%
\definecolor{currentfill}{rgb}{0.121569,0.466667,0.705882}%
\pgfsetfillcolor{currentfill}%
\pgfsetfillopacity{0.827303}%
\pgfsetlinewidth{1.003750pt}%
\definecolor{currentstroke}{rgb}{0.121569,0.466667,0.705882}%
\pgfsetstrokecolor{currentstroke}%
\pgfsetstrokeopacity{0.827303}%
\pgfsetdash{}{0pt}%
\pgfpathmoveto{\pgfqpoint{1.619747in}{1.105464in}}%
\pgfpathcurveto{\pgfqpoint{1.627983in}{1.105464in}}{\pgfqpoint{1.635883in}{1.108736in}}{\pgfqpoint{1.641707in}{1.114560in}}%
\pgfpathcurveto{\pgfqpoint{1.647531in}{1.120384in}}{\pgfqpoint{1.650803in}{1.128284in}}{\pgfqpoint{1.650803in}{1.136521in}}%
\pgfpathcurveto{\pgfqpoint{1.650803in}{1.144757in}}{\pgfqpoint{1.647531in}{1.152657in}}{\pgfqpoint{1.641707in}{1.158481in}}%
\pgfpathcurveto{\pgfqpoint{1.635883in}{1.164305in}}{\pgfqpoint{1.627983in}{1.167577in}}{\pgfqpoint{1.619747in}{1.167577in}}%
\pgfpathcurveto{\pgfqpoint{1.611511in}{1.167577in}}{\pgfqpoint{1.603610in}{1.164305in}}{\pgfqpoint{1.597787in}{1.158481in}}%
\pgfpathcurveto{\pgfqpoint{1.591963in}{1.152657in}}{\pgfqpoint{1.588690in}{1.144757in}}{\pgfqpoint{1.588690in}{1.136521in}}%
\pgfpathcurveto{\pgfqpoint{1.588690in}{1.128284in}}{\pgfqpoint{1.591963in}{1.120384in}}{\pgfqpoint{1.597787in}{1.114560in}}%
\pgfpathcurveto{\pgfqpoint{1.603610in}{1.108736in}}{\pgfqpoint{1.611511in}{1.105464in}}{\pgfqpoint{1.619747in}{1.105464in}}%
\pgfpathclose%
\pgfusepath{stroke,fill}%
\end{pgfscope}%
\begin{pgfscope}%
\pgfpathrectangle{\pgfqpoint{0.100000in}{0.212622in}}{\pgfqpoint{3.696000in}{3.696000in}}%
\pgfusepath{clip}%
\pgfsetbuttcap%
\pgfsetroundjoin%
\definecolor{currentfill}{rgb}{0.121569,0.466667,0.705882}%
\pgfsetfillcolor{currentfill}%
\pgfsetfillopacity{0.829951}%
\pgfsetlinewidth{1.003750pt}%
\definecolor{currentstroke}{rgb}{0.121569,0.466667,0.705882}%
\pgfsetstrokecolor{currentstroke}%
\pgfsetstrokeopacity{0.829951}%
\pgfsetdash{}{0pt}%
\pgfpathmoveto{\pgfqpoint{1.631557in}{1.100739in}}%
\pgfpathcurveto{\pgfqpoint{1.639793in}{1.100739in}}{\pgfqpoint{1.647693in}{1.104012in}}{\pgfqpoint{1.653517in}{1.109836in}}%
\pgfpathcurveto{\pgfqpoint{1.659341in}{1.115659in}}{\pgfqpoint{1.662614in}{1.123560in}}{\pgfqpoint{1.662614in}{1.131796in}}%
\pgfpathcurveto{\pgfqpoint{1.662614in}{1.140032in}}{\pgfqpoint{1.659341in}{1.147932in}}{\pgfqpoint{1.653517in}{1.153756in}}%
\pgfpathcurveto{\pgfqpoint{1.647693in}{1.159580in}}{\pgfqpoint{1.639793in}{1.162852in}}{\pgfqpoint{1.631557in}{1.162852in}}%
\pgfpathcurveto{\pgfqpoint{1.623321in}{1.162852in}}{\pgfqpoint{1.615421in}{1.159580in}}{\pgfqpoint{1.609597in}{1.153756in}}%
\pgfpathcurveto{\pgfqpoint{1.603773in}{1.147932in}}{\pgfqpoint{1.600501in}{1.140032in}}{\pgfqpoint{1.600501in}{1.131796in}}%
\pgfpathcurveto{\pgfqpoint{1.600501in}{1.123560in}}{\pgfqpoint{1.603773in}{1.115659in}}{\pgfqpoint{1.609597in}{1.109836in}}%
\pgfpathcurveto{\pgfqpoint{1.615421in}{1.104012in}}{\pgfqpoint{1.623321in}{1.100739in}}{\pgfqpoint{1.631557in}{1.100739in}}%
\pgfpathclose%
\pgfusepath{stroke,fill}%
\end{pgfscope}%
\begin{pgfscope}%
\pgfpathrectangle{\pgfqpoint{0.100000in}{0.212622in}}{\pgfqpoint{3.696000in}{3.696000in}}%
\pgfusepath{clip}%
\pgfsetbuttcap%
\pgfsetroundjoin%
\definecolor{currentfill}{rgb}{0.121569,0.466667,0.705882}%
\pgfsetfillcolor{currentfill}%
\pgfsetfillopacity{0.832228}%
\pgfsetlinewidth{1.003750pt}%
\definecolor{currentstroke}{rgb}{0.121569,0.466667,0.705882}%
\pgfsetstrokecolor{currentstroke}%
\pgfsetstrokeopacity{0.832228}%
\pgfsetdash{}{0pt}%
\pgfpathmoveto{\pgfqpoint{1.642247in}{1.096723in}}%
\pgfpathcurveto{\pgfqpoint{1.650483in}{1.096723in}}{\pgfqpoint{1.658383in}{1.099996in}}{\pgfqpoint{1.664207in}{1.105819in}}%
\pgfpathcurveto{\pgfqpoint{1.670031in}{1.111643in}}{\pgfqpoint{1.673303in}{1.119543in}}{\pgfqpoint{1.673303in}{1.127780in}}%
\pgfpathcurveto{\pgfqpoint{1.673303in}{1.136016in}}{\pgfqpoint{1.670031in}{1.143916in}}{\pgfqpoint{1.664207in}{1.149740in}}%
\pgfpathcurveto{\pgfqpoint{1.658383in}{1.155564in}}{\pgfqpoint{1.650483in}{1.158836in}}{\pgfqpoint{1.642247in}{1.158836in}}%
\pgfpathcurveto{\pgfqpoint{1.634011in}{1.158836in}}{\pgfqpoint{1.626111in}{1.155564in}}{\pgfqpoint{1.620287in}{1.149740in}}%
\pgfpathcurveto{\pgfqpoint{1.614463in}{1.143916in}}{\pgfqpoint{1.611190in}{1.136016in}}{\pgfqpoint{1.611190in}{1.127780in}}%
\pgfpathcurveto{\pgfqpoint{1.611190in}{1.119543in}}{\pgfqpoint{1.614463in}{1.111643in}}{\pgfqpoint{1.620287in}{1.105819in}}%
\pgfpathcurveto{\pgfqpoint{1.626111in}{1.099996in}}{\pgfqpoint{1.634011in}{1.096723in}}{\pgfqpoint{1.642247in}{1.096723in}}%
\pgfpathclose%
\pgfusepath{stroke,fill}%
\end{pgfscope}%
\begin{pgfscope}%
\pgfpathrectangle{\pgfqpoint{0.100000in}{0.212622in}}{\pgfqpoint{3.696000in}{3.696000in}}%
\pgfusepath{clip}%
\pgfsetbuttcap%
\pgfsetroundjoin%
\definecolor{currentfill}{rgb}{0.121569,0.466667,0.705882}%
\pgfsetfillcolor{currentfill}%
\pgfsetfillopacity{0.832337}%
\pgfsetlinewidth{1.003750pt}%
\definecolor{currentstroke}{rgb}{0.121569,0.466667,0.705882}%
\pgfsetstrokecolor{currentstroke}%
\pgfsetstrokeopacity{0.832337}%
\pgfsetdash{}{0pt}%
\pgfpathmoveto{\pgfqpoint{2.294440in}{1.339369in}}%
\pgfpathcurveto{\pgfqpoint{2.302676in}{1.339369in}}{\pgfqpoint{2.310576in}{1.342641in}}{\pgfqpoint{2.316400in}{1.348465in}}%
\pgfpathcurveto{\pgfqpoint{2.322224in}{1.354289in}}{\pgfqpoint{2.325496in}{1.362189in}}{\pgfqpoint{2.325496in}{1.370426in}}%
\pgfpathcurveto{\pgfqpoint{2.325496in}{1.378662in}}{\pgfqpoint{2.322224in}{1.386562in}}{\pgfqpoint{2.316400in}{1.392386in}}%
\pgfpathcurveto{\pgfqpoint{2.310576in}{1.398210in}}{\pgfqpoint{2.302676in}{1.401482in}}{\pgfqpoint{2.294440in}{1.401482in}}%
\pgfpathcurveto{\pgfqpoint{2.286204in}{1.401482in}}{\pgfqpoint{2.278304in}{1.398210in}}{\pgfqpoint{2.272480in}{1.392386in}}%
\pgfpathcurveto{\pgfqpoint{2.266656in}{1.386562in}}{\pgfqpoint{2.263383in}{1.378662in}}{\pgfqpoint{2.263383in}{1.370426in}}%
\pgfpathcurveto{\pgfqpoint{2.263383in}{1.362189in}}{\pgfqpoint{2.266656in}{1.354289in}}{\pgfqpoint{2.272480in}{1.348465in}}%
\pgfpathcurveto{\pgfqpoint{2.278304in}{1.342641in}}{\pgfqpoint{2.286204in}{1.339369in}}{\pgfqpoint{2.294440in}{1.339369in}}%
\pgfpathclose%
\pgfusepath{stroke,fill}%
\end{pgfscope}%
\begin{pgfscope}%
\pgfpathrectangle{\pgfqpoint{0.100000in}{0.212622in}}{\pgfqpoint{3.696000in}{3.696000in}}%
\pgfusepath{clip}%
\pgfsetbuttcap%
\pgfsetroundjoin%
\definecolor{currentfill}{rgb}{0.121569,0.466667,0.705882}%
\pgfsetfillcolor{currentfill}%
\pgfsetfillopacity{0.834437}%
\pgfsetlinewidth{1.003750pt}%
\definecolor{currentstroke}{rgb}{0.121569,0.466667,0.705882}%
\pgfsetstrokecolor{currentstroke}%
\pgfsetstrokeopacity{0.834437}%
\pgfsetdash{}{0pt}%
\pgfpathmoveto{\pgfqpoint{1.651936in}{1.093699in}}%
\pgfpathcurveto{\pgfqpoint{1.660173in}{1.093699in}}{\pgfqpoint{1.668073in}{1.096971in}}{\pgfqpoint{1.673897in}{1.102795in}}%
\pgfpathcurveto{\pgfqpoint{1.679721in}{1.108619in}}{\pgfqpoint{1.682993in}{1.116519in}}{\pgfqpoint{1.682993in}{1.124755in}}%
\pgfpathcurveto{\pgfqpoint{1.682993in}{1.132992in}}{\pgfqpoint{1.679721in}{1.140892in}}{\pgfqpoint{1.673897in}{1.146716in}}%
\pgfpathcurveto{\pgfqpoint{1.668073in}{1.152540in}}{\pgfqpoint{1.660173in}{1.155812in}}{\pgfqpoint{1.651936in}{1.155812in}}%
\pgfpathcurveto{\pgfqpoint{1.643700in}{1.155812in}}{\pgfqpoint{1.635800in}{1.152540in}}{\pgfqpoint{1.629976in}{1.146716in}}%
\pgfpathcurveto{\pgfqpoint{1.624152in}{1.140892in}}{\pgfqpoint{1.620880in}{1.132992in}}{\pgfqpoint{1.620880in}{1.124755in}}%
\pgfpathcurveto{\pgfqpoint{1.620880in}{1.116519in}}{\pgfqpoint{1.624152in}{1.108619in}}{\pgfqpoint{1.629976in}{1.102795in}}%
\pgfpathcurveto{\pgfqpoint{1.635800in}{1.096971in}}{\pgfqpoint{1.643700in}{1.093699in}}{\pgfqpoint{1.651936in}{1.093699in}}%
\pgfpathclose%
\pgfusepath{stroke,fill}%
\end{pgfscope}%
\begin{pgfscope}%
\pgfpathrectangle{\pgfqpoint{0.100000in}{0.212622in}}{\pgfqpoint{3.696000in}{3.696000in}}%
\pgfusepath{clip}%
\pgfsetbuttcap%
\pgfsetroundjoin%
\definecolor{currentfill}{rgb}{0.121569,0.466667,0.705882}%
\pgfsetfillcolor{currentfill}%
\pgfsetfillopacity{0.838407}%
\pgfsetlinewidth{1.003750pt}%
\definecolor{currentstroke}{rgb}{0.121569,0.466667,0.705882}%
\pgfsetstrokecolor{currentstroke}%
\pgfsetstrokeopacity{0.838407}%
\pgfsetdash{}{0pt}%
\pgfpathmoveto{\pgfqpoint{1.669778in}{1.088775in}}%
\pgfpathcurveto{\pgfqpoint{1.678014in}{1.088775in}}{\pgfqpoint{1.685914in}{1.092048in}}{\pgfqpoint{1.691738in}{1.097872in}}%
\pgfpathcurveto{\pgfqpoint{1.697562in}{1.103696in}}{\pgfqpoint{1.700834in}{1.111596in}}{\pgfqpoint{1.700834in}{1.119832in}}%
\pgfpathcurveto{\pgfqpoint{1.700834in}{1.128068in}}{\pgfqpoint{1.697562in}{1.135968in}}{\pgfqpoint{1.691738in}{1.141792in}}%
\pgfpathcurveto{\pgfqpoint{1.685914in}{1.147616in}}{\pgfqpoint{1.678014in}{1.150888in}}{\pgfqpoint{1.669778in}{1.150888in}}%
\pgfpathcurveto{\pgfqpoint{1.661542in}{1.150888in}}{\pgfqpoint{1.653641in}{1.147616in}}{\pgfqpoint{1.647818in}{1.141792in}}%
\pgfpathcurveto{\pgfqpoint{1.641994in}{1.135968in}}{\pgfqpoint{1.638721in}{1.128068in}}{\pgfqpoint{1.638721in}{1.119832in}}%
\pgfpathcurveto{\pgfqpoint{1.638721in}{1.111596in}}{\pgfqpoint{1.641994in}{1.103696in}}{\pgfqpoint{1.647818in}{1.097872in}}%
\pgfpathcurveto{\pgfqpoint{1.653641in}{1.092048in}}{\pgfqpoint{1.661542in}{1.088775in}}{\pgfqpoint{1.669778in}{1.088775in}}%
\pgfpathclose%
\pgfusepath{stroke,fill}%
\end{pgfscope}%
\begin{pgfscope}%
\pgfpathrectangle{\pgfqpoint{0.100000in}{0.212622in}}{\pgfqpoint{3.696000in}{3.696000in}}%
\pgfusepath{clip}%
\pgfsetbuttcap%
\pgfsetroundjoin%
\definecolor{currentfill}{rgb}{0.121569,0.466667,0.705882}%
\pgfsetfillcolor{currentfill}%
\pgfsetfillopacity{0.839193}%
\pgfsetlinewidth{1.003750pt}%
\definecolor{currentstroke}{rgb}{0.121569,0.466667,0.705882}%
\pgfsetstrokecolor{currentstroke}%
\pgfsetstrokeopacity{0.839193}%
\pgfsetdash{}{0pt}%
\pgfpathmoveto{\pgfqpoint{2.300775in}{1.315213in}}%
\pgfpathcurveto{\pgfqpoint{2.309011in}{1.315213in}}{\pgfqpoint{2.316911in}{1.318485in}}{\pgfqpoint{2.322735in}{1.324309in}}%
\pgfpathcurveto{\pgfqpoint{2.328559in}{1.330133in}}{\pgfqpoint{2.331831in}{1.338033in}}{\pgfqpoint{2.331831in}{1.346270in}}%
\pgfpathcurveto{\pgfqpoint{2.331831in}{1.354506in}}{\pgfqpoint{2.328559in}{1.362406in}}{\pgfqpoint{2.322735in}{1.368230in}}%
\pgfpathcurveto{\pgfqpoint{2.316911in}{1.374054in}}{\pgfqpoint{2.309011in}{1.377326in}}{\pgfqpoint{2.300775in}{1.377326in}}%
\pgfpathcurveto{\pgfqpoint{2.292539in}{1.377326in}}{\pgfqpoint{2.284639in}{1.374054in}}{\pgfqpoint{2.278815in}{1.368230in}}%
\pgfpathcurveto{\pgfqpoint{2.272991in}{1.362406in}}{\pgfqpoint{2.269718in}{1.354506in}}{\pgfqpoint{2.269718in}{1.346270in}}%
\pgfpathcurveto{\pgfqpoint{2.269718in}{1.338033in}}{\pgfqpoint{2.272991in}{1.330133in}}{\pgfqpoint{2.278815in}{1.324309in}}%
\pgfpathcurveto{\pgfqpoint{2.284639in}{1.318485in}}{\pgfqpoint{2.292539in}{1.315213in}}{\pgfqpoint{2.300775in}{1.315213in}}%
\pgfpathclose%
\pgfusepath{stroke,fill}%
\end{pgfscope}%
\begin{pgfscope}%
\pgfpathrectangle{\pgfqpoint{0.100000in}{0.212622in}}{\pgfqpoint{3.696000in}{3.696000in}}%
\pgfusepath{clip}%
\pgfsetbuttcap%
\pgfsetroundjoin%
\definecolor{currentfill}{rgb}{0.121569,0.466667,0.705882}%
\pgfsetfillcolor{currentfill}%
\pgfsetfillopacity{0.841861}%
\pgfsetlinewidth{1.003750pt}%
\definecolor{currentstroke}{rgb}{0.121569,0.466667,0.705882}%
\pgfsetstrokecolor{currentstroke}%
\pgfsetstrokeopacity{0.841861}%
\pgfsetdash{}{0pt}%
\pgfpathmoveto{\pgfqpoint{1.685912in}{1.082975in}}%
\pgfpathcurveto{\pgfqpoint{1.694148in}{1.082975in}}{\pgfqpoint{1.702048in}{1.086247in}}{\pgfqpoint{1.707872in}{1.092071in}}%
\pgfpathcurveto{\pgfqpoint{1.713696in}{1.097895in}}{\pgfqpoint{1.716968in}{1.105795in}}{\pgfqpoint{1.716968in}{1.114032in}}%
\pgfpathcurveto{\pgfqpoint{1.716968in}{1.122268in}}{\pgfqpoint{1.713696in}{1.130168in}}{\pgfqpoint{1.707872in}{1.135992in}}%
\pgfpathcurveto{\pgfqpoint{1.702048in}{1.141816in}}{\pgfqpoint{1.694148in}{1.145088in}}{\pgfqpoint{1.685912in}{1.145088in}}%
\pgfpathcurveto{\pgfqpoint{1.677675in}{1.145088in}}{\pgfqpoint{1.669775in}{1.141816in}}{\pgfqpoint{1.663951in}{1.135992in}}%
\pgfpathcurveto{\pgfqpoint{1.658127in}{1.130168in}}{\pgfqpoint{1.654855in}{1.122268in}}{\pgfqpoint{1.654855in}{1.114032in}}%
\pgfpathcurveto{\pgfqpoint{1.654855in}{1.105795in}}{\pgfqpoint{1.658127in}{1.097895in}}{\pgfqpoint{1.663951in}{1.092071in}}%
\pgfpathcurveto{\pgfqpoint{1.669775in}{1.086247in}}{\pgfqpoint{1.677675in}{1.082975in}}{\pgfqpoint{1.685912in}{1.082975in}}%
\pgfpathclose%
\pgfusepath{stroke,fill}%
\end{pgfscope}%
\begin{pgfscope}%
\pgfpathrectangle{\pgfqpoint{0.100000in}{0.212622in}}{\pgfqpoint{3.696000in}{3.696000in}}%
\pgfusepath{clip}%
\pgfsetbuttcap%
\pgfsetroundjoin%
\definecolor{currentfill}{rgb}{0.121569,0.466667,0.705882}%
\pgfsetfillcolor{currentfill}%
\pgfsetfillopacity{0.845255}%
\pgfsetlinewidth{1.003750pt}%
\definecolor{currentstroke}{rgb}{0.121569,0.466667,0.705882}%
\pgfsetstrokecolor{currentstroke}%
\pgfsetstrokeopacity{0.845255}%
\pgfsetdash{}{0pt}%
\pgfpathmoveto{\pgfqpoint{1.701708in}{1.076960in}}%
\pgfpathcurveto{\pgfqpoint{1.709944in}{1.076960in}}{\pgfqpoint{1.717844in}{1.080233in}}{\pgfqpoint{1.723668in}{1.086057in}}%
\pgfpathcurveto{\pgfqpoint{1.729492in}{1.091881in}}{\pgfqpoint{1.732764in}{1.099781in}}{\pgfqpoint{1.732764in}{1.108017in}}%
\pgfpathcurveto{\pgfqpoint{1.732764in}{1.116253in}}{\pgfqpoint{1.729492in}{1.124153in}}{\pgfqpoint{1.723668in}{1.129977in}}%
\pgfpathcurveto{\pgfqpoint{1.717844in}{1.135801in}}{\pgfqpoint{1.709944in}{1.139073in}}{\pgfqpoint{1.701708in}{1.139073in}}%
\pgfpathcurveto{\pgfqpoint{1.693472in}{1.139073in}}{\pgfqpoint{1.685572in}{1.135801in}}{\pgfqpoint{1.679748in}{1.129977in}}%
\pgfpathcurveto{\pgfqpoint{1.673924in}{1.124153in}}{\pgfqpoint{1.670651in}{1.116253in}}{\pgfqpoint{1.670651in}{1.108017in}}%
\pgfpathcurveto{\pgfqpoint{1.670651in}{1.099781in}}{\pgfqpoint{1.673924in}{1.091881in}}{\pgfqpoint{1.679748in}{1.086057in}}%
\pgfpathcurveto{\pgfqpoint{1.685572in}{1.080233in}}{\pgfqpoint{1.693472in}{1.076960in}}{\pgfqpoint{1.701708in}{1.076960in}}%
\pgfpathclose%
\pgfusepath{stroke,fill}%
\end{pgfscope}%
\begin{pgfscope}%
\pgfpathrectangle{\pgfqpoint{0.100000in}{0.212622in}}{\pgfqpoint{3.696000in}{3.696000in}}%
\pgfusepath{clip}%
\pgfsetbuttcap%
\pgfsetroundjoin%
\definecolor{currentfill}{rgb}{0.121569,0.466667,0.705882}%
\pgfsetfillcolor{currentfill}%
\pgfsetfillopacity{0.845575}%
\pgfsetlinewidth{1.003750pt}%
\definecolor{currentstroke}{rgb}{0.121569,0.466667,0.705882}%
\pgfsetstrokecolor{currentstroke}%
\pgfsetstrokeopacity{0.845575}%
\pgfsetdash{}{0pt}%
\pgfpathmoveto{\pgfqpoint{2.308324in}{1.289722in}}%
\pgfpathcurveto{\pgfqpoint{2.316560in}{1.289722in}}{\pgfqpoint{2.324460in}{1.292995in}}{\pgfqpoint{2.330284in}{1.298819in}}%
\pgfpathcurveto{\pgfqpoint{2.336108in}{1.304643in}}{\pgfqpoint{2.339381in}{1.312543in}}{\pgfqpoint{2.339381in}{1.320779in}}%
\pgfpathcurveto{\pgfqpoint{2.339381in}{1.329015in}}{\pgfqpoint{2.336108in}{1.336915in}}{\pgfqpoint{2.330284in}{1.342739in}}%
\pgfpathcurveto{\pgfqpoint{2.324460in}{1.348563in}}{\pgfqpoint{2.316560in}{1.351835in}}{\pgfqpoint{2.308324in}{1.351835in}}%
\pgfpathcurveto{\pgfqpoint{2.300088in}{1.351835in}}{\pgfqpoint{2.292188in}{1.348563in}}{\pgfqpoint{2.286364in}{1.342739in}}%
\pgfpathcurveto{\pgfqpoint{2.280540in}{1.336915in}}{\pgfqpoint{2.277268in}{1.329015in}}{\pgfqpoint{2.277268in}{1.320779in}}%
\pgfpathcurveto{\pgfqpoint{2.277268in}{1.312543in}}{\pgfqpoint{2.280540in}{1.304643in}}{\pgfqpoint{2.286364in}{1.298819in}}%
\pgfpathcurveto{\pgfqpoint{2.292188in}{1.292995in}}{\pgfqpoint{2.300088in}{1.289722in}}{\pgfqpoint{2.308324in}{1.289722in}}%
\pgfpathclose%
\pgfusepath{stroke,fill}%
\end{pgfscope}%
\begin{pgfscope}%
\pgfpathrectangle{\pgfqpoint{0.100000in}{0.212622in}}{\pgfqpoint{3.696000in}{3.696000in}}%
\pgfusepath{clip}%
\pgfsetbuttcap%
\pgfsetroundjoin%
\definecolor{currentfill}{rgb}{0.121569,0.466667,0.705882}%
\pgfsetfillcolor{currentfill}%
\pgfsetfillopacity{0.848412}%
\pgfsetlinewidth{1.003750pt}%
\definecolor{currentstroke}{rgb}{0.121569,0.466667,0.705882}%
\pgfsetstrokecolor{currentstroke}%
\pgfsetstrokeopacity{0.848412}%
\pgfsetdash{}{0pt}%
\pgfpathmoveto{\pgfqpoint{1.715851in}{1.072419in}}%
\pgfpathcurveto{\pgfqpoint{1.724087in}{1.072419in}}{\pgfqpoint{1.731987in}{1.075691in}}{\pgfqpoint{1.737811in}{1.081515in}}%
\pgfpathcurveto{\pgfqpoint{1.743635in}{1.087339in}}{\pgfqpoint{1.746907in}{1.095239in}}{\pgfqpoint{1.746907in}{1.103475in}}%
\pgfpathcurveto{\pgfqpoint{1.746907in}{1.111712in}}{\pgfqpoint{1.743635in}{1.119612in}}{\pgfqpoint{1.737811in}{1.125436in}}%
\pgfpathcurveto{\pgfqpoint{1.731987in}{1.131260in}}{\pgfqpoint{1.724087in}{1.134532in}}{\pgfqpoint{1.715851in}{1.134532in}}%
\pgfpathcurveto{\pgfqpoint{1.707615in}{1.134532in}}{\pgfqpoint{1.699715in}{1.131260in}}{\pgfqpoint{1.693891in}{1.125436in}}%
\pgfpathcurveto{\pgfqpoint{1.688067in}{1.119612in}}{\pgfqpoint{1.684794in}{1.111712in}}{\pgfqpoint{1.684794in}{1.103475in}}%
\pgfpathcurveto{\pgfqpoint{1.684794in}{1.095239in}}{\pgfqpoint{1.688067in}{1.087339in}}{\pgfqpoint{1.693891in}{1.081515in}}%
\pgfpathcurveto{\pgfqpoint{1.699715in}{1.075691in}}{\pgfqpoint{1.707615in}{1.072419in}}{\pgfqpoint{1.715851in}{1.072419in}}%
\pgfpathclose%
\pgfusepath{stroke,fill}%
\end{pgfscope}%
\begin{pgfscope}%
\pgfpathrectangle{\pgfqpoint{0.100000in}{0.212622in}}{\pgfqpoint{3.696000in}{3.696000in}}%
\pgfusepath{clip}%
\pgfsetbuttcap%
\pgfsetroundjoin%
\definecolor{currentfill}{rgb}{0.121569,0.466667,0.705882}%
\pgfsetfillcolor{currentfill}%
\pgfsetfillopacity{0.851267}%
\pgfsetlinewidth{1.003750pt}%
\definecolor{currentstroke}{rgb}{0.121569,0.466667,0.705882}%
\pgfsetstrokecolor{currentstroke}%
\pgfsetstrokeopacity{0.851267}%
\pgfsetdash{}{0pt}%
\pgfpathmoveto{\pgfqpoint{1.729567in}{1.068098in}}%
\pgfpathcurveto{\pgfqpoint{1.737803in}{1.068098in}}{\pgfqpoint{1.745703in}{1.071371in}}{\pgfqpoint{1.751527in}{1.077195in}}%
\pgfpathcurveto{\pgfqpoint{1.757351in}{1.083018in}}{\pgfqpoint{1.760623in}{1.090919in}}{\pgfqpoint{1.760623in}{1.099155in}}%
\pgfpathcurveto{\pgfqpoint{1.760623in}{1.107391in}}{\pgfqpoint{1.757351in}{1.115291in}}{\pgfqpoint{1.751527in}{1.121115in}}%
\pgfpathcurveto{\pgfqpoint{1.745703in}{1.126939in}}{\pgfqpoint{1.737803in}{1.130211in}}{\pgfqpoint{1.729567in}{1.130211in}}%
\pgfpathcurveto{\pgfqpoint{1.721330in}{1.130211in}}{\pgfqpoint{1.713430in}{1.126939in}}{\pgfqpoint{1.707606in}{1.121115in}}%
\pgfpathcurveto{\pgfqpoint{1.701782in}{1.115291in}}{\pgfqpoint{1.698510in}{1.107391in}}{\pgfqpoint{1.698510in}{1.099155in}}%
\pgfpathcurveto{\pgfqpoint{1.698510in}{1.090919in}}{\pgfqpoint{1.701782in}{1.083018in}}{\pgfqpoint{1.707606in}{1.077195in}}%
\pgfpathcurveto{\pgfqpoint{1.713430in}{1.071371in}}{\pgfqpoint{1.721330in}{1.068098in}}{\pgfqpoint{1.729567in}{1.068098in}}%
\pgfpathclose%
\pgfusepath{stroke,fill}%
\end{pgfscope}%
\begin{pgfscope}%
\pgfpathrectangle{\pgfqpoint{0.100000in}{0.212622in}}{\pgfqpoint{3.696000in}{3.696000in}}%
\pgfusepath{clip}%
\pgfsetbuttcap%
\pgfsetroundjoin%
\definecolor{currentfill}{rgb}{0.121569,0.466667,0.705882}%
\pgfsetfillcolor{currentfill}%
\pgfsetfillopacity{0.853031}%
\pgfsetlinewidth{1.003750pt}%
\definecolor{currentstroke}{rgb}{0.121569,0.466667,0.705882}%
\pgfsetstrokecolor{currentstroke}%
\pgfsetstrokeopacity{0.853031}%
\pgfsetdash{}{0pt}%
\pgfpathmoveto{\pgfqpoint{2.314129in}{1.264505in}}%
\pgfpathcurveto{\pgfqpoint{2.322366in}{1.264505in}}{\pgfqpoint{2.330266in}{1.267777in}}{\pgfqpoint{2.336090in}{1.273601in}}%
\pgfpathcurveto{\pgfqpoint{2.341914in}{1.279425in}}{\pgfqpoint{2.345186in}{1.287325in}}{\pgfqpoint{2.345186in}{1.295562in}}%
\pgfpathcurveto{\pgfqpoint{2.345186in}{1.303798in}}{\pgfqpoint{2.341914in}{1.311698in}}{\pgfqpoint{2.336090in}{1.317522in}}%
\pgfpathcurveto{\pgfqpoint{2.330266in}{1.323346in}}{\pgfqpoint{2.322366in}{1.326618in}}{\pgfqpoint{2.314129in}{1.326618in}}%
\pgfpathcurveto{\pgfqpoint{2.305893in}{1.326618in}}{\pgfqpoint{2.297993in}{1.323346in}}{\pgfqpoint{2.292169in}{1.317522in}}%
\pgfpathcurveto{\pgfqpoint{2.286345in}{1.311698in}}{\pgfqpoint{2.283073in}{1.303798in}}{\pgfqpoint{2.283073in}{1.295562in}}%
\pgfpathcurveto{\pgfqpoint{2.283073in}{1.287325in}}{\pgfqpoint{2.286345in}{1.279425in}}{\pgfqpoint{2.292169in}{1.273601in}}%
\pgfpathcurveto{\pgfqpoint{2.297993in}{1.267777in}}{\pgfqpoint{2.305893in}{1.264505in}}{\pgfqpoint{2.314129in}{1.264505in}}%
\pgfpathclose%
\pgfusepath{stroke,fill}%
\end{pgfscope}%
\begin{pgfscope}%
\pgfpathrectangle{\pgfqpoint{0.100000in}{0.212622in}}{\pgfqpoint{3.696000in}{3.696000in}}%
\pgfusepath{clip}%
\pgfsetbuttcap%
\pgfsetroundjoin%
\definecolor{currentfill}{rgb}{0.121569,0.466667,0.705882}%
\pgfsetfillcolor{currentfill}%
\pgfsetfillopacity{0.854106}%
\pgfsetlinewidth{1.003750pt}%
\definecolor{currentstroke}{rgb}{0.121569,0.466667,0.705882}%
\pgfsetstrokecolor{currentstroke}%
\pgfsetstrokeopacity{0.854106}%
\pgfsetdash{}{0pt}%
\pgfpathmoveto{\pgfqpoint{1.742974in}{1.063901in}}%
\pgfpathcurveto{\pgfqpoint{1.751210in}{1.063901in}}{\pgfqpoint{1.759110in}{1.067173in}}{\pgfqpoint{1.764934in}{1.072997in}}%
\pgfpathcurveto{\pgfqpoint{1.770758in}{1.078821in}}{\pgfqpoint{1.774030in}{1.086721in}}{\pgfqpoint{1.774030in}{1.094957in}}%
\pgfpathcurveto{\pgfqpoint{1.774030in}{1.103194in}}{\pgfqpoint{1.770758in}{1.111094in}}{\pgfqpoint{1.764934in}{1.116918in}}%
\pgfpathcurveto{\pgfqpoint{1.759110in}{1.122741in}}{\pgfqpoint{1.751210in}{1.126014in}}{\pgfqpoint{1.742974in}{1.126014in}}%
\pgfpathcurveto{\pgfqpoint{1.734738in}{1.126014in}}{\pgfqpoint{1.726838in}{1.122741in}}{\pgfqpoint{1.721014in}{1.116918in}}%
\pgfpathcurveto{\pgfqpoint{1.715190in}{1.111094in}}{\pgfqpoint{1.711917in}{1.103194in}}{\pgfqpoint{1.711917in}{1.094957in}}%
\pgfpathcurveto{\pgfqpoint{1.711917in}{1.086721in}}{\pgfqpoint{1.715190in}{1.078821in}}{\pgfqpoint{1.721014in}{1.072997in}}%
\pgfpathcurveto{\pgfqpoint{1.726838in}{1.067173in}}{\pgfqpoint{1.734738in}{1.063901in}}{\pgfqpoint{1.742974in}{1.063901in}}%
\pgfpathclose%
\pgfusepath{stroke,fill}%
\end{pgfscope}%
\begin{pgfscope}%
\pgfpathrectangle{\pgfqpoint{0.100000in}{0.212622in}}{\pgfqpoint{3.696000in}{3.696000in}}%
\pgfusepath{clip}%
\pgfsetbuttcap%
\pgfsetroundjoin%
\definecolor{currentfill}{rgb}{0.121569,0.466667,0.705882}%
\pgfsetfillcolor{currentfill}%
\pgfsetfillopacity{0.856644}%
\pgfsetlinewidth{1.003750pt}%
\definecolor{currentstroke}{rgb}{0.121569,0.466667,0.705882}%
\pgfsetstrokecolor{currentstroke}%
\pgfsetstrokeopacity{0.856644}%
\pgfsetdash{}{0pt}%
\pgfpathmoveto{\pgfqpoint{1.754419in}{1.060556in}}%
\pgfpathcurveto{\pgfqpoint{1.762655in}{1.060556in}}{\pgfqpoint{1.770555in}{1.063828in}}{\pgfqpoint{1.776379in}{1.069652in}}%
\pgfpathcurveto{\pgfqpoint{1.782203in}{1.075476in}}{\pgfqpoint{1.785475in}{1.083376in}}{\pgfqpoint{1.785475in}{1.091612in}}%
\pgfpathcurveto{\pgfqpoint{1.785475in}{1.099848in}}{\pgfqpoint{1.782203in}{1.107748in}}{\pgfqpoint{1.776379in}{1.113572in}}%
\pgfpathcurveto{\pgfqpoint{1.770555in}{1.119396in}}{\pgfqpoint{1.762655in}{1.122669in}}{\pgfqpoint{1.754419in}{1.122669in}}%
\pgfpathcurveto{\pgfqpoint{1.746182in}{1.122669in}}{\pgfqpoint{1.738282in}{1.119396in}}{\pgfqpoint{1.732458in}{1.113572in}}%
\pgfpathcurveto{\pgfqpoint{1.726634in}{1.107748in}}{\pgfqpoint{1.723362in}{1.099848in}}{\pgfqpoint{1.723362in}{1.091612in}}%
\pgfpathcurveto{\pgfqpoint{1.723362in}{1.083376in}}{\pgfqpoint{1.726634in}{1.075476in}}{\pgfqpoint{1.732458in}{1.069652in}}%
\pgfpathcurveto{\pgfqpoint{1.738282in}{1.063828in}}{\pgfqpoint{1.746182in}{1.060556in}}{\pgfqpoint{1.754419in}{1.060556in}}%
\pgfpathclose%
\pgfusepath{stroke,fill}%
\end{pgfscope}%
\begin{pgfscope}%
\pgfpathrectangle{\pgfqpoint{0.100000in}{0.212622in}}{\pgfqpoint{3.696000in}{3.696000in}}%
\pgfusepath{clip}%
\pgfsetbuttcap%
\pgfsetroundjoin%
\definecolor{currentfill}{rgb}{0.121569,0.466667,0.705882}%
\pgfsetfillcolor{currentfill}%
\pgfsetfillopacity{0.859006}%
\pgfsetlinewidth{1.003750pt}%
\definecolor{currentstroke}{rgb}{0.121569,0.466667,0.705882}%
\pgfsetstrokecolor{currentstroke}%
\pgfsetstrokeopacity{0.859006}%
\pgfsetdash{}{0pt}%
\pgfpathmoveto{\pgfqpoint{1.765604in}{1.057092in}}%
\pgfpathcurveto{\pgfqpoint{1.773840in}{1.057092in}}{\pgfqpoint{1.781740in}{1.060364in}}{\pgfqpoint{1.787564in}{1.066188in}}%
\pgfpathcurveto{\pgfqpoint{1.793388in}{1.072012in}}{\pgfqpoint{1.796660in}{1.079912in}}{\pgfqpoint{1.796660in}{1.088148in}}%
\pgfpathcurveto{\pgfqpoint{1.796660in}{1.096384in}}{\pgfqpoint{1.793388in}{1.104284in}}{\pgfqpoint{1.787564in}{1.110108in}}%
\pgfpathcurveto{\pgfqpoint{1.781740in}{1.115932in}}{\pgfqpoint{1.773840in}{1.119205in}}{\pgfqpoint{1.765604in}{1.119205in}}%
\pgfpathcurveto{\pgfqpoint{1.757368in}{1.119205in}}{\pgfqpoint{1.749468in}{1.115932in}}{\pgfqpoint{1.743644in}{1.110108in}}%
\pgfpathcurveto{\pgfqpoint{1.737820in}{1.104284in}}{\pgfqpoint{1.734547in}{1.096384in}}{\pgfqpoint{1.734547in}{1.088148in}}%
\pgfpathcurveto{\pgfqpoint{1.734547in}{1.079912in}}{\pgfqpoint{1.737820in}{1.072012in}}{\pgfqpoint{1.743644in}{1.066188in}}%
\pgfpathcurveto{\pgfqpoint{1.749468in}{1.060364in}}{\pgfqpoint{1.757368in}{1.057092in}}{\pgfqpoint{1.765604in}{1.057092in}}%
\pgfpathclose%
\pgfusepath{stroke,fill}%
\end{pgfscope}%
\begin{pgfscope}%
\pgfpathrectangle{\pgfqpoint{0.100000in}{0.212622in}}{\pgfqpoint{3.696000in}{3.696000in}}%
\pgfusepath{clip}%
\pgfsetbuttcap%
\pgfsetroundjoin%
\definecolor{currentfill}{rgb}{0.121569,0.466667,0.705882}%
\pgfsetfillcolor{currentfill}%
\pgfsetfillopacity{0.860669}%
\pgfsetlinewidth{1.003750pt}%
\definecolor{currentstroke}{rgb}{0.121569,0.466667,0.705882}%
\pgfsetstrokecolor{currentstroke}%
\pgfsetstrokeopacity{0.860669}%
\pgfsetdash{}{0pt}%
\pgfpathmoveto{\pgfqpoint{2.320609in}{1.238862in}}%
\pgfpathcurveto{\pgfqpoint{2.328845in}{1.238862in}}{\pgfqpoint{2.336745in}{1.242134in}}{\pgfqpoint{2.342569in}{1.247958in}}%
\pgfpathcurveto{\pgfqpoint{2.348393in}{1.253782in}}{\pgfqpoint{2.351665in}{1.261682in}}{\pgfqpoint{2.351665in}{1.269918in}}%
\pgfpathcurveto{\pgfqpoint{2.351665in}{1.278154in}}{\pgfqpoint{2.348393in}{1.286054in}}{\pgfqpoint{2.342569in}{1.291878in}}%
\pgfpathcurveto{\pgfqpoint{2.336745in}{1.297702in}}{\pgfqpoint{2.328845in}{1.300975in}}{\pgfqpoint{2.320609in}{1.300975in}}%
\pgfpathcurveto{\pgfqpoint{2.312373in}{1.300975in}}{\pgfqpoint{2.304473in}{1.297702in}}{\pgfqpoint{2.298649in}{1.291878in}}%
\pgfpathcurveto{\pgfqpoint{2.292825in}{1.286054in}}{\pgfqpoint{2.289552in}{1.278154in}}{\pgfqpoint{2.289552in}{1.269918in}}%
\pgfpathcurveto{\pgfqpoint{2.289552in}{1.261682in}}{\pgfqpoint{2.292825in}{1.253782in}}{\pgfqpoint{2.298649in}{1.247958in}}%
\pgfpathcurveto{\pgfqpoint{2.304473in}{1.242134in}}{\pgfqpoint{2.312373in}{1.238862in}}{\pgfqpoint{2.320609in}{1.238862in}}%
\pgfpathclose%
\pgfusepath{stroke,fill}%
\end{pgfscope}%
\begin{pgfscope}%
\pgfpathrectangle{\pgfqpoint{0.100000in}{0.212622in}}{\pgfqpoint{3.696000in}{3.696000in}}%
\pgfusepath{clip}%
\pgfsetbuttcap%
\pgfsetroundjoin%
\definecolor{currentfill}{rgb}{0.121569,0.466667,0.705882}%
\pgfsetfillcolor{currentfill}%
\pgfsetfillopacity{0.861125}%
\pgfsetlinewidth{1.003750pt}%
\definecolor{currentstroke}{rgb}{0.121569,0.466667,0.705882}%
\pgfsetstrokecolor{currentstroke}%
\pgfsetstrokeopacity{0.861125}%
\pgfsetdash{}{0pt}%
\pgfpathmoveto{\pgfqpoint{1.775242in}{1.054312in}}%
\pgfpathcurveto{\pgfqpoint{1.783478in}{1.054312in}}{\pgfqpoint{1.791378in}{1.057584in}}{\pgfqpoint{1.797202in}{1.063408in}}%
\pgfpathcurveto{\pgfqpoint{1.803026in}{1.069232in}}{\pgfqpoint{1.806298in}{1.077132in}}{\pgfqpoint{1.806298in}{1.085368in}}%
\pgfpathcurveto{\pgfqpoint{1.806298in}{1.093605in}}{\pgfqpoint{1.803026in}{1.101505in}}{\pgfqpoint{1.797202in}{1.107329in}}%
\pgfpathcurveto{\pgfqpoint{1.791378in}{1.113153in}}{\pgfqpoint{1.783478in}{1.116425in}}{\pgfqpoint{1.775242in}{1.116425in}}%
\pgfpathcurveto{\pgfqpoint{1.767006in}{1.116425in}}{\pgfqpoint{1.759105in}{1.113153in}}{\pgfqpoint{1.753282in}{1.107329in}}%
\pgfpathcurveto{\pgfqpoint{1.747458in}{1.101505in}}{\pgfqpoint{1.744185in}{1.093605in}}{\pgfqpoint{1.744185in}{1.085368in}}%
\pgfpathcurveto{\pgfqpoint{1.744185in}{1.077132in}}{\pgfqpoint{1.747458in}{1.069232in}}{\pgfqpoint{1.753282in}{1.063408in}}%
\pgfpathcurveto{\pgfqpoint{1.759105in}{1.057584in}}{\pgfqpoint{1.767006in}{1.054312in}}{\pgfqpoint{1.775242in}{1.054312in}}%
\pgfpathclose%
\pgfusepath{stroke,fill}%
\end{pgfscope}%
\begin{pgfscope}%
\pgfpathrectangle{\pgfqpoint{0.100000in}{0.212622in}}{\pgfqpoint{3.696000in}{3.696000in}}%
\pgfusepath{clip}%
\pgfsetbuttcap%
\pgfsetroundjoin%
\definecolor{currentfill}{rgb}{0.121569,0.466667,0.705882}%
\pgfsetfillcolor{currentfill}%
\pgfsetfillopacity{0.862970}%
\pgfsetlinewidth{1.003750pt}%
\definecolor{currentstroke}{rgb}{0.121569,0.466667,0.705882}%
\pgfsetstrokecolor{currentstroke}%
\pgfsetstrokeopacity{0.862970}%
\pgfsetdash{}{0pt}%
\pgfpathmoveto{\pgfqpoint{1.784139in}{1.051677in}}%
\pgfpathcurveto{\pgfqpoint{1.792375in}{1.051677in}}{\pgfqpoint{1.800276in}{1.054949in}}{\pgfqpoint{1.806099in}{1.060773in}}%
\pgfpathcurveto{\pgfqpoint{1.811923in}{1.066597in}}{\pgfqpoint{1.815196in}{1.074497in}}{\pgfqpoint{1.815196in}{1.082733in}}%
\pgfpathcurveto{\pgfqpoint{1.815196in}{1.090970in}}{\pgfqpoint{1.811923in}{1.098870in}}{\pgfqpoint{1.806099in}{1.104694in}}%
\pgfpathcurveto{\pgfqpoint{1.800276in}{1.110517in}}{\pgfqpoint{1.792375in}{1.113790in}}{\pgfqpoint{1.784139in}{1.113790in}}%
\pgfpathcurveto{\pgfqpoint{1.775903in}{1.113790in}}{\pgfqpoint{1.768003in}{1.110517in}}{\pgfqpoint{1.762179in}{1.104694in}}%
\pgfpathcurveto{\pgfqpoint{1.756355in}{1.098870in}}{\pgfqpoint{1.753083in}{1.090970in}}{\pgfqpoint{1.753083in}{1.082733in}}%
\pgfpathcurveto{\pgfqpoint{1.753083in}{1.074497in}}{\pgfqpoint{1.756355in}{1.066597in}}{\pgfqpoint{1.762179in}{1.060773in}}%
\pgfpathcurveto{\pgfqpoint{1.768003in}{1.054949in}}{\pgfqpoint{1.775903in}{1.051677in}}{\pgfqpoint{1.784139in}{1.051677in}}%
\pgfpathclose%
\pgfusepath{stroke,fill}%
\end{pgfscope}%
\begin{pgfscope}%
\pgfpathrectangle{\pgfqpoint{0.100000in}{0.212622in}}{\pgfqpoint{3.696000in}{3.696000in}}%
\pgfusepath{clip}%
\pgfsetbuttcap%
\pgfsetroundjoin%
\definecolor{currentfill}{rgb}{0.121569,0.466667,0.705882}%
\pgfsetfillcolor{currentfill}%
\pgfsetfillopacity{0.864814}%
\pgfsetlinewidth{1.003750pt}%
\definecolor{currentstroke}{rgb}{0.121569,0.466667,0.705882}%
\pgfsetstrokecolor{currentstroke}%
\pgfsetstrokeopacity{0.864814}%
\pgfsetdash{}{0pt}%
\pgfpathmoveto{\pgfqpoint{1.792719in}{1.049148in}}%
\pgfpathcurveto{\pgfqpoint{1.800955in}{1.049148in}}{\pgfqpoint{1.808856in}{1.052420in}}{\pgfqpoint{1.814679in}{1.058244in}}%
\pgfpathcurveto{\pgfqpoint{1.820503in}{1.064068in}}{\pgfqpoint{1.823776in}{1.071968in}}{\pgfqpoint{1.823776in}{1.080205in}}%
\pgfpathcurveto{\pgfqpoint{1.823776in}{1.088441in}}{\pgfqpoint{1.820503in}{1.096341in}}{\pgfqpoint{1.814679in}{1.102165in}}%
\pgfpathcurveto{\pgfqpoint{1.808856in}{1.107989in}}{\pgfqpoint{1.800955in}{1.111261in}}{\pgfqpoint{1.792719in}{1.111261in}}%
\pgfpathcurveto{\pgfqpoint{1.784483in}{1.111261in}}{\pgfqpoint{1.776583in}{1.107989in}}{\pgfqpoint{1.770759in}{1.102165in}}%
\pgfpathcurveto{\pgfqpoint{1.764935in}{1.096341in}}{\pgfqpoint{1.761663in}{1.088441in}}{\pgfqpoint{1.761663in}{1.080205in}}%
\pgfpathcurveto{\pgfqpoint{1.761663in}{1.071968in}}{\pgfqpoint{1.764935in}{1.064068in}}{\pgfqpoint{1.770759in}{1.058244in}}%
\pgfpathcurveto{\pgfqpoint{1.776583in}{1.052420in}}{\pgfqpoint{1.784483in}{1.049148in}}{\pgfqpoint{1.792719in}{1.049148in}}%
\pgfpathclose%
\pgfusepath{stroke,fill}%
\end{pgfscope}%
\begin{pgfscope}%
\pgfpathrectangle{\pgfqpoint{0.100000in}{0.212622in}}{\pgfqpoint{3.696000in}{3.696000in}}%
\pgfusepath{clip}%
\pgfsetbuttcap%
\pgfsetroundjoin%
\definecolor{currentfill}{rgb}{0.121569,0.466667,0.705882}%
\pgfsetfillcolor{currentfill}%
\pgfsetfillopacity{0.866382}%
\pgfsetlinewidth{1.003750pt}%
\definecolor{currentstroke}{rgb}{0.121569,0.466667,0.705882}%
\pgfsetstrokecolor{currentstroke}%
\pgfsetstrokeopacity{0.866382}%
\pgfsetdash{}{0pt}%
\pgfpathmoveto{\pgfqpoint{1.799422in}{1.047472in}}%
\pgfpathcurveto{\pgfqpoint{1.807658in}{1.047472in}}{\pgfqpoint{1.815559in}{1.050744in}}{\pgfqpoint{1.821382in}{1.056568in}}%
\pgfpathcurveto{\pgfqpoint{1.827206in}{1.062392in}}{\pgfqpoint{1.830479in}{1.070292in}}{\pgfqpoint{1.830479in}{1.078528in}}%
\pgfpathcurveto{\pgfqpoint{1.830479in}{1.086764in}}{\pgfqpoint{1.827206in}{1.094664in}}{\pgfqpoint{1.821382in}{1.100488in}}%
\pgfpathcurveto{\pgfqpoint{1.815559in}{1.106312in}}{\pgfqpoint{1.807658in}{1.109585in}}{\pgfqpoint{1.799422in}{1.109585in}}%
\pgfpathcurveto{\pgfqpoint{1.791186in}{1.109585in}}{\pgfqpoint{1.783286in}{1.106312in}}{\pgfqpoint{1.777462in}{1.100488in}}%
\pgfpathcurveto{\pgfqpoint{1.771638in}{1.094664in}}{\pgfqpoint{1.768366in}{1.086764in}}{\pgfqpoint{1.768366in}{1.078528in}}%
\pgfpathcurveto{\pgfqpoint{1.768366in}{1.070292in}}{\pgfqpoint{1.771638in}{1.062392in}}{\pgfqpoint{1.777462in}{1.056568in}}%
\pgfpathcurveto{\pgfqpoint{1.783286in}{1.050744in}}{\pgfqpoint{1.791186in}{1.047472in}}{\pgfqpoint{1.799422in}{1.047472in}}%
\pgfpathclose%
\pgfusepath{stroke,fill}%
\end{pgfscope}%
\begin{pgfscope}%
\pgfpathrectangle{\pgfqpoint{0.100000in}{0.212622in}}{\pgfqpoint{3.696000in}{3.696000in}}%
\pgfusepath{clip}%
\pgfsetbuttcap%
\pgfsetroundjoin%
\definecolor{currentfill}{rgb}{0.121569,0.466667,0.705882}%
\pgfsetfillcolor{currentfill}%
\pgfsetfillopacity{0.867845}%
\pgfsetlinewidth{1.003750pt}%
\definecolor{currentstroke}{rgb}{0.121569,0.466667,0.705882}%
\pgfsetstrokecolor{currentstroke}%
\pgfsetstrokeopacity{0.867845}%
\pgfsetdash{}{0pt}%
\pgfpathmoveto{\pgfqpoint{1.805761in}{1.045213in}}%
\pgfpathcurveto{\pgfqpoint{1.813998in}{1.045213in}}{\pgfqpoint{1.821898in}{1.048486in}}{\pgfqpoint{1.827722in}{1.054309in}}%
\pgfpathcurveto{\pgfqpoint{1.833545in}{1.060133in}}{\pgfqpoint{1.836818in}{1.068033in}}{\pgfqpoint{1.836818in}{1.076270in}}%
\pgfpathcurveto{\pgfqpoint{1.836818in}{1.084506in}}{\pgfqpoint{1.833545in}{1.092406in}}{\pgfqpoint{1.827722in}{1.098230in}}%
\pgfpathcurveto{\pgfqpoint{1.821898in}{1.104054in}}{\pgfqpoint{1.813998in}{1.107326in}}{\pgfqpoint{1.805761in}{1.107326in}}%
\pgfpathcurveto{\pgfqpoint{1.797525in}{1.107326in}}{\pgfqpoint{1.789625in}{1.104054in}}{\pgfqpoint{1.783801in}{1.098230in}}%
\pgfpathcurveto{\pgfqpoint{1.777977in}{1.092406in}}{\pgfqpoint{1.774705in}{1.084506in}}{\pgfqpoint{1.774705in}{1.076270in}}%
\pgfpathcurveto{\pgfqpoint{1.774705in}{1.068033in}}{\pgfqpoint{1.777977in}{1.060133in}}{\pgfqpoint{1.783801in}{1.054309in}}%
\pgfpathcurveto{\pgfqpoint{1.789625in}{1.048486in}}{\pgfqpoint{1.797525in}{1.045213in}}{\pgfqpoint{1.805761in}{1.045213in}}%
\pgfpathclose%
\pgfusepath{stroke,fill}%
\end{pgfscope}%
\begin{pgfscope}%
\pgfpathrectangle{\pgfqpoint{0.100000in}{0.212622in}}{\pgfqpoint{3.696000in}{3.696000in}}%
\pgfusepath{clip}%
\pgfsetbuttcap%
\pgfsetroundjoin%
\definecolor{currentfill}{rgb}{0.121569,0.466667,0.705882}%
\pgfsetfillcolor{currentfill}%
\pgfsetfillopacity{0.867984}%
\pgfsetlinewidth{1.003750pt}%
\definecolor{currentstroke}{rgb}{0.121569,0.466667,0.705882}%
\pgfsetstrokecolor{currentstroke}%
\pgfsetstrokeopacity{0.867984}%
\pgfsetdash{}{0pt}%
\pgfpathmoveto{\pgfqpoint{2.329155in}{1.212663in}}%
\pgfpathcurveto{\pgfqpoint{2.337391in}{1.212663in}}{\pgfqpoint{2.345291in}{1.215935in}}{\pgfqpoint{2.351115in}{1.221759in}}%
\pgfpathcurveto{\pgfqpoint{2.356939in}{1.227583in}}{\pgfqpoint{2.360211in}{1.235483in}}{\pgfqpoint{2.360211in}{1.243719in}}%
\pgfpathcurveto{\pgfqpoint{2.360211in}{1.251955in}}{\pgfqpoint{2.356939in}{1.259855in}}{\pgfqpoint{2.351115in}{1.265679in}}%
\pgfpathcurveto{\pgfqpoint{2.345291in}{1.271503in}}{\pgfqpoint{2.337391in}{1.274776in}}{\pgfqpoint{2.329155in}{1.274776in}}%
\pgfpathcurveto{\pgfqpoint{2.320919in}{1.274776in}}{\pgfqpoint{2.313018in}{1.271503in}}{\pgfqpoint{2.307195in}{1.265679in}}%
\pgfpathcurveto{\pgfqpoint{2.301371in}{1.259855in}}{\pgfqpoint{2.298098in}{1.251955in}}{\pgfqpoint{2.298098in}{1.243719in}}%
\pgfpathcurveto{\pgfqpoint{2.298098in}{1.235483in}}{\pgfqpoint{2.301371in}{1.227583in}}{\pgfqpoint{2.307195in}{1.221759in}}%
\pgfpathcurveto{\pgfqpoint{2.313018in}{1.215935in}}{\pgfqpoint{2.320919in}{1.212663in}}{\pgfqpoint{2.329155in}{1.212663in}}%
\pgfpathclose%
\pgfusepath{stroke,fill}%
\end{pgfscope}%
\begin{pgfscope}%
\pgfpathrectangle{\pgfqpoint{0.100000in}{0.212622in}}{\pgfqpoint{3.696000in}{3.696000in}}%
\pgfusepath{clip}%
\pgfsetbuttcap%
\pgfsetroundjoin%
\definecolor{currentfill}{rgb}{0.121569,0.466667,0.705882}%
\pgfsetfillcolor{currentfill}%
\pgfsetfillopacity{0.868941}%
\pgfsetlinewidth{1.003750pt}%
\definecolor{currentstroke}{rgb}{0.121569,0.466667,0.705882}%
\pgfsetstrokecolor{currentstroke}%
\pgfsetstrokeopacity{0.868941}%
\pgfsetdash{}{0pt}%
\pgfpathmoveto{\pgfqpoint{1.811365in}{1.043150in}}%
\pgfpathcurveto{\pgfqpoint{1.819602in}{1.043150in}}{\pgfqpoint{1.827502in}{1.046423in}}{\pgfqpoint{1.833326in}{1.052247in}}%
\pgfpathcurveto{\pgfqpoint{1.839150in}{1.058071in}}{\pgfqpoint{1.842422in}{1.065971in}}{\pgfqpoint{1.842422in}{1.074207in}}%
\pgfpathcurveto{\pgfqpoint{1.842422in}{1.082443in}}{\pgfqpoint{1.839150in}{1.090343in}}{\pgfqpoint{1.833326in}{1.096167in}}%
\pgfpathcurveto{\pgfqpoint{1.827502in}{1.101991in}}{\pgfqpoint{1.819602in}{1.105263in}}{\pgfqpoint{1.811365in}{1.105263in}}%
\pgfpathcurveto{\pgfqpoint{1.803129in}{1.105263in}}{\pgfqpoint{1.795229in}{1.101991in}}{\pgfqpoint{1.789405in}{1.096167in}}%
\pgfpathcurveto{\pgfqpoint{1.783581in}{1.090343in}}{\pgfqpoint{1.780309in}{1.082443in}}{\pgfqpoint{1.780309in}{1.074207in}}%
\pgfpathcurveto{\pgfqpoint{1.780309in}{1.065971in}}{\pgfqpoint{1.783581in}{1.058071in}}{\pgfqpoint{1.789405in}{1.052247in}}%
\pgfpathcurveto{\pgfqpoint{1.795229in}{1.046423in}}{\pgfqpoint{1.803129in}{1.043150in}}{\pgfqpoint{1.811365in}{1.043150in}}%
\pgfpathclose%
\pgfusepath{stroke,fill}%
\end{pgfscope}%
\begin{pgfscope}%
\pgfpathrectangle{\pgfqpoint{0.100000in}{0.212622in}}{\pgfqpoint{3.696000in}{3.696000in}}%
\pgfusepath{clip}%
\pgfsetbuttcap%
\pgfsetroundjoin%
\definecolor{currentfill}{rgb}{0.121569,0.466667,0.705882}%
\pgfsetfillcolor{currentfill}%
\pgfsetfillopacity{0.869810}%
\pgfsetlinewidth{1.003750pt}%
\definecolor{currentstroke}{rgb}{0.121569,0.466667,0.705882}%
\pgfsetstrokecolor{currentstroke}%
\pgfsetstrokeopacity{0.869810}%
\pgfsetdash{}{0pt}%
\pgfpathmoveto{\pgfqpoint{1.815344in}{1.041690in}}%
\pgfpathcurveto{\pgfqpoint{1.823581in}{1.041690in}}{\pgfqpoint{1.831481in}{1.044962in}}{\pgfqpoint{1.837305in}{1.050786in}}%
\pgfpathcurveto{\pgfqpoint{1.843129in}{1.056610in}}{\pgfqpoint{1.846401in}{1.064510in}}{\pgfqpoint{1.846401in}{1.072746in}}%
\pgfpathcurveto{\pgfqpoint{1.846401in}{1.080983in}}{\pgfqpoint{1.843129in}{1.088883in}}{\pgfqpoint{1.837305in}{1.094706in}}%
\pgfpathcurveto{\pgfqpoint{1.831481in}{1.100530in}}{\pgfqpoint{1.823581in}{1.103803in}}{\pgfqpoint{1.815344in}{1.103803in}}%
\pgfpathcurveto{\pgfqpoint{1.807108in}{1.103803in}}{\pgfqpoint{1.799208in}{1.100530in}}{\pgfqpoint{1.793384in}{1.094706in}}%
\pgfpathcurveto{\pgfqpoint{1.787560in}{1.088883in}}{\pgfqpoint{1.784288in}{1.080983in}}{\pgfqpoint{1.784288in}{1.072746in}}%
\pgfpathcurveto{\pgfqpoint{1.784288in}{1.064510in}}{\pgfqpoint{1.787560in}{1.056610in}}{\pgfqpoint{1.793384in}{1.050786in}}%
\pgfpathcurveto{\pgfqpoint{1.799208in}{1.044962in}}{\pgfqpoint{1.807108in}{1.041690in}}{\pgfqpoint{1.815344in}{1.041690in}}%
\pgfpathclose%
\pgfusepath{stroke,fill}%
\end{pgfscope}%
\begin{pgfscope}%
\pgfpathrectangle{\pgfqpoint{0.100000in}{0.212622in}}{\pgfqpoint{3.696000in}{3.696000in}}%
\pgfusepath{clip}%
\pgfsetbuttcap%
\pgfsetroundjoin%
\definecolor{currentfill}{rgb}{0.121569,0.466667,0.705882}%
\pgfsetfillcolor{currentfill}%
\pgfsetfillopacity{0.871188}%
\pgfsetlinewidth{1.003750pt}%
\definecolor{currentstroke}{rgb}{0.121569,0.466667,0.705882}%
\pgfsetstrokecolor{currentstroke}%
\pgfsetstrokeopacity{0.871188}%
\pgfsetdash{}{0pt}%
\pgfpathmoveto{\pgfqpoint{1.822759in}{1.038935in}}%
\pgfpathcurveto{\pgfqpoint{1.830995in}{1.038935in}}{\pgfqpoint{1.838895in}{1.042207in}}{\pgfqpoint{1.844719in}{1.048031in}}%
\pgfpathcurveto{\pgfqpoint{1.850543in}{1.053855in}}{\pgfqpoint{1.853816in}{1.061755in}}{\pgfqpoint{1.853816in}{1.069991in}}%
\pgfpathcurveto{\pgfqpoint{1.853816in}{1.078228in}}{\pgfqpoint{1.850543in}{1.086128in}}{\pgfqpoint{1.844719in}{1.091952in}}%
\pgfpathcurveto{\pgfqpoint{1.838895in}{1.097776in}}{\pgfqpoint{1.830995in}{1.101048in}}{\pgfqpoint{1.822759in}{1.101048in}}%
\pgfpathcurveto{\pgfqpoint{1.814523in}{1.101048in}}{\pgfqpoint{1.806623in}{1.097776in}}{\pgfqpoint{1.800799in}{1.091952in}}%
\pgfpathcurveto{\pgfqpoint{1.794975in}{1.086128in}}{\pgfqpoint{1.791703in}{1.078228in}}{\pgfqpoint{1.791703in}{1.069991in}}%
\pgfpathcurveto{\pgfqpoint{1.791703in}{1.061755in}}{\pgfqpoint{1.794975in}{1.053855in}}{\pgfqpoint{1.800799in}{1.048031in}}%
\pgfpathcurveto{\pgfqpoint{1.806623in}{1.042207in}}{\pgfqpoint{1.814523in}{1.038935in}}{\pgfqpoint{1.822759in}{1.038935in}}%
\pgfpathclose%
\pgfusepath{stroke,fill}%
\end{pgfscope}%
\begin{pgfscope}%
\pgfpathrectangle{\pgfqpoint{0.100000in}{0.212622in}}{\pgfqpoint{3.696000in}{3.696000in}}%
\pgfusepath{clip}%
\pgfsetbuttcap%
\pgfsetroundjoin%
\definecolor{currentfill}{rgb}{0.121569,0.466667,0.705882}%
\pgfsetfillcolor{currentfill}%
\pgfsetfillopacity{0.872226}%
\pgfsetlinewidth{1.003750pt}%
\definecolor{currentstroke}{rgb}{0.121569,0.466667,0.705882}%
\pgfsetstrokecolor{currentstroke}%
\pgfsetstrokeopacity{0.872226}%
\pgfsetdash{}{0pt}%
\pgfpathmoveto{\pgfqpoint{1.828165in}{1.036862in}}%
\pgfpathcurveto{\pgfqpoint{1.836402in}{1.036862in}}{\pgfqpoint{1.844302in}{1.040134in}}{\pgfqpoint{1.850126in}{1.045958in}}%
\pgfpathcurveto{\pgfqpoint{1.855950in}{1.051782in}}{\pgfqpoint{1.859222in}{1.059682in}}{\pgfqpoint{1.859222in}{1.067918in}}%
\pgfpathcurveto{\pgfqpoint{1.859222in}{1.076155in}}{\pgfqpoint{1.855950in}{1.084055in}}{\pgfqpoint{1.850126in}{1.089879in}}%
\pgfpathcurveto{\pgfqpoint{1.844302in}{1.095702in}}{\pgfqpoint{1.836402in}{1.098975in}}{\pgfqpoint{1.828165in}{1.098975in}}%
\pgfpathcurveto{\pgfqpoint{1.819929in}{1.098975in}}{\pgfqpoint{1.812029in}{1.095702in}}{\pgfqpoint{1.806205in}{1.089879in}}%
\pgfpathcurveto{\pgfqpoint{1.800381in}{1.084055in}}{\pgfqpoint{1.797109in}{1.076155in}}{\pgfqpoint{1.797109in}{1.067918in}}%
\pgfpathcurveto{\pgfqpoint{1.797109in}{1.059682in}}{\pgfqpoint{1.800381in}{1.051782in}}{\pgfqpoint{1.806205in}{1.045958in}}%
\pgfpathcurveto{\pgfqpoint{1.812029in}{1.040134in}}{\pgfqpoint{1.819929in}{1.036862in}}{\pgfqpoint{1.828165in}{1.036862in}}%
\pgfpathclose%
\pgfusepath{stroke,fill}%
\end{pgfscope}%
\begin{pgfscope}%
\pgfpathrectangle{\pgfqpoint{0.100000in}{0.212622in}}{\pgfqpoint{3.696000in}{3.696000in}}%
\pgfusepath{clip}%
\pgfsetbuttcap%
\pgfsetroundjoin%
\definecolor{currentfill}{rgb}{0.121569,0.466667,0.705882}%
\pgfsetfillcolor{currentfill}%
\pgfsetfillopacity{0.873039}%
\pgfsetlinewidth{1.003750pt}%
\definecolor{currentstroke}{rgb}{0.121569,0.466667,0.705882}%
\pgfsetstrokecolor{currentstroke}%
\pgfsetstrokeopacity{0.873039}%
\pgfsetdash{}{0pt}%
\pgfpathmoveto{\pgfqpoint{1.832537in}{1.035147in}}%
\pgfpathcurveto{\pgfqpoint{1.840773in}{1.035147in}}{\pgfqpoint{1.848673in}{1.038419in}}{\pgfqpoint{1.854497in}{1.044243in}}%
\pgfpathcurveto{\pgfqpoint{1.860321in}{1.050067in}}{\pgfqpoint{1.863593in}{1.057967in}}{\pgfqpoint{1.863593in}{1.066204in}}%
\pgfpathcurveto{\pgfqpoint{1.863593in}{1.074440in}}{\pgfqpoint{1.860321in}{1.082340in}}{\pgfqpoint{1.854497in}{1.088164in}}%
\pgfpathcurveto{\pgfqpoint{1.848673in}{1.093988in}}{\pgfqpoint{1.840773in}{1.097260in}}{\pgfqpoint{1.832537in}{1.097260in}}%
\pgfpathcurveto{\pgfqpoint{1.824301in}{1.097260in}}{\pgfqpoint{1.816400in}{1.093988in}}{\pgfqpoint{1.810577in}{1.088164in}}%
\pgfpathcurveto{\pgfqpoint{1.804753in}{1.082340in}}{\pgfqpoint{1.801480in}{1.074440in}}{\pgfqpoint{1.801480in}{1.066204in}}%
\pgfpathcurveto{\pgfqpoint{1.801480in}{1.057967in}}{\pgfqpoint{1.804753in}{1.050067in}}{\pgfqpoint{1.810577in}{1.044243in}}%
\pgfpathcurveto{\pgfqpoint{1.816400in}{1.038419in}}{\pgfqpoint{1.824301in}{1.035147in}}{\pgfqpoint{1.832537in}{1.035147in}}%
\pgfpathclose%
\pgfusepath{stroke,fill}%
\end{pgfscope}%
\begin{pgfscope}%
\pgfpathrectangle{\pgfqpoint{0.100000in}{0.212622in}}{\pgfqpoint{3.696000in}{3.696000in}}%
\pgfusepath{clip}%
\pgfsetbuttcap%
\pgfsetroundjoin%
\definecolor{currentfill}{rgb}{0.121569,0.466667,0.705882}%
\pgfsetfillcolor{currentfill}%
\pgfsetfillopacity{0.874615}%
\pgfsetlinewidth{1.003750pt}%
\definecolor{currentstroke}{rgb}{0.121569,0.466667,0.705882}%
\pgfsetstrokecolor{currentstroke}%
\pgfsetstrokeopacity{0.874615}%
\pgfsetdash{}{0pt}%
\pgfpathmoveto{\pgfqpoint{1.840407in}{1.032096in}}%
\pgfpathcurveto{\pgfqpoint{1.848643in}{1.032096in}}{\pgfqpoint{1.856544in}{1.035369in}}{\pgfqpoint{1.862367in}{1.041193in}}%
\pgfpathcurveto{\pgfqpoint{1.868191in}{1.047017in}}{\pgfqpoint{1.871464in}{1.054917in}}{\pgfqpoint{1.871464in}{1.063153in}}%
\pgfpathcurveto{\pgfqpoint{1.871464in}{1.071389in}}{\pgfqpoint{1.868191in}{1.079289in}}{\pgfqpoint{1.862367in}{1.085113in}}%
\pgfpathcurveto{\pgfqpoint{1.856544in}{1.090937in}}{\pgfqpoint{1.848643in}{1.094209in}}{\pgfqpoint{1.840407in}{1.094209in}}%
\pgfpathcurveto{\pgfqpoint{1.832171in}{1.094209in}}{\pgfqpoint{1.824271in}{1.090937in}}{\pgfqpoint{1.818447in}{1.085113in}}%
\pgfpathcurveto{\pgfqpoint{1.812623in}{1.079289in}}{\pgfqpoint{1.809351in}{1.071389in}}{\pgfqpoint{1.809351in}{1.063153in}}%
\pgfpathcurveto{\pgfqpoint{1.809351in}{1.054917in}}{\pgfqpoint{1.812623in}{1.047017in}}{\pgfqpoint{1.818447in}{1.041193in}}%
\pgfpathcurveto{\pgfqpoint{1.824271in}{1.035369in}}{\pgfqpoint{1.832171in}{1.032096in}}{\pgfqpoint{1.840407in}{1.032096in}}%
\pgfpathclose%
\pgfusepath{stroke,fill}%
\end{pgfscope}%
\begin{pgfscope}%
\pgfpathrectangle{\pgfqpoint{0.100000in}{0.212622in}}{\pgfqpoint{3.696000in}{3.696000in}}%
\pgfusepath{clip}%
\pgfsetbuttcap%
\pgfsetroundjoin%
\definecolor{currentfill}{rgb}{0.121569,0.466667,0.705882}%
\pgfsetfillcolor{currentfill}%
\pgfsetfillopacity{0.875900}%
\pgfsetlinewidth{1.003750pt}%
\definecolor{currentstroke}{rgb}{0.121569,0.466667,0.705882}%
\pgfsetstrokecolor{currentstroke}%
\pgfsetstrokeopacity{0.875900}%
\pgfsetdash{}{0pt}%
\pgfpathmoveto{\pgfqpoint{1.846587in}{1.030134in}}%
\pgfpathcurveto{\pgfqpoint{1.854823in}{1.030134in}}{\pgfqpoint{1.862723in}{1.033407in}}{\pgfqpoint{1.868547in}{1.039231in}}%
\pgfpathcurveto{\pgfqpoint{1.874371in}{1.045054in}}{\pgfqpoint{1.877644in}{1.052954in}}{\pgfqpoint{1.877644in}{1.061191in}}%
\pgfpathcurveto{\pgfqpoint{1.877644in}{1.069427in}}{\pgfqpoint{1.874371in}{1.077327in}}{\pgfqpoint{1.868547in}{1.083151in}}%
\pgfpathcurveto{\pgfqpoint{1.862723in}{1.088975in}}{\pgfqpoint{1.854823in}{1.092247in}}{\pgfqpoint{1.846587in}{1.092247in}}%
\pgfpathcurveto{\pgfqpoint{1.838351in}{1.092247in}}{\pgfqpoint{1.830451in}{1.088975in}}{\pgfqpoint{1.824627in}{1.083151in}}%
\pgfpathcurveto{\pgfqpoint{1.818803in}{1.077327in}}{\pgfqpoint{1.815531in}{1.069427in}}{\pgfqpoint{1.815531in}{1.061191in}}%
\pgfpathcurveto{\pgfqpoint{1.815531in}{1.052954in}}{\pgfqpoint{1.818803in}{1.045054in}}{\pgfqpoint{1.824627in}{1.039231in}}%
\pgfpathcurveto{\pgfqpoint{1.830451in}{1.033407in}}{\pgfqpoint{1.838351in}{1.030134in}}{\pgfqpoint{1.846587in}{1.030134in}}%
\pgfpathclose%
\pgfusepath{stroke,fill}%
\end{pgfscope}%
\begin{pgfscope}%
\pgfpathrectangle{\pgfqpoint{0.100000in}{0.212622in}}{\pgfqpoint{3.696000in}{3.696000in}}%
\pgfusepath{clip}%
\pgfsetbuttcap%
\pgfsetroundjoin%
\definecolor{currentfill}{rgb}{0.121569,0.466667,0.705882}%
\pgfsetfillcolor{currentfill}%
\pgfsetfillopacity{0.876442}%
\pgfsetlinewidth{1.003750pt}%
\definecolor{currentstroke}{rgb}{0.121569,0.466667,0.705882}%
\pgfsetstrokecolor{currentstroke}%
\pgfsetstrokeopacity{0.876442}%
\pgfsetdash{}{0pt}%
\pgfpathmoveto{\pgfqpoint{2.336502in}{1.186055in}}%
\pgfpathcurveto{\pgfqpoint{2.344739in}{1.186055in}}{\pgfqpoint{2.352639in}{1.189328in}}{\pgfqpoint{2.358463in}{1.195152in}}%
\pgfpathcurveto{\pgfqpoint{2.364286in}{1.200976in}}{\pgfqpoint{2.367559in}{1.208876in}}{\pgfqpoint{2.367559in}{1.217112in}}%
\pgfpathcurveto{\pgfqpoint{2.367559in}{1.225348in}}{\pgfqpoint{2.364286in}{1.233248in}}{\pgfqpoint{2.358463in}{1.239072in}}%
\pgfpathcurveto{\pgfqpoint{2.352639in}{1.244896in}}{\pgfqpoint{2.344739in}{1.248168in}}{\pgfqpoint{2.336502in}{1.248168in}}%
\pgfpathcurveto{\pgfqpoint{2.328266in}{1.248168in}}{\pgfqpoint{2.320366in}{1.244896in}}{\pgfqpoint{2.314542in}{1.239072in}}%
\pgfpathcurveto{\pgfqpoint{2.308718in}{1.233248in}}{\pgfqpoint{2.305446in}{1.225348in}}{\pgfqpoint{2.305446in}{1.217112in}}%
\pgfpathcurveto{\pgfqpoint{2.305446in}{1.208876in}}{\pgfqpoint{2.308718in}{1.200976in}}{\pgfqpoint{2.314542in}{1.195152in}}%
\pgfpathcurveto{\pgfqpoint{2.320366in}{1.189328in}}{\pgfqpoint{2.328266in}{1.186055in}}{\pgfqpoint{2.336502in}{1.186055in}}%
\pgfpathclose%
\pgfusepath{stroke,fill}%
\end{pgfscope}%
\begin{pgfscope}%
\pgfpathrectangle{\pgfqpoint{0.100000in}{0.212622in}}{\pgfqpoint{3.696000in}{3.696000in}}%
\pgfusepath{clip}%
\pgfsetbuttcap%
\pgfsetroundjoin%
\definecolor{currentfill}{rgb}{0.121569,0.466667,0.705882}%
\pgfsetfillcolor{currentfill}%
\pgfsetfillopacity{0.877014}%
\pgfsetlinewidth{1.003750pt}%
\definecolor{currentstroke}{rgb}{0.121569,0.466667,0.705882}%
\pgfsetstrokecolor{currentstroke}%
\pgfsetstrokeopacity{0.877014}%
\pgfsetdash{}{0pt}%
\pgfpathmoveto{\pgfqpoint{1.851837in}{1.028696in}}%
\pgfpathcurveto{\pgfqpoint{1.860073in}{1.028696in}}{\pgfqpoint{1.867973in}{1.031968in}}{\pgfqpoint{1.873797in}{1.037792in}}%
\pgfpathcurveto{\pgfqpoint{1.879621in}{1.043616in}}{\pgfqpoint{1.882893in}{1.051516in}}{\pgfqpoint{1.882893in}{1.059752in}}%
\pgfpathcurveto{\pgfqpoint{1.882893in}{1.067988in}}{\pgfqpoint{1.879621in}{1.075888in}}{\pgfqpoint{1.873797in}{1.081712in}}%
\pgfpathcurveto{\pgfqpoint{1.867973in}{1.087536in}}{\pgfqpoint{1.860073in}{1.090809in}}{\pgfqpoint{1.851837in}{1.090809in}}%
\pgfpathcurveto{\pgfqpoint{1.843600in}{1.090809in}}{\pgfqpoint{1.835700in}{1.087536in}}{\pgfqpoint{1.829877in}{1.081712in}}%
\pgfpathcurveto{\pgfqpoint{1.824053in}{1.075888in}}{\pgfqpoint{1.820780in}{1.067988in}}{\pgfqpoint{1.820780in}{1.059752in}}%
\pgfpathcurveto{\pgfqpoint{1.820780in}{1.051516in}}{\pgfqpoint{1.824053in}{1.043616in}}{\pgfqpoint{1.829877in}{1.037792in}}%
\pgfpathcurveto{\pgfqpoint{1.835700in}{1.031968in}}{\pgfqpoint{1.843600in}{1.028696in}}{\pgfqpoint{1.851837in}{1.028696in}}%
\pgfpathclose%
\pgfusepath{stroke,fill}%
\end{pgfscope}%
\begin{pgfscope}%
\pgfpathrectangle{\pgfqpoint{0.100000in}{0.212622in}}{\pgfqpoint{3.696000in}{3.696000in}}%
\pgfusepath{clip}%
\pgfsetbuttcap%
\pgfsetroundjoin%
\definecolor{currentfill}{rgb}{0.121569,0.466667,0.705882}%
\pgfsetfillcolor{currentfill}%
\pgfsetfillopacity{0.878814}%
\pgfsetlinewidth{1.003750pt}%
\definecolor{currentstroke}{rgb}{0.121569,0.466667,0.705882}%
\pgfsetstrokecolor{currentstroke}%
\pgfsetstrokeopacity{0.878814}%
\pgfsetdash{}{0pt}%
\pgfpathmoveto{\pgfqpoint{1.861391in}{1.025288in}}%
\pgfpathcurveto{\pgfqpoint{1.869627in}{1.025288in}}{\pgfqpoint{1.877527in}{1.028561in}}{\pgfqpoint{1.883351in}{1.034384in}}%
\pgfpathcurveto{\pgfqpoint{1.889175in}{1.040208in}}{\pgfqpoint{1.892448in}{1.048108in}}{\pgfqpoint{1.892448in}{1.056345in}}%
\pgfpathcurveto{\pgfqpoint{1.892448in}{1.064581in}}{\pgfqpoint{1.889175in}{1.072481in}}{\pgfqpoint{1.883351in}{1.078305in}}%
\pgfpathcurveto{\pgfqpoint{1.877527in}{1.084129in}}{\pgfqpoint{1.869627in}{1.087401in}}{\pgfqpoint{1.861391in}{1.087401in}}%
\pgfpathcurveto{\pgfqpoint{1.853155in}{1.087401in}}{\pgfqpoint{1.845255in}{1.084129in}}{\pgfqpoint{1.839431in}{1.078305in}}%
\pgfpathcurveto{\pgfqpoint{1.833607in}{1.072481in}}{\pgfqpoint{1.830335in}{1.064581in}}{\pgfqpoint{1.830335in}{1.056345in}}%
\pgfpathcurveto{\pgfqpoint{1.830335in}{1.048108in}}{\pgfqpoint{1.833607in}{1.040208in}}{\pgfqpoint{1.839431in}{1.034384in}}%
\pgfpathcurveto{\pgfqpoint{1.845255in}{1.028561in}}{\pgfqpoint{1.853155in}{1.025288in}}{\pgfqpoint{1.861391in}{1.025288in}}%
\pgfpathclose%
\pgfusepath{stroke,fill}%
\end{pgfscope}%
\begin{pgfscope}%
\pgfpathrectangle{\pgfqpoint{0.100000in}{0.212622in}}{\pgfqpoint{3.696000in}{3.696000in}}%
\pgfusepath{clip}%
\pgfsetbuttcap%
\pgfsetroundjoin%
\definecolor{currentfill}{rgb}{0.121569,0.466667,0.705882}%
\pgfsetfillcolor{currentfill}%
\pgfsetfillopacity{0.880249}%
\pgfsetlinewidth{1.003750pt}%
\definecolor{currentstroke}{rgb}{0.121569,0.466667,0.705882}%
\pgfsetstrokecolor{currentstroke}%
\pgfsetstrokeopacity{0.880249}%
\pgfsetdash{}{0pt}%
\pgfpathmoveto{\pgfqpoint{1.868763in}{1.022618in}}%
\pgfpathcurveto{\pgfqpoint{1.876999in}{1.022618in}}{\pgfqpoint{1.884899in}{1.025891in}}{\pgfqpoint{1.890723in}{1.031715in}}%
\pgfpathcurveto{\pgfqpoint{1.896547in}{1.037538in}}{\pgfqpoint{1.899819in}{1.045439in}}{\pgfqpoint{1.899819in}{1.053675in}}%
\pgfpathcurveto{\pgfqpoint{1.899819in}{1.061911in}}{\pgfqpoint{1.896547in}{1.069811in}}{\pgfqpoint{1.890723in}{1.075635in}}%
\pgfpathcurveto{\pgfqpoint{1.884899in}{1.081459in}}{\pgfqpoint{1.876999in}{1.084731in}}{\pgfqpoint{1.868763in}{1.084731in}}%
\pgfpathcurveto{\pgfqpoint{1.860527in}{1.084731in}}{\pgfqpoint{1.852627in}{1.081459in}}{\pgfqpoint{1.846803in}{1.075635in}}%
\pgfpathcurveto{\pgfqpoint{1.840979in}{1.069811in}}{\pgfqpoint{1.837706in}{1.061911in}}{\pgfqpoint{1.837706in}{1.053675in}}%
\pgfpathcurveto{\pgfqpoint{1.837706in}{1.045439in}}{\pgfqpoint{1.840979in}{1.037538in}}{\pgfqpoint{1.846803in}{1.031715in}}%
\pgfpathcurveto{\pgfqpoint{1.852627in}{1.025891in}}{\pgfqpoint{1.860527in}{1.022618in}}{\pgfqpoint{1.868763in}{1.022618in}}%
\pgfpathclose%
\pgfusepath{stroke,fill}%
\end{pgfscope}%
\begin{pgfscope}%
\pgfpathrectangle{\pgfqpoint{0.100000in}{0.212622in}}{\pgfqpoint{3.696000in}{3.696000in}}%
\pgfusepath{clip}%
\pgfsetbuttcap%
\pgfsetroundjoin%
\definecolor{currentfill}{rgb}{0.121569,0.466667,0.705882}%
\pgfsetfillcolor{currentfill}%
\pgfsetfillopacity{0.881337}%
\pgfsetlinewidth{1.003750pt}%
\definecolor{currentstroke}{rgb}{0.121569,0.466667,0.705882}%
\pgfsetstrokecolor{currentstroke}%
\pgfsetstrokeopacity{0.881337}%
\pgfsetdash{}{0pt}%
\pgfpathmoveto{\pgfqpoint{2.340050in}{1.171720in}}%
\pgfpathcurveto{\pgfqpoint{2.348287in}{1.171720in}}{\pgfqpoint{2.356187in}{1.174993in}}{\pgfqpoint{2.362011in}{1.180817in}}%
\pgfpathcurveto{\pgfqpoint{2.367835in}{1.186641in}}{\pgfqpoint{2.371107in}{1.194541in}}{\pgfqpoint{2.371107in}{1.202777in}}%
\pgfpathcurveto{\pgfqpoint{2.371107in}{1.211013in}}{\pgfqpoint{2.367835in}{1.218913in}}{\pgfqpoint{2.362011in}{1.224737in}}%
\pgfpathcurveto{\pgfqpoint{2.356187in}{1.230561in}}{\pgfqpoint{2.348287in}{1.233833in}}{\pgfqpoint{2.340050in}{1.233833in}}%
\pgfpathcurveto{\pgfqpoint{2.331814in}{1.233833in}}{\pgfqpoint{2.323914in}{1.230561in}}{\pgfqpoint{2.318090in}{1.224737in}}%
\pgfpathcurveto{\pgfqpoint{2.312266in}{1.218913in}}{\pgfqpoint{2.308994in}{1.211013in}}{\pgfqpoint{2.308994in}{1.202777in}}%
\pgfpathcurveto{\pgfqpoint{2.308994in}{1.194541in}}{\pgfqpoint{2.312266in}{1.186641in}}{\pgfqpoint{2.318090in}{1.180817in}}%
\pgfpathcurveto{\pgfqpoint{2.323914in}{1.174993in}}{\pgfqpoint{2.331814in}{1.171720in}}{\pgfqpoint{2.340050in}{1.171720in}}%
\pgfpathclose%
\pgfusepath{stroke,fill}%
\end{pgfscope}%
\begin{pgfscope}%
\pgfpathrectangle{\pgfqpoint{0.100000in}{0.212622in}}{\pgfqpoint{3.696000in}{3.696000in}}%
\pgfusepath{clip}%
\pgfsetbuttcap%
\pgfsetroundjoin%
\definecolor{currentfill}{rgb}{0.121569,0.466667,0.705882}%
\pgfsetfillcolor{currentfill}%
\pgfsetfillopacity{0.881593}%
\pgfsetlinewidth{1.003750pt}%
\definecolor{currentstroke}{rgb}{0.121569,0.466667,0.705882}%
\pgfsetstrokecolor{currentstroke}%
\pgfsetstrokeopacity{0.881593}%
\pgfsetdash{}{0pt}%
\pgfpathmoveto{\pgfqpoint{1.875708in}{1.020443in}}%
\pgfpathcurveto{\pgfqpoint{1.883944in}{1.020443in}}{\pgfqpoint{1.891844in}{1.023715in}}{\pgfqpoint{1.897668in}{1.029539in}}%
\pgfpathcurveto{\pgfqpoint{1.903492in}{1.035363in}}{\pgfqpoint{1.906765in}{1.043263in}}{\pgfqpoint{1.906765in}{1.051499in}}%
\pgfpathcurveto{\pgfqpoint{1.906765in}{1.059735in}}{\pgfqpoint{1.903492in}{1.067635in}}{\pgfqpoint{1.897668in}{1.073459in}}%
\pgfpathcurveto{\pgfqpoint{1.891844in}{1.079283in}}{\pgfqpoint{1.883944in}{1.082556in}}{\pgfqpoint{1.875708in}{1.082556in}}%
\pgfpathcurveto{\pgfqpoint{1.867472in}{1.082556in}}{\pgfqpoint{1.859572in}{1.079283in}}{\pgfqpoint{1.853748in}{1.073459in}}%
\pgfpathcurveto{\pgfqpoint{1.847924in}{1.067635in}}{\pgfqpoint{1.844652in}{1.059735in}}{\pgfqpoint{1.844652in}{1.051499in}}%
\pgfpathcurveto{\pgfqpoint{1.844652in}{1.043263in}}{\pgfqpoint{1.847924in}{1.035363in}}{\pgfqpoint{1.853748in}{1.029539in}}%
\pgfpathcurveto{\pgfqpoint{1.859572in}{1.023715in}}{\pgfqpoint{1.867472in}{1.020443in}}{\pgfqpoint{1.875708in}{1.020443in}}%
\pgfpathclose%
\pgfusepath{stroke,fill}%
\end{pgfscope}%
\begin{pgfscope}%
\pgfpathrectangle{\pgfqpoint{0.100000in}{0.212622in}}{\pgfqpoint{3.696000in}{3.696000in}}%
\pgfusepath{clip}%
\pgfsetbuttcap%
\pgfsetroundjoin%
\definecolor{currentfill}{rgb}{0.121569,0.466667,0.705882}%
\pgfsetfillcolor{currentfill}%
\pgfsetfillopacity{0.882784}%
\pgfsetlinewidth{1.003750pt}%
\definecolor{currentstroke}{rgb}{0.121569,0.466667,0.705882}%
\pgfsetstrokecolor{currentstroke}%
\pgfsetstrokeopacity{0.882784}%
\pgfsetdash{}{0pt}%
\pgfpathmoveto{\pgfqpoint{1.881675in}{1.018597in}}%
\pgfpathcurveto{\pgfqpoint{1.889912in}{1.018597in}}{\pgfqpoint{1.897812in}{1.021870in}}{\pgfqpoint{1.903636in}{1.027694in}}%
\pgfpathcurveto{\pgfqpoint{1.909460in}{1.033517in}}{\pgfqpoint{1.912732in}{1.041418in}}{\pgfqpoint{1.912732in}{1.049654in}}%
\pgfpathcurveto{\pgfqpoint{1.912732in}{1.057890in}}{\pgfqpoint{1.909460in}{1.065790in}}{\pgfqpoint{1.903636in}{1.071614in}}%
\pgfpathcurveto{\pgfqpoint{1.897812in}{1.077438in}}{\pgfqpoint{1.889912in}{1.080710in}}{\pgfqpoint{1.881675in}{1.080710in}}%
\pgfpathcurveto{\pgfqpoint{1.873439in}{1.080710in}}{\pgfqpoint{1.865539in}{1.077438in}}{\pgfqpoint{1.859715in}{1.071614in}}%
\pgfpathcurveto{\pgfqpoint{1.853891in}{1.065790in}}{\pgfqpoint{1.850619in}{1.057890in}}{\pgfqpoint{1.850619in}{1.049654in}}%
\pgfpathcurveto{\pgfqpoint{1.850619in}{1.041418in}}{\pgfqpoint{1.853891in}{1.033517in}}{\pgfqpoint{1.859715in}{1.027694in}}%
\pgfpathcurveto{\pgfqpoint{1.865539in}{1.021870in}}{\pgfqpoint{1.873439in}{1.018597in}}{\pgfqpoint{1.881675in}{1.018597in}}%
\pgfpathclose%
\pgfusepath{stroke,fill}%
\end{pgfscope}%
\begin{pgfscope}%
\pgfpathrectangle{\pgfqpoint{0.100000in}{0.212622in}}{\pgfqpoint{3.696000in}{3.696000in}}%
\pgfusepath{clip}%
\pgfsetbuttcap%
\pgfsetroundjoin%
\definecolor{currentfill}{rgb}{0.121569,0.466667,0.705882}%
\pgfsetfillcolor{currentfill}%
\pgfsetfillopacity{0.883702}%
\pgfsetlinewidth{1.003750pt}%
\definecolor{currentstroke}{rgb}{0.121569,0.466667,0.705882}%
\pgfsetstrokecolor{currentstroke}%
\pgfsetstrokeopacity{0.883702}%
\pgfsetdash{}{0pt}%
\pgfpathmoveto{\pgfqpoint{1.886533in}{1.017051in}}%
\pgfpathcurveto{\pgfqpoint{1.894769in}{1.017051in}}{\pgfqpoint{1.902669in}{1.020323in}}{\pgfqpoint{1.908493in}{1.026147in}}%
\pgfpathcurveto{\pgfqpoint{1.914317in}{1.031971in}}{\pgfqpoint{1.917589in}{1.039871in}}{\pgfqpoint{1.917589in}{1.048107in}}%
\pgfpathcurveto{\pgfqpoint{1.917589in}{1.056344in}}{\pgfqpoint{1.914317in}{1.064244in}}{\pgfqpoint{1.908493in}{1.070068in}}%
\pgfpathcurveto{\pgfqpoint{1.902669in}{1.075891in}}{\pgfqpoint{1.894769in}{1.079164in}}{\pgfqpoint{1.886533in}{1.079164in}}%
\pgfpathcurveto{\pgfqpoint{1.878297in}{1.079164in}}{\pgfqpoint{1.870397in}{1.075891in}}{\pgfqpoint{1.864573in}{1.070068in}}%
\pgfpathcurveto{\pgfqpoint{1.858749in}{1.064244in}}{\pgfqpoint{1.855476in}{1.056344in}}{\pgfqpoint{1.855476in}{1.048107in}}%
\pgfpathcurveto{\pgfqpoint{1.855476in}{1.039871in}}{\pgfqpoint{1.858749in}{1.031971in}}{\pgfqpoint{1.864573in}{1.026147in}}%
\pgfpathcurveto{\pgfqpoint{1.870397in}{1.020323in}}{\pgfqpoint{1.878297in}{1.017051in}}{\pgfqpoint{1.886533in}{1.017051in}}%
\pgfpathclose%
\pgfusepath{stroke,fill}%
\end{pgfscope}%
\begin{pgfscope}%
\pgfpathrectangle{\pgfqpoint{0.100000in}{0.212622in}}{\pgfqpoint{3.696000in}{3.696000in}}%
\pgfusepath{clip}%
\pgfsetbuttcap%
\pgfsetroundjoin%
\definecolor{currentfill}{rgb}{0.121569,0.466667,0.705882}%
\pgfsetfillcolor{currentfill}%
\pgfsetfillopacity{0.885444}%
\pgfsetlinewidth{1.003750pt}%
\definecolor{currentstroke}{rgb}{0.121569,0.466667,0.705882}%
\pgfsetstrokecolor{currentstroke}%
\pgfsetstrokeopacity{0.885444}%
\pgfsetdash{}{0pt}%
\pgfpathmoveto{\pgfqpoint{1.895371in}{1.014516in}}%
\pgfpathcurveto{\pgfqpoint{1.903608in}{1.014516in}}{\pgfqpoint{1.911508in}{1.017789in}}{\pgfqpoint{1.917332in}{1.023612in}}%
\pgfpathcurveto{\pgfqpoint{1.923155in}{1.029436in}}{\pgfqpoint{1.926428in}{1.037336in}}{\pgfqpoint{1.926428in}{1.045573in}}%
\pgfpathcurveto{\pgfqpoint{1.926428in}{1.053809in}}{\pgfqpoint{1.923155in}{1.061709in}}{\pgfqpoint{1.917332in}{1.067533in}}%
\pgfpathcurveto{\pgfqpoint{1.911508in}{1.073357in}}{\pgfqpoint{1.903608in}{1.076629in}}{\pgfqpoint{1.895371in}{1.076629in}}%
\pgfpathcurveto{\pgfqpoint{1.887135in}{1.076629in}}{\pgfqpoint{1.879235in}{1.073357in}}{\pgfqpoint{1.873411in}{1.067533in}}%
\pgfpathcurveto{\pgfqpoint{1.867587in}{1.061709in}}{\pgfqpoint{1.864315in}{1.053809in}}{\pgfqpoint{1.864315in}{1.045573in}}%
\pgfpathcurveto{\pgfqpoint{1.864315in}{1.037336in}}{\pgfqpoint{1.867587in}{1.029436in}}{\pgfqpoint{1.873411in}{1.023612in}}%
\pgfpathcurveto{\pgfqpoint{1.879235in}{1.017789in}}{\pgfqpoint{1.887135in}{1.014516in}}{\pgfqpoint{1.895371in}{1.014516in}}%
\pgfpathclose%
\pgfusepath{stroke,fill}%
\end{pgfscope}%
\begin{pgfscope}%
\pgfpathrectangle{\pgfqpoint{0.100000in}{0.212622in}}{\pgfqpoint{3.696000in}{3.696000in}}%
\pgfusepath{clip}%
\pgfsetbuttcap%
\pgfsetroundjoin%
\definecolor{currentfill}{rgb}{0.121569,0.466667,0.705882}%
\pgfsetfillcolor{currentfill}%
\pgfsetfillopacity{0.886154}%
\pgfsetlinewidth{1.003750pt}%
\definecolor{currentstroke}{rgb}{0.121569,0.466667,0.705882}%
\pgfsetstrokecolor{currentstroke}%
\pgfsetstrokeopacity{0.886154}%
\pgfsetdash{}{0pt}%
\pgfpathmoveto{\pgfqpoint{2.345049in}{1.156764in}}%
\pgfpathcurveto{\pgfqpoint{2.353285in}{1.156764in}}{\pgfqpoint{2.361185in}{1.160036in}}{\pgfqpoint{2.367009in}{1.165860in}}%
\pgfpathcurveto{\pgfqpoint{2.372833in}{1.171684in}}{\pgfqpoint{2.376105in}{1.179584in}}{\pgfqpoint{2.376105in}{1.187820in}}%
\pgfpathcurveto{\pgfqpoint{2.376105in}{1.196056in}}{\pgfqpoint{2.372833in}{1.203956in}}{\pgfqpoint{2.367009in}{1.209780in}}%
\pgfpathcurveto{\pgfqpoint{2.361185in}{1.215604in}}{\pgfqpoint{2.353285in}{1.218877in}}{\pgfqpoint{2.345049in}{1.218877in}}%
\pgfpathcurveto{\pgfqpoint{2.336812in}{1.218877in}}{\pgfqpoint{2.328912in}{1.215604in}}{\pgfqpoint{2.323088in}{1.209780in}}%
\pgfpathcurveto{\pgfqpoint{2.317264in}{1.203956in}}{\pgfqpoint{2.313992in}{1.196056in}}{\pgfqpoint{2.313992in}{1.187820in}}%
\pgfpathcurveto{\pgfqpoint{2.313992in}{1.179584in}}{\pgfqpoint{2.317264in}{1.171684in}}{\pgfqpoint{2.323088in}{1.165860in}}%
\pgfpathcurveto{\pgfqpoint{2.328912in}{1.160036in}}{\pgfqpoint{2.336812in}{1.156764in}}{\pgfqpoint{2.345049in}{1.156764in}}%
\pgfpathclose%
\pgfusepath{stroke,fill}%
\end{pgfscope}%
\begin{pgfscope}%
\pgfpathrectangle{\pgfqpoint{0.100000in}{0.212622in}}{\pgfqpoint{3.696000in}{3.696000in}}%
\pgfusepath{clip}%
\pgfsetbuttcap%
\pgfsetroundjoin%
\definecolor{currentfill}{rgb}{0.121569,0.466667,0.705882}%
\pgfsetfillcolor{currentfill}%
\pgfsetfillopacity{0.886829}%
\pgfsetlinewidth{1.003750pt}%
\definecolor{currentstroke}{rgb}{0.121569,0.466667,0.705882}%
\pgfsetstrokecolor{currentstroke}%
\pgfsetstrokeopacity{0.886829}%
\pgfsetdash{}{0pt}%
\pgfpathmoveto{\pgfqpoint{1.902050in}{1.012387in}}%
\pgfpathcurveto{\pgfqpoint{1.910286in}{1.012387in}}{\pgfqpoint{1.918186in}{1.015659in}}{\pgfqpoint{1.924010in}{1.021483in}}%
\pgfpathcurveto{\pgfqpoint{1.929834in}{1.027307in}}{\pgfqpoint{1.933106in}{1.035207in}}{\pgfqpoint{1.933106in}{1.043443in}}%
\pgfpathcurveto{\pgfqpoint{1.933106in}{1.051680in}}{\pgfqpoint{1.929834in}{1.059580in}}{\pgfqpoint{1.924010in}{1.065404in}}%
\pgfpathcurveto{\pgfqpoint{1.918186in}{1.071227in}}{\pgfqpoint{1.910286in}{1.074500in}}{\pgfqpoint{1.902050in}{1.074500in}}%
\pgfpathcurveto{\pgfqpoint{1.893813in}{1.074500in}}{\pgfqpoint{1.885913in}{1.071227in}}{\pgfqpoint{1.880089in}{1.065404in}}%
\pgfpathcurveto{\pgfqpoint{1.874265in}{1.059580in}}{\pgfqpoint{1.870993in}{1.051680in}}{\pgfqpoint{1.870993in}{1.043443in}}%
\pgfpathcurveto{\pgfqpoint{1.870993in}{1.035207in}}{\pgfqpoint{1.874265in}{1.027307in}}{\pgfqpoint{1.880089in}{1.021483in}}%
\pgfpathcurveto{\pgfqpoint{1.885913in}{1.015659in}}{\pgfqpoint{1.893813in}{1.012387in}}{\pgfqpoint{1.902050in}{1.012387in}}%
\pgfpathclose%
\pgfusepath{stroke,fill}%
\end{pgfscope}%
\begin{pgfscope}%
\pgfpathrectangle{\pgfqpoint{0.100000in}{0.212622in}}{\pgfqpoint{3.696000in}{3.696000in}}%
\pgfusepath{clip}%
\pgfsetbuttcap%
\pgfsetroundjoin%
\definecolor{currentfill}{rgb}{0.121569,0.466667,0.705882}%
\pgfsetfillcolor{currentfill}%
\pgfsetfillopacity{0.888062}%
\pgfsetlinewidth{1.003750pt}%
\definecolor{currentstroke}{rgb}{0.121569,0.466667,0.705882}%
\pgfsetstrokecolor{currentstroke}%
\pgfsetstrokeopacity{0.888062}%
\pgfsetdash{}{0pt}%
\pgfpathmoveto{\pgfqpoint{1.908211in}{1.010324in}}%
\pgfpathcurveto{\pgfqpoint{1.916447in}{1.010324in}}{\pgfqpoint{1.924347in}{1.013596in}}{\pgfqpoint{1.930171in}{1.019420in}}%
\pgfpathcurveto{\pgfqpoint{1.935995in}{1.025244in}}{\pgfqpoint{1.939267in}{1.033144in}}{\pgfqpoint{1.939267in}{1.041380in}}%
\pgfpathcurveto{\pgfqpoint{1.939267in}{1.049617in}}{\pgfqpoint{1.935995in}{1.057517in}}{\pgfqpoint{1.930171in}{1.063341in}}%
\pgfpathcurveto{\pgfqpoint{1.924347in}{1.069164in}}{\pgfqpoint{1.916447in}{1.072437in}}{\pgfqpoint{1.908211in}{1.072437in}}%
\pgfpathcurveto{\pgfqpoint{1.899975in}{1.072437in}}{\pgfqpoint{1.892074in}{1.069164in}}{\pgfqpoint{1.886251in}{1.063341in}}%
\pgfpathcurveto{\pgfqpoint{1.880427in}{1.057517in}}{\pgfqpoint{1.877154in}{1.049617in}}{\pgfqpoint{1.877154in}{1.041380in}}%
\pgfpathcurveto{\pgfqpoint{1.877154in}{1.033144in}}{\pgfqpoint{1.880427in}{1.025244in}}{\pgfqpoint{1.886251in}{1.019420in}}%
\pgfpathcurveto{\pgfqpoint{1.892074in}{1.013596in}}{\pgfqpoint{1.899975in}{1.010324in}}{\pgfqpoint{1.908211in}{1.010324in}}%
\pgfpathclose%
\pgfusepath{stroke,fill}%
\end{pgfscope}%
\begin{pgfscope}%
\pgfpathrectangle{\pgfqpoint{0.100000in}{0.212622in}}{\pgfqpoint{3.696000in}{3.696000in}}%
\pgfusepath{clip}%
\pgfsetbuttcap%
\pgfsetroundjoin%
\definecolor{currentfill}{rgb}{0.121569,0.466667,0.705882}%
\pgfsetfillcolor{currentfill}%
\pgfsetfillopacity{0.888995}%
\pgfsetlinewidth{1.003750pt}%
\definecolor{currentstroke}{rgb}{0.121569,0.466667,0.705882}%
\pgfsetstrokecolor{currentstroke}%
\pgfsetstrokeopacity{0.888995}%
\pgfsetdash{}{0pt}%
\pgfpathmoveto{\pgfqpoint{1.913073in}{1.008581in}}%
\pgfpathcurveto{\pgfqpoint{1.921309in}{1.008581in}}{\pgfqpoint{1.929209in}{1.011854in}}{\pgfqpoint{1.935033in}{1.017678in}}%
\pgfpathcurveto{\pgfqpoint{1.940857in}{1.023502in}}{\pgfqpoint{1.944130in}{1.031402in}}{\pgfqpoint{1.944130in}{1.039638in}}%
\pgfpathcurveto{\pgfqpoint{1.944130in}{1.047874in}}{\pgfqpoint{1.940857in}{1.055774in}}{\pgfqpoint{1.935033in}{1.061598in}}%
\pgfpathcurveto{\pgfqpoint{1.929209in}{1.067422in}}{\pgfqpoint{1.921309in}{1.070694in}}{\pgfqpoint{1.913073in}{1.070694in}}%
\pgfpathcurveto{\pgfqpoint{1.904837in}{1.070694in}}{\pgfqpoint{1.896937in}{1.067422in}}{\pgfqpoint{1.891113in}{1.061598in}}%
\pgfpathcurveto{\pgfqpoint{1.885289in}{1.055774in}}{\pgfqpoint{1.882017in}{1.047874in}}{\pgfqpoint{1.882017in}{1.039638in}}%
\pgfpathcurveto{\pgfqpoint{1.882017in}{1.031402in}}{\pgfqpoint{1.885289in}{1.023502in}}{\pgfqpoint{1.891113in}{1.017678in}}%
\pgfpathcurveto{\pgfqpoint{1.896937in}{1.011854in}}{\pgfqpoint{1.904837in}{1.008581in}}{\pgfqpoint{1.913073in}{1.008581in}}%
\pgfpathclose%
\pgfusepath{stroke,fill}%
\end{pgfscope}%
\begin{pgfscope}%
\pgfpathrectangle{\pgfqpoint{0.100000in}{0.212622in}}{\pgfqpoint{3.696000in}{3.696000in}}%
\pgfusepath{clip}%
\pgfsetbuttcap%
\pgfsetroundjoin%
\definecolor{currentfill}{rgb}{0.121569,0.466667,0.705882}%
\pgfsetfillcolor{currentfill}%
\pgfsetfillopacity{0.889698}%
\pgfsetlinewidth{1.003750pt}%
\definecolor{currentstroke}{rgb}{0.121569,0.466667,0.705882}%
\pgfsetstrokecolor{currentstroke}%
\pgfsetstrokeopacity{0.889698}%
\pgfsetdash{}{0pt}%
\pgfpathmoveto{\pgfqpoint{1.916537in}{1.007472in}}%
\pgfpathcurveto{\pgfqpoint{1.924774in}{1.007472in}}{\pgfqpoint{1.932674in}{1.010745in}}{\pgfqpoint{1.938498in}{1.016569in}}%
\pgfpathcurveto{\pgfqpoint{1.944321in}{1.022393in}}{\pgfqpoint{1.947594in}{1.030293in}}{\pgfqpoint{1.947594in}{1.038529in}}%
\pgfpathcurveto{\pgfqpoint{1.947594in}{1.046765in}}{\pgfqpoint{1.944321in}{1.054665in}}{\pgfqpoint{1.938498in}{1.060489in}}%
\pgfpathcurveto{\pgfqpoint{1.932674in}{1.066313in}}{\pgfqpoint{1.924774in}{1.069585in}}{\pgfqpoint{1.916537in}{1.069585in}}%
\pgfpathcurveto{\pgfqpoint{1.908301in}{1.069585in}}{\pgfqpoint{1.900401in}{1.066313in}}{\pgfqpoint{1.894577in}{1.060489in}}%
\pgfpathcurveto{\pgfqpoint{1.888753in}{1.054665in}}{\pgfqpoint{1.885481in}{1.046765in}}{\pgfqpoint{1.885481in}{1.038529in}}%
\pgfpathcurveto{\pgfqpoint{1.885481in}{1.030293in}}{\pgfqpoint{1.888753in}{1.022393in}}{\pgfqpoint{1.894577in}{1.016569in}}%
\pgfpathcurveto{\pgfqpoint{1.900401in}{1.010745in}}{\pgfqpoint{1.908301in}{1.007472in}}{\pgfqpoint{1.916537in}{1.007472in}}%
\pgfpathclose%
\pgfusepath{stroke,fill}%
\end{pgfscope}%
\begin{pgfscope}%
\pgfpathrectangle{\pgfqpoint{0.100000in}{0.212622in}}{\pgfqpoint{3.696000in}{3.696000in}}%
\pgfusepath{clip}%
\pgfsetbuttcap%
\pgfsetroundjoin%
\definecolor{currentfill}{rgb}{0.121569,0.466667,0.705882}%
\pgfsetfillcolor{currentfill}%
\pgfsetfillopacity{0.890730}%
\pgfsetlinewidth{1.003750pt}%
\definecolor{currentstroke}{rgb}{0.121569,0.466667,0.705882}%
\pgfsetstrokecolor{currentstroke}%
\pgfsetstrokeopacity{0.890730}%
\pgfsetdash{}{0pt}%
\pgfpathmoveto{\pgfqpoint{2.350131in}{1.139959in}}%
\pgfpathcurveto{\pgfqpoint{2.358367in}{1.139959in}}{\pgfqpoint{2.366267in}{1.143231in}}{\pgfqpoint{2.372091in}{1.149055in}}%
\pgfpathcurveto{\pgfqpoint{2.377915in}{1.154879in}}{\pgfqpoint{2.381187in}{1.162779in}}{\pgfqpoint{2.381187in}{1.171015in}}%
\pgfpathcurveto{\pgfqpoint{2.381187in}{1.179252in}}{\pgfqpoint{2.377915in}{1.187152in}}{\pgfqpoint{2.372091in}{1.192976in}}%
\pgfpathcurveto{\pgfqpoint{2.366267in}{1.198799in}}{\pgfqpoint{2.358367in}{1.202072in}}{\pgfqpoint{2.350131in}{1.202072in}}%
\pgfpathcurveto{\pgfqpoint{2.341895in}{1.202072in}}{\pgfqpoint{2.333995in}{1.198799in}}{\pgfqpoint{2.328171in}{1.192976in}}%
\pgfpathcurveto{\pgfqpoint{2.322347in}{1.187152in}}{\pgfqpoint{2.319074in}{1.179252in}}{\pgfqpoint{2.319074in}{1.171015in}}%
\pgfpathcurveto{\pgfqpoint{2.319074in}{1.162779in}}{\pgfqpoint{2.322347in}{1.154879in}}{\pgfqpoint{2.328171in}{1.149055in}}%
\pgfpathcurveto{\pgfqpoint{2.333995in}{1.143231in}}{\pgfqpoint{2.341895in}{1.139959in}}{\pgfqpoint{2.350131in}{1.139959in}}%
\pgfpathclose%
\pgfusepath{stroke,fill}%
\end{pgfscope}%
\begin{pgfscope}%
\pgfpathrectangle{\pgfqpoint{0.100000in}{0.212622in}}{\pgfqpoint{3.696000in}{3.696000in}}%
\pgfusepath{clip}%
\pgfsetbuttcap%
\pgfsetroundjoin%
\definecolor{currentfill}{rgb}{0.121569,0.466667,0.705882}%
\pgfsetfillcolor{currentfill}%
\pgfsetfillopacity{0.890957}%
\pgfsetlinewidth{1.003750pt}%
\definecolor{currentstroke}{rgb}{0.121569,0.466667,0.705882}%
\pgfsetstrokecolor{currentstroke}%
\pgfsetstrokeopacity{0.890957}%
\pgfsetdash{}{0pt}%
\pgfpathmoveto{\pgfqpoint{1.922871in}{1.005512in}}%
\pgfpathcurveto{\pgfqpoint{1.931108in}{1.005512in}}{\pgfqpoint{1.939008in}{1.008785in}}{\pgfqpoint{1.944832in}{1.014608in}}%
\pgfpathcurveto{\pgfqpoint{1.950655in}{1.020432in}}{\pgfqpoint{1.953928in}{1.028332in}}{\pgfqpoint{1.953928in}{1.036569in}}%
\pgfpathcurveto{\pgfqpoint{1.953928in}{1.044805in}}{\pgfqpoint{1.950655in}{1.052705in}}{\pgfqpoint{1.944832in}{1.058529in}}%
\pgfpathcurveto{\pgfqpoint{1.939008in}{1.064353in}}{\pgfqpoint{1.931108in}{1.067625in}}{\pgfqpoint{1.922871in}{1.067625in}}%
\pgfpathcurveto{\pgfqpoint{1.914635in}{1.067625in}}{\pgfqpoint{1.906735in}{1.064353in}}{\pgfqpoint{1.900911in}{1.058529in}}%
\pgfpathcurveto{\pgfqpoint{1.895087in}{1.052705in}}{\pgfqpoint{1.891815in}{1.044805in}}{\pgfqpoint{1.891815in}{1.036569in}}%
\pgfpathcurveto{\pgfqpoint{1.891815in}{1.028332in}}{\pgfqpoint{1.895087in}{1.020432in}}{\pgfqpoint{1.900911in}{1.014608in}}%
\pgfpathcurveto{\pgfqpoint{1.906735in}{1.008785in}}{\pgfqpoint{1.914635in}{1.005512in}}{\pgfqpoint{1.922871in}{1.005512in}}%
\pgfpathclose%
\pgfusepath{stroke,fill}%
\end{pgfscope}%
\begin{pgfscope}%
\pgfpathrectangle{\pgfqpoint{0.100000in}{0.212622in}}{\pgfqpoint{3.696000in}{3.696000in}}%
\pgfusepath{clip}%
\pgfsetbuttcap%
\pgfsetroundjoin%
\definecolor{currentfill}{rgb}{0.121569,0.466667,0.705882}%
\pgfsetfillcolor{currentfill}%
\pgfsetfillopacity{0.891902}%
\pgfsetlinewidth{1.003750pt}%
\definecolor{currentstroke}{rgb}{0.121569,0.466667,0.705882}%
\pgfsetstrokecolor{currentstroke}%
\pgfsetstrokeopacity{0.891902}%
\pgfsetdash{}{0pt}%
\pgfpathmoveto{\pgfqpoint{1.927908in}{1.004057in}}%
\pgfpathcurveto{\pgfqpoint{1.936144in}{1.004057in}}{\pgfqpoint{1.944044in}{1.007329in}}{\pgfqpoint{1.949868in}{1.013153in}}%
\pgfpathcurveto{\pgfqpoint{1.955692in}{1.018977in}}{\pgfqpoint{1.958964in}{1.026877in}}{\pgfqpoint{1.958964in}{1.035113in}}%
\pgfpathcurveto{\pgfqpoint{1.958964in}{1.043350in}}{\pgfqpoint{1.955692in}{1.051250in}}{\pgfqpoint{1.949868in}{1.057074in}}%
\pgfpathcurveto{\pgfqpoint{1.944044in}{1.062898in}}{\pgfqpoint{1.936144in}{1.066170in}}{\pgfqpoint{1.927908in}{1.066170in}}%
\pgfpathcurveto{\pgfqpoint{1.919672in}{1.066170in}}{\pgfqpoint{1.911772in}{1.062898in}}{\pgfqpoint{1.905948in}{1.057074in}}%
\pgfpathcurveto{\pgfqpoint{1.900124in}{1.051250in}}{\pgfqpoint{1.896851in}{1.043350in}}{\pgfqpoint{1.896851in}{1.035113in}}%
\pgfpathcurveto{\pgfqpoint{1.896851in}{1.026877in}}{\pgfqpoint{1.900124in}{1.018977in}}{\pgfqpoint{1.905948in}{1.013153in}}%
\pgfpathcurveto{\pgfqpoint{1.911772in}{1.007329in}}{\pgfqpoint{1.919672in}{1.004057in}}{\pgfqpoint{1.927908in}{1.004057in}}%
\pgfpathclose%
\pgfusepath{stroke,fill}%
\end{pgfscope}%
\begin{pgfscope}%
\pgfpathrectangle{\pgfqpoint{0.100000in}{0.212622in}}{\pgfqpoint{3.696000in}{3.696000in}}%
\pgfusepath{clip}%
\pgfsetbuttcap%
\pgfsetroundjoin%
\definecolor{currentfill}{rgb}{0.121569,0.466667,0.705882}%
\pgfsetfillcolor{currentfill}%
\pgfsetfillopacity{0.892727}%
\pgfsetlinewidth{1.003750pt}%
\definecolor{currentstroke}{rgb}{0.121569,0.466667,0.705882}%
\pgfsetstrokecolor{currentstroke}%
\pgfsetstrokeopacity{0.892727}%
\pgfsetdash{}{0pt}%
\pgfpathmoveto{\pgfqpoint{1.931876in}{1.003060in}}%
\pgfpathcurveto{\pgfqpoint{1.940112in}{1.003060in}}{\pgfqpoint{1.948012in}{1.006333in}}{\pgfqpoint{1.953836in}{1.012156in}}%
\pgfpathcurveto{\pgfqpoint{1.959660in}{1.017980in}}{\pgfqpoint{1.962932in}{1.025880in}}{\pgfqpoint{1.962932in}{1.034117in}}%
\pgfpathcurveto{\pgfqpoint{1.962932in}{1.042353in}}{\pgfqpoint{1.959660in}{1.050253in}}{\pgfqpoint{1.953836in}{1.056077in}}%
\pgfpathcurveto{\pgfqpoint{1.948012in}{1.061901in}}{\pgfqpoint{1.940112in}{1.065173in}}{\pgfqpoint{1.931876in}{1.065173in}}%
\pgfpathcurveto{\pgfqpoint{1.923639in}{1.065173in}}{\pgfqpoint{1.915739in}{1.061901in}}{\pgfqpoint{1.909915in}{1.056077in}}%
\pgfpathcurveto{\pgfqpoint{1.904092in}{1.050253in}}{\pgfqpoint{1.900819in}{1.042353in}}{\pgfqpoint{1.900819in}{1.034117in}}%
\pgfpathcurveto{\pgfqpoint{1.900819in}{1.025880in}}{\pgfqpoint{1.904092in}{1.017980in}}{\pgfqpoint{1.909915in}{1.012156in}}%
\pgfpathcurveto{\pgfqpoint{1.915739in}{1.006333in}}{\pgfqpoint{1.923639in}{1.003060in}}{\pgfqpoint{1.931876in}{1.003060in}}%
\pgfpathclose%
\pgfusepath{stroke,fill}%
\end{pgfscope}%
\begin{pgfscope}%
\pgfpathrectangle{\pgfqpoint{0.100000in}{0.212622in}}{\pgfqpoint{3.696000in}{3.696000in}}%
\pgfusepath{clip}%
\pgfsetbuttcap%
\pgfsetroundjoin%
\definecolor{currentfill}{rgb}{0.121569,0.466667,0.705882}%
\pgfsetfillcolor{currentfill}%
\pgfsetfillopacity{0.893545}%
\pgfsetlinewidth{1.003750pt}%
\definecolor{currentstroke}{rgb}{0.121569,0.466667,0.705882}%
\pgfsetstrokecolor{currentstroke}%
\pgfsetstrokeopacity{0.893545}%
\pgfsetdash{}{0pt}%
\pgfpathmoveto{\pgfqpoint{2.352433in}{1.131206in}}%
\pgfpathcurveto{\pgfqpoint{2.360669in}{1.131206in}}{\pgfqpoint{2.368569in}{1.134479in}}{\pgfqpoint{2.374393in}{1.140303in}}%
\pgfpathcurveto{\pgfqpoint{2.380217in}{1.146127in}}{\pgfqpoint{2.383489in}{1.154027in}}{\pgfqpoint{2.383489in}{1.162263in}}%
\pgfpathcurveto{\pgfqpoint{2.383489in}{1.170499in}}{\pgfqpoint{2.380217in}{1.178399in}}{\pgfqpoint{2.374393in}{1.184223in}}%
\pgfpathcurveto{\pgfqpoint{2.368569in}{1.190047in}}{\pgfqpoint{2.360669in}{1.193319in}}{\pgfqpoint{2.352433in}{1.193319in}}%
\pgfpathcurveto{\pgfqpoint{2.344196in}{1.193319in}}{\pgfqpoint{2.336296in}{1.190047in}}{\pgfqpoint{2.330472in}{1.184223in}}%
\pgfpathcurveto{\pgfqpoint{2.324648in}{1.178399in}}{\pgfqpoint{2.321376in}{1.170499in}}{\pgfqpoint{2.321376in}{1.162263in}}%
\pgfpathcurveto{\pgfqpoint{2.321376in}{1.154027in}}{\pgfqpoint{2.324648in}{1.146127in}}{\pgfqpoint{2.330472in}{1.140303in}}%
\pgfpathcurveto{\pgfqpoint{2.336296in}{1.134479in}}{\pgfqpoint{2.344196in}{1.131206in}}{\pgfqpoint{2.352433in}{1.131206in}}%
\pgfpathclose%
\pgfusepath{stroke,fill}%
\end{pgfscope}%
\begin{pgfscope}%
\pgfpathrectangle{\pgfqpoint{0.100000in}{0.212622in}}{\pgfqpoint{3.696000in}{3.696000in}}%
\pgfusepath{clip}%
\pgfsetbuttcap%
\pgfsetroundjoin%
\definecolor{currentfill}{rgb}{0.121569,0.466667,0.705882}%
\pgfsetfillcolor{currentfill}%
\pgfsetfillopacity{0.894184}%
\pgfsetlinewidth{1.003750pt}%
\definecolor{currentstroke}{rgb}{0.121569,0.466667,0.705882}%
\pgfsetstrokecolor{currentstroke}%
\pgfsetstrokeopacity{0.894184}%
\pgfsetdash{}{0pt}%
\pgfpathmoveto{\pgfqpoint{1.939057in}{1.000965in}}%
\pgfpathcurveto{\pgfqpoint{1.947293in}{1.000965in}}{\pgfqpoint{1.955193in}{1.004237in}}{\pgfqpoint{1.961017in}{1.010061in}}%
\pgfpathcurveto{\pgfqpoint{1.966841in}{1.015885in}}{\pgfqpoint{1.970114in}{1.023785in}}{\pgfqpoint{1.970114in}{1.032021in}}%
\pgfpathcurveto{\pgfqpoint{1.970114in}{1.040258in}}{\pgfqpoint{1.966841in}{1.048158in}}{\pgfqpoint{1.961017in}{1.053982in}}%
\pgfpathcurveto{\pgfqpoint{1.955193in}{1.059806in}}{\pgfqpoint{1.947293in}{1.063078in}}{\pgfqpoint{1.939057in}{1.063078in}}%
\pgfpathcurveto{\pgfqpoint{1.930821in}{1.063078in}}{\pgfqpoint{1.922921in}{1.059806in}}{\pgfqpoint{1.917097in}{1.053982in}}%
\pgfpathcurveto{\pgfqpoint{1.911273in}{1.048158in}}{\pgfqpoint{1.908001in}{1.040258in}}{\pgfqpoint{1.908001in}{1.032021in}}%
\pgfpathcurveto{\pgfqpoint{1.908001in}{1.023785in}}{\pgfqpoint{1.911273in}{1.015885in}}{\pgfqpoint{1.917097in}{1.010061in}}%
\pgfpathcurveto{\pgfqpoint{1.922921in}{1.004237in}}{\pgfqpoint{1.930821in}{1.000965in}}{\pgfqpoint{1.939057in}{1.000965in}}%
\pgfpathclose%
\pgfusepath{stroke,fill}%
\end{pgfscope}%
\begin{pgfscope}%
\pgfpathrectangle{\pgfqpoint{0.100000in}{0.212622in}}{\pgfqpoint{3.696000in}{3.696000in}}%
\pgfusepath{clip}%
\pgfsetbuttcap%
\pgfsetroundjoin%
\definecolor{currentfill}{rgb}{0.121569,0.466667,0.705882}%
\pgfsetfillcolor{currentfill}%
\pgfsetfillopacity{0.895296}%
\pgfsetlinewidth{1.003750pt}%
\definecolor{currentstroke}{rgb}{0.121569,0.466667,0.705882}%
\pgfsetstrokecolor{currentstroke}%
\pgfsetstrokeopacity{0.895296}%
\pgfsetdash{}{0pt}%
\pgfpathmoveto{\pgfqpoint{1.944951in}{0.999371in}}%
\pgfpathcurveto{\pgfqpoint{1.953187in}{0.999371in}}{\pgfqpoint{1.961087in}{1.002644in}}{\pgfqpoint{1.966911in}{1.008468in}}%
\pgfpathcurveto{\pgfqpoint{1.972735in}{1.014291in}}{\pgfqpoint{1.976008in}{1.022192in}}{\pgfqpoint{1.976008in}{1.030428in}}%
\pgfpathcurveto{\pgfqpoint{1.976008in}{1.038664in}}{\pgfqpoint{1.972735in}{1.046564in}}{\pgfqpoint{1.966911in}{1.052388in}}%
\pgfpathcurveto{\pgfqpoint{1.961087in}{1.058212in}}{\pgfqpoint{1.953187in}{1.061484in}}{\pgfqpoint{1.944951in}{1.061484in}}%
\pgfpathcurveto{\pgfqpoint{1.936715in}{1.061484in}}{\pgfqpoint{1.928815in}{1.058212in}}{\pgfqpoint{1.922991in}{1.052388in}}%
\pgfpathcurveto{\pgfqpoint{1.917167in}{1.046564in}}{\pgfqpoint{1.913895in}{1.038664in}}{\pgfqpoint{1.913895in}{1.030428in}}%
\pgfpathcurveto{\pgfqpoint{1.913895in}{1.022192in}}{\pgfqpoint{1.917167in}{1.014291in}}{\pgfqpoint{1.922991in}{1.008468in}}%
\pgfpathcurveto{\pgfqpoint{1.928815in}{1.002644in}}{\pgfqpoint{1.936715in}{0.999371in}}{\pgfqpoint{1.944951in}{0.999371in}}%
\pgfpathclose%
\pgfusepath{stroke,fill}%
\end{pgfscope}%
\begin{pgfscope}%
\pgfpathrectangle{\pgfqpoint{0.100000in}{0.212622in}}{\pgfqpoint{3.696000in}{3.696000in}}%
\pgfusepath{clip}%
\pgfsetbuttcap%
\pgfsetroundjoin%
\definecolor{currentfill}{rgb}{0.121569,0.466667,0.705882}%
\pgfsetfillcolor{currentfill}%
\pgfsetfillopacity{0.896386}%
\pgfsetlinewidth{1.003750pt}%
\definecolor{currentstroke}{rgb}{0.121569,0.466667,0.705882}%
\pgfsetstrokecolor{currentstroke}%
\pgfsetstrokeopacity{0.896386}%
\pgfsetdash{}{0pt}%
\pgfpathmoveto{\pgfqpoint{2.355054in}{1.121723in}}%
\pgfpathcurveto{\pgfqpoint{2.363291in}{1.121723in}}{\pgfqpoint{2.371191in}{1.124995in}}{\pgfqpoint{2.377015in}{1.130819in}}%
\pgfpathcurveto{\pgfqpoint{2.382838in}{1.136643in}}{\pgfqpoint{2.386111in}{1.144543in}}{\pgfqpoint{2.386111in}{1.152780in}}%
\pgfpathcurveto{\pgfqpoint{2.386111in}{1.161016in}}{\pgfqpoint{2.382838in}{1.168916in}}{\pgfqpoint{2.377015in}{1.174740in}}%
\pgfpathcurveto{\pgfqpoint{2.371191in}{1.180564in}}{\pgfqpoint{2.363291in}{1.183836in}}{\pgfqpoint{2.355054in}{1.183836in}}%
\pgfpathcurveto{\pgfqpoint{2.346818in}{1.183836in}}{\pgfqpoint{2.338918in}{1.180564in}}{\pgfqpoint{2.333094in}{1.174740in}}%
\pgfpathcurveto{\pgfqpoint{2.327270in}{1.168916in}}{\pgfqpoint{2.323998in}{1.161016in}}{\pgfqpoint{2.323998in}{1.152780in}}%
\pgfpathcurveto{\pgfqpoint{2.323998in}{1.144543in}}{\pgfqpoint{2.327270in}{1.136643in}}{\pgfqpoint{2.333094in}{1.130819in}}%
\pgfpathcurveto{\pgfqpoint{2.338918in}{1.124995in}}{\pgfqpoint{2.346818in}{1.121723in}}{\pgfqpoint{2.355054in}{1.121723in}}%
\pgfpathclose%
\pgfusepath{stroke,fill}%
\end{pgfscope}%
\begin{pgfscope}%
\pgfpathrectangle{\pgfqpoint{0.100000in}{0.212622in}}{\pgfqpoint{3.696000in}{3.696000in}}%
\pgfusepath{clip}%
\pgfsetbuttcap%
\pgfsetroundjoin%
\definecolor{currentfill}{rgb}{0.121569,0.466667,0.705882}%
\pgfsetfillcolor{currentfill}%
\pgfsetfillopacity{0.896502}%
\pgfsetlinewidth{1.003750pt}%
\definecolor{currentstroke}{rgb}{0.121569,0.466667,0.705882}%
\pgfsetstrokecolor{currentstroke}%
\pgfsetstrokeopacity{0.896502}%
\pgfsetdash{}{0pt}%
\pgfpathmoveto{\pgfqpoint{1.950356in}{0.998224in}}%
\pgfpathcurveto{\pgfqpoint{1.958592in}{0.998224in}}{\pgfqpoint{1.966492in}{1.001497in}}{\pgfqpoint{1.972316in}{1.007321in}}%
\pgfpathcurveto{\pgfqpoint{1.978140in}{1.013144in}}{\pgfqpoint{1.981413in}{1.021045in}}{\pgfqpoint{1.981413in}{1.029281in}}%
\pgfpathcurveto{\pgfqpoint{1.981413in}{1.037517in}}{\pgfqpoint{1.978140in}{1.045417in}}{\pgfqpoint{1.972316in}{1.051241in}}%
\pgfpathcurveto{\pgfqpoint{1.966492in}{1.057065in}}{\pgfqpoint{1.958592in}{1.060337in}}{\pgfqpoint{1.950356in}{1.060337in}}%
\pgfpathcurveto{\pgfqpoint{1.942120in}{1.060337in}}{\pgfqpoint{1.934220in}{1.057065in}}{\pgfqpoint{1.928396in}{1.051241in}}%
\pgfpathcurveto{\pgfqpoint{1.922572in}{1.045417in}}{\pgfqpoint{1.919300in}{1.037517in}}{\pgfqpoint{1.919300in}{1.029281in}}%
\pgfpathcurveto{\pgfqpoint{1.919300in}{1.021045in}}{\pgfqpoint{1.922572in}{1.013144in}}{\pgfqpoint{1.928396in}{1.007321in}}%
\pgfpathcurveto{\pgfqpoint{1.934220in}{1.001497in}}{\pgfqpoint{1.942120in}{0.998224in}}{\pgfqpoint{1.950356in}{0.998224in}}%
\pgfpathclose%
\pgfusepath{stroke,fill}%
\end{pgfscope}%
\begin{pgfscope}%
\pgfpathrectangle{\pgfqpoint{0.100000in}{0.212622in}}{\pgfqpoint{3.696000in}{3.696000in}}%
\pgfusepath{clip}%
\pgfsetbuttcap%
\pgfsetroundjoin%
\definecolor{currentfill}{rgb}{0.121569,0.466667,0.705882}%
\pgfsetfillcolor{currentfill}%
\pgfsetfillopacity{0.897533}%
\pgfsetlinewidth{1.003750pt}%
\definecolor{currentstroke}{rgb}{0.121569,0.466667,0.705882}%
\pgfsetstrokecolor{currentstroke}%
\pgfsetstrokeopacity{0.897533}%
\pgfsetdash{}{0pt}%
\pgfpathmoveto{\pgfqpoint{1.955556in}{0.996934in}}%
\pgfpathcurveto{\pgfqpoint{1.963793in}{0.996934in}}{\pgfqpoint{1.971693in}{1.000206in}}{\pgfqpoint{1.977517in}{1.006030in}}%
\pgfpathcurveto{\pgfqpoint{1.983341in}{1.011854in}}{\pgfqpoint{1.986613in}{1.019754in}}{\pgfqpoint{1.986613in}{1.027991in}}%
\pgfpathcurveto{\pgfqpoint{1.986613in}{1.036227in}}{\pgfqpoint{1.983341in}{1.044127in}}{\pgfqpoint{1.977517in}{1.049951in}}%
\pgfpathcurveto{\pgfqpoint{1.971693in}{1.055775in}}{\pgfqpoint{1.963793in}{1.059047in}}{\pgfqpoint{1.955556in}{1.059047in}}%
\pgfpathcurveto{\pgfqpoint{1.947320in}{1.059047in}}{\pgfqpoint{1.939420in}{1.055775in}}{\pgfqpoint{1.933596in}{1.049951in}}%
\pgfpathcurveto{\pgfqpoint{1.927772in}{1.044127in}}{\pgfqpoint{1.924500in}{1.036227in}}{\pgfqpoint{1.924500in}{1.027991in}}%
\pgfpathcurveto{\pgfqpoint{1.924500in}{1.019754in}}{\pgfqpoint{1.927772in}{1.011854in}}{\pgfqpoint{1.933596in}{1.006030in}}%
\pgfpathcurveto{\pgfqpoint{1.939420in}{1.000206in}}{\pgfqpoint{1.947320in}{0.996934in}}{\pgfqpoint{1.955556in}{0.996934in}}%
\pgfpathclose%
\pgfusepath{stroke,fill}%
\end{pgfscope}%
\begin{pgfscope}%
\pgfpathrectangle{\pgfqpoint{0.100000in}{0.212622in}}{\pgfqpoint{3.696000in}{3.696000in}}%
\pgfusepath{clip}%
\pgfsetbuttcap%
\pgfsetroundjoin%
\definecolor{currentfill}{rgb}{0.121569,0.466667,0.705882}%
\pgfsetfillcolor{currentfill}%
\pgfsetfillopacity{0.898342}%
\pgfsetlinewidth{1.003750pt}%
\definecolor{currentstroke}{rgb}{0.121569,0.466667,0.705882}%
\pgfsetstrokecolor{currentstroke}%
\pgfsetstrokeopacity{0.898342}%
\pgfsetdash{}{0pt}%
\pgfpathmoveto{\pgfqpoint{1.959767in}{0.995841in}}%
\pgfpathcurveto{\pgfqpoint{1.968003in}{0.995841in}}{\pgfqpoint{1.975903in}{0.999114in}}{\pgfqpoint{1.981727in}{1.004938in}}%
\pgfpathcurveto{\pgfqpoint{1.987551in}{1.010762in}}{\pgfqpoint{1.990823in}{1.018662in}}{\pgfqpoint{1.990823in}{1.026898in}}%
\pgfpathcurveto{\pgfqpoint{1.990823in}{1.035134in}}{\pgfqpoint{1.987551in}{1.043034in}}{\pgfqpoint{1.981727in}{1.048858in}}%
\pgfpathcurveto{\pgfqpoint{1.975903in}{1.054682in}}{\pgfqpoint{1.968003in}{1.057954in}}{\pgfqpoint{1.959767in}{1.057954in}}%
\pgfpathcurveto{\pgfqpoint{1.951530in}{1.057954in}}{\pgfqpoint{1.943630in}{1.054682in}}{\pgfqpoint{1.937806in}{1.048858in}}%
\pgfpathcurveto{\pgfqpoint{1.931983in}{1.043034in}}{\pgfqpoint{1.928710in}{1.035134in}}{\pgfqpoint{1.928710in}{1.026898in}}%
\pgfpathcurveto{\pgfqpoint{1.928710in}{1.018662in}}{\pgfqpoint{1.931983in}{1.010762in}}{\pgfqpoint{1.937806in}{1.004938in}}%
\pgfpathcurveto{\pgfqpoint{1.943630in}{0.999114in}}{\pgfqpoint{1.951530in}{0.995841in}}{\pgfqpoint{1.959767in}{0.995841in}}%
\pgfpathclose%
\pgfusepath{stroke,fill}%
\end{pgfscope}%
\begin{pgfscope}%
\pgfpathrectangle{\pgfqpoint{0.100000in}{0.212622in}}{\pgfqpoint{3.696000in}{3.696000in}}%
\pgfusepath{clip}%
\pgfsetbuttcap%
\pgfsetroundjoin%
\definecolor{currentfill}{rgb}{0.121569,0.466667,0.705882}%
\pgfsetfillcolor{currentfill}%
\pgfsetfillopacity{0.899168}%
\pgfsetlinewidth{1.003750pt}%
\definecolor{currentstroke}{rgb}{0.121569,0.466667,0.705882}%
\pgfsetstrokecolor{currentstroke}%
\pgfsetstrokeopacity{0.899168}%
\pgfsetdash{}{0pt}%
\pgfpathmoveto{\pgfqpoint{1.963454in}{0.994960in}}%
\pgfpathcurveto{\pgfqpoint{1.971691in}{0.994960in}}{\pgfqpoint{1.979591in}{0.998232in}}{\pgfqpoint{1.985415in}{1.004056in}}%
\pgfpathcurveto{\pgfqpoint{1.991239in}{1.009880in}}{\pgfqpoint{1.994511in}{1.017780in}}{\pgfqpoint{1.994511in}{1.026016in}}%
\pgfpathcurveto{\pgfqpoint{1.994511in}{1.034253in}}{\pgfqpoint{1.991239in}{1.042153in}}{\pgfqpoint{1.985415in}{1.047977in}}%
\pgfpathcurveto{\pgfqpoint{1.979591in}{1.053801in}}{\pgfqpoint{1.971691in}{1.057073in}}{\pgfqpoint{1.963454in}{1.057073in}}%
\pgfpathcurveto{\pgfqpoint{1.955218in}{1.057073in}}{\pgfqpoint{1.947318in}{1.053801in}}{\pgfqpoint{1.941494in}{1.047977in}}%
\pgfpathcurveto{\pgfqpoint{1.935670in}{1.042153in}}{\pgfqpoint{1.932398in}{1.034253in}}{\pgfqpoint{1.932398in}{1.026016in}}%
\pgfpathcurveto{\pgfqpoint{1.932398in}{1.017780in}}{\pgfqpoint{1.935670in}{1.009880in}}{\pgfqpoint{1.941494in}{1.004056in}}%
\pgfpathcurveto{\pgfqpoint{1.947318in}{0.998232in}}{\pgfqpoint{1.955218in}{0.994960in}}{\pgfqpoint{1.963454in}{0.994960in}}%
\pgfpathclose%
\pgfusepath{stroke,fill}%
\end{pgfscope}%
\begin{pgfscope}%
\pgfpathrectangle{\pgfqpoint{0.100000in}{0.212622in}}{\pgfqpoint{3.696000in}{3.696000in}}%
\pgfusepath{clip}%
\pgfsetbuttcap%
\pgfsetroundjoin%
\definecolor{currentfill}{rgb}{0.121569,0.466667,0.705882}%
\pgfsetfillcolor{currentfill}%
\pgfsetfillopacity{0.899212}%
\pgfsetlinewidth{1.003750pt}%
\definecolor{currentstroke}{rgb}{0.121569,0.466667,0.705882}%
\pgfsetstrokecolor{currentstroke}%
\pgfsetstrokeopacity{0.899212}%
\pgfsetdash{}{0pt}%
\pgfpathmoveto{\pgfqpoint{2.358268in}{1.111558in}}%
\pgfpathcurveto{\pgfqpoint{2.366505in}{1.111558in}}{\pgfqpoint{2.374405in}{1.114830in}}{\pgfqpoint{2.380229in}{1.120654in}}%
\pgfpathcurveto{\pgfqpoint{2.386053in}{1.126478in}}{\pgfqpoint{2.389325in}{1.134378in}}{\pgfqpoint{2.389325in}{1.142615in}}%
\pgfpathcurveto{\pgfqpoint{2.389325in}{1.150851in}}{\pgfqpoint{2.386053in}{1.158751in}}{\pgfqpoint{2.380229in}{1.164575in}}%
\pgfpathcurveto{\pgfqpoint{2.374405in}{1.170399in}}{\pgfqpoint{2.366505in}{1.173671in}}{\pgfqpoint{2.358268in}{1.173671in}}%
\pgfpathcurveto{\pgfqpoint{2.350032in}{1.173671in}}{\pgfqpoint{2.342132in}{1.170399in}}{\pgfqpoint{2.336308in}{1.164575in}}%
\pgfpathcurveto{\pgfqpoint{2.330484in}{1.158751in}}{\pgfqpoint{2.327212in}{1.150851in}}{\pgfqpoint{2.327212in}{1.142615in}}%
\pgfpathcurveto{\pgfqpoint{2.327212in}{1.134378in}}{\pgfqpoint{2.330484in}{1.126478in}}{\pgfqpoint{2.336308in}{1.120654in}}%
\pgfpathcurveto{\pgfqpoint{2.342132in}{1.114830in}}{\pgfqpoint{2.350032in}{1.111558in}}{\pgfqpoint{2.358268in}{1.111558in}}%
\pgfpathclose%
\pgfusepath{stroke,fill}%
\end{pgfscope}%
\begin{pgfscope}%
\pgfpathrectangle{\pgfqpoint{0.100000in}{0.212622in}}{\pgfqpoint{3.696000in}{3.696000in}}%
\pgfusepath{clip}%
\pgfsetbuttcap%
\pgfsetroundjoin%
\definecolor{currentfill}{rgb}{0.121569,0.466667,0.705882}%
\pgfsetfillcolor{currentfill}%
\pgfsetfillopacity{0.900537}%
\pgfsetlinewidth{1.003750pt}%
\definecolor{currentstroke}{rgb}{0.121569,0.466667,0.705882}%
\pgfsetstrokecolor{currentstroke}%
\pgfsetstrokeopacity{0.900537}%
\pgfsetdash{}{0pt}%
\pgfpathmoveto{\pgfqpoint{1.970237in}{0.993170in}}%
\pgfpathcurveto{\pgfqpoint{1.978473in}{0.993170in}}{\pgfqpoint{1.986373in}{0.996442in}}{\pgfqpoint{1.992197in}{1.002266in}}%
\pgfpathcurveto{\pgfqpoint{1.998021in}{1.008090in}}{\pgfqpoint{2.001294in}{1.015990in}}{\pgfqpoint{2.001294in}{1.024226in}}%
\pgfpathcurveto{\pgfqpoint{2.001294in}{1.032463in}}{\pgfqpoint{1.998021in}{1.040363in}}{\pgfqpoint{1.992197in}{1.046187in}}%
\pgfpathcurveto{\pgfqpoint{1.986373in}{1.052010in}}{\pgfqpoint{1.978473in}{1.055283in}}{\pgfqpoint{1.970237in}{1.055283in}}%
\pgfpathcurveto{\pgfqpoint{1.962001in}{1.055283in}}{\pgfqpoint{1.954101in}{1.052010in}}{\pgfqpoint{1.948277in}{1.046187in}}%
\pgfpathcurveto{\pgfqpoint{1.942453in}{1.040363in}}{\pgfqpoint{1.939181in}{1.032463in}}{\pgfqpoint{1.939181in}{1.024226in}}%
\pgfpathcurveto{\pgfqpoint{1.939181in}{1.015990in}}{\pgfqpoint{1.942453in}{1.008090in}}{\pgfqpoint{1.948277in}{1.002266in}}%
\pgfpathcurveto{\pgfqpoint{1.954101in}{0.996442in}}{\pgfqpoint{1.962001in}{0.993170in}}{\pgfqpoint{1.970237in}{0.993170in}}%
\pgfpathclose%
\pgfusepath{stroke,fill}%
\end{pgfscope}%
\begin{pgfscope}%
\pgfpathrectangle{\pgfqpoint{0.100000in}{0.212622in}}{\pgfqpoint{3.696000in}{3.696000in}}%
\pgfusepath{clip}%
\pgfsetbuttcap%
\pgfsetroundjoin%
\definecolor{currentfill}{rgb}{0.121569,0.466667,0.705882}%
\pgfsetfillcolor{currentfill}%
\pgfsetfillopacity{0.901542}%
\pgfsetlinewidth{1.003750pt}%
\definecolor{currentstroke}{rgb}{0.121569,0.466667,0.705882}%
\pgfsetstrokecolor{currentstroke}%
\pgfsetstrokeopacity{0.901542}%
\pgfsetdash{}{0pt}%
\pgfpathmoveto{\pgfqpoint{1.975658in}{0.991611in}}%
\pgfpathcurveto{\pgfqpoint{1.983895in}{0.991611in}}{\pgfqpoint{1.991795in}{0.994883in}}{\pgfqpoint{1.997619in}{1.000707in}}%
\pgfpathcurveto{\pgfqpoint{2.003442in}{1.006531in}}{\pgfqpoint{2.006715in}{1.014431in}}{\pgfqpoint{2.006715in}{1.022668in}}%
\pgfpathcurveto{\pgfqpoint{2.006715in}{1.030904in}}{\pgfqpoint{2.003442in}{1.038804in}}{\pgfqpoint{1.997619in}{1.044628in}}%
\pgfpathcurveto{\pgfqpoint{1.991795in}{1.050452in}}{\pgfqpoint{1.983895in}{1.053724in}}{\pgfqpoint{1.975658in}{1.053724in}}%
\pgfpathcurveto{\pgfqpoint{1.967422in}{1.053724in}}{\pgfqpoint{1.959522in}{1.050452in}}{\pgfqpoint{1.953698in}{1.044628in}}%
\pgfpathcurveto{\pgfqpoint{1.947874in}{1.038804in}}{\pgfqpoint{1.944602in}{1.030904in}}{\pgfqpoint{1.944602in}{1.022668in}}%
\pgfpathcurveto{\pgfqpoint{1.944602in}{1.014431in}}{\pgfqpoint{1.947874in}{1.006531in}}{\pgfqpoint{1.953698in}{1.000707in}}%
\pgfpathcurveto{\pgfqpoint{1.959522in}{0.994883in}}{\pgfqpoint{1.967422in}{0.991611in}}{\pgfqpoint{1.975658in}{0.991611in}}%
\pgfpathclose%
\pgfusepath{stroke,fill}%
\end{pgfscope}%
\begin{pgfscope}%
\pgfpathrectangle{\pgfqpoint{0.100000in}{0.212622in}}{\pgfqpoint{3.696000in}{3.696000in}}%
\pgfusepath{clip}%
\pgfsetbuttcap%
\pgfsetroundjoin%
\definecolor{currentfill}{rgb}{0.121569,0.466667,0.705882}%
\pgfsetfillcolor{currentfill}%
\pgfsetfillopacity{0.902620}%
\pgfsetlinewidth{1.003750pt}%
\definecolor{currentstroke}{rgb}{0.121569,0.466667,0.705882}%
\pgfsetstrokecolor{currentstroke}%
\pgfsetstrokeopacity{0.902620}%
\pgfsetdash{}{0pt}%
\pgfpathmoveto{\pgfqpoint{1.980609in}{0.990604in}}%
\pgfpathcurveto{\pgfqpoint{1.988845in}{0.990604in}}{\pgfqpoint{1.996745in}{0.993877in}}{\pgfqpoint{2.002569in}{0.999701in}}%
\pgfpathcurveto{\pgfqpoint{2.008393in}{1.005524in}}{\pgfqpoint{2.011665in}{1.013425in}}{\pgfqpoint{2.011665in}{1.021661in}}%
\pgfpathcurveto{\pgfqpoint{2.011665in}{1.029897in}}{\pgfqpoint{2.008393in}{1.037797in}}{\pgfqpoint{2.002569in}{1.043621in}}%
\pgfpathcurveto{\pgfqpoint{1.996745in}{1.049445in}}{\pgfqpoint{1.988845in}{1.052717in}}{\pgfqpoint{1.980609in}{1.052717in}}%
\pgfpathcurveto{\pgfqpoint{1.972373in}{1.052717in}}{\pgfqpoint{1.964473in}{1.049445in}}{\pgfqpoint{1.958649in}{1.043621in}}%
\pgfpathcurveto{\pgfqpoint{1.952825in}{1.037797in}}{\pgfqpoint{1.949552in}{1.029897in}}{\pgfqpoint{1.949552in}{1.021661in}}%
\pgfpathcurveto{\pgfqpoint{1.949552in}{1.013425in}}{\pgfqpoint{1.952825in}{1.005524in}}{\pgfqpoint{1.958649in}{0.999701in}}%
\pgfpathcurveto{\pgfqpoint{1.964473in}{0.993877in}}{\pgfqpoint{1.972373in}{0.990604in}}{\pgfqpoint{1.980609in}{0.990604in}}%
\pgfpathclose%
\pgfusepath{stroke,fill}%
\end{pgfscope}%
\begin{pgfscope}%
\pgfpathrectangle{\pgfqpoint{0.100000in}{0.212622in}}{\pgfqpoint{3.696000in}{3.696000in}}%
\pgfusepath{clip}%
\pgfsetbuttcap%
\pgfsetroundjoin%
\definecolor{currentfill}{rgb}{0.121569,0.466667,0.705882}%
\pgfsetfillcolor{currentfill}%
\pgfsetfillopacity{0.902674}%
\pgfsetlinewidth{1.003750pt}%
\definecolor{currentstroke}{rgb}{0.121569,0.466667,0.705882}%
\pgfsetstrokecolor{currentstroke}%
\pgfsetstrokeopacity{0.902674}%
\pgfsetdash{}{0pt}%
\pgfpathmoveto{\pgfqpoint{2.361246in}{1.100252in}}%
\pgfpathcurveto{\pgfqpoint{2.369482in}{1.100252in}}{\pgfqpoint{2.377382in}{1.103524in}}{\pgfqpoint{2.383206in}{1.109348in}}%
\pgfpathcurveto{\pgfqpoint{2.389030in}{1.115172in}}{\pgfqpoint{2.392302in}{1.123072in}}{\pgfqpoint{2.392302in}{1.131309in}}%
\pgfpathcurveto{\pgfqpoint{2.392302in}{1.139545in}}{\pgfqpoint{2.389030in}{1.147445in}}{\pgfqpoint{2.383206in}{1.153269in}}%
\pgfpathcurveto{\pgfqpoint{2.377382in}{1.159093in}}{\pgfqpoint{2.369482in}{1.162365in}}{\pgfqpoint{2.361246in}{1.162365in}}%
\pgfpathcurveto{\pgfqpoint{2.353010in}{1.162365in}}{\pgfqpoint{2.345110in}{1.159093in}}{\pgfqpoint{2.339286in}{1.153269in}}%
\pgfpathcurveto{\pgfqpoint{2.333462in}{1.147445in}}{\pgfqpoint{2.330189in}{1.139545in}}{\pgfqpoint{2.330189in}{1.131309in}}%
\pgfpathcurveto{\pgfqpoint{2.330189in}{1.123072in}}{\pgfqpoint{2.333462in}{1.115172in}}{\pgfqpoint{2.339286in}{1.109348in}}%
\pgfpathcurveto{\pgfqpoint{2.345110in}{1.103524in}}{\pgfqpoint{2.353010in}{1.100252in}}{\pgfqpoint{2.361246in}{1.100252in}}%
\pgfpathclose%
\pgfusepath{stroke,fill}%
\end{pgfscope}%
\begin{pgfscope}%
\pgfpathrectangle{\pgfqpoint{0.100000in}{0.212622in}}{\pgfqpoint{3.696000in}{3.696000in}}%
\pgfusepath{clip}%
\pgfsetbuttcap%
\pgfsetroundjoin%
\definecolor{currentfill}{rgb}{0.121569,0.466667,0.705882}%
\pgfsetfillcolor{currentfill}%
\pgfsetfillopacity{0.903640}%
\pgfsetlinewidth{1.003750pt}%
\definecolor{currentstroke}{rgb}{0.121569,0.466667,0.705882}%
\pgfsetstrokecolor{currentstroke}%
\pgfsetstrokeopacity{0.903640}%
\pgfsetdash{}{0pt}%
\pgfpathmoveto{\pgfqpoint{1.985102in}{0.989646in}}%
\pgfpathcurveto{\pgfqpoint{1.993339in}{0.989646in}}{\pgfqpoint{2.001239in}{0.992918in}}{\pgfqpoint{2.007063in}{0.998742in}}%
\pgfpathcurveto{\pgfqpoint{2.012886in}{1.004566in}}{\pgfqpoint{2.016159in}{1.012466in}}{\pgfqpoint{2.016159in}{1.020703in}}%
\pgfpathcurveto{\pgfqpoint{2.016159in}{1.028939in}}{\pgfqpoint{2.012886in}{1.036839in}}{\pgfqpoint{2.007063in}{1.042663in}}%
\pgfpathcurveto{\pgfqpoint{2.001239in}{1.048487in}}{\pgfqpoint{1.993339in}{1.051759in}}{\pgfqpoint{1.985102in}{1.051759in}}%
\pgfpathcurveto{\pgfqpoint{1.976866in}{1.051759in}}{\pgfqpoint{1.968966in}{1.048487in}}{\pgfqpoint{1.963142in}{1.042663in}}%
\pgfpathcurveto{\pgfqpoint{1.957318in}{1.036839in}}{\pgfqpoint{1.954046in}{1.028939in}}{\pgfqpoint{1.954046in}{1.020703in}}%
\pgfpathcurveto{\pgfqpoint{1.954046in}{1.012466in}}{\pgfqpoint{1.957318in}{1.004566in}}{\pgfqpoint{1.963142in}{0.998742in}}%
\pgfpathcurveto{\pgfqpoint{1.968966in}{0.992918in}}{\pgfqpoint{1.976866in}{0.989646in}}{\pgfqpoint{1.985102in}{0.989646in}}%
\pgfpathclose%
\pgfusepath{stroke,fill}%
\end{pgfscope}%
\begin{pgfscope}%
\pgfpathrectangle{\pgfqpoint{0.100000in}{0.212622in}}{\pgfqpoint{3.696000in}{3.696000in}}%
\pgfusepath{clip}%
\pgfsetbuttcap%
\pgfsetroundjoin%
\definecolor{currentfill}{rgb}{0.121569,0.466667,0.705882}%
\pgfsetfillcolor{currentfill}%
\pgfsetfillopacity{0.904259}%
\pgfsetlinewidth{1.003750pt}%
\definecolor{currentstroke}{rgb}{0.121569,0.466667,0.705882}%
\pgfsetstrokecolor{currentstroke}%
\pgfsetstrokeopacity{0.904259}%
\pgfsetdash{}{0pt}%
\pgfpathmoveto{\pgfqpoint{1.988261in}{0.988853in}}%
\pgfpathcurveto{\pgfqpoint{1.996497in}{0.988853in}}{\pgfqpoint{2.004397in}{0.992125in}}{\pgfqpoint{2.010221in}{0.997949in}}%
\pgfpathcurveto{\pgfqpoint{2.016045in}{1.003773in}}{\pgfqpoint{2.019317in}{1.011673in}}{\pgfqpoint{2.019317in}{1.019909in}}%
\pgfpathcurveto{\pgfqpoint{2.019317in}{1.028146in}}{\pgfqpoint{2.016045in}{1.036046in}}{\pgfqpoint{2.010221in}{1.041870in}}%
\pgfpathcurveto{\pgfqpoint{2.004397in}{1.047694in}}{\pgfqpoint{1.996497in}{1.050966in}}{\pgfqpoint{1.988261in}{1.050966in}}%
\pgfpathcurveto{\pgfqpoint{1.980024in}{1.050966in}}{\pgfqpoint{1.972124in}{1.047694in}}{\pgfqpoint{1.966300in}{1.041870in}}%
\pgfpathcurveto{\pgfqpoint{1.960476in}{1.036046in}}{\pgfqpoint{1.957204in}{1.028146in}}{\pgfqpoint{1.957204in}{1.019909in}}%
\pgfpathcurveto{\pgfqpoint{1.957204in}{1.011673in}}{\pgfqpoint{1.960476in}{1.003773in}}{\pgfqpoint{1.966300in}{0.997949in}}%
\pgfpathcurveto{\pgfqpoint{1.972124in}{0.992125in}}{\pgfqpoint{1.980024in}{0.988853in}}{\pgfqpoint{1.988261in}{0.988853in}}%
\pgfpathclose%
\pgfusepath{stroke,fill}%
\end{pgfscope}%
\begin{pgfscope}%
\pgfpathrectangle{\pgfqpoint{0.100000in}{0.212622in}}{\pgfqpoint{3.696000in}{3.696000in}}%
\pgfusepath{clip}%
\pgfsetbuttcap%
\pgfsetroundjoin%
\definecolor{currentfill}{rgb}{0.121569,0.466667,0.705882}%
\pgfsetfillcolor{currentfill}%
\pgfsetfillopacity{0.905472}%
\pgfsetlinewidth{1.003750pt}%
\definecolor{currentstroke}{rgb}{0.121569,0.466667,0.705882}%
\pgfsetstrokecolor{currentstroke}%
\pgfsetstrokeopacity{0.905472}%
\pgfsetdash{}{0pt}%
\pgfpathmoveto{\pgfqpoint{1.993931in}{0.987443in}}%
\pgfpathcurveto{\pgfqpoint{2.002168in}{0.987443in}}{\pgfqpoint{2.010068in}{0.990715in}}{\pgfqpoint{2.015892in}{0.996539in}}%
\pgfpathcurveto{\pgfqpoint{2.021715in}{1.002363in}}{\pgfqpoint{2.024988in}{1.010263in}}{\pgfqpoint{2.024988in}{1.018499in}}%
\pgfpathcurveto{\pgfqpoint{2.024988in}{1.026736in}}{\pgfqpoint{2.021715in}{1.034636in}}{\pgfqpoint{2.015892in}{1.040460in}}%
\pgfpathcurveto{\pgfqpoint{2.010068in}{1.046283in}}{\pgfqpoint{2.002168in}{1.049556in}}{\pgfqpoint{1.993931in}{1.049556in}}%
\pgfpathcurveto{\pgfqpoint{1.985695in}{1.049556in}}{\pgfqpoint{1.977795in}{1.046283in}}{\pgfqpoint{1.971971in}{1.040460in}}%
\pgfpathcurveto{\pgfqpoint{1.966147in}{1.034636in}}{\pgfqpoint{1.962875in}{1.026736in}}{\pgfqpoint{1.962875in}{1.018499in}}%
\pgfpathcurveto{\pgfqpoint{1.962875in}{1.010263in}}{\pgfqpoint{1.966147in}{1.002363in}}{\pgfqpoint{1.971971in}{0.996539in}}%
\pgfpathcurveto{\pgfqpoint{1.977795in}{0.990715in}}{\pgfqpoint{1.985695in}{0.987443in}}{\pgfqpoint{1.993931in}{0.987443in}}%
\pgfpathclose%
\pgfusepath{stroke,fill}%
\end{pgfscope}%
\begin{pgfscope}%
\pgfpathrectangle{\pgfqpoint{0.100000in}{0.212622in}}{\pgfqpoint{3.696000in}{3.696000in}}%
\pgfusepath{clip}%
\pgfsetbuttcap%
\pgfsetroundjoin%
\definecolor{currentfill}{rgb}{0.121569,0.466667,0.705882}%
\pgfsetfillcolor{currentfill}%
\pgfsetfillopacity{0.906398}%
\pgfsetlinewidth{1.003750pt}%
\definecolor{currentstroke}{rgb}{0.121569,0.466667,0.705882}%
\pgfsetstrokecolor{currentstroke}%
\pgfsetstrokeopacity{0.906398}%
\pgfsetdash{}{0pt}%
\pgfpathmoveto{\pgfqpoint{2.364357in}{1.087949in}}%
\pgfpathcurveto{\pgfqpoint{2.372593in}{1.087949in}}{\pgfqpoint{2.380493in}{1.091221in}}{\pgfqpoint{2.386317in}{1.097045in}}%
\pgfpathcurveto{\pgfqpoint{2.392141in}{1.102869in}}{\pgfqpoint{2.395413in}{1.110769in}}{\pgfqpoint{2.395413in}{1.119005in}}%
\pgfpathcurveto{\pgfqpoint{2.395413in}{1.127241in}}{\pgfqpoint{2.392141in}{1.135141in}}{\pgfqpoint{2.386317in}{1.140965in}}%
\pgfpathcurveto{\pgfqpoint{2.380493in}{1.146789in}}{\pgfqpoint{2.372593in}{1.150062in}}{\pgfqpoint{2.364357in}{1.150062in}}%
\pgfpathcurveto{\pgfqpoint{2.356121in}{1.150062in}}{\pgfqpoint{2.348221in}{1.146789in}}{\pgfqpoint{2.342397in}{1.140965in}}%
\pgfpathcurveto{\pgfqpoint{2.336573in}{1.135141in}}{\pgfqpoint{2.333300in}{1.127241in}}{\pgfqpoint{2.333300in}{1.119005in}}%
\pgfpathcurveto{\pgfqpoint{2.333300in}{1.110769in}}{\pgfqpoint{2.336573in}{1.102869in}}{\pgfqpoint{2.342397in}{1.097045in}}%
\pgfpathcurveto{\pgfqpoint{2.348221in}{1.091221in}}{\pgfqpoint{2.356121in}{1.087949in}}{\pgfqpoint{2.364357in}{1.087949in}}%
\pgfpathclose%
\pgfusepath{stroke,fill}%
\end{pgfscope}%
\begin{pgfscope}%
\pgfpathrectangle{\pgfqpoint{0.100000in}{0.212622in}}{\pgfqpoint{3.696000in}{3.696000in}}%
\pgfusepath{clip}%
\pgfsetbuttcap%
\pgfsetroundjoin%
\definecolor{currentfill}{rgb}{0.121569,0.466667,0.705882}%
\pgfsetfillcolor{currentfill}%
\pgfsetfillopacity{0.906565}%
\pgfsetlinewidth{1.003750pt}%
\definecolor{currentstroke}{rgb}{0.121569,0.466667,0.705882}%
\pgfsetstrokecolor{currentstroke}%
\pgfsetstrokeopacity{0.906565}%
\pgfsetdash{}{0pt}%
\pgfpathmoveto{\pgfqpoint{1.999003in}{0.986319in}}%
\pgfpathcurveto{\pgfqpoint{2.007239in}{0.986319in}}{\pgfqpoint{2.015139in}{0.989591in}}{\pgfqpoint{2.020963in}{0.995415in}}%
\pgfpathcurveto{\pgfqpoint{2.026787in}{1.001239in}}{\pgfqpoint{2.030060in}{1.009139in}}{\pgfqpoint{2.030060in}{1.017375in}}%
\pgfpathcurveto{\pgfqpoint{2.030060in}{1.025612in}}{\pgfqpoint{2.026787in}{1.033512in}}{\pgfqpoint{2.020963in}{1.039336in}}%
\pgfpathcurveto{\pgfqpoint{2.015139in}{1.045160in}}{\pgfqpoint{2.007239in}{1.048432in}}{\pgfqpoint{1.999003in}{1.048432in}}%
\pgfpathcurveto{\pgfqpoint{1.990767in}{1.048432in}}{\pgfqpoint{1.982867in}{1.045160in}}{\pgfqpoint{1.977043in}{1.039336in}}%
\pgfpathcurveto{\pgfqpoint{1.971219in}{1.033512in}}{\pgfqpoint{1.967947in}{1.025612in}}{\pgfqpoint{1.967947in}{1.017375in}}%
\pgfpathcurveto{\pgfqpoint{1.967947in}{1.009139in}}{\pgfqpoint{1.971219in}{1.001239in}}{\pgfqpoint{1.977043in}{0.995415in}}%
\pgfpathcurveto{\pgfqpoint{1.982867in}{0.989591in}}{\pgfqpoint{1.990767in}{0.986319in}}{\pgfqpoint{1.999003in}{0.986319in}}%
\pgfpathclose%
\pgfusepath{stroke,fill}%
\end{pgfscope}%
\begin{pgfscope}%
\pgfpathrectangle{\pgfqpoint{0.100000in}{0.212622in}}{\pgfqpoint{3.696000in}{3.696000in}}%
\pgfusepath{clip}%
\pgfsetbuttcap%
\pgfsetroundjoin%
\definecolor{currentfill}{rgb}{0.121569,0.466667,0.705882}%
\pgfsetfillcolor{currentfill}%
\pgfsetfillopacity{0.907322}%
\pgfsetlinewidth{1.003750pt}%
\definecolor{currentstroke}{rgb}{0.121569,0.466667,0.705882}%
\pgfsetstrokecolor{currentstroke}%
\pgfsetstrokeopacity{0.907322}%
\pgfsetdash{}{0pt}%
\pgfpathmoveto{\pgfqpoint{2.002924in}{0.985293in}}%
\pgfpathcurveto{\pgfqpoint{2.011161in}{0.985293in}}{\pgfqpoint{2.019061in}{0.988565in}}{\pgfqpoint{2.024885in}{0.994389in}}%
\pgfpathcurveto{\pgfqpoint{2.030709in}{1.000213in}}{\pgfqpoint{2.033981in}{1.008113in}}{\pgfqpoint{2.033981in}{1.016349in}}%
\pgfpathcurveto{\pgfqpoint{2.033981in}{1.024586in}}{\pgfqpoint{2.030709in}{1.032486in}}{\pgfqpoint{2.024885in}{1.038310in}}%
\pgfpathcurveto{\pgfqpoint{2.019061in}{1.044134in}}{\pgfqpoint{2.011161in}{1.047406in}}{\pgfqpoint{2.002924in}{1.047406in}}%
\pgfpathcurveto{\pgfqpoint{1.994688in}{1.047406in}}{\pgfqpoint{1.986788in}{1.044134in}}{\pgfqpoint{1.980964in}{1.038310in}}%
\pgfpathcurveto{\pgfqpoint{1.975140in}{1.032486in}}{\pgfqpoint{1.971868in}{1.024586in}}{\pgfqpoint{1.971868in}{1.016349in}}%
\pgfpathcurveto{\pgfqpoint{1.971868in}{1.008113in}}{\pgfqpoint{1.975140in}{1.000213in}}{\pgfqpoint{1.980964in}{0.994389in}}%
\pgfpathcurveto{\pgfqpoint{1.986788in}{0.988565in}}{\pgfqpoint{1.994688in}{0.985293in}}{\pgfqpoint{2.002924in}{0.985293in}}%
\pgfpathclose%
\pgfusepath{stroke,fill}%
\end{pgfscope}%
\begin{pgfscope}%
\pgfpathrectangle{\pgfqpoint{0.100000in}{0.212622in}}{\pgfqpoint{3.696000in}{3.696000in}}%
\pgfusepath{clip}%
\pgfsetbuttcap%
\pgfsetroundjoin%
\definecolor{currentfill}{rgb}{0.121569,0.466667,0.705882}%
\pgfsetfillcolor{currentfill}%
\pgfsetfillopacity{0.908820}%
\pgfsetlinewidth{1.003750pt}%
\definecolor{currentstroke}{rgb}{0.121569,0.466667,0.705882}%
\pgfsetstrokecolor{currentstroke}%
\pgfsetstrokeopacity{0.908820}%
\pgfsetdash{}{0pt}%
\pgfpathmoveto{\pgfqpoint{2.010031in}{0.983764in}}%
\pgfpathcurveto{\pgfqpoint{2.018267in}{0.983764in}}{\pgfqpoint{2.026167in}{0.987036in}}{\pgfqpoint{2.031991in}{0.992860in}}%
\pgfpathcurveto{\pgfqpoint{2.037815in}{0.998684in}}{\pgfqpoint{2.041087in}{1.006584in}}{\pgfqpoint{2.041087in}{1.014820in}}%
\pgfpathcurveto{\pgfqpoint{2.041087in}{1.023057in}}{\pgfqpoint{2.037815in}{1.030957in}}{\pgfqpoint{2.031991in}{1.036781in}}%
\pgfpathcurveto{\pgfqpoint{2.026167in}{1.042605in}}{\pgfqpoint{2.018267in}{1.045877in}}{\pgfqpoint{2.010031in}{1.045877in}}%
\pgfpathcurveto{\pgfqpoint{2.001794in}{1.045877in}}{\pgfqpoint{1.993894in}{1.042605in}}{\pgfqpoint{1.988070in}{1.036781in}}%
\pgfpathcurveto{\pgfqpoint{1.982246in}{1.030957in}}{\pgfqpoint{1.978974in}{1.023057in}}{\pgfqpoint{1.978974in}{1.014820in}}%
\pgfpathcurveto{\pgfqpoint{1.978974in}{1.006584in}}{\pgfqpoint{1.982246in}{0.998684in}}{\pgfqpoint{1.988070in}{0.992860in}}%
\pgfpathcurveto{\pgfqpoint{1.993894in}{0.987036in}}{\pgfqpoint{2.001794in}{0.983764in}}{\pgfqpoint{2.010031in}{0.983764in}}%
\pgfpathclose%
\pgfusepath{stroke,fill}%
\end{pgfscope}%
\begin{pgfscope}%
\pgfpathrectangle{\pgfqpoint{0.100000in}{0.212622in}}{\pgfqpoint{3.696000in}{3.696000in}}%
\pgfusepath{clip}%
\pgfsetbuttcap%
\pgfsetroundjoin%
\definecolor{currentfill}{rgb}{0.121569,0.466667,0.705882}%
\pgfsetfillcolor{currentfill}%
\pgfsetfillopacity{0.910180}%
\pgfsetlinewidth{1.003750pt}%
\definecolor{currentstroke}{rgb}{0.121569,0.466667,0.705882}%
\pgfsetstrokecolor{currentstroke}%
\pgfsetstrokeopacity{0.910180}%
\pgfsetdash{}{0pt}%
\pgfpathmoveto{\pgfqpoint{2.368197in}{1.075035in}}%
\pgfpathcurveto{\pgfqpoint{2.376434in}{1.075035in}}{\pgfqpoint{2.384334in}{1.078308in}}{\pgfqpoint{2.390158in}{1.084132in}}%
\pgfpathcurveto{\pgfqpoint{2.395982in}{1.089955in}}{\pgfqpoint{2.399254in}{1.097856in}}{\pgfqpoint{2.399254in}{1.106092in}}%
\pgfpathcurveto{\pgfqpoint{2.399254in}{1.114328in}}{\pgfqpoint{2.395982in}{1.122228in}}{\pgfqpoint{2.390158in}{1.128052in}}%
\pgfpathcurveto{\pgfqpoint{2.384334in}{1.133876in}}{\pgfqpoint{2.376434in}{1.137148in}}{\pgfqpoint{2.368197in}{1.137148in}}%
\pgfpathcurveto{\pgfqpoint{2.359961in}{1.137148in}}{\pgfqpoint{2.352061in}{1.133876in}}{\pgfqpoint{2.346237in}{1.128052in}}%
\pgfpathcurveto{\pgfqpoint{2.340413in}{1.122228in}}{\pgfqpoint{2.337141in}{1.114328in}}{\pgfqpoint{2.337141in}{1.106092in}}%
\pgfpathcurveto{\pgfqpoint{2.337141in}{1.097856in}}{\pgfqpoint{2.340413in}{1.089955in}}{\pgfqpoint{2.346237in}{1.084132in}}%
\pgfpathcurveto{\pgfqpoint{2.352061in}{1.078308in}}{\pgfqpoint{2.359961in}{1.075035in}}{\pgfqpoint{2.368197in}{1.075035in}}%
\pgfpathclose%
\pgfusepath{stroke,fill}%
\end{pgfscope}%
\begin{pgfscope}%
\pgfpathrectangle{\pgfqpoint{0.100000in}{0.212622in}}{\pgfqpoint{3.696000in}{3.696000in}}%
\pgfusepath{clip}%
\pgfsetbuttcap%
\pgfsetroundjoin%
\definecolor{currentfill}{rgb}{0.121569,0.466667,0.705882}%
\pgfsetfillcolor{currentfill}%
\pgfsetfillopacity{0.910248}%
\pgfsetlinewidth{1.003750pt}%
\definecolor{currentstroke}{rgb}{0.121569,0.466667,0.705882}%
\pgfsetstrokecolor{currentstroke}%
\pgfsetstrokeopacity{0.910248}%
\pgfsetdash{}{0pt}%
\pgfpathmoveto{\pgfqpoint{2.016367in}{0.982370in}}%
\pgfpathcurveto{\pgfqpoint{2.024603in}{0.982370in}}{\pgfqpoint{2.032503in}{0.985642in}}{\pgfqpoint{2.038327in}{0.991466in}}%
\pgfpathcurveto{\pgfqpoint{2.044151in}{0.997290in}}{\pgfqpoint{2.047423in}{1.005190in}}{\pgfqpoint{2.047423in}{1.013427in}}%
\pgfpathcurveto{\pgfqpoint{2.047423in}{1.021663in}}{\pgfqpoint{2.044151in}{1.029563in}}{\pgfqpoint{2.038327in}{1.035387in}}%
\pgfpathcurveto{\pgfqpoint{2.032503in}{1.041211in}}{\pgfqpoint{2.024603in}{1.044483in}}{\pgfqpoint{2.016367in}{1.044483in}}%
\pgfpathcurveto{\pgfqpoint{2.008130in}{1.044483in}}{\pgfqpoint{2.000230in}{1.041211in}}{\pgfqpoint{1.994406in}{1.035387in}}%
\pgfpathcurveto{\pgfqpoint{1.988583in}{1.029563in}}{\pgfqpoint{1.985310in}{1.021663in}}{\pgfqpoint{1.985310in}{1.013427in}}%
\pgfpathcurveto{\pgfqpoint{1.985310in}{1.005190in}}{\pgfqpoint{1.988583in}{0.997290in}}{\pgfqpoint{1.994406in}{0.991466in}}%
\pgfpathcurveto{\pgfqpoint{2.000230in}{0.985642in}}{\pgfqpoint{2.008130in}{0.982370in}}{\pgfqpoint{2.016367in}{0.982370in}}%
\pgfpathclose%
\pgfusepath{stroke,fill}%
\end{pgfscope}%
\begin{pgfscope}%
\pgfpathrectangle{\pgfqpoint{0.100000in}{0.212622in}}{\pgfqpoint{3.696000in}{3.696000in}}%
\pgfusepath{clip}%
\pgfsetbuttcap%
\pgfsetroundjoin%
\definecolor{currentfill}{rgb}{0.121569,0.466667,0.705882}%
\pgfsetfillcolor{currentfill}%
\pgfsetfillopacity{0.911294}%
\pgfsetlinewidth{1.003750pt}%
\definecolor{currentstroke}{rgb}{0.121569,0.466667,0.705882}%
\pgfsetstrokecolor{currentstroke}%
\pgfsetstrokeopacity{0.911294}%
\pgfsetdash{}{0pt}%
\pgfpathmoveto{\pgfqpoint{2.021490in}{0.980858in}}%
\pgfpathcurveto{\pgfqpoint{2.029726in}{0.980858in}}{\pgfqpoint{2.037626in}{0.984130in}}{\pgfqpoint{2.043450in}{0.989954in}}%
\pgfpathcurveto{\pgfqpoint{2.049274in}{0.995778in}}{\pgfqpoint{2.052546in}{1.003678in}}{\pgfqpoint{2.052546in}{1.011914in}}%
\pgfpathcurveto{\pgfqpoint{2.052546in}{1.020151in}}{\pgfqpoint{2.049274in}{1.028051in}}{\pgfqpoint{2.043450in}{1.033875in}}%
\pgfpathcurveto{\pgfqpoint{2.037626in}{1.039699in}}{\pgfqpoint{2.029726in}{1.042971in}}{\pgfqpoint{2.021490in}{1.042971in}}%
\pgfpathcurveto{\pgfqpoint{2.013253in}{1.042971in}}{\pgfqpoint{2.005353in}{1.039699in}}{\pgfqpoint{1.999529in}{1.033875in}}%
\pgfpathcurveto{\pgfqpoint{1.993705in}{1.028051in}}{\pgfqpoint{1.990433in}{1.020151in}}{\pgfqpoint{1.990433in}{1.011914in}}%
\pgfpathcurveto{\pgfqpoint{1.990433in}{1.003678in}}{\pgfqpoint{1.993705in}{0.995778in}}{\pgfqpoint{1.999529in}{0.989954in}}%
\pgfpathcurveto{\pgfqpoint{2.005353in}{0.984130in}}{\pgfqpoint{2.013253in}{0.980858in}}{\pgfqpoint{2.021490in}{0.980858in}}%
\pgfpathclose%
\pgfusepath{stroke,fill}%
\end{pgfscope}%
\begin{pgfscope}%
\pgfpathrectangle{\pgfqpoint{0.100000in}{0.212622in}}{\pgfqpoint{3.696000in}{3.696000in}}%
\pgfusepath{clip}%
\pgfsetbuttcap%
\pgfsetroundjoin%
\definecolor{currentfill}{rgb}{0.121569,0.466667,0.705882}%
\pgfsetfillcolor{currentfill}%
\pgfsetfillopacity{0.913365}%
\pgfsetlinewidth{1.003750pt}%
\definecolor{currentstroke}{rgb}{0.121569,0.466667,0.705882}%
\pgfsetstrokecolor{currentstroke}%
\pgfsetstrokeopacity{0.913365}%
\pgfsetdash{}{0pt}%
\pgfpathmoveto{\pgfqpoint{2.030798in}{0.978672in}}%
\pgfpathcurveto{\pgfqpoint{2.039034in}{0.978672in}}{\pgfqpoint{2.046934in}{0.981944in}}{\pgfqpoint{2.052758in}{0.987768in}}%
\pgfpathcurveto{\pgfqpoint{2.058582in}{0.993592in}}{\pgfqpoint{2.061854in}{1.001492in}}{\pgfqpoint{2.061854in}{1.009728in}}%
\pgfpathcurveto{\pgfqpoint{2.061854in}{1.017964in}}{\pgfqpoint{2.058582in}{1.025865in}}{\pgfqpoint{2.052758in}{1.031688in}}%
\pgfpathcurveto{\pgfqpoint{2.046934in}{1.037512in}}{\pgfqpoint{2.039034in}{1.040785in}}{\pgfqpoint{2.030798in}{1.040785in}}%
\pgfpathcurveto{\pgfqpoint{2.022562in}{1.040785in}}{\pgfqpoint{2.014661in}{1.037512in}}{\pgfqpoint{2.008838in}{1.031688in}}%
\pgfpathcurveto{\pgfqpoint{2.003014in}{1.025865in}}{\pgfqpoint{1.999741in}{1.017964in}}{\pgfqpoint{1.999741in}{1.009728in}}%
\pgfpathcurveto{\pgfqpoint{1.999741in}{1.001492in}}{\pgfqpoint{2.003014in}{0.993592in}}{\pgfqpoint{2.008838in}{0.987768in}}%
\pgfpathcurveto{\pgfqpoint{2.014661in}{0.981944in}}{\pgfqpoint{2.022562in}{0.978672in}}{\pgfqpoint{2.030798in}{0.978672in}}%
\pgfpathclose%
\pgfusepath{stroke,fill}%
\end{pgfscope}%
\begin{pgfscope}%
\pgfpathrectangle{\pgfqpoint{0.100000in}{0.212622in}}{\pgfqpoint{3.696000in}{3.696000in}}%
\pgfusepath{clip}%
\pgfsetbuttcap%
\pgfsetroundjoin%
\definecolor{currentfill}{rgb}{0.121569,0.466667,0.705882}%
\pgfsetfillcolor{currentfill}%
\pgfsetfillopacity{0.914288}%
\pgfsetlinewidth{1.003750pt}%
\definecolor{currentstroke}{rgb}{0.121569,0.466667,0.705882}%
\pgfsetstrokecolor{currentstroke}%
\pgfsetstrokeopacity{0.914288}%
\pgfsetdash{}{0pt}%
\pgfpathmoveto{\pgfqpoint{2.372298in}{1.061014in}}%
\pgfpathcurveto{\pgfqpoint{2.380534in}{1.061014in}}{\pgfqpoint{2.388435in}{1.064287in}}{\pgfqpoint{2.394258in}{1.070111in}}%
\pgfpathcurveto{\pgfqpoint{2.400082in}{1.075935in}}{\pgfqpoint{2.403355in}{1.083835in}}{\pgfqpoint{2.403355in}{1.092071in}}%
\pgfpathcurveto{\pgfqpoint{2.403355in}{1.100307in}}{\pgfqpoint{2.400082in}{1.108207in}}{\pgfqpoint{2.394258in}{1.114031in}}%
\pgfpathcurveto{\pgfqpoint{2.388435in}{1.119855in}}{\pgfqpoint{2.380534in}{1.123127in}}{\pgfqpoint{2.372298in}{1.123127in}}%
\pgfpathcurveto{\pgfqpoint{2.364062in}{1.123127in}}{\pgfqpoint{2.356162in}{1.119855in}}{\pgfqpoint{2.350338in}{1.114031in}}%
\pgfpathcurveto{\pgfqpoint{2.344514in}{1.108207in}}{\pgfqpoint{2.341242in}{1.100307in}}{\pgfqpoint{2.341242in}{1.092071in}}%
\pgfpathcurveto{\pgfqpoint{2.341242in}{1.083835in}}{\pgfqpoint{2.344514in}{1.075935in}}{\pgfqpoint{2.350338in}{1.070111in}}%
\pgfpathcurveto{\pgfqpoint{2.356162in}{1.064287in}}{\pgfqpoint{2.364062in}{1.061014in}}{\pgfqpoint{2.372298in}{1.061014in}}%
\pgfpathclose%
\pgfusepath{stroke,fill}%
\end{pgfscope}%
\begin{pgfscope}%
\pgfpathrectangle{\pgfqpoint{0.100000in}{0.212622in}}{\pgfqpoint{3.696000in}{3.696000in}}%
\pgfusepath{clip}%
\pgfsetbuttcap%
\pgfsetroundjoin%
\definecolor{currentfill}{rgb}{0.121569,0.466667,0.705882}%
\pgfsetfillcolor{currentfill}%
\pgfsetfillopacity{0.915323}%
\pgfsetlinewidth{1.003750pt}%
\definecolor{currentstroke}{rgb}{0.121569,0.466667,0.705882}%
\pgfsetstrokecolor{currentstroke}%
\pgfsetstrokeopacity{0.915323}%
\pgfsetdash{}{0pt}%
\pgfpathmoveto{\pgfqpoint{2.039131in}{0.976680in}}%
\pgfpathcurveto{\pgfqpoint{2.047367in}{0.976680in}}{\pgfqpoint{2.055267in}{0.979953in}}{\pgfqpoint{2.061091in}{0.985776in}}%
\pgfpathcurveto{\pgfqpoint{2.066915in}{0.991600in}}{\pgfqpoint{2.070187in}{0.999500in}}{\pgfqpoint{2.070187in}{1.007737in}}%
\pgfpathcurveto{\pgfqpoint{2.070187in}{1.015973in}}{\pgfqpoint{2.066915in}{1.023873in}}{\pgfqpoint{2.061091in}{1.029697in}}%
\pgfpathcurveto{\pgfqpoint{2.055267in}{1.035521in}}{\pgfqpoint{2.047367in}{1.038793in}}{\pgfqpoint{2.039131in}{1.038793in}}%
\pgfpathcurveto{\pgfqpoint{2.030894in}{1.038793in}}{\pgfqpoint{2.022994in}{1.035521in}}{\pgfqpoint{2.017170in}{1.029697in}}%
\pgfpathcurveto{\pgfqpoint{2.011346in}{1.023873in}}{\pgfqpoint{2.008074in}{1.015973in}}{\pgfqpoint{2.008074in}{1.007737in}}%
\pgfpathcurveto{\pgfqpoint{2.008074in}{0.999500in}}{\pgfqpoint{2.011346in}{0.991600in}}{\pgfqpoint{2.017170in}{0.985776in}}%
\pgfpathcurveto{\pgfqpoint{2.022994in}{0.979953in}}{\pgfqpoint{2.030894in}{0.976680in}}{\pgfqpoint{2.039131in}{0.976680in}}%
\pgfpathclose%
\pgfusepath{stroke,fill}%
\end{pgfscope}%
\begin{pgfscope}%
\pgfpathrectangle{\pgfqpoint{0.100000in}{0.212622in}}{\pgfqpoint{3.696000in}{3.696000in}}%
\pgfusepath{clip}%
\pgfsetbuttcap%
\pgfsetroundjoin%
\definecolor{currentfill}{rgb}{0.121569,0.466667,0.705882}%
\pgfsetfillcolor{currentfill}%
\pgfsetfillopacity{0.916864}%
\pgfsetlinewidth{1.003750pt}%
\definecolor{currentstroke}{rgb}{0.121569,0.466667,0.705882}%
\pgfsetstrokecolor{currentstroke}%
\pgfsetstrokeopacity{0.916864}%
\pgfsetdash{}{0pt}%
\pgfpathmoveto{\pgfqpoint{2.046649in}{0.974603in}}%
\pgfpathcurveto{\pgfqpoint{2.054885in}{0.974603in}}{\pgfqpoint{2.062785in}{0.977875in}}{\pgfqpoint{2.068609in}{0.983699in}}%
\pgfpathcurveto{\pgfqpoint{2.074433in}{0.989523in}}{\pgfqpoint{2.077705in}{0.997423in}}{\pgfqpoint{2.077705in}{1.005659in}}%
\pgfpathcurveto{\pgfqpoint{2.077705in}{1.013895in}}{\pgfqpoint{2.074433in}{1.021795in}}{\pgfqpoint{2.068609in}{1.027619in}}%
\pgfpathcurveto{\pgfqpoint{2.062785in}{1.033443in}}{\pgfqpoint{2.054885in}{1.036716in}}{\pgfqpoint{2.046649in}{1.036716in}}%
\pgfpathcurveto{\pgfqpoint{2.038413in}{1.036716in}}{\pgfqpoint{2.030513in}{1.033443in}}{\pgfqpoint{2.024689in}{1.027619in}}%
\pgfpathcurveto{\pgfqpoint{2.018865in}{1.021795in}}{\pgfqpoint{2.015592in}{1.013895in}}{\pgfqpoint{2.015592in}{1.005659in}}%
\pgfpathcurveto{\pgfqpoint{2.015592in}{0.997423in}}{\pgfqpoint{2.018865in}{0.989523in}}{\pgfqpoint{2.024689in}{0.983699in}}%
\pgfpathcurveto{\pgfqpoint{2.030513in}{0.977875in}}{\pgfqpoint{2.038413in}{0.974603in}}{\pgfqpoint{2.046649in}{0.974603in}}%
\pgfpathclose%
\pgfusepath{stroke,fill}%
\end{pgfscope}%
\begin{pgfscope}%
\pgfpathrectangle{\pgfqpoint{0.100000in}{0.212622in}}{\pgfqpoint{3.696000in}{3.696000in}}%
\pgfusepath{clip}%
\pgfsetbuttcap%
\pgfsetroundjoin%
\definecolor{currentfill}{rgb}{0.121569,0.466667,0.705882}%
\pgfsetfillcolor{currentfill}%
\pgfsetfillopacity{0.918367}%
\pgfsetlinewidth{1.003750pt}%
\definecolor{currentstroke}{rgb}{0.121569,0.466667,0.705882}%
\pgfsetstrokecolor{currentstroke}%
\pgfsetstrokeopacity{0.918367}%
\pgfsetdash{}{0pt}%
\pgfpathmoveto{\pgfqpoint{2.053094in}{0.972887in}}%
\pgfpathcurveto{\pgfqpoint{2.061330in}{0.972887in}}{\pgfqpoint{2.069230in}{0.976160in}}{\pgfqpoint{2.075054in}{0.981984in}}%
\pgfpathcurveto{\pgfqpoint{2.080878in}{0.987808in}}{\pgfqpoint{2.084150in}{0.995708in}}{\pgfqpoint{2.084150in}{1.003944in}}%
\pgfpathcurveto{\pgfqpoint{2.084150in}{1.012180in}}{\pgfqpoint{2.080878in}{1.020080in}}{\pgfqpoint{2.075054in}{1.025904in}}%
\pgfpathcurveto{\pgfqpoint{2.069230in}{1.031728in}}{\pgfqpoint{2.061330in}{1.035000in}}{\pgfqpoint{2.053094in}{1.035000in}}%
\pgfpathcurveto{\pgfqpoint{2.044858in}{1.035000in}}{\pgfqpoint{2.036957in}{1.031728in}}{\pgfqpoint{2.031134in}{1.025904in}}%
\pgfpathcurveto{\pgfqpoint{2.025310in}{1.020080in}}{\pgfqpoint{2.022037in}{1.012180in}}{\pgfqpoint{2.022037in}{1.003944in}}%
\pgfpathcurveto{\pgfqpoint{2.022037in}{0.995708in}}{\pgfqpoint{2.025310in}{0.987808in}}{\pgfqpoint{2.031134in}{0.981984in}}%
\pgfpathcurveto{\pgfqpoint{2.036957in}{0.976160in}}{\pgfqpoint{2.044858in}{0.972887in}}{\pgfqpoint{2.053094in}{0.972887in}}%
\pgfpathclose%
\pgfusepath{stroke,fill}%
\end{pgfscope}%
\begin{pgfscope}%
\pgfpathrectangle{\pgfqpoint{0.100000in}{0.212622in}}{\pgfqpoint{3.696000in}{3.696000in}}%
\pgfusepath{clip}%
\pgfsetbuttcap%
\pgfsetroundjoin%
\definecolor{currentfill}{rgb}{0.121569,0.466667,0.705882}%
\pgfsetfillcolor{currentfill}%
\pgfsetfillopacity{0.918542}%
\pgfsetlinewidth{1.003750pt}%
\definecolor{currentstroke}{rgb}{0.121569,0.466667,0.705882}%
\pgfsetstrokecolor{currentstroke}%
\pgfsetstrokeopacity{0.918542}%
\pgfsetdash{}{0pt}%
\pgfpathmoveto{\pgfqpoint{2.376667in}{1.045838in}}%
\pgfpathcurveto{\pgfqpoint{2.384903in}{1.045838in}}{\pgfqpoint{2.392803in}{1.049110in}}{\pgfqpoint{2.398627in}{1.054934in}}%
\pgfpathcurveto{\pgfqpoint{2.404451in}{1.060758in}}{\pgfqpoint{2.407724in}{1.068658in}}{\pgfqpoint{2.407724in}{1.076894in}}%
\pgfpathcurveto{\pgfqpoint{2.407724in}{1.085130in}}{\pgfqpoint{2.404451in}{1.093030in}}{\pgfqpoint{2.398627in}{1.098854in}}%
\pgfpathcurveto{\pgfqpoint{2.392803in}{1.104678in}}{\pgfqpoint{2.384903in}{1.107951in}}{\pgfqpoint{2.376667in}{1.107951in}}%
\pgfpathcurveto{\pgfqpoint{2.368431in}{1.107951in}}{\pgfqpoint{2.360531in}{1.104678in}}{\pgfqpoint{2.354707in}{1.098854in}}%
\pgfpathcurveto{\pgfqpoint{2.348883in}{1.093030in}}{\pgfqpoint{2.345611in}{1.085130in}}{\pgfqpoint{2.345611in}{1.076894in}}%
\pgfpathcurveto{\pgfqpoint{2.345611in}{1.068658in}}{\pgfqpoint{2.348883in}{1.060758in}}{\pgfqpoint{2.354707in}{1.054934in}}%
\pgfpathcurveto{\pgfqpoint{2.360531in}{1.049110in}}{\pgfqpoint{2.368431in}{1.045838in}}{\pgfqpoint{2.376667in}{1.045838in}}%
\pgfpathclose%
\pgfusepath{stroke,fill}%
\end{pgfscope}%
\begin{pgfscope}%
\pgfpathrectangle{\pgfqpoint{0.100000in}{0.212622in}}{\pgfqpoint{3.696000in}{3.696000in}}%
\pgfusepath{clip}%
\pgfsetbuttcap%
\pgfsetroundjoin%
\definecolor{currentfill}{rgb}{0.121569,0.466667,0.705882}%
\pgfsetfillcolor{currentfill}%
\pgfsetfillopacity{0.920916}%
\pgfsetlinewidth{1.003750pt}%
\definecolor{currentstroke}{rgb}{0.121569,0.466667,0.705882}%
\pgfsetstrokecolor{currentstroke}%
\pgfsetstrokeopacity{0.920916}%
\pgfsetdash{}{0pt}%
\pgfpathmoveto{\pgfqpoint{2.064871in}{0.969310in}}%
\pgfpathcurveto{\pgfqpoint{2.073107in}{0.969310in}}{\pgfqpoint{2.081007in}{0.972582in}}{\pgfqpoint{2.086831in}{0.978406in}}%
\pgfpathcurveto{\pgfqpoint{2.092655in}{0.984230in}}{\pgfqpoint{2.095927in}{0.992130in}}{\pgfqpoint{2.095927in}{1.000366in}}%
\pgfpathcurveto{\pgfqpoint{2.095927in}{1.008603in}}{\pgfqpoint{2.092655in}{1.016503in}}{\pgfqpoint{2.086831in}{1.022327in}}%
\pgfpathcurveto{\pgfqpoint{2.081007in}{1.028151in}}{\pgfqpoint{2.073107in}{1.031423in}}{\pgfqpoint{2.064871in}{1.031423in}}%
\pgfpathcurveto{\pgfqpoint{2.056635in}{1.031423in}}{\pgfqpoint{2.048735in}{1.028151in}}{\pgfqpoint{2.042911in}{1.022327in}}%
\pgfpathcurveto{\pgfqpoint{2.037087in}{1.016503in}}{\pgfqpoint{2.033814in}{1.008603in}}{\pgfqpoint{2.033814in}{1.000366in}}%
\pgfpathcurveto{\pgfqpoint{2.033814in}{0.992130in}}{\pgfqpoint{2.037087in}{0.984230in}}{\pgfqpoint{2.042911in}{0.978406in}}%
\pgfpathcurveto{\pgfqpoint{2.048735in}{0.972582in}}{\pgfqpoint{2.056635in}{0.969310in}}{\pgfqpoint{2.064871in}{0.969310in}}%
\pgfpathclose%
\pgfusepath{stroke,fill}%
\end{pgfscope}%
\begin{pgfscope}%
\pgfpathrectangle{\pgfqpoint{0.100000in}{0.212622in}}{\pgfqpoint{3.696000in}{3.696000in}}%
\pgfusepath{clip}%
\pgfsetbuttcap%
\pgfsetroundjoin%
\definecolor{currentfill}{rgb}{0.121569,0.466667,0.705882}%
\pgfsetfillcolor{currentfill}%
\pgfsetfillopacity{0.923123}%
\pgfsetlinewidth{1.003750pt}%
\definecolor{currentstroke}{rgb}{0.121569,0.466667,0.705882}%
\pgfsetstrokecolor{currentstroke}%
\pgfsetstrokeopacity{0.923123}%
\pgfsetdash{}{0pt}%
\pgfpathmoveto{\pgfqpoint{2.075447in}{0.965535in}}%
\pgfpathcurveto{\pgfqpoint{2.083684in}{0.965535in}}{\pgfqpoint{2.091584in}{0.968808in}}{\pgfqpoint{2.097408in}{0.974632in}}%
\pgfpathcurveto{\pgfqpoint{2.103232in}{0.980456in}}{\pgfqpoint{2.106504in}{0.988356in}}{\pgfqpoint{2.106504in}{0.996592in}}%
\pgfpathcurveto{\pgfqpoint{2.106504in}{1.004828in}}{\pgfqpoint{2.103232in}{1.012728in}}{\pgfqpoint{2.097408in}{1.018552in}}%
\pgfpathcurveto{\pgfqpoint{2.091584in}{1.024376in}}{\pgfqpoint{2.083684in}{1.027648in}}{\pgfqpoint{2.075447in}{1.027648in}}%
\pgfpathcurveto{\pgfqpoint{2.067211in}{1.027648in}}{\pgfqpoint{2.059311in}{1.024376in}}{\pgfqpoint{2.053487in}{1.018552in}}%
\pgfpathcurveto{\pgfqpoint{2.047663in}{1.012728in}}{\pgfqpoint{2.044391in}{1.004828in}}{\pgfqpoint{2.044391in}{0.996592in}}%
\pgfpathcurveto{\pgfqpoint{2.044391in}{0.988356in}}{\pgfqpoint{2.047663in}{0.980456in}}{\pgfqpoint{2.053487in}{0.974632in}}%
\pgfpathcurveto{\pgfqpoint{2.059311in}{0.968808in}}{\pgfqpoint{2.067211in}{0.965535in}}{\pgfqpoint{2.075447in}{0.965535in}}%
\pgfpathclose%
\pgfusepath{stroke,fill}%
\end{pgfscope}%
\begin{pgfscope}%
\pgfpathrectangle{\pgfqpoint{0.100000in}{0.212622in}}{\pgfqpoint{3.696000in}{3.696000in}}%
\pgfusepath{clip}%
\pgfsetbuttcap%
\pgfsetroundjoin%
\definecolor{currentfill}{rgb}{0.121569,0.466667,0.705882}%
\pgfsetfillcolor{currentfill}%
\pgfsetfillopacity{0.923139}%
\pgfsetlinewidth{1.003750pt}%
\definecolor{currentstroke}{rgb}{0.121569,0.466667,0.705882}%
\pgfsetstrokecolor{currentstroke}%
\pgfsetstrokeopacity{0.923139}%
\pgfsetdash{}{0pt}%
\pgfpathmoveto{\pgfqpoint{2.381097in}{1.030114in}}%
\pgfpathcurveto{\pgfqpoint{2.389334in}{1.030114in}}{\pgfqpoint{2.397234in}{1.033386in}}{\pgfqpoint{2.403058in}{1.039210in}}%
\pgfpathcurveto{\pgfqpoint{2.408882in}{1.045034in}}{\pgfqpoint{2.412154in}{1.052934in}}{\pgfqpoint{2.412154in}{1.061170in}}%
\pgfpathcurveto{\pgfqpoint{2.412154in}{1.069407in}}{\pgfqpoint{2.408882in}{1.077307in}}{\pgfqpoint{2.403058in}{1.083131in}}%
\pgfpathcurveto{\pgfqpoint{2.397234in}{1.088955in}}{\pgfqpoint{2.389334in}{1.092227in}}{\pgfqpoint{2.381097in}{1.092227in}}%
\pgfpathcurveto{\pgfqpoint{2.372861in}{1.092227in}}{\pgfqpoint{2.364961in}{1.088955in}}{\pgfqpoint{2.359137in}{1.083131in}}%
\pgfpathcurveto{\pgfqpoint{2.353313in}{1.077307in}}{\pgfqpoint{2.350041in}{1.069407in}}{\pgfqpoint{2.350041in}{1.061170in}}%
\pgfpathcurveto{\pgfqpoint{2.350041in}{1.052934in}}{\pgfqpoint{2.353313in}{1.045034in}}{\pgfqpoint{2.359137in}{1.039210in}}%
\pgfpathcurveto{\pgfqpoint{2.364961in}{1.033386in}}{\pgfqpoint{2.372861in}{1.030114in}}{\pgfqpoint{2.381097in}{1.030114in}}%
\pgfpathclose%
\pgfusepath{stroke,fill}%
\end{pgfscope}%
\begin{pgfscope}%
\pgfpathrectangle{\pgfqpoint{0.100000in}{0.212622in}}{\pgfqpoint{3.696000in}{3.696000in}}%
\pgfusepath{clip}%
\pgfsetbuttcap%
\pgfsetroundjoin%
\definecolor{currentfill}{rgb}{0.121569,0.466667,0.705882}%
\pgfsetfillcolor{currentfill}%
\pgfsetfillopacity{0.925426}%
\pgfsetlinewidth{1.003750pt}%
\definecolor{currentstroke}{rgb}{0.121569,0.466667,0.705882}%
\pgfsetstrokecolor{currentstroke}%
\pgfsetstrokeopacity{0.925426}%
\pgfsetdash{}{0pt}%
\pgfpathmoveto{\pgfqpoint{2.084811in}{0.962275in}}%
\pgfpathcurveto{\pgfqpoint{2.093047in}{0.962275in}}{\pgfqpoint{2.100947in}{0.965547in}}{\pgfqpoint{2.106771in}{0.971371in}}%
\pgfpathcurveto{\pgfqpoint{2.112595in}{0.977195in}}{\pgfqpoint{2.115867in}{0.985095in}}{\pgfqpoint{2.115867in}{0.993331in}}%
\pgfpathcurveto{\pgfqpoint{2.115867in}{1.001567in}}{\pgfqpoint{2.112595in}{1.009468in}}{\pgfqpoint{2.106771in}{1.015291in}}%
\pgfpathcurveto{\pgfqpoint{2.100947in}{1.021115in}}{\pgfqpoint{2.093047in}{1.024388in}}{\pgfqpoint{2.084811in}{1.024388in}}%
\pgfpathcurveto{\pgfqpoint{2.076575in}{1.024388in}}{\pgfqpoint{2.068674in}{1.021115in}}{\pgfqpoint{2.062851in}{1.015291in}}%
\pgfpathcurveto{\pgfqpoint{2.057027in}{1.009468in}}{\pgfqpoint{2.053754in}{1.001567in}}{\pgfqpoint{2.053754in}{0.993331in}}%
\pgfpathcurveto{\pgfqpoint{2.053754in}{0.985095in}}{\pgfqpoint{2.057027in}{0.977195in}}{\pgfqpoint{2.062851in}{0.971371in}}%
\pgfpathcurveto{\pgfqpoint{2.068674in}{0.965547in}}{\pgfqpoint{2.076575in}{0.962275in}}{\pgfqpoint{2.084811in}{0.962275in}}%
\pgfpathclose%
\pgfusepath{stroke,fill}%
\end{pgfscope}%
\begin{pgfscope}%
\pgfpathrectangle{\pgfqpoint{0.100000in}{0.212622in}}{\pgfqpoint{3.696000in}{3.696000in}}%
\pgfusepath{clip}%
\pgfsetbuttcap%
\pgfsetroundjoin%
\definecolor{currentfill}{rgb}{0.121569,0.466667,0.705882}%
\pgfsetfillcolor{currentfill}%
\pgfsetfillopacity{0.927478}%
\pgfsetlinewidth{1.003750pt}%
\definecolor{currentstroke}{rgb}{0.121569,0.466667,0.705882}%
\pgfsetstrokecolor{currentstroke}%
\pgfsetstrokeopacity{0.927478}%
\pgfsetdash{}{0pt}%
\pgfpathmoveto{\pgfqpoint{2.093997in}{0.958628in}}%
\pgfpathcurveto{\pgfqpoint{2.102233in}{0.958628in}}{\pgfqpoint{2.110133in}{0.961900in}}{\pgfqpoint{2.115957in}{0.967724in}}%
\pgfpathcurveto{\pgfqpoint{2.121781in}{0.973548in}}{\pgfqpoint{2.125053in}{0.981448in}}{\pgfqpoint{2.125053in}{0.989684in}}%
\pgfpathcurveto{\pgfqpoint{2.125053in}{0.997920in}}{\pgfqpoint{2.121781in}{1.005820in}}{\pgfqpoint{2.115957in}{1.011644in}}%
\pgfpathcurveto{\pgfqpoint{2.110133in}{1.017468in}}{\pgfqpoint{2.102233in}{1.020741in}}{\pgfqpoint{2.093997in}{1.020741in}}%
\pgfpathcurveto{\pgfqpoint{2.085760in}{1.020741in}}{\pgfqpoint{2.077860in}{1.017468in}}{\pgfqpoint{2.072036in}{1.011644in}}%
\pgfpathcurveto{\pgfqpoint{2.066213in}{1.005820in}}{\pgfqpoint{2.062940in}{0.997920in}}{\pgfqpoint{2.062940in}{0.989684in}}%
\pgfpathcurveto{\pgfqpoint{2.062940in}{0.981448in}}{\pgfqpoint{2.066213in}{0.973548in}}{\pgfqpoint{2.072036in}{0.967724in}}%
\pgfpathcurveto{\pgfqpoint{2.077860in}{0.961900in}}{\pgfqpoint{2.085760in}{0.958628in}}{\pgfqpoint{2.093997in}{0.958628in}}%
\pgfpathclose%
\pgfusepath{stroke,fill}%
\end{pgfscope}%
\begin{pgfscope}%
\pgfpathrectangle{\pgfqpoint{0.100000in}{0.212622in}}{\pgfqpoint{3.696000in}{3.696000in}}%
\pgfusepath{clip}%
\pgfsetbuttcap%
\pgfsetroundjoin%
\definecolor{currentfill}{rgb}{0.121569,0.466667,0.705882}%
\pgfsetfillcolor{currentfill}%
\pgfsetfillopacity{0.927937}%
\pgfsetlinewidth{1.003750pt}%
\definecolor{currentstroke}{rgb}{0.121569,0.466667,0.705882}%
\pgfsetstrokecolor{currentstroke}%
\pgfsetstrokeopacity{0.927937}%
\pgfsetdash{}{0pt}%
\pgfpathmoveto{\pgfqpoint{2.385870in}{1.013422in}}%
\pgfpathcurveto{\pgfqpoint{2.394106in}{1.013422in}}{\pgfqpoint{2.402006in}{1.016694in}}{\pgfqpoint{2.407830in}{1.022518in}}%
\pgfpathcurveto{\pgfqpoint{2.413654in}{1.028342in}}{\pgfqpoint{2.416926in}{1.036242in}}{\pgfqpoint{2.416926in}{1.044479in}}%
\pgfpathcurveto{\pgfqpoint{2.416926in}{1.052715in}}{\pgfqpoint{2.413654in}{1.060615in}}{\pgfqpoint{2.407830in}{1.066439in}}%
\pgfpathcurveto{\pgfqpoint{2.402006in}{1.072263in}}{\pgfqpoint{2.394106in}{1.075535in}}{\pgfqpoint{2.385870in}{1.075535in}}%
\pgfpathcurveto{\pgfqpoint{2.377634in}{1.075535in}}{\pgfqpoint{2.369734in}{1.072263in}}{\pgfqpoint{2.363910in}{1.066439in}}%
\pgfpathcurveto{\pgfqpoint{2.358086in}{1.060615in}}{\pgfqpoint{2.354813in}{1.052715in}}{\pgfqpoint{2.354813in}{1.044479in}}%
\pgfpathcurveto{\pgfqpoint{2.354813in}{1.036242in}}{\pgfqpoint{2.358086in}{1.028342in}}{\pgfqpoint{2.363910in}{1.022518in}}%
\pgfpathcurveto{\pgfqpoint{2.369734in}{1.016694in}}{\pgfqpoint{2.377634in}{1.013422in}}{\pgfqpoint{2.385870in}{1.013422in}}%
\pgfpathclose%
\pgfusepath{stroke,fill}%
\end{pgfscope}%
\begin{pgfscope}%
\pgfpathrectangle{\pgfqpoint{0.100000in}{0.212622in}}{\pgfqpoint{3.696000in}{3.696000in}}%
\pgfusepath{clip}%
\pgfsetbuttcap%
\pgfsetroundjoin%
\definecolor{currentfill}{rgb}{0.121569,0.466667,0.705882}%
\pgfsetfillcolor{currentfill}%
\pgfsetfillopacity{0.929472}%
\pgfsetlinewidth{1.003750pt}%
\definecolor{currentstroke}{rgb}{0.121569,0.466667,0.705882}%
\pgfsetstrokecolor{currentstroke}%
\pgfsetstrokeopacity{0.929472}%
\pgfsetdash{}{0pt}%
\pgfpathmoveto{\pgfqpoint{2.102360in}{0.954425in}}%
\pgfpathcurveto{\pgfqpoint{2.110596in}{0.954425in}}{\pgfqpoint{2.118496in}{0.957698in}}{\pgfqpoint{2.124320in}{0.963521in}}%
\pgfpathcurveto{\pgfqpoint{2.130144in}{0.969345in}}{\pgfqpoint{2.133417in}{0.977245in}}{\pgfqpoint{2.133417in}{0.985482in}}%
\pgfpathcurveto{\pgfqpoint{2.133417in}{0.993718in}}{\pgfqpoint{2.130144in}{1.001618in}}{\pgfqpoint{2.124320in}{1.007442in}}%
\pgfpathcurveto{\pgfqpoint{2.118496in}{1.013266in}}{\pgfqpoint{2.110596in}{1.016538in}}{\pgfqpoint{2.102360in}{1.016538in}}%
\pgfpathcurveto{\pgfqpoint{2.094124in}{1.016538in}}{\pgfqpoint{2.086224in}{1.013266in}}{\pgfqpoint{2.080400in}{1.007442in}}%
\pgfpathcurveto{\pgfqpoint{2.074576in}{1.001618in}}{\pgfqpoint{2.071304in}{0.993718in}}{\pgfqpoint{2.071304in}{0.985482in}}%
\pgfpathcurveto{\pgfqpoint{2.071304in}{0.977245in}}{\pgfqpoint{2.074576in}{0.969345in}}{\pgfqpoint{2.080400in}{0.963521in}}%
\pgfpathcurveto{\pgfqpoint{2.086224in}{0.957698in}}{\pgfqpoint{2.094124in}{0.954425in}}{\pgfqpoint{2.102360in}{0.954425in}}%
\pgfpathclose%
\pgfusepath{stroke,fill}%
\end{pgfscope}%
\begin{pgfscope}%
\pgfpathrectangle{\pgfqpoint{0.100000in}{0.212622in}}{\pgfqpoint{3.696000in}{3.696000in}}%
\pgfusepath{clip}%
\pgfsetbuttcap%
\pgfsetroundjoin%
\definecolor{currentfill}{rgb}{0.121569,0.466667,0.705882}%
\pgfsetfillcolor{currentfill}%
\pgfsetfillopacity{0.931298}%
\pgfsetlinewidth{1.003750pt}%
\definecolor{currentstroke}{rgb}{0.121569,0.466667,0.705882}%
\pgfsetstrokecolor{currentstroke}%
\pgfsetstrokeopacity{0.931298}%
\pgfsetdash{}{0pt}%
\pgfpathmoveto{\pgfqpoint{2.109525in}{0.950839in}}%
\pgfpathcurveto{\pgfqpoint{2.117761in}{0.950839in}}{\pgfqpoint{2.125661in}{0.954111in}}{\pgfqpoint{2.131485in}{0.959935in}}%
\pgfpathcurveto{\pgfqpoint{2.137309in}{0.965759in}}{\pgfqpoint{2.140581in}{0.973659in}}{\pgfqpoint{2.140581in}{0.981895in}}%
\pgfpathcurveto{\pgfqpoint{2.140581in}{0.990131in}}{\pgfqpoint{2.137309in}{0.998032in}}{\pgfqpoint{2.131485in}{1.003855in}}%
\pgfpathcurveto{\pgfqpoint{2.125661in}{1.009679in}}{\pgfqpoint{2.117761in}{1.012952in}}{\pgfqpoint{2.109525in}{1.012952in}}%
\pgfpathcurveto{\pgfqpoint{2.101288in}{1.012952in}}{\pgfqpoint{2.093388in}{1.009679in}}{\pgfqpoint{2.087564in}{1.003855in}}%
\pgfpathcurveto{\pgfqpoint{2.081740in}{0.998032in}}{\pgfqpoint{2.078468in}{0.990131in}}{\pgfqpoint{2.078468in}{0.981895in}}%
\pgfpathcurveto{\pgfqpoint{2.078468in}{0.973659in}}{\pgfqpoint{2.081740in}{0.965759in}}{\pgfqpoint{2.087564in}{0.959935in}}%
\pgfpathcurveto{\pgfqpoint{2.093388in}{0.954111in}}{\pgfqpoint{2.101288in}{0.950839in}}{\pgfqpoint{2.109525in}{0.950839in}}%
\pgfpathclose%
\pgfusepath{stroke,fill}%
\end{pgfscope}%
\begin{pgfscope}%
\pgfpathrectangle{\pgfqpoint{0.100000in}{0.212622in}}{\pgfqpoint{3.696000in}{3.696000in}}%
\pgfusepath{clip}%
\pgfsetbuttcap%
\pgfsetroundjoin%
\definecolor{currentfill}{rgb}{0.121569,0.466667,0.705882}%
\pgfsetfillcolor{currentfill}%
\pgfsetfillopacity{0.933018}%
\pgfsetlinewidth{1.003750pt}%
\definecolor{currentstroke}{rgb}{0.121569,0.466667,0.705882}%
\pgfsetstrokecolor{currentstroke}%
\pgfsetstrokeopacity{0.933018}%
\pgfsetdash{}{0pt}%
\pgfpathmoveto{\pgfqpoint{2.390638in}{0.995816in}}%
\pgfpathcurveto{\pgfqpoint{2.398875in}{0.995816in}}{\pgfqpoint{2.406775in}{0.999088in}}{\pgfqpoint{2.412599in}{1.004912in}}%
\pgfpathcurveto{\pgfqpoint{2.418423in}{1.010736in}}{\pgfqpoint{2.421695in}{1.018636in}}{\pgfqpoint{2.421695in}{1.026872in}}%
\pgfpathcurveto{\pgfqpoint{2.421695in}{1.035108in}}{\pgfqpoint{2.418423in}{1.043008in}}{\pgfqpoint{2.412599in}{1.048832in}}%
\pgfpathcurveto{\pgfqpoint{2.406775in}{1.054656in}}{\pgfqpoint{2.398875in}{1.057929in}}{\pgfqpoint{2.390638in}{1.057929in}}%
\pgfpathcurveto{\pgfqpoint{2.382402in}{1.057929in}}{\pgfqpoint{2.374502in}{1.054656in}}{\pgfqpoint{2.368678in}{1.048832in}}%
\pgfpathcurveto{\pgfqpoint{2.362854in}{1.043008in}}{\pgfqpoint{2.359582in}{1.035108in}}{\pgfqpoint{2.359582in}{1.026872in}}%
\pgfpathcurveto{\pgfqpoint{2.359582in}{1.018636in}}{\pgfqpoint{2.362854in}{1.010736in}}{\pgfqpoint{2.368678in}{1.004912in}}%
\pgfpathcurveto{\pgfqpoint{2.374502in}{0.999088in}}{\pgfqpoint{2.382402in}{0.995816in}}{\pgfqpoint{2.390638in}{0.995816in}}%
\pgfpathclose%
\pgfusepath{stroke,fill}%
\end{pgfscope}%
\begin{pgfscope}%
\pgfpathrectangle{\pgfqpoint{0.100000in}{0.212622in}}{\pgfqpoint{3.696000in}{3.696000in}}%
\pgfusepath{clip}%
\pgfsetbuttcap%
\pgfsetroundjoin%
\definecolor{currentfill}{rgb}{0.121569,0.466667,0.705882}%
\pgfsetfillcolor{currentfill}%
\pgfsetfillopacity{0.934277}%
\pgfsetlinewidth{1.003750pt}%
\definecolor{currentstroke}{rgb}{0.121569,0.466667,0.705882}%
\pgfsetstrokecolor{currentstroke}%
\pgfsetstrokeopacity{0.934277}%
\pgfsetdash{}{0pt}%
\pgfpathmoveto{\pgfqpoint{2.122394in}{0.942617in}}%
\pgfpathcurveto{\pgfqpoint{2.130630in}{0.942617in}}{\pgfqpoint{2.138530in}{0.945890in}}{\pgfqpoint{2.144354in}{0.951714in}}%
\pgfpathcurveto{\pgfqpoint{2.150178in}{0.957538in}}{\pgfqpoint{2.153450in}{0.965438in}}{\pgfqpoint{2.153450in}{0.973674in}}%
\pgfpathcurveto{\pgfqpoint{2.153450in}{0.981910in}}{\pgfqpoint{2.150178in}{0.989810in}}{\pgfqpoint{2.144354in}{0.995634in}}%
\pgfpathcurveto{\pgfqpoint{2.138530in}{1.001458in}}{\pgfqpoint{2.130630in}{1.004730in}}{\pgfqpoint{2.122394in}{1.004730in}}%
\pgfpathcurveto{\pgfqpoint{2.114157in}{1.004730in}}{\pgfqpoint{2.106257in}{1.001458in}}{\pgfqpoint{2.100433in}{0.995634in}}%
\pgfpathcurveto{\pgfqpoint{2.094609in}{0.989810in}}{\pgfqpoint{2.091337in}{0.981910in}}{\pgfqpoint{2.091337in}{0.973674in}}%
\pgfpathcurveto{\pgfqpoint{2.091337in}{0.965438in}}{\pgfqpoint{2.094609in}{0.957538in}}{\pgfqpoint{2.100433in}{0.951714in}}%
\pgfpathcurveto{\pgfqpoint{2.106257in}{0.945890in}}{\pgfqpoint{2.114157in}{0.942617in}}{\pgfqpoint{2.122394in}{0.942617in}}%
\pgfpathclose%
\pgfusepath{stroke,fill}%
\end{pgfscope}%
\begin{pgfscope}%
\pgfpathrectangle{\pgfqpoint{0.100000in}{0.212622in}}{\pgfqpoint{3.696000in}{3.696000in}}%
\pgfusepath{clip}%
\pgfsetbuttcap%
\pgfsetroundjoin%
\definecolor{currentfill}{rgb}{0.121569,0.466667,0.705882}%
\pgfsetfillcolor{currentfill}%
\pgfsetfillopacity{0.937251}%
\pgfsetlinewidth{1.003750pt}%
\definecolor{currentstroke}{rgb}{0.121569,0.466667,0.705882}%
\pgfsetstrokecolor{currentstroke}%
\pgfsetstrokeopacity{0.937251}%
\pgfsetdash{}{0pt}%
\pgfpathmoveto{\pgfqpoint{2.134790in}{0.934476in}}%
\pgfpathcurveto{\pgfqpoint{2.143026in}{0.934476in}}{\pgfqpoint{2.150926in}{0.937749in}}{\pgfqpoint{2.156750in}{0.943572in}}%
\pgfpathcurveto{\pgfqpoint{2.162574in}{0.949396in}}{\pgfqpoint{2.165847in}{0.957296in}}{\pgfqpoint{2.165847in}{0.965533in}}%
\pgfpathcurveto{\pgfqpoint{2.165847in}{0.973769in}}{\pgfqpoint{2.162574in}{0.981669in}}{\pgfqpoint{2.156750in}{0.987493in}}%
\pgfpathcurveto{\pgfqpoint{2.150926in}{0.993317in}}{\pgfqpoint{2.143026in}{0.996589in}}{\pgfqpoint{2.134790in}{0.996589in}}%
\pgfpathcurveto{\pgfqpoint{2.126554in}{0.996589in}}{\pgfqpoint{2.118654in}{0.993317in}}{\pgfqpoint{2.112830in}{0.987493in}}%
\pgfpathcurveto{\pgfqpoint{2.107006in}{0.981669in}}{\pgfqpoint{2.103734in}{0.973769in}}{\pgfqpoint{2.103734in}{0.965533in}}%
\pgfpathcurveto{\pgfqpoint{2.103734in}{0.957296in}}{\pgfqpoint{2.107006in}{0.949396in}}{\pgfqpoint{2.112830in}{0.943572in}}%
\pgfpathcurveto{\pgfqpoint{2.118654in}{0.937749in}}{\pgfqpoint{2.126554in}{0.934476in}}{\pgfqpoint{2.134790in}{0.934476in}}%
\pgfpathclose%
\pgfusepath{stroke,fill}%
\end{pgfscope}%
\begin{pgfscope}%
\pgfpathrectangle{\pgfqpoint{0.100000in}{0.212622in}}{\pgfqpoint{3.696000in}{3.696000in}}%
\pgfusepath{clip}%
\pgfsetbuttcap%
\pgfsetroundjoin%
\definecolor{currentfill}{rgb}{0.121569,0.466667,0.705882}%
\pgfsetfillcolor{currentfill}%
\pgfsetfillopacity{0.938192}%
\pgfsetlinewidth{1.003750pt}%
\definecolor{currentstroke}{rgb}{0.121569,0.466667,0.705882}%
\pgfsetstrokecolor{currentstroke}%
\pgfsetstrokeopacity{0.938192}%
\pgfsetdash{}{0pt}%
\pgfpathmoveto{\pgfqpoint{2.395506in}{0.977147in}}%
\pgfpathcurveto{\pgfqpoint{2.403742in}{0.977147in}}{\pgfqpoint{2.411642in}{0.980419in}}{\pgfqpoint{2.417466in}{0.986243in}}%
\pgfpathcurveto{\pgfqpoint{2.423290in}{0.992067in}}{\pgfqpoint{2.426562in}{0.999967in}}{\pgfqpoint{2.426562in}{1.008203in}}%
\pgfpathcurveto{\pgfqpoint{2.426562in}{1.016440in}}{\pgfqpoint{2.423290in}{1.024340in}}{\pgfqpoint{2.417466in}{1.030164in}}%
\pgfpathcurveto{\pgfqpoint{2.411642in}{1.035987in}}{\pgfqpoint{2.403742in}{1.039260in}}{\pgfqpoint{2.395506in}{1.039260in}}%
\pgfpathcurveto{\pgfqpoint{2.387269in}{1.039260in}}{\pgfqpoint{2.379369in}{1.035987in}}{\pgfqpoint{2.373545in}{1.030164in}}%
\pgfpathcurveto{\pgfqpoint{2.367722in}{1.024340in}}{\pgfqpoint{2.364449in}{1.016440in}}{\pgfqpoint{2.364449in}{1.008203in}}%
\pgfpathcurveto{\pgfqpoint{2.364449in}{0.999967in}}{\pgfqpoint{2.367722in}{0.992067in}}{\pgfqpoint{2.373545in}{0.986243in}}%
\pgfpathcurveto{\pgfqpoint{2.379369in}{0.980419in}}{\pgfqpoint{2.387269in}{0.977147in}}{\pgfqpoint{2.395506in}{0.977147in}}%
\pgfpathclose%
\pgfusepath{stroke,fill}%
\end{pgfscope}%
\begin{pgfscope}%
\pgfpathrectangle{\pgfqpoint{0.100000in}{0.212622in}}{\pgfqpoint{3.696000in}{3.696000in}}%
\pgfusepath{clip}%
\pgfsetbuttcap%
\pgfsetroundjoin%
\definecolor{currentfill}{rgb}{0.121569,0.466667,0.705882}%
\pgfsetfillcolor{currentfill}%
\pgfsetfillopacity{0.939995}%
\pgfsetlinewidth{1.003750pt}%
\definecolor{currentstroke}{rgb}{0.121569,0.466667,0.705882}%
\pgfsetstrokecolor{currentstroke}%
\pgfsetstrokeopacity{0.939995}%
\pgfsetdash{}{0pt}%
\pgfpathmoveto{\pgfqpoint{2.146830in}{0.925935in}}%
\pgfpathcurveto{\pgfqpoint{2.155067in}{0.925935in}}{\pgfqpoint{2.162967in}{0.929208in}}{\pgfqpoint{2.168791in}{0.935032in}}%
\pgfpathcurveto{\pgfqpoint{2.174614in}{0.940856in}}{\pgfqpoint{2.177887in}{0.948756in}}{\pgfqpoint{2.177887in}{0.956992in}}%
\pgfpathcurveto{\pgfqpoint{2.177887in}{0.965228in}}{\pgfqpoint{2.174614in}{0.973128in}}{\pgfqpoint{2.168791in}{0.978952in}}%
\pgfpathcurveto{\pgfqpoint{2.162967in}{0.984776in}}{\pgfqpoint{2.155067in}{0.988048in}}{\pgfqpoint{2.146830in}{0.988048in}}%
\pgfpathcurveto{\pgfqpoint{2.138594in}{0.988048in}}{\pgfqpoint{2.130694in}{0.984776in}}{\pgfqpoint{2.124870in}{0.978952in}}%
\pgfpathcurveto{\pgfqpoint{2.119046in}{0.973128in}}{\pgfqpoint{2.115774in}{0.965228in}}{\pgfqpoint{2.115774in}{0.956992in}}%
\pgfpathcurveto{\pgfqpoint{2.115774in}{0.948756in}}{\pgfqpoint{2.119046in}{0.940856in}}{\pgfqpoint{2.124870in}{0.935032in}}%
\pgfpathcurveto{\pgfqpoint{2.130694in}{0.929208in}}{\pgfqpoint{2.138594in}{0.925935in}}{\pgfqpoint{2.146830in}{0.925935in}}%
\pgfpathclose%
\pgfusepath{stroke,fill}%
\end{pgfscope}%
\begin{pgfscope}%
\pgfpathrectangle{\pgfqpoint{0.100000in}{0.212622in}}{\pgfqpoint{3.696000in}{3.696000in}}%
\pgfusepath{clip}%
\pgfsetbuttcap%
\pgfsetroundjoin%
\definecolor{currentfill}{rgb}{0.121569,0.466667,0.705882}%
\pgfsetfillcolor{currentfill}%
\pgfsetfillopacity{0.942742}%
\pgfsetlinewidth{1.003750pt}%
\definecolor{currentstroke}{rgb}{0.121569,0.466667,0.705882}%
\pgfsetstrokecolor{currentstroke}%
\pgfsetstrokeopacity{0.942742}%
\pgfsetdash{}{0pt}%
\pgfpathmoveto{\pgfqpoint{2.158385in}{0.917603in}}%
\pgfpathcurveto{\pgfqpoint{2.166621in}{0.917603in}}{\pgfqpoint{2.174521in}{0.920876in}}{\pgfqpoint{2.180345in}{0.926700in}}%
\pgfpathcurveto{\pgfqpoint{2.186169in}{0.932524in}}{\pgfqpoint{2.189441in}{0.940424in}}{\pgfqpoint{2.189441in}{0.948660in}}%
\pgfpathcurveto{\pgfqpoint{2.189441in}{0.956896in}}{\pgfqpoint{2.186169in}{0.964796in}}{\pgfqpoint{2.180345in}{0.970620in}}%
\pgfpathcurveto{\pgfqpoint{2.174521in}{0.976444in}}{\pgfqpoint{2.166621in}{0.979716in}}{\pgfqpoint{2.158385in}{0.979716in}}%
\pgfpathcurveto{\pgfqpoint{2.150148in}{0.979716in}}{\pgfqpoint{2.142248in}{0.976444in}}{\pgfqpoint{2.136424in}{0.970620in}}%
\pgfpathcurveto{\pgfqpoint{2.130600in}{0.964796in}}{\pgfqpoint{2.127328in}{0.956896in}}{\pgfqpoint{2.127328in}{0.948660in}}%
\pgfpathcurveto{\pgfqpoint{2.127328in}{0.940424in}}{\pgfqpoint{2.130600in}{0.932524in}}{\pgfqpoint{2.136424in}{0.926700in}}%
\pgfpathcurveto{\pgfqpoint{2.142248in}{0.920876in}}{\pgfqpoint{2.150148in}{0.917603in}}{\pgfqpoint{2.158385in}{0.917603in}}%
\pgfpathclose%
\pgfusepath{stroke,fill}%
\end{pgfscope}%
\begin{pgfscope}%
\pgfpathrectangle{\pgfqpoint{0.100000in}{0.212622in}}{\pgfqpoint{3.696000in}{3.696000in}}%
\pgfusepath{clip}%
\pgfsetbuttcap%
\pgfsetroundjoin%
\definecolor{currentfill}{rgb}{0.121569,0.466667,0.705882}%
\pgfsetfillcolor{currentfill}%
\pgfsetfillopacity{0.943575}%
\pgfsetlinewidth{1.003750pt}%
\definecolor{currentstroke}{rgb}{0.121569,0.466667,0.705882}%
\pgfsetstrokecolor{currentstroke}%
\pgfsetstrokeopacity{0.943575}%
\pgfsetdash{}{0pt}%
\pgfpathmoveto{\pgfqpoint{2.400514in}{0.957689in}}%
\pgfpathcurveto{\pgfqpoint{2.408750in}{0.957689in}}{\pgfqpoint{2.416650in}{0.960961in}}{\pgfqpoint{2.422474in}{0.966785in}}%
\pgfpathcurveto{\pgfqpoint{2.428298in}{0.972609in}}{\pgfqpoint{2.431571in}{0.980509in}}{\pgfqpoint{2.431571in}{0.988745in}}%
\pgfpathcurveto{\pgfqpoint{2.431571in}{0.996982in}}{\pgfqpoint{2.428298in}{1.004882in}}{\pgfqpoint{2.422474in}{1.010706in}}%
\pgfpathcurveto{\pgfqpoint{2.416650in}{1.016529in}}{\pgfqpoint{2.408750in}{1.019802in}}{\pgfqpoint{2.400514in}{1.019802in}}%
\pgfpathcurveto{\pgfqpoint{2.392278in}{1.019802in}}{\pgfqpoint{2.384378in}{1.016529in}}{\pgfqpoint{2.378554in}{1.010706in}}%
\pgfpathcurveto{\pgfqpoint{2.372730in}{1.004882in}}{\pgfqpoint{2.369458in}{0.996982in}}{\pgfqpoint{2.369458in}{0.988745in}}%
\pgfpathcurveto{\pgfqpoint{2.369458in}{0.980509in}}{\pgfqpoint{2.372730in}{0.972609in}}{\pgfqpoint{2.378554in}{0.966785in}}%
\pgfpathcurveto{\pgfqpoint{2.384378in}{0.960961in}}{\pgfqpoint{2.392278in}{0.957689in}}{\pgfqpoint{2.400514in}{0.957689in}}%
\pgfpathclose%
\pgfusepath{stroke,fill}%
\end{pgfscope}%
\begin{pgfscope}%
\pgfpathrectangle{\pgfqpoint{0.100000in}{0.212622in}}{\pgfqpoint{3.696000in}{3.696000in}}%
\pgfusepath{clip}%
\pgfsetbuttcap%
\pgfsetroundjoin%
\definecolor{currentfill}{rgb}{0.121569,0.466667,0.705882}%
\pgfsetfillcolor{currentfill}%
\pgfsetfillopacity{0.945378}%
\pgfsetlinewidth{1.003750pt}%
\definecolor{currentstroke}{rgb}{0.121569,0.466667,0.705882}%
\pgfsetstrokecolor{currentstroke}%
\pgfsetstrokeopacity{0.945378}%
\pgfsetdash{}{0pt}%
\pgfpathmoveto{\pgfqpoint{2.169570in}{0.909508in}}%
\pgfpathcurveto{\pgfqpoint{2.177806in}{0.909508in}}{\pgfqpoint{2.185707in}{0.912781in}}{\pgfqpoint{2.191530in}{0.918605in}}%
\pgfpathcurveto{\pgfqpoint{2.197354in}{0.924429in}}{\pgfqpoint{2.200627in}{0.932329in}}{\pgfqpoint{2.200627in}{0.940565in}}%
\pgfpathcurveto{\pgfqpoint{2.200627in}{0.948801in}}{\pgfqpoint{2.197354in}{0.956701in}}{\pgfqpoint{2.191530in}{0.962525in}}%
\pgfpathcurveto{\pgfqpoint{2.185707in}{0.968349in}}{\pgfqpoint{2.177806in}{0.971621in}}{\pgfqpoint{2.169570in}{0.971621in}}%
\pgfpathcurveto{\pgfqpoint{2.161334in}{0.971621in}}{\pgfqpoint{2.153434in}{0.968349in}}{\pgfqpoint{2.147610in}{0.962525in}}%
\pgfpathcurveto{\pgfqpoint{2.141786in}{0.956701in}}{\pgfqpoint{2.138514in}{0.948801in}}{\pgfqpoint{2.138514in}{0.940565in}}%
\pgfpathcurveto{\pgfqpoint{2.138514in}{0.932329in}}{\pgfqpoint{2.141786in}{0.924429in}}{\pgfqpoint{2.147610in}{0.918605in}}%
\pgfpathcurveto{\pgfqpoint{2.153434in}{0.912781in}}{\pgfqpoint{2.161334in}{0.909508in}}{\pgfqpoint{2.169570in}{0.909508in}}%
\pgfpathclose%
\pgfusepath{stroke,fill}%
\end{pgfscope}%
\begin{pgfscope}%
\pgfpathrectangle{\pgfqpoint{0.100000in}{0.212622in}}{\pgfqpoint{3.696000in}{3.696000in}}%
\pgfusepath{clip}%
\pgfsetbuttcap%
\pgfsetroundjoin%
\definecolor{currentfill}{rgb}{0.121569,0.466667,0.705882}%
\pgfsetfillcolor{currentfill}%
\pgfsetfillopacity{0.947960}%
\pgfsetlinewidth{1.003750pt}%
\definecolor{currentstroke}{rgb}{0.121569,0.466667,0.705882}%
\pgfsetstrokecolor{currentstroke}%
\pgfsetstrokeopacity{0.947960}%
\pgfsetdash{}{0pt}%
\pgfpathmoveto{\pgfqpoint{2.180468in}{0.901937in}}%
\pgfpathcurveto{\pgfqpoint{2.188704in}{0.901937in}}{\pgfqpoint{2.196604in}{0.905210in}}{\pgfqpoint{2.202428in}{0.911034in}}%
\pgfpathcurveto{\pgfqpoint{2.208252in}{0.916857in}}{\pgfqpoint{2.211524in}{0.924757in}}{\pgfqpoint{2.211524in}{0.932994in}}%
\pgfpathcurveto{\pgfqpoint{2.211524in}{0.941230in}}{\pgfqpoint{2.208252in}{0.949130in}}{\pgfqpoint{2.202428in}{0.954954in}}%
\pgfpathcurveto{\pgfqpoint{2.196604in}{0.960778in}}{\pgfqpoint{2.188704in}{0.964050in}}{\pgfqpoint{2.180468in}{0.964050in}}%
\pgfpathcurveto{\pgfqpoint{2.172231in}{0.964050in}}{\pgfqpoint{2.164331in}{0.960778in}}{\pgfqpoint{2.158507in}{0.954954in}}%
\pgfpathcurveto{\pgfqpoint{2.152683in}{0.949130in}}{\pgfqpoint{2.149411in}{0.941230in}}{\pgfqpoint{2.149411in}{0.932994in}}%
\pgfpathcurveto{\pgfqpoint{2.149411in}{0.924757in}}{\pgfqpoint{2.152683in}{0.916857in}}{\pgfqpoint{2.158507in}{0.911034in}}%
\pgfpathcurveto{\pgfqpoint{2.164331in}{0.905210in}}{\pgfqpoint{2.172231in}{0.901937in}}{\pgfqpoint{2.180468in}{0.901937in}}%
\pgfpathclose%
\pgfusepath{stroke,fill}%
\end{pgfscope}%
\begin{pgfscope}%
\pgfpathrectangle{\pgfqpoint{0.100000in}{0.212622in}}{\pgfqpoint{3.696000in}{3.696000in}}%
\pgfusepath{clip}%
\pgfsetbuttcap%
\pgfsetroundjoin%
\definecolor{currentfill}{rgb}{0.121569,0.466667,0.705882}%
\pgfsetfillcolor{currentfill}%
\pgfsetfillopacity{0.949340}%
\pgfsetlinewidth{1.003750pt}%
\definecolor{currentstroke}{rgb}{0.121569,0.466667,0.705882}%
\pgfsetstrokecolor{currentstroke}%
\pgfsetstrokeopacity{0.949340}%
\pgfsetdash{}{0pt}%
\pgfpathmoveto{\pgfqpoint{2.405615in}{0.937573in}}%
\pgfpathcurveto{\pgfqpoint{2.413851in}{0.937573in}}{\pgfqpoint{2.421751in}{0.940846in}}{\pgfqpoint{2.427575in}{0.946669in}}%
\pgfpathcurveto{\pgfqpoint{2.433399in}{0.952493in}}{\pgfqpoint{2.436671in}{0.960393in}}{\pgfqpoint{2.436671in}{0.968630in}}%
\pgfpathcurveto{\pgfqpoint{2.436671in}{0.976866in}}{\pgfqpoint{2.433399in}{0.984766in}}{\pgfqpoint{2.427575in}{0.990590in}}%
\pgfpathcurveto{\pgfqpoint{2.421751in}{0.996414in}}{\pgfqpoint{2.413851in}{0.999686in}}{\pgfqpoint{2.405615in}{0.999686in}}%
\pgfpathcurveto{\pgfqpoint{2.397378in}{0.999686in}}{\pgfqpoint{2.389478in}{0.996414in}}{\pgfqpoint{2.383654in}{0.990590in}}%
\pgfpathcurveto{\pgfqpoint{2.377831in}{0.984766in}}{\pgfqpoint{2.374558in}{0.976866in}}{\pgfqpoint{2.374558in}{0.968630in}}%
\pgfpathcurveto{\pgfqpoint{2.374558in}{0.960393in}}{\pgfqpoint{2.377831in}{0.952493in}}{\pgfqpoint{2.383654in}{0.946669in}}%
\pgfpathcurveto{\pgfqpoint{2.389478in}{0.940846in}}{\pgfqpoint{2.397378in}{0.937573in}}{\pgfqpoint{2.405615in}{0.937573in}}%
\pgfpathclose%
\pgfusepath{stroke,fill}%
\end{pgfscope}%
\begin{pgfscope}%
\pgfpathrectangle{\pgfqpoint{0.100000in}{0.212622in}}{\pgfqpoint{3.696000in}{3.696000in}}%
\pgfusepath{clip}%
\pgfsetbuttcap%
\pgfsetroundjoin%
\definecolor{currentfill}{rgb}{0.121569,0.466667,0.705882}%
\pgfsetfillcolor{currentfill}%
\pgfsetfillopacity{0.950449}%
\pgfsetlinewidth{1.003750pt}%
\definecolor{currentstroke}{rgb}{0.121569,0.466667,0.705882}%
\pgfsetstrokecolor{currentstroke}%
\pgfsetstrokeopacity{0.950449}%
\pgfsetdash{}{0pt}%
\pgfpathmoveto{\pgfqpoint{2.191084in}{0.894759in}}%
\pgfpathcurveto{\pgfqpoint{2.199320in}{0.894759in}}{\pgfqpoint{2.207220in}{0.898032in}}{\pgfqpoint{2.213044in}{0.903855in}}%
\pgfpathcurveto{\pgfqpoint{2.218868in}{0.909679in}}{\pgfqpoint{2.222141in}{0.917579in}}{\pgfqpoint{2.222141in}{0.925816in}}%
\pgfpathcurveto{\pgfqpoint{2.222141in}{0.934052in}}{\pgfqpoint{2.218868in}{0.941952in}}{\pgfqpoint{2.213044in}{0.947776in}}%
\pgfpathcurveto{\pgfqpoint{2.207220in}{0.953600in}}{\pgfqpoint{2.199320in}{0.956872in}}{\pgfqpoint{2.191084in}{0.956872in}}%
\pgfpathcurveto{\pgfqpoint{2.182848in}{0.956872in}}{\pgfqpoint{2.174948in}{0.953600in}}{\pgfqpoint{2.169124in}{0.947776in}}%
\pgfpathcurveto{\pgfqpoint{2.163300in}{0.941952in}}{\pgfqpoint{2.160028in}{0.934052in}}{\pgfqpoint{2.160028in}{0.925816in}}%
\pgfpathcurveto{\pgfqpoint{2.160028in}{0.917579in}}{\pgfqpoint{2.163300in}{0.909679in}}{\pgfqpoint{2.169124in}{0.903855in}}%
\pgfpathcurveto{\pgfqpoint{2.174948in}{0.898032in}}{\pgfqpoint{2.182848in}{0.894759in}}{\pgfqpoint{2.191084in}{0.894759in}}%
\pgfpathclose%
\pgfusepath{stroke,fill}%
\end{pgfscope}%
\begin{pgfscope}%
\pgfpathrectangle{\pgfqpoint{0.100000in}{0.212622in}}{\pgfqpoint{3.696000in}{3.696000in}}%
\pgfusepath{clip}%
\pgfsetbuttcap%
\pgfsetroundjoin%
\definecolor{currentfill}{rgb}{0.121569,0.466667,0.705882}%
\pgfsetfillcolor{currentfill}%
\pgfsetfillopacity{0.952819}%
\pgfsetlinewidth{1.003750pt}%
\definecolor{currentstroke}{rgb}{0.121569,0.466667,0.705882}%
\pgfsetstrokecolor{currentstroke}%
\pgfsetstrokeopacity{0.952819}%
\pgfsetdash{}{0pt}%
\pgfpathmoveto{\pgfqpoint{2.201416in}{0.887891in}}%
\pgfpathcurveto{\pgfqpoint{2.209652in}{0.887891in}}{\pgfqpoint{2.217552in}{0.891163in}}{\pgfqpoint{2.223376in}{0.896987in}}%
\pgfpathcurveto{\pgfqpoint{2.229200in}{0.902811in}}{\pgfqpoint{2.232472in}{0.910711in}}{\pgfqpoint{2.232472in}{0.918948in}}%
\pgfpathcurveto{\pgfqpoint{2.232472in}{0.927184in}}{\pgfqpoint{2.229200in}{0.935084in}}{\pgfqpoint{2.223376in}{0.940908in}}%
\pgfpathcurveto{\pgfqpoint{2.217552in}{0.946732in}}{\pgfqpoint{2.209652in}{0.950004in}}{\pgfqpoint{2.201416in}{0.950004in}}%
\pgfpathcurveto{\pgfqpoint{2.193180in}{0.950004in}}{\pgfqpoint{2.185280in}{0.946732in}}{\pgfqpoint{2.179456in}{0.940908in}}%
\pgfpathcurveto{\pgfqpoint{2.173632in}{0.935084in}}{\pgfqpoint{2.170359in}{0.927184in}}{\pgfqpoint{2.170359in}{0.918948in}}%
\pgfpathcurveto{\pgfqpoint{2.170359in}{0.910711in}}{\pgfqpoint{2.173632in}{0.902811in}}{\pgfqpoint{2.179456in}{0.896987in}}%
\pgfpathcurveto{\pgfqpoint{2.185280in}{0.891163in}}{\pgfqpoint{2.193180in}{0.887891in}}{\pgfqpoint{2.201416in}{0.887891in}}%
\pgfpathclose%
\pgfusepath{stroke,fill}%
\end{pgfscope}%
\begin{pgfscope}%
\pgfpathrectangle{\pgfqpoint{0.100000in}{0.212622in}}{\pgfqpoint{3.696000in}{3.696000in}}%
\pgfusepath{clip}%
\pgfsetbuttcap%
\pgfsetroundjoin%
\definecolor{currentfill}{rgb}{0.121569,0.466667,0.705882}%
\pgfsetfillcolor{currentfill}%
\pgfsetfillopacity{0.955120}%
\pgfsetlinewidth{1.003750pt}%
\definecolor{currentstroke}{rgb}{0.121569,0.466667,0.705882}%
\pgfsetstrokecolor{currentstroke}%
\pgfsetstrokeopacity{0.955120}%
\pgfsetdash{}{0pt}%
\pgfpathmoveto{\pgfqpoint{2.211360in}{0.881231in}}%
\pgfpathcurveto{\pgfqpoint{2.219596in}{0.881231in}}{\pgfqpoint{2.227496in}{0.884503in}}{\pgfqpoint{2.233320in}{0.890327in}}%
\pgfpathcurveto{\pgfqpoint{2.239144in}{0.896151in}}{\pgfqpoint{2.242416in}{0.904051in}}{\pgfqpoint{2.242416in}{0.912287in}}%
\pgfpathcurveto{\pgfqpoint{2.242416in}{0.920523in}}{\pgfqpoint{2.239144in}{0.928423in}}{\pgfqpoint{2.233320in}{0.934247in}}%
\pgfpathcurveto{\pgfqpoint{2.227496in}{0.940071in}}{\pgfqpoint{2.219596in}{0.943344in}}{\pgfqpoint{2.211360in}{0.943344in}}%
\pgfpathcurveto{\pgfqpoint{2.203123in}{0.943344in}}{\pgfqpoint{2.195223in}{0.940071in}}{\pgfqpoint{2.189400in}{0.934247in}}%
\pgfpathcurveto{\pgfqpoint{2.183576in}{0.928423in}}{\pgfqpoint{2.180303in}{0.920523in}}{\pgfqpoint{2.180303in}{0.912287in}}%
\pgfpathcurveto{\pgfqpoint{2.180303in}{0.904051in}}{\pgfqpoint{2.183576in}{0.896151in}}{\pgfqpoint{2.189400in}{0.890327in}}%
\pgfpathcurveto{\pgfqpoint{2.195223in}{0.884503in}}{\pgfqpoint{2.203123in}{0.881231in}}{\pgfqpoint{2.211360in}{0.881231in}}%
\pgfpathclose%
\pgfusepath{stroke,fill}%
\end{pgfscope}%
\begin{pgfscope}%
\pgfpathrectangle{\pgfqpoint{0.100000in}{0.212622in}}{\pgfqpoint{3.696000in}{3.696000in}}%
\pgfusepath{clip}%
\pgfsetbuttcap%
\pgfsetroundjoin%
\definecolor{currentfill}{rgb}{0.121569,0.466667,0.705882}%
\pgfsetfillcolor{currentfill}%
\pgfsetfillopacity{0.955237}%
\pgfsetlinewidth{1.003750pt}%
\definecolor{currentstroke}{rgb}{0.121569,0.466667,0.705882}%
\pgfsetstrokecolor{currentstroke}%
\pgfsetstrokeopacity{0.955237}%
\pgfsetdash{}{0pt}%
\pgfpathmoveto{\pgfqpoint{2.410885in}{0.915716in}}%
\pgfpathcurveto{\pgfqpoint{2.419121in}{0.915716in}}{\pgfqpoint{2.427021in}{0.918988in}}{\pgfqpoint{2.432845in}{0.924812in}}%
\pgfpathcurveto{\pgfqpoint{2.438669in}{0.930636in}}{\pgfqpoint{2.441941in}{0.938536in}}{\pgfqpoint{2.441941in}{0.946773in}}%
\pgfpathcurveto{\pgfqpoint{2.441941in}{0.955009in}}{\pgfqpoint{2.438669in}{0.962909in}}{\pgfqpoint{2.432845in}{0.968733in}}%
\pgfpathcurveto{\pgfqpoint{2.427021in}{0.974557in}}{\pgfqpoint{2.419121in}{0.977829in}}{\pgfqpoint{2.410885in}{0.977829in}}%
\pgfpathcurveto{\pgfqpoint{2.402648in}{0.977829in}}{\pgfqpoint{2.394748in}{0.974557in}}{\pgfqpoint{2.388924in}{0.968733in}}%
\pgfpathcurveto{\pgfqpoint{2.383101in}{0.962909in}}{\pgfqpoint{2.379828in}{0.955009in}}{\pgfqpoint{2.379828in}{0.946773in}}%
\pgfpathcurveto{\pgfqpoint{2.379828in}{0.938536in}}{\pgfqpoint{2.383101in}{0.930636in}}{\pgfqpoint{2.388924in}{0.924812in}}%
\pgfpathcurveto{\pgfqpoint{2.394748in}{0.918988in}}{\pgfqpoint{2.402648in}{0.915716in}}{\pgfqpoint{2.410885in}{0.915716in}}%
\pgfpathclose%
\pgfusepath{stroke,fill}%
\end{pgfscope}%
\begin{pgfscope}%
\pgfpathrectangle{\pgfqpoint{0.100000in}{0.212622in}}{\pgfqpoint{3.696000in}{3.696000in}}%
\pgfusepath{clip}%
\pgfsetbuttcap%
\pgfsetroundjoin%
\definecolor{currentfill}{rgb}{0.121569,0.466667,0.705882}%
\pgfsetfillcolor{currentfill}%
\pgfsetfillopacity{0.957397}%
\pgfsetlinewidth{1.003750pt}%
\definecolor{currentstroke}{rgb}{0.121569,0.466667,0.705882}%
\pgfsetstrokecolor{currentstroke}%
\pgfsetstrokeopacity{0.957397}%
\pgfsetdash{}{0pt}%
\pgfpathmoveto{\pgfqpoint{2.220864in}{0.874683in}}%
\pgfpathcurveto{\pgfqpoint{2.229100in}{0.874683in}}{\pgfqpoint{2.237000in}{0.877955in}}{\pgfqpoint{2.242824in}{0.883779in}}%
\pgfpathcurveto{\pgfqpoint{2.248648in}{0.889603in}}{\pgfqpoint{2.251920in}{0.897503in}}{\pgfqpoint{2.251920in}{0.905740in}}%
\pgfpathcurveto{\pgfqpoint{2.251920in}{0.913976in}}{\pgfqpoint{2.248648in}{0.921876in}}{\pgfqpoint{2.242824in}{0.927700in}}%
\pgfpathcurveto{\pgfqpoint{2.237000in}{0.933524in}}{\pgfqpoint{2.229100in}{0.936796in}}{\pgfqpoint{2.220864in}{0.936796in}}%
\pgfpathcurveto{\pgfqpoint{2.212628in}{0.936796in}}{\pgfqpoint{2.204728in}{0.933524in}}{\pgfqpoint{2.198904in}{0.927700in}}%
\pgfpathcurveto{\pgfqpoint{2.193080in}{0.921876in}}{\pgfqpoint{2.189807in}{0.913976in}}{\pgfqpoint{2.189807in}{0.905740in}}%
\pgfpathcurveto{\pgfqpoint{2.189807in}{0.897503in}}{\pgfqpoint{2.193080in}{0.889603in}}{\pgfqpoint{2.198904in}{0.883779in}}%
\pgfpathcurveto{\pgfqpoint{2.204728in}{0.877955in}}{\pgfqpoint{2.212628in}{0.874683in}}{\pgfqpoint{2.220864in}{0.874683in}}%
\pgfpathclose%
\pgfusepath{stroke,fill}%
\end{pgfscope}%
\begin{pgfscope}%
\pgfpathrectangle{\pgfqpoint{0.100000in}{0.212622in}}{\pgfqpoint{3.696000in}{3.696000in}}%
\pgfusepath{clip}%
\pgfsetbuttcap%
\pgfsetroundjoin%
\definecolor{currentfill}{rgb}{0.121569,0.466667,0.705882}%
\pgfsetfillcolor{currentfill}%
\pgfsetfillopacity{0.959502}%
\pgfsetlinewidth{1.003750pt}%
\definecolor{currentstroke}{rgb}{0.121569,0.466667,0.705882}%
\pgfsetstrokecolor{currentstroke}%
\pgfsetstrokeopacity{0.959502}%
\pgfsetdash{}{0pt}%
\pgfpathmoveto{\pgfqpoint{2.230117in}{0.868191in}}%
\pgfpathcurveto{\pgfqpoint{2.238353in}{0.868191in}}{\pgfqpoint{2.246253in}{0.871464in}}{\pgfqpoint{2.252077in}{0.877288in}}%
\pgfpathcurveto{\pgfqpoint{2.257901in}{0.883111in}}{\pgfqpoint{2.261173in}{0.891012in}}{\pgfqpoint{2.261173in}{0.899248in}}%
\pgfpathcurveto{\pgfqpoint{2.261173in}{0.907484in}}{\pgfqpoint{2.257901in}{0.915384in}}{\pgfqpoint{2.252077in}{0.921208in}}%
\pgfpathcurveto{\pgfqpoint{2.246253in}{0.927032in}}{\pgfqpoint{2.238353in}{0.930304in}}{\pgfqpoint{2.230117in}{0.930304in}}%
\pgfpathcurveto{\pgfqpoint{2.221880in}{0.930304in}}{\pgfqpoint{2.213980in}{0.927032in}}{\pgfqpoint{2.208157in}{0.921208in}}%
\pgfpathcurveto{\pgfqpoint{2.202333in}{0.915384in}}{\pgfqpoint{2.199060in}{0.907484in}}{\pgfqpoint{2.199060in}{0.899248in}}%
\pgfpathcurveto{\pgfqpoint{2.199060in}{0.891012in}}{\pgfqpoint{2.202333in}{0.883111in}}{\pgfqpoint{2.208157in}{0.877288in}}%
\pgfpathcurveto{\pgfqpoint{2.213980in}{0.871464in}}{\pgfqpoint{2.221880in}{0.868191in}}{\pgfqpoint{2.230117in}{0.868191in}}%
\pgfpathclose%
\pgfusepath{stroke,fill}%
\end{pgfscope}%
\begin{pgfscope}%
\pgfpathrectangle{\pgfqpoint{0.100000in}{0.212622in}}{\pgfqpoint{3.696000in}{3.696000in}}%
\pgfusepath{clip}%
\pgfsetbuttcap%
\pgfsetroundjoin%
\definecolor{currentfill}{rgb}{0.121569,0.466667,0.705882}%
\pgfsetfillcolor{currentfill}%
\pgfsetfillopacity{0.961351}%
\pgfsetlinewidth{1.003750pt}%
\definecolor{currentstroke}{rgb}{0.121569,0.466667,0.705882}%
\pgfsetstrokecolor{currentstroke}%
\pgfsetstrokeopacity{0.961351}%
\pgfsetdash{}{0pt}%
\pgfpathmoveto{\pgfqpoint{2.416615in}{0.893089in}}%
\pgfpathcurveto{\pgfqpoint{2.424851in}{0.893089in}}{\pgfqpoint{2.432751in}{0.896361in}}{\pgfqpoint{2.438575in}{0.902185in}}%
\pgfpathcurveto{\pgfqpoint{2.444399in}{0.908009in}}{\pgfqpoint{2.447671in}{0.915909in}}{\pgfqpoint{2.447671in}{0.924145in}}%
\pgfpathcurveto{\pgfqpoint{2.447671in}{0.932381in}}{\pgfqpoint{2.444399in}{0.940281in}}{\pgfqpoint{2.438575in}{0.946105in}}%
\pgfpathcurveto{\pgfqpoint{2.432751in}{0.951929in}}{\pgfqpoint{2.424851in}{0.955202in}}{\pgfqpoint{2.416615in}{0.955202in}}%
\pgfpathcurveto{\pgfqpoint{2.408378in}{0.955202in}}{\pgfqpoint{2.400478in}{0.951929in}}{\pgfqpoint{2.394654in}{0.946105in}}%
\pgfpathcurveto{\pgfqpoint{2.388831in}{0.940281in}}{\pgfqpoint{2.385558in}{0.932381in}}{\pgfqpoint{2.385558in}{0.924145in}}%
\pgfpathcurveto{\pgfqpoint{2.385558in}{0.915909in}}{\pgfqpoint{2.388831in}{0.908009in}}{\pgfqpoint{2.394654in}{0.902185in}}%
\pgfpathcurveto{\pgfqpoint{2.400478in}{0.896361in}}{\pgfqpoint{2.408378in}{0.893089in}}{\pgfqpoint{2.416615in}{0.893089in}}%
\pgfpathclose%
\pgfusepath{stroke,fill}%
\end{pgfscope}%
\begin{pgfscope}%
\pgfpathrectangle{\pgfqpoint{0.100000in}{0.212622in}}{\pgfqpoint{3.696000in}{3.696000in}}%
\pgfusepath{clip}%
\pgfsetbuttcap%
\pgfsetroundjoin%
\definecolor{currentfill}{rgb}{0.121569,0.466667,0.705882}%
\pgfsetfillcolor{currentfill}%
\pgfsetfillopacity{0.961595}%
\pgfsetlinewidth{1.003750pt}%
\definecolor{currentstroke}{rgb}{0.121569,0.466667,0.705882}%
\pgfsetstrokecolor{currentstroke}%
\pgfsetstrokeopacity{0.961595}%
\pgfsetdash{}{0pt}%
\pgfpathmoveto{\pgfqpoint{2.239032in}{0.862224in}}%
\pgfpathcurveto{\pgfqpoint{2.247268in}{0.862224in}}{\pgfqpoint{2.255168in}{0.865497in}}{\pgfqpoint{2.260992in}{0.871320in}}%
\pgfpathcurveto{\pgfqpoint{2.266816in}{0.877144in}}{\pgfqpoint{2.270089in}{0.885044in}}{\pgfqpoint{2.270089in}{0.893281in}}%
\pgfpathcurveto{\pgfqpoint{2.270089in}{0.901517in}}{\pgfqpoint{2.266816in}{0.909417in}}{\pgfqpoint{2.260992in}{0.915241in}}%
\pgfpathcurveto{\pgfqpoint{2.255168in}{0.921065in}}{\pgfqpoint{2.247268in}{0.924337in}}{\pgfqpoint{2.239032in}{0.924337in}}%
\pgfpathcurveto{\pgfqpoint{2.230796in}{0.924337in}}{\pgfqpoint{2.222896in}{0.921065in}}{\pgfqpoint{2.217072in}{0.915241in}}%
\pgfpathcurveto{\pgfqpoint{2.211248in}{0.909417in}}{\pgfqpoint{2.207976in}{0.901517in}}{\pgfqpoint{2.207976in}{0.893281in}}%
\pgfpathcurveto{\pgfqpoint{2.207976in}{0.885044in}}{\pgfqpoint{2.211248in}{0.877144in}}{\pgfqpoint{2.217072in}{0.871320in}}%
\pgfpathcurveto{\pgfqpoint{2.222896in}{0.865497in}}{\pgfqpoint{2.230796in}{0.862224in}}{\pgfqpoint{2.239032in}{0.862224in}}%
\pgfpathclose%
\pgfusepath{stroke,fill}%
\end{pgfscope}%
\begin{pgfscope}%
\pgfpathrectangle{\pgfqpoint{0.100000in}{0.212622in}}{\pgfqpoint{3.696000in}{3.696000in}}%
\pgfusepath{clip}%
\pgfsetbuttcap%
\pgfsetroundjoin%
\definecolor{currentfill}{rgb}{0.121569,0.466667,0.705882}%
\pgfsetfillcolor{currentfill}%
\pgfsetfillopacity{0.963553}%
\pgfsetlinewidth{1.003750pt}%
\definecolor{currentstroke}{rgb}{0.121569,0.466667,0.705882}%
\pgfsetstrokecolor{currentstroke}%
\pgfsetstrokeopacity{0.963553}%
\pgfsetdash{}{0pt}%
\pgfpathmoveto{\pgfqpoint{2.247629in}{0.856445in}}%
\pgfpathcurveto{\pgfqpoint{2.255865in}{0.856445in}}{\pgfqpoint{2.263765in}{0.859717in}}{\pgfqpoint{2.269589in}{0.865541in}}%
\pgfpathcurveto{\pgfqpoint{2.275413in}{0.871365in}}{\pgfqpoint{2.278685in}{0.879265in}}{\pgfqpoint{2.278685in}{0.887501in}}%
\pgfpathcurveto{\pgfqpoint{2.278685in}{0.895738in}}{\pgfqpoint{2.275413in}{0.903638in}}{\pgfqpoint{2.269589in}{0.909462in}}%
\pgfpathcurveto{\pgfqpoint{2.263765in}{0.915286in}}{\pgfqpoint{2.255865in}{0.918558in}}{\pgfqpoint{2.247629in}{0.918558in}}%
\pgfpathcurveto{\pgfqpoint{2.239392in}{0.918558in}}{\pgfqpoint{2.231492in}{0.915286in}}{\pgfqpoint{2.225668in}{0.909462in}}%
\pgfpathcurveto{\pgfqpoint{2.219844in}{0.903638in}}{\pgfqpoint{2.216572in}{0.895738in}}{\pgfqpoint{2.216572in}{0.887501in}}%
\pgfpathcurveto{\pgfqpoint{2.216572in}{0.879265in}}{\pgfqpoint{2.219844in}{0.871365in}}{\pgfqpoint{2.225668in}{0.865541in}}%
\pgfpathcurveto{\pgfqpoint{2.231492in}{0.859717in}}{\pgfqpoint{2.239392in}{0.856445in}}{\pgfqpoint{2.247629in}{0.856445in}}%
\pgfpathclose%
\pgfusepath{stroke,fill}%
\end{pgfscope}%
\begin{pgfscope}%
\pgfpathrectangle{\pgfqpoint{0.100000in}{0.212622in}}{\pgfqpoint{3.696000in}{3.696000in}}%
\pgfusepath{clip}%
\pgfsetbuttcap%
\pgfsetroundjoin%
\definecolor{currentfill}{rgb}{0.121569,0.466667,0.705882}%
\pgfsetfillcolor{currentfill}%
\pgfsetfillopacity{0.965405}%
\pgfsetlinewidth{1.003750pt}%
\definecolor{currentstroke}{rgb}{0.121569,0.466667,0.705882}%
\pgfsetstrokecolor{currentstroke}%
\pgfsetstrokeopacity{0.965405}%
\pgfsetdash{}{0pt}%
\pgfpathmoveto{\pgfqpoint{2.255915in}{0.850802in}}%
\pgfpathcurveto{\pgfqpoint{2.264152in}{0.850802in}}{\pgfqpoint{2.272052in}{0.854075in}}{\pgfqpoint{2.277876in}{0.859898in}}%
\pgfpathcurveto{\pgfqpoint{2.283700in}{0.865722in}}{\pgfqpoint{2.286972in}{0.873622in}}{\pgfqpoint{2.286972in}{0.881859in}}%
\pgfpathcurveto{\pgfqpoint{2.286972in}{0.890095in}}{\pgfqpoint{2.283700in}{0.897995in}}{\pgfqpoint{2.277876in}{0.903819in}}%
\pgfpathcurveto{\pgfqpoint{2.272052in}{0.909643in}}{\pgfqpoint{2.264152in}{0.912915in}}{\pgfqpoint{2.255915in}{0.912915in}}%
\pgfpathcurveto{\pgfqpoint{2.247679in}{0.912915in}}{\pgfqpoint{2.239779in}{0.909643in}}{\pgfqpoint{2.233955in}{0.903819in}}%
\pgfpathcurveto{\pgfqpoint{2.228131in}{0.897995in}}{\pgfqpoint{2.224859in}{0.890095in}}{\pgfqpoint{2.224859in}{0.881859in}}%
\pgfpathcurveto{\pgfqpoint{2.224859in}{0.873622in}}{\pgfqpoint{2.228131in}{0.865722in}}{\pgfqpoint{2.233955in}{0.859898in}}%
\pgfpathcurveto{\pgfqpoint{2.239779in}{0.854075in}}{\pgfqpoint{2.247679in}{0.850802in}}{\pgfqpoint{2.255915in}{0.850802in}}%
\pgfpathclose%
\pgfusepath{stroke,fill}%
\end{pgfscope}%
\begin{pgfscope}%
\pgfpathrectangle{\pgfqpoint{0.100000in}{0.212622in}}{\pgfqpoint{3.696000in}{3.696000in}}%
\pgfusepath{clip}%
\pgfsetbuttcap%
\pgfsetroundjoin%
\definecolor{currentfill}{rgb}{0.121569,0.466667,0.705882}%
\pgfsetfillcolor{currentfill}%
\pgfsetfillopacity{0.967149}%
\pgfsetlinewidth{1.003750pt}%
\definecolor{currentstroke}{rgb}{0.121569,0.466667,0.705882}%
\pgfsetstrokecolor{currentstroke}%
\pgfsetstrokeopacity{0.967149}%
\pgfsetdash{}{0pt}%
\pgfpathmoveto{\pgfqpoint{2.263928in}{0.845359in}}%
\pgfpathcurveto{\pgfqpoint{2.272165in}{0.845359in}}{\pgfqpoint{2.280065in}{0.848631in}}{\pgfqpoint{2.285889in}{0.854455in}}%
\pgfpathcurveto{\pgfqpoint{2.291713in}{0.860279in}}{\pgfqpoint{2.294985in}{0.868179in}}{\pgfqpoint{2.294985in}{0.876416in}}%
\pgfpathcurveto{\pgfqpoint{2.294985in}{0.884652in}}{\pgfqpoint{2.291713in}{0.892552in}}{\pgfqpoint{2.285889in}{0.898376in}}%
\pgfpathcurveto{\pgfqpoint{2.280065in}{0.904200in}}{\pgfqpoint{2.272165in}{0.907472in}}{\pgfqpoint{2.263928in}{0.907472in}}%
\pgfpathcurveto{\pgfqpoint{2.255692in}{0.907472in}}{\pgfqpoint{2.247792in}{0.904200in}}{\pgfqpoint{2.241968in}{0.898376in}}%
\pgfpathcurveto{\pgfqpoint{2.236144in}{0.892552in}}{\pgfqpoint{2.232872in}{0.884652in}}{\pgfqpoint{2.232872in}{0.876416in}}%
\pgfpathcurveto{\pgfqpoint{2.232872in}{0.868179in}}{\pgfqpoint{2.236144in}{0.860279in}}{\pgfqpoint{2.241968in}{0.854455in}}%
\pgfpathcurveto{\pgfqpoint{2.247792in}{0.848631in}}{\pgfqpoint{2.255692in}{0.845359in}}{\pgfqpoint{2.263928in}{0.845359in}}%
\pgfpathclose%
\pgfusepath{stroke,fill}%
\end{pgfscope}%
\begin{pgfscope}%
\pgfpathrectangle{\pgfqpoint{0.100000in}{0.212622in}}{\pgfqpoint{3.696000in}{3.696000in}}%
\pgfusepath{clip}%
\pgfsetbuttcap%
\pgfsetroundjoin%
\definecolor{currentfill}{rgb}{0.121569,0.466667,0.705882}%
\pgfsetfillcolor{currentfill}%
\pgfsetfillopacity{0.967716}%
\pgfsetlinewidth{1.003750pt}%
\definecolor{currentstroke}{rgb}{0.121569,0.466667,0.705882}%
\pgfsetstrokecolor{currentstroke}%
\pgfsetstrokeopacity{0.967716}%
\pgfsetdash{}{0pt}%
\pgfpathmoveto{\pgfqpoint{2.422557in}{0.869581in}}%
\pgfpathcurveto{\pgfqpoint{2.430794in}{0.869581in}}{\pgfqpoint{2.438694in}{0.872853in}}{\pgfqpoint{2.444518in}{0.878677in}}%
\pgfpathcurveto{\pgfqpoint{2.450341in}{0.884501in}}{\pgfqpoint{2.453614in}{0.892401in}}{\pgfqpoint{2.453614in}{0.900637in}}%
\pgfpathcurveto{\pgfqpoint{2.453614in}{0.908873in}}{\pgfqpoint{2.450341in}{0.916773in}}{\pgfqpoint{2.444518in}{0.922597in}}%
\pgfpathcurveto{\pgfqpoint{2.438694in}{0.928421in}}{\pgfqpoint{2.430794in}{0.931694in}}{\pgfqpoint{2.422557in}{0.931694in}}%
\pgfpathcurveto{\pgfqpoint{2.414321in}{0.931694in}}{\pgfqpoint{2.406421in}{0.928421in}}{\pgfqpoint{2.400597in}{0.922597in}}%
\pgfpathcurveto{\pgfqpoint{2.394773in}{0.916773in}}{\pgfqpoint{2.391501in}{0.908873in}}{\pgfqpoint{2.391501in}{0.900637in}}%
\pgfpathcurveto{\pgfqpoint{2.391501in}{0.892401in}}{\pgfqpoint{2.394773in}{0.884501in}}{\pgfqpoint{2.400597in}{0.878677in}}%
\pgfpathcurveto{\pgfqpoint{2.406421in}{0.872853in}}{\pgfqpoint{2.414321in}{0.869581in}}{\pgfqpoint{2.422557in}{0.869581in}}%
\pgfpathclose%
\pgfusepath{stroke,fill}%
\end{pgfscope}%
\begin{pgfscope}%
\pgfpathrectangle{\pgfqpoint{0.100000in}{0.212622in}}{\pgfqpoint{3.696000in}{3.696000in}}%
\pgfusepath{clip}%
\pgfsetbuttcap%
\pgfsetroundjoin%
\definecolor{currentfill}{rgb}{0.121569,0.466667,0.705882}%
\pgfsetfillcolor{currentfill}%
\pgfsetfillopacity{0.968772}%
\pgfsetlinewidth{1.003750pt}%
\definecolor{currentstroke}{rgb}{0.121569,0.466667,0.705882}%
\pgfsetstrokecolor{currentstroke}%
\pgfsetstrokeopacity{0.968772}%
\pgfsetdash{}{0pt}%
\pgfpathmoveto{\pgfqpoint{2.271715in}{0.840240in}}%
\pgfpathcurveto{\pgfqpoint{2.279951in}{0.840240in}}{\pgfqpoint{2.287851in}{0.843513in}}{\pgfqpoint{2.293675in}{0.849336in}}%
\pgfpathcurveto{\pgfqpoint{2.299499in}{0.855160in}}{\pgfqpoint{2.302772in}{0.863060in}}{\pgfqpoint{2.302772in}{0.871297in}}%
\pgfpathcurveto{\pgfqpoint{2.302772in}{0.879533in}}{\pgfqpoint{2.299499in}{0.887433in}}{\pgfqpoint{2.293675in}{0.893257in}}%
\pgfpathcurveto{\pgfqpoint{2.287851in}{0.899081in}}{\pgfqpoint{2.279951in}{0.902353in}}{\pgfqpoint{2.271715in}{0.902353in}}%
\pgfpathcurveto{\pgfqpoint{2.263479in}{0.902353in}}{\pgfqpoint{2.255579in}{0.899081in}}{\pgfqpoint{2.249755in}{0.893257in}}%
\pgfpathcurveto{\pgfqpoint{2.243931in}{0.887433in}}{\pgfqpoint{2.240659in}{0.879533in}}{\pgfqpoint{2.240659in}{0.871297in}}%
\pgfpathcurveto{\pgfqpoint{2.240659in}{0.863060in}}{\pgfqpoint{2.243931in}{0.855160in}}{\pgfqpoint{2.249755in}{0.849336in}}%
\pgfpathcurveto{\pgfqpoint{2.255579in}{0.843513in}}{\pgfqpoint{2.263479in}{0.840240in}}{\pgfqpoint{2.271715in}{0.840240in}}%
\pgfpathclose%
\pgfusepath{stroke,fill}%
\end{pgfscope}%
\begin{pgfscope}%
\pgfpathrectangle{\pgfqpoint{0.100000in}{0.212622in}}{\pgfqpoint{3.696000in}{3.696000in}}%
\pgfusepath{clip}%
\pgfsetbuttcap%
\pgfsetroundjoin%
\definecolor{currentfill}{rgb}{0.121569,0.466667,0.705882}%
\pgfsetfillcolor{currentfill}%
\pgfsetfillopacity{0.970288}%
\pgfsetlinewidth{1.003750pt}%
\definecolor{currentstroke}{rgb}{0.121569,0.466667,0.705882}%
\pgfsetstrokecolor{currentstroke}%
\pgfsetstrokeopacity{0.970288}%
\pgfsetdash{}{0pt}%
\pgfpathmoveto{\pgfqpoint{2.279190in}{0.835328in}}%
\pgfpathcurveto{\pgfqpoint{2.287426in}{0.835328in}}{\pgfqpoint{2.295326in}{0.838600in}}{\pgfqpoint{2.301150in}{0.844424in}}%
\pgfpathcurveto{\pgfqpoint{2.306974in}{0.850248in}}{\pgfqpoint{2.310247in}{0.858148in}}{\pgfqpoint{2.310247in}{0.866384in}}%
\pgfpathcurveto{\pgfqpoint{2.310247in}{0.874620in}}{\pgfqpoint{2.306974in}{0.882521in}}{\pgfqpoint{2.301150in}{0.888344in}}%
\pgfpathcurveto{\pgfqpoint{2.295326in}{0.894168in}}{\pgfqpoint{2.287426in}{0.897441in}}{\pgfqpoint{2.279190in}{0.897441in}}%
\pgfpathcurveto{\pgfqpoint{2.270954in}{0.897441in}}{\pgfqpoint{2.263054in}{0.894168in}}{\pgfqpoint{2.257230in}{0.888344in}}%
\pgfpathcurveto{\pgfqpoint{2.251406in}{0.882521in}}{\pgfqpoint{2.248134in}{0.874620in}}{\pgfqpoint{2.248134in}{0.866384in}}%
\pgfpathcurveto{\pgfqpoint{2.248134in}{0.858148in}}{\pgfqpoint{2.251406in}{0.850248in}}{\pgfqpoint{2.257230in}{0.844424in}}%
\pgfpathcurveto{\pgfqpoint{2.263054in}{0.838600in}}{\pgfqpoint{2.270954in}{0.835328in}}{\pgfqpoint{2.279190in}{0.835328in}}%
\pgfpathclose%
\pgfusepath{stroke,fill}%
\end{pgfscope}%
\begin{pgfscope}%
\pgfpathrectangle{\pgfqpoint{0.100000in}{0.212622in}}{\pgfqpoint{3.696000in}{3.696000in}}%
\pgfusepath{clip}%
\pgfsetbuttcap%
\pgfsetroundjoin%
\definecolor{currentfill}{rgb}{0.121569,0.466667,0.705882}%
\pgfsetfillcolor{currentfill}%
\pgfsetfillopacity{0.971728}%
\pgfsetlinewidth{1.003750pt}%
\definecolor{currentstroke}{rgb}{0.121569,0.466667,0.705882}%
\pgfsetstrokecolor{currentstroke}%
\pgfsetstrokeopacity{0.971728}%
\pgfsetdash{}{0pt}%
\pgfpathmoveto{\pgfqpoint{2.286329in}{0.830572in}}%
\pgfpathcurveto{\pgfqpoint{2.294565in}{0.830572in}}{\pgfqpoint{2.302465in}{0.833844in}}{\pgfqpoint{2.308289in}{0.839668in}}%
\pgfpathcurveto{\pgfqpoint{2.314113in}{0.845492in}}{\pgfqpoint{2.317386in}{0.853392in}}{\pgfqpoint{2.317386in}{0.861629in}}%
\pgfpathcurveto{\pgfqpoint{2.317386in}{0.869865in}}{\pgfqpoint{2.314113in}{0.877765in}}{\pgfqpoint{2.308289in}{0.883589in}}%
\pgfpathcurveto{\pgfqpoint{2.302465in}{0.889413in}}{\pgfqpoint{2.294565in}{0.892685in}}{\pgfqpoint{2.286329in}{0.892685in}}%
\pgfpathcurveto{\pgfqpoint{2.278093in}{0.892685in}}{\pgfqpoint{2.270193in}{0.889413in}}{\pgfqpoint{2.264369in}{0.883589in}}%
\pgfpathcurveto{\pgfqpoint{2.258545in}{0.877765in}}{\pgfqpoint{2.255273in}{0.869865in}}{\pgfqpoint{2.255273in}{0.861629in}}%
\pgfpathcurveto{\pgfqpoint{2.255273in}{0.853392in}}{\pgfqpoint{2.258545in}{0.845492in}}{\pgfqpoint{2.264369in}{0.839668in}}%
\pgfpathcurveto{\pgfqpoint{2.270193in}{0.833844in}}{\pgfqpoint{2.278093in}{0.830572in}}{\pgfqpoint{2.286329in}{0.830572in}}%
\pgfpathclose%
\pgfusepath{stroke,fill}%
\end{pgfscope}%
\begin{pgfscope}%
\pgfpathrectangle{\pgfqpoint{0.100000in}{0.212622in}}{\pgfqpoint{3.696000in}{3.696000in}}%
\pgfusepath{clip}%
\pgfsetbuttcap%
\pgfsetroundjoin%
\definecolor{currentfill}{rgb}{0.121569,0.466667,0.705882}%
\pgfsetfillcolor{currentfill}%
\pgfsetfillopacity{0.973146}%
\pgfsetlinewidth{1.003750pt}%
\definecolor{currentstroke}{rgb}{0.121569,0.466667,0.705882}%
\pgfsetstrokecolor{currentstroke}%
\pgfsetstrokeopacity{0.973146}%
\pgfsetdash{}{0pt}%
\pgfpathmoveto{\pgfqpoint{2.293081in}{0.826006in}}%
\pgfpathcurveto{\pgfqpoint{2.301318in}{0.826006in}}{\pgfqpoint{2.309218in}{0.829278in}}{\pgfqpoint{2.315042in}{0.835102in}}%
\pgfpathcurveto{\pgfqpoint{2.320866in}{0.840926in}}{\pgfqpoint{2.324138in}{0.848826in}}{\pgfqpoint{2.324138in}{0.857062in}}%
\pgfpathcurveto{\pgfqpoint{2.324138in}{0.865298in}}{\pgfqpoint{2.320866in}{0.873198in}}{\pgfqpoint{2.315042in}{0.879022in}}%
\pgfpathcurveto{\pgfqpoint{2.309218in}{0.884846in}}{\pgfqpoint{2.301318in}{0.888119in}}{\pgfqpoint{2.293081in}{0.888119in}}%
\pgfpathcurveto{\pgfqpoint{2.284845in}{0.888119in}}{\pgfqpoint{2.276945in}{0.884846in}}{\pgfqpoint{2.271121in}{0.879022in}}%
\pgfpathcurveto{\pgfqpoint{2.265297in}{0.873198in}}{\pgfqpoint{2.262025in}{0.865298in}}{\pgfqpoint{2.262025in}{0.857062in}}%
\pgfpathcurveto{\pgfqpoint{2.262025in}{0.848826in}}{\pgfqpoint{2.265297in}{0.840926in}}{\pgfqpoint{2.271121in}{0.835102in}}%
\pgfpathcurveto{\pgfqpoint{2.276945in}{0.829278in}}{\pgfqpoint{2.284845in}{0.826006in}}{\pgfqpoint{2.293081in}{0.826006in}}%
\pgfpathclose%
\pgfusepath{stroke,fill}%
\end{pgfscope}%
\begin{pgfscope}%
\pgfpathrectangle{\pgfqpoint{0.100000in}{0.212622in}}{\pgfqpoint{3.696000in}{3.696000in}}%
\pgfusepath{clip}%
\pgfsetbuttcap%
\pgfsetroundjoin%
\definecolor{currentfill}{rgb}{0.121569,0.466667,0.705882}%
\pgfsetfillcolor{currentfill}%
\pgfsetfillopacity{0.974499}%
\pgfsetlinewidth{1.003750pt}%
\definecolor{currentstroke}{rgb}{0.121569,0.466667,0.705882}%
\pgfsetstrokecolor{currentstroke}%
\pgfsetstrokeopacity{0.974499}%
\pgfsetdash{}{0pt}%
\pgfpathmoveto{\pgfqpoint{2.299480in}{0.821756in}}%
\pgfpathcurveto{\pgfqpoint{2.307716in}{0.821756in}}{\pgfqpoint{2.315616in}{0.825028in}}{\pgfqpoint{2.321440in}{0.830852in}}%
\pgfpathcurveto{\pgfqpoint{2.327264in}{0.836676in}}{\pgfqpoint{2.330537in}{0.844576in}}{\pgfqpoint{2.330537in}{0.852812in}}%
\pgfpathcurveto{\pgfqpoint{2.330537in}{0.861048in}}{\pgfqpoint{2.327264in}{0.868948in}}{\pgfqpoint{2.321440in}{0.874772in}}%
\pgfpathcurveto{\pgfqpoint{2.315616in}{0.880596in}}{\pgfqpoint{2.307716in}{0.883869in}}{\pgfqpoint{2.299480in}{0.883869in}}%
\pgfpathcurveto{\pgfqpoint{2.291244in}{0.883869in}}{\pgfqpoint{2.283344in}{0.880596in}}{\pgfqpoint{2.277520in}{0.874772in}}%
\pgfpathcurveto{\pgfqpoint{2.271696in}{0.868948in}}{\pgfqpoint{2.268424in}{0.861048in}}{\pgfqpoint{2.268424in}{0.852812in}}%
\pgfpathcurveto{\pgfqpoint{2.268424in}{0.844576in}}{\pgfqpoint{2.271696in}{0.836676in}}{\pgfqpoint{2.277520in}{0.830852in}}%
\pgfpathcurveto{\pgfqpoint{2.283344in}{0.825028in}}{\pgfqpoint{2.291244in}{0.821756in}}{\pgfqpoint{2.299480in}{0.821756in}}%
\pgfpathclose%
\pgfusepath{stroke,fill}%
\end{pgfscope}%
\begin{pgfscope}%
\pgfpathrectangle{\pgfqpoint{0.100000in}{0.212622in}}{\pgfqpoint{3.696000in}{3.696000in}}%
\pgfusepath{clip}%
\pgfsetbuttcap%
\pgfsetroundjoin%
\definecolor{currentfill}{rgb}{0.121569,0.466667,0.705882}%
\pgfsetfillcolor{currentfill}%
\pgfsetfillopacity{0.974652}%
\pgfsetlinewidth{1.003750pt}%
\definecolor{currentstroke}{rgb}{0.121569,0.466667,0.705882}%
\pgfsetstrokecolor{currentstroke}%
\pgfsetstrokeopacity{0.974652}%
\pgfsetdash{}{0pt}%
\pgfpathmoveto{\pgfqpoint{2.428877in}{0.846494in}}%
\pgfpathcurveto{\pgfqpoint{2.437114in}{0.846494in}}{\pgfqpoint{2.445014in}{0.849766in}}{\pgfqpoint{2.450838in}{0.855590in}}%
\pgfpathcurveto{\pgfqpoint{2.456661in}{0.861414in}}{\pgfqpoint{2.459934in}{0.869314in}}{\pgfqpoint{2.459934in}{0.877550in}}%
\pgfpathcurveto{\pgfqpoint{2.459934in}{0.885786in}}{\pgfqpoint{2.456661in}{0.893687in}}{\pgfqpoint{2.450838in}{0.899510in}}%
\pgfpathcurveto{\pgfqpoint{2.445014in}{0.905334in}}{\pgfqpoint{2.437114in}{0.908607in}}{\pgfqpoint{2.428877in}{0.908607in}}%
\pgfpathcurveto{\pgfqpoint{2.420641in}{0.908607in}}{\pgfqpoint{2.412741in}{0.905334in}}{\pgfqpoint{2.406917in}{0.899510in}}%
\pgfpathcurveto{\pgfqpoint{2.401093in}{0.893687in}}{\pgfqpoint{2.397821in}{0.885786in}}{\pgfqpoint{2.397821in}{0.877550in}}%
\pgfpathcurveto{\pgfqpoint{2.397821in}{0.869314in}}{\pgfqpoint{2.401093in}{0.861414in}}{\pgfqpoint{2.406917in}{0.855590in}}%
\pgfpathcurveto{\pgfqpoint{2.412741in}{0.849766in}}{\pgfqpoint{2.420641in}{0.846494in}}{\pgfqpoint{2.428877in}{0.846494in}}%
\pgfpathclose%
\pgfusepath{stroke,fill}%
\end{pgfscope}%
\begin{pgfscope}%
\pgfpathrectangle{\pgfqpoint{0.100000in}{0.212622in}}{\pgfqpoint{3.696000in}{3.696000in}}%
\pgfusepath{clip}%
\pgfsetbuttcap%
\pgfsetroundjoin%
\definecolor{currentfill}{rgb}{0.121569,0.466667,0.705882}%
\pgfsetfillcolor{currentfill}%
\pgfsetfillopacity{0.975769}%
\pgfsetlinewidth{1.003750pt}%
\definecolor{currentstroke}{rgb}{0.121569,0.466667,0.705882}%
\pgfsetstrokecolor{currentstroke}%
\pgfsetstrokeopacity{0.975769}%
\pgfsetdash{}{0pt}%
\pgfpathmoveto{\pgfqpoint{2.305561in}{0.817739in}}%
\pgfpathcurveto{\pgfqpoint{2.313797in}{0.817739in}}{\pgfqpoint{2.321697in}{0.821012in}}{\pgfqpoint{2.327521in}{0.826835in}}%
\pgfpathcurveto{\pgfqpoint{2.333345in}{0.832659in}}{\pgfqpoint{2.336617in}{0.840559in}}{\pgfqpoint{2.336617in}{0.848796in}}%
\pgfpathcurveto{\pgfqpoint{2.336617in}{0.857032in}}{\pgfqpoint{2.333345in}{0.864932in}}{\pgfqpoint{2.327521in}{0.870756in}}%
\pgfpathcurveto{\pgfqpoint{2.321697in}{0.876580in}}{\pgfqpoint{2.313797in}{0.879852in}}{\pgfqpoint{2.305561in}{0.879852in}}%
\pgfpathcurveto{\pgfqpoint{2.297324in}{0.879852in}}{\pgfqpoint{2.289424in}{0.876580in}}{\pgfqpoint{2.283600in}{0.870756in}}%
\pgfpathcurveto{\pgfqpoint{2.277776in}{0.864932in}}{\pgfqpoint{2.274504in}{0.857032in}}{\pgfqpoint{2.274504in}{0.848796in}}%
\pgfpathcurveto{\pgfqpoint{2.274504in}{0.840559in}}{\pgfqpoint{2.277776in}{0.832659in}}{\pgfqpoint{2.283600in}{0.826835in}}%
\pgfpathcurveto{\pgfqpoint{2.289424in}{0.821012in}}{\pgfqpoint{2.297324in}{0.817739in}}{\pgfqpoint{2.305561in}{0.817739in}}%
\pgfpathclose%
\pgfusepath{stroke,fill}%
\end{pgfscope}%
\begin{pgfscope}%
\pgfpathrectangle{\pgfqpoint{0.100000in}{0.212622in}}{\pgfqpoint{3.696000in}{3.696000in}}%
\pgfusepath{clip}%
\pgfsetbuttcap%
\pgfsetroundjoin%
\definecolor{currentfill}{rgb}{0.121569,0.466667,0.705882}%
\pgfsetfillcolor{currentfill}%
\pgfsetfillopacity{0.976963}%
\pgfsetlinewidth{1.003750pt}%
\definecolor{currentstroke}{rgb}{0.121569,0.466667,0.705882}%
\pgfsetstrokecolor{currentstroke}%
\pgfsetstrokeopacity{0.976963}%
\pgfsetdash{}{0pt}%
\pgfpathmoveto{\pgfqpoint{2.311357in}{0.814137in}}%
\pgfpathcurveto{\pgfqpoint{2.319593in}{0.814137in}}{\pgfqpoint{2.327493in}{0.817410in}}{\pgfqpoint{2.333317in}{0.823234in}}%
\pgfpathcurveto{\pgfqpoint{2.339141in}{0.829058in}}{\pgfqpoint{2.342413in}{0.836958in}}{\pgfqpoint{2.342413in}{0.845194in}}%
\pgfpathcurveto{\pgfqpoint{2.342413in}{0.853430in}}{\pgfqpoint{2.339141in}{0.861330in}}{\pgfqpoint{2.333317in}{0.867154in}}%
\pgfpathcurveto{\pgfqpoint{2.327493in}{0.872978in}}{\pgfqpoint{2.319593in}{0.876250in}}{\pgfqpoint{2.311357in}{0.876250in}}%
\pgfpathcurveto{\pgfqpoint{2.303121in}{0.876250in}}{\pgfqpoint{2.295221in}{0.872978in}}{\pgfqpoint{2.289397in}{0.867154in}}%
\pgfpathcurveto{\pgfqpoint{2.283573in}{0.861330in}}{\pgfqpoint{2.280300in}{0.853430in}}{\pgfqpoint{2.280300in}{0.845194in}}%
\pgfpathcurveto{\pgfqpoint{2.280300in}{0.836958in}}{\pgfqpoint{2.283573in}{0.829058in}}{\pgfqpoint{2.289397in}{0.823234in}}%
\pgfpathcurveto{\pgfqpoint{2.295221in}{0.817410in}}{\pgfqpoint{2.303121in}{0.814137in}}{\pgfqpoint{2.311357in}{0.814137in}}%
\pgfpathclose%
\pgfusepath{stroke,fill}%
\end{pgfscope}%
\begin{pgfscope}%
\pgfpathrectangle{\pgfqpoint{0.100000in}{0.212622in}}{\pgfqpoint{3.696000in}{3.696000in}}%
\pgfusepath{clip}%
\pgfsetbuttcap%
\pgfsetroundjoin%
\definecolor{currentfill}{rgb}{0.121569,0.466667,0.705882}%
\pgfsetfillcolor{currentfill}%
\pgfsetfillopacity{0.978108}%
\pgfsetlinewidth{1.003750pt}%
\definecolor{currentstroke}{rgb}{0.121569,0.466667,0.705882}%
\pgfsetstrokecolor{currentstroke}%
\pgfsetstrokeopacity{0.978108}%
\pgfsetdash{}{0pt}%
\pgfpathmoveto{\pgfqpoint{2.316787in}{0.810749in}}%
\pgfpathcurveto{\pgfqpoint{2.325024in}{0.810749in}}{\pgfqpoint{2.332924in}{0.814021in}}{\pgfqpoint{2.338748in}{0.819845in}}%
\pgfpathcurveto{\pgfqpoint{2.344572in}{0.825669in}}{\pgfqpoint{2.347844in}{0.833569in}}{\pgfqpoint{2.347844in}{0.841805in}}%
\pgfpathcurveto{\pgfqpoint{2.347844in}{0.850041in}}{\pgfqpoint{2.344572in}{0.857941in}}{\pgfqpoint{2.338748in}{0.863765in}}%
\pgfpathcurveto{\pgfqpoint{2.332924in}{0.869589in}}{\pgfqpoint{2.325024in}{0.872862in}}{\pgfqpoint{2.316787in}{0.872862in}}%
\pgfpathcurveto{\pgfqpoint{2.308551in}{0.872862in}}{\pgfqpoint{2.300651in}{0.869589in}}{\pgfqpoint{2.294827in}{0.863765in}}%
\pgfpathcurveto{\pgfqpoint{2.289003in}{0.857941in}}{\pgfqpoint{2.285731in}{0.850041in}}{\pgfqpoint{2.285731in}{0.841805in}}%
\pgfpathcurveto{\pgfqpoint{2.285731in}{0.833569in}}{\pgfqpoint{2.289003in}{0.825669in}}{\pgfqpoint{2.294827in}{0.819845in}}%
\pgfpathcurveto{\pgfqpoint{2.300651in}{0.814021in}}{\pgfqpoint{2.308551in}{0.810749in}}{\pgfqpoint{2.316787in}{0.810749in}}%
\pgfpathclose%
\pgfusepath{stroke,fill}%
\end{pgfscope}%
\begin{pgfscope}%
\pgfpathrectangle{\pgfqpoint{0.100000in}{0.212622in}}{\pgfqpoint{3.696000in}{3.696000in}}%
\pgfusepath{clip}%
\pgfsetbuttcap%
\pgfsetroundjoin%
\definecolor{currentfill}{rgb}{0.121569,0.466667,0.705882}%
\pgfsetfillcolor{currentfill}%
\pgfsetfillopacity{0.979200}%
\pgfsetlinewidth{1.003750pt}%
\definecolor{currentstroke}{rgb}{0.121569,0.466667,0.705882}%
\pgfsetstrokecolor{currentstroke}%
\pgfsetstrokeopacity{0.979200}%
\pgfsetdash{}{0pt}%
\pgfpathmoveto{\pgfqpoint{2.321927in}{0.807758in}}%
\pgfpathcurveto{\pgfqpoint{2.330163in}{0.807758in}}{\pgfqpoint{2.338063in}{0.811030in}}{\pgfqpoint{2.343887in}{0.816854in}}%
\pgfpathcurveto{\pgfqpoint{2.349711in}{0.822678in}}{\pgfqpoint{2.352983in}{0.830578in}}{\pgfqpoint{2.352983in}{0.838814in}}%
\pgfpathcurveto{\pgfqpoint{2.352983in}{0.847051in}}{\pgfqpoint{2.349711in}{0.854951in}}{\pgfqpoint{2.343887in}{0.860775in}}%
\pgfpathcurveto{\pgfqpoint{2.338063in}{0.866599in}}{\pgfqpoint{2.330163in}{0.869871in}}{\pgfqpoint{2.321927in}{0.869871in}}%
\pgfpathcurveto{\pgfqpoint{2.313690in}{0.869871in}}{\pgfqpoint{2.305790in}{0.866599in}}{\pgfqpoint{2.299966in}{0.860775in}}%
\pgfpathcurveto{\pgfqpoint{2.294142in}{0.854951in}}{\pgfqpoint{2.290870in}{0.847051in}}{\pgfqpoint{2.290870in}{0.838814in}}%
\pgfpathcurveto{\pgfqpoint{2.290870in}{0.830578in}}{\pgfqpoint{2.294142in}{0.822678in}}{\pgfqpoint{2.299966in}{0.816854in}}%
\pgfpathcurveto{\pgfqpoint{2.305790in}{0.811030in}}{\pgfqpoint{2.313690in}{0.807758in}}{\pgfqpoint{2.321927in}{0.807758in}}%
\pgfpathclose%
\pgfusepath{stroke,fill}%
\end{pgfscope}%
\begin{pgfscope}%
\pgfpathrectangle{\pgfqpoint{0.100000in}{0.212622in}}{\pgfqpoint{3.696000in}{3.696000in}}%
\pgfusepath{clip}%
\pgfsetbuttcap%
\pgfsetroundjoin%
\definecolor{currentfill}{rgb}{0.121569,0.466667,0.705882}%
\pgfsetfillcolor{currentfill}%
\pgfsetfillopacity{0.980240}%
\pgfsetlinewidth{1.003750pt}%
\definecolor{currentstroke}{rgb}{0.121569,0.466667,0.705882}%
\pgfsetstrokecolor{currentstroke}%
\pgfsetstrokeopacity{0.980240}%
\pgfsetdash{}{0pt}%
\pgfpathmoveto{\pgfqpoint{2.326768in}{0.805114in}}%
\pgfpathcurveto{\pgfqpoint{2.335004in}{0.805114in}}{\pgfqpoint{2.342904in}{0.808386in}}{\pgfqpoint{2.348728in}{0.814210in}}%
\pgfpathcurveto{\pgfqpoint{2.354552in}{0.820034in}}{\pgfqpoint{2.357824in}{0.827934in}}{\pgfqpoint{2.357824in}{0.836170in}}%
\pgfpathcurveto{\pgfqpoint{2.357824in}{0.844407in}}{\pgfqpoint{2.354552in}{0.852307in}}{\pgfqpoint{2.348728in}{0.858131in}}%
\pgfpathcurveto{\pgfqpoint{2.342904in}{0.863955in}}{\pgfqpoint{2.335004in}{0.867227in}}{\pgfqpoint{2.326768in}{0.867227in}}%
\pgfpathcurveto{\pgfqpoint{2.318532in}{0.867227in}}{\pgfqpoint{2.310631in}{0.863955in}}{\pgfqpoint{2.304808in}{0.858131in}}%
\pgfpathcurveto{\pgfqpoint{2.298984in}{0.852307in}}{\pgfqpoint{2.295711in}{0.844407in}}{\pgfqpoint{2.295711in}{0.836170in}}%
\pgfpathcurveto{\pgfqpoint{2.295711in}{0.827934in}}{\pgfqpoint{2.298984in}{0.820034in}}{\pgfqpoint{2.304808in}{0.814210in}}%
\pgfpathcurveto{\pgfqpoint{2.310631in}{0.808386in}}{\pgfqpoint{2.318532in}{0.805114in}}{\pgfqpoint{2.326768in}{0.805114in}}%
\pgfpathclose%
\pgfusepath{stroke,fill}%
\end{pgfscope}%
\begin{pgfscope}%
\pgfpathrectangle{\pgfqpoint{0.100000in}{0.212622in}}{\pgfqpoint{3.696000in}{3.696000in}}%
\pgfusepath{clip}%
\pgfsetbuttcap%
\pgfsetroundjoin%
\definecolor{currentfill}{rgb}{0.121569,0.466667,0.705882}%
\pgfsetfillcolor{currentfill}%
\pgfsetfillopacity{0.981198}%
\pgfsetlinewidth{1.003750pt}%
\definecolor{currentstroke}{rgb}{0.121569,0.466667,0.705882}%
\pgfsetstrokecolor{currentstroke}%
\pgfsetstrokeopacity{0.981198}%
\pgfsetdash{}{0pt}%
\pgfpathmoveto{\pgfqpoint{2.331297in}{0.802743in}}%
\pgfpathcurveto{\pgfqpoint{2.339534in}{0.802743in}}{\pgfqpoint{2.347434in}{0.806015in}}{\pgfqpoint{2.353258in}{0.811839in}}%
\pgfpathcurveto{\pgfqpoint{2.359082in}{0.817663in}}{\pgfqpoint{2.362354in}{0.825563in}}{\pgfqpoint{2.362354in}{0.833799in}}%
\pgfpathcurveto{\pgfqpoint{2.362354in}{0.842035in}}{\pgfqpoint{2.359082in}{0.849936in}}{\pgfqpoint{2.353258in}{0.855759in}}%
\pgfpathcurveto{\pgfqpoint{2.347434in}{0.861583in}}{\pgfqpoint{2.339534in}{0.864856in}}{\pgfqpoint{2.331297in}{0.864856in}}%
\pgfpathcurveto{\pgfqpoint{2.323061in}{0.864856in}}{\pgfqpoint{2.315161in}{0.861583in}}{\pgfqpoint{2.309337in}{0.855759in}}%
\pgfpathcurveto{\pgfqpoint{2.303513in}{0.849936in}}{\pgfqpoint{2.300241in}{0.842035in}}{\pgfqpoint{2.300241in}{0.833799in}}%
\pgfpathcurveto{\pgfqpoint{2.300241in}{0.825563in}}{\pgfqpoint{2.303513in}{0.817663in}}{\pgfqpoint{2.309337in}{0.811839in}}%
\pgfpathcurveto{\pgfqpoint{2.315161in}{0.806015in}}{\pgfqpoint{2.323061in}{0.802743in}}{\pgfqpoint{2.331297in}{0.802743in}}%
\pgfpathclose%
\pgfusepath{stroke,fill}%
\end{pgfscope}%
\begin{pgfscope}%
\pgfpathrectangle{\pgfqpoint{0.100000in}{0.212622in}}{\pgfqpoint{3.696000in}{3.696000in}}%
\pgfusepath{clip}%
\pgfsetbuttcap%
\pgfsetroundjoin%
\definecolor{currentfill}{rgb}{0.121569,0.466667,0.705882}%
\pgfsetfillcolor{currentfill}%
\pgfsetfillopacity{0.981654}%
\pgfsetlinewidth{1.003750pt}%
\definecolor{currentstroke}{rgb}{0.121569,0.466667,0.705882}%
\pgfsetstrokecolor{currentstroke}%
\pgfsetstrokeopacity{0.981654}%
\pgfsetdash{}{0pt}%
\pgfpathmoveto{\pgfqpoint{2.435095in}{0.821830in}}%
\pgfpathcurveto{\pgfqpoint{2.443331in}{0.821830in}}{\pgfqpoint{2.451231in}{0.825103in}}{\pgfqpoint{2.457055in}{0.830926in}}%
\pgfpathcurveto{\pgfqpoint{2.462879in}{0.836750in}}{\pgfqpoint{2.466151in}{0.844650in}}{\pgfqpoint{2.466151in}{0.852887in}}%
\pgfpathcurveto{\pgfqpoint{2.466151in}{0.861123in}}{\pgfqpoint{2.462879in}{0.869023in}}{\pgfqpoint{2.457055in}{0.874847in}}%
\pgfpathcurveto{\pgfqpoint{2.451231in}{0.880671in}}{\pgfqpoint{2.443331in}{0.883943in}}{\pgfqpoint{2.435095in}{0.883943in}}%
\pgfpathcurveto{\pgfqpoint{2.426859in}{0.883943in}}{\pgfqpoint{2.418959in}{0.880671in}}{\pgfqpoint{2.413135in}{0.874847in}}%
\pgfpathcurveto{\pgfqpoint{2.407311in}{0.869023in}}{\pgfqpoint{2.404038in}{0.861123in}}{\pgfqpoint{2.404038in}{0.852887in}}%
\pgfpathcurveto{\pgfqpoint{2.404038in}{0.844650in}}{\pgfqpoint{2.407311in}{0.836750in}}{\pgfqpoint{2.413135in}{0.830926in}}%
\pgfpathcurveto{\pgfqpoint{2.418959in}{0.825103in}}{\pgfqpoint{2.426859in}{0.821830in}}{\pgfqpoint{2.435095in}{0.821830in}}%
\pgfpathclose%
\pgfusepath{stroke,fill}%
\end{pgfscope}%
\begin{pgfscope}%
\pgfpathrectangle{\pgfqpoint{0.100000in}{0.212622in}}{\pgfqpoint{3.696000in}{3.696000in}}%
\pgfusepath{clip}%
\pgfsetbuttcap%
\pgfsetroundjoin%
\definecolor{currentfill}{rgb}{0.121569,0.466667,0.705882}%
\pgfsetfillcolor{currentfill}%
\pgfsetfillopacity{0.982123}%
\pgfsetlinewidth{1.003750pt}%
\definecolor{currentstroke}{rgb}{0.121569,0.466667,0.705882}%
\pgfsetstrokecolor{currentstroke}%
\pgfsetstrokeopacity{0.982123}%
\pgfsetdash{}{0pt}%
\pgfpathmoveto{\pgfqpoint{2.335600in}{0.800662in}}%
\pgfpathcurveto{\pgfqpoint{2.343836in}{0.800662in}}{\pgfqpoint{2.351736in}{0.803934in}}{\pgfqpoint{2.357560in}{0.809758in}}%
\pgfpathcurveto{\pgfqpoint{2.363384in}{0.815582in}}{\pgfqpoint{2.366656in}{0.823482in}}{\pgfqpoint{2.366656in}{0.831718in}}%
\pgfpathcurveto{\pgfqpoint{2.366656in}{0.839955in}}{\pgfqpoint{2.363384in}{0.847855in}}{\pgfqpoint{2.357560in}{0.853679in}}%
\pgfpathcurveto{\pgfqpoint{2.351736in}{0.859503in}}{\pgfqpoint{2.343836in}{0.862775in}}{\pgfqpoint{2.335600in}{0.862775in}}%
\pgfpathcurveto{\pgfqpoint{2.327364in}{0.862775in}}{\pgfqpoint{2.319464in}{0.859503in}}{\pgfqpoint{2.313640in}{0.853679in}}%
\pgfpathcurveto{\pgfqpoint{2.307816in}{0.847855in}}{\pgfqpoint{2.304543in}{0.839955in}}{\pgfqpoint{2.304543in}{0.831718in}}%
\pgfpathcurveto{\pgfqpoint{2.304543in}{0.823482in}}{\pgfqpoint{2.307816in}{0.815582in}}{\pgfqpoint{2.313640in}{0.809758in}}%
\pgfpathcurveto{\pgfqpoint{2.319464in}{0.803934in}}{\pgfqpoint{2.327364in}{0.800662in}}{\pgfqpoint{2.335600in}{0.800662in}}%
\pgfpathclose%
\pgfusepath{stroke,fill}%
\end{pgfscope}%
\begin{pgfscope}%
\pgfpathrectangle{\pgfqpoint{0.100000in}{0.212622in}}{\pgfqpoint{3.696000in}{3.696000in}}%
\pgfusepath{clip}%
\pgfsetbuttcap%
\pgfsetroundjoin%
\definecolor{currentfill}{rgb}{0.121569,0.466667,0.705882}%
\pgfsetfillcolor{currentfill}%
\pgfsetfillopacity{0.982981}%
\pgfsetlinewidth{1.003750pt}%
\definecolor{currentstroke}{rgb}{0.121569,0.466667,0.705882}%
\pgfsetstrokecolor{currentstroke}%
\pgfsetstrokeopacity{0.982981}%
\pgfsetdash{}{0pt}%
\pgfpathmoveto{\pgfqpoint{2.339511in}{0.798929in}}%
\pgfpathcurveto{\pgfqpoint{2.347747in}{0.798929in}}{\pgfqpoint{2.355647in}{0.802202in}}{\pgfqpoint{2.361471in}{0.808025in}}%
\pgfpathcurveto{\pgfqpoint{2.367295in}{0.813849in}}{\pgfqpoint{2.370568in}{0.821749in}}{\pgfqpoint{2.370568in}{0.829986in}}%
\pgfpathcurveto{\pgfqpoint{2.370568in}{0.838222in}}{\pgfqpoint{2.367295in}{0.846122in}}{\pgfqpoint{2.361471in}{0.851946in}}%
\pgfpathcurveto{\pgfqpoint{2.355647in}{0.857770in}}{\pgfqpoint{2.347747in}{0.861042in}}{\pgfqpoint{2.339511in}{0.861042in}}%
\pgfpathcurveto{\pgfqpoint{2.331275in}{0.861042in}}{\pgfqpoint{2.323375in}{0.857770in}}{\pgfqpoint{2.317551in}{0.851946in}}%
\pgfpathcurveto{\pgfqpoint{2.311727in}{0.846122in}}{\pgfqpoint{2.308455in}{0.838222in}}{\pgfqpoint{2.308455in}{0.829986in}}%
\pgfpathcurveto{\pgfqpoint{2.308455in}{0.821749in}}{\pgfqpoint{2.311727in}{0.813849in}}{\pgfqpoint{2.317551in}{0.808025in}}%
\pgfpathcurveto{\pgfqpoint{2.323375in}{0.802202in}}{\pgfqpoint{2.331275in}{0.798929in}}{\pgfqpoint{2.339511in}{0.798929in}}%
\pgfpathclose%
\pgfusepath{stroke,fill}%
\end{pgfscope}%
\begin{pgfscope}%
\pgfpathrectangle{\pgfqpoint{0.100000in}{0.212622in}}{\pgfqpoint{3.696000in}{3.696000in}}%
\pgfusepath{clip}%
\pgfsetbuttcap%
\pgfsetroundjoin%
\definecolor{currentfill}{rgb}{0.121569,0.466667,0.705882}%
\pgfsetfillcolor{currentfill}%
\pgfsetfillopacity{0.984549}%
\pgfsetlinewidth{1.003750pt}%
\definecolor{currentstroke}{rgb}{0.121569,0.466667,0.705882}%
\pgfsetstrokecolor{currentstroke}%
\pgfsetstrokeopacity{0.984549}%
\pgfsetdash{}{0pt}%
\pgfpathmoveto{\pgfqpoint{2.346628in}{0.795816in}}%
\pgfpathcurveto{\pgfqpoint{2.354864in}{0.795816in}}{\pgfqpoint{2.362765in}{0.799088in}}{\pgfqpoint{2.368588in}{0.804912in}}%
\pgfpathcurveto{\pgfqpoint{2.374412in}{0.810736in}}{\pgfqpoint{2.377685in}{0.818636in}}{\pgfqpoint{2.377685in}{0.826872in}}%
\pgfpathcurveto{\pgfqpoint{2.377685in}{0.835109in}}{\pgfqpoint{2.374412in}{0.843009in}}{\pgfqpoint{2.368588in}{0.848833in}}%
\pgfpathcurveto{\pgfqpoint{2.362765in}{0.854656in}}{\pgfqpoint{2.354864in}{0.857929in}}{\pgfqpoint{2.346628in}{0.857929in}}%
\pgfpathcurveto{\pgfqpoint{2.338392in}{0.857929in}}{\pgfqpoint{2.330492in}{0.854656in}}{\pgfqpoint{2.324668in}{0.848833in}}%
\pgfpathcurveto{\pgfqpoint{2.318844in}{0.843009in}}{\pgfqpoint{2.315572in}{0.835109in}}{\pgfqpoint{2.315572in}{0.826872in}}%
\pgfpathcurveto{\pgfqpoint{2.315572in}{0.818636in}}{\pgfqpoint{2.318844in}{0.810736in}}{\pgfqpoint{2.324668in}{0.804912in}}%
\pgfpathcurveto{\pgfqpoint{2.330492in}{0.799088in}}{\pgfqpoint{2.338392in}{0.795816in}}{\pgfqpoint{2.346628in}{0.795816in}}%
\pgfpathclose%
\pgfusepath{stroke,fill}%
\end{pgfscope}%
\begin{pgfscope}%
\pgfpathrectangle{\pgfqpoint{0.100000in}{0.212622in}}{\pgfqpoint{3.696000in}{3.696000in}}%
\pgfusepath{clip}%
\pgfsetbuttcap%
\pgfsetroundjoin%
\definecolor{currentfill}{rgb}{0.121569,0.466667,0.705882}%
\pgfsetfillcolor{currentfill}%
\pgfsetfillopacity{0.985914}%
\pgfsetlinewidth{1.003750pt}%
\definecolor{currentstroke}{rgb}{0.121569,0.466667,0.705882}%
\pgfsetstrokecolor{currentstroke}%
\pgfsetstrokeopacity{0.985914}%
\pgfsetdash{}{0pt}%
\pgfpathmoveto{\pgfqpoint{2.353496in}{0.792860in}}%
\pgfpathcurveto{\pgfqpoint{2.361732in}{0.792860in}}{\pgfqpoint{2.369632in}{0.796132in}}{\pgfqpoint{2.375456in}{0.801956in}}%
\pgfpathcurveto{\pgfqpoint{2.381280in}{0.807780in}}{\pgfqpoint{2.384552in}{0.815680in}}{\pgfqpoint{2.384552in}{0.823916in}}%
\pgfpathcurveto{\pgfqpoint{2.384552in}{0.832153in}}{\pgfqpoint{2.381280in}{0.840053in}}{\pgfqpoint{2.375456in}{0.845877in}}%
\pgfpathcurveto{\pgfqpoint{2.369632in}{0.851700in}}{\pgfqpoint{2.361732in}{0.854973in}}{\pgfqpoint{2.353496in}{0.854973in}}%
\pgfpathcurveto{\pgfqpoint{2.345259in}{0.854973in}}{\pgfqpoint{2.337359in}{0.851700in}}{\pgfqpoint{2.331535in}{0.845877in}}%
\pgfpathcurveto{\pgfqpoint{2.325711in}{0.840053in}}{\pgfqpoint{2.322439in}{0.832153in}}{\pgfqpoint{2.322439in}{0.823916in}}%
\pgfpathcurveto{\pgfqpoint{2.322439in}{0.815680in}}{\pgfqpoint{2.325711in}{0.807780in}}{\pgfqpoint{2.331535in}{0.801956in}}%
\pgfpathcurveto{\pgfqpoint{2.337359in}{0.796132in}}{\pgfqpoint{2.345259in}{0.792860in}}{\pgfqpoint{2.353496in}{0.792860in}}%
\pgfpathclose%
\pgfusepath{stroke,fill}%
\end{pgfscope}%
\begin{pgfscope}%
\pgfpathrectangle{\pgfqpoint{0.100000in}{0.212622in}}{\pgfqpoint{3.696000in}{3.696000in}}%
\pgfusepath{clip}%
\pgfsetbuttcap%
\pgfsetroundjoin%
\definecolor{currentfill}{rgb}{0.121569,0.466667,0.705882}%
\pgfsetfillcolor{currentfill}%
\pgfsetfillopacity{0.987254}%
\pgfsetlinewidth{1.003750pt}%
\definecolor{currentstroke}{rgb}{0.121569,0.466667,0.705882}%
\pgfsetstrokecolor{currentstroke}%
\pgfsetstrokeopacity{0.987254}%
\pgfsetdash{}{0pt}%
\pgfpathmoveto{\pgfqpoint{2.360062in}{0.790242in}}%
\pgfpathcurveto{\pgfqpoint{2.368299in}{0.790242in}}{\pgfqpoint{2.376199in}{0.793514in}}{\pgfqpoint{2.382023in}{0.799338in}}%
\pgfpathcurveto{\pgfqpoint{2.387847in}{0.805162in}}{\pgfqpoint{2.391119in}{0.813062in}}{\pgfqpoint{2.391119in}{0.821299in}}%
\pgfpathcurveto{\pgfqpoint{2.391119in}{0.829535in}}{\pgfqpoint{2.387847in}{0.837435in}}{\pgfqpoint{2.382023in}{0.843259in}}%
\pgfpathcurveto{\pgfqpoint{2.376199in}{0.849083in}}{\pgfqpoint{2.368299in}{0.852355in}}{\pgfqpoint{2.360062in}{0.852355in}}%
\pgfpathcurveto{\pgfqpoint{2.351826in}{0.852355in}}{\pgfqpoint{2.343926in}{0.849083in}}{\pgfqpoint{2.338102in}{0.843259in}}%
\pgfpathcurveto{\pgfqpoint{2.332278in}{0.837435in}}{\pgfqpoint{2.329006in}{0.829535in}}{\pgfqpoint{2.329006in}{0.821299in}}%
\pgfpathcurveto{\pgfqpoint{2.329006in}{0.813062in}}{\pgfqpoint{2.332278in}{0.805162in}}{\pgfqpoint{2.338102in}{0.799338in}}%
\pgfpathcurveto{\pgfqpoint{2.343926in}{0.793514in}}{\pgfqpoint{2.351826in}{0.790242in}}{\pgfqpoint{2.360062in}{0.790242in}}%
\pgfpathclose%
\pgfusepath{stroke,fill}%
\end{pgfscope}%
\begin{pgfscope}%
\pgfpathrectangle{\pgfqpoint{0.100000in}{0.212622in}}{\pgfqpoint{3.696000in}{3.696000in}}%
\pgfusepath{clip}%
\pgfsetbuttcap%
\pgfsetroundjoin%
\definecolor{currentfill}{rgb}{0.121569,0.466667,0.705882}%
\pgfsetfillcolor{currentfill}%
\pgfsetfillopacity{0.988545}%
\pgfsetlinewidth{1.003750pt}%
\definecolor{currentstroke}{rgb}{0.121569,0.466667,0.705882}%
\pgfsetstrokecolor{currentstroke}%
\pgfsetstrokeopacity{0.988545}%
\pgfsetdash{}{0pt}%
\pgfpathmoveto{\pgfqpoint{2.366404in}{0.787824in}}%
\pgfpathcurveto{\pgfqpoint{2.374640in}{0.787824in}}{\pgfqpoint{2.382540in}{0.791096in}}{\pgfqpoint{2.388364in}{0.796920in}}%
\pgfpathcurveto{\pgfqpoint{2.394188in}{0.802744in}}{\pgfqpoint{2.397460in}{0.810644in}}{\pgfqpoint{2.397460in}{0.818881in}}%
\pgfpathcurveto{\pgfqpoint{2.397460in}{0.827117in}}{\pgfqpoint{2.394188in}{0.835017in}}{\pgfqpoint{2.388364in}{0.840841in}}%
\pgfpathcurveto{\pgfqpoint{2.382540in}{0.846665in}}{\pgfqpoint{2.374640in}{0.849937in}}{\pgfqpoint{2.366404in}{0.849937in}}%
\pgfpathcurveto{\pgfqpoint{2.358167in}{0.849937in}}{\pgfqpoint{2.350267in}{0.846665in}}{\pgfqpoint{2.344443in}{0.840841in}}%
\pgfpathcurveto{\pgfqpoint{2.338619in}{0.835017in}}{\pgfqpoint{2.335347in}{0.827117in}}{\pgfqpoint{2.335347in}{0.818881in}}%
\pgfpathcurveto{\pgfqpoint{2.335347in}{0.810644in}}{\pgfqpoint{2.338619in}{0.802744in}}{\pgfqpoint{2.344443in}{0.796920in}}%
\pgfpathcurveto{\pgfqpoint{2.350267in}{0.791096in}}{\pgfqpoint{2.358167in}{0.787824in}}{\pgfqpoint{2.366404in}{0.787824in}}%
\pgfpathclose%
\pgfusepath{stroke,fill}%
\end{pgfscope}%
\begin{pgfscope}%
\pgfpathrectangle{\pgfqpoint{0.100000in}{0.212622in}}{\pgfqpoint{3.696000in}{3.696000in}}%
\pgfusepath{clip}%
\pgfsetbuttcap%
\pgfsetroundjoin%
\definecolor{currentfill}{rgb}{0.121569,0.466667,0.705882}%
\pgfsetfillcolor{currentfill}%
\pgfsetfillopacity{0.988943}%
\pgfsetlinewidth{1.003750pt}%
\definecolor{currentstroke}{rgb}{0.121569,0.466667,0.705882}%
\pgfsetstrokecolor{currentstroke}%
\pgfsetstrokeopacity{0.988943}%
\pgfsetdash{}{0pt}%
\pgfpathmoveto{\pgfqpoint{2.441306in}{0.796156in}}%
\pgfpathcurveto{\pgfqpoint{2.449543in}{0.796156in}}{\pgfqpoint{2.457443in}{0.799428in}}{\pgfqpoint{2.463267in}{0.805252in}}%
\pgfpathcurveto{\pgfqpoint{2.469091in}{0.811076in}}{\pgfqpoint{2.472363in}{0.818976in}}{\pgfqpoint{2.472363in}{0.827212in}}%
\pgfpathcurveto{\pgfqpoint{2.472363in}{0.835448in}}{\pgfqpoint{2.469091in}{0.843348in}}{\pgfqpoint{2.463267in}{0.849172in}}%
\pgfpathcurveto{\pgfqpoint{2.457443in}{0.854996in}}{\pgfqpoint{2.449543in}{0.858269in}}{\pgfqpoint{2.441306in}{0.858269in}}%
\pgfpathcurveto{\pgfqpoint{2.433070in}{0.858269in}}{\pgfqpoint{2.425170in}{0.854996in}}{\pgfqpoint{2.419346in}{0.849172in}}%
\pgfpathcurveto{\pgfqpoint{2.413522in}{0.843348in}}{\pgfqpoint{2.410250in}{0.835448in}}{\pgfqpoint{2.410250in}{0.827212in}}%
\pgfpathcurveto{\pgfqpoint{2.410250in}{0.818976in}}{\pgfqpoint{2.413522in}{0.811076in}}{\pgfqpoint{2.419346in}{0.805252in}}%
\pgfpathcurveto{\pgfqpoint{2.425170in}{0.799428in}}{\pgfqpoint{2.433070in}{0.796156in}}{\pgfqpoint{2.441306in}{0.796156in}}%
\pgfpathclose%
\pgfusepath{stroke,fill}%
\end{pgfscope}%
\begin{pgfscope}%
\pgfpathrectangle{\pgfqpoint{0.100000in}{0.212622in}}{\pgfqpoint{3.696000in}{3.696000in}}%
\pgfusepath{clip}%
\pgfsetbuttcap%
\pgfsetroundjoin%
\definecolor{currentfill}{rgb}{0.121569,0.466667,0.705882}%
\pgfsetfillcolor{currentfill}%
\pgfsetfillopacity{0.989793}%
\pgfsetlinewidth{1.003750pt}%
\definecolor{currentstroke}{rgb}{0.121569,0.466667,0.705882}%
\pgfsetstrokecolor{currentstroke}%
\pgfsetstrokeopacity{0.989793}%
\pgfsetdash{}{0pt}%
\pgfpathmoveto{\pgfqpoint{2.372467in}{0.785488in}}%
\pgfpathcurveto{\pgfqpoint{2.380703in}{0.785488in}}{\pgfqpoint{2.388603in}{0.788760in}}{\pgfqpoint{2.394427in}{0.794584in}}%
\pgfpathcurveto{\pgfqpoint{2.400251in}{0.800408in}}{\pgfqpoint{2.403523in}{0.808308in}}{\pgfqpoint{2.403523in}{0.816545in}}%
\pgfpathcurveto{\pgfqpoint{2.403523in}{0.824781in}}{\pgfqpoint{2.400251in}{0.832681in}}{\pgfqpoint{2.394427in}{0.838505in}}%
\pgfpathcurveto{\pgfqpoint{2.388603in}{0.844329in}}{\pgfqpoint{2.380703in}{0.847601in}}{\pgfqpoint{2.372467in}{0.847601in}}%
\pgfpathcurveto{\pgfqpoint{2.364231in}{0.847601in}}{\pgfqpoint{2.356331in}{0.844329in}}{\pgfqpoint{2.350507in}{0.838505in}}%
\pgfpathcurveto{\pgfqpoint{2.344683in}{0.832681in}}{\pgfqpoint{2.341410in}{0.824781in}}{\pgfqpoint{2.341410in}{0.816545in}}%
\pgfpathcurveto{\pgfqpoint{2.341410in}{0.808308in}}{\pgfqpoint{2.344683in}{0.800408in}}{\pgfqpoint{2.350507in}{0.794584in}}%
\pgfpathcurveto{\pgfqpoint{2.356331in}{0.788760in}}{\pgfqpoint{2.364231in}{0.785488in}}{\pgfqpoint{2.372467in}{0.785488in}}%
\pgfpathclose%
\pgfusepath{stroke,fill}%
\end{pgfscope}%
\begin{pgfscope}%
\pgfpathrectangle{\pgfqpoint{0.100000in}{0.212622in}}{\pgfqpoint{3.696000in}{3.696000in}}%
\pgfusepath{clip}%
\pgfsetbuttcap%
\pgfsetroundjoin%
\definecolor{currentfill}{rgb}{0.121569,0.466667,0.705882}%
\pgfsetfillcolor{currentfill}%
\pgfsetfillopacity{0.990948}%
\pgfsetlinewidth{1.003750pt}%
\definecolor{currentstroke}{rgb}{0.121569,0.466667,0.705882}%
\pgfsetstrokecolor{currentstroke}%
\pgfsetstrokeopacity{0.990948}%
\pgfsetdash{}{0pt}%
\pgfpathmoveto{\pgfqpoint{2.378288in}{0.783315in}}%
\pgfpathcurveto{\pgfqpoint{2.386524in}{0.783315in}}{\pgfqpoint{2.394424in}{0.786588in}}{\pgfqpoint{2.400248in}{0.792412in}}%
\pgfpathcurveto{\pgfqpoint{2.406072in}{0.798235in}}{\pgfqpoint{2.409344in}{0.806135in}}{\pgfqpoint{2.409344in}{0.814372in}}%
\pgfpathcurveto{\pgfqpoint{2.409344in}{0.822608in}}{\pgfqpoint{2.406072in}{0.830508in}}{\pgfqpoint{2.400248in}{0.836332in}}%
\pgfpathcurveto{\pgfqpoint{2.394424in}{0.842156in}}{\pgfqpoint{2.386524in}{0.845428in}}{\pgfqpoint{2.378288in}{0.845428in}}%
\pgfpathcurveto{\pgfqpoint{2.370052in}{0.845428in}}{\pgfqpoint{2.362152in}{0.842156in}}{\pgfqpoint{2.356328in}{0.836332in}}%
\pgfpathcurveto{\pgfqpoint{2.350504in}{0.830508in}}{\pgfqpoint{2.347231in}{0.822608in}}{\pgfqpoint{2.347231in}{0.814372in}}%
\pgfpathcurveto{\pgfqpoint{2.347231in}{0.806135in}}{\pgfqpoint{2.350504in}{0.798235in}}{\pgfqpoint{2.356328in}{0.792412in}}%
\pgfpathcurveto{\pgfqpoint{2.362152in}{0.786588in}}{\pgfqpoint{2.370052in}{0.783315in}}{\pgfqpoint{2.378288in}{0.783315in}}%
\pgfpathclose%
\pgfusepath{stroke,fill}%
\end{pgfscope}%
\begin{pgfscope}%
\pgfpathrectangle{\pgfqpoint{0.100000in}{0.212622in}}{\pgfqpoint{3.696000in}{3.696000in}}%
\pgfusepath{clip}%
\pgfsetbuttcap%
\pgfsetroundjoin%
\definecolor{currentfill}{rgb}{0.121569,0.466667,0.705882}%
\pgfsetfillcolor{currentfill}%
\pgfsetfillopacity{0.991992}%
\pgfsetlinewidth{1.003750pt}%
\definecolor{currentstroke}{rgb}{0.121569,0.466667,0.705882}%
\pgfsetstrokecolor{currentstroke}%
\pgfsetstrokeopacity{0.991992}%
\pgfsetdash{}{0pt}%
\pgfpathmoveto{\pgfqpoint{2.383846in}{0.781280in}}%
\pgfpathcurveto{\pgfqpoint{2.392083in}{0.781280in}}{\pgfqpoint{2.399983in}{0.784552in}}{\pgfqpoint{2.405807in}{0.790376in}}%
\pgfpathcurveto{\pgfqpoint{2.411630in}{0.796200in}}{\pgfqpoint{2.414903in}{0.804100in}}{\pgfqpoint{2.414903in}{0.812337in}}%
\pgfpathcurveto{\pgfqpoint{2.414903in}{0.820573in}}{\pgfqpoint{2.411630in}{0.828473in}}{\pgfqpoint{2.405807in}{0.834297in}}%
\pgfpathcurveto{\pgfqpoint{2.399983in}{0.840121in}}{\pgfqpoint{2.392083in}{0.843393in}}{\pgfqpoint{2.383846in}{0.843393in}}%
\pgfpathcurveto{\pgfqpoint{2.375610in}{0.843393in}}{\pgfqpoint{2.367710in}{0.840121in}}{\pgfqpoint{2.361886in}{0.834297in}}%
\pgfpathcurveto{\pgfqpoint{2.356062in}{0.828473in}}{\pgfqpoint{2.352790in}{0.820573in}}{\pgfqpoint{2.352790in}{0.812337in}}%
\pgfpathcurveto{\pgfqpoint{2.352790in}{0.804100in}}{\pgfqpoint{2.356062in}{0.796200in}}{\pgfqpoint{2.361886in}{0.790376in}}%
\pgfpathcurveto{\pgfqpoint{2.367710in}{0.784552in}}{\pgfqpoint{2.375610in}{0.781280in}}{\pgfqpoint{2.383846in}{0.781280in}}%
\pgfpathclose%
\pgfusepath{stroke,fill}%
\end{pgfscope}%
\begin{pgfscope}%
\pgfpathrectangle{\pgfqpoint{0.100000in}{0.212622in}}{\pgfqpoint{3.696000in}{3.696000in}}%
\pgfusepath{clip}%
\pgfsetbuttcap%
\pgfsetroundjoin%
\definecolor{currentfill}{rgb}{0.121569,0.466667,0.705882}%
\pgfsetfillcolor{currentfill}%
\pgfsetfillopacity{0.992814}%
\pgfsetlinewidth{1.003750pt}%
\definecolor{currentstroke}{rgb}{0.121569,0.466667,0.705882}%
\pgfsetstrokecolor{currentstroke}%
\pgfsetstrokeopacity{0.992814}%
\pgfsetdash{}{0pt}%
\pgfpathmoveto{\pgfqpoint{2.444547in}{0.781342in}}%
\pgfpathcurveto{\pgfqpoint{2.452784in}{0.781342in}}{\pgfqpoint{2.460684in}{0.784614in}}{\pgfqpoint{2.466508in}{0.790438in}}%
\pgfpathcurveto{\pgfqpoint{2.472332in}{0.796262in}}{\pgfqpoint{2.475604in}{0.804162in}}{\pgfqpoint{2.475604in}{0.812399in}}%
\pgfpathcurveto{\pgfqpoint{2.475604in}{0.820635in}}{\pgfqpoint{2.472332in}{0.828535in}}{\pgfqpoint{2.466508in}{0.834359in}}%
\pgfpathcurveto{\pgfqpoint{2.460684in}{0.840183in}}{\pgfqpoint{2.452784in}{0.843455in}}{\pgfqpoint{2.444547in}{0.843455in}}%
\pgfpathcurveto{\pgfqpoint{2.436311in}{0.843455in}}{\pgfqpoint{2.428411in}{0.840183in}}{\pgfqpoint{2.422587in}{0.834359in}}%
\pgfpathcurveto{\pgfqpoint{2.416763in}{0.828535in}}{\pgfqpoint{2.413491in}{0.820635in}}{\pgfqpoint{2.413491in}{0.812399in}}%
\pgfpathcurveto{\pgfqpoint{2.413491in}{0.804162in}}{\pgfqpoint{2.416763in}{0.796262in}}{\pgfqpoint{2.422587in}{0.790438in}}%
\pgfpathcurveto{\pgfqpoint{2.428411in}{0.784614in}}{\pgfqpoint{2.436311in}{0.781342in}}{\pgfqpoint{2.444547in}{0.781342in}}%
\pgfpathclose%
\pgfusepath{stroke,fill}%
\end{pgfscope}%
\begin{pgfscope}%
\pgfpathrectangle{\pgfqpoint{0.100000in}{0.212622in}}{\pgfqpoint{3.696000in}{3.696000in}}%
\pgfusepath{clip}%
\pgfsetbuttcap%
\pgfsetroundjoin%
\definecolor{currentfill}{rgb}{0.121569,0.466667,0.705882}%
\pgfsetfillcolor{currentfill}%
\pgfsetfillopacity{0.992987}%
\pgfsetlinewidth{1.003750pt}%
\definecolor{currentstroke}{rgb}{0.121569,0.466667,0.705882}%
\pgfsetstrokecolor{currentstroke}%
\pgfsetstrokeopacity{0.992987}%
\pgfsetdash{}{0pt}%
\pgfpathmoveto{\pgfqpoint{2.389050in}{0.779213in}}%
\pgfpathcurveto{\pgfqpoint{2.397286in}{0.779213in}}{\pgfqpoint{2.405186in}{0.782485in}}{\pgfqpoint{2.411010in}{0.788309in}}%
\pgfpathcurveto{\pgfqpoint{2.416834in}{0.794133in}}{\pgfqpoint{2.420106in}{0.802033in}}{\pgfqpoint{2.420106in}{0.810269in}}%
\pgfpathcurveto{\pgfqpoint{2.420106in}{0.818505in}}{\pgfqpoint{2.416834in}{0.826405in}}{\pgfqpoint{2.411010in}{0.832229in}}%
\pgfpathcurveto{\pgfqpoint{2.405186in}{0.838053in}}{\pgfqpoint{2.397286in}{0.841326in}}{\pgfqpoint{2.389050in}{0.841326in}}%
\pgfpathcurveto{\pgfqpoint{2.380814in}{0.841326in}}{\pgfqpoint{2.372914in}{0.838053in}}{\pgfqpoint{2.367090in}{0.832229in}}%
\pgfpathcurveto{\pgfqpoint{2.361266in}{0.826405in}}{\pgfqpoint{2.357993in}{0.818505in}}{\pgfqpoint{2.357993in}{0.810269in}}%
\pgfpathcurveto{\pgfqpoint{2.357993in}{0.802033in}}{\pgfqpoint{2.361266in}{0.794133in}}{\pgfqpoint{2.367090in}{0.788309in}}%
\pgfpathcurveto{\pgfqpoint{2.372914in}{0.782485in}}{\pgfqpoint{2.380814in}{0.779213in}}{\pgfqpoint{2.389050in}{0.779213in}}%
\pgfpathclose%
\pgfusepath{stroke,fill}%
\end{pgfscope}%
\begin{pgfscope}%
\pgfpathrectangle{\pgfqpoint{0.100000in}{0.212622in}}{\pgfqpoint{3.696000in}{3.696000in}}%
\pgfusepath{clip}%
\pgfsetbuttcap%
\pgfsetroundjoin%
\definecolor{currentfill}{rgb}{0.121569,0.466667,0.705882}%
\pgfsetfillcolor{currentfill}%
\pgfsetfillopacity{0.993893}%
\pgfsetlinewidth{1.003750pt}%
\definecolor{currentstroke}{rgb}{0.121569,0.466667,0.705882}%
\pgfsetstrokecolor{currentstroke}%
\pgfsetstrokeopacity{0.993893}%
\pgfsetdash{}{0pt}%
\pgfpathmoveto{\pgfqpoint{2.393901in}{0.777161in}}%
\pgfpathcurveto{\pgfqpoint{2.402137in}{0.777161in}}{\pgfqpoint{2.410037in}{0.780433in}}{\pgfqpoint{2.415861in}{0.786257in}}%
\pgfpathcurveto{\pgfqpoint{2.421685in}{0.792081in}}{\pgfqpoint{2.424957in}{0.799981in}}{\pgfqpoint{2.424957in}{0.808218in}}%
\pgfpathcurveto{\pgfqpoint{2.424957in}{0.816454in}}{\pgfqpoint{2.421685in}{0.824354in}}{\pgfqpoint{2.415861in}{0.830178in}}%
\pgfpathcurveto{\pgfqpoint{2.410037in}{0.836002in}}{\pgfqpoint{2.402137in}{0.839274in}}{\pgfqpoint{2.393901in}{0.839274in}}%
\pgfpathcurveto{\pgfqpoint{2.385665in}{0.839274in}}{\pgfqpoint{2.377764in}{0.836002in}}{\pgfqpoint{2.371941in}{0.830178in}}%
\pgfpathcurveto{\pgfqpoint{2.366117in}{0.824354in}}{\pgfqpoint{2.362844in}{0.816454in}}{\pgfqpoint{2.362844in}{0.808218in}}%
\pgfpathcurveto{\pgfqpoint{2.362844in}{0.799981in}}{\pgfqpoint{2.366117in}{0.792081in}}{\pgfqpoint{2.371941in}{0.786257in}}%
\pgfpathcurveto{\pgfqpoint{2.377764in}{0.780433in}}{\pgfqpoint{2.385665in}{0.777161in}}{\pgfqpoint{2.393901in}{0.777161in}}%
\pgfpathclose%
\pgfusepath{stroke,fill}%
\end{pgfscope}%
\begin{pgfscope}%
\pgfpathrectangle{\pgfqpoint{0.100000in}{0.212622in}}{\pgfqpoint{3.696000in}{3.696000in}}%
\pgfusepath{clip}%
\pgfsetbuttcap%
\pgfsetroundjoin%
\definecolor{currentfill}{rgb}{0.121569,0.466667,0.705882}%
\pgfsetfillcolor{currentfill}%
\pgfsetfillopacity{0.994728}%
\pgfsetlinewidth{1.003750pt}%
\definecolor{currentstroke}{rgb}{0.121569,0.466667,0.705882}%
\pgfsetstrokecolor{currentstroke}%
\pgfsetstrokeopacity{0.994728}%
\pgfsetdash{}{0pt}%
\pgfpathmoveto{\pgfqpoint{2.398435in}{0.775152in}}%
\pgfpathcurveto{\pgfqpoint{2.406671in}{0.775152in}}{\pgfqpoint{2.414571in}{0.778424in}}{\pgfqpoint{2.420395in}{0.784248in}}%
\pgfpathcurveto{\pgfqpoint{2.426219in}{0.790072in}}{\pgfqpoint{2.429492in}{0.797972in}}{\pgfqpoint{2.429492in}{0.806209in}}%
\pgfpathcurveto{\pgfqpoint{2.429492in}{0.814445in}}{\pgfqpoint{2.426219in}{0.822345in}}{\pgfqpoint{2.420395in}{0.828169in}}%
\pgfpathcurveto{\pgfqpoint{2.414571in}{0.833993in}}{\pgfqpoint{2.406671in}{0.837265in}}{\pgfqpoint{2.398435in}{0.837265in}}%
\pgfpathcurveto{\pgfqpoint{2.390199in}{0.837265in}}{\pgfqpoint{2.382299in}{0.833993in}}{\pgfqpoint{2.376475in}{0.828169in}}%
\pgfpathcurveto{\pgfqpoint{2.370651in}{0.822345in}}{\pgfqpoint{2.367379in}{0.814445in}}{\pgfqpoint{2.367379in}{0.806209in}}%
\pgfpathcurveto{\pgfqpoint{2.367379in}{0.797972in}}{\pgfqpoint{2.370651in}{0.790072in}}{\pgfqpoint{2.376475in}{0.784248in}}%
\pgfpathcurveto{\pgfqpoint{2.382299in}{0.778424in}}{\pgfqpoint{2.390199in}{0.775152in}}{\pgfqpoint{2.398435in}{0.775152in}}%
\pgfpathclose%
\pgfusepath{stroke,fill}%
\end{pgfscope}%
\begin{pgfscope}%
\pgfpathrectangle{\pgfqpoint{0.100000in}{0.212622in}}{\pgfqpoint{3.696000in}{3.696000in}}%
\pgfusepath{clip}%
\pgfsetbuttcap%
\pgfsetroundjoin%
\definecolor{currentfill}{rgb}{0.121569,0.466667,0.705882}%
\pgfsetfillcolor{currentfill}%
\pgfsetfillopacity{0.995049}%
\pgfsetlinewidth{1.003750pt}%
\definecolor{currentstroke}{rgb}{0.121569,0.466667,0.705882}%
\pgfsetstrokecolor{currentstroke}%
\pgfsetstrokeopacity{0.995049}%
\pgfsetdash{}{0pt}%
\pgfpathmoveto{\pgfqpoint{2.446247in}{0.773466in}}%
\pgfpathcurveto{\pgfqpoint{2.454483in}{0.773466in}}{\pgfqpoint{2.462383in}{0.776738in}}{\pgfqpoint{2.468207in}{0.782562in}}%
\pgfpathcurveto{\pgfqpoint{2.474031in}{0.788386in}}{\pgfqpoint{2.477303in}{0.796286in}}{\pgfqpoint{2.477303in}{0.804522in}}%
\pgfpathcurveto{\pgfqpoint{2.477303in}{0.812759in}}{\pgfqpoint{2.474031in}{0.820659in}}{\pgfqpoint{2.468207in}{0.826483in}}%
\pgfpathcurveto{\pgfqpoint{2.462383in}{0.832306in}}{\pgfqpoint{2.454483in}{0.835579in}}{\pgfqpoint{2.446247in}{0.835579in}}%
\pgfpathcurveto{\pgfqpoint{2.438011in}{0.835579in}}{\pgfqpoint{2.430111in}{0.832306in}}{\pgfqpoint{2.424287in}{0.826483in}}%
\pgfpathcurveto{\pgfqpoint{2.418463in}{0.820659in}}{\pgfqpoint{2.415190in}{0.812759in}}{\pgfqpoint{2.415190in}{0.804522in}}%
\pgfpathcurveto{\pgfqpoint{2.415190in}{0.796286in}}{\pgfqpoint{2.418463in}{0.788386in}}{\pgfqpoint{2.424287in}{0.782562in}}%
\pgfpathcurveto{\pgfqpoint{2.430111in}{0.776738in}}{\pgfqpoint{2.438011in}{0.773466in}}{\pgfqpoint{2.446247in}{0.773466in}}%
\pgfpathclose%
\pgfusepath{stroke,fill}%
\end{pgfscope}%
\begin{pgfscope}%
\pgfpathrectangle{\pgfqpoint{0.100000in}{0.212622in}}{\pgfqpoint{3.696000in}{3.696000in}}%
\pgfusepath{clip}%
\pgfsetbuttcap%
\pgfsetroundjoin%
\definecolor{currentfill}{rgb}{0.121569,0.466667,0.705882}%
\pgfsetfillcolor{currentfill}%
\pgfsetfillopacity{0.995509}%
\pgfsetlinewidth{1.003750pt}%
\definecolor{currentstroke}{rgb}{0.121569,0.466667,0.705882}%
\pgfsetstrokecolor{currentstroke}%
\pgfsetstrokeopacity{0.995509}%
\pgfsetdash{}{0pt}%
\pgfpathmoveto{\pgfqpoint{2.402629in}{0.773197in}}%
\pgfpathcurveto{\pgfqpoint{2.410865in}{0.773197in}}{\pgfqpoint{2.418765in}{0.776469in}}{\pgfqpoint{2.424589in}{0.782293in}}%
\pgfpathcurveto{\pgfqpoint{2.430413in}{0.788117in}}{\pgfqpoint{2.433685in}{0.796017in}}{\pgfqpoint{2.433685in}{0.804253in}}%
\pgfpathcurveto{\pgfqpoint{2.433685in}{0.812490in}}{\pgfqpoint{2.430413in}{0.820390in}}{\pgfqpoint{2.424589in}{0.826214in}}%
\pgfpathcurveto{\pgfqpoint{2.418765in}{0.832037in}}{\pgfqpoint{2.410865in}{0.835310in}}{\pgfqpoint{2.402629in}{0.835310in}}%
\pgfpathcurveto{\pgfqpoint{2.394392in}{0.835310in}}{\pgfqpoint{2.386492in}{0.832037in}}{\pgfqpoint{2.380668in}{0.826214in}}%
\pgfpathcurveto{\pgfqpoint{2.374844in}{0.820390in}}{\pgfqpoint{2.371572in}{0.812490in}}{\pgfqpoint{2.371572in}{0.804253in}}%
\pgfpathcurveto{\pgfqpoint{2.371572in}{0.796017in}}{\pgfqpoint{2.374844in}{0.788117in}}{\pgfqpoint{2.380668in}{0.782293in}}%
\pgfpathcurveto{\pgfqpoint{2.386492in}{0.776469in}}{\pgfqpoint{2.394392in}{0.773197in}}{\pgfqpoint{2.402629in}{0.773197in}}%
\pgfpathclose%
\pgfusepath{stroke,fill}%
\end{pgfscope}%
\begin{pgfscope}%
\pgfpathrectangle{\pgfqpoint{0.100000in}{0.212622in}}{\pgfqpoint{3.696000in}{3.696000in}}%
\pgfusepath{clip}%
\pgfsetbuttcap%
\pgfsetroundjoin%
\definecolor{currentfill}{rgb}{0.121569,0.466667,0.705882}%
\pgfsetfillcolor{currentfill}%
\pgfsetfillopacity{0.996209}%
\pgfsetlinewidth{1.003750pt}%
\definecolor{currentstroke}{rgb}{0.121569,0.466667,0.705882}%
\pgfsetstrokecolor{currentstroke}%
\pgfsetstrokeopacity{0.996209}%
\pgfsetdash{}{0pt}%
\pgfpathmoveto{\pgfqpoint{2.406527in}{0.771395in}}%
\pgfpathcurveto{\pgfqpoint{2.414764in}{0.771395in}}{\pgfqpoint{2.422664in}{0.774667in}}{\pgfqpoint{2.428488in}{0.780491in}}%
\pgfpathcurveto{\pgfqpoint{2.434311in}{0.786315in}}{\pgfqpoint{2.437584in}{0.794215in}}{\pgfqpoint{2.437584in}{0.802451in}}%
\pgfpathcurveto{\pgfqpoint{2.437584in}{0.810687in}}{\pgfqpoint{2.434311in}{0.818588in}}{\pgfqpoint{2.428488in}{0.824411in}}%
\pgfpathcurveto{\pgfqpoint{2.422664in}{0.830235in}}{\pgfqpoint{2.414764in}{0.833508in}}{\pgfqpoint{2.406527in}{0.833508in}}%
\pgfpathcurveto{\pgfqpoint{2.398291in}{0.833508in}}{\pgfqpoint{2.390391in}{0.830235in}}{\pgfqpoint{2.384567in}{0.824411in}}%
\pgfpathcurveto{\pgfqpoint{2.378743in}{0.818588in}}{\pgfqpoint{2.375471in}{0.810687in}}{\pgfqpoint{2.375471in}{0.802451in}}%
\pgfpathcurveto{\pgfqpoint{2.375471in}{0.794215in}}{\pgfqpoint{2.378743in}{0.786315in}}{\pgfqpoint{2.384567in}{0.780491in}}%
\pgfpathcurveto{\pgfqpoint{2.390391in}{0.774667in}}{\pgfqpoint{2.398291in}{0.771395in}}{\pgfqpoint{2.406527in}{0.771395in}}%
\pgfpathclose%
\pgfusepath{stroke,fill}%
\end{pgfscope}%
\begin{pgfscope}%
\pgfpathrectangle{\pgfqpoint{0.100000in}{0.212622in}}{\pgfqpoint{3.696000in}{3.696000in}}%
\pgfusepath{clip}%
\pgfsetbuttcap%
\pgfsetroundjoin%
\definecolor{currentfill}{rgb}{0.121569,0.466667,0.705882}%
\pgfsetfillcolor{currentfill}%
\pgfsetfillopacity{0.996280}%
\pgfsetlinewidth{1.003750pt}%
\definecolor{currentstroke}{rgb}{0.121569,0.466667,0.705882}%
\pgfsetstrokecolor{currentstroke}%
\pgfsetstrokeopacity{0.996280}%
\pgfsetdash{}{0pt}%
\pgfpathmoveto{\pgfqpoint{2.447060in}{0.769024in}}%
\pgfpathcurveto{\pgfqpoint{2.455297in}{0.769024in}}{\pgfqpoint{2.463197in}{0.772296in}}{\pgfqpoint{2.469021in}{0.778120in}}%
\pgfpathcurveto{\pgfqpoint{2.474844in}{0.783944in}}{\pgfqpoint{2.478117in}{0.791844in}}{\pgfqpoint{2.478117in}{0.800080in}}%
\pgfpathcurveto{\pgfqpoint{2.478117in}{0.808316in}}{\pgfqpoint{2.474844in}{0.816216in}}{\pgfqpoint{2.469021in}{0.822040in}}%
\pgfpathcurveto{\pgfqpoint{2.463197in}{0.827864in}}{\pgfqpoint{2.455297in}{0.831137in}}{\pgfqpoint{2.447060in}{0.831137in}}%
\pgfpathcurveto{\pgfqpoint{2.438824in}{0.831137in}}{\pgfqpoint{2.430924in}{0.827864in}}{\pgfqpoint{2.425100in}{0.822040in}}%
\pgfpathcurveto{\pgfqpoint{2.419276in}{0.816216in}}{\pgfqpoint{2.416004in}{0.808316in}}{\pgfqpoint{2.416004in}{0.800080in}}%
\pgfpathcurveto{\pgfqpoint{2.416004in}{0.791844in}}{\pgfqpoint{2.419276in}{0.783944in}}{\pgfqpoint{2.425100in}{0.778120in}}%
\pgfpathcurveto{\pgfqpoint{2.430924in}{0.772296in}}{\pgfqpoint{2.438824in}{0.769024in}}{\pgfqpoint{2.447060in}{0.769024in}}%
\pgfpathclose%
\pgfusepath{stroke,fill}%
\end{pgfscope}%
\begin{pgfscope}%
\pgfpathrectangle{\pgfqpoint{0.100000in}{0.212622in}}{\pgfqpoint{3.696000in}{3.696000in}}%
\pgfusepath{clip}%
\pgfsetbuttcap%
\pgfsetroundjoin%
\definecolor{currentfill}{rgb}{0.121569,0.466667,0.705882}%
\pgfsetfillcolor{currentfill}%
\pgfsetfillopacity{0.996827}%
\pgfsetlinewidth{1.003750pt}%
\definecolor{currentstroke}{rgb}{0.121569,0.466667,0.705882}%
\pgfsetstrokecolor{currentstroke}%
\pgfsetstrokeopacity{0.996827}%
\pgfsetdash{}{0pt}%
\pgfpathmoveto{\pgfqpoint{2.410166in}{0.769680in}}%
\pgfpathcurveto{\pgfqpoint{2.418402in}{0.769680in}}{\pgfqpoint{2.426302in}{0.772952in}}{\pgfqpoint{2.432126in}{0.778776in}}%
\pgfpathcurveto{\pgfqpoint{2.437950in}{0.784600in}}{\pgfqpoint{2.441222in}{0.792500in}}{\pgfqpoint{2.441222in}{0.800737in}}%
\pgfpathcurveto{\pgfqpoint{2.441222in}{0.808973in}}{\pgfqpoint{2.437950in}{0.816873in}}{\pgfqpoint{2.432126in}{0.822697in}}%
\pgfpathcurveto{\pgfqpoint{2.426302in}{0.828521in}}{\pgfqpoint{2.418402in}{0.831793in}}{\pgfqpoint{2.410166in}{0.831793in}}%
\pgfpathcurveto{\pgfqpoint{2.401930in}{0.831793in}}{\pgfqpoint{2.394030in}{0.828521in}}{\pgfqpoint{2.388206in}{0.822697in}}%
\pgfpathcurveto{\pgfqpoint{2.382382in}{0.816873in}}{\pgfqpoint{2.379109in}{0.808973in}}{\pgfqpoint{2.379109in}{0.800737in}}%
\pgfpathcurveto{\pgfqpoint{2.379109in}{0.792500in}}{\pgfqpoint{2.382382in}{0.784600in}}{\pgfqpoint{2.388206in}{0.778776in}}%
\pgfpathcurveto{\pgfqpoint{2.394030in}{0.772952in}}{\pgfqpoint{2.401930in}{0.769680in}}{\pgfqpoint{2.410166in}{0.769680in}}%
\pgfpathclose%
\pgfusepath{stroke,fill}%
\end{pgfscope}%
\begin{pgfscope}%
\pgfpathrectangle{\pgfqpoint{0.100000in}{0.212622in}}{\pgfqpoint{3.696000in}{3.696000in}}%
\pgfusepath{clip}%
\pgfsetbuttcap%
\pgfsetroundjoin%
\definecolor{currentfill}{rgb}{0.121569,0.466667,0.705882}%
\pgfsetfillcolor{currentfill}%
\pgfsetfillopacity{0.996864}%
\pgfsetlinewidth{1.003750pt}%
\definecolor{currentstroke}{rgb}{0.121569,0.466667,0.705882}%
\pgfsetstrokecolor{currentstroke}%
\pgfsetstrokeopacity{0.996864}%
\pgfsetdash{}{0pt}%
\pgfpathmoveto{\pgfqpoint{2.447395in}{0.766159in}}%
\pgfpathcurveto{\pgfqpoint{2.455631in}{0.766159in}}{\pgfqpoint{2.463531in}{0.769431in}}{\pgfqpoint{2.469355in}{0.775255in}}%
\pgfpathcurveto{\pgfqpoint{2.475179in}{0.781079in}}{\pgfqpoint{2.478451in}{0.788979in}}{\pgfqpoint{2.478451in}{0.797215in}}%
\pgfpathcurveto{\pgfqpoint{2.478451in}{0.805452in}}{\pgfqpoint{2.475179in}{0.813352in}}{\pgfqpoint{2.469355in}{0.819176in}}%
\pgfpathcurveto{\pgfqpoint{2.463531in}{0.825000in}}{\pgfqpoint{2.455631in}{0.828272in}}{\pgfqpoint{2.447395in}{0.828272in}}%
\pgfpathcurveto{\pgfqpoint{2.439158in}{0.828272in}}{\pgfqpoint{2.431258in}{0.825000in}}{\pgfqpoint{2.425435in}{0.819176in}}%
\pgfpathcurveto{\pgfqpoint{2.419611in}{0.813352in}}{\pgfqpoint{2.416338in}{0.805452in}}{\pgfqpoint{2.416338in}{0.797215in}}%
\pgfpathcurveto{\pgfqpoint{2.416338in}{0.788979in}}{\pgfqpoint{2.419611in}{0.781079in}}{\pgfqpoint{2.425435in}{0.775255in}}%
\pgfpathcurveto{\pgfqpoint{2.431258in}{0.769431in}}{\pgfqpoint{2.439158in}{0.766159in}}{\pgfqpoint{2.447395in}{0.766159in}}%
\pgfpathclose%
\pgfusepath{stroke,fill}%
\end{pgfscope}%
\begin{pgfscope}%
\pgfpathrectangle{\pgfqpoint{0.100000in}{0.212622in}}{\pgfqpoint{3.696000in}{3.696000in}}%
\pgfusepath{clip}%
\pgfsetbuttcap%
\pgfsetroundjoin%
\definecolor{currentfill}{rgb}{0.121569,0.466667,0.705882}%
\pgfsetfillcolor{currentfill}%
\pgfsetfillopacity{0.997204}%
\pgfsetlinewidth{1.003750pt}%
\definecolor{currentstroke}{rgb}{0.121569,0.466667,0.705882}%
\pgfsetstrokecolor{currentstroke}%
\pgfsetstrokeopacity{0.997204}%
\pgfsetdash{}{0pt}%
\pgfpathmoveto{\pgfqpoint{2.447463in}{0.764568in}}%
\pgfpathcurveto{\pgfqpoint{2.455699in}{0.764568in}}{\pgfqpoint{2.463599in}{0.767841in}}{\pgfqpoint{2.469423in}{0.773665in}}%
\pgfpathcurveto{\pgfqpoint{2.475247in}{0.779489in}}{\pgfqpoint{2.478519in}{0.787389in}}{\pgfqpoint{2.478519in}{0.795625in}}%
\pgfpathcurveto{\pgfqpoint{2.478519in}{0.803861in}}{\pgfqpoint{2.475247in}{0.811761in}}{\pgfqpoint{2.469423in}{0.817585in}}%
\pgfpathcurveto{\pgfqpoint{2.463599in}{0.823409in}}{\pgfqpoint{2.455699in}{0.826681in}}{\pgfqpoint{2.447463in}{0.826681in}}%
\pgfpathcurveto{\pgfqpoint{2.439227in}{0.826681in}}{\pgfqpoint{2.431327in}{0.823409in}}{\pgfqpoint{2.425503in}{0.817585in}}%
\pgfpathcurveto{\pgfqpoint{2.419679in}{0.811761in}}{\pgfqpoint{2.416406in}{0.803861in}}{\pgfqpoint{2.416406in}{0.795625in}}%
\pgfpathcurveto{\pgfqpoint{2.416406in}{0.787389in}}{\pgfqpoint{2.419679in}{0.779489in}}{\pgfqpoint{2.425503in}{0.773665in}}%
\pgfpathcurveto{\pgfqpoint{2.431327in}{0.767841in}}{\pgfqpoint{2.439227in}{0.764568in}}{\pgfqpoint{2.447463in}{0.764568in}}%
\pgfpathclose%
\pgfusepath{stroke,fill}%
\end{pgfscope}%
\begin{pgfscope}%
\pgfpathrectangle{\pgfqpoint{0.100000in}{0.212622in}}{\pgfqpoint{3.696000in}{3.696000in}}%
\pgfusepath{clip}%
\pgfsetbuttcap%
\pgfsetroundjoin%
\definecolor{currentfill}{rgb}{0.121569,0.466667,0.705882}%
\pgfsetfillcolor{currentfill}%
\pgfsetfillopacity{0.997384}%
\pgfsetlinewidth{1.003750pt}%
\definecolor{currentstroke}{rgb}{0.121569,0.466667,0.705882}%
\pgfsetstrokecolor{currentstroke}%
\pgfsetstrokeopacity{0.997384}%
\pgfsetdash{}{0pt}%
\pgfpathmoveto{\pgfqpoint{2.447451in}{0.763641in}}%
\pgfpathcurveto{\pgfqpoint{2.455687in}{0.763641in}}{\pgfqpoint{2.463587in}{0.766914in}}{\pgfqpoint{2.469411in}{0.772738in}}%
\pgfpathcurveto{\pgfqpoint{2.475235in}{0.778562in}}{\pgfqpoint{2.478507in}{0.786462in}}{\pgfqpoint{2.478507in}{0.794698in}}%
\pgfpathcurveto{\pgfqpoint{2.478507in}{0.802934in}}{\pgfqpoint{2.475235in}{0.810834in}}{\pgfqpoint{2.469411in}{0.816658in}}%
\pgfpathcurveto{\pgfqpoint{2.463587in}{0.822482in}}{\pgfqpoint{2.455687in}{0.825754in}}{\pgfqpoint{2.447451in}{0.825754in}}%
\pgfpathcurveto{\pgfqpoint{2.439215in}{0.825754in}}{\pgfqpoint{2.431315in}{0.822482in}}{\pgfqpoint{2.425491in}{0.816658in}}%
\pgfpathcurveto{\pgfqpoint{2.419667in}{0.810834in}}{\pgfqpoint{2.416394in}{0.802934in}}{\pgfqpoint{2.416394in}{0.794698in}}%
\pgfpathcurveto{\pgfqpoint{2.416394in}{0.786462in}}{\pgfqpoint{2.419667in}{0.778562in}}{\pgfqpoint{2.425491in}{0.772738in}}%
\pgfpathcurveto{\pgfqpoint{2.431315in}{0.766914in}}{\pgfqpoint{2.439215in}{0.763641in}}{\pgfqpoint{2.447451in}{0.763641in}}%
\pgfpathclose%
\pgfusepath{stroke,fill}%
\end{pgfscope}%
\begin{pgfscope}%
\pgfpathrectangle{\pgfqpoint{0.100000in}{0.212622in}}{\pgfqpoint{3.696000in}{3.696000in}}%
\pgfusepath{clip}%
\pgfsetbuttcap%
\pgfsetroundjoin%
\definecolor{currentfill}{rgb}{0.121569,0.466667,0.705882}%
\pgfsetfillcolor{currentfill}%
\pgfsetfillopacity{0.997388}%
\pgfsetlinewidth{1.003750pt}%
\definecolor{currentstroke}{rgb}{0.121569,0.466667,0.705882}%
\pgfsetstrokecolor{currentstroke}%
\pgfsetstrokeopacity{0.997388}%
\pgfsetdash{}{0pt}%
\pgfpathmoveto{\pgfqpoint{2.413545in}{0.768115in}}%
\pgfpathcurveto{\pgfqpoint{2.421782in}{0.768115in}}{\pgfqpoint{2.429682in}{0.771387in}}{\pgfqpoint{2.435506in}{0.777211in}}%
\pgfpathcurveto{\pgfqpoint{2.441329in}{0.783035in}}{\pgfqpoint{2.444602in}{0.790935in}}{\pgfqpoint{2.444602in}{0.799171in}}%
\pgfpathcurveto{\pgfqpoint{2.444602in}{0.807408in}}{\pgfqpoint{2.441329in}{0.815308in}}{\pgfqpoint{2.435506in}{0.821132in}}%
\pgfpathcurveto{\pgfqpoint{2.429682in}{0.826956in}}{\pgfqpoint{2.421782in}{0.830228in}}{\pgfqpoint{2.413545in}{0.830228in}}%
\pgfpathcurveto{\pgfqpoint{2.405309in}{0.830228in}}{\pgfqpoint{2.397409in}{0.826956in}}{\pgfqpoint{2.391585in}{0.821132in}}%
\pgfpathcurveto{\pgfqpoint{2.385761in}{0.815308in}}{\pgfqpoint{2.382489in}{0.807408in}}{\pgfqpoint{2.382489in}{0.799171in}}%
\pgfpathcurveto{\pgfqpoint{2.382489in}{0.790935in}}{\pgfqpoint{2.385761in}{0.783035in}}{\pgfqpoint{2.391585in}{0.777211in}}%
\pgfpathcurveto{\pgfqpoint{2.397409in}{0.771387in}}{\pgfqpoint{2.405309in}{0.768115in}}{\pgfqpoint{2.413545in}{0.768115in}}%
\pgfpathclose%
\pgfusepath{stroke,fill}%
\end{pgfscope}%
\begin{pgfscope}%
\pgfpathrectangle{\pgfqpoint{0.100000in}{0.212622in}}{\pgfqpoint{3.696000in}{3.696000in}}%
\pgfusepath{clip}%
\pgfsetbuttcap%
\pgfsetroundjoin%
\definecolor{currentfill}{rgb}{0.121569,0.466667,0.705882}%
\pgfsetfillcolor{currentfill}%
\pgfsetfillopacity{0.997484}%
\pgfsetlinewidth{1.003750pt}%
\definecolor{currentstroke}{rgb}{0.121569,0.466667,0.705882}%
\pgfsetstrokecolor{currentstroke}%
\pgfsetstrokeopacity{0.997484}%
\pgfsetdash{}{0pt}%
\pgfpathmoveto{\pgfqpoint{2.447424in}{0.763130in}}%
\pgfpathcurveto{\pgfqpoint{2.455660in}{0.763130in}}{\pgfqpoint{2.463560in}{0.766402in}}{\pgfqpoint{2.469384in}{0.772226in}}%
\pgfpathcurveto{\pgfqpoint{2.475208in}{0.778050in}}{\pgfqpoint{2.478480in}{0.785950in}}{\pgfqpoint{2.478480in}{0.794186in}}%
\pgfpathcurveto{\pgfqpoint{2.478480in}{0.802423in}}{\pgfqpoint{2.475208in}{0.810323in}}{\pgfqpoint{2.469384in}{0.816147in}}%
\pgfpathcurveto{\pgfqpoint{2.463560in}{0.821970in}}{\pgfqpoint{2.455660in}{0.825243in}}{\pgfqpoint{2.447424in}{0.825243in}}%
\pgfpathcurveto{\pgfqpoint{2.439187in}{0.825243in}}{\pgfqpoint{2.431287in}{0.821970in}}{\pgfqpoint{2.425463in}{0.816147in}}%
\pgfpathcurveto{\pgfqpoint{2.419639in}{0.810323in}}{\pgfqpoint{2.416367in}{0.802423in}}{\pgfqpoint{2.416367in}{0.794186in}}%
\pgfpathcurveto{\pgfqpoint{2.416367in}{0.785950in}}{\pgfqpoint{2.419639in}{0.778050in}}{\pgfqpoint{2.425463in}{0.772226in}}%
\pgfpathcurveto{\pgfqpoint{2.431287in}{0.766402in}}{\pgfqpoint{2.439187in}{0.763130in}}{\pgfqpoint{2.447424in}{0.763130in}}%
\pgfpathclose%
\pgfusepath{stroke,fill}%
\end{pgfscope}%
\begin{pgfscope}%
\pgfpathrectangle{\pgfqpoint{0.100000in}{0.212622in}}{\pgfqpoint{3.696000in}{3.696000in}}%
\pgfusepath{clip}%
\pgfsetbuttcap%
\pgfsetroundjoin%
\definecolor{currentfill}{rgb}{0.121569,0.466667,0.705882}%
\pgfsetfillcolor{currentfill}%
\pgfsetfillopacity{0.997542}%
\pgfsetlinewidth{1.003750pt}%
\definecolor{currentstroke}{rgb}{0.121569,0.466667,0.705882}%
\pgfsetstrokecolor{currentstroke}%
\pgfsetstrokeopacity{0.997542}%
\pgfsetdash{}{0pt}%
\pgfpathmoveto{\pgfqpoint{2.447393in}{0.762855in}}%
\pgfpathcurveto{\pgfqpoint{2.455630in}{0.762855in}}{\pgfqpoint{2.463530in}{0.766127in}}{\pgfqpoint{2.469353in}{0.771951in}}%
\pgfpathcurveto{\pgfqpoint{2.475177in}{0.777775in}}{\pgfqpoint{2.478450in}{0.785675in}}{\pgfqpoint{2.478450in}{0.793911in}}%
\pgfpathcurveto{\pgfqpoint{2.478450in}{0.802147in}}{\pgfqpoint{2.475177in}{0.810047in}}{\pgfqpoint{2.469353in}{0.815871in}}%
\pgfpathcurveto{\pgfqpoint{2.463530in}{0.821695in}}{\pgfqpoint{2.455630in}{0.824968in}}{\pgfqpoint{2.447393in}{0.824968in}}%
\pgfpathcurveto{\pgfqpoint{2.439157in}{0.824968in}}{\pgfqpoint{2.431257in}{0.821695in}}{\pgfqpoint{2.425433in}{0.815871in}}%
\pgfpathcurveto{\pgfqpoint{2.419609in}{0.810047in}}{\pgfqpoint{2.416337in}{0.802147in}}{\pgfqpoint{2.416337in}{0.793911in}}%
\pgfpathcurveto{\pgfqpoint{2.416337in}{0.785675in}}{\pgfqpoint{2.419609in}{0.777775in}}{\pgfqpoint{2.425433in}{0.771951in}}%
\pgfpathcurveto{\pgfqpoint{2.431257in}{0.766127in}}{\pgfqpoint{2.439157in}{0.762855in}}{\pgfqpoint{2.447393in}{0.762855in}}%
\pgfpathclose%
\pgfusepath{stroke,fill}%
\end{pgfscope}%
\begin{pgfscope}%
\pgfpathrectangle{\pgfqpoint{0.100000in}{0.212622in}}{\pgfqpoint{3.696000in}{3.696000in}}%
\pgfusepath{clip}%
\pgfsetbuttcap%
\pgfsetroundjoin%
\definecolor{currentfill}{rgb}{0.121569,0.466667,0.705882}%
\pgfsetfillcolor{currentfill}%
\pgfsetfillopacity{0.997712}%
\pgfsetlinewidth{1.003750pt}%
\definecolor{currentstroke}{rgb}{0.121569,0.466667,0.705882}%
\pgfsetstrokecolor{currentstroke}%
\pgfsetstrokeopacity{0.997712}%
\pgfsetdash{}{0pt}%
\pgfpathmoveto{\pgfqpoint{2.447270in}{0.762104in}}%
\pgfpathcurveto{\pgfqpoint{2.455507in}{0.762104in}}{\pgfqpoint{2.463407in}{0.765376in}}{\pgfqpoint{2.469231in}{0.771200in}}%
\pgfpathcurveto{\pgfqpoint{2.475055in}{0.777024in}}{\pgfqpoint{2.478327in}{0.784924in}}{\pgfqpoint{2.478327in}{0.793160in}}%
\pgfpathcurveto{\pgfqpoint{2.478327in}{0.801396in}}{\pgfqpoint{2.475055in}{0.809296in}}{\pgfqpoint{2.469231in}{0.815120in}}%
\pgfpathcurveto{\pgfqpoint{2.463407in}{0.820944in}}{\pgfqpoint{2.455507in}{0.824217in}}{\pgfqpoint{2.447270in}{0.824217in}}%
\pgfpathcurveto{\pgfqpoint{2.439034in}{0.824217in}}{\pgfqpoint{2.431134in}{0.820944in}}{\pgfqpoint{2.425310in}{0.815120in}}%
\pgfpathcurveto{\pgfqpoint{2.419486in}{0.809296in}}{\pgfqpoint{2.416214in}{0.801396in}}{\pgfqpoint{2.416214in}{0.793160in}}%
\pgfpathcurveto{\pgfqpoint{2.416214in}{0.784924in}}{\pgfqpoint{2.419486in}{0.777024in}}{\pgfqpoint{2.425310in}{0.771200in}}%
\pgfpathcurveto{\pgfqpoint{2.431134in}{0.765376in}}{\pgfqpoint{2.439034in}{0.762104in}}{\pgfqpoint{2.447270in}{0.762104in}}%
\pgfpathclose%
\pgfusepath{stroke,fill}%
\end{pgfscope}%
\begin{pgfscope}%
\pgfpathrectangle{\pgfqpoint{0.100000in}{0.212622in}}{\pgfqpoint{3.696000in}{3.696000in}}%
\pgfusepath{clip}%
\pgfsetbuttcap%
\pgfsetroundjoin%
\definecolor{currentfill}{rgb}{0.121569,0.466667,0.705882}%
\pgfsetfillcolor{currentfill}%
\pgfsetfillopacity{0.997804}%
\pgfsetlinewidth{1.003750pt}%
\definecolor{currentstroke}{rgb}{0.121569,0.466667,0.705882}%
\pgfsetstrokecolor{currentstroke}%
\pgfsetstrokeopacity{0.997804}%
\pgfsetdash{}{0pt}%
\pgfpathmoveto{\pgfqpoint{2.447183in}{0.761688in}}%
\pgfpathcurveto{\pgfqpoint{2.455420in}{0.761688in}}{\pgfqpoint{2.463320in}{0.764961in}}{\pgfqpoint{2.469144in}{0.770785in}}%
\pgfpathcurveto{\pgfqpoint{2.474967in}{0.776608in}}{\pgfqpoint{2.478240in}{0.784509in}}{\pgfqpoint{2.478240in}{0.792745in}}%
\pgfpathcurveto{\pgfqpoint{2.478240in}{0.800981in}}{\pgfqpoint{2.474967in}{0.808881in}}{\pgfqpoint{2.469144in}{0.814705in}}%
\pgfpathcurveto{\pgfqpoint{2.463320in}{0.820529in}}{\pgfqpoint{2.455420in}{0.823801in}}{\pgfqpoint{2.447183in}{0.823801in}}%
\pgfpathcurveto{\pgfqpoint{2.438947in}{0.823801in}}{\pgfqpoint{2.431047in}{0.820529in}}{\pgfqpoint{2.425223in}{0.814705in}}%
\pgfpathcurveto{\pgfqpoint{2.419399in}{0.808881in}}{\pgfqpoint{2.416127in}{0.800981in}}{\pgfqpoint{2.416127in}{0.792745in}}%
\pgfpathcurveto{\pgfqpoint{2.416127in}{0.784509in}}{\pgfqpoint{2.419399in}{0.776608in}}{\pgfqpoint{2.425223in}{0.770785in}}%
\pgfpathcurveto{\pgfqpoint{2.431047in}{0.764961in}}{\pgfqpoint{2.438947in}{0.761688in}}{\pgfqpoint{2.447183in}{0.761688in}}%
\pgfpathclose%
\pgfusepath{stroke,fill}%
\end{pgfscope}%
\begin{pgfscope}%
\pgfpathrectangle{\pgfqpoint{0.100000in}{0.212622in}}{\pgfqpoint{3.696000in}{3.696000in}}%
\pgfusepath{clip}%
\pgfsetbuttcap%
\pgfsetroundjoin%
\definecolor{currentfill}{rgb}{0.121569,0.466667,0.705882}%
\pgfsetfillcolor{currentfill}%
\pgfsetfillopacity{0.997854}%
\pgfsetlinewidth{1.003750pt}%
\definecolor{currentstroke}{rgb}{0.121569,0.466667,0.705882}%
\pgfsetstrokecolor{currentstroke}%
\pgfsetstrokeopacity{0.997854}%
\pgfsetdash{}{0pt}%
\pgfpathmoveto{\pgfqpoint{2.447119in}{0.761464in}}%
\pgfpathcurveto{\pgfqpoint{2.455355in}{0.761464in}}{\pgfqpoint{2.463255in}{0.764737in}}{\pgfqpoint{2.469079in}{0.770561in}}%
\pgfpathcurveto{\pgfqpoint{2.474903in}{0.776384in}}{\pgfqpoint{2.478175in}{0.784285in}}{\pgfqpoint{2.478175in}{0.792521in}}%
\pgfpathcurveto{\pgfqpoint{2.478175in}{0.800757in}}{\pgfqpoint{2.474903in}{0.808657in}}{\pgfqpoint{2.469079in}{0.814481in}}%
\pgfpathcurveto{\pgfqpoint{2.463255in}{0.820305in}}{\pgfqpoint{2.455355in}{0.823577in}}{\pgfqpoint{2.447119in}{0.823577in}}%
\pgfpathcurveto{\pgfqpoint{2.438882in}{0.823577in}}{\pgfqpoint{2.430982in}{0.820305in}}{\pgfqpoint{2.425158in}{0.814481in}}%
\pgfpathcurveto{\pgfqpoint{2.419334in}{0.808657in}}{\pgfqpoint{2.416062in}{0.800757in}}{\pgfqpoint{2.416062in}{0.792521in}}%
\pgfpathcurveto{\pgfqpoint{2.416062in}{0.784285in}}{\pgfqpoint{2.419334in}{0.776384in}}{\pgfqpoint{2.425158in}{0.770561in}}%
\pgfpathcurveto{\pgfqpoint{2.430982in}{0.764737in}}{\pgfqpoint{2.438882in}{0.761464in}}{\pgfqpoint{2.447119in}{0.761464in}}%
\pgfpathclose%
\pgfusepath{stroke,fill}%
\end{pgfscope}%
\begin{pgfscope}%
\pgfpathrectangle{\pgfqpoint{0.100000in}{0.212622in}}{\pgfqpoint{3.696000in}{3.696000in}}%
\pgfusepath{clip}%
\pgfsetbuttcap%
\pgfsetroundjoin%
\definecolor{currentfill}{rgb}{0.121569,0.466667,0.705882}%
\pgfsetfillcolor{currentfill}%
\pgfsetfillopacity{0.997881}%
\pgfsetlinewidth{1.003750pt}%
\definecolor{currentstroke}{rgb}{0.121569,0.466667,0.705882}%
\pgfsetstrokecolor{currentstroke}%
\pgfsetstrokeopacity{0.997881}%
\pgfsetdash{}{0pt}%
\pgfpathmoveto{\pgfqpoint{2.416674in}{0.766660in}}%
\pgfpathcurveto{\pgfqpoint{2.424910in}{0.766660in}}{\pgfqpoint{2.432810in}{0.769932in}}{\pgfqpoint{2.438634in}{0.775756in}}%
\pgfpathcurveto{\pgfqpoint{2.444458in}{0.781580in}}{\pgfqpoint{2.447731in}{0.789480in}}{\pgfqpoint{2.447731in}{0.797717in}}%
\pgfpathcurveto{\pgfqpoint{2.447731in}{0.805953in}}{\pgfqpoint{2.444458in}{0.813853in}}{\pgfqpoint{2.438634in}{0.819677in}}%
\pgfpathcurveto{\pgfqpoint{2.432810in}{0.825501in}}{\pgfqpoint{2.424910in}{0.828773in}}{\pgfqpoint{2.416674in}{0.828773in}}%
\pgfpathcurveto{\pgfqpoint{2.408438in}{0.828773in}}{\pgfqpoint{2.400538in}{0.825501in}}{\pgfqpoint{2.394714in}{0.819677in}}%
\pgfpathcurveto{\pgfqpoint{2.388890in}{0.813853in}}{\pgfqpoint{2.385618in}{0.805953in}}{\pgfqpoint{2.385618in}{0.797717in}}%
\pgfpathcurveto{\pgfqpoint{2.385618in}{0.789480in}}{\pgfqpoint{2.388890in}{0.781580in}}{\pgfqpoint{2.394714in}{0.775756in}}%
\pgfpathcurveto{\pgfqpoint{2.400538in}{0.769932in}}{\pgfqpoint{2.408438in}{0.766660in}}{\pgfqpoint{2.416674in}{0.766660in}}%
\pgfpathclose%
\pgfusepath{stroke,fill}%
\end{pgfscope}%
\begin{pgfscope}%
\pgfpathrectangle{\pgfqpoint{0.100000in}{0.212622in}}{\pgfqpoint{3.696000in}{3.696000in}}%
\pgfusepath{clip}%
\pgfsetbuttcap%
\pgfsetroundjoin%
\definecolor{currentfill}{rgb}{0.121569,0.466667,0.705882}%
\pgfsetfillcolor{currentfill}%
\pgfsetfillopacity{0.998043}%
\pgfsetlinewidth{1.003750pt}%
\definecolor{currentstroke}{rgb}{0.121569,0.466667,0.705882}%
\pgfsetstrokecolor{currentstroke}%
\pgfsetstrokeopacity{0.998043}%
\pgfsetdash{}{0pt}%
\pgfpathmoveto{\pgfqpoint{2.446830in}{0.760679in}}%
\pgfpathcurveto{\pgfqpoint{2.455066in}{0.760679in}}{\pgfqpoint{2.462966in}{0.763952in}}{\pgfqpoint{2.468790in}{0.769776in}}%
\pgfpathcurveto{\pgfqpoint{2.474614in}{0.775600in}}{\pgfqpoint{2.477887in}{0.783500in}}{\pgfqpoint{2.477887in}{0.791736in}}%
\pgfpathcurveto{\pgfqpoint{2.477887in}{0.799972in}}{\pgfqpoint{2.474614in}{0.807872in}}{\pgfqpoint{2.468790in}{0.813696in}}%
\pgfpathcurveto{\pgfqpoint{2.462966in}{0.819520in}}{\pgfqpoint{2.455066in}{0.822792in}}{\pgfqpoint{2.446830in}{0.822792in}}%
\pgfpathcurveto{\pgfqpoint{2.438594in}{0.822792in}}{\pgfqpoint{2.430694in}{0.819520in}}{\pgfqpoint{2.424870in}{0.813696in}}%
\pgfpathcurveto{\pgfqpoint{2.419046in}{0.807872in}}{\pgfqpoint{2.415774in}{0.799972in}}{\pgfqpoint{2.415774in}{0.791736in}}%
\pgfpathcurveto{\pgfqpoint{2.415774in}{0.783500in}}{\pgfqpoint{2.419046in}{0.775600in}}{\pgfqpoint{2.424870in}{0.769776in}}%
\pgfpathcurveto{\pgfqpoint{2.430694in}{0.763952in}}{\pgfqpoint{2.438594in}{0.760679in}}{\pgfqpoint{2.446830in}{0.760679in}}%
\pgfpathclose%
\pgfusepath{stroke,fill}%
\end{pgfscope}%
\begin{pgfscope}%
\pgfpathrectangle{\pgfqpoint{0.100000in}{0.212622in}}{\pgfqpoint{3.696000in}{3.696000in}}%
\pgfusepath{clip}%
\pgfsetbuttcap%
\pgfsetroundjoin%
\definecolor{currentfill}{rgb}{0.121569,0.466667,0.705882}%
\pgfsetfillcolor{currentfill}%
\pgfsetfillopacity{0.998315}%
\pgfsetlinewidth{1.003750pt}%
\definecolor{currentstroke}{rgb}{0.121569,0.466667,0.705882}%
\pgfsetstrokecolor{currentstroke}%
\pgfsetstrokeopacity{0.998315}%
\pgfsetdash{}{0pt}%
\pgfpathmoveto{\pgfqpoint{2.419508in}{0.765293in}}%
\pgfpathcurveto{\pgfqpoint{2.427745in}{0.765293in}}{\pgfqpoint{2.435645in}{0.768565in}}{\pgfqpoint{2.441469in}{0.774389in}}%
\pgfpathcurveto{\pgfqpoint{2.447292in}{0.780213in}}{\pgfqpoint{2.450565in}{0.788113in}}{\pgfqpoint{2.450565in}{0.796350in}}%
\pgfpathcurveto{\pgfqpoint{2.450565in}{0.804586in}}{\pgfqpoint{2.447292in}{0.812486in}}{\pgfqpoint{2.441469in}{0.818310in}}%
\pgfpathcurveto{\pgfqpoint{2.435645in}{0.824134in}}{\pgfqpoint{2.427745in}{0.827406in}}{\pgfqpoint{2.419508in}{0.827406in}}%
\pgfpathcurveto{\pgfqpoint{2.411272in}{0.827406in}}{\pgfqpoint{2.403372in}{0.824134in}}{\pgfqpoint{2.397548in}{0.818310in}}%
\pgfpathcurveto{\pgfqpoint{2.391724in}{0.812486in}}{\pgfqpoint{2.388452in}{0.804586in}}{\pgfqpoint{2.388452in}{0.796350in}}%
\pgfpathcurveto{\pgfqpoint{2.388452in}{0.788113in}}{\pgfqpoint{2.391724in}{0.780213in}}{\pgfqpoint{2.397548in}{0.774389in}}%
\pgfpathcurveto{\pgfqpoint{2.403372in}{0.768565in}}{\pgfqpoint{2.411272in}{0.765293in}}{\pgfqpoint{2.419508in}{0.765293in}}%
\pgfpathclose%
\pgfusepath{stroke,fill}%
\end{pgfscope}%
\begin{pgfscope}%
\pgfpathrectangle{\pgfqpoint{0.100000in}{0.212622in}}{\pgfqpoint{3.696000in}{3.696000in}}%
\pgfusepath{clip}%
\pgfsetbuttcap%
\pgfsetroundjoin%
\definecolor{currentfill}{rgb}{0.121569,0.466667,0.705882}%
\pgfsetfillcolor{currentfill}%
\pgfsetfillopacity{0.998397}%
\pgfsetlinewidth{1.003750pt}%
\definecolor{currentstroke}{rgb}{0.121569,0.466667,0.705882}%
\pgfsetstrokecolor{currentstroke}%
\pgfsetstrokeopacity{0.998397}%
\pgfsetdash{}{0pt}%
\pgfpathmoveto{\pgfqpoint{2.446193in}{0.759361in}}%
\pgfpathcurveto{\pgfqpoint{2.454430in}{0.759361in}}{\pgfqpoint{2.462330in}{0.762633in}}{\pgfqpoint{2.468154in}{0.768457in}}%
\pgfpathcurveto{\pgfqpoint{2.473978in}{0.774281in}}{\pgfqpoint{2.477250in}{0.782181in}}{\pgfqpoint{2.477250in}{0.790417in}}%
\pgfpathcurveto{\pgfqpoint{2.477250in}{0.798654in}}{\pgfqpoint{2.473978in}{0.806554in}}{\pgfqpoint{2.468154in}{0.812378in}}%
\pgfpathcurveto{\pgfqpoint{2.462330in}{0.818202in}}{\pgfqpoint{2.454430in}{0.821474in}}{\pgfqpoint{2.446193in}{0.821474in}}%
\pgfpathcurveto{\pgfqpoint{2.437957in}{0.821474in}}{\pgfqpoint{2.430057in}{0.818202in}}{\pgfqpoint{2.424233in}{0.812378in}}%
\pgfpathcurveto{\pgfqpoint{2.418409in}{0.806554in}}{\pgfqpoint{2.415137in}{0.798654in}}{\pgfqpoint{2.415137in}{0.790417in}}%
\pgfpathcurveto{\pgfqpoint{2.415137in}{0.782181in}}{\pgfqpoint{2.418409in}{0.774281in}}{\pgfqpoint{2.424233in}{0.768457in}}%
\pgfpathcurveto{\pgfqpoint{2.430057in}{0.762633in}}{\pgfqpoint{2.437957in}{0.759361in}}{\pgfqpoint{2.446193in}{0.759361in}}%
\pgfpathclose%
\pgfusepath{stroke,fill}%
\end{pgfscope}%
\begin{pgfscope}%
\pgfpathrectangle{\pgfqpoint{0.100000in}{0.212622in}}{\pgfqpoint{3.696000in}{3.696000in}}%
\pgfusepath{clip}%
\pgfsetbuttcap%
\pgfsetroundjoin%
\definecolor{currentfill}{rgb}{0.121569,0.466667,0.705882}%
\pgfsetfillcolor{currentfill}%
\pgfsetfillopacity{0.998673}%
\pgfsetlinewidth{1.003750pt}%
\definecolor{currentstroke}{rgb}{0.121569,0.466667,0.705882}%
\pgfsetstrokecolor{currentstroke}%
\pgfsetstrokeopacity{0.998673}%
\pgfsetdash{}{0pt}%
\pgfpathmoveto{\pgfqpoint{2.422047in}{0.764098in}}%
\pgfpathcurveto{\pgfqpoint{2.430283in}{0.764098in}}{\pgfqpoint{2.438183in}{0.767370in}}{\pgfqpoint{2.444007in}{0.773194in}}%
\pgfpathcurveto{\pgfqpoint{2.449831in}{0.779018in}}{\pgfqpoint{2.453103in}{0.786918in}}{\pgfqpoint{2.453103in}{0.795154in}}%
\pgfpathcurveto{\pgfqpoint{2.453103in}{0.803390in}}{\pgfqpoint{2.449831in}{0.811291in}}{\pgfqpoint{2.444007in}{0.817114in}}%
\pgfpathcurveto{\pgfqpoint{2.438183in}{0.822938in}}{\pgfqpoint{2.430283in}{0.826211in}}{\pgfqpoint{2.422047in}{0.826211in}}%
\pgfpathcurveto{\pgfqpoint{2.413810in}{0.826211in}}{\pgfqpoint{2.405910in}{0.822938in}}{\pgfqpoint{2.400086in}{0.817114in}}%
\pgfpathcurveto{\pgfqpoint{2.394262in}{0.811291in}}{\pgfqpoint{2.390990in}{0.803390in}}{\pgfqpoint{2.390990in}{0.795154in}}%
\pgfpathcurveto{\pgfqpoint{2.390990in}{0.786918in}}{\pgfqpoint{2.394262in}{0.779018in}}{\pgfqpoint{2.400086in}{0.773194in}}%
\pgfpathcurveto{\pgfqpoint{2.405910in}{0.767370in}}{\pgfqpoint{2.413810in}{0.764098in}}{\pgfqpoint{2.422047in}{0.764098in}}%
\pgfpathclose%
\pgfusepath{stroke,fill}%
\end{pgfscope}%
\begin{pgfscope}%
\pgfpathrectangle{\pgfqpoint{0.100000in}{0.212622in}}{\pgfqpoint{3.696000in}{3.696000in}}%
\pgfusepath{clip}%
\pgfsetbuttcap%
\pgfsetroundjoin%
\definecolor{currentfill}{rgb}{0.121569,0.466667,0.705882}%
\pgfsetfillcolor{currentfill}%
\pgfsetfillopacity{0.998940}%
\pgfsetlinewidth{1.003750pt}%
\definecolor{currentstroke}{rgb}{0.121569,0.466667,0.705882}%
\pgfsetstrokecolor{currentstroke}%
\pgfsetstrokeopacity{0.998940}%
\pgfsetdash{}{0pt}%
\pgfpathmoveto{\pgfqpoint{2.445165in}{0.757736in}}%
\pgfpathcurveto{\pgfqpoint{2.453401in}{0.757736in}}{\pgfqpoint{2.461301in}{0.761008in}}{\pgfqpoint{2.467125in}{0.766832in}}%
\pgfpathcurveto{\pgfqpoint{2.472949in}{0.772656in}}{\pgfqpoint{2.476221in}{0.780556in}}{\pgfqpoint{2.476221in}{0.788792in}}%
\pgfpathcurveto{\pgfqpoint{2.476221in}{0.797029in}}{\pgfqpoint{2.472949in}{0.804929in}}{\pgfqpoint{2.467125in}{0.810753in}}%
\pgfpathcurveto{\pgfqpoint{2.461301in}{0.816577in}}{\pgfqpoint{2.453401in}{0.819849in}}{\pgfqpoint{2.445165in}{0.819849in}}%
\pgfpathcurveto{\pgfqpoint{2.436929in}{0.819849in}}{\pgfqpoint{2.429029in}{0.816577in}}{\pgfqpoint{2.423205in}{0.810753in}}%
\pgfpathcurveto{\pgfqpoint{2.417381in}{0.804929in}}{\pgfqpoint{2.414108in}{0.797029in}}{\pgfqpoint{2.414108in}{0.788792in}}%
\pgfpathcurveto{\pgfqpoint{2.414108in}{0.780556in}}{\pgfqpoint{2.417381in}{0.772656in}}{\pgfqpoint{2.423205in}{0.766832in}}%
\pgfpathcurveto{\pgfqpoint{2.429029in}{0.761008in}}{\pgfqpoint{2.436929in}{0.757736in}}{\pgfqpoint{2.445165in}{0.757736in}}%
\pgfpathclose%
\pgfusepath{stroke,fill}%
\end{pgfscope}%
\begin{pgfscope}%
\pgfpathrectangle{\pgfqpoint{0.100000in}{0.212622in}}{\pgfqpoint{3.696000in}{3.696000in}}%
\pgfusepath{clip}%
\pgfsetbuttcap%
\pgfsetroundjoin%
\definecolor{currentfill}{rgb}{0.121569,0.466667,0.705882}%
\pgfsetfillcolor{currentfill}%
\pgfsetfillopacity{0.998972}%
\pgfsetlinewidth{1.003750pt}%
\definecolor{currentstroke}{rgb}{0.121569,0.466667,0.705882}%
\pgfsetstrokecolor{currentstroke}%
\pgfsetstrokeopacity{0.998972}%
\pgfsetdash{}{0pt}%
\pgfpathmoveto{\pgfqpoint{2.424293in}{0.762984in}}%
\pgfpathcurveto{\pgfqpoint{2.432530in}{0.762984in}}{\pgfqpoint{2.440430in}{0.766256in}}{\pgfqpoint{2.446254in}{0.772080in}}%
\pgfpathcurveto{\pgfqpoint{2.452078in}{0.777904in}}{\pgfqpoint{2.455350in}{0.785804in}}{\pgfqpoint{2.455350in}{0.794040in}}%
\pgfpathcurveto{\pgfqpoint{2.455350in}{0.802277in}}{\pgfqpoint{2.452078in}{0.810177in}}{\pgfqpoint{2.446254in}{0.816001in}}%
\pgfpathcurveto{\pgfqpoint{2.440430in}{0.821824in}}{\pgfqpoint{2.432530in}{0.825097in}}{\pgfqpoint{2.424293in}{0.825097in}}%
\pgfpathcurveto{\pgfqpoint{2.416057in}{0.825097in}}{\pgfqpoint{2.408157in}{0.821824in}}{\pgfqpoint{2.402333in}{0.816001in}}%
\pgfpathcurveto{\pgfqpoint{2.396509in}{0.810177in}}{\pgfqpoint{2.393237in}{0.802277in}}{\pgfqpoint{2.393237in}{0.794040in}}%
\pgfpathcurveto{\pgfqpoint{2.393237in}{0.785804in}}{\pgfqpoint{2.396509in}{0.777904in}}{\pgfqpoint{2.402333in}{0.772080in}}%
\pgfpathcurveto{\pgfqpoint{2.408157in}{0.766256in}}{\pgfqpoint{2.416057in}{0.762984in}}{\pgfqpoint{2.424293in}{0.762984in}}%
\pgfpathclose%
\pgfusepath{stroke,fill}%
\end{pgfscope}%
\begin{pgfscope}%
\pgfpathrectangle{\pgfqpoint{0.100000in}{0.212622in}}{\pgfqpoint{3.696000in}{3.696000in}}%
\pgfusepath{clip}%
\pgfsetbuttcap%
\pgfsetroundjoin%
\definecolor{currentfill}{rgb}{0.121569,0.466667,0.705882}%
\pgfsetfillcolor{currentfill}%
\pgfsetfillopacity{0.999223}%
\pgfsetlinewidth{1.003750pt}%
\definecolor{currentstroke}{rgb}{0.121569,0.466667,0.705882}%
\pgfsetstrokecolor{currentstroke}%
\pgfsetstrokeopacity{0.999223}%
\pgfsetdash{}{0pt}%
\pgfpathmoveto{\pgfqpoint{2.426226in}{0.761988in}}%
\pgfpathcurveto{\pgfqpoint{2.434463in}{0.761988in}}{\pgfqpoint{2.442363in}{0.765261in}}{\pgfqpoint{2.448187in}{0.771085in}}%
\pgfpathcurveto{\pgfqpoint{2.454011in}{0.776909in}}{\pgfqpoint{2.457283in}{0.784809in}}{\pgfqpoint{2.457283in}{0.793045in}}%
\pgfpathcurveto{\pgfqpoint{2.457283in}{0.801281in}}{\pgfqpoint{2.454011in}{0.809181in}}{\pgfqpoint{2.448187in}{0.815005in}}%
\pgfpathcurveto{\pgfqpoint{2.442363in}{0.820829in}}{\pgfqpoint{2.434463in}{0.824101in}}{\pgfqpoint{2.426226in}{0.824101in}}%
\pgfpathcurveto{\pgfqpoint{2.417990in}{0.824101in}}{\pgfqpoint{2.410090in}{0.820829in}}{\pgfqpoint{2.404266in}{0.815005in}}%
\pgfpathcurveto{\pgfqpoint{2.398442in}{0.809181in}}{\pgfqpoint{2.395170in}{0.801281in}}{\pgfqpoint{2.395170in}{0.793045in}}%
\pgfpathcurveto{\pgfqpoint{2.395170in}{0.784809in}}{\pgfqpoint{2.398442in}{0.776909in}}{\pgfqpoint{2.404266in}{0.771085in}}%
\pgfpathcurveto{\pgfqpoint{2.410090in}{0.765261in}}{\pgfqpoint{2.417990in}{0.761988in}}{\pgfqpoint{2.426226in}{0.761988in}}%
\pgfpathclose%
\pgfusepath{stroke,fill}%
\end{pgfscope}%
\begin{pgfscope}%
\pgfpathrectangle{\pgfqpoint{0.100000in}{0.212622in}}{\pgfqpoint{3.696000in}{3.696000in}}%
\pgfusepath{clip}%
\pgfsetbuttcap%
\pgfsetroundjoin%
\definecolor{currentfill}{rgb}{0.121569,0.466667,0.705882}%
\pgfsetfillcolor{currentfill}%
\pgfsetfillopacity{0.999245}%
\pgfsetlinewidth{1.003750pt}%
\definecolor{currentstroke}{rgb}{0.121569,0.466667,0.705882}%
\pgfsetstrokecolor{currentstroke}%
\pgfsetstrokeopacity{0.999245}%
\pgfsetdash{}{0pt}%
\pgfpathmoveto{\pgfqpoint{2.444525in}{0.756972in}}%
\pgfpathcurveto{\pgfqpoint{2.452762in}{0.756972in}}{\pgfqpoint{2.460662in}{0.760244in}}{\pgfqpoint{2.466486in}{0.766068in}}%
\pgfpathcurveto{\pgfqpoint{2.472310in}{0.771892in}}{\pgfqpoint{2.475582in}{0.779792in}}{\pgfqpoint{2.475582in}{0.788028in}}%
\pgfpathcurveto{\pgfqpoint{2.475582in}{0.796264in}}{\pgfqpoint{2.472310in}{0.804165in}}{\pgfqpoint{2.466486in}{0.809988in}}%
\pgfpathcurveto{\pgfqpoint{2.460662in}{0.815812in}}{\pgfqpoint{2.452762in}{0.819085in}}{\pgfqpoint{2.444525in}{0.819085in}}%
\pgfpathcurveto{\pgfqpoint{2.436289in}{0.819085in}}{\pgfqpoint{2.428389in}{0.815812in}}{\pgfqpoint{2.422565in}{0.809988in}}%
\pgfpathcurveto{\pgfqpoint{2.416741in}{0.804165in}}{\pgfqpoint{2.413469in}{0.796264in}}{\pgfqpoint{2.413469in}{0.788028in}}%
\pgfpathcurveto{\pgfqpoint{2.413469in}{0.779792in}}{\pgfqpoint{2.416741in}{0.771892in}}{\pgfqpoint{2.422565in}{0.766068in}}%
\pgfpathcurveto{\pgfqpoint{2.428389in}{0.760244in}}{\pgfqpoint{2.436289in}{0.756972in}}{\pgfqpoint{2.444525in}{0.756972in}}%
\pgfpathclose%
\pgfusepath{stroke,fill}%
\end{pgfscope}%
\begin{pgfscope}%
\pgfpathrectangle{\pgfqpoint{0.100000in}{0.212622in}}{\pgfqpoint{3.696000in}{3.696000in}}%
\pgfusepath{clip}%
\pgfsetbuttcap%
\pgfsetroundjoin%
\definecolor{currentfill}{rgb}{0.121569,0.466667,0.705882}%
\pgfsetfillcolor{currentfill}%
\pgfsetfillopacity{0.999412}%
\pgfsetlinewidth{1.003750pt}%
\definecolor{currentstroke}{rgb}{0.121569,0.466667,0.705882}%
\pgfsetstrokecolor{currentstroke}%
\pgfsetstrokeopacity{0.999412}%
\pgfsetdash{}{0pt}%
\pgfpathmoveto{\pgfqpoint{2.444136in}{0.756615in}}%
\pgfpathcurveto{\pgfqpoint{2.452372in}{0.756615in}}{\pgfqpoint{2.460272in}{0.759888in}}{\pgfqpoint{2.466096in}{0.765712in}}%
\pgfpathcurveto{\pgfqpoint{2.471920in}{0.771536in}}{\pgfqpoint{2.475192in}{0.779436in}}{\pgfqpoint{2.475192in}{0.787672in}}%
\pgfpathcurveto{\pgfqpoint{2.475192in}{0.795908in}}{\pgfqpoint{2.471920in}{0.803808in}}{\pgfqpoint{2.466096in}{0.809632in}}%
\pgfpathcurveto{\pgfqpoint{2.460272in}{0.815456in}}{\pgfqpoint{2.452372in}{0.818728in}}{\pgfqpoint{2.444136in}{0.818728in}}%
\pgfpathcurveto{\pgfqpoint{2.435899in}{0.818728in}}{\pgfqpoint{2.427999in}{0.815456in}}{\pgfqpoint{2.422175in}{0.809632in}}%
\pgfpathcurveto{\pgfqpoint{2.416352in}{0.803808in}}{\pgfqpoint{2.413079in}{0.795908in}}{\pgfqpoint{2.413079in}{0.787672in}}%
\pgfpathcurveto{\pgfqpoint{2.413079in}{0.779436in}}{\pgfqpoint{2.416352in}{0.771536in}}{\pgfqpoint{2.422175in}{0.765712in}}%
\pgfpathcurveto{\pgfqpoint{2.427999in}{0.759888in}}{\pgfqpoint{2.435899in}{0.756615in}}{\pgfqpoint{2.444136in}{0.756615in}}%
\pgfpathclose%
\pgfusepath{stroke,fill}%
\end{pgfscope}%
\begin{pgfscope}%
\pgfpathrectangle{\pgfqpoint{0.100000in}{0.212622in}}{\pgfqpoint{3.696000in}{3.696000in}}%
\pgfusepath{clip}%
\pgfsetbuttcap%
\pgfsetroundjoin%
\definecolor{currentfill}{rgb}{0.121569,0.466667,0.705882}%
\pgfsetfillcolor{currentfill}%
\pgfsetfillopacity{0.999414}%
\pgfsetlinewidth{1.003750pt}%
\definecolor{currentstroke}{rgb}{0.121569,0.466667,0.705882}%
\pgfsetstrokecolor{currentstroke}%
\pgfsetstrokeopacity{0.999414}%
\pgfsetdash{}{0pt}%
\pgfpathmoveto{\pgfqpoint{2.427888in}{0.761112in}}%
\pgfpathcurveto{\pgfqpoint{2.436125in}{0.761112in}}{\pgfqpoint{2.444025in}{0.764385in}}{\pgfqpoint{2.449849in}{0.770209in}}%
\pgfpathcurveto{\pgfqpoint{2.455673in}{0.776033in}}{\pgfqpoint{2.458945in}{0.783933in}}{\pgfqpoint{2.458945in}{0.792169in}}%
\pgfpathcurveto{\pgfqpoint{2.458945in}{0.800405in}}{\pgfqpoint{2.455673in}{0.808305in}}{\pgfqpoint{2.449849in}{0.814129in}}%
\pgfpathcurveto{\pgfqpoint{2.444025in}{0.819953in}}{\pgfqpoint{2.436125in}{0.823225in}}{\pgfqpoint{2.427888in}{0.823225in}}%
\pgfpathcurveto{\pgfqpoint{2.419652in}{0.823225in}}{\pgfqpoint{2.411752in}{0.819953in}}{\pgfqpoint{2.405928in}{0.814129in}}%
\pgfpathcurveto{\pgfqpoint{2.400104in}{0.808305in}}{\pgfqpoint{2.396832in}{0.800405in}}{\pgfqpoint{2.396832in}{0.792169in}}%
\pgfpathcurveto{\pgfqpoint{2.396832in}{0.783933in}}{\pgfqpoint{2.400104in}{0.776033in}}{\pgfqpoint{2.405928in}{0.770209in}}%
\pgfpathcurveto{\pgfqpoint{2.411752in}{0.764385in}}{\pgfqpoint{2.419652in}{0.761112in}}{\pgfqpoint{2.427888in}{0.761112in}}%
\pgfpathclose%
\pgfusepath{stroke,fill}%
\end{pgfscope}%
\begin{pgfscope}%
\pgfpathrectangle{\pgfqpoint{0.100000in}{0.212622in}}{\pgfqpoint{3.696000in}{3.696000in}}%
\pgfusepath{clip}%
\pgfsetbuttcap%
\pgfsetroundjoin%
\definecolor{currentfill}{rgb}{0.121569,0.466667,0.705882}%
\pgfsetfillcolor{currentfill}%
\pgfsetfillopacity{0.999498}%
\pgfsetlinewidth{1.003750pt}%
\definecolor{currentstroke}{rgb}{0.121569,0.466667,0.705882}%
\pgfsetstrokecolor{currentstroke}%
\pgfsetstrokeopacity{0.999498}%
\pgfsetdash{}{0pt}%
\pgfpathmoveto{\pgfqpoint{2.443904in}{0.756446in}}%
\pgfpathcurveto{\pgfqpoint{2.452140in}{0.756446in}}{\pgfqpoint{2.460040in}{0.759719in}}{\pgfqpoint{2.465864in}{0.765543in}}%
\pgfpathcurveto{\pgfqpoint{2.471688in}{0.771367in}}{\pgfqpoint{2.474960in}{0.779267in}}{\pgfqpoint{2.474960in}{0.787503in}}%
\pgfpathcurveto{\pgfqpoint{2.474960in}{0.795739in}}{\pgfqpoint{2.471688in}{0.803639in}}{\pgfqpoint{2.465864in}{0.809463in}}%
\pgfpathcurveto{\pgfqpoint{2.460040in}{0.815287in}}{\pgfqpoint{2.452140in}{0.818559in}}{\pgfqpoint{2.443904in}{0.818559in}}%
\pgfpathcurveto{\pgfqpoint{2.435667in}{0.818559in}}{\pgfqpoint{2.427767in}{0.815287in}}{\pgfqpoint{2.421943in}{0.809463in}}%
\pgfpathcurveto{\pgfqpoint{2.416119in}{0.803639in}}{\pgfqpoint{2.412847in}{0.795739in}}{\pgfqpoint{2.412847in}{0.787503in}}%
\pgfpathcurveto{\pgfqpoint{2.412847in}{0.779267in}}{\pgfqpoint{2.416119in}{0.771367in}}{\pgfqpoint{2.421943in}{0.765543in}}%
\pgfpathcurveto{\pgfqpoint{2.427767in}{0.759719in}}{\pgfqpoint{2.435667in}{0.756446in}}{\pgfqpoint{2.443904in}{0.756446in}}%
\pgfpathclose%
\pgfusepath{stroke,fill}%
\end{pgfscope}%
\begin{pgfscope}%
\pgfpathrectangle{\pgfqpoint{0.100000in}{0.212622in}}{\pgfqpoint{3.696000in}{3.696000in}}%
\pgfusepath{clip}%
\pgfsetbuttcap%
\pgfsetroundjoin%
\definecolor{currentfill}{rgb}{0.121569,0.466667,0.705882}%
\pgfsetfillcolor{currentfill}%
\pgfsetfillopacity{0.999544}%
\pgfsetlinewidth{1.003750pt}%
\definecolor{currentstroke}{rgb}{0.121569,0.466667,0.705882}%
\pgfsetstrokecolor{currentstroke}%
\pgfsetstrokeopacity{0.999544}%
\pgfsetdash{}{0pt}%
\pgfpathmoveto{\pgfqpoint{2.443768in}{0.756369in}}%
\pgfpathcurveto{\pgfqpoint{2.452005in}{0.756369in}}{\pgfqpoint{2.459905in}{0.759641in}}{\pgfqpoint{2.465729in}{0.765465in}}%
\pgfpathcurveto{\pgfqpoint{2.471553in}{0.771289in}}{\pgfqpoint{2.474825in}{0.779189in}}{\pgfqpoint{2.474825in}{0.787426in}}%
\pgfpathcurveto{\pgfqpoint{2.474825in}{0.795662in}}{\pgfqpoint{2.471553in}{0.803562in}}{\pgfqpoint{2.465729in}{0.809386in}}%
\pgfpathcurveto{\pgfqpoint{2.459905in}{0.815210in}}{\pgfqpoint{2.452005in}{0.818482in}}{\pgfqpoint{2.443768in}{0.818482in}}%
\pgfpathcurveto{\pgfqpoint{2.435532in}{0.818482in}}{\pgfqpoint{2.427632in}{0.815210in}}{\pgfqpoint{2.421808in}{0.809386in}}%
\pgfpathcurveto{\pgfqpoint{2.415984in}{0.803562in}}{\pgfqpoint{2.412712in}{0.795662in}}{\pgfqpoint{2.412712in}{0.787426in}}%
\pgfpathcurveto{\pgfqpoint{2.412712in}{0.779189in}}{\pgfqpoint{2.415984in}{0.771289in}}{\pgfqpoint{2.421808in}{0.765465in}}%
\pgfpathcurveto{\pgfqpoint{2.427632in}{0.759641in}}{\pgfqpoint{2.435532in}{0.756369in}}{\pgfqpoint{2.443768in}{0.756369in}}%
\pgfpathclose%
\pgfusepath{stroke,fill}%
\end{pgfscope}%
\begin{pgfscope}%
\pgfpathrectangle{\pgfqpoint{0.100000in}{0.212622in}}{\pgfqpoint{3.696000in}{3.696000in}}%
\pgfusepath{clip}%
\pgfsetbuttcap%
\pgfsetroundjoin%
\definecolor{currentfill}{rgb}{0.121569,0.466667,0.705882}%
\pgfsetfillcolor{currentfill}%
\pgfsetfillopacity{0.999562}%
\pgfsetlinewidth{1.003750pt}%
\definecolor{currentstroke}{rgb}{0.121569,0.466667,0.705882}%
\pgfsetstrokecolor{currentstroke}%
\pgfsetstrokeopacity{0.999562}%
\pgfsetdash{}{0pt}%
\pgfpathmoveto{\pgfqpoint{2.429266in}{0.760417in}}%
\pgfpathcurveto{\pgfqpoint{2.437502in}{0.760417in}}{\pgfqpoint{2.445402in}{0.763690in}}{\pgfqpoint{2.451226in}{0.769514in}}%
\pgfpathcurveto{\pgfqpoint{2.457050in}{0.775338in}}{\pgfqpoint{2.460322in}{0.783238in}}{\pgfqpoint{2.460322in}{0.791474in}}%
\pgfpathcurveto{\pgfqpoint{2.460322in}{0.799710in}}{\pgfqpoint{2.457050in}{0.807610in}}{\pgfqpoint{2.451226in}{0.813434in}}%
\pgfpathcurveto{\pgfqpoint{2.445402in}{0.819258in}}{\pgfqpoint{2.437502in}{0.822530in}}{\pgfqpoint{2.429266in}{0.822530in}}%
\pgfpathcurveto{\pgfqpoint{2.421030in}{0.822530in}}{\pgfqpoint{2.413130in}{0.819258in}}{\pgfqpoint{2.407306in}{0.813434in}}%
\pgfpathcurveto{\pgfqpoint{2.401482in}{0.807610in}}{\pgfqpoint{2.398209in}{0.799710in}}{\pgfqpoint{2.398209in}{0.791474in}}%
\pgfpathcurveto{\pgfqpoint{2.398209in}{0.783238in}}{\pgfqpoint{2.401482in}{0.775338in}}{\pgfqpoint{2.407306in}{0.769514in}}%
\pgfpathcurveto{\pgfqpoint{2.413130in}{0.763690in}}{\pgfqpoint{2.421030in}{0.760417in}}{\pgfqpoint{2.429266in}{0.760417in}}%
\pgfpathclose%
\pgfusepath{stroke,fill}%
\end{pgfscope}%
\begin{pgfscope}%
\pgfpathrectangle{\pgfqpoint{0.100000in}{0.212622in}}{\pgfqpoint{3.696000in}{3.696000in}}%
\pgfusepath{clip}%
\pgfsetbuttcap%
\pgfsetroundjoin%
\definecolor{currentfill}{rgb}{0.121569,0.466667,0.705882}%
\pgfsetfillcolor{currentfill}%
\pgfsetfillopacity{0.999567}%
\pgfsetlinewidth{1.003750pt}%
\definecolor{currentstroke}{rgb}{0.121569,0.466667,0.705882}%
\pgfsetstrokecolor{currentstroke}%
\pgfsetstrokeopacity{0.999567}%
\pgfsetdash{}{0pt}%
\pgfpathmoveto{\pgfqpoint{2.443691in}{0.756333in}}%
\pgfpathcurveto{\pgfqpoint{2.451927in}{0.756333in}}{\pgfqpoint{2.459827in}{0.759605in}}{\pgfqpoint{2.465651in}{0.765429in}}%
\pgfpathcurveto{\pgfqpoint{2.471475in}{0.771253in}}{\pgfqpoint{2.474747in}{0.779153in}}{\pgfqpoint{2.474747in}{0.787390in}}%
\pgfpathcurveto{\pgfqpoint{2.474747in}{0.795626in}}{\pgfqpoint{2.471475in}{0.803526in}}{\pgfqpoint{2.465651in}{0.809350in}}%
\pgfpathcurveto{\pgfqpoint{2.459827in}{0.815174in}}{\pgfqpoint{2.451927in}{0.818446in}}{\pgfqpoint{2.443691in}{0.818446in}}%
\pgfpathcurveto{\pgfqpoint{2.435454in}{0.818446in}}{\pgfqpoint{2.427554in}{0.815174in}}{\pgfqpoint{2.421730in}{0.809350in}}%
\pgfpathcurveto{\pgfqpoint{2.415906in}{0.803526in}}{\pgfqpoint{2.412634in}{0.795626in}}{\pgfqpoint{2.412634in}{0.787390in}}%
\pgfpathcurveto{\pgfqpoint{2.412634in}{0.779153in}}{\pgfqpoint{2.415906in}{0.771253in}}{\pgfqpoint{2.421730in}{0.765429in}}%
\pgfpathcurveto{\pgfqpoint{2.427554in}{0.759605in}}{\pgfqpoint{2.435454in}{0.756333in}}{\pgfqpoint{2.443691in}{0.756333in}}%
\pgfpathclose%
\pgfusepath{stroke,fill}%
\end{pgfscope}%
\begin{pgfscope}%
\pgfpathrectangle{\pgfqpoint{0.100000in}{0.212622in}}{\pgfqpoint{3.696000in}{3.696000in}}%
\pgfusepath{clip}%
\pgfsetbuttcap%
\pgfsetroundjoin%
\definecolor{currentfill}{rgb}{0.121569,0.466667,0.705882}%
\pgfsetfillcolor{currentfill}%
\pgfsetfillopacity{0.999579}%
\pgfsetlinewidth{1.003750pt}%
\definecolor{currentstroke}{rgb}{0.121569,0.466667,0.705882}%
\pgfsetstrokecolor{currentstroke}%
\pgfsetstrokeopacity{0.999579}%
\pgfsetdash{}{0pt}%
\pgfpathmoveto{\pgfqpoint{2.443646in}{0.756318in}}%
\pgfpathcurveto{\pgfqpoint{2.451882in}{0.756318in}}{\pgfqpoint{2.459782in}{0.759590in}}{\pgfqpoint{2.465606in}{0.765414in}}%
\pgfpathcurveto{\pgfqpoint{2.471430in}{0.771238in}}{\pgfqpoint{2.474703in}{0.779138in}}{\pgfqpoint{2.474703in}{0.787375in}}%
\pgfpathcurveto{\pgfqpoint{2.474703in}{0.795611in}}{\pgfqpoint{2.471430in}{0.803511in}}{\pgfqpoint{2.465606in}{0.809335in}}%
\pgfpathcurveto{\pgfqpoint{2.459782in}{0.815159in}}{\pgfqpoint{2.451882in}{0.818431in}}{\pgfqpoint{2.443646in}{0.818431in}}%
\pgfpathcurveto{\pgfqpoint{2.435410in}{0.818431in}}{\pgfqpoint{2.427510in}{0.815159in}}{\pgfqpoint{2.421686in}{0.809335in}}%
\pgfpathcurveto{\pgfqpoint{2.415862in}{0.803511in}}{\pgfqpoint{2.412590in}{0.795611in}}{\pgfqpoint{2.412590in}{0.787375in}}%
\pgfpathcurveto{\pgfqpoint{2.412590in}{0.779138in}}{\pgfqpoint{2.415862in}{0.771238in}}{\pgfqpoint{2.421686in}{0.765414in}}%
\pgfpathcurveto{\pgfqpoint{2.427510in}{0.759590in}}{\pgfqpoint{2.435410in}{0.756318in}}{\pgfqpoint{2.443646in}{0.756318in}}%
\pgfpathclose%
\pgfusepath{stroke,fill}%
\end{pgfscope}%
\begin{pgfscope}%
\pgfpathrectangle{\pgfqpoint{0.100000in}{0.212622in}}{\pgfqpoint{3.696000in}{3.696000in}}%
\pgfusepath{clip}%
\pgfsetbuttcap%
\pgfsetroundjoin%
\definecolor{currentfill}{rgb}{0.121569,0.466667,0.705882}%
\pgfsetfillcolor{currentfill}%
\pgfsetfillopacity{0.999584}%
\pgfsetlinewidth{1.003750pt}%
\definecolor{currentstroke}{rgb}{0.121569,0.466667,0.705882}%
\pgfsetstrokecolor{currentstroke}%
\pgfsetstrokeopacity{0.999584}%
\pgfsetdash{}{0pt}%
\pgfpathmoveto{\pgfqpoint{2.443621in}{0.756312in}}%
\pgfpathcurveto{\pgfqpoint{2.451857in}{0.756312in}}{\pgfqpoint{2.459757in}{0.759584in}}{\pgfqpoint{2.465581in}{0.765408in}}%
\pgfpathcurveto{\pgfqpoint{2.471405in}{0.771232in}}{\pgfqpoint{2.474678in}{0.779132in}}{\pgfqpoint{2.474678in}{0.787368in}}%
\pgfpathcurveto{\pgfqpoint{2.474678in}{0.795604in}}{\pgfqpoint{2.471405in}{0.803505in}}{\pgfqpoint{2.465581in}{0.809328in}}%
\pgfpathcurveto{\pgfqpoint{2.459757in}{0.815152in}}{\pgfqpoint{2.451857in}{0.818425in}}{\pgfqpoint{2.443621in}{0.818425in}}%
\pgfpathcurveto{\pgfqpoint{2.435385in}{0.818425in}}{\pgfqpoint{2.427485in}{0.815152in}}{\pgfqpoint{2.421661in}{0.809328in}}%
\pgfpathcurveto{\pgfqpoint{2.415837in}{0.803505in}}{\pgfqpoint{2.412565in}{0.795604in}}{\pgfqpoint{2.412565in}{0.787368in}}%
\pgfpathcurveto{\pgfqpoint{2.412565in}{0.779132in}}{\pgfqpoint{2.415837in}{0.771232in}}{\pgfqpoint{2.421661in}{0.765408in}}%
\pgfpathcurveto{\pgfqpoint{2.427485in}{0.759584in}}{\pgfqpoint{2.435385in}{0.756312in}}{\pgfqpoint{2.443621in}{0.756312in}}%
\pgfpathclose%
\pgfusepath{stroke,fill}%
\end{pgfscope}%
\begin{pgfscope}%
\pgfpathrectangle{\pgfqpoint{0.100000in}{0.212622in}}{\pgfqpoint{3.696000in}{3.696000in}}%
\pgfusepath{clip}%
\pgfsetbuttcap%
\pgfsetroundjoin%
\definecolor{currentfill}{rgb}{0.121569,0.466667,0.705882}%
\pgfsetfillcolor{currentfill}%
\pgfsetfillopacity{0.999587}%
\pgfsetlinewidth{1.003750pt}%
\definecolor{currentstroke}{rgb}{0.121569,0.466667,0.705882}%
\pgfsetstrokecolor{currentstroke}%
\pgfsetstrokeopacity{0.999587}%
\pgfsetdash{}{0pt}%
\pgfpathmoveto{\pgfqpoint{2.443607in}{0.756309in}}%
\pgfpathcurveto{\pgfqpoint{2.451843in}{0.756309in}}{\pgfqpoint{2.459743in}{0.759582in}}{\pgfqpoint{2.465567in}{0.765406in}}%
\pgfpathcurveto{\pgfqpoint{2.471391in}{0.771230in}}{\pgfqpoint{2.474663in}{0.779130in}}{\pgfqpoint{2.474663in}{0.787366in}}%
\pgfpathcurveto{\pgfqpoint{2.474663in}{0.795602in}}{\pgfqpoint{2.471391in}{0.803502in}}{\pgfqpoint{2.465567in}{0.809326in}}%
\pgfpathcurveto{\pgfqpoint{2.459743in}{0.815150in}}{\pgfqpoint{2.451843in}{0.818422in}}{\pgfqpoint{2.443607in}{0.818422in}}%
\pgfpathcurveto{\pgfqpoint{2.435371in}{0.818422in}}{\pgfqpoint{2.427471in}{0.815150in}}{\pgfqpoint{2.421647in}{0.809326in}}%
\pgfpathcurveto{\pgfqpoint{2.415823in}{0.803502in}}{\pgfqpoint{2.412550in}{0.795602in}}{\pgfqpoint{2.412550in}{0.787366in}}%
\pgfpathcurveto{\pgfqpoint{2.412550in}{0.779130in}}{\pgfqpoint{2.415823in}{0.771230in}}{\pgfqpoint{2.421647in}{0.765406in}}%
\pgfpathcurveto{\pgfqpoint{2.427471in}{0.759582in}}{\pgfqpoint{2.435371in}{0.756309in}}{\pgfqpoint{2.443607in}{0.756309in}}%
\pgfpathclose%
\pgfusepath{stroke,fill}%
\end{pgfscope}%
\begin{pgfscope}%
\pgfpathrectangle{\pgfqpoint{0.100000in}{0.212622in}}{\pgfqpoint{3.696000in}{3.696000in}}%
\pgfusepath{clip}%
\pgfsetbuttcap%
\pgfsetroundjoin%
\definecolor{currentfill}{rgb}{0.121569,0.466667,0.705882}%
\pgfsetfillcolor{currentfill}%
\pgfsetfillopacity{0.999588}%
\pgfsetlinewidth{1.003750pt}%
\definecolor{currentstroke}{rgb}{0.121569,0.466667,0.705882}%
\pgfsetstrokecolor{currentstroke}%
\pgfsetstrokeopacity{0.999588}%
\pgfsetdash{}{0pt}%
\pgfpathmoveto{\pgfqpoint{2.443599in}{0.756309in}}%
\pgfpathcurveto{\pgfqpoint{2.451835in}{0.756309in}}{\pgfqpoint{2.459735in}{0.759581in}}{\pgfqpoint{2.465559in}{0.765405in}}%
\pgfpathcurveto{\pgfqpoint{2.471383in}{0.771229in}}{\pgfqpoint{2.474656in}{0.779129in}}{\pgfqpoint{2.474656in}{0.787365in}}%
\pgfpathcurveto{\pgfqpoint{2.474656in}{0.795601in}}{\pgfqpoint{2.471383in}{0.803501in}}{\pgfqpoint{2.465559in}{0.809325in}}%
\pgfpathcurveto{\pgfqpoint{2.459735in}{0.815149in}}{\pgfqpoint{2.451835in}{0.818422in}}{\pgfqpoint{2.443599in}{0.818422in}}%
\pgfpathcurveto{\pgfqpoint{2.435363in}{0.818422in}}{\pgfqpoint{2.427463in}{0.815149in}}{\pgfqpoint{2.421639in}{0.809325in}}%
\pgfpathcurveto{\pgfqpoint{2.415815in}{0.803501in}}{\pgfqpoint{2.412543in}{0.795601in}}{\pgfqpoint{2.412543in}{0.787365in}}%
\pgfpathcurveto{\pgfqpoint{2.412543in}{0.779129in}}{\pgfqpoint{2.415815in}{0.771229in}}{\pgfqpoint{2.421639in}{0.765405in}}%
\pgfpathcurveto{\pgfqpoint{2.427463in}{0.759581in}}{\pgfqpoint{2.435363in}{0.756309in}}{\pgfqpoint{2.443599in}{0.756309in}}%
\pgfpathclose%
\pgfusepath{stroke,fill}%
\end{pgfscope}%
\begin{pgfscope}%
\pgfpathrectangle{\pgfqpoint{0.100000in}{0.212622in}}{\pgfqpoint{3.696000in}{3.696000in}}%
\pgfusepath{clip}%
\pgfsetbuttcap%
\pgfsetroundjoin%
\definecolor{currentfill}{rgb}{0.121569,0.466667,0.705882}%
\pgfsetfillcolor{currentfill}%
\pgfsetfillopacity{0.999633}%
\pgfsetlinewidth{1.003750pt}%
\definecolor{currentstroke}{rgb}{0.121569,0.466667,0.705882}%
\pgfsetstrokecolor{currentstroke}%
\pgfsetstrokeopacity{0.999633}%
\pgfsetdash{}{0pt}%
\pgfpathmoveto{\pgfqpoint{2.443304in}{0.756301in}}%
\pgfpathcurveto{\pgfqpoint{2.451541in}{0.756301in}}{\pgfqpoint{2.459441in}{0.759574in}}{\pgfqpoint{2.465265in}{0.765397in}}%
\pgfpathcurveto{\pgfqpoint{2.471089in}{0.771221in}}{\pgfqpoint{2.474361in}{0.779121in}}{\pgfqpoint{2.474361in}{0.787358in}}%
\pgfpathcurveto{\pgfqpoint{2.474361in}{0.795594in}}{\pgfqpoint{2.471089in}{0.803494in}}{\pgfqpoint{2.465265in}{0.809318in}}%
\pgfpathcurveto{\pgfqpoint{2.459441in}{0.815142in}}{\pgfqpoint{2.451541in}{0.818414in}}{\pgfqpoint{2.443304in}{0.818414in}}%
\pgfpathcurveto{\pgfqpoint{2.435068in}{0.818414in}}{\pgfqpoint{2.427168in}{0.815142in}}{\pgfqpoint{2.421344in}{0.809318in}}%
\pgfpathcurveto{\pgfqpoint{2.415520in}{0.803494in}}{\pgfqpoint{2.412248in}{0.795594in}}{\pgfqpoint{2.412248in}{0.787358in}}%
\pgfpathcurveto{\pgfqpoint{2.412248in}{0.779121in}}{\pgfqpoint{2.415520in}{0.771221in}}{\pgfqpoint{2.421344in}{0.765397in}}%
\pgfpathcurveto{\pgfqpoint{2.427168in}{0.759574in}}{\pgfqpoint{2.435068in}{0.756301in}}{\pgfqpoint{2.443304in}{0.756301in}}%
\pgfpathclose%
\pgfusepath{stroke,fill}%
\end{pgfscope}%
\begin{pgfscope}%
\pgfpathrectangle{\pgfqpoint{0.100000in}{0.212622in}}{\pgfqpoint{3.696000in}{3.696000in}}%
\pgfusepath{clip}%
\pgfsetbuttcap%
\pgfsetroundjoin%
\definecolor{currentfill}{rgb}{0.121569,0.466667,0.705882}%
\pgfsetfillcolor{currentfill}%
\pgfsetfillopacity{0.999678}%
\pgfsetlinewidth{1.003750pt}%
\definecolor{currentstroke}{rgb}{0.121569,0.466667,0.705882}%
\pgfsetstrokecolor{currentstroke}%
\pgfsetstrokeopacity{0.999678}%
\pgfsetdash{}{0pt}%
\pgfpathmoveto{\pgfqpoint{2.430405in}{0.759866in}}%
\pgfpathcurveto{\pgfqpoint{2.438642in}{0.759866in}}{\pgfqpoint{2.446542in}{0.763138in}}{\pgfqpoint{2.452366in}{0.768962in}}%
\pgfpathcurveto{\pgfqpoint{2.458190in}{0.774786in}}{\pgfqpoint{2.461462in}{0.782686in}}{\pgfqpoint{2.461462in}{0.790922in}}%
\pgfpathcurveto{\pgfqpoint{2.461462in}{0.799159in}}{\pgfqpoint{2.458190in}{0.807059in}}{\pgfqpoint{2.452366in}{0.812882in}}%
\pgfpathcurveto{\pgfqpoint{2.446542in}{0.818706in}}{\pgfqpoint{2.438642in}{0.821979in}}{\pgfqpoint{2.430405in}{0.821979in}}%
\pgfpathcurveto{\pgfqpoint{2.422169in}{0.821979in}}{\pgfqpoint{2.414269in}{0.818706in}}{\pgfqpoint{2.408445in}{0.812882in}}%
\pgfpathcurveto{\pgfqpoint{2.402621in}{0.807059in}}{\pgfqpoint{2.399349in}{0.799159in}}{\pgfqpoint{2.399349in}{0.790922in}}%
\pgfpathcurveto{\pgfqpoint{2.399349in}{0.782686in}}{\pgfqpoint{2.402621in}{0.774786in}}{\pgfqpoint{2.408445in}{0.768962in}}%
\pgfpathcurveto{\pgfqpoint{2.414269in}{0.763138in}}{\pgfqpoint{2.422169in}{0.759866in}}{\pgfqpoint{2.430405in}{0.759866in}}%
\pgfpathclose%
\pgfusepath{stroke,fill}%
\end{pgfscope}%
\begin{pgfscope}%
\pgfpathrectangle{\pgfqpoint{0.100000in}{0.212622in}}{\pgfqpoint{3.696000in}{3.696000in}}%
\pgfusepath{clip}%
\pgfsetbuttcap%
\pgfsetroundjoin%
\definecolor{currentfill}{rgb}{0.121569,0.466667,0.705882}%
\pgfsetfillcolor{currentfill}%
\pgfsetfillopacity{0.999713}%
\pgfsetlinewidth{1.003750pt}%
\definecolor{currentstroke}{rgb}{0.121569,0.466667,0.705882}%
\pgfsetstrokecolor{currentstroke}%
\pgfsetstrokeopacity{0.999713}%
\pgfsetdash{}{0pt}%
\pgfpathmoveto{\pgfqpoint{2.442713in}{0.756336in}}%
\pgfpathcurveto{\pgfqpoint{2.450949in}{0.756336in}}{\pgfqpoint{2.458849in}{0.759609in}}{\pgfqpoint{2.464673in}{0.765433in}}%
\pgfpathcurveto{\pgfqpoint{2.470497in}{0.771256in}}{\pgfqpoint{2.473769in}{0.779157in}}{\pgfqpoint{2.473769in}{0.787393in}}%
\pgfpathcurveto{\pgfqpoint{2.473769in}{0.795629in}}{\pgfqpoint{2.470497in}{0.803529in}}{\pgfqpoint{2.464673in}{0.809353in}}%
\pgfpathcurveto{\pgfqpoint{2.458849in}{0.815177in}}{\pgfqpoint{2.450949in}{0.818449in}}{\pgfqpoint{2.442713in}{0.818449in}}%
\pgfpathcurveto{\pgfqpoint{2.434477in}{0.818449in}}{\pgfqpoint{2.426577in}{0.815177in}}{\pgfqpoint{2.420753in}{0.809353in}}%
\pgfpathcurveto{\pgfqpoint{2.414929in}{0.803529in}}{\pgfqpoint{2.411656in}{0.795629in}}{\pgfqpoint{2.411656in}{0.787393in}}%
\pgfpathcurveto{\pgfqpoint{2.411656in}{0.779157in}}{\pgfqpoint{2.414929in}{0.771256in}}{\pgfqpoint{2.420753in}{0.765433in}}%
\pgfpathcurveto{\pgfqpoint{2.426577in}{0.759609in}}{\pgfqpoint{2.434477in}{0.756336in}}{\pgfqpoint{2.442713in}{0.756336in}}%
\pgfpathclose%
\pgfusepath{stroke,fill}%
\end{pgfscope}%
\begin{pgfscope}%
\pgfpathrectangle{\pgfqpoint{0.100000in}{0.212622in}}{\pgfqpoint{3.696000in}{3.696000in}}%
\pgfusepath{clip}%
\pgfsetbuttcap%
\pgfsetroundjoin%
\definecolor{currentfill}{rgb}{0.121569,0.466667,0.705882}%
\pgfsetfillcolor{currentfill}%
\pgfsetfillopacity{0.999758}%
\pgfsetlinewidth{1.003750pt}%
\definecolor{currentstroke}{rgb}{0.121569,0.466667,0.705882}%
\pgfsetstrokecolor{currentstroke}%
\pgfsetstrokeopacity{0.999758}%
\pgfsetdash{}{0pt}%
\pgfpathmoveto{\pgfqpoint{2.431291in}{0.759460in}}%
\pgfpathcurveto{\pgfqpoint{2.439527in}{0.759460in}}{\pgfqpoint{2.447427in}{0.762733in}}{\pgfqpoint{2.453251in}{0.768556in}}%
\pgfpathcurveto{\pgfqpoint{2.459075in}{0.774380in}}{\pgfqpoint{2.462347in}{0.782280in}}{\pgfqpoint{2.462347in}{0.790517in}}%
\pgfpathcurveto{\pgfqpoint{2.462347in}{0.798753in}}{\pgfqpoint{2.459075in}{0.806653in}}{\pgfqpoint{2.453251in}{0.812477in}}%
\pgfpathcurveto{\pgfqpoint{2.447427in}{0.818301in}}{\pgfqpoint{2.439527in}{0.821573in}}{\pgfqpoint{2.431291in}{0.821573in}}%
\pgfpathcurveto{\pgfqpoint{2.423054in}{0.821573in}}{\pgfqpoint{2.415154in}{0.818301in}}{\pgfqpoint{2.409330in}{0.812477in}}%
\pgfpathcurveto{\pgfqpoint{2.403506in}{0.806653in}}{\pgfqpoint{2.400234in}{0.798753in}}{\pgfqpoint{2.400234in}{0.790517in}}%
\pgfpathcurveto{\pgfqpoint{2.400234in}{0.782280in}}{\pgfqpoint{2.403506in}{0.774380in}}{\pgfqpoint{2.409330in}{0.768556in}}%
\pgfpathcurveto{\pgfqpoint{2.415154in}{0.762733in}}{\pgfqpoint{2.423054in}{0.759460in}}{\pgfqpoint{2.431291in}{0.759460in}}%
\pgfpathclose%
\pgfusepath{stroke,fill}%
\end{pgfscope}%
\begin{pgfscope}%
\pgfpathrectangle{\pgfqpoint{0.100000in}{0.212622in}}{\pgfqpoint{3.696000in}{3.696000in}}%
\pgfusepath{clip}%
\pgfsetbuttcap%
\pgfsetroundjoin%
\definecolor{currentfill}{rgb}{0.121569,0.466667,0.705882}%
\pgfsetfillcolor{currentfill}%
\pgfsetfillopacity{0.999809}%
\pgfsetlinewidth{1.003750pt}%
\definecolor{currentstroke}{rgb}{0.121569,0.466667,0.705882}%
\pgfsetstrokecolor{currentstroke}%
\pgfsetstrokeopacity{0.999809}%
\pgfsetdash{}{0pt}%
\pgfpathmoveto{\pgfqpoint{2.431943in}{0.759195in}}%
\pgfpathcurveto{\pgfqpoint{2.440179in}{0.759195in}}{\pgfqpoint{2.448079in}{0.762467in}}{\pgfqpoint{2.453903in}{0.768291in}}%
\pgfpathcurveto{\pgfqpoint{2.459727in}{0.774115in}}{\pgfqpoint{2.462999in}{0.782015in}}{\pgfqpoint{2.462999in}{0.790251in}}%
\pgfpathcurveto{\pgfqpoint{2.462999in}{0.798487in}}{\pgfqpoint{2.459727in}{0.806387in}}{\pgfqpoint{2.453903in}{0.812211in}}%
\pgfpathcurveto{\pgfqpoint{2.448079in}{0.818035in}}{\pgfqpoint{2.440179in}{0.821308in}}{\pgfqpoint{2.431943in}{0.821308in}}%
\pgfpathcurveto{\pgfqpoint{2.423706in}{0.821308in}}{\pgfqpoint{2.415806in}{0.818035in}}{\pgfqpoint{2.409982in}{0.812211in}}%
\pgfpathcurveto{\pgfqpoint{2.404158in}{0.806387in}}{\pgfqpoint{2.400886in}{0.798487in}}{\pgfqpoint{2.400886in}{0.790251in}}%
\pgfpathcurveto{\pgfqpoint{2.400886in}{0.782015in}}{\pgfqpoint{2.404158in}{0.774115in}}{\pgfqpoint{2.409982in}{0.768291in}}%
\pgfpathcurveto{\pgfqpoint{2.415806in}{0.762467in}}{\pgfqpoint{2.423706in}{0.759195in}}{\pgfqpoint{2.431943in}{0.759195in}}%
\pgfpathclose%
\pgfusepath{stroke,fill}%
\end{pgfscope}%
\begin{pgfscope}%
\pgfpathrectangle{\pgfqpoint{0.100000in}{0.212622in}}{\pgfqpoint{3.696000in}{3.696000in}}%
\pgfusepath{clip}%
\pgfsetbuttcap%
\pgfsetroundjoin%
\definecolor{currentfill}{rgb}{0.121569,0.466667,0.705882}%
\pgfsetfillcolor{currentfill}%
\pgfsetfillopacity{0.999816}%
\pgfsetlinewidth{1.003750pt}%
\definecolor{currentstroke}{rgb}{0.121569,0.466667,0.705882}%
\pgfsetstrokecolor{currentstroke}%
\pgfsetstrokeopacity{0.999816}%
\pgfsetdash{}{0pt}%
\pgfpathmoveto{\pgfqpoint{2.441703in}{0.756434in}}%
\pgfpathcurveto{\pgfqpoint{2.449939in}{0.756434in}}{\pgfqpoint{2.457839in}{0.759706in}}{\pgfqpoint{2.463663in}{0.765530in}}%
\pgfpathcurveto{\pgfqpoint{2.469487in}{0.771354in}}{\pgfqpoint{2.472760in}{0.779254in}}{\pgfqpoint{2.472760in}{0.787490in}}%
\pgfpathcurveto{\pgfqpoint{2.472760in}{0.795727in}}{\pgfqpoint{2.469487in}{0.803627in}}{\pgfqpoint{2.463663in}{0.809451in}}%
\pgfpathcurveto{\pgfqpoint{2.457839in}{0.815275in}}{\pgfqpoint{2.449939in}{0.818547in}}{\pgfqpoint{2.441703in}{0.818547in}}%
\pgfpathcurveto{\pgfqpoint{2.433467in}{0.818547in}}{\pgfqpoint{2.425567in}{0.815275in}}{\pgfqpoint{2.419743in}{0.809451in}}%
\pgfpathcurveto{\pgfqpoint{2.413919in}{0.803627in}}{\pgfqpoint{2.410647in}{0.795727in}}{\pgfqpoint{2.410647in}{0.787490in}}%
\pgfpathcurveto{\pgfqpoint{2.410647in}{0.779254in}}{\pgfqpoint{2.413919in}{0.771354in}}{\pgfqpoint{2.419743in}{0.765530in}}%
\pgfpathcurveto{\pgfqpoint{2.425567in}{0.759706in}}{\pgfqpoint{2.433467in}{0.756434in}}{\pgfqpoint{2.441703in}{0.756434in}}%
\pgfpathclose%
\pgfusepath{stroke,fill}%
\end{pgfscope}%
\begin{pgfscope}%
\pgfpathrectangle{\pgfqpoint{0.100000in}{0.212622in}}{\pgfqpoint{3.696000in}{3.696000in}}%
\pgfusepath{clip}%
\pgfsetbuttcap%
\pgfsetroundjoin%
\definecolor{currentfill}{rgb}{0.121569,0.466667,0.705882}%
\pgfsetfillcolor{currentfill}%
\pgfsetfillopacity{0.999884}%
\pgfsetlinewidth{1.003750pt}%
\definecolor{currentstroke}{rgb}{0.121569,0.466667,0.705882}%
\pgfsetstrokecolor{currentstroke}%
\pgfsetstrokeopacity{0.999884}%
\pgfsetdash{}{0pt}%
\pgfpathmoveto{\pgfqpoint{2.433146in}{0.758739in}}%
\pgfpathcurveto{\pgfqpoint{2.441382in}{0.758739in}}{\pgfqpoint{2.449282in}{0.762012in}}{\pgfqpoint{2.455106in}{0.767836in}}%
\pgfpathcurveto{\pgfqpoint{2.460930in}{0.773660in}}{\pgfqpoint{2.464203in}{0.781560in}}{\pgfqpoint{2.464203in}{0.789796in}}%
\pgfpathcurveto{\pgfqpoint{2.464203in}{0.798032in}}{\pgfqpoint{2.460930in}{0.805932in}}{\pgfqpoint{2.455106in}{0.811756in}}%
\pgfpathcurveto{\pgfqpoint{2.449282in}{0.817580in}}{\pgfqpoint{2.441382in}{0.820852in}}{\pgfqpoint{2.433146in}{0.820852in}}%
\pgfpathcurveto{\pgfqpoint{2.424910in}{0.820852in}}{\pgfqpoint{2.417010in}{0.817580in}}{\pgfqpoint{2.411186in}{0.811756in}}%
\pgfpathcurveto{\pgfqpoint{2.405362in}{0.805932in}}{\pgfqpoint{2.402090in}{0.798032in}}{\pgfqpoint{2.402090in}{0.789796in}}%
\pgfpathcurveto{\pgfqpoint{2.402090in}{0.781560in}}{\pgfqpoint{2.405362in}{0.773660in}}{\pgfqpoint{2.411186in}{0.767836in}}%
\pgfpathcurveto{\pgfqpoint{2.417010in}{0.762012in}}{\pgfqpoint{2.424910in}{0.758739in}}{\pgfqpoint{2.433146in}{0.758739in}}%
\pgfpathclose%
\pgfusepath{stroke,fill}%
\end{pgfscope}%
\begin{pgfscope}%
\pgfpathrectangle{\pgfqpoint{0.100000in}{0.212622in}}{\pgfqpoint{3.696000in}{3.696000in}}%
\pgfusepath{clip}%
\pgfsetbuttcap%
\pgfsetroundjoin%
\definecolor{currentfill}{rgb}{0.121569,0.466667,0.705882}%
\pgfsetfillcolor{currentfill}%
\pgfsetfillopacity{0.999921}%
\pgfsetlinewidth{1.003750pt}%
\definecolor{currentstroke}{rgb}{0.121569,0.466667,0.705882}%
\pgfsetstrokecolor{currentstroke}%
\pgfsetstrokeopacity{0.999921}%
\pgfsetdash{}{0pt}%
\pgfpathmoveto{\pgfqpoint{2.440352in}{0.756621in}}%
\pgfpathcurveto{\pgfqpoint{2.448589in}{0.756621in}}{\pgfqpoint{2.456489in}{0.759894in}}{\pgfqpoint{2.462313in}{0.765717in}}%
\pgfpathcurveto{\pgfqpoint{2.468137in}{0.771541in}}{\pgfqpoint{2.471409in}{0.779441in}}{\pgfqpoint{2.471409in}{0.787678in}}%
\pgfpathcurveto{\pgfqpoint{2.471409in}{0.795914in}}{\pgfqpoint{2.468137in}{0.803814in}}{\pgfqpoint{2.462313in}{0.809638in}}%
\pgfpathcurveto{\pgfqpoint{2.456489in}{0.815462in}}{\pgfqpoint{2.448589in}{0.818734in}}{\pgfqpoint{2.440352in}{0.818734in}}%
\pgfpathcurveto{\pgfqpoint{2.432116in}{0.818734in}}{\pgfqpoint{2.424216in}{0.815462in}}{\pgfqpoint{2.418392in}{0.809638in}}%
\pgfpathcurveto{\pgfqpoint{2.412568in}{0.803814in}}{\pgfqpoint{2.409296in}{0.795914in}}{\pgfqpoint{2.409296in}{0.787678in}}%
\pgfpathcurveto{\pgfqpoint{2.409296in}{0.779441in}}{\pgfqpoint{2.412568in}{0.771541in}}{\pgfqpoint{2.418392in}{0.765717in}}%
\pgfpathcurveto{\pgfqpoint{2.424216in}{0.759894in}}{\pgfqpoint{2.432116in}{0.756621in}}{\pgfqpoint{2.440352in}{0.756621in}}%
\pgfpathclose%
\pgfusepath{stroke,fill}%
\end{pgfscope}%
\begin{pgfscope}%
\pgfpathrectangle{\pgfqpoint{0.100000in}{0.212622in}}{\pgfqpoint{3.696000in}{3.696000in}}%
\pgfusepath{clip}%
\pgfsetbuttcap%
\pgfsetroundjoin%
\definecolor{currentfill}{rgb}{0.121569,0.466667,0.705882}%
\pgfsetfillcolor{currentfill}%
\pgfsetfillopacity{0.999928}%
\pgfsetlinewidth{1.003750pt}%
\definecolor{currentstroke}{rgb}{0.121569,0.466667,0.705882}%
\pgfsetstrokecolor{currentstroke}%
\pgfsetstrokeopacity{0.999928}%
\pgfsetdash{}{0pt}%
\pgfpathmoveto{\pgfqpoint{2.434102in}{0.758385in}}%
\pgfpathcurveto{\pgfqpoint{2.442338in}{0.758385in}}{\pgfqpoint{2.450238in}{0.761658in}}{\pgfqpoint{2.456062in}{0.767482in}}%
\pgfpathcurveto{\pgfqpoint{2.461886in}{0.773306in}}{\pgfqpoint{2.465159in}{0.781206in}}{\pgfqpoint{2.465159in}{0.789442in}}%
\pgfpathcurveto{\pgfqpoint{2.465159in}{0.797678in}}{\pgfqpoint{2.461886in}{0.805578in}}{\pgfqpoint{2.456062in}{0.811402in}}%
\pgfpathcurveto{\pgfqpoint{2.450238in}{0.817226in}}{\pgfqpoint{2.442338in}{0.820498in}}{\pgfqpoint{2.434102in}{0.820498in}}%
\pgfpathcurveto{\pgfqpoint{2.425866in}{0.820498in}}{\pgfqpoint{2.417966in}{0.817226in}}{\pgfqpoint{2.412142in}{0.811402in}}%
\pgfpathcurveto{\pgfqpoint{2.406318in}{0.805578in}}{\pgfqpoint{2.403046in}{0.797678in}}{\pgfqpoint{2.403046in}{0.789442in}}%
\pgfpathcurveto{\pgfqpoint{2.403046in}{0.781206in}}{\pgfqpoint{2.406318in}{0.773306in}}{\pgfqpoint{2.412142in}{0.767482in}}%
\pgfpathcurveto{\pgfqpoint{2.417966in}{0.761658in}}{\pgfqpoint{2.425866in}{0.758385in}}{\pgfqpoint{2.434102in}{0.758385in}}%
\pgfpathclose%
\pgfusepath{stroke,fill}%
\end{pgfscope}%
\begin{pgfscope}%
\pgfpathrectangle{\pgfqpoint{0.100000in}{0.212622in}}{\pgfqpoint{3.696000in}{3.696000in}}%
\pgfusepath{clip}%
\pgfsetbuttcap%
\pgfsetroundjoin%
\definecolor{currentfill}{rgb}{0.121569,0.466667,0.705882}%
\pgfsetfillcolor{currentfill}%
\pgfsetfillopacity{0.999959}%
\pgfsetlinewidth{1.003750pt}%
\definecolor{currentstroke}{rgb}{0.121569,0.466667,0.705882}%
\pgfsetstrokecolor{currentstroke}%
\pgfsetstrokeopacity{0.999959}%
\pgfsetdash{}{0pt}%
\pgfpathmoveto{\pgfqpoint{2.439617in}{0.756750in}}%
\pgfpathcurveto{\pgfqpoint{2.447853in}{0.756750in}}{\pgfqpoint{2.455753in}{0.760023in}}{\pgfqpoint{2.461577in}{0.765847in}}%
\pgfpathcurveto{\pgfqpoint{2.467401in}{0.771671in}}{\pgfqpoint{2.470673in}{0.779571in}}{\pgfqpoint{2.470673in}{0.787807in}}%
\pgfpathcurveto{\pgfqpoint{2.470673in}{0.796043in}}{\pgfqpoint{2.467401in}{0.803943in}}{\pgfqpoint{2.461577in}{0.809767in}}%
\pgfpathcurveto{\pgfqpoint{2.455753in}{0.815591in}}{\pgfqpoint{2.447853in}{0.818863in}}{\pgfqpoint{2.439617in}{0.818863in}}%
\pgfpathcurveto{\pgfqpoint{2.431380in}{0.818863in}}{\pgfqpoint{2.423480in}{0.815591in}}{\pgfqpoint{2.417656in}{0.809767in}}%
\pgfpathcurveto{\pgfqpoint{2.411832in}{0.803943in}}{\pgfqpoint{2.408560in}{0.796043in}}{\pgfqpoint{2.408560in}{0.787807in}}%
\pgfpathcurveto{\pgfqpoint{2.408560in}{0.779571in}}{\pgfqpoint{2.411832in}{0.771671in}}{\pgfqpoint{2.417656in}{0.765847in}}%
\pgfpathcurveto{\pgfqpoint{2.423480in}{0.760023in}}{\pgfqpoint{2.431380in}{0.756750in}}{\pgfqpoint{2.439617in}{0.756750in}}%
\pgfpathclose%
\pgfusepath{stroke,fill}%
\end{pgfscope}%
\begin{pgfscope}%
\pgfpathrectangle{\pgfqpoint{0.100000in}{0.212622in}}{\pgfqpoint{3.696000in}{3.696000in}}%
\pgfusepath{clip}%
\pgfsetbuttcap%
\pgfsetroundjoin%
\definecolor{currentfill}{rgb}{0.121569,0.466667,0.705882}%
\pgfsetfillcolor{currentfill}%
\pgfsetfillopacity{0.999983}%
\pgfsetlinewidth{1.003750pt}%
\definecolor{currentstroke}{rgb}{0.121569,0.466667,0.705882}%
\pgfsetstrokecolor{currentstroke}%
\pgfsetstrokeopacity{0.999983}%
\pgfsetdash{}{0pt}%
\pgfpathmoveto{\pgfqpoint{2.435859in}{0.757764in}}%
\pgfpathcurveto{\pgfqpoint{2.444095in}{0.757764in}}{\pgfqpoint{2.451995in}{0.761036in}}{\pgfqpoint{2.457819in}{0.766860in}}%
\pgfpathcurveto{\pgfqpoint{2.463643in}{0.772684in}}{\pgfqpoint{2.466916in}{0.780584in}}{\pgfqpoint{2.466916in}{0.788820in}}%
\pgfpathcurveto{\pgfqpoint{2.466916in}{0.797056in}}{\pgfqpoint{2.463643in}{0.804956in}}{\pgfqpoint{2.457819in}{0.810780in}}%
\pgfpathcurveto{\pgfqpoint{2.451995in}{0.816604in}}{\pgfqpoint{2.444095in}{0.819877in}}{\pgfqpoint{2.435859in}{0.819877in}}%
\pgfpathcurveto{\pgfqpoint{2.427623in}{0.819877in}}{\pgfqpoint{2.419723in}{0.816604in}}{\pgfqpoint{2.413899in}{0.810780in}}%
\pgfpathcurveto{\pgfqpoint{2.408075in}{0.804956in}}{\pgfqpoint{2.404803in}{0.797056in}}{\pgfqpoint{2.404803in}{0.788820in}}%
\pgfpathcurveto{\pgfqpoint{2.404803in}{0.780584in}}{\pgfqpoint{2.408075in}{0.772684in}}{\pgfqpoint{2.413899in}{0.766860in}}%
\pgfpathcurveto{\pgfqpoint{2.419723in}{0.761036in}}{\pgfqpoint{2.427623in}{0.757764in}}{\pgfqpoint{2.435859in}{0.757764in}}%
\pgfpathclose%
\pgfusepath{stroke,fill}%
\end{pgfscope}%
\begin{pgfscope}%
\pgfpathrectangle{\pgfqpoint{0.100000in}{0.212622in}}{\pgfqpoint{3.696000in}{3.696000in}}%
\pgfusepath{clip}%
\pgfsetbuttcap%
\pgfsetroundjoin%
\definecolor{currentfill}{rgb}{0.121569,0.466667,0.705882}%
\pgfsetfillcolor{currentfill}%
\pgfsetfillopacity{0.999989}%
\pgfsetlinewidth{1.003750pt}%
\definecolor{currentstroke}{rgb}{0.121569,0.466667,0.705882}%
\pgfsetstrokecolor{currentstroke}%
\pgfsetstrokeopacity{0.999989}%
\pgfsetdash{}{0pt}%
\pgfpathmoveto{\pgfqpoint{2.438625in}{0.756982in}}%
\pgfpathcurveto{\pgfqpoint{2.446861in}{0.756982in}}{\pgfqpoint{2.454761in}{0.760254in}}{\pgfqpoint{2.460585in}{0.766078in}}%
\pgfpathcurveto{\pgfqpoint{2.466409in}{0.771902in}}{\pgfqpoint{2.469681in}{0.779802in}}{\pgfqpoint{2.469681in}{0.788038in}}%
\pgfpathcurveto{\pgfqpoint{2.469681in}{0.796275in}}{\pgfqpoint{2.466409in}{0.804175in}}{\pgfqpoint{2.460585in}{0.809999in}}%
\pgfpathcurveto{\pgfqpoint{2.454761in}{0.815823in}}{\pgfqpoint{2.446861in}{0.819095in}}{\pgfqpoint{2.438625in}{0.819095in}}%
\pgfpathcurveto{\pgfqpoint{2.430389in}{0.819095in}}{\pgfqpoint{2.422488in}{0.815823in}}{\pgfqpoint{2.416665in}{0.809999in}}%
\pgfpathcurveto{\pgfqpoint{2.410841in}{0.804175in}}{\pgfqpoint{2.407568in}{0.796275in}}{\pgfqpoint{2.407568in}{0.788038in}}%
\pgfpathcurveto{\pgfqpoint{2.407568in}{0.779802in}}{\pgfqpoint{2.410841in}{0.771902in}}{\pgfqpoint{2.416665in}{0.766078in}}%
\pgfpathcurveto{\pgfqpoint{2.422488in}{0.760254in}}{\pgfqpoint{2.430389in}{0.756982in}}{\pgfqpoint{2.438625in}{0.756982in}}%
\pgfpathclose%
\pgfusepath{stroke,fill}%
\end{pgfscope}%
\begin{pgfscope}%
\pgfpathrectangle{\pgfqpoint{0.100000in}{0.212622in}}{\pgfqpoint{3.696000in}{3.696000in}}%
\pgfusepath{clip}%
\pgfsetbuttcap%
\pgfsetroundjoin%
\definecolor{currentfill}{rgb}{0.121569,0.466667,0.705882}%
\pgfsetfillcolor{currentfill}%
\pgfsetlinewidth{1.003750pt}%
\definecolor{currentstroke}{rgb}{0.121569,0.466667,0.705882}%
\pgfsetstrokecolor{currentstroke}%
\pgfsetdash{}{0pt}%
\pgfpathmoveto{\pgfqpoint{2.437375in}{0.757319in}}%
\pgfpathcurveto{\pgfqpoint{2.445612in}{0.757319in}}{\pgfqpoint{2.453512in}{0.760591in}}{\pgfqpoint{2.459336in}{0.766415in}}%
\pgfpathcurveto{\pgfqpoint{2.465160in}{0.772239in}}{\pgfqpoint{2.468432in}{0.780139in}}{\pgfqpoint{2.468432in}{0.788375in}}%
\pgfpathcurveto{\pgfqpoint{2.468432in}{0.796611in}}{\pgfqpoint{2.465160in}{0.804511in}}{\pgfqpoint{2.459336in}{0.810335in}}%
\pgfpathcurveto{\pgfqpoint{2.453512in}{0.816159in}}{\pgfqpoint{2.445612in}{0.819432in}}{\pgfqpoint{2.437375in}{0.819432in}}%
\pgfpathcurveto{\pgfqpoint{2.429139in}{0.819432in}}{\pgfqpoint{2.421239in}{0.816159in}}{\pgfqpoint{2.415415in}{0.810335in}}%
\pgfpathcurveto{\pgfqpoint{2.409591in}{0.804511in}}{\pgfqpoint{2.406319in}{0.796611in}}{\pgfqpoint{2.406319in}{0.788375in}}%
\pgfpathcurveto{\pgfqpoint{2.406319in}{0.780139in}}{\pgfqpoint{2.409591in}{0.772239in}}{\pgfqpoint{2.415415in}{0.766415in}}%
\pgfpathcurveto{\pgfqpoint{2.421239in}{0.760591in}}{\pgfqpoint{2.429139in}{0.757319in}}{\pgfqpoint{2.437375in}{0.757319in}}%
\pgfpathclose%
\pgfusepath{stroke,fill}%
\end{pgfscope}%
\begin{pgfscope}%
\pgfsetbuttcap%
\pgfsetmiterjoin%
\definecolor{currentfill}{rgb}{1.000000,1.000000,1.000000}%
\pgfsetfillcolor{currentfill}%
\pgfsetfillopacity{0.800000}%
\pgfsetlinewidth{1.003750pt}%
\definecolor{currentstroke}{rgb}{0.800000,0.800000,0.800000}%
\pgfsetstrokecolor{currentstroke}%
\pgfsetstrokeopacity{0.800000}%
\pgfsetdash{}{0pt}%
\pgfpathmoveto{\pgfqpoint{2.104889in}{3.216678in}}%
\pgfpathlineto{\pgfqpoint{3.698778in}{3.216678in}}%
\pgfpathquadraticcurveto{\pgfqpoint{3.726556in}{3.216678in}}{\pgfqpoint{3.726556in}{3.244456in}}%
\pgfpathlineto{\pgfqpoint{3.726556in}{3.811400in}}%
\pgfpathquadraticcurveto{\pgfqpoint{3.726556in}{3.839178in}}{\pgfqpoint{3.698778in}{3.839178in}}%
\pgfpathlineto{\pgfqpoint{2.104889in}{3.839178in}}%
\pgfpathquadraticcurveto{\pgfqpoint{2.077111in}{3.839178in}}{\pgfqpoint{2.077111in}{3.811400in}}%
\pgfpathlineto{\pgfqpoint{2.077111in}{3.244456in}}%
\pgfpathquadraticcurveto{\pgfqpoint{2.077111in}{3.216678in}}{\pgfqpoint{2.104889in}{3.216678in}}%
\pgfpathclose%
\pgfusepath{stroke,fill}%
\end{pgfscope}%
\begin{pgfscope}%
\pgfsetrectcap%
\pgfsetroundjoin%
\pgfsetlinewidth{1.505625pt}%
\definecolor{currentstroke}{rgb}{0.121569,0.466667,0.705882}%
\pgfsetstrokecolor{currentstroke}%
\pgfsetdash{}{0pt}%
\pgfpathmoveto{\pgfqpoint{2.132667in}{3.735011in}}%
\pgfpathlineto{\pgfqpoint{2.410444in}{3.735011in}}%
\pgfusepath{stroke}%
\end{pgfscope}%
\begin{pgfscope}%
\definecolor{textcolor}{rgb}{0.000000,0.000000,0.000000}%
\pgfsetstrokecolor{textcolor}%
\pgfsetfillcolor{textcolor}%
\pgftext[x=2.521555in,y=3.686400in,left,base]{\color{textcolor}\rmfamily\fontsize{10.000000}{12.000000}\selectfont Ground truth}%
\end{pgfscope}%
\begin{pgfscope}%
\pgfsetbuttcap%
\pgfsetroundjoin%
\definecolor{currentfill}{rgb}{0.121569,0.466667,0.705882}%
\pgfsetfillcolor{currentfill}%
\pgfsetlinewidth{1.003750pt}%
\definecolor{currentstroke}{rgb}{0.121569,0.466667,0.705882}%
\pgfsetstrokecolor{currentstroke}%
\pgfsetdash{}{0pt}%
\pgfsys@defobject{currentmarker}{\pgfqpoint{-0.031056in}{-0.031056in}}{\pgfqpoint{0.031056in}{0.031056in}}{%
\pgfpathmoveto{\pgfqpoint{0.000000in}{-0.031056in}}%
\pgfpathcurveto{\pgfqpoint{0.008236in}{-0.031056in}}{\pgfqpoint{0.016136in}{-0.027784in}}{\pgfqpoint{0.021960in}{-0.021960in}}%
\pgfpathcurveto{\pgfqpoint{0.027784in}{-0.016136in}}{\pgfqpoint{0.031056in}{-0.008236in}}{\pgfqpoint{0.031056in}{0.000000in}}%
\pgfpathcurveto{\pgfqpoint{0.031056in}{0.008236in}}{\pgfqpoint{0.027784in}{0.016136in}}{\pgfqpoint{0.021960in}{0.021960in}}%
\pgfpathcurveto{\pgfqpoint{0.016136in}{0.027784in}}{\pgfqpoint{0.008236in}{0.031056in}}{\pgfqpoint{0.000000in}{0.031056in}}%
\pgfpathcurveto{\pgfqpoint{-0.008236in}{0.031056in}}{\pgfqpoint{-0.016136in}{0.027784in}}{\pgfqpoint{-0.021960in}{0.021960in}}%
\pgfpathcurveto{\pgfqpoint{-0.027784in}{0.016136in}}{\pgfqpoint{-0.031056in}{0.008236in}}{\pgfqpoint{-0.031056in}{0.000000in}}%
\pgfpathcurveto{\pgfqpoint{-0.031056in}{-0.008236in}}{\pgfqpoint{-0.027784in}{-0.016136in}}{\pgfqpoint{-0.021960in}{-0.021960in}}%
\pgfpathcurveto{\pgfqpoint{-0.016136in}{-0.027784in}}{\pgfqpoint{-0.008236in}{-0.031056in}}{\pgfqpoint{0.000000in}{-0.031056in}}%
\pgfpathclose%
\pgfusepath{stroke,fill}%
}%
\begin{pgfscope}%
\pgfsys@transformshift{2.271555in}{3.529248in}%
\pgfsys@useobject{currentmarker}{}%
\end{pgfscope}%
\end{pgfscope}%
\begin{pgfscope}%
\definecolor{textcolor}{rgb}{0.000000,0.000000,0.000000}%
\pgfsetstrokecolor{textcolor}%
\pgfsetfillcolor{textcolor}%
\pgftext[x=2.521555in,y=3.492789in,left,base]{\color{textcolor}\rmfamily\fontsize{10.000000}{12.000000}\selectfont Estimated position}%
\end{pgfscope}%
\begin{pgfscope}%
\pgfsetbuttcap%
\pgfsetroundjoin%
\definecolor{currentfill}{rgb}{1.000000,0.498039,0.054902}%
\pgfsetfillcolor{currentfill}%
\pgfsetlinewidth{1.003750pt}%
\definecolor{currentstroke}{rgb}{1.000000,0.498039,0.054902}%
\pgfsetstrokecolor{currentstroke}%
\pgfsetdash{}{0pt}%
\pgfsys@defobject{currentmarker}{\pgfqpoint{-0.031056in}{-0.031056in}}{\pgfqpoint{0.031056in}{0.031056in}}{%
\pgfpathmoveto{\pgfqpoint{0.000000in}{-0.031056in}}%
\pgfpathcurveto{\pgfqpoint{0.008236in}{-0.031056in}}{\pgfqpoint{0.016136in}{-0.027784in}}{\pgfqpoint{0.021960in}{-0.021960in}}%
\pgfpathcurveto{\pgfqpoint{0.027784in}{-0.016136in}}{\pgfqpoint{0.031056in}{-0.008236in}}{\pgfqpoint{0.031056in}{0.000000in}}%
\pgfpathcurveto{\pgfqpoint{0.031056in}{0.008236in}}{\pgfqpoint{0.027784in}{0.016136in}}{\pgfqpoint{0.021960in}{0.021960in}}%
\pgfpathcurveto{\pgfqpoint{0.016136in}{0.027784in}}{\pgfqpoint{0.008236in}{0.031056in}}{\pgfqpoint{0.000000in}{0.031056in}}%
\pgfpathcurveto{\pgfqpoint{-0.008236in}{0.031056in}}{\pgfqpoint{-0.016136in}{0.027784in}}{\pgfqpoint{-0.021960in}{0.021960in}}%
\pgfpathcurveto{\pgfqpoint{-0.027784in}{0.016136in}}{\pgfqpoint{-0.031056in}{0.008236in}}{\pgfqpoint{-0.031056in}{0.000000in}}%
\pgfpathcurveto{\pgfqpoint{-0.031056in}{-0.008236in}}{\pgfqpoint{-0.027784in}{-0.016136in}}{\pgfqpoint{-0.021960in}{-0.021960in}}%
\pgfpathcurveto{\pgfqpoint{-0.016136in}{-0.027784in}}{\pgfqpoint{-0.008236in}{-0.031056in}}{\pgfqpoint{0.000000in}{-0.031056in}}%
\pgfpathclose%
\pgfusepath{stroke,fill}%
}%
\begin{pgfscope}%
\pgfsys@transformshift{2.271555in}{3.335637in}%
\pgfsys@useobject{currentmarker}{}%
\end{pgfscope}%
\end{pgfscope}%
\begin{pgfscope}%
\definecolor{textcolor}{rgb}{0.000000,0.000000,0.000000}%
\pgfsetstrokecolor{textcolor}%
\pgfsetfillcolor{textcolor}%
\pgftext[x=2.521555in,y=3.299178in,left,base]{\color{textcolor}\rmfamily\fontsize{10.000000}{12.000000}\selectfont Estimated turn}%
\end{pgfscope}%
\end{pgfpicture}%
\makeatother%
\endgroup%
}
%         \caption{EKF's 3D position estimation had the lowest turn error for the 28-meter side triangle experiment.}
%         \label{fig:triangle28_3D}
%     \end{subfigure}
%     \caption{Position estimation by the best performing algorithms in the 28-meter side triangle experiment.}
%     \label{fig:triangle28}
% \end{figure}

% \subsection{Square}

% The square shape consisted of moving the inertial system in a square patter for a determined distance. 3 line distances were tested: 4, 16, and 28 meter. The results are shown below:

% \subsubsection{4 meter}

% For the 16-meter square experiment, the Tilt algorithm which had the lowest displacement error with an average of 0.30 meters (16\% of error margin), and FLAE with an average of 0.24 meters of turn error (7\% of error margin).


% \begin{figure}[!h]
%     \centering
%     \begin{table}[H]
    \begin{center}
        \resizebox{1\linewidth}{!}{

            \begin{tabular}[t]{lcccc}
                \hline
                Algorithm     & Displacement Error[$m$] & Displacement Error[\%] & Turn Error[$m$] & Turn Error[\%] \\
                \hline
                AngularRate   & 5.97                    & 37.29                  & 7.29            & 45.58          \\
                AQUA          & 4.12                    & 25.75                  & 4.20            & 26.23          \\
                Complementary & 1.20                    & 7.50                   & 1.46            & 9.13           \\
                Davenport     & 0.69                    & 4.31                   & 0.43            & 2.70           \\
                EKF           & 1.09                    & 6.79                   & 1.53            & 9.56           \\
                FAMC          & 6.01                    & 37.57                  & 6.98            & 43.61          \\
                FLAE          & 0.69                    & 4.32                   & 0.38            & 2.37           \\
                Fourati       & 9.08                    & 56.78                  & 9.74            & 60.87          \\
                Madgwick      & 1.02                    & 6.40                   & 1.18            & 7.36           \\
                Mahony        & 0.53                    & 3.34                   & 0.37            & 2.29           \\
                OLEQ          & 0.60                    & 3.74                   & 0.52            & 3.23           \\
                QUEST         & 5.07                    & 31.71                  & 5.57            & 34.79          \\
                ROLEQ         & 0.78                    & 4.90                   & 0.88            & 5.48           \\
                SAAM          & 0.59                    & 3.69                   & 0.37            & 2.33           \\
                Tilt          & 0.59                    & 3.69                   & 0.37            & 2.33           \\

                \hline
                Average       & 2.54                    & 15.85                  & 2.75            & 17.19
            \end{tabular}
        }
        \caption{4 meter square position estimation error (displacement and turn) of the sensor fusion algorithms. }
        \label{tab:4_square}
    \end{center}
\end{table}
% \end{figure}

% \begin{figure}[!h]
%     \centering
%     \begin{subfigure}{0.49\textwidth}
%         \centering
%         \resizebox{1\linewidth}{!}{%% Creator: Matplotlib, PGF backend
%%
%% To include the figure in your LaTeX document, write
%%   \input{<filename>.pgf}
%%
%% Make sure the required packages are loaded in your preamble
%%   \usepackage{pgf}
%%
%% and, on pdftex
%%   \usepackage[utf8]{inputenc}\DeclareUnicodeCharacter{2212}{-}
%%
%% or, on luatex and xetex
%%   \usepackage{unicode-math}
%%
%% Figures using additional raster images can only be included by \input if
%% they are in the same directory as the main LaTeX file. For loading figures
%% from other directories you can use the `import` package
%%   \usepackage{import}
%%
%% and then include the figures with
%%   \import{<path to file>}{<filename>.pgf}
%%
%% Matplotlib used the following preamble
%%   \usepackage{fontspec}
%%
\begingroup%
\makeatletter%
\begin{pgfpicture}%
\pgfpathrectangle{\pgfpointorigin}{\pgfqpoint{4.342355in}{4.207622in}}%
\pgfusepath{use as bounding box, clip}%
\begin{pgfscope}%
\pgfsetbuttcap%
\pgfsetmiterjoin%
\definecolor{currentfill}{rgb}{1.000000,1.000000,1.000000}%
\pgfsetfillcolor{currentfill}%
\pgfsetlinewidth{0.000000pt}%
\definecolor{currentstroke}{rgb}{1.000000,1.000000,1.000000}%
\pgfsetstrokecolor{currentstroke}%
\pgfsetdash{}{0pt}%
\pgfpathmoveto{\pgfqpoint{0.000000in}{0.000000in}}%
\pgfpathlineto{\pgfqpoint{4.342355in}{0.000000in}}%
\pgfpathlineto{\pgfqpoint{4.342355in}{4.207622in}}%
\pgfpathlineto{\pgfqpoint{0.000000in}{4.207622in}}%
\pgfpathclose%
\pgfusepath{fill}%
\end{pgfscope}%
\begin{pgfscope}%
\pgfsetbuttcap%
\pgfsetmiterjoin%
\definecolor{currentfill}{rgb}{1.000000,1.000000,1.000000}%
\pgfsetfillcolor{currentfill}%
\pgfsetlinewidth{0.000000pt}%
\definecolor{currentstroke}{rgb}{0.000000,0.000000,0.000000}%
\pgfsetstrokecolor{currentstroke}%
\pgfsetstrokeopacity{0.000000}%
\pgfsetdash{}{0pt}%
\pgfpathmoveto{\pgfqpoint{0.100000in}{0.212622in}}%
\pgfpathlineto{\pgfqpoint{3.796000in}{0.212622in}}%
\pgfpathlineto{\pgfqpoint{3.796000in}{3.908622in}}%
\pgfpathlineto{\pgfqpoint{0.100000in}{3.908622in}}%
\pgfpathclose%
\pgfusepath{fill}%
\end{pgfscope}%
\begin{pgfscope}%
\pgfsetbuttcap%
\pgfsetmiterjoin%
\definecolor{currentfill}{rgb}{0.950000,0.950000,0.950000}%
\pgfsetfillcolor{currentfill}%
\pgfsetfillopacity{0.500000}%
\pgfsetlinewidth{1.003750pt}%
\definecolor{currentstroke}{rgb}{0.950000,0.950000,0.950000}%
\pgfsetstrokecolor{currentstroke}%
\pgfsetstrokeopacity{0.500000}%
\pgfsetdash{}{0pt}%
\pgfpathmoveto{\pgfqpoint{0.379073in}{1.123938in}}%
\pgfpathlineto{\pgfqpoint{1.599613in}{2.147018in}}%
\pgfpathlineto{\pgfqpoint{1.582647in}{3.622484in}}%
\pgfpathlineto{\pgfqpoint{0.303698in}{2.689165in}}%
\pgfusepath{stroke,fill}%
\end{pgfscope}%
\begin{pgfscope}%
\pgfsetbuttcap%
\pgfsetmiterjoin%
\definecolor{currentfill}{rgb}{0.900000,0.900000,0.900000}%
\pgfsetfillcolor{currentfill}%
\pgfsetfillopacity{0.500000}%
\pgfsetlinewidth{1.003750pt}%
\definecolor{currentstroke}{rgb}{0.900000,0.900000,0.900000}%
\pgfsetstrokecolor{currentstroke}%
\pgfsetstrokeopacity{0.500000}%
\pgfsetdash{}{0pt}%
\pgfpathmoveto{\pgfqpoint{1.599613in}{2.147018in}}%
\pgfpathlineto{\pgfqpoint{3.558144in}{1.577751in}}%
\pgfpathlineto{\pgfqpoint{3.628038in}{3.104037in}}%
\pgfpathlineto{\pgfqpoint{1.582647in}{3.622484in}}%
\pgfusepath{stroke,fill}%
\end{pgfscope}%
\begin{pgfscope}%
\pgfsetbuttcap%
\pgfsetmiterjoin%
\definecolor{currentfill}{rgb}{0.925000,0.925000,0.925000}%
\pgfsetfillcolor{currentfill}%
\pgfsetfillopacity{0.500000}%
\pgfsetlinewidth{1.003750pt}%
\definecolor{currentstroke}{rgb}{0.925000,0.925000,0.925000}%
\pgfsetstrokecolor{currentstroke}%
\pgfsetstrokeopacity{0.500000}%
\pgfsetdash{}{0pt}%
\pgfpathmoveto{\pgfqpoint{0.379073in}{1.123938in}}%
\pgfpathlineto{\pgfqpoint{2.455212in}{0.445871in}}%
\pgfpathlineto{\pgfqpoint{3.558144in}{1.577751in}}%
\pgfpathlineto{\pgfqpoint{1.599613in}{2.147018in}}%
\pgfusepath{stroke,fill}%
\end{pgfscope}%
\begin{pgfscope}%
\pgfsetrectcap%
\pgfsetroundjoin%
\pgfsetlinewidth{0.803000pt}%
\definecolor{currentstroke}{rgb}{0.000000,0.000000,0.000000}%
\pgfsetstrokecolor{currentstroke}%
\pgfsetdash{}{0pt}%
\pgfpathmoveto{\pgfqpoint{0.379073in}{1.123938in}}%
\pgfpathlineto{\pgfqpoint{2.455212in}{0.445871in}}%
\pgfusepath{stroke}%
\end{pgfscope}%
\begin{pgfscope}%
\definecolor{textcolor}{rgb}{0.000000,0.000000,0.000000}%
\pgfsetstrokecolor{textcolor}%
\pgfsetfillcolor{textcolor}%
\pgftext[x=0.730374in, y=0.408886in, left, base,rotate=341.912962]{\color{textcolor}\rmfamily\fontsize{10.000000}{12.000000}\selectfont Position X [\(\displaystyle m\)]}%
\end{pgfscope}%
\begin{pgfscope}%
\pgfsetbuttcap%
\pgfsetroundjoin%
\pgfsetlinewidth{0.803000pt}%
\definecolor{currentstroke}{rgb}{0.690196,0.690196,0.690196}%
\pgfsetstrokecolor{currentstroke}%
\pgfsetdash{}{0pt}%
\pgfpathmoveto{\pgfqpoint{0.710603in}{1.015660in}}%
\pgfpathlineto{\pgfqpoint{1.913525in}{2.055777in}}%
\pgfpathlineto{\pgfqpoint{1.909899in}{3.539535in}}%
\pgfusepath{stroke}%
\end{pgfscope}%
\begin{pgfscope}%
\pgfsetbuttcap%
\pgfsetroundjoin%
\pgfsetlinewidth{0.803000pt}%
\definecolor{currentstroke}{rgb}{0.690196,0.690196,0.690196}%
\pgfsetstrokecolor{currentstroke}%
\pgfsetdash{}{0pt}%
\pgfpathmoveto{\pgfqpoint{1.057043in}{0.902513in}}%
\pgfpathlineto{\pgfqpoint{2.241081in}{1.960569in}}%
\pgfpathlineto{\pgfqpoint{2.251612in}{3.452920in}}%
\pgfusepath{stroke}%
\end{pgfscope}%
\begin{pgfscope}%
\pgfsetbuttcap%
\pgfsetroundjoin%
\pgfsetlinewidth{0.803000pt}%
\definecolor{currentstroke}{rgb}{0.690196,0.690196,0.690196}%
\pgfsetstrokecolor{currentstroke}%
\pgfsetdash{}{0pt}%
\pgfpathmoveto{\pgfqpoint{1.409746in}{0.787320in}}%
\pgfpathlineto{\pgfqpoint{2.574063in}{1.863784in}}%
\pgfpathlineto{\pgfqpoint{2.599233in}{3.364809in}}%
\pgfusepath{stroke}%
\end{pgfscope}%
\begin{pgfscope}%
\pgfsetbuttcap%
\pgfsetroundjoin%
\pgfsetlinewidth{0.803000pt}%
\definecolor{currentstroke}{rgb}{0.690196,0.690196,0.690196}%
\pgfsetstrokecolor{currentstroke}%
\pgfsetdash{}{0pt}%
\pgfpathmoveto{\pgfqpoint{1.768885in}{0.670025in}}%
\pgfpathlineto{\pgfqpoint{2.912608in}{1.765383in}}%
\pgfpathlineto{\pgfqpoint{2.952917in}{3.275160in}}%
\pgfusepath{stroke}%
\end{pgfscope}%
\begin{pgfscope}%
\pgfsetbuttcap%
\pgfsetroundjoin%
\pgfsetlinewidth{0.803000pt}%
\definecolor{currentstroke}{rgb}{0.690196,0.690196,0.690196}%
\pgfsetstrokecolor{currentstroke}%
\pgfsetdash{}{0pt}%
\pgfpathmoveto{\pgfqpoint{2.134636in}{0.550571in}}%
\pgfpathlineto{\pgfqpoint{3.256855in}{1.665324in}}%
\pgfpathlineto{\pgfqpoint{3.312824in}{3.183934in}}%
\pgfusepath{stroke}%
\end{pgfscope}%
\begin{pgfscope}%
\pgfsetrectcap%
\pgfsetroundjoin%
\pgfsetlinewidth{0.803000pt}%
\definecolor{currentstroke}{rgb}{0.000000,0.000000,0.000000}%
\pgfsetstrokecolor{currentstroke}%
\pgfsetdash{}{0pt}%
\pgfpathmoveto{\pgfqpoint{0.721083in}{1.024721in}}%
\pgfpathlineto{\pgfqpoint{0.689599in}{0.997499in}}%
\pgfusepath{stroke}%
\end{pgfscope}%
\begin{pgfscope}%
\definecolor{textcolor}{rgb}{0.000000,0.000000,0.000000}%
\pgfsetstrokecolor{textcolor}%
\pgfsetfillcolor{textcolor}%
\pgftext[x=0.606240in,y=0.796033in,,top]{\color{textcolor}\rmfamily\fontsize{10.000000}{12.000000}\selectfont \(\displaystyle {0}\)}%
\end{pgfscope}%
\begin{pgfscope}%
\pgfsetrectcap%
\pgfsetroundjoin%
\pgfsetlinewidth{0.803000pt}%
\definecolor{currentstroke}{rgb}{0.000000,0.000000,0.000000}%
\pgfsetstrokecolor{currentstroke}%
\pgfsetdash{}{0pt}%
\pgfpathmoveto{\pgfqpoint{1.067365in}{0.911737in}}%
\pgfpathlineto{\pgfqpoint{1.036353in}{0.884024in}}%
\pgfusepath{stroke}%
\end{pgfscope}%
\begin{pgfscope}%
\definecolor{textcolor}{rgb}{0.000000,0.000000,0.000000}%
\pgfsetstrokecolor{textcolor}%
\pgfsetfillcolor{textcolor}%
\pgftext[x=0.953048in,y=0.680484in,,top]{\color{textcolor}\rmfamily\fontsize{10.000000}{12.000000}\selectfont \(\displaystyle {1}\)}%
\end{pgfscope}%
\begin{pgfscope}%
\pgfsetrectcap%
\pgfsetroundjoin%
\pgfsetlinewidth{0.803000pt}%
\definecolor{currentstroke}{rgb}{0.000000,0.000000,0.000000}%
\pgfsetstrokecolor{currentstroke}%
\pgfsetdash{}{0pt}%
\pgfpathmoveto{\pgfqpoint{1.419904in}{0.796712in}}%
\pgfpathlineto{\pgfqpoint{1.389385in}{0.768495in}}%
\pgfusepath{stroke}%
\end{pgfscope}%
\begin{pgfscope}%
\definecolor{textcolor}{rgb}{0.000000,0.000000,0.000000}%
\pgfsetstrokecolor{textcolor}%
\pgfsetfillcolor{textcolor}%
\pgftext[x=1.306150in,y=0.562838in,,top]{\color{textcolor}\rmfamily\fontsize{10.000000}{12.000000}\selectfont \(\displaystyle {2}\)}%
\end{pgfscope}%
\begin{pgfscope}%
\pgfsetrectcap%
\pgfsetroundjoin%
\pgfsetlinewidth{0.803000pt}%
\definecolor{currentstroke}{rgb}{0.000000,0.000000,0.000000}%
\pgfsetstrokecolor{currentstroke}%
\pgfsetdash{}{0pt}%
\pgfpathmoveto{\pgfqpoint{1.778871in}{0.679589in}}%
\pgfpathlineto{\pgfqpoint{1.748868in}{0.650855in}}%
\pgfusepath{stroke}%
\end{pgfscope}%
\begin{pgfscope}%
\definecolor{textcolor}{rgb}{0.000000,0.000000,0.000000}%
\pgfsetstrokecolor{textcolor}%
\pgfsetfillcolor{textcolor}%
\pgftext[x=1.665718in,y=0.443037in,,top]{\color{textcolor}\rmfamily\fontsize{10.000000}{12.000000}\selectfont \(\displaystyle {3}\)}%
\end{pgfscope}%
\begin{pgfscope}%
\pgfsetrectcap%
\pgfsetroundjoin%
\pgfsetlinewidth{0.803000pt}%
\definecolor{currentstroke}{rgb}{0.000000,0.000000,0.000000}%
\pgfsetstrokecolor{currentstroke}%
\pgfsetdash{}{0pt}%
\pgfpathmoveto{\pgfqpoint{2.144442in}{0.560312in}}%
\pgfpathlineto{\pgfqpoint{2.114980in}{0.531046in}}%
\pgfusepath{stroke}%
\end{pgfscope}%
\begin{pgfscope}%
\definecolor{textcolor}{rgb}{0.000000,0.000000,0.000000}%
\pgfsetstrokecolor{textcolor}%
\pgfsetfillcolor{textcolor}%
\pgftext[x=2.031934in,y=0.321022in,,top]{\color{textcolor}\rmfamily\fontsize{10.000000}{12.000000}\selectfont \(\displaystyle {4}\)}%
\end{pgfscope}%
\begin{pgfscope}%
\pgfsetrectcap%
\pgfsetroundjoin%
\pgfsetlinewidth{0.803000pt}%
\definecolor{currentstroke}{rgb}{0.000000,0.000000,0.000000}%
\pgfsetstrokecolor{currentstroke}%
\pgfsetdash{}{0pt}%
\pgfpathmoveto{\pgfqpoint{3.558144in}{1.577751in}}%
\pgfpathlineto{\pgfqpoint{2.455212in}{0.445871in}}%
\pgfusepath{stroke}%
\end{pgfscope}%
\begin{pgfscope}%
\definecolor{textcolor}{rgb}{0.000000,0.000000,0.000000}%
\pgfsetstrokecolor{textcolor}%
\pgfsetfillcolor{textcolor}%
\pgftext[x=3.120747in, y=0.305657in, left, base,rotate=45.742112]{\color{textcolor}\rmfamily\fontsize{10.000000}{12.000000}\selectfont Position Y [\(\displaystyle m\)]}%
\end{pgfscope}%
\begin{pgfscope}%
\pgfsetbuttcap%
\pgfsetroundjoin%
\pgfsetlinewidth{0.803000pt}%
\definecolor{currentstroke}{rgb}{0.690196,0.690196,0.690196}%
\pgfsetstrokecolor{currentstroke}%
\pgfsetdash{}{0pt}%
\pgfpathmoveto{\pgfqpoint{0.526775in}{2.851957in}}%
\pgfpathlineto{\pgfqpoint{0.591297in}{1.301827in}}%
\pgfpathlineto{\pgfqpoint{2.647687in}{0.643397in}}%
\pgfusepath{stroke}%
\end{pgfscope}%
\begin{pgfscope}%
\pgfsetbuttcap%
\pgfsetroundjoin%
\pgfsetlinewidth{0.803000pt}%
\definecolor{currentstroke}{rgb}{0.690196,0.690196,0.690196}%
\pgfsetstrokecolor{currentstroke}%
\pgfsetdash{}{0pt}%
\pgfpathmoveto{\pgfqpoint{0.771236in}{3.030353in}}%
\pgfpathlineto{\pgfqpoint{0.824185in}{1.497039in}}%
\pgfpathlineto{\pgfqpoint{2.858563in}{0.859808in}}%
\pgfusepath{stroke}%
\end{pgfscope}%
\begin{pgfscope}%
\pgfsetbuttcap%
\pgfsetroundjoin%
\pgfsetlinewidth{0.803000pt}%
\definecolor{currentstroke}{rgb}{0.690196,0.690196,0.690196}%
\pgfsetstrokecolor{currentstroke}%
\pgfsetdash{}{0pt}%
\pgfpathmoveto{\pgfqpoint{1.007746in}{3.202947in}}%
\pgfpathlineto{\pgfqpoint{1.049821in}{1.686172in}}%
\pgfpathlineto{\pgfqpoint{3.062534in}{1.069132in}}%
\pgfusepath{stroke}%
\end{pgfscope}%
\begin{pgfscope}%
\pgfsetbuttcap%
\pgfsetroundjoin%
\pgfsetlinewidth{0.803000pt}%
\definecolor{currentstroke}{rgb}{0.690196,0.690196,0.690196}%
\pgfsetstrokecolor{currentstroke}%
\pgfsetdash{}{0pt}%
\pgfpathmoveto{\pgfqpoint{1.236688in}{3.370019in}}%
\pgfpathlineto{\pgfqpoint{1.268540in}{1.869506in}}%
\pgfpathlineto{\pgfqpoint{3.259934in}{1.271713in}}%
\pgfusepath{stroke}%
\end{pgfscope}%
\begin{pgfscope}%
\pgfsetbuttcap%
\pgfsetroundjoin%
\pgfsetlinewidth{0.803000pt}%
\definecolor{currentstroke}{rgb}{0.690196,0.690196,0.690196}%
\pgfsetstrokecolor{currentstroke}%
\pgfsetdash{}{0pt}%
\pgfpathmoveto{\pgfqpoint{1.458420in}{3.531829in}}%
\pgfpathlineto{\pgfqpoint{1.480653in}{2.047303in}}%
\pgfpathlineto{\pgfqpoint{3.451074in}{1.467870in}}%
\pgfusepath{stroke}%
\end{pgfscope}%
\begin{pgfscope}%
\pgfsetrectcap%
\pgfsetroundjoin%
\pgfsetlinewidth{0.803000pt}%
\definecolor{currentstroke}{rgb}{0.000000,0.000000,0.000000}%
\pgfsetstrokecolor{currentstroke}%
\pgfsetdash{}{0pt}%
\pgfpathmoveto{\pgfqpoint{2.630365in}{0.648943in}}%
\pgfpathlineto{\pgfqpoint{2.682374in}{0.632291in}}%
\pgfusepath{stroke}%
\end{pgfscope}%
\begin{pgfscope}%
\definecolor{textcolor}{rgb}{0.000000,0.000000,0.000000}%
\pgfsetstrokecolor{textcolor}%
\pgfsetfillcolor{textcolor}%
\pgftext[x=2.825027in,y=0.458728in,,top]{\color{textcolor}\rmfamily\fontsize{10.000000}{12.000000}\selectfont \(\displaystyle {0}\)}%
\end{pgfscope}%
\begin{pgfscope}%
\pgfsetrectcap%
\pgfsetroundjoin%
\pgfsetlinewidth{0.803000pt}%
\definecolor{currentstroke}{rgb}{0.000000,0.000000,0.000000}%
\pgfsetstrokecolor{currentstroke}%
\pgfsetdash{}{0pt}%
\pgfpathmoveto{\pgfqpoint{2.841441in}{0.865171in}}%
\pgfpathlineto{\pgfqpoint{2.892849in}{0.849068in}}%
\pgfusepath{stroke}%
\end{pgfscope}%
\begin{pgfscope}%
\definecolor{textcolor}{rgb}{0.000000,0.000000,0.000000}%
\pgfsetstrokecolor{textcolor}%
\pgfsetfillcolor{textcolor}%
\pgftext[x=3.033073in,y=0.678341in,,top]{\color{textcolor}\rmfamily\fontsize{10.000000}{12.000000}\selectfont \(\displaystyle {1}\)}%
\end{pgfscope}%
\begin{pgfscope}%
\pgfsetrectcap%
\pgfsetroundjoin%
\pgfsetlinewidth{0.803000pt}%
\definecolor{currentstroke}{rgb}{0.000000,0.000000,0.000000}%
\pgfsetstrokecolor{currentstroke}%
\pgfsetdash{}{0pt}%
\pgfpathmoveto{\pgfqpoint{3.045608in}{1.074321in}}%
\pgfpathlineto{\pgfqpoint{3.096427in}{1.058742in}}%
\pgfusepath{stroke}%
\end{pgfscope}%
\begin{pgfscope}%
\definecolor{textcolor}{rgb}{0.000000,0.000000,0.000000}%
\pgfsetstrokecolor{textcolor}%
\pgfsetfillcolor{textcolor}%
\pgftext[x=3.234302in,y=0.890758in,,top]{\color{textcolor}\rmfamily\fontsize{10.000000}{12.000000}\selectfont \(\displaystyle {2}\)}%
\end{pgfscope}%
\begin{pgfscope}%
\pgfsetrectcap%
\pgfsetroundjoin%
\pgfsetlinewidth{0.803000pt}%
\definecolor{currentstroke}{rgb}{0.000000,0.000000,0.000000}%
\pgfsetstrokecolor{currentstroke}%
\pgfsetdash{}{0pt}%
\pgfpathmoveto{\pgfqpoint{3.243200in}{1.276736in}}%
\pgfpathlineto{\pgfqpoint{3.293441in}{1.261654in}}%
\pgfusepath{stroke}%
\end{pgfscope}%
\begin{pgfscope}%
\definecolor{textcolor}{rgb}{0.000000,0.000000,0.000000}%
\pgfsetstrokecolor{textcolor}%
\pgfsetfillcolor{textcolor}%
\pgftext[x=3.429045in,y=1.096328in,,top]{\color{textcolor}\rmfamily\fontsize{10.000000}{12.000000}\selectfont \(\displaystyle {3}\)}%
\end{pgfscope}%
\begin{pgfscope}%
\pgfsetrectcap%
\pgfsetroundjoin%
\pgfsetlinewidth{0.803000pt}%
\definecolor{currentstroke}{rgb}{0.000000,0.000000,0.000000}%
\pgfsetstrokecolor{currentstroke}%
\pgfsetdash{}{0pt}%
\pgfpathmoveto{\pgfqpoint{3.434530in}{1.472736in}}%
\pgfpathlineto{\pgfqpoint{3.484203in}{1.458129in}}%
\pgfusepath{stroke}%
\end{pgfscope}%
\begin{pgfscope}%
\definecolor{textcolor}{rgb}{0.000000,0.000000,0.000000}%
\pgfsetstrokecolor{textcolor}%
\pgfsetfillcolor{textcolor}%
\pgftext[x=3.617609in,y=1.295376in,,top]{\color{textcolor}\rmfamily\fontsize{10.000000}{12.000000}\selectfont \(\displaystyle {4}\)}%
\end{pgfscope}%
\begin{pgfscope}%
\pgfsetrectcap%
\pgfsetroundjoin%
\pgfsetlinewidth{0.803000pt}%
\definecolor{currentstroke}{rgb}{0.000000,0.000000,0.000000}%
\pgfsetstrokecolor{currentstroke}%
\pgfsetdash{}{0pt}%
\pgfpathmoveto{\pgfqpoint{3.558144in}{1.577751in}}%
\pgfpathlineto{\pgfqpoint{3.628038in}{3.104037in}}%
\pgfusepath{stroke}%
\end{pgfscope}%
\begin{pgfscope}%
\definecolor{textcolor}{rgb}{0.000000,0.000000,0.000000}%
\pgfsetstrokecolor{textcolor}%
\pgfsetfillcolor{textcolor}%
\pgftext[x=4.167903in, y=1.963517in, left, base,rotate=87.378092]{\color{textcolor}\rmfamily\fontsize{10.000000}{12.000000}\selectfont Position Z [\(\displaystyle m\)]}%
\end{pgfscope}%
\begin{pgfscope}%
\pgfsetbuttcap%
\pgfsetroundjoin%
\pgfsetlinewidth{0.803000pt}%
\definecolor{currentstroke}{rgb}{0.690196,0.690196,0.690196}%
\pgfsetstrokecolor{currentstroke}%
\pgfsetdash{}{0pt}%
\pgfpathmoveto{\pgfqpoint{3.562413in}{1.670968in}}%
\pgfpathlineto{\pgfqpoint{1.598575in}{2.237310in}}%
\pgfpathlineto{\pgfqpoint{0.374477in}{1.219382in}}%
\pgfusepath{stroke}%
\end{pgfscope}%
\begin{pgfscope}%
\pgfsetbuttcap%
\pgfsetroundjoin%
\pgfsetlinewidth{0.803000pt}%
\definecolor{currentstroke}{rgb}{0.690196,0.690196,0.690196}%
\pgfsetstrokecolor{currentstroke}%
\pgfsetdash{}{0pt}%
\pgfpathmoveto{\pgfqpoint{3.573719in}{1.917855in}}%
\pgfpathlineto{\pgfqpoint{1.595826in}{2.476338in}}%
\pgfpathlineto{\pgfqpoint{0.362299in}{1.472263in}}%
\pgfusepath{stroke}%
\end{pgfscope}%
\begin{pgfscope}%
\pgfsetbuttcap%
\pgfsetroundjoin%
\pgfsetlinewidth{0.803000pt}%
\definecolor{currentstroke}{rgb}{0.690196,0.690196,0.690196}%
\pgfsetstrokecolor{currentstroke}%
\pgfsetdash{}{0pt}%
\pgfpathmoveto{\pgfqpoint{3.585189in}{2.168337in}}%
\pgfpathlineto{\pgfqpoint{1.593040in}{2.718679in}}%
\pgfpathlineto{\pgfqpoint{0.349937in}{1.728966in}}%
\pgfusepath{stroke}%
\end{pgfscope}%
\begin{pgfscope}%
\pgfsetbuttcap%
\pgfsetroundjoin%
\pgfsetlinewidth{0.803000pt}%
\definecolor{currentstroke}{rgb}{0.690196,0.690196,0.690196}%
\pgfsetstrokecolor{currentstroke}%
\pgfsetdash{}{0pt}%
\pgfpathmoveto{\pgfqpoint{3.596828in}{2.422492in}}%
\pgfpathlineto{\pgfqpoint{1.590214in}{2.964402in}}%
\pgfpathlineto{\pgfqpoint{0.337387in}{1.989580in}}%
\pgfusepath{stroke}%
\end{pgfscope}%
\begin{pgfscope}%
\pgfsetbuttcap%
\pgfsetroundjoin%
\pgfsetlinewidth{0.803000pt}%
\definecolor{currentstroke}{rgb}{0.690196,0.690196,0.690196}%
\pgfsetstrokecolor{currentstroke}%
\pgfsetdash{}{0pt}%
\pgfpathmoveto{\pgfqpoint{3.608638in}{2.680402in}}%
\pgfpathlineto{\pgfqpoint{1.587349in}{3.213579in}}%
\pgfpathlineto{\pgfqpoint{0.324644in}{2.254193in}}%
\pgfusepath{stroke}%
\end{pgfscope}%
\begin{pgfscope}%
\pgfsetbuttcap%
\pgfsetroundjoin%
\pgfsetlinewidth{0.803000pt}%
\definecolor{currentstroke}{rgb}{0.690196,0.690196,0.690196}%
\pgfsetstrokecolor{currentstroke}%
\pgfsetdash{}{0pt}%
\pgfpathmoveto{\pgfqpoint{3.620624in}{2.942151in}}%
\pgfpathlineto{\pgfqpoint{1.584443in}{3.466283in}}%
\pgfpathlineto{\pgfqpoint{0.311704in}{2.522898in}}%
\pgfusepath{stroke}%
\end{pgfscope}%
\begin{pgfscope}%
\pgfsetrectcap%
\pgfsetroundjoin%
\pgfsetlinewidth{0.803000pt}%
\definecolor{currentstroke}{rgb}{0.000000,0.000000,0.000000}%
\pgfsetstrokecolor{currentstroke}%
\pgfsetdash{}{0pt}%
\pgfpathmoveto{\pgfqpoint{3.545929in}{1.675722in}}%
\pgfpathlineto{\pgfqpoint{3.595421in}{1.661449in}}%
\pgfusepath{stroke}%
\end{pgfscope}%
\begin{pgfscope}%
\definecolor{textcolor}{rgb}{0.000000,0.000000,0.000000}%
\pgfsetstrokecolor{textcolor}%
\pgfsetfillcolor{textcolor}%
\pgftext[x=3.816545in,y=1.706967in,,top]{\color{textcolor}\rmfamily\fontsize{10.000000}{12.000000}\selectfont \(\displaystyle {0.0}\)}%
\end{pgfscope}%
\begin{pgfscope}%
\pgfsetrectcap%
\pgfsetroundjoin%
\pgfsetlinewidth{0.803000pt}%
\definecolor{currentstroke}{rgb}{0.000000,0.000000,0.000000}%
\pgfsetstrokecolor{currentstroke}%
\pgfsetdash{}{0pt}%
\pgfpathmoveto{\pgfqpoint{3.557111in}{1.922545in}}%
\pgfpathlineto{\pgfqpoint{3.606974in}{1.908465in}}%
\pgfusepath{stroke}%
\end{pgfscope}%
\begin{pgfscope}%
\definecolor{textcolor}{rgb}{0.000000,0.000000,0.000000}%
\pgfsetstrokecolor{textcolor}%
\pgfsetfillcolor{textcolor}%
\pgftext[x=3.829647in,y=1.953366in,,top]{\color{textcolor}\rmfamily\fontsize{10.000000}{12.000000}\selectfont \(\displaystyle {0.1}\)}%
\end{pgfscope}%
\begin{pgfscope}%
\pgfsetrectcap%
\pgfsetroundjoin%
\pgfsetlinewidth{0.803000pt}%
\definecolor{currentstroke}{rgb}{0.000000,0.000000,0.000000}%
\pgfsetstrokecolor{currentstroke}%
\pgfsetdash{}{0pt}%
\pgfpathmoveto{\pgfqpoint{3.568456in}{2.172960in}}%
\pgfpathlineto{\pgfqpoint{3.618696in}{2.159081in}}%
\pgfusepath{stroke}%
\end{pgfscope}%
\begin{pgfscope}%
\definecolor{textcolor}{rgb}{0.000000,0.000000,0.000000}%
\pgfsetstrokecolor{textcolor}%
\pgfsetfillcolor{textcolor}%
\pgftext[x=3.842940in,y=2.203342in,,top]{\color{textcolor}\rmfamily\fontsize{10.000000}{12.000000}\selectfont \(\displaystyle {0.2}\)}%
\end{pgfscope}%
\begin{pgfscope}%
\pgfsetrectcap%
\pgfsetroundjoin%
\pgfsetlinewidth{0.803000pt}%
\definecolor{currentstroke}{rgb}{0.000000,0.000000,0.000000}%
\pgfsetstrokecolor{currentstroke}%
\pgfsetdash{}{0pt}%
\pgfpathmoveto{\pgfqpoint{3.579967in}{2.427046in}}%
\pgfpathlineto{\pgfqpoint{3.630590in}{2.413375in}}%
\pgfusepath{stroke}%
\end{pgfscope}%
\begin{pgfscope}%
\definecolor{textcolor}{rgb}{0.000000,0.000000,0.000000}%
\pgfsetstrokecolor{textcolor}%
\pgfsetfillcolor{textcolor}%
\pgftext[x=3.856427in,y=2.456972in,,top]{\color{textcolor}\rmfamily\fontsize{10.000000}{12.000000}\selectfont \(\displaystyle {0.3}\)}%
\end{pgfscope}%
\begin{pgfscope}%
\pgfsetrectcap%
\pgfsetroundjoin%
\pgfsetlinewidth{0.803000pt}%
\definecolor{currentstroke}{rgb}{0.000000,0.000000,0.000000}%
\pgfsetstrokecolor{currentstroke}%
\pgfsetdash{}{0pt}%
\pgfpathmoveto{\pgfqpoint{3.591648in}{2.684884in}}%
\pgfpathlineto{\pgfqpoint{3.642660in}{2.671428in}}%
\pgfusepath{stroke}%
\end{pgfscope}%
\begin{pgfscope}%
\definecolor{textcolor}{rgb}{0.000000,0.000000,0.000000}%
\pgfsetstrokecolor{textcolor}%
\pgfsetfillcolor{textcolor}%
\pgftext[x=3.870112in,y=2.714338in,,top]{\color{textcolor}\rmfamily\fontsize{10.000000}{12.000000}\selectfont \(\displaystyle {0.4}\)}%
\end{pgfscope}%
\begin{pgfscope}%
\pgfsetrectcap%
\pgfsetroundjoin%
\pgfsetlinewidth{0.803000pt}%
\definecolor{currentstroke}{rgb}{0.000000,0.000000,0.000000}%
\pgfsetstrokecolor{currentstroke}%
\pgfsetdash{}{0pt}%
\pgfpathmoveto{\pgfqpoint{3.603503in}{2.946558in}}%
\pgfpathlineto{\pgfqpoint{3.654909in}{2.933326in}}%
\pgfusepath{stroke}%
\end{pgfscope}%
\begin{pgfscope}%
\definecolor{textcolor}{rgb}{0.000000,0.000000,0.000000}%
\pgfsetstrokecolor{textcolor}%
\pgfsetfillcolor{textcolor}%
\pgftext[x=3.884001in,y=2.975522in,,top]{\color{textcolor}\rmfamily\fontsize{10.000000}{12.000000}\selectfont \(\displaystyle {0.5}\)}%
\end{pgfscope}%
\begin{pgfscope}%
\pgfpathrectangle{\pgfqpoint{0.100000in}{0.212622in}}{\pgfqpoint{3.696000in}{3.696000in}}%
\pgfusepath{clip}%
\pgfsetrectcap%
\pgfsetroundjoin%
\pgfsetlinewidth{1.505625pt}%
\definecolor{currentstroke}{rgb}{0.121569,0.466667,0.705882}%
\pgfsetstrokecolor{currentstroke}%
\pgfsetdash{}{0pt}%
\pgfpathmoveto{\pgfqpoint{0.916842in}{1.291711in}}%
\pgfpathlineto{\pgfqpoint{1.795822in}{2.045764in}}%
\pgfpathlineto{\pgfqpoint{3.150997in}{1.650360in}}%
\pgfpathlineto{\pgfqpoint{2.331339in}{0.842232in}}%
\pgfpathlineto{\pgfqpoint{0.916842in}{1.291711in}}%
\pgfusepath{stroke}%
\end{pgfscope}%
\begin{pgfscope}%
\pgfpathrectangle{\pgfqpoint{0.100000in}{0.212622in}}{\pgfqpoint{3.696000in}{3.696000in}}%
\pgfusepath{clip}%
\pgfsetrectcap%
\pgfsetroundjoin%
\pgfsetlinewidth{1.505625pt}%
\definecolor{currentstroke}{rgb}{1.000000,0.000000,0.000000}%
\pgfsetstrokecolor{currentstroke}%
\pgfsetdash{}{0pt}%
\pgfpathmoveto{\pgfqpoint{0.916842in}{1.291711in}}%
\pgfpathlineto{\pgfqpoint{0.916842in}{1.291711in}}%
\pgfusepath{stroke}%
\end{pgfscope}%
\begin{pgfscope}%
\pgfpathrectangle{\pgfqpoint{0.100000in}{0.212622in}}{\pgfqpoint{3.696000in}{3.696000in}}%
\pgfusepath{clip}%
\pgfsetrectcap%
\pgfsetroundjoin%
\pgfsetlinewidth{1.505625pt}%
\definecolor{currentstroke}{rgb}{1.000000,0.000000,0.000000}%
\pgfsetstrokecolor{currentstroke}%
\pgfsetdash{}{0pt}%
\pgfpathmoveto{\pgfqpoint{0.913953in}{1.291267in}}%
\pgfpathlineto{\pgfqpoint{0.916842in}{1.291711in}}%
\pgfusepath{stroke}%
\end{pgfscope}%
\begin{pgfscope}%
\pgfpathrectangle{\pgfqpoint{0.100000in}{0.212622in}}{\pgfqpoint{3.696000in}{3.696000in}}%
\pgfusepath{clip}%
\pgfsetrectcap%
\pgfsetroundjoin%
\pgfsetlinewidth{1.505625pt}%
\definecolor{currentstroke}{rgb}{1.000000,0.000000,0.000000}%
\pgfsetstrokecolor{currentstroke}%
\pgfsetdash{}{0pt}%
\pgfpathmoveto{\pgfqpoint{0.906395in}{1.289089in}}%
\pgfpathlineto{\pgfqpoint{0.916842in}{1.291711in}}%
\pgfusepath{stroke}%
\end{pgfscope}%
\begin{pgfscope}%
\pgfpathrectangle{\pgfqpoint{0.100000in}{0.212622in}}{\pgfqpoint{3.696000in}{3.696000in}}%
\pgfusepath{clip}%
\pgfsetrectcap%
\pgfsetroundjoin%
\pgfsetlinewidth{1.505625pt}%
\definecolor{currentstroke}{rgb}{1.000000,0.000000,0.000000}%
\pgfsetstrokecolor{currentstroke}%
\pgfsetdash{}{0pt}%
\pgfpathmoveto{\pgfqpoint{0.892987in}{1.285240in}}%
\pgfpathlineto{\pgfqpoint{0.916842in}{1.291711in}}%
\pgfusepath{stroke}%
\end{pgfscope}%
\begin{pgfscope}%
\pgfpathrectangle{\pgfqpoint{0.100000in}{0.212622in}}{\pgfqpoint{3.696000in}{3.696000in}}%
\pgfusepath{clip}%
\pgfsetrectcap%
\pgfsetroundjoin%
\pgfsetlinewidth{1.505625pt}%
\definecolor{currentstroke}{rgb}{1.000000,0.000000,0.000000}%
\pgfsetstrokecolor{currentstroke}%
\pgfsetdash{}{0pt}%
\pgfpathmoveto{\pgfqpoint{0.875380in}{1.280545in}}%
\pgfpathlineto{\pgfqpoint{0.916842in}{1.291711in}}%
\pgfusepath{stroke}%
\end{pgfscope}%
\begin{pgfscope}%
\pgfpathrectangle{\pgfqpoint{0.100000in}{0.212622in}}{\pgfqpoint{3.696000in}{3.696000in}}%
\pgfusepath{clip}%
\pgfsetrectcap%
\pgfsetroundjoin%
\pgfsetlinewidth{1.505625pt}%
\definecolor{currentstroke}{rgb}{1.000000,0.000000,0.000000}%
\pgfsetstrokecolor{currentstroke}%
\pgfsetdash{}{0pt}%
\pgfpathmoveto{\pgfqpoint{0.851587in}{1.275014in}}%
\pgfpathlineto{\pgfqpoint{0.916842in}{1.291711in}}%
\pgfusepath{stroke}%
\end{pgfscope}%
\begin{pgfscope}%
\pgfpathrectangle{\pgfqpoint{0.100000in}{0.212622in}}{\pgfqpoint{3.696000in}{3.696000in}}%
\pgfusepath{clip}%
\pgfsetrectcap%
\pgfsetroundjoin%
\pgfsetlinewidth{1.505625pt}%
\definecolor{currentstroke}{rgb}{1.000000,0.000000,0.000000}%
\pgfsetstrokecolor{currentstroke}%
\pgfsetdash{}{0pt}%
\pgfpathmoveto{\pgfqpoint{0.824958in}{1.269252in}}%
\pgfpathlineto{\pgfqpoint{0.916842in}{1.291711in}}%
\pgfusepath{stroke}%
\end{pgfscope}%
\begin{pgfscope}%
\pgfpathrectangle{\pgfqpoint{0.100000in}{0.212622in}}{\pgfqpoint{3.696000in}{3.696000in}}%
\pgfusepath{clip}%
\pgfsetrectcap%
\pgfsetroundjoin%
\pgfsetlinewidth{1.505625pt}%
\definecolor{currentstroke}{rgb}{1.000000,0.000000,0.000000}%
\pgfsetstrokecolor{currentstroke}%
\pgfsetdash{}{0pt}%
\pgfpathmoveto{\pgfqpoint{0.796032in}{1.261512in}}%
\pgfpathlineto{\pgfqpoint{0.916842in}{1.291711in}}%
\pgfusepath{stroke}%
\end{pgfscope}%
\begin{pgfscope}%
\pgfpathrectangle{\pgfqpoint{0.100000in}{0.212622in}}{\pgfqpoint{3.696000in}{3.696000in}}%
\pgfusepath{clip}%
\pgfsetrectcap%
\pgfsetroundjoin%
\pgfsetlinewidth{1.505625pt}%
\definecolor{currentstroke}{rgb}{1.000000,0.000000,0.000000}%
\pgfsetstrokecolor{currentstroke}%
\pgfsetdash{}{0pt}%
\pgfpathmoveto{\pgfqpoint{0.764760in}{1.253804in}}%
\pgfpathlineto{\pgfqpoint{0.916842in}{1.291711in}}%
\pgfusepath{stroke}%
\end{pgfscope}%
\begin{pgfscope}%
\pgfpathrectangle{\pgfqpoint{0.100000in}{0.212622in}}{\pgfqpoint{3.696000in}{3.696000in}}%
\pgfusepath{clip}%
\pgfsetrectcap%
\pgfsetroundjoin%
\pgfsetlinewidth{1.505625pt}%
\definecolor{currentstroke}{rgb}{1.000000,0.000000,0.000000}%
\pgfsetstrokecolor{currentstroke}%
\pgfsetdash{}{0pt}%
\pgfpathmoveto{\pgfqpoint{0.747797in}{1.249252in}}%
\pgfpathlineto{\pgfqpoint{0.916842in}{1.291711in}}%
\pgfusepath{stroke}%
\end{pgfscope}%
\begin{pgfscope}%
\pgfpathrectangle{\pgfqpoint{0.100000in}{0.212622in}}{\pgfqpoint{3.696000in}{3.696000in}}%
\pgfusepath{clip}%
\pgfsetrectcap%
\pgfsetroundjoin%
\pgfsetlinewidth{1.505625pt}%
\definecolor{currentstroke}{rgb}{1.000000,0.000000,0.000000}%
\pgfsetstrokecolor{currentstroke}%
\pgfsetdash{}{0pt}%
\pgfpathmoveto{\pgfqpoint{0.738388in}{1.246685in}}%
\pgfpathlineto{\pgfqpoint{0.916842in}{1.291711in}}%
\pgfusepath{stroke}%
\end{pgfscope}%
\begin{pgfscope}%
\pgfpathrectangle{\pgfqpoint{0.100000in}{0.212622in}}{\pgfqpoint{3.696000in}{3.696000in}}%
\pgfusepath{clip}%
\pgfsetrectcap%
\pgfsetroundjoin%
\pgfsetlinewidth{1.505625pt}%
\definecolor{currentstroke}{rgb}{1.000000,0.000000,0.000000}%
\pgfsetstrokecolor{currentstroke}%
\pgfsetdash{}{0pt}%
\pgfpathmoveto{\pgfqpoint{0.733294in}{1.245314in}}%
\pgfpathlineto{\pgfqpoint{0.916842in}{1.291711in}}%
\pgfusepath{stroke}%
\end{pgfscope}%
\begin{pgfscope}%
\pgfpathrectangle{\pgfqpoint{0.100000in}{0.212622in}}{\pgfqpoint{3.696000in}{3.696000in}}%
\pgfusepath{clip}%
\pgfsetrectcap%
\pgfsetroundjoin%
\pgfsetlinewidth{1.505625pt}%
\definecolor{currentstroke}{rgb}{1.000000,0.000000,0.000000}%
\pgfsetstrokecolor{currentstroke}%
\pgfsetdash{}{0pt}%
\pgfpathmoveto{\pgfqpoint{0.730443in}{1.244530in}}%
\pgfpathlineto{\pgfqpoint{0.916842in}{1.291711in}}%
\pgfusepath{stroke}%
\end{pgfscope}%
\begin{pgfscope}%
\pgfpathrectangle{\pgfqpoint{0.100000in}{0.212622in}}{\pgfqpoint{3.696000in}{3.696000in}}%
\pgfusepath{clip}%
\pgfsetrectcap%
\pgfsetroundjoin%
\pgfsetlinewidth{1.505625pt}%
\definecolor{currentstroke}{rgb}{1.000000,0.000000,0.000000}%
\pgfsetstrokecolor{currentstroke}%
\pgfsetdash{}{0pt}%
\pgfpathmoveto{\pgfqpoint{0.728826in}{1.244140in}}%
\pgfpathlineto{\pgfqpoint{0.916842in}{1.291711in}}%
\pgfusepath{stroke}%
\end{pgfscope}%
\begin{pgfscope}%
\pgfpathrectangle{\pgfqpoint{0.100000in}{0.212622in}}{\pgfqpoint{3.696000in}{3.696000in}}%
\pgfusepath{clip}%
\pgfsetrectcap%
\pgfsetroundjoin%
\pgfsetlinewidth{1.505625pt}%
\definecolor{currentstroke}{rgb}{1.000000,0.000000,0.000000}%
\pgfsetstrokecolor{currentstroke}%
\pgfsetdash{}{0pt}%
\pgfpathmoveto{\pgfqpoint{0.727967in}{1.243923in}}%
\pgfpathlineto{\pgfqpoint{0.916842in}{1.291711in}}%
\pgfusepath{stroke}%
\end{pgfscope}%
\begin{pgfscope}%
\pgfpathrectangle{\pgfqpoint{0.100000in}{0.212622in}}{\pgfqpoint{3.696000in}{3.696000in}}%
\pgfusepath{clip}%
\pgfsetrectcap%
\pgfsetroundjoin%
\pgfsetlinewidth{1.505625pt}%
\definecolor{currentstroke}{rgb}{1.000000,0.000000,0.000000}%
\pgfsetstrokecolor{currentstroke}%
\pgfsetdash{}{0pt}%
\pgfpathmoveto{\pgfqpoint{0.727473in}{1.243823in}}%
\pgfpathlineto{\pgfqpoint{0.916842in}{1.291711in}}%
\pgfusepath{stroke}%
\end{pgfscope}%
\begin{pgfscope}%
\pgfpathrectangle{\pgfqpoint{0.100000in}{0.212622in}}{\pgfqpoint{3.696000in}{3.696000in}}%
\pgfusepath{clip}%
\pgfsetrectcap%
\pgfsetroundjoin%
\pgfsetlinewidth{1.505625pt}%
\definecolor{currentstroke}{rgb}{1.000000,0.000000,0.000000}%
\pgfsetstrokecolor{currentstroke}%
\pgfsetdash{}{0pt}%
\pgfpathmoveto{\pgfqpoint{0.727214in}{1.243780in}}%
\pgfpathlineto{\pgfqpoint{0.916842in}{1.291711in}}%
\pgfusepath{stroke}%
\end{pgfscope}%
\begin{pgfscope}%
\pgfpathrectangle{\pgfqpoint{0.100000in}{0.212622in}}{\pgfqpoint{3.696000in}{3.696000in}}%
\pgfusepath{clip}%
\pgfsetrectcap%
\pgfsetroundjoin%
\pgfsetlinewidth{1.505625pt}%
\definecolor{currentstroke}{rgb}{1.000000,0.000000,0.000000}%
\pgfsetstrokecolor{currentstroke}%
\pgfsetdash{}{0pt}%
\pgfpathmoveto{\pgfqpoint{0.727075in}{1.243760in}}%
\pgfpathlineto{\pgfqpoint{0.916842in}{1.291711in}}%
\pgfusepath{stroke}%
\end{pgfscope}%
\begin{pgfscope}%
\pgfpathrectangle{\pgfqpoint{0.100000in}{0.212622in}}{\pgfqpoint{3.696000in}{3.696000in}}%
\pgfusepath{clip}%
\pgfsetrectcap%
\pgfsetroundjoin%
\pgfsetlinewidth{1.505625pt}%
\definecolor{currentstroke}{rgb}{1.000000,0.000000,0.000000}%
\pgfsetstrokecolor{currentstroke}%
\pgfsetdash{}{0pt}%
\pgfpathmoveto{\pgfqpoint{0.726998in}{1.243749in}}%
\pgfpathlineto{\pgfqpoint{0.916842in}{1.291711in}}%
\pgfusepath{stroke}%
\end{pgfscope}%
\begin{pgfscope}%
\pgfpathrectangle{\pgfqpoint{0.100000in}{0.212622in}}{\pgfqpoint{3.696000in}{3.696000in}}%
\pgfusepath{clip}%
\pgfsetrectcap%
\pgfsetroundjoin%
\pgfsetlinewidth{1.505625pt}%
\definecolor{currentstroke}{rgb}{1.000000,0.000000,0.000000}%
\pgfsetstrokecolor{currentstroke}%
\pgfsetdash{}{0pt}%
\pgfpathmoveto{\pgfqpoint{0.726955in}{1.243743in}}%
\pgfpathlineto{\pgfqpoint{0.916842in}{1.291711in}}%
\pgfusepath{stroke}%
\end{pgfscope}%
\begin{pgfscope}%
\pgfpathrectangle{\pgfqpoint{0.100000in}{0.212622in}}{\pgfqpoint{3.696000in}{3.696000in}}%
\pgfusepath{clip}%
\pgfsetrectcap%
\pgfsetroundjoin%
\pgfsetlinewidth{1.505625pt}%
\definecolor{currentstroke}{rgb}{1.000000,0.000000,0.000000}%
\pgfsetstrokecolor{currentstroke}%
\pgfsetdash{}{0pt}%
\pgfpathmoveto{\pgfqpoint{0.726932in}{1.243739in}}%
\pgfpathlineto{\pgfqpoint{0.916842in}{1.291711in}}%
\pgfusepath{stroke}%
\end{pgfscope}%
\begin{pgfscope}%
\pgfpathrectangle{\pgfqpoint{0.100000in}{0.212622in}}{\pgfqpoint{3.696000in}{3.696000in}}%
\pgfusepath{clip}%
\pgfsetrectcap%
\pgfsetroundjoin%
\pgfsetlinewidth{1.505625pt}%
\definecolor{currentstroke}{rgb}{1.000000,0.000000,0.000000}%
\pgfsetstrokecolor{currentstroke}%
\pgfsetdash{}{0pt}%
\pgfpathmoveto{\pgfqpoint{0.726919in}{1.243737in}}%
\pgfpathlineto{\pgfqpoint{0.916842in}{1.291711in}}%
\pgfusepath{stroke}%
\end{pgfscope}%
\begin{pgfscope}%
\pgfpathrectangle{\pgfqpoint{0.100000in}{0.212622in}}{\pgfqpoint{3.696000in}{3.696000in}}%
\pgfusepath{clip}%
\pgfsetrectcap%
\pgfsetroundjoin%
\pgfsetlinewidth{1.505625pt}%
\definecolor{currentstroke}{rgb}{1.000000,0.000000,0.000000}%
\pgfsetstrokecolor{currentstroke}%
\pgfsetdash{}{0pt}%
\pgfpathmoveto{\pgfqpoint{0.726912in}{1.243737in}}%
\pgfpathlineto{\pgfqpoint{0.916842in}{1.291711in}}%
\pgfusepath{stroke}%
\end{pgfscope}%
\begin{pgfscope}%
\pgfpathrectangle{\pgfqpoint{0.100000in}{0.212622in}}{\pgfqpoint{3.696000in}{3.696000in}}%
\pgfusepath{clip}%
\pgfsetrectcap%
\pgfsetroundjoin%
\pgfsetlinewidth{1.505625pt}%
\definecolor{currentstroke}{rgb}{1.000000,0.000000,0.000000}%
\pgfsetstrokecolor{currentstroke}%
\pgfsetdash{}{0pt}%
\pgfpathmoveto{\pgfqpoint{0.726908in}{1.243736in}}%
\pgfpathlineto{\pgfqpoint{0.916842in}{1.291711in}}%
\pgfusepath{stroke}%
\end{pgfscope}%
\begin{pgfscope}%
\pgfpathrectangle{\pgfqpoint{0.100000in}{0.212622in}}{\pgfqpoint{3.696000in}{3.696000in}}%
\pgfusepath{clip}%
\pgfsetrectcap%
\pgfsetroundjoin%
\pgfsetlinewidth{1.505625pt}%
\definecolor{currentstroke}{rgb}{1.000000,0.000000,0.000000}%
\pgfsetstrokecolor{currentstroke}%
\pgfsetdash{}{0pt}%
\pgfpathmoveto{\pgfqpoint{0.726906in}{1.243736in}}%
\pgfpathlineto{\pgfqpoint{0.916842in}{1.291711in}}%
\pgfusepath{stroke}%
\end{pgfscope}%
\begin{pgfscope}%
\pgfpathrectangle{\pgfqpoint{0.100000in}{0.212622in}}{\pgfqpoint{3.696000in}{3.696000in}}%
\pgfusepath{clip}%
\pgfsetrectcap%
\pgfsetroundjoin%
\pgfsetlinewidth{1.505625pt}%
\definecolor{currentstroke}{rgb}{1.000000,0.000000,0.000000}%
\pgfsetstrokecolor{currentstroke}%
\pgfsetdash{}{0pt}%
\pgfpathmoveto{\pgfqpoint{0.726905in}{1.243736in}}%
\pgfpathlineto{\pgfqpoint{0.916842in}{1.291711in}}%
\pgfusepath{stroke}%
\end{pgfscope}%
\begin{pgfscope}%
\pgfpathrectangle{\pgfqpoint{0.100000in}{0.212622in}}{\pgfqpoint{3.696000in}{3.696000in}}%
\pgfusepath{clip}%
\pgfsetrectcap%
\pgfsetroundjoin%
\pgfsetlinewidth{1.505625pt}%
\definecolor{currentstroke}{rgb}{1.000000,0.000000,0.000000}%
\pgfsetstrokecolor{currentstroke}%
\pgfsetdash{}{0pt}%
\pgfpathmoveto{\pgfqpoint{0.726904in}{1.243736in}}%
\pgfpathlineto{\pgfqpoint{0.916842in}{1.291711in}}%
\pgfusepath{stroke}%
\end{pgfscope}%
\begin{pgfscope}%
\pgfpathrectangle{\pgfqpoint{0.100000in}{0.212622in}}{\pgfqpoint{3.696000in}{3.696000in}}%
\pgfusepath{clip}%
\pgfsetrectcap%
\pgfsetroundjoin%
\pgfsetlinewidth{1.505625pt}%
\definecolor{currentstroke}{rgb}{1.000000,0.000000,0.000000}%
\pgfsetstrokecolor{currentstroke}%
\pgfsetdash{}{0pt}%
\pgfpathmoveto{\pgfqpoint{0.726904in}{1.243736in}}%
\pgfpathlineto{\pgfqpoint{0.916842in}{1.291711in}}%
\pgfusepath{stroke}%
\end{pgfscope}%
\begin{pgfscope}%
\pgfpathrectangle{\pgfqpoint{0.100000in}{0.212622in}}{\pgfqpoint{3.696000in}{3.696000in}}%
\pgfusepath{clip}%
\pgfsetrectcap%
\pgfsetroundjoin%
\pgfsetlinewidth{1.505625pt}%
\definecolor{currentstroke}{rgb}{1.000000,0.000000,0.000000}%
\pgfsetstrokecolor{currentstroke}%
\pgfsetdash{}{0pt}%
\pgfpathmoveto{\pgfqpoint{0.726904in}{1.243736in}}%
\pgfpathlineto{\pgfqpoint{0.916842in}{1.291711in}}%
\pgfusepath{stroke}%
\end{pgfscope}%
\begin{pgfscope}%
\pgfpathrectangle{\pgfqpoint{0.100000in}{0.212622in}}{\pgfqpoint{3.696000in}{3.696000in}}%
\pgfusepath{clip}%
\pgfsetrectcap%
\pgfsetroundjoin%
\pgfsetlinewidth{1.505625pt}%
\definecolor{currentstroke}{rgb}{1.000000,0.000000,0.000000}%
\pgfsetstrokecolor{currentstroke}%
\pgfsetdash{}{0pt}%
\pgfpathmoveto{\pgfqpoint{0.726904in}{1.243736in}}%
\pgfpathlineto{\pgfqpoint{0.916842in}{1.291711in}}%
\pgfusepath{stroke}%
\end{pgfscope}%
\begin{pgfscope}%
\pgfpathrectangle{\pgfqpoint{0.100000in}{0.212622in}}{\pgfqpoint{3.696000in}{3.696000in}}%
\pgfusepath{clip}%
\pgfsetrectcap%
\pgfsetroundjoin%
\pgfsetlinewidth{1.505625pt}%
\definecolor{currentstroke}{rgb}{1.000000,0.000000,0.000000}%
\pgfsetstrokecolor{currentstroke}%
\pgfsetdash{}{0pt}%
\pgfpathmoveto{\pgfqpoint{0.726904in}{1.243736in}}%
\pgfpathlineto{\pgfqpoint{0.916842in}{1.291711in}}%
\pgfusepath{stroke}%
\end{pgfscope}%
\begin{pgfscope}%
\pgfpathrectangle{\pgfqpoint{0.100000in}{0.212622in}}{\pgfqpoint{3.696000in}{3.696000in}}%
\pgfusepath{clip}%
\pgfsetrectcap%
\pgfsetroundjoin%
\pgfsetlinewidth{1.505625pt}%
\definecolor{currentstroke}{rgb}{1.000000,0.000000,0.000000}%
\pgfsetstrokecolor{currentstroke}%
\pgfsetdash{}{0pt}%
\pgfpathmoveto{\pgfqpoint{0.726903in}{1.243736in}}%
\pgfpathlineto{\pgfqpoint{0.916842in}{1.291711in}}%
\pgfusepath{stroke}%
\end{pgfscope}%
\begin{pgfscope}%
\pgfpathrectangle{\pgfqpoint{0.100000in}{0.212622in}}{\pgfqpoint{3.696000in}{3.696000in}}%
\pgfusepath{clip}%
\pgfsetrectcap%
\pgfsetroundjoin%
\pgfsetlinewidth{1.505625pt}%
\definecolor{currentstroke}{rgb}{1.000000,0.000000,0.000000}%
\pgfsetstrokecolor{currentstroke}%
\pgfsetdash{}{0pt}%
\pgfpathmoveto{\pgfqpoint{0.726903in}{1.243736in}}%
\pgfpathlineto{\pgfqpoint{0.916842in}{1.291711in}}%
\pgfusepath{stroke}%
\end{pgfscope}%
\begin{pgfscope}%
\pgfpathrectangle{\pgfqpoint{0.100000in}{0.212622in}}{\pgfqpoint{3.696000in}{3.696000in}}%
\pgfusepath{clip}%
\pgfsetrectcap%
\pgfsetroundjoin%
\pgfsetlinewidth{1.505625pt}%
\definecolor{currentstroke}{rgb}{1.000000,0.000000,0.000000}%
\pgfsetstrokecolor{currentstroke}%
\pgfsetdash{}{0pt}%
\pgfpathmoveto{\pgfqpoint{0.726903in}{1.243736in}}%
\pgfpathlineto{\pgfqpoint{0.916842in}{1.291711in}}%
\pgfusepath{stroke}%
\end{pgfscope}%
\begin{pgfscope}%
\pgfpathrectangle{\pgfqpoint{0.100000in}{0.212622in}}{\pgfqpoint{3.696000in}{3.696000in}}%
\pgfusepath{clip}%
\pgfsetrectcap%
\pgfsetroundjoin%
\pgfsetlinewidth{1.505625pt}%
\definecolor{currentstroke}{rgb}{1.000000,0.000000,0.000000}%
\pgfsetstrokecolor{currentstroke}%
\pgfsetdash{}{0pt}%
\pgfpathmoveto{\pgfqpoint{0.726903in}{1.243736in}}%
\pgfpathlineto{\pgfqpoint{0.916842in}{1.291711in}}%
\pgfusepath{stroke}%
\end{pgfscope}%
\begin{pgfscope}%
\pgfpathrectangle{\pgfqpoint{0.100000in}{0.212622in}}{\pgfqpoint{3.696000in}{3.696000in}}%
\pgfusepath{clip}%
\pgfsetrectcap%
\pgfsetroundjoin%
\pgfsetlinewidth{1.505625pt}%
\definecolor{currentstroke}{rgb}{1.000000,0.000000,0.000000}%
\pgfsetstrokecolor{currentstroke}%
\pgfsetdash{}{0pt}%
\pgfpathmoveto{\pgfqpoint{0.726903in}{1.243736in}}%
\pgfpathlineto{\pgfqpoint{0.916842in}{1.291711in}}%
\pgfusepath{stroke}%
\end{pgfscope}%
\begin{pgfscope}%
\pgfpathrectangle{\pgfqpoint{0.100000in}{0.212622in}}{\pgfqpoint{3.696000in}{3.696000in}}%
\pgfusepath{clip}%
\pgfsetrectcap%
\pgfsetroundjoin%
\pgfsetlinewidth{1.505625pt}%
\definecolor{currentstroke}{rgb}{1.000000,0.000000,0.000000}%
\pgfsetstrokecolor{currentstroke}%
\pgfsetdash{}{0pt}%
\pgfpathmoveto{\pgfqpoint{0.726903in}{1.243736in}}%
\pgfpathlineto{\pgfqpoint{0.916842in}{1.291711in}}%
\pgfusepath{stroke}%
\end{pgfscope}%
\begin{pgfscope}%
\pgfpathrectangle{\pgfqpoint{0.100000in}{0.212622in}}{\pgfqpoint{3.696000in}{3.696000in}}%
\pgfusepath{clip}%
\pgfsetrectcap%
\pgfsetroundjoin%
\pgfsetlinewidth{1.505625pt}%
\definecolor{currentstroke}{rgb}{1.000000,0.000000,0.000000}%
\pgfsetstrokecolor{currentstroke}%
\pgfsetdash{}{0pt}%
\pgfpathmoveto{\pgfqpoint{0.726903in}{1.243736in}}%
\pgfpathlineto{\pgfqpoint{0.916842in}{1.291711in}}%
\pgfusepath{stroke}%
\end{pgfscope}%
\begin{pgfscope}%
\pgfpathrectangle{\pgfqpoint{0.100000in}{0.212622in}}{\pgfqpoint{3.696000in}{3.696000in}}%
\pgfusepath{clip}%
\pgfsetrectcap%
\pgfsetroundjoin%
\pgfsetlinewidth{1.505625pt}%
\definecolor{currentstroke}{rgb}{1.000000,0.000000,0.000000}%
\pgfsetstrokecolor{currentstroke}%
\pgfsetdash{}{0pt}%
\pgfpathmoveto{\pgfqpoint{0.726903in}{1.243736in}}%
\pgfpathlineto{\pgfqpoint{0.916842in}{1.291711in}}%
\pgfusepath{stroke}%
\end{pgfscope}%
\begin{pgfscope}%
\pgfpathrectangle{\pgfqpoint{0.100000in}{0.212622in}}{\pgfqpoint{3.696000in}{3.696000in}}%
\pgfusepath{clip}%
\pgfsetrectcap%
\pgfsetroundjoin%
\pgfsetlinewidth{1.505625pt}%
\definecolor{currentstroke}{rgb}{1.000000,0.000000,0.000000}%
\pgfsetstrokecolor{currentstroke}%
\pgfsetdash{}{0pt}%
\pgfpathmoveto{\pgfqpoint{0.726903in}{1.243736in}}%
\pgfpathlineto{\pgfqpoint{0.916842in}{1.291711in}}%
\pgfusepath{stroke}%
\end{pgfscope}%
\begin{pgfscope}%
\pgfpathrectangle{\pgfqpoint{0.100000in}{0.212622in}}{\pgfqpoint{3.696000in}{3.696000in}}%
\pgfusepath{clip}%
\pgfsetrectcap%
\pgfsetroundjoin%
\pgfsetlinewidth{1.505625pt}%
\definecolor{currentstroke}{rgb}{1.000000,0.000000,0.000000}%
\pgfsetstrokecolor{currentstroke}%
\pgfsetdash{}{0pt}%
\pgfpathmoveto{\pgfqpoint{0.726903in}{1.243736in}}%
\pgfpathlineto{\pgfqpoint{0.916842in}{1.291711in}}%
\pgfusepath{stroke}%
\end{pgfscope}%
\begin{pgfscope}%
\pgfpathrectangle{\pgfqpoint{0.100000in}{0.212622in}}{\pgfqpoint{3.696000in}{3.696000in}}%
\pgfusepath{clip}%
\pgfsetrectcap%
\pgfsetroundjoin%
\pgfsetlinewidth{1.505625pt}%
\definecolor{currentstroke}{rgb}{1.000000,0.000000,0.000000}%
\pgfsetstrokecolor{currentstroke}%
\pgfsetdash{}{0pt}%
\pgfpathmoveto{\pgfqpoint{0.726903in}{1.243736in}}%
\pgfpathlineto{\pgfqpoint{0.916842in}{1.291711in}}%
\pgfusepath{stroke}%
\end{pgfscope}%
\begin{pgfscope}%
\pgfpathrectangle{\pgfqpoint{0.100000in}{0.212622in}}{\pgfqpoint{3.696000in}{3.696000in}}%
\pgfusepath{clip}%
\pgfsetrectcap%
\pgfsetroundjoin%
\pgfsetlinewidth{1.505625pt}%
\definecolor{currentstroke}{rgb}{1.000000,0.000000,0.000000}%
\pgfsetstrokecolor{currentstroke}%
\pgfsetdash{}{0pt}%
\pgfpathmoveto{\pgfqpoint{0.726903in}{1.243736in}}%
\pgfpathlineto{\pgfqpoint{0.916842in}{1.291711in}}%
\pgfusepath{stroke}%
\end{pgfscope}%
\begin{pgfscope}%
\pgfpathrectangle{\pgfqpoint{0.100000in}{0.212622in}}{\pgfqpoint{3.696000in}{3.696000in}}%
\pgfusepath{clip}%
\pgfsetrectcap%
\pgfsetroundjoin%
\pgfsetlinewidth{1.505625pt}%
\definecolor{currentstroke}{rgb}{1.000000,0.000000,0.000000}%
\pgfsetstrokecolor{currentstroke}%
\pgfsetdash{}{0pt}%
\pgfpathmoveto{\pgfqpoint{0.726903in}{1.243736in}}%
\pgfpathlineto{\pgfqpoint{0.916842in}{1.291711in}}%
\pgfusepath{stroke}%
\end{pgfscope}%
\begin{pgfscope}%
\pgfpathrectangle{\pgfqpoint{0.100000in}{0.212622in}}{\pgfqpoint{3.696000in}{3.696000in}}%
\pgfusepath{clip}%
\pgfsetrectcap%
\pgfsetroundjoin%
\pgfsetlinewidth{1.505625pt}%
\definecolor{currentstroke}{rgb}{1.000000,0.000000,0.000000}%
\pgfsetstrokecolor{currentstroke}%
\pgfsetdash{}{0pt}%
\pgfpathmoveto{\pgfqpoint{0.726903in}{1.243736in}}%
\pgfpathlineto{\pgfqpoint{0.916842in}{1.291711in}}%
\pgfusepath{stroke}%
\end{pgfscope}%
\begin{pgfscope}%
\pgfpathrectangle{\pgfqpoint{0.100000in}{0.212622in}}{\pgfqpoint{3.696000in}{3.696000in}}%
\pgfusepath{clip}%
\pgfsetrectcap%
\pgfsetroundjoin%
\pgfsetlinewidth{1.505625pt}%
\definecolor{currentstroke}{rgb}{1.000000,0.000000,0.000000}%
\pgfsetstrokecolor{currentstroke}%
\pgfsetdash{}{0pt}%
\pgfpathmoveto{\pgfqpoint{0.726903in}{1.243736in}}%
\pgfpathlineto{\pgfqpoint{0.916842in}{1.291711in}}%
\pgfusepath{stroke}%
\end{pgfscope}%
\begin{pgfscope}%
\pgfpathrectangle{\pgfqpoint{0.100000in}{0.212622in}}{\pgfqpoint{3.696000in}{3.696000in}}%
\pgfusepath{clip}%
\pgfsetrectcap%
\pgfsetroundjoin%
\pgfsetlinewidth{1.505625pt}%
\definecolor{currentstroke}{rgb}{1.000000,0.000000,0.000000}%
\pgfsetstrokecolor{currentstroke}%
\pgfsetdash{}{0pt}%
\pgfpathmoveto{\pgfqpoint{0.726903in}{1.243736in}}%
\pgfpathlineto{\pgfqpoint{0.916842in}{1.291711in}}%
\pgfusepath{stroke}%
\end{pgfscope}%
\begin{pgfscope}%
\pgfpathrectangle{\pgfqpoint{0.100000in}{0.212622in}}{\pgfqpoint{3.696000in}{3.696000in}}%
\pgfusepath{clip}%
\pgfsetrectcap%
\pgfsetroundjoin%
\pgfsetlinewidth{1.505625pt}%
\definecolor{currentstroke}{rgb}{1.000000,0.000000,0.000000}%
\pgfsetstrokecolor{currentstroke}%
\pgfsetdash{}{0pt}%
\pgfpathmoveto{\pgfqpoint{0.726903in}{1.243736in}}%
\pgfpathlineto{\pgfqpoint{0.916842in}{1.291711in}}%
\pgfusepath{stroke}%
\end{pgfscope}%
\begin{pgfscope}%
\pgfpathrectangle{\pgfqpoint{0.100000in}{0.212622in}}{\pgfqpoint{3.696000in}{3.696000in}}%
\pgfusepath{clip}%
\pgfsetrectcap%
\pgfsetroundjoin%
\pgfsetlinewidth{1.505625pt}%
\definecolor{currentstroke}{rgb}{1.000000,0.000000,0.000000}%
\pgfsetstrokecolor{currentstroke}%
\pgfsetdash{}{0pt}%
\pgfpathmoveto{\pgfqpoint{0.726903in}{1.243736in}}%
\pgfpathlineto{\pgfqpoint{0.916842in}{1.291711in}}%
\pgfusepath{stroke}%
\end{pgfscope}%
\begin{pgfscope}%
\pgfpathrectangle{\pgfqpoint{0.100000in}{0.212622in}}{\pgfqpoint{3.696000in}{3.696000in}}%
\pgfusepath{clip}%
\pgfsetrectcap%
\pgfsetroundjoin%
\pgfsetlinewidth{1.505625pt}%
\definecolor{currentstroke}{rgb}{1.000000,0.000000,0.000000}%
\pgfsetstrokecolor{currentstroke}%
\pgfsetdash{}{0pt}%
\pgfpathmoveto{\pgfqpoint{0.726903in}{1.243736in}}%
\pgfpathlineto{\pgfqpoint{0.916842in}{1.291711in}}%
\pgfusepath{stroke}%
\end{pgfscope}%
\begin{pgfscope}%
\pgfpathrectangle{\pgfqpoint{0.100000in}{0.212622in}}{\pgfqpoint{3.696000in}{3.696000in}}%
\pgfusepath{clip}%
\pgfsetrectcap%
\pgfsetroundjoin%
\pgfsetlinewidth{1.505625pt}%
\definecolor{currentstroke}{rgb}{1.000000,0.000000,0.000000}%
\pgfsetstrokecolor{currentstroke}%
\pgfsetdash{}{0pt}%
\pgfpathmoveto{\pgfqpoint{0.726903in}{1.243736in}}%
\pgfpathlineto{\pgfqpoint{0.916842in}{1.291711in}}%
\pgfusepath{stroke}%
\end{pgfscope}%
\begin{pgfscope}%
\pgfpathrectangle{\pgfqpoint{0.100000in}{0.212622in}}{\pgfqpoint{3.696000in}{3.696000in}}%
\pgfusepath{clip}%
\pgfsetrectcap%
\pgfsetroundjoin%
\pgfsetlinewidth{1.505625pt}%
\definecolor{currentstroke}{rgb}{1.000000,0.000000,0.000000}%
\pgfsetstrokecolor{currentstroke}%
\pgfsetdash{}{0pt}%
\pgfpathmoveto{\pgfqpoint{0.726903in}{1.243736in}}%
\pgfpathlineto{\pgfqpoint{0.916842in}{1.291711in}}%
\pgfusepath{stroke}%
\end{pgfscope}%
\begin{pgfscope}%
\pgfpathrectangle{\pgfqpoint{0.100000in}{0.212622in}}{\pgfqpoint{3.696000in}{3.696000in}}%
\pgfusepath{clip}%
\pgfsetrectcap%
\pgfsetroundjoin%
\pgfsetlinewidth{1.505625pt}%
\definecolor{currentstroke}{rgb}{1.000000,0.000000,0.000000}%
\pgfsetstrokecolor{currentstroke}%
\pgfsetdash{}{0pt}%
\pgfpathmoveto{\pgfqpoint{0.726903in}{1.243736in}}%
\pgfpathlineto{\pgfqpoint{0.916842in}{1.291711in}}%
\pgfusepath{stroke}%
\end{pgfscope}%
\begin{pgfscope}%
\pgfpathrectangle{\pgfqpoint{0.100000in}{0.212622in}}{\pgfqpoint{3.696000in}{3.696000in}}%
\pgfusepath{clip}%
\pgfsetrectcap%
\pgfsetroundjoin%
\pgfsetlinewidth{1.505625pt}%
\definecolor{currentstroke}{rgb}{1.000000,0.000000,0.000000}%
\pgfsetstrokecolor{currentstroke}%
\pgfsetdash{}{0pt}%
\pgfpathmoveto{\pgfqpoint{0.726903in}{1.243736in}}%
\pgfpathlineto{\pgfqpoint{0.916842in}{1.291711in}}%
\pgfusepath{stroke}%
\end{pgfscope}%
\begin{pgfscope}%
\pgfpathrectangle{\pgfqpoint{0.100000in}{0.212622in}}{\pgfqpoint{3.696000in}{3.696000in}}%
\pgfusepath{clip}%
\pgfsetrectcap%
\pgfsetroundjoin%
\pgfsetlinewidth{1.505625pt}%
\definecolor{currentstroke}{rgb}{1.000000,0.000000,0.000000}%
\pgfsetstrokecolor{currentstroke}%
\pgfsetdash{}{0pt}%
\pgfpathmoveto{\pgfqpoint{0.726903in}{1.243736in}}%
\pgfpathlineto{\pgfqpoint{0.916842in}{1.291711in}}%
\pgfusepath{stroke}%
\end{pgfscope}%
\begin{pgfscope}%
\pgfpathrectangle{\pgfqpoint{0.100000in}{0.212622in}}{\pgfqpoint{3.696000in}{3.696000in}}%
\pgfusepath{clip}%
\pgfsetrectcap%
\pgfsetroundjoin%
\pgfsetlinewidth{1.505625pt}%
\definecolor{currentstroke}{rgb}{1.000000,0.000000,0.000000}%
\pgfsetstrokecolor{currentstroke}%
\pgfsetdash{}{0pt}%
\pgfpathmoveto{\pgfqpoint{0.726903in}{1.243736in}}%
\pgfpathlineto{\pgfqpoint{0.916842in}{1.291711in}}%
\pgfusepath{stroke}%
\end{pgfscope}%
\begin{pgfscope}%
\pgfpathrectangle{\pgfqpoint{0.100000in}{0.212622in}}{\pgfqpoint{3.696000in}{3.696000in}}%
\pgfusepath{clip}%
\pgfsetrectcap%
\pgfsetroundjoin%
\pgfsetlinewidth{1.505625pt}%
\definecolor{currentstroke}{rgb}{1.000000,0.000000,0.000000}%
\pgfsetstrokecolor{currentstroke}%
\pgfsetdash{}{0pt}%
\pgfpathmoveto{\pgfqpoint{0.726903in}{1.243736in}}%
\pgfpathlineto{\pgfqpoint{0.916842in}{1.291711in}}%
\pgfusepath{stroke}%
\end{pgfscope}%
\begin{pgfscope}%
\pgfpathrectangle{\pgfqpoint{0.100000in}{0.212622in}}{\pgfqpoint{3.696000in}{3.696000in}}%
\pgfusepath{clip}%
\pgfsetrectcap%
\pgfsetroundjoin%
\pgfsetlinewidth{1.505625pt}%
\definecolor{currentstroke}{rgb}{1.000000,0.000000,0.000000}%
\pgfsetstrokecolor{currentstroke}%
\pgfsetdash{}{0pt}%
\pgfpathmoveto{\pgfqpoint{0.726903in}{1.243736in}}%
\pgfpathlineto{\pgfqpoint{0.916842in}{1.291711in}}%
\pgfusepath{stroke}%
\end{pgfscope}%
\begin{pgfscope}%
\pgfpathrectangle{\pgfqpoint{0.100000in}{0.212622in}}{\pgfqpoint{3.696000in}{3.696000in}}%
\pgfusepath{clip}%
\pgfsetrectcap%
\pgfsetroundjoin%
\pgfsetlinewidth{1.505625pt}%
\definecolor{currentstroke}{rgb}{1.000000,0.000000,0.000000}%
\pgfsetstrokecolor{currentstroke}%
\pgfsetdash{}{0pt}%
\pgfpathmoveto{\pgfqpoint{0.726903in}{1.243736in}}%
\pgfpathlineto{\pgfqpoint{0.916842in}{1.291711in}}%
\pgfusepath{stroke}%
\end{pgfscope}%
\begin{pgfscope}%
\pgfpathrectangle{\pgfqpoint{0.100000in}{0.212622in}}{\pgfqpoint{3.696000in}{3.696000in}}%
\pgfusepath{clip}%
\pgfsetrectcap%
\pgfsetroundjoin%
\pgfsetlinewidth{1.505625pt}%
\definecolor{currentstroke}{rgb}{1.000000,0.000000,0.000000}%
\pgfsetstrokecolor{currentstroke}%
\pgfsetdash{}{0pt}%
\pgfpathmoveto{\pgfqpoint{0.726903in}{1.243736in}}%
\pgfpathlineto{\pgfqpoint{0.916842in}{1.291711in}}%
\pgfusepath{stroke}%
\end{pgfscope}%
\begin{pgfscope}%
\pgfpathrectangle{\pgfqpoint{0.100000in}{0.212622in}}{\pgfqpoint{3.696000in}{3.696000in}}%
\pgfusepath{clip}%
\pgfsetrectcap%
\pgfsetroundjoin%
\pgfsetlinewidth{1.505625pt}%
\definecolor{currentstroke}{rgb}{1.000000,0.000000,0.000000}%
\pgfsetstrokecolor{currentstroke}%
\pgfsetdash{}{0pt}%
\pgfpathmoveto{\pgfqpoint{0.726903in}{1.243736in}}%
\pgfpathlineto{\pgfqpoint{0.916842in}{1.291711in}}%
\pgfusepath{stroke}%
\end{pgfscope}%
\begin{pgfscope}%
\pgfpathrectangle{\pgfqpoint{0.100000in}{0.212622in}}{\pgfqpoint{3.696000in}{3.696000in}}%
\pgfusepath{clip}%
\pgfsetrectcap%
\pgfsetroundjoin%
\pgfsetlinewidth{1.505625pt}%
\definecolor{currentstroke}{rgb}{1.000000,0.000000,0.000000}%
\pgfsetstrokecolor{currentstroke}%
\pgfsetdash{}{0pt}%
\pgfpathmoveto{\pgfqpoint{0.726903in}{1.243736in}}%
\pgfpathlineto{\pgfqpoint{0.916842in}{1.291711in}}%
\pgfusepath{stroke}%
\end{pgfscope}%
\begin{pgfscope}%
\pgfpathrectangle{\pgfqpoint{0.100000in}{0.212622in}}{\pgfqpoint{3.696000in}{3.696000in}}%
\pgfusepath{clip}%
\pgfsetrectcap%
\pgfsetroundjoin%
\pgfsetlinewidth{1.505625pt}%
\definecolor{currentstroke}{rgb}{1.000000,0.000000,0.000000}%
\pgfsetstrokecolor{currentstroke}%
\pgfsetdash{}{0pt}%
\pgfpathmoveto{\pgfqpoint{0.726903in}{1.243736in}}%
\pgfpathlineto{\pgfqpoint{0.916842in}{1.291711in}}%
\pgfusepath{stroke}%
\end{pgfscope}%
\begin{pgfscope}%
\pgfpathrectangle{\pgfqpoint{0.100000in}{0.212622in}}{\pgfqpoint{3.696000in}{3.696000in}}%
\pgfusepath{clip}%
\pgfsetrectcap%
\pgfsetroundjoin%
\pgfsetlinewidth{1.505625pt}%
\definecolor{currentstroke}{rgb}{1.000000,0.000000,0.000000}%
\pgfsetstrokecolor{currentstroke}%
\pgfsetdash{}{0pt}%
\pgfpathmoveto{\pgfqpoint{0.726903in}{1.243736in}}%
\pgfpathlineto{\pgfqpoint{0.916842in}{1.291711in}}%
\pgfusepath{stroke}%
\end{pgfscope}%
\begin{pgfscope}%
\pgfpathrectangle{\pgfqpoint{0.100000in}{0.212622in}}{\pgfqpoint{3.696000in}{3.696000in}}%
\pgfusepath{clip}%
\pgfsetrectcap%
\pgfsetroundjoin%
\pgfsetlinewidth{1.505625pt}%
\definecolor{currentstroke}{rgb}{1.000000,0.000000,0.000000}%
\pgfsetstrokecolor{currentstroke}%
\pgfsetdash{}{0pt}%
\pgfpathmoveto{\pgfqpoint{0.726903in}{1.243736in}}%
\pgfpathlineto{\pgfqpoint{0.916842in}{1.291711in}}%
\pgfusepath{stroke}%
\end{pgfscope}%
\begin{pgfscope}%
\pgfpathrectangle{\pgfqpoint{0.100000in}{0.212622in}}{\pgfqpoint{3.696000in}{3.696000in}}%
\pgfusepath{clip}%
\pgfsetrectcap%
\pgfsetroundjoin%
\pgfsetlinewidth{1.505625pt}%
\definecolor{currentstroke}{rgb}{1.000000,0.000000,0.000000}%
\pgfsetstrokecolor{currentstroke}%
\pgfsetdash{}{0pt}%
\pgfpathmoveto{\pgfqpoint{0.726903in}{1.243736in}}%
\pgfpathlineto{\pgfqpoint{0.916842in}{1.291711in}}%
\pgfusepath{stroke}%
\end{pgfscope}%
\begin{pgfscope}%
\pgfpathrectangle{\pgfqpoint{0.100000in}{0.212622in}}{\pgfqpoint{3.696000in}{3.696000in}}%
\pgfusepath{clip}%
\pgfsetrectcap%
\pgfsetroundjoin%
\pgfsetlinewidth{1.505625pt}%
\definecolor{currentstroke}{rgb}{1.000000,0.000000,0.000000}%
\pgfsetstrokecolor{currentstroke}%
\pgfsetdash{}{0pt}%
\pgfpathmoveto{\pgfqpoint{0.726903in}{1.243736in}}%
\pgfpathlineto{\pgfqpoint{0.916842in}{1.291711in}}%
\pgfusepath{stroke}%
\end{pgfscope}%
\begin{pgfscope}%
\pgfpathrectangle{\pgfqpoint{0.100000in}{0.212622in}}{\pgfqpoint{3.696000in}{3.696000in}}%
\pgfusepath{clip}%
\pgfsetrectcap%
\pgfsetroundjoin%
\pgfsetlinewidth{1.505625pt}%
\definecolor{currentstroke}{rgb}{1.000000,0.000000,0.000000}%
\pgfsetstrokecolor{currentstroke}%
\pgfsetdash{}{0pt}%
\pgfpathmoveto{\pgfqpoint{0.726903in}{1.243736in}}%
\pgfpathlineto{\pgfqpoint{0.916842in}{1.291711in}}%
\pgfusepath{stroke}%
\end{pgfscope}%
\begin{pgfscope}%
\pgfpathrectangle{\pgfqpoint{0.100000in}{0.212622in}}{\pgfqpoint{3.696000in}{3.696000in}}%
\pgfusepath{clip}%
\pgfsetrectcap%
\pgfsetroundjoin%
\pgfsetlinewidth{1.505625pt}%
\definecolor{currentstroke}{rgb}{1.000000,0.000000,0.000000}%
\pgfsetstrokecolor{currentstroke}%
\pgfsetdash{}{0pt}%
\pgfpathmoveto{\pgfqpoint{0.726903in}{1.243736in}}%
\pgfpathlineto{\pgfqpoint{0.916842in}{1.291711in}}%
\pgfusepath{stroke}%
\end{pgfscope}%
\begin{pgfscope}%
\pgfpathrectangle{\pgfqpoint{0.100000in}{0.212622in}}{\pgfqpoint{3.696000in}{3.696000in}}%
\pgfusepath{clip}%
\pgfsetrectcap%
\pgfsetroundjoin%
\pgfsetlinewidth{1.505625pt}%
\definecolor{currentstroke}{rgb}{1.000000,0.000000,0.000000}%
\pgfsetstrokecolor{currentstroke}%
\pgfsetdash{}{0pt}%
\pgfpathmoveto{\pgfqpoint{0.726903in}{1.243736in}}%
\pgfpathlineto{\pgfqpoint{0.916842in}{1.291711in}}%
\pgfusepath{stroke}%
\end{pgfscope}%
\begin{pgfscope}%
\pgfpathrectangle{\pgfqpoint{0.100000in}{0.212622in}}{\pgfqpoint{3.696000in}{3.696000in}}%
\pgfusepath{clip}%
\pgfsetrectcap%
\pgfsetroundjoin%
\pgfsetlinewidth{1.505625pt}%
\definecolor{currentstroke}{rgb}{1.000000,0.000000,0.000000}%
\pgfsetstrokecolor{currentstroke}%
\pgfsetdash{}{0pt}%
\pgfpathmoveto{\pgfqpoint{0.726903in}{1.243736in}}%
\pgfpathlineto{\pgfqpoint{0.916842in}{1.291711in}}%
\pgfusepath{stroke}%
\end{pgfscope}%
\begin{pgfscope}%
\pgfpathrectangle{\pgfqpoint{0.100000in}{0.212622in}}{\pgfqpoint{3.696000in}{3.696000in}}%
\pgfusepath{clip}%
\pgfsetrectcap%
\pgfsetroundjoin%
\pgfsetlinewidth{1.505625pt}%
\definecolor{currentstroke}{rgb}{1.000000,0.000000,0.000000}%
\pgfsetstrokecolor{currentstroke}%
\pgfsetdash{}{0pt}%
\pgfpathmoveto{\pgfqpoint{0.726903in}{1.243736in}}%
\pgfpathlineto{\pgfqpoint{0.916842in}{1.291711in}}%
\pgfusepath{stroke}%
\end{pgfscope}%
\begin{pgfscope}%
\pgfpathrectangle{\pgfqpoint{0.100000in}{0.212622in}}{\pgfqpoint{3.696000in}{3.696000in}}%
\pgfusepath{clip}%
\pgfsetrectcap%
\pgfsetroundjoin%
\pgfsetlinewidth{1.505625pt}%
\definecolor{currentstroke}{rgb}{1.000000,0.000000,0.000000}%
\pgfsetstrokecolor{currentstroke}%
\pgfsetdash{}{0pt}%
\pgfpathmoveto{\pgfqpoint{0.726903in}{1.243736in}}%
\pgfpathlineto{\pgfqpoint{0.916842in}{1.291711in}}%
\pgfusepath{stroke}%
\end{pgfscope}%
\begin{pgfscope}%
\pgfpathrectangle{\pgfqpoint{0.100000in}{0.212622in}}{\pgfqpoint{3.696000in}{3.696000in}}%
\pgfusepath{clip}%
\pgfsetrectcap%
\pgfsetroundjoin%
\pgfsetlinewidth{1.505625pt}%
\definecolor{currentstroke}{rgb}{1.000000,0.000000,0.000000}%
\pgfsetstrokecolor{currentstroke}%
\pgfsetdash{}{0pt}%
\pgfpathmoveto{\pgfqpoint{0.726903in}{1.243736in}}%
\pgfpathlineto{\pgfqpoint{0.916842in}{1.291711in}}%
\pgfusepath{stroke}%
\end{pgfscope}%
\begin{pgfscope}%
\pgfpathrectangle{\pgfqpoint{0.100000in}{0.212622in}}{\pgfqpoint{3.696000in}{3.696000in}}%
\pgfusepath{clip}%
\pgfsetrectcap%
\pgfsetroundjoin%
\pgfsetlinewidth{1.505625pt}%
\definecolor{currentstroke}{rgb}{1.000000,0.000000,0.000000}%
\pgfsetstrokecolor{currentstroke}%
\pgfsetdash{}{0pt}%
\pgfpathmoveto{\pgfqpoint{0.726903in}{1.243736in}}%
\pgfpathlineto{\pgfqpoint{0.916842in}{1.291711in}}%
\pgfusepath{stroke}%
\end{pgfscope}%
\begin{pgfscope}%
\pgfpathrectangle{\pgfqpoint{0.100000in}{0.212622in}}{\pgfqpoint{3.696000in}{3.696000in}}%
\pgfusepath{clip}%
\pgfsetrectcap%
\pgfsetroundjoin%
\pgfsetlinewidth{1.505625pt}%
\definecolor{currentstroke}{rgb}{1.000000,0.000000,0.000000}%
\pgfsetstrokecolor{currentstroke}%
\pgfsetdash{}{0pt}%
\pgfpathmoveto{\pgfqpoint{0.726903in}{1.243736in}}%
\pgfpathlineto{\pgfqpoint{0.916842in}{1.291711in}}%
\pgfusepath{stroke}%
\end{pgfscope}%
\begin{pgfscope}%
\pgfpathrectangle{\pgfqpoint{0.100000in}{0.212622in}}{\pgfqpoint{3.696000in}{3.696000in}}%
\pgfusepath{clip}%
\pgfsetrectcap%
\pgfsetroundjoin%
\pgfsetlinewidth{1.505625pt}%
\definecolor{currentstroke}{rgb}{1.000000,0.000000,0.000000}%
\pgfsetstrokecolor{currentstroke}%
\pgfsetdash{}{0pt}%
\pgfpathmoveto{\pgfqpoint{0.724743in}{1.243364in}}%
\pgfpathlineto{\pgfqpoint{0.916842in}{1.291711in}}%
\pgfusepath{stroke}%
\end{pgfscope}%
\begin{pgfscope}%
\pgfpathrectangle{\pgfqpoint{0.100000in}{0.212622in}}{\pgfqpoint{3.696000in}{3.696000in}}%
\pgfusepath{clip}%
\pgfsetrectcap%
\pgfsetroundjoin%
\pgfsetlinewidth{1.505625pt}%
\definecolor{currentstroke}{rgb}{1.000000,0.000000,0.000000}%
\pgfsetstrokecolor{currentstroke}%
\pgfsetdash{}{0pt}%
\pgfpathmoveto{\pgfqpoint{0.723456in}{1.243873in}}%
\pgfpathlineto{\pgfqpoint{0.916842in}{1.291711in}}%
\pgfusepath{stroke}%
\end{pgfscope}%
\begin{pgfscope}%
\pgfpathrectangle{\pgfqpoint{0.100000in}{0.212622in}}{\pgfqpoint{3.696000in}{3.696000in}}%
\pgfusepath{clip}%
\pgfsetrectcap%
\pgfsetroundjoin%
\pgfsetlinewidth{1.505625pt}%
\definecolor{currentstroke}{rgb}{1.000000,0.000000,0.000000}%
\pgfsetstrokecolor{currentstroke}%
\pgfsetdash{}{0pt}%
\pgfpathmoveto{\pgfqpoint{0.722800in}{1.243850in}}%
\pgfpathlineto{\pgfqpoint{0.916842in}{1.291711in}}%
\pgfusepath{stroke}%
\end{pgfscope}%
\begin{pgfscope}%
\pgfpathrectangle{\pgfqpoint{0.100000in}{0.212622in}}{\pgfqpoint{3.696000in}{3.696000in}}%
\pgfusepath{clip}%
\pgfsetrectcap%
\pgfsetroundjoin%
\pgfsetlinewidth{1.505625pt}%
\definecolor{currentstroke}{rgb}{1.000000,0.000000,0.000000}%
\pgfsetstrokecolor{currentstroke}%
\pgfsetdash{}{0pt}%
\pgfpathmoveto{\pgfqpoint{0.722412in}{1.244032in}}%
\pgfpathlineto{\pgfqpoint{0.916842in}{1.291711in}}%
\pgfusepath{stroke}%
\end{pgfscope}%
\begin{pgfscope}%
\pgfpathrectangle{\pgfqpoint{0.100000in}{0.212622in}}{\pgfqpoint{3.696000in}{3.696000in}}%
\pgfusepath{clip}%
\pgfsetrectcap%
\pgfsetroundjoin%
\pgfsetlinewidth{1.505625pt}%
\definecolor{currentstroke}{rgb}{1.000000,0.000000,0.000000}%
\pgfsetstrokecolor{currentstroke}%
\pgfsetdash{}{0pt}%
\pgfpathmoveto{\pgfqpoint{0.722221in}{1.244031in}}%
\pgfpathlineto{\pgfqpoint{0.916842in}{1.291711in}}%
\pgfusepath{stroke}%
\end{pgfscope}%
\begin{pgfscope}%
\pgfpathrectangle{\pgfqpoint{0.100000in}{0.212622in}}{\pgfqpoint{3.696000in}{3.696000in}}%
\pgfusepath{clip}%
\pgfsetrectcap%
\pgfsetroundjoin%
\pgfsetlinewidth{1.505625pt}%
\definecolor{currentstroke}{rgb}{1.000000,0.000000,0.000000}%
\pgfsetstrokecolor{currentstroke}%
\pgfsetdash{}{0pt}%
\pgfpathmoveto{\pgfqpoint{0.722105in}{1.244102in}}%
\pgfpathlineto{\pgfqpoint{0.916842in}{1.291711in}}%
\pgfusepath{stroke}%
\end{pgfscope}%
\begin{pgfscope}%
\pgfpathrectangle{\pgfqpoint{0.100000in}{0.212622in}}{\pgfqpoint{3.696000in}{3.696000in}}%
\pgfusepath{clip}%
\pgfsetrectcap%
\pgfsetroundjoin%
\pgfsetlinewidth{1.505625pt}%
\definecolor{currentstroke}{rgb}{1.000000,0.000000,0.000000}%
\pgfsetstrokecolor{currentstroke}%
\pgfsetdash{}{0pt}%
\pgfpathmoveto{\pgfqpoint{0.720189in}{1.244071in}}%
\pgfpathlineto{\pgfqpoint{0.916842in}{1.291711in}}%
\pgfusepath{stroke}%
\end{pgfscope}%
\begin{pgfscope}%
\pgfpathrectangle{\pgfqpoint{0.100000in}{0.212622in}}{\pgfqpoint{3.696000in}{3.696000in}}%
\pgfusepath{clip}%
\pgfsetrectcap%
\pgfsetroundjoin%
\pgfsetlinewidth{1.505625pt}%
\definecolor{currentstroke}{rgb}{1.000000,0.000000,0.000000}%
\pgfsetstrokecolor{currentstroke}%
\pgfsetdash{}{0pt}%
\pgfpathmoveto{\pgfqpoint{0.718872in}{1.244675in}}%
\pgfpathlineto{\pgfqpoint{0.916842in}{1.291711in}}%
\pgfusepath{stroke}%
\end{pgfscope}%
\begin{pgfscope}%
\pgfpathrectangle{\pgfqpoint{0.100000in}{0.212622in}}{\pgfqpoint{3.696000in}{3.696000in}}%
\pgfusepath{clip}%
\pgfsetrectcap%
\pgfsetroundjoin%
\pgfsetlinewidth{1.505625pt}%
\definecolor{currentstroke}{rgb}{1.000000,0.000000,0.000000}%
\pgfsetstrokecolor{currentstroke}%
\pgfsetdash{}{0pt}%
\pgfpathmoveto{\pgfqpoint{0.718219in}{1.244668in}}%
\pgfpathlineto{\pgfqpoint{0.916842in}{1.291711in}}%
\pgfusepath{stroke}%
\end{pgfscope}%
\begin{pgfscope}%
\pgfpathrectangle{\pgfqpoint{0.100000in}{0.212622in}}{\pgfqpoint{3.696000in}{3.696000in}}%
\pgfusepath{clip}%
\pgfsetrectcap%
\pgfsetroundjoin%
\pgfsetlinewidth{1.505625pt}%
\definecolor{currentstroke}{rgb}{1.000000,0.000000,0.000000}%
\pgfsetstrokecolor{currentstroke}%
\pgfsetdash{}{0pt}%
\pgfpathmoveto{\pgfqpoint{0.717963in}{1.244850in}}%
\pgfpathlineto{\pgfqpoint{0.916842in}{1.291711in}}%
\pgfusepath{stroke}%
\end{pgfscope}%
\begin{pgfscope}%
\pgfpathrectangle{\pgfqpoint{0.100000in}{0.212622in}}{\pgfqpoint{3.696000in}{3.696000in}}%
\pgfusepath{clip}%
\pgfsetrectcap%
\pgfsetroundjoin%
\pgfsetlinewidth{1.505625pt}%
\definecolor{currentstroke}{rgb}{1.000000,0.000000,0.000000}%
\pgfsetstrokecolor{currentstroke}%
\pgfsetdash{}{0pt}%
\pgfpathmoveto{\pgfqpoint{0.717860in}{1.244971in}}%
\pgfpathlineto{\pgfqpoint{0.916842in}{1.291711in}}%
\pgfusepath{stroke}%
\end{pgfscope}%
\begin{pgfscope}%
\pgfpathrectangle{\pgfqpoint{0.100000in}{0.212622in}}{\pgfqpoint{3.696000in}{3.696000in}}%
\pgfusepath{clip}%
\pgfsetrectcap%
\pgfsetroundjoin%
\pgfsetlinewidth{1.505625pt}%
\definecolor{currentstroke}{rgb}{1.000000,0.000000,0.000000}%
\pgfsetstrokecolor{currentstroke}%
\pgfsetdash{}{0pt}%
\pgfpathmoveto{\pgfqpoint{0.717845in}{1.245040in}}%
\pgfpathlineto{\pgfqpoint{0.916842in}{1.291711in}}%
\pgfusepath{stroke}%
\end{pgfscope}%
\begin{pgfscope}%
\pgfpathrectangle{\pgfqpoint{0.100000in}{0.212622in}}{\pgfqpoint{3.696000in}{3.696000in}}%
\pgfusepath{clip}%
\pgfsetrectcap%
\pgfsetroundjoin%
\pgfsetlinewidth{1.505625pt}%
\definecolor{currentstroke}{rgb}{1.000000,0.000000,0.000000}%
\pgfsetstrokecolor{currentstroke}%
\pgfsetdash{}{0pt}%
\pgfpathmoveto{\pgfqpoint{0.717858in}{1.245091in}}%
\pgfpathlineto{\pgfqpoint{0.916842in}{1.291711in}}%
\pgfusepath{stroke}%
\end{pgfscope}%
\begin{pgfscope}%
\pgfpathrectangle{\pgfqpoint{0.100000in}{0.212622in}}{\pgfqpoint{3.696000in}{3.696000in}}%
\pgfusepath{clip}%
\pgfsetrectcap%
\pgfsetroundjoin%
\pgfsetlinewidth{1.505625pt}%
\definecolor{currentstroke}{rgb}{1.000000,0.000000,0.000000}%
\pgfsetstrokecolor{currentstroke}%
\pgfsetdash{}{0pt}%
\pgfpathmoveto{\pgfqpoint{0.717875in}{1.245116in}}%
\pgfpathlineto{\pgfqpoint{0.916842in}{1.291711in}}%
\pgfusepath{stroke}%
\end{pgfscope}%
\begin{pgfscope}%
\pgfpathrectangle{\pgfqpoint{0.100000in}{0.212622in}}{\pgfqpoint{3.696000in}{3.696000in}}%
\pgfusepath{clip}%
\pgfsetrectcap%
\pgfsetroundjoin%
\pgfsetlinewidth{1.505625pt}%
\definecolor{currentstroke}{rgb}{1.000000,0.000000,0.000000}%
\pgfsetstrokecolor{currentstroke}%
\pgfsetdash{}{0pt}%
\pgfpathmoveto{\pgfqpoint{0.717884in}{1.245132in}}%
\pgfpathlineto{\pgfqpoint{0.916842in}{1.291711in}}%
\pgfusepath{stroke}%
\end{pgfscope}%
\begin{pgfscope}%
\pgfpathrectangle{\pgfqpoint{0.100000in}{0.212622in}}{\pgfqpoint{3.696000in}{3.696000in}}%
\pgfusepath{clip}%
\pgfsetrectcap%
\pgfsetroundjoin%
\pgfsetlinewidth{1.505625pt}%
\definecolor{currentstroke}{rgb}{1.000000,0.000000,0.000000}%
\pgfsetstrokecolor{currentstroke}%
\pgfsetdash{}{0pt}%
\pgfpathmoveto{\pgfqpoint{0.719143in}{1.247070in}}%
\pgfpathlineto{\pgfqpoint{0.916842in}{1.291711in}}%
\pgfusepath{stroke}%
\end{pgfscope}%
\begin{pgfscope}%
\pgfpathrectangle{\pgfqpoint{0.100000in}{0.212622in}}{\pgfqpoint{3.696000in}{3.696000in}}%
\pgfusepath{clip}%
\pgfsetrectcap%
\pgfsetroundjoin%
\pgfsetlinewidth{1.505625pt}%
\definecolor{currentstroke}{rgb}{1.000000,0.000000,0.000000}%
\pgfsetstrokecolor{currentstroke}%
\pgfsetdash{}{0pt}%
\pgfpathmoveto{\pgfqpoint{0.719935in}{1.248111in}}%
\pgfpathlineto{\pgfqpoint{0.916842in}{1.291711in}}%
\pgfusepath{stroke}%
\end{pgfscope}%
\begin{pgfscope}%
\pgfpathrectangle{\pgfqpoint{0.100000in}{0.212622in}}{\pgfqpoint{3.696000in}{3.696000in}}%
\pgfusepath{clip}%
\pgfsetrectcap%
\pgfsetroundjoin%
\pgfsetlinewidth{1.505625pt}%
\definecolor{currentstroke}{rgb}{1.000000,0.000000,0.000000}%
\pgfsetstrokecolor{currentstroke}%
\pgfsetdash{}{0pt}%
\pgfpathmoveto{\pgfqpoint{0.720371in}{1.248668in}}%
\pgfpathlineto{\pgfqpoint{0.916842in}{1.291711in}}%
\pgfusepath{stroke}%
\end{pgfscope}%
\begin{pgfscope}%
\pgfpathrectangle{\pgfqpoint{0.100000in}{0.212622in}}{\pgfqpoint{3.696000in}{3.696000in}}%
\pgfusepath{clip}%
\pgfsetrectcap%
\pgfsetroundjoin%
\pgfsetlinewidth{1.505625pt}%
\definecolor{currentstroke}{rgb}{1.000000,0.000000,0.000000}%
\pgfsetstrokecolor{currentstroke}%
\pgfsetdash{}{0pt}%
\pgfpathmoveto{\pgfqpoint{0.720574in}{1.249018in}}%
\pgfpathlineto{\pgfqpoint{0.916842in}{1.291711in}}%
\pgfusepath{stroke}%
\end{pgfscope}%
\begin{pgfscope}%
\pgfpathrectangle{\pgfqpoint{0.100000in}{0.212622in}}{\pgfqpoint{3.696000in}{3.696000in}}%
\pgfusepath{clip}%
\pgfsetrectcap%
\pgfsetroundjoin%
\pgfsetlinewidth{1.505625pt}%
\definecolor{currentstroke}{rgb}{1.000000,0.000000,0.000000}%
\pgfsetstrokecolor{currentstroke}%
\pgfsetdash{}{0pt}%
\pgfpathmoveto{\pgfqpoint{0.720708in}{1.249201in}}%
\pgfpathlineto{\pgfqpoint{0.916842in}{1.291711in}}%
\pgfusepath{stroke}%
\end{pgfscope}%
\begin{pgfscope}%
\pgfpathrectangle{\pgfqpoint{0.100000in}{0.212622in}}{\pgfqpoint{3.696000in}{3.696000in}}%
\pgfusepath{clip}%
\pgfsetrectcap%
\pgfsetroundjoin%
\pgfsetlinewidth{1.505625pt}%
\definecolor{currentstroke}{rgb}{1.000000,0.000000,0.000000}%
\pgfsetstrokecolor{currentstroke}%
\pgfsetdash{}{0pt}%
\pgfpathmoveto{\pgfqpoint{0.720771in}{1.249301in}}%
\pgfpathlineto{\pgfqpoint{0.916842in}{1.291711in}}%
\pgfusepath{stroke}%
\end{pgfscope}%
\begin{pgfscope}%
\pgfpathrectangle{\pgfqpoint{0.100000in}{0.212622in}}{\pgfqpoint{3.696000in}{3.696000in}}%
\pgfusepath{clip}%
\pgfsetrectcap%
\pgfsetroundjoin%
\pgfsetlinewidth{1.505625pt}%
\definecolor{currentstroke}{rgb}{1.000000,0.000000,0.000000}%
\pgfsetstrokecolor{currentstroke}%
\pgfsetdash{}{0pt}%
\pgfpathmoveto{\pgfqpoint{0.720804in}{1.249351in}}%
\pgfpathlineto{\pgfqpoint{0.916842in}{1.291711in}}%
\pgfusepath{stroke}%
\end{pgfscope}%
\begin{pgfscope}%
\pgfpathrectangle{\pgfqpoint{0.100000in}{0.212622in}}{\pgfqpoint{3.696000in}{3.696000in}}%
\pgfusepath{clip}%
\pgfsetrectcap%
\pgfsetroundjoin%
\pgfsetlinewidth{1.505625pt}%
\definecolor{currentstroke}{rgb}{1.000000,0.000000,0.000000}%
\pgfsetstrokecolor{currentstroke}%
\pgfsetdash{}{0pt}%
\pgfpathmoveto{\pgfqpoint{0.720826in}{1.249384in}}%
\pgfpathlineto{\pgfqpoint{0.916842in}{1.291711in}}%
\pgfusepath{stroke}%
\end{pgfscope}%
\begin{pgfscope}%
\pgfpathrectangle{\pgfqpoint{0.100000in}{0.212622in}}{\pgfqpoint{3.696000in}{3.696000in}}%
\pgfusepath{clip}%
\pgfsetrectcap%
\pgfsetroundjoin%
\pgfsetlinewidth{1.505625pt}%
\definecolor{currentstroke}{rgb}{1.000000,0.000000,0.000000}%
\pgfsetstrokecolor{currentstroke}%
\pgfsetdash{}{0pt}%
\pgfpathmoveto{\pgfqpoint{0.720838in}{1.249400in}}%
\pgfpathlineto{\pgfqpoint{0.916842in}{1.291711in}}%
\pgfusepath{stroke}%
\end{pgfscope}%
\begin{pgfscope}%
\pgfpathrectangle{\pgfqpoint{0.100000in}{0.212622in}}{\pgfqpoint{3.696000in}{3.696000in}}%
\pgfusepath{clip}%
\pgfsetrectcap%
\pgfsetroundjoin%
\pgfsetlinewidth{1.505625pt}%
\definecolor{currentstroke}{rgb}{1.000000,0.000000,0.000000}%
\pgfsetstrokecolor{currentstroke}%
\pgfsetdash{}{0pt}%
\pgfpathmoveto{\pgfqpoint{0.720844in}{1.249408in}}%
\pgfpathlineto{\pgfqpoint{0.916842in}{1.291711in}}%
\pgfusepath{stroke}%
\end{pgfscope}%
\begin{pgfscope}%
\pgfpathrectangle{\pgfqpoint{0.100000in}{0.212622in}}{\pgfqpoint{3.696000in}{3.696000in}}%
\pgfusepath{clip}%
\pgfsetrectcap%
\pgfsetroundjoin%
\pgfsetlinewidth{1.505625pt}%
\definecolor{currentstroke}{rgb}{1.000000,0.000000,0.000000}%
\pgfsetstrokecolor{currentstroke}%
\pgfsetdash{}{0pt}%
\pgfpathmoveto{\pgfqpoint{0.720848in}{1.249413in}}%
\pgfpathlineto{\pgfqpoint{0.916842in}{1.291711in}}%
\pgfusepath{stroke}%
\end{pgfscope}%
\begin{pgfscope}%
\pgfpathrectangle{\pgfqpoint{0.100000in}{0.212622in}}{\pgfqpoint{3.696000in}{3.696000in}}%
\pgfusepath{clip}%
\pgfsetrectcap%
\pgfsetroundjoin%
\pgfsetlinewidth{1.505625pt}%
\definecolor{currentstroke}{rgb}{1.000000,0.000000,0.000000}%
\pgfsetstrokecolor{currentstroke}%
\pgfsetdash{}{0pt}%
\pgfpathmoveto{\pgfqpoint{0.720850in}{1.249415in}}%
\pgfpathlineto{\pgfqpoint{0.916842in}{1.291711in}}%
\pgfusepath{stroke}%
\end{pgfscope}%
\begin{pgfscope}%
\pgfpathrectangle{\pgfqpoint{0.100000in}{0.212622in}}{\pgfqpoint{3.696000in}{3.696000in}}%
\pgfusepath{clip}%
\pgfsetrectcap%
\pgfsetroundjoin%
\pgfsetlinewidth{1.505625pt}%
\definecolor{currentstroke}{rgb}{1.000000,0.000000,0.000000}%
\pgfsetstrokecolor{currentstroke}%
\pgfsetdash{}{0pt}%
\pgfpathmoveto{\pgfqpoint{0.720851in}{1.249417in}}%
\pgfpathlineto{\pgfqpoint{0.916842in}{1.291711in}}%
\pgfusepath{stroke}%
\end{pgfscope}%
\begin{pgfscope}%
\pgfpathrectangle{\pgfqpoint{0.100000in}{0.212622in}}{\pgfqpoint{3.696000in}{3.696000in}}%
\pgfusepath{clip}%
\pgfsetrectcap%
\pgfsetroundjoin%
\pgfsetlinewidth{1.505625pt}%
\definecolor{currentstroke}{rgb}{1.000000,0.000000,0.000000}%
\pgfsetstrokecolor{currentstroke}%
\pgfsetdash{}{0pt}%
\pgfpathmoveto{\pgfqpoint{0.720851in}{1.249417in}}%
\pgfpathlineto{\pgfqpoint{0.916842in}{1.291711in}}%
\pgfusepath{stroke}%
\end{pgfscope}%
\begin{pgfscope}%
\pgfpathrectangle{\pgfqpoint{0.100000in}{0.212622in}}{\pgfqpoint{3.696000in}{3.696000in}}%
\pgfusepath{clip}%
\pgfsetrectcap%
\pgfsetroundjoin%
\pgfsetlinewidth{1.505625pt}%
\definecolor{currentstroke}{rgb}{1.000000,0.000000,0.000000}%
\pgfsetstrokecolor{currentstroke}%
\pgfsetdash{}{0pt}%
\pgfpathmoveto{\pgfqpoint{0.722047in}{1.250896in}}%
\pgfpathlineto{\pgfqpoint{0.916842in}{1.291711in}}%
\pgfusepath{stroke}%
\end{pgfscope}%
\begin{pgfscope}%
\pgfpathrectangle{\pgfqpoint{0.100000in}{0.212622in}}{\pgfqpoint{3.696000in}{3.696000in}}%
\pgfusepath{clip}%
\pgfsetrectcap%
\pgfsetroundjoin%
\pgfsetlinewidth{1.505625pt}%
\definecolor{currentstroke}{rgb}{1.000000,0.000000,0.000000}%
\pgfsetstrokecolor{currentstroke}%
\pgfsetdash{}{0pt}%
\pgfpathmoveto{\pgfqpoint{0.722413in}{1.251829in}}%
\pgfpathlineto{\pgfqpoint{0.916842in}{1.291711in}}%
\pgfusepath{stroke}%
\end{pgfscope}%
\begin{pgfscope}%
\pgfpathrectangle{\pgfqpoint{0.100000in}{0.212622in}}{\pgfqpoint{3.696000in}{3.696000in}}%
\pgfusepath{clip}%
\pgfsetrectcap%
\pgfsetroundjoin%
\pgfsetlinewidth{1.505625pt}%
\definecolor{currentstroke}{rgb}{1.000000,0.000000,0.000000}%
\pgfsetstrokecolor{currentstroke}%
\pgfsetdash{}{0pt}%
\pgfpathmoveto{\pgfqpoint{0.722726in}{1.252441in}}%
\pgfpathlineto{\pgfqpoint{0.916842in}{1.291711in}}%
\pgfusepath{stroke}%
\end{pgfscope}%
\begin{pgfscope}%
\pgfpathrectangle{\pgfqpoint{0.100000in}{0.212622in}}{\pgfqpoint{3.696000in}{3.696000in}}%
\pgfusepath{clip}%
\pgfsetrectcap%
\pgfsetroundjoin%
\pgfsetlinewidth{1.505625pt}%
\definecolor{currentstroke}{rgb}{1.000000,0.000000,0.000000}%
\pgfsetstrokecolor{currentstroke}%
\pgfsetdash{}{0pt}%
\pgfpathmoveto{\pgfqpoint{0.722868in}{1.252791in}}%
\pgfpathlineto{\pgfqpoint{0.916842in}{1.291711in}}%
\pgfusepath{stroke}%
\end{pgfscope}%
\begin{pgfscope}%
\pgfpathrectangle{\pgfqpoint{0.100000in}{0.212622in}}{\pgfqpoint{3.696000in}{3.696000in}}%
\pgfusepath{clip}%
\pgfsetrectcap%
\pgfsetroundjoin%
\pgfsetlinewidth{1.505625pt}%
\definecolor{currentstroke}{rgb}{1.000000,0.000000,0.000000}%
\pgfsetstrokecolor{currentstroke}%
\pgfsetdash{}{0pt}%
\pgfpathmoveto{\pgfqpoint{0.722954in}{1.252981in}}%
\pgfpathlineto{\pgfqpoint{0.916842in}{1.291711in}}%
\pgfusepath{stroke}%
\end{pgfscope}%
\begin{pgfscope}%
\pgfpathrectangle{\pgfqpoint{0.100000in}{0.212622in}}{\pgfqpoint{3.696000in}{3.696000in}}%
\pgfusepath{clip}%
\pgfsetrectcap%
\pgfsetroundjoin%
\pgfsetlinewidth{1.505625pt}%
\definecolor{currentstroke}{rgb}{1.000000,0.000000,0.000000}%
\pgfsetstrokecolor{currentstroke}%
\pgfsetdash{}{0pt}%
\pgfpathmoveto{\pgfqpoint{0.723011in}{1.253042in}}%
\pgfpathlineto{\pgfqpoint{0.916842in}{1.291711in}}%
\pgfusepath{stroke}%
\end{pgfscope}%
\begin{pgfscope}%
\pgfpathrectangle{\pgfqpoint{0.100000in}{0.212622in}}{\pgfqpoint{3.696000in}{3.696000in}}%
\pgfusepath{clip}%
\pgfsetrectcap%
\pgfsetroundjoin%
\pgfsetlinewidth{1.505625pt}%
\definecolor{currentstroke}{rgb}{1.000000,0.000000,0.000000}%
\pgfsetstrokecolor{currentstroke}%
\pgfsetdash{}{0pt}%
\pgfpathmoveto{\pgfqpoint{0.723040in}{1.253082in}}%
\pgfpathlineto{\pgfqpoint{0.916842in}{1.291711in}}%
\pgfusepath{stroke}%
\end{pgfscope}%
\begin{pgfscope}%
\pgfpathrectangle{\pgfqpoint{0.100000in}{0.212622in}}{\pgfqpoint{3.696000in}{3.696000in}}%
\pgfusepath{clip}%
\pgfsetrectcap%
\pgfsetroundjoin%
\pgfsetlinewidth{1.505625pt}%
\definecolor{currentstroke}{rgb}{1.000000,0.000000,0.000000}%
\pgfsetstrokecolor{currentstroke}%
\pgfsetdash{}{0pt}%
\pgfpathmoveto{\pgfqpoint{0.723060in}{1.253106in}}%
\pgfpathlineto{\pgfqpoint{0.916842in}{1.291711in}}%
\pgfusepath{stroke}%
\end{pgfscope}%
\begin{pgfscope}%
\pgfpathrectangle{\pgfqpoint{0.100000in}{0.212622in}}{\pgfqpoint{3.696000in}{3.696000in}}%
\pgfusepath{clip}%
\pgfsetrectcap%
\pgfsetroundjoin%
\pgfsetlinewidth{1.505625pt}%
\definecolor{currentstroke}{rgb}{1.000000,0.000000,0.000000}%
\pgfsetstrokecolor{currentstroke}%
\pgfsetdash{}{0pt}%
\pgfpathmoveto{\pgfqpoint{0.723069in}{1.253115in}}%
\pgfpathlineto{\pgfqpoint{0.916842in}{1.291711in}}%
\pgfusepath{stroke}%
\end{pgfscope}%
\begin{pgfscope}%
\pgfpathrectangle{\pgfqpoint{0.100000in}{0.212622in}}{\pgfqpoint{3.696000in}{3.696000in}}%
\pgfusepath{clip}%
\pgfsetrectcap%
\pgfsetroundjoin%
\pgfsetlinewidth{1.505625pt}%
\definecolor{currentstroke}{rgb}{1.000000,0.000000,0.000000}%
\pgfsetstrokecolor{currentstroke}%
\pgfsetdash{}{0pt}%
\pgfpathmoveto{\pgfqpoint{0.723075in}{1.253122in}}%
\pgfpathlineto{\pgfqpoint{0.916842in}{1.291711in}}%
\pgfusepath{stroke}%
\end{pgfscope}%
\begin{pgfscope}%
\pgfpathrectangle{\pgfqpoint{0.100000in}{0.212622in}}{\pgfqpoint{3.696000in}{3.696000in}}%
\pgfusepath{clip}%
\pgfsetrectcap%
\pgfsetroundjoin%
\pgfsetlinewidth{1.505625pt}%
\definecolor{currentstroke}{rgb}{1.000000,0.000000,0.000000}%
\pgfsetstrokecolor{currentstroke}%
\pgfsetdash{}{0pt}%
\pgfpathmoveto{\pgfqpoint{0.723078in}{1.253125in}}%
\pgfpathlineto{\pgfqpoint{0.916842in}{1.291711in}}%
\pgfusepath{stroke}%
\end{pgfscope}%
\begin{pgfscope}%
\pgfpathrectangle{\pgfqpoint{0.100000in}{0.212622in}}{\pgfqpoint{3.696000in}{3.696000in}}%
\pgfusepath{clip}%
\pgfsetrectcap%
\pgfsetroundjoin%
\pgfsetlinewidth{1.505625pt}%
\definecolor{currentstroke}{rgb}{1.000000,0.000000,0.000000}%
\pgfsetstrokecolor{currentstroke}%
\pgfsetdash{}{0pt}%
\pgfpathmoveto{\pgfqpoint{0.723079in}{1.253126in}}%
\pgfpathlineto{\pgfqpoint{0.916842in}{1.291711in}}%
\pgfusepath{stroke}%
\end{pgfscope}%
\begin{pgfscope}%
\pgfpathrectangle{\pgfqpoint{0.100000in}{0.212622in}}{\pgfqpoint{3.696000in}{3.696000in}}%
\pgfusepath{clip}%
\pgfsetrectcap%
\pgfsetroundjoin%
\pgfsetlinewidth{1.505625pt}%
\definecolor{currentstroke}{rgb}{1.000000,0.000000,0.000000}%
\pgfsetstrokecolor{currentstroke}%
\pgfsetdash{}{0pt}%
\pgfpathmoveto{\pgfqpoint{0.723080in}{1.253127in}}%
\pgfpathlineto{\pgfqpoint{0.916842in}{1.291711in}}%
\pgfusepath{stroke}%
\end{pgfscope}%
\begin{pgfscope}%
\pgfpathrectangle{\pgfqpoint{0.100000in}{0.212622in}}{\pgfqpoint{3.696000in}{3.696000in}}%
\pgfusepath{clip}%
\pgfsetrectcap%
\pgfsetroundjoin%
\pgfsetlinewidth{1.505625pt}%
\definecolor{currentstroke}{rgb}{1.000000,0.000000,0.000000}%
\pgfsetstrokecolor{currentstroke}%
\pgfsetdash{}{0pt}%
\pgfpathmoveto{\pgfqpoint{0.723081in}{1.253128in}}%
\pgfpathlineto{\pgfqpoint{0.916842in}{1.291711in}}%
\pgfusepath{stroke}%
\end{pgfscope}%
\begin{pgfscope}%
\pgfpathrectangle{\pgfqpoint{0.100000in}{0.212622in}}{\pgfqpoint{3.696000in}{3.696000in}}%
\pgfusepath{clip}%
\pgfsetrectcap%
\pgfsetroundjoin%
\pgfsetlinewidth{1.505625pt}%
\definecolor{currentstroke}{rgb}{1.000000,0.000000,0.000000}%
\pgfsetstrokecolor{currentstroke}%
\pgfsetdash{}{0pt}%
\pgfpathmoveto{\pgfqpoint{0.724348in}{1.254292in}}%
\pgfpathlineto{\pgfqpoint{0.916842in}{1.291711in}}%
\pgfusepath{stroke}%
\end{pgfscope}%
\begin{pgfscope}%
\pgfpathrectangle{\pgfqpoint{0.100000in}{0.212622in}}{\pgfqpoint{3.696000in}{3.696000in}}%
\pgfusepath{clip}%
\pgfsetrectcap%
\pgfsetroundjoin%
\pgfsetlinewidth{1.505625pt}%
\definecolor{currentstroke}{rgb}{1.000000,0.000000,0.000000}%
\pgfsetstrokecolor{currentstroke}%
\pgfsetdash{}{0pt}%
\pgfpathmoveto{\pgfqpoint{0.725009in}{1.254856in}}%
\pgfpathlineto{\pgfqpoint{0.916842in}{1.291711in}}%
\pgfusepath{stroke}%
\end{pgfscope}%
\begin{pgfscope}%
\pgfpathrectangle{\pgfqpoint{0.100000in}{0.212622in}}{\pgfqpoint{3.696000in}{3.696000in}}%
\pgfusepath{clip}%
\pgfsetrectcap%
\pgfsetroundjoin%
\pgfsetlinewidth{1.505625pt}%
\definecolor{currentstroke}{rgb}{1.000000,0.000000,0.000000}%
\pgfsetstrokecolor{currentstroke}%
\pgfsetdash{}{0pt}%
\pgfpathmoveto{\pgfqpoint{0.725401in}{1.255192in}}%
\pgfpathlineto{\pgfqpoint{0.916842in}{1.291711in}}%
\pgfusepath{stroke}%
\end{pgfscope}%
\begin{pgfscope}%
\pgfpathrectangle{\pgfqpoint{0.100000in}{0.212622in}}{\pgfqpoint{3.696000in}{3.696000in}}%
\pgfusepath{clip}%
\pgfsetrectcap%
\pgfsetroundjoin%
\pgfsetlinewidth{1.505625pt}%
\definecolor{currentstroke}{rgb}{1.000000,0.000000,0.000000}%
\pgfsetstrokecolor{currentstroke}%
\pgfsetdash{}{0pt}%
\pgfpathmoveto{\pgfqpoint{0.725587in}{1.255390in}}%
\pgfpathlineto{\pgfqpoint{0.916842in}{1.291711in}}%
\pgfusepath{stroke}%
\end{pgfscope}%
\begin{pgfscope}%
\pgfpathrectangle{\pgfqpoint{0.100000in}{0.212622in}}{\pgfqpoint{3.696000in}{3.696000in}}%
\pgfusepath{clip}%
\pgfsetrectcap%
\pgfsetroundjoin%
\pgfsetlinewidth{1.505625pt}%
\definecolor{currentstroke}{rgb}{1.000000,0.000000,0.000000}%
\pgfsetstrokecolor{currentstroke}%
\pgfsetdash{}{0pt}%
\pgfpathmoveto{\pgfqpoint{0.725697in}{1.255497in}}%
\pgfpathlineto{\pgfqpoint{0.916842in}{1.291711in}}%
\pgfusepath{stroke}%
\end{pgfscope}%
\begin{pgfscope}%
\pgfpathrectangle{\pgfqpoint{0.100000in}{0.212622in}}{\pgfqpoint{3.696000in}{3.696000in}}%
\pgfusepath{clip}%
\pgfsetrectcap%
\pgfsetroundjoin%
\pgfsetlinewidth{1.505625pt}%
\definecolor{currentstroke}{rgb}{1.000000,0.000000,0.000000}%
\pgfsetstrokecolor{currentstroke}%
\pgfsetdash{}{0pt}%
\pgfpathmoveto{\pgfqpoint{0.725753in}{1.255558in}}%
\pgfpathlineto{\pgfqpoint{0.916842in}{1.291711in}}%
\pgfusepath{stroke}%
\end{pgfscope}%
\begin{pgfscope}%
\pgfpathrectangle{\pgfqpoint{0.100000in}{0.212622in}}{\pgfqpoint{3.696000in}{3.696000in}}%
\pgfusepath{clip}%
\pgfsetrectcap%
\pgfsetroundjoin%
\pgfsetlinewidth{1.505625pt}%
\definecolor{currentstroke}{rgb}{1.000000,0.000000,0.000000}%
\pgfsetstrokecolor{currentstroke}%
\pgfsetdash{}{0pt}%
\pgfpathmoveto{\pgfqpoint{0.725783in}{1.255597in}}%
\pgfpathlineto{\pgfqpoint{0.916842in}{1.291711in}}%
\pgfusepath{stroke}%
\end{pgfscope}%
\begin{pgfscope}%
\pgfpathrectangle{\pgfqpoint{0.100000in}{0.212622in}}{\pgfqpoint{3.696000in}{3.696000in}}%
\pgfusepath{clip}%
\pgfsetrectcap%
\pgfsetroundjoin%
\pgfsetlinewidth{1.505625pt}%
\definecolor{currentstroke}{rgb}{1.000000,0.000000,0.000000}%
\pgfsetstrokecolor{currentstroke}%
\pgfsetdash{}{0pt}%
\pgfpathmoveto{\pgfqpoint{0.725803in}{1.255613in}}%
\pgfpathlineto{\pgfqpoint{0.916842in}{1.291711in}}%
\pgfusepath{stroke}%
\end{pgfscope}%
\begin{pgfscope}%
\pgfpathrectangle{\pgfqpoint{0.100000in}{0.212622in}}{\pgfqpoint{3.696000in}{3.696000in}}%
\pgfusepath{clip}%
\pgfsetrectcap%
\pgfsetroundjoin%
\pgfsetlinewidth{1.505625pt}%
\definecolor{currentstroke}{rgb}{1.000000,0.000000,0.000000}%
\pgfsetstrokecolor{currentstroke}%
\pgfsetdash{}{0pt}%
\pgfpathmoveto{\pgfqpoint{0.725810in}{1.255623in}}%
\pgfpathlineto{\pgfqpoint{0.916842in}{1.291711in}}%
\pgfusepath{stroke}%
\end{pgfscope}%
\begin{pgfscope}%
\pgfpathrectangle{\pgfqpoint{0.100000in}{0.212622in}}{\pgfqpoint{3.696000in}{3.696000in}}%
\pgfusepath{clip}%
\pgfsetrectcap%
\pgfsetroundjoin%
\pgfsetlinewidth{1.505625pt}%
\definecolor{currentstroke}{rgb}{1.000000,0.000000,0.000000}%
\pgfsetstrokecolor{currentstroke}%
\pgfsetdash{}{0pt}%
\pgfpathmoveto{\pgfqpoint{0.731305in}{1.259461in}}%
\pgfpathlineto{\pgfqpoint{0.916842in}{1.291711in}}%
\pgfusepath{stroke}%
\end{pgfscope}%
\begin{pgfscope}%
\pgfpathrectangle{\pgfqpoint{0.100000in}{0.212622in}}{\pgfqpoint{3.696000in}{3.696000in}}%
\pgfusepath{clip}%
\pgfsetrectcap%
\pgfsetroundjoin%
\pgfsetlinewidth{1.505625pt}%
\definecolor{currentstroke}{rgb}{1.000000,0.000000,0.000000}%
\pgfsetstrokecolor{currentstroke}%
\pgfsetdash{}{0pt}%
\pgfpathmoveto{\pgfqpoint{0.733659in}{1.261909in}}%
\pgfpathlineto{\pgfqpoint{0.916842in}{1.291711in}}%
\pgfusepath{stroke}%
\end{pgfscope}%
\begin{pgfscope}%
\pgfpathrectangle{\pgfqpoint{0.100000in}{0.212622in}}{\pgfqpoint{3.696000in}{3.696000in}}%
\pgfusepath{clip}%
\pgfsetrectcap%
\pgfsetroundjoin%
\pgfsetlinewidth{1.505625pt}%
\definecolor{currentstroke}{rgb}{1.000000,0.000000,0.000000}%
\pgfsetstrokecolor{currentstroke}%
\pgfsetdash{}{0pt}%
\pgfpathmoveto{\pgfqpoint{0.737769in}{1.267531in}}%
\pgfpathlineto{\pgfqpoint{0.916842in}{1.291711in}}%
\pgfusepath{stroke}%
\end{pgfscope}%
\begin{pgfscope}%
\pgfpathrectangle{\pgfqpoint{0.100000in}{0.212622in}}{\pgfqpoint{3.696000in}{3.696000in}}%
\pgfusepath{clip}%
\pgfsetrectcap%
\pgfsetroundjoin%
\pgfsetlinewidth{1.505625pt}%
\definecolor{currentstroke}{rgb}{1.000000,0.000000,0.000000}%
\pgfsetstrokecolor{currentstroke}%
\pgfsetdash{}{0pt}%
\pgfpathmoveto{\pgfqpoint{0.751127in}{1.275524in}}%
\pgfpathlineto{\pgfqpoint{0.916842in}{1.291711in}}%
\pgfusepath{stroke}%
\end{pgfscope}%
\begin{pgfscope}%
\pgfpathrectangle{\pgfqpoint{0.100000in}{0.212622in}}{\pgfqpoint{3.696000in}{3.696000in}}%
\pgfusepath{clip}%
\pgfsetrectcap%
\pgfsetroundjoin%
\pgfsetlinewidth{1.505625pt}%
\definecolor{currentstroke}{rgb}{1.000000,0.000000,0.000000}%
\pgfsetstrokecolor{currentstroke}%
\pgfsetdash{}{0pt}%
\pgfpathmoveto{\pgfqpoint{0.762702in}{1.287923in}}%
\pgfpathlineto{\pgfqpoint{0.916842in}{1.291711in}}%
\pgfusepath{stroke}%
\end{pgfscope}%
\begin{pgfscope}%
\pgfpathrectangle{\pgfqpoint{0.100000in}{0.212622in}}{\pgfqpoint{3.696000in}{3.696000in}}%
\pgfusepath{clip}%
\pgfsetrectcap%
\pgfsetroundjoin%
\pgfsetlinewidth{1.505625pt}%
\definecolor{currentstroke}{rgb}{1.000000,0.000000,0.000000}%
\pgfsetstrokecolor{currentstroke}%
\pgfsetdash{}{0pt}%
\pgfpathmoveto{\pgfqpoint{0.786323in}{1.300686in}}%
\pgfpathlineto{\pgfqpoint{0.916842in}{1.291711in}}%
\pgfusepath{stroke}%
\end{pgfscope}%
\begin{pgfscope}%
\pgfpathrectangle{\pgfqpoint{0.100000in}{0.212622in}}{\pgfqpoint{3.696000in}{3.696000in}}%
\pgfusepath{clip}%
\pgfsetrectcap%
\pgfsetroundjoin%
\pgfsetlinewidth{1.505625pt}%
\definecolor{currentstroke}{rgb}{1.000000,0.000000,0.000000}%
\pgfsetstrokecolor{currentstroke}%
\pgfsetdash{}{0pt}%
\pgfpathmoveto{\pgfqpoint{0.793122in}{1.311976in}}%
\pgfpathlineto{\pgfqpoint{0.916842in}{1.291711in}}%
\pgfusepath{stroke}%
\end{pgfscope}%
\begin{pgfscope}%
\pgfpathrectangle{\pgfqpoint{0.100000in}{0.212622in}}{\pgfqpoint{3.696000in}{3.696000in}}%
\pgfusepath{clip}%
\pgfsetrectcap%
\pgfsetroundjoin%
\pgfsetlinewidth{1.505625pt}%
\definecolor{currentstroke}{rgb}{1.000000,0.000000,0.000000}%
\pgfsetstrokecolor{currentstroke}%
\pgfsetdash{}{0pt}%
\pgfpathmoveto{\pgfqpoint{0.799605in}{1.318223in}}%
\pgfpathlineto{\pgfqpoint{0.916842in}{1.291711in}}%
\pgfusepath{stroke}%
\end{pgfscope}%
\begin{pgfscope}%
\pgfpathrectangle{\pgfqpoint{0.100000in}{0.212622in}}{\pgfqpoint{3.696000in}{3.696000in}}%
\pgfusepath{clip}%
\pgfsetrectcap%
\pgfsetroundjoin%
\pgfsetlinewidth{1.505625pt}%
\definecolor{currentstroke}{rgb}{1.000000,0.000000,0.000000}%
\pgfsetstrokecolor{currentstroke}%
\pgfsetdash{}{0pt}%
\pgfpathmoveto{\pgfqpoint{0.802360in}{1.321315in}}%
\pgfpathlineto{\pgfqpoint{0.916842in}{1.291711in}}%
\pgfusepath{stroke}%
\end{pgfscope}%
\begin{pgfscope}%
\pgfpathrectangle{\pgfqpoint{0.100000in}{0.212622in}}{\pgfqpoint{3.696000in}{3.696000in}}%
\pgfusepath{clip}%
\pgfsetrectcap%
\pgfsetroundjoin%
\pgfsetlinewidth{1.505625pt}%
\definecolor{currentstroke}{rgb}{1.000000,0.000000,0.000000}%
\pgfsetstrokecolor{currentstroke}%
\pgfsetdash{}{0pt}%
\pgfpathmoveto{\pgfqpoint{0.807483in}{1.330105in}}%
\pgfpathlineto{\pgfqpoint{0.916842in}{1.291711in}}%
\pgfusepath{stroke}%
\end{pgfscope}%
\begin{pgfscope}%
\pgfpathrectangle{\pgfqpoint{0.100000in}{0.212622in}}{\pgfqpoint{3.696000in}{3.696000in}}%
\pgfusepath{clip}%
\pgfsetrectcap%
\pgfsetroundjoin%
\pgfsetlinewidth{1.505625pt}%
\definecolor{currentstroke}{rgb}{1.000000,0.000000,0.000000}%
\pgfsetstrokecolor{currentstroke}%
\pgfsetdash{}{0pt}%
\pgfpathmoveto{\pgfqpoint{0.824991in}{1.335319in}}%
\pgfpathlineto{\pgfqpoint{0.916842in}{1.291711in}}%
\pgfusepath{stroke}%
\end{pgfscope}%
\begin{pgfscope}%
\pgfpathrectangle{\pgfqpoint{0.100000in}{0.212622in}}{\pgfqpoint{3.696000in}{3.696000in}}%
\pgfusepath{clip}%
\pgfsetrectcap%
\pgfsetroundjoin%
\pgfsetlinewidth{1.505625pt}%
\definecolor{currentstroke}{rgb}{1.000000,0.000000,0.000000}%
\pgfsetstrokecolor{currentstroke}%
\pgfsetdash{}{0pt}%
\pgfpathmoveto{\pgfqpoint{0.837149in}{1.350082in}}%
\pgfpathlineto{\pgfqpoint{0.916842in}{1.291711in}}%
\pgfusepath{stroke}%
\end{pgfscope}%
\begin{pgfscope}%
\pgfpathrectangle{\pgfqpoint{0.100000in}{0.212622in}}{\pgfqpoint{3.696000in}{3.696000in}}%
\pgfusepath{clip}%
\pgfsetrectcap%
\pgfsetroundjoin%
\pgfsetlinewidth{1.505625pt}%
\definecolor{currentstroke}{rgb}{1.000000,0.000000,0.000000}%
\pgfsetstrokecolor{currentstroke}%
\pgfsetdash{}{0pt}%
\pgfpathmoveto{\pgfqpoint{0.865979in}{1.365128in}}%
\pgfpathlineto{\pgfqpoint{0.935729in}{1.307914in}}%
\pgfusepath{stroke}%
\end{pgfscope}%
\begin{pgfscope}%
\pgfpathrectangle{\pgfqpoint{0.100000in}{0.212622in}}{\pgfqpoint{3.696000in}{3.696000in}}%
\pgfusepath{clip}%
\pgfsetrectcap%
\pgfsetroundjoin%
\pgfsetlinewidth{1.505625pt}%
\definecolor{currentstroke}{rgb}{1.000000,0.000000,0.000000}%
\pgfsetstrokecolor{currentstroke}%
\pgfsetdash{}{0pt}%
\pgfpathmoveto{\pgfqpoint{0.886413in}{1.393411in}}%
\pgfpathlineto{\pgfqpoint{0.963966in}{1.332138in}}%
\pgfusepath{stroke}%
\end{pgfscope}%
\begin{pgfscope}%
\pgfpathrectangle{\pgfqpoint{0.100000in}{0.212622in}}{\pgfqpoint{3.696000in}{3.696000in}}%
\pgfusepath{clip}%
\pgfsetrectcap%
\pgfsetroundjoin%
\pgfsetlinewidth{1.505625pt}%
\definecolor{currentstroke}{rgb}{1.000000,0.000000,0.000000}%
\pgfsetstrokecolor{currentstroke}%
\pgfsetdash{}{0pt}%
\pgfpathmoveto{\pgfqpoint{0.904840in}{1.406328in}}%
\pgfpathlineto{\pgfqpoint{0.973354in}{1.340191in}}%
\pgfusepath{stroke}%
\end{pgfscope}%
\begin{pgfscope}%
\pgfpathrectangle{\pgfqpoint{0.100000in}{0.212622in}}{\pgfqpoint{3.696000in}{3.696000in}}%
\pgfusepath{clip}%
\pgfsetrectcap%
\pgfsetroundjoin%
\pgfsetlinewidth{1.505625pt}%
\definecolor{currentstroke}{rgb}{1.000000,0.000000,0.000000}%
\pgfsetstrokecolor{currentstroke}%
\pgfsetdash{}{0pt}%
\pgfpathmoveto{\pgfqpoint{0.910448in}{1.413626in}}%
\pgfpathlineto{\pgfqpoint{0.982730in}{1.348234in}}%
\pgfusepath{stroke}%
\end{pgfscope}%
\begin{pgfscope}%
\pgfpathrectangle{\pgfqpoint{0.100000in}{0.212622in}}{\pgfqpoint{3.696000in}{3.696000in}}%
\pgfusepath{clip}%
\pgfsetrectcap%
\pgfsetroundjoin%
\pgfsetlinewidth{1.505625pt}%
\definecolor{currentstroke}{rgb}{1.000000,0.000000,0.000000}%
\pgfsetstrokecolor{currentstroke}%
\pgfsetdash{}{0pt}%
\pgfpathmoveto{\pgfqpoint{0.914399in}{1.419857in}}%
\pgfpathlineto{\pgfqpoint{0.992093in}{1.356267in}}%
\pgfusepath{stroke}%
\end{pgfscope}%
\begin{pgfscope}%
\pgfpathrectangle{\pgfqpoint{0.100000in}{0.212622in}}{\pgfqpoint{3.696000in}{3.696000in}}%
\pgfusepath{clip}%
\pgfsetrectcap%
\pgfsetroundjoin%
\pgfsetlinewidth{1.505625pt}%
\definecolor{currentstroke}{rgb}{1.000000,0.000000,0.000000}%
\pgfsetstrokecolor{currentstroke}%
\pgfsetdash{}{0pt}%
\pgfpathmoveto{\pgfqpoint{0.929392in}{1.428418in}}%
\pgfpathlineto{\pgfqpoint{1.001444in}{1.364289in}}%
\pgfusepath{stroke}%
\end{pgfscope}%
\begin{pgfscope}%
\pgfpathrectangle{\pgfqpoint{0.100000in}{0.212622in}}{\pgfqpoint{3.696000in}{3.696000in}}%
\pgfusepath{clip}%
\pgfsetrectcap%
\pgfsetroundjoin%
\pgfsetlinewidth{1.505625pt}%
\definecolor{currentstroke}{rgb}{1.000000,0.000000,0.000000}%
\pgfsetstrokecolor{currentstroke}%
\pgfsetdash{}{0pt}%
\pgfpathmoveto{\pgfqpoint{0.937391in}{1.440413in}}%
\pgfpathlineto{\pgfqpoint{1.010783in}{1.372301in}}%
\pgfusepath{stroke}%
\end{pgfscope}%
\begin{pgfscope}%
\pgfpathrectangle{\pgfqpoint{0.100000in}{0.212622in}}{\pgfqpoint{3.696000in}{3.696000in}}%
\pgfusepath{clip}%
\pgfsetrectcap%
\pgfsetroundjoin%
\pgfsetlinewidth{1.505625pt}%
\definecolor{currentstroke}{rgb}{1.000000,0.000000,0.000000}%
\pgfsetstrokecolor{currentstroke}%
\pgfsetdash{}{0pt}%
\pgfpathmoveto{\pgfqpoint{0.957619in}{1.457192in}}%
\pgfpathlineto{\pgfqpoint{1.029424in}{1.388292in}}%
\pgfusepath{stroke}%
\end{pgfscope}%
\begin{pgfscope}%
\pgfpathrectangle{\pgfqpoint{0.100000in}{0.212622in}}{\pgfqpoint{3.696000in}{3.696000in}}%
\pgfusepath{clip}%
\pgfsetrectcap%
\pgfsetroundjoin%
\pgfsetlinewidth{1.505625pt}%
\definecolor{currentstroke}{rgb}{1.000000,0.000000,0.000000}%
\pgfsetstrokecolor{currentstroke}%
\pgfsetdash{}{0pt}%
\pgfpathmoveto{\pgfqpoint{0.965894in}{1.471065in}}%
\pgfpathlineto{\pgfqpoint{1.038727in}{1.396273in}}%
\pgfusepath{stroke}%
\end{pgfscope}%
\begin{pgfscope}%
\pgfpathrectangle{\pgfqpoint{0.100000in}{0.212622in}}{\pgfqpoint{3.696000in}{3.696000in}}%
\pgfusepath{clip}%
\pgfsetrectcap%
\pgfsetroundjoin%
\pgfsetlinewidth{1.505625pt}%
\definecolor{currentstroke}{rgb}{1.000000,0.000000,0.000000}%
\pgfsetstrokecolor{currentstroke}%
\pgfsetdash{}{0pt}%
\pgfpathmoveto{\pgfqpoint{0.971437in}{1.477777in}}%
\pgfpathlineto{\pgfqpoint{1.048017in}{1.404243in}}%
\pgfusepath{stroke}%
\end{pgfscope}%
\begin{pgfscope}%
\pgfpathrectangle{\pgfqpoint{0.100000in}{0.212622in}}{\pgfqpoint{3.696000in}{3.696000in}}%
\pgfusepath{clip}%
\pgfsetrectcap%
\pgfsetroundjoin%
\pgfsetlinewidth{1.505625pt}%
\definecolor{currentstroke}{rgb}{1.000000,0.000000,0.000000}%
\pgfsetstrokecolor{currentstroke}%
\pgfsetdash{}{0pt}%
\pgfpathmoveto{\pgfqpoint{0.974384in}{1.481275in}}%
\pgfpathlineto{\pgfqpoint{1.048017in}{1.404243in}}%
\pgfusepath{stroke}%
\end{pgfscope}%
\begin{pgfscope}%
\pgfpathrectangle{\pgfqpoint{0.100000in}{0.212622in}}{\pgfqpoint{3.696000in}{3.696000in}}%
\pgfusepath{clip}%
\pgfsetrectcap%
\pgfsetroundjoin%
\pgfsetlinewidth{1.505625pt}%
\definecolor{currentstroke}{rgb}{1.000000,0.000000,0.000000}%
\pgfsetstrokecolor{currentstroke}%
\pgfsetdash{}{0pt}%
\pgfpathmoveto{\pgfqpoint{0.979176in}{1.488163in}}%
\pgfpathlineto{\pgfqpoint{1.057295in}{1.412202in}}%
\pgfusepath{stroke}%
\end{pgfscope}%
\begin{pgfscope}%
\pgfpathrectangle{\pgfqpoint{0.100000in}{0.212622in}}{\pgfqpoint{3.696000in}{3.696000in}}%
\pgfusepath{clip}%
\pgfsetrectcap%
\pgfsetroundjoin%
\pgfsetlinewidth{1.505625pt}%
\definecolor{currentstroke}{rgb}{1.000000,0.000000,0.000000}%
\pgfsetstrokecolor{currentstroke}%
\pgfsetdash{}{0pt}%
\pgfpathmoveto{\pgfqpoint{0.994537in}{1.496652in}}%
\pgfpathlineto{\pgfqpoint{1.066561in}{1.420151in}}%
\pgfusepath{stroke}%
\end{pgfscope}%
\begin{pgfscope}%
\pgfpathrectangle{\pgfqpoint{0.100000in}{0.212622in}}{\pgfqpoint{3.696000in}{3.696000in}}%
\pgfusepath{clip}%
\pgfsetrectcap%
\pgfsetroundjoin%
\pgfsetlinewidth{1.505625pt}%
\definecolor{currentstroke}{rgb}{1.000000,0.000000,0.000000}%
\pgfsetstrokecolor{currentstroke}%
\pgfsetdash{}{0pt}%
\pgfpathmoveto{\pgfqpoint{1.006223in}{1.514565in}}%
\pgfpathlineto{\pgfqpoint{1.085058in}{1.436019in}}%
\pgfusepath{stroke}%
\end{pgfscope}%
\begin{pgfscope}%
\pgfpathrectangle{\pgfqpoint{0.100000in}{0.212622in}}{\pgfqpoint{3.696000in}{3.696000in}}%
\pgfusepath{clip}%
\pgfsetrectcap%
\pgfsetroundjoin%
\pgfsetlinewidth{1.505625pt}%
\definecolor{currentstroke}{rgb}{1.000000,0.000000,0.000000}%
\pgfsetstrokecolor{currentstroke}%
\pgfsetdash{}{0pt}%
\pgfpathmoveto{\pgfqpoint{1.031730in}{1.533504in}}%
\pgfpathlineto{\pgfqpoint{1.112712in}{1.459743in}}%
\pgfusepath{stroke}%
\end{pgfscope}%
\begin{pgfscope}%
\pgfpathrectangle{\pgfqpoint{0.100000in}{0.212622in}}{\pgfqpoint{3.696000in}{3.696000in}}%
\pgfusepath{clip}%
\pgfsetrectcap%
\pgfsetroundjoin%
\pgfsetlinewidth{1.505625pt}%
\definecolor{currentstroke}{rgb}{1.000000,0.000000,0.000000}%
\pgfsetstrokecolor{currentstroke}%
\pgfsetdash{}{0pt}%
\pgfpathmoveto{\pgfqpoint{1.050409in}{1.563542in}}%
\pgfpathlineto{\pgfqpoint{1.140260in}{1.483375in}}%
\pgfusepath{stroke}%
\end{pgfscope}%
\begin{pgfscope}%
\pgfpathrectangle{\pgfqpoint{0.100000in}{0.212622in}}{\pgfqpoint{3.696000in}{3.696000in}}%
\pgfusepath{clip}%
\pgfsetrectcap%
\pgfsetroundjoin%
\pgfsetlinewidth{1.505625pt}%
\definecolor{currentstroke}{rgb}{1.000000,0.000000,0.000000}%
\pgfsetstrokecolor{currentstroke}%
\pgfsetdash{}{0pt}%
\pgfpathmoveto{\pgfqpoint{1.072678in}{1.597130in}}%
\pgfpathlineto{\pgfqpoint{1.167700in}{1.506916in}}%
\pgfusepath{stroke}%
\end{pgfscope}%
\begin{pgfscope}%
\pgfpathrectangle{\pgfqpoint{0.100000in}{0.212622in}}{\pgfqpoint{3.696000in}{3.696000in}}%
\pgfusepath{clip}%
\pgfsetrectcap%
\pgfsetroundjoin%
\pgfsetlinewidth{1.505625pt}%
\definecolor{currentstroke}{rgb}{1.000000,0.000000,0.000000}%
\pgfsetstrokecolor{currentstroke}%
\pgfsetdash{}{0pt}%
\pgfpathmoveto{\pgfqpoint{1.099420in}{1.643610in}}%
\pgfpathlineto{\pgfqpoint{1.195035in}{1.530365in}}%
\pgfusepath{stroke}%
\end{pgfscope}%
\begin{pgfscope}%
\pgfpathrectangle{\pgfqpoint{0.100000in}{0.212622in}}{\pgfqpoint{3.696000in}{3.696000in}}%
\pgfusepath{clip}%
\pgfsetrectcap%
\pgfsetroundjoin%
\pgfsetlinewidth{1.505625pt}%
\definecolor{currentstroke}{rgb}{1.000000,0.000000,0.000000}%
\pgfsetstrokecolor{currentstroke}%
\pgfsetdash{}{0pt}%
\pgfpathmoveto{\pgfqpoint{1.117823in}{1.661595in}}%
\pgfpathlineto{\pgfqpoint{1.213199in}{1.545948in}}%
\pgfusepath{stroke}%
\end{pgfscope}%
\begin{pgfscope}%
\pgfpathrectangle{\pgfqpoint{0.100000in}{0.212622in}}{\pgfqpoint{3.696000in}{3.696000in}}%
\pgfusepath{clip}%
\pgfsetrectcap%
\pgfsetroundjoin%
\pgfsetlinewidth{1.505625pt}%
\definecolor{currentstroke}{rgb}{1.000000,0.000000,0.000000}%
\pgfsetstrokecolor{currentstroke}%
\pgfsetdash{}{0pt}%
\pgfpathmoveto{\pgfqpoint{1.125702in}{1.676269in}}%
\pgfpathlineto{\pgfqpoint{1.231317in}{1.561491in}}%
\pgfusepath{stroke}%
\end{pgfscope}%
\begin{pgfscope}%
\pgfpathrectangle{\pgfqpoint{0.100000in}{0.212622in}}{\pgfqpoint{3.696000in}{3.696000in}}%
\pgfusepath{clip}%
\pgfsetrectcap%
\pgfsetroundjoin%
\pgfsetlinewidth{1.505625pt}%
\definecolor{currentstroke}{rgb}{1.000000,0.000000,0.000000}%
\pgfsetstrokecolor{currentstroke}%
\pgfsetdash{}{0pt}%
\pgfpathmoveto{\pgfqpoint{1.130400in}{1.683256in}}%
\pgfpathlineto{\pgfqpoint{1.231317in}{1.561491in}}%
\pgfusepath{stroke}%
\end{pgfscope}%
\begin{pgfscope}%
\pgfpathrectangle{\pgfqpoint{0.100000in}{0.212622in}}{\pgfqpoint{3.696000in}{3.696000in}}%
\pgfusepath{clip}%
\pgfsetrectcap%
\pgfsetroundjoin%
\pgfsetlinewidth{1.505625pt}%
\definecolor{currentstroke}{rgb}{1.000000,0.000000,0.000000}%
\pgfsetstrokecolor{currentstroke}%
\pgfsetdash{}{0pt}%
\pgfpathmoveto{\pgfqpoint{1.133108in}{1.686880in}}%
\pgfpathlineto{\pgfqpoint{1.240359in}{1.569247in}}%
\pgfusepath{stroke}%
\end{pgfscope}%
\begin{pgfscope}%
\pgfpathrectangle{\pgfqpoint{0.100000in}{0.212622in}}{\pgfqpoint{3.696000in}{3.696000in}}%
\pgfusepath{clip}%
\pgfsetrectcap%
\pgfsetroundjoin%
\pgfsetlinewidth{1.505625pt}%
\definecolor{currentstroke}{rgb}{1.000000,0.000000,0.000000}%
\pgfsetstrokecolor{currentstroke}%
\pgfsetdash{}{0pt}%
\pgfpathmoveto{\pgfqpoint{1.134404in}{1.689193in}}%
\pgfpathlineto{\pgfqpoint{1.240359in}{1.569247in}}%
\pgfusepath{stroke}%
\end{pgfscope}%
\begin{pgfscope}%
\pgfpathrectangle{\pgfqpoint{0.100000in}{0.212622in}}{\pgfqpoint{3.696000in}{3.696000in}}%
\pgfusepath{clip}%
\pgfsetrectcap%
\pgfsetroundjoin%
\pgfsetlinewidth{1.505625pt}%
\definecolor{currentstroke}{rgb}{1.000000,0.000000,0.000000}%
\pgfsetstrokecolor{currentstroke}%
\pgfsetdash{}{0pt}%
\pgfpathmoveto{\pgfqpoint{1.138963in}{1.693179in}}%
\pgfpathlineto{\pgfqpoint{1.240359in}{1.569247in}}%
\pgfusepath{stroke}%
\end{pgfscope}%
\begin{pgfscope}%
\pgfpathrectangle{\pgfqpoint{0.100000in}{0.212622in}}{\pgfqpoint{3.696000in}{3.696000in}}%
\pgfusepath{clip}%
\pgfsetrectcap%
\pgfsetroundjoin%
\pgfsetlinewidth{1.505625pt}%
\definecolor{currentstroke}{rgb}{1.000000,0.000000,0.000000}%
\pgfsetstrokecolor{currentstroke}%
\pgfsetdash{}{0pt}%
\pgfpathmoveto{\pgfqpoint{1.146426in}{1.705269in}}%
\pgfpathlineto{\pgfqpoint{1.249388in}{1.576994in}}%
\pgfusepath{stroke}%
\end{pgfscope}%
\begin{pgfscope}%
\pgfpathrectangle{\pgfqpoint{0.100000in}{0.212622in}}{\pgfqpoint{3.696000in}{3.696000in}}%
\pgfusepath{clip}%
\pgfsetrectcap%
\pgfsetroundjoin%
\pgfsetlinewidth{1.505625pt}%
\definecolor{currentstroke}{rgb}{1.000000,0.000000,0.000000}%
\pgfsetstrokecolor{currentstroke}%
\pgfsetdash{}{0pt}%
\pgfpathmoveto{\pgfqpoint{1.159444in}{1.723285in}}%
\pgfpathlineto{\pgfqpoint{1.267414in}{1.592457in}}%
\pgfusepath{stroke}%
\end{pgfscope}%
\begin{pgfscope}%
\pgfpathrectangle{\pgfqpoint{0.100000in}{0.212622in}}{\pgfqpoint{3.696000in}{3.696000in}}%
\pgfusepath{clip}%
\pgfsetrectcap%
\pgfsetroundjoin%
\pgfsetlinewidth{1.505625pt}%
\definecolor{currentstroke}{rgb}{1.000000,0.000000,0.000000}%
\pgfsetstrokecolor{currentstroke}%
\pgfsetdash{}{0pt}%
\pgfpathmoveto{\pgfqpoint{1.170038in}{1.744949in}}%
\pgfpathlineto{\pgfqpoint{1.285392in}{1.607880in}}%
\pgfusepath{stroke}%
\end{pgfscope}%
\begin{pgfscope}%
\pgfpathrectangle{\pgfqpoint{0.100000in}{0.212622in}}{\pgfqpoint{3.696000in}{3.696000in}}%
\pgfusepath{clip}%
\pgfsetrectcap%
\pgfsetroundjoin%
\pgfsetlinewidth{1.505625pt}%
\definecolor{currentstroke}{rgb}{1.000000,0.000000,0.000000}%
\pgfsetstrokecolor{currentstroke}%
\pgfsetdash{}{0pt}%
\pgfpathmoveto{\pgfqpoint{1.194157in}{1.766083in}}%
\pgfpathlineto{\pgfqpoint{1.312275in}{1.630942in}}%
\pgfusepath{stroke}%
\end{pgfscope}%
\begin{pgfscope}%
\pgfpathrectangle{\pgfqpoint{0.100000in}{0.212622in}}{\pgfqpoint{3.696000in}{3.696000in}}%
\pgfusepath{clip}%
\pgfsetrectcap%
\pgfsetroundjoin%
\pgfsetlinewidth{1.505625pt}%
\definecolor{currentstroke}{rgb}{1.000000,0.000000,0.000000}%
\pgfsetstrokecolor{currentstroke}%
\pgfsetdash{}{0pt}%
\pgfpathmoveto{\pgfqpoint{1.205755in}{1.783218in}}%
\pgfpathlineto{\pgfqpoint{1.321212in}{1.638610in}}%
\pgfusepath{stroke}%
\end{pgfscope}%
\begin{pgfscope}%
\pgfpathrectangle{\pgfqpoint{0.100000in}{0.212622in}}{\pgfqpoint{3.696000in}{3.696000in}}%
\pgfusepath{clip}%
\pgfsetrectcap%
\pgfsetroundjoin%
\pgfsetlinewidth{1.505625pt}%
\definecolor{currentstroke}{rgb}{1.000000,0.000000,0.000000}%
\pgfsetstrokecolor{currentstroke}%
\pgfsetdash{}{0pt}%
\pgfpathmoveto{\pgfqpoint{1.211774in}{1.792157in}}%
\pgfpathlineto{\pgfqpoint{1.330139in}{1.646267in}}%
\pgfusepath{stroke}%
\end{pgfscope}%
\begin{pgfscope}%
\pgfpathrectangle{\pgfqpoint{0.100000in}{0.212622in}}{\pgfqpoint{3.696000in}{3.696000in}}%
\pgfusepath{clip}%
\pgfsetrectcap%
\pgfsetroundjoin%
\pgfsetlinewidth{1.505625pt}%
\definecolor{currentstroke}{rgb}{1.000000,0.000000,0.000000}%
\pgfsetstrokecolor{currentstroke}%
\pgfsetdash{}{0pt}%
\pgfpathmoveto{\pgfqpoint{1.214570in}{1.796730in}}%
\pgfpathlineto{\pgfqpoint{1.330139in}{1.646267in}}%
\pgfusepath{stroke}%
\end{pgfscope}%
\begin{pgfscope}%
\pgfpathrectangle{\pgfqpoint{0.100000in}{0.212622in}}{\pgfqpoint{3.696000in}{3.696000in}}%
\pgfusepath{clip}%
\pgfsetrectcap%
\pgfsetroundjoin%
\pgfsetlinewidth{1.505625pt}%
\definecolor{currentstroke}{rgb}{1.000000,0.000000,0.000000}%
\pgfsetstrokecolor{currentstroke}%
\pgfsetdash{}{0pt}%
\pgfpathmoveto{\pgfqpoint{1.216007in}{1.799773in}}%
\pgfpathlineto{\pgfqpoint{1.330139in}{1.646267in}}%
\pgfusepath{stroke}%
\end{pgfscope}%
\begin{pgfscope}%
\pgfpathrectangle{\pgfqpoint{0.100000in}{0.212622in}}{\pgfqpoint{3.696000in}{3.696000in}}%
\pgfusepath{clip}%
\pgfsetrectcap%
\pgfsetroundjoin%
\pgfsetlinewidth{1.505625pt}%
\definecolor{currentstroke}{rgb}{1.000000,0.000000,0.000000}%
\pgfsetstrokecolor{currentstroke}%
\pgfsetdash{}{0pt}%
\pgfpathmoveto{\pgfqpoint{1.217369in}{1.800860in}}%
\pgfpathlineto{\pgfqpoint{1.339054in}{1.653915in}}%
\pgfusepath{stroke}%
\end{pgfscope}%
\begin{pgfscope}%
\pgfpathrectangle{\pgfqpoint{0.100000in}{0.212622in}}{\pgfqpoint{3.696000in}{3.696000in}}%
\pgfusepath{clip}%
\pgfsetrectcap%
\pgfsetroundjoin%
\pgfsetlinewidth{1.505625pt}%
\definecolor{currentstroke}{rgb}{1.000000,0.000000,0.000000}%
\pgfsetstrokecolor{currentstroke}%
\pgfsetdash{}{0pt}%
\pgfpathmoveto{\pgfqpoint{1.221217in}{1.806727in}}%
\pgfpathlineto{\pgfqpoint{1.339054in}{1.653915in}}%
\pgfusepath{stroke}%
\end{pgfscope}%
\begin{pgfscope}%
\pgfpathrectangle{\pgfqpoint{0.100000in}{0.212622in}}{\pgfqpoint{3.696000in}{3.696000in}}%
\pgfusepath{clip}%
\pgfsetrectcap%
\pgfsetroundjoin%
\pgfsetlinewidth{1.505625pt}%
\definecolor{currentstroke}{rgb}{1.000000,0.000000,0.000000}%
\pgfsetstrokecolor{currentstroke}%
\pgfsetdash{}{0pt}%
\pgfpathmoveto{\pgfqpoint{1.231141in}{1.815567in}}%
\pgfpathlineto{\pgfqpoint{1.347958in}{1.661554in}}%
\pgfusepath{stroke}%
\end{pgfscope}%
\begin{pgfscope}%
\pgfpathrectangle{\pgfqpoint{0.100000in}{0.212622in}}{\pgfqpoint{3.696000in}{3.696000in}}%
\pgfusepath{clip}%
\pgfsetrectcap%
\pgfsetroundjoin%
\pgfsetlinewidth{1.505625pt}%
\definecolor{currentstroke}{rgb}{1.000000,0.000000,0.000000}%
\pgfsetstrokecolor{currentstroke}%
\pgfsetdash{}{0pt}%
\pgfpathmoveto{\pgfqpoint{1.239892in}{1.832569in}}%
\pgfpathlineto{\pgfqpoint{1.365731in}{1.676801in}}%
\pgfusepath{stroke}%
\end{pgfscope}%
\begin{pgfscope}%
\pgfpathrectangle{\pgfqpoint{0.100000in}{0.212622in}}{\pgfqpoint{3.696000in}{3.696000in}}%
\pgfusepath{clip}%
\pgfsetrectcap%
\pgfsetroundjoin%
\pgfsetlinewidth{1.505625pt}%
\definecolor{currentstroke}{rgb}{1.000000,0.000000,0.000000}%
\pgfsetstrokecolor{currentstroke}%
\pgfsetdash{}{0pt}%
\pgfpathmoveto{\pgfqpoint{1.259613in}{1.851270in}}%
\pgfpathlineto{\pgfqpoint{1.383460in}{1.692010in}}%
\pgfusepath{stroke}%
\end{pgfscope}%
\begin{pgfscope}%
\pgfpathrectangle{\pgfqpoint{0.100000in}{0.212622in}}{\pgfqpoint{3.696000in}{3.696000in}}%
\pgfusepath{clip}%
\pgfsetrectcap%
\pgfsetroundjoin%
\pgfsetlinewidth{1.505625pt}%
\definecolor{currentstroke}{rgb}{1.000000,0.000000,0.000000}%
\pgfsetstrokecolor{currentstroke}%
\pgfsetdash{}{0pt}%
\pgfpathmoveto{\pgfqpoint{1.283087in}{1.884256in}}%
\pgfpathlineto{\pgfqpoint{1.409968in}{1.714750in}}%
\pgfusepath{stroke}%
\end{pgfscope}%
\begin{pgfscope}%
\pgfpathrectangle{\pgfqpoint{0.100000in}{0.212622in}}{\pgfqpoint{3.696000in}{3.696000in}}%
\pgfusepath{clip}%
\pgfsetrectcap%
\pgfsetroundjoin%
\pgfsetlinewidth{1.505625pt}%
\definecolor{currentstroke}{rgb}{1.000000,0.000000,0.000000}%
\pgfsetstrokecolor{currentstroke}%
\pgfsetdash{}{0pt}%
\pgfpathmoveto{\pgfqpoint{1.312856in}{1.924766in}}%
\pgfpathlineto{\pgfqpoint{1.445155in}{1.744937in}}%
\pgfusepath{stroke}%
\end{pgfscope}%
\begin{pgfscope}%
\pgfpathrectangle{\pgfqpoint{0.100000in}{0.212622in}}{\pgfqpoint{3.696000in}{3.696000in}}%
\pgfusepath{clip}%
\pgfsetrectcap%
\pgfsetroundjoin%
\pgfsetlinewidth{1.505625pt}%
\definecolor{currentstroke}{rgb}{1.000000,0.000000,0.000000}%
\pgfsetstrokecolor{currentstroke}%
\pgfsetdash{}{0pt}%
\pgfpathmoveto{\pgfqpoint{1.341704in}{1.972195in}}%
\pgfpathlineto{\pgfqpoint{1.480165in}{1.774971in}}%
\pgfusepath{stroke}%
\end{pgfscope}%
\begin{pgfscope}%
\pgfpathrectangle{\pgfqpoint{0.100000in}{0.212622in}}{\pgfqpoint{3.696000in}{3.696000in}}%
\pgfusepath{clip}%
\pgfsetrectcap%
\pgfsetroundjoin%
\pgfsetlinewidth{1.505625pt}%
\definecolor{currentstroke}{rgb}{1.000000,0.000000,0.000000}%
\pgfsetstrokecolor{currentstroke}%
\pgfsetdash{}{0pt}%
\pgfpathmoveto{\pgfqpoint{1.379277in}{2.006665in}}%
\pgfpathlineto{\pgfqpoint{1.514999in}{1.804854in}}%
\pgfusepath{stroke}%
\end{pgfscope}%
\begin{pgfscope}%
\pgfpathrectangle{\pgfqpoint{0.100000in}{0.212622in}}{\pgfqpoint{3.696000in}{3.696000in}}%
\pgfusepath{clip}%
\pgfsetrectcap%
\pgfsetroundjoin%
\pgfsetlinewidth{1.505625pt}%
\definecolor{currentstroke}{rgb}{1.000000,0.000000,0.000000}%
\pgfsetstrokecolor{currentstroke}%
\pgfsetdash{}{0pt}%
\pgfpathmoveto{\pgfqpoint{1.394484in}{2.029991in}}%
\pgfpathlineto{\pgfqpoint{1.532350in}{1.819739in}}%
\pgfusepath{stroke}%
\end{pgfscope}%
\begin{pgfscope}%
\pgfpathrectangle{\pgfqpoint{0.100000in}{0.212622in}}{\pgfqpoint{3.696000in}{3.696000in}}%
\pgfusepath{clip}%
\pgfsetrectcap%
\pgfsetroundjoin%
\pgfsetlinewidth{1.505625pt}%
\definecolor{currentstroke}{rgb}{1.000000,0.000000,0.000000}%
\pgfsetstrokecolor{currentstroke}%
\pgfsetdash{}{0pt}%
\pgfpathmoveto{\pgfqpoint{1.403713in}{2.043121in}}%
\pgfpathlineto{\pgfqpoint{1.541010in}{1.827167in}}%
\pgfusepath{stroke}%
\end{pgfscope}%
\begin{pgfscope}%
\pgfpathrectangle{\pgfqpoint{0.100000in}{0.212622in}}{\pgfqpoint{3.696000in}{3.696000in}}%
\pgfusepath{clip}%
\pgfsetrectcap%
\pgfsetroundjoin%
\pgfsetlinewidth{1.505625pt}%
\definecolor{currentstroke}{rgb}{1.000000,0.000000,0.000000}%
\pgfsetstrokecolor{currentstroke}%
\pgfsetdash{}{0pt}%
\pgfpathmoveto{\pgfqpoint{1.407306in}{2.050121in}}%
\pgfpathlineto{\pgfqpoint{1.549658in}{1.834587in}}%
\pgfusepath{stroke}%
\end{pgfscope}%
\begin{pgfscope}%
\pgfpathrectangle{\pgfqpoint{0.100000in}{0.212622in}}{\pgfqpoint{3.696000in}{3.696000in}}%
\pgfusepath{clip}%
\pgfsetrectcap%
\pgfsetroundjoin%
\pgfsetlinewidth{1.505625pt}%
\definecolor{currentstroke}{rgb}{1.000000,0.000000,0.000000}%
\pgfsetstrokecolor{currentstroke}%
\pgfsetdash{}{0pt}%
\pgfpathmoveto{\pgfqpoint{1.410240in}{2.054669in}}%
\pgfpathlineto{\pgfqpoint{1.549658in}{1.834587in}}%
\pgfusepath{stroke}%
\end{pgfscope}%
\begin{pgfscope}%
\pgfpathrectangle{\pgfqpoint{0.100000in}{0.212622in}}{\pgfqpoint{3.696000in}{3.696000in}}%
\pgfusepath{clip}%
\pgfsetrectcap%
\pgfsetroundjoin%
\pgfsetlinewidth{1.505625pt}%
\definecolor{currentstroke}{rgb}{1.000000,0.000000,0.000000}%
\pgfsetstrokecolor{currentstroke}%
\pgfsetdash{}{0pt}%
\pgfpathmoveto{\pgfqpoint{1.411679in}{2.056813in}}%
\pgfpathlineto{\pgfqpoint{1.558296in}{1.841997in}}%
\pgfusepath{stroke}%
\end{pgfscope}%
\begin{pgfscope}%
\pgfpathrectangle{\pgfqpoint{0.100000in}{0.212622in}}{\pgfqpoint{3.696000in}{3.696000in}}%
\pgfusepath{clip}%
\pgfsetrectcap%
\pgfsetroundjoin%
\pgfsetlinewidth{1.505625pt}%
\definecolor{currentstroke}{rgb}{1.000000,0.000000,0.000000}%
\pgfsetstrokecolor{currentstroke}%
\pgfsetdash{}{0pt}%
\pgfpathmoveto{\pgfqpoint{1.414851in}{2.062317in}}%
\pgfpathlineto{\pgfqpoint{1.558296in}{1.841997in}}%
\pgfusepath{stroke}%
\end{pgfscope}%
\begin{pgfscope}%
\pgfpathrectangle{\pgfqpoint{0.100000in}{0.212622in}}{\pgfqpoint{3.696000in}{3.696000in}}%
\pgfusepath{clip}%
\pgfsetrectcap%
\pgfsetroundjoin%
\pgfsetlinewidth{1.505625pt}%
\definecolor{currentstroke}{rgb}{1.000000,0.000000,0.000000}%
\pgfsetstrokecolor{currentstroke}%
\pgfsetdash{}{0pt}%
\pgfpathmoveto{\pgfqpoint{1.422380in}{2.071197in}}%
\pgfpathlineto{\pgfqpoint{1.566922in}{1.849397in}}%
\pgfusepath{stroke}%
\end{pgfscope}%
\begin{pgfscope}%
\pgfpathrectangle{\pgfqpoint{0.100000in}{0.212622in}}{\pgfqpoint{3.696000in}{3.696000in}}%
\pgfusepath{clip}%
\pgfsetrectcap%
\pgfsetroundjoin%
\pgfsetlinewidth{1.505625pt}%
\definecolor{currentstroke}{rgb}{1.000000,0.000000,0.000000}%
\pgfsetstrokecolor{currentstroke}%
\pgfsetdash{}{0pt}%
\pgfpathmoveto{\pgfqpoint{1.428736in}{2.084012in}}%
\pgfpathlineto{\pgfqpoint{1.575538in}{1.856788in}}%
\pgfusepath{stroke}%
\end{pgfscope}%
\begin{pgfscope}%
\pgfpathrectangle{\pgfqpoint{0.100000in}{0.212622in}}{\pgfqpoint{3.696000in}{3.696000in}}%
\pgfusepath{clip}%
\pgfsetrectcap%
\pgfsetroundjoin%
\pgfsetlinewidth{1.505625pt}%
\definecolor{currentstroke}{rgb}{1.000000,0.000000,0.000000}%
\pgfsetstrokecolor{currentstroke}%
\pgfsetdash{}{0pt}%
\pgfpathmoveto{\pgfqpoint{1.443062in}{2.095446in}}%
\pgfpathlineto{\pgfqpoint{1.592738in}{1.871543in}}%
\pgfusepath{stroke}%
\end{pgfscope}%
\begin{pgfscope}%
\pgfpathrectangle{\pgfqpoint{0.100000in}{0.212622in}}{\pgfqpoint{3.696000in}{3.696000in}}%
\pgfusepath{clip}%
\pgfsetrectcap%
\pgfsetroundjoin%
\pgfsetlinewidth{1.505625pt}%
\definecolor{currentstroke}{rgb}{1.000000,0.000000,0.000000}%
\pgfsetstrokecolor{currentstroke}%
\pgfsetdash{}{0pt}%
\pgfpathmoveto{\pgfqpoint{1.456090in}{2.116154in}}%
\pgfpathlineto{\pgfqpoint{1.609894in}{1.886261in}}%
\pgfusepath{stroke}%
\end{pgfscope}%
\begin{pgfscope}%
\pgfpathrectangle{\pgfqpoint{0.100000in}{0.212622in}}{\pgfqpoint{3.696000in}{3.696000in}}%
\pgfusepath{clip}%
\pgfsetrectcap%
\pgfsetroundjoin%
\pgfsetlinewidth{1.505625pt}%
\definecolor{currentstroke}{rgb}{1.000000,0.000000,0.000000}%
\pgfsetstrokecolor{currentstroke}%
\pgfsetdash{}{0pt}%
\pgfpathmoveto{\pgfqpoint{1.478360in}{2.137110in}}%
\pgfpathlineto{\pgfqpoint{1.627008in}{1.900943in}}%
\pgfusepath{stroke}%
\end{pgfscope}%
\begin{pgfscope}%
\pgfpathrectangle{\pgfqpoint{0.100000in}{0.212622in}}{\pgfqpoint{3.696000in}{3.696000in}}%
\pgfusepath{clip}%
\pgfsetrectcap%
\pgfsetroundjoin%
\pgfsetlinewidth{1.505625pt}%
\definecolor{currentstroke}{rgb}{1.000000,0.000000,0.000000}%
\pgfsetstrokecolor{currentstroke}%
\pgfsetdash{}{0pt}%
\pgfpathmoveto{\pgfqpoint{1.497048in}{2.166672in}}%
\pgfpathlineto{\pgfqpoint{1.652598in}{1.922896in}}%
\pgfusepath{stroke}%
\end{pgfscope}%
\begin{pgfscope}%
\pgfpathrectangle{\pgfqpoint{0.100000in}{0.212622in}}{\pgfqpoint{3.696000in}{3.696000in}}%
\pgfusepath{clip}%
\pgfsetrectcap%
\pgfsetroundjoin%
\pgfsetlinewidth{1.505625pt}%
\definecolor{currentstroke}{rgb}{1.000000,0.000000,0.000000}%
\pgfsetstrokecolor{currentstroke}%
\pgfsetdash{}{0pt}%
\pgfpathmoveto{\pgfqpoint{1.517958in}{2.200180in}}%
\pgfpathlineto{\pgfqpoint{1.678093in}{1.944767in}}%
\pgfusepath{stroke}%
\end{pgfscope}%
\begin{pgfscope}%
\pgfpathrectangle{\pgfqpoint{0.100000in}{0.212622in}}{\pgfqpoint{3.696000in}{3.696000in}}%
\pgfusepath{clip}%
\pgfsetrectcap%
\pgfsetroundjoin%
\pgfsetlinewidth{1.505625pt}%
\definecolor{currentstroke}{rgb}{1.000000,0.000000,0.000000}%
\pgfsetstrokecolor{currentstroke}%
\pgfsetdash{}{0pt}%
\pgfpathmoveto{\pgfqpoint{1.540360in}{2.239412in}}%
\pgfpathlineto{\pgfqpoint{1.711938in}{1.973802in}}%
\pgfusepath{stroke}%
\end{pgfscope}%
\begin{pgfscope}%
\pgfpathrectangle{\pgfqpoint{0.100000in}{0.212622in}}{\pgfqpoint{3.696000in}{3.696000in}}%
\pgfusepath{clip}%
\pgfsetrectcap%
\pgfsetroundjoin%
\pgfsetlinewidth{1.505625pt}%
\definecolor{currentstroke}{rgb}{1.000000,0.000000,0.000000}%
\pgfsetstrokecolor{currentstroke}%
\pgfsetdash{}{0pt}%
\pgfpathmoveto{\pgfqpoint{1.554180in}{2.257041in}}%
\pgfpathlineto{\pgfqpoint{1.728798in}{1.988266in}}%
\pgfusepath{stroke}%
\end{pgfscope}%
\begin{pgfscope}%
\pgfpathrectangle{\pgfqpoint{0.100000in}{0.212622in}}{\pgfqpoint{3.696000in}{3.696000in}}%
\pgfusepath{clip}%
\pgfsetrectcap%
\pgfsetroundjoin%
\pgfsetlinewidth{1.505625pt}%
\definecolor{currentstroke}{rgb}{1.000000,0.000000,0.000000}%
\pgfsetstrokecolor{currentstroke}%
\pgfsetdash{}{0pt}%
\pgfpathmoveto{\pgfqpoint{1.561122in}{2.267913in}}%
\pgfpathlineto{\pgfqpoint{1.737213in}{1.995484in}}%
\pgfusepath{stroke}%
\end{pgfscope}%
\begin{pgfscope}%
\pgfpathrectangle{\pgfqpoint{0.100000in}{0.212622in}}{\pgfqpoint{3.696000in}{3.696000in}}%
\pgfusepath{clip}%
\pgfsetrectcap%
\pgfsetroundjoin%
\pgfsetlinewidth{1.505625pt}%
\definecolor{currentstroke}{rgb}{1.000000,0.000000,0.000000}%
\pgfsetstrokecolor{currentstroke}%
\pgfsetdash{}{0pt}%
\pgfpathmoveto{\pgfqpoint{1.565025in}{2.273575in}}%
\pgfpathlineto{\pgfqpoint{1.737213in}{1.995484in}}%
\pgfusepath{stroke}%
\end{pgfscope}%
\begin{pgfscope}%
\pgfpathrectangle{\pgfqpoint{0.100000in}{0.212622in}}{\pgfqpoint{3.696000in}{3.696000in}}%
\pgfusepath{clip}%
\pgfsetrectcap%
\pgfsetroundjoin%
\pgfsetlinewidth{1.505625pt}%
\definecolor{currentstroke}{rgb}{1.000000,0.000000,0.000000}%
\pgfsetstrokecolor{currentstroke}%
\pgfsetdash{}{0pt}%
\pgfpathmoveto{\pgfqpoint{1.566917in}{2.276560in}}%
\pgfpathlineto{\pgfqpoint{1.745616in}{2.002694in}}%
\pgfusepath{stroke}%
\end{pgfscope}%
\begin{pgfscope}%
\pgfpathrectangle{\pgfqpoint{0.100000in}{0.212622in}}{\pgfqpoint{3.696000in}{3.696000in}}%
\pgfusepath{clip}%
\pgfsetrectcap%
\pgfsetroundjoin%
\pgfsetlinewidth{1.505625pt}%
\definecolor{currentstroke}{rgb}{1.000000,0.000000,0.000000}%
\pgfsetstrokecolor{currentstroke}%
\pgfsetdash{}{0pt}%
\pgfpathmoveto{\pgfqpoint{1.568468in}{2.277807in}}%
\pgfpathlineto{\pgfqpoint{1.745616in}{2.002694in}}%
\pgfusepath{stroke}%
\end{pgfscope}%
\begin{pgfscope}%
\pgfpathrectangle{\pgfqpoint{0.100000in}{0.212622in}}{\pgfqpoint{3.696000in}{3.696000in}}%
\pgfusepath{clip}%
\pgfsetrectcap%
\pgfsetroundjoin%
\pgfsetlinewidth{1.505625pt}%
\definecolor{currentstroke}{rgb}{1.000000,0.000000,0.000000}%
\pgfsetstrokecolor{currentstroke}%
\pgfsetdash{}{0pt}%
\pgfpathmoveto{\pgfqpoint{1.569069in}{2.278887in}}%
\pgfpathlineto{\pgfqpoint{1.745616in}{2.002694in}}%
\pgfusepath{stroke}%
\end{pgfscope}%
\begin{pgfscope}%
\pgfpathrectangle{\pgfqpoint{0.100000in}{0.212622in}}{\pgfqpoint{3.696000in}{3.696000in}}%
\pgfusepath{clip}%
\pgfsetrectcap%
\pgfsetroundjoin%
\pgfsetlinewidth{1.505625pt}%
\definecolor{currentstroke}{rgb}{1.000000,0.000000,0.000000}%
\pgfsetstrokecolor{currentstroke}%
\pgfsetdash{}{0pt}%
\pgfpathmoveto{\pgfqpoint{1.574186in}{2.283079in}}%
\pgfpathlineto{\pgfqpoint{1.754010in}{2.009894in}}%
\pgfusepath{stroke}%
\end{pgfscope}%
\begin{pgfscope}%
\pgfpathrectangle{\pgfqpoint{0.100000in}{0.212622in}}{\pgfqpoint{3.696000in}{3.696000in}}%
\pgfusepath{clip}%
\pgfsetrectcap%
\pgfsetroundjoin%
\pgfsetlinewidth{1.505625pt}%
\definecolor{currentstroke}{rgb}{1.000000,0.000000,0.000000}%
\pgfsetstrokecolor{currentstroke}%
\pgfsetdash{}{0pt}%
\pgfpathmoveto{\pgfqpoint{1.582011in}{2.293152in}}%
\pgfpathlineto{\pgfqpoint{1.762393in}{2.017086in}}%
\pgfusepath{stroke}%
\end{pgfscope}%
\begin{pgfscope}%
\pgfpathrectangle{\pgfqpoint{0.100000in}{0.212622in}}{\pgfqpoint{3.696000in}{3.696000in}}%
\pgfusepath{clip}%
\pgfsetrectcap%
\pgfsetroundjoin%
\pgfsetlinewidth{1.505625pt}%
\definecolor{currentstroke}{rgb}{1.000000,0.000000,0.000000}%
\pgfsetstrokecolor{currentstroke}%
\pgfsetdash{}{0pt}%
\pgfpathmoveto{\pgfqpoint{1.594926in}{2.305251in}}%
\pgfpathlineto{\pgfqpoint{1.770766in}{2.024269in}}%
\pgfusepath{stroke}%
\end{pgfscope}%
\begin{pgfscope}%
\pgfpathrectangle{\pgfqpoint{0.100000in}{0.212622in}}{\pgfqpoint{3.696000in}{3.696000in}}%
\pgfusepath{clip}%
\pgfsetrectcap%
\pgfsetroundjoin%
\pgfsetlinewidth{1.505625pt}%
\definecolor{currentstroke}{rgb}{1.000000,0.000000,0.000000}%
\pgfsetstrokecolor{currentstroke}%
\pgfsetdash{}{0pt}%
\pgfpathmoveto{\pgfqpoint{1.609202in}{2.325194in}}%
\pgfpathlineto{\pgfqpoint{1.787480in}{2.038608in}}%
\pgfusepath{stroke}%
\end{pgfscope}%
\begin{pgfscope}%
\pgfpathrectangle{\pgfqpoint{0.100000in}{0.212622in}}{\pgfqpoint{3.696000in}{3.696000in}}%
\pgfusepath{clip}%
\pgfsetrectcap%
\pgfsetroundjoin%
\pgfsetlinewidth{1.505625pt}%
\definecolor{currentstroke}{rgb}{1.000000,0.000000,0.000000}%
\pgfsetstrokecolor{currentstroke}%
\pgfsetdash{}{0pt}%
\pgfpathmoveto{\pgfqpoint{1.627511in}{2.344082in}}%
\pgfpathlineto{\pgfqpoint{1.795822in}{2.045764in}}%
\pgfusepath{stroke}%
\end{pgfscope}%
\begin{pgfscope}%
\pgfpathrectangle{\pgfqpoint{0.100000in}{0.212622in}}{\pgfqpoint{3.696000in}{3.696000in}}%
\pgfusepath{clip}%
\pgfsetrectcap%
\pgfsetroundjoin%
\pgfsetlinewidth{1.505625pt}%
\definecolor{currentstroke}{rgb}{1.000000,0.000000,0.000000}%
\pgfsetstrokecolor{currentstroke}%
\pgfsetdash{}{0pt}%
\pgfpathmoveto{\pgfqpoint{1.636061in}{2.357154in}}%
\pgfpathlineto{\pgfqpoint{1.795822in}{2.045764in}}%
\pgfusepath{stroke}%
\end{pgfscope}%
\begin{pgfscope}%
\pgfpathrectangle{\pgfqpoint{0.100000in}{0.212622in}}{\pgfqpoint{3.696000in}{3.696000in}}%
\pgfusepath{clip}%
\pgfsetrectcap%
\pgfsetroundjoin%
\pgfsetlinewidth{1.505625pt}%
\definecolor{currentstroke}{rgb}{1.000000,0.000000,0.000000}%
\pgfsetstrokecolor{currentstroke}%
\pgfsetdash{}{0pt}%
\pgfpathmoveto{\pgfqpoint{1.640755in}{2.363684in}}%
\pgfpathlineto{\pgfqpoint{1.795822in}{2.045764in}}%
\pgfusepath{stroke}%
\end{pgfscope}%
\begin{pgfscope}%
\pgfpathrectangle{\pgfqpoint{0.100000in}{0.212622in}}{\pgfqpoint{3.696000in}{3.696000in}}%
\pgfusepath{clip}%
\pgfsetrectcap%
\pgfsetroundjoin%
\pgfsetlinewidth{1.505625pt}%
\definecolor{currentstroke}{rgb}{1.000000,0.000000,0.000000}%
\pgfsetstrokecolor{currentstroke}%
\pgfsetdash{}{0pt}%
\pgfpathmoveto{\pgfqpoint{1.642958in}{2.367119in}}%
\pgfpathlineto{\pgfqpoint{1.795822in}{2.045764in}}%
\pgfusepath{stroke}%
\end{pgfscope}%
\begin{pgfscope}%
\pgfpathrectangle{\pgfqpoint{0.100000in}{0.212622in}}{\pgfqpoint{3.696000in}{3.696000in}}%
\pgfusepath{clip}%
\pgfsetrectcap%
\pgfsetroundjoin%
\pgfsetlinewidth{1.505625pt}%
\definecolor{currentstroke}{rgb}{1.000000,0.000000,0.000000}%
\pgfsetstrokecolor{currentstroke}%
\pgfsetdash{}{0pt}%
\pgfpathmoveto{\pgfqpoint{1.644355in}{2.369364in}}%
\pgfpathlineto{\pgfqpoint{1.795822in}{2.045764in}}%
\pgfusepath{stroke}%
\end{pgfscope}%
\begin{pgfscope}%
\pgfpathrectangle{\pgfqpoint{0.100000in}{0.212622in}}{\pgfqpoint{3.696000in}{3.696000in}}%
\pgfusepath{clip}%
\pgfsetrectcap%
\pgfsetroundjoin%
\pgfsetlinewidth{1.505625pt}%
\definecolor{currentstroke}{rgb}{1.000000,0.000000,0.000000}%
\pgfsetstrokecolor{currentstroke}%
\pgfsetdash{}{0pt}%
\pgfpathmoveto{\pgfqpoint{1.645011in}{2.370480in}}%
\pgfpathlineto{\pgfqpoint{1.795822in}{2.045764in}}%
\pgfusepath{stroke}%
\end{pgfscope}%
\begin{pgfscope}%
\pgfpathrectangle{\pgfqpoint{0.100000in}{0.212622in}}{\pgfqpoint{3.696000in}{3.696000in}}%
\pgfusepath{clip}%
\pgfsetrectcap%
\pgfsetroundjoin%
\pgfsetlinewidth{1.505625pt}%
\definecolor{currentstroke}{rgb}{1.000000,0.000000,0.000000}%
\pgfsetstrokecolor{currentstroke}%
\pgfsetdash{}{0pt}%
\pgfpathmoveto{\pgfqpoint{1.645419in}{2.371061in}}%
\pgfpathlineto{\pgfqpoint{1.795822in}{2.045764in}}%
\pgfusepath{stroke}%
\end{pgfscope}%
\begin{pgfscope}%
\pgfpathrectangle{\pgfqpoint{0.100000in}{0.212622in}}{\pgfqpoint{3.696000in}{3.696000in}}%
\pgfusepath{clip}%
\pgfsetrectcap%
\pgfsetroundjoin%
\pgfsetlinewidth{1.505625pt}%
\definecolor{currentstroke}{rgb}{1.000000,0.000000,0.000000}%
\pgfsetstrokecolor{currentstroke}%
\pgfsetdash{}{0pt}%
\pgfpathmoveto{\pgfqpoint{1.645646in}{2.371425in}}%
\pgfpathlineto{\pgfqpoint{1.795822in}{2.045764in}}%
\pgfusepath{stroke}%
\end{pgfscope}%
\begin{pgfscope}%
\pgfpathrectangle{\pgfqpoint{0.100000in}{0.212622in}}{\pgfqpoint{3.696000in}{3.696000in}}%
\pgfusepath{clip}%
\pgfsetrectcap%
\pgfsetroundjoin%
\pgfsetlinewidth{1.505625pt}%
\definecolor{currentstroke}{rgb}{1.000000,0.000000,0.000000}%
\pgfsetstrokecolor{currentstroke}%
\pgfsetdash{}{0pt}%
\pgfpathmoveto{\pgfqpoint{1.645772in}{2.371614in}}%
\pgfpathlineto{\pgfqpoint{1.795822in}{2.045764in}}%
\pgfusepath{stroke}%
\end{pgfscope}%
\begin{pgfscope}%
\pgfpathrectangle{\pgfqpoint{0.100000in}{0.212622in}}{\pgfqpoint{3.696000in}{3.696000in}}%
\pgfusepath{clip}%
\pgfsetrectcap%
\pgfsetroundjoin%
\pgfsetlinewidth{1.505625pt}%
\definecolor{currentstroke}{rgb}{1.000000,0.000000,0.000000}%
\pgfsetstrokecolor{currentstroke}%
\pgfsetdash{}{0pt}%
\pgfpathmoveto{\pgfqpoint{1.645841in}{2.371717in}}%
\pgfpathlineto{\pgfqpoint{1.795822in}{2.045764in}}%
\pgfusepath{stroke}%
\end{pgfscope}%
\begin{pgfscope}%
\pgfpathrectangle{\pgfqpoint{0.100000in}{0.212622in}}{\pgfqpoint{3.696000in}{3.696000in}}%
\pgfusepath{clip}%
\pgfsetrectcap%
\pgfsetroundjoin%
\pgfsetlinewidth{1.505625pt}%
\definecolor{currentstroke}{rgb}{1.000000,0.000000,0.000000}%
\pgfsetstrokecolor{currentstroke}%
\pgfsetdash{}{0pt}%
\pgfpathmoveto{\pgfqpoint{1.645878in}{2.371773in}}%
\pgfpathlineto{\pgfqpoint{1.795822in}{2.045764in}}%
\pgfusepath{stroke}%
\end{pgfscope}%
\begin{pgfscope}%
\pgfpathrectangle{\pgfqpoint{0.100000in}{0.212622in}}{\pgfqpoint{3.696000in}{3.696000in}}%
\pgfusepath{clip}%
\pgfsetrectcap%
\pgfsetroundjoin%
\pgfsetlinewidth{1.505625pt}%
\definecolor{currentstroke}{rgb}{1.000000,0.000000,0.000000}%
\pgfsetstrokecolor{currentstroke}%
\pgfsetdash{}{0pt}%
\pgfpathmoveto{\pgfqpoint{1.645899in}{2.371804in}}%
\pgfpathlineto{\pgfqpoint{1.795822in}{2.045764in}}%
\pgfusepath{stroke}%
\end{pgfscope}%
\begin{pgfscope}%
\pgfpathrectangle{\pgfqpoint{0.100000in}{0.212622in}}{\pgfqpoint{3.696000in}{3.696000in}}%
\pgfusepath{clip}%
\pgfsetrectcap%
\pgfsetroundjoin%
\pgfsetlinewidth{1.505625pt}%
\definecolor{currentstroke}{rgb}{1.000000,0.000000,0.000000}%
\pgfsetstrokecolor{currentstroke}%
\pgfsetdash{}{0pt}%
\pgfpathmoveto{\pgfqpoint{1.645909in}{2.371821in}}%
\pgfpathlineto{\pgfqpoint{1.795822in}{2.045764in}}%
\pgfusepath{stroke}%
\end{pgfscope}%
\begin{pgfscope}%
\pgfpathrectangle{\pgfqpoint{0.100000in}{0.212622in}}{\pgfqpoint{3.696000in}{3.696000in}}%
\pgfusepath{clip}%
\pgfsetrectcap%
\pgfsetroundjoin%
\pgfsetlinewidth{1.505625pt}%
\definecolor{currentstroke}{rgb}{1.000000,0.000000,0.000000}%
\pgfsetstrokecolor{currentstroke}%
\pgfsetdash{}{0pt}%
\pgfpathmoveto{\pgfqpoint{1.645917in}{2.371830in}}%
\pgfpathlineto{\pgfqpoint{1.795822in}{2.045764in}}%
\pgfusepath{stroke}%
\end{pgfscope}%
\begin{pgfscope}%
\pgfpathrectangle{\pgfqpoint{0.100000in}{0.212622in}}{\pgfqpoint{3.696000in}{3.696000in}}%
\pgfusepath{clip}%
\pgfsetrectcap%
\pgfsetroundjoin%
\pgfsetlinewidth{1.505625pt}%
\definecolor{currentstroke}{rgb}{1.000000,0.000000,0.000000}%
\pgfsetstrokecolor{currentstroke}%
\pgfsetdash{}{0pt}%
\pgfpathmoveto{\pgfqpoint{1.645920in}{2.371836in}}%
\pgfpathlineto{\pgfqpoint{1.795822in}{2.045764in}}%
\pgfusepath{stroke}%
\end{pgfscope}%
\begin{pgfscope}%
\pgfpathrectangle{\pgfqpoint{0.100000in}{0.212622in}}{\pgfqpoint{3.696000in}{3.696000in}}%
\pgfusepath{clip}%
\pgfsetrectcap%
\pgfsetroundjoin%
\pgfsetlinewidth{1.505625pt}%
\definecolor{currentstroke}{rgb}{1.000000,0.000000,0.000000}%
\pgfsetstrokecolor{currentstroke}%
\pgfsetdash{}{0pt}%
\pgfpathmoveto{\pgfqpoint{1.645923in}{2.371839in}}%
\pgfpathlineto{\pgfqpoint{1.795822in}{2.045764in}}%
\pgfusepath{stroke}%
\end{pgfscope}%
\begin{pgfscope}%
\pgfpathrectangle{\pgfqpoint{0.100000in}{0.212622in}}{\pgfqpoint{3.696000in}{3.696000in}}%
\pgfusepath{clip}%
\pgfsetrectcap%
\pgfsetroundjoin%
\pgfsetlinewidth{1.505625pt}%
\definecolor{currentstroke}{rgb}{1.000000,0.000000,0.000000}%
\pgfsetstrokecolor{currentstroke}%
\pgfsetdash{}{0pt}%
\pgfpathmoveto{\pgfqpoint{1.645924in}{2.371840in}}%
\pgfpathlineto{\pgfqpoint{1.795822in}{2.045764in}}%
\pgfusepath{stroke}%
\end{pgfscope}%
\begin{pgfscope}%
\pgfpathrectangle{\pgfqpoint{0.100000in}{0.212622in}}{\pgfqpoint{3.696000in}{3.696000in}}%
\pgfusepath{clip}%
\pgfsetrectcap%
\pgfsetroundjoin%
\pgfsetlinewidth{1.505625pt}%
\definecolor{currentstroke}{rgb}{1.000000,0.000000,0.000000}%
\pgfsetstrokecolor{currentstroke}%
\pgfsetdash{}{0pt}%
\pgfpathmoveto{\pgfqpoint{1.648292in}{2.373429in}}%
\pgfpathlineto{\pgfqpoint{1.795822in}{2.045764in}}%
\pgfusepath{stroke}%
\end{pgfscope}%
\begin{pgfscope}%
\pgfpathrectangle{\pgfqpoint{0.100000in}{0.212622in}}{\pgfqpoint{3.696000in}{3.696000in}}%
\pgfusepath{clip}%
\pgfsetrectcap%
\pgfsetroundjoin%
\pgfsetlinewidth{1.505625pt}%
\definecolor{currentstroke}{rgb}{1.000000,0.000000,0.000000}%
\pgfsetstrokecolor{currentstroke}%
\pgfsetdash{}{0pt}%
\pgfpathmoveto{\pgfqpoint{1.649688in}{2.374513in}}%
\pgfpathlineto{\pgfqpoint{1.795822in}{2.045764in}}%
\pgfusepath{stroke}%
\end{pgfscope}%
\begin{pgfscope}%
\pgfpathrectangle{\pgfqpoint{0.100000in}{0.212622in}}{\pgfqpoint{3.696000in}{3.696000in}}%
\pgfusepath{clip}%
\pgfsetrectcap%
\pgfsetroundjoin%
\pgfsetlinewidth{1.505625pt}%
\definecolor{currentstroke}{rgb}{1.000000,0.000000,0.000000}%
\pgfsetstrokecolor{currentstroke}%
\pgfsetdash{}{0pt}%
\pgfpathmoveto{\pgfqpoint{1.654583in}{2.376030in}}%
\pgfpathlineto{\pgfqpoint{1.795822in}{2.045764in}}%
\pgfusepath{stroke}%
\end{pgfscope}%
\begin{pgfscope}%
\pgfpathrectangle{\pgfqpoint{0.100000in}{0.212622in}}{\pgfqpoint{3.696000in}{3.696000in}}%
\pgfusepath{clip}%
\pgfsetrectcap%
\pgfsetroundjoin%
\pgfsetlinewidth{1.505625pt}%
\definecolor{currentstroke}{rgb}{1.000000,0.000000,0.000000}%
\pgfsetstrokecolor{currentstroke}%
\pgfsetdash{}{0pt}%
\pgfpathmoveto{\pgfqpoint{1.661802in}{2.378586in}}%
\pgfpathlineto{\pgfqpoint{1.795822in}{2.045764in}}%
\pgfusepath{stroke}%
\end{pgfscope}%
\begin{pgfscope}%
\pgfpathrectangle{\pgfqpoint{0.100000in}{0.212622in}}{\pgfqpoint{3.696000in}{3.696000in}}%
\pgfusepath{clip}%
\pgfsetrectcap%
\pgfsetroundjoin%
\pgfsetlinewidth{1.505625pt}%
\definecolor{currentstroke}{rgb}{1.000000,0.000000,0.000000}%
\pgfsetstrokecolor{currentstroke}%
\pgfsetdash{}{0pt}%
\pgfpathmoveto{\pgfqpoint{1.672363in}{2.380011in}}%
\pgfpathlineto{\pgfqpoint{1.795822in}{2.045764in}}%
\pgfusepath{stroke}%
\end{pgfscope}%
\begin{pgfscope}%
\pgfpathrectangle{\pgfqpoint{0.100000in}{0.212622in}}{\pgfqpoint{3.696000in}{3.696000in}}%
\pgfusepath{clip}%
\pgfsetrectcap%
\pgfsetroundjoin%
\pgfsetlinewidth{1.505625pt}%
\definecolor{currentstroke}{rgb}{1.000000,0.000000,0.000000}%
\pgfsetstrokecolor{currentstroke}%
\pgfsetdash{}{0pt}%
\pgfpathmoveto{\pgfqpoint{1.684693in}{2.380609in}}%
\pgfpathlineto{\pgfqpoint{1.795822in}{2.045764in}}%
\pgfusepath{stroke}%
\end{pgfscope}%
\begin{pgfscope}%
\pgfpathrectangle{\pgfqpoint{0.100000in}{0.212622in}}{\pgfqpoint{3.696000in}{3.696000in}}%
\pgfusepath{clip}%
\pgfsetrectcap%
\pgfsetroundjoin%
\pgfsetlinewidth{1.505625pt}%
\definecolor{currentstroke}{rgb}{1.000000,0.000000,0.000000}%
\pgfsetstrokecolor{currentstroke}%
\pgfsetdash{}{0pt}%
\pgfpathmoveto{\pgfqpoint{1.698426in}{2.381170in}}%
\pgfpathlineto{\pgfqpoint{1.795822in}{2.045764in}}%
\pgfusepath{stroke}%
\end{pgfscope}%
\begin{pgfscope}%
\pgfpathrectangle{\pgfqpoint{0.100000in}{0.212622in}}{\pgfqpoint{3.696000in}{3.696000in}}%
\pgfusepath{clip}%
\pgfsetrectcap%
\pgfsetroundjoin%
\pgfsetlinewidth{1.505625pt}%
\definecolor{currentstroke}{rgb}{1.000000,0.000000,0.000000}%
\pgfsetstrokecolor{currentstroke}%
\pgfsetdash{}{0pt}%
\pgfpathmoveto{\pgfqpoint{1.714482in}{2.379732in}}%
\pgfpathlineto{\pgfqpoint{1.795822in}{2.045764in}}%
\pgfusepath{stroke}%
\end{pgfscope}%
\begin{pgfscope}%
\pgfpathrectangle{\pgfqpoint{0.100000in}{0.212622in}}{\pgfqpoint{3.696000in}{3.696000in}}%
\pgfusepath{clip}%
\pgfsetrectcap%
\pgfsetroundjoin%
\pgfsetlinewidth{1.505625pt}%
\definecolor{currentstroke}{rgb}{1.000000,0.000000,0.000000}%
\pgfsetstrokecolor{currentstroke}%
\pgfsetdash{}{0pt}%
\pgfpathmoveto{\pgfqpoint{1.733905in}{2.380460in}}%
\pgfpathlineto{\pgfqpoint{1.795822in}{2.045764in}}%
\pgfusepath{stroke}%
\end{pgfscope}%
\begin{pgfscope}%
\pgfpathrectangle{\pgfqpoint{0.100000in}{0.212622in}}{\pgfqpoint{3.696000in}{3.696000in}}%
\pgfusepath{clip}%
\pgfsetrectcap%
\pgfsetroundjoin%
\pgfsetlinewidth{1.505625pt}%
\definecolor{currentstroke}{rgb}{1.000000,0.000000,0.000000}%
\pgfsetstrokecolor{currentstroke}%
\pgfsetdash{}{0pt}%
\pgfpathmoveto{\pgfqpoint{1.755907in}{2.375849in}}%
\pgfpathlineto{\pgfqpoint{1.795822in}{2.045764in}}%
\pgfusepath{stroke}%
\end{pgfscope}%
\begin{pgfscope}%
\pgfpathrectangle{\pgfqpoint{0.100000in}{0.212622in}}{\pgfqpoint{3.696000in}{3.696000in}}%
\pgfusepath{clip}%
\pgfsetrectcap%
\pgfsetroundjoin%
\pgfsetlinewidth{1.505625pt}%
\definecolor{currentstroke}{rgb}{1.000000,0.000000,0.000000}%
\pgfsetstrokecolor{currentstroke}%
\pgfsetdash{}{0pt}%
\pgfpathmoveto{\pgfqpoint{1.767961in}{2.373870in}}%
\pgfpathlineto{\pgfqpoint{1.795822in}{2.045764in}}%
\pgfusepath{stroke}%
\end{pgfscope}%
\begin{pgfscope}%
\pgfpathrectangle{\pgfqpoint{0.100000in}{0.212622in}}{\pgfqpoint{3.696000in}{3.696000in}}%
\pgfusepath{clip}%
\pgfsetrectcap%
\pgfsetroundjoin%
\pgfsetlinewidth{1.505625pt}%
\definecolor{currentstroke}{rgb}{1.000000,0.000000,0.000000}%
\pgfsetstrokecolor{currentstroke}%
\pgfsetdash{}{0pt}%
\pgfpathmoveto{\pgfqpoint{1.782353in}{2.372794in}}%
\pgfpathlineto{\pgfqpoint{1.795822in}{2.045764in}}%
\pgfusepath{stroke}%
\end{pgfscope}%
\begin{pgfscope}%
\pgfpathrectangle{\pgfqpoint{0.100000in}{0.212622in}}{\pgfqpoint{3.696000in}{3.696000in}}%
\pgfusepath{clip}%
\pgfsetrectcap%
\pgfsetroundjoin%
\pgfsetlinewidth{1.505625pt}%
\definecolor{currentstroke}{rgb}{1.000000,0.000000,0.000000}%
\pgfsetstrokecolor{currentstroke}%
\pgfsetdash{}{0pt}%
\pgfpathmoveto{\pgfqpoint{1.790160in}{2.373233in}}%
\pgfpathlineto{\pgfqpoint{1.795822in}{2.045764in}}%
\pgfusepath{stroke}%
\end{pgfscope}%
\begin{pgfscope}%
\pgfpathrectangle{\pgfqpoint{0.100000in}{0.212622in}}{\pgfqpoint{3.696000in}{3.696000in}}%
\pgfusepath{clip}%
\pgfsetrectcap%
\pgfsetroundjoin%
\pgfsetlinewidth{1.505625pt}%
\definecolor{currentstroke}{rgb}{1.000000,0.000000,0.000000}%
\pgfsetstrokecolor{currentstroke}%
\pgfsetdash{}{0pt}%
\pgfpathmoveto{\pgfqpoint{1.800055in}{2.371832in}}%
\pgfpathlineto{\pgfqpoint{1.795822in}{2.045764in}}%
\pgfusepath{stroke}%
\end{pgfscope}%
\begin{pgfscope}%
\pgfpathrectangle{\pgfqpoint{0.100000in}{0.212622in}}{\pgfqpoint{3.696000in}{3.696000in}}%
\pgfusepath{clip}%
\pgfsetrectcap%
\pgfsetroundjoin%
\pgfsetlinewidth{1.505625pt}%
\definecolor{currentstroke}{rgb}{1.000000,0.000000,0.000000}%
\pgfsetstrokecolor{currentstroke}%
\pgfsetdash{}{0pt}%
\pgfpathmoveto{\pgfqpoint{1.818161in}{2.370564in}}%
\pgfpathlineto{\pgfqpoint{1.795822in}{2.045764in}}%
\pgfusepath{stroke}%
\end{pgfscope}%
\begin{pgfscope}%
\pgfpathrectangle{\pgfqpoint{0.100000in}{0.212622in}}{\pgfqpoint{3.696000in}{3.696000in}}%
\pgfusepath{clip}%
\pgfsetrectcap%
\pgfsetroundjoin%
\pgfsetlinewidth{1.505625pt}%
\definecolor{currentstroke}{rgb}{1.000000,0.000000,0.000000}%
\pgfsetstrokecolor{currentstroke}%
\pgfsetdash{}{0pt}%
\pgfpathmoveto{\pgfqpoint{1.842617in}{2.363516in}}%
\pgfpathlineto{\pgfqpoint{1.795822in}{2.045764in}}%
\pgfusepath{stroke}%
\end{pgfscope}%
\begin{pgfscope}%
\pgfpathrectangle{\pgfqpoint{0.100000in}{0.212622in}}{\pgfqpoint{3.696000in}{3.696000in}}%
\pgfusepath{clip}%
\pgfsetrectcap%
\pgfsetroundjoin%
\pgfsetlinewidth{1.505625pt}%
\definecolor{currentstroke}{rgb}{1.000000,0.000000,0.000000}%
\pgfsetstrokecolor{currentstroke}%
\pgfsetdash{}{0pt}%
\pgfpathmoveto{\pgfqpoint{1.873323in}{2.362067in}}%
\pgfpathlineto{\pgfqpoint{1.835574in}{2.034165in}}%
\pgfusepath{stroke}%
\end{pgfscope}%
\begin{pgfscope}%
\pgfpathrectangle{\pgfqpoint{0.100000in}{0.212622in}}{\pgfqpoint{3.696000in}{3.696000in}}%
\pgfusepath{clip}%
\pgfsetrectcap%
\pgfsetroundjoin%
\pgfsetlinewidth{1.505625pt}%
\definecolor{currentstroke}{rgb}{1.000000,0.000000,0.000000}%
\pgfsetstrokecolor{currentstroke}%
\pgfsetdash{}{0pt}%
\pgfpathmoveto{\pgfqpoint{1.905601in}{2.363422in}}%
\pgfpathlineto{\pgfqpoint{1.862119in}{2.026420in}}%
\pgfusepath{stroke}%
\end{pgfscope}%
\begin{pgfscope}%
\pgfpathrectangle{\pgfqpoint{0.100000in}{0.212622in}}{\pgfqpoint{3.696000in}{3.696000in}}%
\pgfusepath{clip}%
\pgfsetrectcap%
\pgfsetroundjoin%
\pgfsetlinewidth{1.505625pt}%
\definecolor{currentstroke}{rgb}{1.000000,0.000000,0.000000}%
\pgfsetstrokecolor{currentstroke}%
\pgfsetdash{}{0pt}%
\pgfpathmoveto{\pgfqpoint{1.945921in}{2.364690in}}%
\pgfpathlineto{\pgfqpoint{1.915317in}{2.010898in}}%
\pgfusepath{stroke}%
\end{pgfscope}%
\begin{pgfscope}%
\pgfpathrectangle{\pgfqpoint{0.100000in}{0.212622in}}{\pgfqpoint{3.696000in}{3.696000in}}%
\pgfusepath{clip}%
\pgfsetrectcap%
\pgfsetroundjoin%
\pgfsetlinewidth{1.505625pt}%
\definecolor{currentstroke}{rgb}{1.000000,0.000000,0.000000}%
\pgfsetstrokecolor{currentstroke}%
\pgfsetdash{}{0pt}%
\pgfpathmoveto{\pgfqpoint{1.968858in}{2.363314in}}%
\pgfpathlineto{\pgfqpoint{1.928638in}{2.007012in}}%
\pgfusepath{stroke}%
\end{pgfscope}%
\begin{pgfscope}%
\pgfpathrectangle{\pgfqpoint{0.100000in}{0.212622in}}{\pgfqpoint{3.696000in}{3.696000in}}%
\pgfusepath{clip}%
\pgfsetrectcap%
\pgfsetroundjoin%
\pgfsetlinewidth{1.505625pt}%
\definecolor{currentstroke}{rgb}{1.000000,0.000000,0.000000}%
\pgfsetstrokecolor{currentstroke}%
\pgfsetdash{}{0pt}%
\pgfpathmoveto{\pgfqpoint{1.999923in}{2.361574in}}%
\pgfpathlineto{\pgfqpoint{1.968656in}{1.995335in}}%
\pgfusepath{stroke}%
\end{pgfscope}%
\begin{pgfscope}%
\pgfpathrectangle{\pgfqpoint{0.100000in}{0.212622in}}{\pgfqpoint{3.696000in}{3.696000in}}%
\pgfusepath{clip}%
\pgfsetrectcap%
\pgfsetroundjoin%
\pgfsetlinewidth{1.505625pt}%
\definecolor{currentstroke}{rgb}{1.000000,0.000000,0.000000}%
\pgfsetstrokecolor{currentstroke}%
\pgfsetdash{}{0pt}%
\pgfpathmoveto{\pgfqpoint{2.015633in}{2.359314in}}%
\pgfpathlineto{\pgfqpoint{1.982013in}{1.991438in}}%
\pgfusepath{stroke}%
\end{pgfscope}%
\begin{pgfscope}%
\pgfpathrectangle{\pgfqpoint{0.100000in}{0.212622in}}{\pgfqpoint{3.696000in}{3.696000in}}%
\pgfusepath{clip}%
\pgfsetrectcap%
\pgfsetroundjoin%
\pgfsetlinewidth{1.505625pt}%
\definecolor{currentstroke}{rgb}{1.000000,0.000000,0.000000}%
\pgfsetstrokecolor{currentstroke}%
\pgfsetdash{}{0pt}%
\pgfpathmoveto{\pgfqpoint{2.039161in}{2.354632in}}%
\pgfpathlineto{\pgfqpoint{2.008755in}{1.983636in}}%
\pgfusepath{stroke}%
\end{pgfscope}%
\begin{pgfscope}%
\pgfpathrectangle{\pgfqpoint{0.100000in}{0.212622in}}{\pgfqpoint{3.696000in}{3.696000in}}%
\pgfusepath{clip}%
\pgfsetrectcap%
\pgfsetroundjoin%
\pgfsetlinewidth{1.505625pt}%
\definecolor{currentstroke}{rgb}{1.000000,0.000000,0.000000}%
\pgfsetstrokecolor{currentstroke}%
\pgfsetdash{}{0pt}%
\pgfpathmoveto{\pgfqpoint{2.052721in}{2.354903in}}%
\pgfpathlineto{\pgfqpoint{2.022139in}{1.979731in}}%
\pgfusepath{stroke}%
\end{pgfscope}%
\begin{pgfscope}%
\pgfpathrectangle{\pgfqpoint{0.100000in}{0.212622in}}{\pgfqpoint{3.696000in}{3.696000in}}%
\pgfusepath{clip}%
\pgfsetrectcap%
\pgfsetroundjoin%
\pgfsetlinewidth{1.505625pt}%
\definecolor{currentstroke}{rgb}{1.000000,0.000000,0.000000}%
\pgfsetstrokecolor{currentstroke}%
\pgfsetdash{}{0pt}%
\pgfpathmoveto{\pgfqpoint{2.069877in}{2.351267in}}%
\pgfpathlineto{\pgfqpoint{2.035532in}{1.975823in}}%
\pgfusepath{stroke}%
\end{pgfscope}%
\begin{pgfscope}%
\pgfpathrectangle{\pgfqpoint{0.100000in}{0.212622in}}{\pgfqpoint{3.696000in}{3.696000in}}%
\pgfusepath{clip}%
\pgfsetrectcap%
\pgfsetroundjoin%
\pgfsetlinewidth{1.505625pt}%
\definecolor{currentstroke}{rgb}{1.000000,0.000000,0.000000}%
\pgfsetstrokecolor{currentstroke}%
\pgfsetdash{}{0pt}%
\pgfpathmoveto{\pgfqpoint{2.092679in}{2.350523in}}%
\pgfpathlineto{\pgfqpoint{2.062345in}{1.968000in}}%
\pgfusepath{stroke}%
\end{pgfscope}%
\begin{pgfscope}%
\pgfpathrectangle{\pgfqpoint{0.100000in}{0.212622in}}{\pgfqpoint{3.696000in}{3.696000in}}%
\pgfusepath{clip}%
\pgfsetrectcap%
\pgfsetroundjoin%
\pgfsetlinewidth{1.505625pt}%
\definecolor{currentstroke}{rgb}{1.000000,0.000000,0.000000}%
\pgfsetstrokecolor{currentstroke}%
\pgfsetdash{}{0pt}%
\pgfpathmoveto{\pgfqpoint{2.121682in}{2.357047in}}%
\pgfpathlineto{\pgfqpoint{2.089194in}{1.960166in}}%
\pgfusepath{stroke}%
\end{pgfscope}%
\begin{pgfscope}%
\pgfpathrectangle{\pgfqpoint{0.100000in}{0.212622in}}{\pgfqpoint{3.696000in}{3.696000in}}%
\pgfusepath{clip}%
\pgfsetrectcap%
\pgfsetroundjoin%
\pgfsetlinewidth{1.505625pt}%
\definecolor{currentstroke}{rgb}{1.000000,0.000000,0.000000}%
\pgfsetstrokecolor{currentstroke}%
\pgfsetdash{}{0pt}%
\pgfpathmoveto{\pgfqpoint{2.150230in}{2.348622in}}%
\pgfpathlineto{\pgfqpoint{2.129535in}{1.948395in}}%
\pgfusepath{stroke}%
\end{pgfscope}%
\begin{pgfscope}%
\pgfpathrectangle{\pgfqpoint{0.100000in}{0.212622in}}{\pgfqpoint{3.696000in}{3.696000in}}%
\pgfusepath{clip}%
\pgfsetrectcap%
\pgfsetroundjoin%
\pgfsetlinewidth{1.505625pt}%
\definecolor{currentstroke}{rgb}{1.000000,0.000000,0.000000}%
\pgfsetstrokecolor{currentstroke}%
\pgfsetdash{}{0pt}%
\pgfpathmoveto{\pgfqpoint{2.189221in}{2.354429in}}%
\pgfpathlineto{\pgfqpoint{2.156474in}{1.940535in}}%
\pgfusepath{stroke}%
\end{pgfscope}%
\begin{pgfscope}%
\pgfpathrectangle{\pgfqpoint{0.100000in}{0.212622in}}{\pgfqpoint{3.696000in}{3.696000in}}%
\pgfusepath{clip}%
\pgfsetrectcap%
\pgfsetroundjoin%
\pgfsetlinewidth{1.505625pt}%
\definecolor{currentstroke}{rgb}{1.000000,0.000000,0.000000}%
\pgfsetstrokecolor{currentstroke}%
\pgfsetdash{}{0pt}%
\pgfpathmoveto{\pgfqpoint{2.229220in}{2.333922in}}%
\pgfpathlineto{\pgfqpoint{2.210461in}{1.924783in}}%
\pgfusepath{stroke}%
\end{pgfscope}%
\begin{pgfscope}%
\pgfpathrectangle{\pgfqpoint{0.100000in}{0.212622in}}{\pgfqpoint{3.696000in}{3.696000in}}%
\pgfusepath{clip}%
\pgfsetrectcap%
\pgfsetroundjoin%
\pgfsetlinewidth{1.505625pt}%
\definecolor{currentstroke}{rgb}{1.000000,0.000000,0.000000}%
\pgfsetstrokecolor{currentstroke}%
\pgfsetdash{}{0pt}%
\pgfpathmoveto{\pgfqpoint{2.254208in}{2.340503in}}%
\pgfpathlineto{\pgfqpoint{2.223981in}{1.920839in}}%
\pgfusepath{stroke}%
\end{pgfscope}%
\begin{pgfscope}%
\pgfpathrectangle{\pgfqpoint{0.100000in}{0.212622in}}{\pgfqpoint{3.696000in}{3.696000in}}%
\pgfusepath{clip}%
\pgfsetrectcap%
\pgfsetroundjoin%
\pgfsetlinewidth{1.505625pt}%
\definecolor{currentstroke}{rgb}{1.000000,0.000000,0.000000}%
\pgfsetstrokecolor{currentstroke}%
\pgfsetdash{}{0pt}%
\pgfpathmoveto{\pgfqpoint{2.284698in}{2.329634in}}%
\pgfpathlineto{\pgfqpoint{2.264594in}{1.908989in}}%
\pgfusepath{stroke}%
\end{pgfscope}%
\begin{pgfscope}%
\pgfpathrectangle{\pgfqpoint{0.100000in}{0.212622in}}{\pgfqpoint{3.696000in}{3.696000in}}%
\pgfusepath{clip}%
\pgfsetrectcap%
\pgfsetroundjoin%
\pgfsetlinewidth{1.505625pt}%
\definecolor{currentstroke}{rgb}{1.000000,0.000000,0.000000}%
\pgfsetstrokecolor{currentstroke}%
\pgfsetdash{}{0pt}%
\pgfpathmoveto{\pgfqpoint{2.322059in}{2.342759in}}%
\pgfpathlineto{\pgfqpoint{2.291715in}{1.901076in}}%
\pgfusepath{stroke}%
\end{pgfscope}%
\begin{pgfscope}%
\pgfpathrectangle{\pgfqpoint{0.100000in}{0.212622in}}{\pgfqpoint{3.696000in}{3.696000in}}%
\pgfusepath{clip}%
\pgfsetrectcap%
\pgfsetroundjoin%
\pgfsetlinewidth{1.505625pt}%
\definecolor{currentstroke}{rgb}{1.000000,0.000000,0.000000}%
\pgfsetstrokecolor{currentstroke}%
\pgfsetdash{}{0pt}%
\pgfpathmoveto{\pgfqpoint{2.339211in}{2.332368in}}%
\pgfpathlineto{\pgfqpoint{2.318873in}{1.893152in}}%
\pgfusepath{stroke}%
\end{pgfscope}%
\begin{pgfscope}%
\pgfpathrectangle{\pgfqpoint{0.100000in}{0.212622in}}{\pgfqpoint{3.696000in}{3.696000in}}%
\pgfusepath{clip}%
\pgfsetrectcap%
\pgfsetroundjoin%
\pgfsetlinewidth{1.505625pt}%
\definecolor{currentstroke}{rgb}{1.000000,0.000000,0.000000}%
\pgfsetstrokecolor{currentstroke}%
\pgfsetdash{}{0pt}%
\pgfpathmoveto{\pgfqpoint{2.350630in}{2.335807in}}%
\pgfpathlineto{\pgfqpoint{2.332465in}{1.889186in}}%
\pgfusepath{stroke}%
\end{pgfscope}%
\begin{pgfscope}%
\pgfpathrectangle{\pgfqpoint{0.100000in}{0.212622in}}{\pgfqpoint{3.696000in}{3.696000in}}%
\pgfusepath{clip}%
\pgfsetrectcap%
\pgfsetroundjoin%
\pgfsetlinewidth{1.505625pt}%
\definecolor{currentstroke}{rgb}{1.000000,0.000000,0.000000}%
\pgfsetstrokecolor{currentstroke}%
\pgfsetdash{}{0pt}%
\pgfpathmoveto{\pgfqpoint{2.355962in}{2.333303in}}%
\pgfpathlineto{\pgfqpoint{2.332465in}{1.889186in}}%
\pgfusepath{stroke}%
\end{pgfscope}%
\begin{pgfscope}%
\pgfpathrectangle{\pgfqpoint{0.100000in}{0.212622in}}{\pgfqpoint{3.696000in}{3.696000in}}%
\pgfusepath{clip}%
\pgfsetrectcap%
\pgfsetroundjoin%
\pgfsetlinewidth{1.505625pt}%
\definecolor{currentstroke}{rgb}{1.000000,0.000000,0.000000}%
\pgfsetstrokecolor{currentstroke}%
\pgfsetdash{}{0pt}%
\pgfpathmoveto{\pgfqpoint{2.359352in}{2.334187in}}%
\pgfpathlineto{\pgfqpoint{2.332465in}{1.889186in}}%
\pgfusepath{stroke}%
\end{pgfscope}%
\begin{pgfscope}%
\pgfpathrectangle{\pgfqpoint{0.100000in}{0.212622in}}{\pgfqpoint{3.696000in}{3.696000in}}%
\pgfusepath{clip}%
\pgfsetrectcap%
\pgfsetroundjoin%
\pgfsetlinewidth{1.505625pt}%
\definecolor{currentstroke}{rgb}{1.000000,0.000000,0.000000}%
\pgfsetstrokecolor{currentstroke}%
\pgfsetdash{}{0pt}%
\pgfpathmoveto{\pgfqpoint{2.364209in}{2.331866in}}%
\pgfpathlineto{\pgfqpoint{2.346067in}{1.885217in}}%
\pgfusepath{stroke}%
\end{pgfscope}%
\begin{pgfscope}%
\pgfpathrectangle{\pgfqpoint{0.100000in}{0.212622in}}{\pgfqpoint{3.696000in}{3.696000in}}%
\pgfusepath{clip}%
\pgfsetrectcap%
\pgfsetroundjoin%
\pgfsetlinewidth{1.505625pt}%
\definecolor{currentstroke}{rgb}{1.000000,0.000000,0.000000}%
\pgfsetstrokecolor{currentstroke}%
\pgfsetdash{}{0pt}%
\pgfpathmoveto{\pgfqpoint{2.374386in}{2.335356in}}%
\pgfpathlineto{\pgfqpoint{2.346067in}{1.885217in}}%
\pgfusepath{stroke}%
\end{pgfscope}%
\begin{pgfscope}%
\pgfpathrectangle{\pgfqpoint{0.100000in}{0.212622in}}{\pgfqpoint{3.696000in}{3.696000in}}%
\pgfusepath{clip}%
\pgfsetrectcap%
\pgfsetroundjoin%
\pgfsetlinewidth{1.505625pt}%
\definecolor{currentstroke}{rgb}{1.000000,0.000000,0.000000}%
\pgfsetstrokecolor{currentstroke}%
\pgfsetdash{}{0pt}%
\pgfpathmoveto{\pgfqpoint{2.385007in}{2.332710in}}%
\pgfpathlineto{\pgfqpoint{2.359678in}{1.881246in}}%
\pgfusepath{stroke}%
\end{pgfscope}%
\begin{pgfscope}%
\pgfpathrectangle{\pgfqpoint{0.100000in}{0.212622in}}{\pgfqpoint{3.696000in}{3.696000in}}%
\pgfusepath{clip}%
\pgfsetrectcap%
\pgfsetroundjoin%
\pgfsetlinewidth{1.505625pt}%
\definecolor{currentstroke}{rgb}{1.000000,0.000000,0.000000}%
\pgfsetstrokecolor{currentstroke}%
\pgfsetdash{}{0pt}%
\pgfpathmoveto{\pgfqpoint{2.400930in}{2.334964in}}%
\pgfpathlineto{\pgfqpoint{2.373298in}{1.877272in}}%
\pgfusepath{stroke}%
\end{pgfscope}%
\begin{pgfscope}%
\pgfpathrectangle{\pgfqpoint{0.100000in}{0.212622in}}{\pgfqpoint{3.696000in}{3.696000in}}%
\pgfusepath{clip}%
\pgfsetrectcap%
\pgfsetroundjoin%
\pgfsetlinewidth{1.505625pt}%
\definecolor{currentstroke}{rgb}{1.000000,0.000000,0.000000}%
\pgfsetstrokecolor{currentstroke}%
\pgfsetdash{}{0pt}%
\pgfpathmoveto{\pgfqpoint{2.409444in}{2.333765in}}%
\pgfpathlineto{\pgfqpoint{2.386928in}{1.873295in}}%
\pgfusepath{stroke}%
\end{pgfscope}%
\begin{pgfscope}%
\pgfpathrectangle{\pgfqpoint{0.100000in}{0.212622in}}{\pgfqpoint{3.696000in}{3.696000in}}%
\pgfusepath{clip}%
\pgfsetrectcap%
\pgfsetroundjoin%
\pgfsetlinewidth{1.505625pt}%
\definecolor{currentstroke}{rgb}{1.000000,0.000000,0.000000}%
\pgfsetstrokecolor{currentstroke}%
\pgfsetdash{}{0pt}%
\pgfpathmoveto{\pgfqpoint{2.423509in}{2.334798in}}%
\pgfpathlineto{\pgfqpoint{2.400566in}{1.869316in}}%
\pgfusepath{stroke}%
\end{pgfscope}%
\begin{pgfscope}%
\pgfpathrectangle{\pgfqpoint{0.100000in}{0.212622in}}{\pgfqpoint{3.696000in}{3.696000in}}%
\pgfusepath{clip}%
\pgfsetrectcap%
\pgfsetroundjoin%
\pgfsetlinewidth{1.505625pt}%
\definecolor{currentstroke}{rgb}{1.000000,0.000000,0.000000}%
\pgfsetstrokecolor{currentstroke}%
\pgfsetdash{}{0pt}%
\pgfpathmoveto{\pgfqpoint{2.438650in}{2.330474in}}%
\pgfpathlineto{\pgfqpoint{2.414214in}{1.865334in}}%
\pgfusepath{stroke}%
\end{pgfscope}%
\begin{pgfscope}%
\pgfpathrectangle{\pgfqpoint{0.100000in}{0.212622in}}{\pgfqpoint{3.696000in}{3.696000in}}%
\pgfusepath{clip}%
\pgfsetrectcap%
\pgfsetroundjoin%
\pgfsetlinewidth{1.505625pt}%
\definecolor{currentstroke}{rgb}{1.000000,0.000000,0.000000}%
\pgfsetstrokecolor{currentstroke}%
\pgfsetdash{}{0pt}%
\pgfpathmoveto{\pgfqpoint{2.463517in}{2.332776in}}%
\pgfpathlineto{\pgfqpoint{2.441537in}{1.857361in}}%
\pgfusepath{stroke}%
\end{pgfscope}%
\begin{pgfscope}%
\pgfpathrectangle{\pgfqpoint{0.100000in}{0.212622in}}{\pgfqpoint{3.696000in}{3.696000in}}%
\pgfusepath{clip}%
\pgfsetrectcap%
\pgfsetroundjoin%
\pgfsetlinewidth{1.505625pt}%
\definecolor{currentstroke}{rgb}{1.000000,0.000000,0.000000}%
\pgfsetstrokecolor{currentstroke}%
\pgfsetdash{}{0pt}%
\pgfpathmoveto{\pgfqpoint{2.476034in}{2.330449in}}%
\pgfpathlineto{\pgfqpoint{2.455213in}{1.853371in}}%
\pgfusepath{stroke}%
\end{pgfscope}%
\begin{pgfscope}%
\pgfpathrectangle{\pgfqpoint{0.100000in}{0.212622in}}{\pgfqpoint{3.696000in}{3.696000in}}%
\pgfusepath{clip}%
\pgfsetrectcap%
\pgfsetroundjoin%
\pgfsetlinewidth{1.505625pt}%
\definecolor{currentstroke}{rgb}{1.000000,0.000000,0.000000}%
\pgfsetstrokecolor{currentstroke}%
\pgfsetdash{}{0pt}%
\pgfpathmoveto{\pgfqpoint{2.493633in}{2.332720in}}%
\pgfpathlineto{\pgfqpoint{2.468898in}{1.849378in}}%
\pgfusepath{stroke}%
\end{pgfscope}%
\begin{pgfscope}%
\pgfpathrectangle{\pgfqpoint{0.100000in}{0.212622in}}{\pgfqpoint{3.696000in}{3.696000in}}%
\pgfusepath{clip}%
\pgfsetrectcap%
\pgfsetroundjoin%
\pgfsetlinewidth{1.505625pt}%
\definecolor{currentstroke}{rgb}{1.000000,0.000000,0.000000}%
\pgfsetstrokecolor{currentstroke}%
\pgfsetdash{}{0pt}%
\pgfpathmoveto{\pgfqpoint{2.509696in}{2.324093in}}%
\pgfpathlineto{\pgfqpoint{2.496295in}{1.841385in}}%
\pgfusepath{stroke}%
\end{pgfscope}%
\begin{pgfscope}%
\pgfpathrectangle{\pgfqpoint{0.100000in}{0.212622in}}{\pgfqpoint{3.696000in}{3.696000in}}%
\pgfusepath{clip}%
\pgfsetrectcap%
\pgfsetroundjoin%
\pgfsetlinewidth{1.505625pt}%
\definecolor{currentstroke}{rgb}{1.000000,0.000000,0.000000}%
\pgfsetstrokecolor{currentstroke}%
\pgfsetdash{}{0pt}%
\pgfpathmoveto{\pgfqpoint{2.533270in}{2.327371in}}%
\pgfpathlineto{\pgfqpoint{2.523730in}{1.833380in}}%
\pgfusepath{stroke}%
\end{pgfscope}%
\begin{pgfscope}%
\pgfpathrectangle{\pgfqpoint{0.100000in}{0.212622in}}{\pgfqpoint{3.696000in}{3.696000in}}%
\pgfusepath{clip}%
\pgfsetrectcap%
\pgfsetroundjoin%
\pgfsetlinewidth{1.505625pt}%
\definecolor{currentstroke}{rgb}{1.000000,0.000000,0.000000}%
\pgfsetstrokecolor{currentstroke}%
\pgfsetdash{}{0pt}%
\pgfpathmoveto{\pgfqpoint{2.554501in}{2.321949in}}%
\pgfpathlineto{\pgfqpoint{2.537461in}{1.829374in}}%
\pgfusepath{stroke}%
\end{pgfscope}%
\begin{pgfscope}%
\pgfpathrectangle{\pgfqpoint{0.100000in}{0.212622in}}{\pgfqpoint{3.696000in}{3.696000in}}%
\pgfusepath{clip}%
\pgfsetrectcap%
\pgfsetroundjoin%
\pgfsetlinewidth{1.505625pt}%
\definecolor{currentstroke}{rgb}{1.000000,0.000000,0.000000}%
\pgfsetstrokecolor{currentstroke}%
\pgfsetdash{}{0pt}%
\pgfpathmoveto{\pgfqpoint{2.567954in}{2.323266in}}%
\pgfpathlineto{\pgfqpoint{2.551202in}{1.825364in}}%
\pgfusepath{stroke}%
\end{pgfscope}%
\begin{pgfscope}%
\pgfpathrectangle{\pgfqpoint{0.100000in}{0.212622in}}{\pgfqpoint{3.696000in}{3.696000in}}%
\pgfusepath{clip}%
\pgfsetrectcap%
\pgfsetroundjoin%
\pgfsetlinewidth{1.505625pt}%
\definecolor{currentstroke}{rgb}{1.000000,0.000000,0.000000}%
\pgfsetstrokecolor{currentstroke}%
\pgfsetdash{}{0pt}%
\pgfpathmoveto{\pgfqpoint{2.583359in}{2.321428in}}%
\pgfpathlineto{\pgfqpoint{2.578711in}{1.817338in}}%
\pgfusepath{stroke}%
\end{pgfscope}%
\begin{pgfscope}%
\pgfpathrectangle{\pgfqpoint{0.100000in}{0.212622in}}{\pgfqpoint{3.696000in}{3.696000in}}%
\pgfusepath{clip}%
\pgfsetrectcap%
\pgfsetroundjoin%
\pgfsetlinewidth{1.505625pt}%
\definecolor{currentstroke}{rgb}{1.000000,0.000000,0.000000}%
\pgfsetstrokecolor{currentstroke}%
\pgfsetdash{}{0pt}%
\pgfpathmoveto{\pgfqpoint{2.602977in}{2.322608in}}%
\pgfpathlineto{\pgfqpoint{2.592479in}{1.813321in}}%
\pgfusepath{stroke}%
\end{pgfscope}%
\begin{pgfscope}%
\pgfpathrectangle{\pgfqpoint{0.100000in}{0.212622in}}{\pgfqpoint{3.696000in}{3.696000in}}%
\pgfusepath{clip}%
\pgfsetrectcap%
\pgfsetroundjoin%
\pgfsetlinewidth{1.505625pt}%
\definecolor{currentstroke}{rgb}{1.000000,0.000000,0.000000}%
\pgfsetstrokecolor{currentstroke}%
\pgfsetdash{}{0pt}%
\pgfpathmoveto{\pgfqpoint{2.613995in}{2.322554in}}%
\pgfpathlineto{\pgfqpoint{2.606257in}{1.809301in}}%
\pgfusepath{stroke}%
\end{pgfscope}%
\begin{pgfscope}%
\pgfpathrectangle{\pgfqpoint{0.100000in}{0.212622in}}{\pgfqpoint{3.696000in}{3.696000in}}%
\pgfusepath{clip}%
\pgfsetrectcap%
\pgfsetroundjoin%
\pgfsetlinewidth{1.505625pt}%
\definecolor{currentstroke}{rgb}{1.000000,0.000000,0.000000}%
\pgfsetstrokecolor{currentstroke}%
\pgfsetdash{}{0pt}%
\pgfpathmoveto{\pgfqpoint{2.620426in}{2.321983in}}%
\pgfpathlineto{\pgfqpoint{2.606257in}{1.809301in}}%
\pgfusepath{stroke}%
\end{pgfscope}%
\begin{pgfscope}%
\pgfpathrectangle{\pgfqpoint{0.100000in}{0.212622in}}{\pgfqpoint{3.696000in}{3.696000in}}%
\pgfusepath{clip}%
\pgfsetrectcap%
\pgfsetroundjoin%
\pgfsetlinewidth{1.505625pt}%
\definecolor{currentstroke}{rgb}{1.000000,0.000000,0.000000}%
\pgfsetstrokecolor{currentstroke}%
\pgfsetdash{}{0pt}%
\pgfpathmoveto{\pgfqpoint{2.628931in}{2.323740in}}%
\pgfpathlineto{\pgfqpoint{2.620044in}{1.805278in}}%
\pgfusepath{stroke}%
\end{pgfscope}%
\begin{pgfscope}%
\pgfpathrectangle{\pgfqpoint{0.100000in}{0.212622in}}{\pgfqpoint{3.696000in}{3.696000in}}%
\pgfusepath{clip}%
\pgfsetrectcap%
\pgfsetroundjoin%
\pgfsetlinewidth{1.505625pt}%
\definecolor{currentstroke}{rgb}{1.000000,0.000000,0.000000}%
\pgfsetstrokecolor{currentstroke}%
\pgfsetdash{}{0pt}%
\pgfpathmoveto{\pgfqpoint{2.643425in}{2.321297in}}%
\pgfpathlineto{\pgfqpoint{2.633841in}{1.801252in}}%
\pgfusepath{stroke}%
\end{pgfscope}%
\begin{pgfscope}%
\pgfpathrectangle{\pgfqpoint{0.100000in}{0.212622in}}{\pgfqpoint{3.696000in}{3.696000in}}%
\pgfusepath{clip}%
\pgfsetrectcap%
\pgfsetroundjoin%
\pgfsetlinewidth{1.505625pt}%
\definecolor{currentstroke}{rgb}{1.000000,0.000000,0.000000}%
\pgfsetstrokecolor{currentstroke}%
\pgfsetdash{}{0pt}%
\pgfpathmoveto{\pgfqpoint{2.660963in}{2.324575in}}%
\pgfpathlineto{\pgfqpoint{2.647647in}{1.797224in}}%
\pgfusepath{stroke}%
\end{pgfscope}%
\begin{pgfscope}%
\pgfpathrectangle{\pgfqpoint{0.100000in}{0.212622in}}{\pgfqpoint{3.696000in}{3.696000in}}%
\pgfusepath{clip}%
\pgfsetrectcap%
\pgfsetroundjoin%
\pgfsetlinewidth{1.505625pt}%
\definecolor{currentstroke}{rgb}{1.000000,0.000000,0.000000}%
\pgfsetstrokecolor{currentstroke}%
\pgfsetdash{}{0pt}%
\pgfpathmoveto{\pgfqpoint{2.680969in}{2.320377in}}%
\pgfpathlineto{\pgfqpoint{2.675287in}{1.789160in}}%
\pgfusepath{stroke}%
\end{pgfscope}%
\begin{pgfscope}%
\pgfpathrectangle{\pgfqpoint{0.100000in}{0.212622in}}{\pgfqpoint{3.696000in}{3.696000in}}%
\pgfusepath{clip}%
\pgfsetrectcap%
\pgfsetroundjoin%
\pgfsetlinewidth{1.505625pt}%
\definecolor{currentstroke}{rgb}{1.000000,0.000000,0.000000}%
\pgfsetstrokecolor{currentstroke}%
\pgfsetdash{}{0pt}%
\pgfpathmoveto{\pgfqpoint{2.693582in}{2.323380in}}%
\pgfpathlineto{\pgfqpoint{2.689122in}{1.785123in}}%
\pgfusepath{stroke}%
\end{pgfscope}%
\begin{pgfscope}%
\pgfpathrectangle{\pgfqpoint{0.100000in}{0.212622in}}{\pgfqpoint{3.696000in}{3.696000in}}%
\pgfusepath{clip}%
\pgfsetrectcap%
\pgfsetroundjoin%
\pgfsetlinewidth{1.505625pt}%
\definecolor{currentstroke}{rgb}{1.000000,0.000000,0.000000}%
\pgfsetstrokecolor{currentstroke}%
\pgfsetdash{}{0pt}%
\pgfpathmoveto{\pgfqpoint{2.707530in}{2.318415in}}%
\pgfpathlineto{\pgfqpoint{2.702965in}{1.781084in}}%
\pgfusepath{stroke}%
\end{pgfscope}%
\begin{pgfscope}%
\pgfpathrectangle{\pgfqpoint{0.100000in}{0.212622in}}{\pgfqpoint{3.696000in}{3.696000in}}%
\pgfusepath{clip}%
\pgfsetrectcap%
\pgfsetroundjoin%
\pgfsetlinewidth{1.505625pt}%
\definecolor{currentstroke}{rgb}{1.000000,0.000000,0.000000}%
\pgfsetstrokecolor{currentstroke}%
\pgfsetdash{}{0pt}%
\pgfpathmoveto{\pgfqpoint{2.726375in}{2.322237in}}%
\pgfpathlineto{\pgfqpoint{2.716819in}{1.777042in}}%
\pgfusepath{stroke}%
\end{pgfscope}%
\begin{pgfscope}%
\pgfpathrectangle{\pgfqpoint{0.100000in}{0.212622in}}{\pgfqpoint{3.696000in}{3.696000in}}%
\pgfusepath{clip}%
\pgfsetrectcap%
\pgfsetroundjoin%
\pgfsetlinewidth{1.505625pt}%
\definecolor{currentstroke}{rgb}{1.000000,0.000000,0.000000}%
\pgfsetstrokecolor{currentstroke}%
\pgfsetdash{}{0pt}%
\pgfpathmoveto{\pgfqpoint{2.735648in}{2.319942in}}%
\pgfpathlineto{\pgfqpoint{2.730681in}{1.772997in}}%
\pgfusepath{stroke}%
\end{pgfscope}%
\begin{pgfscope}%
\pgfpathrectangle{\pgfqpoint{0.100000in}{0.212622in}}{\pgfqpoint{3.696000in}{3.696000in}}%
\pgfusepath{clip}%
\pgfsetrectcap%
\pgfsetroundjoin%
\pgfsetlinewidth{1.505625pt}%
\definecolor{currentstroke}{rgb}{1.000000,0.000000,0.000000}%
\pgfsetstrokecolor{currentstroke}%
\pgfsetdash{}{0pt}%
\pgfpathmoveto{\pgfqpoint{2.750327in}{2.316736in}}%
\pgfpathlineto{\pgfqpoint{2.744553in}{1.768950in}}%
\pgfusepath{stroke}%
\end{pgfscope}%
\begin{pgfscope}%
\pgfpathrectangle{\pgfqpoint{0.100000in}{0.212622in}}{\pgfqpoint{3.696000in}{3.696000in}}%
\pgfusepath{clip}%
\pgfsetrectcap%
\pgfsetroundjoin%
\pgfsetlinewidth{1.505625pt}%
\definecolor{currentstroke}{rgb}{1.000000,0.000000,0.000000}%
\pgfsetstrokecolor{currentstroke}%
\pgfsetdash{}{0pt}%
\pgfpathmoveto{\pgfqpoint{2.759416in}{2.319709in}}%
\pgfpathlineto{\pgfqpoint{2.758435in}{1.764899in}}%
\pgfusepath{stroke}%
\end{pgfscope}%
\begin{pgfscope}%
\pgfpathrectangle{\pgfqpoint{0.100000in}{0.212622in}}{\pgfqpoint{3.696000in}{3.696000in}}%
\pgfusepath{clip}%
\pgfsetrectcap%
\pgfsetroundjoin%
\pgfsetlinewidth{1.505625pt}%
\definecolor{currentstroke}{rgb}{1.000000,0.000000,0.000000}%
\pgfsetstrokecolor{currentstroke}%
\pgfsetdash{}{0pt}%
\pgfpathmoveto{\pgfqpoint{2.763765in}{2.318552in}}%
\pgfpathlineto{\pgfqpoint{2.758435in}{1.764899in}}%
\pgfusepath{stroke}%
\end{pgfscope}%
\begin{pgfscope}%
\pgfpathrectangle{\pgfqpoint{0.100000in}{0.212622in}}{\pgfqpoint{3.696000in}{3.696000in}}%
\pgfusepath{clip}%
\pgfsetrectcap%
\pgfsetroundjoin%
\pgfsetlinewidth{1.505625pt}%
\definecolor{currentstroke}{rgb}{1.000000,0.000000,0.000000}%
\pgfsetstrokecolor{currentstroke}%
\pgfsetdash{}{0pt}%
\pgfpathmoveto{\pgfqpoint{2.770632in}{2.321077in}}%
\pgfpathlineto{\pgfqpoint{2.758435in}{1.764899in}}%
\pgfusepath{stroke}%
\end{pgfscope}%
\begin{pgfscope}%
\pgfpathrectangle{\pgfqpoint{0.100000in}{0.212622in}}{\pgfqpoint{3.696000in}{3.696000in}}%
\pgfusepath{clip}%
\pgfsetrectcap%
\pgfsetroundjoin%
\pgfsetlinewidth{1.505625pt}%
\definecolor{currentstroke}{rgb}{1.000000,0.000000,0.000000}%
\pgfsetstrokecolor{currentstroke}%
\pgfsetdash{}{0pt}%
\pgfpathmoveto{\pgfqpoint{2.778375in}{2.317259in}}%
\pgfpathlineto{\pgfqpoint{2.772326in}{1.760846in}}%
\pgfusepath{stroke}%
\end{pgfscope}%
\begin{pgfscope}%
\pgfpathrectangle{\pgfqpoint{0.100000in}{0.212622in}}{\pgfqpoint{3.696000in}{3.696000in}}%
\pgfusepath{clip}%
\pgfsetrectcap%
\pgfsetroundjoin%
\pgfsetlinewidth{1.505625pt}%
\definecolor{currentstroke}{rgb}{1.000000,0.000000,0.000000}%
\pgfsetstrokecolor{currentstroke}%
\pgfsetdash{}{0pt}%
\pgfpathmoveto{\pgfqpoint{2.791109in}{2.319552in}}%
\pgfpathlineto{\pgfqpoint{2.786226in}{1.756791in}}%
\pgfusepath{stroke}%
\end{pgfscope}%
\begin{pgfscope}%
\pgfpathrectangle{\pgfqpoint{0.100000in}{0.212622in}}{\pgfqpoint{3.696000in}{3.696000in}}%
\pgfusepath{clip}%
\pgfsetrectcap%
\pgfsetroundjoin%
\pgfsetlinewidth{1.505625pt}%
\definecolor{currentstroke}{rgb}{1.000000,0.000000,0.000000}%
\pgfsetstrokecolor{currentstroke}%
\pgfsetdash{}{0pt}%
\pgfpathmoveto{\pgfqpoint{2.797493in}{2.317899in}}%
\pgfpathlineto{\pgfqpoint{2.786226in}{1.756791in}}%
\pgfusepath{stroke}%
\end{pgfscope}%
\begin{pgfscope}%
\pgfpathrectangle{\pgfqpoint{0.100000in}{0.212622in}}{\pgfqpoint{3.696000in}{3.696000in}}%
\pgfusepath{clip}%
\pgfsetrectcap%
\pgfsetroundjoin%
\pgfsetlinewidth{1.505625pt}%
\definecolor{currentstroke}{rgb}{1.000000,0.000000,0.000000}%
\pgfsetstrokecolor{currentstroke}%
\pgfsetdash{}{0pt}%
\pgfpathmoveto{\pgfqpoint{2.801364in}{2.318725in}}%
\pgfpathlineto{\pgfqpoint{2.800136in}{1.752732in}}%
\pgfusepath{stroke}%
\end{pgfscope}%
\begin{pgfscope}%
\pgfpathrectangle{\pgfqpoint{0.100000in}{0.212622in}}{\pgfqpoint{3.696000in}{3.696000in}}%
\pgfusepath{clip}%
\pgfsetrectcap%
\pgfsetroundjoin%
\pgfsetlinewidth{1.505625pt}%
\definecolor{currentstroke}{rgb}{1.000000,0.000000,0.000000}%
\pgfsetstrokecolor{currentstroke}%
\pgfsetdash{}{0pt}%
\pgfpathmoveto{\pgfqpoint{2.803199in}{2.318179in}}%
\pgfpathlineto{\pgfqpoint{2.800136in}{1.752732in}}%
\pgfusepath{stroke}%
\end{pgfscope}%
\begin{pgfscope}%
\pgfpathrectangle{\pgfqpoint{0.100000in}{0.212622in}}{\pgfqpoint{3.696000in}{3.696000in}}%
\pgfusepath{clip}%
\pgfsetrectcap%
\pgfsetroundjoin%
\pgfsetlinewidth{1.505625pt}%
\definecolor{currentstroke}{rgb}{1.000000,0.000000,0.000000}%
\pgfsetstrokecolor{currentstroke}%
\pgfsetdash{}{0pt}%
\pgfpathmoveto{\pgfqpoint{2.810371in}{2.318337in}}%
\pgfpathlineto{\pgfqpoint{2.800136in}{1.752732in}}%
\pgfusepath{stroke}%
\end{pgfscope}%
\begin{pgfscope}%
\pgfpathrectangle{\pgfqpoint{0.100000in}{0.212622in}}{\pgfqpoint{3.696000in}{3.696000in}}%
\pgfusepath{clip}%
\pgfsetrectcap%
\pgfsetroundjoin%
\pgfsetlinewidth{1.505625pt}%
\definecolor{currentstroke}{rgb}{1.000000,0.000000,0.000000}%
\pgfsetstrokecolor{currentstroke}%
\pgfsetdash{}{0pt}%
\pgfpathmoveto{\pgfqpoint{2.823270in}{2.318659in}}%
\pgfpathlineto{\pgfqpoint{2.814055in}{1.748671in}}%
\pgfusepath{stroke}%
\end{pgfscope}%
\begin{pgfscope}%
\pgfpathrectangle{\pgfqpoint{0.100000in}{0.212622in}}{\pgfqpoint{3.696000in}{3.696000in}}%
\pgfusepath{clip}%
\pgfsetrectcap%
\pgfsetroundjoin%
\pgfsetlinewidth{1.505625pt}%
\definecolor{currentstroke}{rgb}{1.000000,0.000000,0.000000}%
\pgfsetstrokecolor{currentstroke}%
\pgfsetdash{}{0pt}%
\pgfpathmoveto{\pgfqpoint{2.842238in}{2.314966in}}%
\pgfpathlineto{\pgfqpoint{2.841923in}{1.740540in}}%
\pgfusepath{stroke}%
\end{pgfscope}%
\begin{pgfscope}%
\pgfpathrectangle{\pgfqpoint{0.100000in}{0.212622in}}{\pgfqpoint{3.696000in}{3.696000in}}%
\pgfusepath{clip}%
\pgfsetrectcap%
\pgfsetroundjoin%
\pgfsetlinewidth{1.505625pt}%
\definecolor{currentstroke}{rgb}{1.000000,0.000000,0.000000}%
\pgfsetstrokecolor{currentstroke}%
\pgfsetdash{}{0pt}%
\pgfpathmoveto{\pgfqpoint{2.864753in}{2.317627in}}%
\pgfpathlineto{\pgfqpoint{2.855871in}{1.736470in}}%
\pgfusepath{stroke}%
\end{pgfscope}%
\begin{pgfscope}%
\pgfpathrectangle{\pgfqpoint{0.100000in}{0.212622in}}{\pgfqpoint{3.696000in}{3.696000in}}%
\pgfusepath{clip}%
\pgfsetrectcap%
\pgfsetroundjoin%
\pgfsetlinewidth{1.505625pt}%
\definecolor{currentstroke}{rgb}{1.000000,0.000000,0.000000}%
\pgfsetstrokecolor{currentstroke}%
\pgfsetdash{}{0pt}%
\pgfpathmoveto{\pgfqpoint{2.876255in}{2.315960in}}%
\pgfpathlineto{\pgfqpoint{2.869828in}{1.732398in}}%
\pgfusepath{stroke}%
\end{pgfscope}%
\begin{pgfscope}%
\pgfpathrectangle{\pgfqpoint{0.100000in}{0.212622in}}{\pgfqpoint{3.696000in}{3.696000in}}%
\pgfusepath{clip}%
\pgfsetrectcap%
\pgfsetroundjoin%
\pgfsetlinewidth{1.505625pt}%
\definecolor{currentstroke}{rgb}{1.000000,0.000000,0.000000}%
\pgfsetstrokecolor{currentstroke}%
\pgfsetdash{}{0pt}%
\pgfpathmoveto{\pgfqpoint{2.893089in}{2.319673in}}%
\pgfpathlineto{\pgfqpoint{2.883795in}{1.728323in}}%
\pgfusepath{stroke}%
\end{pgfscope}%
\begin{pgfscope}%
\pgfpathrectangle{\pgfqpoint{0.100000in}{0.212622in}}{\pgfqpoint{3.696000in}{3.696000in}}%
\pgfusepath{clip}%
\pgfsetrectcap%
\pgfsetroundjoin%
\pgfsetlinewidth{1.505625pt}%
\definecolor{currentstroke}{rgb}{1.000000,0.000000,0.000000}%
\pgfsetstrokecolor{currentstroke}%
\pgfsetdash{}{0pt}%
\pgfpathmoveto{\pgfqpoint{2.901661in}{2.317732in}}%
\pgfpathlineto{\pgfqpoint{2.897772in}{1.724245in}}%
\pgfusepath{stroke}%
\end{pgfscope}%
\begin{pgfscope}%
\pgfpathrectangle{\pgfqpoint{0.100000in}{0.212622in}}{\pgfqpoint{3.696000in}{3.696000in}}%
\pgfusepath{clip}%
\pgfsetrectcap%
\pgfsetroundjoin%
\pgfsetlinewidth{1.505625pt}%
\definecolor{currentstroke}{rgb}{1.000000,0.000000,0.000000}%
\pgfsetstrokecolor{currentstroke}%
\pgfsetdash{}{0pt}%
\pgfpathmoveto{\pgfqpoint{2.915327in}{2.320604in}}%
\pgfpathlineto{\pgfqpoint{2.911758in}{1.720164in}}%
\pgfusepath{stroke}%
\end{pgfscope}%
\begin{pgfscope}%
\pgfpathrectangle{\pgfqpoint{0.100000in}{0.212622in}}{\pgfqpoint{3.696000in}{3.696000in}}%
\pgfusepath{clip}%
\pgfsetrectcap%
\pgfsetroundjoin%
\pgfsetlinewidth{1.505625pt}%
\definecolor{currentstroke}{rgb}{1.000000,0.000000,0.000000}%
\pgfsetstrokecolor{currentstroke}%
\pgfsetdash{}{0pt}%
\pgfpathmoveto{\pgfqpoint{2.930813in}{2.316505in}}%
\pgfpathlineto{\pgfqpoint{2.925754in}{1.716080in}}%
\pgfusepath{stroke}%
\end{pgfscope}%
\begin{pgfscope}%
\pgfpathrectangle{\pgfqpoint{0.100000in}{0.212622in}}{\pgfqpoint{3.696000in}{3.696000in}}%
\pgfusepath{clip}%
\pgfsetrectcap%
\pgfsetroundjoin%
\pgfsetlinewidth{1.505625pt}%
\definecolor{currentstroke}{rgb}{1.000000,0.000000,0.000000}%
\pgfsetstrokecolor{currentstroke}%
\pgfsetdash{}{0pt}%
\pgfpathmoveto{\pgfqpoint{2.939837in}{2.315458in}}%
\pgfpathlineto{\pgfqpoint{2.939759in}{1.711994in}}%
\pgfusepath{stroke}%
\end{pgfscope}%
\begin{pgfscope}%
\pgfpathrectangle{\pgfqpoint{0.100000in}{0.212622in}}{\pgfqpoint{3.696000in}{3.696000in}}%
\pgfusepath{clip}%
\pgfsetrectcap%
\pgfsetroundjoin%
\pgfsetlinewidth{1.505625pt}%
\definecolor{currentstroke}{rgb}{1.000000,0.000000,0.000000}%
\pgfsetstrokecolor{currentstroke}%
\pgfsetdash{}{0pt}%
\pgfpathmoveto{\pgfqpoint{2.953835in}{2.315934in}}%
\pgfpathlineto{\pgfqpoint{2.953774in}{1.707905in}}%
\pgfusepath{stroke}%
\end{pgfscope}%
\begin{pgfscope}%
\pgfpathrectangle{\pgfqpoint{0.100000in}{0.212622in}}{\pgfqpoint{3.696000in}{3.696000in}}%
\pgfusepath{clip}%
\pgfsetrectcap%
\pgfsetroundjoin%
\pgfsetlinewidth{1.505625pt}%
\definecolor{currentstroke}{rgb}{1.000000,0.000000,0.000000}%
\pgfsetstrokecolor{currentstroke}%
\pgfsetdash{}{0pt}%
\pgfpathmoveto{\pgfqpoint{2.961323in}{2.315159in}}%
\pgfpathlineto{\pgfqpoint{2.953774in}{1.707905in}}%
\pgfusepath{stroke}%
\end{pgfscope}%
\begin{pgfscope}%
\pgfpathrectangle{\pgfqpoint{0.100000in}{0.212622in}}{\pgfqpoint{3.696000in}{3.696000in}}%
\pgfusepath{clip}%
\pgfsetrectcap%
\pgfsetroundjoin%
\pgfsetlinewidth{1.505625pt}%
\definecolor{currentstroke}{rgb}{1.000000,0.000000,0.000000}%
\pgfsetstrokecolor{currentstroke}%
\pgfsetdash{}{0pt}%
\pgfpathmoveto{\pgfqpoint{2.972486in}{2.317223in}}%
\pgfpathlineto{\pgfqpoint{2.967799in}{1.703813in}}%
\pgfusepath{stroke}%
\end{pgfscope}%
\begin{pgfscope}%
\pgfpathrectangle{\pgfqpoint{0.100000in}{0.212622in}}{\pgfqpoint{3.696000in}{3.696000in}}%
\pgfusepath{clip}%
\pgfsetrectcap%
\pgfsetroundjoin%
\pgfsetlinewidth{1.505625pt}%
\definecolor{currentstroke}{rgb}{1.000000,0.000000,0.000000}%
\pgfsetstrokecolor{currentstroke}%
\pgfsetdash{}{0pt}%
\pgfpathmoveto{\pgfqpoint{2.989859in}{2.314265in}}%
\pgfpathlineto{\pgfqpoint{2.981833in}{1.699718in}}%
\pgfusepath{stroke}%
\end{pgfscope}%
\begin{pgfscope}%
\pgfpathrectangle{\pgfqpoint{0.100000in}{0.212622in}}{\pgfqpoint{3.696000in}{3.696000in}}%
\pgfusepath{clip}%
\pgfsetrectcap%
\pgfsetroundjoin%
\pgfsetlinewidth{1.505625pt}%
\definecolor{currentstroke}{rgb}{1.000000,0.000000,0.000000}%
\pgfsetstrokecolor{currentstroke}%
\pgfsetdash{}{0pt}%
\pgfpathmoveto{\pgfqpoint{3.009801in}{2.315248in}}%
\pgfpathlineto{\pgfqpoint{3.009930in}{1.691520in}}%
\pgfusepath{stroke}%
\end{pgfscope}%
\begin{pgfscope}%
\pgfpathrectangle{\pgfqpoint{0.100000in}{0.212622in}}{\pgfqpoint{3.696000in}{3.696000in}}%
\pgfusepath{clip}%
\pgfsetrectcap%
\pgfsetroundjoin%
\pgfsetlinewidth{1.505625pt}%
\definecolor{currentstroke}{rgb}{1.000000,0.000000,0.000000}%
\pgfsetstrokecolor{currentstroke}%
\pgfsetdash{}{0pt}%
\pgfpathmoveto{\pgfqpoint{3.032017in}{2.311989in}}%
\pgfpathlineto{\pgfqpoint{3.023993in}{1.687417in}}%
\pgfusepath{stroke}%
\end{pgfscope}%
\begin{pgfscope}%
\pgfpathrectangle{\pgfqpoint{0.100000in}{0.212622in}}{\pgfqpoint{3.696000in}{3.696000in}}%
\pgfusepath{clip}%
\pgfsetrectcap%
\pgfsetroundjoin%
\pgfsetlinewidth{1.505625pt}%
\definecolor{currentstroke}{rgb}{1.000000,0.000000,0.000000}%
\pgfsetstrokecolor{currentstroke}%
\pgfsetdash{}{0pt}%
\pgfpathmoveto{\pgfqpoint{3.060012in}{2.315919in}}%
\pgfpathlineto{\pgfqpoint{3.052149in}{1.679202in}}%
\pgfusepath{stroke}%
\end{pgfscope}%
\begin{pgfscope}%
\pgfpathrectangle{\pgfqpoint{0.100000in}{0.212622in}}{\pgfqpoint{3.696000in}{3.696000in}}%
\pgfusepath{clip}%
\pgfsetrectcap%
\pgfsetroundjoin%
\pgfsetlinewidth{1.505625pt}%
\definecolor{currentstroke}{rgb}{1.000000,0.000000,0.000000}%
\pgfsetstrokecolor{currentstroke}%
\pgfsetdash{}{0pt}%
\pgfpathmoveto{\pgfqpoint{3.088619in}{2.309748in}}%
\pgfpathlineto{\pgfqpoint{3.080343in}{1.670975in}}%
\pgfusepath{stroke}%
\end{pgfscope}%
\begin{pgfscope}%
\pgfpathrectangle{\pgfqpoint{0.100000in}{0.212622in}}{\pgfqpoint{3.696000in}{3.696000in}}%
\pgfusepath{clip}%
\pgfsetrectcap%
\pgfsetroundjoin%
\pgfsetlinewidth{1.505625pt}%
\definecolor{currentstroke}{rgb}{1.000000,0.000000,0.000000}%
\pgfsetstrokecolor{currentstroke}%
\pgfsetdash{}{0pt}%
\pgfpathmoveto{\pgfqpoint{3.126861in}{2.306567in}}%
\pgfpathlineto{\pgfqpoint{3.127682in}{1.627373in}}%
\pgfusepath{stroke}%
\end{pgfscope}%
\begin{pgfscope}%
\pgfpathrectangle{\pgfqpoint{0.100000in}{0.212622in}}{\pgfqpoint{3.696000in}{3.696000in}}%
\pgfusepath{clip}%
\pgfsetrectcap%
\pgfsetroundjoin%
\pgfsetlinewidth{1.505625pt}%
\definecolor{currentstroke}{rgb}{1.000000,0.000000,0.000000}%
\pgfsetstrokecolor{currentstroke}%
\pgfsetdash{}{0pt}%
\pgfpathmoveto{\pgfqpoint{3.166746in}{2.306824in}}%
\pgfpathlineto{\pgfqpoint{3.127682in}{1.627373in}}%
\pgfusepath{stroke}%
\end{pgfscope}%
\begin{pgfscope}%
\pgfpathrectangle{\pgfqpoint{0.100000in}{0.212622in}}{\pgfqpoint{3.696000in}{3.696000in}}%
\pgfusepath{clip}%
\pgfsetrectcap%
\pgfsetroundjoin%
\pgfsetlinewidth{1.505625pt}%
\definecolor{currentstroke}{rgb}{1.000000,0.000000,0.000000}%
\pgfsetstrokecolor{currentstroke}%
\pgfsetdash{}{0pt}%
\pgfpathmoveto{\pgfqpoint{3.210538in}{2.307332in}}%
\pgfpathlineto{\pgfqpoint{3.127682in}{1.627373in}}%
\pgfusepath{stroke}%
\end{pgfscope}%
\begin{pgfscope}%
\pgfpathrectangle{\pgfqpoint{0.100000in}{0.212622in}}{\pgfqpoint{3.696000in}{3.696000in}}%
\pgfusepath{clip}%
\pgfsetrectcap%
\pgfsetroundjoin%
\pgfsetlinewidth{1.505625pt}%
\definecolor{currentstroke}{rgb}{1.000000,0.000000,0.000000}%
\pgfsetstrokecolor{currentstroke}%
\pgfsetdash{}{0pt}%
\pgfpathmoveto{\pgfqpoint{3.233608in}{2.304526in}}%
\pgfpathlineto{\pgfqpoint{3.127682in}{1.627373in}}%
\pgfusepath{stroke}%
\end{pgfscope}%
\begin{pgfscope}%
\pgfpathrectangle{\pgfqpoint{0.100000in}{0.212622in}}{\pgfqpoint{3.696000in}{3.696000in}}%
\pgfusepath{clip}%
\pgfsetrectcap%
\pgfsetroundjoin%
\pgfsetlinewidth{1.505625pt}%
\definecolor{currentstroke}{rgb}{1.000000,0.000000,0.000000}%
\pgfsetstrokecolor{currentstroke}%
\pgfsetdash{}{0pt}%
\pgfpathmoveto{\pgfqpoint{3.258866in}{2.302542in}}%
\pgfpathlineto{\pgfqpoint{3.119891in}{1.619691in}}%
\pgfusepath{stroke}%
\end{pgfscope}%
\begin{pgfscope}%
\pgfpathrectangle{\pgfqpoint{0.100000in}{0.212622in}}{\pgfqpoint{3.696000in}{3.696000in}}%
\pgfusepath{clip}%
\pgfsetrectcap%
\pgfsetroundjoin%
\pgfsetlinewidth{1.505625pt}%
\definecolor{currentstroke}{rgb}{1.000000,0.000000,0.000000}%
\pgfsetstrokecolor{currentstroke}%
\pgfsetdash{}{0pt}%
\pgfpathmoveto{\pgfqpoint{3.273190in}{2.302308in}}%
\pgfpathlineto{\pgfqpoint{3.119891in}{1.619691in}}%
\pgfusepath{stroke}%
\end{pgfscope}%
\begin{pgfscope}%
\pgfpathrectangle{\pgfqpoint{0.100000in}{0.212622in}}{\pgfqpoint{3.696000in}{3.696000in}}%
\pgfusepath{clip}%
\pgfsetrectcap%
\pgfsetroundjoin%
\pgfsetlinewidth{1.505625pt}%
\definecolor{currentstroke}{rgb}{1.000000,0.000000,0.000000}%
\pgfsetstrokecolor{currentstroke}%
\pgfsetdash{}{0pt}%
\pgfpathmoveto{\pgfqpoint{3.291242in}{2.298587in}}%
\pgfpathlineto{\pgfqpoint{3.119891in}{1.619691in}}%
\pgfusepath{stroke}%
\end{pgfscope}%
\begin{pgfscope}%
\pgfpathrectangle{\pgfqpoint{0.100000in}{0.212622in}}{\pgfqpoint{3.696000in}{3.696000in}}%
\pgfusepath{clip}%
\pgfsetrectcap%
\pgfsetroundjoin%
\pgfsetlinewidth{1.505625pt}%
\definecolor{currentstroke}{rgb}{1.000000,0.000000,0.000000}%
\pgfsetstrokecolor{currentstroke}%
\pgfsetdash{}{0pt}%
\pgfpathmoveto{\pgfqpoint{3.299873in}{2.296551in}}%
\pgfpathlineto{\pgfqpoint{3.119891in}{1.619691in}}%
\pgfusepath{stroke}%
\end{pgfscope}%
\begin{pgfscope}%
\pgfpathrectangle{\pgfqpoint{0.100000in}{0.212622in}}{\pgfqpoint{3.696000in}{3.696000in}}%
\pgfusepath{clip}%
\pgfsetrectcap%
\pgfsetroundjoin%
\pgfsetlinewidth{1.505625pt}%
\definecolor{currentstroke}{rgb}{1.000000,0.000000,0.000000}%
\pgfsetstrokecolor{currentstroke}%
\pgfsetdash{}{0pt}%
\pgfpathmoveto{\pgfqpoint{3.302797in}{2.292263in}}%
\pgfpathlineto{\pgfqpoint{3.119891in}{1.619691in}}%
\pgfusepath{stroke}%
\end{pgfscope}%
\begin{pgfscope}%
\pgfpathrectangle{\pgfqpoint{0.100000in}{0.212622in}}{\pgfqpoint{3.696000in}{3.696000in}}%
\pgfusepath{clip}%
\pgfsetrectcap%
\pgfsetroundjoin%
\pgfsetlinewidth{1.505625pt}%
\definecolor{currentstroke}{rgb}{1.000000,0.000000,0.000000}%
\pgfsetstrokecolor{currentstroke}%
\pgfsetdash{}{0pt}%
\pgfpathmoveto{\pgfqpoint{3.306089in}{2.287826in}}%
\pgfpathlineto{\pgfqpoint{3.112089in}{1.612000in}}%
\pgfusepath{stroke}%
\end{pgfscope}%
\begin{pgfscope}%
\pgfpathrectangle{\pgfqpoint{0.100000in}{0.212622in}}{\pgfqpoint{3.696000in}{3.696000in}}%
\pgfusepath{clip}%
\pgfsetrectcap%
\pgfsetroundjoin%
\pgfsetlinewidth{1.505625pt}%
\definecolor{currentstroke}{rgb}{1.000000,0.000000,0.000000}%
\pgfsetstrokecolor{currentstroke}%
\pgfsetdash{}{0pt}%
\pgfpathmoveto{\pgfqpoint{3.306915in}{2.283615in}}%
\pgfpathlineto{\pgfqpoint{3.112089in}{1.612000in}}%
\pgfusepath{stroke}%
\end{pgfscope}%
\begin{pgfscope}%
\pgfpathrectangle{\pgfqpoint{0.100000in}{0.212622in}}{\pgfqpoint{3.696000in}{3.696000in}}%
\pgfusepath{clip}%
\pgfsetrectcap%
\pgfsetroundjoin%
\pgfsetlinewidth{1.505625pt}%
\definecolor{currentstroke}{rgb}{1.000000,0.000000,0.000000}%
\pgfsetstrokecolor{currentstroke}%
\pgfsetdash{}{0pt}%
\pgfpathmoveto{\pgfqpoint{3.305075in}{2.282414in}}%
\pgfpathlineto{\pgfqpoint{3.104278in}{1.604298in}}%
\pgfusepath{stroke}%
\end{pgfscope}%
\begin{pgfscope}%
\pgfpathrectangle{\pgfqpoint{0.100000in}{0.212622in}}{\pgfqpoint{3.696000in}{3.696000in}}%
\pgfusepath{clip}%
\pgfsetrectcap%
\pgfsetroundjoin%
\pgfsetlinewidth{1.505625pt}%
\definecolor{currentstroke}{rgb}{1.000000,0.000000,0.000000}%
\pgfsetstrokecolor{currentstroke}%
\pgfsetdash{}{0pt}%
\pgfpathmoveto{\pgfqpoint{3.300833in}{2.279814in}}%
\pgfpathlineto{\pgfqpoint{3.104278in}{1.604298in}}%
\pgfusepath{stroke}%
\end{pgfscope}%
\begin{pgfscope}%
\pgfpathrectangle{\pgfqpoint{0.100000in}{0.212622in}}{\pgfqpoint{3.696000in}{3.696000in}}%
\pgfusepath{clip}%
\pgfsetrectcap%
\pgfsetroundjoin%
\pgfsetlinewidth{1.505625pt}%
\definecolor{currentstroke}{rgb}{1.000000,0.000000,0.000000}%
\pgfsetstrokecolor{currentstroke}%
\pgfsetdash{}{0pt}%
\pgfpathmoveto{\pgfqpoint{3.295651in}{2.277738in}}%
\pgfpathlineto{\pgfqpoint{3.096456in}{1.596587in}}%
\pgfusepath{stroke}%
\end{pgfscope}%
\begin{pgfscope}%
\pgfpathrectangle{\pgfqpoint{0.100000in}{0.212622in}}{\pgfqpoint{3.696000in}{3.696000in}}%
\pgfusepath{clip}%
\pgfsetrectcap%
\pgfsetroundjoin%
\pgfsetlinewidth{1.505625pt}%
\definecolor{currentstroke}{rgb}{1.000000,0.000000,0.000000}%
\pgfsetstrokecolor{currentstroke}%
\pgfsetdash{}{0pt}%
\pgfpathmoveto{\pgfqpoint{3.287113in}{2.274365in}}%
\pgfpathlineto{\pgfqpoint{3.088625in}{1.588865in}}%
\pgfusepath{stroke}%
\end{pgfscope}%
\begin{pgfscope}%
\pgfpathrectangle{\pgfqpoint{0.100000in}{0.212622in}}{\pgfqpoint{3.696000in}{3.696000in}}%
\pgfusepath{clip}%
\pgfsetrectcap%
\pgfsetroundjoin%
\pgfsetlinewidth{1.505625pt}%
\definecolor{currentstroke}{rgb}{1.000000,0.000000,0.000000}%
\pgfsetstrokecolor{currentstroke}%
\pgfsetdash{}{0pt}%
\pgfpathmoveto{\pgfqpoint{3.277214in}{2.270541in}}%
\pgfpathlineto{\pgfqpoint{3.080783in}{1.581134in}}%
\pgfusepath{stroke}%
\end{pgfscope}%
\begin{pgfscope}%
\pgfpathrectangle{\pgfqpoint{0.100000in}{0.212622in}}{\pgfqpoint{3.696000in}{3.696000in}}%
\pgfusepath{clip}%
\pgfsetrectcap%
\pgfsetroundjoin%
\pgfsetlinewidth{1.505625pt}%
\definecolor{currentstroke}{rgb}{1.000000,0.000000,0.000000}%
\pgfsetstrokecolor{currentstroke}%
\pgfsetdash{}{0pt}%
\pgfpathmoveto{\pgfqpoint{3.271036in}{2.267848in}}%
\pgfpathlineto{\pgfqpoint{3.072932in}{1.573393in}}%
\pgfusepath{stroke}%
\end{pgfscope}%
\begin{pgfscope}%
\pgfpathrectangle{\pgfqpoint{0.100000in}{0.212622in}}{\pgfqpoint{3.696000in}{3.696000in}}%
\pgfusepath{clip}%
\pgfsetrectcap%
\pgfsetroundjoin%
\pgfsetlinewidth{1.505625pt}%
\definecolor{currentstroke}{rgb}{1.000000,0.000000,0.000000}%
\pgfsetstrokecolor{currentstroke}%
\pgfsetdash{}{0pt}%
\pgfpathmoveto{\pgfqpoint{3.267853in}{2.266918in}}%
\pgfpathlineto{\pgfqpoint{3.072932in}{1.573393in}}%
\pgfusepath{stroke}%
\end{pgfscope}%
\begin{pgfscope}%
\pgfpathrectangle{\pgfqpoint{0.100000in}{0.212622in}}{\pgfqpoint{3.696000in}{3.696000in}}%
\pgfusepath{clip}%
\pgfsetrectcap%
\pgfsetroundjoin%
\pgfsetlinewidth{1.505625pt}%
\definecolor{currentstroke}{rgb}{1.000000,0.000000,0.000000}%
\pgfsetstrokecolor{currentstroke}%
\pgfsetdash{}{0pt}%
\pgfpathmoveto{\pgfqpoint{3.263091in}{2.265854in}}%
\pgfpathlineto{\pgfqpoint{3.065070in}{1.565642in}}%
\pgfusepath{stroke}%
\end{pgfscope}%
\begin{pgfscope}%
\pgfpathrectangle{\pgfqpoint{0.100000in}{0.212622in}}{\pgfqpoint{3.696000in}{3.696000in}}%
\pgfusepath{clip}%
\pgfsetrectcap%
\pgfsetroundjoin%
\pgfsetlinewidth{1.505625pt}%
\definecolor{currentstroke}{rgb}{1.000000,0.000000,0.000000}%
\pgfsetstrokecolor{currentstroke}%
\pgfsetdash{}{0pt}%
\pgfpathmoveto{\pgfqpoint{3.260143in}{2.264675in}}%
\pgfpathlineto{\pgfqpoint{3.065070in}{1.565642in}}%
\pgfusepath{stroke}%
\end{pgfscope}%
\begin{pgfscope}%
\pgfpathrectangle{\pgfqpoint{0.100000in}{0.212622in}}{\pgfqpoint{3.696000in}{3.696000in}}%
\pgfusepath{clip}%
\pgfsetrectcap%
\pgfsetroundjoin%
\pgfsetlinewidth{1.505625pt}%
\definecolor{currentstroke}{rgb}{1.000000,0.000000,0.000000}%
\pgfsetstrokecolor{currentstroke}%
\pgfsetdash{}{0pt}%
\pgfpathmoveto{\pgfqpoint{3.255978in}{2.262394in}}%
\pgfpathlineto{\pgfqpoint{3.057198in}{1.557881in}}%
\pgfusepath{stroke}%
\end{pgfscope}%
\begin{pgfscope}%
\pgfpathrectangle{\pgfqpoint{0.100000in}{0.212622in}}{\pgfqpoint{3.696000in}{3.696000in}}%
\pgfusepath{clip}%
\pgfsetrectcap%
\pgfsetroundjoin%
\pgfsetlinewidth{1.505625pt}%
\definecolor{currentstroke}{rgb}{1.000000,0.000000,0.000000}%
\pgfsetstrokecolor{currentstroke}%
\pgfsetdash{}{0pt}%
\pgfpathmoveto{\pgfqpoint{3.247332in}{2.259555in}}%
\pgfpathlineto{\pgfqpoint{3.057198in}{1.557881in}}%
\pgfusepath{stroke}%
\end{pgfscope}%
\begin{pgfscope}%
\pgfpathrectangle{\pgfqpoint{0.100000in}{0.212622in}}{\pgfqpoint{3.696000in}{3.696000in}}%
\pgfusepath{clip}%
\pgfsetrectcap%
\pgfsetroundjoin%
\pgfsetlinewidth{1.505625pt}%
\definecolor{currentstroke}{rgb}{1.000000,0.000000,0.000000}%
\pgfsetstrokecolor{currentstroke}%
\pgfsetdash{}{0pt}%
\pgfpathmoveto{\pgfqpoint{3.243650in}{2.258372in}}%
\pgfpathlineto{\pgfqpoint{3.049316in}{1.550110in}}%
\pgfusepath{stroke}%
\end{pgfscope}%
\begin{pgfscope}%
\pgfpathrectangle{\pgfqpoint{0.100000in}{0.212622in}}{\pgfqpoint{3.696000in}{3.696000in}}%
\pgfusepath{clip}%
\pgfsetrectcap%
\pgfsetroundjoin%
\pgfsetlinewidth{1.505625pt}%
\definecolor{currentstroke}{rgb}{1.000000,0.000000,0.000000}%
\pgfsetstrokecolor{currentstroke}%
\pgfsetdash{}{0pt}%
\pgfpathmoveto{\pgfqpoint{3.241034in}{2.257506in}}%
\pgfpathlineto{\pgfqpoint{3.049316in}{1.550110in}}%
\pgfusepath{stroke}%
\end{pgfscope}%
\begin{pgfscope}%
\pgfpathrectangle{\pgfqpoint{0.100000in}{0.212622in}}{\pgfqpoint{3.696000in}{3.696000in}}%
\pgfusepath{clip}%
\pgfsetrectcap%
\pgfsetroundjoin%
\pgfsetlinewidth{1.505625pt}%
\definecolor{currentstroke}{rgb}{1.000000,0.000000,0.000000}%
\pgfsetstrokecolor{currentstroke}%
\pgfsetdash{}{0pt}%
\pgfpathmoveto{\pgfqpoint{3.239863in}{2.257294in}}%
\pgfpathlineto{\pgfqpoint{3.049316in}{1.550110in}}%
\pgfusepath{stroke}%
\end{pgfscope}%
\begin{pgfscope}%
\pgfpathrectangle{\pgfqpoint{0.100000in}{0.212622in}}{\pgfqpoint{3.696000in}{3.696000in}}%
\pgfusepath{clip}%
\pgfsetrectcap%
\pgfsetroundjoin%
\pgfsetlinewidth{1.505625pt}%
\definecolor{currentstroke}{rgb}{1.000000,0.000000,0.000000}%
\pgfsetstrokecolor{currentstroke}%
\pgfsetdash{}{0pt}%
\pgfpathmoveto{\pgfqpoint{3.239092in}{2.257001in}}%
\pgfpathlineto{\pgfqpoint{3.049316in}{1.550110in}}%
\pgfusepath{stroke}%
\end{pgfscope}%
\begin{pgfscope}%
\pgfpathrectangle{\pgfqpoint{0.100000in}{0.212622in}}{\pgfqpoint{3.696000in}{3.696000in}}%
\pgfusepath{clip}%
\pgfsetrectcap%
\pgfsetroundjoin%
\pgfsetlinewidth{1.505625pt}%
\definecolor{currentstroke}{rgb}{1.000000,0.000000,0.000000}%
\pgfsetstrokecolor{currentstroke}%
\pgfsetdash{}{0pt}%
\pgfpathmoveto{\pgfqpoint{3.238746in}{2.256893in}}%
\pgfpathlineto{\pgfqpoint{3.049316in}{1.550110in}}%
\pgfusepath{stroke}%
\end{pgfscope}%
\begin{pgfscope}%
\pgfpathrectangle{\pgfqpoint{0.100000in}{0.212622in}}{\pgfqpoint{3.696000in}{3.696000in}}%
\pgfusepath{clip}%
\pgfsetrectcap%
\pgfsetroundjoin%
\pgfsetlinewidth{1.505625pt}%
\definecolor{currentstroke}{rgb}{1.000000,0.000000,0.000000}%
\pgfsetstrokecolor{currentstroke}%
\pgfsetdash{}{0pt}%
\pgfpathmoveto{\pgfqpoint{3.238584in}{2.256832in}}%
\pgfpathlineto{\pgfqpoint{3.049316in}{1.550110in}}%
\pgfusepath{stroke}%
\end{pgfscope}%
\begin{pgfscope}%
\pgfpathrectangle{\pgfqpoint{0.100000in}{0.212622in}}{\pgfqpoint{3.696000in}{3.696000in}}%
\pgfusepath{clip}%
\pgfsetrectcap%
\pgfsetroundjoin%
\pgfsetlinewidth{1.505625pt}%
\definecolor{currentstroke}{rgb}{1.000000,0.000000,0.000000}%
\pgfsetstrokecolor{currentstroke}%
\pgfsetdash{}{0pt}%
\pgfpathmoveto{\pgfqpoint{3.236871in}{2.256688in}}%
\pgfpathlineto{\pgfqpoint{3.049316in}{1.550110in}}%
\pgfusepath{stroke}%
\end{pgfscope}%
\begin{pgfscope}%
\pgfpathrectangle{\pgfqpoint{0.100000in}{0.212622in}}{\pgfqpoint{3.696000in}{3.696000in}}%
\pgfusepath{clip}%
\pgfsetrectcap%
\pgfsetroundjoin%
\pgfsetlinewidth{1.505625pt}%
\definecolor{currentstroke}{rgb}{1.000000,0.000000,0.000000}%
\pgfsetstrokecolor{currentstroke}%
\pgfsetdash{}{0pt}%
\pgfpathmoveto{\pgfqpoint{3.233596in}{2.252836in}}%
\pgfpathlineto{\pgfqpoint{3.041424in}{1.542329in}}%
\pgfusepath{stroke}%
\end{pgfscope}%
\begin{pgfscope}%
\pgfpathrectangle{\pgfqpoint{0.100000in}{0.212622in}}{\pgfqpoint{3.696000in}{3.696000in}}%
\pgfusepath{clip}%
\pgfsetrectcap%
\pgfsetroundjoin%
\pgfsetlinewidth{1.505625pt}%
\definecolor{currentstroke}{rgb}{1.000000,0.000000,0.000000}%
\pgfsetstrokecolor{currentstroke}%
\pgfsetdash{}{0pt}%
\pgfpathmoveto{\pgfqpoint{3.225840in}{2.252520in}}%
\pgfpathlineto{\pgfqpoint{3.033522in}{1.534538in}}%
\pgfusepath{stroke}%
\end{pgfscope}%
\begin{pgfscope}%
\pgfpathrectangle{\pgfqpoint{0.100000in}{0.212622in}}{\pgfqpoint{3.696000in}{3.696000in}}%
\pgfusepath{clip}%
\pgfsetrectcap%
\pgfsetroundjoin%
\pgfsetlinewidth{1.505625pt}%
\definecolor{currentstroke}{rgb}{1.000000,0.000000,0.000000}%
\pgfsetstrokecolor{currentstroke}%
\pgfsetdash{}{0pt}%
\pgfpathmoveto{\pgfqpoint{3.216928in}{2.246784in}}%
\pgfpathlineto{\pgfqpoint{3.025610in}{1.526737in}}%
\pgfusepath{stroke}%
\end{pgfscope}%
\begin{pgfscope}%
\pgfpathrectangle{\pgfqpoint{0.100000in}{0.212622in}}{\pgfqpoint{3.696000in}{3.696000in}}%
\pgfusepath{clip}%
\pgfsetrectcap%
\pgfsetroundjoin%
\pgfsetlinewidth{1.505625pt}%
\definecolor{currentstroke}{rgb}{1.000000,0.000000,0.000000}%
\pgfsetstrokecolor{currentstroke}%
\pgfsetdash{}{0pt}%
\pgfpathmoveto{\pgfqpoint{3.200978in}{2.243154in}}%
\pgfpathlineto{\pgfqpoint{3.017687in}{1.518925in}}%
\pgfusepath{stroke}%
\end{pgfscope}%
\begin{pgfscope}%
\pgfpathrectangle{\pgfqpoint{0.100000in}{0.212622in}}{\pgfqpoint{3.696000in}{3.696000in}}%
\pgfusepath{clip}%
\pgfsetrectcap%
\pgfsetroundjoin%
\pgfsetlinewidth{1.505625pt}%
\definecolor{currentstroke}{rgb}{1.000000,0.000000,0.000000}%
\pgfsetstrokecolor{currentstroke}%
\pgfsetdash{}{0pt}%
\pgfpathmoveto{\pgfqpoint{3.189546in}{2.234460in}}%
\pgfpathlineto{\pgfqpoint{3.001811in}{1.503273in}}%
\pgfusepath{stroke}%
\end{pgfscope}%
\begin{pgfscope}%
\pgfpathrectangle{\pgfqpoint{0.100000in}{0.212622in}}{\pgfqpoint{3.696000in}{3.696000in}}%
\pgfusepath{clip}%
\pgfsetrectcap%
\pgfsetroundjoin%
\pgfsetlinewidth{1.505625pt}%
\definecolor{currentstroke}{rgb}{1.000000,0.000000,0.000000}%
\pgfsetstrokecolor{currentstroke}%
\pgfsetdash{}{0pt}%
\pgfpathmoveto{\pgfqpoint{3.167423in}{2.227283in}}%
\pgfpathlineto{\pgfqpoint{2.993858in}{1.495431in}}%
\pgfusepath{stroke}%
\end{pgfscope}%
\begin{pgfscope}%
\pgfpathrectangle{\pgfqpoint{0.100000in}{0.212622in}}{\pgfqpoint{3.696000in}{3.696000in}}%
\pgfusepath{clip}%
\pgfsetrectcap%
\pgfsetroundjoin%
\pgfsetlinewidth{1.505625pt}%
\definecolor{currentstroke}{rgb}{1.000000,0.000000,0.000000}%
\pgfsetstrokecolor{currentstroke}%
\pgfsetdash{}{0pt}%
\pgfpathmoveto{\pgfqpoint{3.149094in}{2.214548in}}%
\pgfpathlineto{\pgfqpoint{2.969936in}{1.471846in}}%
\pgfusepath{stroke}%
\end{pgfscope}%
\begin{pgfscope}%
\pgfpathrectangle{\pgfqpoint{0.100000in}{0.212622in}}{\pgfqpoint{3.696000in}{3.696000in}}%
\pgfusepath{clip}%
\pgfsetrectcap%
\pgfsetroundjoin%
\pgfsetlinewidth{1.505625pt}%
\definecolor{currentstroke}{rgb}{1.000000,0.000000,0.000000}%
\pgfsetstrokecolor{currentstroke}%
\pgfsetdash{}{0pt}%
\pgfpathmoveto{\pgfqpoint{3.121159in}{2.207515in}}%
\pgfpathlineto{\pgfqpoint{2.953937in}{1.456072in}}%
\pgfusepath{stroke}%
\end{pgfscope}%
\begin{pgfscope}%
\pgfpathrectangle{\pgfqpoint{0.100000in}{0.212622in}}{\pgfqpoint{3.696000in}{3.696000in}}%
\pgfusepath{clip}%
\pgfsetrectcap%
\pgfsetroundjoin%
\pgfsetlinewidth{1.505625pt}%
\definecolor{currentstroke}{rgb}{1.000000,0.000000,0.000000}%
\pgfsetstrokecolor{currentstroke}%
\pgfsetdash{}{0pt}%
\pgfpathmoveto{\pgfqpoint{3.109794in}{2.205463in}}%
\pgfpathlineto{\pgfqpoint{2.945921in}{1.448169in}}%
\pgfusepath{stroke}%
\end{pgfscope}%
\begin{pgfscope}%
\pgfpathrectangle{\pgfqpoint{0.100000in}{0.212622in}}{\pgfqpoint{3.696000in}{3.696000in}}%
\pgfusepath{clip}%
\pgfsetrectcap%
\pgfsetroundjoin%
\pgfsetlinewidth{1.505625pt}%
\definecolor{currentstroke}{rgb}{1.000000,0.000000,0.000000}%
\pgfsetstrokecolor{currentstroke}%
\pgfsetdash{}{0pt}%
\pgfpathmoveto{\pgfqpoint{3.101195in}{2.202204in}}%
\pgfpathlineto{\pgfqpoint{2.937896in}{1.440256in}}%
\pgfusepath{stroke}%
\end{pgfscope}%
\begin{pgfscope}%
\pgfpathrectangle{\pgfqpoint{0.100000in}{0.212622in}}{\pgfqpoint{3.696000in}{3.696000in}}%
\pgfusepath{clip}%
\pgfsetrectcap%
\pgfsetroundjoin%
\pgfsetlinewidth{1.505625pt}%
\definecolor{currentstroke}{rgb}{1.000000,0.000000,0.000000}%
\pgfsetstrokecolor{currentstroke}%
\pgfsetdash{}{0pt}%
\pgfpathmoveto{\pgfqpoint{3.097933in}{2.201624in}}%
\pgfpathlineto{\pgfqpoint{2.937896in}{1.440256in}}%
\pgfusepath{stroke}%
\end{pgfscope}%
\begin{pgfscope}%
\pgfpathrectangle{\pgfqpoint{0.100000in}{0.212622in}}{\pgfqpoint{3.696000in}{3.696000in}}%
\pgfusepath{clip}%
\pgfsetrectcap%
\pgfsetroundjoin%
\pgfsetlinewidth{1.505625pt}%
\definecolor{currentstroke}{rgb}{1.000000,0.000000,0.000000}%
\pgfsetstrokecolor{currentstroke}%
\pgfsetdash{}{0pt}%
\pgfpathmoveto{\pgfqpoint{3.095579in}{2.201261in}}%
\pgfpathlineto{\pgfqpoint{2.929860in}{1.432333in}}%
\pgfusepath{stroke}%
\end{pgfscope}%
\begin{pgfscope}%
\pgfpathrectangle{\pgfqpoint{0.100000in}{0.212622in}}{\pgfqpoint{3.696000in}{3.696000in}}%
\pgfusepath{clip}%
\pgfsetrectcap%
\pgfsetroundjoin%
\pgfsetlinewidth{1.505625pt}%
\definecolor{currentstroke}{rgb}{1.000000,0.000000,0.000000}%
\pgfsetstrokecolor{currentstroke}%
\pgfsetdash{}{0pt}%
\pgfpathmoveto{\pgfqpoint{3.094279in}{2.201083in}}%
\pgfpathlineto{\pgfqpoint{2.929860in}{1.432333in}}%
\pgfusepath{stroke}%
\end{pgfscope}%
\begin{pgfscope}%
\pgfpathrectangle{\pgfqpoint{0.100000in}{0.212622in}}{\pgfqpoint{3.696000in}{3.696000in}}%
\pgfusepath{clip}%
\pgfsetrectcap%
\pgfsetroundjoin%
\pgfsetlinewidth{1.505625pt}%
\definecolor{currentstroke}{rgb}{1.000000,0.000000,0.000000}%
\pgfsetstrokecolor{currentstroke}%
\pgfsetdash{}{0pt}%
\pgfpathmoveto{\pgfqpoint{3.093702in}{2.200811in}}%
\pgfpathlineto{\pgfqpoint{2.929860in}{1.432333in}}%
\pgfusepath{stroke}%
\end{pgfscope}%
\begin{pgfscope}%
\pgfpathrectangle{\pgfqpoint{0.100000in}{0.212622in}}{\pgfqpoint{3.696000in}{3.696000in}}%
\pgfusepath{clip}%
\pgfsetrectcap%
\pgfsetroundjoin%
\pgfsetlinewidth{1.505625pt}%
\definecolor{currentstroke}{rgb}{1.000000,0.000000,0.000000}%
\pgfsetstrokecolor{currentstroke}%
\pgfsetdash{}{0pt}%
\pgfpathmoveto{\pgfqpoint{3.093276in}{2.200840in}}%
\pgfpathlineto{\pgfqpoint{2.929860in}{1.432333in}}%
\pgfusepath{stroke}%
\end{pgfscope}%
\begin{pgfscope}%
\pgfpathrectangle{\pgfqpoint{0.100000in}{0.212622in}}{\pgfqpoint{3.696000in}{3.696000in}}%
\pgfusepath{clip}%
\pgfsetrectcap%
\pgfsetroundjoin%
\pgfsetlinewidth{1.505625pt}%
\definecolor{currentstroke}{rgb}{1.000000,0.000000,0.000000}%
\pgfsetstrokecolor{currentstroke}%
\pgfsetdash{}{0pt}%
\pgfpathmoveto{\pgfqpoint{3.093123in}{2.200747in}}%
\pgfpathlineto{\pgfqpoint{2.929860in}{1.432333in}}%
\pgfusepath{stroke}%
\end{pgfscope}%
\begin{pgfscope}%
\pgfpathrectangle{\pgfqpoint{0.100000in}{0.212622in}}{\pgfqpoint{3.696000in}{3.696000in}}%
\pgfusepath{clip}%
\pgfsetrectcap%
\pgfsetroundjoin%
\pgfsetlinewidth{1.505625pt}%
\definecolor{currentstroke}{rgb}{1.000000,0.000000,0.000000}%
\pgfsetstrokecolor{currentstroke}%
\pgfsetdash{}{0pt}%
\pgfpathmoveto{\pgfqpoint{3.090362in}{2.199531in}}%
\pgfpathlineto{\pgfqpoint{2.929860in}{1.432333in}}%
\pgfusepath{stroke}%
\end{pgfscope}%
\begin{pgfscope}%
\pgfpathrectangle{\pgfqpoint{0.100000in}{0.212622in}}{\pgfqpoint{3.696000in}{3.696000in}}%
\pgfusepath{clip}%
\pgfsetrectcap%
\pgfsetroundjoin%
\pgfsetlinewidth{1.505625pt}%
\definecolor{currentstroke}{rgb}{1.000000,0.000000,0.000000}%
\pgfsetstrokecolor{currentstroke}%
\pgfsetdash{}{0pt}%
\pgfpathmoveto{\pgfqpoint{3.085849in}{2.195821in}}%
\pgfpathlineto{\pgfqpoint{2.921813in}{1.424400in}}%
\pgfusepath{stroke}%
\end{pgfscope}%
\begin{pgfscope}%
\pgfpathrectangle{\pgfqpoint{0.100000in}{0.212622in}}{\pgfqpoint{3.696000in}{3.696000in}}%
\pgfusepath{clip}%
\pgfsetrectcap%
\pgfsetroundjoin%
\pgfsetlinewidth{1.505625pt}%
\definecolor{currentstroke}{rgb}{1.000000,0.000000,0.000000}%
\pgfsetstrokecolor{currentstroke}%
\pgfsetdash{}{0pt}%
\pgfpathmoveto{\pgfqpoint{3.081963in}{2.195287in}}%
\pgfpathlineto{\pgfqpoint{2.921813in}{1.424400in}}%
\pgfusepath{stroke}%
\end{pgfscope}%
\begin{pgfscope}%
\pgfpathrectangle{\pgfqpoint{0.100000in}{0.212622in}}{\pgfqpoint{3.696000in}{3.696000in}}%
\pgfusepath{clip}%
\pgfsetrectcap%
\pgfsetroundjoin%
\pgfsetlinewidth{1.505625pt}%
\definecolor{currentstroke}{rgb}{1.000000,0.000000,0.000000}%
\pgfsetstrokecolor{currentstroke}%
\pgfsetdash{}{0pt}%
\pgfpathmoveto{\pgfqpoint{3.077137in}{2.190915in}}%
\pgfpathlineto{\pgfqpoint{2.913756in}{1.416456in}}%
\pgfusepath{stroke}%
\end{pgfscope}%
\begin{pgfscope}%
\pgfpathrectangle{\pgfqpoint{0.100000in}{0.212622in}}{\pgfqpoint{3.696000in}{3.696000in}}%
\pgfusepath{clip}%
\pgfsetrectcap%
\pgfsetroundjoin%
\pgfsetlinewidth{1.505625pt}%
\definecolor{currentstroke}{rgb}{1.000000,0.000000,0.000000}%
\pgfsetstrokecolor{currentstroke}%
\pgfsetdash{}{0pt}%
\pgfpathmoveto{\pgfqpoint{3.069423in}{2.189273in}}%
\pgfpathlineto{\pgfqpoint{2.913756in}{1.416456in}}%
\pgfusepath{stroke}%
\end{pgfscope}%
\begin{pgfscope}%
\pgfpathrectangle{\pgfqpoint{0.100000in}{0.212622in}}{\pgfqpoint{3.696000in}{3.696000in}}%
\pgfusepath{clip}%
\pgfsetrectcap%
\pgfsetroundjoin%
\pgfsetlinewidth{1.505625pt}%
\definecolor{currentstroke}{rgb}{1.000000,0.000000,0.000000}%
\pgfsetstrokecolor{currentstroke}%
\pgfsetdash{}{0pt}%
\pgfpathmoveto{\pgfqpoint{3.059546in}{2.184394in}}%
\pgfpathlineto{\pgfqpoint{2.897611in}{1.400538in}}%
\pgfusepath{stroke}%
\end{pgfscope}%
\begin{pgfscope}%
\pgfpathrectangle{\pgfqpoint{0.100000in}{0.212622in}}{\pgfqpoint{3.696000in}{3.696000in}}%
\pgfusepath{clip}%
\pgfsetrectcap%
\pgfsetroundjoin%
\pgfsetlinewidth{1.505625pt}%
\definecolor{currentstroke}{rgb}{1.000000,0.000000,0.000000}%
\pgfsetstrokecolor{currentstroke}%
\pgfsetdash{}{0pt}%
\pgfpathmoveto{\pgfqpoint{3.045755in}{2.179895in}}%
\pgfpathlineto{\pgfqpoint{2.889522in}{1.392563in}}%
\pgfusepath{stroke}%
\end{pgfscope}%
\begin{pgfscope}%
\pgfpathrectangle{\pgfqpoint{0.100000in}{0.212622in}}{\pgfqpoint{3.696000in}{3.696000in}}%
\pgfusepath{clip}%
\pgfsetrectcap%
\pgfsetroundjoin%
\pgfsetlinewidth{1.505625pt}%
\definecolor{currentstroke}{rgb}{1.000000,0.000000,0.000000}%
\pgfsetstrokecolor{currentstroke}%
\pgfsetdash{}{0pt}%
\pgfpathmoveto{\pgfqpoint{3.039477in}{2.177744in}}%
\pgfpathlineto{\pgfqpoint{2.881423in}{1.384578in}}%
\pgfusepath{stroke}%
\end{pgfscope}%
\begin{pgfscope}%
\pgfpathrectangle{\pgfqpoint{0.100000in}{0.212622in}}{\pgfqpoint{3.696000in}{3.696000in}}%
\pgfusepath{clip}%
\pgfsetrectcap%
\pgfsetroundjoin%
\pgfsetlinewidth{1.505625pt}%
\definecolor{currentstroke}{rgb}{1.000000,0.000000,0.000000}%
\pgfsetstrokecolor{currentstroke}%
\pgfsetdash{}{0pt}%
\pgfpathmoveto{\pgfqpoint{3.035990in}{2.175980in}}%
\pgfpathlineto{\pgfqpoint{2.881423in}{1.384578in}}%
\pgfusepath{stroke}%
\end{pgfscope}%
\begin{pgfscope}%
\pgfpathrectangle{\pgfqpoint{0.100000in}{0.212622in}}{\pgfqpoint{3.696000in}{3.696000in}}%
\pgfusepath{clip}%
\pgfsetrectcap%
\pgfsetroundjoin%
\pgfsetlinewidth{1.505625pt}%
\definecolor{currentstroke}{rgb}{1.000000,0.000000,0.000000}%
\pgfsetstrokecolor{currentstroke}%
\pgfsetdash{}{0pt}%
\pgfpathmoveto{\pgfqpoint{3.034087in}{2.176006in}}%
\pgfpathlineto{\pgfqpoint{2.881423in}{1.384578in}}%
\pgfusepath{stroke}%
\end{pgfscope}%
\begin{pgfscope}%
\pgfpathrectangle{\pgfqpoint{0.100000in}{0.212622in}}{\pgfqpoint{3.696000in}{3.696000in}}%
\pgfusepath{clip}%
\pgfsetrectcap%
\pgfsetroundjoin%
\pgfsetlinewidth{1.505625pt}%
\definecolor{currentstroke}{rgb}{1.000000,0.000000,0.000000}%
\pgfsetstrokecolor{currentstroke}%
\pgfsetdash{}{0pt}%
\pgfpathmoveto{\pgfqpoint{3.033106in}{2.175585in}}%
\pgfpathlineto{\pgfqpoint{2.881423in}{1.384578in}}%
\pgfusepath{stroke}%
\end{pgfscope}%
\begin{pgfscope}%
\pgfpathrectangle{\pgfqpoint{0.100000in}{0.212622in}}{\pgfqpoint{3.696000in}{3.696000in}}%
\pgfusepath{clip}%
\pgfsetrectcap%
\pgfsetroundjoin%
\pgfsetlinewidth{1.505625pt}%
\definecolor{currentstroke}{rgb}{1.000000,0.000000,0.000000}%
\pgfsetstrokecolor{currentstroke}%
\pgfsetdash{}{0pt}%
\pgfpathmoveto{\pgfqpoint{3.032435in}{2.175536in}}%
\pgfpathlineto{\pgfqpoint{2.881423in}{1.384578in}}%
\pgfusepath{stroke}%
\end{pgfscope}%
\begin{pgfscope}%
\pgfpathrectangle{\pgfqpoint{0.100000in}{0.212622in}}{\pgfqpoint{3.696000in}{3.696000in}}%
\pgfusepath{clip}%
\pgfsetrectcap%
\pgfsetroundjoin%
\pgfsetlinewidth{1.505625pt}%
\definecolor{currentstroke}{rgb}{1.000000,0.000000,0.000000}%
\pgfsetstrokecolor{currentstroke}%
\pgfsetdash{}{0pt}%
\pgfpathmoveto{\pgfqpoint{3.030421in}{2.173364in}}%
\pgfpathlineto{\pgfqpoint{2.873313in}{1.376583in}}%
\pgfusepath{stroke}%
\end{pgfscope}%
\begin{pgfscope}%
\pgfpathrectangle{\pgfqpoint{0.100000in}{0.212622in}}{\pgfqpoint{3.696000in}{3.696000in}}%
\pgfusepath{clip}%
\pgfsetrectcap%
\pgfsetroundjoin%
\pgfsetlinewidth{1.505625pt}%
\definecolor{currentstroke}{rgb}{1.000000,0.000000,0.000000}%
\pgfsetstrokecolor{currentstroke}%
\pgfsetdash{}{0pt}%
\pgfpathmoveto{\pgfqpoint{3.028801in}{2.173171in}}%
\pgfpathlineto{\pgfqpoint{2.873313in}{1.376583in}}%
\pgfusepath{stroke}%
\end{pgfscope}%
\begin{pgfscope}%
\pgfpathrectangle{\pgfqpoint{0.100000in}{0.212622in}}{\pgfqpoint{3.696000in}{3.696000in}}%
\pgfusepath{clip}%
\pgfsetrectcap%
\pgfsetroundjoin%
\pgfsetlinewidth{1.505625pt}%
\definecolor{currentstroke}{rgb}{1.000000,0.000000,0.000000}%
\pgfsetstrokecolor{currentstroke}%
\pgfsetdash{}{0pt}%
\pgfpathmoveto{\pgfqpoint{3.025859in}{2.169276in}}%
\pgfpathlineto{\pgfqpoint{2.873313in}{1.376583in}}%
\pgfusepath{stroke}%
\end{pgfscope}%
\begin{pgfscope}%
\pgfpathrectangle{\pgfqpoint{0.100000in}{0.212622in}}{\pgfqpoint{3.696000in}{3.696000in}}%
\pgfusepath{clip}%
\pgfsetrectcap%
\pgfsetroundjoin%
\pgfsetlinewidth{1.505625pt}%
\definecolor{currentstroke}{rgb}{1.000000,0.000000,0.000000}%
\pgfsetstrokecolor{currentstroke}%
\pgfsetdash{}{0pt}%
\pgfpathmoveto{\pgfqpoint{3.018903in}{2.167173in}}%
\pgfpathlineto{\pgfqpoint{2.865193in}{1.368577in}}%
\pgfusepath{stroke}%
\end{pgfscope}%
\begin{pgfscope}%
\pgfpathrectangle{\pgfqpoint{0.100000in}{0.212622in}}{\pgfqpoint{3.696000in}{3.696000in}}%
\pgfusepath{clip}%
\pgfsetrectcap%
\pgfsetroundjoin%
\pgfsetlinewidth{1.505625pt}%
\definecolor{currentstroke}{rgb}{1.000000,0.000000,0.000000}%
\pgfsetstrokecolor{currentstroke}%
\pgfsetdash{}{0pt}%
\pgfpathmoveto{\pgfqpoint{3.012668in}{2.160572in}}%
\pgfpathlineto{\pgfqpoint{2.857062in}{1.360560in}}%
\pgfusepath{stroke}%
\end{pgfscope}%
\begin{pgfscope}%
\pgfpathrectangle{\pgfqpoint{0.100000in}{0.212622in}}{\pgfqpoint{3.696000in}{3.696000in}}%
\pgfusepath{clip}%
\pgfsetrectcap%
\pgfsetroundjoin%
\pgfsetlinewidth{1.505625pt}%
\definecolor{currentstroke}{rgb}{1.000000,0.000000,0.000000}%
\pgfsetstrokecolor{currentstroke}%
\pgfsetdash{}{0pt}%
\pgfpathmoveto{\pgfqpoint{2.999911in}{2.156279in}}%
\pgfpathlineto{\pgfqpoint{2.848921in}{1.352533in}}%
\pgfusepath{stroke}%
\end{pgfscope}%
\begin{pgfscope}%
\pgfpathrectangle{\pgfqpoint{0.100000in}{0.212622in}}{\pgfqpoint{3.696000in}{3.696000in}}%
\pgfusepath{clip}%
\pgfsetrectcap%
\pgfsetroundjoin%
\pgfsetlinewidth{1.505625pt}%
\definecolor{currentstroke}{rgb}{1.000000,0.000000,0.000000}%
\pgfsetstrokecolor{currentstroke}%
\pgfsetdash{}{0pt}%
\pgfpathmoveto{\pgfqpoint{2.990124in}{2.149960in}}%
\pgfpathlineto{\pgfqpoint{2.840769in}{1.344496in}}%
\pgfusepath{stroke}%
\end{pgfscope}%
\begin{pgfscope}%
\pgfpathrectangle{\pgfqpoint{0.100000in}{0.212622in}}{\pgfqpoint{3.696000in}{3.696000in}}%
\pgfusepath{clip}%
\pgfsetrectcap%
\pgfsetroundjoin%
\pgfsetlinewidth{1.505625pt}%
\definecolor{currentstroke}{rgb}{1.000000,0.000000,0.000000}%
\pgfsetstrokecolor{currentstroke}%
\pgfsetdash{}{0pt}%
\pgfpathmoveto{\pgfqpoint{2.982595in}{2.146379in}}%
\pgfpathlineto{\pgfqpoint{2.832606in}{1.336448in}}%
\pgfusepath{stroke}%
\end{pgfscope}%
\begin{pgfscope}%
\pgfpathrectangle{\pgfqpoint{0.100000in}{0.212622in}}{\pgfqpoint{3.696000in}{3.696000in}}%
\pgfusepath{clip}%
\pgfsetrectcap%
\pgfsetroundjoin%
\pgfsetlinewidth{1.505625pt}%
\definecolor{currentstroke}{rgb}{1.000000,0.000000,0.000000}%
\pgfsetstrokecolor{currentstroke}%
\pgfsetdash{}{0pt}%
\pgfpathmoveto{\pgfqpoint{2.979083in}{2.145306in}}%
\pgfpathlineto{\pgfqpoint{2.832606in}{1.336448in}}%
\pgfusepath{stroke}%
\end{pgfscope}%
\begin{pgfscope}%
\pgfpathrectangle{\pgfqpoint{0.100000in}{0.212622in}}{\pgfqpoint{3.696000in}{3.696000in}}%
\pgfusepath{clip}%
\pgfsetrectcap%
\pgfsetroundjoin%
\pgfsetlinewidth{1.505625pt}%
\definecolor{currentstroke}{rgb}{1.000000,0.000000,0.000000}%
\pgfsetstrokecolor{currentstroke}%
\pgfsetdash{}{0pt}%
\pgfpathmoveto{\pgfqpoint{2.977162in}{2.144702in}}%
\pgfpathlineto{\pgfqpoint{2.832606in}{1.336448in}}%
\pgfusepath{stroke}%
\end{pgfscope}%
\begin{pgfscope}%
\pgfpathrectangle{\pgfqpoint{0.100000in}{0.212622in}}{\pgfqpoint{3.696000in}{3.696000in}}%
\pgfusepath{clip}%
\pgfsetrectcap%
\pgfsetroundjoin%
\pgfsetlinewidth{1.505625pt}%
\definecolor{currentstroke}{rgb}{1.000000,0.000000,0.000000}%
\pgfsetstrokecolor{currentstroke}%
\pgfsetdash{}{0pt}%
\pgfpathmoveto{\pgfqpoint{2.975953in}{2.144492in}}%
\pgfpathlineto{\pgfqpoint{2.824433in}{1.328389in}}%
\pgfusepath{stroke}%
\end{pgfscope}%
\begin{pgfscope}%
\pgfpathrectangle{\pgfqpoint{0.100000in}{0.212622in}}{\pgfqpoint{3.696000in}{3.696000in}}%
\pgfusepath{clip}%
\pgfsetrectcap%
\pgfsetroundjoin%
\pgfsetlinewidth{1.505625pt}%
\definecolor{currentstroke}{rgb}{1.000000,0.000000,0.000000}%
\pgfsetstrokecolor{currentstroke}%
\pgfsetdash{}{0pt}%
\pgfpathmoveto{\pgfqpoint{2.973921in}{2.142537in}}%
\pgfpathlineto{\pgfqpoint{2.824433in}{1.328389in}}%
\pgfusepath{stroke}%
\end{pgfscope}%
\begin{pgfscope}%
\pgfpathrectangle{\pgfqpoint{0.100000in}{0.212622in}}{\pgfqpoint{3.696000in}{3.696000in}}%
\pgfusepath{clip}%
\pgfsetrectcap%
\pgfsetroundjoin%
\pgfsetlinewidth{1.505625pt}%
\definecolor{currentstroke}{rgb}{1.000000,0.000000,0.000000}%
\pgfsetstrokecolor{currentstroke}%
\pgfsetdash{}{0pt}%
\pgfpathmoveto{\pgfqpoint{2.969857in}{2.141447in}}%
\pgfpathlineto{\pgfqpoint{2.824433in}{1.328389in}}%
\pgfusepath{stroke}%
\end{pgfscope}%
\begin{pgfscope}%
\pgfpathrectangle{\pgfqpoint{0.100000in}{0.212622in}}{\pgfqpoint{3.696000in}{3.696000in}}%
\pgfusepath{clip}%
\pgfsetrectcap%
\pgfsetroundjoin%
\pgfsetlinewidth{1.505625pt}%
\definecolor{currentstroke}{rgb}{1.000000,0.000000,0.000000}%
\pgfsetstrokecolor{currentstroke}%
\pgfsetdash{}{0pt}%
\pgfpathmoveto{\pgfqpoint{2.964766in}{2.136056in}}%
\pgfpathlineto{\pgfqpoint{2.816248in}{1.320320in}}%
\pgfusepath{stroke}%
\end{pgfscope}%
\begin{pgfscope}%
\pgfpathrectangle{\pgfqpoint{0.100000in}{0.212622in}}{\pgfqpoint{3.696000in}{3.696000in}}%
\pgfusepath{clip}%
\pgfsetrectcap%
\pgfsetroundjoin%
\pgfsetlinewidth{1.505625pt}%
\definecolor{currentstroke}{rgb}{1.000000,0.000000,0.000000}%
\pgfsetstrokecolor{currentstroke}%
\pgfsetdash{}{0pt}%
\pgfpathmoveto{\pgfqpoint{2.960517in}{2.134572in}}%
\pgfpathlineto{\pgfqpoint{2.816248in}{1.320320in}}%
\pgfusepath{stroke}%
\end{pgfscope}%
\begin{pgfscope}%
\pgfpathrectangle{\pgfqpoint{0.100000in}{0.212622in}}{\pgfqpoint{3.696000in}{3.696000in}}%
\pgfusepath{clip}%
\pgfsetrectcap%
\pgfsetroundjoin%
\pgfsetlinewidth{1.505625pt}%
\definecolor{currentstroke}{rgb}{1.000000,0.000000,0.000000}%
\pgfsetstrokecolor{currentstroke}%
\pgfsetdash{}{0pt}%
\pgfpathmoveto{\pgfqpoint{2.955825in}{2.132297in}}%
\pgfpathlineto{\pgfqpoint{2.808054in}{1.312241in}}%
\pgfusepath{stroke}%
\end{pgfscope}%
\begin{pgfscope}%
\pgfpathrectangle{\pgfqpoint{0.100000in}{0.212622in}}{\pgfqpoint{3.696000in}{3.696000in}}%
\pgfusepath{clip}%
\pgfsetrectcap%
\pgfsetroundjoin%
\pgfsetlinewidth{1.505625pt}%
\definecolor{currentstroke}{rgb}{1.000000,0.000000,0.000000}%
\pgfsetstrokecolor{currentstroke}%
\pgfsetdash{}{0pt}%
\pgfpathmoveto{\pgfqpoint{2.952598in}{2.131068in}}%
\pgfpathlineto{\pgfqpoint{2.808054in}{1.312241in}}%
\pgfusepath{stroke}%
\end{pgfscope}%
\begin{pgfscope}%
\pgfpathrectangle{\pgfqpoint{0.100000in}{0.212622in}}{\pgfqpoint{3.696000in}{3.696000in}}%
\pgfusepath{clip}%
\pgfsetrectcap%
\pgfsetroundjoin%
\pgfsetlinewidth{1.505625pt}%
\definecolor{currentstroke}{rgb}{1.000000,0.000000,0.000000}%
\pgfsetstrokecolor{currentstroke}%
\pgfsetdash{}{0pt}%
\pgfpathmoveto{\pgfqpoint{2.951073in}{2.130025in}}%
\pgfpathlineto{\pgfqpoint{2.808054in}{1.312241in}}%
\pgfusepath{stroke}%
\end{pgfscope}%
\begin{pgfscope}%
\pgfpathrectangle{\pgfqpoint{0.100000in}{0.212622in}}{\pgfqpoint{3.696000in}{3.696000in}}%
\pgfusepath{clip}%
\pgfsetrectcap%
\pgfsetroundjoin%
\pgfsetlinewidth{1.505625pt}%
\definecolor{currentstroke}{rgb}{1.000000,0.000000,0.000000}%
\pgfsetstrokecolor{currentstroke}%
\pgfsetdash{}{0pt}%
\pgfpathmoveto{\pgfqpoint{2.950075in}{2.129876in}}%
\pgfpathlineto{\pgfqpoint{2.799848in}{1.304151in}}%
\pgfusepath{stroke}%
\end{pgfscope}%
\begin{pgfscope}%
\pgfpathrectangle{\pgfqpoint{0.100000in}{0.212622in}}{\pgfqpoint{3.696000in}{3.696000in}}%
\pgfusepath{clip}%
\pgfsetrectcap%
\pgfsetroundjoin%
\pgfsetlinewidth{1.505625pt}%
\definecolor{currentstroke}{rgb}{1.000000,0.000000,0.000000}%
\pgfsetstrokecolor{currentstroke}%
\pgfsetdash{}{0pt}%
\pgfpathmoveto{\pgfqpoint{2.949520in}{2.129666in}}%
\pgfpathlineto{\pgfqpoint{2.799848in}{1.304151in}}%
\pgfusepath{stroke}%
\end{pgfscope}%
\begin{pgfscope}%
\pgfpathrectangle{\pgfqpoint{0.100000in}{0.212622in}}{\pgfqpoint{3.696000in}{3.696000in}}%
\pgfusepath{clip}%
\pgfsetrectcap%
\pgfsetroundjoin%
\pgfsetlinewidth{1.505625pt}%
\definecolor{currentstroke}{rgb}{1.000000,0.000000,0.000000}%
\pgfsetstrokecolor{currentstroke}%
\pgfsetdash{}{0pt}%
\pgfpathmoveto{\pgfqpoint{2.949287in}{2.129603in}}%
\pgfpathlineto{\pgfqpoint{2.799848in}{1.304151in}}%
\pgfusepath{stroke}%
\end{pgfscope}%
\begin{pgfscope}%
\pgfpathrectangle{\pgfqpoint{0.100000in}{0.212622in}}{\pgfqpoint{3.696000in}{3.696000in}}%
\pgfusepath{clip}%
\pgfsetrectcap%
\pgfsetroundjoin%
\pgfsetlinewidth{1.505625pt}%
\definecolor{currentstroke}{rgb}{1.000000,0.000000,0.000000}%
\pgfsetstrokecolor{currentstroke}%
\pgfsetdash{}{0pt}%
\pgfpathmoveto{\pgfqpoint{2.947326in}{2.129049in}}%
\pgfpathlineto{\pgfqpoint{2.799848in}{1.304151in}}%
\pgfusepath{stroke}%
\end{pgfscope}%
\begin{pgfscope}%
\pgfpathrectangle{\pgfqpoint{0.100000in}{0.212622in}}{\pgfqpoint{3.696000in}{3.696000in}}%
\pgfusepath{clip}%
\pgfsetrectcap%
\pgfsetroundjoin%
\pgfsetlinewidth{1.505625pt}%
\definecolor{currentstroke}{rgb}{1.000000,0.000000,0.000000}%
\pgfsetstrokecolor{currentstroke}%
\pgfsetdash{}{0pt}%
\pgfpathmoveto{\pgfqpoint{2.944452in}{2.126125in}}%
\pgfpathlineto{\pgfqpoint{2.799848in}{1.304151in}}%
\pgfusepath{stroke}%
\end{pgfscope}%
\begin{pgfscope}%
\pgfpathrectangle{\pgfqpoint{0.100000in}{0.212622in}}{\pgfqpoint{3.696000in}{3.696000in}}%
\pgfusepath{clip}%
\pgfsetrectcap%
\pgfsetroundjoin%
\pgfsetlinewidth{1.505625pt}%
\definecolor{currentstroke}{rgb}{1.000000,0.000000,0.000000}%
\pgfsetstrokecolor{currentstroke}%
\pgfsetdash{}{0pt}%
\pgfpathmoveto{\pgfqpoint{2.936809in}{2.123666in}}%
\pgfpathlineto{\pgfqpoint{2.791632in}{1.296050in}}%
\pgfusepath{stroke}%
\end{pgfscope}%
\begin{pgfscope}%
\pgfpathrectangle{\pgfqpoint{0.100000in}{0.212622in}}{\pgfqpoint{3.696000in}{3.696000in}}%
\pgfusepath{clip}%
\pgfsetrectcap%
\pgfsetroundjoin%
\pgfsetlinewidth{1.505625pt}%
\definecolor{currentstroke}{rgb}{1.000000,0.000000,0.000000}%
\pgfsetstrokecolor{currentstroke}%
\pgfsetdash{}{0pt}%
\pgfpathmoveto{\pgfqpoint{2.928235in}{2.118249in}}%
\pgfpathlineto{\pgfqpoint{2.783404in}{1.287938in}}%
\pgfusepath{stroke}%
\end{pgfscope}%
\begin{pgfscope}%
\pgfpathrectangle{\pgfqpoint{0.100000in}{0.212622in}}{\pgfqpoint{3.696000in}{3.696000in}}%
\pgfusepath{clip}%
\pgfsetrectcap%
\pgfsetroundjoin%
\pgfsetlinewidth{1.505625pt}%
\definecolor{currentstroke}{rgb}{1.000000,0.000000,0.000000}%
\pgfsetstrokecolor{currentstroke}%
\pgfsetdash{}{0pt}%
\pgfpathmoveto{\pgfqpoint{2.913346in}{2.113265in}}%
\pgfpathlineto{\pgfqpoint{2.775166in}{1.279816in}}%
\pgfusepath{stroke}%
\end{pgfscope}%
\begin{pgfscope}%
\pgfpathrectangle{\pgfqpoint{0.100000in}{0.212622in}}{\pgfqpoint{3.696000in}{3.696000in}}%
\pgfusepath{clip}%
\pgfsetrectcap%
\pgfsetroundjoin%
\pgfsetlinewidth{1.505625pt}%
\definecolor{currentstroke}{rgb}{1.000000,0.000000,0.000000}%
\pgfsetstrokecolor{currentstroke}%
\pgfsetdash{}{0pt}%
\pgfpathmoveto{\pgfqpoint{2.900326in}{2.105090in}}%
\pgfpathlineto{\pgfqpoint{2.758658in}{1.263540in}}%
\pgfusepath{stroke}%
\end{pgfscope}%
\begin{pgfscope}%
\pgfpathrectangle{\pgfqpoint{0.100000in}{0.212622in}}{\pgfqpoint{3.696000in}{3.696000in}}%
\pgfusepath{clip}%
\pgfsetrectcap%
\pgfsetroundjoin%
\pgfsetlinewidth{1.505625pt}%
\definecolor{currentstroke}{rgb}{1.000000,0.000000,0.000000}%
\pgfsetstrokecolor{currentstroke}%
\pgfsetdash{}{0pt}%
\pgfpathmoveto{\pgfqpoint{2.880711in}{2.098071in}}%
\pgfpathlineto{\pgfqpoint{2.742106in}{1.247221in}}%
\pgfusepath{stroke}%
\end{pgfscope}%
\begin{pgfscope}%
\pgfpathrectangle{\pgfqpoint{0.100000in}{0.212622in}}{\pgfqpoint{3.696000in}{3.696000in}}%
\pgfusepath{clip}%
\pgfsetrectcap%
\pgfsetroundjoin%
\pgfsetlinewidth{1.505625pt}%
\definecolor{currentstroke}{rgb}{1.000000,0.000000,0.000000}%
\pgfsetstrokecolor{currentstroke}%
\pgfsetdash{}{0pt}%
\pgfpathmoveto{\pgfqpoint{2.872468in}{2.093807in}}%
\pgfpathlineto{\pgfqpoint{2.733813in}{1.239045in}}%
\pgfusepath{stroke}%
\end{pgfscope}%
\begin{pgfscope}%
\pgfpathrectangle{\pgfqpoint{0.100000in}{0.212622in}}{\pgfqpoint{3.696000in}{3.696000in}}%
\pgfusepath{clip}%
\pgfsetrectcap%
\pgfsetroundjoin%
\pgfsetlinewidth{1.505625pt}%
\definecolor{currentstroke}{rgb}{1.000000,0.000000,0.000000}%
\pgfsetstrokecolor{currentstroke}%
\pgfsetdash{}{0pt}%
\pgfpathmoveto{\pgfqpoint{2.867300in}{2.093163in}}%
\pgfpathlineto{\pgfqpoint{2.733813in}{1.239045in}}%
\pgfusepath{stroke}%
\end{pgfscope}%
\begin{pgfscope}%
\pgfpathrectangle{\pgfqpoint{0.100000in}{0.212622in}}{\pgfqpoint{3.696000in}{3.696000in}}%
\pgfusepath{clip}%
\pgfsetrectcap%
\pgfsetroundjoin%
\pgfsetlinewidth{1.505625pt}%
\definecolor{currentstroke}{rgb}{1.000000,0.000000,0.000000}%
\pgfsetstrokecolor{currentstroke}%
\pgfsetdash{}{0pt}%
\pgfpathmoveto{\pgfqpoint{2.864104in}{2.092542in}}%
\pgfpathlineto{\pgfqpoint{2.733813in}{1.239045in}}%
\pgfusepath{stroke}%
\end{pgfscope}%
\begin{pgfscope}%
\pgfpathrectangle{\pgfqpoint{0.100000in}{0.212622in}}{\pgfqpoint{3.696000in}{3.696000in}}%
\pgfusepath{clip}%
\pgfsetrectcap%
\pgfsetroundjoin%
\pgfsetlinewidth{1.505625pt}%
\definecolor{currentstroke}{rgb}{1.000000,0.000000,0.000000}%
\pgfsetstrokecolor{currentstroke}%
\pgfsetdash{}{0pt}%
\pgfpathmoveto{\pgfqpoint{2.862894in}{2.091500in}}%
\pgfpathlineto{\pgfqpoint{2.733813in}{1.239045in}}%
\pgfusepath{stroke}%
\end{pgfscope}%
\begin{pgfscope}%
\pgfpathrectangle{\pgfqpoint{0.100000in}{0.212622in}}{\pgfqpoint{3.696000in}{3.696000in}}%
\pgfusepath{clip}%
\pgfsetrectcap%
\pgfsetroundjoin%
\pgfsetlinewidth{1.505625pt}%
\definecolor{currentstroke}{rgb}{1.000000,0.000000,0.000000}%
\pgfsetstrokecolor{currentstroke}%
\pgfsetdash{}{0pt}%
\pgfpathmoveto{\pgfqpoint{2.862064in}{2.091137in}}%
\pgfpathlineto{\pgfqpoint{2.725510in}{1.230858in}}%
\pgfusepath{stroke}%
\end{pgfscope}%
\begin{pgfscope}%
\pgfpathrectangle{\pgfqpoint{0.100000in}{0.212622in}}{\pgfqpoint{3.696000in}{3.696000in}}%
\pgfusepath{clip}%
\pgfsetrectcap%
\pgfsetroundjoin%
\pgfsetlinewidth{1.505625pt}%
\definecolor{currentstroke}{rgb}{1.000000,0.000000,0.000000}%
\pgfsetstrokecolor{currentstroke}%
\pgfsetdash{}{0pt}%
\pgfpathmoveto{\pgfqpoint{2.859849in}{2.088737in}}%
\pgfpathlineto{\pgfqpoint{2.725510in}{1.230858in}}%
\pgfusepath{stroke}%
\end{pgfscope}%
\begin{pgfscope}%
\pgfpathrectangle{\pgfqpoint{0.100000in}{0.212622in}}{\pgfqpoint{3.696000in}{3.696000in}}%
\pgfusepath{clip}%
\pgfsetrectcap%
\pgfsetroundjoin%
\pgfsetlinewidth{1.505625pt}%
\definecolor{currentstroke}{rgb}{1.000000,0.000000,0.000000}%
\pgfsetstrokecolor{currentstroke}%
\pgfsetdash{}{0pt}%
\pgfpathmoveto{\pgfqpoint{2.855005in}{2.086741in}}%
\pgfpathlineto{\pgfqpoint{2.725510in}{1.230858in}}%
\pgfusepath{stroke}%
\end{pgfscope}%
\begin{pgfscope}%
\pgfpathrectangle{\pgfqpoint{0.100000in}{0.212622in}}{\pgfqpoint{3.696000in}{3.696000in}}%
\pgfusepath{clip}%
\pgfsetrectcap%
\pgfsetroundjoin%
\pgfsetlinewidth{1.505625pt}%
\definecolor{currentstroke}{rgb}{1.000000,0.000000,0.000000}%
\pgfsetstrokecolor{currentstroke}%
\pgfsetdash{}{0pt}%
\pgfpathmoveto{\pgfqpoint{2.849418in}{2.082820in}}%
\pgfpathlineto{\pgfqpoint{2.717196in}{1.222661in}}%
\pgfusepath{stroke}%
\end{pgfscope}%
\begin{pgfscope}%
\pgfpathrectangle{\pgfqpoint{0.100000in}{0.212622in}}{\pgfqpoint{3.696000in}{3.696000in}}%
\pgfusepath{clip}%
\pgfsetrectcap%
\pgfsetroundjoin%
\pgfsetlinewidth{1.505625pt}%
\definecolor{currentstroke}{rgb}{1.000000,0.000000,0.000000}%
\pgfsetstrokecolor{currentstroke}%
\pgfsetdash{}{0pt}%
\pgfpathmoveto{\pgfqpoint{2.839405in}{2.079006in}}%
\pgfpathlineto{\pgfqpoint{2.708870in}{1.214453in}}%
\pgfusepath{stroke}%
\end{pgfscope}%
\begin{pgfscope}%
\pgfpathrectangle{\pgfqpoint{0.100000in}{0.212622in}}{\pgfqpoint{3.696000in}{3.696000in}}%
\pgfusepath{clip}%
\pgfsetrectcap%
\pgfsetroundjoin%
\pgfsetlinewidth{1.505625pt}%
\definecolor{currentstroke}{rgb}{1.000000,0.000000,0.000000}%
\pgfsetstrokecolor{currentstroke}%
\pgfsetdash{}{0pt}%
\pgfpathmoveto{\pgfqpoint{2.829916in}{2.074616in}}%
\pgfpathlineto{\pgfqpoint{2.700534in}{1.206234in}}%
\pgfusepath{stroke}%
\end{pgfscope}%
\begin{pgfscope}%
\pgfpathrectangle{\pgfqpoint{0.100000in}{0.212622in}}{\pgfqpoint{3.696000in}{3.696000in}}%
\pgfusepath{clip}%
\pgfsetrectcap%
\pgfsetroundjoin%
\pgfsetlinewidth{1.505625pt}%
\definecolor{currentstroke}{rgb}{1.000000,0.000000,0.000000}%
\pgfsetstrokecolor{currentstroke}%
\pgfsetdash{}{0pt}%
\pgfpathmoveto{\pgfqpoint{2.817734in}{2.068806in}}%
\pgfpathlineto{\pgfqpoint{2.692187in}{1.198004in}}%
\pgfusepath{stroke}%
\end{pgfscope}%
\begin{pgfscope}%
\pgfpathrectangle{\pgfqpoint{0.100000in}{0.212622in}}{\pgfqpoint{3.696000in}{3.696000in}}%
\pgfusepath{clip}%
\pgfsetrectcap%
\pgfsetroundjoin%
\pgfsetlinewidth{1.505625pt}%
\definecolor{currentstroke}{rgb}{1.000000,0.000000,0.000000}%
\pgfsetstrokecolor{currentstroke}%
\pgfsetdash{}{0pt}%
\pgfpathmoveto{\pgfqpoint{2.812014in}{2.066701in}}%
\pgfpathlineto{\pgfqpoint{2.683828in}{1.189763in}}%
\pgfusepath{stroke}%
\end{pgfscope}%
\begin{pgfscope}%
\pgfpathrectangle{\pgfqpoint{0.100000in}{0.212622in}}{\pgfqpoint{3.696000in}{3.696000in}}%
\pgfusepath{clip}%
\pgfsetrectcap%
\pgfsetroundjoin%
\pgfsetlinewidth{1.505625pt}%
\definecolor{currentstroke}{rgb}{1.000000,0.000000,0.000000}%
\pgfsetstrokecolor{currentstroke}%
\pgfsetdash{}{0pt}%
\pgfpathmoveto{\pgfqpoint{2.808488in}{2.065502in}}%
\pgfpathlineto{\pgfqpoint{2.683828in}{1.189763in}}%
\pgfusepath{stroke}%
\end{pgfscope}%
\begin{pgfscope}%
\pgfpathrectangle{\pgfqpoint{0.100000in}{0.212622in}}{\pgfqpoint{3.696000in}{3.696000in}}%
\pgfusepath{clip}%
\pgfsetrectcap%
\pgfsetroundjoin%
\pgfsetlinewidth{1.505625pt}%
\definecolor{currentstroke}{rgb}{1.000000,0.000000,0.000000}%
\pgfsetstrokecolor{currentstroke}%
\pgfsetdash{}{0pt}%
\pgfpathmoveto{\pgfqpoint{2.806427in}{2.064691in}}%
\pgfpathlineto{\pgfqpoint{2.675459in}{1.181511in}}%
\pgfusepath{stroke}%
\end{pgfscope}%
\begin{pgfscope}%
\pgfpathrectangle{\pgfqpoint{0.100000in}{0.212622in}}{\pgfqpoint{3.696000in}{3.696000in}}%
\pgfusepath{clip}%
\pgfsetrectcap%
\pgfsetroundjoin%
\pgfsetlinewidth{1.505625pt}%
\definecolor{currentstroke}{rgb}{1.000000,0.000000,0.000000}%
\pgfsetstrokecolor{currentstroke}%
\pgfsetdash{}{0pt}%
\pgfpathmoveto{\pgfqpoint{2.805715in}{2.063906in}}%
\pgfpathlineto{\pgfqpoint{2.675459in}{1.181511in}}%
\pgfusepath{stroke}%
\end{pgfscope}%
\begin{pgfscope}%
\pgfpathrectangle{\pgfqpoint{0.100000in}{0.212622in}}{\pgfqpoint{3.696000in}{3.696000in}}%
\pgfusepath{clip}%
\pgfsetrectcap%
\pgfsetroundjoin%
\pgfsetlinewidth{1.505625pt}%
\definecolor{currentstroke}{rgb}{1.000000,0.000000,0.000000}%
\pgfsetstrokecolor{currentstroke}%
\pgfsetdash{}{0pt}%
\pgfpathmoveto{\pgfqpoint{2.800412in}{2.062049in}}%
\pgfpathlineto{\pgfqpoint{2.675459in}{1.181511in}}%
\pgfusepath{stroke}%
\end{pgfscope}%
\begin{pgfscope}%
\pgfpathrectangle{\pgfqpoint{0.100000in}{0.212622in}}{\pgfqpoint{3.696000in}{3.696000in}}%
\pgfusepath{clip}%
\pgfsetrectcap%
\pgfsetroundjoin%
\pgfsetlinewidth{1.505625pt}%
\definecolor{currentstroke}{rgb}{1.000000,0.000000,0.000000}%
\pgfsetstrokecolor{currentstroke}%
\pgfsetdash{}{0pt}%
\pgfpathmoveto{\pgfqpoint{2.794047in}{2.055411in}}%
\pgfpathlineto{\pgfqpoint{2.667078in}{1.173249in}}%
\pgfusepath{stroke}%
\end{pgfscope}%
\begin{pgfscope}%
\pgfpathrectangle{\pgfqpoint{0.100000in}{0.212622in}}{\pgfqpoint{3.696000in}{3.696000in}}%
\pgfusepath{clip}%
\pgfsetrectcap%
\pgfsetroundjoin%
\pgfsetlinewidth{1.505625pt}%
\definecolor{currentstroke}{rgb}{1.000000,0.000000,0.000000}%
\pgfsetstrokecolor{currentstroke}%
\pgfsetdash{}{0pt}%
\pgfpathmoveto{\pgfqpoint{2.782218in}{2.051724in}}%
\pgfpathlineto{\pgfqpoint{2.658687in}{1.164975in}}%
\pgfusepath{stroke}%
\end{pgfscope}%
\begin{pgfscope}%
\pgfpathrectangle{\pgfqpoint{0.100000in}{0.212622in}}{\pgfqpoint{3.696000in}{3.696000in}}%
\pgfusepath{clip}%
\pgfsetrectcap%
\pgfsetroundjoin%
\pgfsetlinewidth{1.505625pt}%
\definecolor{currentstroke}{rgb}{1.000000,0.000000,0.000000}%
\pgfsetstrokecolor{currentstroke}%
\pgfsetdash{}{0pt}%
\pgfpathmoveto{\pgfqpoint{2.771357in}{2.043761in}}%
\pgfpathlineto{\pgfqpoint{2.650284in}{1.156690in}}%
\pgfusepath{stroke}%
\end{pgfscope}%
\begin{pgfscope}%
\pgfpathrectangle{\pgfqpoint{0.100000in}{0.212622in}}{\pgfqpoint{3.696000in}{3.696000in}}%
\pgfusepath{clip}%
\pgfsetrectcap%
\pgfsetroundjoin%
\pgfsetlinewidth{1.505625pt}%
\definecolor{currentstroke}{rgb}{1.000000,0.000000,0.000000}%
\pgfsetstrokecolor{currentstroke}%
\pgfsetdash{}{0pt}%
\pgfpathmoveto{\pgfqpoint{2.752887in}{2.038146in}}%
\pgfpathlineto{\pgfqpoint{2.633444in}{1.140088in}}%
\pgfusepath{stroke}%
\end{pgfscope}%
\begin{pgfscope}%
\pgfpathrectangle{\pgfqpoint{0.100000in}{0.212622in}}{\pgfqpoint{3.696000in}{3.696000in}}%
\pgfusepath{clip}%
\pgfsetrectcap%
\pgfsetroundjoin%
\pgfsetlinewidth{1.505625pt}%
\definecolor{currentstroke}{rgb}{1.000000,0.000000,0.000000}%
\pgfsetstrokecolor{currentstroke}%
\pgfsetdash{}{0pt}%
\pgfpathmoveto{\pgfqpoint{2.738655in}{2.027941in}}%
\pgfpathlineto{\pgfqpoint{2.616560in}{1.123441in}}%
\pgfusepath{stroke}%
\end{pgfscope}%
\begin{pgfscope}%
\pgfpathrectangle{\pgfqpoint{0.100000in}{0.212622in}}{\pgfqpoint{3.696000in}{3.696000in}}%
\pgfusepath{clip}%
\pgfsetrectcap%
\pgfsetroundjoin%
\pgfsetlinewidth{1.505625pt}%
\definecolor{currentstroke}{rgb}{1.000000,0.000000,0.000000}%
\pgfsetstrokecolor{currentstroke}%
\pgfsetdash{}{0pt}%
\pgfpathmoveto{\pgfqpoint{2.728781in}{2.023455in}}%
\pgfpathlineto{\pgfqpoint{2.608102in}{1.115102in}}%
\pgfusepath{stroke}%
\end{pgfscope}%
\begin{pgfscope}%
\pgfpathrectangle{\pgfqpoint{0.100000in}{0.212622in}}{\pgfqpoint{3.696000in}{3.696000in}}%
\pgfusepath{clip}%
\pgfsetrectcap%
\pgfsetroundjoin%
\pgfsetlinewidth{1.505625pt}%
\definecolor{currentstroke}{rgb}{1.000000,0.000000,0.000000}%
\pgfsetstrokecolor{currentstroke}%
\pgfsetdash{}{0pt}%
\pgfpathmoveto{\pgfqpoint{2.723485in}{2.021516in}}%
\pgfpathlineto{\pgfqpoint{2.608102in}{1.115102in}}%
\pgfusepath{stroke}%
\end{pgfscope}%
\begin{pgfscope}%
\pgfpathrectangle{\pgfqpoint{0.100000in}{0.212622in}}{\pgfqpoint{3.696000in}{3.696000in}}%
\pgfusepath{clip}%
\pgfsetrectcap%
\pgfsetroundjoin%
\pgfsetlinewidth{1.505625pt}%
\definecolor{currentstroke}{rgb}{1.000000,0.000000,0.000000}%
\pgfsetstrokecolor{currentstroke}%
\pgfsetdash{}{0pt}%
\pgfpathmoveto{\pgfqpoint{2.720689in}{2.020259in}}%
\pgfpathlineto{\pgfqpoint{2.608102in}{1.115102in}}%
\pgfusepath{stroke}%
\end{pgfscope}%
\begin{pgfscope}%
\pgfpathrectangle{\pgfqpoint{0.100000in}{0.212622in}}{\pgfqpoint{3.696000in}{3.696000in}}%
\pgfusepath{clip}%
\pgfsetrectcap%
\pgfsetroundjoin%
\pgfsetlinewidth{1.505625pt}%
\definecolor{currentstroke}{rgb}{1.000000,0.000000,0.000000}%
\pgfsetstrokecolor{currentstroke}%
\pgfsetdash{}{0pt}%
\pgfpathmoveto{\pgfqpoint{2.719155in}{2.020161in}}%
\pgfpathlineto{\pgfqpoint{2.599631in}{1.106751in}}%
\pgfusepath{stroke}%
\end{pgfscope}%
\begin{pgfscope}%
\pgfpathrectangle{\pgfqpoint{0.100000in}{0.212622in}}{\pgfqpoint{3.696000in}{3.696000in}}%
\pgfusepath{clip}%
\pgfsetrectcap%
\pgfsetroundjoin%
\pgfsetlinewidth{1.505625pt}%
\definecolor{currentstroke}{rgb}{1.000000,0.000000,0.000000}%
\pgfsetstrokecolor{currentstroke}%
\pgfsetdash{}{0pt}%
\pgfpathmoveto{\pgfqpoint{2.718516in}{2.019461in}}%
\pgfpathlineto{\pgfqpoint{2.599631in}{1.106751in}}%
\pgfusepath{stroke}%
\end{pgfscope}%
\begin{pgfscope}%
\pgfpathrectangle{\pgfqpoint{0.100000in}{0.212622in}}{\pgfqpoint{3.696000in}{3.696000in}}%
\pgfusepath{clip}%
\pgfsetrectcap%
\pgfsetroundjoin%
\pgfsetlinewidth{1.505625pt}%
\definecolor{currentstroke}{rgb}{1.000000,0.000000,0.000000}%
\pgfsetstrokecolor{currentstroke}%
\pgfsetdash{}{0pt}%
\pgfpathmoveto{\pgfqpoint{2.718074in}{2.019308in}}%
\pgfpathlineto{\pgfqpoint{2.599631in}{1.106751in}}%
\pgfusepath{stroke}%
\end{pgfscope}%
\begin{pgfscope}%
\pgfpathrectangle{\pgfqpoint{0.100000in}{0.212622in}}{\pgfqpoint{3.696000in}{3.696000in}}%
\pgfusepath{clip}%
\pgfsetrectcap%
\pgfsetroundjoin%
\pgfsetlinewidth{1.505625pt}%
\definecolor{currentstroke}{rgb}{1.000000,0.000000,0.000000}%
\pgfsetstrokecolor{currentstroke}%
\pgfsetdash{}{0pt}%
\pgfpathmoveto{\pgfqpoint{2.715126in}{2.015650in}}%
\pgfpathlineto{\pgfqpoint{2.599631in}{1.106751in}}%
\pgfusepath{stroke}%
\end{pgfscope}%
\begin{pgfscope}%
\pgfpathrectangle{\pgfqpoint{0.100000in}{0.212622in}}{\pgfqpoint{3.696000in}{3.696000in}}%
\pgfusepath{clip}%
\pgfsetrectcap%
\pgfsetroundjoin%
\pgfsetlinewidth{1.505625pt}%
\definecolor{currentstroke}{rgb}{1.000000,0.000000,0.000000}%
\pgfsetstrokecolor{currentstroke}%
\pgfsetdash{}{0pt}%
\pgfpathmoveto{\pgfqpoint{2.710125in}{2.015924in}}%
\pgfpathlineto{\pgfqpoint{2.591150in}{1.098388in}}%
\pgfusepath{stroke}%
\end{pgfscope}%
\begin{pgfscope}%
\pgfpathrectangle{\pgfqpoint{0.100000in}{0.212622in}}{\pgfqpoint{3.696000in}{3.696000in}}%
\pgfusepath{clip}%
\pgfsetrectcap%
\pgfsetroundjoin%
\pgfsetlinewidth{1.505625pt}%
\definecolor{currentstroke}{rgb}{1.000000,0.000000,0.000000}%
\pgfsetstrokecolor{currentstroke}%
\pgfsetdash{}{0pt}%
\pgfpathmoveto{\pgfqpoint{2.701946in}{2.004843in}}%
\pgfpathlineto{\pgfqpoint{2.582657in}{1.090015in}}%
\pgfusepath{stroke}%
\end{pgfscope}%
\begin{pgfscope}%
\pgfpathrectangle{\pgfqpoint{0.100000in}{0.212622in}}{\pgfqpoint{3.696000in}{3.696000in}}%
\pgfusepath{clip}%
\pgfsetrectcap%
\pgfsetroundjoin%
\pgfsetlinewidth{1.505625pt}%
\definecolor{currentstroke}{rgb}{1.000000,0.000000,0.000000}%
\pgfsetstrokecolor{currentstroke}%
\pgfsetdash{}{0pt}%
\pgfpathmoveto{\pgfqpoint{2.689869in}{2.001199in}}%
\pgfpathlineto{\pgfqpoint{2.574153in}{1.081631in}}%
\pgfusepath{stroke}%
\end{pgfscope}%
\begin{pgfscope}%
\pgfpathrectangle{\pgfqpoint{0.100000in}{0.212622in}}{\pgfqpoint{3.696000in}{3.696000in}}%
\pgfusepath{clip}%
\pgfsetrectcap%
\pgfsetroundjoin%
\pgfsetlinewidth{1.505625pt}%
\definecolor{currentstroke}{rgb}{1.000000,0.000000,0.000000}%
\pgfsetstrokecolor{currentstroke}%
\pgfsetdash{}{0pt}%
\pgfpathmoveto{\pgfqpoint{2.678161in}{1.989933in}}%
\pgfpathlineto{\pgfqpoint{2.557111in}{1.064828in}}%
\pgfusepath{stroke}%
\end{pgfscope}%
\begin{pgfscope}%
\pgfpathrectangle{\pgfqpoint{0.100000in}{0.212622in}}{\pgfqpoint{3.696000in}{3.696000in}}%
\pgfusepath{clip}%
\pgfsetrectcap%
\pgfsetroundjoin%
\pgfsetlinewidth{1.505625pt}%
\definecolor{currentstroke}{rgb}{1.000000,0.000000,0.000000}%
\pgfsetstrokecolor{currentstroke}%
\pgfsetdash{}{0pt}%
\pgfpathmoveto{\pgfqpoint{2.662128in}{1.981202in}}%
\pgfpathlineto{\pgfqpoint{2.548573in}{1.056410in}}%
\pgfusepath{stroke}%
\end{pgfscope}%
\begin{pgfscope}%
\pgfpathrectangle{\pgfqpoint{0.100000in}{0.212622in}}{\pgfqpoint{3.696000in}{3.696000in}}%
\pgfusepath{clip}%
\pgfsetrectcap%
\pgfsetroundjoin%
\pgfsetlinewidth{1.505625pt}%
\definecolor{currentstroke}{rgb}{1.000000,0.000000,0.000000}%
\pgfsetstrokecolor{currentstroke}%
\pgfsetdash{}{0pt}%
\pgfpathmoveto{\pgfqpoint{2.653374in}{1.976731in}}%
\pgfpathlineto{\pgfqpoint{2.540023in}{1.047980in}}%
\pgfusepath{stroke}%
\end{pgfscope}%
\begin{pgfscope}%
\pgfpathrectangle{\pgfqpoint{0.100000in}{0.212622in}}{\pgfqpoint{3.696000in}{3.696000in}}%
\pgfusepath{clip}%
\pgfsetrectcap%
\pgfsetroundjoin%
\pgfsetlinewidth{1.505625pt}%
\definecolor{currentstroke}{rgb}{1.000000,0.000000,0.000000}%
\pgfsetstrokecolor{currentstroke}%
\pgfsetdash{}{0pt}%
\pgfpathmoveto{\pgfqpoint{2.648447in}{1.974258in}}%
\pgfpathlineto{\pgfqpoint{2.531462in}{1.039540in}}%
\pgfusepath{stroke}%
\end{pgfscope}%
\begin{pgfscope}%
\pgfpathrectangle{\pgfqpoint{0.100000in}{0.212622in}}{\pgfqpoint{3.696000in}{3.696000in}}%
\pgfusepath{clip}%
\pgfsetrectcap%
\pgfsetroundjoin%
\pgfsetlinewidth{1.505625pt}%
\definecolor{currentstroke}{rgb}{1.000000,0.000000,0.000000}%
\pgfsetstrokecolor{currentstroke}%
\pgfsetdash{}{0pt}%
\pgfpathmoveto{\pgfqpoint{2.645333in}{1.972773in}}%
\pgfpathlineto{\pgfqpoint{2.531462in}{1.039540in}}%
\pgfusepath{stroke}%
\end{pgfscope}%
\begin{pgfscope}%
\pgfpathrectangle{\pgfqpoint{0.100000in}{0.212622in}}{\pgfqpoint{3.696000in}{3.696000in}}%
\pgfusepath{clip}%
\pgfsetrectcap%
\pgfsetroundjoin%
\pgfsetlinewidth{1.505625pt}%
\definecolor{currentstroke}{rgb}{1.000000,0.000000,0.000000}%
\pgfsetstrokecolor{currentstroke}%
\pgfsetdash{}{0pt}%
\pgfpathmoveto{\pgfqpoint{2.643691in}{1.971976in}}%
\pgfpathlineto{\pgfqpoint{2.531462in}{1.039540in}}%
\pgfusepath{stroke}%
\end{pgfscope}%
\begin{pgfscope}%
\pgfpathrectangle{\pgfqpoint{0.100000in}{0.212622in}}{\pgfqpoint{3.696000in}{3.696000in}}%
\pgfusepath{clip}%
\pgfsetrectcap%
\pgfsetroundjoin%
\pgfsetlinewidth{1.505625pt}%
\definecolor{currentstroke}{rgb}{1.000000,0.000000,0.000000}%
\pgfsetstrokecolor{currentstroke}%
\pgfsetdash{}{0pt}%
\pgfpathmoveto{\pgfqpoint{2.642883in}{1.971858in}}%
\pgfpathlineto{\pgfqpoint{2.531462in}{1.039540in}}%
\pgfusepath{stroke}%
\end{pgfscope}%
\begin{pgfscope}%
\pgfpathrectangle{\pgfqpoint{0.100000in}{0.212622in}}{\pgfqpoint{3.696000in}{3.696000in}}%
\pgfusepath{clip}%
\pgfsetrectcap%
\pgfsetroundjoin%
\pgfsetlinewidth{1.505625pt}%
\definecolor{currentstroke}{rgb}{1.000000,0.000000,0.000000}%
\pgfsetstrokecolor{currentstroke}%
\pgfsetdash{}{0pt}%
\pgfpathmoveto{\pgfqpoint{2.640882in}{1.969250in}}%
\pgfpathlineto{\pgfqpoint{2.531462in}{1.039540in}}%
\pgfusepath{stroke}%
\end{pgfscope}%
\begin{pgfscope}%
\pgfpathrectangle{\pgfqpoint{0.100000in}{0.212622in}}{\pgfqpoint{3.696000in}{3.696000in}}%
\pgfusepath{clip}%
\pgfsetrectcap%
\pgfsetroundjoin%
\pgfsetlinewidth{1.505625pt}%
\definecolor{currentstroke}{rgb}{1.000000,0.000000,0.000000}%
\pgfsetstrokecolor{currentstroke}%
\pgfsetdash{}{0pt}%
\pgfpathmoveto{\pgfqpoint{2.636345in}{1.968078in}}%
\pgfpathlineto{\pgfqpoint{2.522889in}{1.031088in}}%
\pgfusepath{stroke}%
\end{pgfscope}%
\begin{pgfscope}%
\pgfpathrectangle{\pgfqpoint{0.100000in}{0.212622in}}{\pgfqpoint{3.696000in}{3.696000in}}%
\pgfusepath{clip}%
\pgfsetrectcap%
\pgfsetroundjoin%
\pgfsetlinewidth{1.505625pt}%
\definecolor{currentstroke}{rgb}{1.000000,0.000000,0.000000}%
\pgfsetstrokecolor{currentstroke}%
\pgfsetdash{}{0pt}%
\pgfpathmoveto{\pgfqpoint{2.630133in}{1.964574in}}%
\pgfpathlineto{\pgfqpoint{2.514305in}{1.022624in}}%
\pgfusepath{stroke}%
\end{pgfscope}%
\begin{pgfscope}%
\pgfpathrectangle{\pgfqpoint{0.100000in}{0.212622in}}{\pgfqpoint{3.696000in}{3.696000in}}%
\pgfusepath{clip}%
\pgfsetrectcap%
\pgfsetroundjoin%
\pgfsetlinewidth{1.505625pt}%
\definecolor{currentstroke}{rgb}{1.000000,0.000000,0.000000}%
\pgfsetstrokecolor{currentstroke}%
\pgfsetdash{}{0pt}%
\pgfpathmoveto{\pgfqpoint{2.620329in}{1.961676in}}%
\pgfpathlineto{\pgfqpoint{2.514305in}{1.022624in}}%
\pgfusepath{stroke}%
\end{pgfscope}%
\begin{pgfscope}%
\pgfpathrectangle{\pgfqpoint{0.100000in}{0.212622in}}{\pgfqpoint{3.696000in}{3.696000in}}%
\pgfusepath{clip}%
\pgfsetrectcap%
\pgfsetroundjoin%
\pgfsetlinewidth{1.505625pt}%
\definecolor{currentstroke}{rgb}{1.000000,0.000000,0.000000}%
\pgfsetstrokecolor{currentstroke}%
\pgfsetdash{}{0pt}%
\pgfpathmoveto{\pgfqpoint{2.609915in}{1.956116in}}%
\pgfpathlineto{\pgfqpoint{2.497102in}{1.005663in}}%
\pgfusepath{stroke}%
\end{pgfscope}%
\begin{pgfscope}%
\pgfpathrectangle{\pgfqpoint{0.100000in}{0.212622in}}{\pgfqpoint{3.696000in}{3.696000in}}%
\pgfusepath{clip}%
\pgfsetrectcap%
\pgfsetroundjoin%
\pgfsetlinewidth{1.505625pt}%
\definecolor{currentstroke}{rgb}{1.000000,0.000000,0.000000}%
\pgfsetstrokecolor{currentstroke}%
\pgfsetdash{}{0pt}%
\pgfpathmoveto{\pgfqpoint{2.596745in}{1.950138in}}%
\pgfpathlineto{\pgfqpoint{2.488483in}{0.997166in}}%
\pgfusepath{stroke}%
\end{pgfscope}%
\begin{pgfscope}%
\pgfpathrectangle{\pgfqpoint{0.100000in}{0.212622in}}{\pgfqpoint{3.696000in}{3.696000in}}%
\pgfusepath{clip}%
\pgfsetrectcap%
\pgfsetroundjoin%
\pgfsetlinewidth{1.505625pt}%
\definecolor{currentstroke}{rgb}{1.000000,0.000000,0.000000}%
\pgfsetstrokecolor{currentstroke}%
\pgfsetdash{}{0pt}%
\pgfpathmoveto{\pgfqpoint{2.590690in}{1.948383in}}%
\pgfpathlineto{\pgfqpoint{2.488483in}{0.997166in}}%
\pgfusepath{stroke}%
\end{pgfscope}%
\begin{pgfscope}%
\pgfpathrectangle{\pgfqpoint{0.100000in}{0.212622in}}{\pgfqpoint{3.696000in}{3.696000in}}%
\pgfusepath{clip}%
\pgfsetrectcap%
\pgfsetroundjoin%
\pgfsetlinewidth{1.505625pt}%
\definecolor{currentstroke}{rgb}{1.000000,0.000000,0.000000}%
\pgfsetstrokecolor{currentstroke}%
\pgfsetdash{}{0pt}%
\pgfpathmoveto{\pgfqpoint{2.586525in}{1.946460in}}%
\pgfpathlineto{\pgfqpoint{2.479853in}{0.988657in}}%
\pgfusepath{stroke}%
\end{pgfscope}%
\begin{pgfscope}%
\pgfpathrectangle{\pgfqpoint{0.100000in}{0.212622in}}{\pgfqpoint{3.696000in}{3.696000in}}%
\pgfusepath{clip}%
\pgfsetrectcap%
\pgfsetroundjoin%
\pgfsetlinewidth{1.505625pt}%
\definecolor{currentstroke}{rgb}{1.000000,0.000000,0.000000}%
\pgfsetstrokecolor{currentstroke}%
\pgfsetdash{}{0pt}%
\pgfpathmoveto{\pgfqpoint{2.584653in}{1.946094in}}%
\pgfpathlineto{\pgfqpoint{2.479853in}{0.988657in}}%
\pgfusepath{stroke}%
\end{pgfscope}%
\begin{pgfscope}%
\pgfpathrectangle{\pgfqpoint{0.100000in}{0.212622in}}{\pgfqpoint{3.696000in}{3.696000in}}%
\pgfusepath{clip}%
\pgfsetrectcap%
\pgfsetroundjoin%
\pgfsetlinewidth{1.505625pt}%
\definecolor{currentstroke}{rgb}{1.000000,0.000000,0.000000}%
\pgfsetstrokecolor{currentstroke}%
\pgfsetdash{}{0pt}%
\pgfpathmoveto{\pgfqpoint{2.583624in}{1.945433in}}%
\pgfpathlineto{\pgfqpoint{2.479853in}{0.988657in}}%
\pgfusepath{stroke}%
\end{pgfscope}%
\begin{pgfscope}%
\pgfpathrectangle{\pgfqpoint{0.100000in}{0.212622in}}{\pgfqpoint{3.696000in}{3.696000in}}%
\pgfusepath{clip}%
\pgfsetrectcap%
\pgfsetroundjoin%
\pgfsetlinewidth{1.505625pt}%
\definecolor{currentstroke}{rgb}{1.000000,0.000000,0.000000}%
\pgfsetstrokecolor{currentstroke}%
\pgfsetdash{}{0pt}%
\pgfpathmoveto{\pgfqpoint{2.583013in}{1.945277in}}%
\pgfpathlineto{\pgfqpoint{2.479853in}{0.988657in}}%
\pgfusepath{stroke}%
\end{pgfscope}%
\begin{pgfscope}%
\pgfpathrectangle{\pgfqpoint{0.100000in}{0.212622in}}{\pgfqpoint{3.696000in}{3.696000in}}%
\pgfusepath{clip}%
\pgfsetrectcap%
\pgfsetroundjoin%
\pgfsetlinewidth{1.505625pt}%
\definecolor{currentstroke}{rgb}{1.000000,0.000000,0.000000}%
\pgfsetstrokecolor{currentstroke}%
\pgfsetdash{}{0pt}%
\pgfpathmoveto{\pgfqpoint{2.579480in}{1.941753in}}%
\pgfpathlineto{\pgfqpoint{2.471211in}{0.980136in}}%
\pgfusepath{stroke}%
\end{pgfscope}%
\begin{pgfscope}%
\pgfpathrectangle{\pgfqpoint{0.100000in}{0.212622in}}{\pgfqpoint{3.696000in}{3.696000in}}%
\pgfusepath{clip}%
\pgfsetrectcap%
\pgfsetroundjoin%
\pgfsetlinewidth{1.505625pt}%
\definecolor{currentstroke}{rgb}{1.000000,0.000000,0.000000}%
\pgfsetstrokecolor{currentstroke}%
\pgfsetdash{}{0pt}%
\pgfpathmoveto{\pgfqpoint{2.573391in}{1.940029in}}%
\pgfpathlineto{\pgfqpoint{2.471211in}{0.980136in}}%
\pgfusepath{stroke}%
\end{pgfscope}%
\begin{pgfscope}%
\pgfpathrectangle{\pgfqpoint{0.100000in}{0.212622in}}{\pgfqpoint{3.696000in}{3.696000in}}%
\pgfusepath{clip}%
\pgfsetrectcap%
\pgfsetroundjoin%
\pgfsetlinewidth{1.505625pt}%
\definecolor{currentstroke}{rgb}{1.000000,0.000000,0.000000}%
\pgfsetstrokecolor{currentstroke}%
\pgfsetdash{}{0pt}%
\pgfpathmoveto{\pgfqpoint{2.566111in}{1.933413in}}%
\pgfpathlineto{\pgfqpoint{2.462557in}{0.971604in}}%
\pgfusepath{stroke}%
\end{pgfscope}%
\begin{pgfscope}%
\pgfpathrectangle{\pgfqpoint{0.100000in}{0.212622in}}{\pgfqpoint{3.696000in}{3.696000in}}%
\pgfusepath{clip}%
\pgfsetrectcap%
\pgfsetroundjoin%
\pgfsetlinewidth{1.505625pt}%
\definecolor{currentstroke}{rgb}{1.000000,0.000000,0.000000}%
\pgfsetstrokecolor{currentstroke}%
\pgfsetdash{}{0pt}%
\pgfpathmoveto{\pgfqpoint{2.561468in}{1.931013in}}%
\pgfpathlineto{\pgfqpoint{2.462557in}{0.971604in}}%
\pgfusepath{stroke}%
\end{pgfscope}%
\begin{pgfscope}%
\pgfpathrectangle{\pgfqpoint{0.100000in}{0.212622in}}{\pgfqpoint{3.696000in}{3.696000in}}%
\pgfusepath{clip}%
\pgfsetrectcap%
\pgfsetroundjoin%
\pgfsetlinewidth{1.505625pt}%
\definecolor{currentstroke}{rgb}{1.000000,0.000000,0.000000}%
\pgfsetstrokecolor{currentstroke}%
\pgfsetdash{}{0pt}%
\pgfpathmoveto{\pgfqpoint{2.558921in}{1.929920in}}%
\pgfpathlineto{\pgfqpoint{2.453892in}{0.963061in}}%
\pgfusepath{stroke}%
\end{pgfscope}%
\begin{pgfscope}%
\pgfpathrectangle{\pgfqpoint{0.100000in}{0.212622in}}{\pgfqpoint{3.696000in}{3.696000in}}%
\pgfusepath{clip}%
\pgfsetrectcap%
\pgfsetroundjoin%
\pgfsetlinewidth{1.505625pt}%
\definecolor{currentstroke}{rgb}{1.000000,0.000000,0.000000}%
\pgfsetstrokecolor{currentstroke}%
\pgfsetdash{}{0pt}%
\pgfpathmoveto{\pgfqpoint{2.557552in}{1.929327in}}%
\pgfpathlineto{\pgfqpoint{2.453892in}{0.963061in}}%
\pgfusepath{stroke}%
\end{pgfscope}%
\begin{pgfscope}%
\pgfpathrectangle{\pgfqpoint{0.100000in}{0.212622in}}{\pgfqpoint{3.696000in}{3.696000in}}%
\pgfusepath{clip}%
\pgfsetrectcap%
\pgfsetroundjoin%
\pgfsetlinewidth{1.505625pt}%
\definecolor{currentstroke}{rgb}{1.000000,0.000000,0.000000}%
\pgfsetstrokecolor{currentstroke}%
\pgfsetdash{}{0pt}%
\pgfpathmoveto{\pgfqpoint{2.556648in}{1.929202in}}%
\pgfpathlineto{\pgfqpoint{2.453892in}{0.963061in}}%
\pgfusepath{stroke}%
\end{pgfscope}%
\begin{pgfscope}%
\pgfpathrectangle{\pgfqpoint{0.100000in}{0.212622in}}{\pgfqpoint{3.696000in}{3.696000in}}%
\pgfusepath{clip}%
\pgfsetrectcap%
\pgfsetroundjoin%
\pgfsetlinewidth{1.505625pt}%
\definecolor{currentstroke}{rgb}{1.000000,0.000000,0.000000}%
\pgfsetstrokecolor{currentstroke}%
\pgfsetdash{}{0pt}%
\pgfpathmoveto{\pgfqpoint{2.556274in}{1.928810in}}%
\pgfpathlineto{\pgfqpoint{2.453892in}{0.963061in}}%
\pgfusepath{stroke}%
\end{pgfscope}%
\begin{pgfscope}%
\pgfpathrectangle{\pgfqpoint{0.100000in}{0.212622in}}{\pgfqpoint{3.696000in}{3.696000in}}%
\pgfusepath{clip}%
\pgfsetrectcap%
\pgfsetroundjoin%
\pgfsetlinewidth{1.505625pt}%
\definecolor{currentstroke}{rgb}{1.000000,0.000000,0.000000}%
\pgfsetstrokecolor{currentstroke}%
\pgfsetdash{}{0pt}%
\pgfpathmoveto{\pgfqpoint{2.556089in}{1.928813in}}%
\pgfpathlineto{\pgfqpoint{2.453892in}{0.963061in}}%
\pgfusepath{stroke}%
\end{pgfscope}%
\begin{pgfscope}%
\pgfpathrectangle{\pgfqpoint{0.100000in}{0.212622in}}{\pgfqpoint{3.696000in}{3.696000in}}%
\pgfusepath{clip}%
\pgfsetrectcap%
\pgfsetroundjoin%
\pgfsetlinewidth{1.505625pt}%
\definecolor{currentstroke}{rgb}{1.000000,0.000000,0.000000}%
\pgfsetstrokecolor{currentstroke}%
\pgfsetdash{}{0pt}%
\pgfpathmoveto{\pgfqpoint{2.554277in}{1.926619in}}%
\pgfpathlineto{\pgfqpoint{2.453892in}{0.963061in}}%
\pgfusepath{stroke}%
\end{pgfscope}%
\begin{pgfscope}%
\pgfpathrectangle{\pgfqpoint{0.100000in}{0.212622in}}{\pgfqpoint{3.696000in}{3.696000in}}%
\pgfusepath{clip}%
\pgfsetrectcap%
\pgfsetroundjoin%
\pgfsetlinewidth{1.505625pt}%
\definecolor{currentstroke}{rgb}{1.000000,0.000000,0.000000}%
\pgfsetstrokecolor{currentstroke}%
\pgfsetdash{}{0pt}%
\pgfpathmoveto{\pgfqpoint{2.549799in}{1.925112in}}%
\pgfpathlineto{\pgfqpoint{2.445215in}{0.954506in}}%
\pgfusepath{stroke}%
\end{pgfscope}%
\begin{pgfscope}%
\pgfpathrectangle{\pgfqpoint{0.100000in}{0.212622in}}{\pgfqpoint{3.696000in}{3.696000in}}%
\pgfusepath{clip}%
\pgfsetrectcap%
\pgfsetroundjoin%
\pgfsetlinewidth{1.505625pt}%
\definecolor{currentstroke}{rgb}{1.000000,0.000000,0.000000}%
\pgfsetstrokecolor{currentstroke}%
\pgfsetdash{}{0pt}%
\pgfpathmoveto{\pgfqpoint{2.542731in}{1.918839in}}%
\pgfpathlineto{\pgfqpoint{2.445215in}{0.954506in}}%
\pgfusepath{stroke}%
\end{pgfscope}%
\begin{pgfscope}%
\pgfpathrectangle{\pgfqpoint{0.100000in}{0.212622in}}{\pgfqpoint{3.696000in}{3.696000in}}%
\pgfusepath{clip}%
\pgfsetrectcap%
\pgfsetroundjoin%
\pgfsetlinewidth{1.505625pt}%
\definecolor{currentstroke}{rgb}{1.000000,0.000000,0.000000}%
\pgfsetstrokecolor{currentstroke}%
\pgfsetdash{}{0pt}%
\pgfpathmoveto{\pgfqpoint{2.529836in}{1.913764in}}%
\pgfpathlineto{\pgfqpoint{2.427825in}{0.937361in}}%
\pgfusepath{stroke}%
\end{pgfscope}%
\begin{pgfscope}%
\pgfpathrectangle{\pgfqpoint{0.100000in}{0.212622in}}{\pgfqpoint{3.696000in}{3.696000in}}%
\pgfusepath{clip}%
\pgfsetrectcap%
\pgfsetroundjoin%
\pgfsetlinewidth{1.505625pt}%
\definecolor{currentstroke}{rgb}{1.000000,0.000000,0.000000}%
\pgfsetstrokecolor{currentstroke}%
\pgfsetdash{}{0pt}%
\pgfpathmoveto{\pgfqpoint{2.519174in}{1.903529in}}%
\pgfpathlineto{\pgfqpoint{2.419113in}{0.928772in}}%
\pgfusepath{stroke}%
\end{pgfscope}%
\begin{pgfscope}%
\pgfpathrectangle{\pgfqpoint{0.100000in}{0.212622in}}{\pgfqpoint{3.696000in}{3.696000in}}%
\pgfusepath{clip}%
\pgfsetrectcap%
\pgfsetroundjoin%
\pgfsetlinewidth{1.505625pt}%
\definecolor{currentstroke}{rgb}{1.000000,0.000000,0.000000}%
\pgfsetstrokecolor{currentstroke}%
\pgfsetdash{}{0pt}%
\pgfpathmoveto{\pgfqpoint{2.501084in}{1.894786in}}%
\pgfpathlineto{\pgfqpoint{2.410389in}{0.920170in}}%
\pgfusepath{stroke}%
\end{pgfscope}%
\begin{pgfscope}%
\pgfpathrectangle{\pgfqpoint{0.100000in}{0.212622in}}{\pgfqpoint{3.696000in}{3.696000in}}%
\pgfusepath{clip}%
\pgfsetrectcap%
\pgfsetroundjoin%
\pgfsetlinewidth{1.505625pt}%
\definecolor{currentstroke}{rgb}{1.000000,0.000000,0.000000}%
\pgfsetstrokecolor{currentstroke}%
\pgfsetdash{}{0pt}%
\pgfpathmoveto{\pgfqpoint{2.493448in}{1.890634in}}%
\pgfpathlineto{\pgfqpoint{2.401653in}{0.911557in}}%
\pgfusepath{stroke}%
\end{pgfscope}%
\begin{pgfscope}%
\pgfpathrectangle{\pgfqpoint{0.100000in}{0.212622in}}{\pgfqpoint{3.696000in}{3.696000in}}%
\pgfusepath{clip}%
\pgfsetrectcap%
\pgfsetroundjoin%
\pgfsetlinewidth{1.505625pt}%
\definecolor{currentstroke}{rgb}{1.000000,0.000000,0.000000}%
\pgfsetstrokecolor{currentstroke}%
\pgfsetdash{}{0pt}%
\pgfpathmoveto{\pgfqpoint{2.488975in}{1.888104in}}%
\pgfpathlineto{\pgfqpoint{2.392906in}{0.902933in}}%
\pgfusepath{stroke}%
\end{pgfscope}%
\begin{pgfscope}%
\pgfpathrectangle{\pgfqpoint{0.100000in}{0.212622in}}{\pgfqpoint{3.696000in}{3.696000in}}%
\pgfusepath{clip}%
\pgfsetrectcap%
\pgfsetroundjoin%
\pgfsetlinewidth{1.505625pt}%
\definecolor{currentstroke}{rgb}{1.000000,0.000000,0.000000}%
\pgfsetstrokecolor{currentstroke}%
\pgfsetdash{}{0pt}%
\pgfpathmoveto{\pgfqpoint{2.486606in}{1.887445in}}%
\pgfpathlineto{\pgfqpoint{2.392906in}{0.902933in}}%
\pgfusepath{stroke}%
\end{pgfscope}%
\begin{pgfscope}%
\pgfpathrectangle{\pgfqpoint{0.100000in}{0.212622in}}{\pgfqpoint{3.696000in}{3.696000in}}%
\pgfusepath{clip}%
\pgfsetrectcap%
\pgfsetroundjoin%
\pgfsetlinewidth{1.505625pt}%
\definecolor{currentstroke}{rgb}{1.000000,0.000000,0.000000}%
\pgfsetstrokecolor{currentstroke}%
\pgfsetdash{}{0pt}%
\pgfpathmoveto{\pgfqpoint{2.485305in}{1.886389in}}%
\pgfpathlineto{\pgfqpoint{2.392906in}{0.902933in}}%
\pgfusepath{stroke}%
\end{pgfscope}%
\begin{pgfscope}%
\pgfpathrectangle{\pgfqpoint{0.100000in}{0.212622in}}{\pgfqpoint{3.696000in}{3.696000in}}%
\pgfusepath{clip}%
\pgfsetrectcap%
\pgfsetroundjoin%
\pgfsetlinewidth{1.505625pt}%
\definecolor{currentstroke}{rgb}{1.000000,0.000000,0.000000}%
\pgfsetstrokecolor{currentstroke}%
\pgfsetdash{}{0pt}%
\pgfpathmoveto{\pgfqpoint{2.484536in}{1.886124in}}%
\pgfpathlineto{\pgfqpoint{2.392906in}{0.902933in}}%
\pgfusepath{stroke}%
\end{pgfscope}%
\begin{pgfscope}%
\pgfpathrectangle{\pgfqpoint{0.100000in}{0.212622in}}{\pgfqpoint{3.696000in}{3.696000in}}%
\pgfusepath{clip}%
\pgfsetrectcap%
\pgfsetroundjoin%
\pgfsetlinewidth{1.505625pt}%
\definecolor{currentstroke}{rgb}{1.000000,0.000000,0.000000}%
\pgfsetstrokecolor{currentstroke}%
\pgfsetdash{}{0pt}%
\pgfpathmoveto{\pgfqpoint{2.481419in}{1.883269in}}%
\pgfpathlineto{\pgfqpoint{2.384146in}{0.894296in}}%
\pgfusepath{stroke}%
\end{pgfscope}%
\begin{pgfscope}%
\pgfpathrectangle{\pgfqpoint{0.100000in}{0.212622in}}{\pgfqpoint{3.696000in}{3.696000in}}%
\pgfusepath{clip}%
\pgfsetrectcap%
\pgfsetroundjoin%
\pgfsetlinewidth{1.505625pt}%
\definecolor{currentstroke}{rgb}{1.000000,0.000000,0.000000}%
\pgfsetstrokecolor{currentstroke}%
\pgfsetdash{}{0pt}%
\pgfpathmoveto{\pgfqpoint{2.476110in}{1.881145in}}%
\pgfpathlineto{\pgfqpoint{2.384146in}{0.894296in}}%
\pgfusepath{stroke}%
\end{pgfscope}%
\begin{pgfscope}%
\pgfpathrectangle{\pgfqpoint{0.100000in}{0.212622in}}{\pgfqpoint{3.696000in}{3.696000in}}%
\pgfusepath{clip}%
\pgfsetrectcap%
\pgfsetroundjoin%
\pgfsetlinewidth{1.505625pt}%
\definecolor{currentstroke}{rgb}{1.000000,0.000000,0.000000}%
\pgfsetstrokecolor{currentstroke}%
\pgfsetdash{}{0pt}%
\pgfpathmoveto{\pgfqpoint{2.469091in}{1.876212in}}%
\pgfpathlineto{\pgfqpoint{2.375375in}{0.885648in}}%
\pgfusepath{stroke}%
\end{pgfscope}%
\begin{pgfscope}%
\pgfpathrectangle{\pgfqpoint{0.100000in}{0.212622in}}{\pgfqpoint{3.696000in}{3.696000in}}%
\pgfusepath{clip}%
\pgfsetrectcap%
\pgfsetroundjoin%
\pgfsetlinewidth{1.505625pt}%
\definecolor{currentstroke}{rgb}{1.000000,0.000000,0.000000}%
\pgfsetstrokecolor{currentstroke}%
\pgfsetdash{}{0pt}%
\pgfpathmoveto{\pgfqpoint{2.464762in}{1.874126in}}%
\pgfpathlineto{\pgfqpoint{2.375375in}{0.885648in}}%
\pgfusepath{stroke}%
\end{pgfscope}%
\begin{pgfscope}%
\pgfpathrectangle{\pgfqpoint{0.100000in}{0.212622in}}{\pgfqpoint{3.696000in}{3.696000in}}%
\pgfusepath{clip}%
\pgfsetrectcap%
\pgfsetroundjoin%
\pgfsetlinewidth{1.505625pt}%
\definecolor{currentstroke}{rgb}{1.000000,0.000000,0.000000}%
\pgfsetstrokecolor{currentstroke}%
\pgfsetdash{}{0pt}%
\pgfpathmoveto{\pgfqpoint{2.462289in}{1.873586in}}%
\pgfpathlineto{\pgfqpoint{2.375375in}{0.885648in}}%
\pgfusepath{stroke}%
\end{pgfscope}%
\begin{pgfscope}%
\pgfpathrectangle{\pgfqpoint{0.100000in}{0.212622in}}{\pgfqpoint{3.696000in}{3.696000in}}%
\pgfusepath{clip}%
\pgfsetrectcap%
\pgfsetroundjoin%
\pgfsetlinewidth{1.505625pt}%
\definecolor{currentstroke}{rgb}{1.000000,0.000000,0.000000}%
\pgfsetstrokecolor{currentstroke}%
\pgfsetdash{}{0pt}%
\pgfpathmoveto{\pgfqpoint{2.461125in}{1.872753in}}%
\pgfpathlineto{\pgfqpoint{2.375375in}{0.885648in}}%
\pgfusepath{stroke}%
\end{pgfscope}%
\begin{pgfscope}%
\pgfpathrectangle{\pgfqpoint{0.100000in}{0.212622in}}{\pgfqpoint{3.696000in}{3.696000in}}%
\pgfusepath{clip}%
\pgfsetrectcap%
\pgfsetroundjoin%
\pgfsetlinewidth{1.505625pt}%
\definecolor{currentstroke}{rgb}{1.000000,0.000000,0.000000}%
\pgfsetstrokecolor{currentstroke}%
\pgfsetdash{}{0pt}%
\pgfpathmoveto{\pgfqpoint{2.460421in}{1.872563in}}%
\pgfpathlineto{\pgfqpoint{2.366591in}{0.876989in}}%
\pgfusepath{stroke}%
\end{pgfscope}%
\begin{pgfscope}%
\pgfpathrectangle{\pgfqpoint{0.100000in}{0.212622in}}{\pgfqpoint{3.696000in}{3.696000in}}%
\pgfusepath{clip}%
\pgfsetrectcap%
\pgfsetroundjoin%
\pgfsetlinewidth{1.505625pt}%
\definecolor{currentstroke}{rgb}{1.000000,0.000000,0.000000}%
\pgfsetstrokecolor{currentstroke}%
\pgfsetdash{}{0pt}%
\pgfpathmoveto{\pgfqpoint{2.460096in}{1.872316in}}%
\pgfpathlineto{\pgfqpoint{2.366591in}{0.876989in}}%
\pgfusepath{stroke}%
\end{pgfscope}%
\begin{pgfscope}%
\pgfpathrectangle{\pgfqpoint{0.100000in}{0.212622in}}{\pgfqpoint{3.696000in}{3.696000in}}%
\pgfusepath{clip}%
\pgfsetrectcap%
\pgfsetroundjoin%
\pgfsetlinewidth{1.505625pt}%
\definecolor{currentstroke}{rgb}{1.000000,0.000000,0.000000}%
\pgfsetstrokecolor{currentstroke}%
\pgfsetdash{}{0pt}%
\pgfpathmoveto{\pgfqpoint{2.459903in}{1.872236in}}%
\pgfpathlineto{\pgfqpoint{2.366591in}{0.876989in}}%
\pgfusepath{stroke}%
\end{pgfscope}%
\begin{pgfscope}%
\pgfpathrectangle{\pgfqpoint{0.100000in}{0.212622in}}{\pgfqpoint{3.696000in}{3.696000in}}%
\pgfusepath{clip}%
\pgfsetrectcap%
\pgfsetroundjoin%
\pgfsetlinewidth{1.505625pt}%
\definecolor{currentstroke}{rgb}{1.000000,0.000000,0.000000}%
\pgfsetstrokecolor{currentstroke}%
\pgfsetdash{}{0pt}%
\pgfpathmoveto{\pgfqpoint{2.457980in}{1.870395in}}%
\pgfpathlineto{\pgfqpoint{2.366591in}{0.876989in}}%
\pgfusepath{stroke}%
\end{pgfscope}%
\begin{pgfscope}%
\pgfpathrectangle{\pgfqpoint{0.100000in}{0.212622in}}{\pgfqpoint{3.696000in}{3.696000in}}%
\pgfusepath{clip}%
\pgfsetrectcap%
\pgfsetroundjoin%
\pgfsetlinewidth{1.505625pt}%
\definecolor{currentstroke}{rgb}{1.000000,0.000000,0.000000}%
\pgfsetstrokecolor{currentstroke}%
\pgfsetdash{}{0pt}%
\pgfpathmoveto{\pgfqpoint{2.454047in}{1.868536in}}%
\pgfpathlineto{\pgfqpoint{2.366591in}{0.876989in}}%
\pgfusepath{stroke}%
\end{pgfscope}%
\begin{pgfscope}%
\pgfpathrectangle{\pgfqpoint{0.100000in}{0.212622in}}{\pgfqpoint{3.696000in}{3.696000in}}%
\pgfusepath{clip}%
\pgfsetrectcap%
\pgfsetroundjoin%
\pgfsetlinewidth{1.505625pt}%
\definecolor{currentstroke}{rgb}{1.000000,0.000000,0.000000}%
\pgfsetstrokecolor{currentstroke}%
\pgfsetdash{}{0pt}%
\pgfpathmoveto{\pgfqpoint{2.448119in}{1.865715in}}%
\pgfpathlineto{\pgfqpoint{2.357796in}{0.868317in}}%
\pgfusepath{stroke}%
\end{pgfscope}%
\begin{pgfscope}%
\pgfpathrectangle{\pgfqpoint{0.100000in}{0.212622in}}{\pgfqpoint{3.696000in}{3.696000in}}%
\pgfusepath{clip}%
\pgfsetrectcap%
\pgfsetroundjoin%
\pgfsetlinewidth{1.505625pt}%
\definecolor{currentstroke}{rgb}{1.000000,0.000000,0.000000}%
\pgfsetstrokecolor{currentstroke}%
\pgfsetdash{}{0pt}%
\pgfpathmoveto{\pgfqpoint{2.439689in}{1.861640in}}%
\pgfpathlineto{\pgfqpoint{2.348989in}{0.859634in}}%
\pgfusepath{stroke}%
\end{pgfscope}%
\begin{pgfscope}%
\pgfpathrectangle{\pgfqpoint{0.100000in}{0.212622in}}{\pgfqpoint{3.696000in}{3.696000in}}%
\pgfusepath{clip}%
\pgfsetrectcap%
\pgfsetroundjoin%
\pgfsetlinewidth{1.505625pt}%
\definecolor{currentstroke}{rgb}{1.000000,0.000000,0.000000}%
\pgfsetstrokecolor{currentstroke}%
\pgfsetdash{}{0pt}%
\pgfpathmoveto{\pgfqpoint{2.435957in}{1.859321in}}%
\pgfpathlineto{\pgfqpoint{2.348989in}{0.859634in}}%
\pgfusepath{stroke}%
\end{pgfscope}%
\begin{pgfscope}%
\pgfpathrectangle{\pgfqpoint{0.100000in}{0.212622in}}{\pgfqpoint{3.696000in}{3.696000in}}%
\pgfusepath{clip}%
\pgfsetrectcap%
\pgfsetroundjoin%
\pgfsetlinewidth{1.505625pt}%
\definecolor{currentstroke}{rgb}{1.000000,0.000000,0.000000}%
\pgfsetstrokecolor{currentstroke}%
\pgfsetdash{}{0pt}%
\pgfpathmoveto{\pgfqpoint{2.433216in}{1.858149in}}%
\pgfpathlineto{\pgfqpoint{2.348989in}{0.859634in}}%
\pgfusepath{stroke}%
\end{pgfscope}%
\begin{pgfscope}%
\pgfpathrectangle{\pgfqpoint{0.100000in}{0.212622in}}{\pgfqpoint{3.696000in}{3.696000in}}%
\pgfusepath{clip}%
\pgfsetrectcap%
\pgfsetroundjoin%
\pgfsetlinewidth{1.505625pt}%
\definecolor{currentstroke}{rgb}{1.000000,0.000000,0.000000}%
\pgfsetstrokecolor{currentstroke}%
\pgfsetdash{}{0pt}%
\pgfpathmoveto{\pgfqpoint{2.432073in}{1.857922in}}%
\pgfpathlineto{\pgfqpoint{2.348989in}{0.859634in}}%
\pgfusepath{stroke}%
\end{pgfscope}%
\begin{pgfscope}%
\pgfpathrectangle{\pgfqpoint{0.100000in}{0.212622in}}{\pgfqpoint{3.696000in}{3.696000in}}%
\pgfusepath{clip}%
\pgfsetrectcap%
\pgfsetroundjoin%
\pgfsetlinewidth{1.505625pt}%
\definecolor{currentstroke}{rgb}{1.000000,0.000000,0.000000}%
\pgfsetstrokecolor{currentstroke}%
\pgfsetdash{}{0pt}%
\pgfpathmoveto{\pgfqpoint{2.431457in}{1.857393in}}%
\pgfpathlineto{\pgfqpoint{2.348989in}{0.859634in}}%
\pgfusepath{stroke}%
\end{pgfscope}%
\begin{pgfscope}%
\pgfpathrectangle{\pgfqpoint{0.100000in}{0.212622in}}{\pgfqpoint{3.696000in}{3.696000in}}%
\pgfusepath{clip}%
\pgfsetrectcap%
\pgfsetroundjoin%
\pgfsetlinewidth{1.505625pt}%
\definecolor{currentstroke}{rgb}{1.000000,0.000000,0.000000}%
\pgfsetstrokecolor{currentstroke}%
\pgfsetdash{}{0pt}%
\pgfpathmoveto{\pgfqpoint{2.431081in}{1.857278in}}%
\pgfpathlineto{\pgfqpoint{2.348989in}{0.859634in}}%
\pgfusepath{stroke}%
\end{pgfscope}%
\begin{pgfscope}%
\pgfpathrectangle{\pgfqpoint{0.100000in}{0.212622in}}{\pgfqpoint{3.696000in}{3.696000in}}%
\pgfusepath{clip}%
\pgfsetrectcap%
\pgfsetroundjoin%
\pgfsetlinewidth{1.505625pt}%
\definecolor{currentstroke}{rgb}{1.000000,0.000000,0.000000}%
\pgfsetstrokecolor{currentstroke}%
\pgfsetdash{}{0pt}%
\pgfpathmoveto{\pgfqpoint{2.428622in}{1.855066in}}%
\pgfpathlineto{\pgfqpoint{2.340170in}{0.850939in}}%
\pgfusepath{stroke}%
\end{pgfscope}%
\begin{pgfscope}%
\pgfpathrectangle{\pgfqpoint{0.100000in}{0.212622in}}{\pgfqpoint{3.696000in}{3.696000in}}%
\pgfusepath{clip}%
\pgfsetrectcap%
\pgfsetroundjoin%
\pgfsetlinewidth{1.505625pt}%
\definecolor{currentstroke}{rgb}{1.000000,0.000000,0.000000}%
\pgfsetstrokecolor{currentstroke}%
\pgfsetdash{}{0pt}%
\pgfpathmoveto{\pgfqpoint{2.426848in}{1.854279in}}%
\pgfpathlineto{\pgfqpoint{2.340170in}{0.850939in}}%
\pgfusepath{stroke}%
\end{pgfscope}%
\begin{pgfscope}%
\pgfpathrectangle{\pgfqpoint{0.100000in}{0.212622in}}{\pgfqpoint{3.696000in}{3.696000in}}%
\pgfusepath{clip}%
\pgfsetrectcap%
\pgfsetroundjoin%
\pgfsetlinewidth{1.505625pt}%
\definecolor{currentstroke}{rgb}{1.000000,0.000000,0.000000}%
\pgfsetstrokecolor{currentstroke}%
\pgfsetdash{}{0pt}%
\pgfpathmoveto{\pgfqpoint{2.423420in}{1.852144in}}%
\pgfpathlineto{\pgfqpoint{2.340170in}{0.850939in}}%
\pgfusepath{stroke}%
\end{pgfscope}%
\begin{pgfscope}%
\pgfpathrectangle{\pgfqpoint{0.100000in}{0.212622in}}{\pgfqpoint{3.696000in}{3.696000in}}%
\pgfusepath{clip}%
\pgfsetrectcap%
\pgfsetroundjoin%
\pgfsetlinewidth{1.505625pt}%
\definecolor{currentstroke}{rgb}{1.000000,0.000000,0.000000}%
\pgfsetstrokecolor{currentstroke}%
\pgfsetdash{}{0pt}%
\pgfpathmoveto{\pgfqpoint{2.420954in}{1.851021in}}%
\pgfpathlineto{\pgfqpoint{2.331339in}{0.842232in}}%
\pgfusepath{stroke}%
\end{pgfscope}%
\begin{pgfscope}%
\pgfpathrectangle{\pgfqpoint{0.100000in}{0.212622in}}{\pgfqpoint{3.696000in}{3.696000in}}%
\pgfusepath{clip}%
\pgfsetrectcap%
\pgfsetroundjoin%
\pgfsetlinewidth{1.505625pt}%
\definecolor{currentstroke}{rgb}{1.000000,0.000000,0.000000}%
\pgfsetstrokecolor{currentstroke}%
\pgfsetdash{}{0pt}%
\pgfpathmoveto{\pgfqpoint{2.419889in}{1.850879in}}%
\pgfpathlineto{\pgfqpoint{2.331339in}{0.842232in}}%
\pgfusepath{stroke}%
\end{pgfscope}%
\begin{pgfscope}%
\pgfpathrectangle{\pgfqpoint{0.100000in}{0.212622in}}{\pgfqpoint{3.696000in}{3.696000in}}%
\pgfusepath{clip}%
\pgfsetrectcap%
\pgfsetroundjoin%
\pgfsetlinewidth{1.505625pt}%
\definecolor{currentstroke}{rgb}{1.000000,0.000000,0.000000}%
\pgfsetstrokecolor{currentstroke}%
\pgfsetdash{}{0pt}%
\pgfpathmoveto{\pgfqpoint{2.419255in}{1.850547in}}%
\pgfpathlineto{\pgfqpoint{2.331339in}{0.842232in}}%
\pgfusepath{stroke}%
\end{pgfscope}%
\begin{pgfscope}%
\pgfpathrectangle{\pgfqpoint{0.100000in}{0.212622in}}{\pgfqpoint{3.696000in}{3.696000in}}%
\pgfusepath{clip}%
\pgfsetrectcap%
\pgfsetroundjoin%
\pgfsetlinewidth{1.505625pt}%
\definecolor{currentstroke}{rgb}{1.000000,0.000000,0.000000}%
\pgfsetstrokecolor{currentstroke}%
\pgfsetdash{}{0pt}%
\pgfpathmoveto{\pgfqpoint{2.418912in}{1.850476in}}%
\pgfpathlineto{\pgfqpoint{2.331339in}{0.842232in}}%
\pgfusepath{stroke}%
\end{pgfscope}%
\begin{pgfscope}%
\pgfpathrectangle{\pgfqpoint{0.100000in}{0.212622in}}{\pgfqpoint{3.696000in}{3.696000in}}%
\pgfusepath{clip}%
\pgfsetrectcap%
\pgfsetroundjoin%
\pgfsetlinewidth{1.505625pt}%
\definecolor{currentstroke}{rgb}{1.000000,0.000000,0.000000}%
\pgfsetstrokecolor{currentstroke}%
\pgfsetdash{}{0pt}%
\pgfpathmoveto{\pgfqpoint{2.418749in}{1.850313in}}%
\pgfpathlineto{\pgfqpoint{2.331339in}{0.842232in}}%
\pgfusepath{stroke}%
\end{pgfscope}%
\begin{pgfscope}%
\pgfpathrectangle{\pgfqpoint{0.100000in}{0.212622in}}{\pgfqpoint{3.696000in}{3.696000in}}%
\pgfusepath{clip}%
\pgfsetrectcap%
\pgfsetroundjoin%
\pgfsetlinewidth{1.505625pt}%
\definecolor{currentstroke}{rgb}{1.000000,0.000000,0.000000}%
\pgfsetstrokecolor{currentstroke}%
\pgfsetdash{}{0pt}%
\pgfpathmoveto{\pgfqpoint{2.418641in}{1.850267in}}%
\pgfpathlineto{\pgfqpoint{2.331339in}{0.842232in}}%
\pgfusepath{stroke}%
\end{pgfscope}%
\begin{pgfscope}%
\pgfpathrectangle{\pgfqpoint{0.100000in}{0.212622in}}{\pgfqpoint{3.696000in}{3.696000in}}%
\pgfusepath{clip}%
\pgfsetrectcap%
\pgfsetroundjoin%
\pgfsetlinewidth{1.505625pt}%
\definecolor{currentstroke}{rgb}{1.000000,0.000000,0.000000}%
\pgfsetstrokecolor{currentstroke}%
\pgfsetdash{}{0pt}%
\pgfpathmoveto{\pgfqpoint{2.417329in}{1.849342in}}%
\pgfpathlineto{\pgfqpoint{2.331339in}{0.842232in}}%
\pgfusepath{stroke}%
\end{pgfscope}%
\begin{pgfscope}%
\pgfpathrectangle{\pgfqpoint{0.100000in}{0.212622in}}{\pgfqpoint{3.696000in}{3.696000in}}%
\pgfusepath{clip}%
\pgfsetrectcap%
\pgfsetroundjoin%
\pgfsetlinewidth{1.505625pt}%
\definecolor{currentstroke}{rgb}{1.000000,0.000000,0.000000}%
\pgfsetstrokecolor{currentstroke}%
\pgfsetdash{}{0pt}%
\pgfpathmoveto{\pgfqpoint{2.416376in}{1.848897in}}%
\pgfpathlineto{\pgfqpoint{2.331339in}{0.842232in}}%
\pgfusepath{stroke}%
\end{pgfscope}%
\begin{pgfscope}%
\pgfpathrectangle{\pgfqpoint{0.100000in}{0.212622in}}{\pgfqpoint{3.696000in}{3.696000in}}%
\pgfusepath{clip}%
\pgfsetrectcap%
\pgfsetroundjoin%
\pgfsetlinewidth{1.505625pt}%
\definecolor{currentstroke}{rgb}{1.000000,0.000000,0.000000}%
\pgfsetstrokecolor{currentstroke}%
\pgfsetdash{}{0pt}%
\pgfpathmoveto{\pgfqpoint{2.416011in}{1.848692in}}%
\pgfpathlineto{\pgfqpoint{2.331339in}{0.842232in}}%
\pgfusepath{stroke}%
\end{pgfscope}%
\begin{pgfscope}%
\pgfpathrectangle{\pgfqpoint{0.100000in}{0.212622in}}{\pgfqpoint{3.696000in}{3.696000in}}%
\pgfusepath{clip}%
\pgfsetrectcap%
\pgfsetroundjoin%
\pgfsetlinewidth{1.505625pt}%
\definecolor{currentstroke}{rgb}{1.000000,0.000000,0.000000}%
\pgfsetstrokecolor{currentstroke}%
\pgfsetdash{}{0pt}%
\pgfpathmoveto{\pgfqpoint{2.415729in}{1.848566in}}%
\pgfpathlineto{\pgfqpoint{2.331339in}{0.842232in}}%
\pgfusepath{stroke}%
\end{pgfscope}%
\begin{pgfscope}%
\pgfpathrectangle{\pgfqpoint{0.100000in}{0.212622in}}{\pgfqpoint{3.696000in}{3.696000in}}%
\pgfusepath{clip}%
\pgfsetrectcap%
\pgfsetroundjoin%
\pgfsetlinewidth{1.505625pt}%
\definecolor{currentstroke}{rgb}{1.000000,0.000000,0.000000}%
\pgfsetstrokecolor{currentstroke}%
\pgfsetdash{}{0pt}%
\pgfpathmoveto{\pgfqpoint{2.415609in}{1.848542in}}%
\pgfpathlineto{\pgfqpoint{2.331339in}{0.842232in}}%
\pgfusepath{stroke}%
\end{pgfscope}%
\begin{pgfscope}%
\pgfpathrectangle{\pgfqpoint{0.100000in}{0.212622in}}{\pgfqpoint{3.696000in}{3.696000in}}%
\pgfusepath{clip}%
\pgfsetrectcap%
\pgfsetroundjoin%
\pgfsetlinewidth{1.505625pt}%
\definecolor{currentstroke}{rgb}{1.000000,0.000000,0.000000}%
\pgfsetstrokecolor{currentstroke}%
\pgfsetdash{}{0pt}%
\pgfpathmoveto{\pgfqpoint{2.415542in}{1.848484in}}%
\pgfpathlineto{\pgfqpoint{2.331339in}{0.842232in}}%
\pgfusepath{stroke}%
\end{pgfscope}%
\begin{pgfscope}%
\pgfpathrectangle{\pgfqpoint{0.100000in}{0.212622in}}{\pgfqpoint{3.696000in}{3.696000in}}%
\pgfusepath{clip}%
\pgfsetrectcap%
\pgfsetroundjoin%
\pgfsetlinewidth{1.505625pt}%
\definecolor{currentstroke}{rgb}{1.000000,0.000000,0.000000}%
\pgfsetstrokecolor{currentstroke}%
\pgfsetdash{}{0pt}%
\pgfpathmoveto{\pgfqpoint{2.413905in}{1.847793in}}%
\pgfpathlineto{\pgfqpoint{2.331339in}{0.842232in}}%
\pgfusepath{stroke}%
\end{pgfscope}%
\begin{pgfscope}%
\pgfpathrectangle{\pgfqpoint{0.100000in}{0.212622in}}{\pgfqpoint{3.696000in}{3.696000in}}%
\pgfusepath{clip}%
\pgfsetrectcap%
\pgfsetroundjoin%
\pgfsetlinewidth{1.505625pt}%
\definecolor{currentstroke}{rgb}{1.000000,0.000000,0.000000}%
\pgfsetstrokecolor{currentstroke}%
\pgfsetdash{}{0pt}%
\pgfpathmoveto{\pgfqpoint{2.410481in}{1.845465in}}%
\pgfpathlineto{\pgfqpoint{2.331339in}{0.842232in}}%
\pgfusepath{stroke}%
\end{pgfscope}%
\begin{pgfscope}%
\pgfpathrectangle{\pgfqpoint{0.100000in}{0.212622in}}{\pgfqpoint{3.696000in}{3.696000in}}%
\pgfusepath{clip}%
\pgfsetrectcap%
\pgfsetroundjoin%
\pgfsetlinewidth{1.505625pt}%
\definecolor{currentstroke}{rgb}{1.000000,0.000000,0.000000}%
\pgfsetstrokecolor{currentstroke}%
\pgfsetdash{}{0pt}%
\pgfpathmoveto{\pgfqpoint{2.403343in}{1.842232in}}%
\pgfpathlineto{\pgfqpoint{2.331339in}{0.842232in}}%
\pgfusepath{stroke}%
\end{pgfscope}%
\begin{pgfscope}%
\pgfpathrectangle{\pgfqpoint{0.100000in}{0.212622in}}{\pgfqpoint{3.696000in}{3.696000in}}%
\pgfusepath{clip}%
\pgfsetrectcap%
\pgfsetroundjoin%
\pgfsetlinewidth{1.505625pt}%
\definecolor{currentstroke}{rgb}{1.000000,0.000000,0.000000}%
\pgfsetstrokecolor{currentstroke}%
\pgfsetdash{}{0pt}%
\pgfpathmoveto{\pgfqpoint{2.400035in}{1.840997in}}%
\pgfpathlineto{\pgfqpoint{2.331339in}{0.842232in}}%
\pgfusepath{stroke}%
\end{pgfscope}%
\begin{pgfscope}%
\pgfpathrectangle{\pgfqpoint{0.100000in}{0.212622in}}{\pgfqpoint{3.696000in}{3.696000in}}%
\pgfusepath{clip}%
\pgfsetrectcap%
\pgfsetroundjoin%
\pgfsetlinewidth{1.505625pt}%
\definecolor{currentstroke}{rgb}{1.000000,0.000000,0.000000}%
\pgfsetstrokecolor{currentstroke}%
\pgfsetdash{}{0pt}%
\pgfpathmoveto{\pgfqpoint{2.397947in}{1.839550in}}%
\pgfpathlineto{\pgfqpoint{2.331339in}{0.842232in}}%
\pgfusepath{stroke}%
\end{pgfscope}%
\begin{pgfscope}%
\pgfpathrectangle{\pgfqpoint{0.100000in}{0.212622in}}{\pgfqpoint{3.696000in}{3.696000in}}%
\pgfusepath{clip}%
\pgfsetrectcap%
\pgfsetroundjoin%
\pgfsetlinewidth{1.505625pt}%
\definecolor{currentstroke}{rgb}{1.000000,0.000000,0.000000}%
\pgfsetstrokecolor{currentstroke}%
\pgfsetdash{}{0pt}%
\pgfpathmoveto{\pgfqpoint{2.397132in}{1.839246in}}%
\pgfpathlineto{\pgfqpoint{2.331339in}{0.842232in}}%
\pgfusepath{stroke}%
\end{pgfscope}%
\begin{pgfscope}%
\pgfpathrectangle{\pgfqpoint{0.100000in}{0.212622in}}{\pgfqpoint{3.696000in}{3.696000in}}%
\pgfusepath{clip}%
\pgfsetrectcap%
\pgfsetroundjoin%
\pgfsetlinewidth{1.505625pt}%
\definecolor{currentstroke}{rgb}{1.000000,0.000000,0.000000}%
\pgfsetstrokecolor{currentstroke}%
\pgfsetdash{}{0pt}%
\pgfpathmoveto{\pgfqpoint{2.396539in}{1.838927in}}%
\pgfpathlineto{\pgfqpoint{2.331339in}{0.842232in}}%
\pgfusepath{stroke}%
\end{pgfscope}%
\begin{pgfscope}%
\pgfpathrectangle{\pgfqpoint{0.100000in}{0.212622in}}{\pgfqpoint{3.696000in}{3.696000in}}%
\pgfusepath{clip}%
\pgfsetrectcap%
\pgfsetroundjoin%
\pgfsetlinewidth{1.505625pt}%
\definecolor{currentstroke}{rgb}{1.000000,0.000000,0.000000}%
\pgfsetstrokecolor{currentstroke}%
\pgfsetdash{}{0pt}%
\pgfpathmoveto{\pgfqpoint{2.396201in}{1.838813in}}%
\pgfpathlineto{\pgfqpoint{2.331339in}{0.842232in}}%
\pgfusepath{stroke}%
\end{pgfscope}%
\begin{pgfscope}%
\pgfpathrectangle{\pgfqpoint{0.100000in}{0.212622in}}{\pgfqpoint{3.696000in}{3.696000in}}%
\pgfusepath{clip}%
\pgfsetrectcap%
\pgfsetroundjoin%
\pgfsetlinewidth{1.505625pt}%
\definecolor{currentstroke}{rgb}{1.000000,0.000000,0.000000}%
\pgfsetstrokecolor{currentstroke}%
\pgfsetdash{}{0pt}%
\pgfpathmoveto{\pgfqpoint{2.396049in}{1.838697in}}%
\pgfpathlineto{\pgfqpoint{2.331339in}{0.842232in}}%
\pgfusepath{stroke}%
\end{pgfscope}%
\begin{pgfscope}%
\pgfpathrectangle{\pgfqpoint{0.100000in}{0.212622in}}{\pgfqpoint{3.696000in}{3.696000in}}%
\pgfusepath{clip}%
\pgfsetrectcap%
\pgfsetroundjoin%
\pgfsetlinewidth{1.505625pt}%
\definecolor{currentstroke}{rgb}{1.000000,0.000000,0.000000}%
\pgfsetstrokecolor{currentstroke}%
\pgfsetdash{}{0pt}%
\pgfpathmoveto{\pgfqpoint{2.395944in}{1.838668in}}%
\pgfpathlineto{\pgfqpoint{2.331339in}{0.842232in}}%
\pgfusepath{stroke}%
\end{pgfscope}%
\begin{pgfscope}%
\pgfpathrectangle{\pgfqpoint{0.100000in}{0.212622in}}{\pgfqpoint{3.696000in}{3.696000in}}%
\pgfusepath{clip}%
\pgfsetrectcap%
\pgfsetroundjoin%
\pgfsetlinewidth{1.505625pt}%
\definecolor{currentstroke}{rgb}{1.000000,0.000000,0.000000}%
\pgfsetstrokecolor{currentstroke}%
\pgfsetdash{}{0pt}%
\pgfpathmoveto{\pgfqpoint{2.395893in}{1.838643in}}%
\pgfpathlineto{\pgfqpoint{2.331339in}{0.842232in}}%
\pgfusepath{stroke}%
\end{pgfscope}%
\begin{pgfscope}%
\pgfpathrectangle{\pgfqpoint{0.100000in}{0.212622in}}{\pgfqpoint{3.696000in}{3.696000in}}%
\pgfusepath{clip}%
\pgfsetrectcap%
\pgfsetroundjoin%
\pgfsetlinewidth{1.505625pt}%
\definecolor{currentstroke}{rgb}{1.000000,0.000000,0.000000}%
\pgfsetstrokecolor{currentstroke}%
\pgfsetdash{}{0pt}%
\pgfpathmoveto{\pgfqpoint{2.395864in}{1.838631in}}%
\pgfpathlineto{\pgfqpoint{2.331339in}{0.842232in}}%
\pgfusepath{stroke}%
\end{pgfscope}%
\begin{pgfscope}%
\pgfpathrectangle{\pgfqpoint{0.100000in}{0.212622in}}{\pgfqpoint{3.696000in}{3.696000in}}%
\pgfusepath{clip}%
\pgfsetrectcap%
\pgfsetroundjoin%
\pgfsetlinewidth{1.505625pt}%
\definecolor{currentstroke}{rgb}{1.000000,0.000000,0.000000}%
\pgfsetstrokecolor{currentstroke}%
\pgfsetdash{}{0pt}%
\pgfpathmoveto{\pgfqpoint{2.395847in}{1.838625in}}%
\pgfpathlineto{\pgfqpoint{2.331339in}{0.842232in}}%
\pgfusepath{stroke}%
\end{pgfscope}%
\begin{pgfscope}%
\pgfpathrectangle{\pgfqpoint{0.100000in}{0.212622in}}{\pgfqpoint{3.696000in}{3.696000in}}%
\pgfusepath{clip}%
\pgfsetrectcap%
\pgfsetroundjoin%
\pgfsetlinewidth{1.505625pt}%
\definecolor{currentstroke}{rgb}{1.000000,0.000000,0.000000}%
\pgfsetstrokecolor{currentstroke}%
\pgfsetdash{}{0pt}%
\pgfpathmoveto{\pgfqpoint{2.395837in}{1.838622in}}%
\pgfpathlineto{\pgfqpoint{2.331339in}{0.842232in}}%
\pgfusepath{stroke}%
\end{pgfscope}%
\begin{pgfscope}%
\pgfpathrectangle{\pgfqpoint{0.100000in}{0.212622in}}{\pgfqpoint{3.696000in}{3.696000in}}%
\pgfusepath{clip}%
\pgfsetrectcap%
\pgfsetroundjoin%
\pgfsetlinewidth{1.505625pt}%
\definecolor{currentstroke}{rgb}{1.000000,0.000000,0.000000}%
\pgfsetstrokecolor{currentstroke}%
\pgfsetdash{}{0pt}%
\pgfpathmoveto{\pgfqpoint{2.395831in}{1.838622in}}%
\pgfpathlineto{\pgfqpoint{2.331339in}{0.842232in}}%
\pgfusepath{stroke}%
\end{pgfscope}%
\begin{pgfscope}%
\pgfpathrectangle{\pgfqpoint{0.100000in}{0.212622in}}{\pgfqpoint{3.696000in}{3.696000in}}%
\pgfusepath{clip}%
\pgfsetrectcap%
\pgfsetroundjoin%
\pgfsetlinewidth{1.505625pt}%
\definecolor{currentstroke}{rgb}{1.000000,0.000000,0.000000}%
\pgfsetstrokecolor{currentstroke}%
\pgfsetdash{}{0pt}%
\pgfpathmoveto{\pgfqpoint{2.393989in}{1.838141in}}%
\pgfpathlineto{\pgfqpoint{2.331339in}{0.842232in}}%
\pgfusepath{stroke}%
\end{pgfscope}%
\begin{pgfscope}%
\pgfpathrectangle{\pgfqpoint{0.100000in}{0.212622in}}{\pgfqpoint{3.696000in}{3.696000in}}%
\pgfusepath{clip}%
\pgfsetrectcap%
\pgfsetroundjoin%
\pgfsetlinewidth{1.505625pt}%
\definecolor{currentstroke}{rgb}{1.000000,0.000000,0.000000}%
\pgfsetstrokecolor{currentstroke}%
\pgfsetdash{}{0pt}%
\pgfpathmoveto{\pgfqpoint{2.389510in}{1.837929in}}%
\pgfpathlineto{\pgfqpoint{2.331339in}{0.842232in}}%
\pgfusepath{stroke}%
\end{pgfscope}%
\begin{pgfscope}%
\pgfpathrectangle{\pgfqpoint{0.100000in}{0.212622in}}{\pgfqpoint{3.696000in}{3.696000in}}%
\pgfusepath{clip}%
\pgfsetrectcap%
\pgfsetroundjoin%
\pgfsetlinewidth{1.505625pt}%
\definecolor{currentstroke}{rgb}{1.000000,0.000000,0.000000}%
\pgfsetstrokecolor{currentstroke}%
\pgfsetdash{}{0pt}%
\pgfpathmoveto{\pgfqpoint{2.381163in}{1.836950in}}%
\pgfpathlineto{\pgfqpoint{2.331339in}{0.842232in}}%
\pgfusepath{stroke}%
\end{pgfscope}%
\begin{pgfscope}%
\pgfpathrectangle{\pgfqpoint{0.100000in}{0.212622in}}{\pgfqpoint{3.696000in}{3.696000in}}%
\pgfusepath{clip}%
\pgfsetrectcap%
\pgfsetroundjoin%
\pgfsetlinewidth{1.505625pt}%
\definecolor{currentstroke}{rgb}{1.000000,0.000000,0.000000}%
\pgfsetstrokecolor{currentstroke}%
\pgfsetdash{}{0pt}%
\pgfpathmoveto{\pgfqpoint{2.376496in}{1.837417in}}%
\pgfpathlineto{\pgfqpoint{2.331339in}{0.842232in}}%
\pgfusepath{stroke}%
\end{pgfscope}%
\begin{pgfscope}%
\pgfpathrectangle{\pgfqpoint{0.100000in}{0.212622in}}{\pgfqpoint{3.696000in}{3.696000in}}%
\pgfusepath{clip}%
\pgfsetrectcap%
\pgfsetroundjoin%
\pgfsetlinewidth{1.505625pt}%
\definecolor{currentstroke}{rgb}{1.000000,0.000000,0.000000}%
\pgfsetstrokecolor{currentstroke}%
\pgfsetdash{}{0pt}%
\pgfpathmoveto{\pgfqpoint{2.366282in}{1.839525in}}%
\pgfpathlineto{\pgfqpoint{2.331339in}{0.842232in}}%
\pgfusepath{stroke}%
\end{pgfscope}%
\begin{pgfscope}%
\pgfpathrectangle{\pgfqpoint{0.100000in}{0.212622in}}{\pgfqpoint{3.696000in}{3.696000in}}%
\pgfusepath{clip}%
\pgfsetrectcap%
\pgfsetroundjoin%
\pgfsetlinewidth{1.505625pt}%
\definecolor{currentstroke}{rgb}{1.000000,0.000000,0.000000}%
\pgfsetstrokecolor{currentstroke}%
\pgfsetdash{}{0pt}%
\pgfpathmoveto{\pgfqpoint{2.350975in}{1.845768in}}%
\pgfpathlineto{\pgfqpoint{2.331339in}{0.842232in}}%
\pgfusepath{stroke}%
\end{pgfscope}%
\begin{pgfscope}%
\pgfpathrectangle{\pgfqpoint{0.100000in}{0.212622in}}{\pgfqpoint{3.696000in}{3.696000in}}%
\pgfusepath{clip}%
\pgfsetrectcap%
\pgfsetroundjoin%
\pgfsetlinewidth{1.505625pt}%
\definecolor{currentstroke}{rgb}{1.000000,0.000000,0.000000}%
\pgfsetstrokecolor{currentstroke}%
\pgfsetdash{}{0pt}%
\pgfpathmoveto{\pgfqpoint{2.332392in}{1.860789in}}%
\pgfpathlineto{\pgfqpoint{2.331339in}{0.842232in}}%
\pgfusepath{stroke}%
\end{pgfscope}%
\begin{pgfscope}%
\pgfpathrectangle{\pgfqpoint{0.100000in}{0.212622in}}{\pgfqpoint{3.696000in}{3.696000in}}%
\pgfusepath{clip}%
\pgfsetrectcap%
\pgfsetroundjoin%
\pgfsetlinewidth{1.505625pt}%
\definecolor{currentstroke}{rgb}{1.000000,0.000000,0.000000}%
\pgfsetstrokecolor{currentstroke}%
\pgfsetdash{}{0pt}%
\pgfpathmoveto{\pgfqpoint{2.323888in}{1.870060in}}%
\pgfpathlineto{\pgfqpoint{2.331339in}{0.842232in}}%
\pgfusepath{stroke}%
\end{pgfscope}%
\begin{pgfscope}%
\pgfpathrectangle{\pgfqpoint{0.100000in}{0.212622in}}{\pgfqpoint{3.696000in}{3.696000in}}%
\pgfusepath{clip}%
\pgfsetrectcap%
\pgfsetroundjoin%
\pgfsetlinewidth{1.505625pt}%
\definecolor{currentstroke}{rgb}{1.000000,0.000000,0.000000}%
\pgfsetstrokecolor{currentstroke}%
\pgfsetdash{}{0pt}%
\pgfpathmoveto{\pgfqpoint{2.319176in}{1.874832in}}%
\pgfpathlineto{\pgfqpoint{2.331339in}{0.842232in}}%
\pgfusepath{stroke}%
\end{pgfscope}%
\begin{pgfscope}%
\pgfpathrectangle{\pgfqpoint{0.100000in}{0.212622in}}{\pgfqpoint{3.696000in}{3.696000in}}%
\pgfusepath{clip}%
\pgfsetrectcap%
\pgfsetroundjoin%
\pgfsetlinewidth{1.505625pt}%
\definecolor{currentstroke}{rgb}{1.000000,0.000000,0.000000}%
\pgfsetstrokecolor{currentstroke}%
\pgfsetdash{}{0pt}%
\pgfpathmoveto{\pgfqpoint{2.316733in}{1.877727in}}%
\pgfpathlineto{\pgfqpoint{2.331339in}{0.842232in}}%
\pgfusepath{stroke}%
\end{pgfscope}%
\begin{pgfscope}%
\pgfpathrectangle{\pgfqpoint{0.100000in}{0.212622in}}{\pgfqpoint{3.696000in}{3.696000in}}%
\pgfusepath{clip}%
\pgfsetrectcap%
\pgfsetroundjoin%
\pgfsetlinewidth{1.505625pt}%
\definecolor{currentstroke}{rgb}{1.000000,0.000000,0.000000}%
\pgfsetstrokecolor{currentstroke}%
\pgfsetdash{}{0pt}%
\pgfpathmoveto{\pgfqpoint{2.310735in}{1.883273in}}%
\pgfpathlineto{\pgfqpoint{2.316536in}{0.846935in}}%
\pgfusepath{stroke}%
\end{pgfscope}%
\begin{pgfscope}%
\pgfpathrectangle{\pgfqpoint{0.100000in}{0.212622in}}{\pgfqpoint{3.696000in}{3.696000in}}%
\pgfusepath{clip}%
\pgfsetrectcap%
\pgfsetroundjoin%
\pgfsetlinewidth{1.505625pt}%
\definecolor{currentstroke}{rgb}{1.000000,0.000000,0.000000}%
\pgfsetstrokecolor{currentstroke}%
\pgfsetdash{}{0pt}%
\pgfpathmoveto{\pgfqpoint{2.306833in}{1.884961in}}%
\pgfpathlineto{\pgfqpoint{2.316536in}{0.846935in}}%
\pgfusepath{stroke}%
\end{pgfscope}%
\begin{pgfscope}%
\pgfpathrectangle{\pgfqpoint{0.100000in}{0.212622in}}{\pgfqpoint{3.696000in}{3.696000in}}%
\pgfusepath{clip}%
\pgfsetrectcap%
\pgfsetroundjoin%
\pgfsetlinewidth{1.505625pt}%
\definecolor{currentstroke}{rgb}{1.000000,0.000000,0.000000}%
\pgfsetstrokecolor{currentstroke}%
\pgfsetdash{}{0pt}%
\pgfpathmoveto{\pgfqpoint{2.304804in}{1.886362in}}%
\pgfpathlineto{\pgfqpoint{2.316536in}{0.846935in}}%
\pgfusepath{stroke}%
\end{pgfscope}%
\begin{pgfscope}%
\pgfpathrectangle{\pgfqpoint{0.100000in}{0.212622in}}{\pgfqpoint{3.696000in}{3.696000in}}%
\pgfusepath{clip}%
\pgfsetrectcap%
\pgfsetroundjoin%
\pgfsetlinewidth{1.505625pt}%
\definecolor{currentstroke}{rgb}{1.000000,0.000000,0.000000}%
\pgfsetstrokecolor{currentstroke}%
\pgfsetdash{}{0pt}%
\pgfpathmoveto{\pgfqpoint{2.296007in}{1.889448in}}%
\pgfpathlineto{\pgfqpoint{2.301745in}{0.851636in}}%
\pgfusepath{stroke}%
\end{pgfscope}%
\begin{pgfscope}%
\pgfpathrectangle{\pgfqpoint{0.100000in}{0.212622in}}{\pgfqpoint{3.696000in}{3.696000in}}%
\pgfusepath{clip}%
\pgfsetrectcap%
\pgfsetroundjoin%
\pgfsetlinewidth{1.505625pt}%
\definecolor{currentstroke}{rgb}{1.000000,0.000000,0.000000}%
\pgfsetstrokecolor{currentstroke}%
\pgfsetdash{}{0pt}%
\pgfpathmoveto{\pgfqpoint{2.283786in}{1.895577in}}%
\pgfpathlineto{\pgfqpoint{2.301745in}{0.851636in}}%
\pgfusepath{stroke}%
\end{pgfscope}%
\begin{pgfscope}%
\pgfpathrectangle{\pgfqpoint{0.100000in}{0.212622in}}{\pgfqpoint{3.696000in}{3.696000in}}%
\pgfusepath{clip}%
\pgfsetrectcap%
\pgfsetroundjoin%
\pgfsetlinewidth{1.505625pt}%
\definecolor{currentstroke}{rgb}{1.000000,0.000000,0.000000}%
\pgfsetstrokecolor{currentstroke}%
\pgfsetdash{}{0pt}%
\pgfpathmoveto{\pgfqpoint{2.262824in}{1.905248in}}%
\pgfpathlineto{\pgfqpoint{2.272195in}{0.861026in}}%
\pgfusepath{stroke}%
\end{pgfscope}%
\begin{pgfscope}%
\pgfpathrectangle{\pgfqpoint{0.100000in}{0.212622in}}{\pgfqpoint{3.696000in}{3.696000in}}%
\pgfusepath{clip}%
\pgfsetrectcap%
\pgfsetroundjoin%
\pgfsetlinewidth{1.505625pt}%
\definecolor{currentstroke}{rgb}{1.000000,0.000000,0.000000}%
\pgfsetstrokecolor{currentstroke}%
\pgfsetdash{}{0pt}%
\pgfpathmoveto{\pgfqpoint{2.236776in}{1.920208in}}%
\pgfpathlineto{\pgfqpoint{2.242688in}{0.870402in}}%
\pgfusepath{stroke}%
\end{pgfscope}%
\begin{pgfscope}%
\pgfpathrectangle{\pgfqpoint{0.100000in}{0.212622in}}{\pgfqpoint{3.696000in}{3.696000in}}%
\pgfusepath{clip}%
\pgfsetrectcap%
\pgfsetroundjoin%
\pgfsetlinewidth{1.505625pt}%
\definecolor{currentstroke}{rgb}{1.000000,0.000000,0.000000}%
\pgfsetstrokecolor{currentstroke}%
\pgfsetdash{}{0pt}%
\pgfpathmoveto{\pgfqpoint{2.207074in}{1.933096in}}%
\pgfpathlineto{\pgfqpoint{2.213224in}{0.879765in}}%
\pgfusepath{stroke}%
\end{pgfscope}%
\begin{pgfscope}%
\pgfpathrectangle{\pgfqpoint{0.100000in}{0.212622in}}{\pgfqpoint{3.696000in}{3.696000in}}%
\pgfusepath{clip}%
\pgfsetrectcap%
\pgfsetroundjoin%
\pgfsetlinewidth{1.505625pt}%
\definecolor{currentstroke}{rgb}{1.000000,0.000000,0.000000}%
\pgfsetstrokecolor{currentstroke}%
\pgfsetdash{}{0pt}%
\pgfpathmoveto{\pgfqpoint{2.188848in}{1.939449in}}%
\pgfpathlineto{\pgfqpoint{2.198509in}{0.884441in}}%
\pgfusepath{stroke}%
\end{pgfscope}%
\begin{pgfscope}%
\pgfpathrectangle{\pgfqpoint{0.100000in}{0.212622in}}{\pgfqpoint{3.696000in}{3.696000in}}%
\pgfusepath{clip}%
\pgfsetrectcap%
\pgfsetroundjoin%
\pgfsetlinewidth{1.505625pt}%
\definecolor{currentstroke}{rgb}{1.000000,0.000000,0.000000}%
\pgfsetstrokecolor{currentstroke}%
\pgfsetdash{}{0pt}%
\pgfpathmoveto{\pgfqpoint{2.166009in}{1.954595in}}%
\pgfpathlineto{\pgfqpoint{2.183804in}{0.889113in}}%
\pgfusepath{stroke}%
\end{pgfscope}%
\begin{pgfscope}%
\pgfpathrectangle{\pgfqpoint{0.100000in}{0.212622in}}{\pgfqpoint{3.696000in}{3.696000in}}%
\pgfusepath{clip}%
\pgfsetrectcap%
\pgfsetroundjoin%
\pgfsetlinewidth{1.505625pt}%
\definecolor{currentstroke}{rgb}{1.000000,0.000000,0.000000}%
\pgfsetstrokecolor{currentstroke}%
\pgfsetdash{}{0pt}%
\pgfpathmoveto{\pgfqpoint{2.142868in}{1.958735in}}%
\pgfpathlineto{\pgfqpoint{2.154426in}{0.898449in}}%
\pgfusepath{stroke}%
\end{pgfscope}%
\begin{pgfscope}%
\pgfpathrectangle{\pgfqpoint{0.100000in}{0.212622in}}{\pgfqpoint{3.696000in}{3.696000in}}%
\pgfusepath{clip}%
\pgfsetrectcap%
\pgfsetroundjoin%
\pgfsetlinewidth{1.505625pt}%
\definecolor{currentstroke}{rgb}{1.000000,0.000000,0.000000}%
\pgfsetstrokecolor{currentstroke}%
\pgfsetdash{}{0pt}%
\pgfpathmoveto{\pgfqpoint{2.128507in}{1.968681in}}%
\pgfpathlineto{\pgfqpoint{2.139754in}{0.903111in}}%
\pgfusepath{stroke}%
\end{pgfscope}%
\begin{pgfscope}%
\pgfpathrectangle{\pgfqpoint{0.100000in}{0.212622in}}{\pgfqpoint{3.696000in}{3.696000in}}%
\pgfusepath{clip}%
\pgfsetrectcap%
\pgfsetroundjoin%
\pgfsetlinewidth{1.505625pt}%
\definecolor{currentstroke}{rgb}{1.000000,0.000000,0.000000}%
\pgfsetstrokecolor{currentstroke}%
\pgfsetdash{}{0pt}%
\pgfpathmoveto{\pgfqpoint{2.121227in}{1.970050in}}%
\pgfpathlineto{\pgfqpoint{2.125092in}{0.907770in}}%
\pgfusepath{stroke}%
\end{pgfscope}%
\begin{pgfscope}%
\pgfpathrectangle{\pgfqpoint{0.100000in}{0.212622in}}{\pgfqpoint{3.696000in}{3.696000in}}%
\pgfusepath{clip}%
\pgfsetrectcap%
\pgfsetroundjoin%
\pgfsetlinewidth{1.505625pt}%
\definecolor{currentstroke}{rgb}{1.000000,0.000000,0.000000}%
\pgfsetstrokecolor{currentstroke}%
\pgfsetdash{}{0pt}%
\pgfpathmoveto{\pgfqpoint{2.111102in}{1.974194in}}%
\pgfpathlineto{\pgfqpoint{2.125092in}{0.907770in}}%
\pgfusepath{stroke}%
\end{pgfscope}%
\begin{pgfscope}%
\pgfpathrectangle{\pgfqpoint{0.100000in}{0.212622in}}{\pgfqpoint{3.696000in}{3.696000in}}%
\pgfusepath{clip}%
\pgfsetrectcap%
\pgfsetroundjoin%
\pgfsetlinewidth{1.505625pt}%
\definecolor{currentstroke}{rgb}{1.000000,0.000000,0.000000}%
\pgfsetstrokecolor{currentstroke}%
\pgfsetdash{}{0pt}%
\pgfpathmoveto{\pgfqpoint{2.097616in}{1.978420in}}%
\pgfpathlineto{\pgfqpoint{2.110441in}{0.912426in}}%
\pgfusepath{stroke}%
\end{pgfscope}%
\begin{pgfscope}%
\pgfpathrectangle{\pgfqpoint{0.100000in}{0.212622in}}{\pgfqpoint{3.696000in}{3.696000in}}%
\pgfusepath{clip}%
\pgfsetrectcap%
\pgfsetroundjoin%
\pgfsetlinewidth{1.505625pt}%
\definecolor{currentstroke}{rgb}{1.000000,0.000000,0.000000}%
\pgfsetstrokecolor{currentstroke}%
\pgfsetdash{}{0pt}%
\pgfpathmoveto{\pgfqpoint{2.079999in}{1.985564in}}%
\pgfpathlineto{\pgfqpoint{2.095800in}{0.917078in}}%
\pgfusepath{stroke}%
\end{pgfscope}%
\begin{pgfscope}%
\pgfpathrectangle{\pgfqpoint{0.100000in}{0.212622in}}{\pgfqpoint{3.696000in}{3.696000in}}%
\pgfusepath{clip}%
\pgfsetrectcap%
\pgfsetroundjoin%
\pgfsetlinewidth{1.505625pt}%
\definecolor{currentstroke}{rgb}{1.000000,0.000000,0.000000}%
\pgfsetstrokecolor{currentstroke}%
\pgfsetdash{}{0pt}%
\pgfpathmoveto{\pgfqpoint{2.054452in}{1.996968in}}%
\pgfpathlineto{\pgfqpoint{2.066552in}{0.926372in}}%
\pgfusepath{stroke}%
\end{pgfscope}%
\begin{pgfscope}%
\pgfpathrectangle{\pgfqpoint{0.100000in}{0.212622in}}{\pgfqpoint{3.696000in}{3.696000in}}%
\pgfusepath{clip}%
\pgfsetrectcap%
\pgfsetroundjoin%
\pgfsetlinewidth{1.505625pt}%
\definecolor{currentstroke}{rgb}{1.000000,0.000000,0.000000}%
\pgfsetstrokecolor{currentstroke}%
\pgfsetdash{}{0pt}%
\pgfpathmoveto{\pgfqpoint{2.020791in}{2.010327in}}%
\pgfpathlineto{\pgfqpoint{2.037345in}{0.935653in}}%
\pgfusepath{stroke}%
\end{pgfscope}%
\begin{pgfscope}%
\pgfpathrectangle{\pgfqpoint{0.100000in}{0.212622in}}{\pgfqpoint{3.696000in}{3.696000in}}%
\pgfusepath{clip}%
\pgfsetrectcap%
\pgfsetroundjoin%
\pgfsetlinewidth{1.505625pt}%
\definecolor{currentstroke}{rgb}{1.000000,0.000000,0.000000}%
\pgfsetstrokecolor{currentstroke}%
\pgfsetdash{}{0pt}%
\pgfpathmoveto{\pgfqpoint{1.982786in}{2.035519in}}%
\pgfpathlineto{\pgfqpoint{1.993616in}{0.949549in}}%
\pgfusepath{stroke}%
\end{pgfscope}%
\begin{pgfscope}%
\pgfpathrectangle{\pgfqpoint{0.100000in}{0.212622in}}{\pgfqpoint{3.696000in}{3.696000in}}%
\pgfusepath{clip}%
\pgfsetrectcap%
\pgfsetroundjoin%
\pgfsetlinewidth{1.505625pt}%
\definecolor{currentstroke}{rgb}{1.000000,0.000000,0.000000}%
\pgfsetstrokecolor{currentstroke}%
\pgfsetdash{}{0pt}%
\pgfpathmoveto{\pgfqpoint{1.947006in}{2.052233in}}%
\pgfpathlineto{\pgfqpoint{1.964516in}{0.958796in}}%
\pgfusepath{stroke}%
\end{pgfscope}%
\begin{pgfscope}%
\pgfpathrectangle{\pgfqpoint{0.100000in}{0.212622in}}{\pgfqpoint{3.696000in}{3.696000in}}%
\pgfusepath{clip}%
\pgfsetrectcap%
\pgfsetroundjoin%
\pgfsetlinewidth{1.505625pt}%
\definecolor{currentstroke}{rgb}{1.000000,0.000000,0.000000}%
\pgfsetstrokecolor{currentstroke}%
\pgfsetdash{}{0pt}%
\pgfpathmoveto{\pgfqpoint{1.924767in}{2.067610in}}%
\pgfpathlineto{\pgfqpoint{1.935459in}{0.968029in}}%
\pgfusepath{stroke}%
\end{pgfscope}%
\begin{pgfscope}%
\pgfpathrectangle{\pgfqpoint{0.100000in}{0.212622in}}{\pgfqpoint{3.696000in}{3.696000in}}%
\pgfusepath{clip}%
\pgfsetrectcap%
\pgfsetroundjoin%
\pgfsetlinewidth{1.505625pt}%
\definecolor{currentstroke}{rgb}{1.000000,0.000000,0.000000}%
\pgfsetstrokecolor{currentstroke}%
\pgfsetdash{}{0pt}%
\pgfpathmoveto{\pgfqpoint{1.913105in}{2.071860in}}%
\pgfpathlineto{\pgfqpoint{1.920946in}{0.972641in}}%
\pgfusepath{stroke}%
\end{pgfscope}%
\begin{pgfscope}%
\pgfpathrectangle{\pgfqpoint{0.100000in}{0.212622in}}{\pgfqpoint{3.696000in}{3.696000in}}%
\pgfusepath{clip}%
\pgfsetrectcap%
\pgfsetroundjoin%
\pgfsetlinewidth{1.505625pt}%
\definecolor{currentstroke}{rgb}{1.000000,0.000000,0.000000}%
\pgfsetstrokecolor{currentstroke}%
\pgfsetdash{}{0pt}%
\pgfpathmoveto{\pgfqpoint{1.906427in}{2.075747in}}%
\pgfpathlineto{\pgfqpoint{1.920946in}{0.972641in}}%
\pgfusepath{stroke}%
\end{pgfscope}%
\begin{pgfscope}%
\pgfpathrectangle{\pgfqpoint{0.100000in}{0.212622in}}{\pgfqpoint{3.696000in}{3.696000in}}%
\pgfusepath{clip}%
\pgfsetrectcap%
\pgfsetroundjoin%
\pgfsetlinewidth{1.505625pt}%
\definecolor{currentstroke}{rgb}{1.000000,0.000000,0.000000}%
\pgfsetstrokecolor{currentstroke}%
\pgfsetdash{}{0pt}%
\pgfpathmoveto{\pgfqpoint{1.895117in}{2.080024in}}%
\pgfpathlineto{\pgfqpoint{1.906444in}{0.977249in}}%
\pgfusepath{stroke}%
\end{pgfscope}%
\begin{pgfscope}%
\pgfpathrectangle{\pgfqpoint{0.100000in}{0.212622in}}{\pgfqpoint{3.696000in}{3.696000in}}%
\pgfusepath{clip}%
\pgfsetrectcap%
\pgfsetroundjoin%
\pgfsetlinewidth{1.505625pt}%
\definecolor{currentstroke}{rgb}{1.000000,0.000000,0.000000}%
\pgfsetstrokecolor{currentstroke}%
\pgfsetdash{}{0pt}%
\pgfpathmoveto{\pgfqpoint{1.879480in}{2.090704in}}%
\pgfpathlineto{\pgfqpoint{1.891952in}{0.981854in}}%
\pgfusepath{stroke}%
\end{pgfscope}%
\begin{pgfscope}%
\pgfpathrectangle{\pgfqpoint{0.100000in}{0.212622in}}{\pgfqpoint{3.696000in}{3.696000in}}%
\pgfusepath{clip}%
\pgfsetrectcap%
\pgfsetroundjoin%
\pgfsetlinewidth{1.505625pt}%
\definecolor{currentstroke}{rgb}{1.000000,0.000000,0.000000}%
\pgfsetstrokecolor{currentstroke}%
\pgfsetdash{}{0pt}%
\pgfpathmoveto{\pgfqpoint{1.861053in}{2.095401in}}%
\pgfpathlineto{\pgfqpoint{1.877471in}{0.986456in}}%
\pgfusepath{stroke}%
\end{pgfscope}%
\begin{pgfscope}%
\pgfpathrectangle{\pgfqpoint{0.100000in}{0.212622in}}{\pgfqpoint{3.696000in}{3.696000in}}%
\pgfusepath{clip}%
\pgfsetrectcap%
\pgfsetroundjoin%
\pgfsetlinewidth{1.505625pt}%
\definecolor{currentstroke}{rgb}{1.000000,0.000000,0.000000}%
\pgfsetstrokecolor{currentstroke}%
\pgfsetdash{}{0pt}%
\pgfpathmoveto{\pgfqpoint{1.833729in}{2.115909in}}%
\pgfpathlineto{\pgfqpoint{1.848540in}{0.995649in}}%
\pgfusepath{stroke}%
\end{pgfscope}%
\begin{pgfscope}%
\pgfpathrectangle{\pgfqpoint{0.100000in}{0.212622in}}{\pgfqpoint{3.696000in}{3.696000in}}%
\pgfusepath{clip}%
\pgfsetrectcap%
\pgfsetroundjoin%
\pgfsetlinewidth{1.505625pt}%
\definecolor{currentstroke}{rgb}{1.000000,0.000000,0.000000}%
\pgfsetstrokecolor{currentstroke}%
\pgfsetdash{}{0pt}%
\pgfpathmoveto{\pgfqpoint{1.806982in}{2.120212in}}%
\pgfpathlineto{\pgfqpoint{1.819651in}{1.004829in}}%
\pgfusepath{stroke}%
\end{pgfscope}%
\begin{pgfscope}%
\pgfpathrectangle{\pgfqpoint{0.100000in}{0.212622in}}{\pgfqpoint{3.696000in}{3.696000in}}%
\pgfusepath{clip}%
\pgfsetrectcap%
\pgfsetroundjoin%
\pgfsetlinewidth{1.505625pt}%
\definecolor{currentstroke}{rgb}{1.000000,0.000000,0.000000}%
\pgfsetstrokecolor{currentstroke}%
\pgfsetdash{}{0pt}%
\pgfpathmoveto{\pgfqpoint{1.788430in}{2.132043in}}%
\pgfpathlineto{\pgfqpoint{1.805222in}{1.009414in}}%
\pgfusepath{stroke}%
\end{pgfscope}%
\begin{pgfscope}%
\pgfpathrectangle{\pgfqpoint{0.100000in}{0.212622in}}{\pgfqpoint{3.696000in}{3.696000in}}%
\pgfusepath{clip}%
\pgfsetrectcap%
\pgfsetroundjoin%
\pgfsetlinewidth{1.505625pt}%
\definecolor{currentstroke}{rgb}{1.000000,0.000000,0.000000}%
\pgfsetstrokecolor{currentstroke}%
\pgfsetdash{}{0pt}%
\pgfpathmoveto{\pgfqpoint{1.780026in}{2.133867in}}%
\pgfpathlineto{\pgfqpoint{1.790804in}{1.013996in}}%
\pgfusepath{stroke}%
\end{pgfscope}%
\begin{pgfscope}%
\pgfpathrectangle{\pgfqpoint{0.100000in}{0.212622in}}{\pgfqpoint{3.696000in}{3.696000in}}%
\pgfusepath{clip}%
\pgfsetrectcap%
\pgfsetroundjoin%
\pgfsetlinewidth{1.505625pt}%
\definecolor{currentstroke}{rgb}{1.000000,0.000000,0.000000}%
\pgfsetstrokecolor{currentstroke}%
\pgfsetdash{}{0pt}%
\pgfpathmoveto{\pgfqpoint{1.767935in}{2.141397in}}%
\pgfpathlineto{\pgfqpoint{1.790804in}{1.013996in}}%
\pgfusepath{stroke}%
\end{pgfscope}%
\begin{pgfscope}%
\pgfpathrectangle{\pgfqpoint{0.100000in}{0.212622in}}{\pgfqpoint{3.696000in}{3.696000in}}%
\pgfusepath{clip}%
\pgfsetrectcap%
\pgfsetroundjoin%
\pgfsetlinewidth{1.505625pt}%
\definecolor{currentstroke}{rgb}{1.000000,0.000000,0.000000}%
\pgfsetstrokecolor{currentstroke}%
\pgfsetdash{}{0pt}%
\pgfpathmoveto{\pgfqpoint{1.755069in}{2.149398in}}%
\pgfpathlineto{\pgfqpoint{1.776396in}{1.018574in}}%
\pgfusepath{stroke}%
\end{pgfscope}%
\begin{pgfscope}%
\pgfpathrectangle{\pgfqpoint{0.100000in}{0.212622in}}{\pgfqpoint{3.696000in}{3.696000in}}%
\pgfusepath{clip}%
\pgfsetrectcap%
\pgfsetroundjoin%
\pgfsetlinewidth{1.505625pt}%
\definecolor{currentstroke}{rgb}{1.000000,0.000000,0.000000}%
\pgfsetstrokecolor{currentstroke}%
\pgfsetdash{}{0pt}%
\pgfpathmoveto{\pgfqpoint{1.740093in}{2.157378in}}%
\pgfpathlineto{\pgfqpoint{1.761999in}{1.023149in}}%
\pgfusepath{stroke}%
\end{pgfscope}%
\begin{pgfscope}%
\pgfpathrectangle{\pgfqpoint{0.100000in}{0.212622in}}{\pgfqpoint{3.696000in}{3.696000in}}%
\pgfusepath{clip}%
\pgfsetrectcap%
\pgfsetroundjoin%
\pgfsetlinewidth{1.505625pt}%
\definecolor{currentstroke}{rgb}{1.000000,0.000000,0.000000}%
\pgfsetstrokecolor{currentstroke}%
\pgfsetdash{}{0pt}%
\pgfpathmoveto{\pgfqpoint{1.720057in}{2.167601in}}%
\pgfpathlineto{\pgfqpoint{1.733235in}{1.032289in}}%
\pgfusepath{stroke}%
\end{pgfscope}%
\begin{pgfscope}%
\pgfpathrectangle{\pgfqpoint{0.100000in}{0.212622in}}{\pgfqpoint{3.696000in}{3.696000in}}%
\pgfusepath{clip}%
\pgfsetrectcap%
\pgfsetroundjoin%
\pgfsetlinewidth{1.505625pt}%
\definecolor{currentstroke}{rgb}{1.000000,0.000000,0.000000}%
\pgfsetstrokecolor{currentstroke}%
\pgfsetdash{}{0pt}%
\pgfpathmoveto{\pgfqpoint{1.696836in}{2.178847in}}%
\pgfpathlineto{\pgfqpoint{1.718869in}{1.036854in}}%
\pgfusepath{stroke}%
\end{pgfscope}%
\begin{pgfscope}%
\pgfpathrectangle{\pgfqpoint{0.100000in}{0.212622in}}{\pgfqpoint{3.696000in}{3.696000in}}%
\pgfusepath{clip}%
\pgfsetrectcap%
\pgfsetroundjoin%
\pgfsetlinewidth{1.505625pt}%
\definecolor{currentstroke}{rgb}{1.000000,0.000000,0.000000}%
\pgfsetstrokecolor{currentstroke}%
\pgfsetdash{}{0pt}%
\pgfpathmoveto{\pgfqpoint{1.666674in}{2.198950in}}%
\pgfpathlineto{\pgfqpoint{1.690168in}{1.045974in}}%
\pgfusepath{stroke}%
\end{pgfscope}%
\begin{pgfscope}%
\pgfpathrectangle{\pgfqpoint{0.100000in}{0.212622in}}{\pgfqpoint{3.696000in}{3.696000in}}%
\pgfusepath{clip}%
\pgfsetrectcap%
\pgfsetroundjoin%
\pgfsetlinewidth{1.505625pt}%
\definecolor{currentstroke}{rgb}{1.000000,0.000000,0.000000}%
\pgfsetstrokecolor{currentstroke}%
\pgfsetdash{}{0pt}%
\pgfpathmoveto{\pgfqpoint{1.633678in}{2.213323in}}%
\pgfpathlineto{\pgfqpoint{1.647194in}{1.059630in}}%
\pgfusepath{stroke}%
\end{pgfscope}%
\begin{pgfscope}%
\pgfpathrectangle{\pgfqpoint{0.100000in}{0.212622in}}{\pgfqpoint{3.696000in}{3.696000in}}%
\pgfusepath{clip}%
\pgfsetrectcap%
\pgfsetroundjoin%
\pgfsetlinewidth{1.505625pt}%
\definecolor{currentstroke}{rgb}{1.000000,0.000000,0.000000}%
\pgfsetstrokecolor{currentstroke}%
\pgfsetdash{}{0pt}%
\pgfpathmoveto{\pgfqpoint{1.592452in}{2.236842in}}%
\pgfpathlineto{\pgfqpoint{1.618596in}{1.068717in}}%
\pgfusepath{stroke}%
\end{pgfscope}%
\begin{pgfscope}%
\pgfpathrectangle{\pgfqpoint{0.100000in}{0.212622in}}{\pgfqpoint{3.696000in}{3.696000in}}%
\pgfusepath{clip}%
\pgfsetrectcap%
\pgfsetroundjoin%
\pgfsetlinewidth{1.505625pt}%
\definecolor{currentstroke}{rgb}{1.000000,0.000000,0.000000}%
\pgfsetstrokecolor{currentstroke}%
\pgfsetdash{}{0pt}%
\pgfpathmoveto{\pgfqpoint{1.551380in}{2.248832in}}%
\pgfpathlineto{\pgfqpoint{1.575778in}{1.082324in}}%
\pgfusepath{stroke}%
\end{pgfscope}%
\begin{pgfscope}%
\pgfpathrectangle{\pgfqpoint{0.100000in}{0.212622in}}{\pgfqpoint{3.696000in}{3.696000in}}%
\pgfusepath{clip}%
\pgfsetrectcap%
\pgfsetroundjoin%
\pgfsetlinewidth{1.505625pt}%
\definecolor{currentstroke}{rgb}{1.000000,0.000000,0.000000}%
\pgfsetstrokecolor{currentstroke}%
\pgfsetdash{}{0pt}%
\pgfpathmoveto{\pgfqpoint{1.502832in}{2.281767in}}%
\pgfpathlineto{\pgfqpoint{1.518830in}{1.100420in}}%
\pgfusepath{stroke}%
\end{pgfscope}%
\begin{pgfscope}%
\pgfpathrectangle{\pgfqpoint{0.100000in}{0.212622in}}{\pgfqpoint{3.696000in}{3.696000in}}%
\pgfusepath{clip}%
\pgfsetrectcap%
\pgfsetroundjoin%
\pgfsetlinewidth{1.505625pt}%
\definecolor{currentstroke}{rgb}{1.000000,0.000000,0.000000}%
\pgfsetstrokecolor{currentstroke}%
\pgfsetdash{}{0pt}%
\pgfpathmoveto{\pgfqpoint{1.479158in}{2.293545in}}%
\pgfpathlineto{\pgfqpoint{1.504618in}{1.104936in}}%
\pgfusepath{stroke}%
\end{pgfscope}%
\begin{pgfscope}%
\pgfpathrectangle{\pgfqpoint{0.100000in}{0.212622in}}{\pgfqpoint{3.696000in}{3.696000in}}%
\pgfusepath{clip}%
\pgfsetrectcap%
\pgfsetroundjoin%
\pgfsetlinewidth{1.505625pt}%
\definecolor{currentstroke}{rgb}{1.000000,0.000000,0.000000}%
\pgfsetstrokecolor{currentstroke}%
\pgfsetdash{}{0pt}%
\pgfpathmoveto{\pgfqpoint{1.465250in}{2.298229in}}%
\pgfpathlineto{\pgfqpoint{1.490417in}{1.109448in}}%
\pgfusepath{stroke}%
\end{pgfscope}%
\begin{pgfscope}%
\pgfpathrectangle{\pgfqpoint{0.100000in}{0.212622in}}{\pgfqpoint{3.696000in}{3.696000in}}%
\pgfusepath{clip}%
\pgfsetrectcap%
\pgfsetroundjoin%
\pgfsetlinewidth{1.505625pt}%
\definecolor{currentstroke}{rgb}{1.000000,0.000000,0.000000}%
\pgfsetstrokecolor{currentstroke}%
\pgfsetdash{}{0pt}%
\pgfpathmoveto{\pgfqpoint{1.448551in}{2.305796in}}%
\pgfpathlineto{\pgfqpoint{1.476226in}{1.113958in}}%
\pgfusepath{stroke}%
\end{pgfscope}%
\begin{pgfscope}%
\pgfpathrectangle{\pgfqpoint{0.100000in}{0.212622in}}{\pgfqpoint{3.696000in}{3.696000in}}%
\pgfusepath{clip}%
\pgfsetrectcap%
\pgfsetroundjoin%
\pgfsetlinewidth{1.505625pt}%
\definecolor{currentstroke}{rgb}{1.000000,0.000000,0.000000}%
\pgfsetstrokecolor{currentstroke}%
\pgfsetdash{}{0pt}%
\pgfpathmoveto{\pgfqpoint{1.428873in}{2.314579in}}%
\pgfpathlineto{\pgfqpoint{1.447875in}{1.122967in}}%
\pgfusepath{stroke}%
\end{pgfscope}%
\begin{pgfscope}%
\pgfpathrectangle{\pgfqpoint{0.100000in}{0.212622in}}{\pgfqpoint{3.696000in}{3.696000in}}%
\pgfusepath{clip}%
\pgfsetrectcap%
\pgfsetroundjoin%
\pgfsetlinewidth{1.505625pt}%
\definecolor{currentstroke}{rgb}{1.000000,0.000000,0.000000}%
\pgfsetstrokecolor{currentstroke}%
\pgfsetdash{}{0pt}%
\pgfpathmoveto{\pgfqpoint{1.405994in}{2.330454in}}%
\pgfpathlineto{\pgfqpoint{1.433714in}{1.127467in}}%
\pgfusepath{stroke}%
\end{pgfscope}%
\begin{pgfscope}%
\pgfpathrectangle{\pgfqpoint{0.100000in}{0.212622in}}{\pgfqpoint{3.696000in}{3.696000in}}%
\pgfusepath{clip}%
\pgfsetrectcap%
\pgfsetroundjoin%
\pgfsetlinewidth{1.505625pt}%
\definecolor{currentstroke}{rgb}{1.000000,0.000000,0.000000}%
\pgfsetstrokecolor{currentstroke}%
\pgfsetdash{}{0pt}%
\pgfpathmoveto{\pgfqpoint{1.375672in}{2.345826in}}%
\pgfpathlineto{\pgfqpoint{1.405424in}{1.136456in}}%
\pgfusepath{stroke}%
\end{pgfscope}%
\begin{pgfscope}%
\pgfpathrectangle{\pgfqpoint{0.100000in}{0.212622in}}{\pgfqpoint{3.696000in}{3.696000in}}%
\pgfusepath{clip}%
\pgfsetrectcap%
\pgfsetroundjoin%
\pgfsetlinewidth{1.505625pt}%
\definecolor{currentstroke}{rgb}{1.000000,0.000000,0.000000}%
\pgfsetstrokecolor{currentstroke}%
\pgfsetdash{}{0pt}%
\pgfpathmoveto{\pgfqpoint{1.340107in}{2.368208in}}%
\pgfpathlineto{\pgfqpoint{1.363065in}{1.149917in}}%
\pgfusepath{stroke}%
\end{pgfscope}%
\begin{pgfscope}%
\pgfpathrectangle{\pgfqpoint{0.100000in}{0.212622in}}{\pgfqpoint{3.696000in}{3.696000in}}%
\pgfusepath{clip}%
\pgfsetrectcap%
\pgfsetroundjoin%
\pgfsetlinewidth{1.505625pt}%
\definecolor{currentstroke}{rgb}{1.000000,0.000000,0.000000}%
\pgfsetstrokecolor{currentstroke}%
\pgfsetdash{}{0pt}%
\pgfpathmoveto{\pgfqpoint{1.303222in}{2.390575in}}%
\pgfpathlineto{\pgfqpoint{1.334876in}{1.158874in}}%
\pgfusepath{stroke}%
\end{pgfscope}%
\begin{pgfscope}%
\pgfpathrectangle{\pgfqpoint{0.100000in}{0.212622in}}{\pgfqpoint{3.696000in}{3.696000in}}%
\pgfusepath{clip}%
\pgfsetrectcap%
\pgfsetroundjoin%
\pgfsetlinewidth{1.505625pt}%
\definecolor{currentstroke}{rgb}{1.000000,0.000000,0.000000}%
\pgfsetstrokecolor{currentstroke}%
\pgfsetdash{}{0pt}%
\pgfpathmoveto{\pgfqpoint{1.282281in}{2.401600in}}%
\pgfpathlineto{\pgfqpoint{1.306728in}{1.167819in}}%
\pgfusepath{stroke}%
\end{pgfscope}%
\begin{pgfscope}%
\pgfpathrectangle{\pgfqpoint{0.100000in}{0.212622in}}{\pgfqpoint{3.696000in}{3.696000in}}%
\pgfusepath{clip}%
\pgfsetrectcap%
\pgfsetroundjoin%
\pgfsetlinewidth{1.505625pt}%
\definecolor{currentstroke}{rgb}{1.000000,0.000000,0.000000}%
\pgfsetstrokecolor{currentstroke}%
\pgfsetdash{}{0pt}%
\pgfpathmoveto{\pgfqpoint{1.271026in}{2.408322in}}%
\pgfpathlineto{\pgfqpoint{1.292669in}{1.172286in}}%
\pgfusepath{stroke}%
\end{pgfscope}%
\begin{pgfscope}%
\pgfpathrectangle{\pgfqpoint{0.100000in}{0.212622in}}{\pgfqpoint{3.696000in}{3.696000in}}%
\pgfusepath{clip}%
\pgfsetrectcap%
\pgfsetroundjoin%
\pgfsetlinewidth{1.505625pt}%
\definecolor{currentstroke}{rgb}{1.000000,0.000000,0.000000}%
\pgfsetstrokecolor{currentstroke}%
\pgfsetdash{}{0pt}%
\pgfpathmoveto{\pgfqpoint{1.264901in}{2.409821in}}%
\pgfpathlineto{\pgfqpoint{1.292669in}{1.172286in}}%
\pgfusepath{stroke}%
\end{pgfscope}%
\begin{pgfscope}%
\pgfpathrectangle{\pgfqpoint{0.100000in}{0.212622in}}{\pgfqpoint{3.696000in}{3.696000in}}%
\pgfusepath{clip}%
\pgfsetrectcap%
\pgfsetroundjoin%
\pgfsetlinewidth{1.505625pt}%
\definecolor{currentstroke}{rgb}{1.000000,0.000000,0.000000}%
\pgfsetstrokecolor{currentstroke}%
\pgfsetdash{}{0pt}%
\pgfpathmoveto{\pgfqpoint{1.261205in}{2.412618in}}%
\pgfpathlineto{\pgfqpoint{1.292669in}{1.172286in}}%
\pgfusepath{stroke}%
\end{pgfscope}%
\begin{pgfscope}%
\pgfpathrectangle{\pgfqpoint{0.100000in}{0.212622in}}{\pgfqpoint{3.696000in}{3.696000in}}%
\pgfusepath{clip}%
\pgfsetrectcap%
\pgfsetroundjoin%
\pgfsetlinewidth{1.505625pt}%
\definecolor{currentstroke}{rgb}{1.000000,0.000000,0.000000}%
\pgfsetstrokecolor{currentstroke}%
\pgfsetdash{}{0pt}%
\pgfpathmoveto{\pgfqpoint{1.259359in}{2.412889in}}%
\pgfpathlineto{\pgfqpoint{1.292669in}{1.172286in}}%
\pgfusepath{stroke}%
\end{pgfscope}%
\begin{pgfscope}%
\pgfpathrectangle{\pgfqpoint{0.100000in}{0.212622in}}{\pgfqpoint{3.696000in}{3.696000in}}%
\pgfusepath{clip}%
\pgfsetrectcap%
\pgfsetroundjoin%
\pgfsetlinewidth{1.505625pt}%
\definecolor{currentstroke}{rgb}{1.000000,0.000000,0.000000}%
\pgfsetstrokecolor{currentstroke}%
\pgfsetdash{}{0pt}%
\pgfpathmoveto{\pgfqpoint{1.258229in}{2.413468in}}%
\pgfpathlineto{\pgfqpoint{1.292669in}{1.172286in}}%
\pgfusepath{stroke}%
\end{pgfscope}%
\begin{pgfscope}%
\pgfpathrectangle{\pgfqpoint{0.100000in}{0.212622in}}{\pgfqpoint{3.696000in}{3.696000in}}%
\pgfusepath{clip}%
\pgfsetrectcap%
\pgfsetroundjoin%
\pgfsetlinewidth{1.505625pt}%
\definecolor{currentstroke}{rgb}{1.000000,0.000000,0.000000}%
\pgfsetstrokecolor{currentstroke}%
\pgfsetdash{}{0pt}%
\pgfpathmoveto{\pgfqpoint{1.253137in}{2.414922in}}%
\pgfpathlineto{\pgfqpoint{1.278620in}{1.176751in}}%
\pgfusepath{stroke}%
\end{pgfscope}%
\begin{pgfscope}%
\pgfpathrectangle{\pgfqpoint{0.100000in}{0.212622in}}{\pgfqpoint{3.696000in}{3.696000in}}%
\pgfusepath{clip}%
\pgfsetrectcap%
\pgfsetroundjoin%
\pgfsetlinewidth{1.505625pt}%
\definecolor{currentstroke}{rgb}{1.000000,0.000000,0.000000}%
\pgfsetstrokecolor{currentstroke}%
\pgfsetdash{}{0pt}%
\pgfpathmoveto{\pgfqpoint{1.241896in}{2.420590in}}%
\pgfpathlineto{\pgfqpoint{1.278620in}{1.176751in}}%
\pgfusepath{stroke}%
\end{pgfscope}%
\begin{pgfscope}%
\pgfpathrectangle{\pgfqpoint{0.100000in}{0.212622in}}{\pgfqpoint{3.696000in}{3.696000in}}%
\pgfusepath{clip}%
\pgfsetrectcap%
\pgfsetroundjoin%
\pgfsetlinewidth{1.505625pt}%
\definecolor{currentstroke}{rgb}{1.000000,0.000000,0.000000}%
\pgfsetstrokecolor{currentstroke}%
\pgfsetdash{}{0pt}%
\pgfpathmoveto{\pgfqpoint{1.225639in}{2.426094in}}%
\pgfpathlineto{\pgfqpoint{1.250552in}{1.185670in}}%
\pgfusepath{stroke}%
\end{pgfscope}%
\begin{pgfscope}%
\pgfpathrectangle{\pgfqpoint{0.100000in}{0.212622in}}{\pgfqpoint{3.696000in}{3.696000in}}%
\pgfusepath{clip}%
\pgfsetrectcap%
\pgfsetroundjoin%
\pgfsetlinewidth{1.505625pt}%
\definecolor{currentstroke}{rgb}{1.000000,0.000000,0.000000}%
\pgfsetstrokecolor{currentstroke}%
\pgfsetdash{}{0pt}%
\pgfpathmoveto{\pgfqpoint{1.204895in}{2.437182in}}%
\pgfpathlineto{\pgfqpoint{1.236533in}{1.190124in}}%
\pgfusepath{stroke}%
\end{pgfscope}%
\begin{pgfscope}%
\pgfpathrectangle{\pgfqpoint{0.100000in}{0.212622in}}{\pgfqpoint{3.696000in}{3.696000in}}%
\pgfusepath{clip}%
\pgfsetrectcap%
\pgfsetroundjoin%
\pgfsetlinewidth{1.505625pt}%
\definecolor{currentstroke}{rgb}{1.000000,0.000000,0.000000}%
\pgfsetstrokecolor{currentstroke}%
\pgfsetdash{}{0pt}%
\pgfpathmoveto{\pgfqpoint{1.183636in}{2.446261in}}%
\pgfpathlineto{\pgfqpoint{1.222524in}{1.194576in}}%
\pgfusepath{stroke}%
\end{pgfscope}%
\begin{pgfscope}%
\pgfpathrectangle{\pgfqpoint{0.100000in}{0.212622in}}{\pgfqpoint{3.696000in}{3.696000in}}%
\pgfusepath{clip}%
\pgfsetrectcap%
\pgfsetroundjoin%
\pgfsetlinewidth{1.505625pt}%
\definecolor{currentstroke}{rgb}{1.000000,0.000000,0.000000}%
\pgfsetstrokecolor{currentstroke}%
\pgfsetdash{}{0pt}%
\pgfpathmoveto{\pgfqpoint{1.159633in}{2.458690in}}%
\pgfpathlineto{\pgfqpoint{1.194536in}{1.203469in}}%
\pgfusepath{stroke}%
\end{pgfscope}%
\begin{pgfscope}%
\pgfpathrectangle{\pgfqpoint{0.100000in}{0.212622in}}{\pgfqpoint{3.696000in}{3.696000in}}%
\pgfusepath{clip}%
\pgfsetrectcap%
\pgfsetroundjoin%
\pgfsetlinewidth{1.505625pt}%
\definecolor{currentstroke}{rgb}{1.000000,0.000000,0.000000}%
\pgfsetstrokecolor{currentstroke}%
\pgfsetdash{}{0pt}%
\pgfpathmoveto{\pgfqpoint{1.147167in}{2.465878in}}%
\pgfpathlineto{\pgfqpoint{1.180557in}{1.207912in}}%
\pgfusepath{stroke}%
\end{pgfscope}%
\begin{pgfscope}%
\pgfpathrectangle{\pgfqpoint{0.100000in}{0.212622in}}{\pgfqpoint{3.696000in}{3.696000in}}%
\pgfusepath{clip}%
\pgfsetrectcap%
\pgfsetroundjoin%
\pgfsetlinewidth{1.505625pt}%
\definecolor{currentstroke}{rgb}{1.000000,0.000000,0.000000}%
\pgfsetstrokecolor{currentstroke}%
\pgfsetdash{}{0pt}%
\pgfpathmoveto{\pgfqpoint{1.139823in}{2.469054in}}%
\pgfpathlineto{\pgfqpoint{1.180557in}{1.207912in}}%
\pgfusepath{stroke}%
\end{pgfscope}%
\begin{pgfscope}%
\pgfpathrectangle{\pgfqpoint{0.100000in}{0.212622in}}{\pgfqpoint{3.696000in}{3.696000in}}%
\pgfusepath{clip}%
\pgfsetrectcap%
\pgfsetroundjoin%
\pgfsetlinewidth{1.505625pt}%
\definecolor{currentstroke}{rgb}{1.000000,0.000000,0.000000}%
\pgfsetstrokecolor{currentstroke}%
\pgfsetdash{}{0pt}%
\pgfpathmoveto{\pgfqpoint{1.129259in}{2.472262in}}%
\pgfpathlineto{\pgfqpoint{1.166588in}{1.212350in}}%
\pgfusepath{stroke}%
\end{pgfscope}%
\begin{pgfscope}%
\pgfpathrectangle{\pgfqpoint{0.100000in}{0.212622in}}{\pgfqpoint{3.696000in}{3.696000in}}%
\pgfusepath{clip}%
\pgfsetrectcap%
\pgfsetroundjoin%
\pgfsetlinewidth{1.505625pt}%
\definecolor{currentstroke}{rgb}{1.000000,0.000000,0.000000}%
\pgfsetstrokecolor{currentstroke}%
\pgfsetdash{}{0pt}%
\pgfpathmoveto{\pgfqpoint{1.112046in}{2.481907in}}%
\pgfpathlineto{\pgfqpoint{1.152629in}{1.216786in}}%
\pgfusepath{stroke}%
\end{pgfscope}%
\begin{pgfscope}%
\pgfpathrectangle{\pgfqpoint{0.100000in}{0.212622in}}{\pgfqpoint{3.696000in}{3.696000in}}%
\pgfusepath{clip}%
\pgfsetrectcap%
\pgfsetroundjoin%
\pgfsetlinewidth{1.505625pt}%
\definecolor{currentstroke}{rgb}{1.000000,0.000000,0.000000}%
\pgfsetstrokecolor{currentstroke}%
\pgfsetdash{}{0pt}%
\pgfpathmoveto{\pgfqpoint{1.092741in}{2.491948in}}%
\pgfpathlineto{\pgfqpoint{1.124741in}{1.225648in}}%
\pgfusepath{stroke}%
\end{pgfscope}%
\begin{pgfscope}%
\pgfpathrectangle{\pgfqpoint{0.100000in}{0.212622in}}{\pgfqpoint{3.696000in}{3.696000in}}%
\pgfusepath{clip}%
\pgfsetrectcap%
\pgfsetroundjoin%
\pgfsetlinewidth{1.505625pt}%
\definecolor{currentstroke}{rgb}{1.000000,0.000000,0.000000}%
\pgfsetstrokecolor{currentstroke}%
\pgfsetdash{}{0pt}%
\pgfpathmoveto{\pgfqpoint{1.069656in}{2.503877in}}%
\pgfpathlineto{\pgfqpoint{1.110812in}{1.230074in}}%
\pgfusepath{stroke}%
\end{pgfscope}%
\begin{pgfscope}%
\pgfpathrectangle{\pgfqpoint{0.100000in}{0.212622in}}{\pgfqpoint{3.696000in}{3.696000in}}%
\pgfusepath{clip}%
\pgfsetrectcap%
\pgfsetroundjoin%
\pgfsetlinewidth{1.505625pt}%
\definecolor{currentstroke}{rgb}{1.000000,0.000000,0.000000}%
\pgfsetstrokecolor{currentstroke}%
\pgfsetdash{}{0pt}%
\pgfpathmoveto{\pgfqpoint{1.057717in}{2.506458in}}%
\pgfpathlineto{\pgfqpoint{1.096893in}{1.234497in}}%
\pgfusepath{stroke}%
\end{pgfscope}%
\begin{pgfscope}%
\pgfpathrectangle{\pgfqpoint{0.100000in}{0.212622in}}{\pgfqpoint{3.696000in}{3.696000in}}%
\pgfusepath{clip}%
\pgfsetrectcap%
\pgfsetroundjoin%
\pgfsetlinewidth{1.505625pt}%
\definecolor{currentstroke}{rgb}{1.000000,0.000000,0.000000}%
\pgfsetstrokecolor{currentstroke}%
\pgfsetdash{}{0pt}%
\pgfpathmoveto{\pgfqpoint{1.050512in}{2.509772in}}%
\pgfpathlineto{\pgfqpoint{1.082984in}{1.238917in}}%
\pgfusepath{stroke}%
\end{pgfscope}%
\begin{pgfscope}%
\pgfpathrectangle{\pgfqpoint{0.100000in}{0.212622in}}{\pgfqpoint{3.696000in}{3.696000in}}%
\pgfusepath{clip}%
\pgfsetrectcap%
\pgfsetroundjoin%
\pgfsetlinewidth{1.505625pt}%
\definecolor{currentstroke}{rgb}{1.000000,0.000000,0.000000}%
\pgfsetstrokecolor{currentstroke}%
\pgfsetdash{}{0pt}%
\pgfpathmoveto{\pgfqpoint{1.046932in}{2.511268in}}%
\pgfpathlineto{\pgfqpoint{1.082984in}{1.238917in}}%
\pgfusepath{stroke}%
\end{pgfscope}%
\begin{pgfscope}%
\pgfpathrectangle{\pgfqpoint{0.100000in}{0.212622in}}{\pgfqpoint{3.696000in}{3.696000in}}%
\pgfusepath{clip}%
\pgfsetrectcap%
\pgfsetroundjoin%
\pgfsetlinewidth{1.505625pt}%
\definecolor{currentstroke}{rgb}{1.000000,0.000000,0.000000}%
\pgfsetstrokecolor{currentstroke}%
\pgfsetdash{}{0pt}%
\pgfpathmoveto{\pgfqpoint{1.038550in}{2.516092in}}%
\pgfpathlineto{\pgfqpoint{1.069084in}{1.243334in}}%
\pgfusepath{stroke}%
\end{pgfscope}%
\begin{pgfscope}%
\pgfpathrectangle{\pgfqpoint{0.100000in}{0.212622in}}{\pgfqpoint{3.696000in}{3.696000in}}%
\pgfusepath{clip}%
\pgfsetrectcap%
\pgfsetroundjoin%
\pgfsetlinewidth{1.505625pt}%
\definecolor{currentstroke}{rgb}{1.000000,0.000000,0.000000}%
\pgfsetstrokecolor{currentstroke}%
\pgfsetdash{}{0pt}%
\pgfpathmoveto{\pgfqpoint{1.025719in}{2.519746in}}%
\pgfpathlineto{\pgfqpoint{1.069084in}{1.243334in}}%
\pgfusepath{stroke}%
\end{pgfscope}%
\begin{pgfscope}%
\pgfpathrectangle{\pgfqpoint{0.100000in}{0.212622in}}{\pgfqpoint{3.696000in}{3.696000in}}%
\pgfusepath{clip}%
\pgfsetrectcap%
\pgfsetroundjoin%
\pgfsetlinewidth{1.505625pt}%
\definecolor{currentstroke}{rgb}{1.000000,0.000000,0.000000}%
\pgfsetstrokecolor{currentstroke}%
\pgfsetdash{}{0pt}%
\pgfpathmoveto{\pgfqpoint{1.006031in}{2.530981in}}%
\pgfpathlineto{\pgfqpoint{1.041315in}{1.252158in}}%
\pgfusepath{stroke}%
\end{pgfscope}%
\begin{pgfscope}%
\pgfpathrectangle{\pgfqpoint{0.100000in}{0.212622in}}{\pgfqpoint{3.696000in}{3.696000in}}%
\pgfusepath{clip}%
\pgfsetrectcap%
\pgfsetroundjoin%
\pgfsetlinewidth{1.505625pt}%
\definecolor{currentstroke}{rgb}{1.000000,0.000000,0.000000}%
\pgfsetstrokecolor{currentstroke}%
\pgfsetdash{}{0pt}%
\pgfpathmoveto{\pgfqpoint{0.982843in}{2.543464in}}%
\pgfpathlineto{\pgfqpoint{1.027446in}{1.256565in}}%
\pgfusepath{stroke}%
\end{pgfscope}%
\begin{pgfscope}%
\pgfpathrectangle{\pgfqpoint{0.100000in}{0.212622in}}{\pgfqpoint{3.696000in}{3.696000in}}%
\pgfusepath{clip}%
\pgfsetrectcap%
\pgfsetroundjoin%
\pgfsetlinewidth{1.505625pt}%
\definecolor{currentstroke}{rgb}{1.000000,0.000000,0.000000}%
\pgfsetstrokecolor{currentstroke}%
\pgfsetdash{}{0pt}%
\pgfpathmoveto{\pgfqpoint{0.969254in}{2.553983in}}%
\pgfpathlineto{\pgfqpoint{1.013586in}{1.260969in}}%
\pgfusepath{stroke}%
\end{pgfscope}%
\begin{pgfscope}%
\pgfpathrectangle{\pgfqpoint{0.100000in}{0.212622in}}{\pgfqpoint{3.696000in}{3.696000in}}%
\pgfusepath{clip}%
\pgfsetrectcap%
\pgfsetroundjoin%
\pgfsetlinewidth{1.505625pt}%
\definecolor{currentstroke}{rgb}{1.000000,0.000000,0.000000}%
\pgfsetstrokecolor{currentstroke}%
\pgfsetdash{}{0pt}%
\pgfpathmoveto{\pgfqpoint{0.963187in}{2.557802in}}%
\pgfpathlineto{\pgfqpoint{0.999736in}{1.265371in}}%
\pgfusepath{stroke}%
\end{pgfscope}%
\begin{pgfscope}%
\pgfpathrectangle{\pgfqpoint{0.100000in}{0.212622in}}{\pgfqpoint{3.696000in}{3.696000in}}%
\pgfusepath{clip}%
\pgfsetrectcap%
\pgfsetroundjoin%
\pgfsetlinewidth{1.505625pt}%
\definecolor{currentstroke}{rgb}{1.000000,0.000000,0.000000}%
\pgfsetstrokecolor{currentstroke}%
\pgfsetdash{}{0pt}%
\pgfpathmoveto{\pgfqpoint{0.959419in}{2.561194in}}%
\pgfpathlineto{\pgfqpoint{0.999736in}{1.265371in}}%
\pgfusepath{stroke}%
\end{pgfscope}%
\begin{pgfscope}%
\pgfpathrectangle{\pgfqpoint{0.100000in}{0.212622in}}{\pgfqpoint{3.696000in}{3.696000in}}%
\pgfusepath{clip}%
\pgfsetrectcap%
\pgfsetroundjoin%
\pgfsetlinewidth{1.505625pt}%
\definecolor{currentstroke}{rgb}{1.000000,0.000000,0.000000}%
\pgfsetstrokecolor{currentstroke}%
\pgfsetdash{}{0pt}%
\pgfpathmoveto{\pgfqpoint{0.957570in}{2.562510in}}%
\pgfpathlineto{\pgfqpoint{0.999736in}{1.265371in}}%
\pgfusepath{stroke}%
\end{pgfscope}%
\begin{pgfscope}%
\pgfpathrectangle{\pgfqpoint{0.100000in}{0.212622in}}{\pgfqpoint{3.696000in}{3.696000in}}%
\pgfusepath{clip}%
\pgfsetrectcap%
\pgfsetroundjoin%
\pgfsetlinewidth{1.505625pt}%
\definecolor{currentstroke}{rgb}{1.000000,0.000000,0.000000}%
\pgfsetstrokecolor{currentstroke}%
\pgfsetdash{}{0pt}%
\pgfpathmoveto{\pgfqpoint{0.954018in}{2.565493in}}%
\pgfpathlineto{\pgfqpoint{0.985896in}{1.269768in}}%
\pgfusepath{stroke}%
\end{pgfscope}%
\begin{pgfscope}%
\pgfpathrectangle{\pgfqpoint{0.100000in}{0.212622in}}{\pgfqpoint{3.696000in}{3.696000in}}%
\pgfusepath{clip}%
\pgfsetrectcap%
\pgfsetroundjoin%
\pgfsetlinewidth{1.505625pt}%
\definecolor{currentstroke}{rgb}{1.000000,0.000000,0.000000}%
\pgfsetstrokecolor{currentstroke}%
\pgfsetdash{}{0pt}%
\pgfpathmoveto{\pgfqpoint{0.946902in}{2.568664in}}%
\pgfpathlineto{\pgfqpoint{0.985896in}{1.269768in}}%
\pgfusepath{stroke}%
\end{pgfscope}%
\begin{pgfscope}%
\pgfpathrectangle{\pgfqpoint{0.100000in}{0.212622in}}{\pgfqpoint{3.696000in}{3.696000in}}%
\pgfusepath{clip}%
\pgfsetrectcap%
\pgfsetroundjoin%
\pgfsetlinewidth{1.505625pt}%
\definecolor{currentstroke}{rgb}{1.000000,0.000000,0.000000}%
\pgfsetstrokecolor{currentstroke}%
\pgfsetdash{}{0pt}%
\pgfpathmoveto{\pgfqpoint{0.936273in}{2.575289in}}%
\pgfpathlineto{\pgfqpoint{0.972065in}{1.274163in}}%
\pgfusepath{stroke}%
\end{pgfscope}%
\begin{pgfscope}%
\pgfpathrectangle{\pgfqpoint{0.100000in}{0.212622in}}{\pgfqpoint{3.696000in}{3.696000in}}%
\pgfusepath{clip}%
\pgfsetrectcap%
\pgfsetroundjoin%
\pgfsetlinewidth{1.505625pt}%
\definecolor{currentstroke}{rgb}{1.000000,0.000000,0.000000}%
\pgfsetstrokecolor{currentstroke}%
\pgfsetdash{}{0pt}%
\pgfpathmoveto{\pgfqpoint{0.923220in}{2.585337in}}%
\pgfpathlineto{\pgfqpoint{0.958245in}{1.278555in}}%
\pgfusepath{stroke}%
\end{pgfscope}%
\begin{pgfscope}%
\pgfpathrectangle{\pgfqpoint{0.100000in}{0.212622in}}{\pgfqpoint{3.696000in}{3.696000in}}%
\pgfusepath{clip}%
\pgfsetrectcap%
\pgfsetroundjoin%
\pgfsetlinewidth{1.505625pt}%
\definecolor{currentstroke}{rgb}{1.000000,0.000000,0.000000}%
\pgfsetstrokecolor{currentstroke}%
\pgfsetdash{}{0pt}%
\pgfpathmoveto{\pgfqpoint{0.907671in}{2.594636in}}%
\pgfpathlineto{\pgfqpoint{0.944434in}{1.282944in}}%
\pgfusepath{stroke}%
\end{pgfscope}%
\begin{pgfscope}%
\pgfpathrectangle{\pgfqpoint{0.100000in}{0.212622in}}{\pgfqpoint{3.696000in}{3.696000in}}%
\pgfusepath{clip}%
\pgfsetrectcap%
\pgfsetroundjoin%
\pgfsetlinewidth{1.505625pt}%
\definecolor{currentstroke}{rgb}{1.000000,0.000000,0.000000}%
\pgfsetstrokecolor{currentstroke}%
\pgfsetdash{}{0pt}%
\pgfpathmoveto{\pgfqpoint{0.898638in}{2.601059in}}%
\pgfpathlineto{\pgfqpoint{0.930633in}{1.287329in}}%
\pgfusepath{stroke}%
\end{pgfscope}%
\begin{pgfscope}%
\pgfpathrectangle{\pgfqpoint{0.100000in}{0.212622in}}{\pgfqpoint{3.696000in}{3.696000in}}%
\pgfusepath{clip}%
\pgfsetrectcap%
\pgfsetroundjoin%
\pgfsetlinewidth{1.505625pt}%
\definecolor{currentstroke}{rgb}{1.000000,0.000000,0.000000}%
\pgfsetstrokecolor{currentstroke}%
\pgfsetdash{}{0pt}%
\pgfpathmoveto{\pgfqpoint{0.893712in}{2.603246in}}%
\pgfpathlineto{\pgfqpoint{0.930633in}{1.287329in}}%
\pgfusepath{stroke}%
\end{pgfscope}%
\begin{pgfscope}%
\pgfpathrectangle{\pgfqpoint{0.100000in}{0.212622in}}{\pgfqpoint{3.696000in}{3.696000in}}%
\pgfusepath{clip}%
\pgfsetrectcap%
\pgfsetroundjoin%
\pgfsetlinewidth{1.505625pt}%
\definecolor{currentstroke}{rgb}{1.000000,0.000000,0.000000}%
\pgfsetstrokecolor{currentstroke}%
\pgfsetdash{}{0pt}%
\pgfpathmoveto{\pgfqpoint{0.884801in}{2.609154in}}%
\pgfpathlineto{\pgfqpoint{0.926292in}{1.299818in}}%
\pgfusepath{stroke}%
\end{pgfscope}%
\begin{pgfscope}%
\pgfpathrectangle{\pgfqpoint{0.100000in}{0.212622in}}{\pgfqpoint{3.696000in}{3.696000in}}%
\pgfusepath{clip}%
\pgfsetrectcap%
\pgfsetroundjoin%
\pgfsetlinewidth{1.505625pt}%
\definecolor{currentstroke}{rgb}{1.000000,0.000000,0.000000}%
\pgfsetstrokecolor{currentstroke}%
\pgfsetdash{}{0pt}%
\pgfpathmoveto{\pgfqpoint{0.873621in}{2.613784in}}%
\pgfpathlineto{\pgfqpoint{0.926292in}{1.299818in}}%
\pgfusepath{stroke}%
\end{pgfscope}%
\begin{pgfscope}%
\pgfpathrectangle{\pgfqpoint{0.100000in}{0.212622in}}{\pgfqpoint{3.696000in}{3.696000in}}%
\pgfusepath{clip}%
\pgfsetrectcap%
\pgfsetroundjoin%
\pgfsetlinewidth{1.505625pt}%
\definecolor{currentstroke}{rgb}{1.000000,0.000000,0.000000}%
\pgfsetstrokecolor{currentstroke}%
\pgfsetdash{}{0pt}%
\pgfpathmoveto{\pgfqpoint{0.858322in}{2.623683in}}%
\pgfpathlineto{\pgfqpoint{0.926292in}{1.299818in}}%
\pgfusepath{stroke}%
\end{pgfscope}%
\begin{pgfscope}%
\pgfpathrectangle{\pgfqpoint{0.100000in}{0.212622in}}{\pgfqpoint{3.696000in}{3.696000in}}%
\pgfusepath{clip}%
\pgfsetrectcap%
\pgfsetroundjoin%
\pgfsetlinewidth{1.505625pt}%
\definecolor{currentstroke}{rgb}{1.000000,0.000000,0.000000}%
\pgfsetstrokecolor{currentstroke}%
\pgfsetdash{}{0pt}%
\pgfpathmoveto{\pgfqpoint{0.842016in}{2.632033in}}%
\pgfpathlineto{\pgfqpoint{0.926292in}{1.299818in}}%
\pgfusepath{stroke}%
\end{pgfscope}%
\begin{pgfscope}%
\pgfpathrectangle{\pgfqpoint{0.100000in}{0.212622in}}{\pgfqpoint{3.696000in}{3.696000in}}%
\pgfusepath{clip}%
\pgfsetrectcap%
\pgfsetroundjoin%
\pgfsetlinewidth{1.505625pt}%
\definecolor{currentstroke}{rgb}{1.000000,0.000000,0.000000}%
\pgfsetstrokecolor{currentstroke}%
\pgfsetdash{}{0pt}%
\pgfpathmoveto{\pgfqpoint{0.822793in}{2.643856in}}%
\pgfpathlineto{\pgfqpoint{0.926292in}{1.299818in}}%
\pgfusepath{stroke}%
\end{pgfscope}%
\begin{pgfscope}%
\pgfpathrectangle{\pgfqpoint{0.100000in}{0.212622in}}{\pgfqpoint{3.696000in}{3.696000in}}%
\pgfusepath{clip}%
\pgfsetrectcap%
\pgfsetroundjoin%
\pgfsetlinewidth{1.505625pt}%
\definecolor{currentstroke}{rgb}{1.000000,0.000000,0.000000}%
\pgfsetstrokecolor{currentstroke}%
\pgfsetdash{}{0pt}%
\pgfpathmoveto{\pgfqpoint{0.813048in}{2.650307in}}%
\pgfpathlineto{\pgfqpoint{0.926292in}{1.299818in}}%
\pgfusepath{stroke}%
\end{pgfscope}%
\begin{pgfscope}%
\pgfpathrectangle{\pgfqpoint{0.100000in}{0.212622in}}{\pgfqpoint{3.696000in}{3.696000in}}%
\pgfusepath{clip}%
\pgfsetrectcap%
\pgfsetroundjoin%
\pgfsetlinewidth{1.505625pt}%
\definecolor{currentstroke}{rgb}{1.000000,0.000000,0.000000}%
\pgfsetstrokecolor{currentstroke}%
\pgfsetdash{}{0pt}%
\pgfpathmoveto{\pgfqpoint{0.807586in}{2.654794in}}%
\pgfpathlineto{\pgfqpoint{0.926292in}{1.299818in}}%
\pgfusepath{stroke}%
\end{pgfscope}%
\begin{pgfscope}%
\pgfpathrectangle{\pgfqpoint{0.100000in}{0.212622in}}{\pgfqpoint{3.696000in}{3.696000in}}%
\pgfusepath{clip}%
\pgfsetrectcap%
\pgfsetroundjoin%
\pgfsetlinewidth{1.505625pt}%
\definecolor{currentstroke}{rgb}{1.000000,0.000000,0.000000}%
\pgfsetstrokecolor{currentstroke}%
\pgfsetdash{}{0pt}%
\pgfpathmoveto{\pgfqpoint{0.804874in}{2.656435in}}%
\pgfpathlineto{\pgfqpoint{0.926292in}{1.299818in}}%
\pgfusepath{stroke}%
\end{pgfscope}%
\begin{pgfscope}%
\pgfpathrectangle{\pgfqpoint{0.100000in}{0.212622in}}{\pgfqpoint{3.696000in}{3.696000in}}%
\pgfusepath{clip}%
\pgfsetrectcap%
\pgfsetroundjoin%
\pgfsetlinewidth{1.505625pt}%
\definecolor{currentstroke}{rgb}{1.000000,0.000000,0.000000}%
\pgfsetstrokecolor{currentstroke}%
\pgfsetdash{}{0pt}%
\pgfpathmoveto{\pgfqpoint{0.803139in}{2.657988in}}%
\pgfpathlineto{\pgfqpoint{0.926292in}{1.299818in}}%
\pgfusepath{stroke}%
\end{pgfscope}%
\begin{pgfscope}%
\pgfpathrectangle{\pgfqpoint{0.100000in}{0.212622in}}{\pgfqpoint{3.696000in}{3.696000in}}%
\pgfusepath{clip}%
\pgfsetrectcap%
\pgfsetroundjoin%
\pgfsetlinewidth{1.505625pt}%
\definecolor{currentstroke}{rgb}{1.000000,0.000000,0.000000}%
\pgfsetstrokecolor{currentstroke}%
\pgfsetdash{}{0pt}%
\pgfpathmoveto{\pgfqpoint{0.802304in}{2.658734in}}%
\pgfpathlineto{\pgfqpoint{0.926292in}{1.299818in}}%
\pgfusepath{stroke}%
\end{pgfscope}%
\begin{pgfscope}%
\pgfpathrectangle{\pgfqpoint{0.100000in}{0.212622in}}{\pgfqpoint{3.696000in}{3.696000in}}%
\pgfusepath{clip}%
\pgfsetrectcap%
\pgfsetroundjoin%
\pgfsetlinewidth{1.505625pt}%
\definecolor{currentstroke}{rgb}{1.000000,0.000000,0.000000}%
\pgfsetstrokecolor{currentstroke}%
\pgfsetdash{}{0pt}%
\pgfpathmoveto{\pgfqpoint{0.796931in}{2.662279in}}%
\pgfpathlineto{\pgfqpoint{0.926292in}{1.299818in}}%
\pgfusepath{stroke}%
\end{pgfscope}%
\begin{pgfscope}%
\pgfpathrectangle{\pgfqpoint{0.100000in}{0.212622in}}{\pgfqpoint{3.696000in}{3.696000in}}%
\pgfusepath{clip}%
\pgfsetrectcap%
\pgfsetroundjoin%
\pgfsetlinewidth{1.505625pt}%
\definecolor{currentstroke}{rgb}{1.000000,0.000000,0.000000}%
\pgfsetstrokecolor{currentstroke}%
\pgfsetdash{}{0pt}%
\pgfpathmoveto{\pgfqpoint{0.790396in}{2.668025in}}%
\pgfpathlineto{\pgfqpoint{0.926292in}{1.299818in}}%
\pgfusepath{stroke}%
\end{pgfscope}%
\begin{pgfscope}%
\pgfpathrectangle{\pgfqpoint{0.100000in}{0.212622in}}{\pgfqpoint{3.696000in}{3.696000in}}%
\pgfusepath{clip}%
\pgfsetrectcap%
\pgfsetroundjoin%
\pgfsetlinewidth{1.505625pt}%
\definecolor{currentstroke}{rgb}{1.000000,0.000000,0.000000}%
\pgfsetstrokecolor{currentstroke}%
\pgfsetdash{}{0pt}%
\pgfpathmoveto{\pgfqpoint{0.786008in}{2.670435in}}%
\pgfpathlineto{\pgfqpoint{0.926292in}{1.299818in}}%
\pgfusepath{stroke}%
\end{pgfscope}%
\begin{pgfscope}%
\pgfpathrectangle{\pgfqpoint{0.100000in}{0.212622in}}{\pgfqpoint{3.696000in}{3.696000in}}%
\pgfusepath{clip}%
\pgfsetrectcap%
\pgfsetroundjoin%
\pgfsetlinewidth{1.505625pt}%
\definecolor{currentstroke}{rgb}{1.000000,0.000000,0.000000}%
\pgfsetstrokecolor{currentstroke}%
\pgfsetdash{}{0pt}%
\pgfpathmoveto{\pgfqpoint{0.776362in}{2.676748in}}%
\pgfpathlineto{\pgfqpoint{0.926292in}{1.299818in}}%
\pgfusepath{stroke}%
\end{pgfscope}%
\begin{pgfscope}%
\pgfpathrectangle{\pgfqpoint{0.100000in}{0.212622in}}{\pgfqpoint{3.696000in}{3.696000in}}%
\pgfusepath{clip}%
\pgfsetrectcap%
\pgfsetroundjoin%
\pgfsetlinewidth{1.505625pt}%
\definecolor{currentstroke}{rgb}{1.000000,0.000000,0.000000}%
\pgfsetstrokecolor{currentstroke}%
\pgfsetdash{}{0pt}%
\pgfpathmoveto{\pgfqpoint{0.760543in}{2.687532in}}%
\pgfpathlineto{\pgfqpoint{0.926292in}{1.299818in}}%
\pgfusepath{stroke}%
\end{pgfscope}%
\begin{pgfscope}%
\pgfpathrectangle{\pgfqpoint{0.100000in}{0.212622in}}{\pgfqpoint{3.696000in}{3.696000in}}%
\pgfusepath{clip}%
\pgfsetrectcap%
\pgfsetroundjoin%
\pgfsetlinewidth{1.505625pt}%
\definecolor{currentstroke}{rgb}{1.000000,0.000000,0.000000}%
\pgfsetstrokecolor{currentstroke}%
\pgfsetdash{}{0pt}%
\pgfpathmoveto{\pgfqpoint{0.751516in}{2.684448in}}%
\pgfpathlineto{\pgfqpoint{0.926292in}{1.299818in}}%
\pgfusepath{stroke}%
\end{pgfscope}%
\begin{pgfscope}%
\pgfpathrectangle{\pgfqpoint{0.100000in}{0.212622in}}{\pgfqpoint{3.696000in}{3.696000in}}%
\pgfusepath{clip}%
\pgfsetrectcap%
\pgfsetroundjoin%
\pgfsetlinewidth{1.505625pt}%
\definecolor{currentstroke}{rgb}{1.000000,0.000000,0.000000}%
\pgfsetstrokecolor{currentstroke}%
\pgfsetdash{}{0pt}%
\pgfpathmoveto{\pgfqpoint{0.737995in}{2.693019in}}%
\pgfpathlineto{\pgfqpoint{0.926292in}{1.299818in}}%
\pgfusepath{stroke}%
\end{pgfscope}%
\begin{pgfscope}%
\pgfpathrectangle{\pgfqpoint{0.100000in}{0.212622in}}{\pgfqpoint{3.696000in}{3.696000in}}%
\pgfusepath{clip}%
\pgfsetrectcap%
\pgfsetroundjoin%
\pgfsetlinewidth{1.505625pt}%
\definecolor{currentstroke}{rgb}{1.000000,0.000000,0.000000}%
\pgfsetstrokecolor{currentstroke}%
\pgfsetdash{}{0pt}%
\pgfpathmoveto{\pgfqpoint{0.724919in}{2.691199in}}%
\pgfpathlineto{\pgfqpoint{0.926292in}{1.299818in}}%
\pgfusepath{stroke}%
\end{pgfscope}%
\begin{pgfscope}%
\pgfpathrectangle{\pgfqpoint{0.100000in}{0.212622in}}{\pgfqpoint{3.696000in}{3.696000in}}%
\pgfusepath{clip}%
\pgfsetrectcap%
\pgfsetroundjoin%
\pgfsetlinewidth{1.505625pt}%
\definecolor{currentstroke}{rgb}{1.000000,0.000000,0.000000}%
\pgfsetstrokecolor{currentstroke}%
\pgfsetdash{}{0pt}%
\pgfpathmoveto{\pgfqpoint{0.705888in}{2.706198in}}%
\pgfpathlineto{\pgfqpoint{0.926292in}{1.299818in}}%
\pgfusepath{stroke}%
\end{pgfscope}%
\begin{pgfscope}%
\pgfpathrectangle{\pgfqpoint{0.100000in}{0.212622in}}{\pgfqpoint{3.696000in}{3.696000in}}%
\pgfusepath{clip}%
\pgfsetrectcap%
\pgfsetroundjoin%
\pgfsetlinewidth{1.505625pt}%
\definecolor{currentstroke}{rgb}{1.000000,0.000000,0.000000}%
\pgfsetstrokecolor{currentstroke}%
\pgfsetdash{}{0pt}%
\pgfpathmoveto{\pgfqpoint{0.688958in}{2.703228in}}%
\pgfpathlineto{\pgfqpoint{0.926292in}{1.299818in}}%
\pgfusepath{stroke}%
\end{pgfscope}%
\begin{pgfscope}%
\pgfpathrectangle{\pgfqpoint{0.100000in}{0.212622in}}{\pgfqpoint{3.696000in}{3.696000in}}%
\pgfusepath{clip}%
\pgfsetrectcap%
\pgfsetroundjoin%
\pgfsetlinewidth{1.505625pt}%
\definecolor{currentstroke}{rgb}{1.000000,0.000000,0.000000}%
\pgfsetstrokecolor{currentstroke}%
\pgfsetdash{}{0pt}%
\pgfpathmoveto{\pgfqpoint{0.668713in}{2.699677in}}%
\pgfpathlineto{\pgfqpoint{0.926292in}{1.299818in}}%
\pgfusepath{stroke}%
\end{pgfscope}%
\begin{pgfscope}%
\pgfpathrectangle{\pgfqpoint{0.100000in}{0.212622in}}{\pgfqpoint{3.696000in}{3.696000in}}%
\pgfusepath{clip}%
\pgfsetbuttcap%
\pgfsetroundjoin%
\definecolor{currentfill}{rgb}{0.121569,0.466667,0.705882}%
\pgfsetfillcolor{currentfill}%
\pgfsetfillopacity{0.300000}%
\pgfsetlinewidth{1.003750pt}%
\definecolor{currentstroke}{rgb}{0.121569,0.466667,0.705882}%
\pgfsetstrokecolor{currentstroke}%
\pgfsetstrokeopacity{0.300000}%
\pgfsetdash{}{0pt}%
\pgfpathmoveto{\pgfqpoint{1.672363in}{2.348955in}}%
\pgfpathcurveto{\pgfqpoint{1.680599in}{2.348955in}}{\pgfqpoint{1.688499in}{2.352227in}}{\pgfqpoint{1.694323in}{2.358051in}}%
\pgfpathcurveto{\pgfqpoint{1.700147in}{2.363875in}}{\pgfqpoint{1.703419in}{2.371775in}}{\pgfqpoint{1.703419in}{2.380011in}}%
\pgfpathcurveto{\pgfqpoint{1.703419in}{2.388248in}}{\pgfqpoint{1.700147in}{2.396148in}}{\pgfqpoint{1.694323in}{2.401972in}}%
\pgfpathcurveto{\pgfqpoint{1.688499in}{2.407795in}}{\pgfqpoint{1.680599in}{2.411068in}}{\pgfqpoint{1.672363in}{2.411068in}}%
\pgfpathcurveto{\pgfqpoint{1.664127in}{2.411068in}}{\pgfqpoint{1.656227in}{2.407795in}}{\pgfqpoint{1.650403in}{2.401972in}}%
\pgfpathcurveto{\pgfqpoint{1.644579in}{2.396148in}}{\pgfqpoint{1.641306in}{2.388248in}}{\pgfqpoint{1.641306in}{2.380011in}}%
\pgfpathcurveto{\pgfqpoint{1.641306in}{2.371775in}}{\pgfqpoint{1.644579in}{2.363875in}}{\pgfqpoint{1.650403in}{2.358051in}}%
\pgfpathcurveto{\pgfqpoint{1.656227in}{2.352227in}}{\pgfqpoint{1.664127in}{2.348955in}}{\pgfqpoint{1.672363in}{2.348955in}}%
\pgfpathclose%
\pgfusepath{stroke,fill}%
\end{pgfscope}%
\begin{pgfscope}%
\pgfpathrectangle{\pgfqpoint{0.100000in}{0.212622in}}{\pgfqpoint{3.696000in}{3.696000in}}%
\pgfusepath{clip}%
\pgfsetbuttcap%
\pgfsetroundjoin%
\definecolor{currentfill}{rgb}{0.121569,0.466667,0.705882}%
\pgfsetfillcolor{currentfill}%
\pgfsetfillopacity{0.300103}%
\pgfsetlinewidth{1.003750pt}%
\definecolor{currentstroke}{rgb}{0.121569,0.466667,0.705882}%
\pgfsetstrokecolor{currentstroke}%
\pgfsetstrokeopacity{0.300103}%
\pgfsetdash{}{0pt}%
\pgfpathmoveto{\pgfqpoint{1.661802in}{2.347529in}}%
\pgfpathcurveto{\pgfqpoint{1.670038in}{2.347529in}}{\pgfqpoint{1.677938in}{2.350802in}}{\pgfqpoint{1.683762in}{2.356626in}}%
\pgfpathcurveto{\pgfqpoint{1.689586in}{2.362450in}}{\pgfqpoint{1.692858in}{2.370350in}}{\pgfqpoint{1.692858in}{2.378586in}}%
\pgfpathcurveto{\pgfqpoint{1.692858in}{2.386822in}}{\pgfqpoint{1.689586in}{2.394722in}}{\pgfqpoint{1.683762in}{2.400546in}}%
\pgfpathcurveto{\pgfqpoint{1.677938in}{2.406370in}}{\pgfqpoint{1.670038in}{2.409642in}}{\pgfqpoint{1.661802in}{2.409642in}}%
\pgfpathcurveto{\pgfqpoint{1.653566in}{2.409642in}}{\pgfqpoint{1.645666in}{2.406370in}}{\pgfqpoint{1.639842in}{2.400546in}}%
\pgfpathcurveto{\pgfqpoint{1.634018in}{2.394722in}}{\pgfqpoint{1.630745in}{2.386822in}}{\pgfqpoint{1.630745in}{2.378586in}}%
\pgfpathcurveto{\pgfqpoint{1.630745in}{2.370350in}}{\pgfqpoint{1.634018in}{2.362450in}}{\pgfqpoint{1.639842in}{2.356626in}}%
\pgfpathcurveto{\pgfqpoint{1.645666in}{2.350802in}}{\pgfqpoint{1.653566in}{2.347529in}}{\pgfqpoint{1.661802in}{2.347529in}}%
\pgfpathclose%
\pgfusepath{stroke,fill}%
\end{pgfscope}%
\begin{pgfscope}%
\pgfpathrectangle{\pgfqpoint{0.100000in}{0.212622in}}{\pgfqpoint{3.696000in}{3.696000in}}%
\pgfusepath{clip}%
\pgfsetbuttcap%
\pgfsetroundjoin%
\definecolor{currentfill}{rgb}{0.121569,0.466667,0.705882}%
\pgfsetfillcolor{currentfill}%
\pgfsetfillopacity{0.300287}%
\pgfsetlinewidth{1.003750pt}%
\definecolor{currentstroke}{rgb}{0.121569,0.466667,0.705882}%
\pgfsetstrokecolor{currentstroke}%
\pgfsetstrokeopacity{0.300287}%
\pgfsetdash{}{0pt}%
\pgfpathmoveto{\pgfqpoint{1.654583in}{2.344973in}}%
\pgfpathcurveto{\pgfqpoint{1.662819in}{2.344973in}}{\pgfqpoint{1.670719in}{2.348246in}}{\pgfqpoint{1.676543in}{2.354070in}}%
\pgfpathcurveto{\pgfqpoint{1.682367in}{2.359894in}}{\pgfqpoint{1.685639in}{2.367794in}}{\pgfqpoint{1.685639in}{2.376030in}}%
\pgfpathcurveto{\pgfqpoint{1.685639in}{2.384266in}}{\pgfqpoint{1.682367in}{2.392166in}}{\pgfqpoint{1.676543in}{2.397990in}}%
\pgfpathcurveto{\pgfqpoint{1.670719in}{2.403814in}}{\pgfqpoint{1.662819in}{2.407086in}}{\pgfqpoint{1.654583in}{2.407086in}}%
\pgfpathcurveto{\pgfqpoint{1.646347in}{2.407086in}}{\pgfqpoint{1.638447in}{2.403814in}}{\pgfqpoint{1.632623in}{2.397990in}}%
\pgfpathcurveto{\pgfqpoint{1.626799in}{2.392166in}}{\pgfqpoint{1.623526in}{2.384266in}}{\pgfqpoint{1.623526in}{2.376030in}}%
\pgfpathcurveto{\pgfqpoint{1.623526in}{2.367794in}}{\pgfqpoint{1.626799in}{2.359894in}}{\pgfqpoint{1.632623in}{2.354070in}}%
\pgfpathcurveto{\pgfqpoint{1.638447in}{2.348246in}}{\pgfqpoint{1.646347in}{2.344973in}}{\pgfqpoint{1.654583in}{2.344973in}}%
\pgfpathclose%
\pgfusepath{stroke,fill}%
\end{pgfscope}%
\begin{pgfscope}%
\pgfpathrectangle{\pgfqpoint{0.100000in}{0.212622in}}{\pgfqpoint{3.696000in}{3.696000in}}%
\pgfusepath{clip}%
\pgfsetbuttcap%
\pgfsetroundjoin%
\definecolor{currentfill}{rgb}{0.121569,0.466667,0.705882}%
\pgfsetfillcolor{currentfill}%
\pgfsetfillopacity{0.300602}%
\pgfsetlinewidth{1.003750pt}%
\definecolor{currentstroke}{rgb}{0.121569,0.466667,0.705882}%
\pgfsetstrokecolor{currentstroke}%
\pgfsetstrokeopacity{0.300602}%
\pgfsetdash{}{0pt}%
\pgfpathmoveto{\pgfqpoint{1.649688in}{2.343457in}}%
\pgfpathcurveto{\pgfqpoint{1.657924in}{2.343457in}}{\pgfqpoint{1.665824in}{2.346729in}}{\pgfqpoint{1.671648in}{2.352553in}}%
\pgfpathcurveto{\pgfqpoint{1.677472in}{2.358377in}}{\pgfqpoint{1.680744in}{2.366277in}}{\pgfqpoint{1.680744in}{2.374513in}}%
\pgfpathcurveto{\pgfqpoint{1.680744in}{2.382749in}}{\pgfqpoint{1.677472in}{2.390649in}}{\pgfqpoint{1.671648in}{2.396473in}}%
\pgfpathcurveto{\pgfqpoint{1.665824in}{2.402297in}}{\pgfqpoint{1.657924in}{2.405570in}}{\pgfqpoint{1.649688in}{2.405570in}}%
\pgfpathcurveto{\pgfqpoint{1.641451in}{2.405570in}}{\pgfqpoint{1.633551in}{2.402297in}}{\pgfqpoint{1.627727in}{2.396473in}}%
\pgfpathcurveto{\pgfqpoint{1.621904in}{2.390649in}}{\pgfqpoint{1.618631in}{2.382749in}}{\pgfqpoint{1.618631in}{2.374513in}}%
\pgfpathcurveto{\pgfqpoint{1.618631in}{2.366277in}}{\pgfqpoint{1.621904in}{2.358377in}}{\pgfqpoint{1.627727in}{2.352553in}}%
\pgfpathcurveto{\pgfqpoint{1.633551in}{2.346729in}}{\pgfqpoint{1.641451in}{2.343457in}}{\pgfqpoint{1.649688in}{2.343457in}}%
\pgfpathclose%
\pgfusepath{stroke,fill}%
\end{pgfscope}%
\begin{pgfscope}%
\pgfpathrectangle{\pgfqpoint{0.100000in}{0.212622in}}{\pgfqpoint{3.696000in}{3.696000in}}%
\pgfusepath{clip}%
\pgfsetbuttcap%
\pgfsetroundjoin%
\definecolor{currentfill}{rgb}{0.121569,0.466667,0.705882}%
\pgfsetfillcolor{currentfill}%
\pgfsetfillopacity{0.300747}%
\pgfsetlinewidth{1.003750pt}%
\definecolor{currentstroke}{rgb}{0.121569,0.466667,0.705882}%
\pgfsetstrokecolor{currentstroke}%
\pgfsetstrokeopacity{0.300747}%
\pgfsetdash{}{0pt}%
\pgfpathmoveto{\pgfqpoint{1.648292in}{2.342373in}}%
\pgfpathcurveto{\pgfqpoint{1.656529in}{2.342373in}}{\pgfqpoint{1.664429in}{2.345645in}}{\pgfqpoint{1.670253in}{2.351469in}}%
\pgfpathcurveto{\pgfqpoint{1.676077in}{2.357293in}}{\pgfqpoint{1.679349in}{2.365193in}}{\pgfqpoint{1.679349in}{2.373429in}}%
\pgfpathcurveto{\pgfqpoint{1.679349in}{2.381666in}}{\pgfqpoint{1.676077in}{2.389566in}}{\pgfqpoint{1.670253in}{2.395390in}}%
\pgfpathcurveto{\pgfqpoint{1.664429in}{2.401214in}}{\pgfqpoint{1.656529in}{2.404486in}}{\pgfqpoint{1.648292in}{2.404486in}}%
\pgfpathcurveto{\pgfqpoint{1.640056in}{2.404486in}}{\pgfqpoint{1.632156in}{2.401214in}}{\pgfqpoint{1.626332in}{2.395390in}}%
\pgfpathcurveto{\pgfqpoint{1.620508in}{2.389566in}}{\pgfqpoint{1.617236in}{2.381666in}}{\pgfqpoint{1.617236in}{2.373429in}}%
\pgfpathcurveto{\pgfqpoint{1.617236in}{2.365193in}}{\pgfqpoint{1.620508in}{2.357293in}}{\pgfqpoint{1.626332in}{2.351469in}}%
\pgfpathcurveto{\pgfqpoint{1.632156in}{2.345645in}}{\pgfqpoint{1.640056in}{2.342373in}}{\pgfqpoint{1.648292in}{2.342373in}}%
\pgfpathclose%
\pgfusepath{stroke,fill}%
\end{pgfscope}%
\begin{pgfscope}%
\pgfpathrectangle{\pgfqpoint{0.100000in}{0.212622in}}{\pgfqpoint{3.696000in}{3.696000in}}%
\pgfusepath{clip}%
\pgfsetbuttcap%
\pgfsetroundjoin%
\definecolor{currentfill}{rgb}{0.121569,0.466667,0.705882}%
\pgfsetfillcolor{currentfill}%
\pgfsetfillopacity{0.300828}%
\pgfsetlinewidth{1.003750pt}%
\definecolor{currentstroke}{rgb}{0.121569,0.466667,0.705882}%
\pgfsetstrokecolor{currentstroke}%
\pgfsetstrokeopacity{0.300828}%
\pgfsetdash{}{0pt}%
\pgfpathmoveto{\pgfqpoint{1.684693in}{2.349553in}}%
\pgfpathcurveto{\pgfqpoint{1.692930in}{2.349553in}}{\pgfqpoint{1.700830in}{2.352825in}}{\pgfqpoint{1.706654in}{2.358649in}}%
\pgfpathcurveto{\pgfqpoint{1.712478in}{2.364473in}}{\pgfqpoint{1.715750in}{2.372373in}}{\pgfqpoint{1.715750in}{2.380609in}}%
\pgfpathcurveto{\pgfqpoint{1.715750in}{2.388845in}}{\pgfqpoint{1.712478in}{2.396745in}}{\pgfqpoint{1.706654in}{2.402569in}}%
\pgfpathcurveto{\pgfqpoint{1.700830in}{2.408393in}}{\pgfqpoint{1.692930in}{2.411666in}}{\pgfqpoint{1.684693in}{2.411666in}}%
\pgfpathcurveto{\pgfqpoint{1.676457in}{2.411666in}}{\pgfqpoint{1.668557in}{2.408393in}}{\pgfqpoint{1.662733in}{2.402569in}}%
\pgfpathcurveto{\pgfqpoint{1.656909in}{2.396745in}}{\pgfqpoint{1.653637in}{2.388845in}}{\pgfqpoint{1.653637in}{2.380609in}}%
\pgfpathcurveto{\pgfqpoint{1.653637in}{2.372373in}}{\pgfqpoint{1.656909in}{2.364473in}}{\pgfqpoint{1.662733in}{2.358649in}}%
\pgfpathcurveto{\pgfqpoint{1.668557in}{2.352825in}}{\pgfqpoint{1.676457in}{2.349553in}}{\pgfqpoint{1.684693in}{2.349553in}}%
\pgfpathclose%
\pgfusepath{stroke,fill}%
\end{pgfscope}%
\begin{pgfscope}%
\pgfpathrectangle{\pgfqpoint{0.100000in}{0.212622in}}{\pgfqpoint{3.696000in}{3.696000in}}%
\pgfusepath{clip}%
\pgfsetbuttcap%
\pgfsetroundjoin%
\definecolor{currentfill}{rgb}{0.121569,0.466667,0.705882}%
\pgfsetfillcolor{currentfill}%
\pgfsetfillopacity{0.301130}%
\pgfsetlinewidth{1.003750pt}%
\definecolor{currentstroke}{rgb}{0.121569,0.466667,0.705882}%
\pgfsetstrokecolor{currentstroke}%
\pgfsetstrokeopacity{0.301130}%
\pgfsetdash{}{0pt}%
\pgfpathmoveto{\pgfqpoint{1.645924in}{2.340784in}}%
\pgfpathcurveto{\pgfqpoint{1.654161in}{2.340784in}}{\pgfqpoint{1.662061in}{2.344056in}}{\pgfqpoint{1.667885in}{2.349880in}}%
\pgfpathcurveto{\pgfqpoint{1.673709in}{2.355704in}}{\pgfqpoint{1.676981in}{2.363604in}}{\pgfqpoint{1.676981in}{2.371840in}}%
\pgfpathcurveto{\pgfqpoint{1.676981in}{2.380076in}}{\pgfqpoint{1.673709in}{2.387976in}}{\pgfqpoint{1.667885in}{2.393800in}}%
\pgfpathcurveto{\pgfqpoint{1.662061in}{2.399624in}}{\pgfqpoint{1.654161in}{2.402897in}}{\pgfqpoint{1.645924in}{2.402897in}}%
\pgfpathcurveto{\pgfqpoint{1.637688in}{2.402897in}}{\pgfqpoint{1.629788in}{2.399624in}}{\pgfqpoint{1.623964in}{2.393800in}}%
\pgfpathcurveto{\pgfqpoint{1.618140in}{2.387976in}}{\pgfqpoint{1.614868in}{2.380076in}}{\pgfqpoint{1.614868in}{2.371840in}}%
\pgfpathcurveto{\pgfqpoint{1.614868in}{2.363604in}}{\pgfqpoint{1.618140in}{2.355704in}}{\pgfqpoint{1.623964in}{2.349880in}}%
\pgfpathcurveto{\pgfqpoint{1.629788in}{2.344056in}}{\pgfqpoint{1.637688in}{2.340784in}}{\pgfqpoint{1.645924in}{2.340784in}}%
\pgfpathclose%
\pgfusepath{stroke,fill}%
\end{pgfscope}%
\begin{pgfscope}%
\pgfpathrectangle{\pgfqpoint{0.100000in}{0.212622in}}{\pgfqpoint{3.696000in}{3.696000in}}%
\pgfusepath{clip}%
\pgfsetbuttcap%
\pgfsetroundjoin%
\definecolor{currentfill}{rgb}{0.121569,0.466667,0.705882}%
\pgfsetfillcolor{currentfill}%
\pgfsetfillopacity{0.301130}%
\pgfsetlinewidth{1.003750pt}%
\definecolor{currentstroke}{rgb}{0.121569,0.466667,0.705882}%
\pgfsetstrokecolor{currentstroke}%
\pgfsetstrokeopacity{0.301130}%
\pgfsetdash{}{0pt}%
\pgfpathmoveto{\pgfqpoint{1.645923in}{2.340782in}}%
\pgfpathcurveto{\pgfqpoint{1.654159in}{2.340782in}}{\pgfqpoint{1.662059in}{2.344054in}}{\pgfqpoint{1.667883in}{2.349878in}}%
\pgfpathcurveto{\pgfqpoint{1.673707in}{2.355702in}}{\pgfqpoint{1.676979in}{2.363602in}}{\pgfqpoint{1.676979in}{2.371839in}}%
\pgfpathcurveto{\pgfqpoint{1.676979in}{2.380075in}}{\pgfqpoint{1.673707in}{2.387975in}}{\pgfqpoint{1.667883in}{2.393799in}}%
\pgfpathcurveto{\pgfqpoint{1.662059in}{2.399623in}}{\pgfqpoint{1.654159in}{2.402895in}}{\pgfqpoint{1.645923in}{2.402895in}}%
\pgfpathcurveto{\pgfqpoint{1.637687in}{2.402895in}}{\pgfqpoint{1.629787in}{2.399623in}}{\pgfqpoint{1.623963in}{2.393799in}}%
\pgfpathcurveto{\pgfqpoint{1.618139in}{2.387975in}}{\pgfqpoint{1.614866in}{2.380075in}}{\pgfqpoint{1.614866in}{2.371839in}}%
\pgfpathcurveto{\pgfqpoint{1.614866in}{2.363602in}}{\pgfqpoint{1.618139in}{2.355702in}}{\pgfqpoint{1.623963in}{2.349878in}}%
\pgfpathcurveto{\pgfqpoint{1.629787in}{2.344054in}}{\pgfqpoint{1.637687in}{2.340782in}}{\pgfqpoint{1.645923in}{2.340782in}}%
\pgfpathclose%
\pgfusepath{stroke,fill}%
\end{pgfscope}%
\begin{pgfscope}%
\pgfpathrectangle{\pgfqpoint{0.100000in}{0.212622in}}{\pgfqpoint{3.696000in}{3.696000in}}%
\pgfusepath{clip}%
\pgfsetbuttcap%
\pgfsetroundjoin%
\definecolor{currentfill}{rgb}{0.121569,0.466667,0.705882}%
\pgfsetfillcolor{currentfill}%
\pgfsetfillopacity{0.301131}%
\pgfsetlinewidth{1.003750pt}%
\definecolor{currentstroke}{rgb}{0.121569,0.466667,0.705882}%
\pgfsetstrokecolor{currentstroke}%
\pgfsetstrokeopacity{0.301131}%
\pgfsetdash{}{0pt}%
\pgfpathmoveto{\pgfqpoint{1.645920in}{2.340779in}}%
\pgfpathcurveto{\pgfqpoint{1.654156in}{2.340779in}}{\pgfqpoint{1.662056in}{2.344051in}}{\pgfqpoint{1.667880in}{2.349875in}}%
\pgfpathcurveto{\pgfqpoint{1.673704in}{2.355699in}}{\pgfqpoint{1.676977in}{2.363599in}}{\pgfqpoint{1.676977in}{2.371836in}}%
\pgfpathcurveto{\pgfqpoint{1.676977in}{2.380072in}}{\pgfqpoint{1.673704in}{2.387972in}}{\pgfqpoint{1.667880in}{2.393796in}}%
\pgfpathcurveto{\pgfqpoint{1.662056in}{2.399620in}}{\pgfqpoint{1.654156in}{2.402892in}}{\pgfqpoint{1.645920in}{2.402892in}}%
\pgfpathcurveto{\pgfqpoint{1.637684in}{2.402892in}}{\pgfqpoint{1.629784in}{2.399620in}}{\pgfqpoint{1.623960in}{2.393796in}}%
\pgfpathcurveto{\pgfqpoint{1.618136in}{2.387972in}}{\pgfqpoint{1.614864in}{2.380072in}}{\pgfqpoint{1.614864in}{2.371836in}}%
\pgfpathcurveto{\pgfqpoint{1.614864in}{2.363599in}}{\pgfqpoint{1.618136in}{2.355699in}}{\pgfqpoint{1.623960in}{2.349875in}}%
\pgfpathcurveto{\pgfqpoint{1.629784in}{2.344051in}}{\pgfqpoint{1.637684in}{2.340779in}}{\pgfqpoint{1.645920in}{2.340779in}}%
\pgfpathclose%
\pgfusepath{stroke,fill}%
\end{pgfscope}%
\begin{pgfscope}%
\pgfpathrectangle{\pgfqpoint{0.100000in}{0.212622in}}{\pgfqpoint{3.696000in}{3.696000in}}%
\pgfusepath{clip}%
\pgfsetbuttcap%
\pgfsetroundjoin%
\definecolor{currentfill}{rgb}{0.121569,0.466667,0.705882}%
\pgfsetfillcolor{currentfill}%
\pgfsetfillopacity{0.301133}%
\pgfsetlinewidth{1.003750pt}%
\definecolor{currentstroke}{rgb}{0.121569,0.466667,0.705882}%
\pgfsetstrokecolor{currentstroke}%
\pgfsetstrokeopacity{0.301133}%
\pgfsetdash{}{0pt}%
\pgfpathmoveto{\pgfqpoint{1.645917in}{2.340774in}}%
\pgfpathcurveto{\pgfqpoint{1.654153in}{2.340774in}}{\pgfqpoint{1.662053in}{2.344046in}}{\pgfqpoint{1.667877in}{2.349870in}}%
\pgfpathcurveto{\pgfqpoint{1.673701in}{2.355694in}}{\pgfqpoint{1.676973in}{2.363594in}}{\pgfqpoint{1.676973in}{2.371830in}}%
\pgfpathcurveto{\pgfqpoint{1.676973in}{2.380067in}}{\pgfqpoint{1.673701in}{2.387967in}}{\pgfqpoint{1.667877in}{2.393791in}}%
\pgfpathcurveto{\pgfqpoint{1.662053in}{2.399614in}}{\pgfqpoint{1.654153in}{2.402887in}}{\pgfqpoint{1.645917in}{2.402887in}}%
\pgfpathcurveto{\pgfqpoint{1.637680in}{2.402887in}}{\pgfqpoint{1.629780in}{2.399614in}}{\pgfqpoint{1.623956in}{2.393791in}}%
\pgfpathcurveto{\pgfqpoint{1.618132in}{2.387967in}}{\pgfqpoint{1.614860in}{2.380067in}}{\pgfqpoint{1.614860in}{2.371830in}}%
\pgfpathcurveto{\pgfqpoint{1.614860in}{2.363594in}}{\pgfqpoint{1.618132in}{2.355694in}}{\pgfqpoint{1.623956in}{2.349870in}}%
\pgfpathcurveto{\pgfqpoint{1.629780in}{2.344046in}}{\pgfqpoint{1.637680in}{2.340774in}}{\pgfqpoint{1.645917in}{2.340774in}}%
\pgfpathclose%
\pgfusepath{stroke,fill}%
\end{pgfscope}%
\begin{pgfscope}%
\pgfpathrectangle{\pgfqpoint{0.100000in}{0.212622in}}{\pgfqpoint{3.696000in}{3.696000in}}%
\pgfusepath{clip}%
\pgfsetbuttcap%
\pgfsetroundjoin%
\definecolor{currentfill}{rgb}{0.121569,0.466667,0.705882}%
\pgfsetfillcolor{currentfill}%
\pgfsetfillopacity{0.301135}%
\pgfsetlinewidth{1.003750pt}%
\definecolor{currentstroke}{rgb}{0.121569,0.466667,0.705882}%
\pgfsetstrokecolor{currentstroke}%
\pgfsetstrokeopacity{0.301135}%
\pgfsetdash{}{0pt}%
\pgfpathmoveto{\pgfqpoint{1.645909in}{2.340764in}}%
\pgfpathcurveto{\pgfqpoint{1.654145in}{2.340764in}}{\pgfqpoint{1.662045in}{2.344037in}}{\pgfqpoint{1.667869in}{2.349861in}}%
\pgfpathcurveto{\pgfqpoint{1.673693in}{2.355685in}}{\pgfqpoint{1.676965in}{2.363585in}}{\pgfqpoint{1.676965in}{2.371821in}}%
\pgfpathcurveto{\pgfqpoint{1.676965in}{2.380057in}}{\pgfqpoint{1.673693in}{2.387957in}}{\pgfqpoint{1.667869in}{2.393781in}}%
\pgfpathcurveto{\pgfqpoint{1.662045in}{2.399605in}}{\pgfqpoint{1.654145in}{2.402877in}}{\pgfqpoint{1.645909in}{2.402877in}}%
\pgfpathcurveto{\pgfqpoint{1.637673in}{2.402877in}}{\pgfqpoint{1.629773in}{2.399605in}}{\pgfqpoint{1.623949in}{2.393781in}}%
\pgfpathcurveto{\pgfqpoint{1.618125in}{2.387957in}}{\pgfqpoint{1.614852in}{2.380057in}}{\pgfqpoint{1.614852in}{2.371821in}}%
\pgfpathcurveto{\pgfqpoint{1.614852in}{2.363585in}}{\pgfqpoint{1.618125in}{2.355685in}}{\pgfqpoint{1.623949in}{2.349861in}}%
\pgfpathcurveto{\pgfqpoint{1.629773in}{2.344037in}}{\pgfqpoint{1.637673in}{2.340764in}}{\pgfqpoint{1.645909in}{2.340764in}}%
\pgfpathclose%
\pgfusepath{stroke,fill}%
\end{pgfscope}%
\begin{pgfscope}%
\pgfpathrectangle{\pgfqpoint{0.100000in}{0.212622in}}{\pgfqpoint{3.696000in}{3.696000in}}%
\pgfusepath{clip}%
\pgfsetbuttcap%
\pgfsetroundjoin%
\definecolor{currentfill}{rgb}{0.121569,0.466667,0.705882}%
\pgfsetfillcolor{currentfill}%
\pgfsetfillopacity{0.301140}%
\pgfsetlinewidth{1.003750pt}%
\definecolor{currentstroke}{rgb}{0.121569,0.466667,0.705882}%
\pgfsetstrokecolor{currentstroke}%
\pgfsetstrokeopacity{0.301140}%
\pgfsetdash{}{0pt}%
\pgfpathmoveto{\pgfqpoint{1.645899in}{2.340747in}}%
\pgfpathcurveto{\pgfqpoint{1.654135in}{2.340747in}}{\pgfqpoint{1.662035in}{2.344020in}}{\pgfqpoint{1.667859in}{2.349844in}}%
\pgfpathcurveto{\pgfqpoint{1.673683in}{2.355668in}}{\pgfqpoint{1.676955in}{2.363568in}}{\pgfqpoint{1.676955in}{2.371804in}}%
\pgfpathcurveto{\pgfqpoint{1.676955in}{2.380040in}}{\pgfqpoint{1.673683in}{2.387940in}}{\pgfqpoint{1.667859in}{2.393764in}}%
\pgfpathcurveto{\pgfqpoint{1.662035in}{2.399588in}}{\pgfqpoint{1.654135in}{2.402860in}}{\pgfqpoint{1.645899in}{2.402860in}}%
\pgfpathcurveto{\pgfqpoint{1.637662in}{2.402860in}}{\pgfqpoint{1.629762in}{2.399588in}}{\pgfqpoint{1.623938in}{2.393764in}}%
\pgfpathcurveto{\pgfqpoint{1.618114in}{2.387940in}}{\pgfqpoint{1.614842in}{2.380040in}}{\pgfqpoint{1.614842in}{2.371804in}}%
\pgfpathcurveto{\pgfqpoint{1.614842in}{2.363568in}}{\pgfqpoint{1.618114in}{2.355668in}}{\pgfqpoint{1.623938in}{2.349844in}}%
\pgfpathcurveto{\pgfqpoint{1.629762in}{2.344020in}}{\pgfqpoint{1.637662in}{2.340747in}}{\pgfqpoint{1.645899in}{2.340747in}}%
\pgfpathclose%
\pgfusepath{stroke,fill}%
\end{pgfscope}%
\begin{pgfscope}%
\pgfpathrectangle{\pgfqpoint{0.100000in}{0.212622in}}{\pgfqpoint{3.696000in}{3.696000in}}%
\pgfusepath{clip}%
\pgfsetbuttcap%
\pgfsetroundjoin%
\definecolor{currentfill}{rgb}{0.121569,0.466667,0.705882}%
\pgfsetfillcolor{currentfill}%
\pgfsetfillopacity{0.301149}%
\pgfsetlinewidth{1.003750pt}%
\definecolor{currentstroke}{rgb}{0.121569,0.466667,0.705882}%
\pgfsetstrokecolor{currentstroke}%
\pgfsetstrokeopacity{0.301149}%
\pgfsetdash{}{0pt}%
\pgfpathmoveto{\pgfqpoint{1.645878in}{2.340716in}}%
\pgfpathcurveto{\pgfqpoint{1.654115in}{2.340716in}}{\pgfqpoint{1.662015in}{2.343989in}}{\pgfqpoint{1.667839in}{2.349812in}}%
\pgfpathcurveto{\pgfqpoint{1.673663in}{2.355636in}}{\pgfqpoint{1.676935in}{2.363536in}}{\pgfqpoint{1.676935in}{2.371773in}}%
\pgfpathcurveto{\pgfqpoint{1.676935in}{2.380009in}}{\pgfqpoint{1.673663in}{2.387909in}}{\pgfqpoint{1.667839in}{2.393733in}}%
\pgfpathcurveto{\pgfqpoint{1.662015in}{2.399557in}}{\pgfqpoint{1.654115in}{2.402829in}}{\pgfqpoint{1.645878in}{2.402829in}}%
\pgfpathcurveto{\pgfqpoint{1.637642in}{2.402829in}}{\pgfqpoint{1.629742in}{2.399557in}}{\pgfqpoint{1.623918in}{2.393733in}}%
\pgfpathcurveto{\pgfqpoint{1.618094in}{2.387909in}}{\pgfqpoint{1.614822in}{2.380009in}}{\pgfqpoint{1.614822in}{2.371773in}}%
\pgfpathcurveto{\pgfqpoint{1.614822in}{2.363536in}}{\pgfqpoint{1.618094in}{2.355636in}}{\pgfqpoint{1.623918in}{2.349812in}}%
\pgfpathcurveto{\pgfqpoint{1.629742in}{2.343989in}}{\pgfqpoint{1.637642in}{2.340716in}}{\pgfqpoint{1.645878in}{2.340716in}}%
\pgfpathclose%
\pgfusepath{stroke,fill}%
\end{pgfscope}%
\begin{pgfscope}%
\pgfpathrectangle{\pgfqpoint{0.100000in}{0.212622in}}{\pgfqpoint{3.696000in}{3.696000in}}%
\pgfusepath{clip}%
\pgfsetbuttcap%
\pgfsetroundjoin%
\definecolor{currentfill}{rgb}{0.121569,0.466667,0.705882}%
\pgfsetfillcolor{currentfill}%
\pgfsetfillopacity{0.301165}%
\pgfsetlinewidth{1.003750pt}%
\definecolor{currentstroke}{rgb}{0.121569,0.466667,0.705882}%
\pgfsetstrokecolor{currentstroke}%
\pgfsetstrokeopacity{0.301165}%
\pgfsetdash{}{0pt}%
\pgfpathmoveto{\pgfqpoint{1.645841in}{2.340660in}}%
\pgfpathcurveto{\pgfqpoint{1.654077in}{2.340660in}}{\pgfqpoint{1.661977in}{2.343932in}}{\pgfqpoint{1.667801in}{2.349756in}}%
\pgfpathcurveto{\pgfqpoint{1.673625in}{2.355580in}}{\pgfqpoint{1.676898in}{2.363480in}}{\pgfqpoint{1.676898in}{2.371717in}}%
\pgfpathcurveto{\pgfqpoint{1.676898in}{2.379953in}}{\pgfqpoint{1.673625in}{2.387853in}}{\pgfqpoint{1.667801in}{2.393677in}}%
\pgfpathcurveto{\pgfqpoint{1.661977in}{2.399501in}}{\pgfqpoint{1.654077in}{2.402773in}}{\pgfqpoint{1.645841in}{2.402773in}}%
\pgfpathcurveto{\pgfqpoint{1.637605in}{2.402773in}}{\pgfqpoint{1.629705in}{2.399501in}}{\pgfqpoint{1.623881in}{2.393677in}}%
\pgfpathcurveto{\pgfqpoint{1.618057in}{2.387853in}}{\pgfqpoint{1.614785in}{2.379953in}}{\pgfqpoint{1.614785in}{2.371717in}}%
\pgfpathcurveto{\pgfqpoint{1.614785in}{2.363480in}}{\pgfqpoint{1.618057in}{2.355580in}}{\pgfqpoint{1.623881in}{2.349756in}}%
\pgfpathcurveto{\pgfqpoint{1.629705in}{2.343932in}}{\pgfqpoint{1.637605in}{2.340660in}}{\pgfqpoint{1.645841in}{2.340660in}}%
\pgfpathclose%
\pgfusepath{stroke,fill}%
\end{pgfscope}%
\begin{pgfscope}%
\pgfpathrectangle{\pgfqpoint{0.100000in}{0.212622in}}{\pgfqpoint{3.696000in}{3.696000in}}%
\pgfusepath{clip}%
\pgfsetbuttcap%
\pgfsetroundjoin%
\definecolor{currentfill}{rgb}{0.121569,0.466667,0.705882}%
\pgfsetfillcolor{currentfill}%
\pgfsetfillopacity{0.301194}%
\pgfsetlinewidth{1.003750pt}%
\definecolor{currentstroke}{rgb}{0.121569,0.466667,0.705882}%
\pgfsetstrokecolor{currentstroke}%
\pgfsetstrokeopacity{0.301194}%
\pgfsetdash{}{0pt}%
\pgfpathmoveto{\pgfqpoint{1.645772in}{2.340558in}}%
\pgfpathcurveto{\pgfqpoint{1.654009in}{2.340558in}}{\pgfqpoint{1.661909in}{2.343830in}}{\pgfqpoint{1.667733in}{2.349654in}}%
\pgfpathcurveto{\pgfqpoint{1.673557in}{2.355478in}}{\pgfqpoint{1.676829in}{2.363378in}}{\pgfqpoint{1.676829in}{2.371614in}}%
\pgfpathcurveto{\pgfqpoint{1.676829in}{2.379851in}}{\pgfqpoint{1.673557in}{2.387751in}}{\pgfqpoint{1.667733in}{2.393575in}}%
\pgfpathcurveto{\pgfqpoint{1.661909in}{2.399399in}}{\pgfqpoint{1.654009in}{2.402671in}}{\pgfqpoint{1.645772in}{2.402671in}}%
\pgfpathcurveto{\pgfqpoint{1.637536in}{2.402671in}}{\pgfqpoint{1.629636in}{2.399399in}}{\pgfqpoint{1.623812in}{2.393575in}}%
\pgfpathcurveto{\pgfqpoint{1.617988in}{2.387751in}}{\pgfqpoint{1.614716in}{2.379851in}}{\pgfqpoint{1.614716in}{2.371614in}}%
\pgfpathcurveto{\pgfqpoint{1.614716in}{2.363378in}}{\pgfqpoint{1.617988in}{2.355478in}}{\pgfqpoint{1.623812in}{2.349654in}}%
\pgfpathcurveto{\pgfqpoint{1.629636in}{2.343830in}}{\pgfqpoint{1.637536in}{2.340558in}}{\pgfqpoint{1.645772in}{2.340558in}}%
\pgfpathclose%
\pgfusepath{stroke,fill}%
\end{pgfscope}%
\begin{pgfscope}%
\pgfpathrectangle{\pgfqpoint{0.100000in}{0.212622in}}{\pgfqpoint{3.696000in}{3.696000in}}%
\pgfusepath{clip}%
\pgfsetbuttcap%
\pgfsetroundjoin%
\definecolor{currentfill}{rgb}{0.121569,0.466667,0.705882}%
\pgfsetfillcolor{currentfill}%
\pgfsetfillopacity{0.301246}%
\pgfsetlinewidth{1.003750pt}%
\definecolor{currentstroke}{rgb}{0.121569,0.466667,0.705882}%
\pgfsetstrokecolor{currentstroke}%
\pgfsetstrokeopacity{0.301246}%
\pgfsetdash{}{0pt}%
\pgfpathmoveto{\pgfqpoint{1.645646in}{2.340369in}}%
\pgfpathcurveto{\pgfqpoint{1.653882in}{2.340369in}}{\pgfqpoint{1.661782in}{2.343641in}}{\pgfqpoint{1.667606in}{2.349465in}}%
\pgfpathcurveto{\pgfqpoint{1.673430in}{2.355289in}}{\pgfqpoint{1.676703in}{2.363189in}}{\pgfqpoint{1.676703in}{2.371425in}}%
\pgfpathcurveto{\pgfqpoint{1.676703in}{2.379662in}}{\pgfqpoint{1.673430in}{2.387562in}}{\pgfqpoint{1.667606in}{2.393386in}}%
\pgfpathcurveto{\pgfqpoint{1.661782in}{2.399210in}}{\pgfqpoint{1.653882in}{2.402482in}}{\pgfqpoint{1.645646in}{2.402482in}}%
\pgfpathcurveto{\pgfqpoint{1.637410in}{2.402482in}}{\pgfqpoint{1.629510in}{2.399210in}}{\pgfqpoint{1.623686in}{2.393386in}}%
\pgfpathcurveto{\pgfqpoint{1.617862in}{2.387562in}}{\pgfqpoint{1.614590in}{2.379662in}}{\pgfqpoint{1.614590in}{2.371425in}}%
\pgfpathcurveto{\pgfqpoint{1.614590in}{2.363189in}}{\pgfqpoint{1.617862in}{2.355289in}}{\pgfqpoint{1.623686in}{2.349465in}}%
\pgfpathcurveto{\pgfqpoint{1.629510in}{2.343641in}}{\pgfqpoint{1.637410in}{2.340369in}}{\pgfqpoint{1.645646in}{2.340369in}}%
\pgfpathclose%
\pgfusepath{stroke,fill}%
\end{pgfscope}%
\begin{pgfscope}%
\pgfpathrectangle{\pgfqpoint{0.100000in}{0.212622in}}{\pgfqpoint{3.696000in}{3.696000in}}%
\pgfusepath{clip}%
\pgfsetbuttcap%
\pgfsetroundjoin%
\definecolor{currentfill}{rgb}{0.121569,0.466667,0.705882}%
\pgfsetfillcolor{currentfill}%
\pgfsetfillopacity{0.301338}%
\pgfsetlinewidth{1.003750pt}%
\definecolor{currentstroke}{rgb}{0.121569,0.466667,0.705882}%
\pgfsetstrokecolor{currentstroke}%
\pgfsetstrokeopacity{0.301338}%
\pgfsetdash{}{0pt}%
\pgfpathmoveto{\pgfqpoint{1.645419in}{2.340005in}}%
\pgfpathcurveto{\pgfqpoint{1.653655in}{2.340005in}}{\pgfqpoint{1.661555in}{2.343277in}}{\pgfqpoint{1.667379in}{2.349101in}}%
\pgfpathcurveto{\pgfqpoint{1.673203in}{2.354925in}}{\pgfqpoint{1.676475in}{2.362825in}}{\pgfqpoint{1.676475in}{2.371061in}}%
\pgfpathcurveto{\pgfqpoint{1.676475in}{2.379298in}}{\pgfqpoint{1.673203in}{2.387198in}}{\pgfqpoint{1.667379in}{2.393022in}}%
\pgfpathcurveto{\pgfqpoint{1.661555in}{2.398846in}}{\pgfqpoint{1.653655in}{2.402118in}}{\pgfqpoint{1.645419in}{2.402118in}}%
\pgfpathcurveto{\pgfqpoint{1.637182in}{2.402118in}}{\pgfqpoint{1.629282in}{2.398846in}}{\pgfqpoint{1.623458in}{2.393022in}}%
\pgfpathcurveto{\pgfqpoint{1.617634in}{2.387198in}}{\pgfqpoint{1.614362in}{2.379298in}}{\pgfqpoint{1.614362in}{2.371061in}}%
\pgfpathcurveto{\pgfqpoint{1.614362in}{2.362825in}}{\pgfqpoint{1.617634in}{2.354925in}}{\pgfqpoint{1.623458in}{2.349101in}}%
\pgfpathcurveto{\pgfqpoint{1.629282in}{2.343277in}}{\pgfqpoint{1.637182in}{2.340005in}}{\pgfqpoint{1.645419in}{2.340005in}}%
\pgfpathclose%
\pgfusepath{stroke,fill}%
\end{pgfscope}%
\begin{pgfscope}%
\pgfpathrectangle{\pgfqpoint{0.100000in}{0.212622in}}{\pgfqpoint{3.696000in}{3.696000in}}%
\pgfusepath{clip}%
\pgfsetbuttcap%
\pgfsetroundjoin%
\definecolor{currentfill}{rgb}{0.121569,0.466667,0.705882}%
\pgfsetfillcolor{currentfill}%
\pgfsetfillopacity{0.301517}%
\pgfsetlinewidth{1.003750pt}%
\definecolor{currentstroke}{rgb}{0.121569,0.466667,0.705882}%
\pgfsetstrokecolor{currentstroke}%
\pgfsetstrokeopacity{0.301517}%
\pgfsetdash{}{0pt}%
\pgfpathmoveto{\pgfqpoint{1.645011in}{2.339423in}}%
\pgfpathcurveto{\pgfqpoint{1.653247in}{2.339423in}}{\pgfqpoint{1.661147in}{2.342695in}}{\pgfqpoint{1.666971in}{2.348519in}}%
\pgfpathcurveto{\pgfqpoint{1.672795in}{2.354343in}}{\pgfqpoint{1.676067in}{2.362243in}}{\pgfqpoint{1.676067in}{2.370480in}}%
\pgfpathcurveto{\pgfqpoint{1.676067in}{2.378716in}}{\pgfqpoint{1.672795in}{2.386616in}}{\pgfqpoint{1.666971in}{2.392440in}}%
\pgfpathcurveto{\pgfqpoint{1.661147in}{2.398264in}}{\pgfqpoint{1.653247in}{2.401536in}}{\pgfqpoint{1.645011in}{2.401536in}}%
\pgfpathcurveto{\pgfqpoint{1.636774in}{2.401536in}}{\pgfqpoint{1.628874in}{2.398264in}}{\pgfqpoint{1.623050in}{2.392440in}}%
\pgfpathcurveto{\pgfqpoint{1.617226in}{2.386616in}}{\pgfqpoint{1.613954in}{2.378716in}}{\pgfqpoint{1.613954in}{2.370480in}}%
\pgfpathcurveto{\pgfqpoint{1.613954in}{2.362243in}}{\pgfqpoint{1.617226in}{2.354343in}}{\pgfqpoint{1.623050in}{2.348519in}}%
\pgfpathcurveto{\pgfqpoint{1.628874in}{2.342695in}}{\pgfqpoint{1.636774in}{2.339423in}}{\pgfqpoint{1.645011in}{2.339423in}}%
\pgfpathclose%
\pgfusepath{stroke,fill}%
\end{pgfscope}%
\begin{pgfscope}%
\pgfpathrectangle{\pgfqpoint{0.100000in}{0.212622in}}{\pgfqpoint{3.696000in}{3.696000in}}%
\pgfusepath{clip}%
\pgfsetbuttcap%
\pgfsetroundjoin%
\definecolor{currentfill}{rgb}{0.121569,0.466667,0.705882}%
\pgfsetfillcolor{currentfill}%
\pgfsetfillopacity{0.301845}%
\pgfsetlinewidth{1.003750pt}%
\definecolor{currentstroke}{rgb}{0.121569,0.466667,0.705882}%
\pgfsetstrokecolor{currentstroke}%
\pgfsetstrokeopacity{0.301845}%
\pgfsetdash{}{0pt}%
\pgfpathmoveto{\pgfqpoint{1.644355in}{2.338308in}}%
\pgfpathcurveto{\pgfqpoint{1.652592in}{2.338308in}}{\pgfqpoint{1.660492in}{2.341580in}}{\pgfqpoint{1.666316in}{2.347404in}}%
\pgfpathcurveto{\pgfqpoint{1.672140in}{2.353228in}}{\pgfqpoint{1.675412in}{2.361128in}}{\pgfqpoint{1.675412in}{2.369364in}}%
\pgfpathcurveto{\pgfqpoint{1.675412in}{2.377601in}}{\pgfqpoint{1.672140in}{2.385501in}}{\pgfqpoint{1.666316in}{2.391325in}}%
\pgfpathcurveto{\pgfqpoint{1.660492in}{2.397149in}}{\pgfqpoint{1.652592in}{2.400421in}}{\pgfqpoint{1.644355in}{2.400421in}}%
\pgfpathcurveto{\pgfqpoint{1.636119in}{2.400421in}}{\pgfqpoint{1.628219in}{2.397149in}}{\pgfqpoint{1.622395in}{2.391325in}}%
\pgfpathcurveto{\pgfqpoint{1.616571in}{2.385501in}}{\pgfqpoint{1.613299in}{2.377601in}}{\pgfqpoint{1.613299in}{2.369364in}}%
\pgfpathcurveto{\pgfqpoint{1.613299in}{2.361128in}}{\pgfqpoint{1.616571in}{2.353228in}}{\pgfqpoint{1.622395in}{2.347404in}}%
\pgfpathcurveto{\pgfqpoint{1.628219in}{2.341580in}}{\pgfqpoint{1.636119in}{2.338308in}}{\pgfqpoint{1.644355in}{2.338308in}}%
\pgfpathclose%
\pgfusepath{stroke,fill}%
\end{pgfscope}%
\begin{pgfscope}%
\pgfpathrectangle{\pgfqpoint{0.100000in}{0.212622in}}{\pgfqpoint{3.696000in}{3.696000in}}%
\pgfusepath{clip}%
\pgfsetbuttcap%
\pgfsetroundjoin%
\definecolor{currentfill}{rgb}{0.121569,0.466667,0.705882}%
\pgfsetfillcolor{currentfill}%
\pgfsetfillopacity{0.302253}%
\pgfsetlinewidth{1.003750pt}%
\definecolor{currentstroke}{rgb}{0.121569,0.466667,0.705882}%
\pgfsetstrokecolor{currentstroke}%
\pgfsetstrokeopacity{0.302253}%
\pgfsetdash{}{0pt}%
\pgfpathmoveto{\pgfqpoint{1.698426in}{2.350113in}}%
\pgfpathcurveto{\pgfqpoint{1.706662in}{2.350113in}}{\pgfqpoint{1.714562in}{2.353386in}}{\pgfqpoint{1.720386in}{2.359209in}}%
\pgfpathcurveto{\pgfqpoint{1.726210in}{2.365033in}}{\pgfqpoint{1.729483in}{2.372933in}}{\pgfqpoint{1.729483in}{2.381170in}}%
\pgfpathcurveto{\pgfqpoint{1.729483in}{2.389406in}}{\pgfqpoint{1.726210in}{2.397306in}}{\pgfqpoint{1.720386in}{2.403130in}}%
\pgfpathcurveto{\pgfqpoint{1.714562in}{2.408954in}}{\pgfqpoint{1.706662in}{2.412226in}}{\pgfqpoint{1.698426in}{2.412226in}}%
\pgfpathcurveto{\pgfqpoint{1.690190in}{2.412226in}}{\pgfqpoint{1.682290in}{2.408954in}}{\pgfqpoint{1.676466in}{2.403130in}}%
\pgfpathcurveto{\pgfqpoint{1.670642in}{2.397306in}}{\pgfqpoint{1.667370in}{2.389406in}}{\pgfqpoint{1.667370in}{2.381170in}}%
\pgfpathcurveto{\pgfqpoint{1.667370in}{2.372933in}}{\pgfqpoint{1.670642in}{2.365033in}}{\pgfqpoint{1.676466in}{2.359209in}}%
\pgfpathcurveto{\pgfqpoint{1.682290in}{2.353386in}}{\pgfqpoint{1.690190in}{2.350113in}}{\pgfqpoint{1.698426in}{2.350113in}}%
\pgfpathclose%
\pgfusepath{stroke,fill}%
\end{pgfscope}%
\begin{pgfscope}%
\pgfpathrectangle{\pgfqpoint{0.100000in}{0.212622in}}{\pgfqpoint{3.696000in}{3.696000in}}%
\pgfusepath{clip}%
\pgfsetbuttcap%
\pgfsetroundjoin%
\definecolor{currentfill}{rgb}{0.121569,0.466667,0.705882}%
\pgfsetfillcolor{currentfill}%
\pgfsetfillopacity{0.302387}%
\pgfsetlinewidth{1.003750pt}%
\definecolor{currentstroke}{rgb}{0.121569,0.466667,0.705882}%
\pgfsetstrokecolor{currentstroke}%
\pgfsetstrokeopacity{0.302387}%
\pgfsetdash{}{0pt}%
\pgfpathmoveto{\pgfqpoint{1.642958in}{2.336062in}}%
\pgfpathcurveto{\pgfqpoint{1.651194in}{2.336062in}}{\pgfqpoint{1.659094in}{2.339335in}}{\pgfqpoint{1.664918in}{2.345158in}}%
\pgfpathcurveto{\pgfqpoint{1.670742in}{2.350982in}}{\pgfqpoint{1.674015in}{2.358882in}}{\pgfqpoint{1.674015in}{2.367119in}}%
\pgfpathcurveto{\pgfqpoint{1.674015in}{2.375355in}}{\pgfqpoint{1.670742in}{2.383255in}}{\pgfqpoint{1.664918in}{2.389079in}}%
\pgfpathcurveto{\pgfqpoint{1.659094in}{2.394903in}}{\pgfqpoint{1.651194in}{2.398175in}}{\pgfqpoint{1.642958in}{2.398175in}}%
\pgfpathcurveto{\pgfqpoint{1.634722in}{2.398175in}}{\pgfqpoint{1.626822in}{2.394903in}}{\pgfqpoint{1.620998in}{2.389079in}}%
\pgfpathcurveto{\pgfqpoint{1.615174in}{2.383255in}}{\pgfqpoint{1.611902in}{2.375355in}}{\pgfqpoint{1.611902in}{2.367119in}}%
\pgfpathcurveto{\pgfqpoint{1.611902in}{2.358882in}}{\pgfqpoint{1.615174in}{2.350982in}}{\pgfqpoint{1.620998in}{2.345158in}}%
\pgfpathcurveto{\pgfqpoint{1.626822in}{2.339335in}}{\pgfqpoint{1.634722in}{2.336062in}}{\pgfqpoint{1.642958in}{2.336062in}}%
\pgfpathclose%
\pgfusepath{stroke,fill}%
\end{pgfscope}%
\begin{pgfscope}%
\pgfpathrectangle{\pgfqpoint{0.100000in}{0.212622in}}{\pgfqpoint{3.696000in}{3.696000in}}%
\pgfusepath{clip}%
\pgfsetbuttcap%
\pgfsetroundjoin%
\definecolor{currentfill}{rgb}{0.121569,0.466667,0.705882}%
\pgfsetfillcolor{currentfill}%
\pgfsetfillopacity{0.303503}%
\pgfsetlinewidth{1.003750pt}%
\definecolor{currentstroke}{rgb}{0.121569,0.466667,0.705882}%
\pgfsetstrokecolor{currentstroke}%
\pgfsetstrokeopacity{0.303503}%
\pgfsetdash{}{0pt}%
\pgfpathmoveto{\pgfqpoint{1.640755in}{2.332628in}}%
\pgfpathcurveto{\pgfqpoint{1.648991in}{2.332628in}}{\pgfqpoint{1.656891in}{2.335900in}}{\pgfqpoint{1.662715in}{2.341724in}}%
\pgfpathcurveto{\pgfqpoint{1.668539in}{2.347548in}}{\pgfqpoint{1.671811in}{2.355448in}}{\pgfqpoint{1.671811in}{2.363684in}}%
\pgfpathcurveto{\pgfqpoint{1.671811in}{2.371920in}}{\pgfqpoint{1.668539in}{2.379820in}}{\pgfqpoint{1.662715in}{2.385644in}}%
\pgfpathcurveto{\pgfqpoint{1.656891in}{2.391468in}}{\pgfqpoint{1.648991in}{2.394741in}}{\pgfqpoint{1.640755in}{2.394741in}}%
\pgfpathcurveto{\pgfqpoint{1.632519in}{2.394741in}}{\pgfqpoint{1.624619in}{2.391468in}}{\pgfqpoint{1.618795in}{2.385644in}}%
\pgfpathcurveto{\pgfqpoint{1.612971in}{2.379820in}}{\pgfqpoint{1.609698in}{2.371920in}}{\pgfqpoint{1.609698in}{2.363684in}}%
\pgfpathcurveto{\pgfqpoint{1.609698in}{2.355448in}}{\pgfqpoint{1.612971in}{2.347548in}}{\pgfqpoint{1.618795in}{2.341724in}}%
\pgfpathcurveto{\pgfqpoint{1.624619in}{2.335900in}}{\pgfqpoint{1.632519in}{2.332628in}}{\pgfqpoint{1.640755in}{2.332628in}}%
\pgfpathclose%
\pgfusepath{stroke,fill}%
\end{pgfscope}%
\begin{pgfscope}%
\pgfpathrectangle{\pgfqpoint{0.100000in}{0.212622in}}{\pgfqpoint{3.696000in}{3.696000in}}%
\pgfusepath{clip}%
\pgfsetbuttcap%
\pgfsetroundjoin%
\definecolor{currentfill}{rgb}{0.121569,0.466667,0.705882}%
\pgfsetfillcolor{currentfill}%
\pgfsetfillopacity{0.304887}%
\pgfsetlinewidth{1.003750pt}%
\definecolor{currentstroke}{rgb}{0.121569,0.466667,0.705882}%
\pgfsetstrokecolor{currentstroke}%
\pgfsetstrokeopacity{0.304887}%
\pgfsetdash{}{0pt}%
\pgfpathmoveto{\pgfqpoint{1.714482in}{2.348675in}}%
\pgfpathcurveto{\pgfqpoint{1.722718in}{2.348675in}}{\pgfqpoint{1.730618in}{2.351947in}}{\pgfqpoint{1.736442in}{2.357771in}}%
\pgfpathcurveto{\pgfqpoint{1.742266in}{2.363595in}}{\pgfqpoint{1.745538in}{2.371495in}}{\pgfqpoint{1.745538in}{2.379732in}}%
\pgfpathcurveto{\pgfqpoint{1.745538in}{2.387968in}}{\pgfqpoint{1.742266in}{2.395868in}}{\pgfqpoint{1.736442in}{2.401692in}}%
\pgfpathcurveto{\pgfqpoint{1.730618in}{2.407516in}}{\pgfqpoint{1.722718in}{2.410788in}}{\pgfqpoint{1.714482in}{2.410788in}}%
\pgfpathcurveto{\pgfqpoint{1.706245in}{2.410788in}}{\pgfqpoint{1.698345in}{2.407516in}}{\pgfqpoint{1.692522in}{2.401692in}}%
\pgfpathcurveto{\pgfqpoint{1.686698in}{2.395868in}}{\pgfqpoint{1.683425in}{2.387968in}}{\pgfqpoint{1.683425in}{2.379732in}}%
\pgfpathcurveto{\pgfqpoint{1.683425in}{2.371495in}}{\pgfqpoint{1.686698in}{2.363595in}}{\pgfqpoint{1.692522in}{2.357771in}}%
\pgfpathcurveto{\pgfqpoint{1.698345in}{2.351947in}}{\pgfqpoint{1.706245in}{2.348675in}}{\pgfqpoint{1.714482in}{2.348675in}}%
\pgfpathclose%
\pgfusepath{stroke,fill}%
\end{pgfscope}%
\begin{pgfscope}%
\pgfpathrectangle{\pgfqpoint{0.100000in}{0.212622in}}{\pgfqpoint{3.696000in}{3.696000in}}%
\pgfusepath{clip}%
\pgfsetbuttcap%
\pgfsetroundjoin%
\definecolor{currentfill}{rgb}{0.121569,0.466667,0.705882}%
\pgfsetfillcolor{currentfill}%
\pgfsetfillopacity{0.305417}%
\pgfsetlinewidth{1.003750pt}%
\definecolor{currentstroke}{rgb}{0.121569,0.466667,0.705882}%
\pgfsetstrokecolor{currentstroke}%
\pgfsetstrokeopacity{0.305417}%
\pgfsetdash{}{0pt}%
\pgfpathmoveto{\pgfqpoint{1.636061in}{2.326097in}}%
\pgfpathcurveto{\pgfqpoint{1.644298in}{2.326097in}}{\pgfqpoint{1.652198in}{2.329370in}}{\pgfqpoint{1.658022in}{2.335193in}}%
\pgfpathcurveto{\pgfqpoint{1.663845in}{2.341017in}}{\pgfqpoint{1.667118in}{2.348917in}}{\pgfqpoint{1.667118in}{2.357154in}}%
\pgfpathcurveto{\pgfqpoint{1.667118in}{2.365390in}}{\pgfqpoint{1.663845in}{2.373290in}}{\pgfqpoint{1.658022in}{2.379114in}}%
\pgfpathcurveto{\pgfqpoint{1.652198in}{2.384938in}}{\pgfqpoint{1.644298in}{2.388210in}}{\pgfqpoint{1.636061in}{2.388210in}}%
\pgfpathcurveto{\pgfqpoint{1.627825in}{2.388210in}}{\pgfqpoint{1.619925in}{2.384938in}}{\pgfqpoint{1.614101in}{2.379114in}}%
\pgfpathcurveto{\pgfqpoint{1.608277in}{2.373290in}}{\pgfqpoint{1.605005in}{2.365390in}}{\pgfqpoint{1.605005in}{2.357154in}}%
\pgfpathcurveto{\pgfqpoint{1.605005in}{2.348917in}}{\pgfqpoint{1.608277in}{2.341017in}}{\pgfqpoint{1.614101in}{2.335193in}}%
\pgfpathcurveto{\pgfqpoint{1.619925in}{2.329370in}}{\pgfqpoint{1.627825in}{2.326097in}}{\pgfqpoint{1.636061in}{2.326097in}}%
\pgfpathclose%
\pgfusepath{stroke,fill}%
\end{pgfscope}%
\begin{pgfscope}%
\pgfpathrectangle{\pgfqpoint{0.100000in}{0.212622in}}{\pgfqpoint{3.696000in}{3.696000in}}%
\pgfusepath{clip}%
\pgfsetbuttcap%
\pgfsetroundjoin%
\definecolor{currentfill}{rgb}{0.121569,0.466667,0.705882}%
\pgfsetfillcolor{currentfill}%
\pgfsetfillopacity{0.307800}%
\pgfsetlinewidth{1.003750pt}%
\definecolor{currentstroke}{rgb}{0.121569,0.466667,0.705882}%
\pgfsetstrokecolor{currentstroke}%
\pgfsetstrokeopacity{0.307800}%
\pgfsetdash{}{0pt}%
\pgfpathmoveto{\pgfqpoint{1.733905in}{2.349403in}}%
\pgfpathcurveto{\pgfqpoint{1.742141in}{2.349403in}}{\pgfqpoint{1.750041in}{2.352676in}}{\pgfqpoint{1.755865in}{2.358500in}}%
\pgfpathcurveto{\pgfqpoint{1.761689in}{2.364324in}}{\pgfqpoint{1.764961in}{2.372224in}}{\pgfqpoint{1.764961in}{2.380460in}}%
\pgfpathcurveto{\pgfqpoint{1.764961in}{2.388696in}}{\pgfqpoint{1.761689in}{2.396596in}}{\pgfqpoint{1.755865in}{2.402420in}}%
\pgfpathcurveto{\pgfqpoint{1.750041in}{2.408244in}}{\pgfqpoint{1.742141in}{2.411516in}}{\pgfqpoint{1.733905in}{2.411516in}}%
\pgfpathcurveto{\pgfqpoint{1.725668in}{2.411516in}}{\pgfqpoint{1.717768in}{2.408244in}}{\pgfqpoint{1.711944in}{2.402420in}}%
\pgfpathcurveto{\pgfqpoint{1.706120in}{2.396596in}}{\pgfqpoint{1.702848in}{2.388696in}}{\pgfqpoint{1.702848in}{2.380460in}}%
\pgfpathcurveto{\pgfqpoint{1.702848in}{2.372224in}}{\pgfqpoint{1.706120in}{2.364324in}}{\pgfqpoint{1.711944in}{2.358500in}}%
\pgfpathcurveto{\pgfqpoint{1.717768in}{2.352676in}}{\pgfqpoint{1.725668in}{2.349403in}}{\pgfqpoint{1.733905in}{2.349403in}}%
\pgfpathclose%
\pgfusepath{stroke,fill}%
\end{pgfscope}%
\begin{pgfscope}%
\pgfpathrectangle{\pgfqpoint{0.100000in}{0.212622in}}{\pgfqpoint{3.696000in}{3.696000in}}%
\pgfusepath{clip}%
\pgfsetbuttcap%
\pgfsetroundjoin%
\definecolor{currentfill}{rgb}{0.121569,0.466667,0.705882}%
\pgfsetfillcolor{currentfill}%
\pgfsetfillopacity{0.308733}%
\pgfsetlinewidth{1.003750pt}%
\definecolor{currentstroke}{rgb}{0.121569,0.466667,0.705882}%
\pgfsetstrokecolor{currentstroke}%
\pgfsetstrokeopacity{0.308733}%
\pgfsetdash{}{0pt}%
\pgfpathmoveto{\pgfqpoint{1.627511in}{2.313025in}}%
\pgfpathcurveto{\pgfqpoint{1.635747in}{2.313025in}}{\pgfqpoint{1.643647in}{2.316298in}}{\pgfqpoint{1.649471in}{2.322122in}}%
\pgfpathcurveto{\pgfqpoint{1.655295in}{2.327946in}}{\pgfqpoint{1.658567in}{2.335846in}}{\pgfqpoint{1.658567in}{2.344082in}}%
\pgfpathcurveto{\pgfqpoint{1.658567in}{2.352318in}}{\pgfqpoint{1.655295in}{2.360218in}}{\pgfqpoint{1.649471in}{2.366042in}}%
\pgfpathcurveto{\pgfqpoint{1.643647in}{2.371866in}}{\pgfqpoint{1.635747in}{2.375138in}}{\pgfqpoint{1.627511in}{2.375138in}}%
\pgfpathcurveto{\pgfqpoint{1.619275in}{2.375138in}}{\pgfqpoint{1.611374in}{2.371866in}}{\pgfqpoint{1.605551in}{2.366042in}}%
\pgfpathcurveto{\pgfqpoint{1.599727in}{2.360218in}}{\pgfqpoint{1.596454in}{2.352318in}}{\pgfqpoint{1.596454in}{2.344082in}}%
\pgfpathcurveto{\pgfqpoint{1.596454in}{2.335846in}}{\pgfqpoint{1.599727in}{2.327946in}}{\pgfqpoint{1.605551in}{2.322122in}}%
\pgfpathcurveto{\pgfqpoint{1.611374in}{2.316298in}}{\pgfqpoint{1.619275in}{2.313025in}}{\pgfqpoint{1.627511in}{2.313025in}}%
\pgfpathclose%
\pgfusepath{stroke,fill}%
\end{pgfscope}%
\begin{pgfscope}%
\pgfpathrectangle{\pgfqpoint{0.100000in}{0.212622in}}{\pgfqpoint{3.696000in}{3.696000in}}%
\pgfusepath{clip}%
\pgfsetbuttcap%
\pgfsetroundjoin%
\definecolor{currentfill}{rgb}{0.121569,0.466667,0.705882}%
\pgfsetfillcolor{currentfill}%
\pgfsetfillopacity{0.310918}%
\pgfsetlinewidth{1.003750pt}%
\definecolor{currentstroke}{rgb}{0.121569,0.466667,0.705882}%
\pgfsetstrokecolor{currentstroke}%
\pgfsetstrokeopacity{0.310918}%
\pgfsetdash{}{0pt}%
\pgfpathmoveto{\pgfqpoint{1.755907in}{2.344792in}}%
\pgfpathcurveto{\pgfqpoint{1.764143in}{2.344792in}}{\pgfqpoint{1.772043in}{2.348065in}}{\pgfqpoint{1.777867in}{2.353889in}}%
\pgfpathcurveto{\pgfqpoint{1.783691in}{2.359712in}}{\pgfqpoint{1.786963in}{2.367613in}}{\pgfqpoint{1.786963in}{2.375849in}}%
\pgfpathcurveto{\pgfqpoint{1.786963in}{2.384085in}}{\pgfqpoint{1.783691in}{2.391985in}}{\pgfqpoint{1.777867in}{2.397809in}}%
\pgfpathcurveto{\pgfqpoint{1.772043in}{2.403633in}}{\pgfqpoint{1.764143in}{2.406905in}}{\pgfqpoint{1.755907in}{2.406905in}}%
\pgfpathcurveto{\pgfqpoint{1.747671in}{2.406905in}}{\pgfqpoint{1.739771in}{2.403633in}}{\pgfqpoint{1.733947in}{2.397809in}}%
\pgfpathcurveto{\pgfqpoint{1.728123in}{2.391985in}}{\pgfqpoint{1.724850in}{2.384085in}}{\pgfqpoint{1.724850in}{2.375849in}}%
\pgfpathcurveto{\pgfqpoint{1.724850in}{2.367613in}}{\pgfqpoint{1.728123in}{2.359712in}}{\pgfqpoint{1.733947in}{2.353889in}}%
\pgfpathcurveto{\pgfqpoint{1.739771in}{2.348065in}}{\pgfqpoint{1.747671in}{2.344792in}}{\pgfqpoint{1.755907in}{2.344792in}}%
\pgfpathclose%
\pgfusepath{stroke,fill}%
\end{pgfscope}%
\begin{pgfscope}%
\pgfpathrectangle{\pgfqpoint{0.100000in}{0.212622in}}{\pgfqpoint{3.696000in}{3.696000in}}%
\pgfusepath{clip}%
\pgfsetbuttcap%
\pgfsetroundjoin%
\definecolor{currentfill}{rgb}{0.121569,0.466667,0.705882}%
\pgfsetfillcolor{currentfill}%
\pgfsetfillopacity{0.312739}%
\pgfsetlinewidth{1.003750pt}%
\definecolor{currentstroke}{rgb}{0.121569,0.466667,0.705882}%
\pgfsetstrokecolor{currentstroke}%
\pgfsetstrokeopacity{0.312739}%
\pgfsetdash{}{0pt}%
\pgfpathmoveto{\pgfqpoint{1.767961in}{2.342813in}}%
\pgfpathcurveto{\pgfqpoint{1.776198in}{2.342813in}}{\pgfqpoint{1.784098in}{2.346085in}}{\pgfqpoint{1.789921in}{2.351909in}}%
\pgfpathcurveto{\pgfqpoint{1.795745in}{2.357733in}}{\pgfqpoint{1.799018in}{2.365633in}}{\pgfqpoint{1.799018in}{2.373870in}}%
\pgfpathcurveto{\pgfqpoint{1.799018in}{2.382106in}}{\pgfqpoint{1.795745in}{2.390006in}}{\pgfqpoint{1.789921in}{2.395830in}}%
\pgfpathcurveto{\pgfqpoint{1.784098in}{2.401654in}}{\pgfqpoint{1.776198in}{2.404926in}}{\pgfqpoint{1.767961in}{2.404926in}}%
\pgfpathcurveto{\pgfqpoint{1.759725in}{2.404926in}}{\pgfqpoint{1.751825in}{2.401654in}}{\pgfqpoint{1.746001in}{2.395830in}}%
\pgfpathcurveto{\pgfqpoint{1.740177in}{2.390006in}}{\pgfqpoint{1.736905in}{2.382106in}}{\pgfqpoint{1.736905in}{2.373870in}}%
\pgfpathcurveto{\pgfqpoint{1.736905in}{2.365633in}}{\pgfqpoint{1.740177in}{2.357733in}}{\pgfqpoint{1.746001in}{2.351909in}}%
\pgfpathcurveto{\pgfqpoint{1.751825in}{2.346085in}}{\pgfqpoint{1.759725in}{2.342813in}}{\pgfqpoint{1.767961in}{2.342813in}}%
\pgfpathclose%
\pgfusepath{stroke,fill}%
\end{pgfscope}%
\begin{pgfscope}%
\pgfpathrectangle{\pgfqpoint{0.100000in}{0.212622in}}{\pgfqpoint{3.696000in}{3.696000in}}%
\pgfusepath{clip}%
\pgfsetbuttcap%
\pgfsetroundjoin%
\definecolor{currentfill}{rgb}{0.121569,0.466667,0.705882}%
\pgfsetfillcolor{currentfill}%
\pgfsetfillopacity{0.315087}%
\pgfsetlinewidth{1.003750pt}%
\definecolor{currentstroke}{rgb}{0.121569,0.466667,0.705882}%
\pgfsetstrokecolor{currentstroke}%
\pgfsetstrokeopacity{0.315087}%
\pgfsetdash{}{0pt}%
\pgfpathmoveto{\pgfqpoint{1.609202in}{2.294138in}}%
\pgfpathcurveto{\pgfqpoint{1.617438in}{2.294138in}}{\pgfqpoint{1.625338in}{2.297410in}}{\pgfqpoint{1.631162in}{2.303234in}}%
\pgfpathcurveto{\pgfqpoint{1.636986in}{2.309058in}}{\pgfqpoint{1.640258in}{2.316958in}}{\pgfqpoint{1.640258in}{2.325194in}}%
\pgfpathcurveto{\pgfqpoint{1.640258in}{2.333431in}}{\pgfqpoint{1.636986in}{2.341331in}}{\pgfqpoint{1.631162in}{2.347155in}}%
\pgfpathcurveto{\pgfqpoint{1.625338in}{2.352979in}}{\pgfqpoint{1.617438in}{2.356251in}}{\pgfqpoint{1.609202in}{2.356251in}}%
\pgfpathcurveto{\pgfqpoint{1.600965in}{2.356251in}}{\pgfqpoint{1.593065in}{2.352979in}}{\pgfqpoint{1.587241in}{2.347155in}}%
\pgfpathcurveto{\pgfqpoint{1.581417in}{2.341331in}}{\pgfqpoint{1.578145in}{2.333431in}}{\pgfqpoint{1.578145in}{2.325194in}}%
\pgfpathcurveto{\pgfqpoint{1.578145in}{2.316958in}}{\pgfqpoint{1.581417in}{2.309058in}}{\pgfqpoint{1.587241in}{2.303234in}}%
\pgfpathcurveto{\pgfqpoint{1.593065in}{2.297410in}}{\pgfqpoint{1.600965in}{2.294138in}}{\pgfqpoint{1.609202in}{2.294138in}}%
\pgfpathclose%
\pgfusepath{stroke,fill}%
\end{pgfscope}%
\begin{pgfscope}%
\pgfpathrectangle{\pgfqpoint{0.100000in}{0.212622in}}{\pgfqpoint{3.696000in}{3.696000in}}%
\pgfusepath{clip}%
\pgfsetbuttcap%
\pgfsetroundjoin%
\definecolor{currentfill}{rgb}{0.121569,0.466667,0.705882}%
\pgfsetfillcolor{currentfill}%
\pgfsetfillopacity{0.315432}%
\pgfsetlinewidth{1.003750pt}%
\definecolor{currentstroke}{rgb}{0.121569,0.466667,0.705882}%
\pgfsetstrokecolor{currentstroke}%
\pgfsetstrokeopacity{0.315432}%
\pgfsetdash{}{0pt}%
\pgfpathmoveto{\pgfqpoint{1.782353in}{2.341738in}}%
\pgfpathcurveto{\pgfqpoint{1.790590in}{2.341738in}}{\pgfqpoint{1.798490in}{2.345010in}}{\pgfqpoint{1.804314in}{2.350834in}}%
\pgfpathcurveto{\pgfqpoint{1.810137in}{2.356658in}}{\pgfqpoint{1.813410in}{2.364558in}}{\pgfqpoint{1.813410in}{2.372794in}}%
\pgfpathcurveto{\pgfqpoint{1.813410in}{2.381030in}}{\pgfqpoint{1.810137in}{2.388930in}}{\pgfqpoint{1.804314in}{2.394754in}}%
\pgfpathcurveto{\pgfqpoint{1.798490in}{2.400578in}}{\pgfqpoint{1.790590in}{2.403851in}}{\pgfqpoint{1.782353in}{2.403851in}}%
\pgfpathcurveto{\pgfqpoint{1.774117in}{2.403851in}}{\pgfqpoint{1.766217in}{2.400578in}}{\pgfqpoint{1.760393in}{2.394754in}}%
\pgfpathcurveto{\pgfqpoint{1.754569in}{2.388930in}}{\pgfqpoint{1.751297in}{2.381030in}}{\pgfqpoint{1.751297in}{2.372794in}}%
\pgfpathcurveto{\pgfqpoint{1.751297in}{2.364558in}}{\pgfqpoint{1.754569in}{2.356658in}}{\pgfqpoint{1.760393in}{2.350834in}}%
\pgfpathcurveto{\pgfqpoint{1.766217in}{2.345010in}}{\pgfqpoint{1.774117in}{2.341738in}}{\pgfqpoint{1.782353in}{2.341738in}}%
\pgfpathclose%
\pgfusepath{stroke,fill}%
\end{pgfscope}%
\begin{pgfscope}%
\pgfpathrectangle{\pgfqpoint{0.100000in}{0.212622in}}{\pgfqpoint{3.696000in}{3.696000in}}%
\pgfusepath{clip}%
\pgfsetbuttcap%
\pgfsetroundjoin%
\definecolor{currentfill}{rgb}{0.121569,0.466667,0.705882}%
\pgfsetfillcolor{currentfill}%
\pgfsetfillopacity{0.317105}%
\pgfsetlinewidth{1.003750pt}%
\definecolor{currentstroke}{rgb}{0.121569,0.466667,0.705882}%
\pgfsetstrokecolor{currentstroke}%
\pgfsetstrokeopacity{0.317105}%
\pgfsetdash{}{0pt}%
\pgfpathmoveto{\pgfqpoint{1.790160in}{2.342177in}}%
\pgfpathcurveto{\pgfqpoint{1.798396in}{2.342177in}}{\pgfqpoint{1.806296in}{2.345449in}}{\pgfqpoint{1.812120in}{2.351273in}}%
\pgfpathcurveto{\pgfqpoint{1.817944in}{2.357097in}}{\pgfqpoint{1.821216in}{2.364997in}}{\pgfqpoint{1.821216in}{2.373233in}}%
\pgfpathcurveto{\pgfqpoint{1.821216in}{2.381469in}}{\pgfqpoint{1.817944in}{2.389369in}}{\pgfqpoint{1.812120in}{2.395193in}}%
\pgfpathcurveto{\pgfqpoint{1.806296in}{2.401017in}}{\pgfqpoint{1.798396in}{2.404290in}}{\pgfqpoint{1.790160in}{2.404290in}}%
\pgfpathcurveto{\pgfqpoint{1.781924in}{2.404290in}}{\pgfqpoint{1.774024in}{2.401017in}}{\pgfqpoint{1.768200in}{2.395193in}}%
\pgfpathcurveto{\pgfqpoint{1.762376in}{2.389369in}}{\pgfqpoint{1.759103in}{2.381469in}}{\pgfqpoint{1.759103in}{2.373233in}}%
\pgfpathcurveto{\pgfqpoint{1.759103in}{2.364997in}}{\pgfqpoint{1.762376in}{2.357097in}}{\pgfqpoint{1.768200in}{2.351273in}}%
\pgfpathcurveto{\pgfqpoint{1.774024in}{2.345449in}}{\pgfqpoint{1.781924in}{2.342177in}}{\pgfqpoint{1.790160in}{2.342177in}}%
\pgfpathclose%
\pgfusepath{stroke,fill}%
\end{pgfscope}%
\begin{pgfscope}%
\pgfpathrectangle{\pgfqpoint{0.100000in}{0.212622in}}{\pgfqpoint{3.696000in}{3.696000in}}%
\pgfusepath{clip}%
\pgfsetbuttcap%
\pgfsetroundjoin%
\definecolor{currentfill}{rgb}{0.121569,0.466667,0.705882}%
\pgfsetfillcolor{currentfill}%
\pgfsetfillopacity{0.318823}%
\pgfsetlinewidth{1.003750pt}%
\definecolor{currentstroke}{rgb}{0.121569,0.466667,0.705882}%
\pgfsetstrokecolor{currentstroke}%
\pgfsetstrokeopacity{0.318823}%
\pgfsetdash{}{0pt}%
\pgfpathmoveto{\pgfqpoint{1.800055in}{2.340775in}}%
\pgfpathcurveto{\pgfqpoint{1.808291in}{2.340775in}}{\pgfqpoint{1.816191in}{2.344047in}}{\pgfqpoint{1.822015in}{2.349871in}}%
\pgfpathcurveto{\pgfqpoint{1.827839in}{2.355695in}}{\pgfqpoint{1.831111in}{2.363595in}}{\pgfqpoint{1.831111in}{2.371832in}}%
\pgfpathcurveto{\pgfqpoint{1.831111in}{2.380068in}}{\pgfqpoint{1.827839in}{2.387968in}}{\pgfqpoint{1.822015in}{2.393792in}}%
\pgfpathcurveto{\pgfqpoint{1.816191in}{2.399616in}}{\pgfqpoint{1.808291in}{2.402888in}}{\pgfqpoint{1.800055in}{2.402888in}}%
\pgfpathcurveto{\pgfqpoint{1.791818in}{2.402888in}}{\pgfqpoint{1.783918in}{2.399616in}}{\pgfqpoint{1.778094in}{2.393792in}}%
\pgfpathcurveto{\pgfqpoint{1.772270in}{2.387968in}}{\pgfqpoint{1.768998in}{2.380068in}}{\pgfqpoint{1.768998in}{2.371832in}}%
\pgfpathcurveto{\pgfqpoint{1.768998in}{2.363595in}}{\pgfqpoint{1.772270in}{2.355695in}}{\pgfqpoint{1.778094in}{2.349871in}}%
\pgfpathcurveto{\pgfqpoint{1.783918in}{2.344047in}}{\pgfqpoint{1.791818in}{2.340775in}}{\pgfqpoint{1.800055in}{2.340775in}}%
\pgfpathclose%
\pgfusepath{stroke,fill}%
\end{pgfscope}%
\begin{pgfscope}%
\pgfpathrectangle{\pgfqpoint{0.100000in}{0.212622in}}{\pgfqpoint{3.696000in}{3.696000in}}%
\pgfusepath{clip}%
\pgfsetbuttcap%
\pgfsetroundjoin%
\definecolor{currentfill}{rgb}{0.121569,0.466667,0.705882}%
\pgfsetfillcolor{currentfill}%
\pgfsetfillopacity{0.320230}%
\pgfsetlinewidth{1.003750pt}%
\definecolor{currentstroke}{rgb}{0.121569,0.466667,0.705882}%
\pgfsetstrokecolor{currentstroke}%
\pgfsetstrokeopacity{0.320230}%
\pgfsetdash{}{0pt}%
\pgfpathmoveto{\pgfqpoint{1.594926in}{2.274195in}}%
\pgfpathcurveto{\pgfqpoint{1.603163in}{2.274195in}}{\pgfqpoint{1.611063in}{2.277467in}}{\pgfqpoint{1.616887in}{2.283291in}}%
\pgfpathcurveto{\pgfqpoint{1.622711in}{2.289115in}}{\pgfqpoint{1.625983in}{2.297015in}}{\pgfqpoint{1.625983in}{2.305251in}}%
\pgfpathcurveto{\pgfqpoint{1.625983in}{2.313487in}}{\pgfqpoint{1.622711in}{2.321387in}}{\pgfqpoint{1.616887in}{2.327211in}}%
\pgfpathcurveto{\pgfqpoint{1.611063in}{2.333035in}}{\pgfqpoint{1.603163in}{2.336308in}}{\pgfqpoint{1.594926in}{2.336308in}}%
\pgfpathcurveto{\pgfqpoint{1.586690in}{2.336308in}}{\pgfqpoint{1.578790in}{2.333035in}}{\pgfqpoint{1.572966in}{2.327211in}}%
\pgfpathcurveto{\pgfqpoint{1.567142in}{2.321387in}}{\pgfqpoint{1.563870in}{2.313487in}}{\pgfqpoint{1.563870in}{2.305251in}}%
\pgfpathcurveto{\pgfqpoint{1.563870in}{2.297015in}}{\pgfqpoint{1.567142in}{2.289115in}}{\pgfqpoint{1.572966in}{2.283291in}}%
\pgfpathcurveto{\pgfqpoint{1.578790in}{2.277467in}}{\pgfqpoint{1.586690in}{2.274195in}}{\pgfqpoint{1.594926in}{2.274195in}}%
\pgfpathclose%
\pgfusepath{stroke,fill}%
\end{pgfscope}%
\begin{pgfscope}%
\pgfpathrectangle{\pgfqpoint{0.100000in}{0.212622in}}{\pgfqpoint{3.696000in}{3.696000in}}%
\pgfusepath{clip}%
\pgfsetbuttcap%
\pgfsetroundjoin%
\definecolor{currentfill}{rgb}{0.121569,0.466667,0.705882}%
\pgfsetfillcolor{currentfill}%
\pgfsetfillopacity{0.322550}%
\pgfsetlinewidth{1.003750pt}%
\definecolor{currentstroke}{rgb}{0.121569,0.466667,0.705882}%
\pgfsetstrokecolor{currentstroke}%
\pgfsetstrokeopacity{0.322550}%
\pgfsetdash{}{0pt}%
\pgfpathmoveto{\pgfqpoint{1.818161in}{2.339507in}}%
\pgfpathcurveto{\pgfqpoint{1.826397in}{2.339507in}}{\pgfqpoint{1.834297in}{2.342780in}}{\pgfqpoint{1.840121in}{2.348604in}}%
\pgfpathcurveto{\pgfqpoint{1.845945in}{2.354427in}}{\pgfqpoint{1.849217in}{2.362328in}}{\pgfqpoint{1.849217in}{2.370564in}}%
\pgfpathcurveto{\pgfqpoint{1.849217in}{2.378800in}}{\pgfqpoint{1.845945in}{2.386700in}}{\pgfqpoint{1.840121in}{2.392524in}}%
\pgfpathcurveto{\pgfqpoint{1.834297in}{2.398348in}}{\pgfqpoint{1.826397in}{2.401620in}}{\pgfqpoint{1.818161in}{2.401620in}}%
\pgfpathcurveto{\pgfqpoint{1.809925in}{2.401620in}}{\pgfqpoint{1.802025in}{2.398348in}}{\pgfqpoint{1.796201in}{2.392524in}}%
\pgfpathcurveto{\pgfqpoint{1.790377in}{2.386700in}}{\pgfqpoint{1.787104in}{2.378800in}}{\pgfqpoint{1.787104in}{2.370564in}}%
\pgfpathcurveto{\pgfqpoint{1.787104in}{2.362328in}}{\pgfqpoint{1.790377in}{2.354427in}}{\pgfqpoint{1.796201in}{2.348604in}}%
\pgfpathcurveto{\pgfqpoint{1.802025in}{2.342780in}}{\pgfqpoint{1.809925in}{2.339507in}}{\pgfqpoint{1.818161in}{2.339507in}}%
\pgfpathclose%
\pgfusepath{stroke,fill}%
\end{pgfscope}%
\begin{pgfscope}%
\pgfpathrectangle{\pgfqpoint{0.100000in}{0.212622in}}{\pgfqpoint{3.696000in}{3.696000in}}%
\pgfusepath{clip}%
\pgfsetbuttcap%
\pgfsetroundjoin%
\definecolor{currentfill}{rgb}{0.121569,0.466667,0.705882}%
\pgfsetfillcolor{currentfill}%
\pgfsetfillopacity{0.325086}%
\pgfsetlinewidth{1.003750pt}%
\definecolor{currentstroke}{rgb}{0.121569,0.466667,0.705882}%
\pgfsetstrokecolor{currentstroke}%
\pgfsetstrokeopacity{0.325086}%
\pgfsetdash{}{0pt}%
\pgfpathmoveto{\pgfqpoint{1.582011in}{2.262096in}}%
\pgfpathcurveto{\pgfqpoint{1.590248in}{2.262096in}}{\pgfqpoint{1.598148in}{2.265368in}}{\pgfqpoint{1.603972in}{2.271192in}}%
\pgfpathcurveto{\pgfqpoint{1.609796in}{2.277016in}}{\pgfqpoint{1.613068in}{2.284916in}}{\pgfqpoint{1.613068in}{2.293152in}}%
\pgfpathcurveto{\pgfqpoint{1.613068in}{2.301389in}}{\pgfqpoint{1.609796in}{2.309289in}}{\pgfqpoint{1.603972in}{2.315113in}}%
\pgfpathcurveto{\pgfqpoint{1.598148in}{2.320937in}}{\pgfqpoint{1.590248in}{2.324209in}}{\pgfqpoint{1.582011in}{2.324209in}}%
\pgfpathcurveto{\pgfqpoint{1.573775in}{2.324209in}}{\pgfqpoint{1.565875in}{2.320937in}}{\pgfqpoint{1.560051in}{2.315113in}}%
\pgfpathcurveto{\pgfqpoint{1.554227in}{2.309289in}}{\pgfqpoint{1.550955in}{2.301389in}}{\pgfqpoint{1.550955in}{2.293152in}}%
\pgfpathcurveto{\pgfqpoint{1.550955in}{2.284916in}}{\pgfqpoint{1.554227in}{2.277016in}}{\pgfqpoint{1.560051in}{2.271192in}}%
\pgfpathcurveto{\pgfqpoint{1.565875in}{2.265368in}}{\pgfqpoint{1.573775in}{2.262096in}}{\pgfqpoint{1.582011in}{2.262096in}}%
\pgfpathclose%
\pgfusepath{stroke,fill}%
\end{pgfscope}%
\begin{pgfscope}%
\pgfpathrectangle{\pgfqpoint{0.100000in}{0.212622in}}{\pgfqpoint{3.696000in}{3.696000in}}%
\pgfusepath{clip}%
\pgfsetbuttcap%
\pgfsetroundjoin%
\definecolor{currentfill}{rgb}{0.121569,0.466667,0.705882}%
\pgfsetfillcolor{currentfill}%
\pgfsetfillopacity{0.326157}%
\pgfsetlinewidth{1.003750pt}%
\definecolor{currentstroke}{rgb}{0.121569,0.466667,0.705882}%
\pgfsetstrokecolor{currentstroke}%
\pgfsetstrokeopacity{0.326157}%
\pgfsetdash{}{0pt}%
\pgfpathmoveto{\pgfqpoint{1.842617in}{2.332459in}}%
\pgfpathcurveto{\pgfqpoint{1.850853in}{2.332459in}}{\pgfqpoint{1.858753in}{2.335731in}}{\pgfqpoint{1.864577in}{2.341555in}}%
\pgfpathcurveto{\pgfqpoint{1.870401in}{2.347379in}}{\pgfqpoint{1.873674in}{2.355279in}}{\pgfqpoint{1.873674in}{2.363516in}}%
\pgfpathcurveto{\pgfqpoint{1.873674in}{2.371752in}}{\pgfqpoint{1.870401in}{2.379652in}}{\pgfqpoint{1.864577in}{2.385476in}}%
\pgfpathcurveto{\pgfqpoint{1.858753in}{2.391300in}}{\pgfqpoint{1.850853in}{2.394572in}}{\pgfqpoint{1.842617in}{2.394572in}}%
\pgfpathcurveto{\pgfqpoint{1.834381in}{2.394572in}}{\pgfqpoint{1.826481in}{2.391300in}}{\pgfqpoint{1.820657in}{2.385476in}}%
\pgfpathcurveto{\pgfqpoint{1.814833in}{2.379652in}}{\pgfqpoint{1.811561in}{2.371752in}}{\pgfqpoint{1.811561in}{2.363516in}}%
\pgfpathcurveto{\pgfqpoint{1.811561in}{2.355279in}}{\pgfqpoint{1.814833in}{2.347379in}}{\pgfqpoint{1.820657in}{2.341555in}}%
\pgfpathcurveto{\pgfqpoint{1.826481in}{2.335731in}}{\pgfqpoint{1.834381in}{2.332459in}}{\pgfqpoint{1.842617in}{2.332459in}}%
\pgfpathclose%
\pgfusepath{stroke,fill}%
\end{pgfscope}%
\begin{pgfscope}%
\pgfpathrectangle{\pgfqpoint{0.100000in}{0.212622in}}{\pgfqpoint{3.696000in}{3.696000in}}%
\pgfusepath{clip}%
\pgfsetbuttcap%
\pgfsetroundjoin%
\definecolor{currentfill}{rgb}{0.121569,0.466667,0.705882}%
\pgfsetfillcolor{currentfill}%
\pgfsetfillopacity{0.328086}%
\pgfsetlinewidth{1.003750pt}%
\definecolor{currentstroke}{rgb}{0.121569,0.466667,0.705882}%
\pgfsetstrokecolor{currentstroke}%
\pgfsetstrokeopacity{0.328086}%
\pgfsetdash{}{0pt}%
\pgfpathmoveto{\pgfqpoint{1.574186in}{2.252022in}}%
\pgfpathcurveto{\pgfqpoint{1.582423in}{2.252022in}}{\pgfqpoint{1.590323in}{2.255294in}}{\pgfqpoint{1.596147in}{2.261118in}}%
\pgfpathcurveto{\pgfqpoint{1.601970in}{2.266942in}}{\pgfqpoint{1.605243in}{2.274842in}}{\pgfqpoint{1.605243in}{2.283079in}}%
\pgfpathcurveto{\pgfqpoint{1.605243in}{2.291315in}}{\pgfqpoint{1.601970in}{2.299215in}}{\pgfqpoint{1.596147in}{2.305039in}}%
\pgfpathcurveto{\pgfqpoint{1.590323in}{2.310863in}}{\pgfqpoint{1.582423in}{2.314135in}}{\pgfqpoint{1.574186in}{2.314135in}}%
\pgfpathcurveto{\pgfqpoint{1.565950in}{2.314135in}}{\pgfqpoint{1.558050in}{2.310863in}}{\pgfqpoint{1.552226in}{2.305039in}}%
\pgfpathcurveto{\pgfqpoint{1.546402in}{2.299215in}}{\pgfqpoint{1.543130in}{2.291315in}}{\pgfqpoint{1.543130in}{2.283079in}}%
\pgfpathcurveto{\pgfqpoint{1.543130in}{2.274842in}}{\pgfqpoint{1.546402in}{2.266942in}}{\pgfqpoint{1.552226in}{2.261118in}}%
\pgfpathcurveto{\pgfqpoint{1.558050in}{2.255294in}}{\pgfqpoint{1.565950in}{2.252022in}}{\pgfqpoint{1.574186in}{2.252022in}}%
\pgfpathclose%
\pgfusepath{stroke,fill}%
\end{pgfscope}%
\begin{pgfscope}%
\pgfpathrectangle{\pgfqpoint{0.100000in}{0.212622in}}{\pgfqpoint{3.696000in}{3.696000in}}%
\pgfusepath{clip}%
\pgfsetbuttcap%
\pgfsetroundjoin%
\definecolor{currentfill}{rgb}{0.121569,0.466667,0.705882}%
\pgfsetfillcolor{currentfill}%
\pgfsetfillopacity{0.329762}%
\pgfsetlinewidth{1.003750pt}%
\definecolor{currentstroke}{rgb}{0.121569,0.466667,0.705882}%
\pgfsetstrokecolor{currentstroke}%
\pgfsetstrokeopacity{0.329762}%
\pgfsetdash{}{0pt}%
\pgfpathmoveto{\pgfqpoint{1.569069in}{2.247830in}}%
\pgfpathcurveto{\pgfqpoint{1.577305in}{2.247830in}}{\pgfqpoint{1.585205in}{2.251102in}}{\pgfqpoint{1.591029in}{2.256926in}}%
\pgfpathcurveto{\pgfqpoint{1.596853in}{2.262750in}}{\pgfqpoint{1.600125in}{2.270650in}}{\pgfqpoint{1.600125in}{2.278887in}}%
\pgfpathcurveto{\pgfqpoint{1.600125in}{2.287123in}}{\pgfqpoint{1.596853in}{2.295023in}}{\pgfqpoint{1.591029in}{2.300847in}}%
\pgfpathcurveto{\pgfqpoint{1.585205in}{2.306671in}}{\pgfqpoint{1.577305in}{2.309943in}}{\pgfqpoint{1.569069in}{2.309943in}}%
\pgfpathcurveto{\pgfqpoint{1.560832in}{2.309943in}}{\pgfqpoint{1.552932in}{2.306671in}}{\pgfqpoint{1.547108in}{2.300847in}}%
\pgfpathcurveto{\pgfqpoint{1.541285in}{2.295023in}}{\pgfqpoint{1.538012in}{2.287123in}}{\pgfqpoint{1.538012in}{2.278887in}}%
\pgfpathcurveto{\pgfqpoint{1.538012in}{2.270650in}}{\pgfqpoint{1.541285in}{2.262750in}}{\pgfqpoint{1.547108in}{2.256926in}}%
\pgfpathcurveto{\pgfqpoint{1.552932in}{2.251102in}}{\pgfqpoint{1.560832in}{2.247830in}}{\pgfqpoint{1.569069in}{2.247830in}}%
\pgfpathclose%
\pgfusepath{stroke,fill}%
\end{pgfscope}%
\begin{pgfscope}%
\pgfpathrectangle{\pgfqpoint{0.100000in}{0.212622in}}{\pgfqpoint{3.696000in}{3.696000in}}%
\pgfusepath{clip}%
\pgfsetbuttcap%
\pgfsetroundjoin%
\definecolor{currentfill}{rgb}{0.121569,0.466667,0.705882}%
\pgfsetfillcolor{currentfill}%
\pgfsetfillopacity{0.330025}%
\pgfsetlinewidth{1.003750pt}%
\definecolor{currentstroke}{rgb}{0.121569,0.466667,0.705882}%
\pgfsetstrokecolor{currentstroke}%
\pgfsetstrokeopacity{0.330025}%
\pgfsetdash{}{0pt}%
\pgfpathmoveto{\pgfqpoint{1.568468in}{2.246750in}}%
\pgfpathcurveto{\pgfqpoint{1.576704in}{2.246750in}}{\pgfqpoint{1.584604in}{2.250022in}}{\pgfqpoint{1.590428in}{2.255846in}}%
\pgfpathcurveto{\pgfqpoint{1.596252in}{2.261670in}}{\pgfqpoint{1.599525in}{2.269570in}}{\pgfqpoint{1.599525in}{2.277807in}}%
\pgfpathcurveto{\pgfqpoint{1.599525in}{2.286043in}}{\pgfqpoint{1.596252in}{2.293943in}}{\pgfqpoint{1.590428in}{2.299767in}}%
\pgfpathcurveto{\pgfqpoint{1.584604in}{2.305591in}}{\pgfqpoint{1.576704in}{2.308863in}}{\pgfqpoint{1.568468in}{2.308863in}}%
\pgfpathcurveto{\pgfqpoint{1.560232in}{2.308863in}}{\pgfqpoint{1.552332in}{2.305591in}}{\pgfqpoint{1.546508in}{2.299767in}}%
\pgfpathcurveto{\pgfqpoint{1.540684in}{2.293943in}}{\pgfqpoint{1.537412in}{2.286043in}}{\pgfqpoint{1.537412in}{2.277807in}}%
\pgfpathcurveto{\pgfqpoint{1.537412in}{2.269570in}}{\pgfqpoint{1.540684in}{2.261670in}}{\pgfqpoint{1.546508in}{2.255846in}}%
\pgfpathcurveto{\pgfqpoint{1.552332in}{2.250022in}}{\pgfqpoint{1.560232in}{2.246750in}}{\pgfqpoint{1.568468in}{2.246750in}}%
\pgfpathclose%
\pgfusepath{stroke,fill}%
\end{pgfscope}%
\begin{pgfscope}%
\pgfpathrectangle{\pgfqpoint{0.100000in}{0.212622in}}{\pgfqpoint{3.696000in}{3.696000in}}%
\pgfusepath{clip}%
\pgfsetbuttcap%
\pgfsetroundjoin%
\definecolor{currentfill}{rgb}{0.121569,0.466667,0.705882}%
\pgfsetfillcolor{currentfill}%
\pgfsetfillopacity{0.330540}%
\pgfsetlinewidth{1.003750pt}%
\definecolor{currentstroke}{rgb}{0.121569,0.466667,0.705882}%
\pgfsetstrokecolor{currentstroke}%
\pgfsetstrokeopacity{0.330540}%
\pgfsetdash{}{0pt}%
\pgfpathmoveto{\pgfqpoint{1.566917in}{2.245504in}}%
\pgfpathcurveto{\pgfqpoint{1.575154in}{2.245504in}}{\pgfqpoint{1.583054in}{2.248776in}}{\pgfqpoint{1.588878in}{2.254600in}}%
\pgfpathcurveto{\pgfqpoint{1.594702in}{2.260424in}}{\pgfqpoint{1.597974in}{2.268324in}}{\pgfqpoint{1.597974in}{2.276560in}}%
\pgfpathcurveto{\pgfqpoint{1.597974in}{2.284796in}}{\pgfqpoint{1.594702in}{2.292696in}}{\pgfqpoint{1.588878in}{2.298520in}}%
\pgfpathcurveto{\pgfqpoint{1.583054in}{2.304344in}}{\pgfqpoint{1.575154in}{2.307617in}}{\pgfqpoint{1.566917in}{2.307617in}}%
\pgfpathcurveto{\pgfqpoint{1.558681in}{2.307617in}}{\pgfqpoint{1.550781in}{2.304344in}}{\pgfqpoint{1.544957in}{2.298520in}}%
\pgfpathcurveto{\pgfqpoint{1.539133in}{2.292696in}}{\pgfqpoint{1.535861in}{2.284796in}}{\pgfqpoint{1.535861in}{2.276560in}}%
\pgfpathcurveto{\pgfqpoint{1.535861in}{2.268324in}}{\pgfqpoint{1.539133in}{2.260424in}}{\pgfqpoint{1.544957in}{2.254600in}}%
\pgfpathcurveto{\pgfqpoint{1.550781in}{2.248776in}}{\pgfqpoint{1.558681in}{2.245504in}}{\pgfqpoint{1.566917in}{2.245504in}}%
\pgfpathclose%
\pgfusepath{stroke,fill}%
\end{pgfscope}%
\begin{pgfscope}%
\pgfpathrectangle{\pgfqpoint{0.100000in}{0.212622in}}{\pgfqpoint{3.696000in}{3.696000in}}%
\pgfusepath{clip}%
\pgfsetbuttcap%
\pgfsetroundjoin%
\definecolor{currentfill}{rgb}{0.121569,0.466667,0.705882}%
\pgfsetfillcolor{currentfill}%
\pgfsetfillopacity{0.331499}%
\pgfsetlinewidth{1.003750pt}%
\definecolor{currentstroke}{rgb}{0.121569,0.466667,0.705882}%
\pgfsetstrokecolor{currentstroke}%
\pgfsetstrokeopacity{0.331499}%
\pgfsetdash{}{0pt}%
\pgfpathmoveto{\pgfqpoint{1.565025in}{2.242518in}}%
\pgfpathcurveto{\pgfqpoint{1.573261in}{2.242518in}}{\pgfqpoint{1.581161in}{2.245791in}}{\pgfqpoint{1.586985in}{2.251615in}}%
\pgfpathcurveto{\pgfqpoint{1.592809in}{2.257439in}}{\pgfqpoint{1.596082in}{2.265339in}}{\pgfqpoint{1.596082in}{2.273575in}}%
\pgfpathcurveto{\pgfqpoint{1.596082in}{2.281811in}}{\pgfqpoint{1.592809in}{2.289711in}}{\pgfqpoint{1.586985in}{2.295535in}}%
\pgfpathcurveto{\pgfqpoint{1.581161in}{2.301359in}}{\pgfqpoint{1.573261in}{2.304631in}}{\pgfqpoint{1.565025in}{2.304631in}}%
\pgfpathcurveto{\pgfqpoint{1.556789in}{2.304631in}}{\pgfqpoint{1.548889in}{2.301359in}}{\pgfqpoint{1.543065in}{2.295535in}}%
\pgfpathcurveto{\pgfqpoint{1.537241in}{2.289711in}}{\pgfqpoint{1.533969in}{2.281811in}}{\pgfqpoint{1.533969in}{2.273575in}}%
\pgfpathcurveto{\pgfqpoint{1.533969in}{2.265339in}}{\pgfqpoint{1.537241in}{2.257439in}}{\pgfqpoint{1.543065in}{2.251615in}}%
\pgfpathcurveto{\pgfqpoint{1.548889in}{2.245791in}}{\pgfqpoint{1.556789in}{2.242518in}}{\pgfqpoint{1.565025in}{2.242518in}}%
\pgfpathclose%
\pgfusepath{stroke,fill}%
\end{pgfscope}%
\begin{pgfscope}%
\pgfpathrectangle{\pgfqpoint{0.100000in}{0.212622in}}{\pgfqpoint{3.696000in}{3.696000in}}%
\pgfusepath{clip}%
\pgfsetbuttcap%
\pgfsetroundjoin%
\definecolor{currentfill}{rgb}{0.121569,0.466667,0.705882}%
\pgfsetfillcolor{currentfill}%
\pgfsetfillopacity{0.331745}%
\pgfsetlinewidth{1.003750pt}%
\definecolor{currentstroke}{rgb}{0.121569,0.466667,0.705882}%
\pgfsetstrokecolor{currentstroke}%
\pgfsetstrokeopacity{0.331745}%
\pgfsetdash{}{0pt}%
\pgfpathmoveto{\pgfqpoint{1.873323in}{2.331010in}}%
\pgfpathcurveto{\pgfqpoint{1.881560in}{2.331010in}}{\pgfqpoint{1.889460in}{2.334282in}}{\pgfqpoint{1.895284in}{2.340106in}}%
\pgfpathcurveto{\pgfqpoint{1.901108in}{2.345930in}}{\pgfqpoint{1.904380in}{2.353830in}}{\pgfqpoint{1.904380in}{2.362067in}}%
\pgfpathcurveto{\pgfqpoint{1.904380in}{2.370303in}}{\pgfqpoint{1.901108in}{2.378203in}}{\pgfqpoint{1.895284in}{2.384027in}}%
\pgfpathcurveto{\pgfqpoint{1.889460in}{2.389851in}}{\pgfqpoint{1.881560in}{2.393123in}}{\pgfqpoint{1.873323in}{2.393123in}}%
\pgfpathcurveto{\pgfqpoint{1.865087in}{2.393123in}}{\pgfqpoint{1.857187in}{2.389851in}}{\pgfqpoint{1.851363in}{2.384027in}}%
\pgfpathcurveto{\pgfqpoint{1.845539in}{2.378203in}}{\pgfqpoint{1.842267in}{2.370303in}}{\pgfqpoint{1.842267in}{2.362067in}}%
\pgfpathcurveto{\pgfqpoint{1.842267in}{2.353830in}}{\pgfqpoint{1.845539in}{2.345930in}}{\pgfqpoint{1.851363in}{2.340106in}}%
\pgfpathcurveto{\pgfqpoint{1.857187in}{2.334282in}}{\pgfqpoint{1.865087in}{2.331010in}}{\pgfqpoint{1.873323in}{2.331010in}}%
\pgfpathclose%
\pgfusepath{stroke,fill}%
\end{pgfscope}%
\begin{pgfscope}%
\pgfpathrectangle{\pgfqpoint{0.100000in}{0.212622in}}{\pgfqpoint{3.696000in}{3.696000in}}%
\pgfusepath{clip}%
\pgfsetbuttcap%
\pgfsetroundjoin%
\definecolor{currentfill}{rgb}{0.121569,0.466667,0.705882}%
\pgfsetfillcolor{currentfill}%
\pgfsetfillopacity{0.333162}%
\pgfsetlinewidth{1.003750pt}%
\definecolor{currentstroke}{rgb}{0.121569,0.466667,0.705882}%
\pgfsetstrokecolor{currentstroke}%
\pgfsetstrokeopacity{0.333162}%
\pgfsetdash{}{0pt}%
\pgfpathmoveto{\pgfqpoint{1.561122in}{2.236856in}}%
\pgfpathcurveto{\pgfqpoint{1.569358in}{2.236856in}}{\pgfqpoint{1.577258in}{2.240129in}}{\pgfqpoint{1.583082in}{2.245952in}}%
\pgfpathcurveto{\pgfqpoint{1.588906in}{2.251776in}}{\pgfqpoint{1.592178in}{2.259676in}}{\pgfqpoint{1.592178in}{2.267913in}}%
\pgfpathcurveto{\pgfqpoint{1.592178in}{2.276149in}}{\pgfqpoint{1.588906in}{2.284049in}}{\pgfqpoint{1.583082in}{2.289873in}}%
\pgfpathcurveto{\pgfqpoint{1.577258in}{2.295697in}}{\pgfqpoint{1.569358in}{2.298969in}}{\pgfqpoint{1.561122in}{2.298969in}}%
\pgfpathcurveto{\pgfqpoint{1.552885in}{2.298969in}}{\pgfqpoint{1.544985in}{2.295697in}}{\pgfqpoint{1.539161in}{2.289873in}}%
\pgfpathcurveto{\pgfqpoint{1.533337in}{2.284049in}}{\pgfqpoint{1.530065in}{2.276149in}}{\pgfqpoint{1.530065in}{2.267913in}}%
\pgfpathcurveto{\pgfqpoint{1.530065in}{2.259676in}}{\pgfqpoint{1.533337in}{2.251776in}}{\pgfqpoint{1.539161in}{2.245952in}}%
\pgfpathcurveto{\pgfqpoint{1.544985in}{2.240129in}}{\pgfqpoint{1.552885in}{2.236856in}}{\pgfqpoint{1.561122in}{2.236856in}}%
\pgfpathclose%
\pgfusepath{stroke,fill}%
\end{pgfscope}%
\begin{pgfscope}%
\pgfpathrectangle{\pgfqpoint{0.100000in}{0.212622in}}{\pgfqpoint{3.696000in}{3.696000in}}%
\pgfusepath{clip}%
\pgfsetbuttcap%
\pgfsetroundjoin%
\definecolor{currentfill}{rgb}{0.121569,0.466667,0.705882}%
\pgfsetfillcolor{currentfill}%
\pgfsetfillopacity{0.336125}%
\pgfsetlinewidth{1.003750pt}%
\definecolor{currentstroke}{rgb}{0.121569,0.466667,0.705882}%
\pgfsetstrokecolor{currentstroke}%
\pgfsetstrokeopacity{0.336125}%
\pgfsetdash{}{0pt}%
\pgfpathmoveto{\pgfqpoint{1.554180in}{2.225985in}}%
\pgfpathcurveto{\pgfqpoint{1.562417in}{2.225985in}}{\pgfqpoint{1.570317in}{2.229257in}}{\pgfqpoint{1.576141in}{2.235081in}}%
\pgfpathcurveto{\pgfqpoint{1.581964in}{2.240905in}}{\pgfqpoint{1.585237in}{2.248805in}}{\pgfqpoint{1.585237in}{2.257041in}}%
\pgfpathcurveto{\pgfqpoint{1.585237in}{2.265278in}}{\pgfqpoint{1.581964in}{2.273178in}}{\pgfqpoint{1.576141in}{2.279002in}}%
\pgfpathcurveto{\pgfqpoint{1.570317in}{2.284826in}}{\pgfqpoint{1.562417in}{2.288098in}}{\pgfqpoint{1.554180in}{2.288098in}}%
\pgfpathcurveto{\pgfqpoint{1.545944in}{2.288098in}}{\pgfqpoint{1.538044in}{2.284826in}}{\pgfqpoint{1.532220in}{2.279002in}}%
\pgfpathcurveto{\pgfqpoint{1.526396in}{2.273178in}}{\pgfqpoint{1.523124in}{2.265278in}}{\pgfqpoint{1.523124in}{2.257041in}}%
\pgfpathcurveto{\pgfqpoint{1.523124in}{2.248805in}}{\pgfqpoint{1.526396in}{2.240905in}}{\pgfqpoint{1.532220in}{2.235081in}}%
\pgfpathcurveto{\pgfqpoint{1.538044in}{2.229257in}}{\pgfqpoint{1.545944in}{2.225985in}}{\pgfqpoint{1.554180in}{2.225985in}}%
\pgfpathclose%
\pgfusepath{stroke,fill}%
\end{pgfscope}%
\begin{pgfscope}%
\pgfpathrectangle{\pgfqpoint{0.100000in}{0.212622in}}{\pgfqpoint{3.696000in}{3.696000in}}%
\pgfusepath{clip}%
\pgfsetbuttcap%
\pgfsetroundjoin%
\definecolor{currentfill}{rgb}{0.121569,0.466667,0.705882}%
\pgfsetfillcolor{currentfill}%
\pgfsetfillopacity{0.338338}%
\pgfsetlinewidth{1.003750pt}%
\definecolor{currentstroke}{rgb}{0.121569,0.466667,0.705882}%
\pgfsetstrokecolor{currentstroke}%
\pgfsetstrokeopacity{0.338338}%
\pgfsetdash{}{0pt}%
\pgfpathmoveto{\pgfqpoint{1.905601in}{2.332366in}}%
\pgfpathcurveto{\pgfqpoint{1.913837in}{2.332366in}}{\pgfqpoint{1.921737in}{2.335638in}}{\pgfqpoint{1.927561in}{2.341462in}}%
\pgfpathcurveto{\pgfqpoint{1.933385in}{2.347286in}}{\pgfqpoint{1.936657in}{2.355186in}}{\pgfqpoint{1.936657in}{2.363422in}}%
\pgfpathcurveto{\pgfqpoint{1.936657in}{2.371659in}}{\pgfqpoint{1.933385in}{2.379559in}}{\pgfqpoint{1.927561in}{2.385383in}}%
\pgfpathcurveto{\pgfqpoint{1.921737in}{2.391206in}}{\pgfqpoint{1.913837in}{2.394479in}}{\pgfqpoint{1.905601in}{2.394479in}}%
\pgfpathcurveto{\pgfqpoint{1.897364in}{2.394479in}}{\pgfqpoint{1.889464in}{2.391206in}}{\pgfqpoint{1.883640in}{2.385383in}}%
\pgfpathcurveto{\pgfqpoint{1.877816in}{2.379559in}}{\pgfqpoint{1.874544in}{2.371659in}}{\pgfqpoint{1.874544in}{2.363422in}}%
\pgfpathcurveto{\pgfqpoint{1.874544in}{2.355186in}}{\pgfqpoint{1.877816in}{2.347286in}}{\pgfqpoint{1.883640in}{2.341462in}}%
\pgfpathcurveto{\pgfqpoint{1.889464in}{2.335638in}}{\pgfqpoint{1.897364in}{2.332366in}}{\pgfqpoint{1.905601in}{2.332366in}}%
\pgfpathclose%
\pgfusepath{stroke,fill}%
\end{pgfscope}%
\begin{pgfscope}%
\pgfpathrectangle{\pgfqpoint{0.100000in}{0.212622in}}{\pgfqpoint{3.696000in}{3.696000in}}%
\pgfusepath{clip}%
\pgfsetbuttcap%
\pgfsetroundjoin%
\definecolor{currentfill}{rgb}{0.121569,0.466667,0.705882}%
\pgfsetfillcolor{currentfill}%
\pgfsetfillopacity{0.341694}%
\pgfsetlinewidth{1.003750pt}%
\definecolor{currentstroke}{rgb}{0.121569,0.466667,0.705882}%
\pgfsetstrokecolor{currentstroke}%
\pgfsetstrokeopacity{0.341694}%
\pgfsetdash{}{0pt}%
\pgfpathmoveto{\pgfqpoint{1.540360in}{2.208355in}}%
\pgfpathcurveto{\pgfqpoint{1.548596in}{2.208355in}}{\pgfqpoint{1.556496in}{2.211628in}}{\pgfqpoint{1.562320in}{2.217452in}}%
\pgfpathcurveto{\pgfqpoint{1.568144in}{2.223275in}}{\pgfqpoint{1.571416in}{2.231175in}}{\pgfqpoint{1.571416in}{2.239412in}}%
\pgfpathcurveto{\pgfqpoint{1.571416in}{2.247648in}}{\pgfqpoint{1.568144in}{2.255548in}}{\pgfqpoint{1.562320in}{2.261372in}}%
\pgfpathcurveto{\pgfqpoint{1.556496in}{2.267196in}}{\pgfqpoint{1.548596in}{2.270468in}}{\pgfqpoint{1.540360in}{2.270468in}}%
\pgfpathcurveto{\pgfqpoint{1.532123in}{2.270468in}}{\pgfqpoint{1.524223in}{2.267196in}}{\pgfqpoint{1.518399in}{2.261372in}}%
\pgfpathcurveto{\pgfqpoint{1.512576in}{2.255548in}}{\pgfqpoint{1.509303in}{2.247648in}}{\pgfqpoint{1.509303in}{2.239412in}}%
\pgfpathcurveto{\pgfqpoint{1.509303in}{2.231175in}}{\pgfqpoint{1.512576in}{2.223275in}}{\pgfqpoint{1.518399in}{2.217452in}}%
\pgfpathcurveto{\pgfqpoint{1.524223in}{2.211628in}}{\pgfqpoint{1.532123in}{2.208355in}}{\pgfqpoint{1.540360in}{2.208355in}}%
\pgfpathclose%
\pgfusepath{stroke,fill}%
\end{pgfscope}%
\begin{pgfscope}%
\pgfpathrectangle{\pgfqpoint{0.100000in}{0.212622in}}{\pgfqpoint{3.696000in}{3.696000in}}%
\pgfusepath{clip}%
\pgfsetbuttcap%
\pgfsetroundjoin%
\definecolor{currentfill}{rgb}{0.121569,0.466667,0.705882}%
\pgfsetfillcolor{currentfill}%
\pgfsetfillopacity{0.346834}%
\pgfsetlinewidth{1.003750pt}%
\definecolor{currentstroke}{rgb}{0.121569,0.466667,0.705882}%
\pgfsetstrokecolor{currentstroke}%
\pgfsetstrokeopacity{0.346834}%
\pgfsetdash{}{0pt}%
\pgfpathmoveto{\pgfqpoint{1.945921in}{2.333634in}}%
\pgfpathcurveto{\pgfqpoint{1.954158in}{2.333634in}}{\pgfqpoint{1.962058in}{2.336906in}}{\pgfqpoint{1.967882in}{2.342730in}}%
\pgfpathcurveto{\pgfqpoint{1.973705in}{2.348554in}}{\pgfqpoint{1.976978in}{2.356454in}}{\pgfqpoint{1.976978in}{2.364690in}}%
\pgfpathcurveto{\pgfqpoint{1.976978in}{2.372926in}}{\pgfqpoint{1.973705in}{2.380826in}}{\pgfqpoint{1.967882in}{2.386650in}}%
\pgfpathcurveto{\pgfqpoint{1.962058in}{2.392474in}}{\pgfqpoint{1.954158in}{2.395747in}}{\pgfqpoint{1.945921in}{2.395747in}}%
\pgfpathcurveto{\pgfqpoint{1.937685in}{2.395747in}}{\pgfqpoint{1.929785in}{2.392474in}}{\pgfqpoint{1.923961in}{2.386650in}}%
\pgfpathcurveto{\pgfqpoint{1.918137in}{2.380826in}}{\pgfqpoint{1.914865in}{2.372926in}}{\pgfqpoint{1.914865in}{2.364690in}}%
\pgfpathcurveto{\pgfqpoint{1.914865in}{2.356454in}}{\pgfqpoint{1.918137in}{2.348554in}}{\pgfqpoint{1.923961in}{2.342730in}}%
\pgfpathcurveto{\pgfqpoint{1.929785in}{2.336906in}}{\pgfqpoint{1.937685in}{2.333634in}}{\pgfqpoint{1.945921in}{2.333634in}}%
\pgfpathclose%
\pgfusepath{stroke,fill}%
\end{pgfscope}%
\begin{pgfscope}%
\pgfpathrectangle{\pgfqpoint{0.100000in}{0.212622in}}{\pgfqpoint{3.696000in}{3.696000in}}%
\pgfusepath{clip}%
\pgfsetbuttcap%
\pgfsetroundjoin%
\definecolor{currentfill}{rgb}{0.121569,0.466667,0.705882}%
\pgfsetfillcolor{currentfill}%
\pgfsetfillopacity{0.350927}%
\pgfsetlinewidth{1.003750pt}%
\definecolor{currentstroke}{rgb}{0.121569,0.466667,0.705882}%
\pgfsetstrokecolor{currentstroke}%
\pgfsetstrokeopacity{0.350927}%
\pgfsetdash{}{0pt}%
\pgfpathmoveto{\pgfqpoint{1.968858in}{2.332258in}}%
\pgfpathcurveto{\pgfqpoint{1.977094in}{2.332258in}}{\pgfqpoint{1.984994in}{2.335530in}}{\pgfqpoint{1.990818in}{2.341354in}}%
\pgfpathcurveto{\pgfqpoint{1.996642in}{2.347178in}}{\pgfqpoint{1.999914in}{2.355078in}}{\pgfqpoint{1.999914in}{2.363314in}}%
\pgfpathcurveto{\pgfqpoint{1.999914in}{2.371550in}}{\pgfqpoint{1.996642in}{2.379450in}}{\pgfqpoint{1.990818in}{2.385274in}}%
\pgfpathcurveto{\pgfqpoint{1.984994in}{2.391098in}}{\pgfqpoint{1.977094in}{2.394371in}}{\pgfqpoint{1.968858in}{2.394371in}}%
\pgfpathcurveto{\pgfqpoint{1.960621in}{2.394371in}}{\pgfqpoint{1.952721in}{2.391098in}}{\pgfqpoint{1.946897in}{2.385274in}}%
\pgfpathcurveto{\pgfqpoint{1.941073in}{2.379450in}}{\pgfqpoint{1.937801in}{2.371550in}}{\pgfqpoint{1.937801in}{2.363314in}}%
\pgfpathcurveto{\pgfqpoint{1.937801in}{2.355078in}}{\pgfqpoint{1.941073in}{2.347178in}}{\pgfqpoint{1.946897in}{2.341354in}}%
\pgfpathcurveto{\pgfqpoint{1.952721in}{2.335530in}}{\pgfqpoint{1.960621in}{2.332258in}}{\pgfqpoint{1.968858in}{2.332258in}}%
\pgfpathclose%
\pgfusepath{stroke,fill}%
\end{pgfscope}%
\begin{pgfscope}%
\pgfpathrectangle{\pgfqpoint{0.100000in}{0.212622in}}{\pgfqpoint{3.696000in}{3.696000in}}%
\pgfusepath{clip}%
\pgfsetbuttcap%
\pgfsetroundjoin%
\definecolor{currentfill}{rgb}{0.121569,0.466667,0.705882}%
\pgfsetfillcolor{currentfill}%
\pgfsetfillopacity{0.351180}%
\pgfsetlinewidth{1.003750pt}%
\definecolor{currentstroke}{rgb}{0.121569,0.466667,0.705882}%
\pgfsetstrokecolor{currentstroke}%
\pgfsetstrokeopacity{0.351180}%
\pgfsetdash{}{0pt}%
\pgfpathmoveto{\pgfqpoint{1.517958in}{2.169123in}}%
\pgfpathcurveto{\pgfqpoint{1.526194in}{2.169123in}}{\pgfqpoint{1.534094in}{2.172396in}}{\pgfqpoint{1.539918in}{2.178220in}}%
\pgfpathcurveto{\pgfqpoint{1.545742in}{2.184044in}}{\pgfqpoint{1.549014in}{2.191944in}}{\pgfqpoint{1.549014in}{2.200180in}}%
\pgfpathcurveto{\pgfqpoint{1.549014in}{2.208416in}}{\pgfqpoint{1.545742in}{2.216316in}}{\pgfqpoint{1.539918in}{2.222140in}}%
\pgfpathcurveto{\pgfqpoint{1.534094in}{2.227964in}}{\pgfqpoint{1.526194in}{2.231236in}}{\pgfqpoint{1.517958in}{2.231236in}}%
\pgfpathcurveto{\pgfqpoint{1.509721in}{2.231236in}}{\pgfqpoint{1.501821in}{2.227964in}}{\pgfqpoint{1.495997in}{2.222140in}}%
\pgfpathcurveto{\pgfqpoint{1.490173in}{2.216316in}}{\pgfqpoint{1.486901in}{2.208416in}}{\pgfqpoint{1.486901in}{2.200180in}}%
\pgfpathcurveto{\pgfqpoint{1.486901in}{2.191944in}}{\pgfqpoint{1.490173in}{2.184044in}}{\pgfqpoint{1.495997in}{2.178220in}}%
\pgfpathcurveto{\pgfqpoint{1.501821in}{2.172396in}}{\pgfqpoint{1.509721in}{2.169123in}}{\pgfqpoint{1.517958in}{2.169123in}}%
\pgfpathclose%
\pgfusepath{stroke,fill}%
\end{pgfscope}%
\begin{pgfscope}%
\pgfpathrectangle{\pgfqpoint{0.100000in}{0.212622in}}{\pgfqpoint{3.696000in}{3.696000in}}%
\pgfusepath{clip}%
\pgfsetbuttcap%
\pgfsetroundjoin%
\definecolor{currentfill}{rgb}{0.121569,0.466667,0.705882}%
\pgfsetfillcolor{currentfill}%
\pgfsetfillopacity{0.356416}%
\pgfsetlinewidth{1.003750pt}%
\definecolor{currentstroke}{rgb}{0.121569,0.466667,0.705882}%
\pgfsetstrokecolor{currentstroke}%
\pgfsetstrokeopacity{0.356416}%
\pgfsetdash{}{0pt}%
\pgfpathmoveto{\pgfqpoint{1.999923in}{2.330517in}}%
\pgfpathcurveto{\pgfqpoint{2.008160in}{2.330517in}}{\pgfqpoint{2.016060in}{2.333790in}}{\pgfqpoint{2.021884in}{2.339614in}}%
\pgfpathcurveto{\pgfqpoint{2.027708in}{2.345438in}}{\pgfqpoint{2.030980in}{2.353338in}}{\pgfqpoint{2.030980in}{2.361574in}}%
\pgfpathcurveto{\pgfqpoint{2.030980in}{2.369810in}}{\pgfqpoint{2.027708in}{2.377710in}}{\pgfqpoint{2.021884in}{2.383534in}}%
\pgfpathcurveto{\pgfqpoint{2.016060in}{2.389358in}}{\pgfqpoint{2.008160in}{2.392630in}}{\pgfqpoint{1.999923in}{2.392630in}}%
\pgfpathcurveto{\pgfqpoint{1.991687in}{2.392630in}}{\pgfqpoint{1.983787in}{2.389358in}}{\pgfqpoint{1.977963in}{2.383534in}}%
\pgfpathcurveto{\pgfqpoint{1.972139in}{2.377710in}}{\pgfqpoint{1.968867in}{2.369810in}}{\pgfqpoint{1.968867in}{2.361574in}}%
\pgfpathcurveto{\pgfqpoint{1.968867in}{2.353338in}}{\pgfqpoint{1.972139in}{2.345438in}}{\pgfqpoint{1.977963in}{2.339614in}}%
\pgfpathcurveto{\pgfqpoint{1.983787in}{2.333790in}}{\pgfqpoint{1.991687in}{2.330517in}}{\pgfqpoint{1.999923in}{2.330517in}}%
\pgfpathclose%
\pgfusepath{stroke,fill}%
\end{pgfscope}%
\begin{pgfscope}%
\pgfpathrectangle{\pgfqpoint{0.100000in}{0.212622in}}{\pgfqpoint{3.696000in}{3.696000in}}%
\pgfusepath{clip}%
\pgfsetbuttcap%
\pgfsetroundjoin%
\definecolor{currentfill}{rgb}{0.121569,0.466667,0.705882}%
\pgfsetfillcolor{currentfill}%
\pgfsetfillopacity{0.359819}%
\pgfsetlinewidth{1.003750pt}%
\definecolor{currentstroke}{rgb}{0.121569,0.466667,0.705882}%
\pgfsetstrokecolor{currentstroke}%
\pgfsetstrokeopacity{0.359819}%
\pgfsetdash{}{0pt}%
\pgfpathmoveto{\pgfqpoint{2.015633in}{2.328258in}}%
\pgfpathcurveto{\pgfqpoint{2.023869in}{2.328258in}}{\pgfqpoint{2.031769in}{2.331530in}}{\pgfqpoint{2.037593in}{2.337354in}}%
\pgfpathcurveto{\pgfqpoint{2.043417in}{2.343178in}}{\pgfqpoint{2.046689in}{2.351078in}}{\pgfqpoint{2.046689in}{2.359314in}}%
\pgfpathcurveto{\pgfqpoint{2.046689in}{2.367551in}}{\pgfqpoint{2.043417in}{2.375451in}}{\pgfqpoint{2.037593in}{2.381275in}}%
\pgfpathcurveto{\pgfqpoint{2.031769in}{2.387099in}}{\pgfqpoint{2.023869in}{2.390371in}}{\pgfqpoint{2.015633in}{2.390371in}}%
\pgfpathcurveto{\pgfqpoint{2.007396in}{2.390371in}}{\pgfqpoint{1.999496in}{2.387099in}}{\pgfqpoint{1.993672in}{2.381275in}}%
\pgfpathcurveto{\pgfqpoint{1.987848in}{2.375451in}}{\pgfqpoint{1.984576in}{2.367551in}}{\pgfqpoint{1.984576in}{2.359314in}}%
\pgfpathcurveto{\pgfqpoint{1.984576in}{2.351078in}}{\pgfqpoint{1.987848in}{2.343178in}}{\pgfqpoint{1.993672in}{2.337354in}}%
\pgfpathcurveto{\pgfqpoint{1.999496in}{2.331530in}}{\pgfqpoint{2.007396in}{2.328258in}}{\pgfqpoint{2.015633in}{2.328258in}}%
\pgfpathclose%
\pgfusepath{stroke,fill}%
\end{pgfscope}%
\begin{pgfscope}%
\pgfpathrectangle{\pgfqpoint{0.100000in}{0.212622in}}{\pgfqpoint{3.696000in}{3.696000in}}%
\pgfusepath{clip}%
\pgfsetbuttcap%
\pgfsetroundjoin%
\definecolor{currentfill}{rgb}{0.121569,0.466667,0.705882}%
\pgfsetfillcolor{currentfill}%
\pgfsetfillopacity{0.360707}%
\pgfsetlinewidth{1.003750pt}%
\definecolor{currentstroke}{rgb}{0.121569,0.466667,0.705882}%
\pgfsetstrokecolor{currentstroke}%
\pgfsetstrokeopacity{0.360707}%
\pgfsetdash{}{0pt}%
\pgfpathmoveto{\pgfqpoint{1.497048in}{2.135616in}}%
\pgfpathcurveto{\pgfqpoint{1.505284in}{2.135616in}}{\pgfqpoint{1.513184in}{2.138888in}}{\pgfqpoint{1.519008in}{2.144712in}}%
\pgfpathcurveto{\pgfqpoint{1.524832in}{2.150536in}}{\pgfqpoint{1.528104in}{2.158436in}}{\pgfqpoint{1.528104in}{2.166672in}}%
\pgfpathcurveto{\pgfqpoint{1.528104in}{2.174909in}}{\pgfqpoint{1.524832in}{2.182809in}}{\pgfqpoint{1.519008in}{2.188633in}}%
\pgfpathcurveto{\pgfqpoint{1.513184in}{2.194457in}}{\pgfqpoint{1.505284in}{2.197729in}}{\pgfqpoint{1.497048in}{2.197729in}}%
\pgfpathcurveto{\pgfqpoint{1.488811in}{2.197729in}}{\pgfqpoint{1.480911in}{2.194457in}}{\pgfqpoint{1.475087in}{2.188633in}}%
\pgfpathcurveto{\pgfqpoint{1.469263in}{2.182809in}}{\pgfqpoint{1.465991in}{2.174909in}}{\pgfqpoint{1.465991in}{2.166672in}}%
\pgfpathcurveto{\pgfqpoint{1.465991in}{2.158436in}}{\pgfqpoint{1.469263in}{2.150536in}}{\pgfqpoint{1.475087in}{2.144712in}}%
\pgfpathcurveto{\pgfqpoint{1.480911in}{2.138888in}}{\pgfqpoint{1.488811in}{2.135616in}}{\pgfqpoint{1.497048in}{2.135616in}}%
\pgfpathclose%
\pgfusepath{stroke,fill}%
\end{pgfscope}%
\begin{pgfscope}%
\pgfpathrectangle{\pgfqpoint{0.100000in}{0.212622in}}{\pgfqpoint{3.696000in}{3.696000in}}%
\pgfusepath{clip}%
\pgfsetbuttcap%
\pgfsetroundjoin%
\definecolor{currentfill}{rgb}{0.121569,0.466667,0.705882}%
\pgfsetfillcolor{currentfill}%
\pgfsetfillopacity{0.363944}%
\pgfsetlinewidth{1.003750pt}%
\definecolor{currentstroke}{rgb}{0.121569,0.466667,0.705882}%
\pgfsetstrokecolor{currentstroke}%
\pgfsetstrokeopacity{0.363944}%
\pgfsetdash{}{0pt}%
\pgfpathmoveto{\pgfqpoint{2.039161in}{2.323576in}}%
\pgfpathcurveto{\pgfqpoint{2.047398in}{2.323576in}}{\pgfqpoint{2.055298in}{2.326848in}}{\pgfqpoint{2.061122in}{2.332672in}}%
\pgfpathcurveto{\pgfqpoint{2.066945in}{2.338496in}}{\pgfqpoint{2.070218in}{2.346396in}}{\pgfqpoint{2.070218in}{2.354632in}}%
\pgfpathcurveto{\pgfqpoint{2.070218in}{2.362868in}}{\pgfqpoint{2.066945in}{2.370768in}}{\pgfqpoint{2.061122in}{2.376592in}}%
\pgfpathcurveto{\pgfqpoint{2.055298in}{2.382416in}}{\pgfqpoint{2.047398in}{2.385689in}}{\pgfqpoint{2.039161in}{2.385689in}}%
\pgfpathcurveto{\pgfqpoint{2.030925in}{2.385689in}}{\pgfqpoint{2.023025in}{2.382416in}}{\pgfqpoint{2.017201in}{2.376592in}}%
\pgfpathcurveto{\pgfqpoint{2.011377in}{2.370768in}}{\pgfqpoint{2.008105in}{2.362868in}}{\pgfqpoint{2.008105in}{2.354632in}}%
\pgfpathcurveto{\pgfqpoint{2.008105in}{2.346396in}}{\pgfqpoint{2.011377in}{2.338496in}}{\pgfqpoint{2.017201in}{2.332672in}}%
\pgfpathcurveto{\pgfqpoint{2.023025in}{2.326848in}}{\pgfqpoint{2.030925in}{2.323576in}}{\pgfqpoint{2.039161in}{2.323576in}}%
\pgfpathclose%
\pgfusepath{stroke,fill}%
\end{pgfscope}%
\begin{pgfscope}%
\pgfpathrectangle{\pgfqpoint{0.100000in}{0.212622in}}{\pgfqpoint{3.696000in}{3.696000in}}%
\pgfusepath{clip}%
\pgfsetbuttcap%
\pgfsetroundjoin%
\definecolor{currentfill}{rgb}{0.121569,0.466667,0.705882}%
\pgfsetfillcolor{currentfill}%
\pgfsetfillopacity{0.366360}%
\pgfsetlinewidth{1.003750pt}%
\definecolor{currentstroke}{rgb}{0.121569,0.466667,0.705882}%
\pgfsetstrokecolor{currentstroke}%
\pgfsetstrokeopacity{0.366360}%
\pgfsetdash{}{0pt}%
\pgfpathmoveto{\pgfqpoint{2.052721in}{2.323846in}}%
\pgfpathcurveto{\pgfqpoint{2.060957in}{2.323846in}}{\pgfqpoint{2.068858in}{2.327119in}}{\pgfqpoint{2.074681in}{2.332943in}}%
\pgfpathcurveto{\pgfqpoint{2.080505in}{2.338767in}}{\pgfqpoint{2.083778in}{2.346667in}}{\pgfqpoint{2.083778in}{2.354903in}}%
\pgfpathcurveto{\pgfqpoint{2.083778in}{2.363139in}}{\pgfqpoint{2.080505in}{2.371039in}}{\pgfqpoint{2.074681in}{2.376863in}}%
\pgfpathcurveto{\pgfqpoint{2.068858in}{2.382687in}}{\pgfqpoint{2.060957in}{2.385959in}}{\pgfqpoint{2.052721in}{2.385959in}}%
\pgfpathcurveto{\pgfqpoint{2.044485in}{2.385959in}}{\pgfqpoint{2.036585in}{2.382687in}}{\pgfqpoint{2.030761in}{2.376863in}}%
\pgfpathcurveto{\pgfqpoint{2.024937in}{2.371039in}}{\pgfqpoint{2.021665in}{2.363139in}}{\pgfqpoint{2.021665in}{2.354903in}}%
\pgfpathcurveto{\pgfqpoint{2.021665in}{2.346667in}}{\pgfqpoint{2.024937in}{2.338767in}}{\pgfqpoint{2.030761in}{2.332943in}}%
\pgfpathcurveto{\pgfqpoint{2.036585in}{2.327119in}}{\pgfqpoint{2.044485in}{2.323846in}}{\pgfqpoint{2.052721in}{2.323846in}}%
\pgfpathclose%
\pgfusepath{stroke,fill}%
\end{pgfscope}%
\begin{pgfscope}%
\pgfpathrectangle{\pgfqpoint{0.100000in}{0.212622in}}{\pgfqpoint{3.696000in}{3.696000in}}%
\pgfusepath{clip}%
\pgfsetbuttcap%
\pgfsetroundjoin%
\definecolor{currentfill}{rgb}{0.121569,0.466667,0.705882}%
\pgfsetfillcolor{currentfill}%
\pgfsetfillopacity{0.369106}%
\pgfsetlinewidth{1.003750pt}%
\definecolor{currentstroke}{rgb}{0.121569,0.466667,0.705882}%
\pgfsetstrokecolor{currentstroke}%
\pgfsetstrokeopacity{0.369106}%
\pgfsetdash{}{0pt}%
\pgfpathmoveto{\pgfqpoint{2.069877in}{2.320210in}}%
\pgfpathcurveto{\pgfqpoint{2.078114in}{2.320210in}}{\pgfqpoint{2.086014in}{2.323483in}}{\pgfqpoint{2.091838in}{2.329307in}}%
\pgfpathcurveto{\pgfqpoint{2.097662in}{2.335130in}}{\pgfqpoint{2.100934in}{2.343031in}}{\pgfqpoint{2.100934in}{2.351267in}}%
\pgfpathcurveto{\pgfqpoint{2.100934in}{2.359503in}}{\pgfqpoint{2.097662in}{2.367403in}}{\pgfqpoint{2.091838in}{2.373227in}}%
\pgfpathcurveto{\pgfqpoint{2.086014in}{2.379051in}}{\pgfqpoint{2.078114in}{2.382323in}}{\pgfqpoint{2.069877in}{2.382323in}}%
\pgfpathcurveto{\pgfqpoint{2.061641in}{2.382323in}}{\pgfqpoint{2.053741in}{2.379051in}}{\pgfqpoint{2.047917in}{2.373227in}}%
\pgfpathcurveto{\pgfqpoint{2.042093in}{2.367403in}}{\pgfqpoint{2.038821in}{2.359503in}}{\pgfqpoint{2.038821in}{2.351267in}}%
\pgfpathcurveto{\pgfqpoint{2.038821in}{2.343031in}}{\pgfqpoint{2.042093in}{2.335130in}}{\pgfqpoint{2.047917in}{2.329307in}}%
\pgfpathcurveto{\pgfqpoint{2.053741in}{2.323483in}}{\pgfqpoint{2.061641in}{2.320210in}}{\pgfqpoint{2.069877in}{2.320210in}}%
\pgfpathclose%
\pgfusepath{stroke,fill}%
\end{pgfscope}%
\begin{pgfscope}%
\pgfpathrectangle{\pgfqpoint{0.100000in}{0.212622in}}{\pgfqpoint{3.696000in}{3.696000in}}%
\pgfusepath{clip}%
\pgfsetbuttcap%
\pgfsetroundjoin%
\definecolor{currentfill}{rgb}{0.121569,0.466667,0.705882}%
\pgfsetfillcolor{currentfill}%
\pgfsetfillopacity{0.369323}%
\pgfsetlinewidth{1.003750pt}%
\definecolor{currentstroke}{rgb}{0.121569,0.466667,0.705882}%
\pgfsetstrokecolor{currentstroke}%
\pgfsetstrokeopacity{0.369323}%
\pgfsetdash{}{0pt}%
\pgfpathmoveto{\pgfqpoint{1.478360in}{2.106053in}}%
\pgfpathcurveto{\pgfqpoint{1.486596in}{2.106053in}}{\pgfqpoint{1.494496in}{2.109325in}}{\pgfqpoint{1.500320in}{2.115149in}}%
\pgfpathcurveto{\pgfqpoint{1.506144in}{2.120973in}}{\pgfqpoint{1.509417in}{2.128873in}}{\pgfqpoint{1.509417in}{2.137110in}}%
\pgfpathcurveto{\pgfqpoint{1.509417in}{2.145346in}}{\pgfqpoint{1.506144in}{2.153246in}}{\pgfqpoint{1.500320in}{2.159070in}}%
\pgfpathcurveto{\pgfqpoint{1.494496in}{2.164894in}}{\pgfqpoint{1.486596in}{2.168166in}}{\pgfqpoint{1.478360in}{2.168166in}}%
\pgfpathcurveto{\pgfqpoint{1.470124in}{2.168166in}}{\pgfqpoint{1.462224in}{2.164894in}}{\pgfqpoint{1.456400in}{2.159070in}}%
\pgfpathcurveto{\pgfqpoint{1.450576in}{2.153246in}}{\pgfqpoint{1.447304in}{2.145346in}}{\pgfqpoint{1.447304in}{2.137110in}}%
\pgfpathcurveto{\pgfqpoint{1.447304in}{2.128873in}}{\pgfqpoint{1.450576in}{2.120973in}}{\pgfqpoint{1.456400in}{2.115149in}}%
\pgfpathcurveto{\pgfqpoint{1.462224in}{2.109325in}}{\pgfqpoint{1.470124in}{2.106053in}}{\pgfqpoint{1.478360in}{2.106053in}}%
\pgfpathclose%
\pgfusepath{stroke,fill}%
\end{pgfscope}%
\begin{pgfscope}%
\pgfpathrectangle{\pgfqpoint{0.100000in}{0.212622in}}{\pgfqpoint{3.696000in}{3.696000in}}%
\pgfusepath{clip}%
\pgfsetbuttcap%
\pgfsetroundjoin%
\definecolor{currentfill}{rgb}{0.121569,0.466667,0.705882}%
\pgfsetfillcolor{currentfill}%
\pgfsetfillopacity{0.373110}%
\pgfsetlinewidth{1.003750pt}%
\definecolor{currentstroke}{rgb}{0.121569,0.466667,0.705882}%
\pgfsetstrokecolor{currentstroke}%
\pgfsetstrokeopacity{0.373110}%
\pgfsetdash{}{0pt}%
\pgfpathmoveto{\pgfqpoint{2.092679in}{2.319467in}}%
\pgfpathcurveto{\pgfqpoint{2.100915in}{2.319467in}}{\pgfqpoint{2.108815in}{2.322739in}}{\pgfqpoint{2.114639in}{2.328563in}}%
\pgfpathcurveto{\pgfqpoint{2.120463in}{2.334387in}}{\pgfqpoint{2.123735in}{2.342287in}}{\pgfqpoint{2.123735in}{2.350523in}}%
\pgfpathcurveto{\pgfqpoint{2.123735in}{2.358760in}}{\pgfqpoint{2.120463in}{2.366660in}}{\pgfqpoint{2.114639in}{2.372484in}}%
\pgfpathcurveto{\pgfqpoint{2.108815in}{2.378308in}}{\pgfqpoint{2.100915in}{2.381580in}}{\pgfqpoint{2.092679in}{2.381580in}}%
\pgfpathcurveto{\pgfqpoint{2.084442in}{2.381580in}}{\pgfqpoint{2.076542in}{2.378308in}}{\pgfqpoint{2.070719in}{2.372484in}}%
\pgfpathcurveto{\pgfqpoint{2.064895in}{2.366660in}}{\pgfqpoint{2.061622in}{2.358760in}}{\pgfqpoint{2.061622in}{2.350523in}}%
\pgfpathcurveto{\pgfqpoint{2.061622in}{2.342287in}}{\pgfqpoint{2.064895in}{2.334387in}}{\pgfqpoint{2.070719in}{2.328563in}}%
\pgfpathcurveto{\pgfqpoint{2.076542in}{2.322739in}}{\pgfqpoint{2.084442in}{2.319467in}}{\pgfqpoint{2.092679in}{2.319467in}}%
\pgfpathclose%
\pgfusepath{stroke,fill}%
\end{pgfscope}%
\begin{pgfscope}%
\pgfpathrectangle{\pgfqpoint{0.100000in}{0.212622in}}{\pgfqpoint{3.696000in}{3.696000in}}%
\pgfusepath{clip}%
\pgfsetbuttcap%
\pgfsetroundjoin%
\definecolor{currentfill}{rgb}{0.121569,0.466667,0.705882}%
\pgfsetfillcolor{currentfill}%
\pgfsetfillopacity{0.376741}%
\pgfsetlinewidth{1.003750pt}%
\definecolor{currentstroke}{rgb}{0.121569,0.466667,0.705882}%
\pgfsetstrokecolor{currentstroke}%
\pgfsetstrokeopacity{0.376741}%
\pgfsetdash{}{0pt}%
\pgfpathmoveto{\pgfqpoint{1.456090in}{2.085098in}}%
\pgfpathcurveto{\pgfqpoint{1.464327in}{2.085098in}}{\pgfqpoint{1.472227in}{2.088370in}}{\pgfqpoint{1.478051in}{2.094194in}}%
\pgfpathcurveto{\pgfqpoint{1.483875in}{2.100018in}}{\pgfqpoint{1.487147in}{2.107918in}}{\pgfqpoint{1.487147in}{2.116154in}}%
\pgfpathcurveto{\pgfqpoint{1.487147in}{2.124391in}}{\pgfqpoint{1.483875in}{2.132291in}}{\pgfqpoint{1.478051in}{2.138114in}}%
\pgfpathcurveto{\pgfqpoint{1.472227in}{2.143938in}}{\pgfqpoint{1.464327in}{2.147211in}}{\pgfqpoint{1.456090in}{2.147211in}}%
\pgfpathcurveto{\pgfqpoint{1.447854in}{2.147211in}}{\pgfqpoint{1.439954in}{2.143938in}}{\pgfqpoint{1.434130in}{2.138114in}}%
\pgfpathcurveto{\pgfqpoint{1.428306in}{2.132291in}}{\pgfqpoint{1.425034in}{2.124391in}}{\pgfqpoint{1.425034in}{2.116154in}}%
\pgfpathcurveto{\pgfqpoint{1.425034in}{2.107918in}}{\pgfqpoint{1.428306in}{2.100018in}}{\pgfqpoint{1.434130in}{2.094194in}}%
\pgfpathcurveto{\pgfqpoint{1.439954in}{2.088370in}}{\pgfqpoint{1.447854in}{2.085098in}}{\pgfqpoint{1.456090in}{2.085098in}}%
\pgfpathclose%
\pgfusepath{stroke,fill}%
\end{pgfscope}%
\begin{pgfscope}%
\pgfpathrectangle{\pgfqpoint{0.100000in}{0.212622in}}{\pgfqpoint{3.696000in}{3.696000in}}%
\pgfusepath{clip}%
\pgfsetbuttcap%
\pgfsetroundjoin%
\definecolor{currentfill}{rgb}{0.121569,0.466667,0.705882}%
\pgfsetfillcolor{currentfill}%
\pgfsetfillopacity{0.379446}%
\pgfsetlinewidth{1.003750pt}%
\definecolor{currentstroke}{rgb}{0.121569,0.466667,0.705882}%
\pgfsetstrokecolor{currentstroke}%
\pgfsetstrokeopacity{0.379446}%
\pgfsetdash{}{0pt}%
\pgfpathmoveto{\pgfqpoint{2.121682in}{2.325991in}}%
\pgfpathcurveto{\pgfqpoint{2.129918in}{2.325991in}}{\pgfqpoint{2.137818in}{2.329263in}}{\pgfqpoint{2.143642in}{2.335087in}}%
\pgfpathcurveto{\pgfqpoint{2.149466in}{2.340911in}}{\pgfqpoint{2.152738in}{2.348811in}}{\pgfqpoint{2.152738in}{2.357047in}}%
\pgfpathcurveto{\pgfqpoint{2.152738in}{2.365284in}}{\pgfqpoint{2.149466in}{2.373184in}}{\pgfqpoint{2.143642in}{2.379008in}}%
\pgfpathcurveto{\pgfqpoint{2.137818in}{2.384832in}}{\pgfqpoint{2.129918in}{2.388104in}}{\pgfqpoint{2.121682in}{2.388104in}}%
\pgfpathcurveto{\pgfqpoint{2.113445in}{2.388104in}}{\pgfqpoint{2.105545in}{2.384832in}}{\pgfqpoint{2.099721in}{2.379008in}}%
\pgfpathcurveto{\pgfqpoint{2.093897in}{2.373184in}}{\pgfqpoint{2.090625in}{2.365284in}}{\pgfqpoint{2.090625in}{2.357047in}}%
\pgfpathcurveto{\pgfqpoint{2.090625in}{2.348811in}}{\pgfqpoint{2.093897in}{2.340911in}}{\pgfqpoint{2.099721in}{2.335087in}}%
\pgfpathcurveto{\pgfqpoint{2.105545in}{2.329263in}}{\pgfqpoint{2.113445in}{2.325991in}}{\pgfqpoint{2.121682in}{2.325991in}}%
\pgfpathclose%
\pgfusepath{stroke,fill}%
\end{pgfscope}%
\begin{pgfscope}%
\pgfpathrectangle{\pgfqpoint{0.100000in}{0.212622in}}{\pgfqpoint{3.696000in}{3.696000in}}%
\pgfusepath{clip}%
\pgfsetbuttcap%
\pgfsetroundjoin%
\definecolor{currentfill}{rgb}{0.121569,0.466667,0.705882}%
\pgfsetfillcolor{currentfill}%
\pgfsetfillopacity{0.383618}%
\pgfsetlinewidth{1.003750pt}%
\definecolor{currentstroke}{rgb}{0.121569,0.466667,0.705882}%
\pgfsetstrokecolor{currentstroke}%
\pgfsetstrokeopacity{0.383618}%
\pgfsetdash{}{0pt}%
\pgfpathmoveto{\pgfqpoint{1.443062in}{2.064389in}}%
\pgfpathcurveto{\pgfqpoint{1.451298in}{2.064389in}}{\pgfqpoint{1.459198in}{2.067661in}}{\pgfqpoint{1.465022in}{2.073485in}}%
\pgfpathcurveto{\pgfqpoint{1.470846in}{2.079309in}}{\pgfqpoint{1.474118in}{2.087209in}}{\pgfqpoint{1.474118in}{2.095446in}}%
\pgfpathcurveto{\pgfqpoint{1.474118in}{2.103682in}}{\pgfqpoint{1.470846in}{2.111582in}}{\pgfqpoint{1.465022in}{2.117406in}}%
\pgfpathcurveto{\pgfqpoint{1.459198in}{2.123230in}}{\pgfqpoint{1.451298in}{2.126502in}}{\pgfqpoint{1.443062in}{2.126502in}}%
\pgfpathcurveto{\pgfqpoint{1.434825in}{2.126502in}}{\pgfqpoint{1.426925in}{2.123230in}}{\pgfqpoint{1.421101in}{2.117406in}}%
\pgfpathcurveto{\pgfqpoint{1.415278in}{2.111582in}}{\pgfqpoint{1.412005in}{2.103682in}}{\pgfqpoint{1.412005in}{2.095446in}}%
\pgfpathcurveto{\pgfqpoint{1.412005in}{2.087209in}}{\pgfqpoint{1.415278in}{2.079309in}}{\pgfqpoint{1.421101in}{2.073485in}}%
\pgfpathcurveto{\pgfqpoint{1.426925in}{2.067661in}}{\pgfqpoint{1.434825in}{2.064389in}}{\pgfqpoint{1.443062in}{2.064389in}}%
\pgfpathclose%
\pgfusepath{stroke,fill}%
\end{pgfscope}%
\begin{pgfscope}%
\pgfpathrectangle{\pgfqpoint{0.100000in}{0.212622in}}{\pgfqpoint{3.696000in}{3.696000in}}%
\pgfusepath{clip}%
\pgfsetbuttcap%
\pgfsetroundjoin%
\definecolor{currentfill}{rgb}{0.121569,0.466667,0.705882}%
\pgfsetfillcolor{currentfill}%
\pgfsetfillopacity{0.386045}%
\pgfsetlinewidth{1.003750pt}%
\definecolor{currentstroke}{rgb}{0.121569,0.466667,0.705882}%
\pgfsetstrokecolor{currentstroke}%
\pgfsetstrokeopacity{0.386045}%
\pgfsetdash{}{0pt}%
\pgfpathmoveto{\pgfqpoint{2.150230in}{2.317565in}}%
\pgfpathcurveto{\pgfqpoint{2.158466in}{2.317565in}}{\pgfqpoint{2.166366in}{2.320838in}}{\pgfqpoint{2.172190in}{2.326662in}}%
\pgfpathcurveto{\pgfqpoint{2.178014in}{2.332486in}}{\pgfqpoint{2.181286in}{2.340386in}}{\pgfqpoint{2.181286in}{2.348622in}}%
\pgfpathcurveto{\pgfqpoint{2.181286in}{2.356858in}}{\pgfqpoint{2.178014in}{2.364758in}}{\pgfqpoint{2.172190in}{2.370582in}}%
\pgfpathcurveto{\pgfqpoint{2.166366in}{2.376406in}}{\pgfqpoint{2.158466in}{2.379678in}}{\pgfqpoint{2.150230in}{2.379678in}}%
\pgfpathcurveto{\pgfqpoint{2.141994in}{2.379678in}}{\pgfqpoint{2.134094in}{2.376406in}}{\pgfqpoint{2.128270in}{2.370582in}}%
\pgfpathcurveto{\pgfqpoint{2.122446in}{2.364758in}}{\pgfqpoint{2.119173in}{2.356858in}}{\pgfqpoint{2.119173in}{2.348622in}}%
\pgfpathcurveto{\pgfqpoint{2.119173in}{2.340386in}}{\pgfqpoint{2.122446in}{2.332486in}}{\pgfqpoint{2.128270in}{2.326662in}}%
\pgfpathcurveto{\pgfqpoint{2.134094in}{2.320838in}}{\pgfqpoint{2.141994in}{2.317565in}}{\pgfqpoint{2.150230in}{2.317565in}}%
\pgfpathclose%
\pgfusepath{stroke,fill}%
\end{pgfscope}%
\begin{pgfscope}%
\pgfpathrectangle{\pgfqpoint{0.100000in}{0.212622in}}{\pgfqpoint{3.696000in}{3.696000in}}%
\pgfusepath{clip}%
\pgfsetbuttcap%
\pgfsetroundjoin%
\definecolor{currentfill}{rgb}{0.121569,0.466667,0.705882}%
\pgfsetfillcolor{currentfill}%
\pgfsetfillopacity{0.388249}%
\pgfsetlinewidth{1.003750pt}%
\definecolor{currentstroke}{rgb}{0.121569,0.466667,0.705882}%
\pgfsetstrokecolor{currentstroke}%
\pgfsetstrokeopacity{0.388249}%
\pgfsetdash{}{0pt}%
\pgfpathmoveto{\pgfqpoint{1.428736in}{2.052956in}}%
\pgfpathcurveto{\pgfqpoint{1.436972in}{2.052956in}}{\pgfqpoint{1.444872in}{2.056228in}}{\pgfqpoint{1.450696in}{2.062052in}}%
\pgfpathcurveto{\pgfqpoint{1.456520in}{2.067876in}}{\pgfqpoint{1.459792in}{2.075776in}}{\pgfqpoint{1.459792in}{2.084012in}}%
\pgfpathcurveto{\pgfqpoint{1.459792in}{2.092248in}}{\pgfqpoint{1.456520in}{2.100148in}}{\pgfqpoint{1.450696in}{2.105972in}}%
\pgfpathcurveto{\pgfqpoint{1.444872in}{2.111796in}}{\pgfqpoint{1.436972in}{2.115069in}}{\pgfqpoint{1.428736in}{2.115069in}}%
\pgfpathcurveto{\pgfqpoint{1.420499in}{2.115069in}}{\pgfqpoint{1.412599in}{2.111796in}}{\pgfqpoint{1.406775in}{2.105972in}}%
\pgfpathcurveto{\pgfqpoint{1.400951in}{2.100148in}}{\pgfqpoint{1.397679in}{2.092248in}}{\pgfqpoint{1.397679in}{2.084012in}}%
\pgfpathcurveto{\pgfqpoint{1.397679in}{2.075776in}}{\pgfqpoint{1.400951in}{2.067876in}}{\pgfqpoint{1.406775in}{2.062052in}}%
\pgfpathcurveto{\pgfqpoint{1.412599in}{2.056228in}}{\pgfqpoint{1.420499in}{2.052956in}}{\pgfqpoint{1.428736in}{2.052956in}}%
\pgfpathclose%
\pgfusepath{stroke,fill}%
\end{pgfscope}%
\begin{pgfscope}%
\pgfpathrectangle{\pgfqpoint{0.100000in}{0.212622in}}{\pgfqpoint{3.696000in}{3.696000in}}%
\pgfusepath{clip}%
\pgfsetbuttcap%
\pgfsetroundjoin%
\definecolor{currentfill}{rgb}{0.121569,0.466667,0.705882}%
\pgfsetfillcolor{currentfill}%
\pgfsetfillopacity{0.391743}%
\pgfsetlinewidth{1.003750pt}%
\definecolor{currentstroke}{rgb}{0.121569,0.466667,0.705882}%
\pgfsetstrokecolor{currentstroke}%
\pgfsetstrokeopacity{0.391743}%
\pgfsetdash{}{0pt}%
\pgfpathmoveto{\pgfqpoint{1.422380in}{2.040141in}}%
\pgfpathcurveto{\pgfqpoint{1.430616in}{2.040141in}}{\pgfqpoint{1.438516in}{2.043413in}}{\pgfqpoint{1.444340in}{2.049237in}}%
\pgfpathcurveto{\pgfqpoint{1.450164in}{2.055061in}}{\pgfqpoint{1.453437in}{2.062961in}}{\pgfqpoint{1.453437in}{2.071197in}}%
\pgfpathcurveto{\pgfqpoint{1.453437in}{2.079434in}}{\pgfqpoint{1.450164in}{2.087334in}}{\pgfqpoint{1.444340in}{2.093158in}}%
\pgfpathcurveto{\pgfqpoint{1.438516in}{2.098982in}}{\pgfqpoint{1.430616in}{2.102254in}}{\pgfqpoint{1.422380in}{2.102254in}}%
\pgfpathcurveto{\pgfqpoint{1.414144in}{2.102254in}}{\pgfqpoint{1.406244in}{2.098982in}}{\pgfqpoint{1.400420in}{2.093158in}}%
\pgfpathcurveto{\pgfqpoint{1.394596in}{2.087334in}}{\pgfqpoint{1.391324in}{2.079434in}}{\pgfqpoint{1.391324in}{2.071197in}}%
\pgfpathcurveto{\pgfqpoint{1.391324in}{2.062961in}}{\pgfqpoint{1.394596in}{2.055061in}}{\pgfqpoint{1.400420in}{2.049237in}}%
\pgfpathcurveto{\pgfqpoint{1.406244in}{2.043413in}}{\pgfqpoint{1.414144in}{2.040141in}}{\pgfqpoint{1.422380in}{2.040141in}}%
\pgfpathclose%
\pgfusepath{stroke,fill}%
\end{pgfscope}%
\begin{pgfscope}%
\pgfpathrectangle{\pgfqpoint{0.100000in}{0.212622in}}{\pgfqpoint{3.696000in}{3.696000in}}%
\pgfusepath{clip}%
\pgfsetbuttcap%
\pgfsetroundjoin%
\definecolor{currentfill}{rgb}{0.121569,0.466667,0.705882}%
\pgfsetfillcolor{currentfill}%
\pgfsetfillopacity{0.393465}%
\pgfsetlinewidth{1.003750pt}%
\definecolor{currentstroke}{rgb}{0.121569,0.466667,0.705882}%
\pgfsetstrokecolor{currentstroke}%
\pgfsetstrokeopacity{0.393465}%
\pgfsetdash{}{0pt}%
\pgfpathmoveto{\pgfqpoint{2.189221in}{2.323373in}}%
\pgfpathcurveto{\pgfqpoint{2.197457in}{2.323373in}}{\pgfqpoint{2.205358in}{2.326645in}}{\pgfqpoint{2.211181in}{2.332469in}}%
\pgfpathcurveto{\pgfqpoint{2.217005in}{2.338293in}}{\pgfqpoint{2.220278in}{2.346193in}}{\pgfqpoint{2.220278in}{2.354429in}}%
\pgfpathcurveto{\pgfqpoint{2.220278in}{2.362666in}}{\pgfqpoint{2.217005in}{2.370566in}}{\pgfqpoint{2.211181in}{2.376390in}}%
\pgfpathcurveto{\pgfqpoint{2.205358in}{2.382214in}}{\pgfqpoint{2.197457in}{2.385486in}}{\pgfqpoint{2.189221in}{2.385486in}}%
\pgfpathcurveto{\pgfqpoint{2.180985in}{2.385486in}}{\pgfqpoint{2.173085in}{2.382214in}}{\pgfqpoint{2.167261in}{2.376390in}}%
\pgfpathcurveto{\pgfqpoint{2.161437in}{2.370566in}}{\pgfqpoint{2.158165in}{2.362666in}}{\pgfqpoint{2.158165in}{2.354429in}}%
\pgfpathcurveto{\pgfqpoint{2.158165in}{2.346193in}}{\pgfqpoint{2.161437in}{2.338293in}}{\pgfqpoint{2.167261in}{2.332469in}}%
\pgfpathcurveto{\pgfqpoint{2.173085in}{2.326645in}}{\pgfqpoint{2.180985in}{2.323373in}}{\pgfqpoint{2.189221in}{2.323373in}}%
\pgfpathclose%
\pgfusepath{stroke,fill}%
\end{pgfscope}%
\begin{pgfscope}%
\pgfpathrectangle{\pgfqpoint{0.100000in}{0.212622in}}{\pgfqpoint{3.696000in}{3.696000in}}%
\pgfusepath{clip}%
\pgfsetbuttcap%
\pgfsetroundjoin%
\definecolor{currentfill}{rgb}{0.121569,0.466667,0.705882}%
\pgfsetfillcolor{currentfill}%
\pgfsetfillopacity{0.394720}%
\pgfsetlinewidth{1.003750pt}%
\definecolor{currentstroke}{rgb}{0.121569,0.466667,0.705882}%
\pgfsetstrokecolor{currentstroke}%
\pgfsetstrokeopacity{0.394720}%
\pgfsetdash{}{0pt}%
\pgfpathmoveto{\pgfqpoint{1.414851in}{2.031260in}}%
\pgfpathcurveto{\pgfqpoint{1.423088in}{2.031260in}}{\pgfqpoint{1.430988in}{2.034533in}}{\pgfqpoint{1.436811in}{2.040357in}}%
\pgfpathcurveto{\pgfqpoint{1.442635in}{2.046181in}}{\pgfqpoint{1.445908in}{2.054081in}}{\pgfqpoint{1.445908in}{2.062317in}}%
\pgfpathcurveto{\pgfqpoint{1.445908in}{2.070553in}}{\pgfqpoint{1.442635in}{2.078453in}}{\pgfqpoint{1.436811in}{2.084277in}}%
\pgfpathcurveto{\pgfqpoint{1.430988in}{2.090101in}}{\pgfqpoint{1.423088in}{2.093373in}}{\pgfqpoint{1.414851in}{2.093373in}}%
\pgfpathcurveto{\pgfqpoint{1.406615in}{2.093373in}}{\pgfqpoint{1.398715in}{2.090101in}}{\pgfqpoint{1.392891in}{2.084277in}}%
\pgfpathcurveto{\pgfqpoint{1.387067in}{2.078453in}}{\pgfqpoint{1.383795in}{2.070553in}}{\pgfqpoint{1.383795in}{2.062317in}}%
\pgfpathcurveto{\pgfqpoint{1.383795in}{2.054081in}}{\pgfqpoint{1.387067in}{2.046181in}}{\pgfqpoint{1.392891in}{2.040357in}}%
\pgfpathcurveto{\pgfqpoint{1.398715in}{2.034533in}}{\pgfqpoint{1.406615in}{2.031260in}}{\pgfqpoint{1.414851in}{2.031260in}}%
\pgfpathclose%
\pgfusepath{stroke,fill}%
\end{pgfscope}%
\begin{pgfscope}%
\pgfpathrectangle{\pgfqpoint{0.100000in}{0.212622in}}{\pgfqpoint{3.696000in}{3.696000in}}%
\pgfusepath{clip}%
\pgfsetbuttcap%
\pgfsetroundjoin%
\definecolor{currentfill}{rgb}{0.121569,0.466667,0.705882}%
\pgfsetfillcolor{currentfill}%
\pgfsetfillopacity{0.395977}%
\pgfsetlinewidth{1.003750pt}%
\definecolor{currentstroke}{rgb}{0.121569,0.466667,0.705882}%
\pgfsetstrokecolor{currentstroke}%
\pgfsetstrokeopacity{0.395977}%
\pgfsetdash{}{0pt}%
\pgfpathmoveto{\pgfqpoint{1.411679in}{2.025756in}}%
\pgfpathcurveto{\pgfqpoint{1.419915in}{2.025756in}}{\pgfqpoint{1.427815in}{2.029028in}}{\pgfqpoint{1.433639in}{2.034852in}}%
\pgfpathcurveto{\pgfqpoint{1.439463in}{2.040676in}}{\pgfqpoint{1.442735in}{2.048576in}}{\pgfqpoint{1.442735in}{2.056813in}}%
\pgfpathcurveto{\pgfqpoint{1.442735in}{2.065049in}}{\pgfqpoint{1.439463in}{2.072949in}}{\pgfqpoint{1.433639in}{2.078773in}}%
\pgfpathcurveto{\pgfqpoint{1.427815in}{2.084597in}}{\pgfqpoint{1.419915in}{2.087869in}}{\pgfqpoint{1.411679in}{2.087869in}}%
\pgfpathcurveto{\pgfqpoint{1.403442in}{2.087869in}}{\pgfqpoint{1.395542in}{2.084597in}}{\pgfqpoint{1.389718in}{2.078773in}}%
\pgfpathcurveto{\pgfqpoint{1.383894in}{2.072949in}}{\pgfqpoint{1.380622in}{2.065049in}}{\pgfqpoint{1.380622in}{2.056813in}}%
\pgfpathcurveto{\pgfqpoint{1.380622in}{2.048576in}}{\pgfqpoint{1.383894in}{2.040676in}}{\pgfqpoint{1.389718in}{2.034852in}}%
\pgfpathcurveto{\pgfqpoint{1.395542in}{2.029028in}}{\pgfqpoint{1.403442in}{2.025756in}}{\pgfqpoint{1.411679in}{2.025756in}}%
\pgfpathclose%
\pgfusepath{stroke,fill}%
\end{pgfscope}%
\begin{pgfscope}%
\pgfpathrectangle{\pgfqpoint{0.100000in}{0.212622in}}{\pgfqpoint{3.696000in}{3.696000in}}%
\pgfusepath{clip}%
\pgfsetbuttcap%
\pgfsetroundjoin%
\definecolor{currentfill}{rgb}{0.121569,0.466667,0.705882}%
\pgfsetfillcolor{currentfill}%
\pgfsetfillopacity{0.396600}%
\pgfsetlinewidth{1.003750pt}%
\definecolor{currentstroke}{rgb}{0.121569,0.466667,0.705882}%
\pgfsetstrokecolor{currentstroke}%
\pgfsetstrokeopacity{0.396600}%
\pgfsetdash{}{0pt}%
\pgfpathmoveto{\pgfqpoint{1.410240in}{2.023612in}}%
\pgfpathcurveto{\pgfqpoint{1.418476in}{2.023612in}}{\pgfqpoint{1.426376in}{2.026885in}}{\pgfqpoint{1.432200in}{2.032709in}}%
\pgfpathcurveto{\pgfqpoint{1.438024in}{2.038533in}}{\pgfqpoint{1.441296in}{2.046433in}}{\pgfqpoint{1.441296in}{2.054669in}}%
\pgfpathcurveto{\pgfqpoint{1.441296in}{2.062905in}}{\pgfqpoint{1.438024in}{2.070805in}}{\pgfqpoint{1.432200in}{2.076629in}}%
\pgfpathcurveto{\pgfqpoint{1.426376in}{2.082453in}}{\pgfqpoint{1.418476in}{2.085725in}}{\pgfqpoint{1.410240in}{2.085725in}}%
\pgfpathcurveto{\pgfqpoint{1.402004in}{2.085725in}}{\pgfqpoint{1.394104in}{2.082453in}}{\pgfqpoint{1.388280in}{2.076629in}}%
\pgfpathcurveto{\pgfqpoint{1.382456in}{2.070805in}}{\pgfqpoint{1.379183in}{2.062905in}}{\pgfqpoint{1.379183in}{2.054669in}}%
\pgfpathcurveto{\pgfqpoint{1.379183in}{2.046433in}}{\pgfqpoint{1.382456in}{2.038533in}}{\pgfqpoint{1.388280in}{2.032709in}}%
\pgfpathcurveto{\pgfqpoint{1.394104in}{2.026885in}}{\pgfqpoint{1.402004in}{2.023612in}}{\pgfqpoint{1.410240in}{2.023612in}}%
\pgfpathclose%
\pgfusepath{stroke,fill}%
\end{pgfscope}%
\begin{pgfscope}%
\pgfpathrectangle{\pgfqpoint{0.100000in}{0.212622in}}{\pgfqpoint{3.696000in}{3.696000in}}%
\pgfusepath{clip}%
\pgfsetbuttcap%
\pgfsetroundjoin%
\definecolor{currentfill}{rgb}{0.121569,0.466667,0.705882}%
\pgfsetfillcolor{currentfill}%
\pgfsetfillopacity{0.397589}%
\pgfsetlinewidth{1.003750pt}%
\definecolor{currentstroke}{rgb}{0.121569,0.466667,0.705882}%
\pgfsetstrokecolor{currentstroke}%
\pgfsetstrokeopacity{0.397589}%
\pgfsetdash{}{0pt}%
\pgfpathmoveto{\pgfqpoint{1.407306in}{2.019064in}}%
\pgfpathcurveto{\pgfqpoint{1.415542in}{2.019064in}}{\pgfqpoint{1.423442in}{2.022336in}}{\pgfqpoint{1.429266in}{2.028160in}}%
\pgfpathcurveto{\pgfqpoint{1.435090in}{2.033984in}}{\pgfqpoint{1.438362in}{2.041884in}}{\pgfqpoint{1.438362in}{2.050121in}}%
\pgfpathcurveto{\pgfqpoint{1.438362in}{2.058357in}}{\pgfqpoint{1.435090in}{2.066257in}}{\pgfqpoint{1.429266in}{2.072081in}}%
\pgfpathcurveto{\pgfqpoint{1.423442in}{2.077905in}}{\pgfqpoint{1.415542in}{2.081177in}}{\pgfqpoint{1.407306in}{2.081177in}}%
\pgfpathcurveto{\pgfqpoint{1.399070in}{2.081177in}}{\pgfqpoint{1.391169in}{2.077905in}}{\pgfqpoint{1.385346in}{2.072081in}}%
\pgfpathcurveto{\pgfqpoint{1.379522in}{2.066257in}}{\pgfqpoint{1.376249in}{2.058357in}}{\pgfqpoint{1.376249in}{2.050121in}}%
\pgfpathcurveto{\pgfqpoint{1.376249in}{2.041884in}}{\pgfqpoint{1.379522in}{2.033984in}}{\pgfqpoint{1.385346in}{2.028160in}}%
\pgfpathcurveto{\pgfqpoint{1.391169in}{2.022336in}}{\pgfqpoint{1.399070in}{2.019064in}}{\pgfqpoint{1.407306in}{2.019064in}}%
\pgfpathclose%
\pgfusepath{stroke,fill}%
\end{pgfscope}%
\begin{pgfscope}%
\pgfpathrectangle{\pgfqpoint{0.100000in}{0.212622in}}{\pgfqpoint{3.696000in}{3.696000in}}%
\pgfusepath{clip}%
\pgfsetbuttcap%
\pgfsetroundjoin%
\definecolor{currentfill}{rgb}{0.121569,0.466667,0.705882}%
\pgfsetfillcolor{currentfill}%
\pgfsetfillopacity{0.399780}%
\pgfsetlinewidth{1.003750pt}%
\definecolor{currentstroke}{rgb}{0.121569,0.466667,0.705882}%
\pgfsetstrokecolor{currentstroke}%
\pgfsetstrokeopacity{0.399780}%
\pgfsetdash{}{0pt}%
\pgfpathmoveto{\pgfqpoint{1.403713in}{2.012064in}}%
\pgfpathcurveto{\pgfqpoint{1.411950in}{2.012064in}}{\pgfqpoint{1.419850in}{2.015336in}}{\pgfqpoint{1.425674in}{2.021160in}}%
\pgfpathcurveto{\pgfqpoint{1.431498in}{2.026984in}}{\pgfqpoint{1.434770in}{2.034884in}}{\pgfqpoint{1.434770in}{2.043121in}}%
\pgfpathcurveto{\pgfqpoint{1.434770in}{2.051357in}}{\pgfqpoint{1.431498in}{2.059257in}}{\pgfqpoint{1.425674in}{2.065081in}}%
\pgfpathcurveto{\pgfqpoint{1.419850in}{2.070905in}}{\pgfqpoint{1.411950in}{2.074177in}}{\pgfqpoint{1.403713in}{2.074177in}}%
\pgfpathcurveto{\pgfqpoint{1.395477in}{2.074177in}}{\pgfqpoint{1.387577in}{2.070905in}}{\pgfqpoint{1.381753in}{2.065081in}}%
\pgfpathcurveto{\pgfqpoint{1.375929in}{2.059257in}}{\pgfqpoint{1.372657in}{2.051357in}}{\pgfqpoint{1.372657in}{2.043121in}}%
\pgfpathcurveto{\pgfqpoint{1.372657in}{2.034884in}}{\pgfqpoint{1.375929in}{2.026984in}}{\pgfqpoint{1.381753in}{2.021160in}}%
\pgfpathcurveto{\pgfqpoint{1.387577in}{2.015336in}}{\pgfqpoint{1.395477in}{2.012064in}}{\pgfqpoint{1.403713in}{2.012064in}}%
\pgfpathclose%
\pgfusepath{stroke,fill}%
\end{pgfscope}%
\begin{pgfscope}%
\pgfpathrectangle{\pgfqpoint{0.100000in}{0.212622in}}{\pgfqpoint{3.696000in}{3.696000in}}%
\pgfusepath{clip}%
\pgfsetbuttcap%
\pgfsetroundjoin%
\definecolor{currentfill}{rgb}{0.121569,0.466667,0.705882}%
\pgfsetfillcolor{currentfill}%
\pgfsetfillopacity{0.399872}%
\pgfsetlinewidth{1.003750pt}%
\definecolor{currentstroke}{rgb}{0.121569,0.466667,0.705882}%
\pgfsetstrokecolor{currentstroke}%
\pgfsetstrokeopacity{0.399872}%
\pgfsetdash{}{0pt}%
\pgfpathmoveto{\pgfqpoint{2.229220in}{2.302866in}}%
\pgfpathcurveto{\pgfqpoint{2.237456in}{2.302866in}}{\pgfqpoint{2.245356in}{2.306138in}}{\pgfqpoint{2.251180in}{2.311962in}}%
\pgfpathcurveto{\pgfqpoint{2.257004in}{2.317786in}}{\pgfqpoint{2.260276in}{2.325686in}}{\pgfqpoint{2.260276in}{2.333922in}}%
\pgfpathcurveto{\pgfqpoint{2.260276in}{2.342158in}}{\pgfqpoint{2.257004in}{2.350058in}}{\pgfqpoint{2.251180in}{2.355882in}}%
\pgfpathcurveto{\pgfqpoint{2.245356in}{2.361706in}}{\pgfqpoint{2.237456in}{2.364979in}}{\pgfqpoint{2.229220in}{2.364979in}}%
\pgfpathcurveto{\pgfqpoint{2.220983in}{2.364979in}}{\pgfqpoint{2.213083in}{2.361706in}}{\pgfqpoint{2.207259in}{2.355882in}}%
\pgfpathcurveto{\pgfqpoint{2.201435in}{2.350058in}}{\pgfqpoint{2.198163in}{2.342158in}}{\pgfqpoint{2.198163in}{2.333922in}}%
\pgfpathcurveto{\pgfqpoint{2.198163in}{2.325686in}}{\pgfqpoint{2.201435in}{2.317786in}}{\pgfqpoint{2.207259in}{2.311962in}}%
\pgfpathcurveto{\pgfqpoint{2.213083in}{2.306138in}}{\pgfqpoint{2.220983in}{2.302866in}}{\pgfqpoint{2.229220in}{2.302866in}}%
\pgfpathclose%
\pgfusepath{stroke,fill}%
\end{pgfscope}%
\begin{pgfscope}%
\pgfpathrectangle{\pgfqpoint{0.100000in}{0.212622in}}{\pgfqpoint{3.696000in}{3.696000in}}%
\pgfusepath{clip}%
\pgfsetbuttcap%
\pgfsetroundjoin%
\definecolor{currentfill}{rgb}{0.121569,0.466667,0.705882}%
\pgfsetfillcolor{currentfill}%
\pgfsetfillopacity{0.403426}%
\pgfsetlinewidth{1.003750pt}%
\definecolor{currentstroke}{rgb}{0.121569,0.466667,0.705882}%
\pgfsetstrokecolor{currentstroke}%
\pgfsetstrokeopacity{0.403426}%
\pgfsetdash{}{0pt}%
\pgfpathmoveto{\pgfqpoint{1.394484in}{1.998935in}}%
\pgfpathcurveto{\pgfqpoint{1.402721in}{1.998935in}}{\pgfqpoint{1.410621in}{2.002207in}}{\pgfqpoint{1.416445in}{2.008031in}}%
\pgfpathcurveto{\pgfqpoint{1.422268in}{2.013855in}}{\pgfqpoint{1.425541in}{2.021755in}}{\pgfqpoint{1.425541in}{2.029991in}}%
\pgfpathcurveto{\pgfqpoint{1.425541in}{2.038228in}}{\pgfqpoint{1.422268in}{2.046128in}}{\pgfqpoint{1.416445in}{2.051952in}}%
\pgfpathcurveto{\pgfqpoint{1.410621in}{2.057775in}}{\pgfqpoint{1.402721in}{2.061048in}}{\pgfqpoint{1.394484in}{2.061048in}}%
\pgfpathcurveto{\pgfqpoint{1.386248in}{2.061048in}}{\pgfqpoint{1.378348in}{2.057775in}}{\pgfqpoint{1.372524in}{2.051952in}}%
\pgfpathcurveto{\pgfqpoint{1.366700in}{2.046128in}}{\pgfqpoint{1.363428in}{2.038228in}}{\pgfqpoint{1.363428in}{2.029991in}}%
\pgfpathcurveto{\pgfqpoint{1.363428in}{2.021755in}}{\pgfqpoint{1.366700in}{2.013855in}}{\pgfqpoint{1.372524in}{2.008031in}}%
\pgfpathcurveto{\pgfqpoint{1.378348in}{2.002207in}}{\pgfqpoint{1.386248in}{1.998935in}}{\pgfqpoint{1.394484in}{1.998935in}}%
\pgfpathclose%
\pgfusepath{stroke,fill}%
\end{pgfscope}%
\begin{pgfscope}%
\pgfpathrectangle{\pgfqpoint{0.100000in}{0.212622in}}{\pgfqpoint{3.696000in}{3.696000in}}%
\pgfusepath{clip}%
\pgfsetbuttcap%
\pgfsetroundjoin%
\definecolor{currentfill}{rgb}{0.121569,0.466667,0.705882}%
\pgfsetfillcolor{currentfill}%
\pgfsetfillopacity{0.404727}%
\pgfsetlinewidth{1.003750pt}%
\definecolor{currentstroke}{rgb}{0.121569,0.466667,0.705882}%
\pgfsetstrokecolor{currentstroke}%
\pgfsetstrokeopacity{0.404727}%
\pgfsetdash{}{0pt}%
\pgfpathmoveto{\pgfqpoint{2.254208in}{2.309446in}}%
\pgfpathcurveto{\pgfqpoint{2.262444in}{2.309446in}}{\pgfqpoint{2.270345in}{2.312719in}}{\pgfqpoint{2.276168in}{2.318543in}}%
\pgfpathcurveto{\pgfqpoint{2.281992in}{2.324366in}}{\pgfqpoint{2.285265in}{2.332267in}}{\pgfqpoint{2.285265in}{2.340503in}}%
\pgfpathcurveto{\pgfqpoint{2.285265in}{2.348739in}}{\pgfqpoint{2.281992in}{2.356639in}}{\pgfqpoint{2.276168in}{2.362463in}}%
\pgfpathcurveto{\pgfqpoint{2.270345in}{2.368287in}}{\pgfqpoint{2.262444in}{2.371559in}}{\pgfqpoint{2.254208in}{2.371559in}}%
\pgfpathcurveto{\pgfqpoint{2.245972in}{2.371559in}}{\pgfqpoint{2.238072in}{2.368287in}}{\pgfqpoint{2.232248in}{2.362463in}}%
\pgfpathcurveto{\pgfqpoint{2.226424in}{2.356639in}}{\pgfqpoint{2.223152in}{2.348739in}}{\pgfqpoint{2.223152in}{2.340503in}}%
\pgfpathcurveto{\pgfqpoint{2.223152in}{2.332267in}}{\pgfqpoint{2.226424in}{2.324366in}}{\pgfqpoint{2.232248in}{2.318543in}}%
\pgfpathcurveto{\pgfqpoint{2.238072in}{2.312719in}}{\pgfqpoint{2.245972in}{2.309446in}}{\pgfqpoint{2.254208in}{2.309446in}}%
\pgfpathclose%
\pgfusepath{stroke,fill}%
\end{pgfscope}%
\begin{pgfscope}%
\pgfpathrectangle{\pgfqpoint{0.100000in}{0.212622in}}{\pgfqpoint{3.696000in}{3.696000in}}%
\pgfusepath{clip}%
\pgfsetbuttcap%
\pgfsetroundjoin%
\definecolor{currentfill}{rgb}{0.121569,0.466667,0.705882}%
\pgfsetfillcolor{currentfill}%
\pgfsetfillopacity{0.409687}%
\pgfsetlinewidth{1.003750pt}%
\definecolor{currentstroke}{rgb}{0.121569,0.466667,0.705882}%
\pgfsetstrokecolor{currentstroke}%
\pgfsetstrokeopacity{0.409687}%
\pgfsetdash{}{0pt}%
\pgfpathmoveto{\pgfqpoint{2.284698in}{2.298577in}}%
\pgfpathcurveto{\pgfqpoint{2.292934in}{2.298577in}}{\pgfqpoint{2.300834in}{2.301850in}}{\pgfqpoint{2.306658in}{2.307674in}}%
\pgfpathcurveto{\pgfqpoint{2.312482in}{2.313498in}}{\pgfqpoint{2.315754in}{2.321398in}}{\pgfqpoint{2.315754in}{2.329634in}}%
\pgfpathcurveto{\pgfqpoint{2.315754in}{2.337870in}}{\pgfqpoint{2.312482in}{2.345770in}}{\pgfqpoint{2.306658in}{2.351594in}}%
\pgfpathcurveto{\pgfqpoint{2.300834in}{2.357418in}}{\pgfqpoint{2.292934in}{2.360690in}}{\pgfqpoint{2.284698in}{2.360690in}}%
\pgfpathcurveto{\pgfqpoint{2.276462in}{2.360690in}}{\pgfqpoint{2.268562in}{2.357418in}}{\pgfqpoint{2.262738in}{2.351594in}}%
\pgfpathcurveto{\pgfqpoint{2.256914in}{2.345770in}}{\pgfqpoint{2.253641in}{2.337870in}}{\pgfqpoint{2.253641in}{2.329634in}}%
\pgfpathcurveto{\pgfqpoint{2.253641in}{2.321398in}}{\pgfqpoint{2.256914in}{2.313498in}}{\pgfqpoint{2.262738in}{2.307674in}}%
\pgfpathcurveto{\pgfqpoint{2.268562in}{2.301850in}}{\pgfqpoint{2.276462in}{2.298577in}}{\pgfqpoint{2.284698in}{2.298577in}}%
\pgfpathclose%
\pgfusepath{stroke,fill}%
\end{pgfscope}%
\begin{pgfscope}%
\pgfpathrectangle{\pgfqpoint{0.100000in}{0.212622in}}{\pgfqpoint{3.696000in}{3.696000in}}%
\pgfusepath{clip}%
\pgfsetbuttcap%
\pgfsetroundjoin%
\definecolor{currentfill}{rgb}{0.121569,0.466667,0.705882}%
\pgfsetfillcolor{currentfill}%
\pgfsetfillopacity{0.410357}%
\pgfsetlinewidth{1.003750pt}%
\definecolor{currentstroke}{rgb}{0.121569,0.466667,0.705882}%
\pgfsetstrokecolor{currentstroke}%
\pgfsetstrokeopacity{0.410357}%
\pgfsetdash{}{0pt}%
\pgfpathmoveto{\pgfqpoint{1.379277in}{1.975608in}}%
\pgfpathcurveto{\pgfqpoint{1.387514in}{1.975608in}}{\pgfqpoint{1.395414in}{1.978881in}}{\pgfqpoint{1.401238in}{1.984705in}}%
\pgfpathcurveto{\pgfqpoint{1.407062in}{1.990528in}}{\pgfqpoint{1.410334in}{1.998429in}}{\pgfqpoint{1.410334in}{2.006665in}}%
\pgfpathcurveto{\pgfqpoint{1.410334in}{2.014901in}}{\pgfqpoint{1.407062in}{2.022801in}}{\pgfqpoint{1.401238in}{2.028625in}}%
\pgfpathcurveto{\pgfqpoint{1.395414in}{2.034449in}}{\pgfqpoint{1.387514in}{2.037721in}}{\pgfqpoint{1.379277in}{2.037721in}}%
\pgfpathcurveto{\pgfqpoint{1.371041in}{2.037721in}}{\pgfqpoint{1.363141in}{2.034449in}}{\pgfqpoint{1.357317in}{2.028625in}}%
\pgfpathcurveto{\pgfqpoint{1.351493in}{2.022801in}}{\pgfqpoint{1.348221in}{2.014901in}}{\pgfqpoint{1.348221in}{2.006665in}}%
\pgfpathcurveto{\pgfqpoint{1.348221in}{1.998429in}}{\pgfqpoint{1.351493in}{1.990528in}}{\pgfqpoint{1.357317in}{1.984705in}}%
\pgfpathcurveto{\pgfqpoint{1.363141in}{1.978881in}}{\pgfqpoint{1.371041in}{1.975608in}}{\pgfqpoint{1.379277in}{1.975608in}}%
\pgfpathclose%
\pgfusepath{stroke,fill}%
\end{pgfscope}%
\begin{pgfscope}%
\pgfpathrectangle{\pgfqpoint{0.100000in}{0.212622in}}{\pgfqpoint{3.696000in}{3.696000in}}%
\pgfusepath{clip}%
\pgfsetbuttcap%
\pgfsetroundjoin%
\definecolor{currentfill}{rgb}{0.121569,0.466667,0.705882}%
\pgfsetfillcolor{currentfill}%
\pgfsetfillopacity{0.416007}%
\pgfsetlinewidth{1.003750pt}%
\definecolor{currentstroke}{rgb}{0.121569,0.466667,0.705882}%
\pgfsetstrokecolor{currentstroke}%
\pgfsetstrokeopacity{0.416007}%
\pgfsetdash{}{0pt}%
\pgfpathmoveto{\pgfqpoint{2.322059in}{2.311702in}}%
\pgfpathcurveto{\pgfqpoint{2.330295in}{2.311702in}}{\pgfqpoint{2.338195in}{2.314975in}}{\pgfqpoint{2.344019in}{2.320799in}}%
\pgfpathcurveto{\pgfqpoint{2.349843in}{2.326623in}}{\pgfqpoint{2.353115in}{2.334523in}}{\pgfqpoint{2.353115in}{2.342759in}}%
\pgfpathcurveto{\pgfqpoint{2.353115in}{2.350995in}}{\pgfqpoint{2.349843in}{2.358895in}}{\pgfqpoint{2.344019in}{2.364719in}}%
\pgfpathcurveto{\pgfqpoint{2.338195in}{2.370543in}}{\pgfqpoint{2.330295in}{2.373815in}}{\pgfqpoint{2.322059in}{2.373815in}}%
\pgfpathcurveto{\pgfqpoint{2.313822in}{2.373815in}}{\pgfqpoint{2.305922in}{2.370543in}}{\pgfqpoint{2.300098in}{2.364719in}}%
\pgfpathcurveto{\pgfqpoint{2.294274in}{2.358895in}}{\pgfqpoint{2.291002in}{2.350995in}}{\pgfqpoint{2.291002in}{2.342759in}}%
\pgfpathcurveto{\pgfqpoint{2.291002in}{2.334523in}}{\pgfqpoint{2.294274in}{2.326623in}}{\pgfqpoint{2.300098in}{2.320799in}}%
\pgfpathcurveto{\pgfqpoint{2.305922in}{2.314975in}}{\pgfqpoint{2.313822in}{2.311702in}}{\pgfqpoint{2.322059in}{2.311702in}}%
\pgfpathclose%
\pgfusepath{stroke,fill}%
\end{pgfscope}%
\begin{pgfscope}%
\pgfpathrectangle{\pgfqpoint{0.100000in}{0.212622in}}{\pgfqpoint{3.696000in}{3.696000in}}%
\pgfusepath{clip}%
\pgfsetbuttcap%
\pgfsetroundjoin%
\definecolor{currentfill}{rgb}{0.121569,0.466667,0.705882}%
\pgfsetfillcolor{currentfill}%
\pgfsetfillopacity{0.418570}%
\pgfsetlinewidth{1.003750pt}%
\definecolor{currentstroke}{rgb}{0.121569,0.466667,0.705882}%
\pgfsetstrokecolor{currentstroke}%
\pgfsetstrokeopacity{0.418570}%
\pgfsetdash{}{0pt}%
\pgfpathmoveto{\pgfqpoint{2.339211in}{2.301311in}}%
\pgfpathcurveto{\pgfqpoint{2.347447in}{2.301311in}}{\pgfqpoint{2.355347in}{2.304583in}}{\pgfqpoint{2.361171in}{2.310407in}}%
\pgfpathcurveto{\pgfqpoint{2.366995in}{2.316231in}}{\pgfqpoint{2.370267in}{2.324131in}}{\pgfqpoint{2.370267in}{2.332368in}}%
\pgfpathcurveto{\pgfqpoint{2.370267in}{2.340604in}}{\pgfqpoint{2.366995in}{2.348504in}}{\pgfqpoint{2.361171in}{2.354328in}}%
\pgfpathcurveto{\pgfqpoint{2.355347in}{2.360152in}}{\pgfqpoint{2.347447in}{2.363424in}}{\pgfqpoint{2.339211in}{2.363424in}}%
\pgfpathcurveto{\pgfqpoint{2.330974in}{2.363424in}}{\pgfqpoint{2.323074in}{2.360152in}}{\pgfqpoint{2.317250in}{2.354328in}}%
\pgfpathcurveto{\pgfqpoint{2.311426in}{2.348504in}}{\pgfqpoint{2.308154in}{2.340604in}}{\pgfqpoint{2.308154in}{2.332368in}}%
\pgfpathcurveto{\pgfqpoint{2.308154in}{2.324131in}}{\pgfqpoint{2.311426in}{2.316231in}}{\pgfqpoint{2.317250in}{2.310407in}}%
\pgfpathcurveto{\pgfqpoint{2.323074in}{2.304583in}}{\pgfqpoint{2.330974in}{2.301311in}}{\pgfqpoint{2.339211in}{2.301311in}}%
\pgfpathclose%
\pgfusepath{stroke,fill}%
\end{pgfscope}%
\begin{pgfscope}%
\pgfpathrectangle{\pgfqpoint{0.100000in}{0.212622in}}{\pgfqpoint{3.696000in}{3.696000in}}%
\pgfusepath{clip}%
\pgfsetbuttcap%
\pgfsetroundjoin%
\definecolor{currentfill}{rgb}{0.121569,0.466667,0.705882}%
\pgfsetfillcolor{currentfill}%
\pgfsetfillopacity{0.420342}%
\pgfsetlinewidth{1.003750pt}%
\definecolor{currentstroke}{rgb}{0.121569,0.466667,0.705882}%
\pgfsetstrokecolor{currentstroke}%
\pgfsetstrokeopacity{0.420342}%
\pgfsetdash{}{0pt}%
\pgfpathmoveto{\pgfqpoint{2.350630in}{2.304750in}}%
\pgfpathcurveto{\pgfqpoint{2.358866in}{2.304750in}}{\pgfqpoint{2.366766in}{2.308023in}}{\pgfqpoint{2.372590in}{2.313847in}}%
\pgfpathcurveto{\pgfqpoint{2.378414in}{2.319671in}}{\pgfqpoint{2.381687in}{2.327571in}}{\pgfqpoint{2.381687in}{2.335807in}}%
\pgfpathcurveto{\pgfqpoint{2.381687in}{2.344043in}}{\pgfqpoint{2.378414in}{2.351943in}}{\pgfqpoint{2.372590in}{2.357767in}}%
\pgfpathcurveto{\pgfqpoint{2.366766in}{2.363591in}}{\pgfqpoint{2.358866in}{2.366863in}}{\pgfqpoint{2.350630in}{2.366863in}}%
\pgfpathcurveto{\pgfqpoint{2.342394in}{2.366863in}}{\pgfqpoint{2.334494in}{2.363591in}}{\pgfqpoint{2.328670in}{2.357767in}}%
\pgfpathcurveto{\pgfqpoint{2.322846in}{2.351943in}}{\pgfqpoint{2.319574in}{2.344043in}}{\pgfqpoint{2.319574in}{2.335807in}}%
\pgfpathcurveto{\pgfqpoint{2.319574in}{2.327571in}}{\pgfqpoint{2.322846in}{2.319671in}}{\pgfqpoint{2.328670in}{2.313847in}}%
\pgfpathcurveto{\pgfqpoint{2.334494in}{2.308023in}}{\pgfqpoint{2.342394in}{2.304750in}}{\pgfqpoint{2.350630in}{2.304750in}}%
\pgfpathclose%
\pgfusepath{stroke,fill}%
\end{pgfscope}%
\begin{pgfscope}%
\pgfpathrectangle{\pgfqpoint{0.100000in}{0.212622in}}{\pgfqpoint{3.696000in}{3.696000in}}%
\pgfusepath{clip}%
\pgfsetbuttcap%
\pgfsetroundjoin%
\definecolor{currentfill}{rgb}{0.121569,0.466667,0.705882}%
\pgfsetfillcolor{currentfill}%
\pgfsetfillopacity{0.421161}%
\pgfsetlinewidth{1.003750pt}%
\definecolor{currentstroke}{rgb}{0.121569,0.466667,0.705882}%
\pgfsetstrokecolor{currentstroke}%
\pgfsetstrokeopacity{0.421161}%
\pgfsetdash{}{0pt}%
\pgfpathmoveto{\pgfqpoint{2.355962in}{2.302247in}}%
\pgfpathcurveto{\pgfqpoint{2.364199in}{2.302247in}}{\pgfqpoint{2.372099in}{2.305519in}}{\pgfqpoint{2.377923in}{2.311343in}}%
\pgfpathcurveto{\pgfqpoint{2.383746in}{2.317167in}}{\pgfqpoint{2.387019in}{2.325067in}}{\pgfqpoint{2.387019in}{2.333303in}}%
\pgfpathcurveto{\pgfqpoint{2.387019in}{2.341539in}}{\pgfqpoint{2.383746in}{2.349439in}}{\pgfqpoint{2.377923in}{2.355263in}}%
\pgfpathcurveto{\pgfqpoint{2.372099in}{2.361087in}}{\pgfqpoint{2.364199in}{2.364360in}}{\pgfqpoint{2.355962in}{2.364360in}}%
\pgfpathcurveto{\pgfqpoint{2.347726in}{2.364360in}}{\pgfqpoint{2.339826in}{2.361087in}}{\pgfqpoint{2.334002in}{2.355263in}}%
\pgfpathcurveto{\pgfqpoint{2.328178in}{2.349439in}}{\pgfqpoint{2.324906in}{2.341539in}}{\pgfqpoint{2.324906in}{2.333303in}}%
\pgfpathcurveto{\pgfqpoint{2.324906in}{2.325067in}}{\pgfqpoint{2.328178in}{2.317167in}}{\pgfqpoint{2.334002in}{2.311343in}}%
\pgfpathcurveto{\pgfqpoint{2.339826in}{2.305519in}}{\pgfqpoint{2.347726in}{2.302247in}}{\pgfqpoint{2.355962in}{2.302247in}}%
\pgfpathclose%
\pgfusepath{stroke,fill}%
\end{pgfscope}%
\begin{pgfscope}%
\pgfpathrectangle{\pgfqpoint{0.100000in}{0.212622in}}{\pgfqpoint{3.696000in}{3.696000in}}%
\pgfusepath{clip}%
\pgfsetbuttcap%
\pgfsetroundjoin%
\definecolor{currentfill}{rgb}{0.121569,0.466667,0.705882}%
\pgfsetfillcolor{currentfill}%
\pgfsetfillopacity{0.421720}%
\pgfsetlinewidth{1.003750pt}%
\definecolor{currentstroke}{rgb}{0.121569,0.466667,0.705882}%
\pgfsetstrokecolor{currentstroke}%
\pgfsetstrokeopacity{0.421720}%
\pgfsetdash{}{0pt}%
\pgfpathmoveto{\pgfqpoint{2.359352in}{2.303130in}}%
\pgfpathcurveto{\pgfqpoint{2.367589in}{2.303130in}}{\pgfqpoint{2.375489in}{2.306402in}}{\pgfqpoint{2.381313in}{2.312226in}}%
\pgfpathcurveto{\pgfqpoint{2.387136in}{2.318050in}}{\pgfqpoint{2.390409in}{2.325950in}}{\pgfqpoint{2.390409in}{2.334187in}}%
\pgfpathcurveto{\pgfqpoint{2.390409in}{2.342423in}}{\pgfqpoint{2.387136in}{2.350323in}}{\pgfqpoint{2.381313in}{2.356147in}}%
\pgfpathcurveto{\pgfqpoint{2.375489in}{2.361971in}}{\pgfqpoint{2.367589in}{2.365243in}}{\pgfqpoint{2.359352in}{2.365243in}}%
\pgfpathcurveto{\pgfqpoint{2.351116in}{2.365243in}}{\pgfqpoint{2.343216in}{2.361971in}}{\pgfqpoint{2.337392in}{2.356147in}}%
\pgfpathcurveto{\pgfqpoint{2.331568in}{2.350323in}}{\pgfqpoint{2.328296in}{2.342423in}}{\pgfqpoint{2.328296in}{2.334187in}}%
\pgfpathcurveto{\pgfqpoint{2.328296in}{2.325950in}}{\pgfqpoint{2.331568in}{2.318050in}}{\pgfqpoint{2.337392in}{2.312226in}}%
\pgfpathcurveto{\pgfqpoint{2.343216in}{2.306402in}}{\pgfqpoint{2.351116in}{2.303130in}}{\pgfqpoint{2.359352in}{2.303130in}}%
\pgfpathclose%
\pgfusepath{stroke,fill}%
\end{pgfscope}%
\begin{pgfscope}%
\pgfpathrectangle{\pgfqpoint{0.100000in}{0.212622in}}{\pgfqpoint{3.696000in}{3.696000in}}%
\pgfusepath{clip}%
\pgfsetbuttcap%
\pgfsetroundjoin%
\definecolor{currentfill}{rgb}{0.121569,0.466667,0.705882}%
\pgfsetfillcolor{currentfill}%
\pgfsetfillopacity{0.422651}%
\pgfsetlinewidth{1.003750pt}%
\definecolor{currentstroke}{rgb}{0.121569,0.466667,0.705882}%
\pgfsetstrokecolor{currentstroke}%
\pgfsetstrokeopacity{0.422651}%
\pgfsetdash{}{0pt}%
\pgfpathmoveto{\pgfqpoint{2.364209in}{2.300810in}}%
\pgfpathcurveto{\pgfqpoint{2.372445in}{2.300810in}}{\pgfqpoint{2.380345in}{2.304082in}}{\pgfqpoint{2.386169in}{2.309906in}}%
\pgfpathcurveto{\pgfqpoint{2.391993in}{2.315730in}}{\pgfqpoint{2.395265in}{2.323630in}}{\pgfqpoint{2.395265in}{2.331866in}}%
\pgfpathcurveto{\pgfqpoint{2.395265in}{2.340102in}}{\pgfqpoint{2.391993in}{2.348002in}}{\pgfqpoint{2.386169in}{2.353826in}}%
\pgfpathcurveto{\pgfqpoint{2.380345in}{2.359650in}}{\pgfqpoint{2.372445in}{2.362923in}}{\pgfqpoint{2.364209in}{2.362923in}}%
\pgfpathcurveto{\pgfqpoint{2.355972in}{2.362923in}}{\pgfqpoint{2.348072in}{2.359650in}}{\pgfqpoint{2.342248in}{2.353826in}}%
\pgfpathcurveto{\pgfqpoint{2.336425in}{2.348002in}}{\pgfqpoint{2.333152in}{2.340102in}}{\pgfqpoint{2.333152in}{2.331866in}}%
\pgfpathcurveto{\pgfqpoint{2.333152in}{2.323630in}}{\pgfqpoint{2.336425in}{2.315730in}}{\pgfqpoint{2.342248in}{2.309906in}}%
\pgfpathcurveto{\pgfqpoint{2.348072in}{2.304082in}}{\pgfqpoint{2.355972in}{2.300810in}}{\pgfqpoint{2.364209in}{2.300810in}}%
\pgfpathclose%
\pgfusepath{stroke,fill}%
\end{pgfscope}%
\begin{pgfscope}%
\pgfpathrectangle{\pgfqpoint{0.100000in}{0.212622in}}{\pgfqpoint{3.696000in}{3.696000in}}%
\pgfusepath{clip}%
\pgfsetbuttcap%
\pgfsetroundjoin%
\definecolor{currentfill}{rgb}{0.121569,0.466667,0.705882}%
\pgfsetfillcolor{currentfill}%
\pgfsetfillopacity{0.422836}%
\pgfsetlinewidth{1.003750pt}%
\definecolor{currentstroke}{rgb}{0.121569,0.466667,0.705882}%
\pgfsetstrokecolor{currentstroke}%
\pgfsetstrokeopacity{0.422836}%
\pgfsetdash{}{0pt}%
\pgfpathmoveto{\pgfqpoint{1.341704in}{1.941139in}}%
\pgfpathcurveto{\pgfqpoint{1.349940in}{1.941139in}}{\pgfqpoint{1.357840in}{1.944411in}}{\pgfqpoint{1.363664in}{1.950235in}}%
\pgfpathcurveto{\pgfqpoint{1.369488in}{1.956059in}}{\pgfqpoint{1.372760in}{1.963959in}}{\pgfqpoint{1.372760in}{1.972195in}}%
\pgfpathcurveto{\pgfqpoint{1.372760in}{1.980431in}}{\pgfqpoint{1.369488in}{1.988331in}}{\pgfqpoint{1.363664in}{1.994155in}}%
\pgfpathcurveto{\pgfqpoint{1.357840in}{1.999979in}}{\pgfqpoint{1.349940in}{2.003252in}}{\pgfqpoint{1.341704in}{2.003252in}}%
\pgfpathcurveto{\pgfqpoint{1.333467in}{2.003252in}}{\pgfqpoint{1.325567in}{1.999979in}}{\pgfqpoint{1.319744in}{1.994155in}}%
\pgfpathcurveto{\pgfqpoint{1.313920in}{1.988331in}}{\pgfqpoint{1.310647in}{1.980431in}}{\pgfqpoint{1.310647in}{1.972195in}}%
\pgfpathcurveto{\pgfqpoint{1.310647in}{1.963959in}}{\pgfqpoint{1.313920in}{1.956059in}}{\pgfqpoint{1.319744in}{1.950235in}}%
\pgfpathcurveto{\pgfqpoint{1.325567in}{1.944411in}}{\pgfqpoint{1.333467in}{1.941139in}}{\pgfqpoint{1.341704in}{1.941139in}}%
\pgfpathclose%
\pgfusepath{stroke,fill}%
\end{pgfscope}%
\begin{pgfscope}%
\pgfpathrectangle{\pgfqpoint{0.100000in}{0.212622in}}{\pgfqpoint{3.696000in}{3.696000in}}%
\pgfusepath{clip}%
\pgfsetbuttcap%
\pgfsetroundjoin%
\definecolor{currentfill}{rgb}{0.121569,0.466667,0.705882}%
\pgfsetfillcolor{currentfill}%
\pgfsetfillopacity{0.424657}%
\pgfsetlinewidth{1.003750pt}%
\definecolor{currentstroke}{rgb}{0.121569,0.466667,0.705882}%
\pgfsetstrokecolor{currentstroke}%
\pgfsetstrokeopacity{0.424657}%
\pgfsetdash{}{0pt}%
\pgfpathmoveto{\pgfqpoint{2.374386in}{2.304299in}}%
\pgfpathcurveto{\pgfqpoint{2.382622in}{2.304299in}}{\pgfqpoint{2.390522in}{2.307572in}}{\pgfqpoint{2.396346in}{2.313396in}}%
\pgfpathcurveto{\pgfqpoint{2.402170in}{2.319219in}}{\pgfqpoint{2.405443in}{2.327119in}}{\pgfqpoint{2.405443in}{2.335356in}}%
\pgfpathcurveto{\pgfqpoint{2.405443in}{2.343592in}}{\pgfqpoint{2.402170in}{2.351492in}}{\pgfqpoint{2.396346in}{2.357316in}}%
\pgfpathcurveto{\pgfqpoint{2.390522in}{2.363140in}}{\pgfqpoint{2.382622in}{2.366412in}}{\pgfqpoint{2.374386in}{2.366412in}}%
\pgfpathcurveto{\pgfqpoint{2.366150in}{2.366412in}}{\pgfqpoint{2.358250in}{2.363140in}}{\pgfqpoint{2.352426in}{2.357316in}}%
\pgfpathcurveto{\pgfqpoint{2.346602in}{2.351492in}}{\pgfqpoint{2.343330in}{2.343592in}}{\pgfqpoint{2.343330in}{2.335356in}}%
\pgfpathcurveto{\pgfqpoint{2.343330in}{2.327119in}}{\pgfqpoint{2.346602in}{2.319219in}}{\pgfqpoint{2.352426in}{2.313396in}}%
\pgfpathcurveto{\pgfqpoint{2.358250in}{2.307572in}}{\pgfqpoint{2.366150in}{2.304299in}}{\pgfqpoint{2.374386in}{2.304299in}}%
\pgfpathclose%
\pgfusepath{stroke,fill}%
\end{pgfscope}%
\begin{pgfscope}%
\pgfpathrectangle{\pgfqpoint{0.100000in}{0.212622in}}{\pgfqpoint{3.696000in}{3.696000in}}%
\pgfusepath{clip}%
\pgfsetbuttcap%
\pgfsetroundjoin%
\definecolor{currentfill}{rgb}{0.121569,0.466667,0.705882}%
\pgfsetfillcolor{currentfill}%
\pgfsetfillopacity{0.426965}%
\pgfsetlinewidth{1.003750pt}%
\definecolor{currentstroke}{rgb}{0.121569,0.466667,0.705882}%
\pgfsetstrokecolor{currentstroke}%
\pgfsetstrokeopacity{0.426965}%
\pgfsetdash{}{0pt}%
\pgfpathmoveto{\pgfqpoint{2.385007in}{2.301654in}}%
\pgfpathcurveto{\pgfqpoint{2.393243in}{2.301654in}}{\pgfqpoint{2.401143in}{2.304926in}}{\pgfqpoint{2.406967in}{2.310750in}}%
\pgfpathcurveto{\pgfqpoint{2.412791in}{2.316574in}}{\pgfqpoint{2.416064in}{2.324474in}}{\pgfqpoint{2.416064in}{2.332710in}}%
\pgfpathcurveto{\pgfqpoint{2.416064in}{2.340946in}}{\pgfqpoint{2.412791in}{2.348846in}}{\pgfqpoint{2.406967in}{2.354670in}}%
\pgfpathcurveto{\pgfqpoint{2.401143in}{2.360494in}}{\pgfqpoint{2.393243in}{2.363767in}}{\pgfqpoint{2.385007in}{2.363767in}}%
\pgfpathcurveto{\pgfqpoint{2.376771in}{2.363767in}}{\pgfqpoint{2.368871in}{2.360494in}}{\pgfqpoint{2.363047in}{2.354670in}}%
\pgfpathcurveto{\pgfqpoint{2.357223in}{2.348846in}}{\pgfqpoint{2.353951in}{2.340946in}}{\pgfqpoint{2.353951in}{2.332710in}}%
\pgfpathcurveto{\pgfqpoint{2.353951in}{2.324474in}}{\pgfqpoint{2.357223in}{2.316574in}}{\pgfqpoint{2.363047in}{2.310750in}}%
\pgfpathcurveto{\pgfqpoint{2.368871in}{2.304926in}}{\pgfqpoint{2.376771in}{2.301654in}}{\pgfqpoint{2.385007in}{2.301654in}}%
\pgfpathclose%
\pgfusepath{stroke,fill}%
\end{pgfscope}%
\begin{pgfscope}%
\pgfpathrectangle{\pgfqpoint{0.100000in}{0.212622in}}{\pgfqpoint{3.696000in}{3.696000in}}%
\pgfusepath{clip}%
\pgfsetbuttcap%
\pgfsetroundjoin%
\definecolor{currentfill}{rgb}{0.121569,0.466667,0.705882}%
\pgfsetfillcolor{currentfill}%
\pgfsetfillopacity{0.430220}%
\pgfsetlinewidth{1.003750pt}%
\definecolor{currentstroke}{rgb}{0.121569,0.466667,0.705882}%
\pgfsetstrokecolor{currentstroke}%
\pgfsetstrokeopacity{0.430220}%
\pgfsetdash{}{0pt}%
\pgfpathmoveto{\pgfqpoint{2.400930in}{2.303908in}}%
\pgfpathcurveto{\pgfqpoint{2.409166in}{2.303908in}}{\pgfqpoint{2.417066in}{2.307180in}}{\pgfqpoint{2.422890in}{2.313004in}}%
\pgfpathcurveto{\pgfqpoint{2.428714in}{2.318828in}}{\pgfqpoint{2.431987in}{2.326728in}}{\pgfqpoint{2.431987in}{2.334964in}}%
\pgfpathcurveto{\pgfqpoint{2.431987in}{2.343200in}}{\pgfqpoint{2.428714in}{2.351100in}}{\pgfqpoint{2.422890in}{2.356924in}}%
\pgfpathcurveto{\pgfqpoint{2.417066in}{2.362748in}}{\pgfqpoint{2.409166in}{2.366021in}}{\pgfqpoint{2.400930in}{2.366021in}}%
\pgfpathcurveto{\pgfqpoint{2.392694in}{2.366021in}}{\pgfqpoint{2.384794in}{2.362748in}}{\pgfqpoint{2.378970in}{2.356924in}}%
\pgfpathcurveto{\pgfqpoint{2.373146in}{2.351100in}}{\pgfqpoint{2.369874in}{2.343200in}}{\pgfqpoint{2.369874in}{2.334964in}}%
\pgfpathcurveto{\pgfqpoint{2.369874in}{2.326728in}}{\pgfqpoint{2.373146in}{2.318828in}}{\pgfqpoint{2.378970in}{2.313004in}}%
\pgfpathcurveto{\pgfqpoint{2.384794in}{2.307180in}}{\pgfqpoint{2.392694in}{2.303908in}}{\pgfqpoint{2.400930in}{2.303908in}}%
\pgfpathclose%
\pgfusepath{stroke,fill}%
\end{pgfscope}%
\begin{pgfscope}%
\pgfpathrectangle{\pgfqpoint{0.100000in}{0.212622in}}{\pgfqpoint{3.696000in}{3.696000in}}%
\pgfusepath{clip}%
\pgfsetbuttcap%
\pgfsetroundjoin%
\definecolor{currentfill}{rgb}{0.121569,0.466667,0.705882}%
\pgfsetfillcolor{currentfill}%
\pgfsetfillopacity{0.431760}%
\pgfsetlinewidth{1.003750pt}%
\definecolor{currentstroke}{rgb}{0.121569,0.466667,0.705882}%
\pgfsetstrokecolor{currentstroke}%
\pgfsetstrokeopacity{0.431760}%
\pgfsetdash{}{0pt}%
\pgfpathmoveto{\pgfqpoint{2.409444in}{2.302708in}}%
\pgfpathcurveto{\pgfqpoint{2.417680in}{2.302708in}}{\pgfqpoint{2.425580in}{2.305980in}}{\pgfqpoint{2.431404in}{2.311804in}}%
\pgfpathcurveto{\pgfqpoint{2.437228in}{2.317628in}}{\pgfqpoint{2.440500in}{2.325528in}}{\pgfqpoint{2.440500in}{2.333765in}}%
\pgfpathcurveto{\pgfqpoint{2.440500in}{2.342001in}}{\pgfqpoint{2.437228in}{2.349901in}}{\pgfqpoint{2.431404in}{2.355725in}}%
\pgfpathcurveto{\pgfqpoint{2.425580in}{2.361549in}}{\pgfqpoint{2.417680in}{2.364821in}}{\pgfqpoint{2.409444in}{2.364821in}}%
\pgfpathcurveto{\pgfqpoint{2.401207in}{2.364821in}}{\pgfqpoint{2.393307in}{2.361549in}}{\pgfqpoint{2.387484in}{2.355725in}}%
\pgfpathcurveto{\pgfqpoint{2.381660in}{2.349901in}}{\pgfqpoint{2.378387in}{2.342001in}}{\pgfqpoint{2.378387in}{2.333765in}}%
\pgfpathcurveto{\pgfqpoint{2.378387in}{2.325528in}}{\pgfqpoint{2.381660in}{2.317628in}}{\pgfqpoint{2.387484in}{2.311804in}}%
\pgfpathcurveto{\pgfqpoint{2.393307in}{2.305980in}}{\pgfqpoint{2.401207in}{2.302708in}}{\pgfqpoint{2.409444in}{2.302708in}}%
\pgfpathclose%
\pgfusepath{stroke,fill}%
\end{pgfscope}%
\begin{pgfscope}%
\pgfpathrectangle{\pgfqpoint{0.100000in}{0.212622in}}{\pgfqpoint{3.696000in}{3.696000in}}%
\pgfusepath{clip}%
\pgfsetbuttcap%
\pgfsetroundjoin%
\definecolor{currentfill}{rgb}{0.121569,0.466667,0.705882}%
\pgfsetfillcolor{currentfill}%
\pgfsetfillopacity{0.433848}%
\pgfsetlinewidth{1.003750pt}%
\definecolor{currentstroke}{rgb}{0.121569,0.466667,0.705882}%
\pgfsetstrokecolor{currentstroke}%
\pgfsetstrokeopacity{0.433848}%
\pgfsetdash{}{0pt}%
\pgfpathmoveto{\pgfqpoint{1.312856in}{1.893710in}}%
\pgfpathcurveto{\pgfqpoint{1.321092in}{1.893710in}}{\pgfqpoint{1.328992in}{1.896982in}}{\pgfqpoint{1.334816in}{1.902806in}}%
\pgfpathcurveto{\pgfqpoint{1.340640in}{1.908630in}}{\pgfqpoint{1.343912in}{1.916530in}}{\pgfqpoint{1.343912in}{1.924766in}}%
\pgfpathcurveto{\pgfqpoint{1.343912in}{1.933003in}}{\pgfqpoint{1.340640in}{1.940903in}}{\pgfqpoint{1.334816in}{1.946727in}}%
\pgfpathcurveto{\pgfqpoint{1.328992in}{1.952551in}}{\pgfqpoint{1.321092in}{1.955823in}}{\pgfqpoint{1.312856in}{1.955823in}}%
\pgfpathcurveto{\pgfqpoint{1.304619in}{1.955823in}}{\pgfqpoint{1.296719in}{1.952551in}}{\pgfqpoint{1.290895in}{1.946727in}}%
\pgfpathcurveto{\pgfqpoint{1.285071in}{1.940903in}}{\pgfqpoint{1.281799in}{1.933003in}}{\pgfqpoint{1.281799in}{1.924766in}}%
\pgfpathcurveto{\pgfqpoint{1.281799in}{1.916530in}}{\pgfqpoint{1.285071in}{1.908630in}}{\pgfqpoint{1.290895in}{1.902806in}}%
\pgfpathcurveto{\pgfqpoint{1.296719in}{1.896982in}}{\pgfqpoint{1.304619in}{1.893710in}}{\pgfqpoint{1.312856in}{1.893710in}}%
\pgfpathclose%
\pgfusepath{stroke,fill}%
\end{pgfscope}%
\begin{pgfscope}%
\pgfpathrectangle{\pgfqpoint{0.100000in}{0.212622in}}{\pgfqpoint{3.696000in}{3.696000in}}%
\pgfusepath{clip}%
\pgfsetbuttcap%
\pgfsetroundjoin%
\definecolor{currentfill}{rgb}{0.121569,0.466667,0.705882}%
\pgfsetfillcolor{currentfill}%
\pgfsetfillopacity{0.434806}%
\pgfsetlinewidth{1.003750pt}%
\definecolor{currentstroke}{rgb}{0.121569,0.466667,0.705882}%
\pgfsetstrokecolor{currentstroke}%
\pgfsetstrokeopacity{0.434806}%
\pgfsetdash{}{0pt}%
\pgfpathmoveto{\pgfqpoint{2.423509in}{2.303741in}}%
\pgfpathcurveto{\pgfqpoint{2.431745in}{2.303741in}}{\pgfqpoint{2.439645in}{2.307014in}}{\pgfqpoint{2.445469in}{2.312837in}}%
\pgfpathcurveto{\pgfqpoint{2.451293in}{2.318661in}}{\pgfqpoint{2.454565in}{2.326561in}}{\pgfqpoint{2.454565in}{2.334798in}}%
\pgfpathcurveto{\pgfqpoint{2.454565in}{2.343034in}}{\pgfqpoint{2.451293in}{2.350934in}}{\pgfqpoint{2.445469in}{2.356758in}}%
\pgfpathcurveto{\pgfqpoint{2.439645in}{2.362582in}}{\pgfqpoint{2.431745in}{2.365854in}}{\pgfqpoint{2.423509in}{2.365854in}}%
\pgfpathcurveto{\pgfqpoint{2.415273in}{2.365854in}}{\pgfqpoint{2.407372in}{2.362582in}}{\pgfqpoint{2.401549in}{2.356758in}}%
\pgfpathcurveto{\pgfqpoint{2.395725in}{2.350934in}}{\pgfqpoint{2.392452in}{2.343034in}}{\pgfqpoint{2.392452in}{2.334798in}}%
\pgfpathcurveto{\pgfqpoint{2.392452in}{2.326561in}}{\pgfqpoint{2.395725in}{2.318661in}}{\pgfqpoint{2.401549in}{2.312837in}}%
\pgfpathcurveto{\pgfqpoint{2.407372in}{2.307014in}}{\pgfqpoint{2.415273in}{2.303741in}}{\pgfqpoint{2.423509in}{2.303741in}}%
\pgfpathclose%
\pgfusepath{stroke,fill}%
\end{pgfscope}%
\begin{pgfscope}%
\pgfpathrectangle{\pgfqpoint{0.100000in}{0.212622in}}{\pgfqpoint{3.696000in}{3.696000in}}%
\pgfusepath{clip}%
\pgfsetbuttcap%
\pgfsetroundjoin%
\definecolor{currentfill}{rgb}{0.121569,0.466667,0.705882}%
\pgfsetfillcolor{currentfill}%
\pgfsetfillopacity{0.438373}%
\pgfsetlinewidth{1.003750pt}%
\definecolor{currentstroke}{rgb}{0.121569,0.466667,0.705882}%
\pgfsetstrokecolor{currentstroke}%
\pgfsetstrokeopacity{0.438373}%
\pgfsetdash{}{0pt}%
\pgfpathmoveto{\pgfqpoint{2.438650in}{2.299418in}}%
\pgfpathcurveto{\pgfqpoint{2.446886in}{2.299418in}}{\pgfqpoint{2.454786in}{2.302690in}}{\pgfqpoint{2.460610in}{2.308514in}}%
\pgfpathcurveto{\pgfqpoint{2.466434in}{2.314338in}}{\pgfqpoint{2.469707in}{2.322238in}}{\pgfqpoint{2.469707in}{2.330474in}}%
\pgfpathcurveto{\pgfqpoint{2.469707in}{2.338711in}}{\pgfqpoint{2.466434in}{2.346611in}}{\pgfqpoint{2.460610in}{2.352435in}}%
\pgfpathcurveto{\pgfqpoint{2.454786in}{2.358259in}}{\pgfqpoint{2.446886in}{2.361531in}}{\pgfqpoint{2.438650in}{2.361531in}}%
\pgfpathcurveto{\pgfqpoint{2.430414in}{2.361531in}}{\pgfqpoint{2.422514in}{2.358259in}}{\pgfqpoint{2.416690in}{2.352435in}}%
\pgfpathcurveto{\pgfqpoint{2.410866in}{2.346611in}}{\pgfqpoint{2.407594in}{2.338711in}}{\pgfqpoint{2.407594in}{2.330474in}}%
\pgfpathcurveto{\pgfqpoint{2.407594in}{2.322238in}}{\pgfqpoint{2.410866in}{2.314338in}}{\pgfqpoint{2.416690in}{2.308514in}}%
\pgfpathcurveto{\pgfqpoint{2.422514in}{2.302690in}}{\pgfqpoint{2.430414in}{2.299418in}}{\pgfqpoint{2.438650in}{2.299418in}}%
\pgfpathclose%
\pgfusepath{stroke,fill}%
\end{pgfscope}%
\begin{pgfscope}%
\pgfpathrectangle{\pgfqpoint{0.100000in}{0.212622in}}{\pgfqpoint{3.696000in}{3.696000in}}%
\pgfusepath{clip}%
\pgfsetbuttcap%
\pgfsetroundjoin%
\definecolor{currentfill}{rgb}{0.121569,0.466667,0.705882}%
\pgfsetfillcolor{currentfill}%
\pgfsetfillopacity{0.442987}%
\pgfsetlinewidth{1.003750pt}%
\definecolor{currentstroke}{rgb}{0.121569,0.466667,0.705882}%
\pgfsetstrokecolor{currentstroke}%
\pgfsetstrokeopacity{0.442987}%
\pgfsetdash{}{0pt}%
\pgfpathmoveto{\pgfqpoint{2.463517in}{2.301719in}}%
\pgfpathcurveto{\pgfqpoint{2.471753in}{2.301719in}}{\pgfqpoint{2.479653in}{2.304992in}}{\pgfqpoint{2.485477in}{2.310816in}}%
\pgfpathcurveto{\pgfqpoint{2.491301in}{2.316640in}}{\pgfqpoint{2.494573in}{2.324540in}}{\pgfqpoint{2.494573in}{2.332776in}}%
\pgfpathcurveto{\pgfqpoint{2.494573in}{2.341012in}}{\pgfqpoint{2.491301in}{2.348912in}}{\pgfqpoint{2.485477in}{2.354736in}}%
\pgfpathcurveto{\pgfqpoint{2.479653in}{2.360560in}}{\pgfqpoint{2.471753in}{2.363832in}}{\pgfqpoint{2.463517in}{2.363832in}}%
\pgfpathcurveto{\pgfqpoint{2.455281in}{2.363832in}}{\pgfqpoint{2.447381in}{2.360560in}}{\pgfqpoint{2.441557in}{2.354736in}}%
\pgfpathcurveto{\pgfqpoint{2.435733in}{2.348912in}}{\pgfqpoint{2.432460in}{2.341012in}}{\pgfqpoint{2.432460in}{2.332776in}}%
\pgfpathcurveto{\pgfqpoint{2.432460in}{2.324540in}}{\pgfqpoint{2.435733in}{2.316640in}}{\pgfqpoint{2.441557in}{2.310816in}}%
\pgfpathcurveto{\pgfqpoint{2.447381in}{2.304992in}}{\pgfqpoint{2.455281in}{2.301719in}}{\pgfqpoint{2.463517in}{2.301719in}}%
\pgfpathclose%
\pgfusepath{stroke,fill}%
\end{pgfscope}%
\begin{pgfscope}%
\pgfpathrectangle{\pgfqpoint{0.100000in}{0.212622in}}{\pgfqpoint{3.696000in}{3.696000in}}%
\pgfusepath{clip}%
\pgfsetbuttcap%
\pgfsetroundjoin%
\definecolor{currentfill}{rgb}{0.121569,0.466667,0.705882}%
\pgfsetfillcolor{currentfill}%
\pgfsetfillopacity{0.444487}%
\pgfsetlinewidth{1.003750pt}%
\definecolor{currentstroke}{rgb}{0.121569,0.466667,0.705882}%
\pgfsetstrokecolor{currentstroke}%
\pgfsetstrokeopacity{0.444487}%
\pgfsetdash{}{0pt}%
\pgfpathmoveto{\pgfqpoint{1.283087in}{1.853200in}}%
\pgfpathcurveto{\pgfqpoint{1.291324in}{1.853200in}}{\pgfqpoint{1.299224in}{1.856472in}}{\pgfqpoint{1.305048in}{1.862296in}}%
\pgfpathcurveto{\pgfqpoint{1.310872in}{1.868120in}}{\pgfqpoint{1.314144in}{1.876020in}}{\pgfqpoint{1.314144in}{1.884256in}}%
\pgfpathcurveto{\pgfqpoint{1.314144in}{1.892493in}}{\pgfqpoint{1.310872in}{1.900393in}}{\pgfqpoint{1.305048in}{1.906216in}}%
\pgfpathcurveto{\pgfqpoint{1.299224in}{1.912040in}}{\pgfqpoint{1.291324in}{1.915313in}}{\pgfqpoint{1.283087in}{1.915313in}}%
\pgfpathcurveto{\pgfqpoint{1.274851in}{1.915313in}}{\pgfqpoint{1.266951in}{1.912040in}}{\pgfqpoint{1.261127in}{1.906216in}}%
\pgfpathcurveto{\pgfqpoint{1.255303in}{1.900393in}}{\pgfqpoint{1.252031in}{1.892493in}}{\pgfqpoint{1.252031in}{1.884256in}}%
\pgfpathcurveto{\pgfqpoint{1.252031in}{1.876020in}}{\pgfqpoint{1.255303in}{1.868120in}}{\pgfqpoint{1.261127in}{1.862296in}}%
\pgfpathcurveto{\pgfqpoint{1.266951in}{1.856472in}}{\pgfqpoint{1.274851in}{1.853200in}}{\pgfqpoint{1.283087in}{1.853200in}}%
\pgfpathclose%
\pgfusepath{stroke,fill}%
\end{pgfscope}%
\begin{pgfscope}%
\pgfpathrectangle{\pgfqpoint{0.100000in}{0.212622in}}{\pgfqpoint{3.696000in}{3.696000in}}%
\pgfusepath{clip}%
\pgfsetbuttcap%
\pgfsetroundjoin%
\definecolor{currentfill}{rgb}{0.121569,0.466667,0.705882}%
\pgfsetfillcolor{currentfill}%
\pgfsetfillopacity{0.445508}%
\pgfsetlinewidth{1.003750pt}%
\definecolor{currentstroke}{rgb}{0.121569,0.466667,0.705882}%
\pgfsetstrokecolor{currentstroke}%
\pgfsetstrokeopacity{0.445508}%
\pgfsetdash{}{0pt}%
\pgfpathmoveto{\pgfqpoint{2.476034in}{2.299393in}}%
\pgfpathcurveto{\pgfqpoint{2.484271in}{2.299393in}}{\pgfqpoint{2.492171in}{2.302665in}}{\pgfqpoint{2.497995in}{2.308489in}}%
\pgfpathcurveto{\pgfqpoint{2.503819in}{2.314313in}}{\pgfqpoint{2.507091in}{2.322213in}}{\pgfqpoint{2.507091in}{2.330449in}}%
\pgfpathcurveto{\pgfqpoint{2.507091in}{2.338686in}}{\pgfqpoint{2.503819in}{2.346586in}}{\pgfqpoint{2.497995in}{2.352410in}}%
\pgfpathcurveto{\pgfqpoint{2.492171in}{2.358234in}}{\pgfqpoint{2.484271in}{2.361506in}}{\pgfqpoint{2.476034in}{2.361506in}}%
\pgfpathcurveto{\pgfqpoint{2.467798in}{2.361506in}}{\pgfqpoint{2.459898in}{2.358234in}}{\pgfqpoint{2.454074in}{2.352410in}}%
\pgfpathcurveto{\pgfqpoint{2.448250in}{2.346586in}}{\pgfqpoint{2.444978in}{2.338686in}}{\pgfqpoint{2.444978in}{2.330449in}}%
\pgfpathcurveto{\pgfqpoint{2.444978in}{2.322213in}}{\pgfqpoint{2.448250in}{2.314313in}}{\pgfqpoint{2.454074in}{2.308489in}}%
\pgfpathcurveto{\pgfqpoint{2.459898in}{2.302665in}}{\pgfqpoint{2.467798in}{2.299393in}}{\pgfqpoint{2.476034in}{2.299393in}}%
\pgfpathclose%
\pgfusepath{stroke,fill}%
\end{pgfscope}%
\begin{pgfscope}%
\pgfpathrectangle{\pgfqpoint{0.100000in}{0.212622in}}{\pgfqpoint{3.696000in}{3.696000in}}%
\pgfusepath{clip}%
\pgfsetbuttcap%
\pgfsetroundjoin%
\definecolor{currentfill}{rgb}{0.121569,0.466667,0.705882}%
\pgfsetfillcolor{currentfill}%
\pgfsetfillopacity{0.448953}%
\pgfsetlinewidth{1.003750pt}%
\definecolor{currentstroke}{rgb}{0.121569,0.466667,0.705882}%
\pgfsetstrokecolor{currentstroke}%
\pgfsetstrokeopacity{0.448953}%
\pgfsetdash{}{0pt}%
\pgfpathmoveto{\pgfqpoint{2.493633in}{2.301663in}}%
\pgfpathcurveto{\pgfqpoint{2.501869in}{2.301663in}}{\pgfqpoint{2.509769in}{2.304936in}}{\pgfqpoint{2.515593in}{2.310760in}}%
\pgfpathcurveto{\pgfqpoint{2.521417in}{2.316584in}}{\pgfqpoint{2.524689in}{2.324484in}}{\pgfqpoint{2.524689in}{2.332720in}}%
\pgfpathcurveto{\pgfqpoint{2.524689in}{2.340956in}}{\pgfqpoint{2.521417in}{2.348856in}}{\pgfqpoint{2.515593in}{2.354680in}}%
\pgfpathcurveto{\pgfqpoint{2.509769in}{2.360504in}}{\pgfqpoint{2.501869in}{2.363776in}}{\pgfqpoint{2.493633in}{2.363776in}}%
\pgfpathcurveto{\pgfqpoint{2.485397in}{2.363776in}}{\pgfqpoint{2.477496in}{2.360504in}}{\pgfqpoint{2.471673in}{2.354680in}}%
\pgfpathcurveto{\pgfqpoint{2.465849in}{2.348856in}}{\pgfqpoint{2.462576in}{2.340956in}}{\pgfqpoint{2.462576in}{2.332720in}}%
\pgfpathcurveto{\pgfqpoint{2.462576in}{2.324484in}}{\pgfqpoint{2.465849in}{2.316584in}}{\pgfqpoint{2.471673in}{2.310760in}}%
\pgfpathcurveto{\pgfqpoint{2.477496in}{2.304936in}}{\pgfqpoint{2.485397in}{2.301663in}}{\pgfqpoint{2.493633in}{2.301663in}}%
\pgfpathclose%
\pgfusepath{stroke,fill}%
\end{pgfscope}%
\begin{pgfscope}%
\pgfpathrectangle{\pgfqpoint{0.100000in}{0.212622in}}{\pgfqpoint{3.696000in}{3.696000in}}%
\pgfusepath{clip}%
\pgfsetbuttcap%
\pgfsetroundjoin%
\definecolor{currentfill}{rgb}{0.121569,0.466667,0.705882}%
\pgfsetfillcolor{currentfill}%
\pgfsetfillopacity{0.452607}%
\pgfsetlinewidth{1.003750pt}%
\definecolor{currentstroke}{rgb}{0.121569,0.466667,0.705882}%
\pgfsetstrokecolor{currentstroke}%
\pgfsetstrokeopacity{0.452607}%
\pgfsetdash{}{0pt}%
\pgfpathmoveto{\pgfqpoint{2.509696in}{2.293036in}}%
\pgfpathcurveto{\pgfqpoint{2.517932in}{2.293036in}}{\pgfqpoint{2.525832in}{2.296308in}}{\pgfqpoint{2.531656in}{2.302132in}}%
\pgfpathcurveto{\pgfqpoint{2.537480in}{2.307956in}}{\pgfqpoint{2.540753in}{2.315856in}}{\pgfqpoint{2.540753in}{2.324093in}}%
\pgfpathcurveto{\pgfqpoint{2.540753in}{2.332329in}}{\pgfqpoint{2.537480in}{2.340229in}}{\pgfqpoint{2.531656in}{2.346053in}}%
\pgfpathcurveto{\pgfqpoint{2.525832in}{2.351877in}}{\pgfqpoint{2.517932in}{2.355149in}}{\pgfqpoint{2.509696in}{2.355149in}}%
\pgfpathcurveto{\pgfqpoint{2.501460in}{2.355149in}}{\pgfqpoint{2.493560in}{2.351877in}}{\pgfqpoint{2.487736in}{2.346053in}}%
\pgfpathcurveto{\pgfqpoint{2.481912in}{2.340229in}}{\pgfqpoint{2.478640in}{2.332329in}}{\pgfqpoint{2.478640in}{2.324093in}}%
\pgfpathcurveto{\pgfqpoint{2.478640in}{2.315856in}}{\pgfqpoint{2.481912in}{2.307956in}}{\pgfqpoint{2.487736in}{2.302132in}}%
\pgfpathcurveto{\pgfqpoint{2.493560in}{2.296308in}}{\pgfqpoint{2.501460in}{2.293036in}}{\pgfqpoint{2.509696in}{2.293036in}}%
\pgfpathclose%
\pgfusepath{stroke,fill}%
\end{pgfscope}%
\begin{pgfscope}%
\pgfpathrectangle{\pgfqpoint{0.100000in}{0.212622in}}{\pgfqpoint{3.696000in}{3.696000in}}%
\pgfusepath{clip}%
\pgfsetbuttcap%
\pgfsetroundjoin%
\definecolor{currentfill}{rgb}{0.121569,0.466667,0.705882}%
\pgfsetfillcolor{currentfill}%
\pgfsetfillopacity{0.453192}%
\pgfsetlinewidth{1.003750pt}%
\definecolor{currentstroke}{rgb}{0.121569,0.466667,0.705882}%
\pgfsetstrokecolor{currentstroke}%
\pgfsetstrokeopacity{0.453192}%
\pgfsetdash{}{0pt}%
\pgfpathmoveto{\pgfqpoint{1.259613in}{1.820214in}}%
\pgfpathcurveto{\pgfqpoint{1.267850in}{1.820214in}}{\pgfqpoint{1.275750in}{1.823486in}}{\pgfqpoint{1.281574in}{1.829310in}}%
\pgfpathcurveto{\pgfqpoint{1.287398in}{1.835134in}}{\pgfqpoint{1.290670in}{1.843034in}}{\pgfqpoint{1.290670in}{1.851270in}}%
\pgfpathcurveto{\pgfqpoint{1.290670in}{1.859506in}}{\pgfqpoint{1.287398in}{1.867406in}}{\pgfqpoint{1.281574in}{1.873230in}}%
\pgfpathcurveto{\pgfqpoint{1.275750in}{1.879054in}}{\pgfqpoint{1.267850in}{1.882327in}}{\pgfqpoint{1.259613in}{1.882327in}}%
\pgfpathcurveto{\pgfqpoint{1.251377in}{1.882327in}}{\pgfqpoint{1.243477in}{1.879054in}}{\pgfqpoint{1.237653in}{1.873230in}}%
\pgfpathcurveto{\pgfqpoint{1.231829in}{1.867406in}}{\pgfqpoint{1.228557in}{1.859506in}}{\pgfqpoint{1.228557in}{1.851270in}}%
\pgfpathcurveto{\pgfqpoint{1.228557in}{1.843034in}}{\pgfqpoint{1.231829in}{1.835134in}}{\pgfqpoint{1.237653in}{1.829310in}}%
\pgfpathcurveto{\pgfqpoint{1.243477in}{1.823486in}}{\pgfqpoint{1.251377in}{1.820214in}}{\pgfqpoint{1.259613in}{1.820214in}}%
\pgfpathclose%
\pgfusepath{stroke,fill}%
\end{pgfscope}%
\begin{pgfscope}%
\pgfpathrectangle{\pgfqpoint{0.100000in}{0.212622in}}{\pgfqpoint{3.696000in}{3.696000in}}%
\pgfusepath{clip}%
\pgfsetbuttcap%
\pgfsetroundjoin%
\definecolor{currentfill}{rgb}{0.121569,0.466667,0.705882}%
\pgfsetfillcolor{currentfill}%
\pgfsetfillopacity{0.457290}%
\pgfsetlinewidth{1.003750pt}%
\definecolor{currentstroke}{rgb}{0.121569,0.466667,0.705882}%
\pgfsetstrokecolor{currentstroke}%
\pgfsetstrokeopacity{0.457290}%
\pgfsetdash{}{0pt}%
\pgfpathmoveto{\pgfqpoint{2.533270in}{2.296315in}}%
\pgfpathcurveto{\pgfqpoint{2.541506in}{2.296315in}}{\pgfqpoint{2.549406in}{2.299587in}}{\pgfqpoint{2.555230in}{2.305411in}}%
\pgfpathcurveto{\pgfqpoint{2.561054in}{2.311235in}}{\pgfqpoint{2.564326in}{2.319135in}}{\pgfqpoint{2.564326in}{2.327371in}}%
\pgfpathcurveto{\pgfqpoint{2.564326in}{2.335608in}}{\pgfqpoint{2.561054in}{2.343508in}}{\pgfqpoint{2.555230in}{2.349332in}}%
\pgfpathcurveto{\pgfqpoint{2.549406in}{2.355156in}}{\pgfqpoint{2.541506in}{2.358428in}}{\pgfqpoint{2.533270in}{2.358428in}}%
\pgfpathcurveto{\pgfqpoint{2.525033in}{2.358428in}}{\pgfqpoint{2.517133in}{2.355156in}}{\pgfqpoint{2.511309in}{2.349332in}}%
\pgfpathcurveto{\pgfqpoint{2.505485in}{2.343508in}}{\pgfqpoint{2.502213in}{2.335608in}}{\pgfqpoint{2.502213in}{2.327371in}}%
\pgfpathcurveto{\pgfqpoint{2.502213in}{2.319135in}}{\pgfqpoint{2.505485in}{2.311235in}}{\pgfqpoint{2.511309in}{2.305411in}}%
\pgfpathcurveto{\pgfqpoint{2.517133in}{2.299587in}}{\pgfqpoint{2.525033in}{2.296315in}}{\pgfqpoint{2.533270in}{2.296315in}}%
\pgfpathclose%
\pgfusepath{stroke,fill}%
\end{pgfscope}%
\begin{pgfscope}%
\pgfpathrectangle{\pgfqpoint{0.100000in}{0.212622in}}{\pgfqpoint{3.696000in}{3.696000in}}%
\pgfusepath{clip}%
\pgfsetbuttcap%
\pgfsetroundjoin%
\definecolor{currentfill}{rgb}{0.121569,0.466667,0.705882}%
\pgfsetfillcolor{currentfill}%
\pgfsetfillopacity{0.460478}%
\pgfsetlinewidth{1.003750pt}%
\definecolor{currentstroke}{rgb}{0.121569,0.466667,0.705882}%
\pgfsetstrokecolor{currentstroke}%
\pgfsetstrokeopacity{0.460478}%
\pgfsetdash{}{0pt}%
\pgfpathmoveto{\pgfqpoint{1.239892in}{1.801513in}}%
\pgfpathcurveto{\pgfqpoint{1.248128in}{1.801513in}}{\pgfqpoint{1.256028in}{1.804785in}}{\pgfqpoint{1.261852in}{1.810609in}}%
\pgfpathcurveto{\pgfqpoint{1.267676in}{1.816433in}}{\pgfqpoint{1.270948in}{1.824333in}}{\pgfqpoint{1.270948in}{1.832569in}}%
\pgfpathcurveto{\pgfqpoint{1.270948in}{1.840806in}}{\pgfqpoint{1.267676in}{1.848706in}}{\pgfqpoint{1.261852in}{1.854530in}}%
\pgfpathcurveto{\pgfqpoint{1.256028in}{1.860354in}}{\pgfqpoint{1.248128in}{1.863626in}}{\pgfqpoint{1.239892in}{1.863626in}}%
\pgfpathcurveto{\pgfqpoint{1.231655in}{1.863626in}}{\pgfqpoint{1.223755in}{1.860354in}}{\pgfqpoint{1.217932in}{1.854530in}}%
\pgfpathcurveto{\pgfqpoint{1.212108in}{1.848706in}}{\pgfqpoint{1.208835in}{1.840806in}}{\pgfqpoint{1.208835in}{1.832569in}}%
\pgfpathcurveto{\pgfqpoint{1.208835in}{1.824333in}}{\pgfqpoint{1.212108in}{1.816433in}}{\pgfqpoint{1.217932in}{1.810609in}}%
\pgfpathcurveto{\pgfqpoint{1.223755in}{1.804785in}}{\pgfqpoint{1.231655in}{1.801513in}}{\pgfqpoint{1.239892in}{1.801513in}}%
\pgfpathclose%
\pgfusepath{stroke,fill}%
\end{pgfscope}%
\begin{pgfscope}%
\pgfpathrectangle{\pgfqpoint{0.100000in}{0.212622in}}{\pgfqpoint{3.696000in}{3.696000in}}%
\pgfusepath{clip}%
\pgfsetbuttcap%
\pgfsetroundjoin%
\definecolor{currentfill}{rgb}{0.121569,0.466667,0.705882}%
\pgfsetfillcolor{currentfill}%
\pgfsetfillopacity{0.462852}%
\pgfsetlinewidth{1.003750pt}%
\definecolor{currentstroke}{rgb}{0.121569,0.466667,0.705882}%
\pgfsetstrokecolor{currentstroke}%
\pgfsetstrokeopacity{0.462852}%
\pgfsetdash{}{0pt}%
\pgfpathmoveto{\pgfqpoint{2.554501in}{2.290893in}}%
\pgfpathcurveto{\pgfqpoint{2.562737in}{2.290893in}}{\pgfqpoint{2.570637in}{2.294165in}}{\pgfqpoint{2.576461in}{2.299989in}}%
\pgfpathcurveto{\pgfqpoint{2.582285in}{2.305813in}}{\pgfqpoint{2.585558in}{2.313713in}}{\pgfqpoint{2.585558in}{2.321949in}}%
\pgfpathcurveto{\pgfqpoint{2.585558in}{2.330185in}}{\pgfqpoint{2.582285in}{2.338085in}}{\pgfqpoint{2.576461in}{2.343909in}}%
\pgfpathcurveto{\pgfqpoint{2.570637in}{2.349733in}}{\pgfqpoint{2.562737in}{2.353006in}}{\pgfqpoint{2.554501in}{2.353006in}}%
\pgfpathcurveto{\pgfqpoint{2.546265in}{2.353006in}}{\pgfqpoint{2.538365in}{2.349733in}}{\pgfqpoint{2.532541in}{2.343909in}}%
\pgfpathcurveto{\pgfqpoint{2.526717in}{2.338085in}}{\pgfqpoint{2.523445in}{2.330185in}}{\pgfqpoint{2.523445in}{2.321949in}}%
\pgfpathcurveto{\pgfqpoint{2.523445in}{2.313713in}}{\pgfqpoint{2.526717in}{2.305813in}}{\pgfqpoint{2.532541in}{2.299989in}}%
\pgfpathcurveto{\pgfqpoint{2.538365in}{2.294165in}}{\pgfqpoint{2.546265in}{2.290893in}}{\pgfqpoint{2.554501in}{2.290893in}}%
\pgfpathclose%
\pgfusepath{stroke,fill}%
\end{pgfscope}%
\begin{pgfscope}%
\pgfpathrectangle{\pgfqpoint{0.100000in}{0.212622in}}{\pgfqpoint{3.696000in}{3.696000in}}%
\pgfusepath{clip}%
\pgfsetbuttcap%
\pgfsetroundjoin%
\definecolor{currentfill}{rgb}{0.121569,0.466667,0.705882}%
\pgfsetfillcolor{currentfill}%
\pgfsetfillopacity{0.465942}%
\pgfsetlinewidth{1.003750pt}%
\definecolor{currentstroke}{rgb}{0.121569,0.466667,0.705882}%
\pgfsetstrokecolor{currentstroke}%
\pgfsetstrokeopacity{0.465942}%
\pgfsetdash{}{0pt}%
\pgfpathmoveto{\pgfqpoint{2.567954in}{2.292209in}}%
\pgfpathcurveto{\pgfqpoint{2.576191in}{2.292209in}}{\pgfqpoint{2.584091in}{2.295482in}}{\pgfqpoint{2.589914in}{2.301306in}}%
\pgfpathcurveto{\pgfqpoint{2.595738in}{2.307130in}}{\pgfqpoint{2.599011in}{2.315030in}}{\pgfqpoint{2.599011in}{2.323266in}}%
\pgfpathcurveto{\pgfqpoint{2.599011in}{2.331502in}}{\pgfqpoint{2.595738in}{2.339402in}}{\pgfqpoint{2.589914in}{2.345226in}}%
\pgfpathcurveto{\pgfqpoint{2.584091in}{2.351050in}}{\pgfqpoint{2.576191in}{2.354322in}}{\pgfqpoint{2.567954in}{2.354322in}}%
\pgfpathcurveto{\pgfqpoint{2.559718in}{2.354322in}}{\pgfqpoint{2.551818in}{2.351050in}}{\pgfqpoint{2.545994in}{2.345226in}}%
\pgfpathcurveto{\pgfqpoint{2.540170in}{2.339402in}}{\pgfqpoint{2.536898in}{2.331502in}}{\pgfqpoint{2.536898in}{2.323266in}}%
\pgfpathcurveto{\pgfqpoint{2.536898in}{2.315030in}}{\pgfqpoint{2.540170in}{2.307130in}}{\pgfqpoint{2.545994in}{2.301306in}}%
\pgfpathcurveto{\pgfqpoint{2.551818in}{2.295482in}}{\pgfqpoint{2.559718in}{2.292209in}}{\pgfqpoint{2.567954in}{2.292209in}}%
\pgfpathclose%
\pgfusepath{stroke,fill}%
\end{pgfscope}%
\begin{pgfscope}%
\pgfpathrectangle{\pgfqpoint{0.100000in}{0.212622in}}{\pgfqpoint{3.696000in}{3.696000in}}%
\pgfusepath{clip}%
\pgfsetbuttcap%
\pgfsetroundjoin%
\definecolor{currentfill}{rgb}{0.121569,0.466667,0.705882}%
\pgfsetfillcolor{currentfill}%
\pgfsetfillopacity{0.466082}%
\pgfsetlinewidth{1.003750pt}%
\definecolor{currentstroke}{rgb}{0.121569,0.466667,0.705882}%
\pgfsetstrokecolor{currentstroke}%
\pgfsetstrokeopacity{0.466082}%
\pgfsetdash{}{0pt}%
\pgfpathmoveto{\pgfqpoint{1.231141in}{1.784511in}}%
\pgfpathcurveto{\pgfqpoint{1.239377in}{1.784511in}}{\pgfqpoint{1.247277in}{1.787783in}}{\pgfqpoint{1.253101in}{1.793607in}}%
\pgfpathcurveto{\pgfqpoint{1.258925in}{1.799431in}}{\pgfqpoint{1.262197in}{1.807331in}}{\pgfqpoint{1.262197in}{1.815567in}}%
\pgfpathcurveto{\pgfqpoint{1.262197in}{1.823803in}}{\pgfqpoint{1.258925in}{1.831703in}}{\pgfqpoint{1.253101in}{1.837527in}}%
\pgfpathcurveto{\pgfqpoint{1.247277in}{1.843351in}}{\pgfqpoint{1.239377in}{1.846624in}}{\pgfqpoint{1.231141in}{1.846624in}}%
\pgfpathcurveto{\pgfqpoint{1.222905in}{1.846624in}}{\pgfqpoint{1.215005in}{1.843351in}}{\pgfqpoint{1.209181in}{1.837527in}}%
\pgfpathcurveto{\pgfqpoint{1.203357in}{1.831703in}}{\pgfqpoint{1.200084in}{1.823803in}}{\pgfqpoint{1.200084in}{1.815567in}}%
\pgfpathcurveto{\pgfqpoint{1.200084in}{1.807331in}}{\pgfqpoint{1.203357in}{1.799431in}}{\pgfqpoint{1.209181in}{1.793607in}}%
\pgfpathcurveto{\pgfqpoint{1.215005in}{1.787783in}}{\pgfqpoint{1.222905in}{1.784511in}}{\pgfqpoint{1.231141in}{1.784511in}}%
\pgfpathclose%
\pgfusepath{stroke,fill}%
\end{pgfscope}%
\begin{pgfscope}%
\pgfpathrectangle{\pgfqpoint{0.100000in}{0.212622in}}{\pgfqpoint{3.696000in}{3.696000in}}%
\pgfusepath{clip}%
\pgfsetbuttcap%
\pgfsetroundjoin%
\definecolor{currentfill}{rgb}{0.121569,0.466667,0.705882}%
\pgfsetfillcolor{currentfill}%
\pgfsetfillopacity{0.469179}%
\pgfsetlinewidth{1.003750pt}%
\definecolor{currentstroke}{rgb}{0.121569,0.466667,0.705882}%
\pgfsetstrokecolor{currentstroke}%
\pgfsetstrokeopacity{0.469179}%
\pgfsetdash{}{0pt}%
\pgfpathmoveto{\pgfqpoint{1.221217in}{1.775670in}}%
\pgfpathcurveto{\pgfqpoint{1.229454in}{1.775670in}}{\pgfqpoint{1.237354in}{1.778942in}}{\pgfqpoint{1.243178in}{1.784766in}}%
\pgfpathcurveto{\pgfqpoint{1.249002in}{1.790590in}}{\pgfqpoint{1.252274in}{1.798490in}}{\pgfqpoint{1.252274in}{1.806727in}}%
\pgfpathcurveto{\pgfqpoint{1.252274in}{1.814963in}}{\pgfqpoint{1.249002in}{1.822863in}}{\pgfqpoint{1.243178in}{1.828687in}}%
\pgfpathcurveto{\pgfqpoint{1.237354in}{1.834511in}}{\pgfqpoint{1.229454in}{1.837783in}}{\pgfqpoint{1.221217in}{1.837783in}}%
\pgfpathcurveto{\pgfqpoint{1.212981in}{1.837783in}}{\pgfqpoint{1.205081in}{1.834511in}}{\pgfqpoint{1.199257in}{1.828687in}}%
\pgfpathcurveto{\pgfqpoint{1.193433in}{1.822863in}}{\pgfqpoint{1.190161in}{1.814963in}}{\pgfqpoint{1.190161in}{1.806727in}}%
\pgfpathcurveto{\pgfqpoint{1.190161in}{1.798490in}}{\pgfqpoint{1.193433in}{1.790590in}}{\pgfqpoint{1.199257in}{1.784766in}}%
\pgfpathcurveto{\pgfqpoint{1.205081in}{1.778942in}}{\pgfqpoint{1.212981in}{1.775670in}}{\pgfqpoint{1.221217in}{1.775670in}}%
\pgfpathclose%
\pgfusepath{stroke,fill}%
\end{pgfscope}%
\begin{pgfscope}%
\pgfpathrectangle{\pgfqpoint{0.100000in}{0.212622in}}{\pgfqpoint{3.696000in}{3.696000in}}%
\pgfusepath{clip}%
\pgfsetbuttcap%
\pgfsetroundjoin%
\definecolor{currentfill}{rgb}{0.121569,0.466667,0.705882}%
\pgfsetfillcolor{currentfill}%
\pgfsetfillopacity{0.469275}%
\pgfsetlinewidth{1.003750pt}%
\definecolor{currentstroke}{rgb}{0.121569,0.466667,0.705882}%
\pgfsetstrokecolor{currentstroke}%
\pgfsetstrokeopacity{0.469275}%
\pgfsetdash{}{0pt}%
\pgfpathmoveto{\pgfqpoint{2.583359in}{2.290372in}}%
\pgfpathcurveto{\pgfqpoint{2.591595in}{2.290372in}}{\pgfqpoint{2.599495in}{2.293644in}}{\pgfqpoint{2.605319in}{2.299468in}}%
\pgfpathcurveto{\pgfqpoint{2.611143in}{2.305292in}}{\pgfqpoint{2.614416in}{2.313192in}}{\pgfqpoint{2.614416in}{2.321428in}}%
\pgfpathcurveto{\pgfqpoint{2.614416in}{2.329664in}}{\pgfqpoint{2.611143in}{2.337564in}}{\pgfqpoint{2.605319in}{2.343388in}}%
\pgfpathcurveto{\pgfqpoint{2.599495in}{2.349212in}}{\pgfqpoint{2.591595in}{2.352485in}}{\pgfqpoint{2.583359in}{2.352485in}}%
\pgfpathcurveto{\pgfqpoint{2.575123in}{2.352485in}}{\pgfqpoint{2.567223in}{2.349212in}}{\pgfqpoint{2.561399in}{2.343388in}}%
\pgfpathcurveto{\pgfqpoint{2.555575in}{2.337564in}}{\pgfqpoint{2.552303in}{2.329664in}}{\pgfqpoint{2.552303in}{2.321428in}}%
\pgfpathcurveto{\pgfqpoint{2.552303in}{2.313192in}}{\pgfqpoint{2.555575in}{2.305292in}}{\pgfqpoint{2.561399in}{2.299468in}}%
\pgfpathcurveto{\pgfqpoint{2.567223in}{2.293644in}}{\pgfqpoint{2.575123in}{2.290372in}}{\pgfqpoint{2.583359in}{2.290372in}}%
\pgfpathclose%
\pgfusepath{stroke,fill}%
\end{pgfscope}%
\begin{pgfscope}%
\pgfpathrectangle{\pgfqpoint{0.100000in}{0.212622in}}{\pgfqpoint{3.696000in}{3.696000in}}%
\pgfusepath{clip}%
\pgfsetbuttcap%
\pgfsetroundjoin%
\definecolor{currentfill}{rgb}{0.121569,0.466667,0.705882}%
\pgfsetfillcolor{currentfill}%
\pgfsetfillopacity{0.470573}%
\pgfsetlinewidth{1.003750pt}%
\definecolor{currentstroke}{rgb}{0.121569,0.466667,0.705882}%
\pgfsetstrokecolor{currentstroke}%
\pgfsetstrokeopacity{0.470573}%
\pgfsetdash{}{0pt}%
\pgfpathmoveto{\pgfqpoint{1.217369in}{1.769803in}}%
\pgfpathcurveto{\pgfqpoint{1.225605in}{1.769803in}}{\pgfqpoint{1.233505in}{1.773076in}}{\pgfqpoint{1.239329in}{1.778900in}}%
\pgfpathcurveto{\pgfqpoint{1.245153in}{1.784724in}}{\pgfqpoint{1.248426in}{1.792624in}}{\pgfqpoint{1.248426in}{1.800860in}}%
\pgfpathcurveto{\pgfqpoint{1.248426in}{1.809096in}}{\pgfqpoint{1.245153in}{1.816996in}}{\pgfqpoint{1.239329in}{1.822820in}}%
\pgfpathcurveto{\pgfqpoint{1.233505in}{1.828644in}}{\pgfqpoint{1.225605in}{1.831916in}}{\pgfqpoint{1.217369in}{1.831916in}}%
\pgfpathcurveto{\pgfqpoint{1.209133in}{1.831916in}}{\pgfqpoint{1.201233in}{1.828644in}}{\pgfqpoint{1.195409in}{1.822820in}}%
\pgfpathcurveto{\pgfqpoint{1.189585in}{1.816996in}}{\pgfqpoint{1.186313in}{1.809096in}}{\pgfqpoint{1.186313in}{1.800860in}}%
\pgfpathcurveto{\pgfqpoint{1.186313in}{1.792624in}}{\pgfqpoint{1.189585in}{1.784724in}}{\pgfqpoint{1.195409in}{1.778900in}}%
\pgfpathcurveto{\pgfqpoint{1.201233in}{1.773076in}}{\pgfqpoint{1.209133in}{1.769803in}}{\pgfqpoint{1.217369in}{1.769803in}}%
\pgfpathclose%
\pgfusepath{stroke,fill}%
\end{pgfscope}%
\begin{pgfscope}%
\pgfpathrectangle{\pgfqpoint{0.100000in}{0.212622in}}{\pgfqpoint{3.696000in}{3.696000in}}%
\pgfusepath{clip}%
\pgfsetbuttcap%
\pgfsetroundjoin%
\definecolor{currentfill}{rgb}{0.121569,0.466667,0.705882}%
\pgfsetfillcolor{currentfill}%
\pgfsetfillopacity{0.470966}%
\pgfsetlinewidth{1.003750pt}%
\definecolor{currentstroke}{rgb}{0.121569,0.466667,0.705882}%
\pgfsetstrokecolor{currentstroke}%
\pgfsetstrokeopacity{0.470966}%
\pgfsetdash{}{0pt}%
\pgfpathmoveto{\pgfqpoint{1.216007in}{1.768716in}}%
\pgfpathcurveto{\pgfqpoint{1.224243in}{1.768716in}}{\pgfqpoint{1.232143in}{1.771988in}}{\pgfqpoint{1.237967in}{1.777812in}}%
\pgfpathcurveto{\pgfqpoint{1.243791in}{1.783636in}}{\pgfqpoint{1.247063in}{1.791536in}}{\pgfqpoint{1.247063in}{1.799773in}}%
\pgfpathcurveto{\pgfqpoint{1.247063in}{1.808009in}}{\pgfqpoint{1.243791in}{1.815909in}}{\pgfqpoint{1.237967in}{1.821733in}}%
\pgfpathcurveto{\pgfqpoint{1.232143in}{1.827557in}}{\pgfqpoint{1.224243in}{1.830829in}}{\pgfqpoint{1.216007in}{1.830829in}}%
\pgfpathcurveto{\pgfqpoint{1.207771in}{1.830829in}}{\pgfqpoint{1.199871in}{1.827557in}}{\pgfqpoint{1.194047in}{1.821733in}}%
\pgfpathcurveto{\pgfqpoint{1.188223in}{1.815909in}}{\pgfqpoint{1.184950in}{1.808009in}}{\pgfqpoint{1.184950in}{1.799773in}}%
\pgfpathcurveto{\pgfqpoint{1.184950in}{1.791536in}}{\pgfqpoint{1.188223in}{1.783636in}}{\pgfqpoint{1.194047in}{1.777812in}}%
\pgfpathcurveto{\pgfqpoint{1.199871in}{1.771988in}}{\pgfqpoint{1.207771in}{1.768716in}}{\pgfqpoint{1.216007in}{1.768716in}}%
\pgfpathclose%
\pgfusepath{stroke,fill}%
\end{pgfscope}%
\begin{pgfscope}%
\pgfpathrectangle{\pgfqpoint{0.100000in}{0.212622in}}{\pgfqpoint{3.696000in}{3.696000in}}%
\pgfusepath{clip}%
\pgfsetbuttcap%
\pgfsetroundjoin%
\definecolor{currentfill}{rgb}{0.121569,0.466667,0.705882}%
\pgfsetfillcolor{currentfill}%
\pgfsetfillopacity{0.471675}%
\pgfsetlinewidth{1.003750pt}%
\definecolor{currentstroke}{rgb}{0.121569,0.466667,0.705882}%
\pgfsetstrokecolor{currentstroke}%
\pgfsetstrokeopacity{0.471675}%
\pgfsetdash{}{0pt}%
\pgfpathmoveto{\pgfqpoint{1.214570in}{1.765674in}}%
\pgfpathcurveto{\pgfqpoint{1.222807in}{1.765674in}}{\pgfqpoint{1.230707in}{1.768946in}}{\pgfqpoint{1.236531in}{1.774770in}}%
\pgfpathcurveto{\pgfqpoint{1.242355in}{1.780594in}}{\pgfqpoint{1.245627in}{1.788494in}}{\pgfqpoint{1.245627in}{1.796730in}}%
\pgfpathcurveto{\pgfqpoint{1.245627in}{1.804967in}}{\pgfqpoint{1.242355in}{1.812867in}}{\pgfqpoint{1.236531in}{1.818691in}}%
\pgfpathcurveto{\pgfqpoint{1.230707in}{1.824515in}}{\pgfqpoint{1.222807in}{1.827787in}}{\pgfqpoint{1.214570in}{1.827787in}}%
\pgfpathcurveto{\pgfqpoint{1.206334in}{1.827787in}}{\pgfqpoint{1.198434in}{1.824515in}}{\pgfqpoint{1.192610in}{1.818691in}}%
\pgfpathcurveto{\pgfqpoint{1.186786in}{1.812867in}}{\pgfqpoint{1.183514in}{1.804967in}}{\pgfqpoint{1.183514in}{1.796730in}}%
\pgfpathcurveto{\pgfqpoint{1.183514in}{1.788494in}}{\pgfqpoint{1.186786in}{1.780594in}}{\pgfqpoint{1.192610in}{1.774770in}}%
\pgfpathcurveto{\pgfqpoint{1.198434in}{1.768946in}}{\pgfqpoint{1.206334in}{1.765674in}}{\pgfqpoint{1.214570in}{1.765674in}}%
\pgfpathclose%
\pgfusepath{stroke,fill}%
\end{pgfscope}%
\begin{pgfscope}%
\pgfpathrectangle{\pgfqpoint{0.100000in}{0.212622in}}{\pgfqpoint{3.696000in}{3.696000in}}%
\pgfusepath{clip}%
\pgfsetbuttcap%
\pgfsetroundjoin%
\definecolor{currentfill}{rgb}{0.121569,0.466667,0.705882}%
\pgfsetfillcolor{currentfill}%
\pgfsetfillopacity{0.473093}%
\pgfsetlinewidth{1.003750pt}%
\definecolor{currentstroke}{rgb}{0.121569,0.466667,0.705882}%
\pgfsetstrokecolor{currentstroke}%
\pgfsetstrokeopacity{0.473093}%
\pgfsetdash{}{0pt}%
\pgfpathmoveto{\pgfqpoint{1.211774in}{1.761100in}}%
\pgfpathcurveto{\pgfqpoint{1.220010in}{1.761100in}}{\pgfqpoint{1.227910in}{1.764373in}}{\pgfqpoint{1.233734in}{1.770197in}}%
\pgfpathcurveto{\pgfqpoint{1.239558in}{1.776021in}}{\pgfqpoint{1.242831in}{1.783921in}}{\pgfqpoint{1.242831in}{1.792157in}}%
\pgfpathcurveto{\pgfqpoint{1.242831in}{1.800393in}}{\pgfqpoint{1.239558in}{1.808293in}}{\pgfqpoint{1.233734in}{1.814117in}}%
\pgfpathcurveto{\pgfqpoint{1.227910in}{1.819941in}}{\pgfqpoint{1.220010in}{1.823213in}}{\pgfqpoint{1.211774in}{1.823213in}}%
\pgfpathcurveto{\pgfqpoint{1.203538in}{1.823213in}}{\pgfqpoint{1.195638in}{1.819941in}}{\pgfqpoint{1.189814in}{1.814117in}}%
\pgfpathcurveto{\pgfqpoint{1.183990in}{1.808293in}}{\pgfqpoint{1.180718in}{1.800393in}}{\pgfqpoint{1.180718in}{1.792157in}}%
\pgfpathcurveto{\pgfqpoint{1.180718in}{1.783921in}}{\pgfqpoint{1.183990in}{1.776021in}}{\pgfqpoint{1.189814in}{1.770197in}}%
\pgfpathcurveto{\pgfqpoint{1.195638in}{1.764373in}}{\pgfqpoint{1.203538in}{1.761100in}}{\pgfqpoint{1.211774in}{1.761100in}}%
\pgfpathclose%
\pgfusepath{stroke,fill}%
\end{pgfscope}%
\begin{pgfscope}%
\pgfpathrectangle{\pgfqpoint{0.100000in}{0.212622in}}{\pgfqpoint{3.696000in}{3.696000in}}%
\pgfusepath{clip}%
\pgfsetbuttcap%
\pgfsetroundjoin%
\definecolor{currentfill}{rgb}{0.121569,0.466667,0.705882}%
\pgfsetfillcolor{currentfill}%
\pgfsetfillopacity{0.473663}%
\pgfsetlinewidth{1.003750pt}%
\definecolor{currentstroke}{rgb}{0.121569,0.466667,0.705882}%
\pgfsetstrokecolor{currentstroke}%
\pgfsetstrokeopacity{0.473663}%
\pgfsetdash{}{0pt}%
\pgfpathmoveto{\pgfqpoint{2.602977in}{2.291551in}}%
\pgfpathcurveto{\pgfqpoint{2.611213in}{2.291551in}}{\pgfqpoint{2.619113in}{2.294823in}}{\pgfqpoint{2.624937in}{2.300647in}}%
\pgfpathcurveto{\pgfqpoint{2.630761in}{2.306471in}}{\pgfqpoint{2.634034in}{2.314371in}}{\pgfqpoint{2.634034in}{2.322608in}}%
\pgfpathcurveto{\pgfqpoint{2.634034in}{2.330844in}}{\pgfqpoint{2.630761in}{2.338744in}}{\pgfqpoint{2.624937in}{2.344568in}}%
\pgfpathcurveto{\pgfqpoint{2.619113in}{2.350392in}}{\pgfqpoint{2.611213in}{2.353664in}}{\pgfqpoint{2.602977in}{2.353664in}}%
\pgfpathcurveto{\pgfqpoint{2.594741in}{2.353664in}}{\pgfqpoint{2.586841in}{2.350392in}}{\pgfqpoint{2.581017in}{2.344568in}}%
\pgfpathcurveto{\pgfqpoint{2.575193in}{2.338744in}}{\pgfqpoint{2.571921in}{2.330844in}}{\pgfqpoint{2.571921in}{2.322608in}}%
\pgfpathcurveto{\pgfqpoint{2.571921in}{2.314371in}}{\pgfqpoint{2.575193in}{2.306471in}}{\pgfqpoint{2.581017in}{2.300647in}}%
\pgfpathcurveto{\pgfqpoint{2.586841in}{2.294823in}}{\pgfqpoint{2.594741in}{2.291551in}}{\pgfqpoint{2.602977in}{2.291551in}}%
\pgfpathclose%
\pgfusepath{stroke,fill}%
\end{pgfscope}%
\begin{pgfscope}%
\pgfpathrectangle{\pgfqpoint{0.100000in}{0.212622in}}{\pgfqpoint{3.696000in}{3.696000in}}%
\pgfusepath{clip}%
\pgfsetbuttcap%
\pgfsetroundjoin%
\definecolor{currentfill}{rgb}{0.121569,0.466667,0.705882}%
\pgfsetfillcolor{currentfill}%
\pgfsetfillopacity{0.475466}%
\pgfsetlinewidth{1.003750pt}%
\definecolor{currentstroke}{rgb}{0.121569,0.466667,0.705882}%
\pgfsetstrokecolor{currentstroke}%
\pgfsetstrokeopacity{0.475466}%
\pgfsetdash{}{0pt}%
\pgfpathmoveto{\pgfqpoint{1.205755in}{1.752162in}}%
\pgfpathcurveto{\pgfqpoint{1.213992in}{1.752162in}}{\pgfqpoint{1.221892in}{1.755434in}}{\pgfqpoint{1.227716in}{1.761258in}}%
\pgfpathcurveto{\pgfqpoint{1.233540in}{1.767082in}}{\pgfqpoint{1.236812in}{1.774982in}}{\pgfqpoint{1.236812in}{1.783218in}}%
\pgfpathcurveto{\pgfqpoint{1.236812in}{1.791454in}}{\pgfqpoint{1.233540in}{1.799354in}}{\pgfqpoint{1.227716in}{1.805178in}}%
\pgfpathcurveto{\pgfqpoint{1.221892in}{1.811002in}}{\pgfqpoint{1.213992in}{1.814275in}}{\pgfqpoint{1.205755in}{1.814275in}}%
\pgfpathcurveto{\pgfqpoint{1.197519in}{1.814275in}}{\pgfqpoint{1.189619in}{1.811002in}}{\pgfqpoint{1.183795in}{1.805178in}}%
\pgfpathcurveto{\pgfqpoint{1.177971in}{1.799354in}}{\pgfqpoint{1.174699in}{1.791454in}}{\pgfqpoint{1.174699in}{1.783218in}}%
\pgfpathcurveto{\pgfqpoint{1.174699in}{1.774982in}}{\pgfqpoint{1.177971in}{1.767082in}}{\pgfqpoint{1.183795in}{1.761258in}}%
\pgfpathcurveto{\pgfqpoint{1.189619in}{1.755434in}}{\pgfqpoint{1.197519in}{1.752162in}}{\pgfqpoint{1.205755in}{1.752162in}}%
\pgfpathclose%
\pgfusepath{stroke,fill}%
\end{pgfscope}%
\begin{pgfscope}%
\pgfpathrectangle{\pgfqpoint{0.100000in}{0.212622in}}{\pgfqpoint{3.696000in}{3.696000in}}%
\pgfusepath{clip}%
\pgfsetbuttcap%
\pgfsetroundjoin%
\definecolor{currentfill}{rgb}{0.121569,0.466667,0.705882}%
\pgfsetfillcolor{currentfill}%
\pgfsetfillopacity{0.475879}%
\pgfsetlinewidth{1.003750pt}%
\definecolor{currentstroke}{rgb}{0.121569,0.466667,0.705882}%
\pgfsetstrokecolor{currentstroke}%
\pgfsetstrokeopacity{0.475879}%
\pgfsetdash{}{0pt}%
\pgfpathmoveto{\pgfqpoint{2.613995in}{2.291497in}}%
\pgfpathcurveto{\pgfqpoint{2.622232in}{2.291497in}}{\pgfqpoint{2.630132in}{2.294770in}}{\pgfqpoint{2.635956in}{2.300593in}}%
\pgfpathcurveto{\pgfqpoint{2.641779in}{2.306417in}}{\pgfqpoint{2.645052in}{2.314317in}}{\pgfqpoint{2.645052in}{2.322554in}}%
\pgfpathcurveto{\pgfqpoint{2.645052in}{2.330790in}}{\pgfqpoint{2.641779in}{2.338690in}}{\pgfqpoint{2.635956in}{2.344514in}}%
\pgfpathcurveto{\pgfqpoint{2.630132in}{2.350338in}}{\pgfqpoint{2.622232in}{2.353610in}}{\pgfqpoint{2.613995in}{2.353610in}}%
\pgfpathcurveto{\pgfqpoint{2.605759in}{2.353610in}}{\pgfqpoint{2.597859in}{2.350338in}}{\pgfqpoint{2.592035in}{2.344514in}}%
\pgfpathcurveto{\pgfqpoint{2.586211in}{2.338690in}}{\pgfqpoint{2.582939in}{2.330790in}}{\pgfqpoint{2.582939in}{2.322554in}}%
\pgfpathcurveto{\pgfqpoint{2.582939in}{2.314317in}}{\pgfqpoint{2.586211in}{2.306417in}}{\pgfqpoint{2.592035in}{2.300593in}}%
\pgfpathcurveto{\pgfqpoint{2.597859in}{2.294770in}}{\pgfqpoint{2.605759in}{2.291497in}}{\pgfqpoint{2.613995in}{2.291497in}}%
\pgfpathclose%
\pgfusepath{stroke,fill}%
\end{pgfscope}%
\begin{pgfscope}%
\pgfpathrectangle{\pgfqpoint{0.100000in}{0.212622in}}{\pgfqpoint{3.696000in}{3.696000in}}%
\pgfusepath{clip}%
\pgfsetbuttcap%
\pgfsetroundjoin%
\definecolor{currentfill}{rgb}{0.121569,0.466667,0.705882}%
\pgfsetfillcolor{currentfill}%
\pgfsetfillopacity{0.476831}%
\pgfsetlinewidth{1.003750pt}%
\definecolor{currentstroke}{rgb}{0.121569,0.466667,0.705882}%
\pgfsetstrokecolor{currentstroke}%
\pgfsetstrokeopacity{0.476831}%
\pgfsetdash{}{0pt}%
\pgfpathmoveto{\pgfqpoint{2.620426in}{2.290927in}}%
\pgfpathcurveto{\pgfqpoint{2.628663in}{2.290927in}}{\pgfqpoint{2.636563in}{2.294199in}}{\pgfqpoint{2.642387in}{2.300023in}}%
\pgfpathcurveto{\pgfqpoint{2.648210in}{2.305847in}}{\pgfqpoint{2.651483in}{2.313747in}}{\pgfqpoint{2.651483in}{2.321983in}}%
\pgfpathcurveto{\pgfqpoint{2.651483in}{2.330219in}}{\pgfqpoint{2.648210in}{2.338119in}}{\pgfqpoint{2.642387in}{2.343943in}}%
\pgfpathcurveto{\pgfqpoint{2.636563in}{2.349767in}}{\pgfqpoint{2.628663in}{2.353040in}}{\pgfqpoint{2.620426in}{2.353040in}}%
\pgfpathcurveto{\pgfqpoint{2.612190in}{2.353040in}}{\pgfqpoint{2.604290in}{2.349767in}}{\pgfqpoint{2.598466in}{2.343943in}}%
\pgfpathcurveto{\pgfqpoint{2.592642in}{2.338119in}}{\pgfqpoint{2.589370in}{2.330219in}}{\pgfqpoint{2.589370in}{2.321983in}}%
\pgfpathcurveto{\pgfqpoint{2.589370in}{2.313747in}}{\pgfqpoint{2.592642in}{2.305847in}}{\pgfqpoint{2.598466in}{2.300023in}}%
\pgfpathcurveto{\pgfqpoint{2.604290in}{2.294199in}}{\pgfqpoint{2.612190in}{2.290927in}}{\pgfqpoint{2.620426in}{2.290927in}}%
\pgfpathclose%
\pgfusepath{stroke,fill}%
\end{pgfscope}%
\begin{pgfscope}%
\pgfpathrectangle{\pgfqpoint{0.100000in}{0.212622in}}{\pgfqpoint{3.696000in}{3.696000in}}%
\pgfusepath{clip}%
\pgfsetbuttcap%
\pgfsetroundjoin%
\definecolor{currentfill}{rgb}{0.121569,0.466667,0.705882}%
\pgfsetfillcolor{currentfill}%
\pgfsetfillopacity{0.478467}%
\pgfsetlinewidth{1.003750pt}%
\definecolor{currentstroke}{rgb}{0.121569,0.466667,0.705882}%
\pgfsetstrokecolor{currentstroke}%
\pgfsetstrokeopacity{0.478467}%
\pgfsetdash{}{0pt}%
\pgfpathmoveto{\pgfqpoint{2.628931in}{2.292683in}}%
\pgfpathcurveto{\pgfqpoint{2.637168in}{2.292683in}}{\pgfqpoint{2.645068in}{2.295956in}}{\pgfqpoint{2.650892in}{2.301780in}}%
\pgfpathcurveto{\pgfqpoint{2.656716in}{2.307604in}}{\pgfqpoint{2.659988in}{2.315504in}}{\pgfqpoint{2.659988in}{2.323740in}}%
\pgfpathcurveto{\pgfqpoint{2.659988in}{2.331976in}}{\pgfqpoint{2.656716in}{2.339876in}}{\pgfqpoint{2.650892in}{2.345700in}}%
\pgfpathcurveto{\pgfqpoint{2.645068in}{2.351524in}}{\pgfqpoint{2.637168in}{2.354796in}}{\pgfqpoint{2.628931in}{2.354796in}}%
\pgfpathcurveto{\pgfqpoint{2.620695in}{2.354796in}}{\pgfqpoint{2.612795in}{2.351524in}}{\pgfqpoint{2.606971in}{2.345700in}}%
\pgfpathcurveto{\pgfqpoint{2.601147in}{2.339876in}}{\pgfqpoint{2.597875in}{2.331976in}}{\pgfqpoint{2.597875in}{2.323740in}}%
\pgfpathcurveto{\pgfqpoint{2.597875in}{2.315504in}}{\pgfqpoint{2.601147in}{2.307604in}}{\pgfqpoint{2.606971in}{2.301780in}}%
\pgfpathcurveto{\pgfqpoint{2.612795in}{2.295956in}}{\pgfqpoint{2.620695in}{2.292683in}}{\pgfqpoint{2.628931in}{2.292683in}}%
\pgfpathclose%
\pgfusepath{stroke,fill}%
\end{pgfscope}%
\begin{pgfscope}%
\pgfpathrectangle{\pgfqpoint{0.100000in}{0.212622in}}{\pgfqpoint{3.696000in}{3.696000in}}%
\pgfusepath{clip}%
\pgfsetbuttcap%
\pgfsetroundjoin%
\definecolor{currentfill}{rgb}{0.121569,0.466667,0.705882}%
\pgfsetfillcolor{currentfill}%
\pgfsetfillopacity{0.479565}%
\pgfsetlinewidth{1.003750pt}%
\definecolor{currentstroke}{rgb}{0.121569,0.466667,0.705882}%
\pgfsetstrokecolor{currentstroke}%
\pgfsetstrokeopacity{0.479565}%
\pgfsetdash{}{0pt}%
\pgfpathmoveto{\pgfqpoint{1.194157in}{1.735027in}}%
\pgfpathcurveto{\pgfqpoint{1.202393in}{1.735027in}}{\pgfqpoint{1.210293in}{1.738299in}}{\pgfqpoint{1.216117in}{1.744123in}}%
\pgfpathcurveto{\pgfqpoint{1.221941in}{1.749947in}}{\pgfqpoint{1.225213in}{1.757847in}}{\pgfqpoint{1.225213in}{1.766083in}}%
\pgfpathcurveto{\pgfqpoint{1.225213in}{1.774320in}}{\pgfqpoint{1.221941in}{1.782220in}}{\pgfqpoint{1.216117in}{1.788044in}}%
\pgfpathcurveto{\pgfqpoint{1.210293in}{1.793868in}}{\pgfqpoint{1.202393in}{1.797140in}}{\pgfqpoint{1.194157in}{1.797140in}}%
\pgfpathcurveto{\pgfqpoint{1.185921in}{1.797140in}}{\pgfqpoint{1.178020in}{1.793868in}}{\pgfqpoint{1.172197in}{1.788044in}}%
\pgfpathcurveto{\pgfqpoint{1.166373in}{1.782220in}}{\pgfqpoint{1.163100in}{1.774320in}}{\pgfqpoint{1.163100in}{1.766083in}}%
\pgfpathcurveto{\pgfqpoint{1.163100in}{1.757847in}}{\pgfqpoint{1.166373in}{1.749947in}}{\pgfqpoint{1.172197in}{1.744123in}}%
\pgfpathcurveto{\pgfqpoint{1.178020in}{1.738299in}}{\pgfqpoint{1.185921in}{1.735027in}}{\pgfqpoint{1.194157in}{1.735027in}}%
\pgfpathclose%
\pgfusepath{stroke,fill}%
\end{pgfscope}%
\begin{pgfscope}%
\pgfpathrectangle{\pgfqpoint{0.100000in}{0.212622in}}{\pgfqpoint{3.696000in}{3.696000in}}%
\pgfusepath{clip}%
\pgfsetbuttcap%
\pgfsetroundjoin%
\definecolor{currentfill}{rgb}{0.121569,0.466667,0.705882}%
\pgfsetfillcolor{currentfill}%
\pgfsetfillopacity{0.480995}%
\pgfsetlinewidth{1.003750pt}%
\definecolor{currentstroke}{rgb}{0.121569,0.466667,0.705882}%
\pgfsetstrokecolor{currentstroke}%
\pgfsetstrokeopacity{0.480995}%
\pgfsetdash{}{0pt}%
\pgfpathmoveto{\pgfqpoint{2.643425in}{2.290241in}}%
\pgfpathcurveto{\pgfqpoint{2.651661in}{2.290241in}}{\pgfqpoint{2.659561in}{2.293513in}}{\pgfqpoint{2.665385in}{2.299337in}}%
\pgfpathcurveto{\pgfqpoint{2.671209in}{2.305161in}}{\pgfqpoint{2.674481in}{2.313061in}}{\pgfqpoint{2.674481in}{2.321297in}}%
\pgfpathcurveto{\pgfqpoint{2.674481in}{2.329534in}}{\pgfqpoint{2.671209in}{2.337434in}}{\pgfqpoint{2.665385in}{2.343258in}}%
\pgfpathcurveto{\pgfqpoint{2.659561in}{2.349082in}}{\pgfqpoint{2.651661in}{2.352354in}}{\pgfqpoint{2.643425in}{2.352354in}}%
\pgfpathcurveto{\pgfqpoint{2.635188in}{2.352354in}}{\pgfqpoint{2.627288in}{2.349082in}}{\pgfqpoint{2.621464in}{2.343258in}}%
\pgfpathcurveto{\pgfqpoint{2.615640in}{2.337434in}}{\pgfqpoint{2.612368in}{2.329534in}}{\pgfqpoint{2.612368in}{2.321297in}}%
\pgfpathcurveto{\pgfqpoint{2.612368in}{2.313061in}}{\pgfqpoint{2.615640in}{2.305161in}}{\pgfqpoint{2.621464in}{2.299337in}}%
\pgfpathcurveto{\pgfqpoint{2.627288in}{2.293513in}}{\pgfqpoint{2.635188in}{2.290241in}}{\pgfqpoint{2.643425in}{2.290241in}}%
\pgfpathclose%
\pgfusepath{stroke,fill}%
\end{pgfscope}%
\begin{pgfscope}%
\pgfpathrectangle{\pgfqpoint{0.100000in}{0.212622in}}{\pgfqpoint{3.696000in}{3.696000in}}%
\pgfusepath{clip}%
\pgfsetbuttcap%
\pgfsetroundjoin%
\definecolor{currentfill}{rgb}{0.121569,0.466667,0.705882}%
\pgfsetfillcolor{currentfill}%
\pgfsetfillopacity{0.484786}%
\pgfsetlinewidth{1.003750pt}%
\definecolor{currentstroke}{rgb}{0.121569,0.466667,0.705882}%
\pgfsetstrokecolor{currentstroke}%
\pgfsetstrokeopacity{0.484786}%
\pgfsetdash{}{0pt}%
\pgfpathmoveto{\pgfqpoint{2.660963in}{2.293518in}}%
\pgfpathcurveto{\pgfqpoint{2.669199in}{2.293518in}}{\pgfqpoint{2.677099in}{2.296791in}}{\pgfqpoint{2.682923in}{2.302614in}}%
\pgfpathcurveto{\pgfqpoint{2.688747in}{2.308438in}}{\pgfqpoint{2.692019in}{2.316338in}}{\pgfqpoint{2.692019in}{2.324575in}}%
\pgfpathcurveto{\pgfqpoint{2.692019in}{2.332811in}}{\pgfqpoint{2.688747in}{2.340711in}}{\pgfqpoint{2.682923in}{2.346535in}}%
\pgfpathcurveto{\pgfqpoint{2.677099in}{2.352359in}}{\pgfqpoint{2.669199in}{2.355631in}}{\pgfqpoint{2.660963in}{2.355631in}}%
\pgfpathcurveto{\pgfqpoint{2.652727in}{2.355631in}}{\pgfqpoint{2.644827in}{2.352359in}}{\pgfqpoint{2.639003in}{2.346535in}}%
\pgfpathcurveto{\pgfqpoint{2.633179in}{2.340711in}}{\pgfqpoint{2.629906in}{2.332811in}}{\pgfqpoint{2.629906in}{2.324575in}}%
\pgfpathcurveto{\pgfqpoint{2.629906in}{2.316338in}}{\pgfqpoint{2.633179in}{2.308438in}}{\pgfqpoint{2.639003in}{2.302614in}}%
\pgfpathcurveto{\pgfqpoint{2.644827in}{2.296791in}}{\pgfqpoint{2.652727in}{2.293518in}}{\pgfqpoint{2.660963in}{2.293518in}}%
\pgfpathclose%
\pgfusepath{stroke,fill}%
\end{pgfscope}%
\begin{pgfscope}%
\pgfpathrectangle{\pgfqpoint{0.100000in}{0.212622in}}{\pgfqpoint{3.696000in}{3.696000in}}%
\pgfusepath{clip}%
\pgfsetbuttcap%
\pgfsetroundjoin%
\definecolor{currentfill}{rgb}{0.121569,0.466667,0.705882}%
\pgfsetfillcolor{currentfill}%
\pgfsetfillopacity{0.488180}%
\pgfsetlinewidth{1.003750pt}%
\definecolor{currentstroke}{rgb}{0.121569,0.466667,0.705882}%
\pgfsetstrokecolor{currentstroke}%
\pgfsetstrokeopacity{0.488180}%
\pgfsetdash{}{0pt}%
\pgfpathmoveto{\pgfqpoint{1.170038in}{1.713893in}}%
\pgfpathcurveto{\pgfqpoint{1.178274in}{1.713893in}}{\pgfqpoint{1.186174in}{1.717165in}}{\pgfqpoint{1.191998in}{1.722989in}}%
\pgfpathcurveto{\pgfqpoint{1.197822in}{1.728813in}}{\pgfqpoint{1.201094in}{1.736713in}}{\pgfqpoint{1.201094in}{1.744949in}}%
\pgfpathcurveto{\pgfqpoint{1.201094in}{1.753186in}}{\pgfqpoint{1.197822in}{1.761086in}}{\pgfqpoint{1.191998in}{1.766910in}}%
\pgfpathcurveto{\pgfqpoint{1.186174in}{1.772733in}}{\pgfqpoint{1.178274in}{1.776006in}}{\pgfqpoint{1.170038in}{1.776006in}}%
\pgfpathcurveto{\pgfqpoint{1.161801in}{1.776006in}}{\pgfqpoint{1.153901in}{1.772733in}}{\pgfqpoint{1.148077in}{1.766910in}}%
\pgfpathcurveto{\pgfqpoint{1.142254in}{1.761086in}}{\pgfqpoint{1.138981in}{1.753186in}}{\pgfqpoint{1.138981in}{1.744949in}}%
\pgfpathcurveto{\pgfqpoint{1.138981in}{1.736713in}}{\pgfqpoint{1.142254in}{1.728813in}}{\pgfqpoint{1.148077in}{1.722989in}}%
\pgfpathcurveto{\pgfqpoint{1.153901in}{1.717165in}}{\pgfqpoint{1.161801in}{1.713893in}}{\pgfqpoint{1.170038in}{1.713893in}}%
\pgfpathclose%
\pgfusepath{stroke,fill}%
\end{pgfscope}%
\begin{pgfscope}%
\pgfpathrectangle{\pgfqpoint{0.100000in}{0.212622in}}{\pgfqpoint{3.696000in}{3.696000in}}%
\pgfusepath{clip}%
\pgfsetbuttcap%
\pgfsetroundjoin%
\definecolor{currentfill}{rgb}{0.121569,0.466667,0.705882}%
\pgfsetfillcolor{currentfill}%
\pgfsetfillopacity{0.488805}%
\pgfsetlinewidth{1.003750pt}%
\definecolor{currentstroke}{rgb}{0.121569,0.466667,0.705882}%
\pgfsetstrokecolor{currentstroke}%
\pgfsetstrokeopacity{0.488805}%
\pgfsetdash{}{0pt}%
\pgfpathmoveto{\pgfqpoint{2.680969in}{2.289321in}}%
\pgfpathcurveto{\pgfqpoint{2.689206in}{2.289321in}}{\pgfqpoint{2.697106in}{2.292593in}}{\pgfqpoint{2.702930in}{2.298417in}}%
\pgfpathcurveto{\pgfqpoint{2.708754in}{2.304241in}}{\pgfqpoint{2.712026in}{2.312141in}}{\pgfqpoint{2.712026in}{2.320377in}}%
\pgfpathcurveto{\pgfqpoint{2.712026in}{2.328614in}}{\pgfqpoint{2.708754in}{2.336514in}}{\pgfqpoint{2.702930in}{2.342337in}}%
\pgfpathcurveto{\pgfqpoint{2.697106in}{2.348161in}}{\pgfqpoint{2.689206in}{2.351434in}}{\pgfqpoint{2.680969in}{2.351434in}}%
\pgfpathcurveto{\pgfqpoint{2.672733in}{2.351434in}}{\pgfqpoint{2.664833in}{2.348161in}}{\pgfqpoint{2.659009in}{2.342337in}}%
\pgfpathcurveto{\pgfqpoint{2.653185in}{2.336514in}}{\pgfqpoint{2.649913in}{2.328614in}}{\pgfqpoint{2.649913in}{2.320377in}}%
\pgfpathcurveto{\pgfqpoint{2.649913in}{2.312141in}}{\pgfqpoint{2.653185in}{2.304241in}}{\pgfqpoint{2.659009in}{2.298417in}}%
\pgfpathcurveto{\pgfqpoint{2.664833in}{2.292593in}}{\pgfqpoint{2.672733in}{2.289321in}}{\pgfqpoint{2.680969in}{2.289321in}}%
\pgfpathclose%
\pgfusepath{stroke,fill}%
\end{pgfscope}%
\begin{pgfscope}%
\pgfpathrectangle{\pgfqpoint{0.100000in}{0.212622in}}{\pgfqpoint{3.696000in}{3.696000in}}%
\pgfusepath{clip}%
\pgfsetbuttcap%
\pgfsetroundjoin%
\definecolor{currentfill}{rgb}{0.121569,0.466667,0.705882}%
\pgfsetfillcolor{currentfill}%
\pgfsetfillopacity{0.491067}%
\pgfsetlinewidth{1.003750pt}%
\definecolor{currentstroke}{rgb}{0.121569,0.466667,0.705882}%
\pgfsetstrokecolor{currentstroke}%
\pgfsetstrokeopacity{0.491067}%
\pgfsetdash{}{0pt}%
\pgfpathmoveto{\pgfqpoint{2.693582in}{2.292323in}}%
\pgfpathcurveto{\pgfqpoint{2.701818in}{2.292323in}}{\pgfqpoint{2.709719in}{2.295596in}}{\pgfqpoint{2.715542in}{2.301420in}}%
\pgfpathcurveto{\pgfqpoint{2.721366in}{2.307243in}}{\pgfqpoint{2.724639in}{2.315144in}}{\pgfqpoint{2.724639in}{2.323380in}}%
\pgfpathcurveto{\pgfqpoint{2.724639in}{2.331616in}}{\pgfqpoint{2.721366in}{2.339516in}}{\pgfqpoint{2.715542in}{2.345340in}}%
\pgfpathcurveto{\pgfqpoint{2.709719in}{2.351164in}}{\pgfqpoint{2.701818in}{2.354436in}}{\pgfqpoint{2.693582in}{2.354436in}}%
\pgfpathcurveto{\pgfqpoint{2.685346in}{2.354436in}}{\pgfqpoint{2.677446in}{2.351164in}}{\pgfqpoint{2.671622in}{2.345340in}}%
\pgfpathcurveto{\pgfqpoint{2.665798in}{2.339516in}}{\pgfqpoint{2.662526in}{2.331616in}}{\pgfqpoint{2.662526in}{2.323380in}}%
\pgfpathcurveto{\pgfqpoint{2.662526in}{2.315144in}}{\pgfqpoint{2.665798in}{2.307243in}}{\pgfqpoint{2.671622in}{2.301420in}}%
\pgfpathcurveto{\pgfqpoint{2.677446in}{2.295596in}}{\pgfqpoint{2.685346in}{2.292323in}}{\pgfqpoint{2.693582in}{2.292323in}}%
\pgfpathclose%
\pgfusepath{stroke,fill}%
\end{pgfscope}%
\begin{pgfscope}%
\pgfpathrectangle{\pgfqpoint{0.100000in}{0.212622in}}{\pgfqpoint{3.696000in}{3.696000in}}%
\pgfusepath{clip}%
\pgfsetbuttcap%
\pgfsetroundjoin%
\definecolor{currentfill}{rgb}{0.121569,0.466667,0.705882}%
\pgfsetfillcolor{currentfill}%
\pgfsetfillopacity{0.493490}%
\pgfsetlinewidth{1.003750pt}%
\definecolor{currentstroke}{rgb}{0.121569,0.466667,0.705882}%
\pgfsetstrokecolor{currentstroke}%
\pgfsetstrokeopacity{0.493490}%
\pgfsetdash{}{0pt}%
\pgfpathmoveto{\pgfqpoint{2.707530in}{2.287359in}}%
\pgfpathcurveto{\pgfqpoint{2.715766in}{2.287359in}}{\pgfqpoint{2.723666in}{2.290631in}}{\pgfqpoint{2.729490in}{2.296455in}}%
\pgfpathcurveto{\pgfqpoint{2.735314in}{2.302279in}}{\pgfqpoint{2.738586in}{2.310179in}}{\pgfqpoint{2.738586in}{2.318415in}}%
\pgfpathcurveto{\pgfqpoint{2.738586in}{2.326651in}}{\pgfqpoint{2.735314in}{2.334551in}}{\pgfqpoint{2.729490in}{2.340375in}}%
\pgfpathcurveto{\pgfqpoint{2.723666in}{2.346199in}}{\pgfqpoint{2.715766in}{2.349472in}}{\pgfqpoint{2.707530in}{2.349472in}}%
\pgfpathcurveto{\pgfqpoint{2.699294in}{2.349472in}}{\pgfqpoint{2.691394in}{2.346199in}}{\pgfqpoint{2.685570in}{2.340375in}}%
\pgfpathcurveto{\pgfqpoint{2.679746in}{2.334551in}}{\pgfqpoint{2.676473in}{2.326651in}}{\pgfqpoint{2.676473in}{2.318415in}}%
\pgfpathcurveto{\pgfqpoint{2.676473in}{2.310179in}}{\pgfqpoint{2.679746in}{2.302279in}}{\pgfqpoint{2.685570in}{2.296455in}}%
\pgfpathcurveto{\pgfqpoint{2.691394in}{2.290631in}}{\pgfqpoint{2.699294in}{2.287359in}}{\pgfqpoint{2.707530in}{2.287359in}}%
\pgfpathclose%
\pgfusepath{stroke,fill}%
\end{pgfscope}%
\begin{pgfscope}%
\pgfpathrectangle{\pgfqpoint{0.100000in}{0.212622in}}{\pgfqpoint{3.696000in}{3.696000in}}%
\pgfusepath{clip}%
\pgfsetbuttcap%
\pgfsetroundjoin%
\definecolor{currentfill}{rgb}{0.121569,0.466667,0.705882}%
\pgfsetfillcolor{currentfill}%
\pgfsetfillopacity{0.495609}%
\pgfsetlinewidth{1.003750pt}%
\definecolor{currentstroke}{rgb}{0.121569,0.466667,0.705882}%
\pgfsetstrokecolor{currentstroke}%
\pgfsetstrokeopacity{0.495609}%
\pgfsetdash{}{0pt}%
\pgfpathmoveto{\pgfqpoint{1.159444in}{1.692229in}}%
\pgfpathcurveto{\pgfqpoint{1.167680in}{1.692229in}}{\pgfqpoint{1.175581in}{1.695501in}}{\pgfqpoint{1.181404in}{1.701325in}}%
\pgfpathcurveto{\pgfqpoint{1.187228in}{1.707149in}}{\pgfqpoint{1.190501in}{1.715049in}}{\pgfqpoint{1.190501in}{1.723285in}}%
\pgfpathcurveto{\pgfqpoint{1.190501in}{1.731522in}}{\pgfqpoint{1.187228in}{1.739422in}}{\pgfqpoint{1.181404in}{1.745246in}}%
\pgfpathcurveto{\pgfqpoint{1.175581in}{1.751070in}}{\pgfqpoint{1.167680in}{1.754342in}}{\pgfqpoint{1.159444in}{1.754342in}}%
\pgfpathcurveto{\pgfqpoint{1.151208in}{1.754342in}}{\pgfqpoint{1.143308in}{1.751070in}}{\pgfqpoint{1.137484in}{1.745246in}}%
\pgfpathcurveto{\pgfqpoint{1.131660in}{1.739422in}}{\pgfqpoint{1.128388in}{1.731522in}}{\pgfqpoint{1.128388in}{1.723285in}}%
\pgfpathcurveto{\pgfqpoint{1.128388in}{1.715049in}}{\pgfqpoint{1.131660in}{1.707149in}}{\pgfqpoint{1.137484in}{1.701325in}}%
\pgfpathcurveto{\pgfqpoint{1.143308in}{1.695501in}}{\pgfqpoint{1.151208in}{1.692229in}}{\pgfqpoint{1.159444in}{1.692229in}}%
\pgfpathclose%
\pgfusepath{stroke,fill}%
\end{pgfscope}%
\begin{pgfscope}%
\pgfpathrectangle{\pgfqpoint{0.100000in}{0.212622in}}{\pgfqpoint{3.696000in}{3.696000in}}%
\pgfusepath{clip}%
\pgfsetbuttcap%
\pgfsetroundjoin%
\definecolor{currentfill}{rgb}{0.121569,0.466667,0.705882}%
\pgfsetfillcolor{currentfill}%
\pgfsetfillopacity{0.496840}%
\pgfsetlinewidth{1.003750pt}%
\definecolor{currentstroke}{rgb}{0.121569,0.466667,0.705882}%
\pgfsetstrokecolor{currentstroke}%
\pgfsetstrokeopacity{0.496840}%
\pgfsetdash{}{0pt}%
\pgfpathmoveto{\pgfqpoint{2.726375in}{2.291181in}}%
\pgfpathcurveto{\pgfqpoint{2.734611in}{2.291181in}}{\pgfqpoint{2.742512in}{2.294453in}}{\pgfqpoint{2.748335in}{2.300277in}}%
\pgfpathcurveto{\pgfqpoint{2.754159in}{2.306101in}}{\pgfqpoint{2.757432in}{2.314001in}}{\pgfqpoint{2.757432in}{2.322237in}}%
\pgfpathcurveto{\pgfqpoint{2.757432in}{2.330473in}}{\pgfqpoint{2.754159in}{2.338373in}}{\pgfqpoint{2.748335in}{2.344197in}}%
\pgfpathcurveto{\pgfqpoint{2.742512in}{2.350021in}}{\pgfqpoint{2.734611in}{2.353294in}}{\pgfqpoint{2.726375in}{2.353294in}}%
\pgfpathcurveto{\pgfqpoint{2.718139in}{2.353294in}}{\pgfqpoint{2.710239in}{2.350021in}}{\pgfqpoint{2.704415in}{2.344197in}}%
\pgfpathcurveto{\pgfqpoint{2.698591in}{2.338373in}}{\pgfqpoint{2.695319in}{2.330473in}}{\pgfqpoint{2.695319in}{2.322237in}}%
\pgfpathcurveto{\pgfqpoint{2.695319in}{2.314001in}}{\pgfqpoint{2.698591in}{2.306101in}}{\pgfqpoint{2.704415in}{2.300277in}}%
\pgfpathcurveto{\pgfqpoint{2.710239in}{2.294453in}}{\pgfqpoint{2.718139in}{2.291181in}}{\pgfqpoint{2.726375in}{2.291181in}}%
\pgfpathclose%
\pgfusepath{stroke,fill}%
\end{pgfscope}%
\begin{pgfscope}%
\pgfpathrectangle{\pgfqpoint{0.100000in}{0.212622in}}{\pgfqpoint{3.696000in}{3.696000in}}%
\pgfusepath{clip}%
\pgfsetbuttcap%
\pgfsetroundjoin%
\definecolor{currentfill}{rgb}{0.121569,0.466667,0.705882}%
\pgfsetfillcolor{currentfill}%
\pgfsetfillopacity{0.498544}%
\pgfsetlinewidth{1.003750pt}%
\definecolor{currentstroke}{rgb}{0.121569,0.466667,0.705882}%
\pgfsetstrokecolor{currentstroke}%
\pgfsetstrokeopacity{0.498544}%
\pgfsetdash{}{0pt}%
\pgfpathmoveto{\pgfqpoint{2.735648in}{2.288886in}}%
\pgfpathcurveto{\pgfqpoint{2.743884in}{2.288886in}}{\pgfqpoint{2.751784in}{2.292158in}}{\pgfqpoint{2.757608in}{2.297982in}}%
\pgfpathcurveto{\pgfqpoint{2.763432in}{2.303806in}}{\pgfqpoint{2.766705in}{2.311706in}}{\pgfqpoint{2.766705in}{2.319942in}}%
\pgfpathcurveto{\pgfqpoint{2.766705in}{2.328179in}}{\pgfqpoint{2.763432in}{2.336079in}}{\pgfqpoint{2.757608in}{2.341903in}}%
\pgfpathcurveto{\pgfqpoint{2.751784in}{2.347726in}}{\pgfqpoint{2.743884in}{2.350999in}}{\pgfqpoint{2.735648in}{2.350999in}}%
\pgfpathcurveto{\pgfqpoint{2.727412in}{2.350999in}}{\pgfqpoint{2.719512in}{2.347726in}}{\pgfqpoint{2.713688in}{2.341903in}}%
\pgfpathcurveto{\pgfqpoint{2.707864in}{2.336079in}}{\pgfqpoint{2.704592in}{2.328179in}}{\pgfqpoint{2.704592in}{2.319942in}}%
\pgfpathcurveto{\pgfqpoint{2.704592in}{2.311706in}}{\pgfqpoint{2.707864in}{2.303806in}}{\pgfqpoint{2.713688in}{2.297982in}}%
\pgfpathcurveto{\pgfqpoint{2.719512in}{2.292158in}}{\pgfqpoint{2.727412in}{2.288886in}}{\pgfqpoint{2.735648in}{2.288886in}}%
\pgfpathclose%
\pgfusepath{stroke,fill}%
\end{pgfscope}%
\begin{pgfscope}%
\pgfpathrectangle{\pgfqpoint{0.100000in}{0.212622in}}{\pgfqpoint{3.696000in}{3.696000in}}%
\pgfusepath{clip}%
\pgfsetbuttcap%
\pgfsetroundjoin%
\definecolor{currentfill}{rgb}{0.121569,0.466667,0.705882}%
\pgfsetfillcolor{currentfill}%
\pgfsetfillopacity{0.500117}%
\pgfsetlinewidth{1.003750pt}%
\definecolor{currentstroke}{rgb}{0.121569,0.466667,0.705882}%
\pgfsetstrokecolor{currentstroke}%
\pgfsetstrokeopacity{0.500117}%
\pgfsetdash{}{0pt}%
\pgfpathmoveto{\pgfqpoint{1.146426in}{1.674212in}}%
\pgfpathcurveto{\pgfqpoint{1.154662in}{1.674212in}}{\pgfqpoint{1.162562in}{1.677485in}}{\pgfqpoint{1.168386in}{1.683308in}}%
\pgfpathcurveto{\pgfqpoint{1.174210in}{1.689132in}}{\pgfqpoint{1.177482in}{1.697032in}}{\pgfqpoint{1.177482in}{1.705269in}}%
\pgfpathcurveto{\pgfqpoint{1.177482in}{1.713505in}}{\pgfqpoint{1.174210in}{1.721405in}}{\pgfqpoint{1.168386in}{1.727229in}}%
\pgfpathcurveto{\pgfqpoint{1.162562in}{1.733053in}}{\pgfqpoint{1.154662in}{1.736325in}}{\pgfqpoint{1.146426in}{1.736325in}}%
\pgfpathcurveto{\pgfqpoint{1.138189in}{1.736325in}}{\pgfqpoint{1.130289in}{1.733053in}}{\pgfqpoint{1.124465in}{1.727229in}}%
\pgfpathcurveto{\pgfqpoint{1.118641in}{1.721405in}}{\pgfqpoint{1.115369in}{1.713505in}}{\pgfqpoint{1.115369in}{1.705269in}}%
\pgfpathcurveto{\pgfqpoint{1.115369in}{1.697032in}}{\pgfqpoint{1.118641in}{1.689132in}}{\pgfqpoint{1.124465in}{1.683308in}}%
\pgfpathcurveto{\pgfqpoint{1.130289in}{1.677485in}}{\pgfqpoint{1.138189in}{1.674212in}}{\pgfqpoint{1.146426in}{1.674212in}}%
\pgfpathclose%
\pgfusepath{stroke,fill}%
\end{pgfscope}%
\begin{pgfscope}%
\pgfpathrectangle{\pgfqpoint{0.100000in}{0.212622in}}{\pgfqpoint{3.696000in}{3.696000in}}%
\pgfusepath{clip}%
\pgfsetbuttcap%
\pgfsetroundjoin%
\definecolor{currentfill}{rgb}{0.121569,0.466667,0.705882}%
\pgfsetfillcolor{currentfill}%
\pgfsetfillopacity{0.501243}%
\pgfsetlinewidth{1.003750pt}%
\definecolor{currentstroke}{rgb}{0.121569,0.466667,0.705882}%
\pgfsetstrokecolor{currentstroke}%
\pgfsetstrokeopacity{0.501243}%
\pgfsetdash{}{0pt}%
\pgfpathmoveto{\pgfqpoint{2.750327in}{2.285680in}}%
\pgfpathcurveto{\pgfqpoint{2.758564in}{2.285680in}}{\pgfqpoint{2.766464in}{2.288952in}}{\pgfqpoint{2.772288in}{2.294776in}}%
\pgfpathcurveto{\pgfqpoint{2.778111in}{2.300600in}}{\pgfqpoint{2.781384in}{2.308500in}}{\pgfqpoint{2.781384in}{2.316736in}}%
\pgfpathcurveto{\pgfqpoint{2.781384in}{2.324972in}}{\pgfqpoint{2.778111in}{2.332872in}}{\pgfqpoint{2.772288in}{2.338696in}}%
\pgfpathcurveto{\pgfqpoint{2.766464in}{2.344520in}}{\pgfqpoint{2.758564in}{2.347793in}}{\pgfqpoint{2.750327in}{2.347793in}}%
\pgfpathcurveto{\pgfqpoint{2.742091in}{2.347793in}}{\pgfqpoint{2.734191in}{2.344520in}}{\pgfqpoint{2.728367in}{2.338696in}}%
\pgfpathcurveto{\pgfqpoint{2.722543in}{2.332872in}}{\pgfqpoint{2.719271in}{2.324972in}}{\pgfqpoint{2.719271in}{2.316736in}}%
\pgfpathcurveto{\pgfqpoint{2.719271in}{2.308500in}}{\pgfqpoint{2.722543in}{2.300600in}}{\pgfqpoint{2.728367in}{2.294776in}}%
\pgfpathcurveto{\pgfqpoint{2.734191in}{2.288952in}}{\pgfqpoint{2.742091in}{2.285680in}}{\pgfqpoint{2.750327in}{2.285680in}}%
\pgfpathclose%
\pgfusepath{stroke,fill}%
\end{pgfscope}%
\begin{pgfscope}%
\pgfpathrectangle{\pgfqpoint{0.100000in}{0.212622in}}{\pgfqpoint{3.696000in}{3.696000in}}%
\pgfusepath{clip}%
\pgfsetbuttcap%
\pgfsetroundjoin%
\definecolor{currentfill}{rgb}{0.121569,0.466667,0.705882}%
\pgfsetfillcolor{currentfill}%
\pgfsetfillopacity{0.502906}%
\pgfsetlinewidth{1.003750pt}%
\definecolor{currentstroke}{rgb}{0.121569,0.466667,0.705882}%
\pgfsetstrokecolor{currentstroke}%
\pgfsetstrokeopacity{0.502906}%
\pgfsetdash{}{0pt}%
\pgfpathmoveto{\pgfqpoint{1.138963in}{1.662122in}}%
\pgfpathcurveto{\pgfqpoint{1.147199in}{1.662122in}}{\pgfqpoint{1.155099in}{1.665395in}}{\pgfqpoint{1.160923in}{1.671219in}}%
\pgfpathcurveto{\pgfqpoint{1.166747in}{1.677043in}}{\pgfqpoint{1.170020in}{1.684943in}}{\pgfqpoint{1.170020in}{1.693179in}}%
\pgfpathcurveto{\pgfqpoint{1.170020in}{1.701415in}}{\pgfqpoint{1.166747in}{1.709315in}}{\pgfqpoint{1.160923in}{1.715139in}}%
\pgfpathcurveto{\pgfqpoint{1.155099in}{1.720963in}}{\pgfqpoint{1.147199in}{1.724235in}}{\pgfqpoint{1.138963in}{1.724235in}}%
\pgfpathcurveto{\pgfqpoint{1.130727in}{1.724235in}}{\pgfqpoint{1.122827in}{1.720963in}}{\pgfqpoint{1.117003in}{1.715139in}}%
\pgfpathcurveto{\pgfqpoint{1.111179in}{1.709315in}}{\pgfqpoint{1.107907in}{1.701415in}}{\pgfqpoint{1.107907in}{1.693179in}}%
\pgfpathcurveto{\pgfqpoint{1.107907in}{1.684943in}}{\pgfqpoint{1.111179in}{1.677043in}}{\pgfqpoint{1.117003in}{1.671219in}}%
\pgfpathcurveto{\pgfqpoint{1.122827in}{1.665395in}}{\pgfqpoint{1.130727in}{1.662122in}}{\pgfqpoint{1.138963in}{1.662122in}}%
\pgfpathclose%
\pgfusepath{stroke,fill}%
\end{pgfscope}%
\begin{pgfscope}%
\pgfpathrectangle{\pgfqpoint{0.100000in}{0.212622in}}{\pgfqpoint{3.696000in}{3.696000in}}%
\pgfusepath{clip}%
\pgfsetbuttcap%
\pgfsetroundjoin%
\definecolor{currentfill}{rgb}{0.121569,0.466667,0.705882}%
\pgfsetfillcolor{currentfill}%
\pgfsetfillopacity{0.502956}%
\pgfsetlinewidth{1.003750pt}%
\definecolor{currentstroke}{rgb}{0.121569,0.466667,0.705882}%
\pgfsetstrokecolor{currentstroke}%
\pgfsetstrokeopacity{0.502956}%
\pgfsetdash{}{0pt}%
\pgfpathmoveto{\pgfqpoint{2.759416in}{2.288652in}}%
\pgfpathcurveto{\pgfqpoint{2.767652in}{2.288652in}}{\pgfqpoint{2.775552in}{2.291924in}}{\pgfqpoint{2.781376in}{2.297748in}}%
\pgfpathcurveto{\pgfqpoint{2.787200in}{2.303572in}}{\pgfqpoint{2.790472in}{2.311472in}}{\pgfqpoint{2.790472in}{2.319709in}}%
\pgfpathcurveto{\pgfqpoint{2.790472in}{2.327945in}}{\pgfqpoint{2.787200in}{2.335845in}}{\pgfqpoint{2.781376in}{2.341669in}}%
\pgfpathcurveto{\pgfqpoint{2.775552in}{2.347493in}}{\pgfqpoint{2.767652in}{2.350765in}}{\pgfqpoint{2.759416in}{2.350765in}}%
\pgfpathcurveto{\pgfqpoint{2.751180in}{2.350765in}}{\pgfqpoint{2.743280in}{2.347493in}}{\pgfqpoint{2.737456in}{2.341669in}}%
\pgfpathcurveto{\pgfqpoint{2.731632in}{2.335845in}}{\pgfqpoint{2.728359in}{2.327945in}}{\pgfqpoint{2.728359in}{2.319709in}}%
\pgfpathcurveto{\pgfqpoint{2.728359in}{2.311472in}}{\pgfqpoint{2.731632in}{2.303572in}}{\pgfqpoint{2.737456in}{2.297748in}}%
\pgfpathcurveto{\pgfqpoint{2.743280in}{2.291924in}}{\pgfqpoint{2.751180in}{2.288652in}}{\pgfqpoint{2.759416in}{2.288652in}}%
\pgfpathclose%
\pgfusepath{stroke,fill}%
\end{pgfscope}%
\begin{pgfscope}%
\pgfpathrectangle{\pgfqpoint{0.100000in}{0.212622in}}{\pgfqpoint{3.696000in}{3.696000in}}%
\pgfusepath{clip}%
\pgfsetbuttcap%
\pgfsetroundjoin%
\definecolor{currentfill}{rgb}{0.121569,0.466667,0.705882}%
\pgfsetfillcolor{currentfill}%
\pgfsetfillopacity{0.503785}%
\pgfsetlinewidth{1.003750pt}%
\definecolor{currentstroke}{rgb}{0.121569,0.466667,0.705882}%
\pgfsetstrokecolor{currentstroke}%
\pgfsetstrokeopacity{0.503785}%
\pgfsetdash{}{0pt}%
\pgfpathmoveto{\pgfqpoint{2.763765in}{2.287495in}}%
\pgfpathcurveto{\pgfqpoint{2.772001in}{2.287495in}}{\pgfqpoint{2.779901in}{2.290768in}}{\pgfqpoint{2.785725in}{2.296592in}}%
\pgfpathcurveto{\pgfqpoint{2.791549in}{2.302415in}}{\pgfqpoint{2.794821in}{2.310315in}}{\pgfqpoint{2.794821in}{2.318552in}}%
\pgfpathcurveto{\pgfqpoint{2.794821in}{2.326788in}}{\pgfqpoint{2.791549in}{2.334688in}}{\pgfqpoint{2.785725in}{2.340512in}}%
\pgfpathcurveto{\pgfqpoint{2.779901in}{2.346336in}}{\pgfqpoint{2.772001in}{2.349608in}}{\pgfqpoint{2.763765in}{2.349608in}}%
\pgfpathcurveto{\pgfqpoint{2.755528in}{2.349608in}}{\pgfqpoint{2.747628in}{2.346336in}}{\pgfqpoint{2.741804in}{2.340512in}}%
\pgfpathcurveto{\pgfqpoint{2.735980in}{2.334688in}}{\pgfqpoint{2.732708in}{2.326788in}}{\pgfqpoint{2.732708in}{2.318552in}}%
\pgfpathcurveto{\pgfqpoint{2.732708in}{2.310315in}}{\pgfqpoint{2.735980in}{2.302415in}}{\pgfqpoint{2.741804in}{2.296592in}}%
\pgfpathcurveto{\pgfqpoint{2.747628in}{2.290768in}}{\pgfqpoint{2.755528in}{2.287495in}}{\pgfqpoint{2.763765in}{2.287495in}}%
\pgfpathclose%
\pgfusepath{stroke,fill}%
\end{pgfscope}%
\begin{pgfscope}%
\pgfpathrectangle{\pgfqpoint{0.100000in}{0.212622in}}{\pgfqpoint{3.696000in}{3.696000in}}%
\pgfusepath{clip}%
\pgfsetbuttcap%
\pgfsetroundjoin%
\definecolor{currentfill}{rgb}{0.121569,0.466667,0.705882}%
\pgfsetfillcolor{currentfill}%
\pgfsetfillopacity{0.504296}%
\pgfsetlinewidth{1.003750pt}%
\definecolor{currentstroke}{rgb}{0.121569,0.466667,0.705882}%
\pgfsetstrokecolor{currentstroke}%
\pgfsetstrokeopacity{0.504296}%
\pgfsetdash{}{0pt}%
\pgfpathmoveto{\pgfqpoint{1.134404in}{1.658137in}}%
\pgfpathcurveto{\pgfqpoint{1.142640in}{1.658137in}}{\pgfqpoint{1.150540in}{1.661409in}}{\pgfqpoint{1.156364in}{1.667233in}}%
\pgfpathcurveto{\pgfqpoint{1.162188in}{1.673057in}}{\pgfqpoint{1.165460in}{1.680957in}}{\pgfqpoint{1.165460in}{1.689193in}}%
\pgfpathcurveto{\pgfqpoint{1.165460in}{1.697430in}}{\pgfqpoint{1.162188in}{1.705330in}}{\pgfqpoint{1.156364in}{1.711154in}}%
\pgfpathcurveto{\pgfqpoint{1.150540in}{1.716978in}}{\pgfqpoint{1.142640in}{1.720250in}}{\pgfqpoint{1.134404in}{1.720250in}}%
\pgfpathcurveto{\pgfqpoint{1.126167in}{1.720250in}}{\pgfqpoint{1.118267in}{1.716978in}}{\pgfqpoint{1.112443in}{1.711154in}}%
\pgfpathcurveto{\pgfqpoint{1.106619in}{1.705330in}}{\pgfqpoint{1.103347in}{1.697430in}}{\pgfqpoint{1.103347in}{1.689193in}}%
\pgfpathcurveto{\pgfqpoint{1.103347in}{1.680957in}}{\pgfqpoint{1.106619in}{1.673057in}}{\pgfqpoint{1.112443in}{1.667233in}}%
\pgfpathcurveto{\pgfqpoint{1.118267in}{1.661409in}}{\pgfqpoint{1.126167in}{1.658137in}}{\pgfqpoint{1.134404in}{1.658137in}}%
\pgfpathclose%
\pgfusepath{stroke,fill}%
\end{pgfscope}%
\begin{pgfscope}%
\pgfpathrectangle{\pgfqpoint{0.100000in}{0.212622in}}{\pgfqpoint{3.696000in}{3.696000in}}%
\pgfusepath{clip}%
\pgfsetbuttcap%
\pgfsetroundjoin%
\definecolor{currentfill}{rgb}{0.121569,0.466667,0.705882}%
\pgfsetfillcolor{currentfill}%
\pgfsetfillopacity{0.504824}%
\pgfsetlinewidth{1.003750pt}%
\definecolor{currentstroke}{rgb}{0.121569,0.466667,0.705882}%
\pgfsetstrokecolor{currentstroke}%
\pgfsetstrokeopacity{0.504824}%
\pgfsetdash{}{0pt}%
\pgfpathmoveto{\pgfqpoint{1.133108in}{1.655823in}}%
\pgfpathcurveto{\pgfqpoint{1.141344in}{1.655823in}}{\pgfqpoint{1.149244in}{1.659096in}}{\pgfqpoint{1.155068in}{1.664919in}}%
\pgfpathcurveto{\pgfqpoint{1.160892in}{1.670743in}}{\pgfqpoint{1.164164in}{1.678643in}}{\pgfqpoint{1.164164in}{1.686880in}}%
\pgfpathcurveto{\pgfqpoint{1.164164in}{1.695116in}}{\pgfqpoint{1.160892in}{1.703016in}}{\pgfqpoint{1.155068in}{1.708840in}}%
\pgfpathcurveto{\pgfqpoint{1.149244in}{1.714664in}}{\pgfqpoint{1.141344in}{1.717936in}}{\pgfqpoint{1.133108in}{1.717936in}}%
\pgfpathcurveto{\pgfqpoint{1.124871in}{1.717936in}}{\pgfqpoint{1.116971in}{1.714664in}}{\pgfqpoint{1.111147in}{1.708840in}}%
\pgfpathcurveto{\pgfqpoint{1.105324in}{1.703016in}}{\pgfqpoint{1.102051in}{1.695116in}}{\pgfqpoint{1.102051in}{1.686880in}}%
\pgfpathcurveto{\pgfqpoint{1.102051in}{1.678643in}}{\pgfqpoint{1.105324in}{1.670743in}}{\pgfqpoint{1.111147in}{1.664919in}}%
\pgfpathcurveto{\pgfqpoint{1.116971in}{1.659096in}}{\pgfqpoint{1.124871in}{1.655823in}}{\pgfqpoint{1.133108in}{1.655823in}}%
\pgfpathclose%
\pgfusepath{stroke,fill}%
\end{pgfscope}%
\begin{pgfscope}%
\pgfpathrectangle{\pgfqpoint{0.100000in}{0.212622in}}{\pgfqpoint{3.696000in}{3.696000in}}%
\pgfusepath{clip}%
\pgfsetbuttcap%
\pgfsetroundjoin%
\definecolor{currentfill}{rgb}{0.121569,0.466667,0.705882}%
\pgfsetfillcolor{currentfill}%
\pgfsetfillopacity{0.505136}%
\pgfsetlinewidth{1.003750pt}%
\definecolor{currentstroke}{rgb}{0.121569,0.466667,0.705882}%
\pgfsetstrokecolor{currentstroke}%
\pgfsetstrokeopacity{0.505136}%
\pgfsetdash{}{0pt}%
\pgfpathmoveto{\pgfqpoint{2.770632in}{2.290021in}}%
\pgfpathcurveto{\pgfqpoint{2.778868in}{2.290021in}}{\pgfqpoint{2.786768in}{2.293293in}}{\pgfqpoint{2.792592in}{2.299117in}}%
\pgfpathcurveto{\pgfqpoint{2.798416in}{2.304941in}}{\pgfqpoint{2.801688in}{2.312841in}}{\pgfqpoint{2.801688in}{2.321077in}}%
\pgfpathcurveto{\pgfqpoint{2.801688in}{2.329313in}}{\pgfqpoint{2.798416in}{2.337213in}}{\pgfqpoint{2.792592in}{2.343037in}}%
\pgfpathcurveto{\pgfqpoint{2.786768in}{2.348861in}}{\pgfqpoint{2.778868in}{2.352133in}}{\pgfqpoint{2.770632in}{2.352133in}}%
\pgfpathcurveto{\pgfqpoint{2.762395in}{2.352133in}}{\pgfqpoint{2.754495in}{2.348861in}}{\pgfqpoint{2.748671in}{2.343037in}}%
\pgfpathcurveto{\pgfqpoint{2.742847in}{2.337213in}}{\pgfqpoint{2.739575in}{2.329313in}}{\pgfqpoint{2.739575in}{2.321077in}}%
\pgfpathcurveto{\pgfqpoint{2.739575in}{2.312841in}}{\pgfqpoint{2.742847in}{2.304941in}}{\pgfqpoint{2.748671in}{2.299117in}}%
\pgfpathcurveto{\pgfqpoint{2.754495in}{2.293293in}}{\pgfqpoint{2.762395in}{2.290021in}}{\pgfqpoint{2.770632in}{2.290021in}}%
\pgfpathclose%
\pgfusepath{stroke,fill}%
\end{pgfscope}%
\begin{pgfscope}%
\pgfpathrectangle{\pgfqpoint{0.100000in}{0.212622in}}{\pgfqpoint{3.696000in}{3.696000in}}%
\pgfusepath{clip}%
\pgfsetbuttcap%
\pgfsetroundjoin%
\definecolor{currentfill}{rgb}{0.121569,0.466667,0.705882}%
\pgfsetfillcolor{currentfill}%
\pgfsetfillopacity{0.505831}%
\pgfsetlinewidth{1.003750pt}%
\definecolor{currentstroke}{rgb}{0.121569,0.466667,0.705882}%
\pgfsetstrokecolor{currentstroke}%
\pgfsetstrokeopacity{0.505831}%
\pgfsetdash{}{0pt}%
\pgfpathmoveto{\pgfqpoint{1.130400in}{1.652200in}}%
\pgfpathcurveto{\pgfqpoint{1.138636in}{1.652200in}}{\pgfqpoint{1.146536in}{1.655472in}}{\pgfqpoint{1.152360in}{1.661296in}}%
\pgfpathcurveto{\pgfqpoint{1.158184in}{1.667120in}}{\pgfqpoint{1.161456in}{1.675020in}}{\pgfqpoint{1.161456in}{1.683256in}}%
\pgfpathcurveto{\pgfqpoint{1.161456in}{1.691492in}}{\pgfqpoint{1.158184in}{1.699392in}}{\pgfqpoint{1.152360in}{1.705216in}}%
\pgfpathcurveto{\pgfqpoint{1.146536in}{1.711040in}}{\pgfqpoint{1.138636in}{1.714313in}}{\pgfqpoint{1.130400in}{1.714313in}}%
\pgfpathcurveto{\pgfqpoint{1.122163in}{1.714313in}}{\pgfqpoint{1.114263in}{1.711040in}}{\pgfqpoint{1.108439in}{1.705216in}}%
\pgfpathcurveto{\pgfqpoint{1.102615in}{1.699392in}}{\pgfqpoint{1.099343in}{1.691492in}}{\pgfqpoint{1.099343in}{1.683256in}}%
\pgfpathcurveto{\pgfqpoint{1.099343in}{1.675020in}}{\pgfqpoint{1.102615in}{1.667120in}}{\pgfqpoint{1.108439in}{1.661296in}}%
\pgfpathcurveto{\pgfqpoint{1.114263in}{1.655472in}}{\pgfqpoint{1.122163in}{1.652200in}}{\pgfqpoint{1.130400in}{1.652200in}}%
\pgfpathclose%
\pgfusepath{stroke,fill}%
\end{pgfscope}%
\begin{pgfscope}%
\pgfpathrectangle{\pgfqpoint{0.100000in}{0.212622in}}{\pgfqpoint{3.696000in}{3.696000in}}%
\pgfusepath{clip}%
\pgfsetbuttcap%
\pgfsetroundjoin%
\definecolor{currentfill}{rgb}{0.121569,0.466667,0.705882}%
\pgfsetfillcolor{currentfill}%
\pgfsetfillopacity{0.506402}%
\pgfsetlinewidth{1.003750pt}%
\definecolor{currentstroke}{rgb}{0.121569,0.466667,0.705882}%
\pgfsetstrokecolor{currentstroke}%
\pgfsetstrokeopacity{0.506402}%
\pgfsetdash{}{0pt}%
\pgfpathmoveto{\pgfqpoint{2.778375in}{2.286202in}}%
\pgfpathcurveto{\pgfqpoint{2.786612in}{2.286202in}}{\pgfqpoint{2.794512in}{2.289475in}}{\pgfqpoint{2.800336in}{2.295299in}}%
\pgfpathcurveto{\pgfqpoint{2.806160in}{2.301123in}}{\pgfqpoint{2.809432in}{2.309023in}}{\pgfqpoint{2.809432in}{2.317259in}}%
\pgfpathcurveto{\pgfqpoint{2.809432in}{2.325495in}}{\pgfqpoint{2.806160in}{2.333395in}}{\pgfqpoint{2.800336in}{2.339219in}}%
\pgfpathcurveto{\pgfqpoint{2.794512in}{2.345043in}}{\pgfqpoint{2.786612in}{2.348315in}}{\pgfqpoint{2.778375in}{2.348315in}}%
\pgfpathcurveto{\pgfqpoint{2.770139in}{2.348315in}}{\pgfqpoint{2.762239in}{2.345043in}}{\pgfqpoint{2.756415in}{2.339219in}}%
\pgfpathcurveto{\pgfqpoint{2.750591in}{2.333395in}}{\pgfqpoint{2.747319in}{2.325495in}}{\pgfqpoint{2.747319in}{2.317259in}}%
\pgfpathcurveto{\pgfqpoint{2.747319in}{2.309023in}}{\pgfqpoint{2.750591in}{2.301123in}}{\pgfqpoint{2.756415in}{2.295299in}}%
\pgfpathcurveto{\pgfqpoint{2.762239in}{2.289475in}}{\pgfqpoint{2.770139in}{2.286202in}}{\pgfqpoint{2.778375in}{2.286202in}}%
\pgfpathclose%
\pgfusepath{stroke,fill}%
\end{pgfscope}%
\begin{pgfscope}%
\pgfpathrectangle{\pgfqpoint{0.100000in}{0.212622in}}{\pgfqpoint{3.696000in}{3.696000in}}%
\pgfusepath{clip}%
\pgfsetbuttcap%
\pgfsetroundjoin%
\definecolor{currentfill}{rgb}{0.121569,0.466667,0.705882}%
\pgfsetfillcolor{currentfill}%
\pgfsetfillopacity{0.507631}%
\pgfsetlinewidth{1.003750pt}%
\definecolor{currentstroke}{rgb}{0.121569,0.466667,0.705882}%
\pgfsetstrokecolor{currentstroke}%
\pgfsetstrokeopacity{0.507631}%
\pgfsetdash{}{0pt}%
\pgfpathmoveto{\pgfqpoint{1.125702in}{1.645213in}}%
\pgfpathcurveto{\pgfqpoint{1.133938in}{1.645213in}}{\pgfqpoint{1.141838in}{1.648485in}}{\pgfqpoint{1.147662in}{1.654309in}}%
\pgfpathcurveto{\pgfqpoint{1.153486in}{1.660133in}}{\pgfqpoint{1.156758in}{1.668033in}}{\pgfqpoint{1.156758in}{1.676269in}}%
\pgfpathcurveto{\pgfqpoint{1.156758in}{1.684505in}}{\pgfqpoint{1.153486in}{1.692406in}}{\pgfqpoint{1.147662in}{1.698229in}}%
\pgfpathcurveto{\pgfqpoint{1.141838in}{1.704053in}}{\pgfqpoint{1.133938in}{1.707326in}}{\pgfqpoint{1.125702in}{1.707326in}}%
\pgfpathcurveto{\pgfqpoint{1.117466in}{1.707326in}}{\pgfqpoint{1.109566in}{1.704053in}}{\pgfqpoint{1.103742in}{1.698229in}}%
\pgfpathcurveto{\pgfqpoint{1.097918in}{1.692406in}}{\pgfqpoint{1.094645in}{1.684505in}}{\pgfqpoint{1.094645in}{1.676269in}}%
\pgfpathcurveto{\pgfqpoint{1.094645in}{1.668033in}}{\pgfqpoint{1.097918in}{1.660133in}}{\pgfqpoint{1.103742in}{1.654309in}}%
\pgfpathcurveto{\pgfqpoint{1.109566in}{1.648485in}}{\pgfqpoint{1.117466in}{1.645213in}}{\pgfqpoint{1.125702in}{1.645213in}}%
\pgfpathclose%
\pgfusepath{stroke,fill}%
\end{pgfscope}%
\begin{pgfscope}%
\pgfpathrectangle{\pgfqpoint{0.100000in}{0.212622in}}{\pgfqpoint{3.696000in}{3.696000in}}%
\pgfusepath{clip}%
\pgfsetbuttcap%
\pgfsetroundjoin%
\definecolor{currentfill}{rgb}{0.121569,0.466667,0.705882}%
\pgfsetfillcolor{currentfill}%
\pgfsetfillopacity{0.508817}%
\pgfsetlinewidth{1.003750pt}%
\definecolor{currentstroke}{rgb}{0.121569,0.466667,0.705882}%
\pgfsetstrokecolor{currentstroke}%
\pgfsetstrokeopacity{0.508817}%
\pgfsetdash{}{0pt}%
\pgfpathmoveto{\pgfqpoint{2.791109in}{2.288496in}}%
\pgfpathcurveto{\pgfqpoint{2.799346in}{2.288496in}}{\pgfqpoint{2.807246in}{2.291768in}}{\pgfqpoint{2.813070in}{2.297592in}}%
\pgfpathcurveto{\pgfqpoint{2.818894in}{2.303416in}}{\pgfqpoint{2.822166in}{2.311316in}}{\pgfqpoint{2.822166in}{2.319552in}}%
\pgfpathcurveto{\pgfqpoint{2.822166in}{2.327789in}}{\pgfqpoint{2.818894in}{2.335689in}}{\pgfqpoint{2.813070in}{2.341513in}}%
\pgfpathcurveto{\pgfqpoint{2.807246in}{2.347337in}}{\pgfqpoint{2.799346in}{2.350609in}}{\pgfqpoint{2.791109in}{2.350609in}}%
\pgfpathcurveto{\pgfqpoint{2.782873in}{2.350609in}}{\pgfqpoint{2.774973in}{2.347337in}}{\pgfqpoint{2.769149in}{2.341513in}}%
\pgfpathcurveto{\pgfqpoint{2.763325in}{2.335689in}}{\pgfqpoint{2.760053in}{2.327789in}}{\pgfqpoint{2.760053in}{2.319552in}}%
\pgfpathcurveto{\pgfqpoint{2.760053in}{2.311316in}}{\pgfqpoint{2.763325in}{2.303416in}}{\pgfqpoint{2.769149in}{2.297592in}}%
\pgfpathcurveto{\pgfqpoint{2.774973in}{2.291768in}}{\pgfqpoint{2.782873in}{2.288496in}}{\pgfqpoint{2.791109in}{2.288496in}}%
\pgfpathclose%
\pgfusepath{stroke,fill}%
\end{pgfscope}%
\begin{pgfscope}%
\pgfpathrectangle{\pgfqpoint{0.100000in}{0.212622in}}{\pgfqpoint{3.696000in}{3.696000in}}%
\pgfusepath{clip}%
\pgfsetbuttcap%
\pgfsetroundjoin%
\definecolor{currentfill}{rgb}{0.121569,0.466667,0.705882}%
\pgfsetfillcolor{currentfill}%
\pgfsetfillopacity{0.509986}%
\pgfsetlinewidth{1.003750pt}%
\definecolor{currentstroke}{rgb}{0.121569,0.466667,0.705882}%
\pgfsetstrokecolor{currentstroke}%
\pgfsetstrokeopacity{0.509986}%
\pgfsetdash{}{0pt}%
\pgfpathmoveto{\pgfqpoint{2.797493in}{2.286843in}}%
\pgfpathcurveto{\pgfqpoint{2.805729in}{2.286843in}}{\pgfqpoint{2.813629in}{2.290115in}}{\pgfqpoint{2.819453in}{2.295939in}}%
\pgfpathcurveto{\pgfqpoint{2.825277in}{2.301763in}}{\pgfqpoint{2.828549in}{2.309663in}}{\pgfqpoint{2.828549in}{2.317899in}}%
\pgfpathcurveto{\pgfqpoint{2.828549in}{2.326136in}}{\pgfqpoint{2.825277in}{2.334036in}}{\pgfqpoint{2.819453in}{2.339860in}}%
\pgfpathcurveto{\pgfqpoint{2.813629in}{2.345684in}}{\pgfqpoint{2.805729in}{2.348956in}}{\pgfqpoint{2.797493in}{2.348956in}}%
\pgfpathcurveto{\pgfqpoint{2.789257in}{2.348956in}}{\pgfqpoint{2.781356in}{2.345684in}}{\pgfqpoint{2.775533in}{2.339860in}}%
\pgfpathcurveto{\pgfqpoint{2.769709in}{2.334036in}}{\pgfqpoint{2.766436in}{2.326136in}}{\pgfqpoint{2.766436in}{2.317899in}}%
\pgfpathcurveto{\pgfqpoint{2.766436in}{2.309663in}}{\pgfqpoint{2.769709in}{2.301763in}}{\pgfqpoint{2.775533in}{2.295939in}}%
\pgfpathcurveto{\pgfqpoint{2.781356in}{2.290115in}}{\pgfqpoint{2.789257in}{2.286843in}}{\pgfqpoint{2.797493in}{2.286843in}}%
\pgfpathclose%
\pgfusepath{stroke,fill}%
\end{pgfscope}%
\begin{pgfscope}%
\pgfpathrectangle{\pgfqpoint{0.100000in}{0.212622in}}{\pgfqpoint{3.696000in}{3.696000in}}%
\pgfusepath{clip}%
\pgfsetbuttcap%
\pgfsetroundjoin%
\definecolor{currentfill}{rgb}{0.121569,0.466667,0.705882}%
\pgfsetfillcolor{currentfill}%
\pgfsetfillopacity{0.510691}%
\pgfsetlinewidth{1.003750pt}%
\definecolor{currentstroke}{rgb}{0.121569,0.466667,0.705882}%
\pgfsetstrokecolor{currentstroke}%
\pgfsetstrokeopacity{0.510691}%
\pgfsetdash{}{0pt}%
\pgfpathmoveto{\pgfqpoint{1.117823in}{1.630539in}}%
\pgfpathcurveto{\pgfqpoint{1.126059in}{1.630539in}}{\pgfqpoint{1.133959in}{1.633811in}}{\pgfqpoint{1.139783in}{1.639635in}}%
\pgfpathcurveto{\pgfqpoint{1.145607in}{1.645459in}}{\pgfqpoint{1.148879in}{1.653359in}}{\pgfqpoint{1.148879in}{1.661595in}}%
\pgfpathcurveto{\pgfqpoint{1.148879in}{1.669832in}}{\pgfqpoint{1.145607in}{1.677732in}}{\pgfqpoint{1.139783in}{1.683556in}}%
\pgfpathcurveto{\pgfqpoint{1.133959in}{1.689379in}}{\pgfqpoint{1.126059in}{1.692652in}}{\pgfqpoint{1.117823in}{1.692652in}}%
\pgfpathcurveto{\pgfqpoint{1.109586in}{1.692652in}}{\pgfqpoint{1.101686in}{1.689379in}}{\pgfqpoint{1.095862in}{1.683556in}}%
\pgfpathcurveto{\pgfqpoint{1.090038in}{1.677732in}}{\pgfqpoint{1.086766in}{1.669832in}}{\pgfqpoint{1.086766in}{1.661595in}}%
\pgfpathcurveto{\pgfqpoint{1.086766in}{1.653359in}}{\pgfqpoint{1.090038in}{1.645459in}}{\pgfqpoint{1.095862in}{1.639635in}}%
\pgfpathcurveto{\pgfqpoint{1.101686in}{1.633811in}}{\pgfqpoint{1.109586in}{1.630539in}}{\pgfqpoint{1.117823in}{1.630539in}}%
\pgfpathclose%
\pgfusepath{stroke,fill}%
\end{pgfscope}%
\begin{pgfscope}%
\pgfpathrectangle{\pgfqpoint{0.100000in}{0.212622in}}{\pgfqpoint{3.696000in}{3.696000in}}%
\pgfusepath{clip}%
\pgfsetbuttcap%
\pgfsetroundjoin%
\definecolor{currentfill}{rgb}{0.121569,0.466667,0.705882}%
\pgfsetfillcolor{currentfill}%
\pgfsetfillopacity{0.510730}%
\pgfsetlinewidth{1.003750pt}%
\definecolor{currentstroke}{rgb}{0.121569,0.466667,0.705882}%
\pgfsetstrokecolor{currentstroke}%
\pgfsetstrokeopacity{0.510730}%
\pgfsetdash{}{0pt}%
\pgfpathmoveto{\pgfqpoint{2.801364in}{2.287669in}}%
\pgfpathcurveto{\pgfqpoint{2.809601in}{2.287669in}}{\pgfqpoint{2.817501in}{2.290941in}}{\pgfqpoint{2.823325in}{2.296765in}}%
\pgfpathcurveto{\pgfqpoint{2.829149in}{2.302589in}}{\pgfqpoint{2.832421in}{2.310489in}}{\pgfqpoint{2.832421in}{2.318725in}}%
\pgfpathcurveto{\pgfqpoint{2.832421in}{2.326962in}}{\pgfqpoint{2.829149in}{2.334862in}}{\pgfqpoint{2.823325in}{2.340686in}}%
\pgfpathcurveto{\pgfqpoint{2.817501in}{2.346510in}}{\pgfqpoint{2.809601in}{2.349782in}}{\pgfqpoint{2.801364in}{2.349782in}}%
\pgfpathcurveto{\pgfqpoint{2.793128in}{2.349782in}}{\pgfqpoint{2.785228in}{2.346510in}}{\pgfqpoint{2.779404in}{2.340686in}}%
\pgfpathcurveto{\pgfqpoint{2.773580in}{2.334862in}}{\pgfqpoint{2.770308in}{2.326962in}}{\pgfqpoint{2.770308in}{2.318725in}}%
\pgfpathcurveto{\pgfqpoint{2.770308in}{2.310489in}}{\pgfqpoint{2.773580in}{2.302589in}}{\pgfqpoint{2.779404in}{2.296765in}}%
\pgfpathcurveto{\pgfqpoint{2.785228in}{2.290941in}}{\pgfqpoint{2.793128in}{2.287669in}}{\pgfqpoint{2.801364in}{2.287669in}}%
\pgfpathclose%
\pgfusepath{stroke,fill}%
\end{pgfscope}%
\begin{pgfscope}%
\pgfpathrectangle{\pgfqpoint{0.100000in}{0.212622in}}{\pgfqpoint{3.696000in}{3.696000in}}%
\pgfusepath{clip}%
\pgfsetbuttcap%
\pgfsetroundjoin%
\definecolor{currentfill}{rgb}{0.121569,0.466667,0.705882}%
\pgfsetfillcolor{currentfill}%
\pgfsetfillopacity{0.511115}%
\pgfsetlinewidth{1.003750pt}%
\definecolor{currentstroke}{rgb}{0.121569,0.466667,0.705882}%
\pgfsetstrokecolor{currentstroke}%
\pgfsetstrokeopacity{0.511115}%
\pgfsetdash{}{0pt}%
\pgfpathmoveto{\pgfqpoint{2.803199in}{2.287122in}}%
\pgfpathcurveto{\pgfqpoint{2.811435in}{2.287122in}}{\pgfqpoint{2.819335in}{2.290395in}}{\pgfqpoint{2.825159in}{2.296219in}}%
\pgfpathcurveto{\pgfqpoint{2.830983in}{2.302042in}}{\pgfqpoint{2.834255in}{2.309943in}}{\pgfqpoint{2.834255in}{2.318179in}}%
\pgfpathcurveto{\pgfqpoint{2.834255in}{2.326415in}}{\pgfqpoint{2.830983in}{2.334315in}}{\pgfqpoint{2.825159in}{2.340139in}}%
\pgfpathcurveto{\pgfqpoint{2.819335in}{2.345963in}}{\pgfqpoint{2.811435in}{2.349235in}}{\pgfqpoint{2.803199in}{2.349235in}}%
\pgfpathcurveto{\pgfqpoint{2.794963in}{2.349235in}}{\pgfqpoint{2.787062in}{2.345963in}}{\pgfqpoint{2.781239in}{2.340139in}}%
\pgfpathcurveto{\pgfqpoint{2.775415in}{2.334315in}}{\pgfqpoint{2.772142in}{2.326415in}}{\pgfqpoint{2.772142in}{2.318179in}}%
\pgfpathcurveto{\pgfqpoint{2.772142in}{2.309943in}}{\pgfqpoint{2.775415in}{2.302042in}}{\pgfqpoint{2.781239in}{2.296219in}}%
\pgfpathcurveto{\pgfqpoint{2.787062in}{2.290395in}}{\pgfqpoint{2.794963in}{2.287122in}}{\pgfqpoint{2.803199in}{2.287122in}}%
\pgfpathclose%
\pgfusepath{stroke,fill}%
\end{pgfscope}%
\begin{pgfscope}%
\pgfpathrectangle{\pgfqpoint{0.100000in}{0.212622in}}{\pgfqpoint{3.696000in}{3.696000in}}%
\pgfusepath{clip}%
\pgfsetbuttcap%
\pgfsetroundjoin%
\definecolor{currentfill}{rgb}{0.121569,0.466667,0.705882}%
\pgfsetfillcolor{currentfill}%
\pgfsetfillopacity{0.512315}%
\pgfsetlinewidth{1.003750pt}%
\definecolor{currentstroke}{rgb}{0.121569,0.466667,0.705882}%
\pgfsetstrokecolor{currentstroke}%
\pgfsetstrokeopacity{0.512315}%
\pgfsetdash{}{0pt}%
\pgfpathmoveto{\pgfqpoint{2.810371in}{2.287281in}}%
\pgfpathcurveto{\pgfqpoint{2.818607in}{2.287281in}}{\pgfqpoint{2.826507in}{2.290553in}}{\pgfqpoint{2.832331in}{2.296377in}}%
\pgfpathcurveto{\pgfqpoint{2.838155in}{2.302201in}}{\pgfqpoint{2.841427in}{2.310101in}}{\pgfqpoint{2.841427in}{2.318337in}}%
\pgfpathcurveto{\pgfqpoint{2.841427in}{2.326573in}}{\pgfqpoint{2.838155in}{2.334473in}}{\pgfqpoint{2.832331in}{2.340297in}}%
\pgfpathcurveto{\pgfqpoint{2.826507in}{2.346121in}}{\pgfqpoint{2.818607in}{2.349394in}}{\pgfqpoint{2.810371in}{2.349394in}}%
\pgfpathcurveto{\pgfqpoint{2.802135in}{2.349394in}}{\pgfqpoint{2.794234in}{2.346121in}}{\pgfqpoint{2.788411in}{2.340297in}}%
\pgfpathcurveto{\pgfqpoint{2.782587in}{2.334473in}}{\pgfqpoint{2.779314in}{2.326573in}}{\pgfqpoint{2.779314in}{2.318337in}}%
\pgfpathcurveto{\pgfqpoint{2.779314in}{2.310101in}}{\pgfqpoint{2.782587in}{2.302201in}}{\pgfqpoint{2.788411in}{2.296377in}}%
\pgfpathcurveto{\pgfqpoint{2.794234in}{2.290553in}}{\pgfqpoint{2.802135in}{2.287281in}}{\pgfqpoint{2.810371in}{2.287281in}}%
\pgfpathclose%
\pgfusepath{stroke,fill}%
\end{pgfscope}%
\begin{pgfscope}%
\pgfpathrectangle{\pgfqpoint{0.100000in}{0.212622in}}{\pgfqpoint{3.696000in}{3.696000in}}%
\pgfusepath{clip}%
\pgfsetbuttcap%
\pgfsetroundjoin%
\definecolor{currentfill}{rgb}{0.121569,0.466667,0.705882}%
\pgfsetfillcolor{currentfill}%
\pgfsetfillopacity{0.514858}%
\pgfsetlinewidth{1.003750pt}%
\definecolor{currentstroke}{rgb}{0.121569,0.466667,0.705882}%
\pgfsetstrokecolor{currentstroke}%
\pgfsetstrokeopacity{0.514858}%
\pgfsetdash{}{0pt}%
\pgfpathmoveto{\pgfqpoint{2.823270in}{2.287602in}}%
\pgfpathcurveto{\pgfqpoint{2.831507in}{2.287602in}}{\pgfqpoint{2.839407in}{2.290874in}}{\pgfqpoint{2.845231in}{2.296698in}}%
\pgfpathcurveto{\pgfqpoint{2.851055in}{2.302522in}}{\pgfqpoint{2.854327in}{2.310422in}}{\pgfqpoint{2.854327in}{2.318659in}}%
\pgfpathcurveto{\pgfqpoint{2.854327in}{2.326895in}}{\pgfqpoint{2.851055in}{2.334795in}}{\pgfqpoint{2.845231in}{2.340619in}}%
\pgfpathcurveto{\pgfqpoint{2.839407in}{2.346443in}}{\pgfqpoint{2.831507in}{2.349715in}}{\pgfqpoint{2.823270in}{2.349715in}}%
\pgfpathcurveto{\pgfqpoint{2.815034in}{2.349715in}}{\pgfqpoint{2.807134in}{2.346443in}}{\pgfqpoint{2.801310in}{2.340619in}}%
\pgfpathcurveto{\pgfqpoint{2.795486in}{2.334795in}}{\pgfqpoint{2.792214in}{2.326895in}}{\pgfqpoint{2.792214in}{2.318659in}}%
\pgfpathcurveto{\pgfqpoint{2.792214in}{2.310422in}}{\pgfqpoint{2.795486in}{2.302522in}}{\pgfqpoint{2.801310in}{2.296698in}}%
\pgfpathcurveto{\pgfqpoint{2.807134in}{2.290874in}}{\pgfqpoint{2.815034in}{2.287602in}}{\pgfqpoint{2.823270in}{2.287602in}}%
\pgfpathclose%
\pgfusepath{stroke,fill}%
\end{pgfscope}%
\begin{pgfscope}%
\pgfpathrectangle{\pgfqpoint{0.100000in}{0.212622in}}{\pgfqpoint{3.696000in}{3.696000in}}%
\pgfusepath{clip}%
\pgfsetbuttcap%
\pgfsetroundjoin%
\definecolor{currentfill}{rgb}{0.121569,0.466667,0.705882}%
\pgfsetfillcolor{currentfill}%
\pgfsetfillopacity{0.517003}%
\pgfsetlinewidth{1.003750pt}%
\definecolor{currentstroke}{rgb}{0.121569,0.466667,0.705882}%
\pgfsetstrokecolor{currentstroke}%
\pgfsetstrokeopacity{0.517003}%
\pgfsetdash{}{0pt}%
\pgfpathmoveto{\pgfqpoint{1.099420in}{1.612554in}}%
\pgfpathcurveto{\pgfqpoint{1.107657in}{1.612554in}}{\pgfqpoint{1.115557in}{1.615826in}}{\pgfqpoint{1.121381in}{1.621650in}}%
\pgfpathcurveto{\pgfqpoint{1.127205in}{1.627474in}}{\pgfqpoint{1.130477in}{1.635374in}}{\pgfqpoint{1.130477in}{1.643610in}}%
\pgfpathcurveto{\pgfqpoint{1.130477in}{1.651847in}}{\pgfqpoint{1.127205in}{1.659747in}}{\pgfqpoint{1.121381in}{1.665571in}}%
\pgfpathcurveto{\pgfqpoint{1.115557in}{1.671394in}}{\pgfqpoint{1.107657in}{1.674667in}}{\pgfqpoint{1.099420in}{1.674667in}}%
\pgfpathcurveto{\pgfqpoint{1.091184in}{1.674667in}}{\pgfqpoint{1.083284in}{1.671394in}}{\pgfqpoint{1.077460in}{1.665571in}}%
\pgfpathcurveto{\pgfqpoint{1.071636in}{1.659747in}}{\pgfqpoint{1.068364in}{1.651847in}}{\pgfqpoint{1.068364in}{1.643610in}}%
\pgfpathcurveto{\pgfqpoint{1.068364in}{1.635374in}}{\pgfqpoint{1.071636in}{1.627474in}}{\pgfqpoint{1.077460in}{1.621650in}}%
\pgfpathcurveto{\pgfqpoint{1.083284in}{1.615826in}}{\pgfqpoint{1.091184in}{1.612554in}}{\pgfqpoint{1.099420in}{1.612554in}}%
\pgfpathclose%
\pgfusepath{stroke,fill}%
\end{pgfscope}%
\begin{pgfscope}%
\pgfpathrectangle{\pgfqpoint{0.100000in}{0.212622in}}{\pgfqpoint{3.696000in}{3.696000in}}%
\pgfusepath{clip}%
\pgfsetbuttcap%
\pgfsetroundjoin%
\definecolor{currentfill}{rgb}{0.121569,0.466667,0.705882}%
\pgfsetfillcolor{currentfill}%
\pgfsetfillopacity{0.518163}%
\pgfsetlinewidth{1.003750pt}%
\definecolor{currentstroke}{rgb}{0.121569,0.466667,0.705882}%
\pgfsetstrokecolor{currentstroke}%
\pgfsetstrokeopacity{0.518163}%
\pgfsetdash{}{0pt}%
\pgfpathmoveto{\pgfqpoint{2.842238in}{2.283909in}}%
\pgfpathcurveto{\pgfqpoint{2.850475in}{2.283909in}}{\pgfqpoint{2.858375in}{2.287181in}}{\pgfqpoint{2.864199in}{2.293005in}}%
\pgfpathcurveto{\pgfqpoint{2.870022in}{2.298829in}}{\pgfqpoint{2.873295in}{2.306729in}}{\pgfqpoint{2.873295in}{2.314966in}}%
\pgfpathcurveto{\pgfqpoint{2.873295in}{2.323202in}}{\pgfqpoint{2.870022in}{2.331102in}}{\pgfqpoint{2.864199in}{2.336926in}}%
\pgfpathcurveto{\pgfqpoint{2.858375in}{2.342750in}}{\pgfqpoint{2.850475in}{2.346022in}}{\pgfqpoint{2.842238in}{2.346022in}}%
\pgfpathcurveto{\pgfqpoint{2.834002in}{2.346022in}}{\pgfqpoint{2.826102in}{2.342750in}}{\pgfqpoint{2.820278in}{2.336926in}}%
\pgfpathcurveto{\pgfqpoint{2.814454in}{2.331102in}}{\pgfqpoint{2.811182in}{2.323202in}}{\pgfqpoint{2.811182in}{2.314966in}}%
\pgfpathcurveto{\pgfqpoint{2.811182in}{2.306729in}}{\pgfqpoint{2.814454in}{2.298829in}}{\pgfqpoint{2.820278in}{2.293005in}}%
\pgfpathcurveto{\pgfqpoint{2.826102in}{2.287181in}}{\pgfqpoint{2.834002in}{2.283909in}}{\pgfqpoint{2.842238in}{2.283909in}}%
\pgfpathclose%
\pgfusepath{stroke,fill}%
\end{pgfscope}%
\begin{pgfscope}%
\pgfpathrectangle{\pgfqpoint{0.100000in}{0.212622in}}{\pgfqpoint{3.696000in}{3.696000in}}%
\pgfusepath{clip}%
\pgfsetbuttcap%
\pgfsetroundjoin%
\definecolor{currentfill}{rgb}{0.121569,0.466667,0.705882}%
\pgfsetfillcolor{currentfill}%
\pgfsetfillopacity{0.522807}%
\pgfsetlinewidth{1.003750pt}%
\definecolor{currentstroke}{rgb}{0.121569,0.466667,0.705882}%
\pgfsetstrokecolor{currentstroke}%
\pgfsetstrokeopacity{0.522807}%
\pgfsetdash{}{0pt}%
\pgfpathmoveto{\pgfqpoint{2.864753in}{2.286571in}}%
\pgfpathcurveto{\pgfqpoint{2.872989in}{2.286571in}}{\pgfqpoint{2.880890in}{2.289843in}}{\pgfqpoint{2.886713in}{2.295667in}}%
\pgfpathcurveto{\pgfqpoint{2.892537in}{2.301491in}}{\pgfqpoint{2.895810in}{2.309391in}}{\pgfqpoint{2.895810in}{2.317627in}}%
\pgfpathcurveto{\pgfqpoint{2.895810in}{2.325863in}}{\pgfqpoint{2.892537in}{2.333763in}}{\pgfqpoint{2.886713in}{2.339587in}}%
\pgfpathcurveto{\pgfqpoint{2.880890in}{2.345411in}}{\pgfqpoint{2.872989in}{2.348684in}}{\pgfqpoint{2.864753in}{2.348684in}}%
\pgfpathcurveto{\pgfqpoint{2.856517in}{2.348684in}}{\pgfqpoint{2.848617in}{2.345411in}}{\pgfqpoint{2.842793in}{2.339587in}}%
\pgfpathcurveto{\pgfqpoint{2.836969in}{2.333763in}}{\pgfqpoint{2.833697in}{2.325863in}}{\pgfqpoint{2.833697in}{2.317627in}}%
\pgfpathcurveto{\pgfqpoint{2.833697in}{2.309391in}}{\pgfqpoint{2.836969in}{2.301491in}}{\pgfqpoint{2.842793in}{2.295667in}}%
\pgfpathcurveto{\pgfqpoint{2.848617in}{2.289843in}}{\pgfqpoint{2.856517in}{2.286571in}}{\pgfqpoint{2.864753in}{2.286571in}}%
\pgfpathclose%
\pgfusepath{stroke,fill}%
\end{pgfscope}%
\begin{pgfscope}%
\pgfpathrectangle{\pgfqpoint{0.100000in}{0.212622in}}{\pgfqpoint{3.696000in}{3.696000in}}%
\pgfusepath{clip}%
\pgfsetbuttcap%
\pgfsetroundjoin%
\definecolor{currentfill}{rgb}{0.121569,0.466667,0.705882}%
\pgfsetfillcolor{currentfill}%
\pgfsetfillopacity{0.525265}%
\pgfsetlinewidth{1.003750pt}%
\definecolor{currentstroke}{rgb}{0.121569,0.466667,0.705882}%
\pgfsetstrokecolor{currentstroke}%
\pgfsetstrokeopacity{0.525265}%
\pgfsetdash{}{0pt}%
\pgfpathmoveto{\pgfqpoint{2.876255in}{2.284904in}}%
\pgfpathcurveto{\pgfqpoint{2.884492in}{2.284904in}}{\pgfqpoint{2.892392in}{2.288176in}}{\pgfqpoint{2.898216in}{2.294000in}}%
\pgfpathcurveto{\pgfqpoint{2.904040in}{2.299824in}}{\pgfqpoint{2.907312in}{2.307724in}}{\pgfqpoint{2.907312in}{2.315960in}}%
\pgfpathcurveto{\pgfqpoint{2.907312in}{2.324197in}}{\pgfqpoint{2.904040in}{2.332097in}}{\pgfqpoint{2.898216in}{2.337921in}}%
\pgfpathcurveto{\pgfqpoint{2.892392in}{2.343745in}}{\pgfqpoint{2.884492in}{2.347017in}}{\pgfqpoint{2.876255in}{2.347017in}}%
\pgfpathcurveto{\pgfqpoint{2.868019in}{2.347017in}}{\pgfqpoint{2.860119in}{2.343745in}}{\pgfqpoint{2.854295in}{2.337921in}}%
\pgfpathcurveto{\pgfqpoint{2.848471in}{2.332097in}}{\pgfqpoint{2.845199in}{2.324197in}}{\pgfqpoint{2.845199in}{2.315960in}}%
\pgfpathcurveto{\pgfqpoint{2.845199in}{2.307724in}}{\pgfqpoint{2.848471in}{2.299824in}}{\pgfqpoint{2.854295in}{2.294000in}}%
\pgfpathcurveto{\pgfqpoint{2.860119in}{2.288176in}}{\pgfqpoint{2.868019in}{2.284904in}}{\pgfqpoint{2.876255in}{2.284904in}}%
\pgfpathclose%
\pgfusepath{stroke,fill}%
\end{pgfscope}%
\begin{pgfscope}%
\pgfpathrectangle{\pgfqpoint{0.100000in}{0.212622in}}{\pgfqpoint{3.696000in}{3.696000in}}%
\pgfusepath{clip}%
\pgfsetbuttcap%
\pgfsetroundjoin%
\definecolor{currentfill}{rgb}{0.121569,0.466667,0.705882}%
\pgfsetfillcolor{currentfill}%
\pgfsetfillopacity{0.527389}%
\pgfsetlinewidth{1.003750pt}%
\definecolor{currentstroke}{rgb}{0.121569,0.466667,0.705882}%
\pgfsetstrokecolor{currentstroke}%
\pgfsetstrokeopacity{0.527389}%
\pgfsetdash{}{0pt}%
\pgfpathmoveto{\pgfqpoint{1.072678in}{1.566074in}}%
\pgfpathcurveto{\pgfqpoint{1.080914in}{1.566074in}}{\pgfqpoint{1.088814in}{1.569346in}}{\pgfqpoint{1.094638in}{1.575170in}}%
\pgfpathcurveto{\pgfqpoint{1.100462in}{1.580994in}}{\pgfqpoint{1.103735in}{1.588894in}}{\pgfqpoint{1.103735in}{1.597130in}}%
\pgfpathcurveto{\pgfqpoint{1.103735in}{1.605367in}}{\pgfqpoint{1.100462in}{1.613267in}}{\pgfqpoint{1.094638in}{1.619091in}}%
\pgfpathcurveto{\pgfqpoint{1.088814in}{1.624915in}}{\pgfqpoint{1.080914in}{1.628187in}}{\pgfqpoint{1.072678in}{1.628187in}}%
\pgfpathcurveto{\pgfqpoint{1.064442in}{1.628187in}}{\pgfqpoint{1.056542in}{1.624915in}}{\pgfqpoint{1.050718in}{1.619091in}}%
\pgfpathcurveto{\pgfqpoint{1.044894in}{1.613267in}}{\pgfqpoint{1.041622in}{1.605367in}}{\pgfqpoint{1.041622in}{1.597130in}}%
\pgfpathcurveto{\pgfqpoint{1.041622in}{1.588894in}}{\pgfqpoint{1.044894in}{1.580994in}}{\pgfqpoint{1.050718in}{1.575170in}}%
\pgfpathcurveto{\pgfqpoint{1.056542in}{1.569346in}}{\pgfqpoint{1.064442in}{1.566074in}}{\pgfqpoint{1.072678in}{1.566074in}}%
\pgfpathclose%
\pgfusepath{stroke,fill}%
\end{pgfscope}%
\begin{pgfscope}%
\pgfpathrectangle{\pgfqpoint{0.100000in}{0.212622in}}{\pgfqpoint{3.696000in}{3.696000in}}%
\pgfusepath{clip}%
\pgfsetbuttcap%
\pgfsetroundjoin%
\definecolor{currentfill}{rgb}{0.121569,0.466667,0.705882}%
\pgfsetfillcolor{currentfill}%
\pgfsetfillopacity{0.528663}%
\pgfsetlinewidth{1.003750pt}%
\definecolor{currentstroke}{rgb}{0.121569,0.466667,0.705882}%
\pgfsetstrokecolor{currentstroke}%
\pgfsetstrokeopacity{0.528663}%
\pgfsetdash{}{0pt}%
\pgfpathmoveto{\pgfqpoint{2.893089in}{2.288616in}}%
\pgfpathcurveto{\pgfqpoint{2.901326in}{2.288616in}}{\pgfqpoint{2.909226in}{2.291889in}}{\pgfqpoint{2.915050in}{2.297713in}}%
\pgfpathcurveto{\pgfqpoint{2.920873in}{2.303536in}}{\pgfqpoint{2.924146in}{2.311437in}}{\pgfqpoint{2.924146in}{2.319673in}}%
\pgfpathcurveto{\pgfqpoint{2.924146in}{2.327909in}}{\pgfqpoint{2.920873in}{2.335809in}}{\pgfqpoint{2.915050in}{2.341633in}}%
\pgfpathcurveto{\pgfqpoint{2.909226in}{2.347457in}}{\pgfqpoint{2.901326in}{2.350729in}}{\pgfqpoint{2.893089in}{2.350729in}}%
\pgfpathcurveto{\pgfqpoint{2.884853in}{2.350729in}}{\pgfqpoint{2.876953in}{2.347457in}}{\pgfqpoint{2.871129in}{2.341633in}}%
\pgfpathcurveto{\pgfqpoint{2.865305in}{2.335809in}}{\pgfqpoint{2.862033in}{2.327909in}}{\pgfqpoint{2.862033in}{2.319673in}}%
\pgfpathcurveto{\pgfqpoint{2.862033in}{2.311437in}}{\pgfqpoint{2.865305in}{2.303536in}}{\pgfqpoint{2.871129in}{2.297713in}}%
\pgfpathcurveto{\pgfqpoint{2.876953in}{2.291889in}}{\pgfqpoint{2.884853in}{2.288616in}}{\pgfqpoint{2.893089in}{2.288616in}}%
\pgfpathclose%
\pgfusepath{stroke,fill}%
\end{pgfscope}%
\begin{pgfscope}%
\pgfpathrectangle{\pgfqpoint{0.100000in}{0.212622in}}{\pgfqpoint{3.696000in}{3.696000in}}%
\pgfusepath{clip}%
\pgfsetbuttcap%
\pgfsetroundjoin%
\definecolor{currentfill}{rgb}{0.121569,0.466667,0.705882}%
\pgfsetfillcolor{currentfill}%
\pgfsetfillopacity{0.530232}%
\pgfsetlinewidth{1.003750pt}%
\definecolor{currentstroke}{rgb}{0.121569,0.466667,0.705882}%
\pgfsetstrokecolor{currentstroke}%
\pgfsetstrokeopacity{0.530232}%
\pgfsetdash{}{0pt}%
\pgfpathmoveto{\pgfqpoint{2.901661in}{2.286675in}}%
\pgfpathcurveto{\pgfqpoint{2.909897in}{2.286675in}}{\pgfqpoint{2.917797in}{2.289948in}}{\pgfqpoint{2.923621in}{2.295772in}}%
\pgfpathcurveto{\pgfqpoint{2.929445in}{2.301595in}}{\pgfqpoint{2.932718in}{2.309496in}}{\pgfqpoint{2.932718in}{2.317732in}}%
\pgfpathcurveto{\pgfqpoint{2.932718in}{2.325968in}}{\pgfqpoint{2.929445in}{2.333868in}}{\pgfqpoint{2.923621in}{2.339692in}}%
\pgfpathcurveto{\pgfqpoint{2.917797in}{2.345516in}}{\pgfqpoint{2.909897in}{2.348788in}}{\pgfqpoint{2.901661in}{2.348788in}}%
\pgfpathcurveto{\pgfqpoint{2.893425in}{2.348788in}}{\pgfqpoint{2.885525in}{2.345516in}}{\pgfqpoint{2.879701in}{2.339692in}}%
\pgfpathcurveto{\pgfqpoint{2.873877in}{2.333868in}}{\pgfqpoint{2.870605in}{2.325968in}}{\pgfqpoint{2.870605in}{2.317732in}}%
\pgfpathcurveto{\pgfqpoint{2.870605in}{2.309496in}}{\pgfqpoint{2.873877in}{2.301595in}}{\pgfqpoint{2.879701in}{2.295772in}}%
\pgfpathcurveto{\pgfqpoint{2.885525in}{2.289948in}}{\pgfqpoint{2.893425in}{2.286675in}}{\pgfqpoint{2.901661in}{2.286675in}}%
\pgfpathclose%
\pgfusepath{stroke,fill}%
\end{pgfscope}%
\begin{pgfscope}%
\pgfpathrectangle{\pgfqpoint{0.100000in}{0.212622in}}{\pgfqpoint{3.696000in}{3.696000in}}%
\pgfusepath{clip}%
\pgfsetbuttcap%
\pgfsetroundjoin%
\definecolor{currentfill}{rgb}{0.121569,0.466667,0.705882}%
\pgfsetfillcolor{currentfill}%
\pgfsetfillopacity{0.532852}%
\pgfsetlinewidth{1.003750pt}%
\definecolor{currentstroke}{rgb}{0.121569,0.466667,0.705882}%
\pgfsetstrokecolor{currentstroke}%
\pgfsetstrokeopacity{0.532852}%
\pgfsetdash{}{0pt}%
\pgfpathmoveto{\pgfqpoint{2.915327in}{2.289548in}}%
\pgfpathcurveto{\pgfqpoint{2.923564in}{2.289548in}}{\pgfqpoint{2.931464in}{2.292820in}}{\pgfqpoint{2.937288in}{2.298644in}}%
\pgfpathcurveto{\pgfqpoint{2.943112in}{2.304468in}}{\pgfqpoint{2.946384in}{2.312368in}}{\pgfqpoint{2.946384in}{2.320604in}}%
\pgfpathcurveto{\pgfqpoint{2.946384in}{2.328841in}}{\pgfqpoint{2.943112in}{2.336741in}}{\pgfqpoint{2.937288in}{2.342565in}}%
\pgfpathcurveto{\pgfqpoint{2.931464in}{2.348388in}}{\pgfqpoint{2.923564in}{2.351661in}}{\pgfqpoint{2.915327in}{2.351661in}}%
\pgfpathcurveto{\pgfqpoint{2.907091in}{2.351661in}}{\pgfqpoint{2.899191in}{2.348388in}}{\pgfqpoint{2.893367in}{2.342565in}}%
\pgfpathcurveto{\pgfqpoint{2.887543in}{2.336741in}}{\pgfqpoint{2.884271in}{2.328841in}}{\pgfqpoint{2.884271in}{2.320604in}}%
\pgfpathcurveto{\pgfqpoint{2.884271in}{2.312368in}}{\pgfqpoint{2.887543in}{2.304468in}}{\pgfqpoint{2.893367in}{2.298644in}}%
\pgfpathcurveto{\pgfqpoint{2.899191in}{2.292820in}}{\pgfqpoint{2.907091in}{2.289548in}}{\pgfqpoint{2.915327in}{2.289548in}}%
\pgfpathclose%
\pgfusepath{stroke,fill}%
\end{pgfscope}%
\begin{pgfscope}%
\pgfpathrectangle{\pgfqpoint{0.100000in}{0.212622in}}{\pgfqpoint{3.696000in}{3.696000in}}%
\pgfusepath{clip}%
\pgfsetbuttcap%
\pgfsetroundjoin%
\definecolor{currentfill}{rgb}{0.121569,0.466667,0.705882}%
\pgfsetfillcolor{currentfill}%
\pgfsetfillopacity{0.535932}%
\pgfsetlinewidth{1.003750pt}%
\definecolor{currentstroke}{rgb}{0.121569,0.466667,0.705882}%
\pgfsetstrokecolor{currentstroke}%
\pgfsetstrokeopacity{0.535932}%
\pgfsetdash{}{0pt}%
\pgfpathmoveto{\pgfqpoint{2.930813in}{2.285448in}}%
\pgfpathcurveto{\pgfqpoint{2.939049in}{2.285448in}}{\pgfqpoint{2.946949in}{2.288720in}}{\pgfqpoint{2.952773in}{2.294544in}}%
\pgfpathcurveto{\pgfqpoint{2.958597in}{2.300368in}}{\pgfqpoint{2.961869in}{2.308268in}}{\pgfqpoint{2.961869in}{2.316505in}}%
\pgfpathcurveto{\pgfqpoint{2.961869in}{2.324741in}}{\pgfqpoint{2.958597in}{2.332641in}}{\pgfqpoint{2.952773in}{2.338465in}}%
\pgfpathcurveto{\pgfqpoint{2.946949in}{2.344289in}}{\pgfqpoint{2.939049in}{2.347561in}}{\pgfqpoint{2.930813in}{2.347561in}}%
\pgfpathcurveto{\pgfqpoint{2.922577in}{2.347561in}}{\pgfqpoint{2.914677in}{2.344289in}}{\pgfqpoint{2.908853in}{2.338465in}}%
\pgfpathcurveto{\pgfqpoint{2.903029in}{2.332641in}}{\pgfqpoint{2.899756in}{2.324741in}}{\pgfqpoint{2.899756in}{2.316505in}}%
\pgfpathcurveto{\pgfqpoint{2.899756in}{2.308268in}}{\pgfqpoint{2.903029in}{2.300368in}}{\pgfqpoint{2.908853in}{2.294544in}}%
\pgfpathcurveto{\pgfqpoint{2.914677in}{2.288720in}}{\pgfqpoint{2.922577in}{2.285448in}}{\pgfqpoint{2.930813in}{2.285448in}}%
\pgfpathclose%
\pgfusepath{stroke,fill}%
\end{pgfscope}%
\begin{pgfscope}%
\pgfpathrectangle{\pgfqpoint{0.100000in}{0.212622in}}{\pgfqpoint{3.696000in}{3.696000in}}%
\pgfusepath{clip}%
\pgfsetbuttcap%
\pgfsetroundjoin%
\definecolor{currentfill}{rgb}{0.121569,0.466667,0.705882}%
\pgfsetfillcolor{currentfill}%
\pgfsetfillopacity{0.537569}%
\pgfsetlinewidth{1.003750pt}%
\definecolor{currentstroke}{rgb}{0.121569,0.466667,0.705882}%
\pgfsetstrokecolor{currentstroke}%
\pgfsetstrokeopacity{0.537569}%
\pgfsetdash{}{0pt}%
\pgfpathmoveto{\pgfqpoint{1.050409in}{1.532486in}}%
\pgfpathcurveto{\pgfqpoint{1.058645in}{1.532486in}}{\pgfqpoint{1.066545in}{1.535758in}}{\pgfqpoint{1.072369in}{1.541582in}}%
\pgfpathcurveto{\pgfqpoint{1.078193in}{1.547406in}}{\pgfqpoint{1.081466in}{1.555306in}}{\pgfqpoint{1.081466in}{1.563542in}}%
\pgfpathcurveto{\pgfqpoint{1.081466in}{1.571779in}}{\pgfqpoint{1.078193in}{1.579679in}}{\pgfqpoint{1.072369in}{1.585503in}}%
\pgfpathcurveto{\pgfqpoint{1.066545in}{1.591327in}}{\pgfqpoint{1.058645in}{1.594599in}}{\pgfqpoint{1.050409in}{1.594599in}}%
\pgfpathcurveto{\pgfqpoint{1.042173in}{1.594599in}}{\pgfqpoint{1.034273in}{1.591327in}}{\pgfqpoint{1.028449in}{1.585503in}}%
\pgfpathcurveto{\pgfqpoint{1.022625in}{1.579679in}}{\pgfqpoint{1.019353in}{1.571779in}}{\pgfqpoint{1.019353in}{1.563542in}}%
\pgfpathcurveto{\pgfqpoint{1.019353in}{1.555306in}}{\pgfqpoint{1.022625in}{1.547406in}}{\pgfqpoint{1.028449in}{1.541582in}}%
\pgfpathcurveto{\pgfqpoint{1.034273in}{1.535758in}}{\pgfqpoint{1.042173in}{1.532486in}}{\pgfqpoint{1.050409in}{1.532486in}}%
\pgfpathclose%
\pgfusepath{stroke,fill}%
\end{pgfscope}%
\begin{pgfscope}%
\pgfpathrectangle{\pgfqpoint{0.100000in}{0.212622in}}{\pgfqpoint{3.696000in}{3.696000in}}%
\pgfusepath{clip}%
\pgfsetbuttcap%
\pgfsetroundjoin%
\definecolor{currentfill}{rgb}{0.121569,0.466667,0.705882}%
\pgfsetfillcolor{currentfill}%
\pgfsetfillopacity{0.537606}%
\pgfsetlinewidth{1.003750pt}%
\definecolor{currentstroke}{rgb}{0.121569,0.466667,0.705882}%
\pgfsetstrokecolor{currentstroke}%
\pgfsetstrokeopacity{0.537606}%
\pgfsetdash{}{0pt}%
\pgfpathmoveto{\pgfqpoint{2.939837in}{2.284402in}}%
\pgfpathcurveto{\pgfqpoint{2.948074in}{2.284402in}}{\pgfqpoint{2.955974in}{2.287674in}}{\pgfqpoint{2.961798in}{2.293498in}}%
\pgfpathcurveto{\pgfqpoint{2.967621in}{2.299322in}}{\pgfqpoint{2.970894in}{2.307222in}}{\pgfqpoint{2.970894in}{2.315458in}}%
\pgfpathcurveto{\pgfqpoint{2.970894in}{2.323694in}}{\pgfqpoint{2.967621in}{2.331594in}}{\pgfqpoint{2.961798in}{2.337418in}}%
\pgfpathcurveto{\pgfqpoint{2.955974in}{2.343242in}}{\pgfqpoint{2.948074in}{2.346515in}}{\pgfqpoint{2.939837in}{2.346515in}}%
\pgfpathcurveto{\pgfqpoint{2.931601in}{2.346515in}}{\pgfqpoint{2.923701in}{2.343242in}}{\pgfqpoint{2.917877in}{2.337418in}}%
\pgfpathcurveto{\pgfqpoint{2.912053in}{2.331594in}}{\pgfqpoint{2.908781in}{2.323694in}}{\pgfqpoint{2.908781in}{2.315458in}}%
\pgfpathcurveto{\pgfqpoint{2.908781in}{2.307222in}}{\pgfqpoint{2.912053in}{2.299322in}}{\pgfqpoint{2.917877in}{2.293498in}}%
\pgfpathcurveto{\pgfqpoint{2.923701in}{2.287674in}}{\pgfqpoint{2.931601in}{2.284402in}}{\pgfqpoint{2.939837in}{2.284402in}}%
\pgfpathclose%
\pgfusepath{stroke,fill}%
\end{pgfscope}%
\begin{pgfscope}%
\pgfpathrectangle{\pgfqpoint{0.100000in}{0.212622in}}{\pgfqpoint{3.696000in}{3.696000in}}%
\pgfusepath{clip}%
\pgfsetbuttcap%
\pgfsetroundjoin%
\definecolor{currentfill}{rgb}{0.121569,0.466667,0.705882}%
\pgfsetfillcolor{currentfill}%
\pgfsetfillopacity{0.540260}%
\pgfsetlinewidth{1.003750pt}%
\definecolor{currentstroke}{rgb}{0.121569,0.466667,0.705882}%
\pgfsetstrokecolor{currentstroke}%
\pgfsetstrokeopacity{0.540260}%
\pgfsetdash{}{0pt}%
\pgfpathmoveto{\pgfqpoint{2.953835in}{2.284877in}}%
\pgfpathcurveto{\pgfqpoint{2.962072in}{2.284877in}}{\pgfqpoint{2.969972in}{2.288150in}}{\pgfqpoint{2.975796in}{2.293974in}}%
\pgfpathcurveto{\pgfqpoint{2.981620in}{2.299798in}}{\pgfqpoint{2.984892in}{2.307698in}}{\pgfqpoint{2.984892in}{2.315934in}}%
\pgfpathcurveto{\pgfqpoint{2.984892in}{2.324170in}}{\pgfqpoint{2.981620in}{2.332070in}}{\pgfqpoint{2.975796in}{2.337894in}}%
\pgfpathcurveto{\pgfqpoint{2.969972in}{2.343718in}}{\pgfqpoint{2.962072in}{2.346990in}}{\pgfqpoint{2.953835in}{2.346990in}}%
\pgfpathcurveto{\pgfqpoint{2.945599in}{2.346990in}}{\pgfqpoint{2.937699in}{2.343718in}}{\pgfqpoint{2.931875in}{2.337894in}}%
\pgfpathcurveto{\pgfqpoint{2.926051in}{2.332070in}}{\pgfqpoint{2.922779in}{2.324170in}}{\pgfqpoint{2.922779in}{2.315934in}}%
\pgfpathcurveto{\pgfqpoint{2.922779in}{2.307698in}}{\pgfqpoint{2.926051in}{2.299798in}}{\pgfqpoint{2.931875in}{2.293974in}}%
\pgfpathcurveto{\pgfqpoint{2.937699in}{2.288150in}}{\pgfqpoint{2.945599in}{2.284877in}}{\pgfqpoint{2.953835in}{2.284877in}}%
\pgfpathclose%
\pgfusepath{stroke,fill}%
\end{pgfscope}%
\begin{pgfscope}%
\pgfpathrectangle{\pgfqpoint{0.100000in}{0.212622in}}{\pgfqpoint{3.696000in}{3.696000in}}%
\pgfusepath{clip}%
\pgfsetbuttcap%
\pgfsetroundjoin%
\definecolor{currentfill}{rgb}{0.121569,0.466667,0.705882}%
\pgfsetfillcolor{currentfill}%
\pgfsetfillopacity{0.541659}%
\pgfsetlinewidth{1.003750pt}%
\definecolor{currentstroke}{rgb}{0.121569,0.466667,0.705882}%
\pgfsetstrokecolor{currentstroke}%
\pgfsetstrokeopacity{0.541659}%
\pgfsetdash{}{0pt}%
\pgfpathmoveto{\pgfqpoint{2.961323in}{2.284102in}}%
\pgfpathcurveto{\pgfqpoint{2.969559in}{2.284102in}}{\pgfqpoint{2.977459in}{2.287374in}}{\pgfqpoint{2.983283in}{2.293198in}}%
\pgfpathcurveto{\pgfqpoint{2.989107in}{2.299022in}}{\pgfqpoint{2.992379in}{2.306922in}}{\pgfqpoint{2.992379in}{2.315159in}}%
\pgfpathcurveto{\pgfqpoint{2.992379in}{2.323395in}}{\pgfqpoint{2.989107in}{2.331295in}}{\pgfqpoint{2.983283in}{2.337119in}}%
\pgfpathcurveto{\pgfqpoint{2.977459in}{2.342943in}}{\pgfqpoint{2.969559in}{2.346215in}}{\pgfqpoint{2.961323in}{2.346215in}}%
\pgfpathcurveto{\pgfqpoint{2.953087in}{2.346215in}}{\pgfqpoint{2.945187in}{2.342943in}}{\pgfqpoint{2.939363in}{2.337119in}}%
\pgfpathcurveto{\pgfqpoint{2.933539in}{2.331295in}}{\pgfqpoint{2.930266in}{2.323395in}}{\pgfqpoint{2.930266in}{2.315159in}}%
\pgfpathcurveto{\pgfqpoint{2.930266in}{2.306922in}}{\pgfqpoint{2.933539in}{2.299022in}}{\pgfqpoint{2.939363in}{2.293198in}}%
\pgfpathcurveto{\pgfqpoint{2.945187in}{2.287374in}}{\pgfqpoint{2.953087in}{2.284102in}}{\pgfqpoint{2.961323in}{2.284102in}}%
\pgfpathclose%
\pgfusepath{stroke,fill}%
\end{pgfscope}%
\begin{pgfscope}%
\pgfpathrectangle{\pgfqpoint{0.100000in}{0.212622in}}{\pgfqpoint{3.696000in}{3.696000in}}%
\pgfusepath{clip}%
\pgfsetbuttcap%
\pgfsetroundjoin%
\definecolor{currentfill}{rgb}{0.121569,0.466667,0.705882}%
\pgfsetfillcolor{currentfill}%
\pgfsetfillopacity{0.543612}%
\pgfsetlinewidth{1.003750pt}%
\definecolor{currentstroke}{rgb}{0.121569,0.466667,0.705882}%
\pgfsetstrokecolor{currentstroke}%
\pgfsetstrokeopacity{0.543612}%
\pgfsetdash{}{0pt}%
\pgfpathmoveto{\pgfqpoint{2.972486in}{2.286167in}}%
\pgfpathcurveto{\pgfqpoint{2.980722in}{2.286167in}}{\pgfqpoint{2.988622in}{2.289439in}}{\pgfqpoint{2.994446in}{2.295263in}}%
\pgfpathcurveto{\pgfqpoint{3.000270in}{2.301087in}}{\pgfqpoint{3.003542in}{2.308987in}}{\pgfqpoint{3.003542in}{2.317223in}}%
\pgfpathcurveto{\pgfqpoint{3.003542in}{2.325459in}}{\pgfqpoint{3.000270in}{2.333359in}}{\pgfqpoint{2.994446in}{2.339183in}}%
\pgfpathcurveto{\pgfqpoint{2.988622in}{2.345007in}}{\pgfqpoint{2.980722in}{2.348280in}}{\pgfqpoint{2.972486in}{2.348280in}}%
\pgfpathcurveto{\pgfqpoint{2.964249in}{2.348280in}}{\pgfqpoint{2.956349in}{2.345007in}}{\pgfqpoint{2.950525in}{2.339183in}}%
\pgfpathcurveto{\pgfqpoint{2.944702in}{2.333359in}}{\pgfqpoint{2.941429in}{2.325459in}}{\pgfqpoint{2.941429in}{2.317223in}}%
\pgfpathcurveto{\pgfqpoint{2.941429in}{2.308987in}}{\pgfqpoint{2.944702in}{2.301087in}}{\pgfqpoint{2.950525in}{2.295263in}}%
\pgfpathcurveto{\pgfqpoint{2.956349in}{2.289439in}}{\pgfqpoint{2.964249in}{2.286167in}}{\pgfqpoint{2.972486in}{2.286167in}}%
\pgfpathclose%
\pgfusepath{stroke,fill}%
\end{pgfscope}%
\begin{pgfscope}%
\pgfpathrectangle{\pgfqpoint{0.100000in}{0.212622in}}{\pgfqpoint{3.696000in}{3.696000in}}%
\pgfusepath{clip}%
\pgfsetbuttcap%
\pgfsetroundjoin%
\definecolor{currentfill}{rgb}{0.121569,0.466667,0.705882}%
\pgfsetfillcolor{currentfill}%
\pgfsetfillopacity{0.546560}%
\pgfsetlinewidth{1.003750pt}%
\definecolor{currentstroke}{rgb}{0.121569,0.466667,0.705882}%
\pgfsetstrokecolor{currentstroke}%
\pgfsetstrokeopacity{0.546560}%
\pgfsetdash{}{0pt}%
\pgfpathmoveto{\pgfqpoint{2.989859in}{2.283209in}}%
\pgfpathcurveto{\pgfqpoint{2.998095in}{2.283209in}}{\pgfqpoint{3.005995in}{2.286481in}}{\pgfqpoint{3.011819in}{2.292305in}}%
\pgfpathcurveto{\pgfqpoint{3.017643in}{2.298129in}}{\pgfqpoint{3.020916in}{2.306029in}}{\pgfqpoint{3.020916in}{2.314265in}}%
\pgfpathcurveto{\pgfqpoint{3.020916in}{2.322502in}}{\pgfqpoint{3.017643in}{2.330402in}}{\pgfqpoint{3.011819in}{2.336226in}}%
\pgfpathcurveto{\pgfqpoint{3.005995in}{2.342050in}}{\pgfqpoint{2.998095in}{2.345322in}}{\pgfqpoint{2.989859in}{2.345322in}}%
\pgfpathcurveto{\pgfqpoint{2.981623in}{2.345322in}}{\pgfqpoint{2.973723in}{2.342050in}}{\pgfqpoint{2.967899in}{2.336226in}}%
\pgfpathcurveto{\pgfqpoint{2.962075in}{2.330402in}}{\pgfqpoint{2.958803in}{2.322502in}}{\pgfqpoint{2.958803in}{2.314265in}}%
\pgfpathcurveto{\pgfqpoint{2.958803in}{2.306029in}}{\pgfqpoint{2.962075in}{2.298129in}}{\pgfqpoint{2.967899in}{2.292305in}}%
\pgfpathcurveto{\pgfqpoint{2.973723in}{2.286481in}}{\pgfqpoint{2.981623in}{2.283209in}}{\pgfqpoint{2.989859in}{2.283209in}}%
\pgfpathclose%
\pgfusepath{stroke,fill}%
\end{pgfscope}%
\begin{pgfscope}%
\pgfpathrectangle{\pgfqpoint{0.100000in}{0.212622in}}{\pgfqpoint{3.696000in}{3.696000in}}%
\pgfusepath{clip}%
\pgfsetbuttcap%
\pgfsetroundjoin%
\definecolor{currentfill}{rgb}{0.121569,0.466667,0.705882}%
\pgfsetfillcolor{currentfill}%
\pgfsetfillopacity{0.546668}%
\pgfsetlinewidth{1.003750pt}%
\definecolor{currentstroke}{rgb}{0.121569,0.466667,0.705882}%
\pgfsetstrokecolor{currentstroke}%
\pgfsetstrokeopacity{0.546668}%
\pgfsetdash{}{0pt}%
\pgfpathmoveto{\pgfqpoint{1.031730in}{1.502448in}}%
\pgfpathcurveto{\pgfqpoint{1.039966in}{1.502448in}}{\pgfqpoint{1.047866in}{1.505720in}}{\pgfqpoint{1.053690in}{1.511544in}}%
\pgfpathcurveto{\pgfqpoint{1.059514in}{1.517368in}}{\pgfqpoint{1.062787in}{1.525268in}}{\pgfqpoint{1.062787in}{1.533504in}}%
\pgfpathcurveto{\pgfqpoint{1.062787in}{1.541741in}}{\pgfqpoint{1.059514in}{1.549641in}}{\pgfqpoint{1.053690in}{1.555465in}}%
\pgfpathcurveto{\pgfqpoint{1.047866in}{1.561289in}}{\pgfqpoint{1.039966in}{1.564561in}}{\pgfqpoint{1.031730in}{1.564561in}}%
\pgfpathcurveto{\pgfqpoint{1.023494in}{1.564561in}}{\pgfqpoint{1.015594in}{1.561289in}}{\pgfqpoint{1.009770in}{1.555465in}}%
\pgfpathcurveto{\pgfqpoint{1.003946in}{1.549641in}}{\pgfqpoint{1.000674in}{1.541741in}}{\pgfqpoint{1.000674in}{1.533504in}}%
\pgfpathcurveto{\pgfqpoint{1.000674in}{1.525268in}}{\pgfqpoint{1.003946in}{1.517368in}}{\pgfqpoint{1.009770in}{1.511544in}}%
\pgfpathcurveto{\pgfqpoint{1.015594in}{1.505720in}}{\pgfqpoint{1.023494in}{1.502448in}}{\pgfqpoint{1.031730in}{1.502448in}}%
\pgfpathclose%
\pgfusepath{stroke,fill}%
\end{pgfscope}%
\begin{pgfscope}%
\pgfpathrectangle{\pgfqpoint{0.100000in}{0.212622in}}{\pgfqpoint{3.696000in}{3.696000in}}%
\pgfusepath{clip}%
\pgfsetbuttcap%
\pgfsetroundjoin%
\definecolor{currentfill}{rgb}{0.121569,0.466667,0.705882}%
\pgfsetfillcolor{currentfill}%
\pgfsetfillopacity{0.550303}%
\pgfsetlinewidth{1.003750pt}%
\definecolor{currentstroke}{rgb}{0.121569,0.466667,0.705882}%
\pgfsetstrokecolor{currentstroke}%
\pgfsetstrokeopacity{0.550303}%
\pgfsetdash{}{0pt}%
\pgfpathmoveto{\pgfqpoint{3.009801in}{2.284191in}}%
\pgfpathcurveto{\pgfqpoint{3.018038in}{2.284191in}}{\pgfqpoint{3.025938in}{2.287464in}}{\pgfqpoint{3.031762in}{2.293288in}}%
\pgfpathcurveto{\pgfqpoint{3.037586in}{2.299112in}}{\pgfqpoint{3.040858in}{2.307012in}}{\pgfqpoint{3.040858in}{2.315248in}}%
\pgfpathcurveto{\pgfqpoint{3.040858in}{2.323484in}}{\pgfqpoint{3.037586in}{2.331384in}}{\pgfqpoint{3.031762in}{2.337208in}}%
\pgfpathcurveto{\pgfqpoint{3.025938in}{2.343032in}}{\pgfqpoint{3.018038in}{2.346304in}}{\pgfqpoint{3.009801in}{2.346304in}}%
\pgfpathcurveto{\pgfqpoint{3.001565in}{2.346304in}}{\pgfqpoint{2.993665in}{2.343032in}}{\pgfqpoint{2.987841in}{2.337208in}}%
\pgfpathcurveto{\pgfqpoint{2.982017in}{2.331384in}}{\pgfqpoint{2.978745in}{2.323484in}}{\pgfqpoint{2.978745in}{2.315248in}}%
\pgfpathcurveto{\pgfqpoint{2.978745in}{2.307012in}}{\pgfqpoint{2.982017in}{2.299112in}}{\pgfqpoint{2.987841in}{2.293288in}}%
\pgfpathcurveto{\pgfqpoint{2.993665in}{2.287464in}}{\pgfqpoint{3.001565in}{2.284191in}}{\pgfqpoint{3.009801in}{2.284191in}}%
\pgfpathclose%
\pgfusepath{stroke,fill}%
\end{pgfscope}%
\begin{pgfscope}%
\pgfpathrectangle{\pgfqpoint{0.100000in}{0.212622in}}{\pgfqpoint{3.696000in}{3.696000in}}%
\pgfusepath{clip}%
\pgfsetbuttcap%
\pgfsetroundjoin%
\definecolor{currentfill}{rgb}{0.121569,0.466667,0.705882}%
\pgfsetfillcolor{currentfill}%
\pgfsetfillopacity{0.554726}%
\pgfsetlinewidth{1.003750pt}%
\definecolor{currentstroke}{rgb}{0.121569,0.466667,0.705882}%
\pgfsetstrokecolor{currentstroke}%
\pgfsetstrokeopacity{0.554726}%
\pgfsetdash{}{0pt}%
\pgfpathmoveto{\pgfqpoint{3.032017in}{2.280932in}}%
\pgfpathcurveto{\pgfqpoint{3.040254in}{2.280932in}}{\pgfqpoint{3.048154in}{2.284205in}}{\pgfqpoint{3.053978in}{2.290029in}}%
\pgfpathcurveto{\pgfqpoint{3.059802in}{2.295853in}}{\pgfqpoint{3.063074in}{2.303753in}}{\pgfqpoint{3.063074in}{2.311989in}}%
\pgfpathcurveto{\pgfqpoint{3.063074in}{2.320225in}}{\pgfqpoint{3.059802in}{2.328125in}}{\pgfqpoint{3.053978in}{2.333949in}}%
\pgfpathcurveto{\pgfqpoint{3.048154in}{2.339773in}}{\pgfqpoint{3.040254in}{2.343045in}}{\pgfqpoint{3.032017in}{2.343045in}}%
\pgfpathcurveto{\pgfqpoint{3.023781in}{2.343045in}}{\pgfqpoint{3.015881in}{2.339773in}}{\pgfqpoint{3.010057in}{2.333949in}}%
\pgfpathcurveto{\pgfqpoint{3.004233in}{2.328125in}}{\pgfqpoint{3.000961in}{2.320225in}}{\pgfqpoint{3.000961in}{2.311989in}}%
\pgfpathcurveto{\pgfqpoint{3.000961in}{2.303753in}}{\pgfqpoint{3.004233in}{2.295853in}}{\pgfqpoint{3.010057in}{2.290029in}}%
\pgfpathcurveto{\pgfqpoint{3.015881in}{2.284205in}}{\pgfqpoint{3.023781in}{2.280932in}}{\pgfqpoint{3.032017in}{2.280932in}}%
\pgfpathclose%
\pgfusepath{stroke,fill}%
\end{pgfscope}%
\begin{pgfscope}%
\pgfpathrectangle{\pgfqpoint{0.100000in}{0.212622in}}{\pgfqpoint{3.696000in}{3.696000in}}%
\pgfusepath{clip}%
\pgfsetbuttcap%
\pgfsetroundjoin%
\definecolor{currentfill}{rgb}{0.121569,0.466667,0.705882}%
\pgfsetfillcolor{currentfill}%
\pgfsetfillopacity{0.555017}%
\pgfsetlinewidth{1.003750pt}%
\definecolor{currentstroke}{rgb}{0.121569,0.466667,0.705882}%
\pgfsetstrokecolor{currentstroke}%
\pgfsetstrokeopacity{0.555017}%
\pgfsetdash{}{0pt}%
\pgfpathmoveto{\pgfqpoint{1.006223in}{1.483509in}}%
\pgfpathcurveto{\pgfqpoint{1.014460in}{1.483509in}}{\pgfqpoint{1.022360in}{1.486781in}}{\pgfqpoint{1.028184in}{1.492605in}}%
\pgfpathcurveto{\pgfqpoint{1.034008in}{1.498429in}}{\pgfqpoint{1.037280in}{1.506329in}}{\pgfqpoint{1.037280in}{1.514565in}}%
\pgfpathcurveto{\pgfqpoint{1.037280in}{1.522801in}}{\pgfqpoint{1.034008in}{1.530702in}}{\pgfqpoint{1.028184in}{1.536525in}}%
\pgfpathcurveto{\pgfqpoint{1.022360in}{1.542349in}}{\pgfqpoint{1.014460in}{1.545622in}}{\pgfqpoint{1.006223in}{1.545622in}}%
\pgfpathcurveto{\pgfqpoint{0.997987in}{1.545622in}}{\pgfqpoint{0.990087in}{1.542349in}}{\pgfqpoint{0.984263in}{1.536525in}}%
\pgfpathcurveto{\pgfqpoint{0.978439in}{1.530702in}}{\pgfqpoint{0.975167in}{1.522801in}}{\pgfqpoint{0.975167in}{1.514565in}}%
\pgfpathcurveto{\pgfqpoint{0.975167in}{1.506329in}}{\pgfqpoint{0.978439in}{1.498429in}}{\pgfqpoint{0.984263in}{1.492605in}}%
\pgfpathcurveto{\pgfqpoint{0.990087in}{1.486781in}}{\pgfqpoint{0.997987in}{1.483509in}}{\pgfqpoint{1.006223in}{1.483509in}}%
\pgfpathclose%
\pgfusepath{stroke,fill}%
\end{pgfscope}%
\begin{pgfscope}%
\pgfpathrectangle{\pgfqpoint{0.100000in}{0.212622in}}{\pgfqpoint{3.696000in}{3.696000in}}%
\pgfusepath{clip}%
\pgfsetbuttcap%
\pgfsetroundjoin%
\definecolor{currentfill}{rgb}{0.121569,0.466667,0.705882}%
\pgfsetfillcolor{currentfill}%
\pgfsetfillopacity{0.560305}%
\pgfsetlinewidth{1.003750pt}%
\definecolor{currentstroke}{rgb}{0.121569,0.466667,0.705882}%
\pgfsetstrokecolor{currentstroke}%
\pgfsetstrokeopacity{0.560305}%
\pgfsetdash{}{0pt}%
\pgfpathmoveto{\pgfqpoint{3.060012in}{2.284862in}}%
\pgfpathcurveto{\pgfqpoint{3.068249in}{2.284862in}}{\pgfqpoint{3.076149in}{2.288135in}}{\pgfqpoint{3.081973in}{2.293959in}}%
\pgfpathcurveto{\pgfqpoint{3.087797in}{2.299782in}}{\pgfqpoint{3.091069in}{2.307683in}}{\pgfqpoint{3.091069in}{2.315919in}}%
\pgfpathcurveto{\pgfqpoint{3.091069in}{2.324155in}}{\pgfqpoint{3.087797in}{2.332055in}}{\pgfqpoint{3.081973in}{2.337879in}}%
\pgfpathcurveto{\pgfqpoint{3.076149in}{2.343703in}}{\pgfqpoint{3.068249in}{2.346975in}}{\pgfqpoint{3.060012in}{2.346975in}}%
\pgfpathcurveto{\pgfqpoint{3.051776in}{2.346975in}}{\pgfqpoint{3.043876in}{2.343703in}}{\pgfqpoint{3.038052in}{2.337879in}}%
\pgfpathcurveto{\pgfqpoint{3.032228in}{2.332055in}}{\pgfqpoint{3.028956in}{2.324155in}}{\pgfqpoint{3.028956in}{2.315919in}}%
\pgfpathcurveto{\pgfqpoint{3.028956in}{2.307683in}}{\pgfqpoint{3.032228in}{2.299782in}}{\pgfqpoint{3.038052in}{2.293959in}}%
\pgfpathcurveto{\pgfqpoint{3.043876in}{2.288135in}}{\pgfqpoint{3.051776in}{2.284862in}}{\pgfqpoint{3.060012in}{2.284862in}}%
\pgfpathclose%
\pgfusepath{stroke,fill}%
\end{pgfscope}%
\begin{pgfscope}%
\pgfpathrectangle{\pgfqpoint{0.100000in}{0.212622in}}{\pgfqpoint{3.696000in}{3.696000in}}%
\pgfusepath{clip}%
\pgfsetbuttcap%
\pgfsetroundjoin%
\definecolor{currentfill}{rgb}{0.121569,0.466667,0.705882}%
\pgfsetfillcolor{currentfill}%
\pgfsetfillopacity{0.562722}%
\pgfsetlinewidth{1.003750pt}%
\definecolor{currentstroke}{rgb}{0.121569,0.466667,0.705882}%
\pgfsetstrokecolor{currentstroke}%
\pgfsetstrokeopacity{0.562722}%
\pgfsetdash{}{0pt}%
\pgfpathmoveto{\pgfqpoint{0.994537in}{1.465596in}}%
\pgfpathcurveto{\pgfqpoint{1.002774in}{1.465596in}}{\pgfqpoint{1.010674in}{1.468868in}}{\pgfqpoint{1.016498in}{1.474692in}}%
\pgfpathcurveto{\pgfqpoint{1.022322in}{1.480516in}}{\pgfqpoint{1.025594in}{1.488416in}}{\pgfqpoint{1.025594in}{1.496652in}}%
\pgfpathcurveto{\pgfqpoint{1.025594in}{1.504889in}}{\pgfqpoint{1.022322in}{1.512789in}}{\pgfqpoint{1.016498in}{1.518613in}}%
\pgfpathcurveto{\pgfqpoint{1.010674in}{1.524437in}}{\pgfqpoint{1.002774in}{1.527709in}}{\pgfqpoint{0.994537in}{1.527709in}}%
\pgfpathcurveto{\pgfqpoint{0.986301in}{1.527709in}}{\pgfqpoint{0.978401in}{1.524437in}}{\pgfqpoint{0.972577in}{1.518613in}}%
\pgfpathcurveto{\pgfqpoint{0.966753in}{1.512789in}}{\pgfqpoint{0.963481in}{1.504889in}}{\pgfqpoint{0.963481in}{1.496652in}}%
\pgfpathcurveto{\pgfqpoint{0.963481in}{1.488416in}}{\pgfqpoint{0.966753in}{1.480516in}}{\pgfqpoint{0.972577in}{1.474692in}}%
\pgfpathcurveto{\pgfqpoint{0.978401in}{1.468868in}}{\pgfqpoint{0.986301in}{1.465596in}}{\pgfqpoint{0.994537in}{1.465596in}}%
\pgfpathclose%
\pgfusepath{stroke,fill}%
\end{pgfscope}%
\begin{pgfscope}%
\pgfpathrectangle{\pgfqpoint{0.100000in}{0.212622in}}{\pgfqpoint{3.696000in}{3.696000in}}%
\pgfusepath{clip}%
\pgfsetbuttcap%
\pgfsetroundjoin%
\definecolor{currentfill}{rgb}{0.121569,0.466667,0.705882}%
\pgfsetfillcolor{currentfill}%
\pgfsetfillopacity{0.566306}%
\pgfsetlinewidth{1.003750pt}%
\definecolor{currentstroke}{rgb}{0.121569,0.466667,0.705882}%
\pgfsetstrokecolor{currentstroke}%
\pgfsetstrokeopacity{0.566306}%
\pgfsetdash{}{0pt}%
\pgfpathmoveto{\pgfqpoint{3.088619in}{2.278692in}}%
\pgfpathcurveto{\pgfqpoint{3.096856in}{2.278692in}}{\pgfqpoint{3.104756in}{2.281964in}}{\pgfqpoint{3.110580in}{2.287788in}}%
\pgfpathcurveto{\pgfqpoint{3.116403in}{2.293612in}}{\pgfqpoint{3.119676in}{2.301512in}}{\pgfqpoint{3.119676in}{2.309748in}}%
\pgfpathcurveto{\pgfqpoint{3.119676in}{2.317984in}}{\pgfqpoint{3.116403in}{2.325884in}}{\pgfqpoint{3.110580in}{2.331708in}}%
\pgfpathcurveto{\pgfqpoint{3.104756in}{2.337532in}}{\pgfqpoint{3.096856in}{2.340805in}}{\pgfqpoint{3.088619in}{2.340805in}}%
\pgfpathcurveto{\pgfqpoint{3.080383in}{2.340805in}}{\pgfqpoint{3.072483in}{2.337532in}}{\pgfqpoint{3.066659in}{2.331708in}}%
\pgfpathcurveto{\pgfqpoint{3.060835in}{2.325884in}}{\pgfqpoint{3.057563in}{2.317984in}}{\pgfqpoint{3.057563in}{2.309748in}}%
\pgfpathcurveto{\pgfqpoint{3.057563in}{2.301512in}}{\pgfqpoint{3.060835in}{2.293612in}}{\pgfqpoint{3.066659in}{2.287788in}}%
\pgfpathcurveto{\pgfqpoint{3.072483in}{2.281964in}}{\pgfqpoint{3.080383in}{2.278692in}}{\pgfqpoint{3.088619in}{2.278692in}}%
\pgfpathclose%
\pgfusepath{stroke,fill}%
\end{pgfscope}%
\begin{pgfscope}%
\pgfpathrectangle{\pgfqpoint{0.100000in}{0.212622in}}{\pgfqpoint{3.696000in}{3.696000in}}%
\pgfusepath{clip}%
\pgfsetbuttcap%
\pgfsetroundjoin%
\definecolor{currentfill}{rgb}{0.121569,0.466667,0.705882}%
\pgfsetfillcolor{currentfill}%
\pgfsetfillopacity{0.567272}%
\pgfsetlinewidth{1.003750pt}%
\definecolor{currentstroke}{rgb}{0.121569,0.466667,0.705882}%
\pgfsetstrokecolor{currentstroke}%
\pgfsetstrokeopacity{0.567272}%
\pgfsetdash{}{0pt}%
\pgfpathmoveto{\pgfqpoint{0.979176in}{1.457106in}}%
\pgfpathcurveto{\pgfqpoint{0.987412in}{1.457106in}}{\pgfqpoint{0.995312in}{1.460378in}}{\pgfqpoint{1.001136in}{1.466202in}}%
\pgfpathcurveto{\pgfqpoint{1.006960in}{1.472026in}}{\pgfqpoint{1.010232in}{1.479926in}}{\pgfqpoint{1.010232in}{1.488163in}}%
\pgfpathcurveto{\pgfqpoint{1.010232in}{1.496399in}}{\pgfqpoint{1.006960in}{1.504299in}}{\pgfqpoint{1.001136in}{1.510123in}}%
\pgfpathcurveto{\pgfqpoint{0.995312in}{1.515947in}}{\pgfqpoint{0.987412in}{1.519219in}}{\pgfqpoint{0.979176in}{1.519219in}}%
\pgfpathcurveto{\pgfqpoint{0.970939in}{1.519219in}}{\pgfqpoint{0.963039in}{1.515947in}}{\pgfqpoint{0.957215in}{1.510123in}}%
\pgfpathcurveto{\pgfqpoint{0.951391in}{1.504299in}}{\pgfqpoint{0.948119in}{1.496399in}}{\pgfqpoint{0.948119in}{1.488163in}}%
\pgfpathcurveto{\pgfqpoint{0.948119in}{1.479926in}}{\pgfqpoint{0.951391in}{1.472026in}}{\pgfqpoint{0.957215in}{1.466202in}}%
\pgfpathcurveto{\pgfqpoint{0.963039in}{1.460378in}}{\pgfqpoint{0.970939in}{1.457106in}}{\pgfqpoint{0.979176in}{1.457106in}}%
\pgfpathclose%
\pgfusepath{stroke,fill}%
\end{pgfscope}%
\begin{pgfscope}%
\pgfpathrectangle{\pgfqpoint{0.100000in}{0.212622in}}{\pgfqpoint{3.696000in}{3.696000in}}%
\pgfusepath{clip}%
\pgfsetbuttcap%
\pgfsetroundjoin%
\definecolor{currentfill}{rgb}{0.121569,0.466667,0.705882}%
\pgfsetfillcolor{currentfill}%
\pgfsetfillopacity{0.569875}%
\pgfsetlinewidth{1.003750pt}%
\definecolor{currentstroke}{rgb}{0.121569,0.466667,0.705882}%
\pgfsetstrokecolor{currentstroke}%
\pgfsetstrokeopacity{0.569875}%
\pgfsetdash{}{0pt}%
\pgfpathmoveto{\pgfqpoint{0.974384in}{1.450218in}}%
\pgfpathcurveto{\pgfqpoint{0.982621in}{1.450218in}}{\pgfqpoint{0.990521in}{1.453491in}}{\pgfqpoint{0.996345in}{1.459315in}}%
\pgfpathcurveto{\pgfqpoint{1.002169in}{1.465138in}}{\pgfqpoint{1.005441in}{1.473039in}}{\pgfqpoint{1.005441in}{1.481275in}}%
\pgfpathcurveto{\pgfqpoint{1.005441in}{1.489511in}}{\pgfqpoint{1.002169in}{1.497411in}}{\pgfqpoint{0.996345in}{1.503235in}}%
\pgfpathcurveto{\pgfqpoint{0.990521in}{1.509059in}}{\pgfqpoint{0.982621in}{1.512331in}}{\pgfqpoint{0.974384in}{1.512331in}}%
\pgfpathcurveto{\pgfqpoint{0.966148in}{1.512331in}}{\pgfqpoint{0.958248in}{1.509059in}}{\pgfqpoint{0.952424in}{1.503235in}}%
\pgfpathcurveto{\pgfqpoint{0.946600in}{1.497411in}}{\pgfqpoint{0.943328in}{1.489511in}}{\pgfqpoint{0.943328in}{1.481275in}}%
\pgfpathcurveto{\pgfqpoint{0.943328in}{1.473039in}}{\pgfqpoint{0.946600in}{1.465138in}}{\pgfqpoint{0.952424in}{1.459315in}}%
\pgfpathcurveto{\pgfqpoint{0.958248in}{1.453491in}}{\pgfqpoint{0.966148in}{1.450218in}}{\pgfqpoint{0.974384in}{1.450218in}}%
\pgfpathclose%
\pgfusepath{stroke,fill}%
\end{pgfscope}%
\begin{pgfscope}%
\pgfpathrectangle{\pgfqpoint{0.100000in}{0.212622in}}{\pgfqpoint{3.696000in}{3.696000in}}%
\pgfusepath{clip}%
\pgfsetbuttcap%
\pgfsetroundjoin%
\definecolor{currentfill}{rgb}{0.121569,0.466667,0.705882}%
\pgfsetfillcolor{currentfill}%
\pgfsetfillopacity{0.570883}%
\pgfsetlinewidth{1.003750pt}%
\definecolor{currentstroke}{rgb}{0.121569,0.466667,0.705882}%
\pgfsetstrokecolor{currentstroke}%
\pgfsetstrokeopacity{0.570883}%
\pgfsetdash{}{0pt}%
\pgfpathmoveto{\pgfqpoint{0.971437in}{1.446720in}}%
\pgfpathcurveto{\pgfqpoint{0.979674in}{1.446720in}}{\pgfqpoint{0.987574in}{1.449993in}}{\pgfqpoint{0.993398in}{1.455817in}}%
\pgfpathcurveto{\pgfqpoint{0.999222in}{1.461640in}}{\pgfqpoint{1.002494in}{1.469540in}}{\pgfqpoint{1.002494in}{1.477777in}}%
\pgfpathcurveto{\pgfqpoint{1.002494in}{1.486013in}}{\pgfqpoint{0.999222in}{1.493913in}}{\pgfqpoint{0.993398in}{1.499737in}}%
\pgfpathcurveto{\pgfqpoint{0.987574in}{1.505561in}}{\pgfqpoint{0.979674in}{1.508833in}}{\pgfqpoint{0.971437in}{1.508833in}}%
\pgfpathcurveto{\pgfqpoint{0.963201in}{1.508833in}}{\pgfqpoint{0.955301in}{1.505561in}}{\pgfqpoint{0.949477in}{1.499737in}}%
\pgfpathcurveto{\pgfqpoint{0.943653in}{1.493913in}}{\pgfqpoint{0.940381in}{1.486013in}}{\pgfqpoint{0.940381in}{1.477777in}}%
\pgfpathcurveto{\pgfqpoint{0.940381in}{1.469540in}}{\pgfqpoint{0.943653in}{1.461640in}}{\pgfqpoint{0.949477in}{1.455817in}}%
\pgfpathcurveto{\pgfqpoint{0.955301in}{1.449993in}}{\pgfqpoint{0.963201in}{1.446720in}}{\pgfqpoint{0.971437in}{1.446720in}}%
\pgfpathclose%
\pgfusepath{stroke,fill}%
\end{pgfscope}%
\begin{pgfscope}%
\pgfpathrectangle{\pgfqpoint{0.100000in}{0.212622in}}{\pgfqpoint{3.696000in}{3.696000in}}%
\pgfusepath{clip}%
\pgfsetbuttcap%
\pgfsetroundjoin%
\definecolor{currentfill}{rgb}{0.121569,0.466667,0.705882}%
\pgfsetfillcolor{currentfill}%
\pgfsetfillopacity{0.572635}%
\pgfsetlinewidth{1.003750pt}%
\definecolor{currentstroke}{rgb}{0.121569,0.466667,0.705882}%
\pgfsetstrokecolor{currentstroke}%
\pgfsetstrokeopacity{0.572635}%
\pgfsetdash{}{0pt}%
\pgfpathmoveto{\pgfqpoint{0.965894in}{1.440008in}}%
\pgfpathcurveto{\pgfqpoint{0.974131in}{1.440008in}}{\pgfqpoint{0.982031in}{1.443280in}}{\pgfqpoint{0.987855in}{1.449104in}}%
\pgfpathcurveto{\pgfqpoint{0.993679in}{1.454928in}}{\pgfqpoint{0.996951in}{1.462828in}}{\pgfqpoint{0.996951in}{1.471065in}}%
\pgfpathcurveto{\pgfqpoint{0.996951in}{1.479301in}}{\pgfqpoint{0.993679in}{1.487201in}}{\pgfqpoint{0.987855in}{1.493025in}}%
\pgfpathcurveto{\pgfqpoint{0.982031in}{1.498849in}}{\pgfqpoint{0.974131in}{1.502121in}}{\pgfqpoint{0.965894in}{1.502121in}}%
\pgfpathcurveto{\pgfqpoint{0.957658in}{1.502121in}}{\pgfqpoint{0.949758in}{1.498849in}}{\pgfqpoint{0.943934in}{1.493025in}}%
\pgfpathcurveto{\pgfqpoint{0.938110in}{1.487201in}}{\pgfqpoint{0.934838in}{1.479301in}}{\pgfqpoint{0.934838in}{1.471065in}}%
\pgfpathcurveto{\pgfqpoint{0.934838in}{1.462828in}}{\pgfqpoint{0.938110in}{1.454928in}}{\pgfqpoint{0.943934in}{1.449104in}}%
\pgfpathcurveto{\pgfqpoint{0.949758in}{1.443280in}}{\pgfqpoint{0.957658in}{1.440008in}}{\pgfqpoint{0.965894in}{1.440008in}}%
\pgfpathclose%
\pgfusepath{stroke,fill}%
\end{pgfscope}%
\begin{pgfscope}%
\pgfpathrectangle{\pgfqpoint{0.100000in}{0.212622in}}{\pgfqpoint{3.696000in}{3.696000in}}%
\pgfusepath{clip}%
\pgfsetbuttcap%
\pgfsetroundjoin%
\definecolor{currentfill}{rgb}{0.121569,0.466667,0.705882}%
\pgfsetfillcolor{currentfill}%
\pgfsetfillopacity{0.573175}%
\pgfsetlinewidth{1.003750pt}%
\definecolor{currentstroke}{rgb}{0.121569,0.466667,0.705882}%
\pgfsetstrokecolor{currentstroke}%
\pgfsetstrokeopacity{0.573175}%
\pgfsetdash{}{0pt}%
\pgfpathmoveto{\pgfqpoint{3.126861in}{2.275510in}}%
\pgfpathcurveto{\pgfqpoint{3.135097in}{2.275510in}}{\pgfqpoint{3.142997in}{2.278782in}}{\pgfqpoint{3.148821in}{2.284606in}}%
\pgfpathcurveto{\pgfqpoint{3.154645in}{2.290430in}}{\pgfqpoint{3.157918in}{2.298330in}}{\pgfqpoint{3.157918in}{2.306567in}}%
\pgfpathcurveto{\pgfqpoint{3.157918in}{2.314803in}}{\pgfqpoint{3.154645in}{2.322703in}}{\pgfqpoint{3.148821in}{2.328527in}}%
\pgfpathcurveto{\pgfqpoint{3.142997in}{2.334351in}}{\pgfqpoint{3.135097in}{2.337623in}}{\pgfqpoint{3.126861in}{2.337623in}}%
\pgfpathcurveto{\pgfqpoint{3.118625in}{2.337623in}}{\pgfqpoint{3.110725in}{2.334351in}}{\pgfqpoint{3.104901in}{2.328527in}}%
\pgfpathcurveto{\pgfqpoint{3.099077in}{2.322703in}}{\pgfqpoint{3.095805in}{2.314803in}}{\pgfqpoint{3.095805in}{2.306567in}}%
\pgfpathcurveto{\pgfqpoint{3.095805in}{2.298330in}}{\pgfqpoint{3.099077in}{2.290430in}}{\pgfqpoint{3.104901in}{2.284606in}}%
\pgfpathcurveto{\pgfqpoint{3.110725in}{2.278782in}}{\pgfqpoint{3.118625in}{2.275510in}}{\pgfqpoint{3.126861in}{2.275510in}}%
\pgfpathclose%
\pgfusepath{stroke,fill}%
\end{pgfscope}%
\begin{pgfscope}%
\pgfpathrectangle{\pgfqpoint{0.100000in}{0.212622in}}{\pgfqpoint{3.696000in}{3.696000in}}%
\pgfusepath{clip}%
\pgfsetbuttcap%
\pgfsetroundjoin%
\definecolor{currentfill}{rgb}{0.121569,0.466667,0.705882}%
\pgfsetfillcolor{currentfill}%
\pgfsetfillopacity{0.575850}%
\pgfsetlinewidth{1.003750pt}%
\definecolor{currentstroke}{rgb}{0.121569,0.466667,0.705882}%
\pgfsetstrokecolor{currentstroke}%
\pgfsetstrokeopacity{0.575850}%
\pgfsetdash{}{0pt}%
\pgfpathmoveto{\pgfqpoint{0.957619in}{1.426135in}}%
\pgfpathcurveto{\pgfqpoint{0.965855in}{1.426135in}}{\pgfqpoint{0.973755in}{1.429408in}}{\pgfqpoint{0.979579in}{1.435231in}}%
\pgfpathcurveto{\pgfqpoint{0.985403in}{1.441055in}}{\pgfqpoint{0.988676in}{1.448955in}}{\pgfqpoint{0.988676in}{1.457192in}}%
\pgfpathcurveto{\pgfqpoint{0.988676in}{1.465428in}}{\pgfqpoint{0.985403in}{1.473328in}}{\pgfqpoint{0.979579in}{1.479152in}}%
\pgfpathcurveto{\pgfqpoint{0.973755in}{1.484976in}}{\pgfqpoint{0.965855in}{1.488248in}}{\pgfqpoint{0.957619in}{1.488248in}}%
\pgfpathcurveto{\pgfqpoint{0.949383in}{1.488248in}}{\pgfqpoint{0.941483in}{1.484976in}}{\pgfqpoint{0.935659in}{1.479152in}}%
\pgfpathcurveto{\pgfqpoint{0.929835in}{1.473328in}}{\pgfqpoint{0.926563in}{1.465428in}}{\pgfqpoint{0.926563in}{1.457192in}}%
\pgfpathcurveto{\pgfqpoint{0.926563in}{1.448955in}}{\pgfqpoint{0.929835in}{1.441055in}}{\pgfqpoint{0.935659in}{1.435231in}}%
\pgfpathcurveto{\pgfqpoint{0.941483in}{1.429408in}}{\pgfqpoint{0.949383in}{1.426135in}}{\pgfqpoint{0.957619in}{1.426135in}}%
\pgfpathclose%
\pgfusepath{stroke,fill}%
\end{pgfscope}%
\begin{pgfscope}%
\pgfpathrectangle{\pgfqpoint{0.100000in}{0.212622in}}{\pgfqpoint{3.696000in}{3.696000in}}%
\pgfusepath{clip}%
\pgfsetbuttcap%
\pgfsetroundjoin%
\definecolor{currentfill}{rgb}{0.121569,0.466667,0.705882}%
\pgfsetfillcolor{currentfill}%
\pgfsetfillopacity{0.582040}%
\pgfsetlinewidth{1.003750pt}%
\definecolor{currentstroke}{rgb}{0.121569,0.466667,0.705882}%
\pgfsetstrokecolor{currentstroke}%
\pgfsetstrokeopacity{0.582040}%
\pgfsetdash{}{0pt}%
\pgfpathmoveto{\pgfqpoint{3.166746in}{2.275767in}}%
\pgfpathcurveto{\pgfqpoint{3.174982in}{2.275767in}}{\pgfqpoint{3.182882in}{2.279040in}}{\pgfqpoint{3.188706in}{2.284863in}}%
\pgfpathcurveto{\pgfqpoint{3.194530in}{2.290687in}}{\pgfqpoint{3.197803in}{2.298587in}}{\pgfqpoint{3.197803in}{2.306824in}}%
\pgfpathcurveto{\pgfqpoint{3.197803in}{2.315060in}}{\pgfqpoint{3.194530in}{2.322960in}}{\pgfqpoint{3.188706in}{2.328784in}}%
\pgfpathcurveto{\pgfqpoint{3.182882in}{2.334608in}}{\pgfqpoint{3.174982in}{2.337880in}}{\pgfqpoint{3.166746in}{2.337880in}}%
\pgfpathcurveto{\pgfqpoint{3.158510in}{2.337880in}}{\pgfqpoint{3.150610in}{2.334608in}}{\pgfqpoint{3.144786in}{2.328784in}}%
\pgfpathcurveto{\pgfqpoint{3.138962in}{2.322960in}}{\pgfqpoint{3.135690in}{2.315060in}}{\pgfqpoint{3.135690in}{2.306824in}}%
\pgfpathcurveto{\pgfqpoint{3.135690in}{2.298587in}}{\pgfqpoint{3.138962in}{2.290687in}}{\pgfqpoint{3.144786in}{2.284863in}}%
\pgfpathcurveto{\pgfqpoint{3.150610in}{2.279040in}}{\pgfqpoint{3.158510in}{2.275767in}}{\pgfqpoint{3.166746in}{2.275767in}}%
\pgfpathclose%
\pgfusepath{stroke,fill}%
\end{pgfscope}%
\begin{pgfscope}%
\pgfpathrectangle{\pgfqpoint{0.100000in}{0.212622in}}{\pgfqpoint{3.696000in}{3.696000in}}%
\pgfusepath{clip}%
\pgfsetbuttcap%
\pgfsetroundjoin%
\definecolor{currentfill}{rgb}{0.121569,0.466667,0.705882}%
\pgfsetfillcolor{currentfill}%
\pgfsetfillopacity{0.582172}%
\pgfsetlinewidth{1.003750pt}%
\definecolor{currentstroke}{rgb}{0.121569,0.466667,0.705882}%
\pgfsetstrokecolor{currentstroke}%
\pgfsetstrokeopacity{0.582172}%
\pgfsetdash{}{0pt}%
\pgfpathmoveto{\pgfqpoint{0.937391in}{1.409356in}}%
\pgfpathcurveto{\pgfqpoint{0.945627in}{1.409356in}}{\pgfqpoint{0.953528in}{1.412629in}}{\pgfqpoint{0.959351in}{1.418453in}}%
\pgfpathcurveto{\pgfqpoint{0.965175in}{1.424277in}}{\pgfqpoint{0.968448in}{1.432177in}}{\pgfqpoint{0.968448in}{1.440413in}}%
\pgfpathcurveto{\pgfqpoint{0.968448in}{1.448649in}}{\pgfqpoint{0.965175in}{1.456549in}}{\pgfqpoint{0.959351in}{1.462373in}}%
\pgfpathcurveto{\pgfqpoint{0.953528in}{1.468197in}}{\pgfqpoint{0.945627in}{1.471469in}}{\pgfqpoint{0.937391in}{1.471469in}}%
\pgfpathcurveto{\pgfqpoint{0.929155in}{1.471469in}}{\pgfqpoint{0.921255in}{1.468197in}}{\pgfqpoint{0.915431in}{1.462373in}}%
\pgfpathcurveto{\pgfqpoint{0.909607in}{1.456549in}}{\pgfqpoint{0.906335in}{1.448649in}}{\pgfqpoint{0.906335in}{1.440413in}}%
\pgfpathcurveto{\pgfqpoint{0.906335in}{1.432177in}}{\pgfqpoint{0.909607in}{1.424277in}}{\pgfqpoint{0.915431in}{1.418453in}}%
\pgfpathcurveto{\pgfqpoint{0.921255in}{1.412629in}}{\pgfqpoint{0.929155in}{1.409356in}}{\pgfqpoint{0.937391in}{1.409356in}}%
\pgfpathclose%
\pgfusepath{stroke,fill}%
\end{pgfscope}%
\begin{pgfscope}%
\pgfpathrectangle{\pgfqpoint{0.100000in}{0.212622in}}{\pgfqpoint{3.696000in}{3.696000in}}%
\pgfusepath{clip}%
\pgfsetbuttcap%
\pgfsetroundjoin%
\definecolor{currentfill}{rgb}{0.121569,0.466667,0.705882}%
\pgfsetfillcolor{currentfill}%
\pgfsetfillopacity{0.587906}%
\pgfsetlinewidth{1.003750pt}%
\definecolor{currentstroke}{rgb}{0.121569,0.466667,0.705882}%
\pgfsetstrokecolor{currentstroke}%
\pgfsetstrokeopacity{0.587906}%
\pgfsetdash{}{0pt}%
\pgfpathmoveto{\pgfqpoint{0.929392in}{1.397361in}}%
\pgfpathcurveto{\pgfqpoint{0.937628in}{1.397361in}}{\pgfqpoint{0.945528in}{1.400634in}}{\pgfqpoint{0.951352in}{1.406458in}}%
\pgfpathcurveto{\pgfqpoint{0.957176in}{1.412282in}}{\pgfqpoint{0.960448in}{1.420182in}}{\pgfqpoint{0.960448in}{1.428418in}}%
\pgfpathcurveto{\pgfqpoint{0.960448in}{1.436654in}}{\pgfqpoint{0.957176in}{1.444554in}}{\pgfqpoint{0.951352in}{1.450378in}}%
\pgfpathcurveto{\pgfqpoint{0.945528in}{1.456202in}}{\pgfqpoint{0.937628in}{1.459474in}}{\pgfqpoint{0.929392in}{1.459474in}}%
\pgfpathcurveto{\pgfqpoint{0.921155in}{1.459474in}}{\pgfqpoint{0.913255in}{1.456202in}}{\pgfqpoint{0.907431in}{1.450378in}}%
\pgfpathcurveto{\pgfqpoint{0.901607in}{1.444554in}}{\pgfqpoint{0.898335in}{1.436654in}}{\pgfqpoint{0.898335in}{1.428418in}}%
\pgfpathcurveto{\pgfqpoint{0.898335in}{1.420182in}}{\pgfqpoint{0.901607in}{1.412282in}}{\pgfqpoint{0.907431in}{1.406458in}}%
\pgfpathcurveto{\pgfqpoint{0.913255in}{1.400634in}}{\pgfqpoint{0.921155in}{1.397361in}}{\pgfqpoint{0.929392in}{1.397361in}}%
\pgfpathclose%
\pgfusepath{stroke,fill}%
\end{pgfscope}%
\begin{pgfscope}%
\pgfpathrectangle{\pgfqpoint{0.100000in}{0.212622in}}{\pgfqpoint{3.696000in}{3.696000in}}%
\pgfusepath{clip}%
\pgfsetbuttcap%
\pgfsetroundjoin%
\definecolor{currentfill}{rgb}{0.121569,0.466667,0.705882}%
\pgfsetfillcolor{currentfill}%
\pgfsetfillopacity{0.590485}%
\pgfsetlinewidth{1.003750pt}%
\definecolor{currentstroke}{rgb}{0.121569,0.466667,0.705882}%
\pgfsetstrokecolor{currentstroke}%
\pgfsetstrokeopacity{0.590485}%
\pgfsetdash{}{0pt}%
\pgfpathmoveto{\pgfqpoint{3.210538in}{2.276275in}}%
\pgfpathcurveto{\pgfqpoint{3.218774in}{2.276275in}}{\pgfqpoint{3.226674in}{2.279548in}}{\pgfqpoint{3.232498in}{2.285372in}}%
\pgfpathcurveto{\pgfqpoint{3.238322in}{2.291195in}}{\pgfqpoint{3.241594in}{2.299096in}}{\pgfqpoint{3.241594in}{2.307332in}}%
\pgfpathcurveto{\pgfqpoint{3.241594in}{2.315568in}}{\pgfqpoint{3.238322in}{2.323468in}}{\pgfqpoint{3.232498in}{2.329292in}}%
\pgfpathcurveto{\pgfqpoint{3.226674in}{2.335116in}}{\pgfqpoint{3.218774in}{2.338388in}}{\pgfqpoint{3.210538in}{2.338388in}}%
\pgfpathcurveto{\pgfqpoint{3.202301in}{2.338388in}}{\pgfqpoint{3.194401in}{2.335116in}}{\pgfqpoint{3.188577in}{2.329292in}}%
\pgfpathcurveto{\pgfqpoint{3.182753in}{2.323468in}}{\pgfqpoint{3.179481in}{2.315568in}}{\pgfqpoint{3.179481in}{2.307332in}}%
\pgfpathcurveto{\pgfqpoint{3.179481in}{2.299096in}}{\pgfqpoint{3.182753in}{2.291195in}}{\pgfqpoint{3.188577in}{2.285372in}}%
\pgfpathcurveto{\pgfqpoint{3.194401in}{2.279548in}}{\pgfqpoint{3.202301in}{2.276275in}}{\pgfqpoint{3.210538in}{2.276275in}}%
\pgfpathclose%
\pgfusepath{stroke,fill}%
\end{pgfscope}%
\begin{pgfscope}%
\pgfpathrectangle{\pgfqpoint{0.100000in}{0.212622in}}{\pgfqpoint{3.696000in}{3.696000in}}%
\pgfusepath{clip}%
\pgfsetbuttcap%
\pgfsetroundjoin%
\definecolor{currentfill}{rgb}{0.121569,0.466667,0.705882}%
\pgfsetfillcolor{currentfill}%
\pgfsetfillopacity{0.591580}%
\pgfsetlinewidth{1.003750pt}%
\definecolor{currentstroke}{rgb}{0.121569,0.466667,0.705882}%
\pgfsetstrokecolor{currentstroke}%
\pgfsetstrokeopacity{0.591580}%
\pgfsetdash{}{0pt}%
\pgfpathmoveto{\pgfqpoint{0.914399in}{1.388800in}}%
\pgfpathcurveto{\pgfqpoint{0.922635in}{1.388800in}}{\pgfqpoint{0.930535in}{1.392073in}}{\pgfqpoint{0.936359in}{1.397897in}}%
\pgfpathcurveto{\pgfqpoint{0.942183in}{1.403721in}}{\pgfqpoint{0.945455in}{1.411621in}}{\pgfqpoint{0.945455in}{1.419857in}}%
\pgfpathcurveto{\pgfqpoint{0.945455in}{1.428093in}}{\pgfqpoint{0.942183in}{1.435993in}}{\pgfqpoint{0.936359in}{1.441817in}}%
\pgfpathcurveto{\pgfqpoint{0.930535in}{1.447641in}}{\pgfqpoint{0.922635in}{1.450913in}}{\pgfqpoint{0.914399in}{1.450913in}}%
\pgfpathcurveto{\pgfqpoint{0.906162in}{1.450913in}}{\pgfqpoint{0.898262in}{1.447641in}}{\pgfqpoint{0.892438in}{1.441817in}}%
\pgfpathcurveto{\pgfqpoint{0.886614in}{1.435993in}}{\pgfqpoint{0.883342in}{1.428093in}}{\pgfqpoint{0.883342in}{1.419857in}}%
\pgfpathcurveto{\pgfqpoint{0.883342in}{1.411621in}}{\pgfqpoint{0.886614in}{1.403721in}}{\pgfqpoint{0.892438in}{1.397897in}}%
\pgfpathcurveto{\pgfqpoint{0.898262in}{1.392073in}}{\pgfqpoint{0.906162in}{1.388800in}}{\pgfqpoint{0.914399in}{1.388800in}}%
\pgfpathclose%
\pgfusepath{stroke,fill}%
\end{pgfscope}%
\begin{pgfscope}%
\pgfpathrectangle{\pgfqpoint{0.100000in}{0.212622in}}{\pgfqpoint{3.696000in}{3.696000in}}%
\pgfusepath{clip}%
\pgfsetbuttcap%
\pgfsetroundjoin%
\definecolor{currentfill}{rgb}{0.121569,0.466667,0.705882}%
\pgfsetfillcolor{currentfill}%
\pgfsetfillopacity{0.592868}%
\pgfsetlinewidth{1.003750pt}%
\definecolor{currentstroke}{rgb}{0.121569,0.466667,0.705882}%
\pgfsetstrokecolor{currentstroke}%
\pgfsetstrokeopacity{0.592868}%
\pgfsetdash{}{0pt}%
\pgfpathmoveto{\pgfqpoint{0.910448in}{1.382569in}}%
\pgfpathcurveto{\pgfqpoint{0.918684in}{1.382569in}}{\pgfqpoint{0.926584in}{1.385841in}}{\pgfqpoint{0.932408in}{1.391665in}}%
\pgfpathcurveto{\pgfqpoint{0.938232in}{1.397489in}}{\pgfqpoint{0.941505in}{1.405389in}}{\pgfqpoint{0.941505in}{1.413626in}}%
\pgfpathcurveto{\pgfqpoint{0.941505in}{1.421862in}}{\pgfqpoint{0.938232in}{1.429762in}}{\pgfqpoint{0.932408in}{1.435586in}}%
\pgfpathcurveto{\pgfqpoint{0.926584in}{1.441410in}}{\pgfqpoint{0.918684in}{1.444682in}}{\pgfqpoint{0.910448in}{1.444682in}}%
\pgfpathcurveto{\pgfqpoint{0.902212in}{1.444682in}}{\pgfqpoint{0.894312in}{1.441410in}}{\pgfqpoint{0.888488in}{1.435586in}}%
\pgfpathcurveto{\pgfqpoint{0.882664in}{1.429762in}}{\pgfqpoint{0.879392in}{1.421862in}}{\pgfqpoint{0.879392in}{1.413626in}}%
\pgfpathcurveto{\pgfqpoint{0.879392in}{1.405389in}}{\pgfqpoint{0.882664in}{1.397489in}}{\pgfqpoint{0.888488in}{1.391665in}}%
\pgfpathcurveto{\pgfqpoint{0.894312in}{1.385841in}}{\pgfqpoint{0.902212in}{1.382569in}}{\pgfqpoint{0.910448in}{1.382569in}}%
\pgfpathclose%
\pgfusepath{stroke,fill}%
\end{pgfscope}%
\begin{pgfscope}%
\pgfpathrectangle{\pgfqpoint{0.100000in}{0.212622in}}{\pgfqpoint{3.696000in}{3.696000in}}%
\pgfusepath{clip}%
\pgfsetbuttcap%
\pgfsetroundjoin%
\definecolor{currentfill}{rgb}{0.121569,0.466667,0.705882}%
\pgfsetfillcolor{currentfill}%
\pgfsetfillopacity{0.595140}%
\pgfsetlinewidth{1.003750pt}%
\definecolor{currentstroke}{rgb}{0.121569,0.466667,0.705882}%
\pgfsetstrokecolor{currentstroke}%
\pgfsetstrokeopacity{0.595140}%
\pgfsetdash{}{0pt}%
\pgfpathmoveto{\pgfqpoint{3.233608in}{2.273470in}}%
\pgfpathcurveto{\pgfqpoint{3.241845in}{2.273470in}}{\pgfqpoint{3.249745in}{2.276742in}}{\pgfqpoint{3.255569in}{2.282566in}}%
\pgfpathcurveto{\pgfqpoint{3.261393in}{2.288390in}}{\pgfqpoint{3.264665in}{2.296290in}}{\pgfqpoint{3.264665in}{2.304526in}}%
\pgfpathcurveto{\pgfqpoint{3.264665in}{2.312762in}}{\pgfqpoint{3.261393in}{2.320662in}}{\pgfqpoint{3.255569in}{2.326486in}}%
\pgfpathcurveto{\pgfqpoint{3.249745in}{2.332310in}}{\pgfqpoint{3.241845in}{2.335583in}}{\pgfqpoint{3.233608in}{2.335583in}}%
\pgfpathcurveto{\pgfqpoint{3.225372in}{2.335583in}}{\pgfqpoint{3.217472in}{2.332310in}}{\pgfqpoint{3.211648in}{2.326486in}}%
\pgfpathcurveto{\pgfqpoint{3.205824in}{2.320662in}}{\pgfqpoint{3.202552in}{2.312762in}}{\pgfqpoint{3.202552in}{2.304526in}}%
\pgfpathcurveto{\pgfqpoint{3.202552in}{2.296290in}}{\pgfqpoint{3.205824in}{2.288390in}}{\pgfqpoint{3.211648in}{2.282566in}}%
\pgfpathcurveto{\pgfqpoint{3.217472in}{2.276742in}}{\pgfqpoint{3.225372in}{2.273470in}}{\pgfqpoint{3.233608in}{2.273470in}}%
\pgfpathclose%
\pgfusepath{stroke,fill}%
\end{pgfscope}%
\begin{pgfscope}%
\pgfpathrectangle{\pgfqpoint{0.100000in}{0.212622in}}{\pgfqpoint{3.696000in}{3.696000in}}%
\pgfusepath{clip}%
\pgfsetbuttcap%
\pgfsetroundjoin%
\definecolor{currentfill}{rgb}{0.121569,0.466667,0.705882}%
\pgfsetfillcolor{currentfill}%
\pgfsetfillopacity{0.596047}%
\pgfsetlinewidth{1.003750pt}%
\definecolor{currentstroke}{rgb}{0.121569,0.466667,0.705882}%
\pgfsetstrokecolor{currentstroke}%
\pgfsetstrokeopacity{0.596047}%
\pgfsetdash{}{0pt}%
\pgfpathmoveto{\pgfqpoint{0.904840in}{1.375271in}}%
\pgfpathcurveto{\pgfqpoint{0.913076in}{1.375271in}}{\pgfqpoint{0.920976in}{1.378544in}}{\pgfqpoint{0.926800in}{1.384368in}}%
\pgfpathcurveto{\pgfqpoint{0.932624in}{1.390192in}}{\pgfqpoint{0.935896in}{1.398092in}}{\pgfqpoint{0.935896in}{1.406328in}}%
\pgfpathcurveto{\pgfqpoint{0.935896in}{1.414564in}}{\pgfqpoint{0.932624in}{1.422464in}}{\pgfqpoint{0.926800in}{1.428288in}}%
\pgfpathcurveto{\pgfqpoint{0.920976in}{1.434112in}}{\pgfqpoint{0.913076in}{1.437384in}}{\pgfqpoint{0.904840in}{1.437384in}}%
\pgfpathcurveto{\pgfqpoint{0.896603in}{1.437384in}}{\pgfqpoint{0.888703in}{1.434112in}}{\pgfqpoint{0.882879in}{1.428288in}}%
\pgfpathcurveto{\pgfqpoint{0.877055in}{1.422464in}}{\pgfqpoint{0.873783in}{1.414564in}}{\pgfqpoint{0.873783in}{1.406328in}}%
\pgfpathcurveto{\pgfqpoint{0.873783in}{1.398092in}}{\pgfqpoint{0.877055in}{1.390192in}}{\pgfqpoint{0.882879in}{1.384368in}}%
\pgfpathcurveto{\pgfqpoint{0.888703in}{1.378544in}}{\pgfqpoint{0.896603in}{1.375271in}}{\pgfqpoint{0.904840in}{1.375271in}}%
\pgfpathclose%
\pgfusepath{stroke,fill}%
\end{pgfscope}%
\begin{pgfscope}%
\pgfpathrectangle{\pgfqpoint{0.100000in}{0.212622in}}{\pgfqpoint{3.696000in}{3.696000in}}%
\pgfusepath{clip}%
\pgfsetbuttcap%
\pgfsetroundjoin%
\definecolor{currentfill}{rgb}{0.121569,0.466667,0.705882}%
\pgfsetfillcolor{currentfill}%
\pgfsetfillopacity{0.600375}%
\pgfsetlinewidth{1.003750pt}%
\definecolor{currentstroke}{rgb}{0.121569,0.466667,0.705882}%
\pgfsetstrokecolor{currentstroke}%
\pgfsetstrokeopacity{0.600375}%
\pgfsetdash{}{0pt}%
\pgfpathmoveto{\pgfqpoint{0.886413in}{1.362354in}}%
\pgfpathcurveto{\pgfqpoint{0.894650in}{1.362354in}}{\pgfqpoint{0.902550in}{1.365627in}}{\pgfqpoint{0.908374in}{1.371451in}}%
\pgfpathcurveto{\pgfqpoint{0.914198in}{1.377275in}}{\pgfqpoint{0.917470in}{1.385175in}}{\pgfqpoint{0.917470in}{1.393411in}}%
\pgfpathcurveto{\pgfqpoint{0.917470in}{1.401647in}}{\pgfqpoint{0.914198in}{1.409547in}}{\pgfqpoint{0.908374in}{1.415371in}}%
\pgfpathcurveto{\pgfqpoint{0.902550in}{1.421195in}}{\pgfqpoint{0.894650in}{1.424467in}}{\pgfqpoint{0.886413in}{1.424467in}}%
\pgfpathcurveto{\pgfqpoint{0.878177in}{1.424467in}}{\pgfqpoint{0.870277in}{1.421195in}}{\pgfqpoint{0.864453in}{1.415371in}}%
\pgfpathcurveto{\pgfqpoint{0.858629in}{1.409547in}}{\pgfqpoint{0.855357in}{1.401647in}}{\pgfqpoint{0.855357in}{1.393411in}}%
\pgfpathcurveto{\pgfqpoint{0.855357in}{1.385175in}}{\pgfqpoint{0.858629in}{1.377275in}}{\pgfqpoint{0.864453in}{1.371451in}}%
\pgfpathcurveto{\pgfqpoint{0.870277in}{1.365627in}}{\pgfqpoint{0.878177in}{1.362354in}}{\pgfqpoint{0.886413in}{1.362354in}}%
\pgfpathclose%
\pgfusepath{stroke,fill}%
\end{pgfscope}%
\begin{pgfscope}%
\pgfpathrectangle{\pgfqpoint{0.100000in}{0.212622in}}{\pgfqpoint{3.696000in}{3.696000in}}%
\pgfusepath{clip}%
\pgfsetbuttcap%
\pgfsetroundjoin%
\definecolor{currentfill}{rgb}{0.121569,0.466667,0.705882}%
\pgfsetfillcolor{currentfill}%
\pgfsetfillopacity{0.600513}%
\pgfsetlinewidth{1.003750pt}%
\definecolor{currentstroke}{rgb}{0.121569,0.466667,0.705882}%
\pgfsetstrokecolor{currentstroke}%
\pgfsetstrokeopacity{0.600513}%
\pgfsetdash{}{0pt}%
\pgfpathmoveto{\pgfqpoint{3.258866in}{2.271485in}}%
\pgfpathcurveto{\pgfqpoint{3.267102in}{2.271485in}}{\pgfqpoint{3.275002in}{2.274758in}}{\pgfqpoint{3.280826in}{2.280582in}}%
\pgfpathcurveto{\pgfqpoint{3.286650in}{2.286406in}}{\pgfqpoint{3.289923in}{2.294306in}}{\pgfqpoint{3.289923in}{2.302542in}}%
\pgfpathcurveto{\pgfqpoint{3.289923in}{2.310778in}}{\pgfqpoint{3.286650in}{2.318678in}}{\pgfqpoint{3.280826in}{2.324502in}}%
\pgfpathcurveto{\pgfqpoint{3.275002in}{2.330326in}}{\pgfqpoint{3.267102in}{2.333598in}}{\pgfqpoint{3.258866in}{2.333598in}}%
\pgfpathcurveto{\pgfqpoint{3.250630in}{2.333598in}}{\pgfqpoint{3.242730in}{2.330326in}}{\pgfqpoint{3.236906in}{2.324502in}}%
\pgfpathcurveto{\pgfqpoint{3.231082in}{2.318678in}}{\pgfqpoint{3.227810in}{2.310778in}}{\pgfqpoint{3.227810in}{2.302542in}}%
\pgfpathcurveto{\pgfqpoint{3.227810in}{2.294306in}}{\pgfqpoint{3.231082in}{2.286406in}}{\pgfqpoint{3.236906in}{2.280582in}}%
\pgfpathcurveto{\pgfqpoint{3.242730in}{2.274758in}}{\pgfqpoint{3.250630in}{2.271485in}}{\pgfqpoint{3.258866in}{2.271485in}}%
\pgfpathclose%
\pgfusepath{stroke,fill}%
\end{pgfscope}%
\begin{pgfscope}%
\pgfpathrectangle{\pgfqpoint{0.100000in}{0.212622in}}{\pgfqpoint{3.696000in}{3.696000in}}%
\pgfusepath{clip}%
\pgfsetbuttcap%
\pgfsetroundjoin%
\definecolor{currentfill}{rgb}{0.121569,0.466667,0.705882}%
\pgfsetfillcolor{currentfill}%
\pgfsetfillopacity{0.603430}%
\pgfsetlinewidth{1.003750pt}%
\definecolor{currentstroke}{rgb}{0.121569,0.466667,0.705882}%
\pgfsetstrokecolor{currentstroke}%
\pgfsetstrokeopacity{0.603430}%
\pgfsetdash{}{0pt}%
\pgfpathmoveto{\pgfqpoint{3.273190in}{2.271252in}}%
\pgfpathcurveto{\pgfqpoint{3.281427in}{2.271252in}}{\pgfqpoint{3.289327in}{2.274524in}}{\pgfqpoint{3.295151in}{2.280348in}}%
\pgfpathcurveto{\pgfqpoint{3.300975in}{2.286172in}}{\pgfqpoint{3.304247in}{2.294072in}}{\pgfqpoint{3.304247in}{2.302308in}}%
\pgfpathcurveto{\pgfqpoint{3.304247in}{2.310545in}}{\pgfqpoint{3.300975in}{2.318445in}}{\pgfqpoint{3.295151in}{2.324269in}}%
\pgfpathcurveto{\pgfqpoint{3.289327in}{2.330093in}}{\pgfqpoint{3.281427in}{2.333365in}}{\pgfqpoint{3.273190in}{2.333365in}}%
\pgfpathcurveto{\pgfqpoint{3.264954in}{2.333365in}}{\pgfqpoint{3.257054in}{2.330093in}}{\pgfqpoint{3.251230in}{2.324269in}}%
\pgfpathcurveto{\pgfqpoint{3.245406in}{2.318445in}}{\pgfqpoint{3.242134in}{2.310545in}}{\pgfqpoint{3.242134in}{2.302308in}}%
\pgfpathcurveto{\pgfqpoint{3.242134in}{2.294072in}}{\pgfqpoint{3.245406in}{2.286172in}}{\pgfqpoint{3.251230in}{2.280348in}}%
\pgfpathcurveto{\pgfqpoint{3.257054in}{2.274524in}}{\pgfqpoint{3.264954in}{2.271252in}}{\pgfqpoint{3.273190in}{2.271252in}}%
\pgfpathclose%
\pgfusepath{stroke,fill}%
\end{pgfscope}%
\begin{pgfscope}%
\pgfpathrectangle{\pgfqpoint{0.100000in}{0.212622in}}{\pgfqpoint{3.696000in}{3.696000in}}%
\pgfusepath{clip}%
\pgfsetbuttcap%
\pgfsetroundjoin%
\definecolor{currentfill}{rgb}{0.121569,0.466667,0.705882}%
\pgfsetfillcolor{currentfill}%
\pgfsetfillopacity{0.607358}%
\pgfsetlinewidth{1.003750pt}%
\definecolor{currentstroke}{rgb}{0.121569,0.466667,0.705882}%
\pgfsetstrokecolor{currentstroke}%
\pgfsetstrokeopacity{0.607358}%
\pgfsetdash{}{0pt}%
\pgfpathmoveto{\pgfqpoint{3.291242in}{2.267530in}}%
\pgfpathcurveto{\pgfqpoint{3.299478in}{2.267530in}}{\pgfqpoint{3.307378in}{2.270802in}}{\pgfqpoint{3.313202in}{2.276626in}}%
\pgfpathcurveto{\pgfqpoint{3.319026in}{2.282450in}}{\pgfqpoint{3.322298in}{2.290350in}}{\pgfqpoint{3.322298in}{2.298587in}}%
\pgfpathcurveto{\pgfqpoint{3.322298in}{2.306823in}}{\pgfqpoint{3.319026in}{2.314723in}}{\pgfqpoint{3.313202in}{2.320547in}}%
\pgfpathcurveto{\pgfqpoint{3.307378in}{2.326371in}}{\pgfqpoint{3.299478in}{2.329643in}}{\pgfqpoint{3.291242in}{2.329643in}}%
\pgfpathcurveto{\pgfqpoint{3.283005in}{2.329643in}}{\pgfqpoint{3.275105in}{2.326371in}}{\pgfqpoint{3.269281in}{2.320547in}}%
\pgfpathcurveto{\pgfqpoint{3.263457in}{2.314723in}}{\pgfqpoint{3.260185in}{2.306823in}}{\pgfqpoint{3.260185in}{2.298587in}}%
\pgfpathcurveto{\pgfqpoint{3.260185in}{2.290350in}}{\pgfqpoint{3.263457in}{2.282450in}}{\pgfqpoint{3.269281in}{2.276626in}}%
\pgfpathcurveto{\pgfqpoint{3.275105in}{2.270802in}}{\pgfqpoint{3.283005in}{2.267530in}}{\pgfqpoint{3.291242in}{2.267530in}}%
\pgfpathclose%
\pgfusepath{stroke,fill}%
\end{pgfscope}%
\begin{pgfscope}%
\pgfpathrectangle{\pgfqpoint{0.100000in}{0.212622in}}{\pgfqpoint{3.696000in}{3.696000in}}%
\pgfusepath{clip}%
\pgfsetbuttcap%
\pgfsetroundjoin%
\definecolor{currentfill}{rgb}{0.121569,0.466667,0.705882}%
\pgfsetfillcolor{currentfill}%
\pgfsetfillopacity{0.609966}%
\pgfsetlinewidth{1.003750pt}%
\definecolor{currentstroke}{rgb}{0.121569,0.466667,0.705882}%
\pgfsetstrokecolor{currentstroke}%
\pgfsetstrokeopacity{0.609966}%
\pgfsetdash{}{0pt}%
\pgfpathmoveto{\pgfqpoint{3.299873in}{2.265495in}}%
\pgfpathcurveto{\pgfqpoint{3.308110in}{2.265495in}}{\pgfqpoint{3.316010in}{2.268767in}}{\pgfqpoint{3.321834in}{2.274591in}}%
\pgfpathcurveto{\pgfqpoint{3.327658in}{2.280415in}}{\pgfqpoint{3.330930in}{2.288315in}}{\pgfqpoint{3.330930in}{2.296551in}}%
\pgfpathcurveto{\pgfqpoint{3.330930in}{2.304787in}}{\pgfqpoint{3.327658in}{2.312687in}}{\pgfqpoint{3.321834in}{2.318511in}}%
\pgfpathcurveto{\pgfqpoint{3.316010in}{2.324335in}}{\pgfqpoint{3.308110in}{2.327608in}}{\pgfqpoint{3.299873in}{2.327608in}}%
\pgfpathcurveto{\pgfqpoint{3.291637in}{2.327608in}}{\pgfqpoint{3.283737in}{2.324335in}}{\pgfqpoint{3.277913in}{2.318511in}}%
\pgfpathcurveto{\pgfqpoint{3.272089in}{2.312687in}}{\pgfqpoint{3.268817in}{2.304787in}}{\pgfqpoint{3.268817in}{2.296551in}}%
\pgfpathcurveto{\pgfqpoint{3.268817in}{2.288315in}}{\pgfqpoint{3.272089in}{2.280415in}}{\pgfqpoint{3.277913in}{2.274591in}}%
\pgfpathcurveto{\pgfqpoint{3.283737in}{2.268767in}}{\pgfqpoint{3.291637in}{2.265495in}}{\pgfqpoint{3.299873in}{2.265495in}}%
\pgfpathclose%
\pgfusepath{stroke,fill}%
\end{pgfscope}%
\begin{pgfscope}%
\pgfpathrectangle{\pgfqpoint{0.100000in}{0.212622in}}{\pgfqpoint{3.696000in}{3.696000in}}%
\pgfusepath{clip}%
\pgfsetbuttcap%
\pgfsetroundjoin%
\definecolor{currentfill}{rgb}{0.121569,0.466667,0.705882}%
\pgfsetfillcolor{currentfill}%
\pgfsetfillopacity{0.610083}%
\pgfsetlinewidth{1.003750pt}%
\definecolor{currentstroke}{rgb}{0.121569,0.466667,0.705882}%
\pgfsetstrokecolor{currentstroke}%
\pgfsetstrokeopacity{0.610083}%
\pgfsetdash{}{0pt}%
\pgfpathmoveto{\pgfqpoint{0.865979in}{1.334072in}}%
\pgfpathcurveto{\pgfqpoint{0.874215in}{1.334072in}}{\pgfqpoint{0.882116in}{1.337344in}}{\pgfqpoint{0.887939in}{1.343168in}}%
\pgfpathcurveto{\pgfqpoint{0.893763in}{1.348992in}}{\pgfqpoint{0.897036in}{1.356892in}}{\pgfqpoint{0.897036in}{1.365128in}}%
\pgfpathcurveto{\pgfqpoint{0.897036in}{1.373365in}}{\pgfqpoint{0.893763in}{1.381265in}}{\pgfqpoint{0.887939in}{1.387089in}}%
\pgfpathcurveto{\pgfqpoint{0.882116in}{1.392913in}}{\pgfqpoint{0.874215in}{1.396185in}}{\pgfqpoint{0.865979in}{1.396185in}}%
\pgfpathcurveto{\pgfqpoint{0.857743in}{1.396185in}}{\pgfqpoint{0.849843in}{1.392913in}}{\pgfqpoint{0.844019in}{1.387089in}}%
\pgfpathcurveto{\pgfqpoint{0.838195in}{1.381265in}}{\pgfqpoint{0.834923in}{1.373365in}}{\pgfqpoint{0.834923in}{1.365128in}}%
\pgfpathcurveto{\pgfqpoint{0.834923in}{1.356892in}}{\pgfqpoint{0.838195in}{1.348992in}}{\pgfqpoint{0.844019in}{1.343168in}}%
\pgfpathcurveto{\pgfqpoint{0.849843in}{1.337344in}}{\pgfqpoint{0.857743in}{1.334072in}}{\pgfqpoint{0.865979in}{1.334072in}}%
\pgfpathclose%
\pgfusepath{stroke,fill}%
\end{pgfscope}%
\begin{pgfscope}%
\pgfpathrectangle{\pgfqpoint{0.100000in}{0.212622in}}{\pgfqpoint{3.696000in}{3.696000in}}%
\pgfusepath{clip}%
\pgfsetbuttcap%
\pgfsetroundjoin%
\definecolor{currentfill}{rgb}{0.121569,0.466667,0.705882}%
\pgfsetfillcolor{currentfill}%
\pgfsetfillopacity{0.611286}%
\pgfsetlinewidth{1.003750pt}%
\definecolor{currentstroke}{rgb}{0.121569,0.466667,0.705882}%
\pgfsetstrokecolor{currentstroke}%
\pgfsetstrokeopacity{0.611286}%
\pgfsetdash{}{0pt}%
\pgfpathmoveto{\pgfqpoint{3.302797in}{2.261206in}}%
\pgfpathcurveto{\pgfqpoint{3.311033in}{2.261206in}}{\pgfqpoint{3.318933in}{2.264479in}}{\pgfqpoint{3.324757in}{2.270303in}}%
\pgfpathcurveto{\pgfqpoint{3.330581in}{2.276127in}}{\pgfqpoint{3.333853in}{2.284027in}}{\pgfqpoint{3.333853in}{2.292263in}}%
\pgfpathcurveto{\pgfqpoint{3.333853in}{2.300499in}}{\pgfqpoint{3.330581in}{2.308399in}}{\pgfqpoint{3.324757in}{2.314223in}}%
\pgfpathcurveto{\pgfqpoint{3.318933in}{2.320047in}}{\pgfqpoint{3.311033in}{2.323319in}}{\pgfqpoint{3.302797in}{2.323319in}}%
\pgfpathcurveto{\pgfqpoint{3.294560in}{2.323319in}}{\pgfqpoint{3.286660in}{2.320047in}}{\pgfqpoint{3.280836in}{2.314223in}}%
\pgfpathcurveto{\pgfqpoint{3.275012in}{2.308399in}}{\pgfqpoint{3.271740in}{2.300499in}}{\pgfqpoint{3.271740in}{2.292263in}}%
\pgfpathcurveto{\pgfqpoint{3.271740in}{2.284027in}}{\pgfqpoint{3.275012in}{2.276127in}}{\pgfqpoint{3.280836in}{2.270303in}}%
\pgfpathcurveto{\pgfqpoint{3.286660in}{2.264479in}}{\pgfqpoint{3.294560in}{2.261206in}}{\pgfqpoint{3.302797in}{2.261206in}}%
\pgfpathclose%
\pgfusepath{stroke,fill}%
\end{pgfscope}%
\begin{pgfscope}%
\pgfpathrectangle{\pgfqpoint{0.100000in}{0.212622in}}{\pgfqpoint{3.696000in}{3.696000in}}%
\pgfusepath{clip}%
\pgfsetbuttcap%
\pgfsetroundjoin%
\definecolor{currentfill}{rgb}{0.121569,0.466667,0.705882}%
\pgfsetfillcolor{currentfill}%
\pgfsetfillopacity{0.613492}%
\pgfsetlinewidth{1.003750pt}%
\definecolor{currentstroke}{rgb}{0.121569,0.466667,0.705882}%
\pgfsetstrokecolor{currentstroke}%
\pgfsetstrokeopacity{0.613492}%
\pgfsetdash{}{0pt}%
\pgfpathmoveto{\pgfqpoint{3.306089in}{2.256770in}}%
\pgfpathcurveto{\pgfqpoint{3.314325in}{2.256770in}}{\pgfqpoint{3.322225in}{2.260042in}}{\pgfqpoint{3.328049in}{2.265866in}}%
\pgfpathcurveto{\pgfqpoint{3.333873in}{2.271690in}}{\pgfqpoint{3.337145in}{2.279590in}}{\pgfqpoint{3.337145in}{2.287826in}}%
\pgfpathcurveto{\pgfqpoint{3.337145in}{2.296063in}}{\pgfqpoint{3.333873in}{2.303963in}}{\pgfqpoint{3.328049in}{2.309787in}}%
\pgfpathcurveto{\pgfqpoint{3.322225in}{2.315611in}}{\pgfqpoint{3.314325in}{2.318883in}}{\pgfqpoint{3.306089in}{2.318883in}}%
\pgfpathcurveto{\pgfqpoint{3.297852in}{2.318883in}}{\pgfqpoint{3.289952in}{2.315611in}}{\pgfqpoint{3.284128in}{2.309787in}}%
\pgfpathcurveto{\pgfqpoint{3.278304in}{2.303963in}}{\pgfqpoint{3.275032in}{2.296063in}}{\pgfqpoint{3.275032in}{2.287826in}}%
\pgfpathcurveto{\pgfqpoint{3.275032in}{2.279590in}}{\pgfqpoint{3.278304in}{2.271690in}}{\pgfqpoint{3.284128in}{2.265866in}}%
\pgfpathcurveto{\pgfqpoint{3.289952in}{2.260042in}}{\pgfqpoint{3.297852in}{2.256770in}}{\pgfqpoint{3.306089in}{2.256770in}}%
\pgfpathclose%
\pgfusepath{stroke,fill}%
\end{pgfscope}%
\begin{pgfscope}%
\pgfpathrectangle{\pgfqpoint{0.100000in}{0.212622in}}{\pgfqpoint{3.696000in}{3.696000in}}%
\pgfusepath{clip}%
\pgfsetbuttcap%
\pgfsetroundjoin%
\definecolor{currentfill}{rgb}{0.121569,0.466667,0.705882}%
\pgfsetfillcolor{currentfill}%
\pgfsetfillopacity{0.616758}%
\pgfsetlinewidth{1.003750pt}%
\definecolor{currentstroke}{rgb}{0.121569,0.466667,0.705882}%
\pgfsetstrokecolor{currentstroke}%
\pgfsetstrokeopacity{0.616758}%
\pgfsetdash{}{0pt}%
\pgfpathmoveto{\pgfqpoint{3.306915in}{2.252559in}}%
\pgfpathcurveto{\pgfqpoint{3.315151in}{2.252559in}}{\pgfqpoint{3.323052in}{2.255831in}}{\pgfqpoint{3.328875in}{2.261655in}}%
\pgfpathcurveto{\pgfqpoint{3.334699in}{2.267479in}}{\pgfqpoint{3.337972in}{2.275379in}}{\pgfqpoint{3.337972in}{2.283615in}}%
\pgfpathcurveto{\pgfqpoint{3.337972in}{2.291852in}}{\pgfqpoint{3.334699in}{2.299752in}}{\pgfqpoint{3.328875in}{2.305576in}}%
\pgfpathcurveto{\pgfqpoint{3.323052in}{2.311400in}}{\pgfqpoint{3.315151in}{2.314672in}}{\pgfqpoint{3.306915in}{2.314672in}}%
\pgfpathcurveto{\pgfqpoint{3.298679in}{2.314672in}}{\pgfqpoint{3.290779in}{2.311400in}}{\pgfqpoint{3.284955in}{2.305576in}}%
\pgfpathcurveto{\pgfqpoint{3.279131in}{2.299752in}}{\pgfqpoint{3.275859in}{2.291852in}}{\pgfqpoint{3.275859in}{2.283615in}}%
\pgfpathcurveto{\pgfqpoint{3.275859in}{2.275379in}}{\pgfqpoint{3.279131in}{2.267479in}}{\pgfqpoint{3.284955in}{2.261655in}}%
\pgfpathcurveto{\pgfqpoint{3.290779in}{2.255831in}}{\pgfqpoint{3.298679in}{2.252559in}}{\pgfqpoint{3.306915in}{2.252559in}}%
\pgfpathclose%
\pgfusepath{stroke,fill}%
\end{pgfscope}%
\begin{pgfscope}%
\pgfpathrectangle{\pgfqpoint{0.100000in}{0.212622in}}{\pgfqpoint{3.696000in}{3.696000in}}%
\pgfusepath{clip}%
\pgfsetbuttcap%
\pgfsetroundjoin%
\definecolor{currentfill}{rgb}{0.121569,0.466667,0.705882}%
\pgfsetfillcolor{currentfill}%
\pgfsetfillopacity{0.618643}%
\pgfsetlinewidth{1.003750pt}%
\definecolor{currentstroke}{rgb}{0.121569,0.466667,0.705882}%
\pgfsetstrokecolor{currentstroke}%
\pgfsetstrokeopacity{0.618643}%
\pgfsetdash{}{0pt}%
\pgfpathmoveto{\pgfqpoint{3.305075in}{2.251358in}}%
\pgfpathcurveto{\pgfqpoint{3.313311in}{2.251358in}}{\pgfqpoint{3.321211in}{2.254630in}}{\pgfqpoint{3.327035in}{2.260454in}}%
\pgfpathcurveto{\pgfqpoint{3.332859in}{2.266278in}}{\pgfqpoint{3.336131in}{2.274178in}}{\pgfqpoint{3.336131in}{2.282414in}}%
\pgfpathcurveto{\pgfqpoint{3.336131in}{2.290651in}}{\pgfqpoint{3.332859in}{2.298551in}}{\pgfqpoint{3.327035in}{2.304375in}}%
\pgfpathcurveto{\pgfqpoint{3.321211in}{2.310198in}}{\pgfqpoint{3.313311in}{2.313471in}}{\pgfqpoint{3.305075in}{2.313471in}}%
\pgfpathcurveto{\pgfqpoint{3.296839in}{2.313471in}}{\pgfqpoint{3.288939in}{2.310198in}}{\pgfqpoint{3.283115in}{2.304375in}}%
\pgfpathcurveto{\pgfqpoint{3.277291in}{2.298551in}}{\pgfqpoint{3.274018in}{2.290651in}}{\pgfqpoint{3.274018in}{2.282414in}}%
\pgfpathcurveto{\pgfqpoint{3.274018in}{2.274178in}}{\pgfqpoint{3.277291in}{2.266278in}}{\pgfqpoint{3.283115in}{2.260454in}}%
\pgfpathcurveto{\pgfqpoint{3.288939in}{2.254630in}}{\pgfqpoint{3.296839in}{2.251358in}}{\pgfqpoint{3.305075in}{2.251358in}}%
\pgfpathclose%
\pgfusepath{stroke,fill}%
\end{pgfscope}%
\begin{pgfscope}%
\pgfpathrectangle{\pgfqpoint{0.100000in}{0.212622in}}{\pgfqpoint{3.696000in}{3.696000in}}%
\pgfusepath{clip}%
\pgfsetbuttcap%
\pgfsetroundjoin%
\definecolor{currentfill}{rgb}{0.121569,0.466667,0.705882}%
\pgfsetfillcolor{currentfill}%
\pgfsetfillopacity{0.618891}%
\pgfsetlinewidth{1.003750pt}%
\definecolor{currentstroke}{rgb}{0.121569,0.466667,0.705882}%
\pgfsetstrokecolor{currentstroke}%
\pgfsetstrokeopacity{0.618891}%
\pgfsetdash{}{0pt}%
\pgfpathmoveto{\pgfqpoint{0.837149in}{1.319025in}}%
\pgfpathcurveto{\pgfqpoint{0.845385in}{1.319025in}}{\pgfqpoint{0.853285in}{1.322297in}}{\pgfqpoint{0.859109in}{1.328121in}}%
\pgfpathcurveto{\pgfqpoint{0.864933in}{1.333945in}}{\pgfqpoint{0.868206in}{1.341845in}}{\pgfqpoint{0.868206in}{1.350082in}}%
\pgfpathcurveto{\pgfqpoint{0.868206in}{1.358318in}}{\pgfqpoint{0.864933in}{1.366218in}}{\pgfqpoint{0.859109in}{1.372042in}}%
\pgfpathcurveto{\pgfqpoint{0.853285in}{1.377866in}}{\pgfqpoint{0.845385in}{1.381138in}}{\pgfqpoint{0.837149in}{1.381138in}}%
\pgfpathcurveto{\pgfqpoint{0.828913in}{1.381138in}}{\pgfqpoint{0.821013in}{1.377866in}}{\pgfqpoint{0.815189in}{1.372042in}}%
\pgfpathcurveto{\pgfqpoint{0.809365in}{1.366218in}}{\pgfqpoint{0.806093in}{1.358318in}}{\pgfqpoint{0.806093in}{1.350082in}}%
\pgfpathcurveto{\pgfqpoint{0.806093in}{1.341845in}}{\pgfqpoint{0.809365in}{1.333945in}}{\pgfqpoint{0.815189in}{1.328121in}}%
\pgfpathcurveto{\pgfqpoint{0.821013in}{1.322297in}}{\pgfqpoint{0.828913in}{1.319025in}}{\pgfqpoint{0.837149in}{1.319025in}}%
\pgfpathclose%
\pgfusepath{stroke,fill}%
\end{pgfscope}%
\begin{pgfscope}%
\pgfpathrectangle{\pgfqpoint{0.100000in}{0.212622in}}{\pgfqpoint{3.696000in}{3.696000in}}%
\pgfusepath{clip}%
\pgfsetbuttcap%
\pgfsetroundjoin%
\definecolor{currentfill}{rgb}{0.121569,0.466667,0.705882}%
\pgfsetfillcolor{currentfill}%
\pgfsetfillopacity{0.620815}%
\pgfsetlinewidth{1.003750pt}%
\definecolor{currentstroke}{rgb}{0.121569,0.466667,0.705882}%
\pgfsetstrokecolor{currentstroke}%
\pgfsetstrokeopacity{0.620815}%
\pgfsetdash{}{0pt}%
\pgfpathmoveto{\pgfqpoint{0.916842in}{1.260655in}}%
\pgfpathcurveto{\pgfqpoint{0.925078in}{1.260655in}}{\pgfqpoint{0.932978in}{1.263927in}}{\pgfqpoint{0.938802in}{1.269751in}}%
\pgfpathcurveto{\pgfqpoint{0.944626in}{1.275575in}}{\pgfqpoint{0.947899in}{1.283475in}}{\pgfqpoint{0.947899in}{1.291711in}}%
\pgfpathcurveto{\pgfqpoint{0.947899in}{1.299948in}}{\pgfqpoint{0.944626in}{1.307848in}}{\pgfqpoint{0.938802in}{1.313672in}}%
\pgfpathcurveto{\pgfqpoint{0.932978in}{1.319496in}}{\pgfqpoint{0.925078in}{1.322768in}}{\pgfqpoint{0.916842in}{1.322768in}}%
\pgfpathcurveto{\pgfqpoint{0.908606in}{1.322768in}}{\pgfqpoint{0.900706in}{1.319496in}}{\pgfqpoint{0.894882in}{1.313672in}}%
\pgfpathcurveto{\pgfqpoint{0.889058in}{1.307848in}}{\pgfqpoint{0.885786in}{1.299948in}}{\pgfqpoint{0.885786in}{1.291711in}}%
\pgfpathcurveto{\pgfqpoint{0.885786in}{1.283475in}}{\pgfqpoint{0.889058in}{1.275575in}}{\pgfqpoint{0.894882in}{1.269751in}}%
\pgfpathcurveto{\pgfqpoint{0.900706in}{1.263927in}}{\pgfqpoint{0.908606in}{1.260655in}}{\pgfqpoint{0.916842in}{1.260655in}}%
\pgfpathclose%
\pgfusepath{stroke,fill}%
\end{pgfscope}%
\begin{pgfscope}%
\pgfpathrectangle{\pgfqpoint{0.100000in}{0.212622in}}{\pgfqpoint{3.696000in}{3.696000in}}%
\pgfusepath{clip}%
\pgfsetbuttcap%
\pgfsetroundjoin%
\definecolor{currentfill}{rgb}{0.121569,0.466667,0.705882}%
\pgfsetfillcolor{currentfill}%
\pgfsetfillopacity{0.620848}%
\pgfsetlinewidth{1.003750pt}%
\definecolor{currentstroke}{rgb}{0.121569,0.466667,0.705882}%
\pgfsetstrokecolor{currentstroke}%
\pgfsetstrokeopacity{0.620848}%
\pgfsetdash{}{0pt}%
\pgfpathmoveto{\pgfqpoint{3.300833in}{2.248757in}}%
\pgfpathcurveto{\pgfqpoint{3.309069in}{2.248757in}}{\pgfqpoint{3.316969in}{2.252029in}}{\pgfqpoint{3.322793in}{2.257853in}}%
\pgfpathcurveto{\pgfqpoint{3.328617in}{2.263677in}}{\pgfqpoint{3.331889in}{2.271577in}}{\pgfqpoint{3.331889in}{2.279814in}}%
\pgfpathcurveto{\pgfqpoint{3.331889in}{2.288050in}}{\pgfqpoint{3.328617in}{2.295950in}}{\pgfqpoint{3.322793in}{2.301774in}}%
\pgfpathcurveto{\pgfqpoint{3.316969in}{2.307598in}}{\pgfqpoint{3.309069in}{2.310870in}}{\pgfqpoint{3.300833in}{2.310870in}}%
\pgfpathcurveto{\pgfqpoint{3.292597in}{2.310870in}}{\pgfqpoint{3.284697in}{2.307598in}}{\pgfqpoint{3.278873in}{2.301774in}}%
\pgfpathcurveto{\pgfqpoint{3.273049in}{2.295950in}}{\pgfqpoint{3.269776in}{2.288050in}}{\pgfqpoint{3.269776in}{2.279814in}}%
\pgfpathcurveto{\pgfqpoint{3.269776in}{2.271577in}}{\pgfqpoint{3.273049in}{2.263677in}}{\pgfqpoint{3.278873in}{2.257853in}}%
\pgfpathcurveto{\pgfqpoint{3.284697in}{2.252029in}}{\pgfqpoint{3.292597in}{2.248757in}}{\pgfqpoint{3.300833in}{2.248757in}}%
\pgfpathclose%
\pgfusepath{stroke,fill}%
\end{pgfscope}%
\begin{pgfscope}%
\pgfpathrectangle{\pgfqpoint{0.100000in}{0.212622in}}{\pgfqpoint{3.696000in}{3.696000in}}%
\pgfusepath{clip}%
\pgfsetbuttcap%
\pgfsetroundjoin%
\definecolor{currentfill}{rgb}{0.121569,0.466667,0.705882}%
\pgfsetfillcolor{currentfill}%
\pgfsetfillopacity{0.621536}%
\pgfsetlinewidth{1.003750pt}%
\definecolor{currentstroke}{rgb}{0.121569,0.466667,0.705882}%
\pgfsetstrokecolor{currentstroke}%
\pgfsetstrokeopacity{0.621536}%
\pgfsetdash{}{0pt}%
\pgfpathmoveto{\pgfqpoint{0.913953in}{1.260210in}}%
\pgfpathcurveto{\pgfqpoint{0.922190in}{1.260210in}}{\pgfqpoint{0.930090in}{1.263483in}}{\pgfqpoint{0.935914in}{1.269307in}}%
\pgfpathcurveto{\pgfqpoint{0.941738in}{1.275131in}}{\pgfqpoint{0.945010in}{1.283031in}}{\pgfqpoint{0.945010in}{1.291267in}}%
\pgfpathcurveto{\pgfqpoint{0.945010in}{1.299503in}}{\pgfqpoint{0.941738in}{1.307403in}}{\pgfqpoint{0.935914in}{1.313227in}}%
\pgfpathcurveto{\pgfqpoint{0.930090in}{1.319051in}}{\pgfqpoint{0.922190in}{1.322323in}}{\pgfqpoint{0.913953in}{1.322323in}}%
\pgfpathcurveto{\pgfqpoint{0.905717in}{1.322323in}}{\pgfqpoint{0.897817in}{1.319051in}}{\pgfqpoint{0.891993in}{1.313227in}}%
\pgfpathcurveto{\pgfqpoint{0.886169in}{1.307403in}}{\pgfqpoint{0.882897in}{1.299503in}}{\pgfqpoint{0.882897in}{1.291267in}}%
\pgfpathcurveto{\pgfqpoint{0.882897in}{1.283031in}}{\pgfqpoint{0.886169in}{1.275131in}}{\pgfqpoint{0.891993in}{1.269307in}}%
\pgfpathcurveto{\pgfqpoint{0.897817in}{1.263483in}}{\pgfqpoint{0.905717in}{1.260210in}}{\pgfqpoint{0.913953in}{1.260210in}}%
\pgfpathclose%
\pgfusepath{stroke,fill}%
\end{pgfscope}%
\begin{pgfscope}%
\pgfpathrectangle{\pgfqpoint{0.100000in}{0.212622in}}{\pgfqpoint{3.696000in}{3.696000in}}%
\pgfusepath{clip}%
\pgfsetbuttcap%
\pgfsetroundjoin%
\definecolor{currentfill}{rgb}{0.121569,0.466667,0.705882}%
\pgfsetfillcolor{currentfill}%
\pgfsetfillopacity{0.623314}%
\pgfsetlinewidth{1.003750pt}%
\definecolor{currentstroke}{rgb}{0.121569,0.466667,0.705882}%
\pgfsetstrokecolor{currentstroke}%
\pgfsetstrokeopacity{0.623314}%
\pgfsetdash{}{0pt}%
\pgfpathmoveto{\pgfqpoint{0.906395in}{1.258032in}}%
\pgfpathcurveto{\pgfqpoint{0.914631in}{1.258032in}}{\pgfqpoint{0.922531in}{1.261304in}}{\pgfqpoint{0.928355in}{1.267128in}}%
\pgfpathcurveto{\pgfqpoint{0.934179in}{1.272952in}}{\pgfqpoint{0.937451in}{1.280852in}}{\pgfqpoint{0.937451in}{1.289089in}}%
\pgfpathcurveto{\pgfqpoint{0.937451in}{1.297325in}}{\pgfqpoint{0.934179in}{1.305225in}}{\pgfqpoint{0.928355in}{1.311049in}}%
\pgfpathcurveto{\pgfqpoint{0.922531in}{1.316873in}}{\pgfqpoint{0.914631in}{1.320145in}}{\pgfqpoint{0.906395in}{1.320145in}}%
\pgfpathcurveto{\pgfqpoint{0.898158in}{1.320145in}}{\pgfqpoint{0.890258in}{1.316873in}}{\pgfqpoint{0.884434in}{1.311049in}}%
\pgfpathcurveto{\pgfqpoint{0.878610in}{1.305225in}}{\pgfqpoint{0.875338in}{1.297325in}}{\pgfqpoint{0.875338in}{1.289089in}}%
\pgfpathcurveto{\pgfqpoint{0.875338in}{1.280852in}}{\pgfqpoint{0.878610in}{1.272952in}}{\pgfqpoint{0.884434in}{1.267128in}}%
\pgfpathcurveto{\pgfqpoint{0.890258in}{1.261304in}}{\pgfqpoint{0.898158in}{1.258032in}}{\pgfqpoint{0.906395in}{1.258032in}}%
\pgfpathclose%
\pgfusepath{stroke,fill}%
\end{pgfscope}%
\begin{pgfscope}%
\pgfpathrectangle{\pgfqpoint{0.100000in}{0.212622in}}{\pgfqpoint{3.696000in}{3.696000in}}%
\pgfusepath{clip}%
\pgfsetbuttcap%
\pgfsetroundjoin%
\definecolor{currentfill}{rgb}{0.121569,0.466667,0.705882}%
\pgfsetfillcolor{currentfill}%
\pgfsetfillopacity{0.624095}%
\pgfsetlinewidth{1.003750pt}%
\definecolor{currentstroke}{rgb}{0.121569,0.466667,0.705882}%
\pgfsetstrokecolor{currentstroke}%
\pgfsetstrokeopacity{0.624095}%
\pgfsetdash{}{0pt}%
\pgfpathmoveto{\pgfqpoint{3.295651in}{2.246681in}}%
\pgfpathcurveto{\pgfqpoint{3.303888in}{2.246681in}}{\pgfqpoint{3.311788in}{2.249954in}}{\pgfqpoint{3.317612in}{2.255778in}}%
\pgfpathcurveto{\pgfqpoint{3.323436in}{2.261602in}}{\pgfqpoint{3.326708in}{2.269502in}}{\pgfqpoint{3.326708in}{2.277738in}}%
\pgfpathcurveto{\pgfqpoint{3.326708in}{2.285974in}}{\pgfqpoint{3.323436in}{2.293874in}}{\pgfqpoint{3.317612in}{2.299698in}}%
\pgfpathcurveto{\pgfqpoint{3.311788in}{2.305522in}}{\pgfqpoint{3.303888in}{2.308794in}}{\pgfqpoint{3.295651in}{2.308794in}}%
\pgfpathcurveto{\pgfqpoint{3.287415in}{2.308794in}}{\pgfqpoint{3.279515in}{2.305522in}}{\pgfqpoint{3.273691in}{2.299698in}}%
\pgfpathcurveto{\pgfqpoint{3.267867in}{2.293874in}}{\pgfqpoint{3.264595in}{2.285974in}}{\pgfqpoint{3.264595in}{2.277738in}}%
\pgfpathcurveto{\pgfqpoint{3.264595in}{2.269502in}}{\pgfqpoint{3.267867in}{2.261602in}}{\pgfqpoint{3.273691in}{2.255778in}}%
\pgfpathcurveto{\pgfqpoint{3.279515in}{2.249954in}}{\pgfqpoint{3.287415in}{2.246681in}}{\pgfqpoint{3.295651in}{2.246681in}}%
\pgfpathclose%
\pgfusepath{stroke,fill}%
\end{pgfscope}%
\begin{pgfscope}%
\pgfpathrectangle{\pgfqpoint{0.100000in}{0.212622in}}{\pgfqpoint{3.696000in}{3.696000in}}%
\pgfusepath{clip}%
\pgfsetbuttcap%
\pgfsetroundjoin%
\definecolor{currentfill}{rgb}{0.121569,0.466667,0.705882}%
\pgfsetfillcolor{currentfill}%
\pgfsetfillopacity{0.626369}%
\pgfsetlinewidth{1.003750pt}%
\definecolor{currentstroke}{rgb}{0.121569,0.466667,0.705882}%
\pgfsetstrokecolor{currentstroke}%
\pgfsetstrokeopacity{0.626369}%
\pgfsetdash{}{0pt}%
\pgfpathmoveto{\pgfqpoint{0.892987in}{1.254184in}}%
\pgfpathcurveto{\pgfqpoint{0.901223in}{1.254184in}}{\pgfqpoint{0.909124in}{1.257456in}}{\pgfqpoint{0.914947in}{1.263280in}}%
\pgfpathcurveto{\pgfqpoint{0.920771in}{1.269104in}}{\pgfqpoint{0.924044in}{1.277004in}}{\pgfqpoint{0.924044in}{1.285240in}}%
\pgfpathcurveto{\pgfqpoint{0.924044in}{1.293477in}}{\pgfqpoint{0.920771in}{1.301377in}}{\pgfqpoint{0.914947in}{1.307201in}}%
\pgfpathcurveto{\pgfqpoint{0.909124in}{1.313024in}}{\pgfqpoint{0.901223in}{1.316297in}}{\pgfqpoint{0.892987in}{1.316297in}}%
\pgfpathcurveto{\pgfqpoint{0.884751in}{1.316297in}}{\pgfqpoint{0.876851in}{1.313024in}}{\pgfqpoint{0.871027in}{1.307201in}}%
\pgfpathcurveto{\pgfqpoint{0.865203in}{1.301377in}}{\pgfqpoint{0.861931in}{1.293477in}}{\pgfqpoint{0.861931in}{1.285240in}}%
\pgfpathcurveto{\pgfqpoint{0.861931in}{1.277004in}}{\pgfqpoint{0.865203in}{1.269104in}}{\pgfqpoint{0.871027in}{1.263280in}}%
\pgfpathcurveto{\pgfqpoint{0.876851in}{1.257456in}}{\pgfqpoint{0.884751in}{1.254184in}}{\pgfqpoint{0.892987in}{1.254184in}}%
\pgfpathclose%
\pgfusepath{stroke,fill}%
\end{pgfscope}%
\begin{pgfscope}%
\pgfpathrectangle{\pgfqpoint{0.100000in}{0.212622in}}{\pgfqpoint{3.696000in}{3.696000in}}%
\pgfusepath{clip}%
\pgfsetbuttcap%
\pgfsetroundjoin%
\definecolor{currentfill}{rgb}{0.121569,0.466667,0.705882}%
\pgfsetfillcolor{currentfill}%
\pgfsetfillopacity{0.626744}%
\pgfsetlinewidth{1.003750pt}%
\definecolor{currentstroke}{rgb}{0.121569,0.466667,0.705882}%
\pgfsetstrokecolor{currentstroke}%
\pgfsetstrokeopacity{0.626744}%
\pgfsetdash{}{0pt}%
\pgfpathmoveto{\pgfqpoint{0.824991in}{1.304263in}}%
\pgfpathcurveto{\pgfqpoint{0.833228in}{1.304263in}}{\pgfqpoint{0.841128in}{1.307535in}}{\pgfqpoint{0.846952in}{1.313359in}}%
\pgfpathcurveto{\pgfqpoint{0.852776in}{1.319183in}}{\pgfqpoint{0.856048in}{1.327083in}}{\pgfqpoint{0.856048in}{1.335319in}}%
\pgfpathcurveto{\pgfqpoint{0.856048in}{1.343556in}}{\pgfqpoint{0.852776in}{1.351456in}}{\pgfqpoint{0.846952in}{1.357280in}}%
\pgfpathcurveto{\pgfqpoint{0.841128in}{1.363103in}}{\pgfqpoint{0.833228in}{1.366376in}}{\pgfqpoint{0.824991in}{1.366376in}}%
\pgfpathcurveto{\pgfqpoint{0.816755in}{1.366376in}}{\pgfqpoint{0.808855in}{1.363103in}}{\pgfqpoint{0.803031in}{1.357280in}}%
\pgfpathcurveto{\pgfqpoint{0.797207in}{1.351456in}}{\pgfqpoint{0.793935in}{1.343556in}}{\pgfqpoint{0.793935in}{1.335319in}}%
\pgfpathcurveto{\pgfqpoint{0.793935in}{1.327083in}}{\pgfqpoint{0.797207in}{1.319183in}}{\pgfqpoint{0.803031in}{1.313359in}}%
\pgfpathcurveto{\pgfqpoint{0.808855in}{1.307535in}}{\pgfqpoint{0.816755in}{1.304263in}}{\pgfqpoint{0.824991in}{1.304263in}}%
\pgfpathclose%
\pgfusepath{stroke,fill}%
\end{pgfscope}%
\begin{pgfscope}%
\pgfpathrectangle{\pgfqpoint{0.100000in}{0.212622in}}{\pgfqpoint{3.696000in}{3.696000in}}%
\pgfusepath{clip}%
\pgfsetbuttcap%
\pgfsetroundjoin%
\definecolor{currentfill}{rgb}{0.121569,0.466667,0.705882}%
\pgfsetfillcolor{currentfill}%
\pgfsetfillopacity{0.627814}%
\pgfsetlinewidth{1.003750pt}%
\definecolor{currentstroke}{rgb}{0.121569,0.466667,0.705882}%
\pgfsetstrokecolor{currentstroke}%
\pgfsetstrokeopacity{0.627814}%
\pgfsetdash{}{0pt}%
\pgfpathmoveto{\pgfqpoint{3.287113in}{2.243309in}}%
\pgfpathcurveto{\pgfqpoint{3.295349in}{2.243309in}}{\pgfqpoint{3.303249in}{2.246581in}}{\pgfqpoint{3.309073in}{2.252405in}}%
\pgfpathcurveto{\pgfqpoint{3.314897in}{2.258229in}}{\pgfqpoint{3.318169in}{2.266129in}}{\pgfqpoint{3.318169in}{2.274365in}}%
\pgfpathcurveto{\pgfqpoint{3.318169in}{2.282602in}}{\pgfqpoint{3.314897in}{2.290502in}}{\pgfqpoint{3.309073in}{2.296326in}}%
\pgfpathcurveto{\pgfqpoint{3.303249in}{2.302150in}}{\pgfqpoint{3.295349in}{2.305422in}}{\pgfqpoint{3.287113in}{2.305422in}}%
\pgfpathcurveto{\pgfqpoint{3.278876in}{2.305422in}}{\pgfqpoint{3.270976in}{2.302150in}}{\pgfqpoint{3.265152in}{2.296326in}}%
\pgfpathcurveto{\pgfqpoint{3.259328in}{2.290502in}}{\pgfqpoint{3.256056in}{2.282602in}}{\pgfqpoint{3.256056in}{2.274365in}}%
\pgfpathcurveto{\pgfqpoint{3.256056in}{2.266129in}}{\pgfqpoint{3.259328in}{2.258229in}}{\pgfqpoint{3.265152in}{2.252405in}}%
\pgfpathcurveto{\pgfqpoint{3.270976in}{2.246581in}}{\pgfqpoint{3.278876in}{2.243309in}}{\pgfqpoint{3.287113in}{2.243309in}}%
\pgfpathclose%
\pgfusepath{stroke,fill}%
\end{pgfscope}%
\begin{pgfscope}%
\pgfpathrectangle{\pgfqpoint{0.100000in}{0.212622in}}{\pgfqpoint{3.696000in}{3.696000in}}%
\pgfusepath{clip}%
\pgfsetbuttcap%
\pgfsetroundjoin%
\definecolor{currentfill}{rgb}{0.121569,0.466667,0.705882}%
\pgfsetfillcolor{currentfill}%
\pgfsetfillopacity{0.630592}%
\pgfsetlinewidth{1.003750pt}%
\definecolor{currentstroke}{rgb}{0.121569,0.466667,0.705882}%
\pgfsetstrokecolor{currentstroke}%
\pgfsetstrokeopacity{0.630592}%
\pgfsetdash{}{0pt}%
\pgfpathmoveto{\pgfqpoint{0.875380in}{1.249488in}}%
\pgfpathcurveto{\pgfqpoint{0.883616in}{1.249488in}}{\pgfqpoint{0.891516in}{1.252761in}}{\pgfqpoint{0.897340in}{1.258585in}}%
\pgfpathcurveto{\pgfqpoint{0.903164in}{1.264409in}}{\pgfqpoint{0.906436in}{1.272309in}}{\pgfqpoint{0.906436in}{1.280545in}}%
\pgfpathcurveto{\pgfqpoint{0.906436in}{1.288781in}}{\pgfqpoint{0.903164in}{1.296681in}}{\pgfqpoint{0.897340in}{1.302505in}}%
\pgfpathcurveto{\pgfqpoint{0.891516in}{1.308329in}}{\pgfqpoint{0.883616in}{1.311601in}}{\pgfqpoint{0.875380in}{1.311601in}}%
\pgfpathcurveto{\pgfqpoint{0.867144in}{1.311601in}}{\pgfqpoint{0.859244in}{1.308329in}}{\pgfqpoint{0.853420in}{1.302505in}}%
\pgfpathcurveto{\pgfqpoint{0.847596in}{1.296681in}}{\pgfqpoint{0.844323in}{1.288781in}}{\pgfqpoint{0.844323in}{1.280545in}}%
\pgfpathcurveto{\pgfqpoint{0.844323in}{1.272309in}}{\pgfqpoint{0.847596in}{1.264409in}}{\pgfqpoint{0.853420in}{1.258585in}}%
\pgfpathcurveto{\pgfqpoint{0.859244in}{1.252761in}}{\pgfqpoint{0.867144in}{1.249488in}}{\pgfqpoint{0.875380in}{1.249488in}}%
\pgfpathclose%
\pgfusepath{stroke,fill}%
\end{pgfscope}%
\begin{pgfscope}%
\pgfpathrectangle{\pgfqpoint{0.100000in}{0.212622in}}{\pgfqpoint{3.696000in}{3.696000in}}%
\pgfusepath{clip}%
\pgfsetbuttcap%
\pgfsetroundjoin%
\definecolor{currentfill}{rgb}{0.121569,0.466667,0.705882}%
\pgfsetfillcolor{currentfill}%
\pgfsetfillopacity{0.631050}%
\pgfsetlinewidth{1.003750pt}%
\definecolor{currentstroke}{rgb}{0.121569,0.466667,0.705882}%
\pgfsetstrokecolor{currentstroke}%
\pgfsetstrokeopacity{0.631050}%
\pgfsetdash{}{0pt}%
\pgfpathmoveto{\pgfqpoint{0.807483in}{1.299048in}}%
\pgfpathcurveto{\pgfqpoint{0.815719in}{1.299048in}}{\pgfqpoint{0.823619in}{1.302321in}}{\pgfqpoint{0.829443in}{1.308145in}}%
\pgfpathcurveto{\pgfqpoint{0.835267in}{1.313969in}}{\pgfqpoint{0.838539in}{1.321869in}}{\pgfqpoint{0.838539in}{1.330105in}}%
\pgfpathcurveto{\pgfqpoint{0.838539in}{1.338341in}}{\pgfqpoint{0.835267in}{1.346241in}}{\pgfqpoint{0.829443in}{1.352065in}}%
\pgfpathcurveto{\pgfqpoint{0.823619in}{1.357889in}}{\pgfqpoint{0.815719in}{1.361161in}}{\pgfqpoint{0.807483in}{1.361161in}}%
\pgfpathcurveto{\pgfqpoint{0.799246in}{1.361161in}}{\pgfqpoint{0.791346in}{1.357889in}}{\pgfqpoint{0.785522in}{1.352065in}}%
\pgfpathcurveto{\pgfqpoint{0.779698in}{1.346241in}}{\pgfqpoint{0.776426in}{1.338341in}}{\pgfqpoint{0.776426in}{1.330105in}}%
\pgfpathcurveto{\pgfqpoint{0.776426in}{1.321869in}}{\pgfqpoint{0.779698in}{1.313969in}}{\pgfqpoint{0.785522in}{1.308145in}}%
\pgfpathcurveto{\pgfqpoint{0.791346in}{1.302321in}}{\pgfqpoint{0.799246in}{1.299048in}}{\pgfqpoint{0.807483in}{1.299048in}}%
\pgfpathclose%
\pgfusepath{stroke,fill}%
\end{pgfscope}%
\begin{pgfscope}%
\pgfpathrectangle{\pgfqpoint{0.100000in}{0.212622in}}{\pgfqpoint{3.696000in}{3.696000in}}%
\pgfusepath{clip}%
\pgfsetbuttcap%
\pgfsetroundjoin%
\definecolor{currentfill}{rgb}{0.121569,0.466667,0.705882}%
\pgfsetfillcolor{currentfill}%
\pgfsetfillopacity{0.632731}%
\pgfsetlinewidth{1.003750pt}%
\definecolor{currentstroke}{rgb}{0.121569,0.466667,0.705882}%
\pgfsetstrokecolor{currentstroke}%
\pgfsetstrokeopacity{0.632731}%
\pgfsetdash{}{0pt}%
\pgfpathmoveto{\pgfqpoint{3.277214in}{2.239485in}}%
\pgfpathcurveto{\pgfqpoint{3.285450in}{2.239485in}}{\pgfqpoint{3.293350in}{2.242757in}}{\pgfqpoint{3.299174in}{2.248581in}}%
\pgfpathcurveto{\pgfqpoint{3.304998in}{2.254405in}}{\pgfqpoint{3.308270in}{2.262305in}}{\pgfqpoint{3.308270in}{2.270541in}}%
\pgfpathcurveto{\pgfqpoint{3.308270in}{2.278778in}}{\pgfqpoint{3.304998in}{2.286678in}}{\pgfqpoint{3.299174in}{2.292502in}}%
\pgfpathcurveto{\pgfqpoint{3.293350in}{2.298326in}}{\pgfqpoint{3.285450in}{2.301598in}}{\pgfqpoint{3.277214in}{2.301598in}}%
\pgfpathcurveto{\pgfqpoint{3.268977in}{2.301598in}}{\pgfqpoint{3.261077in}{2.298326in}}{\pgfqpoint{3.255253in}{2.292502in}}%
\pgfpathcurveto{\pgfqpoint{3.249429in}{2.286678in}}{\pgfqpoint{3.246157in}{2.278778in}}{\pgfqpoint{3.246157in}{2.270541in}}%
\pgfpathcurveto{\pgfqpoint{3.246157in}{2.262305in}}{\pgfqpoint{3.249429in}{2.254405in}}{\pgfqpoint{3.255253in}{2.248581in}}%
\pgfpathcurveto{\pgfqpoint{3.261077in}{2.242757in}}{\pgfqpoint{3.268977in}{2.239485in}}{\pgfqpoint{3.277214in}{2.239485in}}%
\pgfpathclose%
\pgfusepath{stroke,fill}%
\end{pgfscope}%
\begin{pgfscope}%
\pgfpathrectangle{\pgfqpoint{0.100000in}{0.212622in}}{\pgfqpoint{3.696000in}{3.696000in}}%
\pgfusepath{clip}%
\pgfsetbuttcap%
\pgfsetroundjoin%
\definecolor{currentfill}{rgb}{0.121569,0.466667,0.705882}%
\pgfsetfillcolor{currentfill}%
\pgfsetfillopacity{0.633566}%
\pgfsetlinewidth{1.003750pt}%
\definecolor{currentstroke}{rgb}{0.121569,0.466667,0.705882}%
\pgfsetstrokecolor{currentstroke}%
\pgfsetstrokeopacity{0.633566}%
\pgfsetdash{}{0pt}%
\pgfpathmoveto{\pgfqpoint{0.802360in}{1.290259in}}%
\pgfpathcurveto{\pgfqpoint{0.810596in}{1.290259in}}{\pgfqpoint{0.818496in}{1.293531in}}{\pgfqpoint{0.824320in}{1.299355in}}%
\pgfpathcurveto{\pgfqpoint{0.830144in}{1.305179in}}{\pgfqpoint{0.833416in}{1.313079in}}{\pgfqpoint{0.833416in}{1.321315in}}%
\pgfpathcurveto{\pgfqpoint{0.833416in}{1.329551in}}{\pgfqpoint{0.830144in}{1.337451in}}{\pgfqpoint{0.824320in}{1.343275in}}%
\pgfpathcurveto{\pgfqpoint{0.818496in}{1.349099in}}{\pgfqpoint{0.810596in}{1.352372in}}{\pgfqpoint{0.802360in}{1.352372in}}%
\pgfpathcurveto{\pgfqpoint{0.794124in}{1.352372in}}{\pgfqpoint{0.786224in}{1.349099in}}{\pgfqpoint{0.780400in}{1.343275in}}%
\pgfpathcurveto{\pgfqpoint{0.774576in}{1.337451in}}{\pgfqpoint{0.771303in}{1.329551in}}{\pgfqpoint{0.771303in}{1.321315in}}%
\pgfpathcurveto{\pgfqpoint{0.771303in}{1.313079in}}{\pgfqpoint{0.774576in}{1.305179in}}{\pgfqpoint{0.780400in}{1.299355in}}%
\pgfpathcurveto{\pgfqpoint{0.786224in}{1.293531in}}{\pgfqpoint{0.794124in}{1.290259in}}{\pgfqpoint{0.802360in}{1.290259in}}%
\pgfpathclose%
\pgfusepath{stroke,fill}%
\end{pgfscope}%
\begin{pgfscope}%
\pgfpathrectangle{\pgfqpoint{0.100000in}{0.212622in}}{\pgfqpoint{3.696000in}{3.696000in}}%
\pgfusepath{clip}%
\pgfsetbuttcap%
\pgfsetroundjoin%
\definecolor{currentfill}{rgb}{0.121569,0.466667,0.705882}%
\pgfsetfillcolor{currentfill}%
\pgfsetfillopacity{0.634704}%
\pgfsetlinewidth{1.003750pt}%
\definecolor{currentstroke}{rgb}{0.121569,0.466667,0.705882}%
\pgfsetstrokecolor{currentstroke}%
\pgfsetstrokeopacity{0.634704}%
\pgfsetdash{}{0pt}%
\pgfpathmoveto{\pgfqpoint{0.799605in}{1.287167in}}%
\pgfpathcurveto{\pgfqpoint{0.807842in}{1.287167in}}{\pgfqpoint{0.815742in}{1.290439in}}{\pgfqpoint{0.821566in}{1.296263in}}%
\pgfpathcurveto{\pgfqpoint{0.827390in}{1.302087in}}{\pgfqpoint{0.830662in}{1.309987in}}{\pgfqpoint{0.830662in}{1.318223in}}%
\pgfpathcurveto{\pgfqpoint{0.830662in}{1.326460in}}{\pgfqpoint{0.827390in}{1.334360in}}{\pgfqpoint{0.821566in}{1.340184in}}%
\pgfpathcurveto{\pgfqpoint{0.815742in}{1.346008in}}{\pgfqpoint{0.807842in}{1.349280in}}{\pgfqpoint{0.799605in}{1.349280in}}%
\pgfpathcurveto{\pgfqpoint{0.791369in}{1.349280in}}{\pgfqpoint{0.783469in}{1.346008in}}{\pgfqpoint{0.777645in}{1.340184in}}%
\pgfpathcurveto{\pgfqpoint{0.771821in}{1.334360in}}{\pgfqpoint{0.768549in}{1.326460in}}{\pgfqpoint{0.768549in}{1.318223in}}%
\pgfpathcurveto{\pgfqpoint{0.768549in}{1.309987in}}{\pgfqpoint{0.771821in}{1.302087in}}{\pgfqpoint{0.777645in}{1.296263in}}%
\pgfpathcurveto{\pgfqpoint{0.783469in}{1.290439in}}{\pgfqpoint{0.791369in}{1.287167in}}{\pgfqpoint{0.799605in}{1.287167in}}%
\pgfpathclose%
\pgfusepath{stroke,fill}%
\end{pgfscope}%
\begin{pgfscope}%
\pgfpathrectangle{\pgfqpoint{0.100000in}{0.212622in}}{\pgfqpoint{3.696000in}{3.696000in}}%
\pgfusepath{clip}%
\pgfsetbuttcap%
\pgfsetroundjoin%
\definecolor{currentfill}{rgb}{0.121569,0.466667,0.705882}%
\pgfsetfillcolor{currentfill}%
\pgfsetfillopacity{0.635213}%
\pgfsetlinewidth{1.003750pt}%
\definecolor{currentstroke}{rgb}{0.121569,0.466667,0.705882}%
\pgfsetstrokecolor{currentstroke}%
\pgfsetstrokeopacity{0.635213}%
\pgfsetdash{}{0pt}%
\pgfpathmoveto{\pgfqpoint{3.271036in}{2.236792in}}%
\pgfpathcurveto{\pgfqpoint{3.279272in}{2.236792in}}{\pgfqpoint{3.287172in}{2.240064in}}{\pgfqpoint{3.292996in}{2.245888in}}%
\pgfpathcurveto{\pgfqpoint{3.298820in}{2.251712in}}{\pgfqpoint{3.302093in}{2.259612in}}{\pgfqpoint{3.302093in}{2.267848in}}%
\pgfpathcurveto{\pgfqpoint{3.302093in}{2.276084in}}{\pgfqpoint{3.298820in}{2.283984in}}{\pgfqpoint{3.292996in}{2.289808in}}%
\pgfpathcurveto{\pgfqpoint{3.287172in}{2.295632in}}{\pgfqpoint{3.279272in}{2.298905in}}{\pgfqpoint{3.271036in}{2.298905in}}%
\pgfpathcurveto{\pgfqpoint{3.262800in}{2.298905in}}{\pgfqpoint{3.254900in}{2.295632in}}{\pgfqpoint{3.249076in}{2.289808in}}%
\pgfpathcurveto{\pgfqpoint{3.243252in}{2.283984in}}{\pgfqpoint{3.239980in}{2.276084in}}{\pgfqpoint{3.239980in}{2.267848in}}%
\pgfpathcurveto{\pgfqpoint{3.239980in}{2.259612in}}{\pgfqpoint{3.243252in}{2.251712in}}{\pgfqpoint{3.249076in}{2.245888in}}%
\pgfpathcurveto{\pgfqpoint{3.254900in}{2.240064in}}{\pgfqpoint{3.262800in}{2.236792in}}{\pgfqpoint{3.271036in}{2.236792in}}%
\pgfpathclose%
\pgfusepath{stroke,fill}%
\end{pgfscope}%
\begin{pgfscope}%
\pgfpathrectangle{\pgfqpoint{0.100000in}{0.212622in}}{\pgfqpoint{3.696000in}{3.696000in}}%
\pgfusepath{clip}%
\pgfsetbuttcap%
\pgfsetroundjoin%
\definecolor{currentfill}{rgb}{0.121569,0.466667,0.705882}%
\pgfsetfillcolor{currentfill}%
\pgfsetfillopacity{0.635375}%
\pgfsetlinewidth{1.003750pt}%
\definecolor{currentstroke}{rgb}{0.121569,0.466667,0.705882}%
\pgfsetstrokecolor{currentstroke}%
\pgfsetstrokeopacity{0.635375}%
\pgfsetdash{}{0pt}%
\pgfpathmoveto{\pgfqpoint{0.851587in}{1.243957in}}%
\pgfpathcurveto{\pgfqpoint{0.859823in}{1.243957in}}{\pgfqpoint{0.867723in}{1.247230in}}{\pgfqpoint{0.873547in}{1.253054in}}%
\pgfpathcurveto{\pgfqpoint{0.879371in}{1.258878in}}{\pgfqpoint{0.882644in}{1.266778in}}{\pgfqpoint{0.882644in}{1.275014in}}%
\pgfpathcurveto{\pgfqpoint{0.882644in}{1.283250in}}{\pgfqpoint{0.879371in}{1.291150in}}{\pgfqpoint{0.873547in}{1.296974in}}%
\pgfpathcurveto{\pgfqpoint{0.867723in}{1.302798in}}{\pgfqpoint{0.859823in}{1.306070in}}{\pgfqpoint{0.851587in}{1.306070in}}%
\pgfpathcurveto{\pgfqpoint{0.843351in}{1.306070in}}{\pgfqpoint{0.835451in}{1.302798in}}{\pgfqpoint{0.829627in}{1.296974in}}%
\pgfpathcurveto{\pgfqpoint{0.823803in}{1.291150in}}{\pgfqpoint{0.820531in}{1.283250in}}{\pgfqpoint{0.820531in}{1.275014in}}%
\pgfpathcurveto{\pgfqpoint{0.820531in}{1.266778in}}{\pgfqpoint{0.823803in}{1.258878in}}{\pgfqpoint{0.829627in}{1.253054in}}%
\pgfpathcurveto{\pgfqpoint{0.835451in}{1.247230in}}{\pgfqpoint{0.843351in}{1.243957in}}{\pgfqpoint{0.851587in}{1.243957in}}%
\pgfpathclose%
\pgfusepath{stroke,fill}%
\end{pgfscope}%
\begin{pgfscope}%
\pgfpathrectangle{\pgfqpoint{0.100000in}{0.212622in}}{\pgfqpoint{3.696000in}{3.696000in}}%
\pgfusepath{clip}%
\pgfsetbuttcap%
\pgfsetroundjoin%
\definecolor{currentfill}{rgb}{0.121569,0.466667,0.705882}%
\pgfsetfillcolor{currentfill}%
\pgfsetfillopacity{0.636391}%
\pgfsetlinewidth{1.003750pt}%
\definecolor{currentstroke}{rgb}{0.121569,0.466667,0.705882}%
\pgfsetstrokecolor{currentstroke}%
\pgfsetstrokeopacity{0.636391}%
\pgfsetdash{}{0pt}%
\pgfpathmoveto{\pgfqpoint{0.793122in}{1.280919in}}%
\pgfpathcurveto{\pgfqpoint{0.801359in}{1.280919in}}{\pgfqpoint{0.809259in}{1.284191in}}{\pgfqpoint{0.815083in}{1.290015in}}%
\pgfpathcurveto{\pgfqpoint{0.820906in}{1.295839in}}{\pgfqpoint{0.824179in}{1.303739in}}{\pgfqpoint{0.824179in}{1.311976in}}%
\pgfpathcurveto{\pgfqpoint{0.824179in}{1.320212in}}{\pgfqpoint{0.820906in}{1.328112in}}{\pgfqpoint{0.815083in}{1.333936in}}%
\pgfpathcurveto{\pgfqpoint{0.809259in}{1.339760in}}{\pgfqpoint{0.801359in}{1.343032in}}{\pgfqpoint{0.793122in}{1.343032in}}%
\pgfpathcurveto{\pgfqpoint{0.784886in}{1.343032in}}{\pgfqpoint{0.776986in}{1.339760in}}{\pgfqpoint{0.771162in}{1.333936in}}%
\pgfpathcurveto{\pgfqpoint{0.765338in}{1.328112in}}{\pgfqpoint{0.762066in}{1.320212in}}{\pgfqpoint{0.762066in}{1.311976in}}%
\pgfpathcurveto{\pgfqpoint{0.762066in}{1.303739in}}{\pgfqpoint{0.765338in}{1.295839in}}{\pgfqpoint{0.771162in}{1.290015in}}%
\pgfpathcurveto{\pgfqpoint{0.776986in}{1.284191in}}{\pgfqpoint{0.784886in}{1.280919in}}{\pgfqpoint{0.793122in}{1.280919in}}%
\pgfpathclose%
\pgfusepath{stroke,fill}%
\end{pgfscope}%
\begin{pgfscope}%
\pgfpathrectangle{\pgfqpoint{0.100000in}{0.212622in}}{\pgfqpoint{3.696000in}{3.696000in}}%
\pgfusepath{clip}%
\pgfsetbuttcap%
\pgfsetroundjoin%
\definecolor{currentfill}{rgb}{0.121569,0.466667,0.705882}%
\pgfsetfillcolor{currentfill}%
\pgfsetfillopacity{0.636703}%
\pgfsetlinewidth{1.003750pt}%
\definecolor{currentstroke}{rgb}{0.121569,0.466667,0.705882}%
\pgfsetstrokecolor{currentstroke}%
\pgfsetstrokeopacity{0.636703}%
\pgfsetdash{}{0pt}%
\pgfpathmoveto{\pgfqpoint{3.267853in}{2.235862in}}%
\pgfpathcurveto{\pgfqpoint{3.276089in}{2.235862in}}{\pgfqpoint{3.283989in}{2.239134in}}{\pgfqpoint{3.289813in}{2.244958in}}%
\pgfpathcurveto{\pgfqpoint{3.295637in}{2.250782in}}{\pgfqpoint{3.298910in}{2.258682in}}{\pgfqpoint{3.298910in}{2.266918in}}%
\pgfpathcurveto{\pgfqpoint{3.298910in}{2.275155in}}{\pgfqpoint{3.295637in}{2.283055in}}{\pgfqpoint{3.289813in}{2.288879in}}%
\pgfpathcurveto{\pgfqpoint{3.283989in}{2.294703in}}{\pgfqpoint{3.276089in}{2.297975in}}{\pgfqpoint{3.267853in}{2.297975in}}%
\pgfpathcurveto{\pgfqpoint{3.259617in}{2.297975in}}{\pgfqpoint{3.251717in}{2.294703in}}{\pgfqpoint{3.245893in}{2.288879in}}%
\pgfpathcurveto{\pgfqpoint{3.240069in}{2.283055in}}{\pgfqpoint{3.236797in}{2.275155in}}{\pgfqpoint{3.236797in}{2.266918in}}%
\pgfpathcurveto{\pgfqpoint{3.236797in}{2.258682in}}{\pgfqpoint{3.240069in}{2.250782in}}{\pgfqpoint{3.245893in}{2.244958in}}%
\pgfpathcurveto{\pgfqpoint{3.251717in}{2.239134in}}{\pgfqpoint{3.259617in}{2.235862in}}{\pgfqpoint{3.267853in}{2.235862in}}%
\pgfpathclose%
\pgfusepath{stroke,fill}%
\end{pgfscope}%
\begin{pgfscope}%
\pgfpathrectangle{\pgfqpoint{0.100000in}{0.212622in}}{\pgfqpoint{3.696000in}{3.696000in}}%
\pgfusepath{clip}%
\pgfsetbuttcap%
\pgfsetroundjoin%
\definecolor{currentfill}{rgb}{0.121569,0.466667,0.705882}%
\pgfsetfillcolor{currentfill}%
\pgfsetfillopacity{0.638789}%
\pgfsetlinewidth{1.003750pt}%
\definecolor{currentstroke}{rgb}{0.121569,0.466667,0.705882}%
\pgfsetstrokecolor{currentstroke}%
\pgfsetstrokeopacity{0.638789}%
\pgfsetdash{}{0pt}%
\pgfpathmoveto{\pgfqpoint{3.263091in}{2.234798in}}%
\pgfpathcurveto{\pgfqpoint{3.271328in}{2.234798in}}{\pgfqpoint{3.279228in}{2.238070in}}{\pgfqpoint{3.285052in}{2.243894in}}%
\pgfpathcurveto{\pgfqpoint{3.290876in}{2.249718in}}{\pgfqpoint{3.294148in}{2.257618in}}{\pgfqpoint{3.294148in}{2.265854in}}%
\pgfpathcurveto{\pgfqpoint{3.294148in}{2.274090in}}{\pgfqpoint{3.290876in}{2.281991in}}{\pgfqpoint{3.285052in}{2.287814in}}%
\pgfpathcurveto{\pgfqpoint{3.279228in}{2.293638in}}{\pgfqpoint{3.271328in}{2.296911in}}{\pgfqpoint{3.263091in}{2.296911in}}%
\pgfpathcurveto{\pgfqpoint{3.254855in}{2.296911in}}{\pgfqpoint{3.246955in}{2.293638in}}{\pgfqpoint{3.241131in}{2.287814in}}%
\pgfpathcurveto{\pgfqpoint{3.235307in}{2.281991in}}{\pgfqpoint{3.232035in}{2.274090in}}{\pgfqpoint{3.232035in}{2.265854in}}%
\pgfpathcurveto{\pgfqpoint{3.232035in}{2.257618in}}{\pgfqpoint{3.235307in}{2.249718in}}{\pgfqpoint{3.241131in}{2.243894in}}%
\pgfpathcurveto{\pgfqpoint{3.246955in}{2.238070in}}{\pgfqpoint{3.254855in}{2.234798in}}{\pgfqpoint{3.263091in}{2.234798in}}%
\pgfpathclose%
\pgfusepath{stroke,fill}%
\end{pgfscope}%
\begin{pgfscope}%
\pgfpathrectangle{\pgfqpoint{0.100000in}{0.212622in}}{\pgfqpoint{3.696000in}{3.696000in}}%
\pgfusepath{clip}%
\pgfsetbuttcap%
\pgfsetroundjoin%
\definecolor{currentfill}{rgb}{0.121569,0.466667,0.705882}%
\pgfsetfillcolor{currentfill}%
\pgfsetfillopacity{0.639776}%
\pgfsetlinewidth{1.003750pt}%
\definecolor{currentstroke}{rgb}{0.121569,0.466667,0.705882}%
\pgfsetstrokecolor{currentstroke}%
\pgfsetstrokeopacity{0.639776}%
\pgfsetdash{}{0pt}%
\pgfpathmoveto{\pgfqpoint{3.260143in}{2.233619in}}%
\pgfpathcurveto{\pgfqpoint{3.268379in}{2.233619in}}{\pgfqpoint{3.276279in}{2.236891in}}{\pgfqpoint{3.282103in}{2.242715in}}%
\pgfpathcurveto{\pgfqpoint{3.287927in}{2.248539in}}{\pgfqpoint{3.291199in}{2.256439in}}{\pgfqpoint{3.291199in}{2.264675in}}%
\pgfpathcurveto{\pgfqpoint{3.291199in}{2.272911in}}{\pgfqpoint{3.287927in}{2.280811in}}{\pgfqpoint{3.282103in}{2.286635in}}%
\pgfpathcurveto{\pgfqpoint{3.276279in}{2.292459in}}{\pgfqpoint{3.268379in}{2.295732in}}{\pgfqpoint{3.260143in}{2.295732in}}%
\pgfpathcurveto{\pgfqpoint{3.251907in}{2.295732in}}{\pgfqpoint{3.244007in}{2.292459in}}{\pgfqpoint{3.238183in}{2.286635in}}%
\pgfpathcurveto{\pgfqpoint{3.232359in}{2.280811in}}{\pgfqpoint{3.229086in}{2.272911in}}{\pgfqpoint{3.229086in}{2.264675in}}%
\pgfpathcurveto{\pgfqpoint{3.229086in}{2.256439in}}{\pgfqpoint{3.232359in}{2.248539in}}{\pgfqpoint{3.238183in}{2.242715in}}%
\pgfpathcurveto{\pgfqpoint{3.244007in}{2.236891in}}{\pgfqpoint{3.251907in}{2.233619in}}{\pgfqpoint{3.260143in}{2.233619in}}%
\pgfpathclose%
\pgfusepath{stroke,fill}%
\end{pgfscope}%
\begin{pgfscope}%
\pgfpathrectangle{\pgfqpoint{0.100000in}{0.212622in}}{\pgfqpoint{3.696000in}{3.696000in}}%
\pgfusepath{clip}%
\pgfsetbuttcap%
\pgfsetroundjoin%
\definecolor{currentfill}{rgb}{0.121569,0.466667,0.705882}%
\pgfsetfillcolor{currentfill}%
\pgfsetfillopacity{0.640274}%
\pgfsetlinewidth{1.003750pt}%
\definecolor{currentstroke}{rgb}{0.121569,0.466667,0.705882}%
\pgfsetstrokecolor{currentstroke}%
\pgfsetstrokeopacity{0.640274}%
\pgfsetdash{}{0pt}%
\pgfpathmoveto{\pgfqpoint{0.786323in}{1.269630in}}%
\pgfpathcurveto{\pgfqpoint{0.794559in}{1.269630in}}{\pgfqpoint{0.802459in}{1.272902in}}{\pgfqpoint{0.808283in}{1.278726in}}%
\pgfpathcurveto{\pgfqpoint{0.814107in}{1.284550in}}{\pgfqpoint{0.817380in}{1.292450in}}{\pgfqpoint{0.817380in}{1.300686in}}%
\pgfpathcurveto{\pgfqpoint{0.817380in}{1.308923in}}{\pgfqpoint{0.814107in}{1.316823in}}{\pgfqpoint{0.808283in}{1.322647in}}%
\pgfpathcurveto{\pgfqpoint{0.802459in}{1.328471in}}{\pgfqpoint{0.794559in}{1.331743in}}{\pgfqpoint{0.786323in}{1.331743in}}%
\pgfpathcurveto{\pgfqpoint{0.778087in}{1.331743in}}{\pgfqpoint{0.770187in}{1.328471in}}{\pgfqpoint{0.764363in}{1.322647in}}%
\pgfpathcurveto{\pgfqpoint{0.758539in}{1.316823in}}{\pgfqpoint{0.755267in}{1.308923in}}{\pgfqpoint{0.755267in}{1.300686in}}%
\pgfpathcurveto{\pgfqpoint{0.755267in}{1.292450in}}{\pgfqpoint{0.758539in}{1.284550in}}{\pgfqpoint{0.764363in}{1.278726in}}%
\pgfpathcurveto{\pgfqpoint{0.770187in}{1.272902in}}{\pgfqpoint{0.778087in}{1.269630in}}{\pgfqpoint{0.786323in}{1.269630in}}%
\pgfpathclose%
\pgfusepath{stroke,fill}%
\end{pgfscope}%
\begin{pgfscope}%
\pgfpathrectangle{\pgfqpoint{0.100000in}{0.212622in}}{\pgfqpoint{3.696000in}{3.696000in}}%
\pgfusepath{clip}%
\pgfsetbuttcap%
\pgfsetroundjoin%
\definecolor{currentfill}{rgb}{0.121569,0.466667,0.705882}%
\pgfsetfillcolor{currentfill}%
\pgfsetfillopacity{0.641450}%
\pgfsetlinewidth{1.003750pt}%
\definecolor{currentstroke}{rgb}{0.121569,0.466667,0.705882}%
\pgfsetstrokecolor{currentstroke}%
\pgfsetstrokeopacity{0.641450}%
\pgfsetdash{}{0pt}%
\pgfpathmoveto{\pgfqpoint{0.824958in}{1.238196in}}%
\pgfpathcurveto{\pgfqpoint{0.833194in}{1.238196in}}{\pgfqpoint{0.841094in}{1.241468in}}{\pgfqpoint{0.846918in}{1.247292in}}%
\pgfpathcurveto{\pgfqpoint{0.852742in}{1.253116in}}{\pgfqpoint{0.856015in}{1.261016in}}{\pgfqpoint{0.856015in}{1.269252in}}%
\pgfpathcurveto{\pgfqpoint{0.856015in}{1.277488in}}{\pgfqpoint{0.852742in}{1.285388in}}{\pgfqpoint{0.846918in}{1.291212in}}%
\pgfpathcurveto{\pgfqpoint{0.841094in}{1.297036in}}{\pgfqpoint{0.833194in}{1.300309in}}{\pgfqpoint{0.824958in}{1.300309in}}%
\pgfpathcurveto{\pgfqpoint{0.816722in}{1.300309in}}{\pgfqpoint{0.808822in}{1.297036in}}{\pgfqpoint{0.802998in}{1.291212in}}%
\pgfpathcurveto{\pgfqpoint{0.797174in}{1.285388in}}{\pgfqpoint{0.793902in}{1.277488in}}{\pgfqpoint{0.793902in}{1.269252in}}%
\pgfpathcurveto{\pgfqpoint{0.793902in}{1.261016in}}{\pgfqpoint{0.797174in}{1.253116in}}{\pgfqpoint{0.802998in}{1.247292in}}%
\pgfpathcurveto{\pgfqpoint{0.808822in}{1.241468in}}{\pgfqpoint{0.816722in}{1.238196in}}{\pgfqpoint{0.824958in}{1.238196in}}%
\pgfpathclose%
\pgfusepath{stroke,fill}%
\end{pgfscope}%
\begin{pgfscope}%
\pgfpathrectangle{\pgfqpoint{0.100000in}{0.212622in}}{\pgfqpoint{3.696000in}{3.696000in}}%
\pgfusepath{clip}%
\pgfsetbuttcap%
\pgfsetroundjoin%
\definecolor{currentfill}{rgb}{0.121569,0.466667,0.705882}%
\pgfsetfillcolor{currentfill}%
\pgfsetfillopacity{0.641661}%
\pgfsetlinewidth{1.003750pt}%
\definecolor{currentstroke}{rgb}{0.121569,0.466667,0.705882}%
\pgfsetstrokecolor{currentstroke}%
\pgfsetstrokeopacity{0.641661}%
\pgfsetdash{}{0pt}%
\pgfpathmoveto{\pgfqpoint{3.255978in}{2.231338in}}%
\pgfpathcurveto{\pgfqpoint{3.264214in}{2.231338in}}{\pgfqpoint{3.272114in}{2.234610in}}{\pgfqpoint{3.277938in}{2.240434in}}%
\pgfpathcurveto{\pgfqpoint{3.283762in}{2.246258in}}{\pgfqpoint{3.287034in}{2.254158in}}{\pgfqpoint{3.287034in}{2.262394in}}%
\pgfpathcurveto{\pgfqpoint{3.287034in}{2.270631in}}{\pgfqpoint{3.283762in}{2.278531in}}{\pgfqpoint{3.277938in}{2.284355in}}%
\pgfpathcurveto{\pgfqpoint{3.272114in}{2.290179in}}{\pgfqpoint{3.264214in}{2.293451in}}{\pgfqpoint{3.255978in}{2.293451in}}%
\pgfpathcurveto{\pgfqpoint{3.247741in}{2.293451in}}{\pgfqpoint{3.239841in}{2.290179in}}{\pgfqpoint{3.234017in}{2.284355in}}%
\pgfpathcurveto{\pgfqpoint{3.228193in}{2.278531in}}{\pgfqpoint{3.224921in}{2.270631in}}{\pgfqpoint{3.224921in}{2.262394in}}%
\pgfpathcurveto{\pgfqpoint{3.224921in}{2.254158in}}{\pgfqpoint{3.228193in}{2.246258in}}{\pgfqpoint{3.234017in}{2.240434in}}%
\pgfpathcurveto{\pgfqpoint{3.239841in}{2.234610in}}{\pgfqpoint{3.247741in}{2.231338in}}{\pgfqpoint{3.255978in}{2.231338in}}%
\pgfpathclose%
\pgfusepath{stroke,fill}%
\end{pgfscope}%
\begin{pgfscope}%
\pgfpathrectangle{\pgfqpoint{0.100000in}{0.212622in}}{\pgfqpoint{3.696000in}{3.696000in}}%
\pgfusepath{clip}%
\pgfsetbuttcap%
\pgfsetroundjoin%
\definecolor{currentfill}{rgb}{0.121569,0.466667,0.705882}%
\pgfsetfillcolor{currentfill}%
\pgfsetfillopacity{0.643860}%
\pgfsetlinewidth{1.003750pt}%
\definecolor{currentstroke}{rgb}{0.121569,0.466667,0.705882}%
\pgfsetstrokecolor{currentstroke}%
\pgfsetstrokeopacity{0.643860}%
\pgfsetdash{}{0pt}%
\pgfpathmoveto{\pgfqpoint{3.247332in}{2.228499in}}%
\pgfpathcurveto{\pgfqpoint{3.255568in}{2.228499in}}{\pgfqpoint{3.263468in}{2.231771in}}{\pgfqpoint{3.269292in}{2.237595in}}%
\pgfpathcurveto{\pgfqpoint{3.275116in}{2.243419in}}{\pgfqpoint{3.278388in}{2.251319in}}{\pgfqpoint{3.278388in}{2.259555in}}%
\pgfpathcurveto{\pgfqpoint{3.278388in}{2.267792in}}{\pgfqpoint{3.275116in}{2.275692in}}{\pgfqpoint{3.269292in}{2.281516in}}%
\pgfpathcurveto{\pgfqpoint{3.263468in}{2.287340in}}{\pgfqpoint{3.255568in}{2.290612in}}{\pgfqpoint{3.247332in}{2.290612in}}%
\pgfpathcurveto{\pgfqpoint{3.239095in}{2.290612in}}{\pgfqpoint{3.231195in}{2.287340in}}{\pgfqpoint{3.225371in}{2.281516in}}%
\pgfpathcurveto{\pgfqpoint{3.219547in}{2.275692in}}{\pgfqpoint{3.216275in}{2.267792in}}{\pgfqpoint{3.216275in}{2.259555in}}%
\pgfpathcurveto{\pgfqpoint{3.216275in}{2.251319in}}{\pgfqpoint{3.219547in}{2.243419in}}{\pgfqpoint{3.225371in}{2.237595in}}%
\pgfpathcurveto{\pgfqpoint{3.231195in}{2.231771in}}{\pgfqpoint{3.239095in}{2.228499in}}{\pgfqpoint{3.247332in}{2.228499in}}%
\pgfpathclose%
\pgfusepath{stroke,fill}%
\end{pgfscope}%
\begin{pgfscope}%
\pgfpathrectangle{\pgfqpoint{0.100000in}{0.212622in}}{\pgfqpoint{3.696000in}{3.696000in}}%
\pgfusepath{clip}%
\pgfsetbuttcap%
\pgfsetroundjoin%
\definecolor{currentfill}{rgb}{0.121569,0.466667,0.705882}%
\pgfsetfillcolor{currentfill}%
\pgfsetfillopacity{0.645406}%
\pgfsetlinewidth{1.003750pt}%
\definecolor{currentstroke}{rgb}{0.121569,0.466667,0.705882}%
\pgfsetstrokecolor{currentstroke}%
\pgfsetstrokeopacity{0.645406}%
\pgfsetdash{}{0pt}%
\pgfpathmoveto{\pgfqpoint{3.243650in}{2.227316in}}%
\pgfpathcurveto{\pgfqpoint{3.251886in}{2.227316in}}{\pgfqpoint{3.259786in}{2.230588in}}{\pgfqpoint{3.265610in}{2.236412in}}%
\pgfpathcurveto{\pgfqpoint{3.271434in}{2.242236in}}{\pgfqpoint{3.274706in}{2.250136in}}{\pgfqpoint{3.274706in}{2.258372in}}%
\pgfpathcurveto{\pgfqpoint{3.274706in}{2.266609in}}{\pgfqpoint{3.271434in}{2.274509in}}{\pgfqpoint{3.265610in}{2.280333in}}%
\pgfpathcurveto{\pgfqpoint{3.259786in}{2.286156in}}{\pgfqpoint{3.251886in}{2.289429in}}{\pgfqpoint{3.243650in}{2.289429in}}%
\pgfpathcurveto{\pgfqpoint{3.235413in}{2.289429in}}{\pgfqpoint{3.227513in}{2.286156in}}{\pgfqpoint{3.221689in}{2.280333in}}%
\pgfpathcurveto{\pgfqpoint{3.215865in}{2.274509in}}{\pgfqpoint{3.212593in}{2.266609in}}{\pgfqpoint{3.212593in}{2.258372in}}%
\pgfpathcurveto{\pgfqpoint{3.212593in}{2.250136in}}{\pgfqpoint{3.215865in}{2.242236in}}{\pgfqpoint{3.221689in}{2.236412in}}%
\pgfpathcurveto{\pgfqpoint{3.227513in}{2.230588in}}{\pgfqpoint{3.235413in}{2.227316in}}{\pgfqpoint{3.243650in}{2.227316in}}%
\pgfpathclose%
\pgfusepath{stroke,fill}%
\end{pgfscope}%
\begin{pgfscope}%
\pgfpathrectangle{\pgfqpoint{0.100000in}{0.212622in}}{\pgfqpoint{3.696000in}{3.696000in}}%
\pgfusepath{clip}%
\pgfsetbuttcap%
\pgfsetroundjoin%
\definecolor{currentfill}{rgb}{0.121569,0.466667,0.705882}%
\pgfsetfillcolor{currentfill}%
\pgfsetfillopacity{0.646072}%
\pgfsetlinewidth{1.003750pt}%
\definecolor{currentstroke}{rgb}{0.121569,0.466667,0.705882}%
\pgfsetstrokecolor{currentstroke}%
\pgfsetstrokeopacity{0.646072}%
\pgfsetdash{}{0pt}%
\pgfpathmoveto{\pgfqpoint{3.241034in}{2.226449in}}%
\pgfpathcurveto{\pgfqpoint{3.249271in}{2.226449in}}{\pgfqpoint{3.257171in}{2.229722in}}{\pgfqpoint{3.262995in}{2.235546in}}%
\pgfpathcurveto{\pgfqpoint{3.268819in}{2.241370in}}{\pgfqpoint{3.272091in}{2.249270in}}{\pgfqpoint{3.272091in}{2.257506in}}%
\pgfpathcurveto{\pgfqpoint{3.272091in}{2.265742in}}{\pgfqpoint{3.268819in}{2.273642in}}{\pgfqpoint{3.262995in}{2.279466in}}%
\pgfpathcurveto{\pgfqpoint{3.257171in}{2.285290in}}{\pgfqpoint{3.249271in}{2.288562in}}{\pgfqpoint{3.241034in}{2.288562in}}%
\pgfpathcurveto{\pgfqpoint{3.232798in}{2.288562in}}{\pgfqpoint{3.224898in}{2.285290in}}{\pgfqpoint{3.219074in}{2.279466in}}%
\pgfpathcurveto{\pgfqpoint{3.213250in}{2.273642in}}{\pgfqpoint{3.209978in}{2.265742in}}{\pgfqpoint{3.209978in}{2.257506in}}%
\pgfpathcurveto{\pgfqpoint{3.209978in}{2.249270in}}{\pgfqpoint{3.213250in}{2.241370in}}{\pgfqpoint{3.219074in}{2.235546in}}%
\pgfpathcurveto{\pgfqpoint{3.224898in}{2.229722in}}{\pgfqpoint{3.232798in}{2.226449in}}{\pgfqpoint{3.241034in}{2.226449in}}%
\pgfpathclose%
\pgfusepath{stroke,fill}%
\end{pgfscope}%
\begin{pgfscope}%
\pgfpathrectangle{\pgfqpoint{0.100000in}{0.212622in}}{\pgfqpoint{3.696000in}{3.696000in}}%
\pgfusepath{clip}%
\pgfsetbuttcap%
\pgfsetroundjoin%
\definecolor{currentfill}{rgb}{0.121569,0.466667,0.705882}%
\pgfsetfillcolor{currentfill}%
\pgfsetfillopacity{0.646548}%
\pgfsetlinewidth{1.003750pt}%
\definecolor{currentstroke}{rgb}{0.121569,0.466667,0.705882}%
\pgfsetstrokecolor{currentstroke}%
\pgfsetstrokeopacity{0.646548}%
\pgfsetdash{}{0pt}%
\pgfpathmoveto{\pgfqpoint{3.239863in}{2.226237in}}%
\pgfpathcurveto{\pgfqpoint{3.248100in}{2.226237in}}{\pgfqpoint{3.256000in}{2.229509in}}{\pgfqpoint{3.261824in}{2.235333in}}%
\pgfpathcurveto{\pgfqpoint{3.267648in}{2.241157in}}{\pgfqpoint{3.270920in}{2.249057in}}{\pgfqpoint{3.270920in}{2.257294in}}%
\pgfpathcurveto{\pgfqpoint{3.270920in}{2.265530in}}{\pgfqpoint{3.267648in}{2.273430in}}{\pgfqpoint{3.261824in}{2.279254in}}%
\pgfpathcurveto{\pgfqpoint{3.256000in}{2.285078in}}{\pgfqpoint{3.248100in}{2.288350in}}{\pgfqpoint{3.239863in}{2.288350in}}%
\pgfpathcurveto{\pgfqpoint{3.231627in}{2.288350in}}{\pgfqpoint{3.223727in}{2.285078in}}{\pgfqpoint{3.217903in}{2.279254in}}%
\pgfpathcurveto{\pgfqpoint{3.212079in}{2.273430in}}{\pgfqpoint{3.208807in}{2.265530in}}{\pgfqpoint{3.208807in}{2.257294in}}%
\pgfpathcurveto{\pgfqpoint{3.208807in}{2.249057in}}{\pgfqpoint{3.212079in}{2.241157in}}{\pgfqpoint{3.217903in}{2.235333in}}%
\pgfpathcurveto{\pgfqpoint{3.223727in}{2.229509in}}{\pgfqpoint{3.231627in}{2.226237in}}{\pgfqpoint{3.239863in}{2.226237in}}%
\pgfpathclose%
\pgfusepath{stroke,fill}%
\end{pgfscope}%
\begin{pgfscope}%
\pgfpathrectangle{\pgfqpoint{0.100000in}{0.212622in}}{\pgfqpoint{3.696000in}{3.696000in}}%
\pgfusepath{clip}%
\pgfsetbuttcap%
\pgfsetroundjoin%
\definecolor{currentfill}{rgb}{0.121569,0.466667,0.705882}%
\pgfsetfillcolor{currentfill}%
\pgfsetfillopacity{0.646639}%
\pgfsetlinewidth{1.003750pt}%
\definecolor{currentstroke}{rgb}{0.121569,0.466667,0.705882}%
\pgfsetstrokecolor{currentstroke}%
\pgfsetstrokeopacity{0.646639}%
\pgfsetdash{}{0pt}%
\pgfpathmoveto{\pgfqpoint{0.762702in}{1.256866in}}%
\pgfpathcurveto{\pgfqpoint{0.770938in}{1.256866in}}{\pgfqpoint{0.778838in}{1.260139in}}{\pgfqpoint{0.784662in}{1.265963in}}%
\pgfpathcurveto{\pgfqpoint{0.790486in}{1.271787in}}{\pgfqpoint{0.793759in}{1.279687in}}{\pgfqpoint{0.793759in}{1.287923in}}%
\pgfpathcurveto{\pgfqpoint{0.793759in}{1.296159in}}{\pgfqpoint{0.790486in}{1.304059in}}{\pgfqpoint{0.784662in}{1.309883in}}%
\pgfpathcurveto{\pgfqpoint{0.778838in}{1.315707in}}{\pgfqpoint{0.770938in}{1.318979in}}{\pgfqpoint{0.762702in}{1.318979in}}%
\pgfpathcurveto{\pgfqpoint{0.754466in}{1.318979in}}{\pgfqpoint{0.746566in}{1.315707in}}{\pgfqpoint{0.740742in}{1.309883in}}%
\pgfpathcurveto{\pgfqpoint{0.734918in}{1.304059in}}{\pgfqpoint{0.731646in}{1.296159in}}{\pgfqpoint{0.731646in}{1.287923in}}%
\pgfpathcurveto{\pgfqpoint{0.731646in}{1.279687in}}{\pgfqpoint{0.734918in}{1.271787in}}{\pgfqpoint{0.740742in}{1.265963in}}%
\pgfpathcurveto{\pgfqpoint{0.746566in}{1.260139in}}{\pgfqpoint{0.754466in}{1.256866in}}{\pgfqpoint{0.762702in}{1.256866in}}%
\pgfpathclose%
\pgfusepath{stroke,fill}%
\end{pgfscope}%
\begin{pgfscope}%
\pgfpathrectangle{\pgfqpoint{0.100000in}{0.212622in}}{\pgfqpoint{3.696000in}{3.696000in}}%
\pgfusepath{clip}%
\pgfsetbuttcap%
\pgfsetroundjoin%
\definecolor{currentfill}{rgb}{0.121569,0.466667,0.705882}%
\pgfsetfillcolor{currentfill}%
\pgfsetfillopacity{0.646752}%
\pgfsetlinewidth{1.003750pt}%
\definecolor{currentstroke}{rgb}{0.121569,0.466667,0.705882}%
\pgfsetstrokecolor{currentstroke}%
\pgfsetstrokeopacity{0.646752}%
\pgfsetdash{}{0pt}%
\pgfpathmoveto{\pgfqpoint{3.239092in}{2.225945in}}%
\pgfpathcurveto{\pgfqpoint{3.247328in}{2.225945in}}{\pgfqpoint{3.255228in}{2.229217in}}{\pgfqpoint{3.261052in}{2.235041in}}%
\pgfpathcurveto{\pgfqpoint{3.266876in}{2.240865in}}{\pgfqpoint{3.270148in}{2.248765in}}{\pgfqpoint{3.270148in}{2.257001in}}%
\pgfpathcurveto{\pgfqpoint{3.270148in}{2.265238in}}{\pgfqpoint{3.266876in}{2.273138in}}{\pgfqpoint{3.261052in}{2.278962in}}%
\pgfpathcurveto{\pgfqpoint{3.255228in}{2.284786in}}{\pgfqpoint{3.247328in}{2.288058in}}{\pgfqpoint{3.239092in}{2.288058in}}%
\pgfpathcurveto{\pgfqpoint{3.230856in}{2.288058in}}{\pgfqpoint{3.222955in}{2.284786in}}{\pgfqpoint{3.217132in}{2.278962in}}%
\pgfpathcurveto{\pgfqpoint{3.211308in}{2.273138in}}{\pgfqpoint{3.208035in}{2.265238in}}{\pgfqpoint{3.208035in}{2.257001in}}%
\pgfpathcurveto{\pgfqpoint{3.208035in}{2.248765in}}{\pgfqpoint{3.211308in}{2.240865in}}{\pgfqpoint{3.217132in}{2.235041in}}%
\pgfpathcurveto{\pgfqpoint{3.222955in}{2.229217in}}{\pgfqpoint{3.230856in}{2.225945in}}{\pgfqpoint{3.239092in}{2.225945in}}%
\pgfpathclose%
\pgfusepath{stroke,fill}%
\end{pgfscope}%
\begin{pgfscope}%
\pgfpathrectangle{\pgfqpoint{0.100000in}{0.212622in}}{\pgfqpoint{3.696000in}{3.696000in}}%
\pgfusepath{clip}%
\pgfsetbuttcap%
\pgfsetroundjoin%
\definecolor{currentfill}{rgb}{0.121569,0.466667,0.705882}%
\pgfsetfillcolor{currentfill}%
\pgfsetfillopacity{0.646891}%
\pgfsetlinewidth{1.003750pt}%
\definecolor{currentstroke}{rgb}{0.121569,0.466667,0.705882}%
\pgfsetstrokecolor{currentstroke}%
\pgfsetstrokeopacity{0.646891}%
\pgfsetdash{}{0pt}%
\pgfpathmoveto{\pgfqpoint{3.238746in}{2.225837in}}%
\pgfpathcurveto{\pgfqpoint{3.246982in}{2.225837in}}{\pgfqpoint{3.254882in}{2.229109in}}{\pgfqpoint{3.260706in}{2.234933in}}%
\pgfpathcurveto{\pgfqpoint{3.266530in}{2.240757in}}{\pgfqpoint{3.269802in}{2.248657in}}{\pgfqpoint{3.269802in}{2.256893in}}%
\pgfpathcurveto{\pgfqpoint{3.269802in}{2.265130in}}{\pgfqpoint{3.266530in}{2.273030in}}{\pgfqpoint{3.260706in}{2.278854in}}%
\pgfpathcurveto{\pgfqpoint{3.254882in}{2.284678in}}{\pgfqpoint{3.246982in}{2.287950in}}{\pgfqpoint{3.238746in}{2.287950in}}%
\pgfpathcurveto{\pgfqpoint{3.230510in}{2.287950in}}{\pgfqpoint{3.222610in}{2.284678in}}{\pgfqpoint{3.216786in}{2.278854in}}%
\pgfpathcurveto{\pgfqpoint{3.210962in}{2.273030in}}{\pgfqpoint{3.207689in}{2.265130in}}{\pgfqpoint{3.207689in}{2.256893in}}%
\pgfpathcurveto{\pgfqpoint{3.207689in}{2.248657in}}{\pgfqpoint{3.210962in}{2.240757in}}{\pgfqpoint{3.216786in}{2.234933in}}%
\pgfpathcurveto{\pgfqpoint{3.222610in}{2.229109in}}{\pgfqpoint{3.230510in}{2.225837in}}{\pgfqpoint{3.238746in}{2.225837in}}%
\pgfpathclose%
\pgfusepath{stroke,fill}%
\end{pgfscope}%
\begin{pgfscope}%
\pgfpathrectangle{\pgfqpoint{0.100000in}{0.212622in}}{\pgfqpoint{3.696000in}{3.696000in}}%
\pgfusepath{clip}%
\pgfsetbuttcap%
\pgfsetroundjoin%
\definecolor{currentfill}{rgb}{0.121569,0.466667,0.705882}%
\pgfsetfillcolor{currentfill}%
\pgfsetfillopacity{0.646974}%
\pgfsetlinewidth{1.003750pt}%
\definecolor{currentstroke}{rgb}{0.121569,0.466667,0.705882}%
\pgfsetstrokecolor{currentstroke}%
\pgfsetstrokeopacity{0.646974}%
\pgfsetdash{}{0pt}%
\pgfpathmoveto{\pgfqpoint{3.238584in}{2.225776in}}%
\pgfpathcurveto{\pgfqpoint{3.246820in}{2.225776in}}{\pgfqpoint{3.254720in}{2.229048in}}{\pgfqpoint{3.260544in}{2.234872in}}%
\pgfpathcurveto{\pgfqpoint{3.266368in}{2.240696in}}{\pgfqpoint{3.269640in}{2.248596in}}{\pgfqpoint{3.269640in}{2.256832in}}%
\pgfpathcurveto{\pgfqpoint{3.269640in}{2.265069in}}{\pgfqpoint{3.266368in}{2.272969in}}{\pgfqpoint{3.260544in}{2.278793in}}%
\pgfpathcurveto{\pgfqpoint{3.254720in}{2.284617in}}{\pgfqpoint{3.246820in}{2.287889in}}{\pgfqpoint{3.238584in}{2.287889in}}%
\pgfpathcurveto{\pgfqpoint{3.230347in}{2.287889in}}{\pgfqpoint{3.222447in}{2.284617in}}{\pgfqpoint{3.216623in}{2.278793in}}%
\pgfpathcurveto{\pgfqpoint{3.210799in}{2.272969in}}{\pgfqpoint{3.207527in}{2.265069in}}{\pgfqpoint{3.207527in}{2.256832in}}%
\pgfpathcurveto{\pgfqpoint{3.207527in}{2.248596in}}{\pgfqpoint{3.210799in}{2.240696in}}{\pgfqpoint{3.216623in}{2.234872in}}%
\pgfpathcurveto{\pgfqpoint{3.222447in}{2.229048in}}{\pgfqpoint{3.230347in}{2.225776in}}{\pgfqpoint{3.238584in}{2.225776in}}%
\pgfpathclose%
\pgfusepath{stroke,fill}%
\end{pgfscope}%
\begin{pgfscope}%
\pgfpathrectangle{\pgfqpoint{0.100000in}{0.212622in}}{\pgfqpoint{3.696000in}{3.696000in}}%
\pgfusepath{clip}%
\pgfsetbuttcap%
\pgfsetroundjoin%
\definecolor{currentfill}{rgb}{0.121569,0.466667,0.705882}%
\pgfsetfillcolor{currentfill}%
\pgfsetfillopacity{0.647516}%
\pgfsetlinewidth{1.003750pt}%
\definecolor{currentstroke}{rgb}{0.121569,0.466667,0.705882}%
\pgfsetstrokecolor{currentstroke}%
\pgfsetstrokeopacity{0.647516}%
\pgfsetdash{}{0pt}%
\pgfpathmoveto{\pgfqpoint{3.236871in}{2.225631in}}%
\pgfpathcurveto{\pgfqpoint{3.245107in}{2.225631in}}{\pgfqpoint{3.253007in}{2.228904in}}{\pgfqpoint{3.258831in}{2.234728in}}%
\pgfpathcurveto{\pgfqpoint{3.264655in}{2.240552in}}{\pgfqpoint{3.267927in}{2.248452in}}{\pgfqpoint{3.267927in}{2.256688in}}%
\pgfpathcurveto{\pgfqpoint{3.267927in}{2.264924in}}{\pgfqpoint{3.264655in}{2.272824in}}{\pgfqpoint{3.258831in}{2.278648in}}%
\pgfpathcurveto{\pgfqpoint{3.253007in}{2.284472in}}{\pgfqpoint{3.245107in}{2.287744in}}{\pgfqpoint{3.236871in}{2.287744in}}%
\pgfpathcurveto{\pgfqpoint{3.228634in}{2.287744in}}{\pgfqpoint{3.220734in}{2.284472in}}{\pgfqpoint{3.214910in}{2.278648in}}%
\pgfpathcurveto{\pgfqpoint{3.209087in}{2.272824in}}{\pgfqpoint{3.205814in}{2.264924in}}{\pgfqpoint{3.205814in}{2.256688in}}%
\pgfpathcurveto{\pgfqpoint{3.205814in}{2.248452in}}{\pgfqpoint{3.209087in}{2.240552in}}{\pgfqpoint{3.214910in}{2.234728in}}%
\pgfpathcurveto{\pgfqpoint{3.220734in}{2.228904in}}{\pgfqpoint{3.228634in}{2.225631in}}{\pgfqpoint{3.236871in}{2.225631in}}%
\pgfpathclose%
\pgfusepath{stroke,fill}%
\end{pgfscope}%
\begin{pgfscope}%
\pgfpathrectangle{\pgfqpoint{0.100000in}{0.212622in}}{\pgfqpoint{3.696000in}{3.696000in}}%
\pgfusepath{clip}%
\pgfsetbuttcap%
\pgfsetroundjoin%
\definecolor{currentfill}{rgb}{0.121569,0.466667,0.705882}%
\pgfsetfillcolor{currentfill}%
\pgfsetfillopacity{0.648170}%
\pgfsetlinewidth{1.003750pt}%
\definecolor{currentstroke}{rgb}{0.121569,0.466667,0.705882}%
\pgfsetstrokecolor{currentstroke}%
\pgfsetstrokeopacity{0.648170}%
\pgfsetdash{}{0pt}%
\pgfpathmoveto{\pgfqpoint{0.796032in}{1.230456in}}%
\pgfpathcurveto{\pgfqpoint{0.804268in}{1.230456in}}{\pgfqpoint{0.812168in}{1.233728in}}{\pgfqpoint{0.817992in}{1.239552in}}%
\pgfpathcurveto{\pgfqpoint{0.823816in}{1.245376in}}{\pgfqpoint{0.827088in}{1.253276in}}{\pgfqpoint{0.827088in}{1.261512in}}%
\pgfpathcurveto{\pgfqpoint{0.827088in}{1.269748in}}{\pgfqpoint{0.823816in}{1.277648in}}{\pgfqpoint{0.817992in}{1.283472in}}%
\pgfpathcurveto{\pgfqpoint{0.812168in}{1.289296in}}{\pgfqpoint{0.804268in}{1.292569in}}{\pgfqpoint{0.796032in}{1.292569in}}%
\pgfpathcurveto{\pgfqpoint{0.787796in}{1.292569in}}{\pgfqpoint{0.779896in}{1.289296in}}{\pgfqpoint{0.774072in}{1.283472in}}%
\pgfpathcurveto{\pgfqpoint{0.768248in}{1.277648in}}{\pgfqpoint{0.764975in}{1.269748in}}{\pgfqpoint{0.764975in}{1.261512in}}%
\pgfpathcurveto{\pgfqpoint{0.764975in}{1.253276in}}{\pgfqpoint{0.768248in}{1.245376in}}{\pgfqpoint{0.774072in}{1.239552in}}%
\pgfpathcurveto{\pgfqpoint{0.779896in}{1.233728in}}{\pgfqpoint{0.787796in}{1.230456in}}{\pgfqpoint{0.796032in}{1.230456in}}%
\pgfpathclose%
\pgfusepath{stroke,fill}%
\end{pgfscope}%
\begin{pgfscope}%
\pgfpathrectangle{\pgfqpoint{0.100000in}{0.212622in}}{\pgfqpoint{3.696000in}{3.696000in}}%
\pgfusepath{clip}%
\pgfsetbuttcap%
\pgfsetroundjoin%
\definecolor{currentfill}{rgb}{0.121569,0.466667,0.705882}%
\pgfsetfillcolor{currentfill}%
\pgfsetfillopacity{0.649123}%
\pgfsetlinewidth{1.003750pt}%
\definecolor{currentstroke}{rgb}{0.121569,0.466667,0.705882}%
\pgfsetstrokecolor{currentstroke}%
\pgfsetstrokeopacity{0.649123}%
\pgfsetdash{}{0pt}%
\pgfpathmoveto{\pgfqpoint{3.233596in}{2.221780in}}%
\pgfpathcurveto{\pgfqpoint{3.241832in}{2.221780in}}{\pgfqpoint{3.249732in}{2.225052in}}{\pgfqpoint{3.255556in}{2.230876in}}%
\pgfpathcurveto{\pgfqpoint{3.261380in}{2.236700in}}{\pgfqpoint{3.264653in}{2.244600in}}{\pgfqpoint{3.264653in}{2.252836in}}%
\pgfpathcurveto{\pgfqpoint{3.264653in}{2.261073in}}{\pgfqpoint{3.261380in}{2.268973in}}{\pgfqpoint{3.255556in}{2.274797in}}%
\pgfpathcurveto{\pgfqpoint{3.249732in}{2.280621in}}{\pgfqpoint{3.241832in}{2.283893in}}{\pgfqpoint{3.233596in}{2.283893in}}%
\pgfpathcurveto{\pgfqpoint{3.225360in}{2.283893in}}{\pgfqpoint{3.217460in}{2.280621in}}{\pgfqpoint{3.211636in}{2.274797in}}%
\pgfpathcurveto{\pgfqpoint{3.205812in}{2.268973in}}{\pgfqpoint{3.202540in}{2.261073in}}{\pgfqpoint{3.202540in}{2.252836in}}%
\pgfpathcurveto{\pgfqpoint{3.202540in}{2.244600in}}{\pgfqpoint{3.205812in}{2.236700in}}{\pgfqpoint{3.211636in}{2.230876in}}%
\pgfpathcurveto{\pgfqpoint{3.217460in}{2.225052in}}{\pgfqpoint{3.225360in}{2.221780in}}{\pgfqpoint{3.233596in}{2.221780in}}%
\pgfpathclose%
\pgfusepath{stroke,fill}%
\end{pgfscope}%
\begin{pgfscope}%
\pgfpathrectangle{\pgfqpoint{0.100000in}{0.212622in}}{\pgfqpoint{3.696000in}{3.696000in}}%
\pgfusepath{clip}%
\pgfsetbuttcap%
\pgfsetroundjoin%
\definecolor{currentfill}{rgb}{0.121569,0.466667,0.705882}%
\pgfsetfillcolor{currentfill}%
\pgfsetfillopacity{0.652203}%
\pgfsetlinewidth{1.003750pt}%
\definecolor{currentstroke}{rgb}{0.121569,0.466667,0.705882}%
\pgfsetstrokecolor{currentstroke}%
\pgfsetstrokeopacity{0.652203}%
\pgfsetdash{}{0pt}%
\pgfpathmoveto{\pgfqpoint{3.225840in}{2.221464in}}%
\pgfpathcurveto{\pgfqpoint{3.234076in}{2.221464in}}{\pgfqpoint{3.241976in}{2.224736in}}{\pgfqpoint{3.247800in}{2.230560in}}%
\pgfpathcurveto{\pgfqpoint{3.253624in}{2.236384in}}{\pgfqpoint{3.256897in}{2.244284in}}{\pgfqpoint{3.256897in}{2.252520in}}%
\pgfpathcurveto{\pgfqpoint{3.256897in}{2.260757in}}{\pgfqpoint{3.253624in}{2.268657in}}{\pgfqpoint{3.247800in}{2.274481in}}%
\pgfpathcurveto{\pgfqpoint{3.241976in}{2.280305in}}{\pgfqpoint{3.234076in}{2.283577in}}{\pgfqpoint{3.225840in}{2.283577in}}%
\pgfpathcurveto{\pgfqpoint{3.217604in}{2.283577in}}{\pgfqpoint{3.209704in}{2.280305in}}{\pgfqpoint{3.203880in}{2.274481in}}%
\pgfpathcurveto{\pgfqpoint{3.198056in}{2.268657in}}{\pgfqpoint{3.194784in}{2.260757in}}{\pgfqpoint{3.194784in}{2.252520in}}%
\pgfpathcurveto{\pgfqpoint{3.194784in}{2.244284in}}{\pgfqpoint{3.198056in}{2.236384in}}{\pgfqpoint{3.203880in}{2.230560in}}%
\pgfpathcurveto{\pgfqpoint{3.209704in}{2.224736in}}{\pgfqpoint{3.217604in}{2.221464in}}{\pgfqpoint{3.225840in}{2.221464in}}%
\pgfpathclose%
\pgfusepath{stroke,fill}%
\end{pgfscope}%
\begin{pgfscope}%
\pgfpathrectangle{\pgfqpoint{0.100000in}{0.212622in}}{\pgfqpoint{3.696000in}{3.696000in}}%
\pgfusepath{clip}%
\pgfsetbuttcap%
\pgfsetroundjoin%
\definecolor{currentfill}{rgb}{0.121569,0.466667,0.705882}%
\pgfsetfillcolor{currentfill}%
\pgfsetfillopacity{0.652941}%
\pgfsetlinewidth{1.003750pt}%
\definecolor{currentstroke}{rgb}{0.121569,0.466667,0.705882}%
\pgfsetstrokecolor{currentstroke}%
\pgfsetstrokeopacity{0.652941}%
\pgfsetdash{}{0pt}%
\pgfpathmoveto{\pgfqpoint{0.751127in}{1.244467in}}%
\pgfpathcurveto{\pgfqpoint{0.759364in}{1.244467in}}{\pgfqpoint{0.767264in}{1.247740in}}{\pgfqpoint{0.773088in}{1.253563in}}%
\pgfpathcurveto{\pgfqpoint{0.778912in}{1.259387in}}{\pgfqpoint{0.782184in}{1.267287in}}{\pgfqpoint{0.782184in}{1.275524in}}%
\pgfpathcurveto{\pgfqpoint{0.782184in}{1.283760in}}{\pgfqpoint{0.778912in}{1.291660in}}{\pgfqpoint{0.773088in}{1.297484in}}%
\pgfpathcurveto{\pgfqpoint{0.767264in}{1.303308in}}{\pgfqpoint{0.759364in}{1.306580in}}{\pgfqpoint{0.751127in}{1.306580in}}%
\pgfpathcurveto{\pgfqpoint{0.742891in}{1.306580in}}{\pgfqpoint{0.734991in}{1.303308in}}{\pgfqpoint{0.729167in}{1.297484in}}%
\pgfpathcurveto{\pgfqpoint{0.723343in}{1.291660in}}{\pgfqpoint{0.720071in}{1.283760in}}{\pgfqpoint{0.720071in}{1.275524in}}%
\pgfpathcurveto{\pgfqpoint{0.720071in}{1.267287in}}{\pgfqpoint{0.723343in}{1.259387in}}{\pgfqpoint{0.729167in}{1.253563in}}%
\pgfpathcurveto{\pgfqpoint{0.734991in}{1.247740in}}{\pgfqpoint{0.742891in}{1.244467in}}{\pgfqpoint{0.751127in}{1.244467in}}%
\pgfpathclose%
\pgfusepath{stroke,fill}%
\end{pgfscope}%
\begin{pgfscope}%
\pgfpathrectangle{\pgfqpoint{0.100000in}{0.212622in}}{\pgfqpoint{3.696000in}{3.696000in}}%
\pgfusepath{clip}%
\pgfsetbuttcap%
\pgfsetroundjoin%
\definecolor{currentfill}{rgb}{0.121569,0.466667,0.705882}%
\pgfsetfillcolor{currentfill}%
\pgfsetfillopacity{0.655195}%
\pgfsetlinewidth{1.003750pt}%
\definecolor{currentstroke}{rgb}{0.121569,0.466667,0.705882}%
\pgfsetstrokecolor{currentstroke}%
\pgfsetstrokeopacity{0.655195}%
\pgfsetdash{}{0pt}%
\pgfpathmoveto{\pgfqpoint{0.764760in}{1.222747in}}%
\pgfpathcurveto{\pgfqpoint{0.772996in}{1.222747in}}{\pgfqpoint{0.780896in}{1.226020in}}{\pgfqpoint{0.786720in}{1.231844in}}%
\pgfpathcurveto{\pgfqpoint{0.792544in}{1.237668in}}{\pgfqpoint{0.795816in}{1.245568in}}{\pgfqpoint{0.795816in}{1.253804in}}%
\pgfpathcurveto{\pgfqpoint{0.795816in}{1.262040in}}{\pgfqpoint{0.792544in}{1.269940in}}{\pgfqpoint{0.786720in}{1.275764in}}%
\pgfpathcurveto{\pgfqpoint{0.780896in}{1.281588in}}{\pgfqpoint{0.772996in}{1.284860in}}{\pgfqpoint{0.764760in}{1.284860in}}%
\pgfpathcurveto{\pgfqpoint{0.756524in}{1.284860in}}{\pgfqpoint{0.748624in}{1.281588in}}{\pgfqpoint{0.742800in}{1.275764in}}%
\pgfpathcurveto{\pgfqpoint{0.736976in}{1.269940in}}{\pgfqpoint{0.733703in}{1.262040in}}{\pgfqpoint{0.733703in}{1.253804in}}%
\pgfpathcurveto{\pgfqpoint{0.733703in}{1.245568in}}{\pgfqpoint{0.736976in}{1.237668in}}{\pgfqpoint{0.742800in}{1.231844in}}%
\pgfpathcurveto{\pgfqpoint{0.748624in}{1.226020in}}{\pgfqpoint{0.756524in}{1.222747in}}{\pgfqpoint{0.764760in}{1.222747in}}%
\pgfpathclose%
\pgfusepath{stroke,fill}%
\end{pgfscope}%
\begin{pgfscope}%
\pgfpathrectangle{\pgfqpoint{0.100000in}{0.212622in}}{\pgfqpoint{3.696000in}{3.696000in}}%
\pgfusepath{clip}%
\pgfsetbuttcap%
\pgfsetroundjoin%
\definecolor{currentfill}{rgb}{0.121569,0.466667,0.705882}%
\pgfsetfillcolor{currentfill}%
\pgfsetfillopacity{0.656311}%
\pgfsetlinewidth{1.003750pt}%
\definecolor{currentstroke}{rgb}{0.121569,0.466667,0.705882}%
\pgfsetstrokecolor{currentstroke}%
\pgfsetstrokeopacity{0.656311}%
\pgfsetdash{}{0pt}%
\pgfpathmoveto{\pgfqpoint{3.216928in}{2.215728in}}%
\pgfpathcurveto{\pgfqpoint{3.225165in}{2.215728in}}{\pgfqpoint{3.233065in}{2.219000in}}{\pgfqpoint{3.238888in}{2.224824in}}%
\pgfpathcurveto{\pgfqpoint{3.244712in}{2.230648in}}{\pgfqpoint{3.247985in}{2.238548in}}{\pgfqpoint{3.247985in}{2.246784in}}%
\pgfpathcurveto{\pgfqpoint{3.247985in}{2.255020in}}{\pgfqpoint{3.244712in}{2.262920in}}{\pgfqpoint{3.238888in}{2.268744in}}%
\pgfpathcurveto{\pgfqpoint{3.233065in}{2.274568in}}{\pgfqpoint{3.225165in}{2.277841in}}{\pgfqpoint{3.216928in}{2.277841in}}%
\pgfpathcurveto{\pgfqpoint{3.208692in}{2.277841in}}{\pgfqpoint{3.200792in}{2.274568in}}{\pgfqpoint{3.194968in}{2.268744in}}%
\pgfpathcurveto{\pgfqpoint{3.189144in}{2.262920in}}{\pgfqpoint{3.185872in}{2.255020in}}{\pgfqpoint{3.185872in}{2.246784in}}%
\pgfpathcurveto{\pgfqpoint{3.185872in}{2.238548in}}{\pgfqpoint{3.189144in}{2.230648in}}{\pgfqpoint{3.194968in}{2.224824in}}%
\pgfpathcurveto{\pgfqpoint{3.200792in}{2.219000in}}{\pgfqpoint{3.208692in}{2.215728in}}{\pgfqpoint{3.216928in}{2.215728in}}%
\pgfpathclose%
\pgfusepath{stroke,fill}%
\end{pgfscope}%
\begin{pgfscope}%
\pgfpathrectangle{\pgfqpoint{0.100000in}{0.212622in}}{\pgfqpoint{3.696000in}{3.696000in}}%
\pgfusepath{clip}%
\pgfsetbuttcap%
\pgfsetroundjoin%
\definecolor{currentfill}{rgb}{0.121569,0.466667,0.705882}%
\pgfsetfillcolor{currentfill}%
\pgfsetfillopacity{0.656465}%
\pgfsetlinewidth{1.003750pt}%
\definecolor{currentstroke}{rgb}{0.121569,0.466667,0.705882}%
\pgfsetstrokecolor{currentstroke}%
\pgfsetstrokeopacity{0.656465}%
\pgfsetdash{}{0pt}%
\pgfpathmoveto{\pgfqpoint{0.737769in}{1.236475in}}%
\pgfpathcurveto{\pgfqpoint{0.746006in}{1.236475in}}{\pgfqpoint{0.753906in}{1.239747in}}{\pgfqpoint{0.759730in}{1.245571in}}%
\pgfpathcurveto{\pgfqpoint{0.765554in}{1.251395in}}{\pgfqpoint{0.768826in}{1.259295in}}{\pgfqpoint{0.768826in}{1.267531in}}%
\pgfpathcurveto{\pgfqpoint{0.768826in}{1.275767in}}{\pgfqpoint{0.765554in}{1.283667in}}{\pgfqpoint{0.759730in}{1.289491in}}%
\pgfpathcurveto{\pgfqpoint{0.753906in}{1.295315in}}{\pgfqpoint{0.746006in}{1.298588in}}{\pgfqpoint{0.737769in}{1.298588in}}%
\pgfpathcurveto{\pgfqpoint{0.729533in}{1.298588in}}{\pgfqpoint{0.721633in}{1.295315in}}{\pgfqpoint{0.715809in}{1.289491in}}%
\pgfpathcurveto{\pgfqpoint{0.709985in}{1.283667in}}{\pgfqpoint{0.706713in}{1.275767in}}{\pgfqpoint{0.706713in}{1.267531in}}%
\pgfpathcurveto{\pgfqpoint{0.706713in}{1.259295in}}{\pgfqpoint{0.709985in}{1.251395in}}{\pgfqpoint{0.715809in}{1.245571in}}%
\pgfpathcurveto{\pgfqpoint{0.721633in}{1.239747in}}{\pgfqpoint{0.729533in}{1.236475in}}{\pgfqpoint{0.737769in}{1.236475in}}%
\pgfpathclose%
\pgfusepath{stroke,fill}%
\end{pgfscope}%
\begin{pgfscope}%
\pgfpathrectangle{\pgfqpoint{0.100000in}{0.212622in}}{\pgfqpoint{3.696000in}{3.696000in}}%
\pgfusepath{clip}%
\pgfsetbuttcap%
\pgfsetroundjoin%
\definecolor{currentfill}{rgb}{0.121569,0.466667,0.705882}%
\pgfsetfillcolor{currentfill}%
\pgfsetfillopacity{0.657940}%
\pgfsetlinewidth{1.003750pt}%
\definecolor{currentstroke}{rgb}{0.121569,0.466667,0.705882}%
\pgfsetstrokecolor{currentstroke}%
\pgfsetstrokeopacity{0.657940}%
\pgfsetdash{}{0pt}%
\pgfpathmoveto{\pgfqpoint{0.733659in}{1.230853in}}%
\pgfpathcurveto{\pgfqpoint{0.741895in}{1.230853in}}{\pgfqpoint{0.749795in}{1.234125in}}{\pgfqpoint{0.755619in}{1.239949in}}%
\pgfpathcurveto{\pgfqpoint{0.761443in}{1.245773in}}{\pgfqpoint{0.764715in}{1.253673in}}{\pgfqpoint{0.764715in}{1.261909in}}%
\pgfpathcurveto{\pgfqpoint{0.764715in}{1.270146in}}{\pgfqpoint{0.761443in}{1.278046in}}{\pgfqpoint{0.755619in}{1.283870in}}%
\pgfpathcurveto{\pgfqpoint{0.749795in}{1.289694in}}{\pgfqpoint{0.741895in}{1.292966in}}{\pgfqpoint{0.733659in}{1.292966in}}%
\pgfpathcurveto{\pgfqpoint{0.725422in}{1.292966in}}{\pgfqpoint{0.717522in}{1.289694in}}{\pgfqpoint{0.711698in}{1.283870in}}%
\pgfpathcurveto{\pgfqpoint{0.705875in}{1.278046in}}{\pgfqpoint{0.702602in}{1.270146in}}{\pgfqpoint{0.702602in}{1.261909in}}%
\pgfpathcurveto{\pgfqpoint{0.702602in}{1.253673in}}{\pgfqpoint{0.705875in}{1.245773in}}{\pgfqpoint{0.711698in}{1.239949in}}%
\pgfpathcurveto{\pgfqpoint{0.717522in}{1.234125in}}{\pgfqpoint{0.725422in}{1.230853in}}{\pgfqpoint{0.733659in}{1.230853in}}%
\pgfpathclose%
\pgfusepath{stroke,fill}%
\end{pgfscope}%
\begin{pgfscope}%
\pgfpathrectangle{\pgfqpoint{0.100000in}{0.212622in}}{\pgfqpoint{3.696000in}{3.696000in}}%
\pgfusepath{clip}%
\pgfsetbuttcap%
\pgfsetroundjoin%
\definecolor{currentfill}{rgb}{0.121569,0.466667,0.705882}%
\pgfsetfillcolor{currentfill}%
\pgfsetfillopacity{0.658856}%
\pgfsetlinewidth{1.003750pt}%
\definecolor{currentstroke}{rgb}{0.121569,0.466667,0.705882}%
\pgfsetstrokecolor{currentstroke}%
\pgfsetstrokeopacity{0.658856}%
\pgfsetdash{}{0pt}%
\pgfpathmoveto{\pgfqpoint{0.731305in}{1.228405in}}%
\pgfpathcurveto{\pgfqpoint{0.739541in}{1.228405in}}{\pgfqpoint{0.747441in}{1.231677in}}{\pgfqpoint{0.753265in}{1.237501in}}%
\pgfpathcurveto{\pgfqpoint{0.759089in}{1.243325in}}{\pgfqpoint{0.762361in}{1.251225in}}{\pgfqpoint{0.762361in}{1.259461in}}%
\pgfpathcurveto{\pgfqpoint{0.762361in}{1.267697in}}{\pgfqpoint{0.759089in}{1.275598in}}{\pgfqpoint{0.753265in}{1.281421in}}%
\pgfpathcurveto{\pgfqpoint{0.747441in}{1.287245in}}{\pgfqpoint{0.739541in}{1.290518in}}{\pgfqpoint{0.731305in}{1.290518in}}%
\pgfpathcurveto{\pgfqpoint{0.723068in}{1.290518in}}{\pgfqpoint{0.715168in}{1.287245in}}{\pgfqpoint{0.709344in}{1.281421in}}%
\pgfpathcurveto{\pgfqpoint{0.703520in}{1.275598in}}{\pgfqpoint{0.700248in}{1.267697in}}{\pgfqpoint{0.700248in}{1.259461in}}%
\pgfpathcurveto{\pgfqpoint{0.700248in}{1.251225in}}{\pgfqpoint{0.703520in}{1.243325in}}{\pgfqpoint{0.709344in}{1.237501in}}%
\pgfpathcurveto{\pgfqpoint{0.715168in}{1.231677in}}{\pgfqpoint{0.723068in}{1.228405in}}{\pgfqpoint{0.731305in}{1.228405in}}%
\pgfpathclose%
\pgfusepath{stroke,fill}%
\end{pgfscope}%
\begin{pgfscope}%
\pgfpathrectangle{\pgfqpoint{0.100000in}{0.212622in}}{\pgfqpoint{3.696000in}{3.696000in}}%
\pgfusepath{clip}%
\pgfsetbuttcap%
\pgfsetroundjoin%
\definecolor{currentfill}{rgb}{0.121569,0.466667,0.705882}%
\pgfsetfillcolor{currentfill}%
\pgfsetfillopacity{0.659128}%
\pgfsetlinewidth{1.003750pt}%
\definecolor{currentstroke}{rgb}{0.121569,0.466667,0.705882}%
\pgfsetstrokecolor{currentstroke}%
\pgfsetstrokeopacity{0.659128}%
\pgfsetdash{}{0pt}%
\pgfpathmoveto{\pgfqpoint{0.747797in}{1.218196in}}%
\pgfpathcurveto{\pgfqpoint{0.756034in}{1.218196in}}{\pgfqpoint{0.763934in}{1.221468in}}{\pgfqpoint{0.769758in}{1.227292in}}%
\pgfpathcurveto{\pgfqpoint{0.775582in}{1.233116in}}{\pgfqpoint{0.778854in}{1.241016in}}{\pgfqpoint{0.778854in}{1.249252in}}%
\pgfpathcurveto{\pgfqpoint{0.778854in}{1.257488in}}{\pgfqpoint{0.775582in}{1.265389in}}{\pgfqpoint{0.769758in}{1.271212in}}%
\pgfpathcurveto{\pgfqpoint{0.763934in}{1.277036in}}{\pgfqpoint{0.756034in}{1.280309in}}{\pgfqpoint{0.747797in}{1.280309in}}%
\pgfpathcurveto{\pgfqpoint{0.739561in}{1.280309in}}{\pgfqpoint{0.731661in}{1.277036in}}{\pgfqpoint{0.725837in}{1.271212in}}%
\pgfpathcurveto{\pgfqpoint{0.720013in}{1.265389in}}{\pgfqpoint{0.716741in}{1.257488in}}{\pgfqpoint{0.716741in}{1.249252in}}%
\pgfpathcurveto{\pgfqpoint{0.716741in}{1.241016in}}{\pgfqpoint{0.720013in}{1.233116in}}{\pgfqpoint{0.725837in}{1.227292in}}%
\pgfpathcurveto{\pgfqpoint{0.731661in}{1.221468in}}{\pgfqpoint{0.739561in}{1.218196in}}{\pgfqpoint{0.747797in}{1.218196in}}%
\pgfpathclose%
\pgfusepath{stroke,fill}%
\end{pgfscope}%
\begin{pgfscope}%
\pgfpathrectangle{\pgfqpoint{0.100000in}{0.212622in}}{\pgfqpoint{3.696000in}{3.696000in}}%
\pgfusepath{clip}%
\pgfsetbuttcap%
\pgfsetroundjoin%
\definecolor{currentfill}{rgb}{0.121569,0.466667,0.705882}%
\pgfsetfillcolor{currentfill}%
\pgfsetfillopacity{0.660375}%
\pgfsetlinewidth{1.003750pt}%
\definecolor{currentstroke}{rgb}{0.121569,0.466667,0.705882}%
\pgfsetstrokecolor{currentstroke}%
\pgfsetstrokeopacity{0.660375}%
\pgfsetdash{}{0pt}%
\pgfpathmoveto{\pgfqpoint{0.725810in}{1.224566in}}%
\pgfpathcurveto{\pgfqpoint{0.734046in}{1.224566in}}{\pgfqpoint{0.741946in}{1.227839in}}{\pgfqpoint{0.747770in}{1.233663in}}%
\pgfpathcurveto{\pgfqpoint{0.753594in}{1.239486in}}{\pgfqpoint{0.756866in}{1.247387in}}{\pgfqpoint{0.756866in}{1.255623in}}%
\pgfpathcurveto{\pgfqpoint{0.756866in}{1.263859in}}{\pgfqpoint{0.753594in}{1.271759in}}{\pgfqpoint{0.747770in}{1.277583in}}%
\pgfpathcurveto{\pgfqpoint{0.741946in}{1.283407in}}{\pgfqpoint{0.734046in}{1.286679in}}{\pgfqpoint{0.725810in}{1.286679in}}%
\pgfpathcurveto{\pgfqpoint{0.717574in}{1.286679in}}{\pgfqpoint{0.709673in}{1.283407in}}{\pgfqpoint{0.703850in}{1.277583in}}%
\pgfpathcurveto{\pgfqpoint{0.698026in}{1.271759in}}{\pgfqpoint{0.694753in}{1.263859in}}{\pgfqpoint{0.694753in}{1.255623in}}%
\pgfpathcurveto{\pgfqpoint{0.694753in}{1.247387in}}{\pgfqpoint{0.698026in}{1.239486in}}{\pgfqpoint{0.703850in}{1.233663in}}%
\pgfpathcurveto{\pgfqpoint{0.709673in}{1.227839in}}{\pgfqpoint{0.717574in}{1.224566in}}{\pgfqpoint{0.725810in}{1.224566in}}%
\pgfpathclose%
\pgfusepath{stroke,fill}%
\end{pgfscope}%
\begin{pgfscope}%
\pgfpathrectangle{\pgfqpoint{0.100000in}{0.212622in}}{\pgfqpoint{3.696000in}{3.696000in}}%
\pgfusepath{clip}%
\pgfsetbuttcap%
\pgfsetroundjoin%
\definecolor{currentfill}{rgb}{0.121569,0.466667,0.705882}%
\pgfsetfillcolor{currentfill}%
\pgfsetfillopacity{0.660379}%
\pgfsetlinewidth{1.003750pt}%
\definecolor{currentstroke}{rgb}{0.121569,0.466667,0.705882}%
\pgfsetstrokecolor{currentstroke}%
\pgfsetstrokeopacity{0.660379}%
\pgfsetdash{}{0pt}%
\pgfpathmoveto{\pgfqpoint{0.725803in}{1.224556in}}%
\pgfpathcurveto{\pgfqpoint{0.734039in}{1.224556in}}{\pgfqpoint{0.741939in}{1.227829in}}{\pgfqpoint{0.747763in}{1.233653in}}%
\pgfpathcurveto{\pgfqpoint{0.753587in}{1.239477in}}{\pgfqpoint{0.756859in}{1.247377in}}{\pgfqpoint{0.756859in}{1.255613in}}%
\pgfpathcurveto{\pgfqpoint{0.756859in}{1.263849in}}{\pgfqpoint{0.753587in}{1.271749in}}{\pgfqpoint{0.747763in}{1.277573in}}%
\pgfpathcurveto{\pgfqpoint{0.741939in}{1.283397in}}{\pgfqpoint{0.734039in}{1.286669in}}{\pgfqpoint{0.725803in}{1.286669in}}%
\pgfpathcurveto{\pgfqpoint{0.717567in}{1.286669in}}{\pgfqpoint{0.709666in}{1.283397in}}{\pgfqpoint{0.703843in}{1.277573in}}%
\pgfpathcurveto{\pgfqpoint{0.698019in}{1.271749in}}{\pgfqpoint{0.694746in}{1.263849in}}{\pgfqpoint{0.694746in}{1.255613in}}%
\pgfpathcurveto{\pgfqpoint{0.694746in}{1.247377in}}{\pgfqpoint{0.698019in}{1.239477in}}{\pgfqpoint{0.703843in}{1.233653in}}%
\pgfpathcurveto{\pgfqpoint{0.709666in}{1.227829in}}{\pgfqpoint{0.717567in}{1.224556in}}{\pgfqpoint{0.725803in}{1.224556in}}%
\pgfpathclose%
\pgfusepath{stroke,fill}%
\end{pgfscope}%
\begin{pgfscope}%
\pgfpathrectangle{\pgfqpoint{0.100000in}{0.212622in}}{\pgfqpoint{3.696000in}{3.696000in}}%
\pgfusepath{clip}%
\pgfsetbuttcap%
\pgfsetroundjoin%
\definecolor{currentfill}{rgb}{0.121569,0.466667,0.705882}%
\pgfsetfillcolor{currentfill}%
\pgfsetfillopacity{0.660385}%
\pgfsetlinewidth{1.003750pt}%
\definecolor{currentstroke}{rgb}{0.121569,0.466667,0.705882}%
\pgfsetstrokecolor{currentstroke}%
\pgfsetstrokeopacity{0.660385}%
\pgfsetdash{}{0pt}%
\pgfpathmoveto{\pgfqpoint{0.725783in}{1.224541in}}%
\pgfpathcurveto{\pgfqpoint{0.734019in}{1.224541in}}{\pgfqpoint{0.741919in}{1.227813in}}{\pgfqpoint{0.747743in}{1.233637in}}%
\pgfpathcurveto{\pgfqpoint{0.753567in}{1.239461in}}{\pgfqpoint{0.756839in}{1.247361in}}{\pgfqpoint{0.756839in}{1.255597in}}%
\pgfpathcurveto{\pgfqpoint{0.756839in}{1.263833in}}{\pgfqpoint{0.753567in}{1.271733in}}{\pgfqpoint{0.747743in}{1.277557in}}%
\pgfpathcurveto{\pgfqpoint{0.741919in}{1.283381in}}{\pgfqpoint{0.734019in}{1.286654in}}{\pgfqpoint{0.725783in}{1.286654in}}%
\pgfpathcurveto{\pgfqpoint{0.717546in}{1.286654in}}{\pgfqpoint{0.709646in}{1.283381in}}{\pgfqpoint{0.703822in}{1.277557in}}%
\pgfpathcurveto{\pgfqpoint{0.697998in}{1.271733in}}{\pgfqpoint{0.694726in}{1.263833in}}{\pgfqpoint{0.694726in}{1.255597in}}%
\pgfpathcurveto{\pgfqpoint{0.694726in}{1.247361in}}{\pgfqpoint{0.697998in}{1.239461in}}{\pgfqpoint{0.703822in}{1.233637in}}%
\pgfpathcurveto{\pgfqpoint{0.709646in}{1.227813in}}{\pgfqpoint{0.717546in}{1.224541in}}{\pgfqpoint{0.725783in}{1.224541in}}%
\pgfpathclose%
\pgfusepath{stroke,fill}%
\end{pgfscope}%
\begin{pgfscope}%
\pgfpathrectangle{\pgfqpoint{0.100000in}{0.212622in}}{\pgfqpoint{3.696000in}{3.696000in}}%
\pgfusepath{clip}%
\pgfsetbuttcap%
\pgfsetroundjoin%
\definecolor{currentfill}{rgb}{0.121569,0.466667,0.705882}%
\pgfsetfillcolor{currentfill}%
\pgfsetfillopacity{0.660396}%
\pgfsetlinewidth{1.003750pt}%
\definecolor{currentstroke}{rgb}{0.121569,0.466667,0.705882}%
\pgfsetstrokecolor{currentstroke}%
\pgfsetstrokeopacity{0.660396}%
\pgfsetdash{}{0pt}%
\pgfpathmoveto{\pgfqpoint{0.725753in}{1.224502in}}%
\pgfpathcurveto{\pgfqpoint{0.733989in}{1.224502in}}{\pgfqpoint{0.741889in}{1.227774in}}{\pgfqpoint{0.747713in}{1.233598in}}%
\pgfpathcurveto{\pgfqpoint{0.753537in}{1.239422in}}{\pgfqpoint{0.756810in}{1.247322in}}{\pgfqpoint{0.756810in}{1.255558in}}%
\pgfpathcurveto{\pgfqpoint{0.756810in}{1.263795in}}{\pgfqpoint{0.753537in}{1.271695in}}{\pgfqpoint{0.747713in}{1.277519in}}%
\pgfpathcurveto{\pgfqpoint{0.741889in}{1.283342in}}{\pgfqpoint{0.733989in}{1.286615in}}{\pgfqpoint{0.725753in}{1.286615in}}%
\pgfpathcurveto{\pgfqpoint{0.717517in}{1.286615in}}{\pgfqpoint{0.709617in}{1.283342in}}{\pgfqpoint{0.703793in}{1.277519in}}%
\pgfpathcurveto{\pgfqpoint{0.697969in}{1.271695in}}{\pgfqpoint{0.694697in}{1.263795in}}{\pgfqpoint{0.694697in}{1.255558in}}%
\pgfpathcurveto{\pgfqpoint{0.694697in}{1.247322in}}{\pgfqpoint{0.697969in}{1.239422in}}{\pgfqpoint{0.703793in}{1.233598in}}%
\pgfpathcurveto{\pgfqpoint{0.709617in}{1.227774in}}{\pgfqpoint{0.717517in}{1.224502in}}{\pgfqpoint{0.725753in}{1.224502in}}%
\pgfpathclose%
\pgfusepath{stroke,fill}%
\end{pgfscope}%
\begin{pgfscope}%
\pgfpathrectangle{\pgfqpoint{0.100000in}{0.212622in}}{\pgfqpoint{3.696000in}{3.696000in}}%
\pgfusepath{clip}%
\pgfsetbuttcap%
\pgfsetroundjoin%
\definecolor{currentfill}{rgb}{0.121569,0.466667,0.705882}%
\pgfsetfillcolor{currentfill}%
\pgfsetfillopacity{0.660417}%
\pgfsetlinewidth{1.003750pt}%
\definecolor{currentstroke}{rgb}{0.121569,0.466667,0.705882}%
\pgfsetstrokecolor{currentstroke}%
\pgfsetstrokeopacity{0.660417}%
\pgfsetdash{}{0pt}%
\pgfpathmoveto{\pgfqpoint{0.725697in}{1.224440in}}%
\pgfpathcurveto{\pgfqpoint{0.733933in}{1.224440in}}{\pgfqpoint{0.741833in}{1.227713in}}{\pgfqpoint{0.747657in}{1.233536in}}%
\pgfpathcurveto{\pgfqpoint{0.753481in}{1.239360in}}{\pgfqpoint{0.756753in}{1.247260in}}{\pgfqpoint{0.756753in}{1.255497in}}%
\pgfpathcurveto{\pgfqpoint{0.756753in}{1.263733in}}{\pgfqpoint{0.753481in}{1.271633in}}{\pgfqpoint{0.747657in}{1.277457in}}%
\pgfpathcurveto{\pgfqpoint{0.741833in}{1.283281in}}{\pgfqpoint{0.733933in}{1.286553in}}{\pgfqpoint{0.725697in}{1.286553in}}%
\pgfpathcurveto{\pgfqpoint{0.717460in}{1.286553in}}{\pgfqpoint{0.709560in}{1.283281in}}{\pgfqpoint{0.703736in}{1.277457in}}%
\pgfpathcurveto{\pgfqpoint{0.697912in}{1.271633in}}{\pgfqpoint{0.694640in}{1.263733in}}{\pgfqpoint{0.694640in}{1.255497in}}%
\pgfpathcurveto{\pgfqpoint{0.694640in}{1.247260in}}{\pgfqpoint{0.697912in}{1.239360in}}{\pgfqpoint{0.703736in}{1.233536in}}%
\pgfpathcurveto{\pgfqpoint{0.709560in}{1.227713in}}{\pgfqpoint{0.717460in}{1.224440in}}{\pgfqpoint{0.725697in}{1.224440in}}%
\pgfpathclose%
\pgfusepath{stroke,fill}%
\end{pgfscope}%
\begin{pgfscope}%
\pgfpathrectangle{\pgfqpoint{0.100000in}{0.212622in}}{\pgfqpoint{3.696000in}{3.696000in}}%
\pgfusepath{clip}%
\pgfsetbuttcap%
\pgfsetroundjoin%
\definecolor{currentfill}{rgb}{0.121569,0.466667,0.705882}%
\pgfsetfillcolor{currentfill}%
\pgfsetfillopacity{0.660455}%
\pgfsetlinewidth{1.003750pt}%
\definecolor{currentstroke}{rgb}{0.121569,0.466667,0.705882}%
\pgfsetstrokecolor{currentstroke}%
\pgfsetstrokeopacity{0.660455}%
\pgfsetdash{}{0pt}%
\pgfpathmoveto{\pgfqpoint{0.725587in}{1.224333in}}%
\pgfpathcurveto{\pgfqpoint{0.733823in}{1.224333in}}{\pgfqpoint{0.741723in}{1.227606in}}{\pgfqpoint{0.747547in}{1.233430in}}%
\pgfpathcurveto{\pgfqpoint{0.753371in}{1.239254in}}{\pgfqpoint{0.756643in}{1.247154in}}{\pgfqpoint{0.756643in}{1.255390in}}%
\pgfpathcurveto{\pgfqpoint{0.756643in}{1.263626in}}{\pgfqpoint{0.753371in}{1.271526in}}{\pgfqpoint{0.747547in}{1.277350in}}%
\pgfpathcurveto{\pgfqpoint{0.741723in}{1.283174in}}{\pgfqpoint{0.733823in}{1.286446in}}{\pgfqpoint{0.725587in}{1.286446in}}%
\pgfpathcurveto{\pgfqpoint{0.717350in}{1.286446in}}{\pgfqpoint{0.709450in}{1.283174in}}{\pgfqpoint{0.703626in}{1.277350in}}%
\pgfpathcurveto{\pgfqpoint{0.697802in}{1.271526in}}{\pgfqpoint{0.694530in}{1.263626in}}{\pgfqpoint{0.694530in}{1.255390in}}%
\pgfpathcurveto{\pgfqpoint{0.694530in}{1.247154in}}{\pgfqpoint{0.697802in}{1.239254in}}{\pgfqpoint{0.703626in}{1.233430in}}%
\pgfpathcurveto{\pgfqpoint{0.709450in}{1.227606in}}{\pgfqpoint{0.717350in}{1.224333in}}{\pgfqpoint{0.725587in}{1.224333in}}%
\pgfpathclose%
\pgfusepath{stroke,fill}%
\end{pgfscope}%
\begin{pgfscope}%
\pgfpathrectangle{\pgfqpoint{0.100000in}{0.212622in}}{\pgfqpoint{3.696000in}{3.696000in}}%
\pgfusepath{clip}%
\pgfsetbuttcap%
\pgfsetroundjoin%
\definecolor{currentfill}{rgb}{0.121569,0.466667,0.705882}%
\pgfsetfillcolor{currentfill}%
\pgfsetfillopacity{0.660526}%
\pgfsetlinewidth{1.003750pt}%
\definecolor{currentstroke}{rgb}{0.121569,0.466667,0.705882}%
\pgfsetstrokecolor{currentstroke}%
\pgfsetstrokeopacity{0.660526}%
\pgfsetdash{}{0pt}%
\pgfpathmoveto{\pgfqpoint{0.725401in}{1.224136in}}%
\pgfpathcurveto{\pgfqpoint{0.733637in}{1.224136in}}{\pgfqpoint{0.741537in}{1.227408in}}{\pgfqpoint{0.747361in}{1.233232in}}%
\pgfpathcurveto{\pgfqpoint{0.753185in}{1.239056in}}{\pgfqpoint{0.756458in}{1.246956in}}{\pgfqpoint{0.756458in}{1.255192in}}%
\pgfpathcurveto{\pgfqpoint{0.756458in}{1.263429in}}{\pgfqpoint{0.753185in}{1.271329in}}{\pgfqpoint{0.747361in}{1.277153in}}%
\pgfpathcurveto{\pgfqpoint{0.741537in}{1.282976in}}{\pgfqpoint{0.733637in}{1.286249in}}{\pgfqpoint{0.725401in}{1.286249in}}%
\pgfpathcurveto{\pgfqpoint{0.717165in}{1.286249in}}{\pgfqpoint{0.709265in}{1.282976in}}{\pgfqpoint{0.703441in}{1.277153in}}%
\pgfpathcurveto{\pgfqpoint{0.697617in}{1.271329in}}{\pgfqpoint{0.694345in}{1.263429in}}{\pgfqpoint{0.694345in}{1.255192in}}%
\pgfpathcurveto{\pgfqpoint{0.694345in}{1.246956in}}{\pgfqpoint{0.697617in}{1.239056in}}{\pgfqpoint{0.703441in}{1.233232in}}%
\pgfpathcurveto{\pgfqpoint{0.709265in}{1.227408in}}{\pgfqpoint{0.717165in}{1.224136in}}{\pgfqpoint{0.725401in}{1.224136in}}%
\pgfpathclose%
\pgfusepath{stroke,fill}%
\end{pgfscope}%
\begin{pgfscope}%
\pgfpathrectangle{\pgfqpoint{0.100000in}{0.212622in}}{\pgfqpoint{3.696000in}{3.696000in}}%
\pgfusepath{clip}%
\pgfsetbuttcap%
\pgfsetroundjoin%
\definecolor{currentfill}{rgb}{0.121569,0.466667,0.705882}%
\pgfsetfillcolor{currentfill}%
\pgfsetfillopacity{0.660648}%
\pgfsetlinewidth{1.003750pt}%
\definecolor{currentstroke}{rgb}{0.121569,0.466667,0.705882}%
\pgfsetstrokecolor{currentstroke}%
\pgfsetstrokeopacity{0.660648}%
\pgfsetdash{}{0pt}%
\pgfpathmoveto{\pgfqpoint{0.725009in}{1.223799in}}%
\pgfpathcurveto{\pgfqpoint{0.733245in}{1.223799in}}{\pgfqpoint{0.741145in}{1.227071in}}{\pgfqpoint{0.746969in}{1.232895in}}%
\pgfpathcurveto{\pgfqpoint{0.752793in}{1.238719in}}{\pgfqpoint{0.756066in}{1.246619in}}{\pgfqpoint{0.756066in}{1.254856in}}%
\pgfpathcurveto{\pgfqpoint{0.756066in}{1.263092in}}{\pgfqpoint{0.752793in}{1.270992in}}{\pgfqpoint{0.746969in}{1.276816in}}%
\pgfpathcurveto{\pgfqpoint{0.741145in}{1.282640in}}{\pgfqpoint{0.733245in}{1.285912in}}{\pgfqpoint{0.725009in}{1.285912in}}%
\pgfpathcurveto{\pgfqpoint{0.716773in}{1.285912in}}{\pgfqpoint{0.708873in}{1.282640in}}{\pgfqpoint{0.703049in}{1.276816in}}%
\pgfpathcurveto{\pgfqpoint{0.697225in}{1.270992in}}{\pgfqpoint{0.693953in}{1.263092in}}{\pgfqpoint{0.693953in}{1.254856in}}%
\pgfpathcurveto{\pgfqpoint{0.693953in}{1.246619in}}{\pgfqpoint{0.697225in}{1.238719in}}{\pgfqpoint{0.703049in}{1.232895in}}%
\pgfpathcurveto{\pgfqpoint{0.708873in}{1.227071in}}{\pgfqpoint{0.716773in}{1.223799in}}{\pgfqpoint{0.725009in}{1.223799in}}%
\pgfpathclose%
\pgfusepath{stroke,fill}%
\end{pgfscope}%
\begin{pgfscope}%
\pgfpathrectangle{\pgfqpoint{0.100000in}{0.212622in}}{\pgfqpoint{3.696000in}{3.696000in}}%
\pgfusepath{clip}%
\pgfsetbuttcap%
\pgfsetroundjoin%
\definecolor{currentfill}{rgb}{0.121569,0.466667,0.705882}%
\pgfsetfillcolor{currentfill}%
\pgfsetfillopacity{0.660888}%
\pgfsetlinewidth{1.003750pt}%
\definecolor{currentstroke}{rgb}{0.121569,0.466667,0.705882}%
\pgfsetstrokecolor{currentstroke}%
\pgfsetstrokeopacity{0.660888}%
\pgfsetdash{}{0pt}%
\pgfpathmoveto{\pgfqpoint{0.724348in}{1.223236in}}%
\pgfpathcurveto{\pgfqpoint{0.732584in}{1.223236in}}{\pgfqpoint{0.740484in}{1.226508in}}{\pgfqpoint{0.746308in}{1.232332in}}%
\pgfpathcurveto{\pgfqpoint{0.752132in}{1.238156in}}{\pgfqpoint{0.755404in}{1.246056in}}{\pgfqpoint{0.755404in}{1.254292in}}%
\pgfpathcurveto{\pgfqpoint{0.755404in}{1.262529in}}{\pgfqpoint{0.752132in}{1.270429in}}{\pgfqpoint{0.746308in}{1.276253in}}%
\pgfpathcurveto{\pgfqpoint{0.740484in}{1.282077in}}{\pgfqpoint{0.732584in}{1.285349in}}{\pgfqpoint{0.724348in}{1.285349in}}%
\pgfpathcurveto{\pgfqpoint{0.716112in}{1.285349in}}{\pgfqpoint{0.708212in}{1.282077in}}{\pgfqpoint{0.702388in}{1.276253in}}%
\pgfpathcurveto{\pgfqpoint{0.696564in}{1.270429in}}{\pgfqpoint{0.693291in}{1.262529in}}{\pgfqpoint{0.693291in}{1.254292in}}%
\pgfpathcurveto{\pgfqpoint{0.693291in}{1.246056in}}{\pgfqpoint{0.696564in}{1.238156in}}{\pgfqpoint{0.702388in}{1.232332in}}%
\pgfpathcurveto{\pgfqpoint{0.708212in}{1.226508in}}{\pgfqpoint{0.716112in}{1.223236in}}{\pgfqpoint{0.724348in}{1.223236in}}%
\pgfpathclose%
\pgfusepath{stroke,fill}%
\end{pgfscope}%
\begin{pgfscope}%
\pgfpathrectangle{\pgfqpoint{0.100000in}{0.212622in}}{\pgfqpoint{3.696000in}{3.696000in}}%
\pgfusepath{clip}%
\pgfsetbuttcap%
\pgfsetroundjoin%
\definecolor{currentfill}{rgb}{0.121569,0.466667,0.705882}%
\pgfsetfillcolor{currentfill}%
\pgfsetfillopacity{0.660927}%
\pgfsetlinewidth{1.003750pt}%
\definecolor{currentstroke}{rgb}{0.121569,0.466667,0.705882}%
\pgfsetstrokecolor{currentstroke}%
\pgfsetstrokeopacity{0.660927}%
\pgfsetdash{}{0pt}%
\pgfpathmoveto{\pgfqpoint{3.200978in}{2.212097in}}%
\pgfpathcurveto{\pgfqpoint{3.209214in}{2.212097in}}{\pgfqpoint{3.217114in}{2.215369in}}{\pgfqpoint{3.222938in}{2.221193in}}%
\pgfpathcurveto{\pgfqpoint{3.228762in}{2.227017in}}{\pgfqpoint{3.232034in}{2.234917in}}{\pgfqpoint{3.232034in}{2.243154in}}%
\pgfpathcurveto{\pgfqpoint{3.232034in}{2.251390in}}{\pgfqpoint{3.228762in}{2.259290in}}{\pgfqpoint{3.222938in}{2.265114in}}%
\pgfpathcurveto{\pgfqpoint{3.217114in}{2.270938in}}{\pgfqpoint{3.209214in}{2.274210in}}{\pgfqpoint{3.200978in}{2.274210in}}%
\pgfpathcurveto{\pgfqpoint{3.192741in}{2.274210in}}{\pgfqpoint{3.184841in}{2.270938in}}{\pgfqpoint{3.179017in}{2.265114in}}%
\pgfpathcurveto{\pgfqpoint{3.173193in}{2.259290in}}{\pgfqpoint{3.169921in}{2.251390in}}{\pgfqpoint{3.169921in}{2.243154in}}%
\pgfpathcurveto{\pgfqpoint{3.169921in}{2.234917in}}{\pgfqpoint{3.173193in}{2.227017in}}{\pgfqpoint{3.179017in}{2.221193in}}%
\pgfpathcurveto{\pgfqpoint{3.184841in}{2.215369in}}{\pgfqpoint{3.192741in}{2.212097in}}{\pgfqpoint{3.200978in}{2.212097in}}%
\pgfpathclose%
\pgfusepath{stroke,fill}%
\end{pgfscope}%
\begin{pgfscope}%
\pgfpathrectangle{\pgfqpoint{0.100000in}{0.212622in}}{\pgfqpoint{3.696000in}{3.696000in}}%
\pgfusepath{clip}%
\pgfsetbuttcap%
\pgfsetroundjoin%
\definecolor{currentfill}{rgb}{0.121569,0.466667,0.705882}%
\pgfsetfillcolor{currentfill}%
\pgfsetfillopacity{0.661254}%
\pgfsetlinewidth{1.003750pt}%
\definecolor{currentstroke}{rgb}{0.121569,0.466667,0.705882}%
\pgfsetstrokecolor{currentstroke}%
\pgfsetstrokeopacity{0.661254}%
\pgfsetdash{}{0pt}%
\pgfpathmoveto{\pgfqpoint{0.738388in}{1.215628in}}%
\pgfpathcurveto{\pgfqpoint{0.746624in}{1.215628in}}{\pgfqpoint{0.754524in}{1.218901in}}{\pgfqpoint{0.760348in}{1.224725in}}%
\pgfpathcurveto{\pgfqpoint{0.766172in}{1.230548in}}{\pgfqpoint{0.769444in}{1.238448in}}{\pgfqpoint{0.769444in}{1.246685in}}%
\pgfpathcurveto{\pgfqpoint{0.769444in}{1.254921in}}{\pgfqpoint{0.766172in}{1.262821in}}{\pgfqpoint{0.760348in}{1.268645in}}%
\pgfpathcurveto{\pgfqpoint{0.754524in}{1.274469in}}{\pgfqpoint{0.746624in}{1.277741in}}{\pgfqpoint{0.738388in}{1.277741in}}%
\pgfpathcurveto{\pgfqpoint{0.730152in}{1.277741in}}{\pgfqpoint{0.722251in}{1.274469in}}{\pgfqpoint{0.716428in}{1.268645in}}%
\pgfpathcurveto{\pgfqpoint{0.710604in}{1.262821in}}{\pgfqpoint{0.707331in}{1.254921in}}{\pgfqpoint{0.707331in}{1.246685in}}%
\pgfpathcurveto{\pgfqpoint{0.707331in}{1.238448in}}{\pgfqpoint{0.710604in}{1.230548in}}{\pgfqpoint{0.716428in}{1.224725in}}%
\pgfpathcurveto{\pgfqpoint{0.722251in}{1.218901in}}{\pgfqpoint{0.730152in}{1.215628in}}{\pgfqpoint{0.738388in}{1.215628in}}%
\pgfpathclose%
\pgfusepath{stroke,fill}%
\end{pgfscope}%
\begin{pgfscope}%
\pgfpathrectangle{\pgfqpoint{0.100000in}{0.212622in}}{\pgfqpoint{3.696000in}{3.696000in}}%
\pgfusepath{clip}%
\pgfsetbuttcap%
\pgfsetroundjoin%
\definecolor{currentfill}{rgb}{0.121569,0.466667,0.705882}%
\pgfsetfillcolor{currentfill}%
\pgfsetfillopacity{0.661289}%
\pgfsetlinewidth{1.003750pt}%
\definecolor{currentstroke}{rgb}{0.121569,0.466667,0.705882}%
\pgfsetstrokecolor{currentstroke}%
\pgfsetstrokeopacity{0.661289}%
\pgfsetdash{}{0pt}%
\pgfpathmoveto{\pgfqpoint{0.723081in}{1.222071in}}%
\pgfpathcurveto{\pgfqpoint{0.731317in}{1.222071in}}{\pgfqpoint{0.739217in}{1.225344in}}{\pgfqpoint{0.745041in}{1.231168in}}%
\pgfpathcurveto{\pgfqpoint{0.750865in}{1.236991in}}{\pgfqpoint{0.754137in}{1.244892in}}{\pgfqpoint{0.754137in}{1.253128in}}%
\pgfpathcurveto{\pgfqpoint{0.754137in}{1.261364in}}{\pgfqpoint{0.750865in}{1.269264in}}{\pgfqpoint{0.745041in}{1.275088in}}%
\pgfpathcurveto{\pgfqpoint{0.739217in}{1.280912in}}{\pgfqpoint{0.731317in}{1.284184in}}{\pgfqpoint{0.723081in}{1.284184in}}%
\pgfpathcurveto{\pgfqpoint{0.714845in}{1.284184in}}{\pgfqpoint{0.706945in}{1.280912in}}{\pgfqpoint{0.701121in}{1.275088in}}%
\pgfpathcurveto{\pgfqpoint{0.695297in}{1.269264in}}{\pgfqpoint{0.692024in}{1.261364in}}{\pgfqpoint{0.692024in}{1.253128in}}%
\pgfpathcurveto{\pgfqpoint{0.692024in}{1.244892in}}{\pgfqpoint{0.695297in}{1.236991in}}{\pgfqpoint{0.701121in}{1.231168in}}%
\pgfpathcurveto{\pgfqpoint{0.706945in}{1.225344in}}{\pgfqpoint{0.714845in}{1.222071in}}{\pgfqpoint{0.723081in}{1.222071in}}%
\pgfpathclose%
\pgfusepath{stroke,fill}%
\end{pgfscope}%
\begin{pgfscope}%
\pgfpathrectangle{\pgfqpoint{0.100000in}{0.212622in}}{\pgfqpoint{3.696000in}{3.696000in}}%
\pgfusepath{clip}%
\pgfsetbuttcap%
\pgfsetroundjoin%
\definecolor{currentfill}{rgb}{0.121569,0.466667,0.705882}%
\pgfsetfillcolor{currentfill}%
\pgfsetfillopacity{0.661289}%
\pgfsetlinewidth{1.003750pt}%
\definecolor{currentstroke}{rgb}{0.121569,0.466667,0.705882}%
\pgfsetstrokecolor{currentstroke}%
\pgfsetstrokeopacity{0.661289}%
\pgfsetdash{}{0pt}%
\pgfpathmoveto{\pgfqpoint{0.723080in}{1.222071in}}%
\pgfpathcurveto{\pgfqpoint{0.731317in}{1.222071in}}{\pgfqpoint{0.739217in}{1.225343in}}{\pgfqpoint{0.745041in}{1.231167in}}%
\pgfpathcurveto{\pgfqpoint{0.750865in}{1.236991in}}{\pgfqpoint{0.754137in}{1.244891in}}{\pgfqpoint{0.754137in}{1.253127in}}%
\pgfpathcurveto{\pgfqpoint{0.754137in}{1.261364in}}{\pgfqpoint{0.750865in}{1.269264in}}{\pgfqpoint{0.745041in}{1.275088in}}%
\pgfpathcurveto{\pgfqpoint{0.739217in}{1.280912in}}{\pgfqpoint{0.731317in}{1.284184in}}{\pgfqpoint{0.723080in}{1.284184in}}%
\pgfpathcurveto{\pgfqpoint{0.714844in}{1.284184in}}{\pgfqpoint{0.706944in}{1.280912in}}{\pgfqpoint{0.701120in}{1.275088in}}%
\pgfpathcurveto{\pgfqpoint{0.695296in}{1.269264in}}{\pgfqpoint{0.692024in}{1.261364in}}{\pgfqpoint{0.692024in}{1.253127in}}%
\pgfpathcurveto{\pgfqpoint{0.692024in}{1.244891in}}{\pgfqpoint{0.695296in}{1.236991in}}{\pgfqpoint{0.701120in}{1.231167in}}%
\pgfpathcurveto{\pgfqpoint{0.706944in}{1.225343in}}{\pgfqpoint{0.714844in}{1.222071in}}{\pgfqpoint{0.723080in}{1.222071in}}%
\pgfpathclose%
\pgfusepath{stroke,fill}%
\end{pgfscope}%
\begin{pgfscope}%
\pgfpathrectangle{\pgfqpoint{0.100000in}{0.212622in}}{\pgfqpoint{3.696000in}{3.696000in}}%
\pgfusepath{clip}%
\pgfsetbuttcap%
\pgfsetroundjoin%
\definecolor{currentfill}{rgb}{0.121569,0.466667,0.705882}%
\pgfsetfillcolor{currentfill}%
\pgfsetfillopacity{0.661289}%
\pgfsetlinewidth{1.003750pt}%
\definecolor{currentstroke}{rgb}{0.121569,0.466667,0.705882}%
\pgfsetstrokecolor{currentstroke}%
\pgfsetstrokeopacity{0.661289}%
\pgfsetdash{}{0pt}%
\pgfpathmoveto{\pgfqpoint{0.723079in}{1.222070in}}%
\pgfpathcurveto{\pgfqpoint{0.731316in}{1.222070in}}{\pgfqpoint{0.739216in}{1.225342in}}{\pgfqpoint{0.745040in}{1.231166in}}%
\pgfpathcurveto{\pgfqpoint{0.750863in}{1.236990in}}{\pgfqpoint{0.754136in}{1.244890in}}{\pgfqpoint{0.754136in}{1.253126in}}%
\pgfpathcurveto{\pgfqpoint{0.754136in}{1.261363in}}{\pgfqpoint{0.750863in}{1.269263in}}{\pgfqpoint{0.745040in}{1.275087in}}%
\pgfpathcurveto{\pgfqpoint{0.739216in}{1.280911in}}{\pgfqpoint{0.731316in}{1.284183in}}{\pgfqpoint{0.723079in}{1.284183in}}%
\pgfpathcurveto{\pgfqpoint{0.714843in}{1.284183in}}{\pgfqpoint{0.706943in}{1.280911in}}{\pgfqpoint{0.701119in}{1.275087in}}%
\pgfpathcurveto{\pgfqpoint{0.695295in}{1.269263in}}{\pgfqpoint{0.692023in}{1.261363in}}{\pgfqpoint{0.692023in}{1.253126in}}%
\pgfpathcurveto{\pgfqpoint{0.692023in}{1.244890in}}{\pgfqpoint{0.695295in}{1.236990in}}{\pgfqpoint{0.701119in}{1.231166in}}%
\pgfpathcurveto{\pgfqpoint{0.706943in}{1.225342in}}{\pgfqpoint{0.714843in}{1.222070in}}{\pgfqpoint{0.723079in}{1.222070in}}%
\pgfpathclose%
\pgfusepath{stroke,fill}%
\end{pgfscope}%
\begin{pgfscope}%
\pgfpathrectangle{\pgfqpoint{0.100000in}{0.212622in}}{\pgfqpoint{3.696000in}{3.696000in}}%
\pgfusepath{clip}%
\pgfsetbuttcap%
\pgfsetroundjoin%
\definecolor{currentfill}{rgb}{0.121569,0.466667,0.705882}%
\pgfsetfillcolor{currentfill}%
\pgfsetfillopacity{0.661290}%
\pgfsetlinewidth{1.003750pt}%
\definecolor{currentstroke}{rgb}{0.121569,0.466667,0.705882}%
\pgfsetstrokecolor{currentstroke}%
\pgfsetstrokeopacity{0.661290}%
\pgfsetdash{}{0pt}%
\pgfpathmoveto{\pgfqpoint{0.723078in}{1.222068in}}%
\pgfpathcurveto{\pgfqpoint{0.731314in}{1.222068in}}{\pgfqpoint{0.739214in}{1.225340in}}{\pgfqpoint{0.745038in}{1.231164in}}%
\pgfpathcurveto{\pgfqpoint{0.750862in}{1.236988in}}{\pgfqpoint{0.754134in}{1.244888in}}{\pgfqpoint{0.754134in}{1.253125in}}%
\pgfpathcurveto{\pgfqpoint{0.754134in}{1.261361in}}{\pgfqpoint{0.750862in}{1.269261in}}{\pgfqpoint{0.745038in}{1.275085in}}%
\pgfpathcurveto{\pgfqpoint{0.739214in}{1.280909in}}{\pgfqpoint{0.731314in}{1.284181in}}{\pgfqpoint{0.723078in}{1.284181in}}%
\pgfpathcurveto{\pgfqpoint{0.714841in}{1.284181in}}{\pgfqpoint{0.706941in}{1.280909in}}{\pgfqpoint{0.701117in}{1.275085in}}%
\pgfpathcurveto{\pgfqpoint{0.695294in}{1.269261in}}{\pgfqpoint{0.692021in}{1.261361in}}{\pgfqpoint{0.692021in}{1.253125in}}%
\pgfpathcurveto{\pgfqpoint{0.692021in}{1.244888in}}{\pgfqpoint{0.695294in}{1.236988in}}{\pgfqpoint{0.701117in}{1.231164in}}%
\pgfpathcurveto{\pgfqpoint{0.706941in}{1.225340in}}{\pgfqpoint{0.714841in}{1.222068in}}{\pgfqpoint{0.723078in}{1.222068in}}%
\pgfpathclose%
\pgfusepath{stroke,fill}%
\end{pgfscope}%
\begin{pgfscope}%
\pgfpathrectangle{\pgfqpoint{0.100000in}{0.212622in}}{\pgfqpoint{3.696000in}{3.696000in}}%
\pgfusepath{clip}%
\pgfsetbuttcap%
\pgfsetroundjoin%
\definecolor{currentfill}{rgb}{0.121569,0.466667,0.705882}%
\pgfsetfillcolor{currentfill}%
\pgfsetfillopacity{0.661291}%
\pgfsetlinewidth{1.003750pt}%
\definecolor{currentstroke}{rgb}{0.121569,0.466667,0.705882}%
\pgfsetstrokecolor{currentstroke}%
\pgfsetstrokeopacity{0.661291}%
\pgfsetdash{}{0pt}%
\pgfpathmoveto{\pgfqpoint{0.723075in}{1.222065in}}%
\pgfpathcurveto{\pgfqpoint{0.731311in}{1.222065in}}{\pgfqpoint{0.739211in}{1.225337in}}{\pgfqpoint{0.745035in}{1.231161in}}%
\pgfpathcurveto{\pgfqpoint{0.750859in}{1.236985in}}{\pgfqpoint{0.754131in}{1.244885in}}{\pgfqpoint{0.754131in}{1.253122in}}%
\pgfpathcurveto{\pgfqpoint{0.754131in}{1.261358in}}{\pgfqpoint{0.750859in}{1.269258in}}{\pgfqpoint{0.745035in}{1.275082in}}%
\pgfpathcurveto{\pgfqpoint{0.739211in}{1.280906in}}{\pgfqpoint{0.731311in}{1.284178in}}{\pgfqpoint{0.723075in}{1.284178in}}%
\pgfpathcurveto{\pgfqpoint{0.714839in}{1.284178in}}{\pgfqpoint{0.706938in}{1.280906in}}{\pgfqpoint{0.701115in}{1.275082in}}%
\pgfpathcurveto{\pgfqpoint{0.695291in}{1.269258in}}{\pgfqpoint{0.692018in}{1.261358in}}{\pgfqpoint{0.692018in}{1.253122in}}%
\pgfpathcurveto{\pgfqpoint{0.692018in}{1.244885in}}{\pgfqpoint{0.695291in}{1.236985in}}{\pgfqpoint{0.701115in}{1.231161in}}%
\pgfpathcurveto{\pgfqpoint{0.706938in}{1.225337in}}{\pgfqpoint{0.714839in}{1.222065in}}{\pgfqpoint{0.723075in}{1.222065in}}%
\pgfpathclose%
\pgfusepath{stroke,fill}%
\end{pgfscope}%
\begin{pgfscope}%
\pgfpathrectangle{\pgfqpoint{0.100000in}{0.212622in}}{\pgfqpoint{3.696000in}{3.696000in}}%
\pgfusepath{clip}%
\pgfsetbuttcap%
\pgfsetroundjoin%
\definecolor{currentfill}{rgb}{0.121569,0.466667,0.705882}%
\pgfsetfillcolor{currentfill}%
\pgfsetfillopacity{0.661293}%
\pgfsetlinewidth{1.003750pt}%
\definecolor{currentstroke}{rgb}{0.121569,0.466667,0.705882}%
\pgfsetstrokecolor{currentstroke}%
\pgfsetstrokeopacity{0.661293}%
\pgfsetdash{}{0pt}%
\pgfpathmoveto{\pgfqpoint{0.723069in}{1.222058in}}%
\pgfpathcurveto{\pgfqpoint{0.731305in}{1.222058in}}{\pgfqpoint{0.739205in}{1.225331in}}{\pgfqpoint{0.745029in}{1.231155in}}%
\pgfpathcurveto{\pgfqpoint{0.750853in}{1.236979in}}{\pgfqpoint{0.754126in}{1.244879in}}{\pgfqpoint{0.754126in}{1.253115in}}%
\pgfpathcurveto{\pgfqpoint{0.754126in}{1.261351in}}{\pgfqpoint{0.750853in}{1.269251in}}{\pgfqpoint{0.745029in}{1.275075in}}%
\pgfpathcurveto{\pgfqpoint{0.739205in}{1.280899in}}{\pgfqpoint{0.731305in}{1.284171in}}{\pgfqpoint{0.723069in}{1.284171in}}%
\pgfpathcurveto{\pgfqpoint{0.714833in}{1.284171in}}{\pgfqpoint{0.706933in}{1.280899in}}{\pgfqpoint{0.701109in}{1.275075in}}%
\pgfpathcurveto{\pgfqpoint{0.695285in}{1.269251in}}{\pgfqpoint{0.692013in}{1.261351in}}{\pgfqpoint{0.692013in}{1.253115in}}%
\pgfpathcurveto{\pgfqpoint{0.692013in}{1.244879in}}{\pgfqpoint{0.695285in}{1.236979in}}{\pgfqpoint{0.701109in}{1.231155in}}%
\pgfpathcurveto{\pgfqpoint{0.706933in}{1.225331in}}{\pgfqpoint{0.714833in}{1.222058in}}{\pgfqpoint{0.723069in}{1.222058in}}%
\pgfpathclose%
\pgfusepath{stroke,fill}%
\end{pgfscope}%
\begin{pgfscope}%
\pgfpathrectangle{\pgfqpoint{0.100000in}{0.212622in}}{\pgfqpoint{3.696000in}{3.696000in}}%
\pgfusepath{clip}%
\pgfsetbuttcap%
\pgfsetroundjoin%
\definecolor{currentfill}{rgb}{0.121569,0.466667,0.705882}%
\pgfsetfillcolor{currentfill}%
\pgfsetfillopacity{0.661298}%
\pgfsetlinewidth{1.003750pt}%
\definecolor{currentstroke}{rgb}{0.121569,0.466667,0.705882}%
\pgfsetstrokecolor{currentstroke}%
\pgfsetstrokeopacity{0.661298}%
\pgfsetdash{}{0pt}%
\pgfpathmoveto{\pgfqpoint{0.723060in}{1.222050in}}%
\pgfpathcurveto{\pgfqpoint{0.731296in}{1.222050in}}{\pgfqpoint{0.739196in}{1.225322in}}{\pgfqpoint{0.745020in}{1.231146in}}%
\pgfpathcurveto{\pgfqpoint{0.750844in}{1.236970in}}{\pgfqpoint{0.754116in}{1.244870in}}{\pgfqpoint{0.754116in}{1.253106in}}%
\pgfpathcurveto{\pgfqpoint{0.754116in}{1.261343in}}{\pgfqpoint{0.750844in}{1.269243in}}{\pgfqpoint{0.745020in}{1.275067in}}%
\pgfpathcurveto{\pgfqpoint{0.739196in}{1.280890in}}{\pgfqpoint{0.731296in}{1.284163in}}{\pgfqpoint{0.723060in}{1.284163in}}%
\pgfpathcurveto{\pgfqpoint{0.714824in}{1.284163in}}{\pgfqpoint{0.706924in}{1.280890in}}{\pgfqpoint{0.701100in}{1.275067in}}%
\pgfpathcurveto{\pgfqpoint{0.695276in}{1.269243in}}{\pgfqpoint{0.692003in}{1.261343in}}{\pgfqpoint{0.692003in}{1.253106in}}%
\pgfpathcurveto{\pgfqpoint{0.692003in}{1.244870in}}{\pgfqpoint{0.695276in}{1.236970in}}{\pgfqpoint{0.701100in}{1.231146in}}%
\pgfpathcurveto{\pgfqpoint{0.706924in}{1.225322in}}{\pgfqpoint{0.714824in}{1.222050in}}{\pgfqpoint{0.723060in}{1.222050in}}%
\pgfpathclose%
\pgfusepath{stroke,fill}%
\end{pgfscope}%
\begin{pgfscope}%
\pgfpathrectangle{\pgfqpoint{0.100000in}{0.212622in}}{\pgfqpoint{3.696000in}{3.696000in}}%
\pgfusepath{clip}%
\pgfsetbuttcap%
\pgfsetroundjoin%
\definecolor{currentfill}{rgb}{0.121569,0.466667,0.705882}%
\pgfsetfillcolor{currentfill}%
\pgfsetfillopacity{0.661304}%
\pgfsetlinewidth{1.003750pt}%
\definecolor{currentstroke}{rgb}{0.121569,0.466667,0.705882}%
\pgfsetstrokecolor{currentstroke}%
\pgfsetstrokeopacity{0.661304}%
\pgfsetdash{}{0pt}%
\pgfpathmoveto{\pgfqpoint{0.723040in}{1.222026in}}%
\pgfpathcurveto{\pgfqpoint{0.731276in}{1.222026in}}{\pgfqpoint{0.739176in}{1.225298in}}{\pgfqpoint{0.745000in}{1.231122in}}%
\pgfpathcurveto{\pgfqpoint{0.750824in}{1.236946in}}{\pgfqpoint{0.754096in}{1.244846in}}{\pgfqpoint{0.754096in}{1.253082in}}%
\pgfpathcurveto{\pgfqpoint{0.754096in}{1.261318in}}{\pgfqpoint{0.750824in}{1.269218in}}{\pgfqpoint{0.745000in}{1.275042in}}%
\pgfpathcurveto{\pgfqpoint{0.739176in}{1.280866in}}{\pgfqpoint{0.731276in}{1.284139in}}{\pgfqpoint{0.723040in}{1.284139in}}%
\pgfpathcurveto{\pgfqpoint{0.714804in}{1.284139in}}{\pgfqpoint{0.706903in}{1.280866in}}{\pgfqpoint{0.701080in}{1.275042in}}%
\pgfpathcurveto{\pgfqpoint{0.695256in}{1.269218in}}{\pgfqpoint{0.691983in}{1.261318in}}{\pgfqpoint{0.691983in}{1.253082in}}%
\pgfpathcurveto{\pgfqpoint{0.691983in}{1.244846in}}{\pgfqpoint{0.695256in}{1.236946in}}{\pgfqpoint{0.701080in}{1.231122in}}%
\pgfpathcurveto{\pgfqpoint{0.706903in}{1.225298in}}{\pgfqpoint{0.714804in}{1.222026in}}{\pgfqpoint{0.723040in}{1.222026in}}%
\pgfpathclose%
\pgfusepath{stroke,fill}%
\end{pgfscope}%
\begin{pgfscope}%
\pgfpathrectangle{\pgfqpoint{0.100000in}{0.212622in}}{\pgfqpoint{3.696000in}{3.696000in}}%
\pgfusepath{clip}%
\pgfsetbuttcap%
\pgfsetroundjoin%
\definecolor{currentfill}{rgb}{0.121569,0.466667,0.705882}%
\pgfsetfillcolor{currentfill}%
\pgfsetfillopacity{0.661317}%
\pgfsetlinewidth{1.003750pt}%
\definecolor{currentstroke}{rgb}{0.121569,0.466667,0.705882}%
\pgfsetstrokecolor{currentstroke}%
\pgfsetstrokeopacity{0.661317}%
\pgfsetdash{}{0pt}%
\pgfpathmoveto{\pgfqpoint{0.723011in}{1.221986in}}%
\pgfpathcurveto{\pgfqpoint{0.731248in}{1.221986in}}{\pgfqpoint{0.739148in}{1.225258in}}{\pgfqpoint{0.744972in}{1.231082in}}%
\pgfpathcurveto{\pgfqpoint{0.750795in}{1.236906in}}{\pgfqpoint{0.754068in}{1.244806in}}{\pgfqpoint{0.754068in}{1.253042in}}%
\pgfpathcurveto{\pgfqpoint{0.754068in}{1.261279in}}{\pgfqpoint{0.750795in}{1.269179in}}{\pgfqpoint{0.744972in}{1.275003in}}%
\pgfpathcurveto{\pgfqpoint{0.739148in}{1.280827in}}{\pgfqpoint{0.731248in}{1.284099in}}{\pgfqpoint{0.723011in}{1.284099in}}%
\pgfpathcurveto{\pgfqpoint{0.714775in}{1.284099in}}{\pgfqpoint{0.706875in}{1.280827in}}{\pgfqpoint{0.701051in}{1.275003in}}%
\pgfpathcurveto{\pgfqpoint{0.695227in}{1.269179in}}{\pgfqpoint{0.691955in}{1.261279in}}{\pgfqpoint{0.691955in}{1.253042in}}%
\pgfpathcurveto{\pgfqpoint{0.691955in}{1.244806in}}{\pgfqpoint{0.695227in}{1.236906in}}{\pgfqpoint{0.701051in}{1.231082in}}%
\pgfpathcurveto{\pgfqpoint{0.706875in}{1.225258in}}{\pgfqpoint{0.714775in}{1.221986in}}{\pgfqpoint{0.723011in}{1.221986in}}%
\pgfpathclose%
\pgfusepath{stroke,fill}%
\end{pgfscope}%
\begin{pgfscope}%
\pgfpathrectangle{\pgfqpoint{0.100000in}{0.212622in}}{\pgfqpoint{3.696000in}{3.696000in}}%
\pgfusepath{clip}%
\pgfsetbuttcap%
\pgfsetroundjoin%
\definecolor{currentfill}{rgb}{0.121569,0.466667,0.705882}%
\pgfsetfillcolor{currentfill}%
\pgfsetfillopacity{0.661342}%
\pgfsetlinewidth{1.003750pt}%
\definecolor{currentstroke}{rgb}{0.121569,0.466667,0.705882}%
\pgfsetstrokecolor{currentstroke}%
\pgfsetstrokeopacity{0.661342}%
\pgfsetdash{}{0pt}%
\pgfpathmoveto{\pgfqpoint{0.722954in}{1.221924in}}%
\pgfpathcurveto{\pgfqpoint{0.731191in}{1.221924in}}{\pgfqpoint{0.739091in}{1.225197in}}{\pgfqpoint{0.744915in}{1.231021in}}%
\pgfpathcurveto{\pgfqpoint{0.750739in}{1.236844in}}{\pgfqpoint{0.754011in}{1.244744in}}{\pgfqpoint{0.754011in}{1.252981in}}%
\pgfpathcurveto{\pgfqpoint{0.754011in}{1.261217in}}{\pgfqpoint{0.750739in}{1.269117in}}{\pgfqpoint{0.744915in}{1.274941in}}%
\pgfpathcurveto{\pgfqpoint{0.739091in}{1.280765in}}{\pgfqpoint{0.731191in}{1.284037in}}{\pgfqpoint{0.722954in}{1.284037in}}%
\pgfpathcurveto{\pgfqpoint{0.714718in}{1.284037in}}{\pgfqpoint{0.706818in}{1.280765in}}{\pgfqpoint{0.700994in}{1.274941in}}%
\pgfpathcurveto{\pgfqpoint{0.695170in}{1.269117in}}{\pgfqpoint{0.691898in}{1.261217in}}{\pgfqpoint{0.691898in}{1.252981in}}%
\pgfpathcurveto{\pgfqpoint{0.691898in}{1.244744in}}{\pgfqpoint{0.695170in}{1.236844in}}{\pgfqpoint{0.700994in}{1.231021in}}%
\pgfpathcurveto{\pgfqpoint{0.706818in}{1.225197in}}{\pgfqpoint{0.714718in}{1.221924in}}{\pgfqpoint{0.722954in}{1.221924in}}%
\pgfpathclose%
\pgfusepath{stroke,fill}%
\end{pgfscope}%
\begin{pgfscope}%
\pgfpathrectangle{\pgfqpoint{0.100000in}{0.212622in}}{\pgfqpoint{3.696000in}{3.696000in}}%
\pgfusepath{clip}%
\pgfsetbuttcap%
\pgfsetroundjoin%
\definecolor{currentfill}{rgb}{0.121569,0.466667,0.705882}%
\pgfsetfillcolor{currentfill}%
\pgfsetfillopacity{0.661376}%
\pgfsetlinewidth{1.003750pt}%
\definecolor{currentstroke}{rgb}{0.121569,0.466667,0.705882}%
\pgfsetstrokecolor{currentstroke}%
\pgfsetstrokeopacity{0.661376}%
\pgfsetdash{}{0pt}%
\pgfpathmoveto{\pgfqpoint{0.722868in}{1.221735in}}%
\pgfpathcurveto{\pgfqpoint{0.731104in}{1.221735in}}{\pgfqpoint{0.739004in}{1.225007in}}{\pgfqpoint{0.744828in}{1.230831in}}%
\pgfpathcurveto{\pgfqpoint{0.750652in}{1.236655in}}{\pgfqpoint{0.753925in}{1.244555in}}{\pgfqpoint{0.753925in}{1.252791in}}%
\pgfpathcurveto{\pgfqpoint{0.753925in}{1.261027in}}{\pgfqpoint{0.750652in}{1.268927in}}{\pgfqpoint{0.744828in}{1.274751in}}%
\pgfpathcurveto{\pgfqpoint{0.739004in}{1.280575in}}{\pgfqpoint{0.731104in}{1.283848in}}{\pgfqpoint{0.722868in}{1.283848in}}%
\pgfpathcurveto{\pgfqpoint{0.714632in}{1.283848in}}{\pgfqpoint{0.706732in}{1.280575in}}{\pgfqpoint{0.700908in}{1.274751in}}%
\pgfpathcurveto{\pgfqpoint{0.695084in}{1.268927in}}{\pgfqpoint{0.691812in}{1.261027in}}{\pgfqpoint{0.691812in}{1.252791in}}%
\pgfpathcurveto{\pgfqpoint{0.691812in}{1.244555in}}{\pgfqpoint{0.695084in}{1.236655in}}{\pgfqpoint{0.700908in}{1.230831in}}%
\pgfpathcurveto{\pgfqpoint{0.706732in}{1.225007in}}{\pgfqpoint{0.714632in}{1.221735in}}{\pgfqpoint{0.722868in}{1.221735in}}%
\pgfpathclose%
\pgfusepath{stroke,fill}%
\end{pgfscope}%
\begin{pgfscope}%
\pgfpathrectangle{\pgfqpoint{0.100000in}{0.212622in}}{\pgfqpoint{3.696000in}{3.696000in}}%
\pgfusepath{clip}%
\pgfsetbuttcap%
\pgfsetroundjoin%
\definecolor{currentfill}{rgb}{0.121569,0.466667,0.705882}%
\pgfsetfillcolor{currentfill}%
\pgfsetfillopacity{0.661441}%
\pgfsetlinewidth{1.003750pt}%
\definecolor{currentstroke}{rgb}{0.121569,0.466667,0.705882}%
\pgfsetstrokecolor{currentstroke}%
\pgfsetstrokeopacity{0.661441}%
\pgfsetdash{}{0pt}%
\pgfpathmoveto{\pgfqpoint{0.722726in}{1.221384in}}%
\pgfpathcurveto{\pgfqpoint{0.730962in}{1.221384in}}{\pgfqpoint{0.738862in}{1.224657in}}{\pgfqpoint{0.744686in}{1.230480in}}%
\pgfpathcurveto{\pgfqpoint{0.750510in}{1.236304in}}{\pgfqpoint{0.753783in}{1.244204in}}{\pgfqpoint{0.753783in}{1.252441in}}%
\pgfpathcurveto{\pgfqpoint{0.753783in}{1.260677in}}{\pgfqpoint{0.750510in}{1.268577in}}{\pgfqpoint{0.744686in}{1.274401in}}%
\pgfpathcurveto{\pgfqpoint{0.738862in}{1.280225in}}{\pgfqpoint{0.730962in}{1.283497in}}{\pgfqpoint{0.722726in}{1.283497in}}%
\pgfpathcurveto{\pgfqpoint{0.714490in}{1.283497in}}{\pgfqpoint{0.706590in}{1.280225in}}{\pgfqpoint{0.700766in}{1.274401in}}%
\pgfpathcurveto{\pgfqpoint{0.694942in}{1.268577in}}{\pgfqpoint{0.691670in}{1.260677in}}{\pgfqpoint{0.691670in}{1.252441in}}%
\pgfpathcurveto{\pgfqpoint{0.691670in}{1.244204in}}{\pgfqpoint{0.694942in}{1.236304in}}{\pgfqpoint{0.700766in}{1.230480in}}%
\pgfpathcurveto{\pgfqpoint{0.706590in}{1.224657in}}{\pgfqpoint{0.714490in}{1.221384in}}{\pgfqpoint{0.722726in}{1.221384in}}%
\pgfpathclose%
\pgfusepath{stroke,fill}%
\end{pgfscope}%
\begin{pgfscope}%
\pgfpathrectangle{\pgfqpoint{0.100000in}{0.212622in}}{\pgfqpoint{3.696000in}{3.696000in}}%
\pgfusepath{clip}%
\pgfsetbuttcap%
\pgfsetroundjoin%
\definecolor{currentfill}{rgb}{0.121569,0.466667,0.705882}%
\pgfsetfillcolor{currentfill}%
\pgfsetfillopacity{0.661554}%
\pgfsetlinewidth{1.003750pt}%
\definecolor{currentstroke}{rgb}{0.121569,0.466667,0.705882}%
\pgfsetstrokecolor{currentstroke}%
\pgfsetstrokeopacity{0.661554}%
\pgfsetdash{}{0pt}%
\pgfpathmoveto{\pgfqpoint{0.722413in}{1.220772in}}%
\pgfpathcurveto{\pgfqpoint{0.730649in}{1.220772in}}{\pgfqpoint{0.738549in}{1.224045in}}{\pgfqpoint{0.744373in}{1.229869in}}%
\pgfpathcurveto{\pgfqpoint{0.750197in}{1.235693in}}{\pgfqpoint{0.753469in}{1.243593in}}{\pgfqpoint{0.753469in}{1.251829in}}%
\pgfpathcurveto{\pgfqpoint{0.753469in}{1.260065in}}{\pgfqpoint{0.750197in}{1.267965in}}{\pgfqpoint{0.744373in}{1.273789in}}%
\pgfpathcurveto{\pgfqpoint{0.738549in}{1.279613in}}{\pgfqpoint{0.730649in}{1.282885in}}{\pgfqpoint{0.722413in}{1.282885in}}%
\pgfpathcurveto{\pgfqpoint{0.714176in}{1.282885in}}{\pgfqpoint{0.706276in}{1.279613in}}{\pgfqpoint{0.700453in}{1.273789in}}%
\pgfpathcurveto{\pgfqpoint{0.694629in}{1.267965in}}{\pgfqpoint{0.691356in}{1.260065in}}{\pgfqpoint{0.691356in}{1.251829in}}%
\pgfpathcurveto{\pgfqpoint{0.691356in}{1.243593in}}{\pgfqpoint{0.694629in}{1.235693in}}{\pgfqpoint{0.700453in}{1.229869in}}%
\pgfpathcurveto{\pgfqpoint{0.706276in}{1.224045in}}{\pgfqpoint{0.714176in}{1.220772in}}{\pgfqpoint{0.722413in}{1.220772in}}%
\pgfpathclose%
\pgfusepath{stroke,fill}%
\end{pgfscope}%
\begin{pgfscope}%
\pgfpathrectangle{\pgfqpoint{0.100000in}{0.212622in}}{\pgfqpoint{3.696000in}{3.696000in}}%
\pgfusepath{clip}%
\pgfsetbuttcap%
\pgfsetroundjoin%
\definecolor{currentfill}{rgb}{0.121569,0.466667,0.705882}%
\pgfsetfillcolor{currentfill}%
\pgfsetfillopacity{0.661810}%
\pgfsetlinewidth{1.003750pt}%
\definecolor{currentstroke}{rgb}{0.121569,0.466667,0.705882}%
\pgfsetstrokecolor{currentstroke}%
\pgfsetstrokeopacity{0.661810}%
\pgfsetdash{}{0pt}%
\pgfpathmoveto{\pgfqpoint{0.722047in}{1.219839in}}%
\pgfpathcurveto{\pgfqpoint{0.730284in}{1.219839in}}{\pgfqpoint{0.738184in}{1.223111in}}{\pgfqpoint{0.744008in}{1.228935in}}%
\pgfpathcurveto{\pgfqpoint{0.749832in}{1.234759in}}{\pgfqpoint{0.753104in}{1.242659in}}{\pgfqpoint{0.753104in}{1.250896in}}%
\pgfpathcurveto{\pgfqpoint{0.753104in}{1.259132in}}{\pgfqpoint{0.749832in}{1.267032in}}{\pgfqpoint{0.744008in}{1.272856in}}%
\pgfpathcurveto{\pgfqpoint{0.738184in}{1.278680in}}{\pgfqpoint{0.730284in}{1.281952in}}{\pgfqpoint{0.722047in}{1.281952in}}%
\pgfpathcurveto{\pgfqpoint{0.713811in}{1.281952in}}{\pgfqpoint{0.705911in}{1.278680in}}{\pgfqpoint{0.700087in}{1.272856in}}%
\pgfpathcurveto{\pgfqpoint{0.694263in}{1.267032in}}{\pgfqpoint{0.690991in}{1.259132in}}{\pgfqpoint{0.690991in}{1.250896in}}%
\pgfpathcurveto{\pgfqpoint{0.690991in}{1.242659in}}{\pgfqpoint{0.694263in}{1.234759in}}{\pgfqpoint{0.700087in}{1.228935in}}%
\pgfpathcurveto{\pgfqpoint{0.705911in}{1.223111in}}{\pgfqpoint{0.713811in}{1.219839in}}{\pgfqpoint{0.722047in}{1.219839in}}%
\pgfpathclose%
\pgfusepath{stroke,fill}%
\end{pgfscope}%
\begin{pgfscope}%
\pgfpathrectangle{\pgfqpoint{0.100000in}{0.212622in}}{\pgfqpoint{3.696000in}{3.696000in}}%
\pgfusepath{clip}%
\pgfsetbuttcap%
\pgfsetroundjoin%
\definecolor{currentfill}{rgb}{0.121569,0.466667,0.705882}%
\pgfsetfillcolor{currentfill}%
\pgfsetfillopacity{0.662253}%
\pgfsetlinewidth{1.003750pt}%
\definecolor{currentstroke}{rgb}{0.121569,0.466667,0.705882}%
\pgfsetstrokecolor{currentstroke}%
\pgfsetstrokeopacity{0.662253}%
\pgfsetdash{}{0pt}%
\pgfpathmoveto{\pgfqpoint{0.720851in}{1.218361in}}%
\pgfpathcurveto{\pgfqpoint{0.729087in}{1.218361in}}{\pgfqpoint{0.736987in}{1.221633in}}{\pgfqpoint{0.742811in}{1.227457in}}%
\pgfpathcurveto{\pgfqpoint{0.748635in}{1.233281in}}{\pgfqpoint{0.751908in}{1.241181in}}{\pgfqpoint{0.751908in}{1.249417in}}%
\pgfpathcurveto{\pgfqpoint{0.751908in}{1.257654in}}{\pgfqpoint{0.748635in}{1.265554in}}{\pgfqpoint{0.742811in}{1.271378in}}%
\pgfpathcurveto{\pgfqpoint{0.736987in}{1.277201in}}{\pgfqpoint{0.729087in}{1.280474in}}{\pgfqpoint{0.720851in}{1.280474in}}%
\pgfpathcurveto{\pgfqpoint{0.712615in}{1.280474in}}{\pgfqpoint{0.704715in}{1.277201in}}{\pgfqpoint{0.698891in}{1.271378in}}%
\pgfpathcurveto{\pgfqpoint{0.693067in}{1.265554in}}{\pgfqpoint{0.689795in}{1.257654in}}{\pgfqpoint{0.689795in}{1.249417in}}%
\pgfpathcurveto{\pgfqpoint{0.689795in}{1.241181in}}{\pgfqpoint{0.693067in}{1.233281in}}{\pgfqpoint{0.698891in}{1.227457in}}%
\pgfpathcurveto{\pgfqpoint{0.704715in}{1.221633in}}{\pgfqpoint{0.712615in}{1.218361in}}{\pgfqpoint{0.720851in}{1.218361in}}%
\pgfpathclose%
\pgfusepath{stroke,fill}%
\end{pgfscope}%
\begin{pgfscope}%
\pgfpathrectangle{\pgfqpoint{0.100000in}{0.212622in}}{\pgfqpoint{3.696000in}{3.696000in}}%
\pgfusepath{clip}%
\pgfsetbuttcap%
\pgfsetroundjoin%
\definecolor{currentfill}{rgb}{0.121569,0.466667,0.705882}%
\pgfsetfillcolor{currentfill}%
\pgfsetfillopacity{0.662253}%
\pgfsetlinewidth{1.003750pt}%
\definecolor{currentstroke}{rgb}{0.121569,0.466667,0.705882}%
\pgfsetstrokecolor{currentstroke}%
\pgfsetstrokeopacity{0.662253}%
\pgfsetdash{}{0pt}%
\pgfpathmoveto{\pgfqpoint{0.720851in}{1.218360in}}%
\pgfpathcurveto{\pgfqpoint{0.729087in}{1.218360in}}{\pgfqpoint{0.736987in}{1.221632in}}{\pgfqpoint{0.742811in}{1.227456in}}%
\pgfpathcurveto{\pgfqpoint{0.748635in}{1.233280in}}{\pgfqpoint{0.751907in}{1.241180in}}{\pgfqpoint{0.751907in}{1.249417in}}%
\pgfpathcurveto{\pgfqpoint{0.751907in}{1.257653in}}{\pgfqpoint{0.748635in}{1.265553in}}{\pgfqpoint{0.742811in}{1.271377in}}%
\pgfpathcurveto{\pgfqpoint{0.736987in}{1.277201in}}{\pgfqpoint{0.729087in}{1.280473in}}{\pgfqpoint{0.720851in}{1.280473in}}%
\pgfpathcurveto{\pgfqpoint{0.712614in}{1.280473in}}{\pgfqpoint{0.704714in}{1.277201in}}{\pgfqpoint{0.698890in}{1.271377in}}%
\pgfpathcurveto{\pgfqpoint{0.693066in}{1.265553in}}{\pgfqpoint{0.689794in}{1.257653in}}{\pgfqpoint{0.689794in}{1.249417in}}%
\pgfpathcurveto{\pgfqpoint{0.689794in}{1.241180in}}{\pgfqpoint{0.693066in}{1.233280in}}{\pgfqpoint{0.698890in}{1.227456in}}%
\pgfpathcurveto{\pgfqpoint{0.704714in}{1.221632in}}{\pgfqpoint{0.712614in}{1.218360in}}{\pgfqpoint{0.720851in}{1.218360in}}%
\pgfpathclose%
\pgfusepath{stroke,fill}%
\end{pgfscope}%
\begin{pgfscope}%
\pgfpathrectangle{\pgfqpoint{0.100000in}{0.212622in}}{\pgfqpoint{3.696000in}{3.696000in}}%
\pgfusepath{clip}%
\pgfsetbuttcap%
\pgfsetroundjoin%
\definecolor{currentfill}{rgb}{0.121569,0.466667,0.705882}%
\pgfsetfillcolor{currentfill}%
\pgfsetfillopacity{0.662254}%
\pgfsetlinewidth{1.003750pt}%
\definecolor{currentstroke}{rgb}{0.121569,0.466667,0.705882}%
\pgfsetstrokecolor{currentstroke}%
\pgfsetstrokeopacity{0.662254}%
\pgfsetdash{}{0pt}%
\pgfpathmoveto{\pgfqpoint{0.720850in}{1.218359in}}%
\pgfpathcurveto{\pgfqpoint{0.729086in}{1.218359in}}{\pgfqpoint{0.736986in}{1.221631in}}{\pgfqpoint{0.742810in}{1.227455in}}%
\pgfpathcurveto{\pgfqpoint{0.748634in}{1.233279in}}{\pgfqpoint{0.751906in}{1.241179in}}{\pgfqpoint{0.751906in}{1.249415in}}%
\pgfpathcurveto{\pgfqpoint{0.751906in}{1.257651in}}{\pgfqpoint{0.748634in}{1.265552in}}{\pgfqpoint{0.742810in}{1.271375in}}%
\pgfpathcurveto{\pgfqpoint{0.736986in}{1.277199in}}{\pgfqpoint{0.729086in}{1.280472in}}{\pgfqpoint{0.720850in}{1.280472in}}%
\pgfpathcurveto{\pgfqpoint{0.712613in}{1.280472in}}{\pgfqpoint{0.704713in}{1.277199in}}{\pgfqpoint{0.698889in}{1.271375in}}%
\pgfpathcurveto{\pgfqpoint{0.693065in}{1.265552in}}{\pgfqpoint{0.689793in}{1.257651in}}{\pgfqpoint{0.689793in}{1.249415in}}%
\pgfpathcurveto{\pgfqpoint{0.689793in}{1.241179in}}{\pgfqpoint{0.693065in}{1.233279in}}{\pgfqpoint{0.698889in}{1.227455in}}%
\pgfpathcurveto{\pgfqpoint{0.704713in}{1.221631in}}{\pgfqpoint{0.712613in}{1.218359in}}{\pgfqpoint{0.720850in}{1.218359in}}%
\pgfpathclose%
\pgfusepath{stroke,fill}%
\end{pgfscope}%
\begin{pgfscope}%
\pgfpathrectangle{\pgfqpoint{0.100000in}{0.212622in}}{\pgfqpoint{3.696000in}{3.696000in}}%
\pgfusepath{clip}%
\pgfsetbuttcap%
\pgfsetroundjoin%
\definecolor{currentfill}{rgb}{0.121569,0.466667,0.705882}%
\pgfsetfillcolor{currentfill}%
\pgfsetfillopacity{0.662254}%
\pgfsetlinewidth{1.003750pt}%
\definecolor{currentstroke}{rgb}{0.121569,0.466667,0.705882}%
\pgfsetstrokecolor{currentstroke}%
\pgfsetstrokeopacity{0.662254}%
\pgfsetdash{}{0pt}%
\pgfpathmoveto{\pgfqpoint{0.720848in}{1.218356in}}%
\pgfpathcurveto{\pgfqpoint{0.729084in}{1.218356in}}{\pgfqpoint{0.736984in}{1.221629in}}{\pgfqpoint{0.742808in}{1.227453in}}%
\pgfpathcurveto{\pgfqpoint{0.748632in}{1.233276in}}{\pgfqpoint{0.751904in}{1.241177in}}{\pgfqpoint{0.751904in}{1.249413in}}%
\pgfpathcurveto{\pgfqpoint{0.751904in}{1.257649in}}{\pgfqpoint{0.748632in}{1.265549in}}{\pgfqpoint{0.742808in}{1.271373in}}%
\pgfpathcurveto{\pgfqpoint{0.736984in}{1.277197in}}{\pgfqpoint{0.729084in}{1.280469in}}{\pgfqpoint{0.720848in}{1.280469in}}%
\pgfpathcurveto{\pgfqpoint{0.712612in}{1.280469in}}{\pgfqpoint{0.704711in}{1.277197in}}{\pgfqpoint{0.698888in}{1.271373in}}%
\pgfpathcurveto{\pgfqpoint{0.693064in}{1.265549in}}{\pgfqpoint{0.689791in}{1.257649in}}{\pgfqpoint{0.689791in}{1.249413in}}%
\pgfpathcurveto{\pgfqpoint{0.689791in}{1.241177in}}{\pgfqpoint{0.693064in}{1.233276in}}{\pgfqpoint{0.698888in}{1.227453in}}%
\pgfpathcurveto{\pgfqpoint{0.704711in}{1.221629in}}{\pgfqpoint{0.712612in}{1.218356in}}{\pgfqpoint{0.720848in}{1.218356in}}%
\pgfpathclose%
\pgfusepath{stroke,fill}%
\end{pgfscope}%
\begin{pgfscope}%
\pgfpathrectangle{\pgfqpoint{0.100000in}{0.212622in}}{\pgfqpoint{3.696000in}{3.696000in}}%
\pgfusepath{clip}%
\pgfsetbuttcap%
\pgfsetroundjoin%
\definecolor{currentfill}{rgb}{0.121569,0.466667,0.705882}%
\pgfsetfillcolor{currentfill}%
\pgfsetfillopacity{0.662256}%
\pgfsetlinewidth{1.003750pt}%
\definecolor{currentstroke}{rgb}{0.121569,0.466667,0.705882}%
\pgfsetstrokecolor{currentstroke}%
\pgfsetstrokeopacity{0.662256}%
\pgfsetdash{}{0pt}%
\pgfpathmoveto{\pgfqpoint{0.720844in}{1.218351in}}%
\pgfpathcurveto{\pgfqpoint{0.729081in}{1.218351in}}{\pgfqpoint{0.736981in}{1.221624in}}{\pgfqpoint{0.742805in}{1.227448in}}%
\pgfpathcurveto{\pgfqpoint{0.748629in}{1.233272in}}{\pgfqpoint{0.751901in}{1.241172in}}{\pgfqpoint{0.751901in}{1.249408in}}%
\pgfpathcurveto{\pgfqpoint{0.751901in}{1.257644in}}{\pgfqpoint{0.748629in}{1.265544in}}{\pgfqpoint{0.742805in}{1.271368in}}%
\pgfpathcurveto{\pgfqpoint{0.736981in}{1.277192in}}{\pgfqpoint{0.729081in}{1.280464in}}{\pgfqpoint{0.720844in}{1.280464in}}%
\pgfpathcurveto{\pgfqpoint{0.712608in}{1.280464in}}{\pgfqpoint{0.704708in}{1.277192in}}{\pgfqpoint{0.698884in}{1.271368in}}%
\pgfpathcurveto{\pgfqpoint{0.693060in}{1.265544in}}{\pgfqpoint{0.689788in}{1.257644in}}{\pgfqpoint{0.689788in}{1.249408in}}%
\pgfpathcurveto{\pgfqpoint{0.689788in}{1.241172in}}{\pgfqpoint{0.693060in}{1.233272in}}{\pgfqpoint{0.698884in}{1.227448in}}%
\pgfpathcurveto{\pgfqpoint{0.704708in}{1.221624in}}{\pgfqpoint{0.712608in}{1.218351in}}{\pgfqpoint{0.720844in}{1.218351in}}%
\pgfpathclose%
\pgfusepath{stroke,fill}%
\end{pgfscope}%
\begin{pgfscope}%
\pgfpathrectangle{\pgfqpoint{0.100000in}{0.212622in}}{\pgfqpoint{3.696000in}{3.696000in}}%
\pgfusepath{clip}%
\pgfsetbuttcap%
\pgfsetroundjoin%
\definecolor{currentfill}{rgb}{0.121569,0.466667,0.705882}%
\pgfsetfillcolor{currentfill}%
\pgfsetfillopacity{0.662258}%
\pgfsetlinewidth{1.003750pt}%
\definecolor{currentstroke}{rgb}{0.121569,0.466667,0.705882}%
\pgfsetstrokecolor{currentstroke}%
\pgfsetstrokeopacity{0.662258}%
\pgfsetdash{}{0pt}%
\pgfpathmoveto{\pgfqpoint{0.720838in}{1.218343in}}%
\pgfpathcurveto{\pgfqpoint{0.729075in}{1.218343in}}{\pgfqpoint{0.736975in}{1.221616in}}{\pgfqpoint{0.742799in}{1.227440in}}%
\pgfpathcurveto{\pgfqpoint{0.748623in}{1.233263in}}{\pgfqpoint{0.751895in}{1.241164in}}{\pgfqpoint{0.751895in}{1.249400in}}%
\pgfpathcurveto{\pgfqpoint{0.751895in}{1.257636in}}{\pgfqpoint{0.748623in}{1.265536in}}{\pgfqpoint{0.742799in}{1.271360in}}%
\pgfpathcurveto{\pgfqpoint{0.736975in}{1.277184in}}{\pgfqpoint{0.729075in}{1.280456in}}{\pgfqpoint{0.720838in}{1.280456in}}%
\pgfpathcurveto{\pgfqpoint{0.712602in}{1.280456in}}{\pgfqpoint{0.704702in}{1.277184in}}{\pgfqpoint{0.698878in}{1.271360in}}%
\pgfpathcurveto{\pgfqpoint{0.693054in}{1.265536in}}{\pgfqpoint{0.689782in}{1.257636in}}{\pgfqpoint{0.689782in}{1.249400in}}%
\pgfpathcurveto{\pgfqpoint{0.689782in}{1.241164in}}{\pgfqpoint{0.693054in}{1.233263in}}{\pgfqpoint{0.698878in}{1.227440in}}%
\pgfpathcurveto{\pgfqpoint{0.704702in}{1.221616in}}{\pgfqpoint{0.712602in}{1.218343in}}{\pgfqpoint{0.720838in}{1.218343in}}%
\pgfpathclose%
\pgfusepath{stroke,fill}%
\end{pgfscope}%
\begin{pgfscope}%
\pgfpathrectangle{\pgfqpoint{0.100000in}{0.212622in}}{\pgfqpoint{3.696000in}{3.696000in}}%
\pgfusepath{clip}%
\pgfsetbuttcap%
\pgfsetroundjoin%
\definecolor{currentfill}{rgb}{0.121569,0.466667,0.705882}%
\pgfsetfillcolor{currentfill}%
\pgfsetfillopacity{0.662262}%
\pgfsetlinewidth{1.003750pt}%
\definecolor{currentstroke}{rgb}{0.121569,0.466667,0.705882}%
\pgfsetstrokecolor{currentstroke}%
\pgfsetstrokeopacity{0.662262}%
\pgfsetdash{}{0pt}%
\pgfpathmoveto{\pgfqpoint{0.720826in}{1.218327in}}%
\pgfpathcurveto{\pgfqpoint{0.729063in}{1.218327in}}{\pgfqpoint{0.736963in}{1.221600in}}{\pgfqpoint{0.742787in}{1.227424in}}%
\pgfpathcurveto{\pgfqpoint{0.748610in}{1.233248in}}{\pgfqpoint{0.751883in}{1.241148in}}{\pgfqpoint{0.751883in}{1.249384in}}%
\pgfpathcurveto{\pgfqpoint{0.751883in}{1.257620in}}{\pgfqpoint{0.748610in}{1.265520in}}{\pgfqpoint{0.742787in}{1.271344in}}%
\pgfpathcurveto{\pgfqpoint{0.736963in}{1.277168in}}{\pgfqpoint{0.729063in}{1.280440in}}{\pgfqpoint{0.720826in}{1.280440in}}%
\pgfpathcurveto{\pgfqpoint{0.712590in}{1.280440in}}{\pgfqpoint{0.704690in}{1.277168in}}{\pgfqpoint{0.698866in}{1.271344in}}%
\pgfpathcurveto{\pgfqpoint{0.693042in}{1.265520in}}{\pgfqpoint{0.689770in}{1.257620in}}{\pgfqpoint{0.689770in}{1.249384in}}%
\pgfpathcurveto{\pgfqpoint{0.689770in}{1.241148in}}{\pgfqpoint{0.693042in}{1.233248in}}{\pgfqpoint{0.698866in}{1.227424in}}%
\pgfpathcurveto{\pgfqpoint{0.704690in}{1.221600in}}{\pgfqpoint{0.712590in}{1.218327in}}{\pgfqpoint{0.720826in}{1.218327in}}%
\pgfpathclose%
\pgfusepath{stroke,fill}%
\end{pgfscope}%
\begin{pgfscope}%
\pgfpathrectangle{\pgfqpoint{0.100000in}{0.212622in}}{\pgfqpoint{3.696000in}{3.696000in}}%
\pgfusepath{clip}%
\pgfsetbuttcap%
\pgfsetroundjoin%
\definecolor{currentfill}{rgb}{0.121569,0.466667,0.705882}%
\pgfsetfillcolor{currentfill}%
\pgfsetfillopacity{0.662269}%
\pgfsetlinewidth{1.003750pt}%
\definecolor{currentstroke}{rgb}{0.121569,0.466667,0.705882}%
\pgfsetstrokecolor{currentstroke}%
\pgfsetstrokeopacity{0.662269}%
\pgfsetdash{}{0pt}%
\pgfpathmoveto{\pgfqpoint{0.720804in}{1.218295in}}%
\pgfpathcurveto{\pgfqpoint{0.729041in}{1.218295in}}{\pgfqpoint{0.736941in}{1.221567in}}{\pgfqpoint{0.742765in}{1.227391in}}%
\pgfpathcurveto{\pgfqpoint{0.748588in}{1.233215in}}{\pgfqpoint{0.751861in}{1.241115in}}{\pgfqpoint{0.751861in}{1.249351in}}%
\pgfpathcurveto{\pgfqpoint{0.751861in}{1.257588in}}{\pgfqpoint{0.748588in}{1.265488in}}{\pgfqpoint{0.742765in}{1.271312in}}%
\pgfpathcurveto{\pgfqpoint{0.736941in}{1.277136in}}{\pgfqpoint{0.729041in}{1.280408in}}{\pgfqpoint{0.720804in}{1.280408in}}%
\pgfpathcurveto{\pgfqpoint{0.712568in}{1.280408in}}{\pgfqpoint{0.704668in}{1.277136in}}{\pgfqpoint{0.698844in}{1.271312in}}%
\pgfpathcurveto{\pgfqpoint{0.693020in}{1.265488in}}{\pgfqpoint{0.689748in}{1.257588in}}{\pgfqpoint{0.689748in}{1.249351in}}%
\pgfpathcurveto{\pgfqpoint{0.689748in}{1.241115in}}{\pgfqpoint{0.693020in}{1.233215in}}{\pgfqpoint{0.698844in}{1.227391in}}%
\pgfpathcurveto{\pgfqpoint{0.704668in}{1.221567in}}{\pgfqpoint{0.712568in}{1.218295in}}{\pgfqpoint{0.720804in}{1.218295in}}%
\pgfpathclose%
\pgfusepath{stroke,fill}%
\end{pgfscope}%
\begin{pgfscope}%
\pgfpathrectangle{\pgfqpoint{0.100000in}{0.212622in}}{\pgfqpoint{3.696000in}{3.696000in}}%
\pgfusepath{clip}%
\pgfsetbuttcap%
\pgfsetroundjoin%
\definecolor{currentfill}{rgb}{0.121569,0.466667,0.705882}%
\pgfsetfillcolor{currentfill}%
\pgfsetfillopacity{0.662283}%
\pgfsetlinewidth{1.003750pt}%
\definecolor{currentstroke}{rgb}{0.121569,0.466667,0.705882}%
\pgfsetstrokecolor{currentstroke}%
\pgfsetstrokeopacity{0.662283}%
\pgfsetdash{}{0pt}%
\pgfpathmoveto{\pgfqpoint{0.720771in}{1.218245in}}%
\pgfpathcurveto{\pgfqpoint{0.729007in}{1.218245in}}{\pgfqpoint{0.736907in}{1.221517in}}{\pgfqpoint{0.742731in}{1.227341in}}%
\pgfpathcurveto{\pgfqpoint{0.748555in}{1.233165in}}{\pgfqpoint{0.751827in}{1.241065in}}{\pgfqpoint{0.751827in}{1.249301in}}%
\pgfpathcurveto{\pgfqpoint{0.751827in}{1.257537in}}{\pgfqpoint{0.748555in}{1.265437in}}{\pgfqpoint{0.742731in}{1.271261in}}%
\pgfpathcurveto{\pgfqpoint{0.736907in}{1.277085in}}{\pgfqpoint{0.729007in}{1.280358in}}{\pgfqpoint{0.720771in}{1.280358in}}%
\pgfpathcurveto{\pgfqpoint{0.712534in}{1.280358in}}{\pgfqpoint{0.704634in}{1.277085in}}{\pgfqpoint{0.698810in}{1.271261in}}%
\pgfpathcurveto{\pgfqpoint{0.692987in}{1.265437in}}{\pgfqpoint{0.689714in}{1.257537in}}{\pgfqpoint{0.689714in}{1.249301in}}%
\pgfpathcurveto{\pgfqpoint{0.689714in}{1.241065in}}{\pgfqpoint{0.692987in}{1.233165in}}{\pgfqpoint{0.698810in}{1.227341in}}%
\pgfpathcurveto{\pgfqpoint{0.704634in}{1.221517in}}{\pgfqpoint{0.712534in}{1.218245in}}{\pgfqpoint{0.720771in}{1.218245in}}%
\pgfpathclose%
\pgfusepath{stroke,fill}%
\end{pgfscope}%
\begin{pgfscope}%
\pgfpathrectangle{\pgfqpoint{0.100000in}{0.212622in}}{\pgfqpoint{3.696000in}{3.696000in}}%
\pgfusepath{clip}%
\pgfsetbuttcap%
\pgfsetroundjoin%
\definecolor{currentfill}{rgb}{0.121569,0.466667,0.705882}%
\pgfsetfillcolor{currentfill}%
\pgfsetfillopacity{0.662307}%
\pgfsetlinewidth{1.003750pt}%
\definecolor{currentstroke}{rgb}{0.121569,0.466667,0.705882}%
\pgfsetstrokecolor{currentstroke}%
\pgfsetstrokeopacity{0.662307}%
\pgfsetdash{}{0pt}%
\pgfpathmoveto{\pgfqpoint{0.720708in}{1.218144in}}%
\pgfpathcurveto{\pgfqpoint{0.728944in}{1.218144in}}{\pgfqpoint{0.736844in}{1.221416in}}{\pgfqpoint{0.742668in}{1.227240in}}%
\pgfpathcurveto{\pgfqpoint{0.748492in}{1.233064in}}{\pgfqpoint{0.751764in}{1.240964in}}{\pgfqpoint{0.751764in}{1.249201in}}%
\pgfpathcurveto{\pgfqpoint{0.751764in}{1.257437in}}{\pgfqpoint{0.748492in}{1.265337in}}{\pgfqpoint{0.742668in}{1.271161in}}%
\pgfpathcurveto{\pgfqpoint{0.736844in}{1.276985in}}{\pgfqpoint{0.728944in}{1.280257in}}{\pgfqpoint{0.720708in}{1.280257in}}%
\pgfpathcurveto{\pgfqpoint{0.712472in}{1.280257in}}{\pgfqpoint{0.704571in}{1.276985in}}{\pgfqpoint{0.698748in}{1.271161in}}%
\pgfpathcurveto{\pgfqpoint{0.692924in}{1.265337in}}{\pgfqpoint{0.689651in}{1.257437in}}{\pgfqpoint{0.689651in}{1.249201in}}%
\pgfpathcurveto{\pgfqpoint{0.689651in}{1.240964in}}{\pgfqpoint{0.692924in}{1.233064in}}{\pgfqpoint{0.698748in}{1.227240in}}%
\pgfpathcurveto{\pgfqpoint{0.704571in}{1.221416in}}{\pgfqpoint{0.712472in}{1.218144in}}{\pgfqpoint{0.720708in}{1.218144in}}%
\pgfpathclose%
\pgfusepath{stroke,fill}%
\end{pgfscope}%
\begin{pgfscope}%
\pgfpathrectangle{\pgfqpoint{0.100000in}{0.212622in}}{\pgfqpoint{3.696000in}{3.696000in}}%
\pgfusepath{clip}%
\pgfsetbuttcap%
\pgfsetroundjoin%
\definecolor{currentfill}{rgb}{0.121569,0.466667,0.705882}%
\pgfsetfillcolor{currentfill}%
\pgfsetfillopacity{0.662349}%
\pgfsetlinewidth{1.003750pt}%
\definecolor{currentstroke}{rgb}{0.121569,0.466667,0.705882}%
\pgfsetstrokecolor{currentstroke}%
\pgfsetstrokeopacity{0.662349}%
\pgfsetdash{}{0pt}%
\pgfpathmoveto{\pgfqpoint{0.720574in}{1.217962in}}%
\pgfpathcurveto{\pgfqpoint{0.728811in}{1.217962in}}{\pgfqpoint{0.736711in}{1.221234in}}{\pgfqpoint{0.742535in}{1.227058in}}%
\pgfpathcurveto{\pgfqpoint{0.748359in}{1.232882in}}{\pgfqpoint{0.751631in}{1.240782in}}{\pgfqpoint{0.751631in}{1.249018in}}%
\pgfpathcurveto{\pgfqpoint{0.751631in}{1.257254in}}{\pgfqpoint{0.748359in}{1.265154in}}{\pgfqpoint{0.742535in}{1.270978in}}%
\pgfpathcurveto{\pgfqpoint{0.736711in}{1.276802in}}{\pgfqpoint{0.728811in}{1.280075in}}{\pgfqpoint{0.720574in}{1.280075in}}%
\pgfpathcurveto{\pgfqpoint{0.712338in}{1.280075in}}{\pgfqpoint{0.704438in}{1.276802in}}{\pgfqpoint{0.698614in}{1.270978in}}%
\pgfpathcurveto{\pgfqpoint{0.692790in}{1.265154in}}{\pgfqpoint{0.689518in}{1.257254in}}{\pgfqpoint{0.689518in}{1.249018in}}%
\pgfpathcurveto{\pgfqpoint{0.689518in}{1.240782in}}{\pgfqpoint{0.692790in}{1.232882in}}{\pgfqpoint{0.698614in}{1.227058in}}%
\pgfpathcurveto{\pgfqpoint{0.704438in}{1.221234in}}{\pgfqpoint{0.712338in}{1.217962in}}{\pgfqpoint{0.720574in}{1.217962in}}%
\pgfpathclose%
\pgfusepath{stroke,fill}%
\end{pgfscope}%
\begin{pgfscope}%
\pgfpathrectangle{\pgfqpoint{0.100000in}{0.212622in}}{\pgfqpoint{3.696000in}{3.696000in}}%
\pgfusepath{clip}%
\pgfsetbuttcap%
\pgfsetroundjoin%
\definecolor{currentfill}{rgb}{0.121569,0.466667,0.705882}%
\pgfsetfillcolor{currentfill}%
\pgfsetfillopacity{0.662427}%
\pgfsetlinewidth{1.003750pt}%
\definecolor{currentstroke}{rgb}{0.121569,0.466667,0.705882}%
\pgfsetstrokecolor{currentstroke}%
\pgfsetstrokeopacity{0.662427}%
\pgfsetdash{}{0pt}%
\pgfpathmoveto{\pgfqpoint{0.720371in}{1.217612in}}%
\pgfpathcurveto{\pgfqpoint{0.728608in}{1.217612in}}{\pgfqpoint{0.736508in}{1.220884in}}{\pgfqpoint{0.742332in}{1.226708in}}%
\pgfpathcurveto{\pgfqpoint{0.748156in}{1.232532in}}{\pgfqpoint{0.751428in}{1.240432in}}{\pgfqpoint{0.751428in}{1.248668in}}%
\pgfpathcurveto{\pgfqpoint{0.751428in}{1.256904in}}{\pgfqpoint{0.748156in}{1.264804in}}{\pgfqpoint{0.742332in}{1.270628in}}%
\pgfpathcurveto{\pgfqpoint{0.736508in}{1.276452in}}{\pgfqpoint{0.728608in}{1.279725in}}{\pgfqpoint{0.720371in}{1.279725in}}%
\pgfpathcurveto{\pgfqpoint{0.712135in}{1.279725in}}{\pgfqpoint{0.704235in}{1.276452in}}{\pgfqpoint{0.698411in}{1.270628in}}%
\pgfpathcurveto{\pgfqpoint{0.692587in}{1.264804in}}{\pgfqpoint{0.689315in}{1.256904in}}{\pgfqpoint{0.689315in}{1.248668in}}%
\pgfpathcurveto{\pgfqpoint{0.689315in}{1.240432in}}{\pgfqpoint{0.692587in}{1.232532in}}{\pgfqpoint{0.698411in}{1.226708in}}%
\pgfpathcurveto{\pgfqpoint{0.704235in}{1.220884in}}{\pgfqpoint{0.712135in}{1.217612in}}{\pgfqpoint{0.720371in}{1.217612in}}%
\pgfpathclose%
\pgfusepath{stroke,fill}%
\end{pgfscope}%
\begin{pgfscope}%
\pgfpathrectangle{\pgfqpoint{0.100000in}{0.212622in}}{\pgfqpoint{3.696000in}{3.696000in}}%
\pgfusepath{clip}%
\pgfsetbuttcap%
\pgfsetroundjoin%
\definecolor{currentfill}{rgb}{0.121569,0.466667,0.705882}%
\pgfsetfillcolor{currentfill}%
\pgfsetfillopacity{0.662466}%
\pgfsetlinewidth{1.003750pt}%
\definecolor{currentstroke}{rgb}{0.121569,0.466667,0.705882}%
\pgfsetstrokecolor{currentstroke}%
\pgfsetstrokeopacity{0.662466}%
\pgfsetdash{}{0pt}%
\pgfpathmoveto{\pgfqpoint{0.733294in}{1.214258in}}%
\pgfpathcurveto{\pgfqpoint{0.741530in}{1.214258in}}{\pgfqpoint{0.749430in}{1.217530in}}{\pgfqpoint{0.755254in}{1.223354in}}%
\pgfpathcurveto{\pgfqpoint{0.761078in}{1.229178in}}{\pgfqpoint{0.764350in}{1.237078in}}{\pgfqpoint{0.764350in}{1.245314in}}%
\pgfpathcurveto{\pgfqpoint{0.764350in}{1.253551in}}{\pgfqpoint{0.761078in}{1.261451in}}{\pgfqpoint{0.755254in}{1.267275in}}%
\pgfpathcurveto{\pgfqpoint{0.749430in}{1.273099in}}{\pgfqpoint{0.741530in}{1.276371in}}{\pgfqpoint{0.733294in}{1.276371in}}%
\pgfpathcurveto{\pgfqpoint{0.725058in}{1.276371in}}{\pgfqpoint{0.717158in}{1.273099in}}{\pgfqpoint{0.711334in}{1.267275in}}%
\pgfpathcurveto{\pgfqpoint{0.705510in}{1.261451in}}{\pgfqpoint{0.702237in}{1.253551in}}{\pgfqpoint{0.702237in}{1.245314in}}%
\pgfpathcurveto{\pgfqpoint{0.702237in}{1.237078in}}{\pgfqpoint{0.705510in}{1.229178in}}{\pgfqpoint{0.711334in}{1.223354in}}%
\pgfpathcurveto{\pgfqpoint{0.717158in}{1.217530in}}{\pgfqpoint{0.725058in}{1.214258in}}{\pgfqpoint{0.733294in}{1.214258in}}%
\pgfpathclose%
\pgfusepath{stroke,fill}%
\end{pgfscope}%
\begin{pgfscope}%
\pgfpathrectangle{\pgfqpoint{0.100000in}{0.212622in}}{\pgfqpoint{3.696000in}{3.696000in}}%
\pgfusepath{clip}%
\pgfsetbuttcap%
\pgfsetroundjoin%
\definecolor{currentfill}{rgb}{0.121569,0.466667,0.705882}%
\pgfsetfillcolor{currentfill}%
\pgfsetfillopacity{0.662571}%
\pgfsetlinewidth{1.003750pt}%
\definecolor{currentstroke}{rgb}{0.121569,0.466667,0.705882}%
\pgfsetstrokecolor{currentstroke}%
\pgfsetstrokeopacity{0.662571}%
\pgfsetdash{}{0pt}%
\pgfpathmoveto{\pgfqpoint{0.719935in}{1.217054in}}%
\pgfpathcurveto{\pgfqpoint{0.728171in}{1.217054in}}{\pgfqpoint{0.736071in}{1.220326in}}{\pgfqpoint{0.741895in}{1.226150in}}%
\pgfpathcurveto{\pgfqpoint{0.747719in}{1.231974in}}{\pgfqpoint{0.750991in}{1.239874in}}{\pgfqpoint{0.750991in}{1.248111in}}%
\pgfpathcurveto{\pgfqpoint{0.750991in}{1.256347in}}{\pgfqpoint{0.747719in}{1.264247in}}{\pgfqpoint{0.741895in}{1.270071in}}%
\pgfpathcurveto{\pgfqpoint{0.736071in}{1.275895in}}{\pgfqpoint{0.728171in}{1.279167in}}{\pgfqpoint{0.719935in}{1.279167in}}%
\pgfpathcurveto{\pgfqpoint{0.711698in}{1.279167in}}{\pgfqpoint{0.703798in}{1.275895in}}{\pgfqpoint{0.697974in}{1.270071in}}%
\pgfpathcurveto{\pgfqpoint{0.692151in}{1.264247in}}{\pgfqpoint{0.688878in}{1.256347in}}{\pgfqpoint{0.688878in}{1.248111in}}%
\pgfpathcurveto{\pgfqpoint{0.688878in}{1.239874in}}{\pgfqpoint{0.692151in}{1.231974in}}{\pgfqpoint{0.697974in}{1.226150in}}%
\pgfpathcurveto{\pgfqpoint{0.703798in}{1.220326in}}{\pgfqpoint{0.711698in}{1.217054in}}{\pgfqpoint{0.719935in}{1.217054in}}%
\pgfpathclose%
\pgfusepath{stroke,fill}%
\end{pgfscope}%
\begin{pgfscope}%
\pgfpathrectangle{\pgfqpoint{0.100000in}{0.212622in}}{\pgfqpoint{3.696000in}{3.696000in}}%
\pgfusepath{clip}%
\pgfsetbuttcap%
\pgfsetroundjoin%
\definecolor{currentfill}{rgb}{0.121569,0.466667,0.705882}%
\pgfsetfillcolor{currentfill}%
\pgfsetfillopacity{0.662832}%
\pgfsetlinewidth{1.003750pt}%
\definecolor{currentstroke}{rgb}{0.121569,0.466667,0.705882}%
\pgfsetstrokecolor{currentstroke}%
\pgfsetstrokeopacity{0.662832}%
\pgfsetdash{}{0pt}%
\pgfpathmoveto{\pgfqpoint{0.719143in}{1.216014in}}%
\pgfpathcurveto{\pgfqpoint{0.727380in}{1.216014in}}{\pgfqpoint{0.735280in}{1.219286in}}{\pgfqpoint{0.741104in}{1.225110in}}%
\pgfpathcurveto{\pgfqpoint{0.746928in}{1.230934in}}{\pgfqpoint{0.750200in}{1.238834in}}{\pgfqpoint{0.750200in}{1.247070in}}%
\pgfpathcurveto{\pgfqpoint{0.750200in}{1.255306in}}{\pgfqpoint{0.746928in}{1.263206in}}{\pgfqpoint{0.741104in}{1.269030in}}%
\pgfpathcurveto{\pgfqpoint{0.735280in}{1.274854in}}{\pgfqpoint{0.727380in}{1.278127in}}{\pgfqpoint{0.719143in}{1.278127in}}%
\pgfpathcurveto{\pgfqpoint{0.710907in}{1.278127in}}{\pgfqpoint{0.703007in}{1.274854in}}{\pgfqpoint{0.697183in}{1.269030in}}%
\pgfpathcurveto{\pgfqpoint{0.691359in}{1.263206in}}{\pgfqpoint{0.688087in}{1.255306in}}{\pgfqpoint{0.688087in}{1.247070in}}%
\pgfpathcurveto{\pgfqpoint{0.688087in}{1.238834in}}{\pgfqpoint{0.691359in}{1.230934in}}{\pgfqpoint{0.697183in}{1.225110in}}%
\pgfpathcurveto{\pgfqpoint{0.703007in}{1.219286in}}{\pgfqpoint{0.710907in}{1.216014in}}{\pgfqpoint{0.719143in}{1.216014in}}%
\pgfpathclose%
\pgfusepath{stroke,fill}%
\end{pgfscope}%
\begin{pgfscope}%
\pgfpathrectangle{\pgfqpoint{0.100000in}{0.212622in}}{\pgfqpoint{3.696000in}{3.696000in}}%
\pgfusepath{clip}%
\pgfsetbuttcap%
\pgfsetroundjoin%
\definecolor{currentfill}{rgb}{0.121569,0.466667,0.705882}%
\pgfsetfillcolor{currentfill}%
\pgfsetfillopacity{0.663107}%
\pgfsetlinewidth{1.003750pt}%
\definecolor{currentstroke}{rgb}{0.121569,0.466667,0.705882}%
\pgfsetstrokecolor{currentstroke}%
\pgfsetstrokeopacity{0.663107}%
\pgfsetdash{}{0pt}%
\pgfpathmoveto{\pgfqpoint{0.730443in}{1.213473in}}%
\pgfpathcurveto{\pgfqpoint{0.738679in}{1.213473in}}{\pgfqpoint{0.746579in}{1.216745in}}{\pgfqpoint{0.752403in}{1.222569in}}%
\pgfpathcurveto{\pgfqpoint{0.758227in}{1.228393in}}{\pgfqpoint{0.761499in}{1.236293in}}{\pgfqpoint{0.761499in}{1.244530in}}%
\pgfpathcurveto{\pgfqpoint{0.761499in}{1.252766in}}{\pgfqpoint{0.758227in}{1.260666in}}{\pgfqpoint{0.752403in}{1.266490in}}%
\pgfpathcurveto{\pgfqpoint{0.746579in}{1.272314in}}{\pgfqpoint{0.738679in}{1.275586in}}{\pgfqpoint{0.730443in}{1.275586in}}%
\pgfpathcurveto{\pgfqpoint{0.722206in}{1.275586in}}{\pgfqpoint{0.714306in}{1.272314in}}{\pgfqpoint{0.708482in}{1.266490in}}%
\pgfpathcurveto{\pgfqpoint{0.702659in}{1.260666in}}{\pgfqpoint{0.699386in}{1.252766in}}{\pgfqpoint{0.699386in}{1.244530in}}%
\pgfpathcurveto{\pgfqpoint{0.699386in}{1.236293in}}{\pgfqpoint{0.702659in}{1.228393in}}{\pgfqpoint{0.708482in}{1.222569in}}%
\pgfpathcurveto{\pgfqpoint{0.714306in}{1.216745in}}{\pgfqpoint{0.722206in}{1.213473in}}{\pgfqpoint{0.730443in}{1.213473in}}%
\pgfpathclose%
\pgfusepath{stroke,fill}%
\end{pgfscope}%
\begin{pgfscope}%
\pgfpathrectangle{\pgfqpoint{0.100000in}{0.212622in}}{\pgfqpoint{3.696000in}{3.696000in}}%
\pgfusepath{clip}%
\pgfsetbuttcap%
\pgfsetroundjoin%
\definecolor{currentfill}{rgb}{0.121569,0.466667,0.705882}%
\pgfsetfillcolor{currentfill}%
\pgfsetfillopacity{0.663327}%
\pgfsetlinewidth{1.003750pt}%
\definecolor{currentstroke}{rgb}{0.121569,0.466667,0.705882}%
\pgfsetstrokecolor{currentstroke}%
\pgfsetstrokeopacity{0.663327}%
\pgfsetdash{}{0pt}%
\pgfpathmoveto{\pgfqpoint{0.717884in}{1.214075in}}%
\pgfpathcurveto{\pgfqpoint{0.726120in}{1.214075in}}{\pgfqpoint{0.734020in}{1.217348in}}{\pgfqpoint{0.739844in}{1.223172in}}%
\pgfpathcurveto{\pgfqpoint{0.745668in}{1.228996in}}{\pgfqpoint{0.748941in}{1.236896in}}{\pgfqpoint{0.748941in}{1.245132in}}%
\pgfpathcurveto{\pgfqpoint{0.748941in}{1.253368in}}{\pgfqpoint{0.745668in}{1.261268in}}{\pgfqpoint{0.739844in}{1.267092in}}%
\pgfpathcurveto{\pgfqpoint{0.734020in}{1.272916in}}{\pgfqpoint{0.726120in}{1.276188in}}{\pgfqpoint{0.717884in}{1.276188in}}%
\pgfpathcurveto{\pgfqpoint{0.709648in}{1.276188in}}{\pgfqpoint{0.701748in}{1.272916in}}{\pgfqpoint{0.695924in}{1.267092in}}%
\pgfpathcurveto{\pgfqpoint{0.690100in}{1.261268in}}{\pgfqpoint{0.686828in}{1.253368in}}{\pgfqpoint{0.686828in}{1.245132in}}%
\pgfpathcurveto{\pgfqpoint{0.686828in}{1.236896in}}{\pgfqpoint{0.690100in}{1.228996in}}{\pgfqpoint{0.695924in}{1.223172in}}%
\pgfpathcurveto{\pgfqpoint{0.701748in}{1.217348in}}{\pgfqpoint{0.709648in}{1.214075in}}{\pgfqpoint{0.717884in}{1.214075in}}%
\pgfpathclose%
\pgfusepath{stroke,fill}%
\end{pgfscope}%
\begin{pgfscope}%
\pgfpathrectangle{\pgfqpoint{0.100000in}{0.212622in}}{\pgfqpoint{3.696000in}{3.696000in}}%
\pgfusepath{clip}%
\pgfsetbuttcap%
\pgfsetroundjoin%
\definecolor{currentfill}{rgb}{0.121569,0.466667,0.705882}%
\pgfsetfillcolor{currentfill}%
\pgfsetfillopacity{0.663330}%
\pgfsetlinewidth{1.003750pt}%
\definecolor{currentstroke}{rgb}{0.121569,0.466667,0.705882}%
\pgfsetstrokecolor{currentstroke}%
\pgfsetstrokeopacity{0.663330}%
\pgfsetdash{}{0pt}%
\pgfpathmoveto{\pgfqpoint{0.717875in}{1.214060in}}%
\pgfpathcurveto{\pgfqpoint{0.726111in}{1.214060in}}{\pgfqpoint{0.734011in}{1.217332in}}{\pgfqpoint{0.739835in}{1.223156in}}%
\pgfpathcurveto{\pgfqpoint{0.745659in}{1.228980in}}{\pgfqpoint{0.748931in}{1.236880in}}{\pgfqpoint{0.748931in}{1.245116in}}%
\pgfpathcurveto{\pgfqpoint{0.748931in}{1.253352in}}{\pgfqpoint{0.745659in}{1.261252in}}{\pgfqpoint{0.739835in}{1.267076in}}%
\pgfpathcurveto{\pgfqpoint{0.734011in}{1.272900in}}{\pgfqpoint{0.726111in}{1.276173in}}{\pgfqpoint{0.717875in}{1.276173in}}%
\pgfpathcurveto{\pgfqpoint{0.709639in}{1.276173in}}{\pgfqpoint{0.701739in}{1.272900in}}{\pgfqpoint{0.695915in}{1.267076in}}%
\pgfpathcurveto{\pgfqpoint{0.690091in}{1.261252in}}{\pgfqpoint{0.686818in}{1.253352in}}{\pgfqpoint{0.686818in}{1.245116in}}%
\pgfpathcurveto{\pgfqpoint{0.686818in}{1.236880in}}{\pgfqpoint{0.690091in}{1.228980in}}{\pgfqpoint{0.695915in}{1.223156in}}%
\pgfpathcurveto{\pgfqpoint{0.701739in}{1.217332in}}{\pgfqpoint{0.709639in}{1.214060in}}{\pgfqpoint{0.717875in}{1.214060in}}%
\pgfpathclose%
\pgfusepath{stroke,fill}%
\end{pgfscope}%
\begin{pgfscope}%
\pgfpathrectangle{\pgfqpoint{0.100000in}{0.212622in}}{\pgfqpoint{3.696000in}{3.696000in}}%
\pgfusepath{clip}%
\pgfsetbuttcap%
\pgfsetroundjoin%
\definecolor{currentfill}{rgb}{0.121569,0.466667,0.705882}%
\pgfsetfillcolor{currentfill}%
\pgfsetfillopacity{0.663337}%
\pgfsetlinewidth{1.003750pt}%
\definecolor{currentstroke}{rgb}{0.121569,0.466667,0.705882}%
\pgfsetstrokecolor{currentstroke}%
\pgfsetstrokeopacity{0.663337}%
\pgfsetdash{}{0pt}%
\pgfpathmoveto{\pgfqpoint{0.717858in}{1.214035in}}%
\pgfpathcurveto{\pgfqpoint{0.726094in}{1.214035in}}{\pgfqpoint{0.733994in}{1.217307in}}{\pgfqpoint{0.739818in}{1.223131in}}%
\pgfpathcurveto{\pgfqpoint{0.745642in}{1.228955in}}{\pgfqpoint{0.748914in}{1.236855in}}{\pgfqpoint{0.748914in}{1.245091in}}%
\pgfpathcurveto{\pgfqpoint{0.748914in}{1.253328in}}{\pgfqpoint{0.745642in}{1.261228in}}{\pgfqpoint{0.739818in}{1.267052in}}%
\pgfpathcurveto{\pgfqpoint{0.733994in}{1.272875in}}{\pgfqpoint{0.726094in}{1.276148in}}{\pgfqpoint{0.717858in}{1.276148in}}%
\pgfpathcurveto{\pgfqpoint{0.709621in}{1.276148in}}{\pgfqpoint{0.701721in}{1.272875in}}{\pgfqpoint{0.695897in}{1.267052in}}%
\pgfpathcurveto{\pgfqpoint{0.690073in}{1.261228in}}{\pgfqpoint{0.686801in}{1.253328in}}{\pgfqpoint{0.686801in}{1.245091in}}%
\pgfpathcurveto{\pgfqpoint{0.686801in}{1.236855in}}{\pgfqpoint{0.690073in}{1.228955in}}{\pgfqpoint{0.695897in}{1.223131in}}%
\pgfpathcurveto{\pgfqpoint{0.701721in}{1.217307in}}{\pgfqpoint{0.709621in}{1.214035in}}{\pgfqpoint{0.717858in}{1.214035in}}%
\pgfpathclose%
\pgfusepath{stroke,fill}%
\end{pgfscope}%
\begin{pgfscope}%
\pgfpathrectangle{\pgfqpoint{0.100000in}{0.212622in}}{\pgfqpoint{3.696000in}{3.696000in}}%
\pgfusepath{clip}%
\pgfsetbuttcap%
\pgfsetroundjoin%
\definecolor{currentfill}{rgb}{0.121569,0.466667,0.705882}%
\pgfsetfillcolor{currentfill}%
\pgfsetfillopacity{0.663350}%
\pgfsetlinewidth{1.003750pt}%
\definecolor{currentstroke}{rgb}{0.121569,0.466667,0.705882}%
\pgfsetstrokecolor{currentstroke}%
\pgfsetstrokeopacity{0.663350}%
\pgfsetdash{}{0pt}%
\pgfpathmoveto{\pgfqpoint{0.717845in}{1.213984in}}%
\pgfpathcurveto{\pgfqpoint{0.726081in}{1.213984in}}{\pgfqpoint{0.733981in}{1.217256in}}{\pgfqpoint{0.739805in}{1.223080in}}%
\pgfpathcurveto{\pgfqpoint{0.745629in}{1.228904in}}{\pgfqpoint{0.748901in}{1.236804in}}{\pgfqpoint{0.748901in}{1.245040in}}%
\pgfpathcurveto{\pgfqpoint{0.748901in}{1.253277in}}{\pgfqpoint{0.745629in}{1.261177in}}{\pgfqpoint{0.739805in}{1.267001in}}%
\pgfpathcurveto{\pgfqpoint{0.733981in}{1.272824in}}{\pgfqpoint{0.726081in}{1.276097in}}{\pgfqpoint{0.717845in}{1.276097in}}%
\pgfpathcurveto{\pgfqpoint{0.709609in}{1.276097in}}{\pgfqpoint{0.701709in}{1.272824in}}{\pgfqpoint{0.695885in}{1.267001in}}%
\pgfpathcurveto{\pgfqpoint{0.690061in}{1.261177in}}{\pgfqpoint{0.686788in}{1.253277in}}{\pgfqpoint{0.686788in}{1.245040in}}%
\pgfpathcurveto{\pgfqpoint{0.686788in}{1.236804in}}{\pgfqpoint{0.690061in}{1.228904in}}{\pgfqpoint{0.695885in}{1.223080in}}%
\pgfpathcurveto{\pgfqpoint{0.701709in}{1.217256in}}{\pgfqpoint{0.709609in}{1.213984in}}{\pgfqpoint{0.717845in}{1.213984in}}%
\pgfpathclose%
\pgfusepath{stroke,fill}%
\end{pgfscope}%
\begin{pgfscope}%
\pgfpathrectangle{\pgfqpoint{0.100000in}{0.212622in}}{\pgfqpoint{3.696000in}{3.696000in}}%
\pgfusepath{clip}%
\pgfsetbuttcap%
\pgfsetroundjoin%
\definecolor{currentfill}{rgb}{0.121569,0.466667,0.705882}%
\pgfsetfillcolor{currentfill}%
\pgfsetfillopacity{0.663376}%
\pgfsetlinewidth{1.003750pt}%
\definecolor{currentstroke}{rgb}{0.121569,0.466667,0.705882}%
\pgfsetstrokecolor{currentstroke}%
\pgfsetstrokeopacity{0.663376}%
\pgfsetdash{}{0pt}%
\pgfpathmoveto{\pgfqpoint{0.717860in}{1.213914in}}%
\pgfpathcurveto{\pgfqpoint{0.726096in}{1.213914in}}{\pgfqpoint{0.733997in}{1.217186in}}{\pgfqpoint{0.739820in}{1.223010in}}%
\pgfpathcurveto{\pgfqpoint{0.745644in}{1.228834in}}{\pgfqpoint{0.748917in}{1.236734in}}{\pgfqpoint{0.748917in}{1.244971in}}%
\pgfpathcurveto{\pgfqpoint{0.748917in}{1.253207in}}{\pgfqpoint{0.745644in}{1.261107in}}{\pgfqpoint{0.739820in}{1.266931in}}%
\pgfpathcurveto{\pgfqpoint{0.733997in}{1.272755in}}{\pgfqpoint{0.726096in}{1.276027in}}{\pgfqpoint{0.717860in}{1.276027in}}%
\pgfpathcurveto{\pgfqpoint{0.709624in}{1.276027in}}{\pgfqpoint{0.701724in}{1.272755in}}{\pgfqpoint{0.695900in}{1.266931in}}%
\pgfpathcurveto{\pgfqpoint{0.690076in}{1.261107in}}{\pgfqpoint{0.686804in}{1.253207in}}{\pgfqpoint{0.686804in}{1.244971in}}%
\pgfpathcurveto{\pgfqpoint{0.686804in}{1.236734in}}{\pgfqpoint{0.690076in}{1.228834in}}{\pgfqpoint{0.695900in}{1.223010in}}%
\pgfpathcurveto{\pgfqpoint{0.701724in}{1.217186in}}{\pgfqpoint{0.709624in}{1.213914in}}{\pgfqpoint{0.717860in}{1.213914in}}%
\pgfpathclose%
\pgfusepath{stroke,fill}%
\end{pgfscope}%
\begin{pgfscope}%
\pgfpathrectangle{\pgfqpoint{0.100000in}{0.212622in}}{\pgfqpoint{3.696000in}{3.696000in}}%
\pgfusepath{clip}%
\pgfsetbuttcap%
\pgfsetroundjoin%
\definecolor{currentfill}{rgb}{0.121569,0.466667,0.705882}%
\pgfsetfillcolor{currentfill}%
\pgfsetfillopacity{0.663414}%
\pgfsetlinewidth{1.003750pt}%
\definecolor{currentstroke}{rgb}{0.121569,0.466667,0.705882}%
\pgfsetstrokecolor{currentstroke}%
\pgfsetstrokeopacity{0.663414}%
\pgfsetdash{}{0pt}%
\pgfpathmoveto{\pgfqpoint{0.717963in}{1.213793in}}%
\pgfpathcurveto{\pgfqpoint{0.726199in}{1.213793in}}{\pgfqpoint{0.734099in}{1.217066in}}{\pgfqpoint{0.739923in}{1.222890in}}%
\pgfpathcurveto{\pgfqpoint{0.745747in}{1.228713in}}{\pgfqpoint{0.749020in}{1.236614in}}{\pgfqpoint{0.749020in}{1.244850in}}%
\pgfpathcurveto{\pgfqpoint{0.749020in}{1.253086in}}{\pgfqpoint{0.745747in}{1.260986in}}{\pgfqpoint{0.739923in}{1.266810in}}%
\pgfpathcurveto{\pgfqpoint{0.734099in}{1.272634in}}{\pgfqpoint{0.726199in}{1.275906in}}{\pgfqpoint{0.717963in}{1.275906in}}%
\pgfpathcurveto{\pgfqpoint{0.709727in}{1.275906in}}{\pgfqpoint{0.701827in}{1.272634in}}{\pgfqpoint{0.696003in}{1.266810in}}%
\pgfpathcurveto{\pgfqpoint{0.690179in}{1.260986in}}{\pgfqpoint{0.686907in}{1.253086in}}{\pgfqpoint{0.686907in}{1.244850in}}%
\pgfpathcurveto{\pgfqpoint{0.686907in}{1.236614in}}{\pgfqpoint{0.690179in}{1.228713in}}{\pgfqpoint{0.696003in}{1.222890in}}%
\pgfpathcurveto{\pgfqpoint{0.701827in}{1.217066in}}{\pgfqpoint{0.709727in}{1.213793in}}{\pgfqpoint{0.717963in}{1.213793in}}%
\pgfpathclose%
\pgfusepath{stroke,fill}%
\end{pgfscope}%
\begin{pgfscope}%
\pgfpathrectangle{\pgfqpoint{0.100000in}{0.212622in}}{\pgfqpoint{3.696000in}{3.696000in}}%
\pgfusepath{clip}%
\pgfsetbuttcap%
\pgfsetroundjoin%
\definecolor{currentfill}{rgb}{0.121569,0.466667,0.705882}%
\pgfsetfillcolor{currentfill}%
\pgfsetfillopacity{0.663418}%
\pgfsetlinewidth{1.003750pt}%
\definecolor{currentstroke}{rgb}{0.121569,0.466667,0.705882}%
\pgfsetstrokecolor{currentstroke}%
\pgfsetstrokeopacity{0.663418}%
\pgfsetdash{}{0pt}%
\pgfpathmoveto{\pgfqpoint{0.720189in}{1.213014in}}%
\pgfpathcurveto{\pgfqpoint{0.728425in}{1.213014in}}{\pgfqpoint{0.736326in}{1.216287in}}{\pgfqpoint{0.742149in}{1.222110in}}%
\pgfpathcurveto{\pgfqpoint{0.747973in}{1.227934in}}{\pgfqpoint{0.751246in}{1.235834in}}{\pgfqpoint{0.751246in}{1.244071in}}%
\pgfpathcurveto{\pgfqpoint{0.751246in}{1.252307in}}{\pgfqpoint{0.747973in}{1.260207in}}{\pgfqpoint{0.742149in}{1.266031in}}%
\pgfpathcurveto{\pgfqpoint{0.736326in}{1.271855in}}{\pgfqpoint{0.728425in}{1.275127in}}{\pgfqpoint{0.720189in}{1.275127in}}%
\pgfpathcurveto{\pgfqpoint{0.711953in}{1.275127in}}{\pgfqpoint{0.704053in}{1.271855in}}{\pgfqpoint{0.698229in}{1.266031in}}%
\pgfpathcurveto{\pgfqpoint{0.692405in}{1.260207in}}{\pgfqpoint{0.689133in}{1.252307in}}{\pgfqpoint{0.689133in}{1.244071in}}%
\pgfpathcurveto{\pgfqpoint{0.689133in}{1.235834in}}{\pgfqpoint{0.692405in}{1.227934in}}{\pgfqpoint{0.698229in}{1.222110in}}%
\pgfpathcurveto{\pgfqpoint{0.704053in}{1.216287in}}{\pgfqpoint{0.711953in}{1.213014in}}{\pgfqpoint{0.720189in}{1.213014in}}%
\pgfpathclose%
\pgfusepath{stroke,fill}%
\end{pgfscope}%
\begin{pgfscope}%
\pgfpathrectangle{\pgfqpoint{0.100000in}{0.212622in}}{\pgfqpoint{3.696000in}{3.696000in}}%
\pgfusepath{clip}%
\pgfsetbuttcap%
\pgfsetroundjoin%
\definecolor{currentfill}{rgb}{0.121569,0.466667,0.705882}%
\pgfsetfillcolor{currentfill}%
\pgfsetfillopacity{0.663446}%
\pgfsetlinewidth{1.003750pt}%
\definecolor{currentstroke}{rgb}{0.121569,0.466667,0.705882}%
\pgfsetstrokecolor{currentstroke}%
\pgfsetstrokeopacity{0.663446}%
\pgfsetdash{}{0pt}%
\pgfpathmoveto{\pgfqpoint{0.728826in}{1.213083in}}%
\pgfpathcurveto{\pgfqpoint{0.737062in}{1.213083in}}{\pgfqpoint{0.744962in}{1.216355in}}{\pgfqpoint{0.750786in}{1.222179in}}%
\pgfpathcurveto{\pgfqpoint{0.756610in}{1.228003in}}{\pgfqpoint{0.759882in}{1.235903in}}{\pgfqpoint{0.759882in}{1.244140in}}%
\pgfpathcurveto{\pgfqpoint{0.759882in}{1.252376in}}{\pgfqpoint{0.756610in}{1.260276in}}{\pgfqpoint{0.750786in}{1.266100in}}%
\pgfpathcurveto{\pgfqpoint{0.744962in}{1.271924in}}{\pgfqpoint{0.737062in}{1.275196in}}{\pgfqpoint{0.728826in}{1.275196in}}%
\pgfpathcurveto{\pgfqpoint{0.720589in}{1.275196in}}{\pgfqpoint{0.712689in}{1.271924in}}{\pgfqpoint{0.706865in}{1.266100in}}%
\pgfpathcurveto{\pgfqpoint{0.701042in}{1.260276in}}{\pgfqpoint{0.697769in}{1.252376in}}{\pgfqpoint{0.697769in}{1.244140in}}%
\pgfpathcurveto{\pgfqpoint{0.697769in}{1.235903in}}{\pgfqpoint{0.701042in}{1.228003in}}{\pgfqpoint{0.706865in}{1.222179in}}%
\pgfpathcurveto{\pgfqpoint{0.712689in}{1.216355in}}{\pgfqpoint{0.720589in}{1.213083in}}{\pgfqpoint{0.728826in}{1.213083in}}%
\pgfpathclose%
\pgfusepath{stroke,fill}%
\end{pgfscope}%
\begin{pgfscope}%
\pgfpathrectangle{\pgfqpoint{0.100000in}{0.212622in}}{\pgfqpoint{3.696000in}{3.696000in}}%
\pgfusepath{clip}%
\pgfsetbuttcap%
\pgfsetroundjoin%
\definecolor{currentfill}{rgb}{0.121569,0.466667,0.705882}%
\pgfsetfillcolor{currentfill}%
\pgfsetfillopacity{0.663472}%
\pgfsetlinewidth{1.003750pt}%
\definecolor{currentstroke}{rgb}{0.121569,0.466667,0.705882}%
\pgfsetstrokecolor{currentstroke}%
\pgfsetstrokeopacity{0.663472}%
\pgfsetdash{}{0pt}%
\pgfpathmoveto{\pgfqpoint{0.718219in}{1.213611in}}%
\pgfpathcurveto{\pgfqpoint{0.726455in}{1.213611in}}{\pgfqpoint{0.734355in}{1.216884in}}{\pgfqpoint{0.740179in}{1.222707in}}%
\pgfpathcurveto{\pgfqpoint{0.746003in}{1.228531in}}{\pgfqpoint{0.749275in}{1.236431in}}{\pgfqpoint{0.749275in}{1.244668in}}%
\pgfpathcurveto{\pgfqpoint{0.749275in}{1.252904in}}{\pgfqpoint{0.746003in}{1.260804in}}{\pgfqpoint{0.740179in}{1.266628in}}%
\pgfpathcurveto{\pgfqpoint{0.734355in}{1.272452in}}{\pgfqpoint{0.726455in}{1.275724in}}{\pgfqpoint{0.718219in}{1.275724in}}%
\pgfpathcurveto{\pgfqpoint{0.709983in}{1.275724in}}{\pgfqpoint{0.702082in}{1.272452in}}{\pgfqpoint{0.696259in}{1.266628in}}%
\pgfpathcurveto{\pgfqpoint{0.690435in}{1.260804in}}{\pgfqpoint{0.687162in}{1.252904in}}{\pgfqpoint{0.687162in}{1.244668in}}%
\pgfpathcurveto{\pgfqpoint{0.687162in}{1.236431in}}{\pgfqpoint{0.690435in}{1.228531in}}{\pgfqpoint{0.696259in}{1.222707in}}%
\pgfpathcurveto{\pgfqpoint{0.702082in}{1.216884in}}{\pgfqpoint{0.709983in}{1.213611in}}{\pgfqpoint{0.718219in}{1.213611in}}%
\pgfpathclose%
\pgfusepath{stroke,fill}%
\end{pgfscope}%
\begin{pgfscope}%
\pgfpathrectangle{\pgfqpoint{0.100000in}{0.212622in}}{\pgfqpoint{3.696000in}{3.696000in}}%
\pgfusepath{clip}%
\pgfsetbuttcap%
\pgfsetroundjoin%
\definecolor{currentfill}{rgb}{0.121569,0.466667,0.705882}%
\pgfsetfillcolor{currentfill}%
\pgfsetfillopacity{0.663537}%
\pgfsetlinewidth{1.003750pt}%
\definecolor{currentstroke}{rgb}{0.121569,0.466667,0.705882}%
\pgfsetstrokecolor{currentstroke}%
\pgfsetstrokeopacity{0.663537}%
\pgfsetdash{}{0pt}%
\pgfpathmoveto{\pgfqpoint{0.718872in}{1.213618in}}%
\pgfpathcurveto{\pgfqpoint{0.727109in}{1.213618in}}{\pgfqpoint{0.735009in}{1.216890in}}{\pgfqpoint{0.740833in}{1.222714in}}%
\pgfpathcurveto{\pgfqpoint{0.746657in}{1.228538in}}{\pgfqpoint{0.749929in}{1.236438in}}{\pgfqpoint{0.749929in}{1.244675in}}%
\pgfpathcurveto{\pgfqpoint{0.749929in}{1.252911in}}{\pgfqpoint{0.746657in}{1.260811in}}{\pgfqpoint{0.740833in}{1.266635in}}%
\pgfpathcurveto{\pgfqpoint{0.735009in}{1.272459in}}{\pgfqpoint{0.727109in}{1.275731in}}{\pgfqpoint{0.718872in}{1.275731in}}%
\pgfpathcurveto{\pgfqpoint{0.710636in}{1.275731in}}{\pgfqpoint{0.702736in}{1.272459in}}{\pgfqpoint{0.696912in}{1.266635in}}%
\pgfpathcurveto{\pgfqpoint{0.691088in}{1.260811in}}{\pgfqpoint{0.687816in}{1.252911in}}{\pgfqpoint{0.687816in}{1.244675in}}%
\pgfpathcurveto{\pgfqpoint{0.687816in}{1.236438in}}{\pgfqpoint{0.691088in}{1.228538in}}{\pgfqpoint{0.696912in}{1.222714in}}%
\pgfpathcurveto{\pgfqpoint{0.702736in}{1.216890in}}{\pgfqpoint{0.710636in}{1.213618in}}{\pgfqpoint{0.718872in}{1.213618in}}%
\pgfpathclose%
\pgfusepath{stroke,fill}%
\end{pgfscope}%
\begin{pgfscope}%
\pgfpathrectangle{\pgfqpoint{0.100000in}{0.212622in}}{\pgfqpoint{3.696000in}{3.696000in}}%
\pgfusepath{clip}%
\pgfsetbuttcap%
\pgfsetroundjoin%
\definecolor{currentfill}{rgb}{0.121569,0.466667,0.705882}%
\pgfsetfillcolor{currentfill}%
\pgfsetfillopacity{0.663646}%
\pgfsetlinewidth{1.003750pt}%
\definecolor{currentstroke}{rgb}{0.121569,0.466667,0.705882}%
\pgfsetstrokecolor{currentstroke}%
\pgfsetstrokeopacity{0.663646}%
\pgfsetdash{}{0pt}%
\pgfpathmoveto{\pgfqpoint{0.727967in}{1.212867in}}%
\pgfpathcurveto{\pgfqpoint{0.736203in}{1.212867in}}{\pgfqpoint{0.744103in}{1.216139in}}{\pgfqpoint{0.749927in}{1.221963in}}%
\pgfpathcurveto{\pgfqpoint{0.755751in}{1.227787in}}{\pgfqpoint{0.759023in}{1.235687in}}{\pgfqpoint{0.759023in}{1.243923in}}%
\pgfpathcurveto{\pgfqpoint{0.759023in}{1.252160in}}{\pgfqpoint{0.755751in}{1.260060in}}{\pgfqpoint{0.749927in}{1.265884in}}%
\pgfpathcurveto{\pgfqpoint{0.744103in}{1.271708in}}{\pgfqpoint{0.736203in}{1.274980in}}{\pgfqpoint{0.727967in}{1.274980in}}%
\pgfpathcurveto{\pgfqpoint{0.719730in}{1.274980in}}{\pgfqpoint{0.711830in}{1.271708in}}{\pgfqpoint{0.706006in}{1.265884in}}%
\pgfpathcurveto{\pgfqpoint{0.700182in}{1.260060in}}{\pgfqpoint{0.696910in}{1.252160in}}{\pgfqpoint{0.696910in}{1.243923in}}%
\pgfpathcurveto{\pgfqpoint{0.696910in}{1.235687in}}{\pgfqpoint{0.700182in}{1.227787in}}{\pgfqpoint{0.706006in}{1.221963in}}%
\pgfpathcurveto{\pgfqpoint{0.711830in}{1.216139in}}{\pgfqpoint{0.719730in}{1.212867in}}{\pgfqpoint{0.727967in}{1.212867in}}%
\pgfpathclose%
\pgfusepath{stroke,fill}%
\end{pgfscope}%
\begin{pgfscope}%
\pgfpathrectangle{\pgfqpoint{0.100000in}{0.212622in}}{\pgfqpoint{3.696000in}{3.696000in}}%
\pgfusepath{clip}%
\pgfsetbuttcap%
\pgfsetroundjoin%
\definecolor{currentfill}{rgb}{0.121569,0.466667,0.705882}%
\pgfsetfillcolor{currentfill}%
\pgfsetfillopacity{0.663687}%
\pgfsetlinewidth{1.003750pt}%
\definecolor{currentstroke}{rgb}{0.121569,0.466667,0.705882}%
\pgfsetstrokecolor{currentstroke}%
\pgfsetstrokeopacity{0.663687}%
\pgfsetdash{}{0pt}%
\pgfpathmoveto{\pgfqpoint{0.724743in}{1.212308in}}%
\pgfpathcurveto{\pgfqpoint{0.732979in}{1.212308in}}{\pgfqpoint{0.740879in}{1.215580in}}{\pgfqpoint{0.746703in}{1.221404in}}%
\pgfpathcurveto{\pgfqpoint{0.752527in}{1.227228in}}{\pgfqpoint{0.755799in}{1.235128in}}{\pgfqpoint{0.755799in}{1.243364in}}%
\pgfpathcurveto{\pgfqpoint{0.755799in}{1.251601in}}{\pgfqpoint{0.752527in}{1.259501in}}{\pgfqpoint{0.746703in}{1.265325in}}%
\pgfpathcurveto{\pgfqpoint{0.740879in}{1.271149in}}{\pgfqpoint{0.732979in}{1.274421in}}{\pgfqpoint{0.724743in}{1.274421in}}%
\pgfpathcurveto{\pgfqpoint{0.716507in}{1.274421in}}{\pgfqpoint{0.708607in}{1.271149in}}{\pgfqpoint{0.702783in}{1.265325in}}%
\pgfpathcurveto{\pgfqpoint{0.696959in}{1.259501in}}{\pgfqpoint{0.693686in}{1.251601in}}{\pgfqpoint{0.693686in}{1.243364in}}%
\pgfpathcurveto{\pgfqpoint{0.693686in}{1.235128in}}{\pgfqpoint{0.696959in}{1.227228in}}{\pgfqpoint{0.702783in}{1.221404in}}%
\pgfpathcurveto{\pgfqpoint{0.708607in}{1.215580in}}{\pgfqpoint{0.716507in}{1.212308in}}{\pgfqpoint{0.724743in}{1.212308in}}%
\pgfpathclose%
\pgfusepath{stroke,fill}%
\end{pgfscope}%
\begin{pgfscope}%
\pgfpathrectangle{\pgfqpoint{0.100000in}{0.212622in}}{\pgfqpoint{3.696000in}{3.696000in}}%
\pgfusepath{clip}%
\pgfsetbuttcap%
\pgfsetroundjoin%
\definecolor{currentfill}{rgb}{0.121569,0.466667,0.705882}%
\pgfsetfillcolor{currentfill}%
\pgfsetfillopacity{0.663750}%
\pgfsetlinewidth{1.003750pt}%
\definecolor{currentstroke}{rgb}{0.121569,0.466667,0.705882}%
\pgfsetstrokecolor{currentstroke}%
\pgfsetstrokeopacity{0.663750}%
\pgfsetdash{}{0pt}%
\pgfpathmoveto{\pgfqpoint{0.727473in}{1.212767in}}%
\pgfpathcurveto{\pgfqpoint{0.735710in}{1.212767in}}{\pgfqpoint{0.743610in}{1.216039in}}{\pgfqpoint{0.749434in}{1.221863in}}%
\pgfpathcurveto{\pgfqpoint{0.755258in}{1.227687in}}{\pgfqpoint{0.758530in}{1.235587in}}{\pgfqpoint{0.758530in}{1.243823in}}%
\pgfpathcurveto{\pgfqpoint{0.758530in}{1.252060in}}{\pgfqpoint{0.755258in}{1.259960in}}{\pgfqpoint{0.749434in}{1.265784in}}%
\pgfpathcurveto{\pgfqpoint{0.743610in}{1.271608in}}{\pgfqpoint{0.735710in}{1.274880in}}{\pgfqpoint{0.727473in}{1.274880in}}%
\pgfpathcurveto{\pgfqpoint{0.719237in}{1.274880in}}{\pgfqpoint{0.711337in}{1.271608in}}{\pgfqpoint{0.705513in}{1.265784in}}%
\pgfpathcurveto{\pgfqpoint{0.699689in}{1.259960in}}{\pgfqpoint{0.696417in}{1.252060in}}{\pgfqpoint{0.696417in}{1.243823in}}%
\pgfpathcurveto{\pgfqpoint{0.696417in}{1.235587in}}{\pgfqpoint{0.699689in}{1.227687in}}{\pgfqpoint{0.705513in}{1.221863in}}%
\pgfpathcurveto{\pgfqpoint{0.711337in}{1.216039in}}{\pgfqpoint{0.719237in}{1.212767in}}{\pgfqpoint{0.727473in}{1.212767in}}%
\pgfpathclose%
\pgfusepath{stroke,fill}%
\end{pgfscope}%
\begin{pgfscope}%
\pgfpathrectangle{\pgfqpoint{0.100000in}{0.212622in}}{\pgfqpoint{3.696000in}{3.696000in}}%
\pgfusepath{clip}%
\pgfsetbuttcap%
\pgfsetroundjoin%
\definecolor{currentfill}{rgb}{0.121569,0.466667,0.705882}%
\pgfsetfillcolor{currentfill}%
\pgfsetfillopacity{0.663761}%
\pgfsetlinewidth{1.003750pt}%
\definecolor{currentstroke}{rgb}{0.121569,0.466667,0.705882}%
\pgfsetstrokecolor{currentstroke}%
\pgfsetstrokeopacity{0.663761}%
\pgfsetdash{}{0pt}%
\pgfpathmoveto{\pgfqpoint{0.722800in}{1.212793in}}%
\pgfpathcurveto{\pgfqpoint{0.731036in}{1.212793in}}{\pgfqpoint{0.738936in}{1.216066in}}{\pgfqpoint{0.744760in}{1.221890in}}%
\pgfpathcurveto{\pgfqpoint{0.750584in}{1.227714in}}{\pgfqpoint{0.753857in}{1.235614in}}{\pgfqpoint{0.753857in}{1.243850in}}%
\pgfpathcurveto{\pgfqpoint{0.753857in}{1.252086in}}{\pgfqpoint{0.750584in}{1.259986in}}{\pgfqpoint{0.744760in}{1.265810in}}%
\pgfpathcurveto{\pgfqpoint{0.738936in}{1.271634in}}{\pgfqpoint{0.731036in}{1.274906in}}{\pgfqpoint{0.722800in}{1.274906in}}%
\pgfpathcurveto{\pgfqpoint{0.714564in}{1.274906in}}{\pgfqpoint{0.706664in}{1.271634in}}{\pgfqpoint{0.700840in}{1.265810in}}%
\pgfpathcurveto{\pgfqpoint{0.695016in}{1.259986in}}{\pgfqpoint{0.691744in}{1.252086in}}{\pgfqpoint{0.691744in}{1.243850in}}%
\pgfpathcurveto{\pgfqpoint{0.691744in}{1.235614in}}{\pgfqpoint{0.695016in}{1.227714in}}{\pgfqpoint{0.700840in}{1.221890in}}%
\pgfpathcurveto{\pgfqpoint{0.706664in}{1.216066in}}{\pgfqpoint{0.714564in}{1.212793in}}{\pgfqpoint{0.722800in}{1.212793in}}%
\pgfpathclose%
\pgfusepath{stroke,fill}%
\end{pgfscope}%
\begin{pgfscope}%
\pgfpathrectangle{\pgfqpoint{0.100000in}{0.212622in}}{\pgfqpoint{3.696000in}{3.696000in}}%
\pgfusepath{clip}%
\pgfsetbuttcap%
\pgfsetroundjoin%
\definecolor{currentfill}{rgb}{0.121569,0.466667,0.705882}%
\pgfsetfillcolor{currentfill}%
\pgfsetfillopacity{0.663771}%
\pgfsetlinewidth{1.003750pt}%
\definecolor{currentstroke}{rgb}{0.121569,0.466667,0.705882}%
\pgfsetstrokecolor{currentstroke}%
\pgfsetstrokeopacity{0.663771}%
\pgfsetdash{}{0pt}%
\pgfpathmoveto{\pgfqpoint{0.722221in}{1.212974in}}%
\pgfpathcurveto{\pgfqpoint{0.730458in}{1.212974in}}{\pgfqpoint{0.738358in}{1.216247in}}{\pgfqpoint{0.744182in}{1.222071in}}%
\pgfpathcurveto{\pgfqpoint{0.750006in}{1.227895in}}{\pgfqpoint{0.753278in}{1.235795in}}{\pgfqpoint{0.753278in}{1.244031in}}%
\pgfpathcurveto{\pgfqpoint{0.753278in}{1.252267in}}{\pgfqpoint{0.750006in}{1.260167in}}{\pgfqpoint{0.744182in}{1.265991in}}%
\pgfpathcurveto{\pgfqpoint{0.738358in}{1.271815in}}{\pgfqpoint{0.730458in}{1.275087in}}{\pgfqpoint{0.722221in}{1.275087in}}%
\pgfpathcurveto{\pgfqpoint{0.713985in}{1.275087in}}{\pgfqpoint{0.706085in}{1.271815in}}{\pgfqpoint{0.700261in}{1.265991in}}%
\pgfpathcurveto{\pgfqpoint{0.694437in}{1.260167in}}{\pgfqpoint{0.691165in}{1.252267in}}{\pgfqpoint{0.691165in}{1.244031in}}%
\pgfpathcurveto{\pgfqpoint{0.691165in}{1.235795in}}{\pgfqpoint{0.694437in}{1.227895in}}{\pgfqpoint{0.700261in}{1.222071in}}%
\pgfpathcurveto{\pgfqpoint{0.706085in}{1.216247in}}{\pgfqpoint{0.713985in}{1.212974in}}{\pgfqpoint{0.722221in}{1.212974in}}%
\pgfpathclose%
\pgfusepath{stroke,fill}%
\end{pgfscope}%
\begin{pgfscope}%
\pgfpathrectangle{\pgfqpoint{0.100000in}{0.212622in}}{\pgfqpoint{3.696000in}{3.696000in}}%
\pgfusepath{clip}%
\pgfsetbuttcap%
\pgfsetroundjoin%
\definecolor{currentfill}{rgb}{0.121569,0.466667,0.705882}%
\pgfsetfillcolor{currentfill}%
\pgfsetfillopacity{0.663780}%
\pgfsetlinewidth{1.003750pt}%
\definecolor{currentstroke}{rgb}{0.121569,0.466667,0.705882}%
\pgfsetstrokecolor{currentstroke}%
\pgfsetstrokeopacity{0.663780}%
\pgfsetdash{}{0pt}%
\pgfpathmoveto{\pgfqpoint{0.722105in}{1.213045in}}%
\pgfpathcurveto{\pgfqpoint{0.730341in}{1.213045in}}{\pgfqpoint{0.738241in}{1.216318in}}{\pgfqpoint{0.744065in}{1.222142in}}%
\pgfpathcurveto{\pgfqpoint{0.749889in}{1.227966in}}{\pgfqpoint{0.753161in}{1.235866in}}{\pgfqpoint{0.753161in}{1.244102in}}%
\pgfpathcurveto{\pgfqpoint{0.753161in}{1.252338in}}{\pgfqpoint{0.749889in}{1.260238in}}{\pgfqpoint{0.744065in}{1.266062in}}%
\pgfpathcurveto{\pgfqpoint{0.738241in}{1.271886in}}{\pgfqpoint{0.730341in}{1.275158in}}{\pgfqpoint{0.722105in}{1.275158in}}%
\pgfpathcurveto{\pgfqpoint{0.713868in}{1.275158in}}{\pgfqpoint{0.705968in}{1.271886in}}{\pgfqpoint{0.700144in}{1.266062in}}%
\pgfpathcurveto{\pgfqpoint{0.694320in}{1.260238in}}{\pgfqpoint{0.691048in}{1.252338in}}{\pgfqpoint{0.691048in}{1.244102in}}%
\pgfpathcurveto{\pgfqpoint{0.691048in}{1.235866in}}{\pgfqpoint{0.694320in}{1.227966in}}{\pgfqpoint{0.700144in}{1.222142in}}%
\pgfpathcurveto{\pgfqpoint{0.705968in}{1.216318in}}{\pgfqpoint{0.713868in}{1.213045in}}{\pgfqpoint{0.722105in}{1.213045in}}%
\pgfpathclose%
\pgfusepath{stroke,fill}%
\end{pgfscope}%
\begin{pgfscope}%
\pgfpathrectangle{\pgfqpoint{0.100000in}{0.212622in}}{\pgfqpoint{3.696000in}{3.696000in}}%
\pgfusepath{clip}%
\pgfsetbuttcap%
\pgfsetroundjoin%
\definecolor{currentfill}{rgb}{0.121569,0.466667,0.705882}%
\pgfsetfillcolor{currentfill}%
\pgfsetfillopacity{0.663792}%
\pgfsetlinewidth{1.003750pt}%
\definecolor{currentstroke}{rgb}{0.121569,0.466667,0.705882}%
\pgfsetstrokecolor{currentstroke}%
\pgfsetstrokeopacity{0.663792}%
\pgfsetdash{}{0pt}%
\pgfpathmoveto{\pgfqpoint{0.722412in}{1.212975in}}%
\pgfpathcurveto{\pgfqpoint{0.730648in}{1.212975in}}{\pgfqpoint{0.738548in}{1.216248in}}{\pgfqpoint{0.744372in}{1.222072in}}%
\pgfpathcurveto{\pgfqpoint{0.750196in}{1.227896in}}{\pgfqpoint{0.753468in}{1.235796in}}{\pgfqpoint{0.753468in}{1.244032in}}%
\pgfpathcurveto{\pgfqpoint{0.753468in}{1.252268in}}{\pgfqpoint{0.750196in}{1.260168in}}{\pgfqpoint{0.744372in}{1.265992in}}%
\pgfpathcurveto{\pgfqpoint{0.738548in}{1.271816in}}{\pgfqpoint{0.730648in}{1.275088in}}{\pgfqpoint{0.722412in}{1.275088in}}%
\pgfpathcurveto{\pgfqpoint{0.714176in}{1.275088in}}{\pgfqpoint{0.706276in}{1.271816in}}{\pgfqpoint{0.700452in}{1.265992in}}%
\pgfpathcurveto{\pgfqpoint{0.694628in}{1.260168in}}{\pgfqpoint{0.691355in}{1.252268in}}{\pgfqpoint{0.691355in}{1.244032in}}%
\pgfpathcurveto{\pgfqpoint{0.691355in}{1.235796in}}{\pgfqpoint{0.694628in}{1.227896in}}{\pgfqpoint{0.700452in}{1.222072in}}%
\pgfpathcurveto{\pgfqpoint{0.706276in}{1.216248in}}{\pgfqpoint{0.714176in}{1.212975in}}{\pgfqpoint{0.722412in}{1.212975in}}%
\pgfpathclose%
\pgfusepath{stroke,fill}%
\end{pgfscope}%
\begin{pgfscope}%
\pgfpathrectangle{\pgfqpoint{0.100000in}{0.212622in}}{\pgfqpoint{3.696000in}{3.696000in}}%
\pgfusepath{clip}%
\pgfsetbuttcap%
\pgfsetroundjoin%
\definecolor{currentfill}{rgb}{0.121569,0.466667,0.705882}%
\pgfsetfillcolor{currentfill}%
\pgfsetfillopacity{0.663813}%
\pgfsetlinewidth{1.003750pt}%
\definecolor{currentstroke}{rgb}{0.121569,0.466667,0.705882}%
\pgfsetstrokecolor{currentstroke}%
\pgfsetstrokeopacity{0.663813}%
\pgfsetdash{}{0pt}%
\pgfpathmoveto{\pgfqpoint{0.723456in}{1.212817in}}%
\pgfpathcurveto{\pgfqpoint{0.731692in}{1.212817in}}{\pgfqpoint{0.739592in}{1.216089in}}{\pgfqpoint{0.745416in}{1.221913in}}%
\pgfpathcurveto{\pgfqpoint{0.751240in}{1.227737in}}{\pgfqpoint{0.754512in}{1.235637in}}{\pgfqpoint{0.754512in}{1.243873in}}%
\pgfpathcurveto{\pgfqpoint{0.754512in}{1.252109in}}{\pgfqpoint{0.751240in}{1.260009in}}{\pgfqpoint{0.745416in}{1.265833in}}%
\pgfpathcurveto{\pgfqpoint{0.739592in}{1.271657in}}{\pgfqpoint{0.731692in}{1.274930in}}{\pgfqpoint{0.723456in}{1.274930in}}%
\pgfpathcurveto{\pgfqpoint{0.715219in}{1.274930in}}{\pgfqpoint{0.707319in}{1.271657in}}{\pgfqpoint{0.701495in}{1.265833in}}%
\pgfpathcurveto{\pgfqpoint{0.695671in}{1.260009in}}{\pgfqpoint{0.692399in}{1.252109in}}{\pgfqpoint{0.692399in}{1.243873in}}%
\pgfpathcurveto{\pgfqpoint{0.692399in}{1.235637in}}{\pgfqpoint{0.695671in}{1.227737in}}{\pgfqpoint{0.701495in}{1.221913in}}%
\pgfpathcurveto{\pgfqpoint{0.707319in}{1.216089in}}{\pgfqpoint{0.715219in}{1.212817in}}{\pgfqpoint{0.723456in}{1.212817in}}%
\pgfpathclose%
\pgfusepath{stroke,fill}%
\end{pgfscope}%
\begin{pgfscope}%
\pgfpathrectangle{\pgfqpoint{0.100000in}{0.212622in}}{\pgfqpoint{3.696000in}{3.696000in}}%
\pgfusepath{clip}%
\pgfsetbuttcap%
\pgfsetroundjoin%
\definecolor{currentfill}{rgb}{0.121569,0.466667,0.705882}%
\pgfsetfillcolor{currentfill}%
\pgfsetfillopacity{0.663814}%
\pgfsetlinewidth{1.003750pt}%
\definecolor{currentstroke}{rgb}{0.121569,0.466667,0.705882}%
\pgfsetstrokecolor{currentstroke}%
\pgfsetstrokeopacity{0.663814}%
\pgfsetdash{}{0pt}%
\pgfpathmoveto{\pgfqpoint{0.727214in}{1.212723in}}%
\pgfpathcurveto{\pgfqpoint{0.735450in}{1.212723in}}{\pgfqpoint{0.743350in}{1.215996in}}{\pgfqpoint{0.749174in}{1.221820in}}%
\pgfpathcurveto{\pgfqpoint{0.754998in}{1.227644in}}{\pgfqpoint{0.758271in}{1.235544in}}{\pgfqpoint{0.758271in}{1.243780in}}%
\pgfpathcurveto{\pgfqpoint{0.758271in}{1.252016in}}{\pgfqpoint{0.754998in}{1.259916in}}{\pgfqpoint{0.749174in}{1.265740in}}%
\pgfpathcurveto{\pgfqpoint{0.743350in}{1.271564in}}{\pgfqpoint{0.735450in}{1.274836in}}{\pgfqpoint{0.727214in}{1.274836in}}%
\pgfpathcurveto{\pgfqpoint{0.718978in}{1.274836in}}{\pgfqpoint{0.711078in}{1.271564in}}{\pgfqpoint{0.705254in}{1.265740in}}%
\pgfpathcurveto{\pgfqpoint{0.699430in}{1.259916in}}{\pgfqpoint{0.696158in}{1.252016in}}{\pgfqpoint{0.696158in}{1.243780in}}%
\pgfpathcurveto{\pgfqpoint{0.696158in}{1.235544in}}{\pgfqpoint{0.699430in}{1.227644in}}{\pgfqpoint{0.705254in}{1.221820in}}%
\pgfpathcurveto{\pgfqpoint{0.711078in}{1.215996in}}{\pgfqpoint{0.718978in}{1.212723in}}{\pgfqpoint{0.727214in}{1.212723in}}%
\pgfpathclose%
\pgfusepath{stroke,fill}%
\end{pgfscope}%
\begin{pgfscope}%
\pgfpathrectangle{\pgfqpoint{0.100000in}{0.212622in}}{\pgfqpoint{3.696000in}{3.696000in}}%
\pgfusepath{clip}%
\pgfsetbuttcap%
\pgfsetroundjoin%
\definecolor{currentfill}{rgb}{0.121569,0.466667,0.705882}%
\pgfsetfillcolor{currentfill}%
\pgfsetfillopacity{0.663851}%
\pgfsetlinewidth{1.003750pt}%
\definecolor{currentstroke}{rgb}{0.121569,0.466667,0.705882}%
\pgfsetstrokecolor{currentstroke}%
\pgfsetstrokeopacity{0.663851}%
\pgfsetdash{}{0pt}%
\pgfpathmoveto{\pgfqpoint{0.727075in}{1.212703in}}%
\pgfpathcurveto{\pgfqpoint{0.735311in}{1.212703in}}{\pgfqpoint{0.743212in}{1.215976in}}{\pgfqpoint{0.749035in}{1.221799in}}%
\pgfpathcurveto{\pgfqpoint{0.754859in}{1.227623in}}{\pgfqpoint{0.758132in}{1.235523in}}{\pgfqpoint{0.758132in}{1.243760in}}%
\pgfpathcurveto{\pgfqpoint{0.758132in}{1.251996in}}{\pgfqpoint{0.754859in}{1.259896in}}{\pgfqpoint{0.749035in}{1.265720in}}%
\pgfpathcurveto{\pgfqpoint{0.743212in}{1.271544in}}{\pgfqpoint{0.735311in}{1.274816in}}{\pgfqpoint{0.727075in}{1.274816in}}%
\pgfpathcurveto{\pgfqpoint{0.718839in}{1.274816in}}{\pgfqpoint{0.710939in}{1.271544in}}{\pgfqpoint{0.705115in}{1.265720in}}%
\pgfpathcurveto{\pgfqpoint{0.699291in}{1.259896in}}{\pgfqpoint{0.696019in}{1.251996in}}{\pgfqpoint{0.696019in}{1.243760in}}%
\pgfpathcurveto{\pgfqpoint{0.696019in}{1.235523in}}{\pgfqpoint{0.699291in}{1.227623in}}{\pgfqpoint{0.705115in}{1.221799in}}%
\pgfpathcurveto{\pgfqpoint{0.710939in}{1.215976in}}{\pgfqpoint{0.718839in}{1.212703in}}{\pgfqpoint{0.727075in}{1.212703in}}%
\pgfpathclose%
\pgfusepath{stroke,fill}%
\end{pgfscope}%
\begin{pgfscope}%
\pgfpathrectangle{\pgfqpoint{0.100000in}{0.212622in}}{\pgfqpoint{3.696000in}{3.696000in}}%
\pgfusepath{clip}%
\pgfsetbuttcap%
\pgfsetroundjoin%
\definecolor{currentfill}{rgb}{0.121569,0.466667,0.705882}%
\pgfsetfillcolor{currentfill}%
\pgfsetfillopacity{0.663872}%
\pgfsetlinewidth{1.003750pt}%
\definecolor{currentstroke}{rgb}{0.121569,0.466667,0.705882}%
\pgfsetstrokecolor{currentstroke}%
\pgfsetstrokeopacity{0.663872}%
\pgfsetdash{}{0pt}%
\pgfpathmoveto{\pgfqpoint{0.726998in}{1.212693in}}%
\pgfpathcurveto{\pgfqpoint{0.735234in}{1.212693in}}{\pgfqpoint{0.743134in}{1.215965in}}{\pgfqpoint{0.748958in}{1.221789in}}%
\pgfpathcurveto{\pgfqpoint{0.754782in}{1.227613in}}{\pgfqpoint{0.758054in}{1.235513in}}{\pgfqpoint{0.758054in}{1.243749in}}%
\pgfpathcurveto{\pgfqpoint{0.758054in}{1.251985in}}{\pgfqpoint{0.754782in}{1.259885in}}{\pgfqpoint{0.748958in}{1.265709in}}%
\pgfpathcurveto{\pgfqpoint{0.743134in}{1.271533in}}{\pgfqpoint{0.735234in}{1.274806in}}{\pgfqpoint{0.726998in}{1.274806in}}%
\pgfpathcurveto{\pgfqpoint{0.718762in}{1.274806in}}{\pgfqpoint{0.710862in}{1.271533in}}{\pgfqpoint{0.705038in}{1.265709in}}%
\pgfpathcurveto{\pgfqpoint{0.699214in}{1.259885in}}{\pgfqpoint{0.695941in}{1.251985in}}{\pgfqpoint{0.695941in}{1.243749in}}%
\pgfpathcurveto{\pgfqpoint{0.695941in}{1.235513in}}{\pgfqpoint{0.699214in}{1.227613in}}{\pgfqpoint{0.705038in}{1.221789in}}%
\pgfpathcurveto{\pgfqpoint{0.710862in}{1.215965in}}{\pgfqpoint{0.718762in}{1.212693in}}{\pgfqpoint{0.726998in}{1.212693in}}%
\pgfpathclose%
\pgfusepath{stroke,fill}%
\end{pgfscope}%
\begin{pgfscope}%
\pgfpathrectangle{\pgfqpoint{0.100000in}{0.212622in}}{\pgfqpoint{3.696000in}{3.696000in}}%
\pgfusepath{clip}%
\pgfsetbuttcap%
\pgfsetroundjoin%
\definecolor{currentfill}{rgb}{0.121569,0.466667,0.705882}%
\pgfsetfillcolor{currentfill}%
\pgfsetfillopacity{0.663883}%
\pgfsetlinewidth{1.003750pt}%
\definecolor{currentstroke}{rgb}{0.121569,0.466667,0.705882}%
\pgfsetstrokecolor{currentstroke}%
\pgfsetstrokeopacity{0.663883}%
\pgfsetdash{}{0pt}%
\pgfpathmoveto{\pgfqpoint{0.726955in}{1.212687in}}%
\pgfpathcurveto{\pgfqpoint{0.735191in}{1.212687in}}{\pgfqpoint{0.743091in}{1.215959in}}{\pgfqpoint{0.748915in}{1.221783in}}%
\pgfpathcurveto{\pgfqpoint{0.754739in}{1.227607in}}{\pgfqpoint{0.758011in}{1.235507in}}{\pgfqpoint{0.758011in}{1.243743in}}%
\pgfpathcurveto{\pgfqpoint{0.758011in}{1.251980in}}{\pgfqpoint{0.754739in}{1.259880in}}{\pgfqpoint{0.748915in}{1.265704in}}%
\pgfpathcurveto{\pgfqpoint{0.743091in}{1.271527in}}{\pgfqpoint{0.735191in}{1.274800in}}{\pgfqpoint{0.726955in}{1.274800in}}%
\pgfpathcurveto{\pgfqpoint{0.718719in}{1.274800in}}{\pgfqpoint{0.710819in}{1.271527in}}{\pgfqpoint{0.704995in}{1.265704in}}%
\pgfpathcurveto{\pgfqpoint{0.699171in}{1.259880in}}{\pgfqpoint{0.695898in}{1.251980in}}{\pgfqpoint{0.695898in}{1.243743in}}%
\pgfpathcurveto{\pgfqpoint{0.695898in}{1.235507in}}{\pgfqpoint{0.699171in}{1.227607in}}{\pgfqpoint{0.704995in}{1.221783in}}%
\pgfpathcurveto{\pgfqpoint{0.710819in}{1.215959in}}{\pgfqpoint{0.718719in}{1.212687in}}{\pgfqpoint{0.726955in}{1.212687in}}%
\pgfpathclose%
\pgfusepath{stroke,fill}%
\end{pgfscope}%
\begin{pgfscope}%
\pgfpathrectangle{\pgfqpoint{0.100000in}{0.212622in}}{\pgfqpoint{3.696000in}{3.696000in}}%
\pgfusepath{clip}%
\pgfsetbuttcap%
\pgfsetroundjoin%
\definecolor{currentfill}{rgb}{0.121569,0.466667,0.705882}%
\pgfsetfillcolor{currentfill}%
\pgfsetfillopacity{0.663889}%
\pgfsetlinewidth{1.003750pt}%
\definecolor{currentstroke}{rgb}{0.121569,0.466667,0.705882}%
\pgfsetstrokecolor{currentstroke}%
\pgfsetstrokeopacity{0.663889}%
\pgfsetdash{}{0pt}%
\pgfpathmoveto{\pgfqpoint{0.726932in}{1.212683in}}%
\pgfpathcurveto{\pgfqpoint{0.735168in}{1.212683in}}{\pgfqpoint{0.743068in}{1.215955in}}{\pgfqpoint{0.748892in}{1.221779in}}%
\pgfpathcurveto{\pgfqpoint{0.754716in}{1.227603in}}{\pgfqpoint{0.757989in}{1.235503in}}{\pgfqpoint{0.757989in}{1.243739in}}%
\pgfpathcurveto{\pgfqpoint{0.757989in}{1.251976in}}{\pgfqpoint{0.754716in}{1.259876in}}{\pgfqpoint{0.748892in}{1.265700in}}%
\pgfpathcurveto{\pgfqpoint{0.743068in}{1.271524in}}{\pgfqpoint{0.735168in}{1.274796in}}{\pgfqpoint{0.726932in}{1.274796in}}%
\pgfpathcurveto{\pgfqpoint{0.718696in}{1.274796in}}{\pgfqpoint{0.710796in}{1.271524in}}{\pgfqpoint{0.704972in}{1.265700in}}%
\pgfpathcurveto{\pgfqpoint{0.699148in}{1.259876in}}{\pgfqpoint{0.695876in}{1.251976in}}{\pgfqpoint{0.695876in}{1.243739in}}%
\pgfpathcurveto{\pgfqpoint{0.695876in}{1.235503in}}{\pgfqpoint{0.699148in}{1.227603in}}{\pgfqpoint{0.704972in}{1.221779in}}%
\pgfpathcurveto{\pgfqpoint{0.710796in}{1.215955in}}{\pgfqpoint{0.718696in}{1.212683in}}{\pgfqpoint{0.726932in}{1.212683in}}%
\pgfpathclose%
\pgfusepath{stroke,fill}%
\end{pgfscope}%
\begin{pgfscope}%
\pgfpathrectangle{\pgfqpoint{0.100000in}{0.212622in}}{\pgfqpoint{3.696000in}{3.696000in}}%
\pgfusepath{clip}%
\pgfsetbuttcap%
\pgfsetroundjoin%
\definecolor{currentfill}{rgb}{0.121569,0.466667,0.705882}%
\pgfsetfillcolor{currentfill}%
\pgfsetfillopacity{0.663892}%
\pgfsetlinewidth{1.003750pt}%
\definecolor{currentstroke}{rgb}{0.121569,0.466667,0.705882}%
\pgfsetstrokecolor{currentstroke}%
\pgfsetstrokeopacity{0.663892}%
\pgfsetdash{}{0pt}%
\pgfpathmoveto{\pgfqpoint{0.726919in}{1.212681in}}%
\pgfpathcurveto{\pgfqpoint{0.735156in}{1.212681in}}{\pgfqpoint{0.743056in}{1.215953in}}{\pgfqpoint{0.748880in}{1.221777in}}%
\pgfpathcurveto{\pgfqpoint{0.754704in}{1.227601in}}{\pgfqpoint{0.757976in}{1.235501in}}{\pgfqpoint{0.757976in}{1.243737in}}%
\pgfpathcurveto{\pgfqpoint{0.757976in}{1.251974in}}{\pgfqpoint{0.754704in}{1.259874in}}{\pgfqpoint{0.748880in}{1.265698in}}%
\pgfpathcurveto{\pgfqpoint{0.743056in}{1.271522in}}{\pgfqpoint{0.735156in}{1.274794in}}{\pgfqpoint{0.726919in}{1.274794in}}%
\pgfpathcurveto{\pgfqpoint{0.718683in}{1.274794in}}{\pgfqpoint{0.710783in}{1.271522in}}{\pgfqpoint{0.704959in}{1.265698in}}%
\pgfpathcurveto{\pgfqpoint{0.699135in}{1.259874in}}{\pgfqpoint{0.695863in}{1.251974in}}{\pgfqpoint{0.695863in}{1.243737in}}%
\pgfpathcurveto{\pgfqpoint{0.695863in}{1.235501in}}{\pgfqpoint{0.699135in}{1.227601in}}{\pgfqpoint{0.704959in}{1.221777in}}%
\pgfpathcurveto{\pgfqpoint{0.710783in}{1.215953in}}{\pgfqpoint{0.718683in}{1.212681in}}{\pgfqpoint{0.726919in}{1.212681in}}%
\pgfpathclose%
\pgfusepath{stroke,fill}%
\end{pgfscope}%
\begin{pgfscope}%
\pgfpathrectangle{\pgfqpoint{0.100000in}{0.212622in}}{\pgfqpoint{3.696000in}{3.696000in}}%
\pgfusepath{clip}%
\pgfsetbuttcap%
\pgfsetroundjoin%
\definecolor{currentfill}{rgb}{0.121569,0.466667,0.705882}%
\pgfsetfillcolor{currentfill}%
\pgfsetfillopacity{0.663894}%
\pgfsetlinewidth{1.003750pt}%
\definecolor{currentstroke}{rgb}{0.121569,0.466667,0.705882}%
\pgfsetstrokecolor{currentstroke}%
\pgfsetstrokeopacity{0.663894}%
\pgfsetdash{}{0pt}%
\pgfpathmoveto{\pgfqpoint{0.726912in}{1.212680in}}%
\pgfpathcurveto{\pgfqpoint{0.735148in}{1.212680in}}{\pgfqpoint{0.743048in}{1.215952in}}{\pgfqpoint{0.748872in}{1.221776in}}%
\pgfpathcurveto{\pgfqpoint{0.754696in}{1.227600in}}{\pgfqpoint{0.757969in}{1.235500in}}{\pgfqpoint{0.757969in}{1.243737in}}%
\pgfpathcurveto{\pgfqpoint{0.757969in}{1.251973in}}{\pgfqpoint{0.754696in}{1.259873in}}{\pgfqpoint{0.748872in}{1.265697in}}%
\pgfpathcurveto{\pgfqpoint{0.743048in}{1.271521in}}{\pgfqpoint{0.735148in}{1.274793in}}{\pgfqpoint{0.726912in}{1.274793in}}%
\pgfpathcurveto{\pgfqpoint{0.718676in}{1.274793in}}{\pgfqpoint{0.710776in}{1.271521in}}{\pgfqpoint{0.704952in}{1.265697in}}%
\pgfpathcurveto{\pgfqpoint{0.699128in}{1.259873in}}{\pgfqpoint{0.695856in}{1.251973in}}{\pgfqpoint{0.695856in}{1.243737in}}%
\pgfpathcurveto{\pgfqpoint{0.695856in}{1.235500in}}{\pgfqpoint{0.699128in}{1.227600in}}{\pgfqpoint{0.704952in}{1.221776in}}%
\pgfpathcurveto{\pgfqpoint{0.710776in}{1.215952in}}{\pgfqpoint{0.718676in}{1.212680in}}{\pgfqpoint{0.726912in}{1.212680in}}%
\pgfpathclose%
\pgfusepath{stroke,fill}%
\end{pgfscope}%
\begin{pgfscope}%
\pgfpathrectangle{\pgfqpoint{0.100000in}{0.212622in}}{\pgfqpoint{3.696000in}{3.696000in}}%
\pgfusepath{clip}%
\pgfsetbuttcap%
\pgfsetroundjoin%
\definecolor{currentfill}{rgb}{0.121569,0.466667,0.705882}%
\pgfsetfillcolor{currentfill}%
\pgfsetfillopacity{0.663895}%
\pgfsetlinewidth{1.003750pt}%
\definecolor{currentstroke}{rgb}{0.121569,0.466667,0.705882}%
\pgfsetstrokecolor{currentstroke}%
\pgfsetstrokeopacity{0.663895}%
\pgfsetdash{}{0pt}%
\pgfpathmoveto{\pgfqpoint{0.726908in}{1.212680in}}%
\pgfpathcurveto{\pgfqpoint{0.735145in}{1.212680in}}{\pgfqpoint{0.743045in}{1.215952in}}{\pgfqpoint{0.748869in}{1.221776in}}%
\pgfpathcurveto{\pgfqpoint{0.754692in}{1.227600in}}{\pgfqpoint{0.757965in}{1.235500in}}{\pgfqpoint{0.757965in}{1.243736in}}%
\pgfpathcurveto{\pgfqpoint{0.757965in}{1.251972in}}{\pgfqpoint{0.754692in}{1.259872in}}{\pgfqpoint{0.748869in}{1.265696in}}%
\pgfpathcurveto{\pgfqpoint{0.743045in}{1.271520in}}{\pgfqpoint{0.735145in}{1.274793in}}{\pgfqpoint{0.726908in}{1.274793in}}%
\pgfpathcurveto{\pgfqpoint{0.718672in}{1.274793in}}{\pgfqpoint{0.710772in}{1.271520in}}{\pgfqpoint{0.704948in}{1.265696in}}%
\pgfpathcurveto{\pgfqpoint{0.699124in}{1.259872in}}{\pgfqpoint{0.695852in}{1.251972in}}{\pgfqpoint{0.695852in}{1.243736in}}%
\pgfpathcurveto{\pgfqpoint{0.695852in}{1.235500in}}{\pgfqpoint{0.699124in}{1.227600in}}{\pgfqpoint{0.704948in}{1.221776in}}%
\pgfpathcurveto{\pgfqpoint{0.710772in}{1.215952in}}{\pgfqpoint{0.718672in}{1.212680in}}{\pgfqpoint{0.726908in}{1.212680in}}%
\pgfpathclose%
\pgfusepath{stroke,fill}%
\end{pgfscope}%
\begin{pgfscope}%
\pgfpathrectangle{\pgfqpoint{0.100000in}{0.212622in}}{\pgfqpoint{3.696000in}{3.696000in}}%
\pgfusepath{clip}%
\pgfsetbuttcap%
\pgfsetroundjoin%
\definecolor{currentfill}{rgb}{0.121569,0.466667,0.705882}%
\pgfsetfillcolor{currentfill}%
\pgfsetfillopacity{0.663896}%
\pgfsetlinewidth{1.003750pt}%
\definecolor{currentstroke}{rgb}{0.121569,0.466667,0.705882}%
\pgfsetstrokecolor{currentstroke}%
\pgfsetstrokeopacity{0.663896}%
\pgfsetdash{}{0pt}%
\pgfpathmoveto{\pgfqpoint{0.726906in}{1.212679in}}%
\pgfpathcurveto{\pgfqpoint{0.735142in}{1.212679in}}{\pgfqpoint{0.743042in}{1.215952in}}{\pgfqpoint{0.748866in}{1.221776in}}%
\pgfpathcurveto{\pgfqpoint{0.754690in}{1.227600in}}{\pgfqpoint{0.757963in}{1.235500in}}{\pgfqpoint{0.757963in}{1.243736in}}%
\pgfpathcurveto{\pgfqpoint{0.757963in}{1.251972in}}{\pgfqpoint{0.754690in}{1.259872in}}{\pgfqpoint{0.748866in}{1.265696in}}%
\pgfpathcurveto{\pgfqpoint{0.743042in}{1.271520in}}{\pgfqpoint{0.735142in}{1.274792in}}{\pgfqpoint{0.726906in}{1.274792in}}%
\pgfpathcurveto{\pgfqpoint{0.718670in}{1.274792in}}{\pgfqpoint{0.710770in}{1.271520in}}{\pgfqpoint{0.704946in}{1.265696in}}%
\pgfpathcurveto{\pgfqpoint{0.699122in}{1.259872in}}{\pgfqpoint{0.695850in}{1.251972in}}{\pgfqpoint{0.695850in}{1.243736in}}%
\pgfpathcurveto{\pgfqpoint{0.695850in}{1.235500in}}{\pgfqpoint{0.699122in}{1.227600in}}{\pgfqpoint{0.704946in}{1.221776in}}%
\pgfpathcurveto{\pgfqpoint{0.710770in}{1.215952in}}{\pgfqpoint{0.718670in}{1.212679in}}{\pgfqpoint{0.726906in}{1.212679in}}%
\pgfpathclose%
\pgfusepath{stroke,fill}%
\end{pgfscope}%
\begin{pgfscope}%
\pgfpathrectangle{\pgfqpoint{0.100000in}{0.212622in}}{\pgfqpoint{3.696000in}{3.696000in}}%
\pgfusepath{clip}%
\pgfsetbuttcap%
\pgfsetroundjoin%
\definecolor{currentfill}{rgb}{0.121569,0.466667,0.705882}%
\pgfsetfillcolor{currentfill}%
\pgfsetfillopacity{0.663896}%
\pgfsetlinewidth{1.003750pt}%
\definecolor{currentstroke}{rgb}{0.121569,0.466667,0.705882}%
\pgfsetstrokecolor{currentstroke}%
\pgfsetstrokeopacity{0.663896}%
\pgfsetdash{}{0pt}%
\pgfpathmoveto{\pgfqpoint{0.726905in}{1.212679in}}%
\pgfpathcurveto{\pgfqpoint{0.735141in}{1.212679in}}{\pgfqpoint{0.743041in}{1.215952in}}{\pgfqpoint{0.748865in}{1.221776in}}%
\pgfpathcurveto{\pgfqpoint{0.754689in}{1.227600in}}{\pgfqpoint{0.757961in}{1.235500in}}{\pgfqpoint{0.757961in}{1.243736in}}%
\pgfpathcurveto{\pgfqpoint{0.757961in}{1.251972in}}{\pgfqpoint{0.754689in}{1.259872in}}{\pgfqpoint{0.748865in}{1.265696in}}%
\pgfpathcurveto{\pgfqpoint{0.743041in}{1.271520in}}{\pgfqpoint{0.735141in}{1.274792in}}{\pgfqpoint{0.726905in}{1.274792in}}%
\pgfpathcurveto{\pgfqpoint{0.718669in}{1.274792in}}{\pgfqpoint{0.710769in}{1.271520in}}{\pgfqpoint{0.704945in}{1.265696in}}%
\pgfpathcurveto{\pgfqpoint{0.699121in}{1.259872in}}{\pgfqpoint{0.695848in}{1.251972in}}{\pgfqpoint{0.695848in}{1.243736in}}%
\pgfpathcurveto{\pgfqpoint{0.695848in}{1.235500in}}{\pgfqpoint{0.699121in}{1.227600in}}{\pgfqpoint{0.704945in}{1.221776in}}%
\pgfpathcurveto{\pgfqpoint{0.710769in}{1.215952in}}{\pgfqpoint{0.718669in}{1.212679in}}{\pgfqpoint{0.726905in}{1.212679in}}%
\pgfpathclose%
\pgfusepath{stroke,fill}%
\end{pgfscope}%
\begin{pgfscope}%
\pgfpathrectangle{\pgfqpoint{0.100000in}{0.212622in}}{\pgfqpoint{3.696000in}{3.696000in}}%
\pgfusepath{clip}%
\pgfsetbuttcap%
\pgfsetroundjoin%
\definecolor{currentfill}{rgb}{0.121569,0.466667,0.705882}%
\pgfsetfillcolor{currentfill}%
\pgfsetfillopacity{0.663896}%
\pgfsetlinewidth{1.003750pt}%
\definecolor{currentstroke}{rgb}{0.121569,0.466667,0.705882}%
\pgfsetstrokecolor{currentstroke}%
\pgfsetstrokeopacity{0.663896}%
\pgfsetdash{}{0pt}%
\pgfpathmoveto{\pgfqpoint{0.726904in}{1.212679in}}%
\pgfpathcurveto{\pgfqpoint{0.735141in}{1.212679in}}{\pgfqpoint{0.743041in}{1.215952in}}{\pgfqpoint{0.748865in}{1.221776in}}%
\pgfpathcurveto{\pgfqpoint{0.754688in}{1.227599in}}{\pgfqpoint{0.757961in}{1.235500in}}{\pgfqpoint{0.757961in}{1.243736in}}%
\pgfpathcurveto{\pgfqpoint{0.757961in}{1.251972in}}{\pgfqpoint{0.754688in}{1.259872in}}{\pgfqpoint{0.748865in}{1.265696in}}%
\pgfpathcurveto{\pgfqpoint{0.743041in}{1.271520in}}{\pgfqpoint{0.735141in}{1.274792in}}{\pgfqpoint{0.726904in}{1.274792in}}%
\pgfpathcurveto{\pgfqpoint{0.718668in}{1.274792in}}{\pgfqpoint{0.710768in}{1.271520in}}{\pgfqpoint{0.704944in}{1.265696in}}%
\pgfpathcurveto{\pgfqpoint{0.699120in}{1.259872in}}{\pgfqpoint{0.695848in}{1.251972in}}{\pgfqpoint{0.695848in}{1.243736in}}%
\pgfpathcurveto{\pgfqpoint{0.695848in}{1.235500in}}{\pgfqpoint{0.699120in}{1.227599in}}{\pgfqpoint{0.704944in}{1.221776in}}%
\pgfpathcurveto{\pgfqpoint{0.710768in}{1.215952in}}{\pgfqpoint{0.718668in}{1.212679in}}{\pgfqpoint{0.726904in}{1.212679in}}%
\pgfpathclose%
\pgfusepath{stroke,fill}%
\end{pgfscope}%
\begin{pgfscope}%
\pgfpathrectangle{\pgfqpoint{0.100000in}{0.212622in}}{\pgfqpoint{3.696000in}{3.696000in}}%
\pgfusepath{clip}%
\pgfsetbuttcap%
\pgfsetroundjoin%
\definecolor{currentfill}{rgb}{0.121569,0.466667,0.705882}%
\pgfsetfillcolor{currentfill}%
\pgfsetfillopacity{0.663896}%
\pgfsetlinewidth{1.003750pt}%
\definecolor{currentstroke}{rgb}{0.121569,0.466667,0.705882}%
\pgfsetstrokecolor{currentstroke}%
\pgfsetstrokeopacity{0.663896}%
\pgfsetdash{}{0pt}%
\pgfpathmoveto{\pgfqpoint{0.726904in}{1.212679in}}%
\pgfpathcurveto{\pgfqpoint{0.735140in}{1.212679in}}{\pgfqpoint{0.743040in}{1.215952in}}{\pgfqpoint{0.748864in}{1.221776in}}%
\pgfpathcurveto{\pgfqpoint{0.754688in}{1.227599in}}{\pgfqpoint{0.757960in}{1.235500in}}{\pgfqpoint{0.757960in}{1.243736in}}%
\pgfpathcurveto{\pgfqpoint{0.757960in}{1.251972in}}{\pgfqpoint{0.754688in}{1.259872in}}{\pgfqpoint{0.748864in}{1.265696in}}%
\pgfpathcurveto{\pgfqpoint{0.743040in}{1.271520in}}{\pgfqpoint{0.735140in}{1.274792in}}{\pgfqpoint{0.726904in}{1.274792in}}%
\pgfpathcurveto{\pgfqpoint{0.718668in}{1.274792in}}{\pgfqpoint{0.710768in}{1.271520in}}{\pgfqpoint{0.704944in}{1.265696in}}%
\pgfpathcurveto{\pgfqpoint{0.699120in}{1.259872in}}{\pgfqpoint{0.695847in}{1.251972in}}{\pgfqpoint{0.695847in}{1.243736in}}%
\pgfpathcurveto{\pgfqpoint{0.695847in}{1.235500in}}{\pgfqpoint{0.699120in}{1.227599in}}{\pgfqpoint{0.704944in}{1.221776in}}%
\pgfpathcurveto{\pgfqpoint{0.710768in}{1.215952in}}{\pgfqpoint{0.718668in}{1.212679in}}{\pgfqpoint{0.726904in}{1.212679in}}%
\pgfpathclose%
\pgfusepath{stroke,fill}%
\end{pgfscope}%
\begin{pgfscope}%
\pgfpathrectangle{\pgfqpoint{0.100000in}{0.212622in}}{\pgfqpoint{3.696000in}{3.696000in}}%
\pgfusepath{clip}%
\pgfsetbuttcap%
\pgfsetroundjoin%
\definecolor{currentfill}{rgb}{0.121569,0.466667,0.705882}%
\pgfsetfillcolor{currentfill}%
\pgfsetfillopacity{0.663896}%
\pgfsetlinewidth{1.003750pt}%
\definecolor{currentstroke}{rgb}{0.121569,0.466667,0.705882}%
\pgfsetstrokecolor{currentstroke}%
\pgfsetstrokeopacity{0.663896}%
\pgfsetdash{}{0pt}%
\pgfpathmoveto{\pgfqpoint{0.726904in}{1.212679in}}%
\pgfpathcurveto{\pgfqpoint{0.735140in}{1.212679in}}{\pgfqpoint{0.743040in}{1.215952in}}{\pgfqpoint{0.748864in}{1.221776in}}%
\pgfpathcurveto{\pgfqpoint{0.754688in}{1.227599in}}{\pgfqpoint{0.757960in}{1.235500in}}{\pgfqpoint{0.757960in}{1.243736in}}%
\pgfpathcurveto{\pgfqpoint{0.757960in}{1.251972in}}{\pgfqpoint{0.754688in}{1.259872in}}{\pgfqpoint{0.748864in}{1.265696in}}%
\pgfpathcurveto{\pgfqpoint{0.743040in}{1.271520in}}{\pgfqpoint{0.735140in}{1.274792in}}{\pgfqpoint{0.726904in}{1.274792in}}%
\pgfpathcurveto{\pgfqpoint{0.718667in}{1.274792in}}{\pgfqpoint{0.710767in}{1.271520in}}{\pgfqpoint{0.704943in}{1.265696in}}%
\pgfpathcurveto{\pgfqpoint{0.699119in}{1.259872in}}{\pgfqpoint{0.695847in}{1.251972in}}{\pgfqpoint{0.695847in}{1.243736in}}%
\pgfpathcurveto{\pgfqpoint{0.695847in}{1.235500in}}{\pgfqpoint{0.699119in}{1.227599in}}{\pgfqpoint{0.704943in}{1.221776in}}%
\pgfpathcurveto{\pgfqpoint{0.710767in}{1.215952in}}{\pgfqpoint{0.718667in}{1.212679in}}{\pgfqpoint{0.726904in}{1.212679in}}%
\pgfpathclose%
\pgfusepath{stroke,fill}%
\end{pgfscope}%
\begin{pgfscope}%
\pgfpathrectangle{\pgfqpoint{0.100000in}{0.212622in}}{\pgfqpoint{3.696000in}{3.696000in}}%
\pgfusepath{clip}%
\pgfsetbuttcap%
\pgfsetroundjoin%
\definecolor{currentfill}{rgb}{0.121569,0.466667,0.705882}%
\pgfsetfillcolor{currentfill}%
\pgfsetfillopacity{0.663896}%
\pgfsetlinewidth{1.003750pt}%
\definecolor{currentstroke}{rgb}{0.121569,0.466667,0.705882}%
\pgfsetstrokecolor{currentstroke}%
\pgfsetstrokeopacity{0.663896}%
\pgfsetdash{}{0pt}%
\pgfpathmoveto{\pgfqpoint{0.726904in}{1.212679in}}%
\pgfpathcurveto{\pgfqpoint{0.735140in}{1.212679in}}{\pgfqpoint{0.743040in}{1.215952in}}{\pgfqpoint{0.748864in}{1.221776in}}%
\pgfpathcurveto{\pgfqpoint{0.754688in}{1.227599in}}{\pgfqpoint{0.757960in}{1.235500in}}{\pgfqpoint{0.757960in}{1.243736in}}%
\pgfpathcurveto{\pgfqpoint{0.757960in}{1.251972in}}{\pgfqpoint{0.754688in}{1.259872in}}{\pgfqpoint{0.748864in}{1.265696in}}%
\pgfpathcurveto{\pgfqpoint{0.743040in}{1.271520in}}{\pgfqpoint{0.735140in}{1.274792in}}{\pgfqpoint{0.726904in}{1.274792in}}%
\pgfpathcurveto{\pgfqpoint{0.718667in}{1.274792in}}{\pgfqpoint{0.710767in}{1.271520in}}{\pgfqpoint{0.704943in}{1.265696in}}%
\pgfpathcurveto{\pgfqpoint{0.699119in}{1.259872in}}{\pgfqpoint{0.695847in}{1.251972in}}{\pgfqpoint{0.695847in}{1.243736in}}%
\pgfpathcurveto{\pgfqpoint{0.695847in}{1.235500in}}{\pgfqpoint{0.699119in}{1.227599in}}{\pgfqpoint{0.704943in}{1.221776in}}%
\pgfpathcurveto{\pgfqpoint{0.710767in}{1.215952in}}{\pgfqpoint{0.718667in}{1.212679in}}{\pgfqpoint{0.726904in}{1.212679in}}%
\pgfpathclose%
\pgfusepath{stroke,fill}%
\end{pgfscope}%
\begin{pgfscope}%
\pgfpathrectangle{\pgfqpoint{0.100000in}{0.212622in}}{\pgfqpoint{3.696000in}{3.696000in}}%
\pgfusepath{clip}%
\pgfsetbuttcap%
\pgfsetroundjoin%
\definecolor{currentfill}{rgb}{0.121569,0.466667,0.705882}%
\pgfsetfillcolor{currentfill}%
\pgfsetfillopacity{0.663896}%
\pgfsetlinewidth{1.003750pt}%
\definecolor{currentstroke}{rgb}{0.121569,0.466667,0.705882}%
\pgfsetstrokecolor{currentstroke}%
\pgfsetstrokeopacity{0.663896}%
\pgfsetdash{}{0pt}%
\pgfpathmoveto{\pgfqpoint{0.726904in}{1.212679in}}%
\pgfpathcurveto{\pgfqpoint{0.735140in}{1.212679in}}{\pgfqpoint{0.743040in}{1.215952in}}{\pgfqpoint{0.748864in}{1.221776in}}%
\pgfpathcurveto{\pgfqpoint{0.754688in}{1.227599in}}{\pgfqpoint{0.757960in}{1.235500in}}{\pgfqpoint{0.757960in}{1.243736in}}%
\pgfpathcurveto{\pgfqpoint{0.757960in}{1.251972in}}{\pgfqpoint{0.754688in}{1.259872in}}{\pgfqpoint{0.748864in}{1.265696in}}%
\pgfpathcurveto{\pgfqpoint{0.743040in}{1.271520in}}{\pgfqpoint{0.735140in}{1.274792in}}{\pgfqpoint{0.726904in}{1.274792in}}%
\pgfpathcurveto{\pgfqpoint{0.718667in}{1.274792in}}{\pgfqpoint{0.710767in}{1.271520in}}{\pgfqpoint{0.704943in}{1.265696in}}%
\pgfpathcurveto{\pgfqpoint{0.699119in}{1.259872in}}{\pgfqpoint{0.695847in}{1.251972in}}{\pgfqpoint{0.695847in}{1.243736in}}%
\pgfpathcurveto{\pgfqpoint{0.695847in}{1.235500in}}{\pgfqpoint{0.699119in}{1.227599in}}{\pgfqpoint{0.704943in}{1.221776in}}%
\pgfpathcurveto{\pgfqpoint{0.710767in}{1.215952in}}{\pgfqpoint{0.718667in}{1.212679in}}{\pgfqpoint{0.726904in}{1.212679in}}%
\pgfpathclose%
\pgfusepath{stroke,fill}%
\end{pgfscope}%
\begin{pgfscope}%
\pgfpathrectangle{\pgfqpoint{0.100000in}{0.212622in}}{\pgfqpoint{3.696000in}{3.696000in}}%
\pgfusepath{clip}%
\pgfsetbuttcap%
\pgfsetroundjoin%
\definecolor{currentfill}{rgb}{0.121569,0.466667,0.705882}%
\pgfsetfillcolor{currentfill}%
\pgfsetfillopacity{0.663896}%
\pgfsetlinewidth{1.003750pt}%
\definecolor{currentstroke}{rgb}{0.121569,0.466667,0.705882}%
\pgfsetstrokecolor{currentstroke}%
\pgfsetstrokeopacity{0.663896}%
\pgfsetdash{}{0pt}%
\pgfpathmoveto{\pgfqpoint{0.726903in}{1.212679in}}%
\pgfpathcurveto{\pgfqpoint{0.735140in}{1.212679in}}{\pgfqpoint{0.743040in}{1.215952in}}{\pgfqpoint{0.748864in}{1.221776in}}%
\pgfpathcurveto{\pgfqpoint{0.754688in}{1.227599in}}{\pgfqpoint{0.757960in}{1.235500in}}{\pgfqpoint{0.757960in}{1.243736in}}%
\pgfpathcurveto{\pgfqpoint{0.757960in}{1.251972in}}{\pgfqpoint{0.754688in}{1.259872in}}{\pgfqpoint{0.748864in}{1.265696in}}%
\pgfpathcurveto{\pgfqpoint{0.743040in}{1.271520in}}{\pgfqpoint{0.735140in}{1.274792in}}{\pgfqpoint{0.726903in}{1.274792in}}%
\pgfpathcurveto{\pgfqpoint{0.718667in}{1.274792in}}{\pgfqpoint{0.710767in}{1.271520in}}{\pgfqpoint{0.704943in}{1.265696in}}%
\pgfpathcurveto{\pgfqpoint{0.699119in}{1.259872in}}{\pgfqpoint{0.695847in}{1.251972in}}{\pgfqpoint{0.695847in}{1.243736in}}%
\pgfpathcurveto{\pgfqpoint{0.695847in}{1.235500in}}{\pgfqpoint{0.699119in}{1.227599in}}{\pgfqpoint{0.704943in}{1.221776in}}%
\pgfpathcurveto{\pgfqpoint{0.710767in}{1.215952in}}{\pgfqpoint{0.718667in}{1.212679in}}{\pgfqpoint{0.726903in}{1.212679in}}%
\pgfpathclose%
\pgfusepath{stroke,fill}%
\end{pgfscope}%
\begin{pgfscope}%
\pgfpathrectangle{\pgfqpoint{0.100000in}{0.212622in}}{\pgfqpoint{3.696000in}{3.696000in}}%
\pgfusepath{clip}%
\pgfsetbuttcap%
\pgfsetroundjoin%
\definecolor{currentfill}{rgb}{0.121569,0.466667,0.705882}%
\pgfsetfillcolor{currentfill}%
\pgfsetfillopacity{0.663896}%
\pgfsetlinewidth{1.003750pt}%
\definecolor{currentstroke}{rgb}{0.121569,0.466667,0.705882}%
\pgfsetstrokecolor{currentstroke}%
\pgfsetstrokeopacity{0.663896}%
\pgfsetdash{}{0pt}%
\pgfpathmoveto{\pgfqpoint{0.726903in}{1.212679in}}%
\pgfpathcurveto{\pgfqpoint{0.735140in}{1.212679in}}{\pgfqpoint{0.743040in}{1.215952in}}{\pgfqpoint{0.748864in}{1.221776in}}%
\pgfpathcurveto{\pgfqpoint{0.754688in}{1.227599in}}{\pgfqpoint{0.757960in}{1.235500in}}{\pgfqpoint{0.757960in}{1.243736in}}%
\pgfpathcurveto{\pgfqpoint{0.757960in}{1.251972in}}{\pgfqpoint{0.754688in}{1.259872in}}{\pgfqpoint{0.748864in}{1.265696in}}%
\pgfpathcurveto{\pgfqpoint{0.743040in}{1.271520in}}{\pgfqpoint{0.735140in}{1.274792in}}{\pgfqpoint{0.726903in}{1.274792in}}%
\pgfpathcurveto{\pgfqpoint{0.718667in}{1.274792in}}{\pgfqpoint{0.710767in}{1.271520in}}{\pgfqpoint{0.704943in}{1.265696in}}%
\pgfpathcurveto{\pgfqpoint{0.699119in}{1.259872in}}{\pgfqpoint{0.695847in}{1.251972in}}{\pgfqpoint{0.695847in}{1.243736in}}%
\pgfpathcurveto{\pgfqpoint{0.695847in}{1.235500in}}{\pgfqpoint{0.699119in}{1.227599in}}{\pgfqpoint{0.704943in}{1.221776in}}%
\pgfpathcurveto{\pgfqpoint{0.710767in}{1.215952in}}{\pgfqpoint{0.718667in}{1.212679in}}{\pgfqpoint{0.726903in}{1.212679in}}%
\pgfpathclose%
\pgfusepath{stroke,fill}%
\end{pgfscope}%
\begin{pgfscope}%
\pgfpathrectangle{\pgfqpoint{0.100000in}{0.212622in}}{\pgfqpoint{3.696000in}{3.696000in}}%
\pgfusepath{clip}%
\pgfsetbuttcap%
\pgfsetroundjoin%
\definecolor{currentfill}{rgb}{0.121569,0.466667,0.705882}%
\pgfsetfillcolor{currentfill}%
\pgfsetfillopacity{0.663896}%
\pgfsetlinewidth{1.003750pt}%
\definecolor{currentstroke}{rgb}{0.121569,0.466667,0.705882}%
\pgfsetstrokecolor{currentstroke}%
\pgfsetstrokeopacity{0.663896}%
\pgfsetdash{}{0pt}%
\pgfpathmoveto{\pgfqpoint{0.726903in}{1.212679in}}%
\pgfpathcurveto{\pgfqpoint{0.735140in}{1.212679in}}{\pgfqpoint{0.743040in}{1.215952in}}{\pgfqpoint{0.748864in}{1.221776in}}%
\pgfpathcurveto{\pgfqpoint{0.754688in}{1.227599in}}{\pgfqpoint{0.757960in}{1.235500in}}{\pgfqpoint{0.757960in}{1.243736in}}%
\pgfpathcurveto{\pgfqpoint{0.757960in}{1.251972in}}{\pgfqpoint{0.754688in}{1.259872in}}{\pgfqpoint{0.748864in}{1.265696in}}%
\pgfpathcurveto{\pgfqpoint{0.743040in}{1.271520in}}{\pgfqpoint{0.735140in}{1.274792in}}{\pgfqpoint{0.726903in}{1.274792in}}%
\pgfpathcurveto{\pgfqpoint{0.718667in}{1.274792in}}{\pgfqpoint{0.710767in}{1.271520in}}{\pgfqpoint{0.704943in}{1.265696in}}%
\pgfpathcurveto{\pgfqpoint{0.699119in}{1.259872in}}{\pgfqpoint{0.695847in}{1.251972in}}{\pgfqpoint{0.695847in}{1.243736in}}%
\pgfpathcurveto{\pgfqpoint{0.695847in}{1.235500in}}{\pgfqpoint{0.699119in}{1.227599in}}{\pgfqpoint{0.704943in}{1.221776in}}%
\pgfpathcurveto{\pgfqpoint{0.710767in}{1.215952in}}{\pgfqpoint{0.718667in}{1.212679in}}{\pgfqpoint{0.726903in}{1.212679in}}%
\pgfpathclose%
\pgfusepath{stroke,fill}%
\end{pgfscope}%
\begin{pgfscope}%
\pgfpathrectangle{\pgfqpoint{0.100000in}{0.212622in}}{\pgfqpoint{3.696000in}{3.696000in}}%
\pgfusepath{clip}%
\pgfsetbuttcap%
\pgfsetroundjoin%
\definecolor{currentfill}{rgb}{0.121569,0.466667,0.705882}%
\pgfsetfillcolor{currentfill}%
\pgfsetfillopacity{0.663896}%
\pgfsetlinewidth{1.003750pt}%
\definecolor{currentstroke}{rgb}{0.121569,0.466667,0.705882}%
\pgfsetstrokecolor{currentstroke}%
\pgfsetstrokeopacity{0.663896}%
\pgfsetdash{}{0pt}%
\pgfpathmoveto{\pgfqpoint{0.726903in}{1.212679in}}%
\pgfpathcurveto{\pgfqpoint{0.735140in}{1.212679in}}{\pgfqpoint{0.743040in}{1.215952in}}{\pgfqpoint{0.748864in}{1.221776in}}%
\pgfpathcurveto{\pgfqpoint{0.754688in}{1.227599in}}{\pgfqpoint{0.757960in}{1.235500in}}{\pgfqpoint{0.757960in}{1.243736in}}%
\pgfpathcurveto{\pgfqpoint{0.757960in}{1.251972in}}{\pgfqpoint{0.754688in}{1.259872in}}{\pgfqpoint{0.748864in}{1.265696in}}%
\pgfpathcurveto{\pgfqpoint{0.743040in}{1.271520in}}{\pgfqpoint{0.735140in}{1.274792in}}{\pgfqpoint{0.726903in}{1.274792in}}%
\pgfpathcurveto{\pgfqpoint{0.718667in}{1.274792in}}{\pgfqpoint{0.710767in}{1.271520in}}{\pgfqpoint{0.704943in}{1.265696in}}%
\pgfpathcurveto{\pgfqpoint{0.699119in}{1.259872in}}{\pgfqpoint{0.695847in}{1.251972in}}{\pgfqpoint{0.695847in}{1.243736in}}%
\pgfpathcurveto{\pgfqpoint{0.695847in}{1.235500in}}{\pgfqpoint{0.699119in}{1.227599in}}{\pgfqpoint{0.704943in}{1.221776in}}%
\pgfpathcurveto{\pgfqpoint{0.710767in}{1.215952in}}{\pgfqpoint{0.718667in}{1.212679in}}{\pgfqpoint{0.726903in}{1.212679in}}%
\pgfpathclose%
\pgfusepath{stroke,fill}%
\end{pgfscope}%
\begin{pgfscope}%
\pgfpathrectangle{\pgfqpoint{0.100000in}{0.212622in}}{\pgfqpoint{3.696000in}{3.696000in}}%
\pgfusepath{clip}%
\pgfsetbuttcap%
\pgfsetroundjoin%
\definecolor{currentfill}{rgb}{0.121569,0.466667,0.705882}%
\pgfsetfillcolor{currentfill}%
\pgfsetfillopacity{0.663896}%
\pgfsetlinewidth{1.003750pt}%
\definecolor{currentstroke}{rgb}{0.121569,0.466667,0.705882}%
\pgfsetstrokecolor{currentstroke}%
\pgfsetstrokeopacity{0.663896}%
\pgfsetdash{}{0pt}%
\pgfpathmoveto{\pgfqpoint{0.726903in}{1.212679in}}%
\pgfpathcurveto{\pgfqpoint{0.735140in}{1.212679in}}{\pgfqpoint{0.743040in}{1.215952in}}{\pgfqpoint{0.748864in}{1.221776in}}%
\pgfpathcurveto{\pgfqpoint{0.754688in}{1.227599in}}{\pgfqpoint{0.757960in}{1.235500in}}{\pgfqpoint{0.757960in}{1.243736in}}%
\pgfpathcurveto{\pgfqpoint{0.757960in}{1.251972in}}{\pgfqpoint{0.754688in}{1.259872in}}{\pgfqpoint{0.748864in}{1.265696in}}%
\pgfpathcurveto{\pgfqpoint{0.743040in}{1.271520in}}{\pgfqpoint{0.735140in}{1.274792in}}{\pgfqpoint{0.726903in}{1.274792in}}%
\pgfpathcurveto{\pgfqpoint{0.718667in}{1.274792in}}{\pgfqpoint{0.710767in}{1.271520in}}{\pgfqpoint{0.704943in}{1.265696in}}%
\pgfpathcurveto{\pgfqpoint{0.699119in}{1.259872in}}{\pgfqpoint{0.695847in}{1.251972in}}{\pgfqpoint{0.695847in}{1.243736in}}%
\pgfpathcurveto{\pgfqpoint{0.695847in}{1.235500in}}{\pgfqpoint{0.699119in}{1.227599in}}{\pgfqpoint{0.704943in}{1.221776in}}%
\pgfpathcurveto{\pgfqpoint{0.710767in}{1.215952in}}{\pgfqpoint{0.718667in}{1.212679in}}{\pgfqpoint{0.726903in}{1.212679in}}%
\pgfpathclose%
\pgfusepath{stroke,fill}%
\end{pgfscope}%
\begin{pgfscope}%
\pgfpathrectangle{\pgfqpoint{0.100000in}{0.212622in}}{\pgfqpoint{3.696000in}{3.696000in}}%
\pgfusepath{clip}%
\pgfsetbuttcap%
\pgfsetroundjoin%
\definecolor{currentfill}{rgb}{0.121569,0.466667,0.705882}%
\pgfsetfillcolor{currentfill}%
\pgfsetfillopacity{0.663896}%
\pgfsetlinewidth{1.003750pt}%
\definecolor{currentstroke}{rgb}{0.121569,0.466667,0.705882}%
\pgfsetstrokecolor{currentstroke}%
\pgfsetstrokeopacity{0.663896}%
\pgfsetdash{}{0pt}%
\pgfpathmoveto{\pgfqpoint{0.726903in}{1.212679in}}%
\pgfpathcurveto{\pgfqpoint{0.735140in}{1.212679in}}{\pgfqpoint{0.743040in}{1.215952in}}{\pgfqpoint{0.748864in}{1.221776in}}%
\pgfpathcurveto{\pgfqpoint{0.754688in}{1.227599in}}{\pgfqpoint{0.757960in}{1.235500in}}{\pgfqpoint{0.757960in}{1.243736in}}%
\pgfpathcurveto{\pgfqpoint{0.757960in}{1.251972in}}{\pgfqpoint{0.754688in}{1.259872in}}{\pgfqpoint{0.748864in}{1.265696in}}%
\pgfpathcurveto{\pgfqpoint{0.743040in}{1.271520in}}{\pgfqpoint{0.735140in}{1.274792in}}{\pgfqpoint{0.726903in}{1.274792in}}%
\pgfpathcurveto{\pgfqpoint{0.718667in}{1.274792in}}{\pgfqpoint{0.710767in}{1.271520in}}{\pgfqpoint{0.704943in}{1.265696in}}%
\pgfpathcurveto{\pgfqpoint{0.699119in}{1.259872in}}{\pgfqpoint{0.695847in}{1.251972in}}{\pgfqpoint{0.695847in}{1.243736in}}%
\pgfpathcurveto{\pgfqpoint{0.695847in}{1.235500in}}{\pgfqpoint{0.699119in}{1.227599in}}{\pgfqpoint{0.704943in}{1.221776in}}%
\pgfpathcurveto{\pgfqpoint{0.710767in}{1.215952in}}{\pgfqpoint{0.718667in}{1.212679in}}{\pgfqpoint{0.726903in}{1.212679in}}%
\pgfpathclose%
\pgfusepath{stroke,fill}%
\end{pgfscope}%
\begin{pgfscope}%
\pgfpathrectangle{\pgfqpoint{0.100000in}{0.212622in}}{\pgfqpoint{3.696000in}{3.696000in}}%
\pgfusepath{clip}%
\pgfsetbuttcap%
\pgfsetroundjoin%
\definecolor{currentfill}{rgb}{0.121569,0.466667,0.705882}%
\pgfsetfillcolor{currentfill}%
\pgfsetfillopacity{0.663896}%
\pgfsetlinewidth{1.003750pt}%
\definecolor{currentstroke}{rgb}{0.121569,0.466667,0.705882}%
\pgfsetstrokecolor{currentstroke}%
\pgfsetstrokeopacity{0.663896}%
\pgfsetdash{}{0pt}%
\pgfpathmoveto{\pgfqpoint{0.726903in}{1.212679in}}%
\pgfpathcurveto{\pgfqpoint{0.735140in}{1.212679in}}{\pgfqpoint{0.743040in}{1.215952in}}{\pgfqpoint{0.748864in}{1.221776in}}%
\pgfpathcurveto{\pgfqpoint{0.754688in}{1.227599in}}{\pgfqpoint{0.757960in}{1.235500in}}{\pgfqpoint{0.757960in}{1.243736in}}%
\pgfpathcurveto{\pgfqpoint{0.757960in}{1.251972in}}{\pgfqpoint{0.754688in}{1.259872in}}{\pgfqpoint{0.748864in}{1.265696in}}%
\pgfpathcurveto{\pgfqpoint{0.743040in}{1.271520in}}{\pgfqpoint{0.735140in}{1.274792in}}{\pgfqpoint{0.726903in}{1.274792in}}%
\pgfpathcurveto{\pgfqpoint{0.718667in}{1.274792in}}{\pgfqpoint{0.710767in}{1.271520in}}{\pgfqpoint{0.704943in}{1.265696in}}%
\pgfpathcurveto{\pgfqpoint{0.699119in}{1.259872in}}{\pgfqpoint{0.695847in}{1.251972in}}{\pgfqpoint{0.695847in}{1.243736in}}%
\pgfpathcurveto{\pgfqpoint{0.695847in}{1.235500in}}{\pgfqpoint{0.699119in}{1.227599in}}{\pgfqpoint{0.704943in}{1.221776in}}%
\pgfpathcurveto{\pgfqpoint{0.710767in}{1.215952in}}{\pgfqpoint{0.718667in}{1.212679in}}{\pgfqpoint{0.726903in}{1.212679in}}%
\pgfpathclose%
\pgfusepath{stroke,fill}%
\end{pgfscope}%
\begin{pgfscope}%
\pgfpathrectangle{\pgfqpoint{0.100000in}{0.212622in}}{\pgfqpoint{3.696000in}{3.696000in}}%
\pgfusepath{clip}%
\pgfsetbuttcap%
\pgfsetroundjoin%
\definecolor{currentfill}{rgb}{0.121569,0.466667,0.705882}%
\pgfsetfillcolor{currentfill}%
\pgfsetfillopacity{0.663896}%
\pgfsetlinewidth{1.003750pt}%
\definecolor{currentstroke}{rgb}{0.121569,0.466667,0.705882}%
\pgfsetstrokecolor{currentstroke}%
\pgfsetstrokeopacity{0.663896}%
\pgfsetdash{}{0pt}%
\pgfpathmoveto{\pgfqpoint{0.726903in}{1.212679in}}%
\pgfpathcurveto{\pgfqpoint{0.735140in}{1.212679in}}{\pgfqpoint{0.743040in}{1.215952in}}{\pgfqpoint{0.748864in}{1.221776in}}%
\pgfpathcurveto{\pgfqpoint{0.754688in}{1.227599in}}{\pgfqpoint{0.757960in}{1.235500in}}{\pgfqpoint{0.757960in}{1.243736in}}%
\pgfpathcurveto{\pgfqpoint{0.757960in}{1.251972in}}{\pgfqpoint{0.754688in}{1.259872in}}{\pgfqpoint{0.748864in}{1.265696in}}%
\pgfpathcurveto{\pgfqpoint{0.743040in}{1.271520in}}{\pgfqpoint{0.735140in}{1.274792in}}{\pgfqpoint{0.726903in}{1.274792in}}%
\pgfpathcurveto{\pgfqpoint{0.718667in}{1.274792in}}{\pgfqpoint{0.710767in}{1.271520in}}{\pgfqpoint{0.704943in}{1.265696in}}%
\pgfpathcurveto{\pgfqpoint{0.699119in}{1.259872in}}{\pgfqpoint{0.695847in}{1.251972in}}{\pgfqpoint{0.695847in}{1.243736in}}%
\pgfpathcurveto{\pgfqpoint{0.695847in}{1.235500in}}{\pgfqpoint{0.699119in}{1.227599in}}{\pgfqpoint{0.704943in}{1.221776in}}%
\pgfpathcurveto{\pgfqpoint{0.710767in}{1.215952in}}{\pgfqpoint{0.718667in}{1.212679in}}{\pgfqpoint{0.726903in}{1.212679in}}%
\pgfpathclose%
\pgfusepath{stroke,fill}%
\end{pgfscope}%
\begin{pgfscope}%
\pgfpathrectangle{\pgfqpoint{0.100000in}{0.212622in}}{\pgfqpoint{3.696000in}{3.696000in}}%
\pgfusepath{clip}%
\pgfsetbuttcap%
\pgfsetroundjoin%
\definecolor{currentfill}{rgb}{0.121569,0.466667,0.705882}%
\pgfsetfillcolor{currentfill}%
\pgfsetfillopacity{0.663896}%
\pgfsetlinewidth{1.003750pt}%
\definecolor{currentstroke}{rgb}{0.121569,0.466667,0.705882}%
\pgfsetstrokecolor{currentstroke}%
\pgfsetstrokeopacity{0.663896}%
\pgfsetdash{}{0pt}%
\pgfpathmoveto{\pgfqpoint{0.726903in}{1.212679in}}%
\pgfpathcurveto{\pgfqpoint{0.735140in}{1.212679in}}{\pgfqpoint{0.743040in}{1.215952in}}{\pgfqpoint{0.748864in}{1.221776in}}%
\pgfpathcurveto{\pgfqpoint{0.754688in}{1.227599in}}{\pgfqpoint{0.757960in}{1.235500in}}{\pgfqpoint{0.757960in}{1.243736in}}%
\pgfpathcurveto{\pgfqpoint{0.757960in}{1.251972in}}{\pgfqpoint{0.754688in}{1.259872in}}{\pgfqpoint{0.748864in}{1.265696in}}%
\pgfpathcurveto{\pgfqpoint{0.743040in}{1.271520in}}{\pgfqpoint{0.735140in}{1.274792in}}{\pgfqpoint{0.726903in}{1.274792in}}%
\pgfpathcurveto{\pgfqpoint{0.718667in}{1.274792in}}{\pgfqpoint{0.710767in}{1.271520in}}{\pgfqpoint{0.704943in}{1.265696in}}%
\pgfpathcurveto{\pgfqpoint{0.699119in}{1.259872in}}{\pgfqpoint{0.695847in}{1.251972in}}{\pgfqpoint{0.695847in}{1.243736in}}%
\pgfpathcurveto{\pgfqpoint{0.695847in}{1.235500in}}{\pgfqpoint{0.699119in}{1.227599in}}{\pgfqpoint{0.704943in}{1.221776in}}%
\pgfpathcurveto{\pgfqpoint{0.710767in}{1.215952in}}{\pgfqpoint{0.718667in}{1.212679in}}{\pgfqpoint{0.726903in}{1.212679in}}%
\pgfpathclose%
\pgfusepath{stroke,fill}%
\end{pgfscope}%
\begin{pgfscope}%
\pgfpathrectangle{\pgfqpoint{0.100000in}{0.212622in}}{\pgfqpoint{3.696000in}{3.696000in}}%
\pgfusepath{clip}%
\pgfsetbuttcap%
\pgfsetroundjoin%
\definecolor{currentfill}{rgb}{0.121569,0.466667,0.705882}%
\pgfsetfillcolor{currentfill}%
\pgfsetfillopacity{0.663896}%
\pgfsetlinewidth{1.003750pt}%
\definecolor{currentstroke}{rgb}{0.121569,0.466667,0.705882}%
\pgfsetstrokecolor{currentstroke}%
\pgfsetstrokeopacity{0.663896}%
\pgfsetdash{}{0pt}%
\pgfpathmoveto{\pgfqpoint{0.726903in}{1.212679in}}%
\pgfpathcurveto{\pgfqpoint{0.735140in}{1.212679in}}{\pgfqpoint{0.743040in}{1.215952in}}{\pgfqpoint{0.748864in}{1.221776in}}%
\pgfpathcurveto{\pgfqpoint{0.754688in}{1.227599in}}{\pgfqpoint{0.757960in}{1.235500in}}{\pgfqpoint{0.757960in}{1.243736in}}%
\pgfpathcurveto{\pgfqpoint{0.757960in}{1.251972in}}{\pgfqpoint{0.754688in}{1.259872in}}{\pgfqpoint{0.748864in}{1.265696in}}%
\pgfpathcurveto{\pgfqpoint{0.743040in}{1.271520in}}{\pgfqpoint{0.735140in}{1.274792in}}{\pgfqpoint{0.726903in}{1.274792in}}%
\pgfpathcurveto{\pgfqpoint{0.718667in}{1.274792in}}{\pgfqpoint{0.710767in}{1.271520in}}{\pgfqpoint{0.704943in}{1.265696in}}%
\pgfpathcurveto{\pgfqpoint{0.699119in}{1.259872in}}{\pgfqpoint{0.695847in}{1.251972in}}{\pgfqpoint{0.695847in}{1.243736in}}%
\pgfpathcurveto{\pgfqpoint{0.695847in}{1.235500in}}{\pgfqpoint{0.699119in}{1.227599in}}{\pgfqpoint{0.704943in}{1.221776in}}%
\pgfpathcurveto{\pgfqpoint{0.710767in}{1.215952in}}{\pgfqpoint{0.718667in}{1.212679in}}{\pgfqpoint{0.726903in}{1.212679in}}%
\pgfpathclose%
\pgfusepath{stroke,fill}%
\end{pgfscope}%
\begin{pgfscope}%
\pgfpathrectangle{\pgfqpoint{0.100000in}{0.212622in}}{\pgfqpoint{3.696000in}{3.696000in}}%
\pgfusepath{clip}%
\pgfsetbuttcap%
\pgfsetroundjoin%
\definecolor{currentfill}{rgb}{0.121569,0.466667,0.705882}%
\pgfsetfillcolor{currentfill}%
\pgfsetfillopacity{0.663896}%
\pgfsetlinewidth{1.003750pt}%
\definecolor{currentstroke}{rgb}{0.121569,0.466667,0.705882}%
\pgfsetstrokecolor{currentstroke}%
\pgfsetstrokeopacity{0.663896}%
\pgfsetdash{}{0pt}%
\pgfpathmoveto{\pgfqpoint{0.726903in}{1.212679in}}%
\pgfpathcurveto{\pgfqpoint{0.735140in}{1.212679in}}{\pgfqpoint{0.743040in}{1.215952in}}{\pgfqpoint{0.748864in}{1.221776in}}%
\pgfpathcurveto{\pgfqpoint{0.754688in}{1.227599in}}{\pgfqpoint{0.757960in}{1.235500in}}{\pgfqpoint{0.757960in}{1.243736in}}%
\pgfpathcurveto{\pgfqpoint{0.757960in}{1.251972in}}{\pgfqpoint{0.754688in}{1.259872in}}{\pgfqpoint{0.748864in}{1.265696in}}%
\pgfpathcurveto{\pgfqpoint{0.743040in}{1.271520in}}{\pgfqpoint{0.735140in}{1.274792in}}{\pgfqpoint{0.726903in}{1.274792in}}%
\pgfpathcurveto{\pgfqpoint{0.718667in}{1.274792in}}{\pgfqpoint{0.710767in}{1.271520in}}{\pgfqpoint{0.704943in}{1.265696in}}%
\pgfpathcurveto{\pgfqpoint{0.699119in}{1.259872in}}{\pgfqpoint{0.695847in}{1.251972in}}{\pgfqpoint{0.695847in}{1.243736in}}%
\pgfpathcurveto{\pgfqpoint{0.695847in}{1.235500in}}{\pgfqpoint{0.699119in}{1.227599in}}{\pgfqpoint{0.704943in}{1.221776in}}%
\pgfpathcurveto{\pgfqpoint{0.710767in}{1.215952in}}{\pgfqpoint{0.718667in}{1.212679in}}{\pgfqpoint{0.726903in}{1.212679in}}%
\pgfpathclose%
\pgfusepath{stroke,fill}%
\end{pgfscope}%
\begin{pgfscope}%
\pgfpathrectangle{\pgfqpoint{0.100000in}{0.212622in}}{\pgfqpoint{3.696000in}{3.696000in}}%
\pgfusepath{clip}%
\pgfsetbuttcap%
\pgfsetroundjoin%
\definecolor{currentfill}{rgb}{0.121569,0.466667,0.705882}%
\pgfsetfillcolor{currentfill}%
\pgfsetfillopacity{0.663896}%
\pgfsetlinewidth{1.003750pt}%
\definecolor{currentstroke}{rgb}{0.121569,0.466667,0.705882}%
\pgfsetstrokecolor{currentstroke}%
\pgfsetstrokeopacity{0.663896}%
\pgfsetdash{}{0pt}%
\pgfpathmoveto{\pgfqpoint{0.726903in}{1.212679in}}%
\pgfpathcurveto{\pgfqpoint{0.735140in}{1.212679in}}{\pgfqpoint{0.743040in}{1.215952in}}{\pgfqpoint{0.748864in}{1.221776in}}%
\pgfpathcurveto{\pgfqpoint{0.754688in}{1.227599in}}{\pgfqpoint{0.757960in}{1.235500in}}{\pgfqpoint{0.757960in}{1.243736in}}%
\pgfpathcurveto{\pgfqpoint{0.757960in}{1.251972in}}{\pgfqpoint{0.754688in}{1.259872in}}{\pgfqpoint{0.748864in}{1.265696in}}%
\pgfpathcurveto{\pgfqpoint{0.743040in}{1.271520in}}{\pgfqpoint{0.735140in}{1.274792in}}{\pgfqpoint{0.726903in}{1.274792in}}%
\pgfpathcurveto{\pgfqpoint{0.718667in}{1.274792in}}{\pgfqpoint{0.710767in}{1.271520in}}{\pgfqpoint{0.704943in}{1.265696in}}%
\pgfpathcurveto{\pgfqpoint{0.699119in}{1.259872in}}{\pgfqpoint{0.695847in}{1.251972in}}{\pgfqpoint{0.695847in}{1.243736in}}%
\pgfpathcurveto{\pgfqpoint{0.695847in}{1.235500in}}{\pgfqpoint{0.699119in}{1.227599in}}{\pgfqpoint{0.704943in}{1.221776in}}%
\pgfpathcurveto{\pgfqpoint{0.710767in}{1.215952in}}{\pgfqpoint{0.718667in}{1.212679in}}{\pgfqpoint{0.726903in}{1.212679in}}%
\pgfpathclose%
\pgfusepath{stroke,fill}%
\end{pgfscope}%
\begin{pgfscope}%
\pgfpathrectangle{\pgfqpoint{0.100000in}{0.212622in}}{\pgfqpoint{3.696000in}{3.696000in}}%
\pgfusepath{clip}%
\pgfsetbuttcap%
\pgfsetroundjoin%
\definecolor{currentfill}{rgb}{0.121569,0.466667,0.705882}%
\pgfsetfillcolor{currentfill}%
\pgfsetfillopacity{0.663896}%
\pgfsetlinewidth{1.003750pt}%
\definecolor{currentstroke}{rgb}{0.121569,0.466667,0.705882}%
\pgfsetstrokecolor{currentstroke}%
\pgfsetstrokeopacity{0.663896}%
\pgfsetdash{}{0pt}%
\pgfpathmoveto{\pgfqpoint{0.726903in}{1.212679in}}%
\pgfpathcurveto{\pgfqpoint{0.735140in}{1.212679in}}{\pgfqpoint{0.743040in}{1.215952in}}{\pgfqpoint{0.748864in}{1.221776in}}%
\pgfpathcurveto{\pgfqpoint{0.754688in}{1.227599in}}{\pgfqpoint{0.757960in}{1.235500in}}{\pgfqpoint{0.757960in}{1.243736in}}%
\pgfpathcurveto{\pgfqpoint{0.757960in}{1.251972in}}{\pgfqpoint{0.754688in}{1.259872in}}{\pgfqpoint{0.748864in}{1.265696in}}%
\pgfpathcurveto{\pgfqpoint{0.743040in}{1.271520in}}{\pgfqpoint{0.735140in}{1.274792in}}{\pgfqpoint{0.726903in}{1.274792in}}%
\pgfpathcurveto{\pgfqpoint{0.718667in}{1.274792in}}{\pgfqpoint{0.710767in}{1.271520in}}{\pgfqpoint{0.704943in}{1.265696in}}%
\pgfpathcurveto{\pgfqpoint{0.699119in}{1.259872in}}{\pgfqpoint{0.695847in}{1.251972in}}{\pgfqpoint{0.695847in}{1.243736in}}%
\pgfpathcurveto{\pgfqpoint{0.695847in}{1.235500in}}{\pgfqpoint{0.699119in}{1.227599in}}{\pgfqpoint{0.704943in}{1.221776in}}%
\pgfpathcurveto{\pgfqpoint{0.710767in}{1.215952in}}{\pgfqpoint{0.718667in}{1.212679in}}{\pgfqpoint{0.726903in}{1.212679in}}%
\pgfpathclose%
\pgfusepath{stroke,fill}%
\end{pgfscope}%
\begin{pgfscope}%
\pgfpathrectangle{\pgfqpoint{0.100000in}{0.212622in}}{\pgfqpoint{3.696000in}{3.696000in}}%
\pgfusepath{clip}%
\pgfsetbuttcap%
\pgfsetroundjoin%
\definecolor{currentfill}{rgb}{0.121569,0.466667,0.705882}%
\pgfsetfillcolor{currentfill}%
\pgfsetfillopacity{0.663896}%
\pgfsetlinewidth{1.003750pt}%
\definecolor{currentstroke}{rgb}{0.121569,0.466667,0.705882}%
\pgfsetstrokecolor{currentstroke}%
\pgfsetstrokeopacity{0.663896}%
\pgfsetdash{}{0pt}%
\pgfpathmoveto{\pgfqpoint{0.726903in}{1.212679in}}%
\pgfpathcurveto{\pgfqpoint{0.735140in}{1.212679in}}{\pgfqpoint{0.743040in}{1.215952in}}{\pgfqpoint{0.748864in}{1.221776in}}%
\pgfpathcurveto{\pgfqpoint{0.754688in}{1.227599in}}{\pgfqpoint{0.757960in}{1.235500in}}{\pgfqpoint{0.757960in}{1.243736in}}%
\pgfpathcurveto{\pgfqpoint{0.757960in}{1.251972in}}{\pgfqpoint{0.754688in}{1.259872in}}{\pgfqpoint{0.748864in}{1.265696in}}%
\pgfpathcurveto{\pgfqpoint{0.743040in}{1.271520in}}{\pgfqpoint{0.735140in}{1.274792in}}{\pgfqpoint{0.726903in}{1.274792in}}%
\pgfpathcurveto{\pgfqpoint{0.718667in}{1.274792in}}{\pgfqpoint{0.710767in}{1.271520in}}{\pgfqpoint{0.704943in}{1.265696in}}%
\pgfpathcurveto{\pgfqpoint{0.699119in}{1.259872in}}{\pgfqpoint{0.695847in}{1.251972in}}{\pgfqpoint{0.695847in}{1.243736in}}%
\pgfpathcurveto{\pgfqpoint{0.695847in}{1.235500in}}{\pgfqpoint{0.699119in}{1.227599in}}{\pgfqpoint{0.704943in}{1.221776in}}%
\pgfpathcurveto{\pgfqpoint{0.710767in}{1.215952in}}{\pgfqpoint{0.718667in}{1.212679in}}{\pgfqpoint{0.726903in}{1.212679in}}%
\pgfpathclose%
\pgfusepath{stroke,fill}%
\end{pgfscope}%
\begin{pgfscope}%
\pgfpathrectangle{\pgfqpoint{0.100000in}{0.212622in}}{\pgfqpoint{3.696000in}{3.696000in}}%
\pgfusepath{clip}%
\pgfsetbuttcap%
\pgfsetroundjoin%
\definecolor{currentfill}{rgb}{0.121569,0.466667,0.705882}%
\pgfsetfillcolor{currentfill}%
\pgfsetfillopacity{0.663896}%
\pgfsetlinewidth{1.003750pt}%
\definecolor{currentstroke}{rgb}{0.121569,0.466667,0.705882}%
\pgfsetstrokecolor{currentstroke}%
\pgfsetstrokeopacity{0.663896}%
\pgfsetdash{}{0pt}%
\pgfpathmoveto{\pgfqpoint{0.726903in}{1.212679in}}%
\pgfpathcurveto{\pgfqpoint{0.735140in}{1.212679in}}{\pgfqpoint{0.743040in}{1.215952in}}{\pgfqpoint{0.748864in}{1.221776in}}%
\pgfpathcurveto{\pgfqpoint{0.754688in}{1.227599in}}{\pgfqpoint{0.757960in}{1.235500in}}{\pgfqpoint{0.757960in}{1.243736in}}%
\pgfpathcurveto{\pgfqpoint{0.757960in}{1.251972in}}{\pgfqpoint{0.754688in}{1.259872in}}{\pgfqpoint{0.748864in}{1.265696in}}%
\pgfpathcurveto{\pgfqpoint{0.743040in}{1.271520in}}{\pgfqpoint{0.735140in}{1.274792in}}{\pgfqpoint{0.726903in}{1.274792in}}%
\pgfpathcurveto{\pgfqpoint{0.718667in}{1.274792in}}{\pgfqpoint{0.710767in}{1.271520in}}{\pgfqpoint{0.704943in}{1.265696in}}%
\pgfpathcurveto{\pgfqpoint{0.699119in}{1.259872in}}{\pgfqpoint{0.695847in}{1.251972in}}{\pgfqpoint{0.695847in}{1.243736in}}%
\pgfpathcurveto{\pgfqpoint{0.695847in}{1.235500in}}{\pgfqpoint{0.699119in}{1.227599in}}{\pgfqpoint{0.704943in}{1.221776in}}%
\pgfpathcurveto{\pgfqpoint{0.710767in}{1.215952in}}{\pgfqpoint{0.718667in}{1.212679in}}{\pgfqpoint{0.726903in}{1.212679in}}%
\pgfpathclose%
\pgfusepath{stroke,fill}%
\end{pgfscope}%
\begin{pgfscope}%
\pgfpathrectangle{\pgfqpoint{0.100000in}{0.212622in}}{\pgfqpoint{3.696000in}{3.696000in}}%
\pgfusepath{clip}%
\pgfsetbuttcap%
\pgfsetroundjoin%
\definecolor{currentfill}{rgb}{0.121569,0.466667,0.705882}%
\pgfsetfillcolor{currentfill}%
\pgfsetfillopacity{0.663896}%
\pgfsetlinewidth{1.003750pt}%
\definecolor{currentstroke}{rgb}{0.121569,0.466667,0.705882}%
\pgfsetstrokecolor{currentstroke}%
\pgfsetstrokeopacity{0.663896}%
\pgfsetdash{}{0pt}%
\pgfpathmoveto{\pgfqpoint{0.726903in}{1.212679in}}%
\pgfpathcurveto{\pgfqpoint{0.735140in}{1.212679in}}{\pgfqpoint{0.743040in}{1.215952in}}{\pgfqpoint{0.748864in}{1.221776in}}%
\pgfpathcurveto{\pgfqpoint{0.754688in}{1.227599in}}{\pgfqpoint{0.757960in}{1.235500in}}{\pgfqpoint{0.757960in}{1.243736in}}%
\pgfpathcurveto{\pgfqpoint{0.757960in}{1.251972in}}{\pgfqpoint{0.754688in}{1.259872in}}{\pgfqpoint{0.748864in}{1.265696in}}%
\pgfpathcurveto{\pgfqpoint{0.743040in}{1.271520in}}{\pgfqpoint{0.735140in}{1.274792in}}{\pgfqpoint{0.726903in}{1.274792in}}%
\pgfpathcurveto{\pgfqpoint{0.718667in}{1.274792in}}{\pgfqpoint{0.710767in}{1.271520in}}{\pgfqpoint{0.704943in}{1.265696in}}%
\pgfpathcurveto{\pgfqpoint{0.699119in}{1.259872in}}{\pgfqpoint{0.695847in}{1.251972in}}{\pgfqpoint{0.695847in}{1.243736in}}%
\pgfpathcurveto{\pgfqpoint{0.695847in}{1.235500in}}{\pgfqpoint{0.699119in}{1.227599in}}{\pgfqpoint{0.704943in}{1.221776in}}%
\pgfpathcurveto{\pgfqpoint{0.710767in}{1.215952in}}{\pgfqpoint{0.718667in}{1.212679in}}{\pgfqpoint{0.726903in}{1.212679in}}%
\pgfpathclose%
\pgfusepath{stroke,fill}%
\end{pgfscope}%
\begin{pgfscope}%
\pgfpathrectangle{\pgfqpoint{0.100000in}{0.212622in}}{\pgfqpoint{3.696000in}{3.696000in}}%
\pgfusepath{clip}%
\pgfsetbuttcap%
\pgfsetroundjoin%
\definecolor{currentfill}{rgb}{0.121569,0.466667,0.705882}%
\pgfsetfillcolor{currentfill}%
\pgfsetfillopacity{0.663896}%
\pgfsetlinewidth{1.003750pt}%
\definecolor{currentstroke}{rgb}{0.121569,0.466667,0.705882}%
\pgfsetstrokecolor{currentstroke}%
\pgfsetstrokeopacity{0.663896}%
\pgfsetdash{}{0pt}%
\pgfpathmoveto{\pgfqpoint{0.726903in}{1.212679in}}%
\pgfpathcurveto{\pgfqpoint{0.735140in}{1.212679in}}{\pgfqpoint{0.743040in}{1.215952in}}{\pgfqpoint{0.748864in}{1.221776in}}%
\pgfpathcurveto{\pgfqpoint{0.754688in}{1.227599in}}{\pgfqpoint{0.757960in}{1.235500in}}{\pgfqpoint{0.757960in}{1.243736in}}%
\pgfpathcurveto{\pgfqpoint{0.757960in}{1.251972in}}{\pgfqpoint{0.754688in}{1.259872in}}{\pgfqpoint{0.748864in}{1.265696in}}%
\pgfpathcurveto{\pgfqpoint{0.743040in}{1.271520in}}{\pgfqpoint{0.735140in}{1.274792in}}{\pgfqpoint{0.726903in}{1.274792in}}%
\pgfpathcurveto{\pgfqpoint{0.718667in}{1.274792in}}{\pgfqpoint{0.710767in}{1.271520in}}{\pgfqpoint{0.704943in}{1.265696in}}%
\pgfpathcurveto{\pgfqpoint{0.699119in}{1.259872in}}{\pgfqpoint{0.695847in}{1.251972in}}{\pgfqpoint{0.695847in}{1.243736in}}%
\pgfpathcurveto{\pgfqpoint{0.695847in}{1.235500in}}{\pgfqpoint{0.699119in}{1.227599in}}{\pgfqpoint{0.704943in}{1.221776in}}%
\pgfpathcurveto{\pgfqpoint{0.710767in}{1.215952in}}{\pgfqpoint{0.718667in}{1.212679in}}{\pgfqpoint{0.726903in}{1.212679in}}%
\pgfpathclose%
\pgfusepath{stroke,fill}%
\end{pgfscope}%
\begin{pgfscope}%
\pgfpathrectangle{\pgfqpoint{0.100000in}{0.212622in}}{\pgfqpoint{3.696000in}{3.696000in}}%
\pgfusepath{clip}%
\pgfsetbuttcap%
\pgfsetroundjoin%
\definecolor{currentfill}{rgb}{0.121569,0.466667,0.705882}%
\pgfsetfillcolor{currentfill}%
\pgfsetfillopacity{0.663896}%
\pgfsetlinewidth{1.003750pt}%
\definecolor{currentstroke}{rgb}{0.121569,0.466667,0.705882}%
\pgfsetstrokecolor{currentstroke}%
\pgfsetstrokeopacity{0.663896}%
\pgfsetdash{}{0pt}%
\pgfpathmoveto{\pgfqpoint{0.726903in}{1.212679in}}%
\pgfpathcurveto{\pgfqpoint{0.735140in}{1.212679in}}{\pgfqpoint{0.743040in}{1.215952in}}{\pgfqpoint{0.748864in}{1.221776in}}%
\pgfpathcurveto{\pgfqpoint{0.754688in}{1.227599in}}{\pgfqpoint{0.757960in}{1.235500in}}{\pgfqpoint{0.757960in}{1.243736in}}%
\pgfpathcurveto{\pgfqpoint{0.757960in}{1.251972in}}{\pgfqpoint{0.754688in}{1.259872in}}{\pgfqpoint{0.748864in}{1.265696in}}%
\pgfpathcurveto{\pgfqpoint{0.743040in}{1.271520in}}{\pgfqpoint{0.735140in}{1.274792in}}{\pgfqpoint{0.726903in}{1.274792in}}%
\pgfpathcurveto{\pgfqpoint{0.718667in}{1.274792in}}{\pgfqpoint{0.710767in}{1.271520in}}{\pgfqpoint{0.704943in}{1.265696in}}%
\pgfpathcurveto{\pgfqpoint{0.699119in}{1.259872in}}{\pgfqpoint{0.695847in}{1.251972in}}{\pgfqpoint{0.695847in}{1.243736in}}%
\pgfpathcurveto{\pgfqpoint{0.695847in}{1.235500in}}{\pgfqpoint{0.699119in}{1.227599in}}{\pgfqpoint{0.704943in}{1.221776in}}%
\pgfpathcurveto{\pgfqpoint{0.710767in}{1.215952in}}{\pgfqpoint{0.718667in}{1.212679in}}{\pgfqpoint{0.726903in}{1.212679in}}%
\pgfpathclose%
\pgfusepath{stroke,fill}%
\end{pgfscope}%
\begin{pgfscope}%
\pgfpathrectangle{\pgfqpoint{0.100000in}{0.212622in}}{\pgfqpoint{3.696000in}{3.696000in}}%
\pgfusepath{clip}%
\pgfsetbuttcap%
\pgfsetroundjoin%
\definecolor{currentfill}{rgb}{0.121569,0.466667,0.705882}%
\pgfsetfillcolor{currentfill}%
\pgfsetfillopacity{0.663896}%
\pgfsetlinewidth{1.003750pt}%
\definecolor{currentstroke}{rgb}{0.121569,0.466667,0.705882}%
\pgfsetstrokecolor{currentstroke}%
\pgfsetstrokeopacity{0.663896}%
\pgfsetdash{}{0pt}%
\pgfpathmoveto{\pgfqpoint{0.726903in}{1.212679in}}%
\pgfpathcurveto{\pgfqpoint{0.735140in}{1.212679in}}{\pgfqpoint{0.743040in}{1.215952in}}{\pgfqpoint{0.748864in}{1.221776in}}%
\pgfpathcurveto{\pgfqpoint{0.754688in}{1.227599in}}{\pgfqpoint{0.757960in}{1.235500in}}{\pgfqpoint{0.757960in}{1.243736in}}%
\pgfpathcurveto{\pgfqpoint{0.757960in}{1.251972in}}{\pgfqpoint{0.754688in}{1.259872in}}{\pgfqpoint{0.748864in}{1.265696in}}%
\pgfpathcurveto{\pgfqpoint{0.743040in}{1.271520in}}{\pgfqpoint{0.735140in}{1.274792in}}{\pgfqpoint{0.726903in}{1.274792in}}%
\pgfpathcurveto{\pgfqpoint{0.718667in}{1.274792in}}{\pgfqpoint{0.710767in}{1.271520in}}{\pgfqpoint{0.704943in}{1.265696in}}%
\pgfpathcurveto{\pgfqpoint{0.699119in}{1.259872in}}{\pgfqpoint{0.695847in}{1.251972in}}{\pgfqpoint{0.695847in}{1.243736in}}%
\pgfpathcurveto{\pgfqpoint{0.695847in}{1.235500in}}{\pgfqpoint{0.699119in}{1.227599in}}{\pgfqpoint{0.704943in}{1.221776in}}%
\pgfpathcurveto{\pgfqpoint{0.710767in}{1.215952in}}{\pgfqpoint{0.718667in}{1.212679in}}{\pgfqpoint{0.726903in}{1.212679in}}%
\pgfpathclose%
\pgfusepath{stroke,fill}%
\end{pgfscope}%
\begin{pgfscope}%
\pgfpathrectangle{\pgfqpoint{0.100000in}{0.212622in}}{\pgfqpoint{3.696000in}{3.696000in}}%
\pgfusepath{clip}%
\pgfsetbuttcap%
\pgfsetroundjoin%
\definecolor{currentfill}{rgb}{0.121569,0.466667,0.705882}%
\pgfsetfillcolor{currentfill}%
\pgfsetfillopacity{0.663896}%
\pgfsetlinewidth{1.003750pt}%
\definecolor{currentstroke}{rgb}{0.121569,0.466667,0.705882}%
\pgfsetstrokecolor{currentstroke}%
\pgfsetstrokeopacity{0.663896}%
\pgfsetdash{}{0pt}%
\pgfpathmoveto{\pgfqpoint{0.726903in}{1.212679in}}%
\pgfpathcurveto{\pgfqpoint{0.735140in}{1.212679in}}{\pgfqpoint{0.743040in}{1.215952in}}{\pgfqpoint{0.748864in}{1.221776in}}%
\pgfpathcurveto{\pgfqpoint{0.754688in}{1.227599in}}{\pgfqpoint{0.757960in}{1.235500in}}{\pgfqpoint{0.757960in}{1.243736in}}%
\pgfpathcurveto{\pgfqpoint{0.757960in}{1.251972in}}{\pgfqpoint{0.754688in}{1.259872in}}{\pgfqpoint{0.748864in}{1.265696in}}%
\pgfpathcurveto{\pgfqpoint{0.743040in}{1.271520in}}{\pgfqpoint{0.735140in}{1.274792in}}{\pgfqpoint{0.726903in}{1.274792in}}%
\pgfpathcurveto{\pgfqpoint{0.718667in}{1.274792in}}{\pgfqpoint{0.710767in}{1.271520in}}{\pgfqpoint{0.704943in}{1.265696in}}%
\pgfpathcurveto{\pgfqpoint{0.699119in}{1.259872in}}{\pgfqpoint{0.695847in}{1.251972in}}{\pgfqpoint{0.695847in}{1.243736in}}%
\pgfpathcurveto{\pgfqpoint{0.695847in}{1.235500in}}{\pgfqpoint{0.699119in}{1.227599in}}{\pgfqpoint{0.704943in}{1.221776in}}%
\pgfpathcurveto{\pgfqpoint{0.710767in}{1.215952in}}{\pgfqpoint{0.718667in}{1.212679in}}{\pgfqpoint{0.726903in}{1.212679in}}%
\pgfpathclose%
\pgfusepath{stroke,fill}%
\end{pgfscope}%
\begin{pgfscope}%
\pgfpathrectangle{\pgfqpoint{0.100000in}{0.212622in}}{\pgfqpoint{3.696000in}{3.696000in}}%
\pgfusepath{clip}%
\pgfsetbuttcap%
\pgfsetroundjoin%
\definecolor{currentfill}{rgb}{0.121569,0.466667,0.705882}%
\pgfsetfillcolor{currentfill}%
\pgfsetfillopacity{0.663896}%
\pgfsetlinewidth{1.003750pt}%
\definecolor{currentstroke}{rgb}{0.121569,0.466667,0.705882}%
\pgfsetstrokecolor{currentstroke}%
\pgfsetstrokeopacity{0.663896}%
\pgfsetdash{}{0pt}%
\pgfpathmoveto{\pgfqpoint{0.726903in}{1.212679in}}%
\pgfpathcurveto{\pgfqpoint{0.735140in}{1.212679in}}{\pgfqpoint{0.743040in}{1.215952in}}{\pgfqpoint{0.748864in}{1.221776in}}%
\pgfpathcurveto{\pgfqpoint{0.754688in}{1.227599in}}{\pgfqpoint{0.757960in}{1.235500in}}{\pgfqpoint{0.757960in}{1.243736in}}%
\pgfpathcurveto{\pgfqpoint{0.757960in}{1.251972in}}{\pgfqpoint{0.754688in}{1.259872in}}{\pgfqpoint{0.748864in}{1.265696in}}%
\pgfpathcurveto{\pgfqpoint{0.743040in}{1.271520in}}{\pgfqpoint{0.735140in}{1.274792in}}{\pgfqpoint{0.726903in}{1.274792in}}%
\pgfpathcurveto{\pgfqpoint{0.718667in}{1.274792in}}{\pgfqpoint{0.710767in}{1.271520in}}{\pgfqpoint{0.704943in}{1.265696in}}%
\pgfpathcurveto{\pgfqpoint{0.699119in}{1.259872in}}{\pgfqpoint{0.695847in}{1.251972in}}{\pgfqpoint{0.695847in}{1.243736in}}%
\pgfpathcurveto{\pgfqpoint{0.695847in}{1.235500in}}{\pgfqpoint{0.699119in}{1.227599in}}{\pgfqpoint{0.704943in}{1.221776in}}%
\pgfpathcurveto{\pgfqpoint{0.710767in}{1.215952in}}{\pgfqpoint{0.718667in}{1.212679in}}{\pgfqpoint{0.726903in}{1.212679in}}%
\pgfpathclose%
\pgfusepath{stroke,fill}%
\end{pgfscope}%
\begin{pgfscope}%
\pgfpathrectangle{\pgfqpoint{0.100000in}{0.212622in}}{\pgfqpoint{3.696000in}{3.696000in}}%
\pgfusepath{clip}%
\pgfsetbuttcap%
\pgfsetroundjoin%
\definecolor{currentfill}{rgb}{0.121569,0.466667,0.705882}%
\pgfsetfillcolor{currentfill}%
\pgfsetfillopacity{0.663896}%
\pgfsetlinewidth{1.003750pt}%
\definecolor{currentstroke}{rgb}{0.121569,0.466667,0.705882}%
\pgfsetstrokecolor{currentstroke}%
\pgfsetstrokeopacity{0.663896}%
\pgfsetdash{}{0pt}%
\pgfpathmoveto{\pgfqpoint{0.726903in}{1.212679in}}%
\pgfpathcurveto{\pgfqpoint{0.735140in}{1.212679in}}{\pgfqpoint{0.743040in}{1.215952in}}{\pgfqpoint{0.748864in}{1.221776in}}%
\pgfpathcurveto{\pgfqpoint{0.754688in}{1.227599in}}{\pgfqpoint{0.757960in}{1.235500in}}{\pgfqpoint{0.757960in}{1.243736in}}%
\pgfpathcurveto{\pgfqpoint{0.757960in}{1.251972in}}{\pgfqpoint{0.754688in}{1.259872in}}{\pgfqpoint{0.748864in}{1.265696in}}%
\pgfpathcurveto{\pgfqpoint{0.743040in}{1.271520in}}{\pgfqpoint{0.735140in}{1.274792in}}{\pgfqpoint{0.726903in}{1.274792in}}%
\pgfpathcurveto{\pgfqpoint{0.718667in}{1.274792in}}{\pgfqpoint{0.710767in}{1.271520in}}{\pgfqpoint{0.704943in}{1.265696in}}%
\pgfpathcurveto{\pgfqpoint{0.699119in}{1.259872in}}{\pgfqpoint{0.695847in}{1.251972in}}{\pgfqpoint{0.695847in}{1.243736in}}%
\pgfpathcurveto{\pgfqpoint{0.695847in}{1.235500in}}{\pgfqpoint{0.699119in}{1.227599in}}{\pgfqpoint{0.704943in}{1.221776in}}%
\pgfpathcurveto{\pgfqpoint{0.710767in}{1.215952in}}{\pgfqpoint{0.718667in}{1.212679in}}{\pgfqpoint{0.726903in}{1.212679in}}%
\pgfpathclose%
\pgfusepath{stroke,fill}%
\end{pgfscope}%
\begin{pgfscope}%
\pgfpathrectangle{\pgfqpoint{0.100000in}{0.212622in}}{\pgfqpoint{3.696000in}{3.696000in}}%
\pgfusepath{clip}%
\pgfsetbuttcap%
\pgfsetroundjoin%
\definecolor{currentfill}{rgb}{0.121569,0.466667,0.705882}%
\pgfsetfillcolor{currentfill}%
\pgfsetfillopacity{0.663896}%
\pgfsetlinewidth{1.003750pt}%
\definecolor{currentstroke}{rgb}{0.121569,0.466667,0.705882}%
\pgfsetstrokecolor{currentstroke}%
\pgfsetstrokeopacity{0.663896}%
\pgfsetdash{}{0pt}%
\pgfpathmoveto{\pgfqpoint{0.726903in}{1.212679in}}%
\pgfpathcurveto{\pgfqpoint{0.735140in}{1.212679in}}{\pgfqpoint{0.743040in}{1.215952in}}{\pgfqpoint{0.748864in}{1.221776in}}%
\pgfpathcurveto{\pgfqpoint{0.754688in}{1.227599in}}{\pgfqpoint{0.757960in}{1.235500in}}{\pgfqpoint{0.757960in}{1.243736in}}%
\pgfpathcurveto{\pgfqpoint{0.757960in}{1.251972in}}{\pgfqpoint{0.754688in}{1.259872in}}{\pgfqpoint{0.748864in}{1.265696in}}%
\pgfpathcurveto{\pgfqpoint{0.743040in}{1.271520in}}{\pgfqpoint{0.735140in}{1.274792in}}{\pgfqpoint{0.726903in}{1.274792in}}%
\pgfpathcurveto{\pgfqpoint{0.718667in}{1.274792in}}{\pgfqpoint{0.710767in}{1.271520in}}{\pgfqpoint{0.704943in}{1.265696in}}%
\pgfpathcurveto{\pgfqpoint{0.699119in}{1.259872in}}{\pgfqpoint{0.695847in}{1.251972in}}{\pgfqpoint{0.695847in}{1.243736in}}%
\pgfpathcurveto{\pgfqpoint{0.695847in}{1.235500in}}{\pgfqpoint{0.699119in}{1.227599in}}{\pgfqpoint{0.704943in}{1.221776in}}%
\pgfpathcurveto{\pgfqpoint{0.710767in}{1.215952in}}{\pgfqpoint{0.718667in}{1.212679in}}{\pgfqpoint{0.726903in}{1.212679in}}%
\pgfpathclose%
\pgfusepath{stroke,fill}%
\end{pgfscope}%
\begin{pgfscope}%
\pgfpathrectangle{\pgfqpoint{0.100000in}{0.212622in}}{\pgfqpoint{3.696000in}{3.696000in}}%
\pgfusepath{clip}%
\pgfsetbuttcap%
\pgfsetroundjoin%
\definecolor{currentfill}{rgb}{0.121569,0.466667,0.705882}%
\pgfsetfillcolor{currentfill}%
\pgfsetfillopacity{0.663896}%
\pgfsetlinewidth{1.003750pt}%
\definecolor{currentstroke}{rgb}{0.121569,0.466667,0.705882}%
\pgfsetstrokecolor{currentstroke}%
\pgfsetstrokeopacity{0.663896}%
\pgfsetdash{}{0pt}%
\pgfpathmoveto{\pgfqpoint{0.726903in}{1.212679in}}%
\pgfpathcurveto{\pgfqpoint{0.735140in}{1.212679in}}{\pgfqpoint{0.743040in}{1.215952in}}{\pgfqpoint{0.748864in}{1.221776in}}%
\pgfpathcurveto{\pgfqpoint{0.754688in}{1.227599in}}{\pgfqpoint{0.757960in}{1.235500in}}{\pgfqpoint{0.757960in}{1.243736in}}%
\pgfpathcurveto{\pgfqpoint{0.757960in}{1.251972in}}{\pgfqpoint{0.754688in}{1.259872in}}{\pgfqpoint{0.748864in}{1.265696in}}%
\pgfpathcurveto{\pgfqpoint{0.743040in}{1.271520in}}{\pgfqpoint{0.735140in}{1.274792in}}{\pgfqpoint{0.726903in}{1.274792in}}%
\pgfpathcurveto{\pgfqpoint{0.718667in}{1.274792in}}{\pgfqpoint{0.710767in}{1.271520in}}{\pgfqpoint{0.704943in}{1.265696in}}%
\pgfpathcurveto{\pgfqpoint{0.699119in}{1.259872in}}{\pgfqpoint{0.695847in}{1.251972in}}{\pgfqpoint{0.695847in}{1.243736in}}%
\pgfpathcurveto{\pgfqpoint{0.695847in}{1.235500in}}{\pgfqpoint{0.699119in}{1.227599in}}{\pgfqpoint{0.704943in}{1.221776in}}%
\pgfpathcurveto{\pgfqpoint{0.710767in}{1.215952in}}{\pgfqpoint{0.718667in}{1.212679in}}{\pgfqpoint{0.726903in}{1.212679in}}%
\pgfpathclose%
\pgfusepath{stroke,fill}%
\end{pgfscope}%
\begin{pgfscope}%
\pgfpathrectangle{\pgfqpoint{0.100000in}{0.212622in}}{\pgfqpoint{3.696000in}{3.696000in}}%
\pgfusepath{clip}%
\pgfsetbuttcap%
\pgfsetroundjoin%
\definecolor{currentfill}{rgb}{0.121569,0.466667,0.705882}%
\pgfsetfillcolor{currentfill}%
\pgfsetfillopacity{0.663896}%
\pgfsetlinewidth{1.003750pt}%
\definecolor{currentstroke}{rgb}{0.121569,0.466667,0.705882}%
\pgfsetstrokecolor{currentstroke}%
\pgfsetstrokeopacity{0.663896}%
\pgfsetdash{}{0pt}%
\pgfpathmoveto{\pgfqpoint{0.726903in}{1.212679in}}%
\pgfpathcurveto{\pgfqpoint{0.735140in}{1.212679in}}{\pgfqpoint{0.743040in}{1.215952in}}{\pgfqpoint{0.748864in}{1.221776in}}%
\pgfpathcurveto{\pgfqpoint{0.754688in}{1.227599in}}{\pgfqpoint{0.757960in}{1.235500in}}{\pgfqpoint{0.757960in}{1.243736in}}%
\pgfpathcurveto{\pgfqpoint{0.757960in}{1.251972in}}{\pgfqpoint{0.754688in}{1.259872in}}{\pgfqpoint{0.748864in}{1.265696in}}%
\pgfpathcurveto{\pgfqpoint{0.743040in}{1.271520in}}{\pgfqpoint{0.735140in}{1.274792in}}{\pgfqpoint{0.726903in}{1.274792in}}%
\pgfpathcurveto{\pgfqpoint{0.718667in}{1.274792in}}{\pgfqpoint{0.710767in}{1.271520in}}{\pgfqpoint{0.704943in}{1.265696in}}%
\pgfpathcurveto{\pgfqpoint{0.699119in}{1.259872in}}{\pgfqpoint{0.695847in}{1.251972in}}{\pgfqpoint{0.695847in}{1.243736in}}%
\pgfpathcurveto{\pgfqpoint{0.695847in}{1.235500in}}{\pgfqpoint{0.699119in}{1.227599in}}{\pgfqpoint{0.704943in}{1.221776in}}%
\pgfpathcurveto{\pgfqpoint{0.710767in}{1.215952in}}{\pgfqpoint{0.718667in}{1.212679in}}{\pgfqpoint{0.726903in}{1.212679in}}%
\pgfpathclose%
\pgfusepath{stroke,fill}%
\end{pgfscope}%
\begin{pgfscope}%
\pgfpathrectangle{\pgfqpoint{0.100000in}{0.212622in}}{\pgfqpoint{3.696000in}{3.696000in}}%
\pgfusepath{clip}%
\pgfsetbuttcap%
\pgfsetroundjoin%
\definecolor{currentfill}{rgb}{0.121569,0.466667,0.705882}%
\pgfsetfillcolor{currentfill}%
\pgfsetfillopacity{0.663896}%
\pgfsetlinewidth{1.003750pt}%
\definecolor{currentstroke}{rgb}{0.121569,0.466667,0.705882}%
\pgfsetstrokecolor{currentstroke}%
\pgfsetstrokeopacity{0.663896}%
\pgfsetdash{}{0pt}%
\pgfpathmoveto{\pgfqpoint{0.726903in}{1.212679in}}%
\pgfpathcurveto{\pgfqpoint{0.735140in}{1.212679in}}{\pgfqpoint{0.743040in}{1.215952in}}{\pgfqpoint{0.748864in}{1.221776in}}%
\pgfpathcurveto{\pgfqpoint{0.754688in}{1.227599in}}{\pgfqpoint{0.757960in}{1.235500in}}{\pgfqpoint{0.757960in}{1.243736in}}%
\pgfpathcurveto{\pgfqpoint{0.757960in}{1.251972in}}{\pgfqpoint{0.754688in}{1.259872in}}{\pgfqpoint{0.748864in}{1.265696in}}%
\pgfpathcurveto{\pgfqpoint{0.743040in}{1.271520in}}{\pgfqpoint{0.735140in}{1.274792in}}{\pgfqpoint{0.726903in}{1.274792in}}%
\pgfpathcurveto{\pgfqpoint{0.718667in}{1.274792in}}{\pgfqpoint{0.710767in}{1.271520in}}{\pgfqpoint{0.704943in}{1.265696in}}%
\pgfpathcurveto{\pgfqpoint{0.699119in}{1.259872in}}{\pgfqpoint{0.695847in}{1.251972in}}{\pgfqpoint{0.695847in}{1.243736in}}%
\pgfpathcurveto{\pgfqpoint{0.695847in}{1.235500in}}{\pgfqpoint{0.699119in}{1.227599in}}{\pgfqpoint{0.704943in}{1.221776in}}%
\pgfpathcurveto{\pgfqpoint{0.710767in}{1.215952in}}{\pgfqpoint{0.718667in}{1.212679in}}{\pgfqpoint{0.726903in}{1.212679in}}%
\pgfpathclose%
\pgfusepath{stroke,fill}%
\end{pgfscope}%
\begin{pgfscope}%
\pgfpathrectangle{\pgfqpoint{0.100000in}{0.212622in}}{\pgfqpoint{3.696000in}{3.696000in}}%
\pgfusepath{clip}%
\pgfsetbuttcap%
\pgfsetroundjoin%
\definecolor{currentfill}{rgb}{0.121569,0.466667,0.705882}%
\pgfsetfillcolor{currentfill}%
\pgfsetfillopacity{0.663896}%
\pgfsetlinewidth{1.003750pt}%
\definecolor{currentstroke}{rgb}{0.121569,0.466667,0.705882}%
\pgfsetstrokecolor{currentstroke}%
\pgfsetstrokeopacity{0.663896}%
\pgfsetdash{}{0pt}%
\pgfpathmoveto{\pgfqpoint{0.726903in}{1.212679in}}%
\pgfpathcurveto{\pgfqpoint{0.735140in}{1.212679in}}{\pgfqpoint{0.743040in}{1.215952in}}{\pgfqpoint{0.748864in}{1.221776in}}%
\pgfpathcurveto{\pgfqpoint{0.754688in}{1.227599in}}{\pgfqpoint{0.757960in}{1.235500in}}{\pgfqpoint{0.757960in}{1.243736in}}%
\pgfpathcurveto{\pgfqpoint{0.757960in}{1.251972in}}{\pgfqpoint{0.754688in}{1.259872in}}{\pgfqpoint{0.748864in}{1.265696in}}%
\pgfpathcurveto{\pgfqpoint{0.743040in}{1.271520in}}{\pgfqpoint{0.735140in}{1.274792in}}{\pgfqpoint{0.726903in}{1.274792in}}%
\pgfpathcurveto{\pgfqpoint{0.718667in}{1.274792in}}{\pgfqpoint{0.710767in}{1.271520in}}{\pgfqpoint{0.704943in}{1.265696in}}%
\pgfpathcurveto{\pgfqpoint{0.699119in}{1.259872in}}{\pgfqpoint{0.695847in}{1.251972in}}{\pgfqpoint{0.695847in}{1.243736in}}%
\pgfpathcurveto{\pgfqpoint{0.695847in}{1.235500in}}{\pgfqpoint{0.699119in}{1.227599in}}{\pgfqpoint{0.704943in}{1.221776in}}%
\pgfpathcurveto{\pgfqpoint{0.710767in}{1.215952in}}{\pgfqpoint{0.718667in}{1.212679in}}{\pgfqpoint{0.726903in}{1.212679in}}%
\pgfpathclose%
\pgfusepath{stroke,fill}%
\end{pgfscope}%
\begin{pgfscope}%
\pgfpathrectangle{\pgfqpoint{0.100000in}{0.212622in}}{\pgfqpoint{3.696000in}{3.696000in}}%
\pgfusepath{clip}%
\pgfsetbuttcap%
\pgfsetroundjoin%
\definecolor{currentfill}{rgb}{0.121569,0.466667,0.705882}%
\pgfsetfillcolor{currentfill}%
\pgfsetfillopacity{0.663896}%
\pgfsetlinewidth{1.003750pt}%
\definecolor{currentstroke}{rgb}{0.121569,0.466667,0.705882}%
\pgfsetstrokecolor{currentstroke}%
\pgfsetstrokeopacity{0.663896}%
\pgfsetdash{}{0pt}%
\pgfpathmoveto{\pgfqpoint{0.726903in}{1.212679in}}%
\pgfpathcurveto{\pgfqpoint{0.735140in}{1.212679in}}{\pgfqpoint{0.743040in}{1.215952in}}{\pgfqpoint{0.748864in}{1.221776in}}%
\pgfpathcurveto{\pgfqpoint{0.754688in}{1.227599in}}{\pgfqpoint{0.757960in}{1.235500in}}{\pgfqpoint{0.757960in}{1.243736in}}%
\pgfpathcurveto{\pgfqpoint{0.757960in}{1.251972in}}{\pgfqpoint{0.754688in}{1.259872in}}{\pgfqpoint{0.748864in}{1.265696in}}%
\pgfpathcurveto{\pgfqpoint{0.743040in}{1.271520in}}{\pgfqpoint{0.735140in}{1.274792in}}{\pgfqpoint{0.726903in}{1.274792in}}%
\pgfpathcurveto{\pgfqpoint{0.718667in}{1.274792in}}{\pgfqpoint{0.710767in}{1.271520in}}{\pgfqpoint{0.704943in}{1.265696in}}%
\pgfpathcurveto{\pgfqpoint{0.699119in}{1.259872in}}{\pgfqpoint{0.695847in}{1.251972in}}{\pgfqpoint{0.695847in}{1.243736in}}%
\pgfpathcurveto{\pgfqpoint{0.695847in}{1.235500in}}{\pgfqpoint{0.699119in}{1.227599in}}{\pgfqpoint{0.704943in}{1.221776in}}%
\pgfpathcurveto{\pgfqpoint{0.710767in}{1.215952in}}{\pgfqpoint{0.718667in}{1.212679in}}{\pgfqpoint{0.726903in}{1.212679in}}%
\pgfpathclose%
\pgfusepath{stroke,fill}%
\end{pgfscope}%
\begin{pgfscope}%
\pgfpathrectangle{\pgfqpoint{0.100000in}{0.212622in}}{\pgfqpoint{3.696000in}{3.696000in}}%
\pgfusepath{clip}%
\pgfsetbuttcap%
\pgfsetroundjoin%
\definecolor{currentfill}{rgb}{0.121569,0.466667,0.705882}%
\pgfsetfillcolor{currentfill}%
\pgfsetfillopacity{0.663896}%
\pgfsetlinewidth{1.003750pt}%
\definecolor{currentstroke}{rgb}{0.121569,0.466667,0.705882}%
\pgfsetstrokecolor{currentstroke}%
\pgfsetstrokeopacity{0.663896}%
\pgfsetdash{}{0pt}%
\pgfpathmoveto{\pgfqpoint{0.726903in}{1.212679in}}%
\pgfpathcurveto{\pgfqpoint{0.735140in}{1.212679in}}{\pgfqpoint{0.743040in}{1.215952in}}{\pgfqpoint{0.748864in}{1.221776in}}%
\pgfpathcurveto{\pgfqpoint{0.754688in}{1.227599in}}{\pgfqpoint{0.757960in}{1.235500in}}{\pgfqpoint{0.757960in}{1.243736in}}%
\pgfpathcurveto{\pgfqpoint{0.757960in}{1.251972in}}{\pgfqpoint{0.754688in}{1.259872in}}{\pgfqpoint{0.748864in}{1.265696in}}%
\pgfpathcurveto{\pgfqpoint{0.743040in}{1.271520in}}{\pgfqpoint{0.735140in}{1.274792in}}{\pgfqpoint{0.726903in}{1.274792in}}%
\pgfpathcurveto{\pgfqpoint{0.718667in}{1.274792in}}{\pgfqpoint{0.710767in}{1.271520in}}{\pgfqpoint{0.704943in}{1.265696in}}%
\pgfpathcurveto{\pgfqpoint{0.699119in}{1.259872in}}{\pgfqpoint{0.695847in}{1.251972in}}{\pgfqpoint{0.695847in}{1.243736in}}%
\pgfpathcurveto{\pgfqpoint{0.695847in}{1.235500in}}{\pgfqpoint{0.699119in}{1.227599in}}{\pgfqpoint{0.704943in}{1.221776in}}%
\pgfpathcurveto{\pgfqpoint{0.710767in}{1.215952in}}{\pgfqpoint{0.718667in}{1.212679in}}{\pgfqpoint{0.726903in}{1.212679in}}%
\pgfpathclose%
\pgfusepath{stroke,fill}%
\end{pgfscope}%
\begin{pgfscope}%
\pgfpathrectangle{\pgfqpoint{0.100000in}{0.212622in}}{\pgfqpoint{3.696000in}{3.696000in}}%
\pgfusepath{clip}%
\pgfsetbuttcap%
\pgfsetroundjoin%
\definecolor{currentfill}{rgb}{0.121569,0.466667,0.705882}%
\pgfsetfillcolor{currentfill}%
\pgfsetfillopacity{0.663896}%
\pgfsetlinewidth{1.003750pt}%
\definecolor{currentstroke}{rgb}{0.121569,0.466667,0.705882}%
\pgfsetstrokecolor{currentstroke}%
\pgfsetstrokeopacity{0.663896}%
\pgfsetdash{}{0pt}%
\pgfpathmoveto{\pgfqpoint{0.726903in}{1.212679in}}%
\pgfpathcurveto{\pgfqpoint{0.735140in}{1.212679in}}{\pgfqpoint{0.743040in}{1.215952in}}{\pgfqpoint{0.748864in}{1.221776in}}%
\pgfpathcurveto{\pgfqpoint{0.754688in}{1.227599in}}{\pgfqpoint{0.757960in}{1.235500in}}{\pgfqpoint{0.757960in}{1.243736in}}%
\pgfpathcurveto{\pgfqpoint{0.757960in}{1.251972in}}{\pgfqpoint{0.754688in}{1.259872in}}{\pgfqpoint{0.748864in}{1.265696in}}%
\pgfpathcurveto{\pgfqpoint{0.743040in}{1.271520in}}{\pgfqpoint{0.735140in}{1.274792in}}{\pgfqpoint{0.726903in}{1.274792in}}%
\pgfpathcurveto{\pgfqpoint{0.718667in}{1.274792in}}{\pgfqpoint{0.710767in}{1.271520in}}{\pgfqpoint{0.704943in}{1.265696in}}%
\pgfpathcurveto{\pgfqpoint{0.699119in}{1.259872in}}{\pgfqpoint{0.695847in}{1.251972in}}{\pgfqpoint{0.695847in}{1.243736in}}%
\pgfpathcurveto{\pgfqpoint{0.695847in}{1.235500in}}{\pgfqpoint{0.699119in}{1.227599in}}{\pgfqpoint{0.704943in}{1.221776in}}%
\pgfpathcurveto{\pgfqpoint{0.710767in}{1.215952in}}{\pgfqpoint{0.718667in}{1.212679in}}{\pgfqpoint{0.726903in}{1.212679in}}%
\pgfpathclose%
\pgfusepath{stroke,fill}%
\end{pgfscope}%
\begin{pgfscope}%
\pgfpathrectangle{\pgfqpoint{0.100000in}{0.212622in}}{\pgfqpoint{3.696000in}{3.696000in}}%
\pgfusepath{clip}%
\pgfsetbuttcap%
\pgfsetroundjoin%
\definecolor{currentfill}{rgb}{0.121569,0.466667,0.705882}%
\pgfsetfillcolor{currentfill}%
\pgfsetfillopacity{0.663896}%
\pgfsetlinewidth{1.003750pt}%
\definecolor{currentstroke}{rgb}{0.121569,0.466667,0.705882}%
\pgfsetstrokecolor{currentstroke}%
\pgfsetstrokeopacity{0.663896}%
\pgfsetdash{}{0pt}%
\pgfpathmoveto{\pgfqpoint{0.726903in}{1.212679in}}%
\pgfpathcurveto{\pgfqpoint{0.735140in}{1.212679in}}{\pgfqpoint{0.743040in}{1.215952in}}{\pgfqpoint{0.748864in}{1.221776in}}%
\pgfpathcurveto{\pgfqpoint{0.754688in}{1.227599in}}{\pgfqpoint{0.757960in}{1.235500in}}{\pgfqpoint{0.757960in}{1.243736in}}%
\pgfpathcurveto{\pgfqpoint{0.757960in}{1.251972in}}{\pgfqpoint{0.754688in}{1.259872in}}{\pgfqpoint{0.748864in}{1.265696in}}%
\pgfpathcurveto{\pgfqpoint{0.743040in}{1.271520in}}{\pgfqpoint{0.735140in}{1.274792in}}{\pgfqpoint{0.726903in}{1.274792in}}%
\pgfpathcurveto{\pgfqpoint{0.718667in}{1.274792in}}{\pgfqpoint{0.710767in}{1.271520in}}{\pgfqpoint{0.704943in}{1.265696in}}%
\pgfpathcurveto{\pgfqpoint{0.699119in}{1.259872in}}{\pgfqpoint{0.695847in}{1.251972in}}{\pgfqpoint{0.695847in}{1.243736in}}%
\pgfpathcurveto{\pgfqpoint{0.695847in}{1.235500in}}{\pgfqpoint{0.699119in}{1.227599in}}{\pgfqpoint{0.704943in}{1.221776in}}%
\pgfpathcurveto{\pgfqpoint{0.710767in}{1.215952in}}{\pgfqpoint{0.718667in}{1.212679in}}{\pgfqpoint{0.726903in}{1.212679in}}%
\pgfpathclose%
\pgfusepath{stroke,fill}%
\end{pgfscope}%
\begin{pgfscope}%
\pgfpathrectangle{\pgfqpoint{0.100000in}{0.212622in}}{\pgfqpoint{3.696000in}{3.696000in}}%
\pgfusepath{clip}%
\pgfsetbuttcap%
\pgfsetroundjoin%
\definecolor{currentfill}{rgb}{0.121569,0.466667,0.705882}%
\pgfsetfillcolor{currentfill}%
\pgfsetfillopacity{0.663896}%
\pgfsetlinewidth{1.003750pt}%
\definecolor{currentstroke}{rgb}{0.121569,0.466667,0.705882}%
\pgfsetstrokecolor{currentstroke}%
\pgfsetstrokeopacity{0.663896}%
\pgfsetdash{}{0pt}%
\pgfpathmoveto{\pgfqpoint{0.726903in}{1.212679in}}%
\pgfpathcurveto{\pgfqpoint{0.735140in}{1.212679in}}{\pgfqpoint{0.743040in}{1.215952in}}{\pgfqpoint{0.748864in}{1.221776in}}%
\pgfpathcurveto{\pgfqpoint{0.754688in}{1.227599in}}{\pgfqpoint{0.757960in}{1.235500in}}{\pgfqpoint{0.757960in}{1.243736in}}%
\pgfpathcurveto{\pgfqpoint{0.757960in}{1.251972in}}{\pgfqpoint{0.754688in}{1.259872in}}{\pgfqpoint{0.748864in}{1.265696in}}%
\pgfpathcurveto{\pgfqpoint{0.743040in}{1.271520in}}{\pgfqpoint{0.735140in}{1.274792in}}{\pgfqpoint{0.726903in}{1.274792in}}%
\pgfpathcurveto{\pgfqpoint{0.718667in}{1.274792in}}{\pgfqpoint{0.710767in}{1.271520in}}{\pgfqpoint{0.704943in}{1.265696in}}%
\pgfpathcurveto{\pgfqpoint{0.699119in}{1.259872in}}{\pgfqpoint{0.695847in}{1.251972in}}{\pgfqpoint{0.695847in}{1.243736in}}%
\pgfpathcurveto{\pgfqpoint{0.695847in}{1.235500in}}{\pgfqpoint{0.699119in}{1.227599in}}{\pgfqpoint{0.704943in}{1.221776in}}%
\pgfpathcurveto{\pgfqpoint{0.710767in}{1.215952in}}{\pgfqpoint{0.718667in}{1.212679in}}{\pgfqpoint{0.726903in}{1.212679in}}%
\pgfpathclose%
\pgfusepath{stroke,fill}%
\end{pgfscope}%
\begin{pgfscope}%
\pgfpathrectangle{\pgfqpoint{0.100000in}{0.212622in}}{\pgfqpoint{3.696000in}{3.696000in}}%
\pgfusepath{clip}%
\pgfsetbuttcap%
\pgfsetroundjoin%
\definecolor{currentfill}{rgb}{0.121569,0.466667,0.705882}%
\pgfsetfillcolor{currentfill}%
\pgfsetfillopacity{0.663896}%
\pgfsetlinewidth{1.003750pt}%
\definecolor{currentstroke}{rgb}{0.121569,0.466667,0.705882}%
\pgfsetstrokecolor{currentstroke}%
\pgfsetstrokeopacity{0.663896}%
\pgfsetdash{}{0pt}%
\pgfpathmoveto{\pgfqpoint{0.726903in}{1.212679in}}%
\pgfpathcurveto{\pgfqpoint{0.735140in}{1.212679in}}{\pgfqpoint{0.743040in}{1.215952in}}{\pgfqpoint{0.748864in}{1.221776in}}%
\pgfpathcurveto{\pgfqpoint{0.754688in}{1.227599in}}{\pgfqpoint{0.757960in}{1.235500in}}{\pgfqpoint{0.757960in}{1.243736in}}%
\pgfpathcurveto{\pgfqpoint{0.757960in}{1.251972in}}{\pgfqpoint{0.754688in}{1.259872in}}{\pgfqpoint{0.748864in}{1.265696in}}%
\pgfpathcurveto{\pgfqpoint{0.743040in}{1.271520in}}{\pgfqpoint{0.735140in}{1.274792in}}{\pgfqpoint{0.726903in}{1.274792in}}%
\pgfpathcurveto{\pgfqpoint{0.718667in}{1.274792in}}{\pgfqpoint{0.710767in}{1.271520in}}{\pgfqpoint{0.704943in}{1.265696in}}%
\pgfpathcurveto{\pgfqpoint{0.699119in}{1.259872in}}{\pgfqpoint{0.695847in}{1.251972in}}{\pgfqpoint{0.695847in}{1.243736in}}%
\pgfpathcurveto{\pgfqpoint{0.695847in}{1.235500in}}{\pgfqpoint{0.699119in}{1.227599in}}{\pgfqpoint{0.704943in}{1.221776in}}%
\pgfpathcurveto{\pgfqpoint{0.710767in}{1.215952in}}{\pgfqpoint{0.718667in}{1.212679in}}{\pgfqpoint{0.726903in}{1.212679in}}%
\pgfpathclose%
\pgfusepath{stroke,fill}%
\end{pgfscope}%
\begin{pgfscope}%
\pgfpathrectangle{\pgfqpoint{0.100000in}{0.212622in}}{\pgfqpoint{3.696000in}{3.696000in}}%
\pgfusepath{clip}%
\pgfsetbuttcap%
\pgfsetroundjoin%
\definecolor{currentfill}{rgb}{0.121569,0.466667,0.705882}%
\pgfsetfillcolor{currentfill}%
\pgfsetfillopacity{0.663896}%
\pgfsetlinewidth{1.003750pt}%
\definecolor{currentstroke}{rgb}{0.121569,0.466667,0.705882}%
\pgfsetstrokecolor{currentstroke}%
\pgfsetstrokeopacity{0.663896}%
\pgfsetdash{}{0pt}%
\pgfpathmoveto{\pgfqpoint{0.726903in}{1.212679in}}%
\pgfpathcurveto{\pgfqpoint{0.735140in}{1.212679in}}{\pgfqpoint{0.743040in}{1.215952in}}{\pgfqpoint{0.748864in}{1.221776in}}%
\pgfpathcurveto{\pgfqpoint{0.754688in}{1.227599in}}{\pgfqpoint{0.757960in}{1.235500in}}{\pgfqpoint{0.757960in}{1.243736in}}%
\pgfpathcurveto{\pgfqpoint{0.757960in}{1.251972in}}{\pgfqpoint{0.754688in}{1.259872in}}{\pgfqpoint{0.748864in}{1.265696in}}%
\pgfpathcurveto{\pgfqpoint{0.743040in}{1.271520in}}{\pgfqpoint{0.735140in}{1.274792in}}{\pgfqpoint{0.726903in}{1.274792in}}%
\pgfpathcurveto{\pgfqpoint{0.718667in}{1.274792in}}{\pgfqpoint{0.710767in}{1.271520in}}{\pgfqpoint{0.704943in}{1.265696in}}%
\pgfpathcurveto{\pgfqpoint{0.699119in}{1.259872in}}{\pgfqpoint{0.695847in}{1.251972in}}{\pgfqpoint{0.695847in}{1.243736in}}%
\pgfpathcurveto{\pgfqpoint{0.695847in}{1.235500in}}{\pgfqpoint{0.699119in}{1.227599in}}{\pgfqpoint{0.704943in}{1.221776in}}%
\pgfpathcurveto{\pgfqpoint{0.710767in}{1.215952in}}{\pgfqpoint{0.718667in}{1.212679in}}{\pgfqpoint{0.726903in}{1.212679in}}%
\pgfpathclose%
\pgfusepath{stroke,fill}%
\end{pgfscope}%
\begin{pgfscope}%
\pgfpathrectangle{\pgfqpoint{0.100000in}{0.212622in}}{\pgfqpoint{3.696000in}{3.696000in}}%
\pgfusepath{clip}%
\pgfsetbuttcap%
\pgfsetroundjoin%
\definecolor{currentfill}{rgb}{0.121569,0.466667,0.705882}%
\pgfsetfillcolor{currentfill}%
\pgfsetfillopacity{0.663896}%
\pgfsetlinewidth{1.003750pt}%
\definecolor{currentstroke}{rgb}{0.121569,0.466667,0.705882}%
\pgfsetstrokecolor{currentstroke}%
\pgfsetstrokeopacity{0.663896}%
\pgfsetdash{}{0pt}%
\pgfpathmoveto{\pgfqpoint{0.726903in}{1.212679in}}%
\pgfpathcurveto{\pgfqpoint{0.735140in}{1.212679in}}{\pgfqpoint{0.743040in}{1.215952in}}{\pgfqpoint{0.748864in}{1.221776in}}%
\pgfpathcurveto{\pgfqpoint{0.754688in}{1.227599in}}{\pgfqpoint{0.757960in}{1.235500in}}{\pgfqpoint{0.757960in}{1.243736in}}%
\pgfpathcurveto{\pgfqpoint{0.757960in}{1.251972in}}{\pgfqpoint{0.754688in}{1.259872in}}{\pgfqpoint{0.748864in}{1.265696in}}%
\pgfpathcurveto{\pgfqpoint{0.743040in}{1.271520in}}{\pgfqpoint{0.735140in}{1.274792in}}{\pgfqpoint{0.726903in}{1.274792in}}%
\pgfpathcurveto{\pgfqpoint{0.718667in}{1.274792in}}{\pgfqpoint{0.710767in}{1.271520in}}{\pgfqpoint{0.704943in}{1.265696in}}%
\pgfpathcurveto{\pgfqpoint{0.699119in}{1.259872in}}{\pgfqpoint{0.695847in}{1.251972in}}{\pgfqpoint{0.695847in}{1.243736in}}%
\pgfpathcurveto{\pgfqpoint{0.695847in}{1.235500in}}{\pgfqpoint{0.699119in}{1.227599in}}{\pgfqpoint{0.704943in}{1.221776in}}%
\pgfpathcurveto{\pgfqpoint{0.710767in}{1.215952in}}{\pgfqpoint{0.718667in}{1.212679in}}{\pgfqpoint{0.726903in}{1.212679in}}%
\pgfpathclose%
\pgfusepath{stroke,fill}%
\end{pgfscope}%
\begin{pgfscope}%
\pgfpathrectangle{\pgfqpoint{0.100000in}{0.212622in}}{\pgfqpoint{3.696000in}{3.696000in}}%
\pgfusepath{clip}%
\pgfsetbuttcap%
\pgfsetroundjoin%
\definecolor{currentfill}{rgb}{0.121569,0.466667,0.705882}%
\pgfsetfillcolor{currentfill}%
\pgfsetfillopacity{0.663896}%
\pgfsetlinewidth{1.003750pt}%
\definecolor{currentstroke}{rgb}{0.121569,0.466667,0.705882}%
\pgfsetstrokecolor{currentstroke}%
\pgfsetstrokeopacity{0.663896}%
\pgfsetdash{}{0pt}%
\pgfpathmoveto{\pgfqpoint{0.726903in}{1.212679in}}%
\pgfpathcurveto{\pgfqpoint{0.735140in}{1.212679in}}{\pgfqpoint{0.743040in}{1.215952in}}{\pgfqpoint{0.748864in}{1.221776in}}%
\pgfpathcurveto{\pgfqpoint{0.754688in}{1.227599in}}{\pgfqpoint{0.757960in}{1.235500in}}{\pgfqpoint{0.757960in}{1.243736in}}%
\pgfpathcurveto{\pgfqpoint{0.757960in}{1.251972in}}{\pgfqpoint{0.754688in}{1.259872in}}{\pgfqpoint{0.748864in}{1.265696in}}%
\pgfpathcurveto{\pgfqpoint{0.743040in}{1.271520in}}{\pgfqpoint{0.735140in}{1.274792in}}{\pgfqpoint{0.726903in}{1.274792in}}%
\pgfpathcurveto{\pgfqpoint{0.718667in}{1.274792in}}{\pgfqpoint{0.710767in}{1.271520in}}{\pgfqpoint{0.704943in}{1.265696in}}%
\pgfpathcurveto{\pgfqpoint{0.699119in}{1.259872in}}{\pgfqpoint{0.695847in}{1.251972in}}{\pgfqpoint{0.695847in}{1.243736in}}%
\pgfpathcurveto{\pgfqpoint{0.695847in}{1.235500in}}{\pgfqpoint{0.699119in}{1.227599in}}{\pgfqpoint{0.704943in}{1.221776in}}%
\pgfpathcurveto{\pgfqpoint{0.710767in}{1.215952in}}{\pgfqpoint{0.718667in}{1.212679in}}{\pgfqpoint{0.726903in}{1.212679in}}%
\pgfpathclose%
\pgfusepath{stroke,fill}%
\end{pgfscope}%
\begin{pgfscope}%
\pgfpathrectangle{\pgfqpoint{0.100000in}{0.212622in}}{\pgfqpoint{3.696000in}{3.696000in}}%
\pgfusepath{clip}%
\pgfsetbuttcap%
\pgfsetroundjoin%
\definecolor{currentfill}{rgb}{0.121569,0.466667,0.705882}%
\pgfsetfillcolor{currentfill}%
\pgfsetfillopacity{0.663896}%
\pgfsetlinewidth{1.003750pt}%
\definecolor{currentstroke}{rgb}{0.121569,0.466667,0.705882}%
\pgfsetstrokecolor{currentstroke}%
\pgfsetstrokeopacity{0.663896}%
\pgfsetdash{}{0pt}%
\pgfpathmoveto{\pgfqpoint{0.726903in}{1.212679in}}%
\pgfpathcurveto{\pgfqpoint{0.735140in}{1.212679in}}{\pgfqpoint{0.743040in}{1.215952in}}{\pgfqpoint{0.748864in}{1.221776in}}%
\pgfpathcurveto{\pgfqpoint{0.754688in}{1.227599in}}{\pgfqpoint{0.757960in}{1.235500in}}{\pgfqpoint{0.757960in}{1.243736in}}%
\pgfpathcurveto{\pgfqpoint{0.757960in}{1.251972in}}{\pgfqpoint{0.754688in}{1.259872in}}{\pgfqpoint{0.748864in}{1.265696in}}%
\pgfpathcurveto{\pgfqpoint{0.743040in}{1.271520in}}{\pgfqpoint{0.735140in}{1.274792in}}{\pgfqpoint{0.726903in}{1.274792in}}%
\pgfpathcurveto{\pgfqpoint{0.718667in}{1.274792in}}{\pgfqpoint{0.710767in}{1.271520in}}{\pgfqpoint{0.704943in}{1.265696in}}%
\pgfpathcurveto{\pgfqpoint{0.699119in}{1.259872in}}{\pgfqpoint{0.695847in}{1.251972in}}{\pgfqpoint{0.695847in}{1.243736in}}%
\pgfpathcurveto{\pgfqpoint{0.695847in}{1.235500in}}{\pgfqpoint{0.699119in}{1.227599in}}{\pgfqpoint{0.704943in}{1.221776in}}%
\pgfpathcurveto{\pgfqpoint{0.710767in}{1.215952in}}{\pgfqpoint{0.718667in}{1.212679in}}{\pgfqpoint{0.726903in}{1.212679in}}%
\pgfpathclose%
\pgfusepath{stroke,fill}%
\end{pgfscope}%
\begin{pgfscope}%
\pgfpathrectangle{\pgfqpoint{0.100000in}{0.212622in}}{\pgfqpoint{3.696000in}{3.696000in}}%
\pgfusepath{clip}%
\pgfsetbuttcap%
\pgfsetroundjoin%
\definecolor{currentfill}{rgb}{0.121569,0.466667,0.705882}%
\pgfsetfillcolor{currentfill}%
\pgfsetfillopacity{0.663896}%
\pgfsetlinewidth{1.003750pt}%
\definecolor{currentstroke}{rgb}{0.121569,0.466667,0.705882}%
\pgfsetstrokecolor{currentstroke}%
\pgfsetstrokeopacity{0.663896}%
\pgfsetdash{}{0pt}%
\pgfpathmoveto{\pgfqpoint{0.726903in}{1.212679in}}%
\pgfpathcurveto{\pgfqpoint{0.735140in}{1.212679in}}{\pgfqpoint{0.743040in}{1.215952in}}{\pgfqpoint{0.748864in}{1.221776in}}%
\pgfpathcurveto{\pgfqpoint{0.754688in}{1.227599in}}{\pgfqpoint{0.757960in}{1.235500in}}{\pgfqpoint{0.757960in}{1.243736in}}%
\pgfpathcurveto{\pgfqpoint{0.757960in}{1.251972in}}{\pgfqpoint{0.754688in}{1.259872in}}{\pgfqpoint{0.748864in}{1.265696in}}%
\pgfpathcurveto{\pgfqpoint{0.743040in}{1.271520in}}{\pgfqpoint{0.735140in}{1.274792in}}{\pgfqpoint{0.726903in}{1.274792in}}%
\pgfpathcurveto{\pgfqpoint{0.718667in}{1.274792in}}{\pgfqpoint{0.710767in}{1.271520in}}{\pgfqpoint{0.704943in}{1.265696in}}%
\pgfpathcurveto{\pgfqpoint{0.699119in}{1.259872in}}{\pgfqpoint{0.695847in}{1.251972in}}{\pgfqpoint{0.695847in}{1.243736in}}%
\pgfpathcurveto{\pgfqpoint{0.695847in}{1.235500in}}{\pgfqpoint{0.699119in}{1.227599in}}{\pgfqpoint{0.704943in}{1.221776in}}%
\pgfpathcurveto{\pgfqpoint{0.710767in}{1.215952in}}{\pgfqpoint{0.718667in}{1.212679in}}{\pgfqpoint{0.726903in}{1.212679in}}%
\pgfpathclose%
\pgfusepath{stroke,fill}%
\end{pgfscope}%
\begin{pgfscope}%
\pgfpathrectangle{\pgfqpoint{0.100000in}{0.212622in}}{\pgfqpoint{3.696000in}{3.696000in}}%
\pgfusepath{clip}%
\pgfsetbuttcap%
\pgfsetroundjoin%
\definecolor{currentfill}{rgb}{0.121569,0.466667,0.705882}%
\pgfsetfillcolor{currentfill}%
\pgfsetfillopacity{0.663896}%
\pgfsetlinewidth{1.003750pt}%
\definecolor{currentstroke}{rgb}{0.121569,0.466667,0.705882}%
\pgfsetstrokecolor{currentstroke}%
\pgfsetstrokeopacity{0.663896}%
\pgfsetdash{}{0pt}%
\pgfpathmoveto{\pgfqpoint{0.726903in}{1.212679in}}%
\pgfpathcurveto{\pgfqpoint{0.735140in}{1.212679in}}{\pgfqpoint{0.743040in}{1.215952in}}{\pgfqpoint{0.748864in}{1.221776in}}%
\pgfpathcurveto{\pgfqpoint{0.754688in}{1.227599in}}{\pgfqpoint{0.757960in}{1.235500in}}{\pgfqpoint{0.757960in}{1.243736in}}%
\pgfpathcurveto{\pgfqpoint{0.757960in}{1.251972in}}{\pgfqpoint{0.754688in}{1.259872in}}{\pgfqpoint{0.748864in}{1.265696in}}%
\pgfpathcurveto{\pgfqpoint{0.743040in}{1.271520in}}{\pgfqpoint{0.735140in}{1.274792in}}{\pgfqpoint{0.726903in}{1.274792in}}%
\pgfpathcurveto{\pgfqpoint{0.718667in}{1.274792in}}{\pgfqpoint{0.710767in}{1.271520in}}{\pgfqpoint{0.704943in}{1.265696in}}%
\pgfpathcurveto{\pgfqpoint{0.699119in}{1.259872in}}{\pgfqpoint{0.695847in}{1.251972in}}{\pgfqpoint{0.695847in}{1.243736in}}%
\pgfpathcurveto{\pgfqpoint{0.695847in}{1.235500in}}{\pgfqpoint{0.699119in}{1.227599in}}{\pgfqpoint{0.704943in}{1.221776in}}%
\pgfpathcurveto{\pgfqpoint{0.710767in}{1.215952in}}{\pgfqpoint{0.718667in}{1.212679in}}{\pgfqpoint{0.726903in}{1.212679in}}%
\pgfpathclose%
\pgfusepath{stroke,fill}%
\end{pgfscope}%
\begin{pgfscope}%
\pgfpathrectangle{\pgfqpoint{0.100000in}{0.212622in}}{\pgfqpoint{3.696000in}{3.696000in}}%
\pgfusepath{clip}%
\pgfsetbuttcap%
\pgfsetroundjoin%
\definecolor{currentfill}{rgb}{0.121569,0.466667,0.705882}%
\pgfsetfillcolor{currentfill}%
\pgfsetfillopacity{0.663896}%
\pgfsetlinewidth{1.003750pt}%
\definecolor{currentstroke}{rgb}{0.121569,0.466667,0.705882}%
\pgfsetstrokecolor{currentstroke}%
\pgfsetstrokeopacity{0.663896}%
\pgfsetdash{}{0pt}%
\pgfpathmoveto{\pgfqpoint{0.726903in}{1.212679in}}%
\pgfpathcurveto{\pgfqpoint{0.735140in}{1.212679in}}{\pgfqpoint{0.743040in}{1.215952in}}{\pgfqpoint{0.748864in}{1.221776in}}%
\pgfpathcurveto{\pgfqpoint{0.754688in}{1.227599in}}{\pgfqpoint{0.757960in}{1.235500in}}{\pgfqpoint{0.757960in}{1.243736in}}%
\pgfpathcurveto{\pgfqpoint{0.757960in}{1.251972in}}{\pgfqpoint{0.754688in}{1.259872in}}{\pgfqpoint{0.748864in}{1.265696in}}%
\pgfpathcurveto{\pgfqpoint{0.743040in}{1.271520in}}{\pgfqpoint{0.735140in}{1.274792in}}{\pgfqpoint{0.726903in}{1.274792in}}%
\pgfpathcurveto{\pgfqpoint{0.718667in}{1.274792in}}{\pgfqpoint{0.710767in}{1.271520in}}{\pgfqpoint{0.704943in}{1.265696in}}%
\pgfpathcurveto{\pgfqpoint{0.699119in}{1.259872in}}{\pgfqpoint{0.695847in}{1.251972in}}{\pgfqpoint{0.695847in}{1.243736in}}%
\pgfpathcurveto{\pgfqpoint{0.695847in}{1.235500in}}{\pgfqpoint{0.699119in}{1.227599in}}{\pgfqpoint{0.704943in}{1.221776in}}%
\pgfpathcurveto{\pgfqpoint{0.710767in}{1.215952in}}{\pgfqpoint{0.718667in}{1.212679in}}{\pgfqpoint{0.726903in}{1.212679in}}%
\pgfpathclose%
\pgfusepath{stroke,fill}%
\end{pgfscope}%
\begin{pgfscope}%
\pgfpathrectangle{\pgfqpoint{0.100000in}{0.212622in}}{\pgfqpoint{3.696000in}{3.696000in}}%
\pgfusepath{clip}%
\pgfsetbuttcap%
\pgfsetroundjoin%
\definecolor{currentfill}{rgb}{0.121569,0.466667,0.705882}%
\pgfsetfillcolor{currentfill}%
\pgfsetfillopacity{0.663896}%
\pgfsetlinewidth{1.003750pt}%
\definecolor{currentstroke}{rgb}{0.121569,0.466667,0.705882}%
\pgfsetstrokecolor{currentstroke}%
\pgfsetstrokeopacity{0.663896}%
\pgfsetdash{}{0pt}%
\pgfpathmoveto{\pgfqpoint{0.726903in}{1.212679in}}%
\pgfpathcurveto{\pgfqpoint{0.735140in}{1.212679in}}{\pgfqpoint{0.743040in}{1.215952in}}{\pgfqpoint{0.748864in}{1.221776in}}%
\pgfpathcurveto{\pgfqpoint{0.754688in}{1.227599in}}{\pgfqpoint{0.757960in}{1.235500in}}{\pgfqpoint{0.757960in}{1.243736in}}%
\pgfpathcurveto{\pgfqpoint{0.757960in}{1.251972in}}{\pgfqpoint{0.754688in}{1.259872in}}{\pgfqpoint{0.748864in}{1.265696in}}%
\pgfpathcurveto{\pgfqpoint{0.743040in}{1.271520in}}{\pgfqpoint{0.735140in}{1.274792in}}{\pgfqpoint{0.726903in}{1.274792in}}%
\pgfpathcurveto{\pgfqpoint{0.718667in}{1.274792in}}{\pgfqpoint{0.710767in}{1.271520in}}{\pgfqpoint{0.704943in}{1.265696in}}%
\pgfpathcurveto{\pgfqpoint{0.699119in}{1.259872in}}{\pgfqpoint{0.695847in}{1.251972in}}{\pgfqpoint{0.695847in}{1.243736in}}%
\pgfpathcurveto{\pgfqpoint{0.695847in}{1.235500in}}{\pgfqpoint{0.699119in}{1.227599in}}{\pgfqpoint{0.704943in}{1.221776in}}%
\pgfpathcurveto{\pgfqpoint{0.710767in}{1.215952in}}{\pgfqpoint{0.718667in}{1.212679in}}{\pgfqpoint{0.726903in}{1.212679in}}%
\pgfpathclose%
\pgfusepath{stroke,fill}%
\end{pgfscope}%
\begin{pgfscope}%
\pgfpathrectangle{\pgfqpoint{0.100000in}{0.212622in}}{\pgfqpoint{3.696000in}{3.696000in}}%
\pgfusepath{clip}%
\pgfsetbuttcap%
\pgfsetroundjoin%
\definecolor{currentfill}{rgb}{0.121569,0.466667,0.705882}%
\pgfsetfillcolor{currentfill}%
\pgfsetfillopacity{0.663896}%
\pgfsetlinewidth{1.003750pt}%
\definecolor{currentstroke}{rgb}{0.121569,0.466667,0.705882}%
\pgfsetstrokecolor{currentstroke}%
\pgfsetstrokeopacity{0.663896}%
\pgfsetdash{}{0pt}%
\pgfpathmoveto{\pgfqpoint{0.726903in}{1.212679in}}%
\pgfpathcurveto{\pgfqpoint{0.735140in}{1.212679in}}{\pgfqpoint{0.743040in}{1.215952in}}{\pgfqpoint{0.748864in}{1.221776in}}%
\pgfpathcurveto{\pgfqpoint{0.754688in}{1.227599in}}{\pgfqpoint{0.757960in}{1.235500in}}{\pgfqpoint{0.757960in}{1.243736in}}%
\pgfpathcurveto{\pgfqpoint{0.757960in}{1.251972in}}{\pgfqpoint{0.754688in}{1.259872in}}{\pgfqpoint{0.748864in}{1.265696in}}%
\pgfpathcurveto{\pgfqpoint{0.743040in}{1.271520in}}{\pgfqpoint{0.735140in}{1.274792in}}{\pgfqpoint{0.726903in}{1.274792in}}%
\pgfpathcurveto{\pgfqpoint{0.718667in}{1.274792in}}{\pgfqpoint{0.710767in}{1.271520in}}{\pgfqpoint{0.704943in}{1.265696in}}%
\pgfpathcurveto{\pgfqpoint{0.699119in}{1.259872in}}{\pgfqpoint{0.695847in}{1.251972in}}{\pgfqpoint{0.695847in}{1.243736in}}%
\pgfpathcurveto{\pgfqpoint{0.695847in}{1.235500in}}{\pgfqpoint{0.699119in}{1.227599in}}{\pgfqpoint{0.704943in}{1.221776in}}%
\pgfpathcurveto{\pgfqpoint{0.710767in}{1.215952in}}{\pgfqpoint{0.718667in}{1.212679in}}{\pgfqpoint{0.726903in}{1.212679in}}%
\pgfpathclose%
\pgfusepath{stroke,fill}%
\end{pgfscope}%
\begin{pgfscope}%
\pgfpathrectangle{\pgfqpoint{0.100000in}{0.212622in}}{\pgfqpoint{3.696000in}{3.696000in}}%
\pgfusepath{clip}%
\pgfsetbuttcap%
\pgfsetroundjoin%
\definecolor{currentfill}{rgb}{0.121569,0.466667,0.705882}%
\pgfsetfillcolor{currentfill}%
\pgfsetfillopacity{0.663896}%
\pgfsetlinewidth{1.003750pt}%
\definecolor{currentstroke}{rgb}{0.121569,0.466667,0.705882}%
\pgfsetstrokecolor{currentstroke}%
\pgfsetstrokeopacity{0.663896}%
\pgfsetdash{}{0pt}%
\pgfpathmoveto{\pgfqpoint{0.726903in}{1.212679in}}%
\pgfpathcurveto{\pgfqpoint{0.735140in}{1.212679in}}{\pgfqpoint{0.743040in}{1.215952in}}{\pgfqpoint{0.748864in}{1.221776in}}%
\pgfpathcurveto{\pgfqpoint{0.754688in}{1.227599in}}{\pgfqpoint{0.757960in}{1.235500in}}{\pgfqpoint{0.757960in}{1.243736in}}%
\pgfpathcurveto{\pgfqpoint{0.757960in}{1.251972in}}{\pgfqpoint{0.754688in}{1.259872in}}{\pgfqpoint{0.748864in}{1.265696in}}%
\pgfpathcurveto{\pgfqpoint{0.743040in}{1.271520in}}{\pgfqpoint{0.735140in}{1.274792in}}{\pgfqpoint{0.726903in}{1.274792in}}%
\pgfpathcurveto{\pgfqpoint{0.718667in}{1.274792in}}{\pgfqpoint{0.710767in}{1.271520in}}{\pgfqpoint{0.704943in}{1.265696in}}%
\pgfpathcurveto{\pgfqpoint{0.699119in}{1.259872in}}{\pgfqpoint{0.695847in}{1.251972in}}{\pgfqpoint{0.695847in}{1.243736in}}%
\pgfpathcurveto{\pgfqpoint{0.695847in}{1.235500in}}{\pgfqpoint{0.699119in}{1.227599in}}{\pgfqpoint{0.704943in}{1.221776in}}%
\pgfpathcurveto{\pgfqpoint{0.710767in}{1.215952in}}{\pgfqpoint{0.718667in}{1.212679in}}{\pgfqpoint{0.726903in}{1.212679in}}%
\pgfpathclose%
\pgfusepath{stroke,fill}%
\end{pgfscope}%
\begin{pgfscope}%
\pgfpathrectangle{\pgfqpoint{0.100000in}{0.212622in}}{\pgfqpoint{3.696000in}{3.696000in}}%
\pgfusepath{clip}%
\pgfsetbuttcap%
\pgfsetroundjoin%
\definecolor{currentfill}{rgb}{0.121569,0.466667,0.705882}%
\pgfsetfillcolor{currentfill}%
\pgfsetfillopacity{0.663896}%
\pgfsetlinewidth{1.003750pt}%
\definecolor{currentstroke}{rgb}{0.121569,0.466667,0.705882}%
\pgfsetstrokecolor{currentstroke}%
\pgfsetstrokeopacity{0.663896}%
\pgfsetdash{}{0pt}%
\pgfpathmoveto{\pgfqpoint{0.726903in}{1.212679in}}%
\pgfpathcurveto{\pgfqpoint{0.735140in}{1.212679in}}{\pgfqpoint{0.743040in}{1.215952in}}{\pgfqpoint{0.748864in}{1.221776in}}%
\pgfpathcurveto{\pgfqpoint{0.754688in}{1.227599in}}{\pgfqpoint{0.757960in}{1.235500in}}{\pgfqpoint{0.757960in}{1.243736in}}%
\pgfpathcurveto{\pgfqpoint{0.757960in}{1.251972in}}{\pgfqpoint{0.754688in}{1.259872in}}{\pgfqpoint{0.748864in}{1.265696in}}%
\pgfpathcurveto{\pgfqpoint{0.743040in}{1.271520in}}{\pgfqpoint{0.735140in}{1.274792in}}{\pgfqpoint{0.726903in}{1.274792in}}%
\pgfpathcurveto{\pgfqpoint{0.718667in}{1.274792in}}{\pgfqpoint{0.710767in}{1.271520in}}{\pgfqpoint{0.704943in}{1.265696in}}%
\pgfpathcurveto{\pgfqpoint{0.699119in}{1.259872in}}{\pgfqpoint{0.695847in}{1.251972in}}{\pgfqpoint{0.695847in}{1.243736in}}%
\pgfpathcurveto{\pgfqpoint{0.695847in}{1.235500in}}{\pgfqpoint{0.699119in}{1.227599in}}{\pgfqpoint{0.704943in}{1.221776in}}%
\pgfpathcurveto{\pgfqpoint{0.710767in}{1.215952in}}{\pgfqpoint{0.718667in}{1.212679in}}{\pgfqpoint{0.726903in}{1.212679in}}%
\pgfpathclose%
\pgfusepath{stroke,fill}%
\end{pgfscope}%
\begin{pgfscope}%
\pgfpathrectangle{\pgfqpoint{0.100000in}{0.212622in}}{\pgfqpoint{3.696000in}{3.696000in}}%
\pgfusepath{clip}%
\pgfsetbuttcap%
\pgfsetroundjoin%
\definecolor{currentfill}{rgb}{0.121569,0.466667,0.705882}%
\pgfsetfillcolor{currentfill}%
\pgfsetfillopacity{0.663896}%
\pgfsetlinewidth{1.003750pt}%
\definecolor{currentstroke}{rgb}{0.121569,0.466667,0.705882}%
\pgfsetstrokecolor{currentstroke}%
\pgfsetstrokeopacity{0.663896}%
\pgfsetdash{}{0pt}%
\pgfpathmoveto{\pgfqpoint{0.726903in}{1.212679in}}%
\pgfpathcurveto{\pgfqpoint{0.735140in}{1.212679in}}{\pgfqpoint{0.743040in}{1.215952in}}{\pgfqpoint{0.748864in}{1.221776in}}%
\pgfpathcurveto{\pgfqpoint{0.754688in}{1.227599in}}{\pgfqpoint{0.757960in}{1.235500in}}{\pgfqpoint{0.757960in}{1.243736in}}%
\pgfpathcurveto{\pgfqpoint{0.757960in}{1.251972in}}{\pgfqpoint{0.754688in}{1.259872in}}{\pgfqpoint{0.748864in}{1.265696in}}%
\pgfpathcurveto{\pgfqpoint{0.743040in}{1.271520in}}{\pgfqpoint{0.735140in}{1.274792in}}{\pgfqpoint{0.726903in}{1.274792in}}%
\pgfpathcurveto{\pgfqpoint{0.718667in}{1.274792in}}{\pgfqpoint{0.710767in}{1.271520in}}{\pgfqpoint{0.704943in}{1.265696in}}%
\pgfpathcurveto{\pgfqpoint{0.699119in}{1.259872in}}{\pgfqpoint{0.695847in}{1.251972in}}{\pgfqpoint{0.695847in}{1.243736in}}%
\pgfpathcurveto{\pgfqpoint{0.695847in}{1.235500in}}{\pgfqpoint{0.699119in}{1.227599in}}{\pgfqpoint{0.704943in}{1.221776in}}%
\pgfpathcurveto{\pgfqpoint{0.710767in}{1.215952in}}{\pgfqpoint{0.718667in}{1.212679in}}{\pgfqpoint{0.726903in}{1.212679in}}%
\pgfpathclose%
\pgfusepath{stroke,fill}%
\end{pgfscope}%
\begin{pgfscope}%
\pgfpathrectangle{\pgfqpoint{0.100000in}{0.212622in}}{\pgfqpoint{3.696000in}{3.696000in}}%
\pgfusepath{clip}%
\pgfsetbuttcap%
\pgfsetroundjoin%
\definecolor{currentfill}{rgb}{0.121569,0.466667,0.705882}%
\pgfsetfillcolor{currentfill}%
\pgfsetfillopacity{0.667329}%
\pgfsetlinewidth{1.003750pt}%
\definecolor{currentstroke}{rgb}{0.121569,0.466667,0.705882}%
\pgfsetstrokecolor{currentstroke}%
\pgfsetstrokeopacity{0.667329}%
\pgfsetdash{}{0pt}%
\pgfpathmoveto{\pgfqpoint{3.189546in}{2.203403in}}%
\pgfpathcurveto{\pgfqpoint{3.197782in}{2.203403in}}{\pgfqpoint{3.205682in}{2.206675in}}{\pgfqpoint{3.211506in}{2.212499in}}%
\pgfpathcurveto{\pgfqpoint{3.217330in}{2.218323in}}{\pgfqpoint{3.220603in}{2.226223in}}{\pgfqpoint{3.220603in}{2.234460in}}%
\pgfpathcurveto{\pgfqpoint{3.220603in}{2.242696in}}{\pgfqpoint{3.217330in}{2.250596in}}{\pgfqpoint{3.211506in}{2.256420in}}%
\pgfpathcurveto{\pgfqpoint{3.205682in}{2.262244in}}{\pgfqpoint{3.197782in}{2.265516in}}{\pgfqpoint{3.189546in}{2.265516in}}%
\pgfpathcurveto{\pgfqpoint{3.181310in}{2.265516in}}{\pgfqpoint{3.173410in}{2.262244in}}{\pgfqpoint{3.167586in}{2.256420in}}%
\pgfpathcurveto{\pgfqpoint{3.161762in}{2.250596in}}{\pgfqpoint{3.158490in}{2.242696in}}{\pgfqpoint{3.158490in}{2.234460in}}%
\pgfpathcurveto{\pgfqpoint{3.158490in}{2.226223in}}{\pgfqpoint{3.161762in}{2.218323in}}{\pgfqpoint{3.167586in}{2.212499in}}%
\pgfpathcurveto{\pgfqpoint{3.173410in}{2.206675in}}{\pgfqpoint{3.181310in}{2.203403in}}{\pgfqpoint{3.189546in}{2.203403in}}%
\pgfpathclose%
\pgfusepath{stroke,fill}%
\end{pgfscope}%
\begin{pgfscope}%
\pgfpathrectangle{\pgfqpoint{0.100000in}{0.212622in}}{\pgfqpoint{3.696000in}{3.696000in}}%
\pgfusepath{clip}%
\pgfsetbuttcap%
\pgfsetroundjoin%
\definecolor{currentfill}{rgb}{0.121569,0.466667,0.705882}%
\pgfsetfillcolor{currentfill}%
\pgfsetfillopacity{0.673712}%
\pgfsetlinewidth{1.003750pt}%
\definecolor{currentstroke}{rgb}{0.121569,0.466667,0.705882}%
\pgfsetstrokecolor{currentstroke}%
\pgfsetstrokeopacity{0.673712}%
\pgfsetdash{}{0pt}%
\pgfpathmoveto{\pgfqpoint{3.167423in}{2.196227in}}%
\pgfpathcurveto{\pgfqpoint{3.175659in}{2.196227in}}{\pgfqpoint{3.183559in}{2.199499in}}{\pgfqpoint{3.189383in}{2.205323in}}%
\pgfpathcurveto{\pgfqpoint{3.195207in}{2.211147in}}{\pgfqpoint{3.198479in}{2.219047in}}{\pgfqpoint{3.198479in}{2.227283in}}%
\pgfpathcurveto{\pgfqpoint{3.198479in}{2.235520in}}{\pgfqpoint{3.195207in}{2.243420in}}{\pgfqpoint{3.189383in}{2.249244in}}%
\pgfpathcurveto{\pgfqpoint{3.183559in}{2.255068in}}{\pgfqpoint{3.175659in}{2.258340in}}{\pgfqpoint{3.167423in}{2.258340in}}%
\pgfpathcurveto{\pgfqpoint{3.159187in}{2.258340in}}{\pgfqpoint{3.151287in}{2.255068in}}{\pgfqpoint{3.145463in}{2.249244in}}%
\pgfpathcurveto{\pgfqpoint{3.139639in}{2.243420in}}{\pgfqpoint{3.136366in}{2.235520in}}{\pgfqpoint{3.136366in}{2.227283in}}%
\pgfpathcurveto{\pgfqpoint{3.136366in}{2.219047in}}{\pgfqpoint{3.139639in}{2.211147in}}{\pgfqpoint{3.145463in}{2.205323in}}%
\pgfpathcurveto{\pgfqpoint{3.151287in}{2.199499in}}{\pgfqpoint{3.159187in}{2.196227in}}{\pgfqpoint{3.167423in}{2.196227in}}%
\pgfpathclose%
\pgfusepath{stroke,fill}%
\end{pgfscope}%
\begin{pgfscope}%
\pgfpathrectangle{\pgfqpoint{0.100000in}{0.212622in}}{\pgfqpoint{3.696000in}{3.696000in}}%
\pgfusepath{clip}%
\pgfsetbuttcap%
\pgfsetroundjoin%
\definecolor{currentfill}{rgb}{0.121569,0.466667,0.705882}%
\pgfsetfillcolor{currentfill}%
\pgfsetfillopacity{0.681951}%
\pgfsetlinewidth{1.003750pt}%
\definecolor{currentstroke}{rgb}{0.121569,0.466667,0.705882}%
\pgfsetstrokecolor{currentstroke}%
\pgfsetstrokeopacity{0.681951}%
\pgfsetdash{}{0pt}%
\pgfpathmoveto{\pgfqpoint{3.149094in}{2.183491in}}%
\pgfpathcurveto{\pgfqpoint{3.157330in}{2.183491in}}{\pgfqpoint{3.165230in}{2.186763in}}{\pgfqpoint{3.171054in}{2.192587in}}%
\pgfpathcurveto{\pgfqpoint{3.176878in}{2.198411in}}{\pgfqpoint{3.180151in}{2.206311in}}{\pgfqpoint{3.180151in}{2.214548in}}%
\pgfpathcurveto{\pgfqpoint{3.180151in}{2.222784in}}{\pgfqpoint{3.176878in}{2.230684in}}{\pgfqpoint{3.171054in}{2.236508in}}%
\pgfpathcurveto{\pgfqpoint{3.165230in}{2.242332in}}{\pgfqpoint{3.157330in}{2.245604in}}{\pgfqpoint{3.149094in}{2.245604in}}%
\pgfpathcurveto{\pgfqpoint{3.140858in}{2.245604in}}{\pgfqpoint{3.132958in}{2.242332in}}{\pgfqpoint{3.127134in}{2.236508in}}%
\pgfpathcurveto{\pgfqpoint{3.121310in}{2.230684in}}{\pgfqpoint{3.118038in}{2.222784in}}{\pgfqpoint{3.118038in}{2.214548in}}%
\pgfpathcurveto{\pgfqpoint{3.118038in}{2.206311in}}{\pgfqpoint{3.121310in}{2.198411in}}{\pgfqpoint{3.127134in}{2.192587in}}%
\pgfpathcurveto{\pgfqpoint{3.132958in}{2.186763in}}{\pgfqpoint{3.140858in}{2.183491in}}{\pgfqpoint{3.149094in}{2.183491in}}%
\pgfpathclose%
\pgfusepath{stroke,fill}%
\end{pgfscope}%
\begin{pgfscope}%
\pgfpathrectangle{\pgfqpoint{0.100000in}{0.212622in}}{\pgfqpoint{3.696000in}{3.696000in}}%
\pgfusepath{clip}%
\pgfsetbuttcap%
\pgfsetroundjoin%
\definecolor{currentfill}{rgb}{0.121569,0.466667,0.705882}%
\pgfsetfillcolor{currentfill}%
\pgfsetfillopacity{0.690380}%
\pgfsetlinewidth{1.003750pt}%
\definecolor{currentstroke}{rgb}{0.121569,0.466667,0.705882}%
\pgfsetstrokecolor{currentstroke}%
\pgfsetstrokeopacity{0.690380}%
\pgfsetdash{}{0pt}%
\pgfpathmoveto{\pgfqpoint{3.121159in}{2.176459in}}%
\pgfpathcurveto{\pgfqpoint{3.129396in}{2.176459in}}{\pgfqpoint{3.137296in}{2.179731in}}{\pgfqpoint{3.143120in}{2.185555in}}%
\pgfpathcurveto{\pgfqpoint{3.148943in}{2.191379in}}{\pgfqpoint{3.152216in}{2.199279in}}{\pgfqpoint{3.152216in}{2.207515in}}%
\pgfpathcurveto{\pgfqpoint{3.152216in}{2.215752in}}{\pgfqpoint{3.148943in}{2.223652in}}{\pgfqpoint{3.143120in}{2.229476in}}%
\pgfpathcurveto{\pgfqpoint{3.137296in}{2.235300in}}{\pgfqpoint{3.129396in}{2.238572in}}{\pgfqpoint{3.121159in}{2.238572in}}%
\pgfpathcurveto{\pgfqpoint{3.112923in}{2.238572in}}{\pgfqpoint{3.105023in}{2.235300in}}{\pgfqpoint{3.099199in}{2.229476in}}%
\pgfpathcurveto{\pgfqpoint{3.093375in}{2.223652in}}{\pgfqpoint{3.090103in}{2.215752in}}{\pgfqpoint{3.090103in}{2.207515in}}%
\pgfpathcurveto{\pgfqpoint{3.090103in}{2.199279in}}{\pgfqpoint{3.093375in}{2.191379in}}{\pgfqpoint{3.099199in}{2.185555in}}%
\pgfpathcurveto{\pgfqpoint{3.105023in}{2.179731in}}{\pgfqpoint{3.112923in}{2.176459in}}{\pgfqpoint{3.121159in}{2.176459in}}%
\pgfpathclose%
\pgfusepath{stroke,fill}%
\end{pgfscope}%
\begin{pgfscope}%
\pgfpathrectangle{\pgfqpoint{0.100000in}{0.212622in}}{\pgfqpoint{3.696000in}{3.696000in}}%
\pgfusepath{clip}%
\pgfsetbuttcap%
\pgfsetroundjoin%
\definecolor{currentfill}{rgb}{0.121569,0.466667,0.705882}%
\pgfsetfillcolor{currentfill}%
\pgfsetfillopacity{0.696249}%
\pgfsetlinewidth{1.003750pt}%
\definecolor{currentstroke}{rgb}{0.121569,0.466667,0.705882}%
\pgfsetstrokecolor{currentstroke}%
\pgfsetstrokeopacity{0.696249}%
\pgfsetdash{}{0pt}%
\pgfpathmoveto{\pgfqpoint{3.109794in}{2.174407in}}%
\pgfpathcurveto{\pgfqpoint{3.118031in}{2.174407in}}{\pgfqpoint{3.125931in}{2.177679in}}{\pgfqpoint{3.131755in}{2.183503in}}%
\pgfpathcurveto{\pgfqpoint{3.137579in}{2.189327in}}{\pgfqpoint{3.140851in}{2.197227in}}{\pgfqpoint{3.140851in}{2.205463in}}%
\pgfpathcurveto{\pgfqpoint{3.140851in}{2.213699in}}{\pgfqpoint{3.137579in}{2.221599in}}{\pgfqpoint{3.131755in}{2.227423in}}%
\pgfpathcurveto{\pgfqpoint{3.125931in}{2.233247in}}{\pgfqpoint{3.118031in}{2.236520in}}{\pgfqpoint{3.109794in}{2.236520in}}%
\pgfpathcurveto{\pgfqpoint{3.101558in}{2.236520in}}{\pgfqpoint{3.093658in}{2.233247in}}{\pgfqpoint{3.087834in}{2.227423in}}%
\pgfpathcurveto{\pgfqpoint{3.082010in}{2.221599in}}{\pgfqpoint{3.078738in}{2.213699in}}{\pgfqpoint{3.078738in}{2.205463in}}%
\pgfpathcurveto{\pgfqpoint{3.078738in}{2.197227in}}{\pgfqpoint{3.082010in}{2.189327in}}{\pgfqpoint{3.087834in}{2.183503in}}%
\pgfpathcurveto{\pgfqpoint{3.093658in}{2.177679in}}{\pgfqpoint{3.101558in}{2.174407in}}{\pgfqpoint{3.109794in}{2.174407in}}%
\pgfpathclose%
\pgfusepath{stroke,fill}%
\end{pgfscope}%
\begin{pgfscope}%
\pgfpathrectangle{\pgfqpoint{0.100000in}{0.212622in}}{\pgfqpoint{3.696000in}{3.696000in}}%
\pgfusepath{clip}%
\pgfsetbuttcap%
\pgfsetroundjoin%
\definecolor{currentfill}{rgb}{0.121569,0.466667,0.705882}%
\pgfsetfillcolor{currentfill}%
\pgfsetfillopacity{0.698614}%
\pgfsetlinewidth{1.003750pt}%
\definecolor{currentstroke}{rgb}{0.121569,0.466667,0.705882}%
\pgfsetstrokecolor{currentstroke}%
\pgfsetstrokeopacity{0.698614}%
\pgfsetdash{}{0pt}%
\pgfpathmoveto{\pgfqpoint{3.101195in}{2.171147in}}%
\pgfpathcurveto{\pgfqpoint{3.109431in}{2.171147in}}{\pgfqpoint{3.117331in}{2.174420in}}{\pgfqpoint{3.123155in}{2.180244in}}%
\pgfpathcurveto{\pgfqpoint{3.128979in}{2.186068in}}{\pgfqpoint{3.132252in}{2.193968in}}{\pgfqpoint{3.132252in}{2.202204in}}%
\pgfpathcurveto{\pgfqpoint{3.132252in}{2.210440in}}{\pgfqpoint{3.128979in}{2.218340in}}{\pgfqpoint{3.123155in}{2.224164in}}%
\pgfpathcurveto{\pgfqpoint{3.117331in}{2.229988in}}{\pgfqpoint{3.109431in}{2.233260in}}{\pgfqpoint{3.101195in}{2.233260in}}%
\pgfpathcurveto{\pgfqpoint{3.092959in}{2.233260in}}{\pgfqpoint{3.085059in}{2.229988in}}{\pgfqpoint{3.079235in}{2.224164in}}%
\pgfpathcurveto{\pgfqpoint{3.073411in}{2.218340in}}{\pgfqpoint{3.070139in}{2.210440in}}{\pgfqpoint{3.070139in}{2.202204in}}%
\pgfpathcurveto{\pgfqpoint{3.070139in}{2.193968in}}{\pgfqpoint{3.073411in}{2.186068in}}{\pgfqpoint{3.079235in}{2.180244in}}%
\pgfpathcurveto{\pgfqpoint{3.085059in}{2.174420in}}{\pgfqpoint{3.092959in}{2.171147in}}{\pgfqpoint{3.101195in}{2.171147in}}%
\pgfpathclose%
\pgfusepath{stroke,fill}%
\end{pgfscope}%
\begin{pgfscope}%
\pgfpathrectangle{\pgfqpoint{0.100000in}{0.212622in}}{\pgfqpoint{3.696000in}{3.696000in}}%
\pgfusepath{clip}%
\pgfsetbuttcap%
\pgfsetroundjoin%
\definecolor{currentfill}{rgb}{0.121569,0.466667,0.705882}%
\pgfsetfillcolor{currentfill}%
\pgfsetfillopacity{0.700434}%
\pgfsetlinewidth{1.003750pt}%
\definecolor{currentstroke}{rgb}{0.121569,0.466667,0.705882}%
\pgfsetstrokecolor{currentstroke}%
\pgfsetstrokeopacity{0.700434}%
\pgfsetdash{}{0pt}%
\pgfpathmoveto{\pgfqpoint{3.097933in}{2.170568in}}%
\pgfpathcurveto{\pgfqpoint{3.106170in}{2.170568in}}{\pgfqpoint{3.114070in}{2.173840in}}{\pgfqpoint{3.119894in}{2.179664in}}%
\pgfpathcurveto{\pgfqpoint{3.125718in}{2.185488in}}{\pgfqpoint{3.128990in}{2.193388in}}{\pgfqpoint{3.128990in}{2.201624in}}%
\pgfpathcurveto{\pgfqpoint{3.128990in}{2.209861in}}{\pgfqpoint{3.125718in}{2.217761in}}{\pgfqpoint{3.119894in}{2.223585in}}%
\pgfpathcurveto{\pgfqpoint{3.114070in}{2.229409in}}{\pgfqpoint{3.106170in}{2.232681in}}{\pgfqpoint{3.097933in}{2.232681in}}%
\pgfpathcurveto{\pgfqpoint{3.089697in}{2.232681in}}{\pgfqpoint{3.081797in}{2.229409in}}{\pgfqpoint{3.075973in}{2.223585in}}%
\pgfpathcurveto{\pgfqpoint{3.070149in}{2.217761in}}{\pgfqpoint{3.066877in}{2.209861in}}{\pgfqpoint{3.066877in}{2.201624in}}%
\pgfpathcurveto{\pgfqpoint{3.066877in}{2.193388in}}{\pgfqpoint{3.070149in}{2.185488in}}{\pgfqpoint{3.075973in}{2.179664in}}%
\pgfpathcurveto{\pgfqpoint{3.081797in}{2.173840in}}{\pgfqpoint{3.089697in}{2.170568in}}{\pgfqpoint{3.097933in}{2.170568in}}%
\pgfpathclose%
\pgfusepath{stroke,fill}%
\end{pgfscope}%
\begin{pgfscope}%
\pgfpathrectangle{\pgfqpoint{0.100000in}{0.212622in}}{\pgfqpoint{3.696000in}{3.696000in}}%
\pgfusepath{clip}%
\pgfsetbuttcap%
\pgfsetroundjoin%
\definecolor{currentfill}{rgb}{0.121569,0.466667,0.705882}%
\pgfsetfillcolor{currentfill}%
\pgfsetfillopacity{0.701311}%
\pgfsetlinewidth{1.003750pt}%
\definecolor{currentstroke}{rgb}{0.121569,0.466667,0.705882}%
\pgfsetstrokecolor{currentstroke}%
\pgfsetstrokeopacity{0.701311}%
\pgfsetdash{}{0pt}%
\pgfpathmoveto{\pgfqpoint{3.095579in}{2.170204in}}%
\pgfpathcurveto{\pgfqpoint{3.103815in}{2.170204in}}{\pgfqpoint{3.111715in}{2.173477in}}{\pgfqpoint{3.117539in}{2.179301in}}%
\pgfpathcurveto{\pgfqpoint{3.123363in}{2.185125in}}{\pgfqpoint{3.126636in}{2.193025in}}{\pgfqpoint{3.126636in}{2.201261in}}%
\pgfpathcurveto{\pgfqpoint{3.126636in}{2.209497in}}{\pgfqpoint{3.123363in}{2.217397in}}{\pgfqpoint{3.117539in}{2.223221in}}%
\pgfpathcurveto{\pgfqpoint{3.111715in}{2.229045in}}{\pgfqpoint{3.103815in}{2.232317in}}{\pgfqpoint{3.095579in}{2.232317in}}%
\pgfpathcurveto{\pgfqpoint{3.087343in}{2.232317in}}{\pgfqpoint{3.079443in}{2.229045in}}{\pgfqpoint{3.073619in}{2.223221in}}%
\pgfpathcurveto{\pgfqpoint{3.067795in}{2.217397in}}{\pgfqpoint{3.064523in}{2.209497in}}{\pgfqpoint{3.064523in}{2.201261in}}%
\pgfpathcurveto{\pgfqpoint{3.064523in}{2.193025in}}{\pgfqpoint{3.067795in}{2.185125in}}{\pgfqpoint{3.073619in}{2.179301in}}%
\pgfpathcurveto{\pgfqpoint{3.079443in}{2.173477in}}{\pgfqpoint{3.087343in}{2.170204in}}{\pgfqpoint{3.095579in}{2.170204in}}%
\pgfpathclose%
\pgfusepath{stroke,fill}%
\end{pgfscope}%
\begin{pgfscope}%
\pgfpathrectangle{\pgfqpoint{0.100000in}{0.212622in}}{\pgfqpoint{3.696000in}{3.696000in}}%
\pgfusepath{clip}%
\pgfsetbuttcap%
\pgfsetroundjoin%
\definecolor{currentfill}{rgb}{0.121569,0.466667,0.705882}%
\pgfsetfillcolor{currentfill}%
\pgfsetfillopacity{0.701795}%
\pgfsetlinewidth{1.003750pt}%
\definecolor{currentstroke}{rgb}{0.121569,0.466667,0.705882}%
\pgfsetstrokecolor{currentstroke}%
\pgfsetstrokeopacity{0.701795}%
\pgfsetdash{}{0pt}%
\pgfpathmoveto{\pgfqpoint{3.094279in}{2.170027in}}%
\pgfpathcurveto{\pgfqpoint{3.102515in}{2.170027in}}{\pgfqpoint{3.110415in}{2.173299in}}{\pgfqpoint{3.116239in}{2.179123in}}%
\pgfpathcurveto{\pgfqpoint{3.122063in}{2.184947in}}{\pgfqpoint{3.125336in}{2.192847in}}{\pgfqpoint{3.125336in}{2.201083in}}%
\pgfpathcurveto{\pgfqpoint{3.125336in}{2.209320in}}{\pgfqpoint{3.122063in}{2.217220in}}{\pgfqpoint{3.116239in}{2.223043in}}%
\pgfpathcurveto{\pgfqpoint{3.110415in}{2.228867in}}{\pgfqpoint{3.102515in}{2.232140in}}{\pgfqpoint{3.094279in}{2.232140in}}%
\pgfpathcurveto{\pgfqpoint{3.086043in}{2.232140in}}{\pgfqpoint{3.078143in}{2.228867in}}{\pgfqpoint{3.072319in}{2.223043in}}%
\pgfpathcurveto{\pgfqpoint{3.066495in}{2.217220in}}{\pgfqpoint{3.063223in}{2.209320in}}{\pgfqpoint{3.063223in}{2.201083in}}%
\pgfpathcurveto{\pgfqpoint{3.063223in}{2.192847in}}{\pgfqpoint{3.066495in}{2.184947in}}{\pgfqpoint{3.072319in}{2.179123in}}%
\pgfpathcurveto{\pgfqpoint{3.078143in}{2.173299in}}{\pgfqpoint{3.086043in}{2.170027in}}{\pgfqpoint{3.094279in}{2.170027in}}%
\pgfpathclose%
\pgfusepath{stroke,fill}%
\end{pgfscope}%
\begin{pgfscope}%
\pgfpathrectangle{\pgfqpoint{0.100000in}{0.212622in}}{\pgfqpoint{3.696000in}{3.696000in}}%
\pgfusepath{clip}%
\pgfsetbuttcap%
\pgfsetroundjoin%
\definecolor{currentfill}{rgb}{0.121569,0.466667,0.705882}%
\pgfsetfillcolor{currentfill}%
\pgfsetfillopacity{0.702066}%
\pgfsetlinewidth{1.003750pt}%
\definecolor{currentstroke}{rgb}{0.121569,0.466667,0.705882}%
\pgfsetstrokecolor{currentstroke}%
\pgfsetstrokeopacity{0.702066}%
\pgfsetdash{}{0pt}%
\pgfpathmoveto{\pgfqpoint{3.093702in}{2.169755in}}%
\pgfpathcurveto{\pgfqpoint{3.101938in}{2.169755in}}{\pgfqpoint{3.109838in}{2.173027in}}{\pgfqpoint{3.115662in}{2.178851in}}%
\pgfpathcurveto{\pgfqpoint{3.121486in}{2.184675in}}{\pgfqpoint{3.124758in}{2.192575in}}{\pgfqpoint{3.124758in}{2.200811in}}%
\pgfpathcurveto{\pgfqpoint{3.124758in}{2.209048in}}{\pgfqpoint{3.121486in}{2.216948in}}{\pgfqpoint{3.115662in}{2.222772in}}%
\pgfpathcurveto{\pgfqpoint{3.109838in}{2.228596in}}{\pgfqpoint{3.101938in}{2.231868in}}{\pgfqpoint{3.093702in}{2.231868in}}%
\pgfpathcurveto{\pgfqpoint{3.085466in}{2.231868in}}{\pgfqpoint{3.077566in}{2.228596in}}{\pgfqpoint{3.071742in}{2.222772in}}%
\pgfpathcurveto{\pgfqpoint{3.065918in}{2.216948in}}{\pgfqpoint{3.062645in}{2.209048in}}{\pgfqpoint{3.062645in}{2.200811in}}%
\pgfpathcurveto{\pgfqpoint{3.062645in}{2.192575in}}{\pgfqpoint{3.065918in}{2.184675in}}{\pgfqpoint{3.071742in}{2.178851in}}%
\pgfpathcurveto{\pgfqpoint{3.077566in}{2.173027in}}{\pgfqpoint{3.085466in}{2.169755in}}{\pgfqpoint{3.093702in}{2.169755in}}%
\pgfpathclose%
\pgfusepath{stroke,fill}%
\end{pgfscope}%
\begin{pgfscope}%
\pgfpathrectangle{\pgfqpoint{0.100000in}{0.212622in}}{\pgfqpoint{3.696000in}{3.696000in}}%
\pgfusepath{clip}%
\pgfsetbuttcap%
\pgfsetroundjoin%
\definecolor{currentfill}{rgb}{0.121569,0.466667,0.705882}%
\pgfsetfillcolor{currentfill}%
\pgfsetfillopacity{0.702216}%
\pgfsetlinewidth{1.003750pt}%
\definecolor{currentstroke}{rgb}{0.121569,0.466667,0.705882}%
\pgfsetstrokecolor{currentstroke}%
\pgfsetstrokeopacity{0.702216}%
\pgfsetdash{}{0pt}%
\pgfpathmoveto{\pgfqpoint{3.093276in}{2.169784in}}%
\pgfpathcurveto{\pgfqpoint{3.101512in}{2.169784in}}{\pgfqpoint{3.109412in}{2.173056in}}{\pgfqpoint{3.115236in}{2.178880in}}%
\pgfpathcurveto{\pgfqpoint{3.121060in}{2.184704in}}{\pgfqpoint{3.124332in}{2.192604in}}{\pgfqpoint{3.124332in}{2.200840in}}%
\pgfpathcurveto{\pgfqpoint{3.124332in}{2.209077in}}{\pgfqpoint{3.121060in}{2.216977in}}{\pgfqpoint{3.115236in}{2.222801in}}%
\pgfpathcurveto{\pgfqpoint{3.109412in}{2.228624in}}{\pgfqpoint{3.101512in}{2.231897in}}{\pgfqpoint{3.093276in}{2.231897in}}%
\pgfpathcurveto{\pgfqpoint{3.085040in}{2.231897in}}{\pgfqpoint{3.077140in}{2.228624in}}{\pgfqpoint{3.071316in}{2.222801in}}%
\pgfpathcurveto{\pgfqpoint{3.065492in}{2.216977in}}{\pgfqpoint{3.062219in}{2.209077in}}{\pgfqpoint{3.062219in}{2.200840in}}%
\pgfpathcurveto{\pgfqpoint{3.062219in}{2.192604in}}{\pgfqpoint{3.065492in}{2.184704in}}{\pgfqpoint{3.071316in}{2.178880in}}%
\pgfpathcurveto{\pgfqpoint{3.077140in}{2.173056in}}{\pgfqpoint{3.085040in}{2.169784in}}{\pgfqpoint{3.093276in}{2.169784in}}%
\pgfpathclose%
\pgfusepath{stroke,fill}%
\end{pgfscope}%
\begin{pgfscope}%
\pgfpathrectangle{\pgfqpoint{0.100000in}{0.212622in}}{\pgfqpoint{3.696000in}{3.696000in}}%
\pgfusepath{clip}%
\pgfsetbuttcap%
\pgfsetroundjoin%
\definecolor{currentfill}{rgb}{0.121569,0.466667,0.705882}%
\pgfsetfillcolor{currentfill}%
\pgfsetfillopacity{0.702300}%
\pgfsetlinewidth{1.003750pt}%
\definecolor{currentstroke}{rgb}{0.121569,0.466667,0.705882}%
\pgfsetstrokecolor{currentstroke}%
\pgfsetstrokeopacity{0.702300}%
\pgfsetdash{}{0pt}%
\pgfpathmoveto{\pgfqpoint{3.093123in}{2.169691in}}%
\pgfpathcurveto{\pgfqpoint{3.101359in}{2.169691in}}{\pgfqpoint{3.109259in}{2.172963in}}{\pgfqpoint{3.115083in}{2.178787in}}%
\pgfpathcurveto{\pgfqpoint{3.120907in}{2.184611in}}{\pgfqpoint{3.124179in}{2.192511in}}{\pgfqpoint{3.124179in}{2.200747in}}%
\pgfpathcurveto{\pgfqpoint{3.124179in}{2.208984in}}{\pgfqpoint{3.120907in}{2.216884in}}{\pgfqpoint{3.115083in}{2.222708in}}%
\pgfpathcurveto{\pgfqpoint{3.109259in}{2.228532in}}{\pgfqpoint{3.101359in}{2.231804in}}{\pgfqpoint{3.093123in}{2.231804in}}%
\pgfpathcurveto{\pgfqpoint{3.084887in}{2.231804in}}{\pgfqpoint{3.076986in}{2.228532in}}{\pgfqpoint{3.071163in}{2.222708in}}%
\pgfpathcurveto{\pgfqpoint{3.065339in}{2.216884in}}{\pgfqpoint{3.062066in}{2.208984in}}{\pgfqpoint{3.062066in}{2.200747in}}%
\pgfpathcurveto{\pgfqpoint{3.062066in}{2.192511in}}{\pgfqpoint{3.065339in}{2.184611in}}{\pgfqpoint{3.071163in}{2.178787in}}%
\pgfpathcurveto{\pgfqpoint{3.076986in}{2.172963in}}{\pgfqpoint{3.084887in}{2.169691in}}{\pgfqpoint{3.093123in}{2.169691in}}%
\pgfpathclose%
\pgfusepath{stroke,fill}%
\end{pgfscope}%
\begin{pgfscope}%
\pgfpathrectangle{\pgfqpoint{0.100000in}{0.212622in}}{\pgfqpoint{3.696000in}{3.696000in}}%
\pgfusepath{clip}%
\pgfsetbuttcap%
\pgfsetroundjoin%
\definecolor{currentfill}{rgb}{0.121569,0.466667,0.705882}%
\pgfsetfillcolor{currentfill}%
\pgfsetfillopacity{0.703016}%
\pgfsetlinewidth{1.003750pt}%
\definecolor{currentstroke}{rgb}{0.121569,0.466667,0.705882}%
\pgfsetstrokecolor{currentstroke}%
\pgfsetstrokeopacity{0.703016}%
\pgfsetdash{}{0pt}%
\pgfpathmoveto{\pgfqpoint{3.090362in}{2.168475in}}%
\pgfpathcurveto{\pgfqpoint{3.098598in}{2.168475in}}{\pgfqpoint{3.106498in}{2.171747in}}{\pgfqpoint{3.112322in}{2.177571in}}%
\pgfpathcurveto{\pgfqpoint{3.118146in}{2.183395in}}{\pgfqpoint{3.121418in}{2.191295in}}{\pgfqpoint{3.121418in}{2.199531in}}%
\pgfpathcurveto{\pgfqpoint{3.121418in}{2.207768in}}{\pgfqpoint{3.118146in}{2.215668in}}{\pgfqpoint{3.112322in}{2.221492in}}%
\pgfpathcurveto{\pgfqpoint{3.106498in}{2.227316in}}{\pgfqpoint{3.098598in}{2.230588in}}{\pgfqpoint{3.090362in}{2.230588in}}%
\pgfpathcurveto{\pgfqpoint{3.082125in}{2.230588in}}{\pgfqpoint{3.074225in}{2.227316in}}{\pgfqpoint{3.068401in}{2.221492in}}%
\pgfpathcurveto{\pgfqpoint{3.062577in}{2.215668in}}{\pgfqpoint{3.059305in}{2.207768in}}{\pgfqpoint{3.059305in}{2.199531in}}%
\pgfpathcurveto{\pgfqpoint{3.059305in}{2.191295in}}{\pgfqpoint{3.062577in}{2.183395in}}{\pgfqpoint{3.068401in}{2.177571in}}%
\pgfpathcurveto{\pgfqpoint{3.074225in}{2.171747in}}{\pgfqpoint{3.082125in}{2.168475in}}{\pgfqpoint{3.090362in}{2.168475in}}%
\pgfpathclose%
\pgfusepath{stroke,fill}%
\end{pgfscope}%
\begin{pgfscope}%
\pgfpathrectangle{\pgfqpoint{0.100000in}{0.212622in}}{\pgfqpoint{3.696000in}{3.696000in}}%
\pgfusepath{clip}%
\pgfsetbuttcap%
\pgfsetroundjoin%
\definecolor{currentfill}{rgb}{0.121569,0.466667,0.705882}%
\pgfsetfillcolor{currentfill}%
\pgfsetfillopacity{0.705349}%
\pgfsetlinewidth{1.003750pt}%
\definecolor{currentstroke}{rgb}{0.121569,0.466667,0.705882}%
\pgfsetstrokecolor{currentstroke}%
\pgfsetstrokeopacity{0.705349}%
\pgfsetdash{}{0pt}%
\pgfpathmoveto{\pgfqpoint{3.085849in}{2.164764in}}%
\pgfpathcurveto{\pgfqpoint{3.094085in}{2.164764in}}{\pgfqpoint{3.101985in}{2.168037in}}{\pgfqpoint{3.107809in}{2.173861in}}%
\pgfpathcurveto{\pgfqpoint{3.113633in}{2.179684in}}{\pgfqpoint{3.116905in}{2.187584in}}{\pgfqpoint{3.116905in}{2.195821in}}%
\pgfpathcurveto{\pgfqpoint{3.116905in}{2.204057in}}{\pgfqpoint{3.113633in}{2.211957in}}{\pgfqpoint{3.107809in}{2.217781in}}%
\pgfpathcurveto{\pgfqpoint{3.101985in}{2.223605in}}{\pgfqpoint{3.094085in}{2.226877in}}{\pgfqpoint{3.085849in}{2.226877in}}%
\pgfpathcurveto{\pgfqpoint{3.077613in}{2.226877in}}{\pgfqpoint{3.069712in}{2.223605in}}{\pgfqpoint{3.063889in}{2.217781in}}%
\pgfpathcurveto{\pgfqpoint{3.058065in}{2.211957in}}{\pgfqpoint{3.054792in}{2.204057in}}{\pgfqpoint{3.054792in}{2.195821in}}%
\pgfpathcurveto{\pgfqpoint{3.054792in}{2.187584in}}{\pgfqpoint{3.058065in}{2.179684in}}{\pgfqpoint{3.063889in}{2.173861in}}%
\pgfpathcurveto{\pgfqpoint{3.069712in}{2.168037in}}{\pgfqpoint{3.077613in}{2.164764in}}{\pgfqpoint{3.085849in}{2.164764in}}%
\pgfpathclose%
\pgfusepath{stroke,fill}%
\end{pgfscope}%
\begin{pgfscope}%
\pgfpathrectangle{\pgfqpoint{0.100000in}{0.212622in}}{\pgfqpoint{3.696000in}{3.696000in}}%
\pgfusepath{clip}%
\pgfsetbuttcap%
\pgfsetroundjoin%
\definecolor{currentfill}{rgb}{0.121569,0.466667,0.705882}%
\pgfsetfillcolor{currentfill}%
\pgfsetfillopacity{0.706548}%
\pgfsetlinewidth{1.003750pt}%
\definecolor{currentstroke}{rgb}{0.121569,0.466667,0.705882}%
\pgfsetstrokecolor{currentstroke}%
\pgfsetstrokeopacity{0.706548}%
\pgfsetdash{}{0pt}%
\pgfpathmoveto{\pgfqpoint{3.081963in}{2.164231in}}%
\pgfpathcurveto{\pgfqpoint{3.090199in}{2.164231in}}{\pgfqpoint{3.098099in}{2.167503in}}{\pgfqpoint{3.103923in}{2.173327in}}%
\pgfpathcurveto{\pgfqpoint{3.109747in}{2.179151in}}{\pgfqpoint{3.113019in}{2.187051in}}{\pgfqpoint{3.113019in}{2.195287in}}%
\pgfpathcurveto{\pgfqpoint{3.113019in}{2.203524in}}{\pgfqpoint{3.109747in}{2.211424in}}{\pgfqpoint{3.103923in}{2.217248in}}%
\pgfpathcurveto{\pgfqpoint{3.098099in}{2.223072in}}{\pgfqpoint{3.090199in}{2.226344in}}{\pgfqpoint{3.081963in}{2.226344in}}%
\pgfpathcurveto{\pgfqpoint{3.073727in}{2.226344in}}{\pgfqpoint{3.065827in}{2.223072in}}{\pgfqpoint{3.060003in}{2.217248in}}%
\pgfpathcurveto{\pgfqpoint{3.054179in}{2.211424in}}{\pgfqpoint{3.050906in}{2.203524in}}{\pgfqpoint{3.050906in}{2.195287in}}%
\pgfpathcurveto{\pgfqpoint{3.050906in}{2.187051in}}{\pgfqpoint{3.054179in}{2.179151in}}{\pgfqpoint{3.060003in}{2.173327in}}%
\pgfpathcurveto{\pgfqpoint{3.065827in}{2.167503in}}{\pgfqpoint{3.073727in}{2.164231in}}{\pgfqpoint{3.081963in}{2.164231in}}%
\pgfpathclose%
\pgfusepath{stroke,fill}%
\end{pgfscope}%
\begin{pgfscope}%
\pgfpathrectangle{\pgfqpoint{0.100000in}{0.212622in}}{\pgfqpoint{3.696000in}{3.696000in}}%
\pgfusepath{clip}%
\pgfsetbuttcap%
\pgfsetroundjoin%
\definecolor{currentfill}{rgb}{0.121569,0.466667,0.705882}%
\pgfsetfillcolor{currentfill}%
\pgfsetfillopacity{0.708797}%
\pgfsetlinewidth{1.003750pt}%
\definecolor{currentstroke}{rgb}{0.121569,0.466667,0.705882}%
\pgfsetstrokecolor{currentstroke}%
\pgfsetstrokeopacity{0.708797}%
\pgfsetdash{}{0pt}%
\pgfpathmoveto{\pgfqpoint{3.077137in}{2.159858in}}%
\pgfpathcurveto{\pgfqpoint{3.085374in}{2.159858in}}{\pgfqpoint{3.093274in}{2.163131in}}{\pgfqpoint{3.099098in}{2.168955in}}%
\pgfpathcurveto{\pgfqpoint{3.104922in}{2.174779in}}{\pgfqpoint{3.108194in}{2.182679in}}{\pgfqpoint{3.108194in}{2.190915in}}%
\pgfpathcurveto{\pgfqpoint{3.108194in}{2.199151in}}{\pgfqpoint{3.104922in}{2.207051in}}{\pgfqpoint{3.099098in}{2.212875in}}%
\pgfpathcurveto{\pgfqpoint{3.093274in}{2.218699in}}{\pgfqpoint{3.085374in}{2.221971in}}{\pgfqpoint{3.077137in}{2.221971in}}%
\pgfpathcurveto{\pgfqpoint{3.068901in}{2.221971in}}{\pgfqpoint{3.061001in}{2.218699in}}{\pgfqpoint{3.055177in}{2.212875in}}%
\pgfpathcurveto{\pgfqpoint{3.049353in}{2.207051in}}{\pgfqpoint{3.046081in}{2.199151in}}{\pgfqpoint{3.046081in}{2.190915in}}%
\pgfpathcurveto{\pgfqpoint{3.046081in}{2.182679in}}{\pgfqpoint{3.049353in}{2.174779in}}{\pgfqpoint{3.055177in}{2.168955in}}%
\pgfpathcurveto{\pgfqpoint{3.061001in}{2.163131in}}{\pgfqpoint{3.068901in}{2.159858in}}{\pgfqpoint{3.077137in}{2.159858in}}%
\pgfpathclose%
\pgfusepath{stroke,fill}%
\end{pgfscope}%
\begin{pgfscope}%
\pgfpathrectangle{\pgfqpoint{0.100000in}{0.212622in}}{\pgfqpoint{3.696000in}{3.696000in}}%
\pgfusepath{clip}%
\pgfsetbuttcap%
\pgfsetroundjoin%
\definecolor{currentfill}{rgb}{0.121569,0.466667,0.705882}%
\pgfsetfillcolor{currentfill}%
\pgfsetfillopacity{0.712157}%
\pgfsetlinewidth{1.003750pt}%
\definecolor{currentstroke}{rgb}{0.121569,0.466667,0.705882}%
\pgfsetstrokecolor{currentstroke}%
\pgfsetstrokeopacity{0.712157}%
\pgfsetdash{}{0pt}%
\pgfpathmoveto{\pgfqpoint{3.069423in}{2.158217in}}%
\pgfpathcurveto{\pgfqpoint{3.077659in}{2.158217in}}{\pgfqpoint{3.085559in}{2.161489in}}{\pgfqpoint{3.091383in}{2.167313in}}%
\pgfpathcurveto{\pgfqpoint{3.097207in}{2.173137in}}{\pgfqpoint{3.100480in}{2.181037in}}{\pgfqpoint{3.100480in}{2.189273in}}%
\pgfpathcurveto{\pgfqpoint{3.100480in}{2.197510in}}{\pgfqpoint{3.097207in}{2.205410in}}{\pgfqpoint{3.091383in}{2.211234in}}%
\pgfpathcurveto{\pgfqpoint{3.085559in}{2.217058in}}{\pgfqpoint{3.077659in}{2.220330in}}{\pgfqpoint{3.069423in}{2.220330in}}%
\pgfpathcurveto{\pgfqpoint{3.061187in}{2.220330in}}{\pgfqpoint{3.053287in}{2.217058in}}{\pgfqpoint{3.047463in}{2.211234in}}%
\pgfpathcurveto{\pgfqpoint{3.041639in}{2.205410in}}{\pgfqpoint{3.038367in}{2.197510in}}{\pgfqpoint{3.038367in}{2.189273in}}%
\pgfpathcurveto{\pgfqpoint{3.038367in}{2.181037in}}{\pgfqpoint{3.041639in}{2.173137in}}{\pgfqpoint{3.047463in}{2.167313in}}%
\pgfpathcurveto{\pgfqpoint{3.053287in}{2.161489in}}{\pgfqpoint{3.061187in}{2.158217in}}{\pgfqpoint{3.069423in}{2.158217in}}%
\pgfpathclose%
\pgfusepath{stroke,fill}%
\end{pgfscope}%
\begin{pgfscope}%
\pgfpathrectangle{\pgfqpoint{0.100000in}{0.212622in}}{\pgfqpoint{3.696000in}{3.696000in}}%
\pgfusepath{clip}%
\pgfsetbuttcap%
\pgfsetroundjoin%
\definecolor{currentfill}{rgb}{0.121569,0.466667,0.705882}%
\pgfsetfillcolor{currentfill}%
\pgfsetfillopacity{0.716534}%
\pgfsetlinewidth{1.003750pt}%
\definecolor{currentstroke}{rgb}{0.121569,0.466667,0.705882}%
\pgfsetstrokecolor{currentstroke}%
\pgfsetstrokeopacity{0.716534}%
\pgfsetdash{}{0pt}%
\pgfpathmoveto{\pgfqpoint{3.059546in}{2.153338in}}%
\pgfpathcurveto{\pgfqpoint{3.067782in}{2.153338in}}{\pgfqpoint{3.075682in}{2.156610in}}{\pgfqpoint{3.081506in}{2.162434in}}%
\pgfpathcurveto{\pgfqpoint{3.087330in}{2.168258in}}{\pgfqpoint{3.090602in}{2.176158in}}{\pgfqpoint{3.090602in}{2.184394in}}%
\pgfpathcurveto{\pgfqpoint{3.090602in}{2.192631in}}{\pgfqpoint{3.087330in}{2.200531in}}{\pgfqpoint{3.081506in}{2.206355in}}%
\pgfpathcurveto{\pgfqpoint{3.075682in}{2.212179in}}{\pgfqpoint{3.067782in}{2.215451in}}{\pgfqpoint{3.059546in}{2.215451in}}%
\pgfpathcurveto{\pgfqpoint{3.051309in}{2.215451in}}{\pgfqpoint{3.043409in}{2.212179in}}{\pgfqpoint{3.037585in}{2.206355in}}%
\pgfpathcurveto{\pgfqpoint{3.031761in}{2.200531in}}{\pgfqpoint{3.028489in}{2.192631in}}{\pgfqpoint{3.028489in}{2.184394in}}%
\pgfpathcurveto{\pgfqpoint{3.028489in}{2.176158in}}{\pgfqpoint{3.031761in}{2.168258in}}{\pgfqpoint{3.037585in}{2.162434in}}%
\pgfpathcurveto{\pgfqpoint{3.043409in}{2.156610in}}{\pgfqpoint{3.051309in}{2.153338in}}{\pgfqpoint{3.059546in}{2.153338in}}%
\pgfpathclose%
\pgfusepath{stroke,fill}%
\end{pgfscope}%
\begin{pgfscope}%
\pgfpathrectangle{\pgfqpoint{0.100000in}{0.212622in}}{\pgfqpoint{3.696000in}{3.696000in}}%
\pgfusepath{clip}%
\pgfsetbuttcap%
\pgfsetroundjoin%
\definecolor{currentfill}{rgb}{0.121569,0.466667,0.705882}%
\pgfsetfillcolor{currentfill}%
\pgfsetfillopacity{0.721067}%
\pgfsetlinewidth{1.003750pt}%
\definecolor{currentstroke}{rgb}{0.121569,0.466667,0.705882}%
\pgfsetstrokecolor{currentstroke}%
\pgfsetstrokeopacity{0.721067}%
\pgfsetdash{}{0pt}%
\pgfpathmoveto{\pgfqpoint{3.045755in}{2.148839in}}%
\pgfpathcurveto{\pgfqpoint{3.053992in}{2.148839in}}{\pgfqpoint{3.061892in}{2.152111in}}{\pgfqpoint{3.067716in}{2.157935in}}%
\pgfpathcurveto{\pgfqpoint{3.073540in}{2.163759in}}{\pgfqpoint{3.076812in}{2.171659in}}{\pgfqpoint{3.076812in}{2.179895in}}%
\pgfpathcurveto{\pgfqpoint{3.076812in}{2.188132in}}{\pgfqpoint{3.073540in}{2.196032in}}{\pgfqpoint{3.067716in}{2.201856in}}%
\pgfpathcurveto{\pgfqpoint{3.061892in}{2.207680in}}{\pgfqpoint{3.053992in}{2.210952in}}{\pgfqpoint{3.045755in}{2.210952in}}%
\pgfpathcurveto{\pgfqpoint{3.037519in}{2.210952in}}{\pgfqpoint{3.029619in}{2.207680in}}{\pgfqpoint{3.023795in}{2.201856in}}%
\pgfpathcurveto{\pgfqpoint{3.017971in}{2.196032in}}{\pgfqpoint{3.014699in}{2.188132in}}{\pgfqpoint{3.014699in}{2.179895in}}%
\pgfpathcurveto{\pgfqpoint{3.014699in}{2.171659in}}{\pgfqpoint{3.017971in}{2.163759in}}{\pgfqpoint{3.023795in}{2.157935in}}%
\pgfpathcurveto{\pgfqpoint{3.029619in}{2.152111in}}{\pgfqpoint{3.037519in}{2.148839in}}{\pgfqpoint{3.045755in}{2.148839in}}%
\pgfpathclose%
\pgfusepath{stroke,fill}%
\end{pgfscope}%
\begin{pgfscope}%
\pgfpathrectangle{\pgfqpoint{0.100000in}{0.212622in}}{\pgfqpoint{3.696000in}{3.696000in}}%
\pgfusepath{clip}%
\pgfsetbuttcap%
\pgfsetroundjoin%
\definecolor{currentfill}{rgb}{0.121569,0.466667,0.705882}%
\pgfsetfillcolor{currentfill}%
\pgfsetfillopacity{0.723909}%
\pgfsetlinewidth{1.003750pt}%
\definecolor{currentstroke}{rgb}{0.121569,0.466667,0.705882}%
\pgfsetstrokecolor{currentstroke}%
\pgfsetstrokeopacity{0.723909}%
\pgfsetdash{}{0pt}%
\pgfpathmoveto{\pgfqpoint{3.039477in}{2.146687in}}%
\pgfpathcurveto{\pgfqpoint{3.047713in}{2.146687in}}{\pgfqpoint{3.055613in}{2.149960in}}{\pgfqpoint{3.061437in}{2.155783in}}%
\pgfpathcurveto{\pgfqpoint{3.067261in}{2.161607in}}{\pgfqpoint{3.070533in}{2.169507in}}{\pgfqpoint{3.070533in}{2.177744in}}%
\pgfpathcurveto{\pgfqpoint{3.070533in}{2.185980in}}{\pgfqpoint{3.067261in}{2.193880in}}{\pgfqpoint{3.061437in}{2.199704in}}%
\pgfpathcurveto{\pgfqpoint{3.055613in}{2.205528in}}{\pgfqpoint{3.047713in}{2.208800in}}{\pgfqpoint{3.039477in}{2.208800in}}%
\pgfpathcurveto{\pgfqpoint{3.031241in}{2.208800in}}{\pgfqpoint{3.023341in}{2.205528in}}{\pgfqpoint{3.017517in}{2.199704in}}%
\pgfpathcurveto{\pgfqpoint{3.011693in}{2.193880in}}{\pgfqpoint{3.008420in}{2.185980in}}{\pgfqpoint{3.008420in}{2.177744in}}%
\pgfpathcurveto{\pgfqpoint{3.008420in}{2.169507in}}{\pgfqpoint{3.011693in}{2.161607in}}{\pgfqpoint{3.017517in}{2.155783in}}%
\pgfpathcurveto{\pgfqpoint{3.023341in}{2.149960in}}{\pgfqpoint{3.031241in}{2.146687in}}{\pgfqpoint{3.039477in}{2.146687in}}%
\pgfpathclose%
\pgfusepath{stroke,fill}%
\end{pgfscope}%
\begin{pgfscope}%
\pgfpathrectangle{\pgfqpoint{0.100000in}{0.212622in}}{\pgfqpoint{3.696000in}{3.696000in}}%
\pgfusepath{clip}%
\pgfsetbuttcap%
\pgfsetroundjoin%
\definecolor{currentfill}{rgb}{0.121569,0.466667,0.705882}%
\pgfsetfillcolor{currentfill}%
\pgfsetfillopacity{0.725382}%
\pgfsetlinewidth{1.003750pt}%
\definecolor{currentstroke}{rgb}{0.121569,0.466667,0.705882}%
\pgfsetstrokecolor{currentstroke}%
\pgfsetstrokeopacity{0.725382}%
\pgfsetdash{}{0pt}%
\pgfpathmoveto{\pgfqpoint{3.035990in}{2.144923in}}%
\pgfpathcurveto{\pgfqpoint{3.044227in}{2.144923in}}{\pgfqpoint{3.052127in}{2.148195in}}{\pgfqpoint{3.057951in}{2.154019in}}%
\pgfpathcurveto{\pgfqpoint{3.063775in}{2.159843in}}{\pgfqpoint{3.067047in}{2.167743in}}{\pgfqpoint{3.067047in}{2.175980in}}%
\pgfpathcurveto{\pgfqpoint{3.067047in}{2.184216in}}{\pgfqpoint{3.063775in}{2.192116in}}{\pgfqpoint{3.057951in}{2.197940in}}%
\pgfpathcurveto{\pgfqpoint{3.052127in}{2.203764in}}{\pgfqpoint{3.044227in}{2.207036in}}{\pgfqpoint{3.035990in}{2.207036in}}%
\pgfpathcurveto{\pgfqpoint{3.027754in}{2.207036in}}{\pgfqpoint{3.019854in}{2.203764in}}{\pgfqpoint{3.014030in}{2.197940in}}%
\pgfpathcurveto{\pgfqpoint{3.008206in}{2.192116in}}{\pgfqpoint{3.004934in}{2.184216in}}{\pgfqpoint{3.004934in}{2.175980in}}%
\pgfpathcurveto{\pgfqpoint{3.004934in}{2.167743in}}{\pgfqpoint{3.008206in}{2.159843in}}{\pgfqpoint{3.014030in}{2.154019in}}%
\pgfpathcurveto{\pgfqpoint{3.019854in}{2.148195in}}{\pgfqpoint{3.027754in}{2.144923in}}{\pgfqpoint{3.035990in}{2.144923in}}%
\pgfpathclose%
\pgfusepath{stroke,fill}%
\end{pgfscope}%
\begin{pgfscope}%
\pgfpathrectangle{\pgfqpoint{0.100000in}{0.212622in}}{\pgfqpoint{3.696000in}{3.696000in}}%
\pgfusepath{clip}%
\pgfsetbuttcap%
\pgfsetroundjoin%
\definecolor{currentfill}{rgb}{0.121569,0.466667,0.705882}%
\pgfsetfillcolor{currentfill}%
\pgfsetfillopacity{0.726344}%
\pgfsetlinewidth{1.003750pt}%
\definecolor{currentstroke}{rgb}{0.121569,0.466667,0.705882}%
\pgfsetstrokecolor{currentstroke}%
\pgfsetstrokeopacity{0.726344}%
\pgfsetdash{}{0pt}%
\pgfpathmoveto{\pgfqpoint{3.034087in}{2.144950in}}%
\pgfpathcurveto{\pgfqpoint{3.042323in}{2.144950in}}{\pgfqpoint{3.050223in}{2.148222in}}{\pgfqpoint{3.056047in}{2.154046in}}%
\pgfpathcurveto{\pgfqpoint{3.061871in}{2.159870in}}{\pgfqpoint{3.065143in}{2.167770in}}{\pgfqpoint{3.065143in}{2.176006in}}%
\pgfpathcurveto{\pgfqpoint{3.065143in}{2.184243in}}{\pgfqpoint{3.061871in}{2.192143in}}{\pgfqpoint{3.056047in}{2.197967in}}%
\pgfpathcurveto{\pgfqpoint{3.050223in}{2.203791in}}{\pgfqpoint{3.042323in}{2.207063in}}{\pgfqpoint{3.034087in}{2.207063in}}%
\pgfpathcurveto{\pgfqpoint{3.025851in}{2.207063in}}{\pgfqpoint{3.017951in}{2.203791in}}{\pgfqpoint{3.012127in}{2.197967in}}%
\pgfpathcurveto{\pgfqpoint{3.006303in}{2.192143in}}{\pgfqpoint{3.003030in}{2.184243in}}{\pgfqpoint{3.003030in}{2.176006in}}%
\pgfpathcurveto{\pgfqpoint{3.003030in}{2.167770in}}{\pgfqpoint{3.006303in}{2.159870in}}{\pgfqpoint{3.012127in}{2.154046in}}%
\pgfpathcurveto{\pgfqpoint{3.017951in}{2.148222in}}{\pgfqpoint{3.025851in}{2.144950in}}{\pgfqpoint{3.034087in}{2.144950in}}%
\pgfpathclose%
\pgfusepath{stroke,fill}%
\end{pgfscope}%
\begin{pgfscope}%
\pgfpathrectangle{\pgfqpoint{0.100000in}{0.212622in}}{\pgfqpoint{3.696000in}{3.696000in}}%
\pgfusepath{clip}%
\pgfsetbuttcap%
\pgfsetroundjoin%
\definecolor{currentfill}{rgb}{0.121569,0.466667,0.705882}%
\pgfsetfillcolor{currentfill}%
\pgfsetfillopacity{0.726822}%
\pgfsetlinewidth{1.003750pt}%
\definecolor{currentstroke}{rgb}{0.121569,0.466667,0.705882}%
\pgfsetstrokecolor{currentstroke}%
\pgfsetstrokeopacity{0.726822}%
\pgfsetdash{}{0pt}%
\pgfpathmoveto{\pgfqpoint{3.033106in}{2.144528in}}%
\pgfpathcurveto{\pgfqpoint{3.041343in}{2.144528in}}{\pgfqpoint{3.049243in}{2.147800in}}{\pgfqpoint{3.055067in}{2.153624in}}%
\pgfpathcurveto{\pgfqpoint{3.060891in}{2.159448in}}{\pgfqpoint{3.064163in}{2.167348in}}{\pgfqpoint{3.064163in}{2.175585in}}%
\pgfpathcurveto{\pgfqpoint{3.064163in}{2.183821in}}{\pgfqpoint{3.060891in}{2.191721in}}{\pgfqpoint{3.055067in}{2.197545in}}%
\pgfpathcurveto{\pgfqpoint{3.049243in}{2.203369in}}{\pgfqpoint{3.041343in}{2.206641in}}{\pgfqpoint{3.033106in}{2.206641in}}%
\pgfpathcurveto{\pgfqpoint{3.024870in}{2.206641in}}{\pgfqpoint{3.016970in}{2.203369in}}{\pgfqpoint{3.011146in}{2.197545in}}%
\pgfpathcurveto{\pgfqpoint{3.005322in}{2.191721in}}{\pgfqpoint{3.002050in}{2.183821in}}{\pgfqpoint{3.002050in}{2.175585in}}%
\pgfpathcurveto{\pgfqpoint{3.002050in}{2.167348in}}{\pgfqpoint{3.005322in}{2.159448in}}{\pgfqpoint{3.011146in}{2.153624in}}%
\pgfpathcurveto{\pgfqpoint{3.016970in}{2.147800in}}{\pgfqpoint{3.024870in}{2.144528in}}{\pgfqpoint{3.033106in}{2.144528in}}%
\pgfpathclose%
\pgfusepath{stroke,fill}%
\end{pgfscope}%
\begin{pgfscope}%
\pgfpathrectangle{\pgfqpoint{0.100000in}{0.212622in}}{\pgfqpoint{3.696000in}{3.696000in}}%
\pgfusepath{clip}%
\pgfsetbuttcap%
\pgfsetroundjoin%
\definecolor{currentfill}{rgb}{0.121569,0.466667,0.705882}%
\pgfsetfillcolor{currentfill}%
\pgfsetfillopacity{0.727082}%
\pgfsetlinewidth{1.003750pt}%
\definecolor{currentstroke}{rgb}{0.121569,0.466667,0.705882}%
\pgfsetstrokecolor{currentstroke}%
\pgfsetstrokeopacity{0.727082}%
\pgfsetdash{}{0pt}%
\pgfpathmoveto{\pgfqpoint{3.032435in}{2.144480in}}%
\pgfpathcurveto{\pgfqpoint{3.040671in}{2.144480in}}{\pgfqpoint{3.048571in}{2.147752in}}{\pgfqpoint{3.054395in}{2.153576in}}%
\pgfpathcurveto{\pgfqpoint{3.060219in}{2.159400in}}{\pgfqpoint{3.063492in}{2.167300in}}{\pgfqpoint{3.063492in}{2.175536in}}%
\pgfpathcurveto{\pgfqpoint{3.063492in}{2.183773in}}{\pgfqpoint{3.060219in}{2.191673in}}{\pgfqpoint{3.054395in}{2.197497in}}%
\pgfpathcurveto{\pgfqpoint{3.048571in}{2.203320in}}{\pgfqpoint{3.040671in}{2.206593in}}{\pgfqpoint{3.032435in}{2.206593in}}%
\pgfpathcurveto{\pgfqpoint{3.024199in}{2.206593in}}{\pgfqpoint{3.016299in}{2.203320in}}{\pgfqpoint{3.010475in}{2.197497in}}%
\pgfpathcurveto{\pgfqpoint{3.004651in}{2.191673in}}{\pgfqpoint{3.001379in}{2.183773in}}{\pgfqpoint{3.001379in}{2.175536in}}%
\pgfpathcurveto{\pgfqpoint{3.001379in}{2.167300in}}{\pgfqpoint{3.004651in}{2.159400in}}{\pgfqpoint{3.010475in}{2.153576in}}%
\pgfpathcurveto{\pgfqpoint{3.016299in}{2.147752in}}{\pgfqpoint{3.024199in}{2.144480in}}{\pgfqpoint{3.032435in}{2.144480in}}%
\pgfpathclose%
\pgfusepath{stroke,fill}%
\end{pgfscope}%
\begin{pgfscope}%
\pgfpathrectangle{\pgfqpoint{0.100000in}{0.212622in}}{\pgfqpoint{3.696000in}{3.696000in}}%
\pgfusepath{clip}%
\pgfsetbuttcap%
\pgfsetroundjoin%
\definecolor{currentfill}{rgb}{0.121569,0.466667,0.705882}%
\pgfsetfillcolor{currentfill}%
\pgfsetfillopacity{0.728112}%
\pgfsetlinewidth{1.003750pt}%
\definecolor{currentstroke}{rgb}{0.121569,0.466667,0.705882}%
\pgfsetstrokecolor{currentstroke}%
\pgfsetstrokeopacity{0.728112}%
\pgfsetdash{}{0pt}%
\pgfpathmoveto{\pgfqpoint{3.030421in}{2.142307in}}%
\pgfpathcurveto{\pgfqpoint{3.038657in}{2.142307in}}{\pgfqpoint{3.046557in}{2.145580in}}{\pgfqpoint{3.052381in}{2.151404in}}%
\pgfpathcurveto{\pgfqpoint{3.058205in}{2.157227in}}{\pgfqpoint{3.061478in}{2.165128in}}{\pgfqpoint{3.061478in}{2.173364in}}%
\pgfpathcurveto{\pgfqpoint{3.061478in}{2.181600in}}{\pgfqpoint{3.058205in}{2.189500in}}{\pgfqpoint{3.052381in}{2.195324in}}%
\pgfpathcurveto{\pgfqpoint{3.046557in}{2.201148in}}{\pgfqpoint{3.038657in}{2.204420in}}{\pgfqpoint{3.030421in}{2.204420in}}%
\pgfpathcurveto{\pgfqpoint{3.022185in}{2.204420in}}{\pgfqpoint{3.014285in}{2.201148in}}{\pgfqpoint{3.008461in}{2.195324in}}%
\pgfpathcurveto{\pgfqpoint{3.002637in}{2.189500in}}{\pgfqpoint{2.999365in}{2.181600in}}{\pgfqpoint{2.999365in}{2.173364in}}%
\pgfpathcurveto{\pgfqpoint{2.999365in}{2.165128in}}{\pgfqpoint{3.002637in}{2.157227in}}{\pgfqpoint{3.008461in}{2.151404in}}%
\pgfpathcurveto{\pgfqpoint{3.014285in}{2.145580in}}{\pgfqpoint{3.022185in}{2.142307in}}{\pgfqpoint{3.030421in}{2.142307in}}%
\pgfpathclose%
\pgfusepath{stroke,fill}%
\end{pgfscope}%
\begin{pgfscope}%
\pgfpathrectangle{\pgfqpoint{0.100000in}{0.212622in}}{\pgfqpoint{3.696000in}{3.696000in}}%
\pgfusepath{clip}%
\pgfsetbuttcap%
\pgfsetroundjoin%
\definecolor{currentfill}{rgb}{0.121569,0.466667,0.705882}%
\pgfsetfillcolor{currentfill}%
\pgfsetfillopacity{0.728729}%
\pgfsetlinewidth{1.003750pt}%
\definecolor{currentstroke}{rgb}{0.121569,0.466667,0.705882}%
\pgfsetstrokecolor{currentstroke}%
\pgfsetstrokeopacity{0.728729}%
\pgfsetdash{}{0pt}%
\pgfpathmoveto{\pgfqpoint{3.028801in}{2.142114in}}%
\pgfpathcurveto{\pgfqpoint{3.037037in}{2.142114in}}{\pgfqpoint{3.044937in}{2.145387in}}{\pgfqpoint{3.050761in}{2.151211in}}%
\pgfpathcurveto{\pgfqpoint{3.056585in}{2.157035in}}{\pgfqpoint{3.059857in}{2.164935in}}{\pgfqpoint{3.059857in}{2.173171in}}%
\pgfpathcurveto{\pgfqpoint{3.059857in}{2.181407in}}{\pgfqpoint{3.056585in}{2.189307in}}{\pgfqpoint{3.050761in}{2.195131in}}%
\pgfpathcurveto{\pgfqpoint{3.044937in}{2.200955in}}{\pgfqpoint{3.037037in}{2.204227in}}{\pgfqpoint{3.028801in}{2.204227in}}%
\pgfpathcurveto{\pgfqpoint{3.020565in}{2.204227in}}{\pgfqpoint{3.012665in}{2.200955in}}{\pgfqpoint{3.006841in}{2.195131in}}%
\pgfpathcurveto{\pgfqpoint{3.001017in}{2.189307in}}{\pgfqpoint{2.997744in}{2.181407in}}{\pgfqpoint{2.997744in}{2.173171in}}%
\pgfpathcurveto{\pgfqpoint{2.997744in}{2.164935in}}{\pgfqpoint{3.001017in}{2.157035in}}{\pgfqpoint{3.006841in}{2.151211in}}%
\pgfpathcurveto{\pgfqpoint{3.012665in}{2.145387in}}{\pgfqpoint{3.020565in}{2.142114in}}{\pgfqpoint{3.028801in}{2.142114in}}%
\pgfpathclose%
\pgfusepath{stroke,fill}%
\end{pgfscope}%
\begin{pgfscope}%
\pgfpathrectangle{\pgfqpoint{0.100000in}{0.212622in}}{\pgfqpoint{3.696000in}{3.696000in}}%
\pgfusepath{clip}%
\pgfsetbuttcap%
\pgfsetroundjoin%
\definecolor{currentfill}{rgb}{0.121569,0.466667,0.705882}%
\pgfsetfillcolor{currentfill}%
\pgfsetfillopacity{0.730366}%
\pgfsetlinewidth{1.003750pt}%
\definecolor{currentstroke}{rgb}{0.121569,0.466667,0.705882}%
\pgfsetstrokecolor{currentstroke}%
\pgfsetstrokeopacity{0.730366}%
\pgfsetdash{}{0pt}%
\pgfpathmoveto{\pgfqpoint{3.025859in}{2.138219in}}%
\pgfpathcurveto{\pgfqpoint{3.034095in}{2.138219in}}{\pgfqpoint{3.041995in}{2.141492in}}{\pgfqpoint{3.047819in}{2.147316in}}%
\pgfpathcurveto{\pgfqpoint{3.053643in}{2.153139in}}{\pgfqpoint{3.056915in}{2.161039in}}{\pgfqpoint{3.056915in}{2.169276in}}%
\pgfpathcurveto{\pgfqpoint{3.056915in}{2.177512in}}{\pgfqpoint{3.053643in}{2.185412in}}{\pgfqpoint{3.047819in}{2.191236in}}%
\pgfpathcurveto{\pgfqpoint{3.041995in}{2.197060in}}{\pgfqpoint{3.034095in}{2.200332in}}{\pgfqpoint{3.025859in}{2.200332in}}%
\pgfpathcurveto{\pgfqpoint{3.017623in}{2.200332in}}{\pgfqpoint{3.009722in}{2.197060in}}{\pgfqpoint{3.003899in}{2.191236in}}%
\pgfpathcurveto{\pgfqpoint{2.998075in}{2.185412in}}{\pgfqpoint{2.994802in}{2.177512in}}{\pgfqpoint{2.994802in}{2.169276in}}%
\pgfpathcurveto{\pgfqpoint{2.994802in}{2.161039in}}{\pgfqpoint{2.998075in}{2.153139in}}{\pgfqpoint{3.003899in}{2.147316in}}%
\pgfpathcurveto{\pgfqpoint{3.009722in}{2.141492in}}{\pgfqpoint{3.017623in}{2.138219in}}{\pgfqpoint{3.025859in}{2.138219in}}%
\pgfpathclose%
\pgfusepath{stroke,fill}%
\end{pgfscope}%
\begin{pgfscope}%
\pgfpathrectangle{\pgfqpoint{0.100000in}{0.212622in}}{\pgfqpoint{3.696000in}{3.696000in}}%
\pgfusepath{clip}%
\pgfsetbuttcap%
\pgfsetroundjoin%
\definecolor{currentfill}{rgb}{0.121569,0.466667,0.705882}%
\pgfsetfillcolor{currentfill}%
\pgfsetfillopacity{0.732444}%
\pgfsetlinewidth{1.003750pt}%
\definecolor{currentstroke}{rgb}{0.121569,0.466667,0.705882}%
\pgfsetstrokecolor{currentstroke}%
\pgfsetstrokeopacity{0.732444}%
\pgfsetdash{}{0pt}%
\pgfpathmoveto{\pgfqpoint{3.018903in}{2.136116in}}%
\pgfpathcurveto{\pgfqpoint{3.027139in}{2.136116in}}{\pgfqpoint{3.035040in}{2.139389in}}{\pgfqpoint{3.040863in}{2.145213in}}%
\pgfpathcurveto{\pgfqpoint{3.046687in}{2.151036in}}{\pgfqpoint{3.049960in}{2.158937in}}{\pgfqpoint{3.049960in}{2.167173in}}%
\pgfpathcurveto{\pgfqpoint{3.049960in}{2.175409in}}{\pgfqpoint{3.046687in}{2.183309in}}{\pgfqpoint{3.040863in}{2.189133in}}%
\pgfpathcurveto{\pgfqpoint{3.035040in}{2.194957in}}{\pgfqpoint{3.027139in}{2.198229in}}{\pgfqpoint{3.018903in}{2.198229in}}%
\pgfpathcurveto{\pgfqpoint{3.010667in}{2.198229in}}{\pgfqpoint{3.002767in}{2.194957in}}{\pgfqpoint{2.996943in}{2.189133in}}%
\pgfpathcurveto{\pgfqpoint{2.991119in}{2.183309in}}{\pgfqpoint{2.987847in}{2.175409in}}{\pgfqpoint{2.987847in}{2.167173in}}%
\pgfpathcurveto{\pgfqpoint{2.987847in}{2.158937in}}{\pgfqpoint{2.991119in}{2.151036in}}{\pgfqpoint{2.996943in}{2.145213in}}%
\pgfpathcurveto{\pgfqpoint{3.002767in}{2.139389in}}{\pgfqpoint{3.010667in}{2.136116in}}{\pgfqpoint{3.018903in}{2.136116in}}%
\pgfpathclose%
\pgfusepath{stroke,fill}%
\end{pgfscope}%
\begin{pgfscope}%
\pgfpathrectangle{\pgfqpoint{0.100000in}{0.212622in}}{\pgfqpoint{3.696000in}{3.696000in}}%
\pgfusepath{clip}%
\pgfsetbuttcap%
\pgfsetroundjoin%
\definecolor{currentfill}{rgb}{0.121569,0.466667,0.705882}%
\pgfsetfillcolor{currentfill}%
\pgfsetfillopacity{0.735669}%
\pgfsetlinewidth{1.003750pt}%
\definecolor{currentstroke}{rgb}{0.121569,0.466667,0.705882}%
\pgfsetstrokecolor{currentstroke}%
\pgfsetstrokeopacity{0.735669}%
\pgfsetdash{}{0pt}%
\pgfpathmoveto{\pgfqpoint{3.012668in}{2.129516in}}%
\pgfpathcurveto{\pgfqpoint{3.020904in}{2.129516in}}{\pgfqpoint{3.028804in}{2.132788in}}{\pgfqpoint{3.034628in}{2.138612in}}%
\pgfpathcurveto{\pgfqpoint{3.040452in}{2.144436in}}{\pgfqpoint{3.043724in}{2.152336in}}{\pgfqpoint{3.043724in}{2.160572in}}%
\pgfpathcurveto{\pgfqpoint{3.043724in}{2.168809in}}{\pgfqpoint{3.040452in}{2.176709in}}{\pgfqpoint{3.034628in}{2.182533in}}%
\pgfpathcurveto{\pgfqpoint{3.028804in}{2.188357in}}{\pgfqpoint{3.020904in}{2.191629in}}{\pgfqpoint{3.012668in}{2.191629in}}%
\pgfpathcurveto{\pgfqpoint{3.004431in}{2.191629in}}{\pgfqpoint{2.996531in}{2.188357in}}{\pgfqpoint{2.990707in}{2.182533in}}%
\pgfpathcurveto{\pgfqpoint{2.984883in}{2.176709in}}{\pgfqpoint{2.981611in}{2.168809in}}{\pgfqpoint{2.981611in}{2.160572in}}%
\pgfpathcurveto{\pgfqpoint{2.981611in}{2.152336in}}{\pgfqpoint{2.984883in}{2.144436in}}{\pgfqpoint{2.990707in}{2.138612in}}%
\pgfpathcurveto{\pgfqpoint{2.996531in}{2.132788in}}{\pgfqpoint{3.004431in}{2.129516in}}{\pgfqpoint{3.012668in}{2.129516in}}%
\pgfpathclose%
\pgfusepath{stroke,fill}%
\end{pgfscope}%
\begin{pgfscope}%
\pgfpathrectangle{\pgfqpoint{0.100000in}{0.212622in}}{\pgfqpoint{3.696000in}{3.696000in}}%
\pgfusepath{clip}%
\pgfsetbuttcap%
\pgfsetroundjoin%
\definecolor{currentfill}{rgb}{0.121569,0.466667,0.705882}%
\pgfsetfillcolor{currentfill}%
\pgfsetfillopacity{0.739656}%
\pgfsetlinewidth{1.003750pt}%
\definecolor{currentstroke}{rgb}{0.121569,0.466667,0.705882}%
\pgfsetstrokecolor{currentstroke}%
\pgfsetstrokeopacity{0.739656}%
\pgfsetdash{}{0pt}%
\pgfpathmoveto{\pgfqpoint{2.999911in}{2.125222in}}%
\pgfpathcurveto{\pgfqpoint{3.008147in}{2.125222in}}{\pgfqpoint{3.016047in}{2.128494in}}{\pgfqpoint{3.021871in}{2.134318in}}%
\pgfpathcurveto{\pgfqpoint{3.027695in}{2.140142in}}{\pgfqpoint{3.030967in}{2.148042in}}{\pgfqpoint{3.030967in}{2.156279in}}%
\pgfpathcurveto{\pgfqpoint{3.030967in}{2.164515in}}{\pgfqpoint{3.027695in}{2.172415in}}{\pgfqpoint{3.021871in}{2.178239in}}%
\pgfpathcurveto{\pgfqpoint{3.016047in}{2.184063in}}{\pgfqpoint{3.008147in}{2.187335in}}{\pgfqpoint{2.999911in}{2.187335in}}%
\pgfpathcurveto{\pgfqpoint{2.991675in}{2.187335in}}{\pgfqpoint{2.983775in}{2.184063in}}{\pgfqpoint{2.977951in}{2.178239in}}%
\pgfpathcurveto{\pgfqpoint{2.972127in}{2.172415in}}{\pgfqpoint{2.968854in}{2.164515in}}{\pgfqpoint{2.968854in}{2.156279in}}%
\pgfpathcurveto{\pgfqpoint{2.968854in}{2.148042in}}{\pgfqpoint{2.972127in}{2.140142in}}{\pgfqpoint{2.977951in}{2.134318in}}%
\pgfpathcurveto{\pgfqpoint{2.983775in}{2.128494in}}{\pgfqpoint{2.991675in}{2.125222in}}{\pgfqpoint{2.999911in}{2.125222in}}%
\pgfpathclose%
\pgfusepath{stroke,fill}%
\end{pgfscope}%
\begin{pgfscope}%
\pgfpathrectangle{\pgfqpoint{0.100000in}{0.212622in}}{\pgfqpoint{3.696000in}{3.696000in}}%
\pgfusepath{clip}%
\pgfsetbuttcap%
\pgfsetroundjoin%
\definecolor{currentfill}{rgb}{0.121569,0.466667,0.705882}%
\pgfsetfillcolor{currentfill}%
\pgfsetfillopacity{0.745387}%
\pgfsetlinewidth{1.003750pt}%
\definecolor{currentstroke}{rgb}{0.121569,0.466667,0.705882}%
\pgfsetstrokecolor{currentstroke}%
\pgfsetstrokeopacity{0.745387}%
\pgfsetdash{}{0pt}%
\pgfpathmoveto{\pgfqpoint{2.990124in}{2.118903in}}%
\pgfpathcurveto{\pgfqpoint{2.998360in}{2.118903in}}{\pgfqpoint{3.006260in}{2.122176in}}{\pgfqpoint{3.012084in}{2.128000in}}%
\pgfpathcurveto{\pgfqpoint{3.017908in}{2.133823in}}{\pgfqpoint{3.021180in}{2.141724in}}{\pgfqpoint{3.021180in}{2.149960in}}%
\pgfpathcurveto{\pgfqpoint{3.021180in}{2.158196in}}{\pgfqpoint{3.017908in}{2.166096in}}{\pgfqpoint{3.012084in}{2.171920in}}%
\pgfpathcurveto{\pgfqpoint{3.006260in}{2.177744in}}{\pgfqpoint{2.998360in}{2.181016in}}{\pgfqpoint{2.990124in}{2.181016in}}%
\pgfpathcurveto{\pgfqpoint{2.981888in}{2.181016in}}{\pgfqpoint{2.973988in}{2.177744in}}{\pgfqpoint{2.968164in}{2.171920in}}%
\pgfpathcurveto{\pgfqpoint{2.962340in}{2.166096in}}{\pgfqpoint{2.959067in}{2.158196in}}{\pgfqpoint{2.959067in}{2.149960in}}%
\pgfpathcurveto{\pgfqpoint{2.959067in}{2.141724in}}{\pgfqpoint{2.962340in}{2.133823in}}{\pgfqpoint{2.968164in}{2.128000in}}%
\pgfpathcurveto{\pgfqpoint{2.973988in}{2.122176in}}{\pgfqpoint{2.981888in}{2.118903in}}{\pgfqpoint{2.990124in}{2.118903in}}%
\pgfpathclose%
\pgfusepath{stroke,fill}%
\end{pgfscope}%
\begin{pgfscope}%
\pgfpathrectangle{\pgfqpoint{0.100000in}{0.212622in}}{\pgfqpoint{3.696000in}{3.696000in}}%
\pgfusepath{clip}%
\pgfsetbuttcap%
\pgfsetroundjoin%
\definecolor{currentfill}{rgb}{0.121569,0.466667,0.705882}%
\pgfsetfillcolor{currentfill}%
\pgfsetfillopacity{0.748130}%
\pgfsetlinewidth{1.003750pt}%
\definecolor{currentstroke}{rgb}{0.121569,0.466667,0.705882}%
\pgfsetstrokecolor{currentstroke}%
\pgfsetstrokeopacity{0.748130}%
\pgfsetdash{}{0pt}%
\pgfpathmoveto{\pgfqpoint{2.982595in}{2.115322in}}%
\pgfpathcurveto{\pgfqpoint{2.990831in}{2.115322in}}{\pgfqpoint{2.998732in}{2.118595in}}{\pgfqpoint{3.004555in}{2.124419in}}%
\pgfpathcurveto{\pgfqpoint{3.010379in}{2.130243in}}{\pgfqpoint{3.013652in}{2.138143in}}{\pgfqpoint{3.013652in}{2.146379in}}%
\pgfpathcurveto{\pgfqpoint{3.013652in}{2.154615in}}{\pgfqpoint{3.010379in}{2.162515in}}{\pgfqpoint{3.004555in}{2.168339in}}%
\pgfpathcurveto{\pgfqpoint{2.998732in}{2.174163in}}{\pgfqpoint{2.990831in}{2.177435in}}{\pgfqpoint{2.982595in}{2.177435in}}%
\pgfpathcurveto{\pgfqpoint{2.974359in}{2.177435in}}{\pgfqpoint{2.966459in}{2.174163in}}{\pgfqpoint{2.960635in}{2.168339in}}%
\pgfpathcurveto{\pgfqpoint{2.954811in}{2.162515in}}{\pgfqpoint{2.951539in}{2.154615in}}{\pgfqpoint{2.951539in}{2.146379in}}%
\pgfpathcurveto{\pgfqpoint{2.951539in}{2.138143in}}{\pgfqpoint{2.954811in}{2.130243in}}{\pgfqpoint{2.960635in}{2.124419in}}%
\pgfpathcurveto{\pgfqpoint{2.966459in}{2.118595in}}{\pgfqpoint{2.974359in}{2.115322in}}{\pgfqpoint{2.982595in}{2.115322in}}%
\pgfpathclose%
\pgfusepath{stroke,fill}%
\end{pgfscope}%
\begin{pgfscope}%
\pgfpathrectangle{\pgfqpoint{0.100000in}{0.212622in}}{\pgfqpoint{3.696000in}{3.696000in}}%
\pgfusepath{clip}%
\pgfsetbuttcap%
\pgfsetroundjoin%
\definecolor{currentfill}{rgb}{0.121569,0.466667,0.705882}%
\pgfsetfillcolor{currentfill}%
\pgfsetfillopacity{0.749897}%
\pgfsetlinewidth{1.003750pt}%
\definecolor{currentstroke}{rgb}{0.121569,0.466667,0.705882}%
\pgfsetstrokecolor{currentstroke}%
\pgfsetstrokeopacity{0.749897}%
\pgfsetdash{}{0pt}%
\pgfpathmoveto{\pgfqpoint{2.979083in}{2.114250in}}%
\pgfpathcurveto{\pgfqpoint{2.987320in}{2.114250in}}{\pgfqpoint{2.995220in}{2.117522in}}{\pgfqpoint{3.001044in}{2.123346in}}%
\pgfpathcurveto{\pgfqpoint{3.006868in}{2.129170in}}{\pgfqpoint{3.010140in}{2.137070in}}{\pgfqpoint{3.010140in}{2.145306in}}%
\pgfpathcurveto{\pgfqpoint{3.010140in}{2.153542in}}{\pgfqpoint{3.006868in}{2.161442in}}{\pgfqpoint{3.001044in}{2.167266in}}%
\pgfpathcurveto{\pgfqpoint{2.995220in}{2.173090in}}{\pgfqpoint{2.987320in}{2.176363in}}{\pgfqpoint{2.979083in}{2.176363in}}%
\pgfpathcurveto{\pgfqpoint{2.970847in}{2.176363in}}{\pgfqpoint{2.962947in}{2.173090in}}{\pgfqpoint{2.957123in}{2.167266in}}%
\pgfpathcurveto{\pgfqpoint{2.951299in}{2.161442in}}{\pgfqpoint{2.948027in}{2.153542in}}{\pgfqpoint{2.948027in}{2.145306in}}%
\pgfpathcurveto{\pgfqpoint{2.948027in}{2.137070in}}{\pgfqpoint{2.951299in}{2.129170in}}{\pgfqpoint{2.957123in}{2.123346in}}%
\pgfpathcurveto{\pgfqpoint{2.962947in}{2.117522in}}{\pgfqpoint{2.970847in}{2.114250in}}{\pgfqpoint{2.979083in}{2.114250in}}%
\pgfpathclose%
\pgfusepath{stroke,fill}%
\end{pgfscope}%
\begin{pgfscope}%
\pgfpathrectangle{\pgfqpoint{0.100000in}{0.212622in}}{\pgfqpoint{3.696000in}{3.696000in}}%
\pgfusepath{clip}%
\pgfsetbuttcap%
\pgfsetroundjoin%
\definecolor{currentfill}{rgb}{0.121569,0.466667,0.705882}%
\pgfsetfillcolor{currentfill}%
\pgfsetfillopacity{0.750874}%
\pgfsetlinewidth{1.003750pt}%
\definecolor{currentstroke}{rgb}{0.121569,0.466667,0.705882}%
\pgfsetstrokecolor{currentstroke}%
\pgfsetstrokeopacity{0.750874}%
\pgfsetdash{}{0pt}%
\pgfpathmoveto{\pgfqpoint{2.977162in}{2.113645in}}%
\pgfpathcurveto{\pgfqpoint{2.985399in}{2.113645in}}{\pgfqpoint{2.993299in}{2.116918in}}{\pgfqpoint{2.999123in}{2.122741in}}%
\pgfpathcurveto{\pgfqpoint{3.004947in}{2.128565in}}{\pgfqpoint{3.008219in}{2.136465in}}{\pgfqpoint{3.008219in}{2.144702in}}%
\pgfpathcurveto{\pgfqpoint{3.008219in}{2.152938in}}{\pgfqpoint{3.004947in}{2.160838in}}{\pgfqpoint{2.999123in}{2.166662in}}%
\pgfpathcurveto{\pgfqpoint{2.993299in}{2.172486in}}{\pgfqpoint{2.985399in}{2.175758in}}{\pgfqpoint{2.977162in}{2.175758in}}%
\pgfpathcurveto{\pgfqpoint{2.968926in}{2.175758in}}{\pgfqpoint{2.961026in}{2.172486in}}{\pgfqpoint{2.955202in}{2.166662in}}%
\pgfpathcurveto{\pgfqpoint{2.949378in}{2.160838in}}{\pgfqpoint{2.946106in}{2.152938in}}{\pgfqpoint{2.946106in}{2.144702in}}%
\pgfpathcurveto{\pgfqpoint{2.946106in}{2.136465in}}{\pgfqpoint{2.949378in}{2.128565in}}{\pgfqpoint{2.955202in}{2.122741in}}%
\pgfpathcurveto{\pgfqpoint{2.961026in}{2.116918in}}{\pgfqpoint{2.968926in}{2.113645in}}{\pgfqpoint{2.977162in}{2.113645in}}%
\pgfpathclose%
\pgfusepath{stroke,fill}%
\end{pgfscope}%
\begin{pgfscope}%
\pgfpathrectangle{\pgfqpoint{0.100000in}{0.212622in}}{\pgfqpoint{3.696000in}{3.696000in}}%
\pgfusepath{clip}%
\pgfsetbuttcap%
\pgfsetroundjoin%
\definecolor{currentfill}{rgb}{0.121569,0.466667,0.705882}%
\pgfsetfillcolor{currentfill}%
\pgfsetfillopacity{0.751399}%
\pgfsetlinewidth{1.003750pt}%
\definecolor{currentstroke}{rgb}{0.121569,0.466667,0.705882}%
\pgfsetstrokecolor{currentstroke}%
\pgfsetstrokeopacity{0.751399}%
\pgfsetdash{}{0pt}%
\pgfpathmoveto{\pgfqpoint{2.975953in}{2.113436in}}%
\pgfpathcurveto{\pgfqpoint{2.984189in}{2.113436in}}{\pgfqpoint{2.992089in}{2.116708in}}{\pgfqpoint{2.997913in}{2.122532in}}%
\pgfpathcurveto{\pgfqpoint{3.003737in}{2.128356in}}{\pgfqpoint{3.007009in}{2.136256in}}{\pgfqpoint{3.007009in}{2.144492in}}%
\pgfpathcurveto{\pgfqpoint{3.007009in}{2.152729in}}{\pgfqpoint{3.003737in}{2.160629in}}{\pgfqpoint{2.997913in}{2.166453in}}%
\pgfpathcurveto{\pgfqpoint{2.992089in}{2.172276in}}{\pgfqpoint{2.984189in}{2.175549in}}{\pgfqpoint{2.975953in}{2.175549in}}%
\pgfpathcurveto{\pgfqpoint{2.967717in}{2.175549in}}{\pgfqpoint{2.959817in}{2.172276in}}{\pgfqpoint{2.953993in}{2.166453in}}%
\pgfpathcurveto{\pgfqpoint{2.948169in}{2.160629in}}{\pgfqpoint{2.944896in}{2.152729in}}{\pgfqpoint{2.944896in}{2.144492in}}%
\pgfpathcurveto{\pgfqpoint{2.944896in}{2.136256in}}{\pgfqpoint{2.948169in}{2.128356in}}{\pgfqpoint{2.953993in}{2.122532in}}%
\pgfpathcurveto{\pgfqpoint{2.959817in}{2.116708in}}{\pgfqpoint{2.967717in}{2.113436in}}{\pgfqpoint{2.975953in}{2.113436in}}%
\pgfpathclose%
\pgfusepath{stroke,fill}%
\end{pgfscope}%
\begin{pgfscope}%
\pgfpathrectangle{\pgfqpoint{0.100000in}{0.212622in}}{\pgfqpoint{3.696000in}{3.696000in}}%
\pgfusepath{clip}%
\pgfsetbuttcap%
\pgfsetroundjoin%
\definecolor{currentfill}{rgb}{0.121569,0.466667,0.705882}%
\pgfsetfillcolor{currentfill}%
\pgfsetfillopacity{0.752548}%
\pgfsetlinewidth{1.003750pt}%
\definecolor{currentstroke}{rgb}{0.121569,0.466667,0.705882}%
\pgfsetstrokecolor{currentstroke}%
\pgfsetstrokeopacity{0.752548}%
\pgfsetdash{}{0pt}%
\pgfpathmoveto{\pgfqpoint{2.973921in}{2.111480in}}%
\pgfpathcurveto{\pgfqpoint{2.982157in}{2.111480in}}{\pgfqpoint{2.990057in}{2.114753in}}{\pgfqpoint{2.995881in}{2.120577in}}%
\pgfpathcurveto{\pgfqpoint{3.001705in}{2.126401in}}{\pgfqpoint{3.004977in}{2.134301in}}{\pgfqpoint{3.004977in}{2.142537in}}%
\pgfpathcurveto{\pgfqpoint{3.004977in}{2.150773in}}{\pgfqpoint{3.001705in}{2.158673in}}{\pgfqpoint{2.995881in}{2.164497in}}%
\pgfpathcurveto{\pgfqpoint{2.990057in}{2.170321in}}{\pgfqpoint{2.982157in}{2.173593in}}{\pgfqpoint{2.973921in}{2.173593in}}%
\pgfpathcurveto{\pgfqpoint{2.965685in}{2.173593in}}{\pgfqpoint{2.957785in}{2.170321in}}{\pgfqpoint{2.951961in}{2.164497in}}%
\pgfpathcurveto{\pgfqpoint{2.946137in}{2.158673in}}{\pgfqpoint{2.942864in}{2.150773in}}{\pgfqpoint{2.942864in}{2.142537in}}%
\pgfpathcurveto{\pgfqpoint{2.942864in}{2.134301in}}{\pgfqpoint{2.946137in}{2.126401in}}{\pgfqpoint{2.951961in}{2.120577in}}%
\pgfpathcurveto{\pgfqpoint{2.957785in}{2.114753in}}{\pgfqpoint{2.965685in}{2.111480in}}{\pgfqpoint{2.973921in}{2.111480in}}%
\pgfpathclose%
\pgfusepath{stroke,fill}%
\end{pgfscope}%
\begin{pgfscope}%
\pgfpathrectangle{\pgfqpoint{0.100000in}{0.212622in}}{\pgfqpoint{3.696000in}{3.696000in}}%
\pgfusepath{clip}%
\pgfsetbuttcap%
\pgfsetroundjoin%
\definecolor{currentfill}{rgb}{0.121569,0.466667,0.705882}%
\pgfsetfillcolor{currentfill}%
\pgfsetfillopacity{0.754236}%
\pgfsetlinewidth{1.003750pt}%
\definecolor{currentstroke}{rgb}{0.121569,0.466667,0.705882}%
\pgfsetstrokecolor{currentstroke}%
\pgfsetstrokeopacity{0.754236}%
\pgfsetdash{}{0pt}%
\pgfpathmoveto{\pgfqpoint{2.969857in}{2.110390in}}%
\pgfpathcurveto{\pgfqpoint{2.978093in}{2.110390in}}{\pgfqpoint{2.985993in}{2.113663in}}{\pgfqpoint{2.991817in}{2.119487in}}%
\pgfpathcurveto{\pgfqpoint{2.997641in}{2.125310in}}{\pgfqpoint{3.000913in}{2.133211in}}{\pgfqpoint{3.000913in}{2.141447in}}%
\pgfpathcurveto{\pgfqpoint{3.000913in}{2.149683in}}{\pgfqpoint{2.997641in}{2.157583in}}{\pgfqpoint{2.991817in}{2.163407in}}%
\pgfpathcurveto{\pgfqpoint{2.985993in}{2.169231in}}{\pgfqpoint{2.978093in}{2.172503in}}{\pgfqpoint{2.969857in}{2.172503in}}%
\pgfpathcurveto{\pgfqpoint{2.961620in}{2.172503in}}{\pgfqpoint{2.953720in}{2.169231in}}{\pgfqpoint{2.947896in}{2.163407in}}%
\pgfpathcurveto{\pgfqpoint{2.942072in}{2.157583in}}{\pgfqpoint{2.938800in}{2.149683in}}{\pgfqpoint{2.938800in}{2.141447in}}%
\pgfpathcurveto{\pgfqpoint{2.938800in}{2.133211in}}{\pgfqpoint{2.942072in}{2.125310in}}{\pgfqpoint{2.947896in}{2.119487in}}%
\pgfpathcurveto{\pgfqpoint{2.953720in}{2.113663in}}{\pgfqpoint{2.961620in}{2.110390in}}{\pgfqpoint{2.969857in}{2.110390in}}%
\pgfpathclose%
\pgfusepath{stroke,fill}%
\end{pgfscope}%
\begin{pgfscope}%
\pgfpathrectangle{\pgfqpoint{0.100000in}{0.212622in}}{\pgfqpoint{3.696000in}{3.696000in}}%
\pgfusepath{clip}%
\pgfsetbuttcap%
\pgfsetroundjoin%
\definecolor{currentfill}{rgb}{0.121569,0.466667,0.705882}%
\pgfsetfillcolor{currentfill}%
\pgfsetfillopacity{0.756912}%
\pgfsetlinewidth{1.003750pt}%
\definecolor{currentstroke}{rgb}{0.121569,0.466667,0.705882}%
\pgfsetstrokecolor{currentstroke}%
\pgfsetstrokeopacity{0.756912}%
\pgfsetdash{}{0pt}%
\pgfpathmoveto{\pgfqpoint{2.964766in}{2.104999in}}%
\pgfpathcurveto{\pgfqpoint{2.973002in}{2.104999in}}{\pgfqpoint{2.980902in}{2.108271in}}{\pgfqpoint{2.986726in}{2.114095in}}%
\pgfpathcurveto{\pgfqpoint{2.992550in}{2.119919in}}{\pgfqpoint{2.995823in}{2.127819in}}{\pgfqpoint{2.995823in}{2.136056in}}%
\pgfpathcurveto{\pgfqpoint{2.995823in}{2.144292in}}{\pgfqpoint{2.992550in}{2.152192in}}{\pgfqpoint{2.986726in}{2.158016in}}%
\pgfpathcurveto{\pgfqpoint{2.980902in}{2.163840in}}{\pgfqpoint{2.973002in}{2.167112in}}{\pgfqpoint{2.964766in}{2.167112in}}%
\pgfpathcurveto{\pgfqpoint{2.956530in}{2.167112in}}{\pgfqpoint{2.948630in}{2.163840in}}{\pgfqpoint{2.942806in}{2.158016in}}%
\pgfpathcurveto{\pgfqpoint{2.936982in}{2.152192in}}{\pgfqpoint{2.933710in}{2.144292in}}{\pgfqpoint{2.933710in}{2.136056in}}%
\pgfpathcurveto{\pgfqpoint{2.933710in}{2.127819in}}{\pgfqpoint{2.936982in}{2.119919in}}{\pgfqpoint{2.942806in}{2.114095in}}%
\pgfpathcurveto{\pgfqpoint{2.948630in}{2.108271in}}{\pgfqpoint{2.956530in}{2.104999in}}{\pgfqpoint{2.964766in}{2.104999in}}%
\pgfpathclose%
\pgfusepath{stroke,fill}%
\end{pgfscope}%
\begin{pgfscope}%
\pgfpathrectangle{\pgfqpoint{0.100000in}{0.212622in}}{\pgfqpoint{3.696000in}{3.696000in}}%
\pgfusepath{clip}%
\pgfsetbuttcap%
\pgfsetroundjoin%
\definecolor{currentfill}{rgb}{0.121569,0.466667,0.705882}%
\pgfsetfillcolor{currentfill}%
\pgfsetfillopacity{0.758313}%
\pgfsetlinewidth{1.003750pt}%
\definecolor{currentstroke}{rgb}{0.121569,0.466667,0.705882}%
\pgfsetstrokecolor{currentstroke}%
\pgfsetstrokeopacity{0.758313}%
\pgfsetdash{}{0pt}%
\pgfpathmoveto{\pgfqpoint{2.960517in}{2.103516in}}%
\pgfpathcurveto{\pgfqpoint{2.968753in}{2.103516in}}{\pgfqpoint{2.976653in}{2.106788in}}{\pgfqpoint{2.982477in}{2.112612in}}%
\pgfpathcurveto{\pgfqpoint{2.988301in}{2.118436in}}{\pgfqpoint{2.991573in}{2.126336in}}{\pgfqpoint{2.991573in}{2.134572in}}%
\pgfpathcurveto{\pgfqpoint{2.991573in}{2.142808in}}{\pgfqpoint{2.988301in}{2.150708in}}{\pgfqpoint{2.982477in}{2.156532in}}%
\pgfpathcurveto{\pgfqpoint{2.976653in}{2.162356in}}{\pgfqpoint{2.968753in}{2.165629in}}{\pgfqpoint{2.960517in}{2.165629in}}%
\pgfpathcurveto{\pgfqpoint{2.952280in}{2.165629in}}{\pgfqpoint{2.944380in}{2.162356in}}{\pgfqpoint{2.938557in}{2.156532in}}%
\pgfpathcurveto{\pgfqpoint{2.932733in}{2.150708in}}{\pgfqpoint{2.929460in}{2.142808in}}{\pgfqpoint{2.929460in}{2.134572in}}%
\pgfpathcurveto{\pgfqpoint{2.929460in}{2.126336in}}{\pgfqpoint{2.932733in}{2.118436in}}{\pgfqpoint{2.938557in}{2.112612in}}%
\pgfpathcurveto{\pgfqpoint{2.944380in}{2.106788in}}{\pgfqpoint{2.952280in}{2.103516in}}{\pgfqpoint{2.960517in}{2.103516in}}%
\pgfpathclose%
\pgfusepath{stroke,fill}%
\end{pgfscope}%
\begin{pgfscope}%
\pgfpathrectangle{\pgfqpoint{0.100000in}{0.212622in}}{\pgfqpoint{3.696000in}{3.696000in}}%
\pgfusepath{clip}%
\pgfsetbuttcap%
\pgfsetroundjoin%
\definecolor{currentfill}{rgb}{0.121569,0.466667,0.705882}%
\pgfsetfillcolor{currentfill}%
\pgfsetfillopacity{0.760894}%
\pgfsetlinewidth{1.003750pt}%
\definecolor{currentstroke}{rgb}{0.121569,0.466667,0.705882}%
\pgfsetstrokecolor{currentstroke}%
\pgfsetstrokeopacity{0.760894}%
\pgfsetdash{}{0pt}%
\pgfpathmoveto{\pgfqpoint{2.955825in}{2.101240in}}%
\pgfpathcurveto{\pgfqpoint{2.964061in}{2.101240in}}{\pgfqpoint{2.971961in}{2.104513in}}{\pgfqpoint{2.977785in}{2.110337in}}%
\pgfpathcurveto{\pgfqpoint{2.983609in}{2.116160in}}{\pgfqpoint{2.986882in}{2.124060in}}{\pgfqpoint{2.986882in}{2.132297in}}%
\pgfpathcurveto{\pgfqpoint{2.986882in}{2.140533in}}{\pgfqpoint{2.983609in}{2.148433in}}{\pgfqpoint{2.977785in}{2.154257in}}%
\pgfpathcurveto{\pgfqpoint{2.971961in}{2.160081in}}{\pgfqpoint{2.964061in}{2.163353in}}{\pgfqpoint{2.955825in}{2.163353in}}%
\pgfpathcurveto{\pgfqpoint{2.947589in}{2.163353in}}{\pgfqpoint{2.939689in}{2.160081in}}{\pgfqpoint{2.933865in}{2.154257in}}%
\pgfpathcurveto{\pgfqpoint{2.928041in}{2.148433in}}{\pgfqpoint{2.924769in}{2.140533in}}{\pgfqpoint{2.924769in}{2.132297in}}%
\pgfpathcurveto{\pgfqpoint{2.924769in}{2.124060in}}{\pgfqpoint{2.928041in}{2.116160in}}{\pgfqpoint{2.933865in}{2.110337in}}%
\pgfpathcurveto{\pgfqpoint{2.939689in}{2.104513in}}{\pgfqpoint{2.947589in}{2.101240in}}{\pgfqpoint{2.955825in}{2.101240in}}%
\pgfpathclose%
\pgfusepath{stroke,fill}%
\end{pgfscope}%
\begin{pgfscope}%
\pgfpathrectangle{\pgfqpoint{0.100000in}{0.212622in}}{\pgfqpoint{3.696000in}{3.696000in}}%
\pgfusepath{clip}%
\pgfsetbuttcap%
\pgfsetroundjoin%
\definecolor{currentfill}{rgb}{0.121569,0.466667,0.705882}%
\pgfsetfillcolor{currentfill}%
\pgfsetfillopacity{0.762193}%
\pgfsetlinewidth{1.003750pt}%
\definecolor{currentstroke}{rgb}{0.121569,0.466667,0.705882}%
\pgfsetstrokecolor{currentstroke}%
\pgfsetstrokeopacity{0.762193}%
\pgfsetdash{}{0pt}%
\pgfpathmoveto{\pgfqpoint{2.952598in}{2.100012in}}%
\pgfpathcurveto{\pgfqpoint{2.960834in}{2.100012in}}{\pgfqpoint{2.968734in}{2.103284in}}{\pgfqpoint{2.974558in}{2.109108in}}%
\pgfpathcurveto{\pgfqpoint{2.980382in}{2.114932in}}{\pgfqpoint{2.983654in}{2.122832in}}{\pgfqpoint{2.983654in}{2.131068in}}%
\pgfpathcurveto{\pgfqpoint{2.983654in}{2.139305in}}{\pgfqpoint{2.980382in}{2.147205in}}{\pgfqpoint{2.974558in}{2.153029in}}%
\pgfpathcurveto{\pgfqpoint{2.968734in}{2.158853in}}{\pgfqpoint{2.960834in}{2.162125in}}{\pgfqpoint{2.952598in}{2.162125in}}%
\pgfpathcurveto{\pgfqpoint{2.944362in}{2.162125in}}{\pgfqpoint{2.936462in}{2.158853in}}{\pgfqpoint{2.930638in}{2.153029in}}%
\pgfpathcurveto{\pgfqpoint{2.924814in}{2.147205in}}{\pgfqpoint{2.921541in}{2.139305in}}{\pgfqpoint{2.921541in}{2.131068in}}%
\pgfpathcurveto{\pgfqpoint{2.921541in}{2.122832in}}{\pgfqpoint{2.924814in}{2.114932in}}{\pgfqpoint{2.930638in}{2.109108in}}%
\pgfpathcurveto{\pgfqpoint{2.936462in}{2.103284in}}{\pgfqpoint{2.944362in}{2.100012in}}{\pgfqpoint{2.952598in}{2.100012in}}%
\pgfpathclose%
\pgfusepath{stroke,fill}%
\end{pgfscope}%
\begin{pgfscope}%
\pgfpathrectangle{\pgfqpoint{0.100000in}{0.212622in}}{\pgfqpoint{3.696000in}{3.696000in}}%
\pgfusepath{clip}%
\pgfsetbuttcap%
\pgfsetroundjoin%
\definecolor{currentfill}{rgb}{0.121569,0.466667,0.705882}%
\pgfsetfillcolor{currentfill}%
\pgfsetfillopacity{0.762904}%
\pgfsetlinewidth{1.003750pt}%
\definecolor{currentstroke}{rgb}{0.121569,0.466667,0.705882}%
\pgfsetstrokecolor{currentstroke}%
\pgfsetstrokeopacity{0.762904}%
\pgfsetdash{}{0pt}%
\pgfpathmoveto{\pgfqpoint{2.951073in}{2.098969in}}%
\pgfpathcurveto{\pgfqpoint{2.959309in}{2.098969in}}{\pgfqpoint{2.967209in}{2.102241in}}{\pgfqpoint{2.973033in}{2.108065in}}%
\pgfpathcurveto{\pgfqpoint{2.978857in}{2.113889in}}{\pgfqpoint{2.982129in}{2.121789in}}{\pgfqpoint{2.982129in}{2.130025in}}%
\pgfpathcurveto{\pgfqpoint{2.982129in}{2.138262in}}{\pgfqpoint{2.978857in}{2.146162in}}{\pgfqpoint{2.973033in}{2.151986in}}%
\pgfpathcurveto{\pgfqpoint{2.967209in}{2.157810in}}{\pgfqpoint{2.959309in}{2.161082in}}{\pgfqpoint{2.951073in}{2.161082in}}%
\pgfpathcurveto{\pgfqpoint{2.942836in}{2.161082in}}{\pgfqpoint{2.934936in}{2.157810in}}{\pgfqpoint{2.929112in}{2.151986in}}%
\pgfpathcurveto{\pgfqpoint{2.923288in}{2.146162in}}{\pgfqpoint{2.920016in}{2.138262in}}{\pgfqpoint{2.920016in}{2.130025in}}%
\pgfpathcurveto{\pgfqpoint{2.920016in}{2.121789in}}{\pgfqpoint{2.923288in}{2.113889in}}{\pgfqpoint{2.929112in}{2.108065in}}%
\pgfpathcurveto{\pgfqpoint{2.934936in}{2.102241in}}{\pgfqpoint{2.942836in}{2.098969in}}{\pgfqpoint{2.951073in}{2.098969in}}%
\pgfpathclose%
\pgfusepath{stroke,fill}%
\end{pgfscope}%
\begin{pgfscope}%
\pgfpathrectangle{\pgfqpoint{0.100000in}{0.212622in}}{\pgfqpoint{3.696000in}{3.696000in}}%
\pgfusepath{clip}%
\pgfsetbuttcap%
\pgfsetroundjoin%
\definecolor{currentfill}{rgb}{0.121569,0.466667,0.705882}%
\pgfsetfillcolor{currentfill}%
\pgfsetfillopacity{0.763325}%
\pgfsetlinewidth{1.003750pt}%
\definecolor{currentstroke}{rgb}{0.121569,0.466667,0.705882}%
\pgfsetstrokecolor{currentstroke}%
\pgfsetstrokeopacity{0.763325}%
\pgfsetdash{}{0pt}%
\pgfpathmoveto{\pgfqpoint{2.950075in}{2.098820in}}%
\pgfpathcurveto{\pgfqpoint{2.958312in}{2.098820in}}{\pgfqpoint{2.966212in}{2.102092in}}{\pgfqpoint{2.972036in}{2.107916in}}%
\pgfpathcurveto{\pgfqpoint{2.977860in}{2.113740in}}{\pgfqpoint{2.981132in}{2.121640in}}{\pgfqpoint{2.981132in}{2.129876in}}%
\pgfpathcurveto{\pgfqpoint{2.981132in}{2.138112in}}{\pgfqpoint{2.977860in}{2.146012in}}{\pgfqpoint{2.972036in}{2.151836in}}%
\pgfpathcurveto{\pgfqpoint{2.966212in}{2.157660in}}{\pgfqpoint{2.958312in}{2.160933in}}{\pgfqpoint{2.950075in}{2.160933in}}%
\pgfpathcurveto{\pgfqpoint{2.941839in}{2.160933in}}{\pgfqpoint{2.933939in}{2.157660in}}{\pgfqpoint{2.928115in}{2.151836in}}%
\pgfpathcurveto{\pgfqpoint{2.922291in}{2.146012in}}{\pgfqpoint{2.919019in}{2.138112in}}{\pgfqpoint{2.919019in}{2.129876in}}%
\pgfpathcurveto{\pgfqpoint{2.919019in}{2.121640in}}{\pgfqpoint{2.922291in}{2.113740in}}{\pgfqpoint{2.928115in}{2.107916in}}%
\pgfpathcurveto{\pgfqpoint{2.933939in}{2.102092in}}{\pgfqpoint{2.941839in}{2.098820in}}{\pgfqpoint{2.950075in}{2.098820in}}%
\pgfpathclose%
\pgfusepath{stroke,fill}%
\end{pgfscope}%
\begin{pgfscope}%
\pgfpathrectangle{\pgfqpoint{0.100000in}{0.212622in}}{\pgfqpoint{3.696000in}{3.696000in}}%
\pgfusepath{clip}%
\pgfsetbuttcap%
\pgfsetroundjoin%
\definecolor{currentfill}{rgb}{0.121569,0.466667,0.705882}%
\pgfsetfillcolor{currentfill}%
\pgfsetfillopacity{0.763536}%
\pgfsetlinewidth{1.003750pt}%
\definecolor{currentstroke}{rgb}{0.121569,0.466667,0.705882}%
\pgfsetstrokecolor{currentstroke}%
\pgfsetstrokeopacity{0.763536}%
\pgfsetdash{}{0pt}%
\pgfpathmoveto{\pgfqpoint{2.949520in}{2.098609in}}%
\pgfpathcurveto{\pgfqpoint{2.957756in}{2.098609in}}{\pgfqpoint{2.965656in}{2.101881in}}{\pgfqpoint{2.971480in}{2.107705in}}%
\pgfpathcurveto{\pgfqpoint{2.977304in}{2.113529in}}{\pgfqpoint{2.980576in}{2.121429in}}{\pgfqpoint{2.980576in}{2.129666in}}%
\pgfpathcurveto{\pgfqpoint{2.980576in}{2.137902in}}{\pgfqpoint{2.977304in}{2.145802in}}{\pgfqpoint{2.971480in}{2.151626in}}%
\pgfpathcurveto{\pgfqpoint{2.965656in}{2.157450in}}{\pgfqpoint{2.957756in}{2.160722in}}{\pgfqpoint{2.949520in}{2.160722in}}%
\pgfpathcurveto{\pgfqpoint{2.941283in}{2.160722in}}{\pgfqpoint{2.933383in}{2.157450in}}{\pgfqpoint{2.927559in}{2.151626in}}%
\pgfpathcurveto{\pgfqpoint{2.921735in}{2.145802in}}{\pgfqpoint{2.918463in}{2.137902in}}{\pgfqpoint{2.918463in}{2.129666in}}%
\pgfpathcurveto{\pgfqpoint{2.918463in}{2.121429in}}{\pgfqpoint{2.921735in}{2.113529in}}{\pgfqpoint{2.927559in}{2.107705in}}%
\pgfpathcurveto{\pgfqpoint{2.933383in}{2.101881in}}{\pgfqpoint{2.941283in}{2.098609in}}{\pgfqpoint{2.949520in}{2.098609in}}%
\pgfpathclose%
\pgfusepath{stroke,fill}%
\end{pgfscope}%
\begin{pgfscope}%
\pgfpathrectangle{\pgfqpoint{0.100000in}{0.212622in}}{\pgfqpoint{3.696000in}{3.696000in}}%
\pgfusepath{clip}%
\pgfsetbuttcap%
\pgfsetroundjoin%
\definecolor{currentfill}{rgb}{0.121569,0.466667,0.705882}%
\pgfsetfillcolor{currentfill}%
\pgfsetfillopacity{0.763675}%
\pgfsetlinewidth{1.003750pt}%
\definecolor{currentstroke}{rgb}{0.121569,0.466667,0.705882}%
\pgfsetstrokecolor{currentstroke}%
\pgfsetstrokeopacity{0.763675}%
\pgfsetdash{}{0pt}%
\pgfpathmoveto{\pgfqpoint{2.949287in}{2.098546in}}%
\pgfpathcurveto{\pgfqpoint{2.957523in}{2.098546in}}{\pgfqpoint{2.965423in}{2.101818in}}{\pgfqpoint{2.971247in}{2.107642in}}%
\pgfpathcurveto{\pgfqpoint{2.977071in}{2.113466in}}{\pgfqpoint{2.980343in}{2.121366in}}{\pgfqpoint{2.980343in}{2.129603in}}%
\pgfpathcurveto{\pgfqpoint{2.980343in}{2.137839in}}{\pgfqpoint{2.977071in}{2.145739in}}{\pgfqpoint{2.971247in}{2.151563in}}%
\pgfpathcurveto{\pgfqpoint{2.965423in}{2.157387in}}{\pgfqpoint{2.957523in}{2.160659in}}{\pgfqpoint{2.949287in}{2.160659in}}%
\pgfpathcurveto{\pgfqpoint{2.941050in}{2.160659in}}{\pgfqpoint{2.933150in}{2.157387in}}{\pgfqpoint{2.927326in}{2.151563in}}%
\pgfpathcurveto{\pgfqpoint{2.921503in}{2.145739in}}{\pgfqpoint{2.918230in}{2.137839in}}{\pgfqpoint{2.918230in}{2.129603in}}%
\pgfpathcurveto{\pgfqpoint{2.918230in}{2.121366in}}{\pgfqpoint{2.921503in}{2.113466in}}{\pgfqpoint{2.927326in}{2.107642in}}%
\pgfpathcurveto{\pgfqpoint{2.933150in}{2.101818in}}{\pgfqpoint{2.941050in}{2.098546in}}{\pgfqpoint{2.949287in}{2.098546in}}%
\pgfpathclose%
\pgfusepath{stroke,fill}%
\end{pgfscope}%
\begin{pgfscope}%
\pgfpathrectangle{\pgfqpoint{0.100000in}{0.212622in}}{\pgfqpoint{3.696000in}{3.696000in}}%
\pgfusepath{clip}%
\pgfsetbuttcap%
\pgfsetroundjoin%
\definecolor{currentfill}{rgb}{0.121569,0.466667,0.705882}%
\pgfsetfillcolor{currentfill}%
\pgfsetfillopacity{0.764302}%
\pgfsetlinewidth{1.003750pt}%
\definecolor{currentstroke}{rgb}{0.121569,0.466667,0.705882}%
\pgfsetstrokecolor{currentstroke}%
\pgfsetstrokeopacity{0.764302}%
\pgfsetdash{}{0pt}%
\pgfpathmoveto{\pgfqpoint{2.947326in}{2.097992in}}%
\pgfpathcurveto{\pgfqpoint{2.955562in}{2.097992in}}{\pgfqpoint{2.963462in}{2.101264in}}{\pgfqpoint{2.969286in}{2.107088in}}%
\pgfpathcurveto{\pgfqpoint{2.975110in}{2.112912in}}{\pgfqpoint{2.978382in}{2.120812in}}{\pgfqpoint{2.978382in}{2.129049in}}%
\pgfpathcurveto{\pgfqpoint{2.978382in}{2.137285in}}{\pgfqpoint{2.975110in}{2.145185in}}{\pgfqpoint{2.969286in}{2.151009in}}%
\pgfpathcurveto{\pgfqpoint{2.963462in}{2.156833in}}{\pgfqpoint{2.955562in}{2.160105in}}{\pgfqpoint{2.947326in}{2.160105in}}%
\pgfpathcurveto{\pgfqpoint{2.939090in}{2.160105in}}{\pgfqpoint{2.931190in}{2.156833in}}{\pgfqpoint{2.925366in}{2.151009in}}%
\pgfpathcurveto{\pgfqpoint{2.919542in}{2.145185in}}{\pgfqpoint{2.916269in}{2.137285in}}{\pgfqpoint{2.916269in}{2.129049in}}%
\pgfpathcurveto{\pgfqpoint{2.916269in}{2.120812in}}{\pgfqpoint{2.919542in}{2.112912in}}{\pgfqpoint{2.925366in}{2.107088in}}%
\pgfpathcurveto{\pgfqpoint{2.931190in}{2.101264in}}{\pgfqpoint{2.939090in}{2.097992in}}{\pgfqpoint{2.947326in}{2.097992in}}%
\pgfpathclose%
\pgfusepath{stroke,fill}%
\end{pgfscope}%
\begin{pgfscope}%
\pgfpathrectangle{\pgfqpoint{0.100000in}{0.212622in}}{\pgfqpoint{3.696000in}{3.696000in}}%
\pgfusepath{clip}%
\pgfsetbuttcap%
\pgfsetroundjoin%
\definecolor{currentfill}{rgb}{0.121569,0.466667,0.705882}%
\pgfsetfillcolor{currentfill}%
\pgfsetfillopacity{0.765834}%
\pgfsetlinewidth{1.003750pt}%
\definecolor{currentstroke}{rgb}{0.121569,0.466667,0.705882}%
\pgfsetstrokecolor{currentstroke}%
\pgfsetstrokeopacity{0.765834}%
\pgfsetdash{}{0pt}%
\pgfpathmoveto{\pgfqpoint{2.944452in}{2.095069in}}%
\pgfpathcurveto{\pgfqpoint{2.952689in}{2.095069in}}{\pgfqpoint{2.960589in}{2.098341in}}{\pgfqpoint{2.966413in}{2.104165in}}%
\pgfpathcurveto{\pgfqpoint{2.972237in}{2.109989in}}{\pgfqpoint{2.975509in}{2.117889in}}{\pgfqpoint{2.975509in}{2.126125in}}%
\pgfpathcurveto{\pgfqpoint{2.975509in}{2.134362in}}{\pgfqpoint{2.972237in}{2.142262in}}{\pgfqpoint{2.966413in}{2.148086in}}%
\pgfpathcurveto{\pgfqpoint{2.960589in}{2.153910in}}{\pgfqpoint{2.952689in}{2.157182in}}{\pgfqpoint{2.944452in}{2.157182in}}%
\pgfpathcurveto{\pgfqpoint{2.936216in}{2.157182in}}{\pgfqpoint{2.928316in}{2.153910in}}{\pgfqpoint{2.922492in}{2.148086in}}%
\pgfpathcurveto{\pgfqpoint{2.916668in}{2.142262in}}{\pgfqpoint{2.913396in}{2.134362in}}{\pgfqpoint{2.913396in}{2.126125in}}%
\pgfpathcurveto{\pgfqpoint{2.913396in}{2.117889in}}{\pgfqpoint{2.916668in}{2.109989in}}{\pgfqpoint{2.922492in}{2.104165in}}%
\pgfpathcurveto{\pgfqpoint{2.928316in}{2.098341in}}{\pgfqpoint{2.936216in}{2.095069in}}{\pgfqpoint{2.944452in}{2.095069in}}%
\pgfpathclose%
\pgfusepath{stroke,fill}%
\end{pgfscope}%
\begin{pgfscope}%
\pgfpathrectangle{\pgfqpoint{0.100000in}{0.212622in}}{\pgfqpoint{3.696000in}{3.696000in}}%
\pgfusepath{clip}%
\pgfsetbuttcap%
\pgfsetroundjoin%
\definecolor{currentfill}{rgb}{0.121569,0.466667,0.705882}%
\pgfsetfillcolor{currentfill}%
\pgfsetfillopacity{0.768721}%
\pgfsetlinewidth{1.003750pt}%
\definecolor{currentstroke}{rgb}{0.121569,0.466667,0.705882}%
\pgfsetstrokecolor{currentstroke}%
\pgfsetstrokeopacity{0.768721}%
\pgfsetdash{}{0pt}%
\pgfpathmoveto{\pgfqpoint{2.936809in}{2.092610in}}%
\pgfpathcurveto{\pgfqpoint{2.945045in}{2.092610in}}{\pgfqpoint{2.952945in}{2.095882in}}{\pgfqpoint{2.958769in}{2.101706in}}%
\pgfpathcurveto{\pgfqpoint{2.964593in}{2.107530in}}{\pgfqpoint{2.967865in}{2.115430in}}{\pgfqpoint{2.967865in}{2.123666in}}%
\pgfpathcurveto{\pgfqpoint{2.967865in}{2.131902in}}{\pgfqpoint{2.964593in}{2.139802in}}{\pgfqpoint{2.958769in}{2.145626in}}%
\pgfpathcurveto{\pgfqpoint{2.952945in}{2.151450in}}{\pgfqpoint{2.945045in}{2.154723in}}{\pgfqpoint{2.936809in}{2.154723in}}%
\pgfpathcurveto{\pgfqpoint{2.928573in}{2.154723in}}{\pgfqpoint{2.920673in}{2.151450in}}{\pgfqpoint{2.914849in}{2.145626in}}%
\pgfpathcurveto{\pgfqpoint{2.909025in}{2.139802in}}{\pgfqpoint{2.905752in}{2.131902in}}{\pgfqpoint{2.905752in}{2.123666in}}%
\pgfpathcurveto{\pgfqpoint{2.905752in}{2.115430in}}{\pgfqpoint{2.909025in}{2.107530in}}{\pgfqpoint{2.914849in}{2.101706in}}%
\pgfpathcurveto{\pgfqpoint{2.920673in}{2.095882in}}{\pgfqpoint{2.928573in}{2.092610in}}{\pgfqpoint{2.936809in}{2.092610in}}%
\pgfpathclose%
\pgfusepath{stroke,fill}%
\end{pgfscope}%
\begin{pgfscope}%
\pgfpathrectangle{\pgfqpoint{0.100000in}{0.212622in}}{\pgfqpoint{3.696000in}{3.696000in}}%
\pgfusepath{clip}%
\pgfsetbuttcap%
\pgfsetroundjoin%
\definecolor{currentfill}{rgb}{0.121569,0.466667,0.705882}%
\pgfsetfillcolor{currentfill}%
\pgfsetfillopacity{0.773426}%
\pgfsetlinewidth{1.003750pt}%
\definecolor{currentstroke}{rgb}{0.121569,0.466667,0.705882}%
\pgfsetstrokecolor{currentstroke}%
\pgfsetstrokeopacity{0.773426}%
\pgfsetdash{}{0pt}%
\pgfpathmoveto{\pgfqpoint{2.928235in}{2.087193in}}%
\pgfpathcurveto{\pgfqpoint{2.936471in}{2.087193in}}{\pgfqpoint{2.944371in}{2.090465in}}{\pgfqpoint{2.950195in}{2.096289in}}%
\pgfpathcurveto{\pgfqpoint{2.956019in}{2.102113in}}{\pgfqpoint{2.959292in}{2.110013in}}{\pgfqpoint{2.959292in}{2.118249in}}%
\pgfpathcurveto{\pgfqpoint{2.959292in}{2.126485in}}{\pgfqpoint{2.956019in}{2.134385in}}{\pgfqpoint{2.950195in}{2.140209in}}%
\pgfpathcurveto{\pgfqpoint{2.944371in}{2.146033in}}{\pgfqpoint{2.936471in}{2.149306in}}{\pgfqpoint{2.928235in}{2.149306in}}%
\pgfpathcurveto{\pgfqpoint{2.919999in}{2.149306in}}{\pgfqpoint{2.912099in}{2.146033in}}{\pgfqpoint{2.906275in}{2.140209in}}%
\pgfpathcurveto{\pgfqpoint{2.900451in}{2.134385in}}{\pgfqpoint{2.897179in}{2.126485in}}{\pgfqpoint{2.897179in}{2.118249in}}%
\pgfpathcurveto{\pgfqpoint{2.897179in}{2.110013in}}{\pgfqpoint{2.900451in}{2.102113in}}{\pgfqpoint{2.906275in}{2.096289in}}%
\pgfpathcurveto{\pgfqpoint{2.912099in}{2.090465in}}{\pgfqpoint{2.919999in}{2.087193in}}{\pgfqpoint{2.928235in}{2.087193in}}%
\pgfpathclose%
\pgfusepath{stroke,fill}%
\end{pgfscope}%
\begin{pgfscope}%
\pgfpathrectangle{\pgfqpoint{0.100000in}{0.212622in}}{\pgfqpoint{3.696000in}{3.696000in}}%
\pgfusepath{clip}%
\pgfsetbuttcap%
\pgfsetroundjoin%
\definecolor{currentfill}{rgb}{0.121569,0.466667,0.705882}%
\pgfsetfillcolor{currentfill}%
\pgfsetfillopacity{0.778524}%
\pgfsetlinewidth{1.003750pt}%
\definecolor{currentstroke}{rgb}{0.121569,0.466667,0.705882}%
\pgfsetstrokecolor{currentstroke}%
\pgfsetstrokeopacity{0.778524}%
\pgfsetdash{}{0pt}%
\pgfpathmoveto{\pgfqpoint{2.913346in}{2.082209in}}%
\pgfpathcurveto{\pgfqpoint{2.921582in}{2.082209in}}{\pgfqpoint{2.929482in}{2.085481in}}{\pgfqpoint{2.935306in}{2.091305in}}%
\pgfpathcurveto{\pgfqpoint{2.941130in}{2.097129in}}{\pgfqpoint{2.944402in}{2.105029in}}{\pgfqpoint{2.944402in}{2.113265in}}%
\pgfpathcurveto{\pgfqpoint{2.944402in}{2.121502in}}{\pgfqpoint{2.941130in}{2.129402in}}{\pgfqpoint{2.935306in}{2.135226in}}%
\pgfpathcurveto{\pgfqpoint{2.929482in}{2.141049in}}{\pgfqpoint{2.921582in}{2.144322in}}{\pgfqpoint{2.913346in}{2.144322in}}%
\pgfpathcurveto{\pgfqpoint{2.905110in}{2.144322in}}{\pgfqpoint{2.897210in}{2.141049in}}{\pgfqpoint{2.891386in}{2.135226in}}%
\pgfpathcurveto{\pgfqpoint{2.885562in}{2.129402in}}{\pgfqpoint{2.882289in}{2.121502in}}{\pgfqpoint{2.882289in}{2.113265in}}%
\pgfpathcurveto{\pgfqpoint{2.882289in}{2.105029in}}{\pgfqpoint{2.885562in}{2.097129in}}{\pgfqpoint{2.891386in}{2.091305in}}%
\pgfpathcurveto{\pgfqpoint{2.897210in}{2.085481in}}{\pgfqpoint{2.905110in}{2.082209in}}{\pgfqpoint{2.913346in}{2.082209in}}%
\pgfpathclose%
\pgfusepath{stroke,fill}%
\end{pgfscope}%
\begin{pgfscope}%
\pgfpathrectangle{\pgfqpoint{0.100000in}{0.212622in}}{\pgfqpoint{3.696000in}{3.696000in}}%
\pgfusepath{clip}%
\pgfsetbuttcap%
\pgfsetroundjoin%
\definecolor{currentfill}{rgb}{0.121569,0.466667,0.705882}%
\pgfsetfillcolor{currentfill}%
\pgfsetfillopacity{0.784911}%
\pgfsetlinewidth{1.003750pt}%
\definecolor{currentstroke}{rgb}{0.121569,0.466667,0.705882}%
\pgfsetstrokecolor{currentstroke}%
\pgfsetstrokeopacity{0.784911}%
\pgfsetdash{}{0pt}%
\pgfpathmoveto{\pgfqpoint{2.900326in}{2.074034in}}%
\pgfpathcurveto{\pgfqpoint{2.908562in}{2.074034in}}{\pgfqpoint{2.916462in}{2.077306in}}{\pgfqpoint{2.922286in}{2.083130in}}%
\pgfpathcurveto{\pgfqpoint{2.928110in}{2.088954in}}{\pgfqpoint{2.931383in}{2.096854in}}{\pgfqpoint{2.931383in}{2.105090in}}%
\pgfpathcurveto{\pgfqpoint{2.931383in}{2.113327in}}{\pgfqpoint{2.928110in}{2.121227in}}{\pgfqpoint{2.922286in}{2.127051in}}%
\pgfpathcurveto{\pgfqpoint{2.916462in}{2.132874in}}{\pgfqpoint{2.908562in}{2.136147in}}{\pgfqpoint{2.900326in}{2.136147in}}%
\pgfpathcurveto{\pgfqpoint{2.892090in}{2.136147in}}{\pgfqpoint{2.884190in}{2.132874in}}{\pgfqpoint{2.878366in}{2.127051in}}%
\pgfpathcurveto{\pgfqpoint{2.872542in}{2.121227in}}{\pgfqpoint{2.869270in}{2.113327in}}{\pgfqpoint{2.869270in}{2.105090in}}%
\pgfpathcurveto{\pgfqpoint{2.869270in}{2.096854in}}{\pgfqpoint{2.872542in}{2.088954in}}{\pgfqpoint{2.878366in}{2.083130in}}%
\pgfpathcurveto{\pgfqpoint{2.884190in}{2.077306in}}{\pgfqpoint{2.892090in}{2.074034in}}{\pgfqpoint{2.900326in}{2.074034in}}%
\pgfpathclose%
\pgfusepath{stroke,fill}%
\end{pgfscope}%
\begin{pgfscope}%
\pgfpathrectangle{\pgfqpoint{0.100000in}{0.212622in}}{\pgfqpoint{3.696000in}{3.696000in}}%
\pgfusepath{clip}%
\pgfsetbuttcap%
\pgfsetroundjoin%
\definecolor{currentfill}{rgb}{0.121569,0.466667,0.705882}%
\pgfsetfillcolor{currentfill}%
\pgfsetfillopacity{0.790925}%
\pgfsetlinewidth{1.003750pt}%
\definecolor{currentstroke}{rgb}{0.121569,0.466667,0.705882}%
\pgfsetstrokecolor{currentstroke}%
\pgfsetstrokeopacity{0.790925}%
\pgfsetdash{}{0pt}%
\pgfpathmoveto{\pgfqpoint{2.880711in}{2.067015in}}%
\pgfpathcurveto{\pgfqpoint{2.888947in}{2.067015in}}{\pgfqpoint{2.896847in}{2.070287in}}{\pgfqpoint{2.902671in}{2.076111in}}%
\pgfpathcurveto{\pgfqpoint{2.908495in}{2.081935in}}{\pgfqpoint{2.911767in}{2.089835in}}{\pgfqpoint{2.911767in}{2.098071in}}%
\pgfpathcurveto{\pgfqpoint{2.911767in}{2.106307in}}{\pgfqpoint{2.908495in}{2.114207in}}{\pgfqpoint{2.902671in}{2.120031in}}%
\pgfpathcurveto{\pgfqpoint{2.896847in}{2.125855in}}{\pgfqpoint{2.888947in}{2.129128in}}{\pgfqpoint{2.880711in}{2.129128in}}%
\pgfpathcurveto{\pgfqpoint{2.872474in}{2.129128in}}{\pgfqpoint{2.864574in}{2.125855in}}{\pgfqpoint{2.858750in}{2.120031in}}%
\pgfpathcurveto{\pgfqpoint{2.852927in}{2.114207in}}{\pgfqpoint{2.849654in}{2.106307in}}{\pgfqpoint{2.849654in}{2.098071in}}%
\pgfpathcurveto{\pgfqpoint{2.849654in}{2.089835in}}{\pgfqpoint{2.852927in}{2.081935in}}{\pgfqpoint{2.858750in}{2.076111in}}%
\pgfpathcurveto{\pgfqpoint{2.864574in}{2.070287in}}{\pgfqpoint{2.872474in}{2.067015in}}{\pgfqpoint{2.880711in}{2.067015in}}%
\pgfpathclose%
\pgfusepath{stroke,fill}%
\end{pgfscope}%
\begin{pgfscope}%
\pgfpathrectangle{\pgfqpoint{0.100000in}{0.212622in}}{\pgfqpoint{3.696000in}{3.696000in}}%
\pgfusepath{clip}%
\pgfsetbuttcap%
\pgfsetroundjoin%
\definecolor{currentfill}{rgb}{0.121569,0.466667,0.705882}%
\pgfsetfillcolor{currentfill}%
\pgfsetfillopacity{0.794780}%
\pgfsetlinewidth{1.003750pt}%
\definecolor{currentstroke}{rgb}{0.121569,0.466667,0.705882}%
\pgfsetstrokecolor{currentstroke}%
\pgfsetstrokeopacity{0.794780}%
\pgfsetdash{}{0pt}%
\pgfpathmoveto{\pgfqpoint{2.872468in}{2.062750in}}%
\pgfpathcurveto{\pgfqpoint{2.880704in}{2.062750in}}{\pgfqpoint{2.888604in}{2.066023in}}{\pgfqpoint{2.894428in}{2.071846in}}%
\pgfpathcurveto{\pgfqpoint{2.900252in}{2.077670in}}{\pgfqpoint{2.903524in}{2.085570in}}{\pgfqpoint{2.903524in}{2.093807in}}%
\pgfpathcurveto{\pgfqpoint{2.903524in}{2.102043in}}{\pgfqpoint{2.900252in}{2.109943in}}{\pgfqpoint{2.894428in}{2.115767in}}%
\pgfpathcurveto{\pgfqpoint{2.888604in}{2.121591in}}{\pgfqpoint{2.880704in}{2.124863in}}{\pgfqpoint{2.872468in}{2.124863in}}%
\pgfpathcurveto{\pgfqpoint{2.864231in}{2.124863in}}{\pgfqpoint{2.856331in}{2.121591in}}{\pgfqpoint{2.850507in}{2.115767in}}%
\pgfpathcurveto{\pgfqpoint{2.844683in}{2.109943in}}{\pgfqpoint{2.841411in}{2.102043in}}{\pgfqpoint{2.841411in}{2.093807in}}%
\pgfpathcurveto{\pgfqpoint{2.841411in}{2.085570in}}{\pgfqpoint{2.844683in}{2.077670in}}{\pgfqpoint{2.850507in}{2.071846in}}%
\pgfpathcurveto{\pgfqpoint{2.856331in}{2.066023in}}{\pgfqpoint{2.864231in}{2.062750in}}{\pgfqpoint{2.872468in}{2.062750in}}%
\pgfpathclose%
\pgfusepath{stroke,fill}%
\end{pgfscope}%
\begin{pgfscope}%
\pgfpathrectangle{\pgfqpoint{0.100000in}{0.212622in}}{\pgfqpoint{3.696000in}{3.696000in}}%
\pgfusepath{clip}%
\pgfsetbuttcap%
\pgfsetroundjoin%
\definecolor{currentfill}{rgb}{0.121569,0.466667,0.705882}%
\pgfsetfillcolor{currentfill}%
\pgfsetfillopacity{0.796740}%
\pgfsetlinewidth{1.003750pt}%
\definecolor{currentstroke}{rgb}{0.121569,0.466667,0.705882}%
\pgfsetstrokecolor{currentstroke}%
\pgfsetstrokeopacity{0.796740}%
\pgfsetdash{}{0pt}%
\pgfpathmoveto{\pgfqpoint{0.668713in}{2.668621in}}%
\pgfpathcurveto{\pgfqpoint{0.676949in}{2.668621in}}{\pgfqpoint{0.684849in}{2.671893in}}{\pgfqpoint{0.690673in}{2.677717in}}%
\pgfpathcurveto{\pgfqpoint{0.696497in}{2.683541in}}{\pgfqpoint{0.699770in}{2.691441in}}{\pgfqpoint{0.699770in}{2.699677in}}%
\pgfpathcurveto{\pgfqpoint{0.699770in}{2.707914in}}{\pgfqpoint{0.696497in}{2.715814in}}{\pgfqpoint{0.690673in}{2.721638in}}%
\pgfpathcurveto{\pgfqpoint{0.684849in}{2.727462in}}{\pgfqpoint{0.676949in}{2.730734in}}{\pgfqpoint{0.668713in}{2.730734in}}%
\pgfpathcurveto{\pgfqpoint{0.660477in}{2.730734in}}{\pgfqpoint{0.652577in}{2.727462in}}{\pgfqpoint{0.646753in}{2.721638in}}%
\pgfpathcurveto{\pgfqpoint{0.640929in}{2.715814in}}{\pgfqpoint{0.637657in}{2.707914in}}{\pgfqpoint{0.637657in}{2.699677in}}%
\pgfpathcurveto{\pgfqpoint{0.637657in}{2.691441in}}{\pgfqpoint{0.640929in}{2.683541in}}{\pgfqpoint{0.646753in}{2.677717in}}%
\pgfpathcurveto{\pgfqpoint{0.652577in}{2.671893in}}{\pgfqpoint{0.660477in}{2.668621in}}{\pgfqpoint{0.668713in}{2.668621in}}%
\pgfpathclose%
\pgfusepath{stroke,fill}%
\end{pgfscope}%
\begin{pgfscope}%
\pgfpathrectangle{\pgfqpoint{0.100000in}{0.212622in}}{\pgfqpoint{3.696000in}{3.696000in}}%
\pgfusepath{clip}%
\pgfsetbuttcap%
\pgfsetroundjoin%
\definecolor{currentfill}{rgb}{0.121569,0.466667,0.705882}%
\pgfsetfillcolor{currentfill}%
\pgfsetfillopacity{0.797022}%
\pgfsetlinewidth{1.003750pt}%
\definecolor{currentstroke}{rgb}{0.121569,0.466667,0.705882}%
\pgfsetstrokecolor{currentstroke}%
\pgfsetstrokeopacity{0.797022}%
\pgfsetdash{}{0pt}%
\pgfpathmoveto{\pgfqpoint{2.867300in}{2.062106in}}%
\pgfpathcurveto{\pgfqpoint{2.875537in}{2.062106in}}{\pgfqpoint{2.883437in}{2.065379in}}{\pgfqpoint{2.889261in}{2.071203in}}%
\pgfpathcurveto{\pgfqpoint{2.895085in}{2.077027in}}{\pgfqpoint{2.898357in}{2.084927in}}{\pgfqpoint{2.898357in}{2.093163in}}%
\pgfpathcurveto{\pgfqpoint{2.898357in}{2.101399in}}{\pgfqpoint{2.895085in}{2.109299in}}{\pgfqpoint{2.889261in}{2.115123in}}%
\pgfpathcurveto{\pgfqpoint{2.883437in}{2.120947in}}{\pgfqpoint{2.875537in}{2.124219in}}{\pgfqpoint{2.867300in}{2.124219in}}%
\pgfpathcurveto{\pgfqpoint{2.859064in}{2.124219in}}{\pgfqpoint{2.851164in}{2.120947in}}{\pgfqpoint{2.845340in}{2.115123in}}%
\pgfpathcurveto{\pgfqpoint{2.839516in}{2.109299in}}{\pgfqpoint{2.836244in}{2.101399in}}{\pgfqpoint{2.836244in}{2.093163in}}%
\pgfpathcurveto{\pgfqpoint{2.836244in}{2.084927in}}{\pgfqpoint{2.839516in}{2.077027in}}{\pgfqpoint{2.845340in}{2.071203in}}%
\pgfpathcurveto{\pgfqpoint{2.851164in}{2.065379in}}{\pgfqpoint{2.859064in}{2.062106in}}{\pgfqpoint{2.867300in}{2.062106in}}%
\pgfpathclose%
\pgfusepath{stroke,fill}%
\end{pgfscope}%
\begin{pgfscope}%
\pgfpathrectangle{\pgfqpoint{0.100000in}{0.212622in}}{\pgfqpoint{3.696000in}{3.696000in}}%
\pgfusepath{clip}%
\pgfsetbuttcap%
\pgfsetroundjoin%
\definecolor{currentfill}{rgb}{0.121569,0.466667,0.705882}%
\pgfsetfillcolor{currentfill}%
\pgfsetfillopacity{0.798137}%
\pgfsetlinewidth{1.003750pt}%
\definecolor{currentstroke}{rgb}{0.121569,0.466667,0.705882}%
\pgfsetstrokecolor{currentstroke}%
\pgfsetstrokeopacity{0.798137}%
\pgfsetdash{}{0pt}%
\pgfpathmoveto{\pgfqpoint{2.864104in}{2.061485in}}%
\pgfpathcurveto{\pgfqpoint{2.872340in}{2.061485in}}{\pgfqpoint{2.880240in}{2.064758in}}{\pgfqpoint{2.886064in}{2.070582in}}%
\pgfpathcurveto{\pgfqpoint{2.891888in}{2.076406in}}{\pgfqpoint{2.895161in}{2.084306in}}{\pgfqpoint{2.895161in}{2.092542in}}%
\pgfpathcurveto{\pgfqpoint{2.895161in}{2.100778in}}{\pgfqpoint{2.891888in}{2.108678in}}{\pgfqpoint{2.886064in}{2.114502in}}%
\pgfpathcurveto{\pgfqpoint{2.880240in}{2.120326in}}{\pgfqpoint{2.872340in}{2.123598in}}{\pgfqpoint{2.864104in}{2.123598in}}%
\pgfpathcurveto{\pgfqpoint{2.855868in}{2.123598in}}{\pgfqpoint{2.847968in}{2.120326in}}{\pgfqpoint{2.842144in}{2.114502in}}%
\pgfpathcurveto{\pgfqpoint{2.836320in}{2.108678in}}{\pgfqpoint{2.833048in}{2.100778in}}{\pgfqpoint{2.833048in}{2.092542in}}%
\pgfpathcurveto{\pgfqpoint{2.833048in}{2.084306in}}{\pgfqpoint{2.836320in}{2.076406in}}{\pgfqpoint{2.842144in}{2.070582in}}%
\pgfpathcurveto{\pgfqpoint{2.847968in}{2.064758in}}{\pgfqpoint{2.855868in}{2.061485in}}{\pgfqpoint{2.864104in}{2.061485in}}%
\pgfpathclose%
\pgfusepath{stroke,fill}%
\end{pgfscope}%
\begin{pgfscope}%
\pgfpathrectangle{\pgfqpoint{0.100000in}{0.212622in}}{\pgfqpoint{3.696000in}{3.696000in}}%
\pgfusepath{clip}%
\pgfsetbuttcap%
\pgfsetroundjoin%
\definecolor{currentfill}{rgb}{0.121569,0.466667,0.705882}%
\pgfsetfillcolor{currentfill}%
\pgfsetfillopacity{0.798758}%
\pgfsetlinewidth{1.003750pt}%
\definecolor{currentstroke}{rgb}{0.121569,0.466667,0.705882}%
\pgfsetstrokecolor{currentstroke}%
\pgfsetstrokeopacity{0.798758}%
\pgfsetdash{}{0pt}%
\pgfpathmoveto{\pgfqpoint{2.862894in}{2.060443in}}%
\pgfpathcurveto{\pgfqpoint{2.871130in}{2.060443in}}{\pgfqpoint{2.879030in}{2.063715in}}{\pgfqpoint{2.884854in}{2.069539in}}%
\pgfpathcurveto{\pgfqpoint{2.890678in}{2.075363in}}{\pgfqpoint{2.893950in}{2.083263in}}{\pgfqpoint{2.893950in}{2.091500in}}%
\pgfpathcurveto{\pgfqpoint{2.893950in}{2.099736in}}{\pgfqpoint{2.890678in}{2.107636in}}{\pgfqpoint{2.884854in}{2.113460in}}%
\pgfpathcurveto{\pgfqpoint{2.879030in}{2.119284in}}{\pgfqpoint{2.871130in}{2.122556in}}{\pgfqpoint{2.862894in}{2.122556in}}%
\pgfpathcurveto{\pgfqpoint{2.854657in}{2.122556in}}{\pgfqpoint{2.846757in}{2.119284in}}{\pgfqpoint{2.840933in}{2.113460in}}%
\pgfpathcurveto{\pgfqpoint{2.835110in}{2.107636in}}{\pgfqpoint{2.831837in}{2.099736in}}{\pgfqpoint{2.831837in}{2.091500in}}%
\pgfpathcurveto{\pgfqpoint{2.831837in}{2.083263in}}{\pgfqpoint{2.835110in}{2.075363in}}{\pgfqpoint{2.840933in}{2.069539in}}%
\pgfpathcurveto{\pgfqpoint{2.846757in}{2.063715in}}{\pgfqpoint{2.854657in}{2.060443in}}{\pgfqpoint{2.862894in}{2.060443in}}%
\pgfpathclose%
\pgfusepath{stroke,fill}%
\end{pgfscope}%
\begin{pgfscope}%
\pgfpathrectangle{\pgfqpoint{0.100000in}{0.212622in}}{\pgfqpoint{3.696000in}{3.696000in}}%
\pgfusepath{clip}%
\pgfsetbuttcap%
\pgfsetroundjoin%
\definecolor{currentfill}{rgb}{0.121569,0.466667,0.705882}%
\pgfsetfillcolor{currentfill}%
\pgfsetfillopacity{0.799101}%
\pgfsetlinewidth{1.003750pt}%
\definecolor{currentstroke}{rgb}{0.121569,0.466667,0.705882}%
\pgfsetstrokecolor{currentstroke}%
\pgfsetstrokeopacity{0.799101}%
\pgfsetdash{}{0pt}%
\pgfpathmoveto{\pgfqpoint{2.862064in}{2.060081in}}%
\pgfpathcurveto{\pgfqpoint{2.870300in}{2.060081in}}{\pgfqpoint{2.878200in}{2.063353in}}{\pgfqpoint{2.884024in}{2.069177in}}%
\pgfpathcurveto{\pgfqpoint{2.889848in}{2.075001in}}{\pgfqpoint{2.893120in}{2.082901in}}{\pgfqpoint{2.893120in}{2.091137in}}%
\pgfpathcurveto{\pgfqpoint{2.893120in}{2.099374in}}{\pgfqpoint{2.889848in}{2.107274in}}{\pgfqpoint{2.884024in}{2.113098in}}%
\pgfpathcurveto{\pgfqpoint{2.878200in}{2.118922in}}{\pgfqpoint{2.870300in}{2.122194in}}{\pgfqpoint{2.862064in}{2.122194in}}%
\pgfpathcurveto{\pgfqpoint{2.853828in}{2.122194in}}{\pgfqpoint{2.845928in}{2.118922in}}{\pgfqpoint{2.840104in}{2.113098in}}%
\pgfpathcurveto{\pgfqpoint{2.834280in}{2.107274in}}{\pgfqpoint{2.831007in}{2.099374in}}{\pgfqpoint{2.831007in}{2.091137in}}%
\pgfpathcurveto{\pgfqpoint{2.831007in}{2.082901in}}{\pgfqpoint{2.834280in}{2.075001in}}{\pgfqpoint{2.840104in}{2.069177in}}%
\pgfpathcurveto{\pgfqpoint{2.845928in}{2.063353in}}{\pgfqpoint{2.853828in}{2.060081in}}{\pgfqpoint{2.862064in}{2.060081in}}%
\pgfpathclose%
\pgfusepath{stroke,fill}%
\end{pgfscope}%
\begin{pgfscope}%
\pgfpathrectangle{\pgfqpoint{0.100000in}{0.212622in}}{\pgfqpoint{3.696000in}{3.696000in}}%
\pgfusepath{clip}%
\pgfsetbuttcap%
\pgfsetroundjoin%
\definecolor{currentfill}{rgb}{0.121569,0.466667,0.705882}%
\pgfsetfillcolor{currentfill}%
\pgfsetfillopacity{0.800324}%
\pgfsetlinewidth{1.003750pt}%
\definecolor{currentstroke}{rgb}{0.121569,0.466667,0.705882}%
\pgfsetstrokecolor{currentstroke}%
\pgfsetstrokeopacity{0.800324}%
\pgfsetdash{}{0pt}%
\pgfpathmoveto{\pgfqpoint{2.859849in}{2.057680in}}%
\pgfpathcurveto{\pgfqpoint{2.868086in}{2.057680in}}{\pgfqpoint{2.875986in}{2.060952in}}{\pgfqpoint{2.881810in}{2.066776in}}%
\pgfpathcurveto{\pgfqpoint{2.887634in}{2.072600in}}{\pgfqpoint{2.890906in}{2.080500in}}{\pgfqpoint{2.890906in}{2.088737in}}%
\pgfpathcurveto{\pgfqpoint{2.890906in}{2.096973in}}{\pgfqpoint{2.887634in}{2.104873in}}{\pgfqpoint{2.881810in}{2.110697in}}%
\pgfpathcurveto{\pgfqpoint{2.875986in}{2.116521in}}{\pgfqpoint{2.868086in}{2.119793in}}{\pgfqpoint{2.859849in}{2.119793in}}%
\pgfpathcurveto{\pgfqpoint{2.851613in}{2.119793in}}{\pgfqpoint{2.843713in}{2.116521in}}{\pgfqpoint{2.837889in}{2.110697in}}%
\pgfpathcurveto{\pgfqpoint{2.832065in}{2.104873in}}{\pgfqpoint{2.828793in}{2.096973in}}{\pgfqpoint{2.828793in}{2.088737in}}%
\pgfpathcurveto{\pgfqpoint{2.828793in}{2.080500in}}{\pgfqpoint{2.832065in}{2.072600in}}{\pgfqpoint{2.837889in}{2.066776in}}%
\pgfpathcurveto{\pgfqpoint{2.843713in}{2.060952in}}{\pgfqpoint{2.851613in}{2.057680in}}{\pgfqpoint{2.859849in}{2.057680in}}%
\pgfpathclose%
\pgfusepath{stroke,fill}%
\end{pgfscope}%
\begin{pgfscope}%
\pgfpathrectangle{\pgfqpoint{0.100000in}{0.212622in}}{\pgfqpoint{3.696000in}{3.696000in}}%
\pgfusepath{clip}%
\pgfsetbuttcap%
\pgfsetroundjoin%
\definecolor{currentfill}{rgb}{0.121569,0.466667,0.705882}%
\pgfsetfillcolor{currentfill}%
\pgfsetfillopacity{0.801364}%
\pgfsetlinewidth{1.003750pt}%
\definecolor{currentstroke}{rgb}{0.121569,0.466667,0.705882}%
\pgfsetstrokecolor{currentstroke}%
\pgfsetstrokeopacity{0.801364}%
\pgfsetdash{}{0pt}%
\pgfpathmoveto{\pgfqpoint{0.688958in}{2.672172in}}%
\pgfpathcurveto{\pgfqpoint{0.697195in}{2.672172in}}{\pgfqpoint{0.705095in}{2.675444in}}{\pgfqpoint{0.710919in}{2.681268in}}%
\pgfpathcurveto{\pgfqpoint{0.716743in}{2.687092in}}{\pgfqpoint{0.720015in}{2.694992in}}{\pgfqpoint{0.720015in}{2.703228in}}%
\pgfpathcurveto{\pgfqpoint{0.720015in}{2.711465in}}{\pgfqpoint{0.716743in}{2.719365in}}{\pgfqpoint{0.710919in}{2.725189in}}%
\pgfpathcurveto{\pgfqpoint{0.705095in}{2.731013in}}{\pgfqpoint{0.697195in}{2.734285in}}{\pgfqpoint{0.688958in}{2.734285in}}%
\pgfpathcurveto{\pgfqpoint{0.680722in}{2.734285in}}{\pgfqpoint{0.672822in}{2.731013in}}{\pgfqpoint{0.666998in}{2.725189in}}%
\pgfpathcurveto{\pgfqpoint{0.661174in}{2.719365in}}{\pgfqpoint{0.657902in}{2.711465in}}{\pgfqpoint{0.657902in}{2.703228in}}%
\pgfpathcurveto{\pgfqpoint{0.657902in}{2.694992in}}{\pgfqpoint{0.661174in}{2.687092in}}{\pgfqpoint{0.666998in}{2.681268in}}%
\pgfpathcurveto{\pgfqpoint{0.672822in}{2.675444in}}{\pgfqpoint{0.680722in}{2.672172in}}{\pgfqpoint{0.688958in}{2.672172in}}%
\pgfpathclose%
\pgfusepath{stroke,fill}%
\end{pgfscope}%
\begin{pgfscope}%
\pgfpathrectangle{\pgfqpoint{0.100000in}{0.212622in}}{\pgfqpoint{3.696000in}{3.696000in}}%
\pgfusepath{clip}%
\pgfsetbuttcap%
\pgfsetroundjoin%
\definecolor{currentfill}{rgb}{0.121569,0.466667,0.705882}%
\pgfsetfillcolor{currentfill}%
\pgfsetfillopacity{0.802205}%
\pgfsetlinewidth{1.003750pt}%
\definecolor{currentstroke}{rgb}{0.121569,0.466667,0.705882}%
\pgfsetstrokecolor{currentstroke}%
\pgfsetstrokeopacity{0.802205}%
\pgfsetdash{}{0pt}%
\pgfpathmoveto{\pgfqpoint{2.855005in}{2.055684in}}%
\pgfpathcurveto{\pgfqpoint{2.863242in}{2.055684in}}{\pgfqpoint{2.871142in}{2.058957in}}{\pgfqpoint{2.876966in}{2.064780in}}%
\pgfpathcurveto{\pgfqpoint{2.882790in}{2.070604in}}{\pgfqpoint{2.886062in}{2.078504in}}{\pgfqpoint{2.886062in}{2.086741in}}%
\pgfpathcurveto{\pgfqpoint{2.886062in}{2.094977in}}{\pgfqpoint{2.882790in}{2.102877in}}{\pgfqpoint{2.876966in}{2.108701in}}%
\pgfpathcurveto{\pgfqpoint{2.871142in}{2.114525in}}{\pgfqpoint{2.863242in}{2.117797in}}{\pgfqpoint{2.855005in}{2.117797in}}%
\pgfpathcurveto{\pgfqpoint{2.846769in}{2.117797in}}{\pgfqpoint{2.838869in}{2.114525in}}{\pgfqpoint{2.833045in}{2.108701in}}%
\pgfpathcurveto{\pgfqpoint{2.827221in}{2.102877in}}{\pgfqpoint{2.823949in}{2.094977in}}{\pgfqpoint{2.823949in}{2.086741in}}%
\pgfpathcurveto{\pgfqpoint{2.823949in}{2.078504in}}{\pgfqpoint{2.827221in}{2.070604in}}{\pgfqpoint{2.833045in}{2.064780in}}%
\pgfpathcurveto{\pgfqpoint{2.838869in}{2.058957in}}{\pgfqpoint{2.846769in}{2.055684in}}{\pgfqpoint{2.855005in}{2.055684in}}%
\pgfpathclose%
\pgfusepath{stroke,fill}%
\end{pgfscope}%
\begin{pgfscope}%
\pgfpathrectangle{\pgfqpoint{0.100000in}{0.212622in}}{\pgfqpoint{3.696000in}{3.696000in}}%
\pgfusepath{clip}%
\pgfsetbuttcap%
\pgfsetroundjoin%
\definecolor{currentfill}{rgb}{0.121569,0.466667,0.705882}%
\pgfsetfillcolor{currentfill}%
\pgfsetfillopacity{0.802994}%
\pgfsetlinewidth{1.003750pt}%
\definecolor{currentstroke}{rgb}{0.121569,0.466667,0.705882}%
\pgfsetstrokecolor{currentstroke}%
\pgfsetstrokeopacity{0.802994}%
\pgfsetdash{}{0pt}%
\pgfpathmoveto{\pgfqpoint{0.724919in}{2.660142in}}%
\pgfpathcurveto{\pgfqpoint{0.733156in}{2.660142in}}{\pgfqpoint{0.741056in}{2.663415in}}{\pgfqpoint{0.746880in}{2.669239in}}%
\pgfpathcurveto{\pgfqpoint{0.752703in}{2.675063in}}{\pgfqpoint{0.755976in}{2.682963in}}{\pgfqpoint{0.755976in}{2.691199in}}%
\pgfpathcurveto{\pgfqpoint{0.755976in}{2.699435in}}{\pgfqpoint{0.752703in}{2.707335in}}{\pgfqpoint{0.746880in}{2.713159in}}%
\pgfpathcurveto{\pgfqpoint{0.741056in}{2.718983in}}{\pgfqpoint{0.733156in}{2.722255in}}{\pgfqpoint{0.724919in}{2.722255in}}%
\pgfpathcurveto{\pgfqpoint{0.716683in}{2.722255in}}{\pgfqpoint{0.708783in}{2.718983in}}{\pgfqpoint{0.702959in}{2.713159in}}%
\pgfpathcurveto{\pgfqpoint{0.697135in}{2.707335in}}{\pgfqpoint{0.693863in}{2.699435in}}{\pgfqpoint{0.693863in}{2.691199in}}%
\pgfpathcurveto{\pgfqpoint{0.693863in}{2.682963in}}{\pgfqpoint{0.697135in}{2.675063in}}{\pgfqpoint{0.702959in}{2.669239in}}%
\pgfpathcurveto{\pgfqpoint{0.708783in}{2.663415in}}{\pgfqpoint{0.716683in}{2.660142in}}{\pgfqpoint{0.724919in}{2.660142in}}%
\pgfpathclose%
\pgfusepath{stroke,fill}%
\end{pgfscope}%
\begin{pgfscope}%
\pgfpathrectangle{\pgfqpoint{0.100000in}{0.212622in}}{\pgfqpoint{3.696000in}{3.696000in}}%
\pgfusepath{clip}%
\pgfsetbuttcap%
\pgfsetroundjoin%
\definecolor{currentfill}{rgb}{0.121569,0.466667,0.705882}%
\pgfsetfillcolor{currentfill}%
\pgfsetfillopacity{0.805230}%
\pgfsetlinewidth{1.003750pt}%
\definecolor{currentstroke}{rgb}{0.121569,0.466667,0.705882}%
\pgfsetstrokecolor{currentstroke}%
\pgfsetstrokeopacity{0.805230}%
\pgfsetdash{}{0pt}%
\pgfpathmoveto{\pgfqpoint{0.705888in}{2.675141in}}%
\pgfpathcurveto{\pgfqpoint{0.714124in}{2.675141in}}{\pgfqpoint{0.722024in}{2.678414in}}{\pgfqpoint{0.727848in}{2.684238in}}%
\pgfpathcurveto{\pgfqpoint{0.733672in}{2.690062in}}{\pgfqpoint{0.736945in}{2.697962in}}{\pgfqpoint{0.736945in}{2.706198in}}%
\pgfpathcurveto{\pgfqpoint{0.736945in}{2.714434in}}{\pgfqpoint{0.733672in}{2.722334in}}{\pgfqpoint{0.727848in}{2.728158in}}%
\pgfpathcurveto{\pgfqpoint{0.722024in}{2.733982in}}{\pgfqpoint{0.714124in}{2.737254in}}{\pgfqpoint{0.705888in}{2.737254in}}%
\pgfpathcurveto{\pgfqpoint{0.697652in}{2.737254in}}{\pgfqpoint{0.689752in}{2.733982in}}{\pgfqpoint{0.683928in}{2.728158in}}%
\pgfpathcurveto{\pgfqpoint{0.678104in}{2.722334in}}{\pgfqpoint{0.674832in}{2.714434in}}{\pgfqpoint{0.674832in}{2.706198in}}%
\pgfpathcurveto{\pgfqpoint{0.674832in}{2.697962in}}{\pgfqpoint{0.678104in}{2.690062in}}{\pgfqpoint{0.683928in}{2.684238in}}%
\pgfpathcurveto{\pgfqpoint{0.689752in}{2.678414in}}{\pgfqpoint{0.697652in}{2.675141in}}{\pgfqpoint{0.705888in}{2.675141in}}%
\pgfpathclose%
\pgfusepath{stroke,fill}%
\end{pgfscope}%
\begin{pgfscope}%
\pgfpathrectangle{\pgfqpoint{0.100000in}{0.212622in}}{\pgfqpoint{3.696000in}{3.696000in}}%
\pgfusepath{clip}%
\pgfsetbuttcap%
\pgfsetroundjoin%
\definecolor{currentfill}{rgb}{0.121569,0.466667,0.705882}%
\pgfsetfillcolor{currentfill}%
\pgfsetfillopacity{0.805267}%
\pgfsetlinewidth{1.003750pt}%
\definecolor{currentstroke}{rgb}{0.121569,0.466667,0.705882}%
\pgfsetstrokecolor{currentstroke}%
\pgfsetstrokeopacity{0.805267}%
\pgfsetdash{}{0pt}%
\pgfpathmoveto{\pgfqpoint{0.751516in}{2.653391in}}%
\pgfpathcurveto{\pgfqpoint{0.759753in}{2.653391in}}{\pgfqpoint{0.767653in}{2.656663in}}{\pgfqpoint{0.773477in}{2.662487in}}%
\pgfpathcurveto{\pgfqpoint{0.779301in}{2.668311in}}{\pgfqpoint{0.782573in}{2.676211in}}{\pgfqpoint{0.782573in}{2.684448in}}%
\pgfpathcurveto{\pgfqpoint{0.782573in}{2.692684in}}{\pgfqpoint{0.779301in}{2.700584in}}{\pgfqpoint{0.773477in}{2.706408in}}%
\pgfpathcurveto{\pgfqpoint{0.767653in}{2.712232in}}{\pgfqpoint{0.759753in}{2.715504in}}{\pgfqpoint{0.751516in}{2.715504in}}%
\pgfpathcurveto{\pgfqpoint{0.743280in}{2.715504in}}{\pgfqpoint{0.735380in}{2.712232in}}{\pgfqpoint{0.729556in}{2.706408in}}%
\pgfpathcurveto{\pgfqpoint{0.723732in}{2.700584in}}{\pgfqpoint{0.720460in}{2.692684in}}{\pgfqpoint{0.720460in}{2.684448in}}%
\pgfpathcurveto{\pgfqpoint{0.720460in}{2.676211in}}{\pgfqpoint{0.723732in}{2.668311in}}{\pgfqpoint{0.729556in}{2.662487in}}%
\pgfpathcurveto{\pgfqpoint{0.735380in}{2.656663in}}{\pgfqpoint{0.743280in}{2.653391in}}{\pgfqpoint{0.751516in}{2.653391in}}%
\pgfpathclose%
\pgfusepath{stroke,fill}%
\end{pgfscope}%
\begin{pgfscope}%
\pgfpathrectangle{\pgfqpoint{0.100000in}{0.212622in}}{\pgfqpoint{3.696000in}{3.696000in}}%
\pgfusepath{clip}%
\pgfsetbuttcap%
\pgfsetroundjoin%
\definecolor{currentfill}{rgb}{0.121569,0.466667,0.705882}%
\pgfsetfillcolor{currentfill}%
\pgfsetfillopacity{0.805489}%
\pgfsetlinewidth{1.003750pt}%
\definecolor{currentstroke}{rgb}{0.121569,0.466667,0.705882}%
\pgfsetstrokecolor{currentstroke}%
\pgfsetstrokeopacity{0.805489}%
\pgfsetdash{}{0pt}%
\pgfpathmoveto{\pgfqpoint{2.849418in}{2.051764in}}%
\pgfpathcurveto{\pgfqpoint{2.857654in}{2.051764in}}{\pgfqpoint{2.865554in}{2.055036in}}{\pgfqpoint{2.871378in}{2.060860in}}%
\pgfpathcurveto{\pgfqpoint{2.877202in}{2.066684in}}{\pgfqpoint{2.880475in}{2.074584in}}{\pgfqpoint{2.880475in}{2.082820in}}%
\pgfpathcurveto{\pgfqpoint{2.880475in}{2.091057in}}{\pgfqpoint{2.877202in}{2.098957in}}{\pgfqpoint{2.871378in}{2.104781in}}%
\pgfpathcurveto{\pgfqpoint{2.865554in}{2.110605in}}{\pgfqpoint{2.857654in}{2.113877in}}{\pgfqpoint{2.849418in}{2.113877in}}%
\pgfpathcurveto{\pgfqpoint{2.841182in}{2.113877in}}{\pgfqpoint{2.833282in}{2.110605in}}{\pgfqpoint{2.827458in}{2.104781in}}%
\pgfpathcurveto{\pgfqpoint{2.821634in}{2.098957in}}{\pgfqpoint{2.818362in}{2.091057in}}{\pgfqpoint{2.818362in}{2.082820in}}%
\pgfpathcurveto{\pgfqpoint{2.818362in}{2.074584in}}{\pgfqpoint{2.821634in}{2.066684in}}{\pgfqpoint{2.827458in}{2.060860in}}%
\pgfpathcurveto{\pgfqpoint{2.833282in}{2.055036in}}{\pgfqpoint{2.841182in}{2.051764in}}{\pgfqpoint{2.849418in}{2.051764in}}%
\pgfpathclose%
\pgfusepath{stroke,fill}%
\end{pgfscope}%
\begin{pgfscope}%
\pgfpathrectangle{\pgfqpoint{0.100000in}{0.212622in}}{\pgfqpoint{3.696000in}{3.696000in}}%
\pgfusepath{clip}%
\pgfsetbuttcap%
\pgfsetroundjoin%
\definecolor{currentfill}{rgb}{0.121569,0.466667,0.705882}%
\pgfsetfillcolor{currentfill}%
\pgfsetfillopacity{0.805758}%
\pgfsetlinewidth{1.003750pt}%
\definecolor{currentstroke}{rgb}{0.121569,0.466667,0.705882}%
\pgfsetstrokecolor{currentstroke}%
\pgfsetstrokeopacity{0.805758}%
\pgfsetdash{}{0pt}%
\pgfpathmoveto{\pgfqpoint{0.737995in}{2.661962in}}%
\pgfpathcurveto{\pgfqpoint{0.746232in}{2.661962in}}{\pgfqpoint{0.754132in}{2.665235in}}{\pgfqpoint{0.759956in}{2.671059in}}%
\pgfpathcurveto{\pgfqpoint{0.765780in}{2.676883in}}{\pgfqpoint{0.769052in}{2.684783in}}{\pgfqpoint{0.769052in}{2.693019in}}%
\pgfpathcurveto{\pgfqpoint{0.769052in}{2.701255in}}{\pgfqpoint{0.765780in}{2.709155in}}{\pgfqpoint{0.759956in}{2.714979in}}%
\pgfpathcurveto{\pgfqpoint{0.754132in}{2.720803in}}{\pgfqpoint{0.746232in}{2.724075in}}{\pgfqpoint{0.737995in}{2.724075in}}%
\pgfpathcurveto{\pgfqpoint{0.729759in}{2.724075in}}{\pgfqpoint{0.721859in}{2.720803in}}{\pgfqpoint{0.716035in}{2.714979in}}%
\pgfpathcurveto{\pgfqpoint{0.710211in}{2.709155in}}{\pgfqpoint{0.706939in}{2.701255in}}{\pgfqpoint{0.706939in}{2.693019in}}%
\pgfpathcurveto{\pgfqpoint{0.706939in}{2.684783in}}{\pgfqpoint{0.710211in}{2.676883in}}{\pgfqpoint{0.716035in}{2.671059in}}%
\pgfpathcurveto{\pgfqpoint{0.721859in}{2.665235in}}{\pgfqpoint{0.729759in}{2.661962in}}{\pgfqpoint{0.737995in}{2.661962in}}%
\pgfpathclose%
\pgfusepath{stroke,fill}%
\end{pgfscope}%
\begin{pgfscope}%
\pgfpathrectangle{\pgfqpoint{0.100000in}{0.212622in}}{\pgfqpoint{3.696000in}{3.696000in}}%
\pgfusepath{clip}%
\pgfsetbuttcap%
\pgfsetroundjoin%
\definecolor{currentfill}{rgb}{0.121569,0.466667,0.705882}%
\pgfsetfillcolor{currentfill}%
\pgfsetfillopacity{0.807139}%
\pgfsetlinewidth{1.003750pt}%
\definecolor{currentstroke}{rgb}{0.121569,0.466667,0.705882}%
\pgfsetstrokecolor{currentstroke}%
\pgfsetstrokeopacity{0.807139}%
\pgfsetdash{}{0pt}%
\pgfpathmoveto{\pgfqpoint{0.760543in}{2.656476in}}%
\pgfpathcurveto{\pgfqpoint{0.768780in}{2.656476in}}{\pgfqpoint{0.776680in}{2.659748in}}{\pgfqpoint{0.782504in}{2.665572in}}%
\pgfpathcurveto{\pgfqpoint{0.788328in}{2.671396in}}{\pgfqpoint{0.791600in}{2.679296in}}{\pgfqpoint{0.791600in}{2.687532in}}%
\pgfpathcurveto{\pgfqpoint{0.791600in}{2.695769in}}{\pgfqpoint{0.788328in}{2.703669in}}{\pgfqpoint{0.782504in}{2.709493in}}%
\pgfpathcurveto{\pgfqpoint{0.776680in}{2.715317in}}{\pgfqpoint{0.768780in}{2.718589in}}{\pgfqpoint{0.760543in}{2.718589in}}%
\pgfpathcurveto{\pgfqpoint{0.752307in}{2.718589in}}{\pgfqpoint{0.744407in}{2.715317in}}{\pgfqpoint{0.738583in}{2.709493in}}%
\pgfpathcurveto{\pgfqpoint{0.732759in}{2.703669in}}{\pgfqpoint{0.729487in}{2.695769in}}{\pgfqpoint{0.729487in}{2.687532in}}%
\pgfpathcurveto{\pgfqpoint{0.729487in}{2.679296in}}{\pgfqpoint{0.732759in}{2.671396in}}{\pgfqpoint{0.738583in}{2.665572in}}%
\pgfpathcurveto{\pgfqpoint{0.744407in}{2.659748in}}{\pgfqpoint{0.752307in}{2.656476in}}{\pgfqpoint{0.760543in}{2.656476in}}%
\pgfpathclose%
\pgfusepath{stroke,fill}%
\end{pgfscope}%
\begin{pgfscope}%
\pgfpathrectangle{\pgfqpoint{0.100000in}{0.212622in}}{\pgfqpoint{3.696000in}{3.696000in}}%
\pgfusepath{clip}%
\pgfsetbuttcap%
\pgfsetroundjoin%
\definecolor{currentfill}{rgb}{0.121569,0.466667,0.705882}%
\pgfsetfillcolor{currentfill}%
\pgfsetfillopacity{0.808597}%
\pgfsetlinewidth{1.003750pt}%
\definecolor{currentstroke}{rgb}{0.121569,0.466667,0.705882}%
\pgfsetstrokecolor{currentstroke}%
\pgfsetstrokeopacity{0.808597}%
\pgfsetdash{}{0pt}%
\pgfpathmoveto{\pgfqpoint{0.776362in}{2.645692in}}%
\pgfpathcurveto{\pgfqpoint{0.784599in}{2.645692in}}{\pgfqpoint{0.792499in}{2.648964in}}{\pgfqpoint{0.798323in}{2.654788in}}%
\pgfpathcurveto{\pgfqpoint{0.804147in}{2.660612in}}{\pgfqpoint{0.807419in}{2.668512in}}{\pgfqpoint{0.807419in}{2.676748in}}%
\pgfpathcurveto{\pgfqpoint{0.807419in}{2.684984in}}{\pgfqpoint{0.804147in}{2.692884in}}{\pgfqpoint{0.798323in}{2.698708in}}%
\pgfpathcurveto{\pgfqpoint{0.792499in}{2.704532in}}{\pgfqpoint{0.784599in}{2.707805in}}{\pgfqpoint{0.776362in}{2.707805in}}%
\pgfpathcurveto{\pgfqpoint{0.768126in}{2.707805in}}{\pgfqpoint{0.760226in}{2.704532in}}{\pgfqpoint{0.754402in}{2.698708in}}%
\pgfpathcurveto{\pgfqpoint{0.748578in}{2.692884in}}{\pgfqpoint{0.745306in}{2.684984in}}{\pgfqpoint{0.745306in}{2.676748in}}%
\pgfpathcurveto{\pgfqpoint{0.745306in}{2.668512in}}{\pgfqpoint{0.748578in}{2.660612in}}{\pgfqpoint{0.754402in}{2.654788in}}%
\pgfpathcurveto{\pgfqpoint{0.760226in}{2.648964in}}{\pgfqpoint{0.768126in}{2.645692in}}{\pgfqpoint{0.776362in}{2.645692in}}%
\pgfpathclose%
\pgfusepath{stroke,fill}%
\end{pgfscope}%
\begin{pgfscope}%
\pgfpathrectangle{\pgfqpoint{0.100000in}{0.212622in}}{\pgfqpoint{3.696000in}{3.696000in}}%
\pgfusepath{clip}%
\pgfsetbuttcap%
\pgfsetroundjoin%
\definecolor{currentfill}{rgb}{0.121569,0.466667,0.705882}%
\pgfsetfillcolor{currentfill}%
\pgfsetfillopacity{0.809006}%
\pgfsetlinewidth{1.003750pt}%
\definecolor{currentstroke}{rgb}{0.121569,0.466667,0.705882}%
\pgfsetstrokecolor{currentstroke}%
\pgfsetstrokeopacity{0.809006}%
\pgfsetdash{}{0pt}%
\pgfpathmoveto{\pgfqpoint{2.839405in}{2.047949in}}%
\pgfpathcurveto{\pgfqpoint{2.847641in}{2.047949in}}{\pgfqpoint{2.855541in}{2.051222in}}{\pgfqpoint{2.861365in}{2.057046in}}%
\pgfpathcurveto{\pgfqpoint{2.867189in}{2.062870in}}{\pgfqpoint{2.870461in}{2.070770in}}{\pgfqpoint{2.870461in}{2.079006in}}%
\pgfpathcurveto{\pgfqpoint{2.870461in}{2.087242in}}{\pgfqpoint{2.867189in}{2.095142in}}{\pgfqpoint{2.861365in}{2.100966in}}%
\pgfpathcurveto{\pgfqpoint{2.855541in}{2.106790in}}{\pgfqpoint{2.847641in}{2.110062in}}{\pgfqpoint{2.839405in}{2.110062in}}%
\pgfpathcurveto{\pgfqpoint{2.831169in}{2.110062in}}{\pgfqpoint{2.823269in}{2.106790in}}{\pgfqpoint{2.817445in}{2.100966in}}%
\pgfpathcurveto{\pgfqpoint{2.811621in}{2.095142in}}{\pgfqpoint{2.808348in}{2.087242in}}{\pgfqpoint{2.808348in}{2.079006in}}%
\pgfpathcurveto{\pgfqpoint{2.808348in}{2.070770in}}{\pgfqpoint{2.811621in}{2.062870in}}{\pgfqpoint{2.817445in}{2.057046in}}%
\pgfpathcurveto{\pgfqpoint{2.823269in}{2.051222in}}{\pgfqpoint{2.831169in}{2.047949in}}{\pgfqpoint{2.839405in}{2.047949in}}%
\pgfpathclose%
\pgfusepath{stroke,fill}%
\end{pgfscope}%
\begin{pgfscope}%
\pgfpathrectangle{\pgfqpoint{0.100000in}{0.212622in}}{\pgfqpoint{3.696000in}{3.696000in}}%
\pgfusepath{clip}%
\pgfsetbuttcap%
\pgfsetroundjoin%
\definecolor{currentfill}{rgb}{0.121569,0.466667,0.705882}%
\pgfsetfillcolor{currentfill}%
\pgfsetfillopacity{0.809310}%
\pgfsetlinewidth{1.003750pt}%
\definecolor{currentstroke}{rgb}{0.121569,0.466667,0.705882}%
\pgfsetstrokecolor{currentstroke}%
\pgfsetstrokeopacity{0.809310}%
\pgfsetdash{}{0pt}%
\pgfpathmoveto{\pgfqpoint{0.786008in}{2.639378in}}%
\pgfpathcurveto{\pgfqpoint{0.794244in}{2.639378in}}{\pgfqpoint{0.802144in}{2.642651in}}{\pgfqpoint{0.807968in}{2.648475in}}%
\pgfpathcurveto{\pgfqpoint{0.813792in}{2.654299in}}{\pgfqpoint{0.817064in}{2.662199in}}{\pgfqpoint{0.817064in}{2.670435in}}%
\pgfpathcurveto{\pgfqpoint{0.817064in}{2.678671in}}{\pgfqpoint{0.813792in}{2.686571in}}{\pgfqpoint{0.807968in}{2.692395in}}%
\pgfpathcurveto{\pgfqpoint{0.802144in}{2.698219in}}{\pgfqpoint{0.794244in}{2.701491in}}{\pgfqpoint{0.786008in}{2.701491in}}%
\pgfpathcurveto{\pgfqpoint{0.777772in}{2.701491in}}{\pgfqpoint{0.769872in}{2.698219in}}{\pgfqpoint{0.764048in}{2.692395in}}%
\pgfpathcurveto{\pgfqpoint{0.758224in}{2.686571in}}{\pgfqpoint{0.754951in}{2.678671in}}{\pgfqpoint{0.754951in}{2.670435in}}%
\pgfpathcurveto{\pgfqpoint{0.754951in}{2.662199in}}{\pgfqpoint{0.758224in}{2.654299in}}{\pgfqpoint{0.764048in}{2.648475in}}%
\pgfpathcurveto{\pgfqpoint{0.769872in}{2.642651in}}{\pgfqpoint{0.777772in}{2.639378in}}{\pgfqpoint{0.786008in}{2.639378in}}%
\pgfpathclose%
\pgfusepath{stroke,fill}%
\end{pgfscope}%
\begin{pgfscope}%
\pgfpathrectangle{\pgfqpoint{0.100000in}{0.212622in}}{\pgfqpoint{3.696000in}{3.696000in}}%
\pgfusepath{clip}%
\pgfsetbuttcap%
\pgfsetroundjoin%
\definecolor{currentfill}{rgb}{0.121569,0.466667,0.705882}%
\pgfsetfillcolor{currentfill}%
\pgfsetfillopacity{0.809760}%
\pgfsetlinewidth{1.003750pt}%
\definecolor{currentstroke}{rgb}{0.121569,0.466667,0.705882}%
\pgfsetstrokecolor{currentstroke}%
\pgfsetstrokeopacity{0.809760}%
\pgfsetdash{}{0pt}%
\pgfpathmoveto{\pgfqpoint{0.790396in}{2.636968in}}%
\pgfpathcurveto{\pgfqpoint{0.798633in}{2.636968in}}{\pgfqpoint{0.806533in}{2.640240in}}{\pgfqpoint{0.812357in}{2.646064in}}%
\pgfpathcurveto{\pgfqpoint{0.818180in}{2.651888in}}{\pgfqpoint{0.821453in}{2.659788in}}{\pgfqpoint{0.821453in}{2.668025in}}%
\pgfpathcurveto{\pgfqpoint{0.821453in}{2.676261in}}{\pgfqpoint{0.818180in}{2.684161in}}{\pgfqpoint{0.812357in}{2.689985in}}%
\pgfpathcurveto{\pgfqpoint{0.806533in}{2.695809in}}{\pgfqpoint{0.798633in}{2.699081in}}{\pgfqpoint{0.790396in}{2.699081in}}%
\pgfpathcurveto{\pgfqpoint{0.782160in}{2.699081in}}{\pgfqpoint{0.774260in}{2.695809in}}{\pgfqpoint{0.768436in}{2.689985in}}%
\pgfpathcurveto{\pgfqpoint{0.762612in}{2.684161in}}{\pgfqpoint{0.759340in}{2.676261in}}{\pgfqpoint{0.759340in}{2.668025in}}%
\pgfpathcurveto{\pgfqpoint{0.759340in}{2.659788in}}{\pgfqpoint{0.762612in}{2.651888in}}{\pgfqpoint{0.768436in}{2.646064in}}%
\pgfpathcurveto{\pgfqpoint{0.774260in}{2.640240in}}{\pgfqpoint{0.782160in}{2.636968in}}{\pgfqpoint{0.790396in}{2.636968in}}%
\pgfpathclose%
\pgfusepath{stroke,fill}%
\end{pgfscope}%
\begin{pgfscope}%
\pgfpathrectangle{\pgfqpoint{0.100000in}{0.212622in}}{\pgfqpoint{3.696000in}{3.696000in}}%
\pgfusepath{clip}%
\pgfsetbuttcap%
\pgfsetroundjoin%
\definecolor{currentfill}{rgb}{0.121569,0.466667,0.705882}%
\pgfsetfillcolor{currentfill}%
\pgfsetfillopacity{0.810935}%
\pgfsetlinewidth{1.003750pt}%
\definecolor{currentstroke}{rgb}{0.121569,0.466667,0.705882}%
\pgfsetstrokecolor{currentstroke}%
\pgfsetstrokeopacity{0.810935}%
\pgfsetdash{}{0pt}%
\pgfpathmoveto{\pgfqpoint{0.796931in}{2.631222in}}%
\pgfpathcurveto{\pgfqpoint{0.805168in}{2.631222in}}{\pgfqpoint{0.813068in}{2.634494in}}{\pgfqpoint{0.818892in}{2.640318in}}%
\pgfpathcurveto{\pgfqpoint{0.824716in}{2.646142in}}{\pgfqpoint{0.827988in}{2.654042in}}{\pgfqpoint{0.827988in}{2.662279in}}%
\pgfpathcurveto{\pgfqpoint{0.827988in}{2.670515in}}{\pgfqpoint{0.824716in}{2.678415in}}{\pgfqpoint{0.818892in}{2.684239in}}%
\pgfpathcurveto{\pgfqpoint{0.813068in}{2.690063in}}{\pgfqpoint{0.805168in}{2.693335in}}{\pgfqpoint{0.796931in}{2.693335in}}%
\pgfpathcurveto{\pgfqpoint{0.788695in}{2.693335in}}{\pgfqpoint{0.780795in}{2.690063in}}{\pgfqpoint{0.774971in}{2.684239in}}%
\pgfpathcurveto{\pgfqpoint{0.769147in}{2.678415in}}{\pgfqpoint{0.765875in}{2.670515in}}{\pgfqpoint{0.765875in}{2.662279in}}%
\pgfpathcurveto{\pgfqpoint{0.765875in}{2.654042in}}{\pgfqpoint{0.769147in}{2.646142in}}{\pgfqpoint{0.774971in}{2.640318in}}%
\pgfpathcurveto{\pgfqpoint{0.780795in}{2.634494in}}{\pgfqpoint{0.788695in}{2.631222in}}{\pgfqpoint{0.796931in}{2.631222in}}%
\pgfpathclose%
\pgfusepath{stroke,fill}%
\end{pgfscope}%
\begin{pgfscope}%
\pgfpathrectangle{\pgfqpoint{0.100000in}{0.212622in}}{\pgfqpoint{3.696000in}{3.696000in}}%
\pgfusepath{clip}%
\pgfsetbuttcap%
\pgfsetroundjoin%
\definecolor{currentfill}{rgb}{0.121569,0.466667,0.705882}%
\pgfsetfillcolor{currentfill}%
\pgfsetfillopacity{0.811853}%
\pgfsetlinewidth{1.003750pt}%
\definecolor{currentstroke}{rgb}{0.121569,0.466667,0.705882}%
\pgfsetstrokecolor{currentstroke}%
\pgfsetstrokeopacity{0.811853}%
\pgfsetdash{}{0pt}%
\pgfpathmoveto{\pgfqpoint{0.802304in}{2.627677in}}%
\pgfpathcurveto{\pgfqpoint{0.810540in}{2.627677in}}{\pgfqpoint{0.818440in}{2.630950in}}{\pgfqpoint{0.824264in}{2.636774in}}%
\pgfpathcurveto{\pgfqpoint{0.830088in}{2.642597in}}{\pgfqpoint{0.833360in}{2.650498in}}{\pgfqpoint{0.833360in}{2.658734in}}%
\pgfpathcurveto{\pgfqpoint{0.833360in}{2.666970in}}{\pgfqpoint{0.830088in}{2.674870in}}{\pgfqpoint{0.824264in}{2.680694in}}%
\pgfpathcurveto{\pgfqpoint{0.818440in}{2.686518in}}{\pgfqpoint{0.810540in}{2.689790in}}{\pgfqpoint{0.802304in}{2.689790in}}%
\pgfpathcurveto{\pgfqpoint{0.794067in}{2.689790in}}{\pgfqpoint{0.786167in}{2.686518in}}{\pgfqpoint{0.780343in}{2.680694in}}%
\pgfpathcurveto{\pgfqpoint{0.774520in}{2.674870in}}{\pgfqpoint{0.771247in}{2.666970in}}{\pgfqpoint{0.771247in}{2.658734in}}%
\pgfpathcurveto{\pgfqpoint{0.771247in}{2.650498in}}{\pgfqpoint{0.774520in}{2.642597in}}{\pgfqpoint{0.780343in}{2.636774in}}%
\pgfpathcurveto{\pgfqpoint{0.786167in}{2.630950in}}{\pgfqpoint{0.794067in}{2.627677in}}{\pgfqpoint{0.802304in}{2.627677in}}%
\pgfpathclose%
\pgfusepath{stroke,fill}%
\end{pgfscope}%
\begin{pgfscope}%
\pgfpathrectangle{\pgfqpoint{0.100000in}{0.212622in}}{\pgfqpoint{3.696000in}{3.696000in}}%
\pgfusepath{clip}%
\pgfsetbuttcap%
\pgfsetroundjoin%
\definecolor{currentfill}{rgb}{0.121569,0.466667,0.705882}%
\pgfsetfillcolor{currentfill}%
\pgfsetfillopacity{0.811984}%
\pgfsetlinewidth{1.003750pt}%
\definecolor{currentstroke}{rgb}{0.121569,0.466667,0.705882}%
\pgfsetstrokecolor{currentstroke}%
\pgfsetstrokeopacity{0.811984}%
\pgfsetdash{}{0pt}%
\pgfpathmoveto{\pgfqpoint{0.803139in}{2.626931in}}%
\pgfpathcurveto{\pgfqpoint{0.811375in}{2.626931in}}{\pgfqpoint{0.819275in}{2.630204in}}{\pgfqpoint{0.825099in}{2.636028in}}%
\pgfpathcurveto{\pgfqpoint{0.830923in}{2.641852in}}{\pgfqpoint{0.834195in}{2.649752in}}{\pgfqpoint{0.834195in}{2.657988in}}%
\pgfpathcurveto{\pgfqpoint{0.834195in}{2.666224in}}{\pgfqpoint{0.830923in}{2.674124in}}{\pgfqpoint{0.825099in}{2.679948in}}%
\pgfpathcurveto{\pgfqpoint{0.819275in}{2.685772in}}{\pgfqpoint{0.811375in}{2.689044in}}{\pgfqpoint{0.803139in}{2.689044in}}%
\pgfpathcurveto{\pgfqpoint{0.794903in}{2.689044in}}{\pgfqpoint{0.787002in}{2.685772in}}{\pgfqpoint{0.781179in}{2.679948in}}%
\pgfpathcurveto{\pgfqpoint{0.775355in}{2.674124in}}{\pgfqpoint{0.772082in}{2.666224in}}{\pgfqpoint{0.772082in}{2.657988in}}%
\pgfpathcurveto{\pgfqpoint{0.772082in}{2.649752in}}{\pgfqpoint{0.775355in}{2.641852in}}{\pgfqpoint{0.781179in}{2.636028in}}%
\pgfpathcurveto{\pgfqpoint{0.787002in}{2.630204in}}{\pgfqpoint{0.794903in}{2.626931in}}{\pgfqpoint{0.803139in}{2.626931in}}%
\pgfpathclose%
\pgfusepath{stroke,fill}%
\end{pgfscope}%
\begin{pgfscope}%
\pgfpathrectangle{\pgfqpoint{0.100000in}{0.212622in}}{\pgfqpoint{3.696000in}{3.696000in}}%
\pgfusepath{clip}%
\pgfsetbuttcap%
\pgfsetroundjoin%
\definecolor{currentfill}{rgb}{0.121569,0.466667,0.705882}%
\pgfsetfillcolor{currentfill}%
\pgfsetfillopacity{0.812114}%
\pgfsetlinewidth{1.003750pt}%
\definecolor{currentstroke}{rgb}{0.121569,0.466667,0.705882}%
\pgfsetstrokecolor{currentstroke}%
\pgfsetstrokeopacity{0.812114}%
\pgfsetdash{}{0pt}%
\pgfpathmoveto{\pgfqpoint{0.804874in}{2.625378in}}%
\pgfpathcurveto{\pgfqpoint{0.813110in}{2.625378in}}{\pgfqpoint{0.821010in}{2.628651in}}{\pgfqpoint{0.826834in}{2.634475in}}%
\pgfpathcurveto{\pgfqpoint{0.832658in}{2.640299in}}{\pgfqpoint{0.835930in}{2.648199in}}{\pgfqpoint{0.835930in}{2.656435in}}%
\pgfpathcurveto{\pgfqpoint{0.835930in}{2.664671in}}{\pgfqpoint{0.832658in}{2.672571in}}{\pgfqpoint{0.826834in}{2.678395in}}%
\pgfpathcurveto{\pgfqpoint{0.821010in}{2.684219in}}{\pgfqpoint{0.813110in}{2.687491in}}{\pgfqpoint{0.804874in}{2.687491in}}%
\pgfpathcurveto{\pgfqpoint{0.796637in}{2.687491in}}{\pgfqpoint{0.788737in}{2.684219in}}{\pgfqpoint{0.782913in}{2.678395in}}%
\pgfpathcurveto{\pgfqpoint{0.777089in}{2.672571in}}{\pgfqpoint{0.773817in}{2.664671in}}{\pgfqpoint{0.773817in}{2.656435in}}%
\pgfpathcurveto{\pgfqpoint{0.773817in}{2.648199in}}{\pgfqpoint{0.777089in}{2.640299in}}{\pgfqpoint{0.782913in}{2.634475in}}%
\pgfpathcurveto{\pgfqpoint{0.788737in}{2.628651in}}{\pgfqpoint{0.796637in}{2.625378in}}{\pgfqpoint{0.804874in}{2.625378in}}%
\pgfpathclose%
\pgfusepath{stroke,fill}%
\end{pgfscope}%
\begin{pgfscope}%
\pgfpathrectangle{\pgfqpoint{0.100000in}{0.212622in}}{\pgfqpoint{3.696000in}{3.696000in}}%
\pgfusepath{clip}%
\pgfsetbuttcap%
\pgfsetroundjoin%
\definecolor{currentfill}{rgb}{0.121569,0.466667,0.705882}%
\pgfsetfillcolor{currentfill}%
\pgfsetfillopacity{0.812683}%
\pgfsetlinewidth{1.003750pt}%
\definecolor{currentstroke}{rgb}{0.121569,0.466667,0.705882}%
\pgfsetstrokecolor{currentstroke}%
\pgfsetstrokeopacity{0.812683}%
\pgfsetdash{}{0pt}%
\pgfpathmoveto{\pgfqpoint{0.807586in}{2.623738in}}%
\pgfpathcurveto{\pgfqpoint{0.815823in}{2.623738in}}{\pgfqpoint{0.823723in}{2.627010in}}{\pgfqpoint{0.829547in}{2.632834in}}%
\pgfpathcurveto{\pgfqpoint{0.835370in}{2.638658in}}{\pgfqpoint{0.838643in}{2.646558in}}{\pgfqpoint{0.838643in}{2.654794in}}%
\pgfpathcurveto{\pgfqpoint{0.838643in}{2.663031in}}{\pgfqpoint{0.835370in}{2.670931in}}{\pgfqpoint{0.829547in}{2.676755in}}%
\pgfpathcurveto{\pgfqpoint{0.823723in}{2.682579in}}{\pgfqpoint{0.815823in}{2.685851in}}{\pgfqpoint{0.807586in}{2.685851in}}%
\pgfpathcurveto{\pgfqpoint{0.799350in}{2.685851in}}{\pgfqpoint{0.791450in}{2.682579in}}{\pgfqpoint{0.785626in}{2.676755in}}%
\pgfpathcurveto{\pgfqpoint{0.779802in}{2.670931in}}{\pgfqpoint{0.776530in}{2.663031in}}{\pgfqpoint{0.776530in}{2.654794in}}%
\pgfpathcurveto{\pgfqpoint{0.776530in}{2.646558in}}{\pgfqpoint{0.779802in}{2.638658in}}{\pgfqpoint{0.785626in}{2.632834in}}%
\pgfpathcurveto{\pgfqpoint{0.791450in}{2.627010in}}{\pgfqpoint{0.799350in}{2.623738in}}{\pgfqpoint{0.807586in}{2.623738in}}%
\pgfpathclose%
\pgfusepath{stroke,fill}%
\end{pgfscope}%
\begin{pgfscope}%
\pgfpathrectangle{\pgfqpoint{0.100000in}{0.212622in}}{\pgfqpoint{3.696000in}{3.696000in}}%
\pgfusepath{clip}%
\pgfsetbuttcap%
\pgfsetroundjoin%
\definecolor{currentfill}{rgb}{0.121569,0.466667,0.705882}%
\pgfsetfillcolor{currentfill}%
\pgfsetfillopacity{0.813318}%
\pgfsetlinewidth{1.003750pt}%
\definecolor{currentstroke}{rgb}{0.121569,0.466667,0.705882}%
\pgfsetstrokecolor{currentstroke}%
\pgfsetstrokeopacity{0.813318}%
\pgfsetdash{}{0pt}%
\pgfpathmoveto{\pgfqpoint{0.813048in}{2.619251in}}%
\pgfpathcurveto{\pgfqpoint{0.821284in}{2.619251in}}{\pgfqpoint{0.829184in}{2.622523in}}{\pgfqpoint{0.835008in}{2.628347in}}%
\pgfpathcurveto{\pgfqpoint{0.840832in}{2.634171in}}{\pgfqpoint{0.844104in}{2.642071in}}{\pgfqpoint{0.844104in}{2.650307in}}%
\pgfpathcurveto{\pgfqpoint{0.844104in}{2.658544in}}{\pgfqpoint{0.840832in}{2.666444in}}{\pgfqpoint{0.835008in}{2.672268in}}%
\pgfpathcurveto{\pgfqpoint{0.829184in}{2.678091in}}{\pgfqpoint{0.821284in}{2.681364in}}{\pgfqpoint{0.813048in}{2.681364in}}%
\pgfpathcurveto{\pgfqpoint{0.804812in}{2.681364in}}{\pgfqpoint{0.796912in}{2.678091in}}{\pgfqpoint{0.791088in}{2.672268in}}%
\pgfpathcurveto{\pgfqpoint{0.785264in}{2.666444in}}{\pgfqpoint{0.781991in}{2.658544in}}{\pgfqpoint{0.781991in}{2.650307in}}%
\pgfpathcurveto{\pgfqpoint{0.781991in}{2.642071in}}{\pgfqpoint{0.785264in}{2.634171in}}{\pgfqpoint{0.791088in}{2.628347in}}%
\pgfpathcurveto{\pgfqpoint{0.796912in}{2.622523in}}{\pgfqpoint{0.804812in}{2.619251in}}{\pgfqpoint{0.813048in}{2.619251in}}%
\pgfpathclose%
\pgfusepath{stroke,fill}%
\end{pgfscope}%
\begin{pgfscope}%
\pgfpathrectangle{\pgfqpoint{0.100000in}{0.212622in}}{\pgfqpoint{3.696000in}{3.696000in}}%
\pgfusepath{clip}%
\pgfsetbuttcap%
\pgfsetroundjoin%
\definecolor{currentfill}{rgb}{0.121569,0.466667,0.705882}%
\pgfsetfillcolor{currentfill}%
\pgfsetfillopacity{0.813768}%
\pgfsetlinewidth{1.003750pt}%
\definecolor{currentstroke}{rgb}{0.121569,0.466667,0.705882}%
\pgfsetstrokecolor{currentstroke}%
\pgfsetstrokeopacity{0.813768}%
\pgfsetdash{}{0pt}%
\pgfpathmoveto{\pgfqpoint{2.829916in}{2.043560in}}%
\pgfpathcurveto{\pgfqpoint{2.838153in}{2.043560in}}{\pgfqpoint{2.846053in}{2.046832in}}{\pgfqpoint{2.851877in}{2.052656in}}%
\pgfpathcurveto{\pgfqpoint{2.857701in}{2.058480in}}{\pgfqpoint{2.860973in}{2.066380in}}{\pgfqpoint{2.860973in}{2.074616in}}%
\pgfpathcurveto{\pgfqpoint{2.860973in}{2.082853in}}{\pgfqpoint{2.857701in}{2.090753in}}{\pgfqpoint{2.851877in}{2.096576in}}%
\pgfpathcurveto{\pgfqpoint{2.846053in}{2.102400in}}{\pgfqpoint{2.838153in}{2.105673in}}{\pgfqpoint{2.829916in}{2.105673in}}%
\pgfpathcurveto{\pgfqpoint{2.821680in}{2.105673in}}{\pgfqpoint{2.813780in}{2.102400in}}{\pgfqpoint{2.807956in}{2.096576in}}%
\pgfpathcurveto{\pgfqpoint{2.802132in}{2.090753in}}{\pgfqpoint{2.798860in}{2.082853in}}{\pgfqpoint{2.798860in}{2.074616in}}%
\pgfpathcurveto{\pgfqpoint{2.798860in}{2.066380in}}{\pgfqpoint{2.802132in}{2.058480in}}{\pgfqpoint{2.807956in}{2.052656in}}%
\pgfpathcurveto{\pgfqpoint{2.813780in}{2.046832in}}{\pgfqpoint{2.821680in}{2.043560in}}{\pgfqpoint{2.829916in}{2.043560in}}%
\pgfpathclose%
\pgfusepath{stroke,fill}%
\end{pgfscope}%
\begin{pgfscope}%
\pgfpathrectangle{\pgfqpoint{0.100000in}{0.212622in}}{\pgfqpoint{3.696000in}{3.696000in}}%
\pgfusepath{clip}%
\pgfsetbuttcap%
\pgfsetroundjoin%
\definecolor{currentfill}{rgb}{0.121569,0.466667,0.705882}%
\pgfsetfillcolor{currentfill}%
\pgfsetfillopacity{0.814780}%
\pgfsetlinewidth{1.003750pt}%
\definecolor{currentstroke}{rgb}{0.121569,0.466667,0.705882}%
\pgfsetstrokecolor{currentstroke}%
\pgfsetstrokeopacity{0.814780}%
\pgfsetdash{}{0pt}%
\pgfpathmoveto{\pgfqpoint{0.822793in}{2.612799in}}%
\pgfpathcurveto{\pgfqpoint{0.831029in}{2.612799in}}{\pgfqpoint{0.838930in}{2.616072in}}{\pgfqpoint{0.844753in}{2.621896in}}%
\pgfpathcurveto{\pgfqpoint{0.850577in}{2.627719in}}{\pgfqpoint{0.853850in}{2.635620in}}{\pgfqpoint{0.853850in}{2.643856in}}%
\pgfpathcurveto{\pgfqpoint{0.853850in}{2.652092in}}{\pgfqpoint{0.850577in}{2.659992in}}{\pgfqpoint{0.844753in}{2.665816in}}%
\pgfpathcurveto{\pgfqpoint{0.838930in}{2.671640in}}{\pgfqpoint{0.831029in}{2.674912in}}{\pgfqpoint{0.822793in}{2.674912in}}%
\pgfpathcurveto{\pgfqpoint{0.814557in}{2.674912in}}{\pgfqpoint{0.806657in}{2.671640in}}{\pgfqpoint{0.800833in}{2.665816in}}%
\pgfpathcurveto{\pgfqpoint{0.795009in}{2.659992in}}{\pgfqpoint{0.791737in}{2.652092in}}{\pgfqpoint{0.791737in}{2.643856in}}%
\pgfpathcurveto{\pgfqpoint{0.791737in}{2.635620in}}{\pgfqpoint{0.795009in}{2.627719in}}{\pgfqpoint{0.800833in}{2.621896in}}%
\pgfpathcurveto{\pgfqpoint{0.806657in}{2.616072in}}{\pgfqpoint{0.814557in}{2.612799in}}{\pgfqpoint{0.822793in}{2.612799in}}%
\pgfpathclose%
\pgfusepath{stroke,fill}%
\end{pgfscope}%
\begin{pgfscope}%
\pgfpathrectangle{\pgfqpoint{0.100000in}{0.212622in}}{\pgfqpoint{3.696000in}{3.696000in}}%
\pgfusepath{clip}%
\pgfsetbuttcap%
\pgfsetroundjoin%
\definecolor{currentfill}{rgb}{0.121569,0.466667,0.705882}%
\pgfsetfillcolor{currentfill}%
\pgfsetfillopacity{0.816829}%
\pgfsetlinewidth{1.003750pt}%
\definecolor{currentstroke}{rgb}{0.121569,0.466667,0.705882}%
\pgfsetstrokecolor{currentstroke}%
\pgfsetstrokeopacity{0.816829}%
\pgfsetdash{}{0pt}%
\pgfpathmoveto{\pgfqpoint{0.842016in}{2.600977in}}%
\pgfpathcurveto{\pgfqpoint{0.850252in}{2.600977in}}{\pgfqpoint{0.858152in}{2.604249in}}{\pgfqpoint{0.863976in}{2.610073in}}%
\pgfpathcurveto{\pgfqpoint{0.869800in}{2.615897in}}{\pgfqpoint{0.873072in}{2.623797in}}{\pgfqpoint{0.873072in}{2.632033in}}%
\pgfpathcurveto{\pgfqpoint{0.873072in}{2.640270in}}{\pgfqpoint{0.869800in}{2.648170in}}{\pgfqpoint{0.863976in}{2.653994in}}%
\pgfpathcurveto{\pgfqpoint{0.858152in}{2.659818in}}{\pgfqpoint{0.850252in}{2.663090in}}{\pgfqpoint{0.842016in}{2.663090in}}%
\pgfpathcurveto{\pgfqpoint{0.833779in}{2.663090in}}{\pgfqpoint{0.825879in}{2.659818in}}{\pgfqpoint{0.820055in}{2.653994in}}%
\pgfpathcurveto{\pgfqpoint{0.814232in}{2.648170in}}{\pgfqpoint{0.810959in}{2.640270in}}{\pgfqpoint{0.810959in}{2.632033in}}%
\pgfpathcurveto{\pgfqpoint{0.810959in}{2.623797in}}{\pgfqpoint{0.814232in}{2.615897in}}{\pgfqpoint{0.820055in}{2.610073in}}%
\pgfpathcurveto{\pgfqpoint{0.825879in}{2.604249in}}{\pgfqpoint{0.833779in}{2.600977in}}{\pgfqpoint{0.842016in}{2.600977in}}%
\pgfpathclose%
\pgfusepath{stroke,fill}%
\end{pgfscope}%
\begin{pgfscope}%
\pgfpathrectangle{\pgfqpoint{0.100000in}{0.212622in}}{\pgfqpoint{3.696000in}{3.696000in}}%
\pgfusepath{clip}%
\pgfsetbuttcap%
\pgfsetroundjoin%
\definecolor{currentfill}{rgb}{0.121569,0.466667,0.705882}%
\pgfsetfillcolor{currentfill}%
\pgfsetfillopacity{0.818635}%
\pgfsetlinewidth{1.003750pt}%
\definecolor{currentstroke}{rgb}{0.121569,0.466667,0.705882}%
\pgfsetstrokecolor{currentstroke}%
\pgfsetstrokeopacity{0.818635}%
\pgfsetdash{}{0pt}%
\pgfpathmoveto{\pgfqpoint{2.817734in}{2.037750in}}%
\pgfpathcurveto{\pgfqpoint{2.825970in}{2.037750in}}{\pgfqpoint{2.833870in}{2.041022in}}{\pgfqpoint{2.839694in}{2.046846in}}%
\pgfpathcurveto{\pgfqpoint{2.845518in}{2.052670in}}{\pgfqpoint{2.848791in}{2.060570in}}{\pgfqpoint{2.848791in}{2.068806in}}%
\pgfpathcurveto{\pgfqpoint{2.848791in}{2.077042in}}{\pgfqpoint{2.845518in}{2.084942in}}{\pgfqpoint{2.839694in}{2.090766in}}%
\pgfpathcurveto{\pgfqpoint{2.833870in}{2.096590in}}{\pgfqpoint{2.825970in}{2.099863in}}{\pgfqpoint{2.817734in}{2.099863in}}%
\pgfpathcurveto{\pgfqpoint{2.809498in}{2.099863in}}{\pgfqpoint{2.801598in}{2.096590in}}{\pgfqpoint{2.795774in}{2.090766in}}%
\pgfpathcurveto{\pgfqpoint{2.789950in}{2.084942in}}{\pgfqpoint{2.786678in}{2.077042in}}{\pgfqpoint{2.786678in}{2.068806in}}%
\pgfpathcurveto{\pgfqpoint{2.786678in}{2.060570in}}{\pgfqpoint{2.789950in}{2.052670in}}{\pgfqpoint{2.795774in}{2.046846in}}%
\pgfpathcurveto{\pgfqpoint{2.801598in}{2.041022in}}{\pgfqpoint{2.809498in}{2.037750in}}{\pgfqpoint{2.817734in}{2.037750in}}%
\pgfpathclose%
\pgfusepath{stroke,fill}%
\end{pgfscope}%
\begin{pgfscope}%
\pgfpathrectangle{\pgfqpoint{0.100000in}{0.212622in}}{\pgfqpoint{3.696000in}{3.696000in}}%
\pgfusepath{clip}%
\pgfsetbuttcap%
\pgfsetroundjoin%
\definecolor{currentfill}{rgb}{0.121569,0.466667,0.705882}%
\pgfsetfillcolor{currentfill}%
\pgfsetfillopacity{0.819513}%
\pgfsetlinewidth{1.003750pt}%
\definecolor{currentstroke}{rgb}{0.121569,0.466667,0.705882}%
\pgfsetstrokecolor{currentstroke}%
\pgfsetstrokeopacity{0.819513}%
\pgfsetdash{}{0pt}%
\pgfpathmoveto{\pgfqpoint{0.858322in}{2.592626in}}%
\pgfpathcurveto{\pgfqpoint{0.866558in}{2.592626in}}{\pgfqpoint{0.874458in}{2.595899in}}{\pgfqpoint{0.880282in}{2.601723in}}%
\pgfpathcurveto{\pgfqpoint{0.886106in}{2.607547in}}{\pgfqpoint{0.889378in}{2.615447in}}{\pgfqpoint{0.889378in}{2.623683in}}%
\pgfpathcurveto{\pgfqpoint{0.889378in}{2.631919in}}{\pgfqpoint{0.886106in}{2.639819in}}{\pgfqpoint{0.880282in}{2.645643in}}%
\pgfpathcurveto{\pgfqpoint{0.874458in}{2.651467in}}{\pgfqpoint{0.866558in}{2.654739in}}{\pgfqpoint{0.858322in}{2.654739in}}%
\pgfpathcurveto{\pgfqpoint{0.850085in}{2.654739in}}{\pgfqpoint{0.842185in}{2.651467in}}{\pgfqpoint{0.836361in}{2.645643in}}%
\pgfpathcurveto{\pgfqpoint{0.830537in}{2.639819in}}{\pgfqpoint{0.827265in}{2.631919in}}{\pgfqpoint{0.827265in}{2.623683in}}%
\pgfpathcurveto{\pgfqpoint{0.827265in}{2.615447in}}{\pgfqpoint{0.830537in}{2.607547in}}{\pgfqpoint{0.836361in}{2.601723in}}%
\pgfpathcurveto{\pgfqpoint{0.842185in}{2.595899in}}{\pgfqpoint{0.850085in}{2.592626in}}{\pgfqpoint{0.858322in}{2.592626in}}%
\pgfpathclose%
\pgfusepath{stroke,fill}%
\end{pgfscope}%
\begin{pgfscope}%
\pgfpathrectangle{\pgfqpoint{0.100000in}{0.212622in}}{\pgfqpoint{3.696000in}{3.696000in}}%
\pgfusepath{clip}%
\pgfsetbuttcap%
\pgfsetroundjoin%
\definecolor{currentfill}{rgb}{0.121569,0.466667,0.705882}%
\pgfsetfillcolor{currentfill}%
\pgfsetfillopacity{0.821006}%
\pgfsetlinewidth{1.003750pt}%
\definecolor{currentstroke}{rgb}{0.121569,0.466667,0.705882}%
\pgfsetstrokecolor{currentstroke}%
\pgfsetstrokeopacity{0.821006}%
\pgfsetdash{}{0pt}%
\pgfpathmoveto{\pgfqpoint{0.873621in}{2.582728in}}%
\pgfpathcurveto{\pgfqpoint{0.881857in}{2.582728in}}{\pgfqpoint{0.889757in}{2.586000in}}{\pgfqpoint{0.895581in}{2.591824in}}%
\pgfpathcurveto{\pgfqpoint{0.901405in}{2.597648in}}{\pgfqpoint{0.904678in}{2.605548in}}{\pgfqpoint{0.904678in}{2.613784in}}%
\pgfpathcurveto{\pgfqpoint{0.904678in}{2.622021in}}{\pgfqpoint{0.901405in}{2.629921in}}{\pgfqpoint{0.895581in}{2.635745in}}%
\pgfpathcurveto{\pgfqpoint{0.889757in}{2.641569in}}{\pgfqpoint{0.881857in}{2.644841in}}{\pgfqpoint{0.873621in}{2.644841in}}%
\pgfpathcurveto{\pgfqpoint{0.865385in}{2.644841in}}{\pgfqpoint{0.857485in}{2.641569in}}{\pgfqpoint{0.851661in}{2.635745in}}%
\pgfpathcurveto{\pgfqpoint{0.845837in}{2.629921in}}{\pgfqpoint{0.842565in}{2.622021in}}{\pgfqpoint{0.842565in}{2.613784in}}%
\pgfpathcurveto{\pgfqpoint{0.842565in}{2.605548in}}{\pgfqpoint{0.845837in}{2.597648in}}{\pgfqpoint{0.851661in}{2.591824in}}%
\pgfpathcurveto{\pgfqpoint{0.857485in}{2.586000in}}{\pgfqpoint{0.865385in}{2.582728in}}{\pgfqpoint{0.873621in}{2.582728in}}%
\pgfpathclose%
\pgfusepath{stroke,fill}%
\end{pgfscope}%
\begin{pgfscope}%
\pgfpathrectangle{\pgfqpoint{0.100000in}{0.212622in}}{\pgfqpoint{3.696000in}{3.696000in}}%
\pgfusepath{clip}%
\pgfsetbuttcap%
\pgfsetroundjoin%
\definecolor{currentfill}{rgb}{0.121569,0.466667,0.705882}%
\pgfsetfillcolor{currentfill}%
\pgfsetfillopacity{0.821678}%
\pgfsetlinewidth{1.003750pt}%
\definecolor{currentstroke}{rgb}{0.121569,0.466667,0.705882}%
\pgfsetstrokecolor{currentstroke}%
\pgfsetstrokeopacity{0.821678}%
\pgfsetdash{}{0pt}%
\pgfpathmoveto{\pgfqpoint{2.812014in}{2.035645in}}%
\pgfpathcurveto{\pgfqpoint{2.820250in}{2.035645in}}{\pgfqpoint{2.828150in}{2.038917in}}{\pgfqpoint{2.833974in}{2.044741in}}%
\pgfpathcurveto{\pgfqpoint{2.839798in}{2.050565in}}{\pgfqpoint{2.843070in}{2.058465in}}{\pgfqpoint{2.843070in}{2.066701in}}%
\pgfpathcurveto{\pgfqpoint{2.843070in}{2.074937in}}{\pgfqpoint{2.839798in}{2.082838in}}{\pgfqpoint{2.833974in}{2.088661in}}%
\pgfpathcurveto{\pgfqpoint{2.828150in}{2.094485in}}{\pgfqpoint{2.820250in}{2.097758in}}{\pgfqpoint{2.812014in}{2.097758in}}%
\pgfpathcurveto{\pgfqpoint{2.803777in}{2.097758in}}{\pgfqpoint{2.795877in}{2.094485in}}{\pgfqpoint{2.790053in}{2.088661in}}%
\pgfpathcurveto{\pgfqpoint{2.784229in}{2.082838in}}{\pgfqpoint{2.780957in}{2.074937in}}{\pgfqpoint{2.780957in}{2.066701in}}%
\pgfpathcurveto{\pgfqpoint{2.780957in}{2.058465in}}{\pgfqpoint{2.784229in}{2.050565in}}{\pgfqpoint{2.790053in}{2.044741in}}%
\pgfpathcurveto{\pgfqpoint{2.795877in}{2.038917in}}{\pgfqpoint{2.803777in}{2.035645in}}{\pgfqpoint{2.812014in}{2.035645in}}%
\pgfpathclose%
\pgfusepath{stroke,fill}%
\end{pgfscope}%
\begin{pgfscope}%
\pgfpathrectangle{\pgfqpoint{0.100000in}{0.212622in}}{\pgfqpoint{3.696000in}{3.696000in}}%
\pgfusepath{clip}%
\pgfsetbuttcap%
\pgfsetroundjoin%
\definecolor{currentfill}{rgb}{0.121569,0.466667,0.705882}%
\pgfsetfillcolor{currentfill}%
\pgfsetfillopacity{0.822816}%
\pgfsetlinewidth{1.003750pt}%
\definecolor{currentstroke}{rgb}{0.121569,0.466667,0.705882}%
\pgfsetstrokecolor{currentstroke}%
\pgfsetstrokeopacity{0.822816}%
\pgfsetdash{}{0pt}%
\pgfpathmoveto{\pgfqpoint{0.884801in}{2.578097in}}%
\pgfpathcurveto{\pgfqpoint{0.893037in}{2.578097in}}{\pgfqpoint{0.900937in}{2.581369in}}{\pgfqpoint{0.906761in}{2.587193in}}%
\pgfpathcurveto{\pgfqpoint{0.912585in}{2.593017in}}{\pgfqpoint{0.915857in}{2.600917in}}{\pgfqpoint{0.915857in}{2.609154in}}%
\pgfpathcurveto{\pgfqpoint{0.915857in}{2.617390in}}{\pgfqpoint{0.912585in}{2.625290in}}{\pgfqpoint{0.906761in}{2.631114in}}%
\pgfpathcurveto{\pgfqpoint{0.900937in}{2.636938in}}{\pgfqpoint{0.893037in}{2.640210in}}{\pgfqpoint{0.884801in}{2.640210in}}%
\pgfpathcurveto{\pgfqpoint{0.876564in}{2.640210in}}{\pgfqpoint{0.868664in}{2.636938in}}{\pgfqpoint{0.862840in}{2.631114in}}%
\pgfpathcurveto{\pgfqpoint{0.857017in}{2.625290in}}{\pgfqpoint{0.853744in}{2.617390in}}{\pgfqpoint{0.853744in}{2.609154in}}%
\pgfpathcurveto{\pgfqpoint{0.853744in}{2.600917in}}{\pgfqpoint{0.857017in}{2.593017in}}{\pgfqpoint{0.862840in}{2.587193in}}%
\pgfpathcurveto{\pgfqpoint{0.868664in}{2.581369in}}{\pgfqpoint{0.876564in}{2.578097in}}{\pgfqpoint{0.884801in}{2.578097in}}%
\pgfpathclose%
\pgfusepath{stroke,fill}%
\end{pgfscope}%
\begin{pgfscope}%
\pgfpathrectangle{\pgfqpoint{0.100000in}{0.212622in}}{\pgfqpoint{3.696000in}{3.696000in}}%
\pgfusepath{clip}%
\pgfsetbuttcap%
\pgfsetroundjoin%
\definecolor{currentfill}{rgb}{0.121569,0.466667,0.705882}%
\pgfsetfillcolor{currentfill}%
\pgfsetfillopacity{0.823275}%
\pgfsetlinewidth{1.003750pt}%
\definecolor{currentstroke}{rgb}{0.121569,0.466667,0.705882}%
\pgfsetstrokecolor{currentstroke}%
\pgfsetstrokeopacity{0.823275}%
\pgfsetdash{}{0pt}%
\pgfpathmoveto{\pgfqpoint{2.808488in}{2.034445in}}%
\pgfpathcurveto{\pgfqpoint{2.816725in}{2.034445in}}{\pgfqpoint{2.824625in}{2.037718in}}{\pgfqpoint{2.830449in}{2.043542in}}%
\pgfpathcurveto{\pgfqpoint{2.836273in}{2.049366in}}{\pgfqpoint{2.839545in}{2.057266in}}{\pgfqpoint{2.839545in}{2.065502in}}%
\pgfpathcurveto{\pgfqpoint{2.839545in}{2.073738in}}{\pgfqpoint{2.836273in}{2.081638in}}{\pgfqpoint{2.830449in}{2.087462in}}%
\pgfpathcurveto{\pgfqpoint{2.824625in}{2.093286in}}{\pgfqpoint{2.816725in}{2.096558in}}{\pgfqpoint{2.808488in}{2.096558in}}%
\pgfpathcurveto{\pgfqpoint{2.800252in}{2.096558in}}{\pgfqpoint{2.792352in}{2.093286in}}{\pgfqpoint{2.786528in}{2.087462in}}%
\pgfpathcurveto{\pgfqpoint{2.780704in}{2.081638in}}{\pgfqpoint{2.777432in}{2.073738in}}{\pgfqpoint{2.777432in}{2.065502in}}%
\pgfpathcurveto{\pgfqpoint{2.777432in}{2.057266in}}{\pgfqpoint{2.780704in}{2.049366in}}{\pgfqpoint{2.786528in}{2.043542in}}%
\pgfpathcurveto{\pgfqpoint{2.792352in}{2.037718in}}{\pgfqpoint{2.800252in}{2.034445in}}{\pgfqpoint{2.808488in}{2.034445in}}%
\pgfpathclose%
\pgfusepath{stroke,fill}%
\end{pgfscope}%
\begin{pgfscope}%
\pgfpathrectangle{\pgfqpoint{0.100000in}{0.212622in}}{\pgfqpoint{3.696000in}{3.696000in}}%
\pgfusepath{clip}%
\pgfsetbuttcap%
\pgfsetroundjoin%
\definecolor{currentfill}{rgb}{0.121569,0.466667,0.705882}%
\pgfsetfillcolor{currentfill}%
\pgfsetfillopacity{0.823822}%
\pgfsetlinewidth{1.003750pt}%
\definecolor{currentstroke}{rgb}{0.121569,0.466667,0.705882}%
\pgfsetstrokecolor{currentstroke}%
\pgfsetstrokeopacity{0.823822}%
\pgfsetdash{}{0pt}%
\pgfpathmoveto{\pgfqpoint{0.893712in}{2.572189in}}%
\pgfpathcurveto{\pgfqpoint{0.901948in}{2.572189in}}{\pgfqpoint{0.909848in}{2.575462in}}{\pgfqpoint{0.915672in}{2.581285in}}%
\pgfpathcurveto{\pgfqpoint{0.921496in}{2.587109in}}{\pgfqpoint{0.924768in}{2.595009in}}{\pgfqpoint{0.924768in}{2.603246in}}%
\pgfpathcurveto{\pgfqpoint{0.924768in}{2.611482in}}{\pgfqpoint{0.921496in}{2.619382in}}{\pgfqpoint{0.915672in}{2.625206in}}%
\pgfpathcurveto{\pgfqpoint{0.909848in}{2.631030in}}{\pgfqpoint{0.901948in}{2.634302in}}{\pgfqpoint{0.893712in}{2.634302in}}%
\pgfpathcurveto{\pgfqpoint{0.885476in}{2.634302in}}{\pgfqpoint{0.877576in}{2.631030in}}{\pgfqpoint{0.871752in}{2.625206in}}%
\pgfpathcurveto{\pgfqpoint{0.865928in}{2.619382in}}{\pgfqpoint{0.862655in}{2.611482in}}{\pgfqpoint{0.862655in}{2.603246in}}%
\pgfpathcurveto{\pgfqpoint{0.862655in}{2.595009in}}{\pgfqpoint{0.865928in}{2.587109in}}{\pgfqpoint{0.871752in}{2.581285in}}%
\pgfpathcurveto{\pgfqpoint{0.877576in}{2.575462in}}{\pgfqpoint{0.885476in}{2.572189in}}{\pgfqpoint{0.893712in}{2.572189in}}%
\pgfpathclose%
\pgfusepath{stroke,fill}%
\end{pgfscope}%
\begin{pgfscope}%
\pgfpathrectangle{\pgfqpoint{0.100000in}{0.212622in}}{\pgfqpoint{3.696000in}{3.696000in}}%
\pgfusepath{clip}%
\pgfsetbuttcap%
\pgfsetroundjoin%
\definecolor{currentfill}{rgb}{0.121569,0.466667,0.705882}%
\pgfsetfillcolor{currentfill}%
\pgfsetfillopacity{0.824105}%
\pgfsetlinewidth{1.003750pt}%
\definecolor{currentstroke}{rgb}{0.121569,0.466667,0.705882}%
\pgfsetstrokecolor{currentstroke}%
\pgfsetstrokeopacity{0.824105}%
\pgfsetdash{}{0pt}%
\pgfpathmoveto{\pgfqpoint{2.806427in}{2.033634in}}%
\pgfpathcurveto{\pgfqpoint{2.814663in}{2.033634in}}{\pgfqpoint{2.822563in}{2.036907in}}{\pgfqpoint{2.828387in}{2.042731in}}%
\pgfpathcurveto{\pgfqpoint{2.834211in}{2.048555in}}{\pgfqpoint{2.837483in}{2.056455in}}{\pgfqpoint{2.837483in}{2.064691in}}%
\pgfpathcurveto{\pgfqpoint{2.837483in}{2.072927in}}{\pgfqpoint{2.834211in}{2.080827in}}{\pgfqpoint{2.828387in}{2.086651in}}%
\pgfpathcurveto{\pgfqpoint{2.822563in}{2.092475in}}{\pgfqpoint{2.814663in}{2.095747in}}{\pgfqpoint{2.806427in}{2.095747in}}%
\pgfpathcurveto{\pgfqpoint{2.798190in}{2.095747in}}{\pgfqpoint{2.790290in}{2.092475in}}{\pgfqpoint{2.784466in}{2.086651in}}%
\pgfpathcurveto{\pgfqpoint{2.778642in}{2.080827in}}{\pgfqpoint{2.775370in}{2.072927in}}{\pgfqpoint{2.775370in}{2.064691in}}%
\pgfpathcurveto{\pgfqpoint{2.775370in}{2.056455in}}{\pgfqpoint{2.778642in}{2.048555in}}{\pgfqpoint{2.784466in}{2.042731in}}%
\pgfpathcurveto{\pgfqpoint{2.790290in}{2.036907in}}{\pgfqpoint{2.798190in}{2.033634in}}{\pgfqpoint{2.806427in}{2.033634in}}%
\pgfpathclose%
\pgfusepath{stroke,fill}%
\end{pgfscope}%
\begin{pgfscope}%
\pgfpathrectangle{\pgfqpoint{0.100000in}{0.212622in}}{\pgfqpoint{3.696000in}{3.696000in}}%
\pgfusepath{clip}%
\pgfsetbuttcap%
\pgfsetroundjoin%
\definecolor{currentfill}{rgb}{0.121569,0.466667,0.705882}%
\pgfsetfillcolor{currentfill}%
\pgfsetfillopacity{0.824558}%
\pgfsetlinewidth{1.003750pt}%
\definecolor{currentstroke}{rgb}{0.121569,0.466667,0.705882}%
\pgfsetstrokecolor{currentstroke}%
\pgfsetstrokeopacity{0.824558}%
\pgfsetdash{}{0pt}%
\pgfpathmoveto{\pgfqpoint{0.898638in}{2.570002in}}%
\pgfpathcurveto{\pgfqpoint{0.906874in}{2.570002in}}{\pgfqpoint{0.914774in}{2.573275in}}{\pgfqpoint{0.920598in}{2.579098in}}%
\pgfpathcurveto{\pgfqpoint{0.926422in}{2.584922in}}{\pgfqpoint{0.929694in}{2.592822in}}{\pgfqpoint{0.929694in}{2.601059in}}%
\pgfpathcurveto{\pgfqpoint{0.929694in}{2.609295in}}{\pgfqpoint{0.926422in}{2.617195in}}{\pgfqpoint{0.920598in}{2.623019in}}%
\pgfpathcurveto{\pgfqpoint{0.914774in}{2.628843in}}{\pgfqpoint{0.906874in}{2.632115in}}{\pgfqpoint{0.898638in}{2.632115in}}%
\pgfpathcurveto{\pgfqpoint{0.890402in}{2.632115in}}{\pgfqpoint{0.882502in}{2.628843in}}{\pgfqpoint{0.876678in}{2.623019in}}%
\pgfpathcurveto{\pgfqpoint{0.870854in}{2.617195in}}{\pgfqpoint{0.867581in}{2.609295in}}{\pgfqpoint{0.867581in}{2.601059in}}%
\pgfpathcurveto{\pgfqpoint{0.867581in}{2.592822in}}{\pgfqpoint{0.870854in}{2.584922in}}{\pgfqpoint{0.876678in}{2.579098in}}%
\pgfpathcurveto{\pgfqpoint{0.882502in}{2.573275in}}{\pgfqpoint{0.890402in}{2.570002in}}{\pgfqpoint{0.898638in}{2.570002in}}%
\pgfpathclose%
\pgfusepath{stroke,fill}%
\end{pgfscope}%
\begin{pgfscope}%
\pgfpathrectangle{\pgfqpoint{0.100000in}{0.212622in}}{\pgfqpoint{3.696000in}{3.696000in}}%
\pgfusepath{clip}%
\pgfsetbuttcap%
\pgfsetroundjoin%
\definecolor{currentfill}{rgb}{0.121569,0.466667,0.705882}%
\pgfsetfillcolor{currentfill}%
\pgfsetfillopacity{0.824581}%
\pgfsetlinewidth{1.003750pt}%
\definecolor{currentstroke}{rgb}{0.121569,0.466667,0.705882}%
\pgfsetstrokecolor{currentstroke}%
\pgfsetstrokeopacity{0.824581}%
\pgfsetdash{}{0pt}%
\pgfpathmoveto{\pgfqpoint{2.805715in}{2.032849in}}%
\pgfpathcurveto{\pgfqpoint{2.813952in}{2.032849in}}{\pgfqpoint{2.821852in}{2.036121in}}{\pgfqpoint{2.827676in}{2.041945in}}%
\pgfpathcurveto{\pgfqpoint{2.833499in}{2.047769in}}{\pgfqpoint{2.836772in}{2.055669in}}{\pgfqpoint{2.836772in}{2.063906in}}%
\pgfpathcurveto{\pgfqpoint{2.836772in}{2.072142in}}{\pgfqpoint{2.833499in}{2.080042in}}{\pgfqpoint{2.827676in}{2.085866in}}%
\pgfpathcurveto{\pgfqpoint{2.821852in}{2.091690in}}{\pgfqpoint{2.813952in}{2.094962in}}{\pgfqpoint{2.805715in}{2.094962in}}%
\pgfpathcurveto{\pgfqpoint{2.797479in}{2.094962in}}{\pgfqpoint{2.789579in}{2.091690in}}{\pgfqpoint{2.783755in}{2.085866in}}%
\pgfpathcurveto{\pgfqpoint{2.777931in}{2.080042in}}{\pgfqpoint{2.774659in}{2.072142in}}{\pgfqpoint{2.774659in}{2.063906in}}%
\pgfpathcurveto{\pgfqpoint{2.774659in}{2.055669in}}{\pgfqpoint{2.777931in}{2.047769in}}{\pgfqpoint{2.783755in}{2.041945in}}%
\pgfpathcurveto{\pgfqpoint{2.789579in}{2.036121in}}{\pgfqpoint{2.797479in}{2.032849in}}{\pgfqpoint{2.805715in}{2.032849in}}%
\pgfpathclose%
\pgfusepath{stroke,fill}%
\end{pgfscope}%
\begin{pgfscope}%
\pgfpathrectangle{\pgfqpoint{0.100000in}{0.212622in}}{\pgfqpoint{3.696000in}{3.696000in}}%
\pgfusepath{clip}%
\pgfsetbuttcap%
\pgfsetroundjoin%
\definecolor{currentfill}{rgb}{0.121569,0.466667,0.705882}%
\pgfsetfillcolor{currentfill}%
\pgfsetfillopacity{0.825529}%
\pgfsetlinewidth{1.003750pt}%
\definecolor{currentstroke}{rgb}{0.121569,0.466667,0.705882}%
\pgfsetstrokecolor{currentstroke}%
\pgfsetstrokeopacity{0.825529}%
\pgfsetdash{}{0pt}%
\pgfpathmoveto{\pgfqpoint{0.907671in}{2.563580in}}%
\pgfpathcurveto{\pgfqpoint{0.915907in}{2.563580in}}{\pgfqpoint{0.923807in}{2.566852in}}{\pgfqpoint{0.929631in}{2.572676in}}%
\pgfpathcurveto{\pgfqpoint{0.935455in}{2.578500in}}{\pgfqpoint{0.938727in}{2.586400in}}{\pgfqpoint{0.938727in}{2.594636in}}%
\pgfpathcurveto{\pgfqpoint{0.938727in}{2.602872in}}{\pgfqpoint{0.935455in}{2.610772in}}{\pgfqpoint{0.929631in}{2.616596in}}%
\pgfpathcurveto{\pgfqpoint{0.923807in}{2.622420in}}{\pgfqpoint{0.915907in}{2.625693in}}{\pgfqpoint{0.907671in}{2.625693in}}%
\pgfpathcurveto{\pgfqpoint{0.899434in}{2.625693in}}{\pgfqpoint{0.891534in}{2.622420in}}{\pgfqpoint{0.885710in}{2.616596in}}%
\pgfpathcurveto{\pgfqpoint{0.879886in}{2.610772in}}{\pgfqpoint{0.876614in}{2.602872in}}{\pgfqpoint{0.876614in}{2.594636in}}%
\pgfpathcurveto{\pgfqpoint{0.876614in}{2.586400in}}{\pgfqpoint{0.879886in}{2.578500in}}{\pgfqpoint{0.885710in}{2.572676in}}%
\pgfpathcurveto{\pgfqpoint{0.891534in}{2.566852in}}{\pgfqpoint{0.899434in}{2.563580in}}{\pgfqpoint{0.907671in}{2.563580in}}%
\pgfpathclose%
\pgfusepath{stroke,fill}%
\end{pgfscope}%
\begin{pgfscope}%
\pgfpathrectangle{\pgfqpoint{0.100000in}{0.212622in}}{\pgfqpoint{3.696000in}{3.696000in}}%
\pgfusepath{clip}%
\pgfsetbuttcap%
\pgfsetroundjoin%
\definecolor{currentfill}{rgb}{0.121569,0.466667,0.705882}%
\pgfsetfillcolor{currentfill}%
\pgfsetfillopacity{0.826338}%
\pgfsetlinewidth{1.003750pt}%
\definecolor{currentstroke}{rgb}{0.121569,0.466667,0.705882}%
\pgfsetstrokecolor{currentstroke}%
\pgfsetstrokeopacity{0.826338}%
\pgfsetdash{}{0pt}%
\pgfpathmoveto{\pgfqpoint{2.800412in}{2.030993in}}%
\pgfpathcurveto{\pgfqpoint{2.808648in}{2.030993in}}{\pgfqpoint{2.816548in}{2.034265in}}{\pgfqpoint{2.822372in}{2.040089in}}%
\pgfpathcurveto{\pgfqpoint{2.828196in}{2.045913in}}{\pgfqpoint{2.831468in}{2.053813in}}{\pgfqpoint{2.831468in}{2.062049in}}%
\pgfpathcurveto{\pgfqpoint{2.831468in}{2.070286in}}{\pgfqpoint{2.828196in}{2.078186in}}{\pgfqpoint{2.822372in}{2.084010in}}%
\pgfpathcurveto{\pgfqpoint{2.816548in}{2.089834in}}{\pgfqpoint{2.808648in}{2.093106in}}{\pgfqpoint{2.800412in}{2.093106in}}%
\pgfpathcurveto{\pgfqpoint{2.792175in}{2.093106in}}{\pgfqpoint{2.784275in}{2.089834in}}{\pgfqpoint{2.778451in}{2.084010in}}%
\pgfpathcurveto{\pgfqpoint{2.772628in}{2.078186in}}{\pgfqpoint{2.769355in}{2.070286in}}{\pgfqpoint{2.769355in}{2.062049in}}%
\pgfpathcurveto{\pgfqpoint{2.769355in}{2.053813in}}{\pgfqpoint{2.772628in}{2.045913in}}{\pgfqpoint{2.778451in}{2.040089in}}%
\pgfpathcurveto{\pgfqpoint{2.784275in}{2.034265in}}{\pgfqpoint{2.792175in}{2.030993in}}{\pgfqpoint{2.800412in}{2.030993in}}%
\pgfpathclose%
\pgfusepath{stroke,fill}%
\end{pgfscope}%
\begin{pgfscope}%
\pgfpathrectangle{\pgfqpoint{0.100000in}{0.212622in}}{\pgfqpoint{3.696000in}{3.696000in}}%
\pgfusepath{clip}%
\pgfsetbuttcap%
\pgfsetroundjoin%
\definecolor{currentfill}{rgb}{0.121569,0.466667,0.705882}%
\pgfsetfillcolor{currentfill}%
\pgfsetfillopacity{0.827979}%
\pgfsetlinewidth{1.003750pt}%
\definecolor{currentstroke}{rgb}{0.121569,0.466667,0.705882}%
\pgfsetstrokecolor{currentstroke}%
\pgfsetstrokeopacity{0.827979}%
\pgfsetdash{}{0pt}%
\pgfpathmoveto{\pgfqpoint{0.923220in}{2.554281in}}%
\pgfpathcurveto{\pgfqpoint{0.931456in}{2.554281in}}{\pgfqpoint{0.939356in}{2.557553in}}{\pgfqpoint{0.945180in}{2.563377in}}%
\pgfpathcurveto{\pgfqpoint{0.951004in}{2.569201in}}{\pgfqpoint{0.954276in}{2.577101in}}{\pgfqpoint{0.954276in}{2.585337in}}%
\pgfpathcurveto{\pgfqpoint{0.954276in}{2.593574in}}{\pgfqpoint{0.951004in}{2.601474in}}{\pgfqpoint{0.945180in}{2.607297in}}%
\pgfpathcurveto{\pgfqpoint{0.939356in}{2.613121in}}{\pgfqpoint{0.931456in}{2.616394in}}{\pgfqpoint{0.923220in}{2.616394in}}%
\pgfpathcurveto{\pgfqpoint{0.914983in}{2.616394in}}{\pgfqpoint{0.907083in}{2.613121in}}{\pgfqpoint{0.901259in}{2.607297in}}%
\pgfpathcurveto{\pgfqpoint{0.895435in}{2.601474in}}{\pgfqpoint{0.892163in}{2.593574in}}{\pgfqpoint{0.892163in}{2.585337in}}%
\pgfpathcurveto{\pgfqpoint{0.892163in}{2.577101in}}{\pgfqpoint{0.895435in}{2.569201in}}{\pgfqpoint{0.901259in}{2.563377in}}%
\pgfpathcurveto{\pgfqpoint{0.907083in}{2.557553in}}{\pgfqpoint{0.914983in}{2.554281in}}{\pgfqpoint{0.923220in}{2.554281in}}%
\pgfpathclose%
\pgfusepath{stroke,fill}%
\end{pgfscope}%
\begin{pgfscope}%
\pgfpathrectangle{\pgfqpoint{0.100000in}{0.212622in}}{\pgfqpoint{3.696000in}{3.696000in}}%
\pgfusepath{clip}%
\pgfsetbuttcap%
\pgfsetroundjoin%
\definecolor{currentfill}{rgb}{0.121569,0.466667,0.705882}%
\pgfsetfillcolor{currentfill}%
\pgfsetfillopacity{0.829296}%
\pgfsetlinewidth{1.003750pt}%
\definecolor{currentstroke}{rgb}{0.121569,0.466667,0.705882}%
\pgfsetstrokecolor{currentstroke}%
\pgfsetstrokeopacity{0.829296}%
\pgfsetdash{}{0pt}%
\pgfpathmoveto{\pgfqpoint{2.794047in}{2.024354in}}%
\pgfpathcurveto{\pgfqpoint{2.802283in}{2.024354in}}{\pgfqpoint{2.810183in}{2.027626in}}{\pgfqpoint{2.816007in}{2.033450in}}%
\pgfpathcurveto{\pgfqpoint{2.821831in}{2.039274in}}{\pgfqpoint{2.825103in}{2.047174in}}{\pgfqpoint{2.825103in}{2.055411in}}%
\pgfpathcurveto{\pgfqpoint{2.825103in}{2.063647in}}{\pgfqpoint{2.821831in}{2.071547in}}{\pgfqpoint{2.816007in}{2.077371in}}%
\pgfpathcurveto{\pgfqpoint{2.810183in}{2.083195in}}{\pgfqpoint{2.802283in}{2.086467in}}{\pgfqpoint{2.794047in}{2.086467in}}%
\pgfpathcurveto{\pgfqpoint{2.785810in}{2.086467in}}{\pgfqpoint{2.777910in}{2.083195in}}{\pgfqpoint{2.772086in}{2.077371in}}%
\pgfpathcurveto{\pgfqpoint{2.766263in}{2.071547in}}{\pgfqpoint{2.762990in}{2.063647in}}{\pgfqpoint{2.762990in}{2.055411in}}%
\pgfpathcurveto{\pgfqpoint{2.762990in}{2.047174in}}{\pgfqpoint{2.766263in}{2.039274in}}{\pgfqpoint{2.772086in}{2.033450in}}%
\pgfpathcurveto{\pgfqpoint{2.777910in}{2.027626in}}{\pgfqpoint{2.785810in}{2.024354in}}{\pgfqpoint{2.794047in}{2.024354in}}%
\pgfpathclose%
\pgfusepath{stroke,fill}%
\end{pgfscope}%
\begin{pgfscope}%
\pgfpathrectangle{\pgfqpoint{0.100000in}{0.212622in}}{\pgfqpoint{3.696000in}{3.696000in}}%
\pgfusepath{clip}%
\pgfsetbuttcap%
\pgfsetroundjoin%
\definecolor{currentfill}{rgb}{0.121569,0.466667,0.705882}%
\pgfsetfillcolor{currentfill}%
\pgfsetfillopacity{0.830004}%
\pgfsetlinewidth{1.003750pt}%
\definecolor{currentstroke}{rgb}{0.121569,0.466667,0.705882}%
\pgfsetstrokecolor{currentstroke}%
\pgfsetstrokeopacity{0.830004}%
\pgfsetdash{}{0pt}%
\pgfpathmoveto{\pgfqpoint{0.936273in}{2.544233in}}%
\pgfpathcurveto{\pgfqpoint{0.944510in}{2.544233in}}{\pgfqpoint{0.952410in}{2.547505in}}{\pgfqpoint{0.958234in}{2.553329in}}%
\pgfpathcurveto{\pgfqpoint{0.964058in}{2.559153in}}{\pgfqpoint{0.967330in}{2.567053in}}{\pgfqpoint{0.967330in}{2.575289in}}%
\pgfpathcurveto{\pgfqpoint{0.967330in}{2.583526in}}{\pgfqpoint{0.964058in}{2.591426in}}{\pgfqpoint{0.958234in}{2.597250in}}%
\pgfpathcurveto{\pgfqpoint{0.952410in}{2.603073in}}{\pgfqpoint{0.944510in}{2.606346in}}{\pgfqpoint{0.936273in}{2.606346in}}%
\pgfpathcurveto{\pgfqpoint{0.928037in}{2.606346in}}{\pgfqpoint{0.920137in}{2.603073in}}{\pgfqpoint{0.914313in}{2.597250in}}%
\pgfpathcurveto{\pgfqpoint{0.908489in}{2.591426in}}{\pgfqpoint{0.905217in}{2.583526in}}{\pgfqpoint{0.905217in}{2.575289in}}%
\pgfpathcurveto{\pgfqpoint{0.905217in}{2.567053in}}{\pgfqpoint{0.908489in}{2.559153in}}{\pgfqpoint{0.914313in}{2.553329in}}%
\pgfpathcurveto{\pgfqpoint{0.920137in}{2.547505in}}{\pgfqpoint{0.928037in}{2.544233in}}{\pgfqpoint{0.936273in}{2.544233in}}%
\pgfpathclose%
\pgfusepath{stroke,fill}%
\end{pgfscope}%
\begin{pgfscope}%
\pgfpathrectangle{\pgfqpoint{0.100000in}{0.212622in}}{\pgfqpoint{3.696000in}{3.696000in}}%
\pgfusepath{clip}%
\pgfsetbuttcap%
\pgfsetroundjoin%
\definecolor{currentfill}{rgb}{0.121569,0.466667,0.705882}%
\pgfsetfillcolor{currentfill}%
\pgfsetfillopacity{0.831483}%
\pgfsetlinewidth{1.003750pt}%
\definecolor{currentstroke}{rgb}{0.121569,0.466667,0.705882}%
\pgfsetstrokecolor{currentstroke}%
\pgfsetstrokeopacity{0.831483}%
\pgfsetdash{}{0pt}%
\pgfpathmoveto{\pgfqpoint{0.946902in}{2.537608in}}%
\pgfpathcurveto{\pgfqpoint{0.955139in}{2.537608in}}{\pgfqpoint{0.963039in}{2.540880in}}{\pgfqpoint{0.968863in}{2.546704in}}%
\pgfpathcurveto{\pgfqpoint{0.974687in}{2.552528in}}{\pgfqpoint{0.977959in}{2.560428in}}{\pgfqpoint{0.977959in}{2.568664in}}%
\pgfpathcurveto{\pgfqpoint{0.977959in}{2.576900in}}{\pgfqpoint{0.974687in}{2.584801in}}{\pgfqpoint{0.968863in}{2.590624in}}%
\pgfpathcurveto{\pgfqpoint{0.963039in}{2.596448in}}{\pgfqpoint{0.955139in}{2.599721in}}{\pgfqpoint{0.946902in}{2.599721in}}%
\pgfpathcurveto{\pgfqpoint{0.938666in}{2.599721in}}{\pgfqpoint{0.930766in}{2.596448in}}{\pgfqpoint{0.924942in}{2.590624in}}%
\pgfpathcurveto{\pgfqpoint{0.919118in}{2.584801in}}{\pgfqpoint{0.915846in}{2.576900in}}{\pgfqpoint{0.915846in}{2.568664in}}%
\pgfpathcurveto{\pgfqpoint{0.915846in}{2.560428in}}{\pgfqpoint{0.919118in}{2.552528in}}{\pgfqpoint{0.924942in}{2.546704in}}%
\pgfpathcurveto{\pgfqpoint{0.930766in}{2.540880in}}{\pgfqpoint{0.938666in}{2.537608in}}{\pgfqpoint{0.946902in}{2.537608in}}%
\pgfpathclose%
\pgfusepath{stroke,fill}%
\end{pgfscope}%
\begin{pgfscope}%
\pgfpathrectangle{\pgfqpoint{0.100000in}{0.212622in}}{\pgfqpoint{3.696000in}{3.696000in}}%
\pgfusepath{clip}%
\pgfsetbuttcap%
\pgfsetroundjoin%
\definecolor{currentfill}{rgb}{0.121569,0.466667,0.705882}%
\pgfsetfillcolor{currentfill}%
\pgfsetfillopacity{0.832624}%
\pgfsetlinewidth{1.003750pt}%
\definecolor{currentstroke}{rgb}{0.121569,0.466667,0.705882}%
\pgfsetstrokecolor{currentstroke}%
\pgfsetstrokeopacity{0.832624}%
\pgfsetdash{}{0pt}%
\pgfpathmoveto{\pgfqpoint{0.954018in}{2.534437in}}%
\pgfpathcurveto{\pgfqpoint{0.962254in}{2.534437in}}{\pgfqpoint{0.970154in}{2.537709in}}{\pgfqpoint{0.975978in}{2.543533in}}%
\pgfpathcurveto{\pgfqpoint{0.981802in}{2.549357in}}{\pgfqpoint{0.985074in}{2.557257in}}{\pgfqpoint{0.985074in}{2.565493in}}%
\pgfpathcurveto{\pgfqpoint{0.985074in}{2.573730in}}{\pgfqpoint{0.981802in}{2.581630in}}{\pgfqpoint{0.975978in}{2.587454in}}%
\pgfpathcurveto{\pgfqpoint{0.970154in}{2.593278in}}{\pgfqpoint{0.962254in}{2.596550in}}{\pgfqpoint{0.954018in}{2.596550in}}%
\pgfpathcurveto{\pgfqpoint{0.945782in}{2.596550in}}{\pgfqpoint{0.937882in}{2.593278in}}{\pgfqpoint{0.932058in}{2.587454in}}%
\pgfpathcurveto{\pgfqpoint{0.926234in}{2.581630in}}{\pgfqpoint{0.922961in}{2.573730in}}{\pgfqpoint{0.922961in}{2.565493in}}%
\pgfpathcurveto{\pgfqpoint{0.922961in}{2.557257in}}{\pgfqpoint{0.926234in}{2.549357in}}{\pgfqpoint{0.932058in}{2.543533in}}%
\pgfpathcurveto{\pgfqpoint{0.937882in}{2.537709in}}{\pgfqpoint{0.945782in}{2.534437in}}{\pgfqpoint{0.954018in}{2.534437in}}%
\pgfpathclose%
\pgfusepath{stroke,fill}%
\end{pgfscope}%
\begin{pgfscope}%
\pgfpathrectangle{\pgfqpoint{0.100000in}{0.212622in}}{\pgfqpoint{3.696000in}{3.696000in}}%
\pgfusepath{clip}%
\pgfsetbuttcap%
\pgfsetroundjoin%
\definecolor{currentfill}{rgb}{0.121569,0.466667,0.705882}%
\pgfsetfillcolor{currentfill}%
\pgfsetfillopacity{0.833156}%
\pgfsetlinewidth{1.003750pt}%
\definecolor{currentstroke}{rgb}{0.121569,0.466667,0.705882}%
\pgfsetstrokecolor{currentstroke}%
\pgfsetstrokeopacity{0.833156}%
\pgfsetdash{}{0pt}%
\pgfpathmoveto{\pgfqpoint{0.957570in}{2.531453in}}%
\pgfpathcurveto{\pgfqpoint{0.965806in}{2.531453in}}{\pgfqpoint{0.973706in}{2.534726in}}{\pgfqpoint{0.979530in}{2.540549in}}%
\pgfpathcurveto{\pgfqpoint{0.985354in}{2.546373in}}{\pgfqpoint{0.988626in}{2.554273in}}{\pgfqpoint{0.988626in}{2.562510in}}%
\pgfpathcurveto{\pgfqpoint{0.988626in}{2.570746in}}{\pgfqpoint{0.985354in}{2.578646in}}{\pgfqpoint{0.979530in}{2.584470in}}%
\pgfpathcurveto{\pgfqpoint{0.973706in}{2.590294in}}{\pgfqpoint{0.965806in}{2.593566in}}{\pgfqpoint{0.957570in}{2.593566in}}%
\pgfpathcurveto{\pgfqpoint{0.949333in}{2.593566in}}{\pgfqpoint{0.941433in}{2.590294in}}{\pgfqpoint{0.935609in}{2.584470in}}%
\pgfpathcurveto{\pgfqpoint{0.929786in}{2.578646in}}{\pgfqpoint{0.926513in}{2.570746in}}{\pgfqpoint{0.926513in}{2.562510in}}%
\pgfpathcurveto{\pgfqpoint{0.926513in}{2.554273in}}{\pgfqpoint{0.929786in}{2.546373in}}{\pgfqpoint{0.935609in}{2.540549in}}%
\pgfpathcurveto{\pgfqpoint{0.941433in}{2.534726in}}{\pgfqpoint{0.949333in}{2.531453in}}{\pgfqpoint{0.957570in}{2.531453in}}%
\pgfpathclose%
\pgfusepath{stroke,fill}%
\end{pgfscope}%
\begin{pgfscope}%
\pgfpathrectangle{\pgfqpoint{0.100000in}{0.212622in}}{\pgfqpoint{3.696000in}{3.696000in}}%
\pgfusepath{clip}%
\pgfsetbuttcap%
\pgfsetroundjoin%
\definecolor{currentfill}{rgb}{0.121569,0.466667,0.705882}%
\pgfsetfillcolor{currentfill}%
\pgfsetfillopacity{0.833443}%
\pgfsetlinewidth{1.003750pt}%
\definecolor{currentstroke}{rgb}{0.121569,0.466667,0.705882}%
\pgfsetstrokecolor{currentstroke}%
\pgfsetstrokeopacity{0.833443}%
\pgfsetdash{}{0pt}%
\pgfpathmoveto{\pgfqpoint{2.782218in}{2.020667in}}%
\pgfpathcurveto{\pgfqpoint{2.790454in}{2.020667in}}{\pgfqpoint{2.798354in}{2.023940in}}{\pgfqpoint{2.804178in}{2.029763in}}%
\pgfpathcurveto{\pgfqpoint{2.810002in}{2.035587in}}{\pgfqpoint{2.813274in}{2.043487in}}{\pgfqpoint{2.813274in}{2.051724in}}%
\pgfpathcurveto{\pgfqpoint{2.813274in}{2.059960in}}{\pgfqpoint{2.810002in}{2.067860in}}{\pgfqpoint{2.804178in}{2.073684in}}%
\pgfpathcurveto{\pgfqpoint{2.798354in}{2.079508in}}{\pgfqpoint{2.790454in}{2.082780in}}{\pgfqpoint{2.782218in}{2.082780in}}%
\pgfpathcurveto{\pgfqpoint{2.773982in}{2.082780in}}{\pgfqpoint{2.766082in}{2.079508in}}{\pgfqpoint{2.760258in}{2.073684in}}%
\pgfpathcurveto{\pgfqpoint{2.754434in}{2.067860in}}{\pgfqpoint{2.751161in}{2.059960in}}{\pgfqpoint{2.751161in}{2.051724in}}%
\pgfpathcurveto{\pgfqpoint{2.751161in}{2.043487in}}{\pgfqpoint{2.754434in}{2.035587in}}{\pgfqpoint{2.760258in}{2.029763in}}%
\pgfpathcurveto{\pgfqpoint{2.766082in}{2.023940in}}{\pgfqpoint{2.773982in}{2.020667in}}{\pgfqpoint{2.782218in}{2.020667in}}%
\pgfpathclose%
\pgfusepath{stroke,fill}%
\end{pgfscope}%
\begin{pgfscope}%
\pgfpathrectangle{\pgfqpoint{0.100000in}{0.212622in}}{\pgfqpoint{3.696000in}{3.696000in}}%
\pgfusepath{clip}%
\pgfsetbuttcap%
\pgfsetroundjoin%
\definecolor{currentfill}{rgb}{0.121569,0.466667,0.705882}%
\pgfsetfillcolor{currentfill}%
\pgfsetfillopacity{0.833521}%
\pgfsetlinewidth{1.003750pt}%
\definecolor{currentstroke}{rgb}{0.121569,0.466667,0.705882}%
\pgfsetstrokecolor{currentstroke}%
\pgfsetstrokeopacity{0.833521}%
\pgfsetdash{}{0pt}%
\pgfpathmoveto{\pgfqpoint{0.959419in}{2.530137in}}%
\pgfpathcurveto{\pgfqpoint{0.967655in}{2.530137in}}{\pgfqpoint{0.975555in}{2.533409in}}{\pgfqpoint{0.981379in}{2.539233in}}%
\pgfpathcurveto{\pgfqpoint{0.987203in}{2.545057in}}{\pgfqpoint{0.990475in}{2.552957in}}{\pgfqpoint{0.990475in}{2.561194in}}%
\pgfpathcurveto{\pgfqpoint{0.990475in}{2.569430in}}{\pgfqpoint{0.987203in}{2.577330in}}{\pgfqpoint{0.981379in}{2.583154in}}%
\pgfpathcurveto{\pgfqpoint{0.975555in}{2.588978in}}{\pgfqpoint{0.967655in}{2.592250in}}{\pgfqpoint{0.959419in}{2.592250in}}%
\pgfpathcurveto{\pgfqpoint{0.951182in}{2.592250in}}{\pgfqpoint{0.943282in}{2.588978in}}{\pgfqpoint{0.937458in}{2.583154in}}%
\pgfpathcurveto{\pgfqpoint{0.931634in}{2.577330in}}{\pgfqpoint{0.928362in}{2.569430in}}{\pgfqpoint{0.928362in}{2.561194in}}%
\pgfpathcurveto{\pgfqpoint{0.928362in}{2.552957in}}{\pgfqpoint{0.931634in}{2.545057in}}{\pgfqpoint{0.937458in}{2.539233in}}%
\pgfpathcurveto{\pgfqpoint{0.943282in}{2.533409in}}{\pgfqpoint{0.951182in}{2.530137in}}{\pgfqpoint{0.959419in}{2.530137in}}%
\pgfpathclose%
\pgfusepath{stroke,fill}%
\end{pgfscope}%
\begin{pgfscope}%
\pgfpathrectangle{\pgfqpoint{0.100000in}{0.212622in}}{\pgfqpoint{3.696000in}{3.696000in}}%
\pgfusepath{clip}%
\pgfsetbuttcap%
\pgfsetroundjoin%
\definecolor{currentfill}{rgb}{0.121569,0.466667,0.705882}%
\pgfsetfillcolor{currentfill}%
\pgfsetfillopacity{0.833894}%
\pgfsetlinewidth{1.003750pt}%
\definecolor{currentstroke}{rgb}{0.121569,0.466667,0.705882}%
\pgfsetstrokecolor{currentstroke}%
\pgfsetstrokeopacity{0.833894}%
\pgfsetdash{}{0pt}%
\pgfpathmoveto{\pgfqpoint{0.963187in}{2.526746in}}%
\pgfpathcurveto{\pgfqpoint{0.971423in}{2.526746in}}{\pgfqpoint{0.979323in}{2.530018in}}{\pgfqpoint{0.985147in}{2.535842in}}%
\pgfpathcurveto{\pgfqpoint{0.990971in}{2.541666in}}{\pgfqpoint{0.994243in}{2.549566in}}{\pgfqpoint{0.994243in}{2.557802in}}%
\pgfpathcurveto{\pgfqpoint{0.994243in}{2.566039in}}{\pgfqpoint{0.990971in}{2.573939in}}{\pgfqpoint{0.985147in}{2.579763in}}%
\pgfpathcurveto{\pgfqpoint{0.979323in}{2.585586in}}{\pgfqpoint{0.971423in}{2.588859in}}{\pgfqpoint{0.963187in}{2.588859in}}%
\pgfpathcurveto{\pgfqpoint{0.954950in}{2.588859in}}{\pgfqpoint{0.947050in}{2.585586in}}{\pgfqpoint{0.941226in}{2.579763in}}%
\pgfpathcurveto{\pgfqpoint{0.935403in}{2.573939in}}{\pgfqpoint{0.932130in}{2.566039in}}{\pgfqpoint{0.932130in}{2.557802in}}%
\pgfpathcurveto{\pgfqpoint{0.932130in}{2.549566in}}{\pgfqpoint{0.935403in}{2.541666in}}{\pgfqpoint{0.941226in}{2.535842in}}%
\pgfpathcurveto{\pgfqpoint{0.947050in}{2.530018in}}{\pgfqpoint{0.954950in}{2.526746in}}{\pgfqpoint{0.963187in}{2.526746in}}%
\pgfpathclose%
\pgfusepath{stroke,fill}%
\end{pgfscope}%
\begin{pgfscope}%
\pgfpathrectangle{\pgfqpoint{0.100000in}{0.212622in}}{\pgfqpoint{3.696000in}{3.696000in}}%
\pgfusepath{clip}%
\pgfsetbuttcap%
\pgfsetroundjoin%
\definecolor{currentfill}{rgb}{0.121569,0.466667,0.705882}%
\pgfsetfillcolor{currentfill}%
\pgfsetfillopacity{0.835190}%
\pgfsetlinewidth{1.003750pt}%
\definecolor{currentstroke}{rgb}{0.121569,0.466667,0.705882}%
\pgfsetstrokecolor{currentstroke}%
\pgfsetstrokeopacity{0.835190}%
\pgfsetdash{}{0pt}%
\pgfpathmoveto{\pgfqpoint{0.969254in}{2.522927in}}%
\pgfpathcurveto{\pgfqpoint{0.977490in}{2.522927in}}{\pgfqpoint{0.985390in}{2.526199in}}{\pgfqpoint{0.991214in}{2.532023in}}%
\pgfpathcurveto{\pgfqpoint{0.997038in}{2.537847in}}{\pgfqpoint{1.000310in}{2.545747in}}{\pgfqpoint{1.000310in}{2.553983in}}%
\pgfpathcurveto{\pgfqpoint{1.000310in}{2.562220in}}{\pgfqpoint{0.997038in}{2.570120in}}{\pgfqpoint{0.991214in}{2.575944in}}%
\pgfpathcurveto{\pgfqpoint{0.985390in}{2.581768in}}{\pgfqpoint{0.977490in}{2.585040in}}{\pgfqpoint{0.969254in}{2.585040in}}%
\pgfpathcurveto{\pgfqpoint{0.961018in}{2.585040in}}{\pgfqpoint{0.953118in}{2.581768in}}{\pgfqpoint{0.947294in}{2.575944in}}%
\pgfpathcurveto{\pgfqpoint{0.941470in}{2.570120in}}{\pgfqpoint{0.938197in}{2.562220in}}{\pgfqpoint{0.938197in}{2.553983in}}%
\pgfpathcurveto{\pgfqpoint{0.938197in}{2.545747in}}{\pgfqpoint{0.941470in}{2.537847in}}{\pgfqpoint{0.947294in}{2.532023in}}%
\pgfpathcurveto{\pgfqpoint{0.953118in}{2.526199in}}{\pgfqpoint{0.961018in}{2.522927in}}{\pgfqpoint{0.969254in}{2.522927in}}%
\pgfpathclose%
\pgfusepath{stroke,fill}%
\end{pgfscope}%
\begin{pgfscope}%
\pgfpathrectangle{\pgfqpoint{0.100000in}{0.212622in}}{\pgfqpoint{3.696000in}{3.696000in}}%
\pgfusepath{clip}%
\pgfsetbuttcap%
\pgfsetroundjoin%
\definecolor{currentfill}{rgb}{0.121569,0.466667,0.705882}%
\pgfsetfillcolor{currentfill}%
\pgfsetfillopacity{0.836035}%
\pgfsetlinewidth{1.003750pt}%
\definecolor{currentstroke}{rgb}{0.121569,0.466667,0.705882}%
\pgfsetstrokecolor{currentstroke}%
\pgfsetstrokeopacity{0.836035}%
\pgfsetdash{}{0pt}%
\pgfpathmoveto{\pgfqpoint{0.982843in}{2.512408in}}%
\pgfpathcurveto{\pgfqpoint{0.991079in}{2.512408in}}{\pgfqpoint{0.998979in}{2.515680in}}{\pgfqpoint{1.004803in}{2.521504in}}%
\pgfpathcurveto{\pgfqpoint{1.010627in}{2.527328in}}{\pgfqpoint{1.013899in}{2.535228in}}{\pgfqpoint{1.013899in}{2.543464in}}%
\pgfpathcurveto{\pgfqpoint{1.013899in}{2.551700in}}{\pgfqpoint{1.010627in}{2.559601in}}{\pgfqpoint{1.004803in}{2.565424in}}%
\pgfpathcurveto{\pgfqpoint{0.998979in}{2.571248in}}{\pgfqpoint{0.991079in}{2.574521in}}{\pgfqpoint{0.982843in}{2.574521in}}%
\pgfpathcurveto{\pgfqpoint{0.974607in}{2.574521in}}{\pgfqpoint{0.966707in}{2.571248in}}{\pgfqpoint{0.960883in}{2.565424in}}%
\pgfpathcurveto{\pgfqpoint{0.955059in}{2.559601in}}{\pgfqpoint{0.951786in}{2.551700in}}{\pgfqpoint{0.951786in}{2.543464in}}%
\pgfpathcurveto{\pgfqpoint{0.951786in}{2.535228in}}{\pgfqpoint{0.955059in}{2.527328in}}{\pgfqpoint{0.960883in}{2.521504in}}%
\pgfpathcurveto{\pgfqpoint{0.966707in}{2.515680in}}{\pgfqpoint{0.974607in}{2.512408in}}{\pgfqpoint{0.982843in}{2.512408in}}%
\pgfpathclose%
\pgfusepath{stroke,fill}%
\end{pgfscope}%
\begin{pgfscope}%
\pgfpathrectangle{\pgfqpoint{0.100000in}{0.212622in}}{\pgfqpoint{3.696000in}{3.696000in}}%
\pgfusepath{clip}%
\pgfsetbuttcap%
\pgfsetroundjoin%
\definecolor{currentfill}{rgb}{0.121569,0.466667,0.705882}%
\pgfsetfillcolor{currentfill}%
\pgfsetfillopacity{0.838861}%
\pgfsetlinewidth{1.003750pt}%
\definecolor{currentstroke}{rgb}{0.121569,0.466667,0.705882}%
\pgfsetstrokecolor{currentstroke}%
\pgfsetstrokeopacity{0.838861}%
\pgfsetdash{}{0pt}%
\pgfpathmoveto{\pgfqpoint{2.771357in}{2.012704in}}%
\pgfpathcurveto{\pgfqpoint{2.779594in}{2.012704in}}{\pgfqpoint{2.787494in}{2.015976in}}{\pgfqpoint{2.793318in}{2.021800in}}%
\pgfpathcurveto{\pgfqpoint{2.799141in}{2.027624in}}{\pgfqpoint{2.802414in}{2.035524in}}{\pgfqpoint{2.802414in}{2.043761in}}%
\pgfpathcurveto{\pgfqpoint{2.802414in}{2.051997in}}{\pgfqpoint{2.799141in}{2.059897in}}{\pgfqpoint{2.793318in}{2.065721in}}%
\pgfpathcurveto{\pgfqpoint{2.787494in}{2.071545in}}{\pgfqpoint{2.779594in}{2.074817in}}{\pgfqpoint{2.771357in}{2.074817in}}%
\pgfpathcurveto{\pgfqpoint{2.763121in}{2.074817in}}{\pgfqpoint{2.755221in}{2.071545in}}{\pgfqpoint{2.749397in}{2.065721in}}%
\pgfpathcurveto{\pgfqpoint{2.743573in}{2.059897in}}{\pgfqpoint{2.740301in}{2.051997in}}{\pgfqpoint{2.740301in}{2.043761in}}%
\pgfpathcurveto{\pgfqpoint{2.740301in}{2.035524in}}{\pgfqpoint{2.743573in}{2.027624in}}{\pgfqpoint{2.749397in}{2.021800in}}%
\pgfpathcurveto{\pgfqpoint{2.755221in}{2.015976in}}{\pgfqpoint{2.763121in}{2.012704in}}{\pgfqpoint{2.771357in}{2.012704in}}%
\pgfpathclose%
\pgfusepath{stroke,fill}%
\end{pgfscope}%
\begin{pgfscope}%
\pgfpathrectangle{\pgfqpoint{0.100000in}{0.212622in}}{\pgfqpoint{3.696000in}{3.696000in}}%
\pgfusepath{clip}%
\pgfsetbuttcap%
\pgfsetroundjoin%
\definecolor{currentfill}{rgb}{0.121569,0.466667,0.705882}%
\pgfsetfillcolor{currentfill}%
\pgfsetfillopacity{0.839213}%
\pgfsetlinewidth{1.003750pt}%
\definecolor{currentstroke}{rgb}{0.121569,0.466667,0.705882}%
\pgfsetstrokecolor{currentstroke}%
\pgfsetstrokeopacity{0.839213}%
\pgfsetdash{}{0pt}%
\pgfpathmoveto{\pgfqpoint{1.006031in}{2.499924in}}%
\pgfpathcurveto{\pgfqpoint{1.014267in}{2.499924in}}{\pgfqpoint{1.022167in}{2.503196in}}{\pgfqpoint{1.027991in}{2.509020in}}%
\pgfpathcurveto{\pgfqpoint{1.033815in}{2.514844in}}{\pgfqpoint{1.037088in}{2.522744in}}{\pgfqpoint{1.037088in}{2.530981in}}%
\pgfpathcurveto{\pgfqpoint{1.037088in}{2.539217in}}{\pgfqpoint{1.033815in}{2.547117in}}{\pgfqpoint{1.027991in}{2.552941in}}%
\pgfpathcurveto{\pgfqpoint{1.022167in}{2.558765in}}{\pgfqpoint{1.014267in}{2.562037in}}{\pgfqpoint{1.006031in}{2.562037in}}%
\pgfpathcurveto{\pgfqpoint{0.997795in}{2.562037in}}{\pgfqpoint{0.989895in}{2.558765in}}{\pgfqpoint{0.984071in}{2.552941in}}%
\pgfpathcurveto{\pgfqpoint{0.978247in}{2.547117in}}{\pgfqpoint{0.974975in}{2.539217in}}{\pgfqpoint{0.974975in}{2.530981in}}%
\pgfpathcurveto{\pgfqpoint{0.974975in}{2.522744in}}{\pgfqpoint{0.978247in}{2.514844in}}{\pgfqpoint{0.984071in}{2.509020in}}%
\pgfpathcurveto{\pgfqpoint{0.989895in}{2.503196in}}{\pgfqpoint{0.997795in}{2.499924in}}{\pgfqpoint{1.006031in}{2.499924in}}%
\pgfpathclose%
\pgfusepath{stroke,fill}%
\end{pgfscope}%
\begin{pgfscope}%
\pgfpathrectangle{\pgfqpoint{0.100000in}{0.212622in}}{\pgfqpoint{3.696000in}{3.696000in}}%
\pgfusepath{clip}%
\pgfsetbuttcap%
\pgfsetroundjoin%
\definecolor{currentfill}{rgb}{0.121569,0.466667,0.705882}%
\pgfsetfillcolor{currentfill}%
\pgfsetfillopacity{0.840593}%
\pgfsetlinewidth{1.003750pt}%
\definecolor{currentstroke}{rgb}{0.121569,0.466667,0.705882}%
\pgfsetstrokecolor{currentstroke}%
\pgfsetstrokeopacity{0.840593}%
\pgfsetdash{}{0pt}%
\pgfpathmoveto{\pgfqpoint{1.025719in}{2.488689in}}%
\pgfpathcurveto{\pgfqpoint{1.033955in}{2.488689in}}{\pgfqpoint{1.041855in}{2.491962in}}{\pgfqpoint{1.047679in}{2.497786in}}%
\pgfpathcurveto{\pgfqpoint{1.053503in}{2.503609in}}{\pgfqpoint{1.056776in}{2.511509in}}{\pgfqpoint{1.056776in}{2.519746in}}%
\pgfpathcurveto{\pgfqpoint{1.056776in}{2.527982in}}{\pgfqpoint{1.053503in}{2.535882in}}{\pgfqpoint{1.047679in}{2.541706in}}%
\pgfpathcurveto{\pgfqpoint{1.041855in}{2.547530in}}{\pgfqpoint{1.033955in}{2.550802in}}{\pgfqpoint{1.025719in}{2.550802in}}%
\pgfpathcurveto{\pgfqpoint{1.017483in}{2.550802in}}{\pgfqpoint{1.009583in}{2.547530in}}{\pgfqpoint{1.003759in}{2.541706in}}%
\pgfpathcurveto{\pgfqpoint{0.997935in}{2.535882in}}{\pgfqpoint{0.994663in}{2.527982in}}{\pgfqpoint{0.994663in}{2.519746in}}%
\pgfpathcurveto{\pgfqpoint{0.994663in}{2.511509in}}{\pgfqpoint{0.997935in}{2.503609in}}{\pgfqpoint{1.003759in}{2.497786in}}%
\pgfpathcurveto{\pgfqpoint{1.009583in}{2.491962in}}{\pgfqpoint{1.017483in}{2.488689in}}{\pgfqpoint{1.025719in}{2.488689in}}%
\pgfpathclose%
\pgfusepath{stroke,fill}%
\end{pgfscope}%
\begin{pgfscope}%
\pgfpathrectangle{\pgfqpoint{0.100000in}{0.212622in}}{\pgfqpoint{3.696000in}{3.696000in}}%
\pgfusepath{clip}%
\pgfsetbuttcap%
\pgfsetroundjoin%
\definecolor{currentfill}{rgb}{0.121569,0.466667,0.705882}%
\pgfsetfillcolor{currentfill}%
\pgfsetfillopacity{0.842586}%
\pgfsetlinewidth{1.003750pt}%
\definecolor{currentstroke}{rgb}{0.121569,0.466667,0.705882}%
\pgfsetstrokecolor{currentstroke}%
\pgfsetstrokeopacity{0.842586}%
\pgfsetdash{}{0pt}%
\pgfpathmoveto{\pgfqpoint{1.038550in}{2.485036in}}%
\pgfpathcurveto{\pgfqpoint{1.046786in}{2.485036in}}{\pgfqpoint{1.054686in}{2.488308in}}{\pgfqpoint{1.060510in}{2.494132in}}%
\pgfpathcurveto{\pgfqpoint{1.066334in}{2.499956in}}{\pgfqpoint{1.069606in}{2.507856in}}{\pgfqpoint{1.069606in}{2.516092in}}%
\pgfpathcurveto{\pgfqpoint{1.069606in}{2.524329in}}{\pgfqpoint{1.066334in}{2.532229in}}{\pgfqpoint{1.060510in}{2.538053in}}%
\pgfpathcurveto{\pgfqpoint{1.054686in}{2.543876in}}{\pgfqpoint{1.046786in}{2.547149in}}{\pgfqpoint{1.038550in}{2.547149in}}%
\pgfpathcurveto{\pgfqpoint{1.030313in}{2.547149in}}{\pgfqpoint{1.022413in}{2.543876in}}{\pgfqpoint{1.016589in}{2.538053in}}%
\pgfpathcurveto{\pgfqpoint{1.010765in}{2.532229in}}{\pgfqpoint{1.007493in}{2.524329in}}{\pgfqpoint{1.007493in}{2.516092in}}%
\pgfpathcurveto{\pgfqpoint{1.007493in}{2.507856in}}{\pgfqpoint{1.010765in}{2.499956in}}{\pgfqpoint{1.016589in}{2.494132in}}%
\pgfpathcurveto{\pgfqpoint{1.022413in}{2.488308in}}{\pgfqpoint{1.030313in}{2.485036in}}{\pgfqpoint{1.038550in}{2.485036in}}%
\pgfpathclose%
\pgfusepath{stroke,fill}%
\end{pgfscope}%
\begin{pgfscope}%
\pgfpathrectangle{\pgfqpoint{0.100000in}{0.212622in}}{\pgfqpoint{3.696000in}{3.696000in}}%
\pgfusepath{clip}%
\pgfsetbuttcap%
\pgfsetroundjoin%
\definecolor{currentfill}{rgb}{0.121569,0.466667,0.705882}%
\pgfsetfillcolor{currentfill}%
\pgfsetfillopacity{0.843141}%
\pgfsetlinewidth{1.003750pt}%
\definecolor{currentstroke}{rgb}{0.121569,0.466667,0.705882}%
\pgfsetstrokecolor{currentstroke}%
\pgfsetstrokeopacity{0.843141}%
\pgfsetdash{}{0pt}%
\pgfpathmoveto{\pgfqpoint{1.046932in}{2.480212in}}%
\pgfpathcurveto{\pgfqpoint{1.055168in}{2.480212in}}{\pgfqpoint{1.063068in}{2.483484in}}{\pgfqpoint{1.068892in}{2.489308in}}%
\pgfpathcurveto{\pgfqpoint{1.074716in}{2.495132in}}{\pgfqpoint{1.077988in}{2.503032in}}{\pgfqpoint{1.077988in}{2.511268in}}%
\pgfpathcurveto{\pgfqpoint{1.077988in}{2.519504in}}{\pgfqpoint{1.074716in}{2.527405in}}{\pgfqpoint{1.068892in}{2.533228in}}%
\pgfpathcurveto{\pgfqpoint{1.063068in}{2.539052in}}{\pgfqpoint{1.055168in}{2.542325in}}{\pgfqpoint{1.046932in}{2.542325in}}%
\pgfpathcurveto{\pgfqpoint{1.038696in}{2.542325in}}{\pgfqpoint{1.030796in}{2.539052in}}{\pgfqpoint{1.024972in}{2.533228in}}%
\pgfpathcurveto{\pgfqpoint{1.019148in}{2.527405in}}{\pgfqpoint{1.015875in}{2.519504in}}{\pgfqpoint{1.015875in}{2.511268in}}%
\pgfpathcurveto{\pgfqpoint{1.015875in}{2.503032in}}{\pgfqpoint{1.019148in}{2.495132in}}{\pgfqpoint{1.024972in}{2.489308in}}%
\pgfpathcurveto{\pgfqpoint{1.030796in}{2.483484in}}{\pgfqpoint{1.038696in}{2.480212in}}{\pgfqpoint{1.046932in}{2.480212in}}%
\pgfpathclose%
\pgfusepath{stroke,fill}%
\end{pgfscope}%
\begin{pgfscope}%
\pgfpathrectangle{\pgfqpoint{0.100000in}{0.212622in}}{\pgfqpoint{3.696000in}{3.696000in}}%
\pgfusepath{clip}%
\pgfsetbuttcap%
\pgfsetroundjoin%
\definecolor{currentfill}{rgb}{0.121569,0.466667,0.705882}%
\pgfsetfillcolor{currentfill}%
\pgfsetfillopacity{0.843704}%
\pgfsetlinewidth{1.003750pt}%
\definecolor{currentstroke}{rgb}{0.121569,0.466667,0.705882}%
\pgfsetstrokecolor{currentstroke}%
\pgfsetstrokeopacity{0.843704}%
\pgfsetdash{}{0pt}%
\pgfpathmoveto{\pgfqpoint{1.050512in}{2.478716in}}%
\pgfpathcurveto{\pgfqpoint{1.058748in}{2.478716in}}{\pgfqpoint{1.066649in}{2.481988in}}{\pgfqpoint{1.072472in}{2.487812in}}%
\pgfpathcurveto{\pgfqpoint{1.078296in}{2.493636in}}{\pgfqpoint{1.081569in}{2.501536in}}{\pgfqpoint{1.081569in}{2.509772in}}%
\pgfpathcurveto{\pgfqpoint{1.081569in}{2.518008in}}{\pgfqpoint{1.078296in}{2.525908in}}{\pgfqpoint{1.072472in}{2.531732in}}%
\pgfpathcurveto{\pgfqpoint{1.066649in}{2.537556in}}{\pgfqpoint{1.058748in}{2.540829in}}{\pgfqpoint{1.050512in}{2.540829in}}%
\pgfpathcurveto{\pgfqpoint{1.042276in}{2.540829in}}{\pgfqpoint{1.034376in}{2.537556in}}{\pgfqpoint{1.028552in}{2.531732in}}%
\pgfpathcurveto{\pgfqpoint{1.022728in}{2.525908in}}{\pgfqpoint{1.019456in}{2.518008in}}{\pgfqpoint{1.019456in}{2.509772in}}%
\pgfpathcurveto{\pgfqpoint{1.019456in}{2.501536in}}{\pgfqpoint{1.022728in}{2.493636in}}{\pgfqpoint{1.028552in}{2.487812in}}%
\pgfpathcurveto{\pgfqpoint{1.034376in}{2.481988in}}{\pgfqpoint{1.042276in}{2.478716in}}{\pgfqpoint{1.050512in}{2.478716in}}%
\pgfpathclose%
\pgfusepath{stroke,fill}%
\end{pgfscope}%
\begin{pgfscope}%
\pgfpathrectangle{\pgfqpoint{0.100000in}{0.212622in}}{\pgfqpoint{3.696000in}{3.696000in}}%
\pgfusepath{clip}%
\pgfsetbuttcap%
\pgfsetroundjoin%
\definecolor{currentfill}{rgb}{0.121569,0.466667,0.705882}%
\pgfsetfillcolor{currentfill}%
\pgfsetfillopacity{0.844286}%
\pgfsetlinewidth{1.003750pt}%
\definecolor{currentstroke}{rgb}{0.121569,0.466667,0.705882}%
\pgfsetstrokecolor{currentstroke}%
\pgfsetstrokeopacity{0.844286}%
\pgfsetdash{}{0pt}%
\pgfpathmoveto{\pgfqpoint{1.057717in}{2.475402in}}%
\pgfpathcurveto{\pgfqpoint{1.065953in}{2.475402in}}{\pgfqpoint{1.073853in}{2.478674in}}{\pgfqpoint{1.079677in}{2.484498in}}%
\pgfpathcurveto{\pgfqpoint{1.085501in}{2.490322in}}{\pgfqpoint{1.088773in}{2.498222in}}{\pgfqpoint{1.088773in}{2.506458in}}%
\pgfpathcurveto{\pgfqpoint{1.088773in}{2.514695in}}{\pgfqpoint{1.085501in}{2.522595in}}{\pgfqpoint{1.079677in}{2.528418in}}%
\pgfpathcurveto{\pgfqpoint{1.073853in}{2.534242in}}{\pgfqpoint{1.065953in}{2.537515in}}{\pgfqpoint{1.057717in}{2.537515in}}%
\pgfpathcurveto{\pgfqpoint{1.049481in}{2.537515in}}{\pgfqpoint{1.041581in}{2.534242in}}{\pgfqpoint{1.035757in}{2.528418in}}%
\pgfpathcurveto{\pgfqpoint{1.029933in}{2.522595in}}{\pgfqpoint{1.026660in}{2.514695in}}{\pgfqpoint{1.026660in}{2.506458in}}%
\pgfpathcurveto{\pgfqpoint{1.026660in}{2.498222in}}{\pgfqpoint{1.029933in}{2.490322in}}{\pgfqpoint{1.035757in}{2.484498in}}%
\pgfpathcurveto{\pgfqpoint{1.041581in}{2.478674in}}{\pgfqpoint{1.049481in}{2.475402in}}{\pgfqpoint{1.057717in}{2.475402in}}%
\pgfpathclose%
\pgfusepath{stroke,fill}%
\end{pgfscope}%
\begin{pgfscope}%
\pgfpathrectangle{\pgfqpoint{0.100000in}{0.212622in}}{\pgfqpoint{3.696000in}{3.696000in}}%
\pgfusepath{clip}%
\pgfsetbuttcap%
\pgfsetroundjoin%
\definecolor{currentfill}{rgb}{0.121569,0.466667,0.705882}%
\pgfsetfillcolor{currentfill}%
\pgfsetfillopacity{0.844754}%
\pgfsetlinewidth{1.003750pt}%
\definecolor{currentstroke}{rgb}{0.121569,0.466667,0.705882}%
\pgfsetstrokecolor{currentstroke}%
\pgfsetstrokeopacity{0.844754}%
\pgfsetdash{}{0pt}%
\pgfpathmoveto{\pgfqpoint{2.752887in}{2.007090in}}%
\pgfpathcurveto{\pgfqpoint{2.761123in}{2.007090in}}{\pgfqpoint{2.769023in}{2.010362in}}{\pgfqpoint{2.774847in}{2.016186in}}%
\pgfpathcurveto{\pgfqpoint{2.780671in}{2.022010in}}{\pgfqpoint{2.783943in}{2.029910in}}{\pgfqpoint{2.783943in}{2.038146in}}%
\pgfpathcurveto{\pgfqpoint{2.783943in}{2.046383in}}{\pgfqpoint{2.780671in}{2.054283in}}{\pgfqpoint{2.774847in}{2.060107in}}%
\pgfpathcurveto{\pgfqpoint{2.769023in}{2.065931in}}{\pgfqpoint{2.761123in}{2.069203in}}{\pgfqpoint{2.752887in}{2.069203in}}%
\pgfpathcurveto{\pgfqpoint{2.744650in}{2.069203in}}{\pgfqpoint{2.736750in}{2.065931in}}{\pgfqpoint{2.730926in}{2.060107in}}%
\pgfpathcurveto{\pgfqpoint{2.725102in}{2.054283in}}{\pgfqpoint{2.721830in}{2.046383in}}{\pgfqpoint{2.721830in}{2.038146in}}%
\pgfpathcurveto{\pgfqpoint{2.721830in}{2.029910in}}{\pgfqpoint{2.725102in}{2.022010in}}{\pgfqpoint{2.730926in}{2.016186in}}%
\pgfpathcurveto{\pgfqpoint{2.736750in}{2.010362in}}{\pgfqpoint{2.744650in}{2.007090in}}{\pgfqpoint{2.752887in}{2.007090in}}%
\pgfpathclose%
\pgfusepath{stroke,fill}%
\end{pgfscope}%
\begin{pgfscope}%
\pgfpathrectangle{\pgfqpoint{0.100000in}{0.212622in}}{\pgfqpoint{3.696000in}{3.696000in}}%
\pgfusepath{clip}%
\pgfsetbuttcap%
\pgfsetroundjoin%
\definecolor{currentfill}{rgb}{0.121569,0.466667,0.705882}%
\pgfsetfillcolor{currentfill}%
\pgfsetfillopacity{0.846421}%
\pgfsetlinewidth{1.003750pt}%
\definecolor{currentstroke}{rgb}{0.121569,0.466667,0.705882}%
\pgfsetstrokecolor{currentstroke}%
\pgfsetstrokeopacity{0.846421}%
\pgfsetdash{}{0pt}%
\pgfpathmoveto{\pgfqpoint{1.069656in}{2.472821in}}%
\pgfpathcurveto{\pgfqpoint{1.077892in}{2.472821in}}{\pgfqpoint{1.085792in}{2.476093in}}{\pgfqpoint{1.091616in}{2.481917in}}%
\pgfpathcurveto{\pgfqpoint{1.097440in}{2.487741in}}{\pgfqpoint{1.100712in}{2.495641in}}{\pgfqpoint{1.100712in}{2.503877in}}%
\pgfpathcurveto{\pgfqpoint{1.100712in}{2.512114in}}{\pgfqpoint{1.097440in}{2.520014in}}{\pgfqpoint{1.091616in}{2.525838in}}%
\pgfpathcurveto{\pgfqpoint{1.085792in}{2.531662in}}{\pgfqpoint{1.077892in}{2.534934in}}{\pgfqpoint{1.069656in}{2.534934in}}%
\pgfpathcurveto{\pgfqpoint{1.061419in}{2.534934in}}{\pgfqpoint{1.053519in}{2.531662in}}{\pgfqpoint{1.047695in}{2.525838in}}%
\pgfpathcurveto{\pgfqpoint{1.041871in}{2.520014in}}{\pgfqpoint{1.038599in}{2.512114in}}{\pgfqpoint{1.038599in}{2.503877in}}%
\pgfpathcurveto{\pgfqpoint{1.038599in}{2.495641in}}{\pgfqpoint{1.041871in}{2.487741in}}{\pgfqpoint{1.047695in}{2.481917in}}%
\pgfpathcurveto{\pgfqpoint{1.053519in}{2.476093in}}{\pgfqpoint{1.061419in}{2.472821in}}{\pgfqpoint{1.069656in}{2.472821in}}%
\pgfpathclose%
\pgfusepath{stroke,fill}%
\end{pgfscope}%
\begin{pgfscope}%
\pgfpathrectangle{\pgfqpoint{0.100000in}{0.212622in}}{\pgfqpoint{3.696000in}{3.696000in}}%
\pgfusepath{clip}%
\pgfsetbuttcap%
\pgfsetroundjoin%
\definecolor{currentfill}{rgb}{0.121569,0.466667,0.705882}%
\pgfsetfillcolor{currentfill}%
\pgfsetfillopacity{0.848627}%
\pgfsetlinewidth{1.003750pt}%
\definecolor{currentstroke}{rgb}{0.121569,0.466667,0.705882}%
\pgfsetstrokecolor{currentstroke}%
\pgfsetstrokeopacity{0.848627}%
\pgfsetdash{}{0pt}%
\pgfpathmoveto{\pgfqpoint{1.092741in}{2.460891in}}%
\pgfpathcurveto{\pgfqpoint{1.100977in}{2.460891in}}{\pgfqpoint{1.108877in}{2.464164in}}{\pgfqpoint{1.114701in}{2.469988in}}%
\pgfpathcurveto{\pgfqpoint{1.120525in}{2.475812in}}{\pgfqpoint{1.123797in}{2.483712in}}{\pgfqpoint{1.123797in}{2.491948in}}%
\pgfpathcurveto{\pgfqpoint{1.123797in}{2.500184in}}{\pgfqpoint{1.120525in}{2.508084in}}{\pgfqpoint{1.114701in}{2.513908in}}%
\pgfpathcurveto{\pgfqpoint{1.108877in}{2.519732in}}{\pgfqpoint{1.100977in}{2.523004in}}{\pgfqpoint{1.092741in}{2.523004in}}%
\pgfpathcurveto{\pgfqpoint{1.084505in}{2.523004in}}{\pgfqpoint{1.076605in}{2.519732in}}{\pgfqpoint{1.070781in}{2.513908in}}%
\pgfpathcurveto{\pgfqpoint{1.064957in}{2.508084in}}{\pgfqpoint{1.061684in}{2.500184in}}{\pgfqpoint{1.061684in}{2.491948in}}%
\pgfpathcurveto{\pgfqpoint{1.061684in}{2.483712in}}{\pgfqpoint{1.064957in}{2.475812in}}{\pgfqpoint{1.070781in}{2.469988in}}%
\pgfpathcurveto{\pgfqpoint{1.076605in}{2.464164in}}{\pgfqpoint{1.084505in}{2.460891in}}{\pgfqpoint{1.092741in}{2.460891in}}%
\pgfpathclose%
\pgfusepath{stroke,fill}%
\end{pgfscope}%
\begin{pgfscope}%
\pgfpathrectangle{\pgfqpoint{0.100000in}{0.212622in}}{\pgfqpoint{3.696000in}{3.696000in}}%
\pgfusepath{clip}%
\pgfsetbuttcap%
\pgfsetroundjoin%
\definecolor{currentfill}{rgb}{0.121569,0.466667,0.705882}%
\pgfsetfillcolor{currentfill}%
\pgfsetfillopacity{0.851290}%
\pgfsetlinewidth{1.003750pt}%
\definecolor{currentstroke}{rgb}{0.121569,0.466667,0.705882}%
\pgfsetstrokecolor{currentstroke}%
\pgfsetstrokeopacity{0.851290}%
\pgfsetdash{}{0pt}%
\pgfpathmoveto{\pgfqpoint{1.112046in}{2.450850in}}%
\pgfpathcurveto{\pgfqpoint{1.120282in}{2.450850in}}{\pgfqpoint{1.128182in}{2.454123in}}{\pgfqpoint{1.134006in}{2.459947in}}%
\pgfpathcurveto{\pgfqpoint{1.139830in}{2.465771in}}{\pgfqpoint{1.143102in}{2.473671in}}{\pgfqpoint{1.143102in}{2.481907in}}%
\pgfpathcurveto{\pgfqpoint{1.143102in}{2.490143in}}{\pgfqpoint{1.139830in}{2.498043in}}{\pgfqpoint{1.134006in}{2.503867in}}%
\pgfpathcurveto{\pgfqpoint{1.128182in}{2.509691in}}{\pgfqpoint{1.120282in}{2.512963in}}{\pgfqpoint{1.112046in}{2.512963in}}%
\pgfpathcurveto{\pgfqpoint{1.103810in}{2.512963in}}{\pgfqpoint{1.095909in}{2.509691in}}{\pgfqpoint{1.090086in}{2.503867in}}%
\pgfpathcurveto{\pgfqpoint{1.084262in}{2.498043in}}{\pgfqpoint{1.080989in}{2.490143in}}{\pgfqpoint{1.080989in}{2.481907in}}%
\pgfpathcurveto{\pgfqpoint{1.080989in}{2.473671in}}{\pgfqpoint{1.084262in}{2.465771in}}{\pgfqpoint{1.090086in}{2.459947in}}%
\pgfpathcurveto{\pgfqpoint{1.095909in}{2.454123in}}{\pgfqpoint{1.103810in}{2.450850in}}{\pgfqpoint{1.112046in}{2.450850in}}%
\pgfpathclose%
\pgfusepath{stroke,fill}%
\end{pgfscope}%
\begin{pgfscope}%
\pgfpathrectangle{\pgfqpoint{0.100000in}{0.212622in}}{\pgfqpoint{3.696000in}{3.696000in}}%
\pgfusepath{clip}%
\pgfsetbuttcap%
\pgfsetroundjoin%
\definecolor{currentfill}{rgb}{0.121569,0.466667,0.705882}%
\pgfsetfillcolor{currentfill}%
\pgfsetfillopacity{0.851844}%
\pgfsetlinewidth{1.003750pt}%
\definecolor{currentstroke}{rgb}{0.121569,0.466667,0.705882}%
\pgfsetstrokecolor{currentstroke}%
\pgfsetstrokeopacity{0.851844}%
\pgfsetdash{}{0pt}%
\pgfpathmoveto{\pgfqpoint{2.738655in}{1.996885in}}%
\pgfpathcurveto{\pgfqpoint{2.746891in}{1.996885in}}{\pgfqpoint{2.754792in}{2.000157in}}{\pgfqpoint{2.760615in}{2.005981in}}%
\pgfpathcurveto{\pgfqpoint{2.766439in}{2.011805in}}{\pgfqpoint{2.769712in}{2.019705in}}{\pgfqpoint{2.769712in}{2.027941in}}%
\pgfpathcurveto{\pgfqpoint{2.769712in}{2.036178in}}{\pgfqpoint{2.766439in}{2.044078in}}{\pgfqpoint{2.760615in}{2.049902in}}%
\pgfpathcurveto{\pgfqpoint{2.754792in}{2.055726in}}{\pgfqpoint{2.746891in}{2.058998in}}{\pgfqpoint{2.738655in}{2.058998in}}%
\pgfpathcurveto{\pgfqpoint{2.730419in}{2.058998in}}{\pgfqpoint{2.722519in}{2.055726in}}{\pgfqpoint{2.716695in}{2.049902in}}%
\pgfpathcurveto{\pgfqpoint{2.710871in}{2.044078in}}{\pgfqpoint{2.707599in}{2.036178in}}{\pgfqpoint{2.707599in}{2.027941in}}%
\pgfpathcurveto{\pgfqpoint{2.707599in}{2.019705in}}{\pgfqpoint{2.710871in}{2.011805in}}{\pgfqpoint{2.716695in}{2.005981in}}%
\pgfpathcurveto{\pgfqpoint{2.722519in}{2.000157in}}{\pgfqpoint{2.730419in}{1.996885in}}{\pgfqpoint{2.738655in}{1.996885in}}%
\pgfpathclose%
\pgfusepath{stroke,fill}%
\end{pgfscope}%
\begin{pgfscope}%
\pgfpathrectangle{\pgfqpoint{0.100000in}{0.212622in}}{\pgfqpoint{3.696000in}{3.696000in}}%
\pgfusepath{clip}%
\pgfsetbuttcap%
\pgfsetroundjoin%
\definecolor{currentfill}{rgb}{0.121569,0.466667,0.705882}%
\pgfsetfillcolor{currentfill}%
\pgfsetfillopacity{0.852707}%
\pgfsetlinewidth{1.003750pt}%
\definecolor{currentstroke}{rgb}{0.121569,0.466667,0.705882}%
\pgfsetstrokecolor{currentstroke}%
\pgfsetstrokeopacity{0.852707}%
\pgfsetdash{}{0pt}%
\pgfpathmoveto{\pgfqpoint{1.129259in}{2.441205in}}%
\pgfpathcurveto{\pgfqpoint{1.137496in}{2.441205in}}{\pgfqpoint{1.145396in}{2.444478in}}{\pgfqpoint{1.151220in}{2.450302in}}%
\pgfpathcurveto{\pgfqpoint{1.157043in}{2.456126in}}{\pgfqpoint{1.160316in}{2.464026in}}{\pgfqpoint{1.160316in}{2.472262in}}%
\pgfpathcurveto{\pgfqpoint{1.160316in}{2.480498in}}{\pgfqpoint{1.157043in}{2.488398in}}{\pgfqpoint{1.151220in}{2.494222in}}%
\pgfpathcurveto{\pgfqpoint{1.145396in}{2.500046in}}{\pgfqpoint{1.137496in}{2.503318in}}{\pgfqpoint{1.129259in}{2.503318in}}%
\pgfpathcurveto{\pgfqpoint{1.121023in}{2.503318in}}{\pgfqpoint{1.113123in}{2.500046in}}{\pgfqpoint{1.107299in}{2.494222in}}%
\pgfpathcurveto{\pgfqpoint{1.101475in}{2.488398in}}{\pgfqpoint{1.098203in}{2.480498in}}{\pgfqpoint{1.098203in}{2.472262in}}%
\pgfpathcurveto{\pgfqpoint{1.098203in}{2.464026in}}{\pgfqpoint{1.101475in}{2.456126in}}{\pgfqpoint{1.107299in}{2.450302in}}%
\pgfpathcurveto{\pgfqpoint{1.113123in}{2.444478in}}{\pgfqpoint{1.121023in}{2.441205in}}{\pgfqpoint{1.129259in}{2.441205in}}%
\pgfpathclose%
\pgfusepath{stroke,fill}%
\end{pgfscope}%
\begin{pgfscope}%
\pgfpathrectangle{\pgfqpoint{0.100000in}{0.212622in}}{\pgfqpoint{3.696000in}{3.696000in}}%
\pgfusepath{clip}%
\pgfsetbuttcap%
\pgfsetroundjoin%
\definecolor{currentfill}{rgb}{0.121569,0.466667,0.705882}%
\pgfsetfillcolor{currentfill}%
\pgfsetfillopacity{0.854447}%
\pgfsetlinewidth{1.003750pt}%
\definecolor{currentstroke}{rgb}{0.121569,0.466667,0.705882}%
\pgfsetstrokecolor{currentstroke}%
\pgfsetstrokeopacity{0.854447}%
\pgfsetdash{}{0pt}%
\pgfpathmoveto{\pgfqpoint{1.139823in}{2.437997in}}%
\pgfpathcurveto{\pgfqpoint{1.148059in}{2.437997in}}{\pgfqpoint{1.155959in}{2.441270in}}{\pgfqpoint{1.161783in}{2.447093in}}%
\pgfpathcurveto{\pgfqpoint{1.167607in}{2.452917in}}{\pgfqpoint{1.170879in}{2.460817in}}{\pgfqpoint{1.170879in}{2.469054in}}%
\pgfpathcurveto{\pgfqpoint{1.170879in}{2.477290in}}{\pgfqpoint{1.167607in}{2.485190in}}{\pgfqpoint{1.161783in}{2.491014in}}%
\pgfpathcurveto{\pgfqpoint{1.155959in}{2.496838in}}{\pgfqpoint{1.148059in}{2.500110in}}{\pgfqpoint{1.139823in}{2.500110in}}%
\pgfpathcurveto{\pgfqpoint{1.131586in}{2.500110in}}{\pgfqpoint{1.123686in}{2.496838in}}{\pgfqpoint{1.117862in}{2.491014in}}%
\pgfpathcurveto{\pgfqpoint{1.112039in}{2.485190in}}{\pgfqpoint{1.108766in}{2.477290in}}{\pgfqpoint{1.108766in}{2.469054in}}%
\pgfpathcurveto{\pgfqpoint{1.108766in}{2.460817in}}{\pgfqpoint{1.112039in}{2.452917in}}{\pgfqpoint{1.117862in}{2.447093in}}%
\pgfpathcurveto{\pgfqpoint{1.123686in}{2.441270in}}{\pgfqpoint{1.131586in}{2.437997in}}{\pgfqpoint{1.139823in}{2.437997in}}%
\pgfpathclose%
\pgfusepath{stroke,fill}%
\end{pgfscope}%
\begin{pgfscope}%
\pgfpathrectangle{\pgfqpoint{0.100000in}{0.212622in}}{\pgfqpoint{3.696000in}{3.696000in}}%
\pgfusepath{clip}%
\pgfsetbuttcap%
\pgfsetroundjoin%
\definecolor{currentfill}{rgb}{0.121569,0.466667,0.705882}%
\pgfsetfillcolor{currentfill}%
\pgfsetfillopacity{0.855200}%
\pgfsetlinewidth{1.003750pt}%
\definecolor{currentstroke}{rgb}{0.121569,0.466667,0.705882}%
\pgfsetstrokecolor{currentstroke}%
\pgfsetstrokeopacity{0.855200}%
\pgfsetdash{}{0pt}%
\pgfpathmoveto{\pgfqpoint{1.147167in}{2.434821in}}%
\pgfpathcurveto{\pgfqpoint{1.155403in}{2.434821in}}{\pgfqpoint{1.163303in}{2.438094in}}{\pgfqpoint{1.169127in}{2.443918in}}%
\pgfpathcurveto{\pgfqpoint{1.174951in}{2.449741in}}{\pgfqpoint{1.178223in}{2.457641in}}{\pgfqpoint{1.178223in}{2.465878in}}%
\pgfpathcurveto{\pgfqpoint{1.178223in}{2.474114in}}{\pgfqpoint{1.174951in}{2.482014in}}{\pgfqpoint{1.169127in}{2.487838in}}%
\pgfpathcurveto{\pgfqpoint{1.163303in}{2.493662in}}{\pgfqpoint{1.155403in}{2.496934in}}{\pgfqpoint{1.147167in}{2.496934in}}%
\pgfpathcurveto{\pgfqpoint{1.138930in}{2.496934in}}{\pgfqpoint{1.131030in}{2.493662in}}{\pgfqpoint{1.125206in}{2.487838in}}%
\pgfpathcurveto{\pgfqpoint{1.119382in}{2.482014in}}{\pgfqpoint{1.116110in}{2.474114in}}{\pgfqpoint{1.116110in}{2.465878in}}%
\pgfpathcurveto{\pgfqpoint{1.116110in}{2.457641in}}{\pgfqpoint{1.119382in}{2.449741in}}{\pgfqpoint{1.125206in}{2.443918in}}%
\pgfpathcurveto{\pgfqpoint{1.131030in}{2.438094in}}{\pgfqpoint{1.138930in}{2.434821in}}{\pgfqpoint{1.147167in}{2.434821in}}%
\pgfpathclose%
\pgfusepath{stroke,fill}%
\end{pgfscope}%
\begin{pgfscope}%
\pgfpathrectangle{\pgfqpoint{0.100000in}{0.212622in}}{\pgfqpoint{3.696000in}{3.696000in}}%
\pgfusepath{clip}%
\pgfsetbuttcap%
\pgfsetroundjoin%
\definecolor{currentfill}{rgb}{0.121569,0.466667,0.705882}%
\pgfsetfillcolor{currentfill}%
\pgfsetfillopacity{0.855509}%
\pgfsetlinewidth{1.003750pt}%
\definecolor{currentstroke}{rgb}{0.121569,0.466667,0.705882}%
\pgfsetstrokecolor{currentstroke}%
\pgfsetstrokeopacity{0.855509}%
\pgfsetdash{}{0pt}%
\pgfpathmoveto{\pgfqpoint{2.728781in}{1.992398in}}%
\pgfpathcurveto{\pgfqpoint{2.737018in}{1.992398in}}{\pgfqpoint{2.744918in}{1.995670in}}{\pgfqpoint{2.750742in}{2.001494in}}%
\pgfpathcurveto{\pgfqpoint{2.756565in}{2.007318in}}{\pgfqpoint{2.759838in}{2.015218in}}{\pgfqpoint{2.759838in}{2.023455in}}%
\pgfpathcurveto{\pgfqpoint{2.759838in}{2.031691in}}{\pgfqpoint{2.756565in}{2.039591in}}{\pgfqpoint{2.750742in}{2.045415in}}%
\pgfpathcurveto{\pgfqpoint{2.744918in}{2.051239in}}{\pgfqpoint{2.737018in}{2.054511in}}{\pgfqpoint{2.728781in}{2.054511in}}%
\pgfpathcurveto{\pgfqpoint{2.720545in}{2.054511in}}{\pgfqpoint{2.712645in}{2.051239in}}{\pgfqpoint{2.706821in}{2.045415in}}%
\pgfpathcurveto{\pgfqpoint{2.700997in}{2.039591in}}{\pgfqpoint{2.697725in}{2.031691in}}{\pgfqpoint{2.697725in}{2.023455in}}%
\pgfpathcurveto{\pgfqpoint{2.697725in}{2.015218in}}{\pgfqpoint{2.700997in}{2.007318in}}{\pgfqpoint{2.706821in}{2.001494in}}%
\pgfpathcurveto{\pgfqpoint{2.712645in}{1.995670in}}{\pgfqpoint{2.720545in}{1.992398in}}{\pgfqpoint{2.728781in}{1.992398in}}%
\pgfpathclose%
\pgfusepath{stroke,fill}%
\end{pgfscope}%
\begin{pgfscope}%
\pgfpathrectangle{\pgfqpoint{0.100000in}{0.212622in}}{\pgfqpoint{3.696000in}{3.696000in}}%
\pgfusepath{clip}%
\pgfsetbuttcap%
\pgfsetroundjoin%
\definecolor{currentfill}{rgb}{0.121569,0.466667,0.705882}%
\pgfsetfillcolor{currentfill}%
\pgfsetfillopacity{0.856779}%
\pgfsetlinewidth{1.003750pt}%
\definecolor{currentstroke}{rgb}{0.121569,0.466667,0.705882}%
\pgfsetstrokecolor{currentstroke}%
\pgfsetstrokeopacity{0.856779}%
\pgfsetdash{}{0pt}%
\pgfpathmoveto{\pgfqpoint{1.159633in}{2.427633in}}%
\pgfpathcurveto{\pgfqpoint{1.167870in}{2.427633in}}{\pgfqpoint{1.175770in}{2.430906in}}{\pgfqpoint{1.181593in}{2.436730in}}%
\pgfpathcurveto{\pgfqpoint{1.187417in}{2.442553in}}{\pgfqpoint{1.190690in}{2.450454in}}{\pgfqpoint{1.190690in}{2.458690in}}%
\pgfpathcurveto{\pgfqpoint{1.190690in}{2.466926in}}{\pgfqpoint{1.187417in}{2.474826in}}{\pgfqpoint{1.181593in}{2.480650in}}%
\pgfpathcurveto{\pgfqpoint{1.175770in}{2.486474in}}{\pgfqpoint{1.167870in}{2.489746in}}{\pgfqpoint{1.159633in}{2.489746in}}%
\pgfpathcurveto{\pgfqpoint{1.151397in}{2.489746in}}{\pgfqpoint{1.143497in}{2.486474in}}{\pgfqpoint{1.137673in}{2.480650in}}%
\pgfpathcurveto{\pgfqpoint{1.131849in}{2.474826in}}{\pgfqpoint{1.128577in}{2.466926in}}{\pgfqpoint{1.128577in}{2.458690in}}%
\pgfpathcurveto{\pgfqpoint{1.128577in}{2.450454in}}{\pgfqpoint{1.131849in}{2.442553in}}{\pgfqpoint{1.137673in}{2.436730in}}%
\pgfpathcurveto{\pgfqpoint{1.143497in}{2.430906in}}{\pgfqpoint{1.151397in}{2.427633in}}{\pgfqpoint{1.159633in}{2.427633in}}%
\pgfpathclose%
\pgfusepath{stroke,fill}%
\end{pgfscope}%
\begin{pgfscope}%
\pgfpathrectangle{\pgfqpoint{0.100000in}{0.212622in}}{\pgfqpoint{3.696000in}{3.696000in}}%
\pgfusepath{clip}%
\pgfsetbuttcap%
\pgfsetroundjoin%
\definecolor{currentfill}{rgb}{0.121569,0.466667,0.705882}%
\pgfsetfillcolor{currentfill}%
\pgfsetfillopacity{0.857641}%
\pgfsetlinewidth{1.003750pt}%
\definecolor{currentstroke}{rgb}{0.121569,0.466667,0.705882}%
\pgfsetstrokecolor{currentstroke}%
\pgfsetstrokeopacity{0.857641}%
\pgfsetdash{}{0pt}%
\pgfpathmoveto{\pgfqpoint{2.723485in}{1.990460in}}%
\pgfpathcurveto{\pgfqpoint{2.731721in}{1.990460in}}{\pgfqpoint{2.739621in}{1.993732in}}{\pgfqpoint{2.745445in}{1.999556in}}%
\pgfpathcurveto{\pgfqpoint{2.751269in}{2.005380in}}{\pgfqpoint{2.754542in}{2.013280in}}{\pgfqpoint{2.754542in}{2.021516in}}%
\pgfpathcurveto{\pgfqpoint{2.754542in}{2.029752in}}{\pgfqpoint{2.751269in}{2.037652in}}{\pgfqpoint{2.745445in}{2.043476in}}%
\pgfpathcurveto{\pgfqpoint{2.739621in}{2.049300in}}{\pgfqpoint{2.731721in}{2.052573in}}{\pgfqpoint{2.723485in}{2.052573in}}%
\pgfpathcurveto{\pgfqpoint{2.715249in}{2.052573in}}{\pgfqpoint{2.707349in}{2.049300in}}{\pgfqpoint{2.701525in}{2.043476in}}%
\pgfpathcurveto{\pgfqpoint{2.695701in}{2.037652in}}{\pgfqpoint{2.692429in}{2.029752in}}{\pgfqpoint{2.692429in}{2.021516in}}%
\pgfpathcurveto{\pgfqpoint{2.692429in}{2.013280in}}{\pgfqpoint{2.695701in}{2.005380in}}{\pgfqpoint{2.701525in}{1.999556in}}%
\pgfpathcurveto{\pgfqpoint{2.707349in}{1.993732in}}{\pgfqpoint{2.715249in}{1.990460in}}{\pgfqpoint{2.723485in}{1.990460in}}%
\pgfpathclose%
\pgfusepath{stroke,fill}%
\end{pgfscope}%
\begin{pgfscope}%
\pgfpathrectangle{\pgfqpoint{0.100000in}{0.212622in}}{\pgfqpoint{3.696000in}{3.696000in}}%
\pgfusepath{clip}%
\pgfsetbuttcap%
\pgfsetroundjoin%
\definecolor{currentfill}{rgb}{0.121569,0.466667,0.705882}%
\pgfsetfillcolor{currentfill}%
\pgfsetfillopacity{0.858819}%
\pgfsetlinewidth{1.003750pt}%
\definecolor{currentstroke}{rgb}{0.121569,0.466667,0.705882}%
\pgfsetstrokecolor{currentstroke}%
\pgfsetstrokeopacity{0.858819}%
\pgfsetdash{}{0pt}%
\pgfpathmoveto{\pgfqpoint{2.720689in}{1.989203in}}%
\pgfpathcurveto{\pgfqpoint{2.728925in}{1.989203in}}{\pgfqpoint{2.736825in}{1.992475in}}{\pgfqpoint{2.742649in}{1.998299in}}%
\pgfpathcurveto{\pgfqpoint{2.748473in}{2.004123in}}{\pgfqpoint{2.751745in}{2.012023in}}{\pgfqpoint{2.751745in}{2.020259in}}%
\pgfpathcurveto{\pgfqpoint{2.751745in}{2.028496in}}{\pgfqpoint{2.748473in}{2.036396in}}{\pgfqpoint{2.742649in}{2.042220in}}%
\pgfpathcurveto{\pgfqpoint{2.736825in}{2.048044in}}{\pgfqpoint{2.728925in}{2.051316in}}{\pgfqpoint{2.720689in}{2.051316in}}%
\pgfpathcurveto{\pgfqpoint{2.712452in}{2.051316in}}{\pgfqpoint{2.704552in}{2.048044in}}{\pgfqpoint{2.698729in}{2.042220in}}%
\pgfpathcurveto{\pgfqpoint{2.692905in}{2.036396in}}{\pgfqpoint{2.689632in}{2.028496in}}{\pgfqpoint{2.689632in}{2.020259in}}%
\pgfpathcurveto{\pgfqpoint{2.689632in}{2.012023in}}{\pgfqpoint{2.692905in}{2.004123in}}{\pgfqpoint{2.698729in}{1.998299in}}%
\pgfpathcurveto{\pgfqpoint{2.704552in}{1.992475in}}{\pgfqpoint{2.712452in}{1.989203in}}{\pgfqpoint{2.720689in}{1.989203in}}%
\pgfpathclose%
\pgfusepath{stroke,fill}%
\end{pgfscope}%
\begin{pgfscope}%
\pgfpathrectangle{\pgfqpoint{0.100000in}{0.212622in}}{\pgfqpoint{3.696000in}{3.696000in}}%
\pgfusepath{clip}%
\pgfsetbuttcap%
\pgfsetroundjoin%
\definecolor{currentfill}{rgb}{0.121569,0.466667,0.705882}%
\pgfsetfillcolor{currentfill}%
\pgfsetfillopacity{0.859122}%
\pgfsetlinewidth{1.003750pt}%
\definecolor{currentstroke}{rgb}{0.121569,0.466667,0.705882}%
\pgfsetstrokecolor{currentstroke}%
\pgfsetstrokeopacity{0.859122}%
\pgfsetdash{}{0pt}%
\pgfpathmoveto{\pgfqpoint{1.183636in}{2.415204in}}%
\pgfpathcurveto{\pgfqpoint{1.191873in}{2.415204in}}{\pgfqpoint{1.199773in}{2.418477in}}{\pgfqpoint{1.205597in}{2.424301in}}%
\pgfpathcurveto{\pgfqpoint{1.211421in}{2.430125in}}{\pgfqpoint{1.214693in}{2.438025in}}{\pgfqpoint{1.214693in}{2.446261in}}%
\pgfpathcurveto{\pgfqpoint{1.214693in}{2.454497in}}{\pgfqpoint{1.211421in}{2.462397in}}{\pgfqpoint{1.205597in}{2.468221in}}%
\pgfpathcurveto{\pgfqpoint{1.199773in}{2.474045in}}{\pgfqpoint{1.191873in}{2.477317in}}{\pgfqpoint{1.183636in}{2.477317in}}%
\pgfpathcurveto{\pgfqpoint{1.175400in}{2.477317in}}{\pgfqpoint{1.167500in}{2.474045in}}{\pgfqpoint{1.161676in}{2.468221in}}%
\pgfpathcurveto{\pgfqpoint{1.155852in}{2.462397in}}{\pgfqpoint{1.152580in}{2.454497in}}{\pgfqpoint{1.152580in}{2.446261in}}%
\pgfpathcurveto{\pgfqpoint{1.152580in}{2.438025in}}{\pgfqpoint{1.155852in}{2.430125in}}{\pgfqpoint{1.161676in}{2.424301in}}%
\pgfpathcurveto{\pgfqpoint{1.167500in}{2.418477in}}{\pgfqpoint{1.175400in}{2.415204in}}{\pgfqpoint{1.183636in}{2.415204in}}%
\pgfpathclose%
\pgfusepath{stroke,fill}%
\end{pgfscope}%
\begin{pgfscope}%
\pgfpathrectangle{\pgfqpoint{0.100000in}{0.212622in}}{\pgfqpoint{3.696000in}{3.696000in}}%
\pgfusepath{clip}%
\pgfsetbuttcap%
\pgfsetroundjoin%
\definecolor{currentfill}{rgb}{0.121569,0.466667,0.705882}%
\pgfsetfillcolor{currentfill}%
\pgfsetfillopacity{0.859557}%
\pgfsetlinewidth{1.003750pt}%
\definecolor{currentstroke}{rgb}{0.121569,0.466667,0.705882}%
\pgfsetstrokecolor{currentstroke}%
\pgfsetstrokeopacity{0.859557}%
\pgfsetdash{}{0pt}%
\pgfpathmoveto{\pgfqpoint{2.719155in}{1.989104in}}%
\pgfpathcurveto{\pgfqpoint{2.727391in}{1.989104in}}{\pgfqpoint{2.735291in}{1.992376in}}{\pgfqpoint{2.741115in}{1.998200in}}%
\pgfpathcurveto{\pgfqpoint{2.746939in}{2.004024in}}{\pgfqpoint{2.750211in}{2.011924in}}{\pgfqpoint{2.750211in}{2.020161in}}%
\pgfpathcurveto{\pgfqpoint{2.750211in}{2.028397in}}{\pgfqpoint{2.746939in}{2.036297in}}{\pgfqpoint{2.741115in}{2.042121in}}%
\pgfpathcurveto{\pgfqpoint{2.735291in}{2.047945in}}{\pgfqpoint{2.727391in}{2.051217in}}{\pgfqpoint{2.719155in}{2.051217in}}%
\pgfpathcurveto{\pgfqpoint{2.710918in}{2.051217in}}{\pgfqpoint{2.703018in}{2.047945in}}{\pgfqpoint{2.697194in}{2.042121in}}%
\pgfpathcurveto{\pgfqpoint{2.691370in}{2.036297in}}{\pgfqpoint{2.688098in}{2.028397in}}{\pgfqpoint{2.688098in}{2.020161in}}%
\pgfpathcurveto{\pgfqpoint{2.688098in}{2.011924in}}{\pgfqpoint{2.691370in}{2.004024in}}{\pgfqpoint{2.697194in}{1.998200in}}%
\pgfpathcurveto{\pgfqpoint{2.703018in}{1.992376in}}{\pgfqpoint{2.710918in}{1.989104in}}{\pgfqpoint{2.719155in}{1.989104in}}%
\pgfpathclose%
\pgfusepath{stroke,fill}%
\end{pgfscope}%
\begin{pgfscope}%
\pgfpathrectangle{\pgfqpoint{0.100000in}{0.212622in}}{\pgfqpoint{3.696000in}{3.696000in}}%
\pgfusepath{clip}%
\pgfsetbuttcap%
\pgfsetroundjoin%
\definecolor{currentfill}{rgb}{0.121569,0.466667,0.705882}%
\pgfsetfillcolor{currentfill}%
\pgfsetfillopacity{0.859899}%
\pgfsetlinewidth{1.003750pt}%
\definecolor{currentstroke}{rgb}{0.121569,0.466667,0.705882}%
\pgfsetstrokecolor{currentstroke}%
\pgfsetstrokeopacity{0.859899}%
\pgfsetdash{}{0pt}%
\pgfpathmoveto{\pgfqpoint{2.718516in}{1.988405in}}%
\pgfpathcurveto{\pgfqpoint{2.726752in}{1.988405in}}{\pgfqpoint{2.734652in}{1.991677in}}{\pgfqpoint{2.740476in}{1.997501in}}%
\pgfpathcurveto{\pgfqpoint{2.746300in}{2.003325in}}{\pgfqpoint{2.749572in}{2.011225in}}{\pgfqpoint{2.749572in}{2.019461in}}%
\pgfpathcurveto{\pgfqpoint{2.749572in}{2.027698in}}{\pgfqpoint{2.746300in}{2.035598in}}{\pgfqpoint{2.740476in}{2.041422in}}%
\pgfpathcurveto{\pgfqpoint{2.734652in}{2.047246in}}{\pgfqpoint{2.726752in}{2.050518in}}{\pgfqpoint{2.718516in}{2.050518in}}%
\pgfpathcurveto{\pgfqpoint{2.710280in}{2.050518in}}{\pgfqpoint{2.702380in}{2.047246in}}{\pgfqpoint{2.696556in}{2.041422in}}%
\pgfpathcurveto{\pgfqpoint{2.690732in}{2.035598in}}{\pgfqpoint{2.687459in}{2.027698in}}{\pgfqpoint{2.687459in}{2.019461in}}%
\pgfpathcurveto{\pgfqpoint{2.687459in}{2.011225in}}{\pgfqpoint{2.690732in}{2.003325in}}{\pgfqpoint{2.696556in}{1.997501in}}%
\pgfpathcurveto{\pgfqpoint{2.702380in}{1.991677in}}{\pgfqpoint{2.710280in}{1.988405in}}{\pgfqpoint{2.718516in}{1.988405in}}%
\pgfpathclose%
\pgfusepath{stroke,fill}%
\end{pgfscope}%
\begin{pgfscope}%
\pgfpathrectangle{\pgfqpoint{0.100000in}{0.212622in}}{\pgfqpoint{3.696000in}{3.696000in}}%
\pgfusepath{clip}%
\pgfsetbuttcap%
\pgfsetroundjoin%
\definecolor{currentfill}{rgb}{0.121569,0.466667,0.705882}%
\pgfsetfillcolor{currentfill}%
\pgfsetfillopacity{0.860107}%
\pgfsetlinewidth{1.003750pt}%
\definecolor{currentstroke}{rgb}{0.121569,0.466667,0.705882}%
\pgfsetstrokecolor{currentstroke}%
\pgfsetstrokeopacity{0.860107}%
\pgfsetdash{}{0pt}%
\pgfpathmoveto{\pgfqpoint{2.718074in}{1.988251in}}%
\pgfpathcurveto{\pgfqpoint{2.726310in}{1.988251in}}{\pgfqpoint{2.734210in}{1.991524in}}{\pgfqpoint{2.740034in}{1.997347in}}%
\pgfpathcurveto{\pgfqpoint{2.745858in}{2.003171in}}{\pgfqpoint{2.749130in}{2.011071in}}{\pgfqpoint{2.749130in}{2.019308in}}%
\pgfpathcurveto{\pgfqpoint{2.749130in}{2.027544in}}{\pgfqpoint{2.745858in}{2.035444in}}{\pgfqpoint{2.740034in}{2.041268in}}%
\pgfpathcurveto{\pgfqpoint{2.734210in}{2.047092in}}{\pgfqpoint{2.726310in}{2.050364in}}{\pgfqpoint{2.718074in}{2.050364in}}%
\pgfpathcurveto{\pgfqpoint{2.709837in}{2.050364in}}{\pgfqpoint{2.701937in}{2.047092in}}{\pgfqpoint{2.696113in}{2.041268in}}%
\pgfpathcurveto{\pgfqpoint{2.690289in}{2.035444in}}{\pgfqpoint{2.687017in}{2.027544in}}{\pgfqpoint{2.687017in}{2.019308in}}%
\pgfpathcurveto{\pgfqpoint{2.687017in}{2.011071in}}{\pgfqpoint{2.690289in}{2.003171in}}{\pgfqpoint{2.696113in}{1.997347in}}%
\pgfpathcurveto{\pgfqpoint{2.701937in}{1.991524in}}{\pgfqpoint{2.709837in}{1.988251in}}{\pgfqpoint{2.718074in}{1.988251in}}%
\pgfpathclose%
\pgfusepath{stroke,fill}%
\end{pgfscope}%
\begin{pgfscope}%
\pgfpathrectangle{\pgfqpoint{0.100000in}{0.212622in}}{\pgfqpoint{3.696000in}{3.696000in}}%
\pgfusepath{clip}%
\pgfsetbuttcap%
\pgfsetroundjoin%
\definecolor{currentfill}{rgb}{0.121569,0.466667,0.705882}%
\pgfsetfillcolor{currentfill}%
\pgfsetfillopacity{0.861490}%
\pgfsetlinewidth{1.003750pt}%
\definecolor{currentstroke}{rgb}{0.121569,0.466667,0.705882}%
\pgfsetstrokecolor{currentstroke}%
\pgfsetstrokeopacity{0.861490}%
\pgfsetdash{}{0pt}%
\pgfpathmoveto{\pgfqpoint{2.715126in}{1.984593in}}%
\pgfpathcurveto{\pgfqpoint{2.723362in}{1.984593in}}{\pgfqpoint{2.731262in}{1.987866in}}{\pgfqpoint{2.737086in}{1.993690in}}%
\pgfpathcurveto{\pgfqpoint{2.742910in}{1.999514in}}{\pgfqpoint{2.746182in}{2.007414in}}{\pgfqpoint{2.746182in}{2.015650in}}%
\pgfpathcurveto{\pgfqpoint{2.746182in}{2.023886in}}{\pgfqpoint{2.742910in}{2.031786in}}{\pgfqpoint{2.737086in}{2.037610in}}%
\pgfpathcurveto{\pgfqpoint{2.731262in}{2.043434in}}{\pgfqpoint{2.723362in}{2.046706in}}{\pgfqpoint{2.715126in}{2.046706in}}%
\pgfpathcurveto{\pgfqpoint{2.706890in}{2.046706in}}{\pgfqpoint{2.698990in}{2.043434in}}{\pgfqpoint{2.693166in}{2.037610in}}%
\pgfpathcurveto{\pgfqpoint{2.687342in}{2.031786in}}{\pgfqpoint{2.684069in}{2.023886in}}{\pgfqpoint{2.684069in}{2.015650in}}%
\pgfpathcurveto{\pgfqpoint{2.684069in}{2.007414in}}{\pgfqpoint{2.687342in}{1.999514in}}{\pgfqpoint{2.693166in}{1.993690in}}%
\pgfpathcurveto{\pgfqpoint{2.698990in}{1.987866in}}{\pgfqpoint{2.706890in}{1.984593in}}{\pgfqpoint{2.715126in}{1.984593in}}%
\pgfpathclose%
\pgfusepath{stroke,fill}%
\end{pgfscope}%
\begin{pgfscope}%
\pgfpathrectangle{\pgfqpoint{0.100000in}{0.212622in}}{\pgfqpoint{3.696000in}{3.696000in}}%
\pgfusepath{clip}%
\pgfsetbuttcap%
\pgfsetroundjoin%
\definecolor{currentfill}{rgb}{0.121569,0.466667,0.705882}%
\pgfsetfillcolor{currentfill}%
\pgfsetfillopacity{0.861914}%
\pgfsetlinewidth{1.003750pt}%
\definecolor{currentstroke}{rgb}{0.121569,0.466667,0.705882}%
\pgfsetstrokecolor{currentstroke}%
\pgfsetstrokeopacity{0.861914}%
\pgfsetdash{}{0pt}%
\pgfpathmoveto{\pgfqpoint{1.204895in}{2.406125in}}%
\pgfpathcurveto{\pgfqpoint{1.213132in}{2.406125in}}{\pgfqpoint{1.221032in}{2.409397in}}{\pgfqpoint{1.226856in}{2.415221in}}%
\pgfpathcurveto{\pgfqpoint{1.232679in}{2.421045in}}{\pgfqpoint{1.235952in}{2.428945in}}{\pgfqpoint{1.235952in}{2.437182in}}%
\pgfpathcurveto{\pgfqpoint{1.235952in}{2.445418in}}{\pgfqpoint{1.232679in}{2.453318in}}{\pgfqpoint{1.226856in}{2.459142in}}%
\pgfpathcurveto{\pgfqpoint{1.221032in}{2.464966in}}{\pgfqpoint{1.213132in}{2.468238in}}{\pgfqpoint{1.204895in}{2.468238in}}%
\pgfpathcurveto{\pgfqpoint{1.196659in}{2.468238in}}{\pgfqpoint{1.188759in}{2.464966in}}{\pgfqpoint{1.182935in}{2.459142in}}%
\pgfpathcurveto{\pgfqpoint{1.177111in}{2.453318in}}{\pgfqpoint{1.173839in}{2.445418in}}{\pgfqpoint{1.173839in}{2.437182in}}%
\pgfpathcurveto{\pgfqpoint{1.173839in}{2.428945in}}{\pgfqpoint{1.177111in}{2.421045in}}{\pgfqpoint{1.182935in}{2.415221in}}%
\pgfpathcurveto{\pgfqpoint{1.188759in}{2.409397in}}{\pgfqpoint{1.196659in}{2.406125in}}{\pgfqpoint{1.204895in}{2.406125in}}%
\pgfpathclose%
\pgfusepath{stroke,fill}%
\end{pgfscope}%
\begin{pgfscope}%
\pgfpathrectangle{\pgfqpoint{0.100000in}{0.212622in}}{\pgfqpoint{3.696000in}{3.696000in}}%
\pgfusepath{clip}%
\pgfsetbuttcap%
\pgfsetroundjoin%
\definecolor{currentfill}{rgb}{0.121569,0.466667,0.705882}%
\pgfsetfillcolor{currentfill}%
\pgfsetfillopacity{0.862734}%
\pgfsetlinewidth{1.003750pt}%
\definecolor{currentstroke}{rgb}{0.121569,0.466667,0.705882}%
\pgfsetstrokecolor{currentstroke}%
\pgfsetstrokeopacity{0.862734}%
\pgfsetdash{}{0pt}%
\pgfpathmoveto{\pgfqpoint{1.225639in}{2.395037in}}%
\pgfpathcurveto{\pgfqpoint{1.233875in}{2.395037in}}{\pgfqpoint{1.241775in}{2.398310in}}{\pgfqpoint{1.247599in}{2.404134in}}%
\pgfpathcurveto{\pgfqpoint{1.253423in}{2.409957in}}{\pgfqpoint{1.256695in}{2.417857in}}{\pgfqpoint{1.256695in}{2.426094in}}%
\pgfpathcurveto{\pgfqpoint{1.256695in}{2.434330in}}{\pgfqpoint{1.253423in}{2.442230in}}{\pgfqpoint{1.247599in}{2.448054in}}%
\pgfpathcurveto{\pgfqpoint{1.241775in}{2.453878in}}{\pgfqpoint{1.233875in}{2.457150in}}{\pgfqpoint{1.225639in}{2.457150in}}%
\pgfpathcurveto{\pgfqpoint{1.217402in}{2.457150in}}{\pgfqpoint{1.209502in}{2.453878in}}{\pgfqpoint{1.203678in}{2.448054in}}%
\pgfpathcurveto{\pgfqpoint{1.197854in}{2.442230in}}{\pgfqpoint{1.194582in}{2.434330in}}{\pgfqpoint{1.194582in}{2.426094in}}%
\pgfpathcurveto{\pgfqpoint{1.194582in}{2.417857in}}{\pgfqpoint{1.197854in}{2.409957in}}{\pgfqpoint{1.203678in}{2.404134in}}%
\pgfpathcurveto{\pgfqpoint{1.209502in}{2.398310in}}{\pgfqpoint{1.217402in}{2.395037in}}{\pgfqpoint{1.225639in}{2.395037in}}%
\pgfpathclose%
\pgfusepath{stroke,fill}%
\end{pgfscope}%
\begin{pgfscope}%
\pgfpathrectangle{\pgfqpoint{0.100000in}{0.212622in}}{\pgfqpoint{3.696000in}{3.696000in}}%
\pgfusepath{clip}%
\pgfsetbuttcap%
\pgfsetroundjoin%
\definecolor{currentfill}{rgb}{0.121569,0.466667,0.705882}%
\pgfsetfillcolor{currentfill}%
\pgfsetfillopacity{0.864823}%
\pgfsetlinewidth{1.003750pt}%
\definecolor{currentstroke}{rgb}{0.121569,0.466667,0.705882}%
\pgfsetstrokecolor{currentstroke}%
\pgfsetstrokeopacity{0.864823}%
\pgfsetdash{}{0pt}%
\pgfpathmoveto{\pgfqpoint{2.710125in}{1.984868in}}%
\pgfpathcurveto{\pgfqpoint{2.718362in}{1.984868in}}{\pgfqpoint{2.726262in}{1.988140in}}{\pgfqpoint{2.732086in}{1.993964in}}%
\pgfpathcurveto{\pgfqpoint{2.737910in}{1.999788in}}{\pgfqpoint{2.741182in}{2.007688in}}{\pgfqpoint{2.741182in}{2.015924in}}%
\pgfpathcurveto{\pgfqpoint{2.741182in}{2.024161in}}{\pgfqpoint{2.737910in}{2.032061in}}{\pgfqpoint{2.732086in}{2.037885in}}%
\pgfpathcurveto{\pgfqpoint{2.726262in}{2.043708in}}{\pgfqpoint{2.718362in}{2.046981in}}{\pgfqpoint{2.710125in}{2.046981in}}%
\pgfpathcurveto{\pgfqpoint{2.701889in}{2.046981in}}{\pgfqpoint{2.693989in}{2.043708in}}{\pgfqpoint{2.688165in}{2.037885in}}%
\pgfpathcurveto{\pgfqpoint{2.682341in}{2.032061in}}{\pgfqpoint{2.679069in}{2.024161in}}{\pgfqpoint{2.679069in}{2.015924in}}%
\pgfpathcurveto{\pgfqpoint{2.679069in}{2.007688in}}{\pgfqpoint{2.682341in}{1.999788in}}{\pgfqpoint{2.688165in}{1.993964in}}%
\pgfpathcurveto{\pgfqpoint{2.693989in}{1.988140in}}{\pgfqpoint{2.701889in}{1.984868in}}{\pgfqpoint{2.710125in}{1.984868in}}%
\pgfpathclose%
\pgfusepath{stroke,fill}%
\end{pgfscope}%
\begin{pgfscope}%
\pgfpathrectangle{\pgfqpoint{0.100000in}{0.212622in}}{\pgfqpoint{3.696000in}{3.696000in}}%
\pgfusepath{clip}%
\pgfsetbuttcap%
\pgfsetroundjoin%
\definecolor{currentfill}{rgb}{0.121569,0.466667,0.705882}%
\pgfsetfillcolor{currentfill}%
\pgfsetfillopacity{0.865345}%
\pgfsetlinewidth{1.003750pt}%
\definecolor{currentstroke}{rgb}{0.121569,0.466667,0.705882}%
\pgfsetstrokecolor{currentstroke}%
\pgfsetstrokeopacity{0.865345}%
\pgfsetdash{}{0pt}%
\pgfpathmoveto{\pgfqpoint{1.241896in}{2.389533in}}%
\pgfpathcurveto{\pgfqpoint{1.250132in}{2.389533in}}{\pgfqpoint{1.258032in}{2.392806in}}{\pgfqpoint{1.263856in}{2.398630in}}%
\pgfpathcurveto{\pgfqpoint{1.269680in}{2.404453in}}{\pgfqpoint{1.272952in}{2.412354in}}{\pgfqpoint{1.272952in}{2.420590in}}%
\pgfpathcurveto{\pgfqpoint{1.272952in}{2.428826in}}{\pgfqpoint{1.269680in}{2.436726in}}{\pgfqpoint{1.263856in}{2.442550in}}%
\pgfpathcurveto{\pgfqpoint{1.258032in}{2.448374in}}{\pgfqpoint{1.250132in}{2.451646in}}{\pgfqpoint{1.241896in}{2.451646in}}%
\pgfpathcurveto{\pgfqpoint{1.233659in}{2.451646in}}{\pgfqpoint{1.225759in}{2.448374in}}{\pgfqpoint{1.219935in}{2.442550in}}%
\pgfpathcurveto{\pgfqpoint{1.214111in}{2.436726in}}{\pgfqpoint{1.210839in}{2.428826in}}{\pgfqpoint{1.210839in}{2.420590in}}%
\pgfpathcurveto{\pgfqpoint{1.210839in}{2.412354in}}{\pgfqpoint{1.214111in}{2.404453in}}{\pgfqpoint{1.219935in}{2.398630in}}%
\pgfpathcurveto{\pgfqpoint{1.225759in}{2.392806in}}{\pgfqpoint{1.233659in}{2.389533in}}{\pgfqpoint{1.241896in}{2.389533in}}%
\pgfpathclose%
\pgfusepath{stroke,fill}%
\end{pgfscope}%
\begin{pgfscope}%
\pgfpathrectangle{\pgfqpoint{0.100000in}{0.212622in}}{\pgfqpoint{3.696000in}{3.696000in}}%
\pgfusepath{clip}%
\pgfsetbuttcap%
\pgfsetroundjoin%
\definecolor{currentfill}{rgb}{0.121569,0.466667,0.705882}%
\pgfsetfillcolor{currentfill}%
\pgfsetfillopacity{0.866167}%
\pgfsetlinewidth{1.003750pt}%
\definecolor{currentstroke}{rgb}{0.121569,0.466667,0.705882}%
\pgfsetstrokecolor{currentstroke}%
\pgfsetstrokeopacity{0.866167}%
\pgfsetdash{}{0pt}%
\pgfpathmoveto{\pgfqpoint{1.253137in}{2.383866in}}%
\pgfpathcurveto{\pgfqpoint{1.261374in}{2.383866in}}{\pgfqpoint{1.269274in}{2.387138in}}{\pgfqpoint{1.275098in}{2.392962in}}%
\pgfpathcurveto{\pgfqpoint{1.280922in}{2.398786in}}{\pgfqpoint{1.284194in}{2.406686in}}{\pgfqpoint{1.284194in}{2.414922in}}%
\pgfpathcurveto{\pgfqpoint{1.284194in}{2.423158in}}{\pgfqpoint{1.280922in}{2.431058in}}{\pgfqpoint{1.275098in}{2.436882in}}%
\pgfpathcurveto{\pgfqpoint{1.269274in}{2.442706in}}{\pgfqpoint{1.261374in}{2.445979in}}{\pgfqpoint{1.253137in}{2.445979in}}%
\pgfpathcurveto{\pgfqpoint{1.244901in}{2.445979in}}{\pgfqpoint{1.237001in}{2.442706in}}{\pgfqpoint{1.231177in}{2.436882in}}%
\pgfpathcurveto{\pgfqpoint{1.225353in}{2.431058in}}{\pgfqpoint{1.222081in}{2.423158in}}{\pgfqpoint{1.222081in}{2.414922in}}%
\pgfpathcurveto{\pgfqpoint{1.222081in}{2.406686in}}{\pgfqpoint{1.225353in}{2.398786in}}{\pgfqpoint{1.231177in}{2.392962in}}%
\pgfpathcurveto{\pgfqpoint{1.237001in}{2.387138in}}{\pgfqpoint{1.244901in}{2.383866in}}{\pgfqpoint{1.253137in}{2.383866in}}%
\pgfpathclose%
\pgfusepath{stroke,fill}%
\end{pgfscope}%
\begin{pgfscope}%
\pgfpathrectangle{\pgfqpoint{0.100000in}{0.212622in}}{\pgfqpoint{3.696000in}{3.696000in}}%
\pgfusepath{clip}%
\pgfsetbuttcap%
\pgfsetroundjoin%
\definecolor{currentfill}{rgb}{0.121569,0.466667,0.705882}%
\pgfsetfillcolor{currentfill}%
\pgfsetfillopacity{0.866796}%
\pgfsetlinewidth{1.003750pt}%
\definecolor{currentstroke}{rgb}{0.121569,0.466667,0.705882}%
\pgfsetstrokecolor{currentstroke}%
\pgfsetstrokeopacity{0.866796}%
\pgfsetdash{}{0pt}%
\pgfpathmoveto{\pgfqpoint{1.258229in}{2.382412in}}%
\pgfpathcurveto{\pgfqpoint{1.266466in}{2.382412in}}{\pgfqpoint{1.274366in}{2.385684in}}{\pgfqpoint{1.280190in}{2.391508in}}%
\pgfpathcurveto{\pgfqpoint{1.286014in}{2.397332in}}{\pgfqpoint{1.289286in}{2.405232in}}{\pgfqpoint{1.289286in}{2.413468in}}%
\pgfpathcurveto{\pgfqpoint{1.289286in}{2.421705in}}{\pgfqpoint{1.286014in}{2.429605in}}{\pgfqpoint{1.280190in}{2.435429in}}%
\pgfpathcurveto{\pgfqpoint{1.274366in}{2.441253in}}{\pgfqpoint{1.266466in}{2.444525in}}{\pgfqpoint{1.258229in}{2.444525in}}%
\pgfpathcurveto{\pgfqpoint{1.249993in}{2.444525in}}{\pgfqpoint{1.242093in}{2.441253in}}{\pgfqpoint{1.236269in}{2.435429in}}%
\pgfpathcurveto{\pgfqpoint{1.230445in}{2.429605in}}{\pgfqpoint{1.227173in}{2.421705in}}{\pgfqpoint{1.227173in}{2.413468in}}%
\pgfpathcurveto{\pgfqpoint{1.227173in}{2.405232in}}{\pgfqpoint{1.230445in}{2.397332in}}{\pgfqpoint{1.236269in}{2.391508in}}%
\pgfpathcurveto{\pgfqpoint{1.242093in}{2.385684in}}{\pgfqpoint{1.249993in}{2.382412in}}{\pgfqpoint{1.258229in}{2.382412in}}%
\pgfpathclose%
\pgfusepath{stroke,fill}%
\end{pgfscope}%
\begin{pgfscope}%
\pgfpathrectangle{\pgfqpoint{0.100000in}{0.212622in}}{\pgfqpoint{3.696000in}{3.696000in}}%
\pgfusepath{clip}%
\pgfsetbuttcap%
\pgfsetroundjoin%
\definecolor{currentfill}{rgb}{0.121569,0.466667,0.705882}%
\pgfsetfillcolor{currentfill}%
\pgfsetfillopacity{0.866850}%
\pgfsetlinewidth{1.003750pt}%
\definecolor{currentstroke}{rgb}{0.121569,0.466667,0.705882}%
\pgfsetstrokecolor{currentstroke}%
\pgfsetstrokeopacity{0.866850}%
\pgfsetdash{}{0pt}%
\pgfpathmoveto{\pgfqpoint{1.259359in}{2.381832in}}%
\pgfpathcurveto{\pgfqpoint{1.267595in}{2.381832in}}{\pgfqpoint{1.275495in}{2.385105in}}{\pgfqpoint{1.281319in}{2.390929in}}%
\pgfpathcurveto{\pgfqpoint{1.287143in}{2.396753in}}{\pgfqpoint{1.290416in}{2.404653in}}{\pgfqpoint{1.290416in}{2.412889in}}%
\pgfpathcurveto{\pgfqpoint{1.290416in}{2.421125in}}{\pgfqpoint{1.287143in}{2.429025in}}{\pgfqpoint{1.281319in}{2.434849in}}%
\pgfpathcurveto{\pgfqpoint{1.275495in}{2.440673in}}{\pgfqpoint{1.267595in}{2.443945in}}{\pgfqpoint{1.259359in}{2.443945in}}%
\pgfpathcurveto{\pgfqpoint{1.251123in}{2.443945in}}{\pgfqpoint{1.243223in}{2.440673in}}{\pgfqpoint{1.237399in}{2.434849in}}%
\pgfpathcurveto{\pgfqpoint{1.231575in}{2.429025in}}{\pgfqpoint{1.228303in}{2.421125in}}{\pgfqpoint{1.228303in}{2.412889in}}%
\pgfpathcurveto{\pgfqpoint{1.228303in}{2.404653in}}{\pgfqpoint{1.231575in}{2.396753in}}{\pgfqpoint{1.237399in}{2.390929in}}%
\pgfpathcurveto{\pgfqpoint{1.243223in}{2.385105in}}{\pgfqpoint{1.251123in}{2.381832in}}{\pgfqpoint{1.259359in}{2.381832in}}%
\pgfpathclose%
\pgfusepath{stroke,fill}%
\end{pgfscope}%
\begin{pgfscope}%
\pgfpathrectangle{\pgfqpoint{0.100000in}{0.212622in}}{\pgfqpoint{3.696000in}{3.696000in}}%
\pgfusepath{clip}%
\pgfsetbuttcap%
\pgfsetroundjoin%
\definecolor{currentfill}{rgb}{0.121569,0.466667,0.705882}%
\pgfsetfillcolor{currentfill}%
\pgfsetfillopacity{0.867177}%
\pgfsetlinewidth{1.003750pt}%
\definecolor{currentstroke}{rgb}{0.121569,0.466667,0.705882}%
\pgfsetstrokecolor{currentstroke}%
\pgfsetstrokeopacity{0.867177}%
\pgfsetdash{}{0pt}%
\pgfpathmoveto{\pgfqpoint{1.261205in}{2.381562in}}%
\pgfpathcurveto{\pgfqpoint{1.269442in}{2.381562in}}{\pgfqpoint{1.277342in}{2.384834in}}{\pgfqpoint{1.283166in}{2.390658in}}%
\pgfpathcurveto{\pgfqpoint{1.288989in}{2.396482in}}{\pgfqpoint{1.292262in}{2.404382in}}{\pgfqpoint{1.292262in}{2.412618in}}%
\pgfpathcurveto{\pgfqpoint{1.292262in}{2.420854in}}{\pgfqpoint{1.288989in}{2.428754in}}{\pgfqpoint{1.283166in}{2.434578in}}%
\pgfpathcurveto{\pgfqpoint{1.277342in}{2.440402in}}{\pgfqpoint{1.269442in}{2.443675in}}{\pgfqpoint{1.261205in}{2.443675in}}%
\pgfpathcurveto{\pgfqpoint{1.252969in}{2.443675in}}{\pgfqpoint{1.245069in}{2.440402in}}{\pgfqpoint{1.239245in}{2.434578in}}%
\pgfpathcurveto{\pgfqpoint{1.233421in}{2.428754in}}{\pgfqpoint{1.230149in}{2.420854in}}{\pgfqpoint{1.230149in}{2.412618in}}%
\pgfpathcurveto{\pgfqpoint{1.230149in}{2.404382in}}{\pgfqpoint{1.233421in}{2.396482in}}{\pgfqpoint{1.239245in}{2.390658in}}%
\pgfpathcurveto{\pgfqpoint{1.245069in}{2.384834in}}{\pgfqpoint{1.252969in}{2.381562in}}{\pgfqpoint{1.261205in}{2.381562in}}%
\pgfpathclose%
\pgfusepath{stroke,fill}%
\end{pgfscope}%
\begin{pgfscope}%
\pgfpathrectangle{\pgfqpoint{0.100000in}{0.212622in}}{\pgfqpoint{3.696000in}{3.696000in}}%
\pgfusepath{clip}%
\pgfsetbuttcap%
\pgfsetroundjoin%
\definecolor{currentfill}{rgb}{0.121569,0.466667,0.705882}%
\pgfsetfillcolor{currentfill}%
\pgfsetfillopacity{0.867250}%
\pgfsetlinewidth{1.003750pt}%
\definecolor{currentstroke}{rgb}{0.121569,0.466667,0.705882}%
\pgfsetstrokecolor{currentstroke}%
\pgfsetstrokeopacity{0.867250}%
\pgfsetdash{}{0pt}%
\pgfpathmoveto{\pgfqpoint{1.264901in}{2.378765in}}%
\pgfpathcurveto{\pgfqpoint{1.273137in}{2.378765in}}{\pgfqpoint{1.281037in}{2.382037in}}{\pgfqpoint{1.286861in}{2.387861in}}%
\pgfpathcurveto{\pgfqpoint{1.292685in}{2.393685in}}{\pgfqpoint{1.295958in}{2.401585in}}{\pgfqpoint{1.295958in}{2.409821in}}%
\pgfpathcurveto{\pgfqpoint{1.295958in}{2.418057in}}{\pgfqpoint{1.292685in}{2.425957in}}{\pgfqpoint{1.286861in}{2.431781in}}%
\pgfpathcurveto{\pgfqpoint{1.281037in}{2.437605in}}{\pgfqpoint{1.273137in}{2.440878in}}{\pgfqpoint{1.264901in}{2.440878in}}%
\pgfpathcurveto{\pgfqpoint{1.256665in}{2.440878in}}{\pgfqpoint{1.248765in}{2.437605in}}{\pgfqpoint{1.242941in}{2.431781in}}%
\pgfpathcurveto{\pgfqpoint{1.237117in}{2.425957in}}{\pgfqpoint{1.233845in}{2.418057in}}{\pgfqpoint{1.233845in}{2.409821in}}%
\pgfpathcurveto{\pgfqpoint{1.233845in}{2.401585in}}{\pgfqpoint{1.237117in}{2.393685in}}{\pgfqpoint{1.242941in}{2.387861in}}%
\pgfpathcurveto{\pgfqpoint{1.248765in}{2.382037in}}{\pgfqpoint{1.256665in}{2.378765in}}{\pgfqpoint{1.264901in}{2.378765in}}%
\pgfpathclose%
\pgfusepath{stroke,fill}%
\end{pgfscope}%
\begin{pgfscope}%
\pgfpathrectangle{\pgfqpoint{0.100000in}{0.212622in}}{\pgfqpoint{3.696000in}{3.696000in}}%
\pgfusepath{clip}%
\pgfsetbuttcap%
\pgfsetroundjoin%
\definecolor{currentfill}{rgb}{0.121569,0.466667,0.705882}%
\pgfsetfillcolor{currentfill}%
\pgfsetfillopacity{0.868244}%
\pgfsetlinewidth{1.003750pt}%
\definecolor{currentstroke}{rgb}{0.121569,0.466667,0.705882}%
\pgfsetstrokecolor{currentstroke}%
\pgfsetstrokeopacity{0.868244}%
\pgfsetdash{}{0pt}%
\pgfpathmoveto{\pgfqpoint{1.271026in}{2.377265in}}%
\pgfpathcurveto{\pgfqpoint{1.279262in}{2.377265in}}{\pgfqpoint{1.287162in}{2.380538in}}{\pgfqpoint{1.292986in}{2.386362in}}%
\pgfpathcurveto{\pgfqpoint{1.298810in}{2.392186in}}{\pgfqpoint{1.302083in}{2.400086in}}{\pgfqpoint{1.302083in}{2.408322in}}%
\pgfpathcurveto{\pgfqpoint{1.302083in}{2.416558in}}{\pgfqpoint{1.298810in}{2.424458in}}{\pgfqpoint{1.292986in}{2.430282in}}%
\pgfpathcurveto{\pgfqpoint{1.287162in}{2.436106in}}{\pgfqpoint{1.279262in}{2.439378in}}{\pgfqpoint{1.271026in}{2.439378in}}%
\pgfpathcurveto{\pgfqpoint{1.262790in}{2.439378in}}{\pgfqpoint{1.254890in}{2.436106in}}{\pgfqpoint{1.249066in}{2.430282in}}%
\pgfpathcurveto{\pgfqpoint{1.243242in}{2.424458in}}{\pgfqpoint{1.239970in}{2.416558in}}{\pgfqpoint{1.239970in}{2.408322in}}%
\pgfpathcurveto{\pgfqpoint{1.239970in}{2.400086in}}{\pgfqpoint{1.243242in}{2.392186in}}{\pgfqpoint{1.249066in}{2.386362in}}%
\pgfpathcurveto{\pgfqpoint{1.254890in}{2.380538in}}{\pgfqpoint{1.262790in}{2.377265in}}{\pgfqpoint{1.271026in}{2.377265in}}%
\pgfpathclose%
\pgfusepath{stroke,fill}%
\end{pgfscope}%
\begin{pgfscope}%
\pgfpathrectangle{\pgfqpoint{0.100000in}{0.212622in}}{\pgfqpoint{3.696000in}{3.696000in}}%
\pgfusepath{clip}%
\pgfsetbuttcap%
\pgfsetroundjoin%
\definecolor{currentfill}{rgb}{0.121569,0.466667,0.705882}%
\pgfsetfillcolor{currentfill}%
\pgfsetfillopacity{0.868801}%
\pgfsetlinewidth{1.003750pt}%
\definecolor{currentstroke}{rgb}{0.121569,0.466667,0.705882}%
\pgfsetstrokecolor{currentstroke}%
\pgfsetstrokeopacity{0.868801}%
\pgfsetdash{}{0pt}%
\pgfpathmoveto{\pgfqpoint{2.701946in}{1.973787in}}%
\pgfpathcurveto{\pgfqpoint{2.710182in}{1.973787in}}{\pgfqpoint{2.718082in}{1.977059in}}{\pgfqpoint{2.723906in}{1.982883in}}%
\pgfpathcurveto{\pgfqpoint{2.729730in}{1.988707in}}{\pgfqpoint{2.733002in}{1.996607in}}{\pgfqpoint{2.733002in}{2.004843in}}%
\pgfpathcurveto{\pgfqpoint{2.733002in}{2.013079in}}{\pgfqpoint{2.729730in}{2.020979in}}{\pgfqpoint{2.723906in}{2.026803in}}%
\pgfpathcurveto{\pgfqpoint{2.718082in}{2.032627in}}{\pgfqpoint{2.710182in}{2.035900in}}{\pgfqpoint{2.701946in}{2.035900in}}%
\pgfpathcurveto{\pgfqpoint{2.693710in}{2.035900in}}{\pgfqpoint{2.685810in}{2.032627in}}{\pgfqpoint{2.679986in}{2.026803in}}%
\pgfpathcurveto{\pgfqpoint{2.674162in}{2.020979in}}{\pgfqpoint{2.670889in}{2.013079in}}{\pgfqpoint{2.670889in}{2.004843in}}%
\pgfpathcurveto{\pgfqpoint{2.670889in}{1.996607in}}{\pgfqpoint{2.674162in}{1.988707in}}{\pgfqpoint{2.679986in}{1.982883in}}%
\pgfpathcurveto{\pgfqpoint{2.685810in}{1.977059in}}{\pgfqpoint{2.693710in}{1.973787in}}{\pgfqpoint{2.701946in}{1.973787in}}%
\pgfpathclose%
\pgfusepath{stroke,fill}%
\end{pgfscope}%
\begin{pgfscope}%
\pgfpathrectangle{\pgfqpoint{0.100000in}{0.212622in}}{\pgfqpoint{3.696000in}{3.696000in}}%
\pgfusepath{clip}%
\pgfsetbuttcap%
\pgfsetroundjoin%
\definecolor{currentfill}{rgb}{0.121569,0.466667,0.705882}%
\pgfsetfillcolor{currentfill}%
\pgfsetfillopacity{0.869377}%
\pgfsetlinewidth{1.003750pt}%
\definecolor{currentstroke}{rgb}{0.121569,0.466667,0.705882}%
\pgfsetstrokecolor{currentstroke}%
\pgfsetstrokeopacity{0.869377}%
\pgfsetdash{}{0pt}%
\pgfpathmoveto{\pgfqpoint{1.282281in}{2.370543in}}%
\pgfpathcurveto{\pgfqpoint{1.290517in}{2.370543in}}{\pgfqpoint{1.298417in}{2.373815in}}{\pgfqpoint{1.304241in}{2.379639in}}%
\pgfpathcurveto{\pgfqpoint{1.310065in}{2.385463in}}{\pgfqpoint{1.313338in}{2.393363in}}{\pgfqpoint{1.313338in}{2.401600in}}%
\pgfpathcurveto{\pgfqpoint{1.313338in}{2.409836in}}{\pgfqpoint{1.310065in}{2.417736in}}{\pgfqpoint{1.304241in}{2.423560in}}%
\pgfpathcurveto{\pgfqpoint{1.298417in}{2.429384in}}{\pgfqpoint{1.290517in}{2.432656in}}{\pgfqpoint{1.282281in}{2.432656in}}%
\pgfpathcurveto{\pgfqpoint{1.274045in}{2.432656in}}{\pgfqpoint{1.266145in}{2.429384in}}{\pgfqpoint{1.260321in}{2.423560in}}%
\pgfpathcurveto{\pgfqpoint{1.254497in}{2.417736in}}{\pgfqpoint{1.251225in}{2.409836in}}{\pgfqpoint{1.251225in}{2.401600in}}%
\pgfpathcurveto{\pgfqpoint{1.251225in}{2.393363in}}{\pgfqpoint{1.254497in}{2.385463in}}{\pgfqpoint{1.260321in}{2.379639in}}%
\pgfpathcurveto{\pgfqpoint{1.266145in}{2.373815in}}{\pgfqpoint{1.274045in}{2.370543in}}{\pgfqpoint{1.282281in}{2.370543in}}%
\pgfpathclose%
\pgfusepath{stroke,fill}%
\end{pgfscope}%
\begin{pgfscope}%
\pgfpathrectangle{\pgfqpoint{0.100000in}{0.212622in}}{\pgfqpoint{3.696000in}{3.696000in}}%
\pgfusepath{clip}%
\pgfsetbuttcap%
\pgfsetroundjoin%
\definecolor{currentfill}{rgb}{0.121569,0.466667,0.705882}%
\pgfsetfillcolor{currentfill}%
\pgfsetfillopacity{0.871413}%
\pgfsetlinewidth{1.003750pt}%
\definecolor{currentstroke}{rgb}{0.121569,0.466667,0.705882}%
\pgfsetstrokecolor{currentstroke}%
\pgfsetstrokeopacity{0.871413}%
\pgfsetdash{}{0pt}%
\pgfpathmoveto{\pgfqpoint{1.303222in}{2.359518in}}%
\pgfpathcurveto{\pgfqpoint{1.311459in}{2.359518in}}{\pgfqpoint{1.319359in}{2.362790in}}{\pgfqpoint{1.325182in}{2.368614in}}%
\pgfpathcurveto{\pgfqpoint{1.331006in}{2.374438in}}{\pgfqpoint{1.334279in}{2.382338in}}{\pgfqpoint{1.334279in}{2.390575in}}%
\pgfpathcurveto{\pgfqpoint{1.334279in}{2.398811in}}{\pgfqpoint{1.331006in}{2.406711in}}{\pgfqpoint{1.325182in}{2.412535in}}%
\pgfpathcurveto{\pgfqpoint{1.319359in}{2.418359in}}{\pgfqpoint{1.311459in}{2.421631in}}{\pgfqpoint{1.303222in}{2.421631in}}%
\pgfpathcurveto{\pgfqpoint{1.294986in}{2.421631in}}{\pgfqpoint{1.287086in}{2.418359in}}{\pgfqpoint{1.281262in}{2.412535in}}%
\pgfpathcurveto{\pgfqpoint{1.275438in}{2.406711in}}{\pgfqpoint{1.272166in}{2.398811in}}{\pgfqpoint{1.272166in}{2.390575in}}%
\pgfpathcurveto{\pgfqpoint{1.272166in}{2.382338in}}{\pgfqpoint{1.275438in}{2.374438in}}{\pgfqpoint{1.281262in}{2.368614in}}%
\pgfpathcurveto{\pgfqpoint{1.287086in}{2.362790in}}{\pgfqpoint{1.294986in}{2.359518in}}{\pgfqpoint{1.303222in}{2.359518in}}%
\pgfpathclose%
\pgfusepath{stroke,fill}%
\end{pgfscope}%
\begin{pgfscope}%
\pgfpathrectangle{\pgfqpoint{0.100000in}{0.212622in}}{\pgfqpoint{3.696000in}{3.696000in}}%
\pgfusepath{clip}%
\pgfsetbuttcap%
\pgfsetroundjoin%
\definecolor{currentfill}{rgb}{0.121569,0.466667,0.705882}%
\pgfsetfillcolor{currentfill}%
\pgfsetfillopacity{0.874123}%
\pgfsetlinewidth{1.003750pt}%
\definecolor{currentstroke}{rgb}{0.121569,0.466667,0.705882}%
\pgfsetstrokecolor{currentstroke}%
\pgfsetstrokeopacity{0.874123}%
\pgfsetdash{}{0pt}%
\pgfpathmoveto{\pgfqpoint{2.689869in}{1.970142in}}%
\pgfpathcurveto{\pgfqpoint{2.698106in}{1.970142in}}{\pgfqpoint{2.706006in}{1.973414in}}{\pgfqpoint{2.711830in}{1.979238in}}%
\pgfpathcurveto{\pgfqpoint{2.717654in}{1.985062in}}{\pgfqpoint{2.720926in}{1.992962in}}{\pgfqpoint{2.720926in}{2.001199in}}%
\pgfpathcurveto{\pgfqpoint{2.720926in}{2.009435in}}{\pgfqpoint{2.717654in}{2.017335in}}{\pgfqpoint{2.711830in}{2.023159in}}%
\pgfpathcurveto{\pgfqpoint{2.706006in}{2.028983in}}{\pgfqpoint{2.698106in}{2.032255in}}{\pgfqpoint{2.689869in}{2.032255in}}%
\pgfpathcurveto{\pgfqpoint{2.681633in}{2.032255in}}{\pgfqpoint{2.673733in}{2.028983in}}{\pgfqpoint{2.667909in}{2.023159in}}%
\pgfpathcurveto{\pgfqpoint{2.662085in}{2.017335in}}{\pgfqpoint{2.658813in}{2.009435in}}{\pgfqpoint{2.658813in}{2.001199in}}%
\pgfpathcurveto{\pgfqpoint{2.658813in}{1.992962in}}{\pgfqpoint{2.662085in}{1.985062in}}{\pgfqpoint{2.667909in}{1.979238in}}%
\pgfpathcurveto{\pgfqpoint{2.673733in}{1.973414in}}{\pgfqpoint{2.681633in}{1.970142in}}{\pgfqpoint{2.689869in}{1.970142in}}%
\pgfpathclose%
\pgfusepath{stroke,fill}%
\end{pgfscope}%
\begin{pgfscope}%
\pgfpathrectangle{\pgfqpoint{0.100000in}{0.212622in}}{\pgfqpoint{3.696000in}{3.696000in}}%
\pgfusepath{clip}%
\pgfsetbuttcap%
\pgfsetroundjoin%
\definecolor{currentfill}{rgb}{0.121569,0.466667,0.705882}%
\pgfsetfillcolor{currentfill}%
\pgfsetfillopacity{0.875301}%
\pgfsetlinewidth{1.003750pt}%
\definecolor{currentstroke}{rgb}{0.121569,0.466667,0.705882}%
\pgfsetstrokecolor{currentstroke}%
\pgfsetstrokeopacity{0.875301}%
\pgfsetdash{}{0pt}%
\pgfpathmoveto{\pgfqpoint{1.340107in}{2.337151in}}%
\pgfpathcurveto{\pgfqpoint{1.348343in}{2.337151in}}{\pgfqpoint{1.356243in}{2.340424in}}{\pgfqpoint{1.362067in}{2.346247in}}%
\pgfpathcurveto{\pgfqpoint{1.367891in}{2.352071in}}{\pgfqpoint{1.371163in}{2.359971in}}{\pgfqpoint{1.371163in}{2.368208in}}%
\pgfpathcurveto{\pgfqpoint{1.371163in}{2.376444in}}{\pgfqpoint{1.367891in}{2.384344in}}{\pgfqpoint{1.362067in}{2.390168in}}%
\pgfpathcurveto{\pgfqpoint{1.356243in}{2.395992in}}{\pgfqpoint{1.348343in}{2.399264in}}{\pgfqpoint{1.340107in}{2.399264in}}%
\pgfpathcurveto{\pgfqpoint{1.331870in}{2.399264in}}{\pgfqpoint{1.323970in}{2.395992in}}{\pgfqpoint{1.318147in}{2.390168in}}%
\pgfpathcurveto{\pgfqpoint{1.312323in}{2.384344in}}{\pgfqpoint{1.309050in}{2.376444in}}{\pgfqpoint{1.309050in}{2.368208in}}%
\pgfpathcurveto{\pgfqpoint{1.309050in}{2.359971in}}{\pgfqpoint{1.312323in}{2.352071in}}{\pgfqpoint{1.318147in}{2.346247in}}%
\pgfpathcurveto{\pgfqpoint{1.323970in}{2.340424in}}{\pgfqpoint{1.331870in}{2.337151in}}{\pgfqpoint{1.340107in}{2.337151in}}%
\pgfpathclose%
\pgfusepath{stroke,fill}%
\end{pgfscope}%
\begin{pgfscope}%
\pgfpathrectangle{\pgfqpoint{0.100000in}{0.212622in}}{\pgfqpoint{3.696000in}{3.696000in}}%
\pgfusepath{clip}%
\pgfsetbuttcap%
\pgfsetroundjoin%
\definecolor{currentfill}{rgb}{0.121569,0.466667,0.705882}%
\pgfsetfillcolor{currentfill}%
\pgfsetfillopacity{0.878602}%
\pgfsetlinewidth{1.003750pt}%
\definecolor{currentstroke}{rgb}{0.121569,0.466667,0.705882}%
\pgfsetstrokecolor{currentstroke}%
\pgfsetstrokeopacity{0.878602}%
\pgfsetdash{}{0pt}%
\pgfpathmoveto{\pgfqpoint{1.375672in}{2.314769in}}%
\pgfpathcurveto{\pgfqpoint{1.383908in}{2.314769in}}{\pgfqpoint{1.391808in}{2.318041in}}{\pgfqpoint{1.397632in}{2.323865in}}%
\pgfpathcurveto{\pgfqpoint{1.403456in}{2.329689in}}{\pgfqpoint{1.406728in}{2.337589in}}{\pgfqpoint{1.406728in}{2.345826in}}%
\pgfpathcurveto{\pgfqpoint{1.406728in}{2.354062in}}{\pgfqpoint{1.403456in}{2.361962in}}{\pgfqpoint{1.397632in}{2.367786in}}%
\pgfpathcurveto{\pgfqpoint{1.391808in}{2.373610in}}{\pgfqpoint{1.383908in}{2.376882in}}{\pgfqpoint{1.375672in}{2.376882in}}%
\pgfpathcurveto{\pgfqpoint{1.367436in}{2.376882in}}{\pgfqpoint{1.359536in}{2.373610in}}{\pgfqpoint{1.353712in}{2.367786in}}%
\pgfpathcurveto{\pgfqpoint{1.347888in}{2.361962in}}{\pgfqpoint{1.344615in}{2.354062in}}{\pgfqpoint{1.344615in}{2.345826in}}%
\pgfpathcurveto{\pgfqpoint{1.344615in}{2.337589in}}{\pgfqpoint{1.347888in}{2.329689in}}{\pgfqpoint{1.353712in}{2.323865in}}%
\pgfpathcurveto{\pgfqpoint{1.359536in}{2.318041in}}{\pgfqpoint{1.367436in}{2.314769in}}{\pgfqpoint{1.375672in}{2.314769in}}%
\pgfpathclose%
\pgfusepath{stroke,fill}%
\end{pgfscope}%
\begin{pgfscope}%
\pgfpathrectangle{\pgfqpoint{0.100000in}{0.212622in}}{\pgfqpoint{3.696000in}{3.696000in}}%
\pgfusepath{clip}%
\pgfsetbuttcap%
\pgfsetroundjoin%
\definecolor{currentfill}{rgb}{0.121569,0.466667,0.705882}%
\pgfsetfillcolor{currentfill}%
\pgfsetfillopacity{0.880230}%
\pgfsetlinewidth{1.003750pt}%
\definecolor{currentstroke}{rgb}{0.121569,0.466667,0.705882}%
\pgfsetstrokecolor{currentstroke}%
\pgfsetstrokeopacity{0.880230}%
\pgfsetdash{}{0pt}%
\pgfpathmoveto{\pgfqpoint{2.678161in}{1.958876in}}%
\pgfpathcurveto{\pgfqpoint{2.686397in}{1.958876in}}{\pgfqpoint{2.694297in}{1.962149in}}{\pgfqpoint{2.700121in}{1.967972in}}%
\pgfpathcurveto{\pgfqpoint{2.705945in}{1.973796in}}{\pgfqpoint{2.709218in}{1.981696in}}{\pgfqpoint{2.709218in}{1.989933in}}%
\pgfpathcurveto{\pgfqpoint{2.709218in}{1.998169in}}{\pgfqpoint{2.705945in}{2.006069in}}{\pgfqpoint{2.700121in}{2.011893in}}%
\pgfpathcurveto{\pgfqpoint{2.694297in}{2.017717in}}{\pgfqpoint{2.686397in}{2.020989in}}{\pgfqpoint{2.678161in}{2.020989in}}%
\pgfpathcurveto{\pgfqpoint{2.669925in}{2.020989in}}{\pgfqpoint{2.662025in}{2.017717in}}{\pgfqpoint{2.656201in}{2.011893in}}%
\pgfpathcurveto{\pgfqpoint{2.650377in}{2.006069in}}{\pgfqpoint{2.647105in}{1.998169in}}{\pgfqpoint{2.647105in}{1.989933in}}%
\pgfpathcurveto{\pgfqpoint{2.647105in}{1.981696in}}{\pgfqpoint{2.650377in}{1.973796in}}{\pgfqpoint{2.656201in}{1.967972in}}%
\pgfpathcurveto{\pgfqpoint{2.662025in}{1.962149in}}{\pgfqpoint{2.669925in}{1.958876in}}{\pgfqpoint{2.678161in}{1.958876in}}%
\pgfpathclose%
\pgfusepath{stroke,fill}%
\end{pgfscope}%
\begin{pgfscope}%
\pgfpathrectangle{\pgfqpoint{0.100000in}{0.212622in}}{\pgfqpoint{3.696000in}{3.696000in}}%
\pgfusepath{clip}%
\pgfsetbuttcap%
\pgfsetroundjoin%
\definecolor{currentfill}{rgb}{0.121569,0.466667,0.705882}%
\pgfsetfillcolor{currentfill}%
\pgfsetfillopacity{0.882776}%
\pgfsetlinewidth{1.003750pt}%
\definecolor{currentstroke}{rgb}{0.121569,0.466667,0.705882}%
\pgfsetstrokecolor{currentstroke}%
\pgfsetstrokeopacity{0.882776}%
\pgfsetdash{}{0pt}%
\pgfpathmoveto{\pgfqpoint{1.405994in}{2.299397in}}%
\pgfpathcurveto{\pgfqpoint{1.414230in}{2.299397in}}{\pgfqpoint{1.422130in}{2.302670in}}{\pgfqpoint{1.427954in}{2.308494in}}%
\pgfpathcurveto{\pgfqpoint{1.433778in}{2.314318in}}{\pgfqpoint{1.437050in}{2.322218in}}{\pgfqpoint{1.437050in}{2.330454in}}%
\pgfpathcurveto{\pgfqpoint{1.437050in}{2.338690in}}{\pgfqpoint{1.433778in}{2.346590in}}{\pgfqpoint{1.427954in}{2.352414in}}%
\pgfpathcurveto{\pgfqpoint{1.422130in}{2.358238in}}{\pgfqpoint{1.414230in}{2.361510in}}{\pgfqpoint{1.405994in}{2.361510in}}%
\pgfpathcurveto{\pgfqpoint{1.397757in}{2.361510in}}{\pgfqpoint{1.389857in}{2.358238in}}{\pgfqpoint{1.384033in}{2.352414in}}%
\pgfpathcurveto{\pgfqpoint{1.378209in}{2.346590in}}{\pgfqpoint{1.374937in}{2.338690in}}{\pgfqpoint{1.374937in}{2.330454in}}%
\pgfpathcurveto{\pgfqpoint{1.374937in}{2.322218in}}{\pgfqpoint{1.378209in}{2.314318in}}{\pgfqpoint{1.384033in}{2.308494in}}%
\pgfpathcurveto{\pgfqpoint{1.389857in}{2.302670in}}{\pgfqpoint{1.397757in}{2.299397in}}{\pgfqpoint{1.405994in}{2.299397in}}%
\pgfpathclose%
\pgfusepath{stroke,fill}%
\end{pgfscope}%
\begin{pgfscope}%
\pgfpathrectangle{\pgfqpoint{0.100000in}{0.212622in}}{\pgfqpoint{3.696000in}{3.696000in}}%
\pgfusepath{clip}%
\pgfsetbuttcap%
\pgfsetroundjoin%
\definecolor{currentfill}{rgb}{0.121569,0.466667,0.705882}%
\pgfsetfillcolor{currentfill}%
\pgfsetfillopacity{0.884396}%
\pgfsetlinewidth{1.003750pt}%
\definecolor{currentstroke}{rgb}{0.121569,0.466667,0.705882}%
\pgfsetstrokecolor{currentstroke}%
\pgfsetstrokeopacity{0.884396}%
\pgfsetdash{}{0pt}%
\pgfpathmoveto{\pgfqpoint{1.428873in}{2.283523in}}%
\pgfpathcurveto{\pgfqpoint{1.437109in}{2.283523in}}{\pgfqpoint{1.445009in}{2.286795in}}{\pgfqpoint{1.450833in}{2.292619in}}%
\pgfpathcurveto{\pgfqpoint{1.456657in}{2.298443in}}{\pgfqpoint{1.459930in}{2.306343in}}{\pgfqpoint{1.459930in}{2.314579in}}%
\pgfpathcurveto{\pgfqpoint{1.459930in}{2.322816in}}{\pgfqpoint{1.456657in}{2.330716in}}{\pgfqpoint{1.450833in}{2.336540in}}%
\pgfpathcurveto{\pgfqpoint{1.445009in}{2.342364in}}{\pgfqpoint{1.437109in}{2.345636in}}{\pgfqpoint{1.428873in}{2.345636in}}%
\pgfpathcurveto{\pgfqpoint{1.420637in}{2.345636in}}{\pgfqpoint{1.412737in}{2.342364in}}{\pgfqpoint{1.406913in}{2.336540in}}%
\pgfpathcurveto{\pgfqpoint{1.401089in}{2.330716in}}{\pgfqpoint{1.397817in}{2.322816in}}{\pgfqpoint{1.397817in}{2.314579in}}%
\pgfpathcurveto{\pgfqpoint{1.397817in}{2.306343in}}{\pgfqpoint{1.401089in}{2.298443in}}{\pgfqpoint{1.406913in}{2.292619in}}%
\pgfpathcurveto{\pgfqpoint{1.412737in}{2.286795in}}{\pgfqpoint{1.420637in}{2.283523in}}{\pgfqpoint{1.428873in}{2.283523in}}%
\pgfpathclose%
\pgfusepath{stroke,fill}%
\end{pgfscope}%
\begin{pgfscope}%
\pgfpathrectangle{\pgfqpoint{0.100000in}{0.212622in}}{\pgfqpoint{3.696000in}{3.696000in}}%
\pgfusepath{clip}%
\pgfsetbuttcap%
\pgfsetroundjoin%
\definecolor{currentfill}{rgb}{0.121569,0.466667,0.705882}%
\pgfsetfillcolor{currentfill}%
\pgfsetfillopacity{0.886401}%
\pgfsetlinewidth{1.003750pt}%
\definecolor{currentstroke}{rgb}{0.121569,0.466667,0.705882}%
\pgfsetstrokecolor{currentstroke}%
\pgfsetstrokeopacity{0.886401}%
\pgfsetdash{}{0pt}%
\pgfpathmoveto{\pgfqpoint{1.448551in}{2.274740in}}%
\pgfpathcurveto{\pgfqpoint{1.456787in}{2.274740in}}{\pgfqpoint{1.464687in}{2.278012in}}{\pgfqpoint{1.470511in}{2.283836in}}%
\pgfpathcurveto{\pgfqpoint{1.476335in}{2.289660in}}{\pgfqpoint{1.479607in}{2.297560in}}{\pgfqpoint{1.479607in}{2.305796in}}%
\pgfpathcurveto{\pgfqpoint{1.479607in}{2.314032in}}{\pgfqpoint{1.476335in}{2.321932in}}{\pgfqpoint{1.470511in}{2.327756in}}%
\pgfpathcurveto{\pgfqpoint{1.464687in}{2.333580in}}{\pgfqpoint{1.456787in}{2.336853in}}{\pgfqpoint{1.448551in}{2.336853in}}%
\pgfpathcurveto{\pgfqpoint{1.440314in}{2.336853in}}{\pgfqpoint{1.432414in}{2.333580in}}{\pgfqpoint{1.426590in}{2.327756in}}%
\pgfpathcurveto{\pgfqpoint{1.420766in}{2.321932in}}{\pgfqpoint{1.417494in}{2.314032in}}{\pgfqpoint{1.417494in}{2.305796in}}%
\pgfpathcurveto{\pgfqpoint{1.417494in}{2.297560in}}{\pgfqpoint{1.420766in}{2.289660in}}{\pgfqpoint{1.426590in}{2.283836in}}%
\pgfpathcurveto{\pgfqpoint{1.432414in}{2.278012in}}{\pgfqpoint{1.440314in}{2.274740in}}{\pgfqpoint{1.448551in}{2.274740in}}%
\pgfpathclose%
\pgfusepath{stroke,fill}%
\end{pgfscope}%
\begin{pgfscope}%
\pgfpathrectangle{\pgfqpoint{0.100000in}{0.212622in}}{\pgfqpoint{3.696000in}{3.696000in}}%
\pgfusepath{clip}%
\pgfsetbuttcap%
\pgfsetroundjoin%
\definecolor{currentfill}{rgb}{0.121569,0.466667,0.705882}%
\pgfsetfillcolor{currentfill}%
\pgfsetfillopacity{0.887092}%
\pgfsetlinewidth{1.003750pt}%
\definecolor{currentstroke}{rgb}{0.121569,0.466667,0.705882}%
\pgfsetstrokecolor{currentstroke}%
\pgfsetstrokeopacity{0.887092}%
\pgfsetdash{}{0pt}%
\pgfpathmoveto{\pgfqpoint{2.662128in}{1.950146in}}%
\pgfpathcurveto{\pgfqpoint{2.670364in}{1.950146in}}{\pgfqpoint{2.678264in}{1.953418in}}{\pgfqpoint{2.684088in}{1.959242in}}%
\pgfpathcurveto{\pgfqpoint{2.689912in}{1.965066in}}{\pgfqpoint{2.693185in}{1.972966in}}{\pgfqpoint{2.693185in}{1.981202in}}%
\pgfpathcurveto{\pgfqpoint{2.693185in}{1.989438in}}{\pgfqpoint{2.689912in}{1.997338in}}{\pgfqpoint{2.684088in}{2.003162in}}%
\pgfpathcurveto{\pgfqpoint{2.678264in}{2.008986in}}{\pgfqpoint{2.670364in}{2.012259in}}{\pgfqpoint{2.662128in}{2.012259in}}%
\pgfpathcurveto{\pgfqpoint{2.653892in}{2.012259in}}{\pgfqpoint{2.645992in}{2.008986in}}{\pgfqpoint{2.640168in}{2.003162in}}%
\pgfpathcurveto{\pgfqpoint{2.634344in}{1.997338in}}{\pgfqpoint{2.631072in}{1.989438in}}{\pgfqpoint{2.631072in}{1.981202in}}%
\pgfpathcurveto{\pgfqpoint{2.631072in}{1.972966in}}{\pgfqpoint{2.634344in}{1.965066in}}{\pgfqpoint{2.640168in}{1.959242in}}%
\pgfpathcurveto{\pgfqpoint{2.645992in}{1.953418in}}{\pgfqpoint{2.653892in}{1.950146in}}{\pgfqpoint{2.662128in}{1.950146in}}%
\pgfpathclose%
\pgfusepath{stroke,fill}%
\end{pgfscope}%
\begin{pgfscope}%
\pgfpathrectangle{\pgfqpoint{0.100000in}{0.212622in}}{\pgfqpoint{3.696000in}{3.696000in}}%
\pgfusepath{clip}%
\pgfsetbuttcap%
\pgfsetroundjoin%
\definecolor{currentfill}{rgb}{0.121569,0.466667,0.705882}%
\pgfsetfillcolor{currentfill}%
\pgfsetfillopacity{0.887858}%
\pgfsetlinewidth{1.003750pt}%
\definecolor{currentstroke}{rgb}{0.121569,0.466667,0.705882}%
\pgfsetstrokecolor{currentstroke}%
\pgfsetstrokeopacity{0.887858}%
\pgfsetdash{}{0pt}%
\pgfpathmoveto{\pgfqpoint{1.465250in}{2.267173in}}%
\pgfpathcurveto{\pgfqpoint{1.473486in}{2.267173in}}{\pgfqpoint{1.481386in}{2.270445in}}{\pgfqpoint{1.487210in}{2.276269in}}%
\pgfpathcurveto{\pgfqpoint{1.493034in}{2.282093in}}{\pgfqpoint{1.496306in}{2.289993in}}{\pgfqpoint{1.496306in}{2.298229in}}%
\pgfpathcurveto{\pgfqpoint{1.496306in}{2.306466in}}{\pgfqpoint{1.493034in}{2.314366in}}{\pgfqpoint{1.487210in}{2.320190in}}%
\pgfpathcurveto{\pgfqpoint{1.481386in}{2.326014in}}{\pgfqpoint{1.473486in}{2.329286in}}{\pgfqpoint{1.465250in}{2.329286in}}%
\pgfpathcurveto{\pgfqpoint{1.457013in}{2.329286in}}{\pgfqpoint{1.449113in}{2.326014in}}{\pgfqpoint{1.443289in}{2.320190in}}%
\pgfpathcurveto{\pgfqpoint{1.437465in}{2.314366in}}{\pgfqpoint{1.434193in}{2.306466in}}{\pgfqpoint{1.434193in}{2.298229in}}%
\pgfpathcurveto{\pgfqpoint{1.434193in}{2.289993in}}{\pgfqpoint{1.437465in}{2.282093in}}{\pgfqpoint{1.443289in}{2.276269in}}%
\pgfpathcurveto{\pgfqpoint{1.449113in}{2.270445in}}{\pgfqpoint{1.457013in}{2.267173in}}{\pgfqpoint{1.465250in}{2.267173in}}%
\pgfpathclose%
\pgfusepath{stroke,fill}%
\end{pgfscope}%
\begin{pgfscope}%
\pgfpathrectangle{\pgfqpoint{0.100000in}{0.212622in}}{\pgfqpoint{3.696000in}{3.696000in}}%
\pgfusepath{clip}%
\pgfsetbuttcap%
\pgfsetroundjoin%
\definecolor{currentfill}{rgb}{0.121569,0.466667,0.705882}%
\pgfsetfillcolor{currentfill}%
\pgfsetfillopacity{0.889691}%
\pgfsetlinewidth{1.003750pt}%
\definecolor{currentstroke}{rgb}{0.121569,0.466667,0.705882}%
\pgfsetstrokecolor{currentstroke}%
\pgfsetstrokeopacity{0.889691}%
\pgfsetdash{}{0pt}%
\pgfpathmoveto{\pgfqpoint{1.479158in}{2.262488in}}%
\pgfpathcurveto{\pgfqpoint{1.487394in}{2.262488in}}{\pgfqpoint{1.495295in}{2.265761in}}{\pgfqpoint{1.501118in}{2.271584in}}%
\pgfpathcurveto{\pgfqpoint{1.506942in}{2.277408in}}{\pgfqpoint{1.510215in}{2.285308in}}{\pgfqpoint{1.510215in}{2.293545in}}%
\pgfpathcurveto{\pgfqpoint{1.510215in}{2.301781in}}{\pgfqpoint{1.506942in}{2.309681in}}{\pgfqpoint{1.501118in}{2.315505in}}%
\pgfpathcurveto{\pgfqpoint{1.495295in}{2.321329in}}{\pgfqpoint{1.487394in}{2.324601in}}{\pgfqpoint{1.479158in}{2.324601in}}%
\pgfpathcurveto{\pgfqpoint{1.470922in}{2.324601in}}{\pgfqpoint{1.463022in}{2.321329in}}{\pgfqpoint{1.457198in}{2.315505in}}%
\pgfpathcurveto{\pgfqpoint{1.451374in}{2.309681in}}{\pgfqpoint{1.448102in}{2.301781in}}{\pgfqpoint{1.448102in}{2.293545in}}%
\pgfpathcurveto{\pgfqpoint{1.448102in}{2.285308in}}{\pgfqpoint{1.451374in}{2.277408in}}{\pgfqpoint{1.457198in}{2.271584in}}%
\pgfpathcurveto{\pgfqpoint{1.463022in}{2.265761in}}{\pgfqpoint{1.470922in}{2.262488in}}{\pgfqpoint{1.479158in}{2.262488in}}%
\pgfpathclose%
\pgfusepath{stroke,fill}%
\end{pgfscope}%
\begin{pgfscope}%
\pgfpathrectangle{\pgfqpoint{0.100000in}{0.212622in}}{\pgfqpoint{3.696000in}{3.696000in}}%
\pgfusepath{clip}%
\pgfsetbuttcap%
\pgfsetroundjoin%
\definecolor{currentfill}{rgb}{0.121569,0.466667,0.705882}%
\pgfsetfillcolor{currentfill}%
\pgfsetfillopacity{0.890953}%
\pgfsetlinewidth{1.003750pt}%
\definecolor{currentstroke}{rgb}{0.121569,0.466667,0.705882}%
\pgfsetstrokecolor{currentstroke}%
\pgfsetstrokeopacity{0.890953}%
\pgfsetdash{}{0pt}%
\pgfpathmoveto{\pgfqpoint{2.653374in}{1.945675in}}%
\pgfpathcurveto{\pgfqpoint{2.661610in}{1.945675in}}{\pgfqpoint{2.669510in}{1.948947in}}{\pgfqpoint{2.675334in}{1.954771in}}%
\pgfpathcurveto{\pgfqpoint{2.681158in}{1.960595in}}{\pgfqpoint{2.684430in}{1.968495in}}{\pgfqpoint{2.684430in}{1.976731in}}%
\pgfpathcurveto{\pgfqpoint{2.684430in}{1.984967in}}{\pgfqpoint{2.681158in}{1.992867in}}{\pgfqpoint{2.675334in}{1.998691in}}%
\pgfpathcurveto{\pgfqpoint{2.669510in}{2.004515in}}{\pgfqpoint{2.661610in}{2.007788in}}{\pgfqpoint{2.653374in}{2.007788in}}%
\pgfpathcurveto{\pgfqpoint{2.645137in}{2.007788in}}{\pgfqpoint{2.637237in}{2.004515in}}{\pgfqpoint{2.631414in}{1.998691in}}%
\pgfpathcurveto{\pgfqpoint{2.625590in}{1.992867in}}{\pgfqpoint{2.622317in}{1.984967in}}{\pgfqpoint{2.622317in}{1.976731in}}%
\pgfpathcurveto{\pgfqpoint{2.622317in}{1.968495in}}{\pgfqpoint{2.625590in}{1.960595in}}{\pgfqpoint{2.631414in}{1.954771in}}%
\pgfpathcurveto{\pgfqpoint{2.637237in}{1.948947in}}{\pgfqpoint{2.645137in}{1.945675in}}{\pgfqpoint{2.653374in}{1.945675in}}%
\pgfpathclose%
\pgfusepath{stroke,fill}%
\end{pgfscope}%
\begin{pgfscope}%
\pgfpathrectangle{\pgfqpoint{0.100000in}{0.212622in}}{\pgfqpoint{3.696000in}{3.696000in}}%
\pgfusepath{clip}%
\pgfsetbuttcap%
\pgfsetroundjoin%
\definecolor{currentfill}{rgb}{0.121569,0.466667,0.705882}%
\pgfsetfillcolor{currentfill}%
\pgfsetfillopacity{0.893059}%
\pgfsetlinewidth{1.003750pt}%
\definecolor{currentstroke}{rgb}{0.121569,0.466667,0.705882}%
\pgfsetstrokecolor{currentstroke}%
\pgfsetstrokeopacity{0.893059}%
\pgfsetdash{}{0pt}%
\pgfpathmoveto{\pgfqpoint{2.648447in}{1.943202in}}%
\pgfpathcurveto{\pgfqpoint{2.656683in}{1.943202in}}{\pgfqpoint{2.664583in}{1.946474in}}{\pgfqpoint{2.670407in}{1.952298in}}%
\pgfpathcurveto{\pgfqpoint{2.676231in}{1.958122in}}{\pgfqpoint{2.679503in}{1.966022in}}{\pgfqpoint{2.679503in}{1.974258in}}%
\pgfpathcurveto{\pgfqpoint{2.679503in}{1.982495in}}{\pgfqpoint{2.676231in}{1.990395in}}{\pgfqpoint{2.670407in}{1.996219in}}%
\pgfpathcurveto{\pgfqpoint{2.664583in}{2.002043in}}{\pgfqpoint{2.656683in}{2.005315in}}{\pgfqpoint{2.648447in}{2.005315in}}%
\pgfpathcurveto{\pgfqpoint{2.640210in}{2.005315in}}{\pgfqpoint{2.632310in}{2.002043in}}{\pgfqpoint{2.626486in}{1.996219in}}%
\pgfpathcurveto{\pgfqpoint{2.620662in}{1.990395in}}{\pgfqpoint{2.617390in}{1.982495in}}{\pgfqpoint{2.617390in}{1.974258in}}%
\pgfpathcurveto{\pgfqpoint{2.617390in}{1.966022in}}{\pgfqpoint{2.620662in}{1.958122in}}{\pgfqpoint{2.626486in}{1.952298in}}%
\pgfpathcurveto{\pgfqpoint{2.632310in}{1.946474in}}{\pgfqpoint{2.640210in}{1.943202in}}{\pgfqpoint{2.648447in}{1.943202in}}%
\pgfpathclose%
\pgfusepath{stroke,fill}%
\end{pgfscope}%
\begin{pgfscope}%
\pgfpathrectangle{\pgfqpoint{0.100000in}{0.212622in}}{\pgfqpoint{3.696000in}{3.696000in}}%
\pgfusepath{clip}%
\pgfsetbuttcap%
\pgfsetroundjoin%
\definecolor{currentfill}{rgb}{0.121569,0.466667,0.705882}%
\pgfsetfillcolor{currentfill}%
\pgfsetfillopacity{0.893323}%
\pgfsetlinewidth{1.003750pt}%
\definecolor{currentstroke}{rgb}{0.121569,0.466667,0.705882}%
\pgfsetstrokecolor{currentstroke}%
\pgfsetstrokeopacity{0.893323}%
\pgfsetdash{}{0pt}%
\pgfpathmoveto{\pgfqpoint{1.502832in}{2.250711in}}%
\pgfpathcurveto{\pgfqpoint{1.511068in}{2.250711in}}{\pgfqpoint{1.518968in}{2.253983in}}{\pgfqpoint{1.524792in}{2.259807in}}%
\pgfpathcurveto{\pgfqpoint{1.530616in}{2.265631in}}{\pgfqpoint{1.533888in}{2.273531in}}{\pgfqpoint{1.533888in}{2.281767in}}%
\pgfpathcurveto{\pgfqpoint{1.533888in}{2.290004in}}{\pgfqpoint{1.530616in}{2.297904in}}{\pgfqpoint{1.524792in}{2.303727in}}%
\pgfpathcurveto{\pgfqpoint{1.518968in}{2.309551in}}{\pgfqpoint{1.511068in}{2.312824in}}{\pgfqpoint{1.502832in}{2.312824in}}%
\pgfpathcurveto{\pgfqpoint{1.494596in}{2.312824in}}{\pgfqpoint{1.486696in}{2.309551in}}{\pgfqpoint{1.480872in}{2.303727in}}%
\pgfpathcurveto{\pgfqpoint{1.475048in}{2.297904in}}{\pgfqpoint{1.471775in}{2.290004in}}{\pgfqpoint{1.471775in}{2.281767in}}%
\pgfpathcurveto{\pgfqpoint{1.471775in}{2.273531in}}{\pgfqpoint{1.475048in}{2.265631in}}{\pgfqpoint{1.480872in}{2.259807in}}%
\pgfpathcurveto{\pgfqpoint{1.486696in}{2.253983in}}{\pgfqpoint{1.494596in}{2.250711in}}{\pgfqpoint{1.502832in}{2.250711in}}%
\pgfpathclose%
\pgfusepath{stroke,fill}%
\end{pgfscope}%
\begin{pgfscope}%
\pgfpathrectangle{\pgfqpoint{0.100000in}{0.212622in}}{\pgfqpoint{3.696000in}{3.696000in}}%
\pgfusepath{clip}%
\pgfsetbuttcap%
\pgfsetroundjoin%
\definecolor{currentfill}{rgb}{0.121569,0.466667,0.705882}%
\pgfsetfillcolor{currentfill}%
\pgfsetfillopacity{0.894099}%
\pgfsetlinewidth{1.003750pt}%
\definecolor{currentstroke}{rgb}{0.121569,0.466667,0.705882}%
\pgfsetstrokecolor{currentstroke}%
\pgfsetstrokeopacity{0.894099}%
\pgfsetdash{}{0pt}%
\pgfpathmoveto{\pgfqpoint{2.645333in}{1.941717in}}%
\pgfpathcurveto{\pgfqpoint{2.653569in}{1.941717in}}{\pgfqpoint{2.661469in}{1.944989in}}{\pgfqpoint{2.667293in}{1.950813in}}%
\pgfpathcurveto{\pgfqpoint{2.673117in}{1.956637in}}{\pgfqpoint{2.676389in}{1.964537in}}{\pgfqpoint{2.676389in}{1.972773in}}%
\pgfpathcurveto{\pgfqpoint{2.676389in}{1.981009in}}{\pgfqpoint{2.673117in}{1.988909in}}{\pgfqpoint{2.667293in}{1.994733in}}%
\pgfpathcurveto{\pgfqpoint{2.661469in}{2.000557in}}{\pgfqpoint{2.653569in}{2.003830in}}{\pgfqpoint{2.645333in}{2.003830in}}%
\pgfpathcurveto{\pgfqpoint{2.637096in}{2.003830in}}{\pgfqpoint{2.629196in}{2.000557in}}{\pgfqpoint{2.623372in}{1.994733in}}%
\pgfpathcurveto{\pgfqpoint{2.617549in}{1.988909in}}{\pgfqpoint{2.614276in}{1.981009in}}{\pgfqpoint{2.614276in}{1.972773in}}%
\pgfpathcurveto{\pgfqpoint{2.614276in}{1.964537in}}{\pgfqpoint{2.617549in}{1.956637in}}{\pgfqpoint{2.623372in}{1.950813in}}%
\pgfpathcurveto{\pgfqpoint{2.629196in}{1.944989in}}{\pgfqpoint{2.637096in}{1.941717in}}{\pgfqpoint{2.645333in}{1.941717in}}%
\pgfpathclose%
\pgfusepath{stroke,fill}%
\end{pgfscope}%
\begin{pgfscope}%
\pgfpathrectangle{\pgfqpoint{0.100000in}{0.212622in}}{\pgfqpoint{3.696000in}{3.696000in}}%
\pgfusepath{clip}%
\pgfsetbuttcap%
\pgfsetroundjoin%
\definecolor{currentfill}{rgb}{0.121569,0.466667,0.705882}%
\pgfsetfillcolor{currentfill}%
\pgfsetfillopacity{0.894693}%
\pgfsetlinewidth{1.003750pt}%
\definecolor{currentstroke}{rgb}{0.121569,0.466667,0.705882}%
\pgfsetstrokecolor{currentstroke}%
\pgfsetstrokeopacity{0.894693}%
\pgfsetdash{}{0pt}%
\pgfpathmoveto{\pgfqpoint{2.643691in}{1.940919in}}%
\pgfpathcurveto{\pgfqpoint{2.651927in}{1.940919in}}{\pgfqpoint{2.659827in}{1.944192in}}{\pgfqpoint{2.665651in}{1.950015in}}%
\pgfpathcurveto{\pgfqpoint{2.671475in}{1.955839in}}{\pgfqpoint{2.674747in}{1.963739in}}{\pgfqpoint{2.674747in}{1.971976in}}%
\pgfpathcurveto{\pgfqpoint{2.674747in}{1.980212in}}{\pgfqpoint{2.671475in}{1.988112in}}{\pgfqpoint{2.665651in}{1.993936in}}%
\pgfpathcurveto{\pgfqpoint{2.659827in}{1.999760in}}{\pgfqpoint{2.651927in}{2.003032in}}{\pgfqpoint{2.643691in}{2.003032in}}%
\pgfpathcurveto{\pgfqpoint{2.635454in}{2.003032in}}{\pgfqpoint{2.627554in}{1.999760in}}{\pgfqpoint{2.621730in}{1.993936in}}%
\pgfpathcurveto{\pgfqpoint{2.615906in}{1.988112in}}{\pgfqpoint{2.612634in}{1.980212in}}{\pgfqpoint{2.612634in}{1.971976in}}%
\pgfpathcurveto{\pgfqpoint{2.612634in}{1.963739in}}{\pgfqpoint{2.615906in}{1.955839in}}{\pgfqpoint{2.621730in}{1.950015in}}%
\pgfpathcurveto{\pgfqpoint{2.627554in}{1.944192in}}{\pgfqpoint{2.635454in}{1.940919in}}{\pgfqpoint{2.643691in}{1.940919in}}%
\pgfpathclose%
\pgfusepath{stroke,fill}%
\end{pgfscope}%
\begin{pgfscope}%
\pgfpathrectangle{\pgfqpoint{0.100000in}{0.212622in}}{\pgfqpoint{3.696000in}{3.696000in}}%
\pgfusepath{clip}%
\pgfsetbuttcap%
\pgfsetroundjoin%
\definecolor{currentfill}{rgb}{0.121569,0.466667,0.705882}%
\pgfsetfillcolor{currentfill}%
\pgfsetfillopacity{0.895092}%
\pgfsetlinewidth{1.003750pt}%
\definecolor{currentstroke}{rgb}{0.121569,0.466667,0.705882}%
\pgfsetstrokecolor{currentstroke}%
\pgfsetstrokeopacity{0.895092}%
\pgfsetdash{}{0pt}%
\pgfpathmoveto{\pgfqpoint{2.642883in}{1.940801in}}%
\pgfpathcurveto{\pgfqpoint{2.651119in}{1.940801in}}{\pgfqpoint{2.659019in}{1.944073in}}{\pgfqpoint{2.664843in}{1.949897in}}%
\pgfpathcurveto{\pgfqpoint{2.670667in}{1.955721in}}{\pgfqpoint{2.673939in}{1.963621in}}{\pgfqpoint{2.673939in}{1.971858in}}%
\pgfpathcurveto{\pgfqpoint{2.673939in}{1.980094in}}{\pgfqpoint{2.670667in}{1.987994in}}{\pgfqpoint{2.664843in}{1.993818in}}%
\pgfpathcurveto{\pgfqpoint{2.659019in}{1.999642in}}{\pgfqpoint{2.651119in}{2.002914in}}{\pgfqpoint{2.642883in}{2.002914in}}%
\pgfpathcurveto{\pgfqpoint{2.634647in}{2.002914in}}{\pgfqpoint{2.626746in}{1.999642in}}{\pgfqpoint{2.620923in}{1.993818in}}%
\pgfpathcurveto{\pgfqpoint{2.615099in}{1.987994in}}{\pgfqpoint{2.611826in}{1.980094in}}{\pgfqpoint{2.611826in}{1.971858in}}%
\pgfpathcurveto{\pgfqpoint{2.611826in}{1.963621in}}{\pgfqpoint{2.615099in}{1.955721in}}{\pgfqpoint{2.620923in}{1.949897in}}%
\pgfpathcurveto{\pgfqpoint{2.626746in}{1.944073in}}{\pgfqpoint{2.634647in}{1.940801in}}{\pgfqpoint{2.642883in}{1.940801in}}%
\pgfpathclose%
\pgfusepath{stroke,fill}%
\end{pgfscope}%
\begin{pgfscope}%
\pgfpathrectangle{\pgfqpoint{0.100000in}{0.212622in}}{\pgfqpoint{3.696000in}{3.696000in}}%
\pgfusepath{clip}%
\pgfsetbuttcap%
\pgfsetroundjoin%
\definecolor{currentfill}{rgb}{0.121569,0.466667,0.705882}%
\pgfsetfillcolor{currentfill}%
\pgfsetfillopacity{0.895277}%
\pgfsetlinewidth{1.003750pt}%
\definecolor{currentstroke}{rgb}{0.121569,0.466667,0.705882}%
\pgfsetstrokecolor{currentstroke}%
\pgfsetstrokeopacity{0.895277}%
\pgfsetdash{}{0pt}%
\pgfpathmoveto{\pgfqpoint{1.551380in}{2.217775in}}%
\pgfpathcurveto{\pgfqpoint{1.559616in}{2.217775in}}{\pgfqpoint{1.567516in}{2.221047in}}{\pgfqpoint{1.573340in}{2.226871in}}%
\pgfpathcurveto{\pgfqpoint{1.579164in}{2.232695in}}{\pgfqpoint{1.582436in}{2.240595in}}{\pgfqpoint{1.582436in}{2.248832in}}%
\pgfpathcurveto{\pgfqpoint{1.582436in}{2.257068in}}{\pgfqpoint{1.579164in}{2.264968in}}{\pgfqpoint{1.573340in}{2.270792in}}%
\pgfpathcurveto{\pgfqpoint{1.567516in}{2.276616in}}{\pgfqpoint{1.559616in}{2.279888in}}{\pgfqpoint{1.551380in}{2.279888in}}%
\pgfpathcurveto{\pgfqpoint{1.543143in}{2.279888in}}{\pgfqpoint{1.535243in}{2.276616in}}{\pgfqpoint{1.529419in}{2.270792in}}%
\pgfpathcurveto{\pgfqpoint{1.523595in}{2.264968in}}{\pgfqpoint{1.520323in}{2.257068in}}{\pgfqpoint{1.520323in}{2.248832in}}%
\pgfpathcurveto{\pgfqpoint{1.520323in}{2.240595in}}{\pgfqpoint{1.523595in}{2.232695in}}{\pgfqpoint{1.529419in}{2.226871in}}%
\pgfpathcurveto{\pgfqpoint{1.535243in}{2.221047in}}{\pgfqpoint{1.543143in}{2.217775in}}{\pgfqpoint{1.551380in}{2.217775in}}%
\pgfpathclose%
\pgfusepath{stroke,fill}%
\end{pgfscope}%
\begin{pgfscope}%
\pgfpathrectangle{\pgfqpoint{0.100000in}{0.212622in}}{\pgfqpoint{3.696000in}{3.696000in}}%
\pgfusepath{clip}%
\pgfsetbuttcap%
\pgfsetroundjoin%
\definecolor{currentfill}{rgb}{0.121569,0.466667,0.705882}%
\pgfsetfillcolor{currentfill}%
\pgfsetfillopacity{0.896032}%
\pgfsetlinewidth{1.003750pt}%
\definecolor{currentstroke}{rgb}{0.121569,0.466667,0.705882}%
\pgfsetstrokecolor{currentstroke}%
\pgfsetstrokeopacity{0.896032}%
\pgfsetdash{}{0pt}%
\pgfpathmoveto{\pgfqpoint{2.640882in}{1.938194in}}%
\pgfpathcurveto{\pgfqpoint{2.649119in}{1.938194in}}{\pgfqpoint{2.657019in}{1.941466in}}{\pgfqpoint{2.662842in}{1.947290in}}%
\pgfpathcurveto{\pgfqpoint{2.668666in}{1.953114in}}{\pgfqpoint{2.671939in}{1.961014in}}{\pgfqpoint{2.671939in}{1.969250in}}%
\pgfpathcurveto{\pgfqpoint{2.671939in}{1.977487in}}{\pgfqpoint{2.668666in}{1.985387in}}{\pgfqpoint{2.662842in}{1.991211in}}%
\pgfpathcurveto{\pgfqpoint{2.657019in}{1.997035in}}{\pgfqpoint{2.649119in}{2.000307in}}{\pgfqpoint{2.640882in}{2.000307in}}%
\pgfpathcurveto{\pgfqpoint{2.632646in}{2.000307in}}{\pgfqpoint{2.624746in}{1.997035in}}{\pgfqpoint{2.618922in}{1.991211in}}%
\pgfpathcurveto{\pgfqpoint{2.613098in}{1.985387in}}{\pgfqpoint{2.609826in}{1.977487in}}{\pgfqpoint{2.609826in}{1.969250in}}%
\pgfpathcurveto{\pgfqpoint{2.609826in}{1.961014in}}{\pgfqpoint{2.613098in}{1.953114in}}{\pgfqpoint{2.618922in}{1.947290in}}%
\pgfpathcurveto{\pgfqpoint{2.624746in}{1.941466in}}{\pgfqpoint{2.632646in}{1.938194in}}{\pgfqpoint{2.640882in}{1.938194in}}%
\pgfpathclose%
\pgfusepath{stroke,fill}%
\end{pgfscope}%
\begin{pgfscope}%
\pgfpathrectangle{\pgfqpoint{0.100000in}{0.212622in}}{\pgfqpoint{3.696000in}{3.696000in}}%
\pgfusepath{clip}%
\pgfsetbuttcap%
\pgfsetroundjoin%
\definecolor{currentfill}{rgb}{0.121569,0.466667,0.705882}%
\pgfsetfillcolor{currentfill}%
\pgfsetfillopacity{0.898276}%
\pgfsetlinewidth{1.003750pt}%
\definecolor{currentstroke}{rgb}{0.121569,0.466667,0.705882}%
\pgfsetstrokecolor{currentstroke}%
\pgfsetstrokeopacity{0.898276}%
\pgfsetdash{}{0pt}%
\pgfpathmoveto{\pgfqpoint{2.636345in}{1.937022in}}%
\pgfpathcurveto{\pgfqpoint{2.644581in}{1.937022in}}{\pgfqpoint{2.652482in}{1.940294in}}{\pgfqpoint{2.658305in}{1.946118in}}%
\pgfpathcurveto{\pgfqpoint{2.664129in}{1.951942in}}{\pgfqpoint{2.667402in}{1.959842in}}{\pgfqpoint{2.667402in}{1.968078in}}%
\pgfpathcurveto{\pgfqpoint{2.667402in}{1.976315in}}{\pgfqpoint{2.664129in}{1.984215in}}{\pgfqpoint{2.658305in}{1.990038in}}%
\pgfpathcurveto{\pgfqpoint{2.652482in}{1.995862in}}{\pgfqpoint{2.644581in}{1.999135in}}{\pgfqpoint{2.636345in}{1.999135in}}%
\pgfpathcurveto{\pgfqpoint{2.628109in}{1.999135in}}{\pgfqpoint{2.620209in}{1.995862in}}{\pgfqpoint{2.614385in}{1.990038in}}%
\pgfpathcurveto{\pgfqpoint{2.608561in}{1.984215in}}{\pgfqpoint{2.605289in}{1.976315in}}{\pgfqpoint{2.605289in}{1.968078in}}%
\pgfpathcurveto{\pgfqpoint{2.605289in}{1.959842in}}{\pgfqpoint{2.608561in}{1.951942in}}{\pgfqpoint{2.614385in}{1.946118in}}%
\pgfpathcurveto{\pgfqpoint{2.620209in}{1.940294in}}{\pgfqpoint{2.628109in}{1.937022in}}{\pgfqpoint{2.636345in}{1.937022in}}%
\pgfpathclose%
\pgfusepath{stroke,fill}%
\end{pgfscope}%
\begin{pgfscope}%
\pgfpathrectangle{\pgfqpoint{0.100000in}{0.212622in}}{\pgfqpoint{3.696000in}{3.696000in}}%
\pgfusepath{clip}%
\pgfsetbuttcap%
\pgfsetroundjoin%
\definecolor{currentfill}{rgb}{0.121569,0.466667,0.705882}%
\pgfsetfillcolor{currentfill}%
\pgfsetfillopacity{0.901110}%
\pgfsetlinewidth{1.003750pt}%
\definecolor{currentstroke}{rgb}{0.121569,0.466667,0.705882}%
\pgfsetstrokecolor{currentstroke}%
\pgfsetstrokeopacity{0.901110}%
\pgfsetdash{}{0pt}%
\pgfpathmoveto{\pgfqpoint{2.630133in}{1.933517in}}%
\pgfpathcurveto{\pgfqpoint{2.638369in}{1.933517in}}{\pgfqpoint{2.646269in}{1.936789in}}{\pgfqpoint{2.652093in}{1.942613in}}%
\pgfpathcurveto{\pgfqpoint{2.657917in}{1.948437in}}{\pgfqpoint{2.661189in}{1.956337in}}{\pgfqpoint{2.661189in}{1.964574in}}%
\pgfpathcurveto{\pgfqpoint{2.661189in}{1.972810in}}{\pgfqpoint{2.657917in}{1.980710in}}{\pgfqpoint{2.652093in}{1.986534in}}%
\pgfpathcurveto{\pgfqpoint{2.646269in}{1.992358in}}{\pgfqpoint{2.638369in}{1.995630in}}{\pgfqpoint{2.630133in}{1.995630in}}%
\pgfpathcurveto{\pgfqpoint{2.621896in}{1.995630in}}{\pgfqpoint{2.613996in}{1.992358in}}{\pgfqpoint{2.608173in}{1.986534in}}%
\pgfpathcurveto{\pgfqpoint{2.602349in}{1.980710in}}{\pgfqpoint{2.599076in}{1.972810in}}{\pgfqpoint{2.599076in}{1.964574in}}%
\pgfpathcurveto{\pgfqpoint{2.599076in}{1.956337in}}{\pgfqpoint{2.602349in}{1.948437in}}{\pgfqpoint{2.608173in}{1.942613in}}%
\pgfpathcurveto{\pgfqpoint{2.613996in}{1.936789in}}{\pgfqpoint{2.621896in}{1.933517in}}{\pgfqpoint{2.630133in}{1.933517in}}%
\pgfpathclose%
\pgfusepath{stroke,fill}%
\end{pgfscope}%
\begin{pgfscope}%
\pgfpathrectangle{\pgfqpoint{0.100000in}{0.212622in}}{\pgfqpoint{3.696000in}{3.696000in}}%
\pgfusepath{clip}%
\pgfsetbuttcap%
\pgfsetroundjoin%
\definecolor{currentfill}{rgb}{0.121569,0.466667,0.705882}%
\pgfsetfillcolor{currentfill}%
\pgfsetfillopacity{0.903068}%
\pgfsetlinewidth{1.003750pt}%
\definecolor{currentstroke}{rgb}{0.121569,0.466667,0.705882}%
\pgfsetstrokecolor{currentstroke}%
\pgfsetstrokeopacity{0.903068}%
\pgfsetdash{}{0pt}%
\pgfpathmoveto{\pgfqpoint{1.592452in}{2.205786in}}%
\pgfpathcurveto{\pgfqpoint{1.600689in}{2.205786in}}{\pgfqpoint{1.608589in}{2.209058in}}{\pgfqpoint{1.614413in}{2.214882in}}%
\pgfpathcurveto{\pgfqpoint{1.620237in}{2.220706in}}{\pgfqpoint{1.623509in}{2.228606in}}{\pgfqpoint{1.623509in}{2.236842in}}%
\pgfpathcurveto{\pgfqpoint{1.623509in}{2.245079in}}{\pgfqpoint{1.620237in}{2.252979in}}{\pgfqpoint{1.614413in}{2.258803in}}%
\pgfpathcurveto{\pgfqpoint{1.608589in}{2.264626in}}{\pgfqpoint{1.600689in}{2.267899in}}{\pgfqpoint{1.592452in}{2.267899in}}%
\pgfpathcurveto{\pgfqpoint{1.584216in}{2.267899in}}{\pgfqpoint{1.576316in}{2.264626in}}{\pgfqpoint{1.570492in}{2.258803in}}%
\pgfpathcurveto{\pgfqpoint{1.564668in}{2.252979in}}{\pgfqpoint{1.561396in}{2.245079in}}{\pgfqpoint{1.561396in}{2.236842in}}%
\pgfpathcurveto{\pgfqpoint{1.561396in}{2.228606in}}{\pgfqpoint{1.564668in}{2.220706in}}{\pgfqpoint{1.570492in}{2.214882in}}%
\pgfpathcurveto{\pgfqpoint{1.576316in}{2.209058in}}{\pgfqpoint{1.584216in}{2.205786in}}{\pgfqpoint{1.592452in}{2.205786in}}%
\pgfpathclose%
\pgfusepath{stroke,fill}%
\end{pgfscope}%
\begin{pgfscope}%
\pgfpathrectangle{\pgfqpoint{0.100000in}{0.212622in}}{\pgfqpoint{3.696000in}{3.696000in}}%
\pgfusepath{clip}%
\pgfsetbuttcap%
\pgfsetroundjoin%
\definecolor{currentfill}{rgb}{0.121569,0.466667,0.705882}%
\pgfsetfillcolor{currentfill}%
\pgfsetfillopacity{0.904942}%
\pgfsetlinewidth{1.003750pt}%
\definecolor{currentstroke}{rgb}{0.121569,0.466667,0.705882}%
\pgfsetstrokecolor{currentstroke}%
\pgfsetstrokeopacity{0.904942}%
\pgfsetdash{}{0pt}%
\pgfpathmoveto{\pgfqpoint{2.620329in}{1.930620in}}%
\pgfpathcurveto{\pgfqpoint{2.628565in}{1.930620in}}{\pgfqpoint{2.636465in}{1.933892in}}{\pgfqpoint{2.642289in}{1.939716in}}%
\pgfpathcurveto{\pgfqpoint{2.648113in}{1.945540in}}{\pgfqpoint{2.651385in}{1.953440in}}{\pgfqpoint{2.651385in}{1.961676in}}%
\pgfpathcurveto{\pgfqpoint{2.651385in}{1.969913in}}{\pgfqpoint{2.648113in}{1.977813in}}{\pgfqpoint{2.642289in}{1.983637in}}%
\pgfpathcurveto{\pgfqpoint{2.636465in}{1.989461in}}{\pgfqpoint{2.628565in}{1.992733in}}{\pgfqpoint{2.620329in}{1.992733in}}%
\pgfpathcurveto{\pgfqpoint{2.612093in}{1.992733in}}{\pgfqpoint{2.604193in}{1.989461in}}{\pgfqpoint{2.598369in}{1.983637in}}%
\pgfpathcurveto{\pgfqpoint{2.592545in}{1.977813in}}{\pgfqpoint{2.589272in}{1.969913in}}{\pgfqpoint{2.589272in}{1.961676in}}%
\pgfpathcurveto{\pgfqpoint{2.589272in}{1.953440in}}{\pgfqpoint{2.592545in}{1.945540in}}{\pgfqpoint{2.598369in}{1.939716in}}%
\pgfpathcurveto{\pgfqpoint{2.604193in}{1.933892in}}{\pgfqpoint{2.612093in}{1.930620in}}{\pgfqpoint{2.620329in}{1.930620in}}%
\pgfpathclose%
\pgfusepath{stroke,fill}%
\end{pgfscope}%
\begin{pgfscope}%
\pgfpathrectangle{\pgfqpoint{0.100000in}{0.212622in}}{\pgfqpoint{3.696000in}{3.696000in}}%
\pgfusepath{clip}%
\pgfsetbuttcap%
\pgfsetroundjoin%
\definecolor{currentfill}{rgb}{0.121569,0.466667,0.705882}%
\pgfsetfillcolor{currentfill}%
\pgfsetfillopacity{0.906069}%
\pgfsetlinewidth{1.003750pt}%
\definecolor{currentstroke}{rgb}{0.121569,0.466667,0.705882}%
\pgfsetstrokecolor{currentstroke}%
\pgfsetstrokeopacity{0.906069}%
\pgfsetdash{}{0pt}%
\pgfpathmoveto{\pgfqpoint{1.633678in}{2.182266in}}%
\pgfpathcurveto{\pgfqpoint{1.641915in}{2.182266in}}{\pgfqpoint{1.649815in}{2.185539in}}{\pgfqpoint{1.655639in}{2.191363in}}%
\pgfpathcurveto{\pgfqpoint{1.661462in}{2.197186in}}{\pgfqpoint{1.664735in}{2.205086in}}{\pgfqpoint{1.664735in}{2.213323in}}%
\pgfpathcurveto{\pgfqpoint{1.664735in}{2.221559in}}{\pgfqpoint{1.661462in}{2.229459in}}{\pgfqpoint{1.655639in}{2.235283in}}%
\pgfpathcurveto{\pgfqpoint{1.649815in}{2.241107in}}{\pgfqpoint{1.641915in}{2.244379in}}{\pgfqpoint{1.633678in}{2.244379in}}%
\pgfpathcurveto{\pgfqpoint{1.625442in}{2.244379in}}{\pgfqpoint{1.617542in}{2.241107in}}{\pgfqpoint{1.611718in}{2.235283in}}%
\pgfpathcurveto{\pgfqpoint{1.605894in}{2.229459in}}{\pgfqpoint{1.602622in}{2.221559in}}{\pgfqpoint{1.602622in}{2.213323in}}%
\pgfpathcurveto{\pgfqpoint{1.602622in}{2.205086in}}{\pgfqpoint{1.605894in}{2.197186in}}{\pgfqpoint{1.611718in}{2.191363in}}%
\pgfpathcurveto{\pgfqpoint{1.617542in}{2.185539in}}{\pgfqpoint{1.625442in}{2.182266in}}{\pgfqpoint{1.633678in}{2.182266in}}%
\pgfpathclose%
\pgfusepath{stroke,fill}%
\end{pgfscope}%
\begin{pgfscope}%
\pgfpathrectangle{\pgfqpoint{0.100000in}{0.212622in}}{\pgfqpoint{3.696000in}{3.696000in}}%
\pgfusepath{clip}%
\pgfsetbuttcap%
\pgfsetroundjoin%
\definecolor{currentfill}{rgb}{0.121569,0.466667,0.705882}%
\pgfsetfillcolor{currentfill}%
\pgfsetfillopacity{0.909394}%
\pgfsetlinewidth{1.003750pt}%
\definecolor{currentstroke}{rgb}{0.121569,0.466667,0.705882}%
\pgfsetstrokecolor{currentstroke}%
\pgfsetstrokeopacity{0.909394}%
\pgfsetdash{}{0pt}%
\pgfpathmoveto{\pgfqpoint{2.609915in}{1.925060in}}%
\pgfpathcurveto{\pgfqpoint{2.618151in}{1.925060in}}{\pgfqpoint{2.626051in}{1.928332in}}{\pgfqpoint{2.631875in}{1.934156in}}%
\pgfpathcurveto{\pgfqpoint{2.637699in}{1.939980in}}{\pgfqpoint{2.640971in}{1.947880in}}{\pgfqpoint{2.640971in}{1.956116in}}%
\pgfpathcurveto{\pgfqpoint{2.640971in}{1.964353in}}{\pgfqpoint{2.637699in}{1.972253in}}{\pgfqpoint{2.631875in}{1.978077in}}%
\pgfpathcurveto{\pgfqpoint{2.626051in}{1.983900in}}{\pgfqpoint{2.618151in}{1.987173in}}{\pgfqpoint{2.609915in}{1.987173in}}%
\pgfpathcurveto{\pgfqpoint{2.601678in}{1.987173in}}{\pgfqpoint{2.593778in}{1.983900in}}{\pgfqpoint{2.587954in}{1.978077in}}%
\pgfpathcurveto{\pgfqpoint{2.582130in}{1.972253in}}{\pgfqpoint{2.578858in}{1.964353in}}{\pgfqpoint{2.578858in}{1.956116in}}%
\pgfpathcurveto{\pgfqpoint{2.578858in}{1.947880in}}{\pgfqpoint{2.582130in}{1.939980in}}{\pgfqpoint{2.587954in}{1.934156in}}%
\pgfpathcurveto{\pgfqpoint{2.593778in}{1.928332in}}{\pgfqpoint{2.601678in}{1.925060in}}{\pgfqpoint{2.609915in}{1.925060in}}%
\pgfpathclose%
\pgfusepath{stroke,fill}%
\end{pgfscope}%
\begin{pgfscope}%
\pgfpathrectangle{\pgfqpoint{0.100000in}{0.212622in}}{\pgfqpoint{3.696000in}{3.696000in}}%
\pgfusepath{clip}%
\pgfsetbuttcap%
\pgfsetroundjoin%
\definecolor{currentfill}{rgb}{0.121569,0.466667,0.705882}%
\pgfsetfillcolor{currentfill}%
\pgfsetfillopacity{0.911548}%
\pgfsetlinewidth{1.003750pt}%
\definecolor{currentstroke}{rgb}{0.121569,0.466667,0.705882}%
\pgfsetstrokecolor{currentstroke}%
\pgfsetstrokeopacity{0.911548}%
\pgfsetdash{}{0pt}%
\pgfpathmoveto{\pgfqpoint{1.666674in}{2.167893in}}%
\pgfpathcurveto{\pgfqpoint{1.674911in}{2.167893in}}{\pgfqpoint{1.682811in}{2.171166in}}{\pgfqpoint{1.688635in}{2.176990in}}%
\pgfpathcurveto{\pgfqpoint{1.694459in}{2.182814in}}{\pgfqpoint{1.697731in}{2.190714in}}{\pgfqpoint{1.697731in}{2.198950in}}%
\pgfpathcurveto{\pgfqpoint{1.697731in}{2.207186in}}{\pgfqpoint{1.694459in}{2.215086in}}{\pgfqpoint{1.688635in}{2.220910in}}%
\pgfpathcurveto{\pgfqpoint{1.682811in}{2.226734in}}{\pgfqpoint{1.674911in}{2.230006in}}{\pgfqpoint{1.666674in}{2.230006in}}%
\pgfpathcurveto{\pgfqpoint{1.658438in}{2.230006in}}{\pgfqpoint{1.650538in}{2.226734in}}{\pgfqpoint{1.644714in}{2.220910in}}%
\pgfpathcurveto{\pgfqpoint{1.638890in}{2.215086in}}{\pgfqpoint{1.635618in}{2.207186in}}{\pgfqpoint{1.635618in}{2.198950in}}%
\pgfpathcurveto{\pgfqpoint{1.635618in}{2.190714in}}{\pgfqpoint{1.638890in}{2.182814in}}{\pgfqpoint{1.644714in}{2.176990in}}%
\pgfpathcurveto{\pgfqpoint{1.650538in}{2.171166in}}{\pgfqpoint{1.658438in}{2.167893in}}{\pgfqpoint{1.666674in}{2.167893in}}%
\pgfpathclose%
\pgfusepath{stroke,fill}%
\end{pgfscope}%
\begin{pgfscope}%
\pgfpathrectangle{\pgfqpoint{0.100000in}{0.212622in}}{\pgfqpoint{3.696000in}{3.696000in}}%
\pgfusepath{clip}%
\pgfsetbuttcap%
\pgfsetroundjoin%
\definecolor{currentfill}{rgb}{0.121569,0.466667,0.705882}%
\pgfsetfillcolor{currentfill}%
\pgfsetfillopacity{0.914046}%
\pgfsetlinewidth{1.003750pt}%
\definecolor{currentstroke}{rgb}{0.121569,0.466667,0.705882}%
\pgfsetstrokecolor{currentstroke}%
\pgfsetstrokeopacity{0.914046}%
\pgfsetdash{}{0pt}%
\pgfpathmoveto{\pgfqpoint{1.696836in}{2.147790in}}%
\pgfpathcurveto{\pgfqpoint{1.705072in}{2.147790in}}{\pgfqpoint{1.712972in}{2.151062in}}{\pgfqpoint{1.718796in}{2.156886in}}%
\pgfpathcurveto{\pgfqpoint{1.724620in}{2.162710in}}{\pgfqpoint{1.727892in}{2.170610in}}{\pgfqpoint{1.727892in}{2.178847in}}%
\pgfpathcurveto{\pgfqpoint{1.727892in}{2.187083in}}{\pgfqpoint{1.724620in}{2.194983in}}{\pgfqpoint{1.718796in}{2.200807in}}%
\pgfpathcurveto{\pgfqpoint{1.712972in}{2.206631in}}{\pgfqpoint{1.705072in}{2.209903in}}{\pgfqpoint{1.696836in}{2.209903in}}%
\pgfpathcurveto{\pgfqpoint{1.688599in}{2.209903in}}{\pgfqpoint{1.680699in}{2.206631in}}{\pgfqpoint{1.674875in}{2.200807in}}%
\pgfpathcurveto{\pgfqpoint{1.669052in}{2.194983in}}{\pgfqpoint{1.665779in}{2.187083in}}{\pgfqpoint{1.665779in}{2.178847in}}%
\pgfpathcurveto{\pgfqpoint{1.665779in}{2.170610in}}{\pgfqpoint{1.669052in}{2.162710in}}{\pgfqpoint{1.674875in}{2.156886in}}%
\pgfpathcurveto{\pgfqpoint{1.680699in}{2.151062in}}{\pgfqpoint{1.688599in}{2.147790in}}{\pgfqpoint{1.696836in}{2.147790in}}%
\pgfpathclose%
\pgfusepath{stroke,fill}%
\end{pgfscope}%
\begin{pgfscope}%
\pgfpathrectangle{\pgfqpoint{0.100000in}{0.212622in}}{\pgfqpoint{3.696000in}{3.696000in}}%
\pgfusepath{clip}%
\pgfsetbuttcap%
\pgfsetroundjoin%
\definecolor{currentfill}{rgb}{0.121569,0.466667,0.705882}%
\pgfsetfillcolor{currentfill}%
\pgfsetfillopacity{0.914136}%
\pgfsetlinewidth{1.003750pt}%
\definecolor{currentstroke}{rgb}{0.121569,0.466667,0.705882}%
\pgfsetstrokecolor{currentstroke}%
\pgfsetstrokeopacity{0.914136}%
\pgfsetdash{}{0pt}%
\pgfpathmoveto{\pgfqpoint{2.596745in}{1.919081in}}%
\pgfpathcurveto{\pgfqpoint{2.604981in}{1.919081in}}{\pgfqpoint{2.612882in}{1.922354in}}{\pgfqpoint{2.618705in}{1.928178in}}%
\pgfpathcurveto{\pgfqpoint{2.624529in}{1.934001in}}{\pgfqpoint{2.627802in}{1.941901in}}{\pgfqpoint{2.627802in}{1.950138in}}%
\pgfpathcurveto{\pgfqpoint{2.627802in}{1.958374in}}{\pgfqpoint{2.624529in}{1.966274in}}{\pgfqpoint{2.618705in}{1.972098in}}%
\pgfpathcurveto{\pgfqpoint{2.612882in}{1.977922in}}{\pgfqpoint{2.604981in}{1.981194in}}{\pgfqpoint{2.596745in}{1.981194in}}%
\pgfpathcurveto{\pgfqpoint{2.588509in}{1.981194in}}{\pgfqpoint{2.580609in}{1.977922in}}{\pgfqpoint{2.574785in}{1.972098in}}%
\pgfpathcurveto{\pgfqpoint{2.568961in}{1.966274in}}{\pgfqpoint{2.565689in}{1.958374in}}{\pgfqpoint{2.565689in}{1.950138in}}%
\pgfpathcurveto{\pgfqpoint{2.565689in}{1.941901in}}{\pgfqpoint{2.568961in}{1.934001in}}{\pgfqpoint{2.574785in}{1.928178in}}%
\pgfpathcurveto{\pgfqpoint{2.580609in}{1.922354in}}{\pgfqpoint{2.588509in}{1.919081in}}{\pgfqpoint{2.596745in}{1.919081in}}%
\pgfpathclose%
\pgfusepath{stroke,fill}%
\end{pgfscope}%
\begin{pgfscope}%
\pgfpathrectangle{\pgfqpoint{0.100000in}{0.212622in}}{\pgfqpoint{3.696000in}{3.696000in}}%
\pgfusepath{clip}%
\pgfsetbuttcap%
\pgfsetroundjoin%
\definecolor{currentfill}{rgb}{0.121569,0.466667,0.705882}%
\pgfsetfillcolor{currentfill}%
\pgfsetfillopacity{0.916729}%
\pgfsetlinewidth{1.003750pt}%
\definecolor{currentstroke}{rgb}{0.121569,0.466667,0.705882}%
\pgfsetstrokecolor{currentstroke}%
\pgfsetstrokeopacity{0.916729}%
\pgfsetdash{}{0pt}%
\pgfpathmoveto{\pgfqpoint{1.720057in}{2.136545in}}%
\pgfpathcurveto{\pgfqpoint{1.728293in}{2.136545in}}{\pgfqpoint{1.736193in}{2.139817in}}{\pgfqpoint{1.742017in}{2.145641in}}%
\pgfpathcurveto{\pgfqpoint{1.747841in}{2.151465in}}{\pgfqpoint{1.751113in}{2.159365in}}{\pgfqpoint{1.751113in}{2.167601in}}%
\pgfpathcurveto{\pgfqpoint{1.751113in}{2.175838in}}{\pgfqpoint{1.747841in}{2.183738in}}{\pgfqpoint{1.742017in}{2.189562in}}%
\pgfpathcurveto{\pgfqpoint{1.736193in}{2.195386in}}{\pgfqpoint{1.728293in}{2.198658in}}{\pgfqpoint{1.720057in}{2.198658in}}%
\pgfpathcurveto{\pgfqpoint{1.711820in}{2.198658in}}{\pgfqpoint{1.703920in}{2.195386in}}{\pgfqpoint{1.698096in}{2.189562in}}%
\pgfpathcurveto{\pgfqpoint{1.692272in}{2.183738in}}{\pgfqpoint{1.689000in}{2.175838in}}{\pgfqpoint{1.689000in}{2.167601in}}%
\pgfpathcurveto{\pgfqpoint{1.689000in}{2.159365in}}{\pgfqpoint{1.692272in}{2.151465in}}{\pgfqpoint{1.698096in}{2.145641in}}%
\pgfpathcurveto{\pgfqpoint{1.703920in}{2.139817in}}{\pgfqpoint{1.711820in}{2.136545in}}{\pgfqpoint{1.720057in}{2.136545in}}%
\pgfpathclose%
\pgfusepath{stroke,fill}%
\end{pgfscope}%
\begin{pgfscope}%
\pgfpathrectangle{\pgfqpoint{0.100000in}{0.212622in}}{\pgfqpoint{3.696000in}{3.696000in}}%
\pgfusepath{clip}%
\pgfsetbuttcap%
\pgfsetroundjoin%
\definecolor{currentfill}{rgb}{0.121569,0.466667,0.705882}%
\pgfsetfillcolor{currentfill}%
\pgfsetfillopacity{0.917251}%
\pgfsetlinewidth{1.003750pt}%
\definecolor{currentstroke}{rgb}{0.121569,0.466667,0.705882}%
\pgfsetstrokecolor{currentstroke}%
\pgfsetstrokeopacity{0.917251}%
\pgfsetdash{}{0pt}%
\pgfpathmoveto{\pgfqpoint{2.590690in}{1.917327in}}%
\pgfpathcurveto{\pgfqpoint{2.598926in}{1.917327in}}{\pgfqpoint{2.606827in}{1.920599in}}{\pgfqpoint{2.612650in}{1.926423in}}%
\pgfpathcurveto{\pgfqpoint{2.618474in}{1.932247in}}{\pgfqpoint{2.621747in}{1.940147in}}{\pgfqpoint{2.621747in}{1.948383in}}%
\pgfpathcurveto{\pgfqpoint{2.621747in}{1.956620in}}{\pgfqpoint{2.618474in}{1.964520in}}{\pgfqpoint{2.612650in}{1.970344in}}%
\pgfpathcurveto{\pgfqpoint{2.606827in}{1.976168in}}{\pgfqpoint{2.598926in}{1.979440in}}{\pgfqpoint{2.590690in}{1.979440in}}%
\pgfpathcurveto{\pgfqpoint{2.582454in}{1.979440in}}{\pgfqpoint{2.574554in}{1.976168in}}{\pgfqpoint{2.568730in}{1.970344in}}%
\pgfpathcurveto{\pgfqpoint{2.562906in}{1.964520in}}{\pgfqpoint{2.559634in}{1.956620in}}{\pgfqpoint{2.559634in}{1.948383in}}%
\pgfpathcurveto{\pgfqpoint{2.559634in}{1.940147in}}{\pgfqpoint{2.562906in}{1.932247in}}{\pgfqpoint{2.568730in}{1.926423in}}%
\pgfpathcurveto{\pgfqpoint{2.574554in}{1.920599in}}{\pgfqpoint{2.582454in}{1.917327in}}{\pgfqpoint{2.590690in}{1.917327in}}%
\pgfpathclose%
\pgfusepath{stroke,fill}%
\end{pgfscope}%
\begin{pgfscope}%
\pgfpathrectangle{\pgfqpoint{0.100000in}{0.212622in}}{\pgfqpoint{3.696000in}{3.696000in}}%
\pgfusepath{clip}%
\pgfsetbuttcap%
\pgfsetroundjoin%
\definecolor{currentfill}{rgb}{0.121569,0.466667,0.705882}%
\pgfsetfillcolor{currentfill}%
\pgfsetfillopacity{0.918635}%
\pgfsetlinewidth{1.003750pt}%
\definecolor{currentstroke}{rgb}{0.121569,0.466667,0.705882}%
\pgfsetstrokecolor{currentstroke}%
\pgfsetstrokeopacity{0.918635}%
\pgfsetdash{}{0pt}%
\pgfpathmoveto{\pgfqpoint{2.586525in}{1.915404in}}%
\pgfpathcurveto{\pgfqpoint{2.594761in}{1.915404in}}{\pgfqpoint{2.602661in}{1.918676in}}{\pgfqpoint{2.608485in}{1.924500in}}%
\pgfpathcurveto{\pgfqpoint{2.614309in}{1.930324in}}{\pgfqpoint{2.617581in}{1.938224in}}{\pgfqpoint{2.617581in}{1.946460in}}%
\pgfpathcurveto{\pgfqpoint{2.617581in}{1.954697in}}{\pgfqpoint{2.614309in}{1.962597in}}{\pgfqpoint{2.608485in}{1.968421in}}%
\pgfpathcurveto{\pgfqpoint{2.602661in}{1.974245in}}{\pgfqpoint{2.594761in}{1.977517in}}{\pgfqpoint{2.586525in}{1.977517in}}%
\pgfpathcurveto{\pgfqpoint{2.578289in}{1.977517in}}{\pgfqpoint{2.570388in}{1.974245in}}{\pgfqpoint{2.564565in}{1.968421in}}%
\pgfpathcurveto{\pgfqpoint{2.558741in}{1.962597in}}{\pgfqpoint{2.555468in}{1.954697in}}{\pgfqpoint{2.555468in}{1.946460in}}%
\pgfpathcurveto{\pgfqpoint{2.555468in}{1.938224in}}{\pgfqpoint{2.558741in}{1.930324in}}{\pgfqpoint{2.564565in}{1.924500in}}%
\pgfpathcurveto{\pgfqpoint{2.570388in}{1.918676in}}{\pgfqpoint{2.578289in}{1.915404in}}{\pgfqpoint{2.586525in}{1.915404in}}%
\pgfpathclose%
\pgfusepath{stroke,fill}%
\end{pgfscope}%
\begin{pgfscope}%
\pgfpathrectangle{\pgfqpoint{0.100000in}{0.212622in}}{\pgfqpoint{3.696000in}{3.696000in}}%
\pgfusepath{clip}%
\pgfsetbuttcap%
\pgfsetroundjoin%
\definecolor{currentfill}{rgb}{0.121569,0.466667,0.705882}%
\pgfsetfillcolor{currentfill}%
\pgfsetfillopacity{0.918855}%
\pgfsetlinewidth{1.003750pt}%
\definecolor{currentstroke}{rgb}{0.121569,0.466667,0.705882}%
\pgfsetstrokecolor{currentstroke}%
\pgfsetstrokeopacity{0.918855}%
\pgfsetdash{}{0pt}%
\pgfpathmoveto{\pgfqpoint{1.740093in}{2.126322in}}%
\pgfpathcurveto{\pgfqpoint{1.748329in}{2.126322in}}{\pgfqpoint{1.756229in}{2.129594in}}{\pgfqpoint{1.762053in}{2.135418in}}%
\pgfpathcurveto{\pgfqpoint{1.767877in}{2.141242in}}{\pgfqpoint{1.771150in}{2.149142in}}{\pgfqpoint{1.771150in}{2.157378in}}%
\pgfpathcurveto{\pgfqpoint{1.771150in}{2.165614in}}{\pgfqpoint{1.767877in}{2.173514in}}{\pgfqpoint{1.762053in}{2.179338in}}%
\pgfpathcurveto{\pgfqpoint{1.756229in}{2.185162in}}{\pgfqpoint{1.748329in}{2.188435in}}{\pgfqpoint{1.740093in}{2.188435in}}%
\pgfpathcurveto{\pgfqpoint{1.731857in}{2.188435in}}{\pgfqpoint{1.723957in}{2.185162in}}{\pgfqpoint{1.718133in}{2.179338in}}%
\pgfpathcurveto{\pgfqpoint{1.712309in}{2.173514in}}{\pgfqpoint{1.709037in}{2.165614in}}{\pgfqpoint{1.709037in}{2.157378in}}%
\pgfpathcurveto{\pgfqpoint{1.709037in}{2.149142in}}{\pgfqpoint{1.712309in}{2.141242in}}{\pgfqpoint{1.718133in}{2.135418in}}%
\pgfpathcurveto{\pgfqpoint{1.723957in}{2.129594in}}{\pgfqpoint{1.731857in}{2.126322in}}{\pgfqpoint{1.740093in}{2.126322in}}%
\pgfpathclose%
\pgfusepath{stroke,fill}%
\end{pgfscope}%
\begin{pgfscope}%
\pgfpathrectangle{\pgfqpoint{0.100000in}{0.212622in}}{\pgfqpoint{3.696000in}{3.696000in}}%
\pgfusepath{clip}%
\pgfsetbuttcap%
\pgfsetroundjoin%
\definecolor{currentfill}{rgb}{0.121569,0.466667,0.705882}%
\pgfsetfillcolor{currentfill}%
\pgfsetfillopacity{0.919598}%
\pgfsetlinewidth{1.003750pt}%
\definecolor{currentstroke}{rgb}{0.121569,0.466667,0.705882}%
\pgfsetstrokecolor{currentstroke}%
\pgfsetstrokeopacity{0.919598}%
\pgfsetdash{}{0pt}%
\pgfpathmoveto{\pgfqpoint{2.584653in}{1.915038in}}%
\pgfpathcurveto{\pgfqpoint{2.592889in}{1.915038in}}{\pgfqpoint{2.600789in}{1.918310in}}{\pgfqpoint{2.606613in}{1.924134in}}%
\pgfpathcurveto{\pgfqpoint{2.612437in}{1.929958in}}{\pgfqpoint{2.615709in}{1.937858in}}{\pgfqpoint{2.615709in}{1.946094in}}%
\pgfpathcurveto{\pgfqpoint{2.615709in}{1.954331in}}{\pgfqpoint{2.612437in}{1.962231in}}{\pgfqpoint{2.606613in}{1.968055in}}%
\pgfpathcurveto{\pgfqpoint{2.600789in}{1.973879in}}{\pgfqpoint{2.592889in}{1.977151in}}{\pgfqpoint{2.584653in}{1.977151in}}%
\pgfpathcurveto{\pgfqpoint{2.576416in}{1.977151in}}{\pgfqpoint{2.568516in}{1.973879in}}{\pgfqpoint{2.562692in}{1.968055in}}%
\pgfpathcurveto{\pgfqpoint{2.556868in}{1.962231in}}{\pgfqpoint{2.553596in}{1.954331in}}{\pgfqpoint{2.553596in}{1.946094in}}%
\pgfpathcurveto{\pgfqpoint{2.553596in}{1.937858in}}{\pgfqpoint{2.556868in}{1.929958in}}{\pgfqpoint{2.562692in}{1.924134in}}%
\pgfpathcurveto{\pgfqpoint{2.568516in}{1.918310in}}{\pgfqpoint{2.576416in}{1.915038in}}{\pgfqpoint{2.584653in}{1.915038in}}%
\pgfpathclose%
\pgfusepath{stroke,fill}%
\end{pgfscope}%
\begin{pgfscope}%
\pgfpathrectangle{\pgfqpoint{0.100000in}{0.212622in}}{\pgfqpoint{3.696000in}{3.696000in}}%
\pgfusepath{clip}%
\pgfsetbuttcap%
\pgfsetroundjoin%
\definecolor{currentfill}{rgb}{0.121569,0.466667,0.705882}%
\pgfsetfillcolor{currentfill}%
\pgfsetfillopacity{0.920057}%
\pgfsetlinewidth{1.003750pt}%
\definecolor{currentstroke}{rgb}{0.121569,0.466667,0.705882}%
\pgfsetstrokecolor{currentstroke}%
\pgfsetstrokeopacity{0.920057}%
\pgfsetdash{}{0pt}%
\pgfpathmoveto{\pgfqpoint{2.583624in}{1.914376in}}%
\pgfpathcurveto{\pgfqpoint{2.591861in}{1.914376in}}{\pgfqpoint{2.599761in}{1.917648in}}{\pgfqpoint{2.605585in}{1.923472in}}%
\pgfpathcurveto{\pgfqpoint{2.611409in}{1.929296in}}{\pgfqpoint{2.614681in}{1.937196in}}{\pgfqpoint{2.614681in}{1.945433in}}%
\pgfpathcurveto{\pgfqpoint{2.614681in}{1.953669in}}{\pgfqpoint{2.611409in}{1.961569in}}{\pgfqpoint{2.605585in}{1.967393in}}%
\pgfpathcurveto{\pgfqpoint{2.599761in}{1.973217in}}{\pgfqpoint{2.591861in}{1.976489in}}{\pgfqpoint{2.583624in}{1.976489in}}%
\pgfpathcurveto{\pgfqpoint{2.575388in}{1.976489in}}{\pgfqpoint{2.567488in}{1.973217in}}{\pgfqpoint{2.561664in}{1.967393in}}%
\pgfpathcurveto{\pgfqpoint{2.555840in}{1.961569in}}{\pgfqpoint{2.552568in}{1.953669in}}{\pgfqpoint{2.552568in}{1.945433in}}%
\pgfpathcurveto{\pgfqpoint{2.552568in}{1.937196in}}{\pgfqpoint{2.555840in}{1.929296in}}{\pgfqpoint{2.561664in}{1.923472in}}%
\pgfpathcurveto{\pgfqpoint{2.567488in}{1.917648in}}{\pgfqpoint{2.575388in}{1.914376in}}{\pgfqpoint{2.583624in}{1.914376in}}%
\pgfpathclose%
\pgfusepath{stroke,fill}%
\end{pgfscope}%
\begin{pgfscope}%
\pgfpathrectangle{\pgfqpoint{0.100000in}{0.212622in}}{\pgfqpoint{3.696000in}{3.696000in}}%
\pgfusepath{clip}%
\pgfsetbuttcap%
\pgfsetroundjoin%
\definecolor{currentfill}{rgb}{0.121569,0.466667,0.705882}%
\pgfsetfillcolor{currentfill}%
\pgfsetfillopacity{0.920332}%
\pgfsetlinewidth{1.003750pt}%
\definecolor{currentstroke}{rgb}{0.121569,0.466667,0.705882}%
\pgfsetstrokecolor{currentstroke}%
\pgfsetstrokeopacity{0.920332}%
\pgfsetdash{}{0pt}%
\pgfpathmoveto{\pgfqpoint{2.583013in}{1.914221in}}%
\pgfpathcurveto{\pgfqpoint{2.591250in}{1.914221in}}{\pgfqpoint{2.599150in}{1.917493in}}{\pgfqpoint{2.604974in}{1.923317in}}%
\pgfpathcurveto{\pgfqpoint{2.610798in}{1.929141in}}{\pgfqpoint{2.614070in}{1.937041in}}{\pgfqpoint{2.614070in}{1.945277in}}%
\pgfpathcurveto{\pgfqpoint{2.614070in}{1.953513in}}{\pgfqpoint{2.610798in}{1.961413in}}{\pgfqpoint{2.604974in}{1.967237in}}%
\pgfpathcurveto{\pgfqpoint{2.599150in}{1.973061in}}{\pgfqpoint{2.591250in}{1.976334in}}{\pgfqpoint{2.583013in}{1.976334in}}%
\pgfpathcurveto{\pgfqpoint{2.574777in}{1.976334in}}{\pgfqpoint{2.566877in}{1.973061in}}{\pgfqpoint{2.561053in}{1.967237in}}%
\pgfpathcurveto{\pgfqpoint{2.555229in}{1.961413in}}{\pgfqpoint{2.551957in}{1.953513in}}{\pgfqpoint{2.551957in}{1.945277in}}%
\pgfpathcurveto{\pgfqpoint{2.551957in}{1.937041in}}{\pgfqpoint{2.555229in}{1.929141in}}{\pgfqpoint{2.561053in}{1.923317in}}%
\pgfpathcurveto{\pgfqpoint{2.566877in}{1.917493in}}{\pgfqpoint{2.574777in}{1.914221in}}{\pgfqpoint{2.583013in}{1.914221in}}%
\pgfpathclose%
\pgfusepath{stroke,fill}%
\end{pgfscope}%
\begin{pgfscope}%
\pgfpathrectangle{\pgfqpoint{0.100000in}{0.212622in}}{\pgfqpoint{3.696000in}{3.696000in}}%
\pgfusepath{clip}%
\pgfsetbuttcap%
\pgfsetroundjoin%
\definecolor{currentfill}{rgb}{0.121569,0.466667,0.705882}%
\pgfsetfillcolor{currentfill}%
\pgfsetfillopacity{0.921011}%
\pgfsetlinewidth{1.003750pt}%
\definecolor{currentstroke}{rgb}{0.121569,0.466667,0.705882}%
\pgfsetstrokecolor{currentstroke}%
\pgfsetstrokeopacity{0.921011}%
\pgfsetdash{}{0pt}%
\pgfpathmoveto{\pgfqpoint{1.755069in}{2.118341in}}%
\pgfpathcurveto{\pgfqpoint{1.763306in}{2.118341in}}{\pgfqpoint{1.771206in}{2.121613in}}{\pgfqpoint{1.777030in}{2.127437in}}%
\pgfpathcurveto{\pgfqpoint{1.782854in}{2.133261in}}{\pgfqpoint{1.786126in}{2.141161in}}{\pgfqpoint{1.786126in}{2.149398in}}%
\pgfpathcurveto{\pgfqpoint{1.786126in}{2.157634in}}{\pgfqpoint{1.782854in}{2.165534in}}{\pgfqpoint{1.777030in}{2.171358in}}%
\pgfpathcurveto{\pgfqpoint{1.771206in}{2.177182in}}{\pgfqpoint{1.763306in}{2.180454in}}{\pgfqpoint{1.755069in}{2.180454in}}%
\pgfpathcurveto{\pgfqpoint{1.746833in}{2.180454in}}{\pgfqpoint{1.738933in}{2.177182in}}{\pgfqpoint{1.733109in}{2.171358in}}%
\pgfpathcurveto{\pgfqpoint{1.727285in}{2.165534in}}{\pgfqpoint{1.724013in}{2.157634in}}{\pgfqpoint{1.724013in}{2.149398in}}%
\pgfpathcurveto{\pgfqpoint{1.724013in}{2.141161in}}{\pgfqpoint{1.727285in}{2.133261in}}{\pgfqpoint{1.733109in}{2.127437in}}%
\pgfpathcurveto{\pgfqpoint{1.738933in}{2.121613in}}{\pgfqpoint{1.746833in}{2.118341in}}{\pgfqpoint{1.755069in}{2.118341in}}%
\pgfpathclose%
\pgfusepath{stroke,fill}%
\end{pgfscope}%
\begin{pgfscope}%
\pgfpathrectangle{\pgfqpoint{0.100000in}{0.212622in}}{\pgfqpoint{3.696000in}{3.696000in}}%
\pgfusepath{clip}%
\pgfsetbuttcap%
\pgfsetroundjoin%
\definecolor{currentfill}{rgb}{0.121569,0.466667,0.705882}%
\pgfsetfillcolor{currentfill}%
\pgfsetfillopacity{0.921901}%
\pgfsetlinewidth{1.003750pt}%
\definecolor{currentstroke}{rgb}{0.121569,0.466667,0.705882}%
\pgfsetstrokecolor{currentstroke}%
\pgfsetstrokeopacity{0.921901}%
\pgfsetdash{}{0pt}%
\pgfpathmoveto{\pgfqpoint{2.579480in}{1.910696in}}%
\pgfpathcurveto{\pgfqpoint{2.587716in}{1.910696in}}{\pgfqpoint{2.595616in}{1.913968in}}{\pgfqpoint{2.601440in}{1.919792in}}%
\pgfpathcurveto{\pgfqpoint{2.607264in}{1.925616in}}{\pgfqpoint{2.610537in}{1.933516in}}{\pgfqpoint{2.610537in}{1.941753in}}%
\pgfpathcurveto{\pgfqpoint{2.610537in}{1.949989in}}{\pgfqpoint{2.607264in}{1.957889in}}{\pgfqpoint{2.601440in}{1.963713in}}%
\pgfpathcurveto{\pgfqpoint{2.595616in}{1.969537in}}{\pgfqpoint{2.587716in}{1.972809in}}{\pgfqpoint{2.579480in}{1.972809in}}%
\pgfpathcurveto{\pgfqpoint{2.571244in}{1.972809in}}{\pgfqpoint{2.563344in}{1.969537in}}{\pgfqpoint{2.557520in}{1.963713in}}%
\pgfpathcurveto{\pgfqpoint{2.551696in}{1.957889in}}{\pgfqpoint{2.548424in}{1.949989in}}{\pgfqpoint{2.548424in}{1.941753in}}%
\pgfpathcurveto{\pgfqpoint{2.548424in}{1.933516in}}{\pgfqpoint{2.551696in}{1.925616in}}{\pgfqpoint{2.557520in}{1.919792in}}%
\pgfpathcurveto{\pgfqpoint{2.563344in}{1.913968in}}{\pgfqpoint{2.571244in}{1.910696in}}{\pgfqpoint{2.579480in}{1.910696in}}%
\pgfpathclose%
\pgfusepath{stroke,fill}%
\end{pgfscope}%
\begin{pgfscope}%
\pgfpathrectangle{\pgfqpoint{0.100000in}{0.212622in}}{\pgfqpoint{3.696000in}{3.696000in}}%
\pgfusepath{clip}%
\pgfsetbuttcap%
\pgfsetroundjoin%
\definecolor{currentfill}{rgb}{0.121569,0.466667,0.705882}%
\pgfsetfillcolor{currentfill}%
\pgfsetfillopacity{0.922758}%
\pgfsetlinewidth{1.003750pt}%
\definecolor{currentstroke}{rgb}{0.121569,0.466667,0.705882}%
\pgfsetstrokecolor{currentstroke}%
\pgfsetstrokeopacity{0.922758}%
\pgfsetdash{}{0pt}%
\pgfpathmoveto{\pgfqpoint{1.767935in}{2.110341in}}%
\pgfpathcurveto{\pgfqpoint{1.776171in}{2.110341in}}{\pgfqpoint{1.784071in}{2.113613in}}{\pgfqpoint{1.789895in}{2.119437in}}%
\pgfpathcurveto{\pgfqpoint{1.795719in}{2.125261in}}{\pgfqpoint{1.798992in}{2.133161in}}{\pgfqpoint{1.798992in}{2.141397in}}%
\pgfpathcurveto{\pgfqpoint{1.798992in}{2.149633in}}{\pgfqpoint{1.795719in}{2.157534in}}{\pgfqpoint{1.789895in}{2.163357in}}%
\pgfpathcurveto{\pgfqpoint{1.784071in}{2.169181in}}{\pgfqpoint{1.776171in}{2.172454in}}{\pgfqpoint{1.767935in}{2.172454in}}%
\pgfpathcurveto{\pgfqpoint{1.759699in}{2.172454in}}{\pgfqpoint{1.751799in}{2.169181in}}{\pgfqpoint{1.745975in}{2.163357in}}%
\pgfpathcurveto{\pgfqpoint{1.740151in}{2.157534in}}{\pgfqpoint{1.736879in}{2.149633in}}{\pgfqpoint{1.736879in}{2.141397in}}%
\pgfpathcurveto{\pgfqpoint{1.736879in}{2.133161in}}{\pgfqpoint{1.740151in}{2.125261in}}{\pgfqpoint{1.745975in}{2.119437in}}%
\pgfpathcurveto{\pgfqpoint{1.751799in}{2.113613in}}{\pgfqpoint{1.759699in}{2.110341in}}{\pgfqpoint{1.767935in}{2.110341in}}%
\pgfpathclose%
\pgfusepath{stroke,fill}%
\end{pgfscope}%
\begin{pgfscope}%
\pgfpathrectangle{\pgfqpoint{0.100000in}{0.212622in}}{\pgfqpoint{3.696000in}{3.696000in}}%
\pgfusepath{clip}%
\pgfsetbuttcap%
\pgfsetroundjoin%
\definecolor{currentfill}{rgb}{0.121569,0.466667,0.705882}%
\pgfsetfillcolor{currentfill}%
\pgfsetfillopacity{0.923599}%
\pgfsetlinewidth{1.003750pt}%
\definecolor{currentstroke}{rgb}{0.121569,0.466667,0.705882}%
\pgfsetstrokecolor{currentstroke}%
\pgfsetstrokeopacity{0.923599}%
\pgfsetdash{}{0pt}%
\pgfpathmoveto{\pgfqpoint{1.780026in}{2.102811in}}%
\pgfpathcurveto{\pgfqpoint{1.788263in}{2.102811in}}{\pgfqpoint{1.796163in}{2.106083in}}{\pgfqpoint{1.801987in}{2.111907in}}%
\pgfpathcurveto{\pgfqpoint{1.807811in}{2.117731in}}{\pgfqpoint{1.811083in}{2.125631in}}{\pgfqpoint{1.811083in}{2.133867in}}%
\pgfpathcurveto{\pgfqpoint{1.811083in}{2.142104in}}{\pgfqpoint{1.807811in}{2.150004in}}{\pgfqpoint{1.801987in}{2.155828in}}%
\pgfpathcurveto{\pgfqpoint{1.796163in}{2.161652in}}{\pgfqpoint{1.788263in}{2.164924in}}{\pgfqpoint{1.780026in}{2.164924in}}%
\pgfpathcurveto{\pgfqpoint{1.771790in}{2.164924in}}{\pgfqpoint{1.763890in}{2.161652in}}{\pgfqpoint{1.758066in}{2.155828in}}%
\pgfpathcurveto{\pgfqpoint{1.752242in}{2.150004in}}{\pgfqpoint{1.748970in}{2.142104in}}{\pgfqpoint{1.748970in}{2.133867in}}%
\pgfpathcurveto{\pgfqpoint{1.748970in}{2.125631in}}{\pgfqpoint{1.752242in}{2.117731in}}{\pgfqpoint{1.758066in}{2.111907in}}%
\pgfpathcurveto{\pgfqpoint{1.763890in}{2.106083in}}{\pgfqpoint{1.771790in}{2.102811in}}{\pgfqpoint{1.780026in}{2.102811in}}%
\pgfpathclose%
\pgfusepath{stroke,fill}%
\end{pgfscope}%
\begin{pgfscope}%
\pgfpathrectangle{\pgfqpoint{0.100000in}{0.212622in}}{\pgfqpoint{3.696000in}{3.696000in}}%
\pgfusepath{clip}%
\pgfsetbuttcap%
\pgfsetroundjoin%
\definecolor{currentfill}{rgb}{0.121569,0.466667,0.705882}%
\pgfsetfillcolor{currentfill}%
\pgfsetfillopacity{0.924508}%
\pgfsetlinewidth{1.003750pt}%
\definecolor{currentstroke}{rgb}{0.121569,0.466667,0.705882}%
\pgfsetstrokecolor{currentstroke}%
\pgfsetstrokeopacity{0.924508}%
\pgfsetdash{}{0pt}%
\pgfpathmoveto{\pgfqpoint{2.573391in}{1.908973in}}%
\pgfpathcurveto{\pgfqpoint{2.581627in}{1.908973in}}{\pgfqpoint{2.589527in}{1.912245in}}{\pgfqpoint{2.595351in}{1.918069in}}%
\pgfpathcurveto{\pgfqpoint{2.601175in}{1.923893in}}{\pgfqpoint{2.604447in}{1.931793in}}{\pgfqpoint{2.604447in}{1.940029in}}%
\pgfpathcurveto{\pgfqpoint{2.604447in}{1.948265in}}{\pgfqpoint{2.601175in}{1.956165in}}{\pgfqpoint{2.595351in}{1.961989in}}%
\pgfpathcurveto{\pgfqpoint{2.589527in}{1.967813in}}{\pgfqpoint{2.581627in}{1.971086in}}{\pgfqpoint{2.573391in}{1.971086in}}%
\pgfpathcurveto{\pgfqpoint{2.565155in}{1.971086in}}{\pgfqpoint{2.557254in}{1.967813in}}{\pgfqpoint{2.551431in}{1.961989in}}%
\pgfpathcurveto{\pgfqpoint{2.545607in}{1.956165in}}{\pgfqpoint{2.542334in}{1.948265in}}{\pgfqpoint{2.542334in}{1.940029in}}%
\pgfpathcurveto{\pgfqpoint{2.542334in}{1.931793in}}{\pgfqpoint{2.545607in}{1.923893in}}{\pgfqpoint{2.551431in}{1.918069in}}%
\pgfpathcurveto{\pgfqpoint{2.557254in}{1.912245in}}{\pgfqpoint{2.565155in}{1.908973in}}{\pgfqpoint{2.573391in}{1.908973in}}%
\pgfpathclose%
\pgfusepath{stroke,fill}%
\end{pgfscope}%
\begin{pgfscope}%
\pgfpathrectangle{\pgfqpoint{0.100000in}{0.212622in}}{\pgfqpoint{3.696000in}{3.696000in}}%
\pgfusepath{clip}%
\pgfsetbuttcap%
\pgfsetroundjoin%
\definecolor{currentfill}{rgb}{0.121569,0.466667,0.705882}%
\pgfsetfillcolor{currentfill}%
\pgfsetfillopacity{0.925524}%
\pgfsetlinewidth{1.003750pt}%
\definecolor{currentstroke}{rgb}{0.121569,0.466667,0.705882}%
\pgfsetstrokecolor{currentstroke}%
\pgfsetstrokeopacity{0.925524}%
\pgfsetdash{}{0pt}%
\pgfpathmoveto{\pgfqpoint{1.788430in}{2.100987in}}%
\pgfpathcurveto{\pgfqpoint{1.796667in}{2.100987in}}{\pgfqpoint{1.804567in}{2.104259in}}{\pgfqpoint{1.810391in}{2.110083in}}%
\pgfpathcurveto{\pgfqpoint{1.816214in}{2.115907in}}{\pgfqpoint{1.819487in}{2.123807in}}{\pgfqpoint{1.819487in}{2.132043in}}%
\pgfpathcurveto{\pgfqpoint{1.819487in}{2.140279in}}{\pgfqpoint{1.816214in}{2.148179in}}{\pgfqpoint{1.810391in}{2.154003in}}%
\pgfpathcurveto{\pgfqpoint{1.804567in}{2.159827in}}{\pgfqpoint{1.796667in}{2.163100in}}{\pgfqpoint{1.788430in}{2.163100in}}%
\pgfpathcurveto{\pgfqpoint{1.780194in}{2.163100in}}{\pgfqpoint{1.772294in}{2.159827in}}{\pgfqpoint{1.766470in}{2.154003in}}%
\pgfpathcurveto{\pgfqpoint{1.760646in}{2.148179in}}{\pgfqpoint{1.757374in}{2.140279in}}{\pgfqpoint{1.757374in}{2.132043in}}%
\pgfpathcurveto{\pgfqpoint{1.757374in}{2.123807in}}{\pgfqpoint{1.760646in}{2.115907in}}{\pgfqpoint{1.766470in}{2.110083in}}%
\pgfpathcurveto{\pgfqpoint{1.772294in}{2.104259in}}{\pgfqpoint{1.780194in}{2.100987in}}{\pgfqpoint{1.788430in}{2.100987in}}%
\pgfpathclose%
\pgfusepath{stroke,fill}%
\end{pgfscope}%
\begin{pgfscope}%
\pgfpathrectangle{\pgfqpoint{0.100000in}{0.212622in}}{\pgfqpoint{3.696000in}{3.696000in}}%
\pgfusepath{clip}%
\pgfsetbuttcap%
\pgfsetroundjoin%
\definecolor{currentfill}{rgb}{0.121569,0.466667,0.705882}%
\pgfsetfillcolor{currentfill}%
\pgfsetfillopacity{0.925898}%
\pgfsetlinewidth{1.003750pt}%
\definecolor{currentstroke}{rgb}{0.121569,0.466667,0.705882}%
\pgfsetstrokecolor{currentstroke}%
\pgfsetstrokeopacity{0.925898}%
\pgfsetdash{}{0pt}%
\pgfpathmoveto{\pgfqpoint{1.806982in}{2.089156in}}%
\pgfpathcurveto{\pgfqpoint{1.815218in}{2.089156in}}{\pgfqpoint{1.823118in}{2.092428in}}{\pgfqpoint{1.828942in}{2.098252in}}%
\pgfpathcurveto{\pgfqpoint{1.834766in}{2.104076in}}{\pgfqpoint{1.838038in}{2.111976in}}{\pgfqpoint{1.838038in}{2.120212in}}%
\pgfpathcurveto{\pgfqpoint{1.838038in}{2.128448in}}{\pgfqpoint{1.834766in}{2.136348in}}{\pgfqpoint{1.828942in}{2.142172in}}%
\pgfpathcurveto{\pgfqpoint{1.823118in}{2.147996in}}{\pgfqpoint{1.815218in}{2.151269in}}{\pgfqpoint{1.806982in}{2.151269in}}%
\pgfpathcurveto{\pgfqpoint{1.798746in}{2.151269in}}{\pgfqpoint{1.790846in}{2.147996in}}{\pgfqpoint{1.785022in}{2.142172in}}%
\pgfpathcurveto{\pgfqpoint{1.779198in}{2.136348in}}{\pgfqpoint{1.775925in}{2.128448in}}{\pgfqpoint{1.775925in}{2.120212in}}%
\pgfpathcurveto{\pgfqpoint{1.775925in}{2.111976in}}{\pgfqpoint{1.779198in}{2.104076in}}{\pgfqpoint{1.785022in}{2.098252in}}%
\pgfpathcurveto{\pgfqpoint{1.790846in}{2.092428in}}{\pgfqpoint{1.798746in}{2.089156in}}{\pgfqpoint{1.806982in}{2.089156in}}%
\pgfpathclose%
\pgfusepath{stroke,fill}%
\end{pgfscope}%
\begin{pgfscope}%
\pgfpathrectangle{\pgfqpoint{0.100000in}{0.212622in}}{\pgfqpoint{3.696000in}{3.696000in}}%
\pgfusepath{clip}%
\pgfsetbuttcap%
\pgfsetroundjoin%
\definecolor{currentfill}{rgb}{0.121569,0.466667,0.705882}%
\pgfsetfillcolor{currentfill}%
\pgfsetfillopacity{0.927838}%
\pgfsetlinewidth{1.003750pt}%
\definecolor{currentstroke}{rgb}{0.121569,0.466667,0.705882}%
\pgfsetstrokecolor{currentstroke}%
\pgfsetstrokeopacity{0.927838}%
\pgfsetdash{}{0pt}%
\pgfpathmoveto{\pgfqpoint{2.566111in}{1.902356in}}%
\pgfpathcurveto{\pgfqpoint{2.574347in}{1.902356in}}{\pgfqpoint{2.582247in}{1.905628in}}{\pgfqpoint{2.588071in}{1.911452in}}%
\pgfpathcurveto{\pgfqpoint{2.593895in}{1.917276in}}{\pgfqpoint{2.597167in}{1.925176in}}{\pgfqpoint{2.597167in}{1.933413in}}%
\pgfpathcurveto{\pgfqpoint{2.597167in}{1.941649in}}{\pgfqpoint{2.593895in}{1.949549in}}{\pgfqpoint{2.588071in}{1.955373in}}%
\pgfpathcurveto{\pgfqpoint{2.582247in}{1.961197in}}{\pgfqpoint{2.574347in}{1.964469in}}{\pgfqpoint{2.566111in}{1.964469in}}%
\pgfpathcurveto{\pgfqpoint{2.557875in}{1.964469in}}{\pgfqpoint{2.549975in}{1.961197in}}{\pgfqpoint{2.544151in}{1.955373in}}%
\pgfpathcurveto{\pgfqpoint{2.538327in}{1.949549in}}{\pgfqpoint{2.535054in}{1.941649in}}{\pgfqpoint{2.535054in}{1.933413in}}%
\pgfpathcurveto{\pgfqpoint{2.535054in}{1.925176in}}{\pgfqpoint{2.538327in}{1.917276in}}{\pgfqpoint{2.544151in}{1.911452in}}%
\pgfpathcurveto{\pgfqpoint{2.549975in}{1.905628in}}{\pgfqpoint{2.557875in}{1.902356in}}{\pgfqpoint{2.566111in}{1.902356in}}%
\pgfpathclose%
\pgfusepath{stroke,fill}%
\end{pgfscope}%
\begin{pgfscope}%
\pgfpathrectangle{\pgfqpoint{0.100000in}{0.212622in}}{\pgfqpoint{3.696000in}{3.696000in}}%
\pgfusepath{clip}%
\pgfsetbuttcap%
\pgfsetroundjoin%
\definecolor{currentfill}{rgb}{0.121569,0.466667,0.705882}%
\pgfsetfillcolor{currentfill}%
\pgfsetfillopacity{0.929750}%
\pgfsetlinewidth{1.003750pt}%
\definecolor{currentstroke}{rgb}{0.121569,0.466667,0.705882}%
\pgfsetstrokecolor{currentstroke}%
\pgfsetstrokeopacity{0.929750}%
\pgfsetdash{}{0pt}%
\pgfpathmoveto{\pgfqpoint{2.561468in}{1.899956in}}%
\pgfpathcurveto{\pgfqpoint{2.569704in}{1.899956in}}{\pgfqpoint{2.577604in}{1.903228in}}{\pgfqpoint{2.583428in}{1.909052in}}%
\pgfpathcurveto{\pgfqpoint{2.589252in}{1.914876in}}{\pgfqpoint{2.592524in}{1.922776in}}{\pgfqpoint{2.592524in}{1.931013in}}%
\pgfpathcurveto{\pgfqpoint{2.592524in}{1.939249in}}{\pgfqpoint{2.589252in}{1.947149in}}{\pgfqpoint{2.583428in}{1.952973in}}%
\pgfpathcurveto{\pgfqpoint{2.577604in}{1.958797in}}{\pgfqpoint{2.569704in}{1.962069in}}{\pgfqpoint{2.561468in}{1.962069in}}%
\pgfpathcurveto{\pgfqpoint{2.553231in}{1.962069in}}{\pgfqpoint{2.545331in}{1.958797in}}{\pgfqpoint{2.539507in}{1.952973in}}%
\pgfpathcurveto{\pgfqpoint{2.533684in}{1.947149in}}{\pgfqpoint{2.530411in}{1.939249in}}{\pgfqpoint{2.530411in}{1.931013in}}%
\pgfpathcurveto{\pgfqpoint{2.530411in}{1.922776in}}{\pgfqpoint{2.533684in}{1.914876in}}{\pgfqpoint{2.539507in}{1.909052in}}%
\pgfpathcurveto{\pgfqpoint{2.545331in}{1.903228in}}{\pgfqpoint{2.553231in}{1.899956in}}{\pgfqpoint{2.561468in}{1.899956in}}%
\pgfpathclose%
\pgfusepath{stroke,fill}%
\end{pgfscope}%
\begin{pgfscope}%
\pgfpathrectangle{\pgfqpoint{0.100000in}{0.212622in}}{\pgfqpoint{3.696000in}{3.696000in}}%
\pgfusepath{clip}%
\pgfsetbuttcap%
\pgfsetroundjoin%
\definecolor{currentfill}{rgb}{0.121569,0.466667,0.705882}%
\pgfsetfillcolor{currentfill}%
\pgfsetfillopacity{0.930833}%
\pgfsetlinewidth{1.003750pt}%
\definecolor{currentstroke}{rgb}{0.121569,0.466667,0.705882}%
\pgfsetstrokecolor{currentstroke}%
\pgfsetstrokeopacity{0.930833}%
\pgfsetdash{}{0pt}%
\pgfpathmoveto{\pgfqpoint{2.558921in}{1.898864in}}%
\pgfpathcurveto{\pgfqpoint{2.567157in}{1.898864in}}{\pgfqpoint{2.575057in}{1.902136in}}{\pgfqpoint{2.580881in}{1.907960in}}%
\pgfpathcurveto{\pgfqpoint{2.586705in}{1.913784in}}{\pgfqpoint{2.589977in}{1.921684in}}{\pgfqpoint{2.589977in}{1.929920in}}%
\pgfpathcurveto{\pgfqpoint{2.589977in}{1.938156in}}{\pgfqpoint{2.586705in}{1.946056in}}{\pgfqpoint{2.580881in}{1.951880in}}%
\pgfpathcurveto{\pgfqpoint{2.575057in}{1.957704in}}{\pgfqpoint{2.567157in}{1.960977in}}{\pgfqpoint{2.558921in}{1.960977in}}%
\pgfpathcurveto{\pgfqpoint{2.550685in}{1.960977in}}{\pgfqpoint{2.542785in}{1.957704in}}{\pgfqpoint{2.536961in}{1.951880in}}%
\pgfpathcurveto{\pgfqpoint{2.531137in}{1.946056in}}{\pgfqpoint{2.527864in}{1.938156in}}{\pgfqpoint{2.527864in}{1.929920in}}%
\pgfpathcurveto{\pgfqpoint{2.527864in}{1.921684in}}{\pgfqpoint{2.531137in}{1.913784in}}{\pgfqpoint{2.536961in}{1.907960in}}%
\pgfpathcurveto{\pgfqpoint{2.542785in}{1.902136in}}{\pgfqpoint{2.550685in}{1.898864in}}{\pgfqpoint{2.558921in}{1.898864in}}%
\pgfpathclose%
\pgfusepath{stroke,fill}%
\end{pgfscope}%
\begin{pgfscope}%
\pgfpathrectangle{\pgfqpoint{0.100000in}{0.212622in}}{\pgfqpoint{3.696000in}{3.696000in}}%
\pgfusepath{clip}%
\pgfsetbuttcap%
\pgfsetroundjoin%
\definecolor{currentfill}{rgb}{0.121569,0.466667,0.705882}%
\pgfsetfillcolor{currentfill}%
\pgfsetfillopacity{0.931440}%
\pgfsetlinewidth{1.003750pt}%
\definecolor{currentstroke}{rgb}{0.121569,0.466667,0.705882}%
\pgfsetstrokecolor{currentstroke}%
\pgfsetstrokeopacity{0.931440}%
\pgfsetdash{}{0pt}%
\pgfpathmoveto{\pgfqpoint{2.557552in}{1.898270in}}%
\pgfpathcurveto{\pgfqpoint{2.565788in}{1.898270in}}{\pgfqpoint{2.573688in}{1.901542in}}{\pgfqpoint{2.579512in}{1.907366in}}%
\pgfpathcurveto{\pgfqpoint{2.585336in}{1.913190in}}{\pgfqpoint{2.588608in}{1.921090in}}{\pgfqpoint{2.588608in}{1.929327in}}%
\pgfpathcurveto{\pgfqpoint{2.588608in}{1.937563in}}{\pgfqpoint{2.585336in}{1.945463in}}{\pgfqpoint{2.579512in}{1.951287in}}%
\pgfpathcurveto{\pgfqpoint{2.573688in}{1.957111in}}{\pgfqpoint{2.565788in}{1.960383in}}{\pgfqpoint{2.557552in}{1.960383in}}%
\pgfpathcurveto{\pgfqpoint{2.549316in}{1.960383in}}{\pgfqpoint{2.541416in}{1.957111in}}{\pgfqpoint{2.535592in}{1.951287in}}%
\pgfpathcurveto{\pgfqpoint{2.529768in}{1.945463in}}{\pgfqpoint{2.526495in}{1.937563in}}{\pgfqpoint{2.526495in}{1.929327in}}%
\pgfpathcurveto{\pgfqpoint{2.526495in}{1.921090in}}{\pgfqpoint{2.529768in}{1.913190in}}{\pgfqpoint{2.535592in}{1.907366in}}%
\pgfpathcurveto{\pgfqpoint{2.541416in}{1.901542in}}{\pgfqpoint{2.549316in}{1.898270in}}{\pgfqpoint{2.557552in}{1.898270in}}%
\pgfpathclose%
\pgfusepath{stroke,fill}%
\end{pgfscope}%
\begin{pgfscope}%
\pgfpathrectangle{\pgfqpoint{0.100000in}{0.212622in}}{\pgfqpoint{3.696000in}{3.696000in}}%
\pgfusepath{clip}%
\pgfsetbuttcap%
\pgfsetroundjoin%
\definecolor{currentfill}{rgb}{0.121569,0.466667,0.705882}%
\pgfsetfillcolor{currentfill}%
\pgfsetfillopacity{0.931767}%
\pgfsetlinewidth{1.003750pt}%
\definecolor{currentstroke}{rgb}{0.121569,0.466667,0.705882}%
\pgfsetstrokecolor{currentstroke}%
\pgfsetstrokeopacity{0.931767}%
\pgfsetdash{}{0pt}%
\pgfpathmoveto{\pgfqpoint{2.556648in}{1.898145in}}%
\pgfpathcurveto{\pgfqpoint{2.564884in}{1.898145in}}{\pgfqpoint{2.572784in}{1.901418in}}{\pgfqpoint{2.578608in}{1.907242in}}%
\pgfpathcurveto{\pgfqpoint{2.584432in}{1.913066in}}{\pgfqpoint{2.587704in}{1.920966in}}{\pgfqpoint{2.587704in}{1.929202in}}%
\pgfpathcurveto{\pgfqpoint{2.587704in}{1.937438in}}{\pgfqpoint{2.584432in}{1.945338in}}{\pgfqpoint{2.578608in}{1.951162in}}%
\pgfpathcurveto{\pgfqpoint{2.572784in}{1.956986in}}{\pgfqpoint{2.564884in}{1.960258in}}{\pgfqpoint{2.556648in}{1.960258in}}%
\pgfpathcurveto{\pgfqpoint{2.548412in}{1.960258in}}{\pgfqpoint{2.540512in}{1.956986in}}{\pgfqpoint{2.534688in}{1.951162in}}%
\pgfpathcurveto{\pgfqpoint{2.528864in}{1.945338in}}{\pgfqpoint{2.525591in}{1.937438in}}{\pgfqpoint{2.525591in}{1.929202in}}%
\pgfpathcurveto{\pgfqpoint{2.525591in}{1.920966in}}{\pgfqpoint{2.528864in}{1.913066in}}{\pgfqpoint{2.534688in}{1.907242in}}%
\pgfpathcurveto{\pgfqpoint{2.540512in}{1.901418in}}{\pgfqpoint{2.548412in}{1.898145in}}{\pgfqpoint{2.556648in}{1.898145in}}%
\pgfpathclose%
\pgfusepath{stroke,fill}%
\end{pgfscope}%
\begin{pgfscope}%
\pgfpathrectangle{\pgfqpoint{0.100000in}{0.212622in}}{\pgfqpoint{3.696000in}{3.696000in}}%
\pgfusepath{clip}%
\pgfsetbuttcap%
\pgfsetroundjoin%
\definecolor{currentfill}{rgb}{0.121569,0.466667,0.705882}%
\pgfsetfillcolor{currentfill}%
\pgfsetfillopacity{0.931925}%
\pgfsetlinewidth{1.003750pt}%
\definecolor{currentstroke}{rgb}{0.121569,0.466667,0.705882}%
\pgfsetstrokecolor{currentstroke}%
\pgfsetstrokeopacity{0.931925}%
\pgfsetdash{}{0pt}%
\pgfpathmoveto{\pgfqpoint{2.556274in}{1.897753in}}%
\pgfpathcurveto{\pgfqpoint{2.564510in}{1.897753in}}{\pgfqpoint{2.572410in}{1.901025in}}{\pgfqpoint{2.578234in}{1.906849in}}%
\pgfpathcurveto{\pgfqpoint{2.584058in}{1.912673in}}{\pgfqpoint{2.587330in}{1.920573in}}{\pgfqpoint{2.587330in}{1.928810in}}%
\pgfpathcurveto{\pgfqpoint{2.587330in}{1.937046in}}{\pgfqpoint{2.584058in}{1.944946in}}{\pgfqpoint{2.578234in}{1.950770in}}%
\pgfpathcurveto{\pgfqpoint{2.572410in}{1.956594in}}{\pgfqpoint{2.564510in}{1.959866in}}{\pgfqpoint{2.556274in}{1.959866in}}%
\pgfpathcurveto{\pgfqpoint{2.548037in}{1.959866in}}{\pgfqpoint{2.540137in}{1.956594in}}{\pgfqpoint{2.534313in}{1.950770in}}%
\pgfpathcurveto{\pgfqpoint{2.528490in}{1.944946in}}{\pgfqpoint{2.525217in}{1.937046in}}{\pgfqpoint{2.525217in}{1.928810in}}%
\pgfpathcurveto{\pgfqpoint{2.525217in}{1.920573in}}{\pgfqpoint{2.528490in}{1.912673in}}{\pgfqpoint{2.534313in}{1.906849in}}%
\pgfpathcurveto{\pgfqpoint{2.540137in}{1.901025in}}{\pgfqpoint{2.548037in}{1.897753in}}{\pgfqpoint{2.556274in}{1.897753in}}%
\pgfpathclose%
\pgfusepath{stroke,fill}%
\end{pgfscope}%
\begin{pgfscope}%
\pgfpathrectangle{\pgfqpoint{0.100000in}{0.212622in}}{\pgfqpoint{3.696000in}{3.696000in}}%
\pgfusepath{clip}%
\pgfsetbuttcap%
\pgfsetroundjoin%
\definecolor{currentfill}{rgb}{0.121569,0.466667,0.705882}%
\pgfsetfillcolor{currentfill}%
\pgfsetfillopacity{0.932050}%
\pgfsetlinewidth{1.003750pt}%
\definecolor{currentstroke}{rgb}{0.121569,0.466667,0.705882}%
\pgfsetstrokecolor{currentstroke}%
\pgfsetstrokeopacity{0.932050}%
\pgfsetdash{}{0pt}%
\pgfpathmoveto{\pgfqpoint{2.556089in}{1.897756in}}%
\pgfpathcurveto{\pgfqpoint{2.564325in}{1.897756in}}{\pgfqpoint{2.572225in}{1.901028in}}{\pgfqpoint{2.578049in}{1.906852in}}%
\pgfpathcurveto{\pgfqpoint{2.583873in}{1.912676in}}{\pgfqpoint{2.587145in}{1.920576in}}{\pgfqpoint{2.587145in}{1.928813in}}%
\pgfpathcurveto{\pgfqpoint{2.587145in}{1.937049in}}{\pgfqpoint{2.583873in}{1.944949in}}{\pgfqpoint{2.578049in}{1.950773in}}%
\pgfpathcurveto{\pgfqpoint{2.572225in}{1.956597in}}{\pgfqpoint{2.564325in}{1.959869in}}{\pgfqpoint{2.556089in}{1.959869in}}%
\pgfpathcurveto{\pgfqpoint{2.547852in}{1.959869in}}{\pgfqpoint{2.539952in}{1.956597in}}{\pgfqpoint{2.534129in}{1.950773in}}%
\pgfpathcurveto{\pgfqpoint{2.528305in}{1.944949in}}{\pgfqpoint{2.525032in}{1.937049in}}{\pgfqpoint{2.525032in}{1.928813in}}%
\pgfpathcurveto{\pgfqpoint{2.525032in}{1.920576in}}{\pgfqpoint{2.528305in}{1.912676in}}{\pgfqpoint{2.534129in}{1.906852in}}%
\pgfpathcurveto{\pgfqpoint{2.539952in}{1.901028in}}{\pgfqpoint{2.547852in}{1.897756in}}{\pgfqpoint{2.556089in}{1.897756in}}%
\pgfpathclose%
\pgfusepath{stroke,fill}%
\end{pgfscope}%
\begin{pgfscope}%
\pgfpathrectangle{\pgfqpoint{0.100000in}{0.212622in}}{\pgfqpoint{3.696000in}{3.696000in}}%
\pgfusepath{clip}%
\pgfsetbuttcap%
\pgfsetroundjoin%
\definecolor{currentfill}{rgb}{0.121569,0.466667,0.705882}%
\pgfsetfillcolor{currentfill}%
\pgfsetfillopacity{0.932951}%
\pgfsetlinewidth{1.003750pt}%
\definecolor{currentstroke}{rgb}{0.121569,0.466667,0.705882}%
\pgfsetstrokecolor{currentstroke}%
\pgfsetstrokeopacity{0.932951}%
\pgfsetdash{}{0pt}%
\pgfpathmoveto{\pgfqpoint{2.554277in}{1.895562in}}%
\pgfpathcurveto{\pgfqpoint{2.562513in}{1.895562in}}{\pgfqpoint{2.570413in}{1.898835in}}{\pgfqpoint{2.576237in}{1.904658in}}%
\pgfpathcurveto{\pgfqpoint{2.582061in}{1.910482in}}{\pgfqpoint{2.585334in}{1.918382in}}{\pgfqpoint{2.585334in}{1.926619in}}%
\pgfpathcurveto{\pgfqpoint{2.585334in}{1.934855in}}{\pgfqpoint{2.582061in}{1.942755in}}{\pgfqpoint{2.576237in}{1.948579in}}%
\pgfpathcurveto{\pgfqpoint{2.570413in}{1.954403in}}{\pgfqpoint{2.562513in}{1.957675in}}{\pgfqpoint{2.554277in}{1.957675in}}%
\pgfpathcurveto{\pgfqpoint{2.546041in}{1.957675in}}{\pgfqpoint{2.538141in}{1.954403in}}{\pgfqpoint{2.532317in}{1.948579in}}%
\pgfpathcurveto{\pgfqpoint{2.526493in}{1.942755in}}{\pgfqpoint{2.523221in}{1.934855in}}{\pgfqpoint{2.523221in}{1.926619in}}%
\pgfpathcurveto{\pgfqpoint{2.523221in}{1.918382in}}{\pgfqpoint{2.526493in}{1.910482in}}{\pgfqpoint{2.532317in}{1.904658in}}%
\pgfpathcurveto{\pgfqpoint{2.538141in}{1.898835in}}{\pgfqpoint{2.546041in}{1.895562in}}{\pgfqpoint{2.554277in}{1.895562in}}%
\pgfpathclose%
\pgfusepath{stroke,fill}%
\end{pgfscope}%
\begin{pgfscope}%
\pgfpathrectangle{\pgfqpoint{0.100000in}{0.212622in}}{\pgfqpoint{3.696000in}{3.696000in}}%
\pgfusepath{clip}%
\pgfsetbuttcap%
\pgfsetroundjoin%
\definecolor{currentfill}{rgb}{0.121569,0.466667,0.705882}%
\pgfsetfillcolor{currentfill}%
\pgfsetfillopacity{0.932955}%
\pgfsetlinewidth{1.003750pt}%
\definecolor{currentstroke}{rgb}{0.121569,0.466667,0.705882}%
\pgfsetstrokecolor{currentstroke}%
\pgfsetstrokeopacity{0.932955}%
\pgfsetdash{}{0pt}%
\pgfpathmoveto{\pgfqpoint{1.833729in}{2.084853in}}%
\pgfpathcurveto{\pgfqpoint{1.841965in}{2.084853in}}{\pgfqpoint{1.849865in}{2.088125in}}{\pgfqpoint{1.855689in}{2.093949in}}%
\pgfpathcurveto{\pgfqpoint{1.861513in}{2.099773in}}{\pgfqpoint{1.864785in}{2.107673in}}{\pgfqpoint{1.864785in}{2.115909in}}%
\pgfpathcurveto{\pgfqpoint{1.864785in}{2.124145in}}{\pgfqpoint{1.861513in}{2.132046in}}{\pgfqpoint{1.855689in}{2.137869in}}%
\pgfpathcurveto{\pgfqpoint{1.849865in}{2.143693in}}{\pgfqpoint{1.841965in}{2.146966in}}{\pgfqpoint{1.833729in}{2.146966in}}%
\pgfpathcurveto{\pgfqpoint{1.825493in}{2.146966in}}{\pgfqpoint{1.817593in}{2.143693in}}{\pgfqpoint{1.811769in}{2.137869in}}%
\pgfpathcurveto{\pgfqpoint{1.805945in}{2.132046in}}{\pgfqpoint{1.802672in}{2.124145in}}{\pgfqpoint{1.802672in}{2.115909in}}%
\pgfpathcurveto{\pgfqpoint{1.802672in}{2.107673in}}{\pgfqpoint{1.805945in}{2.099773in}}{\pgfqpoint{1.811769in}{2.093949in}}%
\pgfpathcurveto{\pgfqpoint{1.817593in}{2.088125in}}{\pgfqpoint{1.825493in}{2.084853in}}{\pgfqpoint{1.833729in}{2.084853in}}%
\pgfpathclose%
\pgfusepath{stroke,fill}%
\end{pgfscope}%
\begin{pgfscope}%
\pgfpathrectangle{\pgfqpoint{0.100000in}{0.212622in}}{\pgfqpoint{3.696000in}{3.696000in}}%
\pgfusepath{clip}%
\pgfsetbuttcap%
\pgfsetroundjoin%
\definecolor{currentfill}{rgb}{0.121569,0.466667,0.705882}%
\pgfsetfillcolor{currentfill}%
\pgfsetfillopacity{0.934186}%
\pgfsetlinewidth{1.003750pt}%
\definecolor{currentstroke}{rgb}{0.121569,0.466667,0.705882}%
\pgfsetstrokecolor{currentstroke}%
\pgfsetstrokeopacity{0.934186}%
\pgfsetdash{}{0pt}%
\pgfpathmoveto{\pgfqpoint{1.861053in}{2.064345in}}%
\pgfpathcurveto{\pgfqpoint{1.869290in}{2.064345in}}{\pgfqpoint{1.877190in}{2.067617in}}{\pgfqpoint{1.883014in}{2.073441in}}%
\pgfpathcurveto{\pgfqpoint{1.888838in}{2.079265in}}{\pgfqpoint{1.892110in}{2.087165in}}{\pgfqpoint{1.892110in}{2.095401in}}%
\pgfpathcurveto{\pgfqpoint{1.892110in}{2.103638in}}{\pgfqpoint{1.888838in}{2.111538in}}{\pgfqpoint{1.883014in}{2.117362in}}%
\pgfpathcurveto{\pgfqpoint{1.877190in}{2.123186in}}{\pgfqpoint{1.869290in}{2.126458in}}{\pgfqpoint{1.861053in}{2.126458in}}%
\pgfpathcurveto{\pgfqpoint{1.852817in}{2.126458in}}{\pgfqpoint{1.844917in}{2.123186in}}{\pgfqpoint{1.839093in}{2.117362in}}%
\pgfpathcurveto{\pgfqpoint{1.833269in}{2.111538in}}{\pgfqpoint{1.829997in}{2.103638in}}{\pgfqpoint{1.829997in}{2.095401in}}%
\pgfpathcurveto{\pgfqpoint{1.829997in}{2.087165in}}{\pgfqpoint{1.833269in}{2.079265in}}{\pgfqpoint{1.839093in}{2.073441in}}%
\pgfpathcurveto{\pgfqpoint{1.844917in}{2.067617in}}{\pgfqpoint{1.852817in}{2.064345in}}{\pgfqpoint{1.861053in}{2.064345in}}%
\pgfpathclose%
\pgfusepath{stroke,fill}%
\end{pgfscope}%
\begin{pgfscope}%
\pgfpathrectangle{\pgfqpoint{0.100000in}{0.212622in}}{\pgfqpoint{3.696000in}{3.696000in}}%
\pgfusepath{clip}%
\pgfsetbuttcap%
\pgfsetroundjoin%
\definecolor{currentfill}{rgb}{0.121569,0.466667,0.705882}%
\pgfsetfillcolor{currentfill}%
\pgfsetfillopacity{0.934976}%
\pgfsetlinewidth{1.003750pt}%
\definecolor{currentstroke}{rgb}{0.121569,0.466667,0.705882}%
\pgfsetstrokecolor{currentstroke}%
\pgfsetstrokeopacity{0.934976}%
\pgfsetdash{}{0pt}%
\pgfpathmoveto{\pgfqpoint{2.549799in}{1.894055in}}%
\pgfpathcurveto{\pgfqpoint{2.558036in}{1.894055in}}{\pgfqpoint{2.565936in}{1.897328in}}{\pgfqpoint{2.571760in}{1.903151in}}%
\pgfpathcurveto{\pgfqpoint{2.577584in}{1.908975in}}{\pgfqpoint{2.580856in}{1.916875in}}{\pgfqpoint{2.580856in}{1.925112in}}%
\pgfpathcurveto{\pgfqpoint{2.580856in}{1.933348in}}{\pgfqpoint{2.577584in}{1.941248in}}{\pgfqpoint{2.571760in}{1.947072in}}%
\pgfpathcurveto{\pgfqpoint{2.565936in}{1.952896in}}{\pgfqpoint{2.558036in}{1.956168in}}{\pgfqpoint{2.549799in}{1.956168in}}%
\pgfpathcurveto{\pgfqpoint{2.541563in}{1.956168in}}{\pgfqpoint{2.533663in}{1.952896in}}{\pgfqpoint{2.527839in}{1.947072in}}%
\pgfpathcurveto{\pgfqpoint{2.522015in}{1.941248in}}{\pgfqpoint{2.518743in}{1.933348in}}{\pgfqpoint{2.518743in}{1.925112in}}%
\pgfpathcurveto{\pgfqpoint{2.518743in}{1.916875in}}{\pgfqpoint{2.522015in}{1.908975in}}{\pgfqpoint{2.527839in}{1.903151in}}%
\pgfpathcurveto{\pgfqpoint{2.533663in}{1.897328in}}{\pgfqpoint{2.541563in}{1.894055in}}{\pgfqpoint{2.549799in}{1.894055in}}%
\pgfpathclose%
\pgfusepath{stroke,fill}%
\end{pgfscope}%
\begin{pgfscope}%
\pgfpathrectangle{\pgfqpoint{0.100000in}{0.212622in}}{\pgfqpoint{3.696000in}{3.696000in}}%
\pgfusepath{clip}%
\pgfsetbuttcap%
\pgfsetroundjoin%
\definecolor{currentfill}{rgb}{0.121569,0.466667,0.705882}%
\pgfsetfillcolor{currentfill}%
\pgfsetfillopacity{0.937669}%
\pgfsetlinewidth{1.003750pt}%
\definecolor{currentstroke}{rgb}{0.121569,0.466667,0.705882}%
\pgfsetstrokecolor{currentstroke}%
\pgfsetstrokeopacity{0.937669}%
\pgfsetdash{}{0pt}%
\pgfpathmoveto{\pgfqpoint{1.879480in}{2.059647in}}%
\pgfpathcurveto{\pgfqpoint{1.887716in}{2.059647in}}{\pgfqpoint{1.895617in}{2.062919in}}{\pgfqpoint{1.901440in}{2.068743in}}%
\pgfpathcurveto{\pgfqpoint{1.907264in}{2.074567in}}{\pgfqpoint{1.910537in}{2.082467in}}{\pgfqpoint{1.910537in}{2.090704in}}%
\pgfpathcurveto{\pgfqpoint{1.910537in}{2.098940in}}{\pgfqpoint{1.907264in}{2.106840in}}{\pgfqpoint{1.901440in}{2.112664in}}%
\pgfpathcurveto{\pgfqpoint{1.895617in}{2.118488in}}{\pgfqpoint{1.887716in}{2.121760in}}{\pgfqpoint{1.879480in}{2.121760in}}%
\pgfpathcurveto{\pgfqpoint{1.871244in}{2.121760in}}{\pgfqpoint{1.863344in}{2.118488in}}{\pgfqpoint{1.857520in}{2.112664in}}%
\pgfpathcurveto{\pgfqpoint{1.851696in}{2.106840in}}{\pgfqpoint{1.848424in}{2.098940in}}{\pgfqpoint{1.848424in}{2.090704in}}%
\pgfpathcurveto{\pgfqpoint{1.848424in}{2.082467in}}{\pgfqpoint{1.851696in}{2.074567in}}{\pgfqpoint{1.857520in}{2.068743in}}%
\pgfpathcurveto{\pgfqpoint{1.863344in}{2.062919in}}{\pgfqpoint{1.871244in}{2.059647in}}{\pgfqpoint{1.879480in}{2.059647in}}%
\pgfpathclose%
\pgfusepath{stroke,fill}%
\end{pgfscope}%
\begin{pgfscope}%
\pgfpathrectangle{\pgfqpoint{0.100000in}{0.212622in}}{\pgfqpoint{3.696000in}{3.696000in}}%
\pgfusepath{clip}%
\pgfsetbuttcap%
\pgfsetroundjoin%
\definecolor{currentfill}{rgb}{0.121569,0.466667,0.705882}%
\pgfsetfillcolor{currentfill}%
\pgfsetfillopacity{0.937966}%
\pgfsetlinewidth{1.003750pt}%
\definecolor{currentstroke}{rgb}{0.121569,0.466667,0.705882}%
\pgfsetstrokecolor{currentstroke}%
\pgfsetstrokeopacity{0.937966}%
\pgfsetdash{}{0pt}%
\pgfpathmoveto{\pgfqpoint{2.542731in}{1.887782in}}%
\pgfpathcurveto{\pgfqpoint{2.550967in}{1.887782in}}{\pgfqpoint{2.558867in}{1.891055in}}{\pgfqpoint{2.564691in}{1.896879in}}%
\pgfpathcurveto{\pgfqpoint{2.570515in}{1.902702in}}{\pgfqpoint{2.573788in}{1.910603in}}{\pgfqpoint{2.573788in}{1.918839in}}%
\pgfpathcurveto{\pgfqpoint{2.573788in}{1.927075in}}{\pgfqpoint{2.570515in}{1.934975in}}{\pgfqpoint{2.564691in}{1.940799in}}%
\pgfpathcurveto{\pgfqpoint{2.558867in}{1.946623in}}{\pgfqpoint{2.550967in}{1.949895in}}{\pgfqpoint{2.542731in}{1.949895in}}%
\pgfpathcurveto{\pgfqpoint{2.534495in}{1.949895in}}{\pgfqpoint{2.526595in}{1.946623in}}{\pgfqpoint{2.520771in}{1.940799in}}%
\pgfpathcurveto{\pgfqpoint{2.514947in}{1.934975in}}{\pgfqpoint{2.511675in}{1.927075in}}{\pgfqpoint{2.511675in}{1.918839in}}%
\pgfpathcurveto{\pgfqpoint{2.511675in}{1.910603in}}{\pgfqpoint{2.514947in}{1.902702in}}{\pgfqpoint{2.520771in}{1.896879in}}%
\pgfpathcurveto{\pgfqpoint{2.526595in}{1.891055in}}{\pgfqpoint{2.534495in}{1.887782in}}{\pgfqpoint{2.542731in}{1.887782in}}%
\pgfpathclose%
\pgfusepath{stroke,fill}%
\end{pgfscope}%
\begin{pgfscope}%
\pgfpathrectangle{\pgfqpoint{0.100000in}{0.212622in}}{\pgfqpoint{3.696000in}{3.696000in}}%
\pgfusepath{clip}%
\pgfsetbuttcap%
\pgfsetroundjoin%
\definecolor{currentfill}{rgb}{0.121569,0.466667,0.705882}%
\pgfsetfillcolor{currentfill}%
\pgfsetfillopacity{0.938302}%
\pgfsetlinewidth{1.003750pt}%
\definecolor{currentstroke}{rgb}{0.121569,0.466667,0.705882}%
\pgfsetstrokecolor{currentstroke}%
\pgfsetstrokeopacity{0.938302}%
\pgfsetdash{}{0pt}%
\pgfpathmoveto{\pgfqpoint{1.895117in}{2.048968in}}%
\pgfpathcurveto{\pgfqpoint{1.903353in}{2.048968in}}{\pgfqpoint{1.911253in}{2.052240in}}{\pgfqpoint{1.917077in}{2.058064in}}%
\pgfpathcurveto{\pgfqpoint{1.922901in}{2.063888in}}{\pgfqpoint{1.926174in}{2.071788in}}{\pgfqpoint{1.926174in}{2.080024in}}%
\pgfpathcurveto{\pgfqpoint{1.926174in}{2.088260in}}{\pgfqpoint{1.922901in}{2.096160in}}{\pgfqpoint{1.917077in}{2.101984in}}%
\pgfpathcurveto{\pgfqpoint{1.911253in}{2.107808in}}{\pgfqpoint{1.903353in}{2.111081in}}{\pgfqpoint{1.895117in}{2.111081in}}%
\pgfpathcurveto{\pgfqpoint{1.886881in}{2.111081in}}{\pgfqpoint{1.878981in}{2.107808in}}{\pgfqpoint{1.873157in}{2.101984in}}%
\pgfpathcurveto{\pgfqpoint{1.867333in}{2.096160in}}{\pgfqpoint{1.864061in}{2.088260in}}{\pgfqpoint{1.864061in}{2.080024in}}%
\pgfpathcurveto{\pgfqpoint{1.864061in}{2.071788in}}{\pgfqpoint{1.867333in}{2.063888in}}{\pgfqpoint{1.873157in}{2.058064in}}%
\pgfpathcurveto{\pgfqpoint{1.878981in}{2.052240in}}{\pgfqpoint{1.886881in}{2.048968in}}{\pgfqpoint{1.895117in}{2.048968in}}%
\pgfpathclose%
\pgfusepath{stroke,fill}%
\end{pgfscope}%
\begin{pgfscope}%
\pgfpathrectangle{\pgfqpoint{0.100000in}{0.212622in}}{\pgfqpoint{3.696000in}{3.696000in}}%
\pgfusepath{clip}%
\pgfsetbuttcap%
\pgfsetroundjoin%
\definecolor{currentfill}{rgb}{0.121569,0.466667,0.705882}%
\pgfsetfillcolor{currentfill}%
\pgfsetfillopacity{0.940141}%
\pgfsetlinewidth{1.003750pt}%
\definecolor{currentstroke}{rgb}{0.121569,0.466667,0.705882}%
\pgfsetstrokecolor{currentstroke}%
\pgfsetstrokeopacity{0.940141}%
\pgfsetdash{}{0pt}%
\pgfpathmoveto{\pgfqpoint{1.906427in}{2.044691in}}%
\pgfpathcurveto{\pgfqpoint{1.914664in}{2.044691in}}{\pgfqpoint{1.922564in}{2.047963in}}{\pgfqpoint{1.928387in}{2.053787in}}%
\pgfpathcurveto{\pgfqpoint{1.934211in}{2.059611in}}{\pgfqpoint{1.937484in}{2.067511in}}{\pgfqpoint{1.937484in}{2.075747in}}%
\pgfpathcurveto{\pgfqpoint{1.937484in}{2.083983in}}{\pgfqpoint{1.934211in}{2.091883in}}{\pgfqpoint{1.928387in}{2.097707in}}%
\pgfpathcurveto{\pgfqpoint{1.922564in}{2.103531in}}{\pgfqpoint{1.914664in}{2.106804in}}{\pgfqpoint{1.906427in}{2.106804in}}%
\pgfpathcurveto{\pgfqpoint{1.898191in}{2.106804in}}{\pgfqpoint{1.890291in}{2.103531in}}{\pgfqpoint{1.884467in}{2.097707in}}%
\pgfpathcurveto{\pgfqpoint{1.878643in}{2.091883in}}{\pgfqpoint{1.875371in}{2.083983in}}{\pgfqpoint{1.875371in}{2.075747in}}%
\pgfpathcurveto{\pgfqpoint{1.875371in}{2.067511in}}{\pgfqpoint{1.878643in}{2.059611in}}{\pgfqpoint{1.884467in}{2.053787in}}%
\pgfpathcurveto{\pgfqpoint{1.890291in}{2.047963in}}{\pgfqpoint{1.898191in}{2.044691in}}{\pgfqpoint{1.906427in}{2.044691in}}%
\pgfpathclose%
\pgfusepath{stroke,fill}%
\end{pgfscope}%
\begin{pgfscope}%
\pgfpathrectangle{\pgfqpoint{0.100000in}{0.212622in}}{\pgfqpoint{3.696000in}{3.696000in}}%
\pgfusepath{clip}%
\pgfsetbuttcap%
\pgfsetroundjoin%
\definecolor{currentfill}{rgb}{0.121569,0.466667,0.705882}%
\pgfsetfillcolor{currentfill}%
\pgfsetfillopacity{0.940738}%
\pgfsetlinewidth{1.003750pt}%
\definecolor{currentstroke}{rgb}{0.121569,0.466667,0.705882}%
\pgfsetstrokecolor{currentstroke}%
\pgfsetstrokeopacity{0.940738}%
\pgfsetdash{}{0pt}%
\pgfpathmoveto{\pgfqpoint{1.913105in}{2.040803in}}%
\pgfpathcurveto{\pgfqpoint{1.921341in}{2.040803in}}{\pgfqpoint{1.929241in}{2.044075in}}{\pgfqpoint{1.935065in}{2.049899in}}%
\pgfpathcurveto{\pgfqpoint{1.940889in}{2.055723in}}{\pgfqpoint{1.944161in}{2.063623in}}{\pgfqpoint{1.944161in}{2.071860in}}%
\pgfpathcurveto{\pgfqpoint{1.944161in}{2.080096in}}{\pgfqpoint{1.940889in}{2.087996in}}{\pgfqpoint{1.935065in}{2.093820in}}%
\pgfpathcurveto{\pgfqpoint{1.929241in}{2.099644in}}{\pgfqpoint{1.921341in}{2.102916in}}{\pgfqpoint{1.913105in}{2.102916in}}%
\pgfpathcurveto{\pgfqpoint{1.904869in}{2.102916in}}{\pgfqpoint{1.896969in}{2.099644in}}{\pgfqpoint{1.891145in}{2.093820in}}%
\pgfpathcurveto{\pgfqpoint{1.885321in}{2.087996in}}{\pgfqpoint{1.882048in}{2.080096in}}{\pgfqpoint{1.882048in}{2.071860in}}%
\pgfpathcurveto{\pgfqpoint{1.882048in}{2.063623in}}{\pgfqpoint{1.885321in}{2.055723in}}{\pgfqpoint{1.891145in}{2.049899in}}%
\pgfpathcurveto{\pgfqpoint{1.896969in}{2.044075in}}{\pgfqpoint{1.904869in}{2.040803in}}{\pgfqpoint{1.913105in}{2.040803in}}%
\pgfpathclose%
\pgfusepath{stroke,fill}%
\end{pgfscope}%
\begin{pgfscope}%
\pgfpathrectangle{\pgfqpoint{0.100000in}{0.212622in}}{\pgfqpoint{3.696000in}{3.696000in}}%
\pgfusepath{clip}%
\pgfsetbuttcap%
\pgfsetroundjoin%
\definecolor{currentfill}{rgb}{0.121569,0.466667,0.705882}%
\pgfsetfillcolor{currentfill}%
\pgfsetfillopacity{0.942431}%
\pgfsetlinewidth{1.003750pt}%
\definecolor{currentstroke}{rgb}{0.121569,0.466667,0.705882}%
\pgfsetstrokecolor{currentstroke}%
\pgfsetstrokeopacity{0.942431}%
\pgfsetdash{}{0pt}%
\pgfpathmoveto{\pgfqpoint{2.529836in}{1.882708in}}%
\pgfpathcurveto{\pgfqpoint{2.538072in}{1.882708in}}{\pgfqpoint{2.545972in}{1.885980in}}{\pgfqpoint{2.551796in}{1.891804in}}%
\pgfpathcurveto{\pgfqpoint{2.557620in}{1.897628in}}{\pgfqpoint{2.560892in}{1.905528in}}{\pgfqpoint{2.560892in}{1.913764in}}%
\pgfpathcurveto{\pgfqpoint{2.560892in}{1.922000in}}{\pgfqpoint{2.557620in}{1.929900in}}{\pgfqpoint{2.551796in}{1.935724in}}%
\pgfpathcurveto{\pgfqpoint{2.545972in}{1.941548in}}{\pgfqpoint{2.538072in}{1.944821in}}{\pgfqpoint{2.529836in}{1.944821in}}%
\pgfpathcurveto{\pgfqpoint{2.521599in}{1.944821in}}{\pgfqpoint{2.513699in}{1.941548in}}{\pgfqpoint{2.507875in}{1.935724in}}%
\pgfpathcurveto{\pgfqpoint{2.502052in}{1.929900in}}{\pgfqpoint{2.498779in}{1.922000in}}{\pgfqpoint{2.498779in}{1.913764in}}%
\pgfpathcurveto{\pgfqpoint{2.498779in}{1.905528in}}{\pgfqpoint{2.502052in}{1.897628in}}{\pgfqpoint{2.507875in}{1.891804in}}%
\pgfpathcurveto{\pgfqpoint{2.513699in}{1.885980in}}{\pgfqpoint{2.521599in}{1.882708in}}{\pgfqpoint{2.529836in}{1.882708in}}%
\pgfpathclose%
\pgfusepath{stroke,fill}%
\end{pgfscope}%
\begin{pgfscope}%
\pgfpathrectangle{\pgfqpoint{0.100000in}{0.212622in}}{\pgfqpoint{3.696000in}{3.696000in}}%
\pgfusepath{clip}%
\pgfsetbuttcap%
\pgfsetroundjoin%
\definecolor{currentfill}{rgb}{0.121569,0.466667,0.705882}%
\pgfsetfillcolor{currentfill}%
\pgfsetfillopacity{0.942543}%
\pgfsetlinewidth{1.003750pt}%
\definecolor{currentstroke}{rgb}{0.121569,0.466667,0.705882}%
\pgfsetstrokecolor{currentstroke}%
\pgfsetstrokeopacity{0.942543}%
\pgfsetdash{}{0pt}%
\pgfpathmoveto{\pgfqpoint{1.924767in}{2.036554in}}%
\pgfpathcurveto{\pgfqpoint{1.933003in}{2.036554in}}{\pgfqpoint{1.940903in}{2.039826in}}{\pgfqpoint{1.946727in}{2.045650in}}%
\pgfpathcurveto{\pgfqpoint{1.952551in}{2.051474in}}{\pgfqpoint{1.955824in}{2.059374in}}{\pgfqpoint{1.955824in}{2.067610in}}%
\pgfpathcurveto{\pgfqpoint{1.955824in}{2.075847in}}{\pgfqpoint{1.952551in}{2.083747in}}{\pgfqpoint{1.946727in}{2.089571in}}%
\pgfpathcurveto{\pgfqpoint{1.940903in}{2.095395in}}{\pgfqpoint{1.933003in}{2.098667in}}{\pgfqpoint{1.924767in}{2.098667in}}%
\pgfpathcurveto{\pgfqpoint{1.916531in}{2.098667in}}{\pgfqpoint{1.908631in}{2.095395in}}{\pgfqpoint{1.902807in}{2.089571in}}%
\pgfpathcurveto{\pgfqpoint{1.896983in}{2.083747in}}{\pgfqpoint{1.893711in}{2.075847in}}{\pgfqpoint{1.893711in}{2.067610in}}%
\pgfpathcurveto{\pgfqpoint{1.893711in}{2.059374in}}{\pgfqpoint{1.896983in}{2.051474in}}{\pgfqpoint{1.902807in}{2.045650in}}%
\pgfpathcurveto{\pgfqpoint{1.908631in}{2.039826in}}{\pgfqpoint{1.916531in}{2.036554in}}{\pgfqpoint{1.924767in}{2.036554in}}%
\pgfpathclose%
\pgfusepath{stroke,fill}%
\end{pgfscope}%
\begin{pgfscope}%
\pgfpathrectangle{\pgfqpoint{0.100000in}{0.212622in}}{\pgfqpoint{3.696000in}{3.696000in}}%
\pgfusepath{clip}%
\pgfsetbuttcap%
\pgfsetroundjoin%
\definecolor{currentfill}{rgb}{0.121569,0.466667,0.705882}%
\pgfsetfillcolor{currentfill}%
\pgfsetfillopacity{0.944070}%
\pgfsetlinewidth{1.003750pt}%
\definecolor{currentstroke}{rgb}{0.121569,0.466667,0.705882}%
\pgfsetstrokecolor{currentstroke}%
\pgfsetstrokeopacity{0.944070}%
\pgfsetdash{}{0pt}%
\pgfpathmoveto{\pgfqpoint{1.947006in}{2.021177in}}%
\pgfpathcurveto{\pgfqpoint{1.955242in}{2.021177in}}{\pgfqpoint{1.963143in}{2.024449in}}{\pgfqpoint{1.968966in}{2.030273in}}%
\pgfpathcurveto{\pgfqpoint{1.974790in}{2.036097in}}{\pgfqpoint{1.978063in}{2.043997in}}{\pgfqpoint{1.978063in}{2.052233in}}%
\pgfpathcurveto{\pgfqpoint{1.978063in}{2.060469in}}{\pgfqpoint{1.974790in}{2.068370in}}{\pgfqpoint{1.968966in}{2.074193in}}%
\pgfpathcurveto{\pgfqpoint{1.963143in}{2.080017in}}{\pgfqpoint{1.955242in}{2.083290in}}{\pgfqpoint{1.947006in}{2.083290in}}%
\pgfpathcurveto{\pgfqpoint{1.938770in}{2.083290in}}{\pgfqpoint{1.930870in}{2.080017in}}{\pgfqpoint{1.925046in}{2.074193in}}%
\pgfpathcurveto{\pgfqpoint{1.919222in}{2.068370in}}{\pgfqpoint{1.915950in}{2.060469in}}{\pgfqpoint{1.915950in}{2.052233in}}%
\pgfpathcurveto{\pgfqpoint{1.915950in}{2.043997in}}{\pgfqpoint{1.919222in}{2.036097in}}{\pgfqpoint{1.925046in}{2.030273in}}%
\pgfpathcurveto{\pgfqpoint{1.930870in}{2.024449in}}{\pgfqpoint{1.938770in}{2.021177in}}{\pgfqpoint{1.947006in}{2.021177in}}%
\pgfpathclose%
\pgfusepath{stroke,fill}%
\end{pgfscope}%
\begin{pgfscope}%
\pgfpathrectangle{\pgfqpoint{0.100000in}{0.212622in}}{\pgfqpoint{3.696000in}{3.696000in}}%
\pgfusepath{clip}%
\pgfsetbuttcap%
\pgfsetroundjoin%
\definecolor{currentfill}{rgb}{0.121569,0.466667,0.705882}%
\pgfsetfillcolor{currentfill}%
\pgfsetfillopacity{0.947833}%
\pgfsetlinewidth{1.003750pt}%
\definecolor{currentstroke}{rgb}{0.121569,0.466667,0.705882}%
\pgfsetstrokecolor{currentstroke}%
\pgfsetstrokeopacity{0.947833}%
\pgfsetdash{}{0pt}%
\pgfpathmoveto{\pgfqpoint{2.519174in}{1.872472in}}%
\pgfpathcurveto{\pgfqpoint{2.527410in}{1.872472in}}{\pgfqpoint{2.535310in}{1.875745in}}{\pgfqpoint{2.541134in}{1.881569in}}%
\pgfpathcurveto{\pgfqpoint{2.546958in}{1.887393in}}{\pgfqpoint{2.550230in}{1.895293in}}{\pgfqpoint{2.550230in}{1.903529in}}%
\pgfpathcurveto{\pgfqpoint{2.550230in}{1.911765in}}{\pgfqpoint{2.546958in}{1.919665in}}{\pgfqpoint{2.541134in}{1.925489in}}%
\pgfpathcurveto{\pgfqpoint{2.535310in}{1.931313in}}{\pgfqpoint{2.527410in}{1.934585in}}{\pgfqpoint{2.519174in}{1.934585in}}%
\pgfpathcurveto{\pgfqpoint{2.510937in}{1.934585in}}{\pgfqpoint{2.503037in}{1.931313in}}{\pgfqpoint{2.497213in}{1.925489in}}%
\pgfpathcurveto{\pgfqpoint{2.491389in}{1.919665in}}{\pgfqpoint{2.488117in}{1.911765in}}{\pgfqpoint{2.488117in}{1.903529in}}%
\pgfpathcurveto{\pgfqpoint{2.488117in}{1.895293in}}{\pgfqpoint{2.491389in}{1.887393in}}{\pgfqpoint{2.497213in}{1.881569in}}%
\pgfpathcurveto{\pgfqpoint{2.503037in}{1.875745in}}{\pgfqpoint{2.510937in}{1.872472in}}{\pgfqpoint{2.519174in}{1.872472in}}%
\pgfpathclose%
\pgfusepath{stroke,fill}%
\end{pgfscope}%
\begin{pgfscope}%
\pgfpathrectangle{\pgfqpoint{0.100000in}{0.212622in}}{\pgfqpoint{3.696000in}{3.696000in}}%
\pgfusepath{clip}%
\pgfsetbuttcap%
\pgfsetroundjoin%
\definecolor{currentfill}{rgb}{0.121569,0.466667,0.705882}%
\pgfsetfillcolor{currentfill}%
\pgfsetfillopacity{0.950961}%
\pgfsetlinewidth{1.003750pt}%
\definecolor{currentstroke}{rgb}{0.121569,0.466667,0.705882}%
\pgfsetstrokecolor{currentstroke}%
\pgfsetstrokeopacity{0.950961}%
\pgfsetdash{}{0pt}%
\pgfpathmoveto{\pgfqpoint{1.982786in}{2.004462in}}%
\pgfpathcurveto{\pgfqpoint{1.991022in}{2.004462in}}{\pgfqpoint{1.998922in}{2.007735in}}{\pgfqpoint{2.004746in}{2.013559in}}%
\pgfpathcurveto{\pgfqpoint{2.010570in}{2.019382in}}{\pgfqpoint{2.013842in}{2.027283in}}{\pgfqpoint{2.013842in}{2.035519in}}%
\pgfpathcurveto{\pgfqpoint{2.013842in}{2.043755in}}{\pgfqpoint{2.010570in}{2.051655in}}{\pgfqpoint{2.004746in}{2.057479in}}%
\pgfpathcurveto{\pgfqpoint{1.998922in}{2.063303in}}{\pgfqpoint{1.991022in}{2.066575in}}{\pgfqpoint{1.982786in}{2.066575in}}%
\pgfpathcurveto{\pgfqpoint{1.974549in}{2.066575in}}{\pgfqpoint{1.966649in}{2.063303in}}{\pgfqpoint{1.960825in}{2.057479in}}%
\pgfpathcurveto{\pgfqpoint{1.955001in}{2.051655in}}{\pgfqpoint{1.951729in}{2.043755in}}{\pgfqpoint{1.951729in}{2.035519in}}%
\pgfpathcurveto{\pgfqpoint{1.951729in}{2.027283in}}{\pgfqpoint{1.955001in}{2.019382in}}{\pgfqpoint{1.960825in}{2.013559in}}%
\pgfpathcurveto{\pgfqpoint{1.966649in}{2.007735in}}{\pgfqpoint{1.974549in}{2.004462in}}{\pgfqpoint{1.982786in}{2.004462in}}%
\pgfpathclose%
\pgfusepath{stroke,fill}%
\end{pgfscope}%
\begin{pgfscope}%
\pgfpathrectangle{\pgfqpoint{0.100000in}{0.212622in}}{\pgfqpoint{3.696000in}{3.696000in}}%
\pgfusepath{clip}%
\pgfsetbuttcap%
\pgfsetroundjoin%
\definecolor{currentfill}{rgb}{0.121569,0.466667,0.705882}%
\pgfsetfillcolor{currentfill}%
\pgfsetfillopacity{0.953279}%
\pgfsetlinewidth{1.003750pt}%
\definecolor{currentstroke}{rgb}{0.121569,0.466667,0.705882}%
\pgfsetstrokecolor{currentstroke}%
\pgfsetstrokeopacity{0.953279}%
\pgfsetdash{}{0pt}%
\pgfpathmoveto{\pgfqpoint{2.501084in}{1.863729in}}%
\pgfpathcurveto{\pgfqpoint{2.509320in}{1.863729in}}{\pgfqpoint{2.517220in}{1.867002in}}{\pgfqpoint{2.523044in}{1.872826in}}%
\pgfpathcurveto{\pgfqpoint{2.528868in}{1.878650in}}{\pgfqpoint{2.532140in}{1.886550in}}{\pgfqpoint{2.532140in}{1.894786in}}%
\pgfpathcurveto{\pgfqpoint{2.532140in}{1.903022in}}{\pgfqpoint{2.528868in}{1.910922in}}{\pgfqpoint{2.523044in}{1.916746in}}%
\pgfpathcurveto{\pgfqpoint{2.517220in}{1.922570in}}{\pgfqpoint{2.509320in}{1.925842in}}{\pgfqpoint{2.501084in}{1.925842in}}%
\pgfpathcurveto{\pgfqpoint{2.492847in}{1.925842in}}{\pgfqpoint{2.484947in}{1.922570in}}{\pgfqpoint{2.479123in}{1.916746in}}%
\pgfpathcurveto{\pgfqpoint{2.473300in}{1.910922in}}{\pgfqpoint{2.470027in}{1.903022in}}{\pgfqpoint{2.470027in}{1.894786in}}%
\pgfpathcurveto{\pgfqpoint{2.470027in}{1.886550in}}{\pgfqpoint{2.473300in}{1.878650in}}{\pgfqpoint{2.479123in}{1.872826in}}%
\pgfpathcurveto{\pgfqpoint{2.484947in}{1.867002in}}{\pgfqpoint{2.492847in}{1.863729in}}{\pgfqpoint{2.501084in}{1.863729in}}%
\pgfpathclose%
\pgfusepath{stroke,fill}%
\end{pgfscope}%
\begin{pgfscope}%
\pgfpathrectangle{\pgfqpoint{0.100000in}{0.212622in}}{\pgfqpoint{3.696000in}{3.696000in}}%
\pgfusepath{clip}%
\pgfsetbuttcap%
\pgfsetroundjoin%
\definecolor{currentfill}{rgb}{0.121569,0.466667,0.705882}%
\pgfsetfillcolor{currentfill}%
\pgfsetfillopacity{0.953721}%
\pgfsetlinewidth{1.003750pt}%
\definecolor{currentstroke}{rgb}{0.121569,0.466667,0.705882}%
\pgfsetstrokecolor{currentstroke}%
\pgfsetstrokeopacity{0.953721}%
\pgfsetdash{}{0pt}%
\pgfpathmoveto{\pgfqpoint{2.020791in}{1.979270in}}%
\pgfpathcurveto{\pgfqpoint{2.029027in}{1.979270in}}{\pgfqpoint{2.036927in}{1.982542in}}{\pgfqpoint{2.042751in}{1.988366in}}%
\pgfpathcurveto{\pgfqpoint{2.048575in}{1.994190in}}{\pgfqpoint{2.051848in}{2.002090in}}{\pgfqpoint{2.051848in}{2.010327in}}%
\pgfpathcurveto{\pgfqpoint{2.051848in}{2.018563in}}{\pgfqpoint{2.048575in}{2.026463in}}{\pgfqpoint{2.042751in}{2.032287in}}%
\pgfpathcurveto{\pgfqpoint{2.036927in}{2.038111in}}{\pgfqpoint{2.029027in}{2.041383in}}{\pgfqpoint{2.020791in}{2.041383in}}%
\pgfpathcurveto{\pgfqpoint{2.012555in}{2.041383in}}{\pgfqpoint{2.004655in}{2.038111in}}{\pgfqpoint{1.998831in}{2.032287in}}%
\pgfpathcurveto{\pgfqpoint{1.993007in}{2.026463in}}{\pgfqpoint{1.989735in}{2.018563in}}{\pgfqpoint{1.989735in}{2.010327in}}%
\pgfpathcurveto{\pgfqpoint{1.989735in}{2.002090in}}{\pgfqpoint{1.993007in}{1.994190in}}{\pgfqpoint{1.998831in}{1.988366in}}%
\pgfpathcurveto{\pgfqpoint{2.004655in}{1.982542in}}{\pgfqpoint{2.012555in}{1.979270in}}{\pgfqpoint{2.020791in}{1.979270in}}%
\pgfpathclose%
\pgfusepath{stroke,fill}%
\end{pgfscope}%
\begin{pgfscope}%
\pgfpathrectangle{\pgfqpoint{0.100000in}{0.212622in}}{\pgfqpoint{3.696000in}{3.696000in}}%
\pgfusepath{clip}%
\pgfsetbuttcap%
\pgfsetroundjoin%
\definecolor{currentfill}{rgb}{0.121569,0.466667,0.705882}%
\pgfsetfillcolor{currentfill}%
\pgfsetfillopacity{0.956945}%
\pgfsetlinewidth{1.003750pt}%
\definecolor{currentstroke}{rgb}{0.121569,0.466667,0.705882}%
\pgfsetstrokecolor{currentstroke}%
\pgfsetstrokeopacity{0.956945}%
\pgfsetdash{}{0pt}%
\pgfpathmoveto{\pgfqpoint{2.493448in}{1.859577in}}%
\pgfpathcurveto{\pgfqpoint{2.501684in}{1.859577in}}{\pgfqpoint{2.509584in}{1.862850in}}{\pgfqpoint{2.515408in}{1.868674in}}%
\pgfpathcurveto{\pgfqpoint{2.521232in}{1.874498in}}{\pgfqpoint{2.524504in}{1.882398in}}{\pgfqpoint{2.524504in}{1.890634in}}%
\pgfpathcurveto{\pgfqpoint{2.524504in}{1.898870in}}{\pgfqpoint{2.521232in}{1.906770in}}{\pgfqpoint{2.515408in}{1.912594in}}%
\pgfpathcurveto{\pgfqpoint{2.509584in}{1.918418in}}{\pgfqpoint{2.501684in}{1.921690in}}{\pgfqpoint{2.493448in}{1.921690in}}%
\pgfpathcurveto{\pgfqpoint{2.485212in}{1.921690in}}{\pgfqpoint{2.477311in}{1.918418in}}{\pgfqpoint{2.471488in}{1.912594in}}%
\pgfpathcurveto{\pgfqpoint{2.465664in}{1.906770in}}{\pgfqpoint{2.462391in}{1.898870in}}{\pgfqpoint{2.462391in}{1.890634in}}%
\pgfpathcurveto{\pgfqpoint{2.462391in}{1.882398in}}{\pgfqpoint{2.465664in}{1.874498in}}{\pgfqpoint{2.471488in}{1.868674in}}%
\pgfpathcurveto{\pgfqpoint{2.477311in}{1.862850in}}{\pgfqpoint{2.485212in}{1.859577in}}{\pgfqpoint{2.493448in}{1.859577in}}%
\pgfpathclose%
\pgfusepath{stroke,fill}%
\end{pgfscope}%
\begin{pgfscope}%
\pgfpathrectangle{\pgfqpoint{0.100000in}{0.212622in}}{\pgfqpoint{3.696000in}{3.696000in}}%
\pgfusepath{clip}%
\pgfsetbuttcap%
\pgfsetroundjoin%
\definecolor{currentfill}{rgb}{0.121569,0.466667,0.705882}%
\pgfsetfillcolor{currentfill}%
\pgfsetfillopacity{0.957762}%
\pgfsetlinewidth{1.003750pt}%
\definecolor{currentstroke}{rgb}{0.121569,0.466667,0.705882}%
\pgfsetstrokecolor{currentstroke}%
\pgfsetstrokeopacity{0.957762}%
\pgfsetdash{}{0pt}%
\pgfpathmoveto{\pgfqpoint{2.054452in}{1.965912in}}%
\pgfpathcurveto{\pgfqpoint{2.062688in}{1.965912in}}{\pgfqpoint{2.070588in}{1.969184in}}{\pgfqpoint{2.076412in}{1.975008in}}%
\pgfpathcurveto{\pgfqpoint{2.082236in}{1.980832in}}{\pgfqpoint{2.085508in}{1.988732in}}{\pgfqpoint{2.085508in}{1.996968in}}%
\pgfpathcurveto{\pgfqpoint{2.085508in}{2.005204in}}{\pgfqpoint{2.082236in}{2.013104in}}{\pgfqpoint{2.076412in}{2.018928in}}%
\pgfpathcurveto{\pgfqpoint{2.070588in}{2.024752in}}{\pgfqpoint{2.062688in}{2.028025in}}{\pgfqpoint{2.054452in}{2.028025in}}%
\pgfpathcurveto{\pgfqpoint{2.046216in}{2.028025in}}{\pgfqpoint{2.038315in}{2.024752in}}{\pgfqpoint{2.032492in}{2.018928in}}%
\pgfpathcurveto{\pgfqpoint{2.026668in}{2.013104in}}{\pgfqpoint{2.023395in}{2.005204in}}{\pgfqpoint{2.023395in}{1.996968in}}%
\pgfpathcurveto{\pgfqpoint{2.023395in}{1.988732in}}{\pgfqpoint{2.026668in}{1.980832in}}{\pgfqpoint{2.032492in}{1.975008in}}%
\pgfpathcurveto{\pgfqpoint{2.038315in}{1.969184in}}{\pgfqpoint{2.046216in}{1.965912in}}{\pgfqpoint{2.054452in}{1.965912in}}%
\pgfpathclose%
\pgfusepath{stroke,fill}%
\end{pgfscope}%
\begin{pgfscope}%
\pgfpathrectangle{\pgfqpoint{0.100000in}{0.212622in}}{\pgfqpoint{3.696000in}{3.696000in}}%
\pgfusepath{clip}%
\pgfsetbuttcap%
\pgfsetroundjoin%
\definecolor{currentfill}{rgb}{0.121569,0.466667,0.705882}%
\pgfsetfillcolor{currentfill}%
\pgfsetfillopacity{0.958873}%
\pgfsetlinewidth{1.003750pt}%
\definecolor{currentstroke}{rgb}{0.121569,0.466667,0.705882}%
\pgfsetstrokecolor{currentstroke}%
\pgfsetstrokeopacity{0.958873}%
\pgfsetdash{}{0pt}%
\pgfpathmoveto{\pgfqpoint{2.488975in}{1.857048in}}%
\pgfpathcurveto{\pgfqpoint{2.497211in}{1.857048in}}{\pgfqpoint{2.505111in}{1.860320in}}{\pgfqpoint{2.510935in}{1.866144in}}%
\pgfpathcurveto{\pgfqpoint{2.516759in}{1.871968in}}{\pgfqpoint{2.520031in}{1.879868in}}{\pgfqpoint{2.520031in}{1.888104in}}%
\pgfpathcurveto{\pgfqpoint{2.520031in}{1.896340in}}{\pgfqpoint{2.516759in}{1.904240in}}{\pgfqpoint{2.510935in}{1.910064in}}%
\pgfpathcurveto{\pgfqpoint{2.505111in}{1.915888in}}{\pgfqpoint{2.497211in}{1.919161in}}{\pgfqpoint{2.488975in}{1.919161in}}%
\pgfpathcurveto{\pgfqpoint{2.480739in}{1.919161in}}{\pgfqpoint{2.472839in}{1.915888in}}{\pgfqpoint{2.467015in}{1.910064in}}%
\pgfpathcurveto{\pgfqpoint{2.461191in}{1.904240in}}{\pgfqpoint{2.457918in}{1.896340in}}{\pgfqpoint{2.457918in}{1.888104in}}%
\pgfpathcurveto{\pgfqpoint{2.457918in}{1.879868in}}{\pgfqpoint{2.461191in}{1.871968in}}{\pgfqpoint{2.467015in}{1.866144in}}%
\pgfpathcurveto{\pgfqpoint{2.472839in}{1.860320in}}{\pgfqpoint{2.480739in}{1.857048in}}{\pgfqpoint{2.488975in}{1.857048in}}%
\pgfpathclose%
\pgfusepath{stroke,fill}%
\end{pgfscope}%
\begin{pgfscope}%
\pgfpathrectangle{\pgfqpoint{0.100000in}{0.212622in}}{\pgfqpoint{3.696000in}{3.696000in}}%
\pgfusepath{clip}%
\pgfsetbuttcap%
\pgfsetroundjoin%
\definecolor{currentfill}{rgb}{0.121569,0.466667,0.705882}%
\pgfsetfillcolor{currentfill}%
\pgfsetfillopacity{0.960064}%
\pgfsetlinewidth{1.003750pt}%
\definecolor{currentstroke}{rgb}{0.121569,0.466667,0.705882}%
\pgfsetstrokecolor{currentstroke}%
\pgfsetstrokeopacity{0.960064}%
\pgfsetdash{}{0pt}%
\pgfpathmoveto{\pgfqpoint{2.486606in}{1.856389in}}%
\pgfpathcurveto{\pgfqpoint{2.494842in}{1.856389in}}{\pgfqpoint{2.502742in}{1.859661in}}{\pgfqpoint{2.508566in}{1.865485in}}%
\pgfpathcurveto{\pgfqpoint{2.514390in}{1.871309in}}{\pgfqpoint{2.517662in}{1.879209in}}{\pgfqpoint{2.517662in}{1.887445in}}%
\pgfpathcurveto{\pgfqpoint{2.517662in}{1.895682in}}{\pgfqpoint{2.514390in}{1.903582in}}{\pgfqpoint{2.508566in}{1.909406in}}%
\pgfpathcurveto{\pgfqpoint{2.502742in}{1.915230in}}{\pgfqpoint{2.494842in}{1.918502in}}{\pgfqpoint{2.486606in}{1.918502in}}%
\pgfpathcurveto{\pgfqpoint{2.478369in}{1.918502in}}{\pgfqpoint{2.470469in}{1.915230in}}{\pgfqpoint{2.464645in}{1.909406in}}%
\pgfpathcurveto{\pgfqpoint{2.458821in}{1.903582in}}{\pgfqpoint{2.455549in}{1.895682in}}{\pgfqpoint{2.455549in}{1.887445in}}%
\pgfpathcurveto{\pgfqpoint{2.455549in}{1.879209in}}{\pgfqpoint{2.458821in}{1.871309in}}{\pgfqpoint{2.464645in}{1.865485in}}%
\pgfpathcurveto{\pgfqpoint{2.470469in}{1.859661in}}{\pgfqpoint{2.478369in}{1.856389in}}{\pgfqpoint{2.486606in}{1.856389in}}%
\pgfpathclose%
\pgfusepath{stroke,fill}%
\end{pgfscope}%
\begin{pgfscope}%
\pgfpathrectangle{\pgfqpoint{0.100000in}{0.212622in}}{\pgfqpoint{3.696000in}{3.696000in}}%
\pgfusepath{clip}%
\pgfsetbuttcap%
\pgfsetroundjoin%
\definecolor{currentfill}{rgb}{0.121569,0.466667,0.705882}%
\pgfsetfillcolor{currentfill}%
\pgfsetfillopacity{0.960611}%
\pgfsetlinewidth{1.003750pt}%
\definecolor{currentstroke}{rgb}{0.121569,0.466667,0.705882}%
\pgfsetstrokecolor{currentstroke}%
\pgfsetstrokeopacity{0.960611}%
\pgfsetdash{}{0pt}%
\pgfpathmoveto{\pgfqpoint{2.485305in}{1.855332in}}%
\pgfpathcurveto{\pgfqpoint{2.493541in}{1.855332in}}{\pgfqpoint{2.501441in}{1.858604in}}{\pgfqpoint{2.507265in}{1.864428in}}%
\pgfpathcurveto{\pgfqpoint{2.513089in}{1.870252in}}{\pgfqpoint{2.516361in}{1.878152in}}{\pgfqpoint{2.516361in}{1.886389in}}%
\pgfpathcurveto{\pgfqpoint{2.516361in}{1.894625in}}{\pgfqpoint{2.513089in}{1.902525in}}{\pgfqpoint{2.507265in}{1.908349in}}%
\pgfpathcurveto{\pgfqpoint{2.501441in}{1.914173in}}{\pgfqpoint{2.493541in}{1.917445in}}{\pgfqpoint{2.485305in}{1.917445in}}%
\pgfpathcurveto{\pgfqpoint{2.477068in}{1.917445in}}{\pgfqpoint{2.469168in}{1.914173in}}{\pgfqpoint{2.463344in}{1.908349in}}%
\pgfpathcurveto{\pgfqpoint{2.457520in}{1.902525in}}{\pgfqpoint{2.454248in}{1.894625in}}{\pgfqpoint{2.454248in}{1.886389in}}%
\pgfpathcurveto{\pgfqpoint{2.454248in}{1.878152in}}{\pgfqpoint{2.457520in}{1.870252in}}{\pgfqpoint{2.463344in}{1.864428in}}%
\pgfpathcurveto{\pgfqpoint{2.469168in}{1.858604in}}{\pgfqpoint{2.477068in}{1.855332in}}{\pgfqpoint{2.485305in}{1.855332in}}%
\pgfpathclose%
\pgfusepath{stroke,fill}%
\end{pgfscope}%
\begin{pgfscope}%
\pgfpathrectangle{\pgfqpoint{0.100000in}{0.212622in}}{\pgfqpoint{3.696000in}{3.696000in}}%
\pgfusepath{clip}%
\pgfsetbuttcap%
\pgfsetroundjoin%
\definecolor{currentfill}{rgb}{0.121569,0.466667,0.705882}%
\pgfsetfillcolor{currentfill}%
\pgfsetfillopacity{0.960711}%
\pgfsetlinewidth{1.003750pt}%
\definecolor{currentstroke}{rgb}{0.121569,0.466667,0.705882}%
\pgfsetstrokecolor{currentstroke}%
\pgfsetstrokeopacity{0.960711}%
\pgfsetdash{}{0pt}%
\pgfpathmoveto{\pgfqpoint{2.079999in}{1.954508in}}%
\pgfpathcurveto{\pgfqpoint{2.088235in}{1.954508in}}{\pgfqpoint{2.096135in}{1.957780in}}{\pgfqpoint{2.101959in}{1.963604in}}%
\pgfpathcurveto{\pgfqpoint{2.107783in}{1.969428in}}{\pgfqpoint{2.111055in}{1.977328in}}{\pgfqpoint{2.111055in}{1.985564in}}%
\pgfpathcurveto{\pgfqpoint{2.111055in}{1.993800in}}{\pgfqpoint{2.107783in}{2.001701in}}{\pgfqpoint{2.101959in}{2.007524in}}%
\pgfpathcurveto{\pgfqpoint{2.096135in}{2.013348in}}{\pgfqpoint{2.088235in}{2.016621in}}{\pgfqpoint{2.079999in}{2.016621in}}%
\pgfpathcurveto{\pgfqpoint{2.071763in}{2.016621in}}{\pgfqpoint{2.063863in}{2.013348in}}{\pgfqpoint{2.058039in}{2.007524in}}%
\pgfpathcurveto{\pgfqpoint{2.052215in}{2.001701in}}{\pgfqpoint{2.048942in}{1.993800in}}{\pgfqpoint{2.048942in}{1.985564in}}%
\pgfpathcurveto{\pgfqpoint{2.048942in}{1.977328in}}{\pgfqpoint{2.052215in}{1.969428in}}{\pgfqpoint{2.058039in}{1.963604in}}%
\pgfpathcurveto{\pgfqpoint{2.063863in}{1.957780in}}{\pgfqpoint{2.071763in}{1.954508in}}{\pgfqpoint{2.079999in}{1.954508in}}%
\pgfpathclose%
\pgfusepath{stroke,fill}%
\end{pgfscope}%
\begin{pgfscope}%
\pgfpathrectangle{\pgfqpoint{0.100000in}{0.212622in}}{\pgfqpoint{3.696000in}{3.696000in}}%
\pgfusepath{clip}%
\pgfsetbuttcap%
\pgfsetroundjoin%
\definecolor{currentfill}{rgb}{0.121569,0.466667,0.705882}%
\pgfsetfillcolor{currentfill}%
\pgfsetfillopacity{0.960951}%
\pgfsetlinewidth{1.003750pt}%
\definecolor{currentstroke}{rgb}{0.121569,0.466667,0.705882}%
\pgfsetstrokecolor{currentstroke}%
\pgfsetstrokeopacity{0.960951}%
\pgfsetdash{}{0pt}%
\pgfpathmoveto{\pgfqpoint{2.484536in}{1.855068in}}%
\pgfpathcurveto{\pgfqpoint{2.492772in}{1.855068in}}{\pgfqpoint{2.500672in}{1.858340in}}{\pgfqpoint{2.506496in}{1.864164in}}%
\pgfpathcurveto{\pgfqpoint{2.512320in}{1.869988in}}{\pgfqpoint{2.515592in}{1.877888in}}{\pgfqpoint{2.515592in}{1.886124in}}%
\pgfpathcurveto{\pgfqpoint{2.515592in}{1.894360in}}{\pgfqpoint{2.512320in}{1.902260in}}{\pgfqpoint{2.506496in}{1.908084in}}%
\pgfpathcurveto{\pgfqpoint{2.500672in}{1.913908in}}{\pgfqpoint{2.492772in}{1.917181in}}{\pgfqpoint{2.484536in}{1.917181in}}%
\pgfpathcurveto{\pgfqpoint{2.476299in}{1.917181in}}{\pgfqpoint{2.468399in}{1.913908in}}{\pgfqpoint{2.462575in}{1.908084in}}%
\pgfpathcurveto{\pgfqpoint{2.456751in}{1.902260in}}{\pgfqpoint{2.453479in}{1.894360in}}{\pgfqpoint{2.453479in}{1.886124in}}%
\pgfpathcurveto{\pgfqpoint{2.453479in}{1.877888in}}{\pgfqpoint{2.456751in}{1.869988in}}{\pgfqpoint{2.462575in}{1.864164in}}%
\pgfpathcurveto{\pgfqpoint{2.468399in}{1.858340in}}{\pgfqpoint{2.476299in}{1.855068in}}{\pgfqpoint{2.484536in}{1.855068in}}%
\pgfpathclose%
\pgfusepath{stroke,fill}%
\end{pgfscope}%
\begin{pgfscope}%
\pgfpathrectangle{\pgfqpoint{0.100000in}{0.212622in}}{\pgfqpoint{3.696000in}{3.696000in}}%
\pgfusepath{clip}%
\pgfsetbuttcap%
\pgfsetroundjoin%
\definecolor{currentfill}{rgb}{0.121569,0.466667,0.705882}%
\pgfsetfillcolor{currentfill}%
\pgfsetfillopacity{0.962337}%
\pgfsetlinewidth{1.003750pt}%
\definecolor{currentstroke}{rgb}{0.121569,0.466667,0.705882}%
\pgfsetstrokecolor{currentstroke}%
\pgfsetstrokeopacity{0.962337}%
\pgfsetdash{}{0pt}%
\pgfpathmoveto{\pgfqpoint{2.481419in}{1.852212in}}%
\pgfpathcurveto{\pgfqpoint{2.489655in}{1.852212in}}{\pgfqpoint{2.497555in}{1.855485in}}{\pgfqpoint{2.503379in}{1.861309in}}%
\pgfpathcurveto{\pgfqpoint{2.509203in}{1.867133in}}{\pgfqpoint{2.512475in}{1.875033in}}{\pgfqpoint{2.512475in}{1.883269in}}%
\pgfpathcurveto{\pgfqpoint{2.512475in}{1.891505in}}{\pgfqpoint{2.509203in}{1.899405in}}{\pgfqpoint{2.503379in}{1.905229in}}%
\pgfpathcurveto{\pgfqpoint{2.497555in}{1.911053in}}{\pgfqpoint{2.489655in}{1.914325in}}{\pgfqpoint{2.481419in}{1.914325in}}%
\pgfpathcurveto{\pgfqpoint{2.473183in}{1.914325in}}{\pgfqpoint{2.465283in}{1.911053in}}{\pgfqpoint{2.459459in}{1.905229in}}%
\pgfpathcurveto{\pgfqpoint{2.453635in}{1.899405in}}{\pgfqpoint{2.450362in}{1.891505in}}{\pgfqpoint{2.450362in}{1.883269in}}%
\pgfpathcurveto{\pgfqpoint{2.450362in}{1.875033in}}{\pgfqpoint{2.453635in}{1.867133in}}{\pgfqpoint{2.459459in}{1.861309in}}%
\pgfpathcurveto{\pgfqpoint{2.465283in}{1.855485in}}{\pgfqpoint{2.473183in}{1.852212in}}{\pgfqpoint{2.481419in}{1.852212in}}%
\pgfpathclose%
\pgfusepath{stroke,fill}%
\end{pgfscope}%
\begin{pgfscope}%
\pgfpathrectangle{\pgfqpoint{0.100000in}{0.212622in}}{\pgfqpoint{3.696000in}{3.696000in}}%
\pgfusepath{clip}%
\pgfsetbuttcap%
\pgfsetroundjoin%
\definecolor{currentfill}{rgb}{0.121569,0.466667,0.705882}%
\pgfsetfillcolor{currentfill}%
\pgfsetfillopacity{0.963055}%
\pgfsetlinewidth{1.003750pt}%
\definecolor{currentstroke}{rgb}{0.121569,0.466667,0.705882}%
\pgfsetstrokecolor{currentstroke}%
\pgfsetstrokeopacity{0.963055}%
\pgfsetdash{}{0pt}%
\pgfpathmoveto{\pgfqpoint{2.097616in}{1.947364in}}%
\pgfpathcurveto{\pgfqpoint{2.105852in}{1.947364in}}{\pgfqpoint{2.113752in}{1.950636in}}{\pgfqpoint{2.119576in}{1.956460in}}%
\pgfpathcurveto{\pgfqpoint{2.125400in}{1.962284in}}{\pgfqpoint{2.128672in}{1.970184in}}{\pgfqpoint{2.128672in}{1.978420in}}%
\pgfpathcurveto{\pgfqpoint{2.128672in}{1.986656in}}{\pgfqpoint{2.125400in}{1.994556in}}{\pgfqpoint{2.119576in}{2.000380in}}%
\pgfpathcurveto{\pgfqpoint{2.113752in}{2.006204in}}{\pgfqpoint{2.105852in}{2.009477in}}{\pgfqpoint{2.097616in}{2.009477in}}%
\pgfpathcurveto{\pgfqpoint{2.089379in}{2.009477in}}{\pgfqpoint{2.081479in}{2.006204in}}{\pgfqpoint{2.075655in}{2.000380in}}%
\pgfpathcurveto{\pgfqpoint{2.069832in}{1.994556in}}{\pgfqpoint{2.066559in}{1.986656in}}{\pgfqpoint{2.066559in}{1.978420in}}%
\pgfpathcurveto{\pgfqpoint{2.066559in}{1.970184in}}{\pgfqpoint{2.069832in}{1.962284in}}{\pgfqpoint{2.075655in}{1.956460in}}%
\pgfpathcurveto{\pgfqpoint{2.081479in}{1.950636in}}{\pgfqpoint{2.089379in}{1.947364in}}{\pgfqpoint{2.097616in}{1.947364in}}%
\pgfpathclose%
\pgfusepath{stroke,fill}%
\end{pgfscope}%
\begin{pgfscope}%
\pgfpathrectangle{\pgfqpoint{0.100000in}{0.212622in}}{\pgfqpoint{3.696000in}{3.696000in}}%
\pgfusepath{clip}%
\pgfsetbuttcap%
\pgfsetroundjoin%
\definecolor{currentfill}{rgb}{0.121569,0.466667,0.705882}%
\pgfsetfillcolor{currentfill}%
\pgfsetfillopacity{0.964324}%
\pgfsetlinewidth{1.003750pt}%
\definecolor{currentstroke}{rgb}{0.121569,0.466667,0.705882}%
\pgfsetstrokecolor{currentstroke}%
\pgfsetstrokeopacity{0.964324}%
\pgfsetdash{}{0pt}%
\pgfpathmoveto{\pgfqpoint{2.476110in}{1.850088in}}%
\pgfpathcurveto{\pgfqpoint{2.484346in}{1.850088in}}{\pgfqpoint{2.492246in}{1.853360in}}{\pgfqpoint{2.498070in}{1.859184in}}%
\pgfpathcurveto{\pgfqpoint{2.503894in}{1.865008in}}{\pgfqpoint{2.507166in}{1.872908in}}{\pgfqpoint{2.507166in}{1.881145in}}%
\pgfpathcurveto{\pgfqpoint{2.507166in}{1.889381in}}{\pgfqpoint{2.503894in}{1.897281in}}{\pgfqpoint{2.498070in}{1.903105in}}%
\pgfpathcurveto{\pgfqpoint{2.492246in}{1.908929in}}{\pgfqpoint{2.484346in}{1.912201in}}{\pgfqpoint{2.476110in}{1.912201in}}%
\pgfpathcurveto{\pgfqpoint{2.467873in}{1.912201in}}{\pgfqpoint{2.459973in}{1.908929in}}{\pgfqpoint{2.454149in}{1.903105in}}%
\pgfpathcurveto{\pgfqpoint{2.448325in}{1.897281in}}{\pgfqpoint{2.445053in}{1.889381in}}{\pgfqpoint{2.445053in}{1.881145in}}%
\pgfpathcurveto{\pgfqpoint{2.445053in}{1.872908in}}{\pgfqpoint{2.448325in}{1.865008in}}{\pgfqpoint{2.454149in}{1.859184in}}%
\pgfpathcurveto{\pgfqpoint{2.459973in}{1.853360in}}{\pgfqpoint{2.467873in}{1.850088in}}{\pgfqpoint{2.476110in}{1.850088in}}%
\pgfpathclose%
\pgfusepath{stroke,fill}%
\end{pgfscope}%
\begin{pgfscope}%
\pgfpathrectangle{\pgfqpoint{0.100000in}{0.212622in}}{\pgfqpoint{3.696000in}{3.696000in}}%
\pgfusepath{clip}%
\pgfsetbuttcap%
\pgfsetroundjoin%
\definecolor{currentfill}{rgb}{0.121569,0.466667,0.705882}%
\pgfsetfillcolor{currentfill}%
\pgfsetfillopacity{0.964915}%
\pgfsetlinewidth{1.003750pt}%
\definecolor{currentstroke}{rgb}{0.121569,0.466667,0.705882}%
\pgfsetstrokecolor{currentstroke}%
\pgfsetstrokeopacity{0.964915}%
\pgfsetdash{}{0pt}%
\pgfpathmoveto{\pgfqpoint{2.111102in}{1.943138in}}%
\pgfpathcurveto{\pgfqpoint{2.119339in}{1.943138in}}{\pgfqpoint{2.127239in}{1.946410in}}{\pgfqpoint{2.133063in}{1.952234in}}%
\pgfpathcurveto{\pgfqpoint{2.138887in}{1.958058in}}{\pgfqpoint{2.142159in}{1.965958in}}{\pgfqpoint{2.142159in}{1.974194in}}%
\pgfpathcurveto{\pgfqpoint{2.142159in}{1.982430in}}{\pgfqpoint{2.138887in}{1.990330in}}{\pgfqpoint{2.133063in}{1.996154in}}%
\pgfpathcurveto{\pgfqpoint{2.127239in}{2.001978in}}{\pgfqpoint{2.119339in}{2.005251in}}{\pgfqpoint{2.111102in}{2.005251in}}%
\pgfpathcurveto{\pgfqpoint{2.102866in}{2.005251in}}{\pgfqpoint{2.094966in}{2.001978in}}{\pgfqpoint{2.089142in}{1.996154in}}%
\pgfpathcurveto{\pgfqpoint{2.083318in}{1.990330in}}{\pgfqpoint{2.080046in}{1.982430in}}{\pgfqpoint{2.080046in}{1.974194in}}%
\pgfpathcurveto{\pgfqpoint{2.080046in}{1.965958in}}{\pgfqpoint{2.083318in}{1.958058in}}{\pgfqpoint{2.089142in}{1.952234in}}%
\pgfpathcurveto{\pgfqpoint{2.094966in}{1.946410in}}{\pgfqpoint{2.102866in}{1.943138in}}{\pgfqpoint{2.111102in}{1.943138in}}%
\pgfpathclose%
\pgfusepath{stroke,fill}%
\end{pgfscope}%
\begin{pgfscope}%
\pgfpathrectangle{\pgfqpoint{0.100000in}{0.212622in}}{\pgfqpoint{3.696000in}{3.696000in}}%
\pgfusepath{clip}%
\pgfsetbuttcap%
\pgfsetroundjoin%
\definecolor{currentfill}{rgb}{0.121569,0.466667,0.705882}%
\pgfsetfillcolor{currentfill}%
\pgfsetfillopacity{0.965812}%
\pgfsetlinewidth{1.003750pt}%
\definecolor{currentstroke}{rgb}{0.121569,0.466667,0.705882}%
\pgfsetstrokecolor{currentstroke}%
\pgfsetstrokeopacity{0.965812}%
\pgfsetdash{}{0pt}%
\pgfpathmoveto{\pgfqpoint{2.121227in}{1.938993in}}%
\pgfpathcurveto{\pgfqpoint{2.129463in}{1.938993in}}{\pgfqpoint{2.137363in}{1.942266in}}{\pgfqpoint{2.143187in}{1.948090in}}%
\pgfpathcurveto{\pgfqpoint{2.149011in}{1.953914in}}{\pgfqpoint{2.152284in}{1.961814in}}{\pgfqpoint{2.152284in}{1.970050in}}%
\pgfpathcurveto{\pgfqpoint{2.152284in}{1.978286in}}{\pgfqpoint{2.149011in}{1.986186in}}{\pgfqpoint{2.143187in}{1.992010in}}%
\pgfpathcurveto{\pgfqpoint{2.137363in}{1.997834in}}{\pgfqpoint{2.129463in}{2.001106in}}{\pgfqpoint{2.121227in}{2.001106in}}%
\pgfpathcurveto{\pgfqpoint{2.112991in}{2.001106in}}{\pgfqpoint{2.105091in}{1.997834in}}{\pgfqpoint{2.099267in}{1.992010in}}%
\pgfpathcurveto{\pgfqpoint{2.093443in}{1.986186in}}{\pgfqpoint{2.090171in}{1.978286in}}{\pgfqpoint{2.090171in}{1.970050in}}%
\pgfpathcurveto{\pgfqpoint{2.090171in}{1.961814in}}{\pgfqpoint{2.093443in}{1.953914in}}{\pgfqpoint{2.099267in}{1.948090in}}%
\pgfpathcurveto{\pgfqpoint{2.105091in}{1.942266in}}{\pgfqpoint{2.112991in}{1.938993in}}{\pgfqpoint{2.121227in}{1.938993in}}%
\pgfpathclose%
\pgfusepath{stroke,fill}%
\end{pgfscope}%
\begin{pgfscope}%
\pgfpathrectangle{\pgfqpoint{0.100000in}{0.212622in}}{\pgfqpoint{3.696000in}{3.696000in}}%
\pgfusepath{clip}%
\pgfsetbuttcap%
\pgfsetroundjoin%
\definecolor{currentfill}{rgb}{0.121569,0.466667,0.705882}%
\pgfsetfillcolor{currentfill}%
\pgfsetfillopacity{0.967128}%
\pgfsetlinewidth{1.003750pt}%
\definecolor{currentstroke}{rgb}{0.121569,0.466667,0.705882}%
\pgfsetstrokecolor{currentstroke}%
\pgfsetstrokeopacity{0.967128}%
\pgfsetdash{}{0pt}%
\pgfpathmoveto{\pgfqpoint{2.128507in}{1.937625in}}%
\pgfpathcurveto{\pgfqpoint{2.136743in}{1.937625in}}{\pgfqpoint{2.144643in}{1.940897in}}{\pgfqpoint{2.150467in}{1.946721in}}%
\pgfpathcurveto{\pgfqpoint{2.156291in}{1.952545in}}{\pgfqpoint{2.159563in}{1.960445in}}{\pgfqpoint{2.159563in}{1.968681in}}%
\pgfpathcurveto{\pgfqpoint{2.159563in}{1.976918in}}{\pgfqpoint{2.156291in}{1.984818in}}{\pgfqpoint{2.150467in}{1.990642in}}%
\pgfpathcurveto{\pgfqpoint{2.144643in}{1.996466in}}{\pgfqpoint{2.136743in}{1.999738in}}{\pgfqpoint{2.128507in}{1.999738in}}%
\pgfpathcurveto{\pgfqpoint{2.120271in}{1.999738in}}{\pgfqpoint{2.112371in}{1.996466in}}{\pgfqpoint{2.106547in}{1.990642in}}%
\pgfpathcurveto{\pgfqpoint{2.100723in}{1.984818in}}{\pgfqpoint{2.097450in}{1.976918in}}{\pgfqpoint{2.097450in}{1.968681in}}%
\pgfpathcurveto{\pgfqpoint{2.097450in}{1.960445in}}{\pgfqpoint{2.100723in}{1.952545in}}{\pgfqpoint{2.106547in}{1.946721in}}%
\pgfpathcurveto{\pgfqpoint{2.112371in}{1.940897in}}{\pgfqpoint{2.120271in}{1.937625in}}{\pgfqpoint{2.128507in}{1.937625in}}%
\pgfpathclose%
\pgfusepath{stroke,fill}%
\end{pgfscope}%
\begin{pgfscope}%
\pgfpathrectangle{\pgfqpoint{0.100000in}{0.212622in}}{\pgfqpoint{3.696000in}{3.696000in}}%
\pgfusepath{clip}%
\pgfsetbuttcap%
\pgfsetroundjoin%
\definecolor{currentfill}{rgb}{0.121569,0.466667,0.705882}%
\pgfsetfillcolor{currentfill}%
\pgfsetfillopacity{0.967299}%
\pgfsetlinewidth{1.003750pt}%
\definecolor{currentstroke}{rgb}{0.121569,0.466667,0.705882}%
\pgfsetstrokecolor{currentstroke}%
\pgfsetstrokeopacity{0.967299}%
\pgfsetdash{}{0pt}%
\pgfpathmoveto{\pgfqpoint{2.469091in}{1.845156in}}%
\pgfpathcurveto{\pgfqpoint{2.477328in}{1.845156in}}{\pgfqpoint{2.485228in}{1.848428in}}{\pgfqpoint{2.491051in}{1.854252in}}%
\pgfpathcurveto{\pgfqpoint{2.496875in}{1.860076in}}{\pgfqpoint{2.500148in}{1.867976in}}{\pgfqpoint{2.500148in}{1.876212in}}%
\pgfpathcurveto{\pgfqpoint{2.500148in}{1.884448in}}{\pgfqpoint{2.496875in}{1.892348in}}{\pgfqpoint{2.491051in}{1.898172in}}%
\pgfpathcurveto{\pgfqpoint{2.485228in}{1.903996in}}{\pgfqpoint{2.477328in}{1.907269in}}{\pgfqpoint{2.469091in}{1.907269in}}%
\pgfpathcurveto{\pgfqpoint{2.460855in}{1.907269in}}{\pgfqpoint{2.452955in}{1.903996in}}{\pgfqpoint{2.447131in}{1.898172in}}%
\pgfpathcurveto{\pgfqpoint{2.441307in}{1.892348in}}{\pgfqpoint{2.438035in}{1.884448in}}{\pgfqpoint{2.438035in}{1.876212in}}%
\pgfpathcurveto{\pgfqpoint{2.438035in}{1.867976in}}{\pgfqpoint{2.441307in}{1.860076in}}{\pgfqpoint{2.447131in}{1.854252in}}%
\pgfpathcurveto{\pgfqpoint{2.452955in}{1.848428in}}{\pgfqpoint{2.460855in}{1.845156in}}{\pgfqpoint{2.469091in}{1.845156in}}%
\pgfpathclose%
\pgfusepath{stroke,fill}%
\end{pgfscope}%
\begin{pgfscope}%
\pgfpathrectangle{\pgfqpoint{0.100000in}{0.212622in}}{\pgfqpoint{3.696000in}{3.696000in}}%
\pgfusepath{clip}%
\pgfsetbuttcap%
\pgfsetroundjoin%
\definecolor{currentfill}{rgb}{0.121569,0.466667,0.705882}%
\pgfsetfillcolor{currentfill}%
\pgfsetfillopacity{0.967629}%
\pgfsetlinewidth{1.003750pt}%
\definecolor{currentstroke}{rgb}{0.121569,0.466667,0.705882}%
\pgfsetstrokecolor{currentstroke}%
\pgfsetstrokeopacity{0.967629}%
\pgfsetdash{}{0pt}%
\pgfpathmoveto{\pgfqpoint{2.142868in}{1.927679in}}%
\pgfpathcurveto{\pgfqpoint{2.151104in}{1.927679in}}{\pgfqpoint{2.159004in}{1.930951in}}{\pgfqpoint{2.164828in}{1.936775in}}%
\pgfpathcurveto{\pgfqpoint{2.170652in}{1.942599in}}{\pgfqpoint{2.173924in}{1.950499in}}{\pgfqpoint{2.173924in}{1.958735in}}%
\pgfpathcurveto{\pgfqpoint{2.173924in}{1.966971in}}{\pgfqpoint{2.170652in}{1.974872in}}{\pgfqpoint{2.164828in}{1.980695in}}%
\pgfpathcurveto{\pgfqpoint{2.159004in}{1.986519in}}{\pgfqpoint{2.151104in}{1.989792in}}{\pgfqpoint{2.142868in}{1.989792in}}%
\pgfpathcurveto{\pgfqpoint{2.134632in}{1.989792in}}{\pgfqpoint{2.126731in}{1.986519in}}{\pgfqpoint{2.120908in}{1.980695in}}%
\pgfpathcurveto{\pgfqpoint{2.115084in}{1.974872in}}{\pgfqpoint{2.111811in}{1.966971in}}{\pgfqpoint{2.111811in}{1.958735in}}%
\pgfpathcurveto{\pgfqpoint{2.111811in}{1.950499in}}{\pgfqpoint{2.115084in}{1.942599in}}{\pgfqpoint{2.120908in}{1.936775in}}%
\pgfpathcurveto{\pgfqpoint{2.126731in}{1.930951in}}{\pgfqpoint{2.134632in}{1.927679in}}{\pgfqpoint{2.142868in}{1.927679in}}%
\pgfpathclose%
\pgfusepath{stroke,fill}%
\end{pgfscope}%
\begin{pgfscope}%
\pgfpathrectangle{\pgfqpoint{0.100000in}{0.212622in}}{\pgfqpoint{3.696000in}{3.696000in}}%
\pgfusepath{clip}%
\pgfsetbuttcap%
\pgfsetroundjoin%
\definecolor{currentfill}{rgb}{0.121569,0.466667,0.705882}%
\pgfsetfillcolor{currentfill}%
\pgfsetfillopacity{0.968928}%
\pgfsetlinewidth{1.003750pt}%
\definecolor{currentstroke}{rgb}{0.121569,0.466667,0.705882}%
\pgfsetstrokecolor{currentstroke}%
\pgfsetstrokeopacity{0.968928}%
\pgfsetdash{}{0pt}%
\pgfpathmoveto{\pgfqpoint{2.464762in}{1.843070in}}%
\pgfpathcurveto{\pgfqpoint{2.472998in}{1.843070in}}{\pgfqpoint{2.480898in}{1.846342in}}{\pgfqpoint{2.486722in}{1.852166in}}%
\pgfpathcurveto{\pgfqpoint{2.492546in}{1.857990in}}{\pgfqpoint{2.495819in}{1.865890in}}{\pgfqpoint{2.495819in}{1.874126in}}%
\pgfpathcurveto{\pgfqpoint{2.495819in}{1.882362in}}{\pgfqpoint{2.492546in}{1.890262in}}{\pgfqpoint{2.486722in}{1.896086in}}%
\pgfpathcurveto{\pgfqpoint{2.480898in}{1.901910in}}{\pgfqpoint{2.472998in}{1.905183in}}{\pgfqpoint{2.464762in}{1.905183in}}%
\pgfpathcurveto{\pgfqpoint{2.456526in}{1.905183in}}{\pgfqpoint{2.448626in}{1.901910in}}{\pgfqpoint{2.442802in}{1.896086in}}%
\pgfpathcurveto{\pgfqpoint{2.436978in}{1.890262in}}{\pgfqpoint{2.433706in}{1.882362in}}{\pgfqpoint{2.433706in}{1.874126in}}%
\pgfpathcurveto{\pgfqpoint{2.433706in}{1.865890in}}{\pgfqpoint{2.436978in}{1.857990in}}{\pgfqpoint{2.442802in}{1.852166in}}%
\pgfpathcurveto{\pgfqpoint{2.448626in}{1.846342in}}{\pgfqpoint{2.456526in}{1.843070in}}{\pgfqpoint{2.464762in}{1.843070in}}%
\pgfpathclose%
\pgfusepath{stroke,fill}%
\end{pgfscope}%
\begin{pgfscope}%
\pgfpathrectangle{\pgfqpoint{0.100000in}{0.212622in}}{\pgfqpoint{3.696000in}{3.696000in}}%
\pgfusepath{clip}%
\pgfsetbuttcap%
\pgfsetroundjoin%
\definecolor{currentfill}{rgb}{0.121569,0.466667,0.705882}%
\pgfsetfillcolor{currentfill}%
\pgfsetfillopacity{0.969903}%
\pgfsetlinewidth{1.003750pt}%
\definecolor{currentstroke}{rgb}{0.121569,0.466667,0.705882}%
\pgfsetstrokecolor{currentstroke}%
\pgfsetstrokeopacity{0.969903}%
\pgfsetdash{}{0pt}%
\pgfpathmoveto{\pgfqpoint{2.462289in}{1.842530in}}%
\pgfpathcurveto{\pgfqpoint{2.470525in}{1.842530in}}{\pgfqpoint{2.478425in}{1.845802in}}{\pgfqpoint{2.484249in}{1.851626in}}%
\pgfpathcurveto{\pgfqpoint{2.490073in}{1.857450in}}{\pgfqpoint{2.493345in}{1.865350in}}{\pgfqpoint{2.493345in}{1.873586in}}%
\pgfpathcurveto{\pgfqpoint{2.493345in}{1.881823in}}{\pgfqpoint{2.490073in}{1.889723in}}{\pgfqpoint{2.484249in}{1.895547in}}%
\pgfpathcurveto{\pgfqpoint{2.478425in}{1.901370in}}{\pgfqpoint{2.470525in}{1.904643in}}{\pgfqpoint{2.462289in}{1.904643in}}%
\pgfpathcurveto{\pgfqpoint{2.454053in}{1.904643in}}{\pgfqpoint{2.446153in}{1.901370in}}{\pgfqpoint{2.440329in}{1.895547in}}%
\pgfpathcurveto{\pgfqpoint{2.434505in}{1.889723in}}{\pgfqpoint{2.431232in}{1.881823in}}{\pgfqpoint{2.431232in}{1.873586in}}%
\pgfpathcurveto{\pgfqpoint{2.431232in}{1.865350in}}{\pgfqpoint{2.434505in}{1.857450in}}{\pgfqpoint{2.440329in}{1.851626in}}%
\pgfpathcurveto{\pgfqpoint{2.446153in}{1.845802in}}{\pgfqpoint{2.454053in}{1.842530in}}{\pgfqpoint{2.462289in}{1.842530in}}%
\pgfpathclose%
\pgfusepath{stroke,fill}%
\end{pgfscope}%
\begin{pgfscope}%
\pgfpathrectangle{\pgfqpoint{0.100000in}{0.212622in}}{\pgfqpoint{3.696000in}{3.696000in}}%
\pgfusepath{clip}%
\pgfsetbuttcap%
\pgfsetroundjoin%
\definecolor{currentfill}{rgb}{0.121569,0.466667,0.705882}%
\pgfsetfillcolor{currentfill}%
\pgfsetfillopacity{0.970397}%
\pgfsetlinewidth{1.003750pt}%
\definecolor{currentstroke}{rgb}{0.121569,0.466667,0.705882}%
\pgfsetstrokecolor{currentstroke}%
\pgfsetstrokeopacity{0.970397}%
\pgfsetdash{}{0pt}%
\pgfpathmoveto{\pgfqpoint{2.461125in}{1.841697in}}%
\pgfpathcurveto{\pgfqpoint{2.469361in}{1.841697in}}{\pgfqpoint{2.477261in}{1.844969in}}{\pgfqpoint{2.483085in}{1.850793in}}%
\pgfpathcurveto{\pgfqpoint{2.488909in}{1.856617in}}{\pgfqpoint{2.492181in}{1.864517in}}{\pgfqpoint{2.492181in}{1.872753in}}%
\pgfpathcurveto{\pgfqpoint{2.492181in}{1.880990in}}{\pgfqpoint{2.488909in}{1.888890in}}{\pgfqpoint{2.483085in}{1.894714in}}%
\pgfpathcurveto{\pgfqpoint{2.477261in}{1.900537in}}{\pgfqpoint{2.469361in}{1.903810in}}{\pgfqpoint{2.461125in}{1.903810in}}%
\pgfpathcurveto{\pgfqpoint{2.452888in}{1.903810in}}{\pgfqpoint{2.444988in}{1.900537in}}{\pgfqpoint{2.439164in}{1.894714in}}%
\pgfpathcurveto{\pgfqpoint{2.433340in}{1.888890in}}{\pgfqpoint{2.430068in}{1.880990in}}{\pgfqpoint{2.430068in}{1.872753in}}%
\pgfpathcurveto{\pgfqpoint{2.430068in}{1.864517in}}{\pgfqpoint{2.433340in}{1.856617in}}{\pgfqpoint{2.439164in}{1.850793in}}%
\pgfpathcurveto{\pgfqpoint{2.444988in}{1.844969in}}{\pgfqpoint{2.452888in}{1.841697in}}{\pgfqpoint{2.461125in}{1.841697in}}%
\pgfpathclose%
\pgfusepath{stroke,fill}%
\end{pgfscope}%
\begin{pgfscope}%
\pgfpathrectangle{\pgfqpoint{0.100000in}{0.212622in}}{\pgfqpoint{3.696000in}{3.696000in}}%
\pgfusepath{clip}%
\pgfsetbuttcap%
\pgfsetroundjoin%
\definecolor{currentfill}{rgb}{0.121569,0.466667,0.705882}%
\pgfsetfillcolor{currentfill}%
\pgfsetfillopacity{0.970699}%
\pgfsetlinewidth{1.003750pt}%
\definecolor{currentstroke}{rgb}{0.121569,0.466667,0.705882}%
\pgfsetstrokecolor{currentstroke}%
\pgfsetstrokeopacity{0.970699}%
\pgfsetdash{}{0pt}%
\pgfpathmoveto{\pgfqpoint{2.460421in}{1.841507in}}%
\pgfpathcurveto{\pgfqpoint{2.468657in}{1.841507in}}{\pgfqpoint{2.476557in}{1.844779in}}{\pgfqpoint{2.482381in}{1.850603in}}%
\pgfpathcurveto{\pgfqpoint{2.488205in}{1.856427in}}{\pgfqpoint{2.491478in}{1.864327in}}{\pgfqpoint{2.491478in}{1.872563in}}%
\pgfpathcurveto{\pgfqpoint{2.491478in}{1.880800in}}{\pgfqpoint{2.488205in}{1.888700in}}{\pgfqpoint{2.482381in}{1.894524in}}%
\pgfpathcurveto{\pgfqpoint{2.476557in}{1.900348in}}{\pgfqpoint{2.468657in}{1.903620in}}{\pgfqpoint{2.460421in}{1.903620in}}%
\pgfpathcurveto{\pgfqpoint{2.452185in}{1.903620in}}{\pgfqpoint{2.444285in}{1.900348in}}{\pgfqpoint{2.438461in}{1.894524in}}%
\pgfpathcurveto{\pgfqpoint{2.432637in}{1.888700in}}{\pgfqpoint{2.429365in}{1.880800in}}{\pgfqpoint{2.429365in}{1.872563in}}%
\pgfpathcurveto{\pgfqpoint{2.429365in}{1.864327in}}{\pgfqpoint{2.432637in}{1.856427in}}{\pgfqpoint{2.438461in}{1.850603in}}%
\pgfpathcurveto{\pgfqpoint{2.444285in}{1.844779in}}{\pgfqpoint{2.452185in}{1.841507in}}{\pgfqpoint{2.460421in}{1.841507in}}%
\pgfpathclose%
\pgfusepath{stroke,fill}%
\end{pgfscope}%
\begin{pgfscope}%
\pgfpathrectangle{\pgfqpoint{0.100000in}{0.212622in}}{\pgfqpoint{3.696000in}{3.696000in}}%
\pgfusepath{clip}%
\pgfsetbuttcap%
\pgfsetroundjoin%
\definecolor{currentfill}{rgb}{0.121569,0.466667,0.705882}%
\pgfsetfillcolor{currentfill}%
\pgfsetfillopacity{0.970856}%
\pgfsetlinewidth{1.003750pt}%
\definecolor{currentstroke}{rgb}{0.121569,0.466667,0.705882}%
\pgfsetstrokecolor{currentstroke}%
\pgfsetstrokeopacity{0.970856}%
\pgfsetdash{}{0pt}%
\pgfpathmoveto{\pgfqpoint{2.460096in}{1.841259in}}%
\pgfpathcurveto{\pgfqpoint{2.468332in}{1.841259in}}{\pgfqpoint{2.476232in}{1.844531in}}{\pgfqpoint{2.482056in}{1.850355in}}%
\pgfpathcurveto{\pgfqpoint{2.487880in}{1.856179in}}{\pgfqpoint{2.491153in}{1.864079in}}{\pgfqpoint{2.491153in}{1.872316in}}%
\pgfpathcurveto{\pgfqpoint{2.491153in}{1.880552in}}{\pgfqpoint{2.487880in}{1.888452in}}{\pgfqpoint{2.482056in}{1.894276in}}%
\pgfpathcurveto{\pgfqpoint{2.476232in}{1.900100in}}{\pgfqpoint{2.468332in}{1.903372in}}{\pgfqpoint{2.460096in}{1.903372in}}%
\pgfpathcurveto{\pgfqpoint{2.451860in}{1.903372in}}{\pgfqpoint{2.443960in}{1.900100in}}{\pgfqpoint{2.438136in}{1.894276in}}%
\pgfpathcurveto{\pgfqpoint{2.432312in}{1.888452in}}{\pgfqpoint{2.429040in}{1.880552in}}{\pgfqpoint{2.429040in}{1.872316in}}%
\pgfpathcurveto{\pgfqpoint{2.429040in}{1.864079in}}{\pgfqpoint{2.432312in}{1.856179in}}{\pgfqpoint{2.438136in}{1.850355in}}%
\pgfpathcurveto{\pgfqpoint{2.443960in}{1.844531in}}{\pgfqpoint{2.451860in}{1.841259in}}{\pgfqpoint{2.460096in}{1.841259in}}%
\pgfpathclose%
\pgfusepath{stroke,fill}%
\end{pgfscope}%
\begin{pgfscope}%
\pgfpathrectangle{\pgfqpoint{0.100000in}{0.212622in}}{\pgfqpoint{3.696000in}{3.696000in}}%
\pgfusepath{clip}%
\pgfsetbuttcap%
\pgfsetroundjoin%
\definecolor{currentfill}{rgb}{0.121569,0.466667,0.705882}%
\pgfsetfillcolor{currentfill}%
\pgfsetfillopacity{0.970948}%
\pgfsetlinewidth{1.003750pt}%
\definecolor{currentstroke}{rgb}{0.121569,0.466667,0.705882}%
\pgfsetstrokecolor{currentstroke}%
\pgfsetstrokeopacity{0.970948}%
\pgfsetdash{}{0pt}%
\pgfpathmoveto{\pgfqpoint{2.459903in}{1.841180in}}%
\pgfpathcurveto{\pgfqpoint{2.468139in}{1.841180in}}{\pgfqpoint{2.476039in}{1.844452in}}{\pgfqpoint{2.481863in}{1.850276in}}%
\pgfpathcurveto{\pgfqpoint{2.487687in}{1.856100in}}{\pgfqpoint{2.490959in}{1.864000in}}{\pgfqpoint{2.490959in}{1.872236in}}%
\pgfpathcurveto{\pgfqpoint{2.490959in}{1.880473in}}{\pgfqpoint{2.487687in}{1.888373in}}{\pgfqpoint{2.481863in}{1.894197in}}%
\pgfpathcurveto{\pgfqpoint{2.476039in}{1.900021in}}{\pgfqpoint{2.468139in}{1.903293in}}{\pgfqpoint{2.459903in}{1.903293in}}%
\pgfpathcurveto{\pgfqpoint{2.451666in}{1.903293in}}{\pgfqpoint{2.443766in}{1.900021in}}{\pgfqpoint{2.437942in}{1.894197in}}%
\pgfpathcurveto{\pgfqpoint{2.432119in}{1.888373in}}{\pgfqpoint{2.428846in}{1.880473in}}{\pgfqpoint{2.428846in}{1.872236in}}%
\pgfpathcurveto{\pgfqpoint{2.428846in}{1.864000in}}{\pgfqpoint{2.432119in}{1.856100in}}{\pgfqpoint{2.437942in}{1.850276in}}%
\pgfpathcurveto{\pgfqpoint{2.443766in}{1.844452in}}{\pgfqpoint{2.451666in}{1.841180in}}{\pgfqpoint{2.459903in}{1.841180in}}%
\pgfpathclose%
\pgfusepath{stroke,fill}%
\end{pgfscope}%
\begin{pgfscope}%
\pgfpathrectangle{\pgfqpoint{0.100000in}{0.212622in}}{\pgfqpoint{3.696000in}{3.696000in}}%
\pgfusepath{clip}%
\pgfsetbuttcap%
\pgfsetroundjoin%
\definecolor{currentfill}{rgb}{0.121569,0.466667,0.705882}%
\pgfsetfillcolor{currentfill}%
\pgfsetfillopacity{0.971961}%
\pgfsetlinewidth{1.003750pt}%
\definecolor{currentstroke}{rgb}{0.121569,0.466667,0.705882}%
\pgfsetstrokecolor{currentstroke}%
\pgfsetstrokeopacity{0.971961}%
\pgfsetdash{}{0pt}%
\pgfpathmoveto{\pgfqpoint{2.457980in}{1.839338in}}%
\pgfpathcurveto{\pgfqpoint{2.466217in}{1.839338in}}{\pgfqpoint{2.474117in}{1.842611in}}{\pgfqpoint{2.479941in}{1.848435in}}%
\pgfpathcurveto{\pgfqpoint{2.485765in}{1.854259in}}{\pgfqpoint{2.489037in}{1.862159in}}{\pgfqpoint{2.489037in}{1.870395in}}%
\pgfpathcurveto{\pgfqpoint{2.489037in}{1.878631in}}{\pgfqpoint{2.485765in}{1.886531in}}{\pgfqpoint{2.479941in}{1.892355in}}%
\pgfpathcurveto{\pgfqpoint{2.474117in}{1.898179in}}{\pgfqpoint{2.466217in}{1.901451in}}{\pgfqpoint{2.457980in}{1.901451in}}%
\pgfpathcurveto{\pgfqpoint{2.449744in}{1.901451in}}{\pgfqpoint{2.441844in}{1.898179in}}{\pgfqpoint{2.436020in}{1.892355in}}%
\pgfpathcurveto{\pgfqpoint{2.430196in}{1.886531in}}{\pgfqpoint{2.426924in}{1.878631in}}{\pgfqpoint{2.426924in}{1.870395in}}%
\pgfpathcurveto{\pgfqpoint{2.426924in}{1.862159in}}{\pgfqpoint{2.430196in}{1.854259in}}{\pgfqpoint{2.436020in}{1.848435in}}%
\pgfpathcurveto{\pgfqpoint{2.441844in}{1.842611in}}{\pgfqpoint{2.449744in}{1.839338in}}{\pgfqpoint{2.457980in}{1.839338in}}%
\pgfpathclose%
\pgfusepath{stroke,fill}%
\end{pgfscope}%
\begin{pgfscope}%
\pgfpathrectangle{\pgfqpoint{0.100000in}{0.212622in}}{\pgfqpoint{3.696000in}{3.696000in}}%
\pgfusepath{clip}%
\pgfsetbuttcap%
\pgfsetroundjoin%
\definecolor{currentfill}{rgb}{0.121569,0.466667,0.705882}%
\pgfsetfillcolor{currentfill}%
\pgfsetfillopacity{0.972508}%
\pgfsetlinewidth{1.003750pt}%
\definecolor{currentstroke}{rgb}{0.121569,0.466667,0.705882}%
\pgfsetstrokecolor{currentstroke}%
\pgfsetstrokeopacity{0.972508}%
\pgfsetdash{}{0pt}%
\pgfpathmoveto{\pgfqpoint{2.166009in}{1.923538in}}%
\pgfpathcurveto{\pgfqpoint{2.174246in}{1.923538in}}{\pgfqpoint{2.182146in}{1.926811in}}{\pgfqpoint{2.187970in}{1.932635in}}%
\pgfpathcurveto{\pgfqpoint{2.193794in}{1.938459in}}{\pgfqpoint{2.197066in}{1.946359in}}{\pgfqpoint{2.197066in}{1.954595in}}%
\pgfpathcurveto{\pgfqpoint{2.197066in}{1.962831in}}{\pgfqpoint{2.193794in}{1.970731in}}{\pgfqpoint{2.187970in}{1.976555in}}%
\pgfpathcurveto{\pgfqpoint{2.182146in}{1.982379in}}{\pgfqpoint{2.174246in}{1.985651in}}{\pgfqpoint{2.166009in}{1.985651in}}%
\pgfpathcurveto{\pgfqpoint{2.157773in}{1.985651in}}{\pgfqpoint{2.149873in}{1.982379in}}{\pgfqpoint{2.144049in}{1.976555in}}%
\pgfpathcurveto{\pgfqpoint{2.138225in}{1.970731in}}{\pgfqpoint{2.134953in}{1.962831in}}{\pgfqpoint{2.134953in}{1.954595in}}%
\pgfpathcurveto{\pgfqpoint{2.134953in}{1.946359in}}{\pgfqpoint{2.138225in}{1.938459in}}{\pgfqpoint{2.144049in}{1.932635in}}%
\pgfpathcurveto{\pgfqpoint{2.149873in}{1.926811in}}{\pgfqpoint{2.157773in}{1.923538in}}{\pgfqpoint{2.166009in}{1.923538in}}%
\pgfpathclose%
\pgfusepath{stroke,fill}%
\end{pgfscope}%
\begin{pgfscope}%
\pgfpathrectangle{\pgfqpoint{0.100000in}{0.212622in}}{\pgfqpoint{3.696000in}{3.696000in}}%
\pgfusepath{clip}%
\pgfsetbuttcap%
\pgfsetroundjoin%
\definecolor{currentfill}{rgb}{0.121569,0.466667,0.705882}%
\pgfsetfillcolor{currentfill}%
\pgfsetfillopacity{0.973702}%
\pgfsetlinewidth{1.003750pt}%
\definecolor{currentstroke}{rgb}{0.121569,0.466667,0.705882}%
\pgfsetstrokecolor{currentstroke}%
\pgfsetstrokeopacity{0.973702}%
\pgfsetdash{}{0pt}%
\pgfpathmoveto{\pgfqpoint{2.454047in}{1.837479in}}%
\pgfpathcurveto{\pgfqpoint{2.462283in}{1.837479in}}{\pgfqpoint{2.470183in}{1.840752in}}{\pgfqpoint{2.476007in}{1.846576in}}%
\pgfpathcurveto{\pgfqpoint{2.481831in}{1.852400in}}{\pgfqpoint{2.485103in}{1.860300in}}{\pgfqpoint{2.485103in}{1.868536in}}%
\pgfpathcurveto{\pgfqpoint{2.485103in}{1.876772in}}{\pgfqpoint{2.481831in}{1.884672in}}{\pgfqpoint{2.476007in}{1.890496in}}%
\pgfpathcurveto{\pgfqpoint{2.470183in}{1.896320in}}{\pgfqpoint{2.462283in}{1.899592in}}{\pgfqpoint{2.454047in}{1.899592in}}%
\pgfpathcurveto{\pgfqpoint{2.445810in}{1.899592in}}{\pgfqpoint{2.437910in}{1.896320in}}{\pgfqpoint{2.432086in}{1.890496in}}%
\pgfpathcurveto{\pgfqpoint{2.426262in}{1.884672in}}{\pgfqpoint{2.422990in}{1.876772in}}{\pgfqpoint{2.422990in}{1.868536in}}%
\pgfpathcurveto{\pgfqpoint{2.422990in}{1.860300in}}{\pgfqpoint{2.426262in}{1.852400in}}{\pgfqpoint{2.432086in}{1.846576in}}%
\pgfpathcurveto{\pgfqpoint{2.437910in}{1.840752in}}{\pgfqpoint{2.445810in}{1.837479in}}{\pgfqpoint{2.454047in}{1.837479in}}%
\pgfpathclose%
\pgfusepath{stroke,fill}%
\end{pgfscope}%
\begin{pgfscope}%
\pgfpathrectangle{\pgfqpoint{0.100000in}{0.212622in}}{\pgfqpoint{3.696000in}{3.696000in}}%
\pgfusepath{clip}%
\pgfsetbuttcap%
\pgfsetroundjoin%
\definecolor{currentfill}{rgb}{0.121569,0.466667,0.705882}%
\pgfsetfillcolor{currentfill}%
\pgfsetfillopacity{0.973917}%
\pgfsetlinewidth{1.003750pt}%
\definecolor{currentstroke}{rgb}{0.121569,0.466667,0.705882}%
\pgfsetstrokecolor{currentstroke}%
\pgfsetstrokeopacity{0.973917}%
\pgfsetdash{}{0pt}%
\pgfpathmoveto{\pgfqpoint{2.188848in}{1.908392in}}%
\pgfpathcurveto{\pgfqpoint{2.197084in}{1.908392in}}{\pgfqpoint{2.204984in}{1.911665in}}{\pgfqpoint{2.210808in}{1.917489in}}%
\pgfpathcurveto{\pgfqpoint{2.216632in}{1.923313in}}{\pgfqpoint{2.219905in}{1.931213in}}{\pgfqpoint{2.219905in}{1.939449in}}%
\pgfpathcurveto{\pgfqpoint{2.219905in}{1.947685in}}{\pgfqpoint{2.216632in}{1.955585in}}{\pgfqpoint{2.210808in}{1.961409in}}%
\pgfpathcurveto{\pgfqpoint{2.204984in}{1.967233in}}{\pgfqpoint{2.197084in}{1.970505in}}{\pgfqpoint{2.188848in}{1.970505in}}%
\pgfpathcurveto{\pgfqpoint{2.180612in}{1.970505in}}{\pgfqpoint{2.172712in}{1.967233in}}{\pgfqpoint{2.166888in}{1.961409in}}%
\pgfpathcurveto{\pgfqpoint{2.161064in}{1.955585in}}{\pgfqpoint{2.157792in}{1.947685in}}{\pgfqpoint{2.157792in}{1.939449in}}%
\pgfpathcurveto{\pgfqpoint{2.157792in}{1.931213in}}{\pgfqpoint{2.161064in}{1.923313in}}{\pgfqpoint{2.166888in}{1.917489in}}%
\pgfpathcurveto{\pgfqpoint{2.172712in}{1.911665in}}{\pgfqpoint{2.180612in}{1.908392in}}{\pgfqpoint{2.188848in}{1.908392in}}%
\pgfpathclose%
\pgfusepath{stroke,fill}%
\end{pgfscope}%
\begin{pgfscope}%
\pgfpathrectangle{\pgfqpoint{0.100000in}{0.212622in}}{\pgfqpoint{3.696000in}{3.696000in}}%
\pgfusepath{clip}%
\pgfsetbuttcap%
\pgfsetroundjoin%
\definecolor{currentfill}{rgb}{0.121569,0.466667,0.705882}%
\pgfsetfillcolor{currentfill}%
\pgfsetfillopacity{0.976188}%
\pgfsetlinewidth{1.003750pt}%
\definecolor{currentstroke}{rgb}{0.121569,0.466667,0.705882}%
\pgfsetstrokecolor{currentstroke}%
\pgfsetstrokeopacity{0.976188}%
\pgfsetdash{}{0pt}%
\pgfpathmoveto{\pgfqpoint{2.207074in}{1.902040in}}%
\pgfpathcurveto{\pgfqpoint{2.215310in}{1.902040in}}{\pgfqpoint{2.223210in}{1.905312in}}{\pgfqpoint{2.229034in}{1.911136in}}%
\pgfpathcurveto{\pgfqpoint{2.234858in}{1.916960in}}{\pgfqpoint{2.238130in}{1.924860in}}{\pgfqpoint{2.238130in}{1.933096in}}%
\pgfpathcurveto{\pgfqpoint{2.238130in}{1.941332in}}{\pgfqpoint{2.234858in}{1.949233in}}{\pgfqpoint{2.229034in}{1.955056in}}%
\pgfpathcurveto{\pgfqpoint{2.223210in}{1.960880in}}{\pgfqpoint{2.215310in}{1.964153in}}{\pgfqpoint{2.207074in}{1.964153in}}%
\pgfpathcurveto{\pgfqpoint{2.198838in}{1.964153in}}{\pgfqpoint{2.190938in}{1.960880in}}{\pgfqpoint{2.185114in}{1.955056in}}%
\pgfpathcurveto{\pgfqpoint{2.179290in}{1.949233in}}{\pgfqpoint{2.176017in}{1.941332in}}{\pgfqpoint{2.176017in}{1.933096in}}%
\pgfpathcurveto{\pgfqpoint{2.176017in}{1.924860in}}{\pgfqpoint{2.179290in}{1.916960in}}{\pgfqpoint{2.185114in}{1.911136in}}%
\pgfpathcurveto{\pgfqpoint{2.190938in}{1.905312in}}{\pgfqpoint{2.198838in}{1.902040in}}{\pgfqpoint{2.207074in}{1.902040in}}%
\pgfpathclose%
\pgfusepath{stroke,fill}%
\end{pgfscope}%
\begin{pgfscope}%
\pgfpathrectangle{\pgfqpoint{0.100000in}{0.212622in}}{\pgfqpoint{3.696000in}{3.696000in}}%
\pgfusepath{clip}%
\pgfsetbuttcap%
\pgfsetroundjoin%
\definecolor{currentfill}{rgb}{0.121569,0.466667,0.705882}%
\pgfsetfillcolor{currentfill}%
\pgfsetfillopacity{0.976517}%
\pgfsetlinewidth{1.003750pt}%
\definecolor{currentstroke}{rgb}{0.121569,0.466667,0.705882}%
\pgfsetstrokecolor{currentstroke}%
\pgfsetstrokeopacity{0.976517}%
\pgfsetdash{}{0pt}%
\pgfpathmoveto{\pgfqpoint{2.448119in}{1.834659in}}%
\pgfpathcurveto{\pgfqpoint{2.456355in}{1.834659in}}{\pgfqpoint{2.464255in}{1.837931in}}{\pgfqpoint{2.470079in}{1.843755in}}%
\pgfpathcurveto{\pgfqpoint{2.475903in}{1.849579in}}{\pgfqpoint{2.479175in}{1.857479in}}{\pgfqpoint{2.479175in}{1.865715in}}%
\pgfpathcurveto{\pgfqpoint{2.479175in}{1.873951in}}{\pgfqpoint{2.475903in}{1.881851in}}{\pgfqpoint{2.470079in}{1.887675in}}%
\pgfpathcurveto{\pgfqpoint{2.464255in}{1.893499in}}{\pgfqpoint{2.456355in}{1.896772in}}{\pgfqpoint{2.448119in}{1.896772in}}%
\pgfpathcurveto{\pgfqpoint{2.439882in}{1.896772in}}{\pgfqpoint{2.431982in}{1.893499in}}{\pgfqpoint{2.426158in}{1.887675in}}%
\pgfpathcurveto{\pgfqpoint{2.420334in}{1.881851in}}{\pgfqpoint{2.417062in}{1.873951in}}{\pgfqpoint{2.417062in}{1.865715in}}%
\pgfpathcurveto{\pgfqpoint{2.417062in}{1.857479in}}{\pgfqpoint{2.420334in}{1.849579in}}{\pgfqpoint{2.426158in}{1.843755in}}%
\pgfpathcurveto{\pgfqpoint{2.431982in}{1.837931in}}{\pgfqpoint{2.439882in}{1.834659in}}{\pgfqpoint{2.448119in}{1.834659in}}%
\pgfpathclose%
\pgfusepath{stroke,fill}%
\end{pgfscope}%
\begin{pgfscope}%
\pgfpathrectangle{\pgfqpoint{0.100000in}{0.212622in}}{\pgfqpoint{3.696000in}{3.696000in}}%
\pgfusepath{clip}%
\pgfsetbuttcap%
\pgfsetroundjoin%
\definecolor{currentfill}{rgb}{0.121569,0.466667,0.705882}%
\pgfsetfillcolor{currentfill}%
\pgfsetfillopacity{0.979569}%
\pgfsetlinewidth{1.003750pt}%
\definecolor{currentstroke}{rgb}{0.121569,0.466667,0.705882}%
\pgfsetstrokecolor{currentstroke}%
\pgfsetstrokeopacity{0.979569}%
\pgfsetdash{}{0pt}%
\pgfpathmoveto{\pgfqpoint{2.439689in}{1.830584in}}%
\pgfpathcurveto{\pgfqpoint{2.447926in}{1.830584in}}{\pgfqpoint{2.455826in}{1.833856in}}{\pgfqpoint{2.461650in}{1.839680in}}%
\pgfpathcurveto{\pgfqpoint{2.467473in}{1.845504in}}{\pgfqpoint{2.470746in}{1.853404in}}{\pgfqpoint{2.470746in}{1.861640in}}%
\pgfpathcurveto{\pgfqpoint{2.470746in}{1.869877in}}{\pgfqpoint{2.467473in}{1.877777in}}{\pgfqpoint{2.461650in}{1.883601in}}%
\pgfpathcurveto{\pgfqpoint{2.455826in}{1.889425in}}{\pgfqpoint{2.447926in}{1.892697in}}{\pgfqpoint{2.439689in}{1.892697in}}%
\pgfpathcurveto{\pgfqpoint{2.431453in}{1.892697in}}{\pgfqpoint{2.423553in}{1.889425in}}{\pgfqpoint{2.417729in}{1.883601in}}%
\pgfpathcurveto{\pgfqpoint{2.411905in}{1.877777in}}{\pgfqpoint{2.408633in}{1.869877in}}{\pgfqpoint{2.408633in}{1.861640in}}%
\pgfpathcurveto{\pgfqpoint{2.408633in}{1.853404in}}{\pgfqpoint{2.411905in}{1.845504in}}{\pgfqpoint{2.417729in}{1.839680in}}%
\pgfpathcurveto{\pgfqpoint{2.423553in}{1.833856in}}{\pgfqpoint{2.431453in}{1.830584in}}{\pgfqpoint{2.439689in}{1.830584in}}%
\pgfpathclose%
\pgfusepath{stroke,fill}%
\end{pgfscope}%
\begin{pgfscope}%
\pgfpathrectangle{\pgfqpoint{0.100000in}{0.212622in}}{\pgfqpoint{3.696000in}{3.696000in}}%
\pgfusepath{clip}%
\pgfsetbuttcap%
\pgfsetroundjoin%
\definecolor{currentfill}{rgb}{0.121569,0.466667,0.705882}%
\pgfsetfillcolor{currentfill}%
\pgfsetfillopacity{0.981423}%
\pgfsetlinewidth{1.003750pt}%
\definecolor{currentstroke}{rgb}{0.121569,0.466667,0.705882}%
\pgfsetstrokecolor{currentstroke}%
\pgfsetstrokeopacity{0.981423}%
\pgfsetdash{}{0pt}%
\pgfpathmoveto{\pgfqpoint{2.435957in}{1.828264in}}%
\pgfpathcurveto{\pgfqpoint{2.444193in}{1.828264in}}{\pgfqpoint{2.452093in}{1.831537in}}{\pgfqpoint{2.457917in}{1.837361in}}%
\pgfpathcurveto{\pgfqpoint{2.463741in}{1.843185in}}{\pgfqpoint{2.467014in}{1.851085in}}{\pgfqpoint{2.467014in}{1.859321in}}%
\pgfpathcurveto{\pgfqpoint{2.467014in}{1.867557in}}{\pgfqpoint{2.463741in}{1.875457in}}{\pgfqpoint{2.457917in}{1.881281in}}%
\pgfpathcurveto{\pgfqpoint{2.452093in}{1.887105in}}{\pgfqpoint{2.444193in}{1.890377in}}{\pgfqpoint{2.435957in}{1.890377in}}%
\pgfpathcurveto{\pgfqpoint{2.427721in}{1.890377in}}{\pgfqpoint{2.419821in}{1.887105in}}{\pgfqpoint{2.413997in}{1.881281in}}%
\pgfpathcurveto{\pgfqpoint{2.408173in}{1.875457in}}{\pgfqpoint{2.404901in}{1.867557in}}{\pgfqpoint{2.404901in}{1.859321in}}%
\pgfpathcurveto{\pgfqpoint{2.404901in}{1.851085in}}{\pgfqpoint{2.408173in}{1.843185in}}{\pgfqpoint{2.413997in}{1.837361in}}%
\pgfpathcurveto{\pgfqpoint{2.419821in}{1.831537in}}{\pgfqpoint{2.427721in}{1.828264in}}{\pgfqpoint{2.435957in}{1.828264in}}%
\pgfpathclose%
\pgfusepath{stroke,fill}%
\end{pgfscope}%
\begin{pgfscope}%
\pgfpathrectangle{\pgfqpoint{0.100000in}{0.212622in}}{\pgfqpoint{3.696000in}{3.696000in}}%
\pgfusepath{clip}%
\pgfsetbuttcap%
\pgfsetroundjoin%
\definecolor{currentfill}{rgb}{0.121569,0.466667,0.705882}%
\pgfsetfillcolor{currentfill}%
\pgfsetfillopacity{0.981828}%
\pgfsetlinewidth{1.003750pt}%
\definecolor{currentstroke}{rgb}{0.121569,0.466667,0.705882}%
\pgfsetstrokecolor{currentstroke}%
\pgfsetstrokeopacity{0.981828}%
\pgfsetdash{}{0pt}%
\pgfpathmoveto{\pgfqpoint{2.236776in}{1.889152in}}%
\pgfpathcurveto{\pgfqpoint{2.245012in}{1.889152in}}{\pgfqpoint{2.252912in}{1.892424in}}{\pgfqpoint{2.258736in}{1.898248in}}%
\pgfpathcurveto{\pgfqpoint{2.264560in}{1.904072in}}{\pgfqpoint{2.267833in}{1.911972in}}{\pgfqpoint{2.267833in}{1.920208in}}%
\pgfpathcurveto{\pgfqpoint{2.267833in}{1.928445in}}{\pgfqpoint{2.264560in}{1.936345in}}{\pgfqpoint{2.258736in}{1.942169in}}%
\pgfpathcurveto{\pgfqpoint{2.252912in}{1.947993in}}{\pgfqpoint{2.245012in}{1.951265in}}{\pgfqpoint{2.236776in}{1.951265in}}%
\pgfpathcurveto{\pgfqpoint{2.228540in}{1.951265in}}{\pgfqpoint{2.220640in}{1.947993in}}{\pgfqpoint{2.214816in}{1.942169in}}%
\pgfpathcurveto{\pgfqpoint{2.208992in}{1.936345in}}{\pgfqpoint{2.205720in}{1.928445in}}{\pgfqpoint{2.205720in}{1.920208in}}%
\pgfpathcurveto{\pgfqpoint{2.205720in}{1.911972in}}{\pgfqpoint{2.208992in}{1.904072in}}{\pgfqpoint{2.214816in}{1.898248in}}%
\pgfpathcurveto{\pgfqpoint{2.220640in}{1.892424in}}{\pgfqpoint{2.228540in}{1.889152in}}{\pgfqpoint{2.236776in}{1.889152in}}%
\pgfpathclose%
\pgfusepath{stroke,fill}%
\end{pgfscope}%
\begin{pgfscope}%
\pgfpathrectangle{\pgfqpoint{0.100000in}{0.212622in}}{\pgfqpoint{3.696000in}{3.696000in}}%
\pgfusepath{clip}%
\pgfsetbuttcap%
\pgfsetroundjoin%
\definecolor{currentfill}{rgb}{0.121569,0.466667,0.705882}%
\pgfsetfillcolor{currentfill}%
\pgfsetfillopacity{0.982305}%
\pgfsetlinewidth{1.003750pt}%
\definecolor{currentstroke}{rgb}{0.121569,0.466667,0.705882}%
\pgfsetstrokecolor{currentstroke}%
\pgfsetstrokeopacity{0.982305}%
\pgfsetdash{}{0pt}%
\pgfpathmoveto{\pgfqpoint{2.433216in}{1.827093in}}%
\pgfpathcurveto{\pgfqpoint{2.441453in}{1.827093in}}{\pgfqpoint{2.449353in}{1.830365in}}{\pgfqpoint{2.455177in}{1.836189in}}%
\pgfpathcurveto{\pgfqpoint{2.461001in}{1.842013in}}{\pgfqpoint{2.464273in}{1.849913in}}{\pgfqpoint{2.464273in}{1.858149in}}%
\pgfpathcurveto{\pgfqpoint{2.464273in}{1.866385in}}{\pgfqpoint{2.461001in}{1.874285in}}{\pgfqpoint{2.455177in}{1.880109in}}%
\pgfpathcurveto{\pgfqpoint{2.449353in}{1.885933in}}{\pgfqpoint{2.441453in}{1.889206in}}{\pgfqpoint{2.433216in}{1.889206in}}%
\pgfpathcurveto{\pgfqpoint{2.424980in}{1.889206in}}{\pgfqpoint{2.417080in}{1.885933in}}{\pgfqpoint{2.411256in}{1.880109in}}%
\pgfpathcurveto{\pgfqpoint{2.405432in}{1.874285in}}{\pgfqpoint{2.402160in}{1.866385in}}{\pgfqpoint{2.402160in}{1.858149in}}%
\pgfpathcurveto{\pgfqpoint{2.402160in}{1.849913in}}{\pgfqpoint{2.405432in}{1.842013in}}{\pgfqpoint{2.411256in}{1.836189in}}%
\pgfpathcurveto{\pgfqpoint{2.417080in}{1.830365in}}{\pgfqpoint{2.424980in}{1.827093in}}{\pgfqpoint{2.433216in}{1.827093in}}%
\pgfpathclose%
\pgfusepath{stroke,fill}%
\end{pgfscope}%
\begin{pgfscope}%
\pgfpathrectangle{\pgfqpoint{0.100000in}{0.212622in}}{\pgfqpoint{3.696000in}{3.696000in}}%
\pgfusepath{clip}%
\pgfsetbuttcap%
\pgfsetroundjoin%
\definecolor{currentfill}{rgb}{0.121569,0.466667,0.705882}%
\pgfsetfillcolor{currentfill}%
\pgfsetfillopacity{0.982941}%
\pgfsetlinewidth{1.003750pt}%
\definecolor{currentstroke}{rgb}{0.121569,0.466667,0.705882}%
\pgfsetstrokecolor{currentstroke}%
\pgfsetstrokeopacity{0.982941}%
\pgfsetdash{}{0pt}%
\pgfpathmoveto{\pgfqpoint{2.432073in}{1.826866in}}%
\pgfpathcurveto{\pgfqpoint{2.440309in}{1.826866in}}{\pgfqpoint{2.448209in}{1.830138in}}{\pgfqpoint{2.454033in}{1.835962in}}%
\pgfpathcurveto{\pgfqpoint{2.459857in}{1.841786in}}{\pgfqpoint{2.463130in}{1.849686in}}{\pgfqpoint{2.463130in}{1.857922in}}%
\pgfpathcurveto{\pgfqpoint{2.463130in}{1.866159in}}{\pgfqpoint{2.459857in}{1.874059in}}{\pgfqpoint{2.454033in}{1.879883in}}%
\pgfpathcurveto{\pgfqpoint{2.448209in}{1.885707in}}{\pgfqpoint{2.440309in}{1.888979in}}{\pgfqpoint{2.432073in}{1.888979in}}%
\pgfpathcurveto{\pgfqpoint{2.423837in}{1.888979in}}{\pgfqpoint{2.415937in}{1.885707in}}{\pgfqpoint{2.410113in}{1.879883in}}%
\pgfpathcurveto{\pgfqpoint{2.404289in}{1.874059in}}{\pgfqpoint{2.401017in}{1.866159in}}{\pgfqpoint{2.401017in}{1.857922in}}%
\pgfpathcurveto{\pgfqpoint{2.401017in}{1.849686in}}{\pgfqpoint{2.404289in}{1.841786in}}{\pgfqpoint{2.410113in}{1.835962in}}%
\pgfpathcurveto{\pgfqpoint{2.415937in}{1.830138in}}{\pgfqpoint{2.423837in}{1.826866in}}{\pgfqpoint{2.432073in}{1.826866in}}%
\pgfpathclose%
\pgfusepath{stroke,fill}%
\end{pgfscope}%
\begin{pgfscope}%
\pgfpathrectangle{\pgfqpoint{0.100000in}{0.212622in}}{\pgfqpoint{3.696000in}{3.696000in}}%
\pgfusepath{clip}%
\pgfsetbuttcap%
\pgfsetroundjoin%
\definecolor{currentfill}{rgb}{0.121569,0.466667,0.705882}%
\pgfsetfillcolor{currentfill}%
\pgfsetfillopacity{0.983228}%
\pgfsetlinewidth{1.003750pt}%
\definecolor{currentstroke}{rgb}{0.121569,0.466667,0.705882}%
\pgfsetstrokecolor{currentstroke}%
\pgfsetstrokeopacity{0.983228}%
\pgfsetdash{}{0pt}%
\pgfpathmoveto{\pgfqpoint{2.431457in}{1.826336in}}%
\pgfpathcurveto{\pgfqpoint{2.439694in}{1.826336in}}{\pgfqpoint{2.447594in}{1.829608in}}{\pgfqpoint{2.453418in}{1.835432in}}%
\pgfpathcurveto{\pgfqpoint{2.459241in}{1.841256in}}{\pgfqpoint{2.462514in}{1.849156in}}{\pgfqpoint{2.462514in}{1.857393in}}%
\pgfpathcurveto{\pgfqpoint{2.462514in}{1.865629in}}{\pgfqpoint{2.459241in}{1.873529in}}{\pgfqpoint{2.453418in}{1.879353in}}%
\pgfpathcurveto{\pgfqpoint{2.447594in}{1.885177in}}{\pgfqpoint{2.439694in}{1.888449in}}{\pgfqpoint{2.431457in}{1.888449in}}%
\pgfpathcurveto{\pgfqpoint{2.423221in}{1.888449in}}{\pgfqpoint{2.415321in}{1.885177in}}{\pgfqpoint{2.409497in}{1.879353in}}%
\pgfpathcurveto{\pgfqpoint{2.403673in}{1.873529in}}{\pgfqpoint{2.400401in}{1.865629in}}{\pgfqpoint{2.400401in}{1.857393in}}%
\pgfpathcurveto{\pgfqpoint{2.400401in}{1.849156in}}{\pgfqpoint{2.403673in}{1.841256in}}{\pgfqpoint{2.409497in}{1.835432in}}%
\pgfpathcurveto{\pgfqpoint{2.415321in}{1.829608in}}{\pgfqpoint{2.423221in}{1.826336in}}{\pgfqpoint{2.431457in}{1.826336in}}%
\pgfpathclose%
\pgfusepath{stroke,fill}%
\end{pgfscope}%
\begin{pgfscope}%
\pgfpathrectangle{\pgfqpoint{0.100000in}{0.212622in}}{\pgfqpoint{3.696000in}{3.696000in}}%
\pgfusepath{clip}%
\pgfsetbuttcap%
\pgfsetroundjoin%
\definecolor{currentfill}{rgb}{0.121569,0.466667,0.705882}%
\pgfsetfillcolor{currentfill}%
\pgfsetfillopacity{0.983408}%
\pgfsetlinewidth{1.003750pt}%
\definecolor{currentstroke}{rgb}{0.121569,0.466667,0.705882}%
\pgfsetstrokecolor{currentstroke}%
\pgfsetstrokeopacity{0.983408}%
\pgfsetdash{}{0pt}%
\pgfpathmoveto{\pgfqpoint{2.431081in}{1.826221in}}%
\pgfpathcurveto{\pgfqpoint{2.439317in}{1.826221in}}{\pgfqpoint{2.447217in}{1.829494in}}{\pgfqpoint{2.453041in}{1.835317in}}%
\pgfpathcurveto{\pgfqpoint{2.458865in}{1.841141in}}{\pgfqpoint{2.462138in}{1.849041in}}{\pgfqpoint{2.462138in}{1.857278in}}%
\pgfpathcurveto{\pgfqpoint{2.462138in}{1.865514in}}{\pgfqpoint{2.458865in}{1.873414in}}{\pgfqpoint{2.453041in}{1.879238in}}%
\pgfpathcurveto{\pgfqpoint{2.447217in}{1.885062in}}{\pgfqpoint{2.439317in}{1.888334in}}{\pgfqpoint{2.431081in}{1.888334in}}%
\pgfpathcurveto{\pgfqpoint{2.422845in}{1.888334in}}{\pgfqpoint{2.414945in}{1.885062in}}{\pgfqpoint{2.409121in}{1.879238in}}%
\pgfpathcurveto{\pgfqpoint{2.403297in}{1.873414in}}{\pgfqpoint{2.400025in}{1.865514in}}{\pgfqpoint{2.400025in}{1.857278in}}%
\pgfpathcurveto{\pgfqpoint{2.400025in}{1.849041in}}{\pgfqpoint{2.403297in}{1.841141in}}{\pgfqpoint{2.409121in}{1.835317in}}%
\pgfpathcurveto{\pgfqpoint{2.414945in}{1.829494in}}{\pgfqpoint{2.422845in}{1.826221in}}{\pgfqpoint{2.431081in}{1.826221in}}%
\pgfpathclose%
\pgfusepath{stroke,fill}%
\end{pgfscope}%
\begin{pgfscope}%
\pgfpathrectangle{\pgfqpoint{0.100000in}{0.212622in}}{\pgfqpoint{3.696000in}{3.696000in}}%
\pgfusepath{clip}%
\pgfsetbuttcap%
\pgfsetroundjoin%
\definecolor{currentfill}{rgb}{0.121569,0.466667,0.705882}%
\pgfsetfillcolor{currentfill}%
\pgfsetfillopacity{0.984741}%
\pgfsetlinewidth{1.003750pt}%
\definecolor{currentstroke}{rgb}{0.121569,0.466667,0.705882}%
\pgfsetstrokecolor{currentstroke}%
\pgfsetstrokeopacity{0.984741}%
\pgfsetdash{}{0pt}%
\pgfpathmoveto{\pgfqpoint{2.428622in}{1.824010in}}%
\pgfpathcurveto{\pgfqpoint{2.436858in}{1.824010in}}{\pgfqpoint{2.444759in}{1.827282in}}{\pgfqpoint{2.450582in}{1.833106in}}%
\pgfpathcurveto{\pgfqpoint{2.456406in}{1.838930in}}{\pgfqpoint{2.459679in}{1.846830in}}{\pgfqpoint{2.459679in}{1.855066in}}%
\pgfpathcurveto{\pgfqpoint{2.459679in}{1.863303in}}{\pgfqpoint{2.456406in}{1.871203in}}{\pgfqpoint{2.450582in}{1.877027in}}%
\pgfpathcurveto{\pgfqpoint{2.444759in}{1.882851in}}{\pgfqpoint{2.436858in}{1.886123in}}{\pgfqpoint{2.428622in}{1.886123in}}%
\pgfpathcurveto{\pgfqpoint{2.420386in}{1.886123in}}{\pgfqpoint{2.412486in}{1.882851in}}{\pgfqpoint{2.406662in}{1.877027in}}%
\pgfpathcurveto{\pgfqpoint{2.400838in}{1.871203in}}{\pgfqpoint{2.397566in}{1.863303in}}{\pgfqpoint{2.397566in}{1.855066in}}%
\pgfpathcurveto{\pgfqpoint{2.397566in}{1.846830in}}{\pgfqpoint{2.400838in}{1.838930in}}{\pgfqpoint{2.406662in}{1.833106in}}%
\pgfpathcurveto{\pgfqpoint{2.412486in}{1.827282in}}{\pgfqpoint{2.420386in}{1.824010in}}{\pgfqpoint{2.428622in}{1.824010in}}%
\pgfpathclose%
\pgfusepath{stroke,fill}%
\end{pgfscope}%
\begin{pgfscope}%
\pgfpathrectangle{\pgfqpoint{0.100000in}{0.212622in}}{\pgfqpoint{3.696000in}{3.696000in}}%
\pgfusepath{clip}%
\pgfsetbuttcap%
\pgfsetroundjoin%
\definecolor{currentfill}{rgb}{0.121569,0.466667,0.705882}%
\pgfsetfillcolor{currentfill}%
\pgfsetfillopacity{0.984915}%
\pgfsetlinewidth{1.003750pt}%
\definecolor{currentstroke}{rgb}{0.121569,0.466667,0.705882}%
\pgfsetstrokecolor{currentstroke}%
\pgfsetstrokeopacity{0.984915}%
\pgfsetdash{}{0pt}%
\pgfpathmoveto{\pgfqpoint{2.262824in}{1.874191in}}%
\pgfpathcurveto{\pgfqpoint{2.271061in}{1.874191in}}{\pgfqpoint{2.278961in}{1.877463in}}{\pgfqpoint{2.284785in}{1.883287in}}%
\pgfpathcurveto{\pgfqpoint{2.290609in}{1.889111in}}{\pgfqpoint{2.293881in}{1.897011in}}{\pgfqpoint{2.293881in}{1.905248in}}%
\pgfpathcurveto{\pgfqpoint{2.293881in}{1.913484in}}{\pgfqpoint{2.290609in}{1.921384in}}{\pgfqpoint{2.284785in}{1.927208in}}%
\pgfpathcurveto{\pgfqpoint{2.278961in}{1.933032in}}{\pgfqpoint{2.271061in}{1.936304in}}{\pgfqpoint{2.262824in}{1.936304in}}%
\pgfpathcurveto{\pgfqpoint{2.254588in}{1.936304in}}{\pgfqpoint{2.246688in}{1.933032in}}{\pgfqpoint{2.240864in}{1.927208in}}%
\pgfpathcurveto{\pgfqpoint{2.235040in}{1.921384in}}{\pgfqpoint{2.231768in}{1.913484in}}{\pgfqpoint{2.231768in}{1.905248in}}%
\pgfpathcurveto{\pgfqpoint{2.231768in}{1.897011in}}{\pgfqpoint{2.235040in}{1.889111in}}{\pgfqpoint{2.240864in}{1.883287in}}%
\pgfpathcurveto{\pgfqpoint{2.246688in}{1.877463in}}{\pgfqpoint{2.254588in}{1.874191in}}{\pgfqpoint{2.262824in}{1.874191in}}%
\pgfpathclose%
\pgfusepath{stroke,fill}%
\end{pgfscope}%
\begin{pgfscope}%
\pgfpathrectangle{\pgfqpoint{0.100000in}{0.212622in}}{\pgfqpoint{3.696000in}{3.696000in}}%
\pgfusepath{clip}%
\pgfsetbuttcap%
\pgfsetroundjoin%
\definecolor{currentfill}{rgb}{0.121569,0.466667,0.705882}%
\pgfsetfillcolor{currentfill}%
\pgfsetfillopacity{0.985467}%
\pgfsetlinewidth{1.003750pt}%
\definecolor{currentstroke}{rgb}{0.121569,0.466667,0.705882}%
\pgfsetstrokecolor{currentstroke}%
\pgfsetstrokeopacity{0.985467}%
\pgfsetdash{}{0pt}%
\pgfpathmoveto{\pgfqpoint{2.426848in}{1.823222in}}%
\pgfpathcurveto{\pgfqpoint{2.435085in}{1.823222in}}{\pgfqpoint{2.442985in}{1.826494in}}{\pgfqpoint{2.448809in}{1.832318in}}%
\pgfpathcurveto{\pgfqpoint{2.454633in}{1.838142in}}{\pgfqpoint{2.457905in}{1.846042in}}{\pgfqpoint{2.457905in}{1.854279in}}%
\pgfpathcurveto{\pgfqpoint{2.457905in}{1.862515in}}{\pgfqpoint{2.454633in}{1.870415in}}{\pgfqpoint{2.448809in}{1.876239in}}%
\pgfpathcurveto{\pgfqpoint{2.442985in}{1.882063in}}{\pgfqpoint{2.435085in}{1.885335in}}{\pgfqpoint{2.426848in}{1.885335in}}%
\pgfpathcurveto{\pgfqpoint{2.418612in}{1.885335in}}{\pgfqpoint{2.410712in}{1.882063in}}{\pgfqpoint{2.404888in}{1.876239in}}%
\pgfpathcurveto{\pgfqpoint{2.399064in}{1.870415in}}{\pgfqpoint{2.395792in}{1.862515in}}{\pgfqpoint{2.395792in}{1.854279in}}%
\pgfpathcurveto{\pgfqpoint{2.395792in}{1.846042in}}{\pgfqpoint{2.399064in}{1.838142in}}{\pgfqpoint{2.404888in}{1.832318in}}%
\pgfpathcurveto{\pgfqpoint{2.410712in}{1.826494in}}{\pgfqpoint{2.418612in}{1.823222in}}{\pgfqpoint{2.426848in}{1.823222in}}%
\pgfpathclose%
\pgfusepath{stroke,fill}%
\end{pgfscope}%
\begin{pgfscope}%
\pgfpathrectangle{\pgfqpoint{0.100000in}{0.212622in}}{\pgfqpoint{3.696000in}{3.696000in}}%
\pgfusepath{clip}%
\pgfsetbuttcap%
\pgfsetroundjoin%
\definecolor{currentfill}{rgb}{0.121569,0.466667,0.705882}%
\pgfsetfillcolor{currentfill}%
\pgfsetfillopacity{0.987051}%
\pgfsetlinewidth{1.003750pt}%
\definecolor{currentstroke}{rgb}{0.121569,0.466667,0.705882}%
\pgfsetstrokecolor{currentstroke}%
\pgfsetstrokeopacity{0.987051}%
\pgfsetdash{}{0pt}%
\pgfpathmoveto{\pgfqpoint{2.423420in}{1.821087in}}%
\pgfpathcurveto{\pgfqpoint{2.431657in}{1.821087in}}{\pgfqpoint{2.439557in}{1.824360in}}{\pgfqpoint{2.445380in}{1.830184in}}%
\pgfpathcurveto{\pgfqpoint{2.451204in}{1.836008in}}{\pgfqpoint{2.454477in}{1.843908in}}{\pgfqpoint{2.454477in}{1.852144in}}%
\pgfpathcurveto{\pgfqpoint{2.454477in}{1.860380in}}{\pgfqpoint{2.451204in}{1.868280in}}{\pgfqpoint{2.445380in}{1.874104in}}%
\pgfpathcurveto{\pgfqpoint{2.439557in}{1.879928in}}{\pgfqpoint{2.431657in}{1.883200in}}{\pgfqpoint{2.423420in}{1.883200in}}%
\pgfpathcurveto{\pgfqpoint{2.415184in}{1.883200in}}{\pgfqpoint{2.407284in}{1.879928in}}{\pgfqpoint{2.401460in}{1.874104in}}%
\pgfpathcurveto{\pgfqpoint{2.395636in}{1.868280in}}{\pgfqpoint{2.392364in}{1.860380in}}{\pgfqpoint{2.392364in}{1.852144in}}%
\pgfpathcurveto{\pgfqpoint{2.392364in}{1.843908in}}{\pgfqpoint{2.395636in}{1.836008in}}{\pgfqpoint{2.401460in}{1.830184in}}%
\pgfpathcurveto{\pgfqpoint{2.407284in}{1.824360in}}{\pgfqpoint{2.415184in}{1.821087in}}{\pgfqpoint{2.423420in}{1.821087in}}%
\pgfpathclose%
\pgfusepath{stroke,fill}%
\end{pgfscope}%
\begin{pgfscope}%
\pgfpathrectangle{\pgfqpoint{0.100000in}{0.212622in}}{\pgfqpoint{3.696000in}{3.696000in}}%
\pgfusepath{clip}%
\pgfsetbuttcap%
\pgfsetroundjoin%
\definecolor{currentfill}{rgb}{0.121569,0.466667,0.705882}%
\pgfsetfillcolor{currentfill}%
\pgfsetfillopacity{0.987795}%
\pgfsetlinewidth{1.003750pt}%
\definecolor{currentstroke}{rgb}{0.121569,0.466667,0.705882}%
\pgfsetstrokecolor{currentstroke}%
\pgfsetstrokeopacity{0.987795}%
\pgfsetdash{}{0pt}%
\pgfpathmoveto{\pgfqpoint{2.420954in}{1.819965in}}%
\pgfpathcurveto{\pgfqpoint{2.429191in}{1.819965in}}{\pgfqpoint{2.437091in}{1.823237in}}{\pgfqpoint{2.442915in}{1.829061in}}%
\pgfpathcurveto{\pgfqpoint{2.448739in}{1.834885in}}{\pgfqpoint{2.452011in}{1.842785in}}{\pgfqpoint{2.452011in}{1.851021in}}%
\pgfpathcurveto{\pgfqpoint{2.452011in}{1.859258in}}{\pgfqpoint{2.448739in}{1.867158in}}{\pgfqpoint{2.442915in}{1.872982in}}%
\pgfpathcurveto{\pgfqpoint{2.437091in}{1.878806in}}{\pgfqpoint{2.429191in}{1.882078in}}{\pgfqpoint{2.420954in}{1.882078in}}%
\pgfpathcurveto{\pgfqpoint{2.412718in}{1.882078in}}{\pgfqpoint{2.404818in}{1.878806in}}{\pgfqpoint{2.398994in}{1.872982in}}%
\pgfpathcurveto{\pgfqpoint{2.393170in}{1.867158in}}{\pgfqpoint{2.389898in}{1.859258in}}{\pgfqpoint{2.389898in}{1.851021in}}%
\pgfpathcurveto{\pgfqpoint{2.389898in}{1.842785in}}{\pgfqpoint{2.393170in}{1.834885in}}{\pgfqpoint{2.398994in}{1.829061in}}%
\pgfpathcurveto{\pgfqpoint{2.404818in}{1.823237in}}{\pgfqpoint{2.412718in}{1.819965in}}{\pgfqpoint{2.420954in}{1.819965in}}%
\pgfpathclose%
\pgfusepath{stroke,fill}%
\end{pgfscope}%
\begin{pgfscope}%
\pgfpathrectangle{\pgfqpoint{0.100000in}{0.212622in}}{\pgfqpoint{3.696000in}{3.696000in}}%
\pgfusepath{clip}%
\pgfsetbuttcap%
\pgfsetroundjoin%
\definecolor{currentfill}{rgb}{0.121569,0.466667,0.705882}%
\pgfsetfillcolor{currentfill}%
\pgfsetfillopacity{0.988292}%
\pgfsetlinewidth{1.003750pt}%
\definecolor{currentstroke}{rgb}{0.121569,0.466667,0.705882}%
\pgfsetstrokecolor{currentstroke}%
\pgfsetstrokeopacity{0.988292}%
\pgfsetdash{}{0pt}%
\pgfpathmoveto{\pgfqpoint{2.283786in}{1.864520in}}%
\pgfpathcurveto{\pgfqpoint{2.292022in}{1.864520in}}{\pgfqpoint{2.299922in}{1.867792in}}{\pgfqpoint{2.305746in}{1.873616in}}%
\pgfpathcurveto{\pgfqpoint{2.311570in}{1.879440in}}{\pgfqpoint{2.314842in}{1.887340in}}{\pgfqpoint{2.314842in}{1.895577in}}%
\pgfpathcurveto{\pgfqpoint{2.314842in}{1.903813in}}{\pgfqpoint{2.311570in}{1.911713in}}{\pgfqpoint{2.305746in}{1.917537in}}%
\pgfpathcurveto{\pgfqpoint{2.299922in}{1.923361in}}{\pgfqpoint{2.292022in}{1.926633in}}{\pgfqpoint{2.283786in}{1.926633in}}%
\pgfpathcurveto{\pgfqpoint{2.275550in}{1.926633in}}{\pgfqpoint{2.267650in}{1.923361in}}{\pgfqpoint{2.261826in}{1.917537in}}%
\pgfpathcurveto{\pgfqpoint{2.256002in}{1.911713in}}{\pgfqpoint{2.252729in}{1.903813in}}{\pgfqpoint{2.252729in}{1.895577in}}%
\pgfpathcurveto{\pgfqpoint{2.252729in}{1.887340in}}{\pgfqpoint{2.256002in}{1.879440in}}{\pgfqpoint{2.261826in}{1.873616in}}%
\pgfpathcurveto{\pgfqpoint{2.267650in}{1.867792in}}{\pgfqpoint{2.275550in}{1.864520in}}{\pgfqpoint{2.283786in}{1.864520in}}%
\pgfpathclose%
\pgfusepath{stroke,fill}%
\end{pgfscope}%
\begin{pgfscope}%
\pgfpathrectangle{\pgfqpoint{0.100000in}{0.212622in}}{\pgfqpoint{3.696000in}{3.696000in}}%
\pgfusepath{clip}%
\pgfsetbuttcap%
\pgfsetroundjoin%
\definecolor{currentfill}{rgb}{0.121569,0.466667,0.705882}%
\pgfsetfillcolor{currentfill}%
\pgfsetfillopacity{0.988352}%
\pgfsetlinewidth{1.003750pt}%
\definecolor{currentstroke}{rgb}{0.121569,0.466667,0.705882}%
\pgfsetstrokecolor{currentstroke}%
\pgfsetstrokeopacity{0.988352}%
\pgfsetdash{}{0pt}%
\pgfpathmoveto{\pgfqpoint{2.419889in}{1.819823in}}%
\pgfpathcurveto{\pgfqpoint{2.428126in}{1.819823in}}{\pgfqpoint{2.436026in}{1.823095in}}{\pgfqpoint{2.441850in}{1.828919in}}%
\pgfpathcurveto{\pgfqpoint{2.447674in}{1.834743in}}{\pgfqpoint{2.450946in}{1.842643in}}{\pgfqpoint{2.450946in}{1.850879in}}%
\pgfpathcurveto{\pgfqpoint{2.450946in}{1.859115in}}{\pgfqpoint{2.447674in}{1.867015in}}{\pgfqpoint{2.441850in}{1.872839in}}%
\pgfpathcurveto{\pgfqpoint{2.436026in}{1.878663in}}{\pgfqpoint{2.428126in}{1.881936in}}{\pgfqpoint{2.419889in}{1.881936in}}%
\pgfpathcurveto{\pgfqpoint{2.411653in}{1.881936in}}{\pgfqpoint{2.403753in}{1.878663in}}{\pgfqpoint{2.397929in}{1.872839in}}%
\pgfpathcurveto{\pgfqpoint{2.392105in}{1.867015in}}{\pgfqpoint{2.388833in}{1.859115in}}{\pgfqpoint{2.388833in}{1.850879in}}%
\pgfpathcurveto{\pgfqpoint{2.388833in}{1.842643in}}{\pgfqpoint{2.392105in}{1.834743in}}{\pgfqpoint{2.397929in}{1.828919in}}%
\pgfpathcurveto{\pgfqpoint{2.403753in}{1.823095in}}{\pgfqpoint{2.411653in}{1.819823in}}{\pgfqpoint{2.419889in}{1.819823in}}%
\pgfpathclose%
\pgfusepath{stroke,fill}%
\end{pgfscope}%
\begin{pgfscope}%
\pgfpathrectangle{\pgfqpoint{0.100000in}{0.212622in}}{\pgfqpoint{3.696000in}{3.696000in}}%
\pgfusepath{clip}%
\pgfsetbuttcap%
\pgfsetroundjoin%
\definecolor{currentfill}{rgb}{0.121569,0.466667,0.705882}%
\pgfsetfillcolor{currentfill}%
\pgfsetfillopacity{0.988609}%
\pgfsetlinewidth{1.003750pt}%
\definecolor{currentstroke}{rgb}{0.121569,0.466667,0.705882}%
\pgfsetstrokecolor{currentstroke}%
\pgfsetstrokeopacity{0.988609}%
\pgfsetdash{}{0pt}%
\pgfpathmoveto{\pgfqpoint{2.419255in}{1.819490in}}%
\pgfpathcurveto{\pgfqpoint{2.427491in}{1.819490in}}{\pgfqpoint{2.435391in}{1.822763in}}{\pgfqpoint{2.441215in}{1.828586in}}%
\pgfpathcurveto{\pgfqpoint{2.447039in}{1.834410in}}{\pgfqpoint{2.450311in}{1.842310in}}{\pgfqpoint{2.450311in}{1.850547in}}%
\pgfpathcurveto{\pgfqpoint{2.450311in}{1.858783in}}{\pgfqpoint{2.447039in}{1.866683in}}{\pgfqpoint{2.441215in}{1.872507in}}%
\pgfpathcurveto{\pgfqpoint{2.435391in}{1.878331in}}{\pgfqpoint{2.427491in}{1.881603in}}{\pgfqpoint{2.419255in}{1.881603in}}%
\pgfpathcurveto{\pgfqpoint{2.411018in}{1.881603in}}{\pgfqpoint{2.403118in}{1.878331in}}{\pgfqpoint{2.397295in}{1.872507in}}%
\pgfpathcurveto{\pgfqpoint{2.391471in}{1.866683in}}{\pgfqpoint{2.388198in}{1.858783in}}{\pgfqpoint{2.388198in}{1.850547in}}%
\pgfpathcurveto{\pgfqpoint{2.388198in}{1.842310in}}{\pgfqpoint{2.391471in}{1.834410in}}{\pgfqpoint{2.397295in}{1.828586in}}%
\pgfpathcurveto{\pgfqpoint{2.403118in}{1.822763in}}{\pgfqpoint{2.411018in}{1.819490in}}{\pgfqpoint{2.419255in}{1.819490in}}%
\pgfpathclose%
\pgfusepath{stroke,fill}%
\end{pgfscope}%
\begin{pgfscope}%
\pgfpathrectangle{\pgfqpoint{0.100000in}{0.212622in}}{\pgfqpoint{3.696000in}{3.696000in}}%
\pgfusepath{clip}%
\pgfsetbuttcap%
\pgfsetroundjoin%
\definecolor{currentfill}{rgb}{0.121569,0.466667,0.705882}%
\pgfsetfillcolor{currentfill}%
\pgfsetfillopacity{0.988769}%
\pgfsetlinewidth{1.003750pt}%
\definecolor{currentstroke}{rgb}{0.121569,0.466667,0.705882}%
\pgfsetstrokecolor{currentstroke}%
\pgfsetstrokeopacity{0.988769}%
\pgfsetdash{}{0pt}%
\pgfpathmoveto{\pgfqpoint{2.418912in}{1.819420in}}%
\pgfpathcurveto{\pgfqpoint{2.427148in}{1.819420in}}{\pgfqpoint{2.435048in}{1.822692in}}{\pgfqpoint{2.440872in}{1.828516in}}%
\pgfpathcurveto{\pgfqpoint{2.446696in}{1.834340in}}{\pgfqpoint{2.449969in}{1.842240in}}{\pgfqpoint{2.449969in}{1.850476in}}%
\pgfpathcurveto{\pgfqpoint{2.449969in}{1.858712in}}{\pgfqpoint{2.446696in}{1.866613in}}{\pgfqpoint{2.440872in}{1.872436in}}%
\pgfpathcurveto{\pgfqpoint{2.435048in}{1.878260in}}{\pgfqpoint{2.427148in}{1.881533in}}{\pgfqpoint{2.418912in}{1.881533in}}%
\pgfpathcurveto{\pgfqpoint{2.410676in}{1.881533in}}{\pgfqpoint{2.402776in}{1.878260in}}{\pgfqpoint{2.396952in}{1.872436in}}%
\pgfpathcurveto{\pgfqpoint{2.391128in}{1.866613in}}{\pgfqpoint{2.387856in}{1.858712in}}{\pgfqpoint{2.387856in}{1.850476in}}%
\pgfpathcurveto{\pgfqpoint{2.387856in}{1.842240in}}{\pgfqpoint{2.391128in}{1.834340in}}{\pgfqpoint{2.396952in}{1.828516in}}%
\pgfpathcurveto{\pgfqpoint{2.402776in}{1.822692in}}{\pgfqpoint{2.410676in}{1.819420in}}{\pgfqpoint{2.418912in}{1.819420in}}%
\pgfpathclose%
\pgfusepath{stroke,fill}%
\end{pgfscope}%
\begin{pgfscope}%
\pgfpathrectangle{\pgfqpoint{0.100000in}{0.212622in}}{\pgfqpoint{3.696000in}{3.696000in}}%
\pgfusepath{clip}%
\pgfsetbuttcap%
\pgfsetroundjoin%
\definecolor{currentfill}{rgb}{0.121569,0.466667,0.705882}%
\pgfsetfillcolor{currentfill}%
\pgfsetfillopacity{0.988842}%
\pgfsetlinewidth{1.003750pt}%
\definecolor{currentstroke}{rgb}{0.121569,0.466667,0.705882}%
\pgfsetstrokecolor{currentstroke}%
\pgfsetstrokeopacity{0.988842}%
\pgfsetdash{}{0pt}%
\pgfpathmoveto{\pgfqpoint{2.418749in}{1.819256in}}%
\pgfpathcurveto{\pgfqpoint{2.426985in}{1.819256in}}{\pgfqpoint{2.434885in}{1.822529in}}{\pgfqpoint{2.440709in}{1.828353in}}%
\pgfpathcurveto{\pgfqpoint{2.446533in}{1.834176in}}{\pgfqpoint{2.449805in}{1.842077in}}{\pgfqpoint{2.449805in}{1.850313in}}%
\pgfpathcurveto{\pgfqpoint{2.449805in}{1.858549in}}{\pgfqpoint{2.446533in}{1.866449in}}{\pgfqpoint{2.440709in}{1.872273in}}%
\pgfpathcurveto{\pgfqpoint{2.434885in}{1.878097in}}{\pgfqpoint{2.426985in}{1.881369in}}{\pgfqpoint{2.418749in}{1.881369in}}%
\pgfpathcurveto{\pgfqpoint{2.410512in}{1.881369in}}{\pgfqpoint{2.402612in}{1.878097in}}{\pgfqpoint{2.396788in}{1.872273in}}%
\pgfpathcurveto{\pgfqpoint{2.390964in}{1.866449in}}{\pgfqpoint{2.387692in}{1.858549in}}{\pgfqpoint{2.387692in}{1.850313in}}%
\pgfpathcurveto{\pgfqpoint{2.387692in}{1.842077in}}{\pgfqpoint{2.390964in}{1.834176in}}{\pgfqpoint{2.396788in}{1.828353in}}%
\pgfpathcurveto{\pgfqpoint{2.402612in}{1.822529in}}{\pgfqpoint{2.410512in}{1.819256in}}{\pgfqpoint{2.418749in}{1.819256in}}%
\pgfpathclose%
\pgfusepath{stroke,fill}%
\end{pgfscope}%
\begin{pgfscope}%
\pgfpathrectangle{\pgfqpoint{0.100000in}{0.212622in}}{\pgfqpoint{3.696000in}{3.696000in}}%
\pgfusepath{clip}%
\pgfsetbuttcap%
\pgfsetroundjoin%
\definecolor{currentfill}{rgb}{0.121569,0.466667,0.705882}%
\pgfsetfillcolor{currentfill}%
\pgfsetfillopacity{0.988885}%
\pgfsetlinewidth{1.003750pt}%
\definecolor{currentstroke}{rgb}{0.121569,0.466667,0.705882}%
\pgfsetstrokecolor{currentstroke}%
\pgfsetstrokeopacity{0.988885}%
\pgfsetdash{}{0pt}%
\pgfpathmoveto{\pgfqpoint{2.418641in}{1.819210in}}%
\pgfpathcurveto{\pgfqpoint{2.426877in}{1.819210in}}{\pgfqpoint{2.434777in}{1.822482in}}{\pgfqpoint{2.440601in}{1.828306in}}%
\pgfpathcurveto{\pgfqpoint{2.446425in}{1.834130in}}{\pgfqpoint{2.449697in}{1.842030in}}{\pgfqpoint{2.449697in}{1.850267in}}%
\pgfpathcurveto{\pgfqpoint{2.449697in}{1.858503in}}{\pgfqpoint{2.446425in}{1.866403in}}{\pgfqpoint{2.440601in}{1.872227in}}%
\pgfpathcurveto{\pgfqpoint{2.434777in}{1.878051in}}{\pgfqpoint{2.426877in}{1.881323in}}{\pgfqpoint{2.418641in}{1.881323in}}%
\pgfpathcurveto{\pgfqpoint{2.410405in}{1.881323in}}{\pgfqpoint{2.402505in}{1.878051in}}{\pgfqpoint{2.396681in}{1.872227in}}%
\pgfpathcurveto{\pgfqpoint{2.390857in}{1.866403in}}{\pgfqpoint{2.387584in}{1.858503in}}{\pgfqpoint{2.387584in}{1.850267in}}%
\pgfpathcurveto{\pgfqpoint{2.387584in}{1.842030in}}{\pgfqpoint{2.390857in}{1.834130in}}{\pgfqpoint{2.396681in}{1.828306in}}%
\pgfpathcurveto{\pgfqpoint{2.402505in}{1.822482in}}{\pgfqpoint{2.410405in}{1.819210in}}{\pgfqpoint{2.418641in}{1.819210in}}%
\pgfpathclose%
\pgfusepath{stroke,fill}%
\end{pgfscope}%
\begin{pgfscope}%
\pgfpathrectangle{\pgfqpoint{0.100000in}{0.212622in}}{\pgfqpoint{3.696000in}{3.696000in}}%
\pgfusepath{clip}%
\pgfsetbuttcap%
\pgfsetroundjoin%
\definecolor{currentfill}{rgb}{0.121569,0.466667,0.705882}%
\pgfsetfillcolor{currentfill}%
\pgfsetfillopacity{0.989510}%
\pgfsetlinewidth{1.003750pt}%
\definecolor{currentstroke}{rgb}{0.121569,0.466667,0.705882}%
\pgfsetstrokecolor{currentstroke}%
\pgfsetstrokeopacity{0.989510}%
\pgfsetdash{}{0pt}%
\pgfpathmoveto{\pgfqpoint{2.417329in}{1.818285in}}%
\pgfpathcurveto{\pgfqpoint{2.425565in}{1.818285in}}{\pgfqpoint{2.433465in}{1.821557in}}{\pgfqpoint{2.439289in}{1.827381in}}%
\pgfpathcurveto{\pgfqpoint{2.445113in}{1.833205in}}{\pgfqpoint{2.448386in}{1.841105in}}{\pgfqpoint{2.448386in}{1.849342in}}%
\pgfpathcurveto{\pgfqpoint{2.448386in}{1.857578in}}{\pgfqpoint{2.445113in}{1.865478in}}{\pgfqpoint{2.439289in}{1.871302in}}%
\pgfpathcurveto{\pgfqpoint{2.433465in}{1.877126in}}{\pgfqpoint{2.425565in}{1.880398in}}{\pgfqpoint{2.417329in}{1.880398in}}%
\pgfpathcurveto{\pgfqpoint{2.409093in}{1.880398in}}{\pgfqpoint{2.401193in}{1.877126in}}{\pgfqpoint{2.395369in}{1.871302in}}%
\pgfpathcurveto{\pgfqpoint{2.389545in}{1.865478in}}{\pgfqpoint{2.386273in}{1.857578in}}{\pgfqpoint{2.386273in}{1.849342in}}%
\pgfpathcurveto{\pgfqpoint{2.386273in}{1.841105in}}{\pgfqpoint{2.389545in}{1.833205in}}{\pgfqpoint{2.395369in}{1.827381in}}%
\pgfpathcurveto{\pgfqpoint{2.401193in}{1.821557in}}{\pgfqpoint{2.409093in}{1.818285in}}{\pgfqpoint{2.417329in}{1.818285in}}%
\pgfpathclose%
\pgfusepath{stroke,fill}%
\end{pgfscope}%
\begin{pgfscope}%
\pgfpathrectangle{\pgfqpoint{0.100000in}{0.212622in}}{\pgfqpoint{3.696000in}{3.696000in}}%
\pgfusepath{clip}%
\pgfsetbuttcap%
\pgfsetroundjoin%
\definecolor{currentfill}{rgb}{0.121569,0.466667,0.705882}%
\pgfsetfillcolor{currentfill}%
\pgfsetfillopacity{0.989809}%
\pgfsetlinewidth{1.003750pt}%
\definecolor{currentstroke}{rgb}{0.121569,0.466667,0.705882}%
\pgfsetstrokecolor{currentstroke}%
\pgfsetstrokeopacity{0.989809}%
\pgfsetdash{}{0pt}%
\pgfpathmoveto{\pgfqpoint{2.416376in}{1.817841in}}%
\pgfpathcurveto{\pgfqpoint{2.424613in}{1.817841in}}{\pgfqpoint{2.432513in}{1.821113in}}{\pgfqpoint{2.438337in}{1.826937in}}%
\pgfpathcurveto{\pgfqpoint{2.444161in}{1.832761in}}{\pgfqpoint{2.447433in}{1.840661in}}{\pgfqpoint{2.447433in}{1.848897in}}%
\pgfpathcurveto{\pgfqpoint{2.447433in}{1.857134in}}{\pgfqpoint{2.444161in}{1.865034in}}{\pgfqpoint{2.438337in}{1.870858in}}%
\pgfpathcurveto{\pgfqpoint{2.432513in}{1.876682in}}{\pgfqpoint{2.424613in}{1.879954in}}{\pgfqpoint{2.416376in}{1.879954in}}%
\pgfpathcurveto{\pgfqpoint{2.408140in}{1.879954in}}{\pgfqpoint{2.400240in}{1.876682in}}{\pgfqpoint{2.394416in}{1.870858in}}%
\pgfpathcurveto{\pgfqpoint{2.388592in}{1.865034in}}{\pgfqpoint{2.385320in}{1.857134in}}{\pgfqpoint{2.385320in}{1.848897in}}%
\pgfpathcurveto{\pgfqpoint{2.385320in}{1.840661in}}{\pgfqpoint{2.388592in}{1.832761in}}{\pgfqpoint{2.394416in}{1.826937in}}%
\pgfpathcurveto{\pgfqpoint{2.400240in}{1.821113in}}{\pgfqpoint{2.408140in}{1.817841in}}{\pgfqpoint{2.416376in}{1.817841in}}%
\pgfpathclose%
\pgfusepath{stroke,fill}%
\end{pgfscope}%
\begin{pgfscope}%
\pgfpathrectangle{\pgfqpoint{0.100000in}{0.212622in}}{\pgfqpoint{3.696000in}{3.696000in}}%
\pgfusepath{clip}%
\pgfsetbuttcap%
\pgfsetroundjoin%
\definecolor{currentfill}{rgb}{0.121569,0.466667,0.705882}%
\pgfsetfillcolor{currentfill}%
\pgfsetfillopacity{0.989921}%
\pgfsetlinewidth{1.003750pt}%
\definecolor{currentstroke}{rgb}{0.121569,0.466667,0.705882}%
\pgfsetstrokecolor{currentstroke}%
\pgfsetstrokeopacity{0.989921}%
\pgfsetdash{}{0pt}%
\pgfpathmoveto{\pgfqpoint{2.296007in}{1.858391in}}%
\pgfpathcurveto{\pgfqpoint{2.304243in}{1.858391in}}{\pgfqpoint{2.312143in}{1.861664in}}{\pgfqpoint{2.317967in}{1.867488in}}%
\pgfpathcurveto{\pgfqpoint{2.323791in}{1.873312in}}{\pgfqpoint{2.327063in}{1.881212in}}{\pgfqpoint{2.327063in}{1.889448in}}%
\pgfpathcurveto{\pgfqpoint{2.327063in}{1.897684in}}{\pgfqpoint{2.323791in}{1.905584in}}{\pgfqpoint{2.317967in}{1.911408in}}%
\pgfpathcurveto{\pgfqpoint{2.312143in}{1.917232in}}{\pgfqpoint{2.304243in}{1.920504in}}{\pgfqpoint{2.296007in}{1.920504in}}%
\pgfpathcurveto{\pgfqpoint{2.287771in}{1.920504in}}{\pgfqpoint{2.279871in}{1.917232in}}{\pgfqpoint{2.274047in}{1.911408in}}%
\pgfpathcurveto{\pgfqpoint{2.268223in}{1.905584in}}{\pgfqpoint{2.264950in}{1.897684in}}{\pgfqpoint{2.264950in}{1.889448in}}%
\pgfpathcurveto{\pgfqpoint{2.264950in}{1.881212in}}{\pgfqpoint{2.268223in}{1.873312in}}{\pgfqpoint{2.274047in}{1.867488in}}%
\pgfpathcurveto{\pgfqpoint{2.279871in}{1.861664in}}{\pgfqpoint{2.287771in}{1.858391in}}{\pgfqpoint{2.296007in}{1.858391in}}%
\pgfpathclose%
\pgfusepath{stroke,fill}%
\end{pgfscope}%
\begin{pgfscope}%
\pgfpathrectangle{\pgfqpoint{0.100000in}{0.212622in}}{\pgfqpoint{3.696000in}{3.696000in}}%
\pgfusepath{clip}%
\pgfsetbuttcap%
\pgfsetroundjoin%
\definecolor{currentfill}{rgb}{0.121569,0.466667,0.705882}%
\pgfsetfillcolor{currentfill}%
\pgfsetfillopacity{0.990015}%
\pgfsetlinewidth{1.003750pt}%
\definecolor{currentstroke}{rgb}{0.121569,0.466667,0.705882}%
\pgfsetstrokecolor{currentstroke}%
\pgfsetstrokeopacity{0.990015}%
\pgfsetdash{}{0pt}%
\pgfpathmoveto{\pgfqpoint{2.416011in}{1.817636in}}%
\pgfpathcurveto{\pgfqpoint{2.424248in}{1.817636in}}{\pgfqpoint{2.432148in}{1.820908in}}{\pgfqpoint{2.437972in}{1.826732in}}%
\pgfpathcurveto{\pgfqpoint{2.443796in}{1.832556in}}{\pgfqpoint{2.447068in}{1.840456in}}{\pgfqpoint{2.447068in}{1.848692in}}%
\pgfpathcurveto{\pgfqpoint{2.447068in}{1.856929in}}{\pgfqpoint{2.443796in}{1.864829in}}{\pgfqpoint{2.437972in}{1.870653in}}%
\pgfpathcurveto{\pgfqpoint{2.432148in}{1.876477in}}{\pgfqpoint{2.424248in}{1.879749in}}{\pgfqpoint{2.416011in}{1.879749in}}%
\pgfpathcurveto{\pgfqpoint{2.407775in}{1.879749in}}{\pgfqpoint{2.399875in}{1.876477in}}{\pgfqpoint{2.394051in}{1.870653in}}%
\pgfpathcurveto{\pgfqpoint{2.388227in}{1.864829in}}{\pgfqpoint{2.384955in}{1.856929in}}{\pgfqpoint{2.384955in}{1.848692in}}%
\pgfpathcurveto{\pgfqpoint{2.384955in}{1.840456in}}{\pgfqpoint{2.388227in}{1.832556in}}{\pgfqpoint{2.394051in}{1.826732in}}%
\pgfpathcurveto{\pgfqpoint{2.399875in}{1.820908in}}{\pgfqpoint{2.407775in}{1.817636in}}{\pgfqpoint{2.416011in}{1.817636in}}%
\pgfpathclose%
\pgfusepath{stroke,fill}%
\end{pgfscope}%
\begin{pgfscope}%
\pgfpathrectangle{\pgfqpoint{0.100000in}{0.212622in}}{\pgfqpoint{3.696000in}{3.696000in}}%
\pgfusepath{clip}%
\pgfsetbuttcap%
\pgfsetroundjoin%
\definecolor{currentfill}{rgb}{0.121569,0.466667,0.705882}%
\pgfsetfillcolor{currentfill}%
\pgfsetfillopacity{0.990108}%
\pgfsetlinewidth{1.003750pt}%
\definecolor{currentstroke}{rgb}{0.121569,0.466667,0.705882}%
\pgfsetstrokecolor{currentstroke}%
\pgfsetstrokeopacity{0.990108}%
\pgfsetdash{}{0pt}%
\pgfpathmoveto{\pgfqpoint{2.415729in}{1.817510in}}%
\pgfpathcurveto{\pgfqpoint{2.423965in}{1.817510in}}{\pgfqpoint{2.431865in}{1.820782in}}{\pgfqpoint{2.437689in}{1.826606in}}%
\pgfpathcurveto{\pgfqpoint{2.443513in}{1.832430in}}{\pgfqpoint{2.446786in}{1.840330in}}{\pgfqpoint{2.446786in}{1.848566in}}%
\pgfpathcurveto{\pgfqpoint{2.446786in}{1.856803in}}{\pgfqpoint{2.443513in}{1.864703in}}{\pgfqpoint{2.437689in}{1.870526in}}%
\pgfpathcurveto{\pgfqpoint{2.431865in}{1.876350in}}{\pgfqpoint{2.423965in}{1.879623in}}{\pgfqpoint{2.415729in}{1.879623in}}%
\pgfpathcurveto{\pgfqpoint{2.407493in}{1.879623in}}{\pgfqpoint{2.399593in}{1.876350in}}{\pgfqpoint{2.393769in}{1.870526in}}%
\pgfpathcurveto{\pgfqpoint{2.387945in}{1.864703in}}{\pgfqpoint{2.384673in}{1.856803in}}{\pgfqpoint{2.384673in}{1.848566in}}%
\pgfpathcurveto{\pgfqpoint{2.384673in}{1.840330in}}{\pgfqpoint{2.387945in}{1.832430in}}{\pgfqpoint{2.393769in}{1.826606in}}%
\pgfpathcurveto{\pgfqpoint{2.399593in}{1.820782in}}{\pgfqpoint{2.407493in}{1.817510in}}{\pgfqpoint{2.415729in}{1.817510in}}%
\pgfpathclose%
\pgfusepath{stroke,fill}%
\end{pgfscope}%
\begin{pgfscope}%
\pgfpathrectangle{\pgfqpoint{0.100000in}{0.212622in}}{\pgfqpoint{3.696000in}{3.696000in}}%
\pgfusepath{clip}%
\pgfsetbuttcap%
\pgfsetroundjoin%
\definecolor{currentfill}{rgb}{0.121569,0.466667,0.705882}%
\pgfsetfillcolor{currentfill}%
\pgfsetfillopacity{0.990175}%
\pgfsetlinewidth{1.003750pt}%
\definecolor{currentstroke}{rgb}{0.121569,0.466667,0.705882}%
\pgfsetstrokecolor{currentstroke}%
\pgfsetstrokeopacity{0.990175}%
\pgfsetdash{}{0pt}%
\pgfpathmoveto{\pgfqpoint{2.415609in}{1.817486in}}%
\pgfpathcurveto{\pgfqpoint{2.423845in}{1.817486in}}{\pgfqpoint{2.431745in}{1.820758in}}{\pgfqpoint{2.437569in}{1.826582in}}%
\pgfpathcurveto{\pgfqpoint{2.443393in}{1.832406in}}{\pgfqpoint{2.446665in}{1.840306in}}{\pgfqpoint{2.446665in}{1.848542in}}%
\pgfpathcurveto{\pgfqpoint{2.446665in}{1.856779in}}{\pgfqpoint{2.443393in}{1.864679in}}{\pgfqpoint{2.437569in}{1.870503in}}%
\pgfpathcurveto{\pgfqpoint{2.431745in}{1.876326in}}{\pgfqpoint{2.423845in}{1.879599in}}{\pgfqpoint{2.415609in}{1.879599in}}%
\pgfpathcurveto{\pgfqpoint{2.407372in}{1.879599in}}{\pgfqpoint{2.399472in}{1.876326in}}{\pgfqpoint{2.393648in}{1.870503in}}%
\pgfpathcurveto{\pgfqpoint{2.387825in}{1.864679in}}{\pgfqpoint{2.384552in}{1.856779in}}{\pgfqpoint{2.384552in}{1.848542in}}%
\pgfpathcurveto{\pgfqpoint{2.384552in}{1.840306in}}{\pgfqpoint{2.387825in}{1.832406in}}{\pgfqpoint{2.393648in}{1.826582in}}%
\pgfpathcurveto{\pgfqpoint{2.399472in}{1.820758in}}{\pgfqpoint{2.407372in}{1.817486in}}{\pgfqpoint{2.415609in}{1.817486in}}%
\pgfpathclose%
\pgfusepath{stroke,fill}%
\end{pgfscope}%
\begin{pgfscope}%
\pgfpathrectangle{\pgfqpoint{0.100000in}{0.212622in}}{\pgfqpoint{3.696000in}{3.696000in}}%
\pgfusepath{clip}%
\pgfsetbuttcap%
\pgfsetroundjoin%
\definecolor{currentfill}{rgb}{0.121569,0.466667,0.705882}%
\pgfsetfillcolor{currentfill}%
\pgfsetfillopacity{0.990205}%
\pgfsetlinewidth{1.003750pt}%
\definecolor{currentstroke}{rgb}{0.121569,0.466667,0.705882}%
\pgfsetstrokecolor{currentstroke}%
\pgfsetstrokeopacity{0.990205}%
\pgfsetdash{}{0pt}%
\pgfpathmoveto{\pgfqpoint{2.415542in}{1.817427in}}%
\pgfpathcurveto{\pgfqpoint{2.423778in}{1.817427in}}{\pgfqpoint{2.431678in}{1.820700in}}{\pgfqpoint{2.437502in}{1.826524in}}%
\pgfpathcurveto{\pgfqpoint{2.443326in}{1.832348in}}{\pgfqpoint{2.446598in}{1.840248in}}{\pgfqpoint{2.446598in}{1.848484in}}%
\pgfpathcurveto{\pgfqpoint{2.446598in}{1.856720in}}{\pgfqpoint{2.443326in}{1.864620in}}{\pgfqpoint{2.437502in}{1.870444in}}%
\pgfpathcurveto{\pgfqpoint{2.431678in}{1.876268in}}{\pgfqpoint{2.423778in}{1.879540in}}{\pgfqpoint{2.415542in}{1.879540in}}%
\pgfpathcurveto{\pgfqpoint{2.407306in}{1.879540in}}{\pgfqpoint{2.399406in}{1.876268in}}{\pgfqpoint{2.393582in}{1.870444in}}%
\pgfpathcurveto{\pgfqpoint{2.387758in}{1.864620in}}{\pgfqpoint{2.384485in}{1.856720in}}{\pgfqpoint{2.384485in}{1.848484in}}%
\pgfpathcurveto{\pgfqpoint{2.384485in}{1.840248in}}{\pgfqpoint{2.387758in}{1.832348in}}{\pgfqpoint{2.393582in}{1.826524in}}%
\pgfpathcurveto{\pgfqpoint{2.399406in}{1.820700in}}{\pgfqpoint{2.407306in}{1.817427in}}{\pgfqpoint{2.415542in}{1.817427in}}%
\pgfpathclose%
\pgfusepath{stroke,fill}%
\end{pgfscope}%
\begin{pgfscope}%
\pgfpathrectangle{\pgfqpoint{0.100000in}{0.212622in}}{\pgfqpoint{3.696000in}{3.696000in}}%
\pgfusepath{clip}%
\pgfsetbuttcap%
\pgfsetroundjoin%
\definecolor{currentfill}{rgb}{0.121569,0.466667,0.705882}%
\pgfsetfillcolor{currentfill}%
\pgfsetfillopacity{0.990952}%
\pgfsetlinewidth{1.003750pt}%
\definecolor{currentstroke}{rgb}{0.121569,0.466667,0.705882}%
\pgfsetstrokecolor{currentstroke}%
\pgfsetstrokeopacity{0.990952}%
\pgfsetdash{}{0pt}%
\pgfpathmoveto{\pgfqpoint{2.413905in}{1.816737in}}%
\pgfpathcurveto{\pgfqpoint{2.422141in}{1.816737in}}{\pgfqpoint{2.430041in}{1.820009in}}{\pgfqpoint{2.435865in}{1.825833in}}%
\pgfpathcurveto{\pgfqpoint{2.441689in}{1.831657in}}{\pgfqpoint{2.444961in}{1.839557in}}{\pgfqpoint{2.444961in}{1.847793in}}%
\pgfpathcurveto{\pgfqpoint{2.444961in}{1.856029in}}{\pgfqpoint{2.441689in}{1.863929in}}{\pgfqpoint{2.435865in}{1.869753in}}%
\pgfpathcurveto{\pgfqpoint{2.430041in}{1.875577in}}{\pgfqpoint{2.422141in}{1.878850in}}{\pgfqpoint{2.413905in}{1.878850in}}%
\pgfpathcurveto{\pgfqpoint{2.405668in}{1.878850in}}{\pgfqpoint{2.397768in}{1.875577in}}{\pgfqpoint{2.391944in}{1.869753in}}%
\pgfpathcurveto{\pgfqpoint{2.386120in}{1.863929in}}{\pgfqpoint{2.382848in}{1.856029in}}{\pgfqpoint{2.382848in}{1.847793in}}%
\pgfpathcurveto{\pgfqpoint{2.382848in}{1.839557in}}{\pgfqpoint{2.386120in}{1.831657in}}{\pgfqpoint{2.391944in}{1.825833in}}%
\pgfpathcurveto{\pgfqpoint{2.397768in}{1.820009in}}{\pgfqpoint{2.405668in}{1.816737in}}{\pgfqpoint{2.413905in}{1.816737in}}%
\pgfpathclose%
\pgfusepath{stroke,fill}%
\end{pgfscope}%
\begin{pgfscope}%
\pgfpathrectangle{\pgfqpoint{0.100000in}{0.212622in}}{\pgfqpoint{3.696000in}{3.696000in}}%
\pgfusepath{clip}%
\pgfsetbuttcap%
\pgfsetroundjoin%
\definecolor{currentfill}{rgb}{0.121569,0.466667,0.705882}%
\pgfsetfillcolor{currentfill}%
\pgfsetfillopacity{0.991349}%
\pgfsetlinewidth{1.003750pt}%
\definecolor{currentstroke}{rgb}{0.121569,0.466667,0.705882}%
\pgfsetstrokecolor{currentstroke}%
\pgfsetstrokeopacity{0.991349}%
\pgfsetdash{}{0pt}%
\pgfpathmoveto{\pgfqpoint{2.304804in}{1.855306in}}%
\pgfpathcurveto{\pgfqpoint{2.313040in}{1.855306in}}{\pgfqpoint{2.320941in}{1.858578in}}{\pgfqpoint{2.326764in}{1.864402in}}%
\pgfpathcurveto{\pgfqpoint{2.332588in}{1.870226in}}{\pgfqpoint{2.335861in}{1.878126in}}{\pgfqpoint{2.335861in}{1.886362in}}%
\pgfpathcurveto{\pgfqpoint{2.335861in}{1.894598in}}{\pgfqpoint{2.332588in}{1.902498in}}{\pgfqpoint{2.326764in}{1.908322in}}%
\pgfpathcurveto{\pgfqpoint{2.320941in}{1.914146in}}{\pgfqpoint{2.313040in}{1.917419in}}{\pgfqpoint{2.304804in}{1.917419in}}%
\pgfpathcurveto{\pgfqpoint{2.296568in}{1.917419in}}{\pgfqpoint{2.288668in}{1.914146in}}{\pgfqpoint{2.282844in}{1.908322in}}%
\pgfpathcurveto{\pgfqpoint{2.277020in}{1.902498in}}{\pgfqpoint{2.273748in}{1.894598in}}{\pgfqpoint{2.273748in}{1.886362in}}%
\pgfpathcurveto{\pgfqpoint{2.273748in}{1.878126in}}{\pgfqpoint{2.277020in}{1.870226in}}{\pgfqpoint{2.282844in}{1.864402in}}%
\pgfpathcurveto{\pgfqpoint{2.288668in}{1.858578in}}{\pgfqpoint{2.296568in}{1.855306in}}{\pgfqpoint{2.304804in}{1.855306in}}%
\pgfpathclose%
\pgfusepath{stroke,fill}%
\end{pgfscope}%
\begin{pgfscope}%
\pgfpathrectangle{\pgfqpoint{0.100000in}{0.212622in}}{\pgfqpoint{3.696000in}{3.696000in}}%
\pgfusepath{clip}%
\pgfsetbuttcap%
\pgfsetroundjoin%
\definecolor{currentfill}{rgb}{0.121569,0.466667,0.705882}%
\pgfsetfillcolor{currentfill}%
\pgfsetfillopacity{0.991683}%
\pgfsetlinewidth{1.003750pt}%
\definecolor{currentstroke}{rgb}{0.121569,0.466667,0.705882}%
\pgfsetstrokecolor{currentstroke}%
\pgfsetstrokeopacity{0.991683}%
\pgfsetdash{}{0pt}%
\pgfpathmoveto{\pgfqpoint{2.306833in}{1.853905in}}%
\pgfpathcurveto{\pgfqpoint{2.315069in}{1.853905in}}{\pgfqpoint{2.322970in}{1.857177in}}{\pgfqpoint{2.328793in}{1.863001in}}%
\pgfpathcurveto{\pgfqpoint{2.334617in}{1.868825in}}{\pgfqpoint{2.337890in}{1.876725in}}{\pgfqpoint{2.337890in}{1.884961in}}%
\pgfpathcurveto{\pgfqpoint{2.337890in}{1.893198in}}{\pgfqpoint{2.334617in}{1.901098in}}{\pgfqpoint{2.328793in}{1.906922in}}%
\pgfpathcurveto{\pgfqpoint{2.322970in}{1.912746in}}{\pgfqpoint{2.315069in}{1.916018in}}{\pgfqpoint{2.306833in}{1.916018in}}%
\pgfpathcurveto{\pgfqpoint{2.298597in}{1.916018in}}{\pgfqpoint{2.290697in}{1.912746in}}{\pgfqpoint{2.284873in}{1.906922in}}%
\pgfpathcurveto{\pgfqpoint{2.279049in}{1.901098in}}{\pgfqpoint{2.275777in}{1.893198in}}{\pgfqpoint{2.275777in}{1.884961in}}%
\pgfpathcurveto{\pgfqpoint{2.275777in}{1.876725in}}{\pgfqpoint{2.279049in}{1.868825in}}{\pgfqpoint{2.284873in}{1.863001in}}%
\pgfpathcurveto{\pgfqpoint{2.290697in}{1.857177in}}{\pgfqpoint{2.298597in}{1.853905in}}{\pgfqpoint{2.306833in}{1.853905in}}%
\pgfpathclose%
\pgfusepath{stroke,fill}%
\end{pgfscope}%
\begin{pgfscope}%
\pgfpathrectangle{\pgfqpoint{0.100000in}{0.212622in}}{\pgfqpoint{3.696000in}{3.696000in}}%
\pgfusepath{clip}%
\pgfsetbuttcap%
\pgfsetroundjoin%
\definecolor{currentfill}{rgb}{0.121569,0.466667,0.705882}%
\pgfsetfillcolor{currentfill}%
\pgfsetfillopacity{0.992346}%
\pgfsetlinewidth{1.003750pt}%
\definecolor{currentstroke}{rgb}{0.121569,0.466667,0.705882}%
\pgfsetstrokecolor{currentstroke}%
\pgfsetstrokeopacity{0.992346}%
\pgfsetdash{}{0pt}%
\pgfpathmoveto{\pgfqpoint{2.310735in}{1.852216in}}%
\pgfpathcurveto{\pgfqpoint{2.318971in}{1.852216in}}{\pgfqpoint{2.326872in}{1.855488in}}{\pgfqpoint{2.332695in}{1.861312in}}%
\pgfpathcurveto{\pgfqpoint{2.338519in}{1.867136in}}{\pgfqpoint{2.341792in}{1.875036in}}{\pgfqpoint{2.341792in}{1.883273in}}%
\pgfpathcurveto{\pgfqpoint{2.341792in}{1.891509in}}{\pgfqpoint{2.338519in}{1.899409in}}{\pgfqpoint{2.332695in}{1.905233in}}%
\pgfpathcurveto{\pgfqpoint{2.326872in}{1.911057in}}{\pgfqpoint{2.318971in}{1.914329in}}{\pgfqpoint{2.310735in}{1.914329in}}%
\pgfpathcurveto{\pgfqpoint{2.302499in}{1.914329in}}{\pgfqpoint{2.294599in}{1.911057in}}{\pgfqpoint{2.288775in}{1.905233in}}%
\pgfpathcurveto{\pgfqpoint{2.282951in}{1.899409in}}{\pgfqpoint{2.279679in}{1.891509in}}{\pgfqpoint{2.279679in}{1.883273in}}%
\pgfpathcurveto{\pgfqpoint{2.279679in}{1.875036in}}{\pgfqpoint{2.282951in}{1.867136in}}{\pgfqpoint{2.288775in}{1.861312in}}%
\pgfpathcurveto{\pgfqpoint{2.294599in}{1.855488in}}{\pgfqpoint{2.302499in}{1.852216in}}{\pgfqpoint{2.310735in}{1.852216in}}%
\pgfpathclose%
\pgfusepath{stroke,fill}%
\end{pgfscope}%
\begin{pgfscope}%
\pgfpathrectangle{\pgfqpoint{0.100000in}{0.212622in}}{\pgfqpoint{3.696000in}{3.696000in}}%
\pgfusepath{clip}%
\pgfsetbuttcap%
\pgfsetroundjoin%
\definecolor{currentfill}{rgb}{0.121569,0.466667,0.705882}%
\pgfsetfillcolor{currentfill}%
\pgfsetfillopacity{0.992580}%
\pgfsetlinewidth{1.003750pt}%
\definecolor{currentstroke}{rgb}{0.121569,0.466667,0.705882}%
\pgfsetstrokecolor{currentstroke}%
\pgfsetstrokeopacity{0.992580}%
\pgfsetdash{}{0pt}%
\pgfpathmoveto{\pgfqpoint{2.410481in}{1.814409in}}%
\pgfpathcurveto{\pgfqpoint{2.418717in}{1.814409in}}{\pgfqpoint{2.426617in}{1.817681in}}{\pgfqpoint{2.432441in}{1.823505in}}%
\pgfpathcurveto{\pgfqpoint{2.438265in}{1.829329in}}{\pgfqpoint{2.441538in}{1.837229in}}{\pgfqpoint{2.441538in}{1.845465in}}%
\pgfpathcurveto{\pgfqpoint{2.441538in}{1.853701in}}{\pgfqpoint{2.438265in}{1.861601in}}{\pgfqpoint{2.432441in}{1.867425in}}%
\pgfpathcurveto{\pgfqpoint{2.426617in}{1.873249in}}{\pgfqpoint{2.418717in}{1.876522in}}{\pgfqpoint{2.410481in}{1.876522in}}%
\pgfpathcurveto{\pgfqpoint{2.402245in}{1.876522in}}{\pgfqpoint{2.394345in}{1.873249in}}{\pgfqpoint{2.388521in}{1.867425in}}%
\pgfpathcurveto{\pgfqpoint{2.382697in}{1.861601in}}{\pgfqpoint{2.379425in}{1.853701in}}{\pgfqpoint{2.379425in}{1.845465in}}%
\pgfpathcurveto{\pgfqpoint{2.379425in}{1.837229in}}{\pgfqpoint{2.382697in}{1.829329in}}{\pgfqpoint{2.388521in}{1.823505in}}%
\pgfpathcurveto{\pgfqpoint{2.394345in}{1.817681in}}{\pgfqpoint{2.402245in}{1.814409in}}{\pgfqpoint{2.410481in}{1.814409in}}%
\pgfpathclose%
\pgfusepath{stroke,fill}%
\end{pgfscope}%
\begin{pgfscope}%
\pgfpathrectangle{\pgfqpoint{0.100000in}{0.212622in}}{\pgfqpoint{3.696000in}{3.696000in}}%
\pgfusepath{clip}%
\pgfsetbuttcap%
\pgfsetroundjoin%
\definecolor{currentfill}{rgb}{0.121569,0.466667,0.705882}%
\pgfsetfillcolor{currentfill}%
\pgfsetfillopacity{0.993550}%
\pgfsetlinewidth{1.003750pt}%
\definecolor{currentstroke}{rgb}{0.121569,0.466667,0.705882}%
\pgfsetstrokecolor{currentstroke}%
\pgfsetstrokeopacity{0.993550}%
\pgfsetdash{}{0pt}%
\pgfpathmoveto{\pgfqpoint{2.316733in}{1.846670in}}%
\pgfpathcurveto{\pgfqpoint{2.324969in}{1.846670in}}{\pgfqpoint{2.332869in}{1.849942in}}{\pgfqpoint{2.338693in}{1.855766in}}%
\pgfpathcurveto{\pgfqpoint{2.344517in}{1.861590in}}{\pgfqpoint{2.347789in}{1.869490in}}{\pgfqpoint{2.347789in}{1.877727in}}%
\pgfpathcurveto{\pgfqpoint{2.347789in}{1.885963in}}{\pgfqpoint{2.344517in}{1.893863in}}{\pgfqpoint{2.338693in}{1.899687in}}%
\pgfpathcurveto{\pgfqpoint{2.332869in}{1.905511in}}{\pgfqpoint{2.324969in}{1.908783in}}{\pgfqpoint{2.316733in}{1.908783in}}%
\pgfpathcurveto{\pgfqpoint{2.308496in}{1.908783in}}{\pgfqpoint{2.300596in}{1.905511in}}{\pgfqpoint{2.294772in}{1.899687in}}%
\pgfpathcurveto{\pgfqpoint{2.288949in}{1.893863in}}{\pgfqpoint{2.285676in}{1.885963in}}{\pgfqpoint{2.285676in}{1.877727in}}%
\pgfpathcurveto{\pgfqpoint{2.285676in}{1.869490in}}{\pgfqpoint{2.288949in}{1.861590in}}{\pgfqpoint{2.294772in}{1.855766in}}%
\pgfpathcurveto{\pgfqpoint{2.300596in}{1.849942in}}{\pgfqpoint{2.308496in}{1.846670in}}{\pgfqpoint{2.316733in}{1.846670in}}%
\pgfpathclose%
\pgfusepath{stroke,fill}%
\end{pgfscope}%
\begin{pgfscope}%
\pgfpathrectangle{\pgfqpoint{0.100000in}{0.212622in}}{\pgfqpoint{3.696000in}{3.696000in}}%
\pgfusepath{clip}%
\pgfsetbuttcap%
\pgfsetroundjoin%
\definecolor{currentfill}{rgb}{0.121569,0.466667,0.705882}%
\pgfsetfillcolor{currentfill}%
\pgfsetfillopacity{0.994043}%
\pgfsetlinewidth{1.003750pt}%
\definecolor{currentstroke}{rgb}{0.121569,0.466667,0.705882}%
\pgfsetstrokecolor{currentstroke}%
\pgfsetstrokeopacity{0.994043}%
\pgfsetdash{}{0pt}%
\pgfpathmoveto{\pgfqpoint{2.319176in}{1.843776in}}%
\pgfpathcurveto{\pgfqpoint{2.327412in}{1.843776in}}{\pgfqpoint{2.335312in}{1.847048in}}{\pgfqpoint{2.341136in}{1.852872in}}%
\pgfpathcurveto{\pgfqpoint{2.346960in}{1.858696in}}{\pgfqpoint{2.350232in}{1.866596in}}{\pgfqpoint{2.350232in}{1.874832in}}%
\pgfpathcurveto{\pgfqpoint{2.350232in}{1.883069in}}{\pgfqpoint{2.346960in}{1.890969in}}{\pgfqpoint{2.341136in}{1.896793in}}%
\pgfpathcurveto{\pgfqpoint{2.335312in}{1.902617in}}{\pgfqpoint{2.327412in}{1.905889in}}{\pgfqpoint{2.319176in}{1.905889in}}%
\pgfpathcurveto{\pgfqpoint{2.310939in}{1.905889in}}{\pgfqpoint{2.303039in}{1.902617in}}{\pgfqpoint{2.297215in}{1.896793in}}%
\pgfpathcurveto{\pgfqpoint{2.291391in}{1.890969in}}{\pgfqpoint{2.288119in}{1.883069in}}{\pgfqpoint{2.288119in}{1.874832in}}%
\pgfpathcurveto{\pgfqpoint{2.288119in}{1.866596in}}{\pgfqpoint{2.291391in}{1.858696in}}{\pgfqpoint{2.297215in}{1.852872in}}%
\pgfpathcurveto{\pgfqpoint{2.303039in}{1.847048in}}{\pgfqpoint{2.310939in}{1.843776in}}{\pgfqpoint{2.319176in}{1.843776in}}%
\pgfpathclose%
\pgfusepath{stroke,fill}%
\end{pgfscope}%
\begin{pgfscope}%
\pgfpathrectangle{\pgfqpoint{0.100000in}{0.212622in}}{\pgfqpoint{3.696000in}{3.696000in}}%
\pgfusepath{clip}%
\pgfsetbuttcap%
\pgfsetroundjoin%
\definecolor{currentfill}{rgb}{0.121569,0.466667,0.705882}%
\pgfsetfillcolor{currentfill}%
\pgfsetfillopacity{0.994887}%
\pgfsetlinewidth{1.003750pt}%
\definecolor{currentstroke}{rgb}{0.121569,0.466667,0.705882}%
\pgfsetstrokecolor{currentstroke}%
\pgfsetstrokeopacity{0.994887}%
\pgfsetdash{}{0pt}%
\pgfpathmoveto{\pgfqpoint{2.403343in}{1.811176in}}%
\pgfpathcurveto{\pgfqpoint{2.411579in}{1.811176in}}{\pgfqpoint{2.419479in}{1.814448in}}{\pgfqpoint{2.425303in}{1.820272in}}%
\pgfpathcurveto{\pgfqpoint{2.431127in}{1.826096in}}{\pgfqpoint{2.434399in}{1.833996in}}{\pgfqpoint{2.434399in}{1.842232in}}%
\pgfpathcurveto{\pgfqpoint{2.434399in}{1.850469in}}{\pgfqpoint{2.431127in}{1.858369in}}{\pgfqpoint{2.425303in}{1.864193in}}%
\pgfpathcurveto{\pgfqpoint{2.419479in}{1.870017in}}{\pgfqpoint{2.411579in}{1.873289in}}{\pgfqpoint{2.403343in}{1.873289in}}%
\pgfpathcurveto{\pgfqpoint{2.395107in}{1.873289in}}{\pgfqpoint{2.387207in}{1.870017in}}{\pgfqpoint{2.381383in}{1.864193in}}%
\pgfpathcurveto{\pgfqpoint{2.375559in}{1.858369in}}{\pgfqpoint{2.372286in}{1.850469in}}{\pgfqpoint{2.372286in}{1.842232in}}%
\pgfpathcurveto{\pgfqpoint{2.372286in}{1.833996in}}{\pgfqpoint{2.375559in}{1.826096in}}{\pgfqpoint{2.381383in}{1.820272in}}%
\pgfpathcurveto{\pgfqpoint{2.387207in}{1.814448in}}{\pgfqpoint{2.395107in}{1.811176in}}{\pgfqpoint{2.403343in}{1.811176in}}%
\pgfpathclose%
\pgfusepath{stroke,fill}%
\end{pgfscope}%
\begin{pgfscope}%
\pgfpathrectangle{\pgfqpoint{0.100000in}{0.212622in}}{\pgfqpoint{3.696000in}{3.696000in}}%
\pgfusepath{clip}%
\pgfsetbuttcap%
\pgfsetroundjoin%
\definecolor{currentfill}{rgb}{0.121569,0.466667,0.705882}%
\pgfsetfillcolor{currentfill}%
\pgfsetfillopacity{0.994943}%
\pgfsetlinewidth{1.003750pt}%
\definecolor{currentstroke}{rgb}{0.121569,0.466667,0.705882}%
\pgfsetstrokecolor{currentstroke}%
\pgfsetstrokeopacity{0.994943}%
\pgfsetdash{}{0pt}%
\pgfpathmoveto{\pgfqpoint{2.323888in}{1.839003in}}%
\pgfpathcurveto{\pgfqpoint{2.332124in}{1.839003in}}{\pgfqpoint{2.340024in}{1.842275in}}{\pgfqpoint{2.345848in}{1.848099in}}%
\pgfpathcurveto{\pgfqpoint{2.351672in}{1.853923in}}{\pgfqpoint{2.354945in}{1.861823in}}{\pgfqpoint{2.354945in}{1.870060in}}%
\pgfpathcurveto{\pgfqpoint{2.354945in}{1.878296in}}{\pgfqpoint{2.351672in}{1.886196in}}{\pgfqpoint{2.345848in}{1.892020in}}%
\pgfpathcurveto{\pgfqpoint{2.340024in}{1.897844in}}{\pgfqpoint{2.332124in}{1.901116in}}{\pgfqpoint{2.323888in}{1.901116in}}%
\pgfpathcurveto{\pgfqpoint{2.315652in}{1.901116in}}{\pgfqpoint{2.307752in}{1.897844in}}{\pgfqpoint{2.301928in}{1.892020in}}%
\pgfpathcurveto{\pgfqpoint{2.296104in}{1.886196in}}{\pgfqpoint{2.292832in}{1.878296in}}{\pgfqpoint{2.292832in}{1.870060in}}%
\pgfpathcurveto{\pgfqpoint{2.292832in}{1.861823in}}{\pgfqpoint{2.296104in}{1.853923in}}{\pgfqpoint{2.301928in}{1.848099in}}%
\pgfpathcurveto{\pgfqpoint{2.307752in}{1.842275in}}{\pgfqpoint{2.315652in}{1.839003in}}{\pgfqpoint{2.323888in}{1.839003in}}%
\pgfpathclose%
\pgfusepath{stroke,fill}%
\end{pgfscope}%
\begin{pgfscope}%
\pgfpathrectangle{\pgfqpoint{0.100000in}{0.212622in}}{\pgfqpoint{3.696000in}{3.696000in}}%
\pgfusepath{clip}%
\pgfsetbuttcap%
\pgfsetroundjoin%
\definecolor{currentfill}{rgb}{0.121569,0.466667,0.705882}%
\pgfsetfillcolor{currentfill}%
\pgfsetfillopacity{0.996403}%
\pgfsetlinewidth{1.003750pt}%
\definecolor{currentstroke}{rgb}{0.121569,0.466667,0.705882}%
\pgfsetstrokecolor{currentstroke}%
\pgfsetstrokeopacity{0.996403}%
\pgfsetdash{}{0pt}%
\pgfpathmoveto{\pgfqpoint{2.400035in}{1.809941in}}%
\pgfpathcurveto{\pgfqpoint{2.408271in}{1.809941in}}{\pgfqpoint{2.416171in}{1.813213in}}{\pgfqpoint{2.421995in}{1.819037in}}%
\pgfpathcurveto{\pgfqpoint{2.427819in}{1.824861in}}{\pgfqpoint{2.431091in}{1.832761in}}{\pgfqpoint{2.431091in}{1.840997in}}%
\pgfpathcurveto{\pgfqpoint{2.431091in}{1.849234in}}{\pgfqpoint{2.427819in}{1.857134in}}{\pgfqpoint{2.421995in}{1.862958in}}%
\pgfpathcurveto{\pgfqpoint{2.416171in}{1.868782in}}{\pgfqpoint{2.408271in}{1.872054in}}{\pgfqpoint{2.400035in}{1.872054in}}%
\pgfpathcurveto{\pgfqpoint{2.391798in}{1.872054in}}{\pgfqpoint{2.383898in}{1.868782in}}{\pgfqpoint{2.378074in}{1.862958in}}%
\pgfpathcurveto{\pgfqpoint{2.372251in}{1.857134in}}{\pgfqpoint{2.368978in}{1.849234in}}{\pgfqpoint{2.368978in}{1.840997in}}%
\pgfpathcurveto{\pgfqpoint{2.368978in}{1.832761in}}{\pgfqpoint{2.372251in}{1.824861in}}{\pgfqpoint{2.378074in}{1.819037in}}%
\pgfpathcurveto{\pgfqpoint{2.383898in}{1.813213in}}{\pgfqpoint{2.391798in}{1.809941in}}{\pgfqpoint{2.400035in}{1.809941in}}%
\pgfpathclose%
\pgfusepath{stroke,fill}%
\end{pgfscope}%
\begin{pgfscope}%
\pgfpathrectangle{\pgfqpoint{0.100000in}{0.212622in}}{\pgfqpoint{3.696000in}{3.696000in}}%
\pgfusepath{clip}%
\pgfsetbuttcap%
\pgfsetroundjoin%
\definecolor{currentfill}{rgb}{0.121569,0.466667,0.705882}%
\pgfsetfillcolor{currentfill}%
\pgfsetfillopacity{0.996503}%
\pgfsetlinewidth{1.003750pt}%
\definecolor{currentstroke}{rgb}{0.121569,0.466667,0.705882}%
\pgfsetstrokecolor{currentstroke}%
\pgfsetstrokeopacity{0.996503}%
\pgfsetdash{}{0pt}%
\pgfpathmoveto{\pgfqpoint{2.332392in}{1.829733in}}%
\pgfpathcurveto{\pgfqpoint{2.340628in}{1.829733in}}{\pgfqpoint{2.348528in}{1.833005in}}{\pgfqpoint{2.354352in}{1.838829in}}%
\pgfpathcurveto{\pgfqpoint{2.360176in}{1.844653in}}{\pgfqpoint{2.363448in}{1.852553in}}{\pgfqpoint{2.363448in}{1.860789in}}%
\pgfpathcurveto{\pgfqpoint{2.363448in}{1.869026in}}{\pgfqpoint{2.360176in}{1.876926in}}{\pgfqpoint{2.354352in}{1.882750in}}%
\pgfpathcurveto{\pgfqpoint{2.348528in}{1.888574in}}{\pgfqpoint{2.340628in}{1.891846in}}{\pgfqpoint{2.332392in}{1.891846in}}%
\pgfpathcurveto{\pgfqpoint{2.324155in}{1.891846in}}{\pgfqpoint{2.316255in}{1.888574in}}{\pgfqpoint{2.310432in}{1.882750in}}%
\pgfpathcurveto{\pgfqpoint{2.304608in}{1.876926in}}{\pgfqpoint{2.301335in}{1.869026in}}{\pgfqpoint{2.301335in}{1.860789in}}%
\pgfpathcurveto{\pgfqpoint{2.301335in}{1.852553in}}{\pgfqpoint{2.304608in}{1.844653in}}{\pgfqpoint{2.310432in}{1.838829in}}%
\pgfpathcurveto{\pgfqpoint{2.316255in}{1.833005in}}{\pgfqpoint{2.324155in}{1.829733in}}{\pgfqpoint{2.332392in}{1.829733in}}%
\pgfpathclose%
\pgfusepath{stroke,fill}%
\end{pgfscope}%
\begin{pgfscope}%
\pgfpathrectangle{\pgfqpoint{0.100000in}{0.212622in}}{\pgfqpoint{3.696000in}{3.696000in}}%
\pgfusepath{clip}%
\pgfsetbuttcap%
\pgfsetroundjoin%
\definecolor{currentfill}{rgb}{0.121569,0.466667,0.705882}%
\pgfsetfillcolor{currentfill}%
\pgfsetfillopacity{0.997048}%
\pgfsetlinewidth{1.003750pt}%
\definecolor{currentstroke}{rgb}{0.121569,0.466667,0.705882}%
\pgfsetstrokecolor{currentstroke}%
\pgfsetstrokeopacity{0.997048}%
\pgfsetdash{}{0pt}%
\pgfpathmoveto{\pgfqpoint{2.397947in}{1.808494in}}%
\pgfpathcurveto{\pgfqpoint{2.406184in}{1.808494in}}{\pgfqpoint{2.414084in}{1.811766in}}{\pgfqpoint{2.419908in}{1.817590in}}%
\pgfpathcurveto{\pgfqpoint{2.425731in}{1.823414in}}{\pgfqpoint{2.429004in}{1.831314in}}{\pgfqpoint{2.429004in}{1.839550in}}%
\pgfpathcurveto{\pgfqpoint{2.429004in}{1.847787in}}{\pgfqpoint{2.425731in}{1.855687in}}{\pgfqpoint{2.419908in}{1.861511in}}%
\pgfpathcurveto{\pgfqpoint{2.414084in}{1.867335in}}{\pgfqpoint{2.406184in}{1.870607in}}{\pgfqpoint{2.397947in}{1.870607in}}%
\pgfpathcurveto{\pgfqpoint{2.389711in}{1.870607in}}{\pgfqpoint{2.381811in}{1.867335in}}{\pgfqpoint{2.375987in}{1.861511in}}%
\pgfpathcurveto{\pgfqpoint{2.370163in}{1.855687in}}{\pgfqpoint{2.366891in}{1.847787in}}{\pgfqpoint{2.366891in}{1.839550in}}%
\pgfpathcurveto{\pgfqpoint{2.366891in}{1.831314in}}{\pgfqpoint{2.370163in}{1.823414in}}{\pgfqpoint{2.375987in}{1.817590in}}%
\pgfpathcurveto{\pgfqpoint{2.381811in}{1.811766in}}{\pgfqpoint{2.389711in}{1.808494in}}{\pgfqpoint{2.397947in}{1.808494in}}%
\pgfpathclose%
\pgfusepath{stroke,fill}%
\end{pgfscope}%
\begin{pgfscope}%
\pgfpathrectangle{\pgfqpoint{0.100000in}{0.212622in}}{\pgfqpoint{3.696000in}{3.696000in}}%
\pgfusepath{clip}%
\pgfsetbuttcap%
\pgfsetroundjoin%
\definecolor{currentfill}{rgb}{0.121569,0.466667,0.705882}%
\pgfsetfillcolor{currentfill}%
\pgfsetfillopacity{0.997552}%
\pgfsetlinewidth{1.003750pt}%
\definecolor{currentstroke}{rgb}{0.121569,0.466667,0.705882}%
\pgfsetstrokecolor{currentstroke}%
\pgfsetstrokeopacity{0.997552}%
\pgfsetdash{}{0pt}%
\pgfpathmoveto{\pgfqpoint{2.397132in}{1.808189in}}%
\pgfpathcurveto{\pgfqpoint{2.405368in}{1.808189in}}{\pgfqpoint{2.413269in}{1.811462in}}{\pgfqpoint{2.419092in}{1.817286in}}%
\pgfpathcurveto{\pgfqpoint{2.424916in}{1.823110in}}{\pgfqpoint{2.428189in}{1.831010in}}{\pgfqpoint{2.428189in}{1.839246in}}%
\pgfpathcurveto{\pgfqpoint{2.428189in}{1.847482in}}{\pgfqpoint{2.424916in}{1.855382in}}{\pgfqpoint{2.419092in}{1.861206in}}%
\pgfpathcurveto{\pgfqpoint{2.413269in}{1.867030in}}{\pgfqpoint{2.405368in}{1.870302in}}{\pgfqpoint{2.397132in}{1.870302in}}%
\pgfpathcurveto{\pgfqpoint{2.388896in}{1.870302in}}{\pgfqpoint{2.380996in}{1.867030in}}{\pgfqpoint{2.375172in}{1.861206in}}%
\pgfpathcurveto{\pgfqpoint{2.369348in}{1.855382in}}{\pgfqpoint{2.366076in}{1.847482in}}{\pgfqpoint{2.366076in}{1.839246in}}%
\pgfpathcurveto{\pgfqpoint{2.366076in}{1.831010in}}{\pgfqpoint{2.369348in}{1.823110in}}{\pgfqpoint{2.375172in}{1.817286in}}%
\pgfpathcurveto{\pgfqpoint{2.380996in}{1.811462in}}{\pgfqpoint{2.388896in}{1.808189in}}{\pgfqpoint{2.397132in}{1.808189in}}%
\pgfpathclose%
\pgfusepath{stroke,fill}%
\end{pgfscope}%
\begin{pgfscope}%
\pgfpathrectangle{\pgfqpoint{0.100000in}{0.212622in}}{\pgfqpoint{3.696000in}{3.696000in}}%
\pgfusepath{clip}%
\pgfsetbuttcap%
\pgfsetroundjoin%
\definecolor{currentfill}{rgb}{0.121569,0.466667,0.705882}%
\pgfsetfillcolor{currentfill}%
\pgfsetfillopacity{0.997777}%
\pgfsetlinewidth{1.003750pt}%
\definecolor{currentstroke}{rgb}{0.121569,0.466667,0.705882}%
\pgfsetstrokecolor{currentstroke}%
\pgfsetstrokeopacity{0.997777}%
\pgfsetdash{}{0pt}%
\pgfpathmoveto{\pgfqpoint{2.396539in}{1.807871in}}%
\pgfpathcurveto{\pgfqpoint{2.404776in}{1.807871in}}{\pgfqpoint{2.412676in}{1.811143in}}{\pgfqpoint{2.418500in}{1.816967in}}%
\pgfpathcurveto{\pgfqpoint{2.424324in}{1.822791in}}{\pgfqpoint{2.427596in}{1.830691in}}{\pgfqpoint{2.427596in}{1.838927in}}%
\pgfpathcurveto{\pgfqpoint{2.427596in}{1.847164in}}{\pgfqpoint{2.424324in}{1.855064in}}{\pgfqpoint{2.418500in}{1.860888in}}%
\pgfpathcurveto{\pgfqpoint{2.412676in}{1.866712in}}{\pgfqpoint{2.404776in}{1.869984in}}{\pgfqpoint{2.396539in}{1.869984in}}%
\pgfpathcurveto{\pgfqpoint{2.388303in}{1.869984in}}{\pgfqpoint{2.380403in}{1.866712in}}{\pgfqpoint{2.374579in}{1.860888in}}%
\pgfpathcurveto{\pgfqpoint{2.368755in}{1.855064in}}{\pgfqpoint{2.365483in}{1.847164in}}{\pgfqpoint{2.365483in}{1.838927in}}%
\pgfpathcurveto{\pgfqpoint{2.365483in}{1.830691in}}{\pgfqpoint{2.368755in}{1.822791in}}{\pgfqpoint{2.374579in}{1.816967in}}%
\pgfpathcurveto{\pgfqpoint{2.380403in}{1.811143in}}{\pgfqpoint{2.388303in}{1.807871in}}{\pgfqpoint{2.396539in}{1.807871in}}%
\pgfpathclose%
\pgfusepath{stroke,fill}%
\end{pgfscope}%
\begin{pgfscope}%
\pgfpathrectangle{\pgfqpoint{0.100000in}{0.212622in}}{\pgfqpoint{3.696000in}{3.696000in}}%
\pgfusepath{clip}%
\pgfsetbuttcap%
\pgfsetroundjoin%
\definecolor{currentfill}{rgb}{0.121569,0.466667,0.705882}%
\pgfsetfillcolor{currentfill}%
\pgfsetfillopacity{0.997908}%
\pgfsetlinewidth{1.003750pt}%
\definecolor{currentstroke}{rgb}{0.121569,0.466667,0.705882}%
\pgfsetstrokecolor{currentstroke}%
\pgfsetstrokeopacity{0.997908}%
\pgfsetdash{}{0pt}%
\pgfpathmoveto{\pgfqpoint{2.396201in}{1.807756in}}%
\pgfpathcurveto{\pgfqpoint{2.404437in}{1.807756in}}{\pgfqpoint{2.412337in}{1.811029in}}{\pgfqpoint{2.418161in}{1.816852in}}%
\pgfpathcurveto{\pgfqpoint{2.423985in}{1.822676in}}{\pgfqpoint{2.427257in}{1.830576in}}{\pgfqpoint{2.427257in}{1.838813in}}%
\pgfpathcurveto{\pgfqpoint{2.427257in}{1.847049in}}{\pgfqpoint{2.423985in}{1.854949in}}{\pgfqpoint{2.418161in}{1.860773in}}%
\pgfpathcurveto{\pgfqpoint{2.412337in}{1.866597in}}{\pgfqpoint{2.404437in}{1.869869in}}{\pgfqpoint{2.396201in}{1.869869in}}%
\pgfpathcurveto{\pgfqpoint{2.387964in}{1.869869in}}{\pgfqpoint{2.380064in}{1.866597in}}{\pgfqpoint{2.374241in}{1.860773in}}%
\pgfpathcurveto{\pgfqpoint{2.368417in}{1.854949in}}{\pgfqpoint{2.365144in}{1.847049in}}{\pgfqpoint{2.365144in}{1.838813in}}%
\pgfpathcurveto{\pgfqpoint{2.365144in}{1.830576in}}{\pgfqpoint{2.368417in}{1.822676in}}{\pgfqpoint{2.374241in}{1.816852in}}%
\pgfpathcurveto{\pgfqpoint{2.380064in}{1.811029in}}{\pgfqpoint{2.387964in}{1.807756in}}{\pgfqpoint{2.396201in}{1.807756in}}%
\pgfpathclose%
\pgfusepath{stroke,fill}%
\end{pgfscope}%
\begin{pgfscope}%
\pgfpathrectangle{\pgfqpoint{0.100000in}{0.212622in}}{\pgfqpoint{3.696000in}{3.696000in}}%
\pgfusepath{clip}%
\pgfsetbuttcap%
\pgfsetroundjoin%
\definecolor{currentfill}{rgb}{0.121569,0.466667,0.705882}%
\pgfsetfillcolor{currentfill}%
\pgfsetfillopacity{0.997979}%
\pgfsetlinewidth{1.003750pt}%
\definecolor{currentstroke}{rgb}{0.121569,0.466667,0.705882}%
\pgfsetstrokecolor{currentstroke}%
\pgfsetstrokeopacity{0.997979}%
\pgfsetdash{}{0pt}%
\pgfpathmoveto{\pgfqpoint{2.396049in}{1.807641in}}%
\pgfpathcurveto{\pgfqpoint{2.404285in}{1.807641in}}{\pgfqpoint{2.412185in}{1.810913in}}{\pgfqpoint{2.418009in}{1.816737in}}%
\pgfpathcurveto{\pgfqpoint{2.423833in}{1.822561in}}{\pgfqpoint{2.427105in}{1.830461in}}{\pgfqpoint{2.427105in}{1.838697in}}%
\pgfpathcurveto{\pgfqpoint{2.427105in}{1.846934in}}{\pgfqpoint{2.423833in}{1.854834in}}{\pgfqpoint{2.418009in}{1.860658in}}%
\pgfpathcurveto{\pgfqpoint{2.412185in}{1.866482in}}{\pgfqpoint{2.404285in}{1.869754in}}{\pgfqpoint{2.396049in}{1.869754in}}%
\pgfpathcurveto{\pgfqpoint{2.387812in}{1.869754in}}{\pgfqpoint{2.379912in}{1.866482in}}{\pgfqpoint{2.374088in}{1.860658in}}%
\pgfpathcurveto{\pgfqpoint{2.368264in}{1.854834in}}{\pgfqpoint{2.364992in}{1.846934in}}{\pgfqpoint{2.364992in}{1.838697in}}%
\pgfpathcurveto{\pgfqpoint{2.364992in}{1.830461in}}{\pgfqpoint{2.368264in}{1.822561in}}{\pgfqpoint{2.374088in}{1.816737in}}%
\pgfpathcurveto{\pgfqpoint{2.379912in}{1.810913in}}{\pgfqpoint{2.387812in}{1.807641in}}{\pgfqpoint{2.396049in}{1.807641in}}%
\pgfpathclose%
\pgfusepath{stroke,fill}%
\end{pgfscope}%
\begin{pgfscope}%
\pgfpathrectangle{\pgfqpoint{0.100000in}{0.212622in}}{\pgfqpoint{3.696000in}{3.696000in}}%
\pgfusepath{clip}%
\pgfsetbuttcap%
\pgfsetroundjoin%
\definecolor{currentfill}{rgb}{0.121569,0.466667,0.705882}%
\pgfsetfillcolor{currentfill}%
\pgfsetfillopacity{0.998019}%
\pgfsetlinewidth{1.003750pt}%
\definecolor{currentstroke}{rgb}{0.121569,0.466667,0.705882}%
\pgfsetstrokecolor{currentstroke}%
\pgfsetstrokeopacity{0.998019}%
\pgfsetdash{}{0pt}%
\pgfpathmoveto{\pgfqpoint{2.395944in}{1.807611in}}%
\pgfpathcurveto{\pgfqpoint{2.404181in}{1.807611in}}{\pgfqpoint{2.412081in}{1.810883in}}{\pgfqpoint{2.417905in}{1.816707in}}%
\pgfpathcurveto{\pgfqpoint{2.423729in}{1.822531in}}{\pgfqpoint{2.427001in}{1.830431in}}{\pgfqpoint{2.427001in}{1.838668in}}%
\pgfpathcurveto{\pgfqpoint{2.427001in}{1.846904in}}{\pgfqpoint{2.423729in}{1.854804in}}{\pgfqpoint{2.417905in}{1.860628in}}%
\pgfpathcurveto{\pgfqpoint{2.412081in}{1.866452in}}{\pgfqpoint{2.404181in}{1.869724in}}{\pgfqpoint{2.395944in}{1.869724in}}%
\pgfpathcurveto{\pgfqpoint{2.387708in}{1.869724in}}{\pgfqpoint{2.379808in}{1.866452in}}{\pgfqpoint{2.373984in}{1.860628in}}%
\pgfpathcurveto{\pgfqpoint{2.368160in}{1.854804in}}{\pgfqpoint{2.364888in}{1.846904in}}{\pgfqpoint{2.364888in}{1.838668in}}%
\pgfpathcurveto{\pgfqpoint{2.364888in}{1.830431in}}{\pgfqpoint{2.368160in}{1.822531in}}{\pgfqpoint{2.373984in}{1.816707in}}%
\pgfpathcurveto{\pgfqpoint{2.379808in}{1.810883in}}{\pgfqpoint{2.387708in}{1.807611in}}{\pgfqpoint{2.395944in}{1.807611in}}%
\pgfpathclose%
\pgfusepath{stroke,fill}%
\end{pgfscope}%
\begin{pgfscope}%
\pgfpathrectangle{\pgfqpoint{0.100000in}{0.212622in}}{\pgfqpoint{3.696000in}{3.696000in}}%
\pgfusepath{clip}%
\pgfsetbuttcap%
\pgfsetroundjoin%
\definecolor{currentfill}{rgb}{0.121569,0.466667,0.705882}%
\pgfsetfillcolor{currentfill}%
\pgfsetfillopacity{0.998041}%
\pgfsetlinewidth{1.003750pt}%
\definecolor{currentstroke}{rgb}{0.121569,0.466667,0.705882}%
\pgfsetstrokecolor{currentstroke}%
\pgfsetstrokeopacity{0.998041}%
\pgfsetdash{}{0pt}%
\pgfpathmoveto{\pgfqpoint{2.395893in}{1.807586in}}%
\pgfpathcurveto{\pgfqpoint{2.404130in}{1.807586in}}{\pgfqpoint{2.412030in}{1.810859in}}{\pgfqpoint{2.417854in}{1.816683in}}%
\pgfpathcurveto{\pgfqpoint{2.423677in}{1.822507in}}{\pgfqpoint{2.426950in}{1.830407in}}{\pgfqpoint{2.426950in}{1.838643in}}%
\pgfpathcurveto{\pgfqpoint{2.426950in}{1.846879in}}{\pgfqpoint{2.423677in}{1.854779in}}{\pgfqpoint{2.417854in}{1.860603in}}%
\pgfpathcurveto{\pgfqpoint{2.412030in}{1.866427in}}{\pgfqpoint{2.404130in}{1.869699in}}{\pgfqpoint{2.395893in}{1.869699in}}%
\pgfpathcurveto{\pgfqpoint{2.387657in}{1.869699in}}{\pgfqpoint{2.379757in}{1.866427in}}{\pgfqpoint{2.373933in}{1.860603in}}%
\pgfpathcurveto{\pgfqpoint{2.368109in}{1.854779in}}{\pgfqpoint{2.364837in}{1.846879in}}{\pgfqpoint{2.364837in}{1.838643in}}%
\pgfpathcurveto{\pgfqpoint{2.364837in}{1.830407in}}{\pgfqpoint{2.368109in}{1.822507in}}{\pgfqpoint{2.373933in}{1.816683in}}%
\pgfpathcurveto{\pgfqpoint{2.379757in}{1.810859in}}{\pgfqpoint{2.387657in}{1.807586in}}{\pgfqpoint{2.395893in}{1.807586in}}%
\pgfpathclose%
\pgfusepath{stroke,fill}%
\end{pgfscope}%
\begin{pgfscope}%
\pgfpathrectangle{\pgfqpoint{0.100000in}{0.212622in}}{\pgfqpoint{3.696000in}{3.696000in}}%
\pgfusepath{clip}%
\pgfsetbuttcap%
\pgfsetroundjoin%
\definecolor{currentfill}{rgb}{0.121569,0.466667,0.705882}%
\pgfsetfillcolor{currentfill}%
\pgfsetfillopacity{0.998053}%
\pgfsetlinewidth{1.003750pt}%
\definecolor{currentstroke}{rgb}{0.121569,0.466667,0.705882}%
\pgfsetstrokecolor{currentstroke}%
\pgfsetstrokeopacity{0.998053}%
\pgfsetdash{}{0pt}%
\pgfpathmoveto{\pgfqpoint{2.395864in}{1.807574in}}%
\pgfpathcurveto{\pgfqpoint{2.404100in}{1.807574in}}{\pgfqpoint{2.412000in}{1.810846in}}{\pgfqpoint{2.417824in}{1.816670in}}%
\pgfpathcurveto{\pgfqpoint{2.423648in}{1.822494in}}{\pgfqpoint{2.426920in}{1.830394in}}{\pgfqpoint{2.426920in}{1.838631in}}%
\pgfpathcurveto{\pgfqpoint{2.426920in}{1.846867in}}{\pgfqpoint{2.423648in}{1.854767in}}{\pgfqpoint{2.417824in}{1.860591in}}%
\pgfpathcurveto{\pgfqpoint{2.412000in}{1.866415in}}{\pgfqpoint{2.404100in}{1.869687in}}{\pgfqpoint{2.395864in}{1.869687in}}%
\pgfpathcurveto{\pgfqpoint{2.387628in}{1.869687in}}{\pgfqpoint{2.379728in}{1.866415in}}{\pgfqpoint{2.373904in}{1.860591in}}%
\pgfpathcurveto{\pgfqpoint{2.368080in}{1.854767in}}{\pgfqpoint{2.364807in}{1.846867in}}{\pgfqpoint{2.364807in}{1.838631in}}%
\pgfpathcurveto{\pgfqpoint{2.364807in}{1.830394in}}{\pgfqpoint{2.368080in}{1.822494in}}{\pgfqpoint{2.373904in}{1.816670in}}%
\pgfpathcurveto{\pgfqpoint{2.379728in}{1.810846in}}{\pgfqpoint{2.387628in}{1.807574in}}{\pgfqpoint{2.395864in}{1.807574in}}%
\pgfpathclose%
\pgfusepath{stroke,fill}%
\end{pgfscope}%
\begin{pgfscope}%
\pgfpathrectangle{\pgfqpoint{0.100000in}{0.212622in}}{\pgfqpoint{3.696000in}{3.696000in}}%
\pgfusepath{clip}%
\pgfsetbuttcap%
\pgfsetroundjoin%
\definecolor{currentfill}{rgb}{0.121569,0.466667,0.705882}%
\pgfsetfillcolor{currentfill}%
\pgfsetfillopacity{0.998059}%
\pgfsetlinewidth{1.003750pt}%
\definecolor{currentstroke}{rgb}{0.121569,0.466667,0.705882}%
\pgfsetstrokecolor{currentstroke}%
\pgfsetstrokeopacity{0.998059}%
\pgfsetdash{}{0pt}%
\pgfpathmoveto{\pgfqpoint{2.395847in}{1.807568in}}%
\pgfpathcurveto{\pgfqpoint{2.404083in}{1.807568in}}{\pgfqpoint{2.411983in}{1.810841in}}{\pgfqpoint{2.417807in}{1.816664in}}%
\pgfpathcurveto{\pgfqpoint{2.423631in}{1.822488in}}{\pgfqpoint{2.426903in}{1.830388in}}{\pgfqpoint{2.426903in}{1.838625in}}%
\pgfpathcurveto{\pgfqpoint{2.426903in}{1.846861in}}{\pgfqpoint{2.423631in}{1.854761in}}{\pgfqpoint{2.417807in}{1.860585in}}%
\pgfpathcurveto{\pgfqpoint{2.411983in}{1.866409in}}{\pgfqpoint{2.404083in}{1.869681in}}{\pgfqpoint{2.395847in}{1.869681in}}%
\pgfpathcurveto{\pgfqpoint{2.387610in}{1.869681in}}{\pgfqpoint{2.379710in}{1.866409in}}{\pgfqpoint{2.373887in}{1.860585in}}%
\pgfpathcurveto{\pgfqpoint{2.368063in}{1.854761in}}{\pgfqpoint{2.364790in}{1.846861in}}{\pgfqpoint{2.364790in}{1.838625in}}%
\pgfpathcurveto{\pgfqpoint{2.364790in}{1.830388in}}{\pgfqpoint{2.368063in}{1.822488in}}{\pgfqpoint{2.373887in}{1.816664in}}%
\pgfpathcurveto{\pgfqpoint{2.379710in}{1.810841in}}{\pgfqpoint{2.387610in}{1.807568in}}{\pgfqpoint{2.395847in}{1.807568in}}%
\pgfpathclose%
\pgfusepath{stroke,fill}%
\end{pgfscope}%
\begin{pgfscope}%
\pgfpathrectangle{\pgfqpoint{0.100000in}{0.212622in}}{\pgfqpoint{3.696000in}{3.696000in}}%
\pgfusepath{clip}%
\pgfsetbuttcap%
\pgfsetroundjoin%
\definecolor{currentfill}{rgb}{0.121569,0.466667,0.705882}%
\pgfsetfillcolor{currentfill}%
\pgfsetfillopacity{0.998063}%
\pgfsetlinewidth{1.003750pt}%
\definecolor{currentstroke}{rgb}{0.121569,0.466667,0.705882}%
\pgfsetstrokecolor{currentstroke}%
\pgfsetstrokeopacity{0.998063}%
\pgfsetdash{}{0pt}%
\pgfpathmoveto{\pgfqpoint{2.395837in}{1.807566in}}%
\pgfpathcurveto{\pgfqpoint{2.404073in}{1.807566in}}{\pgfqpoint{2.411974in}{1.810838in}}{\pgfqpoint{2.417797in}{1.816662in}}%
\pgfpathcurveto{\pgfqpoint{2.423621in}{1.822486in}}{\pgfqpoint{2.426894in}{1.830386in}}{\pgfqpoint{2.426894in}{1.838622in}}%
\pgfpathcurveto{\pgfqpoint{2.426894in}{1.846859in}}{\pgfqpoint{2.423621in}{1.854759in}}{\pgfqpoint{2.417797in}{1.860583in}}%
\pgfpathcurveto{\pgfqpoint{2.411974in}{1.866407in}}{\pgfqpoint{2.404073in}{1.869679in}}{\pgfqpoint{2.395837in}{1.869679in}}%
\pgfpathcurveto{\pgfqpoint{2.387601in}{1.869679in}}{\pgfqpoint{2.379701in}{1.866407in}}{\pgfqpoint{2.373877in}{1.860583in}}%
\pgfpathcurveto{\pgfqpoint{2.368053in}{1.854759in}}{\pgfqpoint{2.364781in}{1.846859in}}{\pgfqpoint{2.364781in}{1.838622in}}%
\pgfpathcurveto{\pgfqpoint{2.364781in}{1.830386in}}{\pgfqpoint{2.368053in}{1.822486in}}{\pgfqpoint{2.373877in}{1.816662in}}%
\pgfpathcurveto{\pgfqpoint{2.379701in}{1.810838in}}{\pgfqpoint{2.387601in}{1.807566in}}{\pgfqpoint{2.395837in}{1.807566in}}%
\pgfpathclose%
\pgfusepath{stroke,fill}%
\end{pgfscope}%
\begin{pgfscope}%
\pgfpathrectangle{\pgfqpoint{0.100000in}{0.212622in}}{\pgfqpoint{3.696000in}{3.696000in}}%
\pgfusepath{clip}%
\pgfsetbuttcap%
\pgfsetroundjoin%
\definecolor{currentfill}{rgb}{0.121569,0.466667,0.705882}%
\pgfsetfillcolor{currentfill}%
\pgfsetfillopacity{0.998065}%
\pgfsetlinewidth{1.003750pt}%
\definecolor{currentstroke}{rgb}{0.121569,0.466667,0.705882}%
\pgfsetstrokecolor{currentstroke}%
\pgfsetstrokeopacity{0.998065}%
\pgfsetdash{}{0pt}%
\pgfpathmoveto{\pgfqpoint{2.395831in}{1.807566in}}%
\pgfpathcurveto{\pgfqpoint{2.404068in}{1.807566in}}{\pgfqpoint{2.411968in}{1.810838in}}{\pgfqpoint{2.417792in}{1.816662in}}%
\pgfpathcurveto{\pgfqpoint{2.423615in}{1.822486in}}{\pgfqpoint{2.426888in}{1.830386in}}{\pgfqpoint{2.426888in}{1.838622in}}%
\pgfpathcurveto{\pgfqpoint{2.426888in}{1.846859in}}{\pgfqpoint{2.423615in}{1.854759in}}{\pgfqpoint{2.417792in}{1.860583in}}%
\pgfpathcurveto{\pgfqpoint{2.411968in}{1.866407in}}{\pgfqpoint{2.404068in}{1.869679in}}{\pgfqpoint{2.395831in}{1.869679in}}%
\pgfpathcurveto{\pgfqpoint{2.387595in}{1.869679in}}{\pgfqpoint{2.379695in}{1.866407in}}{\pgfqpoint{2.373871in}{1.860583in}}%
\pgfpathcurveto{\pgfqpoint{2.368047in}{1.854759in}}{\pgfqpoint{2.364775in}{1.846859in}}{\pgfqpoint{2.364775in}{1.838622in}}%
\pgfpathcurveto{\pgfqpoint{2.364775in}{1.830386in}}{\pgfqpoint{2.368047in}{1.822486in}}{\pgfqpoint{2.373871in}{1.816662in}}%
\pgfpathcurveto{\pgfqpoint{2.379695in}{1.810838in}}{\pgfqpoint{2.387595in}{1.807566in}}{\pgfqpoint{2.395831in}{1.807566in}}%
\pgfpathclose%
\pgfusepath{stroke,fill}%
\end{pgfscope}%
\begin{pgfscope}%
\pgfpathrectangle{\pgfqpoint{0.100000in}{0.212622in}}{\pgfqpoint{3.696000in}{3.696000in}}%
\pgfusepath{clip}%
\pgfsetbuttcap%
\pgfsetroundjoin%
\definecolor{currentfill}{rgb}{0.121569,0.466667,0.705882}%
\pgfsetfillcolor{currentfill}%
\pgfsetfillopacity{0.998423}%
\pgfsetlinewidth{1.003750pt}%
\definecolor{currentstroke}{rgb}{0.121569,0.466667,0.705882}%
\pgfsetstrokecolor{currentstroke}%
\pgfsetstrokeopacity{0.998423}%
\pgfsetdash{}{0pt}%
\pgfpathmoveto{\pgfqpoint{2.350975in}{1.814711in}}%
\pgfpathcurveto{\pgfqpoint{2.359212in}{1.814711in}}{\pgfqpoint{2.367112in}{1.817984in}}{\pgfqpoint{2.372935in}{1.823808in}}%
\pgfpathcurveto{\pgfqpoint{2.378759in}{1.829632in}}{\pgfqpoint{2.382032in}{1.837532in}}{\pgfqpoint{2.382032in}{1.845768in}}%
\pgfpathcurveto{\pgfqpoint{2.382032in}{1.854004in}}{\pgfqpoint{2.378759in}{1.861904in}}{\pgfqpoint{2.372935in}{1.867728in}}%
\pgfpathcurveto{\pgfqpoint{2.367112in}{1.873552in}}{\pgfqpoint{2.359212in}{1.876824in}}{\pgfqpoint{2.350975in}{1.876824in}}%
\pgfpathcurveto{\pgfqpoint{2.342739in}{1.876824in}}{\pgfqpoint{2.334839in}{1.873552in}}{\pgfqpoint{2.329015in}{1.867728in}}%
\pgfpathcurveto{\pgfqpoint{2.323191in}{1.861904in}}{\pgfqpoint{2.319919in}{1.854004in}}{\pgfqpoint{2.319919in}{1.845768in}}%
\pgfpathcurveto{\pgfqpoint{2.319919in}{1.837532in}}{\pgfqpoint{2.323191in}{1.829632in}}{\pgfqpoint{2.329015in}{1.823808in}}%
\pgfpathcurveto{\pgfqpoint{2.334839in}{1.817984in}}{\pgfqpoint{2.342739in}{1.814711in}}{\pgfqpoint{2.350975in}{1.814711in}}%
\pgfpathclose%
\pgfusepath{stroke,fill}%
\end{pgfscope}%
\begin{pgfscope}%
\pgfpathrectangle{\pgfqpoint{0.100000in}{0.212622in}}{\pgfqpoint{3.696000in}{3.696000in}}%
\pgfusepath{clip}%
\pgfsetbuttcap%
\pgfsetroundjoin%
\definecolor{currentfill}{rgb}{0.121569,0.466667,0.705882}%
\pgfsetfillcolor{currentfill}%
\pgfsetfillopacity{0.998561}%
\pgfsetlinewidth{1.003750pt}%
\definecolor{currentstroke}{rgb}{0.121569,0.466667,0.705882}%
\pgfsetstrokecolor{currentstroke}%
\pgfsetstrokeopacity{0.998561}%
\pgfsetdash{}{0pt}%
\pgfpathmoveto{\pgfqpoint{2.393989in}{1.807084in}}%
\pgfpathcurveto{\pgfqpoint{2.402226in}{1.807084in}}{\pgfqpoint{2.410126in}{1.810356in}}{\pgfqpoint{2.415949in}{1.816180in}}%
\pgfpathcurveto{\pgfqpoint{2.421773in}{1.822004in}}{\pgfqpoint{2.425046in}{1.829904in}}{\pgfqpoint{2.425046in}{1.838141in}}%
\pgfpathcurveto{\pgfqpoint{2.425046in}{1.846377in}}{\pgfqpoint{2.421773in}{1.854277in}}{\pgfqpoint{2.415949in}{1.860101in}}%
\pgfpathcurveto{\pgfqpoint{2.410126in}{1.865925in}}{\pgfqpoint{2.402226in}{1.869197in}}{\pgfqpoint{2.393989in}{1.869197in}}%
\pgfpathcurveto{\pgfqpoint{2.385753in}{1.869197in}}{\pgfqpoint{2.377853in}{1.865925in}}{\pgfqpoint{2.372029in}{1.860101in}}%
\pgfpathcurveto{\pgfqpoint{2.366205in}{1.854277in}}{\pgfqpoint{2.362933in}{1.846377in}}{\pgfqpoint{2.362933in}{1.838141in}}%
\pgfpathcurveto{\pgfqpoint{2.362933in}{1.829904in}}{\pgfqpoint{2.366205in}{1.822004in}}{\pgfqpoint{2.372029in}{1.816180in}}%
\pgfpathcurveto{\pgfqpoint{2.377853in}{1.810356in}}{\pgfqpoint{2.385753in}{1.807084in}}{\pgfqpoint{2.393989in}{1.807084in}}%
\pgfpathclose%
\pgfusepath{stroke,fill}%
\end{pgfscope}%
\begin{pgfscope}%
\pgfpathrectangle{\pgfqpoint{0.100000in}{0.212622in}}{\pgfqpoint{3.696000in}{3.696000in}}%
\pgfusepath{clip}%
\pgfsetbuttcap%
\pgfsetroundjoin%
\definecolor{currentfill}{rgb}{0.121569,0.466667,0.705882}%
\pgfsetfillcolor{currentfill}%
\pgfsetfillopacity{0.999323}%
\pgfsetlinewidth{1.003750pt}%
\definecolor{currentstroke}{rgb}{0.121569,0.466667,0.705882}%
\pgfsetstrokecolor{currentstroke}%
\pgfsetstrokeopacity{0.999323}%
\pgfsetdash{}{0pt}%
\pgfpathmoveto{\pgfqpoint{2.389510in}{1.806872in}}%
\pgfpathcurveto{\pgfqpoint{2.397746in}{1.806872in}}{\pgfqpoint{2.405646in}{1.810144in}}{\pgfqpoint{2.411470in}{1.815968in}}%
\pgfpathcurveto{\pgfqpoint{2.417294in}{1.821792in}}{\pgfqpoint{2.420567in}{1.829692in}}{\pgfqpoint{2.420567in}{1.837929in}}%
\pgfpathcurveto{\pgfqpoint{2.420567in}{1.846165in}}{\pgfqpoint{2.417294in}{1.854065in}}{\pgfqpoint{2.411470in}{1.859889in}}%
\pgfpathcurveto{\pgfqpoint{2.405646in}{1.865713in}}{\pgfqpoint{2.397746in}{1.868985in}}{\pgfqpoint{2.389510in}{1.868985in}}%
\pgfpathcurveto{\pgfqpoint{2.381274in}{1.868985in}}{\pgfqpoint{2.373374in}{1.865713in}}{\pgfqpoint{2.367550in}{1.859889in}}%
\pgfpathcurveto{\pgfqpoint{2.361726in}{1.854065in}}{\pgfqpoint{2.358454in}{1.846165in}}{\pgfqpoint{2.358454in}{1.837929in}}%
\pgfpathcurveto{\pgfqpoint{2.358454in}{1.829692in}}{\pgfqpoint{2.361726in}{1.821792in}}{\pgfqpoint{2.367550in}{1.815968in}}%
\pgfpathcurveto{\pgfqpoint{2.373374in}{1.810144in}}{\pgfqpoint{2.381274in}{1.806872in}}{\pgfqpoint{2.389510in}{1.806872in}}%
\pgfpathclose%
\pgfusepath{stroke,fill}%
\end{pgfscope}%
\begin{pgfscope}%
\pgfpathrectangle{\pgfqpoint{0.100000in}{0.212622in}}{\pgfqpoint{3.696000in}{3.696000in}}%
\pgfusepath{clip}%
\pgfsetbuttcap%
\pgfsetroundjoin%
\definecolor{currentfill}{rgb}{0.121569,0.466667,0.705882}%
\pgfsetfillcolor{currentfill}%
\pgfsetfillopacity{0.999777}%
\pgfsetlinewidth{1.003750pt}%
\definecolor{currentstroke}{rgb}{0.121569,0.466667,0.705882}%
\pgfsetstrokecolor{currentstroke}%
\pgfsetstrokeopacity{0.999777}%
\pgfsetdash{}{0pt}%
\pgfpathmoveto{\pgfqpoint{2.381163in}{1.805893in}}%
\pgfpathcurveto{\pgfqpoint{2.389399in}{1.805893in}}{\pgfqpoint{2.397299in}{1.809166in}}{\pgfqpoint{2.403123in}{1.814990in}}%
\pgfpathcurveto{\pgfqpoint{2.408947in}{1.820814in}}{\pgfqpoint{2.412219in}{1.828714in}}{\pgfqpoint{2.412219in}{1.836950in}}%
\pgfpathcurveto{\pgfqpoint{2.412219in}{1.845186in}}{\pgfqpoint{2.408947in}{1.853086in}}{\pgfqpoint{2.403123in}{1.858910in}}%
\pgfpathcurveto{\pgfqpoint{2.397299in}{1.864734in}}{\pgfqpoint{2.389399in}{1.868006in}}{\pgfqpoint{2.381163in}{1.868006in}}%
\pgfpathcurveto{\pgfqpoint{2.372926in}{1.868006in}}{\pgfqpoint{2.365026in}{1.864734in}}{\pgfqpoint{2.359203in}{1.858910in}}%
\pgfpathcurveto{\pgfqpoint{2.353379in}{1.853086in}}{\pgfqpoint{2.350106in}{1.845186in}}{\pgfqpoint{2.350106in}{1.836950in}}%
\pgfpathcurveto{\pgfqpoint{2.350106in}{1.828714in}}{\pgfqpoint{2.353379in}{1.820814in}}{\pgfqpoint{2.359203in}{1.814990in}}%
\pgfpathcurveto{\pgfqpoint{2.365026in}{1.809166in}}{\pgfqpoint{2.372926in}{1.805893in}}{\pgfqpoint{2.381163in}{1.805893in}}%
\pgfpathclose%
\pgfusepath{stroke,fill}%
\end{pgfscope}%
\begin{pgfscope}%
\pgfpathrectangle{\pgfqpoint{0.100000in}{0.212622in}}{\pgfqpoint{3.696000in}{3.696000in}}%
\pgfusepath{clip}%
\pgfsetbuttcap%
\pgfsetroundjoin%
\definecolor{currentfill}{rgb}{0.121569,0.466667,0.705882}%
\pgfsetfillcolor{currentfill}%
\pgfsetfillopacity{1.000000}%
\pgfsetlinewidth{1.003750pt}%
\definecolor{currentstroke}{rgb}{0.121569,0.466667,0.705882}%
\pgfsetstrokecolor{currentstroke}%
\pgfsetstrokeopacity{1.000000}%
\pgfsetdash{}{0pt}%
\pgfpathmoveto{\pgfqpoint{2.366282in}{1.808468in}}%
\pgfpathcurveto{\pgfqpoint{2.374518in}{1.808468in}}{\pgfqpoint{2.382418in}{1.811741in}}{\pgfqpoint{2.388242in}{1.817565in}}%
\pgfpathcurveto{\pgfqpoint{2.394066in}{1.823388in}}{\pgfqpoint{2.397339in}{1.831289in}}{\pgfqpoint{2.397339in}{1.839525in}}%
\pgfpathcurveto{\pgfqpoint{2.397339in}{1.847761in}}{\pgfqpoint{2.394066in}{1.855661in}}{\pgfqpoint{2.388242in}{1.861485in}}%
\pgfpathcurveto{\pgfqpoint{2.382418in}{1.867309in}}{\pgfqpoint{2.374518in}{1.870581in}}{\pgfqpoint{2.366282in}{1.870581in}}%
\pgfpathcurveto{\pgfqpoint{2.358046in}{1.870581in}}{\pgfqpoint{2.350146in}{1.867309in}}{\pgfqpoint{2.344322in}{1.861485in}}%
\pgfpathcurveto{\pgfqpoint{2.338498in}{1.855661in}}{\pgfqpoint{2.335226in}{1.847761in}}{\pgfqpoint{2.335226in}{1.839525in}}%
\pgfpathcurveto{\pgfqpoint{2.335226in}{1.831289in}}{\pgfqpoint{2.338498in}{1.823388in}}{\pgfqpoint{2.344322in}{1.817565in}}%
\pgfpathcurveto{\pgfqpoint{2.350146in}{1.811741in}}{\pgfqpoint{2.358046in}{1.808468in}}{\pgfqpoint{2.366282in}{1.808468in}}%
\pgfpathclose%
\pgfusepath{stroke,fill}%
\end{pgfscope}%
\begin{pgfscope}%
\pgfpathrectangle{\pgfqpoint{0.100000in}{0.212622in}}{\pgfqpoint{3.696000in}{3.696000in}}%
\pgfusepath{clip}%
\pgfsetbuttcap%
\pgfsetroundjoin%
\definecolor{currentfill}{rgb}{0.121569,0.466667,0.705882}%
\pgfsetfillcolor{currentfill}%
\pgfsetlinewidth{1.003750pt}%
\definecolor{currentstroke}{rgb}{0.121569,0.466667,0.705882}%
\pgfsetstrokecolor{currentstroke}%
\pgfsetdash{}{0pt}%
\pgfpathmoveto{\pgfqpoint{2.376496in}{1.806361in}}%
\pgfpathcurveto{\pgfqpoint{2.384733in}{1.806361in}}{\pgfqpoint{2.392633in}{1.809633in}}{\pgfqpoint{2.398457in}{1.815457in}}%
\pgfpathcurveto{\pgfqpoint{2.404281in}{1.821281in}}{\pgfqpoint{2.407553in}{1.829181in}}{\pgfqpoint{2.407553in}{1.837417in}}%
\pgfpathcurveto{\pgfqpoint{2.407553in}{1.845654in}}{\pgfqpoint{2.404281in}{1.853554in}}{\pgfqpoint{2.398457in}{1.859378in}}%
\pgfpathcurveto{\pgfqpoint{2.392633in}{1.865202in}}{\pgfqpoint{2.384733in}{1.868474in}}{\pgfqpoint{2.376496in}{1.868474in}}%
\pgfpathcurveto{\pgfqpoint{2.368260in}{1.868474in}}{\pgfqpoint{2.360360in}{1.865202in}}{\pgfqpoint{2.354536in}{1.859378in}}%
\pgfpathcurveto{\pgfqpoint{2.348712in}{1.853554in}}{\pgfqpoint{2.345440in}{1.845654in}}{\pgfqpoint{2.345440in}{1.837417in}}%
\pgfpathcurveto{\pgfqpoint{2.345440in}{1.829181in}}{\pgfqpoint{2.348712in}{1.821281in}}{\pgfqpoint{2.354536in}{1.815457in}}%
\pgfpathcurveto{\pgfqpoint{2.360360in}{1.809633in}}{\pgfqpoint{2.368260in}{1.806361in}}{\pgfqpoint{2.376496in}{1.806361in}}%
\pgfpathclose%
\pgfusepath{stroke,fill}%
\end{pgfscope}%
\begin{pgfscope}%
\definecolor{textcolor}{rgb}{0.000000,0.000000,0.000000}%
\pgfsetstrokecolor{textcolor}%
\pgfsetfillcolor{textcolor}%
\pgftext[x=1.948000in,y=3.991956in,,base]{\color{textcolor}\rmfamily\fontsize{12.000000}{14.400000}\selectfont Mahony}%
\end{pgfscope}%
\begin{pgfscope}%
\pgfsetbuttcap%
\pgfsetmiterjoin%
\definecolor{currentfill}{rgb}{1.000000,1.000000,1.000000}%
\pgfsetfillcolor{currentfill}%
\pgfsetfillopacity{0.800000}%
\pgfsetlinewidth{1.003750pt}%
\definecolor{currentstroke}{rgb}{0.800000,0.800000,0.800000}%
\pgfsetstrokecolor{currentstroke}%
\pgfsetstrokeopacity{0.800000}%
\pgfsetdash{}{0pt}%
\pgfpathmoveto{\pgfqpoint{2.104889in}{3.410289in}}%
\pgfpathlineto{\pgfqpoint{3.698778in}{3.410289in}}%
\pgfpathquadraticcurveto{\pgfqpoint{3.726556in}{3.410289in}}{\pgfqpoint{3.726556in}{3.438067in}}%
\pgfpathlineto{\pgfqpoint{3.726556in}{3.811400in}}%
\pgfpathquadraticcurveto{\pgfqpoint{3.726556in}{3.839178in}}{\pgfqpoint{3.698778in}{3.839178in}}%
\pgfpathlineto{\pgfqpoint{2.104889in}{3.839178in}}%
\pgfpathquadraticcurveto{\pgfqpoint{2.077111in}{3.839178in}}{\pgfqpoint{2.077111in}{3.811400in}}%
\pgfpathlineto{\pgfqpoint{2.077111in}{3.438067in}}%
\pgfpathquadraticcurveto{\pgfqpoint{2.077111in}{3.410289in}}{\pgfqpoint{2.104889in}{3.410289in}}%
\pgfpathclose%
\pgfusepath{stroke,fill}%
\end{pgfscope}%
\begin{pgfscope}%
\pgfsetrectcap%
\pgfsetroundjoin%
\pgfsetlinewidth{1.505625pt}%
\definecolor{currentstroke}{rgb}{0.121569,0.466667,0.705882}%
\pgfsetstrokecolor{currentstroke}%
\pgfsetdash{}{0pt}%
\pgfpathmoveto{\pgfqpoint{2.132667in}{3.735011in}}%
\pgfpathlineto{\pgfqpoint{2.410444in}{3.735011in}}%
\pgfusepath{stroke}%
\end{pgfscope}%
\begin{pgfscope}%
\definecolor{textcolor}{rgb}{0.000000,0.000000,0.000000}%
\pgfsetstrokecolor{textcolor}%
\pgfsetfillcolor{textcolor}%
\pgftext[x=2.521555in,y=3.686400in,left,base]{\color{textcolor}\rmfamily\fontsize{10.000000}{12.000000}\selectfont Ground truth}%
\end{pgfscope}%
\begin{pgfscope}%
\pgfsetbuttcap%
\pgfsetroundjoin%
\definecolor{currentfill}{rgb}{0.121569,0.466667,0.705882}%
\pgfsetfillcolor{currentfill}%
\pgfsetlinewidth{1.003750pt}%
\definecolor{currentstroke}{rgb}{0.121569,0.466667,0.705882}%
\pgfsetstrokecolor{currentstroke}%
\pgfsetdash{}{0pt}%
\pgfsys@defobject{currentmarker}{\pgfqpoint{-0.031056in}{-0.031056in}}{\pgfqpoint{0.031056in}{0.031056in}}{%
\pgfpathmoveto{\pgfqpoint{0.000000in}{-0.031056in}}%
\pgfpathcurveto{\pgfqpoint{0.008236in}{-0.031056in}}{\pgfqpoint{0.016136in}{-0.027784in}}{\pgfqpoint{0.021960in}{-0.021960in}}%
\pgfpathcurveto{\pgfqpoint{0.027784in}{-0.016136in}}{\pgfqpoint{0.031056in}{-0.008236in}}{\pgfqpoint{0.031056in}{0.000000in}}%
\pgfpathcurveto{\pgfqpoint{0.031056in}{0.008236in}}{\pgfqpoint{0.027784in}{0.016136in}}{\pgfqpoint{0.021960in}{0.021960in}}%
\pgfpathcurveto{\pgfqpoint{0.016136in}{0.027784in}}{\pgfqpoint{0.008236in}{0.031056in}}{\pgfqpoint{0.000000in}{0.031056in}}%
\pgfpathcurveto{\pgfqpoint{-0.008236in}{0.031056in}}{\pgfqpoint{-0.016136in}{0.027784in}}{\pgfqpoint{-0.021960in}{0.021960in}}%
\pgfpathcurveto{\pgfqpoint{-0.027784in}{0.016136in}}{\pgfqpoint{-0.031056in}{0.008236in}}{\pgfqpoint{-0.031056in}{0.000000in}}%
\pgfpathcurveto{\pgfqpoint{-0.031056in}{-0.008236in}}{\pgfqpoint{-0.027784in}{-0.016136in}}{\pgfqpoint{-0.021960in}{-0.021960in}}%
\pgfpathcurveto{\pgfqpoint{-0.016136in}{-0.027784in}}{\pgfqpoint{-0.008236in}{-0.031056in}}{\pgfqpoint{0.000000in}{-0.031056in}}%
\pgfpathclose%
\pgfusepath{stroke,fill}%
}%
\begin{pgfscope}%
\pgfsys@transformshift{2.271555in}{3.529248in}%
\pgfsys@useobject{currentmarker}{}%
\end{pgfscope}%
\end{pgfscope}%
\begin{pgfscope}%
\definecolor{textcolor}{rgb}{0.000000,0.000000,0.000000}%
\pgfsetstrokecolor{textcolor}%
\pgfsetfillcolor{textcolor}%
\pgftext[x=2.521555in,y=3.492789in,left,base]{\color{textcolor}\rmfamily\fontsize{10.000000}{12.000000}\selectfont Estimated position}%
\end{pgfscope}%
\end{pgfpicture}%
\makeatother%
\endgroup%
}
%         \caption{Tilt's 3D position estimation had the lowest displacement error for the 4-meter side square experiment.}

%         \label{fig:square42D}
%     \end{subfigure}
%     \begin{subfigure}{0.49\textwidth}
%         \centering
%         \resizebox{1\linewidth}{!}{%% Creator: Matplotlib, PGF backend
%%
%% To include the figure in your LaTeX document, write
%%   \input{<filename>.pgf}
%%
%% Make sure the required packages are loaded in your preamble
%%   \usepackage{pgf}
%%
%% and, on pdftex
%%   \usepackage[utf8]{inputenc}\DeclareUnicodeCharacter{2212}{-}
%%
%% or, on luatex and xetex
%%   \usepackage{unicode-math}
%%
%% Figures using additional raster images can only be included by \input if
%% they are in the same directory as the main LaTeX file. For loading figures
%% from other directories you can use the `import` package
%%   \usepackage{import}
%%
%% and then include the figures with
%%   \import{<path to file>}{<filename>.pgf}
%%
%% Matplotlib used the following preamble
%%   \usepackage{fontspec}
%%   \setmainfont{DejaVuSerif.ttf}[Path=C:/Users/Claudio/AppData/Local/Programs/Python/Python39/Lib/site-packages/matplotlib/mpl-data/fonts/ttf/]
%%   \setsansfont{DejaVuSans.ttf}[Path=C:/Users/Claudio/AppData/Local/Programs/Python/Python39/Lib/site-packages/matplotlib/mpl-data/fonts/ttf/]
%%   \setmonofont{DejaVuSansMono.ttf}[Path=C:/Users/Claudio/AppData/Local/Programs/Python/Python39/Lib/site-packages/matplotlib/mpl-data/fonts/ttf/]
%%
\begingroup%
\makeatletter%
\begin{pgfpicture}%
\pgfpathrectangle{\pgfpointorigin}{\pgfqpoint{4.342069in}{4.226689in}}%
\pgfusepath{use as bounding box, clip}%
\begin{pgfscope}%
\pgfsetbuttcap%
\pgfsetmiterjoin%
\definecolor{currentfill}{rgb}{1.000000,1.000000,1.000000}%
\pgfsetfillcolor{currentfill}%
\pgfsetlinewidth{0.000000pt}%
\definecolor{currentstroke}{rgb}{1.000000,1.000000,1.000000}%
\pgfsetstrokecolor{currentstroke}%
\pgfsetdash{}{0pt}%
\pgfpathmoveto{\pgfqpoint{0.000000in}{0.000000in}}%
\pgfpathlineto{\pgfqpoint{4.342069in}{0.000000in}}%
\pgfpathlineto{\pgfqpoint{4.342069in}{4.226689in}}%
\pgfpathlineto{\pgfqpoint{0.000000in}{4.226689in}}%
\pgfpathclose%
\pgfusepath{fill}%
\end{pgfscope}%
\begin{pgfscope}%
\pgfsetbuttcap%
\pgfsetmiterjoin%
\definecolor{currentfill}{rgb}{1.000000,1.000000,1.000000}%
\pgfsetfillcolor{currentfill}%
\pgfsetlinewidth{0.000000pt}%
\definecolor{currentstroke}{rgb}{0.000000,0.000000,0.000000}%
\pgfsetstrokecolor{currentstroke}%
\pgfsetstrokeopacity{0.000000}%
\pgfsetdash{}{0pt}%
\pgfpathmoveto{\pgfqpoint{0.100000in}{0.220728in}}%
\pgfpathlineto{\pgfqpoint{3.796000in}{0.220728in}}%
\pgfpathlineto{\pgfqpoint{3.796000in}{3.916728in}}%
\pgfpathlineto{\pgfqpoint{0.100000in}{3.916728in}}%
\pgfpathclose%
\pgfusepath{fill}%
\end{pgfscope}%
\begin{pgfscope}%
\pgfsetbuttcap%
\pgfsetmiterjoin%
\definecolor{currentfill}{rgb}{0.950000,0.950000,0.950000}%
\pgfsetfillcolor{currentfill}%
\pgfsetfillopacity{0.500000}%
\pgfsetlinewidth{1.003750pt}%
\definecolor{currentstroke}{rgb}{0.950000,0.950000,0.950000}%
\pgfsetstrokecolor{currentstroke}%
\pgfsetstrokeopacity{0.500000}%
\pgfsetdash{}{0pt}%
\pgfpathmoveto{\pgfqpoint{0.379073in}{1.132043in}}%
\pgfpathlineto{\pgfqpoint{1.599613in}{2.155124in}}%
\pgfpathlineto{\pgfqpoint{1.582647in}{3.630589in}}%
\pgfpathlineto{\pgfqpoint{0.303698in}{2.697271in}}%
\pgfusepath{stroke,fill}%
\end{pgfscope}%
\begin{pgfscope}%
\pgfsetbuttcap%
\pgfsetmiterjoin%
\definecolor{currentfill}{rgb}{0.900000,0.900000,0.900000}%
\pgfsetfillcolor{currentfill}%
\pgfsetfillopacity{0.500000}%
\pgfsetlinewidth{1.003750pt}%
\definecolor{currentstroke}{rgb}{0.900000,0.900000,0.900000}%
\pgfsetstrokecolor{currentstroke}%
\pgfsetstrokeopacity{0.500000}%
\pgfsetdash{}{0pt}%
\pgfpathmoveto{\pgfqpoint{1.599613in}{2.155124in}}%
\pgfpathlineto{\pgfqpoint{3.558144in}{1.585856in}}%
\pgfpathlineto{\pgfqpoint{3.628038in}{3.112142in}}%
\pgfpathlineto{\pgfqpoint{1.582647in}{3.630589in}}%
\pgfusepath{stroke,fill}%
\end{pgfscope}%
\begin{pgfscope}%
\pgfsetbuttcap%
\pgfsetmiterjoin%
\definecolor{currentfill}{rgb}{0.925000,0.925000,0.925000}%
\pgfsetfillcolor{currentfill}%
\pgfsetfillopacity{0.500000}%
\pgfsetlinewidth{1.003750pt}%
\definecolor{currentstroke}{rgb}{0.925000,0.925000,0.925000}%
\pgfsetstrokecolor{currentstroke}%
\pgfsetstrokeopacity{0.500000}%
\pgfsetdash{}{0pt}%
\pgfpathmoveto{\pgfqpoint{0.379073in}{1.132043in}}%
\pgfpathlineto{\pgfqpoint{2.455212in}{0.453976in}}%
\pgfpathlineto{\pgfqpoint{3.558144in}{1.585856in}}%
\pgfpathlineto{\pgfqpoint{1.599613in}{2.155124in}}%
\pgfusepath{stroke,fill}%
\end{pgfscope}%
\begin{pgfscope}%
\pgfsetrectcap%
\pgfsetroundjoin%
\pgfsetlinewidth{0.803000pt}%
\definecolor{currentstroke}{rgb}{0.000000,0.000000,0.000000}%
\pgfsetstrokecolor{currentstroke}%
\pgfsetdash{}{0pt}%
\pgfpathmoveto{\pgfqpoint{0.379073in}{1.132043in}}%
\pgfpathlineto{\pgfqpoint{2.455212in}{0.453976in}}%
\pgfusepath{stroke}%
\end{pgfscope}%
\begin{pgfscope}%
\definecolor{textcolor}{rgb}{0.000000,0.000000,0.000000}%
\pgfsetstrokecolor{textcolor}%
\pgfsetfillcolor{textcolor}%
\pgftext[x=0.697927in, y=0.423808in, left, base,rotate=341.912962]{\color{textcolor}\sffamily\fontsize{10.000000}{12.000000}\selectfont Position X [\(\displaystyle m\)]}%
\end{pgfscope}%
\begin{pgfscope}%
\pgfsetbuttcap%
\pgfsetroundjoin%
\pgfsetlinewidth{0.803000pt}%
\definecolor{currentstroke}{rgb}{0.690196,0.690196,0.690196}%
\pgfsetstrokecolor{currentstroke}%
\pgfsetdash{}{0pt}%
\pgfpathmoveto{\pgfqpoint{0.648810in}{1.043947in}}%
\pgfpathlineto{\pgfqpoint{1.855049in}{2.080879in}}%
\pgfpathlineto{\pgfqpoint{1.848922in}{3.563096in}}%
\pgfusepath{stroke}%
\end{pgfscope}%
\begin{pgfscope}%
\pgfsetbuttcap%
\pgfsetroundjoin%
\pgfsetlinewidth{0.803000pt}%
\definecolor{currentstroke}{rgb}{0.690196,0.690196,0.690196}%
\pgfsetstrokecolor{currentstroke}%
\pgfsetdash{}{0pt}%
\pgfpathmoveto{\pgfqpoint{0.995809in}{0.930617in}}%
\pgfpathlineto{\pgfqpoint{2.183220in}{1.985493in}}%
\pgfpathlineto{\pgfqpoint{2.191233in}{3.476330in}}%
\pgfusepath{stroke}%
\end{pgfscope}%
\begin{pgfscope}%
\pgfsetbuttcap%
\pgfsetroundjoin%
\pgfsetlinewidth{0.803000pt}%
\definecolor{currentstroke}{rgb}{0.690196,0.690196,0.690196}%
\pgfsetstrokecolor{currentstroke}%
\pgfsetdash{}{0pt}%
\pgfpathmoveto{\pgfqpoint{1.349102in}{0.815232in}}%
\pgfpathlineto{\pgfqpoint{2.516845in}{1.888521in}}%
\pgfpathlineto{\pgfqpoint{2.539482in}{3.388059in}}%
\pgfusepath{stroke}%
\end{pgfscope}%
\begin{pgfscope}%
\pgfsetbuttcap%
\pgfsetroundjoin%
\pgfsetlinewidth{0.803000pt}%
\definecolor{currentstroke}{rgb}{0.690196,0.690196,0.690196}%
\pgfsetstrokecolor{currentstroke}%
\pgfsetdash{}{0pt}%
\pgfpathmoveto{\pgfqpoint{1.708862in}{0.697734in}}%
\pgfpathlineto{\pgfqpoint{2.856063in}{1.789924in}}%
\pgfpathlineto{\pgfqpoint{2.893826in}{3.298244in}}%
\pgfusepath{stroke}%
\end{pgfscope}%
\begin{pgfscope}%
\pgfsetbuttcap%
\pgfsetroundjoin%
\pgfsetlinewidth{0.803000pt}%
\definecolor{currentstroke}{rgb}{0.690196,0.690196,0.690196}%
\pgfsetstrokecolor{currentstroke}%
\pgfsetdash{}{0pt}%
\pgfpathmoveto{\pgfqpoint{2.075268in}{0.578066in}}%
\pgfpathlineto{\pgfqpoint{3.201014in}{1.689660in}}%
\pgfpathlineto{\pgfqpoint{3.254425in}{3.206842in}}%
\pgfusepath{stroke}%
\end{pgfscope}%
\begin{pgfscope}%
\pgfsetrectcap%
\pgfsetroundjoin%
\pgfsetlinewidth{0.803000pt}%
\definecolor{currentstroke}{rgb}{0.000000,0.000000,0.000000}%
\pgfsetstrokecolor{currentstroke}%
\pgfsetdash{}{0pt}%
\pgfpathmoveto{\pgfqpoint{0.659317in}{1.052979in}}%
\pgfpathlineto{\pgfqpoint{0.627751in}{1.025844in}}%
\pgfusepath{stroke}%
\end{pgfscope}%
\begin{pgfscope}%
\definecolor{textcolor}{rgb}{0.000000,0.000000,0.000000}%
\pgfsetstrokecolor{textcolor}%
\pgfsetfillcolor{textcolor}%
\pgftext[x=0.544384in,y=0.824747in,,top]{\color{textcolor}\sffamily\fontsize{10.000000}{12.000000}\selectfont 0}%
\end{pgfscope}%
\begin{pgfscope}%
\pgfsetrectcap%
\pgfsetroundjoin%
\pgfsetlinewidth{0.803000pt}%
\definecolor{currentstroke}{rgb}{0.000000,0.000000,0.000000}%
\pgfsetstrokecolor{currentstroke}%
\pgfsetdash{}{0pt}%
\pgfpathmoveto{\pgfqpoint{1.006160in}{0.939813in}}%
\pgfpathlineto{\pgfqpoint{0.975063in}{0.912187in}}%
\pgfusepath{stroke}%
\end{pgfscope}%
\begin{pgfscope}%
\definecolor{textcolor}{rgb}{0.000000,0.000000,0.000000}%
\pgfsetstrokecolor{textcolor}%
\pgfsetfillcolor{textcolor}%
\pgftext[x=0.891748in,y=0.709013in,,top]{\color{textcolor}\sffamily\fontsize{10.000000}{12.000000}\selectfont 1}%
\end{pgfscope}%
\begin{pgfscope}%
\pgfsetrectcap%
\pgfsetroundjoin%
\pgfsetlinewidth{0.803000pt}%
\definecolor{currentstroke}{rgb}{0.000000,0.000000,0.000000}%
\pgfsetstrokecolor{currentstroke}%
\pgfsetdash{}{0pt}%
\pgfpathmoveto{\pgfqpoint{1.359289in}{0.824595in}}%
\pgfpathlineto{\pgfqpoint{1.328684in}{0.796465in}}%
\pgfusepath{stroke}%
\end{pgfscope}%
\begin{pgfscope}%
\definecolor{textcolor}{rgb}{0.000000,0.000000,0.000000}%
\pgfsetstrokecolor{textcolor}%
\pgfsetfillcolor{textcolor}%
\pgftext[x=1.245436in,y=0.591172in,,top]{\color{textcolor}\sffamily\fontsize{10.000000}{12.000000}\selectfont 2}%
\end{pgfscope}%
\begin{pgfscope}%
\pgfsetrectcap%
\pgfsetroundjoin%
\pgfsetlinewidth{0.803000pt}%
\definecolor{currentstroke}{rgb}{0.000000,0.000000,0.000000}%
\pgfsetstrokecolor{currentstroke}%
\pgfsetdash{}{0pt}%
\pgfpathmoveto{\pgfqpoint{1.718877in}{0.707269in}}%
\pgfpathlineto{\pgfqpoint{1.688788in}{0.678622in}}%
\pgfusepath{stroke}%
\end{pgfscope}%
\begin{pgfscope}%
\definecolor{textcolor}{rgb}{0.000000,0.000000,0.000000}%
\pgfsetstrokecolor{textcolor}%
\pgfsetfillcolor{textcolor}%
\pgftext[x=1.605622in,y=0.471165in,,top]{\color{textcolor}\sffamily\fontsize{10.000000}{12.000000}\selectfont 3}%
\end{pgfscope}%
\begin{pgfscope}%
\pgfsetrectcap%
\pgfsetroundjoin%
\pgfsetlinewidth{0.803000pt}%
\definecolor{currentstroke}{rgb}{0.000000,0.000000,0.000000}%
\pgfsetstrokecolor{currentstroke}%
\pgfsetdash{}{0pt}%
\pgfpathmoveto{\pgfqpoint{2.085104in}{0.587778in}}%
\pgfpathlineto{\pgfqpoint{2.055553in}{0.558599in}}%
\pgfusepath{stroke}%
\end{pgfscope}%
\begin{pgfscope}%
\definecolor{textcolor}{rgb}{0.000000,0.000000,0.000000}%
\pgfsetstrokecolor{textcolor}%
\pgfsetfillcolor{textcolor}%
\pgftext[x=1.972489in,y=0.348933in,,top]{\color{textcolor}\sffamily\fontsize{10.000000}{12.000000}\selectfont 4}%
\end{pgfscope}%
\begin{pgfscope}%
\pgfsetrectcap%
\pgfsetroundjoin%
\pgfsetlinewidth{0.803000pt}%
\definecolor{currentstroke}{rgb}{0.000000,0.000000,0.000000}%
\pgfsetstrokecolor{currentstroke}%
\pgfsetdash{}{0pt}%
\pgfpathmoveto{\pgfqpoint{3.558144in}{1.585856in}}%
\pgfpathlineto{\pgfqpoint{2.455212in}{0.453976in}}%
\pgfusepath{stroke}%
\end{pgfscope}%
\begin{pgfscope}%
\definecolor{textcolor}{rgb}{0.000000,0.000000,0.000000}%
\pgfsetstrokecolor{textcolor}%
\pgfsetfillcolor{textcolor}%
\pgftext[x=3.103916in, y=0.291339in, left, base,rotate=45.742112]{\color{textcolor}\sffamily\fontsize{10.000000}{12.000000}\selectfont Position Y [\(\displaystyle m\)]}%
\end{pgfscope}%
\begin{pgfscope}%
\pgfsetbuttcap%
\pgfsetroundjoin%
\pgfsetlinewidth{0.803000pt}%
\definecolor{currentstroke}{rgb}{0.690196,0.690196,0.690196}%
\pgfsetstrokecolor{currentstroke}%
\pgfsetdash{}{0pt}%
\pgfpathmoveto{\pgfqpoint{0.529420in}{2.861993in}}%
\pgfpathlineto{\pgfqpoint{0.593815in}{1.312044in}}%
\pgfpathlineto{\pgfqpoint{2.649969in}{0.653844in}}%
\pgfusepath{stroke}%
\end{pgfscope}%
\begin{pgfscope}%
\pgfsetbuttcap%
\pgfsetroundjoin%
\pgfsetlinewidth{0.803000pt}%
\definecolor{currentstroke}{rgb}{0.690196,0.690196,0.690196}%
\pgfsetstrokecolor{currentstroke}%
\pgfsetdash{}{0pt}%
\pgfpathmoveto{\pgfqpoint{0.772526in}{3.039400in}}%
\pgfpathlineto{\pgfqpoint{0.825415in}{1.506176in}}%
\pgfpathlineto{\pgfqpoint{2.859676in}{0.869056in}}%
\pgfusepath{stroke}%
\end{pgfscope}%
\begin{pgfscope}%
\pgfsetbuttcap%
\pgfsetroundjoin%
\pgfsetlinewidth{0.803000pt}%
\definecolor{currentstroke}{rgb}{0.690196,0.690196,0.690196}%
\pgfsetstrokecolor{currentstroke}%
\pgfsetdash{}{0pt}%
\pgfpathmoveto{\pgfqpoint{1.007768in}{3.211069in}}%
\pgfpathlineto{\pgfqpoint{1.049842in}{1.694295in}}%
\pgfpathlineto{\pgfqpoint{3.062553in}{1.077257in}}%
\pgfusepath{stroke}%
\end{pgfscope}%
\begin{pgfscope}%
\pgfsetbuttcap%
\pgfsetroundjoin%
\pgfsetlinewidth{0.803000pt}%
\definecolor{currentstroke}{rgb}{0.690196,0.690196,0.690196}%
\pgfsetstrokecolor{currentstroke}%
\pgfsetdash{}{0pt}%
\pgfpathmoveto{\pgfqpoint{1.235521in}{3.377273in}}%
\pgfpathlineto{\pgfqpoint{1.267424in}{1.876676in}}%
\pgfpathlineto{\pgfqpoint{3.258927in}{1.278786in}}%
\pgfusepath{stroke}%
\end{pgfscope}%
\begin{pgfscope}%
\pgfsetbuttcap%
\pgfsetroundjoin%
\pgfsetlinewidth{0.803000pt}%
\definecolor{currentstroke}{rgb}{0.690196,0.690196,0.690196}%
\pgfsetstrokecolor{currentstroke}%
\pgfsetdash{}{0pt}%
\pgfpathmoveto{\pgfqpoint{1.456138in}{3.538269in}}%
\pgfpathlineto{\pgfqpoint{1.478469in}{2.053578in}}%
\pgfpathlineto{\pgfqpoint{3.449107in}{1.473957in}}%
\pgfusepath{stroke}%
\end{pgfscope}%
\begin{pgfscope}%
\pgfsetrectcap%
\pgfsetroundjoin%
\pgfsetlinewidth{0.803000pt}%
\definecolor{currentstroke}{rgb}{0.000000,0.000000,0.000000}%
\pgfsetstrokecolor{currentstroke}%
\pgfsetdash{}{0pt}%
\pgfpathmoveto{\pgfqpoint{2.632649in}{0.659388in}}%
\pgfpathlineto{\pgfqpoint{2.684651in}{0.642742in}}%
\pgfusepath{stroke}%
\end{pgfscope}%
\begin{pgfscope}%
\definecolor{textcolor}{rgb}{0.000000,0.000000,0.000000}%
\pgfsetstrokecolor{textcolor}%
\pgfsetfillcolor{textcolor}%
\pgftext[x=2.827278in,y=0.469210in,,top]{\color{textcolor}\sffamily\fontsize{10.000000}{12.000000}\selectfont 0}%
\end{pgfscope}%
\begin{pgfscope}%
\pgfsetrectcap%
\pgfsetroundjoin%
\pgfsetlinewidth{0.803000pt}%
\definecolor{currentstroke}{rgb}{0.000000,0.000000,0.000000}%
\pgfsetstrokecolor{currentstroke}%
\pgfsetdash{}{0pt}%
\pgfpathmoveto{\pgfqpoint{2.842555in}{0.874418in}}%
\pgfpathlineto{\pgfqpoint{2.893960in}{0.858318in}}%
\pgfusepath{stroke}%
\end{pgfscope}%
\begin{pgfscope}%
\definecolor{textcolor}{rgb}{0.000000,0.000000,0.000000}%
\pgfsetstrokecolor{textcolor}%
\pgfsetfillcolor{textcolor}%
\pgftext[x=3.034171in,y=0.687605in,,top]{\color{textcolor}\sffamily\fontsize{10.000000}{12.000000}\selectfont 1}%
\end{pgfscope}%
\begin{pgfscope}%
\pgfsetrectcap%
\pgfsetroundjoin%
\pgfsetlinewidth{0.803000pt}%
\definecolor{currentstroke}{rgb}{0.000000,0.000000,0.000000}%
\pgfsetstrokecolor{currentstroke}%
\pgfsetdash{}{0pt}%
\pgfpathmoveto{\pgfqpoint{3.045627in}{1.082446in}}%
\pgfpathlineto{\pgfqpoint{3.096446in}{1.066867in}}%
\pgfusepath{stroke}%
\end{pgfscope}%
\begin{pgfscope}%
\definecolor{textcolor}{rgb}{0.000000,0.000000,0.000000}%
\pgfsetstrokecolor{textcolor}%
\pgfsetfillcolor{textcolor}%
\pgftext[x=3.234321in,y=0.898883in,,top]{\color{textcolor}\sffamily\fontsize{10.000000}{12.000000}\selectfont 2}%
\end{pgfscope}%
\begin{pgfscope}%
\pgfsetrectcap%
\pgfsetroundjoin%
\pgfsetlinewidth{0.803000pt}%
\definecolor{currentstroke}{rgb}{0.000000,0.000000,0.000000}%
\pgfsetstrokecolor{currentstroke}%
\pgfsetdash{}{0pt}%
\pgfpathmoveto{\pgfqpoint{3.242193in}{1.283810in}}%
\pgfpathlineto{\pgfqpoint{3.292436in}{1.268726in}}%
\pgfusepath{stroke}%
\end{pgfscope}%
\begin{pgfscope}%
\definecolor{textcolor}{rgb}{0.000000,0.000000,0.000000}%
\pgfsetstrokecolor{textcolor}%
\pgfsetfillcolor{textcolor}%
\pgftext[x=3.428052in,y=1.103386in,,top]{\color{textcolor}\sffamily\fontsize{10.000000}{12.000000}\selectfont 3}%
\end{pgfscope}%
\begin{pgfscope}%
\pgfsetrectcap%
\pgfsetroundjoin%
\pgfsetlinewidth{0.803000pt}%
\definecolor{currentstroke}{rgb}{0.000000,0.000000,0.000000}%
\pgfsetstrokecolor{currentstroke}%
\pgfsetdash{}{0pt}%
\pgfpathmoveto{\pgfqpoint{3.432561in}{1.478824in}}%
\pgfpathlineto{\pgfqpoint{3.482240in}{1.464212in}}%
\pgfusepath{stroke}%
\end{pgfscope}%
\begin{pgfscope}%
\definecolor{textcolor}{rgb}{0.000000,0.000000,0.000000}%
\pgfsetstrokecolor{textcolor}%
\pgfsetfillcolor{textcolor}%
\pgftext[x=3.615669in,y=1.301434in,,top]{\color{textcolor}\sffamily\fontsize{10.000000}{12.000000}\selectfont 4}%
\end{pgfscope}%
\begin{pgfscope}%
\pgfsetrectcap%
\pgfsetroundjoin%
\pgfsetlinewidth{0.803000pt}%
\definecolor{currentstroke}{rgb}{0.000000,0.000000,0.000000}%
\pgfsetstrokecolor{currentstroke}%
\pgfsetdash{}{0pt}%
\pgfpathmoveto{\pgfqpoint{3.558144in}{1.585856in}}%
\pgfpathlineto{\pgfqpoint{3.628038in}{3.112142in}}%
\pgfusepath{stroke}%
\end{pgfscope}%
\begin{pgfscope}%
\definecolor{textcolor}{rgb}{0.000000,0.000000,0.000000}%
\pgfsetstrokecolor{textcolor}%
\pgfsetfillcolor{textcolor}%
\pgftext[x=4.169544in, y=1.928890in, left, base,rotate=87.378092]{\color{textcolor}\sffamily\fontsize{10.000000}{12.000000}\selectfont Position Z [\(\displaystyle m\)]}%
\end{pgfscope}%
\begin{pgfscope}%
\pgfsetbuttcap%
\pgfsetroundjoin%
\pgfsetlinewidth{0.803000pt}%
\definecolor{currentstroke}{rgb}{0.690196,0.690196,0.690196}%
\pgfsetstrokecolor{currentstroke}%
\pgfsetdash{}{0pt}%
\pgfpathmoveto{\pgfqpoint{3.566860in}{1.776185in}}%
\pgfpathlineto{\pgfqpoint{1.597494in}{2.339455in}}%
\pgfpathlineto{\pgfqpoint{0.369688in}{1.326940in}}%
\pgfusepath{stroke}%
\end{pgfscope}%
\begin{pgfscope}%
\pgfsetbuttcap%
\pgfsetroundjoin%
\pgfsetlinewidth{0.803000pt}%
\definecolor{currentstroke}{rgb}{0.690196,0.690196,0.690196}%
\pgfsetstrokecolor{currentstroke}%
\pgfsetdash{}{0pt}%
\pgfpathmoveto{\pgfqpoint{3.582846in}{2.125284in}}%
\pgfpathlineto{\pgfqpoint{1.593609in}{2.677303in}}%
\pgfpathlineto{\pgfqpoint{0.352463in}{1.684631in}}%
\pgfusepath{stroke}%
\end{pgfscope}%
\begin{pgfscope}%
\pgfsetbuttcap%
\pgfsetroundjoin%
\pgfsetlinewidth{0.803000pt}%
\definecolor{currentstroke}{rgb}{0.690196,0.690196,0.690196}%
\pgfsetstrokecolor{currentstroke}%
\pgfsetdash{}{0pt}%
\pgfpathmoveto{\pgfqpoint{3.599162in}{2.481571in}}%
\pgfpathlineto{\pgfqpoint{1.589648in}{3.021769in}}%
\pgfpathlineto{\pgfqpoint{0.334869in}{2.049971in}}%
\pgfusepath{stroke}%
\end{pgfscope}%
\begin{pgfscope}%
\pgfsetbuttcap%
\pgfsetroundjoin%
\pgfsetlinewidth{0.803000pt}%
\definecolor{currentstroke}{rgb}{0.690196,0.690196,0.690196}%
\pgfsetstrokecolor{currentstroke}%
\pgfsetdash{}{0pt}%
\pgfpathmoveto{\pgfqpoint{3.615817in}{2.845270in}}%
\pgfpathlineto{\pgfqpoint{1.585608in}{3.373052in}}%
\pgfpathlineto{\pgfqpoint{0.316896in}{2.423208in}}%
\pgfusepath{stroke}%
\end{pgfscope}%
\begin{pgfscope}%
\pgfsetrectcap%
\pgfsetroundjoin%
\pgfsetlinewidth{0.803000pt}%
\definecolor{currentstroke}{rgb}{0.000000,0.000000,0.000000}%
\pgfsetstrokecolor{currentstroke}%
\pgfsetdash{}{0pt}%
\pgfpathmoveto{\pgfqpoint{3.550327in}{1.780913in}}%
\pgfpathlineto{\pgfqpoint{3.599965in}{1.766716in}}%
\pgfusepath{stroke}%
\end{pgfscope}%
\begin{pgfscope}%
\definecolor{textcolor}{rgb}{0.000000,0.000000,0.000000}%
\pgfsetstrokecolor{textcolor}%
\pgfsetfillcolor{textcolor}%
\pgftext[x=3.821699in,y=1.811993in,,top]{\color{textcolor}\sffamily\fontsize{10.000000}{12.000000}\selectfont 0.0}%
\end{pgfscope}%
\begin{pgfscope}%
\pgfsetrectcap%
\pgfsetroundjoin%
\pgfsetlinewidth{0.803000pt}%
\definecolor{currentstroke}{rgb}{0.000000,0.000000,0.000000}%
\pgfsetstrokecolor{currentstroke}%
\pgfsetdash{}{0pt}%
\pgfpathmoveto{\pgfqpoint{3.566139in}{2.129921in}}%
\pgfpathlineto{\pgfqpoint{3.616302in}{2.116000in}}%
\pgfusepath{stroke}%
\end{pgfscope}%
\begin{pgfscope}%
\definecolor{textcolor}{rgb}{0.000000,0.000000,0.000000}%
\pgfsetstrokecolor{textcolor}%
\pgfsetfillcolor{textcolor}%
\pgftext[x=3.840225in,y=2.160393in,,top]{\color{textcolor}\sffamily\fontsize{10.000000}{12.000000}\selectfont 0.1}%
\end{pgfscope}%
\begin{pgfscope}%
\pgfsetrectcap%
\pgfsetroundjoin%
\pgfsetlinewidth{0.803000pt}%
\definecolor{currentstroke}{rgb}{0.000000,0.000000,0.000000}%
\pgfsetstrokecolor{currentstroke}%
\pgfsetdash{}{0pt}%
\pgfpathmoveto{\pgfqpoint{3.582276in}{2.486110in}}%
\pgfpathlineto{\pgfqpoint{3.632975in}{2.472481in}}%
\pgfusepath{stroke}%
\end{pgfscope}%
\begin{pgfscope}%
\definecolor{textcolor}{rgb}{0.000000,0.000000,0.000000}%
\pgfsetstrokecolor{textcolor}%
\pgfsetfillcolor{textcolor}%
\pgftext[x=3.859132in,y=2.515944in,,top]{\color{textcolor}\sffamily\fontsize{10.000000}{12.000000}\selectfont 0.2}%
\end{pgfscope}%
\begin{pgfscope}%
\pgfsetrectcap%
\pgfsetroundjoin%
\pgfsetlinewidth{0.803000pt}%
\definecolor{currentstroke}{rgb}{0.000000,0.000000,0.000000}%
\pgfsetstrokecolor{currentstroke}%
\pgfsetdash{}{0pt}%
\pgfpathmoveto{\pgfqpoint{3.598748in}{2.849707in}}%
\pgfpathlineto{\pgfqpoint{3.649996in}{2.836385in}}%
\pgfusepath{stroke}%
\end{pgfscope}%
\begin{pgfscope}%
\definecolor{textcolor}{rgb}{0.000000,0.000000,0.000000}%
\pgfsetstrokecolor{textcolor}%
\pgfsetfillcolor{textcolor}%
\pgftext[x=3.878430in,y=2.878868in,,top]{\color{textcolor}\sffamily\fontsize{10.000000}{12.000000}\selectfont 0.3}%
\end{pgfscope}%
\begin{pgfscope}%
\pgfpathrectangle{\pgfqpoint{0.100000in}{0.220728in}}{\pgfqpoint{3.696000in}{3.696000in}}%
\pgfusepath{clip}%
\pgfsetrectcap%
\pgfsetroundjoin%
\pgfsetlinewidth{1.505625pt}%
\definecolor{currentstroke}{rgb}{1.000000,0.000000,0.000000}%
\pgfsetstrokecolor{currentstroke}%
\pgfsetdash{}{0pt}%
\pgfpathmoveto{\pgfqpoint{0.854595in}{1.420432in}}%
\pgfpathlineto{\pgfqpoint{0.854595in}{1.420432in}}%
\pgfusepath{stroke}%
\end{pgfscope}%
\begin{pgfscope}%
\pgfpathrectangle{\pgfqpoint{0.100000in}{0.220728in}}{\pgfqpoint{3.696000in}{3.696000in}}%
\pgfusepath{clip}%
\pgfsetrectcap%
\pgfsetroundjoin%
\pgfsetlinewidth{1.505625pt}%
\definecolor{currentstroke}{rgb}{1.000000,0.000000,0.000000}%
\pgfsetstrokecolor{currentstroke}%
\pgfsetdash{}{0pt}%
\pgfpathmoveto{\pgfqpoint{1.654126in}{3.157081in}}%
\pgfpathlineto{\pgfqpoint{1.733967in}{2.164240in}}%
\pgfusepath{stroke}%
\end{pgfscope}%
\begin{pgfscope}%
\pgfpathrectangle{\pgfqpoint{0.100000in}{0.220728in}}{\pgfqpoint{3.696000in}{3.696000in}}%
\pgfusepath{clip}%
\pgfsetrectcap%
\pgfsetroundjoin%
\pgfsetlinewidth{1.505625pt}%
\definecolor{currentstroke}{rgb}{1.000000,0.000000,0.000000}%
\pgfsetstrokecolor{currentstroke}%
\pgfsetdash{}{0pt}%
\pgfpathmoveto{\pgfqpoint{3.311522in}{2.907170in}}%
\pgfpathlineto{\pgfqpoint{3.095743in}{1.770078in}}%
\pgfusepath{stroke}%
\end{pgfscope}%
\begin{pgfscope}%
\pgfpathrectangle{\pgfqpoint{0.100000in}{0.220728in}}{\pgfqpoint{3.696000in}{3.696000in}}%
\pgfusepath{clip}%
\pgfsetrectcap%
\pgfsetroundjoin%
\pgfsetlinewidth{1.505625pt}%
\definecolor{currentstroke}{rgb}{1.000000,0.000000,0.000000}%
\pgfsetstrokecolor{currentstroke}%
\pgfsetdash{}{0pt}%
\pgfpathmoveto{\pgfqpoint{2.349891in}{1.416973in}}%
\pgfpathlineto{\pgfqpoint{2.275567in}{0.972587in}}%
\pgfusepath{stroke}%
\end{pgfscope}%
\begin{pgfscope}%
\pgfpathrectangle{\pgfqpoint{0.100000in}{0.220728in}}{\pgfqpoint{3.696000in}{3.696000in}}%
\pgfusepath{clip}%
\pgfsetrectcap%
\pgfsetroundjoin%
\pgfsetlinewidth{1.505625pt}%
\definecolor{currentstroke}{rgb}{1.000000,0.000000,0.000000}%
\pgfsetstrokecolor{currentstroke}%
\pgfsetdash{}{0pt}%
\pgfpathmoveto{\pgfqpoint{0.754893in}{2.350324in}}%
\pgfpathlineto{\pgfqpoint{0.854595in}{1.420432in}}%
\pgfusepath{stroke}%
\end{pgfscope}%
\begin{pgfscope}%
\pgfpathrectangle{\pgfqpoint{0.100000in}{0.220728in}}{\pgfqpoint{3.696000in}{3.696000in}}%
\pgfusepath{clip}%
\pgfsetrectcap%
\pgfsetroundjoin%
\pgfsetlinewidth{1.505625pt}%
\definecolor{currentstroke}{rgb}{0.121569,0.466667,0.705882}%
\pgfsetstrokecolor{currentstroke}%
\pgfsetdash{}{0pt}%
\pgfpathmoveto{\pgfqpoint{0.854595in}{1.420432in}}%
\pgfpathlineto{\pgfqpoint{1.733967in}{2.164240in}}%
\pgfpathlineto{\pgfqpoint{3.095743in}{1.770078in}}%
\pgfpathlineto{\pgfqpoint{2.275567in}{0.972587in}}%
\pgfpathlineto{\pgfqpoint{0.854595in}{1.420432in}}%
\pgfusepath{stroke}%
\end{pgfscope}%
\begin{pgfscope}%
\pgfpathrectangle{\pgfqpoint{0.100000in}{0.220728in}}{\pgfqpoint{3.696000in}{3.696000in}}%
\pgfusepath{clip}%
\pgfsetbuttcap%
\pgfsetroundjoin%
\definecolor{currentfill}{rgb}{0.121569,0.466667,0.705882}%
\pgfsetfillcolor{currentfill}%
\pgfsetfillopacity{0.300000}%
\pgfsetlinewidth{1.003750pt}%
\definecolor{currentstroke}{rgb}{0.121569,0.466667,0.705882}%
\pgfsetstrokecolor{currentstroke}%
\pgfsetstrokeopacity{0.300000}%
\pgfsetdash{}{0pt}%
\pgfpathmoveto{\pgfqpoint{1.643450in}{3.120439in}}%
\pgfpathcurveto{\pgfqpoint{1.651687in}{3.120439in}}{\pgfqpoint{1.659587in}{3.123711in}}{\pgfqpoint{1.665411in}{3.129535in}}%
\pgfpathcurveto{\pgfqpoint{1.671235in}{3.135359in}}{\pgfqpoint{1.674507in}{3.143259in}}{\pgfqpoint{1.674507in}{3.151495in}}%
\pgfpathcurveto{\pgfqpoint{1.674507in}{3.159732in}}{\pgfqpoint{1.671235in}{3.167632in}}{\pgfqpoint{1.665411in}{3.173456in}}%
\pgfpathcurveto{\pgfqpoint{1.659587in}{3.179280in}}{\pgfqpoint{1.651687in}{3.182552in}}{\pgfqpoint{1.643450in}{3.182552in}}%
\pgfpathcurveto{\pgfqpoint{1.635214in}{3.182552in}}{\pgfqpoint{1.627314in}{3.179280in}}{\pgfqpoint{1.621490in}{3.173456in}}%
\pgfpathcurveto{\pgfqpoint{1.615666in}{3.167632in}}{\pgfqpoint{1.612394in}{3.159732in}}{\pgfqpoint{1.612394in}{3.151495in}}%
\pgfpathcurveto{\pgfqpoint{1.612394in}{3.143259in}}{\pgfqpoint{1.615666in}{3.135359in}}{\pgfqpoint{1.621490in}{3.129535in}}%
\pgfpathcurveto{\pgfqpoint{1.627314in}{3.123711in}}{\pgfqpoint{1.635214in}{3.120439in}}{\pgfqpoint{1.643450in}{3.120439in}}%
\pgfpathclose%
\pgfusepath{stroke,fill}%
\end{pgfscope}%
\begin{pgfscope}%
\pgfpathrectangle{\pgfqpoint{0.100000in}{0.220728in}}{\pgfqpoint{3.696000in}{3.696000in}}%
\pgfusepath{clip}%
\pgfsetbuttcap%
\pgfsetroundjoin%
\definecolor{currentfill}{rgb}{0.121569,0.466667,0.705882}%
\pgfsetfillcolor{currentfill}%
\pgfsetfillopacity{0.300180}%
\pgfsetlinewidth{1.003750pt}%
\definecolor{currentstroke}{rgb}{0.121569,0.466667,0.705882}%
\pgfsetstrokecolor{currentstroke}%
\pgfsetstrokeopacity{0.300180}%
\pgfsetdash{}{0pt}%
\pgfpathmoveto{\pgfqpoint{1.654126in}{3.126024in}}%
\pgfpathcurveto{\pgfqpoint{1.662362in}{3.126024in}}{\pgfqpoint{1.670263in}{3.129297in}}{\pgfqpoint{1.676086in}{3.135121in}}%
\pgfpathcurveto{\pgfqpoint{1.681910in}{3.140945in}}{\pgfqpoint{1.685183in}{3.148845in}}{\pgfqpoint{1.685183in}{3.157081in}}%
\pgfpathcurveto{\pgfqpoint{1.685183in}{3.165317in}}{\pgfqpoint{1.681910in}{3.173217in}}{\pgfqpoint{1.676086in}{3.179041in}}%
\pgfpathcurveto{\pgfqpoint{1.670263in}{3.184865in}}{\pgfqpoint{1.662362in}{3.188137in}}{\pgfqpoint{1.654126in}{3.188137in}}%
\pgfpathcurveto{\pgfqpoint{1.645890in}{3.188137in}}{\pgfqpoint{1.637990in}{3.184865in}}{\pgfqpoint{1.632166in}{3.179041in}}%
\pgfpathcurveto{\pgfqpoint{1.626342in}{3.173217in}}{\pgfqpoint{1.623070in}{3.165317in}}{\pgfqpoint{1.623070in}{3.157081in}}%
\pgfpathcurveto{\pgfqpoint{1.623070in}{3.148845in}}{\pgfqpoint{1.626342in}{3.140945in}}{\pgfqpoint{1.632166in}{3.135121in}}%
\pgfpathcurveto{\pgfqpoint{1.637990in}{3.129297in}}{\pgfqpoint{1.645890in}{3.126024in}}{\pgfqpoint{1.654126in}{3.126024in}}%
\pgfpathclose%
\pgfusepath{stroke,fill}%
\end{pgfscope}%
\begin{pgfscope}%
\pgfpathrectangle{\pgfqpoint{0.100000in}{0.220728in}}{\pgfqpoint{3.696000in}{3.696000in}}%
\pgfusepath{clip}%
\pgfsetbuttcap%
\pgfsetroundjoin%
\definecolor{currentfill}{rgb}{0.121569,0.466667,0.705882}%
\pgfsetfillcolor{currentfill}%
\pgfsetfillopacity{0.300241}%
\pgfsetlinewidth{1.003750pt}%
\definecolor{currentstroke}{rgb}{0.121569,0.466667,0.705882}%
\pgfsetstrokecolor{currentstroke}%
\pgfsetstrokeopacity{0.300241}%
\pgfsetdash{}{0pt}%
\pgfpathmoveto{\pgfqpoint{1.636383in}{3.115577in}}%
\pgfpathcurveto{\pgfqpoint{1.644619in}{3.115577in}}{\pgfqpoint{1.652519in}{3.118849in}}{\pgfqpoint{1.658343in}{3.124673in}}%
\pgfpathcurveto{\pgfqpoint{1.664167in}{3.130497in}}{\pgfqpoint{1.667439in}{3.138397in}}{\pgfqpoint{1.667439in}{3.146634in}}%
\pgfpathcurveto{\pgfqpoint{1.667439in}{3.154870in}}{\pgfqpoint{1.664167in}{3.162770in}}{\pgfqpoint{1.658343in}{3.168594in}}%
\pgfpathcurveto{\pgfqpoint{1.652519in}{3.174418in}}{\pgfqpoint{1.644619in}{3.177690in}}{\pgfqpoint{1.636383in}{3.177690in}}%
\pgfpathcurveto{\pgfqpoint{1.628146in}{3.177690in}}{\pgfqpoint{1.620246in}{3.174418in}}{\pgfqpoint{1.614422in}{3.168594in}}%
\pgfpathcurveto{\pgfqpoint{1.608598in}{3.162770in}}{\pgfqpoint{1.605326in}{3.154870in}}{\pgfqpoint{1.605326in}{3.146634in}}%
\pgfpathcurveto{\pgfqpoint{1.605326in}{3.138397in}}{\pgfqpoint{1.608598in}{3.130497in}}{\pgfqpoint{1.614422in}{3.124673in}}%
\pgfpathcurveto{\pgfqpoint{1.620246in}{3.118849in}}{\pgfqpoint{1.628146in}{3.115577in}}{\pgfqpoint{1.636383in}{3.115577in}}%
\pgfpathclose%
\pgfusepath{stroke,fill}%
\end{pgfscope}%
\begin{pgfscope}%
\pgfpathrectangle{\pgfqpoint{0.100000in}{0.220728in}}{\pgfqpoint{3.696000in}{3.696000in}}%
\pgfusepath{clip}%
\pgfsetbuttcap%
\pgfsetroundjoin%
\definecolor{currentfill}{rgb}{0.121569,0.466667,0.705882}%
\pgfsetfillcolor{currentfill}%
\pgfsetfillopacity{0.300584}%
\pgfsetlinewidth{1.003750pt}%
\definecolor{currentstroke}{rgb}{0.121569,0.466667,0.705882}%
\pgfsetstrokecolor{currentstroke}%
\pgfsetstrokeopacity{0.300584}%
\pgfsetdash{}{0pt}%
\pgfpathmoveto{\pgfqpoint{1.632008in}{3.110993in}}%
\pgfpathcurveto{\pgfqpoint{1.640244in}{3.110993in}}{\pgfqpoint{1.648144in}{3.114265in}}{\pgfqpoint{1.653968in}{3.120089in}}%
\pgfpathcurveto{\pgfqpoint{1.659792in}{3.125913in}}{\pgfqpoint{1.663065in}{3.133813in}}{\pgfqpoint{1.663065in}{3.142049in}}%
\pgfpathcurveto{\pgfqpoint{1.663065in}{3.150285in}}{\pgfqpoint{1.659792in}{3.158185in}}{\pgfqpoint{1.653968in}{3.164009in}}%
\pgfpathcurveto{\pgfqpoint{1.648144in}{3.169833in}}{\pgfqpoint{1.640244in}{3.173106in}}{\pgfqpoint{1.632008in}{3.173106in}}%
\pgfpathcurveto{\pgfqpoint{1.623772in}{3.173106in}}{\pgfqpoint{1.615872in}{3.169833in}}{\pgfqpoint{1.610048in}{3.164009in}}%
\pgfpathcurveto{\pgfqpoint{1.604224in}{3.158185in}}{\pgfqpoint{1.600952in}{3.150285in}}{\pgfqpoint{1.600952in}{3.142049in}}%
\pgfpathcurveto{\pgfqpoint{1.600952in}{3.133813in}}{\pgfqpoint{1.604224in}{3.125913in}}{\pgfqpoint{1.610048in}{3.120089in}}%
\pgfpathcurveto{\pgfqpoint{1.615872in}{3.114265in}}{\pgfqpoint{1.623772in}{3.110993in}}{\pgfqpoint{1.632008in}{3.110993in}}%
\pgfpathclose%
\pgfusepath{stroke,fill}%
\end{pgfscope}%
\begin{pgfscope}%
\pgfpathrectangle{\pgfqpoint{0.100000in}{0.220728in}}{\pgfqpoint{3.696000in}{3.696000in}}%
\pgfusepath{clip}%
\pgfsetbuttcap%
\pgfsetroundjoin%
\definecolor{currentfill}{rgb}{0.121569,0.466667,0.705882}%
\pgfsetfillcolor{currentfill}%
\pgfsetfillopacity{0.300641}%
\pgfsetlinewidth{1.003750pt}%
\definecolor{currentstroke}{rgb}{0.121569,0.466667,0.705882}%
\pgfsetstrokecolor{currentstroke}%
\pgfsetstrokeopacity{0.300641}%
\pgfsetdash{}{0pt}%
\pgfpathmoveto{\pgfqpoint{1.667209in}{3.131193in}}%
\pgfpathcurveto{\pgfqpoint{1.675446in}{3.131193in}}{\pgfqpoint{1.683346in}{3.134466in}}{\pgfqpoint{1.689170in}{3.140289in}}%
\pgfpathcurveto{\pgfqpoint{1.694994in}{3.146113in}}{\pgfqpoint{1.698266in}{3.154013in}}{\pgfqpoint{1.698266in}{3.162250in}}%
\pgfpathcurveto{\pgfqpoint{1.698266in}{3.170486in}}{\pgfqpoint{1.694994in}{3.178386in}}{\pgfqpoint{1.689170in}{3.184210in}}%
\pgfpathcurveto{\pgfqpoint{1.683346in}{3.190034in}}{\pgfqpoint{1.675446in}{3.193306in}}{\pgfqpoint{1.667209in}{3.193306in}}%
\pgfpathcurveto{\pgfqpoint{1.658973in}{3.193306in}}{\pgfqpoint{1.651073in}{3.190034in}}{\pgfqpoint{1.645249in}{3.184210in}}%
\pgfpathcurveto{\pgfqpoint{1.639425in}{3.178386in}}{\pgfqpoint{1.636153in}{3.170486in}}{\pgfqpoint{1.636153in}{3.162250in}}%
\pgfpathcurveto{\pgfqpoint{1.636153in}{3.154013in}}{\pgfqpoint{1.639425in}{3.146113in}}{\pgfqpoint{1.645249in}{3.140289in}}%
\pgfpathcurveto{\pgfqpoint{1.651073in}{3.134466in}}{\pgfqpoint{1.658973in}{3.131193in}}{\pgfqpoint{1.667209in}{3.131193in}}%
\pgfpathclose%
\pgfusepath{stroke,fill}%
\end{pgfscope}%
\begin{pgfscope}%
\pgfpathrectangle{\pgfqpoint{0.100000in}{0.220728in}}{\pgfqpoint{3.696000in}{3.696000in}}%
\pgfusepath{clip}%
\pgfsetbuttcap%
\pgfsetroundjoin%
\definecolor{currentfill}{rgb}{0.121569,0.466667,0.705882}%
\pgfsetfillcolor{currentfill}%
\pgfsetfillopacity{0.300751}%
\pgfsetlinewidth{1.003750pt}%
\definecolor{currentstroke}{rgb}{0.121569,0.466667,0.705882}%
\pgfsetstrokecolor{currentstroke}%
\pgfsetstrokeopacity{0.300751}%
\pgfsetdash{}{0pt}%
\pgfpathmoveto{\pgfqpoint{1.630711in}{3.109411in}}%
\pgfpathcurveto{\pgfqpoint{1.638947in}{3.109411in}}{\pgfqpoint{1.646847in}{3.112683in}}{\pgfqpoint{1.652671in}{3.118507in}}%
\pgfpathcurveto{\pgfqpoint{1.658495in}{3.124331in}}{\pgfqpoint{1.661768in}{3.132231in}}{\pgfqpoint{1.661768in}{3.140467in}}%
\pgfpathcurveto{\pgfqpoint{1.661768in}{3.148703in}}{\pgfqpoint{1.658495in}{3.156603in}}{\pgfqpoint{1.652671in}{3.162427in}}%
\pgfpathcurveto{\pgfqpoint{1.646847in}{3.168251in}}{\pgfqpoint{1.638947in}{3.171524in}}{\pgfqpoint{1.630711in}{3.171524in}}%
\pgfpathcurveto{\pgfqpoint{1.622475in}{3.171524in}}{\pgfqpoint{1.614575in}{3.168251in}}{\pgfqpoint{1.608751in}{3.162427in}}%
\pgfpathcurveto{\pgfqpoint{1.602927in}{3.156603in}}{\pgfqpoint{1.599655in}{3.148703in}}{\pgfqpoint{1.599655in}{3.140467in}}%
\pgfpathcurveto{\pgfqpoint{1.599655in}{3.132231in}}{\pgfqpoint{1.602927in}{3.124331in}}{\pgfqpoint{1.608751in}{3.118507in}}%
\pgfpathcurveto{\pgfqpoint{1.614575in}{3.112683in}}{\pgfqpoint{1.622475in}{3.109411in}}{\pgfqpoint{1.630711in}{3.109411in}}%
\pgfpathclose%
\pgfusepath{stroke,fill}%
\end{pgfscope}%
\begin{pgfscope}%
\pgfpathrectangle{\pgfqpoint{0.100000in}{0.220728in}}{\pgfqpoint{3.696000in}{3.696000in}}%
\pgfusepath{clip}%
\pgfsetbuttcap%
\pgfsetroundjoin%
\definecolor{currentfill}{rgb}{0.121569,0.466667,0.705882}%
\pgfsetfillcolor{currentfill}%
\pgfsetfillopacity{0.301102}%
\pgfsetlinewidth{1.003750pt}%
\definecolor{currentstroke}{rgb}{0.121569,0.466667,0.705882}%
\pgfsetstrokecolor{currentstroke}%
\pgfsetstrokeopacity{0.301102}%
\pgfsetdash{}{0pt}%
\pgfpathmoveto{\pgfqpoint{1.628590in}{3.106317in}}%
\pgfpathcurveto{\pgfqpoint{1.636826in}{3.106317in}}{\pgfqpoint{1.644726in}{3.109590in}}{\pgfqpoint{1.650550in}{3.115414in}}%
\pgfpathcurveto{\pgfqpoint{1.656374in}{3.121237in}}{\pgfqpoint{1.659646in}{3.129137in}}{\pgfqpoint{1.659646in}{3.137374in}}%
\pgfpathcurveto{\pgfqpoint{1.659646in}{3.145610in}}{\pgfqpoint{1.656374in}{3.153510in}}{\pgfqpoint{1.650550in}{3.159334in}}%
\pgfpathcurveto{\pgfqpoint{1.644726in}{3.165158in}}{\pgfqpoint{1.636826in}{3.168430in}}{\pgfqpoint{1.628590in}{3.168430in}}%
\pgfpathcurveto{\pgfqpoint{1.620353in}{3.168430in}}{\pgfqpoint{1.612453in}{3.165158in}}{\pgfqpoint{1.606629in}{3.159334in}}%
\pgfpathcurveto{\pgfqpoint{1.600805in}{3.153510in}}{\pgfqpoint{1.597533in}{3.145610in}}{\pgfqpoint{1.597533in}{3.137374in}}%
\pgfpathcurveto{\pgfqpoint{1.597533in}{3.129137in}}{\pgfqpoint{1.600805in}{3.121237in}}{\pgfqpoint{1.606629in}{3.115414in}}%
\pgfpathcurveto{\pgfqpoint{1.612453in}{3.109590in}}{\pgfqpoint{1.620353in}{3.106317in}}{\pgfqpoint{1.628590in}{3.106317in}}%
\pgfpathclose%
\pgfusepath{stroke,fill}%
\end{pgfscope}%
\begin{pgfscope}%
\pgfpathrectangle{\pgfqpoint{0.100000in}{0.220728in}}{\pgfqpoint{3.696000in}{3.696000in}}%
\pgfusepath{clip}%
\pgfsetbuttcap%
\pgfsetroundjoin%
\definecolor{currentfill}{rgb}{0.121569,0.466667,0.705882}%
\pgfsetfillcolor{currentfill}%
\pgfsetfillopacity{0.301102}%
\pgfsetlinewidth{1.003750pt}%
\definecolor{currentstroke}{rgb}{0.121569,0.466667,0.705882}%
\pgfsetstrokecolor{currentstroke}%
\pgfsetstrokeopacity{0.301102}%
\pgfsetdash{}{0pt}%
\pgfpathmoveto{\pgfqpoint{1.628588in}{3.106315in}}%
\pgfpathcurveto{\pgfqpoint{1.636824in}{3.106315in}}{\pgfqpoint{1.644724in}{3.109587in}}{\pgfqpoint{1.650548in}{3.115411in}}%
\pgfpathcurveto{\pgfqpoint{1.656372in}{3.121235in}}{\pgfqpoint{1.659645in}{3.129135in}}{\pgfqpoint{1.659645in}{3.137371in}}%
\pgfpathcurveto{\pgfqpoint{1.659645in}{3.145607in}}{\pgfqpoint{1.656372in}{3.153508in}}{\pgfqpoint{1.650548in}{3.159331in}}%
\pgfpathcurveto{\pgfqpoint{1.644724in}{3.165155in}}{\pgfqpoint{1.636824in}{3.168428in}}{\pgfqpoint{1.628588in}{3.168428in}}%
\pgfpathcurveto{\pgfqpoint{1.620352in}{3.168428in}}{\pgfqpoint{1.612452in}{3.165155in}}{\pgfqpoint{1.606628in}{3.159331in}}%
\pgfpathcurveto{\pgfqpoint{1.600804in}{3.153508in}}{\pgfqpoint{1.597532in}{3.145607in}}{\pgfqpoint{1.597532in}{3.137371in}}%
\pgfpathcurveto{\pgfqpoint{1.597532in}{3.129135in}}{\pgfqpoint{1.600804in}{3.121235in}}{\pgfqpoint{1.606628in}{3.115411in}}%
\pgfpathcurveto{\pgfqpoint{1.612452in}{3.109587in}}{\pgfqpoint{1.620352in}{3.106315in}}{\pgfqpoint{1.628588in}{3.106315in}}%
\pgfpathclose%
\pgfusepath{stroke,fill}%
\end{pgfscope}%
\begin{pgfscope}%
\pgfpathrectangle{\pgfqpoint{0.100000in}{0.220728in}}{\pgfqpoint{3.696000in}{3.696000in}}%
\pgfusepath{clip}%
\pgfsetbuttcap%
\pgfsetroundjoin%
\definecolor{currentfill}{rgb}{0.121569,0.466667,0.705882}%
\pgfsetfillcolor{currentfill}%
\pgfsetfillopacity{0.301103}%
\pgfsetlinewidth{1.003750pt}%
\definecolor{currentstroke}{rgb}{0.121569,0.466667,0.705882}%
\pgfsetstrokecolor{currentstroke}%
\pgfsetstrokeopacity{0.301103}%
\pgfsetdash{}{0pt}%
\pgfpathmoveto{\pgfqpoint{1.628586in}{3.106310in}}%
\pgfpathcurveto{\pgfqpoint{1.636822in}{3.106310in}}{\pgfqpoint{1.644722in}{3.109582in}}{\pgfqpoint{1.650546in}{3.115406in}}%
\pgfpathcurveto{\pgfqpoint{1.656370in}{3.121230in}}{\pgfqpoint{1.659642in}{3.129130in}}{\pgfqpoint{1.659642in}{3.137366in}}%
\pgfpathcurveto{\pgfqpoint{1.659642in}{3.145603in}}{\pgfqpoint{1.656370in}{3.153503in}}{\pgfqpoint{1.650546in}{3.159327in}}%
\pgfpathcurveto{\pgfqpoint{1.644722in}{3.165151in}}{\pgfqpoint{1.636822in}{3.168423in}}{\pgfqpoint{1.628586in}{3.168423in}}%
\pgfpathcurveto{\pgfqpoint{1.620349in}{3.168423in}}{\pgfqpoint{1.612449in}{3.165151in}}{\pgfqpoint{1.606625in}{3.159327in}}%
\pgfpathcurveto{\pgfqpoint{1.600801in}{3.153503in}}{\pgfqpoint{1.597529in}{3.145603in}}{\pgfqpoint{1.597529in}{3.137366in}}%
\pgfpathcurveto{\pgfqpoint{1.597529in}{3.129130in}}{\pgfqpoint{1.600801in}{3.121230in}}{\pgfqpoint{1.606625in}{3.115406in}}%
\pgfpathcurveto{\pgfqpoint{1.612449in}{3.109582in}}{\pgfqpoint{1.620349in}{3.106310in}}{\pgfqpoint{1.628586in}{3.106310in}}%
\pgfpathclose%
\pgfusepath{stroke,fill}%
\end{pgfscope}%
\begin{pgfscope}%
\pgfpathrectangle{\pgfqpoint{0.100000in}{0.220728in}}{\pgfqpoint{3.696000in}{3.696000in}}%
\pgfusepath{clip}%
\pgfsetbuttcap%
\pgfsetroundjoin%
\definecolor{currentfill}{rgb}{0.121569,0.466667,0.705882}%
\pgfsetfillcolor{currentfill}%
\pgfsetfillopacity{0.301104}%
\pgfsetlinewidth{1.003750pt}%
\definecolor{currentstroke}{rgb}{0.121569,0.466667,0.705882}%
\pgfsetstrokecolor{currentstroke}%
\pgfsetstrokeopacity{0.301104}%
\pgfsetdash{}{0pt}%
\pgfpathmoveto{\pgfqpoint{1.628581in}{3.106302in}}%
\pgfpathcurveto{\pgfqpoint{1.636817in}{3.106302in}}{\pgfqpoint{1.644717in}{3.109574in}}{\pgfqpoint{1.650541in}{3.115398in}}%
\pgfpathcurveto{\pgfqpoint{1.656365in}{3.121222in}}{\pgfqpoint{1.659637in}{3.129122in}}{\pgfqpoint{1.659637in}{3.137358in}}%
\pgfpathcurveto{\pgfqpoint{1.659637in}{3.145594in}}{\pgfqpoint{1.656365in}{3.153495in}}{\pgfqpoint{1.650541in}{3.159318in}}%
\pgfpathcurveto{\pgfqpoint{1.644717in}{3.165142in}}{\pgfqpoint{1.636817in}{3.168415in}}{\pgfqpoint{1.628581in}{3.168415in}}%
\pgfpathcurveto{\pgfqpoint{1.620345in}{3.168415in}}{\pgfqpoint{1.612445in}{3.165142in}}{\pgfqpoint{1.606621in}{3.159318in}}%
\pgfpathcurveto{\pgfqpoint{1.600797in}{3.153495in}}{\pgfqpoint{1.597524in}{3.145594in}}{\pgfqpoint{1.597524in}{3.137358in}}%
\pgfpathcurveto{\pgfqpoint{1.597524in}{3.129122in}}{\pgfqpoint{1.600797in}{3.121222in}}{\pgfqpoint{1.606621in}{3.115398in}}%
\pgfpathcurveto{\pgfqpoint{1.612445in}{3.109574in}}{\pgfqpoint{1.620345in}{3.106302in}}{\pgfqpoint{1.628581in}{3.106302in}}%
\pgfpathclose%
\pgfusepath{stroke,fill}%
\end{pgfscope}%
\begin{pgfscope}%
\pgfpathrectangle{\pgfqpoint{0.100000in}{0.220728in}}{\pgfqpoint{3.696000in}{3.696000in}}%
\pgfusepath{clip}%
\pgfsetbuttcap%
\pgfsetroundjoin%
\definecolor{currentfill}{rgb}{0.121569,0.466667,0.705882}%
\pgfsetfillcolor{currentfill}%
\pgfsetfillopacity{0.301106}%
\pgfsetlinewidth{1.003750pt}%
\definecolor{currentstroke}{rgb}{0.121569,0.466667,0.705882}%
\pgfsetstrokecolor{currentstroke}%
\pgfsetstrokeopacity{0.301106}%
\pgfsetdash{}{0pt}%
\pgfpathmoveto{\pgfqpoint{1.628574in}{3.106286in}}%
\pgfpathcurveto{\pgfqpoint{1.636810in}{3.106286in}}{\pgfqpoint{1.644710in}{3.109559in}}{\pgfqpoint{1.650534in}{3.115382in}}%
\pgfpathcurveto{\pgfqpoint{1.656358in}{3.121206in}}{\pgfqpoint{1.659630in}{3.129106in}}{\pgfqpoint{1.659630in}{3.137343in}}%
\pgfpathcurveto{\pgfqpoint{1.659630in}{3.145579in}}{\pgfqpoint{1.656358in}{3.153479in}}{\pgfqpoint{1.650534in}{3.159303in}}%
\pgfpathcurveto{\pgfqpoint{1.644710in}{3.165127in}}{\pgfqpoint{1.636810in}{3.168399in}}{\pgfqpoint{1.628574in}{3.168399in}}%
\pgfpathcurveto{\pgfqpoint{1.620338in}{3.168399in}}{\pgfqpoint{1.612438in}{3.165127in}}{\pgfqpoint{1.606614in}{3.159303in}}%
\pgfpathcurveto{\pgfqpoint{1.600790in}{3.153479in}}{\pgfqpoint{1.597517in}{3.145579in}}{\pgfqpoint{1.597517in}{3.137343in}}%
\pgfpathcurveto{\pgfqpoint{1.597517in}{3.129106in}}{\pgfqpoint{1.600790in}{3.121206in}}{\pgfqpoint{1.606614in}{3.115382in}}%
\pgfpathcurveto{\pgfqpoint{1.612438in}{3.109559in}}{\pgfqpoint{1.620338in}{3.106286in}}{\pgfqpoint{1.628574in}{3.106286in}}%
\pgfpathclose%
\pgfusepath{stroke,fill}%
\end{pgfscope}%
\begin{pgfscope}%
\pgfpathrectangle{\pgfqpoint{0.100000in}{0.220728in}}{\pgfqpoint{3.696000in}{3.696000in}}%
\pgfusepath{clip}%
\pgfsetbuttcap%
\pgfsetroundjoin%
\definecolor{currentfill}{rgb}{0.121569,0.466667,0.705882}%
\pgfsetfillcolor{currentfill}%
\pgfsetfillopacity{0.301110}%
\pgfsetlinewidth{1.003750pt}%
\definecolor{currentstroke}{rgb}{0.121569,0.466667,0.705882}%
\pgfsetstrokecolor{currentstroke}%
\pgfsetstrokeopacity{0.301110}%
\pgfsetdash{}{0pt}%
\pgfpathmoveto{\pgfqpoint{1.628560in}{3.106258in}}%
\pgfpathcurveto{\pgfqpoint{1.636797in}{3.106258in}}{\pgfqpoint{1.644697in}{3.109530in}}{\pgfqpoint{1.650521in}{3.115354in}}%
\pgfpathcurveto{\pgfqpoint{1.656345in}{3.121178in}}{\pgfqpoint{1.659617in}{3.129078in}}{\pgfqpoint{1.659617in}{3.137314in}}%
\pgfpathcurveto{\pgfqpoint{1.659617in}{3.145551in}}{\pgfqpoint{1.656345in}{3.153451in}}{\pgfqpoint{1.650521in}{3.159275in}}%
\pgfpathcurveto{\pgfqpoint{1.644697in}{3.165099in}}{\pgfqpoint{1.636797in}{3.168371in}}{\pgfqpoint{1.628560in}{3.168371in}}%
\pgfpathcurveto{\pgfqpoint{1.620324in}{3.168371in}}{\pgfqpoint{1.612424in}{3.165099in}}{\pgfqpoint{1.606600in}{3.159275in}}%
\pgfpathcurveto{\pgfqpoint{1.600776in}{3.153451in}}{\pgfqpoint{1.597504in}{3.145551in}}{\pgfqpoint{1.597504in}{3.137314in}}%
\pgfpathcurveto{\pgfqpoint{1.597504in}{3.129078in}}{\pgfqpoint{1.600776in}{3.121178in}}{\pgfqpoint{1.606600in}{3.115354in}}%
\pgfpathcurveto{\pgfqpoint{1.612424in}{3.109530in}}{\pgfqpoint{1.620324in}{3.106258in}}{\pgfqpoint{1.628560in}{3.106258in}}%
\pgfpathclose%
\pgfusepath{stroke,fill}%
\end{pgfscope}%
\begin{pgfscope}%
\pgfpathrectangle{\pgfqpoint{0.100000in}{0.220728in}}{\pgfqpoint{3.696000in}{3.696000in}}%
\pgfusepath{clip}%
\pgfsetbuttcap%
\pgfsetroundjoin%
\definecolor{currentfill}{rgb}{0.121569,0.466667,0.705882}%
\pgfsetfillcolor{currentfill}%
\pgfsetfillopacity{0.301118}%
\pgfsetlinewidth{1.003750pt}%
\definecolor{currentstroke}{rgb}{0.121569,0.466667,0.705882}%
\pgfsetstrokecolor{currentstroke}%
\pgfsetstrokeopacity{0.301118}%
\pgfsetdash{}{0pt}%
\pgfpathmoveto{\pgfqpoint{1.628537in}{3.106207in}}%
\pgfpathcurveto{\pgfqpoint{1.636773in}{3.106207in}}{\pgfqpoint{1.644673in}{3.109480in}}{\pgfqpoint{1.650497in}{3.115303in}}%
\pgfpathcurveto{\pgfqpoint{1.656321in}{3.121127in}}{\pgfqpoint{1.659593in}{3.129027in}}{\pgfqpoint{1.659593in}{3.137264in}}%
\pgfpathcurveto{\pgfqpoint{1.659593in}{3.145500in}}{\pgfqpoint{1.656321in}{3.153400in}}{\pgfqpoint{1.650497in}{3.159224in}}%
\pgfpathcurveto{\pgfqpoint{1.644673in}{3.165048in}}{\pgfqpoint{1.636773in}{3.168320in}}{\pgfqpoint{1.628537in}{3.168320in}}%
\pgfpathcurveto{\pgfqpoint{1.620300in}{3.168320in}}{\pgfqpoint{1.612400in}{3.165048in}}{\pgfqpoint{1.606577in}{3.159224in}}%
\pgfpathcurveto{\pgfqpoint{1.600753in}{3.153400in}}{\pgfqpoint{1.597480in}{3.145500in}}{\pgfqpoint{1.597480in}{3.137264in}}%
\pgfpathcurveto{\pgfqpoint{1.597480in}{3.129027in}}{\pgfqpoint{1.600753in}{3.121127in}}{\pgfqpoint{1.606577in}{3.115303in}}%
\pgfpathcurveto{\pgfqpoint{1.612400in}{3.109480in}}{\pgfqpoint{1.620300in}{3.106207in}}{\pgfqpoint{1.628537in}{3.106207in}}%
\pgfpathclose%
\pgfusepath{stroke,fill}%
\end{pgfscope}%
\begin{pgfscope}%
\pgfpathrectangle{\pgfqpoint{0.100000in}{0.220728in}}{\pgfqpoint{3.696000in}{3.696000in}}%
\pgfusepath{clip}%
\pgfsetbuttcap%
\pgfsetroundjoin%
\definecolor{currentfill}{rgb}{0.121569,0.466667,0.705882}%
\pgfsetfillcolor{currentfill}%
\pgfsetfillopacity{0.301131}%
\pgfsetlinewidth{1.003750pt}%
\definecolor{currentstroke}{rgb}{0.121569,0.466667,0.705882}%
\pgfsetstrokecolor{currentstroke}%
\pgfsetstrokeopacity{0.301131}%
\pgfsetdash{}{0pt}%
\pgfpathmoveto{\pgfqpoint{1.628492in}{3.106112in}}%
\pgfpathcurveto{\pgfqpoint{1.636729in}{3.106112in}}{\pgfqpoint{1.644629in}{3.109384in}}{\pgfqpoint{1.650453in}{3.115208in}}%
\pgfpathcurveto{\pgfqpoint{1.656276in}{3.121032in}}{\pgfqpoint{1.659549in}{3.128932in}}{\pgfqpoint{1.659549in}{3.137168in}}%
\pgfpathcurveto{\pgfqpoint{1.659549in}{3.145405in}}{\pgfqpoint{1.656276in}{3.153305in}}{\pgfqpoint{1.650453in}{3.159129in}}%
\pgfpathcurveto{\pgfqpoint{1.644629in}{3.164953in}}{\pgfqpoint{1.636729in}{3.168225in}}{\pgfqpoint{1.628492in}{3.168225in}}%
\pgfpathcurveto{\pgfqpoint{1.620256in}{3.168225in}}{\pgfqpoint{1.612356in}{3.164953in}}{\pgfqpoint{1.606532in}{3.159129in}}%
\pgfpathcurveto{\pgfqpoint{1.600708in}{3.153305in}}{\pgfqpoint{1.597436in}{3.145405in}}{\pgfqpoint{1.597436in}{3.137168in}}%
\pgfpathcurveto{\pgfqpoint{1.597436in}{3.128932in}}{\pgfqpoint{1.600708in}{3.121032in}}{\pgfqpoint{1.606532in}{3.115208in}}%
\pgfpathcurveto{\pgfqpoint{1.612356in}{3.109384in}}{\pgfqpoint{1.620256in}{3.106112in}}{\pgfqpoint{1.628492in}{3.106112in}}%
\pgfpathclose%
\pgfusepath{stroke,fill}%
\end{pgfscope}%
\begin{pgfscope}%
\pgfpathrectangle{\pgfqpoint{0.100000in}{0.220728in}}{\pgfqpoint{3.696000in}{3.696000in}}%
\pgfusepath{clip}%
\pgfsetbuttcap%
\pgfsetroundjoin%
\definecolor{currentfill}{rgb}{0.121569,0.466667,0.705882}%
\pgfsetfillcolor{currentfill}%
\pgfsetfillopacity{0.301157}%
\pgfsetlinewidth{1.003750pt}%
\definecolor{currentstroke}{rgb}{0.121569,0.466667,0.705882}%
\pgfsetstrokecolor{currentstroke}%
\pgfsetstrokeopacity{0.301157}%
\pgfsetdash{}{0pt}%
\pgfpathmoveto{\pgfqpoint{1.628412in}{3.105945in}}%
\pgfpathcurveto{\pgfqpoint{1.636649in}{3.105945in}}{\pgfqpoint{1.644549in}{3.109218in}}{\pgfqpoint{1.650373in}{3.115041in}}%
\pgfpathcurveto{\pgfqpoint{1.656197in}{3.120865in}}{\pgfqpoint{1.659469in}{3.128765in}}{\pgfqpoint{1.659469in}{3.137002in}}%
\pgfpathcurveto{\pgfqpoint{1.659469in}{3.145238in}}{\pgfqpoint{1.656197in}{3.153138in}}{\pgfqpoint{1.650373in}{3.158962in}}%
\pgfpathcurveto{\pgfqpoint{1.644549in}{3.164786in}}{\pgfqpoint{1.636649in}{3.168058in}}{\pgfqpoint{1.628412in}{3.168058in}}%
\pgfpathcurveto{\pgfqpoint{1.620176in}{3.168058in}}{\pgfqpoint{1.612276in}{3.164786in}}{\pgfqpoint{1.606452in}{3.158962in}}%
\pgfpathcurveto{\pgfqpoint{1.600628in}{3.153138in}}{\pgfqpoint{1.597356in}{3.145238in}}{\pgfqpoint{1.597356in}{3.137002in}}%
\pgfpathcurveto{\pgfqpoint{1.597356in}{3.128765in}}{\pgfqpoint{1.600628in}{3.120865in}}{\pgfqpoint{1.606452in}{3.115041in}}%
\pgfpathcurveto{\pgfqpoint{1.612276in}{3.109218in}}{\pgfqpoint{1.620176in}{3.105945in}}{\pgfqpoint{1.628412in}{3.105945in}}%
\pgfpathclose%
\pgfusepath{stroke,fill}%
\end{pgfscope}%
\begin{pgfscope}%
\pgfpathrectangle{\pgfqpoint{0.100000in}{0.220728in}}{\pgfqpoint{3.696000in}{3.696000in}}%
\pgfusepath{clip}%
\pgfsetbuttcap%
\pgfsetroundjoin%
\definecolor{currentfill}{rgb}{0.121569,0.466667,0.705882}%
\pgfsetfillcolor{currentfill}%
\pgfsetfillopacity{0.301204}%
\pgfsetlinewidth{1.003750pt}%
\definecolor{currentstroke}{rgb}{0.121569,0.466667,0.705882}%
\pgfsetstrokecolor{currentstroke}%
\pgfsetstrokeopacity{0.301204}%
\pgfsetdash{}{0pt}%
\pgfpathmoveto{\pgfqpoint{1.628272in}{3.105638in}}%
\pgfpathcurveto{\pgfqpoint{1.636509in}{3.105638in}}{\pgfqpoint{1.644409in}{3.108911in}}{\pgfqpoint{1.650233in}{3.114735in}}%
\pgfpathcurveto{\pgfqpoint{1.656056in}{3.120558in}}{\pgfqpoint{1.659329in}{3.128458in}}{\pgfqpoint{1.659329in}{3.136695in}}%
\pgfpathcurveto{\pgfqpoint{1.659329in}{3.144931in}}{\pgfqpoint{1.656056in}{3.152831in}}{\pgfqpoint{1.650233in}{3.158655in}}%
\pgfpathcurveto{\pgfqpoint{1.644409in}{3.164479in}}{\pgfqpoint{1.636509in}{3.167751in}}{\pgfqpoint{1.628272in}{3.167751in}}%
\pgfpathcurveto{\pgfqpoint{1.620036in}{3.167751in}}{\pgfqpoint{1.612136in}{3.164479in}}{\pgfqpoint{1.606312in}{3.158655in}}%
\pgfpathcurveto{\pgfqpoint{1.600488in}{3.152831in}}{\pgfqpoint{1.597216in}{3.144931in}}{\pgfqpoint{1.597216in}{3.136695in}}%
\pgfpathcurveto{\pgfqpoint{1.597216in}{3.128458in}}{\pgfqpoint{1.600488in}{3.120558in}}{\pgfqpoint{1.606312in}{3.114735in}}%
\pgfpathcurveto{\pgfqpoint{1.612136in}{3.108911in}}{\pgfqpoint{1.620036in}{3.105638in}}{\pgfqpoint{1.628272in}{3.105638in}}%
\pgfpathclose%
\pgfusepath{stroke,fill}%
\end{pgfscope}%
\begin{pgfscope}%
\pgfpathrectangle{\pgfqpoint{0.100000in}{0.220728in}}{\pgfqpoint{3.696000in}{3.696000in}}%
\pgfusepath{clip}%
\pgfsetbuttcap%
\pgfsetroundjoin%
\definecolor{currentfill}{rgb}{0.121569,0.466667,0.705882}%
\pgfsetfillcolor{currentfill}%
\pgfsetfillopacity{0.301293}%
\pgfsetlinewidth{1.003750pt}%
\definecolor{currentstroke}{rgb}{0.121569,0.466667,0.705882}%
\pgfsetstrokecolor{currentstroke}%
\pgfsetstrokeopacity{0.301293}%
\pgfsetdash{}{0pt}%
\pgfpathmoveto{\pgfqpoint{1.628022in}{3.105096in}}%
\pgfpathcurveto{\pgfqpoint{1.636259in}{3.105096in}}{\pgfqpoint{1.644159in}{3.108368in}}{\pgfqpoint{1.649983in}{3.114192in}}%
\pgfpathcurveto{\pgfqpoint{1.655807in}{3.120016in}}{\pgfqpoint{1.659079in}{3.127916in}}{\pgfqpoint{1.659079in}{3.136152in}}%
\pgfpathcurveto{\pgfqpoint{1.659079in}{3.144388in}}{\pgfqpoint{1.655807in}{3.152288in}}{\pgfqpoint{1.649983in}{3.158112in}}%
\pgfpathcurveto{\pgfqpoint{1.644159in}{3.163936in}}{\pgfqpoint{1.636259in}{3.167209in}}{\pgfqpoint{1.628022in}{3.167209in}}%
\pgfpathcurveto{\pgfqpoint{1.619786in}{3.167209in}}{\pgfqpoint{1.611886in}{3.163936in}}{\pgfqpoint{1.606062in}{3.158112in}}%
\pgfpathcurveto{\pgfqpoint{1.600238in}{3.152288in}}{\pgfqpoint{1.596966in}{3.144388in}}{\pgfqpoint{1.596966in}{3.136152in}}%
\pgfpathcurveto{\pgfqpoint{1.596966in}{3.127916in}}{\pgfqpoint{1.600238in}{3.120016in}}{\pgfqpoint{1.606062in}{3.114192in}}%
\pgfpathcurveto{\pgfqpoint{1.611886in}{3.108368in}}{\pgfqpoint{1.619786in}{3.105096in}}{\pgfqpoint{1.628022in}{3.105096in}}%
\pgfpathclose%
\pgfusepath{stroke,fill}%
\end{pgfscope}%
\begin{pgfscope}%
\pgfpathrectangle{\pgfqpoint{0.100000in}{0.220728in}}{\pgfqpoint{3.696000in}{3.696000in}}%
\pgfusepath{clip}%
\pgfsetbuttcap%
\pgfsetroundjoin%
\definecolor{currentfill}{rgb}{0.121569,0.466667,0.705882}%
\pgfsetfillcolor{currentfill}%
\pgfsetfillopacity{0.301455}%
\pgfsetlinewidth{1.003750pt}%
\definecolor{currentstroke}{rgb}{0.121569,0.466667,0.705882}%
\pgfsetstrokecolor{currentstroke}%
\pgfsetstrokeopacity{0.301455}%
\pgfsetdash{}{0pt}%
\pgfpathmoveto{\pgfqpoint{1.627582in}{3.104092in}}%
\pgfpathcurveto{\pgfqpoint{1.635818in}{3.104092in}}{\pgfqpoint{1.643718in}{3.107365in}}{\pgfqpoint{1.649542in}{3.113189in}}%
\pgfpathcurveto{\pgfqpoint{1.655366in}{3.119013in}}{\pgfqpoint{1.658638in}{3.126913in}}{\pgfqpoint{1.658638in}{3.135149in}}%
\pgfpathcurveto{\pgfqpoint{1.658638in}{3.143385in}}{\pgfqpoint{1.655366in}{3.151285in}}{\pgfqpoint{1.649542in}{3.157109in}}%
\pgfpathcurveto{\pgfqpoint{1.643718in}{3.162933in}}{\pgfqpoint{1.635818in}{3.166205in}}{\pgfqpoint{1.627582in}{3.166205in}}%
\pgfpathcurveto{\pgfqpoint{1.619345in}{3.166205in}}{\pgfqpoint{1.611445in}{3.162933in}}{\pgfqpoint{1.605621in}{3.157109in}}%
\pgfpathcurveto{\pgfqpoint{1.599797in}{3.151285in}}{\pgfqpoint{1.596525in}{3.143385in}}{\pgfqpoint{1.596525in}{3.135149in}}%
\pgfpathcurveto{\pgfqpoint{1.596525in}{3.126913in}}{\pgfqpoint{1.599797in}{3.119013in}}{\pgfqpoint{1.605621in}{3.113189in}}%
\pgfpathcurveto{\pgfqpoint{1.611445in}{3.107365in}}{\pgfqpoint{1.619345in}{3.104092in}}{\pgfqpoint{1.627582in}{3.104092in}}%
\pgfpathclose%
\pgfusepath{stroke,fill}%
\end{pgfscope}%
\begin{pgfscope}%
\pgfpathrectangle{\pgfqpoint{0.100000in}{0.220728in}}{\pgfqpoint{3.696000in}{3.696000in}}%
\pgfusepath{clip}%
\pgfsetbuttcap%
\pgfsetroundjoin%
\definecolor{currentfill}{rgb}{0.121569,0.466667,0.705882}%
\pgfsetfillcolor{currentfill}%
\pgfsetfillopacity{0.301756}%
\pgfsetlinewidth{1.003750pt}%
\definecolor{currentstroke}{rgb}{0.121569,0.466667,0.705882}%
\pgfsetstrokecolor{currentstroke}%
\pgfsetstrokeopacity{0.301756}%
\pgfsetdash{}{0pt}%
\pgfpathmoveto{\pgfqpoint{1.626774in}{3.102312in}}%
\pgfpathcurveto{\pgfqpoint{1.635010in}{3.102312in}}{\pgfqpoint{1.642910in}{3.105585in}}{\pgfqpoint{1.648734in}{3.111409in}}%
\pgfpathcurveto{\pgfqpoint{1.654558in}{3.117233in}}{\pgfqpoint{1.657830in}{3.125133in}}{\pgfqpoint{1.657830in}{3.133369in}}%
\pgfpathcurveto{\pgfqpoint{1.657830in}{3.141605in}}{\pgfqpoint{1.654558in}{3.149505in}}{\pgfqpoint{1.648734in}{3.155329in}}%
\pgfpathcurveto{\pgfqpoint{1.642910in}{3.161153in}}{\pgfqpoint{1.635010in}{3.164425in}}{\pgfqpoint{1.626774in}{3.164425in}}%
\pgfpathcurveto{\pgfqpoint{1.618537in}{3.164425in}}{\pgfqpoint{1.610637in}{3.161153in}}{\pgfqpoint{1.604813in}{3.155329in}}%
\pgfpathcurveto{\pgfqpoint{1.598990in}{3.149505in}}{\pgfqpoint{1.595717in}{3.141605in}}{\pgfqpoint{1.595717in}{3.133369in}}%
\pgfpathcurveto{\pgfqpoint{1.595717in}{3.125133in}}{\pgfqpoint{1.598990in}{3.117233in}}{\pgfqpoint{1.604813in}{3.111409in}}%
\pgfpathcurveto{\pgfqpoint{1.610637in}{3.105585in}}{\pgfqpoint{1.618537in}{3.102312in}}{\pgfqpoint{1.626774in}{3.102312in}}%
\pgfpathclose%
\pgfusepath{stroke,fill}%
\end{pgfscope}%
\begin{pgfscope}%
\pgfpathrectangle{\pgfqpoint{0.100000in}{0.220728in}}{\pgfqpoint{3.696000in}{3.696000in}}%
\pgfusepath{clip}%
\pgfsetbuttcap%
\pgfsetroundjoin%
\definecolor{currentfill}{rgb}{0.121569,0.466667,0.705882}%
\pgfsetfillcolor{currentfill}%
\pgfsetfillopacity{0.302322}%
\pgfsetlinewidth{1.003750pt}%
\definecolor{currentstroke}{rgb}{0.121569,0.466667,0.705882}%
\pgfsetstrokecolor{currentstroke}%
\pgfsetstrokeopacity{0.302322}%
\pgfsetdash{}{0pt}%
\pgfpathmoveto{\pgfqpoint{1.625340in}{3.099139in}}%
\pgfpathcurveto{\pgfqpoint{1.633576in}{3.099139in}}{\pgfqpoint{1.641476in}{3.102411in}}{\pgfqpoint{1.647300in}{3.108235in}}%
\pgfpathcurveto{\pgfqpoint{1.653124in}{3.114059in}}{\pgfqpoint{1.656396in}{3.121959in}}{\pgfqpoint{1.656396in}{3.130195in}}%
\pgfpathcurveto{\pgfqpoint{1.656396in}{3.138431in}}{\pgfqpoint{1.653124in}{3.146332in}}{\pgfqpoint{1.647300in}{3.152155in}}%
\pgfpathcurveto{\pgfqpoint{1.641476in}{3.157979in}}{\pgfqpoint{1.633576in}{3.161252in}}{\pgfqpoint{1.625340in}{3.161252in}}%
\pgfpathcurveto{\pgfqpoint{1.617104in}{3.161252in}}{\pgfqpoint{1.609204in}{3.157979in}}{\pgfqpoint{1.603380in}{3.152155in}}%
\pgfpathcurveto{\pgfqpoint{1.597556in}{3.146332in}}{\pgfqpoint{1.594283in}{3.138431in}}{\pgfqpoint{1.594283in}{3.130195in}}%
\pgfpathcurveto{\pgfqpoint{1.594283in}{3.121959in}}{\pgfqpoint{1.597556in}{3.114059in}}{\pgfqpoint{1.603380in}{3.108235in}}%
\pgfpathcurveto{\pgfqpoint{1.609204in}{3.102411in}}{\pgfqpoint{1.617104in}{3.099139in}}{\pgfqpoint{1.625340in}{3.099139in}}%
\pgfpathclose%
\pgfusepath{stroke,fill}%
\end{pgfscope}%
\begin{pgfscope}%
\pgfpathrectangle{\pgfqpoint{0.100000in}{0.220728in}}{\pgfqpoint{3.696000in}{3.696000in}}%
\pgfusepath{clip}%
\pgfsetbuttcap%
\pgfsetroundjoin%
\definecolor{currentfill}{rgb}{0.121569,0.466667,0.705882}%
\pgfsetfillcolor{currentfill}%
\pgfsetfillopacity{0.302523}%
\pgfsetlinewidth{1.003750pt}%
\definecolor{currentstroke}{rgb}{0.121569,0.466667,0.705882}%
\pgfsetstrokecolor{currentstroke}%
\pgfsetstrokeopacity{0.302523}%
\pgfsetdash{}{0pt}%
\pgfpathmoveto{\pgfqpoint{1.681683in}{3.133624in}}%
\pgfpathcurveto{\pgfqpoint{1.689920in}{3.133624in}}{\pgfqpoint{1.697820in}{3.136897in}}{\pgfqpoint{1.703644in}{3.142720in}}%
\pgfpathcurveto{\pgfqpoint{1.709468in}{3.148544in}}{\pgfqpoint{1.712740in}{3.156444in}}{\pgfqpoint{1.712740in}{3.164681in}}%
\pgfpathcurveto{\pgfqpoint{1.712740in}{3.172917in}}{\pgfqpoint{1.709468in}{3.180817in}}{\pgfqpoint{1.703644in}{3.186641in}}%
\pgfpathcurveto{\pgfqpoint{1.697820in}{3.192465in}}{\pgfqpoint{1.689920in}{3.195737in}}{\pgfqpoint{1.681683in}{3.195737in}}%
\pgfpathcurveto{\pgfqpoint{1.673447in}{3.195737in}}{\pgfqpoint{1.665547in}{3.192465in}}{\pgfqpoint{1.659723in}{3.186641in}}%
\pgfpathcurveto{\pgfqpoint{1.653899in}{3.180817in}}{\pgfqpoint{1.650627in}{3.172917in}}{\pgfqpoint{1.650627in}{3.164681in}}%
\pgfpathcurveto{\pgfqpoint{1.650627in}{3.156444in}}{\pgfqpoint{1.653899in}{3.148544in}}{\pgfqpoint{1.659723in}{3.142720in}}%
\pgfpathcurveto{\pgfqpoint{1.665547in}{3.136897in}}{\pgfqpoint{1.673447in}{3.133624in}}{\pgfqpoint{1.681683in}{3.133624in}}%
\pgfpathclose%
\pgfusepath{stroke,fill}%
\end{pgfscope}%
\begin{pgfscope}%
\pgfpathrectangle{\pgfqpoint{0.100000in}{0.220728in}}{\pgfqpoint{3.696000in}{3.696000in}}%
\pgfusepath{clip}%
\pgfsetbuttcap%
\pgfsetroundjoin%
\definecolor{currentfill}{rgb}{0.121569,0.466667,0.705882}%
\pgfsetfillcolor{currentfill}%
\pgfsetfillopacity{0.303210}%
\pgfsetlinewidth{1.003750pt}%
\definecolor{currentstroke}{rgb}{0.121569,0.466667,0.705882}%
\pgfsetstrokecolor{currentstroke}%
\pgfsetstrokeopacity{0.303210}%
\pgfsetdash{}{0pt}%
\pgfpathmoveto{\pgfqpoint{1.622037in}{3.093372in}}%
\pgfpathcurveto{\pgfqpoint{1.630273in}{3.093372in}}{\pgfqpoint{1.638173in}{3.096644in}}{\pgfqpoint{1.643997in}{3.102468in}}%
\pgfpathcurveto{\pgfqpoint{1.649821in}{3.108292in}}{\pgfqpoint{1.653093in}{3.116192in}}{\pgfqpoint{1.653093in}{3.124428in}}%
\pgfpathcurveto{\pgfqpoint{1.653093in}{3.132664in}}{\pgfqpoint{1.649821in}{3.140565in}}{\pgfqpoint{1.643997in}{3.146388in}}%
\pgfpathcurveto{\pgfqpoint{1.638173in}{3.152212in}}{\pgfqpoint{1.630273in}{3.155485in}}{\pgfqpoint{1.622037in}{3.155485in}}%
\pgfpathcurveto{\pgfqpoint{1.613801in}{3.155485in}}{\pgfqpoint{1.605901in}{3.152212in}}{\pgfqpoint{1.600077in}{3.146388in}}%
\pgfpathcurveto{\pgfqpoint{1.594253in}{3.140565in}}{\pgfqpoint{1.590980in}{3.132664in}}{\pgfqpoint{1.590980in}{3.124428in}}%
\pgfpathcurveto{\pgfqpoint{1.590980in}{3.116192in}}{\pgfqpoint{1.594253in}{3.108292in}}{\pgfqpoint{1.600077in}{3.102468in}}%
\pgfpathcurveto{\pgfqpoint{1.605901in}{3.096644in}}{\pgfqpoint{1.613801in}{3.093372in}}{\pgfqpoint{1.622037in}{3.093372in}}%
\pgfpathclose%
\pgfusepath{stroke,fill}%
\end{pgfscope}%
\begin{pgfscope}%
\pgfpathrectangle{\pgfqpoint{0.100000in}{0.220728in}}{\pgfqpoint{3.696000in}{3.696000in}}%
\pgfusepath{clip}%
\pgfsetbuttcap%
\pgfsetroundjoin%
\definecolor{currentfill}{rgb}{0.121569,0.466667,0.705882}%
\pgfsetfillcolor{currentfill}%
\pgfsetfillopacity{0.304986}%
\pgfsetlinewidth{1.003750pt}%
\definecolor{currentstroke}{rgb}{0.121569,0.466667,0.705882}%
\pgfsetstrokecolor{currentstroke}%
\pgfsetstrokeopacity{0.304986}%
\pgfsetdash{}{0pt}%
\pgfpathmoveto{\pgfqpoint{1.616681in}{3.082966in}}%
\pgfpathcurveto{\pgfqpoint{1.624917in}{3.082966in}}{\pgfqpoint{1.632817in}{3.086238in}}{\pgfqpoint{1.638641in}{3.092062in}}%
\pgfpathcurveto{\pgfqpoint{1.644465in}{3.097886in}}{\pgfqpoint{1.647738in}{3.105786in}}{\pgfqpoint{1.647738in}{3.114023in}}%
\pgfpathcurveto{\pgfqpoint{1.647738in}{3.122259in}}{\pgfqpoint{1.644465in}{3.130159in}}{\pgfqpoint{1.638641in}{3.135983in}}%
\pgfpathcurveto{\pgfqpoint{1.632817in}{3.141807in}}{\pgfqpoint{1.624917in}{3.145079in}}{\pgfqpoint{1.616681in}{3.145079in}}%
\pgfpathcurveto{\pgfqpoint{1.608445in}{3.145079in}}{\pgfqpoint{1.600545in}{3.141807in}}{\pgfqpoint{1.594721in}{3.135983in}}%
\pgfpathcurveto{\pgfqpoint{1.588897in}{3.130159in}}{\pgfqpoint{1.585625in}{3.122259in}}{\pgfqpoint{1.585625in}{3.114023in}}%
\pgfpathcurveto{\pgfqpoint{1.585625in}{3.105786in}}{\pgfqpoint{1.588897in}{3.097886in}}{\pgfqpoint{1.594721in}{3.092062in}}%
\pgfpathcurveto{\pgfqpoint{1.600545in}{3.086238in}}{\pgfqpoint{1.608445in}{3.082966in}}{\pgfqpoint{1.616681in}{3.082966in}}%
\pgfpathclose%
\pgfusepath{stroke,fill}%
\end{pgfscope}%
\begin{pgfscope}%
\pgfpathrectangle{\pgfqpoint{0.100000in}{0.220728in}}{\pgfqpoint{3.696000in}{3.696000in}}%
\pgfusepath{clip}%
\pgfsetbuttcap%
\pgfsetroundjoin%
\definecolor{currentfill}{rgb}{0.121569,0.466667,0.705882}%
\pgfsetfillcolor{currentfill}%
\pgfsetfillopacity{0.306192}%
\pgfsetlinewidth{1.003750pt}%
\definecolor{currentstroke}{rgb}{0.121569,0.466667,0.705882}%
\pgfsetstrokecolor{currentstroke}%
\pgfsetstrokeopacity{0.306192}%
\pgfsetdash{}{0pt}%
\pgfpathmoveto{\pgfqpoint{1.698646in}{3.134050in}}%
\pgfpathcurveto{\pgfqpoint{1.706883in}{3.134050in}}{\pgfqpoint{1.714783in}{3.137323in}}{\pgfqpoint{1.720607in}{3.143147in}}%
\pgfpathcurveto{\pgfqpoint{1.726430in}{3.148970in}}{\pgfqpoint{1.729703in}{3.156871in}}{\pgfqpoint{1.729703in}{3.165107in}}%
\pgfpathcurveto{\pgfqpoint{1.729703in}{3.173343in}}{\pgfqpoint{1.726430in}{3.181243in}}{\pgfqpoint{1.720607in}{3.187067in}}%
\pgfpathcurveto{\pgfqpoint{1.714783in}{3.192891in}}{\pgfqpoint{1.706883in}{3.196163in}}{\pgfqpoint{1.698646in}{3.196163in}}%
\pgfpathcurveto{\pgfqpoint{1.690410in}{3.196163in}}{\pgfqpoint{1.682510in}{3.192891in}}{\pgfqpoint{1.676686in}{3.187067in}}%
\pgfpathcurveto{\pgfqpoint{1.670862in}{3.181243in}}{\pgfqpoint{1.667590in}{3.173343in}}{\pgfqpoint{1.667590in}{3.165107in}}%
\pgfpathcurveto{\pgfqpoint{1.667590in}{3.156871in}}{\pgfqpoint{1.670862in}{3.148970in}}{\pgfqpoint{1.676686in}{3.143147in}}%
\pgfpathcurveto{\pgfqpoint{1.682510in}{3.137323in}}{\pgfqpoint{1.690410in}{3.134050in}}{\pgfqpoint{1.698646in}{3.134050in}}%
\pgfpathclose%
\pgfusepath{stroke,fill}%
\end{pgfscope}%
\begin{pgfscope}%
\pgfpathrectangle{\pgfqpoint{0.100000in}{0.220728in}}{\pgfqpoint{3.696000in}{3.696000in}}%
\pgfusepath{clip}%
\pgfsetbuttcap%
\pgfsetroundjoin%
\definecolor{currentfill}{rgb}{0.121569,0.466667,0.705882}%
\pgfsetfillcolor{currentfill}%
\pgfsetfillopacity{0.308304}%
\pgfsetlinewidth{1.003750pt}%
\definecolor{currentstroke}{rgb}{0.121569,0.466667,0.705882}%
\pgfsetstrokecolor{currentstroke}%
\pgfsetstrokeopacity{0.308304}%
\pgfsetdash{}{0pt}%
\pgfpathmoveto{\pgfqpoint{1.607209in}{3.064171in}}%
\pgfpathcurveto{\pgfqpoint{1.615446in}{3.064171in}}{\pgfqpoint{1.623346in}{3.067443in}}{\pgfqpoint{1.629169in}{3.073267in}}%
\pgfpathcurveto{\pgfqpoint{1.634993in}{3.079091in}}{\pgfqpoint{1.638266in}{3.086991in}}{\pgfqpoint{1.638266in}{3.095228in}}%
\pgfpathcurveto{\pgfqpoint{1.638266in}{3.103464in}}{\pgfqpoint{1.634993in}{3.111364in}}{\pgfqpoint{1.629169in}{3.117188in}}%
\pgfpathcurveto{\pgfqpoint{1.623346in}{3.123012in}}{\pgfqpoint{1.615446in}{3.126284in}}{\pgfqpoint{1.607209in}{3.126284in}}%
\pgfpathcurveto{\pgfqpoint{1.598973in}{3.126284in}}{\pgfqpoint{1.591073in}{3.123012in}}{\pgfqpoint{1.585249in}{3.117188in}}%
\pgfpathcurveto{\pgfqpoint{1.579425in}{3.111364in}}{\pgfqpoint{1.576153in}{3.103464in}}{\pgfqpoint{1.576153in}{3.095228in}}%
\pgfpathcurveto{\pgfqpoint{1.576153in}{3.086991in}}{\pgfqpoint{1.579425in}{3.079091in}}{\pgfqpoint{1.585249in}{3.073267in}}%
\pgfpathcurveto{\pgfqpoint{1.591073in}{3.067443in}}{\pgfqpoint{1.598973in}{3.064171in}}{\pgfqpoint{1.607209in}{3.064171in}}%
\pgfpathclose%
\pgfusepath{stroke,fill}%
\end{pgfscope}%
\begin{pgfscope}%
\pgfpathrectangle{\pgfqpoint{0.100000in}{0.220728in}}{\pgfqpoint{3.696000in}{3.696000in}}%
\pgfusepath{clip}%
\pgfsetbuttcap%
\pgfsetroundjoin%
\definecolor{currentfill}{rgb}{0.121569,0.466667,0.705882}%
\pgfsetfillcolor{currentfill}%
\pgfsetfillopacity{0.309620}%
\pgfsetlinewidth{1.003750pt}%
\definecolor{currentstroke}{rgb}{0.121569,0.466667,0.705882}%
\pgfsetstrokecolor{currentstroke}%
\pgfsetstrokeopacity{0.309620}%
\pgfsetdash{}{0pt}%
\pgfpathmoveto{\pgfqpoint{1.718933in}{3.133042in}}%
\pgfpathcurveto{\pgfqpoint{1.727169in}{3.133042in}}{\pgfqpoint{1.735069in}{3.136315in}}{\pgfqpoint{1.740893in}{3.142138in}}%
\pgfpathcurveto{\pgfqpoint{1.746717in}{3.147962in}}{\pgfqpoint{1.749989in}{3.155862in}}{\pgfqpoint{1.749989in}{3.164099in}}%
\pgfpathcurveto{\pgfqpoint{1.749989in}{3.172335in}}{\pgfqpoint{1.746717in}{3.180235in}}{\pgfqpoint{1.740893in}{3.186059in}}%
\pgfpathcurveto{\pgfqpoint{1.735069in}{3.191883in}}{\pgfqpoint{1.727169in}{3.195155in}}{\pgfqpoint{1.718933in}{3.195155in}}%
\pgfpathcurveto{\pgfqpoint{1.710697in}{3.195155in}}{\pgfqpoint{1.702797in}{3.191883in}}{\pgfqpoint{1.696973in}{3.186059in}}%
\pgfpathcurveto{\pgfqpoint{1.691149in}{3.180235in}}{\pgfqpoint{1.687876in}{3.172335in}}{\pgfqpoint{1.687876in}{3.164099in}}%
\pgfpathcurveto{\pgfqpoint{1.687876in}{3.155862in}}{\pgfqpoint{1.691149in}{3.147962in}}{\pgfqpoint{1.696973in}{3.142138in}}%
\pgfpathcurveto{\pgfqpoint{1.702797in}{3.136315in}}{\pgfqpoint{1.710697in}{3.133042in}}{\pgfqpoint{1.718933in}{3.133042in}}%
\pgfpathclose%
\pgfusepath{stroke,fill}%
\end{pgfscope}%
\begin{pgfscope}%
\pgfpathrectangle{\pgfqpoint{0.100000in}{0.220728in}}{\pgfqpoint{3.696000in}{3.696000in}}%
\pgfusepath{clip}%
\pgfsetbuttcap%
\pgfsetroundjoin%
\definecolor{currentfill}{rgb}{0.121569,0.466667,0.705882}%
\pgfsetfillcolor{currentfill}%
\pgfsetfillopacity{0.312952}%
\pgfsetlinewidth{1.003750pt}%
\definecolor{currentstroke}{rgb}{0.121569,0.466667,0.705882}%
\pgfsetstrokecolor{currentstroke}%
\pgfsetstrokeopacity{0.312952}%
\pgfsetdash{}{0pt}%
\pgfpathmoveto{\pgfqpoint{1.743901in}{3.132974in}}%
\pgfpathcurveto{\pgfqpoint{1.752137in}{3.132974in}}{\pgfqpoint{1.760037in}{3.136246in}}{\pgfqpoint{1.765861in}{3.142070in}}%
\pgfpathcurveto{\pgfqpoint{1.771685in}{3.147894in}}{\pgfqpoint{1.774957in}{3.155794in}}{\pgfqpoint{1.774957in}{3.164030in}}%
\pgfpathcurveto{\pgfqpoint{1.774957in}{3.172266in}}{\pgfqpoint{1.771685in}{3.180167in}}{\pgfqpoint{1.765861in}{3.185990in}}%
\pgfpathcurveto{\pgfqpoint{1.760037in}{3.191814in}}{\pgfqpoint{1.752137in}{3.195087in}}{\pgfqpoint{1.743901in}{3.195087in}}%
\pgfpathcurveto{\pgfqpoint{1.735664in}{3.195087in}}{\pgfqpoint{1.727764in}{3.191814in}}{\pgfqpoint{1.721940in}{3.185990in}}%
\pgfpathcurveto{\pgfqpoint{1.716116in}{3.180167in}}{\pgfqpoint{1.712844in}{3.172266in}}{\pgfqpoint{1.712844in}{3.164030in}}%
\pgfpathcurveto{\pgfqpoint{1.712844in}{3.155794in}}{\pgfqpoint{1.716116in}{3.147894in}}{\pgfqpoint{1.721940in}{3.142070in}}%
\pgfpathcurveto{\pgfqpoint{1.727764in}{3.136246in}}{\pgfqpoint{1.735664in}{3.132974in}}{\pgfqpoint{1.743901in}{3.132974in}}%
\pgfpathclose%
\pgfusepath{stroke,fill}%
\end{pgfscope}%
\begin{pgfscope}%
\pgfpathrectangle{\pgfqpoint{0.100000in}{0.220728in}}{\pgfqpoint{3.696000in}{3.696000in}}%
\pgfusepath{clip}%
\pgfsetbuttcap%
\pgfsetroundjoin%
\definecolor{currentfill}{rgb}{0.121569,0.466667,0.705882}%
\pgfsetfillcolor{currentfill}%
\pgfsetfillopacity{0.313349}%
\pgfsetlinewidth{1.003750pt}%
\definecolor{currentstroke}{rgb}{0.121569,0.466667,0.705882}%
\pgfsetstrokecolor{currentstroke}%
\pgfsetstrokeopacity{0.313349}%
\pgfsetdash{}{0pt}%
\pgfpathmoveto{\pgfqpoint{1.586499in}{3.028855in}}%
\pgfpathcurveto{\pgfqpoint{1.594735in}{3.028855in}}{\pgfqpoint{1.602635in}{3.032127in}}{\pgfqpoint{1.608459in}{3.037951in}}%
\pgfpathcurveto{\pgfqpoint{1.614283in}{3.043775in}}{\pgfqpoint{1.617555in}{3.051675in}}{\pgfqpoint{1.617555in}{3.059912in}}%
\pgfpathcurveto{\pgfqpoint{1.617555in}{3.068148in}}{\pgfqpoint{1.614283in}{3.076048in}}{\pgfqpoint{1.608459in}{3.081872in}}%
\pgfpathcurveto{\pgfqpoint{1.602635in}{3.087696in}}{\pgfqpoint{1.594735in}{3.090968in}}{\pgfqpoint{1.586499in}{3.090968in}}%
\pgfpathcurveto{\pgfqpoint{1.578262in}{3.090968in}}{\pgfqpoint{1.570362in}{3.087696in}}{\pgfqpoint{1.564538in}{3.081872in}}%
\pgfpathcurveto{\pgfqpoint{1.558714in}{3.076048in}}{\pgfqpoint{1.555442in}{3.068148in}}{\pgfqpoint{1.555442in}{3.059912in}}%
\pgfpathcurveto{\pgfqpoint{1.555442in}{3.051675in}}{\pgfqpoint{1.558714in}{3.043775in}}{\pgfqpoint{1.564538in}{3.037951in}}%
\pgfpathcurveto{\pgfqpoint{1.570362in}{3.032127in}}{\pgfqpoint{1.578262in}{3.028855in}}{\pgfqpoint{1.586499in}{3.028855in}}%
\pgfpathclose%
\pgfusepath{stroke,fill}%
\end{pgfscope}%
\begin{pgfscope}%
\pgfpathrectangle{\pgfqpoint{0.100000in}{0.220728in}}{\pgfqpoint{3.696000in}{3.696000in}}%
\pgfusepath{clip}%
\pgfsetbuttcap%
\pgfsetroundjoin%
\definecolor{currentfill}{rgb}{0.121569,0.466667,0.705882}%
\pgfsetfillcolor{currentfill}%
\pgfsetfillopacity{0.314573}%
\pgfsetlinewidth{1.003750pt}%
\definecolor{currentstroke}{rgb}{0.121569,0.466667,0.705882}%
\pgfsetstrokecolor{currentstroke}%
\pgfsetstrokeopacity{0.314573}%
\pgfsetdash{}{0pt}%
\pgfpathmoveto{\pgfqpoint{1.757613in}{3.131612in}}%
\pgfpathcurveto{\pgfqpoint{1.765849in}{3.131612in}}{\pgfqpoint{1.773749in}{3.134885in}}{\pgfqpoint{1.779573in}{3.140709in}}%
\pgfpathcurveto{\pgfqpoint{1.785397in}{3.146533in}}{\pgfqpoint{1.788669in}{3.154433in}}{\pgfqpoint{1.788669in}{3.162669in}}%
\pgfpathcurveto{\pgfqpoint{1.788669in}{3.170905in}}{\pgfqpoint{1.785397in}{3.178805in}}{\pgfqpoint{1.779573in}{3.184629in}}%
\pgfpathcurveto{\pgfqpoint{1.773749in}{3.190453in}}{\pgfqpoint{1.765849in}{3.193725in}}{\pgfqpoint{1.757613in}{3.193725in}}%
\pgfpathcurveto{\pgfqpoint{1.749377in}{3.193725in}}{\pgfqpoint{1.741477in}{3.190453in}}{\pgfqpoint{1.735653in}{3.184629in}}%
\pgfpathcurveto{\pgfqpoint{1.729829in}{3.178805in}}{\pgfqpoint{1.726556in}{3.170905in}}{\pgfqpoint{1.726556in}{3.162669in}}%
\pgfpathcurveto{\pgfqpoint{1.726556in}{3.154433in}}{\pgfqpoint{1.729829in}{3.146533in}}{\pgfqpoint{1.735653in}{3.140709in}}%
\pgfpathcurveto{\pgfqpoint{1.741477in}{3.134885in}}{\pgfqpoint{1.749377in}{3.131612in}}{\pgfqpoint{1.757613in}{3.131612in}}%
\pgfpathclose%
\pgfusepath{stroke,fill}%
\end{pgfscope}%
\begin{pgfscope}%
\pgfpathrectangle{\pgfqpoint{0.100000in}{0.220728in}}{\pgfqpoint{3.696000in}{3.696000in}}%
\pgfusepath{clip}%
\pgfsetbuttcap%
\pgfsetroundjoin%
\definecolor{currentfill}{rgb}{0.121569,0.466667,0.705882}%
\pgfsetfillcolor{currentfill}%
\pgfsetfillopacity{0.317185}%
\pgfsetlinewidth{1.003750pt}%
\definecolor{currentstroke}{rgb}{0.121569,0.466667,0.705882}%
\pgfsetstrokecolor{currentstroke}%
\pgfsetstrokeopacity{0.317185}%
\pgfsetdash{}{0pt}%
\pgfpathmoveto{\pgfqpoint{1.773563in}{3.128562in}}%
\pgfpathcurveto{\pgfqpoint{1.781799in}{3.128562in}}{\pgfqpoint{1.789699in}{3.131834in}}{\pgfqpoint{1.795523in}{3.137658in}}%
\pgfpathcurveto{\pgfqpoint{1.801347in}{3.143482in}}{\pgfqpoint{1.804620in}{3.151382in}}{\pgfqpoint{1.804620in}{3.159618in}}%
\pgfpathcurveto{\pgfqpoint{1.804620in}{3.167855in}}{\pgfqpoint{1.801347in}{3.175755in}}{\pgfqpoint{1.795523in}{3.181579in}}%
\pgfpathcurveto{\pgfqpoint{1.789699in}{3.187403in}}{\pgfqpoint{1.781799in}{3.190675in}}{\pgfqpoint{1.773563in}{3.190675in}}%
\pgfpathcurveto{\pgfqpoint{1.765327in}{3.190675in}}{\pgfqpoint{1.757427in}{3.187403in}}{\pgfqpoint{1.751603in}{3.181579in}}%
\pgfpathcurveto{\pgfqpoint{1.745779in}{3.175755in}}{\pgfqpoint{1.742507in}{3.167855in}}{\pgfqpoint{1.742507in}{3.159618in}}%
\pgfpathcurveto{\pgfqpoint{1.742507in}{3.151382in}}{\pgfqpoint{1.745779in}{3.143482in}}{\pgfqpoint{1.751603in}{3.137658in}}%
\pgfpathcurveto{\pgfqpoint{1.757427in}{3.131834in}}{\pgfqpoint{1.765327in}{3.128562in}}{\pgfqpoint{1.773563in}{3.128562in}}%
\pgfpathclose%
\pgfusepath{stroke,fill}%
\end{pgfscope}%
\begin{pgfscope}%
\pgfpathrectangle{\pgfqpoint{0.100000in}{0.220728in}}{\pgfqpoint{3.696000in}{3.696000in}}%
\pgfusepath{clip}%
\pgfsetbuttcap%
\pgfsetroundjoin%
\definecolor{currentfill}{rgb}{0.121569,0.466667,0.705882}%
\pgfsetfillcolor{currentfill}%
\pgfsetfillopacity{0.319077}%
\pgfsetlinewidth{1.003750pt}%
\definecolor{currentstroke}{rgb}{0.121569,0.466667,0.705882}%
\pgfsetstrokecolor{currentstroke}%
\pgfsetstrokeopacity{0.319077}%
\pgfsetdash{}{0pt}%
\pgfpathmoveto{\pgfqpoint{1.574698in}{2.999741in}}%
\pgfpathcurveto{\pgfqpoint{1.582934in}{2.999741in}}{\pgfqpoint{1.590834in}{3.003013in}}{\pgfqpoint{1.596658in}{3.008837in}}%
\pgfpathcurveto{\pgfqpoint{1.602482in}{3.014661in}}{\pgfqpoint{1.605754in}{3.022561in}}{\pgfqpoint{1.605754in}{3.030798in}}%
\pgfpathcurveto{\pgfqpoint{1.605754in}{3.039034in}}{\pgfqpoint{1.602482in}{3.046934in}}{\pgfqpoint{1.596658in}{3.052758in}}%
\pgfpathcurveto{\pgfqpoint{1.590834in}{3.058582in}}{\pgfqpoint{1.582934in}{3.061854in}}{\pgfqpoint{1.574698in}{3.061854in}}%
\pgfpathcurveto{\pgfqpoint{1.566461in}{3.061854in}}{\pgfqpoint{1.558561in}{3.058582in}}{\pgfqpoint{1.552737in}{3.052758in}}%
\pgfpathcurveto{\pgfqpoint{1.546913in}{3.046934in}}{\pgfqpoint{1.543641in}{3.039034in}}{\pgfqpoint{1.543641in}{3.030798in}}%
\pgfpathcurveto{\pgfqpoint{1.543641in}{3.022561in}}{\pgfqpoint{1.546913in}{3.014661in}}{\pgfqpoint{1.552737in}{3.008837in}}%
\pgfpathcurveto{\pgfqpoint{1.558561in}{3.003013in}}{\pgfqpoint{1.566461in}{2.999741in}}{\pgfqpoint{1.574698in}{2.999741in}}%
\pgfpathclose%
\pgfusepath{stroke,fill}%
\end{pgfscope}%
\begin{pgfscope}%
\pgfpathrectangle{\pgfqpoint{0.100000in}{0.220728in}}{\pgfqpoint{3.696000in}{3.696000in}}%
\pgfusepath{clip}%
\pgfsetbuttcap%
\pgfsetroundjoin%
\definecolor{currentfill}{rgb}{0.121569,0.466667,0.705882}%
\pgfsetfillcolor{currentfill}%
\pgfsetfillopacity{0.319104}%
\pgfsetlinewidth{1.003750pt}%
\definecolor{currentstroke}{rgb}{0.121569,0.466667,0.705882}%
\pgfsetstrokecolor{currentstroke}%
\pgfsetstrokeopacity{0.319104}%
\pgfsetdash{}{0pt}%
\pgfpathmoveto{\pgfqpoint{1.781565in}{3.126964in}}%
\pgfpathcurveto{\pgfqpoint{1.789801in}{3.126964in}}{\pgfqpoint{1.797701in}{3.130237in}}{\pgfqpoint{1.803525in}{3.136060in}}%
\pgfpathcurveto{\pgfqpoint{1.809349in}{3.141884in}}{\pgfqpoint{1.812621in}{3.149784in}}{\pgfqpoint{1.812621in}{3.158021in}}%
\pgfpathcurveto{\pgfqpoint{1.812621in}{3.166257in}}{\pgfqpoint{1.809349in}{3.174157in}}{\pgfqpoint{1.803525in}{3.179981in}}%
\pgfpathcurveto{\pgfqpoint{1.797701in}{3.185805in}}{\pgfqpoint{1.789801in}{3.189077in}}{\pgfqpoint{1.781565in}{3.189077in}}%
\pgfpathcurveto{\pgfqpoint{1.773329in}{3.189077in}}{\pgfqpoint{1.765429in}{3.185805in}}{\pgfqpoint{1.759605in}{3.179981in}}%
\pgfpathcurveto{\pgfqpoint{1.753781in}{3.174157in}}{\pgfqpoint{1.750508in}{3.166257in}}{\pgfqpoint{1.750508in}{3.158021in}}%
\pgfpathcurveto{\pgfqpoint{1.750508in}{3.149784in}}{\pgfqpoint{1.753781in}{3.141884in}}{\pgfqpoint{1.759605in}{3.136060in}}%
\pgfpathcurveto{\pgfqpoint{1.765429in}{3.130237in}}{\pgfqpoint{1.773329in}{3.126964in}}{\pgfqpoint{1.781565in}{3.126964in}}%
\pgfpathclose%
\pgfusepath{stroke,fill}%
\end{pgfscope}%
\begin{pgfscope}%
\pgfpathrectangle{\pgfqpoint{0.100000in}{0.220728in}}{\pgfqpoint{3.696000in}{3.696000in}}%
\pgfusepath{clip}%
\pgfsetbuttcap%
\pgfsetroundjoin%
\definecolor{currentfill}{rgb}{0.121569,0.466667,0.705882}%
\pgfsetfillcolor{currentfill}%
\pgfsetfillopacity{0.321636}%
\pgfsetlinewidth{1.003750pt}%
\definecolor{currentstroke}{rgb}{0.121569,0.466667,0.705882}%
\pgfsetstrokecolor{currentstroke}%
\pgfsetstrokeopacity{0.321636}%
\pgfsetdash{}{0pt}%
\pgfpathmoveto{\pgfqpoint{1.791471in}{3.125739in}}%
\pgfpathcurveto{\pgfqpoint{1.799707in}{3.125739in}}{\pgfqpoint{1.807607in}{3.129011in}}{\pgfqpoint{1.813431in}{3.134835in}}%
\pgfpathcurveto{\pgfqpoint{1.819255in}{3.140659in}}{\pgfqpoint{1.822527in}{3.148559in}}{\pgfqpoint{1.822527in}{3.156795in}}%
\pgfpathcurveto{\pgfqpoint{1.822527in}{3.165031in}}{\pgfqpoint{1.819255in}{3.172932in}}{\pgfqpoint{1.813431in}{3.178755in}}%
\pgfpathcurveto{\pgfqpoint{1.807607in}{3.184579in}}{\pgfqpoint{1.799707in}{3.187852in}}{\pgfqpoint{1.791471in}{3.187852in}}%
\pgfpathcurveto{\pgfqpoint{1.783235in}{3.187852in}}{\pgfqpoint{1.775335in}{3.184579in}}{\pgfqpoint{1.769511in}{3.178755in}}%
\pgfpathcurveto{\pgfqpoint{1.763687in}{3.172932in}}{\pgfqpoint{1.760414in}{3.165031in}}{\pgfqpoint{1.760414in}{3.156795in}}%
\pgfpathcurveto{\pgfqpoint{1.760414in}{3.148559in}}{\pgfqpoint{1.763687in}{3.140659in}}{\pgfqpoint{1.769511in}{3.134835in}}%
\pgfpathcurveto{\pgfqpoint{1.775335in}{3.129011in}}{\pgfqpoint{1.783235in}{3.125739in}}{\pgfqpoint{1.791471in}{3.125739in}}%
\pgfpathclose%
\pgfusepath{stroke,fill}%
\end{pgfscope}%
\begin{pgfscope}%
\pgfpathrectangle{\pgfqpoint{0.100000in}{0.220728in}}{\pgfqpoint{3.696000in}{3.696000in}}%
\pgfusepath{clip}%
\pgfsetbuttcap%
\pgfsetroundjoin%
\definecolor{currentfill}{rgb}{0.121569,0.466667,0.705882}%
\pgfsetfillcolor{currentfill}%
\pgfsetfillopacity{0.322565}%
\pgfsetlinewidth{1.003750pt}%
\definecolor{currentstroke}{rgb}{0.121569,0.466667,0.705882}%
\pgfsetstrokecolor{currentstroke}%
\pgfsetstrokeopacity{0.322565}%
\pgfsetdash{}{0pt}%
\pgfpathmoveto{\pgfqpoint{1.559236in}{2.974258in}}%
\pgfpathcurveto{\pgfqpoint{1.567472in}{2.974258in}}{\pgfqpoint{1.575372in}{2.977530in}}{\pgfqpoint{1.581196in}{2.983354in}}%
\pgfpathcurveto{\pgfqpoint{1.587020in}{2.989178in}}{\pgfqpoint{1.590292in}{2.997078in}}{\pgfqpoint{1.590292in}{3.005314in}}%
\pgfpathcurveto{\pgfqpoint{1.590292in}{3.013550in}}{\pgfqpoint{1.587020in}{3.021450in}}{\pgfqpoint{1.581196in}{3.027274in}}%
\pgfpathcurveto{\pgfqpoint{1.575372in}{3.033098in}}{\pgfqpoint{1.567472in}{3.036371in}}{\pgfqpoint{1.559236in}{3.036371in}}%
\pgfpathcurveto{\pgfqpoint{1.550999in}{3.036371in}}{\pgfqpoint{1.543099in}{3.033098in}}{\pgfqpoint{1.537275in}{3.027274in}}%
\pgfpathcurveto{\pgfqpoint{1.531451in}{3.021450in}}{\pgfqpoint{1.528179in}{3.013550in}}{\pgfqpoint{1.528179in}{3.005314in}}%
\pgfpathcurveto{\pgfqpoint{1.528179in}{2.997078in}}{\pgfqpoint{1.531451in}{2.989178in}}{\pgfqpoint{1.537275in}{2.983354in}}%
\pgfpathcurveto{\pgfqpoint{1.543099in}{2.977530in}}{\pgfqpoint{1.550999in}{2.974258in}}{\pgfqpoint{1.559236in}{2.974258in}}%
\pgfpathclose%
\pgfusepath{stroke,fill}%
\end{pgfscope}%
\begin{pgfscope}%
\pgfpathrectangle{\pgfqpoint{0.100000in}{0.220728in}}{\pgfqpoint{3.696000in}{3.696000in}}%
\pgfusepath{clip}%
\pgfsetbuttcap%
\pgfsetroundjoin%
\definecolor{currentfill}{rgb}{0.121569,0.466667,0.705882}%
\pgfsetfillcolor{currentfill}%
\pgfsetfillopacity{0.325700}%
\pgfsetlinewidth{1.003750pt}%
\definecolor{currentstroke}{rgb}{0.121569,0.466667,0.705882}%
\pgfsetstrokecolor{currentstroke}%
\pgfsetstrokeopacity{0.325700}%
\pgfsetdash{}{0pt}%
\pgfpathmoveto{\pgfqpoint{1.553572in}{2.957381in}}%
\pgfpathcurveto{\pgfqpoint{1.561809in}{2.957381in}}{\pgfqpoint{1.569709in}{2.960653in}}{\pgfqpoint{1.575533in}{2.966477in}}%
\pgfpathcurveto{\pgfqpoint{1.581357in}{2.972301in}}{\pgfqpoint{1.584629in}{2.980201in}}{\pgfqpoint{1.584629in}{2.988437in}}%
\pgfpathcurveto{\pgfqpoint{1.584629in}{2.996673in}}{\pgfqpoint{1.581357in}{3.004573in}}{\pgfqpoint{1.575533in}{3.010397in}}%
\pgfpathcurveto{\pgfqpoint{1.569709in}{3.016221in}}{\pgfqpoint{1.561809in}{3.019494in}}{\pgfqpoint{1.553572in}{3.019494in}}%
\pgfpathcurveto{\pgfqpoint{1.545336in}{3.019494in}}{\pgfqpoint{1.537436in}{3.016221in}}{\pgfqpoint{1.531612in}{3.010397in}}%
\pgfpathcurveto{\pgfqpoint{1.525788in}{3.004573in}}{\pgfqpoint{1.522516in}{2.996673in}}{\pgfqpoint{1.522516in}{2.988437in}}%
\pgfpathcurveto{\pgfqpoint{1.522516in}{2.980201in}}{\pgfqpoint{1.525788in}{2.972301in}}{\pgfqpoint{1.531612in}{2.966477in}}%
\pgfpathcurveto{\pgfqpoint{1.537436in}{2.960653in}}{\pgfqpoint{1.545336in}{2.957381in}}{\pgfqpoint{1.553572in}{2.957381in}}%
\pgfpathclose%
\pgfusepath{stroke,fill}%
\end{pgfscope}%
\begin{pgfscope}%
\pgfpathrectangle{\pgfqpoint{0.100000in}{0.220728in}}{\pgfqpoint{3.696000in}{3.696000in}}%
\pgfusepath{clip}%
\pgfsetbuttcap%
\pgfsetroundjoin%
\definecolor{currentfill}{rgb}{0.121569,0.466667,0.705882}%
\pgfsetfillcolor{currentfill}%
\pgfsetfillopacity{0.325727}%
\pgfsetlinewidth{1.003750pt}%
\definecolor{currentstroke}{rgb}{0.121569,0.466667,0.705882}%
\pgfsetstrokecolor{currentstroke}%
\pgfsetstrokeopacity{0.325727}%
\pgfsetdash{}{0pt}%
\pgfpathmoveto{\pgfqpoint{1.811274in}{3.122590in}}%
\pgfpathcurveto{\pgfqpoint{1.819511in}{3.122590in}}{\pgfqpoint{1.827411in}{3.125862in}}{\pgfqpoint{1.833234in}{3.131686in}}%
\pgfpathcurveto{\pgfqpoint{1.839058in}{3.137510in}}{\pgfqpoint{1.842331in}{3.145410in}}{\pgfqpoint{1.842331in}{3.153646in}}%
\pgfpathcurveto{\pgfqpoint{1.842331in}{3.161882in}}{\pgfqpoint{1.839058in}{3.169782in}}{\pgfqpoint{1.833234in}{3.175606in}}%
\pgfpathcurveto{\pgfqpoint{1.827411in}{3.181430in}}{\pgfqpoint{1.819511in}{3.184703in}}{\pgfqpoint{1.811274in}{3.184703in}}%
\pgfpathcurveto{\pgfqpoint{1.803038in}{3.184703in}}{\pgfqpoint{1.795138in}{3.181430in}}{\pgfqpoint{1.789314in}{3.175606in}}%
\pgfpathcurveto{\pgfqpoint{1.783490in}{3.169782in}}{\pgfqpoint{1.780218in}{3.161882in}}{\pgfqpoint{1.780218in}{3.153646in}}%
\pgfpathcurveto{\pgfqpoint{1.780218in}{3.145410in}}{\pgfqpoint{1.783490in}{3.137510in}}{\pgfqpoint{1.789314in}{3.131686in}}%
\pgfpathcurveto{\pgfqpoint{1.795138in}{3.125862in}}{\pgfqpoint{1.803038in}{3.122590in}}{\pgfqpoint{1.811274in}{3.122590in}}%
\pgfpathclose%
\pgfusepath{stroke,fill}%
\end{pgfscope}%
\begin{pgfscope}%
\pgfpathrectangle{\pgfqpoint{0.100000in}{0.220728in}}{\pgfqpoint{3.696000in}{3.696000in}}%
\pgfusepath{clip}%
\pgfsetbuttcap%
\pgfsetroundjoin%
\definecolor{currentfill}{rgb}{0.121569,0.466667,0.705882}%
\pgfsetfillcolor{currentfill}%
\pgfsetfillopacity{0.327067}%
\pgfsetlinewidth{1.003750pt}%
\definecolor{currentstroke}{rgb}{0.121569,0.466667,0.705882}%
\pgfsetstrokecolor{currentstroke}%
\pgfsetstrokeopacity{0.327067}%
\pgfsetdash{}{0pt}%
\pgfpathmoveto{\pgfqpoint{1.548607in}{2.948143in}}%
\pgfpathcurveto{\pgfqpoint{1.556844in}{2.948143in}}{\pgfqpoint{1.564744in}{2.951415in}}{\pgfqpoint{1.570568in}{2.957239in}}%
\pgfpathcurveto{\pgfqpoint{1.576392in}{2.963063in}}{\pgfqpoint{1.579664in}{2.970963in}}{\pgfqpoint{1.579664in}{2.979199in}}%
\pgfpathcurveto{\pgfqpoint{1.579664in}{2.987435in}}{\pgfqpoint{1.576392in}{2.995335in}}{\pgfqpoint{1.570568in}{3.001159in}}%
\pgfpathcurveto{\pgfqpoint{1.564744in}{3.006983in}}{\pgfqpoint{1.556844in}{3.010256in}}{\pgfqpoint{1.548607in}{3.010256in}}%
\pgfpathcurveto{\pgfqpoint{1.540371in}{3.010256in}}{\pgfqpoint{1.532471in}{3.006983in}}{\pgfqpoint{1.526647in}{3.001159in}}%
\pgfpathcurveto{\pgfqpoint{1.520823in}{2.995335in}}{\pgfqpoint{1.517551in}{2.987435in}}{\pgfqpoint{1.517551in}{2.979199in}}%
\pgfpathcurveto{\pgfqpoint{1.517551in}{2.970963in}}{\pgfqpoint{1.520823in}{2.963063in}}{\pgfqpoint{1.526647in}{2.957239in}}%
\pgfpathcurveto{\pgfqpoint{1.532471in}{2.951415in}}{\pgfqpoint{1.540371in}{2.948143in}}{\pgfqpoint{1.548607in}{2.948143in}}%
\pgfpathclose%
\pgfusepath{stroke,fill}%
\end{pgfscope}%
\begin{pgfscope}%
\pgfpathrectangle{\pgfqpoint{0.100000in}{0.220728in}}{\pgfqpoint{3.696000in}{3.696000in}}%
\pgfusepath{clip}%
\pgfsetbuttcap%
\pgfsetroundjoin%
\definecolor{currentfill}{rgb}{0.121569,0.466667,0.705882}%
\pgfsetfillcolor{currentfill}%
\pgfsetfillopacity{0.327348}%
\pgfsetlinewidth{1.003750pt}%
\definecolor{currentstroke}{rgb}{0.121569,0.466667,0.705882}%
\pgfsetstrokecolor{currentstroke}%
\pgfsetstrokeopacity{0.327348}%
\pgfsetdash{}{0pt}%
\pgfpathmoveto{\pgfqpoint{1.547983in}{2.946667in}}%
\pgfpathcurveto{\pgfqpoint{1.556220in}{2.946667in}}{\pgfqpoint{1.564120in}{2.949940in}}{\pgfqpoint{1.569944in}{2.955763in}}%
\pgfpathcurveto{\pgfqpoint{1.575768in}{2.961587in}}{\pgfqpoint{1.579040in}{2.969487in}}{\pgfqpoint{1.579040in}{2.977724in}}%
\pgfpathcurveto{\pgfqpoint{1.579040in}{2.985960in}}{\pgfqpoint{1.575768in}{2.993860in}}{\pgfqpoint{1.569944in}{2.999684in}}%
\pgfpathcurveto{\pgfqpoint{1.564120in}{3.005508in}}{\pgfqpoint{1.556220in}{3.008780in}}{\pgfqpoint{1.547983in}{3.008780in}}%
\pgfpathcurveto{\pgfqpoint{1.539747in}{3.008780in}}{\pgfqpoint{1.531847in}{3.005508in}}{\pgfqpoint{1.526023in}{2.999684in}}%
\pgfpathcurveto{\pgfqpoint{1.520199in}{2.993860in}}{\pgfqpoint{1.516927in}{2.985960in}}{\pgfqpoint{1.516927in}{2.977724in}}%
\pgfpathcurveto{\pgfqpoint{1.516927in}{2.969487in}}{\pgfqpoint{1.520199in}{2.961587in}}{\pgfqpoint{1.526023in}{2.955763in}}%
\pgfpathcurveto{\pgfqpoint{1.531847in}{2.949940in}}{\pgfqpoint{1.539747in}{2.946667in}}{\pgfqpoint{1.547983in}{2.946667in}}%
\pgfpathclose%
\pgfusepath{stroke,fill}%
\end{pgfscope}%
\begin{pgfscope}%
\pgfpathrectangle{\pgfqpoint{0.100000in}{0.220728in}}{\pgfqpoint{3.696000in}{3.696000in}}%
\pgfusepath{clip}%
\pgfsetbuttcap%
\pgfsetroundjoin%
\definecolor{currentfill}{rgb}{0.121569,0.466667,0.705882}%
\pgfsetfillcolor{currentfill}%
\pgfsetfillopacity{0.327807}%
\pgfsetlinewidth{1.003750pt}%
\definecolor{currentstroke}{rgb}{0.121569,0.466667,0.705882}%
\pgfsetstrokecolor{currentstroke}%
\pgfsetstrokeopacity{0.327807}%
\pgfsetdash{}{0pt}%
\pgfpathmoveto{\pgfqpoint{1.546542in}{2.944005in}}%
\pgfpathcurveto{\pgfqpoint{1.554778in}{2.944005in}}{\pgfqpoint{1.562678in}{2.947277in}}{\pgfqpoint{1.568502in}{2.953101in}}%
\pgfpathcurveto{\pgfqpoint{1.574326in}{2.958925in}}{\pgfqpoint{1.577598in}{2.966825in}}{\pgfqpoint{1.577598in}{2.975061in}}%
\pgfpathcurveto{\pgfqpoint{1.577598in}{2.983298in}}{\pgfqpoint{1.574326in}{2.991198in}}{\pgfqpoint{1.568502in}{2.997022in}}%
\pgfpathcurveto{\pgfqpoint{1.562678in}{3.002846in}}{\pgfqpoint{1.554778in}{3.006118in}}{\pgfqpoint{1.546542in}{3.006118in}}%
\pgfpathcurveto{\pgfqpoint{1.538306in}{3.006118in}}{\pgfqpoint{1.530405in}{3.002846in}}{\pgfqpoint{1.524582in}{2.997022in}}%
\pgfpathcurveto{\pgfqpoint{1.518758in}{2.991198in}}{\pgfqpoint{1.515485in}{2.983298in}}{\pgfqpoint{1.515485in}{2.975061in}}%
\pgfpathcurveto{\pgfqpoint{1.515485in}{2.966825in}}{\pgfqpoint{1.518758in}{2.958925in}}{\pgfqpoint{1.524582in}{2.953101in}}%
\pgfpathcurveto{\pgfqpoint{1.530405in}{2.947277in}}{\pgfqpoint{1.538306in}{2.944005in}}{\pgfqpoint{1.546542in}{2.944005in}}%
\pgfpathclose%
\pgfusepath{stroke,fill}%
\end{pgfscope}%
\begin{pgfscope}%
\pgfpathrectangle{\pgfqpoint{0.100000in}{0.220728in}}{\pgfqpoint{3.696000in}{3.696000in}}%
\pgfusepath{clip}%
\pgfsetbuttcap%
\pgfsetroundjoin%
\definecolor{currentfill}{rgb}{0.121569,0.466667,0.705882}%
\pgfsetfillcolor{currentfill}%
\pgfsetfillopacity{0.328651}%
\pgfsetlinewidth{1.003750pt}%
\definecolor{currentstroke}{rgb}{0.121569,0.466667,0.705882}%
\pgfsetstrokecolor{currentstroke}%
\pgfsetstrokeopacity{0.328651}%
\pgfsetdash{}{0pt}%
\pgfpathmoveto{\pgfqpoint{1.544158in}{2.938975in}}%
\pgfpathcurveto{\pgfqpoint{1.552395in}{2.938975in}}{\pgfqpoint{1.560295in}{2.942247in}}{\pgfqpoint{1.566118in}{2.948071in}}%
\pgfpathcurveto{\pgfqpoint{1.571942in}{2.953895in}}{\pgfqpoint{1.575215in}{2.961795in}}{\pgfqpoint{1.575215in}{2.970031in}}%
\pgfpathcurveto{\pgfqpoint{1.575215in}{2.978268in}}{\pgfqpoint{1.571942in}{2.986168in}}{\pgfqpoint{1.566118in}{2.991992in}}%
\pgfpathcurveto{\pgfqpoint{1.560295in}{2.997816in}}{\pgfqpoint{1.552395in}{3.001088in}}{\pgfqpoint{1.544158in}{3.001088in}}%
\pgfpathcurveto{\pgfqpoint{1.535922in}{3.001088in}}{\pgfqpoint{1.528022in}{2.997816in}}{\pgfqpoint{1.522198in}{2.991992in}}%
\pgfpathcurveto{\pgfqpoint{1.516374in}{2.986168in}}{\pgfqpoint{1.513102in}{2.978268in}}{\pgfqpoint{1.513102in}{2.970031in}}%
\pgfpathcurveto{\pgfqpoint{1.513102in}{2.961795in}}{\pgfqpoint{1.516374in}{2.953895in}}{\pgfqpoint{1.522198in}{2.948071in}}%
\pgfpathcurveto{\pgfqpoint{1.528022in}{2.942247in}}{\pgfqpoint{1.535922in}{2.938975in}}{\pgfqpoint{1.544158in}{2.938975in}}%
\pgfpathclose%
\pgfusepath{stroke,fill}%
\end{pgfscope}%
\begin{pgfscope}%
\pgfpathrectangle{\pgfqpoint{0.100000in}{0.220728in}}{\pgfqpoint{3.696000in}{3.696000in}}%
\pgfusepath{clip}%
\pgfsetbuttcap%
\pgfsetroundjoin%
\definecolor{currentfill}{rgb}{0.121569,0.466667,0.705882}%
\pgfsetfillcolor{currentfill}%
\pgfsetfillopacity{0.330209}%
\pgfsetlinewidth{1.003750pt}%
\definecolor{currentstroke}{rgb}{0.121569,0.466667,0.705882}%
\pgfsetstrokecolor{currentstroke}%
\pgfsetstrokeopacity{0.330209}%
\pgfsetdash{}{0pt}%
\pgfpathmoveto{\pgfqpoint{1.539770in}{2.929988in}}%
\pgfpathcurveto{\pgfqpoint{1.548007in}{2.929988in}}{\pgfqpoint{1.555907in}{2.933261in}}{\pgfqpoint{1.561731in}{2.939085in}}%
\pgfpathcurveto{\pgfqpoint{1.567555in}{2.944909in}}{\pgfqpoint{1.570827in}{2.952809in}}{\pgfqpoint{1.570827in}{2.961045in}}%
\pgfpathcurveto{\pgfqpoint{1.570827in}{2.969281in}}{\pgfqpoint{1.567555in}{2.977181in}}{\pgfqpoint{1.561731in}{2.983005in}}%
\pgfpathcurveto{\pgfqpoint{1.555907in}{2.988829in}}{\pgfqpoint{1.548007in}{2.992101in}}{\pgfqpoint{1.539770in}{2.992101in}}%
\pgfpathcurveto{\pgfqpoint{1.531534in}{2.992101in}}{\pgfqpoint{1.523634in}{2.988829in}}{\pgfqpoint{1.517810in}{2.983005in}}%
\pgfpathcurveto{\pgfqpoint{1.511986in}{2.977181in}}{\pgfqpoint{1.508714in}{2.969281in}}{\pgfqpoint{1.508714in}{2.961045in}}%
\pgfpathcurveto{\pgfqpoint{1.508714in}{2.952809in}}{\pgfqpoint{1.511986in}{2.944909in}}{\pgfqpoint{1.517810in}{2.939085in}}%
\pgfpathcurveto{\pgfqpoint{1.523634in}{2.933261in}}{\pgfqpoint{1.531534in}{2.929988in}}{\pgfqpoint{1.539770in}{2.929988in}}%
\pgfpathclose%
\pgfusepath{stroke,fill}%
\end{pgfscope}%
\begin{pgfscope}%
\pgfpathrectangle{\pgfqpoint{0.100000in}{0.220728in}}{\pgfqpoint{3.696000in}{3.696000in}}%
\pgfusepath{clip}%
\pgfsetbuttcap%
\pgfsetroundjoin%
\definecolor{currentfill}{rgb}{0.121569,0.466667,0.705882}%
\pgfsetfillcolor{currentfill}%
\pgfsetfillopacity{0.331598}%
\pgfsetlinewidth{1.003750pt}%
\definecolor{currentstroke}{rgb}{0.121569,0.466667,0.705882}%
\pgfsetstrokecolor{currentstroke}%
\pgfsetstrokeopacity{0.331598}%
\pgfsetdash{}{0pt}%
\pgfpathmoveto{\pgfqpoint{1.835466in}{3.117630in}}%
\pgfpathcurveto{\pgfqpoint{1.843703in}{3.117630in}}{\pgfqpoint{1.851603in}{3.120902in}}{\pgfqpoint{1.857427in}{3.126726in}}%
\pgfpathcurveto{\pgfqpoint{1.863250in}{3.132550in}}{\pgfqpoint{1.866523in}{3.140450in}}{\pgfqpoint{1.866523in}{3.148687in}}%
\pgfpathcurveto{\pgfqpoint{1.866523in}{3.156923in}}{\pgfqpoint{1.863250in}{3.164823in}}{\pgfqpoint{1.857427in}{3.170647in}}%
\pgfpathcurveto{\pgfqpoint{1.851603in}{3.176471in}}{\pgfqpoint{1.843703in}{3.179743in}}{\pgfqpoint{1.835466in}{3.179743in}}%
\pgfpathcurveto{\pgfqpoint{1.827230in}{3.179743in}}{\pgfqpoint{1.819330in}{3.176471in}}{\pgfqpoint{1.813506in}{3.170647in}}%
\pgfpathcurveto{\pgfqpoint{1.807682in}{3.164823in}}{\pgfqpoint{1.804410in}{3.156923in}}{\pgfqpoint{1.804410in}{3.148687in}}%
\pgfpathcurveto{\pgfqpoint{1.804410in}{3.140450in}}{\pgfqpoint{1.807682in}{3.132550in}}{\pgfqpoint{1.813506in}{3.126726in}}%
\pgfpathcurveto{\pgfqpoint{1.819330in}{3.120902in}}{\pgfqpoint{1.827230in}{3.117630in}}{\pgfqpoint{1.835466in}{3.117630in}}%
\pgfpathclose%
\pgfusepath{stroke,fill}%
\end{pgfscope}%
\begin{pgfscope}%
\pgfpathrectangle{\pgfqpoint{0.100000in}{0.220728in}}{\pgfqpoint{3.696000in}{3.696000in}}%
\pgfusepath{clip}%
\pgfsetbuttcap%
\pgfsetroundjoin%
\definecolor{currentfill}{rgb}{0.121569,0.466667,0.705882}%
\pgfsetfillcolor{currentfill}%
\pgfsetfillopacity{0.333123}%
\pgfsetlinewidth{1.003750pt}%
\definecolor{currentstroke}{rgb}{0.121569,0.466667,0.705882}%
\pgfsetstrokecolor{currentstroke}%
\pgfsetstrokeopacity{0.333123}%
\pgfsetdash{}{0pt}%
\pgfpathmoveto{\pgfqpoint{1.532305in}{2.913574in}}%
\pgfpathcurveto{\pgfqpoint{1.540541in}{2.913574in}}{\pgfqpoint{1.548441in}{2.916846in}}{\pgfqpoint{1.554265in}{2.922670in}}%
\pgfpathcurveto{\pgfqpoint{1.560089in}{2.928494in}}{\pgfqpoint{1.563362in}{2.936394in}}{\pgfqpoint{1.563362in}{2.944630in}}%
\pgfpathcurveto{\pgfqpoint{1.563362in}{2.952866in}}{\pgfqpoint{1.560089in}{2.960766in}}{\pgfqpoint{1.554265in}{2.966590in}}%
\pgfpathcurveto{\pgfqpoint{1.548441in}{2.972414in}}{\pgfqpoint{1.540541in}{2.975687in}}{\pgfqpoint{1.532305in}{2.975687in}}%
\pgfpathcurveto{\pgfqpoint{1.524069in}{2.975687in}}{\pgfqpoint{1.516169in}{2.972414in}}{\pgfqpoint{1.510345in}{2.966590in}}%
\pgfpathcurveto{\pgfqpoint{1.504521in}{2.960766in}}{\pgfqpoint{1.501249in}{2.952866in}}{\pgfqpoint{1.501249in}{2.944630in}}%
\pgfpathcurveto{\pgfqpoint{1.501249in}{2.936394in}}{\pgfqpoint{1.504521in}{2.928494in}}{\pgfqpoint{1.510345in}{2.922670in}}%
\pgfpathcurveto{\pgfqpoint{1.516169in}{2.916846in}}{\pgfqpoint{1.524069in}{2.913574in}}{\pgfqpoint{1.532305in}{2.913574in}}%
\pgfpathclose%
\pgfusepath{stroke,fill}%
\end{pgfscope}%
\begin{pgfscope}%
\pgfpathrectangle{\pgfqpoint{0.100000in}{0.220728in}}{\pgfqpoint{3.696000in}{3.696000in}}%
\pgfusepath{clip}%
\pgfsetbuttcap%
\pgfsetroundjoin%
\definecolor{currentfill}{rgb}{0.121569,0.466667,0.705882}%
\pgfsetfillcolor{currentfill}%
\pgfsetfillopacity{0.337406}%
\pgfsetlinewidth{1.003750pt}%
\definecolor{currentstroke}{rgb}{0.121569,0.466667,0.705882}%
\pgfsetstrokecolor{currentstroke}%
\pgfsetstrokeopacity{0.337406}%
\pgfsetdash{}{0pt}%
\pgfpathmoveto{\pgfqpoint{1.867871in}{3.109071in}}%
\pgfpathcurveto{\pgfqpoint{1.876108in}{3.109071in}}{\pgfqpoint{1.884008in}{3.112343in}}{\pgfqpoint{1.889832in}{3.118167in}}%
\pgfpathcurveto{\pgfqpoint{1.895656in}{3.123991in}}{\pgfqpoint{1.898928in}{3.131891in}}{\pgfqpoint{1.898928in}{3.140127in}}%
\pgfpathcurveto{\pgfqpoint{1.898928in}{3.148364in}}{\pgfqpoint{1.895656in}{3.156264in}}{\pgfqpoint{1.889832in}{3.162088in}}%
\pgfpathcurveto{\pgfqpoint{1.884008in}{3.167912in}}{\pgfqpoint{1.876108in}{3.171184in}}{\pgfqpoint{1.867871in}{3.171184in}}%
\pgfpathcurveto{\pgfqpoint{1.859635in}{3.171184in}}{\pgfqpoint{1.851735in}{3.167912in}}{\pgfqpoint{1.845911in}{3.162088in}}%
\pgfpathcurveto{\pgfqpoint{1.840087in}{3.156264in}}{\pgfqpoint{1.836815in}{3.148364in}}{\pgfqpoint{1.836815in}{3.140127in}}%
\pgfpathcurveto{\pgfqpoint{1.836815in}{3.131891in}}{\pgfqpoint{1.840087in}{3.123991in}}{\pgfqpoint{1.845911in}{3.118167in}}%
\pgfpathcurveto{\pgfqpoint{1.851735in}{3.112343in}}{\pgfqpoint{1.859635in}{3.109071in}}{\pgfqpoint{1.867871in}{3.109071in}}%
\pgfpathclose%
\pgfusepath{stroke,fill}%
\end{pgfscope}%
\begin{pgfscope}%
\pgfpathrectangle{\pgfqpoint{0.100000in}{0.220728in}}{\pgfqpoint{3.696000in}{3.696000in}}%
\pgfusepath{clip}%
\pgfsetbuttcap%
\pgfsetroundjoin%
\definecolor{currentfill}{rgb}{0.121569,0.466667,0.705882}%
\pgfsetfillcolor{currentfill}%
\pgfsetfillopacity{0.338065}%
\pgfsetlinewidth{1.003750pt}%
\definecolor{currentstroke}{rgb}{0.121569,0.466667,0.705882}%
\pgfsetstrokecolor{currentstroke}%
\pgfsetstrokeopacity{0.338065}%
\pgfsetdash{}{0pt}%
\pgfpathmoveto{\pgfqpoint{1.517530in}{2.882897in}}%
\pgfpathcurveto{\pgfqpoint{1.525767in}{2.882897in}}{\pgfqpoint{1.533667in}{2.886169in}}{\pgfqpoint{1.539491in}{2.891993in}}%
\pgfpathcurveto{\pgfqpoint{1.545315in}{2.897817in}}{\pgfqpoint{1.548587in}{2.905717in}}{\pgfqpoint{1.548587in}{2.913953in}}%
\pgfpathcurveto{\pgfqpoint{1.548587in}{2.922190in}}{\pgfqpoint{1.545315in}{2.930090in}}{\pgfqpoint{1.539491in}{2.935914in}}%
\pgfpathcurveto{\pgfqpoint{1.533667in}{2.941738in}}{\pgfqpoint{1.525767in}{2.945010in}}{\pgfqpoint{1.517530in}{2.945010in}}%
\pgfpathcurveto{\pgfqpoint{1.509294in}{2.945010in}}{\pgfqpoint{1.501394in}{2.941738in}}{\pgfqpoint{1.495570in}{2.935914in}}%
\pgfpathcurveto{\pgfqpoint{1.489746in}{2.930090in}}{\pgfqpoint{1.486474in}{2.922190in}}{\pgfqpoint{1.486474in}{2.913953in}}%
\pgfpathcurveto{\pgfqpoint{1.486474in}{2.905717in}}{\pgfqpoint{1.489746in}{2.897817in}}{\pgfqpoint{1.495570in}{2.891993in}}%
\pgfpathcurveto{\pgfqpoint{1.501394in}{2.886169in}}{\pgfqpoint{1.509294in}{2.882897in}}{\pgfqpoint{1.517530in}{2.882897in}}%
\pgfpathclose%
\pgfusepath{stroke,fill}%
\end{pgfscope}%
\begin{pgfscope}%
\pgfpathrectangle{\pgfqpoint{0.100000in}{0.220728in}}{\pgfqpoint{3.696000in}{3.696000in}}%
\pgfusepath{clip}%
\pgfsetbuttcap%
\pgfsetroundjoin%
\definecolor{currentfill}{rgb}{0.121569,0.466667,0.705882}%
\pgfsetfillcolor{currentfill}%
\pgfsetfillopacity{0.344914}%
\pgfsetlinewidth{1.003750pt}%
\definecolor{currentstroke}{rgb}{0.121569,0.466667,0.705882}%
\pgfsetstrokecolor{currentstroke}%
\pgfsetstrokeopacity{0.344914}%
\pgfsetdash{}{0pt}%
\pgfpathmoveto{\pgfqpoint{1.900911in}{3.102087in}}%
\pgfpathcurveto{\pgfqpoint{1.909148in}{3.102087in}}{\pgfqpoint{1.917048in}{3.105359in}}{\pgfqpoint{1.922872in}{3.111183in}}%
\pgfpathcurveto{\pgfqpoint{1.928696in}{3.117007in}}{\pgfqpoint{1.931968in}{3.124907in}}{\pgfqpoint{1.931968in}{3.133143in}}%
\pgfpathcurveto{\pgfqpoint{1.931968in}{3.141379in}}{\pgfqpoint{1.928696in}{3.149279in}}{\pgfqpoint{1.922872in}{3.155103in}}%
\pgfpathcurveto{\pgfqpoint{1.917048in}{3.160927in}}{\pgfqpoint{1.909148in}{3.164200in}}{\pgfqpoint{1.900911in}{3.164200in}}%
\pgfpathcurveto{\pgfqpoint{1.892675in}{3.164200in}}{\pgfqpoint{1.884775in}{3.160927in}}{\pgfqpoint{1.878951in}{3.155103in}}%
\pgfpathcurveto{\pgfqpoint{1.873127in}{3.149279in}}{\pgfqpoint{1.869855in}{3.141379in}}{\pgfqpoint{1.869855in}{3.133143in}}%
\pgfpathcurveto{\pgfqpoint{1.869855in}{3.124907in}}{\pgfqpoint{1.873127in}{3.117007in}}{\pgfqpoint{1.878951in}{3.111183in}}%
\pgfpathcurveto{\pgfqpoint{1.884775in}{3.105359in}}{\pgfqpoint{1.892675in}{3.102087in}}{\pgfqpoint{1.900911in}{3.102087in}}%
\pgfpathclose%
\pgfusepath{stroke,fill}%
\end{pgfscope}%
\begin{pgfscope}%
\pgfpathrectangle{\pgfqpoint{0.100000in}{0.220728in}}{\pgfqpoint{3.696000in}{3.696000in}}%
\pgfusepath{clip}%
\pgfsetbuttcap%
\pgfsetroundjoin%
\definecolor{currentfill}{rgb}{0.121569,0.466667,0.705882}%
\pgfsetfillcolor{currentfill}%
\pgfsetfillopacity{0.347669}%
\pgfsetlinewidth{1.003750pt}%
\definecolor{currentstroke}{rgb}{0.121569,0.466667,0.705882}%
\pgfsetstrokecolor{currentstroke}%
\pgfsetstrokeopacity{0.347669}%
\pgfsetdash{}{0pt}%
\pgfpathmoveto{\pgfqpoint{1.498457in}{2.823694in}}%
\pgfpathcurveto{\pgfqpoint{1.506693in}{2.823694in}}{\pgfqpoint{1.514593in}{2.826966in}}{\pgfqpoint{1.520417in}{2.832790in}}%
\pgfpathcurveto{\pgfqpoint{1.526241in}{2.838614in}}{\pgfqpoint{1.529513in}{2.846514in}}{\pgfqpoint{1.529513in}{2.854750in}}%
\pgfpathcurveto{\pgfqpoint{1.529513in}{2.862986in}}{\pgfqpoint{1.526241in}{2.870886in}}{\pgfqpoint{1.520417in}{2.876710in}}%
\pgfpathcurveto{\pgfqpoint{1.514593in}{2.882534in}}{\pgfqpoint{1.506693in}{2.885807in}}{\pgfqpoint{1.498457in}{2.885807in}}%
\pgfpathcurveto{\pgfqpoint{1.490220in}{2.885807in}}{\pgfqpoint{1.482320in}{2.882534in}}{\pgfqpoint{1.476496in}{2.876710in}}%
\pgfpathcurveto{\pgfqpoint{1.470672in}{2.870886in}}{\pgfqpoint{1.467400in}{2.862986in}}{\pgfqpoint{1.467400in}{2.854750in}}%
\pgfpathcurveto{\pgfqpoint{1.467400in}{2.846514in}}{\pgfqpoint{1.470672in}{2.838614in}}{\pgfqpoint{1.476496in}{2.832790in}}%
\pgfpathcurveto{\pgfqpoint{1.482320in}{2.826966in}}{\pgfqpoint{1.490220in}{2.823694in}}{\pgfqpoint{1.498457in}{2.823694in}}%
\pgfpathclose%
\pgfusepath{stroke,fill}%
\end{pgfscope}%
\begin{pgfscope}%
\pgfpathrectangle{\pgfqpoint{0.100000in}{0.220728in}}{\pgfqpoint{3.696000in}{3.696000in}}%
\pgfusepath{clip}%
\pgfsetbuttcap%
\pgfsetroundjoin%
\definecolor{currentfill}{rgb}{0.121569,0.466667,0.705882}%
\pgfsetfillcolor{currentfill}%
\pgfsetfillopacity{0.354960}%
\pgfsetlinewidth{1.003750pt}%
\definecolor{currentstroke}{rgb}{0.121569,0.466667,0.705882}%
\pgfsetstrokecolor{currentstroke}%
\pgfsetstrokeopacity{0.354960}%
\pgfsetdash{}{0pt}%
\pgfpathmoveto{\pgfqpoint{1.464572in}{2.775409in}}%
\pgfpathcurveto{\pgfqpoint{1.472808in}{2.775409in}}{\pgfqpoint{1.480709in}{2.778681in}}{\pgfqpoint{1.486532in}{2.784505in}}%
\pgfpathcurveto{\pgfqpoint{1.492356in}{2.790329in}}{\pgfqpoint{1.495629in}{2.798229in}}{\pgfqpoint{1.495629in}{2.806465in}}%
\pgfpathcurveto{\pgfqpoint{1.495629in}{2.814701in}}{\pgfqpoint{1.492356in}{2.822601in}}{\pgfqpoint{1.486532in}{2.828425in}}%
\pgfpathcurveto{\pgfqpoint{1.480709in}{2.834249in}}{\pgfqpoint{1.472808in}{2.837522in}}{\pgfqpoint{1.464572in}{2.837522in}}%
\pgfpathcurveto{\pgfqpoint{1.456336in}{2.837522in}}{\pgfqpoint{1.448436in}{2.834249in}}{\pgfqpoint{1.442612in}{2.828425in}}%
\pgfpathcurveto{\pgfqpoint{1.436788in}{2.822601in}}{\pgfqpoint{1.433516in}{2.814701in}}{\pgfqpoint{1.433516in}{2.806465in}}%
\pgfpathcurveto{\pgfqpoint{1.433516in}{2.798229in}}{\pgfqpoint{1.436788in}{2.790329in}}{\pgfqpoint{1.442612in}{2.784505in}}%
\pgfpathcurveto{\pgfqpoint{1.448436in}{2.778681in}}{\pgfqpoint{1.456336in}{2.775409in}}{\pgfqpoint{1.464572in}{2.775409in}}%
\pgfpathclose%
\pgfusepath{stroke,fill}%
\end{pgfscope}%
\begin{pgfscope}%
\pgfpathrectangle{\pgfqpoint{0.100000in}{0.220728in}}{\pgfqpoint{3.696000in}{3.696000in}}%
\pgfusepath{clip}%
\pgfsetbuttcap%
\pgfsetroundjoin%
\definecolor{currentfill}{rgb}{0.121569,0.466667,0.705882}%
\pgfsetfillcolor{currentfill}%
\pgfsetfillopacity{0.356689}%
\pgfsetlinewidth{1.003750pt}%
\definecolor{currentstroke}{rgb}{0.121569,0.466667,0.705882}%
\pgfsetstrokecolor{currentstroke}%
\pgfsetstrokeopacity{0.356689}%
\pgfsetdash{}{0pt}%
\pgfpathmoveto{\pgfqpoint{1.939397in}{3.096577in}}%
\pgfpathcurveto{\pgfqpoint{1.947633in}{3.096577in}}{\pgfqpoint{1.955533in}{3.099850in}}{\pgfqpoint{1.961357in}{3.105673in}}%
\pgfpathcurveto{\pgfqpoint{1.967181in}{3.111497in}}{\pgfqpoint{1.970453in}{3.119397in}}{\pgfqpoint{1.970453in}{3.127634in}}%
\pgfpathcurveto{\pgfqpoint{1.970453in}{3.135870in}}{\pgfqpoint{1.967181in}{3.143770in}}{\pgfqpoint{1.961357in}{3.149594in}}%
\pgfpathcurveto{\pgfqpoint{1.955533in}{3.155418in}}{\pgfqpoint{1.947633in}{3.158690in}}{\pgfqpoint{1.939397in}{3.158690in}}%
\pgfpathcurveto{\pgfqpoint{1.931160in}{3.158690in}}{\pgfqpoint{1.923260in}{3.155418in}}{\pgfqpoint{1.917436in}{3.149594in}}%
\pgfpathcurveto{\pgfqpoint{1.911612in}{3.143770in}}{\pgfqpoint{1.908340in}{3.135870in}}{\pgfqpoint{1.908340in}{3.127634in}}%
\pgfpathcurveto{\pgfqpoint{1.908340in}{3.119397in}}{\pgfqpoint{1.911612in}{3.111497in}}{\pgfqpoint{1.917436in}{3.105673in}}%
\pgfpathcurveto{\pgfqpoint{1.923260in}{3.099850in}}{\pgfqpoint{1.931160in}{3.096577in}}{\pgfqpoint{1.939397in}{3.096577in}}%
\pgfpathclose%
\pgfusepath{stroke,fill}%
\end{pgfscope}%
\begin{pgfscope}%
\pgfpathrectangle{\pgfqpoint{0.100000in}{0.220728in}}{\pgfqpoint{3.696000in}{3.696000in}}%
\pgfusepath{clip}%
\pgfsetbuttcap%
\pgfsetroundjoin%
\definecolor{currentfill}{rgb}{0.121569,0.466667,0.705882}%
\pgfsetfillcolor{currentfill}%
\pgfsetfillopacity{0.362054}%
\pgfsetlinewidth{1.003750pt}%
\definecolor{currentstroke}{rgb}{0.121569,0.466667,0.705882}%
\pgfsetstrokecolor{currentstroke}%
\pgfsetstrokeopacity{0.362054}%
\pgfsetdash{}{0pt}%
\pgfpathmoveto{\pgfqpoint{1.963250in}{3.094880in}}%
\pgfpathcurveto{\pgfqpoint{1.971486in}{3.094880in}}{\pgfqpoint{1.979386in}{3.098152in}}{\pgfqpoint{1.985210in}{3.103976in}}%
\pgfpathcurveto{\pgfqpoint{1.991034in}{3.109800in}}{\pgfqpoint{1.994306in}{3.117700in}}{\pgfqpoint{1.994306in}{3.125936in}}%
\pgfpathcurveto{\pgfqpoint{1.994306in}{3.134172in}}{\pgfqpoint{1.991034in}{3.142072in}}{\pgfqpoint{1.985210in}{3.147896in}}%
\pgfpathcurveto{\pgfqpoint{1.979386in}{3.153720in}}{\pgfqpoint{1.971486in}{3.156993in}}{\pgfqpoint{1.963250in}{3.156993in}}%
\pgfpathcurveto{\pgfqpoint{1.955014in}{3.156993in}}{\pgfqpoint{1.947113in}{3.153720in}}{\pgfqpoint{1.941290in}{3.147896in}}%
\pgfpathcurveto{\pgfqpoint{1.935466in}{3.142072in}}{\pgfqpoint{1.932193in}{3.134172in}}{\pgfqpoint{1.932193in}{3.125936in}}%
\pgfpathcurveto{\pgfqpoint{1.932193in}{3.117700in}}{\pgfqpoint{1.935466in}{3.109800in}}{\pgfqpoint{1.941290in}{3.103976in}}%
\pgfpathcurveto{\pgfqpoint{1.947113in}{3.098152in}}{\pgfqpoint{1.955014in}{3.094880in}}{\pgfqpoint{1.963250in}{3.094880in}}%
\pgfpathclose%
\pgfusepath{stroke,fill}%
\end{pgfscope}%
\begin{pgfscope}%
\pgfpathrectangle{\pgfqpoint{0.100000in}{0.220728in}}{\pgfqpoint{3.696000in}{3.696000in}}%
\pgfusepath{clip}%
\pgfsetbuttcap%
\pgfsetroundjoin%
\definecolor{currentfill}{rgb}{0.121569,0.466667,0.705882}%
\pgfsetfillcolor{currentfill}%
\pgfsetfillopacity{0.363628}%
\pgfsetlinewidth{1.003750pt}%
\definecolor{currentstroke}{rgb}{0.121569,0.466667,0.705882}%
\pgfsetstrokecolor{currentstroke}%
\pgfsetstrokeopacity{0.363628}%
\pgfsetdash{}{0pt}%
\pgfpathmoveto{\pgfqpoint{1.449718in}{2.727381in}}%
\pgfpathcurveto{\pgfqpoint{1.457955in}{2.727381in}}{\pgfqpoint{1.465855in}{2.730654in}}{\pgfqpoint{1.471679in}{2.736478in}}%
\pgfpathcurveto{\pgfqpoint{1.477502in}{2.742302in}}{\pgfqpoint{1.480775in}{2.750202in}}{\pgfqpoint{1.480775in}{2.758438in}}%
\pgfpathcurveto{\pgfqpoint{1.480775in}{2.766674in}}{\pgfqpoint{1.477502in}{2.774574in}}{\pgfqpoint{1.471679in}{2.780398in}}%
\pgfpathcurveto{\pgfqpoint{1.465855in}{2.786222in}}{\pgfqpoint{1.457955in}{2.789494in}}{\pgfqpoint{1.449718in}{2.789494in}}%
\pgfpathcurveto{\pgfqpoint{1.441482in}{2.789494in}}{\pgfqpoint{1.433582in}{2.786222in}}{\pgfqpoint{1.427758in}{2.780398in}}%
\pgfpathcurveto{\pgfqpoint{1.421934in}{2.774574in}}{\pgfqpoint{1.418662in}{2.766674in}}{\pgfqpoint{1.418662in}{2.758438in}}%
\pgfpathcurveto{\pgfqpoint{1.418662in}{2.750202in}}{\pgfqpoint{1.421934in}{2.742302in}}{\pgfqpoint{1.427758in}{2.736478in}}%
\pgfpathcurveto{\pgfqpoint{1.433582in}{2.730654in}}{\pgfqpoint{1.441482in}{2.727381in}}{\pgfqpoint{1.449718in}{2.727381in}}%
\pgfpathclose%
\pgfusepath{stroke,fill}%
\end{pgfscope}%
\begin{pgfscope}%
\pgfpathrectangle{\pgfqpoint{0.100000in}{0.220728in}}{\pgfqpoint{3.696000in}{3.696000in}}%
\pgfusepath{clip}%
\pgfsetbuttcap%
\pgfsetroundjoin%
\definecolor{currentfill}{rgb}{0.121569,0.466667,0.705882}%
\pgfsetfillcolor{currentfill}%
\pgfsetfillopacity{0.369194}%
\pgfsetlinewidth{1.003750pt}%
\definecolor{currentstroke}{rgb}{0.121569,0.466667,0.705882}%
\pgfsetstrokecolor{currentstroke}%
\pgfsetstrokeopacity{0.369194}%
\pgfsetdash{}{0pt}%
\pgfpathmoveto{\pgfqpoint{1.994890in}{3.090339in}}%
\pgfpathcurveto{\pgfqpoint{2.003127in}{3.090339in}}{\pgfqpoint{2.011027in}{3.093612in}}{\pgfqpoint{2.016851in}{3.099436in}}%
\pgfpathcurveto{\pgfqpoint{2.022675in}{3.105260in}}{\pgfqpoint{2.025947in}{3.113160in}}{\pgfqpoint{2.025947in}{3.121396in}}%
\pgfpathcurveto{\pgfqpoint{2.025947in}{3.129632in}}{\pgfqpoint{2.022675in}{3.137532in}}{\pgfqpoint{2.016851in}{3.143356in}}%
\pgfpathcurveto{\pgfqpoint{2.011027in}{3.149180in}}{\pgfqpoint{2.003127in}{3.152452in}}{\pgfqpoint{1.994890in}{3.152452in}}%
\pgfpathcurveto{\pgfqpoint{1.986654in}{3.152452in}}{\pgfqpoint{1.978754in}{3.149180in}}{\pgfqpoint{1.972930in}{3.143356in}}%
\pgfpathcurveto{\pgfqpoint{1.967106in}{3.137532in}}{\pgfqpoint{1.963834in}{3.129632in}}{\pgfqpoint{1.963834in}{3.121396in}}%
\pgfpathcurveto{\pgfqpoint{1.963834in}{3.113160in}}{\pgfqpoint{1.967106in}{3.105260in}}{\pgfqpoint{1.972930in}{3.099436in}}%
\pgfpathcurveto{\pgfqpoint{1.978754in}{3.093612in}}{\pgfqpoint{1.986654in}{3.090339in}}{\pgfqpoint{1.994890in}{3.090339in}}%
\pgfpathclose%
\pgfusepath{stroke,fill}%
\end{pgfscope}%
\begin{pgfscope}%
\pgfpathrectangle{\pgfqpoint{0.100000in}{0.220728in}}{\pgfqpoint{3.696000in}{3.696000in}}%
\pgfusepath{clip}%
\pgfsetbuttcap%
\pgfsetroundjoin%
\definecolor{currentfill}{rgb}{0.121569,0.466667,0.705882}%
\pgfsetfillcolor{currentfill}%
\pgfsetfillopacity{0.369420}%
\pgfsetlinewidth{1.003750pt}%
\definecolor{currentstroke}{rgb}{0.121569,0.466667,0.705882}%
\pgfsetstrokecolor{currentstroke}%
\pgfsetstrokeopacity{0.369420}%
\pgfsetdash{}{0pt}%
\pgfpathmoveto{\pgfqpoint{1.423733in}{2.689296in}}%
\pgfpathcurveto{\pgfqpoint{1.431969in}{2.689296in}}{\pgfqpoint{1.439870in}{2.692569in}}{\pgfqpoint{1.445693in}{2.698393in}}%
\pgfpathcurveto{\pgfqpoint{1.451517in}{2.704216in}}{\pgfqpoint{1.454790in}{2.712117in}}{\pgfqpoint{1.454790in}{2.720353in}}%
\pgfpathcurveto{\pgfqpoint{1.454790in}{2.728589in}}{\pgfqpoint{1.451517in}{2.736489in}}{\pgfqpoint{1.445693in}{2.742313in}}%
\pgfpathcurveto{\pgfqpoint{1.439870in}{2.748137in}}{\pgfqpoint{1.431969in}{2.751409in}}{\pgfqpoint{1.423733in}{2.751409in}}%
\pgfpathcurveto{\pgfqpoint{1.415497in}{2.751409in}}{\pgfqpoint{1.407597in}{2.748137in}}{\pgfqpoint{1.401773in}{2.742313in}}%
\pgfpathcurveto{\pgfqpoint{1.395949in}{2.736489in}}{\pgfqpoint{1.392677in}{2.728589in}}{\pgfqpoint{1.392677in}{2.720353in}}%
\pgfpathcurveto{\pgfqpoint{1.392677in}{2.712117in}}{\pgfqpoint{1.395949in}{2.704216in}}{\pgfqpoint{1.401773in}{2.698393in}}%
\pgfpathcurveto{\pgfqpoint{1.407597in}{2.692569in}}{\pgfqpoint{1.415497in}{2.689296in}}{\pgfqpoint{1.423733in}{2.689296in}}%
\pgfpathclose%
\pgfusepath{stroke,fill}%
\end{pgfscope}%
\begin{pgfscope}%
\pgfpathrectangle{\pgfqpoint{0.100000in}{0.220728in}}{\pgfqpoint{3.696000in}{3.696000in}}%
\pgfusepath{clip}%
\pgfsetbuttcap%
\pgfsetroundjoin%
\definecolor{currentfill}{rgb}{0.121569,0.466667,0.705882}%
\pgfsetfillcolor{currentfill}%
\pgfsetfillopacity{0.372466}%
\pgfsetlinewidth{1.003750pt}%
\definecolor{currentstroke}{rgb}{0.121569,0.466667,0.705882}%
\pgfsetstrokecolor{currentstroke}%
\pgfsetstrokeopacity{0.372466}%
\pgfsetdash{}{0pt}%
\pgfpathmoveto{\pgfqpoint{2.013517in}{3.088480in}}%
\pgfpathcurveto{\pgfqpoint{2.021753in}{3.088480in}}{\pgfqpoint{2.029653in}{3.091752in}}{\pgfqpoint{2.035477in}{3.097576in}}%
\pgfpathcurveto{\pgfqpoint{2.041301in}{3.103400in}}{\pgfqpoint{2.044573in}{3.111300in}}{\pgfqpoint{2.044573in}{3.119536in}}%
\pgfpathcurveto{\pgfqpoint{2.044573in}{3.127773in}}{\pgfqpoint{2.041301in}{3.135673in}}{\pgfqpoint{2.035477in}{3.141497in}}%
\pgfpathcurveto{\pgfqpoint{2.029653in}{3.147321in}}{\pgfqpoint{2.021753in}{3.150593in}}{\pgfqpoint{2.013517in}{3.150593in}}%
\pgfpathcurveto{\pgfqpoint{2.005281in}{3.150593in}}{\pgfqpoint{1.997381in}{3.147321in}}{\pgfqpoint{1.991557in}{3.141497in}}%
\pgfpathcurveto{\pgfqpoint{1.985733in}{3.135673in}}{\pgfqpoint{1.982460in}{3.127773in}}{\pgfqpoint{1.982460in}{3.119536in}}%
\pgfpathcurveto{\pgfqpoint{1.982460in}{3.111300in}}{\pgfqpoint{1.985733in}{3.103400in}}{\pgfqpoint{1.991557in}{3.097576in}}%
\pgfpathcurveto{\pgfqpoint{1.997381in}{3.091752in}}{\pgfqpoint{2.005281in}{3.088480in}}{\pgfqpoint{2.013517in}{3.088480in}}%
\pgfpathclose%
\pgfusepath{stroke,fill}%
\end{pgfscope}%
\begin{pgfscope}%
\pgfpathrectangle{\pgfqpoint{0.100000in}{0.220728in}}{\pgfqpoint{3.696000in}{3.696000in}}%
\pgfusepath{clip}%
\pgfsetbuttcap%
\pgfsetroundjoin%
\definecolor{currentfill}{rgb}{0.121569,0.466667,0.705882}%
\pgfsetfillcolor{currentfill}%
\pgfsetfillopacity{0.375684}%
\pgfsetlinewidth{1.003750pt}%
\definecolor{currentstroke}{rgb}{0.121569,0.466667,0.705882}%
\pgfsetstrokecolor{currentstroke}%
\pgfsetstrokeopacity{0.375684}%
\pgfsetdash{}{0pt}%
\pgfpathmoveto{\pgfqpoint{1.412509in}{2.651576in}}%
\pgfpathcurveto{\pgfqpoint{1.420746in}{2.651576in}}{\pgfqpoint{1.428646in}{2.654849in}}{\pgfqpoint{1.434470in}{2.660673in}}%
\pgfpathcurveto{\pgfqpoint{1.440294in}{2.666497in}}{\pgfqpoint{1.443566in}{2.674397in}}{\pgfqpoint{1.443566in}{2.682633in}}%
\pgfpathcurveto{\pgfqpoint{1.443566in}{2.690869in}}{\pgfqpoint{1.440294in}{2.698769in}}{\pgfqpoint{1.434470in}{2.704593in}}%
\pgfpathcurveto{\pgfqpoint{1.428646in}{2.710417in}}{\pgfqpoint{1.420746in}{2.713689in}}{\pgfqpoint{1.412509in}{2.713689in}}%
\pgfpathcurveto{\pgfqpoint{1.404273in}{2.713689in}}{\pgfqpoint{1.396373in}{2.710417in}}{\pgfqpoint{1.390549in}{2.704593in}}%
\pgfpathcurveto{\pgfqpoint{1.384725in}{2.698769in}}{\pgfqpoint{1.381453in}{2.690869in}}{\pgfqpoint{1.381453in}{2.682633in}}%
\pgfpathcurveto{\pgfqpoint{1.381453in}{2.674397in}}{\pgfqpoint{1.384725in}{2.666497in}}{\pgfqpoint{1.390549in}{2.660673in}}%
\pgfpathcurveto{\pgfqpoint{1.396373in}{2.654849in}}{\pgfqpoint{1.404273in}{2.651576in}}{\pgfqpoint{1.412509in}{2.651576in}}%
\pgfpathclose%
\pgfusepath{stroke,fill}%
\end{pgfscope}%
\begin{pgfscope}%
\pgfpathrectangle{\pgfqpoint{0.100000in}{0.220728in}}{\pgfqpoint{3.696000in}{3.696000in}}%
\pgfusepath{clip}%
\pgfsetbuttcap%
\pgfsetroundjoin%
\definecolor{currentfill}{rgb}{0.121569,0.466667,0.705882}%
\pgfsetfillcolor{currentfill}%
\pgfsetfillopacity{0.379380}%
\pgfsetlinewidth{1.003750pt}%
\definecolor{currentstroke}{rgb}{0.121569,0.466667,0.705882}%
\pgfsetstrokecolor{currentstroke}%
\pgfsetstrokeopacity{0.379380}%
\pgfsetdash{}{0pt}%
\pgfpathmoveto{\pgfqpoint{2.034812in}{3.083201in}}%
\pgfpathcurveto{\pgfqpoint{2.043049in}{3.083201in}}{\pgfqpoint{2.050949in}{3.086474in}}{\pgfqpoint{2.056773in}{3.092298in}}%
\pgfpathcurveto{\pgfqpoint{2.062596in}{3.098121in}}{\pgfqpoint{2.065869in}{3.106022in}}{\pgfqpoint{2.065869in}{3.114258in}}%
\pgfpathcurveto{\pgfqpoint{2.065869in}{3.122494in}}{\pgfqpoint{2.062596in}{3.130394in}}{\pgfqpoint{2.056773in}{3.136218in}}%
\pgfpathcurveto{\pgfqpoint{2.050949in}{3.142042in}}{\pgfqpoint{2.043049in}{3.145314in}}{\pgfqpoint{2.034812in}{3.145314in}}%
\pgfpathcurveto{\pgfqpoint{2.026576in}{3.145314in}}{\pgfqpoint{2.018676in}{3.142042in}}{\pgfqpoint{2.012852in}{3.136218in}}%
\pgfpathcurveto{\pgfqpoint{2.007028in}{3.130394in}}{\pgfqpoint{2.003756in}{3.122494in}}{\pgfqpoint{2.003756in}{3.114258in}}%
\pgfpathcurveto{\pgfqpoint{2.003756in}{3.106022in}}{\pgfqpoint{2.007028in}{3.098121in}}{\pgfqpoint{2.012852in}{3.092298in}}%
\pgfpathcurveto{\pgfqpoint{2.018676in}{3.086474in}}{\pgfqpoint{2.026576in}{3.083201in}}{\pgfqpoint{2.034812in}{3.083201in}}%
\pgfpathclose%
\pgfusepath{stroke,fill}%
\end{pgfscope}%
\begin{pgfscope}%
\pgfpathrectangle{\pgfqpoint{0.100000in}{0.220728in}}{\pgfqpoint{3.696000in}{3.696000in}}%
\pgfusepath{clip}%
\pgfsetbuttcap%
\pgfsetroundjoin%
\definecolor{currentfill}{rgb}{0.121569,0.466667,0.705882}%
\pgfsetfillcolor{currentfill}%
\pgfsetfillopacity{0.379498}%
\pgfsetlinewidth{1.003750pt}%
\definecolor{currentstroke}{rgb}{0.121569,0.466667,0.705882}%
\pgfsetstrokecolor{currentstroke}%
\pgfsetstrokeopacity{0.379498}%
\pgfsetdash{}{0pt}%
\pgfpathmoveto{\pgfqpoint{1.398055in}{2.627608in}}%
\pgfpathcurveto{\pgfqpoint{1.406291in}{2.627608in}}{\pgfqpoint{1.414191in}{2.630880in}}{\pgfqpoint{1.420015in}{2.636704in}}%
\pgfpathcurveto{\pgfqpoint{1.425839in}{2.642528in}}{\pgfqpoint{1.429111in}{2.650428in}}{\pgfqpoint{1.429111in}{2.658665in}}%
\pgfpathcurveto{\pgfqpoint{1.429111in}{2.666901in}}{\pgfqpoint{1.425839in}{2.674801in}}{\pgfqpoint{1.420015in}{2.680625in}}%
\pgfpathcurveto{\pgfqpoint{1.414191in}{2.686449in}}{\pgfqpoint{1.406291in}{2.689721in}}{\pgfqpoint{1.398055in}{2.689721in}}%
\pgfpathcurveto{\pgfqpoint{1.389819in}{2.689721in}}{\pgfqpoint{1.381919in}{2.686449in}}{\pgfqpoint{1.376095in}{2.680625in}}%
\pgfpathcurveto{\pgfqpoint{1.370271in}{2.674801in}}{\pgfqpoint{1.366998in}{2.666901in}}{\pgfqpoint{1.366998in}{2.658665in}}%
\pgfpathcurveto{\pgfqpoint{1.366998in}{2.650428in}}{\pgfqpoint{1.370271in}{2.642528in}}{\pgfqpoint{1.376095in}{2.636704in}}%
\pgfpathcurveto{\pgfqpoint{1.381919in}{2.630880in}}{\pgfqpoint{1.389819in}{2.627608in}}{\pgfqpoint{1.398055in}{2.627608in}}%
\pgfpathclose%
\pgfusepath{stroke,fill}%
\end{pgfscope}%
\begin{pgfscope}%
\pgfpathrectangle{\pgfqpoint{0.100000in}{0.220728in}}{\pgfqpoint{3.696000in}{3.696000in}}%
\pgfusepath{clip}%
\pgfsetbuttcap%
\pgfsetroundjoin%
\definecolor{currentfill}{rgb}{0.121569,0.466667,0.705882}%
\pgfsetfillcolor{currentfill}%
\pgfsetfillopacity{0.381660}%
\pgfsetlinewidth{1.003750pt}%
\definecolor{currentstroke}{rgb}{0.121569,0.466667,0.705882}%
\pgfsetstrokecolor{currentstroke}%
\pgfsetstrokeopacity{0.381660}%
\pgfsetdash{}{0pt}%
\pgfpathmoveto{\pgfqpoint{2.049376in}{3.080434in}}%
\pgfpathcurveto{\pgfqpoint{2.057612in}{3.080434in}}{\pgfqpoint{2.065512in}{3.083707in}}{\pgfqpoint{2.071336in}{3.089531in}}%
\pgfpathcurveto{\pgfqpoint{2.077160in}{3.095354in}}{\pgfqpoint{2.080432in}{3.103254in}}{\pgfqpoint{2.080432in}{3.111491in}}%
\pgfpathcurveto{\pgfqpoint{2.080432in}{3.119727in}}{\pgfqpoint{2.077160in}{3.127627in}}{\pgfqpoint{2.071336in}{3.133451in}}%
\pgfpathcurveto{\pgfqpoint{2.065512in}{3.139275in}}{\pgfqpoint{2.057612in}{3.142547in}}{\pgfqpoint{2.049376in}{3.142547in}}%
\pgfpathcurveto{\pgfqpoint{2.041140in}{3.142547in}}{\pgfqpoint{2.033240in}{3.139275in}}{\pgfqpoint{2.027416in}{3.133451in}}%
\pgfpathcurveto{\pgfqpoint{2.021592in}{3.127627in}}{\pgfqpoint{2.018319in}{3.119727in}}{\pgfqpoint{2.018319in}{3.111491in}}%
\pgfpathcurveto{\pgfqpoint{2.018319in}{3.103254in}}{\pgfqpoint{2.021592in}{3.095354in}}{\pgfqpoint{2.027416in}{3.089531in}}%
\pgfpathcurveto{\pgfqpoint{2.033240in}{3.083707in}}{\pgfqpoint{2.041140in}{3.080434in}}{\pgfqpoint{2.049376in}{3.080434in}}%
\pgfpathclose%
\pgfusepath{stroke,fill}%
\end{pgfscope}%
\begin{pgfscope}%
\pgfpathrectangle{\pgfqpoint{0.100000in}{0.220728in}}{\pgfqpoint{3.696000in}{3.696000in}}%
\pgfusepath{clip}%
\pgfsetbuttcap%
\pgfsetroundjoin%
\definecolor{currentfill}{rgb}{0.121569,0.466667,0.705882}%
\pgfsetfillcolor{currentfill}%
\pgfsetfillopacity{0.382900}%
\pgfsetlinewidth{1.003750pt}%
\definecolor{currentstroke}{rgb}{0.121569,0.466667,0.705882}%
\pgfsetstrokecolor{currentstroke}%
\pgfsetstrokeopacity{0.382900}%
\pgfsetdash{}{0pt}%
\pgfpathmoveto{\pgfqpoint{1.391517in}{2.607960in}}%
\pgfpathcurveto{\pgfqpoint{1.399753in}{2.607960in}}{\pgfqpoint{1.407654in}{2.611233in}}{\pgfqpoint{1.413477in}{2.617057in}}%
\pgfpathcurveto{\pgfqpoint{1.419301in}{2.622881in}}{\pgfqpoint{1.422574in}{2.630781in}}{\pgfqpoint{1.422574in}{2.639017in}}%
\pgfpathcurveto{\pgfqpoint{1.422574in}{2.647253in}}{\pgfqpoint{1.419301in}{2.655153in}}{\pgfqpoint{1.413477in}{2.660977in}}%
\pgfpathcurveto{\pgfqpoint{1.407654in}{2.666801in}}{\pgfqpoint{1.399753in}{2.670073in}}{\pgfqpoint{1.391517in}{2.670073in}}%
\pgfpathcurveto{\pgfqpoint{1.383281in}{2.670073in}}{\pgfqpoint{1.375381in}{2.666801in}}{\pgfqpoint{1.369557in}{2.660977in}}%
\pgfpathcurveto{\pgfqpoint{1.363733in}{2.655153in}}{\pgfqpoint{1.360461in}{2.647253in}}{\pgfqpoint{1.360461in}{2.639017in}}%
\pgfpathcurveto{\pgfqpoint{1.360461in}{2.630781in}}{\pgfqpoint{1.363733in}{2.622881in}}{\pgfqpoint{1.369557in}{2.617057in}}%
\pgfpathcurveto{\pgfqpoint{1.375381in}{2.611233in}}{\pgfqpoint{1.383281in}{2.607960in}}{\pgfqpoint{1.391517in}{2.607960in}}%
\pgfpathclose%
\pgfusepath{stroke,fill}%
\end{pgfscope}%
\begin{pgfscope}%
\pgfpathrectangle{\pgfqpoint{0.100000in}{0.220728in}}{\pgfqpoint{3.696000in}{3.696000in}}%
\pgfusepath{clip}%
\pgfsetbuttcap%
\pgfsetroundjoin%
\definecolor{currentfill}{rgb}{0.121569,0.466667,0.705882}%
\pgfsetfillcolor{currentfill}%
\pgfsetfillopacity{0.385099}%
\pgfsetlinewidth{1.003750pt}%
\definecolor{currentstroke}{rgb}{0.121569,0.466667,0.705882}%
\pgfsetstrokecolor{currentstroke}%
\pgfsetstrokeopacity{0.385099}%
\pgfsetdash{}{0pt}%
\pgfpathmoveto{\pgfqpoint{1.382540in}{2.591281in}}%
\pgfpathcurveto{\pgfqpoint{1.390776in}{2.591281in}}{\pgfqpoint{1.398676in}{2.594553in}}{\pgfqpoint{1.404500in}{2.600377in}}%
\pgfpathcurveto{\pgfqpoint{1.410324in}{2.606201in}}{\pgfqpoint{1.413596in}{2.614101in}}{\pgfqpoint{1.413596in}{2.622337in}}%
\pgfpathcurveto{\pgfqpoint{1.413596in}{2.630574in}}{\pgfqpoint{1.410324in}{2.638474in}}{\pgfqpoint{1.404500in}{2.644298in}}%
\pgfpathcurveto{\pgfqpoint{1.398676in}{2.650122in}}{\pgfqpoint{1.390776in}{2.653394in}}{\pgfqpoint{1.382540in}{2.653394in}}%
\pgfpathcurveto{\pgfqpoint{1.374304in}{2.653394in}}{\pgfqpoint{1.366404in}{2.650122in}}{\pgfqpoint{1.360580in}{2.644298in}}%
\pgfpathcurveto{\pgfqpoint{1.354756in}{2.638474in}}{\pgfqpoint{1.351483in}{2.630574in}}{\pgfqpoint{1.351483in}{2.622337in}}%
\pgfpathcurveto{\pgfqpoint{1.351483in}{2.614101in}}{\pgfqpoint{1.354756in}{2.606201in}}{\pgfqpoint{1.360580in}{2.600377in}}%
\pgfpathcurveto{\pgfqpoint{1.366404in}{2.594553in}}{\pgfqpoint{1.374304in}{2.591281in}}{\pgfqpoint{1.382540in}{2.591281in}}%
\pgfpathclose%
\pgfusepath{stroke,fill}%
\end{pgfscope}%
\begin{pgfscope}%
\pgfpathrectangle{\pgfqpoint{0.100000in}{0.220728in}}{\pgfqpoint{3.696000in}{3.696000in}}%
\pgfusepath{clip}%
\pgfsetbuttcap%
\pgfsetroundjoin%
\definecolor{currentfill}{rgb}{0.121569,0.466667,0.705882}%
\pgfsetfillcolor{currentfill}%
\pgfsetfillopacity{0.385924}%
\pgfsetlinewidth{1.003750pt}%
\definecolor{currentstroke}{rgb}{0.121569,0.466667,0.705882}%
\pgfsetstrokecolor{currentstroke}%
\pgfsetstrokeopacity{0.385924}%
\pgfsetdash{}{0pt}%
\pgfpathmoveto{\pgfqpoint{2.066324in}{3.077417in}}%
\pgfpathcurveto{\pgfqpoint{2.074560in}{3.077417in}}{\pgfqpoint{2.082460in}{3.080690in}}{\pgfqpoint{2.088284in}{3.086514in}}%
\pgfpathcurveto{\pgfqpoint{2.094108in}{3.092337in}}{\pgfqpoint{2.097381in}{3.100238in}}{\pgfqpoint{2.097381in}{3.108474in}}%
\pgfpathcurveto{\pgfqpoint{2.097381in}{3.116710in}}{\pgfqpoint{2.094108in}{3.124610in}}{\pgfqpoint{2.088284in}{3.130434in}}%
\pgfpathcurveto{\pgfqpoint{2.082460in}{3.136258in}}{\pgfqpoint{2.074560in}{3.139530in}}{\pgfqpoint{2.066324in}{3.139530in}}%
\pgfpathcurveto{\pgfqpoint{2.058088in}{3.139530in}}{\pgfqpoint{2.050188in}{3.136258in}}{\pgfqpoint{2.044364in}{3.130434in}}%
\pgfpathcurveto{\pgfqpoint{2.038540in}{3.124610in}}{\pgfqpoint{2.035268in}{3.116710in}}{\pgfqpoint{2.035268in}{3.108474in}}%
\pgfpathcurveto{\pgfqpoint{2.035268in}{3.100238in}}{\pgfqpoint{2.038540in}{3.092337in}}{\pgfqpoint{2.044364in}{3.086514in}}%
\pgfpathcurveto{\pgfqpoint{2.050188in}{3.080690in}}{\pgfqpoint{2.058088in}{3.077417in}}{\pgfqpoint{2.066324in}{3.077417in}}%
\pgfpathclose%
\pgfusepath{stroke,fill}%
\end{pgfscope}%
\begin{pgfscope}%
\pgfpathrectangle{\pgfqpoint{0.100000in}{0.220728in}}{\pgfqpoint{3.696000in}{3.696000in}}%
\pgfusepath{clip}%
\pgfsetbuttcap%
\pgfsetroundjoin%
\definecolor{currentfill}{rgb}{0.121569,0.466667,0.705882}%
\pgfsetfillcolor{currentfill}%
\pgfsetfillopacity{0.386383}%
\pgfsetlinewidth{1.003750pt}%
\definecolor{currentstroke}{rgb}{0.121569,0.466667,0.705882}%
\pgfsetstrokecolor{currentstroke}%
\pgfsetstrokeopacity{0.386383}%
\pgfsetdash{}{0pt}%
\pgfpathmoveto{\pgfqpoint{1.379686in}{2.583344in}}%
\pgfpathcurveto{\pgfqpoint{1.387922in}{2.583344in}}{\pgfqpoint{1.395823in}{2.586616in}}{\pgfqpoint{1.401646in}{2.592440in}}%
\pgfpathcurveto{\pgfqpoint{1.407470in}{2.598264in}}{\pgfqpoint{1.410743in}{2.606164in}}{\pgfqpoint{1.410743in}{2.614400in}}%
\pgfpathcurveto{\pgfqpoint{1.410743in}{2.622636in}}{\pgfqpoint{1.407470in}{2.630537in}}{\pgfqpoint{1.401646in}{2.636360in}}%
\pgfpathcurveto{\pgfqpoint{1.395823in}{2.642184in}}{\pgfqpoint{1.387922in}{2.645457in}}{\pgfqpoint{1.379686in}{2.645457in}}%
\pgfpathcurveto{\pgfqpoint{1.371450in}{2.645457in}}{\pgfqpoint{1.363550in}{2.642184in}}{\pgfqpoint{1.357726in}{2.636360in}}%
\pgfpathcurveto{\pgfqpoint{1.351902in}{2.630537in}}{\pgfqpoint{1.348630in}{2.622636in}}{\pgfqpoint{1.348630in}{2.614400in}}%
\pgfpathcurveto{\pgfqpoint{1.348630in}{2.606164in}}{\pgfqpoint{1.351902in}{2.598264in}}{\pgfqpoint{1.357726in}{2.592440in}}%
\pgfpathcurveto{\pgfqpoint{1.363550in}{2.586616in}}{\pgfqpoint{1.371450in}{2.583344in}}{\pgfqpoint{1.379686in}{2.583344in}}%
\pgfpathclose%
\pgfusepath{stroke,fill}%
\end{pgfscope}%
\begin{pgfscope}%
\pgfpathrectangle{\pgfqpoint{0.100000in}{0.220728in}}{\pgfqpoint{3.696000in}{3.696000in}}%
\pgfusepath{clip}%
\pgfsetbuttcap%
\pgfsetroundjoin%
\definecolor{currentfill}{rgb}{0.121569,0.466667,0.705882}%
\pgfsetfillcolor{currentfill}%
\pgfsetfillopacity{0.386974}%
\pgfsetlinewidth{1.003750pt}%
\definecolor{currentstroke}{rgb}{0.121569,0.466667,0.705882}%
\pgfsetstrokecolor{currentstroke}%
\pgfsetstrokeopacity{0.386974}%
\pgfsetdash{}{0pt}%
\pgfpathmoveto{\pgfqpoint{1.378249in}{2.579887in}}%
\pgfpathcurveto{\pgfqpoint{1.386485in}{2.579887in}}{\pgfqpoint{1.394385in}{2.583160in}}{\pgfqpoint{1.400209in}{2.588984in}}%
\pgfpathcurveto{\pgfqpoint{1.406033in}{2.594808in}}{\pgfqpoint{1.409306in}{2.602708in}}{\pgfqpoint{1.409306in}{2.610944in}}%
\pgfpathcurveto{\pgfqpoint{1.409306in}{2.619180in}}{\pgfqpoint{1.406033in}{2.627080in}}{\pgfqpoint{1.400209in}{2.632904in}}%
\pgfpathcurveto{\pgfqpoint{1.394385in}{2.638728in}}{\pgfqpoint{1.386485in}{2.642000in}}{\pgfqpoint{1.378249in}{2.642000in}}%
\pgfpathcurveto{\pgfqpoint{1.370013in}{2.642000in}}{\pgfqpoint{1.362113in}{2.638728in}}{\pgfqpoint{1.356289in}{2.632904in}}%
\pgfpathcurveto{\pgfqpoint{1.350465in}{2.627080in}}{\pgfqpoint{1.347193in}{2.619180in}}{\pgfqpoint{1.347193in}{2.610944in}}%
\pgfpathcurveto{\pgfqpoint{1.347193in}{2.602708in}}{\pgfqpoint{1.350465in}{2.594808in}}{\pgfqpoint{1.356289in}{2.588984in}}%
\pgfpathcurveto{\pgfqpoint{1.362113in}{2.583160in}}{\pgfqpoint{1.370013in}{2.579887in}}{\pgfqpoint{1.378249in}{2.579887in}}%
\pgfpathclose%
\pgfusepath{stroke,fill}%
\end{pgfscope}%
\begin{pgfscope}%
\pgfpathrectangle{\pgfqpoint{0.100000in}{0.220728in}}{\pgfqpoint{3.696000in}{3.696000in}}%
\pgfusepath{clip}%
\pgfsetbuttcap%
\pgfsetroundjoin%
\definecolor{currentfill}{rgb}{0.121569,0.466667,0.705882}%
\pgfsetfillcolor{currentfill}%
\pgfsetfillopacity{0.388059}%
\pgfsetlinewidth{1.003750pt}%
\definecolor{currentstroke}{rgb}{0.121569,0.466667,0.705882}%
\pgfsetstrokecolor{currentstroke}%
\pgfsetstrokeopacity{0.388059}%
\pgfsetdash{}{0pt}%
\pgfpathmoveto{\pgfqpoint{1.375191in}{2.574112in}}%
\pgfpathcurveto{\pgfqpoint{1.383427in}{2.574112in}}{\pgfqpoint{1.391327in}{2.577384in}}{\pgfqpoint{1.397151in}{2.583208in}}%
\pgfpathcurveto{\pgfqpoint{1.402975in}{2.589032in}}{\pgfqpoint{1.406247in}{2.596932in}}{\pgfqpoint{1.406247in}{2.605168in}}%
\pgfpathcurveto{\pgfqpoint{1.406247in}{2.613404in}}{\pgfqpoint{1.402975in}{2.621304in}}{\pgfqpoint{1.397151in}{2.627128in}}%
\pgfpathcurveto{\pgfqpoint{1.391327in}{2.632952in}}{\pgfqpoint{1.383427in}{2.636225in}}{\pgfqpoint{1.375191in}{2.636225in}}%
\pgfpathcurveto{\pgfqpoint{1.366955in}{2.636225in}}{\pgfqpoint{1.359055in}{2.632952in}}{\pgfqpoint{1.353231in}{2.627128in}}%
\pgfpathcurveto{\pgfqpoint{1.347407in}{2.621304in}}{\pgfqpoint{1.344134in}{2.613404in}}{\pgfqpoint{1.344134in}{2.605168in}}%
\pgfpathcurveto{\pgfqpoint{1.344134in}{2.596932in}}{\pgfqpoint{1.347407in}{2.589032in}}{\pgfqpoint{1.353231in}{2.583208in}}%
\pgfpathcurveto{\pgfqpoint{1.359055in}{2.577384in}}{\pgfqpoint{1.366955in}{2.574112in}}{\pgfqpoint{1.375191in}{2.574112in}}%
\pgfpathclose%
\pgfusepath{stroke,fill}%
\end{pgfscope}%
\begin{pgfscope}%
\pgfpathrectangle{\pgfqpoint{0.100000in}{0.220728in}}{\pgfqpoint{3.696000in}{3.696000in}}%
\pgfusepath{clip}%
\pgfsetbuttcap%
\pgfsetroundjoin%
\definecolor{currentfill}{rgb}{0.121569,0.466667,0.705882}%
\pgfsetfillcolor{currentfill}%
\pgfsetfillopacity{0.389480}%
\pgfsetlinewidth{1.003750pt}%
\definecolor{currentstroke}{rgb}{0.121569,0.466667,0.705882}%
\pgfsetstrokecolor{currentstroke}%
\pgfsetstrokeopacity{0.389480}%
\pgfsetdash{}{0pt}%
\pgfpathmoveto{\pgfqpoint{2.091781in}{3.074478in}}%
\pgfpathcurveto{\pgfqpoint{2.100018in}{3.074478in}}{\pgfqpoint{2.107918in}{3.077750in}}{\pgfqpoint{2.113742in}{3.083574in}}%
\pgfpathcurveto{\pgfqpoint{2.119566in}{3.089398in}}{\pgfqpoint{2.122838in}{3.097298in}}{\pgfqpoint{2.122838in}{3.105534in}}%
\pgfpathcurveto{\pgfqpoint{2.122838in}{3.113771in}}{\pgfqpoint{2.119566in}{3.121671in}}{\pgfqpoint{2.113742in}{3.127495in}}%
\pgfpathcurveto{\pgfqpoint{2.107918in}{3.133319in}}{\pgfqpoint{2.100018in}{3.136591in}}{\pgfqpoint{2.091781in}{3.136591in}}%
\pgfpathcurveto{\pgfqpoint{2.083545in}{3.136591in}}{\pgfqpoint{2.075645in}{3.133319in}}{\pgfqpoint{2.069821in}{3.127495in}}%
\pgfpathcurveto{\pgfqpoint{2.063997in}{3.121671in}}{\pgfqpoint{2.060725in}{3.113771in}}{\pgfqpoint{2.060725in}{3.105534in}}%
\pgfpathcurveto{\pgfqpoint{2.060725in}{3.097298in}}{\pgfqpoint{2.063997in}{3.089398in}}{\pgfqpoint{2.069821in}{3.083574in}}%
\pgfpathcurveto{\pgfqpoint{2.075645in}{3.077750in}}{\pgfqpoint{2.083545in}{3.074478in}}{\pgfqpoint{2.091781in}{3.074478in}}%
\pgfpathclose%
\pgfusepath{stroke,fill}%
\end{pgfscope}%
\begin{pgfscope}%
\pgfpathrectangle{\pgfqpoint{0.100000in}{0.220728in}}{\pgfqpoint{3.696000in}{3.696000in}}%
\pgfusepath{clip}%
\pgfsetbuttcap%
\pgfsetroundjoin%
\definecolor{currentfill}{rgb}{0.121569,0.466667,0.705882}%
\pgfsetfillcolor{currentfill}%
\pgfsetfillopacity{0.390076}%
\pgfsetlinewidth{1.003750pt}%
\definecolor{currentstroke}{rgb}{0.121569,0.466667,0.705882}%
\pgfsetstrokecolor{currentstroke}%
\pgfsetstrokeopacity{0.390076}%
\pgfsetdash{}{0pt}%
\pgfpathmoveto{\pgfqpoint{1.370786in}{2.562735in}}%
\pgfpathcurveto{\pgfqpoint{1.379022in}{2.562735in}}{\pgfqpoint{1.386923in}{2.566007in}}{\pgfqpoint{1.392746in}{2.571831in}}%
\pgfpathcurveto{\pgfqpoint{1.398570in}{2.577655in}}{\pgfqpoint{1.401843in}{2.585555in}}{\pgfqpoint{1.401843in}{2.593791in}}%
\pgfpathcurveto{\pgfqpoint{1.401843in}{2.602028in}}{\pgfqpoint{1.398570in}{2.609928in}}{\pgfqpoint{1.392746in}{2.615752in}}%
\pgfpathcurveto{\pgfqpoint{1.386923in}{2.621576in}}{\pgfqpoint{1.379022in}{2.624848in}}{\pgfqpoint{1.370786in}{2.624848in}}%
\pgfpathcurveto{\pgfqpoint{1.362550in}{2.624848in}}{\pgfqpoint{1.354650in}{2.621576in}}{\pgfqpoint{1.348826in}{2.615752in}}%
\pgfpathcurveto{\pgfqpoint{1.343002in}{2.609928in}}{\pgfqpoint{1.339730in}{2.602028in}}{\pgfqpoint{1.339730in}{2.593791in}}%
\pgfpathcurveto{\pgfqpoint{1.339730in}{2.585555in}}{\pgfqpoint{1.343002in}{2.577655in}}{\pgfqpoint{1.348826in}{2.571831in}}%
\pgfpathcurveto{\pgfqpoint{1.354650in}{2.566007in}}{\pgfqpoint{1.362550in}{2.562735in}}{\pgfqpoint{1.370786in}{2.562735in}}%
\pgfpathclose%
\pgfusepath{stroke,fill}%
\end{pgfscope}%
\begin{pgfscope}%
\pgfpathrectangle{\pgfqpoint{0.100000in}{0.220728in}}{\pgfqpoint{3.696000in}{3.696000in}}%
\pgfusepath{clip}%
\pgfsetbuttcap%
\pgfsetroundjoin%
\definecolor{currentfill}{rgb}{0.121569,0.466667,0.705882}%
\pgfsetfillcolor{currentfill}%
\pgfsetfillopacity{0.393155}%
\pgfsetlinewidth{1.003750pt}%
\definecolor{currentstroke}{rgb}{0.121569,0.466667,0.705882}%
\pgfsetstrokecolor{currentstroke}%
\pgfsetstrokeopacity{0.393155}%
\pgfsetdash{}{0pt}%
\pgfpathmoveto{\pgfqpoint{1.358081in}{2.544128in}}%
\pgfpathcurveto{\pgfqpoint{1.366317in}{2.544128in}}{\pgfqpoint{1.374217in}{2.547400in}}{\pgfqpoint{1.380041in}{2.553224in}}%
\pgfpathcurveto{\pgfqpoint{1.385865in}{2.559048in}}{\pgfqpoint{1.389137in}{2.566948in}}{\pgfqpoint{1.389137in}{2.575184in}}%
\pgfpathcurveto{\pgfqpoint{1.389137in}{2.583421in}}{\pgfqpoint{1.385865in}{2.591321in}}{\pgfqpoint{1.380041in}{2.597145in}}%
\pgfpathcurveto{\pgfqpoint{1.374217in}{2.602969in}}{\pgfqpoint{1.366317in}{2.606241in}}{\pgfqpoint{1.358081in}{2.606241in}}%
\pgfpathcurveto{\pgfqpoint{1.349845in}{2.606241in}}{\pgfqpoint{1.341945in}{2.602969in}}{\pgfqpoint{1.336121in}{2.597145in}}%
\pgfpathcurveto{\pgfqpoint{1.330297in}{2.591321in}}{\pgfqpoint{1.327024in}{2.583421in}}{\pgfqpoint{1.327024in}{2.575184in}}%
\pgfpathcurveto{\pgfqpoint{1.327024in}{2.566948in}}{\pgfqpoint{1.330297in}{2.559048in}}{\pgfqpoint{1.336121in}{2.553224in}}%
\pgfpathcurveto{\pgfqpoint{1.341945in}{2.547400in}}{\pgfqpoint{1.349845in}{2.544128in}}{\pgfqpoint{1.358081in}{2.544128in}}%
\pgfpathclose%
\pgfusepath{stroke,fill}%
\end{pgfscope}%
\begin{pgfscope}%
\pgfpathrectangle{\pgfqpoint{0.100000in}{0.220728in}}{\pgfqpoint{3.696000in}{3.696000in}}%
\pgfusepath{clip}%
\pgfsetbuttcap%
\pgfsetroundjoin%
\definecolor{currentfill}{rgb}{0.121569,0.466667,0.705882}%
\pgfsetfillcolor{currentfill}%
\pgfsetfillopacity{0.396736}%
\pgfsetlinewidth{1.003750pt}%
\definecolor{currentstroke}{rgb}{0.121569,0.466667,0.705882}%
\pgfsetstrokecolor{currentstroke}%
\pgfsetstrokeopacity{0.396736}%
\pgfsetdash{}{0pt}%
\pgfpathmoveto{\pgfqpoint{2.119596in}{3.068322in}}%
\pgfpathcurveto{\pgfqpoint{2.127832in}{3.068322in}}{\pgfqpoint{2.135732in}{3.071594in}}{\pgfqpoint{2.141556in}{3.077418in}}%
\pgfpathcurveto{\pgfqpoint{2.147380in}{3.083242in}}{\pgfqpoint{2.150652in}{3.091142in}}{\pgfqpoint{2.150652in}{3.099378in}}%
\pgfpathcurveto{\pgfqpoint{2.150652in}{3.107615in}}{\pgfqpoint{2.147380in}{3.115515in}}{\pgfqpoint{2.141556in}{3.121339in}}%
\pgfpathcurveto{\pgfqpoint{2.135732in}{3.127163in}}{\pgfqpoint{2.127832in}{3.130435in}}{\pgfqpoint{2.119596in}{3.130435in}}%
\pgfpathcurveto{\pgfqpoint{2.111360in}{3.130435in}}{\pgfqpoint{2.103459in}{3.127163in}}{\pgfqpoint{2.097636in}{3.121339in}}%
\pgfpathcurveto{\pgfqpoint{2.091812in}{3.115515in}}{\pgfqpoint{2.088539in}{3.107615in}}{\pgfqpoint{2.088539in}{3.099378in}}%
\pgfpathcurveto{\pgfqpoint{2.088539in}{3.091142in}}{\pgfqpoint{2.091812in}{3.083242in}}{\pgfqpoint{2.097636in}{3.077418in}}%
\pgfpathcurveto{\pgfqpoint{2.103459in}{3.071594in}}{\pgfqpoint{2.111360in}{3.068322in}}{\pgfqpoint{2.119596in}{3.068322in}}%
\pgfpathclose%
\pgfusepath{stroke,fill}%
\end{pgfscope}%
\begin{pgfscope}%
\pgfpathrectangle{\pgfqpoint{0.100000in}{0.220728in}}{\pgfqpoint{3.696000in}{3.696000in}}%
\pgfusepath{clip}%
\pgfsetbuttcap%
\pgfsetroundjoin%
\definecolor{currentfill}{rgb}{0.121569,0.466667,0.705882}%
\pgfsetfillcolor{currentfill}%
\pgfsetfillopacity{0.399599}%
\pgfsetlinewidth{1.003750pt}%
\definecolor{currentstroke}{rgb}{0.121569,0.466667,0.705882}%
\pgfsetstrokecolor{currentstroke}%
\pgfsetstrokeopacity{0.399599}%
\pgfsetdash{}{0pt}%
\pgfpathmoveto{\pgfqpoint{1.341270in}{2.507049in}}%
\pgfpathcurveto{\pgfqpoint{1.349506in}{2.507049in}}{\pgfqpoint{1.357406in}{2.510321in}}{\pgfqpoint{1.363230in}{2.516145in}}%
\pgfpathcurveto{\pgfqpoint{1.369054in}{2.521969in}}{\pgfqpoint{1.372326in}{2.529869in}}{\pgfqpoint{1.372326in}{2.538106in}}%
\pgfpathcurveto{\pgfqpoint{1.372326in}{2.546342in}}{\pgfqpoint{1.369054in}{2.554242in}}{\pgfqpoint{1.363230in}{2.560066in}}%
\pgfpathcurveto{\pgfqpoint{1.357406in}{2.565890in}}{\pgfqpoint{1.349506in}{2.569162in}}{\pgfqpoint{1.341270in}{2.569162in}}%
\pgfpathcurveto{\pgfqpoint{1.333033in}{2.569162in}}{\pgfqpoint{1.325133in}{2.565890in}}{\pgfqpoint{1.319309in}{2.560066in}}%
\pgfpathcurveto{\pgfqpoint{1.313485in}{2.554242in}}{\pgfqpoint{1.310213in}{2.546342in}}{\pgfqpoint{1.310213in}{2.538106in}}%
\pgfpathcurveto{\pgfqpoint{1.310213in}{2.529869in}}{\pgfqpoint{1.313485in}{2.521969in}}{\pgfqpoint{1.319309in}{2.516145in}}%
\pgfpathcurveto{\pgfqpoint{1.325133in}{2.510321in}}{\pgfqpoint{1.333033in}{2.507049in}}{\pgfqpoint{1.341270in}{2.507049in}}%
\pgfpathclose%
\pgfusepath{stroke,fill}%
\end{pgfscope}%
\begin{pgfscope}%
\pgfpathrectangle{\pgfqpoint{0.100000in}{0.220728in}}{\pgfqpoint{3.696000in}{3.696000in}}%
\pgfusepath{clip}%
\pgfsetbuttcap%
\pgfsetroundjoin%
\definecolor{currentfill}{rgb}{0.121569,0.466667,0.705882}%
\pgfsetfillcolor{currentfill}%
\pgfsetfillopacity{0.406314}%
\pgfsetlinewidth{1.003750pt}%
\definecolor{currentstroke}{rgb}{0.121569,0.466667,0.705882}%
\pgfsetstrokecolor{currentstroke}%
\pgfsetstrokeopacity{0.406314}%
\pgfsetdash{}{0pt}%
\pgfpathmoveto{\pgfqpoint{2.150808in}{3.067069in}}%
\pgfpathcurveto{\pgfqpoint{2.159044in}{3.067069in}}{\pgfqpoint{2.166944in}{3.070341in}}{\pgfqpoint{2.172768in}{3.076165in}}%
\pgfpathcurveto{\pgfqpoint{2.178592in}{3.081989in}}{\pgfqpoint{2.181864in}{3.089889in}}{\pgfqpoint{2.181864in}{3.098126in}}%
\pgfpathcurveto{\pgfqpoint{2.181864in}{3.106362in}}{\pgfqpoint{2.178592in}{3.114262in}}{\pgfqpoint{2.172768in}{3.120086in}}%
\pgfpathcurveto{\pgfqpoint{2.166944in}{3.125910in}}{\pgfqpoint{2.159044in}{3.129182in}}{\pgfqpoint{2.150808in}{3.129182in}}%
\pgfpathcurveto{\pgfqpoint{2.142571in}{3.129182in}}{\pgfqpoint{2.134671in}{3.125910in}}{\pgfqpoint{2.128847in}{3.120086in}}%
\pgfpathcurveto{\pgfqpoint{2.123023in}{3.114262in}}{\pgfqpoint{2.119751in}{3.106362in}}{\pgfqpoint{2.119751in}{3.098126in}}%
\pgfpathcurveto{\pgfqpoint{2.119751in}{3.089889in}}{\pgfqpoint{2.123023in}{3.081989in}}{\pgfqpoint{2.128847in}{3.076165in}}%
\pgfpathcurveto{\pgfqpoint{2.134671in}{3.070341in}}{\pgfqpoint{2.142571in}{3.067069in}}{\pgfqpoint{2.150808in}{3.067069in}}%
\pgfpathclose%
\pgfusepath{stroke,fill}%
\end{pgfscope}%
\begin{pgfscope}%
\pgfpathrectangle{\pgfqpoint{0.100000in}{0.220728in}}{\pgfqpoint{3.696000in}{3.696000in}}%
\pgfusepath{clip}%
\pgfsetbuttcap%
\pgfsetroundjoin%
\definecolor{currentfill}{rgb}{0.121569,0.466667,0.705882}%
\pgfsetfillcolor{currentfill}%
\pgfsetfillopacity{0.409134}%
\pgfsetlinewidth{1.003750pt}%
\definecolor{currentstroke}{rgb}{0.121569,0.466667,0.705882}%
\pgfsetstrokecolor{currentstroke}%
\pgfsetstrokeopacity{0.409134}%
\pgfsetdash{}{0pt}%
\pgfpathmoveto{\pgfqpoint{1.301557in}{2.438075in}}%
\pgfpathcurveto{\pgfqpoint{1.309794in}{2.438075in}}{\pgfqpoint{1.317694in}{2.441347in}}{\pgfqpoint{1.323518in}{2.447171in}}%
\pgfpathcurveto{\pgfqpoint{1.329342in}{2.452995in}}{\pgfqpoint{1.332614in}{2.460895in}}{\pgfqpoint{1.332614in}{2.469131in}}%
\pgfpathcurveto{\pgfqpoint{1.332614in}{2.477368in}}{\pgfqpoint{1.329342in}{2.485268in}}{\pgfqpoint{1.323518in}{2.491092in}}%
\pgfpathcurveto{\pgfqpoint{1.317694in}{2.496915in}}{\pgfqpoint{1.309794in}{2.500188in}}{\pgfqpoint{1.301557in}{2.500188in}}%
\pgfpathcurveto{\pgfqpoint{1.293321in}{2.500188in}}{\pgfqpoint{1.285421in}{2.496915in}}{\pgfqpoint{1.279597in}{2.491092in}}%
\pgfpathcurveto{\pgfqpoint{1.273773in}{2.485268in}}{\pgfqpoint{1.270501in}{2.477368in}}{\pgfqpoint{1.270501in}{2.469131in}}%
\pgfpathcurveto{\pgfqpoint{1.270501in}{2.460895in}}{\pgfqpoint{1.273773in}{2.452995in}}{\pgfqpoint{1.279597in}{2.447171in}}%
\pgfpathcurveto{\pgfqpoint{1.285421in}{2.441347in}}{\pgfqpoint{1.293321in}{2.438075in}}{\pgfqpoint{1.301557in}{2.438075in}}%
\pgfpathclose%
\pgfusepath{stroke,fill}%
\end{pgfscope}%
\begin{pgfscope}%
\pgfpathrectangle{\pgfqpoint{0.100000in}{0.220728in}}{\pgfqpoint{3.696000in}{3.696000in}}%
\pgfusepath{clip}%
\pgfsetbuttcap%
\pgfsetroundjoin%
\definecolor{currentfill}{rgb}{0.121569,0.466667,0.705882}%
\pgfsetfillcolor{currentfill}%
\pgfsetfillopacity{0.416609}%
\pgfsetlinewidth{1.003750pt}%
\definecolor{currentstroke}{rgb}{0.121569,0.466667,0.705882}%
\pgfsetstrokecolor{currentstroke}%
\pgfsetstrokeopacity{0.416609}%
\pgfsetdash{}{0pt}%
\pgfpathmoveto{\pgfqpoint{2.185957in}{3.059949in}}%
\pgfpathcurveto{\pgfqpoint{2.194193in}{3.059949in}}{\pgfqpoint{2.202094in}{3.063221in}}{\pgfqpoint{2.207917in}{3.069045in}}%
\pgfpathcurveto{\pgfqpoint{2.213741in}{3.074869in}}{\pgfqpoint{2.217014in}{3.082769in}}{\pgfqpoint{2.217014in}{3.091005in}}%
\pgfpathcurveto{\pgfqpoint{2.217014in}{3.099242in}}{\pgfqpoint{2.213741in}{3.107142in}}{\pgfqpoint{2.207917in}{3.112965in}}%
\pgfpathcurveto{\pgfqpoint{2.202094in}{3.118789in}}{\pgfqpoint{2.194193in}{3.122062in}}{\pgfqpoint{2.185957in}{3.122062in}}%
\pgfpathcurveto{\pgfqpoint{2.177721in}{3.122062in}}{\pgfqpoint{2.169821in}{3.118789in}}{\pgfqpoint{2.163997in}{3.112965in}}%
\pgfpathcurveto{\pgfqpoint{2.158173in}{3.107142in}}{\pgfqpoint{2.154901in}{3.099242in}}{\pgfqpoint{2.154901in}{3.091005in}}%
\pgfpathcurveto{\pgfqpoint{2.154901in}{3.082769in}}{\pgfqpoint{2.158173in}{3.074869in}}{\pgfqpoint{2.163997in}{3.069045in}}%
\pgfpathcurveto{\pgfqpoint{2.169821in}{3.063221in}}{\pgfqpoint{2.177721in}{3.059949in}}{\pgfqpoint{2.185957in}{3.059949in}}%
\pgfpathclose%
\pgfusepath{stroke,fill}%
\end{pgfscope}%
\begin{pgfscope}%
\pgfpathrectangle{\pgfqpoint{0.100000in}{0.220728in}}{\pgfqpoint{3.696000in}{3.696000in}}%
\pgfusepath{clip}%
\pgfsetbuttcap%
\pgfsetroundjoin%
\definecolor{currentfill}{rgb}{0.121569,0.466667,0.705882}%
\pgfsetfillcolor{currentfill}%
\pgfsetfillopacity{0.420355}%
\pgfsetlinewidth{1.003750pt}%
\definecolor{currentstroke}{rgb}{0.121569,0.466667,0.705882}%
\pgfsetstrokecolor{currentstroke}%
\pgfsetstrokeopacity{0.420355}%
\pgfsetdash{}{0pt}%
\pgfpathmoveto{\pgfqpoint{1.280202in}{2.366529in}}%
\pgfpathcurveto{\pgfqpoint{1.288438in}{2.366529in}}{\pgfqpoint{1.296338in}{2.369801in}}{\pgfqpoint{1.302162in}{2.375625in}}%
\pgfpathcurveto{\pgfqpoint{1.307986in}{2.381449in}}{\pgfqpoint{1.311259in}{2.389349in}}{\pgfqpoint{1.311259in}{2.397585in}}%
\pgfpathcurveto{\pgfqpoint{1.311259in}{2.405822in}}{\pgfqpoint{1.307986in}{2.413722in}}{\pgfqpoint{1.302162in}{2.419546in}}%
\pgfpathcurveto{\pgfqpoint{1.296338in}{2.425370in}}{\pgfqpoint{1.288438in}{2.428642in}}{\pgfqpoint{1.280202in}{2.428642in}}%
\pgfpathcurveto{\pgfqpoint{1.271966in}{2.428642in}}{\pgfqpoint{1.264066in}{2.425370in}}{\pgfqpoint{1.258242in}{2.419546in}}%
\pgfpathcurveto{\pgfqpoint{1.252418in}{2.413722in}}{\pgfqpoint{1.249146in}{2.405822in}}{\pgfqpoint{1.249146in}{2.397585in}}%
\pgfpathcurveto{\pgfqpoint{1.249146in}{2.389349in}}{\pgfqpoint{1.252418in}{2.381449in}}{\pgfqpoint{1.258242in}{2.375625in}}%
\pgfpathcurveto{\pgfqpoint{1.264066in}{2.369801in}}{\pgfqpoint{1.271966in}{2.366529in}}{\pgfqpoint{1.280202in}{2.366529in}}%
\pgfpathclose%
\pgfusepath{stroke,fill}%
\end{pgfscope}%
\begin{pgfscope}%
\pgfpathrectangle{\pgfqpoint{0.100000in}{0.220728in}}{\pgfqpoint{3.696000in}{3.696000in}}%
\pgfusepath{clip}%
\pgfsetbuttcap%
\pgfsetroundjoin%
\definecolor{currentfill}{rgb}{0.121569,0.466667,0.705882}%
\pgfsetfillcolor{currentfill}%
\pgfsetfillopacity{0.426364}%
\pgfsetlinewidth{1.003750pt}%
\definecolor{currentstroke}{rgb}{0.121569,0.466667,0.705882}%
\pgfsetstrokecolor{currentstroke}%
\pgfsetstrokeopacity{0.426364}%
\pgfsetdash{}{0pt}%
\pgfpathmoveto{\pgfqpoint{2.230405in}{3.053558in}}%
\pgfpathcurveto{\pgfqpoint{2.238641in}{3.053558in}}{\pgfqpoint{2.246542in}{3.056830in}}{\pgfqpoint{2.252365in}{3.062654in}}%
\pgfpathcurveto{\pgfqpoint{2.258189in}{3.068478in}}{\pgfqpoint{2.261462in}{3.076378in}}{\pgfqpoint{2.261462in}{3.084614in}}%
\pgfpathcurveto{\pgfqpoint{2.261462in}{3.092850in}}{\pgfqpoint{2.258189in}{3.100750in}}{\pgfqpoint{2.252365in}{3.106574in}}%
\pgfpathcurveto{\pgfqpoint{2.246542in}{3.112398in}}{\pgfqpoint{2.238641in}{3.115671in}}{\pgfqpoint{2.230405in}{3.115671in}}%
\pgfpathcurveto{\pgfqpoint{2.222169in}{3.115671in}}{\pgfqpoint{2.214269in}{3.112398in}}{\pgfqpoint{2.208445in}{3.106574in}}%
\pgfpathcurveto{\pgfqpoint{2.202621in}{3.100750in}}{\pgfqpoint{2.199349in}{3.092850in}}{\pgfqpoint{2.199349in}{3.084614in}}%
\pgfpathcurveto{\pgfqpoint{2.199349in}{3.076378in}}{\pgfqpoint{2.202621in}{3.068478in}}{\pgfqpoint{2.208445in}{3.062654in}}%
\pgfpathcurveto{\pgfqpoint{2.214269in}{3.056830in}}{\pgfqpoint{2.222169in}{3.053558in}}{\pgfqpoint{2.230405in}{3.053558in}}%
\pgfpathclose%
\pgfusepath{stroke,fill}%
\end{pgfscope}%
\begin{pgfscope}%
\pgfpathrectangle{\pgfqpoint{0.100000in}{0.220728in}}{\pgfqpoint{3.696000in}{3.696000in}}%
\pgfusepath{clip}%
\pgfsetbuttcap%
\pgfsetroundjoin%
\definecolor{currentfill}{rgb}{0.121569,0.466667,0.705882}%
\pgfsetfillcolor{currentfill}%
\pgfsetfillopacity{0.427712}%
\pgfsetlinewidth{1.003750pt}%
\definecolor{currentstroke}{rgb}{0.121569,0.466667,0.705882}%
\pgfsetstrokecolor{currentstroke}%
\pgfsetstrokeopacity{0.427712}%
\pgfsetdash{}{0pt}%
\pgfpathmoveto{\pgfqpoint{1.239820in}{2.304673in}}%
\pgfpathcurveto{\pgfqpoint{1.248057in}{2.304673in}}{\pgfqpoint{1.255957in}{2.307945in}}{\pgfqpoint{1.261781in}{2.313769in}}%
\pgfpathcurveto{\pgfqpoint{1.267605in}{2.319593in}}{\pgfqpoint{1.270877in}{2.327493in}}{\pgfqpoint{1.270877in}{2.335729in}}%
\pgfpathcurveto{\pgfqpoint{1.270877in}{2.343965in}}{\pgfqpoint{1.267605in}{2.351866in}}{\pgfqpoint{1.261781in}{2.357689in}}%
\pgfpathcurveto{\pgfqpoint{1.255957in}{2.363513in}}{\pgfqpoint{1.248057in}{2.366786in}}{\pgfqpoint{1.239820in}{2.366786in}}%
\pgfpathcurveto{\pgfqpoint{1.231584in}{2.366786in}}{\pgfqpoint{1.223684in}{2.363513in}}{\pgfqpoint{1.217860in}{2.357689in}}%
\pgfpathcurveto{\pgfqpoint{1.212036in}{2.351866in}}{\pgfqpoint{1.208764in}{2.343965in}}{\pgfqpoint{1.208764in}{2.335729in}}%
\pgfpathcurveto{\pgfqpoint{1.208764in}{2.327493in}}{\pgfqpoint{1.212036in}{2.319593in}}{\pgfqpoint{1.217860in}{2.313769in}}%
\pgfpathcurveto{\pgfqpoint{1.223684in}{2.307945in}}{\pgfqpoint{1.231584in}{2.304673in}}{\pgfqpoint{1.239820in}{2.304673in}}%
\pgfpathclose%
\pgfusepath{stroke,fill}%
\end{pgfscope}%
\begin{pgfscope}%
\pgfpathrectangle{\pgfqpoint{0.100000in}{0.220728in}}{\pgfqpoint{3.696000in}{3.696000in}}%
\pgfusepath{clip}%
\pgfsetbuttcap%
\pgfsetroundjoin%
\definecolor{currentfill}{rgb}{0.121569,0.466667,0.705882}%
\pgfsetfillcolor{currentfill}%
\pgfsetfillopacity{0.430976}%
\pgfsetlinewidth{1.003750pt}%
\definecolor{currentstroke}{rgb}{0.121569,0.466667,0.705882}%
\pgfsetstrokecolor{currentstroke}%
\pgfsetstrokeopacity{0.430976}%
\pgfsetdash{}{0pt}%
\pgfpathmoveto{\pgfqpoint{2.255535in}{3.048215in}}%
\pgfpathcurveto{\pgfqpoint{2.263771in}{3.048215in}}{\pgfqpoint{2.271671in}{3.051487in}}{\pgfqpoint{2.277495in}{3.057311in}}%
\pgfpathcurveto{\pgfqpoint{2.283319in}{3.063135in}}{\pgfqpoint{2.286591in}{3.071035in}}{\pgfqpoint{2.286591in}{3.079271in}}%
\pgfpathcurveto{\pgfqpoint{2.286591in}{3.087508in}}{\pgfqpoint{2.283319in}{3.095408in}}{\pgfqpoint{2.277495in}{3.101231in}}%
\pgfpathcurveto{\pgfqpoint{2.271671in}{3.107055in}}{\pgfqpoint{2.263771in}{3.110328in}}{\pgfqpoint{2.255535in}{3.110328in}}%
\pgfpathcurveto{\pgfqpoint{2.247298in}{3.110328in}}{\pgfqpoint{2.239398in}{3.107055in}}{\pgfqpoint{2.233574in}{3.101231in}}%
\pgfpathcurveto{\pgfqpoint{2.227750in}{3.095408in}}{\pgfqpoint{2.224478in}{3.087508in}}{\pgfqpoint{2.224478in}{3.079271in}}%
\pgfpathcurveto{\pgfqpoint{2.224478in}{3.071035in}}{\pgfqpoint{2.227750in}{3.063135in}}{\pgfqpoint{2.233574in}{3.057311in}}%
\pgfpathcurveto{\pgfqpoint{2.239398in}{3.051487in}}{\pgfqpoint{2.247298in}{3.048215in}}{\pgfqpoint{2.255535in}{3.048215in}}%
\pgfpathclose%
\pgfusepath{stroke,fill}%
\end{pgfscope}%
\begin{pgfscope}%
\pgfpathrectangle{\pgfqpoint{0.100000in}{0.220728in}}{\pgfqpoint{3.696000in}{3.696000in}}%
\pgfusepath{clip}%
\pgfsetbuttcap%
\pgfsetroundjoin%
\definecolor{currentfill}{rgb}{0.121569,0.466667,0.705882}%
\pgfsetfillcolor{currentfill}%
\pgfsetfillopacity{0.436521}%
\pgfsetlinewidth{1.003750pt}%
\definecolor{currentstroke}{rgb}{0.121569,0.466667,0.705882}%
\pgfsetstrokecolor{currentstroke}%
\pgfsetstrokeopacity{0.436521}%
\pgfsetdash{}{0pt}%
\pgfpathmoveto{\pgfqpoint{1.225028in}{2.250676in}}%
\pgfpathcurveto{\pgfqpoint{1.233265in}{2.250676in}}{\pgfqpoint{1.241165in}{2.253948in}}{\pgfqpoint{1.246989in}{2.259772in}}%
\pgfpathcurveto{\pgfqpoint{1.252813in}{2.265596in}}{\pgfqpoint{1.256085in}{2.273496in}}{\pgfqpoint{1.256085in}{2.281733in}}%
\pgfpathcurveto{\pgfqpoint{1.256085in}{2.289969in}}{\pgfqpoint{1.252813in}{2.297869in}}{\pgfqpoint{1.246989in}{2.303693in}}%
\pgfpathcurveto{\pgfqpoint{1.241165in}{2.309517in}}{\pgfqpoint{1.233265in}{2.312789in}}{\pgfqpoint{1.225028in}{2.312789in}}%
\pgfpathcurveto{\pgfqpoint{1.216792in}{2.312789in}}{\pgfqpoint{1.208892in}{2.309517in}}{\pgfqpoint{1.203068in}{2.303693in}}%
\pgfpathcurveto{\pgfqpoint{1.197244in}{2.297869in}}{\pgfqpoint{1.193972in}{2.289969in}}{\pgfqpoint{1.193972in}{2.281733in}}%
\pgfpathcurveto{\pgfqpoint{1.193972in}{2.273496in}}{\pgfqpoint{1.197244in}{2.265596in}}{\pgfqpoint{1.203068in}{2.259772in}}%
\pgfpathcurveto{\pgfqpoint{1.208892in}{2.253948in}}{\pgfqpoint{1.216792in}{2.250676in}}{\pgfqpoint{1.225028in}{2.250676in}}%
\pgfpathclose%
\pgfusepath{stroke,fill}%
\end{pgfscope}%
\begin{pgfscope}%
\pgfpathrectangle{\pgfqpoint{0.100000in}{0.220728in}}{\pgfqpoint{3.696000in}{3.696000in}}%
\pgfusepath{clip}%
\pgfsetbuttcap%
\pgfsetroundjoin%
\definecolor{currentfill}{rgb}{0.121569,0.466667,0.705882}%
\pgfsetfillcolor{currentfill}%
\pgfsetfillopacity{0.439621}%
\pgfsetlinewidth{1.003750pt}%
\definecolor{currentstroke}{rgb}{0.121569,0.466667,0.705882}%
\pgfsetstrokecolor{currentstroke}%
\pgfsetstrokeopacity{0.439621}%
\pgfsetdash{}{0pt}%
\pgfpathmoveto{\pgfqpoint{2.285518in}{3.043118in}}%
\pgfpathcurveto{\pgfqpoint{2.293754in}{3.043118in}}{\pgfqpoint{2.301654in}{3.046390in}}{\pgfqpoint{2.307478in}{3.052214in}}%
\pgfpathcurveto{\pgfqpoint{2.313302in}{3.058038in}}{\pgfqpoint{2.316574in}{3.065938in}}{\pgfqpoint{2.316574in}{3.074174in}}%
\pgfpathcurveto{\pgfqpoint{2.316574in}{3.082411in}}{\pgfqpoint{2.313302in}{3.090311in}}{\pgfqpoint{2.307478in}{3.096135in}}%
\pgfpathcurveto{\pgfqpoint{2.301654in}{3.101959in}}{\pgfqpoint{2.293754in}{3.105231in}}{\pgfqpoint{2.285518in}{3.105231in}}%
\pgfpathcurveto{\pgfqpoint{2.277282in}{3.105231in}}{\pgfqpoint{2.269381in}{3.101959in}}{\pgfqpoint{2.263558in}{3.096135in}}%
\pgfpathcurveto{\pgfqpoint{2.257734in}{3.090311in}}{\pgfqpoint{2.254461in}{3.082411in}}{\pgfqpoint{2.254461in}{3.074174in}}%
\pgfpathcurveto{\pgfqpoint{2.254461in}{3.065938in}}{\pgfqpoint{2.257734in}{3.058038in}}{\pgfqpoint{2.263558in}{3.052214in}}%
\pgfpathcurveto{\pgfqpoint{2.269381in}{3.046390in}}{\pgfqpoint{2.277282in}{3.043118in}}{\pgfqpoint{2.285518in}{3.043118in}}%
\pgfpathclose%
\pgfusepath{stroke,fill}%
\end{pgfscope}%
\begin{pgfscope}%
\pgfpathrectangle{\pgfqpoint{0.100000in}{0.220728in}}{\pgfqpoint{3.696000in}{3.696000in}}%
\pgfusepath{clip}%
\pgfsetbuttcap%
\pgfsetroundjoin%
\definecolor{currentfill}{rgb}{0.121569,0.466667,0.705882}%
\pgfsetfillcolor{currentfill}%
\pgfsetfillopacity{0.441792}%
\pgfsetlinewidth{1.003750pt}%
\definecolor{currentstroke}{rgb}{0.121569,0.466667,0.705882}%
\pgfsetstrokecolor{currentstroke}%
\pgfsetstrokeopacity{0.441792}%
\pgfsetdash{}{0pt}%
\pgfpathmoveto{\pgfqpoint{1.200484in}{2.215915in}}%
\pgfpathcurveto{\pgfqpoint{1.208720in}{2.215915in}}{\pgfqpoint{1.216620in}{2.219187in}}{\pgfqpoint{1.222444in}{2.225011in}}%
\pgfpathcurveto{\pgfqpoint{1.228268in}{2.230835in}}{\pgfqpoint{1.231541in}{2.238735in}}{\pgfqpoint{1.231541in}{2.246971in}}%
\pgfpathcurveto{\pgfqpoint{1.231541in}{2.255208in}}{\pgfqpoint{1.228268in}{2.263108in}}{\pgfqpoint{1.222444in}{2.268932in}}%
\pgfpathcurveto{\pgfqpoint{1.216620in}{2.274756in}}{\pgfqpoint{1.208720in}{2.278028in}}{\pgfqpoint{1.200484in}{2.278028in}}%
\pgfpathcurveto{\pgfqpoint{1.192248in}{2.278028in}}{\pgfqpoint{1.184348in}{2.274756in}}{\pgfqpoint{1.178524in}{2.268932in}}%
\pgfpathcurveto{\pgfqpoint{1.172700in}{2.263108in}}{\pgfqpoint{1.169428in}{2.255208in}}{\pgfqpoint{1.169428in}{2.246971in}}%
\pgfpathcurveto{\pgfqpoint{1.169428in}{2.238735in}}{\pgfqpoint{1.172700in}{2.230835in}}{\pgfqpoint{1.178524in}{2.225011in}}%
\pgfpathcurveto{\pgfqpoint{1.184348in}{2.219187in}}{\pgfqpoint{1.192248in}{2.215915in}}{\pgfqpoint{1.200484in}{2.215915in}}%
\pgfpathclose%
\pgfusepath{stroke,fill}%
\end{pgfscope}%
\begin{pgfscope}%
\pgfpathrectangle{\pgfqpoint{0.100000in}{0.220728in}}{\pgfqpoint{3.696000in}{3.696000in}}%
\pgfusepath{clip}%
\pgfsetbuttcap%
\pgfsetroundjoin%
\definecolor{currentfill}{rgb}{0.121569,0.466667,0.705882}%
\pgfsetfillcolor{currentfill}%
\pgfsetfillopacity{0.445618}%
\pgfsetlinewidth{1.003750pt}%
\definecolor{currentstroke}{rgb}{0.121569,0.466667,0.705882}%
\pgfsetstrokecolor{currentstroke}%
\pgfsetstrokeopacity{0.445618}%
\pgfsetdash{}{0pt}%
\pgfpathmoveto{\pgfqpoint{2.322475in}{3.037529in}}%
\pgfpathcurveto{\pgfqpoint{2.330712in}{3.037529in}}{\pgfqpoint{2.338612in}{3.040802in}}{\pgfqpoint{2.344436in}{3.046625in}}%
\pgfpathcurveto{\pgfqpoint{2.350260in}{3.052449in}}{\pgfqpoint{2.353532in}{3.060349in}}{\pgfqpoint{2.353532in}{3.068586in}}%
\pgfpathcurveto{\pgfqpoint{2.353532in}{3.076822in}}{\pgfqpoint{2.350260in}{3.084722in}}{\pgfqpoint{2.344436in}{3.090546in}}%
\pgfpathcurveto{\pgfqpoint{2.338612in}{3.096370in}}{\pgfqpoint{2.330712in}{3.099642in}}{\pgfqpoint{2.322475in}{3.099642in}}%
\pgfpathcurveto{\pgfqpoint{2.314239in}{3.099642in}}{\pgfqpoint{2.306339in}{3.096370in}}{\pgfqpoint{2.300515in}{3.090546in}}%
\pgfpathcurveto{\pgfqpoint{2.294691in}{3.084722in}}{\pgfqpoint{2.291419in}{3.076822in}}{\pgfqpoint{2.291419in}{3.068586in}}%
\pgfpathcurveto{\pgfqpoint{2.291419in}{3.060349in}}{\pgfqpoint{2.294691in}{3.052449in}}{\pgfqpoint{2.300515in}{3.046625in}}%
\pgfpathcurveto{\pgfqpoint{2.306339in}{3.040802in}}{\pgfqpoint{2.314239in}{3.037529in}}{\pgfqpoint{2.322475in}{3.037529in}}%
\pgfpathclose%
\pgfusepath{stroke,fill}%
\end{pgfscope}%
\begin{pgfscope}%
\pgfpathrectangle{\pgfqpoint{0.100000in}{0.220728in}}{\pgfqpoint{3.696000in}{3.696000in}}%
\pgfusepath{clip}%
\pgfsetbuttcap%
\pgfsetroundjoin%
\definecolor{currentfill}{rgb}{0.121569,0.466667,0.705882}%
\pgfsetfillcolor{currentfill}%
\pgfsetfillopacity{0.446898}%
\pgfsetlinewidth{1.003750pt}%
\definecolor{currentstroke}{rgb}{0.121569,0.466667,0.705882}%
\pgfsetstrokecolor{currentstroke}%
\pgfsetstrokeopacity{0.446898}%
\pgfsetdash{}{0pt}%
\pgfpathmoveto{\pgfqpoint{1.190520in}{2.187380in}}%
\pgfpathcurveto{\pgfqpoint{1.198757in}{2.187380in}}{\pgfqpoint{1.206657in}{2.190652in}}{\pgfqpoint{1.212481in}{2.196476in}}%
\pgfpathcurveto{\pgfqpoint{1.218305in}{2.202300in}}{\pgfqpoint{1.221577in}{2.210200in}}{\pgfqpoint{1.221577in}{2.218437in}}%
\pgfpathcurveto{\pgfqpoint{1.221577in}{2.226673in}}{\pgfqpoint{1.218305in}{2.234573in}}{\pgfqpoint{1.212481in}{2.240397in}}%
\pgfpathcurveto{\pgfqpoint{1.206657in}{2.246221in}}{\pgfqpoint{1.198757in}{2.249493in}}{\pgfqpoint{1.190520in}{2.249493in}}%
\pgfpathcurveto{\pgfqpoint{1.182284in}{2.249493in}}{\pgfqpoint{1.174384in}{2.246221in}}{\pgfqpoint{1.168560in}{2.240397in}}%
\pgfpathcurveto{\pgfqpoint{1.162736in}{2.234573in}}{\pgfqpoint{1.159464in}{2.226673in}}{\pgfqpoint{1.159464in}{2.218437in}}%
\pgfpathcurveto{\pgfqpoint{1.159464in}{2.210200in}}{\pgfqpoint{1.162736in}{2.202300in}}{\pgfqpoint{1.168560in}{2.196476in}}%
\pgfpathcurveto{\pgfqpoint{1.174384in}{2.190652in}}{\pgfqpoint{1.182284in}{2.187380in}}{\pgfqpoint{1.190520in}{2.187380in}}%
\pgfpathclose%
\pgfusepath{stroke,fill}%
\end{pgfscope}%
\begin{pgfscope}%
\pgfpathrectangle{\pgfqpoint{0.100000in}{0.220728in}}{\pgfqpoint{3.696000in}{3.696000in}}%
\pgfusepath{clip}%
\pgfsetbuttcap%
\pgfsetroundjoin%
\definecolor{currentfill}{rgb}{0.121569,0.466667,0.705882}%
\pgfsetfillcolor{currentfill}%
\pgfsetfillopacity{0.449162}%
\pgfsetlinewidth{1.003750pt}%
\definecolor{currentstroke}{rgb}{0.121569,0.466667,0.705882}%
\pgfsetstrokecolor{currentstroke}%
\pgfsetstrokeopacity{0.449162}%
\pgfsetdash{}{0pt}%
\pgfpathmoveto{\pgfqpoint{1.180608in}{2.169118in}}%
\pgfpathcurveto{\pgfqpoint{1.188844in}{2.169118in}}{\pgfqpoint{1.196744in}{2.172390in}}{\pgfqpoint{1.202568in}{2.178214in}}%
\pgfpathcurveto{\pgfqpoint{1.208392in}{2.184038in}}{\pgfqpoint{1.211664in}{2.191938in}}{\pgfqpoint{1.211664in}{2.200174in}}%
\pgfpathcurveto{\pgfqpoint{1.211664in}{2.208411in}}{\pgfqpoint{1.208392in}{2.216311in}}{\pgfqpoint{1.202568in}{2.222135in}}%
\pgfpathcurveto{\pgfqpoint{1.196744in}{2.227959in}}{\pgfqpoint{1.188844in}{2.231231in}}{\pgfqpoint{1.180608in}{2.231231in}}%
\pgfpathcurveto{\pgfqpoint{1.172372in}{2.231231in}}{\pgfqpoint{1.164472in}{2.227959in}}{\pgfqpoint{1.158648in}{2.222135in}}%
\pgfpathcurveto{\pgfqpoint{1.152824in}{2.216311in}}{\pgfqpoint{1.149551in}{2.208411in}}{\pgfqpoint{1.149551in}{2.200174in}}%
\pgfpathcurveto{\pgfqpoint{1.149551in}{2.191938in}}{\pgfqpoint{1.152824in}{2.184038in}}{\pgfqpoint{1.158648in}{2.178214in}}%
\pgfpathcurveto{\pgfqpoint{1.164472in}{2.172390in}}{\pgfqpoint{1.172372in}{2.169118in}}{\pgfqpoint{1.180608in}{2.169118in}}%
\pgfpathclose%
\pgfusepath{stroke,fill}%
\end{pgfscope}%
\begin{pgfscope}%
\pgfpathrectangle{\pgfqpoint{0.100000in}{0.220728in}}{\pgfqpoint{3.696000in}{3.696000in}}%
\pgfusepath{clip}%
\pgfsetbuttcap%
\pgfsetroundjoin%
\definecolor{currentfill}{rgb}{0.121569,0.466667,0.705882}%
\pgfsetfillcolor{currentfill}%
\pgfsetfillopacity{0.450225}%
\pgfsetlinewidth{1.003750pt}%
\definecolor{currentstroke}{rgb}{0.121569,0.466667,0.705882}%
\pgfsetstrokecolor{currentstroke}%
\pgfsetstrokeopacity{0.450225}%
\pgfsetdash{}{0pt}%
\pgfpathmoveto{\pgfqpoint{2.341815in}{3.037364in}}%
\pgfpathcurveto{\pgfqpoint{2.350051in}{3.037364in}}{\pgfqpoint{2.357951in}{3.040636in}}{\pgfqpoint{2.363775in}{3.046460in}}%
\pgfpathcurveto{\pgfqpoint{2.369599in}{3.052284in}}{\pgfqpoint{2.372872in}{3.060184in}}{\pgfqpoint{2.372872in}{3.068420in}}%
\pgfpathcurveto{\pgfqpoint{2.372872in}{3.076657in}}{\pgfqpoint{2.369599in}{3.084557in}}{\pgfqpoint{2.363775in}{3.090380in}}%
\pgfpathcurveto{\pgfqpoint{2.357951in}{3.096204in}}{\pgfqpoint{2.350051in}{3.099477in}}{\pgfqpoint{2.341815in}{3.099477in}}%
\pgfpathcurveto{\pgfqpoint{2.333579in}{3.099477in}}{\pgfqpoint{2.325679in}{3.096204in}}{\pgfqpoint{2.319855in}{3.090380in}}%
\pgfpathcurveto{\pgfqpoint{2.314031in}{3.084557in}}{\pgfqpoint{2.310759in}{3.076657in}}{\pgfqpoint{2.310759in}{3.068420in}}%
\pgfpathcurveto{\pgfqpoint{2.310759in}{3.060184in}}{\pgfqpoint{2.314031in}{3.052284in}}{\pgfqpoint{2.319855in}{3.046460in}}%
\pgfpathcurveto{\pgfqpoint{2.325679in}{3.040636in}}{\pgfqpoint{2.333579in}{3.037364in}}{\pgfqpoint{2.341815in}{3.037364in}}%
\pgfpathclose%
\pgfusepath{stroke,fill}%
\end{pgfscope}%
\begin{pgfscope}%
\pgfpathrectangle{\pgfqpoint{0.100000in}{0.220728in}}{\pgfqpoint{3.696000in}{3.696000in}}%
\pgfusepath{clip}%
\pgfsetbuttcap%
\pgfsetroundjoin%
\definecolor{currentfill}{rgb}{0.121569,0.466667,0.705882}%
\pgfsetfillcolor{currentfill}%
\pgfsetfillopacity{0.450574}%
\pgfsetlinewidth{1.003750pt}%
\definecolor{currentstroke}{rgb}{0.121569,0.466667,0.705882}%
\pgfsetstrokecolor{currentstroke}%
\pgfsetstrokeopacity{0.450574}%
\pgfsetdash{}{0pt}%
\pgfpathmoveto{\pgfqpoint{1.176780in}{2.160915in}}%
\pgfpathcurveto{\pgfqpoint{1.185017in}{2.160915in}}{\pgfqpoint{1.192917in}{2.164187in}}{\pgfqpoint{1.198741in}{2.170011in}}%
\pgfpathcurveto{\pgfqpoint{1.204565in}{2.175835in}}{\pgfqpoint{1.207837in}{2.183735in}}{\pgfqpoint{1.207837in}{2.191971in}}%
\pgfpathcurveto{\pgfqpoint{1.207837in}{2.200207in}}{\pgfqpoint{1.204565in}{2.208107in}}{\pgfqpoint{1.198741in}{2.213931in}}%
\pgfpathcurveto{\pgfqpoint{1.192917in}{2.219755in}}{\pgfqpoint{1.185017in}{2.223028in}}{\pgfqpoint{1.176780in}{2.223028in}}%
\pgfpathcurveto{\pgfqpoint{1.168544in}{2.223028in}}{\pgfqpoint{1.160644in}{2.219755in}}{\pgfqpoint{1.154820in}{2.213931in}}%
\pgfpathcurveto{\pgfqpoint{1.148996in}{2.208107in}}{\pgfqpoint{1.145724in}{2.200207in}}{\pgfqpoint{1.145724in}{2.191971in}}%
\pgfpathcurveto{\pgfqpoint{1.145724in}{2.183735in}}{\pgfqpoint{1.148996in}{2.175835in}}{\pgfqpoint{1.154820in}{2.170011in}}%
\pgfpathcurveto{\pgfqpoint{1.160644in}{2.164187in}}{\pgfqpoint{1.168544in}{2.160915in}}{\pgfqpoint{1.176780in}{2.160915in}}%
\pgfpathclose%
\pgfusepath{stroke,fill}%
\end{pgfscope}%
\begin{pgfscope}%
\pgfpathrectangle{\pgfqpoint{0.100000in}{0.220728in}}{\pgfqpoint{3.696000in}{3.696000in}}%
\pgfusepath{clip}%
\pgfsetbuttcap%
\pgfsetroundjoin%
\definecolor{currentfill}{rgb}{0.121569,0.466667,0.705882}%
\pgfsetfillcolor{currentfill}%
\pgfsetfillopacity{0.450928}%
\pgfsetlinewidth{1.003750pt}%
\definecolor{currentstroke}{rgb}{0.121569,0.466667,0.705882}%
\pgfsetstrokecolor{currentstroke}%
\pgfsetstrokeopacity{0.450928}%
\pgfsetdash{}{0pt}%
\pgfpathmoveto{\pgfqpoint{1.175713in}{2.158615in}}%
\pgfpathcurveto{\pgfqpoint{1.183949in}{2.158615in}}{\pgfqpoint{1.191849in}{2.161888in}}{\pgfqpoint{1.197673in}{2.167712in}}%
\pgfpathcurveto{\pgfqpoint{1.203497in}{2.173536in}}{\pgfqpoint{1.206769in}{2.181436in}}{\pgfqpoint{1.206769in}{2.189672in}}%
\pgfpathcurveto{\pgfqpoint{1.206769in}{2.197908in}}{\pgfqpoint{1.203497in}{2.205808in}}{\pgfqpoint{1.197673in}{2.211632in}}%
\pgfpathcurveto{\pgfqpoint{1.191849in}{2.217456in}}{\pgfqpoint{1.183949in}{2.220728in}}{\pgfqpoint{1.175713in}{2.220728in}}%
\pgfpathcurveto{\pgfqpoint{1.167476in}{2.220728in}}{\pgfqpoint{1.159576in}{2.217456in}}{\pgfqpoint{1.153752in}{2.211632in}}%
\pgfpathcurveto{\pgfqpoint{1.147929in}{2.205808in}}{\pgfqpoint{1.144656in}{2.197908in}}{\pgfqpoint{1.144656in}{2.189672in}}%
\pgfpathcurveto{\pgfqpoint{1.144656in}{2.181436in}}{\pgfqpoint{1.147929in}{2.173536in}}{\pgfqpoint{1.153752in}{2.167712in}}%
\pgfpathcurveto{\pgfqpoint{1.159576in}{2.161888in}}{\pgfqpoint{1.167476in}{2.158615in}}{\pgfqpoint{1.175713in}{2.158615in}}%
\pgfpathclose%
\pgfusepath{stroke,fill}%
\end{pgfscope}%
\begin{pgfscope}%
\pgfpathrectangle{\pgfqpoint{0.100000in}{0.220728in}}{\pgfqpoint{3.696000in}{3.696000in}}%
\pgfusepath{clip}%
\pgfsetbuttcap%
\pgfsetroundjoin%
\definecolor{currentfill}{rgb}{0.121569,0.466667,0.705882}%
\pgfsetfillcolor{currentfill}%
\pgfsetfillopacity{0.451618}%
\pgfsetlinewidth{1.003750pt}%
\definecolor{currentstroke}{rgb}{0.121569,0.466667,0.705882}%
\pgfsetstrokecolor{currentstroke}%
\pgfsetstrokeopacity{0.451618}%
\pgfsetdash{}{0pt}%
\pgfpathmoveto{\pgfqpoint{1.173722in}{2.154700in}}%
\pgfpathcurveto{\pgfqpoint{1.181958in}{2.154700in}}{\pgfqpoint{1.189858in}{2.157973in}}{\pgfqpoint{1.195682in}{2.163797in}}%
\pgfpathcurveto{\pgfqpoint{1.201506in}{2.169620in}}{\pgfqpoint{1.204778in}{2.177521in}}{\pgfqpoint{1.204778in}{2.185757in}}%
\pgfpathcurveto{\pgfqpoint{1.204778in}{2.193993in}}{\pgfqpoint{1.201506in}{2.201893in}}{\pgfqpoint{1.195682in}{2.207717in}}%
\pgfpathcurveto{\pgfqpoint{1.189858in}{2.213541in}}{\pgfqpoint{1.181958in}{2.216813in}}{\pgfqpoint{1.173722in}{2.216813in}}%
\pgfpathcurveto{\pgfqpoint{1.165485in}{2.216813in}}{\pgfqpoint{1.157585in}{2.213541in}}{\pgfqpoint{1.151761in}{2.207717in}}%
\pgfpathcurveto{\pgfqpoint{1.145937in}{2.201893in}}{\pgfqpoint{1.142665in}{2.193993in}}{\pgfqpoint{1.142665in}{2.185757in}}%
\pgfpathcurveto{\pgfqpoint{1.142665in}{2.177521in}}{\pgfqpoint{1.145937in}{2.169620in}}{\pgfqpoint{1.151761in}{2.163797in}}%
\pgfpathcurveto{\pgfqpoint{1.157585in}{2.157973in}}{\pgfqpoint{1.165485in}{2.154700in}}{\pgfqpoint{1.173722in}{2.154700in}}%
\pgfpathclose%
\pgfusepath{stroke,fill}%
\end{pgfscope}%
\begin{pgfscope}%
\pgfpathrectangle{\pgfqpoint{0.100000in}{0.220728in}}{\pgfqpoint{3.696000in}{3.696000in}}%
\pgfusepath{clip}%
\pgfsetbuttcap%
\pgfsetroundjoin%
\definecolor{currentfill}{rgb}{0.121569,0.466667,0.705882}%
\pgfsetfillcolor{currentfill}%
\pgfsetfillopacity{0.452112}%
\pgfsetlinewidth{1.003750pt}%
\definecolor{currentstroke}{rgb}{0.121569,0.466667,0.705882}%
\pgfsetstrokecolor{currentstroke}%
\pgfsetstrokeopacity{0.452112}%
\pgfsetdash{}{0pt}%
\pgfpathmoveto{\pgfqpoint{2.352801in}{3.035179in}}%
\pgfpathcurveto{\pgfqpoint{2.361037in}{3.035179in}}{\pgfqpoint{2.368937in}{3.038452in}}{\pgfqpoint{2.374761in}{3.044276in}}%
\pgfpathcurveto{\pgfqpoint{2.380585in}{3.050100in}}{\pgfqpoint{2.383857in}{3.058000in}}{\pgfqpoint{2.383857in}{3.066236in}}%
\pgfpathcurveto{\pgfqpoint{2.383857in}{3.074472in}}{\pgfqpoint{2.380585in}{3.082372in}}{\pgfqpoint{2.374761in}{3.088196in}}%
\pgfpathcurveto{\pgfqpoint{2.368937in}{3.094020in}}{\pgfqpoint{2.361037in}{3.097292in}}{\pgfqpoint{2.352801in}{3.097292in}}%
\pgfpathcurveto{\pgfqpoint{2.344565in}{3.097292in}}{\pgfqpoint{2.336664in}{3.094020in}}{\pgfqpoint{2.330841in}{3.088196in}}%
\pgfpathcurveto{\pgfqpoint{2.325017in}{3.082372in}}{\pgfqpoint{2.321744in}{3.074472in}}{\pgfqpoint{2.321744in}{3.066236in}}%
\pgfpathcurveto{\pgfqpoint{2.321744in}{3.058000in}}{\pgfqpoint{2.325017in}{3.050100in}}{\pgfqpoint{2.330841in}{3.044276in}}%
\pgfpathcurveto{\pgfqpoint{2.336664in}{3.038452in}}{\pgfqpoint{2.344565in}{3.035179in}}{\pgfqpoint{2.352801in}{3.035179in}}%
\pgfpathclose%
\pgfusepath{stroke,fill}%
\end{pgfscope}%
\begin{pgfscope}%
\pgfpathrectangle{\pgfqpoint{0.100000in}{0.220728in}}{\pgfqpoint{3.696000in}{3.696000in}}%
\pgfusepath{clip}%
\pgfsetbuttcap%
\pgfsetroundjoin%
\definecolor{currentfill}{rgb}{0.121569,0.466667,0.705882}%
\pgfsetfillcolor{currentfill}%
\pgfsetfillopacity{0.452904}%
\pgfsetlinewidth{1.003750pt}%
\definecolor{currentstroke}{rgb}{0.121569,0.466667,0.705882}%
\pgfsetstrokecolor{currentstroke}%
\pgfsetstrokeopacity{0.452904}%
\pgfsetdash{}{0pt}%
\pgfpathmoveto{\pgfqpoint{1.170631in}{2.147250in}}%
\pgfpathcurveto{\pgfqpoint{1.178868in}{2.147250in}}{\pgfqpoint{1.186768in}{2.150522in}}{\pgfqpoint{1.192592in}{2.156346in}}%
\pgfpathcurveto{\pgfqpoint{1.198416in}{2.162170in}}{\pgfqpoint{1.201688in}{2.170070in}}{\pgfqpoint{1.201688in}{2.178306in}}%
\pgfpathcurveto{\pgfqpoint{1.201688in}{2.186542in}}{\pgfqpoint{1.198416in}{2.194442in}}{\pgfqpoint{1.192592in}{2.200266in}}%
\pgfpathcurveto{\pgfqpoint{1.186768in}{2.206090in}}{\pgfqpoint{1.178868in}{2.209363in}}{\pgfqpoint{1.170631in}{2.209363in}}%
\pgfpathcurveto{\pgfqpoint{1.162395in}{2.209363in}}{\pgfqpoint{1.154495in}{2.206090in}}{\pgfqpoint{1.148671in}{2.200266in}}%
\pgfpathcurveto{\pgfqpoint{1.142847in}{2.194442in}}{\pgfqpoint{1.139575in}{2.186542in}}{\pgfqpoint{1.139575in}{2.178306in}}%
\pgfpathcurveto{\pgfqpoint{1.139575in}{2.170070in}}{\pgfqpoint{1.142847in}{2.162170in}}{\pgfqpoint{1.148671in}{2.156346in}}%
\pgfpathcurveto{\pgfqpoint{1.154495in}{2.150522in}}{\pgfqpoint{1.162395in}{2.147250in}}{\pgfqpoint{1.170631in}{2.147250in}}%
\pgfpathclose%
\pgfusepath{stroke,fill}%
\end{pgfscope}%
\begin{pgfscope}%
\pgfpathrectangle{\pgfqpoint{0.100000in}{0.220728in}}{\pgfqpoint{3.696000in}{3.696000in}}%
\pgfusepath{clip}%
\pgfsetbuttcap%
\pgfsetroundjoin%
\definecolor{currentfill}{rgb}{0.121569,0.466667,0.705882}%
\pgfsetfillcolor{currentfill}%
\pgfsetfillopacity{0.453468}%
\pgfsetlinewidth{1.003750pt}%
\definecolor{currentstroke}{rgb}{0.121569,0.466667,0.705882}%
\pgfsetstrokecolor{currentstroke}%
\pgfsetstrokeopacity{0.453468}%
\pgfsetdash{}{0pt}%
\pgfpathmoveto{\pgfqpoint{2.358490in}{3.034408in}}%
\pgfpathcurveto{\pgfqpoint{2.366727in}{3.034408in}}{\pgfqpoint{2.374627in}{3.037680in}}{\pgfqpoint{2.380451in}{3.043504in}}%
\pgfpathcurveto{\pgfqpoint{2.386275in}{3.049328in}}{\pgfqpoint{2.389547in}{3.057228in}}{\pgfqpoint{2.389547in}{3.065464in}}%
\pgfpathcurveto{\pgfqpoint{2.389547in}{3.073701in}}{\pgfqpoint{2.386275in}{3.081601in}}{\pgfqpoint{2.380451in}{3.087425in}}%
\pgfpathcurveto{\pgfqpoint{2.374627in}{3.093249in}}{\pgfqpoint{2.366727in}{3.096521in}}{\pgfqpoint{2.358490in}{3.096521in}}%
\pgfpathcurveto{\pgfqpoint{2.350254in}{3.096521in}}{\pgfqpoint{2.342354in}{3.093249in}}{\pgfqpoint{2.336530in}{3.087425in}}%
\pgfpathcurveto{\pgfqpoint{2.330706in}{3.081601in}}{\pgfqpoint{2.327434in}{3.073701in}}{\pgfqpoint{2.327434in}{3.065464in}}%
\pgfpathcurveto{\pgfqpoint{2.327434in}{3.057228in}}{\pgfqpoint{2.330706in}{3.049328in}}{\pgfqpoint{2.336530in}{3.043504in}}%
\pgfpathcurveto{\pgfqpoint{2.342354in}{3.037680in}}{\pgfqpoint{2.350254in}{3.034408in}}{\pgfqpoint{2.358490in}{3.034408in}}%
\pgfpathclose%
\pgfusepath{stroke,fill}%
\end{pgfscope}%
\begin{pgfscope}%
\pgfpathrectangle{\pgfqpoint{0.100000in}{0.220728in}}{\pgfqpoint{3.696000in}{3.696000in}}%
\pgfusepath{clip}%
\pgfsetbuttcap%
\pgfsetroundjoin%
\definecolor{currentfill}{rgb}{0.121569,0.466667,0.705882}%
\pgfsetfillcolor{currentfill}%
\pgfsetfillopacity{0.454160}%
\pgfsetlinewidth{1.003750pt}%
\definecolor{currentstroke}{rgb}{0.121569,0.466667,0.705882}%
\pgfsetstrokecolor{currentstroke}%
\pgfsetstrokeopacity{0.454160}%
\pgfsetdash{}{0pt}%
\pgfpathmoveto{\pgfqpoint{2.361652in}{3.033809in}}%
\pgfpathcurveto{\pgfqpoint{2.369888in}{3.033809in}}{\pgfqpoint{2.377788in}{3.037081in}}{\pgfqpoint{2.383612in}{3.042905in}}%
\pgfpathcurveto{\pgfqpoint{2.389436in}{3.048729in}}{\pgfqpoint{2.392708in}{3.056629in}}{\pgfqpoint{2.392708in}{3.064865in}}%
\pgfpathcurveto{\pgfqpoint{2.392708in}{3.073101in}}{\pgfqpoint{2.389436in}{3.081001in}}{\pgfqpoint{2.383612in}{3.086825in}}%
\pgfpathcurveto{\pgfqpoint{2.377788in}{3.092649in}}{\pgfqpoint{2.369888in}{3.095922in}}{\pgfqpoint{2.361652in}{3.095922in}}%
\pgfpathcurveto{\pgfqpoint{2.353415in}{3.095922in}}{\pgfqpoint{2.345515in}{3.092649in}}{\pgfqpoint{2.339691in}{3.086825in}}%
\pgfpathcurveto{\pgfqpoint{2.333868in}{3.081001in}}{\pgfqpoint{2.330595in}{3.073101in}}{\pgfqpoint{2.330595in}{3.064865in}}%
\pgfpathcurveto{\pgfqpoint{2.330595in}{3.056629in}}{\pgfqpoint{2.333868in}{3.048729in}}{\pgfqpoint{2.339691in}{3.042905in}}%
\pgfpathcurveto{\pgfqpoint{2.345515in}{3.037081in}}{\pgfqpoint{2.353415in}{3.033809in}}{\pgfqpoint{2.361652in}{3.033809in}}%
\pgfpathclose%
\pgfusepath{stroke,fill}%
\end{pgfscope}%
\begin{pgfscope}%
\pgfpathrectangle{\pgfqpoint{0.100000in}{0.220728in}}{\pgfqpoint{3.696000in}{3.696000in}}%
\pgfusepath{clip}%
\pgfsetbuttcap%
\pgfsetroundjoin%
\definecolor{currentfill}{rgb}{0.121569,0.466667,0.705882}%
\pgfsetfillcolor{currentfill}%
\pgfsetfillopacity{0.454796}%
\pgfsetlinewidth{1.003750pt}%
\definecolor{currentstroke}{rgb}{0.121569,0.466667,0.705882}%
\pgfsetstrokecolor{currentstroke}%
\pgfsetstrokeopacity{0.454796}%
\pgfsetdash{}{0pt}%
\pgfpathmoveto{\pgfqpoint{1.163123in}{2.133466in}}%
\pgfpathcurveto{\pgfqpoint{1.171359in}{2.133466in}}{\pgfqpoint{1.179259in}{2.136738in}}{\pgfqpoint{1.185083in}{2.142562in}}%
\pgfpathcurveto{\pgfqpoint{1.190907in}{2.148386in}}{\pgfqpoint{1.194179in}{2.156286in}}{\pgfqpoint{1.194179in}{2.164522in}}%
\pgfpathcurveto{\pgfqpoint{1.194179in}{2.172759in}}{\pgfqpoint{1.190907in}{2.180659in}}{\pgfqpoint{1.185083in}{2.186483in}}%
\pgfpathcurveto{\pgfqpoint{1.179259in}{2.192307in}}{\pgfqpoint{1.171359in}{2.195579in}}{\pgfqpoint{1.163123in}{2.195579in}}%
\pgfpathcurveto{\pgfqpoint{1.154886in}{2.195579in}}{\pgfqpoint{1.146986in}{2.192307in}}{\pgfqpoint{1.141162in}{2.186483in}}%
\pgfpathcurveto{\pgfqpoint{1.135338in}{2.180659in}}{\pgfqpoint{1.132066in}{2.172759in}}{\pgfqpoint{1.132066in}{2.164522in}}%
\pgfpathcurveto{\pgfqpoint{1.132066in}{2.156286in}}{\pgfqpoint{1.135338in}{2.148386in}}{\pgfqpoint{1.141162in}{2.142562in}}%
\pgfpathcurveto{\pgfqpoint{1.146986in}{2.136738in}}{\pgfqpoint{1.154886in}{2.133466in}}{\pgfqpoint{1.163123in}{2.133466in}}%
\pgfpathclose%
\pgfusepath{stroke,fill}%
\end{pgfscope}%
\begin{pgfscope}%
\pgfpathrectangle{\pgfqpoint{0.100000in}{0.220728in}}{\pgfqpoint{3.696000in}{3.696000in}}%
\pgfusepath{clip}%
\pgfsetbuttcap%
\pgfsetroundjoin%
\definecolor{currentfill}{rgb}{0.121569,0.466667,0.705882}%
\pgfsetfillcolor{currentfill}%
\pgfsetfillopacity{0.455487}%
\pgfsetlinewidth{1.003750pt}%
\definecolor{currentstroke}{rgb}{0.121569,0.466667,0.705882}%
\pgfsetstrokecolor{currentstroke}%
\pgfsetstrokeopacity{0.455487}%
\pgfsetdash{}{0pt}%
\pgfpathmoveto{\pgfqpoint{2.367293in}{3.033466in}}%
\pgfpathcurveto{\pgfqpoint{2.375529in}{3.033466in}}{\pgfqpoint{2.383429in}{3.036739in}}{\pgfqpoint{2.389253in}{3.042563in}}%
\pgfpathcurveto{\pgfqpoint{2.395077in}{3.048386in}}{\pgfqpoint{2.398349in}{3.056287in}}{\pgfqpoint{2.398349in}{3.064523in}}%
\pgfpathcurveto{\pgfqpoint{2.398349in}{3.072759in}}{\pgfqpoint{2.395077in}{3.080659in}}{\pgfqpoint{2.389253in}{3.086483in}}%
\pgfpathcurveto{\pgfqpoint{2.383429in}{3.092307in}}{\pgfqpoint{2.375529in}{3.095579in}}{\pgfqpoint{2.367293in}{3.095579in}}%
\pgfpathcurveto{\pgfqpoint{2.359056in}{3.095579in}}{\pgfqpoint{2.351156in}{3.092307in}}{\pgfqpoint{2.345332in}{3.086483in}}%
\pgfpathcurveto{\pgfqpoint{2.339509in}{3.080659in}}{\pgfqpoint{2.336236in}{3.072759in}}{\pgfqpoint{2.336236in}{3.064523in}}%
\pgfpathcurveto{\pgfqpoint{2.336236in}{3.056287in}}{\pgfqpoint{2.339509in}{3.048386in}}{\pgfqpoint{2.345332in}{3.042563in}}%
\pgfpathcurveto{\pgfqpoint{2.351156in}{3.036739in}}{\pgfqpoint{2.359056in}{3.033466in}}{\pgfqpoint{2.367293in}{3.033466in}}%
\pgfpathclose%
\pgfusepath{stroke,fill}%
\end{pgfscope}%
\begin{pgfscope}%
\pgfpathrectangle{\pgfqpoint{0.100000in}{0.220728in}}{\pgfqpoint{3.696000in}{3.696000in}}%
\pgfusepath{clip}%
\pgfsetbuttcap%
\pgfsetroundjoin%
\definecolor{currentfill}{rgb}{0.121569,0.466667,0.705882}%
\pgfsetfillcolor{currentfill}%
\pgfsetfillopacity{0.457867}%
\pgfsetlinewidth{1.003750pt}%
\definecolor{currentstroke}{rgb}{0.121569,0.466667,0.705882}%
\pgfsetstrokecolor{currentstroke}%
\pgfsetstrokeopacity{0.457867}%
\pgfsetdash{}{0pt}%
\pgfpathmoveto{\pgfqpoint{2.376296in}{3.030831in}}%
\pgfpathcurveto{\pgfqpoint{2.384532in}{3.030831in}}{\pgfqpoint{2.392432in}{3.034103in}}{\pgfqpoint{2.398256in}{3.039927in}}%
\pgfpathcurveto{\pgfqpoint{2.404080in}{3.045751in}}{\pgfqpoint{2.407352in}{3.053651in}}{\pgfqpoint{2.407352in}{3.061887in}}%
\pgfpathcurveto{\pgfqpoint{2.407352in}{3.070124in}}{\pgfqpoint{2.404080in}{3.078024in}}{\pgfqpoint{2.398256in}{3.083848in}}%
\pgfpathcurveto{\pgfqpoint{2.392432in}{3.089672in}}{\pgfqpoint{2.384532in}{3.092944in}}{\pgfqpoint{2.376296in}{3.092944in}}%
\pgfpathcurveto{\pgfqpoint{2.368060in}{3.092944in}}{\pgfqpoint{2.360160in}{3.089672in}}{\pgfqpoint{2.354336in}{3.083848in}}%
\pgfpathcurveto{\pgfqpoint{2.348512in}{3.078024in}}{\pgfqpoint{2.345239in}{3.070124in}}{\pgfqpoint{2.345239in}{3.061887in}}%
\pgfpathcurveto{\pgfqpoint{2.345239in}{3.053651in}}{\pgfqpoint{2.348512in}{3.045751in}}{\pgfqpoint{2.354336in}{3.039927in}}%
\pgfpathcurveto{\pgfqpoint{2.360160in}{3.034103in}}{\pgfqpoint{2.368060in}{3.030831in}}{\pgfqpoint{2.376296in}{3.030831in}}%
\pgfpathclose%
\pgfusepath{stroke,fill}%
\end{pgfscope}%
\begin{pgfscope}%
\pgfpathrectangle{\pgfqpoint{0.100000in}{0.220728in}}{\pgfqpoint{3.696000in}{3.696000in}}%
\pgfusepath{clip}%
\pgfsetbuttcap%
\pgfsetroundjoin%
\definecolor{currentfill}{rgb}{0.121569,0.466667,0.705882}%
\pgfsetfillcolor{currentfill}%
\pgfsetfillopacity{0.459135}%
\pgfsetlinewidth{1.003750pt}%
\definecolor{currentstroke}{rgb}{0.121569,0.466667,0.705882}%
\pgfsetstrokecolor{currentstroke}%
\pgfsetstrokeopacity{0.459135}%
\pgfsetdash{}{0pt}%
\pgfpathmoveto{\pgfqpoint{1.153616in}{2.108705in}}%
\pgfpathcurveto{\pgfqpoint{1.161853in}{2.108705in}}{\pgfqpoint{1.169753in}{2.111978in}}{\pgfqpoint{1.175577in}{2.117802in}}%
\pgfpathcurveto{\pgfqpoint{1.181400in}{2.123626in}}{\pgfqpoint{1.184673in}{2.131526in}}{\pgfqpoint{1.184673in}{2.139762in}}%
\pgfpathcurveto{\pgfqpoint{1.184673in}{2.147998in}}{\pgfqpoint{1.181400in}{2.155898in}}{\pgfqpoint{1.175577in}{2.161722in}}%
\pgfpathcurveto{\pgfqpoint{1.169753in}{2.167546in}}{\pgfqpoint{1.161853in}{2.170818in}}{\pgfqpoint{1.153616in}{2.170818in}}%
\pgfpathcurveto{\pgfqpoint{1.145380in}{2.170818in}}{\pgfqpoint{1.137480in}{2.167546in}}{\pgfqpoint{1.131656in}{2.161722in}}%
\pgfpathcurveto{\pgfqpoint{1.125832in}{2.155898in}}{\pgfqpoint{1.122560in}{2.147998in}}{\pgfqpoint{1.122560in}{2.139762in}}%
\pgfpathcurveto{\pgfqpoint{1.122560in}{2.131526in}}{\pgfqpoint{1.125832in}{2.123626in}}{\pgfqpoint{1.131656in}{2.117802in}}%
\pgfpathcurveto{\pgfqpoint{1.137480in}{2.111978in}}{\pgfqpoint{1.145380in}{2.108705in}}{\pgfqpoint{1.153616in}{2.108705in}}%
\pgfpathclose%
\pgfusepath{stroke,fill}%
\end{pgfscope}%
\begin{pgfscope}%
\pgfpathrectangle{\pgfqpoint{0.100000in}{0.220728in}}{\pgfqpoint{3.696000in}{3.696000in}}%
\pgfusepath{clip}%
\pgfsetbuttcap%
\pgfsetroundjoin%
\definecolor{currentfill}{rgb}{0.121569,0.466667,0.705882}%
\pgfsetfillcolor{currentfill}%
\pgfsetfillopacity{0.461378}%
\pgfsetlinewidth{1.003750pt}%
\definecolor{currentstroke}{rgb}{0.121569,0.466667,0.705882}%
\pgfsetstrokecolor{currentstroke}%
\pgfsetstrokeopacity{0.461378}%
\pgfsetdash{}{0pt}%
\pgfpathmoveto{\pgfqpoint{2.386940in}{3.029327in}}%
\pgfpathcurveto{\pgfqpoint{2.395177in}{3.029327in}}{\pgfqpoint{2.403077in}{3.032600in}}{\pgfqpoint{2.408901in}{3.038423in}}%
\pgfpathcurveto{\pgfqpoint{2.414724in}{3.044247in}}{\pgfqpoint{2.417997in}{3.052147in}}{\pgfqpoint{2.417997in}{3.060384in}}%
\pgfpathcurveto{\pgfqpoint{2.417997in}{3.068620in}}{\pgfqpoint{2.414724in}{3.076520in}}{\pgfqpoint{2.408901in}{3.082344in}}%
\pgfpathcurveto{\pgfqpoint{2.403077in}{3.088168in}}{\pgfqpoint{2.395177in}{3.091440in}}{\pgfqpoint{2.386940in}{3.091440in}}%
\pgfpathcurveto{\pgfqpoint{2.378704in}{3.091440in}}{\pgfqpoint{2.370804in}{3.088168in}}{\pgfqpoint{2.364980in}{3.082344in}}%
\pgfpathcurveto{\pgfqpoint{2.359156in}{3.076520in}}{\pgfqpoint{2.355884in}{3.068620in}}{\pgfqpoint{2.355884in}{3.060384in}}%
\pgfpathcurveto{\pgfqpoint{2.355884in}{3.052147in}}{\pgfqpoint{2.359156in}{3.044247in}}{\pgfqpoint{2.364980in}{3.038423in}}%
\pgfpathcurveto{\pgfqpoint{2.370804in}{3.032600in}}{\pgfqpoint{2.378704in}{3.029327in}}{\pgfqpoint{2.386940in}{3.029327in}}%
\pgfpathclose%
\pgfusepath{stroke,fill}%
\end{pgfscope}%
\begin{pgfscope}%
\pgfpathrectangle{\pgfqpoint{0.100000in}{0.220728in}}{\pgfqpoint{3.696000in}{3.696000in}}%
\pgfusepath{clip}%
\pgfsetbuttcap%
\pgfsetroundjoin%
\definecolor{currentfill}{rgb}{0.121569,0.466667,0.705882}%
\pgfsetfillcolor{currentfill}%
\pgfsetfillopacity{0.465294}%
\pgfsetlinewidth{1.003750pt}%
\definecolor{currentstroke}{rgb}{0.121569,0.466667,0.705882}%
\pgfsetstrokecolor{currentstroke}%
\pgfsetstrokeopacity{0.465294}%
\pgfsetdash{}{0pt}%
\pgfpathmoveto{\pgfqpoint{2.402489in}{3.026637in}}%
\pgfpathcurveto{\pgfqpoint{2.410725in}{3.026637in}}{\pgfqpoint{2.418625in}{3.029909in}}{\pgfqpoint{2.424449in}{3.035733in}}%
\pgfpathcurveto{\pgfqpoint{2.430273in}{3.041557in}}{\pgfqpoint{2.433546in}{3.049457in}}{\pgfqpoint{2.433546in}{3.057693in}}%
\pgfpathcurveto{\pgfqpoint{2.433546in}{3.065930in}}{\pgfqpoint{2.430273in}{3.073830in}}{\pgfqpoint{2.424449in}{3.079654in}}%
\pgfpathcurveto{\pgfqpoint{2.418625in}{3.085477in}}{\pgfqpoint{2.410725in}{3.088750in}}{\pgfqpoint{2.402489in}{3.088750in}}%
\pgfpathcurveto{\pgfqpoint{2.394253in}{3.088750in}}{\pgfqpoint{2.386353in}{3.085477in}}{\pgfqpoint{2.380529in}{3.079654in}}%
\pgfpathcurveto{\pgfqpoint{2.374705in}{3.073830in}}{\pgfqpoint{2.371433in}{3.065930in}}{\pgfqpoint{2.371433in}{3.057693in}}%
\pgfpathcurveto{\pgfqpoint{2.371433in}{3.049457in}}{\pgfqpoint{2.374705in}{3.041557in}}{\pgfqpoint{2.380529in}{3.035733in}}%
\pgfpathcurveto{\pgfqpoint{2.386353in}{3.029909in}}{\pgfqpoint{2.394253in}{3.026637in}}{\pgfqpoint{2.402489in}{3.026637in}}%
\pgfpathclose%
\pgfusepath{stroke,fill}%
\end{pgfscope}%
\begin{pgfscope}%
\pgfpathrectangle{\pgfqpoint{0.100000in}{0.220728in}}{\pgfqpoint{3.696000in}{3.696000in}}%
\pgfusepath{clip}%
\pgfsetbuttcap%
\pgfsetroundjoin%
\definecolor{currentfill}{rgb}{0.121569,0.466667,0.705882}%
\pgfsetfillcolor{currentfill}%
\pgfsetfillopacity{0.465689}%
\pgfsetlinewidth{1.003750pt}%
\definecolor{currentstroke}{rgb}{0.121569,0.466667,0.705882}%
\pgfsetstrokecolor{currentstroke}%
\pgfsetstrokeopacity{0.465689}%
\pgfsetdash{}{0pt}%
\pgfpathmoveto{\pgfqpoint{1.127028in}{2.066116in}}%
\pgfpathcurveto{\pgfqpoint{1.135264in}{2.066116in}}{\pgfqpoint{1.143164in}{2.069389in}}{\pgfqpoint{1.148988in}{2.075212in}}%
\pgfpathcurveto{\pgfqpoint{1.154812in}{2.081036in}}{\pgfqpoint{1.158084in}{2.088936in}}{\pgfqpoint{1.158084in}{2.097173in}}%
\pgfpathcurveto{\pgfqpoint{1.158084in}{2.105409in}}{\pgfqpoint{1.154812in}{2.113309in}}{\pgfqpoint{1.148988in}{2.119133in}}%
\pgfpathcurveto{\pgfqpoint{1.143164in}{2.124957in}}{\pgfqpoint{1.135264in}{2.128229in}}{\pgfqpoint{1.127028in}{2.128229in}}%
\pgfpathcurveto{\pgfqpoint{1.118791in}{2.128229in}}{\pgfqpoint{1.110891in}{2.124957in}}{\pgfqpoint{1.105067in}{2.119133in}}%
\pgfpathcurveto{\pgfqpoint{1.099243in}{2.113309in}}{\pgfqpoint{1.095971in}{2.105409in}}{\pgfqpoint{1.095971in}{2.097173in}}%
\pgfpathcurveto{\pgfqpoint{1.095971in}{2.088936in}}{\pgfqpoint{1.099243in}{2.081036in}}{\pgfqpoint{1.105067in}{2.075212in}}%
\pgfpathcurveto{\pgfqpoint{1.110891in}{2.069389in}}{\pgfqpoint{1.118791in}{2.066116in}}{\pgfqpoint{1.127028in}{2.066116in}}%
\pgfpathclose%
\pgfusepath{stroke,fill}%
\end{pgfscope}%
\begin{pgfscope}%
\pgfpathrectangle{\pgfqpoint{0.100000in}{0.220728in}}{\pgfqpoint{3.696000in}{3.696000in}}%
\pgfusepath{clip}%
\pgfsetbuttcap%
\pgfsetroundjoin%
\definecolor{currentfill}{rgb}{0.121569,0.466667,0.705882}%
\pgfsetfillcolor{currentfill}%
\pgfsetfillopacity{0.467730}%
\pgfsetlinewidth{1.003750pt}%
\definecolor{currentstroke}{rgb}{0.121569,0.466667,0.705882}%
\pgfsetstrokecolor{currentstroke}%
\pgfsetstrokeopacity{0.467730}%
\pgfsetdash{}{0pt}%
\pgfpathmoveto{\pgfqpoint{2.410538in}{3.025258in}}%
\pgfpathcurveto{\pgfqpoint{2.418774in}{3.025258in}}{\pgfqpoint{2.426674in}{3.028530in}}{\pgfqpoint{2.432498in}{3.034354in}}%
\pgfpathcurveto{\pgfqpoint{2.438322in}{3.040178in}}{\pgfqpoint{2.441594in}{3.048078in}}{\pgfqpoint{2.441594in}{3.056315in}}%
\pgfpathcurveto{\pgfqpoint{2.441594in}{3.064551in}}{\pgfqpoint{2.438322in}{3.072451in}}{\pgfqpoint{2.432498in}{3.078275in}}%
\pgfpathcurveto{\pgfqpoint{2.426674in}{3.084099in}}{\pgfqpoint{2.418774in}{3.087371in}}{\pgfqpoint{2.410538in}{3.087371in}}%
\pgfpathcurveto{\pgfqpoint{2.402301in}{3.087371in}}{\pgfqpoint{2.394401in}{3.084099in}}{\pgfqpoint{2.388577in}{3.078275in}}%
\pgfpathcurveto{\pgfqpoint{2.382753in}{3.072451in}}{\pgfqpoint{2.379481in}{3.064551in}}{\pgfqpoint{2.379481in}{3.056315in}}%
\pgfpathcurveto{\pgfqpoint{2.379481in}{3.048078in}}{\pgfqpoint{2.382753in}{3.040178in}}{\pgfqpoint{2.388577in}{3.034354in}}%
\pgfpathcurveto{\pgfqpoint{2.394401in}{3.028530in}}{\pgfqpoint{2.402301in}{3.025258in}}{\pgfqpoint{2.410538in}{3.025258in}}%
\pgfpathclose%
\pgfusepath{stroke,fill}%
\end{pgfscope}%
\begin{pgfscope}%
\pgfpathrectangle{\pgfqpoint{0.100000in}{0.220728in}}{\pgfqpoint{3.696000in}{3.696000in}}%
\pgfusepath{clip}%
\pgfsetbuttcap%
\pgfsetroundjoin%
\definecolor{currentfill}{rgb}{0.121569,0.466667,0.705882}%
\pgfsetfillcolor{currentfill}%
\pgfsetfillopacity{0.470311}%
\pgfsetlinewidth{1.003750pt}%
\definecolor{currentstroke}{rgb}{0.121569,0.466667,0.705882}%
\pgfsetstrokecolor{currentstroke}%
\pgfsetstrokeopacity{0.470311}%
\pgfsetdash{}{0pt}%
\pgfpathmoveto{\pgfqpoint{2.426416in}{3.023336in}}%
\pgfpathcurveto{\pgfqpoint{2.434652in}{3.023336in}}{\pgfqpoint{2.442552in}{3.026609in}}{\pgfqpoint{2.448376in}{3.032433in}}%
\pgfpathcurveto{\pgfqpoint{2.454200in}{3.038256in}}{\pgfqpoint{2.457472in}{3.046156in}}{\pgfqpoint{2.457472in}{3.054393in}}%
\pgfpathcurveto{\pgfqpoint{2.457472in}{3.062629in}}{\pgfqpoint{2.454200in}{3.070529in}}{\pgfqpoint{2.448376in}{3.076353in}}%
\pgfpathcurveto{\pgfqpoint{2.442552in}{3.082177in}}{\pgfqpoint{2.434652in}{3.085449in}}{\pgfqpoint{2.426416in}{3.085449in}}%
\pgfpathcurveto{\pgfqpoint{2.418179in}{3.085449in}}{\pgfqpoint{2.410279in}{3.082177in}}{\pgfqpoint{2.404455in}{3.076353in}}%
\pgfpathcurveto{\pgfqpoint{2.398631in}{3.070529in}}{\pgfqpoint{2.395359in}{3.062629in}}{\pgfqpoint{2.395359in}{3.054393in}}%
\pgfpathcurveto{\pgfqpoint{2.395359in}{3.046156in}}{\pgfqpoint{2.398631in}{3.038256in}}{\pgfqpoint{2.404455in}{3.032433in}}%
\pgfpathcurveto{\pgfqpoint{2.410279in}{3.026609in}}{\pgfqpoint{2.418179in}{3.023336in}}{\pgfqpoint{2.426416in}{3.023336in}}%
\pgfpathclose%
\pgfusepath{stroke,fill}%
\end{pgfscope}%
\begin{pgfscope}%
\pgfpathrectangle{\pgfqpoint{0.100000in}{0.220728in}}{\pgfqpoint{3.696000in}{3.696000in}}%
\pgfusepath{clip}%
\pgfsetbuttcap%
\pgfsetroundjoin%
\definecolor{currentfill}{rgb}{0.121569,0.466667,0.705882}%
\pgfsetfillcolor{currentfill}%
\pgfsetfillopacity{0.472008}%
\pgfsetlinewidth{1.003750pt}%
\definecolor{currentstroke}{rgb}{0.121569,0.466667,0.705882}%
\pgfsetstrokecolor{currentstroke}%
\pgfsetstrokeopacity{0.472008}%
\pgfsetdash{}{0pt}%
\pgfpathmoveto{\pgfqpoint{1.118449in}{2.025442in}}%
\pgfpathcurveto{\pgfqpoint{1.126685in}{2.025442in}}{\pgfqpoint{1.134585in}{2.028714in}}{\pgfqpoint{1.140409in}{2.034538in}}%
\pgfpathcurveto{\pgfqpoint{1.146233in}{2.040362in}}{\pgfqpoint{1.149506in}{2.048262in}}{\pgfqpoint{1.149506in}{2.056499in}}%
\pgfpathcurveto{\pgfqpoint{1.149506in}{2.064735in}}{\pgfqpoint{1.146233in}{2.072635in}}{\pgfqpoint{1.140409in}{2.078459in}}%
\pgfpathcurveto{\pgfqpoint{1.134585in}{2.084283in}}{\pgfqpoint{1.126685in}{2.087555in}}{\pgfqpoint{1.118449in}{2.087555in}}%
\pgfpathcurveto{\pgfqpoint{1.110213in}{2.087555in}}{\pgfqpoint{1.102313in}{2.084283in}}{\pgfqpoint{1.096489in}{2.078459in}}%
\pgfpathcurveto{\pgfqpoint{1.090665in}{2.072635in}}{\pgfqpoint{1.087393in}{2.064735in}}{\pgfqpoint{1.087393in}{2.056499in}}%
\pgfpathcurveto{\pgfqpoint{1.087393in}{2.048262in}}{\pgfqpoint{1.090665in}{2.040362in}}{\pgfqpoint{1.096489in}{2.034538in}}%
\pgfpathcurveto{\pgfqpoint{1.102313in}{2.028714in}}{\pgfqpoint{1.110213in}{2.025442in}}{\pgfqpoint{1.118449in}{2.025442in}}%
\pgfpathclose%
\pgfusepath{stroke,fill}%
\end{pgfscope}%
\begin{pgfscope}%
\pgfpathrectangle{\pgfqpoint{0.100000in}{0.220728in}}{\pgfqpoint{3.696000in}{3.696000in}}%
\pgfusepath{clip}%
\pgfsetbuttcap%
\pgfsetroundjoin%
\definecolor{currentfill}{rgb}{0.121569,0.466667,0.705882}%
\pgfsetfillcolor{currentfill}%
\pgfsetfillopacity{0.475521}%
\pgfsetlinewidth{1.003750pt}%
\definecolor{currentstroke}{rgb}{0.121569,0.466667,0.705882}%
\pgfsetstrokecolor{currentstroke}%
\pgfsetstrokeopacity{0.475521}%
\pgfsetdash{}{0pt}%
\pgfpathmoveto{\pgfqpoint{1.102289in}{1.998887in}}%
\pgfpathcurveto{\pgfqpoint{1.110526in}{1.998887in}}{\pgfqpoint{1.118426in}{2.002159in}}{\pgfqpoint{1.124250in}{2.007983in}}%
\pgfpathcurveto{\pgfqpoint{1.130073in}{2.013807in}}{\pgfqpoint{1.133346in}{2.021707in}}{\pgfqpoint{1.133346in}{2.029943in}}%
\pgfpathcurveto{\pgfqpoint{1.133346in}{2.038180in}}{\pgfqpoint{1.130073in}{2.046080in}}{\pgfqpoint{1.124250in}{2.051904in}}%
\pgfpathcurveto{\pgfqpoint{1.118426in}{2.057727in}}{\pgfqpoint{1.110526in}{2.061000in}}{\pgfqpoint{1.102289in}{2.061000in}}%
\pgfpathcurveto{\pgfqpoint{1.094053in}{2.061000in}}{\pgfqpoint{1.086153in}{2.057727in}}{\pgfqpoint{1.080329in}{2.051904in}}%
\pgfpathcurveto{\pgfqpoint{1.074505in}{2.046080in}}{\pgfqpoint{1.071233in}{2.038180in}}{\pgfqpoint{1.071233in}{2.029943in}}%
\pgfpathcurveto{\pgfqpoint{1.071233in}{2.021707in}}{\pgfqpoint{1.074505in}{2.013807in}}{\pgfqpoint{1.080329in}{2.007983in}}%
\pgfpathcurveto{\pgfqpoint{1.086153in}{2.002159in}}{\pgfqpoint{1.094053in}{1.998887in}}{\pgfqpoint{1.102289in}{1.998887in}}%
\pgfpathclose%
\pgfusepath{stroke,fill}%
\end{pgfscope}%
\begin{pgfscope}%
\pgfpathrectangle{\pgfqpoint{0.100000in}{0.220728in}}{\pgfqpoint{3.696000in}{3.696000in}}%
\pgfusepath{clip}%
\pgfsetbuttcap%
\pgfsetroundjoin%
\definecolor{currentfill}{rgb}{0.121569,0.466667,0.705882}%
\pgfsetfillcolor{currentfill}%
\pgfsetfillopacity{0.475659}%
\pgfsetlinewidth{1.003750pt}%
\definecolor{currentstroke}{rgb}{0.121569,0.466667,0.705882}%
\pgfsetstrokecolor{currentstroke}%
\pgfsetstrokeopacity{0.475659}%
\pgfsetdash{}{0pt}%
\pgfpathmoveto{\pgfqpoint{2.441575in}{3.020488in}}%
\pgfpathcurveto{\pgfqpoint{2.449811in}{3.020488in}}{\pgfqpoint{2.457711in}{3.023760in}}{\pgfqpoint{2.463535in}{3.029584in}}%
\pgfpathcurveto{\pgfqpoint{2.469359in}{3.035408in}}{\pgfqpoint{2.472631in}{3.043308in}}{\pgfqpoint{2.472631in}{3.051544in}}%
\pgfpathcurveto{\pgfqpoint{2.472631in}{3.059780in}}{\pgfqpoint{2.469359in}{3.067680in}}{\pgfqpoint{2.463535in}{3.073504in}}%
\pgfpathcurveto{\pgfqpoint{2.457711in}{3.079328in}}{\pgfqpoint{2.449811in}{3.082601in}}{\pgfqpoint{2.441575in}{3.082601in}}%
\pgfpathcurveto{\pgfqpoint{2.433338in}{3.082601in}}{\pgfqpoint{2.425438in}{3.079328in}}{\pgfqpoint{2.419615in}{3.073504in}}%
\pgfpathcurveto{\pgfqpoint{2.413791in}{3.067680in}}{\pgfqpoint{2.410518in}{3.059780in}}{\pgfqpoint{2.410518in}{3.051544in}}%
\pgfpathcurveto{\pgfqpoint{2.410518in}{3.043308in}}{\pgfqpoint{2.413791in}{3.035408in}}{\pgfqpoint{2.419615in}{3.029584in}}%
\pgfpathcurveto{\pgfqpoint{2.425438in}{3.023760in}}{\pgfqpoint{2.433338in}{3.020488in}}{\pgfqpoint{2.441575in}{3.020488in}}%
\pgfpathclose%
\pgfusepath{stroke,fill}%
\end{pgfscope}%
\begin{pgfscope}%
\pgfpathrectangle{\pgfqpoint{0.100000in}{0.220728in}}{\pgfqpoint{3.696000in}{3.696000in}}%
\pgfusepath{clip}%
\pgfsetbuttcap%
\pgfsetroundjoin%
\definecolor{currentfill}{rgb}{0.121569,0.466667,0.705882}%
\pgfsetfillcolor{currentfill}%
\pgfsetfillopacity{0.478453}%
\pgfsetlinewidth{1.003750pt}%
\definecolor{currentstroke}{rgb}{0.121569,0.466667,0.705882}%
\pgfsetstrokecolor{currentstroke}%
\pgfsetstrokeopacity{0.478453}%
\pgfsetdash{}{0pt}%
\pgfpathmoveto{\pgfqpoint{1.095742in}{1.982057in}}%
\pgfpathcurveto{\pgfqpoint{1.103978in}{1.982057in}}{\pgfqpoint{1.111878in}{1.985329in}}{\pgfqpoint{1.117702in}{1.991153in}}%
\pgfpathcurveto{\pgfqpoint{1.123526in}{1.996977in}}{\pgfqpoint{1.126798in}{2.004877in}}{\pgfqpoint{1.126798in}{2.013114in}}%
\pgfpathcurveto{\pgfqpoint{1.126798in}{2.021350in}}{\pgfqpoint{1.123526in}{2.029250in}}{\pgfqpoint{1.117702in}{2.035074in}}%
\pgfpathcurveto{\pgfqpoint{1.111878in}{2.040898in}}{\pgfqpoint{1.103978in}{2.044170in}}{\pgfqpoint{1.095742in}{2.044170in}}%
\pgfpathcurveto{\pgfqpoint{1.087505in}{2.044170in}}{\pgfqpoint{1.079605in}{2.040898in}}{\pgfqpoint{1.073782in}{2.035074in}}%
\pgfpathcurveto{\pgfqpoint{1.067958in}{2.029250in}}{\pgfqpoint{1.064685in}{2.021350in}}{\pgfqpoint{1.064685in}{2.013114in}}%
\pgfpathcurveto{\pgfqpoint{1.064685in}{2.004877in}}{\pgfqpoint{1.067958in}{1.996977in}}{\pgfqpoint{1.073782in}{1.991153in}}%
\pgfpathcurveto{\pgfqpoint{1.079605in}{1.985329in}}{\pgfqpoint{1.087505in}{1.982057in}}{\pgfqpoint{1.095742in}{1.982057in}}%
\pgfpathclose%
\pgfusepath{stroke,fill}%
\end{pgfscope}%
\begin{pgfscope}%
\pgfpathrectangle{\pgfqpoint{0.100000in}{0.220728in}}{\pgfqpoint{3.696000in}{3.696000in}}%
\pgfusepath{clip}%
\pgfsetbuttcap%
\pgfsetroundjoin%
\definecolor{currentfill}{rgb}{0.121569,0.466667,0.705882}%
\pgfsetfillcolor{currentfill}%
\pgfsetfillopacity{0.479584}%
\pgfsetlinewidth{1.003750pt}%
\definecolor{currentstroke}{rgb}{0.121569,0.466667,0.705882}%
\pgfsetstrokecolor{currentstroke}%
\pgfsetstrokeopacity{0.479584}%
\pgfsetdash{}{0pt}%
\pgfpathmoveto{\pgfqpoint{1.091490in}{1.974121in}}%
\pgfpathcurveto{\pgfqpoint{1.099726in}{1.974121in}}{\pgfqpoint{1.107627in}{1.977394in}}{\pgfqpoint{1.113450in}{1.983217in}}%
\pgfpathcurveto{\pgfqpoint{1.119274in}{1.989041in}}{\pgfqpoint{1.122547in}{1.996941in}}{\pgfqpoint{1.122547in}{2.005178in}}%
\pgfpathcurveto{\pgfqpoint{1.122547in}{2.013414in}}{\pgfqpoint{1.119274in}{2.021314in}}{\pgfqpoint{1.113450in}{2.027138in}}%
\pgfpathcurveto{\pgfqpoint{1.107627in}{2.032962in}}{\pgfqpoint{1.099726in}{2.036234in}}{\pgfqpoint{1.091490in}{2.036234in}}%
\pgfpathcurveto{\pgfqpoint{1.083254in}{2.036234in}}{\pgfqpoint{1.075354in}{2.032962in}}{\pgfqpoint{1.069530in}{2.027138in}}%
\pgfpathcurveto{\pgfqpoint{1.063706in}{2.021314in}}{\pgfqpoint{1.060434in}{2.013414in}}{\pgfqpoint{1.060434in}{2.005178in}}%
\pgfpathcurveto{\pgfqpoint{1.060434in}{1.996941in}}{\pgfqpoint{1.063706in}{1.989041in}}{\pgfqpoint{1.069530in}{1.983217in}}%
\pgfpathcurveto{\pgfqpoint{1.075354in}{1.977394in}}{\pgfqpoint{1.083254in}{1.974121in}}{\pgfqpoint{1.091490in}{1.974121in}}%
\pgfpathclose%
\pgfusepath{stroke,fill}%
\end{pgfscope}%
\begin{pgfscope}%
\pgfpathrectangle{\pgfqpoint{0.100000in}{0.220728in}}{\pgfqpoint{3.696000in}{3.696000in}}%
\pgfusepath{clip}%
\pgfsetbuttcap%
\pgfsetroundjoin%
\definecolor{currentfill}{rgb}{0.121569,0.466667,0.705882}%
\pgfsetfillcolor{currentfill}%
\pgfsetfillopacity{0.480076}%
\pgfsetlinewidth{1.003750pt}%
\definecolor{currentstroke}{rgb}{0.121569,0.466667,0.705882}%
\pgfsetstrokecolor{currentstroke}%
\pgfsetstrokeopacity{0.480076}%
\pgfsetdash{}{0pt}%
\pgfpathmoveto{\pgfqpoint{1.089970in}{1.970954in}}%
\pgfpathcurveto{\pgfqpoint{1.098206in}{1.970954in}}{\pgfqpoint{1.106106in}{1.974226in}}{\pgfqpoint{1.111930in}{1.980050in}}%
\pgfpathcurveto{\pgfqpoint{1.117754in}{1.985874in}}{\pgfqpoint{1.121027in}{1.993774in}}{\pgfqpoint{1.121027in}{2.002011in}}%
\pgfpathcurveto{\pgfqpoint{1.121027in}{2.010247in}}{\pgfqpoint{1.117754in}{2.018147in}}{\pgfqpoint{1.111930in}{2.023971in}}%
\pgfpathcurveto{\pgfqpoint{1.106106in}{2.029795in}}{\pgfqpoint{1.098206in}{2.033067in}}{\pgfqpoint{1.089970in}{2.033067in}}%
\pgfpathcurveto{\pgfqpoint{1.081734in}{2.033067in}}{\pgfqpoint{1.073834in}{2.029795in}}{\pgfqpoint{1.068010in}{2.023971in}}%
\pgfpathcurveto{\pgfqpoint{1.062186in}{2.018147in}}{\pgfqpoint{1.058914in}{2.010247in}}{\pgfqpoint{1.058914in}{2.002011in}}%
\pgfpathcurveto{\pgfqpoint{1.058914in}{1.993774in}}{\pgfqpoint{1.062186in}{1.985874in}}{\pgfqpoint{1.068010in}{1.980050in}}%
\pgfpathcurveto{\pgfqpoint{1.073834in}{1.974226in}}{\pgfqpoint{1.081734in}{1.970954in}}{\pgfqpoint{1.089970in}{1.970954in}}%
\pgfpathclose%
\pgfusepath{stroke,fill}%
\end{pgfscope}%
\begin{pgfscope}%
\pgfpathrectangle{\pgfqpoint{0.100000in}{0.220728in}}{\pgfqpoint{3.696000in}{3.696000in}}%
\pgfusepath{clip}%
\pgfsetbuttcap%
\pgfsetroundjoin%
\definecolor{currentfill}{rgb}{0.121569,0.466667,0.705882}%
\pgfsetfillcolor{currentfill}%
\pgfsetfillopacity{0.481014}%
\pgfsetlinewidth{1.003750pt}%
\definecolor{currentstroke}{rgb}{0.121569,0.466667,0.705882}%
\pgfsetstrokecolor{currentstroke}%
\pgfsetstrokeopacity{0.481014}%
\pgfsetdash{}{0pt}%
\pgfpathmoveto{\pgfqpoint{1.087367in}{1.965213in}}%
\pgfpathcurveto{\pgfqpoint{1.095603in}{1.965213in}}{\pgfqpoint{1.103503in}{1.968486in}}{\pgfqpoint{1.109327in}{1.974310in}}%
\pgfpathcurveto{\pgfqpoint{1.115151in}{1.980133in}}{\pgfqpoint{1.118423in}{1.988034in}}{\pgfqpoint{1.118423in}{1.996270in}}%
\pgfpathcurveto{\pgfqpoint{1.118423in}{2.004506in}}{\pgfqpoint{1.115151in}{2.012406in}}{\pgfqpoint{1.109327in}{2.018230in}}%
\pgfpathcurveto{\pgfqpoint{1.103503in}{2.024054in}}{\pgfqpoint{1.095603in}{2.027326in}}{\pgfqpoint{1.087367in}{2.027326in}}%
\pgfpathcurveto{\pgfqpoint{1.079131in}{2.027326in}}{\pgfqpoint{1.071231in}{2.024054in}}{\pgfqpoint{1.065407in}{2.018230in}}%
\pgfpathcurveto{\pgfqpoint{1.059583in}{2.012406in}}{\pgfqpoint{1.056310in}{2.004506in}}{\pgfqpoint{1.056310in}{1.996270in}}%
\pgfpathcurveto{\pgfqpoint{1.056310in}{1.988034in}}{\pgfqpoint{1.059583in}{1.980133in}}{\pgfqpoint{1.065407in}{1.974310in}}%
\pgfpathcurveto{\pgfqpoint{1.071231in}{1.968486in}}{\pgfqpoint{1.079131in}{1.965213in}}{\pgfqpoint{1.087367in}{1.965213in}}%
\pgfpathclose%
\pgfusepath{stroke,fill}%
\end{pgfscope}%
\begin{pgfscope}%
\pgfpathrectangle{\pgfqpoint{0.100000in}{0.220728in}}{\pgfqpoint{3.696000in}{3.696000in}}%
\pgfusepath{clip}%
\pgfsetbuttcap%
\pgfsetroundjoin%
\definecolor{currentfill}{rgb}{0.121569,0.466667,0.705882}%
\pgfsetfillcolor{currentfill}%
\pgfsetfillopacity{0.482064}%
\pgfsetlinewidth{1.003750pt}%
\definecolor{currentstroke}{rgb}{0.121569,0.466667,0.705882}%
\pgfsetstrokecolor{currentstroke}%
\pgfsetstrokeopacity{0.482064}%
\pgfsetdash{}{0pt}%
\pgfpathmoveto{\pgfqpoint{2.463963in}{3.014583in}}%
\pgfpathcurveto{\pgfqpoint{2.472199in}{3.014583in}}{\pgfqpoint{2.480099in}{3.017856in}}{\pgfqpoint{2.485923in}{3.023679in}}%
\pgfpathcurveto{\pgfqpoint{2.491747in}{3.029503in}}{\pgfqpoint{2.495020in}{3.037403in}}{\pgfqpoint{2.495020in}{3.045640in}}%
\pgfpathcurveto{\pgfqpoint{2.495020in}{3.053876in}}{\pgfqpoint{2.491747in}{3.061776in}}{\pgfqpoint{2.485923in}{3.067600in}}%
\pgfpathcurveto{\pgfqpoint{2.480099in}{3.073424in}}{\pgfqpoint{2.472199in}{3.076696in}}{\pgfqpoint{2.463963in}{3.076696in}}%
\pgfpathcurveto{\pgfqpoint{2.455727in}{3.076696in}}{\pgfqpoint{2.447827in}{3.073424in}}{\pgfqpoint{2.442003in}{3.067600in}}%
\pgfpathcurveto{\pgfqpoint{2.436179in}{3.061776in}}{\pgfqpoint{2.432907in}{3.053876in}}{\pgfqpoint{2.432907in}{3.045640in}}%
\pgfpathcurveto{\pgfqpoint{2.432907in}{3.037403in}}{\pgfqpoint{2.436179in}{3.029503in}}{\pgfqpoint{2.442003in}{3.023679in}}%
\pgfpathcurveto{\pgfqpoint{2.447827in}{3.017856in}}{\pgfqpoint{2.455727in}{3.014583in}}{\pgfqpoint{2.463963in}{3.014583in}}%
\pgfpathclose%
\pgfusepath{stroke,fill}%
\end{pgfscope}%
\begin{pgfscope}%
\pgfpathrectangle{\pgfqpoint{0.100000in}{0.220728in}}{\pgfqpoint{3.696000in}{3.696000in}}%
\pgfusepath{clip}%
\pgfsetbuttcap%
\pgfsetroundjoin%
\definecolor{currentfill}{rgb}{0.121569,0.466667,0.705882}%
\pgfsetfillcolor{currentfill}%
\pgfsetfillopacity{0.482638}%
\pgfsetlinewidth{1.003750pt}%
\definecolor{currentstroke}{rgb}{0.121569,0.466667,0.705882}%
\pgfsetstrokecolor{currentstroke}%
\pgfsetstrokeopacity{0.482638}%
\pgfsetdash{}{0pt}%
\pgfpathmoveto{\pgfqpoint{1.082495in}{1.954528in}}%
\pgfpathcurveto{\pgfqpoint{1.090731in}{1.954528in}}{\pgfqpoint{1.098631in}{1.957800in}}{\pgfqpoint{1.104455in}{1.963624in}}%
\pgfpathcurveto{\pgfqpoint{1.110279in}{1.969448in}}{\pgfqpoint{1.113551in}{1.977348in}}{\pgfqpoint{1.113551in}{1.985584in}}%
\pgfpathcurveto{\pgfqpoint{1.113551in}{1.993820in}}{\pgfqpoint{1.110279in}{2.001720in}}{\pgfqpoint{1.104455in}{2.007544in}}%
\pgfpathcurveto{\pgfqpoint{1.098631in}{2.013368in}}{\pgfqpoint{1.090731in}{2.016641in}}{\pgfqpoint{1.082495in}{2.016641in}}%
\pgfpathcurveto{\pgfqpoint{1.074259in}{2.016641in}}{\pgfqpoint{1.066359in}{2.013368in}}{\pgfqpoint{1.060535in}{2.007544in}}%
\pgfpathcurveto{\pgfqpoint{1.054711in}{2.001720in}}{\pgfqpoint{1.051438in}{1.993820in}}{\pgfqpoint{1.051438in}{1.985584in}}%
\pgfpathcurveto{\pgfqpoint{1.051438in}{1.977348in}}{\pgfqpoint{1.054711in}{1.969448in}}{\pgfqpoint{1.060535in}{1.963624in}}%
\pgfpathcurveto{\pgfqpoint{1.066359in}{1.957800in}}{\pgfqpoint{1.074259in}{1.954528in}}{\pgfqpoint{1.082495in}{1.954528in}}%
\pgfpathclose%
\pgfusepath{stroke,fill}%
\end{pgfscope}%
\begin{pgfscope}%
\pgfpathrectangle{\pgfqpoint{0.100000in}{0.220728in}}{\pgfqpoint{3.696000in}{3.696000in}}%
\pgfusepath{clip}%
\pgfsetbuttcap%
\pgfsetroundjoin%
\definecolor{currentfill}{rgb}{0.121569,0.466667,0.705882}%
\pgfsetfillcolor{currentfill}%
\pgfsetfillopacity{0.485469}%
\pgfsetlinewidth{1.003750pt}%
\definecolor{currentstroke}{rgb}{0.121569,0.466667,0.705882}%
\pgfsetstrokecolor{currentstroke}%
\pgfsetstrokeopacity{0.485469}%
\pgfsetdash{}{0pt}%
\pgfpathmoveto{\pgfqpoint{2.476841in}{3.012081in}}%
\pgfpathcurveto{\pgfqpoint{2.485077in}{3.012081in}}{\pgfqpoint{2.492977in}{3.015353in}}{\pgfqpoint{2.498801in}{3.021177in}}%
\pgfpathcurveto{\pgfqpoint{2.504625in}{3.027001in}}{\pgfqpoint{2.507897in}{3.034901in}}{\pgfqpoint{2.507897in}{3.043137in}}%
\pgfpathcurveto{\pgfqpoint{2.507897in}{3.051373in}}{\pgfqpoint{2.504625in}{3.059273in}}{\pgfqpoint{2.498801in}{3.065097in}}%
\pgfpathcurveto{\pgfqpoint{2.492977in}{3.070921in}}{\pgfqpoint{2.485077in}{3.074194in}}{\pgfqpoint{2.476841in}{3.074194in}}%
\pgfpathcurveto{\pgfqpoint{2.468604in}{3.074194in}}{\pgfqpoint{2.460704in}{3.070921in}}{\pgfqpoint{2.454880in}{3.065097in}}%
\pgfpathcurveto{\pgfqpoint{2.449056in}{3.059273in}}{\pgfqpoint{2.445784in}{3.051373in}}{\pgfqpoint{2.445784in}{3.043137in}}%
\pgfpathcurveto{\pgfqpoint{2.445784in}{3.034901in}}{\pgfqpoint{2.449056in}{3.027001in}}{\pgfqpoint{2.454880in}{3.021177in}}%
\pgfpathcurveto{\pgfqpoint{2.460704in}{3.015353in}}{\pgfqpoint{2.468604in}{3.012081in}}{\pgfqpoint{2.476841in}{3.012081in}}%
\pgfpathclose%
\pgfusepath{stroke,fill}%
\end{pgfscope}%
\begin{pgfscope}%
\pgfpathrectangle{\pgfqpoint{0.100000in}{0.220728in}}{\pgfqpoint{3.696000in}{3.696000in}}%
\pgfusepath{clip}%
\pgfsetbuttcap%
\pgfsetroundjoin%
\definecolor{currentfill}{rgb}{0.121569,0.466667,0.705882}%
\pgfsetfillcolor{currentfill}%
\pgfsetfillopacity{0.485800}%
\pgfsetlinewidth{1.003750pt}%
\definecolor{currentstroke}{rgb}{0.121569,0.466667,0.705882}%
\pgfsetstrokecolor{currentstroke}%
\pgfsetstrokeopacity{0.485800}%
\pgfsetdash{}{0pt}%
\pgfpathmoveto{\pgfqpoint{1.073756in}{1.935989in}}%
\pgfpathcurveto{\pgfqpoint{1.081992in}{1.935989in}}{\pgfqpoint{1.089892in}{1.939261in}}{\pgfqpoint{1.095716in}{1.945085in}}%
\pgfpathcurveto{\pgfqpoint{1.101540in}{1.950909in}}{\pgfqpoint{1.104812in}{1.958809in}}{\pgfqpoint{1.104812in}{1.967045in}}%
\pgfpathcurveto{\pgfqpoint{1.104812in}{1.975282in}}{\pgfqpoint{1.101540in}{1.983182in}}{\pgfqpoint{1.095716in}{1.989006in}}%
\pgfpathcurveto{\pgfqpoint{1.089892in}{1.994830in}}{\pgfqpoint{1.081992in}{1.998102in}}{\pgfqpoint{1.073756in}{1.998102in}}%
\pgfpathcurveto{\pgfqpoint{1.065520in}{1.998102in}}{\pgfqpoint{1.057620in}{1.994830in}}{\pgfqpoint{1.051796in}{1.989006in}}%
\pgfpathcurveto{\pgfqpoint{1.045972in}{1.983182in}}{\pgfqpoint{1.042699in}{1.975282in}}{\pgfqpoint{1.042699in}{1.967045in}}%
\pgfpathcurveto{\pgfqpoint{1.042699in}{1.958809in}}{\pgfqpoint{1.045972in}{1.950909in}}{\pgfqpoint{1.051796in}{1.945085in}}%
\pgfpathcurveto{\pgfqpoint{1.057620in}{1.939261in}}{\pgfqpoint{1.065520in}{1.935989in}}{\pgfqpoint{1.073756in}{1.935989in}}%
\pgfpathclose%
\pgfusepath{stroke,fill}%
\end{pgfscope}%
\begin{pgfscope}%
\pgfpathrectangle{\pgfqpoint{0.100000in}{0.220728in}}{\pgfqpoint{3.696000in}{3.696000in}}%
\pgfusepath{clip}%
\pgfsetbuttcap%
\pgfsetroundjoin%
\definecolor{currentfill}{rgb}{0.121569,0.466667,0.705882}%
\pgfsetfillcolor{currentfill}%
\pgfsetfillopacity{0.489576}%
\pgfsetlinewidth{1.003750pt}%
\definecolor{currentstroke}{rgb}{0.121569,0.466667,0.705882}%
\pgfsetstrokecolor{currentstroke}%
\pgfsetstrokeopacity{0.489576}%
\pgfsetdash{}{0pt}%
\pgfpathmoveto{\pgfqpoint{2.494091in}{3.008970in}}%
\pgfpathcurveto{\pgfqpoint{2.502327in}{3.008970in}}{\pgfqpoint{2.510227in}{3.012243in}}{\pgfqpoint{2.516051in}{3.018067in}}%
\pgfpathcurveto{\pgfqpoint{2.521875in}{3.023890in}}{\pgfqpoint{2.525147in}{3.031791in}}{\pgfqpoint{2.525147in}{3.040027in}}%
\pgfpathcurveto{\pgfqpoint{2.525147in}{3.048263in}}{\pgfqpoint{2.521875in}{3.056163in}}{\pgfqpoint{2.516051in}{3.061987in}}%
\pgfpathcurveto{\pgfqpoint{2.510227in}{3.067811in}}{\pgfqpoint{2.502327in}{3.071083in}}{\pgfqpoint{2.494091in}{3.071083in}}%
\pgfpathcurveto{\pgfqpoint{2.485854in}{3.071083in}}{\pgfqpoint{2.477954in}{3.067811in}}{\pgfqpoint{2.472130in}{3.061987in}}%
\pgfpathcurveto{\pgfqpoint{2.466306in}{3.056163in}}{\pgfqpoint{2.463034in}{3.048263in}}{\pgfqpoint{2.463034in}{3.040027in}}%
\pgfpathcurveto{\pgfqpoint{2.463034in}{3.031791in}}{\pgfqpoint{2.466306in}{3.023890in}}{\pgfqpoint{2.472130in}{3.018067in}}%
\pgfpathcurveto{\pgfqpoint{2.477954in}{3.012243in}}{\pgfqpoint{2.485854in}{3.008970in}}{\pgfqpoint{2.494091in}{3.008970in}}%
\pgfpathclose%
\pgfusepath{stroke,fill}%
\end{pgfscope}%
\begin{pgfscope}%
\pgfpathrectangle{\pgfqpoint{0.100000in}{0.220728in}}{\pgfqpoint{3.696000in}{3.696000in}}%
\pgfusepath{clip}%
\pgfsetbuttcap%
\pgfsetroundjoin%
\definecolor{currentfill}{rgb}{0.121569,0.466667,0.705882}%
\pgfsetfillcolor{currentfill}%
\pgfsetfillopacity{0.490777}%
\pgfsetlinewidth{1.003750pt}%
\definecolor{currentstroke}{rgb}{0.121569,0.466667,0.705882}%
\pgfsetstrokecolor{currentstroke}%
\pgfsetstrokeopacity{0.490777}%
\pgfsetdash{}{0pt}%
\pgfpathmoveto{\pgfqpoint{1.055254in}{1.901284in}}%
\pgfpathcurveto{\pgfqpoint{1.063490in}{1.901284in}}{\pgfqpoint{1.071390in}{1.904557in}}{\pgfqpoint{1.077214in}{1.910380in}}%
\pgfpathcurveto{\pgfqpoint{1.083038in}{1.916204in}}{\pgfqpoint{1.086310in}{1.924104in}}{\pgfqpoint{1.086310in}{1.932341in}}%
\pgfpathcurveto{\pgfqpoint{1.086310in}{1.940577in}}{\pgfqpoint{1.083038in}{1.948477in}}{\pgfqpoint{1.077214in}{1.954301in}}%
\pgfpathcurveto{\pgfqpoint{1.071390in}{1.960125in}}{\pgfqpoint{1.063490in}{1.963397in}}{\pgfqpoint{1.055254in}{1.963397in}}%
\pgfpathcurveto{\pgfqpoint{1.047017in}{1.963397in}}{\pgfqpoint{1.039117in}{1.960125in}}{\pgfqpoint{1.033293in}{1.954301in}}%
\pgfpathcurveto{\pgfqpoint{1.027469in}{1.948477in}}{\pgfqpoint{1.024197in}{1.940577in}}{\pgfqpoint{1.024197in}{1.932341in}}%
\pgfpathcurveto{\pgfqpoint{1.024197in}{1.924104in}}{\pgfqpoint{1.027469in}{1.916204in}}{\pgfqpoint{1.033293in}{1.910380in}}%
\pgfpathcurveto{\pgfqpoint{1.039117in}{1.904557in}}{\pgfqpoint{1.047017in}{1.901284in}}{\pgfqpoint{1.055254in}{1.901284in}}%
\pgfpathclose%
\pgfusepath{stroke,fill}%
\end{pgfscope}%
\begin{pgfscope}%
\pgfpathrectangle{\pgfqpoint{0.100000in}{0.220728in}}{\pgfqpoint{3.696000in}{3.696000in}}%
\pgfusepath{clip}%
\pgfsetbuttcap%
\pgfsetroundjoin%
\definecolor{currentfill}{rgb}{0.121569,0.466667,0.705882}%
\pgfsetfillcolor{currentfill}%
\pgfsetfillopacity{0.494844}%
\pgfsetlinewidth{1.003750pt}%
\definecolor{currentstroke}{rgb}{0.121569,0.466667,0.705882}%
\pgfsetstrokecolor{currentstroke}%
\pgfsetstrokeopacity{0.494844}%
\pgfsetdash{}{0pt}%
\pgfpathmoveto{\pgfqpoint{2.512681in}{3.006225in}}%
\pgfpathcurveto{\pgfqpoint{2.520917in}{3.006225in}}{\pgfqpoint{2.528817in}{3.009497in}}{\pgfqpoint{2.534641in}{3.015321in}}%
\pgfpathcurveto{\pgfqpoint{2.540465in}{3.021145in}}{\pgfqpoint{2.543738in}{3.029045in}}{\pgfqpoint{2.543738in}{3.037281in}}%
\pgfpathcurveto{\pgfqpoint{2.543738in}{3.045518in}}{\pgfqpoint{2.540465in}{3.053418in}}{\pgfqpoint{2.534641in}{3.059242in}}%
\pgfpathcurveto{\pgfqpoint{2.528817in}{3.065066in}}{\pgfqpoint{2.520917in}{3.068338in}}{\pgfqpoint{2.512681in}{3.068338in}}%
\pgfpathcurveto{\pgfqpoint{2.504445in}{3.068338in}}{\pgfqpoint{2.496545in}{3.065066in}}{\pgfqpoint{2.490721in}{3.059242in}}%
\pgfpathcurveto{\pgfqpoint{2.484897in}{3.053418in}}{\pgfqpoint{2.481625in}{3.045518in}}{\pgfqpoint{2.481625in}{3.037281in}}%
\pgfpathcurveto{\pgfqpoint{2.481625in}{3.029045in}}{\pgfqpoint{2.484897in}{3.021145in}}{\pgfqpoint{2.490721in}{3.015321in}}%
\pgfpathcurveto{\pgfqpoint{2.496545in}{3.009497in}}{\pgfqpoint{2.504445in}{3.006225in}}{\pgfqpoint{2.512681in}{3.006225in}}%
\pgfpathclose%
\pgfusepath{stroke,fill}%
\end{pgfscope}%
\begin{pgfscope}%
\pgfpathrectangle{\pgfqpoint{0.100000in}{0.220728in}}{\pgfqpoint{3.696000in}{3.696000in}}%
\pgfusepath{clip}%
\pgfsetbuttcap%
\pgfsetroundjoin%
\definecolor{currentfill}{rgb}{0.121569,0.466667,0.705882}%
\pgfsetfillcolor{currentfill}%
\pgfsetfillopacity{0.500489}%
\pgfsetlinewidth{1.003750pt}%
\definecolor{currentstroke}{rgb}{0.121569,0.466667,0.705882}%
\pgfsetstrokecolor{currentstroke}%
\pgfsetstrokeopacity{0.500489}%
\pgfsetdash{}{0pt}%
\pgfpathmoveto{\pgfqpoint{1.030248in}{1.832981in}}%
\pgfpathcurveto{\pgfqpoint{1.038484in}{1.832981in}}{\pgfqpoint{1.046385in}{1.836253in}}{\pgfqpoint{1.052208in}{1.842077in}}%
\pgfpathcurveto{\pgfqpoint{1.058032in}{1.847901in}}{\pgfqpoint{1.061305in}{1.855801in}}{\pgfqpoint{1.061305in}{1.864037in}}%
\pgfpathcurveto{\pgfqpoint{1.061305in}{1.872273in}}{\pgfqpoint{1.058032in}{1.880173in}}{\pgfqpoint{1.052208in}{1.885997in}}%
\pgfpathcurveto{\pgfqpoint{1.046385in}{1.891821in}}{\pgfqpoint{1.038484in}{1.895094in}}{\pgfqpoint{1.030248in}{1.895094in}}%
\pgfpathcurveto{\pgfqpoint{1.022012in}{1.895094in}}{\pgfqpoint{1.014112in}{1.891821in}}{\pgfqpoint{1.008288in}{1.885997in}}%
\pgfpathcurveto{\pgfqpoint{1.002464in}{1.880173in}}{\pgfqpoint{0.999192in}{1.872273in}}{\pgfqpoint{0.999192in}{1.864037in}}%
\pgfpathcurveto{\pgfqpoint{0.999192in}{1.855801in}}{\pgfqpoint{1.002464in}{1.847901in}}{\pgfqpoint{1.008288in}{1.842077in}}%
\pgfpathcurveto{\pgfqpoint{1.014112in}{1.836253in}}{\pgfqpoint{1.022012in}{1.832981in}}{\pgfqpoint{1.030248in}{1.832981in}}%
\pgfpathclose%
\pgfusepath{stroke,fill}%
\end{pgfscope}%
\begin{pgfscope}%
\pgfpathrectangle{\pgfqpoint{0.100000in}{0.220728in}}{\pgfqpoint{3.696000in}{3.696000in}}%
\pgfusepath{clip}%
\pgfsetbuttcap%
\pgfsetroundjoin%
\definecolor{currentfill}{rgb}{0.121569,0.466667,0.705882}%
\pgfsetfillcolor{currentfill}%
\pgfsetfillopacity{0.500829}%
\pgfsetlinewidth{1.003750pt}%
\definecolor{currentstroke}{rgb}{0.121569,0.466667,0.705882}%
\pgfsetstrokecolor{currentstroke}%
\pgfsetstrokeopacity{0.500829}%
\pgfsetdash{}{0pt}%
\pgfpathmoveto{\pgfqpoint{2.533461in}{2.998439in}}%
\pgfpathcurveto{\pgfqpoint{2.541697in}{2.998439in}}{\pgfqpoint{2.549597in}{3.001711in}}{\pgfqpoint{2.555421in}{3.007535in}}%
\pgfpathcurveto{\pgfqpoint{2.561245in}{3.013359in}}{\pgfqpoint{2.564518in}{3.021259in}}{\pgfqpoint{2.564518in}{3.029496in}}%
\pgfpathcurveto{\pgfqpoint{2.564518in}{3.037732in}}{\pgfqpoint{2.561245in}{3.045632in}}{\pgfqpoint{2.555421in}{3.051456in}}%
\pgfpathcurveto{\pgfqpoint{2.549597in}{3.057280in}}{\pgfqpoint{2.541697in}{3.060552in}}{\pgfqpoint{2.533461in}{3.060552in}}%
\pgfpathcurveto{\pgfqpoint{2.525225in}{3.060552in}}{\pgfqpoint{2.517325in}{3.057280in}}{\pgfqpoint{2.511501in}{3.051456in}}%
\pgfpathcurveto{\pgfqpoint{2.505677in}{3.045632in}}{\pgfqpoint{2.502405in}{3.037732in}}{\pgfqpoint{2.502405in}{3.029496in}}%
\pgfpathcurveto{\pgfqpoint{2.502405in}{3.021259in}}{\pgfqpoint{2.505677in}{3.013359in}}{\pgfqpoint{2.511501in}{3.007535in}}%
\pgfpathcurveto{\pgfqpoint{2.517325in}{3.001711in}}{\pgfqpoint{2.525225in}{2.998439in}}{\pgfqpoint{2.533461in}{2.998439in}}%
\pgfpathclose%
\pgfusepath{stroke,fill}%
\end{pgfscope}%
\begin{pgfscope}%
\pgfpathrectangle{\pgfqpoint{0.100000in}{0.220728in}}{\pgfqpoint{3.696000in}{3.696000in}}%
\pgfusepath{clip}%
\pgfsetbuttcap%
\pgfsetroundjoin%
\definecolor{currentfill}{rgb}{0.121569,0.466667,0.705882}%
\pgfsetfillcolor{currentfill}%
\pgfsetfillopacity{0.507660}%
\pgfsetlinewidth{1.003750pt}%
\definecolor{currentstroke}{rgb}{0.121569,0.466667,0.705882}%
\pgfsetstrokecolor{currentstroke}%
\pgfsetstrokeopacity{0.507660}%
\pgfsetdash{}{0pt}%
\pgfpathmoveto{\pgfqpoint{2.556606in}{2.992003in}}%
\pgfpathcurveto{\pgfqpoint{2.564843in}{2.992003in}}{\pgfqpoint{2.572743in}{2.995275in}}{\pgfqpoint{2.578567in}{3.001099in}}%
\pgfpathcurveto{\pgfqpoint{2.584391in}{3.006923in}}{\pgfqpoint{2.587663in}{3.014823in}}{\pgfqpoint{2.587663in}{3.023059in}}%
\pgfpathcurveto{\pgfqpoint{2.587663in}{3.031296in}}{\pgfqpoint{2.584391in}{3.039196in}}{\pgfqpoint{2.578567in}{3.045020in}}%
\pgfpathcurveto{\pgfqpoint{2.572743in}{3.050843in}}{\pgfqpoint{2.564843in}{3.054116in}}{\pgfqpoint{2.556606in}{3.054116in}}%
\pgfpathcurveto{\pgfqpoint{2.548370in}{3.054116in}}{\pgfqpoint{2.540470in}{3.050843in}}{\pgfqpoint{2.534646in}{3.045020in}}%
\pgfpathcurveto{\pgfqpoint{2.528822in}{3.039196in}}{\pgfqpoint{2.525550in}{3.031296in}}{\pgfqpoint{2.525550in}{3.023059in}}%
\pgfpathcurveto{\pgfqpoint{2.525550in}{3.014823in}}{\pgfqpoint{2.528822in}{3.006923in}}{\pgfqpoint{2.534646in}{3.001099in}}%
\pgfpathcurveto{\pgfqpoint{2.540470in}{2.995275in}}{\pgfqpoint{2.548370in}{2.992003in}}{\pgfqpoint{2.556606in}{2.992003in}}%
\pgfpathclose%
\pgfusepath{stroke,fill}%
\end{pgfscope}%
\begin{pgfscope}%
\pgfpathrectangle{\pgfqpoint{0.100000in}{0.220728in}}{\pgfqpoint{3.696000in}{3.696000in}}%
\pgfusepath{clip}%
\pgfsetbuttcap%
\pgfsetroundjoin%
\definecolor{currentfill}{rgb}{0.121569,0.466667,0.705882}%
\pgfsetfillcolor{currentfill}%
\pgfsetfillopacity{0.507772}%
\pgfsetlinewidth{1.003750pt}%
\definecolor{currentstroke}{rgb}{0.121569,0.466667,0.705882}%
\pgfsetstrokecolor{currentstroke}%
\pgfsetstrokeopacity{0.507772}%
\pgfsetdash{}{0pt}%
\pgfpathmoveto{\pgfqpoint{0.997693in}{1.780132in}}%
\pgfpathcurveto{\pgfqpoint{1.005930in}{1.780132in}}{\pgfqpoint{1.013830in}{1.783404in}}{\pgfqpoint{1.019654in}{1.789228in}}%
\pgfpathcurveto{\pgfqpoint{1.025478in}{1.795052in}}{\pgfqpoint{1.028750in}{1.802952in}}{\pgfqpoint{1.028750in}{1.811188in}}%
\pgfpathcurveto{\pgfqpoint{1.028750in}{1.819425in}}{\pgfqpoint{1.025478in}{1.827325in}}{\pgfqpoint{1.019654in}{1.833149in}}%
\pgfpathcurveto{\pgfqpoint{1.013830in}{1.838972in}}{\pgfqpoint{1.005930in}{1.842245in}}{\pgfqpoint{0.997693in}{1.842245in}}%
\pgfpathcurveto{\pgfqpoint{0.989457in}{1.842245in}}{\pgfqpoint{0.981557in}{1.838972in}}{\pgfqpoint{0.975733in}{1.833149in}}%
\pgfpathcurveto{\pgfqpoint{0.969909in}{1.827325in}}{\pgfqpoint{0.966637in}{1.819425in}}{\pgfqpoint{0.966637in}{1.811188in}}%
\pgfpathcurveto{\pgfqpoint{0.966637in}{1.802952in}}{\pgfqpoint{0.969909in}{1.795052in}}{\pgfqpoint{0.975733in}{1.789228in}}%
\pgfpathcurveto{\pgfqpoint{0.981557in}{1.783404in}}{\pgfqpoint{0.989457in}{1.780132in}}{\pgfqpoint{0.997693in}{1.780132in}}%
\pgfpathclose%
\pgfusepath{stroke,fill}%
\end{pgfscope}%
\begin{pgfscope}%
\pgfpathrectangle{\pgfqpoint{0.100000in}{0.220728in}}{\pgfqpoint{3.696000in}{3.696000in}}%
\pgfusepath{clip}%
\pgfsetbuttcap%
\pgfsetroundjoin%
\definecolor{currentfill}{rgb}{0.121569,0.466667,0.705882}%
\pgfsetfillcolor{currentfill}%
\pgfsetfillopacity{0.511661}%
\pgfsetlinewidth{1.003750pt}%
\definecolor{currentstroke}{rgb}{0.121569,0.466667,0.705882}%
\pgfsetstrokecolor{currentstroke}%
\pgfsetstrokeopacity{0.511661}%
\pgfsetdash{}{0pt}%
\pgfpathmoveto{\pgfqpoint{2.568819in}{2.988459in}}%
\pgfpathcurveto{\pgfqpoint{2.577055in}{2.988459in}}{\pgfqpoint{2.584955in}{2.991732in}}{\pgfqpoint{2.590779in}{2.997556in}}%
\pgfpathcurveto{\pgfqpoint{2.596603in}{3.003380in}}{\pgfqpoint{2.599876in}{3.011280in}}{\pgfqpoint{2.599876in}{3.019516in}}%
\pgfpathcurveto{\pgfqpoint{2.599876in}{3.027752in}}{\pgfqpoint{2.596603in}{3.035652in}}{\pgfqpoint{2.590779in}{3.041476in}}%
\pgfpathcurveto{\pgfqpoint{2.584955in}{3.047300in}}{\pgfqpoint{2.577055in}{3.050572in}}{\pgfqpoint{2.568819in}{3.050572in}}%
\pgfpathcurveto{\pgfqpoint{2.560583in}{3.050572in}}{\pgfqpoint{2.552683in}{3.047300in}}{\pgfqpoint{2.546859in}{3.041476in}}%
\pgfpathcurveto{\pgfqpoint{2.541035in}{3.035652in}}{\pgfqpoint{2.537763in}{3.027752in}}{\pgfqpoint{2.537763in}{3.019516in}}%
\pgfpathcurveto{\pgfqpoint{2.537763in}{3.011280in}}{\pgfqpoint{2.541035in}{3.003380in}}{\pgfqpoint{2.546859in}{2.997556in}}%
\pgfpathcurveto{\pgfqpoint{2.552683in}{2.991732in}}{\pgfqpoint{2.560583in}{2.988459in}}{\pgfqpoint{2.568819in}{2.988459in}}%
\pgfpathclose%
\pgfusepath{stroke,fill}%
\end{pgfscope}%
\begin{pgfscope}%
\pgfpathrectangle{\pgfqpoint{0.100000in}{0.220728in}}{\pgfqpoint{3.696000in}{3.696000in}}%
\pgfusepath{clip}%
\pgfsetbuttcap%
\pgfsetroundjoin%
\definecolor{currentfill}{rgb}{0.121569,0.466667,0.705882}%
\pgfsetfillcolor{currentfill}%
\pgfsetfillopacity{0.516016}%
\pgfsetlinewidth{1.003750pt}%
\definecolor{currentstroke}{rgb}{0.121569,0.466667,0.705882}%
\pgfsetstrokecolor{currentstroke}%
\pgfsetstrokeopacity{0.516016}%
\pgfsetdash{}{0pt}%
\pgfpathmoveto{\pgfqpoint{2.584485in}{2.985246in}}%
\pgfpathcurveto{\pgfqpoint{2.592721in}{2.985246in}}{\pgfqpoint{2.600621in}{2.988518in}}{\pgfqpoint{2.606445in}{2.994342in}}%
\pgfpathcurveto{\pgfqpoint{2.612269in}{3.000166in}}{\pgfqpoint{2.615542in}{3.008066in}}{\pgfqpoint{2.615542in}{3.016302in}}%
\pgfpathcurveto{\pgfqpoint{2.615542in}{3.024538in}}{\pgfqpoint{2.612269in}{3.032438in}}{\pgfqpoint{2.606445in}{3.038262in}}%
\pgfpathcurveto{\pgfqpoint{2.600621in}{3.044086in}}{\pgfqpoint{2.592721in}{3.047359in}}{\pgfqpoint{2.584485in}{3.047359in}}%
\pgfpathcurveto{\pgfqpoint{2.576249in}{3.047359in}}{\pgfqpoint{2.568349in}{3.044086in}}{\pgfqpoint{2.562525in}{3.038262in}}%
\pgfpathcurveto{\pgfqpoint{2.556701in}{3.032438in}}{\pgfqpoint{2.553429in}{3.024538in}}{\pgfqpoint{2.553429in}{3.016302in}}%
\pgfpathcurveto{\pgfqpoint{2.553429in}{3.008066in}}{\pgfqpoint{2.556701in}{3.000166in}}{\pgfqpoint{2.562525in}{2.994342in}}%
\pgfpathcurveto{\pgfqpoint{2.568349in}{2.988518in}}{\pgfqpoint{2.576249in}{2.985246in}}{\pgfqpoint{2.584485in}{2.985246in}}%
\pgfpathclose%
\pgfusepath{stroke,fill}%
\end{pgfscope}%
\begin{pgfscope}%
\pgfpathrectangle{\pgfqpoint{0.100000in}{0.220728in}}{\pgfqpoint{3.696000in}{3.696000in}}%
\pgfusepath{clip}%
\pgfsetbuttcap%
\pgfsetroundjoin%
\definecolor{currentfill}{rgb}{0.121569,0.466667,0.705882}%
\pgfsetfillcolor{currentfill}%
\pgfsetfillopacity{0.516488}%
\pgfsetlinewidth{1.003750pt}%
\definecolor{currentstroke}{rgb}{0.121569,0.466667,0.705882}%
\pgfsetstrokecolor{currentstroke}%
\pgfsetstrokeopacity{0.516488}%
\pgfsetdash{}{0pt}%
\pgfpathmoveto{\pgfqpoint{0.978285in}{1.734013in}}%
\pgfpathcurveto{\pgfqpoint{0.986521in}{1.734013in}}{\pgfqpoint{0.994421in}{1.737285in}}{\pgfqpoint{1.000245in}{1.743109in}}%
\pgfpathcurveto{\pgfqpoint{1.006069in}{1.748933in}}{\pgfqpoint{1.009342in}{1.756833in}}{\pgfqpoint{1.009342in}{1.765069in}}%
\pgfpathcurveto{\pgfqpoint{1.009342in}{1.773306in}}{\pgfqpoint{1.006069in}{1.781206in}}{\pgfqpoint{1.000245in}{1.787030in}}%
\pgfpathcurveto{\pgfqpoint{0.994421in}{1.792853in}}{\pgfqpoint{0.986521in}{1.796126in}}{\pgfqpoint{0.978285in}{1.796126in}}%
\pgfpathcurveto{\pgfqpoint{0.970049in}{1.796126in}}{\pgfqpoint{0.962149in}{1.792853in}}{\pgfqpoint{0.956325in}{1.787030in}}%
\pgfpathcurveto{\pgfqpoint{0.950501in}{1.781206in}}{\pgfqpoint{0.947229in}{1.773306in}}{\pgfqpoint{0.947229in}{1.765069in}}%
\pgfpathcurveto{\pgfqpoint{0.947229in}{1.756833in}}{\pgfqpoint{0.950501in}{1.748933in}}{\pgfqpoint{0.956325in}{1.743109in}}%
\pgfpathcurveto{\pgfqpoint{0.962149in}{1.737285in}}{\pgfqpoint{0.970049in}{1.734013in}}{\pgfqpoint{0.978285in}{1.734013in}}%
\pgfpathclose%
\pgfusepath{stroke,fill}%
\end{pgfscope}%
\begin{pgfscope}%
\pgfpathrectangle{\pgfqpoint{0.100000in}{0.220728in}}{\pgfqpoint{3.696000in}{3.696000in}}%
\pgfusepath{clip}%
\pgfsetbuttcap%
\pgfsetroundjoin%
\definecolor{currentfill}{rgb}{0.121569,0.466667,0.705882}%
\pgfsetfillcolor{currentfill}%
\pgfsetfillopacity{0.521194}%
\pgfsetlinewidth{1.003750pt}%
\definecolor{currentstroke}{rgb}{0.121569,0.466667,0.705882}%
\pgfsetstrokecolor{currentstroke}%
\pgfsetstrokeopacity{0.521194}%
\pgfsetdash{}{0pt}%
\pgfpathmoveto{\pgfqpoint{2.604319in}{2.981664in}}%
\pgfpathcurveto{\pgfqpoint{2.612555in}{2.981664in}}{\pgfqpoint{2.620455in}{2.984936in}}{\pgfqpoint{2.626279in}{2.990760in}}%
\pgfpathcurveto{\pgfqpoint{2.632103in}{2.996584in}}{\pgfqpoint{2.635376in}{3.004484in}}{\pgfqpoint{2.635376in}{3.012720in}}%
\pgfpathcurveto{\pgfqpoint{2.635376in}{3.020957in}}{\pgfqpoint{2.632103in}{3.028857in}}{\pgfqpoint{2.626279in}{3.034681in}}%
\pgfpathcurveto{\pgfqpoint{2.620455in}{3.040505in}}{\pgfqpoint{2.612555in}{3.043777in}}{\pgfqpoint{2.604319in}{3.043777in}}%
\pgfpathcurveto{\pgfqpoint{2.596083in}{3.043777in}}{\pgfqpoint{2.588183in}{3.040505in}}{\pgfqpoint{2.582359in}{3.034681in}}%
\pgfpathcurveto{\pgfqpoint{2.576535in}{3.028857in}}{\pgfqpoint{2.573263in}{3.020957in}}{\pgfqpoint{2.573263in}{3.012720in}}%
\pgfpathcurveto{\pgfqpoint{2.573263in}{3.004484in}}{\pgfqpoint{2.576535in}{2.996584in}}{\pgfqpoint{2.582359in}{2.990760in}}%
\pgfpathcurveto{\pgfqpoint{2.588183in}{2.984936in}}{\pgfqpoint{2.596083in}{2.981664in}}{\pgfqpoint{2.604319in}{2.981664in}}%
\pgfpathclose%
\pgfusepath{stroke,fill}%
\end{pgfscope}%
\begin{pgfscope}%
\pgfpathrectangle{\pgfqpoint{0.100000in}{0.220728in}}{\pgfqpoint{3.696000in}{3.696000in}}%
\pgfusepath{clip}%
\pgfsetbuttcap%
\pgfsetroundjoin%
\definecolor{currentfill}{rgb}{0.121569,0.466667,0.705882}%
\pgfsetfillcolor{currentfill}%
\pgfsetfillopacity{0.522390}%
\pgfsetlinewidth{1.003750pt}%
\definecolor{currentstroke}{rgb}{0.121569,0.466667,0.705882}%
\pgfsetstrokecolor{currentstroke}%
\pgfsetstrokeopacity{0.522390}%
\pgfsetdash{}{0pt}%
\pgfpathmoveto{\pgfqpoint{0.948010in}{1.698003in}}%
\pgfpathcurveto{\pgfqpoint{0.956246in}{1.698003in}}{\pgfqpoint{0.964146in}{1.701275in}}{\pgfqpoint{0.969970in}{1.707099in}}%
\pgfpathcurveto{\pgfqpoint{0.975794in}{1.712923in}}{\pgfqpoint{0.979066in}{1.720823in}}{\pgfqpoint{0.979066in}{1.729059in}}%
\pgfpathcurveto{\pgfqpoint{0.979066in}{1.737295in}}{\pgfqpoint{0.975794in}{1.745195in}}{\pgfqpoint{0.969970in}{1.751019in}}%
\pgfpathcurveto{\pgfqpoint{0.964146in}{1.756843in}}{\pgfqpoint{0.956246in}{1.760116in}}{\pgfqpoint{0.948010in}{1.760116in}}%
\pgfpathcurveto{\pgfqpoint{0.939774in}{1.760116in}}{\pgfqpoint{0.931874in}{1.756843in}}{\pgfqpoint{0.926050in}{1.751019in}}%
\pgfpathcurveto{\pgfqpoint{0.920226in}{1.745195in}}{\pgfqpoint{0.916953in}{1.737295in}}{\pgfqpoint{0.916953in}{1.729059in}}%
\pgfpathcurveto{\pgfqpoint{0.916953in}{1.720823in}}{\pgfqpoint{0.920226in}{1.712923in}}{\pgfqpoint{0.926050in}{1.707099in}}%
\pgfpathcurveto{\pgfqpoint{0.931874in}{1.701275in}}{\pgfqpoint{0.939774in}{1.698003in}}{\pgfqpoint{0.948010in}{1.698003in}}%
\pgfpathclose%
\pgfusepath{stroke,fill}%
\end{pgfscope}%
\begin{pgfscope}%
\pgfpathrectangle{\pgfqpoint{0.100000in}{0.220728in}}{\pgfqpoint{3.696000in}{3.696000in}}%
\pgfusepath{clip}%
\pgfsetbuttcap%
\pgfsetroundjoin%
\definecolor{currentfill}{rgb}{0.121569,0.466667,0.705882}%
\pgfsetfillcolor{currentfill}%
\pgfsetfillopacity{0.524320}%
\pgfsetlinewidth{1.003750pt}%
\definecolor{currentstroke}{rgb}{0.121569,0.466667,0.705882}%
\pgfsetstrokecolor{currentstroke}%
\pgfsetstrokeopacity{0.524320}%
\pgfsetdash{}{0pt}%
\pgfpathmoveto{\pgfqpoint{2.614749in}{2.979799in}}%
\pgfpathcurveto{\pgfqpoint{2.622985in}{2.979799in}}{\pgfqpoint{2.630885in}{2.983072in}}{\pgfqpoint{2.636709in}{2.988895in}}%
\pgfpathcurveto{\pgfqpoint{2.642533in}{2.994719in}}{\pgfqpoint{2.645805in}{3.002619in}}{\pgfqpoint{2.645805in}{3.010856in}}%
\pgfpathcurveto{\pgfqpoint{2.645805in}{3.019092in}}{\pgfqpoint{2.642533in}{3.026992in}}{\pgfqpoint{2.636709in}{3.032816in}}%
\pgfpathcurveto{\pgfqpoint{2.630885in}{3.038640in}}{\pgfqpoint{2.622985in}{3.041912in}}{\pgfqpoint{2.614749in}{3.041912in}}%
\pgfpathcurveto{\pgfqpoint{2.606513in}{3.041912in}}{\pgfqpoint{2.598613in}{3.038640in}}{\pgfqpoint{2.592789in}{3.032816in}}%
\pgfpathcurveto{\pgfqpoint{2.586965in}{3.026992in}}{\pgfqpoint{2.583692in}{3.019092in}}{\pgfqpoint{2.583692in}{3.010856in}}%
\pgfpathcurveto{\pgfqpoint{2.583692in}{3.002619in}}{\pgfqpoint{2.586965in}{2.994719in}}{\pgfqpoint{2.592789in}{2.988895in}}%
\pgfpathcurveto{\pgfqpoint{2.598613in}{2.983072in}}{\pgfqpoint{2.606513in}{2.979799in}}{\pgfqpoint{2.614749in}{2.979799in}}%
\pgfpathclose%
\pgfusepath{stroke,fill}%
\end{pgfscope}%
\begin{pgfscope}%
\pgfpathrectangle{\pgfqpoint{0.100000in}{0.220728in}}{\pgfqpoint{3.696000in}{3.696000in}}%
\pgfusepath{clip}%
\pgfsetbuttcap%
\pgfsetroundjoin%
\definecolor{currentfill}{rgb}{0.121569,0.466667,0.705882}%
\pgfsetfillcolor{currentfill}%
\pgfsetfillopacity{0.525776}%
\pgfsetlinewidth{1.003750pt}%
\definecolor{currentstroke}{rgb}{0.121569,0.466667,0.705882}%
\pgfsetstrokecolor{currentstroke}%
\pgfsetstrokeopacity{0.525776}%
\pgfsetdash{}{0pt}%
\pgfpathmoveto{\pgfqpoint{2.630927in}{2.979790in}}%
\pgfpathcurveto{\pgfqpoint{2.639163in}{2.979790in}}{\pgfqpoint{2.647064in}{2.983062in}}{\pgfqpoint{2.652887in}{2.988886in}}%
\pgfpathcurveto{\pgfqpoint{2.658711in}{2.994710in}}{\pgfqpoint{2.661984in}{3.002610in}}{\pgfqpoint{2.661984in}{3.010847in}}%
\pgfpathcurveto{\pgfqpoint{2.661984in}{3.019083in}}{\pgfqpoint{2.658711in}{3.026983in}}{\pgfqpoint{2.652887in}{3.032807in}}%
\pgfpathcurveto{\pgfqpoint{2.647064in}{3.038631in}}{\pgfqpoint{2.639163in}{3.041903in}}{\pgfqpoint{2.630927in}{3.041903in}}%
\pgfpathcurveto{\pgfqpoint{2.622691in}{3.041903in}}{\pgfqpoint{2.614791in}{3.038631in}}{\pgfqpoint{2.608967in}{3.032807in}}%
\pgfpathcurveto{\pgfqpoint{2.603143in}{3.026983in}}{\pgfqpoint{2.599871in}{3.019083in}}{\pgfqpoint{2.599871in}{3.010847in}}%
\pgfpathcurveto{\pgfqpoint{2.599871in}{3.002610in}}{\pgfqpoint{2.603143in}{2.994710in}}{\pgfqpoint{2.608967in}{2.988886in}}%
\pgfpathcurveto{\pgfqpoint{2.614791in}{2.983062in}}{\pgfqpoint{2.622691in}{2.979790in}}{\pgfqpoint{2.630927in}{2.979790in}}%
\pgfpathclose%
\pgfusepath{stroke,fill}%
\end{pgfscope}%
\begin{pgfscope}%
\pgfpathrectangle{\pgfqpoint{0.100000in}{0.220728in}}{\pgfqpoint{3.696000in}{3.696000in}}%
\pgfusepath{clip}%
\pgfsetbuttcap%
\pgfsetroundjoin%
\definecolor{currentfill}{rgb}{0.121569,0.466667,0.705882}%
\pgfsetfillcolor{currentfill}%
\pgfsetfillopacity{0.525839}%
\pgfsetlinewidth{1.003750pt}%
\definecolor{currentstroke}{rgb}{0.121569,0.466667,0.705882}%
\pgfsetstrokecolor{currentstroke}%
\pgfsetstrokeopacity{0.525839}%
\pgfsetdash{}{0pt}%
\pgfpathmoveto{\pgfqpoint{2.621005in}{2.979156in}}%
\pgfpathcurveto{\pgfqpoint{2.629241in}{2.979156in}}{\pgfqpoint{2.637141in}{2.982429in}}{\pgfqpoint{2.642965in}{2.988253in}}%
\pgfpathcurveto{\pgfqpoint{2.648789in}{2.994077in}}{\pgfqpoint{2.652061in}{3.001977in}}{\pgfqpoint{2.652061in}{3.010213in}}%
\pgfpathcurveto{\pgfqpoint{2.652061in}{3.018449in}}{\pgfqpoint{2.648789in}{3.026349in}}{\pgfqpoint{2.642965in}{3.032173in}}%
\pgfpathcurveto{\pgfqpoint{2.637141in}{3.037997in}}{\pgfqpoint{2.629241in}{3.041269in}}{\pgfqpoint{2.621005in}{3.041269in}}%
\pgfpathcurveto{\pgfqpoint{2.612769in}{3.041269in}}{\pgfqpoint{2.604869in}{3.037997in}}{\pgfqpoint{2.599045in}{3.032173in}}%
\pgfpathcurveto{\pgfqpoint{2.593221in}{3.026349in}}{\pgfqpoint{2.589948in}{3.018449in}}{\pgfqpoint{2.589948in}{3.010213in}}%
\pgfpathcurveto{\pgfqpoint{2.589948in}{3.001977in}}{\pgfqpoint{2.593221in}{2.994077in}}{\pgfqpoint{2.599045in}{2.988253in}}%
\pgfpathcurveto{\pgfqpoint{2.604869in}{2.982429in}}{\pgfqpoint{2.612769in}{2.979156in}}{\pgfqpoint{2.621005in}{2.979156in}}%
\pgfpathclose%
\pgfusepath{stroke,fill}%
\end{pgfscope}%
\begin{pgfscope}%
\pgfpathrectangle{\pgfqpoint{0.100000in}{0.220728in}}{\pgfqpoint{3.696000in}{3.696000in}}%
\pgfusepath{clip}%
\pgfsetbuttcap%
\pgfsetroundjoin%
\definecolor{currentfill}{rgb}{0.121569,0.466667,0.705882}%
\pgfsetfillcolor{currentfill}%
\pgfsetfillopacity{0.529374}%
\pgfsetlinewidth{1.003750pt}%
\definecolor{currentstroke}{rgb}{0.121569,0.466667,0.705882}%
\pgfsetstrokecolor{currentstroke}%
\pgfsetstrokeopacity{0.529374}%
\pgfsetdash{}{0pt}%
\pgfpathmoveto{\pgfqpoint{0.938171in}{1.663251in}}%
\pgfpathcurveto{\pgfqpoint{0.946407in}{1.663251in}}{\pgfqpoint{0.954307in}{1.666523in}}{\pgfqpoint{0.960131in}{1.672347in}}%
\pgfpathcurveto{\pgfqpoint{0.965955in}{1.678171in}}{\pgfqpoint{0.969227in}{1.686071in}}{\pgfqpoint{0.969227in}{1.694307in}}%
\pgfpathcurveto{\pgfqpoint{0.969227in}{1.702544in}}{\pgfqpoint{0.965955in}{1.710444in}}{\pgfqpoint{0.960131in}{1.716268in}}%
\pgfpathcurveto{\pgfqpoint{0.954307in}{1.722091in}}{\pgfqpoint{0.946407in}{1.725364in}}{\pgfqpoint{0.938171in}{1.725364in}}%
\pgfpathcurveto{\pgfqpoint{0.929935in}{1.725364in}}{\pgfqpoint{0.922034in}{1.722091in}}{\pgfqpoint{0.916211in}{1.716268in}}%
\pgfpathcurveto{\pgfqpoint{0.910387in}{1.710444in}}{\pgfqpoint{0.907114in}{1.702544in}}{\pgfqpoint{0.907114in}{1.694307in}}%
\pgfpathcurveto{\pgfqpoint{0.907114in}{1.686071in}}{\pgfqpoint{0.910387in}{1.678171in}}{\pgfqpoint{0.916211in}{1.672347in}}%
\pgfpathcurveto{\pgfqpoint{0.922034in}{1.666523in}}{\pgfqpoint{0.929935in}{1.663251in}}{\pgfqpoint{0.938171in}{1.663251in}}%
\pgfpathclose%
\pgfusepath{stroke,fill}%
\end{pgfscope}%
\begin{pgfscope}%
\pgfpathrectangle{\pgfqpoint{0.100000in}{0.220728in}}{\pgfqpoint{3.696000in}{3.696000in}}%
\pgfusepath{clip}%
\pgfsetbuttcap%
\pgfsetroundjoin%
\definecolor{currentfill}{rgb}{0.121569,0.466667,0.705882}%
\pgfsetfillcolor{currentfill}%
\pgfsetfillopacity{0.529962}%
\pgfsetlinewidth{1.003750pt}%
\definecolor{currentstroke}{rgb}{0.121569,0.466667,0.705882}%
\pgfsetstrokecolor{currentstroke}%
\pgfsetstrokeopacity{0.529962}%
\pgfsetdash{}{0pt}%
\pgfpathmoveto{\pgfqpoint{2.645184in}{2.979162in}}%
\pgfpathcurveto{\pgfqpoint{2.653420in}{2.979162in}}{\pgfqpoint{2.661320in}{2.982435in}}{\pgfqpoint{2.667144in}{2.988259in}}%
\pgfpathcurveto{\pgfqpoint{2.672968in}{2.994083in}}{\pgfqpoint{2.676240in}{3.001983in}}{\pgfqpoint{2.676240in}{3.010219in}}%
\pgfpathcurveto{\pgfqpoint{2.676240in}{3.018455in}}{\pgfqpoint{2.672968in}{3.026355in}}{\pgfqpoint{2.667144in}{3.032179in}}%
\pgfpathcurveto{\pgfqpoint{2.661320in}{3.038003in}}{\pgfqpoint{2.653420in}{3.041275in}}{\pgfqpoint{2.645184in}{3.041275in}}%
\pgfpathcurveto{\pgfqpoint{2.636947in}{3.041275in}}{\pgfqpoint{2.629047in}{3.038003in}}{\pgfqpoint{2.623223in}{3.032179in}}%
\pgfpathcurveto{\pgfqpoint{2.617399in}{3.026355in}}{\pgfqpoint{2.614127in}{3.018455in}}{\pgfqpoint{2.614127in}{3.010219in}}%
\pgfpathcurveto{\pgfqpoint{2.614127in}{3.001983in}}{\pgfqpoint{2.617399in}{2.994083in}}{\pgfqpoint{2.623223in}{2.988259in}}%
\pgfpathcurveto{\pgfqpoint{2.629047in}{2.982435in}}{\pgfqpoint{2.636947in}{2.979162in}}{\pgfqpoint{2.645184in}{2.979162in}}%
\pgfpathclose%
\pgfusepath{stroke,fill}%
\end{pgfscope}%
\begin{pgfscope}%
\pgfpathrectangle{\pgfqpoint{0.100000in}{0.220728in}}{\pgfqpoint{3.696000in}{3.696000in}}%
\pgfusepath{clip}%
\pgfsetbuttcap%
\pgfsetroundjoin%
\definecolor{currentfill}{rgb}{0.121569,0.466667,0.705882}%
\pgfsetfillcolor{currentfill}%
\pgfsetfillopacity{0.532502}%
\pgfsetlinewidth{1.003750pt}%
\definecolor{currentstroke}{rgb}{0.121569,0.466667,0.705882}%
\pgfsetstrokecolor{currentstroke}%
\pgfsetstrokeopacity{0.532502}%
\pgfsetdash{}{0pt}%
\pgfpathmoveto{\pgfqpoint{0.921844in}{1.642740in}}%
\pgfpathcurveto{\pgfqpoint{0.930080in}{1.642740in}}{\pgfqpoint{0.937980in}{1.646013in}}{\pgfqpoint{0.943804in}{1.651837in}}%
\pgfpathcurveto{\pgfqpoint{0.949628in}{1.657660in}}{\pgfqpoint{0.952900in}{1.665560in}}{\pgfqpoint{0.952900in}{1.673797in}}%
\pgfpathcurveto{\pgfqpoint{0.952900in}{1.682033in}}{\pgfqpoint{0.949628in}{1.689933in}}{\pgfqpoint{0.943804in}{1.695757in}}%
\pgfpathcurveto{\pgfqpoint{0.937980in}{1.701581in}}{\pgfqpoint{0.930080in}{1.704853in}}{\pgfqpoint{0.921844in}{1.704853in}}%
\pgfpathcurveto{\pgfqpoint{0.913607in}{1.704853in}}{\pgfqpoint{0.905707in}{1.701581in}}{\pgfqpoint{0.899883in}{1.695757in}}%
\pgfpathcurveto{\pgfqpoint{0.894060in}{1.689933in}}{\pgfqpoint{0.890787in}{1.682033in}}{\pgfqpoint{0.890787in}{1.673797in}}%
\pgfpathcurveto{\pgfqpoint{0.890787in}{1.665560in}}{\pgfqpoint{0.894060in}{1.657660in}}{\pgfqpoint{0.899883in}{1.651837in}}%
\pgfpathcurveto{\pgfqpoint{0.905707in}{1.646013in}}{\pgfqpoint{0.913607in}{1.642740in}}{\pgfqpoint{0.921844in}{1.642740in}}%
\pgfpathclose%
\pgfusepath{stroke,fill}%
\end{pgfscope}%
\begin{pgfscope}%
\pgfpathrectangle{\pgfqpoint{0.100000in}{0.220728in}}{\pgfqpoint{3.696000in}{3.696000in}}%
\pgfusepath{clip}%
\pgfsetbuttcap%
\pgfsetroundjoin%
\definecolor{currentfill}{rgb}{0.121569,0.466667,0.705882}%
\pgfsetfillcolor{currentfill}%
\pgfsetfillopacity{0.533357}%
\pgfsetlinewidth{1.003750pt}%
\definecolor{currentstroke}{rgb}{0.121569,0.466667,0.705882}%
\pgfsetstrokecolor{currentstroke}%
\pgfsetstrokeopacity{0.533357}%
\pgfsetdash{}{0pt}%
\pgfpathmoveto{\pgfqpoint{2.664278in}{2.977869in}}%
\pgfpathcurveto{\pgfqpoint{2.672515in}{2.977869in}}{\pgfqpoint{2.680415in}{2.981141in}}{\pgfqpoint{2.686239in}{2.986965in}}%
\pgfpathcurveto{\pgfqpoint{2.692062in}{2.992789in}}{\pgfqpoint{2.695335in}{3.000689in}}{\pgfqpoint{2.695335in}{3.008925in}}%
\pgfpathcurveto{\pgfqpoint{2.695335in}{3.017161in}}{\pgfqpoint{2.692062in}{3.025061in}}{\pgfqpoint{2.686239in}{3.030885in}}%
\pgfpathcurveto{\pgfqpoint{2.680415in}{3.036709in}}{\pgfqpoint{2.672515in}{3.039982in}}{\pgfqpoint{2.664278in}{3.039982in}}%
\pgfpathcurveto{\pgfqpoint{2.656042in}{3.039982in}}{\pgfqpoint{2.648142in}{3.036709in}}{\pgfqpoint{2.642318in}{3.030885in}}%
\pgfpathcurveto{\pgfqpoint{2.636494in}{3.025061in}}{\pgfqpoint{2.633222in}{3.017161in}}{\pgfqpoint{2.633222in}{3.008925in}}%
\pgfpathcurveto{\pgfqpoint{2.633222in}{3.000689in}}{\pgfqpoint{2.636494in}{2.992789in}}{\pgfqpoint{2.642318in}{2.986965in}}%
\pgfpathcurveto{\pgfqpoint{2.648142in}{2.981141in}}{\pgfqpoint{2.656042in}{2.977869in}}{\pgfqpoint{2.664278in}{2.977869in}}%
\pgfpathclose%
\pgfusepath{stroke,fill}%
\end{pgfscope}%
\begin{pgfscope}%
\pgfpathrectangle{\pgfqpoint{0.100000in}{0.220728in}}{\pgfqpoint{3.696000in}{3.696000in}}%
\pgfusepath{clip}%
\pgfsetbuttcap%
\pgfsetroundjoin%
\definecolor{currentfill}{rgb}{0.121569,0.466667,0.705882}%
\pgfsetfillcolor{currentfill}%
\pgfsetfillopacity{0.534709}%
\pgfsetlinewidth{1.003750pt}%
\definecolor{currentstroke}{rgb}{0.121569,0.466667,0.705882}%
\pgfsetstrokecolor{currentstroke}%
\pgfsetstrokeopacity{0.534709}%
\pgfsetdash{}{0pt}%
\pgfpathmoveto{\pgfqpoint{0.916152in}{1.630536in}}%
\pgfpathcurveto{\pgfqpoint{0.924389in}{1.630536in}}{\pgfqpoint{0.932289in}{1.633809in}}{\pgfqpoint{0.938113in}{1.639633in}}%
\pgfpathcurveto{\pgfqpoint{0.943937in}{1.645457in}}{\pgfqpoint{0.947209in}{1.653357in}}{\pgfqpoint{0.947209in}{1.661593in}}%
\pgfpathcurveto{\pgfqpoint{0.947209in}{1.669829in}}{\pgfqpoint{0.943937in}{1.677729in}}{\pgfqpoint{0.938113in}{1.683553in}}%
\pgfpathcurveto{\pgfqpoint{0.932289in}{1.689377in}}{\pgfqpoint{0.924389in}{1.692649in}}{\pgfqpoint{0.916152in}{1.692649in}}%
\pgfpathcurveto{\pgfqpoint{0.907916in}{1.692649in}}{\pgfqpoint{0.900016in}{1.689377in}}{\pgfqpoint{0.894192in}{1.683553in}}%
\pgfpathcurveto{\pgfqpoint{0.888368in}{1.677729in}}{\pgfqpoint{0.885096in}{1.669829in}}{\pgfqpoint{0.885096in}{1.661593in}}%
\pgfpathcurveto{\pgfqpoint{0.885096in}{1.653357in}}{\pgfqpoint{0.888368in}{1.645457in}}{\pgfqpoint{0.894192in}{1.639633in}}%
\pgfpathcurveto{\pgfqpoint{0.900016in}{1.633809in}}{\pgfqpoint{0.907916in}{1.630536in}}{\pgfqpoint{0.916152in}{1.630536in}}%
\pgfpathclose%
\pgfusepath{stroke,fill}%
\end{pgfscope}%
\begin{pgfscope}%
\pgfpathrectangle{\pgfqpoint{0.100000in}{0.220728in}}{\pgfqpoint{3.696000in}{3.696000in}}%
\pgfusepath{clip}%
\pgfsetbuttcap%
\pgfsetroundjoin%
\definecolor{currentfill}{rgb}{0.121569,0.466667,0.705882}%
\pgfsetfillcolor{currentfill}%
\pgfsetfillopacity{0.535583}%
\pgfsetlinewidth{1.003750pt}%
\definecolor{currentstroke}{rgb}{0.121569,0.466667,0.705882}%
\pgfsetstrokecolor{currentstroke}%
\pgfsetstrokeopacity{0.535583}%
\pgfsetdash{}{0pt}%
\pgfpathmoveto{\pgfqpoint{0.913032in}{1.625031in}}%
\pgfpathcurveto{\pgfqpoint{0.921269in}{1.625031in}}{\pgfqpoint{0.929169in}{1.628303in}}{\pgfqpoint{0.934993in}{1.634127in}}%
\pgfpathcurveto{\pgfqpoint{0.940817in}{1.639951in}}{\pgfqpoint{0.944089in}{1.647851in}}{\pgfqpoint{0.944089in}{1.656087in}}%
\pgfpathcurveto{\pgfqpoint{0.944089in}{1.664324in}}{\pgfqpoint{0.940817in}{1.672224in}}{\pgfqpoint{0.934993in}{1.678048in}}%
\pgfpathcurveto{\pgfqpoint{0.929169in}{1.683872in}}{\pgfqpoint{0.921269in}{1.687144in}}{\pgfqpoint{0.913032in}{1.687144in}}%
\pgfpathcurveto{\pgfqpoint{0.904796in}{1.687144in}}{\pgfqpoint{0.896896in}{1.683872in}}{\pgfqpoint{0.891072in}{1.678048in}}%
\pgfpathcurveto{\pgfqpoint{0.885248in}{1.672224in}}{\pgfqpoint{0.881976in}{1.664324in}}{\pgfqpoint{0.881976in}{1.656087in}}%
\pgfpathcurveto{\pgfqpoint{0.881976in}{1.647851in}}{\pgfqpoint{0.885248in}{1.639951in}}{\pgfqpoint{0.891072in}{1.634127in}}%
\pgfpathcurveto{\pgfqpoint{0.896896in}{1.628303in}}{\pgfqpoint{0.904796in}{1.625031in}}{\pgfqpoint{0.913032in}{1.625031in}}%
\pgfpathclose%
\pgfusepath{stroke,fill}%
\end{pgfscope}%
\begin{pgfscope}%
\pgfpathrectangle{\pgfqpoint{0.100000in}{0.220728in}}{\pgfqpoint{3.696000in}{3.696000in}}%
\pgfusepath{clip}%
\pgfsetbuttcap%
\pgfsetroundjoin%
\definecolor{currentfill}{rgb}{0.121569,0.466667,0.705882}%
\pgfsetfillcolor{currentfill}%
\pgfsetfillopacity{0.537249}%
\pgfsetlinewidth{1.003750pt}%
\definecolor{currentstroke}{rgb}{0.121569,0.466667,0.705882}%
\pgfsetstrokecolor{currentstroke}%
\pgfsetstrokeopacity{0.537249}%
\pgfsetdash{}{0pt}%
\pgfpathmoveto{\pgfqpoint{0.907282in}{1.615435in}}%
\pgfpathcurveto{\pgfqpoint{0.915518in}{1.615435in}}{\pgfqpoint{0.923418in}{1.618707in}}{\pgfqpoint{0.929242in}{1.624531in}}%
\pgfpathcurveto{\pgfqpoint{0.935066in}{1.630355in}}{\pgfqpoint{0.938338in}{1.638255in}}{\pgfqpoint{0.938338in}{1.646491in}}%
\pgfpathcurveto{\pgfqpoint{0.938338in}{1.654728in}}{\pgfqpoint{0.935066in}{1.662628in}}{\pgfqpoint{0.929242in}{1.668452in}}%
\pgfpathcurveto{\pgfqpoint{0.923418in}{1.674275in}}{\pgfqpoint{0.915518in}{1.677548in}}{\pgfqpoint{0.907282in}{1.677548in}}%
\pgfpathcurveto{\pgfqpoint{0.899045in}{1.677548in}}{\pgfqpoint{0.891145in}{1.674275in}}{\pgfqpoint{0.885322in}{1.668452in}}%
\pgfpathcurveto{\pgfqpoint{0.879498in}{1.662628in}}{\pgfqpoint{0.876225in}{1.654728in}}{\pgfqpoint{0.876225in}{1.646491in}}%
\pgfpathcurveto{\pgfqpoint{0.876225in}{1.638255in}}{\pgfqpoint{0.879498in}{1.630355in}}{\pgfqpoint{0.885322in}{1.624531in}}%
\pgfpathcurveto{\pgfqpoint{0.891145in}{1.618707in}}{\pgfqpoint{0.899045in}{1.615435in}}{\pgfqpoint{0.907282in}{1.615435in}}%
\pgfpathclose%
\pgfusepath{stroke,fill}%
\end{pgfscope}%
\begin{pgfscope}%
\pgfpathrectangle{\pgfqpoint{0.100000in}{0.220728in}}{\pgfqpoint{3.696000in}{3.696000in}}%
\pgfusepath{clip}%
\pgfsetbuttcap%
\pgfsetroundjoin%
\definecolor{currentfill}{rgb}{0.121569,0.466667,0.705882}%
\pgfsetfillcolor{currentfill}%
\pgfsetfillopacity{0.540258}%
\pgfsetlinewidth{1.003750pt}%
\definecolor{currentstroke}{rgb}{0.121569,0.466667,0.705882}%
\pgfsetstrokecolor{currentstroke}%
\pgfsetstrokeopacity{0.540258}%
\pgfsetdash{}{0pt}%
\pgfpathmoveto{\pgfqpoint{2.682332in}{2.974968in}}%
\pgfpathcurveto{\pgfqpoint{2.690568in}{2.974968in}}{\pgfqpoint{2.698468in}{2.978240in}}{\pgfqpoint{2.704292in}{2.984064in}}%
\pgfpathcurveto{\pgfqpoint{2.710116in}{2.989888in}}{\pgfqpoint{2.713389in}{2.997788in}}{\pgfqpoint{2.713389in}{3.006024in}}%
\pgfpathcurveto{\pgfqpoint{2.713389in}{3.014260in}}{\pgfqpoint{2.710116in}{3.022161in}}{\pgfqpoint{2.704292in}{3.027984in}}%
\pgfpathcurveto{\pgfqpoint{2.698468in}{3.033808in}}{\pgfqpoint{2.690568in}{3.037081in}}{\pgfqpoint{2.682332in}{3.037081in}}%
\pgfpathcurveto{\pgfqpoint{2.674096in}{3.037081in}}{\pgfqpoint{2.666196in}{3.033808in}}{\pgfqpoint{2.660372in}{3.027984in}}%
\pgfpathcurveto{\pgfqpoint{2.654548in}{3.022161in}}{\pgfqpoint{2.651276in}{3.014260in}}{\pgfqpoint{2.651276in}{3.006024in}}%
\pgfpathcurveto{\pgfqpoint{2.651276in}{2.997788in}}{\pgfqpoint{2.654548in}{2.989888in}}{\pgfqpoint{2.660372in}{2.984064in}}%
\pgfpathcurveto{\pgfqpoint{2.666196in}{2.978240in}}{\pgfqpoint{2.674096in}{2.974968in}}{\pgfqpoint{2.682332in}{2.974968in}}%
\pgfpathclose%
\pgfusepath{stroke,fill}%
\end{pgfscope}%
\begin{pgfscope}%
\pgfpathrectangle{\pgfqpoint{0.100000in}{0.220728in}}{\pgfqpoint{3.696000in}{3.696000in}}%
\pgfusepath{clip}%
\pgfsetbuttcap%
\pgfsetroundjoin%
\definecolor{currentfill}{rgb}{0.121569,0.466667,0.705882}%
\pgfsetfillcolor{currentfill}%
\pgfsetfillopacity{0.540809}%
\pgfsetlinewidth{1.003750pt}%
\definecolor{currentstroke}{rgb}{0.121569,0.466667,0.705882}%
\pgfsetstrokecolor{currentstroke}%
\pgfsetstrokeopacity{0.540809}%
\pgfsetdash{}{0pt}%
\pgfpathmoveto{\pgfqpoint{0.899425in}{1.597815in}}%
\pgfpathcurveto{\pgfqpoint{0.907662in}{1.597815in}}{\pgfqpoint{0.915562in}{1.601087in}}{\pgfqpoint{0.921386in}{1.606911in}}%
\pgfpathcurveto{\pgfqpoint{0.927209in}{1.612735in}}{\pgfqpoint{0.930482in}{1.620635in}}{\pgfqpoint{0.930482in}{1.628872in}}%
\pgfpathcurveto{\pgfqpoint{0.930482in}{1.637108in}}{\pgfqpoint{0.927209in}{1.645008in}}{\pgfqpoint{0.921386in}{1.650832in}}%
\pgfpathcurveto{\pgfqpoint{0.915562in}{1.656656in}}{\pgfqpoint{0.907662in}{1.659928in}}{\pgfqpoint{0.899425in}{1.659928in}}%
\pgfpathcurveto{\pgfqpoint{0.891189in}{1.659928in}}{\pgfqpoint{0.883289in}{1.656656in}}{\pgfqpoint{0.877465in}{1.650832in}}%
\pgfpathcurveto{\pgfqpoint{0.871641in}{1.645008in}}{\pgfqpoint{0.868369in}{1.637108in}}{\pgfqpoint{0.868369in}{1.628872in}}%
\pgfpathcurveto{\pgfqpoint{0.868369in}{1.620635in}}{\pgfqpoint{0.871641in}{1.612735in}}{\pgfqpoint{0.877465in}{1.606911in}}%
\pgfpathcurveto{\pgfqpoint{0.883289in}{1.601087in}}{\pgfqpoint{0.891189in}{1.597815in}}{\pgfqpoint{0.899425in}{1.597815in}}%
\pgfpathclose%
\pgfusepath{stroke,fill}%
\end{pgfscope}%
\begin{pgfscope}%
\pgfpathrectangle{\pgfqpoint{0.100000in}{0.220728in}}{\pgfqpoint{3.696000in}{3.696000in}}%
\pgfusepath{clip}%
\pgfsetbuttcap%
\pgfsetroundjoin%
\definecolor{currentfill}{rgb}{0.121569,0.466667,0.705882}%
\pgfsetfillcolor{currentfill}%
\pgfsetfillopacity{0.542914}%
\pgfsetlinewidth{1.003750pt}%
\definecolor{currentstroke}{rgb}{0.121569,0.466667,0.705882}%
\pgfsetstrokecolor{currentstroke}%
\pgfsetstrokeopacity{0.542914}%
\pgfsetdash{}{0pt}%
\pgfpathmoveto{\pgfqpoint{2.694538in}{2.973424in}}%
\pgfpathcurveto{\pgfqpoint{2.702774in}{2.973424in}}{\pgfqpoint{2.710674in}{2.976696in}}{\pgfqpoint{2.716498in}{2.982520in}}%
\pgfpathcurveto{\pgfqpoint{2.722322in}{2.988344in}}{\pgfqpoint{2.725595in}{2.996244in}}{\pgfqpoint{2.725595in}{3.004480in}}%
\pgfpathcurveto{\pgfqpoint{2.725595in}{3.012717in}}{\pgfqpoint{2.722322in}{3.020617in}}{\pgfqpoint{2.716498in}{3.026441in}}%
\pgfpathcurveto{\pgfqpoint{2.710674in}{3.032264in}}{\pgfqpoint{2.702774in}{3.035537in}}{\pgfqpoint{2.694538in}{3.035537in}}%
\pgfpathcurveto{\pgfqpoint{2.686302in}{3.035537in}}{\pgfqpoint{2.678402in}{3.032264in}}{\pgfqpoint{2.672578in}{3.026441in}}%
\pgfpathcurveto{\pgfqpoint{2.666754in}{3.020617in}}{\pgfqpoint{2.663482in}{3.012717in}}{\pgfqpoint{2.663482in}{3.004480in}}%
\pgfpathcurveto{\pgfqpoint{2.663482in}{2.996244in}}{\pgfqpoint{2.666754in}{2.988344in}}{\pgfqpoint{2.672578in}{2.982520in}}%
\pgfpathcurveto{\pgfqpoint{2.678402in}{2.976696in}}{\pgfqpoint{2.686302in}{2.973424in}}{\pgfqpoint{2.694538in}{2.973424in}}%
\pgfpathclose%
\pgfusepath{stroke,fill}%
\end{pgfscope}%
\begin{pgfscope}%
\pgfpathrectangle{\pgfqpoint{0.100000in}{0.220728in}}{\pgfqpoint{3.696000in}{3.696000in}}%
\pgfusepath{clip}%
\pgfsetbuttcap%
\pgfsetroundjoin%
\definecolor{currentfill}{rgb}{0.121569,0.466667,0.705882}%
\pgfsetfillcolor{currentfill}%
\pgfsetfillopacity{0.545933}%
\pgfsetlinewidth{1.003750pt}%
\definecolor{currentstroke}{rgb}{0.121569,0.466667,0.705882}%
\pgfsetstrokecolor{currentstroke}%
\pgfsetstrokeopacity{0.545933}%
\pgfsetdash{}{0pt}%
\pgfpathmoveto{\pgfqpoint{0.877551in}{1.568139in}}%
\pgfpathcurveto{\pgfqpoint{0.885787in}{1.568139in}}{\pgfqpoint{0.893687in}{1.571412in}}{\pgfqpoint{0.899511in}{1.577235in}}%
\pgfpathcurveto{\pgfqpoint{0.905335in}{1.583059in}}{\pgfqpoint{0.908607in}{1.590959in}}{\pgfqpoint{0.908607in}{1.599196in}}%
\pgfpathcurveto{\pgfqpoint{0.908607in}{1.607432in}}{\pgfqpoint{0.905335in}{1.615332in}}{\pgfqpoint{0.899511in}{1.621156in}}%
\pgfpathcurveto{\pgfqpoint{0.893687in}{1.626980in}}{\pgfqpoint{0.885787in}{1.630252in}}{\pgfqpoint{0.877551in}{1.630252in}}%
\pgfpathcurveto{\pgfqpoint{0.869314in}{1.630252in}}{\pgfqpoint{0.861414in}{1.626980in}}{\pgfqpoint{0.855590in}{1.621156in}}%
\pgfpathcurveto{\pgfqpoint{0.849766in}{1.615332in}}{\pgfqpoint{0.846494in}{1.607432in}}{\pgfqpoint{0.846494in}{1.599196in}}%
\pgfpathcurveto{\pgfqpoint{0.846494in}{1.590959in}}{\pgfqpoint{0.849766in}{1.583059in}}{\pgfqpoint{0.855590in}{1.577235in}}%
\pgfpathcurveto{\pgfqpoint{0.861414in}{1.571412in}}{\pgfqpoint{0.869314in}{1.568139in}}{\pgfqpoint{0.877551in}{1.568139in}}%
\pgfpathclose%
\pgfusepath{stroke,fill}%
\end{pgfscope}%
\begin{pgfscope}%
\pgfpathrectangle{\pgfqpoint{0.100000in}{0.220728in}}{\pgfqpoint{3.696000in}{3.696000in}}%
\pgfusepath{clip}%
\pgfsetbuttcap%
\pgfsetroundjoin%
\definecolor{currentfill}{rgb}{0.121569,0.466667,0.705882}%
\pgfsetfillcolor{currentfill}%
\pgfsetfillopacity{0.546657}%
\pgfsetlinewidth{1.003750pt}%
\definecolor{currentstroke}{rgb}{0.121569,0.466667,0.705882}%
\pgfsetstrokecolor{currentstroke}%
\pgfsetstrokeopacity{0.546657}%
\pgfsetdash{}{0pt}%
\pgfpathmoveto{\pgfqpoint{2.709388in}{2.971926in}}%
\pgfpathcurveto{\pgfqpoint{2.717624in}{2.971926in}}{\pgfqpoint{2.725524in}{2.975198in}}{\pgfqpoint{2.731348in}{2.981022in}}%
\pgfpathcurveto{\pgfqpoint{2.737172in}{2.986846in}}{\pgfqpoint{2.740445in}{2.994746in}}{\pgfqpoint{2.740445in}{3.002982in}}%
\pgfpathcurveto{\pgfqpoint{2.740445in}{3.011219in}}{\pgfqpoint{2.737172in}{3.019119in}}{\pgfqpoint{2.731348in}{3.024943in}}%
\pgfpathcurveto{\pgfqpoint{2.725524in}{3.030767in}}{\pgfqpoint{2.717624in}{3.034039in}}{\pgfqpoint{2.709388in}{3.034039in}}%
\pgfpathcurveto{\pgfqpoint{2.701152in}{3.034039in}}{\pgfqpoint{2.693252in}{3.030767in}}{\pgfqpoint{2.687428in}{3.024943in}}%
\pgfpathcurveto{\pgfqpoint{2.681604in}{3.019119in}}{\pgfqpoint{2.678332in}{3.011219in}}{\pgfqpoint{2.678332in}{3.002982in}}%
\pgfpathcurveto{\pgfqpoint{2.678332in}{2.994746in}}{\pgfqpoint{2.681604in}{2.986846in}}{\pgfqpoint{2.687428in}{2.981022in}}%
\pgfpathcurveto{\pgfqpoint{2.693252in}{2.975198in}}{\pgfqpoint{2.701152in}{2.971926in}}{\pgfqpoint{2.709388in}{2.971926in}}%
\pgfpathclose%
\pgfusepath{stroke,fill}%
\end{pgfscope}%
\begin{pgfscope}%
\pgfpathrectangle{\pgfqpoint{0.100000in}{0.220728in}}{\pgfqpoint{3.696000in}{3.696000in}}%
\pgfusepath{clip}%
\pgfsetbuttcap%
\pgfsetroundjoin%
\definecolor{currentfill}{rgb}{0.121569,0.466667,0.705882}%
\pgfsetfillcolor{currentfill}%
\pgfsetfillopacity{0.550881}%
\pgfsetlinewidth{1.003750pt}%
\definecolor{currentstroke}{rgb}{0.121569,0.466667,0.705882}%
\pgfsetstrokecolor{currentstroke}%
\pgfsetstrokeopacity{0.550881}%
\pgfsetdash{}{0pt}%
\pgfpathmoveto{\pgfqpoint{2.727295in}{2.969421in}}%
\pgfpathcurveto{\pgfqpoint{2.735532in}{2.969421in}}{\pgfqpoint{2.743432in}{2.972694in}}{\pgfqpoint{2.749256in}{2.978517in}}%
\pgfpathcurveto{\pgfqpoint{2.755080in}{2.984341in}}{\pgfqpoint{2.758352in}{2.992241in}}{\pgfqpoint{2.758352in}{3.000478in}}%
\pgfpathcurveto{\pgfqpoint{2.758352in}{3.008714in}}{\pgfqpoint{2.755080in}{3.016614in}}{\pgfqpoint{2.749256in}{3.022438in}}%
\pgfpathcurveto{\pgfqpoint{2.743432in}{3.028262in}}{\pgfqpoint{2.735532in}{3.031534in}}{\pgfqpoint{2.727295in}{3.031534in}}%
\pgfpathcurveto{\pgfqpoint{2.719059in}{3.031534in}}{\pgfqpoint{2.711159in}{3.028262in}}{\pgfqpoint{2.705335in}{3.022438in}}%
\pgfpathcurveto{\pgfqpoint{2.699511in}{3.016614in}}{\pgfqpoint{2.696239in}{3.008714in}}{\pgfqpoint{2.696239in}{3.000478in}}%
\pgfpathcurveto{\pgfqpoint{2.696239in}{2.992241in}}{\pgfqpoint{2.699511in}{2.984341in}}{\pgfqpoint{2.705335in}{2.978517in}}%
\pgfpathcurveto{\pgfqpoint{2.711159in}{2.972694in}}{\pgfqpoint{2.719059in}{2.969421in}}{\pgfqpoint{2.727295in}{2.969421in}}%
\pgfpathclose%
\pgfusepath{stroke,fill}%
\end{pgfscope}%
\begin{pgfscope}%
\pgfpathrectangle{\pgfqpoint{0.100000in}{0.220728in}}{\pgfqpoint{3.696000in}{3.696000in}}%
\pgfusepath{clip}%
\pgfsetbuttcap%
\pgfsetroundjoin%
\definecolor{currentfill}{rgb}{0.121569,0.466667,0.705882}%
\pgfsetfillcolor{currentfill}%
\pgfsetfillopacity{0.551259}%
\pgfsetlinewidth{1.003750pt}%
\definecolor{currentstroke}{rgb}{0.121569,0.466667,0.705882}%
\pgfsetstrokecolor{currentstroke}%
\pgfsetstrokeopacity{0.551259}%
\pgfsetdash{}{0pt}%
\pgfpathmoveto{\pgfqpoint{0.869845in}{1.544945in}}%
\pgfpathcurveto{\pgfqpoint{0.878081in}{1.544945in}}{\pgfqpoint{0.885981in}{1.548218in}}{\pgfqpoint{0.891805in}{1.554042in}}%
\pgfpathcurveto{\pgfqpoint{0.897629in}{1.559865in}}{\pgfqpoint{0.900901in}{1.567765in}}{\pgfqpoint{0.900901in}{1.576002in}}%
\pgfpathcurveto{\pgfqpoint{0.900901in}{1.584238in}}{\pgfqpoint{0.897629in}{1.592138in}}{\pgfqpoint{0.891805in}{1.597962in}}%
\pgfpathcurveto{\pgfqpoint{0.885981in}{1.603786in}}{\pgfqpoint{0.878081in}{1.607058in}}{\pgfqpoint{0.869845in}{1.607058in}}%
\pgfpathcurveto{\pgfqpoint{0.861609in}{1.607058in}}{\pgfqpoint{0.853708in}{1.603786in}}{\pgfqpoint{0.847885in}{1.597962in}}%
\pgfpathcurveto{\pgfqpoint{0.842061in}{1.592138in}}{\pgfqpoint{0.838788in}{1.584238in}}{\pgfqpoint{0.838788in}{1.576002in}}%
\pgfpathcurveto{\pgfqpoint{0.838788in}{1.567765in}}{\pgfqpoint{0.842061in}{1.559865in}}{\pgfqpoint{0.847885in}{1.554042in}}%
\pgfpathcurveto{\pgfqpoint{0.853708in}{1.548218in}}{\pgfqpoint{0.861609in}{1.544945in}}{\pgfqpoint{0.869845in}{1.544945in}}%
\pgfpathclose%
\pgfusepath{stroke,fill}%
\end{pgfscope}%
\begin{pgfscope}%
\pgfpathrectangle{\pgfqpoint{0.100000in}{0.220728in}}{\pgfqpoint{3.696000in}{3.696000in}}%
\pgfusepath{clip}%
\pgfsetbuttcap%
\pgfsetroundjoin%
\definecolor{currentfill}{rgb}{0.121569,0.466667,0.705882}%
\pgfsetfillcolor{currentfill}%
\pgfsetfillopacity{0.553263}%
\pgfsetlinewidth{1.003750pt}%
\definecolor{currentstroke}{rgb}{0.121569,0.466667,0.705882}%
\pgfsetstrokecolor{currentstroke}%
\pgfsetstrokeopacity{0.553263}%
\pgfsetdash{}{0pt}%
\pgfpathmoveto{\pgfqpoint{2.737015in}{2.967908in}}%
\pgfpathcurveto{\pgfqpoint{2.745251in}{2.967908in}}{\pgfqpoint{2.753151in}{2.971180in}}{\pgfqpoint{2.758975in}{2.977004in}}%
\pgfpathcurveto{\pgfqpoint{2.764799in}{2.982828in}}{\pgfqpoint{2.768071in}{2.990728in}}{\pgfqpoint{2.768071in}{2.998965in}}%
\pgfpathcurveto{\pgfqpoint{2.768071in}{3.007201in}}{\pgfqpoint{2.764799in}{3.015101in}}{\pgfqpoint{2.758975in}{3.020925in}}%
\pgfpathcurveto{\pgfqpoint{2.753151in}{3.026749in}}{\pgfqpoint{2.745251in}{3.030021in}}{\pgfqpoint{2.737015in}{3.030021in}}%
\pgfpathcurveto{\pgfqpoint{2.728779in}{3.030021in}}{\pgfqpoint{2.720878in}{3.026749in}}{\pgfqpoint{2.715055in}{3.020925in}}%
\pgfpathcurveto{\pgfqpoint{2.709231in}{3.015101in}}{\pgfqpoint{2.705958in}{3.007201in}}{\pgfqpoint{2.705958in}{2.998965in}}%
\pgfpathcurveto{\pgfqpoint{2.705958in}{2.990728in}}{\pgfqpoint{2.709231in}{2.982828in}}{\pgfqpoint{2.715055in}{2.977004in}}%
\pgfpathcurveto{\pgfqpoint{2.720878in}{2.971180in}}{\pgfqpoint{2.728779in}{2.967908in}}{\pgfqpoint{2.737015in}{2.967908in}}%
\pgfpathclose%
\pgfusepath{stroke,fill}%
\end{pgfscope}%
\begin{pgfscope}%
\pgfpathrectangle{\pgfqpoint{0.100000in}{0.220728in}}{\pgfqpoint{3.696000in}{3.696000in}}%
\pgfusepath{clip}%
\pgfsetbuttcap%
\pgfsetroundjoin%
\definecolor{currentfill}{rgb}{0.121569,0.466667,0.705882}%
\pgfsetfillcolor{currentfill}%
\pgfsetfillopacity{0.554727}%
\pgfsetlinewidth{1.003750pt}%
\definecolor{currentstroke}{rgb}{0.121569,0.466667,0.705882}%
\pgfsetstrokecolor{currentstroke}%
\pgfsetstrokeopacity{0.554727}%
\pgfsetdash{}{0pt}%
\pgfpathmoveto{\pgfqpoint{0.857642in}{1.525532in}}%
\pgfpathcurveto{\pgfqpoint{0.865878in}{1.525532in}}{\pgfqpoint{0.873778in}{1.528804in}}{\pgfqpoint{0.879602in}{1.534628in}}%
\pgfpathcurveto{\pgfqpoint{0.885426in}{1.540452in}}{\pgfqpoint{0.888698in}{1.548352in}}{\pgfqpoint{0.888698in}{1.556588in}}%
\pgfpathcurveto{\pgfqpoint{0.888698in}{1.564825in}}{\pgfqpoint{0.885426in}{1.572725in}}{\pgfqpoint{0.879602in}{1.578549in}}%
\pgfpathcurveto{\pgfqpoint{0.873778in}{1.584372in}}{\pgfqpoint{0.865878in}{1.587645in}}{\pgfqpoint{0.857642in}{1.587645in}}%
\pgfpathcurveto{\pgfqpoint{0.849406in}{1.587645in}}{\pgfqpoint{0.841506in}{1.584372in}}{\pgfqpoint{0.835682in}{1.578549in}}%
\pgfpathcurveto{\pgfqpoint{0.829858in}{1.572725in}}{\pgfqpoint{0.826585in}{1.564825in}}{\pgfqpoint{0.826585in}{1.556588in}}%
\pgfpathcurveto{\pgfqpoint{0.826585in}{1.548352in}}{\pgfqpoint{0.829858in}{1.540452in}}{\pgfqpoint{0.835682in}{1.534628in}}%
\pgfpathcurveto{\pgfqpoint{0.841506in}{1.528804in}}{\pgfqpoint{0.849406in}{1.525532in}}{\pgfqpoint{0.857642in}{1.525532in}}%
\pgfpathclose%
\pgfusepath{stroke,fill}%
\end{pgfscope}%
\begin{pgfscope}%
\pgfpathrectangle{\pgfqpoint{0.100000in}{0.220728in}}{\pgfqpoint{3.696000in}{3.696000in}}%
\pgfusepath{clip}%
\pgfsetbuttcap%
\pgfsetroundjoin%
\definecolor{currentfill}{rgb}{0.121569,0.466667,0.705882}%
\pgfsetfillcolor{currentfill}%
\pgfsetfillopacity{0.556256}%
\pgfsetlinewidth{1.003750pt}%
\definecolor{currentstroke}{rgb}{0.121569,0.466667,0.705882}%
\pgfsetstrokecolor{currentstroke}%
\pgfsetstrokeopacity{0.556256}%
\pgfsetdash{}{0pt}%
\pgfpathmoveto{\pgfqpoint{0.854061in}{1.518114in}}%
\pgfpathcurveto{\pgfqpoint{0.862297in}{1.518114in}}{\pgfqpoint{0.870197in}{1.521386in}}{\pgfqpoint{0.876021in}{1.527210in}}%
\pgfpathcurveto{\pgfqpoint{0.881845in}{1.533034in}}{\pgfqpoint{0.885118in}{1.540934in}}{\pgfqpoint{0.885118in}{1.549171in}}%
\pgfpathcurveto{\pgfqpoint{0.885118in}{1.557407in}}{\pgfqpoint{0.881845in}{1.565307in}}{\pgfqpoint{0.876021in}{1.571131in}}%
\pgfpathcurveto{\pgfqpoint{0.870197in}{1.576955in}}{\pgfqpoint{0.862297in}{1.580227in}}{\pgfqpoint{0.854061in}{1.580227in}}%
\pgfpathcurveto{\pgfqpoint{0.845825in}{1.580227in}}{\pgfqpoint{0.837925in}{1.576955in}}{\pgfqpoint{0.832101in}{1.571131in}}%
\pgfpathcurveto{\pgfqpoint{0.826277in}{1.565307in}}{\pgfqpoint{0.823005in}{1.557407in}}{\pgfqpoint{0.823005in}{1.549171in}}%
\pgfpathcurveto{\pgfqpoint{0.823005in}{1.540934in}}{\pgfqpoint{0.826277in}{1.533034in}}{\pgfqpoint{0.832101in}{1.527210in}}%
\pgfpathcurveto{\pgfqpoint{0.837925in}{1.521386in}}{\pgfqpoint{0.845825in}{1.518114in}}{\pgfqpoint{0.854061in}{1.518114in}}%
\pgfpathclose%
\pgfusepath{stroke,fill}%
\end{pgfscope}%
\begin{pgfscope}%
\pgfpathrectangle{\pgfqpoint{0.100000in}{0.220728in}}{\pgfqpoint{3.696000in}{3.696000in}}%
\pgfusepath{clip}%
\pgfsetbuttcap%
\pgfsetroundjoin%
\definecolor{currentfill}{rgb}{0.121569,0.466667,0.705882}%
\pgfsetfillcolor{currentfill}%
\pgfsetfillopacity{0.556921}%
\pgfsetlinewidth{1.003750pt}%
\definecolor{currentstroke}{rgb}{0.121569,0.466667,0.705882}%
\pgfsetstrokecolor{currentstroke}%
\pgfsetstrokeopacity{0.556921}%
\pgfsetdash{}{0pt}%
\pgfpathmoveto{\pgfqpoint{2.752662in}{2.966200in}}%
\pgfpathcurveto{\pgfqpoint{2.760898in}{2.966200in}}{\pgfqpoint{2.768798in}{2.969472in}}{\pgfqpoint{2.774622in}{2.975296in}}%
\pgfpathcurveto{\pgfqpoint{2.780446in}{2.981120in}}{\pgfqpoint{2.783719in}{2.989020in}}{\pgfqpoint{2.783719in}{2.997256in}}%
\pgfpathcurveto{\pgfqpoint{2.783719in}{3.005493in}}{\pgfqpoint{2.780446in}{3.013393in}}{\pgfqpoint{2.774622in}{3.019217in}}%
\pgfpathcurveto{\pgfqpoint{2.768798in}{3.025041in}}{\pgfqpoint{2.760898in}{3.028313in}}{\pgfqpoint{2.752662in}{3.028313in}}%
\pgfpathcurveto{\pgfqpoint{2.744426in}{3.028313in}}{\pgfqpoint{2.736526in}{3.025041in}}{\pgfqpoint{2.730702in}{3.019217in}}%
\pgfpathcurveto{\pgfqpoint{2.724878in}{3.013393in}}{\pgfqpoint{2.721606in}{3.005493in}}{\pgfqpoint{2.721606in}{2.997256in}}%
\pgfpathcurveto{\pgfqpoint{2.721606in}{2.989020in}}{\pgfqpoint{2.724878in}{2.981120in}}{\pgfqpoint{2.730702in}{2.975296in}}%
\pgfpathcurveto{\pgfqpoint{2.736526in}{2.969472in}}{\pgfqpoint{2.744426in}{2.966200in}}{\pgfqpoint{2.752662in}{2.966200in}}%
\pgfpathclose%
\pgfusepath{stroke,fill}%
\end{pgfscope}%
\begin{pgfscope}%
\pgfpathrectangle{\pgfqpoint{0.100000in}{0.220728in}}{\pgfqpoint{3.696000in}{3.696000in}}%
\pgfusepath{clip}%
\pgfsetbuttcap%
\pgfsetroundjoin%
\definecolor{currentfill}{rgb}{0.121569,0.466667,0.705882}%
\pgfsetfillcolor{currentfill}%
\pgfsetfillopacity{0.558789}%
\pgfsetlinewidth{1.003750pt}%
\definecolor{currentstroke}{rgb}{0.121569,0.466667,0.705882}%
\pgfsetstrokecolor{currentstroke}%
\pgfsetstrokeopacity{0.558789}%
\pgfsetdash{}{0pt}%
\pgfpathmoveto{\pgfqpoint{2.760981in}{2.963711in}}%
\pgfpathcurveto{\pgfqpoint{2.769217in}{2.963711in}}{\pgfqpoint{2.777117in}{2.966983in}}{\pgfqpoint{2.782941in}{2.972807in}}%
\pgfpathcurveto{\pgfqpoint{2.788765in}{2.978631in}}{\pgfqpoint{2.792037in}{2.986531in}}{\pgfqpoint{2.792037in}{2.994767in}}%
\pgfpathcurveto{\pgfqpoint{2.792037in}{3.003004in}}{\pgfqpoint{2.788765in}{3.010904in}}{\pgfqpoint{2.782941in}{3.016728in}}%
\pgfpathcurveto{\pgfqpoint{2.777117in}{3.022552in}}{\pgfqpoint{2.769217in}{3.025824in}}{\pgfqpoint{2.760981in}{3.025824in}}%
\pgfpathcurveto{\pgfqpoint{2.752745in}{3.025824in}}{\pgfqpoint{2.744845in}{3.022552in}}{\pgfqpoint{2.739021in}{3.016728in}}%
\pgfpathcurveto{\pgfqpoint{2.733197in}{3.010904in}}{\pgfqpoint{2.729924in}{3.003004in}}{\pgfqpoint{2.729924in}{2.994767in}}%
\pgfpathcurveto{\pgfqpoint{2.729924in}{2.986531in}}{\pgfqpoint{2.733197in}{2.978631in}}{\pgfqpoint{2.739021in}{2.972807in}}%
\pgfpathcurveto{\pgfqpoint{2.744845in}{2.966983in}}{\pgfqpoint{2.752745in}{2.963711in}}{\pgfqpoint{2.760981in}{2.963711in}}%
\pgfpathclose%
\pgfusepath{stroke,fill}%
\end{pgfscope}%
\begin{pgfscope}%
\pgfpathrectangle{\pgfqpoint{0.100000in}{0.220728in}}{\pgfqpoint{3.696000in}{3.696000in}}%
\pgfusepath{clip}%
\pgfsetbuttcap%
\pgfsetroundjoin%
\definecolor{currentfill}{rgb}{0.121569,0.466667,0.705882}%
\pgfsetfillcolor{currentfill}%
\pgfsetfillopacity{0.559037}%
\pgfsetlinewidth{1.003750pt}%
\definecolor{currentstroke}{rgb}{0.121569,0.466667,0.705882}%
\pgfsetstrokecolor{currentstroke}%
\pgfsetstrokeopacity{0.559037}%
\pgfsetdash{}{0pt}%
\pgfpathmoveto{\pgfqpoint{0.847374in}{1.504810in}}%
\pgfpathcurveto{\pgfqpoint{0.855610in}{1.504810in}}{\pgfqpoint{0.863510in}{1.508082in}}{\pgfqpoint{0.869334in}{1.513906in}}%
\pgfpathcurveto{\pgfqpoint{0.875158in}{1.519730in}}{\pgfqpoint{0.878430in}{1.527630in}}{\pgfqpoint{0.878430in}{1.535867in}}%
\pgfpathcurveto{\pgfqpoint{0.878430in}{1.544103in}}{\pgfqpoint{0.875158in}{1.552003in}}{\pgfqpoint{0.869334in}{1.557827in}}%
\pgfpathcurveto{\pgfqpoint{0.863510in}{1.563651in}}{\pgfqpoint{0.855610in}{1.566923in}}{\pgfqpoint{0.847374in}{1.566923in}}%
\pgfpathcurveto{\pgfqpoint{0.839138in}{1.566923in}}{\pgfqpoint{0.831238in}{1.563651in}}{\pgfqpoint{0.825414in}{1.557827in}}%
\pgfpathcurveto{\pgfqpoint{0.819590in}{1.552003in}}{\pgfqpoint{0.816317in}{1.544103in}}{\pgfqpoint{0.816317in}{1.535867in}}%
\pgfpathcurveto{\pgfqpoint{0.816317in}{1.527630in}}{\pgfqpoint{0.819590in}{1.519730in}}{\pgfqpoint{0.825414in}{1.513906in}}%
\pgfpathcurveto{\pgfqpoint{0.831238in}{1.508082in}}{\pgfqpoint{0.839138in}{1.504810in}}{\pgfqpoint{0.847374in}{1.504810in}}%
\pgfpathclose%
\pgfusepath{stroke,fill}%
\end{pgfscope}%
\begin{pgfscope}%
\pgfpathrectangle{\pgfqpoint{0.100000in}{0.220728in}}{\pgfqpoint{3.696000in}{3.696000in}}%
\pgfusepath{clip}%
\pgfsetbuttcap%
\pgfsetroundjoin%
\definecolor{currentfill}{rgb}{0.121569,0.466667,0.705882}%
\pgfsetfillcolor{currentfill}%
\pgfsetfillopacity{0.560102}%
\pgfsetlinewidth{1.003750pt}%
\definecolor{currentstroke}{rgb}{0.121569,0.466667,0.705882}%
\pgfsetstrokecolor{currentstroke}%
\pgfsetstrokeopacity{0.560102}%
\pgfsetdash{}{0pt}%
\pgfpathmoveto{\pgfqpoint{2.765414in}{2.963327in}}%
\pgfpathcurveto{\pgfqpoint{2.773650in}{2.963327in}}{\pgfqpoint{2.781550in}{2.966599in}}{\pgfqpoint{2.787374in}{2.972423in}}%
\pgfpathcurveto{\pgfqpoint{2.793198in}{2.978247in}}{\pgfqpoint{2.796471in}{2.986147in}}{\pgfqpoint{2.796471in}{2.994383in}}%
\pgfpathcurveto{\pgfqpoint{2.796471in}{3.002620in}}{\pgfqpoint{2.793198in}{3.010520in}}{\pgfqpoint{2.787374in}{3.016344in}}%
\pgfpathcurveto{\pgfqpoint{2.781550in}{3.022168in}}{\pgfqpoint{2.773650in}{3.025440in}}{\pgfqpoint{2.765414in}{3.025440in}}%
\pgfpathcurveto{\pgfqpoint{2.757178in}{3.025440in}}{\pgfqpoint{2.749278in}{3.022168in}}{\pgfqpoint{2.743454in}{3.016344in}}%
\pgfpathcurveto{\pgfqpoint{2.737630in}{3.010520in}}{\pgfqpoint{2.734358in}{3.002620in}}{\pgfqpoint{2.734358in}{2.994383in}}%
\pgfpathcurveto{\pgfqpoint{2.734358in}{2.986147in}}{\pgfqpoint{2.737630in}{2.978247in}}{\pgfqpoint{2.743454in}{2.972423in}}%
\pgfpathcurveto{\pgfqpoint{2.749278in}{2.966599in}}{\pgfqpoint{2.757178in}{2.963327in}}{\pgfqpoint{2.765414in}{2.963327in}}%
\pgfpathclose%
\pgfusepath{stroke,fill}%
\end{pgfscope}%
\begin{pgfscope}%
\pgfpathrectangle{\pgfqpoint{0.100000in}{0.220728in}}{\pgfqpoint{3.696000in}{3.696000in}}%
\pgfusepath{clip}%
\pgfsetbuttcap%
\pgfsetroundjoin%
\definecolor{currentfill}{rgb}{0.121569,0.466667,0.705882}%
\pgfsetfillcolor{currentfill}%
\pgfsetfillopacity{0.561604}%
\pgfsetlinewidth{1.003750pt}%
\definecolor{currentstroke}{rgb}{0.121569,0.466667,0.705882}%
\pgfsetstrokecolor{currentstroke}%
\pgfsetstrokeopacity{0.561604}%
\pgfsetdash{}{0pt}%
\pgfpathmoveto{\pgfqpoint{2.771947in}{2.962542in}}%
\pgfpathcurveto{\pgfqpoint{2.780183in}{2.962542in}}{\pgfqpoint{2.788083in}{2.965814in}}{\pgfqpoint{2.793907in}{2.971638in}}%
\pgfpathcurveto{\pgfqpoint{2.799731in}{2.977462in}}{\pgfqpoint{2.803003in}{2.985362in}}{\pgfqpoint{2.803003in}{2.993599in}}%
\pgfpathcurveto{\pgfqpoint{2.803003in}{3.001835in}}{\pgfqpoint{2.799731in}{3.009735in}}{\pgfqpoint{2.793907in}{3.015559in}}%
\pgfpathcurveto{\pgfqpoint{2.788083in}{3.021383in}}{\pgfqpoint{2.780183in}{3.024655in}}{\pgfqpoint{2.771947in}{3.024655in}}%
\pgfpathcurveto{\pgfqpoint{2.763710in}{3.024655in}}{\pgfqpoint{2.755810in}{3.021383in}}{\pgfqpoint{2.749986in}{3.015559in}}%
\pgfpathcurveto{\pgfqpoint{2.744162in}{3.009735in}}{\pgfqpoint{2.740890in}{3.001835in}}{\pgfqpoint{2.740890in}{2.993599in}}%
\pgfpathcurveto{\pgfqpoint{2.740890in}{2.985362in}}{\pgfqpoint{2.744162in}{2.977462in}}{\pgfqpoint{2.749986in}{2.971638in}}%
\pgfpathcurveto{\pgfqpoint{2.755810in}{2.965814in}}{\pgfqpoint{2.763710in}{2.962542in}}{\pgfqpoint{2.771947in}{2.962542in}}%
\pgfpathclose%
\pgfusepath{stroke,fill}%
\end{pgfscope}%
\begin{pgfscope}%
\pgfpathrectangle{\pgfqpoint{0.100000in}{0.220728in}}{\pgfqpoint{3.696000in}{3.696000in}}%
\pgfusepath{clip}%
\pgfsetbuttcap%
\pgfsetroundjoin%
\definecolor{currentfill}{rgb}{0.121569,0.466667,0.705882}%
\pgfsetfillcolor{currentfill}%
\pgfsetfillopacity{0.563168}%
\pgfsetlinewidth{1.003750pt}%
\definecolor{currentstroke}{rgb}{0.121569,0.466667,0.705882}%
\pgfsetstrokecolor{currentstroke}%
\pgfsetstrokeopacity{0.563168}%
\pgfsetdash{}{0pt}%
\pgfpathmoveto{\pgfqpoint{0.829748in}{1.483154in}}%
\pgfpathcurveto{\pgfqpoint{0.837985in}{1.483154in}}{\pgfqpoint{0.845885in}{1.486426in}}{\pgfqpoint{0.851709in}{1.492250in}}%
\pgfpathcurveto{\pgfqpoint{0.857533in}{1.498074in}}{\pgfqpoint{0.860805in}{1.505974in}}{\pgfqpoint{0.860805in}{1.514210in}}%
\pgfpathcurveto{\pgfqpoint{0.860805in}{1.522447in}}{\pgfqpoint{0.857533in}{1.530347in}}{\pgfqpoint{0.851709in}{1.536171in}}%
\pgfpathcurveto{\pgfqpoint{0.845885in}{1.541994in}}{\pgfqpoint{0.837985in}{1.545267in}}{\pgfqpoint{0.829748in}{1.545267in}}%
\pgfpathcurveto{\pgfqpoint{0.821512in}{1.545267in}}{\pgfqpoint{0.813612in}{1.541994in}}{\pgfqpoint{0.807788in}{1.536171in}}%
\pgfpathcurveto{\pgfqpoint{0.801964in}{1.530347in}}{\pgfqpoint{0.798692in}{1.522447in}}{\pgfqpoint{0.798692in}{1.514210in}}%
\pgfpathcurveto{\pgfqpoint{0.798692in}{1.505974in}}{\pgfqpoint{0.801964in}{1.498074in}}{\pgfqpoint{0.807788in}{1.492250in}}%
\pgfpathcurveto{\pgfqpoint{0.813612in}{1.486426in}}{\pgfqpoint{0.821512in}{1.483154in}}{\pgfqpoint{0.829748in}{1.483154in}}%
\pgfpathclose%
\pgfusepath{stroke,fill}%
\end{pgfscope}%
\begin{pgfscope}%
\pgfpathrectangle{\pgfqpoint{0.100000in}{0.220728in}}{\pgfqpoint{3.696000in}{3.696000in}}%
\pgfusepath{clip}%
\pgfsetbuttcap%
\pgfsetroundjoin%
\definecolor{currentfill}{rgb}{0.121569,0.466667,0.705882}%
\pgfsetfillcolor{currentfill}%
\pgfsetfillopacity{0.563993}%
\pgfsetlinewidth{1.003750pt}%
\definecolor{currentstroke}{rgb}{0.121569,0.466667,0.705882}%
\pgfsetstrokecolor{currentstroke}%
\pgfsetstrokeopacity{0.563993}%
\pgfsetdash{}{0pt}%
\pgfpathmoveto{\pgfqpoint{2.780013in}{2.962110in}}%
\pgfpathcurveto{\pgfqpoint{2.788249in}{2.962110in}}{\pgfqpoint{2.796149in}{2.965382in}}{\pgfqpoint{2.801973in}{2.971206in}}%
\pgfpathcurveto{\pgfqpoint{2.807797in}{2.977030in}}{\pgfqpoint{2.811069in}{2.984930in}}{\pgfqpoint{2.811069in}{2.993167in}}%
\pgfpathcurveto{\pgfqpoint{2.811069in}{3.001403in}}{\pgfqpoint{2.807797in}{3.009303in}}{\pgfqpoint{2.801973in}{3.015127in}}%
\pgfpathcurveto{\pgfqpoint{2.796149in}{3.020951in}}{\pgfqpoint{2.788249in}{3.024223in}}{\pgfqpoint{2.780013in}{3.024223in}}%
\pgfpathcurveto{\pgfqpoint{2.771776in}{3.024223in}}{\pgfqpoint{2.763876in}{3.020951in}}{\pgfqpoint{2.758052in}{3.015127in}}%
\pgfpathcurveto{\pgfqpoint{2.752228in}{3.009303in}}{\pgfqpoint{2.748956in}{3.001403in}}{\pgfqpoint{2.748956in}{2.993167in}}%
\pgfpathcurveto{\pgfqpoint{2.748956in}{2.984930in}}{\pgfqpoint{2.752228in}{2.977030in}}{\pgfqpoint{2.758052in}{2.971206in}}%
\pgfpathcurveto{\pgfqpoint{2.763876in}{2.965382in}}{\pgfqpoint{2.771776in}{2.962110in}}{\pgfqpoint{2.780013in}{2.962110in}}%
\pgfpathclose%
\pgfusepath{stroke,fill}%
\end{pgfscope}%
\begin{pgfscope}%
\pgfpathrectangle{\pgfqpoint{0.100000in}{0.220728in}}{\pgfqpoint{3.696000in}{3.696000in}}%
\pgfusepath{clip}%
\pgfsetbuttcap%
\pgfsetroundjoin%
\definecolor{currentfill}{rgb}{0.121569,0.466667,0.705882}%
\pgfsetfillcolor{currentfill}%
\pgfsetfillopacity{0.566224}%
\pgfsetlinewidth{1.003750pt}%
\definecolor{currentstroke}{rgb}{0.121569,0.466667,0.705882}%
\pgfsetstrokecolor{currentstroke}%
\pgfsetstrokeopacity{0.566224}%
\pgfsetdash{}{0pt}%
\pgfpathmoveto{\pgfqpoint{2.793267in}{2.960216in}}%
\pgfpathcurveto{\pgfqpoint{2.801503in}{2.960216in}}{\pgfqpoint{2.809403in}{2.963488in}}{\pgfqpoint{2.815227in}{2.969312in}}%
\pgfpathcurveto{\pgfqpoint{2.821051in}{2.975136in}}{\pgfqpoint{2.824323in}{2.983036in}}{\pgfqpoint{2.824323in}{2.991273in}}%
\pgfpathcurveto{\pgfqpoint{2.824323in}{2.999509in}}{\pgfqpoint{2.821051in}{3.007409in}}{\pgfqpoint{2.815227in}{3.013233in}}%
\pgfpathcurveto{\pgfqpoint{2.809403in}{3.019057in}}{\pgfqpoint{2.801503in}{3.022329in}}{\pgfqpoint{2.793267in}{3.022329in}}%
\pgfpathcurveto{\pgfqpoint{2.785030in}{3.022329in}}{\pgfqpoint{2.777130in}{3.019057in}}{\pgfqpoint{2.771306in}{3.013233in}}%
\pgfpathcurveto{\pgfqpoint{2.765483in}{3.007409in}}{\pgfqpoint{2.762210in}{2.999509in}}{\pgfqpoint{2.762210in}{2.991273in}}%
\pgfpathcurveto{\pgfqpoint{2.762210in}{2.983036in}}{\pgfqpoint{2.765483in}{2.975136in}}{\pgfqpoint{2.771306in}{2.969312in}}%
\pgfpathcurveto{\pgfqpoint{2.777130in}{2.963488in}}{\pgfqpoint{2.785030in}{2.960216in}}{\pgfqpoint{2.793267in}{2.960216in}}%
\pgfpathclose%
\pgfusepath{stroke,fill}%
\end{pgfscope}%
\begin{pgfscope}%
\pgfpathrectangle{\pgfqpoint{0.100000in}{0.220728in}}{\pgfqpoint{3.696000in}{3.696000in}}%
\pgfusepath{clip}%
\pgfsetbuttcap%
\pgfsetroundjoin%
\definecolor{currentfill}{rgb}{0.121569,0.466667,0.705882}%
\pgfsetfillcolor{currentfill}%
\pgfsetfillopacity{0.568190}%
\pgfsetlinewidth{1.003750pt}%
\definecolor{currentstroke}{rgb}{0.121569,0.466667,0.705882}%
\pgfsetstrokecolor{currentstroke}%
\pgfsetstrokeopacity{0.568190}%
\pgfsetdash{}{0pt}%
\pgfpathmoveto{\pgfqpoint{2.799099in}{2.958628in}}%
\pgfpathcurveto{\pgfqpoint{2.807335in}{2.958628in}}{\pgfqpoint{2.815235in}{2.961901in}}{\pgfqpoint{2.821059in}{2.967724in}}%
\pgfpathcurveto{\pgfqpoint{2.826883in}{2.973548in}}{\pgfqpoint{2.830155in}{2.981448in}}{\pgfqpoint{2.830155in}{2.989685in}}%
\pgfpathcurveto{\pgfqpoint{2.830155in}{2.997921in}}{\pgfqpoint{2.826883in}{3.005821in}}{\pgfqpoint{2.821059in}{3.011645in}}%
\pgfpathcurveto{\pgfqpoint{2.815235in}{3.017469in}}{\pgfqpoint{2.807335in}{3.020741in}}{\pgfqpoint{2.799099in}{3.020741in}}%
\pgfpathcurveto{\pgfqpoint{2.790863in}{3.020741in}}{\pgfqpoint{2.782962in}{3.017469in}}{\pgfqpoint{2.777139in}{3.011645in}}%
\pgfpathcurveto{\pgfqpoint{2.771315in}{3.005821in}}{\pgfqpoint{2.768042in}{2.997921in}}{\pgfqpoint{2.768042in}{2.989685in}}%
\pgfpathcurveto{\pgfqpoint{2.768042in}{2.981448in}}{\pgfqpoint{2.771315in}{2.973548in}}{\pgfqpoint{2.777139in}{2.967724in}}%
\pgfpathcurveto{\pgfqpoint{2.782962in}{2.961901in}}{\pgfqpoint{2.790863in}{2.958628in}}{\pgfqpoint{2.799099in}{2.958628in}}%
\pgfpathclose%
\pgfusepath{stroke,fill}%
\end{pgfscope}%
\begin{pgfscope}%
\pgfpathrectangle{\pgfqpoint{0.100000in}{0.220728in}}{\pgfqpoint{3.696000in}{3.696000in}}%
\pgfusepath{clip}%
\pgfsetbuttcap%
\pgfsetroundjoin%
\definecolor{currentfill}{rgb}{0.121569,0.466667,0.705882}%
\pgfsetfillcolor{currentfill}%
\pgfsetfillopacity{0.569044}%
\pgfsetlinewidth{1.003750pt}%
\definecolor{currentstroke}{rgb}{0.121569,0.466667,0.705882}%
\pgfsetstrokecolor{currentstroke}%
\pgfsetstrokeopacity{0.569044}%
\pgfsetdash{}{0pt}%
\pgfpathmoveto{\pgfqpoint{2.802894in}{2.958136in}}%
\pgfpathcurveto{\pgfqpoint{2.811130in}{2.958136in}}{\pgfqpoint{2.819030in}{2.961409in}}{\pgfqpoint{2.824854in}{2.967233in}}%
\pgfpathcurveto{\pgfqpoint{2.830678in}{2.973056in}}{\pgfqpoint{2.833950in}{2.980957in}}{\pgfqpoint{2.833950in}{2.989193in}}%
\pgfpathcurveto{\pgfqpoint{2.833950in}{2.997429in}}{\pgfqpoint{2.830678in}{3.005329in}}{\pgfqpoint{2.824854in}{3.011153in}}%
\pgfpathcurveto{\pgfqpoint{2.819030in}{3.016977in}}{\pgfqpoint{2.811130in}{3.020249in}}{\pgfqpoint{2.802894in}{3.020249in}}%
\pgfpathcurveto{\pgfqpoint{2.794657in}{3.020249in}}{\pgfqpoint{2.786757in}{3.016977in}}{\pgfqpoint{2.780933in}{3.011153in}}%
\pgfpathcurveto{\pgfqpoint{2.775109in}{3.005329in}}{\pgfqpoint{2.771837in}{2.997429in}}{\pgfqpoint{2.771837in}{2.989193in}}%
\pgfpathcurveto{\pgfqpoint{2.771837in}{2.980957in}}{\pgfqpoint{2.775109in}{2.973056in}}{\pgfqpoint{2.780933in}{2.967233in}}%
\pgfpathcurveto{\pgfqpoint{2.786757in}{2.961409in}}{\pgfqpoint{2.794657in}{2.958136in}}{\pgfqpoint{2.802894in}{2.958136in}}%
\pgfpathclose%
\pgfusepath{stroke,fill}%
\end{pgfscope}%
\begin{pgfscope}%
\pgfpathrectangle{\pgfqpoint{0.100000in}{0.220728in}}{\pgfqpoint{3.696000in}{3.696000in}}%
\pgfusepath{clip}%
\pgfsetbuttcap%
\pgfsetroundjoin%
\definecolor{currentfill}{rgb}{0.121569,0.466667,0.705882}%
\pgfsetfillcolor{currentfill}%
\pgfsetfillopacity{0.569586}%
\pgfsetlinewidth{1.003750pt}%
\definecolor{currentstroke}{rgb}{0.121569,0.466667,0.705882}%
\pgfsetstrokecolor{currentstroke}%
\pgfsetstrokeopacity{0.569586}%
\pgfsetdash{}{0pt}%
\pgfpathmoveto{\pgfqpoint{2.804876in}{2.957903in}}%
\pgfpathcurveto{\pgfqpoint{2.813112in}{2.957903in}}{\pgfqpoint{2.821012in}{2.961176in}}{\pgfqpoint{2.826836in}{2.967000in}}%
\pgfpathcurveto{\pgfqpoint{2.832660in}{2.972823in}}{\pgfqpoint{2.835932in}{2.980724in}}{\pgfqpoint{2.835932in}{2.988960in}}%
\pgfpathcurveto{\pgfqpoint{2.835932in}{2.997196in}}{\pgfqpoint{2.832660in}{3.005096in}}{\pgfqpoint{2.826836in}{3.010920in}}%
\pgfpathcurveto{\pgfqpoint{2.821012in}{3.016744in}}{\pgfqpoint{2.813112in}{3.020016in}}{\pgfqpoint{2.804876in}{3.020016in}}%
\pgfpathcurveto{\pgfqpoint{2.796640in}{3.020016in}}{\pgfqpoint{2.788740in}{3.016744in}}{\pgfqpoint{2.782916in}{3.010920in}}%
\pgfpathcurveto{\pgfqpoint{2.777092in}{3.005096in}}{\pgfqpoint{2.773819in}{2.997196in}}{\pgfqpoint{2.773819in}{2.988960in}}%
\pgfpathcurveto{\pgfqpoint{2.773819in}{2.980724in}}{\pgfqpoint{2.777092in}{2.972823in}}{\pgfqpoint{2.782916in}{2.967000in}}%
\pgfpathcurveto{\pgfqpoint{2.788740in}{2.961176in}}{\pgfqpoint{2.796640in}{2.957903in}}{\pgfqpoint{2.804876in}{2.957903in}}%
\pgfpathclose%
\pgfusepath{stroke,fill}%
\end{pgfscope}%
\begin{pgfscope}%
\pgfpathrectangle{\pgfqpoint{0.100000in}{0.220728in}}{\pgfqpoint{3.696000in}{3.696000in}}%
\pgfusepath{clip}%
\pgfsetbuttcap%
\pgfsetroundjoin%
\definecolor{currentfill}{rgb}{0.121569,0.466667,0.705882}%
\pgfsetfillcolor{currentfill}%
\pgfsetfillopacity{0.571399}%
\pgfsetlinewidth{1.003750pt}%
\definecolor{currentstroke}{rgb}{0.121569,0.466667,0.705882}%
\pgfsetstrokecolor{currentstroke}%
\pgfsetstrokeopacity{0.571399}%
\pgfsetdash{}{0pt}%
\pgfpathmoveto{\pgfqpoint{2.811542in}{2.956823in}}%
\pgfpathcurveto{\pgfqpoint{2.819779in}{2.956823in}}{\pgfqpoint{2.827679in}{2.960095in}}{\pgfqpoint{2.833503in}{2.965919in}}%
\pgfpathcurveto{\pgfqpoint{2.839326in}{2.971743in}}{\pgfqpoint{2.842599in}{2.979643in}}{\pgfqpoint{2.842599in}{2.987879in}}%
\pgfpathcurveto{\pgfqpoint{2.842599in}{2.996116in}}{\pgfqpoint{2.839326in}{3.004016in}}{\pgfqpoint{2.833503in}{3.009840in}}%
\pgfpathcurveto{\pgfqpoint{2.827679in}{3.015664in}}{\pgfqpoint{2.819779in}{3.018936in}}{\pgfqpoint{2.811542in}{3.018936in}}%
\pgfpathcurveto{\pgfqpoint{2.803306in}{3.018936in}}{\pgfqpoint{2.795406in}{3.015664in}}{\pgfqpoint{2.789582in}{3.009840in}}%
\pgfpathcurveto{\pgfqpoint{2.783758in}{3.004016in}}{\pgfqpoint{2.780486in}{2.996116in}}{\pgfqpoint{2.780486in}{2.987879in}}%
\pgfpathcurveto{\pgfqpoint{2.780486in}{2.979643in}}{\pgfqpoint{2.783758in}{2.971743in}}{\pgfqpoint{2.789582in}{2.965919in}}%
\pgfpathcurveto{\pgfqpoint{2.795406in}{2.960095in}}{\pgfqpoint{2.803306in}{2.956823in}}{\pgfqpoint{2.811542in}{2.956823in}}%
\pgfpathclose%
\pgfusepath{stroke,fill}%
\end{pgfscope}%
\begin{pgfscope}%
\pgfpathrectangle{\pgfqpoint{0.100000in}{0.220728in}}{\pgfqpoint{3.696000in}{3.696000in}}%
\pgfusepath{clip}%
\pgfsetbuttcap%
\pgfsetroundjoin%
\definecolor{currentfill}{rgb}{0.121569,0.466667,0.705882}%
\pgfsetfillcolor{currentfill}%
\pgfsetfillopacity{0.572891}%
\pgfsetlinewidth{1.003750pt}%
\definecolor{currentstroke}{rgb}{0.121569,0.466667,0.705882}%
\pgfsetstrokecolor{currentstroke}%
\pgfsetstrokeopacity{0.572891}%
\pgfsetdash{}{0pt}%
\pgfpathmoveto{\pgfqpoint{0.810121in}{1.439347in}}%
\pgfpathcurveto{\pgfqpoint{0.818357in}{1.439347in}}{\pgfqpoint{0.826257in}{1.442620in}}{\pgfqpoint{0.832081in}{1.448444in}}%
\pgfpathcurveto{\pgfqpoint{0.837905in}{1.454268in}}{\pgfqpoint{0.841177in}{1.462168in}}{\pgfqpoint{0.841177in}{1.470404in}}%
\pgfpathcurveto{\pgfqpoint{0.841177in}{1.478640in}}{\pgfqpoint{0.837905in}{1.486540in}}{\pgfqpoint{0.832081in}{1.492364in}}%
\pgfpathcurveto{\pgfqpoint{0.826257in}{1.498188in}}{\pgfqpoint{0.818357in}{1.501460in}}{\pgfqpoint{0.810121in}{1.501460in}}%
\pgfpathcurveto{\pgfqpoint{0.801884in}{1.501460in}}{\pgfqpoint{0.793984in}{1.498188in}}{\pgfqpoint{0.788160in}{1.492364in}}%
\pgfpathcurveto{\pgfqpoint{0.782336in}{1.486540in}}{\pgfqpoint{0.779064in}{1.478640in}}{\pgfqpoint{0.779064in}{1.470404in}}%
\pgfpathcurveto{\pgfqpoint{0.779064in}{1.462168in}}{\pgfqpoint{0.782336in}{1.454268in}}{\pgfqpoint{0.788160in}{1.448444in}}%
\pgfpathcurveto{\pgfqpoint{0.793984in}{1.442620in}}{\pgfqpoint{0.801884in}{1.439347in}}{\pgfqpoint{0.810121in}{1.439347in}}%
\pgfpathclose%
\pgfusepath{stroke,fill}%
\end{pgfscope}%
\begin{pgfscope}%
\pgfpathrectangle{\pgfqpoint{0.100000in}{0.220728in}}{\pgfqpoint{3.696000in}{3.696000in}}%
\pgfusepath{clip}%
\pgfsetbuttcap%
\pgfsetroundjoin%
\definecolor{currentfill}{rgb}{0.121569,0.466667,0.705882}%
\pgfsetfillcolor{currentfill}%
\pgfsetfillopacity{0.573785}%
\pgfsetlinewidth{1.003750pt}%
\definecolor{currentstroke}{rgb}{0.121569,0.466667,0.705882}%
\pgfsetstrokecolor{currentstroke}%
\pgfsetstrokeopacity{0.573785}%
\pgfsetdash{}{0pt}%
\pgfpathmoveto{\pgfqpoint{2.825554in}{2.954728in}}%
\pgfpathcurveto{\pgfqpoint{2.833790in}{2.954728in}}{\pgfqpoint{2.841690in}{2.958000in}}{\pgfqpoint{2.847514in}{2.963824in}}%
\pgfpathcurveto{\pgfqpoint{2.853338in}{2.969648in}}{\pgfqpoint{2.856610in}{2.977548in}}{\pgfqpoint{2.856610in}{2.985785in}}%
\pgfpathcurveto{\pgfqpoint{2.856610in}{2.994021in}}{\pgfqpoint{2.853338in}{3.001921in}}{\pgfqpoint{2.847514in}{3.007745in}}%
\pgfpathcurveto{\pgfqpoint{2.841690in}{3.013569in}}{\pgfqpoint{2.833790in}{3.016841in}}{\pgfqpoint{2.825554in}{3.016841in}}%
\pgfpathcurveto{\pgfqpoint{2.817317in}{3.016841in}}{\pgfqpoint{2.809417in}{3.013569in}}{\pgfqpoint{2.803593in}{3.007745in}}%
\pgfpathcurveto{\pgfqpoint{2.797770in}{3.001921in}}{\pgfqpoint{2.794497in}{2.994021in}}{\pgfqpoint{2.794497in}{2.985785in}}%
\pgfpathcurveto{\pgfqpoint{2.794497in}{2.977548in}}{\pgfqpoint{2.797770in}{2.969648in}}{\pgfqpoint{2.803593in}{2.963824in}}%
\pgfpathcurveto{\pgfqpoint{2.809417in}{2.958000in}}{\pgfqpoint{2.817317in}{2.954728in}}{\pgfqpoint{2.825554in}{2.954728in}}%
\pgfpathclose%
\pgfusepath{stroke,fill}%
\end{pgfscope}%
\begin{pgfscope}%
\pgfpathrectangle{\pgfqpoint{0.100000in}{0.220728in}}{\pgfqpoint{3.696000in}{3.696000in}}%
\pgfusepath{clip}%
\pgfsetbuttcap%
\pgfsetroundjoin%
\definecolor{currentfill}{rgb}{0.121569,0.466667,0.705882}%
\pgfsetfillcolor{currentfill}%
\pgfsetfillopacity{0.579053}%
\pgfsetlinewidth{1.003750pt}%
\definecolor{currentstroke}{rgb}{0.121569,0.466667,0.705882}%
\pgfsetstrokecolor{currentstroke}%
\pgfsetstrokeopacity{0.579053}%
\pgfsetdash{}{0pt}%
\pgfpathmoveto{\pgfqpoint{2.844434in}{2.952950in}}%
\pgfpathcurveto{\pgfqpoint{2.852671in}{2.952950in}}{\pgfqpoint{2.860571in}{2.956222in}}{\pgfqpoint{2.866395in}{2.962046in}}%
\pgfpathcurveto{\pgfqpoint{2.872218in}{2.967870in}}{\pgfqpoint{2.875491in}{2.975770in}}{\pgfqpoint{2.875491in}{2.984007in}}%
\pgfpathcurveto{\pgfqpoint{2.875491in}{2.992243in}}{\pgfqpoint{2.872218in}{3.000143in}}{\pgfqpoint{2.866395in}{3.005967in}}%
\pgfpathcurveto{\pgfqpoint{2.860571in}{3.011791in}}{\pgfqpoint{2.852671in}{3.015063in}}{\pgfqpoint{2.844434in}{3.015063in}}%
\pgfpathcurveto{\pgfqpoint{2.836198in}{3.015063in}}{\pgfqpoint{2.828298in}{3.011791in}}{\pgfqpoint{2.822474in}{3.005967in}}%
\pgfpathcurveto{\pgfqpoint{2.816650in}{3.000143in}}{\pgfqpoint{2.813378in}{2.992243in}}{\pgfqpoint{2.813378in}{2.984007in}}%
\pgfpathcurveto{\pgfqpoint{2.813378in}{2.975770in}}{\pgfqpoint{2.816650in}{2.967870in}}{\pgfqpoint{2.822474in}{2.962046in}}%
\pgfpathcurveto{\pgfqpoint{2.828298in}{2.956222in}}{\pgfqpoint{2.836198in}{2.952950in}}{\pgfqpoint{2.844434in}{2.952950in}}%
\pgfpathclose%
\pgfusepath{stroke,fill}%
\end{pgfscope}%
\begin{pgfscope}%
\pgfpathrectangle{\pgfqpoint{0.100000in}{0.220728in}}{\pgfqpoint{3.696000in}{3.696000in}}%
\pgfusepath{clip}%
\pgfsetbuttcap%
\pgfsetroundjoin%
\definecolor{currentfill}{rgb}{0.121569,0.466667,0.705882}%
\pgfsetfillcolor{currentfill}%
\pgfsetfillopacity{0.579448}%
\pgfsetlinewidth{1.003750pt}%
\definecolor{currentstroke}{rgb}{0.121569,0.466667,0.705882}%
\pgfsetstrokecolor{currentstroke}%
\pgfsetstrokeopacity{0.579448}%
\pgfsetdash{}{0pt}%
\pgfpathmoveto{\pgfqpoint{0.780815in}{1.401739in}}%
\pgfpathcurveto{\pgfqpoint{0.789051in}{1.401739in}}{\pgfqpoint{0.796951in}{1.405011in}}{\pgfqpoint{0.802775in}{1.410835in}}%
\pgfpathcurveto{\pgfqpoint{0.808599in}{1.416659in}}{\pgfqpoint{0.811871in}{1.424559in}}{\pgfqpoint{0.811871in}{1.432796in}}%
\pgfpathcurveto{\pgfqpoint{0.811871in}{1.441032in}}{\pgfqpoint{0.808599in}{1.448932in}}{\pgfqpoint{0.802775in}{1.454756in}}%
\pgfpathcurveto{\pgfqpoint{0.796951in}{1.460580in}}{\pgfqpoint{0.789051in}{1.463852in}}{\pgfqpoint{0.780815in}{1.463852in}}%
\pgfpathcurveto{\pgfqpoint{0.772578in}{1.463852in}}{\pgfqpoint{0.764678in}{1.460580in}}{\pgfqpoint{0.758854in}{1.454756in}}%
\pgfpathcurveto{\pgfqpoint{0.753030in}{1.448932in}}{\pgfqpoint{0.749758in}{1.441032in}}{\pgfqpoint{0.749758in}{1.432796in}}%
\pgfpathcurveto{\pgfqpoint{0.749758in}{1.424559in}}{\pgfqpoint{0.753030in}{1.416659in}}{\pgfqpoint{0.758854in}{1.410835in}}%
\pgfpathcurveto{\pgfqpoint{0.764678in}{1.405011in}}{\pgfqpoint{0.772578in}{1.401739in}}{\pgfqpoint{0.780815in}{1.401739in}}%
\pgfpathclose%
\pgfusepath{stroke,fill}%
\end{pgfscope}%
\begin{pgfscope}%
\pgfpathrectangle{\pgfqpoint{0.100000in}{0.220728in}}{\pgfqpoint{3.696000in}{3.696000in}}%
\pgfusepath{clip}%
\pgfsetbuttcap%
\pgfsetroundjoin%
\definecolor{currentfill}{rgb}{0.121569,0.466667,0.705882}%
\pgfsetfillcolor{currentfill}%
\pgfsetfillopacity{0.583062}%
\pgfsetlinewidth{1.003750pt}%
\definecolor{currentstroke}{rgb}{0.121569,0.466667,0.705882}%
\pgfsetstrokecolor{currentstroke}%
\pgfsetstrokeopacity{0.583062}%
\pgfsetdash{}{0pt}%
\pgfpathmoveto{\pgfqpoint{2.868918in}{2.949765in}}%
\pgfpathcurveto{\pgfqpoint{2.877154in}{2.949765in}}{\pgfqpoint{2.885054in}{2.953037in}}{\pgfqpoint{2.890878in}{2.958861in}}%
\pgfpathcurveto{\pgfqpoint{2.896702in}{2.964685in}}{\pgfqpoint{2.899974in}{2.972585in}}{\pgfqpoint{2.899974in}{2.980821in}}%
\pgfpathcurveto{\pgfqpoint{2.899974in}{2.989058in}}{\pgfqpoint{2.896702in}{2.996958in}}{\pgfqpoint{2.890878in}{3.002782in}}%
\pgfpathcurveto{\pgfqpoint{2.885054in}{3.008605in}}{\pgfqpoint{2.877154in}{3.011878in}}{\pgfqpoint{2.868918in}{3.011878in}}%
\pgfpathcurveto{\pgfqpoint{2.860682in}{3.011878in}}{\pgfqpoint{2.852781in}{3.008605in}}{\pgfqpoint{2.846958in}{3.002782in}}%
\pgfpathcurveto{\pgfqpoint{2.841134in}{2.996958in}}{\pgfqpoint{2.837861in}{2.989058in}}{\pgfqpoint{2.837861in}{2.980821in}}%
\pgfpathcurveto{\pgfqpoint{2.837861in}{2.972585in}}{\pgfqpoint{2.841134in}{2.964685in}}{\pgfqpoint{2.846958in}{2.958861in}}%
\pgfpathcurveto{\pgfqpoint{2.852781in}{2.953037in}}{\pgfqpoint{2.860682in}{2.949765in}}{\pgfqpoint{2.868918in}{2.949765in}}%
\pgfpathclose%
\pgfusepath{stroke,fill}%
\end{pgfscope}%
\begin{pgfscope}%
\pgfpathrectangle{\pgfqpoint{0.100000in}{0.220728in}}{\pgfqpoint{3.696000in}{3.696000in}}%
\pgfusepath{clip}%
\pgfsetbuttcap%
\pgfsetroundjoin%
\definecolor{currentfill}{rgb}{0.121569,0.466667,0.705882}%
\pgfsetfillcolor{currentfill}%
\pgfsetfillopacity{0.586465}%
\pgfsetlinewidth{1.003750pt}%
\definecolor{currentstroke}{rgb}{0.121569,0.466667,0.705882}%
\pgfsetstrokecolor{currentstroke}%
\pgfsetstrokeopacity{0.586465}%
\pgfsetdash{}{0pt}%
\pgfpathmoveto{\pgfqpoint{0.770520in}{1.369896in}}%
\pgfpathcurveto{\pgfqpoint{0.778756in}{1.369896in}}{\pgfqpoint{0.786656in}{1.373168in}}{\pgfqpoint{0.792480in}{1.378992in}}%
\pgfpathcurveto{\pgfqpoint{0.798304in}{1.384816in}}{\pgfqpoint{0.801577in}{1.392716in}}{\pgfqpoint{0.801577in}{1.400952in}}%
\pgfpathcurveto{\pgfqpoint{0.801577in}{1.409189in}}{\pgfqpoint{0.798304in}{1.417089in}}{\pgfqpoint{0.792480in}{1.422913in}}%
\pgfpathcurveto{\pgfqpoint{0.786656in}{1.428737in}}{\pgfqpoint{0.778756in}{1.432009in}}{\pgfqpoint{0.770520in}{1.432009in}}%
\pgfpathcurveto{\pgfqpoint{0.762284in}{1.432009in}}{\pgfqpoint{0.754384in}{1.428737in}}{\pgfqpoint{0.748560in}{1.422913in}}%
\pgfpathcurveto{\pgfqpoint{0.742736in}{1.417089in}}{\pgfqpoint{0.739464in}{1.409189in}}{\pgfqpoint{0.739464in}{1.400952in}}%
\pgfpathcurveto{\pgfqpoint{0.739464in}{1.392716in}}{\pgfqpoint{0.742736in}{1.384816in}}{\pgfqpoint{0.748560in}{1.378992in}}%
\pgfpathcurveto{\pgfqpoint{0.754384in}{1.373168in}}{\pgfqpoint{0.762284in}{1.369896in}}{\pgfqpoint{0.770520in}{1.369896in}}%
\pgfpathclose%
\pgfusepath{stroke,fill}%
\end{pgfscope}%
\begin{pgfscope}%
\pgfpathrectangle{\pgfqpoint{0.100000in}{0.220728in}}{\pgfqpoint{3.696000in}{3.696000in}}%
\pgfusepath{clip}%
\pgfsetbuttcap%
\pgfsetroundjoin%
\definecolor{currentfill}{rgb}{0.121569,0.466667,0.705882}%
\pgfsetfillcolor{currentfill}%
\pgfsetfillopacity{0.586895}%
\pgfsetlinewidth{1.003750pt}%
\definecolor{currentstroke}{rgb}{0.121569,0.466667,0.705882}%
\pgfsetstrokecolor{currentstroke}%
\pgfsetstrokeopacity{0.586895}%
\pgfsetdash{}{0pt}%
\pgfpathmoveto{\pgfqpoint{2.879509in}{2.947649in}}%
\pgfpathcurveto{\pgfqpoint{2.887745in}{2.947649in}}{\pgfqpoint{2.895645in}{2.950921in}}{\pgfqpoint{2.901469in}{2.956745in}}%
\pgfpathcurveto{\pgfqpoint{2.907293in}{2.962569in}}{\pgfqpoint{2.910565in}{2.970469in}}{\pgfqpoint{2.910565in}{2.978705in}}%
\pgfpathcurveto{\pgfqpoint{2.910565in}{2.986942in}}{\pgfqpoint{2.907293in}{2.994842in}}{\pgfqpoint{2.901469in}{3.000666in}}%
\pgfpathcurveto{\pgfqpoint{2.895645in}{3.006490in}}{\pgfqpoint{2.887745in}{3.009762in}}{\pgfqpoint{2.879509in}{3.009762in}}%
\pgfpathcurveto{\pgfqpoint{2.871272in}{3.009762in}}{\pgfqpoint{2.863372in}{3.006490in}}{\pgfqpoint{2.857548in}{3.000666in}}%
\pgfpathcurveto{\pgfqpoint{2.851724in}{2.994842in}}{\pgfqpoint{2.848452in}{2.986942in}}{\pgfqpoint{2.848452in}{2.978705in}}%
\pgfpathcurveto{\pgfqpoint{2.848452in}{2.970469in}}{\pgfqpoint{2.851724in}{2.962569in}}{\pgfqpoint{2.857548in}{2.956745in}}%
\pgfpathcurveto{\pgfqpoint{2.863372in}{2.950921in}}{\pgfqpoint{2.871272in}{2.947649in}}{\pgfqpoint{2.879509in}{2.947649in}}%
\pgfpathclose%
\pgfusepath{stroke,fill}%
\end{pgfscope}%
\begin{pgfscope}%
\pgfpathrectangle{\pgfqpoint{0.100000in}{0.220728in}}{\pgfqpoint{3.696000in}{3.696000in}}%
\pgfusepath{clip}%
\pgfsetbuttcap%
\pgfsetroundjoin%
\definecolor{currentfill}{rgb}{0.121569,0.466667,0.705882}%
\pgfsetfillcolor{currentfill}%
\pgfsetfillopacity{0.587768}%
\pgfsetlinewidth{1.003750pt}%
\definecolor{currentstroke}{rgb}{0.121569,0.466667,0.705882}%
\pgfsetstrokecolor{currentstroke}%
\pgfsetstrokeopacity{0.587768}%
\pgfsetdash{}{0pt}%
\pgfpathmoveto{\pgfqpoint{0.854595in}{1.389376in}}%
\pgfpathcurveto{\pgfqpoint{0.862831in}{1.389376in}}{\pgfqpoint{0.870732in}{1.392648in}}{\pgfqpoint{0.876555in}{1.398472in}}%
\pgfpathcurveto{\pgfqpoint{0.882379in}{1.404296in}}{\pgfqpoint{0.885652in}{1.412196in}}{\pgfqpoint{0.885652in}{1.420432in}}%
\pgfpathcurveto{\pgfqpoint{0.885652in}{1.428669in}}{\pgfqpoint{0.882379in}{1.436569in}}{\pgfqpoint{0.876555in}{1.442393in}}%
\pgfpathcurveto{\pgfqpoint{0.870732in}{1.448217in}}{\pgfqpoint{0.862831in}{1.451489in}}{\pgfqpoint{0.854595in}{1.451489in}}%
\pgfpathcurveto{\pgfqpoint{0.846359in}{1.451489in}}{\pgfqpoint{0.838459in}{1.448217in}}{\pgfqpoint{0.832635in}{1.442393in}}%
\pgfpathcurveto{\pgfqpoint{0.826811in}{1.436569in}}{\pgfqpoint{0.823539in}{1.428669in}}{\pgfqpoint{0.823539in}{1.420432in}}%
\pgfpathcurveto{\pgfqpoint{0.823539in}{1.412196in}}{\pgfqpoint{0.826811in}{1.404296in}}{\pgfqpoint{0.832635in}{1.398472in}}%
\pgfpathcurveto{\pgfqpoint{0.838459in}{1.392648in}}{\pgfqpoint{0.846359in}{1.389376in}}{\pgfqpoint{0.854595in}{1.389376in}}%
\pgfpathclose%
\pgfusepath{stroke,fill}%
\end{pgfscope}%
\begin{pgfscope}%
\pgfpathrectangle{\pgfqpoint{0.100000in}{0.220728in}}{\pgfqpoint{3.696000in}{3.696000in}}%
\pgfusepath{clip}%
\pgfsetbuttcap%
\pgfsetroundjoin%
\definecolor{currentfill}{rgb}{0.121569,0.466667,0.705882}%
\pgfsetfillcolor{currentfill}%
\pgfsetfillopacity{0.588311}%
\pgfsetlinewidth{1.003750pt}%
\definecolor{currentstroke}{rgb}{0.121569,0.466667,0.705882}%
\pgfsetstrokecolor{currentstroke}%
\pgfsetstrokeopacity{0.588311}%
\pgfsetdash{}{0pt}%
\pgfpathmoveto{\pgfqpoint{0.852136in}{1.386212in}}%
\pgfpathcurveto{\pgfqpoint{0.860372in}{1.386212in}}{\pgfqpoint{0.868272in}{1.389484in}}{\pgfqpoint{0.874096in}{1.395308in}}%
\pgfpathcurveto{\pgfqpoint{0.879920in}{1.401132in}}{\pgfqpoint{0.883192in}{1.409032in}}{\pgfqpoint{0.883192in}{1.417269in}}%
\pgfpathcurveto{\pgfqpoint{0.883192in}{1.425505in}}{\pgfqpoint{0.879920in}{1.433405in}}{\pgfqpoint{0.874096in}{1.439229in}}%
\pgfpathcurveto{\pgfqpoint{0.868272in}{1.445053in}}{\pgfqpoint{0.860372in}{1.448325in}}{\pgfqpoint{0.852136in}{1.448325in}}%
\pgfpathcurveto{\pgfqpoint{0.843899in}{1.448325in}}{\pgfqpoint{0.835999in}{1.445053in}}{\pgfqpoint{0.830175in}{1.439229in}}%
\pgfpathcurveto{\pgfqpoint{0.824351in}{1.433405in}}{\pgfqpoint{0.821079in}{1.425505in}}{\pgfqpoint{0.821079in}{1.417269in}}%
\pgfpathcurveto{\pgfqpoint{0.821079in}{1.409032in}}{\pgfqpoint{0.824351in}{1.401132in}}{\pgfqpoint{0.830175in}{1.395308in}}%
\pgfpathcurveto{\pgfqpoint{0.835999in}{1.389484in}}{\pgfqpoint{0.843899in}{1.386212in}}{\pgfqpoint{0.852136in}{1.386212in}}%
\pgfpathclose%
\pgfusepath{stroke,fill}%
\end{pgfscope}%
\begin{pgfscope}%
\pgfpathrectangle{\pgfqpoint{0.100000in}{0.220728in}}{\pgfqpoint{3.696000in}{3.696000in}}%
\pgfusepath{clip}%
\pgfsetbuttcap%
\pgfsetroundjoin%
\definecolor{currentfill}{rgb}{0.121569,0.466667,0.705882}%
\pgfsetfillcolor{currentfill}%
\pgfsetfillopacity{0.589676}%
\pgfsetlinewidth{1.003750pt}%
\definecolor{currentstroke}{rgb}{0.121569,0.466667,0.705882}%
\pgfsetstrokecolor{currentstroke}%
\pgfsetstrokeopacity{0.589676}%
\pgfsetdash{}{0pt}%
\pgfpathmoveto{\pgfqpoint{0.844956in}{1.378766in}}%
\pgfpathcurveto{\pgfqpoint{0.853193in}{1.378766in}}{\pgfqpoint{0.861093in}{1.382039in}}{\pgfqpoint{0.866917in}{1.387863in}}%
\pgfpathcurveto{\pgfqpoint{0.872740in}{1.393687in}}{\pgfqpoint{0.876013in}{1.401587in}}{\pgfqpoint{0.876013in}{1.409823in}}%
\pgfpathcurveto{\pgfqpoint{0.876013in}{1.418059in}}{\pgfqpoint{0.872740in}{1.425959in}}{\pgfqpoint{0.866917in}{1.431783in}}%
\pgfpathcurveto{\pgfqpoint{0.861093in}{1.437607in}}{\pgfqpoint{0.853193in}{1.440879in}}{\pgfqpoint{0.844956in}{1.440879in}}%
\pgfpathcurveto{\pgfqpoint{0.836720in}{1.440879in}}{\pgfqpoint{0.828820in}{1.437607in}}{\pgfqpoint{0.822996in}{1.431783in}}%
\pgfpathcurveto{\pgfqpoint{0.817172in}{1.425959in}}{\pgfqpoint{0.813900in}{1.418059in}}{\pgfqpoint{0.813900in}{1.409823in}}%
\pgfpathcurveto{\pgfqpoint{0.813900in}{1.401587in}}{\pgfqpoint{0.817172in}{1.393687in}}{\pgfqpoint{0.822996in}{1.387863in}}%
\pgfpathcurveto{\pgfqpoint{0.828820in}{1.382039in}}{\pgfqpoint{0.836720in}{1.378766in}}{\pgfqpoint{0.844956in}{1.378766in}}%
\pgfpathclose%
\pgfusepath{stroke,fill}%
\end{pgfscope}%
\begin{pgfscope}%
\pgfpathrectangle{\pgfqpoint{0.100000in}{0.220728in}}{\pgfqpoint{3.696000in}{3.696000in}}%
\pgfusepath{clip}%
\pgfsetbuttcap%
\pgfsetroundjoin%
\definecolor{currentfill}{rgb}{0.121569,0.466667,0.705882}%
\pgfsetfillcolor{currentfill}%
\pgfsetfillopacity{0.590207}%
\pgfsetlinewidth{1.003750pt}%
\definecolor{currentstroke}{rgb}{0.121569,0.466667,0.705882}%
\pgfsetstrokecolor{currentstroke}%
\pgfsetstrokeopacity{0.590207}%
\pgfsetdash{}{0pt}%
\pgfpathmoveto{\pgfqpoint{0.755558in}{1.351183in}}%
\pgfpathcurveto{\pgfqpoint{0.763794in}{1.351183in}}{\pgfqpoint{0.771694in}{1.354455in}}{\pgfqpoint{0.777518in}{1.360279in}}%
\pgfpathcurveto{\pgfqpoint{0.783342in}{1.366103in}}{\pgfqpoint{0.786615in}{1.374003in}}{\pgfqpoint{0.786615in}{1.382240in}}%
\pgfpathcurveto{\pgfqpoint{0.786615in}{1.390476in}}{\pgfqpoint{0.783342in}{1.398376in}}{\pgfqpoint{0.777518in}{1.404200in}}%
\pgfpathcurveto{\pgfqpoint{0.771694in}{1.410024in}}{\pgfqpoint{0.763794in}{1.413296in}}{\pgfqpoint{0.755558in}{1.413296in}}%
\pgfpathcurveto{\pgfqpoint{0.747322in}{1.413296in}}{\pgfqpoint{0.739422in}{1.410024in}}{\pgfqpoint{0.733598in}{1.404200in}}%
\pgfpathcurveto{\pgfqpoint{0.727774in}{1.398376in}}{\pgfqpoint{0.724502in}{1.390476in}}{\pgfqpoint{0.724502in}{1.382240in}}%
\pgfpathcurveto{\pgfqpoint{0.724502in}{1.374003in}}{\pgfqpoint{0.727774in}{1.366103in}}{\pgfqpoint{0.733598in}{1.360279in}}%
\pgfpathcurveto{\pgfqpoint{0.739422in}{1.354455in}}{\pgfqpoint{0.747322in}{1.351183in}}{\pgfqpoint{0.755558in}{1.351183in}}%
\pgfpathclose%
\pgfusepath{stroke,fill}%
\end{pgfscope}%
\begin{pgfscope}%
\pgfpathrectangle{\pgfqpoint{0.100000in}{0.220728in}}{\pgfqpoint{3.696000in}{3.696000in}}%
\pgfusepath{clip}%
\pgfsetbuttcap%
\pgfsetroundjoin%
\definecolor{currentfill}{rgb}{0.121569,0.466667,0.705882}%
\pgfsetfillcolor{currentfill}%
\pgfsetfillopacity{0.590921}%
\pgfsetlinewidth{1.003750pt}%
\definecolor{currentstroke}{rgb}{0.121569,0.466667,0.705882}%
\pgfsetstrokecolor{currentstroke}%
\pgfsetstrokeopacity{0.590921}%
\pgfsetdash{}{0pt}%
\pgfpathmoveto{\pgfqpoint{2.895674in}{2.945232in}}%
\pgfpathcurveto{\pgfqpoint{2.903910in}{2.945232in}}{\pgfqpoint{2.911810in}{2.948504in}}{\pgfqpoint{2.917634in}{2.954328in}}%
\pgfpathcurveto{\pgfqpoint{2.923458in}{2.960152in}}{\pgfqpoint{2.926731in}{2.968052in}}{\pgfqpoint{2.926731in}{2.976289in}}%
\pgfpathcurveto{\pgfqpoint{2.926731in}{2.984525in}}{\pgfqpoint{2.923458in}{2.992425in}}{\pgfqpoint{2.917634in}{2.998249in}}%
\pgfpathcurveto{\pgfqpoint{2.911810in}{3.004073in}}{\pgfqpoint{2.903910in}{3.007345in}}{\pgfqpoint{2.895674in}{3.007345in}}%
\pgfpathcurveto{\pgfqpoint{2.887438in}{3.007345in}}{\pgfqpoint{2.879538in}{3.004073in}}{\pgfqpoint{2.873714in}{2.998249in}}%
\pgfpathcurveto{\pgfqpoint{2.867890in}{2.992425in}}{\pgfqpoint{2.864618in}{2.984525in}}{\pgfqpoint{2.864618in}{2.976289in}}%
\pgfpathcurveto{\pgfqpoint{2.864618in}{2.968052in}}{\pgfqpoint{2.867890in}{2.960152in}}{\pgfqpoint{2.873714in}{2.954328in}}%
\pgfpathcurveto{\pgfqpoint{2.879538in}{2.948504in}}{\pgfqpoint{2.887438in}{2.945232in}}{\pgfqpoint{2.895674in}{2.945232in}}%
\pgfpathclose%
\pgfusepath{stroke,fill}%
\end{pgfscope}%
\begin{pgfscope}%
\pgfpathrectangle{\pgfqpoint{0.100000in}{0.220728in}}{\pgfqpoint{3.696000in}{3.696000in}}%
\pgfusepath{clip}%
\pgfsetbuttcap%
\pgfsetroundjoin%
\definecolor{currentfill}{rgb}{0.121569,0.466667,0.705882}%
\pgfsetfillcolor{currentfill}%
\pgfsetfillopacity{0.592149}%
\pgfsetlinewidth{1.003750pt}%
\definecolor{currentstroke}{rgb}{0.121569,0.466667,0.705882}%
\pgfsetstrokecolor{currentstroke}%
\pgfsetstrokeopacity{0.592149}%
\pgfsetdash{}{0pt}%
\pgfpathmoveto{\pgfqpoint{0.832530in}{1.365748in}}%
\pgfpathcurveto{\pgfqpoint{0.840766in}{1.365748in}}{\pgfqpoint{0.848666in}{1.369020in}}{\pgfqpoint{0.854490in}{1.374844in}}%
\pgfpathcurveto{\pgfqpoint{0.860314in}{1.380668in}}{\pgfqpoint{0.863586in}{1.388568in}}{\pgfqpoint{0.863586in}{1.396804in}}%
\pgfpathcurveto{\pgfqpoint{0.863586in}{1.405041in}}{\pgfqpoint{0.860314in}{1.412941in}}{\pgfqpoint{0.854490in}{1.418765in}}%
\pgfpathcurveto{\pgfqpoint{0.848666in}{1.424589in}}{\pgfqpoint{0.840766in}{1.427861in}}{\pgfqpoint{0.832530in}{1.427861in}}%
\pgfpathcurveto{\pgfqpoint{0.824293in}{1.427861in}}{\pgfqpoint{0.816393in}{1.424589in}}{\pgfqpoint{0.810569in}{1.418765in}}%
\pgfpathcurveto{\pgfqpoint{0.804745in}{1.412941in}}{\pgfqpoint{0.801473in}{1.405041in}}{\pgfqpoint{0.801473in}{1.396804in}}%
\pgfpathcurveto{\pgfqpoint{0.801473in}{1.388568in}}{\pgfqpoint{0.804745in}{1.380668in}}{\pgfqpoint{0.810569in}{1.374844in}}%
\pgfpathcurveto{\pgfqpoint{0.816393in}{1.369020in}}{\pgfqpoint{0.824293in}{1.365748in}}{\pgfqpoint{0.832530in}{1.365748in}}%
\pgfpathclose%
\pgfusepath{stroke,fill}%
\end{pgfscope}%
\begin{pgfscope}%
\pgfpathrectangle{\pgfqpoint{0.100000in}{0.220728in}}{\pgfqpoint{3.696000in}{3.696000in}}%
\pgfusepath{clip}%
\pgfsetbuttcap%
\pgfsetroundjoin%
\definecolor{currentfill}{rgb}{0.121569,0.466667,0.705882}%
\pgfsetfillcolor{currentfill}%
\pgfsetfillopacity{0.592912}%
\pgfsetlinewidth{1.003750pt}%
\definecolor{currentstroke}{rgb}{0.121569,0.466667,0.705882}%
\pgfsetstrokecolor{currentstroke}%
\pgfsetstrokeopacity{0.592912}%
\pgfsetdash{}{0pt}%
\pgfpathmoveto{\pgfqpoint{0.750216in}{1.339636in}}%
\pgfpathcurveto{\pgfqpoint{0.758452in}{1.339636in}}{\pgfqpoint{0.766352in}{1.342908in}}{\pgfqpoint{0.772176in}{1.348732in}}%
\pgfpathcurveto{\pgfqpoint{0.778000in}{1.354556in}}{\pgfqpoint{0.781272in}{1.362456in}}{\pgfqpoint{0.781272in}{1.370693in}}%
\pgfpathcurveto{\pgfqpoint{0.781272in}{1.378929in}}{\pgfqpoint{0.778000in}{1.386829in}}{\pgfqpoint{0.772176in}{1.392653in}}%
\pgfpathcurveto{\pgfqpoint{0.766352in}{1.398477in}}{\pgfqpoint{0.758452in}{1.401749in}}{\pgfqpoint{0.750216in}{1.401749in}}%
\pgfpathcurveto{\pgfqpoint{0.741980in}{1.401749in}}{\pgfqpoint{0.734079in}{1.398477in}}{\pgfqpoint{0.728256in}{1.392653in}}%
\pgfpathcurveto{\pgfqpoint{0.722432in}{1.386829in}}{\pgfqpoint{0.719159in}{1.378929in}}{\pgfqpoint{0.719159in}{1.370693in}}%
\pgfpathcurveto{\pgfqpoint{0.719159in}{1.362456in}}{\pgfqpoint{0.722432in}{1.354556in}}{\pgfqpoint{0.728256in}{1.348732in}}%
\pgfpathcurveto{\pgfqpoint{0.734079in}{1.342908in}}{\pgfqpoint{0.741980in}{1.339636in}}{\pgfqpoint{0.750216in}{1.339636in}}%
\pgfpathclose%
\pgfusepath{stroke,fill}%
\end{pgfscope}%
\begin{pgfscope}%
\pgfpathrectangle{\pgfqpoint{0.100000in}{0.220728in}}{\pgfqpoint{3.696000in}{3.696000in}}%
\pgfusepath{clip}%
\pgfsetbuttcap%
\pgfsetroundjoin%
\definecolor{currentfill}{rgb}{0.121569,0.466667,0.705882}%
\pgfsetfillcolor{currentfill}%
\pgfsetfillopacity{0.593468}%
\pgfsetlinewidth{1.003750pt}%
\definecolor{currentstroke}{rgb}{0.121569,0.466667,0.705882}%
\pgfsetstrokecolor{currentstroke}%
\pgfsetstrokeopacity{0.593468}%
\pgfsetdash{}{0pt}%
\pgfpathmoveto{\pgfqpoint{2.904129in}{2.944292in}}%
\pgfpathcurveto{\pgfqpoint{2.912366in}{2.944292in}}{\pgfqpoint{2.920266in}{2.947564in}}{\pgfqpoint{2.926090in}{2.953388in}}%
\pgfpathcurveto{\pgfqpoint{2.931914in}{2.959212in}}{\pgfqpoint{2.935186in}{2.967112in}}{\pgfqpoint{2.935186in}{2.975348in}}%
\pgfpathcurveto{\pgfqpoint{2.935186in}{2.983585in}}{\pgfqpoint{2.931914in}{2.991485in}}{\pgfqpoint{2.926090in}{2.997309in}}%
\pgfpathcurveto{\pgfqpoint{2.920266in}{3.003133in}}{\pgfqpoint{2.912366in}{3.006405in}}{\pgfqpoint{2.904129in}{3.006405in}}%
\pgfpathcurveto{\pgfqpoint{2.895893in}{3.006405in}}{\pgfqpoint{2.887993in}{3.003133in}}{\pgfqpoint{2.882169in}{2.997309in}}%
\pgfpathcurveto{\pgfqpoint{2.876345in}{2.991485in}}{\pgfqpoint{2.873073in}{2.983585in}}{\pgfqpoint{2.873073in}{2.975348in}}%
\pgfpathcurveto{\pgfqpoint{2.873073in}{2.967112in}}{\pgfqpoint{2.876345in}{2.959212in}}{\pgfqpoint{2.882169in}{2.953388in}}%
\pgfpathcurveto{\pgfqpoint{2.887993in}{2.947564in}}{\pgfqpoint{2.895893in}{2.944292in}}{\pgfqpoint{2.904129in}{2.944292in}}%
\pgfpathclose%
\pgfusepath{stroke,fill}%
\end{pgfscope}%
\begin{pgfscope}%
\pgfpathrectangle{\pgfqpoint{0.100000in}{0.220728in}}{\pgfqpoint{3.696000in}{3.696000in}}%
\pgfusepath{clip}%
\pgfsetbuttcap%
\pgfsetroundjoin%
\definecolor{currentfill}{rgb}{0.121569,0.466667,0.705882}%
\pgfsetfillcolor{currentfill}%
\pgfsetfillopacity{0.593894}%
\pgfsetlinewidth{1.003750pt}%
\definecolor{currentstroke}{rgb}{0.121569,0.466667,0.705882}%
\pgfsetstrokecolor{currentstroke}%
\pgfsetstrokeopacity{0.593894}%
\pgfsetdash{}{0pt}%
\pgfpathmoveto{\pgfqpoint{0.746950in}{1.334703in}}%
\pgfpathcurveto{\pgfqpoint{0.755186in}{1.334703in}}{\pgfqpoint{0.763086in}{1.337976in}}{\pgfqpoint{0.768910in}{1.343800in}}%
\pgfpathcurveto{\pgfqpoint{0.774734in}{1.349624in}}{\pgfqpoint{0.778006in}{1.357524in}}{\pgfqpoint{0.778006in}{1.365760in}}%
\pgfpathcurveto{\pgfqpoint{0.778006in}{1.373996in}}{\pgfqpoint{0.774734in}{1.381896in}}{\pgfqpoint{0.768910in}{1.387720in}}%
\pgfpathcurveto{\pgfqpoint{0.763086in}{1.393544in}}{\pgfqpoint{0.755186in}{1.396816in}}{\pgfqpoint{0.746950in}{1.396816in}}%
\pgfpathcurveto{\pgfqpoint{0.738714in}{1.396816in}}{\pgfqpoint{0.730814in}{1.393544in}}{\pgfqpoint{0.724990in}{1.387720in}}%
\pgfpathcurveto{\pgfqpoint{0.719166in}{1.381896in}}{\pgfqpoint{0.715893in}{1.373996in}}{\pgfqpoint{0.715893in}{1.365760in}}%
\pgfpathcurveto{\pgfqpoint{0.715893in}{1.357524in}}{\pgfqpoint{0.719166in}{1.349624in}}{\pgfqpoint{0.724990in}{1.343800in}}%
\pgfpathcurveto{\pgfqpoint{0.730814in}{1.337976in}}{\pgfqpoint{0.738714in}{1.334703in}}{\pgfqpoint{0.746950in}{1.334703in}}%
\pgfpathclose%
\pgfusepath{stroke,fill}%
\end{pgfscope}%
\begin{pgfscope}%
\pgfpathrectangle{\pgfqpoint{0.100000in}{0.220728in}}{\pgfqpoint{3.696000in}{3.696000in}}%
\pgfusepath{clip}%
\pgfsetbuttcap%
\pgfsetroundjoin%
\definecolor{currentfill}{rgb}{0.121569,0.466667,0.705882}%
\pgfsetfillcolor{currentfill}%
\pgfsetfillopacity{0.595125}%
\pgfsetlinewidth{1.003750pt}%
\definecolor{currentstroke}{rgb}{0.121569,0.466667,0.705882}%
\pgfsetstrokecolor{currentstroke}%
\pgfsetstrokeopacity{0.595125}%
\pgfsetdash{}{0pt}%
\pgfpathmoveto{\pgfqpoint{0.816546in}{1.345988in}}%
\pgfpathcurveto{\pgfqpoint{0.824782in}{1.345988in}}{\pgfqpoint{0.832682in}{1.349261in}}{\pgfqpoint{0.838506in}{1.355085in}}%
\pgfpathcurveto{\pgfqpoint{0.844330in}{1.360909in}}{\pgfqpoint{0.847602in}{1.368809in}}{\pgfqpoint{0.847602in}{1.377045in}}%
\pgfpathcurveto{\pgfqpoint{0.847602in}{1.385281in}}{\pgfqpoint{0.844330in}{1.393181in}}{\pgfqpoint{0.838506in}{1.399005in}}%
\pgfpathcurveto{\pgfqpoint{0.832682in}{1.404829in}}{\pgfqpoint{0.824782in}{1.408101in}}{\pgfqpoint{0.816546in}{1.408101in}}%
\pgfpathcurveto{\pgfqpoint{0.808309in}{1.408101in}}{\pgfqpoint{0.800409in}{1.404829in}}{\pgfqpoint{0.794585in}{1.399005in}}%
\pgfpathcurveto{\pgfqpoint{0.788762in}{1.393181in}}{\pgfqpoint{0.785489in}{1.385281in}}{\pgfqpoint{0.785489in}{1.377045in}}%
\pgfpathcurveto{\pgfqpoint{0.785489in}{1.368809in}}{\pgfqpoint{0.788762in}{1.360909in}}{\pgfqpoint{0.794585in}{1.355085in}}%
\pgfpathcurveto{\pgfqpoint{0.800409in}{1.349261in}}{\pgfqpoint{0.808309in}{1.345988in}}{\pgfqpoint{0.816546in}{1.345988in}}%
\pgfpathclose%
\pgfusepath{stroke,fill}%
\end{pgfscope}%
\begin{pgfscope}%
\pgfpathrectangle{\pgfqpoint{0.100000in}{0.220728in}}{\pgfqpoint{3.696000in}{3.696000in}}%
\pgfusepath{clip}%
\pgfsetbuttcap%
\pgfsetroundjoin%
\definecolor{currentfill}{rgb}{0.121569,0.466667,0.705882}%
\pgfsetfillcolor{currentfill}%
\pgfsetfillopacity{0.595819}%
\pgfsetlinewidth{1.003750pt}%
\definecolor{currentstroke}{rgb}{0.121569,0.466667,0.705882}%
\pgfsetstrokecolor{currentstroke}%
\pgfsetstrokeopacity{0.595819}%
\pgfsetdash{}{0pt}%
\pgfpathmoveto{\pgfqpoint{0.741806in}{1.325437in}}%
\pgfpathcurveto{\pgfqpoint{0.750042in}{1.325437in}}{\pgfqpoint{0.757942in}{1.328710in}}{\pgfqpoint{0.763766in}{1.334533in}}%
\pgfpathcurveto{\pgfqpoint{0.769590in}{1.340357in}}{\pgfqpoint{0.772863in}{1.348257in}}{\pgfqpoint{0.772863in}{1.356494in}}%
\pgfpathcurveto{\pgfqpoint{0.772863in}{1.364730in}}{\pgfqpoint{0.769590in}{1.372630in}}{\pgfqpoint{0.763766in}{1.378454in}}%
\pgfpathcurveto{\pgfqpoint{0.757942in}{1.384278in}}{\pgfqpoint{0.750042in}{1.387550in}}{\pgfqpoint{0.741806in}{1.387550in}}%
\pgfpathcurveto{\pgfqpoint{0.733570in}{1.387550in}}{\pgfqpoint{0.725670in}{1.384278in}}{\pgfqpoint{0.719846in}{1.378454in}}%
\pgfpathcurveto{\pgfqpoint{0.714022in}{1.372630in}}{\pgfqpoint{0.710750in}{1.364730in}}{\pgfqpoint{0.710750in}{1.356494in}}%
\pgfpathcurveto{\pgfqpoint{0.710750in}{1.348257in}}{\pgfqpoint{0.714022in}{1.340357in}}{\pgfqpoint{0.719846in}{1.334533in}}%
\pgfpathcurveto{\pgfqpoint{0.725670in}{1.328710in}}{\pgfqpoint{0.733570in}{1.325437in}}{\pgfqpoint{0.741806in}{1.325437in}}%
\pgfpathclose%
\pgfusepath{stroke,fill}%
\end{pgfscope}%
\begin{pgfscope}%
\pgfpathrectangle{\pgfqpoint{0.100000in}{0.220728in}}{\pgfqpoint{3.696000in}{3.696000in}}%
\pgfusepath{clip}%
\pgfsetbuttcap%
\pgfsetroundjoin%
\definecolor{currentfill}{rgb}{0.121569,0.466667,0.705882}%
\pgfsetfillcolor{currentfill}%
\pgfsetfillopacity{0.596136}%
\pgfsetlinewidth{1.003750pt}%
\definecolor{currentstroke}{rgb}{0.121569,0.466667,0.705882}%
\pgfsetstrokecolor{currentstroke}%
\pgfsetstrokeopacity{0.596136}%
\pgfsetdash{}{0pt}%
\pgfpathmoveto{\pgfqpoint{2.918091in}{2.942910in}}%
\pgfpathcurveto{\pgfqpoint{2.926327in}{2.942910in}}{\pgfqpoint{2.934227in}{2.946182in}}{\pgfqpoint{2.940051in}{2.952006in}}%
\pgfpathcurveto{\pgfqpoint{2.945875in}{2.957830in}}{\pgfqpoint{2.949147in}{2.965730in}}{\pgfqpoint{2.949147in}{2.973967in}}%
\pgfpathcurveto{\pgfqpoint{2.949147in}{2.982203in}}{\pgfqpoint{2.945875in}{2.990103in}}{\pgfqpoint{2.940051in}{2.995927in}}%
\pgfpathcurveto{\pgfqpoint{2.934227in}{3.001751in}}{\pgfqpoint{2.926327in}{3.005023in}}{\pgfqpoint{2.918091in}{3.005023in}}%
\pgfpathcurveto{\pgfqpoint{2.909855in}{3.005023in}}{\pgfqpoint{2.901955in}{3.001751in}}{\pgfqpoint{2.896131in}{2.995927in}}%
\pgfpathcurveto{\pgfqpoint{2.890307in}{2.990103in}}{\pgfqpoint{2.887034in}{2.982203in}}{\pgfqpoint{2.887034in}{2.973967in}}%
\pgfpathcurveto{\pgfqpoint{2.887034in}{2.965730in}}{\pgfqpoint{2.890307in}{2.957830in}}{\pgfqpoint{2.896131in}{2.952006in}}%
\pgfpathcurveto{\pgfqpoint{2.901955in}{2.946182in}}{\pgfqpoint{2.909855in}{2.942910in}}{\pgfqpoint{2.918091in}{2.942910in}}%
\pgfpathclose%
\pgfusepath{stroke,fill}%
\end{pgfscope}%
\begin{pgfscope}%
\pgfpathrectangle{\pgfqpoint{0.100000in}{0.220728in}}{\pgfqpoint{3.696000in}{3.696000in}}%
\pgfusepath{clip}%
\pgfsetbuttcap%
\pgfsetroundjoin%
\definecolor{currentfill}{rgb}{0.121569,0.466667,0.705882}%
\pgfsetfillcolor{currentfill}%
\pgfsetfillopacity{0.598993}%
\pgfsetlinewidth{1.003750pt}%
\definecolor{currentstroke}{rgb}{0.121569,0.466667,0.705882}%
\pgfsetstrokecolor{currentstroke}%
\pgfsetstrokeopacity{0.598993}%
\pgfsetdash{}{0pt}%
\pgfpathmoveto{\pgfqpoint{0.795940in}{1.322167in}}%
\pgfpathcurveto{\pgfqpoint{0.804177in}{1.322167in}}{\pgfqpoint{0.812077in}{1.325439in}}{\pgfqpoint{0.817901in}{1.331263in}}%
\pgfpathcurveto{\pgfqpoint{0.823724in}{1.337087in}}{\pgfqpoint{0.826997in}{1.344987in}}{\pgfqpoint{0.826997in}{1.353223in}}%
\pgfpathcurveto{\pgfqpoint{0.826997in}{1.361460in}}{\pgfqpoint{0.823724in}{1.369360in}}{\pgfqpoint{0.817901in}{1.375184in}}%
\pgfpathcurveto{\pgfqpoint{0.812077in}{1.381008in}}{\pgfqpoint{0.804177in}{1.384280in}}{\pgfqpoint{0.795940in}{1.384280in}}%
\pgfpathcurveto{\pgfqpoint{0.787704in}{1.384280in}}{\pgfqpoint{0.779804in}{1.381008in}}{\pgfqpoint{0.773980in}{1.375184in}}%
\pgfpathcurveto{\pgfqpoint{0.768156in}{1.369360in}}{\pgfqpoint{0.764884in}{1.361460in}}{\pgfqpoint{0.764884in}{1.353223in}}%
\pgfpathcurveto{\pgfqpoint{0.764884in}{1.344987in}}{\pgfqpoint{0.768156in}{1.337087in}}{\pgfqpoint{0.773980in}{1.331263in}}%
\pgfpathcurveto{\pgfqpoint{0.779804in}{1.325439in}}{\pgfqpoint{0.787704in}{1.322167in}}{\pgfqpoint{0.795940in}{1.322167in}}%
\pgfpathclose%
\pgfusepath{stroke,fill}%
\end{pgfscope}%
\begin{pgfscope}%
\pgfpathrectangle{\pgfqpoint{0.100000in}{0.220728in}}{\pgfqpoint{3.696000in}{3.696000in}}%
\pgfusepath{clip}%
\pgfsetbuttcap%
\pgfsetroundjoin%
\definecolor{currentfill}{rgb}{0.121569,0.466667,0.705882}%
\pgfsetfillcolor{currentfill}%
\pgfsetfillopacity{0.599504}%
\pgfsetlinewidth{1.003750pt}%
\definecolor{currentstroke}{rgb}{0.121569,0.466667,0.705882}%
\pgfsetstrokecolor{currentstroke}%
\pgfsetstrokeopacity{0.599504}%
\pgfsetdash{}{0pt}%
\pgfpathmoveto{\pgfqpoint{0.733353in}{1.308564in}}%
\pgfpathcurveto{\pgfqpoint{0.741590in}{1.308564in}}{\pgfqpoint{0.749490in}{1.311837in}}{\pgfqpoint{0.755314in}{1.317660in}}%
\pgfpathcurveto{\pgfqpoint{0.761138in}{1.323484in}}{\pgfqpoint{0.764410in}{1.331384in}}{\pgfqpoint{0.764410in}{1.339621in}}%
\pgfpathcurveto{\pgfqpoint{0.764410in}{1.347857in}}{\pgfqpoint{0.761138in}{1.355757in}}{\pgfqpoint{0.755314in}{1.361581in}}%
\pgfpathcurveto{\pgfqpoint{0.749490in}{1.367405in}}{\pgfqpoint{0.741590in}{1.370677in}}{\pgfqpoint{0.733353in}{1.370677in}}%
\pgfpathcurveto{\pgfqpoint{0.725117in}{1.370677in}}{\pgfqpoint{0.717217in}{1.367405in}}{\pgfqpoint{0.711393in}{1.361581in}}%
\pgfpathcurveto{\pgfqpoint{0.705569in}{1.355757in}}{\pgfqpoint{0.702297in}{1.347857in}}{\pgfqpoint{0.702297in}{1.339621in}}%
\pgfpathcurveto{\pgfqpoint{0.702297in}{1.331384in}}{\pgfqpoint{0.705569in}{1.323484in}}{\pgfqpoint{0.711393in}{1.317660in}}%
\pgfpathcurveto{\pgfqpoint{0.717217in}{1.311837in}}{\pgfqpoint{0.725117in}{1.308564in}}{\pgfqpoint{0.733353in}{1.308564in}}%
\pgfpathclose%
\pgfusepath{stroke,fill}%
\end{pgfscope}%
\begin{pgfscope}%
\pgfpathrectangle{\pgfqpoint{0.100000in}{0.220728in}}{\pgfqpoint{3.696000in}{3.696000in}}%
\pgfusepath{clip}%
\pgfsetbuttcap%
\pgfsetroundjoin%
\definecolor{currentfill}{rgb}{0.121569,0.466667,0.705882}%
\pgfsetfillcolor{currentfill}%
\pgfsetfillopacity{0.600694}%
\pgfsetlinewidth{1.003750pt}%
\definecolor{currentstroke}{rgb}{0.121569,0.466667,0.705882}%
\pgfsetstrokecolor{currentstroke}%
\pgfsetstrokeopacity{0.600694}%
\pgfsetdash{}{0pt}%
\pgfpathmoveto{\pgfqpoint{2.934184in}{2.940932in}}%
\pgfpathcurveto{\pgfqpoint{2.942420in}{2.940932in}}{\pgfqpoint{2.950320in}{2.944205in}}{\pgfqpoint{2.956144in}{2.950029in}}%
\pgfpathcurveto{\pgfqpoint{2.961968in}{2.955853in}}{\pgfqpoint{2.965240in}{2.963753in}}{\pgfqpoint{2.965240in}{2.971989in}}%
\pgfpathcurveto{\pgfqpoint{2.965240in}{2.980225in}}{\pgfqpoint{2.961968in}{2.988125in}}{\pgfqpoint{2.956144in}{2.993949in}}%
\pgfpathcurveto{\pgfqpoint{2.950320in}{2.999773in}}{\pgfqpoint{2.942420in}{3.003045in}}{\pgfqpoint{2.934184in}{3.003045in}}%
\pgfpathcurveto{\pgfqpoint{2.925947in}{3.003045in}}{\pgfqpoint{2.918047in}{2.999773in}}{\pgfqpoint{2.912223in}{2.993949in}}%
\pgfpathcurveto{\pgfqpoint{2.906399in}{2.988125in}}{\pgfqpoint{2.903127in}{2.980225in}}{\pgfqpoint{2.903127in}{2.971989in}}%
\pgfpathcurveto{\pgfqpoint{2.903127in}{2.963753in}}{\pgfqpoint{2.906399in}{2.955853in}}{\pgfqpoint{2.912223in}{2.950029in}}%
\pgfpathcurveto{\pgfqpoint{2.918047in}{2.944205in}}{\pgfqpoint{2.925947in}{2.940932in}}{\pgfqpoint{2.934184in}{2.940932in}}%
\pgfpathclose%
\pgfusepath{stroke,fill}%
\end{pgfscope}%
\begin{pgfscope}%
\pgfpathrectangle{\pgfqpoint{0.100000in}{0.220728in}}{\pgfqpoint{3.696000in}{3.696000in}}%
\pgfusepath{clip}%
\pgfsetbuttcap%
\pgfsetroundjoin%
\definecolor{currentfill}{rgb}{0.121569,0.466667,0.705882}%
\pgfsetfillcolor{currentfill}%
\pgfsetfillopacity{0.603014}%
\pgfsetlinewidth{1.003750pt}%
\definecolor{currentstroke}{rgb}{0.121569,0.466667,0.705882}%
\pgfsetstrokecolor{currentstroke}%
\pgfsetstrokeopacity{0.603014}%
\pgfsetdash{}{0pt}%
\pgfpathmoveto{\pgfqpoint{2.943320in}{2.939701in}}%
\pgfpathcurveto{\pgfqpoint{2.951557in}{2.939701in}}{\pgfqpoint{2.959457in}{2.942974in}}{\pgfqpoint{2.965281in}{2.948798in}}%
\pgfpathcurveto{\pgfqpoint{2.971104in}{2.954622in}}{\pgfqpoint{2.974377in}{2.962522in}}{\pgfqpoint{2.974377in}{2.970758in}}%
\pgfpathcurveto{\pgfqpoint{2.974377in}{2.978994in}}{\pgfqpoint{2.971104in}{2.986894in}}{\pgfqpoint{2.965281in}{2.992718in}}%
\pgfpathcurveto{\pgfqpoint{2.959457in}{2.998542in}}{\pgfqpoint{2.951557in}{3.001814in}}{\pgfqpoint{2.943320in}{3.001814in}}%
\pgfpathcurveto{\pgfqpoint{2.935084in}{3.001814in}}{\pgfqpoint{2.927184in}{2.998542in}}{\pgfqpoint{2.921360in}{2.992718in}}%
\pgfpathcurveto{\pgfqpoint{2.915536in}{2.986894in}}{\pgfqpoint{2.912264in}{2.978994in}}{\pgfqpoint{2.912264in}{2.970758in}}%
\pgfpathcurveto{\pgfqpoint{2.912264in}{2.962522in}}{\pgfqpoint{2.915536in}{2.954622in}}{\pgfqpoint{2.921360in}{2.948798in}}%
\pgfpathcurveto{\pgfqpoint{2.927184in}{2.942974in}}{\pgfqpoint{2.935084in}{2.939701in}}{\pgfqpoint{2.943320in}{2.939701in}}%
\pgfpathclose%
\pgfusepath{stroke,fill}%
\end{pgfscope}%
\begin{pgfscope}%
\pgfpathrectangle{\pgfqpoint{0.100000in}{0.220728in}}{\pgfqpoint{3.696000in}{3.696000in}}%
\pgfusepath{clip}%
\pgfsetbuttcap%
\pgfsetroundjoin%
\definecolor{currentfill}{rgb}{0.121569,0.466667,0.705882}%
\pgfsetfillcolor{currentfill}%
\pgfsetfillopacity{0.603216}%
\pgfsetlinewidth{1.003750pt}%
\definecolor{currentstroke}{rgb}{0.121569,0.466667,0.705882}%
\pgfsetstrokecolor{currentstroke}%
\pgfsetstrokeopacity{0.603216}%
\pgfsetdash{}{0pt}%
\pgfpathmoveto{\pgfqpoint{0.770798in}{1.296168in}}%
\pgfpathcurveto{\pgfqpoint{0.779035in}{1.296168in}}{\pgfqpoint{0.786935in}{1.299441in}}{\pgfqpoint{0.792759in}{1.305265in}}%
\pgfpathcurveto{\pgfqpoint{0.798583in}{1.311088in}}{\pgfqpoint{0.801855in}{1.318988in}}{\pgfqpoint{0.801855in}{1.327225in}}%
\pgfpathcurveto{\pgfqpoint{0.801855in}{1.335461in}}{\pgfqpoint{0.798583in}{1.343361in}}{\pgfqpoint{0.792759in}{1.349185in}}%
\pgfpathcurveto{\pgfqpoint{0.786935in}{1.355009in}}{\pgfqpoint{0.779035in}{1.358281in}}{\pgfqpoint{0.770798in}{1.358281in}}%
\pgfpathcurveto{\pgfqpoint{0.762562in}{1.358281in}}{\pgfqpoint{0.754662in}{1.355009in}}{\pgfqpoint{0.748838in}{1.349185in}}%
\pgfpathcurveto{\pgfqpoint{0.743014in}{1.343361in}}{\pgfqpoint{0.739742in}{1.335461in}}{\pgfqpoint{0.739742in}{1.327225in}}%
\pgfpathcurveto{\pgfqpoint{0.739742in}{1.318988in}}{\pgfqpoint{0.743014in}{1.311088in}}{\pgfqpoint{0.748838in}{1.305265in}}%
\pgfpathcurveto{\pgfqpoint{0.754662in}{1.299441in}}{\pgfqpoint{0.762562in}{1.296168in}}{\pgfqpoint{0.770798in}{1.296168in}}%
\pgfpathclose%
\pgfusepath{stroke,fill}%
\end{pgfscope}%
\begin{pgfscope}%
\pgfpathrectangle{\pgfqpoint{0.100000in}{0.220728in}}{\pgfqpoint{3.696000in}{3.696000in}}%
\pgfusepath{clip}%
\pgfsetbuttcap%
\pgfsetroundjoin%
\definecolor{currentfill}{rgb}{0.121569,0.466667,0.705882}%
\pgfsetfillcolor{currentfill}%
\pgfsetfillopacity{0.604513}%
\pgfsetlinewidth{1.003750pt}%
\definecolor{currentstroke}{rgb}{0.121569,0.466667,0.705882}%
\pgfsetstrokecolor{currentstroke}%
\pgfsetstrokeopacity{0.604513}%
\pgfsetdash{}{0pt}%
\pgfpathmoveto{\pgfqpoint{0.711218in}{1.278103in}}%
\pgfpathcurveto{\pgfqpoint{0.719454in}{1.278103in}}{\pgfqpoint{0.727354in}{1.281375in}}{\pgfqpoint{0.733178in}{1.287199in}}%
\pgfpathcurveto{\pgfqpoint{0.739002in}{1.293023in}}{\pgfqpoint{0.742274in}{1.300923in}}{\pgfqpoint{0.742274in}{1.309159in}}%
\pgfpathcurveto{\pgfqpoint{0.742274in}{1.317395in}}{\pgfqpoint{0.739002in}{1.325295in}}{\pgfqpoint{0.733178in}{1.331119in}}%
\pgfpathcurveto{\pgfqpoint{0.727354in}{1.336943in}}{\pgfqpoint{0.719454in}{1.340216in}}{\pgfqpoint{0.711218in}{1.340216in}}%
\pgfpathcurveto{\pgfqpoint{0.702981in}{1.340216in}}{\pgfqpoint{0.695081in}{1.336943in}}{\pgfqpoint{0.689257in}{1.331119in}}%
\pgfpathcurveto{\pgfqpoint{0.683434in}{1.325295in}}{\pgfqpoint{0.680161in}{1.317395in}}{\pgfqpoint{0.680161in}{1.309159in}}%
\pgfpathcurveto{\pgfqpoint{0.680161in}{1.300923in}}{\pgfqpoint{0.683434in}{1.293023in}}{\pgfqpoint{0.689257in}{1.287199in}}%
\pgfpathcurveto{\pgfqpoint{0.695081in}{1.281375in}}{\pgfqpoint{0.702981in}{1.278103in}}{\pgfqpoint{0.711218in}{1.278103in}}%
\pgfpathclose%
\pgfusepath{stroke,fill}%
\end{pgfscope}%
\begin{pgfscope}%
\pgfpathrectangle{\pgfqpoint{0.100000in}{0.220728in}}{\pgfqpoint{3.696000in}{3.696000in}}%
\pgfusepath{clip}%
\pgfsetbuttcap%
\pgfsetroundjoin%
\definecolor{currentfill}{rgb}{0.121569,0.466667,0.705882}%
\pgfsetfillcolor{currentfill}%
\pgfsetfillopacity{0.606234}%
\pgfsetlinewidth{1.003750pt}%
\definecolor{currentstroke}{rgb}{0.121569,0.466667,0.705882}%
\pgfsetstrokecolor{currentstroke}%
\pgfsetstrokeopacity{0.606234}%
\pgfsetdash{}{0pt}%
\pgfpathmoveto{\pgfqpoint{2.956866in}{2.935981in}}%
\pgfpathcurveto{\pgfqpoint{2.965103in}{2.935981in}}{\pgfqpoint{2.973003in}{2.939253in}}{\pgfqpoint{2.978827in}{2.945077in}}%
\pgfpathcurveto{\pgfqpoint{2.984650in}{2.950901in}}{\pgfqpoint{2.987923in}{2.958801in}}{\pgfqpoint{2.987923in}{2.967037in}}%
\pgfpathcurveto{\pgfqpoint{2.987923in}{2.975273in}}{\pgfqpoint{2.984650in}{2.983174in}}{\pgfqpoint{2.978827in}{2.988997in}}%
\pgfpathcurveto{\pgfqpoint{2.973003in}{2.994821in}}{\pgfqpoint{2.965103in}{2.998094in}}{\pgfqpoint{2.956866in}{2.998094in}}%
\pgfpathcurveto{\pgfqpoint{2.948630in}{2.998094in}}{\pgfqpoint{2.940730in}{2.994821in}}{\pgfqpoint{2.934906in}{2.988997in}}%
\pgfpathcurveto{\pgfqpoint{2.929082in}{2.983174in}}{\pgfqpoint{2.925810in}{2.975273in}}{\pgfqpoint{2.925810in}{2.967037in}}%
\pgfpathcurveto{\pgfqpoint{2.925810in}{2.958801in}}{\pgfqpoint{2.929082in}{2.950901in}}{\pgfqpoint{2.934906in}{2.945077in}}%
\pgfpathcurveto{\pgfqpoint{2.940730in}{2.939253in}}{\pgfqpoint{2.948630in}{2.935981in}}{\pgfqpoint{2.956866in}{2.935981in}}%
\pgfpathclose%
\pgfusepath{stroke,fill}%
\end{pgfscope}%
\begin{pgfscope}%
\pgfpathrectangle{\pgfqpoint{0.100000in}{0.220728in}}{\pgfqpoint{3.696000in}{3.696000in}}%
\pgfusepath{clip}%
\pgfsetbuttcap%
\pgfsetroundjoin%
\definecolor{currentfill}{rgb}{0.121569,0.466667,0.705882}%
\pgfsetfillcolor{currentfill}%
\pgfsetfillopacity{0.607868}%
\pgfsetlinewidth{1.003750pt}%
\definecolor{currentstroke}{rgb}{0.121569,0.466667,0.705882}%
\pgfsetstrokecolor{currentstroke}%
\pgfsetstrokeopacity{0.607868}%
\pgfsetdash{}{0pt}%
\pgfpathmoveto{\pgfqpoint{0.743806in}{1.265575in}}%
\pgfpathcurveto{\pgfqpoint{0.752042in}{1.265575in}}{\pgfqpoint{0.759942in}{1.268847in}}{\pgfqpoint{0.765766in}{1.274671in}}%
\pgfpathcurveto{\pgfqpoint{0.771590in}{1.280495in}}{\pgfqpoint{0.774862in}{1.288395in}}{\pgfqpoint{0.774862in}{1.296631in}}%
\pgfpathcurveto{\pgfqpoint{0.774862in}{1.304868in}}{\pgfqpoint{0.771590in}{1.312768in}}{\pgfqpoint{0.765766in}{1.318592in}}%
\pgfpathcurveto{\pgfqpoint{0.759942in}{1.324416in}}{\pgfqpoint{0.752042in}{1.327688in}}{\pgfqpoint{0.743806in}{1.327688in}}%
\pgfpathcurveto{\pgfqpoint{0.735570in}{1.327688in}}{\pgfqpoint{0.727670in}{1.324416in}}{\pgfqpoint{0.721846in}{1.318592in}}%
\pgfpathcurveto{\pgfqpoint{0.716022in}{1.312768in}}{\pgfqpoint{0.712749in}{1.304868in}}{\pgfqpoint{0.712749in}{1.296631in}}%
\pgfpathcurveto{\pgfqpoint{0.712749in}{1.288395in}}{\pgfqpoint{0.716022in}{1.280495in}}{\pgfqpoint{0.721846in}{1.274671in}}%
\pgfpathcurveto{\pgfqpoint{0.727670in}{1.268847in}}{\pgfqpoint{0.735570in}{1.265575in}}{\pgfqpoint{0.743806in}{1.265575in}}%
\pgfpathclose%
\pgfusepath{stroke,fill}%
\end{pgfscope}%
\begin{pgfscope}%
\pgfpathrectangle{\pgfqpoint{0.100000in}{0.220728in}}{\pgfqpoint{3.696000in}{3.696000in}}%
\pgfusepath{clip}%
\pgfsetbuttcap%
\pgfsetroundjoin%
\definecolor{currentfill}{rgb}{0.121569,0.466667,0.705882}%
\pgfsetfillcolor{currentfill}%
\pgfsetfillopacity{0.608173}%
\pgfsetlinewidth{1.003750pt}%
\definecolor{currentstroke}{rgb}{0.121569,0.466667,0.705882}%
\pgfsetstrokecolor{currentstroke}%
\pgfsetstrokeopacity{0.608173}%
\pgfsetdash{}{0pt}%
\pgfpathmoveto{\pgfqpoint{2.964366in}{2.934829in}}%
\pgfpathcurveto{\pgfqpoint{2.972602in}{2.934829in}}{\pgfqpoint{2.980503in}{2.938102in}}{\pgfqpoint{2.986326in}{2.943926in}}%
\pgfpathcurveto{\pgfqpoint{2.992150in}{2.949750in}}{\pgfqpoint{2.995423in}{2.957650in}}{\pgfqpoint{2.995423in}{2.965886in}}%
\pgfpathcurveto{\pgfqpoint{2.995423in}{2.974122in}}{\pgfqpoint{2.992150in}{2.982022in}}{\pgfqpoint{2.986326in}{2.987846in}}%
\pgfpathcurveto{\pgfqpoint{2.980503in}{2.993670in}}{\pgfqpoint{2.972602in}{2.996942in}}{\pgfqpoint{2.964366in}{2.996942in}}%
\pgfpathcurveto{\pgfqpoint{2.956130in}{2.996942in}}{\pgfqpoint{2.948230in}{2.993670in}}{\pgfqpoint{2.942406in}{2.987846in}}%
\pgfpathcurveto{\pgfqpoint{2.936582in}{2.982022in}}{\pgfqpoint{2.933310in}{2.974122in}}{\pgfqpoint{2.933310in}{2.965886in}}%
\pgfpathcurveto{\pgfqpoint{2.933310in}{2.957650in}}{\pgfqpoint{2.936582in}{2.949750in}}{\pgfqpoint{2.942406in}{2.943926in}}%
\pgfpathcurveto{\pgfqpoint{2.948230in}{2.938102in}}{\pgfqpoint{2.956130in}{2.934829in}}{\pgfqpoint{2.964366in}{2.934829in}}%
\pgfpathclose%
\pgfusepath{stroke,fill}%
\end{pgfscope}%
\begin{pgfscope}%
\pgfpathrectangle{\pgfqpoint{0.100000in}{0.220728in}}{\pgfqpoint{3.696000in}{3.696000in}}%
\pgfusepath{clip}%
\pgfsetbuttcap%
\pgfsetroundjoin%
\definecolor{currentfill}{rgb}{0.121569,0.466667,0.705882}%
\pgfsetfillcolor{currentfill}%
\pgfsetfillopacity{0.610027}%
\pgfsetlinewidth{1.003750pt}%
\definecolor{currentstroke}{rgb}{0.121569,0.466667,0.705882}%
\pgfsetstrokecolor{currentstroke}%
\pgfsetstrokeopacity{0.610027}%
\pgfsetdash{}{0pt}%
\pgfpathmoveto{\pgfqpoint{0.702529in}{1.250729in}}%
\pgfpathcurveto{\pgfqpoint{0.710765in}{1.250729in}}{\pgfqpoint{0.718665in}{1.254001in}}{\pgfqpoint{0.724489in}{1.259825in}}%
\pgfpathcurveto{\pgfqpoint{0.730313in}{1.265649in}}{\pgfqpoint{0.733586in}{1.273549in}}{\pgfqpoint{0.733586in}{1.281786in}}%
\pgfpathcurveto{\pgfqpoint{0.733586in}{1.290022in}}{\pgfqpoint{0.730313in}{1.297922in}}{\pgfqpoint{0.724489in}{1.303746in}}%
\pgfpathcurveto{\pgfqpoint{0.718665in}{1.309570in}}{\pgfqpoint{0.710765in}{1.312842in}}{\pgfqpoint{0.702529in}{1.312842in}}%
\pgfpathcurveto{\pgfqpoint{0.694293in}{1.312842in}}{\pgfqpoint{0.686393in}{1.309570in}}{\pgfqpoint{0.680569in}{1.303746in}}%
\pgfpathcurveto{\pgfqpoint{0.674745in}{1.297922in}}{\pgfqpoint{0.671473in}{1.290022in}}{\pgfqpoint{0.671473in}{1.281786in}}%
\pgfpathcurveto{\pgfqpoint{0.671473in}{1.273549in}}{\pgfqpoint{0.674745in}{1.265649in}}{\pgfqpoint{0.680569in}{1.259825in}}%
\pgfpathcurveto{\pgfqpoint{0.686393in}{1.254001in}}{\pgfqpoint{0.694293in}{1.250729in}}{\pgfqpoint{0.702529in}{1.250729in}}%
\pgfpathclose%
\pgfusepath{stroke,fill}%
\end{pgfscope}%
\begin{pgfscope}%
\pgfpathrectangle{\pgfqpoint{0.100000in}{0.220728in}}{\pgfqpoint{3.696000in}{3.696000in}}%
\pgfusepath{clip}%
\pgfsetbuttcap%
\pgfsetroundjoin%
\definecolor{currentfill}{rgb}{0.121569,0.466667,0.705882}%
\pgfsetfillcolor{currentfill}%
\pgfsetfillopacity{0.610516}%
\pgfsetlinewidth{1.003750pt}%
\definecolor{currentstroke}{rgb}{0.121569,0.466667,0.705882}%
\pgfsetstrokecolor{currentstroke}%
\pgfsetstrokeopacity{0.610516}%
\pgfsetdash{}{0pt}%
\pgfpathmoveto{\pgfqpoint{2.975252in}{2.933521in}}%
\pgfpathcurveto{\pgfqpoint{2.983488in}{2.933521in}}{\pgfqpoint{2.991388in}{2.936793in}}{\pgfqpoint{2.997212in}{2.942617in}}%
\pgfpathcurveto{\pgfqpoint{3.003036in}{2.948441in}}{\pgfqpoint{3.006308in}{2.956341in}}{\pgfqpoint{3.006308in}{2.964577in}}%
\pgfpathcurveto{\pgfqpoint{3.006308in}{2.972814in}}{\pgfqpoint{3.003036in}{2.980714in}}{\pgfqpoint{2.997212in}{2.986537in}}%
\pgfpathcurveto{\pgfqpoint{2.991388in}{2.992361in}}{\pgfqpoint{2.983488in}{2.995634in}}{\pgfqpoint{2.975252in}{2.995634in}}%
\pgfpathcurveto{\pgfqpoint{2.967016in}{2.995634in}}{\pgfqpoint{2.959115in}{2.992361in}}{\pgfqpoint{2.953292in}{2.986537in}}%
\pgfpathcurveto{\pgfqpoint{2.947468in}{2.980714in}}{\pgfqpoint{2.944195in}{2.972814in}}{\pgfqpoint{2.944195in}{2.964577in}}%
\pgfpathcurveto{\pgfqpoint{2.944195in}{2.956341in}}{\pgfqpoint{2.947468in}{2.948441in}}{\pgfqpoint{2.953292in}{2.942617in}}%
\pgfpathcurveto{\pgfqpoint{2.959115in}{2.936793in}}{\pgfqpoint{2.967016in}{2.933521in}}{\pgfqpoint{2.975252in}{2.933521in}}%
\pgfpathclose%
\pgfusepath{stroke,fill}%
\end{pgfscope}%
\begin{pgfscope}%
\pgfpathrectangle{\pgfqpoint{0.100000in}{0.220728in}}{\pgfqpoint{3.696000in}{3.696000in}}%
\pgfusepath{clip}%
\pgfsetbuttcap%
\pgfsetroundjoin%
\definecolor{currentfill}{rgb}{0.121569,0.466667,0.705882}%
\pgfsetfillcolor{currentfill}%
\pgfsetfillopacity{0.612794}%
\pgfsetlinewidth{1.003750pt}%
\definecolor{currentstroke}{rgb}{0.121569,0.466667,0.705882}%
\pgfsetstrokecolor{currentstroke}%
\pgfsetstrokeopacity{0.612794}%
\pgfsetdash{}{0pt}%
\pgfpathmoveto{\pgfqpoint{0.690628in}{1.232030in}}%
\pgfpathcurveto{\pgfqpoint{0.698865in}{1.232030in}}{\pgfqpoint{0.706765in}{1.235302in}}{\pgfqpoint{0.712589in}{1.241126in}}%
\pgfpathcurveto{\pgfqpoint{0.718413in}{1.246950in}}{\pgfqpoint{0.721685in}{1.254850in}}{\pgfqpoint{0.721685in}{1.263086in}}%
\pgfpathcurveto{\pgfqpoint{0.721685in}{1.271323in}}{\pgfqpoint{0.718413in}{1.279223in}}{\pgfqpoint{0.712589in}{1.285047in}}%
\pgfpathcurveto{\pgfqpoint{0.706765in}{1.290871in}}{\pgfqpoint{0.698865in}{1.294143in}}{\pgfqpoint{0.690628in}{1.294143in}}%
\pgfpathcurveto{\pgfqpoint{0.682392in}{1.294143in}}{\pgfqpoint{0.674492in}{1.290871in}}{\pgfqpoint{0.668668in}{1.285047in}}%
\pgfpathcurveto{\pgfqpoint{0.662844in}{1.279223in}}{\pgfqpoint{0.659572in}{1.271323in}}{\pgfqpoint{0.659572in}{1.263086in}}%
\pgfpathcurveto{\pgfqpoint{0.659572in}{1.254850in}}{\pgfqpoint{0.662844in}{1.246950in}}{\pgfqpoint{0.668668in}{1.241126in}}%
\pgfpathcurveto{\pgfqpoint{0.674492in}{1.235302in}}{\pgfqpoint{0.682392in}{1.232030in}}{\pgfqpoint{0.690628in}{1.232030in}}%
\pgfpathclose%
\pgfusepath{stroke,fill}%
\end{pgfscope}%
\begin{pgfscope}%
\pgfpathrectangle{\pgfqpoint{0.100000in}{0.220728in}}{\pgfqpoint{3.696000in}{3.696000in}}%
\pgfusepath{clip}%
\pgfsetbuttcap%
\pgfsetroundjoin%
\definecolor{currentfill}{rgb}{0.121569,0.466667,0.705882}%
\pgfsetfillcolor{currentfill}%
\pgfsetfillopacity{0.613578}%
\pgfsetlinewidth{1.003750pt}%
\definecolor{currentstroke}{rgb}{0.121569,0.466667,0.705882}%
\pgfsetstrokecolor{currentstroke}%
\pgfsetstrokeopacity{0.613578}%
\pgfsetdash{}{0pt}%
\pgfpathmoveto{\pgfqpoint{0.714371in}{1.238128in}}%
\pgfpathcurveto{\pgfqpoint{0.722607in}{1.238128in}}{\pgfqpoint{0.730507in}{1.241400in}}{\pgfqpoint{0.736331in}{1.247224in}}%
\pgfpathcurveto{\pgfqpoint{0.742155in}{1.253048in}}{\pgfqpoint{0.745427in}{1.260948in}}{\pgfqpoint{0.745427in}{1.269184in}}%
\pgfpathcurveto{\pgfqpoint{0.745427in}{1.277421in}}{\pgfqpoint{0.742155in}{1.285321in}}{\pgfqpoint{0.736331in}{1.291145in}}%
\pgfpathcurveto{\pgfqpoint{0.730507in}{1.296969in}}{\pgfqpoint{0.722607in}{1.300241in}}{\pgfqpoint{0.714371in}{1.300241in}}%
\pgfpathcurveto{\pgfqpoint{0.706135in}{1.300241in}}{\pgfqpoint{0.698234in}{1.296969in}}{\pgfqpoint{0.692411in}{1.291145in}}%
\pgfpathcurveto{\pgfqpoint{0.686587in}{1.285321in}}{\pgfqpoint{0.683314in}{1.277421in}}{\pgfqpoint{0.683314in}{1.269184in}}%
\pgfpathcurveto{\pgfqpoint{0.683314in}{1.260948in}}{\pgfqpoint{0.686587in}{1.253048in}}{\pgfqpoint{0.692411in}{1.247224in}}%
\pgfpathcurveto{\pgfqpoint{0.698234in}{1.241400in}}{\pgfqpoint{0.706135in}{1.238128in}}{\pgfqpoint{0.714371in}{1.238128in}}%
\pgfpathclose%
\pgfusepath{stroke,fill}%
\end{pgfscope}%
\begin{pgfscope}%
\pgfpathrectangle{\pgfqpoint{0.100000in}{0.220728in}}{\pgfqpoint{3.696000in}{3.696000in}}%
\pgfusepath{clip}%
\pgfsetbuttcap%
\pgfsetroundjoin%
\definecolor{currentfill}{rgb}{0.121569,0.466667,0.705882}%
\pgfsetfillcolor{currentfill}%
\pgfsetfillopacity{0.614366}%
\pgfsetlinewidth{1.003750pt}%
\definecolor{currentstroke}{rgb}{0.121569,0.466667,0.705882}%
\pgfsetstrokecolor{currentstroke}%
\pgfsetstrokeopacity{0.614366}%
\pgfsetdash{}{0pt}%
\pgfpathmoveto{\pgfqpoint{0.686478in}{1.224753in}}%
\pgfpathcurveto{\pgfqpoint{0.694714in}{1.224753in}}{\pgfqpoint{0.702614in}{1.228025in}}{\pgfqpoint{0.708438in}{1.233849in}}%
\pgfpathcurveto{\pgfqpoint{0.714262in}{1.239673in}}{\pgfqpoint{0.717534in}{1.247573in}}{\pgfqpoint{0.717534in}{1.255810in}}%
\pgfpathcurveto{\pgfqpoint{0.717534in}{1.264046in}}{\pgfqpoint{0.714262in}{1.271946in}}{\pgfqpoint{0.708438in}{1.277770in}}%
\pgfpathcurveto{\pgfqpoint{0.702614in}{1.283594in}}{\pgfqpoint{0.694714in}{1.286866in}}{\pgfqpoint{0.686478in}{1.286866in}}%
\pgfpathcurveto{\pgfqpoint{0.678242in}{1.286866in}}{\pgfqpoint{0.670342in}{1.283594in}}{\pgfqpoint{0.664518in}{1.277770in}}%
\pgfpathcurveto{\pgfqpoint{0.658694in}{1.271946in}}{\pgfqpoint{0.655421in}{1.264046in}}{\pgfqpoint{0.655421in}{1.255810in}}%
\pgfpathcurveto{\pgfqpoint{0.655421in}{1.247573in}}{\pgfqpoint{0.658694in}{1.239673in}}{\pgfqpoint{0.664518in}{1.233849in}}%
\pgfpathcurveto{\pgfqpoint{0.670342in}{1.228025in}}{\pgfqpoint{0.678242in}{1.224753in}}{\pgfqpoint{0.686478in}{1.224753in}}%
\pgfpathclose%
\pgfusepath{stroke,fill}%
\end{pgfscope}%
\begin{pgfscope}%
\pgfpathrectangle{\pgfqpoint{0.100000in}{0.220728in}}{\pgfqpoint{3.696000in}{3.696000in}}%
\pgfusepath{clip}%
\pgfsetbuttcap%
\pgfsetroundjoin%
\definecolor{currentfill}{rgb}{0.121569,0.466667,0.705882}%
\pgfsetfillcolor{currentfill}%
\pgfsetfillopacity{0.614856}%
\pgfsetlinewidth{1.003750pt}%
\definecolor{currentstroke}{rgb}{0.121569,0.466667,0.705882}%
\pgfsetstrokecolor{currentstroke}%
\pgfsetstrokeopacity{0.614856}%
\pgfsetdash{}{0pt}%
\pgfpathmoveto{\pgfqpoint{2.993130in}{2.932168in}}%
\pgfpathcurveto{\pgfqpoint{3.001367in}{2.932168in}}{\pgfqpoint{3.009267in}{2.935441in}}{\pgfqpoint{3.015091in}{2.941265in}}%
\pgfpathcurveto{\pgfqpoint{3.020915in}{2.947088in}}{\pgfqpoint{3.024187in}{2.954989in}}{\pgfqpoint{3.024187in}{2.963225in}}%
\pgfpathcurveto{\pgfqpoint{3.024187in}{2.971461in}}{\pgfqpoint{3.020915in}{2.979361in}}{\pgfqpoint{3.015091in}{2.985185in}}%
\pgfpathcurveto{\pgfqpoint{3.009267in}{2.991009in}}{\pgfqpoint{3.001367in}{2.994281in}}{\pgfqpoint{2.993130in}{2.994281in}}%
\pgfpathcurveto{\pgfqpoint{2.984894in}{2.994281in}}{\pgfqpoint{2.976994in}{2.991009in}}{\pgfqpoint{2.971170in}{2.985185in}}%
\pgfpathcurveto{\pgfqpoint{2.965346in}{2.979361in}}{\pgfqpoint{2.962074in}{2.971461in}}{\pgfqpoint{2.962074in}{2.963225in}}%
\pgfpathcurveto{\pgfqpoint{2.962074in}{2.954989in}}{\pgfqpoint{2.965346in}{2.947088in}}{\pgfqpoint{2.971170in}{2.941265in}}%
\pgfpathcurveto{\pgfqpoint{2.976994in}{2.935441in}}{\pgfqpoint{2.984894in}{2.932168in}}{\pgfqpoint{2.993130in}{2.932168in}}%
\pgfpathclose%
\pgfusepath{stroke,fill}%
\end{pgfscope}%
\begin{pgfscope}%
\pgfpathrectangle{\pgfqpoint{0.100000in}{0.220728in}}{\pgfqpoint{3.696000in}{3.696000in}}%
\pgfusepath{clip}%
\pgfsetbuttcap%
\pgfsetroundjoin%
\definecolor{currentfill}{rgb}{0.121569,0.466667,0.705882}%
\pgfsetfillcolor{currentfill}%
\pgfsetfillopacity{0.615262}%
\pgfsetlinewidth{1.003750pt}%
\definecolor{currentstroke}{rgb}{0.121569,0.466667,0.705882}%
\pgfsetstrokecolor{currentstroke}%
\pgfsetstrokeopacity{0.615262}%
\pgfsetdash{}{0pt}%
\pgfpathmoveto{\pgfqpoint{0.684262in}{1.220743in}}%
\pgfpathcurveto{\pgfqpoint{0.692499in}{1.220743in}}{\pgfqpoint{0.700399in}{1.224015in}}{\pgfqpoint{0.706223in}{1.229839in}}%
\pgfpathcurveto{\pgfqpoint{0.712047in}{1.235663in}}{\pgfqpoint{0.715319in}{1.243563in}}{\pgfqpoint{0.715319in}{1.251799in}}%
\pgfpathcurveto{\pgfqpoint{0.715319in}{1.260035in}}{\pgfqpoint{0.712047in}{1.267935in}}{\pgfqpoint{0.706223in}{1.273759in}}%
\pgfpathcurveto{\pgfqpoint{0.700399in}{1.279583in}}{\pgfqpoint{0.692499in}{1.282856in}}{\pgfqpoint{0.684262in}{1.282856in}}%
\pgfpathcurveto{\pgfqpoint{0.676026in}{1.282856in}}{\pgfqpoint{0.668126in}{1.279583in}}{\pgfqpoint{0.662302in}{1.273759in}}%
\pgfpathcurveto{\pgfqpoint{0.656478in}{1.267935in}}{\pgfqpoint{0.653206in}{1.260035in}}{\pgfqpoint{0.653206in}{1.251799in}}%
\pgfpathcurveto{\pgfqpoint{0.653206in}{1.243563in}}{\pgfqpoint{0.656478in}{1.235663in}}{\pgfqpoint{0.662302in}{1.229839in}}%
\pgfpathcurveto{\pgfqpoint{0.668126in}{1.224015in}}{\pgfqpoint{0.676026in}{1.220743in}}{\pgfqpoint{0.684262in}{1.220743in}}%
\pgfpathclose%
\pgfusepath{stroke,fill}%
\end{pgfscope}%
\begin{pgfscope}%
\pgfpathrectangle{\pgfqpoint{0.100000in}{0.220728in}}{\pgfqpoint{3.696000in}{3.696000in}}%
\pgfusepath{clip}%
\pgfsetbuttcap%
\pgfsetroundjoin%
\definecolor{currentfill}{rgb}{0.121569,0.466667,0.705882}%
\pgfsetfillcolor{currentfill}%
\pgfsetfillopacity{0.616401}%
\pgfsetlinewidth{1.003750pt}%
\definecolor{currentstroke}{rgb}{0.121569,0.466667,0.705882}%
\pgfsetstrokecolor{currentstroke}%
\pgfsetstrokeopacity{0.616401}%
\pgfsetdash{}{0pt}%
\pgfpathmoveto{\pgfqpoint{0.678016in}{1.214504in}}%
\pgfpathcurveto{\pgfqpoint{0.686252in}{1.214504in}}{\pgfqpoint{0.694152in}{1.217776in}}{\pgfqpoint{0.699976in}{1.223600in}}%
\pgfpathcurveto{\pgfqpoint{0.705800in}{1.229424in}}{\pgfqpoint{0.709072in}{1.237324in}}{\pgfqpoint{0.709072in}{1.245560in}}%
\pgfpathcurveto{\pgfqpoint{0.709072in}{1.253797in}}{\pgfqpoint{0.705800in}{1.261697in}}{\pgfqpoint{0.699976in}{1.267521in}}%
\pgfpathcurveto{\pgfqpoint{0.694152in}{1.273345in}}{\pgfqpoint{0.686252in}{1.276617in}}{\pgfqpoint{0.678016in}{1.276617in}}%
\pgfpathcurveto{\pgfqpoint{0.669780in}{1.276617in}}{\pgfqpoint{0.661880in}{1.273345in}}{\pgfqpoint{0.656056in}{1.267521in}}%
\pgfpathcurveto{\pgfqpoint{0.650232in}{1.261697in}}{\pgfqpoint{0.646959in}{1.253797in}}{\pgfqpoint{0.646959in}{1.245560in}}%
\pgfpathcurveto{\pgfqpoint{0.646959in}{1.237324in}}{\pgfqpoint{0.650232in}{1.229424in}}{\pgfqpoint{0.656056in}{1.223600in}}%
\pgfpathcurveto{\pgfqpoint{0.661880in}{1.217776in}}{\pgfqpoint{0.669780in}{1.214504in}}{\pgfqpoint{0.678016in}{1.214504in}}%
\pgfpathclose%
\pgfusepath{stroke,fill}%
\end{pgfscope}%
\begin{pgfscope}%
\pgfpathrectangle{\pgfqpoint{0.100000in}{0.220728in}}{\pgfqpoint{3.696000in}{3.696000in}}%
\pgfusepath{clip}%
\pgfsetbuttcap%
\pgfsetroundjoin%
\definecolor{currentfill}{rgb}{0.121569,0.466667,0.705882}%
\pgfsetfillcolor{currentfill}%
\pgfsetfillopacity{0.616405}%
\pgfsetlinewidth{1.003750pt}%
\definecolor{currentstroke}{rgb}{0.121569,0.466667,0.705882}%
\pgfsetstrokecolor{currentstroke}%
\pgfsetstrokeopacity{0.616405}%
\pgfsetdash{}{0pt}%
\pgfpathmoveto{\pgfqpoint{0.678010in}{1.214488in}}%
\pgfpathcurveto{\pgfqpoint{0.686246in}{1.214488in}}{\pgfqpoint{0.694146in}{1.217760in}}{\pgfqpoint{0.699970in}{1.223584in}}%
\pgfpathcurveto{\pgfqpoint{0.705794in}{1.229408in}}{\pgfqpoint{0.709067in}{1.237308in}}{\pgfqpoint{0.709067in}{1.245544in}}%
\pgfpathcurveto{\pgfqpoint{0.709067in}{1.253780in}}{\pgfqpoint{0.705794in}{1.261680in}}{\pgfqpoint{0.699970in}{1.267504in}}%
\pgfpathcurveto{\pgfqpoint{0.694146in}{1.273328in}}{\pgfqpoint{0.686246in}{1.276601in}}{\pgfqpoint{0.678010in}{1.276601in}}%
\pgfpathcurveto{\pgfqpoint{0.669774in}{1.276601in}}{\pgfqpoint{0.661874in}{1.273328in}}{\pgfqpoint{0.656050in}{1.267504in}}%
\pgfpathcurveto{\pgfqpoint{0.650226in}{1.261680in}}{\pgfqpoint{0.646954in}{1.253780in}}{\pgfqpoint{0.646954in}{1.245544in}}%
\pgfpathcurveto{\pgfqpoint{0.646954in}{1.237308in}}{\pgfqpoint{0.650226in}{1.229408in}}{\pgfqpoint{0.656050in}{1.223584in}}%
\pgfpathcurveto{\pgfqpoint{0.661874in}{1.217760in}}{\pgfqpoint{0.669774in}{1.214488in}}{\pgfqpoint{0.678010in}{1.214488in}}%
\pgfpathclose%
\pgfusepath{stroke,fill}%
\end{pgfscope}%
\begin{pgfscope}%
\pgfpathrectangle{\pgfqpoint{0.100000in}{0.220728in}}{\pgfqpoint{3.696000in}{3.696000in}}%
\pgfusepath{clip}%
\pgfsetbuttcap%
\pgfsetroundjoin%
\definecolor{currentfill}{rgb}{0.121569,0.466667,0.705882}%
\pgfsetfillcolor{currentfill}%
\pgfsetfillopacity{0.616406}%
\pgfsetlinewidth{1.003750pt}%
\definecolor{currentstroke}{rgb}{0.121569,0.466667,0.705882}%
\pgfsetstrokecolor{currentstroke}%
\pgfsetstrokeopacity{0.616406}%
\pgfsetdash{}{0pt}%
\pgfpathmoveto{\pgfqpoint{0.698593in}{1.220695in}}%
\pgfpathcurveto{\pgfqpoint{0.706829in}{1.220695in}}{\pgfqpoint{0.714729in}{1.223968in}}{\pgfqpoint{0.720553in}{1.229792in}}%
\pgfpathcurveto{\pgfqpoint{0.726377in}{1.235615in}}{\pgfqpoint{0.729649in}{1.243516in}}{\pgfqpoint{0.729649in}{1.251752in}}%
\pgfpathcurveto{\pgfqpoint{0.729649in}{1.259988in}}{\pgfqpoint{0.726377in}{1.267888in}}{\pgfqpoint{0.720553in}{1.273712in}}%
\pgfpathcurveto{\pgfqpoint{0.714729in}{1.279536in}}{\pgfqpoint{0.706829in}{1.282808in}}{\pgfqpoint{0.698593in}{1.282808in}}%
\pgfpathcurveto{\pgfqpoint{0.690357in}{1.282808in}}{\pgfqpoint{0.682456in}{1.279536in}}{\pgfqpoint{0.676633in}{1.273712in}}%
\pgfpathcurveto{\pgfqpoint{0.670809in}{1.267888in}}{\pgfqpoint{0.667536in}{1.259988in}}{\pgfqpoint{0.667536in}{1.251752in}}%
\pgfpathcurveto{\pgfqpoint{0.667536in}{1.243516in}}{\pgfqpoint{0.670809in}{1.235615in}}{\pgfqpoint{0.676633in}{1.229792in}}%
\pgfpathcurveto{\pgfqpoint{0.682456in}{1.223968in}}{\pgfqpoint{0.690357in}{1.220695in}}{\pgfqpoint{0.698593in}{1.220695in}}%
\pgfpathclose%
\pgfusepath{stroke,fill}%
\end{pgfscope}%
\begin{pgfscope}%
\pgfpathrectangle{\pgfqpoint{0.100000in}{0.220728in}}{\pgfqpoint{3.696000in}{3.696000in}}%
\pgfusepath{clip}%
\pgfsetbuttcap%
\pgfsetroundjoin%
\definecolor{currentfill}{rgb}{0.121569,0.466667,0.705882}%
\pgfsetfillcolor{currentfill}%
\pgfsetfillopacity{0.616411}%
\pgfsetlinewidth{1.003750pt}%
\definecolor{currentstroke}{rgb}{0.121569,0.466667,0.705882}%
\pgfsetstrokecolor{currentstroke}%
\pgfsetstrokeopacity{0.616411}%
\pgfsetdash{}{0pt}%
\pgfpathmoveto{\pgfqpoint{0.677994in}{1.214459in}}%
\pgfpathcurveto{\pgfqpoint{0.686230in}{1.214459in}}{\pgfqpoint{0.694130in}{1.217731in}}{\pgfqpoint{0.699954in}{1.223555in}}%
\pgfpathcurveto{\pgfqpoint{0.705778in}{1.229379in}}{\pgfqpoint{0.709050in}{1.237279in}}{\pgfqpoint{0.709050in}{1.245515in}}%
\pgfpathcurveto{\pgfqpoint{0.709050in}{1.253751in}}{\pgfqpoint{0.705778in}{1.261651in}}{\pgfqpoint{0.699954in}{1.267475in}}%
\pgfpathcurveto{\pgfqpoint{0.694130in}{1.273299in}}{\pgfqpoint{0.686230in}{1.276572in}}{\pgfqpoint{0.677994in}{1.276572in}}%
\pgfpathcurveto{\pgfqpoint{0.669758in}{1.276572in}}{\pgfqpoint{0.661858in}{1.273299in}}{\pgfqpoint{0.656034in}{1.267475in}}%
\pgfpathcurveto{\pgfqpoint{0.650210in}{1.261651in}}{\pgfqpoint{0.646937in}{1.253751in}}{\pgfqpoint{0.646937in}{1.245515in}}%
\pgfpathcurveto{\pgfqpoint{0.646937in}{1.237279in}}{\pgfqpoint{0.650210in}{1.229379in}}{\pgfqpoint{0.656034in}{1.223555in}}%
\pgfpathcurveto{\pgfqpoint{0.661858in}{1.217731in}}{\pgfqpoint{0.669758in}{1.214459in}}{\pgfqpoint{0.677994in}{1.214459in}}%
\pgfpathclose%
\pgfusepath{stroke,fill}%
\end{pgfscope}%
\begin{pgfscope}%
\pgfpathrectangle{\pgfqpoint{0.100000in}{0.220728in}}{\pgfqpoint{3.696000in}{3.696000in}}%
\pgfusepath{clip}%
\pgfsetbuttcap%
\pgfsetroundjoin%
\definecolor{currentfill}{rgb}{0.121569,0.466667,0.705882}%
\pgfsetfillcolor{currentfill}%
\pgfsetfillopacity{0.616423}%
\pgfsetlinewidth{1.003750pt}%
\definecolor{currentstroke}{rgb}{0.121569,0.466667,0.705882}%
\pgfsetstrokecolor{currentstroke}%
\pgfsetstrokeopacity{0.616423}%
\pgfsetdash{}{0pt}%
\pgfpathmoveto{\pgfqpoint{0.677963in}{1.214406in}}%
\pgfpathcurveto{\pgfqpoint{0.686200in}{1.214406in}}{\pgfqpoint{0.694100in}{1.217679in}}{\pgfqpoint{0.699924in}{1.223503in}}%
\pgfpathcurveto{\pgfqpoint{0.705747in}{1.229327in}}{\pgfqpoint{0.709020in}{1.237227in}}{\pgfqpoint{0.709020in}{1.245463in}}%
\pgfpathcurveto{\pgfqpoint{0.709020in}{1.253699in}}{\pgfqpoint{0.705747in}{1.261599in}}{\pgfqpoint{0.699924in}{1.267423in}}%
\pgfpathcurveto{\pgfqpoint{0.694100in}{1.273247in}}{\pgfqpoint{0.686200in}{1.276519in}}{\pgfqpoint{0.677963in}{1.276519in}}%
\pgfpathcurveto{\pgfqpoint{0.669727in}{1.276519in}}{\pgfqpoint{0.661827in}{1.273247in}}{\pgfqpoint{0.656003in}{1.267423in}}%
\pgfpathcurveto{\pgfqpoint{0.650179in}{1.261599in}}{\pgfqpoint{0.646907in}{1.253699in}}{\pgfqpoint{0.646907in}{1.245463in}}%
\pgfpathcurveto{\pgfqpoint{0.646907in}{1.237227in}}{\pgfqpoint{0.650179in}{1.229327in}}{\pgfqpoint{0.656003in}{1.223503in}}%
\pgfpathcurveto{\pgfqpoint{0.661827in}{1.217679in}}{\pgfqpoint{0.669727in}{1.214406in}}{\pgfqpoint{0.677963in}{1.214406in}}%
\pgfpathclose%
\pgfusepath{stroke,fill}%
\end{pgfscope}%
\begin{pgfscope}%
\pgfpathrectangle{\pgfqpoint{0.100000in}{0.220728in}}{\pgfqpoint{3.696000in}{3.696000in}}%
\pgfusepath{clip}%
\pgfsetbuttcap%
\pgfsetroundjoin%
\definecolor{currentfill}{rgb}{0.121569,0.466667,0.705882}%
\pgfsetfillcolor{currentfill}%
\pgfsetfillopacity{0.616443}%
\pgfsetlinewidth{1.003750pt}%
\definecolor{currentstroke}{rgb}{0.121569,0.466667,0.705882}%
\pgfsetstrokecolor{currentstroke}%
\pgfsetstrokeopacity{0.616443}%
\pgfsetdash{}{0pt}%
\pgfpathmoveto{\pgfqpoint{0.677904in}{1.214313in}}%
\pgfpathcurveto{\pgfqpoint{0.686140in}{1.214313in}}{\pgfqpoint{0.694040in}{1.217586in}}{\pgfqpoint{0.699864in}{1.223410in}}%
\pgfpathcurveto{\pgfqpoint{0.705688in}{1.229233in}}{\pgfqpoint{0.708961in}{1.237133in}}{\pgfqpoint{0.708961in}{1.245370in}}%
\pgfpathcurveto{\pgfqpoint{0.708961in}{1.253606in}}{\pgfqpoint{0.705688in}{1.261506in}}{\pgfqpoint{0.699864in}{1.267330in}}%
\pgfpathcurveto{\pgfqpoint{0.694040in}{1.273154in}}{\pgfqpoint{0.686140in}{1.276426in}}{\pgfqpoint{0.677904in}{1.276426in}}%
\pgfpathcurveto{\pgfqpoint{0.669668in}{1.276426in}}{\pgfqpoint{0.661768in}{1.273154in}}{\pgfqpoint{0.655944in}{1.267330in}}%
\pgfpathcurveto{\pgfqpoint{0.650120in}{1.261506in}}{\pgfqpoint{0.646848in}{1.253606in}}{\pgfqpoint{0.646848in}{1.245370in}}%
\pgfpathcurveto{\pgfqpoint{0.646848in}{1.237133in}}{\pgfqpoint{0.650120in}{1.229233in}}{\pgfqpoint{0.655944in}{1.223410in}}%
\pgfpathcurveto{\pgfqpoint{0.661768in}{1.217586in}}{\pgfqpoint{0.669668in}{1.214313in}}{\pgfqpoint{0.677904in}{1.214313in}}%
\pgfpathclose%
\pgfusepath{stroke,fill}%
\end{pgfscope}%
\begin{pgfscope}%
\pgfpathrectangle{\pgfqpoint{0.100000in}{0.220728in}}{\pgfqpoint{3.696000in}{3.696000in}}%
\pgfusepath{clip}%
\pgfsetbuttcap%
\pgfsetroundjoin%
\definecolor{currentfill}{rgb}{0.121569,0.466667,0.705882}%
\pgfsetfillcolor{currentfill}%
\pgfsetfillopacity{0.616479}%
\pgfsetlinewidth{1.003750pt}%
\definecolor{currentstroke}{rgb}{0.121569,0.466667,0.705882}%
\pgfsetstrokecolor{currentstroke}%
\pgfsetstrokeopacity{0.616479}%
\pgfsetdash{}{0pt}%
\pgfpathmoveto{\pgfqpoint{0.677800in}{1.214136in}}%
\pgfpathcurveto{\pgfqpoint{0.686036in}{1.214136in}}{\pgfqpoint{0.693936in}{1.217408in}}{\pgfqpoint{0.699760in}{1.223232in}}%
\pgfpathcurveto{\pgfqpoint{0.705584in}{1.229056in}}{\pgfqpoint{0.708856in}{1.236956in}}{\pgfqpoint{0.708856in}{1.245192in}}%
\pgfpathcurveto{\pgfqpoint{0.708856in}{1.253429in}}{\pgfqpoint{0.705584in}{1.261329in}}{\pgfqpoint{0.699760in}{1.267153in}}%
\pgfpathcurveto{\pgfqpoint{0.693936in}{1.272976in}}{\pgfqpoint{0.686036in}{1.276249in}}{\pgfqpoint{0.677800in}{1.276249in}}%
\pgfpathcurveto{\pgfqpoint{0.669564in}{1.276249in}}{\pgfqpoint{0.661664in}{1.272976in}}{\pgfqpoint{0.655840in}{1.267153in}}%
\pgfpathcurveto{\pgfqpoint{0.650016in}{1.261329in}}{\pgfqpoint{0.646743in}{1.253429in}}{\pgfqpoint{0.646743in}{1.245192in}}%
\pgfpathcurveto{\pgfqpoint{0.646743in}{1.236956in}}{\pgfqpoint{0.650016in}{1.229056in}}{\pgfqpoint{0.655840in}{1.223232in}}%
\pgfpathcurveto{\pgfqpoint{0.661664in}{1.217408in}}{\pgfqpoint{0.669564in}{1.214136in}}{\pgfqpoint{0.677800in}{1.214136in}}%
\pgfpathclose%
\pgfusepath{stroke,fill}%
\end{pgfscope}%
\begin{pgfscope}%
\pgfpathrectangle{\pgfqpoint{0.100000in}{0.220728in}}{\pgfqpoint{3.696000in}{3.696000in}}%
\pgfusepath{clip}%
\pgfsetbuttcap%
\pgfsetroundjoin%
\definecolor{currentfill}{rgb}{0.121569,0.466667,0.705882}%
\pgfsetfillcolor{currentfill}%
\pgfsetfillopacity{0.616543}%
\pgfsetlinewidth{1.003750pt}%
\definecolor{currentstroke}{rgb}{0.121569,0.466667,0.705882}%
\pgfsetstrokecolor{currentstroke}%
\pgfsetstrokeopacity{0.616543}%
\pgfsetdash{}{0pt}%
\pgfpathmoveto{\pgfqpoint{0.677596in}{1.213827in}}%
\pgfpathcurveto{\pgfqpoint{0.685833in}{1.213827in}}{\pgfqpoint{0.693733in}{1.217100in}}{\pgfqpoint{0.699556in}{1.222924in}}%
\pgfpathcurveto{\pgfqpoint{0.705380in}{1.228748in}}{\pgfqpoint{0.708653in}{1.236648in}}{\pgfqpoint{0.708653in}{1.244884in}}%
\pgfpathcurveto{\pgfqpoint{0.708653in}{1.253120in}}{\pgfqpoint{0.705380in}{1.261020in}}{\pgfqpoint{0.699556in}{1.266844in}}%
\pgfpathcurveto{\pgfqpoint{0.693733in}{1.272668in}}{\pgfqpoint{0.685833in}{1.275940in}}{\pgfqpoint{0.677596in}{1.275940in}}%
\pgfpathcurveto{\pgfqpoint{0.669360in}{1.275940in}}{\pgfqpoint{0.661460in}{1.272668in}}{\pgfqpoint{0.655636in}{1.266844in}}%
\pgfpathcurveto{\pgfqpoint{0.649812in}{1.261020in}}{\pgfqpoint{0.646540in}{1.253120in}}{\pgfqpoint{0.646540in}{1.244884in}}%
\pgfpathcurveto{\pgfqpoint{0.646540in}{1.236648in}}{\pgfqpoint{0.649812in}{1.228748in}}{\pgfqpoint{0.655636in}{1.222924in}}%
\pgfpathcurveto{\pgfqpoint{0.661460in}{1.217100in}}{\pgfqpoint{0.669360in}{1.213827in}}{\pgfqpoint{0.677596in}{1.213827in}}%
\pgfpathclose%
\pgfusepath{stroke,fill}%
\end{pgfscope}%
\begin{pgfscope}%
\pgfpathrectangle{\pgfqpoint{0.100000in}{0.220728in}}{\pgfqpoint{3.696000in}{3.696000in}}%
\pgfusepath{clip}%
\pgfsetbuttcap%
\pgfsetroundjoin%
\definecolor{currentfill}{rgb}{0.121569,0.466667,0.705882}%
\pgfsetfillcolor{currentfill}%
\pgfsetfillopacity{0.616660}%
\pgfsetlinewidth{1.003750pt}%
\definecolor{currentstroke}{rgb}{0.121569,0.466667,0.705882}%
\pgfsetstrokecolor{currentstroke}%
\pgfsetstrokeopacity{0.616660}%
\pgfsetdash{}{0pt}%
\pgfpathmoveto{\pgfqpoint{0.677227in}{1.213261in}}%
\pgfpathcurveto{\pgfqpoint{0.685463in}{1.213261in}}{\pgfqpoint{0.693363in}{1.216533in}}{\pgfqpoint{0.699187in}{1.222357in}}%
\pgfpathcurveto{\pgfqpoint{0.705011in}{1.228181in}}{\pgfqpoint{0.708283in}{1.236081in}}{\pgfqpoint{0.708283in}{1.244317in}}%
\pgfpathcurveto{\pgfqpoint{0.708283in}{1.252554in}}{\pgfqpoint{0.705011in}{1.260454in}}{\pgfqpoint{0.699187in}{1.266278in}}%
\pgfpathcurveto{\pgfqpoint{0.693363in}{1.272102in}}{\pgfqpoint{0.685463in}{1.275374in}}{\pgfqpoint{0.677227in}{1.275374in}}%
\pgfpathcurveto{\pgfqpoint{0.668990in}{1.275374in}}{\pgfqpoint{0.661090in}{1.272102in}}{\pgfqpoint{0.655266in}{1.266278in}}%
\pgfpathcurveto{\pgfqpoint{0.649442in}{1.260454in}}{\pgfqpoint{0.646170in}{1.252554in}}{\pgfqpoint{0.646170in}{1.244317in}}%
\pgfpathcurveto{\pgfqpoint{0.646170in}{1.236081in}}{\pgfqpoint{0.649442in}{1.228181in}}{\pgfqpoint{0.655266in}{1.222357in}}%
\pgfpathcurveto{\pgfqpoint{0.661090in}{1.216533in}}{\pgfqpoint{0.668990in}{1.213261in}}{\pgfqpoint{0.677227in}{1.213261in}}%
\pgfpathclose%
\pgfusepath{stroke,fill}%
\end{pgfscope}%
\begin{pgfscope}%
\pgfpathrectangle{\pgfqpoint{0.100000in}{0.220728in}}{\pgfqpoint{3.696000in}{3.696000in}}%
\pgfusepath{clip}%
\pgfsetbuttcap%
\pgfsetroundjoin%
\definecolor{currentfill}{rgb}{0.121569,0.466667,0.705882}%
\pgfsetfillcolor{currentfill}%
\pgfsetfillopacity{0.616876}%
\pgfsetlinewidth{1.003750pt}%
\definecolor{currentstroke}{rgb}{0.121569,0.466667,0.705882}%
\pgfsetstrokecolor{currentstroke}%
\pgfsetstrokeopacity{0.616876}%
\pgfsetdash{}{0pt}%
\pgfpathmoveto{\pgfqpoint{0.676568in}{1.212238in}}%
\pgfpathcurveto{\pgfqpoint{0.684804in}{1.212238in}}{\pgfqpoint{0.692704in}{1.215510in}}{\pgfqpoint{0.698528in}{1.221334in}}%
\pgfpathcurveto{\pgfqpoint{0.704352in}{1.227158in}}{\pgfqpoint{0.707624in}{1.235058in}}{\pgfqpoint{0.707624in}{1.243294in}}%
\pgfpathcurveto{\pgfqpoint{0.707624in}{1.251531in}}{\pgfqpoint{0.704352in}{1.259431in}}{\pgfqpoint{0.698528in}{1.265255in}}%
\pgfpathcurveto{\pgfqpoint{0.692704in}{1.271079in}}{\pgfqpoint{0.684804in}{1.274351in}}{\pgfqpoint{0.676568in}{1.274351in}}%
\pgfpathcurveto{\pgfqpoint{0.668332in}{1.274351in}}{\pgfqpoint{0.660431in}{1.271079in}}{\pgfqpoint{0.654608in}{1.265255in}}%
\pgfpathcurveto{\pgfqpoint{0.648784in}{1.259431in}}{\pgfqpoint{0.645511in}{1.251531in}}{\pgfqpoint{0.645511in}{1.243294in}}%
\pgfpathcurveto{\pgfqpoint{0.645511in}{1.235058in}}{\pgfqpoint{0.648784in}{1.227158in}}{\pgfqpoint{0.654608in}{1.221334in}}%
\pgfpathcurveto{\pgfqpoint{0.660431in}{1.215510in}}{\pgfqpoint{0.668332in}{1.212238in}}{\pgfqpoint{0.676568in}{1.212238in}}%
\pgfpathclose%
\pgfusepath{stroke,fill}%
\end{pgfscope}%
\begin{pgfscope}%
\pgfpathrectangle{\pgfqpoint{0.100000in}{0.220728in}}{\pgfqpoint{3.696000in}{3.696000in}}%
\pgfusepath{clip}%
\pgfsetbuttcap%
\pgfsetroundjoin%
\definecolor{currentfill}{rgb}{0.121569,0.466667,0.705882}%
\pgfsetfillcolor{currentfill}%
\pgfsetfillopacity{0.617299}%
\pgfsetlinewidth{1.003750pt}%
\definecolor{currentstroke}{rgb}{0.121569,0.466667,0.705882}%
\pgfsetstrokecolor{currentstroke}%
\pgfsetstrokeopacity{0.617299}%
\pgfsetdash{}{0pt}%
\pgfpathmoveto{\pgfqpoint{0.675405in}{1.210476in}}%
\pgfpathcurveto{\pgfqpoint{0.683642in}{1.210476in}}{\pgfqpoint{0.691542in}{1.213748in}}{\pgfqpoint{0.697366in}{1.219572in}}%
\pgfpathcurveto{\pgfqpoint{0.703190in}{1.225396in}}{\pgfqpoint{0.706462in}{1.233296in}}{\pgfqpoint{0.706462in}{1.241532in}}%
\pgfpathcurveto{\pgfqpoint{0.706462in}{1.249769in}}{\pgfqpoint{0.703190in}{1.257669in}}{\pgfqpoint{0.697366in}{1.263493in}}%
\pgfpathcurveto{\pgfqpoint{0.691542in}{1.269317in}}{\pgfqpoint{0.683642in}{1.272589in}}{\pgfqpoint{0.675405in}{1.272589in}}%
\pgfpathcurveto{\pgfqpoint{0.667169in}{1.272589in}}{\pgfqpoint{0.659269in}{1.269317in}}{\pgfqpoint{0.653445in}{1.263493in}}%
\pgfpathcurveto{\pgfqpoint{0.647621in}{1.257669in}}{\pgfqpoint{0.644349in}{1.249769in}}{\pgfqpoint{0.644349in}{1.241532in}}%
\pgfpathcurveto{\pgfqpoint{0.644349in}{1.233296in}}{\pgfqpoint{0.647621in}{1.225396in}}{\pgfqpoint{0.653445in}{1.219572in}}%
\pgfpathcurveto{\pgfqpoint{0.659269in}{1.213748in}}{\pgfqpoint{0.667169in}{1.210476in}}{\pgfqpoint{0.675405in}{1.210476in}}%
\pgfpathclose%
\pgfusepath{stroke,fill}%
\end{pgfscope}%
\begin{pgfscope}%
\pgfpathrectangle{\pgfqpoint{0.100000in}{0.220728in}}{\pgfqpoint{3.696000in}{3.696000in}}%
\pgfusepath{clip}%
\pgfsetbuttcap%
\pgfsetroundjoin%
\definecolor{currentfill}{rgb}{0.121569,0.466667,0.705882}%
\pgfsetfillcolor{currentfill}%
\pgfsetfillopacity{0.617299}%
\pgfsetlinewidth{1.003750pt}%
\definecolor{currentstroke}{rgb}{0.121569,0.466667,0.705882}%
\pgfsetstrokecolor{currentstroke}%
\pgfsetstrokeopacity{0.617299}%
\pgfsetdash{}{0pt}%
\pgfpathmoveto{\pgfqpoint{0.675405in}{1.210475in}}%
\pgfpathcurveto{\pgfqpoint{0.683641in}{1.210475in}}{\pgfqpoint{0.691541in}{1.213747in}}{\pgfqpoint{0.697365in}{1.219571in}}%
\pgfpathcurveto{\pgfqpoint{0.703189in}{1.225395in}}{\pgfqpoint{0.706461in}{1.233295in}}{\pgfqpoint{0.706461in}{1.241532in}}%
\pgfpathcurveto{\pgfqpoint{0.706461in}{1.249768in}}{\pgfqpoint{0.703189in}{1.257668in}}{\pgfqpoint{0.697365in}{1.263492in}}%
\pgfpathcurveto{\pgfqpoint{0.691541in}{1.269316in}}{\pgfqpoint{0.683641in}{1.272588in}}{\pgfqpoint{0.675405in}{1.272588in}}%
\pgfpathcurveto{\pgfqpoint{0.667169in}{1.272588in}}{\pgfqpoint{0.659268in}{1.269316in}}{\pgfqpoint{0.653445in}{1.263492in}}%
\pgfpathcurveto{\pgfqpoint{0.647621in}{1.257668in}}{\pgfqpoint{0.644348in}{1.249768in}}{\pgfqpoint{0.644348in}{1.241532in}}%
\pgfpathcurveto{\pgfqpoint{0.644348in}{1.233295in}}{\pgfqpoint{0.647621in}{1.225395in}}{\pgfqpoint{0.653445in}{1.219571in}}%
\pgfpathcurveto{\pgfqpoint{0.659268in}{1.213747in}}{\pgfqpoint{0.667169in}{1.210475in}}{\pgfqpoint{0.675405in}{1.210475in}}%
\pgfpathclose%
\pgfusepath{stroke,fill}%
\end{pgfscope}%
\begin{pgfscope}%
\pgfpathrectangle{\pgfqpoint{0.100000in}{0.220728in}}{\pgfqpoint{3.696000in}{3.696000in}}%
\pgfusepath{clip}%
\pgfsetbuttcap%
\pgfsetroundjoin%
\definecolor{currentfill}{rgb}{0.121569,0.466667,0.705882}%
\pgfsetfillcolor{currentfill}%
\pgfsetfillopacity{0.617299}%
\pgfsetlinewidth{1.003750pt}%
\definecolor{currentstroke}{rgb}{0.121569,0.466667,0.705882}%
\pgfsetstrokecolor{currentstroke}%
\pgfsetstrokeopacity{0.617299}%
\pgfsetdash{}{0pt}%
\pgfpathmoveto{\pgfqpoint{0.675404in}{1.210473in}}%
\pgfpathcurveto{\pgfqpoint{0.683640in}{1.210473in}}{\pgfqpoint{0.691540in}{1.213746in}}{\pgfqpoint{0.697364in}{1.219570in}}%
\pgfpathcurveto{\pgfqpoint{0.703188in}{1.225394in}}{\pgfqpoint{0.706460in}{1.233294in}}{\pgfqpoint{0.706460in}{1.241530in}}%
\pgfpathcurveto{\pgfqpoint{0.706460in}{1.249766in}}{\pgfqpoint{0.703188in}{1.257666in}}{\pgfqpoint{0.697364in}{1.263490in}}%
\pgfpathcurveto{\pgfqpoint{0.691540in}{1.269314in}}{\pgfqpoint{0.683640in}{1.272586in}}{\pgfqpoint{0.675404in}{1.272586in}}%
\pgfpathcurveto{\pgfqpoint{0.667168in}{1.272586in}}{\pgfqpoint{0.659268in}{1.269314in}}{\pgfqpoint{0.653444in}{1.263490in}}%
\pgfpathcurveto{\pgfqpoint{0.647620in}{1.257666in}}{\pgfqpoint{0.644347in}{1.249766in}}{\pgfqpoint{0.644347in}{1.241530in}}%
\pgfpathcurveto{\pgfqpoint{0.644347in}{1.233294in}}{\pgfqpoint{0.647620in}{1.225394in}}{\pgfqpoint{0.653444in}{1.219570in}}%
\pgfpathcurveto{\pgfqpoint{0.659268in}{1.213746in}}{\pgfqpoint{0.667168in}{1.210473in}}{\pgfqpoint{0.675404in}{1.210473in}}%
\pgfpathclose%
\pgfusepath{stroke,fill}%
\end{pgfscope}%
\begin{pgfscope}%
\pgfpathrectangle{\pgfqpoint{0.100000in}{0.220728in}}{\pgfqpoint{3.696000in}{3.696000in}}%
\pgfusepath{clip}%
\pgfsetbuttcap%
\pgfsetroundjoin%
\definecolor{currentfill}{rgb}{0.121569,0.466667,0.705882}%
\pgfsetfillcolor{currentfill}%
\pgfsetfillopacity{0.617300}%
\pgfsetlinewidth{1.003750pt}%
\definecolor{currentstroke}{rgb}{0.121569,0.466667,0.705882}%
\pgfsetstrokecolor{currentstroke}%
\pgfsetstrokeopacity{0.617300}%
\pgfsetdash{}{0pt}%
\pgfpathmoveto{\pgfqpoint{0.675402in}{1.210471in}}%
\pgfpathcurveto{\pgfqpoint{0.683639in}{1.210471in}}{\pgfqpoint{0.691539in}{1.213743in}}{\pgfqpoint{0.697363in}{1.219567in}}%
\pgfpathcurveto{\pgfqpoint{0.703186in}{1.225391in}}{\pgfqpoint{0.706459in}{1.233291in}}{\pgfqpoint{0.706459in}{1.241527in}}%
\pgfpathcurveto{\pgfqpoint{0.706459in}{1.249763in}}{\pgfqpoint{0.703186in}{1.257663in}}{\pgfqpoint{0.697363in}{1.263487in}}%
\pgfpathcurveto{\pgfqpoint{0.691539in}{1.269311in}}{\pgfqpoint{0.683639in}{1.272584in}}{\pgfqpoint{0.675402in}{1.272584in}}%
\pgfpathcurveto{\pgfqpoint{0.667166in}{1.272584in}}{\pgfqpoint{0.659266in}{1.269311in}}{\pgfqpoint{0.653442in}{1.263487in}}%
\pgfpathcurveto{\pgfqpoint{0.647618in}{1.257663in}}{\pgfqpoint{0.644346in}{1.249763in}}{\pgfqpoint{0.644346in}{1.241527in}}%
\pgfpathcurveto{\pgfqpoint{0.644346in}{1.233291in}}{\pgfqpoint{0.647618in}{1.225391in}}{\pgfqpoint{0.653442in}{1.219567in}}%
\pgfpathcurveto{\pgfqpoint{0.659266in}{1.213743in}}{\pgfqpoint{0.667166in}{1.210471in}}{\pgfqpoint{0.675402in}{1.210471in}}%
\pgfpathclose%
\pgfusepath{stroke,fill}%
\end{pgfscope}%
\begin{pgfscope}%
\pgfpathrectangle{\pgfqpoint{0.100000in}{0.220728in}}{\pgfqpoint{3.696000in}{3.696000in}}%
\pgfusepath{clip}%
\pgfsetbuttcap%
\pgfsetroundjoin%
\definecolor{currentfill}{rgb}{0.121569,0.466667,0.705882}%
\pgfsetfillcolor{currentfill}%
\pgfsetfillopacity{0.617301}%
\pgfsetlinewidth{1.003750pt}%
\definecolor{currentstroke}{rgb}{0.121569,0.466667,0.705882}%
\pgfsetstrokecolor{currentstroke}%
\pgfsetstrokeopacity{0.617301}%
\pgfsetdash{}{0pt}%
\pgfpathmoveto{\pgfqpoint{0.675399in}{1.210465in}}%
\pgfpathcurveto{\pgfqpoint{0.683636in}{1.210465in}}{\pgfqpoint{0.691536in}{1.213737in}}{\pgfqpoint{0.697360in}{1.219561in}}%
\pgfpathcurveto{\pgfqpoint{0.703183in}{1.225385in}}{\pgfqpoint{0.706456in}{1.233285in}}{\pgfqpoint{0.706456in}{1.241522in}}%
\pgfpathcurveto{\pgfqpoint{0.706456in}{1.249758in}}{\pgfqpoint{0.703183in}{1.257658in}}{\pgfqpoint{0.697360in}{1.263482in}}%
\pgfpathcurveto{\pgfqpoint{0.691536in}{1.269306in}}{\pgfqpoint{0.683636in}{1.272578in}}{\pgfqpoint{0.675399in}{1.272578in}}%
\pgfpathcurveto{\pgfqpoint{0.667163in}{1.272578in}}{\pgfqpoint{0.659263in}{1.269306in}}{\pgfqpoint{0.653439in}{1.263482in}}%
\pgfpathcurveto{\pgfqpoint{0.647615in}{1.257658in}}{\pgfqpoint{0.644343in}{1.249758in}}{\pgfqpoint{0.644343in}{1.241522in}}%
\pgfpathcurveto{\pgfqpoint{0.644343in}{1.233285in}}{\pgfqpoint{0.647615in}{1.225385in}}{\pgfqpoint{0.653439in}{1.219561in}}%
\pgfpathcurveto{\pgfqpoint{0.659263in}{1.213737in}}{\pgfqpoint{0.667163in}{1.210465in}}{\pgfqpoint{0.675399in}{1.210465in}}%
\pgfpathclose%
\pgfusepath{stroke,fill}%
\end{pgfscope}%
\begin{pgfscope}%
\pgfpathrectangle{\pgfqpoint{0.100000in}{0.220728in}}{\pgfqpoint{3.696000in}{3.696000in}}%
\pgfusepath{clip}%
\pgfsetbuttcap%
\pgfsetroundjoin%
\definecolor{currentfill}{rgb}{0.121569,0.466667,0.705882}%
\pgfsetfillcolor{currentfill}%
\pgfsetfillopacity{0.617303}%
\pgfsetlinewidth{1.003750pt}%
\definecolor{currentstroke}{rgb}{0.121569,0.466667,0.705882}%
\pgfsetstrokecolor{currentstroke}%
\pgfsetstrokeopacity{0.617303}%
\pgfsetdash{}{0pt}%
\pgfpathmoveto{\pgfqpoint{0.675394in}{1.210456in}}%
\pgfpathcurveto{\pgfqpoint{0.683630in}{1.210456in}}{\pgfqpoint{0.691530in}{1.213728in}}{\pgfqpoint{0.697354in}{1.219552in}}%
\pgfpathcurveto{\pgfqpoint{0.703178in}{1.225376in}}{\pgfqpoint{0.706450in}{1.233276in}}{\pgfqpoint{0.706450in}{1.241512in}}%
\pgfpathcurveto{\pgfqpoint{0.706450in}{1.249748in}}{\pgfqpoint{0.703178in}{1.257648in}}{\pgfqpoint{0.697354in}{1.263472in}}%
\pgfpathcurveto{\pgfqpoint{0.691530in}{1.269296in}}{\pgfqpoint{0.683630in}{1.272569in}}{\pgfqpoint{0.675394in}{1.272569in}}%
\pgfpathcurveto{\pgfqpoint{0.667158in}{1.272569in}}{\pgfqpoint{0.659257in}{1.269296in}}{\pgfqpoint{0.653434in}{1.263472in}}%
\pgfpathcurveto{\pgfqpoint{0.647610in}{1.257648in}}{\pgfqpoint{0.644337in}{1.249748in}}{\pgfqpoint{0.644337in}{1.241512in}}%
\pgfpathcurveto{\pgfqpoint{0.644337in}{1.233276in}}{\pgfqpoint{0.647610in}{1.225376in}}{\pgfqpoint{0.653434in}{1.219552in}}%
\pgfpathcurveto{\pgfqpoint{0.659257in}{1.213728in}}{\pgfqpoint{0.667158in}{1.210456in}}{\pgfqpoint{0.675394in}{1.210456in}}%
\pgfpathclose%
\pgfusepath{stroke,fill}%
\end{pgfscope}%
\begin{pgfscope}%
\pgfpathrectangle{\pgfqpoint{0.100000in}{0.220728in}}{\pgfqpoint{3.696000in}{3.696000in}}%
\pgfusepath{clip}%
\pgfsetbuttcap%
\pgfsetroundjoin%
\definecolor{currentfill}{rgb}{0.121569,0.466667,0.705882}%
\pgfsetfillcolor{currentfill}%
\pgfsetfillopacity{0.617307}%
\pgfsetlinewidth{1.003750pt}%
\definecolor{currentstroke}{rgb}{0.121569,0.466667,0.705882}%
\pgfsetstrokecolor{currentstroke}%
\pgfsetstrokeopacity{0.617307}%
\pgfsetdash{}{0pt}%
\pgfpathmoveto{\pgfqpoint{0.675383in}{1.210437in}}%
\pgfpathcurveto{\pgfqpoint{0.683620in}{1.210437in}}{\pgfqpoint{0.691520in}{1.213709in}}{\pgfqpoint{0.697344in}{1.219533in}}%
\pgfpathcurveto{\pgfqpoint{0.703168in}{1.225357in}}{\pgfqpoint{0.706440in}{1.233257in}}{\pgfqpoint{0.706440in}{1.241494in}}%
\pgfpathcurveto{\pgfqpoint{0.706440in}{1.249730in}}{\pgfqpoint{0.703168in}{1.257630in}}{\pgfqpoint{0.697344in}{1.263454in}}%
\pgfpathcurveto{\pgfqpoint{0.691520in}{1.269278in}}{\pgfqpoint{0.683620in}{1.272550in}}{\pgfqpoint{0.675383in}{1.272550in}}%
\pgfpathcurveto{\pgfqpoint{0.667147in}{1.272550in}}{\pgfqpoint{0.659247in}{1.269278in}}{\pgfqpoint{0.653423in}{1.263454in}}%
\pgfpathcurveto{\pgfqpoint{0.647599in}{1.257630in}}{\pgfqpoint{0.644327in}{1.249730in}}{\pgfqpoint{0.644327in}{1.241494in}}%
\pgfpathcurveto{\pgfqpoint{0.644327in}{1.233257in}}{\pgfqpoint{0.647599in}{1.225357in}}{\pgfqpoint{0.653423in}{1.219533in}}%
\pgfpathcurveto{\pgfqpoint{0.659247in}{1.213709in}}{\pgfqpoint{0.667147in}{1.210437in}}{\pgfqpoint{0.675383in}{1.210437in}}%
\pgfpathclose%
\pgfusepath{stroke,fill}%
\end{pgfscope}%
\begin{pgfscope}%
\pgfpathrectangle{\pgfqpoint{0.100000in}{0.220728in}}{\pgfqpoint{3.696000in}{3.696000in}}%
\pgfusepath{clip}%
\pgfsetbuttcap%
\pgfsetroundjoin%
\definecolor{currentfill}{rgb}{0.121569,0.466667,0.705882}%
\pgfsetfillcolor{currentfill}%
\pgfsetfillopacity{0.617313}%
\pgfsetlinewidth{1.003750pt}%
\definecolor{currentstroke}{rgb}{0.121569,0.466667,0.705882}%
\pgfsetstrokecolor{currentstroke}%
\pgfsetstrokeopacity{0.617313}%
\pgfsetdash{}{0pt}%
\pgfpathmoveto{\pgfqpoint{0.675366in}{1.210402in}}%
\pgfpathcurveto{\pgfqpoint{0.683602in}{1.210402in}}{\pgfqpoint{0.691502in}{1.213674in}}{\pgfqpoint{0.697326in}{1.219498in}}%
\pgfpathcurveto{\pgfqpoint{0.703150in}{1.225322in}}{\pgfqpoint{0.706422in}{1.233222in}}{\pgfqpoint{0.706422in}{1.241458in}}%
\pgfpathcurveto{\pgfqpoint{0.706422in}{1.249695in}}{\pgfqpoint{0.703150in}{1.257595in}}{\pgfqpoint{0.697326in}{1.263419in}}%
\pgfpathcurveto{\pgfqpoint{0.691502in}{1.269243in}}{\pgfqpoint{0.683602in}{1.272515in}}{\pgfqpoint{0.675366in}{1.272515in}}%
\pgfpathcurveto{\pgfqpoint{0.667130in}{1.272515in}}{\pgfqpoint{0.659230in}{1.269243in}}{\pgfqpoint{0.653406in}{1.263419in}}%
\pgfpathcurveto{\pgfqpoint{0.647582in}{1.257595in}}{\pgfqpoint{0.644309in}{1.249695in}}{\pgfqpoint{0.644309in}{1.241458in}}%
\pgfpathcurveto{\pgfqpoint{0.644309in}{1.233222in}}{\pgfqpoint{0.647582in}{1.225322in}}{\pgfqpoint{0.653406in}{1.219498in}}%
\pgfpathcurveto{\pgfqpoint{0.659230in}{1.213674in}}{\pgfqpoint{0.667130in}{1.210402in}}{\pgfqpoint{0.675366in}{1.210402in}}%
\pgfpathclose%
\pgfusepath{stroke,fill}%
\end{pgfscope}%
\begin{pgfscope}%
\pgfpathrectangle{\pgfqpoint{0.100000in}{0.220728in}}{\pgfqpoint{3.696000in}{3.696000in}}%
\pgfusepath{clip}%
\pgfsetbuttcap%
\pgfsetroundjoin%
\definecolor{currentfill}{rgb}{0.121569,0.466667,0.705882}%
\pgfsetfillcolor{currentfill}%
\pgfsetfillopacity{0.617324}%
\pgfsetlinewidth{1.003750pt}%
\definecolor{currentstroke}{rgb}{0.121569,0.466667,0.705882}%
\pgfsetstrokecolor{currentstroke}%
\pgfsetstrokeopacity{0.617324}%
\pgfsetdash{}{0pt}%
\pgfpathmoveto{\pgfqpoint{0.675330in}{1.210340in}}%
\pgfpathcurveto{\pgfqpoint{0.683566in}{1.210340in}}{\pgfqpoint{0.691466in}{1.213612in}}{\pgfqpoint{0.697290in}{1.219436in}}%
\pgfpathcurveto{\pgfqpoint{0.703114in}{1.225260in}}{\pgfqpoint{0.706386in}{1.233160in}}{\pgfqpoint{0.706386in}{1.241396in}}%
\pgfpathcurveto{\pgfqpoint{0.706386in}{1.249632in}}{\pgfqpoint{0.703114in}{1.257532in}}{\pgfqpoint{0.697290in}{1.263356in}}%
\pgfpathcurveto{\pgfqpoint{0.691466in}{1.269180in}}{\pgfqpoint{0.683566in}{1.272453in}}{\pgfqpoint{0.675330in}{1.272453in}}%
\pgfpathcurveto{\pgfqpoint{0.667093in}{1.272453in}}{\pgfqpoint{0.659193in}{1.269180in}}{\pgfqpoint{0.653369in}{1.263356in}}%
\pgfpathcurveto{\pgfqpoint{0.647545in}{1.257532in}}{\pgfqpoint{0.644273in}{1.249632in}}{\pgfqpoint{0.644273in}{1.241396in}}%
\pgfpathcurveto{\pgfqpoint{0.644273in}{1.233160in}}{\pgfqpoint{0.647545in}{1.225260in}}{\pgfqpoint{0.653369in}{1.219436in}}%
\pgfpathcurveto{\pgfqpoint{0.659193in}{1.213612in}}{\pgfqpoint{0.667093in}{1.210340in}}{\pgfqpoint{0.675330in}{1.210340in}}%
\pgfpathclose%
\pgfusepath{stroke,fill}%
\end{pgfscope}%
\begin{pgfscope}%
\pgfpathrectangle{\pgfqpoint{0.100000in}{0.220728in}}{\pgfqpoint{3.696000in}{3.696000in}}%
\pgfusepath{clip}%
\pgfsetbuttcap%
\pgfsetroundjoin%
\definecolor{currentfill}{rgb}{0.121569,0.466667,0.705882}%
\pgfsetfillcolor{currentfill}%
\pgfsetfillopacity{0.617343}%
\pgfsetlinewidth{1.003750pt}%
\definecolor{currentstroke}{rgb}{0.121569,0.466667,0.705882}%
\pgfsetstrokecolor{currentstroke}%
\pgfsetstrokeopacity{0.617343}%
\pgfsetdash{}{0pt}%
\pgfpathmoveto{\pgfqpoint{0.675279in}{1.210209in}}%
\pgfpathcurveto{\pgfqpoint{0.683516in}{1.210209in}}{\pgfqpoint{0.691416in}{1.213481in}}{\pgfqpoint{0.697240in}{1.219305in}}%
\pgfpathcurveto{\pgfqpoint{0.703064in}{1.225129in}}{\pgfqpoint{0.706336in}{1.233029in}}{\pgfqpoint{0.706336in}{1.241265in}}%
\pgfpathcurveto{\pgfqpoint{0.706336in}{1.249502in}}{\pgfqpoint{0.703064in}{1.257402in}}{\pgfqpoint{0.697240in}{1.263226in}}%
\pgfpathcurveto{\pgfqpoint{0.691416in}{1.269050in}}{\pgfqpoint{0.683516in}{1.272322in}}{\pgfqpoint{0.675279in}{1.272322in}}%
\pgfpathcurveto{\pgfqpoint{0.667043in}{1.272322in}}{\pgfqpoint{0.659143in}{1.269050in}}{\pgfqpoint{0.653319in}{1.263226in}}%
\pgfpathcurveto{\pgfqpoint{0.647495in}{1.257402in}}{\pgfqpoint{0.644223in}{1.249502in}}{\pgfqpoint{0.644223in}{1.241265in}}%
\pgfpathcurveto{\pgfqpoint{0.644223in}{1.233029in}}{\pgfqpoint{0.647495in}{1.225129in}}{\pgfqpoint{0.653319in}{1.219305in}}%
\pgfpathcurveto{\pgfqpoint{0.659143in}{1.213481in}}{\pgfqpoint{0.667043in}{1.210209in}}{\pgfqpoint{0.675279in}{1.210209in}}%
\pgfpathclose%
\pgfusepath{stroke,fill}%
\end{pgfscope}%
\begin{pgfscope}%
\pgfpathrectangle{\pgfqpoint{0.100000in}{0.220728in}}{\pgfqpoint{3.696000in}{3.696000in}}%
\pgfusepath{clip}%
\pgfsetbuttcap%
\pgfsetroundjoin%
\definecolor{currentfill}{rgb}{0.121569,0.466667,0.705882}%
\pgfsetfillcolor{currentfill}%
\pgfsetfillopacity{0.617379}%
\pgfsetlinewidth{1.003750pt}%
\definecolor{currentstroke}{rgb}{0.121569,0.466667,0.705882}%
\pgfsetstrokecolor{currentstroke}%
\pgfsetstrokeopacity{0.617379}%
\pgfsetdash{}{0pt}%
\pgfpathmoveto{\pgfqpoint{0.675194in}{1.209977in}}%
\pgfpathcurveto{\pgfqpoint{0.683430in}{1.209977in}}{\pgfqpoint{0.691330in}{1.213249in}}{\pgfqpoint{0.697154in}{1.219073in}}%
\pgfpathcurveto{\pgfqpoint{0.702978in}{1.224897in}}{\pgfqpoint{0.706250in}{1.232797in}}{\pgfqpoint{0.706250in}{1.241033in}}%
\pgfpathcurveto{\pgfqpoint{0.706250in}{1.249270in}}{\pgfqpoint{0.702978in}{1.257170in}}{\pgfqpoint{0.697154in}{1.262994in}}%
\pgfpathcurveto{\pgfqpoint{0.691330in}{1.268817in}}{\pgfqpoint{0.683430in}{1.272090in}}{\pgfqpoint{0.675194in}{1.272090in}}%
\pgfpathcurveto{\pgfqpoint{0.666958in}{1.272090in}}{\pgfqpoint{0.659057in}{1.268817in}}{\pgfqpoint{0.653234in}{1.262994in}}%
\pgfpathcurveto{\pgfqpoint{0.647410in}{1.257170in}}{\pgfqpoint{0.644137in}{1.249270in}}{\pgfqpoint{0.644137in}{1.241033in}}%
\pgfpathcurveto{\pgfqpoint{0.644137in}{1.232797in}}{\pgfqpoint{0.647410in}{1.224897in}}{\pgfqpoint{0.653234in}{1.219073in}}%
\pgfpathcurveto{\pgfqpoint{0.659057in}{1.213249in}}{\pgfqpoint{0.666958in}{1.209977in}}{\pgfqpoint{0.675194in}{1.209977in}}%
\pgfpathclose%
\pgfusepath{stroke,fill}%
\end{pgfscope}%
\begin{pgfscope}%
\pgfpathrectangle{\pgfqpoint{0.100000in}{0.220728in}}{\pgfqpoint{3.696000in}{3.696000in}}%
\pgfusepath{clip}%
\pgfsetbuttcap%
\pgfsetroundjoin%
\definecolor{currentfill}{rgb}{0.121569,0.466667,0.705882}%
\pgfsetfillcolor{currentfill}%
\pgfsetfillopacity{0.617446}%
\pgfsetlinewidth{1.003750pt}%
\definecolor{currentstroke}{rgb}{0.121569,0.466667,0.705882}%
\pgfsetstrokecolor{currentstroke}%
\pgfsetstrokeopacity{0.617446}%
\pgfsetdash{}{0pt}%
\pgfpathmoveto{\pgfqpoint{0.675026in}{1.209560in}}%
\pgfpathcurveto{\pgfqpoint{0.683262in}{1.209560in}}{\pgfqpoint{0.691162in}{1.212833in}}{\pgfqpoint{0.696986in}{1.218657in}}%
\pgfpathcurveto{\pgfqpoint{0.702810in}{1.224480in}}{\pgfqpoint{0.706082in}{1.232381in}}{\pgfqpoint{0.706082in}{1.240617in}}%
\pgfpathcurveto{\pgfqpoint{0.706082in}{1.248853in}}{\pgfqpoint{0.702810in}{1.256753in}}{\pgfqpoint{0.696986in}{1.262577in}}%
\pgfpathcurveto{\pgfqpoint{0.691162in}{1.268401in}}{\pgfqpoint{0.683262in}{1.271673in}}{\pgfqpoint{0.675026in}{1.271673in}}%
\pgfpathcurveto{\pgfqpoint{0.666789in}{1.271673in}}{\pgfqpoint{0.658889in}{1.268401in}}{\pgfqpoint{0.653065in}{1.262577in}}%
\pgfpathcurveto{\pgfqpoint{0.647241in}{1.256753in}}{\pgfqpoint{0.643969in}{1.248853in}}{\pgfqpoint{0.643969in}{1.240617in}}%
\pgfpathcurveto{\pgfqpoint{0.643969in}{1.232381in}}{\pgfqpoint{0.647241in}{1.224480in}}{\pgfqpoint{0.653065in}{1.218657in}}%
\pgfpathcurveto{\pgfqpoint{0.658889in}{1.212833in}}{\pgfqpoint{0.666789in}{1.209560in}}{\pgfqpoint{0.675026in}{1.209560in}}%
\pgfpathclose%
\pgfusepath{stroke,fill}%
\end{pgfscope}%
\begin{pgfscope}%
\pgfpathrectangle{\pgfqpoint{0.100000in}{0.220728in}}{\pgfqpoint{3.696000in}{3.696000in}}%
\pgfusepath{clip}%
\pgfsetbuttcap%
\pgfsetroundjoin%
\definecolor{currentfill}{rgb}{0.121569,0.466667,0.705882}%
\pgfsetfillcolor{currentfill}%
\pgfsetfillopacity{0.617587}%
\pgfsetlinewidth{1.003750pt}%
\definecolor{currentstroke}{rgb}{0.121569,0.466667,0.705882}%
\pgfsetstrokecolor{currentstroke}%
\pgfsetstrokeopacity{0.617587}%
\pgfsetdash{}{0pt}%
\pgfpathmoveto{\pgfqpoint{0.674784in}{1.208859in}}%
\pgfpathcurveto{\pgfqpoint{0.683020in}{1.208859in}}{\pgfqpoint{0.690920in}{1.212132in}}{\pgfqpoint{0.696744in}{1.217956in}}%
\pgfpathcurveto{\pgfqpoint{0.702568in}{1.223780in}}{\pgfqpoint{0.705841in}{1.231680in}}{\pgfqpoint{0.705841in}{1.239916in}}%
\pgfpathcurveto{\pgfqpoint{0.705841in}{1.248152in}}{\pgfqpoint{0.702568in}{1.256052in}}{\pgfqpoint{0.696744in}{1.261876in}}%
\pgfpathcurveto{\pgfqpoint{0.690920in}{1.267700in}}{\pgfqpoint{0.683020in}{1.270972in}}{\pgfqpoint{0.674784in}{1.270972in}}%
\pgfpathcurveto{\pgfqpoint{0.666548in}{1.270972in}}{\pgfqpoint{0.658648in}{1.267700in}}{\pgfqpoint{0.652824in}{1.261876in}}%
\pgfpathcurveto{\pgfqpoint{0.647000in}{1.256052in}}{\pgfqpoint{0.643728in}{1.248152in}}{\pgfqpoint{0.643728in}{1.239916in}}%
\pgfpathcurveto{\pgfqpoint{0.643728in}{1.231680in}}{\pgfqpoint{0.647000in}{1.223780in}}{\pgfqpoint{0.652824in}{1.217956in}}%
\pgfpathcurveto{\pgfqpoint{0.658648in}{1.212132in}}{\pgfqpoint{0.666548in}{1.208859in}}{\pgfqpoint{0.674784in}{1.208859in}}%
\pgfpathclose%
\pgfusepath{stroke,fill}%
\end{pgfscope}%
\begin{pgfscope}%
\pgfpathrectangle{\pgfqpoint{0.100000in}{0.220728in}}{\pgfqpoint{3.696000in}{3.696000in}}%
\pgfusepath{clip}%
\pgfsetbuttcap%
\pgfsetroundjoin%
\definecolor{currentfill}{rgb}{0.121569,0.466667,0.705882}%
\pgfsetfillcolor{currentfill}%
\pgfsetfillopacity{0.617849}%
\pgfsetlinewidth{1.003750pt}%
\definecolor{currentstroke}{rgb}{0.121569,0.466667,0.705882}%
\pgfsetstrokecolor{currentstroke}%
\pgfsetstrokeopacity{0.617849}%
\pgfsetdash{}{0pt}%
\pgfpathmoveto{\pgfqpoint{0.674191in}{1.207721in}}%
\pgfpathcurveto{\pgfqpoint{0.682427in}{1.207721in}}{\pgfqpoint{0.690327in}{1.210993in}}{\pgfqpoint{0.696151in}{1.216817in}}%
\pgfpathcurveto{\pgfqpoint{0.701975in}{1.222641in}}{\pgfqpoint{0.705248in}{1.230541in}}{\pgfqpoint{0.705248in}{1.238777in}}%
\pgfpathcurveto{\pgfqpoint{0.705248in}{1.247013in}}{\pgfqpoint{0.701975in}{1.254914in}}{\pgfqpoint{0.696151in}{1.260737in}}%
\pgfpathcurveto{\pgfqpoint{0.690327in}{1.266561in}}{\pgfqpoint{0.682427in}{1.269834in}}{\pgfqpoint{0.674191in}{1.269834in}}%
\pgfpathcurveto{\pgfqpoint{0.665955in}{1.269834in}}{\pgfqpoint{0.658055in}{1.266561in}}{\pgfqpoint{0.652231in}{1.260737in}}%
\pgfpathcurveto{\pgfqpoint{0.646407in}{1.254914in}}{\pgfqpoint{0.643135in}{1.247013in}}{\pgfqpoint{0.643135in}{1.238777in}}%
\pgfpathcurveto{\pgfqpoint{0.643135in}{1.230541in}}{\pgfqpoint{0.646407in}{1.222641in}}{\pgfqpoint{0.652231in}{1.216817in}}%
\pgfpathcurveto{\pgfqpoint{0.658055in}{1.210993in}}{\pgfqpoint{0.665955in}{1.207721in}}{\pgfqpoint{0.674191in}{1.207721in}}%
\pgfpathclose%
\pgfusepath{stroke,fill}%
\end{pgfscope}%
\begin{pgfscope}%
\pgfpathrectangle{\pgfqpoint{0.100000in}{0.220728in}}{\pgfqpoint{3.696000in}{3.696000in}}%
\pgfusepath{clip}%
\pgfsetbuttcap%
\pgfsetroundjoin%
\definecolor{currentfill}{rgb}{0.121569,0.466667,0.705882}%
\pgfsetfillcolor{currentfill}%
\pgfsetfillopacity{0.617997}%
\pgfsetlinewidth{1.003750pt}%
\definecolor{currentstroke}{rgb}{0.121569,0.466667,0.705882}%
\pgfsetstrokecolor{currentstroke}%
\pgfsetstrokeopacity{0.617997}%
\pgfsetdash{}{0pt}%
\pgfpathmoveto{\pgfqpoint{0.689870in}{1.211340in}}%
\pgfpathcurveto{\pgfqpoint{0.698106in}{1.211340in}}{\pgfqpoint{0.706006in}{1.214612in}}{\pgfqpoint{0.711830in}{1.220436in}}%
\pgfpathcurveto{\pgfqpoint{0.717654in}{1.226260in}}{\pgfqpoint{0.720926in}{1.234160in}}{\pgfqpoint{0.720926in}{1.242396in}}%
\pgfpathcurveto{\pgfqpoint{0.720926in}{1.250633in}}{\pgfqpoint{0.717654in}{1.258533in}}{\pgfqpoint{0.711830in}{1.264357in}}%
\pgfpathcurveto{\pgfqpoint{0.706006in}{1.270181in}}{\pgfqpoint{0.698106in}{1.273453in}}{\pgfqpoint{0.689870in}{1.273453in}}%
\pgfpathcurveto{\pgfqpoint{0.681633in}{1.273453in}}{\pgfqpoint{0.673733in}{1.270181in}}{\pgfqpoint{0.667909in}{1.264357in}}%
\pgfpathcurveto{\pgfqpoint{0.662085in}{1.258533in}}{\pgfqpoint{0.658813in}{1.250633in}}{\pgfqpoint{0.658813in}{1.242396in}}%
\pgfpathcurveto{\pgfqpoint{0.658813in}{1.234160in}}{\pgfqpoint{0.662085in}{1.226260in}}{\pgfqpoint{0.667909in}{1.220436in}}%
\pgfpathcurveto{\pgfqpoint{0.673733in}{1.214612in}}{\pgfqpoint{0.681633in}{1.211340in}}{\pgfqpoint{0.689870in}{1.211340in}}%
\pgfpathclose%
\pgfusepath{stroke,fill}%
\end{pgfscope}%
\begin{pgfscope}%
\pgfpathrectangle{\pgfqpoint{0.100000in}{0.220728in}}{\pgfqpoint{3.696000in}{3.696000in}}%
\pgfusepath{clip}%
\pgfsetbuttcap%
\pgfsetroundjoin%
\definecolor{currentfill}{rgb}{0.121569,0.466667,0.705882}%
\pgfsetfillcolor{currentfill}%
\pgfsetfillopacity{0.618322}%
\pgfsetlinewidth{1.003750pt}%
\definecolor{currentstroke}{rgb}{0.121569,0.466667,0.705882}%
\pgfsetstrokecolor{currentstroke}%
\pgfsetstrokeopacity{0.618322}%
\pgfsetdash{}{0pt}%
\pgfpathmoveto{\pgfqpoint{0.673127in}{1.205612in}}%
\pgfpathcurveto{\pgfqpoint{0.681363in}{1.205612in}}{\pgfqpoint{0.689263in}{1.208885in}}{\pgfqpoint{0.695087in}{1.214708in}}%
\pgfpathcurveto{\pgfqpoint{0.700911in}{1.220532in}}{\pgfqpoint{0.704183in}{1.228432in}}{\pgfqpoint{0.704183in}{1.236669in}}%
\pgfpathcurveto{\pgfqpoint{0.704183in}{1.244905in}}{\pgfqpoint{0.700911in}{1.252805in}}{\pgfqpoint{0.695087in}{1.258629in}}%
\pgfpathcurveto{\pgfqpoint{0.689263in}{1.264453in}}{\pgfqpoint{0.681363in}{1.267725in}}{\pgfqpoint{0.673127in}{1.267725in}}%
\pgfpathcurveto{\pgfqpoint{0.664891in}{1.267725in}}{\pgfqpoint{0.656991in}{1.264453in}}{\pgfqpoint{0.651167in}{1.258629in}}%
\pgfpathcurveto{\pgfqpoint{0.645343in}{1.252805in}}{\pgfqpoint{0.642070in}{1.244905in}}{\pgfqpoint{0.642070in}{1.236669in}}%
\pgfpathcurveto{\pgfqpoint{0.642070in}{1.228432in}}{\pgfqpoint{0.645343in}{1.220532in}}{\pgfqpoint{0.651167in}{1.214708in}}%
\pgfpathcurveto{\pgfqpoint{0.656991in}{1.208885in}}{\pgfqpoint{0.664891in}{1.205612in}}{\pgfqpoint{0.673127in}{1.205612in}}%
\pgfpathclose%
\pgfusepath{stroke,fill}%
\end{pgfscope}%
\begin{pgfscope}%
\pgfpathrectangle{\pgfqpoint{0.100000in}{0.220728in}}{\pgfqpoint{3.696000in}{3.696000in}}%
\pgfusepath{clip}%
\pgfsetbuttcap%
\pgfsetroundjoin%
\definecolor{currentfill}{rgb}{0.121569,0.466667,0.705882}%
\pgfsetfillcolor{currentfill}%
\pgfsetfillopacity{0.618322}%
\pgfsetlinewidth{1.003750pt}%
\definecolor{currentstroke}{rgb}{0.121569,0.466667,0.705882}%
\pgfsetstrokecolor{currentstroke}%
\pgfsetstrokeopacity{0.618322}%
\pgfsetdash{}{0pt}%
\pgfpathmoveto{\pgfqpoint{0.673126in}{1.205611in}}%
\pgfpathcurveto{\pgfqpoint{0.681363in}{1.205611in}}{\pgfqpoint{0.689263in}{1.208883in}}{\pgfqpoint{0.695087in}{1.214707in}}%
\pgfpathcurveto{\pgfqpoint{0.700911in}{1.220531in}}{\pgfqpoint{0.704183in}{1.228431in}}{\pgfqpoint{0.704183in}{1.236668in}}%
\pgfpathcurveto{\pgfqpoint{0.704183in}{1.244904in}}{\pgfqpoint{0.700911in}{1.252804in}}{\pgfqpoint{0.695087in}{1.258628in}}%
\pgfpathcurveto{\pgfqpoint{0.689263in}{1.264452in}}{\pgfqpoint{0.681363in}{1.267724in}}{\pgfqpoint{0.673126in}{1.267724in}}%
\pgfpathcurveto{\pgfqpoint{0.664890in}{1.267724in}}{\pgfqpoint{0.656990in}{1.264452in}}{\pgfqpoint{0.651166in}{1.258628in}}%
\pgfpathcurveto{\pgfqpoint{0.645342in}{1.252804in}}{\pgfqpoint{0.642070in}{1.244904in}}{\pgfqpoint{0.642070in}{1.236668in}}%
\pgfpathcurveto{\pgfqpoint{0.642070in}{1.228431in}}{\pgfqpoint{0.645342in}{1.220531in}}{\pgfqpoint{0.651166in}{1.214707in}}%
\pgfpathcurveto{\pgfqpoint{0.656990in}{1.208883in}}{\pgfqpoint{0.664890in}{1.205611in}}{\pgfqpoint{0.673126in}{1.205611in}}%
\pgfpathclose%
\pgfusepath{stroke,fill}%
\end{pgfscope}%
\begin{pgfscope}%
\pgfpathrectangle{\pgfqpoint{0.100000in}{0.220728in}}{\pgfqpoint{3.696000in}{3.696000in}}%
\pgfusepath{clip}%
\pgfsetbuttcap%
\pgfsetroundjoin%
\definecolor{currentfill}{rgb}{0.121569,0.466667,0.705882}%
\pgfsetfillcolor{currentfill}%
\pgfsetfillopacity{0.618322}%
\pgfsetlinewidth{1.003750pt}%
\definecolor{currentstroke}{rgb}{0.121569,0.466667,0.705882}%
\pgfsetstrokecolor{currentstroke}%
\pgfsetstrokeopacity{0.618322}%
\pgfsetdash{}{0pt}%
\pgfpathmoveto{\pgfqpoint{0.673125in}{1.205609in}}%
\pgfpathcurveto{\pgfqpoint{0.681362in}{1.205609in}}{\pgfqpoint{0.689262in}{1.208881in}}{\pgfqpoint{0.695086in}{1.214705in}}%
\pgfpathcurveto{\pgfqpoint{0.700910in}{1.220529in}}{\pgfqpoint{0.704182in}{1.228429in}}{\pgfqpoint{0.704182in}{1.236666in}}%
\pgfpathcurveto{\pgfqpoint{0.704182in}{1.244902in}}{\pgfqpoint{0.700910in}{1.252802in}}{\pgfqpoint{0.695086in}{1.258626in}}%
\pgfpathcurveto{\pgfqpoint{0.689262in}{1.264450in}}{\pgfqpoint{0.681362in}{1.267722in}}{\pgfqpoint{0.673125in}{1.267722in}}%
\pgfpathcurveto{\pgfqpoint{0.664889in}{1.267722in}}{\pgfqpoint{0.656989in}{1.264450in}}{\pgfqpoint{0.651165in}{1.258626in}}%
\pgfpathcurveto{\pgfqpoint{0.645341in}{1.252802in}}{\pgfqpoint{0.642069in}{1.244902in}}{\pgfqpoint{0.642069in}{1.236666in}}%
\pgfpathcurveto{\pgfqpoint{0.642069in}{1.228429in}}{\pgfqpoint{0.645341in}{1.220529in}}{\pgfqpoint{0.651165in}{1.214705in}}%
\pgfpathcurveto{\pgfqpoint{0.656989in}{1.208881in}}{\pgfqpoint{0.664889in}{1.205609in}}{\pgfqpoint{0.673125in}{1.205609in}}%
\pgfpathclose%
\pgfusepath{stroke,fill}%
\end{pgfscope}%
\begin{pgfscope}%
\pgfpathrectangle{\pgfqpoint{0.100000in}{0.220728in}}{\pgfqpoint{3.696000in}{3.696000in}}%
\pgfusepath{clip}%
\pgfsetbuttcap%
\pgfsetroundjoin%
\definecolor{currentfill}{rgb}{0.121569,0.466667,0.705882}%
\pgfsetfillcolor{currentfill}%
\pgfsetfillopacity{0.618323}%
\pgfsetlinewidth{1.003750pt}%
\definecolor{currentstroke}{rgb}{0.121569,0.466667,0.705882}%
\pgfsetstrokecolor{currentstroke}%
\pgfsetstrokeopacity{0.618323}%
\pgfsetdash{}{0pt}%
\pgfpathmoveto{\pgfqpoint{0.673124in}{1.205605in}}%
\pgfpathcurveto{\pgfqpoint{0.681360in}{1.205605in}}{\pgfqpoint{0.689260in}{1.208878in}}{\pgfqpoint{0.695084in}{1.214702in}}%
\pgfpathcurveto{\pgfqpoint{0.700908in}{1.220525in}}{\pgfqpoint{0.704180in}{1.228426in}}{\pgfqpoint{0.704180in}{1.236662in}}%
\pgfpathcurveto{\pgfqpoint{0.704180in}{1.244898in}}{\pgfqpoint{0.700908in}{1.252798in}}{\pgfqpoint{0.695084in}{1.258622in}}%
\pgfpathcurveto{\pgfqpoint{0.689260in}{1.264446in}}{\pgfqpoint{0.681360in}{1.267718in}}{\pgfqpoint{0.673124in}{1.267718in}}%
\pgfpathcurveto{\pgfqpoint{0.664887in}{1.267718in}}{\pgfqpoint{0.656987in}{1.264446in}}{\pgfqpoint{0.651163in}{1.258622in}}%
\pgfpathcurveto{\pgfqpoint{0.645339in}{1.252798in}}{\pgfqpoint{0.642067in}{1.244898in}}{\pgfqpoint{0.642067in}{1.236662in}}%
\pgfpathcurveto{\pgfqpoint{0.642067in}{1.228426in}}{\pgfqpoint{0.645339in}{1.220525in}}{\pgfqpoint{0.651163in}{1.214702in}}%
\pgfpathcurveto{\pgfqpoint{0.656987in}{1.208878in}}{\pgfqpoint{0.664887in}{1.205605in}}{\pgfqpoint{0.673124in}{1.205605in}}%
\pgfpathclose%
\pgfusepath{stroke,fill}%
\end{pgfscope}%
\begin{pgfscope}%
\pgfpathrectangle{\pgfqpoint{0.100000in}{0.220728in}}{\pgfqpoint{3.696000in}{3.696000in}}%
\pgfusepath{clip}%
\pgfsetbuttcap%
\pgfsetroundjoin%
\definecolor{currentfill}{rgb}{0.121569,0.466667,0.705882}%
\pgfsetfillcolor{currentfill}%
\pgfsetfillopacity{0.618324}%
\pgfsetlinewidth{1.003750pt}%
\definecolor{currentstroke}{rgb}{0.121569,0.466667,0.705882}%
\pgfsetstrokecolor{currentstroke}%
\pgfsetstrokeopacity{0.618324}%
\pgfsetdash{}{0pt}%
\pgfpathmoveto{\pgfqpoint{0.673120in}{1.205599in}}%
\pgfpathcurveto{\pgfqpoint{0.681356in}{1.205599in}}{\pgfqpoint{0.689256in}{1.208871in}}{\pgfqpoint{0.695080in}{1.214695in}}%
\pgfpathcurveto{\pgfqpoint{0.700904in}{1.220519in}}{\pgfqpoint{0.704176in}{1.228419in}}{\pgfqpoint{0.704176in}{1.236655in}}%
\pgfpathcurveto{\pgfqpoint{0.704176in}{1.244892in}}{\pgfqpoint{0.700904in}{1.252792in}}{\pgfqpoint{0.695080in}{1.258616in}}%
\pgfpathcurveto{\pgfqpoint{0.689256in}{1.264440in}}{\pgfqpoint{0.681356in}{1.267712in}}{\pgfqpoint{0.673120in}{1.267712in}}%
\pgfpathcurveto{\pgfqpoint{0.664884in}{1.267712in}}{\pgfqpoint{0.656984in}{1.264440in}}{\pgfqpoint{0.651160in}{1.258616in}}%
\pgfpathcurveto{\pgfqpoint{0.645336in}{1.252792in}}{\pgfqpoint{0.642063in}{1.244892in}}{\pgfqpoint{0.642063in}{1.236655in}}%
\pgfpathcurveto{\pgfqpoint{0.642063in}{1.228419in}}{\pgfqpoint{0.645336in}{1.220519in}}{\pgfqpoint{0.651160in}{1.214695in}}%
\pgfpathcurveto{\pgfqpoint{0.656984in}{1.208871in}}{\pgfqpoint{0.664884in}{1.205599in}}{\pgfqpoint{0.673120in}{1.205599in}}%
\pgfpathclose%
\pgfusepath{stroke,fill}%
\end{pgfscope}%
\begin{pgfscope}%
\pgfpathrectangle{\pgfqpoint{0.100000in}{0.220728in}}{\pgfqpoint{3.696000in}{3.696000in}}%
\pgfusepath{clip}%
\pgfsetbuttcap%
\pgfsetroundjoin%
\definecolor{currentfill}{rgb}{0.121569,0.466667,0.705882}%
\pgfsetfillcolor{currentfill}%
\pgfsetfillopacity{0.618326}%
\pgfsetlinewidth{1.003750pt}%
\definecolor{currentstroke}{rgb}{0.121569,0.466667,0.705882}%
\pgfsetstrokecolor{currentstroke}%
\pgfsetstrokeopacity{0.618326}%
\pgfsetdash{}{0pt}%
\pgfpathmoveto{\pgfqpoint{0.673113in}{1.205586in}}%
\pgfpathcurveto{\pgfqpoint{0.681349in}{1.205586in}}{\pgfqpoint{0.689250in}{1.208858in}}{\pgfqpoint{0.695073in}{1.214682in}}%
\pgfpathcurveto{\pgfqpoint{0.700897in}{1.220506in}}{\pgfqpoint{0.704170in}{1.228406in}}{\pgfqpoint{0.704170in}{1.236643in}}%
\pgfpathcurveto{\pgfqpoint{0.704170in}{1.244879in}}{\pgfqpoint{0.700897in}{1.252779in}}{\pgfqpoint{0.695073in}{1.258603in}}%
\pgfpathcurveto{\pgfqpoint{0.689250in}{1.264427in}}{\pgfqpoint{0.681349in}{1.267699in}}{\pgfqpoint{0.673113in}{1.267699in}}%
\pgfpathcurveto{\pgfqpoint{0.664877in}{1.267699in}}{\pgfqpoint{0.656977in}{1.264427in}}{\pgfqpoint{0.651153in}{1.258603in}}%
\pgfpathcurveto{\pgfqpoint{0.645329in}{1.252779in}}{\pgfqpoint{0.642057in}{1.244879in}}{\pgfqpoint{0.642057in}{1.236643in}}%
\pgfpathcurveto{\pgfqpoint{0.642057in}{1.228406in}}{\pgfqpoint{0.645329in}{1.220506in}}{\pgfqpoint{0.651153in}{1.214682in}}%
\pgfpathcurveto{\pgfqpoint{0.656977in}{1.208858in}}{\pgfqpoint{0.664877in}{1.205586in}}{\pgfqpoint{0.673113in}{1.205586in}}%
\pgfpathclose%
\pgfusepath{stroke,fill}%
\end{pgfscope}%
\begin{pgfscope}%
\pgfpathrectangle{\pgfqpoint{0.100000in}{0.220728in}}{\pgfqpoint{3.696000in}{3.696000in}}%
\pgfusepath{clip}%
\pgfsetbuttcap%
\pgfsetroundjoin%
\definecolor{currentfill}{rgb}{0.121569,0.466667,0.705882}%
\pgfsetfillcolor{currentfill}%
\pgfsetfillopacity{0.618329}%
\pgfsetlinewidth{1.003750pt}%
\definecolor{currentstroke}{rgb}{0.121569,0.466667,0.705882}%
\pgfsetstrokecolor{currentstroke}%
\pgfsetstrokeopacity{0.618329}%
\pgfsetdash{}{0pt}%
\pgfpathmoveto{\pgfqpoint{0.673100in}{1.205562in}}%
\pgfpathcurveto{\pgfqpoint{0.681337in}{1.205562in}}{\pgfqpoint{0.689237in}{1.208835in}}{\pgfqpoint{0.695061in}{1.214659in}}%
\pgfpathcurveto{\pgfqpoint{0.700884in}{1.220483in}}{\pgfqpoint{0.704157in}{1.228383in}}{\pgfqpoint{0.704157in}{1.236619in}}%
\pgfpathcurveto{\pgfqpoint{0.704157in}{1.244855in}}{\pgfqpoint{0.700884in}{1.252755in}}{\pgfqpoint{0.695061in}{1.258579in}}%
\pgfpathcurveto{\pgfqpoint{0.689237in}{1.264403in}}{\pgfqpoint{0.681337in}{1.267675in}}{\pgfqpoint{0.673100in}{1.267675in}}%
\pgfpathcurveto{\pgfqpoint{0.664864in}{1.267675in}}{\pgfqpoint{0.656964in}{1.264403in}}{\pgfqpoint{0.651140in}{1.258579in}}%
\pgfpathcurveto{\pgfqpoint{0.645316in}{1.252755in}}{\pgfqpoint{0.642044in}{1.244855in}}{\pgfqpoint{0.642044in}{1.236619in}}%
\pgfpathcurveto{\pgfqpoint{0.642044in}{1.228383in}}{\pgfqpoint{0.645316in}{1.220483in}}{\pgfqpoint{0.651140in}{1.214659in}}%
\pgfpathcurveto{\pgfqpoint{0.656964in}{1.208835in}}{\pgfqpoint{0.664864in}{1.205562in}}{\pgfqpoint{0.673100in}{1.205562in}}%
\pgfpathclose%
\pgfusepath{stroke,fill}%
\end{pgfscope}%
\begin{pgfscope}%
\pgfpathrectangle{\pgfqpoint{0.100000in}{0.220728in}}{\pgfqpoint{3.696000in}{3.696000in}}%
\pgfusepath{clip}%
\pgfsetbuttcap%
\pgfsetroundjoin%
\definecolor{currentfill}{rgb}{0.121569,0.466667,0.705882}%
\pgfsetfillcolor{currentfill}%
\pgfsetfillopacity{0.618336}%
\pgfsetlinewidth{1.003750pt}%
\definecolor{currentstroke}{rgb}{0.121569,0.466667,0.705882}%
\pgfsetstrokecolor{currentstroke}%
\pgfsetstrokeopacity{0.618336}%
\pgfsetdash{}{0pt}%
\pgfpathmoveto{\pgfqpoint{0.673080in}{1.205519in}}%
\pgfpathcurveto{\pgfqpoint{0.681316in}{1.205519in}}{\pgfqpoint{0.689216in}{1.208791in}}{\pgfqpoint{0.695040in}{1.214615in}}%
\pgfpathcurveto{\pgfqpoint{0.700864in}{1.220439in}}{\pgfqpoint{0.704136in}{1.228339in}}{\pgfqpoint{0.704136in}{1.236576in}}%
\pgfpathcurveto{\pgfqpoint{0.704136in}{1.244812in}}{\pgfqpoint{0.700864in}{1.252712in}}{\pgfqpoint{0.695040in}{1.258536in}}%
\pgfpathcurveto{\pgfqpoint{0.689216in}{1.264360in}}{\pgfqpoint{0.681316in}{1.267632in}}{\pgfqpoint{0.673080in}{1.267632in}}%
\pgfpathcurveto{\pgfqpoint{0.664843in}{1.267632in}}{\pgfqpoint{0.656943in}{1.264360in}}{\pgfqpoint{0.651119in}{1.258536in}}%
\pgfpathcurveto{\pgfqpoint{0.645296in}{1.252712in}}{\pgfqpoint{0.642023in}{1.244812in}}{\pgfqpoint{0.642023in}{1.236576in}}%
\pgfpathcurveto{\pgfqpoint{0.642023in}{1.228339in}}{\pgfqpoint{0.645296in}{1.220439in}}{\pgfqpoint{0.651119in}{1.214615in}}%
\pgfpathcurveto{\pgfqpoint{0.656943in}{1.208791in}}{\pgfqpoint{0.664843in}{1.205519in}}{\pgfqpoint{0.673080in}{1.205519in}}%
\pgfpathclose%
\pgfusepath{stroke,fill}%
\end{pgfscope}%
\begin{pgfscope}%
\pgfpathrectangle{\pgfqpoint{0.100000in}{0.220728in}}{\pgfqpoint{3.696000in}{3.696000in}}%
\pgfusepath{clip}%
\pgfsetbuttcap%
\pgfsetroundjoin%
\definecolor{currentfill}{rgb}{0.121569,0.466667,0.705882}%
\pgfsetfillcolor{currentfill}%
\pgfsetfillopacity{0.618346}%
\pgfsetlinewidth{1.003750pt}%
\definecolor{currentstroke}{rgb}{0.121569,0.466667,0.705882}%
\pgfsetstrokecolor{currentstroke}%
\pgfsetstrokeopacity{0.618346}%
\pgfsetdash{}{0pt}%
\pgfpathmoveto{\pgfqpoint{0.673040in}{1.205437in}}%
\pgfpathcurveto{\pgfqpoint{0.681276in}{1.205437in}}{\pgfqpoint{0.689176in}{1.208709in}}{\pgfqpoint{0.695000in}{1.214533in}}%
\pgfpathcurveto{\pgfqpoint{0.700824in}{1.220357in}}{\pgfqpoint{0.704096in}{1.228257in}}{\pgfqpoint{0.704096in}{1.236494in}}%
\pgfpathcurveto{\pgfqpoint{0.704096in}{1.244730in}}{\pgfqpoint{0.700824in}{1.252630in}}{\pgfqpoint{0.695000in}{1.258454in}}%
\pgfpathcurveto{\pgfqpoint{0.689176in}{1.264278in}}{\pgfqpoint{0.681276in}{1.267550in}}{\pgfqpoint{0.673040in}{1.267550in}}%
\pgfpathcurveto{\pgfqpoint{0.664803in}{1.267550in}}{\pgfqpoint{0.656903in}{1.264278in}}{\pgfqpoint{0.651079in}{1.258454in}}%
\pgfpathcurveto{\pgfqpoint{0.645256in}{1.252630in}}{\pgfqpoint{0.641983in}{1.244730in}}{\pgfqpoint{0.641983in}{1.236494in}}%
\pgfpathcurveto{\pgfqpoint{0.641983in}{1.228257in}}{\pgfqpoint{0.645256in}{1.220357in}}{\pgfqpoint{0.651079in}{1.214533in}}%
\pgfpathcurveto{\pgfqpoint{0.656903in}{1.208709in}}{\pgfqpoint{0.664803in}{1.205437in}}{\pgfqpoint{0.673040in}{1.205437in}}%
\pgfpathclose%
\pgfusepath{stroke,fill}%
\end{pgfscope}%
\begin{pgfscope}%
\pgfpathrectangle{\pgfqpoint{0.100000in}{0.220728in}}{\pgfqpoint{3.696000in}{3.696000in}}%
\pgfusepath{clip}%
\pgfsetbuttcap%
\pgfsetroundjoin%
\definecolor{currentfill}{rgb}{0.121569,0.466667,0.705882}%
\pgfsetfillcolor{currentfill}%
\pgfsetfillopacity{0.618367}%
\pgfsetlinewidth{1.003750pt}%
\definecolor{currentstroke}{rgb}{0.121569,0.466667,0.705882}%
\pgfsetstrokecolor{currentstroke}%
\pgfsetstrokeopacity{0.618367}%
\pgfsetdash{}{0pt}%
\pgfpathmoveto{\pgfqpoint{0.672962in}{1.205300in}}%
\pgfpathcurveto{\pgfqpoint{0.681198in}{1.205300in}}{\pgfqpoint{0.689098in}{1.208573in}}{\pgfqpoint{0.694922in}{1.214396in}}%
\pgfpathcurveto{\pgfqpoint{0.700746in}{1.220220in}}{\pgfqpoint{0.704018in}{1.228120in}}{\pgfqpoint{0.704018in}{1.236357in}}%
\pgfpathcurveto{\pgfqpoint{0.704018in}{1.244593in}}{\pgfqpoint{0.700746in}{1.252493in}}{\pgfqpoint{0.694922in}{1.258317in}}%
\pgfpathcurveto{\pgfqpoint{0.689098in}{1.264141in}}{\pgfqpoint{0.681198in}{1.267413in}}{\pgfqpoint{0.672962in}{1.267413in}}%
\pgfpathcurveto{\pgfqpoint{0.664726in}{1.267413in}}{\pgfqpoint{0.656826in}{1.264141in}}{\pgfqpoint{0.651002in}{1.258317in}}%
\pgfpathcurveto{\pgfqpoint{0.645178in}{1.252493in}}{\pgfqpoint{0.641905in}{1.244593in}}{\pgfqpoint{0.641905in}{1.236357in}}%
\pgfpathcurveto{\pgfqpoint{0.641905in}{1.228120in}}{\pgfqpoint{0.645178in}{1.220220in}}{\pgfqpoint{0.651002in}{1.214396in}}%
\pgfpathcurveto{\pgfqpoint{0.656826in}{1.208573in}}{\pgfqpoint{0.664726in}{1.205300in}}{\pgfqpoint{0.672962in}{1.205300in}}%
\pgfpathclose%
\pgfusepath{stroke,fill}%
\end{pgfscope}%
\begin{pgfscope}%
\pgfpathrectangle{\pgfqpoint{0.100000in}{0.220728in}}{\pgfqpoint{3.696000in}{3.696000in}}%
\pgfusepath{clip}%
\pgfsetbuttcap%
\pgfsetroundjoin%
\definecolor{currentfill}{rgb}{0.121569,0.466667,0.705882}%
\pgfsetfillcolor{currentfill}%
\pgfsetfillopacity{0.618405}%
\pgfsetlinewidth{1.003750pt}%
\definecolor{currentstroke}{rgb}{0.121569,0.466667,0.705882}%
\pgfsetstrokecolor{currentstroke}%
\pgfsetstrokeopacity{0.618405}%
\pgfsetdash{}{0pt}%
\pgfpathmoveto{\pgfqpoint{0.672832in}{1.205039in}}%
\pgfpathcurveto{\pgfqpoint{0.681068in}{1.205039in}}{\pgfqpoint{0.688968in}{1.208311in}}{\pgfqpoint{0.694792in}{1.214135in}}%
\pgfpathcurveto{\pgfqpoint{0.700616in}{1.219959in}}{\pgfqpoint{0.703888in}{1.227859in}}{\pgfqpoint{0.703888in}{1.236095in}}%
\pgfpathcurveto{\pgfqpoint{0.703888in}{1.244331in}}{\pgfqpoint{0.700616in}{1.252231in}}{\pgfqpoint{0.694792in}{1.258055in}}%
\pgfpathcurveto{\pgfqpoint{0.688968in}{1.263879in}}{\pgfqpoint{0.681068in}{1.267152in}}{\pgfqpoint{0.672832in}{1.267152in}}%
\pgfpathcurveto{\pgfqpoint{0.664595in}{1.267152in}}{\pgfqpoint{0.656695in}{1.263879in}}{\pgfqpoint{0.650871in}{1.258055in}}%
\pgfpathcurveto{\pgfqpoint{0.645047in}{1.252231in}}{\pgfqpoint{0.641775in}{1.244331in}}{\pgfqpoint{0.641775in}{1.236095in}}%
\pgfpathcurveto{\pgfqpoint{0.641775in}{1.227859in}}{\pgfqpoint{0.645047in}{1.219959in}}{\pgfqpoint{0.650871in}{1.214135in}}%
\pgfpathcurveto{\pgfqpoint{0.656695in}{1.208311in}}{\pgfqpoint{0.664595in}{1.205039in}}{\pgfqpoint{0.672832in}{1.205039in}}%
\pgfpathclose%
\pgfusepath{stroke,fill}%
\end{pgfscope}%
\begin{pgfscope}%
\pgfpathrectangle{\pgfqpoint{0.100000in}{0.220728in}}{\pgfqpoint{3.696000in}{3.696000in}}%
\pgfusepath{clip}%
\pgfsetbuttcap%
\pgfsetroundjoin%
\definecolor{currentfill}{rgb}{0.121569,0.466667,0.705882}%
\pgfsetfillcolor{currentfill}%
\pgfsetfillopacity{0.618465}%
\pgfsetlinewidth{1.003750pt}%
\definecolor{currentstroke}{rgb}{0.121569,0.466667,0.705882}%
\pgfsetstrokecolor{currentstroke}%
\pgfsetstrokeopacity{0.618465}%
\pgfsetdash{}{0pt}%
\pgfpathmoveto{\pgfqpoint{0.672557in}{1.204574in}}%
\pgfpathcurveto{\pgfqpoint{0.680793in}{1.204574in}}{\pgfqpoint{0.688693in}{1.207846in}}{\pgfqpoint{0.694517in}{1.213670in}}%
\pgfpathcurveto{\pgfqpoint{0.700341in}{1.219494in}}{\pgfqpoint{0.703613in}{1.227394in}}{\pgfqpoint{0.703613in}{1.235630in}}%
\pgfpathcurveto{\pgfqpoint{0.703613in}{1.243866in}}{\pgfqpoint{0.700341in}{1.251767in}}{\pgfqpoint{0.694517in}{1.257590in}}%
\pgfpathcurveto{\pgfqpoint{0.688693in}{1.263414in}}{\pgfqpoint{0.680793in}{1.266687in}}{\pgfqpoint{0.672557in}{1.266687in}}%
\pgfpathcurveto{\pgfqpoint{0.664321in}{1.266687in}}{\pgfqpoint{0.656421in}{1.263414in}}{\pgfqpoint{0.650597in}{1.257590in}}%
\pgfpathcurveto{\pgfqpoint{0.644773in}{1.251767in}}{\pgfqpoint{0.641500in}{1.243866in}}{\pgfqpoint{0.641500in}{1.235630in}}%
\pgfpathcurveto{\pgfqpoint{0.641500in}{1.227394in}}{\pgfqpoint{0.644773in}{1.219494in}}{\pgfqpoint{0.650597in}{1.213670in}}%
\pgfpathcurveto{\pgfqpoint{0.656421in}{1.207846in}}{\pgfqpoint{0.664321in}{1.204574in}}{\pgfqpoint{0.672557in}{1.204574in}}%
\pgfpathclose%
\pgfusepath{stroke,fill}%
\end{pgfscope}%
\begin{pgfscope}%
\pgfpathrectangle{\pgfqpoint{0.100000in}{0.220728in}}{\pgfqpoint{3.696000in}{3.696000in}}%
\pgfusepath{clip}%
\pgfsetbuttcap%
\pgfsetroundjoin%
\definecolor{currentfill}{rgb}{0.121569,0.466667,0.705882}%
\pgfsetfillcolor{currentfill}%
\pgfsetfillopacity{0.618583}%
\pgfsetlinewidth{1.003750pt}%
\definecolor{currentstroke}{rgb}{0.121569,0.466667,0.705882}%
\pgfsetstrokecolor{currentstroke}%
\pgfsetstrokeopacity{0.618583}%
\pgfsetdash{}{0pt}%
\pgfpathmoveto{\pgfqpoint{0.672093in}{1.203721in}}%
\pgfpathcurveto{\pgfqpoint{0.680329in}{1.203721in}}{\pgfqpoint{0.688229in}{1.206994in}}{\pgfqpoint{0.694053in}{1.212818in}}%
\pgfpathcurveto{\pgfqpoint{0.699877in}{1.218642in}}{\pgfqpoint{0.703149in}{1.226542in}}{\pgfqpoint{0.703149in}{1.234778in}}%
\pgfpathcurveto{\pgfqpoint{0.703149in}{1.243014in}}{\pgfqpoint{0.699877in}{1.250914in}}{\pgfqpoint{0.694053in}{1.256738in}}%
\pgfpathcurveto{\pgfqpoint{0.688229in}{1.262562in}}{\pgfqpoint{0.680329in}{1.265834in}}{\pgfqpoint{0.672093in}{1.265834in}}%
\pgfpathcurveto{\pgfqpoint{0.663856in}{1.265834in}}{\pgfqpoint{0.655956in}{1.262562in}}{\pgfqpoint{0.650132in}{1.256738in}}%
\pgfpathcurveto{\pgfqpoint{0.644308in}{1.250914in}}{\pgfqpoint{0.641036in}{1.243014in}}{\pgfqpoint{0.641036in}{1.234778in}}%
\pgfpathcurveto{\pgfqpoint{0.641036in}{1.226542in}}{\pgfqpoint{0.644308in}{1.218642in}}{\pgfqpoint{0.650132in}{1.212818in}}%
\pgfpathcurveto{\pgfqpoint{0.655956in}{1.206994in}}{\pgfqpoint{0.663856in}{1.203721in}}{\pgfqpoint{0.672093in}{1.203721in}}%
\pgfpathclose%
\pgfusepath{stroke,fill}%
\end{pgfscope}%
\begin{pgfscope}%
\pgfpathrectangle{\pgfqpoint{0.100000in}{0.220728in}}{\pgfqpoint{3.696000in}{3.696000in}}%
\pgfusepath{clip}%
\pgfsetbuttcap%
\pgfsetroundjoin%
\definecolor{currentfill}{rgb}{0.121569,0.466667,0.705882}%
\pgfsetfillcolor{currentfill}%
\pgfsetfillopacity{0.618811}%
\pgfsetlinewidth{1.003750pt}%
\definecolor{currentstroke}{rgb}{0.121569,0.466667,0.705882}%
\pgfsetstrokecolor{currentstroke}%
\pgfsetstrokeopacity{0.618811}%
\pgfsetdash{}{0pt}%
\pgfpathmoveto{\pgfqpoint{0.671296in}{1.202161in}}%
\pgfpathcurveto{\pgfqpoint{0.679532in}{1.202161in}}{\pgfqpoint{0.687432in}{1.205434in}}{\pgfqpoint{0.693256in}{1.211258in}}%
\pgfpathcurveto{\pgfqpoint{0.699080in}{1.217082in}}{\pgfqpoint{0.702352in}{1.224982in}}{\pgfqpoint{0.702352in}{1.233218in}}%
\pgfpathcurveto{\pgfqpoint{0.702352in}{1.241454in}}{\pgfqpoint{0.699080in}{1.249354in}}{\pgfqpoint{0.693256in}{1.255178in}}%
\pgfpathcurveto{\pgfqpoint{0.687432in}{1.261002in}}{\pgfqpoint{0.679532in}{1.264274in}}{\pgfqpoint{0.671296in}{1.264274in}}%
\pgfpathcurveto{\pgfqpoint{0.663060in}{1.264274in}}{\pgfqpoint{0.655159in}{1.261002in}}{\pgfqpoint{0.649336in}{1.255178in}}%
\pgfpathcurveto{\pgfqpoint{0.643512in}{1.249354in}}{\pgfqpoint{0.640239in}{1.241454in}}{\pgfqpoint{0.640239in}{1.233218in}}%
\pgfpathcurveto{\pgfqpoint{0.640239in}{1.224982in}}{\pgfqpoint{0.643512in}{1.217082in}}{\pgfqpoint{0.649336in}{1.211258in}}%
\pgfpathcurveto{\pgfqpoint{0.655159in}{1.205434in}}{\pgfqpoint{0.663060in}{1.202161in}}{\pgfqpoint{0.671296in}{1.202161in}}%
\pgfpathclose%
\pgfusepath{stroke,fill}%
\end{pgfscope}%
\begin{pgfscope}%
\pgfpathrectangle{\pgfqpoint{0.100000in}{0.220728in}}{\pgfqpoint{3.696000in}{3.696000in}}%
\pgfusepath{clip}%
\pgfsetbuttcap%
\pgfsetroundjoin%
\definecolor{currentfill}{rgb}{0.121569,0.466667,0.705882}%
\pgfsetfillcolor{currentfill}%
\pgfsetfillopacity{0.618844}%
\pgfsetlinewidth{1.003750pt}%
\definecolor{currentstroke}{rgb}{0.121569,0.466667,0.705882}%
\pgfsetstrokecolor{currentstroke}%
\pgfsetstrokeopacity{0.618844}%
\pgfsetdash{}{0pt}%
\pgfpathmoveto{\pgfqpoint{0.685043in}{1.206123in}}%
\pgfpathcurveto{\pgfqpoint{0.693280in}{1.206123in}}{\pgfqpoint{0.701180in}{1.209395in}}{\pgfqpoint{0.707004in}{1.215219in}}%
\pgfpathcurveto{\pgfqpoint{0.712827in}{1.221043in}}{\pgfqpoint{0.716100in}{1.228943in}}{\pgfqpoint{0.716100in}{1.237179in}}%
\pgfpathcurveto{\pgfqpoint{0.716100in}{1.245416in}}{\pgfqpoint{0.712827in}{1.253316in}}{\pgfqpoint{0.707004in}{1.259140in}}%
\pgfpathcurveto{\pgfqpoint{0.701180in}{1.264964in}}{\pgfqpoint{0.693280in}{1.268236in}}{\pgfqpoint{0.685043in}{1.268236in}}%
\pgfpathcurveto{\pgfqpoint{0.676807in}{1.268236in}}{\pgfqpoint{0.668907in}{1.264964in}}{\pgfqpoint{0.663083in}{1.259140in}}%
\pgfpathcurveto{\pgfqpoint{0.657259in}{1.253316in}}{\pgfqpoint{0.653987in}{1.245416in}}{\pgfqpoint{0.653987in}{1.237179in}}%
\pgfpathcurveto{\pgfqpoint{0.653987in}{1.228943in}}{\pgfqpoint{0.657259in}{1.221043in}}{\pgfqpoint{0.663083in}{1.215219in}}%
\pgfpathcurveto{\pgfqpoint{0.668907in}{1.209395in}}{\pgfqpoint{0.676807in}{1.206123in}}{\pgfqpoint{0.685043in}{1.206123in}}%
\pgfpathclose%
\pgfusepath{stroke,fill}%
\end{pgfscope}%
\begin{pgfscope}%
\pgfpathrectangle{\pgfqpoint{0.100000in}{0.220728in}}{\pgfqpoint{3.696000in}{3.696000in}}%
\pgfusepath{clip}%
\pgfsetbuttcap%
\pgfsetroundjoin%
\definecolor{currentfill}{rgb}{0.121569,0.466667,0.705882}%
\pgfsetfillcolor{currentfill}%
\pgfsetfillopacity{0.619207}%
\pgfsetlinewidth{1.003750pt}%
\definecolor{currentstroke}{rgb}{0.121569,0.466667,0.705882}%
\pgfsetstrokecolor{currentstroke}%
\pgfsetstrokeopacity{0.619207}%
\pgfsetdash{}{0pt}%
\pgfpathmoveto{\pgfqpoint{0.669849in}{1.199241in}}%
\pgfpathcurveto{\pgfqpoint{0.678085in}{1.199241in}}{\pgfqpoint{0.685985in}{1.202513in}}{\pgfqpoint{0.691809in}{1.208337in}}%
\pgfpathcurveto{\pgfqpoint{0.697633in}{1.214161in}}{\pgfqpoint{0.700906in}{1.222061in}}{\pgfqpoint{0.700906in}{1.230297in}}%
\pgfpathcurveto{\pgfqpoint{0.700906in}{1.238533in}}{\pgfqpoint{0.697633in}{1.246433in}}{\pgfqpoint{0.691809in}{1.252257in}}%
\pgfpathcurveto{\pgfqpoint{0.685985in}{1.258081in}}{\pgfqpoint{0.678085in}{1.261354in}}{\pgfqpoint{0.669849in}{1.261354in}}%
\pgfpathcurveto{\pgfqpoint{0.661613in}{1.261354in}}{\pgfqpoint{0.653713in}{1.258081in}}{\pgfqpoint{0.647889in}{1.252257in}}%
\pgfpathcurveto{\pgfqpoint{0.642065in}{1.246433in}}{\pgfqpoint{0.638793in}{1.238533in}}{\pgfqpoint{0.638793in}{1.230297in}}%
\pgfpathcurveto{\pgfqpoint{0.638793in}{1.222061in}}{\pgfqpoint{0.642065in}{1.214161in}}{\pgfqpoint{0.647889in}{1.208337in}}%
\pgfpathcurveto{\pgfqpoint{0.653713in}{1.202513in}}{\pgfqpoint{0.661613in}{1.199241in}}{\pgfqpoint{0.669849in}{1.199241in}}%
\pgfpathclose%
\pgfusepath{stroke,fill}%
\end{pgfscope}%
\begin{pgfscope}%
\pgfpathrectangle{\pgfqpoint{0.100000in}{0.220728in}}{\pgfqpoint{3.696000in}{3.696000in}}%
\pgfusepath{clip}%
\pgfsetbuttcap%
\pgfsetroundjoin%
\definecolor{currentfill}{rgb}{0.121569,0.466667,0.705882}%
\pgfsetfillcolor{currentfill}%
\pgfsetfillopacity{0.619210}%
\pgfsetlinewidth{1.003750pt}%
\definecolor{currentstroke}{rgb}{0.121569,0.466667,0.705882}%
\pgfsetstrokecolor{currentstroke}%
\pgfsetstrokeopacity{0.619210}%
\pgfsetdash{}{0pt}%
\pgfpathmoveto{\pgfqpoint{0.669840in}{1.199219in}}%
\pgfpathcurveto{\pgfqpoint{0.678076in}{1.199219in}}{\pgfqpoint{0.685976in}{1.202491in}}{\pgfqpoint{0.691800in}{1.208315in}}%
\pgfpathcurveto{\pgfqpoint{0.697624in}{1.214139in}}{\pgfqpoint{0.700896in}{1.222039in}}{\pgfqpoint{0.700896in}{1.230275in}}%
\pgfpathcurveto{\pgfqpoint{0.700896in}{1.238512in}}{\pgfqpoint{0.697624in}{1.246412in}}{\pgfqpoint{0.691800in}{1.252236in}}%
\pgfpathcurveto{\pgfqpoint{0.685976in}{1.258060in}}{\pgfqpoint{0.678076in}{1.261332in}}{\pgfqpoint{0.669840in}{1.261332in}}%
\pgfpathcurveto{\pgfqpoint{0.661603in}{1.261332in}}{\pgfqpoint{0.653703in}{1.258060in}}{\pgfqpoint{0.647879in}{1.252236in}}%
\pgfpathcurveto{\pgfqpoint{0.642055in}{1.246412in}}{\pgfqpoint{0.638783in}{1.238512in}}{\pgfqpoint{0.638783in}{1.230275in}}%
\pgfpathcurveto{\pgfqpoint{0.638783in}{1.222039in}}{\pgfqpoint{0.642055in}{1.214139in}}{\pgfqpoint{0.647879in}{1.208315in}}%
\pgfpathcurveto{\pgfqpoint{0.653703in}{1.202491in}}{\pgfqpoint{0.661603in}{1.199219in}}{\pgfqpoint{0.669840in}{1.199219in}}%
\pgfpathclose%
\pgfusepath{stroke,fill}%
\end{pgfscope}%
\begin{pgfscope}%
\pgfpathrectangle{\pgfqpoint{0.100000in}{0.220728in}}{\pgfqpoint{3.696000in}{3.696000in}}%
\pgfusepath{clip}%
\pgfsetbuttcap%
\pgfsetroundjoin%
\definecolor{currentfill}{rgb}{0.121569,0.466667,0.705882}%
\pgfsetfillcolor{currentfill}%
\pgfsetfillopacity{0.619217}%
\pgfsetlinewidth{1.003750pt}%
\definecolor{currentstroke}{rgb}{0.121569,0.466667,0.705882}%
\pgfsetstrokecolor{currentstroke}%
\pgfsetstrokeopacity{0.619217}%
\pgfsetdash{}{0pt}%
\pgfpathmoveto{\pgfqpoint{0.669831in}{1.199178in}}%
\pgfpathcurveto{\pgfqpoint{0.678067in}{1.199178in}}{\pgfqpoint{0.685967in}{1.202451in}}{\pgfqpoint{0.691791in}{1.208275in}}%
\pgfpathcurveto{\pgfqpoint{0.697615in}{1.214098in}}{\pgfqpoint{0.700887in}{1.221999in}}{\pgfqpoint{0.700887in}{1.230235in}}%
\pgfpathcurveto{\pgfqpoint{0.700887in}{1.238471in}}{\pgfqpoint{0.697615in}{1.246371in}}{\pgfqpoint{0.691791in}{1.252195in}}%
\pgfpathcurveto{\pgfqpoint{0.685967in}{1.258019in}}{\pgfqpoint{0.678067in}{1.261291in}}{\pgfqpoint{0.669831in}{1.261291in}}%
\pgfpathcurveto{\pgfqpoint{0.661595in}{1.261291in}}{\pgfqpoint{0.653695in}{1.258019in}}{\pgfqpoint{0.647871in}{1.252195in}}%
\pgfpathcurveto{\pgfqpoint{0.642047in}{1.246371in}}{\pgfqpoint{0.638774in}{1.238471in}}{\pgfqpoint{0.638774in}{1.230235in}}%
\pgfpathcurveto{\pgfqpoint{0.638774in}{1.221999in}}{\pgfqpoint{0.642047in}{1.214098in}}{\pgfqpoint{0.647871in}{1.208275in}}%
\pgfpathcurveto{\pgfqpoint{0.653695in}{1.202451in}}{\pgfqpoint{0.661595in}{1.199178in}}{\pgfqpoint{0.669831in}{1.199178in}}%
\pgfpathclose%
\pgfusepath{stroke,fill}%
\end{pgfscope}%
\begin{pgfscope}%
\pgfpathrectangle{\pgfqpoint{0.100000in}{0.220728in}}{\pgfqpoint{3.696000in}{3.696000in}}%
\pgfusepath{clip}%
\pgfsetbuttcap%
\pgfsetroundjoin%
\definecolor{currentfill}{rgb}{0.121569,0.466667,0.705882}%
\pgfsetfillcolor{currentfill}%
\pgfsetfillopacity{0.619231}%
\pgfsetlinewidth{1.003750pt}%
\definecolor{currentstroke}{rgb}{0.121569,0.466667,0.705882}%
\pgfsetstrokecolor{currentstroke}%
\pgfsetstrokeopacity{0.619231}%
\pgfsetdash{}{0pt}%
\pgfpathmoveto{\pgfqpoint{0.669833in}{1.199113in}}%
\pgfpathcurveto{\pgfqpoint{0.678069in}{1.199113in}}{\pgfqpoint{0.685969in}{1.202386in}}{\pgfqpoint{0.691793in}{1.208210in}}%
\pgfpathcurveto{\pgfqpoint{0.697617in}{1.214033in}}{\pgfqpoint{0.700889in}{1.221933in}}{\pgfqpoint{0.700889in}{1.230170in}}%
\pgfpathcurveto{\pgfqpoint{0.700889in}{1.238406in}}{\pgfqpoint{0.697617in}{1.246306in}}{\pgfqpoint{0.691793in}{1.252130in}}%
\pgfpathcurveto{\pgfqpoint{0.685969in}{1.257954in}}{\pgfqpoint{0.678069in}{1.261226in}}{\pgfqpoint{0.669833in}{1.261226in}}%
\pgfpathcurveto{\pgfqpoint{0.661597in}{1.261226in}}{\pgfqpoint{0.653697in}{1.257954in}}{\pgfqpoint{0.647873in}{1.252130in}}%
\pgfpathcurveto{\pgfqpoint{0.642049in}{1.246306in}}{\pgfqpoint{0.638776in}{1.238406in}}{\pgfqpoint{0.638776in}{1.230170in}}%
\pgfpathcurveto{\pgfqpoint{0.638776in}{1.221933in}}{\pgfqpoint{0.642049in}{1.214033in}}{\pgfqpoint{0.647873in}{1.208210in}}%
\pgfpathcurveto{\pgfqpoint{0.653697in}{1.202386in}}{\pgfqpoint{0.661597in}{1.199113in}}{\pgfqpoint{0.669833in}{1.199113in}}%
\pgfpathclose%
\pgfusepath{stroke,fill}%
\end{pgfscope}%
\begin{pgfscope}%
\pgfpathrectangle{\pgfqpoint{0.100000in}{0.220728in}}{\pgfqpoint{3.696000in}{3.696000in}}%
\pgfusepath{clip}%
\pgfsetbuttcap%
\pgfsetroundjoin%
\definecolor{currentfill}{rgb}{0.121569,0.466667,0.705882}%
\pgfsetfillcolor{currentfill}%
\pgfsetfillopacity{0.619255}%
\pgfsetlinewidth{1.003750pt}%
\definecolor{currentstroke}{rgb}{0.121569,0.466667,0.705882}%
\pgfsetstrokecolor{currentstroke}%
\pgfsetstrokeopacity{0.619255}%
\pgfsetdash{}{0pt}%
\pgfpathmoveto{\pgfqpoint{0.669865in}{1.199002in}}%
\pgfpathcurveto{\pgfqpoint{0.678102in}{1.199002in}}{\pgfqpoint{0.686002in}{1.202274in}}{\pgfqpoint{0.691826in}{1.208098in}}%
\pgfpathcurveto{\pgfqpoint{0.697650in}{1.213922in}}{\pgfqpoint{0.700922in}{1.221822in}}{\pgfqpoint{0.700922in}{1.230058in}}%
\pgfpathcurveto{\pgfqpoint{0.700922in}{1.238294in}}{\pgfqpoint{0.697650in}{1.246194in}}{\pgfqpoint{0.691826in}{1.252018in}}%
\pgfpathcurveto{\pgfqpoint{0.686002in}{1.257842in}}{\pgfqpoint{0.678102in}{1.261115in}}{\pgfqpoint{0.669865in}{1.261115in}}%
\pgfpathcurveto{\pgfqpoint{0.661629in}{1.261115in}}{\pgfqpoint{0.653729in}{1.257842in}}{\pgfqpoint{0.647905in}{1.252018in}}%
\pgfpathcurveto{\pgfqpoint{0.642081in}{1.246194in}}{\pgfqpoint{0.638809in}{1.238294in}}{\pgfqpoint{0.638809in}{1.230058in}}%
\pgfpathcurveto{\pgfqpoint{0.638809in}{1.221822in}}{\pgfqpoint{0.642081in}{1.213922in}}{\pgfqpoint{0.647905in}{1.208098in}}%
\pgfpathcurveto{\pgfqpoint{0.653729in}{1.202274in}}{\pgfqpoint{0.661629in}{1.199002in}}{\pgfqpoint{0.669865in}{1.199002in}}%
\pgfpathclose%
\pgfusepath{stroke,fill}%
\end{pgfscope}%
\begin{pgfscope}%
\pgfpathrectangle{\pgfqpoint{0.100000in}{0.220728in}}{\pgfqpoint{3.696000in}{3.696000in}}%
\pgfusepath{clip}%
\pgfsetbuttcap%
\pgfsetroundjoin%
\definecolor{currentfill}{rgb}{0.121569,0.466667,0.705882}%
\pgfsetfillcolor{currentfill}%
\pgfsetfillopacity{0.619294}%
\pgfsetlinewidth{1.003750pt}%
\definecolor{currentstroke}{rgb}{0.121569,0.466667,0.705882}%
\pgfsetstrokecolor{currentstroke}%
\pgfsetstrokeopacity{0.619294}%
\pgfsetdash{}{0pt}%
\pgfpathmoveto{\pgfqpoint{0.669974in}{1.198832in}}%
\pgfpathcurveto{\pgfqpoint{0.678210in}{1.198832in}}{\pgfqpoint{0.686111in}{1.202104in}}{\pgfqpoint{0.691934in}{1.207928in}}%
\pgfpathcurveto{\pgfqpoint{0.697758in}{1.213752in}}{\pgfqpoint{0.701031in}{1.221652in}}{\pgfqpoint{0.701031in}{1.229888in}}%
\pgfpathcurveto{\pgfqpoint{0.701031in}{1.238125in}}{\pgfqpoint{0.697758in}{1.246025in}}{\pgfqpoint{0.691934in}{1.251848in}}%
\pgfpathcurveto{\pgfqpoint{0.686111in}{1.257672in}}{\pgfqpoint{0.678210in}{1.260945in}}{\pgfqpoint{0.669974in}{1.260945in}}%
\pgfpathcurveto{\pgfqpoint{0.661738in}{1.260945in}}{\pgfqpoint{0.653838in}{1.257672in}}{\pgfqpoint{0.648014in}{1.251848in}}%
\pgfpathcurveto{\pgfqpoint{0.642190in}{1.246025in}}{\pgfqpoint{0.638918in}{1.238125in}}{\pgfqpoint{0.638918in}{1.229888in}}%
\pgfpathcurveto{\pgfqpoint{0.638918in}{1.221652in}}{\pgfqpoint{0.642190in}{1.213752in}}{\pgfqpoint{0.648014in}{1.207928in}}%
\pgfpathcurveto{\pgfqpoint{0.653838in}{1.202104in}}{\pgfqpoint{0.661738in}{1.198832in}}{\pgfqpoint{0.669974in}{1.198832in}}%
\pgfpathclose%
\pgfusepath{stroke,fill}%
\end{pgfscope}%
\begin{pgfscope}%
\pgfpathrectangle{\pgfqpoint{0.100000in}{0.220728in}}{\pgfqpoint{3.696000in}{3.696000in}}%
\pgfusepath{clip}%
\pgfsetbuttcap%
\pgfsetroundjoin%
\definecolor{currentfill}{rgb}{0.121569,0.466667,0.705882}%
\pgfsetfillcolor{currentfill}%
\pgfsetfillopacity{0.619315}%
\pgfsetlinewidth{1.003750pt}%
\definecolor{currentstroke}{rgb}{0.121569,0.466667,0.705882}%
\pgfsetstrokecolor{currentstroke}%
\pgfsetstrokeopacity{0.619315}%
\pgfsetdash{}{0pt}%
\pgfpathmoveto{\pgfqpoint{0.682427in}{1.203196in}}%
\pgfpathcurveto{\pgfqpoint{0.690663in}{1.203196in}}{\pgfqpoint{0.698563in}{1.206469in}}{\pgfqpoint{0.704387in}{1.212293in}}%
\pgfpathcurveto{\pgfqpoint{0.710211in}{1.218117in}}{\pgfqpoint{0.713483in}{1.226017in}}{\pgfqpoint{0.713483in}{1.234253in}}%
\pgfpathcurveto{\pgfqpoint{0.713483in}{1.242489in}}{\pgfqpoint{0.710211in}{1.250389in}}{\pgfqpoint{0.704387in}{1.256213in}}%
\pgfpathcurveto{\pgfqpoint{0.698563in}{1.262037in}}{\pgfqpoint{0.690663in}{1.265309in}}{\pgfqpoint{0.682427in}{1.265309in}}%
\pgfpathcurveto{\pgfqpoint{0.674190in}{1.265309in}}{\pgfqpoint{0.666290in}{1.262037in}}{\pgfqpoint{0.660466in}{1.256213in}}%
\pgfpathcurveto{\pgfqpoint{0.654642in}{1.250389in}}{\pgfqpoint{0.651370in}{1.242489in}}{\pgfqpoint{0.651370in}{1.234253in}}%
\pgfpathcurveto{\pgfqpoint{0.651370in}{1.226017in}}{\pgfqpoint{0.654642in}{1.218117in}}{\pgfqpoint{0.660466in}{1.212293in}}%
\pgfpathcurveto{\pgfqpoint{0.666290in}{1.206469in}}{\pgfqpoint{0.674190in}{1.203196in}}{\pgfqpoint{0.682427in}{1.203196in}}%
\pgfpathclose%
\pgfusepath{stroke,fill}%
\end{pgfscope}%
\begin{pgfscope}%
\pgfpathrectangle{\pgfqpoint{0.100000in}{0.220728in}}{\pgfqpoint{3.696000in}{3.696000in}}%
\pgfusepath{clip}%
\pgfsetbuttcap%
\pgfsetroundjoin%
\definecolor{currentfill}{rgb}{0.121569,0.466667,0.705882}%
\pgfsetfillcolor{currentfill}%
\pgfsetfillopacity{0.619352}%
\pgfsetlinewidth{1.003750pt}%
\definecolor{currentstroke}{rgb}{0.121569,0.466667,0.705882}%
\pgfsetstrokecolor{currentstroke}%
\pgfsetstrokeopacity{0.619352}%
\pgfsetdash{}{0pt}%
\pgfpathmoveto{\pgfqpoint{0.670252in}{1.198599in}}%
\pgfpathcurveto{\pgfqpoint{0.678488in}{1.198599in}}{\pgfqpoint{0.686388in}{1.201872in}}{\pgfqpoint{0.692212in}{1.207696in}}%
\pgfpathcurveto{\pgfqpoint{0.698036in}{1.213520in}}{\pgfqpoint{0.701309in}{1.221420in}}{\pgfqpoint{0.701309in}{1.229656in}}%
\pgfpathcurveto{\pgfqpoint{0.701309in}{1.237892in}}{\pgfqpoint{0.698036in}{1.245792in}}{\pgfqpoint{0.692212in}{1.251616in}}%
\pgfpathcurveto{\pgfqpoint{0.686388in}{1.257440in}}{\pgfqpoint{0.678488in}{1.260712in}}{\pgfqpoint{0.670252in}{1.260712in}}%
\pgfpathcurveto{\pgfqpoint{0.662016in}{1.260712in}}{\pgfqpoint{0.654116in}{1.257440in}}{\pgfqpoint{0.648292in}{1.251616in}}%
\pgfpathcurveto{\pgfqpoint{0.642468in}{1.245792in}}{\pgfqpoint{0.639196in}{1.237892in}}{\pgfqpoint{0.639196in}{1.229656in}}%
\pgfpathcurveto{\pgfqpoint{0.639196in}{1.221420in}}{\pgfqpoint{0.642468in}{1.213520in}}{\pgfqpoint{0.648292in}{1.207696in}}%
\pgfpathcurveto{\pgfqpoint{0.654116in}{1.201872in}}{\pgfqpoint{0.662016in}{1.198599in}}{\pgfqpoint{0.670252in}{1.198599in}}%
\pgfpathclose%
\pgfusepath{stroke,fill}%
\end{pgfscope}%
\begin{pgfscope}%
\pgfpathrectangle{\pgfqpoint{0.100000in}{0.220728in}}{\pgfqpoint{3.696000in}{3.696000in}}%
\pgfusepath{clip}%
\pgfsetbuttcap%
\pgfsetroundjoin%
\definecolor{currentfill}{rgb}{0.121569,0.466667,0.705882}%
\pgfsetfillcolor{currentfill}%
\pgfsetfillopacity{0.619384}%
\pgfsetlinewidth{1.003750pt}%
\definecolor{currentstroke}{rgb}{0.121569,0.466667,0.705882}%
\pgfsetstrokecolor{currentstroke}%
\pgfsetstrokeopacity{0.619384}%
\pgfsetdash{}{0pt}%
\pgfpathmoveto{\pgfqpoint{3.013487in}{2.930249in}}%
\pgfpathcurveto{\pgfqpoint{3.021723in}{2.930249in}}{\pgfqpoint{3.029623in}{2.933522in}}{\pgfqpoint{3.035447in}{2.939346in}}%
\pgfpathcurveto{\pgfqpoint{3.041271in}{2.945169in}}{\pgfqpoint{3.044543in}{2.953069in}}{\pgfqpoint{3.044543in}{2.961306in}}%
\pgfpathcurveto{\pgfqpoint{3.044543in}{2.969542in}}{\pgfqpoint{3.041271in}{2.977442in}}{\pgfqpoint{3.035447in}{2.983266in}}%
\pgfpathcurveto{\pgfqpoint{3.029623in}{2.989090in}}{\pgfqpoint{3.021723in}{2.992362in}}{\pgfqpoint{3.013487in}{2.992362in}}%
\pgfpathcurveto{\pgfqpoint{3.005251in}{2.992362in}}{\pgfqpoint{2.997351in}{2.989090in}}{\pgfqpoint{2.991527in}{2.983266in}}%
\pgfpathcurveto{\pgfqpoint{2.985703in}{2.977442in}}{\pgfqpoint{2.982430in}{2.969542in}}{\pgfqpoint{2.982430in}{2.961306in}}%
\pgfpathcurveto{\pgfqpoint{2.982430in}{2.953069in}}{\pgfqpoint{2.985703in}{2.945169in}}{\pgfqpoint{2.991527in}{2.939346in}}%
\pgfpathcurveto{\pgfqpoint{2.997351in}{2.933522in}}{\pgfqpoint{3.005251in}{2.930249in}}{\pgfqpoint{3.013487in}{2.930249in}}%
\pgfpathclose%
\pgfusepath{stroke,fill}%
\end{pgfscope}%
\begin{pgfscope}%
\pgfpathrectangle{\pgfqpoint{0.100000in}{0.220728in}}{\pgfqpoint{3.696000in}{3.696000in}}%
\pgfusepath{clip}%
\pgfsetbuttcap%
\pgfsetroundjoin%
\definecolor{currentfill}{rgb}{0.121569,0.466667,0.705882}%
\pgfsetfillcolor{currentfill}%
\pgfsetfillopacity{0.619437}%
\pgfsetlinewidth{1.003750pt}%
\definecolor{currentstroke}{rgb}{0.121569,0.466667,0.705882}%
\pgfsetstrokecolor{currentstroke}%
\pgfsetstrokeopacity{0.619437}%
\pgfsetdash{}{0pt}%
\pgfpathmoveto{\pgfqpoint{0.670911in}{1.198627in}}%
\pgfpathcurveto{\pgfqpoint{0.679148in}{1.198627in}}{\pgfqpoint{0.687048in}{1.201899in}}{\pgfqpoint{0.692872in}{1.207723in}}%
\pgfpathcurveto{\pgfqpoint{0.698696in}{1.213547in}}{\pgfqpoint{0.701968in}{1.221447in}}{\pgfqpoint{0.701968in}{1.229683in}}%
\pgfpathcurveto{\pgfqpoint{0.701968in}{1.237919in}}{\pgfqpoint{0.698696in}{1.245819in}}{\pgfqpoint{0.692872in}{1.251643in}}%
\pgfpathcurveto{\pgfqpoint{0.687048in}{1.257467in}}{\pgfqpoint{0.679148in}{1.260740in}}{\pgfqpoint{0.670911in}{1.260740in}}%
\pgfpathcurveto{\pgfqpoint{0.662675in}{1.260740in}}{\pgfqpoint{0.654775in}{1.257467in}}{\pgfqpoint{0.648951in}{1.251643in}}%
\pgfpathcurveto{\pgfqpoint{0.643127in}{1.245819in}}{\pgfqpoint{0.639855in}{1.237919in}}{\pgfqpoint{0.639855in}{1.229683in}}%
\pgfpathcurveto{\pgfqpoint{0.639855in}{1.221447in}}{\pgfqpoint{0.643127in}{1.213547in}}{\pgfqpoint{0.648951in}{1.207723in}}%
\pgfpathcurveto{\pgfqpoint{0.654775in}{1.201899in}}{\pgfqpoint{0.662675in}{1.198627in}}{\pgfqpoint{0.670911in}{1.198627in}}%
\pgfpathclose%
\pgfusepath{stroke,fill}%
\end{pgfscope}%
\begin{pgfscope}%
\pgfpathrectangle{\pgfqpoint{0.100000in}{0.220728in}}{\pgfqpoint{3.696000in}{3.696000in}}%
\pgfusepath{clip}%
\pgfsetbuttcap%
\pgfsetroundjoin%
\definecolor{currentfill}{rgb}{0.121569,0.466667,0.705882}%
\pgfsetfillcolor{currentfill}%
\pgfsetfillopacity{0.619568}%
\pgfsetlinewidth{1.003750pt}%
\definecolor{currentstroke}{rgb}{0.121569,0.466667,0.705882}%
\pgfsetstrokecolor{currentstroke}%
\pgfsetstrokeopacity{0.619568}%
\pgfsetdash{}{0pt}%
\pgfpathmoveto{\pgfqpoint{0.680991in}{1.201549in}}%
\pgfpathcurveto{\pgfqpoint{0.689227in}{1.201549in}}{\pgfqpoint{0.697127in}{1.204821in}}{\pgfqpoint{0.702951in}{1.210645in}}%
\pgfpathcurveto{\pgfqpoint{0.708775in}{1.216469in}}{\pgfqpoint{0.712047in}{1.224369in}}{\pgfqpoint{0.712047in}{1.232605in}}%
\pgfpathcurveto{\pgfqpoint{0.712047in}{1.240842in}}{\pgfqpoint{0.708775in}{1.248742in}}{\pgfqpoint{0.702951in}{1.254566in}}%
\pgfpathcurveto{\pgfqpoint{0.697127in}{1.260389in}}{\pgfqpoint{0.689227in}{1.263662in}}{\pgfqpoint{0.680991in}{1.263662in}}%
\pgfpathcurveto{\pgfqpoint{0.672754in}{1.263662in}}{\pgfqpoint{0.664854in}{1.260389in}}{\pgfqpoint{0.659030in}{1.254566in}}%
\pgfpathcurveto{\pgfqpoint{0.653206in}{1.248742in}}{\pgfqpoint{0.649934in}{1.240842in}}{\pgfqpoint{0.649934in}{1.232605in}}%
\pgfpathcurveto{\pgfqpoint{0.649934in}{1.224369in}}{\pgfqpoint{0.653206in}{1.216469in}}{\pgfqpoint{0.659030in}{1.210645in}}%
\pgfpathcurveto{\pgfqpoint{0.664854in}{1.204821in}}{\pgfqpoint{0.672754in}{1.201549in}}{\pgfqpoint{0.680991in}{1.201549in}}%
\pgfpathclose%
\pgfusepath{stroke,fill}%
\end{pgfscope}%
\begin{pgfscope}%
\pgfpathrectangle{\pgfqpoint{0.100000in}{0.220728in}}{\pgfqpoint{3.696000in}{3.696000in}}%
\pgfusepath{clip}%
\pgfsetbuttcap%
\pgfsetroundjoin%
\definecolor{currentfill}{rgb}{0.121569,0.466667,0.705882}%
\pgfsetfillcolor{currentfill}%
\pgfsetfillopacity{0.619570}%
\pgfsetlinewidth{1.003750pt}%
\definecolor{currentstroke}{rgb}{0.121569,0.466667,0.705882}%
\pgfsetstrokecolor{currentstroke}%
\pgfsetstrokeopacity{0.619570}%
\pgfsetdash{}{0pt}%
\pgfpathmoveto{\pgfqpoint{0.672084in}{1.198409in}}%
\pgfpathcurveto{\pgfqpoint{0.680320in}{1.198409in}}{\pgfqpoint{0.688220in}{1.201681in}}{\pgfqpoint{0.694044in}{1.207505in}}%
\pgfpathcurveto{\pgfqpoint{0.699868in}{1.213329in}}{\pgfqpoint{0.703140in}{1.221229in}}{\pgfqpoint{0.703140in}{1.229466in}}%
\pgfpathcurveto{\pgfqpoint{0.703140in}{1.237702in}}{\pgfqpoint{0.699868in}{1.245602in}}{\pgfqpoint{0.694044in}{1.251426in}}%
\pgfpathcurveto{\pgfqpoint{0.688220in}{1.257250in}}{\pgfqpoint{0.680320in}{1.260522in}}{\pgfqpoint{0.672084in}{1.260522in}}%
\pgfpathcurveto{\pgfqpoint{0.663848in}{1.260522in}}{\pgfqpoint{0.655947in}{1.257250in}}{\pgfqpoint{0.650124in}{1.251426in}}%
\pgfpathcurveto{\pgfqpoint{0.644300in}{1.245602in}}{\pgfqpoint{0.641027in}{1.237702in}}{\pgfqpoint{0.641027in}{1.229466in}}%
\pgfpathcurveto{\pgfqpoint{0.641027in}{1.221229in}}{\pgfqpoint{0.644300in}{1.213329in}}{\pgfqpoint{0.650124in}{1.207505in}}%
\pgfpathcurveto{\pgfqpoint{0.655947in}{1.201681in}}{\pgfqpoint{0.663848in}{1.198409in}}{\pgfqpoint{0.672084in}{1.198409in}}%
\pgfpathclose%
\pgfusepath{stroke,fill}%
\end{pgfscope}%
\begin{pgfscope}%
\pgfpathrectangle{\pgfqpoint{0.100000in}{0.220728in}}{\pgfqpoint{3.696000in}{3.696000in}}%
\pgfusepath{clip}%
\pgfsetbuttcap%
\pgfsetroundjoin%
\definecolor{currentfill}{rgb}{0.121569,0.466667,0.705882}%
\pgfsetfillcolor{currentfill}%
\pgfsetfillopacity{0.619704}%
\pgfsetlinewidth{1.003750pt}%
\definecolor{currentstroke}{rgb}{0.121569,0.466667,0.705882}%
\pgfsetstrokecolor{currentstroke}%
\pgfsetstrokeopacity{0.619704}%
\pgfsetdash{}{0pt}%
\pgfpathmoveto{\pgfqpoint{0.680173in}{1.200692in}}%
\pgfpathcurveto{\pgfqpoint{0.688409in}{1.200692in}}{\pgfqpoint{0.696309in}{1.203964in}}{\pgfqpoint{0.702133in}{1.209788in}}%
\pgfpathcurveto{\pgfqpoint{0.707957in}{1.215612in}}{\pgfqpoint{0.711229in}{1.223512in}}{\pgfqpoint{0.711229in}{1.231748in}}%
\pgfpathcurveto{\pgfqpoint{0.711229in}{1.239985in}}{\pgfqpoint{0.707957in}{1.247885in}}{\pgfqpoint{0.702133in}{1.253709in}}%
\pgfpathcurveto{\pgfqpoint{0.696309in}{1.259533in}}{\pgfqpoint{0.688409in}{1.262805in}}{\pgfqpoint{0.680173in}{1.262805in}}%
\pgfpathcurveto{\pgfqpoint{0.671936in}{1.262805in}}{\pgfqpoint{0.664036in}{1.259533in}}{\pgfqpoint{0.658212in}{1.253709in}}%
\pgfpathcurveto{\pgfqpoint{0.652388in}{1.247885in}}{\pgfqpoint{0.649116in}{1.239985in}}{\pgfqpoint{0.649116in}{1.231748in}}%
\pgfpathcurveto{\pgfqpoint{0.649116in}{1.223512in}}{\pgfqpoint{0.652388in}{1.215612in}}{\pgfqpoint{0.658212in}{1.209788in}}%
\pgfpathcurveto{\pgfqpoint{0.664036in}{1.203964in}}{\pgfqpoint{0.671936in}{1.200692in}}{\pgfqpoint{0.680173in}{1.200692in}}%
\pgfpathclose%
\pgfusepath{stroke,fill}%
\end{pgfscope}%
\begin{pgfscope}%
\pgfpathrectangle{\pgfqpoint{0.100000in}{0.220728in}}{\pgfqpoint{3.696000in}{3.696000in}}%
\pgfusepath{clip}%
\pgfsetbuttcap%
\pgfsetroundjoin%
\definecolor{currentfill}{rgb}{0.121569,0.466667,0.705882}%
\pgfsetfillcolor{currentfill}%
\pgfsetfillopacity{0.619760}%
\pgfsetlinewidth{1.003750pt}%
\definecolor{currentstroke}{rgb}{0.121569,0.466667,0.705882}%
\pgfsetstrokecolor{currentstroke}%
\pgfsetstrokeopacity{0.619760}%
\pgfsetdash{}{0pt}%
\pgfpathmoveto{\pgfqpoint{0.674287in}{1.198128in}}%
\pgfpathcurveto{\pgfqpoint{0.682523in}{1.198128in}}{\pgfqpoint{0.690423in}{1.201400in}}{\pgfqpoint{0.696247in}{1.207224in}}%
\pgfpathcurveto{\pgfqpoint{0.702071in}{1.213048in}}{\pgfqpoint{0.705344in}{1.220948in}}{\pgfqpoint{0.705344in}{1.229184in}}%
\pgfpathcurveto{\pgfqpoint{0.705344in}{1.237421in}}{\pgfqpoint{0.702071in}{1.245321in}}{\pgfqpoint{0.696247in}{1.251145in}}%
\pgfpathcurveto{\pgfqpoint{0.690423in}{1.256969in}}{\pgfqpoint{0.682523in}{1.260241in}}{\pgfqpoint{0.674287in}{1.260241in}}%
\pgfpathcurveto{\pgfqpoint{0.666051in}{1.260241in}}{\pgfqpoint{0.658151in}{1.256969in}}{\pgfqpoint{0.652327in}{1.251145in}}%
\pgfpathcurveto{\pgfqpoint{0.646503in}{1.245321in}}{\pgfqpoint{0.643231in}{1.237421in}}{\pgfqpoint{0.643231in}{1.229184in}}%
\pgfpathcurveto{\pgfqpoint{0.643231in}{1.220948in}}{\pgfqpoint{0.646503in}{1.213048in}}{\pgfqpoint{0.652327in}{1.207224in}}%
\pgfpathcurveto{\pgfqpoint{0.658151in}{1.201400in}}{\pgfqpoint{0.666051in}{1.198128in}}{\pgfqpoint{0.674287in}{1.198128in}}%
\pgfpathclose%
\pgfusepath{stroke,fill}%
\end{pgfscope}%
\begin{pgfscope}%
\pgfpathrectangle{\pgfqpoint{0.100000in}{0.220728in}}{\pgfqpoint{3.696000in}{3.696000in}}%
\pgfusepath{clip}%
\pgfsetbuttcap%
\pgfsetroundjoin%
\definecolor{currentfill}{rgb}{0.121569,0.466667,0.705882}%
\pgfsetfillcolor{currentfill}%
\pgfsetfillopacity{0.619767}%
\pgfsetlinewidth{1.003750pt}%
\definecolor{currentstroke}{rgb}{0.121569,0.466667,0.705882}%
\pgfsetstrokecolor{currentstroke}%
\pgfsetstrokeopacity{0.619767}%
\pgfsetdash{}{0pt}%
\pgfpathmoveto{\pgfqpoint{0.674400in}{1.198139in}}%
\pgfpathcurveto{\pgfqpoint{0.682636in}{1.198139in}}{\pgfqpoint{0.690536in}{1.201411in}}{\pgfqpoint{0.696360in}{1.207235in}}%
\pgfpathcurveto{\pgfqpoint{0.702184in}{1.213059in}}{\pgfqpoint{0.705457in}{1.220959in}}{\pgfqpoint{0.705457in}{1.229195in}}%
\pgfpathcurveto{\pgfqpoint{0.705457in}{1.237431in}}{\pgfqpoint{0.702184in}{1.245332in}}{\pgfqpoint{0.696360in}{1.251155in}}%
\pgfpathcurveto{\pgfqpoint{0.690536in}{1.256979in}}{\pgfqpoint{0.682636in}{1.260252in}}{\pgfqpoint{0.674400in}{1.260252in}}%
\pgfpathcurveto{\pgfqpoint{0.666164in}{1.260252in}}{\pgfqpoint{0.658264in}{1.256979in}}{\pgfqpoint{0.652440in}{1.251155in}}%
\pgfpathcurveto{\pgfqpoint{0.646616in}{1.245332in}}{\pgfqpoint{0.643344in}{1.237431in}}{\pgfqpoint{0.643344in}{1.229195in}}%
\pgfpathcurveto{\pgfqpoint{0.643344in}{1.220959in}}{\pgfqpoint{0.646616in}{1.213059in}}{\pgfqpoint{0.652440in}{1.207235in}}%
\pgfpathcurveto{\pgfqpoint{0.658264in}{1.201411in}}{\pgfqpoint{0.666164in}{1.198139in}}{\pgfqpoint{0.674400in}{1.198139in}}%
\pgfpathclose%
\pgfusepath{stroke,fill}%
\end{pgfscope}%
\begin{pgfscope}%
\pgfpathrectangle{\pgfqpoint{0.100000in}{0.220728in}}{\pgfqpoint{3.696000in}{3.696000in}}%
\pgfusepath{clip}%
\pgfsetbuttcap%
\pgfsetroundjoin%
\definecolor{currentfill}{rgb}{0.121569,0.466667,0.705882}%
\pgfsetfillcolor{currentfill}%
\pgfsetfillopacity{0.619777}%
\pgfsetlinewidth{1.003750pt}%
\definecolor{currentstroke}{rgb}{0.121569,0.466667,0.705882}%
\pgfsetstrokecolor{currentstroke}%
\pgfsetstrokeopacity{0.619777}%
\pgfsetdash{}{0pt}%
\pgfpathmoveto{\pgfqpoint{0.674607in}{1.198183in}}%
\pgfpathcurveto{\pgfqpoint{0.682844in}{1.198183in}}{\pgfqpoint{0.690744in}{1.201455in}}{\pgfqpoint{0.696567in}{1.207279in}}%
\pgfpathcurveto{\pgfqpoint{0.702391in}{1.213103in}}{\pgfqpoint{0.705664in}{1.221003in}}{\pgfqpoint{0.705664in}{1.229240in}}%
\pgfpathcurveto{\pgfqpoint{0.705664in}{1.237476in}}{\pgfqpoint{0.702391in}{1.245376in}}{\pgfqpoint{0.696567in}{1.251200in}}%
\pgfpathcurveto{\pgfqpoint{0.690744in}{1.257024in}}{\pgfqpoint{0.682844in}{1.260296in}}{\pgfqpoint{0.674607in}{1.260296in}}%
\pgfpathcurveto{\pgfqpoint{0.666371in}{1.260296in}}{\pgfqpoint{0.658471in}{1.257024in}}{\pgfqpoint{0.652647in}{1.251200in}}%
\pgfpathcurveto{\pgfqpoint{0.646823in}{1.245376in}}{\pgfqpoint{0.643551in}{1.237476in}}{\pgfqpoint{0.643551in}{1.229240in}}%
\pgfpathcurveto{\pgfqpoint{0.643551in}{1.221003in}}{\pgfqpoint{0.646823in}{1.213103in}}{\pgfqpoint{0.652647in}{1.207279in}}%
\pgfpathcurveto{\pgfqpoint{0.658471in}{1.201455in}}{\pgfqpoint{0.666371in}{1.198183in}}{\pgfqpoint{0.674607in}{1.198183in}}%
\pgfpathclose%
\pgfusepath{stroke,fill}%
\end{pgfscope}%
\begin{pgfscope}%
\pgfpathrectangle{\pgfqpoint{0.100000in}{0.220728in}}{\pgfqpoint{3.696000in}{3.696000in}}%
\pgfusepath{clip}%
\pgfsetbuttcap%
\pgfsetroundjoin%
\definecolor{currentfill}{rgb}{0.121569,0.466667,0.705882}%
\pgfsetfillcolor{currentfill}%
\pgfsetfillopacity{0.619780}%
\pgfsetlinewidth{1.003750pt}%
\definecolor{currentstroke}{rgb}{0.121569,0.466667,0.705882}%
\pgfsetstrokecolor{currentstroke}%
\pgfsetstrokeopacity{0.619780}%
\pgfsetdash{}{0pt}%
\pgfpathmoveto{\pgfqpoint{0.679736in}{1.200201in}}%
\pgfpathcurveto{\pgfqpoint{0.687972in}{1.200201in}}{\pgfqpoint{0.695872in}{1.203473in}}{\pgfqpoint{0.701696in}{1.209297in}}%
\pgfpathcurveto{\pgfqpoint{0.707520in}{1.215121in}}{\pgfqpoint{0.710792in}{1.223021in}}{\pgfqpoint{0.710792in}{1.231258in}}%
\pgfpathcurveto{\pgfqpoint{0.710792in}{1.239494in}}{\pgfqpoint{0.707520in}{1.247394in}}{\pgfqpoint{0.701696in}{1.253218in}}%
\pgfpathcurveto{\pgfqpoint{0.695872in}{1.259042in}}{\pgfqpoint{0.687972in}{1.262314in}}{\pgfqpoint{0.679736in}{1.262314in}}%
\pgfpathcurveto{\pgfqpoint{0.671499in}{1.262314in}}{\pgfqpoint{0.663599in}{1.259042in}}{\pgfqpoint{0.657775in}{1.253218in}}%
\pgfpathcurveto{\pgfqpoint{0.651951in}{1.247394in}}{\pgfqpoint{0.648679in}{1.239494in}}{\pgfqpoint{0.648679in}{1.231258in}}%
\pgfpathcurveto{\pgfqpoint{0.648679in}{1.223021in}}{\pgfqpoint{0.651951in}{1.215121in}}{\pgfqpoint{0.657775in}{1.209297in}}%
\pgfpathcurveto{\pgfqpoint{0.663599in}{1.203473in}}{\pgfqpoint{0.671499in}{1.200201in}}{\pgfqpoint{0.679736in}{1.200201in}}%
\pgfpathclose%
\pgfusepath{stroke,fill}%
\end{pgfscope}%
\begin{pgfscope}%
\pgfpathrectangle{\pgfqpoint{0.100000in}{0.220728in}}{\pgfqpoint{3.696000in}{3.696000in}}%
\pgfusepath{clip}%
\pgfsetbuttcap%
\pgfsetroundjoin%
\definecolor{currentfill}{rgb}{0.121569,0.466667,0.705882}%
\pgfsetfillcolor{currentfill}%
\pgfsetfillopacity{0.619789}%
\pgfsetlinewidth{1.003750pt}%
\definecolor{currentstroke}{rgb}{0.121569,0.466667,0.705882}%
\pgfsetstrokecolor{currentstroke}%
\pgfsetstrokeopacity{0.619789}%
\pgfsetdash{}{0pt}%
\pgfpathmoveto{\pgfqpoint{0.674986in}{1.198287in}}%
\pgfpathcurveto{\pgfqpoint{0.683222in}{1.198287in}}{\pgfqpoint{0.691122in}{1.201560in}}{\pgfqpoint{0.696946in}{1.207384in}}%
\pgfpathcurveto{\pgfqpoint{0.702770in}{1.213208in}}{\pgfqpoint{0.706042in}{1.221108in}}{\pgfqpoint{0.706042in}{1.229344in}}%
\pgfpathcurveto{\pgfqpoint{0.706042in}{1.237580in}}{\pgfqpoint{0.702770in}{1.245480in}}{\pgfqpoint{0.696946in}{1.251304in}}%
\pgfpathcurveto{\pgfqpoint{0.691122in}{1.257128in}}{\pgfqpoint{0.683222in}{1.260400in}}{\pgfqpoint{0.674986in}{1.260400in}}%
\pgfpathcurveto{\pgfqpoint{0.666750in}{1.260400in}}{\pgfqpoint{0.658850in}{1.257128in}}{\pgfqpoint{0.653026in}{1.251304in}}%
\pgfpathcurveto{\pgfqpoint{0.647202in}{1.245480in}}{\pgfqpoint{0.643929in}{1.237580in}}{\pgfqpoint{0.643929in}{1.229344in}}%
\pgfpathcurveto{\pgfqpoint{0.643929in}{1.221108in}}{\pgfqpoint{0.647202in}{1.213208in}}{\pgfqpoint{0.653026in}{1.207384in}}%
\pgfpathcurveto{\pgfqpoint{0.658850in}{1.201560in}}{\pgfqpoint{0.666750in}{1.198287in}}{\pgfqpoint{0.674986in}{1.198287in}}%
\pgfpathclose%
\pgfusepath{stroke,fill}%
\end{pgfscope}%
\begin{pgfscope}%
\pgfpathrectangle{\pgfqpoint{0.100000in}{0.220728in}}{\pgfqpoint{3.696000in}{3.696000in}}%
\pgfusepath{clip}%
\pgfsetbuttcap%
\pgfsetroundjoin%
\definecolor{currentfill}{rgb}{0.121569,0.466667,0.705882}%
\pgfsetfillcolor{currentfill}%
\pgfsetfillopacity{0.619815}%
\pgfsetlinewidth{1.003750pt}%
\definecolor{currentstroke}{rgb}{0.121569,0.466667,0.705882}%
\pgfsetstrokecolor{currentstroke}%
\pgfsetstrokeopacity{0.619815}%
\pgfsetdash{}{0pt}%
\pgfpathmoveto{\pgfqpoint{0.675675in}{1.198466in}}%
\pgfpathcurveto{\pgfqpoint{0.683911in}{1.198466in}}{\pgfqpoint{0.691811in}{1.201738in}}{\pgfqpoint{0.697635in}{1.207562in}}%
\pgfpathcurveto{\pgfqpoint{0.703459in}{1.213386in}}{\pgfqpoint{0.706731in}{1.221286in}}{\pgfqpoint{0.706731in}{1.229523in}}%
\pgfpathcurveto{\pgfqpoint{0.706731in}{1.237759in}}{\pgfqpoint{0.703459in}{1.245659in}}{\pgfqpoint{0.697635in}{1.251483in}}%
\pgfpathcurveto{\pgfqpoint{0.691811in}{1.257307in}}{\pgfqpoint{0.683911in}{1.260579in}}{\pgfqpoint{0.675675in}{1.260579in}}%
\pgfpathcurveto{\pgfqpoint{0.667438in}{1.260579in}}{\pgfqpoint{0.659538in}{1.257307in}}{\pgfqpoint{0.653714in}{1.251483in}}%
\pgfpathcurveto{\pgfqpoint{0.647890in}{1.245659in}}{\pgfqpoint{0.644618in}{1.237759in}}{\pgfqpoint{0.644618in}{1.229523in}}%
\pgfpathcurveto{\pgfqpoint{0.644618in}{1.221286in}}{\pgfqpoint{0.647890in}{1.213386in}}{\pgfqpoint{0.653714in}{1.207562in}}%
\pgfpathcurveto{\pgfqpoint{0.659538in}{1.201738in}}{\pgfqpoint{0.667438in}{1.198466in}}{\pgfqpoint{0.675675in}{1.198466in}}%
\pgfpathclose%
\pgfusepath{stroke,fill}%
\end{pgfscope}%
\begin{pgfscope}%
\pgfpathrectangle{\pgfqpoint{0.100000in}{0.220728in}}{\pgfqpoint{3.696000in}{3.696000in}}%
\pgfusepath{clip}%
\pgfsetbuttcap%
\pgfsetroundjoin%
\definecolor{currentfill}{rgb}{0.121569,0.466667,0.705882}%
\pgfsetfillcolor{currentfill}%
\pgfsetfillopacity{0.619821}%
\pgfsetlinewidth{1.003750pt}%
\definecolor{currentstroke}{rgb}{0.121569,0.466667,0.705882}%
\pgfsetstrokecolor{currentstroke}%
\pgfsetstrokeopacity{0.619821}%
\pgfsetdash{}{0pt}%
\pgfpathmoveto{\pgfqpoint{0.679489in}{1.199936in}}%
\pgfpathcurveto{\pgfqpoint{0.687726in}{1.199936in}}{\pgfqpoint{0.695626in}{1.203209in}}{\pgfqpoint{0.701450in}{1.209032in}}%
\pgfpathcurveto{\pgfqpoint{0.707274in}{1.214856in}}{\pgfqpoint{0.710546in}{1.222756in}}{\pgfqpoint{0.710546in}{1.230993in}}%
\pgfpathcurveto{\pgfqpoint{0.710546in}{1.239229in}}{\pgfqpoint{0.707274in}{1.247129in}}{\pgfqpoint{0.701450in}{1.252953in}}%
\pgfpathcurveto{\pgfqpoint{0.695626in}{1.258777in}}{\pgfqpoint{0.687726in}{1.262049in}}{\pgfqpoint{0.679489in}{1.262049in}}%
\pgfpathcurveto{\pgfqpoint{0.671253in}{1.262049in}}{\pgfqpoint{0.663353in}{1.258777in}}{\pgfqpoint{0.657529in}{1.252953in}}%
\pgfpathcurveto{\pgfqpoint{0.651705in}{1.247129in}}{\pgfqpoint{0.648433in}{1.239229in}}{\pgfqpoint{0.648433in}{1.230993in}}%
\pgfpathcurveto{\pgfqpoint{0.648433in}{1.222756in}}{\pgfqpoint{0.651705in}{1.214856in}}{\pgfqpoint{0.657529in}{1.209032in}}%
\pgfpathcurveto{\pgfqpoint{0.663353in}{1.203209in}}{\pgfqpoint{0.671253in}{1.199936in}}{\pgfqpoint{0.679489in}{1.199936in}}%
\pgfpathclose%
\pgfusepath{stroke,fill}%
\end{pgfscope}%
\begin{pgfscope}%
\pgfpathrectangle{\pgfqpoint{0.100000in}{0.220728in}}{\pgfqpoint{3.696000in}{3.696000in}}%
\pgfusepath{clip}%
\pgfsetbuttcap%
\pgfsetroundjoin%
\definecolor{currentfill}{rgb}{0.121569,0.466667,0.705882}%
\pgfsetfillcolor{currentfill}%
\pgfsetfillopacity{0.619845}%
\pgfsetlinewidth{1.003750pt}%
\definecolor{currentstroke}{rgb}{0.121569,0.466667,0.705882}%
\pgfsetstrokecolor{currentstroke}%
\pgfsetstrokeopacity{0.619845}%
\pgfsetdash{}{0pt}%
\pgfpathmoveto{\pgfqpoint{0.679356in}{1.199796in}}%
\pgfpathcurveto{\pgfqpoint{0.687592in}{1.199796in}}{\pgfqpoint{0.695492in}{1.203068in}}{\pgfqpoint{0.701316in}{1.208892in}}%
\pgfpathcurveto{\pgfqpoint{0.707140in}{1.214716in}}{\pgfqpoint{0.710412in}{1.222616in}}{\pgfqpoint{0.710412in}{1.230853in}}%
\pgfpathcurveto{\pgfqpoint{0.710412in}{1.239089in}}{\pgfqpoint{0.707140in}{1.246989in}}{\pgfqpoint{0.701316in}{1.252813in}}%
\pgfpathcurveto{\pgfqpoint{0.695492in}{1.258637in}}{\pgfqpoint{0.687592in}{1.261909in}}{\pgfqpoint{0.679356in}{1.261909in}}%
\pgfpathcurveto{\pgfqpoint{0.671119in}{1.261909in}}{\pgfqpoint{0.663219in}{1.258637in}}{\pgfqpoint{0.657395in}{1.252813in}}%
\pgfpathcurveto{\pgfqpoint{0.651571in}{1.246989in}}{\pgfqpoint{0.648299in}{1.239089in}}{\pgfqpoint{0.648299in}{1.230853in}}%
\pgfpathcurveto{\pgfqpoint{0.648299in}{1.222616in}}{\pgfqpoint{0.651571in}{1.214716in}}{\pgfqpoint{0.657395in}{1.208892in}}%
\pgfpathcurveto{\pgfqpoint{0.663219in}{1.203068in}}{\pgfqpoint{0.671119in}{1.199796in}}{\pgfqpoint{0.679356in}{1.199796in}}%
\pgfpathclose%
\pgfusepath{stroke,fill}%
\end{pgfscope}%
\begin{pgfscope}%
\pgfpathrectangle{\pgfqpoint{0.100000in}{0.220728in}}{\pgfqpoint{3.696000in}{3.696000in}}%
\pgfusepath{clip}%
\pgfsetbuttcap%
\pgfsetroundjoin%
\definecolor{currentfill}{rgb}{0.121569,0.466667,0.705882}%
\pgfsetfillcolor{currentfill}%
\pgfsetfillopacity{0.619855}%
\pgfsetlinewidth{1.003750pt}%
\definecolor{currentstroke}{rgb}{0.121569,0.466667,0.705882}%
\pgfsetstrokecolor{currentstroke}%
\pgfsetstrokeopacity{0.619855}%
\pgfsetdash{}{0pt}%
\pgfpathmoveto{\pgfqpoint{0.676925in}{1.198831in}}%
\pgfpathcurveto{\pgfqpoint{0.685162in}{1.198831in}}{\pgfqpoint{0.693062in}{1.202103in}}{\pgfqpoint{0.698886in}{1.207927in}}%
\pgfpathcurveto{\pgfqpoint{0.704710in}{1.213751in}}{\pgfqpoint{0.707982in}{1.221651in}}{\pgfqpoint{0.707982in}{1.229887in}}%
\pgfpathcurveto{\pgfqpoint{0.707982in}{1.238124in}}{\pgfqpoint{0.704710in}{1.246024in}}{\pgfqpoint{0.698886in}{1.251848in}}%
\pgfpathcurveto{\pgfqpoint{0.693062in}{1.257672in}}{\pgfqpoint{0.685162in}{1.260944in}}{\pgfqpoint{0.676925in}{1.260944in}}%
\pgfpathcurveto{\pgfqpoint{0.668689in}{1.260944in}}{\pgfqpoint{0.660789in}{1.257672in}}{\pgfqpoint{0.654965in}{1.251848in}}%
\pgfpathcurveto{\pgfqpoint{0.649141in}{1.246024in}}{\pgfqpoint{0.645869in}{1.238124in}}{\pgfqpoint{0.645869in}{1.229887in}}%
\pgfpathcurveto{\pgfqpoint{0.645869in}{1.221651in}}{\pgfqpoint{0.649141in}{1.213751in}}{\pgfqpoint{0.654965in}{1.207927in}}%
\pgfpathcurveto{\pgfqpoint{0.660789in}{1.202103in}}{\pgfqpoint{0.668689in}{1.198831in}}{\pgfqpoint{0.676925in}{1.198831in}}%
\pgfpathclose%
\pgfusepath{stroke,fill}%
\end{pgfscope}%
\begin{pgfscope}%
\pgfpathrectangle{\pgfqpoint{0.100000in}{0.220728in}}{\pgfqpoint{3.696000in}{3.696000in}}%
\pgfusepath{clip}%
\pgfsetbuttcap%
\pgfsetroundjoin%
\definecolor{currentfill}{rgb}{0.121569,0.466667,0.705882}%
\pgfsetfillcolor{currentfill}%
\pgfsetfillopacity{0.619860}%
\pgfsetlinewidth{1.003750pt}%
\definecolor{currentstroke}{rgb}{0.121569,0.466667,0.705882}%
\pgfsetstrokecolor{currentstroke}%
\pgfsetstrokeopacity{0.619860}%
\pgfsetdash{}{0pt}%
\pgfpathmoveto{\pgfqpoint{0.679285in}{1.199719in}}%
\pgfpathcurveto{\pgfqpoint{0.687521in}{1.199719in}}{\pgfqpoint{0.695421in}{1.202992in}}{\pgfqpoint{0.701245in}{1.208816in}}%
\pgfpathcurveto{\pgfqpoint{0.707069in}{1.214640in}}{\pgfqpoint{0.710341in}{1.222540in}}{\pgfqpoint{0.710341in}{1.230776in}}%
\pgfpathcurveto{\pgfqpoint{0.710341in}{1.239012in}}{\pgfqpoint{0.707069in}{1.246912in}}{\pgfqpoint{0.701245in}{1.252736in}}%
\pgfpathcurveto{\pgfqpoint{0.695421in}{1.258560in}}{\pgfqpoint{0.687521in}{1.261832in}}{\pgfqpoint{0.679285in}{1.261832in}}%
\pgfpathcurveto{\pgfqpoint{0.671048in}{1.261832in}}{\pgfqpoint{0.663148in}{1.258560in}}{\pgfqpoint{0.657324in}{1.252736in}}%
\pgfpathcurveto{\pgfqpoint{0.651500in}{1.246912in}}{\pgfqpoint{0.648228in}{1.239012in}}{\pgfqpoint{0.648228in}{1.230776in}}%
\pgfpathcurveto{\pgfqpoint{0.648228in}{1.222540in}}{\pgfqpoint{0.651500in}{1.214640in}}{\pgfqpoint{0.657324in}{1.208816in}}%
\pgfpathcurveto{\pgfqpoint{0.663148in}{1.202992in}}{\pgfqpoint{0.671048in}{1.199719in}}{\pgfqpoint{0.679285in}{1.199719in}}%
\pgfpathclose%
\pgfusepath{stroke,fill}%
\end{pgfscope}%
\begin{pgfscope}%
\pgfpathrectangle{\pgfqpoint{0.100000in}{0.220728in}}{\pgfqpoint{3.696000in}{3.696000in}}%
\pgfusepath{clip}%
\pgfsetbuttcap%
\pgfsetroundjoin%
\definecolor{currentfill}{rgb}{0.121569,0.466667,0.705882}%
\pgfsetfillcolor{currentfill}%
\pgfsetfillopacity{0.619868}%
\pgfsetlinewidth{1.003750pt}%
\definecolor{currentstroke}{rgb}{0.121569,0.466667,0.705882}%
\pgfsetstrokecolor{currentstroke}%
\pgfsetstrokeopacity{0.619868}%
\pgfsetdash{}{0pt}%
\pgfpathmoveto{\pgfqpoint{0.679246in}{1.199675in}}%
\pgfpathcurveto{\pgfqpoint{0.687482in}{1.199675in}}{\pgfqpoint{0.695382in}{1.202947in}}{\pgfqpoint{0.701206in}{1.208771in}}%
\pgfpathcurveto{\pgfqpoint{0.707030in}{1.214595in}}{\pgfqpoint{0.710303in}{1.222495in}}{\pgfqpoint{0.710303in}{1.230732in}}%
\pgfpathcurveto{\pgfqpoint{0.710303in}{1.238968in}}{\pgfqpoint{0.707030in}{1.246868in}}{\pgfqpoint{0.701206in}{1.252692in}}%
\pgfpathcurveto{\pgfqpoint{0.695382in}{1.258516in}}{\pgfqpoint{0.687482in}{1.261788in}}{\pgfqpoint{0.679246in}{1.261788in}}%
\pgfpathcurveto{\pgfqpoint{0.671010in}{1.261788in}}{\pgfqpoint{0.663110in}{1.258516in}}{\pgfqpoint{0.657286in}{1.252692in}}%
\pgfpathcurveto{\pgfqpoint{0.651462in}{1.246868in}}{\pgfqpoint{0.648190in}{1.238968in}}{\pgfqpoint{0.648190in}{1.230732in}}%
\pgfpathcurveto{\pgfqpoint{0.648190in}{1.222495in}}{\pgfqpoint{0.651462in}{1.214595in}}{\pgfqpoint{0.657286in}{1.208771in}}%
\pgfpathcurveto{\pgfqpoint{0.663110in}{1.202947in}}{\pgfqpoint{0.671010in}{1.199675in}}{\pgfqpoint{0.679246in}{1.199675in}}%
\pgfpathclose%
\pgfusepath{stroke,fill}%
\end{pgfscope}%
\begin{pgfscope}%
\pgfpathrectangle{\pgfqpoint{0.100000in}{0.220728in}}{\pgfqpoint{3.696000in}{3.696000in}}%
\pgfusepath{clip}%
\pgfsetbuttcap%
\pgfsetroundjoin%
\definecolor{currentfill}{rgb}{0.121569,0.466667,0.705882}%
\pgfsetfillcolor{currentfill}%
\pgfsetfillopacity{0.619872}%
\pgfsetlinewidth{1.003750pt}%
\definecolor{currentstroke}{rgb}{0.121569,0.466667,0.705882}%
\pgfsetstrokecolor{currentstroke}%
\pgfsetstrokeopacity{0.619872}%
\pgfsetdash{}{0pt}%
\pgfpathmoveto{\pgfqpoint{0.679225in}{1.199650in}}%
\pgfpathcurveto{\pgfqpoint{0.687461in}{1.199650in}}{\pgfqpoint{0.695361in}{1.202923in}}{\pgfqpoint{0.701185in}{1.208747in}}%
\pgfpathcurveto{\pgfqpoint{0.707009in}{1.214570in}}{\pgfqpoint{0.710281in}{1.222471in}}{\pgfqpoint{0.710281in}{1.230707in}}%
\pgfpathcurveto{\pgfqpoint{0.710281in}{1.238943in}}{\pgfqpoint{0.707009in}{1.246843in}}{\pgfqpoint{0.701185in}{1.252667in}}%
\pgfpathcurveto{\pgfqpoint{0.695361in}{1.258491in}}{\pgfqpoint{0.687461in}{1.261763in}}{\pgfqpoint{0.679225in}{1.261763in}}%
\pgfpathcurveto{\pgfqpoint{0.670989in}{1.261763in}}{\pgfqpoint{0.663089in}{1.258491in}}{\pgfqpoint{0.657265in}{1.252667in}}%
\pgfpathcurveto{\pgfqpoint{0.651441in}{1.246843in}}{\pgfqpoint{0.648168in}{1.238943in}}{\pgfqpoint{0.648168in}{1.230707in}}%
\pgfpathcurveto{\pgfqpoint{0.648168in}{1.222471in}}{\pgfqpoint{0.651441in}{1.214570in}}{\pgfqpoint{0.657265in}{1.208747in}}%
\pgfpathcurveto{\pgfqpoint{0.663089in}{1.202923in}}{\pgfqpoint{0.670989in}{1.199650in}}{\pgfqpoint{0.679225in}{1.199650in}}%
\pgfpathclose%
\pgfusepath{stroke,fill}%
\end{pgfscope}%
\begin{pgfscope}%
\pgfpathrectangle{\pgfqpoint{0.100000in}{0.220728in}}{\pgfqpoint{3.696000in}{3.696000in}}%
\pgfusepath{clip}%
\pgfsetbuttcap%
\pgfsetroundjoin%
\definecolor{currentfill}{rgb}{0.121569,0.466667,0.705882}%
\pgfsetfillcolor{currentfill}%
\pgfsetfillopacity{0.619875}%
\pgfsetlinewidth{1.003750pt}%
\definecolor{currentstroke}{rgb}{0.121569,0.466667,0.705882}%
\pgfsetstrokecolor{currentstroke}%
\pgfsetstrokeopacity{0.619875}%
\pgfsetdash{}{0pt}%
\pgfpathmoveto{\pgfqpoint{0.679213in}{1.199637in}}%
\pgfpathcurveto{\pgfqpoint{0.687450in}{1.199637in}}{\pgfqpoint{0.695350in}{1.202909in}}{\pgfqpoint{0.701174in}{1.208733in}}%
\pgfpathcurveto{\pgfqpoint{0.706998in}{1.214557in}}{\pgfqpoint{0.710270in}{1.222457in}}{\pgfqpoint{0.710270in}{1.230694in}}%
\pgfpathcurveto{\pgfqpoint{0.710270in}{1.238930in}}{\pgfqpoint{0.706998in}{1.246830in}}{\pgfqpoint{0.701174in}{1.252654in}}%
\pgfpathcurveto{\pgfqpoint{0.695350in}{1.258478in}}{\pgfqpoint{0.687450in}{1.261750in}}{\pgfqpoint{0.679213in}{1.261750in}}%
\pgfpathcurveto{\pgfqpoint{0.670977in}{1.261750in}}{\pgfqpoint{0.663077in}{1.258478in}}{\pgfqpoint{0.657253in}{1.252654in}}%
\pgfpathcurveto{\pgfqpoint{0.651429in}{1.246830in}}{\pgfqpoint{0.648157in}{1.238930in}}{\pgfqpoint{0.648157in}{1.230694in}}%
\pgfpathcurveto{\pgfqpoint{0.648157in}{1.222457in}}{\pgfqpoint{0.651429in}{1.214557in}}{\pgfqpoint{0.657253in}{1.208733in}}%
\pgfpathcurveto{\pgfqpoint{0.663077in}{1.202909in}}{\pgfqpoint{0.670977in}{1.199637in}}{\pgfqpoint{0.679213in}{1.199637in}}%
\pgfpathclose%
\pgfusepath{stroke,fill}%
\end{pgfscope}%
\begin{pgfscope}%
\pgfpathrectangle{\pgfqpoint{0.100000in}{0.220728in}}{\pgfqpoint{3.696000in}{3.696000in}}%
\pgfusepath{clip}%
\pgfsetbuttcap%
\pgfsetroundjoin%
\definecolor{currentfill}{rgb}{0.121569,0.466667,0.705882}%
\pgfsetfillcolor{currentfill}%
\pgfsetfillopacity{0.619876}%
\pgfsetlinewidth{1.003750pt}%
\definecolor{currentstroke}{rgb}{0.121569,0.466667,0.705882}%
\pgfsetstrokecolor{currentstroke}%
\pgfsetstrokeopacity{0.619876}%
\pgfsetdash{}{0pt}%
\pgfpathmoveto{\pgfqpoint{0.679207in}{1.199630in}}%
\pgfpathcurveto{\pgfqpoint{0.687443in}{1.199630in}}{\pgfqpoint{0.695343in}{1.202902in}}{\pgfqpoint{0.701167in}{1.208726in}}%
\pgfpathcurveto{\pgfqpoint{0.706991in}{1.214550in}}{\pgfqpoint{0.710263in}{1.222450in}}{\pgfqpoint{0.710263in}{1.230686in}}%
\pgfpathcurveto{\pgfqpoint{0.710263in}{1.238923in}}{\pgfqpoint{0.706991in}{1.246823in}}{\pgfqpoint{0.701167in}{1.252647in}}%
\pgfpathcurveto{\pgfqpoint{0.695343in}{1.258471in}}{\pgfqpoint{0.687443in}{1.261743in}}{\pgfqpoint{0.679207in}{1.261743in}}%
\pgfpathcurveto{\pgfqpoint{0.670971in}{1.261743in}}{\pgfqpoint{0.663071in}{1.258471in}}{\pgfqpoint{0.657247in}{1.252647in}}%
\pgfpathcurveto{\pgfqpoint{0.651423in}{1.246823in}}{\pgfqpoint{0.648150in}{1.238923in}}{\pgfqpoint{0.648150in}{1.230686in}}%
\pgfpathcurveto{\pgfqpoint{0.648150in}{1.222450in}}{\pgfqpoint{0.651423in}{1.214550in}}{\pgfqpoint{0.657247in}{1.208726in}}%
\pgfpathcurveto{\pgfqpoint{0.663071in}{1.202902in}}{\pgfqpoint{0.670971in}{1.199630in}}{\pgfqpoint{0.679207in}{1.199630in}}%
\pgfpathclose%
\pgfusepath{stroke,fill}%
\end{pgfscope}%
\begin{pgfscope}%
\pgfpathrectangle{\pgfqpoint{0.100000in}{0.220728in}}{\pgfqpoint{3.696000in}{3.696000in}}%
\pgfusepath{clip}%
\pgfsetbuttcap%
\pgfsetroundjoin%
\definecolor{currentfill}{rgb}{0.121569,0.466667,0.705882}%
\pgfsetfillcolor{currentfill}%
\pgfsetfillopacity{0.619877}%
\pgfsetlinewidth{1.003750pt}%
\definecolor{currentstroke}{rgb}{0.121569,0.466667,0.705882}%
\pgfsetstrokecolor{currentstroke}%
\pgfsetstrokeopacity{0.619877}%
\pgfsetdash{}{0pt}%
\pgfpathmoveto{\pgfqpoint{0.679203in}{1.199626in}}%
\pgfpathcurveto{\pgfqpoint{0.687440in}{1.199626in}}{\pgfqpoint{0.695340in}{1.202898in}}{\pgfqpoint{0.701164in}{1.208722in}}%
\pgfpathcurveto{\pgfqpoint{0.706988in}{1.214546in}}{\pgfqpoint{0.710260in}{1.222446in}}{\pgfqpoint{0.710260in}{1.230682in}}%
\pgfpathcurveto{\pgfqpoint{0.710260in}{1.238919in}}{\pgfqpoint{0.706988in}{1.246819in}}{\pgfqpoint{0.701164in}{1.252643in}}%
\pgfpathcurveto{\pgfqpoint{0.695340in}{1.258466in}}{\pgfqpoint{0.687440in}{1.261739in}}{\pgfqpoint{0.679203in}{1.261739in}}%
\pgfpathcurveto{\pgfqpoint{0.670967in}{1.261739in}}{\pgfqpoint{0.663067in}{1.258466in}}{\pgfqpoint{0.657243in}{1.252643in}}%
\pgfpathcurveto{\pgfqpoint{0.651419in}{1.246819in}}{\pgfqpoint{0.648147in}{1.238919in}}{\pgfqpoint{0.648147in}{1.230682in}}%
\pgfpathcurveto{\pgfqpoint{0.648147in}{1.222446in}}{\pgfqpoint{0.651419in}{1.214546in}}{\pgfqpoint{0.657243in}{1.208722in}}%
\pgfpathcurveto{\pgfqpoint{0.663067in}{1.202898in}}{\pgfqpoint{0.670967in}{1.199626in}}{\pgfqpoint{0.679203in}{1.199626in}}%
\pgfpathclose%
\pgfusepath{stroke,fill}%
\end{pgfscope}%
\begin{pgfscope}%
\pgfpathrectangle{\pgfqpoint{0.100000in}{0.220728in}}{\pgfqpoint{3.696000in}{3.696000in}}%
\pgfusepath{clip}%
\pgfsetbuttcap%
\pgfsetroundjoin%
\definecolor{currentfill}{rgb}{0.121569,0.466667,0.705882}%
\pgfsetfillcolor{currentfill}%
\pgfsetfillopacity{0.619877}%
\pgfsetlinewidth{1.003750pt}%
\definecolor{currentstroke}{rgb}{0.121569,0.466667,0.705882}%
\pgfsetstrokecolor{currentstroke}%
\pgfsetstrokeopacity{0.619877}%
\pgfsetdash{}{0pt}%
\pgfpathmoveto{\pgfqpoint{0.679201in}{1.199624in}}%
\pgfpathcurveto{\pgfqpoint{0.687438in}{1.199624in}}{\pgfqpoint{0.695338in}{1.202896in}}{\pgfqpoint{0.701162in}{1.208720in}}%
\pgfpathcurveto{\pgfqpoint{0.706986in}{1.214544in}}{\pgfqpoint{0.710258in}{1.222444in}}{\pgfqpoint{0.710258in}{1.230680in}}%
\pgfpathcurveto{\pgfqpoint{0.710258in}{1.238916in}}{\pgfqpoint{0.706986in}{1.246816in}}{\pgfqpoint{0.701162in}{1.252640in}}%
\pgfpathcurveto{\pgfqpoint{0.695338in}{1.258464in}}{\pgfqpoint{0.687438in}{1.261737in}}{\pgfqpoint{0.679201in}{1.261737in}}%
\pgfpathcurveto{\pgfqpoint{0.670965in}{1.261737in}}{\pgfqpoint{0.663065in}{1.258464in}}{\pgfqpoint{0.657241in}{1.252640in}}%
\pgfpathcurveto{\pgfqpoint{0.651417in}{1.246816in}}{\pgfqpoint{0.648145in}{1.238916in}}{\pgfqpoint{0.648145in}{1.230680in}}%
\pgfpathcurveto{\pgfqpoint{0.648145in}{1.222444in}}{\pgfqpoint{0.651417in}{1.214544in}}{\pgfqpoint{0.657241in}{1.208720in}}%
\pgfpathcurveto{\pgfqpoint{0.663065in}{1.202896in}}{\pgfqpoint{0.670965in}{1.199624in}}{\pgfqpoint{0.679201in}{1.199624in}}%
\pgfpathclose%
\pgfusepath{stroke,fill}%
\end{pgfscope}%
\begin{pgfscope}%
\pgfpathrectangle{\pgfqpoint{0.100000in}{0.220728in}}{\pgfqpoint{3.696000in}{3.696000in}}%
\pgfusepath{clip}%
\pgfsetbuttcap%
\pgfsetroundjoin%
\definecolor{currentfill}{rgb}{0.121569,0.466667,0.705882}%
\pgfsetfillcolor{currentfill}%
\pgfsetfillopacity{0.619877}%
\pgfsetlinewidth{1.003750pt}%
\definecolor{currentstroke}{rgb}{0.121569,0.466667,0.705882}%
\pgfsetstrokecolor{currentstroke}%
\pgfsetstrokeopacity{0.619877}%
\pgfsetdash{}{0pt}%
\pgfpathmoveto{\pgfqpoint{0.679200in}{1.199622in}}%
\pgfpathcurveto{\pgfqpoint{0.687437in}{1.199622in}}{\pgfqpoint{0.695337in}{1.202895in}}{\pgfqpoint{0.701161in}{1.208719in}}%
\pgfpathcurveto{\pgfqpoint{0.706985in}{1.214543in}}{\pgfqpoint{0.710257in}{1.222443in}}{\pgfqpoint{0.710257in}{1.230679in}}%
\pgfpathcurveto{\pgfqpoint{0.710257in}{1.238915in}}{\pgfqpoint{0.706985in}{1.246815in}}{\pgfqpoint{0.701161in}{1.252639in}}%
\pgfpathcurveto{\pgfqpoint{0.695337in}{1.258463in}}{\pgfqpoint{0.687437in}{1.261735in}}{\pgfqpoint{0.679200in}{1.261735in}}%
\pgfpathcurveto{\pgfqpoint{0.670964in}{1.261735in}}{\pgfqpoint{0.663064in}{1.258463in}}{\pgfqpoint{0.657240in}{1.252639in}}%
\pgfpathcurveto{\pgfqpoint{0.651416in}{1.246815in}}{\pgfqpoint{0.648144in}{1.238915in}}{\pgfqpoint{0.648144in}{1.230679in}}%
\pgfpathcurveto{\pgfqpoint{0.648144in}{1.222443in}}{\pgfqpoint{0.651416in}{1.214543in}}{\pgfqpoint{0.657240in}{1.208719in}}%
\pgfpathcurveto{\pgfqpoint{0.663064in}{1.202895in}}{\pgfqpoint{0.670964in}{1.199622in}}{\pgfqpoint{0.679200in}{1.199622in}}%
\pgfpathclose%
\pgfusepath{stroke,fill}%
\end{pgfscope}%
\begin{pgfscope}%
\pgfpathrectangle{\pgfqpoint{0.100000in}{0.220728in}}{\pgfqpoint{3.696000in}{3.696000in}}%
\pgfusepath{clip}%
\pgfsetbuttcap%
\pgfsetroundjoin%
\definecolor{currentfill}{rgb}{0.121569,0.466667,0.705882}%
\pgfsetfillcolor{currentfill}%
\pgfsetfillopacity{0.619877}%
\pgfsetlinewidth{1.003750pt}%
\definecolor{currentstroke}{rgb}{0.121569,0.466667,0.705882}%
\pgfsetstrokecolor{currentstroke}%
\pgfsetstrokeopacity{0.619877}%
\pgfsetdash{}{0pt}%
\pgfpathmoveto{\pgfqpoint{0.679200in}{1.199622in}}%
\pgfpathcurveto{\pgfqpoint{0.687436in}{1.199622in}}{\pgfqpoint{0.695336in}{1.202894in}}{\pgfqpoint{0.701160in}{1.208718in}}%
\pgfpathcurveto{\pgfqpoint{0.706984in}{1.214542in}}{\pgfqpoint{0.710256in}{1.222442in}}{\pgfqpoint{0.710256in}{1.230678in}}%
\pgfpathcurveto{\pgfqpoint{0.710256in}{1.238914in}}{\pgfqpoint{0.706984in}{1.246815in}}{\pgfqpoint{0.701160in}{1.252638in}}%
\pgfpathcurveto{\pgfqpoint{0.695336in}{1.258462in}}{\pgfqpoint{0.687436in}{1.261735in}}{\pgfqpoint{0.679200in}{1.261735in}}%
\pgfpathcurveto{\pgfqpoint{0.670963in}{1.261735in}}{\pgfqpoint{0.663063in}{1.258462in}}{\pgfqpoint{0.657239in}{1.252638in}}%
\pgfpathcurveto{\pgfqpoint{0.651416in}{1.246815in}}{\pgfqpoint{0.648143in}{1.238914in}}{\pgfqpoint{0.648143in}{1.230678in}}%
\pgfpathcurveto{\pgfqpoint{0.648143in}{1.222442in}}{\pgfqpoint{0.651416in}{1.214542in}}{\pgfqpoint{0.657239in}{1.208718in}}%
\pgfpathcurveto{\pgfqpoint{0.663063in}{1.202894in}}{\pgfqpoint{0.670963in}{1.199622in}}{\pgfqpoint{0.679200in}{1.199622in}}%
\pgfpathclose%
\pgfusepath{stroke,fill}%
\end{pgfscope}%
\begin{pgfscope}%
\pgfpathrectangle{\pgfqpoint{0.100000in}{0.220728in}}{\pgfqpoint{3.696000in}{3.696000in}}%
\pgfusepath{clip}%
\pgfsetbuttcap%
\pgfsetroundjoin%
\definecolor{currentfill}{rgb}{0.121569,0.466667,0.705882}%
\pgfsetfillcolor{currentfill}%
\pgfsetfillopacity{0.619877}%
\pgfsetlinewidth{1.003750pt}%
\definecolor{currentstroke}{rgb}{0.121569,0.466667,0.705882}%
\pgfsetstrokecolor{currentstroke}%
\pgfsetstrokeopacity{0.619877}%
\pgfsetdash{}{0pt}%
\pgfpathmoveto{\pgfqpoint{0.679199in}{1.199621in}}%
\pgfpathcurveto{\pgfqpoint{0.687436in}{1.199621in}}{\pgfqpoint{0.695336in}{1.202894in}}{\pgfqpoint{0.701160in}{1.208718in}}%
\pgfpathcurveto{\pgfqpoint{0.706984in}{1.214542in}}{\pgfqpoint{0.710256in}{1.222442in}}{\pgfqpoint{0.710256in}{1.230678in}}%
\pgfpathcurveto{\pgfqpoint{0.710256in}{1.238914in}}{\pgfqpoint{0.706984in}{1.246814in}}{\pgfqpoint{0.701160in}{1.252638in}}%
\pgfpathcurveto{\pgfqpoint{0.695336in}{1.258462in}}{\pgfqpoint{0.687436in}{1.261734in}}{\pgfqpoint{0.679199in}{1.261734in}}%
\pgfpathcurveto{\pgfqpoint{0.670963in}{1.261734in}}{\pgfqpoint{0.663063in}{1.258462in}}{\pgfqpoint{0.657239in}{1.252638in}}%
\pgfpathcurveto{\pgfqpoint{0.651415in}{1.246814in}}{\pgfqpoint{0.648143in}{1.238914in}}{\pgfqpoint{0.648143in}{1.230678in}}%
\pgfpathcurveto{\pgfqpoint{0.648143in}{1.222442in}}{\pgfqpoint{0.651415in}{1.214542in}}{\pgfqpoint{0.657239in}{1.208718in}}%
\pgfpathcurveto{\pgfqpoint{0.663063in}{1.202894in}}{\pgfqpoint{0.670963in}{1.199621in}}{\pgfqpoint{0.679199in}{1.199621in}}%
\pgfpathclose%
\pgfusepath{stroke,fill}%
\end{pgfscope}%
\begin{pgfscope}%
\pgfpathrectangle{\pgfqpoint{0.100000in}{0.220728in}}{\pgfqpoint{3.696000in}{3.696000in}}%
\pgfusepath{clip}%
\pgfsetbuttcap%
\pgfsetroundjoin%
\definecolor{currentfill}{rgb}{0.121569,0.466667,0.705882}%
\pgfsetfillcolor{currentfill}%
\pgfsetfillopacity{0.619877}%
\pgfsetlinewidth{1.003750pt}%
\definecolor{currentstroke}{rgb}{0.121569,0.466667,0.705882}%
\pgfsetstrokecolor{currentstroke}%
\pgfsetstrokeopacity{0.619877}%
\pgfsetdash{}{0pt}%
\pgfpathmoveto{\pgfqpoint{0.679199in}{1.199621in}}%
\pgfpathcurveto{\pgfqpoint{0.687435in}{1.199621in}}{\pgfqpoint{0.695336in}{1.202893in}}{\pgfqpoint{0.701159in}{1.208717in}}%
\pgfpathcurveto{\pgfqpoint{0.706983in}{1.214541in}}{\pgfqpoint{0.710256in}{1.222441in}}{\pgfqpoint{0.710256in}{1.230678in}}%
\pgfpathcurveto{\pgfqpoint{0.710256in}{1.238914in}}{\pgfqpoint{0.706983in}{1.246814in}}{\pgfqpoint{0.701159in}{1.252638in}}%
\pgfpathcurveto{\pgfqpoint{0.695336in}{1.258462in}}{\pgfqpoint{0.687435in}{1.261734in}}{\pgfqpoint{0.679199in}{1.261734in}}%
\pgfpathcurveto{\pgfqpoint{0.670963in}{1.261734in}}{\pgfqpoint{0.663063in}{1.258462in}}{\pgfqpoint{0.657239in}{1.252638in}}%
\pgfpathcurveto{\pgfqpoint{0.651415in}{1.246814in}}{\pgfqpoint{0.648143in}{1.238914in}}{\pgfqpoint{0.648143in}{1.230678in}}%
\pgfpathcurveto{\pgfqpoint{0.648143in}{1.222441in}}{\pgfqpoint{0.651415in}{1.214541in}}{\pgfqpoint{0.657239in}{1.208717in}}%
\pgfpathcurveto{\pgfqpoint{0.663063in}{1.202893in}}{\pgfqpoint{0.670963in}{1.199621in}}{\pgfqpoint{0.679199in}{1.199621in}}%
\pgfpathclose%
\pgfusepath{stroke,fill}%
\end{pgfscope}%
\begin{pgfscope}%
\pgfpathrectangle{\pgfqpoint{0.100000in}{0.220728in}}{\pgfqpoint{3.696000in}{3.696000in}}%
\pgfusepath{clip}%
\pgfsetbuttcap%
\pgfsetroundjoin%
\definecolor{currentfill}{rgb}{0.121569,0.466667,0.705882}%
\pgfsetfillcolor{currentfill}%
\pgfsetfillopacity{0.619877}%
\pgfsetlinewidth{1.003750pt}%
\definecolor{currentstroke}{rgb}{0.121569,0.466667,0.705882}%
\pgfsetstrokecolor{currentstroke}%
\pgfsetstrokeopacity{0.619877}%
\pgfsetdash{}{0pt}%
\pgfpathmoveto{\pgfqpoint{0.679199in}{1.199621in}}%
\pgfpathcurveto{\pgfqpoint{0.687435in}{1.199621in}}{\pgfqpoint{0.695335in}{1.202893in}}{\pgfqpoint{0.701159in}{1.208717in}}%
\pgfpathcurveto{\pgfqpoint{0.706983in}{1.214541in}}{\pgfqpoint{0.710256in}{1.222441in}}{\pgfqpoint{0.710256in}{1.230678in}}%
\pgfpathcurveto{\pgfqpoint{0.710256in}{1.238914in}}{\pgfqpoint{0.706983in}{1.246814in}}{\pgfqpoint{0.701159in}{1.252638in}}%
\pgfpathcurveto{\pgfqpoint{0.695335in}{1.258462in}}{\pgfqpoint{0.687435in}{1.261734in}}{\pgfqpoint{0.679199in}{1.261734in}}%
\pgfpathcurveto{\pgfqpoint{0.670963in}{1.261734in}}{\pgfqpoint{0.663063in}{1.258462in}}{\pgfqpoint{0.657239in}{1.252638in}}%
\pgfpathcurveto{\pgfqpoint{0.651415in}{1.246814in}}{\pgfqpoint{0.648143in}{1.238914in}}{\pgfqpoint{0.648143in}{1.230678in}}%
\pgfpathcurveto{\pgfqpoint{0.648143in}{1.222441in}}{\pgfqpoint{0.651415in}{1.214541in}}{\pgfqpoint{0.657239in}{1.208717in}}%
\pgfpathcurveto{\pgfqpoint{0.663063in}{1.202893in}}{\pgfqpoint{0.670963in}{1.199621in}}{\pgfqpoint{0.679199in}{1.199621in}}%
\pgfpathclose%
\pgfusepath{stroke,fill}%
\end{pgfscope}%
\begin{pgfscope}%
\pgfpathrectangle{\pgfqpoint{0.100000in}{0.220728in}}{\pgfqpoint{3.696000in}{3.696000in}}%
\pgfusepath{clip}%
\pgfsetbuttcap%
\pgfsetroundjoin%
\definecolor{currentfill}{rgb}{0.121569,0.466667,0.705882}%
\pgfsetfillcolor{currentfill}%
\pgfsetfillopacity{0.619877}%
\pgfsetlinewidth{1.003750pt}%
\definecolor{currentstroke}{rgb}{0.121569,0.466667,0.705882}%
\pgfsetstrokecolor{currentstroke}%
\pgfsetstrokeopacity{0.619877}%
\pgfsetdash{}{0pt}%
\pgfpathmoveto{\pgfqpoint{0.679199in}{1.199621in}}%
\pgfpathcurveto{\pgfqpoint{0.687435in}{1.199621in}}{\pgfqpoint{0.695335in}{1.202893in}}{\pgfqpoint{0.701159in}{1.208717in}}%
\pgfpathcurveto{\pgfqpoint{0.706983in}{1.214541in}}{\pgfqpoint{0.710256in}{1.222441in}}{\pgfqpoint{0.710256in}{1.230678in}}%
\pgfpathcurveto{\pgfqpoint{0.710256in}{1.238914in}}{\pgfqpoint{0.706983in}{1.246814in}}{\pgfqpoint{0.701159in}{1.252638in}}%
\pgfpathcurveto{\pgfqpoint{0.695335in}{1.258462in}}{\pgfqpoint{0.687435in}{1.261734in}}{\pgfqpoint{0.679199in}{1.261734in}}%
\pgfpathcurveto{\pgfqpoint{0.670963in}{1.261734in}}{\pgfqpoint{0.663063in}{1.258462in}}{\pgfqpoint{0.657239in}{1.252638in}}%
\pgfpathcurveto{\pgfqpoint{0.651415in}{1.246814in}}{\pgfqpoint{0.648143in}{1.238914in}}{\pgfqpoint{0.648143in}{1.230678in}}%
\pgfpathcurveto{\pgfqpoint{0.648143in}{1.222441in}}{\pgfqpoint{0.651415in}{1.214541in}}{\pgfqpoint{0.657239in}{1.208717in}}%
\pgfpathcurveto{\pgfqpoint{0.663063in}{1.202893in}}{\pgfqpoint{0.670963in}{1.199621in}}{\pgfqpoint{0.679199in}{1.199621in}}%
\pgfpathclose%
\pgfusepath{stroke,fill}%
\end{pgfscope}%
\begin{pgfscope}%
\pgfpathrectangle{\pgfqpoint{0.100000in}{0.220728in}}{\pgfqpoint{3.696000in}{3.696000in}}%
\pgfusepath{clip}%
\pgfsetbuttcap%
\pgfsetroundjoin%
\definecolor{currentfill}{rgb}{0.121569,0.466667,0.705882}%
\pgfsetfillcolor{currentfill}%
\pgfsetfillopacity{0.619877}%
\pgfsetlinewidth{1.003750pt}%
\definecolor{currentstroke}{rgb}{0.121569,0.466667,0.705882}%
\pgfsetstrokecolor{currentstroke}%
\pgfsetstrokeopacity{0.619877}%
\pgfsetdash{}{0pt}%
\pgfpathmoveto{\pgfqpoint{0.679199in}{1.199621in}}%
\pgfpathcurveto{\pgfqpoint{0.687435in}{1.199621in}}{\pgfqpoint{0.695335in}{1.202893in}}{\pgfqpoint{0.701159in}{1.208717in}}%
\pgfpathcurveto{\pgfqpoint{0.706983in}{1.214541in}}{\pgfqpoint{0.710256in}{1.222441in}}{\pgfqpoint{0.710256in}{1.230677in}}%
\pgfpathcurveto{\pgfqpoint{0.710256in}{1.238914in}}{\pgfqpoint{0.706983in}{1.246814in}}{\pgfqpoint{0.701159in}{1.252638in}}%
\pgfpathcurveto{\pgfqpoint{0.695335in}{1.258462in}}{\pgfqpoint{0.687435in}{1.261734in}}{\pgfqpoint{0.679199in}{1.261734in}}%
\pgfpathcurveto{\pgfqpoint{0.670963in}{1.261734in}}{\pgfqpoint{0.663063in}{1.258462in}}{\pgfqpoint{0.657239in}{1.252638in}}%
\pgfpathcurveto{\pgfqpoint{0.651415in}{1.246814in}}{\pgfqpoint{0.648143in}{1.238914in}}{\pgfqpoint{0.648143in}{1.230677in}}%
\pgfpathcurveto{\pgfqpoint{0.648143in}{1.222441in}}{\pgfqpoint{0.651415in}{1.214541in}}{\pgfqpoint{0.657239in}{1.208717in}}%
\pgfpathcurveto{\pgfqpoint{0.663063in}{1.202893in}}{\pgfqpoint{0.670963in}{1.199621in}}{\pgfqpoint{0.679199in}{1.199621in}}%
\pgfpathclose%
\pgfusepath{stroke,fill}%
\end{pgfscope}%
\begin{pgfscope}%
\pgfpathrectangle{\pgfqpoint{0.100000in}{0.220728in}}{\pgfqpoint{3.696000in}{3.696000in}}%
\pgfusepath{clip}%
\pgfsetbuttcap%
\pgfsetroundjoin%
\definecolor{currentfill}{rgb}{0.121569,0.466667,0.705882}%
\pgfsetfillcolor{currentfill}%
\pgfsetfillopacity{0.619877}%
\pgfsetlinewidth{1.003750pt}%
\definecolor{currentstroke}{rgb}{0.121569,0.466667,0.705882}%
\pgfsetstrokecolor{currentstroke}%
\pgfsetstrokeopacity{0.619877}%
\pgfsetdash{}{0pt}%
\pgfpathmoveto{\pgfqpoint{0.679199in}{1.199621in}}%
\pgfpathcurveto{\pgfqpoint{0.687435in}{1.199621in}}{\pgfqpoint{0.695335in}{1.202893in}}{\pgfqpoint{0.701159in}{1.208717in}}%
\pgfpathcurveto{\pgfqpoint{0.706983in}{1.214541in}}{\pgfqpoint{0.710256in}{1.222441in}}{\pgfqpoint{0.710256in}{1.230677in}}%
\pgfpathcurveto{\pgfqpoint{0.710256in}{1.238914in}}{\pgfqpoint{0.706983in}{1.246814in}}{\pgfqpoint{0.701159in}{1.252638in}}%
\pgfpathcurveto{\pgfqpoint{0.695335in}{1.258462in}}{\pgfqpoint{0.687435in}{1.261734in}}{\pgfqpoint{0.679199in}{1.261734in}}%
\pgfpathcurveto{\pgfqpoint{0.670963in}{1.261734in}}{\pgfqpoint{0.663063in}{1.258462in}}{\pgfqpoint{0.657239in}{1.252638in}}%
\pgfpathcurveto{\pgfqpoint{0.651415in}{1.246814in}}{\pgfqpoint{0.648143in}{1.238914in}}{\pgfqpoint{0.648143in}{1.230677in}}%
\pgfpathcurveto{\pgfqpoint{0.648143in}{1.222441in}}{\pgfqpoint{0.651415in}{1.214541in}}{\pgfqpoint{0.657239in}{1.208717in}}%
\pgfpathcurveto{\pgfqpoint{0.663063in}{1.202893in}}{\pgfqpoint{0.670963in}{1.199621in}}{\pgfqpoint{0.679199in}{1.199621in}}%
\pgfpathclose%
\pgfusepath{stroke,fill}%
\end{pgfscope}%
\begin{pgfscope}%
\pgfpathrectangle{\pgfqpoint{0.100000in}{0.220728in}}{\pgfqpoint{3.696000in}{3.696000in}}%
\pgfusepath{clip}%
\pgfsetbuttcap%
\pgfsetroundjoin%
\definecolor{currentfill}{rgb}{0.121569,0.466667,0.705882}%
\pgfsetfillcolor{currentfill}%
\pgfsetfillopacity{0.619877}%
\pgfsetlinewidth{1.003750pt}%
\definecolor{currentstroke}{rgb}{0.121569,0.466667,0.705882}%
\pgfsetstrokecolor{currentstroke}%
\pgfsetstrokeopacity{0.619877}%
\pgfsetdash{}{0pt}%
\pgfpathmoveto{\pgfqpoint{0.679199in}{1.199621in}}%
\pgfpathcurveto{\pgfqpoint{0.687435in}{1.199621in}}{\pgfqpoint{0.695335in}{1.202893in}}{\pgfqpoint{0.701159in}{1.208717in}}%
\pgfpathcurveto{\pgfqpoint{0.706983in}{1.214541in}}{\pgfqpoint{0.710255in}{1.222441in}}{\pgfqpoint{0.710255in}{1.230677in}}%
\pgfpathcurveto{\pgfqpoint{0.710255in}{1.238914in}}{\pgfqpoint{0.706983in}{1.246814in}}{\pgfqpoint{0.701159in}{1.252638in}}%
\pgfpathcurveto{\pgfqpoint{0.695335in}{1.258462in}}{\pgfqpoint{0.687435in}{1.261734in}}{\pgfqpoint{0.679199in}{1.261734in}}%
\pgfpathcurveto{\pgfqpoint{0.670963in}{1.261734in}}{\pgfqpoint{0.663063in}{1.258462in}}{\pgfqpoint{0.657239in}{1.252638in}}%
\pgfpathcurveto{\pgfqpoint{0.651415in}{1.246814in}}{\pgfqpoint{0.648143in}{1.238914in}}{\pgfqpoint{0.648143in}{1.230677in}}%
\pgfpathcurveto{\pgfqpoint{0.648143in}{1.222441in}}{\pgfqpoint{0.651415in}{1.214541in}}{\pgfqpoint{0.657239in}{1.208717in}}%
\pgfpathcurveto{\pgfqpoint{0.663063in}{1.202893in}}{\pgfqpoint{0.670963in}{1.199621in}}{\pgfqpoint{0.679199in}{1.199621in}}%
\pgfpathclose%
\pgfusepath{stroke,fill}%
\end{pgfscope}%
\begin{pgfscope}%
\pgfpathrectangle{\pgfqpoint{0.100000in}{0.220728in}}{\pgfqpoint{3.696000in}{3.696000in}}%
\pgfusepath{clip}%
\pgfsetbuttcap%
\pgfsetroundjoin%
\definecolor{currentfill}{rgb}{0.121569,0.466667,0.705882}%
\pgfsetfillcolor{currentfill}%
\pgfsetfillopacity{0.619877}%
\pgfsetlinewidth{1.003750pt}%
\definecolor{currentstroke}{rgb}{0.121569,0.466667,0.705882}%
\pgfsetstrokecolor{currentstroke}%
\pgfsetstrokeopacity{0.619877}%
\pgfsetdash{}{0pt}%
\pgfpathmoveto{\pgfqpoint{0.679199in}{1.199621in}}%
\pgfpathcurveto{\pgfqpoint{0.687435in}{1.199621in}}{\pgfqpoint{0.695335in}{1.202893in}}{\pgfqpoint{0.701159in}{1.208717in}}%
\pgfpathcurveto{\pgfqpoint{0.706983in}{1.214541in}}{\pgfqpoint{0.710255in}{1.222441in}}{\pgfqpoint{0.710255in}{1.230677in}}%
\pgfpathcurveto{\pgfqpoint{0.710255in}{1.238914in}}{\pgfqpoint{0.706983in}{1.246814in}}{\pgfqpoint{0.701159in}{1.252638in}}%
\pgfpathcurveto{\pgfqpoint{0.695335in}{1.258462in}}{\pgfqpoint{0.687435in}{1.261734in}}{\pgfqpoint{0.679199in}{1.261734in}}%
\pgfpathcurveto{\pgfqpoint{0.670963in}{1.261734in}}{\pgfqpoint{0.663063in}{1.258462in}}{\pgfqpoint{0.657239in}{1.252638in}}%
\pgfpathcurveto{\pgfqpoint{0.651415in}{1.246814in}}{\pgfqpoint{0.648142in}{1.238914in}}{\pgfqpoint{0.648142in}{1.230677in}}%
\pgfpathcurveto{\pgfqpoint{0.648142in}{1.222441in}}{\pgfqpoint{0.651415in}{1.214541in}}{\pgfqpoint{0.657239in}{1.208717in}}%
\pgfpathcurveto{\pgfqpoint{0.663063in}{1.202893in}}{\pgfqpoint{0.670963in}{1.199621in}}{\pgfqpoint{0.679199in}{1.199621in}}%
\pgfpathclose%
\pgfusepath{stroke,fill}%
\end{pgfscope}%
\begin{pgfscope}%
\pgfpathrectangle{\pgfqpoint{0.100000in}{0.220728in}}{\pgfqpoint{3.696000in}{3.696000in}}%
\pgfusepath{clip}%
\pgfsetbuttcap%
\pgfsetroundjoin%
\definecolor{currentfill}{rgb}{0.121569,0.466667,0.705882}%
\pgfsetfillcolor{currentfill}%
\pgfsetfillopacity{0.619878}%
\pgfsetlinewidth{1.003750pt}%
\definecolor{currentstroke}{rgb}{0.121569,0.466667,0.705882}%
\pgfsetstrokecolor{currentstroke}%
\pgfsetstrokeopacity{0.619878}%
\pgfsetdash{}{0pt}%
\pgfpathmoveto{\pgfqpoint{0.679199in}{1.199621in}}%
\pgfpathcurveto{\pgfqpoint{0.687435in}{1.199621in}}{\pgfqpoint{0.695335in}{1.202893in}}{\pgfqpoint{0.701159in}{1.208717in}}%
\pgfpathcurveto{\pgfqpoint{0.706983in}{1.214541in}}{\pgfqpoint{0.710255in}{1.222441in}}{\pgfqpoint{0.710255in}{1.230677in}}%
\pgfpathcurveto{\pgfqpoint{0.710255in}{1.238914in}}{\pgfqpoint{0.706983in}{1.246814in}}{\pgfqpoint{0.701159in}{1.252638in}}%
\pgfpathcurveto{\pgfqpoint{0.695335in}{1.258462in}}{\pgfqpoint{0.687435in}{1.261734in}}{\pgfqpoint{0.679199in}{1.261734in}}%
\pgfpathcurveto{\pgfqpoint{0.670963in}{1.261734in}}{\pgfqpoint{0.663063in}{1.258462in}}{\pgfqpoint{0.657239in}{1.252638in}}%
\pgfpathcurveto{\pgfqpoint{0.651415in}{1.246814in}}{\pgfqpoint{0.648142in}{1.238914in}}{\pgfqpoint{0.648142in}{1.230677in}}%
\pgfpathcurveto{\pgfqpoint{0.648142in}{1.222441in}}{\pgfqpoint{0.651415in}{1.214541in}}{\pgfqpoint{0.657239in}{1.208717in}}%
\pgfpathcurveto{\pgfqpoint{0.663063in}{1.202893in}}{\pgfqpoint{0.670963in}{1.199621in}}{\pgfqpoint{0.679199in}{1.199621in}}%
\pgfpathclose%
\pgfusepath{stroke,fill}%
\end{pgfscope}%
\begin{pgfscope}%
\pgfpathrectangle{\pgfqpoint{0.100000in}{0.220728in}}{\pgfqpoint{3.696000in}{3.696000in}}%
\pgfusepath{clip}%
\pgfsetbuttcap%
\pgfsetroundjoin%
\definecolor{currentfill}{rgb}{0.121569,0.466667,0.705882}%
\pgfsetfillcolor{currentfill}%
\pgfsetfillopacity{0.619878}%
\pgfsetlinewidth{1.003750pt}%
\definecolor{currentstroke}{rgb}{0.121569,0.466667,0.705882}%
\pgfsetstrokecolor{currentstroke}%
\pgfsetstrokeopacity{0.619878}%
\pgfsetdash{}{0pt}%
\pgfpathmoveto{\pgfqpoint{0.679199in}{1.199621in}}%
\pgfpathcurveto{\pgfqpoint{0.687435in}{1.199621in}}{\pgfqpoint{0.695335in}{1.202893in}}{\pgfqpoint{0.701159in}{1.208717in}}%
\pgfpathcurveto{\pgfqpoint{0.706983in}{1.214541in}}{\pgfqpoint{0.710255in}{1.222441in}}{\pgfqpoint{0.710255in}{1.230677in}}%
\pgfpathcurveto{\pgfqpoint{0.710255in}{1.238914in}}{\pgfqpoint{0.706983in}{1.246814in}}{\pgfqpoint{0.701159in}{1.252638in}}%
\pgfpathcurveto{\pgfqpoint{0.695335in}{1.258462in}}{\pgfqpoint{0.687435in}{1.261734in}}{\pgfqpoint{0.679199in}{1.261734in}}%
\pgfpathcurveto{\pgfqpoint{0.670963in}{1.261734in}}{\pgfqpoint{0.663063in}{1.258462in}}{\pgfqpoint{0.657239in}{1.252638in}}%
\pgfpathcurveto{\pgfqpoint{0.651415in}{1.246814in}}{\pgfqpoint{0.648142in}{1.238914in}}{\pgfqpoint{0.648142in}{1.230677in}}%
\pgfpathcurveto{\pgfqpoint{0.648142in}{1.222441in}}{\pgfqpoint{0.651415in}{1.214541in}}{\pgfqpoint{0.657239in}{1.208717in}}%
\pgfpathcurveto{\pgfqpoint{0.663063in}{1.202893in}}{\pgfqpoint{0.670963in}{1.199621in}}{\pgfqpoint{0.679199in}{1.199621in}}%
\pgfpathclose%
\pgfusepath{stroke,fill}%
\end{pgfscope}%
\begin{pgfscope}%
\pgfpathrectangle{\pgfqpoint{0.100000in}{0.220728in}}{\pgfqpoint{3.696000in}{3.696000in}}%
\pgfusepath{clip}%
\pgfsetbuttcap%
\pgfsetroundjoin%
\definecolor{currentfill}{rgb}{0.121569,0.466667,0.705882}%
\pgfsetfillcolor{currentfill}%
\pgfsetfillopacity{0.619878}%
\pgfsetlinewidth{1.003750pt}%
\definecolor{currentstroke}{rgb}{0.121569,0.466667,0.705882}%
\pgfsetstrokecolor{currentstroke}%
\pgfsetstrokeopacity{0.619878}%
\pgfsetdash{}{0pt}%
\pgfpathmoveto{\pgfqpoint{0.679199in}{1.199621in}}%
\pgfpathcurveto{\pgfqpoint{0.687435in}{1.199621in}}{\pgfqpoint{0.695335in}{1.202893in}}{\pgfqpoint{0.701159in}{1.208717in}}%
\pgfpathcurveto{\pgfqpoint{0.706983in}{1.214541in}}{\pgfqpoint{0.710255in}{1.222441in}}{\pgfqpoint{0.710255in}{1.230677in}}%
\pgfpathcurveto{\pgfqpoint{0.710255in}{1.238914in}}{\pgfqpoint{0.706983in}{1.246814in}}{\pgfqpoint{0.701159in}{1.252638in}}%
\pgfpathcurveto{\pgfqpoint{0.695335in}{1.258462in}}{\pgfqpoint{0.687435in}{1.261734in}}{\pgfqpoint{0.679199in}{1.261734in}}%
\pgfpathcurveto{\pgfqpoint{0.670963in}{1.261734in}}{\pgfqpoint{0.663063in}{1.258462in}}{\pgfqpoint{0.657239in}{1.252638in}}%
\pgfpathcurveto{\pgfqpoint{0.651415in}{1.246814in}}{\pgfqpoint{0.648142in}{1.238914in}}{\pgfqpoint{0.648142in}{1.230677in}}%
\pgfpathcurveto{\pgfqpoint{0.648142in}{1.222441in}}{\pgfqpoint{0.651415in}{1.214541in}}{\pgfqpoint{0.657239in}{1.208717in}}%
\pgfpathcurveto{\pgfqpoint{0.663063in}{1.202893in}}{\pgfqpoint{0.670963in}{1.199621in}}{\pgfqpoint{0.679199in}{1.199621in}}%
\pgfpathclose%
\pgfusepath{stroke,fill}%
\end{pgfscope}%
\begin{pgfscope}%
\pgfpathrectangle{\pgfqpoint{0.100000in}{0.220728in}}{\pgfqpoint{3.696000in}{3.696000in}}%
\pgfusepath{clip}%
\pgfsetbuttcap%
\pgfsetroundjoin%
\definecolor{currentfill}{rgb}{0.121569,0.466667,0.705882}%
\pgfsetfillcolor{currentfill}%
\pgfsetfillopacity{0.619878}%
\pgfsetlinewidth{1.003750pt}%
\definecolor{currentstroke}{rgb}{0.121569,0.466667,0.705882}%
\pgfsetstrokecolor{currentstroke}%
\pgfsetstrokeopacity{0.619878}%
\pgfsetdash{}{0pt}%
\pgfpathmoveto{\pgfqpoint{0.679199in}{1.199621in}}%
\pgfpathcurveto{\pgfqpoint{0.687435in}{1.199621in}}{\pgfqpoint{0.695335in}{1.202893in}}{\pgfqpoint{0.701159in}{1.208717in}}%
\pgfpathcurveto{\pgfqpoint{0.706983in}{1.214541in}}{\pgfqpoint{0.710255in}{1.222441in}}{\pgfqpoint{0.710255in}{1.230677in}}%
\pgfpathcurveto{\pgfqpoint{0.710255in}{1.238914in}}{\pgfqpoint{0.706983in}{1.246814in}}{\pgfqpoint{0.701159in}{1.252638in}}%
\pgfpathcurveto{\pgfqpoint{0.695335in}{1.258462in}}{\pgfqpoint{0.687435in}{1.261734in}}{\pgfqpoint{0.679199in}{1.261734in}}%
\pgfpathcurveto{\pgfqpoint{0.670963in}{1.261734in}}{\pgfqpoint{0.663063in}{1.258462in}}{\pgfqpoint{0.657239in}{1.252638in}}%
\pgfpathcurveto{\pgfqpoint{0.651415in}{1.246814in}}{\pgfqpoint{0.648142in}{1.238914in}}{\pgfqpoint{0.648142in}{1.230677in}}%
\pgfpathcurveto{\pgfqpoint{0.648142in}{1.222441in}}{\pgfqpoint{0.651415in}{1.214541in}}{\pgfqpoint{0.657239in}{1.208717in}}%
\pgfpathcurveto{\pgfqpoint{0.663063in}{1.202893in}}{\pgfqpoint{0.670963in}{1.199621in}}{\pgfqpoint{0.679199in}{1.199621in}}%
\pgfpathclose%
\pgfusepath{stroke,fill}%
\end{pgfscope}%
\begin{pgfscope}%
\pgfpathrectangle{\pgfqpoint{0.100000in}{0.220728in}}{\pgfqpoint{3.696000in}{3.696000in}}%
\pgfusepath{clip}%
\pgfsetbuttcap%
\pgfsetroundjoin%
\definecolor{currentfill}{rgb}{0.121569,0.466667,0.705882}%
\pgfsetfillcolor{currentfill}%
\pgfsetfillopacity{0.619878}%
\pgfsetlinewidth{1.003750pt}%
\definecolor{currentstroke}{rgb}{0.121569,0.466667,0.705882}%
\pgfsetstrokecolor{currentstroke}%
\pgfsetstrokeopacity{0.619878}%
\pgfsetdash{}{0pt}%
\pgfpathmoveto{\pgfqpoint{0.679199in}{1.199621in}}%
\pgfpathcurveto{\pgfqpoint{0.687435in}{1.199621in}}{\pgfqpoint{0.695335in}{1.202893in}}{\pgfqpoint{0.701159in}{1.208717in}}%
\pgfpathcurveto{\pgfqpoint{0.706983in}{1.214541in}}{\pgfqpoint{0.710255in}{1.222441in}}{\pgfqpoint{0.710255in}{1.230677in}}%
\pgfpathcurveto{\pgfqpoint{0.710255in}{1.238914in}}{\pgfqpoint{0.706983in}{1.246814in}}{\pgfqpoint{0.701159in}{1.252638in}}%
\pgfpathcurveto{\pgfqpoint{0.695335in}{1.258462in}}{\pgfqpoint{0.687435in}{1.261734in}}{\pgfqpoint{0.679199in}{1.261734in}}%
\pgfpathcurveto{\pgfqpoint{0.670963in}{1.261734in}}{\pgfqpoint{0.663063in}{1.258462in}}{\pgfqpoint{0.657239in}{1.252638in}}%
\pgfpathcurveto{\pgfqpoint{0.651415in}{1.246814in}}{\pgfqpoint{0.648142in}{1.238914in}}{\pgfqpoint{0.648142in}{1.230677in}}%
\pgfpathcurveto{\pgfqpoint{0.648142in}{1.222441in}}{\pgfqpoint{0.651415in}{1.214541in}}{\pgfqpoint{0.657239in}{1.208717in}}%
\pgfpathcurveto{\pgfqpoint{0.663063in}{1.202893in}}{\pgfqpoint{0.670963in}{1.199621in}}{\pgfqpoint{0.679199in}{1.199621in}}%
\pgfpathclose%
\pgfusepath{stroke,fill}%
\end{pgfscope}%
\begin{pgfscope}%
\pgfpathrectangle{\pgfqpoint{0.100000in}{0.220728in}}{\pgfqpoint{3.696000in}{3.696000in}}%
\pgfusepath{clip}%
\pgfsetbuttcap%
\pgfsetroundjoin%
\definecolor{currentfill}{rgb}{0.121569,0.466667,0.705882}%
\pgfsetfillcolor{currentfill}%
\pgfsetfillopacity{0.619878}%
\pgfsetlinewidth{1.003750pt}%
\definecolor{currentstroke}{rgb}{0.121569,0.466667,0.705882}%
\pgfsetstrokecolor{currentstroke}%
\pgfsetstrokeopacity{0.619878}%
\pgfsetdash{}{0pt}%
\pgfpathmoveto{\pgfqpoint{0.679199in}{1.199621in}}%
\pgfpathcurveto{\pgfqpoint{0.687435in}{1.199621in}}{\pgfqpoint{0.695335in}{1.202893in}}{\pgfqpoint{0.701159in}{1.208717in}}%
\pgfpathcurveto{\pgfqpoint{0.706983in}{1.214541in}}{\pgfqpoint{0.710255in}{1.222441in}}{\pgfqpoint{0.710255in}{1.230677in}}%
\pgfpathcurveto{\pgfqpoint{0.710255in}{1.238914in}}{\pgfqpoint{0.706983in}{1.246814in}}{\pgfqpoint{0.701159in}{1.252638in}}%
\pgfpathcurveto{\pgfqpoint{0.695335in}{1.258462in}}{\pgfqpoint{0.687435in}{1.261734in}}{\pgfqpoint{0.679199in}{1.261734in}}%
\pgfpathcurveto{\pgfqpoint{0.670963in}{1.261734in}}{\pgfqpoint{0.663063in}{1.258462in}}{\pgfqpoint{0.657239in}{1.252638in}}%
\pgfpathcurveto{\pgfqpoint{0.651415in}{1.246814in}}{\pgfqpoint{0.648142in}{1.238914in}}{\pgfqpoint{0.648142in}{1.230677in}}%
\pgfpathcurveto{\pgfqpoint{0.648142in}{1.222441in}}{\pgfqpoint{0.651415in}{1.214541in}}{\pgfqpoint{0.657239in}{1.208717in}}%
\pgfpathcurveto{\pgfqpoint{0.663063in}{1.202893in}}{\pgfqpoint{0.670963in}{1.199621in}}{\pgfqpoint{0.679199in}{1.199621in}}%
\pgfpathclose%
\pgfusepath{stroke,fill}%
\end{pgfscope}%
\begin{pgfscope}%
\pgfpathrectangle{\pgfqpoint{0.100000in}{0.220728in}}{\pgfqpoint{3.696000in}{3.696000in}}%
\pgfusepath{clip}%
\pgfsetbuttcap%
\pgfsetroundjoin%
\definecolor{currentfill}{rgb}{0.121569,0.466667,0.705882}%
\pgfsetfillcolor{currentfill}%
\pgfsetfillopacity{0.619878}%
\pgfsetlinewidth{1.003750pt}%
\definecolor{currentstroke}{rgb}{0.121569,0.466667,0.705882}%
\pgfsetstrokecolor{currentstroke}%
\pgfsetstrokeopacity{0.619878}%
\pgfsetdash{}{0pt}%
\pgfpathmoveto{\pgfqpoint{0.679199in}{1.199621in}}%
\pgfpathcurveto{\pgfqpoint{0.687435in}{1.199621in}}{\pgfqpoint{0.695335in}{1.202893in}}{\pgfqpoint{0.701159in}{1.208717in}}%
\pgfpathcurveto{\pgfqpoint{0.706983in}{1.214541in}}{\pgfqpoint{0.710255in}{1.222441in}}{\pgfqpoint{0.710255in}{1.230677in}}%
\pgfpathcurveto{\pgfqpoint{0.710255in}{1.238914in}}{\pgfqpoint{0.706983in}{1.246814in}}{\pgfqpoint{0.701159in}{1.252638in}}%
\pgfpathcurveto{\pgfqpoint{0.695335in}{1.258462in}}{\pgfqpoint{0.687435in}{1.261734in}}{\pgfqpoint{0.679199in}{1.261734in}}%
\pgfpathcurveto{\pgfqpoint{0.670963in}{1.261734in}}{\pgfqpoint{0.663063in}{1.258462in}}{\pgfqpoint{0.657239in}{1.252638in}}%
\pgfpathcurveto{\pgfqpoint{0.651415in}{1.246814in}}{\pgfqpoint{0.648142in}{1.238914in}}{\pgfqpoint{0.648142in}{1.230677in}}%
\pgfpathcurveto{\pgfqpoint{0.648142in}{1.222441in}}{\pgfqpoint{0.651415in}{1.214541in}}{\pgfqpoint{0.657239in}{1.208717in}}%
\pgfpathcurveto{\pgfqpoint{0.663063in}{1.202893in}}{\pgfqpoint{0.670963in}{1.199621in}}{\pgfqpoint{0.679199in}{1.199621in}}%
\pgfpathclose%
\pgfusepath{stroke,fill}%
\end{pgfscope}%
\begin{pgfscope}%
\pgfpathrectangle{\pgfqpoint{0.100000in}{0.220728in}}{\pgfqpoint{3.696000in}{3.696000in}}%
\pgfusepath{clip}%
\pgfsetbuttcap%
\pgfsetroundjoin%
\definecolor{currentfill}{rgb}{0.121569,0.466667,0.705882}%
\pgfsetfillcolor{currentfill}%
\pgfsetfillopacity{0.619878}%
\pgfsetlinewidth{1.003750pt}%
\definecolor{currentstroke}{rgb}{0.121569,0.466667,0.705882}%
\pgfsetstrokecolor{currentstroke}%
\pgfsetstrokeopacity{0.619878}%
\pgfsetdash{}{0pt}%
\pgfpathmoveto{\pgfqpoint{0.679199in}{1.199621in}}%
\pgfpathcurveto{\pgfqpoint{0.687435in}{1.199621in}}{\pgfqpoint{0.695335in}{1.202893in}}{\pgfqpoint{0.701159in}{1.208717in}}%
\pgfpathcurveto{\pgfqpoint{0.706983in}{1.214541in}}{\pgfqpoint{0.710255in}{1.222441in}}{\pgfqpoint{0.710255in}{1.230677in}}%
\pgfpathcurveto{\pgfqpoint{0.710255in}{1.238914in}}{\pgfqpoint{0.706983in}{1.246814in}}{\pgfqpoint{0.701159in}{1.252638in}}%
\pgfpathcurveto{\pgfqpoint{0.695335in}{1.258462in}}{\pgfqpoint{0.687435in}{1.261734in}}{\pgfqpoint{0.679199in}{1.261734in}}%
\pgfpathcurveto{\pgfqpoint{0.670963in}{1.261734in}}{\pgfqpoint{0.663063in}{1.258462in}}{\pgfqpoint{0.657239in}{1.252638in}}%
\pgfpathcurveto{\pgfqpoint{0.651415in}{1.246814in}}{\pgfqpoint{0.648142in}{1.238914in}}{\pgfqpoint{0.648142in}{1.230677in}}%
\pgfpathcurveto{\pgfqpoint{0.648142in}{1.222441in}}{\pgfqpoint{0.651415in}{1.214541in}}{\pgfqpoint{0.657239in}{1.208717in}}%
\pgfpathcurveto{\pgfqpoint{0.663063in}{1.202893in}}{\pgfqpoint{0.670963in}{1.199621in}}{\pgfqpoint{0.679199in}{1.199621in}}%
\pgfpathclose%
\pgfusepath{stroke,fill}%
\end{pgfscope}%
\begin{pgfscope}%
\pgfpathrectangle{\pgfqpoint{0.100000in}{0.220728in}}{\pgfqpoint{3.696000in}{3.696000in}}%
\pgfusepath{clip}%
\pgfsetbuttcap%
\pgfsetroundjoin%
\definecolor{currentfill}{rgb}{0.121569,0.466667,0.705882}%
\pgfsetfillcolor{currentfill}%
\pgfsetfillopacity{0.619878}%
\pgfsetlinewidth{1.003750pt}%
\definecolor{currentstroke}{rgb}{0.121569,0.466667,0.705882}%
\pgfsetstrokecolor{currentstroke}%
\pgfsetstrokeopacity{0.619878}%
\pgfsetdash{}{0pt}%
\pgfpathmoveto{\pgfqpoint{0.679199in}{1.199621in}}%
\pgfpathcurveto{\pgfqpoint{0.687435in}{1.199621in}}{\pgfqpoint{0.695335in}{1.202893in}}{\pgfqpoint{0.701159in}{1.208717in}}%
\pgfpathcurveto{\pgfqpoint{0.706983in}{1.214541in}}{\pgfqpoint{0.710255in}{1.222441in}}{\pgfqpoint{0.710255in}{1.230677in}}%
\pgfpathcurveto{\pgfqpoint{0.710255in}{1.238914in}}{\pgfqpoint{0.706983in}{1.246814in}}{\pgfqpoint{0.701159in}{1.252638in}}%
\pgfpathcurveto{\pgfqpoint{0.695335in}{1.258462in}}{\pgfqpoint{0.687435in}{1.261734in}}{\pgfqpoint{0.679199in}{1.261734in}}%
\pgfpathcurveto{\pgfqpoint{0.670963in}{1.261734in}}{\pgfqpoint{0.663063in}{1.258462in}}{\pgfqpoint{0.657239in}{1.252638in}}%
\pgfpathcurveto{\pgfqpoint{0.651415in}{1.246814in}}{\pgfqpoint{0.648142in}{1.238914in}}{\pgfqpoint{0.648142in}{1.230677in}}%
\pgfpathcurveto{\pgfqpoint{0.648142in}{1.222441in}}{\pgfqpoint{0.651415in}{1.214541in}}{\pgfqpoint{0.657239in}{1.208717in}}%
\pgfpathcurveto{\pgfqpoint{0.663063in}{1.202893in}}{\pgfqpoint{0.670963in}{1.199621in}}{\pgfqpoint{0.679199in}{1.199621in}}%
\pgfpathclose%
\pgfusepath{stroke,fill}%
\end{pgfscope}%
\begin{pgfscope}%
\pgfpathrectangle{\pgfqpoint{0.100000in}{0.220728in}}{\pgfqpoint{3.696000in}{3.696000in}}%
\pgfusepath{clip}%
\pgfsetbuttcap%
\pgfsetroundjoin%
\definecolor{currentfill}{rgb}{0.121569,0.466667,0.705882}%
\pgfsetfillcolor{currentfill}%
\pgfsetfillopacity{0.619878}%
\pgfsetlinewidth{1.003750pt}%
\definecolor{currentstroke}{rgb}{0.121569,0.466667,0.705882}%
\pgfsetstrokecolor{currentstroke}%
\pgfsetstrokeopacity{0.619878}%
\pgfsetdash{}{0pt}%
\pgfpathmoveto{\pgfqpoint{0.679199in}{1.199621in}}%
\pgfpathcurveto{\pgfqpoint{0.687435in}{1.199621in}}{\pgfqpoint{0.695335in}{1.202893in}}{\pgfqpoint{0.701159in}{1.208717in}}%
\pgfpathcurveto{\pgfqpoint{0.706983in}{1.214541in}}{\pgfqpoint{0.710255in}{1.222441in}}{\pgfqpoint{0.710255in}{1.230677in}}%
\pgfpathcurveto{\pgfqpoint{0.710255in}{1.238914in}}{\pgfqpoint{0.706983in}{1.246814in}}{\pgfqpoint{0.701159in}{1.252638in}}%
\pgfpathcurveto{\pgfqpoint{0.695335in}{1.258462in}}{\pgfqpoint{0.687435in}{1.261734in}}{\pgfqpoint{0.679199in}{1.261734in}}%
\pgfpathcurveto{\pgfqpoint{0.670963in}{1.261734in}}{\pgfqpoint{0.663063in}{1.258462in}}{\pgfqpoint{0.657239in}{1.252638in}}%
\pgfpathcurveto{\pgfqpoint{0.651415in}{1.246814in}}{\pgfqpoint{0.648142in}{1.238914in}}{\pgfqpoint{0.648142in}{1.230677in}}%
\pgfpathcurveto{\pgfqpoint{0.648142in}{1.222441in}}{\pgfqpoint{0.651415in}{1.214541in}}{\pgfqpoint{0.657239in}{1.208717in}}%
\pgfpathcurveto{\pgfqpoint{0.663063in}{1.202893in}}{\pgfqpoint{0.670963in}{1.199621in}}{\pgfqpoint{0.679199in}{1.199621in}}%
\pgfpathclose%
\pgfusepath{stroke,fill}%
\end{pgfscope}%
\begin{pgfscope}%
\pgfpathrectangle{\pgfqpoint{0.100000in}{0.220728in}}{\pgfqpoint{3.696000in}{3.696000in}}%
\pgfusepath{clip}%
\pgfsetbuttcap%
\pgfsetroundjoin%
\definecolor{currentfill}{rgb}{0.121569,0.466667,0.705882}%
\pgfsetfillcolor{currentfill}%
\pgfsetfillopacity{0.619878}%
\pgfsetlinewidth{1.003750pt}%
\definecolor{currentstroke}{rgb}{0.121569,0.466667,0.705882}%
\pgfsetstrokecolor{currentstroke}%
\pgfsetstrokeopacity{0.619878}%
\pgfsetdash{}{0pt}%
\pgfpathmoveto{\pgfqpoint{0.679199in}{1.199621in}}%
\pgfpathcurveto{\pgfqpoint{0.687435in}{1.199621in}}{\pgfqpoint{0.695335in}{1.202893in}}{\pgfqpoint{0.701159in}{1.208717in}}%
\pgfpathcurveto{\pgfqpoint{0.706983in}{1.214541in}}{\pgfqpoint{0.710255in}{1.222441in}}{\pgfqpoint{0.710255in}{1.230677in}}%
\pgfpathcurveto{\pgfqpoint{0.710255in}{1.238914in}}{\pgfqpoint{0.706983in}{1.246814in}}{\pgfqpoint{0.701159in}{1.252638in}}%
\pgfpathcurveto{\pgfqpoint{0.695335in}{1.258462in}}{\pgfqpoint{0.687435in}{1.261734in}}{\pgfqpoint{0.679199in}{1.261734in}}%
\pgfpathcurveto{\pgfqpoint{0.670963in}{1.261734in}}{\pgfqpoint{0.663063in}{1.258462in}}{\pgfqpoint{0.657239in}{1.252638in}}%
\pgfpathcurveto{\pgfqpoint{0.651415in}{1.246814in}}{\pgfqpoint{0.648142in}{1.238914in}}{\pgfqpoint{0.648142in}{1.230677in}}%
\pgfpathcurveto{\pgfqpoint{0.648142in}{1.222441in}}{\pgfqpoint{0.651415in}{1.214541in}}{\pgfqpoint{0.657239in}{1.208717in}}%
\pgfpathcurveto{\pgfqpoint{0.663063in}{1.202893in}}{\pgfqpoint{0.670963in}{1.199621in}}{\pgfqpoint{0.679199in}{1.199621in}}%
\pgfpathclose%
\pgfusepath{stroke,fill}%
\end{pgfscope}%
\begin{pgfscope}%
\pgfpathrectangle{\pgfqpoint{0.100000in}{0.220728in}}{\pgfqpoint{3.696000in}{3.696000in}}%
\pgfusepath{clip}%
\pgfsetbuttcap%
\pgfsetroundjoin%
\definecolor{currentfill}{rgb}{0.121569,0.466667,0.705882}%
\pgfsetfillcolor{currentfill}%
\pgfsetfillopacity{0.619878}%
\pgfsetlinewidth{1.003750pt}%
\definecolor{currentstroke}{rgb}{0.121569,0.466667,0.705882}%
\pgfsetstrokecolor{currentstroke}%
\pgfsetstrokeopacity{0.619878}%
\pgfsetdash{}{0pt}%
\pgfpathmoveto{\pgfqpoint{0.679199in}{1.199621in}}%
\pgfpathcurveto{\pgfqpoint{0.687435in}{1.199621in}}{\pgfqpoint{0.695335in}{1.202893in}}{\pgfqpoint{0.701159in}{1.208717in}}%
\pgfpathcurveto{\pgfqpoint{0.706983in}{1.214541in}}{\pgfqpoint{0.710255in}{1.222441in}}{\pgfqpoint{0.710255in}{1.230677in}}%
\pgfpathcurveto{\pgfqpoint{0.710255in}{1.238914in}}{\pgfqpoint{0.706983in}{1.246814in}}{\pgfqpoint{0.701159in}{1.252638in}}%
\pgfpathcurveto{\pgfqpoint{0.695335in}{1.258462in}}{\pgfqpoint{0.687435in}{1.261734in}}{\pgfqpoint{0.679199in}{1.261734in}}%
\pgfpathcurveto{\pgfqpoint{0.670963in}{1.261734in}}{\pgfqpoint{0.663063in}{1.258462in}}{\pgfqpoint{0.657239in}{1.252638in}}%
\pgfpathcurveto{\pgfqpoint{0.651415in}{1.246814in}}{\pgfqpoint{0.648142in}{1.238914in}}{\pgfqpoint{0.648142in}{1.230677in}}%
\pgfpathcurveto{\pgfqpoint{0.648142in}{1.222441in}}{\pgfqpoint{0.651415in}{1.214541in}}{\pgfqpoint{0.657239in}{1.208717in}}%
\pgfpathcurveto{\pgfqpoint{0.663063in}{1.202893in}}{\pgfqpoint{0.670963in}{1.199621in}}{\pgfqpoint{0.679199in}{1.199621in}}%
\pgfpathclose%
\pgfusepath{stroke,fill}%
\end{pgfscope}%
\begin{pgfscope}%
\pgfpathrectangle{\pgfqpoint{0.100000in}{0.220728in}}{\pgfqpoint{3.696000in}{3.696000in}}%
\pgfusepath{clip}%
\pgfsetbuttcap%
\pgfsetroundjoin%
\definecolor{currentfill}{rgb}{0.121569,0.466667,0.705882}%
\pgfsetfillcolor{currentfill}%
\pgfsetfillopacity{0.619878}%
\pgfsetlinewidth{1.003750pt}%
\definecolor{currentstroke}{rgb}{0.121569,0.466667,0.705882}%
\pgfsetstrokecolor{currentstroke}%
\pgfsetstrokeopacity{0.619878}%
\pgfsetdash{}{0pt}%
\pgfpathmoveto{\pgfqpoint{0.679199in}{1.199621in}}%
\pgfpathcurveto{\pgfqpoint{0.687435in}{1.199621in}}{\pgfqpoint{0.695335in}{1.202893in}}{\pgfqpoint{0.701159in}{1.208717in}}%
\pgfpathcurveto{\pgfqpoint{0.706983in}{1.214541in}}{\pgfqpoint{0.710255in}{1.222441in}}{\pgfqpoint{0.710255in}{1.230677in}}%
\pgfpathcurveto{\pgfqpoint{0.710255in}{1.238914in}}{\pgfqpoint{0.706983in}{1.246814in}}{\pgfqpoint{0.701159in}{1.252638in}}%
\pgfpathcurveto{\pgfqpoint{0.695335in}{1.258462in}}{\pgfqpoint{0.687435in}{1.261734in}}{\pgfqpoint{0.679199in}{1.261734in}}%
\pgfpathcurveto{\pgfqpoint{0.670963in}{1.261734in}}{\pgfqpoint{0.663063in}{1.258462in}}{\pgfqpoint{0.657239in}{1.252638in}}%
\pgfpathcurveto{\pgfqpoint{0.651415in}{1.246814in}}{\pgfqpoint{0.648142in}{1.238914in}}{\pgfqpoint{0.648142in}{1.230677in}}%
\pgfpathcurveto{\pgfqpoint{0.648142in}{1.222441in}}{\pgfqpoint{0.651415in}{1.214541in}}{\pgfqpoint{0.657239in}{1.208717in}}%
\pgfpathcurveto{\pgfqpoint{0.663063in}{1.202893in}}{\pgfqpoint{0.670963in}{1.199621in}}{\pgfqpoint{0.679199in}{1.199621in}}%
\pgfpathclose%
\pgfusepath{stroke,fill}%
\end{pgfscope}%
\begin{pgfscope}%
\pgfpathrectangle{\pgfqpoint{0.100000in}{0.220728in}}{\pgfqpoint{3.696000in}{3.696000in}}%
\pgfusepath{clip}%
\pgfsetbuttcap%
\pgfsetroundjoin%
\definecolor{currentfill}{rgb}{0.121569,0.466667,0.705882}%
\pgfsetfillcolor{currentfill}%
\pgfsetfillopacity{0.619878}%
\pgfsetlinewidth{1.003750pt}%
\definecolor{currentstroke}{rgb}{0.121569,0.466667,0.705882}%
\pgfsetstrokecolor{currentstroke}%
\pgfsetstrokeopacity{0.619878}%
\pgfsetdash{}{0pt}%
\pgfpathmoveto{\pgfqpoint{0.679199in}{1.199621in}}%
\pgfpathcurveto{\pgfqpoint{0.687435in}{1.199621in}}{\pgfqpoint{0.695335in}{1.202893in}}{\pgfqpoint{0.701159in}{1.208717in}}%
\pgfpathcurveto{\pgfqpoint{0.706983in}{1.214541in}}{\pgfqpoint{0.710255in}{1.222441in}}{\pgfqpoint{0.710255in}{1.230677in}}%
\pgfpathcurveto{\pgfqpoint{0.710255in}{1.238914in}}{\pgfqpoint{0.706983in}{1.246814in}}{\pgfqpoint{0.701159in}{1.252638in}}%
\pgfpathcurveto{\pgfqpoint{0.695335in}{1.258462in}}{\pgfqpoint{0.687435in}{1.261734in}}{\pgfqpoint{0.679199in}{1.261734in}}%
\pgfpathcurveto{\pgfqpoint{0.670963in}{1.261734in}}{\pgfqpoint{0.663063in}{1.258462in}}{\pgfqpoint{0.657239in}{1.252638in}}%
\pgfpathcurveto{\pgfqpoint{0.651415in}{1.246814in}}{\pgfqpoint{0.648142in}{1.238914in}}{\pgfqpoint{0.648142in}{1.230677in}}%
\pgfpathcurveto{\pgfqpoint{0.648142in}{1.222441in}}{\pgfqpoint{0.651415in}{1.214541in}}{\pgfqpoint{0.657239in}{1.208717in}}%
\pgfpathcurveto{\pgfqpoint{0.663063in}{1.202893in}}{\pgfqpoint{0.670963in}{1.199621in}}{\pgfqpoint{0.679199in}{1.199621in}}%
\pgfpathclose%
\pgfusepath{stroke,fill}%
\end{pgfscope}%
\begin{pgfscope}%
\pgfpathrectangle{\pgfqpoint{0.100000in}{0.220728in}}{\pgfqpoint{3.696000in}{3.696000in}}%
\pgfusepath{clip}%
\pgfsetbuttcap%
\pgfsetroundjoin%
\definecolor{currentfill}{rgb}{0.121569,0.466667,0.705882}%
\pgfsetfillcolor{currentfill}%
\pgfsetfillopacity{0.619878}%
\pgfsetlinewidth{1.003750pt}%
\definecolor{currentstroke}{rgb}{0.121569,0.466667,0.705882}%
\pgfsetstrokecolor{currentstroke}%
\pgfsetstrokeopacity{0.619878}%
\pgfsetdash{}{0pt}%
\pgfpathmoveto{\pgfqpoint{0.679199in}{1.199621in}}%
\pgfpathcurveto{\pgfqpoint{0.687435in}{1.199621in}}{\pgfqpoint{0.695335in}{1.202893in}}{\pgfqpoint{0.701159in}{1.208717in}}%
\pgfpathcurveto{\pgfqpoint{0.706983in}{1.214541in}}{\pgfqpoint{0.710255in}{1.222441in}}{\pgfqpoint{0.710255in}{1.230677in}}%
\pgfpathcurveto{\pgfqpoint{0.710255in}{1.238914in}}{\pgfqpoint{0.706983in}{1.246814in}}{\pgfqpoint{0.701159in}{1.252638in}}%
\pgfpathcurveto{\pgfqpoint{0.695335in}{1.258462in}}{\pgfqpoint{0.687435in}{1.261734in}}{\pgfqpoint{0.679199in}{1.261734in}}%
\pgfpathcurveto{\pgfqpoint{0.670963in}{1.261734in}}{\pgfqpoint{0.663063in}{1.258462in}}{\pgfqpoint{0.657239in}{1.252638in}}%
\pgfpathcurveto{\pgfqpoint{0.651415in}{1.246814in}}{\pgfqpoint{0.648142in}{1.238914in}}{\pgfqpoint{0.648142in}{1.230677in}}%
\pgfpathcurveto{\pgfqpoint{0.648142in}{1.222441in}}{\pgfqpoint{0.651415in}{1.214541in}}{\pgfqpoint{0.657239in}{1.208717in}}%
\pgfpathcurveto{\pgfqpoint{0.663063in}{1.202893in}}{\pgfqpoint{0.670963in}{1.199621in}}{\pgfqpoint{0.679199in}{1.199621in}}%
\pgfpathclose%
\pgfusepath{stroke,fill}%
\end{pgfscope}%
\begin{pgfscope}%
\pgfpathrectangle{\pgfqpoint{0.100000in}{0.220728in}}{\pgfqpoint{3.696000in}{3.696000in}}%
\pgfusepath{clip}%
\pgfsetbuttcap%
\pgfsetroundjoin%
\definecolor{currentfill}{rgb}{0.121569,0.466667,0.705882}%
\pgfsetfillcolor{currentfill}%
\pgfsetfillopacity{0.619878}%
\pgfsetlinewidth{1.003750pt}%
\definecolor{currentstroke}{rgb}{0.121569,0.466667,0.705882}%
\pgfsetstrokecolor{currentstroke}%
\pgfsetstrokeopacity{0.619878}%
\pgfsetdash{}{0pt}%
\pgfpathmoveto{\pgfqpoint{0.679199in}{1.199621in}}%
\pgfpathcurveto{\pgfqpoint{0.687435in}{1.199621in}}{\pgfqpoint{0.695335in}{1.202893in}}{\pgfqpoint{0.701159in}{1.208717in}}%
\pgfpathcurveto{\pgfqpoint{0.706983in}{1.214541in}}{\pgfqpoint{0.710255in}{1.222441in}}{\pgfqpoint{0.710255in}{1.230677in}}%
\pgfpathcurveto{\pgfqpoint{0.710255in}{1.238914in}}{\pgfqpoint{0.706983in}{1.246814in}}{\pgfqpoint{0.701159in}{1.252638in}}%
\pgfpathcurveto{\pgfqpoint{0.695335in}{1.258462in}}{\pgfqpoint{0.687435in}{1.261734in}}{\pgfqpoint{0.679199in}{1.261734in}}%
\pgfpathcurveto{\pgfqpoint{0.670963in}{1.261734in}}{\pgfqpoint{0.663063in}{1.258462in}}{\pgfqpoint{0.657239in}{1.252638in}}%
\pgfpathcurveto{\pgfqpoint{0.651415in}{1.246814in}}{\pgfqpoint{0.648142in}{1.238914in}}{\pgfqpoint{0.648142in}{1.230677in}}%
\pgfpathcurveto{\pgfqpoint{0.648142in}{1.222441in}}{\pgfqpoint{0.651415in}{1.214541in}}{\pgfqpoint{0.657239in}{1.208717in}}%
\pgfpathcurveto{\pgfqpoint{0.663063in}{1.202893in}}{\pgfqpoint{0.670963in}{1.199621in}}{\pgfqpoint{0.679199in}{1.199621in}}%
\pgfpathclose%
\pgfusepath{stroke,fill}%
\end{pgfscope}%
\begin{pgfscope}%
\pgfpathrectangle{\pgfqpoint{0.100000in}{0.220728in}}{\pgfqpoint{3.696000in}{3.696000in}}%
\pgfusepath{clip}%
\pgfsetbuttcap%
\pgfsetroundjoin%
\definecolor{currentfill}{rgb}{0.121569,0.466667,0.705882}%
\pgfsetfillcolor{currentfill}%
\pgfsetfillopacity{0.619878}%
\pgfsetlinewidth{1.003750pt}%
\definecolor{currentstroke}{rgb}{0.121569,0.466667,0.705882}%
\pgfsetstrokecolor{currentstroke}%
\pgfsetstrokeopacity{0.619878}%
\pgfsetdash{}{0pt}%
\pgfpathmoveto{\pgfqpoint{0.679199in}{1.199621in}}%
\pgfpathcurveto{\pgfqpoint{0.687435in}{1.199621in}}{\pgfqpoint{0.695335in}{1.202893in}}{\pgfqpoint{0.701159in}{1.208717in}}%
\pgfpathcurveto{\pgfqpoint{0.706983in}{1.214541in}}{\pgfqpoint{0.710255in}{1.222441in}}{\pgfqpoint{0.710255in}{1.230677in}}%
\pgfpathcurveto{\pgfqpoint{0.710255in}{1.238914in}}{\pgfqpoint{0.706983in}{1.246814in}}{\pgfqpoint{0.701159in}{1.252638in}}%
\pgfpathcurveto{\pgfqpoint{0.695335in}{1.258462in}}{\pgfqpoint{0.687435in}{1.261734in}}{\pgfqpoint{0.679199in}{1.261734in}}%
\pgfpathcurveto{\pgfqpoint{0.670963in}{1.261734in}}{\pgfqpoint{0.663063in}{1.258462in}}{\pgfqpoint{0.657239in}{1.252638in}}%
\pgfpathcurveto{\pgfqpoint{0.651415in}{1.246814in}}{\pgfqpoint{0.648142in}{1.238914in}}{\pgfqpoint{0.648142in}{1.230677in}}%
\pgfpathcurveto{\pgfqpoint{0.648142in}{1.222441in}}{\pgfqpoint{0.651415in}{1.214541in}}{\pgfqpoint{0.657239in}{1.208717in}}%
\pgfpathcurveto{\pgfqpoint{0.663063in}{1.202893in}}{\pgfqpoint{0.670963in}{1.199621in}}{\pgfqpoint{0.679199in}{1.199621in}}%
\pgfpathclose%
\pgfusepath{stroke,fill}%
\end{pgfscope}%
\begin{pgfscope}%
\pgfpathrectangle{\pgfqpoint{0.100000in}{0.220728in}}{\pgfqpoint{3.696000in}{3.696000in}}%
\pgfusepath{clip}%
\pgfsetbuttcap%
\pgfsetroundjoin%
\definecolor{currentfill}{rgb}{0.121569,0.466667,0.705882}%
\pgfsetfillcolor{currentfill}%
\pgfsetfillopacity{0.619878}%
\pgfsetlinewidth{1.003750pt}%
\definecolor{currentstroke}{rgb}{0.121569,0.466667,0.705882}%
\pgfsetstrokecolor{currentstroke}%
\pgfsetstrokeopacity{0.619878}%
\pgfsetdash{}{0pt}%
\pgfpathmoveto{\pgfqpoint{0.679199in}{1.199621in}}%
\pgfpathcurveto{\pgfqpoint{0.687435in}{1.199621in}}{\pgfqpoint{0.695335in}{1.202893in}}{\pgfqpoint{0.701159in}{1.208717in}}%
\pgfpathcurveto{\pgfqpoint{0.706983in}{1.214541in}}{\pgfqpoint{0.710255in}{1.222441in}}{\pgfqpoint{0.710255in}{1.230677in}}%
\pgfpathcurveto{\pgfqpoint{0.710255in}{1.238914in}}{\pgfqpoint{0.706983in}{1.246814in}}{\pgfqpoint{0.701159in}{1.252638in}}%
\pgfpathcurveto{\pgfqpoint{0.695335in}{1.258462in}}{\pgfqpoint{0.687435in}{1.261734in}}{\pgfqpoint{0.679199in}{1.261734in}}%
\pgfpathcurveto{\pgfqpoint{0.670963in}{1.261734in}}{\pgfqpoint{0.663063in}{1.258462in}}{\pgfqpoint{0.657239in}{1.252638in}}%
\pgfpathcurveto{\pgfqpoint{0.651415in}{1.246814in}}{\pgfqpoint{0.648142in}{1.238914in}}{\pgfqpoint{0.648142in}{1.230677in}}%
\pgfpathcurveto{\pgfqpoint{0.648142in}{1.222441in}}{\pgfqpoint{0.651415in}{1.214541in}}{\pgfqpoint{0.657239in}{1.208717in}}%
\pgfpathcurveto{\pgfqpoint{0.663063in}{1.202893in}}{\pgfqpoint{0.670963in}{1.199621in}}{\pgfqpoint{0.679199in}{1.199621in}}%
\pgfpathclose%
\pgfusepath{stroke,fill}%
\end{pgfscope}%
\begin{pgfscope}%
\pgfpathrectangle{\pgfqpoint{0.100000in}{0.220728in}}{\pgfqpoint{3.696000in}{3.696000in}}%
\pgfusepath{clip}%
\pgfsetbuttcap%
\pgfsetroundjoin%
\definecolor{currentfill}{rgb}{0.121569,0.466667,0.705882}%
\pgfsetfillcolor{currentfill}%
\pgfsetfillopacity{0.619878}%
\pgfsetlinewidth{1.003750pt}%
\definecolor{currentstroke}{rgb}{0.121569,0.466667,0.705882}%
\pgfsetstrokecolor{currentstroke}%
\pgfsetstrokeopacity{0.619878}%
\pgfsetdash{}{0pt}%
\pgfpathmoveto{\pgfqpoint{0.679199in}{1.199621in}}%
\pgfpathcurveto{\pgfqpoint{0.687435in}{1.199621in}}{\pgfqpoint{0.695335in}{1.202893in}}{\pgfqpoint{0.701159in}{1.208717in}}%
\pgfpathcurveto{\pgfqpoint{0.706983in}{1.214541in}}{\pgfqpoint{0.710255in}{1.222441in}}{\pgfqpoint{0.710255in}{1.230677in}}%
\pgfpathcurveto{\pgfqpoint{0.710255in}{1.238914in}}{\pgfqpoint{0.706983in}{1.246814in}}{\pgfqpoint{0.701159in}{1.252638in}}%
\pgfpathcurveto{\pgfqpoint{0.695335in}{1.258462in}}{\pgfqpoint{0.687435in}{1.261734in}}{\pgfqpoint{0.679199in}{1.261734in}}%
\pgfpathcurveto{\pgfqpoint{0.670963in}{1.261734in}}{\pgfqpoint{0.663063in}{1.258462in}}{\pgfqpoint{0.657239in}{1.252638in}}%
\pgfpathcurveto{\pgfqpoint{0.651415in}{1.246814in}}{\pgfqpoint{0.648142in}{1.238914in}}{\pgfqpoint{0.648142in}{1.230677in}}%
\pgfpathcurveto{\pgfqpoint{0.648142in}{1.222441in}}{\pgfqpoint{0.651415in}{1.214541in}}{\pgfqpoint{0.657239in}{1.208717in}}%
\pgfpathcurveto{\pgfqpoint{0.663063in}{1.202893in}}{\pgfqpoint{0.670963in}{1.199621in}}{\pgfqpoint{0.679199in}{1.199621in}}%
\pgfpathclose%
\pgfusepath{stroke,fill}%
\end{pgfscope}%
\begin{pgfscope}%
\pgfpathrectangle{\pgfqpoint{0.100000in}{0.220728in}}{\pgfqpoint{3.696000in}{3.696000in}}%
\pgfusepath{clip}%
\pgfsetbuttcap%
\pgfsetroundjoin%
\definecolor{currentfill}{rgb}{0.121569,0.466667,0.705882}%
\pgfsetfillcolor{currentfill}%
\pgfsetfillopacity{0.619878}%
\pgfsetlinewidth{1.003750pt}%
\definecolor{currentstroke}{rgb}{0.121569,0.466667,0.705882}%
\pgfsetstrokecolor{currentstroke}%
\pgfsetstrokeopacity{0.619878}%
\pgfsetdash{}{0pt}%
\pgfpathmoveto{\pgfqpoint{0.679199in}{1.199621in}}%
\pgfpathcurveto{\pgfqpoint{0.687435in}{1.199621in}}{\pgfqpoint{0.695335in}{1.202893in}}{\pgfqpoint{0.701159in}{1.208717in}}%
\pgfpathcurveto{\pgfqpoint{0.706983in}{1.214541in}}{\pgfqpoint{0.710255in}{1.222441in}}{\pgfqpoint{0.710255in}{1.230677in}}%
\pgfpathcurveto{\pgfqpoint{0.710255in}{1.238914in}}{\pgfqpoint{0.706983in}{1.246814in}}{\pgfqpoint{0.701159in}{1.252638in}}%
\pgfpathcurveto{\pgfqpoint{0.695335in}{1.258462in}}{\pgfqpoint{0.687435in}{1.261734in}}{\pgfqpoint{0.679199in}{1.261734in}}%
\pgfpathcurveto{\pgfqpoint{0.670963in}{1.261734in}}{\pgfqpoint{0.663063in}{1.258462in}}{\pgfqpoint{0.657239in}{1.252638in}}%
\pgfpathcurveto{\pgfqpoint{0.651415in}{1.246814in}}{\pgfqpoint{0.648142in}{1.238914in}}{\pgfqpoint{0.648142in}{1.230677in}}%
\pgfpathcurveto{\pgfqpoint{0.648142in}{1.222441in}}{\pgfqpoint{0.651415in}{1.214541in}}{\pgfqpoint{0.657239in}{1.208717in}}%
\pgfpathcurveto{\pgfqpoint{0.663063in}{1.202893in}}{\pgfqpoint{0.670963in}{1.199621in}}{\pgfqpoint{0.679199in}{1.199621in}}%
\pgfpathclose%
\pgfusepath{stroke,fill}%
\end{pgfscope}%
\begin{pgfscope}%
\pgfpathrectangle{\pgfqpoint{0.100000in}{0.220728in}}{\pgfqpoint{3.696000in}{3.696000in}}%
\pgfusepath{clip}%
\pgfsetbuttcap%
\pgfsetroundjoin%
\definecolor{currentfill}{rgb}{0.121569,0.466667,0.705882}%
\pgfsetfillcolor{currentfill}%
\pgfsetfillopacity{0.619878}%
\pgfsetlinewidth{1.003750pt}%
\definecolor{currentstroke}{rgb}{0.121569,0.466667,0.705882}%
\pgfsetstrokecolor{currentstroke}%
\pgfsetstrokeopacity{0.619878}%
\pgfsetdash{}{0pt}%
\pgfpathmoveto{\pgfqpoint{0.679199in}{1.199621in}}%
\pgfpathcurveto{\pgfqpoint{0.687435in}{1.199621in}}{\pgfqpoint{0.695335in}{1.202893in}}{\pgfqpoint{0.701159in}{1.208717in}}%
\pgfpathcurveto{\pgfqpoint{0.706983in}{1.214541in}}{\pgfqpoint{0.710255in}{1.222441in}}{\pgfqpoint{0.710255in}{1.230677in}}%
\pgfpathcurveto{\pgfqpoint{0.710255in}{1.238914in}}{\pgfqpoint{0.706983in}{1.246814in}}{\pgfqpoint{0.701159in}{1.252638in}}%
\pgfpathcurveto{\pgfqpoint{0.695335in}{1.258462in}}{\pgfqpoint{0.687435in}{1.261734in}}{\pgfqpoint{0.679199in}{1.261734in}}%
\pgfpathcurveto{\pgfqpoint{0.670963in}{1.261734in}}{\pgfqpoint{0.663063in}{1.258462in}}{\pgfqpoint{0.657239in}{1.252638in}}%
\pgfpathcurveto{\pgfqpoint{0.651415in}{1.246814in}}{\pgfqpoint{0.648142in}{1.238914in}}{\pgfqpoint{0.648142in}{1.230677in}}%
\pgfpathcurveto{\pgfqpoint{0.648142in}{1.222441in}}{\pgfqpoint{0.651415in}{1.214541in}}{\pgfqpoint{0.657239in}{1.208717in}}%
\pgfpathcurveto{\pgfqpoint{0.663063in}{1.202893in}}{\pgfqpoint{0.670963in}{1.199621in}}{\pgfqpoint{0.679199in}{1.199621in}}%
\pgfpathclose%
\pgfusepath{stroke,fill}%
\end{pgfscope}%
\begin{pgfscope}%
\pgfpathrectangle{\pgfqpoint{0.100000in}{0.220728in}}{\pgfqpoint{3.696000in}{3.696000in}}%
\pgfusepath{clip}%
\pgfsetbuttcap%
\pgfsetroundjoin%
\definecolor{currentfill}{rgb}{0.121569,0.466667,0.705882}%
\pgfsetfillcolor{currentfill}%
\pgfsetfillopacity{0.619878}%
\pgfsetlinewidth{1.003750pt}%
\definecolor{currentstroke}{rgb}{0.121569,0.466667,0.705882}%
\pgfsetstrokecolor{currentstroke}%
\pgfsetstrokeopacity{0.619878}%
\pgfsetdash{}{0pt}%
\pgfpathmoveto{\pgfqpoint{0.679199in}{1.199621in}}%
\pgfpathcurveto{\pgfqpoint{0.687435in}{1.199621in}}{\pgfqpoint{0.695335in}{1.202893in}}{\pgfqpoint{0.701159in}{1.208717in}}%
\pgfpathcurveto{\pgfqpoint{0.706983in}{1.214541in}}{\pgfqpoint{0.710255in}{1.222441in}}{\pgfqpoint{0.710255in}{1.230677in}}%
\pgfpathcurveto{\pgfqpoint{0.710255in}{1.238914in}}{\pgfqpoint{0.706983in}{1.246814in}}{\pgfqpoint{0.701159in}{1.252638in}}%
\pgfpathcurveto{\pgfqpoint{0.695335in}{1.258462in}}{\pgfqpoint{0.687435in}{1.261734in}}{\pgfqpoint{0.679199in}{1.261734in}}%
\pgfpathcurveto{\pgfqpoint{0.670963in}{1.261734in}}{\pgfqpoint{0.663063in}{1.258462in}}{\pgfqpoint{0.657239in}{1.252638in}}%
\pgfpathcurveto{\pgfqpoint{0.651415in}{1.246814in}}{\pgfqpoint{0.648142in}{1.238914in}}{\pgfqpoint{0.648142in}{1.230677in}}%
\pgfpathcurveto{\pgfqpoint{0.648142in}{1.222441in}}{\pgfqpoint{0.651415in}{1.214541in}}{\pgfqpoint{0.657239in}{1.208717in}}%
\pgfpathcurveto{\pgfqpoint{0.663063in}{1.202893in}}{\pgfqpoint{0.670963in}{1.199621in}}{\pgfqpoint{0.679199in}{1.199621in}}%
\pgfpathclose%
\pgfusepath{stroke,fill}%
\end{pgfscope}%
\begin{pgfscope}%
\pgfpathrectangle{\pgfqpoint{0.100000in}{0.220728in}}{\pgfqpoint{3.696000in}{3.696000in}}%
\pgfusepath{clip}%
\pgfsetbuttcap%
\pgfsetroundjoin%
\definecolor{currentfill}{rgb}{0.121569,0.466667,0.705882}%
\pgfsetfillcolor{currentfill}%
\pgfsetfillopacity{0.619878}%
\pgfsetlinewidth{1.003750pt}%
\definecolor{currentstroke}{rgb}{0.121569,0.466667,0.705882}%
\pgfsetstrokecolor{currentstroke}%
\pgfsetstrokeopacity{0.619878}%
\pgfsetdash{}{0pt}%
\pgfpathmoveto{\pgfqpoint{0.679199in}{1.199621in}}%
\pgfpathcurveto{\pgfqpoint{0.687435in}{1.199621in}}{\pgfqpoint{0.695335in}{1.202893in}}{\pgfqpoint{0.701159in}{1.208717in}}%
\pgfpathcurveto{\pgfqpoint{0.706983in}{1.214541in}}{\pgfqpoint{0.710255in}{1.222441in}}{\pgfqpoint{0.710255in}{1.230677in}}%
\pgfpathcurveto{\pgfqpoint{0.710255in}{1.238914in}}{\pgfqpoint{0.706983in}{1.246814in}}{\pgfqpoint{0.701159in}{1.252638in}}%
\pgfpathcurveto{\pgfqpoint{0.695335in}{1.258462in}}{\pgfqpoint{0.687435in}{1.261734in}}{\pgfqpoint{0.679199in}{1.261734in}}%
\pgfpathcurveto{\pgfqpoint{0.670963in}{1.261734in}}{\pgfqpoint{0.663063in}{1.258462in}}{\pgfqpoint{0.657239in}{1.252638in}}%
\pgfpathcurveto{\pgfqpoint{0.651415in}{1.246814in}}{\pgfqpoint{0.648142in}{1.238914in}}{\pgfqpoint{0.648142in}{1.230677in}}%
\pgfpathcurveto{\pgfqpoint{0.648142in}{1.222441in}}{\pgfqpoint{0.651415in}{1.214541in}}{\pgfqpoint{0.657239in}{1.208717in}}%
\pgfpathcurveto{\pgfqpoint{0.663063in}{1.202893in}}{\pgfqpoint{0.670963in}{1.199621in}}{\pgfqpoint{0.679199in}{1.199621in}}%
\pgfpathclose%
\pgfusepath{stroke,fill}%
\end{pgfscope}%
\begin{pgfscope}%
\pgfpathrectangle{\pgfqpoint{0.100000in}{0.220728in}}{\pgfqpoint{3.696000in}{3.696000in}}%
\pgfusepath{clip}%
\pgfsetbuttcap%
\pgfsetroundjoin%
\definecolor{currentfill}{rgb}{0.121569,0.466667,0.705882}%
\pgfsetfillcolor{currentfill}%
\pgfsetfillopacity{0.619878}%
\pgfsetlinewidth{1.003750pt}%
\definecolor{currentstroke}{rgb}{0.121569,0.466667,0.705882}%
\pgfsetstrokecolor{currentstroke}%
\pgfsetstrokeopacity{0.619878}%
\pgfsetdash{}{0pt}%
\pgfpathmoveto{\pgfqpoint{0.679199in}{1.199621in}}%
\pgfpathcurveto{\pgfqpoint{0.687435in}{1.199621in}}{\pgfqpoint{0.695335in}{1.202893in}}{\pgfqpoint{0.701159in}{1.208717in}}%
\pgfpathcurveto{\pgfqpoint{0.706983in}{1.214541in}}{\pgfqpoint{0.710255in}{1.222441in}}{\pgfqpoint{0.710255in}{1.230677in}}%
\pgfpathcurveto{\pgfqpoint{0.710255in}{1.238914in}}{\pgfqpoint{0.706983in}{1.246814in}}{\pgfqpoint{0.701159in}{1.252638in}}%
\pgfpathcurveto{\pgfqpoint{0.695335in}{1.258462in}}{\pgfqpoint{0.687435in}{1.261734in}}{\pgfqpoint{0.679199in}{1.261734in}}%
\pgfpathcurveto{\pgfqpoint{0.670963in}{1.261734in}}{\pgfqpoint{0.663063in}{1.258462in}}{\pgfqpoint{0.657239in}{1.252638in}}%
\pgfpathcurveto{\pgfqpoint{0.651415in}{1.246814in}}{\pgfqpoint{0.648142in}{1.238914in}}{\pgfqpoint{0.648142in}{1.230677in}}%
\pgfpathcurveto{\pgfqpoint{0.648142in}{1.222441in}}{\pgfqpoint{0.651415in}{1.214541in}}{\pgfqpoint{0.657239in}{1.208717in}}%
\pgfpathcurveto{\pgfqpoint{0.663063in}{1.202893in}}{\pgfqpoint{0.670963in}{1.199621in}}{\pgfqpoint{0.679199in}{1.199621in}}%
\pgfpathclose%
\pgfusepath{stroke,fill}%
\end{pgfscope}%
\begin{pgfscope}%
\pgfpathrectangle{\pgfqpoint{0.100000in}{0.220728in}}{\pgfqpoint{3.696000in}{3.696000in}}%
\pgfusepath{clip}%
\pgfsetbuttcap%
\pgfsetroundjoin%
\definecolor{currentfill}{rgb}{0.121569,0.466667,0.705882}%
\pgfsetfillcolor{currentfill}%
\pgfsetfillopacity{0.619878}%
\pgfsetlinewidth{1.003750pt}%
\definecolor{currentstroke}{rgb}{0.121569,0.466667,0.705882}%
\pgfsetstrokecolor{currentstroke}%
\pgfsetstrokeopacity{0.619878}%
\pgfsetdash{}{0pt}%
\pgfpathmoveto{\pgfqpoint{0.679199in}{1.199621in}}%
\pgfpathcurveto{\pgfqpoint{0.687435in}{1.199621in}}{\pgfqpoint{0.695335in}{1.202893in}}{\pgfqpoint{0.701159in}{1.208717in}}%
\pgfpathcurveto{\pgfqpoint{0.706983in}{1.214541in}}{\pgfqpoint{0.710255in}{1.222441in}}{\pgfqpoint{0.710255in}{1.230677in}}%
\pgfpathcurveto{\pgfqpoint{0.710255in}{1.238914in}}{\pgfqpoint{0.706983in}{1.246814in}}{\pgfqpoint{0.701159in}{1.252638in}}%
\pgfpathcurveto{\pgfqpoint{0.695335in}{1.258462in}}{\pgfqpoint{0.687435in}{1.261734in}}{\pgfqpoint{0.679199in}{1.261734in}}%
\pgfpathcurveto{\pgfqpoint{0.670963in}{1.261734in}}{\pgfqpoint{0.663063in}{1.258462in}}{\pgfqpoint{0.657239in}{1.252638in}}%
\pgfpathcurveto{\pgfqpoint{0.651415in}{1.246814in}}{\pgfqpoint{0.648142in}{1.238914in}}{\pgfqpoint{0.648142in}{1.230677in}}%
\pgfpathcurveto{\pgfqpoint{0.648142in}{1.222441in}}{\pgfqpoint{0.651415in}{1.214541in}}{\pgfqpoint{0.657239in}{1.208717in}}%
\pgfpathcurveto{\pgfqpoint{0.663063in}{1.202893in}}{\pgfqpoint{0.670963in}{1.199621in}}{\pgfqpoint{0.679199in}{1.199621in}}%
\pgfpathclose%
\pgfusepath{stroke,fill}%
\end{pgfscope}%
\begin{pgfscope}%
\pgfpathrectangle{\pgfqpoint{0.100000in}{0.220728in}}{\pgfqpoint{3.696000in}{3.696000in}}%
\pgfusepath{clip}%
\pgfsetbuttcap%
\pgfsetroundjoin%
\definecolor{currentfill}{rgb}{0.121569,0.466667,0.705882}%
\pgfsetfillcolor{currentfill}%
\pgfsetfillopacity{0.619878}%
\pgfsetlinewidth{1.003750pt}%
\definecolor{currentstroke}{rgb}{0.121569,0.466667,0.705882}%
\pgfsetstrokecolor{currentstroke}%
\pgfsetstrokeopacity{0.619878}%
\pgfsetdash{}{0pt}%
\pgfpathmoveto{\pgfqpoint{0.679199in}{1.199621in}}%
\pgfpathcurveto{\pgfqpoint{0.687435in}{1.199621in}}{\pgfqpoint{0.695335in}{1.202893in}}{\pgfqpoint{0.701159in}{1.208717in}}%
\pgfpathcurveto{\pgfqpoint{0.706983in}{1.214541in}}{\pgfqpoint{0.710255in}{1.222441in}}{\pgfqpoint{0.710255in}{1.230677in}}%
\pgfpathcurveto{\pgfqpoint{0.710255in}{1.238914in}}{\pgfqpoint{0.706983in}{1.246814in}}{\pgfqpoint{0.701159in}{1.252638in}}%
\pgfpathcurveto{\pgfqpoint{0.695335in}{1.258462in}}{\pgfqpoint{0.687435in}{1.261734in}}{\pgfqpoint{0.679199in}{1.261734in}}%
\pgfpathcurveto{\pgfqpoint{0.670963in}{1.261734in}}{\pgfqpoint{0.663063in}{1.258462in}}{\pgfqpoint{0.657239in}{1.252638in}}%
\pgfpathcurveto{\pgfqpoint{0.651415in}{1.246814in}}{\pgfqpoint{0.648142in}{1.238914in}}{\pgfqpoint{0.648142in}{1.230677in}}%
\pgfpathcurveto{\pgfqpoint{0.648142in}{1.222441in}}{\pgfqpoint{0.651415in}{1.214541in}}{\pgfqpoint{0.657239in}{1.208717in}}%
\pgfpathcurveto{\pgfqpoint{0.663063in}{1.202893in}}{\pgfqpoint{0.670963in}{1.199621in}}{\pgfqpoint{0.679199in}{1.199621in}}%
\pgfpathclose%
\pgfusepath{stroke,fill}%
\end{pgfscope}%
\begin{pgfscope}%
\pgfpathrectangle{\pgfqpoint{0.100000in}{0.220728in}}{\pgfqpoint{3.696000in}{3.696000in}}%
\pgfusepath{clip}%
\pgfsetbuttcap%
\pgfsetroundjoin%
\definecolor{currentfill}{rgb}{0.121569,0.466667,0.705882}%
\pgfsetfillcolor{currentfill}%
\pgfsetfillopacity{0.619878}%
\pgfsetlinewidth{1.003750pt}%
\definecolor{currentstroke}{rgb}{0.121569,0.466667,0.705882}%
\pgfsetstrokecolor{currentstroke}%
\pgfsetstrokeopacity{0.619878}%
\pgfsetdash{}{0pt}%
\pgfpathmoveto{\pgfqpoint{0.679199in}{1.199621in}}%
\pgfpathcurveto{\pgfqpoint{0.687435in}{1.199621in}}{\pgfqpoint{0.695335in}{1.202893in}}{\pgfqpoint{0.701159in}{1.208717in}}%
\pgfpathcurveto{\pgfqpoint{0.706983in}{1.214541in}}{\pgfqpoint{0.710255in}{1.222441in}}{\pgfqpoint{0.710255in}{1.230677in}}%
\pgfpathcurveto{\pgfqpoint{0.710255in}{1.238914in}}{\pgfqpoint{0.706983in}{1.246814in}}{\pgfqpoint{0.701159in}{1.252638in}}%
\pgfpathcurveto{\pgfqpoint{0.695335in}{1.258462in}}{\pgfqpoint{0.687435in}{1.261734in}}{\pgfqpoint{0.679199in}{1.261734in}}%
\pgfpathcurveto{\pgfqpoint{0.670963in}{1.261734in}}{\pgfqpoint{0.663063in}{1.258462in}}{\pgfqpoint{0.657239in}{1.252638in}}%
\pgfpathcurveto{\pgfqpoint{0.651415in}{1.246814in}}{\pgfqpoint{0.648142in}{1.238914in}}{\pgfqpoint{0.648142in}{1.230677in}}%
\pgfpathcurveto{\pgfqpoint{0.648142in}{1.222441in}}{\pgfqpoint{0.651415in}{1.214541in}}{\pgfqpoint{0.657239in}{1.208717in}}%
\pgfpathcurveto{\pgfqpoint{0.663063in}{1.202893in}}{\pgfqpoint{0.670963in}{1.199621in}}{\pgfqpoint{0.679199in}{1.199621in}}%
\pgfpathclose%
\pgfusepath{stroke,fill}%
\end{pgfscope}%
\begin{pgfscope}%
\pgfpathrectangle{\pgfqpoint{0.100000in}{0.220728in}}{\pgfqpoint{3.696000in}{3.696000in}}%
\pgfusepath{clip}%
\pgfsetbuttcap%
\pgfsetroundjoin%
\definecolor{currentfill}{rgb}{0.121569,0.466667,0.705882}%
\pgfsetfillcolor{currentfill}%
\pgfsetfillopacity{0.619878}%
\pgfsetlinewidth{1.003750pt}%
\definecolor{currentstroke}{rgb}{0.121569,0.466667,0.705882}%
\pgfsetstrokecolor{currentstroke}%
\pgfsetstrokeopacity{0.619878}%
\pgfsetdash{}{0pt}%
\pgfpathmoveto{\pgfqpoint{0.679199in}{1.199621in}}%
\pgfpathcurveto{\pgfqpoint{0.687435in}{1.199621in}}{\pgfqpoint{0.695335in}{1.202893in}}{\pgfqpoint{0.701159in}{1.208717in}}%
\pgfpathcurveto{\pgfqpoint{0.706983in}{1.214541in}}{\pgfqpoint{0.710255in}{1.222441in}}{\pgfqpoint{0.710255in}{1.230677in}}%
\pgfpathcurveto{\pgfqpoint{0.710255in}{1.238914in}}{\pgfqpoint{0.706983in}{1.246814in}}{\pgfqpoint{0.701159in}{1.252638in}}%
\pgfpathcurveto{\pgfqpoint{0.695335in}{1.258462in}}{\pgfqpoint{0.687435in}{1.261734in}}{\pgfqpoint{0.679199in}{1.261734in}}%
\pgfpathcurveto{\pgfqpoint{0.670963in}{1.261734in}}{\pgfqpoint{0.663063in}{1.258462in}}{\pgfqpoint{0.657239in}{1.252638in}}%
\pgfpathcurveto{\pgfqpoint{0.651415in}{1.246814in}}{\pgfqpoint{0.648142in}{1.238914in}}{\pgfqpoint{0.648142in}{1.230677in}}%
\pgfpathcurveto{\pgfqpoint{0.648142in}{1.222441in}}{\pgfqpoint{0.651415in}{1.214541in}}{\pgfqpoint{0.657239in}{1.208717in}}%
\pgfpathcurveto{\pgfqpoint{0.663063in}{1.202893in}}{\pgfqpoint{0.670963in}{1.199621in}}{\pgfqpoint{0.679199in}{1.199621in}}%
\pgfpathclose%
\pgfusepath{stroke,fill}%
\end{pgfscope}%
\begin{pgfscope}%
\pgfpathrectangle{\pgfqpoint{0.100000in}{0.220728in}}{\pgfqpoint{3.696000in}{3.696000in}}%
\pgfusepath{clip}%
\pgfsetbuttcap%
\pgfsetroundjoin%
\definecolor{currentfill}{rgb}{0.121569,0.466667,0.705882}%
\pgfsetfillcolor{currentfill}%
\pgfsetfillopacity{0.619878}%
\pgfsetlinewidth{1.003750pt}%
\definecolor{currentstroke}{rgb}{0.121569,0.466667,0.705882}%
\pgfsetstrokecolor{currentstroke}%
\pgfsetstrokeopacity{0.619878}%
\pgfsetdash{}{0pt}%
\pgfpathmoveto{\pgfqpoint{0.679199in}{1.199621in}}%
\pgfpathcurveto{\pgfqpoint{0.687435in}{1.199621in}}{\pgfqpoint{0.695335in}{1.202893in}}{\pgfqpoint{0.701159in}{1.208717in}}%
\pgfpathcurveto{\pgfqpoint{0.706983in}{1.214541in}}{\pgfqpoint{0.710255in}{1.222441in}}{\pgfqpoint{0.710255in}{1.230677in}}%
\pgfpathcurveto{\pgfqpoint{0.710255in}{1.238914in}}{\pgfqpoint{0.706983in}{1.246814in}}{\pgfqpoint{0.701159in}{1.252638in}}%
\pgfpathcurveto{\pgfqpoint{0.695335in}{1.258462in}}{\pgfqpoint{0.687435in}{1.261734in}}{\pgfqpoint{0.679199in}{1.261734in}}%
\pgfpathcurveto{\pgfqpoint{0.670963in}{1.261734in}}{\pgfqpoint{0.663063in}{1.258462in}}{\pgfqpoint{0.657239in}{1.252638in}}%
\pgfpathcurveto{\pgfqpoint{0.651415in}{1.246814in}}{\pgfqpoint{0.648142in}{1.238914in}}{\pgfqpoint{0.648142in}{1.230677in}}%
\pgfpathcurveto{\pgfqpoint{0.648142in}{1.222441in}}{\pgfqpoint{0.651415in}{1.214541in}}{\pgfqpoint{0.657239in}{1.208717in}}%
\pgfpathcurveto{\pgfqpoint{0.663063in}{1.202893in}}{\pgfqpoint{0.670963in}{1.199621in}}{\pgfqpoint{0.679199in}{1.199621in}}%
\pgfpathclose%
\pgfusepath{stroke,fill}%
\end{pgfscope}%
\begin{pgfscope}%
\pgfpathrectangle{\pgfqpoint{0.100000in}{0.220728in}}{\pgfqpoint{3.696000in}{3.696000in}}%
\pgfusepath{clip}%
\pgfsetbuttcap%
\pgfsetroundjoin%
\definecolor{currentfill}{rgb}{0.121569,0.466667,0.705882}%
\pgfsetfillcolor{currentfill}%
\pgfsetfillopacity{0.619878}%
\pgfsetlinewidth{1.003750pt}%
\definecolor{currentstroke}{rgb}{0.121569,0.466667,0.705882}%
\pgfsetstrokecolor{currentstroke}%
\pgfsetstrokeopacity{0.619878}%
\pgfsetdash{}{0pt}%
\pgfpathmoveto{\pgfqpoint{0.679199in}{1.199621in}}%
\pgfpathcurveto{\pgfqpoint{0.687435in}{1.199621in}}{\pgfqpoint{0.695335in}{1.202893in}}{\pgfqpoint{0.701159in}{1.208717in}}%
\pgfpathcurveto{\pgfqpoint{0.706983in}{1.214541in}}{\pgfqpoint{0.710255in}{1.222441in}}{\pgfqpoint{0.710255in}{1.230677in}}%
\pgfpathcurveto{\pgfqpoint{0.710255in}{1.238914in}}{\pgfqpoint{0.706983in}{1.246814in}}{\pgfqpoint{0.701159in}{1.252638in}}%
\pgfpathcurveto{\pgfqpoint{0.695335in}{1.258462in}}{\pgfqpoint{0.687435in}{1.261734in}}{\pgfqpoint{0.679199in}{1.261734in}}%
\pgfpathcurveto{\pgfqpoint{0.670963in}{1.261734in}}{\pgfqpoint{0.663063in}{1.258462in}}{\pgfqpoint{0.657239in}{1.252638in}}%
\pgfpathcurveto{\pgfqpoint{0.651415in}{1.246814in}}{\pgfqpoint{0.648142in}{1.238914in}}{\pgfqpoint{0.648142in}{1.230677in}}%
\pgfpathcurveto{\pgfqpoint{0.648142in}{1.222441in}}{\pgfqpoint{0.651415in}{1.214541in}}{\pgfqpoint{0.657239in}{1.208717in}}%
\pgfpathcurveto{\pgfqpoint{0.663063in}{1.202893in}}{\pgfqpoint{0.670963in}{1.199621in}}{\pgfqpoint{0.679199in}{1.199621in}}%
\pgfpathclose%
\pgfusepath{stroke,fill}%
\end{pgfscope}%
\begin{pgfscope}%
\pgfpathrectangle{\pgfqpoint{0.100000in}{0.220728in}}{\pgfqpoint{3.696000in}{3.696000in}}%
\pgfusepath{clip}%
\pgfsetbuttcap%
\pgfsetroundjoin%
\definecolor{currentfill}{rgb}{0.121569,0.466667,0.705882}%
\pgfsetfillcolor{currentfill}%
\pgfsetfillopacity{0.619878}%
\pgfsetlinewidth{1.003750pt}%
\definecolor{currentstroke}{rgb}{0.121569,0.466667,0.705882}%
\pgfsetstrokecolor{currentstroke}%
\pgfsetstrokeopacity{0.619878}%
\pgfsetdash{}{0pt}%
\pgfpathmoveto{\pgfqpoint{0.679199in}{1.199621in}}%
\pgfpathcurveto{\pgfqpoint{0.687435in}{1.199621in}}{\pgfqpoint{0.695335in}{1.202893in}}{\pgfqpoint{0.701159in}{1.208717in}}%
\pgfpathcurveto{\pgfqpoint{0.706983in}{1.214541in}}{\pgfqpoint{0.710255in}{1.222441in}}{\pgfqpoint{0.710255in}{1.230677in}}%
\pgfpathcurveto{\pgfqpoint{0.710255in}{1.238914in}}{\pgfqpoint{0.706983in}{1.246814in}}{\pgfqpoint{0.701159in}{1.252638in}}%
\pgfpathcurveto{\pgfqpoint{0.695335in}{1.258462in}}{\pgfqpoint{0.687435in}{1.261734in}}{\pgfqpoint{0.679199in}{1.261734in}}%
\pgfpathcurveto{\pgfqpoint{0.670963in}{1.261734in}}{\pgfqpoint{0.663063in}{1.258462in}}{\pgfqpoint{0.657239in}{1.252638in}}%
\pgfpathcurveto{\pgfqpoint{0.651415in}{1.246814in}}{\pgfqpoint{0.648142in}{1.238914in}}{\pgfqpoint{0.648142in}{1.230677in}}%
\pgfpathcurveto{\pgfqpoint{0.648142in}{1.222441in}}{\pgfqpoint{0.651415in}{1.214541in}}{\pgfqpoint{0.657239in}{1.208717in}}%
\pgfpathcurveto{\pgfqpoint{0.663063in}{1.202893in}}{\pgfqpoint{0.670963in}{1.199621in}}{\pgfqpoint{0.679199in}{1.199621in}}%
\pgfpathclose%
\pgfusepath{stroke,fill}%
\end{pgfscope}%
\begin{pgfscope}%
\pgfpathrectangle{\pgfqpoint{0.100000in}{0.220728in}}{\pgfqpoint{3.696000in}{3.696000in}}%
\pgfusepath{clip}%
\pgfsetbuttcap%
\pgfsetroundjoin%
\definecolor{currentfill}{rgb}{0.121569,0.466667,0.705882}%
\pgfsetfillcolor{currentfill}%
\pgfsetfillopacity{0.619878}%
\pgfsetlinewidth{1.003750pt}%
\definecolor{currentstroke}{rgb}{0.121569,0.466667,0.705882}%
\pgfsetstrokecolor{currentstroke}%
\pgfsetstrokeopacity{0.619878}%
\pgfsetdash{}{0pt}%
\pgfpathmoveto{\pgfqpoint{0.679199in}{1.199621in}}%
\pgfpathcurveto{\pgfqpoint{0.687435in}{1.199621in}}{\pgfqpoint{0.695335in}{1.202893in}}{\pgfqpoint{0.701159in}{1.208717in}}%
\pgfpathcurveto{\pgfqpoint{0.706983in}{1.214541in}}{\pgfqpoint{0.710255in}{1.222441in}}{\pgfqpoint{0.710255in}{1.230677in}}%
\pgfpathcurveto{\pgfqpoint{0.710255in}{1.238914in}}{\pgfqpoint{0.706983in}{1.246814in}}{\pgfqpoint{0.701159in}{1.252638in}}%
\pgfpathcurveto{\pgfqpoint{0.695335in}{1.258462in}}{\pgfqpoint{0.687435in}{1.261734in}}{\pgfqpoint{0.679199in}{1.261734in}}%
\pgfpathcurveto{\pgfqpoint{0.670963in}{1.261734in}}{\pgfqpoint{0.663063in}{1.258462in}}{\pgfqpoint{0.657239in}{1.252638in}}%
\pgfpathcurveto{\pgfqpoint{0.651415in}{1.246814in}}{\pgfqpoint{0.648142in}{1.238914in}}{\pgfqpoint{0.648142in}{1.230677in}}%
\pgfpathcurveto{\pgfqpoint{0.648142in}{1.222441in}}{\pgfqpoint{0.651415in}{1.214541in}}{\pgfqpoint{0.657239in}{1.208717in}}%
\pgfpathcurveto{\pgfqpoint{0.663063in}{1.202893in}}{\pgfqpoint{0.670963in}{1.199621in}}{\pgfqpoint{0.679199in}{1.199621in}}%
\pgfpathclose%
\pgfusepath{stroke,fill}%
\end{pgfscope}%
\begin{pgfscope}%
\pgfpathrectangle{\pgfqpoint{0.100000in}{0.220728in}}{\pgfqpoint{3.696000in}{3.696000in}}%
\pgfusepath{clip}%
\pgfsetbuttcap%
\pgfsetroundjoin%
\definecolor{currentfill}{rgb}{0.121569,0.466667,0.705882}%
\pgfsetfillcolor{currentfill}%
\pgfsetfillopacity{0.619878}%
\pgfsetlinewidth{1.003750pt}%
\definecolor{currentstroke}{rgb}{0.121569,0.466667,0.705882}%
\pgfsetstrokecolor{currentstroke}%
\pgfsetstrokeopacity{0.619878}%
\pgfsetdash{}{0pt}%
\pgfpathmoveto{\pgfqpoint{0.679199in}{1.199621in}}%
\pgfpathcurveto{\pgfqpoint{0.687435in}{1.199621in}}{\pgfqpoint{0.695335in}{1.202893in}}{\pgfqpoint{0.701159in}{1.208717in}}%
\pgfpathcurveto{\pgfqpoint{0.706983in}{1.214541in}}{\pgfqpoint{0.710255in}{1.222441in}}{\pgfqpoint{0.710255in}{1.230677in}}%
\pgfpathcurveto{\pgfqpoint{0.710255in}{1.238914in}}{\pgfqpoint{0.706983in}{1.246814in}}{\pgfqpoint{0.701159in}{1.252638in}}%
\pgfpathcurveto{\pgfqpoint{0.695335in}{1.258462in}}{\pgfqpoint{0.687435in}{1.261734in}}{\pgfqpoint{0.679199in}{1.261734in}}%
\pgfpathcurveto{\pgfqpoint{0.670963in}{1.261734in}}{\pgfqpoint{0.663063in}{1.258462in}}{\pgfqpoint{0.657239in}{1.252638in}}%
\pgfpathcurveto{\pgfqpoint{0.651415in}{1.246814in}}{\pgfqpoint{0.648142in}{1.238914in}}{\pgfqpoint{0.648142in}{1.230677in}}%
\pgfpathcurveto{\pgfqpoint{0.648142in}{1.222441in}}{\pgfqpoint{0.651415in}{1.214541in}}{\pgfqpoint{0.657239in}{1.208717in}}%
\pgfpathcurveto{\pgfqpoint{0.663063in}{1.202893in}}{\pgfqpoint{0.670963in}{1.199621in}}{\pgfqpoint{0.679199in}{1.199621in}}%
\pgfpathclose%
\pgfusepath{stroke,fill}%
\end{pgfscope}%
\begin{pgfscope}%
\pgfpathrectangle{\pgfqpoint{0.100000in}{0.220728in}}{\pgfqpoint{3.696000in}{3.696000in}}%
\pgfusepath{clip}%
\pgfsetbuttcap%
\pgfsetroundjoin%
\definecolor{currentfill}{rgb}{0.121569,0.466667,0.705882}%
\pgfsetfillcolor{currentfill}%
\pgfsetfillopacity{0.619878}%
\pgfsetlinewidth{1.003750pt}%
\definecolor{currentstroke}{rgb}{0.121569,0.466667,0.705882}%
\pgfsetstrokecolor{currentstroke}%
\pgfsetstrokeopacity{0.619878}%
\pgfsetdash{}{0pt}%
\pgfpathmoveto{\pgfqpoint{0.679199in}{1.199621in}}%
\pgfpathcurveto{\pgfqpoint{0.687435in}{1.199621in}}{\pgfqpoint{0.695335in}{1.202893in}}{\pgfqpoint{0.701159in}{1.208717in}}%
\pgfpathcurveto{\pgfqpoint{0.706983in}{1.214541in}}{\pgfqpoint{0.710255in}{1.222441in}}{\pgfqpoint{0.710255in}{1.230677in}}%
\pgfpathcurveto{\pgfqpoint{0.710255in}{1.238914in}}{\pgfqpoint{0.706983in}{1.246814in}}{\pgfqpoint{0.701159in}{1.252638in}}%
\pgfpathcurveto{\pgfqpoint{0.695335in}{1.258462in}}{\pgfqpoint{0.687435in}{1.261734in}}{\pgfqpoint{0.679199in}{1.261734in}}%
\pgfpathcurveto{\pgfqpoint{0.670963in}{1.261734in}}{\pgfqpoint{0.663063in}{1.258462in}}{\pgfqpoint{0.657239in}{1.252638in}}%
\pgfpathcurveto{\pgfqpoint{0.651415in}{1.246814in}}{\pgfqpoint{0.648142in}{1.238914in}}{\pgfqpoint{0.648142in}{1.230677in}}%
\pgfpathcurveto{\pgfqpoint{0.648142in}{1.222441in}}{\pgfqpoint{0.651415in}{1.214541in}}{\pgfqpoint{0.657239in}{1.208717in}}%
\pgfpathcurveto{\pgfqpoint{0.663063in}{1.202893in}}{\pgfqpoint{0.670963in}{1.199621in}}{\pgfqpoint{0.679199in}{1.199621in}}%
\pgfpathclose%
\pgfusepath{stroke,fill}%
\end{pgfscope}%
\begin{pgfscope}%
\pgfpathrectangle{\pgfqpoint{0.100000in}{0.220728in}}{\pgfqpoint{3.696000in}{3.696000in}}%
\pgfusepath{clip}%
\pgfsetbuttcap%
\pgfsetroundjoin%
\definecolor{currentfill}{rgb}{0.121569,0.466667,0.705882}%
\pgfsetfillcolor{currentfill}%
\pgfsetfillopacity{0.619878}%
\pgfsetlinewidth{1.003750pt}%
\definecolor{currentstroke}{rgb}{0.121569,0.466667,0.705882}%
\pgfsetstrokecolor{currentstroke}%
\pgfsetstrokeopacity{0.619878}%
\pgfsetdash{}{0pt}%
\pgfpathmoveto{\pgfqpoint{0.679199in}{1.199621in}}%
\pgfpathcurveto{\pgfqpoint{0.687435in}{1.199621in}}{\pgfqpoint{0.695335in}{1.202893in}}{\pgfqpoint{0.701159in}{1.208717in}}%
\pgfpathcurveto{\pgfqpoint{0.706983in}{1.214541in}}{\pgfqpoint{0.710255in}{1.222441in}}{\pgfqpoint{0.710255in}{1.230677in}}%
\pgfpathcurveto{\pgfqpoint{0.710255in}{1.238914in}}{\pgfqpoint{0.706983in}{1.246814in}}{\pgfqpoint{0.701159in}{1.252638in}}%
\pgfpathcurveto{\pgfqpoint{0.695335in}{1.258462in}}{\pgfqpoint{0.687435in}{1.261734in}}{\pgfqpoint{0.679199in}{1.261734in}}%
\pgfpathcurveto{\pgfqpoint{0.670963in}{1.261734in}}{\pgfqpoint{0.663063in}{1.258462in}}{\pgfqpoint{0.657239in}{1.252638in}}%
\pgfpathcurveto{\pgfqpoint{0.651415in}{1.246814in}}{\pgfqpoint{0.648142in}{1.238914in}}{\pgfqpoint{0.648142in}{1.230677in}}%
\pgfpathcurveto{\pgfqpoint{0.648142in}{1.222441in}}{\pgfqpoint{0.651415in}{1.214541in}}{\pgfqpoint{0.657239in}{1.208717in}}%
\pgfpathcurveto{\pgfqpoint{0.663063in}{1.202893in}}{\pgfqpoint{0.670963in}{1.199621in}}{\pgfqpoint{0.679199in}{1.199621in}}%
\pgfpathclose%
\pgfusepath{stroke,fill}%
\end{pgfscope}%
\begin{pgfscope}%
\pgfpathrectangle{\pgfqpoint{0.100000in}{0.220728in}}{\pgfqpoint{3.696000in}{3.696000in}}%
\pgfusepath{clip}%
\pgfsetbuttcap%
\pgfsetroundjoin%
\definecolor{currentfill}{rgb}{0.121569,0.466667,0.705882}%
\pgfsetfillcolor{currentfill}%
\pgfsetfillopacity{0.619878}%
\pgfsetlinewidth{1.003750pt}%
\definecolor{currentstroke}{rgb}{0.121569,0.466667,0.705882}%
\pgfsetstrokecolor{currentstroke}%
\pgfsetstrokeopacity{0.619878}%
\pgfsetdash{}{0pt}%
\pgfpathmoveto{\pgfqpoint{0.679199in}{1.199621in}}%
\pgfpathcurveto{\pgfqpoint{0.687435in}{1.199621in}}{\pgfqpoint{0.695335in}{1.202893in}}{\pgfqpoint{0.701159in}{1.208717in}}%
\pgfpathcurveto{\pgfqpoint{0.706983in}{1.214541in}}{\pgfqpoint{0.710255in}{1.222441in}}{\pgfqpoint{0.710255in}{1.230677in}}%
\pgfpathcurveto{\pgfqpoint{0.710255in}{1.238914in}}{\pgfqpoint{0.706983in}{1.246814in}}{\pgfqpoint{0.701159in}{1.252638in}}%
\pgfpathcurveto{\pgfqpoint{0.695335in}{1.258462in}}{\pgfqpoint{0.687435in}{1.261734in}}{\pgfqpoint{0.679199in}{1.261734in}}%
\pgfpathcurveto{\pgfqpoint{0.670963in}{1.261734in}}{\pgfqpoint{0.663063in}{1.258462in}}{\pgfqpoint{0.657239in}{1.252638in}}%
\pgfpathcurveto{\pgfqpoint{0.651415in}{1.246814in}}{\pgfqpoint{0.648142in}{1.238914in}}{\pgfqpoint{0.648142in}{1.230677in}}%
\pgfpathcurveto{\pgfqpoint{0.648142in}{1.222441in}}{\pgfqpoint{0.651415in}{1.214541in}}{\pgfqpoint{0.657239in}{1.208717in}}%
\pgfpathcurveto{\pgfqpoint{0.663063in}{1.202893in}}{\pgfqpoint{0.670963in}{1.199621in}}{\pgfqpoint{0.679199in}{1.199621in}}%
\pgfpathclose%
\pgfusepath{stroke,fill}%
\end{pgfscope}%
\begin{pgfscope}%
\pgfpathrectangle{\pgfqpoint{0.100000in}{0.220728in}}{\pgfqpoint{3.696000in}{3.696000in}}%
\pgfusepath{clip}%
\pgfsetbuttcap%
\pgfsetroundjoin%
\definecolor{currentfill}{rgb}{0.121569,0.466667,0.705882}%
\pgfsetfillcolor{currentfill}%
\pgfsetfillopacity{0.619878}%
\pgfsetlinewidth{1.003750pt}%
\definecolor{currentstroke}{rgb}{0.121569,0.466667,0.705882}%
\pgfsetstrokecolor{currentstroke}%
\pgfsetstrokeopacity{0.619878}%
\pgfsetdash{}{0pt}%
\pgfpathmoveto{\pgfqpoint{0.679199in}{1.199621in}}%
\pgfpathcurveto{\pgfqpoint{0.687435in}{1.199621in}}{\pgfqpoint{0.695335in}{1.202893in}}{\pgfqpoint{0.701159in}{1.208717in}}%
\pgfpathcurveto{\pgfqpoint{0.706983in}{1.214541in}}{\pgfqpoint{0.710255in}{1.222441in}}{\pgfqpoint{0.710255in}{1.230677in}}%
\pgfpathcurveto{\pgfqpoint{0.710255in}{1.238914in}}{\pgfqpoint{0.706983in}{1.246814in}}{\pgfqpoint{0.701159in}{1.252638in}}%
\pgfpathcurveto{\pgfqpoint{0.695335in}{1.258462in}}{\pgfqpoint{0.687435in}{1.261734in}}{\pgfqpoint{0.679199in}{1.261734in}}%
\pgfpathcurveto{\pgfqpoint{0.670963in}{1.261734in}}{\pgfqpoint{0.663063in}{1.258462in}}{\pgfqpoint{0.657239in}{1.252638in}}%
\pgfpathcurveto{\pgfqpoint{0.651415in}{1.246814in}}{\pgfqpoint{0.648142in}{1.238914in}}{\pgfqpoint{0.648142in}{1.230677in}}%
\pgfpathcurveto{\pgfqpoint{0.648142in}{1.222441in}}{\pgfqpoint{0.651415in}{1.214541in}}{\pgfqpoint{0.657239in}{1.208717in}}%
\pgfpathcurveto{\pgfqpoint{0.663063in}{1.202893in}}{\pgfqpoint{0.670963in}{1.199621in}}{\pgfqpoint{0.679199in}{1.199621in}}%
\pgfpathclose%
\pgfusepath{stroke,fill}%
\end{pgfscope}%
\begin{pgfscope}%
\pgfpathrectangle{\pgfqpoint{0.100000in}{0.220728in}}{\pgfqpoint{3.696000in}{3.696000in}}%
\pgfusepath{clip}%
\pgfsetbuttcap%
\pgfsetroundjoin%
\definecolor{currentfill}{rgb}{0.121569,0.466667,0.705882}%
\pgfsetfillcolor{currentfill}%
\pgfsetfillopacity{0.619878}%
\pgfsetlinewidth{1.003750pt}%
\definecolor{currentstroke}{rgb}{0.121569,0.466667,0.705882}%
\pgfsetstrokecolor{currentstroke}%
\pgfsetstrokeopacity{0.619878}%
\pgfsetdash{}{0pt}%
\pgfpathmoveto{\pgfqpoint{0.679199in}{1.199621in}}%
\pgfpathcurveto{\pgfqpoint{0.687435in}{1.199621in}}{\pgfqpoint{0.695335in}{1.202893in}}{\pgfqpoint{0.701159in}{1.208717in}}%
\pgfpathcurveto{\pgfqpoint{0.706983in}{1.214541in}}{\pgfqpoint{0.710255in}{1.222441in}}{\pgfqpoint{0.710255in}{1.230677in}}%
\pgfpathcurveto{\pgfqpoint{0.710255in}{1.238914in}}{\pgfqpoint{0.706983in}{1.246814in}}{\pgfqpoint{0.701159in}{1.252638in}}%
\pgfpathcurveto{\pgfqpoint{0.695335in}{1.258462in}}{\pgfqpoint{0.687435in}{1.261734in}}{\pgfqpoint{0.679199in}{1.261734in}}%
\pgfpathcurveto{\pgfqpoint{0.670963in}{1.261734in}}{\pgfqpoint{0.663063in}{1.258462in}}{\pgfqpoint{0.657239in}{1.252638in}}%
\pgfpathcurveto{\pgfqpoint{0.651415in}{1.246814in}}{\pgfqpoint{0.648142in}{1.238914in}}{\pgfqpoint{0.648142in}{1.230677in}}%
\pgfpathcurveto{\pgfqpoint{0.648142in}{1.222441in}}{\pgfqpoint{0.651415in}{1.214541in}}{\pgfqpoint{0.657239in}{1.208717in}}%
\pgfpathcurveto{\pgfqpoint{0.663063in}{1.202893in}}{\pgfqpoint{0.670963in}{1.199621in}}{\pgfqpoint{0.679199in}{1.199621in}}%
\pgfpathclose%
\pgfusepath{stroke,fill}%
\end{pgfscope}%
\begin{pgfscope}%
\pgfpathrectangle{\pgfqpoint{0.100000in}{0.220728in}}{\pgfqpoint{3.696000in}{3.696000in}}%
\pgfusepath{clip}%
\pgfsetbuttcap%
\pgfsetroundjoin%
\definecolor{currentfill}{rgb}{0.121569,0.466667,0.705882}%
\pgfsetfillcolor{currentfill}%
\pgfsetfillopacity{0.619878}%
\pgfsetlinewidth{1.003750pt}%
\definecolor{currentstroke}{rgb}{0.121569,0.466667,0.705882}%
\pgfsetstrokecolor{currentstroke}%
\pgfsetstrokeopacity{0.619878}%
\pgfsetdash{}{0pt}%
\pgfpathmoveto{\pgfqpoint{0.679199in}{1.199621in}}%
\pgfpathcurveto{\pgfqpoint{0.687435in}{1.199621in}}{\pgfqpoint{0.695335in}{1.202893in}}{\pgfqpoint{0.701159in}{1.208717in}}%
\pgfpathcurveto{\pgfqpoint{0.706983in}{1.214541in}}{\pgfqpoint{0.710255in}{1.222441in}}{\pgfqpoint{0.710255in}{1.230677in}}%
\pgfpathcurveto{\pgfqpoint{0.710255in}{1.238914in}}{\pgfqpoint{0.706983in}{1.246814in}}{\pgfqpoint{0.701159in}{1.252638in}}%
\pgfpathcurveto{\pgfqpoint{0.695335in}{1.258462in}}{\pgfqpoint{0.687435in}{1.261734in}}{\pgfqpoint{0.679199in}{1.261734in}}%
\pgfpathcurveto{\pgfqpoint{0.670963in}{1.261734in}}{\pgfqpoint{0.663063in}{1.258462in}}{\pgfqpoint{0.657239in}{1.252638in}}%
\pgfpathcurveto{\pgfqpoint{0.651415in}{1.246814in}}{\pgfqpoint{0.648142in}{1.238914in}}{\pgfqpoint{0.648142in}{1.230677in}}%
\pgfpathcurveto{\pgfqpoint{0.648142in}{1.222441in}}{\pgfqpoint{0.651415in}{1.214541in}}{\pgfqpoint{0.657239in}{1.208717in}}%
\pgfpathcurveto{\pgfqpoint{0.663063in}{1.202893in}}{\pgfqpoint{0.670963in}{1.199621in}}{\pgfqpoint{0.679199in}{1.199621in}}%
\pgfpathclose%
\pgfusepath{stroke,fill}%
\end{pgfscope}%
\begin{pgfscope}%
\pgfpathrectangle{\pgfqpoint{0.100000in}{0.220728in}}{\pgfqpoint{3.696000in}{3.696000in}}%
\pgfusepath{clip}%
\pgfsetbuttcap%
\pgfsetroundjoin%
\definecolor{currentfill}{rgb}{0.121569,0.466667,0.705882}%
\pgfsetfillcolor{currentfill}%
\pgfsetfillopacity{0.619878}%
\pgfsetlinewidth{1.003750pt}%
\definecolor{currentstroke}{rgb}{0.121569,0.466667,0.705882}%
\pgfsetstrokecolor{currentstroke}%
\pgfsetstrokeopacity{0.619878}%
\pgfsetdash{}{0pt}%
\pgfpathmoveto{\pgfqpoint{0.679199in}{1.199621in}}%
\pgfpathcurveto{\pgfqpoint{0.687435in}{1.199621in}}{\pgfqpoint{0.695335in}{1.202893in}}{\pgfqpoint{0.701159in}{1.208717in}}%
\pgfpathcurveto{\pgfqpoint{0.706983in}{1.214541in}}{\pgfqpoint{0.710255in}{1.222441in}}{\pgfqpoint{0.710255in}{1.230677in}}%
\pgfpathcurveto{\pgfqpoint{0.710255in}{1.238914in}}{\pgfqpoint{0.706983in}{1.246814in}}{\pgfqpoint{0.701159in}{1.252638in}}%
\pgfpathcurveto{\pgfqpoint{0.695335in}{1.258462in}}{\pgfqpoint{0.687435in}{1.261734in}}{\pgfqpoint{0.679199in}{1.261734in}}%
\pgfpathcurveto{\pgfqpoint{0.670963in}{1.261734in}}{\pgfqpoint{0.663063in}{1.258462in}}{\pgfqpoint{0.657239in}{1.252638in}}%
\pgfpathcurveto{\pgfqpoint{0.651415in}{1.246814in}}{\pgfqpoint{0.648142in}{1.238914in}}{\pgfqpoint{0.648142in}{1.230677in}}%
\pgfpathcurveto{\pgfqpoint{0.648142in}{1.222441in}}{\pgfqpoint{0.651415in}{1.214541in}}{\pgfqpoint{0.657239in}{1.208717in}}%
\pgfpathcurveto{\pgfqpoint{0.663063in}{1.202893in}}{\pgfqpoint{0.670963in}{1.199621in}}{\pgfqpoint{0.679199in}{1.199621in}}%
\pgfpathclose%
\pgfusepath{stroke,fill}%
\end{pgfscope}%
\begin{pgfscope}%
\pgfpathrectangle{\pgfqpoint{0.100000in}{0.220728in}}{\pgfqpoint{3.696000in}{3.696000in}}%
\pgfusepath{clip}%
\pgfsetbuttcap%
\pgfsetroundjoin%
\definecolor{currentfill}{rgb}{0.121569,0.466667,0.705882}%
\pgfsetfillcolor{currentfill}%
\pgfsetfillopacity{0.619878}%
\pgfsetlinewidth{1.003750pt}%
\definecolor{currentstroke}{rgb}{0.121569,0.466667,0.705882}%
\pgfsetstrokecolor{currentstroke}%
\pgfsetstrokeopacity{0.619878}%
\pgfsetdash{}{0pt}%
\pgfpathmoveto{\pgfqpoint{0.679199in}{1.199621in}}%
\pgfpathcurveto{\pgfqpoint{0.687435in}{1.199621in}}{\pgfqpoint{0.695335in}{1.202893in}}{\pgfqpoint{0.701159in}{1.208717in}}%
\pgfpathcurveto{\pgfqpoint{0.706983in}{1.214541in}}{\pgfqpoint{0.710255in}{1.222441in}}{\pgfqpoint{0.710255in}{1.230677in}}%
\pgfpathcurveto{\pgfqpoint{0.710255in}{1.238914in}}{\pgfqpoint{0.706983in}{1.246814in}}{\pgfqpoint{0.701159in}{1.252638in}}%
\pgfpathcurveto{\pgfqpoint{0.695335in}{1.258462in}}{\pgfqpoint{0.687435in}{1.261734in}}{\pgfqpoint{0.679199in}{1.261734in}}%
\pgfpathcurveto{\pgfqpoint{0.670963in}{1.261734in}}{\pgfqpoint{0.663063in}{1.258462in}}{\pgfqpoint{0.657239in}{1.252638in}}%
\pgfpathcurveto{\pgfqpoint{0.651415in}{1.246814in}}{\pgfqpoint{0.648142in}{1.238914in}}{\pgfqpoint{0.648142in}{1.230677in}}%
\pgfpathcurveto{\pgfqpoint{0.648142in}{1.222441in}}{\pgfqpoint{0.651415in}{1.214541in}}{\pgfqpoint{0.657239in}{1.208717in}}%
\pgfpathcurveto{\pgfqpoint{0.663063in}{1.202893in}}{\pgfqpoint{0.670963in}{1.199621in}}{\pgfqpoint{0.679199in}{1.199621in}}%
\pgfpathclose%
\pgfusepath{stroke,fill}%
\end{pgfscope}%
\begin{pgfscope}%
\pgfpathrectangle{\pgfqpoint{0.100000in}{0.220728in}}{\pgfqpoint{3.696000in}{3.696000in}}%
\pgfusepath{clip}%
\pgfsetbuttcap%
\pgfsetroundjoin%
\definecolor{currentfill}{rgb}{0.121569,0.466667,0.705882}%
\pgfsetfillcolor{currentfill}%
\pgfsetfillopacity{0.625635}%
\pgfsetlinewidth{1.003750pt}%
\definecolor{currentstroke}{rgb}{0.121569,0.466667,0.705882}%
\pgfsetstrokecolor{currentstroke}%
\pgfsetstrokeopacity{0.625635}%
\pgfsetdash{}{0pt}%
\pgfpathmoveto{\pgfqpoint{3.035698in}{2.926457in}}%
\pgfpathcurveto{\pgfqpoint{3.043934in}{2.926457in}}{\pgfqpoint{3.051834in}{2.929730in}}{\pgfqpoint{3.057658in}{2.935554in}}%
\pgfpathcurveto{\pgfqpoint{3.063482in}{2.941378in}}{\pgfqpoint{3.066754in}{2.949278in}}{\pgfqpoint{3.066754in}{2.957514in}}%
\pgfpathcurveto{\pgfqpoint{3.066754in}{2.965750in}}{\pgfqpoint{3.063482in}{2.973650in}}{\pgfqpoint{3.057658in}{2.979474in}}%
\pgfpathcurveto{\pgfqpoint{3.051834in}{2.985298in}}{\pgfqpoint{3.043934in}{2.988570in}}{\pgfqpoint{3.035698in}{2.988570in}}%
\pgfpathcurveto{\pgfqpoint{3.027461in}{2.988570in}}{\pgfqpoint{3.019561in}{2.985298in}}{\pgfqpoint{3.013737in}{2.979474in}}%
\pgfpathcurveto{\pgfqpoint{3.007913in}{2.973650in}}{\pgfqpoint{3.004641in}{2.965750in}}{\pgfqpoint{3.004641in}{2.957514in}}%
\pgfpathcurveto{\pgfqpoint{3.004641in}{2.949278in}}{\pgfqpoint{3.007913in}{2.941378in}}{\pgfqpoint{3.013737in}{2.935554in}}%
\pgfpathcurveto{\pgfqpoint{3.019561in}{2.929730in}}{\pgfqpoint{3.027461in}{2.926457in}}{\pgfqpoint{3.035698in}{2.926457in}}%
\pgfpathclose%
\pgfusepath{stroke,fill}%
\end{pgfscope}%
\begin{pgfscope}%
\pgfpathrectangle{\pgfqpoint{0.100000in}{0.220728in}}{\pgfqpoint{3.696000in}{3.696000in}}%
\pgfusepath{clip}%
\pgfsetbuttcap%
\pgfsetroundjoin%
\definecolor{currentfill}{rgb}{0.121569,0.466667,0.705882}%
\pgfsetfillcolor{currentfill}%
\pgfsetfillopacity{0.632462}%
\pgfsetlinewidth{1.003750pt}%
\definecolor{currentstroke}{rgb}{0.121569,0.466667,0.705882}%
\pgfsetstrokecolor{currentstroke}%
\pgfsetstrokeopacity{0.632462}%
\pgfsetdash{}{0pt}%
\pgfpathmoveto{\pgfqpoint{3.062807in}{2.922132in}}%
\pgfpathcurveto{\pgfqpoint{3.071044in}{2.922132in}}{\pgfqpoint{3.078944in}{2.925404in}}{\pgfqpoint{3.084768in}{2.931228in}}%
\pgfpathcurveto{\pgfqpoint{3.090591in}{2.937052in}}{\pgfqpoint{3.093864in}{2.944952in}}{\pgfqpoint{3.093864in}{2.953188in}}%
\pgfpathcurveto{\pgfqpoint{3.093864in}{2.961425in}}{\pgfqpoint{3.090591in}{2.969325in}}{\pgfqpoint{3.084768in}{2.975148in}}%
\pgfpathcurveto{\pgfqpoint{3.078944in}{2.980972in}}{\pgfqpoint{3.071044in}{2.984245in}}{\pgfqpoint{3.062807in}{2.984245in}}%
\pgfpathcurveto{\pgfqpoint{3.054571in}{2.984245in}}{\pgfqpoint{3.046671in}{2.980972in}}{\pgfqpoint{3.040847in}{2.975148in}}%
\pgfpathcurveto{\pgfqpoint{3.035023in}{2.969325in}}{\pgfqpoint{3.031751in}{2.961425in}}{\pgfqpoint{3.031751in}{2.953188in}}%
\pgfpathcurveto{\pgfqpoint{3.031751in}{2.944952in}}{\pgfqpoint{3.035023in}{2.937052in}}{\pgfqpoint{3.040847in}{2.931228in}}%
\pgfpathcurveto{\pgfqpoint{3.046671in}{2.925404in}}{\pgfqpoint{3.054571in}{2.922132in}}{\pgfqpoint{3.062807in}{2.922132in}}%
\pgfpathclose%
\pgfusepath{stroke,fill}%
\end{pgfscope}%
\begin{pgfscope}%
\pgfpathrectangle{\pgfqpoint{0.100000in}{0.220728in}}{\pgfqpoint{3.696000in}{3.696000in}}%
\pgfusepath{clip}%
\pgfsetbuttcap%
\pgfsetroundjoin%
\definecolor{currentfill}{rgb}{0.121569,0.466667,0.705882}%
\pgfsetfillcolor{currentfill}%
\pgfsetfillopacity{0.640688}%
\pgfsetlinewidth{1.003750pt}%
\definecolor{currentstroke}{rgb}{0.121569,0.466667,0.705882}%
\pgfsetstrokecolor{currentstroke}%
\pgfsetstrokeopacity{0.640688}%
\pgfsetdash{}{0pt}%
\pgfpathmoveto{\pgfqpoint{3.093070in}{2.918389in}}%
\pgfpathcurveto{\pgfqpoint{3.101307in}{2.918389in}}{\pgfqpoint{3.109207in}{2.921662in}}{\pgfqpoint{3.115031in}{2.927486in}}%
\pgfpathcurveto{\pgfqpoint{3.120855in}{2.933310in}}{\pgfqpoint{3.124127in}{2.941210in}}{\pgfqpoint{3.124127in}{2.949446in}}%
\pgfpathcurveto{\pgfqpoint{3.124127in}{2.957682in}}{\pgfqpoint{3.120855in}{2.965582in}}{\pgfqpoint{3.115031in}{2.971406in}}%
\pgfpathcurveto{\pgfqpoint{3.109207in}{2.977230in}}{\pgfqpoint{3.101307in}{2.980502in}}{\pgfqpoint{3.093070in}{2.980502in}}%
\pgfpathcurveto{\pgfqpoint{3.084834in}{2.980502in}}{\pgfqpoint{3.076934in}{2.977230in}}{\pgfqpoint{3.071110in}{2.971406in}}%
\pgfpathcurveto{\pgfqpoint{3.065286in}{2.965582in}}{\pgfqpoint{3.062014in}{2.957682in}}{\pgfqpoint{3.062014in}{2.949446in}}%
\pgfpathcurveto{\pgfqpoint{3.062014in}{2.941210in}}{\pgfqpoint{3.065286in}{2.933310in}}{\pgfqpoint{3.071110in}{2.927486in}}%
\pgfpathcurveto{\pgfqpoint{3.076934in}{2.921662in}}{\pgfqpoint{3.084834in}{2.918389in}}{\pgfqpoint{3.093070in}{2.918389in}}%
\pgfpathclose%
\pgfusepath{stroke,fill}%
\end{pgfscope}%
\begin{pgfscope}%
\pgfpathrectangle{\pgfqpoint{0.100000in}{0.220728in}}{\pgfqpoint{3.696000in}{3.696000in}}%
\pgfusepath{clip}%
\pgfsetbuttcap%
\pgfsetroundjoin%
\definecolor{currentfill}{rgb}{0.121569,0.466667,0.705882}%
\pgfsetfillcolor{currentfill}%
\pgfsetfillopacity{0.650681}%
\pgfsetlinewidth{1.003750pt}%
\definecolor{currentstroke}{rgb}{0.121569,0.466667,0.705882}%
\pgfsetstrokecolor{currentstroke}%
\pgfsetstrokeopacity{0.650681}%
\pgfsetdash{}{0pt}%
\pgfpathmoveto{\pgfqpoint{3.130267in}{2.912318in}}%
\pgfpathcurveto{\pgfqpoint{3.138503in}{2.912318in}}{\pgfqpoint{3.146403in}{2.915590in}}{\pgfqpoint{3.152227in}{2.921414in}}%
\pgfpathcurveto{\pgfqpoint{3.158051in}{2.927238in}}{\pgfqpoint{3.161324in}{2.935138in}}{\pgfqpoint{3.161324in}{2.943374in}}%
\pgfpathcurveto{\pgfqpoint{3.161324in}{2.951611in}}{\pgfqpoint{3.158051in}{2.959511in}}{\pgfqpoint{3.152227in}{2.965335in}}%
\pgfpathcurveto{\pgfqpoint{3.146403in}{2.971159in}}{\pgfqpoint{3.138503in}{2.974431in}}{\pgfqpoint{3.130267in}{2.974431in}}%
\pgfpathcurveto{\pgfqpoint{3.122031in}{2.974431in}}{\pgfqpoint{3.114131in}{2.971159in}}{\pgfqpoint{3.108307in}{2.965335in}}%
\pgfpathcurveto{\pgfqpoint{3.102483in}{2.959511in}}{\pgfqpoint{3.099211in}{2.951611in}}{\pgfqpoint{3.099211in}{2.943374in}}%
\pgfpathcurveto{\pgfqpoint{3.099211in}{2.935138in}}{\pgfqpoint{3.102483in}{2.927238in}}{\pgfqpoint{3.108307in}{2.921414in}}%
\pgfpathcurveto{\pgfqpoint{3.114131in}{2.915590in}}{\pgfqpoint{3.122031in}{2.912318in}}{\pgfqpoint{3.130267in}{2.912318in}}%
\pgfpathclose%
\pgfusepath{stroke,fill}%
\end{pgfscope}%
\begin{pgfscope}%
\pgfpathrectangle{\pgfqpoint{0.100000in}{0.220728in}}{\pgfqpoint{3.696000in}{3.696000in}}%
\pgfusepath{clip}%
\pgfsetbuttcap%
\pgfsetroundjoin%
\definecolor{currentfill}{rgb}{0.121569,0.466667,0.705882}%
\pgfsetfillcolor{currentfill}%
\pgfsetfillopacity{0.659505}%
\pgfsetlinewidth{1.003750pt}%
\definecolor{currentstroke}{rgb}{0.121569,0.466667,0.705882}%
\pgfsetstrokecolor{currentstroke}%
\pgfsetstrokeopacity{0.659505}%
\pgfsetdash{}{0pt}%
\pgfpathmoveto{\pgfqpoint{3.173077in}{2.903326in}}%
\pgfpathcurveto{\pgfqpoint{3.181313in}{2.903326in}}{\pgfqpoint{3.189213in}{2.906598in}}{\pgfqpoint{3.195037in}{2.912422in}}%
\pgfpathcurveto{\pgfqpoint{3.200861in}{2.918246in}}{\pgfqpoint{3.204133in}{2.926146in}}{\pgfqpoint{3.204133in}{2.934382in}}%
\pgfpathcurveto{\pgfqpoint{3.204133in}{2.942618in}}{\pgfqpoint{3.200861in}{2.950518in}}{\pgfqpoint{3.195037in}{2.956342in}}%
\pgfpathcurveto{\pgfqpoint{3.189213in}{2.962166in}}{\pgfqpoint{3.181313in}{2.965439in}}{\pgfqpoint{3.173077in}{2.965439in}}%
\pgfpathcurveto{\pgfqpoint{3.164841in}{2.965439in}}{\pgfqpoint{3.156940in}{2.962166in}}{\pgfqpoint{3.151117in}{2.956342in}}%
\pgfpathcurveto{\pgfqpoint{3.145293in}{2.950518in}}{\pgfqpoint{3.142020in}{2.942618in}}{\pgfqpoint{3.142020in}{2.934382in}}%
\pgfpathcurveto{\pgfqpoint{3.142020in}{2.926146in}}{\pgfqpoint{3.145293in}{2.918246in}}{\pgfqpoint{3.151117in}{2.912422in}}%
\pgfpathcurveto{\pgfqpoint{3.156940in}{2.906598in}}{\pgfqpoint{3.164841in}{2.903326in}}{\pgfqpoint{3.173077in}{2.903326in}}%
\pgfpathclose%
\pgfusepath{stroke,fill}%
\end{pgfscope}%
\begin{pgfscope}%
\pgfpathrectangle{\pgfqpoint{0.100000in}{0.220728in}}{\pgfqpoint{3.696000in}{3.696000in}}%
\pgfusepath{clip}%
\pgfsetbuttcap%
\pgfsetroundjoin%
\definecolor{currentfill}{rgb}{0.121569,0.466667,0.705882}%
\pgfsetfillcolor{currentfill}%
\pgfsetfillopacity{0.672322}%
\pgfsetlinewidth{1.003750pt}%
\definecolor{currentstroke}{rgb}{0.121569,0.466667,0.705882}%
\pgfsetstrokecolor{currentstroke}%
\pgfsetstrokeopacity{0.672322}%
\pgfsetdash{}{0pt}%
\pgfpathmoveto{\pgfqpoint{3.212527in}{2.895769in}}%
\pgfpathcurveto{\pgfqpoint{3.220763in}{2.895769in}}{\pgfqpoint{3.228663in}{2.899042in}}{\pgfqpoint{3.234487in}{2.904866in}}%
\pgfpathcurveto{\pgfqpoint{3.240311in}{2.910690in}}{\pgfqpoint{3.243583in}{2.918590in}}{\pgfqpoint{3.243583in}{2.926826in}}%
\pgfpathcurveto{\pgfqpoint{3.243583in}{2.935062in}}{\pgfqpoint{3.240311in}{2.942962in}}{\pgfqpoint{3.234487in}{2.948786in}}%
\pgfpathcurveto{\pgfqpoint{3.228663in}{2.954610in}}{\pgfqpoint{3.220763in}{2.957882in}}{\pgfqpoint{3.212527in}{2.957882in}}%
\pgfpathcurveto{\pgfqpoint{3.204290in}{2.957882in}}{\pgfqpoint{3.196390in}{2.954610in}}{\pgfqpoint{3.190566in}{2.948786in}}%
\pgfpathcurveto{\pgfqpoint{3.184742in}{2.942962in}}{\pgfqpoint{3.181470in}{2.935062in}}{\pgfqpoint{3.181470in}{2.926826in}}%
\pgfpathcurveto{\pgfqpoint{3.181470in}{2.918590in}}{\pgfqpoint{3.184742in}{2.910690in}}{\pgfqpoint{3.190566in}{2.904866in}}%
\pgfpathcurveto{\pgfqpoint{3.196390in}{2.899042in}}{\pgfqpoint{3.204290in}{2.895769in}}{\pgfqpoint{3.212527in}{2.895769in}}%
\pgfpathclose%
\pgfusepath{stroke,fill}%
\end{pgfscope}%
\begin{pgfscope}%
\pgfpathrectangle{\pgfqpoint{0.100000in}{0.220728in}}{\pgfqpoint{3.696000in}{3.696000in}}%
\pgfusepath{clip}%
\pgfsetbuttcap%
\pgfsetroundjoin%
\definecolor{currentfill}{rgb}{0.121569,0.466667,0.705882}%
\pgfsetfillcolor{currentfill}%
\pgfsetfillopacity{0.677811}%
\pgfsetlinewidth{1.003750pt}%
\definecolor{currentstroke}{rgb}{0.121569,0.466667,0.705882}%
\pgfsetstrokecolor{currentstroke}%
\pgfsetstrokeopacity{0.677811}%
\pgfsetdash{}{0pt}%
\pgfpathmoveto{\pgfqpoint{3.237616in}{2.893315in}}%
\pgfpathcurveto{\pgfqpoint{3.245852in}{2.893315in}}{\pgfqpoint{3.253752in}{2.896587in}}{\pgfqpoint{3.259576in}{2.902411in}}%
\pgfpathcurveto{\pgfqpoint{3.265400in}{2.908235in}}{\pgfqpoint{3.268672in}{2.916135in}}{\pgfqpoint{3.268672in}{2.924371in}}%
\pgfpathcurveto{\pgfqpoint{3.268672in}{2.932607in}}{\pgfqpoint{3.265400in}{2.940507in}}{\pgfqpoint{3.259576in}{2.946331in}}%
\pgfpathcurveto{\pgfqpoint{3.253752in}{2.952155in}}{\pgfqpoint{3.245852in}{2.955428in}}{\pgfqpoint{3.237616in}{2.955428in}}%
\pgfpathcurveto{\pgfqpoint{3.229379in}{2.955428in}}{\pgfqpoint{3.221479in}{2.952155in}}{\pgfqpoint{3.215655in}{2.946331in}}%
\pgfpathcurveto{\pgfqpoint{3.209831in}{2.940507in}}{\pgfqpoint{3.206559in}{2.932607in}}{\pgfqpoint{3.206559in}{2.924371in}}%
\pgfpathcurveto{\pgfqpoint{3.206559in}{2.916135in}}{\pgfqpoint{3.209831in}{2.908235in}}{\pgfqpoint{3.215655in}{2.902411in}}%
\pgfpathcurveto{\pgfqpoint{3.221479in}{2.896587in}}{\pgfqpoint{3.229379in}{2.893315in}}{\pgfqpoint{3.237616in}{2.893315in}}%
\pgfpathclose%
\pgfusepath{stroke,fill}%
\end{pgfscope}%
\begin{pgfscope}%
\pgfpathrectangle{\pgfqpoint{0.100000in}{0.220728in}}{\pgfqpoint{3.696000in}{3.696000in}}%
\pgfusepath{clip}%
\pgfsetbuttcap%
\pgfsetroundjoin%
\definecolor{currentfill}{rgb}{0.121569,0.466667,0.705882}%
\pgfsetfillcolor{currentfill}%
\pgfsetfillopacity{0.684326}%
\pgfsetlinewidth{1.003750pt}%
\definecolor{currentstroke}{rgb}{0.121569,0.466667,0.705882}%
\pgfsetstrokecolor{currentstroke}%
\pgfsetstrokeopacity{0.684326}%
\pgfsetdash{}{0pt}%
\pgfpathmoveto{\pgfqpoint{3.263794in}{2.888407in}}%
\pgfpathcurveto{\pgfqpoint{3.272030in}{2.888407in}}{\pgfqpoint{3.279930in}{2.891679in}}{\pgfqpoint{3.285754in}{2.897503in}}%
\pgfpathcurveto{\pgfqpoint{3.291578in}{2.903327in}}{\pgfqpoint{3.294850in}{2.911227in}}{\pgfqpoint{3.294850in}{2.919463in}}%
\pgfpathcurveto{\pgfqpoint{3.294850in}{2.927700in}}{\pgfqpoint{3.291578in}{2.935600in}}{\pgfqpoint{3.285754in}{2.941424in}}%
\pgfpathcurveto{\pgfqpoint{3.279930in}{2.947248in}}{\pgfqpoint{3.272030in}{2.950520in}}{\pgfqpoint{3.263794in}{2.950520in}}%
\pgfpathcurveto{\pgfqpoint{3.255557in}{2.950520in}}{\pgfqpoint{3.247657in}{2.947248in}}{\pgfqpoint{3.241833in}{2.941424in}}%
\pgfpathcurveto{\pgfqpoint{3.236009in}{2.935600in}}{\pgfqpoint{3.232737in}{2.927700in}}{\pgfqpoint{3.232737in}{2.919463in}}%
\pgfpathcurveto{\pgfqpoint{3.232737in}{2.911227in}}{\pgfqpoint{3.236009in}{2.903327in}}{\pgfqpoint{3.241833in}{2.897503in}}%
\pgfpathcurveto{\pgfqpoint{3.247657in}{2.891679in}}{\pgfqpoint{3.255557in}{2.888407in}}{\pgfqpoint{3.263794in}{2.888407in}}%
\pgfpathclose%
\pgfusepath{stroke,fill}%
\end{pgfscope}%
\begin{pgfscope}%
\pgfpathrectangle{\pgfqpoint{0.100000in}{0.220728in}}{\pgfqpoint{3.696000in}{3.696000in}}%
\pgfusepath{clip}%
\pgfsetbuttcap%
\pgfsetroundjoin%
\definecolor{currentfill}{rgb}{0.121569,0.466667,0.705882}%
\pgfsetfillcolor{currentfill}%
\pgfsetfillopacity{0.688178}%
\pgfsetlinewidth{1.003750pt}%
\definecolor{currentstroke}{rgb}{0.121569,0.466667,0.705882}%
\pgfsetstrokecolor{currentstroke}%
\pgfsetstrokeopacity{0.688178}%
\pgfsetdash{}{0pt}%
\pgfpathmoveto{\pgfqpoint{3.277560in}{2.885208in}}%
\pgfpathcurveto{\pgfqpoint{3.285797in}{2.885208in}}{\pgfqpoint{3.293697in}{2.888480in}}{\pgfqpoint{3.299521in}{2.894304in}}%
\pgfpathcurveto{\pgfqpoint{3.305345in}{2.900128in}}{\pgfqpoint{3.308617in}{2.908028in}}{\pgfqpoint{3.308617in}{2.916264in}}%
\pgfpathcurveto{\pgfqpoint{3.308617in}{2.924501in}}{\pgfqpoint{3.305345in}{2.932401in}}{\pgfqpoint{3.299521in}{2.938225in}}%
\pgfpathcurveto{\pgfqpoint{3.293697in}{2.944049in}}{\pgfqpoint{3.285797in}{2.947321in}}{\pgfqpoint{3.277560in}{2.947321in}}%
\pgfpathcurveto{\pgfqpoint{3.269324in}{2.947321in}}{\pgfqpoint{3.261424in}{2.944049in}}{\pgfqpoint{3.255600in}{2.938225in}}%
\pgfpathcurveto{\pgfqpoint{3.249776in}{2.932401in}}{\pgfqpoint{3.246504in}{2.924501in}}{\pgfqpoint{3.246504in}{2.916264in}}%
\pgfpathcurveto{\pgfqpoint{3.246504in}{2.908028in}}{\pgfqpoint{3.249776in}{2.900128in}}{\pgfqpoint{3.255600in}{2.894304in}}%
\pgfpathcurveto{\pgfqpoint{3.261424in}{2.888480in}}{\pgfqpoint{3.269324in}{2.885208in}}{\pgfqpoint{3.277560in}{2.885208in}}%
\pgfpathclose%
\pgfusepath{stroke,fill}%
\end{pgfscope}%
\begin{pgfscope}%
\pgfpathrectangle{\pgfqpoint{0.100000in}{0.220728in}}{\pgfqpoint{3.696000in}{3.696000in}}%
\pgfusepath{clip}%
\pgfsetbuttcap%
\pgfsetroundjoin%
\definecolor{currentfill}{rgb}{0.121569,0.466667,0.705882}%
\pgfsetfillcolor{currentfill}%
\pgfsetfillopacity{0.693181}%
\pgfsetlinewidth{1.003750pt}%
\definecolor{currentstroke}{rgb}{0.121569,0.466667,0.705882}%
\pgfsetstrokecolor{currentstroke}%
\pgfsetstrokeopacity{0.693181}%
\pgfsetdash{}{0pt}%
\pgfpathmoveto{\pgfqpoint{3.296852in}{2.881813in}}%
\pgfpathcurveto{\pgfqpoint{3.305089in}{2.881813in}}{\pgfqpoint{3.312989in}{2.885085in}}{\pgfqpoint{3.318813in}{2.890909in}}%
\pgfpathcurveto{\pgfqpoint{3.324637in}{2.896733in}}{\pgfqpoint{3.327909in}{2.904633in}}{\pgfqpoint{3.327909in}{2.912869in}}%
\pgfpathcurveto{\pgfqpoint{3.327909in}{2.921106in}}{\pgfqpoint{3.324637in}{2.929006in}}{\pgfqpoint{3.318813in}{2.934830in}}%
\pgfpathcurveto{\pgfqpoint{3.312989in}{2.940653in}}{\pgfqpoint{3.305089in}{2.943926in}}{\pgfqpoint{3.296852in}{2.943926in}}%
\pgfpathcurveto{\pgfqpoint{3.288616in}{2.943926in}}{\pgfqpoint{3.280716in}{2.940653in}}{\pgfqpoint{3.274892in}{2.934830in}}%
\pgfpathcurveto{\pgfqpoint{3.269068in}{2.929006in}}{\pgfqpoint{3.265796in}{2.921106in}}{\pgfqpoint{3.265796in}{2.912869in}}%
\pgfpathcurveto{\pgfqpoint{3.265796in}{2.904633in}}{\pgfqpoint{3.269068in}{2.896733in}}{\pgfqpoint{3.274892in}{2.890909in}}%
\pgfpathcurveto{\pgfqpoint{3.280716in}{2.885085in}}{\pgfqpoint{3.288616in}{2.881813in}}{\pgfqpoint{3.296852in}{2.881813in}}%
\pgfpathclose%
\pgfusepath{stroke,fill}%
\end{pgfscope}%
\begin{pgfscope}%
\pgfpathrectangle{\pgfqpoint{0.100000in}{0.220728in}}{\pgfqpoint{3.696000in}{3.696000in}}%
\pgfusepath{clip}%
\pgfsetbuttcap%
\pgfsetroundjoin%
\definecolor{currentfill}{rgb}{0.121569,0.466667,0.705882}%
\pgfsetfillcolor{currentfill}%
\pgfsetfillopacity{0.695179}%
\pgfsetlinewidth{1.003750pt}%
\definecolor{currentstroke}{rgb}{0.121569,0.466667,0.705882}%
\pgfsetstrokecolor{currentstroke}%
\pgfsetstrokeopacity{0.695179}%
\pgfsetdash{}{0pt}%
\pgfpathmoveto{\pgfqpoint{0.754893in}{2.319267in}}%
\pgfpathcurveto{\pgfqpoint{0.763129in}{2.319267in}}{\pgfqpoint{0.771029in}{2.322540in}}{\pgfqpoint{0.776853in}{2.328364in}}%
\pgfpathcurveto{\pgfqpoint{0.782677in}{2.334188in}}{\pgfqpoint{0.785949in}{2.342088in}}{\pgfqpoint{0.785949in}{2.350324in}}%
\pgfpathcurveto{\pgfqpoint{0.785949in}{2.358560in}}{\pgfqpoint{0.782677in}{2.366460in}}{\pgfqpoint{0.776853in}{2.372284in}}%
\pgfpathcurveto{\pgfqpoint{0.771029in}{2.378108in}}{\pgfqpoint{0.763129in}{2.381380in}}{\pgfqpoint{0.754893in}{2.381380in}}%
\pgfpathcurveto{\pgfqpoint{0.746657in}{2.381380in}}{\pgfqpoint{0.738757in}{2.378108in}}{\pgfqpoint{0.732933in}{2.372284in}}%
\pgfpathcurveto{\pgfqpoint{0.727109in}{2.366460in}}{\pgfqpoint{0.723836in}{2.358560in}}{\pgfqpoint{0.723836in}{2.350324in}}%
\pgfpathcurveto{\pgfqpoint{0.723836in}{2.342088in}}{\pgfqpoint{0.727109in}{2.334188in}}{\pgfqpoint{0.732933in}{2.328364in}}%
\pgfpathcurveto{\pgfqpoint{0.738757in}{2.322540in}}{\pgfqpoint{0.746657in}{2.319267in}}{\pgfqpoint{0.754893in}{2.319267in}}%
\pgfpathclose%
\pgfusepath{stroke,fill}%
\end{pgfscope}%
\begin{pgfscope}%
\pgfpathrectangle{\pgfqpoint{0.100000in}{0.220728in}}{\pgfqpoint{3.696000in}{3.696000in}}%
\pgfusepath{clip}%
\pgfsetbuttcap%
\pgfsetroundjoin%
\definecolor{currentfill}{rgb}{0.121569,0.466667,0.705882}%
\pgfsetfillcolor{currentfill}%
\pgfsetfillopacity{0.696043}%
\pgfsetlinewidth{1.003750pt}%
\definecolor{currentstroke}{rgb}{0.121569,0.466667,0.705882}%
\pgfsetstrokecolor{currentstroke}%
\pgfsetstrokeopacity{0.696043}%
\pgfsetdash{}{0pt}%
\pgfpathmoveto{\pgfqpoint{3.306938in}{2.879078in}}%
\pgfpathcurveto{\pgfqpoint{3.315174in}{2.879078in}}{\pgfqpoint{3.323075in}{2.882350in}}{\pgfqpoint{3.328898in}{2.888174in}}%
\pgfpathcurveto{\pgfqpoint{3.334722in}{2.893998in}}{\pgfqpoint{3.337995in}{2.901898in}}{\pgfqpoint{3.337995in}{2.910134in}}%
\pgfpathcurveto{\pgfqpoint{3.337995in}{2.918371in}}{\pgfqpoint{3.334722in}{2.926271in}}{\pgfqpoint{3.328898in}{2.932095in}}%
\pgfpathcurveto{\pgfqpoint{3.323075in}{2.937918in}}{\pgfqpoint{3.315174in}{2.941191in}}{\pgfqpoint{3.306938in}{2.941191in}}%
\pgfpathcurveto{\pgfqpoint{3.298702in}{2.941191in}}{\pgfqpoint{3.290802in}{2.937918in}}{\pgfqpoint{3.284978in}{2.932095in}}%
\pgfpathcurveto{\pgfqpoint{3.279154in}{2.926271in}}{\pgfqpoint{3.275882in}{2.918371in}}{\pgfqpoint{3.275882in}{2.910134in}}%
\pgfpathcurveto{\pgfqpoint{3.275882in}{2.901898in}}{\pgfqpoint{3.279154in}{2.893998in}}{\pgfqpoint{3.284978in}{2.888174in}}%
\pgfpathcurveto{\pgfqpoint{3.290802in}{2.882350in}}{\pgfqpoint{3.298702in}{2.879078in}}{\pgfqpoint{3.306938in}{2.879078in}}%
\pgfpathclose%
\pgfusepath{stroke,fill}%
\end{pgfscope}%
\begin{pgfscope}%
\pgfpathrectangle{\pgfqpoint{0.100000in}{0.220728in}}{\pgfqpoint{3.696000in}{3.696000in}}%
\pgfusepath{clip}%
\pgfsetbuttcap%
\pgfsetroundjoin%
\definecolor{currentfill}{rgb}{0.121569,0.466667,0.705882}%
\pgfsetfillcolor{currentfill}%
\pgfsetfillopacity{0.697714}%
\pgfsetlinewidth{1.003750pt}%
\definecolor{currentstroke}{rgb}{0.121569,0.466667,0.705882}%
\pgfsetstrokecolor{currentstroke}%
\pgfsetstrokeopacity{0.697714}%
\pgfsetdash{}{0pt}%
\pgfpathmoveto{\pgfqpoint{3.311522in}{2.876114in}}%
\pgfpathcurveto{\pgfqpoint{3.319758in}{2.876114in}}{\pgfqpoint{3.327658in}{2.879386in}}{\pgfqpoint{3.333482in}{2.885210in}}%
\pgfpathcurveto{\pgfqpoint{3.339306in}{2.891034in}}{\pgfqpoint{3.342578in}{2.898934in}}{\pgfqpoint{3.342578in}{2.907170in}}%
\pgfpathcurveto{\pgfqpoint{3.342578in}{2.915406in}}{\pgfqpoint{3.339306in}{2.923307in}}{\pgfqpoint{3.333482in}{2.929130in}}%
\pgfpathcurveto{\pgfqpoint{3.327658in}{2.934954in}}{\pgfqpoint{3.319758in}{2.938227in}}{\pgfqpoint{3.311522in}{2.938227in}}%
\pgfpathcurveto{\pgfqpoint{3.303285in}{2.938227in}}{\pgfqpoint{3.295385in}{2.934954in}}{\pgfqpoint{3.289561in}{2.929130in}}%
\pgfpathcurveto{\pgfqpoint{3.283738in}{2.923307in}}{\pgfqpoint{3.280465in}{2.915406in}}{\pgfqpoint{3.280465in}{2.907170in}}%
\pgfpathcurveto{\pgfqpoint{3.280465in}{2.898934in}}{\pgfqpoint{3.283738in}{2.891034in}}{\pgfqpoint{3.289561in}{2.885210in}}%
\pgfpathcurveto{\pgfqpoint{3.295385in}{2.879386in}}{\pgfqpoint{3.303285in}{2.876114in}}{\pgfqpoint{3.311522in}{2.876114in}}%
\pgfpathclose%
\pgfusepath{stroke,fill}%
\end{pgfscope}%
\begin{pgfscope}%
\pgfpathrectangle{\pgfqpoint{0.100000in}{0.220728in}}{\pgfqpoint{3.696000in}{3.696000in}}%
\pgfusepath{clip}%
\pgfsetbuttcap%
\pgfsetroundjoin%
\definecolor{currentfill}{rgb}{0.121569,0.466667,0.705882}%
\pgfsetfillcolor{currentfill}%
\pgfsetfillopacity{0.698231}%
\pgfsetlinewidth{1.003750pt}%
\definecolor{currentstroke}{rgb}{0.121569,0.466667,0.705882}%
\pgfsetstrokecolor{currentstroke}%
\pgfsetstrokeopacity{0.698231}%
\pgfsetdash{}{0pt}%
\pgfpathmoveto{\pgfqpoint{0.775879in}{2.309403in}}%
\pgfpathcurveto{\pgfqpoint{0.784115in}{2.309403in}}{\pgfqpoint{0.792015in}{2.312675in}}{\pgfqpoint{0.797839in}{2.318499in}}%
\pgfpathcurveto{\pgfqpoint{0.803663in}{2.324323in}}{\pgfqpoint{0.806935in}{2.332223in}}{\pgfqpoint{0.806935in}{2.340460in}}%
\pgfpathcurveto{\pgfqpoint{0.806935in}{2.348696in}}{\pgfqpoint{0.803663in}{2.356596in}}{\pgfqpoint{0.797839in}{2.362420in}}%
\pgfpathcurveto{\pgfqpoint{0.792015in}{2.368244in}}{\pgfqpoint{0.784115in}{2.371516in}}{\pgfqpoint{0.775879in}{2.371516in}}%
\pgfpathcurveto{\pgfqpoint{0.767643in}{2.371516in}}{\pgfqpoint{0.759743in}{2.368244in}}{\pgfqpoint{0.753919in}{2.362420in}}%
\pgfpathcurveto{\pgfqpoint{0.748095in}{2.356596in}}{\pgfqpoint{0.744822in}{2.348696in}}{\pgfqpoint{0.744822in}{2.340460in}}%
\pgfpathcurveto{\pgfqpoint{0.744822in}{2.332223in}}{\pgfqpoint{0.748095in}{2.324323in}}{\pgfqpoint{0.753919in}{2.318499in}}%
\pgfpathcurveto{\pgfqpoint{0.759743in}{2.312675in}}{\pgfqpoint{0.767643in}{2.309403in}}{\pgfqpoint{0.775879in}{2.309403in}}%
\pgfpathclose%
\pgfusepath{stroke,fill}%
\end{pgfscope}%
\begin{pgfscope}%
\pgfpathrectangle{\pgfqpoint{0.100000in}{0.220728in}}{\pgfqpoint{3.696000in}{3.696000in}}%
\pgfusepath{clip}%
\pgfsetbuttcap%
\pgfsetroundjoin%
\definecolor{currentfill}{rgb}{0.121569,0.466667,0.705882}%
\pgfsetfillcolor{currentfill}%
\pgfsetfillopacity{0.699878}%
\pgfsetlinewidth{1.003750pt}%
\definecolor{currentstroke}{rgb}{0.121569,0.466667,0.705882}%
\pgfsetstrokecolor{currentstroke}%
\pgfsetstrokeopacity{0.699878}%
\pgfsetdash{}{0pt}%
\pgfpathmoveto{\pgfqpoint{3.314132in}{2.867047in}}%
\pgfpathcurveto{\pgfqpoint{3.322369in}{2.867047in}}{\pgfqpoint{3.330269in}{2.870319in}}{\pgfqpoint{3.336093in}{2.876143in}}%
\pgfpathcurveto{\pgfqpoint{3.341917in}{2.881967in}}{\pgfqpoint{3.345189in}{2.889867in}}{\pgfqpoint{3.345189in}{2.898104in}}%
\pgfpathcurveto{\pgfqpoint{3.345189in}{2.906340in}}{\pgfqpoint{3.341917in}{2.914240in}}{\pgfqpoint{3.336093in}{2.920064in}}%
\pgfpathcurveto{\pgfqpoint{3.330269in}{2.925888in}}{\pgfqpoint{3.322369in}{2.929160in}}{\pgfqpoint{3.314132in}{2.929160in}}%
\pgfpathcurveto{\pgfqpoint{3.305896in}{2.929160in}}{\pgfqpoint{3.297996in}{2.925888in}}{\pgfqpoint{3.292172in}{2.920064in}}%
\pgfpathcurveto{\pgfqpoint{3.286348in}{2.914240in}}{\pgfqpoint{3.283076in}{2.906340in}}{\pgfqpoint{3.283076in}{2.898104in}}%
\pgfpathcurveto{\pgfqpoint{3.283076in}{2.889867in}}{\pgfqpoint{3.286348in}{2.881967in}}{\pgfqpoint{3.292172in}{2.876143in}}%
\pgfpathcurveto{\pgfqpoint{3.297996in}{2.870319in}}{\pgfqpoint{3.305896in}{2.867047in}}{\pgfqpoint{3.314132in}{2.867047in}}%
\pgfpathclose%
\pgfusepath{stroke,fill}%
\end{pgfscope}%
\begin{pgfscope}%
\pgfpathrectangle{\pgfqpoint{0.100000in}{0.220728in}}{\pgfqpoint{3.696000in}{3.696000in}}%
\pgfusepath{clip}%
\pgfsetbuttcap%
\pgfsetroundjoin%
\definecolor{currentfill}{rgb}{0.121569,0.466667,0.705882}%
\pgfsetfillcolor{currentfill}%
\pgfsetfillopacity{0.701050}%
\pgfsetlinewidth{1.003750pt}%
\definecolor{currentstroke}{rgb}{0.121569,0.466667,0.705882}%
\pgfsetstrokecolor{currentstroke}%
\pgfsetstrokeopacity{0.701050}%
\pgfsetdash{}{0pt}%
\pgfpathmoveto{\pgfqpoint{0.786823in}{2.288000in}}%
\pgfpathcurveto{\pgfqpoint{0.795059in}{2.288000in}}{\pgfqpoint{0.802960in}{2.291272in}}{\pgfqpoint{0.808783in}{2.297096in}}%
\pgfpathcurveto{\pgfqpoint{0.814607in}{2.302920in}}{\pgfqpoint{0.817880in}{2.310820in}}{\pgfqpoint{0.817880in}{2.319057in}}%
\pgfpathcurveto{\pgfqpoint{0.817880in}{2.327293in}}{\pgfqpoint{0.814607in}{2.335193in}}{\pgfqpoint{0.808783in}{2.341017in}}%
\pgfpathcurveto{\pgfqpoint{0.802960in}{2.346841in}}{\pgfqpoint{0.795059in}{2.350113in}}{\pgfqpoint{0.786823in}{2.350113in}}%
\pgfpathcurveto{\pgfqpoint{0.778587in}{2.350113in}}{\pgfqpoint{0.770687in}{2.346841in}}{\pgfqpoint{0.764863in}{2.341017in}}%
\pgfpathcurveto{\pgfqpoint{0.759039in}{2.335193in}}{\pgfqpoint{0.755767in}{2.327293in}}{\pgfqpoint{0.755767in}{2.319057in}}%
\pgfpathcurveto{\pgfqpoint{0.755767in}{2.310820in}}{\pgfqpoint{0.759039in}{2.302920in}}{\pgfqpoint{0.764863in}{2.297096in}}%
\pgfpathcurveto{\pgfqpoint{0.770687in}{2.291272in}}{\pgfqpoint{0.778587in}{2.288000in}}{\pgfqpoint{0.786823in}{2.288000in}}%
\pgfpathclose%
\pgfusepath{stroke,fill}%
\end{pgfscope}%
\begin{pgfscope}%
\pgfpathrectangle{\pgfqpoint{0.100000in}{0.220728in}}{\pgfqpoint{3.696000in}{3.696000in}}%
\pgfusepath{clip}%
\pgfsetbuttcap%
\pgfsetroundjoin%
\definecolor{currentfill}{rgb}{0.121569,0.466667,0.705882}%
\pgfsetfillcolor{currentfill}%
\pgfsetfillopacity{0.702979}%
\pgfsetlinewidth{1.003750pt}%
\definecolor{currentstroke}{rgb}{0.121569,0.466667,0.705882}%
\pgfsetstrokecolor{currentstroke}%
\pgfsetstrokeopacity{0.702979}%
\pgfsetdash{}{0pt}%
\pgfpathmoveto{\pgfqpoint{0.803873in}{2.287549in}}%
\pgfpathcurveto{\pgfqpoint{0.812109in}{2.287549in}}{\pgfqpoint{0.820009in}{2.290821in}}{\pgfqpoint{0.825833in}{2.296645in}}%
\pgfpathcurveto{\pgfqpoint{0.831657in}{2.302469in}}{\pgfqpoint{0.834929in}{2.310369in}}{\pgfqpoint{0.834929in}{2.318606in}}%
\pgfpathcurveto{\pgfqpoint{0.834929in}{2.326842in}}{\pgfqpoint{0.831657in}{2.334742in}}{\pgfqpoint{0.825833in}{2.340566in}}%
\pgfpathcurveto{\pgfqpoint{0.820009in}{2.346390in}}{\pgfqpoint{0.812109in}{2.349662in}}{\pgfqpoint{0.803873in}{2.349662in}}%
\pgfpathcurveto{\pgfqpoint{0.795636in}{2.349662in}}{\pgfqpoint{0.787736in}{2.346390in}}{\pgfqpoint{0.781912in}{2.340566in}}%
\pgfpathcurveto{\pgfqpoint{0.776088in}{2.334742in}}{\pgfqpoint{0.772816in}{2.326842in}}{\pgfqpoint{0.772816in}{2.318606in}}%
\pgfpathcurveto{\pgfqpoint{0.772816in}{2.310369in}}{\pgfqpoint{0.776088in}{2.302469in}}{\pgfqpoint{0.781912in}{2.296645in}}%
\pgfpathcurveto{\pgfqpoint{0.787736in}{2.290821in}}{\pgfqpoint{0.795636in}{2.287549in}}{\pgfqpoint{0.803873in}{2.287549in}}%
\pgfpathclose%
\pgfusepath{stroke,fill}%
\end{pgfscope}%
\begin{pgfscope}%
\pgfpathrectangle{\pgfqpoint{0.100000in}{0.220728in}}{\pgfqpoint{3.696000in}{3.696000in}}%
\pgfusepath{clip}%
\pgfsetbuttcap%
\pgfsetroundjoin%
\definecolor{currentfill}{rgb}{0.121569,0.466667,0.705882}%
\pgfsetfillcolor{currentfill}%
\pgfsetfillopacity{0.703039}%
\pgfsetlinewidth{1.003750pt}%
\definecolor{currentstroke}{rgb}{0.121569,0.466667,0.705882}%
\pgfsetstrokecolor{currentstroke}%
\pgfsetstrokeopacity{0.703039}%
\pgfsetdash{}{0pt}%
\pgfpathmoveto{\pgfqpoint{3.315105in}{2.856479in}}%
\pgfpathcurveto{\pgfqpoint{3.323341in}{2.856479in}}{\pgfqpoint{3.331241in}{2.859752in}}{\pgfqpoint{3.337065in}{2.865576in}}%
\pgfpathcurveto{\pgfqpoint{3.342889in}{2.871400in}}{\pgfqpoint{3.346161in}{2.879300in}}{\pgfqpoint{3.346161in}{2.887536in}}%
\pgfpathcurveto{\pgfqpoint{3.346161in}{2.895772in}}{\pgfqpoint{3.342889in}{2.903672in}}{\pgfqpoint{3.337065in}{2.909496in}}%
\pgfpathcurveto{\pgfqpoint{3.331241in}{2.915320in}}{\pgfqpoint{3.323341in}{2.918592in}}{\pgfqpoint{3.315105in}{2.918592in}}%
\pgfpathcurveto{\pgfqpoint{3.306868in}{2.918592in}}{\pgfqpoint{3.298968in}{2.915320in}}{\pgfqpoint{3.293144in}{2.909496in}}%
\pgfpathcurveto{\pgfqpoint{3.287321in}{2.903672in}}{\pgfqpoint{3.284048in}{2.895772in}}{\pgfqpoint{3.284048in}{2.887536in}}%
\pgfpathcurveto{\pgfqpoint{3.284048in}{2.879300in}}{\pgfqpoint{3.287321in}{2.871400in}}{\pgfqpoint{3.293144in}{2.865576in}}%
\pgfpathcurveto{\pgfqpoint{3.298968in}{2.859752in}}{\pgfqpoint{3.306868in}{2.856479in}}{\pgfqpoint{3.315105in}{2.856479in}}%
\pgfpathclose%
\pgfusepath{stroke,fill}%
\end{pgfscope}%
\begin{pgfscope}%
\pgfpathrectangle{\pgfqpoint{0.100000in}{0.220728in}}{\pgfqpoint{3.696000in}{3.696000in}}%
\pgfusepath{clip}%
\pgfsetbuttcap%
\pgfsetroundjoin%
\definecolor{currentfill}{rgb}{0.121569,0.466667,0.705882}%
\pgfsetfillcolor{currentfill}%
\pgfsetfillopacity{0.704660}%
\pgfsetlinewidth{1.003750pt}%
\definecolor{currentstroke}{rgb}{0.121569,0.466667,0.705882}%
\pgfsetstrokecolor{currentstroke}%
\pgfsetstrokeopacity{0.704660}%
\pgfsetdash{}{0pt}%
\pgfpathmoveto{\pgfqpoint{3.314643in}{2.850054in}}%
\pgfpathcurveto{\pgfqpoint{3.322880in}{2.850054in}}{\pgfqpoint{3.330780in}{2.853326in}}{\pgfqpoint{3.336604in}{2.859150in}}%
\pgfpathcurveto{\pgfqpoint{3.342428in}{2.864974in}}{\pgfqpoint{3.345700in}{2.872874in}}{\pgfqpoint{3.345700in}{2.881111in}}%
\pgfpathcurveto{\pgfqpoint{3.345700in}{2.889347in}}{\pgfqpoint{3.342428in}{2.897247in}}{\pgfqpoint{3.336604in}{2.903071in}}%
\pgfpathcurveto{\pgfqpoint{3.330780in}{2.908895in}}{\pgfqpoint{3.322880in}{2.912167in}}{\pgfqpoint{3.314643in}{2.912167in}}%
\pgfpathcurveto{\pgfqpoint{3.306407in}{2.912167in}}{\pgfqpoint{3.298507in}{2.908895in}}{\pgfqpoint{3.292683in}{2.903071in}}%
\pgfpathcurveto{\pgfqpoint{3.286859in}{2.897247in}}{\pgfqpoint{3.283587in}{2.889347in}}{\pgfqpoint{3.283587in}{2.881111in}}%
\pgfpathcurveto{\pgfqpoint{3.283587in}{2.872874in}}{\pgfqpoint{3.286859in}{2.864974in}}{\pgfqpoint{3.292683in}{2.859150in}}%
\pgfpathcurveto{\pgfqpoint{3.298507in}{2.853326in}}{\pgfqpoint{3.306407in}{2.850054in}}{\pgfqpoint{3.314643in}{2.850054in}}%
\pgfpathclose%
\pgfusepath{stroke,fill}%
\end{pgfscope}%
\begin{pgfscope}%
\pgfpathrectangle{\pgfqpoint{0.100000in}{0.220728in}}{\pgfqpoint{3.696000in}{3.696000in}}%
\pgfusepath{clip}%
\pgfsetbuttcap%
\pgfsetroundjoin%
\definecolor{currentfill}{rgb}{0.121569,0.466667,0.705882}%
\pgfsetfillcolor{currentfill}%
\pgfsetfillopacity{0.705293}%
\pgfsetlinewidth{1.003750pt}%
\definecolor{currentstroke}{rgb}{0.121569,0.466667,0.705882}%
\pgfsetstrokecolor{currentstroke}%
\pgfsetstrokeopacity{0.705293}%
\pgfsetdash{}{0pt}%
\pgfpathmoveto{\pgfqpoint{0.811111in}{2.270819in}}%
\pgfpathcurveto{\pgfqpoint{0.819348in}{2.270819in}}{\pgfqpoint{0.827248in}{2.274092in}}{\pgfqpoint{0.833072in}{2.279916in}}%
\pgfpathcurveto{\pgfqpoint{0.838896in}{2.285740in}}{\pgfqpoint{0.842168in}{2.293640in}}{\pgfqpoint{0.842168in}{2.301876in}}%
\pgfpathcurveto{\pgfqpoint{0.842168in}{2.310112in}}{\pgfqpoint{0.838896in}{2.318012in}}{\pgfqpoint{0.833072in}{2.323836in}}%
\pgfpathcurveto{\pgfqpoint{0.827248in}{2.329660in}}{\pgfqpoint{0.819348in}{2.332932in}}{\pgfqpoint{0.811111in}{2.332932in}}%
\pgfpathcurveto{\pgfqpoint{0.802875in}{2.332932in}}{\pgfqpoint{0.794975in}{2.329660in}}{\pgfqpoint{0.789151in}{2.323836in}}%
\pgfpathcurveto{\pgfqpoint{0.783327in}{2.318012in}}{\pgfqpoint{0.780055in}{2.310112in}}{\pgfqpoint{0.780055in}{2.301876in}}%
\pgfpathcurveto{\pgfqpoint{0.780055in}{2.293640in}}{\pgfqpoint{0.783327in}{2.285740in}}{\pgfqpoint{0.789151in}{2.279916in}}%
\pgfpathcurveto{\pgfqpoint{0.794975in}{2.274092in}}{\pgfqpoint{0.802875in}{2.270819in}}{\pgfqpoint{0.811111in}{2.270819in}}%
\pgfpathclose%
\pgfusepath{stroke,fill}%
\end{pgfscope}%
\begin{pgfscope}%
\pgfpathrectangle{\pgfqpoint{0.100000in}{0.220728in}}{\pgfqpoint{3.696000in}{3.696000in}}%
\pgfusepath{clip}%
\pgfsetbuttcap%
\pgfsetroundjoin%
\definecolor{currentfill}{rgb}{0.121569,0.466667,0.705882}%
\pgfsetfillcolor{currentfill}%
\pgfsetfillopacity{0.706004}%
\pgfsetlinewidth{1.003750pt}%
\definecolor{currentstroke}{rgb}{0.121569,0.466667,0.705882}%
\pgfsetstrokecolor{currentstroke}%
\pgfsetstrokeopacity{0.706004}%
\pgfsetdash{}{0pt}%
\pgfpathmoveto{\pgfqpoint{0.824289in}{2.281766in}}%
\pgfpathcurveto{\pgfqpoint{0.832525in}{2.281766in}}{\pgfqpoint{0.840425in}{2.285039in}}{\pgfqpoint{0.846249in}{2.290863in}}%
\pgfpathcurveto{\pgfqpoint{0.852073in}{2.296687in}}{\pgfqpoint{0.855346in}{2.304587in}}{\pgfqpoint{0.855346in}{2.312823in}}%
\pgfpathcurveto{\pgfqpoint{0.855346in}{2.321059in}}{\pgfqpoint{0.852073in}{2.328959in}}{\pgfqpoint{0.846249in}{2.334783in}}%
\pgfpathcurveto{\pgfqpoint{0.840425in}{2.340607in}}{\pgfqpoint{0.832525in}{2.343879in}}{\pgfqpoint{0.824289in}{2.343879in}}%
\pgfpathcurveto{\pgfqpoint{0.816053in}{2.343879in}}{\pgfqpoint{0.808153in}{2.340607in}}{\pgfqpoint{0.802329in}{2.334783in}}%
\pgfpathcurveto{\pgfqpoint{0.796505in}{2.328959in}}{\pgfqpoint{0.793233in}{2.321059in}}{\pgfqpoint{0.793233in}{2.312823in}}%
\pgfpathcurveto{\pgfqpoint{0.793233in}{2.304587in}}{\pgfqpoint{0.796505in}{2.296687in}}{\pgfqpoint{0.802329in}{2.290863in}}%
\pgfpathcurveto{\pgfqpoint{0.808153in}{2.285039in}}{\pgfqpoint{0.816053in}{2.281766in}}{\pgfqpoint{0.824289in}{2.281766in}}%
\pgfpathclose%
\pgfusepath{stroke,fill}%
\end{pgfscope}%
\begin{pgfscope}%
\pgfpathrectangle{\pgfqpoint{0.100000in}{0.220728in}}{\pgfqpoint{3.696000in}{3.696000in}}%
\pgfusepath{clip}%
\pgfsetbuttcap%
\pgfsetroundjoin%
\definecolor{currentfill}{rgb}{0.121569,0.466667,0.705882}%
\pgfsetfillcolor{currentfill}%
\pgfsetfillopacity{0.706865}%
\pgfsetlinewidth{1.003750pt}%
\definecolor{currentstroke}{rgb}{0.121569,0.466667,0.705882}%
\pgfsetstrokecolor{currentstroke}%
\pgfsetstrokeopacity{0.706865}%
\pgfsetdash{}{0pt}%
\pgfpathmoveto{\pgfqpoint{3.311924in}{2.842080in}}%
\pgfpathcurveto{\pgfqpoint{3.320160in}{2.842080in}}{\pgfqpoint{3.328060in}{2.845353in}}{\pgfqpoint{3.333884in}{2.851177in}}%
\pgfpathcurveto{\pgfqpoint{3.339708in}{2.857001in}}{\pgfqpoint{3.342980in}{2.864901in}}{\pgfqpoint{3.342980in}{2.873137in}}%
\pgfpathcurveto{\pgfqpoint{3.342980in}{2.881373in}}{\pgfqpoint{3.339708in}{2.889273in}}{\pgfqpoint{3.333884in}{2.895097in}}%
\pgfpathcurveto{\pgfqpoint{3.328060in}{2.900921in}}{\pgfqpoint{3.320160in}{2.904193in}}{\pgfqpoint{3.311924in}{2.904193in}}%
\pgfpathcurveto{\pgfqpoint{3.303687in}{2.904193in}}{\pgfqpoint{3.295787in}{2.900921in}}{\pgfqpoint{3.289963in}{2.895097in}}%
\pgfpathcurveto{\pgfqpoint{3.284140in}{2.889273in}}{\pgfqpoint{3.280867in}{2.881373in}}{\pgfqpoint{3.280867in}{2.873137in}}%
\pgfpathcurveto{\pgfqpoint{3.280867in}{2.864901in}}{\pgfqpoint{3.284140in}{2.857001in}}{\pgfqpoint{3.289963in}{2.851177in}}%
\pgfpathcurveto{\pgfqpoint{3.295787in}{2.845353in}}{\pgfqpoint{3.303687in}{2.842080in}}{\pgfqpoint{3.311924in}{2.842080in}}%
\pgfpathclose%
\pgfusepath{stroke,fill}%
\end{pgfscope}%
\begin{pgfscope}%
\pgfpathrectangle{\pgfqpoint{0.100000in}{0.220728in}}{\pgfqpoint{3.696000in}{3.696000in}}%
\pgfusepath{clip}%
\pgfsetbuttcap%
\pgfsetroundjoin%
\definecolor{currentfill}{rgb}{0.121569,0.466667,0.705882}%
\pgfsetfillcolor{currentfill}%
\pgfsetfillopacity{0.707351}%
\pgfsetlinewidth{1.003750pt}%
\definecolor{currentstroke}{rgb}{0.121569,0.466667,0.705882}%
\pgfsetstrokecolor{currentstroke}%
\pgfsetstrokeopacity{0.707351}%
\pgfsetdash{}{0pt}%
\pgfpathmoveto{\pgfqpoint{0.832055in}{2.275121in}}%
\pgfpathcurveto{\pgfqpoint{0.840291in}{2.275121in}}{\pgfqpoint{0.848191in}{2.278393in}}{\pgfqpoint{0.854015in}{2.284217in}}%
\pgfpathcurveto{\pgfqpoint{0.859839in}{2.290041in}}{\pgfqpoint{0.863112in}{2.297941in}}{\pgfqpoint{0.863112in}{2.306178in}}%
\pgfpathcurveto{\pgfqpoint{0.863112in}{2.314414in}}{\pgfqpoint{0.859839in}{2.322314in}}{\pgfqpoint{0.854015in}{2.328138in}}%
\pgfpathcurveto{\pgfqpoint{0.848191in}{2.333962in}}{\pgfqpoint{0.840291in}{2.337234in}}{\pgfqpoint{0.832055in}{2.337234in}}%
\pgfpathcurveto{\pgfqpoint{0.823819in}{2.337234in}}{\pgfqpoint{0.815919in}{2.333962in}}{\pgfqpoint{0.810095in}{2.328138in}}%
\pgfpathcurveto{\pgfqpoint{0.804271in}{2.322314in}}{\pgfqpoint{0.800999in}{2.314414in}}{\pgfqpoint{0.800999in}{2.306178in}}%
\pgfpathcurveto{\pgfqpoint{0.800999in}{2.297941in}}{\pgfqpoint{0.804271in}{2.290041in}}{\pgfqpoint{0.810095in}{2.284217in}}%
\pgfpathcurveto{\pgfqpoint{0.815919in}{2.278393in}}{\pgfqpoint{0.823819in}{2.275121in}}{\pgfqpoint{0.832055in}{2.275121in}}%
\pgfpathclose%
\pgfusepath{stroke,fill}%
\end{pgfscope}%
\begin{pgfscope}%
\pgfpathrectangle{\pgfqpoint{0.100000in}{0.220728in}}{\pgfqpoint{3.696000in}{3.696000in}}%
\pgfusepath{clip}%
\pgfsetbuttcap%
\pgfsetroundjoin%
\definecolor{currentfill}{rgb}{0.121569,0.466667,0.705882}%
\pgfsetfillcolor{currentfill}%
\pgfsetfillopacity{0.709212}%
\pgfsetlinewidth{1.003750pt}%
\definecolor{currentstroke}{rgb}{0.121569,0.466667,0.705882}%
\pgfsetstrokecolor{currentstroke}%
\pgfsetstrokeopacity{0.709212}%
\pgfsetdash{}{0pt}%
\pgfpathmoveto{\pgfqpoint{3.305062in}{2.831243in}}%
\pgfpathcurveto{\pgfqpoint{3.313299in}{2.831243in}}{\pgfqpoint{3.321199in}{2.834515in}}{\pgfqpoint{3.327023in}{2.840339in}}%
\pgfpathcurveto{\pgfqpoint{3.332847in}{2.846163in}}{\pgfqpoint{3.336119in}{2.854063in}}{\pgfqpoint{3.336119in}{2.862299in}}%
\pgfpathcurveto{\pgfqpoint{3.336119in}{2.870536in}}{\pgfqpoint{3.332847in}{2.878436in}}{\pgfqpoint{3.327023in}{2.884260in}}%
\pgfpathcurveto{\pgfqpoint{3.321199in}{2.890084in}}{\pgfqpoint{3.313299in}{2.893356in}}{\pgfqpoint{3.305062in}{2.893356in}}%
\pgfpathcurveto{\pgfqpoint{3.296826in}{2.893356in}}{\pgfqpoint{3.288926in}{2.890084in}}{\pgfqpoint{3.283102in}{2.884260in}}%
\pgfpathcurveto{\pgfqpoint{3.277278in}{2.878436in}}{\pgfqpoint{3.274006in}{2.870536in}}{\pgfqpoint{3.274006in}{2.862299in}}%
\pgfpathcurveto{\pgfqpoint{3.274006in}{2.854063in}}{\pgfqpoint{3.277278in}{2.846163in}}{\pgfqpoint{3.283102in}{2.840339in}}%
\pgfpathcurveto{\pgfqpoint{3.288926in}{2.834515in}}{\pgfqpoint{3.296826in}{2.831243in}}{\pgfqpoint{3.305062in}{2.831243in}}%
\pgfpathclose%
\pgfusepath{stroke,fill}%
\end{pgfscope}%
\begin{pgfscope}%
\pgfpathrectangle{\pgfqpoint{0.100000in}{0.220728in}}{\pgfqpoint{3.696000in}{3.696000in}}%
\pgfusepath{clip}%
\pgfsetbuttcap%
\pgfsetroundjoin%
\definecolor{currentfill}{rgb}{0.121569,0.466667,0.705882}%
\pgfsetfillcolor{currentfill}%
\pgfsetfillopacity{0.709981}%
\pgfsetlinewidth{1.003750pt}%
\definecolor{currentstroke}{rgb}{0.121569,0.466667,0.705882}%
\pgfsetstrokecolor{currentstroke}%
\pgfsetstrokeopacity{0.709981}%
\pgfsetdash{}{0pt}%
\pgfpathmoveto{\pgfqpoint{0.845581in}{2.262526in}}%
\pgfpathcurveto{\pgfqpoint{0.853817in}{2.262526in}}{\pgfqpoint{0.861717in}{2.265798in}}{\pgfqpoint{0.867541in}{2.271622in}}%
\pgfpathcurveto{\pgfqpoint{0.873365in}{2.277446in}}{\pgfqpoint{0.876637in}{2.285346in}}{\pgfqpoint{0.876637in}{2.293582in}}%
\pgfpathcurveto{\pgfqpoint{0.876637in}{2.301819in}}{\pgfqpoint{0.873365in}{2.309719in}}{\pgfqpoint{0.867541in}{2.315543in}}%
\pgfpathcurveto{\pgfqpoint{0.861717in}{2.321366in}}{\pgfqpoint{0.853817in}{2.324639in}}{\pgfqpoint{0.845581in}{2.324639in}}%
\pgfpathcurveto{\pgfqpoint{0.837344in}{2.324639in}}{\pgfqpoint{0.829444in}{2.321366in}}{\pgfqpoint{0.823620in}{2.315543in}}%
\pgfpathcurveto{\pgfqpoint{0.817796in}{2.309719in}}{\pgfqpoint{0.814524in}{2.301819in}}{\pgfqpoint{0.814524in}{2.293582in}}%
\pgfpathcurveto{\pgfqpoint{0.814524in}{2.285346in}}{\pgfqpoint{0.817796in}{2.277446in}}{\pgfqpoint{0.823620in}{2.271622in}}%
\pgfpathcurveto{\pgfqpoint{0.829444in}{2.265798in}}{\pgfqpoint{0.837344in}{2.262526in}}{\pgfqpoint{0.845581in}{2.262526in}}%
\pgfpathclose%
\pgfusepath{stroke,fill}%
\end{pgfscope}%
\begin{pgfscope}%
\pgfpathrectangle{\pgfqpoint{0.100000in}{0.220728in}}{\pgfqpoint{3.696000in}{3.696000in}}%
\pgfusepath{clip}%
\pgfsetbuttcap%
\pgfsetroundjoin%
\definecolor{currentfill}{rgb}{0.121569,0.466667,0.705882}%
\pgfsetfillcolor{currentfill}%
\pgfsetfillopacity{0.711292}%
\pgfsetlinewidth{1.003750pt}%
\definecolor{currentstroke}{rgb}{0.121569,0.466667,0.705882}%
\pgfsetstrokecolor{currentstroke}%
\pgfsetstrokeopacity{0.711292}%
\pgfsetdash{}{0pt}%
\pgfpathmoveto{\pgfqpoint{0.856261in}{2.264164in}}%
\pgfpathcurveto{\pgfqpoint{0.864497in}{2.264164in}}{\pgfqpoint{0.872397in}{2.267437in}}{\pgfqpoint{0.878221in}{2.273261in}}%
\pgfpathcurveto{\pgfqpoint{0.884045in}{2.279085in}}{\pgfqpoint{0.887317in}{2.286985in}}{\pgfqpoint{0.887317in}{2.295221in}}%
\pgfpathcurveto{\pgfqpoint{0.887317in}{2.303457in}}{\pgfqpoint{0.884045in}{2.311357in}}{\pgfqpoint{0.878221in}{2.317181in}}%
\pgfpathcurveto{\pgfqpoint{0.872397in}{2.323005in}}{\pgfqpoint{0.864497in}{2.326277in}}{\pgfqpoint{0.856261in}{2.326277in}}%
\pgfpathcurveto{\pgfqpoint{0.848024in}{2.326277in}}{\pgfqpoint{0.840124in}{2.323005in}}{\pgfqpoint{0.834300in}{2.317181in}}%
\pgfpathcurveto{\pgfqpoint{0.828476in}{2.311357in}}{\pgfqpoint{0.825204in}{2.303457in}}{\pgfqpoint{0.825204in}{2.295221in}}%
\pgfpathcurveto{\pgfqpoint{0.825204in}{2.286985in}}{\pgfqpoint{0.828476in}{2.279085in}}{\pgfqpoint{0.834300in}{2.273261in}}%
\pgfpathcurveto{\pgfqpoint{0.840124in}{2.267437in}}{\pgfqpoint{0.848024in}{2.264164in}}{\pgfqpoint{0.856261in}{2.264164in}}%
\pgfpathclose%
\pgfusepath{stroke,fill}%
\end{pgfscope}%
\begin{pgfscope}%
\pgfpathrectangle{\pgfqpoint{0.100000in}{0.220728in}}{\pgfqpoint{3.696000in}{3.696000in}}%
\pgfusepath{clip}%
\pgfsetbuttcap%
\pgfsetroundjoin%
\definecolor{currentfill}{rgb}{0.121569,0.466667,0.705882}%
\pgfsetfillcolor{currentfill}%
\pgfsetfillopacity{0.712197}%
\pgfsetlinewidth{1.003750pt}%
\definecolor{currentstroke}{rgb}{0.121569,0.466667,0.705882}%
\pgfsetstrokecolor{currentstroke}%
\pgfsetstrokeopacity{0.712197}%
\pgfsetdash{}{0pt}%
\pgfpathmoveto{\pgfqpoint{0.858387in}{2.259116in}}%
\pgfpathcurveto{\pgfqpoint{0.866624in}{2.259116in}}{\pgfqpoint{0.874524in}{2.262389in}}{\pgfqpoint{0.880348in}{2.268213in}}%
\pgfpathcurveto{\pgfqpoint{0.886172in}{2.274037in}}{\pgfqpoint{0.889444in}{2.281937in}}{\pgfqpoint{0.889444in}{2.290173in}}%
\pgfpathcurveto{\pgfqpoint{0.889444in}{2.298409in}}{\pgfqpoint{0.886172in}{2.306309in}}{\pgfqpoint{0.880348in}{2.312133in}}%
\pgfpathcurveto{\pgfqpoint{0.874524in}{2.317957in}}{\pgfqpoint{0.866624in}{2.321229in}}{\pgfqpoint{0.858387in}{2.321229in}}%
\pgfpathcurveto{\pgfqpoint{0.850151in}{2.321229in}}{\pgfqpoint{0.842251in}{2.317957in}}{\pgfqpoint{0.836427in}{2.312133in}}%
\pgfpathcurveto{\pgfqpoint{0.830603in}{2.306309in}}{\pgfqpoint{0.827331in}{2.298409in}}{\pgfqpoint{0.827331in}{2.290173in}}%
\pgfpathcurveto{\pgfqpoint{0.827331in}{2.281937in}}{\pgfqpoint{0.830603in}{2.274037in}}{\pgfqpoint{0.836427in}{2.268213in}}%
\pgfpathcurveto{\pgfqpoint{0.842251in}{2.262389in}}{\pgfqpoint{0.850151in}{2.259116in}}{\pgfqpoint{0.858387in}{2.259116in}}%
\pgfpathclose%
\pgfusepath{stroke,fill}%
\end{pgfscope}%
\begin{pgfscope}%
\pgfpathrectangle{\pgfqpoint{0.100000in}{0.220728in}}{\pgfqpoint{3.696000in}{3.696000in}}%
\pgfusepath{clip}%
\pgfsetbuttcap%
\pgfsetroundjoin%
\definecolor{currentfill}{rgb}{0.121569,0.466667,0.705882}%
\pgfsetfillcolor{currentfill}%
\pgfsetfillopacity{0.712387}%
\pgfsetlinewidth{1.003750pt}%
\definecolor{currentstroke}{rgb}{0.121569,0.466667,0.705882}%
\pgfsetstrokecolor{currentstroke}%
\pgfsetstrokeopacity{0.712387}%
\pgfsetdash{}{0pt}%
\pgfpathmoveto{\pgfqpoint{3.296522in}{2.818015in}}%
\pgfpathcurveto{\pgfqpoint{3.304758in}{2.818015in}}{\pgfqpoint{3.312658in}{2.821287in}}{\pgfqpoint{3.318482in}{2.827111in}}%
\pgfpathcurveto{\pgfqpoint{3.324306in}{2.832935in}}{\pgfqpoint{3.327578in}{2.840835in}}{\pgfqpoint{3.327578in}{2.849072in}}%
\pgfpathcurveto{\pgfqpoint{3.327578in}{2.857308in}}{\pgfqpoint{3.324306in}{2.865208in}}{\pgfqpoint{3.318482in}{2.871032in}}%
\pgfpathcurveto{\pgfqpoint{3.312658in}{2.876856in}}{\pgfqpoint{3.304758in}{2.880128in}}{\pgfqpoint{3.296522in}{2.880128in}}%
\pgfpathcurveto{\pgfqpoint{3.288285in}{2.880128in}}{\pgfqpoint{3.280385in}{2.876856in}}{\pgfqpoint{3.274561in}{2.871032in}}%
\pgfpathcurveto{\pgfqpoint{3.268737in}{2.865208in}}{\pgfqpoint{3.265465in}{2.857308in}}{\pgfqpoint{3.265465in}{2.849072in}}%
\pgfpathcurveto{\pgfqpoint{3.265465in}{2.840835in}}{\pgfqpoint{3.268737in}{2.832935in}}{\pgfqpoint{3.274561in}{2.827111in}}%
\pgfpathcurveto{\pgfqpoint{3.280385in}{2.821287in}}{\pgfqpoint{3.288285in}{2.818015in}}{\pgfqpoint{3.296522in}{2.818015in}}%
\pgfpathclose%
\pgfusepath{stroke,fill}%
\end{pgfscope}%
\begin{pgfscope}%
\pgfpathrectangle{\pgfqpoint{0.100000in}{0.220728in}}{\pgfqpoint{3.696000in}{3.696000in}}%
\pgfusepath{clip}%
\pgfsetbuttcap%
\pgfsetroundjoin%
\definecolor{currentfill}{rgb}{0.121569,0.466667,0.705882}%
\pgfsetfillcolor{currentfill}%
\pgfsetfillopacity{0.713754}%
\pgfsetlinewidth{1.003750pt}%
\definecolor{currentstroke}{rgb}{0.121569,0.466667,0.705882}%
\pgfsetstrokecolor{currentstroke}%
\pgfsetstrokeopacity{0.713754}%
\pgfsetdash{}{0pt}%
\pgfpathmoveto{\pgfqpoint{0.864646in}{2.253092in}}%
\pgfpathcurveto{\pgfqpoint{0.872883in}{2.253092in}}{\pgfqpoint{0.880783in}{2.256365in}}{\pgfqpoint{0.886607in}{2.262188in}}%
\pgfpathcurveto{\pgfqpoint{0.892431in}{2.268012in}}{\pgfqpoint{0.895703in}{2.275912in}}{\pgfqpoint{0.895703in}{2.284149in}}%
\pgfpathcurveto{\pgfqpoint{0.895703in}{2.292385in}}{\pgfqpoint{0.892431in}{2.300285in}}{\pgfqpoint{0.886607in}{2.306109in}}%
\pgfpathcurveto{\pgfqpoint{0.880783in}{2.311933in}}{\pgfqpoint{0.872883in}{2.315205in}}{\pgfqpoint{0.864646in}{2.315205in}}%
\pgfpathcurveto{\pgfqpoint{0.856410in}{2.315205in}}{\pgfqpoint{0.848510in}{2.311933in}}{\pgfqpoint{0.842686in}{2.306109in}}%
\pgfpathcurveto{\pgfqpoint{0.836862in}{2.300285in}}{\pgfqpoint{0.833590in}{2.292385in}}{\pgfqpoint{0.833590in}{2.284149in}}%
\pgfpathcurveto{\pgfqpoint{0.833590in}{2.275912in}}{\pgfqpoint{0.836862in}{2.268012in}}{\pgfqpoint{0.842686in}{2.262188in}}%
\pgfpathcurveto{\pgfqpoint{0.848510in}{2.256365in}}{\pgfqpoint{0.856410in}{2.253092in}}{\pgfqpoint{0.864646in}{2.253092in}}%
\pgfpathclose%
\pgfusepath{stroke,fill}%
\end{pgfscope}%
\begin{pgfscope}%
\pgfpathrectangle{\pgfqpoint{0.100000in}{0.220728in}}{\pgfqpoint{3.696000in}{3.696000in}}%
\pgfusepath{clip}%
\pgfsetbuttcap%
\pgfsetroundjoin%
\definecolor{currentfill}{rgb}{0.121569,0.466667,0.705882}%
\pgfsetfillcolor{currentfill}%
\pgfsetfillopacity{0.714838}%
\pgfsetlinewidth{1.003750pt}%
\definecolor{currentstroke}{rgb}{0.121569,0.466667,0.705882}%
\pgfsetstrokecolor{currentstroke}%
\pgfsetstrokeopacity{0.714838}%
\pgfsetdash{}{0pt}%
\pgfpathmoveto{\pgfqpoint{0.869069in}{2.247114in}}%
\pgfpathcurveto{\pgfqpoint{0.877306in}{2.247114in}}{\pgfqpoint{0.885206in}{2.250386in}}{\pgfqpoint{0.891030in}{2.256210in}}%
\pgfpathcurveto{\pgfqpoint{0.896854in}{2.262034in}}{\pgfqpoint{0.900126in}{2.269934in}}{\pgfqpoint{0.900126in}{2.278170in}}%
\pgfpathcurveto{\pgfqpoint{0.900126in}{2.286406in}}{\pgfqpoint{0.896854in}{2.294306in}}{\pgfqpoint{0.891030in}{2.300130in}}%
\pgfpathcurveto{\pgfqpoint{0.885206in}{2.305954in}}{\pgfqpoint{0.877306in}{2.309227in}}{\pgfqpoint{0.869069in}{2.309227in}}%
\pgfpathcurveto{\pgfqpoint{0.860833in}{2.309227in}}{\pgfqpoint{0.852933in}{2.305954in}}{\pgfqpoint{0.847109in}{2.300130in}}%
\pgfpathcurveto{\pgfqpoint{0.841285in}{2.294306in}}{\pgfqpoint{0.838013in}{2.286406in}}{\pgfqpoint{0.838013in}{2.278170in}}%
\pgfpathcurveto{\pgfqpoint{0.838013in}{2.269934in}}{\pgfqpoint{0.841285in}{2.262034in}}{\pgfqpoint{0.847109in}{2.256210in}}%
\pgfpathcurveto{\pgfqpoint{0.852933in}{2.250386in}}{\pgfqpoint{0.860833in}{2.247114in}}{\pgfqpoint{0.869069in}{2.247114in}}%
\pgfpathclose%
\pgfusepath{stroke,fill}%
\end{pgfscope}%
\begin{pgfscope}%
\pgfpathrectangle{\pgfqpoint{0.100000in}{0.220728in}}{\pgfqpoint{3.696000in}{3.696000in}}%
\pgfusepath{clip}%
\pgfsetbuttcap%
\pgfsetroundjoin%
\definecolor{currentfill}{rgb}{0.121569,0.466667,0.705882}%
\pgfsetfillcolor{currentfill}%
\pgfsetfillopacity{0.715024}%
\pgfsetlinewidth{1.003750pt}%
\definecolor{currentstroke}{rgb}{0.121569,0.466667,0.705882}%
\pgfsetstrokecolor{currentstroke}%
\pgfsetstrokeopacity{0.715024}%
\pgfsetdash{}{0pt}%
\pgfpathmoveto{\pgfqpoint{0.869748in}{2.246163in}}%
\pgfpathcurveto{\pgfqpoint{0.877985in}{2.246163in}}{\pgfqpoint{0.885885in}{2.249435in}}{\pgfqpoint{0.891709in}{2.255259in}}%
\pgfpathcurveto{\pgfqpoint{0.897532in}{2.261083in}}{\pgfqpoint{0.900805in}{2.268983in}}{\pgfqpoint{0.900805in}{2.277219in}}%
\pgfpathcurveto{\pgfqpoint{0.900805in}{2.285456in}}{\pgfqpoint{0.897532in}{2.293356in}}{\pgfqpoint{0.891709in}{2.299180in}}%
\pgfpathcurveto{\pgfqpoint{0.885885in}{2.305004in}}{\pgfqpoint{0.877985in}{2.308276in}}{\pgfqpoint{0.869748in}{2.308276in}}%
\pgfpathcurveto{\pgfqpoint{0.861512in}{2.308276in}}{\pgfqpoint{0.853612in}{2.305004in}}{\pgfqpoint{0.847788in}{2.299180in}}%
\pgfpathcurveto{\pgfqpoint{0.841964in}{2.293356in}}{\pgfqpoint{0.838692in}{2.285456in}}{\pgfqpoint{0.838692in}{2.277219in}}%
\pgfpathcurveto{\pgfqpoint{0.838692in}{2.268983in}}{\pgfqpoint{0.841964in}{2.261083in}}{\pgfqpoint{0.847788in}{2.255259in}}%
\pgfpathcurveto{\pgfqpoint{0.853612in}{2.249435in}}{\pgfqpoint{0.861512in}{2.246163in}}{\pgfqpoint{0.869748in}{2.246163in}}%
\pgfpathclose%
\pgfusepath{stroke,fill}%
\end{pgfscope}%
\begin{pgfscope}%
\pgfpathrectangle{\pgfqpoint{0.100000in}{0.220728in}}{\pgfqpoint{3.696000in}{3.696000in}}%
\pgfusepath{clip}%
\pgfsetbuttcap%
\pgfsetroundjoin%
\definecolor{currentfill}{rgb}{0.121569,0.466667,0.705882}%
\pgfsetfillcolor{currentfill}%
\pgfsetfillopacity{0.715367}%
\pgfsetlinewidth{1.003750pt}%
\definecolor{currentstroke}{rgb}{0.121569,0.466667,0.705882}%
\pgfsetstrokecolor{currentstroke}%
\pgfsetstrokeopacity{0.715367}%
\pgfsetdash{}{0pt}%
\pgfpathmoveto{\pgfqpoint{0.871196in}{2.244855in}}%
\pgfpathcurveto{\pgfqpoint{0.879432in}{2.244855in}}{\pgfqpoint{0.887332in}{2.248128in}}{\pgfqpoint{0.893156in}{2.253951in}}%
\pgfpathcurveto{\pgfqpoint{0.898980in}{2.259775in}}{\pgfqpoint{0.902252in}{2.267675in}}{\pgfqpoint{0.902252in}{2.275912in}}%
\pgfpathcurveto{\pgfqpoint{0.902252in}{2.284148in}}{\pgfqpoint{0.898980in}{2.292048in}}{\pgfqpoint{0.893156in}{2.297872in}}%
\pgfpathcurveto{\pgfqpoint{0.887332in}{2.303696in}}{\pgfqpoint{0.879432in}{2.306968in}}{\pgfqpoint{0.871196in}{2.306968in}}%
\pgfpathcurveto{\pgfqpoint{0.862960in}{2.306968in}}{\pgfqpoint{0.855060in}{2.303696in}}{\pgfqpoint{0.849236in}{2.297872in}}%
\pgfpathcurveto{\pgfqpoint{0.843412in}{2.292048in}}{\pgfqpoint{0.840139in}{2.284148in}}{\pgfqpoint{0.840139in}{2.275912in}}%
\pgfpathcurveto{\pgfqpoint{0.840139in}{2.267675in}}{\pgfqpoint{0.843412in}{2.259775in}}{\pgfqpoint{0.849236in}{2.253951in}}%
\pgfpathcurveto{\pgfqpoint{0.855060in}{2.248128in}}{\pgfqpoint{0.862960in}{2.244855in}}{\pgfqpoint{0.871196in}{2.244855in}}%
\pgfpathclose%
\pgfusepath{stroke,fill}%
\end{pgfscope}%
\begin{pgfscope}%
\pgfpathrectangle{\pgfqpoint{0.100000in}{0.220728in}}{\pgfqpoint{3.696000in}{3.696000in}}%
\pgfusepath{clip}%
\pgfsetbuttcap%
\pgfsetroundjoin%
\definecolor{currentfill}{rgb}{0.121569,0.466667,0.705882}%
\pgfsetfillcolor{currentfill}%
\pgfsetfillopacity{0.715709}%
\pgfsetlinewidth{1.003750pt}%
\definecolor{currentstroke}{rgb}{0.121569,0.466667,0.705882}%
\pgfsetstrokecolor{currentstroke}%
\pgfsetstrokeopacity{0.715709}%
\pgfsetdash{}{0pt}%
\pgfpathmoveto{\pgfqpoint{3.284508in}{2.799530in}}%
\pgfpathcurveto{\pgfqpoint{3.292744in}{2.799530in}}{\pgfqpoint{3.300644in}{2.802802in}}{\pgfqpoint{3.306468in}{2.808626in}}%
\pgfpathcurveto{\pgfqpoint{3.312292in}{2.814450in}}{\pgfqpoint{3.315565in}{2.822350in}}{\pgfqpoint{3.315565in}{2.830586in}}%
\pgfpathcurveto{\pgfqpoint{3.315565in}{2.838823in}}{\pgfqpoint{3.312292in}{2.846723in}}{\pgfqpoint{3.306468in}{2.852547in}}%
\pgfpathcurveto{\pgfqpoint{3.300644in}{2.858371in}}{\pgfqpoint{3.292744in}{2.861643in}}{\pgfqpoint{3.284508in}{2.861643in}}%
\pgfpathcurveto{\pgfqpoint{3.276272in}{2.861643in}}{\pgfqpoint{3.268372in}{2.858371in}}{\pgfqpoint{3.262548in}{2.852547in}}%
\pgfpathcurveto{\pgfqpoint{3.256724in}{2.846723in}}{\pgfqpoint{3.253452in}{2.838823in}}{\pgfqpoint{3.253452in}{2.830586in}}%
\pgfpathcurveto{\pgfqpoint{3.253452in}{2.822350in}}{\pgfqpoint{3.256724in}{2.814450in}}{\pgfqpoint{3.262548in}{2.808626in}}%
\pgfpathcurveto{\pgfqpoint{3.268372in}{2.802802in}}{\pgfqpoint{3.276272in}{2.799530in}}{\pgfqpoint{3.284508in}{2.799530in}}%
\pgfpathclose%
\pgfusepath{stroke,fill}%
\end{pgfscope}%
\begin{pgfscope}%
\pgfpathrectangle{\pgfqpoint{0.100000in}{0.220728in}}{\pgfqpoint{3.696000in}{3.696000in}}%
\pgfusepath{clip}%
\pgfsetbuttcap%
\pgfsetroundjoin%
\definecolor{currentfill}{rgb}{0.121569,0.466667,0.705882}%
\pgfsetfillcolor{currentfill}%
\pgfsetfillopacity{0.715969}%
\pgfsetlinewidth{1.003750pt}%
\definecolor{currentstroke}{rgb}{0.121569,0.466667,0.705882}%
\pgfsetstrokecolor{currentstroke}%
\pgfsetstrokeopacity{0.715969}%
\pgfsetdash{}{0pt}%
\pgfpathmoveto{\pgfqpoint{0.873526in}{2.241795in}}%
\pgfpathcurveto{\pgfqpoint{0.881762in}{2.241795in}}{\pgfqpoint{0.889662in}{2.245067in}}{\pgfqpoint{0.895486in}{2.250891in}}%
\pgfpathcurveto{\pgfqpoint{0.901310in}{2.256715in}}{\pgfqpoint{0.904582in}{2.264615in}}{\pgfqpoint{0.904582in}{2.272852in}}%
\pgfpathcurveto{\pgfqpoint{0.904582in}{2.281088in}}{\pgfqpoint{0.901310in}{2.288988in}}{\pgfqpoint{0.895486in}{2.294812in}}%
\pgfpathcurveto{\pgfqpoint{0.889662in}{2.300636in}}{\pgfqpoint{0.881762in}{2.303908in}}{\pgfqpoint{0.873526in}{2.303908in}}%
\pgfpathcurveto{\pgfqpoint{0.865290in}{2.303908in}}{\pgfqpoint{0.857390in}{2.300636in}}{\pgfqpoint{0.851566in}{2.294812in}}%
\pgfpathcurveto{\pgfqpoint{0.845742in}{2.288988in}}{\pgfqpoint{0.842469in}{2.281088in}}{\pgfqpoint{0.842469in}{2.272852in}}%
\pgfpathcurveto{\pgfqpoint{0.842469in}{2.264615in}}{\pgfqpoint{0.845742in}{2.256715in}}{\pgfqpoint{0.851566in}{2.250891in}}%
\pgfpathcurveto{\pgfqpoint{0.857390in}{2.245067in}}{\pgfqpoint{0.865290in}{2.241795in}}{\pgfqpoint{0.873526in}{2.241795in}}%
\pgfpathclose%
\pgfusepath{stroke,fill}%
\end{pgfscope}%
\begin{pgfscope}%
\pgfpathrectangle{\pgfqpoint{0.100000in}{0.220728in}}{\pgfqpoint{3.696000in}{3.696000in}}%
\pgfusepath{clip}%
\pgfsetbuttcap%
\pgfsetroundjoin%
\definecolor{currentfill}{rgb}{0.121569,0.466667,0.705882}%
\pgfsetfillcolor{currentfill}%
\pgfsetfillopacity{0.717087}%
\pgfsetlinewidth{1.003750pt}%
\definecolor{currentstroke}{rgb}{0.121569,0.466667,0.705882}%
\pgfsetstrokecolor{currentstroke}%
\pgfsetstrokeopacity{0.717087}%
\pgfsetdash{}{0pt}%
\pgfpathmoveto{\pgfqpoint{0.878185in}{2.237111in}}%
\pgfpathcurveto{\pgfqpoint{0.886421in}{2.237111in}}{\pgfqpoint{0.894321in}{2.240383in}}{\pgfqpoint{0.900145in}{2.246207in}}%
\pgfpathcurveto{\pgfqpoint{0.905969in}{2.252031in}}{\pgfqpoint{0.909241in}{2.259931in}}{\pgfqpoint{0.909241in}{2.268168in}}%
\pgfpathcurveto{\pgfqpoint{0.909241in}{2.276404in}}{\pgfqpoint{0.905969in}{2.284304in}}{\pgfqpoint{0.900145in}{2.290128in}}%
\pgfpathcurveto{\pgfqpoint{0.894321in}{2.295952in}}{\pgfqpoint{0.886421in}{2.299224in}}{\pgfqpoint{0.878185in}{2.299224in}}%
\pgfpathcurveto{\pgfqpoint{0.869949in}{2.299224in}}{\pgfqpoint{0.862049in}{2.295952in}}{\pgfqpoint{0.856225in}{2.290128in}}%
\pgfpathcurveto{\pgfqpoint{0.850401in}{2.284304in}}{\pgfqpoint{0.847128in}{2.276404in}}{\pgfqpoint{0.847128in}{2.268168in}}%
\pgfpathcurveto{\pgfqpoint{0.847128in}{2.259931in}}{\pgfqpoint{0.850401in}{2.252031in}}{\pgfqpoint{0.856225in}{2.246207in}}%
\pgfpathcurveto{\pgfqpoint{0.862049in}{2.240383in}}{\pgfqpoint{0.869949in}{2.237111in}}{\pgfqpoint{0.878185in}{2.237111in}}%
\pgfpathclose%
\pgfusepath{stroke,fill}%
\end{pgfscope}%
\begin{pgfscope}%
\pgfpathrectangle{\pgfqpoint{0.100000in}{0.220728in}}{\pgfqpoint{3.696000in}{3.696000in}}%
\pgfusepath{clip}%
\pgfsetbuttcap%
\pgfsetroundjoin%
\definecolor{currentfill}{rgb}{0.121569,0.466667,0.705882}%
\pgfsetfillcolor{currentfill}%
\pgfsetfillopacity{0.717495}%
\pgfsetlinewidth{1.003750pt}%
\definecolor{currentstroke}{rgb}{0.121569,0.466667,0.705882}%
\pgfsetstrokecolor{currentstroke}%
\pgfsetstrokeopacity{0.717495}%
\pgfsetdash{}{0pt}%
\pgfpathmoveto{\pgfqpoint{3.277599in}{2.789595in}}%
\pgfpathcurveto{\pgfqpoint{3.285835in}{2.789595in}}{\pgfqpoint{3.293735in}{2.792867in}}{\pgfqpoint{3.299559in}{2.798691in}}%
\pgfpathcurveto{\pgfqpoint{3.305383in}{2.804515in}}{\pgfqpoint{3.308656in}{2.812415in}}{\pgfqpoint{3.308656in}{2.820651in}}%
\pgfpathcurveto{\pgfqpoint{3.308656in}{2.828887in}}{\pgfqpoint{3.305383in}{2.836787in}}{\pgfqpoint{3.299559in}{2.842611in}}%
\pgfpathcurveto{\pgfqpoint{3.293735in}{2.848435in}}{\pgfqpoint{3.285835in}{2.851708in}}{\pgfqpoint{3.277599in}{2.851708in}}%
\pgfpathcurveto{\pgfqpoint{3.269363in}{2.851708in}}{\pgfqpoint{3.261463in}{2.848435in}}{\pgfqpoint{3.255639in}{2.842611in}}%
\pgfpathcurveto{\pgfqpoint{3.249815in}{2.836787in}}{\pgfqpoint{3.246543in}{2.828887in}}{\pgfqpoint{3.246543in}{2.820651in}}%
\pgfpathcurveto{\pgfqpoint{3.246543in}{2.812415in}}{\pgfqpoint{3.249815in}{2.804515in}}{\pgfqpoint{3.255639in}{2.798691in}}%
\pgfpathcurveto{\pgfqpoint{3.261463in}{2.792867in}}{\pgfqpoint{3.269363in}{2.789595in}}{\pgfqpoint{3.277599in}{2.789595in}}%
\pgfpathclose%
\pgfusepath{stroke,fill}%
\end{pgfscope}%
\begin{pgfscope}%
\pgfpathrectangle{\pgfqpoint{0.100000in}{0.220728in}}{\pgfqpoint{3.696000in}{3.696000in}}%
\pgfusepath{clip}%
\pgfsetbuttcap%
\pgfsetroundjoin%
\definecolor{currentfill}{rgb}{0.121569,0.466667,0.705882}%
\pgfsetfillcolor{currentfill}%
\pgfsetfillopacity{0.718470}%
\pgfsetlinewidth{1.003750pt}%
\definecolor{currentstroke}{rgb}{0.121569,0.466667,0.705882}%
\pgfsetstrokecolor{currentstroke}%
\pgfsetstrokeopacity{0.718470}%
\pgfsetdash{}{0pt}%
\pgfpathmoveto{\pgfqpoint{3.274036in}{2.783723in}}%
\pgfpathcurveto{\pgfqpoint{3.282273in}{2.783723in}}{\pgfqpoint{3.290173in}{2.786995in}}{\pgfqpoint{3.295997in}{2.792819in}}%
\pgfpathcurveto{\pgfqpoint{3.301820in}{2.798643in}}{\pgfqpoint{3.305093in}{2.806543in}}{\pgfqpoint{3.305093in}{2.814779in}}%
\pgfpathcurveto{\pgfqpoint{3.305093in}{2.823016in}}{\pgfqpoint{3.301820in}{2.830916in}}{\pgfqpoint{3.295997in}{2.836740in}}%
\pgfpathcurveto{\pgfqpoint{3.290173in}{2.842564in}}{\pgfqpoint{3.282273in}{2.845836in}}{\pgfqpoint{3.274036in}{2.845836in}}%
\pgfpathcurveto{\pgfqpoint{3.265800in}{2.845836in}}{\pgfqpoint{3.257900in}{2.842564in}}{\pgfqpoint{3.252076in}{2.836740in}}%
\pgfpathcurveto{\pgfqpoint{3.246252in}{2.830916in}}{\pgfqpoint{3.242980in}{2.823016in}}{\pgfqpoint{3.242980in}{2.814779in}}%
\pgfpathcurveto{\pgfqpoint{3.242980in}{2.806543in}}{\pgfqpoint{3.246252in}{2.798643in}}{\pgfqpoint{3.252076in}{2.792819in}}%
\pgfpathcurveto{\pgfqpoint{3.257900in}{2.786995in}}{\pgfqpoint{3.265800in}{2.783723in}}{\pgfqpoint{3.274036in}{2.783723in}}%
\pgfpathclose%
\pgfusepath{stroke,fill}%
\end{pgfscope}%
\begin{pgfscope}%
\pgfpathrectangle{\pgfqpoint{0.100000in}{0.220728in}}{\pgfqpoint{3.696000in}{3.696000in}}%
\pgfusepath{clip}%
\pgfsetbuttcap%
\pgfsetroundjoin%
\definecolor{currentfill}{rgb}{0.121569,0.466667,0.705882}%
\pgfsetfillcolor{currentfill}%
\pgfsetfillopacity{0.719060}%
\pgfsetlinewidth{1.003750pt}%
\definecolor{currentstroke}{rgb}{0.121569,0.466667,0.705882}%
\pgfsetstrokecolor{currentstroke}%
\pgfsetstrokeopacity{0.719060}%
\pgfsetdash{}{0pt}%
\pgfpathmoveto{\pgfqpoint{0.887763in}{2.230858in}}%
\pgfpathcurveto{\pgfqpoint{0.896000in}{2.230858in}}{\pgfqpoint{0.903900in}{2.234130in}}{\pgfqpoint{0.909724in}{2.239954in}}%
\pgfpathcurveto{\pgfqpoint{0.915548in}{2.245778in}}{\pgfqpoint{0.918820in}{2.253678in}}{\pgfqpoint{0.918820in}{2.261915in}}%
\pgfpathcurveto{\pgfqpoint{0.918820in}{2.270151in}}{\pgfqpoint{0.915548in}{2.278051in}}{\pgfqpoint{0.909724in}{2.283875in}}%
\pgfpathcurveto{\pgfqpoint{0.903900in}{2.289699in}}{\pgfqpoint{0.896000in}{2.292971in}}{\pgfqpoint{0.887763in}{2.292971in}}%
\pgfpathcurveto{\pgfqpoint{0.879527in}{2.292971in}}{\pgfqpoint{0.871627in}{2.289699in}}{\pgfqpoint{0.865803in}{2.283875in}}%
\pgfpathcurveto{\pgfqpoint{0.859979in}{2.278051in}}{\pgfqpoint{0.856707in}{2.270151in}}{\pgfqpoint{0.856707in}{2.261915in}}%
\pgfpathcurveto{\pgfqpoint{0.856707in}{2.253678in}}{\pgfqpoint{0.859979in}{2.245778in}}{\pgfqpoint{0.865803in}{2.239954in}}%
\pgfpathcurveto{\pgfqpoint{0.871627in}{2.234130in}}{\pgfqpoint{0.879527in}{2.230858in}}{\pgfqpoint{0.887763in}{2.230858in}}%
\pgfpathclose%
\pgfusepath{stroke,fill}%
\end{pgfscope}%
\begin{pgfscope}%
\pgfpathrectangle{\pgfqpoint{0.100000in}{0.220728in}}{\pgfqpoint{3.696000in}{3.696000in}}%
\pgfusepath{clip}%
\pgfsetbuttcap%
\pgfsetroundjoin%
\definecolor{currentfill}{rgb}{0.121569,0.466667,0.705882}%
\pgfsetfillcolor{currentfill}%
\pgfsetfillopacity{0.719870}%
\pgfsetlinewidth{1.003750pt}%
\definecolor{currentstroke}{rgb}{0.121569,0.466667,0.705882}%
\pgfsetstrokecolor{currentstroke}%
\pgfsetstrokeopacity{0.719870}%
\pgfsetdash{}{0pt}%
\pgfpathmoveto{\pgfqpoint{3.268763in}{2.775934in}}%
\pgfpathcurveto{\pgfqpoint{3.277000in}{2.775934in}}{\pgfqpoint{3.284900in}{2.779207in}}{\pgfqpoint{3.290724in}{2.785030in}}%
\pgfpathcurveto{\pgfqpoint{3.296548in}{2.790854in}}{\pgfqpoint{3.299820in}{2.798754in}}{\pgfqpoint{3.299820in}{2.806991in}}%
\pgfpathcurveto{\pgfqpoint{3.299820in}{2.815227in}}{\pgfqpoint{3.296548in}{2.823127in}}{\pgfqpoint{3.290724in}{2.828951in}}%
\pgfpathcurveto{\pgfqpoint{3.284900in}{2.834775in}}{\pgfqpoint{3.277000in}{2.838047in}}{\pgfqpoint{3.268763in}{2.838047in}}%
\pgfpathcurveto{\pgfqpoint{3.260527in}{2.838047in}}{\pgfqpoint{3.252627in}{2.834775in}}{\pgfqpoint{3.246803in}{2.828951in}}%
\pgfpathcurveto{\pgfqpoint{3.240979in}{2.823127in}}{\pgfqpoint{3.237707in}{2.815227in}}{\pgfqpoint{3.237707in}{2.806991in}}%
\pgfpathcurveto{\pgfqpoint{3.237707in}{2.798754in}}{\pgfqpoint{3.240979in}{2.790854in}}{\pgfqpoint{3.246803in}{2.785030in}}%
\pgfpathcurveto{\pgfqpoint{3.252627in}{2.779207in}}{\pgfqpoint{3.260527in}{2.775934in}}{\pgfqpoint{3.268763in}{2.775934in}}%
\pgfpathclose%
\pgfusepath{stroke,fill}%
\end{pgfscope}%
\begin{pgfscope}%
\pgfpathrectangle{\pgfqpoint{0.100000in}{0.220728in}}{\pgfqpoint{3.696000in}{3.696000in}}%
\pgfusepath{clip}%
\pgfsetbuttcap%
\pgfsetroundjoin%
\definecolor{currentfill}{rgb}{0.121569,0.466667,0.705882}%
\pgfsetfillcolor{currentfill}%
\pgfsetfillopacity{0.720727}%
\pgfsetlinewidth{1.003750pt}%
\definecolor{currentstroke}{rgb}{0.121569,0.466667,0.705882}%
\pgfsetstrokecolor{currentstroke}%
\pgfsetstrokeopacity{0.720727}%
\pgfsetdash{}{0pt}%
\pgfpathmoveto{\pgfqpoint{3.266119in}{2.771724in}}%
\pgfpathcurveto{\pgfqpoint{3.274355in}{2.771724in}}{\pgfqpoint{3.282255in}{2.774996in}}{\pgfqpoint{3.288079in}{2.780820in}}%
\pgfpathcurveto{\pgfqpoint{3.293903in}{2.786644in}}{\pgfqpoint{3.297175in}{2.794544in}}{\pgfqpoint{3.297175in}{2.802780in}}%
\pgfpathcurveto{\pgfqpoint{3.297175in}{2.811017in}}{\pgfqpoint{3.293903in}{2.818917in}}{\pgfqpoint{3.288079in}{2.824741in}}%
\pgfpathcurveto{\pgfqpoint{3.282255in}{2.830565in}}{\pgfqpoint{3.274355in}{2.833837in}}{\pgfqpoint{3.266119in}{2.833837in}}%
\pgfpathcurveto{\pgfqpoint{3.257882in}{2.833837in}}{\pgfqpoint{3.249982in}{2.830565in}}{\pgfqpoint{3.244158in}{2.824741in}}%
\pgfpathcurveto{\pgfqpoint{3.238334in}{2.818917in}}{\pgfqpoint{3.235062in}{2.811017in}}{\pgfqpoint{3.235062in}{2.802780in}}%
\pgfpathcurveto{\pgfqpoint{3.235062in}{2.794544in}}{\pgfqpoint{3.238334in}{2.786644in}}{\pgfqpoint{3.244158in}{2.780820in}}%
\pgfpathcurveto{\pgfqpoint{3.249982in}{2.774996in}}{\pgfqpoint{3.257882in}{2.771724in}}{\pgfqpoint{3.266119in}{2.771724in}}%
\pgfpathclose%
\pgfusepath{stroke,fill}%
\end{pgfscope}%
\begin{pgfscope}%
\pgfpathrectangle{\pgfqpoint{0.100000in}{0.220728in}}{\pgfqpoint{3.696000in}{3.696000in}}%
\pgfusepath{clip}%
\pgfsetbuttcap%
\pgfsetroundjoin%
\definecolor{currentfill}{rgb}{0.121569,0.466667,0.705882}%
\pgfsetfillcolor{currentfill}%
\pgfsetfillopacity{0.721973}%
\pgfsetlinewidth{1.003750pt}%
\definecolor{currentstroke}{rgb}{0.121569,0.466667,0.705882}%
\pgfsetstrokecolor{currentstroke}%
\pgfsetstrokeopacity{0.721973}%
\pgfsetdash{}{0pt}%
\pgfpathmoveto{\pgfqpoint{3.260894in}{2.764115in}}%
\pgfpathcurveto{\pgfqpoint{3.269131in}{2.764115in}}{\pgfqpoint{3.277031in}{2.767388in}}{\pgfqpoint{3.282855in}{2.773212in}}%
\pgfpathcurveto{\pgfqpoint{3.288679in}{2.779036in}}{\pgfqpoint{3.291951in}{2.786936in}}{\pgfqpoint{3.291951in}{2.795172in}}%
\pgfpathcurveto{\pgfqpoint{3.291951in}{2.803408in}}{\pgfqpoint{3.288679in}{2.811308in}}{\pgfqpoint{3.282855in}{2.817132in}}%
\pgfpathcurveto{\pgfqpoint{3.277031in}{2.822956in}}{\pgfqpoint{3.269131in}{2.826228in}}{\pgfqpoint{3.260894in}{2.826228in}}%
\pgfpathcurveto{\pgfqpoint{3.252658in}{2.826228in}}{\pgfqpoint{3.244758in}{2.822956in}}{\pgfqpoint{3.238934in}{2.817132in}}%
\pgfpathcurveto{\pgfqpoint{3.233110in}{2.811308in}}{\pgfqpoint{3.229838in}{2.803408in}}{\pgfqpoint{3.229838in}{2.795172in}}%
\pgfpathcurveto{\pgfqpoint{3.229838in}{2.786936in}}{\pgfqpoint{3.233110in}{2.779036in}}{\pgfqpoint{3.238934in}{2.773212in}}%
\pgfpathcurveto{\pgfqpoint{3.244758in}{2.767388in}}{\pgfqpoint{3.252658in}{2.764115in}}{\pgfqpoint{3.260894in}{2.764115in}}%
\pgfpathclose%
\pgfusepath{stroke,fill}%
\end{pgfscope}%
\begin{pgfscope}%
\pgfpathrectangle{\pgfqpoint{0.100000in}{0.220728in}}{\pgfqpoint{3.696000in}{3.696000in}}%
\pgfusepath{clip}%
\pgfsetbuttcap%
\pgfsetroundjoin%
\definecolor{currentfill}{rgb}{0.121569,0.466667,0.705882}%
\pgfsetfillcolor{currentfill}%
\pgfsetfillopacity{0.722655}%
\pgfsetlinewidth{1.003750pt}%
\definecolor{currentstroke}{rgb}{0.121569,0.466667,0.705882}%
\pgfsetstrokecolor{currentstroke}%
\pgfsetstrokeopacity{0.722655}%
\pgfsetdash{}{0pt}%
\pgfpathmoveto{\pgfqpoint{0.905104in}{2.219218in}}%
\pgfpathcurveto{\pgfqpoint{0.913340in}{2.219218in}}{\pgfqpoint{0.921240in}{2.222490in}}{\pgfqpoint{0.927064in}{2.228314in}}%
\pgfpathcurveto{\pgfqpoint{0.932888in}{2.234138in}}{\pgfqpoint{0.936160in}{2.242038in}}{\pgfqpoint{0.936160in}{2.250274in}}%
\pgfpathcurveto{\pgfqpoint{0.936160in}{2.258510in}}{\pgfqpoint{0.932888in}{2.266410in}}{\pgfqpoint{0.927064in}{2.272234in}}%
\pgfpathcurveto{\pgfqpoint{0.921240in}{2.278058in}}{\pgfqpoint{0.913340in}{2.281331in}}{\pgfqpoint{0.905104in}{2.281331in}}%
\pgfpathcurveto{\pgfqpoint{0.896868in}{2.281331in}}{\pgfqpoint{0.888968in}{2.278058in}}{\pgfqpoint{0.883144in}{2.272234in}}%
\pgfpathcurveto{\pgfqpoint{0.877320in}{2.266410in}}{\pgfqpoint{0.874047in}{2.258510in}}{\pgfqpoint{0.874047in}{2.250274in}}%
\pgfpathcurveto{\pgfqpoint{0.874047in}{2.242038in}}{\pgfqpoint{0.877320in}{2.234138in}}{\pgfqpoint{0.883144in}{2.228314in}}%
\pgfpathcurveto{\pgfqpoint{0.888968in}{2.222490in}}{\pgfqpoint{0.896868in}{2.219218in}}{\pgfqpoint{0.905104in}{2.219218in}}%
\pgfpathclose%
\pgfusepath{stroke,fill}%
\end{pgfscope}%
\begin{pgfscope}%
\pgfpathrectangle{\pgfqpoint{0.100000in}{0.220728in}}{\pgfqpoint{3.696000in}{3.696000in}}%
\pgfusepath{clip}%
\pgfsetbuttcap%
\pgfsetroundjoin%
\definecolor{currentfill}{rgb}{0.121569,0.466667,0.705882}%
\pgfsetfillcolor{currentfill}%
\pgfsetfillopacity{0.724114}%
\pgfsetlinewidth{1.003750pt}%
\definecolor{currentstroke}{rgb}{0.121569,0.466667,0.705882}%
\pgfsetstrokecolor{currentstroke}%
\pgfsetstrokeopacity{0.724114}%
\pgfsetdash{}{0pt}%
\pgfpathmoveto{\pgfqpoint{3.254299in}{2.753137in}}%
\pgfpathcurveto{\pgfqpoint{3.262535in}{2.753137in}}{\pgfqpoint{3.270435in}{2.756409in}}{\pgfqpoint{3.276259in}{2.762233in}}%
\pgfpathcurveto{\pgfqpoint{3.282083in}{2.768057in}}{\pgfqpoint{3.285355in}{2.775957in}}{\pgfqpoint{3.285355in}{2.784193in}}%
\pgfpathcurveto{\pgfqpoint{3.285355in}{2.792430in}}{\pgfqpoint{3.282083in}{2.800330in}}{\pgfqpoint{3.276259in}{2.806154in}}%
\pgfpathcurveto{\pgfqpoint{3.270435in}{2.811977in}}{\pgfqpoint{3.262535in}{2.815250in}}{\pgfqpoint{3.254299in}{2.815250in}}%
\pgfpathcurveto{\pgfqpoint{3.246063in}{2.815250in}}{\pgfqpoint{3.238163in}{2.811977in}}{\pgfqpoint{3.232339in}{2.806154in}}%
\pgfpathcurveto{\pgfqpoint{3.226515in}{2.800330in}}{\pgfqpoint{3.223242in}{2.792430in}}{\pgfqpoint{3.223242in}{2.784193in}}%
\pgfpathcurveto{\pgfqpoint{3.223242in}{2.775957in}}{\pgfqpoint{3.226515in}{2.768057in}}{\pgfqpoint{3.232339in}{2.762233in}}%
\pgfpathcurveto{\pgfqpoint{3.238163in}{2.756409in}}{\pgfqpoint{3.246063in}{2.753137in}}{\pgfqpoint{3.254299in}{2.753137in}}%
\pgfpathclose%
\pgfusepath{stroke,fill}%
\end{pgfscope}%
\begin{pgfscope}%
\pgfpathrectangle{\pgfqpoint{0.100000in}{0.220728in}}{\pgfqpoint{3.696000in}{3.696000in}}%
\pgfusepath{clip}%
\pgfsetbuttcap%
\pgfsetroundjoin%
\definecolor{currentfill}{rgb}{0.121569,0.466667,0.705882}%
\pgfsetfillcolor{currentfill}%
\pgfsetfillopacity{0.725134}%
\pgfsetlinewidth{1.003750pt}%
\definecolor{currentstroke}{rgb}{0.121569,0.466667,0.705882}%
\pgfsetstrokecolor{currentstroke}%
\pgfsetstrokeopacity{0.725134}%
\pgfsetdash{}{0pt}%
\pgfpathmoveto{\pgfqpoint{3.249939in}{2.747384in}}%
\pgfpathcurveto{\pgfqpoint{3.258175in}{2.747384in}}{\pgfqpoint{3.266075in}{2.750656in}}{\pgfqpoint{3.271899in}{2.756480in}}%
\pgfpathcurveto{\pgfqpoint{3.277723in}{2.762304in}}{\pgfqpoint{3.280995in}{2.770204in}}{\pgfqpoint{3.280995in}{2.778440in}}%
\pgfpathcurveto{\pgfqpoint{3.280995in}{2.786677in}}{\pgfqpoint{3.277723in}{2.794577in}}{\pgfqpoint{3.271899in}{2.800401in}}%
\pgfpathcurveto{\pgfqpoint{3.266075in}{2.806225in}}{\pgfqpoint{3.258175in}{2.809497in}}{\pgfqpoint{3.249939in}{2.809497in}}%
\pgfpathcurveto{\pgfqpoint{3.241702in}{2.809497in}}{\pgfqpoint{3.233802in}{2.806225in}}{\pgfqpoint{3.227978in}{2.800401in}}%
\pgfpathcurveto{\pgfqpoint{3.222154in}{2.794577in}}{\pgfqpoint{3.218882in}{2.786677in}}{\pgfqpoint{3.218882in}{2.778440in}}%
\pgfpathcurveto{\pgfqpoint{3.218882in}{2.770204in}}{\pgfqpoint{3.222154in}{2.762304in}}{\pgfqpoint{3.227978in}{2.756480in}}%
\pgfpathcurveto{\pgfqpoint{3.233802in}{2.750656in}}{\pgfqpoint{3.241702in}{2.747384in}}{\pgfqpoint{3.249939in}{2.747384in}}%
\pgfpathclose%
\pgfusepath{stroke,fill}%
\end{pgfscope}%
\begin{pgfscope}%
\pgfpathrectangle{\pgfqpoint{0.100000in}{0.220728in}}{\pgfqpoint{3.696000in}{3.696000in}}%
\pgfusepath{clip}%
\pgfsetbuttcap%
\pgfsetroundjoin%
\definecolor{currentfill}{rgb}{0.121569,0.466667,0.705882}%
\pgfsetfillcolor{currentfill}%
\pgfsetfillopacity{0.725702}%
\pgfsetlinewidth{1.003750pt}%
\definecolor{currentstroke}{rgb}{0.121569,0.466667,0.705882}%
\pgfsetstrokecolor{currentstroke}%
\pgfsetstrokeopacity{0.725702}%
\pgfsetdash{}{0pt}%
\pgfpathmoveto{\pgfqpoint{3.247626in}{2.744111in}}%
\pgfpathcurveto{\pgfqpoint{3.255862in}{2.744111in}}{\pgfqpoint{3.263762in}{2.747383in}}{\pgfqpoint{3.269586in}{2.753207in}}%
\pgfpathcurveto{\pgfqpoint{3.275410in}{2.759031in}}{\pgfqpoint{3.278682in}{2.766931in}}{\pgfqpoint{3.278682in}{2.775168in}}%
\pgfpathcurveto{\pgfqpoint{3.278682in}{2.783404in}}{\pgfqpoint{3.275410in}{2.791304in}}{\pgfqpoint{3.269586in}{2.797128in}}%
\pgfpathcurveto{\pgfqpoint{3.263762in}{2.802952in}}{\pgfqpoint{3.255862in}{2.806224in}}{\pgfqpoint{3.247626in}{2.806224in}}%
\pgfpathcurveto{\pgfqpoint{3.239390in}{2.806224in}}{\pgfqpoint{3.231490in}{2.802952in}}{\pgfqpoint{3.225666in}{2.797128in}}%
\pgfpathcurveto{\pgfqpoint{3.219842in}{2.791304in}}{\pgfqpoint{3.216569in}{2.783404in}}{\pgfqpoint{3.216569in}{2.775168in}}%
\pgfpathcurveto{\pgfqpoint{3.216569in}{2.766931in}}{\pgfqpoint{3.219842in}{2.759031in}}{\pgfqpoint{3.225666in}{2.753207in}}%
\pgfpathcurveto{\pgfqpoint{3.231490in}{2.747383in}}{\pgfqpoint{3.239390in}{2.744111in}}{\pgfqpoint{3.247626in}{2.744111in}}%
\pgfpathclose%
\pgfusepath{stroke,fill}%
\end{pgfscope}%
\begin{pgfscope}%
\pgfpathrectangle{\pgfqpoint{0.100000in}{0.220728in}}{\pgfqpoint{3.696000in}{3.696000in}}%
\pgfusepath{clip}%
\pgfsetbuttcap%
\pgfsetroundjoin%
\definecolor{currentfill}{rgb}{0.121569,0.466667,0.705882}%
\pgfsetfillcolor{currentfill}%
\pgfsetfillopacity{0.725921}%
\pgfsetlinewidth{1.003750pt}%
\definecolor{currentstroke}{rgb}{0.121569,0.466667,0.705882}%
\pgfsetstrokecolor{currentstroke}%
\pgfsetstrokeopacity{0.725921}%
\pgfsetdash{}{0pt}%
\pgfpathmoveto{\pgfqpoint{0.920563in}{2.207875in}}%
\pgfpathcurveto{\pgfqpoint{0.928800in}{2.207875in}}{\pgfqpoint{0.936700in}{2.211147in}}{\pgfqpoint{0.942524in}{2.216971in}}%
\pgfpathcurveto{\pgfqpoint{0.948348in}{2.222795in}}{\pgfqpoint{0.951620in}{2.230695in}}{\pgfqpoint{0.951620in}{2.238931in}}%
\pgfpathcurveto{\pgfqpoint{0.951620in}{2.247168in}}{\pgfqpoint{0.948348in}{2.255068in}}{\pgfqpoint{0.942524in}{2.260892in}}%
\pgfpathcurveto{\pgfqpoint{0.936700in}{2.266716in}}{\pgfqpoint{0.928800in}{2.269988in}}{\pgfqpoint{0.920563in}{2.269988in}}%
\pgfpathcurveto{\pgfqpoint{0.912327in}{2.269988in}}{\pgfqpoint{0.904427in}{2.266716in}}{\pgfqpoint{0.898603in}{2.260892in}}%
\pgfpathcurveto{\pgfqpoint{0.892779in}{2.255068in}}{\pgfqpoint{0.889507in}{2.247168in}}{\pgfqpoint{0.889507in}{2.238931in}}%
\pgfpathcurveto{\pgfqpoint{0.889507in}{2.230695in}}{\pgfqpoint{0.892779in}{2.222795in}}{\pgfqpoint{0.898603in}{2.216971in}}%
\pgfpathcurveto{\pgfqpoint{0.904427in}{2.211147in}}{\pgfqpoint{0.912327in}{2.207875in}}{\pgfqpoint{0.920563in}{2.207875in}}%
\pgfpathclose%
\pgfusepath{stroke,fill}%
\end{pgfscope}%
\begin{pgfscope}%
\pgfpathrectangle{\pgfqpoint{0.100000in}{0.220728in}}{\pgfqpoint{3.696000in}{3.696000in}}%
\pgfusepath{clip}%
\pgfsetbuttcap%
\pgfsetroundjoin%
\definecolor{currentfill}{rgb}{0.121569,0.466667,0.705882}%
\pgfsetfillcolor{currentfill}%
\pgfsetfillopacity{0.725996}%
\pgfsetlinewidth{1.003750pt}%
\definecolor{currentstroke}{rgb}{0.121569,0.466667,0.705882}%
\pgfsetstrokecolor{currentstroke}%
\pgfsetstrokeopacity{0.725996}%
\pgfsetdash{}{0pt}%
\pgfpathmoveto{\pgfqpoint{3.246356in}{2.742211in}}%
\pgfpathcurveto{\pgfqpoint{3.254592in}{2.742211in}}{\pgfqpoint{3.262492in}{2.745483in}}{\pgfqpoint{3.268316in}{2.751307in}}%
\pgfpathcurveto{\pgfqpoint{3.274140in}{2.757131in}}{\pgfqpoint{3.277412in}{2.765031in}}{\pgfqpoint{3.277412in}{2.773267in}}%
\pgfpathcurveto{\pgfqpoint{3.277412in}{2.781503in}}{\pgfqpoint{3.274140in}{2.789403in}}{\pgfqpoint{3.268316in}{2.795227in}}%
\pgfpathcurveto{\pgfqpoint{3.262492in}{2.801051in}}{\pgfqpoint{3.254592in}{2.804324in}}{\pgfqpoint{3.246356in}{2.804324in}}%
\pgfpathcurveto{\pgfqpoint{3.238119in}{2.804324in}}{\pgfqpoint{3.230219in}{2.801051in}}{\pgfqpoint{3.224395in}{2.795227in}}%
\pgfpathcurveto{\pgfqpoint{3.218571in}{2.789403in}}{\pgfqpoint{3.215299in}{2.781503in}}{\pgfqpoint{3.215299in}{2.773267in}}%
\pgfpathcurveto{\pgfqpoint{3.215299in}{2.765031in}}{\pgfqpoint{3.218571in}{2.757131in}}{\pgfqpoint{3.224395in}{2.751307in}}%
\pgfpathcurveto{\pgfqpoint{3.230219in}{2.745483in}}{\pgfqpoint{3.238119in}{2.742211in}}{\pgfqpoint{3.246356in}{2.742211in}}%
\pgfpathclose%
\pgfusepath{stroke,fill}%
\end{pgfscope}%
\begin{pgfscope}%
\pgfpathrectangle{\pgfqpoint{0.100000in}{0.220728in}}{\pgfqpoint{3.696000in}{3.696000in}}%
\pgfusepath{clip}%
\pgfsetbuttcap%
\pgfsetroundjoin%
\definecolor{currentfill}{rgb}{0.121569,0.466667,0.705882}%
\pgfsetfillcolor{currentfill}%
\pgfsetfillopacity{0.726184}%
\pgfsetlinewidth{1.003750pt}%
\definecolor{currentstroke}{rgb}{0.121569,0.466667,0.705882}%
\pgfsetstrokecolor{currentstroke}%
\pgfsetstrokeopacity{0.726184}%
\pgfsetdash{}{0pt}%
\pgfpathmoveto{\pgfqpoint{3.245649in}{2.741314in}}%
\pgfpathcurveto{\pgfqpoint{3.253886in}{2.741314in}}{\pgfqpoint{3.261786in}{2.744587in}}{\pgfqpoint{3.267610in}{2.750411in}}%
\pgfpathcurveto{\pgfqpoint{3.273433in}{2.756235in}}{\pgfqpoint{3.276706in}{2.764135in}}{\pgfqpoint{3.276706in}{2.772371in}}%
\pgfpathcurveto{\pgfqpoint{3.276706in}{2.780607in}}{\pgfqpoint{3.273433in}{2.788507in}}{\pgfqpoint{3.267610in}{2.794331in}}%
\pgfpathcurveto{\pgfqpoint{3.261786in}{2.800155in}}{\pgfqpoint{3.253886in}{2.803427in}}{\pgfqpoint{3.245649in}{2.803427in}}%
\pgfpathcurveto{\pgfqpoint{3.237413in}{2.803427in}}{\pgfqpoint{3.229513in}{2.800155in}}{\pgfqpoint{3.223689in}{2.794331in}}%
\pgfpathcurveto{\pgfqpoint{3.217865in}{2.788507in}}{\pgfqpoint{3.214593in}{2.780607in}}{\pgfqpoint{3.214593in}{2.772371in}}%
\pgfpathcurveto{\pgfqpoint{3.214593in}{2.764135in}}{\pgfqpoint{3.217865in}{2.756235in}}{\pgfqpoint{3.223689in}{2.750411in}}%
\pgfpathcurveto{\pgfqpoint{3.229513in}{2.744587in}}{\pgfqpoint{3.237413in}{2.741314in}}{\pgfqpoint{3.245649in}{2.741314in}}%
\pgfpathclose%
\pgfusepath{stroke,fill}%
\end{pgfscope}%
\begin{pgfscope}%
\pgfpathrectangle{\pgfqpoint{0.100000in}{0.220728in}}{\pgfqpoint{3.696000in}{3.696000in}}%
\pgfusepath{clip}%
\pgfsetbuttcap%
\pgfsetroundjoin%
\definecolor{currentfill}{rgb}{0.121569,0.466667,0.705882}%
\pgfsetfillcolor{currentfill}%
\pgfsetfillopacity{0.726276}%
\pgfsetlinewidth{1.003750pt}%
\definecolor{currentstroke}{rgb}{0.121569,0.466667,0.705882}%
\pgfsetstrokecolor{currentstroke}%
\pgfsetstrokeopacity{0.726276}%
\pgfsetdash{}{0pt}%
\pgfpathmoveto{\pgfqpoint{3.245269in}{2.740747in}}%
\pgfpathcurveto{\pgfqpoint{3.253505in}{2.740747in}}{\pgfqpoint{3.261405in}{2.744019in}}{\pgfqpoint{3.267229in}{2.749843in}}%
\pgfpathcurveto{\pgfqpoint{3.273053in}{2.755667in}}{\pgfqpoint{3.276326in}{2.763567in}}{\pgfqpoint{3.276326in}{2.771803in}}%
\pgfpathcurveto{\pgfqpoint{3.276326in}{2.780040in}}{\pgfqpoint{3.273053in}{2.787940in}}{\pgfqpoint{3.267229in}{2.793764in}}%
\pgfpathcurveto{\pgfqpoint{3.261405in}{2.799588in}}{\pgfqpoint{3.253505in}{2.802860in}}{\pgfqpoint{3.245269in}{2.802860in}}%
\pgfpathcurveto{\pgfqpoint{3.237033in}{2.802860in}}{\pgfqpoint{3.229133in}{2.799588in}}{\pgfqpoint{3.223309in}{2.793764in}}%
\pgfpathcurveto{\pgfqpoint{3.217485in}{2.787940in}}{\pgfqpoint{3.214213in}{2.780040in}}{\pgfqpoint{3.214213in}{2.771803in}}%
\pgfpathcurveto{\pgfqpoint{3.214213in}{2.763567in}}{\pgfqpoint{3.217485in}{2.755667in}}{\pgfqpoint{3.223309in}{2.749843in}}%
\pgfpathcurveto{\pgfqpoint{3.229133in}{2.744019in}}{\pgfqpoint{3.237033in}{2.740747in}}{\pgfqpoint{3.245269in}{2.740747in}}%
\pgfpathclose%
\pgfusepath{stroke,fill}%
\end{pgfscope}%
\begin{pgfscope}%
\pgfpathrectangle{\pgfqpoint{0.100000in}{0.220728in}}{\pgfqpoint{3.696000in}{3.696000in}}%
\pgfusepath{clip}%
\pgfsetbuttcap%
\pgfsetroundjoin%
\definecolor{currentfill}{rgb}{0.121569,0.466667,0.705882}%
\pgfsetfillcolor{currentfill}%
\pgfsetfillopacity{0.726323}%
\pgfsetlinewidth{1.003750pt}%
\definecolor{currentstroke}{rgb}{0.121569,0.466667,0.705882}%
\pgfsetstrokecolor{currentstroke}%
\pgfsetstrokeopacity{0.726323}%
\pgfsetdash{}{0pt}%
\pgfpathmoveto{\pgfqpoint{3.245031in}{2.740471in}}%
\pgfpathcurveto{\pgfqpoint{3.253268in}{2.740471in}}{\pgfqpoint{3.261168in}{2.743743in}}{\pgfqpoint{3.266992in}{2.749567in}}%
\pgfpathcurveto{\pgfqpoint{3.272816in}{2.755391in}}{\pgfqpoint{3.276088in}{2.763291in}}{\pgfqpoint{3.276088in}{2.771527in}}%
\pgfpathcurveto{\pgfqpoint{3.276088in}{2.779764in}}{\pgfqpoint{3.272816in}{2.787664in}}{\pgfqpoint{3.266992in}{2.793488in}}%
\pgfpathcurveto{\pgfqpoint{3.261168in}{2.799312in}}{\pgfqpoint{3.253268in}{2.802584in}}{\pgfqpoint{3.245031in}{2.802584in}}%
\pgfpathcurveto{\pgfqpoint{3.236795in}{2.802584in}}{\pgfqpoint{3.228895in}{2.799312in}}{\pgfqpoint{3.223071in}{2.793488in}}%
\pgfpathcurveto{\pgfqpoint{3.217247in}{2.787664in}}{\pgfqpoint{3.213975in}{2.779764in}}{\pgfqpoint{3.213975in}{2.771527in}}%
\pgfpathcurveto{\pgfqpoint{3.213975in}{2.763291in}}{\pgfqpoint{3.217247in}{2.755391in}}{\pgfqpoint{3.223071in}{2.749567in}}%
\pgfpathcurveto{\pgfqpoint{3.228895in}{2.743743in}}{\pgfqpoint{3.236795in}{2.740471in}}{\pgfqpoint{3.245031in}{2.740471in}}%
\pgfpathclose%
\pgfusepath{stroke,fill}%
\end{pgfscope}%
\begin{pgfscope}%
\pgfpathrectangle{\pgfqpoint{0.100000in}{0.220728in}}{\pgfqpoint{3.696000in}{3.696000in}}%
\pgfusepath{clip}%
\pgfsetbuttcap%
\pgfsetroundjoin%
\definecolor{currentfill}{rgb}{0.121569,0.466667,0.705882}%
\pgfsetfillcolor{currentfill}%
\pgfsetfillopacity{0.726747}%
\pgfsetlinewidth{1.003750pt}%
\definecolor{currentstroke}{rgb}{0.121569,0.466667,0.705882}%
\pgfsetstrokecolor{currentstroke}%
\pgfsetstrokeopacity{0.726747}%
\pgfsetdash{}{0pt}%
\pgfpathmoveto{\pgfqpoint{3.243620in}{2.738161in}}%
\pgfpathcurveto{\pgfqpoint{3.251856in}{2.738161in}}{\pgfqpoint{3.259756in}{2.741433in}}{\pgfqpoint{3.265580in}{2.747257in}}%
\pgfpathcurveto{\pgfqpoint{3.271404in}{2.753081in}}{\pgfqpoint{3.274676in}{2.760981in}}{\pgfqpoint{3.274676in}{2.769217in}}%
\pgfpathcurveto{\pgfqpoint{3.274676in}{2.777454in}}{\pgfqpoint{3.271404in}{2.785354in}}{\pgfqpoint{3.265580in}{2.791178in}}%
\pgfpathcurveto{\pgfqpoint{3.259756in}{2.797001in}}{\pgfqpoint{3.251856in}{2.800274in}}{\pgfqpoint{3.243620in}{2.800274in}}%
\pgfpathcurveto{\pgfqpoint{3.235384in}{2.800274in}}{\pgfqpoint{3.227483in}{2.797001in}}{\pgfqpoint{3.221660in}{2.791178in}}%
\pgfpathcurveto{\pgfqpoint{3.215836in}{2.785354in}}{\pgfqpoint{3.212563in}{2.777454in}}{\pgfqpoint{3.212563in}{2.769217in}}%
\pgfpathcurveto{\pgfqpoint{3.212563in}{2.760981in}}{\pgfqpoint{3.215836in}{2.753081in}}{\pgfqpoint{3.221660in}{2.747257in}}%
\pgfpathcurveto{\pgfqpoint{3.227483in}{2.741433in}}{\pgfqpoint{3.235384in}{2.738161in}}{\pgfqpoint{3.243620in}{2.738161in}}%
\pgfpathclose%
\pgfusepath{stroke,fill}%
\end{pgfscope}%
\begin{pgfscope}%
\pgfpathrectangle{\pgfqpoint{0.100000in}{0.220728in}}{\pgfqpoint{3.696000in}{3.696000in}}%
\pgfusepath{clip}%
\pgfsetbuttcap%
\pgfsetroundjoin%
\definecolor{currentfill}{rgb}{0.121569,0.466667,0.705882}%
\pgfsetfillcolor{currentfill}%
\pgfsetfillopacity{0.727914}%
\pgfsetlinewidth{1.003750pt}%
\definecolor{currentstroke}{rgb}{0.121569,0.466667,0.705882}%
\pgfsetstrokecolor{currentstroke}%
\pgfsetstrokeopacity{0.727914}%
\pgfsetdash{}{0pt}%
\pgfpathmoveto{\pgfqpoint{3.238242in}{2.732022in}}%
\pgfpathcurveto{\pgfqpoint{3.246478in}{2.732022in}}{\pgfqpoint{3.254378in}{2.735295in}}{\pgfqpoint{3.260202in}{2.741118in}}%
\pgfpathcurveto{\pgfqpoint{3.266026in}{2.746942in}}{\pgfqpoint{3.269299in}{2.754842in}}{\pgfqpoint{3.269299in}{2.763079in}}%
\pgfpathcurveto{\pgfqpoint{3.269299in}{2.771315in}}{\pgfqpoint{3.266026in}{2.779215in}}{\pgfqpoint{3.260202in}{2.785039in}}%
\pgfpathcurveto{\pgfqpoint{3.254378in}{2.790863in}}{\pgfqpoint{3.246478in}{2.794135in}}{\pgfqpoint{3.238242in}{2.794135in}}%
\pgfpathcurveto{\pgfqpoint{3.230006in}{2.794135in}}{\pgfqpoint{3.222106in}{2.790863in}}{\pgfqpoint{3.216282in}{2.785039in}}%
\pgfpathcurveto{\pgfqpoint{3.210458in}{2.779215in}}{\pgfqpoint{3.207186in}{2.771315in}}{\pgfqpoint{3.207186in}{2.763079in}}%
\pgfpathcurveto{\pgfqpoint{3.207186in}{2.754842in}}{\pgfqpoint{3.210458in}{2.746942in}}{\pgfqpoint{3.216282in}{2.741118in}}%
\pgfpathcurveto{\pgfqpoint{3.222106in}{2.735295in}}{\pgfqpoint{3.230006in}{2.732022in}}{\pgfqpoint{3.238242in}{2.732022in}}%
\pgfpathclose%
\pgfusepath{stroke,fill}%
\end{pgfscope}%
\begin{pgfscope}%
\pgfpathrectangle{\pgfqpoint{0.100000in}{0.220728in}}{\pgfqpoint{3.696000in}{3.696000in}}%
\pgfusepath{clip}%
\pgfsetbuttcap%
\pgfsetroundjoin%
\definecolor{currentfill}{rgb}{0.121569,0.466667,0.705882}%
\pgfsetfillcolor{currentfill}%
\pgfsetfillopacity{0.728848}%
\pgfsetlinewidth{1.003750pt}%
\definecolor{currentstroke}{rgb}{0.121569,0.466667,0.705882}%
\pgfsetstrokecolor{currentstroke}%
\pgfsetstrokeopacity{0.728848}%
\pgfsetdash{}{0pt}%
\pgfpathmoveto{\pgfqpoint{0.934862in}{2.200692in}}%
\pgfpathcurveto{\pgfqpoint{0.943098in}{2.200692in}}{\pgfqpoint{0.950998in}{2.203965in}}{\pgfqpoint{0.956822in}{2.209789in}}%
\pgfpathcurveto{\pgfqpoint{0.962646in}{2.215613in}}{\pgfqpoint{0.965918in}{2.223513in}}{\pgfqpoint{0.965918in}{2.231749in}}%
\pgfpathcurveto{\pgfqpoint{0.965918in}{2.239985in}}{\pgfqpoint{0.962646in}{2.247885in}}{\pgfqpoint{0.956822in}{2.253709in}}%
\pgfpathcurveto{\pgfqpoint{0.950998in}{2.259533in}}{\pgfqpoint{0.943098in}{2.262805in}}{\pgfqpoint{0.934862in}{2.262805in}}%
\pgfpathcurveto{\pgfqpoint{0.926625in}{2.262805in}}{\pgfqpoint{0.918725in}{2.259533in}}{\pgfqpoint{0.912901in}{2.253709in}}%
\pgfpathcurveto{\pgfqpoint{0.907077in}{2.247885in}}{\pgfqpoint{0.903805in}{2.239985in}}{\pgfqpoint{0.903805in}{2.231749in}}%
\pgfpathcurveto{\pgfqpoint{0.903805in}{2.223513in}}{\pgfqpoint{0.907077in}{2.215613in}}{\pgfqpoint{0.912901in}{2.209789in}}%
\pgfpathcurveto{\pgfqpoint{0.918725in}{2.203965in}}{\pgfqpoint{0.926625in}{2.200692in}}{\pgfqpoint{0.934862in}{2.200692in}}%
\pgfpathclose%
\pgfusepath{stroke,fill}%
\end{pgfscope}%
\begin{pgfscope}%
\pgfpathrectangle{\pgfqpoint{0.100000in}{0.220728in}}{\pgfqpoint{3.696000in}{3.696000in}}%
\pgfusepath{clip}%
\pgfsetbuttcap%
\pgfsetroundjoin%
\definecolor{currentfill}{rgb}{0.121569,0.466667,0.705882}%
\pgfsetfillcolor{currentfill}%
\pgfsetfillopacity{0.729993}%
\pgfsetlinewidth{1.003750pt}%
\definecolor{currentstroke}{rgb}{0.121569,0.466667,0.705882}%
\pgfsetstrokecolor{currentstroke}%
\pgfsetstrokeopacity{0.729993}%
\pgfsetdash{}{0pt}%
\pgfpathmoveto{\pgfqpoint{3.230944in}{2.720034in}}%
\pgfpathcurveto{\pgfqpoint{3.239180in}{2.720034in}}{\pgfqpoint{3.247080in}{2.723306in}}{\pgfqpoint{3.252904in}{2.729130in}}%
\pgfpathcurveto{\pgfqpoint{3.258728in}{2.734954in}}{\pgfqpoint{3.262000in}{2.742854in}}{\pgfqpoint{3.262000in}{2.751090in}}%
\pgfpathcurveto{\pgfqpoint{3.262000in}{2.759326in}}{\pgfqpoint{3.258728in}{2.767226in}}{\pgfqpoint{3.252904in}{2.773050in}}%
\pgfpathcurveto{\pgfqpoint{3.247080in}{2.778874in}}{\pgfqpoint{3.239180in}{2.782147in}}{\pgfqpoint{3.230944in}{2.782147in}}%
\pgfpathcurveto{\pgfqpoint{3.222707in}{2.782147in}}{\pgfqpoint{3.214807in}{2.778874in}}{\pgfqpoint{3.208983in}{2.773050in}}%
\pgfpathcurveto{\pgfqpoint{3.203159in}{2.767226in}}{\pgfqpoint{3.199887in}{2.759326in}}{\pgfqpoint{3.199887in}{2.751090in}}%
\pgfpathcurveto{\pgfqpoint{3.199887in}{2.742854in}}{\pgfqpoint{3.203159in}{2.734954in}}{\pgfqpoint{3.208983in}{2.729130in}}%
\pgfpathcurveto{\pgfqpoint{3.214807in}{2.723306in}}{\pgfqpoint{3.222707in}{2.720034in}}{\pgfqpoint{3.230944in}{2.720034in}}%
\pgfpathclose%
\pgfusepath{stroke,fill}%
\end{pgfscope}%
\begin{pgfscope}%
\pgfpathrectangle{\pgfqpoint{0.100000in}{0.220728in}}{\pgfqpoint{3.696000in}{3.696000in}}%
\pgfusepath{clip}%
\pgfsetbuttcap%
\pgfsetroundjoin%
\definecolor{currentfill}{rgb}{0.121569,0.466667,0.705882}%
\pgfsetfillcolor{currentfill}%
\pgfsetfillopacity{0.731026}%
\pgfsetlinewidth{1.003750pt}%
\definecolor{currentstroke}{rgb}{0.121569,0.466667,0.705882}%
\pgfsetstrokecolor{currentstroke}%
\pgfsetstrokeopacity{0.731026}%
\pgfsetdash{}{0pt}%
\pgfpathmoveto{\pgfqpoint{0.945180in}{2.193091in}}%
\pgfpathcurveto{\pgfqpoint{0.953416in}{2.193091in}}{\pgfqpoint{0.961317in}{2.196363in}}{\pgfqpoint{0.967140in}{2.202187in}}%
\pgfpathcurveto{\pgfqpoint{0.972964in}{2.208011in}}{\pgfqpoint{0.976237in}{2.215911in}}{\pgfqpoint{0.976237in}{2.224148in}}%
\pgfpathcurveto{\pgfqpoint{0.976237in}{2.232384in}}{\pgfqpoint{0.972964in}{2.240284in}}{\pgfqpoint{0.967140in}{2.246108in}}%
\pgfpathcurveto{\pgfqpoint{0.961317in}{2.251932in}}{\pgfqpoint{0.953416in}{2.255204in}}{\pgfqpoint{0.945180in}{2.255204in}}%
\pgfpathcurveto{\pgfqpoint{0.936944in}{2.255204in}}{\pgfqpoint{0.929044in}{2.251932in}}{\pgfqpoint{0.923220in}{2.246108in}}%
\pgfpathcurveto{\pgfqpoint{0.917396in}{2.240284in}}{\pgfqpoint{0.914124in}{2.232384in}}{\pgfqpoint{0.914124in}{2.224148in}}%
\pgfpathcurveto{\pgfqpoint{0.914124in}{2.215911in}}{\pgfqpoint{0.917396in}{2.208011in}}{\pgfqpoint{0.923220in}{2.202187in}}%
\pgfpathcurveto{\pgfqpoint{0.929044in}{2.196363in}}{\pgfqpoint{0.936944in}{2.193091in}}{\pgfqpoint{0.945180in}{2.193091in}}%
\pgfpathclose%
\pgfusepath{stroke,fill}%
\end{pgfscope}%
\begin{pgfscope}%
\pgfpathrectangle{\pgfqpoint{0.100000in}{0.220728in}}{\pgfqpoint{3.696000in}{3.696000in}}%
\pgfusepath{clip}%
\pgfsetbuttcap%
\pgfsetroundjoin%
\definecolor{currentfill}{rgb}{0.121569,0.466667,0.705882}%
\pgfsetfillcolor{currentfill}%
\pgfsetfillopacity{0.732715}%
\pgfsetlinewidth{1.003750pt}%
\definecolor{currentstroke}{rgb}{0.121569,0.466667,0.705882}%
\pgfsetstrokecolor{currentstroke}%
\pgfsetstrokeopacity{0.732715}%
\pgfsetdash{}{0pt}%
\pgfpathmoveto{\pgfqpoint{0.953874in}{2.188824in}}%
\pgfpathcurveto{\pgfqpoint{0.962111in}{2.188824in}}{\pgfqpoint{0.970011in}{2.192097in}}{\pgfqpoint{0.975835in}{2.197921in}}%
\pgfpathcurveto{\pgfqpoint{0.981659in}{2.203744in}}{\pgfqpoint{0.984931in}{2.211645in}}{\pgfqpoint{0.984931in}{2.219881in}}%
\pgfpathcurveto{\pgfqpoint{0.984931in}{2.228117in}}{\pgfqpoint{0.981659in}{2.236017in}}{\pgfqpoint{0.975835in}{2.241841in}}%
\pgfpathcurveto{\pgfqpoint{0.970011in}{2.247665in}}{\pgfqpoint{0.962111in}{2.250937in}}{\pgfqpoint{0.953874in}{2.250937in}}%
\pgfpathcurveto{\pgfqpoint{0.945638in}{2.250937in}}{\pgfqpoint{0.937738in}{2.247665in}}{\pgfqpoint{0.931914in}{2.241841in}}%
\pgfpathcurveto{\pgfqpoint{0.926090in}{2.236017in}}{\pgfqpoint{0.922818in}{2.228117in}}{\pgfqpoint{0.922818in}{2.219881in}}%
\pgfpathcurveto{\pgfqpoint{0.922818in}{2.211645in}}{\pgfqpoint{0.926090in}{2.203744in}}{\pgfqpoint{0.931914in}{2.197921in}}%
\pgfpathcurveto{\pgfqpoint{0.937738in}{2.192097in}}{\pgfqpoint{0.945638in}{2.188824in}}{\pgfqpoint{0.953874in}{2.188824in}}%
\pgfpathclose%
\pgfusepath{stroke,fill}%
\end{pgfscope}%
\begin{pgfscope}%
\pgfpathrectangle{\pgfqpoint{0.100000in}{0.220728in}}{\pgfqpoint{3.696000in}{3.696000in}}%
\pgfusepath{clip}%
\pgfsetbuttcap%
\pgfsetroundjoin%
\definecolor{currentfill}{rgb}{0.121569,0.466667,0.705882}%
\pgfsetfillcolor{currentfill}%
\pgfsetfillopacity{0.733072}%
\pgfsetlinewidth{1.003750pt}%
\definecolor{currentstroke}{rgb}{0.121569,0.466667,0.705882}%
\pgfsetstrokecolor{currentstroke}%
\pgfsetstrokeopacity{0.733072}%
\pgfsetdash{}{0pt}%
\pgfpathmoveto{\pgfqpoint{3.219356in}{2.705179in}}%
\pgfpathcurveto{\pgfqpoint{3.227592in}{2.705179in}}{\pgfqpoint{3.235492in}{2.708452in}}{\pgfqpoint{3.241316in}{2.714275in}}%
\pgfpathcurveto{\pgfqpoint{3.247140in}{2.720099in}}{\pgfqpoint{3.250412in}{2.727999in}}{\pgfqpoint{3.250412in}{2.736236in}}%
\pgfpathcurveto{\pgfqpoint{3.250412in}{2.744472in}}{\pgfqpoint{3.247140in}{2.752372in}}{\pgfqpoint{3.241316in}{2.758196in}}%
\pgfpathcurveto{\pgfqpoint{3.235492in}{2.764020in}}{\pgfqpoint{3.227592in}{2.767292in}}{\pgfqpoint{3.219356in}{2.767292in}}%
\pgfpathcurveto{\pgfqpoint{3.211120in}{2.767292in}}{\pgfqpoint{3.203219in}{2.764020in}}{\pgfqpoint{3.197396in}{2.758196in}}%
\pgfpathcurveto{\pgfqpoint{3.191572in}{2.752372in}}{\pgfqpoint{3.188299in}{2.744472in}}{\pgfqpoint{3.188299in}{2.736236in}}%
\pgfpathcurveto{\pgfqpoint{3.188299in}{2.727999in}}{\pgfqpoint{3.191572in}{2.720099in}}{\pgfqpoint{3.197396in}{2.714275in}}%
\pgfpathcurveto{\pgfqpoint{3.203219in}{2.708452in}}{\pgfqpoint{3.211120in}{2.705179in}}{\pgfqpoint{3.219356in}{2.705179in}}%
\pgfpathclose%
\pgfusepath{stroke,fill}%
\end{pgfscope}%
\begin{pgfscope}%
\pgfpathrectangle{\pgfqpoint{0.100000in}{0.220728in}}{\pgfqpoint{3.696000in}{3.696000in}}%
\pgfusepath{clip}%
\pgfsetbuttcap%
\pgfsetroundjoin%
\definecolor{currentfill}{rgb}{0.121569,0.466667,0.705882}%
\pgfsetfillcolor{currentfill}%
\pgfsetfillopacity{0.733642}%
\pgfsetlinewidth{1.003750pt}%
\definecolor{currentstroke}{rgb}{0.121569,0.466667,0.705882}%
\pgfsetstrokecolor{currentstroke}%
\pgfsetstrokeopacity{0.733642}%
\pgfsetdash{}{0pt}%
\pgfpathmoveto{\pgfqpoint{0.958367in}{2.185430in}}%
\pgfpathcurveto{\pgfqpoint{0.966603in}{2.185430in}}{\pgfqpoint{0.974503in}{2.188702in}}{\pgfqpoint{0.980327in}{2.194526in}}%
\pgfpathcurveto{\pgfqpoint{0.986151in}{2.200350in}}{\pgfqpoint{0.989423in}{2.208250in}}{\pgfqpoint{0.989423in}{2.216486in}}%
\pgfpathcurveto{\pgfqpoint{0.989423in}{2.224723in}}{\pgfqpoint{0.986151in}{2.232623in}}{\pgfqpoint{0.980327in}{2.238447in}}%
\pgfpathcurveto{\pgfqpoint{0.974503in}{2.244271in}}{\pgfqpoint{0.966603in}{2.247543in}}{\pgfqpoint{0.958367in}{2.247543in}}%
\pgfpathcurveto{\pgfqpoint{0.950130in}{2.247543in}}{\pgfqpoint{0.942230in}{2.244271in}}{\pgfqpoint{0.936406in}{2.238447in}}%
\pgfpathcurveto{\pgfqpoint{0.930582in}{2.232623in}}{\pgfqpoint{0.927310in}{2.224723in}}{\pgfqpoint{0.927310in}{2.216486in}}%
\pgfpathcurveto{\pgfqpoint{0.927310in}{2.208250in}}{\pgfqpoint{0.930582in}{2.200350in}}{\pgfqpoint{0.936406in}{2.194526in}}%
\pgfpathcurveto{\pgfqpoint{0.942230in}{2.188702in}}{\pgfqpoint{0.950130in}{2.185430in}}{\pgfqpoint{0.958367in}{2.185430in}}%
\pgfpathclose%
\pgfusepath{stroke,fill}%
\end{pgfscope}%
\begin{pgfscope}%
\pgfpathrectangle{\pgfqpoint{0.100000in}{0.220728in}}{\pgfqpoint{3.696000in}{3.696000in}}%
\pgfusepath{clip}%
\pgfsetbuttcap%
\pgfsetroundjoin%
\definecolor{currentfill}{rgb}{0.121569,0.466667,0.705882}%
\pgfsetfillcolor{currentfill}%
\pgfsetfillopacity{0.735439}%
\pgfsetlinewidth{1.003750pt}%
\definecolor{currentstroke}{rgb}{0.121569,0.466667,0.705882}%
\pgfsetstrokecolor{currentstroke}%
\pgfsetstrokeopacity{0.735439}%
\pgfsetdash{}{0pt}%
\pgfpathmoveto{\pgfqpoint{0.966564in}{2.179814in}}%
\pgfpathcurveto{\pgfqpoint{0.974800in}{2.179814in}}{\pgfqpoint{0.982700in}{2.183086in}}{\pgfqpoint{0.988524in}{2.188910in}}%
\pgfpathcurveto{\pgfqpoint{0.994348in}{2.194734in}}{\pgfqpoint{0.997620in}{2.202634in}}{\pgfqpoint{0.997620in}{2.210871in}}%
\pgfpathcurveto{\pgfqpoint{0.997620in}{2.219107in}}{\pgfqpoint{0.994348in}{2.227007in}}{\pgfqpoint{0.988524in}{2.232831in}}%
\pgfpathcurveto{\pgfqpoint{0.982700in}{2.238655in}}{\pgfqpoint{0.974800in}{2.241927in}}{\pgfqpoint{0.966564in}{2.241927in}}%
\pgfpathcurveto{\pgfqpoint{0.958327in}{2.241927in}}{\pgfqpoint{0.950427in}{2.238655in}}{\pgfqpoint{0.944603in}{2.232831in}}%
\pgfpathcurveto{\pgfqpoint{0.938780in}{2.227007in}}{\pgfqpoint{0.935507in}{2.219107in}}{\pgfqpoint{0.935507in}{2.210871in}}%
\pgfpathcurveto{\pgfqpoint{0.935507in}{2.202634in}}{\pgfqpoint{0.938780in}{2.194734in}}{\pgfqpoint{0.944603in}{2.188910in}}%
\pgfpathcurveto{\pgfqpoint{0.950427in}{2.183086in}}{\pgfqpoint{0.958327in}{2.179814in}}{\pgfqpoint{0.966564in}{2.179814in}}%
\pgfpathclose%
\pgfusepath{stroke,fill}%
\end{pgfscope}%
\begin{pgfscope}%
\pgfpathrectangle{\pgfqpoint{0.100000in}{0.220728in}}{\pgfqpoint{3.696000in}{3.696000in}}%
\pgfusepath{clip}%
\pgfsetbuttcap%
\pgfsetroundjoin%
\definecolor{currentfill}{rgb}{0.121569,0.466667,0.705882}%
\pgfsetfillcolor{currentfill}%
\pgfsetfillopacity{0.737077}%
\pgfsetlinewidth{1.003750pt}%
\definecolor{currentstroke}{rgb}{0.121569,0.466667,0.705882}%
\pgfsetstrokecolor{currentstroke}%
\pgfsetstrokeopacity{0.737077}%
\pgfsetdash{}{0pt}%
\pgfpathmoveto{\pgfqpoint{3.206166in}{2.684778in}}%
\pgfpathcurveto{\pgfqpoint{3.214403in}{2.684778in}}{\pgfqpoint{3.222303in}{2.688050in}}{\pgfqpoint{3.228127in}{2.693874in}}%
\pgfpathcurveto{\pgfqpoint{3.233951in}{2.699698in}}{\pgfqpoint{3.237223in}{2.707598in}}{\pgfqpoint{3.237223in}{2.715834in}}%
\pgfpathcurveto{\pgfqpoint{3.237223in}{2.724071in}}{\pgfqpoint{3.233951in}{2.731971in}}{\pgfqpoint{3.228127in}{2.737795in}}%
\pgfpathcurveto{\pgfqpoint{3.222303in}{2.743618in}}{\pgfqpoint{3.214403in}{2.746891in}}{\pgfqpoint{3.206166in}{2.746891in}}%
\pgfpathcurveto{\pgfqpoint{3.197930in}{2.746891in}}{\pgfqpoint{3.190030in}{2.743618in}}{\pgfqpoint{3.184206in}{2.737795in}}%
\pgfpathcurveto{\pgfqpoint{3.178382in}{2.731971in}}{\pgfqpoint{3.175110in}{2.724071in}}{\pgfqpoint{3.175110in}{2.715834in}}%
\pgfpathcurveto{\pgfqpoint{3.175110in}{2.707598in}}{\pgfqpoint{3.178382in}{2.699698in}}{\pgfqpoint{3.184206in}{2.693874in}}%
\pgfpathcurveto{\pgfqpoint{3.190030in}{2.688050in}}{\pgfqpoint{3.197930in}{2.684778in}}{\pgfqpoint{3.206166in}{2.684778in}}%
\pgfpathclose%
\pgfusepath{stroke,fill}%
\end{pgfscope}%
\begin{pgfscope}%
\pgfpathrectangle{\pgfqpoint{0.100000in}{0.220728in}}{\pgfqpoint{3.696000in}{3.696000in}}%
\pgfusepath{clip}%
\pgfsetbuttcap%
\pgfsetroundjoin%
\definecolor{currentfill}{rgb}{0.121569,0.466667,0.705882}%
\pgfsetfillcolor{currentfill}%
\pgfsetfillopacity{0.738756}%
\pgfsetlinewidth{1.003750pt}%
\definecolor{currentstroke}{rgb}{0.121569,0.466667,0.705882}%
\pgfsetstrokecolor{currentstroke}%
\pgfsetstrokeopacity{0.738756}%
\pgfsetdash{}{0pt}%
\pgfpathmoveto{\pgfqpoint{0.979293in}{2.165271in}}%
\pgfpathcurveto{\pgfqpoint{0.987530in}{2.165271in}}{\pgfqpoint{0.995430in}{2.168543in}}{\pgfqpoint{1.001254in}{2.174367in}}%
\pgfpathcurveto{\pgfqpoint{1.007077in}{2.180191in}}{\pgfqpoint{1.010350in}{2.188091in}}{\pgfqpoint{1.010350in}{2.196327in}}%
\pgfpathcurveto{\pgfqpoint{1.010350in}{2.204564in}}{\pgfqpoint{1.007077in}{2.212464in}}{\pgfqpoint{1.001254in}{2.218288in}}%
\pgfpathcurveto{\pgfqpoint{0.995430in}{2.224111in}}{\pgfqpoint{0.987530in}{2.227384in}}{\pgfqpoint{0.979293in}{2.227384in}}%
\pgfpathcurveto{\pgfqpoint{0.971057in}{2.227384in}}{\pgfqpoint{0.963157in}{2.224111in}}{\pgfqpoint{0.957333in}{2.218288in}}%
\pgfpathcurveto{\pgfqpoint{0.951509in}{2.212464in}}{\pgfqpoint{0.948237in}{2.204564in}}{\pgfqpoint{0.948237in}{2.196327in}}%
\pgfpathcurveto{\pgfqpoint{0.948237in}{2.188091in}}{\pgfqpoint{0.951509in}{2.180191in}}{\pgfqpoint{0.957333in}{2.174367in}}%
\pgfpathcurveto{\pgfqpoint{0.963157in}{2.168543in}}{\pgfqpoint{0.971057in}{2.165271in}}{\pgfqpoint{0.979293in}{2.165271in}}%
\pgfpathclose%
\pgfusepath{stroke,fill}%
\end{pgfscope}%
\begin{pgfscope}%
\pgfpathrectangle{\pgfqpoint{0.100000in}{0.220728in}}{\pgfqpoint{3.696000in}{3.696000in}}%
\pgfusepath{clip}%
\pgfsetbuttcap%
\pgfsetroundjoin%
\definecolor{currentfill}{rgb}{0.121569,0.466667,0.705882}%
\pgfsetfillcolor{currentfill}%
\pgfsetfillopacity{0.741149}%
\pgfsetlinewidth{1.003750pt}%
\definecolor{currentstroke}{rgb}{0.121569,0.466667,0.705882}%
\pgfsetstrokecolor{currentstroke}%
\pgfsetstrokeopacity{0.741149}%
\pgfsetdash{}{0pt}%
\pgfpathmoveto{\pgfqpoint{3.187946in}{2.661715in}}%
\pgfpathcurveto{\pgfqpoint{3.196182in}{2.661715in}}{\pgfqpoint{3.204082in}{2.664987in}}{\pgfqpoint{3.209906in}{2.670811in}}%
\pgfpathcurveto{\pgfqpoint{3.215730in}{2.676635in}}{\pgfqpoint{3.219002in}{2.684535in}}{\pgfqpoint{3.219002in}{2.692772in}}%
\pgfpathcurveto{\pgfqpoint{3.219002in}{2.701008in}}{\pgfqpoint{3.215730in}{2.708908in}}{\pgfqpoint{3.209906in}{2.714732in}}%
\pgfpathcurveto{\pgfqpoint{3.204082in}{2.720556in}}{\pgfqpoint{3.196182in}{2.723828in}}{\pgfqpoint{3.187946in}{2.723828in}}%
\pgfpathcurveto{\pgfqpoint{3.179709in}{2.723828in}}{\pgfqpoint{3.171809in}{2.720556in}}{\pgfqpoint{3.165985in}{2.714732in}}%
\pgfpathcurveto{\pgfqpoint{3.160161in}{2.708908in}}{\pgfqpoint{3.156889in}{2.701008in}}{\pgfqpoint{3.156889in}{2.692772in}}%
\pgfpathcurveto{\pgfqpoint{3.156889in}{2.684535in}}{\pgfqpoint{3.160161in}{2.676635in}}{\pgfqpoint{3.165985in}{2.670811in}}%
\pgfpathcurveto{\pgfqpoint{3.171809in}{2.664987in}}{\pgfqpoint{3.179709in}{2.661715in}}{\pgfqpoint{3.187946in}{2.661715in}}%
\pgfpathclose%
\pgfusepath{stroke,fill}%
\end{pgfscope}%
\begin{pgfscope}%
\pgfpathrectangle{\pgfqpoint{0.100000in}{0.220728in}}{\pgfqpoint{3.696000in}{3.696000in}}%
\pgfusepath{clip}%
\pgfsetbuttcap%
\pgfsetroundjoin%
\definecolor{currentfill}{rgb}{0.121569,0.466667,0.705882}%
\pgfsetfillcolor{currentfill}%
\pgfsetfillopacity{0.741723}%
\pgfsetlinewidth{1.003750pt}%
\definecolor{currentstroke}{rgb}{0.121569,0.466667,0.705882}%
\pgfsetstrokecolor{currentstroke}%
\pgfsetstrokeopacity{0.741723}%
\pgfsetdash{}{0pt}%
\pgfpathmoveto{\pgfqpoint{0.991296in}{2.154851in}}%
\pgfpathcurveto{\pgfqpoint{0.999532in}{2.154851in}}{\pgfqpoint{1.007432in}{2.158123in}}{\pgfqpoint{1.013256in}{2.163947in}}%
\pgfpathcurveto{\pgfqpoint{1.019080in}{2.169771in}}{\pgfqpoint{1.022352in}{2.177671in}}{\pgfqpoint{1.022352in}{2.185907in}}%
\pgfpathcurveto{\pgfqpoint{1.022352in}{2.194144in}}{\pgfqpoint{1.019080in}{2.202044in}}{\pgfqpoint{1.013256in}{2.207868in}}%
\pgfpathcurveto{\pgfqpoint{1.007432in}{2.213692in}}{\pgfqpoint{0.999532in}{2.216964in}}{\pgfqpoint{0.991296in}{2.216964in}}%
\pgfpathcurveto{\pgfqpoint{0.983059in}{2.216964in}}{\pgfqpoint{0.975159in}{2.213692in}}{\pgfqpoint{0.969335in}{2.207868in}}%
\pgfpathcurveto{\pgfqpoint{0.963511in}{2.202044in}}{\pgfqpoint{0.960239in}{2.194144in}}{\pgfqpoint{0.960239in}{2.185907in}}%
\pgfpathcurveto{\pgfqpoint{0.960239in}{2.177671in}}{\pgfqpoint{0.963511in}{2.169771in}}{\pgfqpoint{0.969335in}{2.163947in}}%
\pgfpathcurveto{\pgfqpoint{0.975159in}{2.158123in}}{\pgfqpoint{0.983059in}{2.154851in}}{\pgfqpoint{0.991296in}{2.154851in}}%
\pgfpathclose%
\pgfusepath{stroke,fill}%
\end{pgfscope}%
\begin{pgfscope}%
\pgfpathrectangle{\pgfqpoint{0.100000in}{0.220728in}}{\pgfqpoint{3.696000in}{3.696000in}}%
\pgfusepath{clip}%
\pgfsetbuttcap%
\pgfsetroundjoin%
\definecolor{currentfill}{rgb}{0.121569,0.466667,0.705882}%
\pgfsetfillcolor{currentfill}%
\pgfsetfillopacity{0.743986}%
\pgfsetlinewidth{1.003750pt}%
\definecolor{currentstroke}{rgb}{0.121569,0.466667,0.705882}%
\pgfsetstrokecolor{currentstroke}%
\pgfsetstrokeopacity{0.743986}%
\pgfsetdash{}{0pt}%
\pgfpathmoveto{\pgfqpoint{1.001523in}{2.149184in}}%
\pgfpathcurveto{\pgfqpoint{1.009759in}{2.149184in}}{\pgfqpoint{1.017659in}{2.152456in}}{\pgfqpoint{1.023483in}{2.158280in}}%
\pgfpathcurveto{\pgfqpoint{1.029307in}{2.164104in}}{\pgfqpoint{1.032579in}{2.172004in}}{\pgfqpoint{1.032579in}{2.180241in}}%
\pgfpathcurveto{\pgfqpoint{1.032579in}{2.188477in}}{\pgfqpoint{1.029307in}{2.196377in}}{\pgfqpoint{1.023483in}{2.202201in}}%
\pgfpathcurveto{\pgfqpoint{1.017659in}{2.208025in}}{\pgfqpoint{1.009759in}{2.211297in}}{\pgfqpoint{1.001523in}{2.211297in}}%
\pgfpathcurveto{\pgfqpoint{0.993286in}{2.211297in}}{\pgfqpoint{0.985386in}{2.208025in}}{\pgfqpoint{0.979562in}{2.202201in}}%
\pgfpathcurveto{\pgfqpoint{0.973739in}{2.196377in}}{\pgfqpoint{0.970466in}{2.188477in}}{\pgfqpoint{0.970466in}{2.180241in}}%
\pgfpathcurveto{\pgfqpoint{0.970466in}{2.172004in}}{\pgfqpoint{0.973739in}{2.164104in}}{\pgfqpoint{0.979562in}{2.158280in}}%
\pgfpathcurveto{\pgfqpoint{0.985386in}{2.152456in}}{\pgfqpoint{0.993286in}{2.149184in}}{\pgfqpoint{1.001523in}{2.149184in}}%
\pgfpathclose%
\pgfusepath{stroke,fill}%
\end{pgfscope}%
\begin{pgfscope}%
\pgfpathrectangle{\pgfqpoint{0.100000in}{0.220728in}}{\pgfqpoint{3.696000in}{3.696000in}}%
\pgfusepath{clip}%
\pgfsetbuttcap%
\pgfsetroundjoin%
\definecolor{currentfill}{rgb}{0.121569,0.466667,0.705882}%
\pgfsetfillcolor{currentfill}%
\pgfsetfillopacity{0.745450}%
\pgfsetlinewidth{1.003750pt}%
\definecolor{currentstroke}{rgb}{0.121569,0.466667,0.705882}%
\pgfsetstrokecolor{currentstroke}%
\pgfsetstrokeopacity{0.745450}%
\pgfsetdash{}{0pt}%
\pgfpathmoveto{\pgfqpoint{1.007544in}{2.143400in}}%
\pgfpathcurveto{\pgfqpoint{1.015780in}{2.143400in}}{\pgfqpoint{1.023680in}{2.146673in}}{\pgfqpoint{1.029504in}{2.152497in}}%
\pgfpathcurveto{\pgfqpoint{1.035328in}{2.158321in}}{\pgfqpoint{1.038600in}{2.166221in}}{\pgfqpoint{1.038600in}{2.174457in}}%
\pgfpathcurveto{\pgfqpoint{1.038600in}{2.182693in}}{\pgfqpoint{1.035328in}{2.190593in}}{\pgfqpoint{1.029504in}{2.196417in}}%
\pgfpathcurveto{\pgfqpoint{1.023680in}{2.202241in}}{\pgfqpoint{1.015780in}{2.205513in}}{\pgfqpoint{1.007544in}{2.205513in}}%
\pgfpathcurveto{\pgfqpoint{0.999307in}{2.205513in}}{\pgfqpoint{0.991407in}{2.202241in}}{\pgfqpoint{0.985583in}{2.196417in}}%
\pgfpathcurveto{\pgfqpoint{0.979759in}{2.190593in}}{\pgfqpoint{0.976487in}{2.182693in}}{\pgfqpoint{0.976487in}{2.174457in}}%
\pgfpathcurveto{\pgfqpoint{0.976487in}{2.166221in}}{\pgfqpoint{0.979759in}{2.158321in}}{\pgfqpoint{0.985583in}{2.152497in}}%
\pgfpathcurveto{\pgfqpoint{0.991407in}{2.146673in}}{\pgfqpoint{0.999307in}{2.143400in}}{\pgfqpoint{1.007544in}{2.143400in}}%
\pgfpathclose%
\pgfusepath{stroke,fill}%
\end{pgfscope}%
\begin{pgfscope}%
\pgfpathrectangle{\pgfqpoint{0.100000in}{0.220728in}}{\pgfqpoint{3.696000in}{3.696000in}}%
\pgfusepath{clip}%
\pgfsetbuttcap%
\pgfsetroundjoin%
\definecolor{currentfill}{rgb}{0.121569,0.466667,0.705882}%
\pgfsetfillcolor{currentfill}%
\pgfsetfillopacity{0.746187}%
\pgfsetlinewidth{1.003750pt}%
\definecolor{currentstroke}{rgb}{0.121569,0.466667,0.705882}%
\pgfsetstrokecolor{currentstroke}%
\pgfsetstrokeopacity{0.746187}%
\pgfsetdash{}{0pt}%
\pgfpathmoveto{\pgfqpoint{1.010753in}{2.139983in}}%
\pgfpathcurveto{\pgfqpoint{1.018989in}{2.139983in}}{\pgfqpoint{1.026889in}{2.143255in}}{\pgfqpoint{1.032713in}{2.149079in}}%
\pgfpathcurveto{\pgfqpoint{1.038537in}{2.154903in}}{\pgfqpoint{1.041809in}{2.162803in}}{\pgfqpoint{1.041809in}{2.171040in}}%
\pgfpathcurveto{\pgfqpoint{1.041809in}{2.179276in}}{\pgfqpoint{1.038537in}{2.187176in}}{\pgfqpoint{1.032713in}{2.193000in}}%
\pgfpathcurveto{\pgfqpoint{1.026889in}{2.198824in}}{\pgfqpoint{1.018989in}{2.202096in}}{\pgfqpoint{1.010753in}{2.202096in}}%
\pgfpathcurveto{\pgfqpoint{1.002516in}{2.202096in}}{\pgfqpoint{0.994616in}{2.198824in}}{\pgfqpoint{0.988792in}{2.193000in}}%
\pgfpathcurveto{\pgfqpoint{0.982968in}{2.187176in}}{\pgfqpoint{0.979696in}{2.179276in}}{\pgfqpoint{0.979696in}{2.171040in}}%
\pgfpathcurveto{\pgfqpoint{0.979696in}{2.162803in}}{\pgfqpoint{0.982968in}{2.154903in}}{\pgfqpoint{0.988792in}{2.149079in}}%
\pgfpathcurveto{\pgfqpoint{0.994616in}{2.143255in}}{\pgfqpoint{1.002516in}{2.139983in}}{\pgfqpoint{1.010753in}{2.139983in}}%
\pgfpathclose%
\pgfusepath{stroke,fill}%
\end{pgfscope}%
\begin{pgfscope}%
\pgfpathrectangle{\pgfqpoint{0.100000in}{0.220728in}}{\pgfqpoint{3.696000in}{3.696000in}}%
\pgfusepath{clip}%
\pgfsetbuttcap%
\pgfsetroundjoin%
\definecolor{currentfill}{rgb}{0.121569,0.466667,0.705882}%
\pgfsetfillcolor{currentfill}%
\pgfsetfillopacity{0.746599}%
\pgfsetlinewidth{1.003750pt}%
\definecolor{currentstroke}{rgb}{0.121569,0.466667,0.705882}%
\pgfsetstrokecolor{currentstroke}%
\pgfsetstrokeopacity{0.746599}%
\pgfsetdash{}{0pt}%
\pgfpathmoveto{\pgfqpoint{1.012533in}{2.138229in}}%
\pgfpathcurveto{\pgfqpoint{1.020769in}{2.138229in}}{\pgfqpoint{1.028669in}{2.141501in}}{\pgfqpoint{1.034493in}{2.147325in}}%
\pgfpathcurveto{\pgfqpoint{1.040317in}{2.153149in}}{\pgfqpoint{1.043589in}{2.161049in}}{\pgfqpoint{1.043589in}{2.169286in}}%
\pgfpathcurveto{\pgfqpoint{1.043589in}{2.177522in}}{\pgfqpoint{1.040317in}{2.185422in}}{\pgfqpoint{1.034493in}{2.191246in}}%
\pgfpathcurveto{\pgfqpoint{1.028669in}{2.197070in}}{\pgfqpoint{1.020769in}{2.200342in}}{\pgfqpoint{1.012533in}{2.200342in}}%
\pgfpathcurveto{\pgfqpoint{1.004296in}{2.200342in}}{\pgfqpoint{0.996396in}{2.197070in}}{\pgfqpoint{0.990572in}{2.191246in}}%
\pgfpathcurveto{\pgfqpoint{0.984748in}{2.185422in}}{\pgfqpoint{0.981476in}{2.177522in}}{\pgfqpoint{0.981476in}{2.169286in}}%
\pgfpathcurveto{\pgfqpoint{0.981476in}{2.161049in}}{\pgfqpoint{0.984748in}{2.153149in}}{\pgfqpoint{0.990572in}{2.147325in}}%
\pgfpathcurveto{\pgfqpoint{0.996396in}{2.141501in}}{\pgfqpoint{1.004296in}{2.138229in}}{\pgfqpoint{1.012533in}{2.138229in}}%
\pgfpathclose%
\pgfusepath{stroke,fill}%
\end{pgfscope}%
\begin{pgfscope}%
\pgfpathrectangle{\pgfqpoint{0.100000in}{0.220728in}}{\pgfqpoint{3.696000in}{3.696000in}}%
\pgfusepath{clip}%
\pgfsetbuttcap%
\pgfsetroundjoin%
\definecolor{currentfill}{rgb}{0.121569,0.466667,0.705882}%
\pgfsetfillcolor{currentfill}%
\pgfsetfillopacity{0.747202}%
\pgfsetlinewidth{1.003750pt}%
\definecolor{currentstroke}{rgb}{0.121569,0.466667,0.705882}%
\pgfsetstrokecolor{currentstroke}%
\pgfsetstrokeopacity{0.747202}%
\pgfsetdash{}{0pt}%
\pgfpathmoveto{\pgfqpoint{3.171045in}{2.632595in}}%
\pgfpathcurveto{\pgfqpoint{3.179281in}{2.632595in}}{\pgfqpoint{3.187181in}{2.635868in}}{\pgfqpoint{3.193005in}{2.641692in}}%
\pgfpathcurveto{\pgfqpoint{3.198829in}{2.647515in}}{\pgfqpoint{3.202101in}{2.655416in}}{\pgfqpoint{3.202101in}{2.663652in}}%
\pgfpathcurveto{\pgfqpoint{3.202101in}{2.671888in}}{\pgfqpoint{3.198829in}{2.679788in}}{\pgfqpoint{3.193005in}{2.685612in}}%
\pgfpathcurveto{\pgfqpoint{3.187181in}{2.691436in}}{\pgfqpoint{3.179281in}{2.694708in}}{\pgfqpoint{3.171045in}{2.694708in}}%
\pgfpathcurveto{\pgfqpoint{3.162809in}{2.694708in}}{\pgfqpoint{3.154909in}{2.691436in}}{\pgfqpoint{3.149085in}{2.685612in}}%
\pgfpathcurveto{\pgfqpoint{3.143261in}{2.679788in}}{\pgfqpoint{3.139988in}{2.671888in}}{\pgfqpoint{3.139988in}{2.663652in}}%
\pgfpathcurveto{\pgfqpoint{3.139988in}{2.655416in}}{\pgfqpoint{3.143261in}{2.647515in}}{\pgfqpoint{3.149085in}{2.641692in}}%
\pgfpathcurveto{\pgfqpoint{3.154909in}{2.635868in}}{\pgfqpoint{3.162809in}{2.632595in}}{\pgfqpoint{3.171045in}{2.632595in}}%
\pgfpathclose%
\pgfusepath{stroke,fill}%
\end{pgfscope}%
\begin{pgfscope}%
\pgfpathrectangle{\pgfqpoint{0.100000in}{0.220728in}}{\pgfqpoint{3.696000in}{3.696000in}}%
\pgfusepath{clip}%
\pgfsetbuttcap%
\pgfsetroundjoin%
\definecolor{currentfill}{rgb}{0.121569,0.466667,0.705882}%
\pgfsetfillcolor{currentfill}%
\pgfsetfillopacity{0.747321}%
\pgfsetlinewidth{1.003750pt}%
\definecolor{currentstroke}{rgb}{0.121569,0.466667,0.705882}%
\pgfsetstrokecolor{currentstroke}%
\pgfsetstrokeopacity{0.747321}%
\pgfsetdash{}{0pt}%
\pgfpathmoveto{\pgfqpoint{1.015725in}{2.134812in}}%
\pgfpathcurveto{\pgfqpoint{1.023962in}{2.134812in}}{\pgfqpoint{1.031862in}{2.138085in}}{\pgfqpoint{1.037686in}{2.143909in}}%
\pgfpathcurveto{\pgfqpoint{1.043510in}{2.149733in}}{\pgfqpoint{1.046782in}{2.157633in}}{\pgfqpoint{1.046782in}{2.165869in}}%
\pgfpathcurveto{\pgfqpoint{1.046782in}{2.174105in}}{\pgfqpoint{1.043510in}{2.182005in}}{\pgfqpoint{1.037686in}{2.187829in}}%
\pgfpathcurveto{\pgfqpoint{1.031862in}{2.193653in}}{\pgfqpoint{1.023962in}{2.196925in}}{\pgfqpoint{1.015725in}{2.196925in}}%
\pgfpathcurveto{\pgfqpoint{1.007489in}{2.196925in}}{\pgfqpoint{0.999589in}{2.193653in}}{\pgfqpoint{0.993765in}{2.187829in}}%
\pgfpathcurveto{\pgfqpoint{0.987941in}{2.182005in}}{\pgfqpoint{0.984669in}{2.174105in}}{\pgfqpoint{0.984669in}{2.165869in}}%
\pgfpathcurveto{\pgfqpoint{0.984669in}{2.157633in}}{\pgfqpoint{0.987941in}{2.149733in}}{\pgfqpoint{0.993765in}{2.143909in}}%
\pgfpathcurveto{\pgfqpoint{0.999589in}{2.138085in}}{\pgfqpoint{1.007489in}{2.134812in}}{\pgfqpoint{1.015725in}{2.134812in}}%
\pgfpathclose%
\pgfusepath{stroke,fill}%
\end{pgfscope}%
\begin{pgfscope}%
\pgfpathrectangle{\pgfqpoint{0.100000in}{0.220728in}}{\pgfqpoint{3.696000in}{3.696000in}}%
\pgfusepath{clip}%
\pgfsetbuttcap%
\pgfsetroundjoin%
\definecolor{currentfill}{rgb}{0.121569,0.466667,0.705882}%
\pgfsetfillcolor{currentfill}%
\pgfsetfillopacity{0.748505}%
\pgfsetlinewidth{1.003750pt}%
\definecolor{currentstroke}{rgb}{0.121569,0.466667,0.705882}%
\pgfsetstrokecolor{currentstroke}%
\pgfsetstrokeopacity{0.748505}%
\pgfsetdash{}{0pt}%
\pgfpathmoveto{\pgfqpoint{1.022185in}{2.129435in}}%
\pgfpathcurveto{\pgfqpoint{1.030421in}{2.129435in}}{\pgfqpoint{1.038321in}{2.132707in}}{\pgfqpoint{1.044145in}{2.138531in}}%
\pgfpathcurveto{\pgfqpoint{1.049969in}{2.144355in}}{\pgfqpoint{1.053241in}{2.152255in}}{\pgfqpoint{1.053241in}{2.160491in}}%
\pgfpathcurveto{\pgfqpoint{1.053241in}{2.168727in}}{\pgfqpoint{1.049969in}{2.176627in}}{\pgfqpoint{1.044145in}{2.182451in}}%
\pgfpathcurveto{\pgfqpoint{1.038321in}{2.188275in}}{\pgfqpoint{1.030421in}{2.191548in}}{\pgfqpoint{1.022185in}{2.191548in}}%
\pgfpathcurveto{\pgfqpoint{1.013948in}{2.191548in}}{\pgfqpoint{1.006048in}{2.188275in}}{\pgfqpoint{1.000224in}{2.182451in}}%
\pgfpathcurveto{\pgfqpoint{0.994400in}{2.176627in}}{\pgfqpoint{0.991128in}{2.168727in}}{\pgfqpoint{0.991128in}{2.160491in}}%
\pgfpathcurveto{\pgfqpoint{0.991128in}{2.152255in}}{\pgfqpoint{0.994400in}{2.144355in}}{\pgfqpoint{1.000224in}{2.138531in}}%
\pgfpathcurveto{\pgfqpoint{1.006048in}{2.132707in}}{\pgfqpoint{1.013948in}{2.129435in}}{\pgfqpoint{1.022185in}{2.129435in}}%
\pgfpathclose%
\pgfusepath{stroke,fill}%
\end{pgfscope}%
\begin{pgfscope}%
\pgfpathrectangle{\pgfqpoint{0.100000in}{0.220728in}}{\pgfqpoint{3.696000in}{3.696000in}}%
\pgfusepath{clip}%
\pgfsetbuttcap%
\pgfsetroundjoin%
\definecolor{currentfill}{rgb}{0.121569,0.466667,0.705882}%
\pgfsetfillcolor{currentfill}%
\pgfsetfillopacity{0.750860}%
\pgfsetlinewidth{1.003750pt}%
\definecolor{currentstroke}{rgb}{0.121569,0.466667,0.705882}%
\pgfsetstrokecolor{currentstroke}%
\pgfsetstrokeopacity{0.750860}%
\pgfsetdash{}{0pt}%
\pgfpathmoveto{\pgfqpoint{1.033710in}{2.120055in}}%
\pgfpathcurveto{\pgfqpoint{1.041946in}{2.120055in}}{\pgfqpoint{1.049846in}{2.123328in}}{\pgfqpoint{1.055670in}{2.129152in}}%
\pgfpathcurveto{\pgfqpoint{1.061494in}{2.134976in}}{\pgfqpoint{1.064766in}{2.142876in}}{\pgfqpoint{1.064766in}{2.151112in}}%
\pgfpathcurveto{\pgfqpoint{1.064766in}{2.159348in}}{\pgfqpoint{1.061494in}{2.167248in}}{\pgfqpoint{1.055670in}{2.173072in}}%
\pgfpathcurveto{\pgfqpoint{1.049846in}{2.178896in}}{\pgfqpoint{1.041946in}{2.182168in}}{\pgfqpoint{1.033710in}{2.182168in}}%
\pgfpathcurveto{\pgfqpoint{1.025473in}{2.182168in}}{\pgfqpoint{1.017573in}{2.178896in}}{\pgfqpoint{1.011749in}{2.173072in}}%
\pgfpathcurveto{\pgfqpoint{1.005925in}{2.167248in}}{\pgfqpoint{1.002653in}{2.159348in}}{\pgfqpoint{1.002653in}{2.151112in}}%
\pgfpathcurveto{\pgfqpoint{1.002653in}{2.142876in}}{\pgfqpoint{1.005925in}{2.134976in}}{\pgfqpoint{1.011749in}{2.129152in}}%
\pgfpathcurveto{\pgfqpoint{1.017573in}{2.123328in}}{\pgfqpoint{1.025473in}{2.120055in}}{\pgfqpoint{1.033710in}{2.120055in}}%
\pgfpathclose%
\pgfusepath{stroke,fill}%
\end{pgfscope}%
\begin{pgfscope}%
\pgfpathrectangle{\pgfqpoint{0.100000in}{0.220728in}}{\pgfqpoint{3.696000in}{3.696000in}}%
\pgfusepath{clip}%
\pgfsetbuttcap%
\pgfsetroundjoin%
\definecolor{currentfill}{rgb}{0.121569,0.466667,0.705882}%
\pgfsetfillcolor{currentfill}%
\pgfsetfillopacity{0.753239}%
\pgfsetlinewidth{1.003750pt}%
\definecolor{currentstroke}{rgb}{0.121569,0.466667,0.705882}%
\pgfsetstrokecolor{currentstroke}%
\pgfsetstrokeopacity{0.753239}%
\pgfsetdash{}{0pt}%
\pgfpathmoveto{\pgfqpoint{3.147593in}{2.600819in}}%
\pgfpathcurveto{\pgfqpoint{3.155830in}{2.600819in}}{\pgfqpoint{3.163730in}{2.604092in}}{\pgfqpoint{3.169554in}{2.609915in}}%
\pgfpathcurveto{\pgfqpoint{3.175378in}{2.615739in}}{\pgfqpoint{3.178650in}{2.623639in}}{\pgfqpoint{3.178650in}{2.631876in}}%
\pgfpathcurveto{\pgfqpoint{3.178650in}{2.640112in}}{\pgfqpoint{3.175378in}{2.648012in}}{\pgfqpoint{3.169554in}{2.653836in}}%
\pgfpathcurveto{\pgfqpoint{3.163730in}{2.659660in}}{\pgfqpoint{3.155830in}{2.662932in}}{\pgfqpoint{3.147593in}{2.662932in}}%
\pgfpathcurveto{\pgfqpoint{3.139357in}{2.662932in}}{\pgfqpoint{3.131457in}{2.659660in}}{\pgfqpoint{3.125633in}{2.653836in}}%
\pgfpathcurveto{\pgfqpoint{3.119809in}{2.648012in}}{\pgfqpoint{3.116537in}{2.640112in}}{\pgfqpoint{3.116537in}{2.631876in}}%
\pgfpathcurveto{\pgfqpoint{3.116537in}{2.623639in}}{\pgfqpoint{3.119809in}{2.615739in}}{\pgfqpoint{3.125633in}{2.609915in}}%
\pgfpathcurveto{\pgfqpoint{3.131457in}{2.604092in}}{\pgfqpoint{3.139357in}{2.600819in}}{\pgfqpoint{3.147593in}{2.600819in}}%
\pgfpathclose%
\pgfusepath{stroke,fill}%
\end{pgfscope}%
\begin{pgfscope}%
\pgfpathrectangle{\pgfqpoint{0.100000in}{0.220728in}}{\pgfqpoint{3.696000in}{3.696000in}}%
\pgfusepath{clip}%
\pgfsetbuttcap%
\pgfsetroundjoin%
\definecolor{currentfill}{rgb}{0.121569,0.466667,0.705882}%
\pgfsetfillcolor{currentfill}%
\pgfsetfillopacity{0.754993}%
\pgfsetlinewidth{1.003750pt}%
\definecolor{currentstroke}{rgb}{0.121569,0.466667,0.705882}%
\pgfsetstrokecolor{currentstroke}%
\pgfsetstrokeopacity{0.754993}%
\pgfsetdash{}{0pt}%
\pgfpathmoveto{\pgfqpoint{1.056653in}{2.107467in}}%
\pgfpathcurveto{\pgfqpoint{1.064889in}{2.107467in}}{\pgfqpoint{1.072789in}{2.110739in}}{\pgfqpoint{1.078613in}{2.116563in}}%
\pgfpathcurveto{\pgfqpoint{1.084437in}{2.122387in}}{\pgfqpoint{1.087709in}{2.130287in}}{\pgfqpoint{1.087709in}{2.138524in}}%
\pgfpathcurveto{\pgfqpoint{1.087709in}{2.146760in}}{\pgfqpoint{1.084437in}{2.154660in}}{\pgfqpoint{1.078613in}{2.160484in}}%
\pgfpathcurveto{\pgfqpoint{1.072789in}{2.166308in}}{\pgfqpoint{1.064889in}{2.169580in}}{\pgfqpoint{1.056653in}{2.169580in}}%
\pgfpathcurveto{\pgfqpoint{1.048416in}{2.169580in}}{\pgfqpoint{1.040516in}{2.166308in}}{\pgfqpoint{1.034692in}{2.160484in}}%
\pgfpathcurveto{\pgfqpoint{1.028868in}{2.154660in}}{\pgfqpoint{1.025596in}{2.146760in}}{\pgfqpoint{1.025596in}{2.138524in}}%
\pgfpathcurveto{\pgfqpoint{1.025596in}{2.130287in}}{\pgfqpoint{1.028868in}{2.122387in}}{\pgfqpoint{1.034692in}{2.116563in}}%
\pgfpathcurveto{\pgfqpoint{1.040516in}{2.110739in}}{\pgfqpoint{1.048416in}{2.107467in}}{\pgfqpoint{1.056653in}{2.107467in}}%
\pgfpathclose%
\pgfusepath{stroke,fill}%
\end{pgfscope}%
\begin{pgfscope}%
\pgfpathrectangle{\pgfqpoint{0.100000in}{0.220728in}}{\pgfqpoint{3.696000in}{3.696000in}}%
\pgfusepath{clip}%
\pgfsetbuttcap%
\pgfsetroundjoin%
\definecolor{currentfill}{rgb}{0.121569,0.466667,0.705882}%
\pgfsetfillcolor{currentfill}%
\pgfsetfillopacity{0.758293}%
\pgfsetlinewidth{1.003750pt}%
\definecolor{currentstroke}{rgb}{0.121569,0.466667,0.705882}%
\pgfsetstrokecolor{currentstroke}%
\pgfsetstrokeopacity{0.758293}%
\pgfsetdash{}{0pt}%
\pgfpathmoveto{\pgfqpoint{1.075398in}{2.101519in}}%
\pgfpathcurveto{\pgfqpoint{1.083634in}{2.101519in}}{\pgfqpoint{1.091534in}{2.104791in}}{\pgfqpoint{1.097358in}{2.110615in}}%
\pgfpathcurveto{\pgfqpoint{1.103182in}{2.116439in}}{\pgfqpoint{1.106454in}{2.124339in}}{\pgfqpoint{1.106454in}{2.132575in}}%
\pgfpathcurveto{\pgfqpoint{1.106454in}{2.140811in}}{\pgfqpoint{1.103182in}{2.148711in}}{\pgfqpoint{1.097358in}{2.154535in}}%
\pgfpathcurveto{\pgfqpoint{1.091534in}{2.160359in}}{\pgfqpoint{1.083634in}{2.163632in}}{\pgfqpoint{1.075398in}{2.163632in}}%
\pgfpathcurveto{\pgfqpoint{1.067161in}{2.163632in}}{\pgfqpoint{1.059261in}{2.160359in}}{\pgfqpoint{1.053437in}{2.154535in}}%
\pgfpathcurveto{\pgfqpoint{1.047614in}{2.148711in}}{\pgfqpoint{1.044341in}{2.140811in}}{\pgfqpoint{1.044341in}{2.132575in}}%
\pgfpathcurveto{\pgfqpoint{1.044341in}{2.124339in}}{\pgfqpoint{1.047614in}{2.116439in}}{\pgfqpoint{1.053437in}{2.110615in}}%
\pgfpathcurveto{\pgfqpoint{1.059261in}{2.104791in}}{\pgfqpoint{1.067161in}{2.101519in}}{\pgfqpoint{1.075398in}{2.101519in}}%
\pgfpathclose%
\pgfusepath{stroke,fill}%
\end{pgfscope}%
\begin{pgfscope}%
\pgfpathrectangle{\pgfqpoint{0.100000in}{0.220728in}}{\pgfqpoint{3.696000in}{3.696000in}}%
\pgfusepath{clip}%
\pgfsetbuttcap%
\pgfsetroundjoin%
\definecolor{currentfill}{rgb}{0.121569,0.466667,0.705882}%
\pgfsetfillcolor{currentfill}%
\pgfsetfillopacity{0.760074}%
\pgfsetlinewidth{1.003750pt}%
\definecolor{currentstroke}{rgb}{0.121569,0.466667,0.705882}%
\pgfsetstrokecolor{currentstroke}%
\pgfsetstrokeopacity{0.760074}%
\pgfsetdash{}{0pt}%
\pgfpathmoveto{\pgfqpoint{3.123285in}{2.563669in}}%
\pgfpathcurveto{\pgfqpoint{3.131522in}{2.563669in}}{\pgfqpoint{3.139422in}{2.566942in}}{\pgfqpoint{3.145246in}{2.572766in}}%
\pgfpathcurveto{\pgfqpoint{3.151070in}{2.578590in}}{\pgfqpoint{3.154342in}{2.586490in}}{\pgfqpoint{3.154342in}{2.594726in}}%
\pgfpathcurveto{\pgfqpoint{3.154342in}{2.602962in}}{\pgfqpoint{3.151070in}{2.610862in}}{\pgfqpoint{3.145246in}{2.616686in}}%
\pgfpathcurveto{\pgfqpoint{3.139422in}{2.622510in}}{\pgfqpoint{3.131522in}{2.625782in}}{\pgfqpoint{3.123285in}{2.625782in}}%
\pgfpathcurveto{\pgfqpoint{3.115049in}{2.625782in}}{\pgfqpoint{3.107149in}{2.622510in}}{\pgfqpoint{3.101325in}{2.616686in}}%
\pgfpathcurveto{\pgfqpoint{3.095501in}{2.610862in}}{\pgfqpoint{3.092229in}{2.602962in}}{\pgfqpoint{3.092229in}{2.594726in}}%
\pgfpathcurveto{\pgfqpoint{3.092229in}{2.586490in}}{\pgfqpoint{3.095501in}{2.578590in}}{\pgfqpoint{3.101325in}{2.572766in}}%
\pgfpathcurveto{\pgfqpoint{3.107149in}{2.566942in}}{\pgfqpoint{3.115049in}{2.563669in}}{\pgfqpoint{3.123285in}{2.563669in}}%
\pgfpathclose%
\pgfusepath{stroke,fill}%
\end{pgfscope}%
\begin{pgfscope}%
\pgfpathrectangle{\pgfqpoint{0.100000in}{0.220728in}}{\pgfqpoint{3.696000in}{3.696000in}}%
\pgfusepath{clip}%
\pgfsetbuttcap%
\pgfsetroundjoin%
\definecolor{currentfill}{rgb}{0.121569,0.466667,0.705882}%
\pgfsetfillcolor{currentfill}%
\pgfsetfillopacity{0.760529}%
\pgfsetlinewidth{1.003750pt}%
\definecolor{currentstroke}{rgb}{0.121569,0.466667,0.705882}%
\pgfsetstrokecolor{currentstroke}%
\pgfsetstrokeopacity{0.760529}%
\pgfsetdash{}{0pt}%
\pgfpathmoveto{\pgfqpoint{1.087690in}{2.094646in}}%
\pgfpathcurveto{\pgfqpoint{1.095927in}{2.094646in}}{\pgfqpoint{1.103827in}{2.097918in}}{\pgfqpoint{1.109651in}{2.103742in}}%
\pgfpathcurveto{\pgfqpoint{1.115475in}{2.109566in}}{\pgfqpoint{1.118747in}{2.117466in}}{\pgfqpoint{1.118747in}{2.125702in}}%
\pgfpathcurveto{\pgfqpoint{1.118747in}{2.133938in}}{\pgfqpoint{1.115475in}{2.141838in}}{\pgfqpoint{1.109651in}{2.147662in}}%
\pgfpathcurveto{\pgfqpoint{1.103827in}{2.153486in}}{\pgfqpoint{1.095927in}{2.156759in}}{\pgfqpoint{1.087690in}{2.156759in}}%
\pgfpathcurveto{\pgfqpoint{1.079454in}{2.156759in}}{\pgfqpoint{1.071554in}{2.153486in}}{\pgfqpoint{1.065730in}{2.147662in}}%
\pgfpathcurveto{\pgfqpoint{1.059906in}{2.141838in}}{\pgfqpoint{1.056634in}{2.133938in}}{\pgfqpoint{1.056634in}{2.125702in}}%
\pgfpathcurveto{\pgfqpoint{1.056634in}{2.117466in}}{\pgfqpoint{1.059906in}{2.109566in}}{\pgfqpoint{1.065730in}{2.103742in}}%
\pgfpathcurveto{\pgfqpoint{1.071554in}{2.097918in}}{\pgfqpoint{1.079454in}{2.094646in}}{\pgfqpoint{1.087690in}{2.094646in}}%
\pgfpathclose%
\pgfusepath{stroke,fill}%
\end{pgfscope}%
\begin{pgfscope}%
\pgfpathrectangle{\pgfqpoint{0.100000in}{0.220728in}}{\pgfqpoint{3.696000in}{3.696000in}}%
\pgfusepath{clip}%
\pgfsetbuttcap%
\pgfsetroundjoin%
\definecolor{currentfill}{rgb}{0.121569,0.466667,0.705882}%
\pgfsetfillcolor{currentfill}%
\pgfsetfillopacity{0.762062}%
\pgfsetlinewidth{1.003750pt}%
\definecolor{currentstroke}{rgb}{0.121569,0.466667,0.705882}%
\pgfsetstrokecolor{currentstroke}%
\pgfsetstrokeopacity{0.762062}%
\pgfsetdash{}{0pt}%
\pgfpathmoveto{\pgfqpoint{1.094792in}{2.090292in}}%
\pgfpathcurveto{\pgfqpoint{1.103028in}{2.090292in}}{\pgfqpoint{1.110928in}{2.093564in}}{\pgfqpoint{1.116752in}{2.099388in}}%
\pgfpathcurveto{\pgfqpoint{1.122576in}{2.105212in}}{\pgfqpoint{1.125848in}{2.113112in}}{\pgfqpoint{1.125848in}{2.121348in}}%
\pgfpathcurveto{\pgfqpoint{1.125848in}{2.129584in}}{\pgfqpoint{1.122576in}{2.137485in}}{\pgfqpoint{1.116752in}{2.143308in}}%
\pgfpathcurveto{\pgfqpoint{1.110928in}{2.149132in}}{\pgfqpoint{1.103028in}{2.152405in}}{\pgfqpoint{1.094792in}{2.152405in}}%
\pgfpathcurveto{\pgfqpoint{1.086555in}{2.152405in}}{\pgfqpoint{1.078655in}{2.149132in}}{\pgfqpoint{1.072831in}{2.143308in}}%
\pgfpathcurveto{\pgfqpoint{1.067007in}{2.137485in}}{\pgfqpoint{1.063735in}{2.129584in}}{\pgfqpoint{1.063735in}{2.121348in}}%
\pgfpathcurveto{\pgfqpoint{1.063735in}{2.113112in}}{\pgfqpoint{1.067007in}{2.105212in}}{\pgfqpoint{1.072831in}{2.099388in}}%
\pgfpathcurveto{\pgfqpoint{1.078655in}{2.093564in}}{\pgfqpoint{1.086555in}{2.090292in}}{\pgfqpoint{1.094792in}{2.090292in}}%
\pgfpathclose%
\pgfusepath{stroke,fill}%
\end{pgfscope}%
\begin{pgfscope}%
\pgfpathrectangle{\pgfqpoint{0.100000in}{0.220728in}}{\pgfqpoint{3.696000in}{3.696000in}}%
\pgfusepath{clip}%
\pgfsetbuttcap%
\pgfsetroundjoin%
\definecolor{currentfill}{rgb}{0.121569,0.466667,0.705882}%
\pgfsetfillcolor{currentfill}%
\pgfsetfillopacity{0.762769}%
\pgfsetlinewidth{1.003750pt}%
\definecolor{currentstroke}{rgb}{0.121569,0.466667,0.705882}%
\pgfsetstrokecolor{currentstroke}%
\pgfsetstrokeopacity{0.762769}%
\pgfsetdash{}{0pt}%
\pgfpathmoveto{\pgfqpoint{1.098355in}{2.088691in}}%
\pgfpathcurveto{\pgfqpoint{1.106592in}{2.088691in}}{\pgfqpoint{1.114492in}{2.091964in}}{\pgfqpoint{1.120316in}{2.097788in}}%
\pgfpathcurveto{\pgfqpoint{1.126139in}{2.103612in}}{\pgfqpoint{1.129412in}{2.111512in}}{\pgfqpoint{1.129412in}{2.119748in}}%
\pgfpathcurveto{\pgfqpoint{1.129412in}{2.127984in}}{\pgfqpoint{1.126139in}{2.135884in}}{\pgfqpoint{1.120316in}{2.141708in}}%
\pgfpathcurveto{\pgfqpoint{1.114492in}{2.147532in}}{\pgfqpoint{1.106592in}{2.150804in}}{\pgfqpoint{1.098355in}{2.150804in}}%
\pgfpathcurveto{\pgfqpoint{1.090119in}{2.150804in}}{\pgfqpoint{1.082219in}{2.147532in}}{\pgfqpoint{1.076395in}{2.141708in}}%
\pgfpathcurveto{\pgfqpoint{1.070571in}{2.135884in}}{\pgfqpoint{1.067299in}{2.127984in}}{\pgfqpoint{1.067299in}{2.119748in}}%
\pgfpathcurveto{\pgfqpoint{1.067299in}{2.111512in}}{\pgfqpoint{1.070571in}{2.103612in}}{\pgfqpoint{1.076395in}{2.097788in}}%
\pgfpathcurveto{\pgfqpoint{1.082219in}{2.091964in}}{\pgfqpoint{1.090119in}{2.088691in}}{\pgfqpoint{1.098355in}{2.088691in}}%
\pgfpathclose%
\pgfusepath{stroke,fill}%
\end{pgfscope}%
\begin{pgfscope}%
\pgfpathrectangle{\pgfqpoint{0.100000in}{0.220728in}}{\pgfqpoint{3.696000in}{3.696000in}}%
\pgfusepath{clip}%
\pgfsetbuttcap%
\pgfsetroundjoin%
\definecolor{currentfill}{rgb}{0.121569,0.466667,0.705882}%
\pgfsetfillcolor{currentfill}%
\pgfsetfillopacity{0.763747}%
\pgfsetlinewidth{1.003750pt}%
\definecolor{currentstroke}{rgb}{0.121569,0.466667,0.705882}%
\pgfsetstrokecolor{currentstroke}%
\pgfsetstrokeopacity{0.763747}%
\pgfsetdash{}{0pt}%
\pgfpathmoveto{\pgfqpoint{3.109113in}{2.543924in}}%
\pgfpathcurveto{\pgfqpoint{3.117349in}{2.543924in}}{\pgfqpoint{3.125249in}{2.547197in}}{\pgfqpoint{3.131073in}{2.553020in}}%
\pgfpathcurveto{\pgfqpoint{3.136897in}{2.558844in}}{\pgfqpoint{3.140169in}{2.566744in}}{\pgfqpoint{3.140169in}{2.574981in}}%
\pgfpathcurveto{\pgfqpoint{3.140169in}{2.583217in}}{\pgfqpoint{3.136897in}{2.591117in}}{\pgfqpoint{3.131073in}{2.596941in}}%
\pgfpathcurveto{\pgfqpoint{3.125249in}{2.602765in}}{\pgfqpoint{3.117349in}{2.606037in}}{\pgfqpoint{3.109113in}{2.606037in}}%
\pgfpathcurveto{\pgfqpoint{3.100876in}{2.606037in}}{\pgfqpoint{3.092976in}{2.602765in}}{\pgfqpoint{3.087152in}{2.596941in}}%
\pgfpathcurveto{\pgfqpoint{3.081328in}{2.591117in}}{\pgfqpoint{3.078056in}{2.583217in}}{\pgfqpoint{3.078056in}{2.574981in}}%
\pgfpathcurveto{\pgfqpoint{3.078056in}{2.566744in}}{\pgfqpoint{3.081328in}{2.558844in}}{\pgfqpoint{3.087152in}{2.553020in}}%
\pgfpathcurveto{\pgfqpoint{3.092976in}{2.547197in}}{\pgfqpoint{3.100876in}{2.543924in}}{\pgfqpoint{3.109113in}{2.543924in}}%
\pgfpathclose%
\pgfusepath{stroke,fill}%
\end{pgfscope}%
\begin{pgfscope}%
\pgfpathrectangle{\pgfqpoint{0.100000in}{0.220728in}}{\pgfqpoint{3.696000in}{3.696000in}}%
\pgfusepath{clip}%
\pgfsetbuttcap%
\pgfsetroundjoin%
\definecolor{currentfill}{rgb}{0.121569,0.466667,0.705882}%
\pgfsetfillcolor{currentfill}%
\pgfsetfillopacity{0.764057}%
\pgfsetlinewidth{1.003750pt}%
\definecolor{currentstroke}{rgb}{0.121569,0.466667,0.705882}%
\pgfsetstrokecolor{currentstroke}%
\pgfsetstrokeopacity{0.764057}%
\pgfsetdash{}{0pt}%
\pgfpathmoveto{\pgfqpoint{1.105155in}{2.086932in}}%
\pgfpathcurveto{\pgfqpoint{1.113392in}{2.086932in}}{\pgfqpoint{1.121292in}{2.090205in}}{\pgfqpoint{1.127116in}{2.096029in}}%
\pgfpathcurveto{\pgfqpoint{1.132939in}{2.101852in}}{\pgfqpoint{1.136212in}{2.109753in}}{\pgfqpoint{1.136212in}{2.117989in}}%
\pgfpathcurveto{\pgfqpoint{1.136212in}{2.126225in}}{\pgfqpoint{1.132939in}{2.134125in}}{\pgfqpoint{1.127116in}{2.139949in}}%
\pgfpathcurveto{\pgfqpoint{1.121292in}{2.145773in}}{\pgfqpoint{1.113392in}{2.149045in}}{\pgfqpoint{1.105155in}{2.149045in}}%
\pgfpathcurveto{\pgfqpoint{1.096919in}{2.149045in}}{\pgfqpoint{1.089019in}{2.145773in}}{\pgfqpoint{1.083195in}{2.139949in}}%
\pgfpathcurveto{\pgfqpoint{1.077371in}{2.134125in}}{\pgfqpoint{1.074099in}{2.126225in}}{\pgfqpoint{1.074099in}{2.117989in}}%
\pgfpathcurveto{\pgfqpoint{1.074099in}{2.109753in}}{\pgfqpoint{1.077371in}{2.101852in}}{\pgfqpoint{1.083195in}{2.096029in}}%
\pgfpathcurveto{\pgfqpoint{1.089019in}{2.090205in}}{\pgfqpoint{1.096919in}{2.086932in}}{\pgfqpoint{1.105155in}{2.086932in}}%
\pgfpathclose%
\pgfusepath{stroke,fill}%
\end{pgfscope}%
\begin{pgfscope}%
\pgfpathrectangle{\pgfqpoint{0.100000in}{0.220728in}}{\pgfqpoint{3.696000in}{3.696000in}}%
\pgfusepath{clip}%
\pgfsetbuttcap%
\pgfsetroundjoin%
\definecolor{currentfill}{rgb}{0.121569,0.466667,0.705882}%
\pgfsetfillcolor{currentfill}%
\pgfsetfillopacity{0.766108}%
\pgfsetlinewidth{1.003750pt}%
\definecolor{currentstroke}{rgb}{0.121569,0.466667,0.705882}%
\pgfsetstrokecolor{currentstroke}%
\pgfsetstrokeopacity{0.766108}%
\pgfsetdash{}{0pt}%
\pgfpathmoveto{\pgfqpoint{3.102056in}{2.533676in}}%
\pgfpathcurveto{\pgfqpoint{3.110292in}{2.533676in}}{\pgfqpoint{3.118192in}{2.536948in}}{\pgfqpoint{3.124016in}{2.542772in}}%
\pgfpathcurveto{\pgfqpoint{3.129840in}{2.548596in}}{\pgfqpoint{3.133112in}{2.556496in}}{\pgfqpoint{3.133112in}{2.564732in}}%
\pgfpathcurveto{\pgfqpoint{3.133112in}{2.572969in}}{\pgfqpoint{3.129840in}{2.580869in}}{\pgfqpoint{3.124016in}{2.586693in}}%
\pgfpathcurveto{\pgfqpoint{3.118192in}{2.592517in}}{\pgfqpoint{3.110292in}{2.595789in}}{\pgfqpoint{3.102056in}{2.595789in}}%
\pgfpathcurveto{\pgfqpoint{3.093819in}{2.595789in}}{\pgfqpoint{3.085919in}{2.592517in}}{\pgfqpoint{3.080095in}{2.586693in}}%
\pgfpathcurveto{\pgfqpoint{3.074271in}{2.580869in}}{\pgfqpoint{3.070999in}{2.572969in}}{\pgfqpoint{3.070999in}{2.564732in}}%
\pgfpathcurveto{\pgfqpoint{3.070999in}{2.556496in}}{\pgfqpoint{3.074271in}{2.548596in}}{\pgfqpoint{3.080095in}{2.542772in}}%
\pgfpathcurveto{\pgfqpoint{3.085919in}{2.536948in}}{\pgfqpoint{3.093819in}{2.533676in}}{\pgfqpoint{3.102056in}{2.533676in}}%
\pgfpathclose%
\pgfusepath{stroke,fill}%
\end{pgfscope}%
\begin{pgfscope}%
\pgfpathrectangle{\pgfqpoint{0.100000in}{0.220728in}}{\pgfqpoint{3.696000in}{3.696000in}}%
\pgfusepath{clip}%
\pgfsetbuttcap%
\pgfsetroundjoin%
\definecolor{currentfill}{rgb}{0.121569,0.466667,0.705882}%
\pgfsetfillcolor{currentfill}%
\pgfsetfillopacity{0.766182}%
\pgfsetlinewidth{1.003750pt}%
\definecolor{currentstroke}{rgb}{0.121569,0.466667,0.705882}%
\pgfsetstrokecolor{currentstroke}%
\pgfsetstrokeopacity{0.766182}%
\pgfsetdash{}{0pt}%
\pgfpathmoveto{\pgfqpoint{1.117221in}{2.081598in}}%
\pgfpathcurveto{\pgfqpoint{1.125458in}{2.081598in}}{\pgfqpoint{1.133358in}{2.084871in}}{\pgfqpoint{1.139182in}{2.090694in}}%
\pgfpathcurveto{\pgfqpoint{1.145006in}{2.096518in}}{\pgfqpoint{1.148278in}{2.104418in}}{\pgfqpoint{1.148278in}{2.112655in}}%
\pgfpathcurveto{\pgfqpoint{1.148278in}{2.120891in}}{\pgfqpoint{1.145006in}{2.128791in}}{\pgfqpoint{1.139182in}{2.134615in}}%
\pgfpathcurveto{\pgfqpoint{1.133358in}{2.140439in}}{\pgfqpoint{1.125458in}{2.143711in}}{\pgfqpoint{1.117221in}{2.143711in}}%
\pgfpathcurveto{\pgfqpoint{1.108985in}{2.143711in}}{\pgfqpoint{1.101085in}{2.140439in}}{\pgfqpoint{1.095261in}{2.134615in}}%
\pgfpathcurveto{\pgfqpoint{1.089437in}{2.128791in}}{\pgfqpoint{1.086165in}{2.120891in}}{\pgfqpoint{1.086165in}{2.112655in}}%
\pgfpathcurveto{\pgfqpoint{1.086165in}{2.104418in}}{\pgfqpoint{1.089437in}{2.096518in}}{\pgfqpoint{1.095261in}{2.090694in}}%
\pgfpathcurveto{\pgfqpoint{1.101085in}{2.084871in}}{\pgfqpoint{1.108985in}{2.081598in}}{\pgfqpoint{1.117221in}{2.081598in}}%
\pgfpathclose%
\pgfusepath{stroke,fill}%
\end{pgfscope}%
\begin{pgfscope}%
\pgfpathrectangle{\pgfqpoint{0.100000in}{0.220728in}}{\pgfqpoint{3.696000in}{3.696000in}}%
\pgfusepath{clip}%
\pgfsetbuttcap%
\pgfsetroundjoin%
\definecolor{currentfill}{rgb}{0.121569,0.466667,0.705882}%
\pgfsetfillcolor{currentfill}%
\pgfsetfillopacity{0.767253}%
\pgfsetlinewidth{1.003750pt}%
\definecolor{currentstroke}{rgb}{0.121569,0.466667,0.705882}%
\pgfsetstrokecolor{currentstroke}%
\pgfsetstrokeopacity{0.767253}%
\pgfsetdash{}{0pt}%
\pgfpathmoveto{\pgfqpoint{3.097962in}{2.527545in}}%
\pgfpathcurveto{\pgfqpoint{3.106198in}{2.527545in}}{\pgfqpoint{3.114098in}{2.530817in}}{\pgfqpoint{3.119922in}{2.536641in}}%
\pgfpathcurveto{\pgfqpoint{3.125746in}{2.542465in}}{\pgfqpoint{3.129019in}{2.550365in}}{\pgfqpoint{3.129019in}{2.558601in}}%
\pgfpathcurveto{\pgfqpoint{3.129019in}{2.566837in}}{\pgfqpoint{3.125746in}{2.574737in}}{\pgfqpoint{3.119922in}{2.580561in}}%
\pgfpathcurveto{\pgfqpoint{3.114098in}{2.586385in}}{\pgfqpoint{3.106198in}{2.589658in}}{\pgfqpoint{3.097962in}{2.589658in}}%
\pgfpathcurveto{\pgfqpoint{3.089726in}{2.589658in}}{\pgfqpoint{3.081826in}{2.586385in}}{\pgfqpoint{3.076002in}{2.580561in}}%
\pgfpathcurveto{\pgfqpoint{3.070178in}{2.574737in}}{\pgfqpoint{3.066906in}{2.566837in}}{\pgfqpoint{3.066906in}{2.558601in}}%
\pgfpathcurveto{\pgfqpoint{3.066906in}{2.550365in}}{\pgfqpoint{3.070178in}{2.542465in}}{\pgfqpoint{3.076002in}{2.536641in}}%
\pgfpathcurveto{\pgfqpoint{3.081826in}{2.530817in}}{\pgfqpoint{3.089726in}{2.527545in}}{\pgfqpoint{3.097962in}{2.527545in}}%
\pgfpathclose%
\pgfusepath{stroke,fill}%
\end{pgfscope}%
\begin{pgfscope}%
\pgfpathrectangle{\pgfqpoint{0.100000in}{0.220728in}}{\pgfqpoint{3.696000in}{3.696000in}}%
\pgfusepath{clip}%
\pgfsetbuttcap%
\pgfsetroundjoin%
\definecolor{currentfill}{rgb}{0.121569,0.466667,0.705882}%
\pgfsetfillcolor{currentfill}%
\pgfsetfillopacity{0.767912}%
\pgfsetlinewidth{1.003750pt}%
\definecolor{currentstroke}{rgb}{0.121569,0.466667,0.705882}%
\pgfsetstrokecolor{currentstroke}%
\pgfsetstrokeopacity{0.767912}%
\pgfsetdash{}{0pt}%
\pgfpathmoveto{\pgfqpoint{3.095654in}{2.524409in}}%
\pgfpathcurveto{\pgfqpoint{3.103890in}{2.524409in}}{\pgfqpoint{3.111790in}{2.527681in}}{\pgfqpoint{3.117614in}{2.533505in}}%
\pgfpathcurveto{\pgfqpoint{3.123438in}{2.539329in}}{\pgfqpoint{3.126710in}{2.547229in}}{\pgfqpoint{3.126710in}{2.555465in}}%
\pgfpathcurveto{\pgfqpoint{3.126710in}{2.563701in}}{\pgfqpoint{3.123438in}{2.571601in}}{\pgfqpoint{3.117614in}{2.577425in}}%
\pgfpathcurveto{\pgfqpoint{3.111790in}{2.583249in}}{\pgfqpoint{3.103890in}{2.586522in}}{\pgfqpoint{3.095654in}{2.586522in}}%
\pgfpathcurveto{\pgfqpoint{3.087417in}{2.586522in}}{\pgfqpoint{3.079517in}{2.583249in}}{\pgfqpoint{3.073693in}{2.577425in}}%
\pgfpathcurveto{\pgfqpoint{3.067869in}{2.571601in}}{\pgfqpoint{3.064597in}{2.563701in}}{\pgfqpoint{3.064597in}{2.555465in}}%
\pgfpathcurveto{\pgfqpoint{3.064597in}{2.547229in}}{\pgfqpoint{3.067869in}{2.539329in}}{\pgfqpoint{3.073693in}{2.533505in}}%
\pgfpathcurveto{\pgfqpoint{3.079517in}{2.527681in}}{\pgfqpoint{3.087417in}{2.524409in}}{\pgfqpoint{3.095654in}{2.524409in}}%
\pgfpathclose%
\pgfusepath{stroke,fill}%
\end{pgfscope}%
\begin{pgfscope}%
\pgfpathrectangle{\pgfqpoint{0.100000in}{0.220728in}}{\pgfqpoint{3.696000in}{3.696000in}}%
\pgfusepath{clip}%
\pgfsetbuttcap%
\pgfsetroundjoin%
\definecolor{currentfill}{rgb}{0.121569,0.466667,0.705882}%
\pgfsetfillcolor{currentfill}%
\pgfsetfillopacity{0.768284}%
\pgfsetlinewidth{1.003750pt}%
\definecolor{currentstroke}{rgb}{0.121569,0.466667,0.705882}%
\pgfsetstrokecolor{currentstroke}%
\pgfsetstrokeopacity{0.768284}%
\pgfsetdash{}{0pt}%
\pgfpathmoveto{\pgfqpoint{3.094457in}{2.522621in}}%
\pgfpathcurveto{\pgfqpoint{3.102693in}{2.522621in}}{\pgfqpoint{3.110593in}{2.525893in}}{\pgfqpoint{3.116417in}{2.531717in}}%
\pgfpathcurveto{\pgfqpoint{3.122241in}{2.537541in}}{\pgfqpoint{3.125513in}{2.545441in}}{\pgfqpoint{3.125513in}{2.553677in}}%
\pgfpathcurveto{\pgfqpoint{3.125513in}{2.561914in}}{\pgfqpoint{3.122241in}{2.569814in}}{\pgfqpoint{3.116417in}{2.575637in}}%
\pgfpathcurveto{\pgfqpoint{3.110593in}{2.581461in}}{\pgfqpoint{3.102693in}{2.584734in}}{\pgfqpoint{3.094457in}{2.584734in}}%
\pgfpathcurveto{\pgfqpoint{3.086220in}{2.584734in}}{\pgfqpoint{3.078320in}{2.581461in}}{\pgfqpoint{3.072497in}{2.575637in}}%
\pgfpathcurveto{\pgfqpoint{3.066673in}{2.569814in}}{\pgfqpoint{3.063400in}{2.561914in}}{\pgfqpoint{3.063400in}{2.553677in}}%
\pgfpathcurveto{\pgfqpoint{3.063400in}{2.545441in}}{\pgfqpoint{3.066673in}{2.537541in}}{\pgfqpoint{3.072497in}{2.531717in}}%
\pgfpathcurveto{\pgfqpoint{3.078320in}{2.525893in}}{\pgfqpoint{3.086220in}{2.522621in}}{\pgfqpoint{3.094457in}{2.522621in}}%
\pgfpathclose%
\pgfusepath{stroke,fill}%
\end{pgfscope}%
\begin{pgfscope}%
\pgfpathrectangle{\pgfqpoint{0.100000in}{0.220728in}}{\pgfqpoint{3.696000in}{3.696000in}}%
\pgfusepath{clip}%
\pgfsetbuttcap%
\pgfsetroundjoin%
\definecolor{currentfill}{rgb}{0.121569,0.466667,0.705882}%
\pgfsetfillcolor{currentfill}%
\pgfsetfillopacity{0.768462}%
\pgfsetlinewidth{1.003750pt}%
\definecolor{currentstroke}{rgb}{0.121569,0.466667,0.705882}%
\pgfsetstrokecolor{currentstroke}%
\pgfsetstrokeopacity{0.768462}%
\pgfsetdash{}{0pt}%
\pgfpathmoveto{\pgfqpoint{3.093676in}{2.521706in}}%
\pgfpathcurveto{\pgfqpoint{3.101912in}{2.521706in}}{\pgfqpoint{3.109812in}{2.524978in}}{\pgfqpoint{3.115636in}{2.530802in}}%
\pgfpathcurveto{\pgfqpoint{3.121460in}{2.536626in}}{\pgfqpoint{3.124732in}{2.544526in}}{\pgfqpoint{3.124732in}{2.552762in}}%
\pgfpathcurveto{\pgfqpoint{3.124732in}{2.560999in}}{\pgfqpoint{3.121460in}{2.568899in}}{\pgfqpoint{3.115636in}{2.574723in}}%
\pgfpathcurveto{\pgfqpoint{3.109812in}{2.580546in}}{\pgfqpoint{3.101912in}{2.583819in}}{\pgfqpoint{3.093676in}{2.583819in}}%
\pgfpathcurveto{\pgfqpoint{3.085440in}{2.583819in}}{\pgfqpoint{3.077539in}{2.580546in}}{\pgfqpoint{3.071716in}{2.574723in}}%
\pgfpathcurveto{\pgfqpoint{3.065892in}{2.568899in}}{\pgfqpoint{3.062619in}{2.560999in}}{\pgfqpoint{3.062619in}{2.552762in}}%
\pgfpathcurveto{\pgfqpoint{3.062619in}{2.544526in}}{\pgfqpoint{3.065892in}{2.536626in}}{\pgfqpoint{3.071716in}{2.530802in}}%
\pgfpathcurveto{\pgfqpoint{3.077539in}{2.524978in}}{\pgfqpoint{3.085440in}{2.521706in}}{\pgfqpoint{3.093676in}{2.521706in}}%
\pgfpathclose%
\pgfusepath{stroke,fill}%
\end{pgfscope}%
\begin{pgfscope}%
\pgfpathrectangle{\pgfqpoint{0.100000in}{0.220728in}}{\pgfqpoint{3.696000in}{3.696000in}}%
\pgfusepath{clip}%
\pgfsetbuttcap%
\pgfsetroundjoin%
\definecolor{currentfill}{rgb}{0.121569,0.466667,0.705882}%
\pgfsetfillcolor{currentfill}%
\pgfsetfillopacity{0.768565}%
\pgfsetlinewidth{1.003750pt}%
\definecolor{currentstroke}{rgb}{0.121569,0.466667,0.705882}%
\pgfsetstrokecolor{currentstroke}%
\pgfsetstrokeopacity{0.768565}%
\pgfsetdash{}{0pt}%
\pgfpathmoveto{\pgfqpoint{3.093293in}{2.521147in}}%
\pgfpathcurveto{\pgfqpoint{3.101529in}{2.521147in}}{\pgfqpoint{3.109429in}{2.524419in}}{\pgfqpoint{3.115253in}{2.530243in}}%
\pgfpathcurveto{\pgfqpoint{3.121077in}{2.536067in}}{\pgfqpoint{3.124350in}{2.543967in}}{\pgfqpoint{3.124350in}{2.552203in}}%
\pgfpathcurveto{\pgfqpoint{3.124350in}{2.560440in}}{\pgfqpoint{3.121077in}{2.568340in}}{\pgfqpoint{3.115253in}{2.574164in}}%
\pgfpathcurveto{\pgfqpoint{3.109429in}{2.579988in}}{\pgfqpoint{3.101529in}{2.583260in}}{\pgfqpoint{3.093293in}{2.583260in}}%
\pgfpathcurveto{\pgfqpoint{3.085057in}{2.583260in}}{\pgfqpoint{3.077157in}{2.579988in}}{\pgfqpoint{3.071333in}{2.574164in}}%
\pgfpathcurveto{\pgfqpoint{3.065509in}{2.568340in}}{\pgfqpoint{3.062237in}{2.560440in}}{\pgfqpoint{3.062237in}{2.552203in}}%
\pgfpathcurveto{\pgfqpoint{3.062237in}{2.543967in}}{\pgfqpoint{3.065509in}{2.536067in}}{\pgfqpoint{3.071333in}{2.530243in}}%
\pgfpathcurveto{\pgfqpoint{3.077157in}{2.524419in}}{\pgfqpoint{3.085057in}{2.521147in}}{\pgfqpoint{3.093293in}{2.521147in}}%
\pgfpathclose%
\pgfusepath{stroke,fill}%
\end{pgfscope}%
\begin{pgfscope}%
\pgfpathrectangle{\pgfqpoint{0.100000in}{0.220728in}}{\pgfqpoint{3.696000in}{3.696000in}}%
\pgfusepath{clip}%
\pgfsetbuttcap%
\pgfsetroundjoin%
\definecolor{currentfill}{rgb}{0.121569,0.466667,0.705882}%
\pgfsetfillcolor{currentfill}%
\pgfsetfillopacity{0.768626}%
\pgfsetlinewidth{1.003750pt}%
\definecolor{currentstroke}{rgb}{0.121569,0.466667,0.705882}%
\pgfsetstrokecolor{currentstroke}%
\pgfsetstrokeopacity{0.768626}%
\pgfsetdash{}{0pt}%
\pgfpathmoveto{\pgfqpoint{3.093065in}{2.520890in}}%
\pgfpathcurveto{\pgfqpoint{3.101302in}{2.520890in}}{\pgfqpoint{3.109202in}{2.524162in}}{\pgfqpoint{3.115026in}{2.529986in}}%
\pgfpathcurveto{\pgfqpoint{3.120850in}{2.535810in}}{\pgfqpoint{3.124122in}{2.543710in}}{\pgfqpoint{3.124122in}{2.551946in}}%
\pgfpathcurveto{\pgfqpoint{3.124122in}{2.560183in}}{\pgfqpoint{3.120850in}{2.568083in}}{\pgfqpoint{3.115026in}{2.573907in}}%
\pgfpathcurveto{\pgfqpoint{3.109202in}{2.579731in}}{\pgfqpoint{3.101302in}{2.583003in}}{\pgfqpoint{3.093065in}{2.583003in}}%
\pgfpathcurveto{\pgfqpoint{3.084829in}{2.583003in}}{\pgfqpoint{3.076929in}{2.579731in}}{\pgfqpoint{3.071105in}{2.573907in}}%
\pgfpathcurveto{\pgfqpoint{3.065281in}{2.568083in}}{\pgfqpoint{3.062009in}{2.560183in}}{\pgfqpoint{3.062009in}{2.551946in}}%
\pgfpathcurveto{\pgfqpoint{3.062009in}{2.543710in}}{\pgfqpoint{3.065281in}{2.535810in}}{\pgfqpoint{3.071105in}{2.529986in}}%
\pgfpathcurveto{\pgfqpoint{3.076929in}{2.524162in}}{\pgfqpoint{3.084829in}{2.520890in}}{\pgfqpoint{3.093065in}{2.520890in}}%
\pgfpathclose%
\pgfusepath{stroke,fill}%
\end{pgfscope}%
\begin{pgfscope}%
\pgfpathrectangle{\pgfqpoint{0.100000in}{0.220728in}}{\pgfqpoint{3.696000in}{3.696000in}}%
\pgfusepath{clip}%
\pgfsetbuttcap%
\pgfsetroundjoin%
\definecolor{currentfill}{rgb}{0.121569,0.466667,0.705882}%
\pgfsetfillcolor{currentfill}%
\pgfsetfillopacity{0.769318}%
\pgfsetlinewidth{1.003750pt}%
\definecolor{currentstroke}{rgb}{0.121569,0.466667,0.705882}%
\pgfsetstrokecolor{currentstroke}%
\pgfsetstrokeopacity{0.769318}%
\pgfsetdash{}{0pt}%
\pgfpathmoveto{\pgfqpoint{3.090934in}{2.517199in}}%
\pgfpathcurveto{\pgfqpoint{3.099170in}{2.517199in}}{\pgfqpoint{3.107070in}{2.520471in}}{\pgfqpoint{3.112894in}{2.526295in}}%
\pgfpathcurveto{\pgfqpoint{3.118718in}{2.532119in}}{\pgfqpoint{3.121990in}{2.540019in}}{\pgfqpoint{3.121990in}{2.548256in}}%
\pgfpathcurveto{\pgfqpoint{3.121990in}{2.556492in}}{\pgfqpoint{3.118718in}{2.564392in}}{\pgfqpoint{3.112894in}{2.570216in}}%
\pgfpathcurveto{\pgfqpoint{3.107070in}{2.576040in}}{\pgfqpoint{3.099170in}{2.579312in}}{\pgfqpoint{3.090934in}{2.579312in}}%
\pgfpathcurveto{\pgfqpoint{3.082697in}{2.579312in}}{\pgfqpoint{3.074797in}{2.576040in}}{\pgfqpoint{3.068974in}{2.570216in}}%
\pgfpathcurveto{\pgfqpoint{3.063150in}{2.564392in}}{\pgfqpoint{3.059877in}{2.556492in}}{\pgfqpoint{3.059877in}{2.548256in}}%
\pgfpathcurveto{\pgfqpoint{3.059877in}{2.540019in}}{\pgfqpoint{3.063150in}{2.532119in}}{\pgfqpoint{3.068974in}{2.526295in}}%
\pgfpathcurveto{\pgfqpoint{3.074797in}{2.520471in}}{\pgfqpoint{3.082697in}{2.517199in}}{\pgfqpoint{3.090934in}{2.517199in}}%
\pgfpathclose%
\pgfusepath{stroke,fill}%
\end{pgfscope}%
\begin{pgfscope}%
\pgfpathrectangle{\pgfqpoint{0.100000in}{0.220728in}}{\pgfqpoint{3.696000in}{3.696000in}}%
\pgfusepath{clip}%
\pgfsetbuttcap%
\pgfsetroundjoin%
\definecolor{currentfill}{rgb}{0.121569,0.466667,0.705882}%
\pgfsetfillcolor{currentfill}%
\pgfsetfillopacity{0.770074}%
\pgfsetlinewidth{1.003750pt}%
\definecolor{currentstroke}{rgb}{0.121569,0.466667,0.705882}%
\pgfsetstrokecolor{currentstroke}%
\pgfsetstrokeopacity{0.770074}%
\pgfsetdash{}{0pt}%
\pgfpathmoveto{\pgfqpoint{1.138150in}{2.068882in}}%
\pgfpathcurveto{\pgfqpoint{1.146386in}{2.068882in}}{\pgfqpoint{1.154286in}{2.072154in}}{\pgfqpoint{1.160110in}{2.077978in}}%
\pgfpathcurveto{\pgfqpoint{1.165934in}{2.083802in}}{\pgfqpoint{1.169206in}{2.091702in}}{\pgfqpoint{1.169206in}{2.099938in}}%
\pgfpathcurveto{\pgfqpoint{1.169206in}{2.108175in}}{\pgfqpoint{1.165934in}{2.116075in}}{\pgfqpoint{1.160110in}{2.121899in}}%
\pgfpathcurveto{\pgfqpoint{1.154286in}{2.127723in}}{\pgfqpoint{1.146386in}{2.130995in}}{\pgfqpoint{1.138150in}{2.130995in}}%
\pgfpathcurveto{\pgfqpoint{1.129914in}{2.130995in}}{\pgfqpoint{1.122014in}{2.127723in}}{\pgfqpoint{1.116190in}{2.121899in}}%
\pgfpathcurveto{\pgfqpoint{1.110366in}{2.116075in}}{\pgfqpoint{1.107093in}{2.108175in}}{\pgfqpoint{1.107093in}{2.099938in}}%
\pgfpathcurveto{\pgfqpoint{1.107093in}{2.091702in}}{\pgfqpoint{1.110366in}{2.083802in}}{\pgfqpoint{1.116190in}{2.077978in}}%
\pgfpathcurveto{\pgfqpoint{1.122014in}{2.072154in}}{\pgfqpoint{1.129914in}{2.068882in}}{\pgfqpoint{1.138150in}{2.068882in}}%
\pgfpathclose%
\pgfusepath{stroke,fill}%
\end{pgfscope}%
\begin{pgfscope}%
\pgfpathrectangle{\pgfqpoint{0.100000in}{0.220728in}}{\pgfqpoint{3.696000in}{3.696000in}}%
\pgfusepath{clip}%
\pgfsetbuttcap%
\pgfsetroundjoin%
\definecolor{currentfill}{rgb}{0.121569,0.466667,0.705882}%
\pgfsetfillcolor{currentfill}%
\pgfsetfillopacity{0.770809}%
\pgfsetlinewidth{1.003750pt}%
\definecolor{currentstroke}{rgb}{0.121569,0.466667,0.705882}%
\pgfsetstrokecolor{currentstroke}%
\pgfsetstrokeopacity{0.770809}%
\pgfsetdash{}{0pt}%
\pgfpathmoveto{\pgfqpoint{3.084046in}{2.508414in}}%
\pgfpathcurveto{\pgfqpoint{3.092282in}{2.508414in}}{\pgfqpoint{3.100182in}{2.511687in}}{\pgfqpoint{3.106006in}{2.517511in}}%
\pgfpathcurveto{\pgfqpoint{3.111830in}{2.523335in}}{\pgfqpoint{3.115102in}{2.531235in}}{\pgfqpoint{3.115102in}{2.539471in}}%
\pgfpathcurveto{\pgfqpoint{3.115102in}{2.547707in}}{\pgfqpoint{3.111830in}{2.555607in}}{\pgfqpoint{3.106006in}{2.561431in}}%
\pgfpathcurveto{\pgfqpoint{3.100182in}{2.567255in}}{\pgfqpoint{3.092282in}{2.570527in}}{\pgfqpoint{3.084046in}{2.570527in}}%
\pgfpathcurveto{\pgfqpoint{3.075809in}{2.570527in}}{\pgfqpoint{3.067909in}{2.567255in}}{\pgfqpoint{3.062085in}{2.561431in}}%
\pgfpathcurveto{\pgfqpoint{3.056262in}{2.555607in}}{\pgfqpoint{3.052989in}{2.547707in}}{\pgfqpoint{3.052989in}{2.539471in}}%
\pgfpathcurveto{\pgfqpoint{3.052989in}{2.531235in}}{\pgfqpoint{3.056262in}{2.523335in}}{\pgfqpoint{3.062085in}{2.517511in}}%
\pgfpathcurveto{\pgfqpoint{3.067909in}{2.511687in}}{\pgfqpoint{3.075809in}{2.508414in}}{\pgfqpoint{3.084046in}{2.508414in}}%
\pgfpathclose%
\pgfusepath{stroke,fill}%
\end{pgfscope}%
\begin{pgfscope}%
\pgfpathrectangle{\pgfqpoint{0.100000in}{0.220728in}}{\pgfqpoint{3.696000in}{3.696000in}}%
\pgfusepath{clip}%
\pgfsetbuttcap%
\pgfsetroundjoin%
\definecolor{currentfill}{rgb}{0.121569,0.466667,0.705882}%
\pgfsetfillcolor{currentfill}%
\pgfsetfillopacity{0.771780}%
\pgfsetlinewidth{1.003750pt}%
\definecolor{currentstroke}{rgb}{0.121569,0.466667,0.705882}%
\pgfsetstrokecolor{currentstroke}%
\pgfsetstrokeopacity{0.771780}%
\pgfsetdash{}{0pt}%
\pgfpathmoveto{\pgfqpoint{3.080965in}{2.503192in}}%
\pgfpathcurveto{\pgfqpoint{3.089201in}{2.503192in}}{\pgfqpoint{3.097101in}{2.506464in}}{\pgfqpoint{3.102925in}{2.512288in}}%
\pgfpathcurveto{\pgfqpoint{3.108749in}{2.518112in}}{\pgfqpoint{3.112021in}{2.526012in}}{\pgfqpoint{3.112021in}{2.534248in}}%
\pgfpathcurveto{\pgfqpoint{3.112021in}{2.542485in}}{\pgfqpoint{3.108749in}{2.550385in}}{\pgfqpoint{3.102925in}{2.556209in}}%
\pgfpathcurveto{\pgfqpoint{3.097101in}{2.562033in}}{\pgfqpoint{3.089201in}{2.565305in}}{\pgfqpoint{3.080965in}{2.565305in}}%
\pgfpathcurveto{\pgfqpoint{3.072729in}{2.565305in}}{\pgfqpoint{3.064829in}{2.562033in}}{\pgfqpoint{3.059005in}{2.556209in}}%
\pgfpathcurveto{\pgfqpoint{3.053181in}{2.550385in}}{\pgfqpoint{3.049908in}{2.542485in}}{\pgfqpoint{3.049908in}{2.534248in}}%
\pgfpathcurveto{\pgfqpoint{3.049908in}{2.526012in}}{\pgfqpoint{3.053181in}{2.518112in}}{\pgfqpoint{3.059005in}{2.512288in}}%
\pgfpathcurveto{\pgfqpoint{3.064829in}{2.506464in}}{\pgfqpoint{3.072729in}{2.503192in}}{\pgfqpoint{3.080965in}{2.503192in}}%
\pgfpathclose%
\pgfusepath{stroke,fill}%
\end{pgfscope}%
\begin{pgfscope}%
\pgfpathrectangle{\pgfqpoint{0.100000in}{0.220728in}}{\pgfqpoint{3.696000in}{3.696000in}}%
\pgfusepath{clip}%
\pgfsetbuttcap%
\pgfsetroundjoin%
\definecolor{currentfill}{rgb}{0.121569,0.466667,0.705882}%
\pgfsetfillcolor{currentfill}%
\pgfsetfillopacity{0.773420}%
\pgfsetlinewidth{1.003750pt}%
\definecolor{currentstroke}{rgb}{0.121569,0.466667,0.705882}%
\pgfsetstrokecolor{currentstroke}%
\pgfsetstrokeopacity{0.773420}%
\pgfsetdash{}{0pt}%
\pgfpathmoveto{\pgfqpoint{3.074175in}{2.494498in}}%
\pgfpathcurveto{\pgfqpoint{3.082412in}{2.494498in}}{\pgfqpoint{3.090312in}{2.497771in}}{\pgfqpoint{3.096136in}{2.503595in}}%
\pgfpathcurveto{\pgfqpoint{3.101960in}{2.509418in}}{\pgfqpoint{3.105232in}{2.517318in}}{\pgfqpoint{3.105232in}{2.525555in}}%
\pgfpathcurveto{\pgfqpoint{3.105232in}{2.533791in}}{\pgfqpoint{3.101960in}{2.541691in}}{\pgfqpoint{3.096136in}{2.547515in}}%
\pgfpathcurveto{\pgfqpoint{3.090312in}{2.553339in}}{\pgfqpoint{3.082412in}{2.556611in}}{\pgfqpoint{3.074175in}{2.556611in}}%
\pgfpathcurveto{\pgfqpoint{3.065939in}{2.556611in}}{\pgfqpoint{3.058039in}{2.553339in}}{\pgfqpoint{3.052215in}{2.547515in}}%
\pgfpathcurveto{\pgfqpoint{3.046391in}{2.541691in}}{\pgfqpoint{3.043119in}{2.533791in}}{\pgfqpoint{3.043119in}{2.525555in}}%
\pgfpathcurveto{\pgfqpoint{3.043119in}{2.517318in}}{\pgfqpoint{3.046391in}{2.509418in}}{\pgfqpoint{3.052215in}{2.503595in}}%
\pgfpathcurveto{\pgfqpoint{3.058039in}{2.497771in}}{\pgfqpoint{3.065939in}{2.494498in}}{\pgfqpoint{3.074175in}{2.494498in}}%
\pgfpathclose%
\pgfusepath{stroke,fill}%
\end{pgfscope}%
\begin{pgfscope}%
\pgfpathrectangle{\pgfqpoint{0.100000in}{0.220728in}}{\pgfqpoint{3.696000in}{3.696000in}}%
\pgfusepath{clip}%
\pgfsetbuttcap%
\pgfsetroundjoin%
\definecolor{currentfill}{rgb}{0.121569,0.466667,0.705882}%
\pgfsetfillcolor{currentfill}%
\pgfsetfillopacity{0.773648}%
\pgfsetlinewidth{1.003750pt}%
\definecolor{currentstroke}{rgb}{0.121569,0.466667,0.705882}%
\pgfsetstrokecolor{currentstroke}%
\pgfsetstrokeopacity{0.773648}%
\pgfsetdash{}{0pt}%
\pgfpathmoveto{\pgfqpoint{1.156960in}{2.058441in}}%
\pgfpathcurveto{\pgfqpoint{1.165197in}{2.058441in}}{\pgfqpoint{1.173097in}{2.061714in}}{\pgfqpoint{1.178921in}{2.067538in}}%
\pgfpathcurveto{\pgfqpoint{1.184744in}{2.073362in}}{\pgfqpoint{1.188017in}{2.081262in}}{\pgfqpoint{1.188017in}{2.089498in}}%
\pgfpathcurveto{\pgfqpoint{1.188017in}{2.097734in}}{\pgfqpoint{1.184744in}{2.105634in}}{\pgfqpoint{1.178921in}{2.111458in}}%
\pgfpathcurveto{\pgfqpoint{1.173097in}{2.117282in}}{\pgfqpoint{1.165197in}{2.120554in}}{\pgfqpoint{1.156960in}{2.120554in}}%
\pgfpathcurveto{\pgfqpoint{1.148724in}{2.120554in}}{\pgfqpoint{1.140824in}{2.117282in}}{\pgfqpoint{1.135000in}{2.111458in}}%
\pgfpathcurveto{\pgfqpoint{1.129176in}{2.105634in}}{\pgfqpoint{1.125904in}{2.097734in}}{\pgfqpoint{1.125904in}{2.089498in}}%
\pgfpathcurveto{\pgfqpoint{1.125904in}{2.081262in}}{\pgfqpoint{1.129176in}{2.073362in}}{\pgfqpoint{1.135000in}{2.067538in}}%
\pgfpathcurveto{\pgfqpoint{1.140824in}{2.061714in}}{\pgfqpoint{1.148724in}{2.058441in}}{\pgfqpoint{1.156960in}{2.058441in}}%
\pgfpathclose%
\pgfusepath{stroke,fill}%
\end{pgfscope}%
\begin{pgfscope}%
\pgfpathrectangle{\pgfqpoint{0.100000in}{0.220728in}}{\pgfqpoint{3.696000in}{3.696000in}}%
\pgfusepath{clip}%
\pgfsetbuttcap%
\pgfsetroundjoin%
\definecolor{currentfill}{rgb}{0.121569,0.466667,0.705882}%
\pgfsetfillcolor{currentfill}%
\pgfsetfillopacity{0.775925}%
\pgfsetlinewidth{1.003750pt}%
\definecolor{currentstroke}{rgb}{0.121569,0.466667,0.705882}%
\pgfsetstrokecolor{currentstroke}%
\pgfsetstrokeopacity{0.775925}%
\pgfsetdash{}{0pt}%
\pgfpathmoveto{\pgfqpoint{3.066792in}{2.482130in}}%
\pgfpathcurveto{\pgfqpoint{3.075029in}{2.482130in}}{\pgfqpoint{3.082929in}{2.485403in}}{\pgfqpoint{3.088753in}{2.491227in}}%
\pgfpathcurveto{\pgfqpoint{3.094577in}{2.497051in}}{\pgfqpoint{3.097849in}{2.504951in}}{\pgfqpoint{3.097849in}{2.513187in}}%
\pgfpathcurveto{\pgfqpoint{3.097849in}{2.521423in}}{\pgfqpoint{3.094577in}{2.529323in}}{\pgfqpoint{3.088753in}{2.535147in}}%
\pgfpathcurveto{\pgfqpoint{3.082929in}{2.540971in}}{\pgfqpoint{3.075029in}{2.544243in}}{\pgfqpoint{3.066792in}{2.544243in}}%
\pgfpathcurveto{\pgfqpoint{3.058556in}{2.544243in}}{\pgfqpoint{3.050656in}{2.540971in}}{\pgfqpoint{3.044832in}{2.535147in}}%
\pgfpathcurveto{\pgfqpoint{3.039008in}{2.529323in}}{\pgfqpoint{3.035736in}{2.521423in}}{\pgfqpoint{3.035736in}{2.513187in}}%
\pgfpathcurveto{\pgfqpoint{3.035736in}{2.504951in}}{\pgfqpoint{3.039008in}{2.497051in}}{\pgfqpoint{3.044832in}{2.491227in}}%
\pgfpathcurveto{\pgfqpoint{3.050656in}{2.485403in}}{\pgfqpoint{3.058556in}{2.482130in}}{\pgfqpoint{3.066792in}{2.482130in}}%
\pgfpathclose%
\pgfusepath{stroke,fill}%
\end{pgfscope}%
\begin{pgfscope}%
\pgfpathrectangle{\pgfqpoint{0.100000in}{0.220728in}}{\pgfqpoint{3.696000in}{3.696000in}}%
\pgfusepath{clip}%
\pgfsetbuttcap%
\pgfsetroundjoin%
\definecolor{currentfill}{rgb}{0.121569,0.466667,0.705882}%
\pgfsetfillcolor{currentfill}%
\pgfsetfillopacity{0.776769}%
\pgfsetlinewidth{1.003750pt}%
\definecolor{currentstroke}{rgb}{0.121569,0.466667,0.705882}%
\pgfsetstrokecolor{currentstroke}%
\pgfsetstrokeopacity{0.776769}%
\pgfsetdash{}{0pt}%
\pgfpathmoveto{\pgfqpoint{1.173422in}{2.053411in}}%
\pgfpathcurveto{\pgfqpoint{1.181658in}{2.053411in}}{\pgfqpoint{1.189559in}{2.056683in}}{\pgfqpoint{1.195382in}{2.062507in}}%
\pgfpathcurveto{\pgfqpoint{1.201206in}{2.068331in}}{\pgfqpoint{1.204479in}{2.076231in}}{\pgfqpoint{1.204479in}{2.084468in}}%
\pgfpathcurveto{\pgfqpoint{1.204479in}{2.092704in}}{\pgfqpoint{1.201206in}{2.100604in}}{\pgfqpoint{1.195382in}{2.106428in}}%
\pgfpathcurveto{\pgfqpoint{1.189559in}{2.112252in}}{\pgfqpoint{1.181658in}{2.115524in}}{\pgfqpoint{1.173422in}{2.115524in}}%
\pgfpathcurveto{\pgfqpoint{1.165186in}{2.115524in}}{\pgfqpoint{1.157286in}{2.112252in}}{\pgfqpoint{1.151462in}{2.106428in}}%
\pgfpathcurveto{\pgfqpoint{1.145638in}{2.100604in}}{\pgfqpoint{1.142366in}{2.092704in}}{\pgfqpoint{1.142366in}{2.084468in}}%
\pgfpathcurveto{\pgfqpoint{1.142366in}{2.076231in}}{\pgfqpoint{1.145638in}{2.068331in}}{\pgfqpoint{1.151462in}{2.062507in}}%
\pgfpathcurveto{\pgfqpoint{1.157286in}{2.056683in}}{\pgfqpoint{1.165186in}{2.053411in}}{\pgfqpoint{1.173422in}{2.053411in}}%
\pgfpathclose%
\pgfusepath{stroke,fill}%
\end{pgfscope}%
\begin{pgfscope}%
\pgfpathrectangle{\pgfqpoint{0.100000in}{0.220728in}}{\pgfqpoint{3.696000in}{3.696000in}}%
\pgfusepath{clip}%
\pgfsetbuttcap%
\pgfsetroundjoin%
\definecolor{currentfill}{rgb}{0.121569,0.466667,0.705882}%
\pgfsetfillcolor{currentfill}%
\pgfsetfillopacity{0.778518}%
\pgfsetlinewidth{1.003750pt}%
\definecolor{currentstroke}{rgb}{0.121569,0.466667,0.705882}%
\pgfsetstrokecolor{currentstroke}%
\pgfsetstrokeopacity{0.778518}%
\pgfsetdash{}{0pt}%
\pgfpathmoveto{\pgfqpoint{1.184725in}{2.050359in}}%
\pgfpathcurveto{\pgfqpoint{1.192961in}{2.050359in}}{\pgfqpoint{1.200861in}{2.053632in}}{\pgfqpoint{1.206685in}{2.059455in}}%
\pgfpathcurveto{\pgfqpoint{1.212509in}{2.065279in}}{\pgfqpoint{1.215782in}{2.073179in}}{\pgfqpoint{1.215782in}{2.081416in}}%
\pgfpathcurveto{\pgfqpoint{1.215782in}{2.089652in}}{\pgfqpoint{1.212509in}{2.097552in}}{\pgfqpoint{1.206685in}{2.103376in}}%
\pgfpathcurveto{\pgfqpoint{1.200861in}{2.109200in}}{\pgfqpoint{1.192961in}{2.112472in}}{\pgfqpoint{1.184725in}{2.112472in}}%
\pgfpathcurveto{\pgfqpoint{1.176489in}{2.112472in}}{\pgfqpoint{1.168589in}{2.109200in}}{\pgfqpoint{1.162765in}{2.103376in}}%
\pgfpathcurveto{\pgfqpoint{1.156941in}{2.097552in}}{\pgfqpoint{1.153669in}{2.089652in}}{\pgfqpoint{1.153669in}{2.081416in}}%
\pgfpathcurveto{\pgfqpoint{1.153669in}{2.073179in}}{\pgfqpoint{1.156941in}{2.065279in}}{\pgfqpoint{1.162765in}{2.059455in}}%
\pgfpathcurveto{\pgfqpoint{1.168589in}{2.053632in}}{\pgfqpoint{1.176489in}{2.050359in}}{\pgfqpoint{1.184725in}{2.050359in}}%
\pgfpathclose%
\pgfusepath{stroke,fill}%
\end{pgfscope}%
\begin{pgfscope}%
\pgfpathrectangle{\pgfqpoint{0.100000in}{0.220728in}}{\pgfqpoint{3.696000in}{3.696000in}}%
\pgfusepath{clip}%
\pgfsetbuttcap%
\pgfsetroundjoin%
\definecolor{currentfill}{rgb}{0.121569,0.466667,0.705882}%
\pgfsetfillcolor{currentfill}%
\pgfsetfillopacity{0.779318}%
\pgfsetlinewidth{1.003750pt}%
\definecolor{currentstroke}{rgb}{0.121569,0.466667,0.705882}%
\pgfsetstrokecolor{currentstroke}%
\pgfsetstrokeopacity{0.779318}%
\pgfsetdash{}{0pt}%
\pgfpathmoveto{\pgfqpoint{3.055192in}{2.467006in}}%
\pgfpathcurveto{\pgfqpoint{3.063428in}{2.467006in}}{\pgfqpoint{3.071328in}{2.470278in}}{\pgfqpoint{3.077152in}{2.476102in}}%
\pgfpathcurveto{\pgfqpoint{3.082976in}{2.481926in}}{\pgfqpoint{3.086248in}{2.489826in}}{\pgfqpoint{3.086248in}{2.498062in}}%
\pgfpathcurveto{\pgfqpoint{3.086248in}{2.506298in}}{\pgfqpoint{3.082976in}{2.514198in}}{\pgfqpoint{3.077152in}{2.520022in}}%
\pgfpathcurveto{\pgfqpoint{3.071328in}{2.525846in}}{\pgfqpoint{3.063428in}{2.529119in}}{\pgfqpoint{3.055192in}{2.529119in}}%
\pgfpathcurveto{\pgfqpoint{3.046956in}{2.529119in}}{\pgfqpoint{3.039056in}{2.525846in}}{\pgfqpoint{3.033232in}{2.520022in}}%
\pgfpathcurveto{\pgfqpoint{3.027408in}{2.514198in}}{\pgfqpoint{3.024135in}{2.506298in}}{\pgfqpoint{3.024135in}{2.498062in}}%
\pgfpathcurveto{\pgfqpoint{3.024135in}{2.489826in}}{\pgfqpoint{3.027408in}{2.481926in}}{\pgfqpoint{3.033232in}{2.476102in}}%
\pgfpathcurveto{\pgfqpoint{3.039056in}{2.470278in}}{\pgfqpoint{3.046956in}{2.467006in}}{\pgfqpoint{3.055192in}{2.467006in}}%
\pgfpathclose%
\pgfusepath{stroke,fill}%
\end{pgfscope}%
\begin{pgfscope}%
\pgfpathrectangle{\pgfqpoint{0.100000in}{0.220728in}}{\pgfqpoint{3.696000in}{3.696000in}}%
\pgfusepath{clip}%
\pgfsetbuttcap%
\pgfsetroundjoin%
\definecolor{currentfill}{rgb}{0.121569,0.466667,0.705882}%
\pgfsetfillcolor{currentfill}%
\pgfsetfillopacity{0.779756}%
\pgfsetlinewidth{1.003750pt}%
\definecolor{currentstroke}{rgb}{0.121569,0.466667,0.705882}%
\pgfsetstrokecolor{currentstroke}%
\pgfsetstrokeopacity{0.779756}%
\pgfsetdash{}{0pt}%
\pgfpathmoveto{\pgfqpoint{1.191034in}{2.045586in}}%
\pgfpathcurveto{\pgfqpoint{1.199270in}{2.045586in}}{\pgfqpoint{1.207170in}{2.048858in}}{\pgfqpoint{1.212994in}{2.054682in}}%
\pgfpathcurveto{\pgfqpoint{1.218818in}{2.060506in}}{\pgfqpoint{1.222091in}{2.068406in}}{\pgfqpoint{1.222091in}{2.076643in}}%
\pgfpathcurveto{\pgfqpoint{1.222091in}{2.084879in}}{\pgfqpoint{1.218818in}{2.092779in}}{\pgfqpoint{1.212994in}{2.098603in}}%
\pgfpathcurveto{\pgfqpoint{1.207170in}{2.104427in}}{\pgfqpoint{1.199270in}{2.107699in}}{\pgfqpoint{1.191034in}{2.107699in}}%
\pgfpathcurveto{\pgfqpoint{1.182798in}{2.107699in}}{\pgfqpoint{1.174898in}{2.104427in}}{\pgfqpoint{1.169074in}{2.098603in}}%
\pgfpathcurveto{\pgfqpoint{1.163250in}{2.092779in}}{\pgfqpoint{1.159978in}{2.084879in}}{\pgfqpoint{1.159978in}{2.076643in}}%
\pgfpathcurveto{\pgfqpoint{1.159978in}{2.068406in}}{\pgfqpoint{1.163250in}{2.060506in}}{\pgfqpoint{1.169074in}{2.054682in}}%
\pgfpathcurveto{\pgfqpoint{1.174898in}{2.048858in}}{\pgfqpoint{1.182798in}{2.045586in}}{\pgfqpoint{1.191034in}{2.045586in}}%
\pgfpathclose%
\pgfusepath{stroke,fill}%
\end{pgfscope}%
\begin{pgfscope}%
\pgfpathrectangle{\pgfqpoint{0.100000in}{0.220728in}}{\pgfqpoint{3.696000in}{3.696000in}}%
\pgfusepath{clip}%
\pgfsetbuttcap%
\pgfsetroundjoin%
\definecolor{currentfill}{rgb}{0.121569,0.466667,0.705882}%
\pgfsetfillcolor{currentfill}%
\pgfsetfillopacity{0.782023}%
\pgfsetlinewidth{1.003750pt}%
\definecolor{currentstroke}{rgb}{0.121569,0.466667,0.705882}%
\pgfsetstrokecolor{currentstroke}%
\pgfsetstrokeopacity{0.782023}%
\pgfsetdash{}{0pt}%
\pgfpathmoveto{\pgfqpoint{1.203375in}{2.039439in}}%
\pgfpathcurveto{\pgfqpoint{1.211611in}{2.039439in}}{\pgfqpoint{1.219512in}{2.042712in}}{\pgfqpoint{1.225335in}{2.048536in}}%
\pgfpathcurveto{\pgfqpoint{1.231159in}{2.054359in}}{\pgfqpoint{1.234432in}{2.062260in}}{\pgfqpoint{1.234432in}{2.070496in}}%
\pgfpathcurveto{\pgfqpoint{1.234432in}{2.078732in}}{\pgfqpoint{1.231159in}{2.086632in}}{\pgfqpoint{1.225335in}{2.092456in}}%
\pgfpathcurveto{\pgfqpoint{1.219512in}{2.098280in}}{\pgfqpoint{1.211611in}{2.101552in}}{\pgfqpoint{1.203375in}{2.101552in}}%
\pgfpathcurveto{\pgfqpoint{1.195139in}{2.101552in}}{\pgfqpoint{1.187239in}{2.098280in}}{\pgfqpoint{1.181415in}{2.092456in}}%
\pgfpathcurveto{\pgfqpoint{1.175591in}{2.086632in}}{\pgfqpoint{1.172319in}{2.078732in}}{\pgfqpoint{1.172319in}{2.070496in}}%
\pgfpathcurveto{\pgfqpoint{1.172319in}{2.062260in}}{\pgfqpoint{1.175591in}{2.054359in}}{\pgfqpoint{1.181415in}{2.048536in}}%
\pgfpathcurveto{\pgfqpoint{1.187239in}{2.042712in}}{\pgfqpoint{1.195139in}{2.039439in}}{\pgfqpoint{1.203375in}{2.039439in}}%
\pgfpathclose%
\pgfusepath{stroke,fill}%
\end{pgfscope}%
\begin{pgfscope}%
\pgfpathrectangle{\pgfqpoint{0.100000in}{0.220728in}}{\pgfqpoint{3.696000in}{3.696000in}}%
\pgfusepath{clip}%
\pgfsetbuttcap%
\pgfsetroundjoin%
\definecolor{currentfill}{rgb}{0.121569,0.466667,0.705882}%
\pgfsetfillcolor{currentfill}%
\pgfsetfillopacity{0.783403}%
\pgfsetlinewidth{1.003750pt}%
\definecolor{currentstroke}{rgb}{0.121569,0.466667,0.705882}%
\pgfsetstrokecolor{currentstroke}%
\pgfsetstrokeopacity{0.783403}%
\pgfsetdash{}{0pt}%
\pgfpathmoveto{\pgfqpoint{3.043571in}{2.448822in}}%
\pgfpathcurveto{\pgfqpoint{3.051807in}{2.448822in}}{\pgfqpoint{3.059707in}{2.452094in}}{\pgfqpoint{3.065531in}{2.457918in}}%
\pgfpathcurveto{\pgfqpoint{3.071355in}{2.463742in}}{\pgfqpoint{3.074628in}{2.471642in}}{\pgfqpoint{3.074628in}{2.479878in}}%
\pgfpathcurveto{\pgfqpoint{3.074628in}{2.488115in}}{\pgfqpoint{3.071355in}{2.496015in}}{\pgfqpoint{3.065531in}{2.501839in}}%
\pgfpathcurveto{\pgfqpoint{3.059707in}{2.507663in}}{\pgfqpoint{3.051807in}{2.510935in}}{\pgfqpoint{3.043571in}{2.510935in}}%
\pgfpathcurveto{\pgfqpoint{3.035335in}{2.510935in}}{\pgfqpoint{3.027435in}{2.507663in}}{\pgfqpoint{3.021611in}{2.501839in}}%
\pgfpathcurveto{\pgfqpoint{3.015787in}{2.496015in}}{\pgfqpoint{3.012515in}{2.488115in}}{\pgfqpoint{3.012515in}{2.479878in}}%
\pgfpathcurveto{\pgfqpoint{3.012515in}{2.471642in}}{\pgfqpoint{3.015787in}{2.463742in}}{\pgfqpoint{3.021611in}{2.457918in}}%
\pgfpathcurveto{\pgfqpoint{3.027435in}{2.452094in}}{\pgfqpoint{3.035335in}{2.448822in}}{\pgfqpoint{3.043571in}{2.448822in}}%
\pgfpathclose%
\pgfusepath{stroke,fill}%
\end{pgfscope}%
\begin{pgfscope}%
\pgfpathrectangle{\pgfqpoint{0.100000in}{0.220728in}}{\pgfqpoint{3.696000in}{3.696000in}}%
\pgfusepath{clip}%
\pgfsetbuttcap%
\pgfsetroundjoin%
\definecolor{currentfill}{rgb}{0.121569,0.466667,0.705882}%
\pgfsetfillcolor{currentfill}%
\pgfsetfillopacity{0.785416}%
\pgfsetlinewidth{1.003750pt}%
\definecolor{currentstroke}{rgb}{0.121569,0.466667,0.705882}%
\pgfsetstrokecolor{currentstroke}%
\pgfsetstrokeopacity{0.785416}%
\pgfsetdash{}{0pt}%
\pgfpathmoveto{\pgfqpoint{3.036538in}{2.438562in}}%
\pgfpathcurveto{\pgfqpoint{3.044775in}{2.438562in}}{\pgfqpoint{3.052675in}{2.441835in}}{\pgfqpoint{3.058499in}{2.447659in}}%
\pgfpathcurveto{\pgfqpoint{3.064323in}{2.453482in}}{\pgfqpoint{3.067595in}{2.461383in}}{\pgfqpoint{3.067595in}{2.469619in}}%
\pgfpathcurveto{\pgfqpoint{3.067595in}{2.477855in}}{\pgfqpoint{3.064323in}{2.485755in}}{\pgfqpoint{3.058499in}{2.491579in}}%
\pgfpathcurveto{\pgfqpoint{3.052675in}{2.497403in}}{\pgfqpoint{3.044775in}{2.500675in}}{\pgfqpoint{3.036538in}{2.500675in}}%
\pgfpathcurveto{\pgfqpoint{3.028302in}{2.500675in}}{\pgfqpoint{3.020402in}{2.497403in}}{\pgfqpoint{3.014578in}{2.491579in}}%
\pgfpathcurveto{\pgfqpoint{3.008754in}{2.485755in}}{\pgfqpoint{3.005482in}{2.477855in}}{\pgfqpoint{3.005482in}{2.469619in}}%
\pgfpathcurveto{\pgfqpoint{3.005482in}{2.461383in}}{\pgfqpoint{3.008754in}{2.453482in}}{\pgfqpoint{3.014578in}{2.447659in}}%
\pgfpathcurveto{\pgfqpoint{3.020402in}{2.441835in}}{\pgfqpoint{3.028302in}{2.438562in}}{\pgfqpoint{3.036538in}{2.438562in}}%
\pgfpathclose%
\pgfusepath{stroke,fill}%
\end{pgfscope}%
\begin{pgfscope}%
\pgfpathrectangle{\pgfqpoint{0.100000in}{0.220728in}}{\pgfqpoint{3.696000in}{3.696000in}}%
\pgfusepath{clip}%
\pgfsetbuttcap%
\pgfsetroundjoin%
\definecolor{currentfill}{rgb}{0.121569,0.466667,0.705882}%
\pgfsetfillcolor{currentfill}%
\pgfsetfillopacity{0.785732}%
\pgfsetlinewidth{1.003750pt}%
\definecolor{currentstroke}{rgb}{0.121569,0.466667,0.705882}%
\pgfsetstrokecolor{currentstroke}%
\pgfsetstrokeopacity{0.785732}%
\pgfsetdash{}{0pt}%
\pgfpathmoveto{\pgfqpoint{1.226379in}{2.027813in}}%
\pgfpathcurveto{\pgfqpoint{1.234615in}{2.027813in}}{\pgfqpoint{1.242515in}{2.031085in}}{\pgfqpoint{1.248339in}{2.036909in}}%
\pgfpathcurveto{\pgfqpoint{1.254163in}{2.042733in}}{\pgfqpoint{1.257435in}{2.050633in}}{\pgfqpoint{1.257435in}{2.058870in}}%
\pgfpathcurveto{\pgfqpoint{1.257435in}{2.067106in}}{\pgfqpoint{1.254163in}{2.075006in}}{\pgfqpoint{1.248339in}{2.080830in}}%
\pgfpathcurveto{\pgfqpoint{1.242515in}{2.086654in}}{\pgfqpoint{1.234615in}{2.089926in}}{\pgfqpoint{1.226379in}{2.089926in}}%
\pgfpathcurveto{\pgfqpoint{1.218143in}{2.089926in}}{\pgfqpoint{1.210243in}{2.086654in}}{\pgfqpoint{1.204419in}{2.080830in}}%
\pgfpathcurveto{\pgfqpoint{1.198595in}{2.075006in}}{\pgfqpoint{1.195322in}{2.067106in}}{\pgfqpoint{1.195322in}{2.058870in}}%
\pgfpathcurveto{\pgfqpoint{1.195322in}{2.050633in}}{\pgfqpoint{1.198595in}{2.042733in}}{\pgfqpoint{1.204419in}{2.036909in}}%
\pgfpathcurveto{\pgfqpoint{1.210243in}{2.031085in}}{\pgfqpoint{1.218143in}{2.027813in}}{\pgfqpoint{1.226379in}{2.027813in}}%
\pgfpathclose%
\pgfusepath{stroke,fill}%
\end{pgfscope}%
\begin{pgfscope}%
\pgfpathrectangle{\pgfqpoint{0.100000in}{0.220728in}}{\pgfqpoint{3.696000in}{3.696000in}}%
\pgfusepath{clip}%
\pgfsetbuttcap%
\pgfsetroundjoin%
\definecolor{currentfill}{rgb}{0.121569,0.466667,0.705882}%
\pgfsetfillcolor{currentfill}%
\pgfsetfillopacity{0.786548}%
\pgfsetlinewidth{1.003750pt}%
\definecolor{currentstroke}{rgb}{0.121569,0.466667,0.705882}%
\pgfsetstrokecolor{currentstroke}%
\pgfsetstrokeopacity{0.786548}%
\pgfsetdash{}{0pt}%
\pgfpathmoveto{\pgfqpoint{3.032664in}{2.433053in}}%
\pgfpathcurveto{\pgfqpoint{3.040900in}{2.433053in}}{\pgfqpoint{3.048800in}{2.436325in}}{\pgfqpoint{3.054624in}{2.442149in}}%
\pgfpathcurveto{\pgfqpoint{3.060448in}{2.447973in}}{\pgfqpoint{3.063720in}{2.455873in}}{\pgfqpoint{3.063720in}{2.464109in}}%
\pgfpathcurveto{\pgfqpoint{3.063720in}{2.472345in}}{\pgfqpoint{3.060448in}{2.480245in}}{\pgfqpoint{3.054624in}{2.486069in}}%
\pgfpathcurveto{\pgfqpoint{3.048800in}{2.491893in}}{\pgfqpoint{3.040900in}{2.495166in}}{\pgfqpoint{3.032664in}{2.495166in}}%
\pgfpathcurveto{\pgfqpoint{3.024427in}{2.495166in}}{\pgfqpoint{3.016527in}{2.491893in}}{\pgfqpoint{3.010703in}{2.486069in}}%
\pgfpathcurveto{\pgfqpoint{3.004880in}{2.480245in}}{\pgfqpoint{3.001607in}{2.472345in}}{\pgfqpoint{3.001607in}{2.464109in}}%
\pgfpathcurveto{\pgfqpoint{3.001607in}{2.455873in}}{\pgfqpoint{3.004880in}{2.447973in}}{\pgfqpoint{3.010703in}{2.442149in}}%
\pgfpathcurveto{\pgfqpoint{3.016527in}{2.436325in}}{\pgfqpoint{3.024427in}{2.433053in}}{\pgfqpoint{3.032664in}{2.433053in}}%
\pgfpathclose%
\pgfusepath{stroke,fill}%
\end{pgfscope}%
\begin{pgfscope}%
\pgfpathrectangle{\pgfqpoint{0.100000in}{0.220728in}}{\pgfqpoint{3.696000in}{3.696000in}}%
\pgfusepath{clip}%
\pgfsetbuttcap%
\pgfsetroundjoin%
\definecolor{currentfill}{rgb}{0.121569,0.466667,0.705882}%
\pgfsetfillcolor{currentfill}%
\pgfsetfillopacity{0.787235}%
\pgfsetlinewidth{1.003750pt}%
\definecolor{currentstroke}{rgb}{0.121569,0.466667,0.705882}%
\pgfsetstrokecolor{currentstroke}%
\pgfsetstrokeopacity{0.787235}%
\pgfsetdash{}{0pt}%
\pgfpathmoveto{\pgfqpoint{3.030750in}{2.430032in}}%
\pgfpathcurveto{\pgfqpoint{3.038986in}{2.430032in}}{\pgfqpoint{3.046886in}{2.433304in}}{\pgfqpoint{3.052710in}{2.439128in}}%
\pgfpathcurveto{\pgfqpoint{3.058534in}{2.444952in}}{\pgfqpoint{3.061806in}{2.452852in}}{\pgfqpoint{3.061806in}{2.461089in}}%
\pgfpathcurveto{\pgfqpoint{3.061806in}{2.469325in}}{\pgfqpoint{3.058534in}{2.477225in}}{\pgfqpoint{3.052710in}{2.483049in}}%
\pgfpathcurveto{\pgfqpoint{3.046886in}{2.488873in}}{\pgfqpoint{3.038986in}{2.492145in}}{\pgfqpoint{3.030750in}{2.492145in}}%
\pgfpathcurveto{\pgfqpoint{3.022514in}{2.492145in}}{\pgfqpoint{3.014614in}{2.488873in}}{\pgfqpoint{3.008790in}{2.483049in}}%
\pgfpathcurveto{\pgfqpoint{3.002966in}{2.477225in}}{\pgfqpoint{2.999693in}{2.469325in}}{\pgfqpoint{2.999693in}{2.461089in}}%
\pgfpathcurveto{\pgfqpoint{2.999693in}{2.452852in}}{\pgfqpoint{3.002966in}{2.444952in}}{\pgfqpoint{3.008790in}{2.439128in}}%
\pgfpathcurveto{\pgfqpoint{3.014614in}{2.433304in}}{\pgfqpoint{3.022514in}{2.430032in}}{\pgfqpoint{3.030750in}{2.430032in}}%
\pgfpathclose%
\pgfusepath{stroke,fill}%
\end{pgfscope}%
\begin{pgfscope}%
\pgfpathrectangle{\pgfqpoint{0.100000in}{0.220728in}}{\pgfqpoint{3.696000in}{3.696000in}}%
\pgfusepath{clip}%
\pgfsetbuttcap%
\pgfsetroundjoin%
\definecolor{currentfill}{rgb}{0.121569,0.466667,0.705882}%
\pgfsetfillcolor{currentfill}%
\pgfsetfillopacity{0.787589}%
\pgfsetlinewidth{1.003750pt}%
\definecolor{currentstroke}{rgb}{0.121569,0.466667,0.705882}%
\pgfsetstrokecolor{currentstroke}%
\pgfsetstrokeopacity{0.787589}%
\pgfsetdash{}{0pt}%
\pgfpathmoveto{\pgfqpoint{3.029540in}{2.428475in}}%
\pgfpathcurveto{\pgfqpoint{3.037776in}{2.428475in}}{\pgfqpoint{3.045676in}{2.431748in}}{\pgfqpoint{3.051500in}{2.437572in}}%
\pgfpathcurveto{\pgfqpoint{3.057324in}{2.443396in}}{\pgfqpoint{3.060596in}{2.451296in}}{\pgfqpoint{3.060596in}{2.459532in}}%
\pgfpathcurveto{\pgfqpoint{3.060596in}{2.467768in}}{\pgfqpoint{3.057324in}{2.475668in}}{\pgfqpoint{3.051500in}{2.481492in}}%
\pgfpathcurveto{\pgfqpoint{3.045676in}{2.487316in}}{\pgfqpoint{3.037776in}{2.490588in}}{\pgfqpoint{3.029540in}{2.490588in}}%
\pgfpathcurveto{\pgfqpoint{3.021303in}{2.490588in}}{\pgfqpoint{3.013403in}{2.487316in}}{\pgfqpoint{3.007579in}{2.481492in}}%
\pgfpathcurveto{\pgfqpoint{3.001756in}{2.475668in}}{\pgfqpoint{2.998483in}{2.467768in}}{\pgfqpoint{2.998483in}{2.459532in}}%
\pgfpathcurveto{\pgfqpoint{2.998483in}{2.451296in}}{\pgfqpoint{3.001756in}{2.443396in}}{\pgfqpoint{3.007579in}{2.437572in}}%
\pgfpathcurveto{\pgfqpoint{3.013403in}{2.431748in}}{\pgfqpoint{3.021303in}{2.428475in}}{\pgfqpoint{3.029540in}{2.428475in}}%
\pgfpathclose%
\pgfusepath{stroke,fill}%
\end{pgfscope}%
\begin{pgfscope}%
\pgfpathrectangle{\pgfqpoint{0.100000in}{0.220728in}}{\pgfqpoint{3.696000in}{3.696000in}}%
\pgfusepath{clip}%
\pgfsetbuttcap%
\pgfsetroundjoin%
\definecolor{currentfill}{rgb}{0.121569,0.466667,0.705882}%
\pgfsetfillcolor{currentfill}%
\pgfsetfillopacity{0.787793}%
\pgfsetlinewidth{1.003750pt}%
\definecolor{currentstroke}{rgb}{0.121569,0.466667,0.705882}%
\pgfsetstrokecolor{currentstroke}%
\pgfsetstrokeopacity{0.787793}%
\pgfsetdash{}{0pt}%
\pgfpathmoveto{\pgfqpoint{3.028984in}{2.427520in}}%
\pgfpathcurveto{\pgfqpoint{3.037220in}{2.427520in}}{\pgfqpoint{3.045120in}{2.430792in}}{\pgfqpoint{3.050944in}{2.436616in}}%
\pgfpathcurveto{\pgfqpoint{3.056768in}{2.442440in}}{\pgfqpoint{3.060040in}{2.450340in}}{\pgfqpoint{3.060040in}{2.458577in}}%
\pgfpathcurveto{\pgfqpoint{3.060040in}{2.466813in}}{\pgfqpoint{3.056768in}{2.474713in}}{\pgfqpoint{3.050944in}{2.480537in}}%
\pgfpathcurveto{\pgfqpoint{3.045120in}{2.486361in}}{\pgfqpoint{3.037220in}{2.489633in}}{\pgfqpoint{3.028984in}{2.489633in}}%
\pgfpathcurveto{\pgfqpoint{3.020747in}{2.489633in}}{\pgfqpoint{3.012847in}{2.486361in}}{\pgfqpoint{3.007023in}{2.480537in}}%
\pgfpathcurveto{\pgfqpoint{3.001200in}{2.474713in}}{\pgfqpoint{2.997927in}{2.466813in}}{\pgfqpoint{2.997927in}{2.458577in}}%
\pgfpathcurveto{\pgfqpoint{2.997927in}{2.450340in}}{\pgfqpoint{3.001200in}{2.442440in}}{\pgfqpoint{3.007023in}{2.436616in}}%
\pgfpathcurveto{\pgfqpoint{3.012847in}{2.430792in}}{\pgfqpoint{3.020747in}{2.427520in}}{\pgfqpoint{3.028984in}{2.427520in}}%
\pgfpathclose%
\pgfusepath{stroke,fill}%
\end{pgfscope}%
\begin{pgfscope}%
\pgfpathrectangle{\pgfqpoint{0.100000in}{0.220728in}}{\pgfqpoint{3.696000in}{3.696000in}}%
\pgfusepath{clip}%
\pgfsetbuttcap%
\pgfsetroundjoin%
\definecolor{currentfill}{rgb}{0.121569,0.466667,0.705882}%
\pgfsetfillcolor{currentfill}%
\pgfsetfillopacity{0.788627}%
\pgfsetlinewidth{1.003750pt}%
\definecolor{currentstroke}{rgb}{0.121569,0.466667,0.705882}%
\pgfsetstrokecolor{currentstroke}%
\pgfsetstrokeopacity{0.788627}%
\pgfsetdash{}{0pt}%
\pgfpathmoveto{\pgfqpoint{3.025894in}{2.424005in}}%
\pgfpathcurveto{\pgfqpoint{3.034130in}{2.424005in}}{\pgfqpoint{3.042030in}{2.427277in}}{\pgfqpoint{3.047854in}{2.433101in}}%
\pgfpathcurveto{\pgfqpoint{3.053678in}{2.438925in}}{\pgfqpoint{3.056950in}{2.446825in}}{\pgfqpoint{3.056950in}{2.455061in}}%
\pgfpathcurveto{\pgfqpoint{3.056950in}{2.463297in}}{\pgfqpoint{3.053678in}{2.471198in}}{\pgfqpoint{3.047854in}{2.477021in}}%
\pgfpathcurveto{\pgfqpoint{3.042030in}{2.482845in}}{\pgfqpoint{3.034130in}{2.486118in}}{\pgfqpoint{3.025894in}{2.486118in}}%
\pgfpathcurveto{\pgfqpoint{3.017657in}{2.486118in}}{\pgfqpoint{3.009757in}{2.482845in}}{\pgfqpoint{3.003933in}{2.477021in}}%
\pgfpathcurveto{\pgfqpoint{2.998110in}{2.471198in}}{\pgfqpoint{2.994837in}{2.463297in}}{\pgfqpoint{2.994837in}{2.455061in}}%
\pgfpathcurveto{\pgfqpoint{2.994837in}{2.446825in}}{\pgfqpoint{2.998110in}{2.438925in}}{\pgfqpoint{3.003933in}{2.433101in}}%
\pgfpathcurveto{\pgfqpoint{3.009757in}{2.427277in}}{\pgfqpoint{3.017657in}{2.424005in}}{\pgfqpoint{3.025894in}{2.424005in}}%
\pgfpathclose%
\pgfusepath{stroke,fill}%
\end{pgfscope}%
\begin{pgfscope}%
\pgfpathrectangle{\pgfqpoint{0.100000in}{0.220728in}}{\pgfqpoint{3.696000in}{3.696000in}}%
\pgfusepath{clip}%
\pgfsetbuttcap%
\pgfsetroundjoin%
\definecolor{currentfill}{rgb}{0.121569,0.466667,0.705882}%
\pgfsetfillcolor{currentfill}%
\pgfsetfillopacity{0.789001}%
\pgfsetlinewidth{1.003750pt}%
\definecolor{currentstroke}{rgb}{0.121569,0.466667,0.705882}%
\pgfsetstrokecolor{currentstroke}%
\pgfsetstrokeopacity{0.789001}%
\pgfsetdash{}{0pt}%
\pgfpathmoveto{\pgfqpoint{1.248181in}{2.019327in}}%
\pgfpathcurveto{\pgfqpoint{1.256418in}{2.019327in}}{\pgfqpoint{1.264318in}{2.022599in}}{\pgfqpoint{1.270142in}{2.028423in}}%
\pgfpathcurveto{\pgfqpoint{1.275965in}{2.034247in}}{\pgfqpoint{1.279238in}{2.042147in}}{\pgfqpoint{1.279238in}{2.050383in}}%
\pgfpathcurveto{\pgfqpoint{1.279238in}{2.058620in}}{\pgfqpoint{1.275965in}{2.066520in}}{\pgfqpoint{1.270142in}{2.072344in}}%
\pgfpathcurveto{\pgfqpoint{1.264318in}{2.078167in}}{\pgfqpoint{1.256418in}{2.081440in}}{\pgfqpoint{1.248181in}{2.081440in}}%
\pgfpathcurveto{\pgfqpoint{1.239945in}{2.081440in}}{\pgfqpoint{1.232045in}{2.078167in}}{\pgfqpoint{1.226221in}{2.072344in}}%
\pgfpathcurveto{\pgfqpoint{1.220397in}{2.066520in}}{\pgfqpoint{1.217125in}{2.058620in}}{\pgfqpoint{1.217125in}{2.050383in}}%
\pgfpathcurveto{\pgfqpoint{1.217125in}{2.042147in}}{\pgfqpoint{1.220397in}{2.034247in}}{\pgfqpoint{1.226221in}{2.028423in}}%
\pgfpathcurveto{\pgfqpoint{1.232045in}{2.022599in}}{\pgfqpoint{1.239945in}{2.019327in}}{\pgfqpoint{1.248181in}{2.019327in}}%
\pgfpathclose%
\pgfusepath{stroke,fill}%
\end{pgfscope}%
\begin{pgfscope}%
\pgfpathrectangle{\pgfqpoint{0.100000in}{0.220728in}}{\pgfqpoint{3.696000in}{3.696000in}}%
\pgfusepath{clip}%
\pgfsetbuttcap%
\pgfsetroundjoin%
\definecolor{currentfill}{rgb}{0.121569,0.466667,0.705882}%
\pgfsetfillcolor{currentfill}%
\pgfsetfillopacity{0.789107}%
\pgfsetlinewidth{1.003750pt}%
\definecolor{currentstroke}{rgb}{0.121569,0.466667,0.705882}%
\pgfsetstrokecolor{currentstroke}%
\pgfsetstrokeopacity{0.789107}%
\pgfsetdash{}{0pt}%
\pgfpathmoveto{\pgfqpoint{3.024556in}{2.421632in}}%
\pgfpathcurveto{\pgfqpoint{3.032792in}{2.421632in}}{\pgfqpoint{3.040692in}{2.424904in}}{\pgfqpoint{3.046516in}{2.430728in}}%
\pgfpathcurveto{\pgfqpoint{3.052340in}{2.436552in}}{\pgfqpoint{3.055612in}{2.444452in}}{\pgfqpoint{3.055612in}{2.452688in}}%
\pgfpathcurveto{\pgfqpoint{3.055612in}{2.460925in}}{\pgfqpoint{3.052340in}{2.468825in}}{\pgfqpoint{3.046516in}{2.474648in}}%
\pgfpathcurveto{\pgfqpoint{3.040692in}{2.480472in}}{\pgfqpoint{3.032792in}{2.483745in}}{\pgfqpoint{3.024556in}{2.483745in}}%
\pgfpathcurveto{\pgfqpoint{3.016319in}{2.483745in}}{\pgfqpoint{3.008419in}{2.480472in}}{\pgfqpoint{3.002595in}{2.474648in}}%
\pgfpathcurveto{\pgfqpoint{2.996771in}{2.468825in}}{\pgfqpoint{2.993499in}{2.460925in}}{\pgfqpoint{2.993499in}{2.452688in}}%
\pgfpathcurveto{\pgfqpoint{2.993499in}{2.444452in}}{\pgfqpoint{2.996771in}{2.436552in}}{\pgfqpoint{3.002595in}{2.430728in}}%
\pgfpathcurveto{\pgfqpoint{3.008419in}{2.424904in}}{\pgfqpoint{3.016319in}{2.421632in}}{\pgfqpoint{3.024556in}{2.421632in}}%
\pgfpathclose%
\pgfusepath{stroke,fill}%
\end{pgfscope}%
\begin{pgfscope}%
\pgfpathrectangle{\pgfqpoint{0.100000in}{0.220728in}}{\pgfqpoint{3.696000in}{3.696000in}}%
\pgfusepath{clip}%
\pgfsetbuttcap%
\pgfsetroundjoin%
\definecolor{currentfill}{rgb}{0.121569,0.466667,0.705882}%
\pgfsetfillcolor{currentfill}%
\pgfsetfillopacity{0.790391}%
\pgfsetlinewidth{1.003750pt}%
\definecolor{currentstroke}{rgb}{0.121569,0.466667,0.705882}%
\pgfsetstrokecolor{currentstroke}%
\pgfsetstrokeopacity{0.790391}%
\pgfsetdash{}{0pt}%
\pgfpathmoveto{\pgfqpoint{3.019687in}{2.415518in}}%
\pgfpathcurveto{\pgfqpoint{3.027924in}{2.415518in}}{\pgfqpoint{3.035824in}{2.418790in}}{\pgfqpoint{3.041648in}{2.424614in}}%
\pgfpathcurveto{\pgfqpoint{3.047472in}{2.430438in}}{\pgfqpoint{3.050744in}{2.438338in}}{\pgfqpoint{3.050744in}{2.446574in}}%
\pgfpathcurveto{\pgfqpoint{3.050744in}{2.454810in}}{\pgfqpoint{3.047472in}{2.462711in}}{\pgfqpoint{3.041648in}{2.468534in}}%
\pgfpathcurveto{\pgfqpoint{3.035824in}{2.474358in}}{\pgfqpoint{3.027924in}{2.477631in}}{\pgfqpoint{3.019687in}{2.477631in}}%
\pgfpathcurveto{\pgfqpoint{3.011451in}{2.477631in}}{\pgfqpoint{3.003551in}{2.474358in}}{\pgfqpoint{2.997727in}{2.468534in}}%
\pgfpathcurveto{\pgfqpoint{2.991903in}{2.462711in}}{\pgfqpoint{2.988631in}{2.454810in}}{\pgfqpoint{2.988631in}{2.446574in}}%
\pgfpathcurveto{\pgfqpoint{2.988631in}{2.438338in}}{\pgfqpoint{2.991903in}{2.430438in}}{\pgfqpoint{2.997727in}{2.424614in}}%
\pgfpathcurveto{\pgfqpoint{3.003551in}{2.418790in}}{\pgfqpoint{3.011451in}{2.415518in}}{\pgfqpoint{3.019687in}{2.415518in}}%
\pgfpathclose%
\pgfusepath{stroke,fill}%
\end{pgfscope}%
\begin{pgfscope}%
\pgfpathrectangle{\pgfqpoint{0.100000in}{0.220728in}}{\pgfqpoint{3.696000in}{3.696000in}}%
\pgfusepath{clip}%
\pgfsetbuttcap%
\pgfsetroundjoin%
\definecolor{currentfill}{rgb}{0.121569,0.466667,0.705882}%
\pgfsetfillcolor{currentfill}%
\pgfsetfillopacity{0.792204}%
\pgfsetlinewidth{1.003750pt}%
\definecolor{currentstroke}{rgb}{0.121569,0.466667,0.705882}%
\pgfsetstrokecolor{currentstroke}%
\pgfsetstrokeopacity{0.792204}%
\pgfsetdash{}{0pt}%
\pgfpathmoveto{\pgfqpoint{3.014584in}{2.405583in}}%
\pgfpathcurveto{\pgfqpoint{3.022821in}{2.405583in}}{\pgfqpoint{3.030721in}{2.408856in}}{\pgfqpoint{3.036545in}{2.414679in}}%
\pgfpathcurveto{\pgfqpoint{3.042369in}{2.420503in}}{\pgfqpoint{3.045641in}{2.428403in}}{\pgfqpoint{3.045641in}{2.436640in}}%
\pgfpathcurveto{\pgfqpoint{3.045641in}{2.444876in}}{\pgfqpoint{3.042369in}{2.452776in}}{\pgfqpoint{3.036545in}{2.458600in}}%
\pgfpathcurveto{\pgfqpoint{3.030721in}{2.464424in}}{\pgfqpoint{3.022821in}{2.467696in}}{\pgfqpoint{3.014584in}{2.467696in}}%
\pgfpathcurveto{\pgfqpoint{3.006348in}{2.467696in}}{\pgfqpoint{2.998448in}{2.464424in}}{\pgfqpoint{2.992624in}{2.458600in}}%
\pgfpathcurveto{\pgfqpoint{2.986800in}{2.452776in}}{\pgfqpoint{2.983528in}{2.444876in}}{\pgfqpoint{2.983528in}{2.436640in}}%
\pgfpathcurveto{\pgfqpoint{2.983528in}{2.428403in}}{\pgfqpoint{2.986800in}{2.420503in}}{\pgfqpoint{2.992624in}{2.414679in}}%
\pgfpathcurveto{\pgfqpoint{2.998448in}{2.408856in}}{\pgfqpoint{3.006348in}{2.405583in}}{\pgfqpoint{3.014584in}{2.405583in}}%
\pgfpathclose%
\pgfusepath{stroke,fill}%
\end{pgfscope}%
\begin{pgfscope}%
\pgfpathrectangle{\pgfqpoint{0.100000in}{0.220728in}}{\pgfqpoint{3.696000in}{3.696000in}}%
\pgfusepath{clip}%
\pgfsetbuttcap%
\pgfsetroundjoin%
\definecolor{currentfill}{rgb}{0.121569,0.466667,0.705882}%
\pgfsetfillcolor{currentfill}%
\pgfsetfillopacity{0.792504}%
\pgfsetlinewidth{1.003750pt}%
\definecolor{currentstroke}{rgb}{0.121569,0.466667,0.705882}%
\pgfsetstrokecolor{currentstroke}%
\pgfsetstrokeopacity{0.792504}%
\pgfsetdash{}{0pt}%
\pgfpathmoveto{\pgfqpoint{1.265641in}{2.009789in}}%
\pgfpathcurveto{\pgfqpoint{1.273877in}{2.009789in}}{\pgfqpoint{1.281778in}{2.013061in}}{\pgfqpoint{1.287601in}{2.018885in}}%
\pgfpathcurveto{\pgfqpoint{1.293425in}{2.024709in}}{\pgfqpoint{1.296698in}{2.032609in}}{\pgfqpoint{1.296698in}{2.040845in}}%
\pgfpathcurveto{\pgfqpoint{1.296698in}{2.049082in}}{\pgfqpoint{1.293425in}{2.056982in}}{\pgfqpoint{1.287601in}{2.062806in}}%
\pgfpathcurveto{\pgfqpoint{1.281778in}{2.068630in}}{\pgfqpoint{1.273877in}{2.071902in}}{\pgfqpoint{1.265641in}{2.071902in}}%
\pgfpathcurveto{\pgfqpoint{1.257405in}{2.071902in}}{\pgfqpoint{1.249505in}{2.068630in}}{\pgfqpoint{1.243681in}{2.062806in}}%
\pgfpathcurveto{\pgfqpoint{1.237857in}{2.056982in}}{\pgfqpoint{1.234585in}{2.049082in}}{\pgfqpoint{1.234585in}{2.040845in}}%
\pgfpathcurveto{\pgfqpoint{1.234585in}{2.032609in}}{\pgfqpoint{1.237857in}{2.024709in}}{\pgfqpoint{1.243681in}{2.018885in}}%
\pgfpathcurveto{\pgfqpoint{1.249505in}{2.013061in}}{\pgfqpoint{1.257405in}{2.009789in}}{\pgfqpoint{1.265641in}{2.009789in}}%
\pgfpathclose%
\pgfusepath{stroke,fill}%
\end{pgfscope}%
\begin{pgfscope}%
\pgfpathrectangle{\pgfqpoint{0.100000in}{0.220728in}}{\pgfqpoint{3.696000in}{3.696000in}}%
\pgfusepath{clip}%
\pgfsetbuttcap%
\pgfsetroundjoin%
\definecolor{currentfill}{rgb}{0.121569,0.466667,0.705882}%
\pgfsetfillcolor{currentfill}%
\pgfsetfillopacity{0.794428}%
\pgfsetlinewidth{1.003750pt}%
\definecolor{currentstroke}{rgb}{0.121569,0.466667,0.705882}%
\pgfsetstrokecolor{currentstroke}%
\pgfsetstrokeopacity{0.794428}%
\pgfsetdash{}{0pt}%
\pgfpathmoveto{\pgfqpoint{3.005319in}{2.392272in}}%
\pgfpathcurveto{\pgfqpoint{3.013555in}{2.392272in}}{\pgfqpoint{3.021455in}{2.395544in}}{\pgfqpoint{3.027279in}{2.401368in}}%
\pgfpathcurveto{\pgfqpoint{3.033103in}{2.407192in}}{\pgfqpoint{3.036375in}{2.415092in}}{\pgfqpoint{3.036375in}{2.423328in}}%
\pgfpathcurveto{\pgfqpoint{3.036375in}{2.431564in}}{\pgfqpoint{3.033103in}{2.439465in}}{\pgfqpoint{3.027279in}{2.445288in}}%
\pgfpathcurveto{\pgfqpoint{3.021455in}{2.451112in}}{\pgfqpoint{3.013555in}{2.454385in}}{\pgfqpoint{3.005319in}{2.454385in}}%
\pgfpathcurveto{\pgfqpoint{2.997082in}{2.454385in}}{\pgfqpoint{2.989182in}{2.451112in}}{\pgfqpoint{2.983358in}{2.445288in}}%
\pgfpathcurveto{\pgfqpoint{2.977534in}{2.439465in}}{\pgfqpoint{2.974262in}{2.431564in}}{\pgfqpoint{2.974262in}{2.423328in}}%
\pgfpathcurveto{\pgfqpoint{2.974262in}{2.415092in}}{\pgfqpoint{2.977534in}{2.407192in}}{\pgfqpoint{2.983358in}{2.401368in}}%
\pgfpathcurveto{\pgfqpoint{2.989182in}{2.395544in}}{\pgfqpoint{2.997082in}{2.392272in}}{\pgfqpoint{3.005319in}{2.392272in}}%
\pgfpathclose%
\pgfusepath{stroke,fill}%
\end{pgfscope}%
\begin{pgfscope}%
\pgfpathrectangle{\pgfqpoint{0.100000in}{0.220728in}}{\pgfqpoint{3.696000in}{3.696000in}}%
\pgfusepath{clip}%
\pgfsetbuttcap%
\pgfsetroundjoin%
\definecolor{currentfill}{rgb}{0.121569,0.466667,0.705882}%
\pgfsetfillcolor{currentfill}%
\pgfsetfillopacity{0.795323}%
\pgfsetlinewidth{1.003750pt}%
\definecolor{currentstroke}{rgb}{0.121569,0.466667,0.705882}%
\pgfsetstrokecolor{currentstroke}%
\pgfsetstrokeopacity{0.795323}%
\pgfsetdash{}{0pt}%
\pgfpathmoveto{\pgfqpoint{1.281922in}{2.001740in}}%
\pgfpathcurveto{\pgfqpoint{1.290158in}{2.001740in}}{\pgfqpoint{1.298058in}{2.005012in}}{\pgfqpoint{1.303882in}{2.010836in}}%
\pgfpathcurveto{\pgfqpoint{1.309706in}{2.016660in}}{\pgfqpoint{1.312978in}{2.024560in}}{\pgfqpoint{1.312978in}{2.032797in}}%
\pgfpathcurveto{\pgfqpoint{1.312978in}{2.041033in}}{\pgfqpoint{1.309706in}{2.048933in}}{\pgfqpoint{1.303882in}{2.054757in}}%
\pgfpathcurveto{\pgfqpoint{1.298058in}{2.060581in}}{\pgfqpoint{1.290158in}{2.063853in}}{\pgfqpoint{1.281922in}{2.063853in}}%
\pgfpathcurveto{\pgfqpoint{1.273686in}{2.063853in}}{\pgfqpoint{1.265785in}{2.060581in}}{\pgfqpoint{1.259962in}{2.054757in}}%
\pgfpathcurveto{\pgfqpoint{1.254138in}{2.048933in}}{\pgfqpoint{1.250865in}{2.041033in}}{\pgfqpoint{1.250865in}{2.032797in}}%
\pgfpathcurveto{\pgfqpoint{1.250865in}{2.024560in}}{\pgfqpoint{1.254138in}{2.016660in}}{\pgfqpoint{1.259962in}{2.010836in}}%
\pgfpathcurveto{\pgfqpoint{1.265785in}{2.005012in}}{\pgfqpoint{1.273686in}{2.001740in}}{\pgfqpoint{1.281922in}{2.001740in}}%
\pgfpathclose%
\pgfusepath{stroke,fill}%
\end{pgfscope}%
\begin{pgfscope}%
\pgfpathrectangle{\pgfqpoint{0.100000in}{0.220728in}}{\pgfqpoint{3.696000in}{3.696000in}}%
\pgfusepath{clip}%
\pgfsetbuttcap%
\pgfsetroundjoin%
\definecolor{currentfill}{rgb}{0.121569,0.466667,0.705882}%
\pgfsetfillcolor{currentfill}%
\pgfsetfillopacity{0.797072}%
\pgfsetlinewidth{1.003750pt}%
\definecolor{currentstroke}{rgb}{0.121569,0.466667,0.705882}%
\pgfsetstrokecolor{currentstroke}%
\pgfsetstrokeopacity{0.797072}%
\pgfsetdash{}{0pt}%
\pgfpathmoveto{\pgfqpoint{1.292660in}{1.998071in}}%
\pgfpathcurveto{\pgfqpoint{1.300896in}{1.998071in}}{\pgfqpoint{1.308796in}{2.001343in}}{\pgfqpoint{1.314620in}{2.007167in}}%
\pgfpathcurveto{\pgfqpoint{1.320444in}{2.012991in}}{\pgfqpoint{1.323716in}{2.020891in}}{\pgfqpoint{1.323716in}{2.029127in}}%
\pgfpathcurveto{\pgfqpoint{1.323716in}{2.037363in}}{\pgfqpoint{1.320444in}{2.045263in}}{\pgfqpoint{1.314620in}{2.051087in}}%
\pgfpathcurveto{\pgfqpoint{1.308796in}{2.056911in}}{\pgfqpoint{1.300896in}{2.060184in}}{\pgfqpoint{1.292660in}{2.060184in}}%
\pgfpathcurveto{\pgfqpoint{1.284424in}{2.060184in}}{\pgfqpoint{1.276524in}{2.056911in}}{\pgfqpoint{1.270700in}{2.051087in}}%
\pgfpathcurveto{\pgfqpoint{1.264876in}{2.045263in}}{\pgfqpoint{1.261603in}{2.037363in}}{\pgfqpoint{1.261603in}{2.029127in}}%
\pgfpathcurveto{\pgfqpoint{1.261603in}{2.020891in}}{\pgfqpoint{1.264876in}{2.012991in}}{\pgfqpoint{1.270700in}{2.007167in}}%
\pgfpathcurveto{\pgfqpoint{1.276524in}{2.001343in}}{\pgfqpoint{1.284424in}{1.998071in}}{\pgfqpoint{1.292660in}{1.998071in}}%
\pgfpathclose%
\pgfusepath{stroke,fill}%
\end{pgfscope}%
\begin{pgfscope}%
\pgfpathrectangle{\pgfqpoint{0.100000in}{0.220728in}}{\pgfqpoint{3.696000in}{3.696000in}}%
\pgfusepath{clip}%
\pgfsetbuttcap%
\pgfsetroundjoin%
\definecolor{currentfill}{rgb}{0.121569,0.466667,0.705882}%
\pgfsetfillcolor{currentfill}%
\pgfsetfillopacity{0.797933}%
\pgfsetlinewidth{1.003750pt}%
\definecolor{currentstroke}{rgb}{0.121569,0.466667,0.705882}%
\pgfsetstrokecolor{currentstroke}%
\pgfsetstrokeopacity{0.797933}%
\pgfsetdash{}{0pt}%
\pgfpathmoveto{\pgfqpoint{1.297348in}{1.995620in}}%
\pgfpathcurveto{\pgfqpoint{1.305584in}{1.995620in}}{\pgfqpoint{1.313484in}{1.998892in}}{\pgfqpoint{1.319308in}{2.004716in}}%
\pgfpathcurveto{\pgfqpoint{1.325132in}{2.010540in}}{\pgfqpoint{1.328404in}{2.018440in}}{\pgfqpoint{1.328404in}{2.026676in}}%
\pgfpathcurveto{\pgfqpoint{1.328404in}{2.034912in}}{\pgfqpoint{1.325132in}{2.042812in}}{\pgfqpoint{1.319308in}{2.048636in}}%
\pgfpathcurveto{\pgfqpoint{1.313484in}{2.054460in}}{\pgfqpoint{1.305584in}{2.057733in}}{\pgfqpoint{1.297348in}{2.057733in}}%
\pgfpathcurveto{\pgfqpoint{1.289111in}{2.057733in}}{\pgfqpoint{1.281211in}{2.054460in}}{\pgfqpoint{1.275387in}{2.048636in}}%
\pgfpathcurveto{\pgfqpoint{1.269563in}{2.042812in}}{\pgfqpoint{1.266291in}{2.034912in}}{\pgfqpoint{1.266291in}{2.026676in}}%
\pgfpathcurveto{\pgfqpoint{1.266291in}{2.018440in}}{\pgfqpoint{1.269563in}{2.010540in}}{\pgfqpoint{1.275387in}{2.004716in}}%
\pgfpathcurveto{\pgfqpoint{1.281211in}{1.998892in}}{\pgfqpoint{1.289111in}{1.995620in}}{\pgfqpoint{1.297348in}{1.995620in}}%
\pgfpathclose%
\pgfusepath{stroke,fill}%
\end{pgfscope}%
\begin{pgfscope}%
\pgfpathrectangle{\pgfqpoint{0.100000in}{0.220728in}}{\pgfqpoint{3.696000in}{3.696000in}}%
\pgfusepath{clip}%
\pgfsetbuttcap%
\pgfsetroundjoin%
\definecolor{currentfill}{rgb}{0.121569,0.466667,0.705882}%
\pgfsetfillcolor{currentfill}%
\pgfsetfillopacity{0.798063}%
\pgfsetlinewidth{1.003750pt}%
\definecolor{currentstroke}{rgb}{0.121569,0.466667,0.705882}%
\pgfsetstrokecolor{currentstroke}%
\pgfsetstrokeopacity{0.798063}%
\pgfsetdash{}{0pt}%
\pgfpathmoveto{\pgfqpoint{2.995727in}{2.374541in}}%
\pgfpathcurveto{\pgfqpoint{3.003963in}{2.374541in}}{\pgfqpoint{3.011863in}{2.377813in}}{\pgfqpoint{3.017687in}{2.383637in}}%
\pgfpathcurveto{\pgfqpoint{3.023511in}{2.389461in}}{\pgfqpoint{3.026783in}{2.397361in}}{\pgfqpoint{3.026783in}{2.405597in}}%
\pgfpathcurveto{\pgfqpoint{3.026783in}{2.413833in}}{\pgfqpoint{3.023511in}{2.421733in}}{\pgfqpoint{3.017687in}{2.427557in}}%
\pgfpathcurveto{\pgfqpoint{3.011863in}{2.433381in}}{\pgfqpoint{3.003963in}{2.436654in}}{\pgfqpoint{2.995727in}{2.436654in}}%
\pgfpathcurveto{\pgfqpoint{2.987490in}{2.436654in}}{\pgfqpoint{2.979590in}{2.433381in}}{\pgfqpoint{2.973766in}{2.427557in}}%
\pgfpathcurveto{\pgfqpoint{2.967942in}{2.421733in}}{\pgfqpoint{2.964670in}{2.413833in}}{\pgfqpoint{2.964670in}{2.405597in}}%
\pgfpathcurveto{\pgfqpoint{2.964670in}{2.397361in}}{\pgfqpoint{2.967942in}{2.389461in}}{\pgfqpoint{2.973766in}{2.383637in}}%
\pgfpathcurveto{\pgfqpoint{2.979590in}{2.377813in}}{\pgfqpoint{2.987490in}{2.374541in}}{\pgfqpoint{2.995727in}{2.374541in}}%
\pgfpathclose%
\pgfusepath{stroke,fill}%
\end{pgfscope}%
\begin{pgfscope}%
\pgfpathrectangle{\pgfqpoint{0.100000in}{0.220728in}}{\pgfqpoint{3.696000in}{3.696000in}}%
\pgfusepath{clip}%
\pgfsetbuttcap%
\pgfsetroundjoin%
\definecolor{currentfill}{rgb}{0.121569,0.466667,0.705882}%
\pgfsetfillcolor{currentfill}%
\pgfsetfillopacity{0.798094}%
\pgfsetlinewidth{1.003750pt}%
\definecolor{currentstroke}{rgb}{0.121569,0.466667,0.705882}%
\pgfsetstrokecolor{currentstroke}%
\pgfsetstrokeopacity{0.798094}%
\pgfsetdash{}{0pt}%
\pgfpathmoveto{\pgfqpoint{1.298454in}{1.995458in}}%
\pgfpathcurveto{\pgfqpoint{1.306690in}{1.995458in}}{\pgfqpoint{1.314590in}{1.998731in}}{\pgfqpoint{1.320414in}{2.004554in}}%
\pgfpathcurveto{\pgfqpoint{1.326238in}{2.010378in}}{\pgfqpoint{1.329510in}{2.018278in}}{\pgfqpoint{1.329510in}{2.026515in}}%
\pgfpathcurveto{\pgfqpoint{1.329510in}{2.034751in}}{\pgfqpoint{1.326238in}{2.042651in}}{\pgfqpoint{1.320414in}{2.048475in}}%
\pgfpathcurveto{\pgfqpoint{1.314590in}{2.054299in}}{\pgfqpoint{1.306690in}{2.057571in}}{\pgfqpoint{1.298454in}{2.057571in}}%
\pgfpathcurveto{\pgfqpoint{1.290217in}{2.057571in}}{\pgfqpoint{1.282317in}{2.054299in}}{\pgfqpoint{1.276493in}{2.048475in}}%
\pgfpathcurveto{\pgfqpoint{1.270669in}{2.042651in}}{\pgfqpoint{1.267397in}{2.034751in}}{\pgfqpoint{1.267397in}{2.026515in}}%
\pgfpathcurveto{\pgfqpoint{1.267397in}{2.018278in}}{\pgfqpoint{1.270669in}{2.010378in}}{\pgfqpoint{1.276493in}{2.004554in}}%
\pgfpathcurveto{\pgfqpoint{1.282317in}{1.998731in}}{\pgfqpoint{1.290217in}{1.995458in}}{\pgfqpoint{1.298454in}{1.995458in}}%
\pgfpathclose%
\pgfusepath{stroke,fill}%
\end{pgfscope}%
\begin{pgfscope}%
\pgfpathrectangle{\pgfqpoint{0.100000in}{0.220728in}}{\pgfqpoint{3.696000in}{3.696000in}}%
\pgfusepath{clip}%
\pgfsetbuttcap%
\pgfsetroundjoin%
\definecolor{currentfill}{rgb}{0.121569,0.466667,0.705882}%
\pgfsetfillcolor{currentfill}%
\pgfsetfillopacity{0.798389}%
\pgfsetlinewidth{1.003750pt}%
\definecolor{currentstroke}{rgb}{0.121569,0.466667,0.705882}%
\pgfsetstrokecolor{currentstroke}%
\pgfsetstrokeopacity{0.798389}%
\pgfsetdash{}{0pt}%
\pgfpathmoveto{\pgfqpoint{1.300146in}{1.994113in}}%
\pgfpathcurveto{\pgfqpoint{1.308383in}{1.994113in}}{\pgfqpoint{1.316283in}{1.997385in}}{\pgfqpoint{1.322107in}{2.003209in}}%
\pgfpathcurveto{\pgfqpoint{1.327931in}{2.009033in}}{\pgfqpoint{1.331203in}{2.016933in}}{\pgfqpoint{1.331203in}{2.025169in}}%
\pgfpathcurveto{\pgfqpoint{1.331203in}{2.033405in}}{\pgfqpoint{1.327931in}{2.041305in}}{\pgfqpoint{1.322107in}{2.047129in}}%
\pgfpathcurveto{\pgfqpoint{1.316283in}{2.052953in}}{\pgfqpoint{1.308383in}{2.056226in}}{\pgfqpoint{1.300146in}{2.056226in}}%
\pgfpathcurveto{\pgfqpoint{1.291910in}{2.056226in}}{\pgfqpoint{1.284010in}{2.052953in}}{\pgfqpoint{1.278186in}{2.047129in}}%
\pgfpathcurveto{\pgfqpoint{1.272362in}{2.041305in}}{\pgfqpoint{1.269090in}{2.033405in}}{\pgfqpoint{1.269090in}{2.025169in}}%
\pgfpathcurveto{\pgfqpoint{1.269090in}{2.016933in}}{\pgfqpoint{1.272362in}{2.009033in}}{\pgfqpoint{1.278186in}{2.003209in}}%
\pgfpathcurveto{\pgfqpoint{1.284010in}{1.997385in}}{\pgfqpoint{1.291910in}{1.994113in}}{\pgfqpoint{1.300146in}{1.994113in}}%
\pgfpathclose%
\pgfusepath{stroke,fill}%
\end{pgfscope}%
\begin{pgfscope}%
\pgfpathrectangle{\pgfqpoint{0.100000in}{0.220728in}}{\pgfqpoint{3.696000in}{3.696000in}}%
\pgfusepath{clip}%
\pgfsetbuttcap%
\pgfsetroundjoin%
\definecolor{currentfill}{rgb}{0.121569,0.466667,0.705882}%
\pgfsetfillcolor{currentfill}%
\pgfsetfillopacity{0.799005}%
\pgfsetlinewidth{1.003750pt}%
\definecolor{currentstroke}{rgb}{0.121569,0.466667,0.705882}%
\pgfsetstrokecolor{currentstroke}%
\pgfsetstrokeopacity{0.799005}%
\pgfsetdash{}{0pt}%
\pgfpathmoveto{\pgfqpoint{1.303655in}{1.993419in}}%
\pgfpathcurveto{\pgfqpoint{1.311891in}{1.993419in}}{\pgfqpoint{1.319791in}{1.996691in}}{\pgfqpoint{1.325615in}{2.002515in}}%
\pgfpathcurveto{\pgfqpoint{1.331439in}{2.008339in}}{\pgfqpoint{1.334712in}{2.016239in}}{\pgfqpoint{1.334712in}{2.024475in}}%
\pgfpathcurveto{\pgfqpoint{1.334712in}{2.032712in}}{\pgfqpoint{1.331439in}{2.040612in}}{\pgfqpoint{1.325615in}{2.046436in}}%
\pgfpathcurveto{\pgfqpoint{1.319791in}{2.052259in}}{\pgfqpoint{1.311891in}{2.055532in}}{\pgfqpoint{1.303655in}{2.055532in}}%
\pgfpathcurveto{\pgfqpoint{1.295419in}{2.055532in}}{\pgfqpoint{1.287519in}{2.052259in}}{\pgfqpoint{1.281695in}{2.046436in}}%
\pgfpathcurveto{\pgfqpoint{1.275871in}{2.040612in}}{\pgfqpoint{1.272599in}{2.032712in}}{\pgfqpoint{1.272599in}{2.024475in}}%
\pgfpathcurveto{\pgfqpoint{1.272599in}{2.016239in}}{\pgfqpoint{1.275871in}{2.008339in}}{\pgfqpoint{1.281695in}{2.002515in}}%
\pgfpathcurveto{\pgfqpoint{1.287519in}{1.996691in}}{\pgfqpoint{1.295419in}{1.993419in}}{\pgfqpoint{1.303655in}{1.993419in}}%
\pgfpathclose%
\pgfusepath{stroke,fill}%
\end{pgfscope}%
\begin{pgfscope}%
\pgfpathrectangle{\pgfqpoint{0.100000in}{0.220728in}}{\pgfqpoint{3.696000in}{3.696000in}}%
\pgfusepath{clip}%
\pgfsetbuttcap%
\pgfsetroundjoin%
\definecolor{currentfill}{rgb}{0.121569,0.466667,0.705882}%
\pgfsetfillcolor{currentfill}%
\pgfsetfillopacity{0.800031}%
\pgfsetlinewidth{1.003750pt}%
\definecolor{currentstroke}{rgb}{0.121569,0.466667,0.705882}%
\pgfsetstrokecolor{currentstroke}%
\pgfsetstrokeopacity{0.800031}%
\pgfsetdash{}{0pt}%
\pgfpathmoveto{\pgfqpoint{1.309460in}{1.989710in}}%
\pgfpathcurveto{\pgfqpoint{1.317696in}{1.989710in}}{\pgfqpoint{1.325596in}{1.992982in}}{\pgfqpoint{1.331420in}{1.998806in}}%
\pgfpathcurveto{\pgfqpoint{1.337244in}{2.004630in}}{\pgfqpoint{1.340516in}{2.012530in}}{\pgfqpoint{1.340516in}{2.020767in}}%
\pgfpathcurveto{\pgfqpoint{1.340516in}{2.029003in}}{\pgfqpoint{1.337244in}{2.036903in}}{\pgfqpoint{1.331420in}{2.042727in}}%
\pgfpathcurveto{\pgfqpoint{1.325596in}{2.048551in}}{\pgfqpoint{1.317696in}{2.051823in}}{\pgfqpoint{1.309460in}{2.051823in}}%
\pgfpathcurveto{\pgfqpoint{1.301224in}{2.051823in}}{\pgfqpoint{1.293324in}{2.048551in}}{\pgfqpoint{1.287500in}{2.042727in}}%
\pgfpathcurveto{\pgfqpoint{1.281676in}{2.036903in}}{\pgfqpoint{1.278403in}{2.029003in}}{\pgfqpoint{1.278403in}{2.020767in}}%
\pgfpathcurveto{\pgfqpoint{1.278403in}{2.012530in}}{\pgfqpoint{1.281676in}{2.004630in}}{\pgfqpoint{1.287500in}{1.998806in}}%
\pgfpathcurveto{\pgfqpoint{1.293324in}{1.992982in}}{\pgfqpoint{1.301224in}{1.989710in}}{\pgfqpoint{1.309460in}{1.989710in}}%
\pgfpathclose%
\pgfusepath{stroke,fill}%
\end{pgfscope}%
\begin{pgfscope}%
\pgfpathrectangle{\pgfqpoint{0.100000in}{0.220728in}}{\pgfqpoint{3.696000in}{3.696000in}}%
\pgfusepath{clip}%
\pgfsetbuttcap%
\pgfsetroundjoin%
\definecolor{currentfill}{rgb}{0.121569,0.466667,0.705882}%
\pgfsetfillcolor{currentfill}%
\pgfsetfillopacity{0.801752}%
\pgfsetlinewidth{1.003750pt}%
\definecolor{currentstroke}{rgb}{0.121569,0.466667,0.705882}%
\pgfsetstrokecolor{currentstroke}%
\pgfsetstrokeopacity{0.801752}%
\pgfsetdash{}{0pt}%
\pgfpathmoveto{\pgfqpoint{2.981172in}{2.354298in}}%
\pgfpathcurveto{\pgfqpoint{2.989408in}{2.354298in}}{\pgfqpoint{2.997309in}{2.357570in}}{\pgfqpoint{3.003132in}{2.363394in}}%
\pgfpathcurveto{\pgfqpoint{3.008956in}{2.369218in}}{\pgfqpoint{3.012229in}{2.377118in}}{\pgfqpoint{3.012229in}{2.385354in}}%
\pgfpathcurveto{\pgfqpoint{3.012229in}{2.393590in}}{\pgfqpoint{3.008956in}{2.401490in}}{\pgfqpoint{3.003132in}{2.407314in}}%
\pgfpathcurveto{\pgfqpoint{2.997309in}{2.413138in}}{\pgfqpoint{2.989408in}{2.416411in}}{\pgfqpoint{2.981172in}{2.416411in}}%
\pgfpathcurveto{\pgfqpoint{2.972936in}{2.416411in}}{\pgfqpoint{2.965036in}{2.413138in}}{\pgfqpoint{2.959212in}{2.407314in}}%
\pgfpathcurveto{\pgfqpoint{2.953388in}{2.401490in}}{\pgfqpoint{2.950116in}{2.393590in}}{\pgfqpoint{2.950116in}{2.385354in}}%
\pgfpathcurveto{\pgfqpoint{2.950116in}{2.377118in}}{\pgfqpoint{2.953388in}{2.369218in}}{\pgfqpoint{2.959212in}{2.363394in}}%
\pgfpathcurveto{\pgfqpoint{2.965036in}{2.357570in}}{\pgfqpoint{2.972936in}{2.354298in}}{\pgfqpoint{2.981172in}{2.354298in}}%
\pgfpathclose%
\pgfusepath{stroke,fill}%
\end{pgfscope}%
\begin{pgfscope}%
\pgfpathrectangle{\pgfqpoint{0.100000in}{0.220728in}}{\pgfqpoint{3.696000in}{3.696000in}}%
\pgfusepath{clip}%
\pgfsetbuttcap%
\pgfsetroundjoin%
\definecolor{currentfill}{rgb}{0.121569,0.466667,0.705882}%
\pgfsetfillcolor{currentfill}%
\pgfsetfillopacity{0.801801}%
\pgfsetlinewidth{1.003750pt}%
\definecolor{currentstroke}{rgb}{0.121569,0.466667,0.705882}%
\pgfsetstrokecolor{currentstroke}%
\pgfsetstrokeopacity{0.801801}%
\pgfsetdash{}{0pt}%
\pgfpathmoveto{\pgfqpoint{1.320780in}{1.985063in}}%
\pgfpathcurveto{\pgfqpoint{1.329016in}{1.985063in}}{\pgfqpoint{1.336916in}{1.988336in}}{\pgfqpoint{1.342740in}{1.994160in}}%
\pgfpathcurveto{\pgfqpoint{1.348564in}{1.999984in}}{\pgfqpoint{1.351836in}{2.007884in}}{\pgfqpoint{1.351836in}{2.016120in}}%
\pgfpathcurveto{\pgfqpoint{1.351836in}{2.024356in}}{\pgfqpoint{1.348564in}{2.032256in}}{\pgfqpoint{1.342740in}{2.038080in}}%
\pgfpathcurveto{\pgfqpoint{1.336916in}{2.043904in}}{\pgfqpoint{1.329016in}{2.047176in}}{\pgfqpoint{1.320780in}{2.047176in}}%
\pgfpathcurveto{\pgfqpoint{1.312544in}{2.047176in}}{\pgfqpoint{1.304644in}{2.043904in}}{\pgfqpoint{1.298820in}{2.038080in}}%
\pgfpathcurveto{\pgfqpoint{1.292996in}{2.032256in}}{\pgfqpoint{1.289723in}{2.024356in}}{\pgfqpoint{1.289723in}{2.016120in}}%
\pgfpathcurveto{\pgfqpoint{1.289723in}{2.007884in}}{\pgfqpoint{1.292996in}{1.999984in}}{\pgfqpoint{1.298820in}{1.994160in}}%
\pgfpathcurveto{\pgfqpoint{1.304644in}{1.988336in}}{\pgfqpoint{1.312544in}{1.985063in}}{\pgfqpoint{1.320780in}{1.985063in}}%
\pgfpathclose%
\pgfusepath{stroke,fill}%
\end{pgfscope}%
\begin{pgfscope}%
\pgfpathrectangle{\pgfqpoint{0.100000in}{0.220728in}}{\pgfqpoint{3.696000in}{3.696000in}}%
\pgfusepath{clip}%
\pgfsetbuttcap%
\pgfsetroundjoin%
\definecolor{currentfill}{rgb}{0.121569,0.466667,0.705882}%
\pgfsetfillcolor{currentfill}%
\pgfsetfillopacity{0.804264}%
\pgfsetlinewidth{1.003750pt}%
\definecolor{currentstroke}{rgb}{0.121569,0.466667,0.705882}%
\pgfsetstrokecolor{currentstroke}%
\pgfsetstrokeopacity{0.804264}%
\pgfsetdash{}{0pt}%
\pgfpathmoveto{\pgfqpoint{2.974544in}{2.343583in}}%
\pgfpathcurveto{\pgfqpoint{2.982780in}{2.343583in}}{\pgfqpoint{2.990680in}{2.346855in}}{\pgfqpoint{2.996504in}{2.352679in}}%
\pgfpathcurveto{\pgfqpoint{3.002328in}{2.358503in}}{\pgfqpoint{3.005600in}{2.366403in}}{\pgfqpoint{3.005600in}{2.374640in}}%
\pgfpathcurveto{\pgfqpoint{3.005600in}{2.382876in}}{\pgfqpoint{3.002328in}{2.390776in}}{\pgfqpoint{2.996504in}{2.396600in}}%
\pgfpathcurveto{\pgfqpoint{2.990680in}{2.402424in}}{\pgfqpoint{2.982780in}{2.405696in}}{\pgfqpoint{2.974544in}{2.405696in}}%
\pgfpathcurveto{\pgfqpoint{2.966307in}{2.405696in}}{\pgfqpoint{2.958407in}{2.402424in}}{\pgfqpoint{2.952583in}{2.396600in}}%
\pgfpathcurveto{\pgfqpoint{2.946760in}{2.390776in}}{\pgfqpoint{2.943487in}{2.382876in}}{\pgfqpoint{2.943487in}{2.374640in}}%
\pgfpathcurveto{\pgfqpoint{2.943487in}{2.366403in}}{\pgfqpoint{2.946760in}{2.358503in}}{\pgfqpoint{2.952583in}{2.352679in}}%
\pgfpathcurveto{\pgfqpoint{2.958407in}{2.346855in}}{\pgfqpoint{2.966307in}{2.343583in}}{\pgfqpoint{2.974544in}{2.343583in}}%
\pgfpathclose%
\pgfusepath{stroke,fill}%
\end{pgfscope}%
\begin{pgfscope}%
\pgfpathrectangle{\pgfqpoint{0.100000in}{0.220728in}}{\pgfqpoint{3.696000in}{3.696000in}}%
\pgfusepath{clip}%
\pgfsetbuttcap%
\pgfsetroundjoin%
\definecolor{currentfill}{rgb}{0.121569,0.466667,0.705882}%
\pgfsetfillcolor{currentfill}%
\pgfsetfillopacity{0.805125}%
\pgfsetlinewidth{1.003750pt}%
\definecolor{currentstroke}{rgb}{0.121569,0.466667,0.705882}%
\pgfsetstrokecolor{currentstroke}%
\pgfsetstrokeopacity{0.805125}%
\pgfsetdash{}{0pt}%
\pgfpathmoveto{\pgfqpoint{1.340080in}{1.972848in}}%
\pgfpathcurveto{\pgfqpoint{1.348316in}{1.972848in}}{\pgfqpoint{1.356216in}{1.976121in}}{\pgfqpoint{1.362040in}{1.981944in}}%
\pgfpathcurveto{\pgfqpoint{1.367864in}{1.987768in}}{\pgfqpoint{1.371136in}{1.995668in}}{\pgfqpoint{1.371136in}{2.003905in}}%
\pgfpathcurveto{\pgfqpoint{1.371136in}{2.012141in}}{\pgfqpoint{1.367864in}{2.020041in}}{\pgfqpoint{1.362040in}{2.025865in}}%
\pgfpathcurveto{\pgfqpoint{1.356216in}{2.031689in}}{\pgfqpoint{1.348316in}{2.034961in}}{\pgfqpoint{1.340080in}{2.034961in}}%
\pgfpathcurveto{\pgfqpoint{1.331843in}{2.034961in}}{\pgfqpoint{1.323943in}{2.031689in}}{\pgfqpoint{1.318119in}{2.025865in}}%
\pgfpathcurveto{\pgfqpoint{1.312295in}{2.020041in}}{\pgfqpoint{1.309023in}{2.012141in}}{\pgfqpoint{1.309023in}{2.003905in}}%
\pgfpathcurveto{\pgfqpoint{1.309023in}{1.995668in}}{\pgfqpoint{1.312295in}{1.987768in}}{\pgfqpoint{1.318119in}{1.981944in}}%
\pgfpathcurveto{\pgfqpoint{1.323943in}{1.976121in}}{\pgfqpoint{1.331843in}{1.972848in}}{\pgfqpoint{1.340080in}{1.972848in}}%
\pgfpathclose%
\pgfusepath{stroke,fill}%
\end{pgfscope}%
\begin{pgfscope}%
\pgfpathrectangle{\pgfqpoint{0.100000in}{0.220728in}}{\pgfqpoint{3.696000in}{3.696000in}}%
\pgfusepath{clip}%
\pgfsetbuttcap%
\pgfsetroundjoin%
\definecolor{currentfill}{rgb}{0.121569,0.466667,0.705882}%
\pgfsetfillcolor{currentfill}%
\pgfsetfillopacity{0.805570}%
\pgfsetlinewidth{1.003750pt}%
\definecolor{currentstroke}{rgb}{0.121569,0.466667,0.705882}%
\pgfsetstrokecolor{currentstroke}%
\pgfsetstrokeopacity{0.805570}%
\pgfsetdash{}{0pt}%
\pgfpathmoveto{\pgfqpoint{2.970775in}{2.337514in}}%
\pgfpathcurveto{\pgfqpoint{2.979011in}{2.337514in}}{\pgfqpoint{2.986911in}{2.340786in}}{\pgfqpoint{2.992735in}{2.346610in}}%
\pgfpathcurveto{\pgfqpoint{2.998559in}{2.352434in}}{\pgfqpoint{3.001832in}{2.360334in}}{\pgfqpoint{3.001832in}{2.368570in}}%
\pgfpathcurveto{\pgfqpoint{3.001832in}{2.376806in}}{\pgfqpoint{2.998559in}{2.384706in}}{\pgfqpoint{2.992735in}{2.390530in}}%
\pgfpathcurveto{\pgfqpoint{2.986911in}{2.396354in}}{\pgfqpoint{2.979011in}{2.399627in}}{\pgfqpoint{2.970775in}{2.399627in}}%
\pgfpathcurveto{\pgfqpoint{2.962539in}{2.399627in}}{\pgfqpoint{2.954639in}{2.396354in}}{\pgfqpoint{2.948815in}{2.390530in}}%
\pgfpathcurveto{\pgfqpoint{2.942991in}{2.384706in}}{\pgfqpoint{2.939719in}{2.376806in}}{\pgfqpoint{2.939719in}{2.368570in}}%
\pgfpathcurveto{\pgfqpoint{2.939719in}{2.360334in}}{\pgfqpoint{2.942991in}{2.352434in}}{\pgfqpoint{2.948815in}{2.346610in}}%
\pgfpathcurveto{\pgfqpoint{2.954639in}{2.340786in}}{\pgfqpoint{2.962539in}{2.337514in}}{\pgfqpoint{2.970775in}{2.337514in}}%
\pgfpathclose%
\pgfusepath{stroke,fill}%
\end{pgfscope}%
\begin{pgfscope}%
\pgfpathrectangle{\pgfqpoint{0.100000in}{0.220728in}}{\pgfqpoint{3.696000in}{3.696000in}}%
\pgfusepath{clip}%
\pgfsetbuttcap%
\pgfsetroundjoin%
\definecolor{currentfill}{rgb}{0.121569,0.466667,0.705882}%
\pgfsetfillcolor{currentfill}%
\pgfsetfillopacity{0.806224}%
\pgfsetlinewidth{1.003750pt}%
\definecolor{currentstroke}{rgb}{0.121569,0.466667,0.705882}%
\pgfsetstrokecolor{currentstroke}%
\pgfsetstrokeopacity{0.806224}%
\pgfsetdash{}{0pt}%
\pgfpathmoveto{\pgfqpoint{2.968488in}{2.334124in}}%
\pgfpathcurveto{\pgfqpoint{2.976724in}{2.334124in}}{\pgfqpoint{2.984624in}{2.337397in}}{\pgfqpoint{2.990448in}{2.343221in}}%
\pgfpathcurveto{\pgfqpoint{2.996272in}{2.349045in}}{\pgfqpoint{2.999544in}{2.356945in}}{\pgfqpoint{2.999544in}{2.365181in}}%
\pgfpathcurveto{\pgfqpoint{2.999544in}{2.373417in}}{\pgfqpoint{2.996272in}{2.381317in}}{\pgfqpoint{2.990448in}{2.387141in}}%
\pgfpathcurveto{\pgfqpoint{2.984624in}{2.392965in}}{\pgfqpoint{2.976724in}{2.396237in}}{\pgfqpoint{2.968488in}{2.396237in}}%
\pgfpathcurveto{\pgfqpoint{2.960252in}{2.396237in}}{\pgfqpoint{2.952351in}{2.392965in}}{\pgfqpoint{2.946528in}{2.387141in}}%
\pgfpathcurveto{\pgfqpoint{2.940704in}{2.381317in}}{\pgfqpoint{2.937431in}{2.373417in}}{\pgfqpoint{2.937431in}{2.365181in}}%
\pgfpathcurveto{\pgfqpoint{2.937431in}{2.356945in}}{\pgfqpoint{2.940704in}{2.349045in}}{\pgfqpoint{2.946528in}{2.343221in}}%
\pgfpathcurveto{\pgfqpoint{2.952351in}{2.337397in}}{\pgfqpoint{2.960252in}{2.334124in}}{\pgfqpoint{2.968488in}{2.334124in}}%
\pgfpathclose%
\pgfusepath{stroke,fill}%
\end{pgfscope}%
\begin{pgfscope}%
\pgfpathrectangle{\pgfqpoint{0.100000in}{0.220728in}}{\pgfqpoint{3.696000in}{3.696000in}}%
\pgfusepath{clip}%
\pgfsetbuttcap%
\pgfsetroundjoin%
\definecolor{currentfill}{rgb}{0.121569,0.466667,0.705882}%
\pgfsetfillcolor{currentfill}%
\pgfsetfillopacity{0.806622}%
\pgfsetlinewidth{1.003750pt}%
\definecolor{currentstroke}{rgb}{0.121569,0.466667,0.705882}%
\pgfsetstrokecolor{currentstroke}%
\pgfsetstrokeopacity{0.806622}%
\pgfsetdash{}{0pt}%
\pgfpathmoveto{\pgfqpoint{2.967442in}{2.332168in}}%
\pgfpathcurveto{\pgfqpoint{2.975679in}{2.332168in}}{\pgfqpoint{2.983579in}{2.335440in}}{\pgfqpoint{2.989403in}{2.341264in}}%
\pgfpathcurveto{\pgfqpoint{2.995227in}{2.347088in}}{\pgfqpoint{2.998499in}{2.354988in}}{\pgfqpoint{2.998499in}{2.363224in}}%
\pgfpathcurveto{\pgfqpoint{2.998499in}{2.371461in}}{\pgfqpoint{2.995227in}{2.379361in}}{\pgfqpoint{2.989403in}{2.385185in}}%
\pgfpathcurveto{\pgfqpoint{2.983579in}{2.391009in}}{\pgfqpoint{2.975679in}{2.394281in}}{\pgfqpoint{2.967442in}{2.394281in}}%
\pgfpathcurveto{\pgfqpoint{2.959206in}{2.394281in}}{\pgfqpoint{2.951306in}{2.391009in}}{\pgfqpoint{2.945482in}{2.385185in}}%
\pgfpathcurveto{\pgfqpoint{2.939658in}{2.379361in}}{\pgfqpoint{2.936386in}{2.371461in}}{\pgfqpoint{2.936386in}{2.363224in}}%
\pgfpathcurveto{\pgfqpoint{2.936386in}{2.354988in}}{\pgfqpoint{2.939658in}{2.347088in}}{\pgfqpoint{2.945482in}{2.341264in}}%
\pgfpathcurveto{\pgfqpoint{2.951306in}{2.335440in}}{\pgfqpoint{2.959206in}{2.332168in}}{\pgfqpoint{2.967442in}{2.332168in}}%
\pgfpathclose%
\pgfusepath{stroke,fill}%
\end{pgfscope}%
\begin{pgfscope}%
\pgfpathrectangle{\pgfqpoint{0.100000in}{0.220728in}}{\pgfqpoint{3.696000in}{3.696000in}}%
\pgfusepath{clip}%
\pgfsetbuttcap%
\pgfsetroundjoin%
\definecolor{currentfill}{rgb}{0.121569,0.466667,0.705882}%
\pgfsetfillcolor{currentfill}%
\pgfsetfillopacity{0.807512}%
\pgfsetlinewidth{1.003750pt}%
\definecolor{currentstroke}{rgb}{0.121569,0.466667,0.705882}%
\pgfsetstrokecolor{currentstroke}%
\pgfsetstrokeopacity{0.807512}%
\pgfsetdash{}{0pt}%
\pgfpathmoveto{\pgfqpoint{2.964401in}{2.328206in}}%
\pgfpathcurveto{\pgfqpoint{2.972638in}{2.328206in}}{\pgfqpoint{2.980538in}{2.331478in}}{\pgfqpoint{2.986362in}{2.337302in}}%
\pgfpathcurveto{\pgfqpoint{2.992185in}{2.343126in}}{\pgfqpoint{2.995458in}{2.351026in}}{\pgfqpoint{2.995458in}{2.359262in}}%
\pgfpathcurveto{\pgfqpoint{2.995458in}{2.367498in}}{\pgfqpoint{2.992185in}{2.375398in}}{\pgfqpoint{2.986362in}{2.381222in}}%
\pgfpathcurveto{\pgfqpoint{2.980538in}{2.387046in}}{\pgfqpoint{2.972638in}{2.390319in}}{\pgfqpoint{2.964401in}{2.390319in}}%
\pgfpathcurveto{\pgfqpoint{2.956165in}{2.390319in}}{\pgfqpoint{2.948265in}{2.387046in}}{\pgfqpoint{2.942441in}{2.381222in}}%
\pgfpathcurveto{\pgfqpoint{2.936617in}{2.375398in}}{\pgfqpoint{2.933345in}{2.367498in}}{\pgfqpoint{2.933345in}{2.359262in}}%
\pgfpathcurveto{\pgfqpoint{2.933345in}{2.351026in}}{\pgfqpoint{2.936617in}{2.343126in}}{\pgfqpoint{2.942441in}{2.337302in}}%
\pgfpathcurveto{\pgfqpoint{2.948265in}{2.331478in}}{\pgfqpoint{2.956165in}{2.328206in}}{\pgfqpoint{2.964401in}{2.328206in}}%
\pgfpathclose%
\pgfusepath{stroke,fill}%
\end{pgfscope}%
\begin{pgfscope}%
\pgfpathrectangle{\pgfqpoint{0.100000in}{0.220728in}}{\pgfqpoint{3.696000in}{3.696000in}}%
\pgfusepath{clip}%
\pgfsetbuttcap%
\pgfsetroundjoin%
\definecolor{currentfill}{rgb}{0.121569,0.466667,0.705882}%
\pgfsetfillcolor{currentfill}%
\pgfsetfillopacity{0.808846}%
\pgfsetlinewidth{1.003750pt}%
\definecolor{currentstroke}{rgb}{0.121569,0.466667,0.705882}%
\pgfsetstrokecolor{currentstroke}%
\pgfsetstrokeopacity{0.808846}%
\pgfsetdash{}{0pt}%
\pgfpathmoveto{\pgfqpoint{2.960964in}{2.321657in}}%
\pgfpathcurveto{\pgfqpoint{2.969200in}{2.321657in}}{\pgfqpoint{2.977100in}{2.324929in}}{\pgfqpoint{2.982924in}{2.330753in}}%
\pgfpathcurveto{\pgfqpoint{2.988748in}{2.336577in}}{\pgfqpoint{2.992021in}{2.344477in}}{\pgfqpoint{2.992021in}{2.352713in}}%
\pgfpathcurveto{\pgfqpoint{2.992021in}{2.360950in}}{\pgfqpoint{2.988748in}{2.368850in}}{\pgfqpoint{2.982924in}{2.374674in}}%
\pgfpathcurveto{\pgfqpoint{2.977100in}{2.380498in}}{\pgfqpoint{2.969200in}{2.383770in}}{\pgfqpoint{2.960964in}{2.383770in}}%
\pgfpathcurveto{\pgfqpoint{2.952728in}{2.383770in}}{\pgfqpoint{2.944828in}{2.380498in}}{\pgfqpoint{2.939004in}{2.374674in}}%
\pgfpathcurveto{\pgfqpoint{2.933180in}{2.368850in}}{\pgfqpoint{2.929908in}{2.360950in}}{\pgfqpoint{2.929908in}{2.352713in}}%
\pgfpathcurveto{\pgfqpoint{2.929908in}{2.344477in}}{\pgfqpoint{2.933180in}{2.336577in}}{\pgfqpoint{2.939004in}{2.330753in}}%
\pgfpathcurveto{\pgfqpoint{2.944828in}{2.324929in}}{\pgfqpoint{2.952728in}{2.321657in}}{\pgfqpoint{2.960964in}{2.321657in}}%
\pgfpathclose%
\pgfusepath{stroke,fill}%
\end{pgfscope}%
\begin{pgfscope}%
\pgfpathrectangle{\pgfqpoint{0.100000in}{0.220728in}}{\pgfqpoint{3.696000in}{3.696000in}}%
\pgfusepath{clip}%
\pgfsetbuttcap%
\pgfsetroundjoin%
\definecolor{currentfill}{rgb}{0.121569,0.466667,0.705882}%
\pgfsetfillcolor{currentfill}%
\pgfsetfillopacity{0.810998}%
\pgfsetlinewidth{1.003750pt}%
\definecolor{currentstroke}{rgb}{0.121569,0.466667,0.705882}%
\pgfsetstrokecolor{currentstroke}%
\pgfsetstrokeopacity{0.810998}%
\pgfsetdash{}{0pt}%
\pgfpathmoveto{\pgfqpoint{2.953597in}{2.311944in}}%
\pgfpathcurveto{\pgfqpoint{2.961833in}{2.311944in}}{\pgfqpoint{2.969733in}{2.315216in}}{\pgfqpoint{2.975557in}{2.321040in}}%
\pgfpathcurveto{\pgfqpoint{2.981381in}{2.326864in}}{\pgfqpoint{2.984653in}{2.334764in}}{\pgfqpoint{2.984653in}{2.343000in}}%
\pgfpathcurveto{\pgfqpoint{2.984653in}{2.351237in}}{\pgfqpoint{2.981381in}{2.359137in}}{\pgfqpoint{2.975557in}{2.364961in}}%
\pgfpathcurveto{\pgfqpoint{2.969733in}{2.370785in}}{\pgfqpoint{2.961833in}{2.374057in}}{\pgfqpoint{2.953597in}{2.374057in}}%
\pgfpathcurveto{\pgfqpoint{2.945360in}{2.374057in}}{\pgfqpoint{2.937460in}{2.370785in}}{\pgfqpoint{2.931636in}{2.364961in}}%
\pgfpathcurveto{\pgfqpoint{2.925812in}{2.359137in}}{\pgfqpoint{2.922540in}{2.351237in}}{\pgfqpoint{2.922540in}{2.343000in}}%
\pgfpathcurveto{\pgfqpoint{2.922540in}{2.334764in}}{\pgfqpoint{2.925812in}{2.326864in}}{\pgfqpoint{2.931636in}{2.321040in}}%
\pgfpathcurveto{\pgfqpoint{2.937460in}{2.315216in}}{\pgfqpoint{2.945360in}{2.311944in}}{\pgfqpoint{2.953597in}{2.311944in}}%
\pgfpathclose%
\pgfusepath{stroke,fill}%
\end{pgfscope}%
\begin{pgfscope}%
\pgfpathrectangle{\pgfqpoint{0.100000in}{0.220728in}}{\pgfqpoint{3.696000in}{3.696000in}}%
\pgfusepath{clip}%
\pgfsetbuttcap%
\pgfsetroundjoin%
\definecolor{currentfill}{rgb}{0.121569,0.466667,0.705882}%
\pgfsetfillcolor{currentfill}%
\pgfsetfillopacity{0.811674}%
\pgfsetlinewidth{1.003750pt}%
\definecolor{currentstroke}{rgb}{0.121569,0.466667,0.705882}%
\pgfsetstrokecolor{currentstroke}%
\pgfsetstrokeopacity{0.811674}%
\pgfsetdash{}{0pt}%
\pgfpathmoveto{\pgfqpoint{1.373675in}{1.949428in}}%
\pgfpathcurveto{\pgfqpoint{1.381912in}{1.949428in}}{\pgfqpoint{1.389812in}{1.952700in}}{\pgfqpoint{1.395636in}{1.958524in}}%
\pgfpathcurveto{\pgfqpoint{1.401459in}{1.964348in}}{\pgfqpoint{1.404732in}{1.972248in}}{\pgfqpoint{1.404732in}{1.980484in}}%
\pgfpathcurveto{\pgfqpoint{1.404732in}{1.988721in}}{\pgfqpoint{1.401459in}{1.996621in}}{\pgfqpoint{1.395636in}{2.002445in}}%
\pgfpathcurveto{\pgfqpoint{1.389812in}{2.008268in}}{\pgfqpoint{1.381912in}{2.011541in}}{\pgfqpoint{1.373675in}{2.011541in}}%
\pgfpathcurveto{\pgfqpoint{1.365439in}{2.011541in}}{\pgfqpoint{1.357539in}{2.008268in}}{\pgfqpoint{1.351715in}{2.002445in}}%
\pgfpathcurveto{\pgfqpoint{1.345891in}{1.996621in}}{\pgfqpoint{1.342619in}{1.988721in}}{\pgfqpoint{1.342619in}{1.980484in}}%
\pgfpathcurveto{\pgfqpoint{1.342619in}{1.972248in}}{\pgfqpoint{1.345891in}{1.964348in}}{\pgfqpoint{1.351715in}{1.958524in}}%
\pgfpathcurveto{\pgfqpoint{1.357539in}{1.952700in}}{\pgfqpoint{1.365439in}{1.949428in}}{\pgfqpoint{1.373675in}{1.949428in}}%
\pgfpathclose%
\pgfusepath{stroke,fill}%
\end{pgfscope}%
\begin{pgfscope}%
\pgfpathrectangle{\pgfqpoint{0.100000in}{0.220728in}}{\pgfqpoint{3.696000in}{3.696000in}}%
\pgfusepath{clip}%
\pgfsetbuttcap%
\pgfsetroundjoin%
\definecolor{currentfill}{rgb}{0.121569,0.466667,0.705882}%
\pgfsetfillcolor{currentfill}%
\pgfsetfillopacity{0.812179}%
\pgfsetlinewidth{1.003750pt}%
\definecolor{currentstroke}{rgb}{0.121569,0.466667,0.705882}%
\pgfsetstrokecolor{currentstroke}%
\pgfsetstrokeopacity{0.812179}%
\pgfsetdash{}{0pt}%
\pgfpathmoveto{\pgfqpoint{2.950296in}{2.305589in}}%
\pgfpathcurveto{\pgfqpoint{2.958533in}{2.305589in}}{\pgfqpoint{2.966433in}{2.308862in}}{\pgfqpoint{2.972257in}{2.314686in}}%
\pgfpathcurveto{\pgfqpoint{2.978080in}{2.320510in}}{\pgfqpoint{2.981353in}{2.328410in}}{\pgfqpoint{2.981353in}{2.336646in}}%
\pgfpathcurveto{\pgfqpoint{2.981353in}{2.344882in}}{\pgfqpoint{2.978080in}{2.352782in}}{\pgfqpoint{2.972257in}{2.358606in}}%
\pgfpathcurveto{\pgfqpoint{2.966433in}{2.364430in}}{\pgfqpoint{2.958533in}{2.367702in}}{\pgfqpoint{2.950296in}{2.367702in}}%
\pgfpathcurveto{\pgfqpoint{2.942060in}{2.367702in}}{\pgfqpoint{2.934160in}{2.364430in}}{\pgfqpoint{2.928336in}{2.358606in}}%
\pgfpathcurveto{\pgfqpoint{2.922512in}{2.352782in}}{\pgfqpoint{2.919240in}{2.344882in}}{\pgfqpoint{2.919240in}{2.336646in}}%
\pgfpathcurveto{\pgfqpoint{2.919240in}{2.328410in}}{\pgfqpoint{2.922512in}{2.320510in}}{\pgfqpoint{2.928336in}{2.314686in}}%
\pgfpathcurveto{\pgfqpoint{2.934160in}{2.308862in}}{\pgfqpoint{2.942060in}{2.305589in}}{\pgfqpoint{2.950296in}{2.305589in}}%
\pgfpathclose%
\pgfusepath{stroke,fill}%
\end{pgfscope}%
\begin{pgfscope}%
\pgfpathrectangle{\pgfqpoint{0.100000in}{0.220728in}}{\pgfqpoint{3.696000in}{3.696000in}}%
\pgfusepath{clip}%
\pgfsetbuttcap%
\pgfsetroundjoin%
\definecolor{currentfill}{rgb}{0.121569,0.466667,0.705882}%
\pgfsetfillcolor{currentfill}%
\pgfsetfillopacity{0.813875}%
\pgfsetlinewidth{1.003750pt}%
\definecolor{currentstroke}{rgb}{0.121569,0.466667,0.705882}%
\pgfsetstrokecolor{currentstroke}%
\pgfsetstrokeopacity{0.813875}%
\pgfsetdash{}{0pt}%
\pgfpathmoveto{\pgfqpoint{2.944005in}{2.296623in}}%
\pgfpathcurveto{\pgfqpoint{2.952241in}{2.296623in}}{\pgfqpoint{2.960141in}{2.299895in}}{\pgfqpoint{2.965965in}{2.305719in}}%
\pgfpathcurveto{\pgfqpoint{2.971789in}{2.311543in}}{\pgfqpoint{2.975061in}{2.319443in}}{\pgfqpoint{2.975061in}{2.327680in}}%
\pgfpathcurveto{\pgfqpoint{2.975061in}{2.335916in}}{\pgfqpoint{2.971789in}{2.343816in}}{\pgfqpoint{2.965965in}{2.349640in}}%
\pgfpathcurveto{\pgfqpoint{2.960141in}{2.355464in}}{\pgfqpoint{2.952241in}{2.358736in}}{\pgfqpoint{2.944005in}{2.358736in}}%
\pgfpathcurveto{\pgfqpoint{2.935769in}{2.358736in}}{\pgfqpoint{2.927868in}{2.355464in}}{\pgfqpoint{2.922045in}{2.349640in}}%
\pgfpathcurveto{\pgfqpoint{2.916221in}{2.343816in}}{\pgfqpoint{2.912948in}{2.335916in}}{\pgfqpoint{2.912948in}{2.327680in}}%
\pgfpathcurveto{\pgfqpoint{2.912948in}{2.319443in}}{\pgfqpoint{2.916221in}{2.311543in}}{\pgfqpoint{2.922045in}{2.305719in}}%
\pgfpathcurveto{\pgfqpoint{2.927868in}{2.299895in}}{\pgfqpoint{2.935769in}{2.296623in}}{\pgfqpoint{2.944005in}{2.296623in}}%
\pgfpathclose%
\pgfusepath{stroke,fill}%
\end{pgfscope}%
\begin{pgfscope}%
\pgfpathrectangle{\pgfqpoint{0.100000in}{0.220728in}}{\pgfqpoint{3.696000in}{3.696000in}}%
\pgfusepath{clip}%
\pgfsetbuttcap%
\pgfsetroundjoin%
\definecolor{currentfill}{rgb}{0.121569,0.466667,0.705882}%
\pgfsetfillcolor{currentfill}%
\pgfsetfillopacity{0.815036}%
\pgfsetlinewidth{1.003750pt}%
\definecolor{currentstroke}{rgb}{0.121569,0.466667,0.705882}%
\pgfsetstrokecolor{currentstroke}%
\pgfsetstrokeopacity{0.815036}%
\pgfsetdash{}{0pt}%
\pgfpathmoveto{\pgfqpoint{2.941197in}{2.291954in}}%
\pgfpathcurveto{\pgfqpoint{2.949433in}{2.291954in}}{\pgfqpoint{2.957333in}{2.295227in}}{\pgfqpoint{2.963157in}{2.301051in}}%
\pgfpathcurveto{\pgfqpoint{2.968981in}{2.306875in}}{\pgfqpoint{2.972253in}{2.314775in}}{\pgfqpoint{2.972253in}{2.323011in}}%
\pgfpathcurveto{\pgfqpoint{2.972253in}{2.331247in}}{\pgfqpoint{2.968981in}{2.339147in}}{\pgfqpoint{2.963157in}{2.344971in}}%
\pgfpathcurveto{\pgfqpoint{2.957333in}{2.350795in}}{\pgfqpoint{2.949433in}{2.354067in}}{\pgfqpoint{2.941197in}{2.354067in}}%
\pgfpathcurveto{\pgfqpoint{2.932961in}{2.354067in}}{\pgfqpoint{2.925061in}{2.350795in}}{\pgfqpoint{2.919237in}{2.344971in}}%
\pgfpathcurveto{\pgfqpoint{2.913413in}{2.339147in}}{\pgfqpoint{2.910140in}{2.331247in}}{\pgfqpoint{2.910140in}{2.323011in}}%
\pgfpathcurveto{\pgfqpoint{2.910140in}{2.314775in}}{\pgfqpoint{2.913413in}{2.306875in}}{\pgfqpoint{2.919237in}{2.301051in}}%
\pgfpathcurveto{\pgfqpoint{2.925061in}{2.295227in}}{\pgfqpoint{2.932961in}{2.291954in}}{\pgfqpoint{2.941197in}{2.291954in}}%
\pgfpathclose%
\pgfusepath{stroke,fill}%
\end{pgfscope}%
\begin{pgfscope}%
\pgfpathrectangle{\pgfqpoint{0.100000in}{0.220728in}}{\pgfqpoint{3.696000in}{3.696000in}}%
\pgfusepath{clip}%
\pgfsetbuttcap%
\pgfsetroundjoin%
\definecolor{currentfill}{rgb}{0.121569,0.466667,0.705882}%
\pgfsetfillcolor{currentfill}%
\pgfsetfillopacity{0.815619}%
\pgfsetlinewidth{1.003750pt}%
\definecolor{currentstroke}{rgb}{0.121569,0.466667,0.705882}%
\pgfsetstrokecolor{currentstroke}%
\pgfsetstrokeopacity{0.815619}%
\pgfsetdash{}{0pt}%
\pgfpathmoveto{\pgfqpoint{2.939589in}{2.289178in}}%
\pgfpathcurveto{\pgfqpoint{2.947825in}{2.289178in}}{\pgfqpoint{2.955725in}{2.292450in}}{\pgfqpoint{2.961549in}{2.298274in}}%
\pgfpathcurveto{\pgfqpoint{2.967373in}{2.304098in}}{\pgfqpoint{2.970646in}{2.311998in}}{\pgfqpoint{2.970646in}{2.320234in}}%
\pgfpathcurveto{\pgfqpoint{2.970646in}{2.328470in}}{\pgfqpoint{2.967373in}{2.336370in}}{\pgfqpoint{2.961549in}{2.342194in}}%
\pgfpathcurveto{\pgfqpoint{2.955725in}{2.348018in}}{\pgfqpoint{2.947825in}{2.351291in}}{\pgfqpoint{2.939589in}{2.351291in}}%
\pgfpathcurveto{\pgfqpoint{2.931353in}{2.351291in}}{\pgfqpoint{2.923453in}{2.348018in}}{\pgfqpoint{2.917629in}{2.342194in}}%
\pgfpathcurveto{\pgfqpoint{2.911805in}{2.336370in}}{\pgfqpoint{2.908533in}{2.328470in}}{\pgfqpoint{2.908533in}{2.320234in}}%
\pgfpathcurveto{\pgfqpoint{2.908533in}{2.311998in}}{\pgfqpoint{2.911805in}{2.304098in}}{\pgfqpoint{2.917629in}{2.298274in}}%
\pgfpathcurveto{\pgfqpoint{2.923453in}{2.292450in}}{\pgfqpoint{2.931353in}{2.289178in}}{\pgfqpoint{2.939589in}{2.289178in}}%
\pgfpathclose%
\pgfusepath{stroke,fill}%
\end{pgfscope}%
\begin{pgfscope}%
\pgfpathrectangle{\pgfqpoint{0.100000in}{0.220728in}}{\pgfqpoint{3.696000in}{3.696000in}}%
\pgfusepath{clip}%
\pgfsetbuttcap%
\pgfsetroundjoin%
\definecolor{currentfill}{rgb}{0.121569,0.466667,0.705882}%
\pgfsetfillcolor{currentfill}%
\pgfsetfillopacity{0.815931}%
\pgfsetlinewidth{1.003750pt}%
\definecolor{currentstroke}{rgb}{0.121569,0.466667,0.705882}%
\pgfsetstrokecolor{currentstroke}%
\pgfsetstrokeopacity{0.815931}%
\pgfsetdash{}{0pt}%
\pgfpathmoveto{\pgfqpoint{2.938626in}{2.287716in}}%
\pgfpathcurveto{\pgfqpoint{2.946862in}{2.287716in}}{\pgfqpoint{2.954762in}{2.290989in}}{\pgfqpoint{2.960586in}{2.296813in}}%
\pgfpathcurveto{\pgfqpoint{2.966410in}{2.302636in}}{\pgfqpoint{2.969682in}{2.310537in}}{\pgfqpoint{2.969682in}{2.318773in}}%
\pgfpathcurveto{\pgfqpoint{2.969682in}{2.327009in}}{\pgfqpoint{2.966410in}{2.334909in}}{\pgfqpoint{2.960586in}{2.340733in}}%
\pgfpathcurveto{\pgfqpoint{2.954762in}{2.346557in}}{\pgfqpoint{2.946862in}{2.349829in}}{\pgfqpoint{2.938626in}{2.349829in}}%
\pgfpathcurveto{\pgfqpoint{2.930390in}{2.349829in}}{\pgfqpoint{2.922490in}{2.346557in}}{\pgfqpoint{2.916666in}{2.340733in}}%
\pgfpathcurveto{\pgfqpoint{2.910842in}{2.334909in}}{\pgfqpoint{2.907569in}{2.327009in}}{\pgfqpoint{2.907569in}{2.318773in}}%
\pgfpathcurveto{\pgfqpoint{2.907569in}{2.310537in}}{\pgfqpoint{2.910842in}{2.302636in}}{\pgfqpoint{2.916666in}{2.296813in}}%
\pgfpathcurveto{\pgfqpoint{2.922490in}{2.290989in}}{\pgfqpoint{2.930390in}{2.287716in}}{\pgfqpoint{2.938626in}{2.287716in}}%
\pgfpathclose%
\pgfusepath{stroke,fill}%
\end{pgfscope}%
\begin{pgfscope}%
\pgfpathrectangle{\pgfqpoint{0.100000in}{0.220728in}}{\pgfqpoint{3.696000in}{3.696000in}}%
\pgfusepath{clip}%
\pgfsetbuttcap%
\pgfsetroundjoin%
\definecolor{currentfill}{rgb}{0.121569,0.466667,0.705882}%
\pgfsetfillcolor{currentfill}%
\pgfsetfillopacity{0.816109}%
\pgfsetlinewidth{1.003750pt}%
\definecolor{currentstroke}{rgb}{0.121569,0.466667,0.705882}%
\pgfsetstrokecolor{currentstroke}%
\pgfsetstrokeopacity{0.816109}%
\pgfsetdash{}{0pt}%
\pgfpathmoveto{\pgfqpoint{2.938130in}{2.286902in}}%
\pgfpathcurveto{\pgfqpoint{2.946366in}{2.286902in}}{\pgfqpoint{2.954266in}{2.290174in}}{\pgfqpoint{2.960090in}{2.295998in}}%
\pgfpathcurveto{\pgfqpoint{2.965914in}{2.301822in}}{\pgfqpoint{2.969186in}{2.309722in}}{\pgfqpoint{2.969186in}{2.317958in}}%
\pgfpathcurveto{\pgfqpoint{2.969186in}{2.326194in}}{\pgfqpoint{2.965914in}{2.334094in}}{\pgfqpoint{2.960090in}{2.339918in}}%
\pgfpathcurveto{\pgfqpoint{2.954266in}{2.345742in}}{\pgfqpoint{2.946366in}{2.349015in}}{\pgfqpoint{2.938130in}{2.349015in}}%
\pgfpathcurveto{\pgfqpoint{2.929894in}{2.349015in}}{\pgfqpoint{2.921994in}{2.345742in}}{\pgfqpoint{2.916170in}{2.339918in}}%
\pgfpathcurveto{\pgfqpoint{2.910346in}{2.334094in}}{\pgfqpoint{2.907073in}{2.326194in}}{\pgfqpoint{2.907073in}{2.317958in}}%
\pgfpathcurveto{\pgfqpoint{2.907073in}{2.309722in}}{\pgfqpoint{2.910346in}{2.301822in}}{\pgfqpoint{2.916170in}{2.295998in}}%
\pgfpathcurveto{\pgfqpoint{2.921994in}{2.290174in}}{\pgfqpoint{2.929894in}{2.286902in}}{\pgfqpoint{2.938130in}{2.286902in}}%
\pgfpathclose%
\pgfusepath{stroke,fill}%
\end{pgfscope}%
\begin{pgfscope}%
\pgfpathrectangle{\pgfqpoint{0.100000in}{0.220728in}}{\pgfqpoint{3.696000in}{3.696000in}}%
\pgfusepath{clip}%
\pgfsetbuttcap%
\pgfsetroundjoin%
\definecolor{currentfill}{rgb}{0.121569,0.466667,0.705882}%
\pgfsetfillcolor{currentfill}%
\pgfsetfillopacity{0.816202}%
\pgfsetlinewidth{1.003750pt}%
\definecolor{currentstroke}{rgb}{0.121569,0.466667,0.705882}%
\pgfsetstrokecolor{currentstroke}%
\pgfsetstrokeopacity{0.816202}%
\pgfsetdash{}{0pt}%
\pgfpathmoveto{\pgfqpoint{2.937801in}{2.286503in}}%
\pgfpathcurveto{\pgfqpoint{2.946037in}{2.286503in}}{\pgfqpoint{2.953937in}{2.289776in}}{\pgfqpoint{2.959761in}{2.295600in}}%
\pgfpathcurveto{\pgfqpoint{2.965585in}{2.301424in}}{\pgfqpoint{2.968857in}{2.309324in}}{\pgfqpoint{2.968857in}{2.317560in}}%
\pgfpathcurveto{\pgfqpoint{2.968857in}{2.325796in}}{\pgfqpoint{2.965585in}{2.333696in}}{\pgfqpoint{2.959761in}{2.339520in}}%
\pgfpathcurveto{\pgfqpoint{2.953937in}{2.345344in}}{\pgfqpoint{2.946037in}{2.348616in}}{\pgfqpoint{2.937801in}{2.348616in}}%
\pgfpathcurveto{\pgfqpoint{2.929564in}{2.348616in}}{\pgfqpoint{2.921664in}{2.345344in}}{\pgfqpoint{2.915840in}{2.339520in}}%
\pgfpathcurveto{\pgfqpoint{2.910016in}{2.333696in}}{\pgfqpoint{2.906744in}{2.325796in}}{\pgfqpoint{2.906744in}{2.317560in}}%
\pgfpathcurveto{\pgfqpoint{2.906744in}{2.309324in}}{\pgfqpoint{2.910016in}{2.301424in}}{\pgfqpoint{2.915840in}{2.295600in}}%
\pgfpathcurveto{\pgfqpoint{2.921664in}{2.289776in}}{\pgfqpoint{2.929564in}{2.286503in}}{\pgfqpoint{2.937801in}{2.286503in}}%
\pgfpathclose%
\pgfusepath{stroke,fill}%
\end{pgfscope}%
\begin{pgfscope}%
\pgfpathrectangle{\pgfqpoint{0.100000in}{0.220728in}}{\pgfqpoint{3.696000in}{3.696000in}}%
\pgfusepath{clip}%
\pgfsetbuttcap%
\pgfsetroundjoin%
\definecolor{currentfill}{rgb}{0.121569,0.466667,0.705882}%
\pgfsetfillcolor{currentfill}%
\pgfsetfillopacity{0.816826}%
\pgfsetlinewidth{1.003750pt}%
\definecolor{currentstroke}{rgb}{0.121569,0.466667,0.705882}%
\pgfsetstrokecolor{currentstroke}%
\pgfsetstrokeopacity{0.816826}%
\pgfsetdash{}{0pt}%
\pgfpathmoveto{\pgfqpoint{2.936486in}{2.283958in}}%
\pgfpathcurveto{\pgfqpoint{2.944723in}{2.283958in}}{\pgfqpoint{2.952623in}{2.287231in}}{\pgfqpoint{2.958447in}{2.293055in}}%
\pgfpathcurveto{\pgfqpoint{2.964270in}{2.298879in}}{\pgfqpoint{2.967543in}{2.306779in}}{\pgfqpoint{2.967543in}{2.315015in}}%
\pgfpathcurveto{\pgfqpoint{2.967543in}{2.323251in}}{\pgfqpoint{2.964270in}{2.331151in}}{\pgfqpoint{2.958447in}{2.336975in}}%
\pgfpathcurveto{\pgfqpoint{2.952623in}{2.342799in}}{\pgfqpoint{2.944723in}{2.346071in}}{\pgfqpoint{2.936486in}{2.346071in}}%
\pgfpathcurveto{\pgfqpoint{2.928250in}{2.346071in}}{\pgfqpoint{2.920350in}{2.342799in}}{\pgfqpoint{2.914526in}{2.336975in}}%
\pgfpathcurveto{\pgfqpoint{2.908702in}{2.331151in}}{\pgfqpoint{2.905430in}{2.323251in}}{\pgfqpoint{2.905430in}{2.315015in}}%
\pgfpathcurveto{\pgfqpoint{2.905430in}{2.306779in}}{\pgfqpoint{2.908702in}{2.298879in}}{\pgfqpoint{2.914526in}{2.293055in}}%
\pgfpathcurveto{\pgfqpoint{2.920350in}{2.287231in}}{\pgfqpoint{2.928250in}{2.283958in}}{\pgfqpoint{2.936486in}{2.283958in}}%
\pgfpathclose%
\pgfusepath{stroke,fill}%
\end{pgfscope}%
\begin{pgfscope}%
\pgfpathrectangle{\pgfqpoint{0.100000in}{0.220728in}}{\pgfqpoint{3.696000in}{3.696000in}}%
\pgfusepath{clip}%
\pgfsetbuttcap%
\pgfsetroundjoin%
\definecolor{currentfill}{rgb}{0.121569,0.466667,0.705882}%
\pgfsetfillcolor{currentfill}%
\pgfsetfillopacity{0.817653}%
\pgfsetlinewidth{1.003750pt}%
\definecolor{currentstroke}{rgb}{0.121569,0.466667,0.705882}%
\pgfsetstrokecolor{currentstroke}%
\pgfsetstrokeopacity{0.817653}%
\pgfsetdash{}{0pt}%
\pgfpathmoveto{\pgfqpoint{1.407366in}{1.930617in}}%
\pgfpathcurveto{\pgfqpoint{1.415602in}{1.930617in}}{\pgfqpoint{1.423502in}{1.933889in}}{\pgfqpoint{1.429326in}{1.939713in}}%
\pgfpathcurveto{\pgfqpoint{1.435150in}{1.945537in}}{\pgfqpoint{1.438423in}{1.953437in}}{\pgfqpoint{1.438423in}{1.961673in}}%
\pgfpathcurveto{\pgfqpoint{1.438423in}{1.969909in}}{\pgfqpoint{1.435150in}{1.977809in}}{\pgfqpoint{1.429326in}{1.983633in}}%
\pgfpathcurveto{\pgfqpoint{1.423502in}{1.989457in}}{\pgfqpoint{1.415602in}{1.992730in}}{\pgfqpoint{1.407366in}{1.992730in}}%
\pgfpathcurveto{\pgfqpoint{1.399130in}{1.992730in}}{\pgfqpoint{1.391230in}{1.989457in}}{\pgfqpoint{1.385406in}{1.983633in}}%
\pgfpathcurveto{\pgfqpoint{1.379582in}{1.977809in}}{\pgfqpoint{1.376310in}{1.969909in}}{\pgfqpoint{1.376310in}{1.961673in}}%
\pgfpathcurveto{\pgfqpoint{1.376310in}{1.953437in}}{\pgfqpoint{1.379582in}{1.945537in}}{\pgfqpoint{1.385406in}{1.939713in}}%
\pgfpathcurveto{\pgfqpoint{1.391230in}{1.933889in}}{\pgfqpoint{1.399130in}{1.930617in}}{\pgfqpoint{1.407366in}{1.930617in}}%
\pgfpathclose%
\pgfusepath{stroke,fill}%
\end{pgfscope}%
\begin{pgfscope}%
\pgfpathrectangle{\pgfqpoint{0.100000in}{0.220728in}}{\pgfqpoint{3.696000in}{3.696000in}}%
\pgfusepath{clip}%
\pgfsetbuttcap%
\pgfsetroundjoin%
\definecolor{currentfill}{rgb}{0.121569,0.466667,0.705882}%
\pgfsetfillcolor{currentfill}%
\pgfsetfillopacity{0.817976}%
\pgfsetlinewidth{1.003750pt}%
\definecolor{currentstroke}{rgb}{0.121569,0.466667,0.705882}%
\pgfsetstrokecolor{currentstroke}%
\pgfsetstrokeopacity{0.817976}%
\pgfsetdash{}{0pt}%
\pgfpathmoveto{\pgfqpoint{2.932053in}{2.278612in}}%
\pgfpathcurveto{\pgfqpoint{2.940289in}{2.278612in}}{\pgfqpoint{2.948189in}{2.281885in}}{\pgfqpoint{2.954013in}{2.287709in}}%
\pgfpathcurveto{\pgfqpoint{2.959837in}{2.293532in}}{\pgfqpoint{2.963110in}{2.301433in}}{\pgfqpoint{2.963110in}{2.309669in}}%
\pgfpathcurveto{\pgfqpoint{2.963110in}{2.317905in}}{\pgfqpoint{2.959837in}{2.325805in}}{\pgfqpoint{2.954013in}{2.331629in}}%
\pgfpathcurveto{\pgfqpoint{2.948189in}{2.337453in}}{\pgfqpoint{2.940289in}{2.340725in}}{\pgfqpoint{2.932053in}{2.340725in}}%
\pgfpathcurveto{\pgfqpoint{2.923817in}{2.340725in}}{\pgfqpoint{2.915917in}{2.337453in}}{\pgfqpoint{2.910093in}{2.331629in}}%
\pgfpathcurveto{\pgfqpoint{2.904269in}{2.325805in}}{\pgfqpoint{2.900997in}{2.317905in}}{\pgfqpoint{2.900997in}{2.309669in}}%
\pgfpathcurveto{\pgfqpoint{2.900997in}{2.301433in}}{\pgfqpoint{2.904269in}{2.293532in}}{\pgfqpoint{2.910093in}{2.287709in}}%
\pgfpathcurveto{\pgfqpoint{2.915917in}{2.281885in}}{\pgfqpoint{2.923817in}{2.278612in}}{\pgfqpoint{2.932053in}{2.278612in}}%
\pgfpathclose%
\pgfusepath{stroke,fill}%
\end{pgfscope}%
\begin{pgfscope}%
\pgfpathrectangle{\pgfqpoint{0.100000in}{0.220728in}}{\pgfqpoint{3.696000in}{3.696000in}}%
\pgfusepath{clip}%
\pgfsetbuttcap%
\pgfsetroundjoin%
\definecolor{currentfill}{rgb}{0.121569,0.466667,0.705882}%
\pgfsetfillcolor{currentfill}%
\pgfsetfillopacity{0.820459}%
\pgfsetlinewidth{1.003750pt}%
\definecolor{currentstroke}{rgb}{0.121569,0.466667,0.705882}%
\pgfsetstrokecolor{currentstroke}%
\pgfsetstrokeopacity{0.820459}%
\pgfsetdash{}{0pt}%
\pgfpathmoveto{\pgfqpoint{2.926164in}{2.266973in}}%
\pgfpathcurveto{\pgfqpoint{2.934400in}{2.266973in}}{\pgfqpoint{2.942300in}{2.270245in}}{\pgfqpoint{2.948124in}{2.276069in}}%
\pgfpathcurveto{\pgfqpoint{2.953948in}{2.281893in}}{\pgfqpoint{2.957220in}{2.289793in}}{\pgfqpoint{2.957220in}{2.298029in}}%
\pgfpathcurveto{\pgfqpoint{2.957220in}{2.306266in}}{\pgfqpoint{2.953948in}{2.314166in}}{\pgfqpoint{2.948124in}{2.319990in}}%
\pgfpathcurveto{\pgfqpoint{2.942300in}{2.325814in}}{\pgfqpoint{2.934400in}{2.329086in}}{\pgfqpoint{2.926164in}{2.329086in}}%
\pgfpathcurveto{\pgfqpoint{2.917927in}{2.329086in}}{\pgfqpoint{2.910027in}{2.325814in}}{\pgfqpoint{2.904203in}{2.319990in}}%
\pgfpathcurveto{\pgfqpoint{2.898379in}{2.314166in}}{\pgfqpoint{2.895107in}{2.306266in}}{\pgfqpoint{2.895107in}{2.298029in}}%
\pgfpathcurveto{\pgfqpoint{2.895107in}{2.289793in}}{\pgfqpoint{2.898379in}{2.281893in}}{\pgfqpoint{2.904203in}{2.276069in}}%
\pgfpathcurveto{\pgfqpoint{2.910027in}{2.270245in}}{\pgfqpoint{2.917927in}{2.266973in}}{\pgfqpoint{2.926164in}{2.266973in}}%
\pgfpathclose%
\pgfusepath{stroke,fill}%
\end{pgfscope}%
\begin{pgfscope}%
\pgfpathrectangle{\pgfqpoint{0.100000in}{0.220728in}}{\pgfqpoint{3.696000in}{3.696000in}}%
\pgfusepath{clip}%
\pgfsetbuttcap%
\pgfsetroundjoin%
\definecolor{currentfill}{rgb}{0.121569,0.466667,0.705882}%
\pgfsetfillcolor{currentfill}%
\pgfsetfillopacity{0.822390}%
\pgfsetlinewidth{1.003750pt}%
\definecolor{currentstroke}{rgb}{0.121569,0.466667,0.705882}%
\pgfsetstrokecolor{currentstroke}%
\pgfsetstrokeopacity{0.822390}%
\pgfsetdash{}{0pt}%
\pgfpathmoveto{\pgfqpoint{1.436953in}{1.910253in}}%
\pgfpathcurveto{\pgfqpoint{1.445189in}{1.910253in}}{\pgfqpoint{1.453089in}{1.913525in}}{\pgfqpoint{1.458913in}{1.919349in}}%
\pgfpathcurveto{\pgfqpoint{1.464737in}{1.925173in}}{\pgfqpoint{1.468009in}{1.933073in}}{\pgfqpoint{1.468009in}{1.941309in}}%
\pgfpathcurveto{\pgfqpoint{1.468009in}{1.949545in}}{\pgfqpoint{1.464737in}{1.957445in}}{\pgfqpoint{1.458913in}{1.963269in}}%
\pgfpathcurveto{\pgfqpoint{1.453089in}{1.969093in}}{\pgfqpoint{1.445189in}{1.972366in}}{\pgfqpoint{1.436953in}{1.972366in}}%
\pgfpathcurveto{\pgfqpoint{1.428716in}{1.972366in}}{\pgfqpoint{1.420816in}{1.969093in}}{\pgfqpoint{1.414992in}{1.963269in}}%
\pgfpathcurveto{\pgfqpoint{1.409168in}{1.957445in}}{\pgfqpoint{1.405896in}{1.949545in}}{\pgfqpoint{1.405896in}{1.941309in}}%
\pgfpathcurveto{\pgfqpoint{1.405896in}{1.933073in}}{\pgfqpoint{1.409168in}{1.925173in}}{\pgfqpoint{1.414992in}{1.919349in}}%
\pgfpathcurveto{\pgfqpoint{1.420816in}{1.913525in}}{\pgfqpoint{1.428716in}{1.910253in}}{\pgfqpoint{1.436953in}{1.910253in}}%
\pgfpathclose%
\pgfusepath{stroke,fill}%
\end{pgfscope}%
\begin{pgfscope}%
\pgfpathrectangle{\pgfqpoint{0.100000in}{0.220728in}}{\pgfqpoint{3.696000in}{3.696000in}}%
\pgfusepath{clip}%
\pgfsetbuttcap%
\pgfsetroundjoin%
\definecolor{currentfill}{rgb}{0.121569,0.466667,0.705882}%
\pgfsetfillcolor{currentfill}%
\pgfsetfillopacity{0.823732}%
\pgfsetlinewidth{1.003750pt}%
\definecolor{currentstroke}{rgb}{0.121569,0.466667,0.705882}%
\pgfsetstrokecolor{currentstroke}%
\pgfsetstrokeopacity{0.823732}%
\pgfsetdash{}{0pt}%
\pgfpathmoveto{\pgfqpoint{2.914043in}{2.251229in}}%
\pgfpathcurveto{\pgfqpoint{2.922279in}{2.251229in}}{\pgfqpoint{2.930179in}{2.254501in}}{\pgfqpoint{2.936003in}{2.260325in}}%
\pgfpathcurveto{\pgfqpoint{2.941827in}{2.266149in}}{\pgfqpoint{2.945099in}{2.274049in}}{\pgfqpoint{2.945099in}{2.282286in}}%
\pgfpathcurveto{\pgfqpoint{2.945099in}{2.290522in}}{\pgfqpoint{2.941827in}{2.298422in}}{\pgfqpoint{2.936003in}{2.304246in}}%
\pgfpathcurveto{\pgfqpoint{2.930179in}{2.310070in}}{\pgfqpoint{2.922279in}{2.313342in}}{\pgfqpoint{2.914043in}{2.313342in}}%
\pgfpathcurveto{\pgfqpoint{2.905806in}{2.313342in}}{\pgfqpoint{2.897906in}{2.310070in}}{\pgfqpoint{2.892082in}{2.304246in}}%
\pgfpathcurveto{\pgfqpoint{2.886259in}{2.298422in}}{\pgfqpoint{2.882986in}{2.290522in}}{\pgfqpoint{2.882986in}{2.282286in}}%
\pgfpathcurveto{\pgfqpoint{2.882986in}{2.274049in}}{\pgfqpoint{2.886259in}{2.266149in}}{\pgfqpoint{2.892082in}{2.260325in}}%
\pgfpathcurveto{\pgfqpoint{2.897906in}{2.254501in}}{\pgfqpoint{2.905806in}{2.251229in}}{\pgfqpoint{2.914043in}{2.251229in}}%
\pgfpathclose%
\pgfusepath{stroke,fill}%
\end{pgfscope}%
\begin{pgfscope}%
\pgfpathrectangle{\pgfqpoint{0.100000in}{0.220728in}}{\pgfqpoint{3.696000in}{3.696000in}}%
\pgfusepath{clip}%
\pgfsetbuttcap%
\pgfsetroundjoin%
\definecolor{currentfill}{rgb}{0.121569,0.466667,0.705882}%
\pgfsetfillcolor{currentfill}%
\pgfsetfillopacity{0.826249}%
\pgfsetlinewidth{1.003750pt}%
\definecolor{currentstroke}{rgb}{0.121569,0.466667,0.705882}%
\pgfsetstrokecolor{currentstroke}%
\pgfsetstrokeopacity{0.826249}%
\pgfsetdash{}{0pt}%
\pgfpathmoveto{\pgfqpoint{1.458352in}{1.899037in}}%
\pgfpathcurveto{\pgfqpoint{1.466588in}{1.899037in}}{\pgfqpoint{1.474488in}{1.902309in}}{\pgfqpoint{1.480312in}{1.908133in}}%
\pgfpathcurveto{\pgfqpoint{1.486136in}{1.913957in}}{\pgfqpoint{1.489409in}{1.921857in}}{\pgfqpoint{1.489409in}{1.930093in}}%
\pgfpathcurveto{\pgfqpoint{1.489409in}{1.938329in}}{\pgfqpoint{1.486136in}{1.946229in}}{\pgfqpoint{1.480312in}{1.952053in}}%
\pgfpathcurveto{\pgfqpoint{1.474488in}{1.957877in}}{\pgfqpoint{1.466588in}{1.961150in}}{\pgfqpoint{1.458352in}{1.961150in}}%
\pgfpathcurveto{\pgfqpoint{1.450116in}{1.961150in}}{\pgfqpoint{1.442216in}{1.957877in}}{\pgfqpoint{1.436392in}{1.952053in}}%
\pgfpathcurveto{\pgfqpoint{1.430568in}{1.946229in}}{\pgfqpoint{1.427296in}{1.938329in}}{\pgfqpoint{1.427296in}{1.930093in}}%
\pgfpathcurveto{\pgfqpoint{1.427296in}{1.921857in}}{\pgfqpoint{1.430568in}{1.913957in}}{\pgfqpoint{1.436392in}{1.908133in}}%
\pgfpathcurveto{\pgfqpoint{1.442216in}{1.902309in}}{\pgfqpoint{1.450116in}{1.899037in}}{\pgfqpoint{1.458352in}{1.899037in}}%
\pgfpathclose%
\pgfusepath{stroke,fill}%
\end{pgfscope}%
\begin{pgfscope}%
\pgfpathrectangle{\pgfqpoint{0.100000in}{0.220728in}}{\pgfqpoint{3.696000in}{3.696000in}}%
\pgfusepath{clip}%
\pgfsetbuttcap%
\pgfsetroundjoin%
\definecolor{currentfill}{rgb}{0.121569,0.466667,0.705882}%
\pgfsetfillcolor{currentfill}%
\pgfsetfillopacity{0.828234}%
\pgfsetlinewidth{1.003750pt}%
\definecolor{currentstroke}{rgb}{0.121569,0.466667,0.705882}%
\pgfsetstrokecolor{currentstroke}%
\pgfsetstrokeopacity{0.828234}%
\pgfsetdash{}{0pt}%
\pgfpathmoveto{\pgfqpoint{2.901987in}{2.230559in}}%
\pgfpathcurveto{\pgfqpoint{2.910223in}{2.230559in}}{\pgfqpoint{2.918123in}{2.233831in}}{\pgfqpoint{2.923947in}{2.239655in}}%
\pgfpathcurveto{\pgfqpoint{2.929771in}{2.245479in}}{\pgfqpoint{2.933043in}{2.253379in}}{\pgfqpoint{2.933043in}{2.261615in}}%
\pgfpathcurveto{\pgfqpoint{2.933043in}{2.269852in}}{\pgfqpoint{2.929771in}{2.277752in}}{\pgfqpoint{2.923947in}{2.283576in}}%
\pgfpathcurveto{\pgfqpoint{2.918123in}{2.289400in}}{\pgfqpoint{2.910223in}{2.292672in}}{\pgfqpoint{2.901987in}{2.292672in}}%
\pgfpathcurveto{\pgfqpoint{2.893751in}{2.292672in}}{\pgfqpoint{2.885851in}{2.289400in}}{\pgfqpoint{2.880027in}{2.283576in}}%
\pgfpathcurveto{\pgfqpoint{2.874203in}{2.277752in}}{\pgfqpoint{2.870930in}{2.269852in}}{\pgfqpoint{2.870930in}{2.261615in}}%
\pgfpathcurveto{\pgfqpoint{2.870930in}{2.253379in}}{\pgfqpoint{2.874203in}{2.245479in}}{\pgfqpoint{2.880027in}{2.239655in}}%
\pgfpathcurveto{\pgfqpoint{2.885851in}{2.233831in}}{\pgfqpoint{2.893751in}{2.230559in}}{\pgfqpoint{2.901987in}{2.230559in}}%
\pgfpathclose%
\pgfusepath{stroke,fill}%
\end{pgfscope}%
\begin{pgfscope}%
\pgfpathrectangle{\pgfqpoint{0.100000in}{0.220728in}}{\pgfqpoint{3.696000in}{3.696000in}}%
\pgfusepath{clip}%
\pgfsetbuttcap%
\pgfsetroundjoin%
\definecolor{currentfill}{rgb}{0.121569,0.466667,0.705882}%
\pgfsetfillcolor{currentfill}%
\pgfsetfillopacity{0.829601}%
\pgfsetlinewidth{1.003750pt}%
\definecolor{currentstroke}{rgb}{0.121569,0.466667,0.705882}%
\pgfsetstrokecolor{currentstroke}%
\pgfsetstrokeopacity{0.829601}%
\pgfsetdash{}{0pt}%
\pgfpathmoveto{\pgfqpoint{1.477025in}{1.891045in}}%
\pgfpathcurveto{\pgfqpoint{1.485261in}{1.891045in}}{\pgfqpoint{1.493161in}{1.894318in}}{\pgfqpoint{1.498985in}{1.900142in}}%
\pgfpathcurveto{\pgfqpoint{1.504809in}{1.905966in}}{\pgfqpoint{1.508081in}{1.913866in}}{\pgfqpoint{1.508081in}{1.922102in}}%
\pgfpathcurveto{\pgfqpoint{1.508081in}{1.930338in}}{\pgfqpoint{1.504809in}{1.938238in}}{\pgfqpoint{1.498985in}{1.944062in}}%
\pgfpathcurveto{\pgfqpoint{1.493161in}{1.949886in}}{\pgfqpoint{1.485261in}{1.953158in}}{\pgfqpoint{1.477025in}{1.953158in}}%
\pgfpathcurveto{\pgfqpoint{1.468788in}{1.953158in}}{\pgfqpoint{1.460888in}{1.949886in}}{\pgfqpoint{1.455064in}{1.944062in}}%
\pgfpathcurveto{\pgfqpoint{1.449240in}{1.938238in}}{\pgfqpoint{1.445968in}{1.930338in}}{\pgfqpoint{1.445968in}{1.922102in}}%
\pgfpathcurveto{\pgfqpoint{1.445968in}{1.913866in}}{\pgfqpoint{1.449240in}{1.905966in}}{\pgfqpoint{1.455064in}{1.900142in}}%
\pgfpathcurveto{\pgfqpoint{1.460888in}{1.894318in}}{\pgfqpoint{1.468788in}{1.891045in}}{\pgfqpoint{1.477025in}{1.891045in}}%
\pgfpathclose%
\pgfusepath{stroke,fill}%
\end{pgfscope}%
\begin{pgfscope}%
\pgfpathrectangle{\pgfqpoint{0.100000in}{0.220728in}}{\pgfqpoint{3.696000in}{3.696000in}}%
\pgfusepath{clip}%
\pgfsetbuttcap%
\pgfsetroundjoin%
\definecolor{currentfill}{rgb}{0.121569,0.466667,0.705882}%
\pgfsetfillcolor{currentfill}%
\pgfsetfillopacity{0.832502}%
\pgfsetlinewidth{1.003750pt}%
\definecolor{currentstroke}{rgb}{0.121569,0.466667,0.705882}%
\pgfsetstrokecolor{currentstroke}%
\pgfsetstrokeopacity{0.832502}%
\pgfsetdash{}{0pt}%
\pgfpathmoveto{\pgfqpoint{1.493409in}{1.887834in}}%
\pgfpathcurveto{\pgfqpoint{1.501646in}{1.887834in}}{\pgfqpoint{1.509546in}{1.891106in}}{\pgfqpoint{1.515370in}{1.896930in}}%
\pgfpathcurveto{\pgfqpoint{1.521194in}{1.902754in}}{\pgfqpoint{1.524466in}{1.910654in}}{\pgfqpoint{1.524466in}{1.918890in}}%
\pgfpathcurveto{\pgfqpoint{1.524466in}{1.927126in}}{\pgfqpoint{1.521194in}{1.935026in}}{\pgfqpoint{1.515370in}{1.940850in}}%
\pgfpathcurveto{\pgfqpoint{1.509546in}{1.946674in}}{\pgfqpoint{1.501646in}{1.949947in}}{\pgfqpoint{1.493409in}{1.949947in}}%
\pgfpathcurveto{\pgfqpoint{1.485173in}{1.949947in}}{\pgfqpoint{1.477273in}{1.946674in}}{\pgfqpoint{1.471449in}{1.940850in}}%
\pgfpathcurveto{\pgfqpoint{1.465625in}{1.935026in}}{\pgfqpoint{1.462353in}{1.927126in}}{\pgfqpoint{1.462353in}{1.918890in}}%
\pgfpathcurveto{\pgfqpoint{1.462353in}{1.910654in}}{\pgfqpoint{1.465625in}{1.902754in}}{\pgfqpoint{1.471449in}{1.896930in}}%
\pgfpathcurveto{\pgfqpoint{1.477273in}{1.891106in}}{\pgfqpoint{1.485173in}{1.887834in}}{\pgfqpoint{1.493409in}{1.887834in}}%
\pgfpathclose%
\pgfusepath{stroke,fill}%
\end{pgfscope}%
\begin{pgfscope}%
\pgfpathrectangle{\pgfqpoint{0.100000in}{0.220728in}}{\pgfqpoint{3.696000in}{3.696000in}}%
\pgfusepath{clip}%
\pgfsetbuttcap%
\pgfsetroundjoin%
\definecolor{currentfill}{rgb}{0.121569,0.466667,0.705882}%
\pgfsetfillcolor{currentfill}%
\pgfsetfillopacity{0.832862}%
\pgfsetlinewidth{1.003750pt}%
\definecolor{currentstroke}{rgb}{0.121569,0.466667,0.705882}%
\pgfsetstrokecolor{currentstroke}%
\pgfsetstrokeopacity{0.832862}%
\pgfsetdash{}{0pt}%
\pgfpathmoveto{\pgfqpoint{2.885490in}{2.207684in}}%
\pgfpathcurveto{\pgfqpoint{2.893726in}{2.207684in}}{\pgfqpoint{2.901626in}{2.210956in}}{\pgfqpoint{2.907450in}{2.216780in}}%
\pgfpathcurveto{\pgfqpoint{2.913274in}{2.222604in}}{\pgfqpoint{2.916546in}{2.230504in}}{\pgfqpoint{2.916546in}{2.238741in}}%
\pgfpathcurveto{\pgfqpoint{2.916546in}{2.246977in}}{\pgfqpoint{2.913274in}{2.254877in}}{\pgfqpoint{2.907450in}{2.260701in}}%
\pgfpathcurveto{\pgfqpoint{2.901626in}{2.266525in}}{\pgfqpoint{2.893726in}{2.269797in}}{\pgfqpoint{2.885490in}{2.269797in}}%
\pgfpathcurveto{\pgfqpoint{2.877254in}{2.269797in}}{\pgfqpoint{2.869354in}{2.266525in}}{\pgfqpoint{2.863530in}{2.260701in}}%
\pgfpathcurveto{\pgfqpoint{2.857706in}{2.254877in}}{\pgfqpoint{2.854433in}{2.246977in}}{\pgfqpoint{2.854433in}{2.238741in}}%
\pgfpathcurveto{\pgfqpoint{2.854433in}{2.230504in}}{\pgfqpoint{2.857706in}{2.222604in}}{\pgfqpoint{2.863530in}{2.216780in}}%
\pgfpathcurveto{\pgfqpoint{2.869354in}{2.210956in}}{\pgfqpoint{2.877254in}{2.207684in}}{\pgfqpoint{2.885490in}{2.207684in}}%
\pgfpathclose%
\pgfusepath{stroke,fill}%
\end{pgfscope}%
\begin{pgfscope}%
\pgfpathrectangle{\pgfqpoint{0.100000in}{0.220728in}}{\pgfqpoint{3.696000in}{3.696000in}}%
\pgfusepath{clip}%
\pgfsetbuttcap%
\pgfsetroundjoin%
\definecolor{currentfill}{rgb}{0.121569,0.466667,0.705882}%
\pgfsetfillcolor{currentfill}%
\pgfsetfillopacity{0.834690}%
\pgfsetlinewidth{1.003750pt}%
\definecolor{currentstroke}{rgb}{0.121569,0.466667,0.705882}%
\pgfsetstrokecolor{currentstroke}%
\pgfsetstrokeopacity{0.834690}%
\pgfsetdash{}{0pt}%
\pgfpathmoveto{\pgfqpoint{1.506606in}{1.880158in}}%
\pgfpathcurveto{\pgfqpoint{1.514842in}{1.880158in}}{\pgfqpoint{1.522742in}{1.883430in}}{\pgfqpoint{1.528566in}{1.889254in}}%
\pgfpathcurveto{\pgfqpoint{1.534390in}{1.895078in}}{\pgfqpoint{1.537663in}{1.902978in}}{\pgfqpoint{1.537663in}{1.911214in}}%
\pgfpathcurveto{\pgfqpoint{1.537663in}{1.919451in}}{\pgfqpoint{1.534390in}{1.927351in}}{\pgfqpoint{1.528566in}{1.933175in}}%
\pgfpathcurveto{\pgfqpoint{1.522742in}{1.938999in}}{\pgfqpoint{1.514842in}{1.942271in}}{\pgfqpoint{1.506606in}{1.942271in}}%
\pgfpathcurveto{\pgfqpoint{1.498370in}{1.942271in}}{\pgfqpoint{1.490470in}{1.938999in}}{\pgfqpoint{1.484646in}{1.933175in}}%
\pgfpathcurveto{\pgfqpoint{1.478822in}{1.927351in}}{\pgfqpoint{1.475550in}{1.919451in}}{\pgfqpoint{1.475550in}{1.911214in}}%
\pgfpathcurveto{\pgfqpoint{1.475550in}{1.902978in}}{\pgfqpoint{1.478822in}{1.895078in}}{\pgfqpoint{1.484646in}{1.889254in}}%
\pgfpathcurveto{\pgfqpoint{1.490470in}{1.883430in}}{\pgfqpoint{1.498370in}{1.880158in}}{\pgfqpoint{1.506606in}{1.880158in}}%
\pgfpathclose%
\pgfusepath{stroke,fill}%
\end{pgfscope}%
\begin{pgfscope}%
\pgfpathrectangle{\pgfqpoint{0.100000in}{0.220728in}}{\pgfqpoint{3.696000in}{3.696000in}}%
\pgfusepath{clip}%
\pgfsetbuttcap%
\pgfsetroundjoin%
\definecolor{currentfill}{rgb}{0.121569,0.466667,0.705882}%
\pgfsetfillcolor{currentfill}%
\pgfsetfillopacity{0.838565}%
\pgfsetlinewidth{1.003750pt}%
\definecolor{currentstroke}{rgb}{0.121569,0.466667,0.705882}%
\pgfsetstrokecolor{currentstroke}%
\pgfsetstrokeopacity{0.838565}%
\pgfsetdash{}{0pt}%
\pgfpathmoveto{\pgfqpoint{2.870149in}{2.182457in}}%
\pgfpathcurveto{\pgfqpoint{2.878385in}{2.182457in}}{\pgfqpoint{2.886285in}{2.185729in}}{\pgfqpoint{2.892109in}{2.191553in}}%
\pgfpathcurveto{\pgfqpoint{2.897933in}{2.197377in}}{\pgfqpoint{2.901206in}{2.205277in}}{\pgfqpoint{2.901206in}{2.213514in}}%
\pgfpathcurveto{\pgfqpoint{2.901206in}{2.221750in}}{\pgfqpoint{2.897933in}{2.229650in}}{\pgfqpoint{2.892109in}{2.235474in}}%
\pgfpathcurveto{\pgfqpoint{2.886285in}{2.241298in}}{\pgfqpoint{2.878385in}{2.244570in}}{\pgfqpoint{2.870149in}{2.244570in}}%
\pgfpathcurveto{\pgfqpoint{2.861913in}{2.244570in}}{\pgfqpoint{2.854013in}{2.241298in}}{\pgfqpoint{2.848189in}{2.235474in}}%
\pgfpathcurveto{\pgfqpoint{2.842365in}{2.229650in}}{\pgfqpoint{2.839093in}{2.221750in}}{\pgfqpoint{2.839093in}{2.213514in}}%
\pgfpathcurveto{\pgfqpoint{2.839093in}{2.205277in}}{\pgfqpoint{2.842365in}{2.197377in}}{\pgfqpoint{2.848189in}{2.191553in}}%
\pgfpathcurveto{\pgfqpoint{2.854013in}{2.185729in}}{\pgfqpoint{2.861913in}{2.182457in}}{\pgfqpoint{2.870149in}{2.182457in}}%
\pgfpathclose%
\pgfusepath{stroke,fill}%
\end{pgfscope}%
\begin{pgfscope}%
\pgfpathrectangle{\pgfqpoint{0.100000in}{0.220728in}}{\pgfqpoint{3.696000in}{3.696000in}}%
\pgfusepath{clip}%
\pgfsetbuttcap%
\pgfsetroundjoin%
\definecolor{currentfill}{rgb}{0.121569,0.466667,0.705882}%
\pgfsetfillcolor{currentfill}%
\pgfsetfillopacity{0.838925}%
\pgfsetlinewidth{1.003750pt}%
\definecolor{currentstroke}{rgb}{0.121569,0.466667,0.705882}%
\pgfsetstrokecolor{currentstroke}%
\pgfsetstrokeopacity{0.838925}%
\pgfsetdash{}{0pt}%
\pgfpathmoveto{\pgfqpoint{1.530527in}{1.867081in}}%
\pgfpathcurveto{\pgfqpoint{1.538763in}{1.867081in}}{\pgfqpoint{1.546663in}{1.870353in}}{\pgfqpoint{1.552487in}{1.876177in}}%
\pgfpathcurveto{\pgfqpoint{1.558311in}{1.882001in}}{\pgfqpoint{1.561583in}{1.889901in}}{\pgfqpoint{1.561583in}{1.898137in}}%
\pgfpathcurveto{\pgfqpoint{1.561583in}{1.906374in}}{\pgfqpoint{1.558311in}{1.914274in}}{\pgfqpoint{1.552487in}{1.920098in}}%
\pgfpathcurveto{\pgfqpoint{1.546663in}{1.925922in}}{\pgfqpoint{1.538763in}{1.929194in}}{\pgfqpoint{1.530527in}{1.929194in}}%
\pgfpathcurveto{\pgfqpoint{1.522291in}{1.929194in}}{\pgfqpoint{1.514390in}{1.925922in}}{\pgfqpoint{1.508567in}{1.920098in}}%
\pgfpathcurveto{\pgfqpoint{1.502743in}{1.914274in}}{\pgfqpoint{1.499470in}{1.906374in}}{\pgfqpoint{1.499470in}{1.898137in}}%
\pgfpathcurveto{\pgfqpoint{1.499470in}{1.889901in}}{\pgfqpoint{1.502743in}{1.882001in}}{\pgfqpoint{1.508567in}{1.876177in}}%
\pgfpathcurveto{\pgfqpoint{1.514390in}{1.870353in}}{\pgfqpoint{1.522291in}{1.867081in}}{\pgfqpoint{1.530527in}{1.867081in}}%
\pgfpathclose%
\pgfusepath{stroke,fill}%
\end{pgfscope}%
\begin{pgfscope}%
\pgfpathrectangle{\pgfqpoint{0.100000in}{0.220728in}}{\pgfqpoint{3.696000in}{3.696000in}}%
\pgfusepath{clip}%
\pgfsetbuttcap%
\pgfsetroundjoin%
\definecolor{currentfill}{rgb}{0.121569,0.466667,0.705882}%
\pgfsetfillcolor{currentfill}%
\pgfsetfillopacity{0.841281}%
\pgfsetlinewidth{1.003750pt}%
\definecolor{currentstroke}{rgb}{0.121569,0.466667,0.705882}%
\pgfsetstrokecolor{currentstroke}%
\pgfsetstrokeopacity{0.841281}%
\pgfsetdash{}{0pt}%
\pgfpathmoveto{\pgfqpoint{2.860345in}{2.168333in}}%
\pgfpathcurveto{\pgfqpoint{2.868581in}{2.168333in}}{\pgfqpoint{2.876481in}{2.171606in}}{\pgfqpoint{2.882305in}{2.177430in}}%
\pgfpathcurveto{\pgfqpoint{2.888129in}{2.183254in}}{\pgfqpoint{2.891401in}{2.191154in}}{\pgfqpoint{2.891401in}{2.199390in}}%
\pgfpathcurveto{\pgfqpoint{2.891401in}{2.207626in}}{\pgfqpoint{2.888129in}{2.215526in}}{\pgfqpoint{2.882305in}{2.221350in}}%
\pgfpathcurveto{\pgfqpoint{2.876481in}{2.227174in}}{\pgfqpoint{2.868581in}{2.230446in}}{\pgfqpoint{2.860345in}{2.230446in}}%
\pgfpathcurveto{\pgfqpoint{2.852109in}{2.230446in}}{\pgfqpoint{2.844209in}{2.227174in}}{\pgfqpoint{2.838385in}{2.221350in}}%
\pgfpathcurveto{\pgfqpoint{2.832561in}{2.215526in}}{\pgfqpoint{2.829288in}{2.207626in}}{\pgfqpoint{2.829288in}{2.199390in}}%
\pgfpathcurveto{\pgfqpoint{2.829288in}{2.191154in}}{\pgfqpoint{2.832561in}{2.183254in}}{\pgfqpoint{2.838385in}{2.177430in}}%
\pgfpathcurveto{\pgfqpoint{2.844209in}{2.171606in}}{\pgfqpoint{2.852109in}{2.168333in}}{\pgfqpoint{2.860345in}{2.168333in}}%
\pgfpathclose%
\pgfusepath{stroke,fill}%
\end{pgfscope}%
\begin{pgfscope}%
\pgfpathrectangle{\pgfqpoint{0.100000in}{0.220728in}}{\pgfqpoint{3.696000in}{3.696000in}}%
\pgfusepath{clip}%
\pgfsetbuttcap%
\pgfsetroundjoin%
\definecolor{currentfill}{rgb}{0.121569,0.466667,0.705882}%
\pgfsetfillcolor{currentfill}%
\pgfsetfillopacity{0.842761}%
\pgfsetlinewidth{1.003750pt}%
\definecolor{currentstroke}{rgb}{0.121569,0.466667,0.705882}%
\pgfsetstrokecolor{currentstroke}%
\pgfsetstrokeopacity{0.842761}%
\pgfsetdash{}{0pt}%
\pgfpathmoveto{\pgfqpoint{2.854773in}{2.160819in}}%
\pgfpathcurveto{\pgfqpoint{2.863009in}{2.160819in}}{\pgfqpoint{2.870909in}{2.164091in}}{\pgfqpoint{2.876733in}{2.169915in}}%
\pgfpathcurveto{\pgfqpoint{2.882557in}{2.175739in}}{\pgfqpoint{2.885829in}{2.183639in}}{\pgfqpoint{2.885829in}{2.191875in}}%
\pgfpathcurveto{\pgfqpoint{2.885829in}{2.200112in}}{\pgfqpoint{2.882557in}{2.208012in}}{\pgfqpoint{2.876733in}{2.213836in}}%
\pgfpathcurveto{\pgfqpoint{2.870909in}{2.219660in}}{\pgfqpoint{2.863009in}{2.222932in}}{\pgfqpoint{2.854773in}{2.222932in}}%
\pgfpathcurveto{\pgfqpoint{2.846536in}{2.222932in}}{\pgfqpoint{2.838636in}{2.219660in}}{\pgfqpoint{2.832812in}{2.213836in}}%
\pgfpathcurveto{\pgfqpoint{2.826988in}{2.208012in}}{\pgfqpoint{2.823716in}{2.200112in}}{\pgfqpoint{2.823716in}{2.191875in}}%
\pgfpathcurveto{\pgfqpoint{2.823716in}{2.183639in}}{\pgfqpoint{2.826988in}{2.175739in}}{\pgfqpoint{2.832812in}{2.169915in}}%
\pgfpathcurveto{\pgfqpoint{2.838636in}{2.164091in}}{\pgfqpoint{2.846536in}{2.160819in}}{\pgfqpoint{2.854773in}{2.160819in}}%
\pgfpathclose%
\pgfusepath{stroke,fill}%
\end{pgfscope}%
\begin{pgfscope}%
\pgfpathrectangle{\pgfqpoint{0.100000in}{0.220728in}}{\pgfqpoint{3.696000in}{3.696000in}}%
\pgfusepath{clip}%
\pgfsetbuttcap%
\pgfsetroundjoin%
\definecolor{currentfill}{rgb}{0.121569,0.466667,0.705882}%
\pgfsetfillcolor{currentfill}%
\pgfsetfillopacity{0.843641}%
\pgfsetlinewidth{1.003750pt}%
\definecolor{currentstroke}{rgb}{0.121569,0.466667,0.705882}%
\pgfsetstrokecolor{currentstroke}%
\pgfsetstrokeopacity{0.843641}%
\pgfsetdash{}{0pt}%
\pgfpathmoveto{\pgfqpoint{2.852069in}{2.156411in}}%
\pgfpathcurveto{\pgfqpoint{2.860305in}{2.156411in}}{\pgfqpoint{2.868205in}{2.159684in}}{\pgfqpoint{2.874029in}{2.165508in}}%
\pgfpathcurveto{\pgfqpoint{2.879853in}{2.171331in}}{\pgfqpoint{2.883125in}{2.179232in}}{\pgfqpoint{2.883125in}{2.187468in}}%
\pgfpathcurveto{\pgfqpoint{2.883125in}{2.195704in}}{\pgfqpoint{2.879853in}{2.203604in}}{\pgfqpoint{2.874029in}{2.209428in}}%
\pgfpathcurveto{\pgfqpoint{2.868205in}{2.215252in}}{\pgfqpoint{2.860305in}{2.218524in}}{\pgfqpoint{2.852069in}{2.218524in}}%
\pgfpathcurveto{\pgfqpoint{2.843832in}{2.218524in}}{\pgfqpoint{2.835932in}{2.215252in}}{\pgfqpoint{2.830108in}{2.209428in}}%
\pgfpathcurveto{\pgfqpoint{2.824285in}{2.203604in}}{\pgfqpoint{2.821012in}{2.195704in}}{\pgfqpoint{2.821012in}{2.187468in}}%
\pgfpathcurveto{\pgfqpoint{2.821012in}{2.179232in}}{\pgfqpoint{2.824285in}{2.171331in}}{\pgfqpoint{2.830108in}{2.165508in}}%
\pgfpathcurveto{\pgfqpoint{2.835932in}{2.159684in}}{\pgfqpoint{2.843832in}{2.156411in}}{\pgfqpoint{2.852069in}{2.156411in}}%
\pgfpathclose%
\pgfusepath{stroke,fill}%
\end{pgfscope}%
\begin{pgfscope}%
\pgfpathrectangle{\pgfqpoint{0.100000in}{0.220728in}}{\pgfqpoint{3.696000in}{3.696000in}}%
\pgfusepath{clip}%
\pgfsetbuttcap%
\pgfsetroundjoin%
\definecolor{currentfill}{rgb}{0.121569,0.466667,0.705882}%
\pgfsetfillcolor{currentfill}%
\pgfsetfillopacity{0.844069}%
\pgfsetlinewidth{1.003750pt}%
\definecolor{currentstroke}{rgb}{0.121569,0.466667,0.705882}%
\pgfsetstrokecolor{currentstroke}%
\pgfsetstrokeopacity{0.844069}%
\pgfsetdash{}{0pt}%
\pgfpathmoveto{\pgfqpoint{2.850305in}{2.154153in}}%
\pgfpathcurveto{\pgfqpoint{2.858541in}{2.154153in}}{\pgfqpoint{2.866441in}{2.157425in}}{\pgfqpoint{2.872265in}{2.163249in}}%
\pgfpathcurveto{\pgfqpoint{2.878089in}{2.169073in}}{\pgfqpoint{2.881362in}{2.176973in}}{\pgfqpoint{2.881362in}{2.185209in}}%
\pgfpathcurveto{\pgfqpoint{2.881362in}{2.193445in}}{\pgfqpoint{2.878089in}{2.201345in}}{\pgfqpoint{2.872265in}{2.207169in}}%
\pgfpathcurveto{\pgfqpoint{2.866441in}{2.212993in}}{\pgfqpoint{2.858541in}{2.216266in}}{\pgfqpoint{2.850305in}{2.216266in}}%
\pgfpathcurveto{\pgfqpoint{2.842069in}{2.216266in}}{\pgfqpoint{2.834169in}{2.212993in}}{\pgfqpoint{2.828345in}{2.207169in}}%
\pgfpathcurveto{\pgfqpoint{2.822521in}{2.201345in}}{\pgfqpoint{2.819249in}{2.193445in}}{\pgfqpoint{2.819249in}{2.185209in}}%
\pgfpathcurveto{\pgfqpoint{2.819249in}{2.176973in}}{\pgfqpoint{2.822521in}{2.169073in}}{\pgfqpoint{2.828345in}{2.163249in}}%
\pgfpathcurveto{\pgfqpoint{2.834169in}{2.157425in}}{\pgfqpoint{2.842069in}{2.154153in}}{\pgfqpoint{2.850305in}{2.154153in}}%
\pgfpathclose%
\pgfusepath{stroke,fill}%
\end{pgfscope}%
\begin{pgfscope}%
\pgfpathrectangle{\pgfqpoint{0.100000in}{0.220728in}}{\pgfqpoint{3.696000in}{3.696000in}}%
\pgfusepath{clip}%
\pgfsetbuttcap%
\pgfsetroundjoin%
\definecolor{currentfill}{rgb}{0.121569,0.466667,0.705882}%
\pgfsetfillcolor{currentfill}%
\pgfsetfillopacity{0.844352}%
\pgfsetlinewidth{1.003750pt}%
\definecolor{currentstroke}{rgb}{0.121569,0.466667,0.705882}%
\pgfsetstrokecolor{currentstroke}%
\pgfsetstrokeopacity{0.844352}%
\pgfsetdash{}{0pt}%
\pgfpathmoveto{\pgfqpoint{2.849563in}{2.152811in}}%
\pgfpathcurveto{\pgfqpoint{2.857800in}{2.152811in}}{\pgfqpoint{2.865700in}{2.156084in}}{\pgfqpoint{2.871524in}{2.161908in}}%
\pgfpathcurveto{\pgfqpoint{2.877348in}{2.167732in}}{\pgfqpoint{2.880620in}{2.175632in}}{\pgfqpoint{2.880620in}{2.183868in}}%
\pgfpathcurveto{\pgfqpoint{2.880620in}{2.192104in}}{\pgfqpoint{2.877348in}{2.200004in}}{\pgfqpoint{2.871524in}{2.205828in}}%
\pgfpathcurveto{\pgfqpoint{2.865700in}{2.211652in}}{\pgfqpoint{2.857800in}{2.214924in}}{\pgfqpoint{2.849563in}{2.214924in}}%
\pgfpathcurveto{\pgfqpoint{2.841327in}{2.214924in}}{\pgfqpoint{2.833427in}{2.211652in}}{\pgfqpoint{2.827603in}{2.205828in}}%
\pgfpathcurveto{\pgfqpoint{2.821779in}{2.200004in}}{\pgfqpoint{2.818507in}{2.192104in}}{\pgfqpoint{2.818507in}{2.183868in}}%
\pgfpathcurveto{\pgfqpoint{2.818507in}{2.175632in}}{\pgfqpoint{2.821779in}{2.167732in}}{\pgfqpoint{2.827603in}{2.161908in}}%
\pgfpathcurveto{\pgfqpoint{2.833427in}{2.156084in}}{\pgfqpoint{2.841327in}{2.152811in}}{\pgfqpoint{2.849563in}{2.152811in}}%
\pgfpathclose%
\pgfusepath{stroke,fill}%
\end{pgfscope}%
\begin{pgfscope}%
\pgfpathrectangle{\pgfqpoint{0.100000in}{0.220728in}}{\pgfqpoint{3.696000in}{3.696000in}}%
\pgfusepath{clip}%
\pgfsetbuttcap%
\pgfsetroundjoin%
\definecolor{currentfill}{rgb}{0.121569,0.466667,0.705882}%
\pgfsetfillcolor{currentfill}%
\pgfsetfillopacity{0.845254}%
\pgfsetlinewidth{1.003750pt}%
\definecolor{currentstroke}{rgb}{0.121569,0.466667,0.705882}%
\pgfsetstrokecolor{currentstroke}%
\pgfsetstrokeopacity{0.845254}%
\pgfsetdash{}{0pt}%
\pgfpathmoveto{\pgfqpoint{2.846166in}{2.148315in}}%
\pgfpathcurveto{\pgfqpoint{2.854402in}{2.148315in}}{\pgfqpoint{2.862302in}{2.151588in}}{\pgfqpoint{2.868126in}{2.157412in}}%
\pgfpathcurveto{\pgfqpoint{2.873950in}{2.163236in}}{\pgfqpoint{2.877222in}{2.171136in}}{\pgfqpoint{2.877222in}{2.179372in}}%
\pgfpathcurveto{\pgfqpoint{2.877222in}{2.187608in}}{\pgfqpoint{2.873950in}{2.195508in}}{\pgfqpoint{2.868126in}{2.201332in}}%
\pgfpathcurveto{\pgfqpoint{2.862302in}{2.207156in}}{\pgfqpoint{2.854402in}{2.210428in}}{\pgfqpoint{2.846166in}{2.210428in}}%
\pgfpathcurveto{\pgfqpoint{2.837930in}{2.210428in}}{\pgfqpoint{2.830030in}{2.207156in}}{\pgfqpoint{2.824206in}{2.201332in}}%
\pgfpathcurveto{\pgfqpoint{2.818382in}{2.195508in}}{\pgfqpoint{2.815109in}{2.187608in}}{\pgfqpoint{2.815109in}{2.179372in}}%
\pgfpathcurveto{\pgfqpoint{2.815109in}{2.171136in}}{\pgfqpoint{2.818382in}{2.163236in}}{\pgfqpoint{2.824206in}{2.157412in}}%
\pgfpathcurveto{\pgfqpoint{2.830030in}{2.151588in}}{\pgfqpoint{2.837930in}{2.148315in}}{\pgfqpoint{2.846166in}{2.148315in}}%
\pgfpathclose%
\pgfusepath{stroke,fill}%
\end{pgfscope}%
\begin{pgfscope}%
\pgfpathrectangle{\pgfqpoint{0.100000in}{0.220728in}}{\pgfqpoint{3.696000in}{3.696000in}}%
\pgfusepath{clip}%
\pgfsetbuttcap%
\pgfsetroundjoin%
\definecolor{currentfill}{rgb}{0.121569,0.466667,0.705882}%
\pgfsetfillcolor{currentfill}%
\pgfsetfillopacity{0.846696}%
\pgfsetlinewidth{1.003750pt}%
\definecolor{currentstroke}{rgb}{0.121569,0.466667,0.705882}%
\pgfsetstrokecolor{currentstroke}%
\pgfsetstrokeopacity{0.846696}%
\pgfsetdash{}{0pt}%
\pgfpathmoveto{\pgfqpoint{2.842069in}{2.140088in}}%
\pgfpathcurveto{\pgfqpoint{2.850306in}{2.140088in}}{\pgfqpoint{2.858206in}{2.143360in}}{\pgfqpoint{2.864030in}{2.149184in}}%
\pgfpathcurveto{\pgfqpoint{2.869853in}{2.155008in}}{\pgfqpoint{2.873126in}{2.162908in}}{\pgfqpoint{2.873126in}{2.171144in}}%
\pgfpathcurveto{\pgfqpoint{2.873126in}{2.179381in}}{\pgfqpoint{2.869853in}{2.187281in}}{\pgfqpoint{2.864030in}{2.193104in}}%
\pgfpathcurveto{\pgfqpoint{2.858206in}{2.198928in}}{\pgfqpoint{2.850306in}{2.202201in}}{\pgfqpoint{2.842069in}{2.202201in}}%
\pgfpathcurveto{\pgfqpoint{2.833833in}{2.202201in}}{\pgfqpoint{2.825933in}{2.198928in}}{\pgfqpoint{2.820109in}{2.193104in}}%
\pgfpathcurveto{\pgfqpoint{2.814285in}{2.187281in}}{\pgfqpoint{2.811013in}{2.179381in}}{\pgfqpoint{2.811013in}{2.171144in}}%
\pgfpathcurveto{\pgfqpoint{2.811013in}{2.162908in}}{\pgfqpoint{2.814285in}{2.155008in}}{\pgfqpoint{2.820109in}{2.149184in}}%
\pgfpathcurveto{\pgfqpoint{2.825933in}{2.143360in}}{\pgfqpoint{2.833833in}{2.140088in}}{\pgfqpoint{2.842069in}{2.140088in}}%
\pgfpathclose%
\pgfusepath{stroke,fill}%
\end{pgfscope}%
\begin{pgfscope}%
\pgfpathrectangle{\pgfqpoint{0.100000in}{0.220728in}}{\pgfqpoint{3.696000in}{3.696000in}}%
\pgfusepath{clip}%
\pgfsetbuttcap%
\pgfsetroundjoin%
\definecolor{currentfill}{rgb}{0.121569,0.466667,0.705882}%
\pgfsetfillcolor{currentfill}%
\pgfsetfillopacity{0.847424}%
\pgfsetlinewidth{1.003750pt}%
\definecolor{currentstroke}{rgb}{0.121569,0.466667,0.705882}%
\pgfsetstrokecolor{currentstroke}%
\pgfsetstrokeopacity{0.847424}%
\pgfsetdash{}{0pt}%
\pgfpathmoveto{\pgfqpoint{1.574362in}{1.848498in}}%
\pgfpathcurveto{\pgfqpoint{1.582599in}{1.848498in}}{\pgfqpoint{1.590499in}{1.851770in}}{\pgfqpoint{1.596323in}{1.857594in}}%
\pgfpathcurveto{\pgfqpoint{1.602147in}{1.863418in}}{\pgfqpoint{1.605419in}{1.871318in}}{\pgfqpoint{1.605419in}{1.879554in}}%
\pgfpathcurveto{\pgfqpoint{1.605419in}{1.887790in}}{\pgfqpoint{1.602147in}{1.895691in}}{\pgfqpoint{1.596323in}{1.901514in}}%
\pgfpathcurveto{\pgfqpoint{1.590499in}{1.907338in}}{\pgfqpoint{1.582599in}{1.910611in}}{\pgfqpoint{1.574362in}{1.910611in}}%
\pgfpathcurveto{\pgfqpoint{1.566126in}{1.910611in}}{\pgfqpoint{1.558226in}{1.907338in}}{\pgfqpoint{1.552402in}{1.901514in}}%
\pgfpathcurveto{\pgfqpoint{1.546578in}{1.895691in}}{\pgfqpoint{1.543306in}{1.887790in}}{\pgfqpoint{1.543306in}{1.879554in}}%
\pgfpathcurveto{\pgfqpoint{1.543306in}{1.871318in}}{\pgfqpoint{1.546578in}{1.863418in}}{\pgfqpoint{1.552402in}{1.857594in}}%
\pgfpathcurveto{\pgfqpoint{1.558226in}{1.851770in}}{\pgfqpoint{1.566126in}{1.848498in}}{\pgfqpoint{1.574362in}{1.848498in}}%
\pgfpathclose%
\pgfusepath{stroke,fill}%
\end{pgfscope}%
\begin{pgfscope}%
\pgfpathrectangle{\pgfqpoint{0.100000in}{0.220728in}}{\pgfqpoint{3.696000in}{3.696000in}}%
\pgfusepath{clip}%
\pgfsetbuttcap%
\pgfsetroundjoin%
\definecolor{currentfill}{rgb}{0.121569,0.466667,0.705882}%
\pgfsetfillcolor{currentfill}%
\pgfsetfillopacity{0.848940}%
\pgfsetlinewidth{1.003750pt}%
\definecolor{currentstroke}{rgb}{0.121569,0.466667,0.705882}%
\pgfsetstrokecolor{currentstroke}%
\pgfsetstrokeopacity{0.848940}%
\pgfsetdash{}{0pt}%
\pgfpathmoveto{\pgfqpoint{2.834196in}{2.128461in}}%
\pgfpathcurveto{\pgfqpoint{2.842432in}{2.128461in}}{\pgfqpoint{2.850332in}{2.131733in}}{\pgfqpoint{2.856156in}{2.137557in}}%
\pgfpathcurveto{\pgfqpoint{2.861980in}{2.143381in}}{\pgfqpoint{2.865252in}{2.151281in}}{\pgfqpoint{2.865252in}{2.159518in}}%
\pgfpathcurveto{\pgfqpoint{2.865252in}{2.167754in}}{\pgfqpoint{2.861980in}{2.175654in}}{\pgfqpoint{2.856156in}{2.181478in}}%
\pgfpathcurveto{\pgfqpoint{2.850332in}{2.187302in}}{\pgfqpoint{2.842432in}{2.190574in}}{\pgfqpoint{2.834196in}{2.190574in}}%
\pgfpathcurveto{\pgfqpoint{2.825959in}{2.190574in}}{\pgfqpoint{2.818059in}{2.187302in}}{\pgfqpoint{2.812235in}{2.181478in}}%
\pgfpathcurveto{\pgfqpoint{2.806411in}{2.175654in}}{\pgfqpoint{2.803139in}{2.167754in}}{\pgfqpoint{2.803139in}{2.159518in}}%
\pgfpathcurveto{\pgfqpoint{2.803139in}{2.151281in}}{\pgfqpoint{2.806411in}{2.143381in}}{\pgfqpoint{2.812235in}{2.137557in}}%
\pgfpathcurveto{\pgfqpoint{2.818059in}{2.131733in}}{\pgfqpoint{2.825959in}{2.128461in}}{\pgfqpoint{2.834196in}{2.128461in}}%
\pgfpathclose%
\pgfusepath{stroke,fill}%
\end{pgfscope}%
\begin{pgfscope}%
\pgfpathrectangle{\pgfqpoint{0.100000in}{0.220728in}}{\pgfqpoint{3.696000in}{3.696000in}}%
\pgfusepath{clip}%
\pgfsetbuttcap%
\pgfsetroundjoin%
\definecolor{currentfill}{rgb}{0.121569,0.466667,0.705882}%
\pgfsetfillcolor{currentfill}%
\pgfsetfillopacity{0.852167}%
\pgfsetlinewidth{1.003750pt}%
\definecolor{currentstroke}{rgb}{0.121569,0.466667,0.705882}%
\pgfsetstrokecolor{currentstroke}%
\pgfsetstrokeopacity{0.852167}%
\pgfsetdash{}{0pt}%
\pgfpathmoveto{\pgfqpoint{2.826405in}{2.114412in}}%
\pgfpathcurveto{\pgfqpoint{2.834641in}{2.114412in}}{\pgfqpoint{2.842541in}{2.117685in}}{\pgfqpoint{2.848365in}{2.123509in}}%
\pgfpathcurveto{\pgfqpoint{2.854189in}{2.129333in}}{\pgfqpoint{2.857462in}{2.137233in}}{\pgfqpoint{2.857462in}{2.145469in}}%
\pgfpathcurveto{\pgfqpoint{2.857462in}{2.153705in}}{\pgfqpoint{2.854189in}{2.161605in}}{\pgfqpoint{2.848365in}{2.167429in}}%
\pgfpathcurveto{\pgfqpoint{2.842541in}{2.173253in}}{\pgfqpoint{2.834641in}{2.176525in}}{\pgfqpoint{2.826405in}{2.176525in}}%
\pgfpathcurveto{\pgfqpoint{2.818169in}{2.176525in}}{\pgfqpoint{2.810269in}{2.173253in}}{\pgfqpoint{2.804445in}{2.167429in}}%
\pgfpathcurveto{\pgfqpoint{2.798621in}{2.161605in}}{\pgfqpoint{2.795349in}{2.153705in}}{\pgfqpoint{2.795349in}{2.145469in}}%
\pgfpathcurveto{\pgfqpoint{2.795349in}{2.137233in}}{\pgfqpoint{2.798621in}{2.129333in}}{\pgfqpoint{2.804445in}{2.123509in}}%
\pgfpathcurveto{\pgfqpoint{2.810269in}{2.117685in}}{\pgfqpoint{2.818169in}{2.114412in}}{\pgfqpoint{2.826405in}{2.114412in}}%
\pgfpathclose%
\pgfusepath{stroke,fill}%
\end{pgfscope}%
\begin{pgfscope}%
\pgfpathrectangle{\pgfqpoint{0.100000in}{0.220728in}}{\pgfqpoint{3.696000in}{3.696000in}}%
\pgfusepath{clip}%
\pgfsetbuttcap%
\pgfsetroundjoin%
\definecolor{currentfill}{rgb}{0.121569,0.466667,0.705882}%
\pgfsetfillcolor{currentfill}%
\pgfsetfillopacity{0.855106}%
\pgfsetlinewidth{1.003750pt}%
\definecolor{currentstroke}{rgb}{0.121569,0.466667,0.705882}%
\pgfsetstrokecolor{currentstroke}%
\pgfsetstrokeopacity{0.855106}%
\pgfsetdash{}{0pt}%
\pgfpathmoveto{\pgfqpoint{1.616057in}{1.826684in}}%
\pgfpathcurveto{\pgfqpoint{1.624293in}{1.826684in}}{\pgfqpoint{1.632193in}{1.829956in}}{\pgfqpoint{1.638017in}{1.835780in}}%
\pgfpathcurveto{\pgfqpoint{1.643841in}{1.841604in}}{\pgfqpoint{1.647113in}{1.849504in}}{\pgfqpoint{1.647113in}{1.857741in}}%
\pgfpathcurveto{\pgfqpoint{1.647113in}{1.865977in}}{\pgfqpoint{1.643841in}{1.873877in}}{\pgfqpoint{1.638017in}{1.879701in}}%
\pgfpathcurveto{\pgfqpoint{1.632193in}{1.885525in}}{\pgfqpoint{1.624293in}{1.888797in}}{\pgfqpoint{1.616057in}{1.888797in}}%
\pgfpathcurveto{\pgfqpoint{1.607821in}{1.888797in}}{\pgfqpoint{1.599921in}{1.885525in}}{\pgfqpoint{1.594097in}{1.879701in}}%
\pgfpathcurveto{\pgfqpoint{1.588273in}{1.873877in}}{\pgfqpoint{1.585000in}{1.865977in}}{\pgfqpoint{1.585000in}{1.857741in}}%
\pgfpathcurveto{\pgfqpoint{1.585000in}{1.849504in}}{\pgfqpoint{1.588273in}{1.841604in}}{\pgfqpoint{1.594097in}{1.835780in}}%
\pgfpathcurveto{\pgfqpoint{1.599921in}{1.829956in}}{\pgfqpoint{1.607821in}{1.826684in}}{\pgfqpoint{1.616057in}{1.826684in}}%
\pgfpathclose%
\pgfusepath{stroke,fill}%
\end{pgfscope}%
\begin{pgfscope}%
\pgfpathrectangle{\pgfqpoint{0.100000in}{0.220728in}}{\pgfqpoint{3.696000in}{3.696000in}}%
\pgfusepath{clip}%
\pgfsetbuttcap%
\pgfsetroundjoin%
\definecolor{currentfill}{rgb}{0.121569,0.466667,0.705882}%
\pgfsetfillcolor{currentfill}%
\pgfsetfillopacity{0.855484}%
\pgfsetlinewidth{1.003750pt}%
\definecolor{currentstroke}{rgb}{0.121569,0.466667,0.705882}%
\pgfsetstrokecolor{currentstroke}%
\pgfsetstrokeopacity{0.855484}%
\pgfsetdash{}{0pt}%
\pgfpathmoveto{\pgfqpoint{2.815089in}{2.097585in}}%
\pgfpathcurveto{\pgfqpoint{2.823325in}{2.097585in}}{\pgfqpoint{2.831225in}{2.100857in}}{\pgfqpoint{2.837049in}{2.106681in}}%
\pgfpathcurveto{\pgfqpoint{2.842873in}{2.112505in}}{\pgfqpoint{2.846145in}{2.120405in}}{\pgfqpoint{2.846145in}{2.128642in}}%
\pgfpathcurveto{\pgfqpoint{2.846145in}{2.136878in}}{\pgfqpoint{2.842873in}{2.144778in}}{\pgfqpoint{2.837049in}{2.150602in}}%
\pgfpathcurveto{\pgfqpoint{2.831225in}{2.156426in}}{\pgfqpoint{2.823325in}{2.159698in}}{\pgfqpoint{2.815089in}{2.159698in}}%
\pgfpathcurveto{\pgfqpoint{2.806852in}{2.159698in}}{\pgfqpoint{2.798952in}{2.156426in}}{\pgfqpoint{2.793128in}{2.150602in}}%
\pgfpathcurveto{\pgfqpoint{2.787304in}{2.144778in}}{\pgfqpoint{2.784032in}{2.136878in}}{\pgfqpoint{2.784032in}{2.128642in}}%
\pgfpathcurveto{\pgfqpoint{2.784032in}{2.120405in}}{\pgfqpoint{2.787304in}{2.112505in}}{\pgfqpoint{2.793128in}{2.106681in}}%
\pgfpathcurveto{\pgfqpoint{2.798952in}{2.100857in}}{\pgfqpoint{2.806852in}{2.097585in}}{\pgfqpoint{2.815089in}{2.097585in}}%
\pgfpathclose%
\pgfusepath{stroke,fill}%
\end{pgfscope}%
\begin{pgfscope}%
\pgfpathrectangle{\pgfqpoint{0.100000in}{0.220728in}}{\pgfqpoint{3.696000in}{3.696000in}}%
\pgfusepath{clip}%
\pgfsetbuttcap%
\pgfsetroundjoin%
\definecolor{currentfill}{rgb}{0.121569,0.466667,0.705882}%
\pgfsetfillcolor{currentfill}%
\pgfsetfillopacity{0.859472}%
\pgfsetlinewidth{1.003750pt}%
\definecolor{currentstroke}{rgb}{0.121569,0.466667,0.705882}%
\pgfsetstrokecolor{currentstroke}%
\pgfsetstrokeopacity{0.859472}%
\pgfsetdash{}{0pt}%
\pgfpathmoveto{\pgfqpoint{2.803080in}{2.079305in}}%
\pgfpathcurveto{\pgfqpoint{2.811316in}{2.079305in}}{\pgfqpoint{2.819216in}{2.082578in}}{\pgfqpoint{2.825040in}{2.088402in}}%
\pgfpathcurveto{\pgfqpoint{2.830864in}{2.094226in}}{\pgfqpoint{2.834136in}{2.102126in}}{\pgfqpoint{2.834136in}{2.110362in}}%
\pgfpathcurveto{\pgfqpoint{2.834136in}{2.118598in}}{\pgfqpoint{2.830864in}{2.126498in}}{\pgfqpoint{2.825040in}{2.132322in}}%
\pgfpathcurveto{\pgfqpoint{2.819216in}{2.138146in}}{\pgfqpoint{2.811316in}{2.141418in}}{\pgfqpoint{2.803080in}{2.141418in}}%
\pgfpathcurveto{\pgfqpoint{2.794844in}{2.141418in}}{\pgfqpoint{2.786944in}{2.138146in}}{\pgfqpoint{2.781120in}{2.132322in}}%
\pgfpathcurveto{\pgfqpoint{2.775296in}{2.126498in}}{\pgfqpoint{2.772023in}{2.118598in}}{\pgfqpoint{2.772023in}{2.110362in}}%
\pgfpathcurveto{\pgfqpoint{2.772023in}{2.102126in}}{\pgfqpoint{2.775296in}{2.094226in}}{\pgfqpoint{2.781120in}{2.088402in}}%
\pgfpathcurveto{\pgfqpoint{2.786944in}{2.082578in}}{\pgfqpoint{2.794844in}{2.079305in}}{\pgfqpoint{2.803080in}{2.079305in}}%
\pgfpathclose%
\pgfusepath{stroke,fill}%
\end{pgfscope}%
\begin{pgfscope}%
\pgfpathrectangle{\pgfqpoint{0.100000in}{0.220728in}}{\pgfqpoint{3.696000in}{3.696000in}}%
\pgfusepath{clip}%
\pgfsetbuttcap%
\pgfsetroundjoin%
\definecolor{currentfill}{rgb}{0.121569,0.466667,0.705882}%
\pgfsetfillcolor{currentfill}%
\pgfsetfillopacity{0.861675}%
\pgfsetlinewidth{1.003750pt}%
\definecolor{currentstroke}{rgb}{0.121569,0.466667,0.705882}%
\pgfsetstrokecolor{currentstroke}%
\pgfsetstrokeopacity{0.861675}%
\pgfsetdash{}{0pt}%
\pgfpathmoveto{\pgfqpoint{2.796655in}{2.069031in}}%
\pgfpathcurveto{\pgfqpoint{2.804891in}{2.069031in}}{\pgfqpoint{2.812791in}{2.072303in}}{\pgfqpoint{2.818615in}{2.078127in}}%
\pgfpathcurveto{\pgfqpoint{2.824439in}{2.083951in}}{\pgfqpoint{2.827711in}{2.091851in}}{\pgfqpoint{2.827711in}{2.100088in}}%
\pgfpathcurveto{\pgfqpoint{2.827711in}{2.108324in}}{\pgfqpoint{2.824439in}{2.116224in}}{\pgfqpoint{2.818615in}{2.122048in}}%
\pgfpathcurveto{\pgfqpoint{2.812791in}{2.127872in}}{\pgfqpoint{2.804891in}{2.131144in}}{\pgfqpoint{2.796655in}{2.131144in}}%
\pgfpathcurveto{\pgfqpoint{2.788419in}{2.131144in}}{\pgfqpoint{2.780519in}{2.127872in}}{\pgfqpoint{2.774695in}{2.122048in}}%
\pgfpathcurveto{\pgfqpoint{2.768871in}{2.116224in}}{\pgfqpoint{2.765598in}{2.108324in}}{\pgfqpoint{2.765598in}{2.100088in}}%
\pgfpathcurveto{\pgfqpoint{2.765598in}{2.091851in}}{\pgfqpoint{2.768871in}{2.083951in}}{\pgfqpoint{2.774695in}{2.078127in}}%
\pgfpathcurveto{\pgfqpoint{2.780519in}{2.072303in}}{\pgfqpoint{2.788419in}{2.069031in}}{\pgfqpoint{2.796655in}{2.069031in}}%
\pgfpathclose%
\pgfusepath{stroke,fill}%
\end{pgfscope}%
\begin{pgfscope}%
\pgfpathrectangle{\pgfqpoint{0.100000in}{0.220728in}}{\pgfqpoint{3.696000in}{3.696000in}}%
\pgfusepath{clip}%
\pgfsetbuttcap%
\pgfsetroundjoin%
\definecolor{currentfill}{rgb}{0.121569,0.466667,0.705882}%
\pgfsetfillcolor{currentfill}%
\pgfsetfillopacity{0.862330}%
\pgfsetlinewidth{1.003750pt}%
\definecolor{currentstroke}{rgb}{0.121569,0.466667,0.705882}%
\pgfsetstrokecolor{currentstroke}%
\pgfsetstrokeopacity{0.862330}%
\pgfsetdash{}{0pt}%
\pgfpathmoveto{\pgfqpoint{1.653350in}{1.807400in}}%
\pgfpathcurveto{\pgfqpoint{1.661586in}{1.807400in}}{\pgfqpoint{1.669486in}{1.810672in}}{\pgfqpoint{1.675310in}{1.816496in}}%
\pgfpathcurveto{\pgfqpoint{1.681134in}{1.822320in}}{\pgfqpoint{1.684406in}{1.830220in}}{\pgfqpoint{1.684406in}{1.838457in}}%
\pgfpathcurveto{\pgfqpoint{1.684406in}{1.846693in}}{\pgfqpoint{1.681134in}{1.854593in}}{\pgfqpoint{1.675310in}{1.860417in}}%
\pgfpathcurveto{\pgfqpoint{1.669486in}{1.866241in}}{\pgfqpoint{1.661586in}{1.869513in}}{\pgfqpoint{1.653350in}{1.869513in}}%
\pgfpathcurveto{\pgfqpoint{1.645114in}{1.869513in}}{\pgfqpoint{1.637214in}{1.866241in}}{\pgfqpoint{1.631390in}{1.860417in}}%
\pgfpathcurveto{\pgfqpoint{1.625566in}{1.854593in}}{\pgfqpoint{1.622293in}{1.846693in}}{\pgfqpoint{1.622293in}{1.838457in}}%
\pgfpathcurveto{\pgfqpoint{1.622293in}{1.830220in}}{\pgfqpoint{1.625566in}{1.822320in}}{\pgfqpoint{1.631390in}{1.816496in}}%
\pgfpathcurveto{\pgfqpoint{1.637214in}{1.810672in}}{\pgfqpoint{1.645114in}{1.807400in}}{\pgfqpoint{1.653350in}{1.807400in}}%
\pgfpathclose%
\pgfusepath{stroke,fill}%
\end{pgfscope}%
\begin{pgfscope}%
\pgfpathrectangle{\pgfqpoint{0.100000in}{0.220728in}}{\pgfqpoint{3.696000in}{3.696000in}}%
\pgfusepath{clip}%
\pgfsetbuttcap%
\pgfsetroundjoin%
\definecolor{currentfill}{rgb}{0.121569,0.466667,0.705882}%
\pgfsetfillcolor{currentfill}%
\pgfsetfillopacity{0.862791}%
\pgfsetlinewidth{1.003750pt}%
\definecolor{currentstroke}{rgb}{0.121569,0.466667,0.705882}%
\pgfsetstrokecolor{currentstroke}%
\pgfsetstrokeopacity{0.862791}%
\pgfsetdash{}{0pt}%
\pgfpathmoveto{\pgfqpoint{2.792796in}{2.063361in}}%
\pgfpathcurveto{\pgfqpoint{2.801032in}{2.063361in}}{\pgfqpoint{2.808933in}{2.066633in}}{\pgfqpoint{2.814756in}{2.072457in}}%
\pgfpathcurveto{\pgfqpoint{2.820580in}{2.078281in}}{\pgfqpoint{2.823853in}{2.086181in}}{\pgfqpoint{2.823853in}{2.094417in}}%
\pgfpathcurveto{\pgfqpoint{2.823853in}{2.102654in}}{\pgfqpoint{2.820580in}{2.110554in}}{\pgfqpoint{2.814756in}{2.116378in}}%
\pgfpathcurveto{\pgfqpoint{2.808933in}{2.122202in}}{\pgfqpoint{2.801032in}{2.125474in}}{\pgfqpoint{2.792796in}{2.125474in}}%
\pgfpathcurveto{\pgfqpoint{2.784560in}{2.125474in}}{\pgfqpoint{2.776660in}{2.122202in}}{\pgfqpoint{2.770836in}{2.116378in}}%
\pgfpathcurveto{\pgfqpoint{2.765012in}{2.110554in}}{\pgfqpoint{2.761740in}{2.102654in}}{\pgfqpoint{2.761740in}{2.094417in}}%
\pgfpathcurveto{\pgfqpoint{2.761740in}{2.086181in}}{\pgfqpoint{2.765012in}{2.078281in}}{\pgfqpoint{2.770836in}{2.072457in}}%
\pgfpathcurveto{\pgfqpoint{2.776660in}{2.066633in}}{\pgfqpoint{2.784560in}{2.063361in}}{\pgfqpoint{2.792796in}{2.063361in}}%
\pgfpathclose%
\pgfusepath{stroke,fill}%
\end{pgfscope}%
\begin{pgfscope}%
\pgfpathrectangle{\pgfqpoint{0.100000in}{0.220728in}}{\pgfqpoint{3.696000in}{3.696000in}}%
\pgfusepath{clip}%
\pgfsetbuttcap%
\pgfsetroundjoin%
\definecolor{currentfill}{rgb}{0.121569,0.466667,0.705882}%
\pgfsetfillcolor{currentfill}%
\pgfsetfillopacity{0.863427}%
\pgfsetlinewidth{1.003750pt}%
\definecolor{currentstroke}{rgb}{0.121569,0.466667,0.705882}%
\pgfsetstrokecolor{currentstroke}%
\pgfsetstrokeopacity{0.863427}%
\pgfsetdash{}{0pt}%
\pgfpathmoveto{\pgfqpoint{2.790820in}{2.060144in}}%
\pgfpathcurveto{\pgfqpoint{2.799056in}{2.060144in}}{\pgfqpoint{2.806957in}{2.063416in}}{\pgfqpoint{2.812780in}{2.069240in}}%
\pgfpathcurveto{\pgfqpoint{2.818604in}{2.075064in}}{\pgfqpoint{2.821877in}{2.082964in}}{\pgfqpoint{2.821877in}{2.091201in}}%
\pgfpathcurveto{\pgfqpoint{2.821877in}{2.099437in}}{\pgfqpoint{2.818604in}{2.107337in}}{\pgfqpoint{2.812780in}{2.113161in}}%
\pgfpathcurveto{\pgfqpoint{2.806957in}{2.118985in}}{\pgfqpoint{2.799056in}{2.122257in}}{\pgfqpoint{2.790820in}{2.122257in}}%
\pgfpathcurveto{\pgfqpoint{2.782584in}{2.122257in}}{\pgfqpoint{2.774684in}{2.118985in}}{\pgfqpoint{2.768860in}{2.113161in}}%
\pgfpathcurveto{\pgfqpoint{2.763036in}{2.107337in}}{\pgfqpoint{2.759764in}{2.099437in}}{\pgfqpoint{2.759764in}{2.091201in}}%
\pgfpathcurveto{\pgfqpoint{2.759764in}{2.082964in}}{\pgfqpoint{2.763036in}{2.075064in}}{\pgfqpoint{2.768860in}{2.069240in}}%
\pgfpathcurveto{\pgfqpoint{2.774684in}{2.063416in}}{\pgfqpoint{2.782584in}{2.060144in}}{\pgfqpoint{2.790820in}{2.060144in}}%
\pgfpathclose%
\pgfusepath{stroke,fill}%
\end{pgfscope}%
\begin{pgfscope}%
\pgfpathrectangle{\pgfqpoint{0.100000in}{0.220728in}}{\pgfqpoint{3.696000in}{3.696000in}}%
\pgfusepath{clip}%
\pgfsetbuttcap%
\pgfsetroundjoin%
\definecolor{currentfill}{rgb}{0.121569,0.466667,0.705882}%
\pgfsetfillcolor{currentfill}%
\pgfsetfillopacity{0.863727}%
\pgfsetlinewidth{1.003750pt}%
\definecolor{currentstroke}{rgb}{0.121569,0.466667,0.705882}%
\pgfsetstrokecolor{currentstroke}%
\pgfsetstrokeopacity{0.863727}%
\pgfsetdash{}{0pt}%
\pgfpathmoveto{\pgfqpoint{2.789534in}{2.058431in}}%
\pgfpathcurveto{\pgfqpoint{2.797770in}{2.058431in}}{\pgfqpoint{2.805670in}{2.061704in}}{\pgfqpoint{2.811494in}{2.067528in}}%
\pgfpathcurveto{\pgfqpoint{2.817318in}{2.073352in}}{\pgfqpoint{2.820590in}{2.081252in}}{\pgfqpoint{2.820590in}{2.089488in}}%
\pgfpathcurveto{\pgfqpoint{2.820590in}{2.097724in}}{\pgfqpoint{2.817318in}{2.105624in}}{\pgfqpoint{2.811494in}{2.111448in}}%
\pgfpathcurveto{\pgfqpoint{2.805670in}{2.117272in}}{\pgfqpoint{2.797770in}{2.120544in}}{\pgfqpoint{2.789534in}{2.120544in}}%
\pgfpathcurveto{\pgfqpoint{2.781298in}{2.120544in}}{\pgfqpoint{2.773398in}{2.117272in}}{\pgfqpoint{2.767574in}{2.111448in}}%
\pgfpathcurveto{\pgfqpoint{2.761750in}{2.105624in}}{\pgfqpoint{2.758477in}{2.097724in}}{\pgfqpoint{2.758477in}{2.089488in}}%
\pgfpathcurveto{\pgfqpoint{2.758477in}{2.081252in}}{\pgfqpoint{2.761750in}{2.073352in}}{\pgfqpoint{2.767574in}{2.067528in}}%
\pgfpathcurveto{\pgfqpoint{2.773398in}{2.061704in}}{\pgfqpoint{2.781298in}{2.058431in}}{\pgfqpoint{2.789534in}{2.058431in}}%
\pgfpathclose%
\pgfusepath{stroke,fill}%
\end{pgfscope}%
\begin{pgfscope}%
\pgfpathrectangle{\pgfqpoint{0.100000in}{0.220728in}}{\pgfqpoint{3.696000in}{3.696000in}}%
\pgfusepath{clip}%
\pgfsetbuttcap%
\pgfsetroundjoin%
\definecolor{currentfill}{rgb}{0.121569,0.466667,0.705882}%
\pgfsetfillcolor{currentfill}%
\pgfsetfillopacity{0.865197}%
\pgfsetlinewidth{1.003750pt}%
\definecolor{currentstroke}{rgb}{0.121569,0.466667,0.705882}%
\pgfsetstrokecolor{currentstroke}%
\pgfsetstrokeopacity{0.865197}%
\pgfsetdash{}{0pt}%
\pgfpathmoveto{\pgfqpoint{2.785570in}{2.050439in}}%
\pgfpathcurveto{\pgfqpoint{2.793806in}{2.050439in}}{\pgfqpoint{2.801706in}{2.053711in}}{\pgfqpoint{2.807530in}{2.059535in}}%
\pgfpathcurveto{\pgfqpoint{2.813354in}{2.065359in}}{\pgfqpoint{2.816626in}{2.073259in}}{\pgfqpoint{2.816626in}{2.081495in}}%
\pgfpathcurveto{\pgfqpoint{2.816626in}{2.089732in}}{\pgfqpoint{2.813354in}{2.097632in}}{\pgfqpoint{2.807530in}{2.103456in}}%
\pgfpathcurveto{\pgfqpoint{2.801706in}{2.109280in}}{\pgfqpoint{2.793806in}{2.112552in}}{\pgfqpoint{2.785570in}{2.112552in}}%
\pgfpathcurveto{\pgfqpoint{2.777334in}{2.112552in}}{\pgfqpoint{2.769434in}{2.109280in}}{\pgfqpoint{2.763610in}{2.103456in}}%
\pgfpathcurveto{\pgfqpoint{2.757786in}{2.097632in}}{\pgfqpoint{2.754513in}{2.089732in}}{\pgfqpoint{2.754513in}{2.081495in}}%
\pgfpathcurveto{\pgfqpoint{2.754513in}{2.073259in}}{\pgfqpoint{2.757786in}{2.065359in}}{\pgfqpoint{2.763610in}{2.059535in}}%
\pgfpathcurveto{\pgfqpoint{2.769434in}{2.053711in}}{\pgfqpoint{2.777334in}{2.050439in}}{\pgfqpoint{2.785570in}{2.050439in}}%
\pgfpathclose%
\pgfusepath{stroke,fill}%
\end{pgfscope}%
\begin{pgfscope}%
\pgfpathrectangle{\pgfqpoint{0.100000in}{0.220728in}}{\pgfqpoint{3.696000in}{3.696000in}}%
\pgfusepath{clip}%
\pgfsetbuttcap%
\pgfsetroundjoin%
\definecolor{currentfill}{rgb}{0.121569,0.466667,0.705882}%
\pgfsetfillcolor{currentfill}%
\pgfsetfillopacity{0.867475}%
\pgfsetlinewidth{1.003750pt}%
\definecolor{currentstroke}{rgb}{0.121569,0.466667,0.705882}%
\pgfsetstrokecolor{currentstroke}%
\pgfsetstrokeopacity{0.867475}%
\pgfsetdash{}{0pt}%
\pgfpathmoveto{\pgfqpoint{2.777215in}{2.038303in}}%
\pgfpathcurveto{\pgfqpoint{2.785451in}{2.038303in}}{\pgfqpoint{2.793351in}{2.041575in}}{\pgfqpoint{2.799175in}{2.047399in}}%
\pgfpathcurveto{\pgfqpoint{2.804999in}{2.053223in}}{\pgfqpoint{2.808271in}{2.061123in}}{\pgfqpoint{2.808271in}{2.069360in}}%
\pgfpathcurveto{\pgfqpoint{2.808271in}{2.077596in}}{\pgfqpoint{2.804999in}{2.085496in}}{\pgfqpoint{2.799175in}{2.091320in}}%
\pgfpathcurveto{\pgfqpoint{2.793351in}{2.097144in}}{\pgfqpoint{2.785451in}{2.100416in}}{\pgfqpoint{2.777215in}{2.100416in}}%
\pgfpathcurveto{\pgfqpoint{2.768978in}{2.100416in}}{\pgfqpoint{2.761078in}{2.097144in}}{\pgfqpoint{2.755254in}{2.091320in}}%
\pgfpathcurveto{\pgfqpoint{2.749430in}{2.085496in}}{\pgfqpoint{2.746158in}{2.077596in}}{\pgfqpoint{2.746158in}{2.069360in}}%
\pgfpathcurveto{\pgfqpoint{2.746158in}{2.061123in}}{\pgfqpoint{2.749430in}{2.053223in}}{\pgfqpoint{2.755254in}{2.047399in}}%
\pgfpathcurveto{\pgfqpoint{2.761078in}{2.041575in}}{\pgfqpoint{2.768978in}{2.038303in}}{\pgfqpoint{2.777215in}{2.038303in}}%
\pgfpathclose%
\pgfusepath{stroke,fill}%
\end{pgfscope}%
\begin{pgfscope}%
\pgfpathrectangle{\pgfqpoint{0.100000in}{0.220728in}}{\pgfqpoint{3.696000in}{3.696000in}}%
\pgfusepath{clip}%
\pgfsetbuttcap%
\pgfsetroundjoin%
\definecolor{currentfill}{rgb}{0.121569,0.466667,0.705882}%
\pgfsetfillcolor{currentfill}%
\pgfsetfillopacity{0.868125}%
\pgfsetlinewidth{1.003750pt}%
\definecolor{currentstroke}{rgb}{0.121569,0.466667,0.705882}%
\pgfsetstrokecolor{currentstroke}%
\pgfsetstrokeopacity{0.868125}%
\pgfsetdash{}{0pt}%
\pgfpathmoveto{\pgfqpoint{1.684981in}{1.783268in}}%
\pgfpathcurveto{\pgfqpoint{1.693217in}{1.783268in}}{\pgfqpoint{1.701117in}{1.786540in}}{\pgfqpoint{1.706941in}{1.792364in}}%
\pgfpathcurveto{\pgfqpoint{1.712765in}{1.798188in}}{\pgfqpoint{1.716037in}{1.806088in}}{\pgfqpoint{1.716037in}{1.814324in}}%
\pgfpathcurveto{\pgfqpoint{1.716037in}{1.822560in}}{\pgfqpoint{1.712765in}{1.830460in}}{\pgfqpoint{1.706941in}{1.836284in}}%
\pgfpathcurveto{\pgfqpoint{1.701117in}{1.842108in}}{\pgfqpoint{1.693217in}{1.845381in}}{\pgfqpoint{1.684981in}{1.845381in}}%
\pgfpathcurveto{\pgfqpoint{1.676744in}{1.845381in}}{\pgfqpoint{1.668844in}{1.842108in}}{\pgfqpoint{1.663020in}{1.836284in}}%
\pgfpathcurveto{\pgfqpoint{1.657196in}{1.830460in}}{\pgfqpoint{1.653924in}{1.822560in}}{\pgfqpoint{1.653924in}{1.814324in}}%
\pgfpathcurveto{\pgfqpoint{1.653924in}{1.806088in}}{\pgfqpoint{1.657196in}{1.798188in}}{\pgfqpoint{1.663020in}{1.792364in}}%
\pgfpathcurveto{\pgfqpoint{1.668844in}{1.786540in}}{\pgfqpoint{1.676744in}{1.783268in}}{\pgfqpoint{1.684981in}{1.783268in}}%
\pgfpathclose%
\pgfusepath{stroke,fill}%
\end{pgfscope}%
\begin{pgfscope}%
\pgfpathrectangle{\pgfqpoint{0.100000in}{0.220728in}}{\pgfqpoint{3.696000in}{3.696000in}}%
\pgfusepath{clip}%
\pgfsetbuttcap%
\pgfsetroundjoin%
\definecolor{currentfill}{rgb}{0.121569,0.466667,0.705882}%
\pgfsetfillcolor{currentfill}%
\pgfsetfillopacity{0.870756}%
\pgfsetlinewidth{1.003750pt}%
\definecolor{currentstroke}{rgb}{0.121569,0.466667,0.705882}%
\pgfsetstrokecolor{currentstroke}%
\pgfsetstrokeopacity{0.870756}%
\pgfsetdash{}{0pt}%
\pgfpathmoveto{\pgfqpoint{2.767389in}{2.020767in}}%
\pgfpathcurveto{\pgfqpoint{2.775626in}{2.020767in}}{\pgfqpoint{2.783526in}{2.024039in}}{\pgfqpoint{2.789349in}{2.029863in}}%
\pgfpathcurveto{\pgfqpoint{2.795173in}{2.035687in}}{\pgfqpoint{2.798446in}{2.043587in}}{\pgfqpoint{2.798446in}{2.051823in}}%
\pgfpathcurveto{\pgfqpoint{2.798446in}{2.060059in}}{\pgfqpoint{2.795173in}{2.067959in}}{\pgfqpoint{2.789349in}{2.073783in}}%
\pgfpathcurveto{\pgfqpoint{2.783526in}{2.079607in}}{\pgfqpoint{2.775626in}{2.082880in}}{\pgfqpoint{2.767389in}{2.082880in}}%
\pgfpathcurveto{\pgfqpoint{2.759153in}{2.082880in}}{\pgfqpoint{2.751253in}{2.079607in}}{\pgfqpoint{2.745429in}{2.073783in}}%
\pgfpathcurveto{\pgfqpoint{2.739605in}{2.067959in}}{\pgfqpoint{2.736333in}{2.060059in}}{\pgfqpoint{2.736333in}{2.051823in}}%
\pgfpathcurveto{\pgfqpoint{2.736333in}{2.043587in}}{\pgfqpoint{2.739605in}{2.035687in}}{\pgfqpoint{2.745429in}{2.029863in}}%
\pgfpathcurveto{\pgfqpoint{2.751253in}{2.024039in}}{\pgfqpoint{2.759153in}{2.020767in}}{\pgfqpoint{2.767389in}{2.020767in}}%
\pgfpathclose%
\pgfusepath{stroke,fill}%
\end{pgfscope}%
\begin{pgfscope}%
\pgfpathrectangle{\pgfqpoint{0.100000in}{0.220728in}}{\pgfqpoint{3.696000in}{3.696000in}}%
\pgfusepath{clip}%
\pgfsetbuttcap%
\pgfsetroundjoin%
\definecolor{currentfill}{rgb}{0.121569,0.466667,0.705882}%
\pgfsetfillcolor{currentfill}%
\pgfsetfillopacity{0.873372}%
\pgfsetlinewidth{1.003750pt}%
\definecolor{currentstroke}{rgb}{0.121569,0.466667,0.705882}%
\pgfsetstrokecolor{currentstroke}%
\pgfsetstrokeopacity{0.873372}%
\pgfsetdash{}{0pt}%
\pgfpathmoveto{\pgfqpoint{1.713514in}{1.768386in}}%
\pgfpathcurveto{\pgfqpoint{1.721750in}{1.768386in}}{\pgfqpoint{1.729650in}{1.771658in}}{\pgfqpoint{1.735474in}{1.777482in}}%
\pgfpathcurveto{\pgfqpoint{1.741298in}{1.783306in}}{\pgfqpoint{1.744570in}{1.791206in}}{\pgfqpoint{1.744570in}{1.799442in}}%
\pgfpathcurveto{\pgfqpoint{1.744570in}{1.807679in}}{\pgfqpoint{1.741298in}{1.815579in}}{\pgfqpoint{1.735474in}{1.821403in}}%
\pgfpathcurveto{\pgfqpoint{1.729650in}{1.827227in}}{\pgfqpoint{1.721750in}{1.830499in}}{\pgfqpoint{1.713514in}{1.830499in}}%
\pgfpathcurveto{\pgfqpoint{1.705278in}{1.830499in}}{\pgfqpoint{1.697378in}{1.827227in}}{\pgfqpoint{1.691554in}{1.821403in}}%
\pgfpathcurveto{\pgfqpoint{1.685730in}{1.815579in}}{\pgfqpoint{1.682457in}{1.807679in}}{\pgfqpoint{1.682457in}{1.799442in}}%
\pgfpathcurveto{\pgfqpoint{1.682457in}{1.791206in}}{\pgfqpoint{1.685730in}{1.783306in}}{\pgfqpoint{1.691554in}{1.777482in}}%
\pgfpathcurveto{\pgfqpoint{1.697378in}{1.771658in}}{\pgfqpoint{1.705278in}{1.768386in}}{\pgfqpoint{1.713514in}{1.768386in}}%
\pgfpathclose%
\pgfusepath{stroke,fill}%
\end{pgfscope}%
\begin{pgfscope}%
\pgfpathrectangle{\pgfqpoint{0.100000in}{0.220728in}}{\pgfqpoint{3.696000in}{3.696000in}}%
\pgfusepath{clip}%
\pgfsetbuttcap%
\pgfsetroundjoin%
\definecolor{currentfill}{rgb}{0.121569,0.466667,0.705882}%
\pgfsetfillcolor{currentfill}%
\pgfsetfillopacity{0.874488}%
\pgfsetlinewidth{1.003750pt}%
\definecolor{currentstroke}{rgb}{0.121569,0.466667,0.705882}%
\pgfsetstrokecolor{currentstroke}%
\pgfsetstrokeopacity{0.874488}%
\pgfsetdash{}{0pt}%
\pgfpathmoveto{\pgfqpoint{2.752576in}{2.001372in}}%
\pgfpathcurveto{\pgfqpoint{2.760812in}{2.001372in}}{\pgfqpoint{2.768713in}{2.004645in}}{\pgfqpoint{2.774536in}{2.010469in}}%
\pgfpathcurveto{\pgfqpoint{2.780360in}{2.016292in}}{\pgfqpoint{2.783633in}{2.024193in}}{\pgfqpoint{2.783633in}{2.032429in}}%
\pgfpathcurveto{\pgfqpoint{2.783633in}{2.040665in}}{\pgfqpoint{2.780360in}{2.048565in}}{\pgfqpoint{2.774536in}{2.054389in}}%
\pgfpathcurveto{\pgfqpoint{2.768713in}{2.060213in}}{\pgfqpoint{2.760812in}{2.063485in}}{\pgfqpoint{2.752576in}{2.063485in}}%
\pgfpathcurveto{\pgfqpoint{2.744340in}{2.063485in}}{\pgfqpoint{2.736440in}{2.060213in}}{\pgfqpoint{2.730616in}{2.054389in}}%
\pgfpathcurveto{\pgfqpoint{2.724792in}{2.048565in}}{\pgfqpoint{2.721520in}{2.040665in}}{\pgfqpoint{2.721520in}{2.032429in}}%
\pgfpathcurveto{\pgfqpoint{2.721520in}{2.024193in}}{\pgfqpoint{2.724792in}{2.016292in}}{\pgfqpoint{2.730616in}{2.010469in}}%
\pgfpathcurveto{\pgfqpoint{2.736440in}{2.004645in}}{\pgfqpoint{2.744340in}{2.001372in}}{\pgfqpoint{2.752576in}{2.001372in}}%
\pgfpathclose%
\pgfusepath{stroke,fill}%
\end{pgfscope}%
\begin{pgfscope}%
\pgfpathrectangle{\pgfqpoint{0.100000in}{0.220728in}}{\pgfqpoint{3.696000in}{3.696000in}}%
\pgfusepath{clip}%
\pgfsetbuttcap%
\pgfsetroundjoin%
\definecolor{currentfill}{rgb}{0.121569,0.466667,0.705882}%
\pgfsetfillcolor{currentfill}%
\pgfsetfillopacity{0.877584}%
\pgfsetlinewidth{1.003750pt}%
\definecolor{currentstroke}{rgb}{0.121569,0.466667,0.705882}%
\pgfsetstrokecolor{currentstroke}%
\pgfsetstrokeopacity{0.877584}%
\pgfsetdash{}{0pt}%
\pgfpathmoveto{\pgfqpoint{1.735548in}{1.757318in}}%
\pgfpathcurveto{\pgfqpoint{1.743785in}{1.757318in}}{\pgfqpoint{1.751685in}{1.760590in}}{\pgfqpoint{1.757509in}{1.766414in}}%
\pgfpathcurveto{\pgfqpoint{1.763333in}{1.772238in}}{\pgfqpoint{1.766605in}{1.780138in}}{\pgfqpoint{1.766605in}{1.788374in}}%
\pgfpathcurveto{\pgfqpoint{1.766605in}{1.796610in}}{\pgfqpoint{1.763333in}{1.804510in}}{\pgfqpoint{1.757509in}{1.810334in}}%
\pgfpathcurveto{\pgfqpoint{1.751685in}{1.816158in}}{\pgfqpoint{1.743785in}{1.819431in}}{\pgfqpoint{1.735548in}{1.819431in}}%
\pgfpathcurveto{\pgfqpoint{1.727312in}{1.819431in}}{\pgfqpoint{1.719412in}{1.816158in}}{\pgfqpoint{1.713588in}{1.810334in}}%
\pgfpathcurveto{\pgfqpoint{1.707764in}{1.804510in}}{\pgfqpoint{1.704492in}{1.796610in}}{\pgfqpoint{1.704492in}{1.788374in}}%
\pgfpathcurveto{\pgfqpoint{1.704492in}{1.780138in}}{\pgfqpoint{1.707764in}{1.772238in}}{\pgfqpoint{1.713588in}{1.766414in}}%
\pgfpathcurveto{\pgfqpoint{1.719412in}{1.760590in}}{\pgfqpoint{1.727312in}{1.757318in}}{\pgfqpoint{1.735548in}{1.757318in}}%
\pgfpathclose%
\pgfusepath{stroke,fill}%
\end{pgfscope}%
\begin{pgfscope}%
\pgfpathrectangle{\pgfqpoint{0.100000in}{0.220728in}}{\pgfqpoint{3.696000in}{3.696000in}}%
\pgfusepath{clip}%
\pgfsetbuttcap%
\pgfsetroundjoin%
\definecolor{currentfill}{rgb}{0.121569,0.466667,0.705882}%
\pgfsetfillcolor{currentfill}%
\pgfsetfillopacity{0.879607}%
\pgfsetlinewidth{1.003750pt}%
\definecolor{currentstroke}{rgb}{0.121569,0.466667,0.705882}%
\pgfsetstrokecolor{currentstroke}%
\pgfsetstrokeopacity{0.879607}%
\pgfsetdash{}{0pt}%
\pgfpathmoveto{\pgfqpoint{2.738348in}{1.976149in}}%
\pgfpathcurveto{\pgfqpoint{2.746584in}{1.976149in}}{\pgfqpoint{2.754484in}{1.979422in}}{\pgfqpoint{2.760308in}{1.985245in}}%
\pgfpathcurveto{\pgfqpoint{2.766132in}{1.991069in}}{\pgfqpoint{2.769404in}{1.998969in}}{\pgfqpoint{2.769404in}{2.007206in}}%
\pgfpathcurveto{\pgfqpoint{2.769404in}{2.015442in}}{\pgfqpoint{2.766132in}{2.023342in}}{\pgfqpoint{2.760308in}{2.029166in}}%
\pgfpathcurveto{\pgfqpoint{2.754484in}{2.034990in}}{\pgfqpoint{2.746584in}{2.038262in}}{\pgfqpoint{2.738348in}{2.038262in}}%
\pgfpathcurveto{\pgfqpoint{2.730112in}{2.038262in}}{\pgfqpoint{2.722212in}{2.034990in}}{\pgfqpoint{2.716388in}{2.029166in}}%
\pgfpathcurveto{\pgfqpoint{2.710564in}{2.023342in}}{\pgfqpoint{2.707291in}{2.015442in}}{\pgfqpoint{2.707291in}{2.007206in}}%
\pgfpathcurveto{\pgfqpoint{2.707291in}{1.998969in}}{\pgfqpoint{2.710564in}{1.991069in}}{\pgfqpoint{2.716388in}{1.985245in}}%
\pgfpathcurveto{\pgfqpoint{2.722212in}{1.979422in}}{\pgfqpoint{2.730112in}{1.976149in}}{\pgfqpoint{2.738348in}{1.976149in}}%
\pgfpathclose%
\pgfusepath{stroke,fill}%
\end{pgfscope}%
\begin{pgfscope}%
\pgfpathrectangle{\pgfqpoint{0.100000in}{0.220728in}}{\pgfqpoint{3.696000in}{3.696000in}}%
\pgfusepath{clip}%
\pgfsetbuttcap%
\pgfsetroundjoin%
\definecolor{currentfill}{rgb}{0.121569,0.466667,0.705882}%
\pgfsetfillcolor{currentfill}%
\pgfsetfillopacity{0.881199}%
\pgfsetlinewidth{1.003750pt}%
\definecolor{currentstroke}{rgb}{0.121569,0.466667,0.705882}%
\pgfsetstrokecolor{currentstroke}%
\pgfsetstrokeopacity{0.881199}%
\pgfsetdash{}{0pt}%
\pgfpathmoveto{\pgfqpoint{1.754474in}{1.748034in}}%
\pgfpathcurveto{\pgfqpoint{1.762710in}{1.748034in}}{\pgfqpoint{1.770610in}{1.751306in}}{\pgfqpoint{1.776434in}{1.757130in}}%
\pgfpathcurveto{\pgfqpoint{1.782258in}{1.762954in}}{\pgfqpoint{1.785530in}{1.770854in}}{\pgfqpoint{1.785530in}{1.779090in}}%
\pgfpathcurveto{\pgfqpoint{1.785530in}{1.787326in}}{\pgfqpoint{1.782258in}{1.795227in}}{\pgfqpoint{1.776434in}{1.801050in}}%
\pgfpathcurveto{\pgfqpoint{1.770610in}{1.806874in}}{\pgfqpoint{1.762710in}{1.810147in}}{\pgfqpoint{1.754474in}{1.810147in}}%
\pgfpathcurveto{\pgfqpoint{1.746238in}{1.810147in}}{\pgfqpoint{1.738337in}{1.806874in}}{\pgfqpoint{1.732514in}{1.801050in}}%
\pgfpathcurveto{\pgfqpoint{1.726690in}{1.795227in}}{\pgfqpoint{1.723417in}{1.787326in}}{\pgfqpoint{1.723417in}{1.779090in}}%
\pgfpathcurveto{\pgfqpoint{1.723417in}{1.770854in}}{\pgfqpoint{1.726690in}{1.762954in}}{\pgfqpoint{1.732514in}{1.757130in}}%
\pgfpathcurveto{\pgfqpoint{1.738337in}{1.751306in}}{\pgfqpoint{1.746238in}{1.748034in}}{\pgfqpoint{1.754474in}{1.748034in}}%
\pgfpathclose%
\pgfusepath{stroke,fill}%
\end{pgfscope}%
\begin{pgfscope}%
\pgfpathrectangle{\pgfqpoint{0.100000in}{0.220728in}}{\pgfqpoint{3.696000in}{3.696000in}}%
\pgfusepath{clip}%
\pgfsetbuttcap%
\pgfsetroundjoin%
\definecolor{currentfill}{rgb}{0.121569,0.466667,0.705882}%
\pgfsetfillcolor{currentfill}%
\pgfsetfillopacity{0.883777}%
\pgfsetlinewidth{1.003750pt}%
\definecolor{currentstroke}{rgb}{0.121569,0.466667,0.705882}%
\pgfsetstrokecolor{currentstroke}%
\pgfsetstrokeopacity{0.883777}%
\pgfsetdash{}{0pt}%
\pgfpathmoveto{\pgfqpoint{1.768765in}{1.737278in}}%
\pgfpathcurveto{\pgfqpoint{1.777001in}{1.737278in}}{\pgfqpoint{1.784902in}{1.740551in}}{\pgfqpoint{1.790725in}{1.746375in}}%
\pgfpathcurveto{\pgfqpoint{1.796549in}{1.752199in}}{\pgfqpoint{1.799822in}{1.760099in}}{\pgfqpoint{1.799822in}{1.768335in}}%
\pgfpathcurveto{\pgfqpoint{1.799822in}{1.776571in}}{\pgfqpoint{1.796549in}{1.784471in}}{\pgfqpoint{1.790725in}{1.790295in}}%
\pgfpathcurveto{\pgfqpoint{1.784902in}{1.796119in}}{\pgfqpoint{1.777001in}{1.799391in}}{\pgfqpoint{1.768765in}{1.799391in}}%
\pgfpathcurveto{\pgfqpoint{1.760529in}{1.799391in}}{\pgfqpoint{1.752629in}{1.796119in}}{\pgfqpoint{1.746805in}{1.790295in}}%
\pgfpathcurveto{\pgfqpoint{1.740981in}{1.784471in}}{\pgfqpoint{1.737709in}{1.776571in}}{\pgfqpoint{1.737709in}{1.768335in}}%
\pgfpathcurveto{\pgfqpoint{1.737709in}{1.760099in}}{\pgfqpoint{1.740981in}{1.752199in}}{\pgfqpoint{1.746805in}{1.746375in}}%
\pgfpathcurveto{\pgfqpoint{1.752629in}{1.740551in}}{\pgfqpoint{1.760529in}{1.737278in}}{\pgfqpoint{1.768765in}{1.737278in}}%
\pgfpathclose%
\pgfusepath{stroke,fill}%
\end{pgfscope}%
\begin{pgfscope}%
\pgfpathrectangle{\pgfqpoint{0.100000in}{0.220728in}}{\pgfqpoint{3.696000in}{3.696000in}}%
\pgfusepath{clip}%
\pgfsetbuttcap%
\pgfsetroundjoin%
\definecolor{currentfill}{rgb}{0.121569,0.466667,0.705882}%
\pgfsetfillcolor{currentfill}%
\pgfsetfillopacity{0.884690}%
\pgfsetlinewidth{1.003750pt}%
\definecolor{currentstroke}{rgb}{0.121569,0.466667,0.705882}%
\pgfsetstrokecolor{currentstroke}%
\pgfsetstrokeopacity{0.884690}%
\pgfsetdash{}{0pt}%
\pgfpathmoveto{\pgfqpoint{2.719958in}{1.950503in}}%
\pgfpathcurveto{\pgfqpoint{2.728194in}{1.950503in}}{\pgfqpoint{2.736094in}{1.953775in}}{\pgfqpoint{2.741918in}{1.959599in}}%
\pgfpathcurveto{\pgfqpoint{2.747742in}{1.965423in}}{\pgfqpoint{2.751014in}{1.973323in}}{\pgfqpoint{2.751014in}{1.981559in}}%
\pgfpathcurveto{\pgfqpoint{2.751014in}{1.989796in}}{\pgfqpoint{2.747742in}{1.997696in}}{\pgfqpoint{2.741918in}{2.003520in}}%
\pgfpathcurveto{\pgfqpoint{2.736094in}{2.009344in}}{\pgfqpoint{2.728194in}{2.012616in}}{\pgfqpoint{2.719958in}{2.012616in}}%
\pgfpathcurveto{\pgfqpoint{2.711721in}{2.012616in}}{\pgfqpoint{2.703821in}{2.009344in}}{\pgfqpoint{2.697997in}{2.003520in}}%
\pgfpathcurveto{\pgfqpoint{2.692173in}{1.997696in}}{\pgfqpoint{2.688901in}{1.989796in}}{\pgfqpoint{2.688901in}{1.981559in}}%
\pgfpathcurveto{\pgfqpoint{2.688901in}{1.973323in}}{\pgfqpoint{2.692173in}{1.965423in}}{\pgfqpoint{2.697997in}{1.959599in}}%
\pgfpathcurveto{\pgfqpoint{2.703821in}{1.953775in}}{\pgfqpoint{2.711721in}{1.950503in}}{\pgfqpoint{2.719958in}{1.950503in}}%
\pgfpathclose%
\pgfusepath{stroke,fill}%
\end{pgfscope}%
\begin{pgfscope}%
\pgfpathrectangle{\pgfqpoint{0.100000in}{0.220728in}}{\pgfqpoint{3.696000in}{3.696000in}}%
\pgfusepath{clip}%
\pgfsetbuttcap%
\pgfsetroundjoin%
\definecolor{currentfill}{rgb}{0.121569,0.466667,0.705882}%
\pgfsetfillcolor{currentfill}%
\pgfsetfillopacity{0.886261}%
\pgfsetlinewidth{1.003750pt}%
\definecolor{currentstroke}{rgb}{0.121569,0.466667,0.705882}%
\pgfsetstrokecolor{currentstroke}%
\pgfsetstrokeopacity{0.886261}%
\pgfsetdash{}{0pt}%
\pgfpathmoveto{\pgfqpoint{1.781823in}{1.731043in}}%
\pgfpathcurveto{\pgfqpoint{1.790059in}{1.731043in}}{\pgfqpoint{1.797959in}{1.734315in}}{\pgfqpoint{1.803783in}{1.740139in}}%
\pgfpathcurveto{\pgfqpoint{1.809607in}{1.745963in}}{\pgfqpoint{1.812879in}{1.753863in}}{\pgfqpoint{1.812879in}{1.762099in}}%
\pgfpathcurveto{\pgfqpoint{1.812879in}{1.770336in}}{\pgfqpoint{1.809607in}{1.778236in}}{\pgfqpoint{1.803783in}{1.784060in}}%
\pgfpathcurveto{\pgfqpoint{1.797959in}{1.789884in}}{\pgfqpoint{1.790059in}{1.793156in}}{\pgfqpoint{1.781823in}{1.793156in}}%
\pgfpathcurveto{\pgfqpoint{1.773586in}{1.793156in}}{\pgfqpoint{1.765686in}{1.789884in}}{\pgfqpoint{1.759862in}{1.784060in}}%
\pgfpathcurveto{\pgfqpoint{1.754038in}{1.778236in}}{\pgfqpoint{1.750766in}{1.770336in}}{\pgfqpoint{1.750766in}{1.762099in}}%
\pgfpathcurveto{\pgfqpoint{1.750766in}{1.753863in}}{\pgfqpoint{1.754038in}{1.745963in}}{\pgfqpoint{1.759862in}{1.740139in}}%
\pgfpathcurveto{\pgfqpoint{1.765686in}{1.734315in}}{\pgfqpoint{1.773586in}{1.731043in}}{\pgfqpoint{1.781823in}{1.731043in}}%
\pgfpathclose%
\pgfusepath{stroke,fill}%
\end{pgfscope}%
\begin{pgfscope}%
\pgfpathrectangle{\pgfqpoint{0.100000in}{0.220728in}}{\pgfqpoint{3.696000in}{3.696000in}}%
\pgfusepath{clip}%
\pgfsetbuttcap%
\pgfsetroundjoin%
\definecolor{currentfill}{rgb}{0.121569,0.466667,0.705882}%
\pgfsetfillcolor{currentfill}%
\pgfsetfillopacity{0.887851}%
\pgfsetlinewidth{1.003750pt}%
\definecolor{currentstroke}{rgb}{0.121569,0.466667,0.705882}%
\pgfsetstrokecolor{currentstroke}%
\pgfsetstrokeopacity{0.887851}%
\pgfsetdash{}{0pt}%
\pgfpathmoveto{\pgfqpoint{2.710986in}{1.936517in}}%
\pgfpathcurveto{\pgfqpoint{2.719222in}{1.936517in}}{\pgfqpoint{2.727122in}{1.939789in}}{\pgfqpoint{2.732946in}{1.945613in}}%
\pgfpathcurveto{\pgfqpoint{2.738770in}{1.951437in}}{\pgfqpoint{2.742042in}{1.959337in}}{\pgfqpoint{2.742042in}{1.967573in}}%
\pgfpathcurveto{\pgfqpoint{2.742042in}{1.975809in}}{\pgfqpoint{2.738770in}{1.983709in}}{\pgfqpoint{2.732946in}{1.989533in}}%
\pgfpathcurveto{\pgfqpoint{2.727122in}{1.995357in}}{\pgfqpoint{2.719222in}{1.998630in}}{\pgfqpoint{2.710986in}{1.998630in}}%
\pgfpathcurveto{\pgfqpoint{2.702749in}{1.998630in}}{\pgfqpoint{2.694849in}{1.995357in}}{\pgfqpoint{2.689025in}{1.989533in}}%
\pgfpathcurveto{\pgfqpoint{2.683201in}{1.983709in}}{\pgfqpoint{2.679929in}{1.975809in}}{\pgfqpoint{2.679929in}{1.967573in}}%
\pgfpathcurveto{\pgfqpoint{2.679929in}{1.959337in}}{\pgfqpoint{2.683201in}{1.951437in}}{\pgfqpoint{2.689025in}{1.945613in}}%
\pgfpathcurveto{\pgfqpoint{2.694849in}{1.939789in}}{\pgfqpoint{2.702749in}{1.936517in}}{\pgfqpoint{2.710986in}{1.936517in}}%
\pgfpathclose%
\pgfusepath{stroke,fill}%
\end{pgfscope}%
\begin{pgfscope}%
\pgfpathrectangle{\pgfqpoint{0.100000in}{0.220728in}}{\pgfqpoint{3.696000in}{3.696000in}}%
\pgfusepath{clip}%
\pgfsetbuttcap%
\pgfsetroundjoin%
\definecolor{currentfill}{rgb}{0.121569,0.466667,0.705882}%
\pgfsetfillcolor{currentfill}%
\pgfsetfillopacity{0.888322}%
\pgfsetlinewidth{1.003750pt}%
\definecolor{currentstroke}{rgb}{0.121569,0.466667,0.705882}%
\pgfsetstrokecolor{currentstroke}%
\pgfsetstrokeopacity{0.888322}%
\pgfsetdash{}{0pt}%
\pgfpathmoveto{\pgfqpoint{1.793426in}{1.726730in}}%
\pgfpathcurveto{\pgfqpoint{1.801662in}{1.726730in}}{\pgfqpoint{1.809562in}{1.730002in}}{\pgfqpoint{1.815386in}{1.735826in}}%
\pgfpathcurveto{\pgfqpoint{1.821210in}{1.741650in}}{\pgfqpoint{1.824482in}{1.749550in}}{\pgfqpoint{1.824482in}{1.757786in}}%
\pgfpathcurveto{\pgfqpoint{1.824482in}{1.766023in}}{\pgfqpoint{1.821210in}{1.773923in}}{\pgfqpoint{1.815386in}{1.779747in}}%
\pgfpathcurveto{\pgfqpoint{1.809562in}{1.785570in}}{\pgfqpoint{1.801662in}{1.788843in}}{\pgfqpoint{1.793426in}{1.788843in}}%
\pgfpathcurveto{\pgfqpoint{1.785189in}{1.788843in}}{\pgfqpoint{1.777289in}{1.785570in}}{\pgfqpoint{1.771465in}{1.779747in}}%
\pgfpathcurveto{\pgfqpoint{1.765642in}{1.773923in}}{\pgfqpoint{1.762369in}{1.766023in}}{\pgfqpoint{1.762369in}{1.757786in}}%
\pgfpathcurveto{\pgfqpoint{1.762369in}{1.749550in}}{\pgfqpoint{1.765642in}{1.741650in}}{\pgfqpoint{1.771465in}{1.735826in}}%
\pgfpathcurveto{\pgfqpoint{1.777289in}{1.730002in}}{\pgfqpoint{1.785189in}{1.726730in}}{\pgfqpoint{1.793426in}{1.726730in}}%
\pgfpathclose%
\pgfusepath{stroke,fill}%
\end{pgfscope}%
\begin{pgfscope}%
\pgfpathrectangle{\pgfqpoint{0.100000in}{0.220728in}}{\pgfqpoint{3.696000in}{3.696000in}}%
\pgfusepath{clip}%
\pgfsetbuttcap%
\pgfsetroundjoin%
\definecolor{currentfill}{rgb}{0.121569,0.466667,0.705882}%
\pgfsetfillcolor{currentfill}%
\pgfsetfillopacity{0.889424}%
\pgfsetlinewidth{1.003750pt}%
\definecolor{currentstroke}{rgb}{0.121569,0.466667,0.705882}%
\pgfsetstrokecolor{currentstroke}%
\pgfsetstrokeopacity{0.889424}%
\pgfsetdash{}{0pt}%
\pgfpathmoveto{\pgfqpoint{2.706029in}{1.928048in}}%
\pgfpathcurveto{\pgfqpoint{2.714266in}{1.928048in}}{\pgfqpoint{2.722166in}{1.931321in}}{\pgfqpoint{2.727990in}{1.937145in}}%
\pgfpathcurveto{\pgfqpoint{2.733814in}{1.942969in}}{\pgfqpoint{2.737086in}{1.950869in}}{\pgfqpoint{2.737086in}{1.959105in}}%
\pgfpathcurveto{\pgfqpoint{2.737086in}{1.967341in}}{\pgfqpoint{2.733814in}{1.975241in}}{\pgfqpoint{2.727990in}{1.981065in}}%
\pgfpathcurveto{\pgfqpoint{2.722166in}{1.986889in}}{\pgfqpoint{2.714266in}{1.990161in}}{\pgfqpoint{2.706029in}{1.990161in}}%
\pgfpathcurveto{\pgfqpoint{2.697793in}{1.990161in}}{\pgfqpoint{2.689893in}{1.986889in}}{\pgfqpoint{2.684069in}{1.981065in}}%
\pgfpathcurveto{\pgfqpoint{2.678245in}{1.975241in}}{\pgfqpoint{2.674973in}{1.967341in}}{\pgfqpoint{2.674973in}{1.959105in}}%
\pgfpathcurveto{\pgfqpoint{2.674973in}{1.950869in}}{\pgfqpoint{2.678245in}{1.942969in}}{\pgfqpoint{2.684069in}{1.937145in}}%
\pgfpathcurveto{\pgfqpoint{2.689893in}{1.931321in}}{\pgfqpoint{2.697793in}{1.928048in}}{\pgfqpoint{2.706029in}{1.928048in}}%
\pgfpathclose%
\pgfusepath{stroke,fill}%
\end{pgfscope}%
\begin{pgfscope}%
\pgfpathrectangle{\pgfqpoint{0.100000in}{0.220728in}}{\pgfqpoint{3.696000in}{3.696000in}}%
\pgfusepath{clip}%
\pgfsetbuttcap%
\pgfsetroundjoin%
\definecolor{currentfill}{rgb}{0.121569,0.466667,0.705882}%
\pgfsetfillcolor{currentfill}%
\pgfsetfillopacity{0.889976}%
\pgfsetlinewidth{1.003750pt}%
\definecolor{currentstroke}{rgb}{0.121569,0.466667,0.705882}%
\pgfsetstrokecolor{currentstroke}%
\pgfsetstrokeopacity{0.889976}%
\pgfsetdash{}{0pt}%
\pgfpathmoveto{\pgfqpoint{1.802344in}{1.722287in}}%
\pgfpathcurveto{\pgfqpoint{1.810581in}{1.722287in}}{\pgfqpoint{1.818481in}{1.725560in}}{\pgfqpoint{1.824305in}{1.731384in}}%
\pgfpathcurveto{\pgfqpoint{1.830129in}{1.737208in}}{\pgfqpoint{1.833401in}{1.745108in}}{\pgfqpoint{1.833401in}{1.753344in}}%
\pgfpathcurveto{\pgfqpoint{1.833401in}{1.761580in}}{\pgfqpoint{1.830129in}{1.769480in}}{\pgfqpoint{1.824305in}{1.775304in}}%
\pgfpathcurveto{\pgfqpoint{1.818481in}{1.781128in}}{\pgfqpoint{1.810581in}{1.784400in}}{\pgfqpoint{1.802344in}{1.784400in}}%
\pgfpathcurveto{\pgfqpoint{1.794108in}{1.784400in}}{\pgfqpoint{1.786208in}{1.781128in}}{\pgfqpoint{1.780384in}{1.775304in}}%
\pgfpathcurveto{\pgfqpoint{1.774560in}{1.769480in}}{\pgfqpoint{1.771288in}{1.761580in}}{\pgfqpoint{1.771288in}{1.753344in}}%
\pgfpathcurveto{\pgfqpoint{1.771288in}{1.745108in}}{\pgfqpoint{1.774560in}{1.737208in}}{\pgfqpoint{1.780384in}{1.731384in}}%
\pgfpathcurveto{\pgfqpoint{1.786208in}{1.725560in}}{\pgfqpoint{1.794108in}{1.722287in}}{\pgfqpoint{1.802344in}{1.722287in}}%
\pgfpathclose%
\pgfusepath{stroke,fill}%
\end{pgfscope}%
\begin{pgfscope}%
\pgfpathrectangle{\pgfqpoint{0.100000in}{0.220728in}}{\pgfqpoint{3.696000in}{3.696000in}}%
\pgfusepath{clip}%
\pgfsetbuttcap%
\pgfsetroundjoin%
\definecolor{currentfill}{rgb}{0.121569,0.466667,0.705882}%
\pgfsetfillcolor{currentfill}%
\pgfsetfillopacity{0.890271}%
\pgfsetlinewidth{1.003750pt}%
\definecolor{currentstroke}{rgb}{0.121569,0.466667,0.705882}%
\pgfsetstrokecolor{currentstroke}%
\pgfsetstrokeopacity{0.890271}%
\pgfsetdash{}{0pt}%
\pgfpathmoveto{\pgfqpoint{2.702841in}{1.923926in}}%
\pgfpathcurveto{\pgfqpoint{2.711077in}{1.923926in}}{\pgfqpoint{2.718978in}{1.927198in}}{\pgfqpoint{2.724801in}{1.933022in}}%
\pgfpathcurveto{\pgfqpoint{2.730625in}{1.938846in}}{\pgfqpoint{2.733898in}{1.946746in}}{\pgfqpoint{2.733898in}{1.954982in}}%
\pgfpathcurveto{\pgfqpoint{2.733898in}{1.963219in}}{\pgfqpoint{2.730625in}{1.971119in}}{\pgfqpoint{2.724801in}{1.976943in}}%
\pgfpathcurveto{\pgfqpoint{2.718978in}{1.982767in}}{\pgfqpoint{2.711077in}{1.986039in}}{\pgfqpoint{2.702841in}{1.986039in}}%
\pgfpathcurveto{\pgfqpoint{2.694605in}{1.986039in}}{\pgfqpoint{2.686705in}{1.982767in}}{\pgfqpoint{2.680881in}{1.976943in}}%
\pgfpathcurveto{\pgfqpoint{2.675057in}{1.971119in}}{\pgfqpoint{2.671785in}{1.963219in}}{\pgfqpoint{2.671785in}{1.954982in}}%
\pgfpathcurveto{\pgfqpoint{2.671785in}{1.946746in}}{\pgfqpoint{2.675057in}{1.938846in}}{\pgfqpoint{2.680881in}{1.933022in}}%
\pgfpathcurveto{\pgfqpoint{2.686705in}{1.927198in}}{\pgfqpoint{2.694605in}{1.923926in}}{\pgfqpoint{2.702841in}{1.923926in}}%
\pgfpathclose%
\pgfusepath{stroke,fill}%
\end{pgfscope}%
\begin{pgfscope}%
\pgfpathrectangle{\pgfqpoint{0.100000in}{0.220728in}}{\pgfqpoint{3.696000in}{3.696000in}}%
\pgfusepath{clip}%
\pgfsetbuttcap%
\pgfsetroundjoin%
\definecolor{currentfill}{rgb}{0.121569,0.466667,0.705882}%
\pgfsetfillcolor{currentfill}%
\pgfsetfillopacity{0.890748}%
\pgfsetlinewidth{1.003750pt}%
\definecolor{currentstroke}{rgb}{0.121569,0.466667,0.705882}%
\pgfsetstrokecolor{currentstroke}%
\pgfsetstrokeopacity{0.890748}%
\pgfsetdash{}{0pt}%
\pgfpathmoveto{\pgfqpoint{2.701303in}{1.921410in}}%
\pgfpathcurveto{\pgfqpoint{2.709539in}{1.921410in}}{\pgfqpoint{2.717439in}{1.924683in}}{\pgfqpoint{2.723263in}{1.930507in}}%
\pgfpathcurveto{\pgfqpoint{2.729087in}{1.936330in}}{\pgfqpoint{2.732359in}{1.944231in}}{\pgfqpoint{2.732359in}{1.952467in}}%
\pgfpathcurveto{\pgfqpoint{2.732359in}{1.960703in}}{\pgfqpoint{2.729087in}{1.968603in}}{\pgfqpoint{2.723263in}{1.974427in}}%
\pgfpathcurveto{\pgfqpoint{2.717439in}{1.980251in}}{\pgfqpoint{2.709539in}{1.983523in}}{\pgfqpoint{2.701303in}{1.983523in}}%
\pgfpathcurveto{\pgfqpoint{2.693066in}{1.983523in}}{\pgfqpoint{2.685166in}{1.980251in}}{\pgfqpoint{2.679342in}{1.974427in}}%
\pgfpathcurveto{\pgfqpoint{2.673518in}{1.968603in}}{\pgfqpoint{2.670246in}{1.960703in}}{\pgfqpoint{2.670246in}{1.952467in}}%
\pgfpathcurveto{\pgfqpoint{2.670246in}{1.944231in}}{\pgfqpoint{2.673518in}{1.936330in}}{\pgfqpoint{2.679342in}{1.930507in}}%
\pgfpathcurveto{\pgfqpoint{2.685166in}{1.924683in}}{\pgfqpoint{2.693066in}{1.921410in}}{\pgfqpoint{2.701303in}{1.921410in}}%
\pgfpathclose%
\pgfusepath{stroke,fill}%
\end{pgfscope}%
\begin{pgfscope}%
\pgfpathrectangle{\pgfqpoint{0.100000in}{0.220728in}}{\pgfqpoint{3.696000in}{3.696000in}}%
\pgfusepath{clip}%
\pgfsetbuttcap%
\pgfsetroundjoin%
\definecolor{currentfill}{rgb}{0.121569,0.466667,0.705882}%
\pgfsetfillcolor{currentfill}%
\pgfsetfillopacity{0.891023}%
\pgfsetlinewidth{1.003750pt}%
\definecolor{currentstroke}{rgb}{0.121569,0.466667,0.705882}%
\pgfsetstrokecolor{currentstroke}%
\pgfsetstrokeopacity{0.891023}%
\pgfsetdash{}{0pt}%
\pgfpathmoveto{\pgfqpoint{2.700407in}{1.920154in}}%
\pgfpathcurveto{\pgfqpoint{2.708643in}{1.920154in}}{\pgfqpoint{2.716544in}{1.923427in}}{\pgfqpoint{2.722367in}{1.929251in}}%
\pgfpathcurveto{\pgfqpoint{2.728191in}{1.935074in}}{\pgfqpoint{2.731464in}{1.942974in}}{\pgfqpoint{2.731464in}{1.951211in}}%
\pgfpathcurveto{\pgfqpoint{2.731464in}{1.959447in}}{\pgfqpoint{2.728191in}{1.967347in}}{\pgfqpoint{2.722367in}{1.973171in}}%
\pgfpathcurveto{\pgfqpoint{2.716544in}{1.978995in}}{\pgfqpoint{2.708643in}{1.982267in}}{\pgfqpoint{2.700407in}{1.982267in}}%
\pgfpathcurveto{\pgfqpoint{2.692171in}{1.982267in}}{\pgfqpoint{2.684271in}{1.978995in}}{\pgfqpoint{2.678447in}{1.973171in}}%
\pgfpathcurveto{\pgfqpoint{2.672623in}{1.967347in}}{\pgfqpoint{2.669351in}{1.959447in}}{\pgfqpoint{2.669351in}{1.951211in}}%
\pgfpathcurveto{\pgfqpoint{2.669351in}{1.942974in}}{\pgfqpoint{2.672623in}{1.935074in}}{\pgfqpoint{2.678447in}{1.929251in}}%
\pgfpathcurveto{\pgfqpoint{2.684271in}{1.923427in}}{\pgfqpoint{2.692171in}{1.920154in}}{\pgfqpoint{2.700407in}{1.920154in}}%
\pgfpathclose%
\pgfusepath{stroke,fill}%
\end{pgfscope}%
\begin{pgfscope}%
\pgfpathrectangle{\pgfqpoint{0.100000in}{0.220728in}}{\pgfqpoint{3.696000in}{3.696000in}}%
\pgfusepath{clip}%
\pgfsetbuttcap%
\pgfsetroundjoin%
\definecolor{currentfill}{rgb}{0.121569,0.466667,0.705882}%
\pgfsetfillcolor{currentfill}%
\pgfsetfillopacity{0.891172}%
\pgfsetlinewidth{1.003750pt}%
\definecolor{currentstroke}{rgb}{0.121569,0.466667,0.705882}%
\pgfsetstrokecolor{currentstroke}%
\pgfsetstrokeopacity{0.891172}%
\pgfsetdash{}{0pt}%
\pgfpathmoveto{\pgfqpoint{2.700002in}{1.919345in}}%
\pgfpathcurveto{\pgfqpoint{2.708238in}{1.919345in}}{\pgfqpoint{2.716138in}{1.922617in}}{\pgfqpoint{2.721962in}{1.928441in}}%
\pgfpathcurveto{\pgfqpoint{2.727786in}{1.934265in}}{\pgfqpoint{2.731058in}{1.942165in}}{\pgfqpoint{2.731058in}{1.950402in}}%
\pgfpathcurveto{\pgfqpoint{2.731058in}{1.958638in}}{\pgfqpoint{2.727786in}{1.966538in}}{\pgfqpoint{2.721962in}{1.972362in}}%
\pgfpathcurveto{\pgfqpoint{2.716138in}{1.978186in}}{\pgfqpoint{2.708238in}{1.981458in}}{\pgfqpoint{2.700002in}{1.981458in}}%
\pgfpathcurveto{\pgfqpoint{2.691766in}{1.981458in}}{\pgfqpoint{2.683866in}{1.978186in}}{\pgfqpoint{2.678042in}{1.972362in}}%
\pgfpathcurveto{\pgfqpoint{2.672218in}{1.966538in}}{\pgfqpoint{2.668945in}{1.958638in}}{\pgfqpoint{2.668945in}{1.950402in}}%
\pgfpathcurveto{\pgfqpoint{2.668945in}{1.942165in}}{\pgfqpoint{2.672218in}{1.934265in}}{\pgfqpoint{2.678042in}{1.928441in}}%
\pgfpathcurveto{\pgfqpoint{2.683866in}{1.922617in}}{\pgfqpoint{2.691766in}{1.919345in}}{\pgfqpoint{2.700002in}{1.919345in}}%
\pgfpathclose%
\pgfusepath{stroke,fill}%
\end{pgfscope}%
\begin{pgfscope}%
\pgfpathrectangle{\pgfqpoint{0.100000in}{0.220728in}}{\pgfqpoint{3.696000in}{3.696000in}}%
\pgfusepath{clip}%
\pgfsetbuttcap%
\pgfsetroundjoin%
\definecolor{currentfill}{rgb}{0.121569,0.466667,0.705882}%
\pgfsetfillcolor{currentfill}%
\pgfsetfillopacity{0.892303}%
\pgfsetlinewidth{1.003750pt}%
\definecolor{currentstroke}{rgb}{0.121569,0.466667,0.705882}%
\pgfsetstrokecolor{currentstroke}%
\pgfsetstrokeopacity{0.892303}%
\pgfsetdash{}{0pt}%
\pgfpathmoveto{\pgfqpoint{2.695755in}{1.914038in}}%
\pgfpathcurveto{\pgfqpoint{2.703991in}{1.914038in}}{\pgfqpoint{2.711891in}{1.917310in}}{\pgfqpoint{2.717715in}{1.923134in}}%
\pgfpathcurveto{\pgfqpoint{2.723539in}{1.928958in}}{\pgfqpoint{2.726811in}{1.936858in}}{\pgfqpoint{2.726811in}{1.945094in}}%
\pgfpathcurveto{\pgfqpoint{2.726811in}{1.953331in}}{\pgfqpoint{2.723539in}{1.961231in}}{\pgfqpoint{2.717715in}{1.967055in}}%
\pgfpathcurveto{\pgfqpoint{2.711891in}{1.972879in}}{\pgfqpoint{2.703991in}{1.976151in}}{\pgfqpoint{2.695755in}{1.976151in}}%
\pgfpathcurveto{\pgfqpoint{2.687519in}{1.976151in}}{\pgfqpoint{2.679618in}{1.972879in}}{\pgfqpoint{2.673795in}{1.967055in}}%
\pgfpathcurveto{\pgfqpoint{2.667971in}{1.961231in}}{\pgfqpoint{2.664698in}{1.953331in}}{\pgfqpoint{2.664698in}{1.945094in}}%
\pgfpathcurveto{\pgfqpoint{2.664698in}{1.936858in}}{\pgfqpoint{2.667971in}{1.928958in}}{\pgfqpoint{2.673795in}{1.923134in}}%
\pgfpathcurveto{\pgfqpoint{2.679618in}{1.917310in}}{\pgfqpoint{2.687519in}{1.914038in}}{\pgfqpoint{2.695755in}{1.914038in}}%
\pgfpathclose%
\pgfusepath{stroke,fill}%
\end{pgfscope}%
\begin{pgfscope}%
\pgfpathrectangle{\pgfqpoint{0.100000in}{0.220728in}}{\pgfqpoint{3.696000in}{3.696000in}}%
\pgfusepath{clip}%
\pgfsetbuttcap%
\pgfsetroundjoin%
\definecolor{currentfill}{rgb}{0.121569,0.466667,0.705882}%
\pgfsetfillcolor{currentfill}%
\pgfsetfillopacity{0.893174}%
\pgfsetlinewidth{1.003750pt}%
\definecolor{currentstroke}{rgb}{0.121569,0.466667,0.705882}%
\pgfsetstrokecolor{currentstroke}%
\pgfsetstrokeopacity{0.893174}%
\pgfsetdash{}{0pt}%
\pgfpathmoveto{\pgfqpoint{1.819268in}{1.717450in}}%
\pgfpathcurveto{\pgfqpoint{1.827504in}{1.717450in}}{\pgfqpoint{1.835404in}{1.720722in}}{\pgfqpoint{1.841228in}{1.726546in}}%
\pgfpathcurveto{\pgfqpoint{1.847052in}{1.732370in}}{\pgfqpoint{1.850324in}{1.740270in}}{\pgfqpoint{1.850324in}{1.748507in}}%
\pgfpathcurveto{\pgfqpoint{1.850324in}{1.756743in}}{\pgfqpoint{1.847052in}{1.764643in}}{\pgfqpoint{1.841228in}{1.770467in}}%
\pgfpathcurveto{\pgfqpoint{1.835404in}{1.776291in}}{\pgfqpoint{1.827504in}{1.779563in}}{\pgfqpoint{1.819268in}{1.779563in}}%
\pgfpathcurveto{\pgfqpoint{1.811031in}{1.779563in}}{\pgfqpoint{1.803131in}{1.776291in}}{\pgfqpoint{1.797307in}{1.770467in}}%
\pgfpathcurveto{\pgfqpoint{1.791484in}{1.764643in}}{\pgfqpoint{1.788211in}{1.756743in}}{\pgfqpoint{1.788211in}{1.748507in}}%
\pgfpathcurveto{\pgfqpoint{1.788211in}{1.740270in}}{\pgfqpoint{1.791484in}{1.732370in}}{\pgfqpoint{1.797307in}{1.726546in}}%
\pgfpathcurveto{\pgfqpoint{1.803131in}{1.720722in}}{\pgfqpoint{1.811031in}{1.717450in}}{\pgfqpoint{1.819268in}{1.717450in}}%
\pgfpathclose%
\pgfusepath{stroke,fill}%
\end{pgfscope}%
\begin{pgfscope}%
\pgfpathrectangle{\pgfqpoint{0.100000in}{0.220728in}}{\pgfqpoint{3.696000in}{3.696000in}}%
\pgfusepath{clip}%
\pgfsetbuttcap%
\pgfsetroundjoin%
\definecolor{currentfill}{rgb}{0.121569,0.466667,0.705882}%
\pgfsetfillcolor{currentfill}%
\pgfsetfillopacity{0.894208}%
\pgfsetlinewidth{1.003750pt}%
\definecolor{currentstroke}{rgb}{0.121569,0.466667,0.705882}%
\pgfsetstrokecolor{currentstroke}%
\pgfsetstrokeopacity{0.894208}%
\pgfsetdash{}{0pt}%
\pgfpathmoveto{\pgfqpoint{2.690072in}{1.902961in}}%
\pgfpathcurveto{\pgfqpoint{2.698309in}{1.902961in}}{\pgfqpoint{2.706209in}{1.906234in}}{\pgfqpoint{2.712033in}{1.912058in}}%
\pgfpathcurveto{\pgfqpoint{2.717857in}{1.917882in}}{\pgfqpoint{2.721129in}{1.925782in}}{\pgfqpoint{2.721129in}{1.934018in}}%
\pgfpathcurveto{\pgfqpoint{2.721129in}{1.942254in}}{\pgfqpoint{2.717857in}{1.950154in}}{\pgfqpoint{2.712033in}{1.955978in}}%
\pgfpathcurveto{\pgfqpoint{2.706209in}{1.961802in}}{\pgfqpoint{2.698309in}{1.965074in}}{\pgfqpoint{2.690072in}{1.965074in}}%
\pgfpathcurveto{\pgfqpoint{2.681836in}{1.965074in}}{\pgfqpoint{2.673936in}{1.961802in}}{\pgfqpoint{2.668112in}{1.955978in}}%
\pgfpathcurveto{\pgfqpoint{2.662288in}{1.950154in}}{\pgfqpoint{2.659016in}{1.942254in}}{\pgfqpoint{2.659016in}{1.934018in}}%
\pgfpathcurveto{\pgfqpoint{2.659016in}{1.925782in}}{\pgfqpoint{2.662288in}{1.917882in}}{\pgfqpoint{2.668112in}{1.912058in}}%
\pgfpathcurveto{\pgfqpoint{2.673936in}{1.906234in}}{\pgfqpoint{2.681836in}{1.902961in}}{\pgfqpoint{2.690072in}{1.902961in}}%
\pgfpathclose%
\pgfusepath{stroke,fill}%
\end{pgfscope}%
\begin{pgfscope}%
\pgfpathrectangle{\pgfqpoint{0.100000in}{0.220728in}}{\pgfqpoint{3.696000in}{3.696000in}}%
\pgfusepath{clip}%
\pgfsetbuttcap%
\pgfsetroundjoin%
\definecolor{currentfill}{rgb}{0.121569,0.466667,0.705882}%
\pgfsetfillcolor{currentfill}%
\pgfsetfillopacity{0.898038}%
\pgfsetlinewidth{1.003750pt}%
\definecolor{currentstroke}{rgb}{0.121569,0.466667,0.705882}%
\pgfsetstrokecolor{currentstroke}%
\pgfsetstrokeopacity{0.898038}%
\pgfsetdash{}{0pt}%
\pgfpathmoveto{\pgfqpoint{2.678907in}{1.888559in}}%
\pgfpathcurveto{\pgfqpoint{2.687143in}{1.888559in}}{\pgfqpoint{2.695043in}{1.891831in}}{\pgfqpoint{2.700867in}{1.897655in}}%
\pgfpathcurveto{\pgfqpoint{2.706691in}{1.903479in}}{\pgfqpoint{2.709963in}{1.911379in}}{\pgfqpoint{2.709963in}{1.919615in}}%
\pgfpathcurveto{\pgfqpoint{2.709963in}{1.927851in}}{\pgfqpoint{2.706691in}{1.935751in}}{\pgfqpoint{2.700867in}{1.941575in}}%
\pgfpathcurveto{\pgfqpoint{2.695043in}{1.947399in}}{\pgfqpoint{2.687143in}{1.950672in}}{\pgfqpoint{2.678907in}{1.950672in}}%
\pgfpathcurveto{\pgfqpoint{2.670670in}{1.950672in}}{\pgfqpoint{2.662770in}{1.947399in}}{\pgfqpoint{2.656946in}{1.941575in}}%
\pgfpathcurveto{\pgfqpoint{2.651122in}{1.935751in}}{\pgfqpoint{2.647850in}{1.927851in}}{\pgfqpoint{2.647850in}{1.919615in}}%
\pgfpathcurveto{\pgfqpoint{2.647850in}{1.911379in}}{\pgfqpoint{2.651122in}{1.903479in}}{\pgfqpoint{2.656946in}{1.897655in}}%
\pgfpathcurveto{\pgfqpoint{2.662770in}{1.891831in}}{\pgfqpoint{2.670670in}{1.888559in}}{\pgfqpoint{2.678907in}{1.888559in}}%
\pgfpathclose%
\pgfusepath{stroke,fill}%
\end{pgfscope}%
\begin{pgfscope}%
\pgfpathrectangle{\pgfqpoint{0.100000in}{0.220728in}}{\pgfqpoint{3.696000in}{3.696000in}}%
\pgfusepath{clip}%
\pgfsetbuttcap%
\pgfsetroundjoin%
\definecolor{currentfill}{rgb}{0.121569,0.466667,0.705882}%
\pgfsetfillcolor{currentfill}%
\pgfsetfillopacity{0.898538}%
\pgfsetlinewidth{1.003750pt}%
\definecolor{currentstroke}{rgb}{0.121569,0.466667,0.705882}%
\pgfsetstrokecolor{currentstroke}%
\pgfsetstrokeopacity{0.898538}%
\pgfsetdash{}{0pt}%
\pgfpathmoveto{\pgfqpoint{1.845619in}{1.694033in}}%
\pgfpathcurveto{\pgfqpoint{1.853855in}{1.694033in}}{\pgfqpoint{1.861756in}{1.697306in}}{\pgfqpoint{1.867579in}{1.703130in}}%
\pgfpathcurveto{\pgfqpoint{1.873403in}{1.708953in}}{\pgfqpoint{1.876676in}{1.716854in}}{\pgfqpoint{1.876676in}{1.725090in}}%
\pgfpathcurveto{\pgfqpoint{1.876676in}{1.733326in}}{\pgfqpoint{1.873403in}{1.741226in}}{\pgfqpoint{1.867579in}{1.747050in}}%
\pgfpathcurveto{\pgfqpoint{1.861756in}{1.752874in}}{\pgfqpoint{1.853855in}{1.756146in}}{\pgfqpoint{1.845619in}{1.756146in}}%
\pgfpathcurveto{\pgfqpoint{1.837383in}{1.756146in}}{\pgfqpoint{1.829483in}{1.752874in}}{\pgfqpoint{1.823659in}{1.747050in}}%
\pgfpathcurveto{\pgfqpoint{1.817835in}{1.741226in}}{\pgfqpoint{1.814563in}{1.733326in}}{\pgfqpoint{1.814563in}{1.725090in}}%
\pgfpathcurveto{\pgfqpoint{1.814563in}{1.716854in}}{\pgfqpoint{1.817835in}{1.708953in}}{\pgfqpoint{1.823659in}{1.703130in}}%
\pgfpathcurveto{\pgfqpoint{1.829483in}{1.697306in}}{\pgfqpoint{1.837383in}{1.694033in}}{\pgfqpoint{1.845619in}{1.694033in}}%
\pgfpathclose%
\pgfusepath{stroke,fill}%
\end{pgfscope}%
\begin{pgfscope}%
\pgfpathrectangle{\pgfqpoint{0.100000in}{0.220728in}}{\pgfqpoint{3.696000in}{3.696000in}}%
\pgfusepath{clip}%
\pgfsetbuttcap%
\pgfsetroundjoin%
\definecolor{currentfill}{rgb}{0.121569,0.466667,0.705882}%
\pgfsetfillcolor{currentfill}%
\pgfsetfillopacity{0.901505}%
\pgfsetlinewidth{1.003750pt}%
\definecolor{currentstroke}{rgb}{0.121569,0.466667,0.705882}%
\pgfsetstrokecolor{currentstroke}%
\pgfsetstrokeopacity{0.901505}%
\pgfsetdash{}{0pt}%
\pgfpathmoveto{\pgfqpoint{2.667353in}{1.867003in}}%
\pgfpathcurveto{\pgfqpoint{2.675589in}{1.867003in}}{\pgfqpoint{2.683489in}{1.870275in}}{\pgfqpoint{2.689313in}{1.876099in}}%
\pgfpathcurveto{\pgfqpoint{2.695137in}{1.881923in}}{\pgfqpoint{2.698409in}{1.889823in}}{\pgfqpoint{2.698409in}{1.898059in}}%
\pgfpathcurveto{\pgfqpoint{2.698409in}{1.906296in}}{\pgfqpoint{2.695137in}{1.914196in}}{\pgfqpoint{2.689313in}{1.920020in}}%
\pgfpathcurveto{\pgfqpoint{2.683489in}{1.925844in}}{\pgfqpoint{2.675589in}{1.929116in}}{\pgfqpoint{2.667353in}{1.929116in}}%
\pgfpathcurveto{\pgfqpoint{2.659117in}{1.929116in}}{\pgfqpoint{2.651216in}{1.925844in}}{\pgfqpoint{2.645393in}{1.920020in}}%
\pgfpathcurveto{\pgfqpoint{2.639569in}{1.914196in}}{\pgfqpoint{2.636296in}{1.906296in}}{\pgfqpoint{2.636296in}{1.898059in}}%
\pgfpathcurveto{\pgfqpoint{2.636296in}{1.889823in}}{\pgfqpoint{2.639569in}{1.881923in}}{\pgfqpoint{2.645393in}{1.876099in}}%
\pgfpathcurveto{\pgfqpoint{2.651216in}{1.870275in}}{\pgfqpoint{2.659117in}{1.867003in}}{\pgfqpoint{2.667353in}{1.867003in}}%
\pgfpathclose%
\pgfusepath{stroke,fill}%
\end{pgfscope}%
\begin{pgfscope}%
\pgfpathrectangle{\pgfqpoint{0.100000in}{0.220728in}}{\pgfqpoint{3.696000in}{3.696000in}}%
\pgfusepath{clip}%
\pgfsetbuttcap%
\pgfsetroundjoin%
\definecolor{currentfill}{rgb}{0.121569,0.466667,0.705882}%
\pgfsetfillcolor{currentfill}%
\pgfsetfillopacity{0.903370}%
\pgfsetlinewidth{1.003750pt}%
\definecolor{currentstroke}{rgb}{0.121569,0.466667,0.705882}%
\pgfsetstrokecolor{currentstroke}%
\pgfsetstrokeopacity{0.903370}%
\pgfsetdash{}{0pt}%
\pgfpathmoveto{\pgfqpoint{1.870981in}{1.682917in}}%
\pgfpathcurveto{\pgfqpoint{1.879217in}{1.682917in}}{\pgfqpoint{1.887117in}{1.686189in}}{\pgfqpoint{1.892941in}{1.692013in}}%
\pgfpathcurveto{\pgfqpoint{1.898765in}{1.697837in}}{\pgfqpoint{1.902037in}{1.705737in}}{\pgfqpoint{1.902037in}{1.713973in}}%
\pgfpathcurveto{\pgfqpoint{1.902037in}{1.722210in}}{\pgfqpoint{1.898765in}{1.730110in}}{\pgfqpoint{1.892941in}{1.735934in}}%
\pgfpathcurveto{\pgfqpoint{1.887117in}{1.741757in}}{\pgfqpoint{1.879217in}{1.745030in}}{\pgfqpoint{1.870981in}{1.745030in}}%
\pgfpathcurveto{\pgfqpoint{1.862745in}{1.745030in}}{\pgfqpoint{1.854845in}{1.741757in}}{\pgfqpoint{1.849021in}{1.735934in}}%
\pgfpathcurveto{\pgfqpoint{1.843197in}{1.730110in}}{\pgfqpoint{1.839924in}{1.722210in}}{\pgfqpoint{1.839924in}{1.713973in}}%
\pgfpathcurveto{\pgfqpoint{1.839924in}{1.705737in}}{\pgfqpoint{1.843197in}{1.697837in}}{\pgfqpoint{1.849021in}{1.692013in}}%
\pgfpathcurveto{\pgfqpoint{1.854845in}{1.686189in}}{\pgfqpoint{1.862745in}{1.682917in}}{\pgfqpoint{1.870981in}{1.682917in}}%
\pgfpathclose%
\pgfusepath{stroke,fill}%
\end{pgfscope}%
\begin{pgfscope}%
\pgfpathrectangle{\pgfqpoint{0.100000in}{0.220728in}}{\pgfqpoint{3.696000in}{3.696000in}}%
\pgfusepath{clip}%
\pgfsetbuttcap%
\pgfsetroundjoin%
\definecolor{currentfill}{rgb}{0.121569,0.466667,0.705882}%
\pgfsetfillcolor{currentfill}%
\pgfsetfillopacity{0.905897}%
\pgfsetlinewidth{1.003750pt}%
\definecolor{currentstroke}{rgb}{0.121569,0.466667,0.705882}%
\pgfsetstrokecolor{currentstroke}%
\pgfsetstrokeopacity{0.905897}%
\pgfsetdash{}{0pt}%
\pgfpathmoveto{\pgfqpoint{2.651120in}{1.843788in}}%
\pgfpathcurveto{\pgfqpoint{2.659356in}{1.843788in}}{\pgfqpoint{2.667256in}{1.847061in}}{\pgfqpoint{2.673080in}{1.852885in}}%
\pgfpathcurveto{\pgfqpoint{2.678904in}{1.858709in}}{\pgfqpoint{2.682176in}{1.866609in}}{\pgfqpoint{2.682176in}{1.874845in}}%
\pgfpathcurveto{\pgfqpoint{2.682176in}{1.883081in}}{\pgfqpoint{2.678904in}{1.890981in}}{\pgfqpoint{2.673080in}{1.896805in}}%
\pgfpathcurveto{\pgfqpoint{2.667256in}{1.902629in}}{\pgfqpoint{2.659356in}{1.905901in}}{\pgfqpoint{2.651120in}{1.905901in}}%
\pgfpathcurveto{\pgfqpoint{2.642884in}{1.905901in}}{\pgfqpoint{2.634984in}{1.902629in}}{\pgfqpoint{2.629160in}{1.896805in}}%
\pgfpathcurveto{\pgfqpoint{2.623336in}{1.890981in}}{\pgfqpoint{2.620063in}{1.883081in}}{\pgfqpoint{2.620063in}{1.874845in}}%
\pgfpathcurveto{\pgfqpoint{2.620063in}{1.866609in}}{\pgfqpoint{2.623336in}{1.858709in}}{\pgfqpoint{2.629160in}{1.852885in}}%
\pgfpathcurveto{\pgfqpoint{2.634984in}{1.847061in}}{\pgfqpoint{2.642884in}{1.843788in}}{\pgfqpoint{2.651120in}{1.843788in}}%
\pgfpathclose%
\pgfusepath{stroke,fill}%
\end{pgfscope}%
\begin{pgfscope}%
\pgfpathrectangle{\pgfqpoint{0.100000in}{0.220728in}}{\pgfqpoint{3.696000in}{3.696000in}}%
\pgfusepath{clip}%
\pgfsetbuttcap%
\pgfsetroundjoin%
\definecolor{currentfill}{rgb}{0.121569,0.466667,0.705882}%
\pgfsetfillcolor{currentfill}%
\pgfsetfillopacity{0.906457}%
\pgfsetlinewidth{1.003750pt}%
\definecolor{currentstroke}{rgb}{0.121569,0.466667,0.705882}%
\pgfsetstrokecolor{currentstroke}%
\pgfsetstrokeopacity{0.906457}%
\pgfsetdash{}{0pt}%
\pgfpathmoveto{\pgfqpoint{1.888988in}{1.670333in}}%
\pgfpathcurveto{\pgfqpoint{1.897224in}{1.670333in}}{\pgfqpoint{1.905124in}{1.673605in}}{\pgfqpoint{1.910948in}{1.679429in}}%
\pgfpathcurveto{\pgfqpoint{1.916772in}{1.685253in}}{\pgfqpoint{1.920044in}{1.693153in}}{\pgfqpoint{1.920044in}{1.701389in}}%
\pgfpathcurveto{\pgfqpoint{1.920044in}{1.709626in}}{\pgfqpoint{1.916772in}{1.717526in}}{\pgfqpoint{1.910948in}{1.723350in}}%
\pgfpathcurveto{\pgfqpoint{1.905124in}{1.729173in}}{\pgfqpoint{1.897224in}{1.732446in}}{\pgfqpoint{1.888988in}{1.732446in}}%
\pgfpathcurveto{\pgfqpoint{1.880752in}{1.732446in}}{\pgfqpoint{1.872852in}{1.729173in}}{\pgfqpoint{1.867028in}{1.723350in}}%
\pgfpathcurveto{\pgfqpoint{1.861204in}{1.717526in}}{\pgfqpoint{1.857931in}{1.709626in}}{\pgfqpoint{1.857931in}{1.701389in}}%
\pgfpathcurveto{\pgfqpoint{1.857931in}{1.693153in}}{\pgfqpoint{1.861204in}{1.685253in}}{\pgfqpoint{1.867028in}{1.679429in}}%
\pgfpathcurveto{\pgfqpoint{1.872852in}{1.673605in}}{\pgfqpoint{1.880752in}{1.670333in}}{\pgfqpoint{1.888988in}{1.670333in}}%
\pgfpathclose%
\pgfusepath{stroke,fill}%
\end{pgfscope}%
\begin{pgfscope}%
\pgfpathrectangle{\pgfqpoint{0.100000in}{0.220728in}}{\pgfqpoint{3.696000in}{3.696000in}}%
\pgfusepath{clip}%
\pgfsetbuttcap%
\pgfsetroundjoin%
\definecolor{currentfill}{rgb}{0.121569,0.466667,0.705882}%
\pgfsetfillcolor{currentfill}%
\pgfsetfillopacity{0.909294}%
\pgfsetlinewidth{1.003750pt}%
\definecolor{currentstroke}{rgb}{0.121569,0.466667,0.705882}%
\pgfsetstrokecolor{currentstroke}%
\pgfsetstrokeopacity{0.909294}%
\pgfsetdash{}{0pt}%
\pgfpathmoveto{\pgfqpoint{1.902352in}{1.662177in}}%
\pgfpathcurveto{\pgfqpoint{1.910588in}{1.662177in}}{\pgfqpoint{1.918488in}{1.665450in}}{\pgfqpoint{1.924312in}{1.671274in}}%
\pgfpathcurveto{\pgfqpoint{1.930136in}{1.677098in}}{\pgfqpoint{1.933408in}{1.684998in}}{\pgfqpoint{1.933408in}{1.693234in}}%
\pgfpathcurveto{\pgfqpoint{1.933408in}{1.701470in}}{\pgfqpoint{1.930136in}{1.709370in}}{\pgfqpoint{1.924312in}{1.715194in}}%
\pgfpathcurveto{\pgfqpoint{1.918488in}{1.721018in}}{\pgfqpoint{1.910588in}{1.724290in}}{\pgfqpoint{1.902352in}{1.724290in}}%
\pgfpathcurveto{\pgfqpoint{1.894115in}{1.724290in}}{\pgfqpoint{1.886215in}{1.721018in}}{\pgfqpoint{1.880391in}{1.715194in}}%
\pgfpathcurveto{\pgfqpoint{1.874567in}{1.709370in}}{\pgfqpoint{1.871295in}{1.701470in}}{\pgfqpoint{1.871295in}{1.693234in}}%
\pgfpathcurveto{\pgfqpoint{1.871295in}{1.684998in}}{\pgfqpoint{1.874567in}{1.677098in}}{\pgfqpoint{1.880391in}{1.671274in}}%
\pgfpathcurveto{\pgfqpoint{1.886215in}{1.665450in}}{\pgfqpoint{1.894115in}{1.662177in}}{\pgfqpoint{1.902352in}{1.662177in}}%
\pgfpathclose%
\pgfusepath{stroke,fill}%
\end{pgfscope}%
\begin{pgfscope}%
\pgfpathrectangle{\pgfqpoint{0.100000in}{0.220728in}}{\pgfqpoint{3.696000in}{3.696000in}}%
\pgfusepath{clip}%
\pgfsetbuttcap%
\pgfsetroundjoin%
\definecolor{currentfill}{rgb}{0.121569,0.466667,0.705882}%
\pgfsetfillcolor{currentfill}%
\pgfsetfillopacity{0.911522}%
\pgfsetlinewidth{1.003750pt}%
\definecolor{currentstroke}{rgb}{0.121569,0.466667,0.705882}%
\pgfsetstrokecolor{currentstroke}%
\pgfsetstrokeopacity{0.911522}%
\pgfsetdash{}{0pt}%
\pgfpathmoveto{\pgfqpoint{1.912203in}{1.653512in}}%
\pgfpathcurveto{\pgfqpoint{1.920439in}{1.653512in}}{\pgfqpoint{1.928339in}{1.656784in}}{\pgfqpoint{1.934163in}{1.662608in}}%
\pgfpathcurveto{\pgfqpoint{1.939987in}{1.668432in}}{\pgfqpoint{1.943259in}{1.676332in}}{\pgfqpoint{1.943259in}{1.684568in}}%
\pgfpathcurveto{\pgfqpoint{1.943259in}{1.692805in}}{\pgfqpoint{1.939987in}{1.700705in}}{\pgfqpoint{1.934163in}{1.706529in}}%
\pgfpathcurveto{\pgfqpoint{1.928339in}{1.712353in}}{\pgfqpoint{1.920439in}{1.715625in}}{\pgfqpoint{1.912203in}{1.715625in}}%
\pgfpathcurveto{\pgfqpoint{1.903967in}{1.715625in}}{\pgfqpoint{1.896066in}{1.712353in}}{\pgfqpoint{1.890243in}{1.706529in}}%
\pgfpathcurveto{\pgfqpoint{1.884419in}{1.700705in}}{\pgfqpoint{1.881146in}{1.692805in}}{\pgfqpoint{1.881146in}{1.684568in}}%
\pgfpathcurveto{\pgfqpoint{1.881146in}{1.676332in}}{\pgfqpoint{1.884419in}{1.668432in}}{\pgfqpoint{1.890243in}{1.662608in}}%
\pgfpathcurveto{\pgfqpoint{1.896066in}{1.656784in}}{\pgfqpoint{1.903967in}{1.653512in}}{\pgfqpoint{1.912203in}{1.653512in}}%
\pgfpathclose%
\pgfusepath{stroke,fill}%
\end{pgfscope}%
\begin{pgfscope}%
\pgfpathrectangle{\pgfqpoint{0.100000in}{0.220728in}}{\pgfqpoint{3.696000in}{3.696000in}}%
\pgfusepath{clip}%
\pgfsetbuttcap%
\pgfsetroundjoin%
\definecolor{currentfill}{rgb}{0.121569,0.466667,0.705882}%
\pgfsetfillcolor{currentfill}%
\pgfsetfillopacity{0.911530}%
\pgfsetlinewidth{1.003750pt}%
\definecolor{currentstroke}{rgb}{0.121569,0.466667,0.705882}%
\pgfsetstrokecolor{currentstroke}%
\pgfsetstrokeopacity{0.911530}%
\pgfsetdash{}{0pt}%
\pgfpathmoveto{\pgfqpoint{2.636435in}{1.816999in}}%
\pgfpathcurveto{\pgfqpoint{2.644671in}{1.816999in}}{\pgfqpoint{2.652571in}{1.820271in}}{\pgfqpoint{2.658395in}{1.826095in}}%
\pgfpathcurveto{\pgfqpoint{2.664219in}{1.831919in}}{\pgfqpoint{2.667492in}{1.839819in}}{\pgfqpoint{2.667492in}{1.848056in}}%
\pgfpathcurveto{\pgfqpoint{2.667492in}{1.856292in}}{\pgfqpoint{2.664219in}{1.864192in}}{\pgfqpoint{2.658395in}{1.870016in}}%
\pgfpathcurveto{\pgfqpoint{2.652571in}{1.875840in}}{\pgfqpoint{2.644671in}{1.879112in}}{\pgfqpoint{2.636435in}{1.879112in}}%
\pgfpathcurveto{\pgfqpoint{2.628199in}{1.879112in}}{\pgfqpoint{2.620299in}{1.875840in}}{\pgfqpoint{2.614475in}{1.870016in}}%
\pgfpathcurveto{\pgfqpoint{2.608651in}{1.864192in}}{\pgfqpoint{2.605379in}{1.856292in}}{\pgfqpoint{2.605379in}{1.848056in}}%
\pgfpathcurveto{\pgfqpoint{2.605379in}{1.839819in}}{\pgfqpoint{2.608651in}{1.831919in}}{\pgfqpoint{2.614475in}{1.826095in}}%
\pgfpathcurveto{\pgfqpoint{2.620299in}{1.820271in}}{\pgfqpoint{2.628199in}{1.816999in}}{\pgfqpoint{2.636435in}{1.816999in}}%
\pgfpathclose%
\pgfusepath{stroke,fill}%
\end{pgfscope}%
\begin{pgfscope}%
\pgfpathrectangle{\pgfqpoint{0.100000in}{0.220728in}}{\pgfqpoint{3.696000in}{3.696000in}}%
\pgfusepath{clip}%
\pgfsetbuttcap%
\pgfsetroundjoin%
\definecolor{currentfill}{rgb}{0.121569,0.466667,0.705882}%
\pgfsetfillcolor{currentfill}%
\pgfsetfillopacity{0.912671}%
\pgfsetlinewidth{1.003750pt}%
\definecolor{currentstroke}{rgb}{0.121569,0.466667,0.705882}%
\pgfsetstrokecolor{currentstroke}%
\pgfsetstrokeopacity{0.912671}%
\pgfsetdash{}{0pt}%
\pgfpathmoveto{\pgfqpoint{1.918699in}{1.650904in}}%
\pgfpathcurveto{\pgfqpoint{1.926935in}{1.650904in}}{\pgfqpoint{1.934835in}{1.654176in}}{\pgfqpoint{1.940659in}{1.660000in}}%
\pgfpathcurveto{\pgfqpoint{1.946483in}{1.665824in}}{\pgfqpoint{1.949756in}{1.673724in}}{\pgfqpoint{1.949756in}{1.681960in}}%
\pgfpathcurveto{\pgfqpoint{1.949756in}{1.690196in}}{\pgfqpoint{1.946483in}{1.698096in}}{\pgfqpoint{1.940659in}{1.703920in}}%
\pgfpathcurveto{\pgfqpoint{1.934835in}{1.709744in}}{\pgfqpoint{1.926935in}{1.713017in}}{\pgfqpoint{1.918699in}{1.713017in}}%
\pgfpathcurveto{\pgfqpoint{1.910463in}{1.713017in}}{\pgfqpoint{1.902563in}{1.709744in}}{\pgfqpoint{1.896739in}{1.703920in}}%
\pgfpathcurveto{\pgfqpoint{1.890915in}{1.698096in}}{\pgfqpoint{1.887643in}{1.690196in}}{\pgfqpoint{1.887643in}{1.681960in}}%
\pgfpathcurveto{\pgfqpoint{1.887643in}{1.673724in}}{\pgfqpoint{1.890915in}{1.665824in}}{\pgfqpoint{1.896739in}{1.660000in}}%
\pgfpathcurveto{\pgfqpoint{1.902563in}{1.654176in}}{\pgfqpoint{1.910463in}{1.650904in}}{\pgfqpoint{1.918699in}{1.650904in}}%
\pgfpathclose%
\pgfusepath{stroke,fill}%
\end{pgfscope}%
\begin{pgfscope}%
\pgfpathrectangle{\pgfqpoint{0.100000in}{0.220728in}}{\pgfqpoint{3.696000in}{3.696000in}}%
\pgfusepath{clip}%
\pgfsetbuttcap%
\pgfsetroundjoin%
\definecolor{currentfill}{rgb}{0.121569,0.466667,0.705882}%
\pgfsetfillcolor{currentfill}%
\pgfsetfillopacity{0.914530}%
\pgfsetlinewidth{1.003750pt}%
\definecolor{currentstroke}{rgb}{0.121569,0.466667,0.705882}%
\pgfsetstrokecolor{currentstroke}%
\pgfsetstrokeopacity{0.914530}%
\pgfsetdash{}{0pt}%
\pgfpathmoveto{\pgfqpoint{2.627258in}{1.803069in}}%
\pgfpathcurveto{\pgfqpoint{2.635494in}{1.803069in}}{\pgfqpoint{2.643394in}{1.806341in}}{\pgfqpoint{2.649218in}{1.812165in}}%
\pgfpathcurveto{\pgfqpoint{2.655042in}{1.817989in}}{\pgfqpoint{2.658314in}{1.825889in}}{\pgfqpoint{2.658314in}{1.834125in}}%
\pgfpathcurveto{\pgfqpoint{2.658314in}{1.842362in}}{\pgfqpoint{2.655042in}{1.850262in}}{\pgfqpoint{2.649218in}{1.856086in}}%
\pgfpathcurveto{\pgfqpoint{2.643394in}{1.861910in}}{\pgfqpoint{2.635494in}{1.865182in}}{\pgfqpoint{2.627258in}{1.865182in}}%
\pgfpathcurveto{\pgfqpoint{2.619021in}{1.865182in}}{\pgfqpoint{2.611121in}{1.861910in}}{\pgfqpoint{2.605297in}{1.856086in}}%
\pgfpathcurveto{\pgfqpoint{2.599473in}{1.850262in}}{\pgfqpoint{2.596201in}{1.842362in}}{\pgfqpoint{2.596201in}{1.834125in}}%
\pgfpathcurveto{\pgfqpoint{2.596201in}{1.825889in}}{\pgfqpoint{2.599473in}{1.817989in}}{\pgfqpoint{2.605297in}{1.812165in}}%
\pgfpathcurveto{\pgfqpoint{2.611121in}{1.806341in}}{\pgfqpoint{2.619021in}{1.803069in}}{\pgfqpoint{2.627258in}{1.803069in}}%
\pgfpathclose%
\pgfusepath{stroke,fill}%
\end{pgfscope}%
\begin{pgfscope}%
\pgfpathrectangle{\pgfqpoint{0.100000in}{0.220728in}}{\pgfqpoint{3.696000in}{3.696000in}}%
\pgfusepath{clip}%
\pgfsetbuttcap%
\pgfsetroundjoin%
\definecolor{currentfill}{rgb}{0.121569,0.466667,0.705882}%
\pgfsetfillcolor{currentfill}%
\pgfsetfillopacity{0.914847}%
\pgfsetlinewidth{1.003750pt}%
\definecolor{currentstroke}{rgb}{0.121569,0.466667,0.705882}%
\pgfsetstrokecolor{currentstroke}%
\pgfsetstrokeopacity{0.914847}%
\pgfsetdash{}{0pt}%
\pgfpathmoveto{\pgfqpoint{1.929276in}{1.642977in}}%
\pgfpathcurveto{\pgfqpoint{1.937512in}{1.642977in}}{\pgfqpoint{1.945412in}{1.646250in}}{\pgfqpoint{1.951236in}{1.652073in}}%
\pgfpathcurveto{\pgfqpoint{1.957060in}{1.657897in}}{\pgfqpoint{1.960333in}{1.665797in}}{\pgfqpoint{1.960333in}{1.674034in}}%
\pgfpathcurveto{\pgfqpoint{1.960333in}{1.682270in}}{\pgfqpoint{1.957060in}{1.690170in}}{\pgfqpoint{1.951236in}{1.695994in}}%
\pgfpathcurveto{\pgfqpoint{1.945412in}{1.701818in}}{\pgfqpoint{1.937512in}{1.705090in}}{\pgfqpoint{1.929276in}{1.705090in}}%
\pgfpathcurveto{\pgfqpoint{1.921040in}{1.705090in}}{\pgfqpoint{1.913140in}{1.701818in}}{\pgfqpoint{1.907316in}{1.695994in}}%
\pgfpathcurveto{\pgfqpoint{1.901492in}{1.690170in}}{\pgfqpoint{1.898220in}{1.682270in}}{\pgfqpoint{1.898220in}{1.674034in}}%
\pgfpathcurveto{\pgfqpoint{1.898220in}{1.665797in}}{\pgfqpoint{1.901492in}{1.657897in}}{\pgfqpoint{1.907316in}{1.652073in}}%
\pgfpathcurveto{\pgfqpoint{1.913140in}{1.646250in}}{\pgfqpoint{1.921040in}{1.642977in}}{\pgfqpoint{1.929276in}{1.642977in}}%
\pgfpathclose%
\pgfusepath{stroke,fill}%
\end{pgfscope}%
\begin{pgfscope}%
\pgfpathrectangle{\pgfqpoint{0.100000in}{0.220728in}}{\pgfqpoint{3.696000in}{3.696000in}}%
\pgfusepath{clip}%
\pgfsetbuttcap%
\pgfsetroundjoin%
\definecolor{currentfill}{rgb}{0.121569,0.466667,0.705882}%
\pgfsetfillcolor{currentfill}%
\pgfsetfillopacity{0.916174}%
\pgfsetlinewidth{1.003750pt}%
\definecolor{currentstroke}{rgb}{0.121569,0.466667,0.705882}%
\pgfsetstrokecolor{currentstroke}%
\pgfsetstrokeopacity{0.916174}%
\pgfsetdash{}{0pt}%
\pgfpathmoveto{\pgfqpoint{2.622399in}{1.795082in}}%
\pgfpathcurveto{\pgfqpoint{2.630636in}{1.795082in}}{\pgfqpoint{2.638536in}{1.798355in}}{\pgfqpoint{2.644360in}{1.804179in}}%
\pgfpathcurveto{\pgfqpoint{2.650184in}{1.810003in}}{\pgfqpoint{2.653456in}{1.817903in}}{\pgfqpoint{2.653456in}{1.826139in}}%
\pgfpathcurveto{\pgfqpoint{2.653456in}{1.834375in}}{\pgfqpoint{2.650184in}{1.842275in}}{\pgfqpoint{2.644360in}{1.848099in}}%
\pgfpathcurveto{\pgfqpoint{2.638536in}{1.853923in}}{\pgfqpoint{2.630636in}{1.857195in}}{\pgfqpoint{2.622399in}{1.857195in}}%
\pgfpathcurveto{\pgfqpoint{2.614163in}{1.857195in}}{\pgfqpoint{2.606263in}{1.853923in}}{\pgfqpoint{2.600439in}{1.848099in}}%
\pgfpathcurveto{\pgfqpoint{2.594615in}{1.842275in}}{\pgfqpoint{2.591343in}{1.834375in}}{\pgfqpoint{2.591343in}{1.826139in}}%
\pgfpathcurveto{\pgfqpoint{2.591343in}{1.817903in}}{\pgfqpoint{2.594615in}{1.810003in}}{\pgfqpoint{2.600439in}{1.804179in}}%
\pgfpathcurveto{\pgfqpoint{2.606263in}{1.798355in}}{\pgfqpoint{2.614163in}{1.795082in}}{\pgfqpoint{2.622399in}{1.795082in}}%
\pgfpathclose%
\pgfusepath{stroke,fill}%
\end{pgfscope}%
\begin{pgfscope}%
\pgfpathrectangle{\pgfqpoint{0.100000in}{0.220728in}}{\pgfqpoint{3.696000in}{3.696000in}}%
\pgfusepath{clip}%
\pgfsetbuttcap%
\pgfsetroundjoin%
\definecolor{currentfill}{rgb}{0.121569,0.466667,0.705882}%
\pgfsetfillcolor{currentfill}%
\pgfsetfillopacity{0.917129}%
\pgfsetlinewidth{1.003750pt}%
\definecolor{currentstroke}{rgb}{0.121569,0.466667,0.705882}%
\pgfsetstrokecolor{currentstroke}%
\pgfsetstrokeopacity{0.917129}%
\pgfsetdash{}{0pt}%
\pgfpathmoveto{\pgfqpoint{2.619800in}{1.790867in}}%
\pgfpathcurveto{\pgfqpoint{2.628036in}{1.790867in}}{\pgfqpoint{2.635936in}{1.794139in}}{\pgfqpoint{2.641760in}{1.799963in}}%
\pgfpathcurveto{\pgfqpoint{2.647584in}{1.805787in}}{\pgfqpoint{2.650856in}{1.813687in}}{\pgfqpoint{2.650856in}{1.821923in}}%
\pgfpathcurveto{\pgfqpoint{2.650856in}{1.830159in}}{\pgfqpoint{2.647584in}{1.838060in}}{\pgfqpoint{2.641760in}{1.843883in}}%
\pgfpathcurveto{\pgfqpoint{2.635936in}{1.849707in}}{\pgfqpoint{2.628036in}{1.852980in}}{\pgfqpoint{2.619800in}{1.852980in}}%
\pgfpathcurveto{\pgfqpoint{2.611564in}{1.852980in}}{\pgfqpoint{2.603664in}{1.849707in}}{\pgfqpoint{2.597840in}{1.843883in}}%
\pgfpathcurveto{\pgfqpoint{2.592016in}{1.838060in}}{\pgfqpoint{2.588743in}{1.830159in}}{\pgfqpoint{2.588743in}{1.821923in}}%
\pgfpathcurveto{\pgfqpoint{2.588743in}{1.813687in}}{\pgfqpoint{2.592016in}{1.805787in}}{\pgfqpoint{2.597840in}{1.799963in}}%
\pgfpathcurveto{\pgfqpoint{2.603664in}{1.794139in}}{\pgfqpoint{2.611564in}{1.790867in}}{\pgfqpoint{2.619800in}{1.790867in}}%
\pgfpathclose%
\pgfusepath{stroke,fill}%
\end{pgfscope}%
\begin{pgfscope}%
\pgfpathrectangle{\pgfqpoint{0.100000in}{0.220728in}}{\pgfqpoint{3.696000in}{3.696000in}}%
\pgfusepath{clip}%
\pgfsetbuttcap%
\pgfsetroundjoin%
\definecolor{currentfill}{rgb}{0.121569,0.466667,0.705882}%
\pgfsetfillcolor{currentfill}%
\pgfsetfillopacity{0.917571}%
\pgfsetlinewidth{1.003750pt}%
\definecolor{currentstroke}{rgb}{0.121569,0.466667,0.705882}%
\pgfsetstrokecolor{currentstroke}%
\pgfsetstrokeopacity{0.917571}%
\pgfsetdash{}{0pt}%
\pgfpathmoveto{\pgfqpoint{2.618047in}{1.788615in}}%
\pgfpathcurveto{\pgfqpoint{2.626284in}{1.788615in}}{\pgfqpoint{2.634184in}{1.791888in}}{\pgfqpoint{2.640008in}{1.797712in}}%
\pgfpathcurveto{\pgfqpoint{2.645832in}{1.803536in}}{\pgfqpoint{2.649104in}{1.811436in}}{\pgfqpoint{2.649104in}{1.819672in}}%
\pgfpathcurveto{\pgfqpoint{2.649104in}{1.827908in}}{\pgfqpoint{2.645832in}{1.835808in}}{\pgfqpoint{2.640008in}{1.841632in}}%
\pgfpathcurveto{\pgfqpoint{2.634184in}{1.847456in}}{\pgfqpoint{2.626284in}{1.850728in}}{\pgfqpoint{2.618047in}{1.850728in}}%
\pgfpathcurveto{\pgfqpoint{2.609811in}{1.850728in}}{\pgfqpoint{2.601911in}{1.847456in}}{\pgfqpoint{2.596087in}{1.841632in}}%
\pgfpathcurveto{\pgfqpoint{2.590263in}{1.835808in}}{\pgfqpoint{2.586991in}{1.827908in}}{\pgfqpoint{2.586991in}{1.819672in}}%
\pgfpathcurveto{\pgfqpoint{2.586991in}{1.811436in}}{\pgfqpoint{2.590263in}{1.803536in}}{\pgfqpoint{2.596087in}{1.797712in}}%
\pgfpathcurveto{\pgfqpoint{2.601911in}{1.791888in}}{\pgfqpoint{2.609811in}{1.788615in}}{\pgfqpoint{2.618047in}{1.788615in}}%
\pgfpathclose%
\pgfusepath{stroke,fill}%
\end{pgfscope}%
\begin{pgfscope}%
\pgfpathrectangle{\pgfqpoint{0.100000in}{0.220728in}}{\pgfqpoint{3.696000in}{3.696000in}}%
\pgfusepath{clip}%
\pgfsetbuttcap%
\pgfsetroundjoin%
\definecolor{currentfill}{rgb}{0.121569,0.466667,0.705882}%
\pgfsetfillcolor{currentfill}%
\pgfsetfillopacity{0.917834}%
\pgfsetlinewidth{1.003750pt}%
\definecolor{currentstroke}{rgb}{0.121569,0.466667,0.705882}%
\pgfsetstrokecolor{currentstroke}%
\pgfsetstrokeopacity{0.917834}%
\pgfsetdash{}{0pt}%
\pgfpathmoveto{\pgfqpoint{2.617193in}{1.787307in}}%
\pgfpathcurveto{\pgfqpoint{2.625429in}{1.787307in}}{\pgfqpoint{2.633329in}{1.790580in}}{\pgfqpoint{2.639153in}{1.796404in}}%
\pgfpathcurveto{\pgfqpoint{2.644977in}{1.802228in}}{\pgfqpoint{2.648249in}{1.810128in}}{\pgfqpoint{2.648249in}{1.818364in}}%
\pgfpathcurveto{\pgfqpoint{2.648249in}{1.826600in}}{\pgfqpoint{2.644977in}{1.834500in}}{\pgfqpoint{2.639153in}{1.840324in}}%
\pgfpathcurveto{\pgfqpoint{2.633329in}{1.846148in}}{\pgfqpoint{2.625429in}{1.849420in}}{\pgfqpoint{2.617193in}{1.849420in}}%
\pgfpathcurveto{\pgfqpoint{2.608957in}{1.849420in}}{\pgfqpoint{2.601057in}{1.846148in}}{\pgfqpoint{2.595233in}{1.840324in}}%
\pgfpathcurveto{\pgfqpoint{2.589409in}{1.834500in}}{\pgfqpoint{2.586136in}{1.826600in}}{\pgfqpoint{2.586136in}{1.818364in}}%
\pgfpathcurveto{\pgfqpoint{2.586136in}{1.810128in}}{\pgfqpoint{2.589409in}{1.802228in}}{\pgfqpoint{2.595233in}{1.796404in}}%
\pgfpathcurveto{\pgfqpoint{2.601057in}{1.790580in}}{\pgfqpoint{2.608957in}{1.787307in}}{\pgfqpoint{2.617193in}{1.787307in}}%
\pgfpathclose%
\pgfusepath{stroke,fill}%
\end{pgfscope}%
\begin{pgfscope}%
\pgfpathrectangle{\pgfqpoint{0.100000in}{0.220728in}}{\pgfqpoint{3.696000in}{3.696000in}}%
\pgfusepath{clip}%
\pgfsetbuttcap%
\pgfsetroundjoin%
\definecolor{currentfill}{rgb}{0.121569,0.466667,0.705882}%
\pgfsetfillcolor{currentfill}%
\pgfsetfillopacity{0.918624}%
\pgfsetlinewidth{1.003750pt}%
\definecolor{currentstroke}{rgb}{0.121569,0.466667,0.705882}%
\pgfsetstrokecolor{currentstroke}%
\pgfsetstrokeopacity{0.918624}%
\pgfsetdash{}{0pt}%
\pgfpathmoveto{\pgfqpoint{2.614177in}{1.783938in}}%
\pgfpathcurveto{\pgfqpoint{2.622413in}{1.783938in}}{\pgfqpoint{2.630313in}{1.787210in}}{\pgfqpoint{2.636137in}{1.793034in}}%
\pgfpathcurveto{\pgfqpoint{2.641961in}{1.798858in}}{\pgfqpoint{2.645234in}{1.806758in}}{\pgfqpoint{2.645234in}{1.814995in}}%
\pgfpathcurveto{\pgfqpoint{2.645234in}{1.823231in}}{\pgfqpoint{2.641961in}{1.831131in}}{\pgfqpoint{2.636137in}{1.836955in}}%
\pgfpathcurveto{\pgfqpoint{2.630313in}{1.842779in}}{\pgfqpoint{2.622413in}{1.846051in}}{\pgfqpoint{2.614177in}{1.846051in}}%
\pgfpathcurveto{\pgfqpoint{2.605941in}{1.846051in}}{\pgfqpoint{2.598041in}{1.842779in}}{\pgfqpoint{2.592217in}{1.836955in}}%
\pgfpathcurveto{\pgfqpoint{2.586393in}{1.831131in}}{\pgfqpoint{2.583121in}{1.823231in}}{\pgfqpoint{2.583121in}{1.814995in}}%
\pgfpathcurveto{\pgfqpoint{2.583121in}{1.806758in}}{\pgfqpoint{2.586393in}{1.798858in}}{\pgfqpoint{2.592217in}{1.793034in}}%
\pgfpathcurveto{\pgfqpoint{2.598041in}{1.787210in}}{\pgfqpoint{2.605941in}{1.783938in}}{\pgfqpoint{2.614177in}{1.783938in}}%
\pgfpathclose%
\pgfusepath{stroke,fill}%
\end{pgfscope}%
\begin{pgfscope}%
\pgfpathrectangle{\pgfqpoint{0.100000in}{0.220728in}}{\pgfqpoint{3.696000in}{3.696000in}}%
\pgfusepath{clip}%
\pgfsetbuttcap%
\pgfsetroundjoin%
\definecolor{currentfill}{rgb}{0.121569,0.466667,0.705882}%
\pgfsetfillcolor{currentfill}%
\pgfsetfillopacity{0.919108}%
\pgfsetlinewidth{1.003750pt}%
\definecolor{currentstroke}{rgb}{0.121569,0.466667,0.705882}%
\pgfsetstrokecolor{currentstroke}%
\pgfsetstrokeopacity{0.919108}%
\pgfsetdash{}{0pt}%
\pgfpathmoveto{\pgfqpoint{1.948522in}{1.629923in}}%
\pgfpathcurveto{\pgfqpoint{1.956758in}{1.629923in}}{\pgfqpoint{1.964658in}{1.633195in}}{\pgfqpoint{1.970482in}{1.639019in}}%
\pgfpathcurveto{\pgfqpoint{1.976306in}{1.644843in}}{\pgfqpoint{1.979578in}{1.652743in}}{\pgfqpoint{1.979578in}{1.660979in}}%
\pgfpathcurveto{\pgfqpoint{1.979578in}{1.669216in}}{\pgfqpoint{1.976306in}{1.677116in}}{\pgfqpoint{1.970482in}{1.682939in}}%
\pgfpathcurveto{\pgfqpoint{1.964658in}{1.688763in}}{\pgfqpoint{1.956758in}{1.692036in}}{\pgfqpoint{1.948522in}{1.692036in}}%
\pgfpathcurveto{\pgfqpoint{1.940285in}{1.692036in}}{\pgfqpoint{1.932385in}{1.688763in}}{\pgfqpoint{1.926561in}{1.682939in}}%
\pgfpathcurveto{\pgfqpoint{1.920738in}{1.677116in}}{\pgfqpoint{1.917465in}{1.669216in}}{\pgfqpoint{1.917465in}{1.660979in}}%
\pgfpathcurveto{\pgfqpoint{1.917465in}{1.652743in}}{\pgfqpoint{1.920738in}{1.644843in}}{\pgfqpoint{1.926561in}{1.639019in}}%
\pgfpathcurveto{\pgfqpoint{1.932385in}{1.633195in}}{\pgfqpoint{1.940285in}{1.629923in}}{\pgfqpoint{1.948522in}{1.629923in}}%
\pgfpathclose%
\pgfusepath{stroke,fill}%
\end{pgfscope}%
\begin{pgfscope}%
\pgfpathrectangle{\pgfqpoint{0.100000in}{0.220728in}}{\pgfqpoint{3.696000in}{3.696000in}}%
\pgfusepath{clip}%
\pgfsetbuttcap%
\pgfsetroundjoin%
\definecolor{currentfill}{rgb}{0.121569,0.466667,0.705882}%
\pgfsetfillcolor{currentfill}%
\pgfsetfillopacity{0.920162}%
\pgfsetlinewidth{1.003750pt}%
\definecolor{currentstroke}{rgb}{0.121569,0.466667,0.705882}%
\pgfsetstrokecolor{currentstroke}%
\pgfsetstrokeopacity{0.920162}%
\pgfsetdash{}{0pt}%
\pgfpathmoveto{\pgfqpoint{2.609941in}{1.775621in}}%
\pgfpathcurveto{\pgfqpoint{2.618178in}{1.775621in}}{\pgfqpoint{2.626078in}{1.778893in}}{\pgfqpoint{2.631902in}{1.784717in}}%
\pgfpathcurveto{\pgfqpoint{2.637726in}{1.790541in}}{\pgfqpoint{2.640998in}{1.798441in}}{\pgfqpoint{2.640998in}{1.806678in}}%
\pgfpathcurveto{\pgfqpoint{2.640998in}{1.814914in}}{\pgfqpoint{2.637726in}{1.822814in}}{\pgfqpoint{2.631902in}{1.828638in}}%
\pgfpathcurveto{\pgfqpoint{2.626078in}{1.834462in}}{\pgfqpoint{2.618178in}{1.837734in}}{\pgfqpoint{2.609941in}{1.837734in}}%
\pgfpathcurveto{\pgfqpoint{2.601705in}{1.837734in}}{\pgfqpoint{2.593805in}{1.834462in}}{\pgfqpoint{2.587981in}{1.828638in}}%
\pgfpathcurveto{\pgfqpoint{2.582157in}{1.822814in}}{\pgfqpoint{2.578885in}{1.814914in}}{\pgfqpoint{2.578885in}{1.806678in}}%
\pgfpathcurveto{\pgfqpoint{2.578885in}{1.798441in}}{\pgfqpoint{2.582157in}{1.790541in}}{\pgfqpoint{2.587981in}{1.784717in}}%
\pgfpathcurveto{\pgfqpoint{2.593805in}{1.778893in}}{\pgfqpoint{2.601705in}{1.775621in}}{\pgfqpoint{2.609941in}{1.775621in}}%
\pgfpathclose%
\pgfusepath{stroke,fill}%
\end{pgfscope}%
\begin{pgfscope}%
\pgfpathrectangle{\pgfqpoint{0.100000in}{0.220728in}}{\pgfqpoint{3.696000in}{3.696000in}}%
\pgfusepath{clip}%
\pgfsetbuttcap%
\pgfsetroundjoin%
\definecolor{currentfill}{rgb}{0.121569,0.466667,0.705882}%
\pgfsetfillcolor{currentfill}%
\pgfsetfillopacity{0.922313}%
\pgfsetlinewidth{1.003750pt}%
\definecolor{currentstroke}{rgb}{0.121569,0.466667,0.705882}%
\pgfsetstrokecolor{currentstroke}%
\pgfsetstrokeopacity{0.922313}%
\pgfsetdash{}{0pt}%
\pgfpathmoveto{\pgfqpoint{2.602591in}{1.766162in}}%
\pgfpathcurveto{\pgfqpoint{2.610828in}{1.766162in}}{\pgfqpoint{2.618728in}{1.769434in}}{\pgfqpoint{2.624552in}{1.775258in}}%
\pgfpathcurveto{\pgfqpoint{2.630376in}{1.781082in}}{\pgfqpoint{2.633648in}{1.788982in}}{\pgfqpoint{2.633648in}{1.797219in}}%
\pgfpathcurveto{\pgfqpoint{2.633648in}{1.805455in}}{\pgfqpoint{2.630376in}{1.813355in}}{\pgfqpoint{2.624552in}{1.819179in}}%
\pgfpathcurveto{\pgfqpoint{2.618728in}{1.825003in}}{\pgfqpoint{2.610828in}{1.828275in}}{\pgfqpoint{2.602591in}{1.828275in}}%
\pgfpathcurveto{\pgfqpoint{2.594355in}{1.828275in}}{\pgfqpoint{2.586455in}{1.825003in}}{\pgfqpoint{2.580631in}{1.819179in}}%
\pgfpathcurveto{\pgfqpoint{2.574807in}{1.813355in}}{\pgfqpoint{2.571535in}{1.805455in}}{\pgfqpoint{2.571535in}{1.797219in}}%
\pgfpathcurveto{\pgfqpoint{2.571535in}{1.788982in}}{\pgfqpoint{2.574807in}{1.781082in}}{\pgfqpoint{2.580631in}{1.775258in}}%
\pgfpathcurveto{\pgfqpoint{2.586455in}{1.769434in}}{\pgfqpoint{2.594355in}{1.766162in}}{\pgfqpoint{2.602591in}{1.766162in}}%
\pgfpathclose%
\pgfusepath{stroke,fill}%
\end{pgfscope}%
\begin{pgfscope}%
\pgfpathrectangle{\pgfqpoint{0.100000in}{0.220728in}}{\pgfqpoint{3.696000in}{3.696000in}}%
\pgfusepath{clip}%
\pgfsetbuttcap%
\pgfsetroundjoin%
\definecolor{currentfill}{rgb}{0.121569,0.466667,0.705882}%
\pgfsetfillcolor{currentfill}%
\pgfsetfillopacity{0.925471}%
\pgfsetlinewidth{1.003750pt}%
\definecolor{currentstroke}{rgb}{0.121569,0.466667,0.705882}%
\pgfsetstrokecolor{currentstroke}%
\pgfsetstrokeopacity{0.925471}%
\pgfsetdash{}{0pt}%
\pgfpathmoveto{\pgfqpoint{2.594113in}{1.752388in}}%
\pgfpathcurveto{\pgfqpoint{2.602349in}{1.752388in}}{\pgfqpoint{2.610249in}{1.755660in}}{\pgfqpoint{2.616073in}{1.761484in}}%
\pgfpathcurveto{\pgfqpoint{2.621897in}{1.767308in}}{\pgfqpoint{2.625170in}{1.775208in}}{\pgfqpoint{2.625170in}{1.783444in}}%
\pgfpathcurveto{\pgfqpoint{2.625170in}{1.791680in}}{\pgfqpoint{2.621897in}{1.799580in}}{\pgfqpoint{2.616073in}{1.805404in}}%
\pgfpathcurveto{\pgfqpoint{2.610249in}{1.811228in}}{\pgfqpoint{2.602349in}{1.814501in}}{\pgfqpoint{2.594113in}{1.814501in}}%
\pgfpathcurveto{\pgfqpoint{2.585877in}{1.814501in}}{\pgfqpoint{2.577977in}{1.811228in}}{\pgfqpoint{2.572153in}{1.805404in}}%
\pgfpathcurveto{\pgfqpoint{2.566329in}{1.799580in}}{\pgfqpoint{2.563057in}{1.791680in}}{\pgfqpoint{2.563057in}{1.783444in}}%
\pgfpathcurveto{\pgfqpoint{2.563057in}{1.775208in}}{\pgfqpoint{2.566329in}{1.767308in}}{\pgfqpoint{2.572153in}{1.761484in}}%
\pgfpathcurveto{\pgfqpoint{2.577977in}{1.755660in}}{\pgfqpoint{2.585877in}{1.752388in}}{\pgfqpoint{2.594113in}{1.752388in}}%
\pgfpathclose%
\pgfusepath{stroke,fill}%
\end{pgfscope}%
\begin{pgfscope}%
\pgfpathrectangle{\pgfqpoint{0.100000in}{0.220728in}}{\pgfqpoint{3.696000in}{3.696000in}}%
\pgfusepath{clip}%
\pgfsetbuttcap%
\pgfsetroundjoin%
\definecolor{currentfill}{rgb}{0.121569,0.466667,0.705882}%
\pgfsetfillcolor{currentfill}%
\pgfsetfillopacity{0.925568}%
\pgfsetlinewidth{1.003750pt}%
\definecolor{currentstroke}{rgb}{0.121569,0.466667,0.705882}%
\pgfsetstrokecolor{currentstroke}%
\pgfsetstrokeopacity{0.925568}%
\pgfsetdash{}{0pt}%
\pgfpathmoveto{\pgfqpoint{1.985308in}{1.604329in}}%
\pgfpathcurveto{\pgfqpoint{1.993544in}{1.604329in}}{\pgfqpoint{2.001444in}{1.607601in}}{\pgfqpoint{2.007268in}{1.613425in}}%
\pgfpathcurveto{\pgfqpoint{2.013092in}{1.619249in}}{\pgfqpoint{2.016364in}{1.627149in}}{\pgfqpoint{2.016364in}{1.635385in}}%
\pgfpathcurveto{\pgfqpoint{2.016364in}{1.643622in}}{\pgfqpoint{2.013092in}{1.651522in}}{\pgfqpoint{2.007268in}{1.657346in}}%
\pgfpathcurveto{\pgfqpoint{2.001444in}{1.663170in}}{\pgfqpoint{1.993544in}{1.666442in}}{\pgfqpoint{1.985308in}{1.666442in}}%
\pgfpathcurveto{\pgfqpoint{1.977072in}{1.666442in}}{\pgfqpoint{1.969172in}{1.663170in}}{\pgfqpoint{1.963348in}{1.657346in}}%
\pgfpathcurveto{\pgfqpoint{1.957524in}{1.651522in}}{\pgfqpoint{1.954251in}{1.643622in}}{\pgfqpoint{1.954251in}{1.635385in}}%
\pgfpathcurveto{\pgfqpoint{1.954251in}{1.627149in}}{\pgfqpoint{1.957524in}{1.619249in}}{\pgfqpoint{1.963348in}{1.613425in}}%
\pgfpathcurveto{\pgfqpoint{1.969172in}{1.607601in}}{\pgfqpoint{1.977072in}{1.604329in}}{\pgfqpoint{1.985308in}{1.604329in}}%
\pgfpathclose%
\pgfusepath{stroke,fill}%
\end{pgfscope}%
\begin{pgfscope}%
\pgfpathrectangle{\pgfqpoint{0.100000in}{0.220728in}}{\pgfqpoint{3.696000in}{3.696000in}}%
\pgfusepath{clip}%
\pgfsetbuttcap%
\pgfsetroundjoin%
\definecolor{currentfill}{rgb}{0.121569,0.466667,0.705882}%
\pgfsetfillcolor{currentfill}%
\pgfsetfillopacity{0.928931}%
\pgfsetlinewidth{1.003750pt}%
\definecolor{currentstroke}{rgb}{0.121569,0.466667,0.705882}%
\pgfsetstrokecolor{currentstroke}%
\pgfsetstrokeopacity{0.928931}%
\pgfsetdash{}{0pt}%
\pgfpathmoveto{\pgfqpoint{2.582389in}{1.737301in}}%
\pgfpathcurveto{\pgfqpoint{2.590625in}{1.737301in}}{\pgfqpoint{2.598525in}{1.740574in}}{\pgfqpoint{2.604349in}{1.746398in}}%
\pgfpathcurveto{\pgfqpoint{2.610173in}{1.752222in}}{\pgfqpoint{2.613445in}{1.760122in}}{\pgfqpoint{2.613445in}{1.768358in}}%
\pgfpathcurveto{\pgfqpoint{2.613445in}{1.776594in}}{\pgfqpoint{2.610173in}{1.784494in}}{\pgfqpoint{2.604349in}{1.790318in}}%
\pgfpathcurveto{\pgfqpoint{2.598525in}{1.796142in}}{\pgfqpoint{2.590625in}{1.799414in}}{\pgfqpoint{2.582389in}{1.799414in}}%
\pgfpathcurveto{\pgfqpoint{2.574152in}{1.799414in}}{\pgfqpoint{2.566252in}{1.796142in}}{\pgfqpoint{2.560428in}{1.790318in}}%
\pgfpathcurveto{\pgfqpoint{2.554604in}{1.784494in}}{\pgfqpoint{2.551332in}{1.776594in}}{\pgfqpoint{2.551332in}{1.768358in}}%
\pgfpathcurveto{\pgfqpoint{2.551332in}{1.760122in}}{\pgfqpoint{2.554604in}{1.752222in}}{\pgfqpoint{2.560428in}{1.746398in}}%
\pgfpathcurveto{\pgfqpoint{2.566252in}{1.740574in}}{\pgfqpoint{2.574152in}{1.737301in}}{\pgfqpoint{2.582389in}{1.737301in}}%
\pgfpathclose%
\pgfusepath{stroke,fill}%
\end{pgfscope}%
\begin{pgfscope}%
\pgfpathrectangle{\pgfqpoint{0.100000in}{0.220728in}}{\pgfqpoint{3.696000in}{3.696000in}}%
\pgfusepath{clip}%
\pgfsetbuttcap%
\pgfsetroundjoin%
\definecolor{currentfill}{rgb}{0.121569,0.466667,0.705882}%
\pgfsetfillcolor{currentfill}%
\pgfsetfillopacity{0.932260}%
\pgfsetlinewidth{1.003750pt}%
\definecolor{currentstroke}{rgb}{0.121569,0.466667,0.705882}%
\pgfsetstrokecolor{currentstroke}%
\pgfsetstrokeopacity{0.932260}%
\pgfsetdash{}{0pt}%
\pgfpathmoveto{\pgfqpoint{2.022224in}{1.590686in}}%
\pgfpathcurveto{\pgfqpoint{2.030460in}{1.590686in}}{\pgfqpoint{2.038360in}{1.593958in}}{\pgfqpoint{2.044184in}{1.599782in}}%
\pgfpathcurveto{\pgfqpoint{2.050008in}{1.605606in}}{\pgfqpoint{2.053280in}{1.613506in}}{\pgfqpoint{2.053280in}{1.621742in}}%
\pgfpathcurveto{\pgfqpoint{2.053280in}{1.629978in}}{\pgfqpoint{2.050008in}{1.637878in}}{\pgfqpoint{2.044184in}{1.643702in}}%
\pgfpathcurveto{\pgfqpoint{2.038360in}{1.649526in}}{\pgfqpoint{2.030460in}{1.652799in}}{\pgfqpoint{2.022224in}{1.652799in}}%
\pgfpathcurveto{\pgfqpoint{2.013988in}{1.652799in}}{\pgfqpoint{2.006088in}{1.649526in}}{\pgfqpoint{2.000264in}{1.643702in}}%
\pgfpathcurveto{\pgfqpoint{1.994440in}{1.637878in}}{\pgfqpoint{1.991167in}{1.629978in}}{\pgfqpoint{1.991167in}{1.621742in}}%
\pgfpathcurveto{\pgfqpoint{1.991167in}{1.613506in}}{\pgfqpoint{1.994440in}{1.605606in}}{\pgfqpoint{2.000264in}{1.599782in}}%
\pgfpathcurveto{\pgfqpoint{2.006088in}{1.593958in}}{\pgfqpoint{2.013988in}{1.590686in}}{\pgfqpoint{2.022224in}{1.590686in}}%
\pgfpathclose%
\pgfusepath{stroke,fill}%
\end{pgfscope}%
\begin{pgfscope}%
\pgfpathrectangle{\pgfqpoint{0.100000in}{0.220728in}}{\pgfqpoint{3.696000in}{3.696000in}}%
\pgfusepath{clip}%
\pgfsetbuttcap%
\pgfsetroundjoin%
\definecolor{currentfill}{rgb}{0.121569,0.466667,0.705882}%
\pgfsetfillcolor{currentfill}%
\pgfsetfillopacity{0.933238}%
\pgfsetlinewidth{1.003750pt}%
\definecolor{currentstroke}{rgb}{0.121569,0.466667,0.705882}%
\pgfsetstrokecolor{currentstroke}%
\pgfsetstrokeopacity{0.933238}%
\pgfsetdash{}{0pt}%
\pgfpathmoveto{\pgfqpoint{2.570996in}{1.719649in}}%
\pgfpathcurveto{\pgfqpoint{2.579232in}{1.719649in}}{\pgfqpoint{2.587132in}{1.722922in}}{\pgfqpoint{2.592956in}{1.728746in}}%
\pgfpathcurveto{\pgfqpoint{2.598780in}{1.734570in}}{\pgfqpoint{2.602052in}{1.742470in}}{\pgfqpoint{2.602052in}{1.750706in}}%
\pgfpathcurveto{\pgfqpoint{2.602052in}{1.758942in}}{\pgfqpoint{2.598780in}{1.766842in}}{\pgfqpoint{2.592956in}{1.772666in}}%
\pgfpathcurveto{\pgfqpoint{2.587132in}{1.778490in}}{\pgfqpoint{2.579232in}{1.781762in}}{\pgfqpoint{2.570996in}{1.781762in}}%
\pgfpathcurveto{\pgfqpoint{2.562760in}{1.781762in}}{\pgfqpoint{2.554860in}{1.778490in}}{\pgfqpoint{2.549036in}{1.772666in}}%
\pgfpathcurveto{\pgfqpoint{2.543212in}{1.766842in}}{\pgfqpoint{2.539939in}{1.758942in}}{\pgfqpoint{2.539939in}{1.750706in}}%
\pgfpathcurveto{\pgfqpoint{2.539939in}{1.742470in}}{\pgfqpoint{2.543212in}{1.734570in}}{\pgfqpoint{2.549036in}{1.728746in}}%
\pgfpathcurveto{\pgfqpoint{2.554860in}{1.722922in}}{\pgfqpoint{2.562760in}{1.719649in}}{\pgfqpoint{2.570996in}{1.719649in}}%
\pgfpathclose%
\pgfusepath{stroke,fill}%
\end{pgfscope}%
\begin{pgfscope}%
\pgfpathrectangle{\pgfqpoint{0.100000in}{0.220728in}}{\pgfqpoint{3.696000in}{3.696000in}}%
\pgfusepath{clip}%
\pgfsetbuttcap%
\pgfsetroundjoin%
\definecolor{currentfill}{rgb}{0.121569,0.466667,0.705882}%
\pgfsetfillcolor{currentfill}%
\pgfsetfillopacity{0.935322}%
\pgfsetlinewidth{1.003750pt}%
\definecolor{currentstroke}{rgb}{0.121569,0.466667,0.705882}%
\pgfsetstrokecolor{currentstroke}%
\pgfsetstrokeopacity{0.935322}%
\pgfsetdash{}{0pt}%
\pgfpathmoveto{\pgfqpoint{2.564396in}{1.708999in}}%
\pgfpathcurveto{\pgfqpoint{2.572633in}{1.708999in}}{\pgfqpoint{2.580533in}{1.712272in}}{\pgfqpoint{2.586357in}{1.718096in}}%
\pgfpathcurveto{\pgfqpoint{2.592180in}{1.723919in}}{\pgfqpoint{2.595453in}{1.731820in}}{\pgfqpoint{2.595453in}{1.740056in}}%
\pgfpathcurveto{\pgfqpoint{2.595453in}{1.748292in}}{\pgfqpoint{2.592180in}{1.756192in}}{\pgfqpoint{2.586357in}{1.762016in}}%
\pgfpathcurveto{\pgfqpoint{2.580533in}{1.767840in}}{\pgfqpoint{2.572633in}{1.771112in}}{\pgfqpoint{2.564396in}{1.771112in}}%
\pgfpathcurveto{\pgfqpoint{2.556160in}{1.771112in}}{\pgfqpoint{2.548260in}{1.767840in}}{\pgfqpoint{2.542436in}{1.762016in}}%
\pgfpathcurveto{\pgfqpoint{2.536612in}{1.756192in}}{\pgfqpoint{2.533340in}{1.748292in}}{\pgfqpoint{2.533340in}{1.740056in}}%
\pgfpathcurveto{\pgfqpoint{2.533340in}{1.731820in}}{\pgfqpoint{2.536612in}{1.723919in}}{\pgfqpoint{2.542436in}{1.718096in}}%
\pgfpathcurveto{\pgfqpoint{2.548260in}{1.712272in}}{\pgfqpoint{2.556160in}{1.708999in}}{\pgfqpoint{2.564396in}{1.708999in}}%
\pgfpathclose%
\pgfusepath{stroke,fill}%
\end{pgfscope}%
\begin{pgfscope}%
\pgfpathrectangle{\pgfqpoint{0.100000in}{0.220728in}}{\pgfqpoint{3.696000in}{3.696000in}}%
\pgfusepath{clip}%
\pgfsetbuttcap%
\pgfsetroundjoin%
\definecolor{currentfill}{rgb}{0.121569,0.466667,0.705882}%
\pgfsetfillcolor{currentfill}%
\pgfsetfillopacity{0.936599}%
\pgfsetlinewidth{1.003750pt}%
\definecolor{currentstroke}{rgb}{0.121569,0.466667,0.705882}%
\pgfsetstrokecolor{currentstroke}%
\pgfsetstrokeopacity{0.936599}%
\pgfsetdash{}{0pt}%
\pgfpathmoveto{\pgfqpoint{2.560737in}{1.703740in}}%
\pgfpathcurveto{\pgfqpoint{2.568974in}{1.703740in}}{\pgfqpoint{2.576874in}{1.707013in}}{\pgfqpoint{2.582698in}{1.712837in}}%
\pgfpathcurveto{\pgfqpoint{2.588521in}{1.718661in}}{\pgfqpoint{2.591794in}{1.726561in}}{\pgfqpoint{2.591794in}{1.734797in}}%
\pgfpathcurveto{\pgfqpoint{2.591794in}{1.743033in}}{\pgfqpoint{2.588521in}{1.750933in}}{\pgfqpoint{2.582698in}{1.756757in}}%
\pgfpathcurveto{\pgfqpoint{2.576874in}{1.762581in}}{\pgfqpoint{2.568974in}{1.765853in}}{\pgfqpoint{2.560737in}{1.765853in}}%
\pgfpathcurveto{\pgfqpoint{2.552501in}{1.765853in}}{\pgfqpoint{2.544601in}{1.762581in}}{\pgfqpoint{2.538777in}{1.756757in}}%
\pgfpathcurveto{\pgfqpoint{2.532953in}{1.750933in}}{\pgfqpoint{2.529681in}{1.743033in}}{\pgfqpoint{2.529681in}{1.734797in}}%
\pgfpathcurveto{\pgfqpoint{2.529681in}{1.726561in}}{\pgfqpoint{2.532953in}{1.718661in}}{\pgfqpoint{2.538777in}{1.712837in}}%
\pgfpathcurveto{\pgfqpoint{2.544601in}{1.707013in}}{\pgfqpoint{2.552501in}{1.703740in}}{\pgfqpoint{2.560737in}{1.703740in}}%
\pgfpathclose%
\pgfusepath{stroke,fill}%
\end{pgfscope}%
\begin{pgfscope}%
\pgfpathrectangle{\pgfqpoint{0.100000in}{0.220728in}}{\pgfqpoint{3.696000in}{3.696000in}}%
\pgfusepath{clip}%
\pgfsetbuttcap%
\pgfsetroundjoin%
\definecolor{currentfill}{rgb}{0.121569,0.466667,0.705882}%
\pgfsetfillcolor{currentfill}%
\pgfsetfillopacity{0.937279}%
\pgfsetlinewidth{1.003750pt}%
\definecolor{currentstroke}{rgb}{0.121569,0.466667,0.705882}%
\pgfsetstrokecolor{currentstroke}%
\pgfsetstrokeopacity{0.937279}%
\pgfsetdash{}{0pt}%
\pgfpathmoveto{\pgfqpoint{2.558712in}{1.700785in}}%
\pgfpathcurveto{\pgfqpoint{2.566948in}{1.700785in}}{\pgfqpoint{2.574849in}{1.704057in}}{\pgfqpoint{2.580672in}{1.709881in}}%
\pgfpathcurveto{\pgfqpoint{2.586496in}{1.715705in}}{\pgfqpoint{2.589769in}{1.723605in}}{\pgfqpoint{2.589769in}{1.731841in}}%
\pgfpathcurveto{\pgfqpoint{2.589769in}{1.740078in}}{\pgfqpoint{2.586496in}{1.747978in}}{\pgfqpoint{2.580672in}{1.753802in}}%
\pgfpathcurveto{\pgfqpoint{2.574849in}{1.759626in}}{\pgfqpoint{2.566948in}{1.762898in}}{\pgfqpoint{2.558712in}{1.762898in}}%
\pgfpathcurveto{\pgfqpoint{2.550476in}{1.762898in}}{\pgfqpoint{2.542576in}{1.759626in}}{\pgfqpoint{2.536752in}{1.753802in}}%
\pgfpathcurveto{\pgfqpoint{2.530928in}{1.747978in}}{\pgfqpoint{2.527656in}{1.740078in}}{\pgfqpoint{2.527656in}{1.731841in}}%
\pgfpathcurveto{\pgfqpoint{2.527656in}{1.723605in}}{\pgfqpoint{2.530928in}{1.715705in}}{\pgfqpoint{2.536752in}{1.709881in}}%
\pgfpathcurveto{\pgfqpoint{2.542576in}{1.704057in}}{\pgfqpoint{2.550476in}{1.700785in}}{\pgfqpoint{2.558712in}{1.700785in}}%
\pgfpathclose%
\pgfusepath{stroke,fill}%
\end{pgfscope}%
\begin{pgfscope}%
\pgfpathrectangle{\pgfqpoint{0.100000in}{0.220728in}}{\pgfqpoint{3.696000in}{3.696000in}}%
\pgfusepath{clip}%
\pgfsetbuttcap%
\pgfsetroundjoin%
\definecolor{currentfill}{rgb}{0.121569,0.466667,0.705882}%
\pgfsetfillcolor{currentfill}%
\pgfsetfillopacity{0.937627}%
\pgfsetlinewidth{1.003750pt}%
\definecolor{currentstroke}{rgb}{0.121569,0.466667,0.705882}%
\pgfsetstrokecolor{currentstroke}%
\pgfsetstrokeopacity{0.937627}%
\pgfsetdash{}{0pt}%
\pgfpathmoveto{\pgfqpoint{2.557483in}{1.699201in}}%
\pgfpathcurveto{\pgfqpoint{2.565720in}{1.699201in}}{\pgfqpoint{2.573620in}{1.702473in}}{\pgfqpoint{2.579444in}{1.708297in}}%
\pgfpathcurveto{\pgfqpoint{2.585268in}{1.714121in}}{\pgfqpoint{2.588540in}{1.722021in}}{\pgfqpoint{2.588540in}{1.730257in}}%
\pgfpathcurveto{\pgfqpoint{2.588540in}{1.738494in}}{\pgfqpoint{2.585268in}{1.746394in}}{\pgfqpoint{2.579444in}{1.752218in}}%
\pgfpathcurveto{\pgfqpoint{2.573620in}{1.758042in}}{\pgfqpoint{2.565720in}{1.761314in}}{\pgfqpoint{2.557483in}{1.761314in}}%
\pgfpathcurveto{\pgfqpoint{2.549247in}{1.761314in}}{\pgfqpoint{2.541347in}{1.758042in}}{\pgfqpoint{2.535523in}{1.752218in}}%
\pgfpathcurveto{\pgfqpoint{2.529699in}{1.746394in}}{\pgfqpoint{2.526427in}{1.738494in}}{\pgfqpoint{2.526427in}{1.730257in}}%
\pgfpathcurveto{\pgfqpoint{2.526427in}{1.722021in}}{\pgfqpoint{2.529699in}{1.714121in}}{\pgfqpoint{2.535523in}{1.708297in}}%
\pgfpathcurveto{\pgfqpoint{2.541347in}{1.702473in}}{\pgfqpoint{2.549247in}{1.699201in}}{\pgfqpoint{2.557483in}{1.699201in}}%
\pgfpathclose%
\pgfusepath{stroke,fill}%
\end{pgfscope}%
\begin{pgfscope}%
\pgfpathrectangle{\pgfqpoint{0.100000in}{0.220728in}}{\pgfqpoint{3.696000in}{3.696000in}}%
\pgfusepath{clip}%
\pgfsetbuttcap%
\pgfsetroundjoin%
\definecolor{currentfill}{rgb}{0.121569,0.466667,0.705882}%
\pgfsetfillcolor{currentfill}%
\pgfsetfillopacity{0.937828}%
\pgfsetlinewidth{1.003750pt}%
\definecolor{currentstroke}{rgb}{0.121569,0.466667,0.705882}%
\pgfsetstrokecolor{currentstroke}%
\pgfsetstrokeopacity{0.937828}%
\pgfsetdash{}{0pt}%
\pgfpathmoveto{\pgfqpoint{2.556924in}{1.698216in}}%
\pgfpathcurveto{\pgfqpoint{2.565160in}{1.698216in}}{\pgfqpoint{2.573060in}{1.701488in}}{\pgfqpoint{2.578884in}{1.707312in}}%
\pgfpathcurveto{\pgfqpoint{2.584708in}{1.713136in}}{\pgfqpoint{2.587980in}{1.721036in}}{\pgfqpoint{2.587980in}{1.729272in}}%
\pgfpathcurveto{\pgfqpoint{2.587980in}{1.737509in}}{\pgfqpoint{2.584708in}{1.745409in}}{\pgfqpoint{2.578884in}{1.751233in}}%
\pgfpathcurveto{\pgfqpoint{2.573060in}{1.757057in}}{\pgfqpoint{2.565160in}{1.760329in}}{\pgfqpoint{2.556924in}{1.760329in}}%
\pgfpathcurveto{\pgfqpoint{2.548687in}{1.760329in}}{\pgfqpoint{2.540787in}{1.757057in}}{\pgfqpoint{2.534963in}{1.751233in}}%
\pgfpathcurveto{\pgfqpoint{2.529140in}{1.745409in}}{\pgfqpoint{2.525867in}{1.737509in}}{\pgfqpoint{2.525867in}{1.729272in}}%
\pgfpathcurveto{\pgfqpoint{2.525867in}{1.721036in}}{\pgfqpoint{2.529140in}{1.713136in}}{\pgfqpoint{2.534963in}{1.707312in}}%
\pgfpathcurveto{\pgfqpoint{2.540787in}{1.701488in}}{\pgfqpoint{2.548687in}{1.698216in}}{\pgfqpoint{2.556924in}{1.698216in}}%
\pgfpathclose%
\pgfusepath{stroke,fill}%
\end{pgfscope}%
\begin{pgfscope}%
\pgfpathrectangle{\pgfqpoint{0.100000in}{0.220728in}}{\pgfqpoint{3.696000in}{3.696000in}}%
\pgfusepath{clip}%
\pgfsetbuttcap%
\pgfsetroundjoin%
\definecolor{currentfill}{rgb}{0.121569,0.466667,0.705882}%
\pgfsetfillcolor{currentfill}%
\pgfsetfillopacity{0.938304}%
\pgfsetlinewidth{1.003750pt}%
\definecolor{currentstroke}{rgb}{0.121569,0.466667,0.705882}%
\pgfsetstrokecolor{currentstroke}%
\pgfsetstrokeopacity{0.938304}%
\pgfsetdash{}{0pt}%
\pgfpathmoveto{\pgfqpoint{2.054475in}{1.576800in}}%
\pgfpathcurveto{\pgfqpoint{2.062711in}{1.576800in}}{\pgfqpoint{2.070611in}{1.580072in}}{\pgfqpoint{2.076435in}{1.585896in}}%
\pgfpathcurveto{\pgfqpoint{2.082259in}{1.591720in}}{\pgfqpoint{2.085531in}{1.599620in}}{\pgfqpoint{2.085531in}{1.607856in}}%
\pgfpathcurveto{\pgfqpoint{2.085531in}{1.616093in}}{\pgfqpoint{2.082259in}{1.623993in}}{\pgfqpoint{2.076435in}{1.629817in}}%
\pgfpathcurveto{\pgfqpoint{2.070611in}{1.635641in}}{\pgfqpoint{2.062711in}{1.638913in}}{\pgfqpoint{2.054475in}{1.638913in}}%
\pgfpathcurveto{\pgfqpoint{2.046238in}{1.638913in}}{\pgfqpoint{2.038338in}{1.635641in}}{\pgfqpoint{2.032515in}{1.629817in}}%
\pgfpathcurveto{\pgfqpoint{2.026691in}{1.623993in}}{\pgfqpoint{2.023418in}{1.616093in}}{\pgfqpoint{2.023418in}{1.607856in}}%
\pgfpathcurveto{\pgfqpoint{2.023418in}{1.599620in}}{\pgfqpoint{2.026691in}{1.591720in}}{\pgfqpoint{2.032515in}{1.585896in}}%
\pgfpathcurveto{\pgfqpoint{2.038338in}{1.580072in}}{\pgfqpoint{2.046238in}{1.576800in}}{\pgfqpoint{2.054475in}{1.576800in}}%
\pgfpathclose%
\pgfusepath{stroke,fill}%
\end{pgfscope}%
\begin{pgfscope}%
\pgfpathrectangle{\pgfqpoint{0.100000in}{0.220728in}}{\pgfqpoint{3.696000in}{3.696000in}}%
\pgfusepath{clip}%
\pgfsetbuttcap%
\pgfsetroundjoin%
\definecolor{currentfill}{rgb}{0.121569,0.466667,0.705882}%
\pgfsetfillcolor{currentfill}%
\pgfsetfillopacity{0.939105}%
\pgfsetlinewidth{1.003750pt}%
\definecolor{currentstroke}{rgb}{0.121569,0.466667,0.705882}%
\pgfsetstrokecolor{currentstroke}%
\pgfsetstrokeopacity{0.939105}%
\pgfsetdash{}{0pt}%
\pgfpathmoveto{\pgfqpoint{2.552428in}{1.692273in}}%
\pgfpathcurveto{\pgfqpoint{2.560664in}{1.692273in}}{\pgfqpoint{2.568564in}{1.695545in}}{\pgfqpoint{2.574388in}{1.701369in}}%
\pgfpathcurveto{\pgfqpoint{2.580212in}{1.707193in}}{\pgfqpoint{2.583485in}{1.715093in}}{\pgfqpoint{2.583485in}{1.723329in}}%
\pgfpathcurveto{\pgfqpoint{2.583485in}{1.731566in}}{\pgfqpoint{2.580212in}{1.739466in}}{\pgfqpoint{2.574388in}{1.745290in}}%
\pgfpathcurveto{\pgfqpoint{2.568564in}{1.751113in}}{\pgfqpoint{2.560664in}{1.754386in}}{\pgfqpoint{2.552428in}{1.754386in}}%
\pgfpathcurveto{\pgfqpoint{2.544192in}{1.754386in}}{\pgfqpoint{2.536292in}{1.751113in}}{\pgfqpoint{2.530468in}{1.745290in}}%
\pgfpathcurveto{\pgfqpoint{2.524644in}{1.739466in}}{\pgfqpoint{2.521372in}{1.731566in}}{\pgfqpoint{2.521372in}{1.723329in}}%
\pgfpathcurveto{\pgfqpoint{2.521372in}{1.715093in}}{\pgfqpoint{2.524644in}{1.707193in}}{\pgfqpoint{2.530468in}{1.701369in}}%
\pgfpathcurveto{\pgfqpoint{2.536292in}{1.695545in}}{\pgfqpoint{2.544192in}{1.692273in}}{\pgfqpoint{2.552428in}{1.692273in}}%
\pgfpathclose%
\pgfusepath{stroke,fill}%
\end{pgfscope}%
\begin{pgfscope}%
\pgfpathrectangle{\pgfqpoint{0.100000in}{0.220728in}}{\pgfqpoint{3.696000in}{3.696000in}}%
\pgfusepath{clip}%
\pgfsetbuttcap%
\pgfsetroundjoin%
\definecolor{currentfill}{rgb}{0.121569,0.466667,0.705882}%
\pgfsetfillcolor{currentfill}%
\pgfsetfillopacity{0.940805}%
\pgfsetlinewidth{1.003750pt}%
\definecolor{currentstroke}{rgb}{0.121569,0.466667,0.705882}%
\pgfsetstrokecolor{currentstroke}%
\pgfsetstrokeopacity{0.940805}%
\pgfsetdash{}{0pt}%
\pgfpathmoveto{\pgfqpoint{2.546587in}{1.682010in}}%
\pgfpathcurveto{\pgfqpoint{2.554823in}{1.682010in}}{\pgfqpoint{2.562723in}{1.685282in}}{\pgfqpoint{2.568547in}{1.691106in}}%
\pgfpathcurveto{\pgfqpoint{2.574371in}{1.696930in}}{\pgfqpoint{2.577643in}{1.704830in}}{\pgfqpoint{2.577643in}{1.713066in}}%
\pgfpathcurveto{\pgfqpoint{2.577643in}{1.721302in}}{\pgfqpoint{2.574371in}{1.729202in}}{\pgfqpoint{2.568547in}{1.735026in}}%
\pgfpathcurveto{\pgfqpoint{2.562723in}{1.740850in}}{\pgfqpoint{2.554823in}{1.744123in}}{\pgfqpoint{2.546587in}{1.744123in}}%
\pgfpathcurveto{\pgfqpoint{2.538350in}{1.744123in}}{\pgfqpoint{2.530450in}{1.740850in}}{\pgfqpoint{2.524626in}{1.735026in}}%
\pgfpathcurveto{\pgfqpoint{2.518802in}{1.729202in}}{\pgfqpoint{2.515530in}{1.721302in}}{\pgfqpoint{2.515530in}{1.713066in}}%
\pgfpathcurveto{\pgfqpoint{2.515530in}{1.704830in}}{\pgfqpoint{2.518802in}{1.696930in}}{\pgfqpoint{2.524626in}{1.691106in}}%
\pgfpathcurveto{\pgfqpoint{2.530450in}{1.685282in}}{\pgfqpoint{2.538350in}{1.682010in}}{\pgfqpoint{2.546587in}{1.682010in}}%
\pgfpathclose%
\pgfusepath{stroke,fill}%
\end{pgfscope}%
\begin{pgfscope}%
\pgfpathrectangle{\pgfqpoint{0.100000in}{0.220728in}}{\pgfqpoint{3.696000in}{3.696000in}}%
\pgfusepath{clip}%
\pgfsetbuttcap%
\pgfsetroundjoin%
\definecolor{currentfill}{rgb}{0.121569,0.466667,0.705882}%
\pgfsetfillcolor{currentfill}%
\pgfsetfillopacity{0.943187}%
\pgfsetlinewidth{1.003750pt}%
\definecolor{currentstroke}{rgb}{0.121569,0.466667,0.705882}%
\pgfsetstrokecolor{currentstroke}%
\pgfsetstrokeopacity{0.943187}%
\pgfsetdash{}{0pt}%
\pgfpathmoveto{\pgfqpoint{2.078117in}{1.564670in}}%
\pgfpathcurveto{\pgfqpoint{2.086353in}{1.564670in}}{\pgfqpoint{2.094253in}{1.567943in}}{\pgfqpoint{2.100077in}{1.573767in}}%
\pgfpathcurveto{\pgfqpoint{2.105901in}{1.579591in}}{\pgfqpoint{2.109173in}{1.587491in}}{\pgfqpoint{2.109173in}{1.595727in}}%
\pgfpathcurveto{\pgfqpoint{2.109173in}{1.603963in}}{\pgfqpoint{2.105901in}{1.611863in}}{\pgfqpoint{2.100077in}{1.617687in}}%
\pgfpathcurveto{\pgfqpoint{2.094253in}{1.623511in}}{\pgfqpoint{2.086353in}{1.626783in}}{\pgfqpoint{2.078117in}{1.626783in}}%
\pgfpathcurveto{\pgfqpoint{2.069880in}{1.626783in}}{\pgfqpoint{2.061980in}{1.623511in}}{\pgfqpoint{2.056156in}{1.617687in}}%
\pgfpathcurveto{\pgfqpoint{2.050333in}{1.611863in}}{\pgfqpoint{2.047060in}{1.603963in}}{\pgfqpoint{2.047060in}{1.595727in}}%
\pgfpathcurveto{\pgfqpoint{2.047060in}{1.587491in}}{\pgfqpoint{2.050333in}{1.579591in}}{\pgfqpoint{2.056156in}{1.573767in}}%
\pgfpathcurveto{\pgfqpoint{2.061980in}{1.567943in}}{\pgfqpoint{2.069880in}{1.564670in}}{\pgfqpoint{2.078117in}{1.564670in}}%
\pgfpathclose%
\pgfusepath{stroke,fill}%
\end{pgfscope}%
\begin{pgfscope}%
\pgfpathrectangle{\pgfqpoint{0.100000in}{0.220728in}}{\pgfqpoint{3.696000in}{3.696000in}}%
\pgfusepath{clip}%
\pgfsetbuttcap%
\pgfsetroundjoin%
\definecolor{currentfill}{rgb}{0.121569,0.466667,0.705882}%
\pgfsetfillcolor{currentfill}%
\pgfsetfillopacity{0.943458}%
\pgfsetlinewidth{1.003750pt}%
\definecolor{currentstroke}{rgb}{0.121569,0.466667,0.705882}%
\pgfsetstrokecolor{currentstroke}%
\pgfsetstrokeopacity{0.943458}%
\pgfsetdash{}{0pt}%
\pgfpathmoveto{\pgfqpoint{2.537491in}{1.669683in}}%
\pgfpathcurveto{\pgfqpoint{2.545727in}{1.669683in}}{\pgfqpoint{2.553627in}{1.672955in}}{\pgfqpoint{2.559451in}{1.678779in}}%
\pgfpathcurveto{\pgfqpoint{2.565275in}{1.684603in}}{\pgfqpoint{2.568547in}{1.692503in}}{\pgfqpoint{2.568547in}{1.700739in}}%
\pgfpathcurveto{\pgfqpoint{2.568547in}{1.708975in}}{\pgfqpoint{2.565275in}{1.716875in}}{\pgfqpoint{2.559451in}{1.722699in}}%
\pgfpathcurveto{\pgfqpoint{2.553627in}{1.728523in}}{\pgfqpoint{2.545727in}{1.731796in}}{\pgfqpoint{2.537491in}{1.731796in}}%
\pgfpathcurveto{\pgfqpoint{2.529254in}{1.731796in}}{\pgfqpoint{2.521354in}{1.728523in}}{\pgfqpoint{2.515530in}{1.722699in}}%
\pgfpathcurveto{\pgfqpoint{2.509706in}{1.716875in}}{\pgfqpoint{2.506434in}{1.708975in}}{\pgfqpoint{2.506434in}{1.700739in}}%
\pgfpathcurveto{\pgfqpoint{2.506434in}{1.692503in}}{\pgfqpoint{2.509706in}{1.684603in}}{\pgfqpoint{2.515530in}{1.678779in}}%
\pgfpathcurveto{\pgfqpoint{2.521354in}{1.672955in}}{\pgfqpoint{2.529254in}{1.669683in}}{\pgfqpoint{2.537491in}{1.669683in}}%
\pgfpathclose%
\pgfusepath{stroke,fill}%
\end{pgfscope}%
\begin{pgfscope}%
\pgfpathrectangle{\pgfqpoint{0.100000in}{0.220728in}}{\pgfqpoint{3.696000in}{3.696000in}}%
\pgfusepath{clip}%
\pgfsetbuttcap%
\pgfsetroundjoin%
\definecolor{currentfill}{rgb}{0.121569,0.466667,0.705882}%
\pgfsetfillcolor{currentfill}%
\pgfsetfillopacity{0.945005}%
\pgfsetlinewidth{1.003750pt}%
\definecolor{currentstroke}{rgb}{0.121569,0.466667,0.705882}%
\pgfsetstrokecolor{currentstroke}%
\pgfsetstrokeopacity{0.945005}%
\pgfsetdash{}{0pt}%
\pgfpathmoveto{\pgfqpoint{2.533269in}{1.662196in}}%
\pgfpathcurveto{\pgfqpoint{2.541506in}{1.662196in}}{\pgfqpoint{2.549406in}{1.665468in}}{\pgfqpoint{2.555230in}{1.671292in}}%
\pgfpathcurveto{\pgfqpoint{2.561054in}{1.677116in}}{\pgfqpoint{2.564326in}{1.685016in}}{\pgfqpoint{2.564326in}{1.693252in}}%
\pgfpathcurveto{\pgfqpoint{2.564326in}{1.701489in}}{\pgfqpoint{2.561054in}{1.709389in}}{\pgfqpoint{2.555230in}{1.715213in}}%
\pgfpathcurveto{\pgfqpoint{2.549406in}{1.721037in}}{\pgfqpoint{2.541506in}{1.724309in}}{\pgfqpoint{2.533269in}{1.724309in}}%
\pgfpathcurveto{\pgfqpoint{2.525033in}{1.724309in}}{\pgfqpoint{2.517133in}{1.721037in}}{\pgfqpoint{2.511309in}{1.715213in}}%
\pgfpathcurveto{\pgfqpoint{2.505485in}{1.709389in}}{\pgfqpoint{2.502213in}{1.701489in}}{\pgfqpoint{2.502213in}{1.693252in}}%
\pgfpathcurveto{\pgfqpoint{2.502213in}{1.685016in}}{\pgfqpoint{2.505485in}{1.677116in}}{\pgfqpoint{2.511309in}{1.671292in}}%
\pgfpathcurveto{\pgfqpoint{2.517133in}{1.665468in}}{\pgfqpoint{2.525033in}{1.662196in}}{\pgfqpoint{2.533269in}{1.662196in}}%
\pgfpathclose%
\pgfusepath{stroke,fill}%
\end{pgfscope}%
\begin{pgfscope}%
\pgfpathrectangle{\pgfqpoint{0.100000in}{0.220728in}}{\pgfqpoint{3.696000in}{3.696000in}}%
\pgfusepath{clip}%
\pgfsetbuttcap%
\pgfsetroundjoin%
\definecolor{currentfill}{rgb}{0.121569,0.466667,0.705882}%
\pgfsetfillcolor{currentfill}%
\pgfsetfillopacity{0.945846}%
\pgfsetlinewidth{1.003750pt}%
\definecolor{currentstroke}{rgb}{0.121569,0.466667,0.705882}%
\pgfsetstrokecolor{currentstroke}%
\pgfsetstrokeopacity{0.945846}%
\pgfsetdash{}{0pt}%
\pgfpathmoveto{\pgfqpoint{2.530713in}{1.658370in}}%
\pgfpathcurveto{\pgfqpoint{2.538950in}{1.658370in}}{\pgfqpoint{2.546850in}{1.661643in}}{\pgfqpoint{2.552674in}{1.667467in}}%
\pgfpathcurveto{\pgfqpoint{2.558497in}{1.673291in}}{\pgfqpoint{2.561770in}{1.681191in}}{\pgfqpoint{2.561770in}{1.689427in}}%
\pgfpathcurveto{\pgfqpoint{2.561770in}{1.697663in}}{\pgfqpoint{2.558497in}{1.705563in}}{\pgfqpoint{2.552674in}{1.711387in}}%
\pgfpathcurveto{\pgfqpoint{2.546850in}{1.717211in}}{\pgfqpoint{2.538950in}{1.720483in}}{\pgfqpoint{2.530713in}{1.720483in}}%
\pgfpathcurveto{\pgfqpoint{2.522477in}{1.720483in}}{\pgfqpoint{2.514577in}{1.717211in}}{\pgfqpoint{2.508753in}{1.711387in}}%
\pgfpathcurveto{\pgfqpoint{2.502929in}{1.705563in}}{\pgfqpoint{2.499657in}{1.697663in}}{\pgfqpoint{2.499657in}{1.689427in}}%
\pgfpathcurveto{\pgfqpoint{2.499657in}{1.681191in}}{\pgfqpoint{2.502929in}{1.673291in}}{\pgfqpoint{2.508753in}{1.667467in}}%
\pgfpathcurveto{\pgfqpoint{2.514577in}{1.661643in}}{\pgfqpoint{2.522477in}{1.658370in}}{\pgfqpoint{2.530713in}{1.658370in}}%
\pgfpathclose%
\pgfusepath{stroke,fill}%
\end{pgfscope}%
\begin{pgfscope}%
\pgfpathrectangle{\pgfqpoint{0.100000in}{0.220728in}}{\pgfqpoint{3.696000in}{3.696000in}}%
\pgfusepath{clip}%
\pgfsetbuttcap%
\pgfsetroundjoin%
\definecolor{currentfill}{rgb}{0.121569,0.466667,0.705882}%
\pgfsetfillcolor{currentfill}%
\pgfsetfillopacity{0.946283}%
\pgfsetlinewidth{1.003750pt}%
\definecolor{currentstroke}{rgb}{0.121569,0.466667,0.705882}%
\pgfsetstrokecolor{currentstroke}%
\pgfsetstrokeopacity{0.946283}%
\pgfsetdash{}{0pt}%
\pgfpathmoveto{\pgfqpoint{2.529200in}{1.656283in}}%
\pgfpathcurveto{\pgfqpoint{2.537437in}{1.656283in}}{\pgfqpoint{2.545337in}{1.659555in}}{\pgfqpoint{2.551161in}{1.665379in}}%
\pgfpathcurveto{\pgfqpoint{2.556985in}{1.671203in}}{\pgfqpoint{2.560257in}{1.679103in}}{\pgfqpoint{2.560257in}{1.687339in}}%
\pgfpathcurveto{\pgfqpoint{2.560257in}{1.695575in}}{\pgfqpoint{2.556985in}{1.703475in}}{\pgfqpoint{2.551161in}{1.709299in}}%
\pgfpathcurveto{\pgfqpoint{2.545337in}{1.715123in}}{\pgfqpoint{2.537437in}{1.718396in}}{\pgfqpoint{2.529200in}{1.718396in}}%
\pgfpathcurveto{\pgfqpoint{2.520964in}{1.718396in}}{\pgfqpoint{2.513064in}{1.715123in}}{\pgfqpoint{2.507240in}{1.709299in}}%
\pgfpathcurveto{\pgfqpoint{2.501416in}{1.703475in}}{\pgfqpoint{2.498144in}{1.695575in}}{\pgfqpoint{2.498144in}{1.687339in}}%
\pgfpathcurveto{\pgfqpoint{2.498144in}{1.679103in}}{\pgfqpoint{2.501416in}{1.671203in}}{\pgfqpoint{2.507240in}{1.665379in}}%
\pgfpathcurveto{\pgfqpoint{2.513064in}{1.659555in}}{\pgfqpoint{2.520964in}{1.656283in}}{\pgfqpoint{2.529200in}{1.656283in}}%
\pgfpathclose%
\pgfusepath{stroke,fill}%
\end{pgfscope}%
\begin{pgfscope}%
\pgfpathrectangle{\pgfqpoint{0.100000in}{0.220728in}}{\pgfqpoint{3.696000in}{3.696000in}}%
\pgfusepath{clip}%
\pgfsetbuttcap%
\pgfsetroundjoin%
\definecolor{currentfill}{rgb}{0.121569,0.466667,0.705882}%
\pgfsetfillcolor{currentfill}%
\pgfsetfillopacity{0.946547}%
\pgfsetlinewidth{1.003750pt}%
\definecolor{currentstroke}{rgb}{0.121569,0.466667,0.705882}%
\pgfsetstrokecolor{currentstroke}%
\pgfsetstrokeopacity{0.946547}%
\pgfsetdash{}{0pt}%
\pgfpathmoveto{\pgfqpoint{2.095235in}{1.557576in}}%
\pgfpathcurveto{\pgfqpoint{2.103472in}{1.557576in}}{\pgfqpoint{2.111372in}{1.560848in}}{\pgfqpoint{2.117196in}{1.566672in}}%
\pgfpathcurveto{\pgfqpoint{2.123020in}{1.572496in}}{\pgfqpoint{2.126292in}{1.580396in}}{\pgfqpoint{2.126292in}{1.588633in}}%
\pgfpathcurveto{\pgfqpoint{2.126292in}{1.596869in}}{\pgfqpoint{2.123020in}{1.604769in}}{\pgfqpoint{2.117196in}{1.610593in}}%
\pgfpathcurveto{\pgfqpoint{2.111372in}{1.616417in}}{\pgfqpoint{2.103472in}{1.619689in}}{\pgfqpoint{2.095235in}{1.619689in}}%
\pgfpathcurveto{\pgfqpoint{2.086999in}{1.619689in}}{\pgfqpoint{2.079099in}{1.616417in}}{\pgfqpoint{2.073275in}{1.610593in}}%
\pgfpathcurveto{\pgfqpoint{2.067451in}{1.604769in}}{\pgfqpoint{2.064179in}{1.596869in}}{\pgfqpoint{2.064179in}{1.588633in}}%
\pgfpathcurveto{\pgfqpoint{2.064179in}{1.580396in}}{\pgfqpoint{2.067451in}{1.572496in}}{\pgfqpoint{2.073275in}{1.566672in}}%
\pgfpathcurveto{\pgfqpoint{2.079099in}{1.560848in}}{\pgfqpoint{2.086999in}{1.557576in}}{\pgfqpoint{2.095235in}{1.557576in}}%
\pgfpathclose%
\pgfusepath{stroke,fill}%
\end{pgfscope}%
\begin{pgfscope}%
\pgfpathrectangle{\pgfqpoint{0.100000in}{0.220728in}}{\pgfqpoint{3.696000in}{3.696000in}}%
\pgfusepath{clip}%
\pgfsetbuttcap%
\pgfsetroundjoin%
\definecolor{currentfill}{rgb}{0.121569,0.466667,0.705882}%
\pgfsetfillcolor{currentfill}%
\pgfsetfillopacity{0.946551}%
\pgfsetlinewidth{1.003750pt}%
\definecolor{currentstroke}{rgb}{0.121569,0.466667,0.705882}%
\pgfsetstrokecolor{currentstroke}%
\pgfsetstrokeopacity{0.946551}%
\pgfsetdash{}{0pt}%
\pgfpathmoveto{\pgfqpoint{2.528472in}{1.655125in}}%
\pgfpathcurveto{\pgfqpoint{2.536708in}{1.655125in}}{\pgfqpoint{2.544608in}{1.658398in}}{\pgfqpoint{2.550432in}{1.664222in}}%
\pgfpathcurveto{\pgfqpoint{2.556256in}{1.670046in}}{\pgfqpoint{2.559528in}{1.677946in}}{\pgfqpoint{2.559528in}{1.686182in}}%
\pgfpathcurveto{\pgfqpoint{2.559528in}{1.694418in}}{\pgfqpoint{2.556256in}{1.702318in}}{\pgfqpoint{2.550432in}{1.708142in}}%
\pgfpathcurveto{\pgfqpoint{2.544608in}{1.713966in}}{\pgfqpoint{2.536708in}{1.717238in}}{\pgfqpoint{2.528472in}{1.717238in}}%
\pgfpathcurveto{\pgfqpoint{2.520236in}{1.717238in}}{\pgfqpoint{2.512335in}{1.713966in}}{\pgfqpoint{2.506512in}{1.708142in}}%
\pgfpathcurveto{\pgfqpoint{2.500688in}{1.702318in}}{\pgfqpoint{2.497415in}{1.694418in}}{\pgfqpoint{2.497415in}{1.686182in}}%
\pgfpathcurveto{\pgfqpoint{2.497415in}{1.677946in}}{\pgfqpoint{2.500688in}{1.670046in}}{\pgfqpoint{2.506512in}{1.664222in}}%
\pgfpathcurveto{\pgfqpoint{2.512335in}{1.658398in}}{\pgfqpoint{2.520236in}{1.655125in}}{\pgfqpoint{2.528472in}{1.655125in}}%
\pgfpathclose%
\pgfusepath{stroke,fill}%
\end{pgfscope}%
\begin{pgfscope}%
\pgfpathrectangle{\pgfqpoint{0.100000in}{0.220728in}}{\pgfqpoint{3.696000in}{3.696000in}}%
\pgfusepath{clip}%
\pgfsetbuttcap%
\pgfsetroundjoin%
\definecolor{currentfill}{rgb}{0.121569,0.466667,0.705882}%
\pgfsetfillcolor{currentfill}%
\pgfsetfillopacity{0.946678}%
\pgfsetlinewidth{1.003750pt}%
\definecolor{currentstroke}{rgb}{0.121569,0.466667,0.705882}%
\pgfsetstrokecolor{currentstroke}%
\pgfsetstrokeopacity{0.946678}%
\pgfsetdash{}{0pt}%
\pgfpathmoveto{\pgfqpoint{2.527992in}{1.654502in}}%
\pgfpathcurveto{\pgfqpoint{2.536228in}{1.654502in}}{\pgfqpoint{2.544128in}{1.657774in}}{\pgfqpoint{2.549952in}{1.663598in}}%
\pgfpathcurveto{\pgfqpoint{2.555776in}{1.669422in}}{\pgfqpoint{2.559049in}{1.677322in}}{\pgfqpoint{2.559049in}{1.685558in}}%
\pgfpathcurveto{\pgfqpoint{2.559049in}{1.693794in}}{\pgfqpoint{2.555776in}{1.701694in}}{\pgfqpoint{2.549952in}{1.707518in}}%
\pgfpathcurveto{\pgfqpoint{2.544128in}{1.713342in}}{\pgfqpoint{2.536228in}{1.716615in}}{\pgfqpoint{2.527992in}{1.716615in}}%
\pgfpathcurveto{\pgfqpoint{2.519756in}{1.716615in}}{\pgfqpoint{2.511856in}{1.713342in}}{\pgfqpoint{2.506032in}{1.707518in}}%
\pgfpathcurveto{\pgfqpoint{2.500208in}{1.701694in}}{\pgfqpoint{2.496936in}{1.693794in}}{\pgfqpoint{2.496936in}{1.685558in}}%
\pgfpathcurveto{\pgfqpoint{2.496936in}{1.677322in}}{\pgfqpoint{2.500208in}{1.669422in}}{\pgfqpoint{2.506032in}{1.663598in}}%
\pgfpathcurveto{\pgfqpoint{2.511856in}{1.657774in}}{\pgfqpoint{2.519756in}{1.654502in}}{\pgfqpoint{2.527992in}{1.654502in}}%
\pgfpathclose%
\pgfusepath{stroke,fill}%
\end{pgfscope}%
\begin{pgfscope}%
\pgfpathrectangle{\pgfqpoint{0.100000in}{0.220728in}}{\pgfqpoint{3.696000in}{3.696000in}}%
\pgfusepath{clip}%
\pgfsetbuttcap%
\pgfsetroundjoin%
\definecolor{currentfill}{rgb}{0.121569,0.466667,0.705882}%
\pgfsetfillcolor{currentfill}%
\pgfsetfillopacity{0.946750}%
\pgfsetlinewidth{1.003750pt}%
\definecolor{currentstroke}{rgb}{0.121569,0.466667,0.705882}%
\pgfsetstrokecolor{currentstroke}%
\pgfsetstrokeopacity{0.946750}%
\pgfsetdash{}{0pt}%
\pgfpathmoveto{\pgfqpoint{2.527770in}{1.654112in}}%
\pgfpathcurveto{\pgfqpoint{2.536007in}{1.654112in}}{\pgfqpoint{2.543907in}{1.657384in}}{\pgfqpoint{2.549730in}{1.663208in}}%
\pgfpathcurveto{\pgfqpoint{2.555554in}{1.669032in}}{\pgfqpoint{2.558827in}{1.676932in}}{\pgfqpoint{2.558827in}{1.685168in}}%
\pgfpathcurveto{\pgfqpoint{2.558827in}{1.693405in}}{\pgfqpoint{2.555554in}{1.701305in}}{\pgfqpoint{2.549730in}{1.707129in}}%
\pgfpathcurveto{\pgfqpoint{2.543907in}{1.712953in}}{\pgfqpoint{2.536007in}{1.716225in}}{\pgfqpoint{2.527770in}{1.716225in}}%
\pgfpathcurveto{\pgfqpoint{2.519534in}{1.716225in}}{\pgfqpoint{2.511634in}{1.712953in}}{\pgfqpoint{2.505810in}{1.707129in}}%
\pgfpathcurveto{\pgfqpoint{2.499986in}{1.701305in}}{\pgfqpoint{2.496714in}{1.693405in}}{\pgfqpoint{2.496714in}{1.685168in}}%
\pgfpathcurveto{\pgfqpoint{2.496714in}{1.676932in}}{\pgfqpoint{2.499986in}{1.669032in}}{\pgfqpoint{2.505810in}{1.663208in}}%
\pgfpathcurveto{\pgfqpoint{2.511634in}{1.657384in}}{\pgfqpoint{2.519534in}{1.654112in}}{\pgfqpoint{2.527770in}{1.654112in}}%
\pgfpathclose%
\pgfusepath{stroke,fill}%
\end{pgfscope}%
\begin{pgfscope}%
\pgfpathrectangle{\pgfqpoint{0.100000in}{0.220728in}}{\pgfqpoint{3.696000in}{3.696000in}}%
\pgfusepath{clip}%
\pgfsetbuttcap%
\pgfsetroundjoin%
\definecolor{currentfill}{rgb}{0.121569,0.466667,0.705882}%
\pgfsetfillcolor{currentfill}%
\pgfsetfillopacity{0.947575}%
\pgfsetlinewidth{1.003750pt}%
\definecolor{currentstroke}{rgb}{0.121569,0.466667,0.705882}%
\pgfsetstrokecolor{currentstroke}%
\pgfsetstrokeopacity{0.947575}%
\pgfsetdash{}{0pt}%
\pgfpathmoveto{\pgfqpoint{2.525322in}{1.651009in}}%
\pgfpathcurveto{\pgfqpoint{2.533558in}{1.651009in}}{\pgfqpoint{2.541458in}{1.654282in}}{\pgfqpoint{2.547282in}{1.660106in}}%
\pgfpathcurveto{\pgfqpoint{2.553106in}{1.665929in}}{\pgfqpoint{2.556378in}{1.673829in}}{\pgfqpoint{2.556378in}{1.682066in}}%
\pgfpathcurveto{\pgfqpoint{2.556378in}{1.690302in}}{\pgfqpoint{2.553106in}{1.698202in}}{\pgfqpoint{2.547282in}{1.704026in}}%
\pgfpathcurveto{\pgfqpoint{2.541458in}{1.709850in}}{\pgfqpoint{2.533558in}{1.713122in}}{\pgfqpoint{2.525322in}{1.713122in}}%
\pgfpathcurveto{\pgfqpoint{2.517086in}{1.713122in}}{\pgfqpoint{2.509185in}{1.709850in}}{\pgfqpoint{2.503362in}{1.704026in}}%
\pgfpathcurveto{\pgfqpoint{2.497538in}{1.698202in}}{\pgfqpoint{2.494265in}{1.690302in}}{\pgfqpoint{2.494265in}{1.682066in}}%
\pgfpathcurveto{\pgfqpoint{2.494265in}{1.673829in}}{\pgfqpoint{2.497538in}{1.665929in}}{\pgfqpoint{2.503362in}{1.660106in}}%
\pgfpathcurveto{\pgfqpoint{2.509185in}{1.654282in}}{\pgfqpoint{2.517086in}{1.651009in}}{\pgfqpoint{2.525322in}{1.651009in}}%
\pgfpathclose%
\pgfusepath{stroke,fill}%
\end{pgfscope}%
\begin{pgfscope}%
\pgfpathrectangle{\pgfqpoint{0.100000in}{0.220728in}}{\pgfqpoint{3.696000in}{3.696000in}}%
\pgfusepath{clip}%
\pgfsetbuttcap%
\pgfsetroundjoin%
\definecolor{currentfill}{rgb}{0.121569,0.466667,0.705882}%
\pgfsetfillcolor{currentfill}%
\pgfsetfillopacity{0.949110}%
\pgfsetlinewidth{1.003750pt}%
\definecolor{currentstroke}{rgb}{0.121569,0.466667,0.705882}%
\pgfsetstrokecolor{currentstroke}%
\pgfsetstrokeopacity{0.949110}%
\pgfsetdash{}{0pt}%
\pgfpathmoveto{\pgfqpoint{2.521607in}{1.643301in}}%
\pgfpathcurveto{\pgfqpoint{2.529843in}{1.643301in}}{\pgfqpoint{2.537743in}{1.646573in}}{\pgfqpoint{2.543567in}{1.652397in}}%
\pgfpathcurveto{\pgfqpoint{2.549391in}{1.658221in}}{\pgfqpoint{2.552663in}{1.666121in}}{\pgfqpoint{2.552663in}{1.674357in}}%
\pgfpathcurveto{\pgfqpoint{2.552663in}{1.682593in}}{\pgfqpoint{2.549391in}{1.690493in}}{\pgfqpoint{2.543567in}{1.696317in}}%
\pgfpathcurveto{\pgfqpoint{2.537743in}{1.702141in}}{\pgfqpoint{2.529843in}{1.705414in}}{\pgfqpoint{2.521607in}{1.705414in}}%
\pgfpathcurveto{\pgfqpoint{2.513371in}{1.705414in}}{\pgfqpoint{2.505471in}{1.702141in}}{\pgfqpoint{2.499647in}{1.696317in}}%
\pgfpathcurveto{\pgfqpoint{2.493823in}{1.690493in}}{\pgfqpoint{2.490550in}{1.682593in}}{\pgfqpoint{2.490550in}{1.674357in}}%
\pgfpathcurveto{\pgfqpoint{2.490550in}{1.666121in}}{\pgfqpoint{2.493823in}{1.658221in}}{\pgfqpoint{2.499647in}{1.652397in}}%
\pgfpathcurveto{\pgfqpoint{2.505471in}{1.646573in}}{\pgfqpoint{2.513371in}{1.643301in}}{\pgfqpoint{2.521607in}{1.643301in}}%
\pgfpathclose%
\pgfusepath{stroke,fill}%
\end{pgfscope}%
\begin{pgfscope}%
\pgfpathrectangle{\pgfqpoint{0.100000in}{0.220728in}}{\pgfqpoint{3.696000in}{3.696000in}}%
\pgfusepath{clip}%
\pgfsetbuttcap%
\pgfsetroundjoin%
\definecolor{currentfill}{rgb}{0.121569,0.466667,0.705882}%
\pgfsetfillcolor{currentfill}%
\pgfsetfillopacity{0.949179}%
\pgfsetlinewidth{1.003750pt}%
\definecolor{currentstroke}{rgb}{0.121569,0.466667,0.705882}%
\pgfsetstrokecolor{currentstroke}%
\pgfsetstrokeopacity{0.949179}%
\pgfsetdash{}{0pt}%
\pgfpathmoveto{\pgfqpoint{2.107860in}{1.551769in}}%
\pgfpathcurveto{\pgfqpoint{2.116097in}{1.551769in}}{\pgfqpoint{2.123997in}{1.555042in}}{\pgfqpoint{2.129821in}{1.560866in}}%
\pgfpathcurveto{\pgfqpoint{2.135644in}{1.566690in}}{\pgfqpoint{2.138917in}{1.574590in}}{\pgfqpoint{2.138917in}{1.582826in}}%
\pgfpathcurveto{\pgfqpoint{2.138917in}{1.591062in}}{\pgfqpoint{2.135644in}{1.598962in}}{\pgfqpoint{2.129821in}{1.604786in}}%
\pgfpathcurveto{\pgfqpoint{2.123997in}{1.610610in}}{\pgfqpoint{2.116097in}{1.613882in}}{\pgfqpoint{2.107860in}{1.613882in}}%
\pgfpathcurveto{\pgfqpoint{2.099624in}{1.613882in}}{\pgfqpoint{2.091724in}{1.610610in}}{\pgfqpoint{2.085900in}{1.604786in}}%
\pgfpathcurveto{\pgfqpoint{2.080076in}{1.598962in}}{\pgfqpoint{2.076804in}{1.591062in}}{\pgfqpoint{2.076804in}{1.582826in}}%
\pgfpathcurveto{\pgfqpoint{2.076804in}{1.574590in}}{\pgfqpoint{2.080076in}{1.566690in}}{\pgfqpoint{2.085900in}{1.560866in}}%
\pgfpathcurveto{\pgfqpoint{2.091724in}{1.555042in}}{\pgfqpoint{2.099624in}{1.551769in}}{\pgfqpoint{2.107860in}{1.551769in}}%
\pgfpathclose%
\pgfusepath{stroke,fill}%
\end{pgfscope}%
\begin{pgfscope}%
\pgfpathrectangle{\pgfqpoint{0.100000in}{0.220728in}}{\pgfqpoint{3.696000in}{3.696000in}}%
\pgfusepath{clip}%
\pgfsetbuttcap%
\pgfsetroundjoin%
\definecolor{currentfill}{rgb}{0.121569,0.466667,0.705882}%
\pgfsetfillcolor{currentfill}%
\pgfsetfillopacity{0.950921}%
\pgfsetlinewidth{1.003750pt}%
\definecolor{currentstroke}{rgb}{0.121569,0.466667,0.705882}%
\pgfsetstrokecolor{currentstroke}%
\pgfsetstrokeopacity{0.950921}%
\pgfsetdash{}{0pt}%
\pgfpathmoveto{\pgfqpoint{2.117611in}{1.549032in}}%
\pgfpathcurveto{\pgfqpoint{2.125847in}{1.549032in}}{\pgfqpoint{2.133748in}{1.552304in}}{\pgfqpoint{2.139571in}{1.558128in}}%
\pgfpathcurveto{\pgfqpoint{2.145395in}{1.563952in}}{\pgfqpoint{2.148668in}{1.571852in}}{\pgfqpoint{2.148668in}{1.580088in}}%
\pgfpathcurveto{\pgfqpoint{2.148668in}{1.588324in}}{\pgfqpoint{2.145395in}{1.596224in}}{\pgfqpoint{2.139571in}{1.602048in}}%
\pgfpathcurveto{\pgfqpoint{2.133748in}{1.607872in}}{\pgfqpoint{2.125847in}{1.611145in}}{\pgfqpoint{2.117611in}{1.611145in}}%
\pgfpathcurveto{\pgfqpoint{2.109375in}{1.611145in}}{\pgfqpoint{2.101475in}{1.607872in}}{\pgfqpoint{2.095651in}{1.602048in}}%
\pgfpathcurveto{\pgfqpoint{2.089827in}{1.596224in}}{\pgfqpoint{2.086555in}{1.588324in}}{\pgfqpoint{2.086555in}{1.580088in}}%
\pgfpathcurveto{\pgfqpoint{2.086555in}{1.571852in}}{\pgfqpoint{2.089827in}{1.563952in}}{\pgfqpoint{2.095651in}{1.558128in}}%
\pgfpathcurveto{\pgfqpoint{2.101475in}{1.552304in}}{\pgfqpoint{2.109375in}{1.549032in}}{\pgfqpoint{2.117611in}{1.549032in}}%
\pgfpathclose%
\pgfusepath{stroke,fill}%
\end{pgfscope}%
\begin{pgfscope}%
\pgfpathrectangle{\pgfqpoint{0.100000in}{0.220728in}}{\pgfqpoint{3.696000in}{3.696000in}}%
\pgfusepath{clip}%
\pgfsetbuttcap%
\pgfsetroundjoin%
\definecolor{currentfill}{rgb}{0.121569,0.466667,0.705882}%
\pgfsetfillcolor{currentfill}%
\pgfsetfillopacity{0.951482}%
\pgfsetlinewidth{1.003750pt}%
\definecolor{currentstroke}{rgb}{0.121569,0.466667,0.705882}%
\pgfsetstrokecolor{currentstroke}%
\pgfsetstrokeopacity{0.951482}%
\pgfsetdash{}{0pt}%
\pgfpathmoveto{\pgfqpoint{2.512740in}{1.632334in}}%
\pgfpathcurveto{\pgfqpoint{2.520977in}{1.632334in}}{\pgfqpoint{2.528877in}{1.635607in}}{\pgfqpoint{2.534701in}{1.641431in}}%
\pgfpathcurveto{\pgfqpoint{2.540525in}{1.647255in}}{\pgfqpoint{2.543797in}{1.655155in}}{\pgfqpoint{2.543797in}{1.663391in}}%
\pgfpathcurveto{\pgfqpoint{2.543797in}{1.671627in}}{\pgfqpoint{2.540525in}{1.679527in}}{\pgfqpoint{2.534701in}{1.685351in}}%
\pgfpathcurveto{\pgfqpoint{2.528877in}{1.691175in}}{\pgfqpoint{2.520977in}{1.694447in}}{\pgfqpoint{2.512740in}{1.694447in}}%
\pgfpathcurveto{\pgfqpoint{2.504504in}{1.694447in}}{\pgfqpoint{2.496604in}{1.691175in}}{\pgfqpoint{2.490780in}{1.685351in}}%
\pgfpathcurveto{\pgfqpoint{2.484956in}{1.679527in}}{\pgfqpoint{2.481684in}{1.671627in}}{\pgfqpoint{2.481684in}{1.663391in}}%
\pgfpathcurveto{\pgfqpoint{2.481684in}{1.655155in}}{\pgfqpoint{2.484956in}{1.647255in}}{\pgfqpoint{2.490780in}{1.641431in}}%
\pgfpathcurveto{\pgfqpoint{2.496604in}{1.635607in}}{\pgfqpoint{2.504504in}{1.632334in}}{\pgfqpoint{2.512740in}{1.632334in}}%
\pgfpathclose%
\pgfusepath{stroke,fill}%
\end{pgfscope}%
\begin{pgfscope}%
\pgfpathrectangle{\pgfqpoint{0.100000in}{0.220728in}}{\pgfqpoint{3.696000in}{3.696000in}}%
\pgfusepath{clip}%
\pgfsetbuttcap%
\pgfsetroundjoin%
\definecolor{currentfill}{rgb}{0.121569,0.466667,0.705882}%
\pgfsetfillcolor{currentfill}%
\pgfsetfillopacity{0.952291}%
\pgfsetlinewidth{1.003750pt}%
\definecolor{currentstroke}{rgb}{0.121569,0.466667,0.705882}%
\pgfsetstrokecolor{currentstroke}%
\pgfsetstrokeopacity{0.952291}%
\pgfsetdash{}{0pt}%
\pgfpathmoveto{\pgfqpoint{2.124054in}{1.543860in}}%
\pgfpathcurveto{\pgfqpoint{2.132290in}{1.543860in}}{\pgfqpoint{2.140190in}{1.547132in}}{\pgfqpoint{2.146014in}{1.552956in}}%
\pgfpathcurveto{\pgfqpoint{2.151838in}{1.558780in}}{\pgfqpoint{2.155110in}{1.566680in}}{\pgfqpoint{2.155110in}{1.574916in}}%
\pgfpathcurveto{\pgfqpoint{2.155110in}{1.583153in}}{\pgfqpoint{2.151838in}{1.591053in}}{\pgfqpoint{2.146014in}{1.596877in}}%
\pgfpathcurveto{\pgfqpoint{2.140190in}{1.602700in}}{\pgfqpoint{2.132290in}{1.605973in}}{\pgfqpoint{2.124054in}{1.605973in}}%
\pgfpathcurveto{\pgfqpoint{2.115818in}{1.605973in}}{\pgfqpoint{2.107917in}{1.602700in}}{\pgfqpoint{2.102094in}{1.596877in}}%
\pgfpathcurveto{\pgfqpoint{2.096270in}{1.591053in}}{\pgfqpoint{2.092997in}{1.583153in}}{\pgfqpoint{2.092997in}{1.574916in}}%
\pgfpathcurveto{\pgfqpoint{2.092997in}{1.566680in}}{\pgfqpoint{2.096270in}{1.558780in}}{\pgfqpoint{2.102094in}{1.552956in}}%
\pgfpathcurveto{\pgfqpoint{2.107917in}{1.547132in}}{\pgfqpoint{2.115818in}{1.543860in}}{\pgfqpoint{2.124054in}{1.543860in}}%
\pgfpathclose%
\pgfusepath{stroke,fill}%
\end{pgfscope}%
\begin{pgfscope}%
\pgfpathrectangle{\pgfqpoint{0.100000in}{0.220728in}}{\pgfqpoint{3.696000in}{3.696000in}}%
\pgfusepath{clip}%
\pgfsetbuttcap%
\pgfsetroundjoin%
\definecolor{currentfill}{rgb}{0.121569,0.466667,0.705882}%
\pgfsetfillcolor{currentfill}%
\pgfsetfillopacity{0.954856}%
\pgfsetlinewidth{1.003750pt}%
\definecolor{currentstroke}{rgb}{0.121569,0.466667,0.705882}%
\pgfsetstrokecolor{currentstroke}%
\pgfsetstrokeopacity{0.954856}%
\pgfsetdash{}{0pt}%
\pgfpathmoveto{\pgfqpoint{2.137661in}{1.540123in}}%
\pgfpathcurveto{\pgfqpoint{2.145897in}{1.540123in}}{\pgfqpoint{2.153797in}{1.543396in}}{\pgfqpoint{2.159621in}{1.549220in}}%
\pgfpathcurveto{\pgfqpoint{2.165445in}{1.555044in}}{\pgfqpoint{2.168717in}{1.562944in}}{\pgfqpoint{2.168717in}{1.571180in}}%
\pgfpathcurveto{\pgfqpoint{2.168717in}{1.579416in}}{\pgfqpoint{2.165445in}{1.587316in}}{\pgfqpoint{2.159621in}{1.593140in}}%
\pgfpathcurveto{\pgfqpoint{2.153797in}{1.598964in}}{\pgfqpoint{2.145897in}{1.602236in}}{\pgfqpoint{2.137661in}{1.602236in}}%
\pgfpathcurveto{\pgfqpoint{2.129424in}{1.602236in}}{\pgfqpoint{2.121524in}{1.598964in}}{\pgfqpoint{2.115700in}{1.593140in}}%
\pgfpathcurveto{\pgfqpoint{2.109877in}{1.587316in}}{\pgfqpoint{2.106604in}{1.579416in}}{\pgfqpoint{2.106604in}{1.571180in}}%
\pgfpathcurveto{\pgfqpoint{2.106604in}{1.562944in}}{\pgfqpoint{2.109877in}{1.555044in}}{\pgfqpoint{2.115700in}{1.549220in}}%
\pgfpathcurveto{\pgfqpoint{2.121524in}{1.543396in}}{\pgfqpoint{2.129424in}{1.540123in}}{\pgfqpoint{2.137661in}{1.540123in}}%
\pgfpathclose%
\pgfusepath{stroke,fill}%
\end{pgfscope}%
\begin{pgfscope}%
\pgfpathrectangle{\pgfqpoint{0.100000in}{0.220728in}}{\pgfqpoint{3.696000in}{3.696000in}}%
\pgfusepath{clip}%
\pgfsetbuttcap%
\pgfsetroundjoin%
\definecolor{currentfill}{rgb}{0.121569,0.466667,0.705882}%
\pgfsetfillcolor{currentfill}%
\pgfsetfillopacity{0.955511}%
\pgfsetlinewidth{1.003750pt}%
\definecolor{currentstroke}{rgb}{0.121569,0.466667,0.705882}%
\pgfsetstrokecolor{currentstroke}%
\pgfsetstrokeopacity{0.955511}%
\pgfsetdash{}{0pt}%
\pgfpathmoveto{\pgfqpoint{2.503258in}{1.614091in}}%
\pgfpathcurveto{\pgfqpoint{2.511494in}{1.614091in}}{\pgfqpoint{2.519394in}{1.617363in}}{\pgfqpoint{2.525218in}{1.623187in}}%
\pgfpathcurveto{\pgfqpoint{2.531042in}{1.629011in}}{\pgfqpoint{2.534314in}{1.636911in}}{\pgfqpoint{2.534314in}{1.645147in}}%
\pgfpathcurveto{\pgfqpoint{2.534314in}{1.653383in}}{\pgfqpoint{2.531042in}{1.661283in}}{\pgfqpoint{2.525218in}{1.667107in}}%
\pgfpathcurveto{\pgfqpoint{2.519394in}{1.672931in}}{\pgfqpoint{2.511494in}{1.676204in}}{\pgfqpoint{2.503258in}{1.676204in}}%
\pgfpathcurveto{\pgfqpoint{2.495021in}{1.676204in}}{\pgfqpoint{2.487121in}{1.672931in}}{\pgfqpoint{2.481297in}{1.667107in}}%
\pgfpathcurveto{\pgfqpoint{2.475473in}{1.661283in}}{\pgfqpoint{2.472201in}{1.653383in}}{\pgfqpoint{2.472201in}{1.645147in}}%
\pgfpathcurveto{\pgfqpoint{2.472201in}{1.636911in}}{\pgfqpoint{2.475473in}{1.629011in}}{\pgfqpoint{2.481297in}{1.623187in}}%
\pgfpathcurveto{\pgfqpoint{2.487121in}{1.617363in}}{\pgfqpoint{2.495021in}{1.614091in}}{\pgfqpoint{2.503258in}{1.614091in}}%
\pgfpathclose%
\pgfusepath{stroke,fill}%
\end{pgfscope}%
\begin{pgfscope}%
\pgfpathrectangle{\pgfqpoint{0.100000in}{0.220728in}}{\pgfqpoint{3.696000in}{3.696000in}}%
\pgfusepath{clip}%
\pgfsetbuttcap%
\pgfsetroundjoin%
\definecolor{currentfill}{rgb}{0.121569,0.466667,0.705882}%
\pgfsetfillcolor{currentfill}%
\pgfsetfillopacity{0.958823}%
\pgfsetlinewidth{1.003750pt}%
\definecolor{currentstroke}{rgb}{0.121569,0.466667,0.705882}%
\pgfsetstrokecolor{currentstroke}%
\pgfsetstrokeopacity{0.958823}%
\pgfsetdash{}{0pt}%
\pgfpathmoveto{\pgfqpoint{2.160544in}{1.524389in}}%
\pgfpathcurveto{\pgfqpoint{2.168780in}{1.524389in}}{\pgfqpoint{2.176680in}{1.527661in}}{\pgfqpoint{2.182504in}{1.533485in}}%
\pgfpathcurveto{\pgfqpoint{2.188328in}{1.539309in}}{\pgfqpoint{2.191601in}{1.547209in}}{\pgfqpoint{2.191601in}{1.555446in}}%
\pgfpathcurveto{\pgfqpoint{2.191601in}{1.563682in}}{\pgfqpoint{2.188328in}{1.571582in}}{\pgfqpoint{2.182504in}{1.577406in}}%
\pgfpathcurveto{\pgfqpoint{2.176680in}{1.583230in}}{\pgfqpoint{2.168780in}{1.586502in}}{\pgfqpoint{2.160544in}{1.586502in}}%
\pgfpathcurveto{\pgfqpoint{2.152308in}{1.586502in}}{\pgfqpoint{2.144408in}{1.583230in}}{\pgfqpoint{2.138584in}{1.577406in}}%
\pgfpathcurveto{\pgfqpoint{2.132760in}{1.571582in}}{\pgfqpoint{2.129488in}{1.563682in}}{\pgfqpoint{2.129488in}{1.555446in}}%
\pgfpathcurveto{\pgfqpoint{2.129488in}{1.547209in}}{\pgfqpoint{2.132760in}{1.539309in}}{\pgfqpoint{2.138584in}{1.533485in}}%
\pgfpathcurveto{\pgfqpoint{2.144408in}{1.527661in}}{\pgfqpoint{2.152308in}{1.524389in}}{\pgfqpoint{2.160544in}{1.524389in}}%
\pgfpathclose%
\pgfusepath{stroke,fill}%
\end{pgfscope}%
\begin{pgfscope}%
\pgfpathrectangle{\pgfqpoint{0.100000in}{0.220728in}}{\pgfqpoint{3.696000in}{3.696000in}}%
\pgfusepath{clip}%
\pgfsetbuttcap%
\pgfsetroundjoin%
\definecolor{currentfill}{rgb}{0.121569,0.466667,0.705882}%
\pgfsetfillcolor{currentfill}%
\pgfsetfillopacity{0.959279}%
\pgfsetlinewidth{1.003750pt}%
\definecolor{currentstroke}{rgb}{0.121569,0.466667,0.705882}%
\pgfsetstrokecolor{currentstroke}%
\pgfsetstrokeopacity{0.959279}%
\pgfsetdash{}{0pt}%
\pgfpathmoveto{\pgfqpoint{2.488956in}{1.592917in}}%
\pgfpathcurveto{\pgfqpoint{2.497193in}{1.592917in}}{\pgfqpoint{2.505093in}{1.596189in}}{\pgfqpoint{2.510917in}{1.602013in}}%
\pgfpathcurveto{\pgfqpoint{2.516741in}{1.607837in}}{\pgfqpoint{2.520013in}{1.615737in}}{\pgfqpoint{2.520013in}{1.623973in}}%
\pgfpathcurveto{\pgfqpoint{2.520013in}{1.632210in}}{\pgfqpoint{2.516741in}{1.640110in}}{\pgfqpoint{2.510917in}{1.645934in}}%
\pgfpathcurveto{\pgfqpoint{2.505093in}{1.651758in}}{\pgfqpoint{2.497193in}{1.655030in}}{\pgfqpoint{2.488956in}{1.655030in}}%
\pgfpathcurveto{\pgfqpoint{2.480720in}{1.655030in}}{\pgfqpoint{2.472820in}{1.651758in}}{\pgfqpoint{2.466996in}{1.645934in}}%
\pgfpathcurveto{\pgfqpoint{2.461172in}{1.640110in}}{\pgfqpoint{2.457900in}{1.632210in}}{\pgfqpoint{2.457900in}{1.623973in}}%
\pgfpathcurveto{\pgfqpoint{2.457900in}{1.615737in}}{\pgfqpoint{2.461172in}{1.607837in}}{\pgfqpoint{2.466996in}{1.602013in}}%
\pgfpathcurveto{\pgfqpoint{2.472820in}{1.596189in}}{\pgfqpoint{2.480720in}{1.592917in}}{\pgfqpoint{2.488956in}{1.592917in}}%
\pgfpathclose%
\pgfusepath{stroke,fill}%
\end{pgfscope}%
\begin{pgfscope}%
\pgfpathrectangle{\pgfqpoint{0.100000in}{0.220728in}}{\pgfqpoint{3.696000in}{3.696000in}}%
\pgfusepath{clip}%
\pgfsetbuttcap%
\pgfsetroundjoin%
\definecolor{currentfill}{rgb}{0.121569,0.466667,0.705882}%
\pgfsetfillcolor{currentfill}%
\pgfsetfillopacity{0.963333}%
\pgfsetlinewidth{1.003750pt}%
\definecolor{currentstroke}{rgb}{0.121569,0.466667,0.705882}%
\pgfsetstrokecolor{currentstroke}%
\pgfsetstrokeopacity{0.963333}%
\pgfsetdash{}{0pt}%
\pgfpathmoveto{\pgfqpoint{2.181700in}{1.516259in}}%
\pgfpathcurveto{\pgfqpoint{2.189937in}{1.516259in}}{\pgfqpoint{2.197837in}{1.519531in}}{\pgfqpoint{2.203661in}{1.525355in}}%
\pgfpathcurveto{\pgfqpoint{2.209485in}{1.531179in}}{\pgfqpoint{2.212757in}{1.539079in}}{\pgfqpoint{2.212757in}{1.547315in}}%
\pgfpathcurveto{\pgfqpoint{2.212757in}{1.555552in}}{\pgfqpoint{2.209485in}{1.563452in}}{\pgfqpoint{2.203661in}{1.569276in}}%
\pgfpathcurveto{\pgfqpoint{2.197837in}{1.575100in}}{\pgfqpoint{2.189937in}{1.578372in}}{\pgfqpoint{2.181700in}{1.578372in}}%
\pgfpathcurveto{\pgfqpoint{2.173464in}{1.578372in}}{\pgfqpoint{2.165564in}{1.575100in}}{\pgfqpoint{2.159740in}{1.569276in}}%
\pgfpathcurveto{\pgfqpoint{2.153916in}{1.563452in}}{\pgfqpoint{2.150644in}{1.555552in}}{\pgfqpoint{2.150644in}{1.547315in}}%
\pgfpathcurveto{\pgfqpoint{2.150644in}{1.539079in}}{\pgfqpoint{2.153916in}{1.531179in}}{\pgfqpoint{2.159740in}{1.525355in}}%
\pgfpathcurveto{\pgfqpoint{2.165564in}{1.519531in}}{\pgfqpoint{2.173464in}{1.516259in}}{\pgfqpoint{2.181700in}{1.516259in}}%
\pgfpathclose%
\pgfusepath{stroke,fill}%
\end{pgfscope}%
\begin{pgfscope}%
\pgfpathrectangle{\pgfqpoint{0.100000in}{0.220728in}}{\pgfqpoint{3.696000in}{3.696000in}}%
\pgfusepath{clip}%
\pgfsetbuttcap%
\pgfsetroundjoin%
\definecolor{currentfill}{rgb}{0.121569,0.466667,0.705882}%
\pgfsetfillcolor{currentfill}%
\pgfsetfillopacity{0.964419}%
\pgfsetlinewidth{1.003750pt}%
\definecolor{currentstroke}{rgb}{0.121569,0.466667,0.705882}%
\pgfsetstrokecolor{currentstroke}%
\pgfsetstrokeopacity{0.964419}%
\pgfsetdash{}{0pt}%
\pgfpathmoveto{\pgfqpoint{2.475215in}{1.568504in}}%
\pgfpathcurveto{\pgfqpoint{2.483451in}{1.568504in}}{\pgfqpoint{2.491351in}{1.571776in}}{\pgfqpoint{2.497175in}{1.577600in}}%
\pgfpathcurveto{\pgfqpoint{2.502999in}{1.583424in}}{\pgfqpoint{2.506271in}{1.591324in}}{\pgfqpoint{2.506271in}{1.599560in}}%
\pgfpathcurveto{\pgfqpoint{2.506271in}{1.607797in}}{\pgfqpoint{2.502999in}{1.615697in}}{\pgfqpoint{2.497175in}{1.621521in}}%
\pgfpathcurveto{\pgfqpoint{2.491351in}{1.627345in}}{\pgfqpoint{2.483451in}{1.630617in}}{\pgfqpoint{2.475215in}{1.630617in}}%
\pgfpathcurveto{\pgfqpoint{2.466978in}{1.630617in}}{\pgfqpoint{2.459078in}{1.627345in}}{\pgfqpoint{2.453254in}{1.621521in}}%
\pgfpathcurveto{\pgfqpoint{2.447430in}{1.615697in}}{\pgfqpoint{2.444158in}{1.607797in}}{\pgfqpoint{2.444158in}{1.599560in}}%
\pgfpathcurveto{\pgfqpoint{2.444158in}{1.591324in}}{\pgfqpoint{2.447430in}{1.583424in}}{\pgfqpoint{2.453254in}{1.577600in}}%
\pgfpathcurveto{\pgfqpoint{2.459078in}{1.571776in}}{\pgfqpoint{2.466978in}{1.568504in}}{\pgfqpoint{2.475215in}{1.568504in}}%
\pgfpathclose%
\pgfusepath{stroke,fill}%
\end{pgfscope}%
\begin{pgfscope}%
\pgfpathrectangle{\pgfqpoint{0.100000in}{0.220728in}}{\pgfqpoint{3.696000in}{3.696000in}}%
\pgfusepath{clip}%
\pgfsetbuttcap%
\pgfsetroundjoin%
\definecolor{currentfill}{rgb}{0.121569,0.466667,0.705882}%
\pgfsetfillcolor{currentfill}%
\pgfsetfillopacity{0.966835}%
\pgfsetlinewidth{1.003750pt}%
\definecolor{currentstroke}{rgb}{0.121569,0.466667,0.705882}%
\pgfsetstrokecolor{currentstroke}%
\pgfsetstrokeopacity{0.966835}%
\pgfsetdash{}{0pt}%
\pgfpathmoveto{\pgfqpoint{2.466436in}{1.554680in}}%
\pgfpathcurveto{\pgfqpoint{2.474672in}{1.554680in}}{\pgfqpoint{2.482572in}{1.557952in}}{\pgfqpoint{2.488396in}{1.563776in}}%
\pgfpathcurveto{\pgfqpoint{2.494220in}{1.569600in}}{\pgfqpoint{2.497492in}{1.577500in}}{\pgfqpoint{2.497492in}{1.585737in}}%
\pgfpathcurveto{\pgfqpoint{2.497492in}{1.593973in}}{\pgfqpoint{2.494220in}{1.601873in}}{\pgfqpoint{2.488396in}{1.607697in}}%
\pgfpathcurveto{\pgfqpoint{2.482572in}{1.613521in}}{\pgfqpoint{2.474672in}{1.616793in}}{\pgfqpoint{2.466436in}{1.616793in}}%
\pgfpathcurveto{\pgfqpoint{2.458199in}{1.616793in}}{\pgfqpoint{2.450299in}{1.613521in}}{\pgfqpoint{2.444475in}{1.607697in}}%
\pgfpathcurveto{\pgfqpoint{2.438651in}{1.601873in}}{\pgfqpoint{2.435379in}{1.593973in}}{\pgfqpoint{2.435379in}{1.585737in}}%
\pgfpathcurveto{\pgfqpoint{2.435379in}{1.577500in}}{\pgfqpoint{2.438651in}{1.569600in}}{\pgfqpoint{2.444475in}{1.563776in}}%
\pgfpathcurveto{\pgfqpoint{2.450299in}{1.557952in}}{\pgfqpoint{2.458199in}{1.554680in}}{\pgfqpoint{2.466436in}{1.554680in}}%
\pgfpathclose%
\pgfusepath{stroke,fill}%
\end{pgfscope}%
\begin{pgfscope}%
\pgfpathrectangle{\pgfqpoint{0.100000in}{0.220728in}}{\pgfqpoint{3.696000in}{3.696000in}}%
\pgfusepath{clip}%
\pgfsetbuttcap%
\pgfsetroundjoin%
\definecolor{currentfill}{rgb}{0.121569,0.466667,0.705882}%
\pgfsetfillcolor{currentfill}%
\pgfsetfillopacity{0.967044}%
\pgfsetlinewidth{1.003750pt}%
\definecolor{currentstroke}{rgb}{0.121569,0.466667,0.705882}%
\pgfsetstrokecolor{currentstroke}%
\pgfsetstrokeopacity{0.967044}%
\pgfsetdash{}{0pt}%
\pgfpathmoveto{\pgfqpoint{2.197571in}{1.506587in}}%
\pgfpathcurveto{\pgfqpoint{2.205807in}{1.506587in}}{\pgfqpoint{2.213707in}{1.509859in}}{\pgfqpoint{2.219531in}{1.515683in}}%
\pgfpathcurveto{\pgfqpoint{2.225355in}{1.521507in}}{\pgfqpoint{2.228627in}{1.529407in}}{\pgfqpoint{2.228627in}{1.537644in}}%
\pgfpathcurveto{\pgfqpoint{2.228627in}{1.545880in}}{\pgfqpoint{2.225355in}{1.553780in}}{\pgfqpoint{2.219531in}{1.559604in}}%
\pgfpathcurveto{\pgfqpoint{2.213707in}{1.565428in}}{\pgfqpoint{2.205807in}{1.568700in}}{\pgfqpoint{2.197571in}{1.568700in}}%
\pgfpathcurveto{\pgfqpoint{2.189335in}{1.568700in}}{\pgfqpoint{2.181434in}{1.565428in}}{\pgfqpoint{2.175611in}{1.559604in}}%
\pgfpathcurveto{\pgfqpoint{2.169787in}{1.553780in}}{\pgfqpoint{2.166514in}{1.545880in}}{\pgfqpoint{2.166514in}{1.537644in}}%
\pgfpathcurveto{\pgfqpoint{2.166514in}{1.529407in}}{\pgfqpoint{2.169787in}{1.521507in}}{\pgfqpoint{2.175611in}{1.515683in}}%
\pgfpathcurveto{\pgfqpoint{2.181434in}{1.509859in}}{\pgfqpoint{2.189335in}{1.506587in}}{\pgfqpoint{2.197571in}{1.506587in}}%
\pgfpathclose%
\pgfusepath{stroke,fill}%
\end{pgfscope}%
\begin{pgfscope}%
\pgfpathrectangle{\pgfqpoint{0.100000in}{0.220728in}}{\pgfqpoint{3.696000in}{3.696000in}}%
\pgfusepath{clip}%
\pgfsetbuttcap%
\pgfsetroundjoin%
\definecolor{currentfill}{rgb}{0.121569,0.466667,0.705882}%
\pgfsetfillcolor{currentfill}%
\pgfsetfillopacity{0.968249}%
\pgfsetlinewidth{1.003750pt}%
\definecolor{currentstroke}{rgb}{0.121569,0.466667,0.705882}%
\pgfsetstrokecolor{currentstroke}%
\pgfsetstrokeopacity{0.968249}%
\pgfsetdash{}{0pt}%
\pgfpathmoveto{\pgfqpoint{2.461430in}{1.547726in}}%
\pgfpathcurveto{\pgfqpoint{2.469666in}{1.547726in}}{\pgfqpoint{2.477566in}{1.550998in}}{\pgfqpoint{2.483390in}{1.556822in}}%
\pgfpathcurveto{\pgfqpoint{2.489214in}{1.562646in}}{\pgfqpoint{2.492486in}{1.570546in}}{\pgfqpoint{2.492486in}{1.578783in}}%
\pgfpathcurveto{\pgfqpoint{2.492486in}{1.587019in}}{\pgfqpoint{2.489214in}{1.594919in}}{\pgfqpoint{2.483390in}{1.600743in}}%
\pgfpathcurveto{\pgfqpoint{2.477566in}{1.606567in}}{\pgfqpoint{2.469666in}{1.609839in}}{\pgfqpoint{2.461430in}{1.609839in}}%
\pgfpathcurveto{\pgfqpoint{2.453194in}{1.609839in}}{\pgfqpoint{2.445294in}{1.606567in}}{\pgfqpoint{2.439470in}{1.600743in}}%
\pgfpathcurveto{\pgfqpoint{2.433646in}{1.594919in}}{\pgfqpoint{2.430373in}{1.587019in}}{\pgfqpoint{2.430373in}{1.578783in}}%
\pgfpathcurveto{\pgfqpoint{2.430373in}{1.570546in}}{\pgfqpoint{2.433646in}{1.562646in}}{\pgfqpoint{2.439470in}{1.556822in}}%
\pgfpathcurveto{\pgfqpoint{2.445294in}{1.550998in}}{\pgfqpoint{2.453194in}{1.547726in}}{\pgfqpoint{2.461430in}{1.547726in}}%
\pgfpathclose%
\pgfusepath{stroke,fill}%
\end{pgfscope}%
\begin{pgfscope}%
\pgfpathrectangle{\pgfqpoint{0.100000in}{0.220728in}}{\pgfqpoint{3.696000in}{3.696000in}}%
\pgfusepath{clip}%
\pgfsetbuttcap%
\pgfsetroundjoin%
\definecolor{currentfill}{rgb}{0.121569,0.466667,0.705882}%
\pgfsetfillcolor{currentfill}%
\pgfsetfillopacity{0.969044}%
\pgfsetlinewidth{1.003750pt}%
\definecolor{currentstroke}{rgb}{0.121569,0.466667,0.705882}%
\pgfsetstrokecolor{currentstroke}%
\pgfsetstrokeopacity{0.969044}%
\pgfsetdash{}{0pt}%
\pgfpathmoveto{\pgfqpoint{2.459049in}{1.543493in}}%
\pgfpathcurveto{\pgfqpoint{2.467286in}{1.543493in}}{\pgfqpoint{2.475186in}{1.546765in}}{\pgfqpoint{2.481009in}{1.552589in}}%
\pgfpathcurveto{\pgfqpoint{2.486833in}{1.558413in}}{\pgfqpoint{2.490106in}{1.566313in}}{\pgfqpoint{2.490106in}{1.574550in}}%
\pgfpathcurveto{\pgfqpoint{2.490106in}{1.582786in}}{\pgfqpoint{2.486833in}{1.590686in}}{\pgfqpoint{2.481009in}{1.596510in}}%
\pgfpathcurveto{\pgfqpoint{2.475186in}{1.602334in}}{\pgfqpoint{2.467286in}{1.605606in}}{\pgfqpoint{2.459049in}{1.605606in}}%
\pgfpathcurveto{\pgfqpoint{2.450813in}{1.605606in}}{\pgfqpoint{2.442913in}{1.602334in}}{\pgfqpoint{2.437089in}{1.596510in}}%
\pgfpathcurveto{\pgfqpoint{2.431265in}{1.590686in}}{\pgfqpoint{2.427993in}{1.582786in}}{\pgfqpoint{2.427993in}{1.574550in}}%
\pgfpathcurveto{\pgfqpoint{2.427993in}{1.566313in}}{\pgfqpoint{2.431265in}{1.558413in}}{\pgfqpoint{2.437089in}{1.552589in}}%
\pgfpathcurveto{\pgfqpoint{2.442913in}{1.546765in}}{\pgfqpoint{2.450813in}{1.543493in}}{\pgfqpoint{2.459049in}{1.543493in}}%
\pgfpathclose%
\pgfusepath{stroke,fill}%
\end{pgfscope}%
\begin{pgfscope}%
\pgfpathrectangle{\pgfqpoint{0.100000in}{0.220728in}}{\pgfqpoint{3.696000in}{3.696000in}}%
\pgfusepath{clip}%
\pgfsetbuttcap%
\pgfsetroundjoin%
\definecolor{currentfill}{rgb}{0.121569,0.466667,0.705882}%
\pgfsetfillcolor{currentfill}%
\pgfsetfillopacity{0.969496}%
\pgfsetlinewidth{1.003750pt}%
\definecolor{currentstroke}{rgb}{0.121569,0.466667,0.705882}%
\pgfsetstrokecolor{currentstroke}%
\pgfsetstrokeopacity{0.969496}%
\pgfsetdash{}{0pt}%
\pgfpathmoveto{\pgfqpoint{2.457512in}{1.541543in}}%
\pgfpathcurveto{\pgfqpoint{2.465748in}{1.541543in}}{\pgfqpoint{2.473648in}{1.544815in}}{\pgfqpoint{2.479472in}{1.550639in}}%
\pgfpathcurveto{\pgfqpoint{2.485296in}{1.556463in}}{\pgfqpoint{2.488569in}{1.564363in}}{\pgfqpoint{2.488569in}{1.572599in}}%
\pgfpathcurveto{\pgfqpoint{2.488569in}{1.580836in}}{\pgfqpoint{2.485296in}{1.588736in}}{\pgfqpoint{2.479472in}{1.594560in}}%
\pgfpathcurveto{\pgfqpoint{2.473648in}{1.600383in}}{\pgfqpoint{2.465748in}{1.603656in}}{\pgfqpoint{2.457512in}{1.603656in}}%
\pgfpathcurveto{\pgfqpoint{2.449276in}{1.603656in}}{\pgfqpoint{2.441376in}{1.600383in}}{\pgfqpoint{2.435552in}{1.594560in}}%
\pgfpathcurveto{\pgfqpoint{2.429728in}{1.588736in}}{\pgfqpoint{2.426456in}{1.580836in}}{\pgfqpoint{2.426456in}{1.572599in}}%
\pgfpathcurveto{\pgfqpoint{2.426456in}{1.564363in}}{\pgfqpoint{2.429728in}{1.556463in}}{\pgfqpoint{2.435552in}{1.550639in}}%
\pgfpathcurveto{\pgfqpoint{2.441376in}{1.544815in}}{\pgfqpoint{2.449276in}{1.541543in}}{\pgfqpoint{2.457512in}{1.541543in}}%
\pgfpathclose%
\pgfusepath{stroke,fill}%
\end{pgfscope}%
\begin{pgfscope}%
\pgfpathrectangle{\pgfqpoint{0.100000in}{0.220728in}}{\pgfqpoint{3.696000in}{3.696000in}}%
\pgfusepath{clip}%
\pgfsetbuttcap%
\pgfsetroundjoin%
\definecolor{currentfill}{rgb}{0.121569,0.466667,0.705882}%
\pgfsetfillcolor{currentfill}%
\pgfsetfillopacity{0.969744}%
\pgfsetlinewidth{1.003750pt}%
\definecolor{currentstroke}{rgb}{0.121569,0.466667,0.705882}%
\pgfsetstrokecolor{currentstroke}%
\pgfsetstrokeopacity{0.969744}%
\pgfsetdash{}{0pt}%
\pgfpathmoveto{\pgfqpoint{2.456786in}{1.540297in}}%
\pgfpathcurveto{\pgfqpoint{2.465022in}{1.540297in}}{\pgfqpoint{2.472922in}{1.543569in}}{\pgfqpoint{2.478746in}{1.549393in}}%
\pgfpathcurveto{\pgfqpoint{2.484570in}{1.555217in}}{\pgfqpoint{2.487843in}{1.563117in}}{\pgfqpoint{2.487843in}{1.571354in}}%
\pgfpathcurveto{\pgfqpoint{2.487843in}{1.579590in}}{\pgfqpoint{2.484570in}{1.587490in}}{\pgfqpoint{2.478746in}{1.593314in}}%
\pgfpathcurveto{\pgfqpoint{2.472922in}{1.599138in}}{\pgfqpoint{2.465022in}{1.602410in}}{\pgfqpoint{2.456786in}{1.602410in}}%
\pgfpathcurveto{\pgfqpoint{2.448550in}{1.602410in}}{\pgfqpoint{2.440650in}{1.599138in}}{\pgfqpoint{2.434826in}{1.593314in}}%
\pgfpathcurveto{\pgfqpoint{2.429002in}{1.587490in}}{\pgfqpoint{2.425730in}{1.579590in}}{\pgfqpoint{2.425730in}{1.571354in}}%
\pgfpathcurveto{\pgfqpoint{2.425730in}{1.563117in}}{\pgfqpoint{2.429002in}{1.555217in}}{\pgfqpoint{2.434826in}{1.549393in}}%
\pgfpathcurveto{\pgfqpoint{2.440650in}{1.543569in}}{\pgfqpoint{2.448550in}{1.540297in}}{\pgfqpoint{2.456786in}{1.540297in}}%
\pgfpathclose%
\pgfusepath{stroke,fill}%
\end{pgfscope}%
\begin{pgfscope}%
\pgfpathrectangle{\pgfqpoint{0.100000in}{0.220728in}}{\pgfqpoint{3.696000in}{3.696000in}}%
\pgfusepath{clip}%
\pgfsetbuttcap%
\pgfsetroundjoin%
\definecolor{currentfill}{rgb}{0.121569,0.466667,0.705882}%
\pgfsetfillcolor{currentfill}%
\pgfsetfillopacity{0.970786}%
\pgfsetlinewidth{1.003750pt}%
\definecolor{currentstroke}{rgb}{0.121569,0.466667,0.705882}%
\pgfsetstrokecolor{currentstroke}%
\pgfsetstrokeopacity{0.970786}%
\pgfsetdash{}{0pt}%
\pgfpathmoveto{\pgfqpoint{2.452776in}{1.535103in}}%
\pgfpathcurveto{\pgfqpoint{2.461013in}{1.535103in}}{\pgfqpoint{2.468913in}{1.538375in}}{\pgfqpoint{2.474737in}{1.544199in}}%
\pgfpathcurveto{\pgfqpoint{2.480561in}{1.550023in}}{\pgfqpoint{2.483833in}{1.557923in}}{\pgfqpoint{2.483833in}{1.566159in}}%
\pgfpathcurveto{\pgfqpoint{2.483833in}{1.574396in}}{\pgfqpoint{2.480561in}{1.582296in}}{\pgfqpoint{2.474737in}{1.588120in}}%
\pgfpathcurveto{\pgfqpoint{2.468913in}{1.593944in}}{\pgfqpoint{2.461013in}{1.597216in}}{\pgfqpoint{2.452776in}{1.597216in}}%
\pgfpathcurveto{\pgfqpoint{2.444540in}{1.597216in}}{\pgfqpoint{2.436640in}{1.593944in}}{\pgfqpoint{2.430816in}{1.588120in}}%
\pgfpathcurveto{\pgfqpoint{2.424992in}{1.582296in}}{\pgfqpoint{2.421720in}{1.574396in}}{\pgfqpoint{2.421720in}{1.566159in}}%
\pgfpathcurveto{\pgfqpoint{2.421720in}{1.557923in}}{\pgfqpoint{2.424992in}{1.550023in}}{\pgfqpoint{2.430816in}{1.544199in}}%
\pgfpathcurveto{\pgfqpoint{2.436640in}{1.538375in}}{\pgfqpoint{2.444540in}{1.535103in}}{\pgfqpoint{2.452776in}{1.535103in}}%
\pgfpathclose%
\pgfusepath{stroke,fill}%
\end{pgfscope}%
\begin{pgfscope}%
\pgfpathrectangle{\pgfqpoint{0.100000in}{0.220728in}}{\pgfqpoint{3.696000in}{3.696000in}}%
\pgfusepath{clip}%
\pgfsetbuttcap%
\pgfsetroundjoin%
\definecolor{currentfill}{rgb}{0.121569,0.466667,0.705882}%
\pgfsetfillcolor{currentfill}%
\pgfsetfillopacity{0.972322}%
\pgfsetlinewidth{1.003750pt}%
\definecolor{currentstroke}{rgb}{0.121569,0.466667,0.705882}%
\pgfsetstrokecolor{currentstroke}%
\pgfsetstrokeopacity{0.972322}%
\pgfsetdash{}{0pt}%
\pgfpathmoveto{\pgfqpoint{2.448253in}{1.526927in}}%
\pgfpathcurveto{\pgfqpoint{2.456489in}{1.526927in}}{\pgfqpoint{2.464389in}{1.530199in}}{\pgfqpoint{2.470213in}{1.536023in}}%
\pgfpathcurveto{\pgfqpoint{2.476037in}{1.541847in}}{\pgfqpoint{2.479309in}{1.549747in}}{\pgfqpoint{2.479309in}{1.557983in}}%
\pgfpathcurveto{\pgfqpoint{2.479309in}{1.566219in}}{\pgfqpoint{2.476037in}{1.574119in}}{\pgfqpoint{2.470213in}{1.579943in}}%
\pgfpathcurveto{\pgfqpoint{2.464389in}{1.585767in}}{\pgfqpoint{2.456489in}{1.589040in}}{\pgfqpoint{2.448253in}{1.589040in}}%
\pgfpathcurveto{\pgfqpoint{2.440017in}{1.589040in}}{\pgfqpoint{2.432117in}{1.585767in}}{\pgfqpoint{2.426293in}{1.579943in}}%
\pgfpathcurveto{\pgfqpoint{2.420469in}{1.574119in}}{\pgfqpoint{2.417196in}{1.566219in}}{\pgfqpoint{2.417196in}{1.557983in}}%
\pgfpathcurveto{\pgfqpoint{2.417196in}{1.549747in}}{\pgfqpoint{2.420469in}{1.541847in}}{\pgfqpoint{2.426293in}{1.536023in}}%
\pgfpathcurveto{\pgfqpoint{2.432117in}{1.530199in}}{\pgfqpoint{2.440017in}{1.526927in}}{\pgfqpoint{2.448253in}{1.526927in}}%
\pgfpathclose%
\pgfusepath{stroke,fill}%
\end{pgfscope}%
\begin{pgfscope}%
\pgfpathrectangle{\pgfqpoint{0.100000in}{0.220728in}}{\pgfqpoint{3.696000in}{3.696000in}}%
\pgfusepath{clip}%
\pgfsetbuttcap%
\pgfsetroundjoin%
\definecolor{currentfill}{rgb}{0.121569,0.466667,0.705882}%
\pgfsetfillcolor{currentfill}%
\pgfsetfillopacity{0.973221}%
\pgfsetlinewidth{1.003750pt}%
\definecolor{currentstroke}{rgb}{0.121569,0.466667,0.705882}%
\pgfsetstrokecolor{currentstroke}%
\pgfsetstrokeopacity{0.973221}%
\pgfsetdash{}{0pt}%
\pgfpathmoveto{\pgfqpoint{2.226433in}{1.486341in}}%
\pgfpathcurveto{\pgfqpoint{2.234669in}{1.486341in}}{\pgfqpoint{2.242569in}{1.489613in}}{\pgfqpoint{2.248393in}{1.495437in}}%
\pgfpathcurveto{\pgfqpoint{2.254217in}{1.501261in}}{\pgfqpoint{2.257489in}{1.509161in}}{\pgfqpoint{2.257489in}{1.517397in}}%
\pgfpathcurveto{\pgfqpoint{2.257489in}{1.525634in}}{\pgfqpoint{2.254217in}{1.533534in}}{\pgfqpoint{2.248393in}{1.539358in}}%
\pgfpathcurveto{\pgfqpoint{2.242569in}{1.545182in}}{\pgfqpoint{2.234669in}{1.548454in}}{\pgfqpoint{2.226433in}{1.548454in}}%
\pgfpathcurveto{\pgfqpoint{2.218197in}{1.548454in}}{\pgfqpoint{2.210297in}{1.545182in}}{\pgfqpoint{2.204473in}{1.539358in}}%
\pgfpathcurveto{\pgfqpoint{2.198649in}{1.533534in}}{\pgfqpoint{2.195376in}{1.525634in}}{\pgfqpoint{2.195376in}{1.517397in}}%
\pgfpathcurveto{\pgfqpoint{2.195376in}{1.509161in}}{\pgfqpoint{2.198649in}{1.501261in}}{\pgfqpoint{2.204473in}{1.495437in}}%
\pgfpathcurveto{\pgfqpoint{2.210297in}{1.489613in}}{\pgfqpoint{2.218197in}{1.486341in}}{\pgfqpoint{2.226433in}{1.486341in}}%
\pgfpathclose%
\pgfusepath{stroke,fill}%
\end{pgfscope}%
\begin{pgfscope}%
\pgfpathrectangle{\pgfqpoint{0.100000in}{0.220728in}}{\pgfqpoint{3.696000in}{3.696000in}}%
\pgfusepath{clip}%
\pgfsetbuttcap%
\pgfsetroundjoin%
\definecolor{currentfill}{rgb}{0.121569,0.466667,0.705882}%
\pgfsetfillcolor{currentfill}%
\pgfsetfillopacity{0.974496}%
\pgfsetlinewidth{1.003750pt}%
\definecolor{currentstroke}{rgb}{0.121569,0.466667,0.705882}%
\pgfsetstrokecolor{currentstroke}%
\pgfsetstrokeopacity{0.974496}%
\pgfsetdash{}{0pt}%
\pgfpathmoveto{\pgfqpoint{2.440096in}{1.515632in}}%
\pgfpathcurveto{\pgfqpoint{2.448333in}{1.515632in}}{\pgfqpoint{2.456233in}{1.518904in}}{\pgfqpoint{2.462057in}{1.524728in}}%
\pgfpathcurveto{\pgfqpoint{2.467881in}{1.530552in}}{\pgfqpoint{2.471153in}{1.538452in}}{\pgfqpoint{2.471153in}{1.546688in}}%
\pgfpathcurveto{\pgfqpoint{2.471153in}{1.554924in}}{\pgfqpoint{2.467881in}{1.562825in}}{\pgfqpoint{2.462057in}{1.568648in}}%
\pgfpathcurveto{\pgfqpoint{2.456233in}{1.574472in}}{\pgfqpoint{2.448333in}{1.577745in}}{\pgfqpoint{2.440096in}{1.577745in}}%
\pgfpathcurveto{\pgfqpoint{2.431860in}{1.577745in}}{\pgfqpoint{2.423960in}{1.574472in}}{\pgfqpoint{2.418136in}{1.568648in}}%
\pgfpathcurveto{\pgfqpoint{2.412312in}{1.562825in}}{\pgfqpoint{2.409040in}{1.554924in}}{\pgfqpoint{2.409040in}{1.546688in}}%
\pgfpathcurveto{\pgfqpoint{2.409040in}{1.538452in}}{\pgfqpoint{2.412312in}{1.530552in}}{\pgfqpoint{2.418136in}{1.524728in}}%
\pgfpathcurveto{\pgfqpoint{2.423960in}{1.518904in}}{\pgfqpoint{2.431860in}{1.515632in}}{\pgfqpoint{2.440096in}{1.515632in}}%
\pgfpathclose%
\pgfusepath{stroke,fill}%
\end{pgfscope}%
\begin{pgfscope}%
\pgfpathrectangle{\pgfqpoint{0.100000in}{0.220728in}}{\pgfqpoint{3.696000in}{3.696000in}}%
\pgfusepath{clip}%
\pgfsetbuttcap%
\pgfsetroundjoin%
\definecolor{currentfill}{rgb}{0.121569,0.466667,0.705882}%
\pgfsetfillcolor{currentfill}%
\pgfsetfillopacity{0.975721}%
\pgfsetlinewidth{1.003750pt}%
\definecolor{currentstroke}{rgb}{0.121569,0.466667,0.705882}%
\pgfsetstrokecolor{currentstroke}%
\pgfsetstrokeopacity{0.975721}%
\pgfsetdash{}{0pt}%
\pgfpathmoveto{\pgfqpoint{2.436060in}{1.508922in}}%
\pgfpathcurveto{\pgfqpoint{2.444296in}{1.508922in}}{\pgfqpoint{2.452196in}{1.512194in}}{\pgfqpoint{2.458020in}{1.518018in}}%
\pgfpathcurveto{\pgfqpoint{2.463844in}{1.523842in}}{\pgfqpoint{2.467116in}{1.531742in}}{\pgfqpoint{2.467116in}{1.539979in}}%
\pgfpathcurveto{\pgfqpoint{2.467116in}{1.548215in}}{\pgfqpoint{2.463844in}{1.556115in}}{\pgfqpoint{2.458020in}{1.561939in}}%
\pgfpathcurveto{\pgfqpoint{2.452196in}{1.567763in}}{\pgfqpoint{2.444296in}{1.571035in}}{\pgfqpoint{2.436060in}{1.571035in}}%
\pgfpathcurveto{\pgfqpoint{2.427824in}{1.571035in}}{\pgfqpoint{2.419924in}{1.567763in}}{\pgfqpoint{2.414100in}{1.561939in}}%
\pgfpathcurveto{\pgfqpoint{2.408276in}{1.556115in}}{\pgfqpoint{2.405003in}{1.548215in}}{\pgfqpoint{2.405003in}{1.539979in}}%
\pgfpathcurveto{\pgfqpoint{2.405003in}{1.531742in}}{\pgfqpoint{2.408276in}{1.523842in}}{\pgfqpoint{2.414100in}{1.518018in}}%
\pgfpathcurveto{\pgfqpoint{2.419924in}{1.512194in}}{\pgfqpoint{2.427824in}{1.508922in}}{\pgfqpoint{2.436060in}{1.508922in}}%
\pgfpathclose%
\pgfusepath{stroke,fill}%
\end{pgfscope}%
\begin{pgfscope}%
\pgfpathrectangle{\pgfqpoint{0.100000in}{0.220728in}}{\pgfqpoint{3.696000in}{3.696000in}}%
\pgfusepath{clip}%
\pgfsetbuttcap%
\pgfsetroundjoin%
\definecolor{currentfill}{rgb}{0.121569,0.466667,0.705882}%
\pgfsetfillcolor{currentfill}%
\pgfsetfillopacity{0.976409}%
\pgfsetlinewidth{1.003750pt}%
\definecolor{currentstroke}{rgb}{0.121569,0.466667,0.705882}%
\pgfsetstrokecolor{currentstroke}%
\pgfsetstrokeopacity{0.976409}%
\pgfsetdash{}{0pt}%
\pgfpathmoveto{\pgfqpoint{2.433828in}{1.505288in}}%
\pgfpathcurveto{\pgfqpoint{2.442064in}{1.505288in}}{\pgfqpoint{2.449964in}{1.508561in}}{\pgfqpoint{2.455788in}{1.514385in}}%
\pgfpathcurveto{\pgfqpoint{2.461612in}{1.520209in}}{\pgfqpoint{2.464885in}{1.528109in}}{\pgfqpoint{2.464885in}{1.536345in}}%
\pgfpathcurveto{\pgfqpoint{2.464885in}{1.544581in}}{\pgfqpoint{2.461612in}{1.552481in}}{\pgfqpoint{2.455788in}{1.558305in}}%
\pgfpathcurveto{\pgfqpoint{2.449964in}{1.564129in}}{\pgfqpoint{2.442064in}{1.567401in}}{\pgfqpoint{2.433828in}{1.567401in}}%
\pgfpathcurveto{\pgfqpoint{2.425592in}{1.567401in}}{\pgfqpoint{2.417692in}{1.564129in}}{\pgfqpoint{2.411868in}{1.558305in}}%
\pgfpathcurveto{\pgfqpoint{2.406044in}{1.552481in}}{\pgfqpoint{2.402772in}{1.544581in}}{\pgfqpoint{2.402772in}{1.536345in}}%
\pgfpathcurveto{\pgfqpoint{2.402772in}{1.528109in}}{\pgfqpoint{2.406044in}{1.520209in}}{\pgfqpoint{2.411868in}{1.514385in}}%
\pgfpathcurveto{\pgfqpoint{2.417692in}{1.508561in}}{\pgfqpoint{2.425592in}{1.505288in}}{\pgfqpoint{2.433828in}{1.505288in}}%
\pgfpathclose%
\pgfusepath{stroke,fill}%
\end{pgfscope}%
\begin{pgfscope}%
\pgfpathrectangle{\pgfqpoint{0.100000in}{0.220728in}}{\pgfqpoint{3.696000in}{3.696000in}}%
\pgfusepath{clip}%
\pgfsetbuttcap%
\pgfsetroundjoin%
\definecolor{currentfill}{rgb}{0.121569,0.466667,0.705882}%
\pgfsetfillcolor{currentfill}%
\pgfsetfillopacity{0.976774}%
\pgfsetlinewidth{1.003750pt}%
\definecolor{currentstroke}{rgb}{0.121569,0.466667,0.705882}%
\pgfsetstrokecolor{currentstroke}%
\pgfsetstrokeopacity{0.976774}%
\pgfsetdash{}{0pt}%
\pgfpathmoveto{\pgfqpoint{2.432421in}{1.503501in}}%
\pgfpathcurveto{\pgfqpoint{2.440657in}{1.503501in}}{\pgfqpoint{2.448557in}{1.506773in}}{\pgfqpoint{2.454381in}{1.512597in}}%
\pgfpathcurveto{\pgfqpoint{2.460205in}{1.518421in}}{\pgfqpoint{2.463477in}{1.526321in}}{\pgfqpoint{2.463477in}{1.534558in}}%
\pgfpathcurveto{\pgfqpoint{2.463477in}{1.542794in}}{\pgfqpoint{2.460205in}{1.550694in}}{\pgfqpoint{2.454381in}{1.556518in}}%
\pgfpathcurveto{\pgfqpoint{2.448557in}{1.562342in}}{\pgfqpoint{2.440657in}{1.565614in}}{\pgfqpoint{2.432421in}{1.565614in}}%
\pgfpathcurveto{\pgfqpoint{2.424185in}{1.565614in}}{\pgfqpoint{2.416285in}{1.562342in}}{\pgfqpoint{2.410461in}{1.556518in}}%
\pgfpathcurveto{\pgfqpoint{2.404637in}{1.550694in}}{\pgfqpoint{2.401364in}{1.542794in}}{\pgfqpoint{2.401364in}{1.534558in}}%
\pgfpathcurveto{\pgfqpoint{2.401364in}{1.526321in}}{\pgfqpoint{2.404637in}{1.518421in}}{\pgfqpoint{2.410461in}{1.512597in}}%
\pgfpathcurveto{\pgfqpoint{2.416285in}{1.506773in}}{\pgfqpoint{2.424185in}{1.503501in}}{\pgfqpoint{2.432421in}{1.503501in}}%
\pgfpathclose%
\pgfusepath{stroke,fill}%
\end{pgfscope}%
\begin{pgfscope}%
\pgfpathrectangle{\pgfqpoint{0.100000in}{0.220728in}}{\pgfqpoint{3.696000in}{3.696000in}}%
\pgfusepath{clip}%
\pgfsetbuttcap%
\pgfsetroundjoin%
\definecolor{currentfill}{rgb}{0.121569,0.466667,0.705882}%
\pgfsetfillcolor{currentfill}%
\pgfsetfillopacity{0.976996}%
\pgfsetlinewidth{1.003750pt}%
\definecolor{currentstroke}{rgb}{0.121569,0.466667,0.705882}%
\pgfsetstrokecolor{currentstroke}%
\pgfsetstrokeopacity{0.976996}%
\pgfsetdash{}{0pt}%
\pgfpathmoveto{\pgfqpoint{2.431798in}{1.502398in}}%
\pgfpathcurveto{\pgfqpoint{2.440034in}{1.502398in}}{\pgfqpoint{2.447934in}{1.505670in}}{\pgfqpoint{2.453758in}{1.511494in}}%
\pgfpathcurveto{\pgfqpoint{2.459582in}{1.517318in}}{\pgfqpoint{2.462854in}{1.525218in}}{\pgfqpoint{2.462854in}{1.533454in}}%
\pgfpathcurveto{\pgfqpoint{2.462854in}{1.541690in}}{\pgfqpoint{2.459582in}{1.549590in}}{\pgfqpoint{2.453758in}{1.555414in}}%
\pgfpathcurveto{\pgfqpoint{2.447934in}{1.561238in}}{\pgfqpoint{2.440034in}{1.564511in}}{\pgfqpoint{2.431798in}{1.564511in}}%
\pgfpathcurveto{\pgfqpoint{2.423561in}{1.564511in}}{\pgfqpoint{2.415661in}{1.561238in}}{\pgfqpoint{2.409837in}{1.555414in}}%
\pgfpathcurveto{\pgfqpoint{2.404013in}{1.549590in}}{\pgfqpoint{2.400741in}{1.541690in}}{\pgfqpoint{2.400741in}{1.533454in}}%
\pgfpathcurveto{\pgfqpoint{2.400741in}{1.525218in}}{\pgfqpoint{2.404013in}{1.517318in}}{\pgfqpoint{2.409837in}{1.511494in}}%
\pgfpathcurveto{\pgfqpoint{2.415661in}{1.505670in}}{\pgfqpoint{2.423561in}{1.502398in}}{\pgfqpoint{2.431798in}{1.502398in}}%
\pgfpathclose%
\pgfusepath{stroke,fill}%
\end{pgfscope}%
\begin{pgfscope}%
\pgfpathrectangle{\pgfqpoint{0.100000in}{0.220728in}}{\pgfqpoint{3.696000in}{3.696000in}}%
\pgfusepath{clip}%
\pgfsetbuttcap%
\pgfsetroundjoin%
\definecolor{currentfill}{rgb}{0.121569,0.466667,0.705882}%
\pgfsetfillcolor{currentfill}%
\pgfsetfillopacity{0.977106}%
\pgfsetlinewidth{1.003750pt}%
\definecolor{currentstroke}{rgb}{0.121569,0.466667,0.705882}%
\pgfsetstrokecolor{currentstroke}%
\pgfsetstrokeopacity{0.977106}%
\pgfsetdash{}{0pt}%
\pgfpathmoveto{\pgfqpoint{2.431366in}{1.501862in}}%
\pgfpathcurveto{\pgfqpoint{2.439602in}{1.501862in}}{\pgfqpoint{2.447502in}{1.505134in}}{\pgfqpoint{2.453326in}{1.510958in}}%
\pgfpathcurveto{\pgfqpoint{2.459150in}{1.516782in}}{\pgfqpoint{2.462423in}{1.524682in}}{\pgfqpoint{2.462423in}{1.532918in}}%
\pgfpathcurveto{\pgfqpoint{2.462423in}{1.541154in}}{\pgfqpoint{2.459150in}{1.549054in}}{\pgfqpoint{2.453326in}{1.554878in}}%
\pgfpathcurveto{\pgfqpoint{2.447502in}{1.560702in}}{\pgfqpoint{2.439602in}{1.563975in}}{\pgfqpoint{2.431366in}{1.563975in}}%
\pgfpathcurveto{\pgfqpoint{2.423130in}{1.563975in}}{\pgfqpoint{2.415230in}{1.560702in}}{\pgfqpoint{2.409406in}{1.554878in}}%
\pgfpathcurveto{\pgfqpoint{2.403582in}{1.549054in}}{\pgfqpoint{2.400310in}{1.541154in}}{\pgfqpoint{2.400310in}{1.532918in}}%
\pgfpathcurveto{\pgfqpoint{2.400310in}{1.524682in}}{\pgfqpoint{2.403582in}{1.516782in}}{\pgfqpoint{2.409406in}{1.510958in}}%
\pgfpathcurveto{\pgfqpoint{2.415230in}{1.505134in}}{\pgfqpoint{2.423130in}{1.501862in}}{\pgfqpoint{2.431366in}{1.501862in}}%
\pgfpathclose%
\pgfusepath{stroke,fill}%
\end{pgfscope}%
\begin{pgfscope}%
\pgfpathrectangle{\pgfqpoint{0.100000in}{0.220728in}}{\pgfqpoint{3.696000in}{3.696000in}}%
\pgfusepath{clip}%
\pgfsetbuttcap%
\pgfsetroundjoin%
\definecolor{currentfill}{rgb}{0.121569,0.466667,0.705882}%
\pgfsetfillcolor{currentfill}%
\pgfsetfillopacity{0.977178}%
\pgfsetlinewidth{1.003750pt}%
\definecolor{currentstroke}{rgb}{0.121569,0.466667,0.705882}%
\pgfsetstrokecolor{currentstroke}%
\pgfsetstrokeopacity{0.977178}%
\pgfsetdash{}{0pt}%
\pgfpathmoveto{\pgfqpoint{2.431188in}{1.501536in}}%
\pgfpathcurveto{\pgfqpoint{2.439424in}{1.501536in}}{\pgfqpoint{2.447324in}{1.504809in}}{\pgfqpoint{2.453148in}{1.510633in}}%
\pgfpathcurveto{\pgfqpoint{2.458972in}{1.516457in}}{\pgfqpoint{2.462244in}{1.524357in}}{\pgfqpoint{2.462244in}{1.532593in}}%
\pgfpathcurveto{\pgfqpoint{2.462244in}{1.540829in}}{\pgfqpoint{2.458972in}{1.548729in}}{\pgfqpoint{2.453148in}{1.554553in}}%
\pgfpathcurveto{\pgfqpoint{2.447324in}{1.560377in}}{\pgfqpoint{2.439424in}{1.563649in}}{\pgfqpoint{2.431188in}{1.563649in}}%
\pgfpathcurveto{\pgfqpoint{2.422952in}{1.563649in}}{\pgfqpoint{2.415052in}{1.560377in}}{\pgfqpoint{2.409228in}{1.554553in}}%
\pgfpathcurveto{\pgfqpoint{2.403404in}{1.548729in}}{\pgfqpoint{2.400131in}{1.540829in}}{\pgfqpoint{2.400131in}{1.532593in}}%
\pgfpathcurveto{\pgfqpoint{2.400131in}{1.524357in}}{\pgfqpoint{2.403404in}{1.516457in}}{\pgfqpoint{2.409228in}{1.510633in}}%
\pgfpathcurveto{\pgfqpoint{2.415052in}{1.504809in}}{\pgfqpoint{2.422952in}{1.501536in}}{\pgfqpoint{2.431188in}{1.501536in}}%
\pgfpathclose%
\pgfusepath{stroke,fill}%
\end{pgfscope}%
\begin{pgfscope}%
\pgfpathrectangle{\pgfqpoint{0.100000in}{0.220728in}}{\pgfqpoint{3.696000in}{3.696000in}}%
\pgfusepath{clip}%
\pgfsetbuttcap%
\pgfsetroundjoin%
\definecolor{currentfill}{rgb}{0.121569,0.466667,0.705882}%
\pgfsetfillcolor{currentfill}%
\pgfsetfillopacity{0.977927}%
\pgfsetlinewidth{1.003750pt}%
\definecolor{currentstroke}{rgb}{0.121569,0.466667,0.705882}%
\pgfsetstrokecolor{currentstroke}%
\pgfsetstrokeopacity{0.977927}%
\pgfsetdash{}{0pt}%
\pgfpathmoveto{\pgfqpoint{2.428405in}{1.498159in}}%
\pgfpathcurveto{\pgfqpoint{2.436641in}{1.498159in}}{\pgfqpoint{2.444541in}{1.501431in}}{\pgfqpoint{2.450365in}{1.507255in}}%
\pgfpathcurveto{\pgfqpoint{2.456189in}{1.513079in}}{\pgfqpoint{2.459462in}{1.520979in}}{\pgfqpoint{2.459462in}{1.529216in}}%
\pgfpathcurveto{\pgfqpoint{2.459462in}{1.537452in}}{\pgfqpoint{2.456189in}{1.545352in}}{\pgfqpoint{2.450365in}{1.551176in}}%
\pgfpathcurveto{\pgfqpoint{2.444541in}{1.557000in}}{\pgfqpoint{2.436641in}{1.560272in}}{\pgfqpoint{2.428405in}{1.560272in}}%
\pgfpathcurveto{\pgfqpoint{2.420169in}{1.560272in}}{\pgfqpoint{2.412269in}{1.557000in}}{\pgfqpoint{2.406445in}{1.551176in}}%
\pgfpathcurveto{\pgfqpoint{2.400621in}{1.545352in}}{\pgfqpoint{2.397349in}{1.537452in}}{\pgfqpoint{2.397349in}{1.529216in}}%
\pgfpathcurveto{\pgfqpoint{2.397349in}{1.520979in}}{\pgfqpoint{2.400621in}{1.513079in}}{\pgfqpoint{2.406445in}{1.507255in}}%
\pgfpathcurveto{\pgfqpoint{2.412269in}{1.501431in}}{\pgfqpoint{2.420169in}{1.498159in}}{\pgfqpoint{2.428405in}{1.498159in}}%
\pgfpathclose%
\pgfusepath{stroke,fill}%
\end{pgfscope}%
\begin{pgfscope}%
\pgfpathrectangle{\pgfqpoint{0.100000in}{0.220728in}}{\pgfqpoint{3.696000in}{3.696000in}}%
\pgfusepath{clip}%
\pgfsetbuttcap%
\pgfsetroundjoin%
\definecolor{currentfill}{rgb}{0.121569,0.466667,0.705882}%
\pgfsetfillcolor{currentfill}%
\pgfsetfillopacity{0.978629}%
\pgfsetlinewidth{1.003750pt}%
\definecolor{currentstroke}{rgb}{0.121569,0.466667,0.705882}%
\pgfsetstrokecolor{currentstroke}%
\pgfsetstrokeopacity{0.978629}%
\pgfsetdash{}{0pt}%
\pgfpathmoveto{\pgfqpoint{2.249916in}{1.470923in}}%
\pgfpathcurveto{\pgfqpoint{2.258153in}{1.470923in}}{\pgfqpoint{2.266053in}{1.474195in}}{\pgfqpoint{2.271877in}{1.480019in}}%
\pgfpathcurveto{\pgfqpoint{2.277701in}{1.485843in}}{\pgfqpoint{2.280973in}{1.493743in}}{\pgfqpoint{2.280973in}{1.501979in}}%
\pgfpathcurveto{\pgfqpoint{2.280973in}{1.510216in}}{\pgfqpoint{2.277701in}{1.518116in}}{\pgfqpoint{2.271877in}{1.523939in}}%
\pgfpathcurveto{\pgfqpoint{2.266053in}{1.529763in}}{\pgfqpoint{2.258153in}{1.533036in}}{\pgfqpoint{2.249916in}{1.533036in}}%
\pgfpathcurveto{\pgfqpoint{2.241680in}{1.533036in}}{\pgfqpoint{2.233780in}{1.529763in}}{\pgfqpoint{2.227956in}{1.523939in}}%
\pgfpathcurveto{\pgfqpoint{2.222132in}{1.518116in}}{\pgfqpoint{2.218860in}{1.510216in}}{\pgfqpoint{2.218860in}{1.501979in}}%
\pgfpathcurveto{\pgfqpoint{2.218860in}{1.493743in}}{\pgfqpoint{2.222132in}{1.485843in}}{\pgfqpoint{2.227956in}{1.480019in}}%
\pgfpathcurveto{\pgfqpoint{2.233780in}{1.474195in}}{\pgfqpoint{2.241680in}{1.470923in}}{\pgfqpoint{2.249916in}{1.470923in}}%
\pgfpathclose%
\pgfusepath{stroke,fill}%
\end{pgfscope}%
\begin{pgfscope}%
\pgfpathrectangle{\pgfqpoint{0.100000in}{0.220728in}}{\pgfqpoint{3.696000in}{3.696000in}}%
\pgfusepath{clip}%
\pgfsetbuttcap%
\pgfsetroundjoin%
\definecolor{currentfill}{rgb}{0.121569,0.466667,0.705882}%
\pgfsetfillcolor{currentfill}%
\pgfsetfillopacity{0.979238}%
\pgfsetlinewidth{1.003750pt}%
\definecolor{currentstroke}{rgb}{0.121569,0.466667,0.705882}%
\pgfsetstrokecolor{currentstroke}%
\pgfsetstrokeopacity{0.979238}%
\pgfsetdash{}{0pt}%
\pgfpathmoveto{\pgfqpoint{2.424860in}{1.491335in}}%
\pgfpathcurveto{\pgfqpoint{2.433096in}{1.491335in}}{\pgfqpoint{2.440997in}{1.494607in}}{\pgfqpoint{2.446820in}{1.500431in}}%
\pgfpathcurveto{\pgfqpoint{2.452644in}{1.506255in}}{\pgfqpoint{2.455917in}{1.514155in}}{\pgfqpoint{2.455917in}{1.522391in}}%
\pgfpathcurveto{\pgfqpoint{2.455917in}{1.530627in}}{\pgfqpoint{2.452644in}{1.538527in}}{\pgfqpoint{2.446820in}{1.544351in}}%
\pgfpathcurveto{\pgfqpoint{2.440997in}{1.550175in}}{\pgfqpoint{2.433096in}{1.553448in}}{\pgfqpoint{2.424860in}{1.553448in}}%
\pgfpathcurveto{\pgfqpoint{2.416624in}{1.553448in}}{\pgfqpoint{2.408724in}{1.550175in}}{\pgfqpoint{2.402900in}{1.544351in}}%
\pgfpathcurveto{\pgfqpoint{2.397076in}{1.538527in}}{\pgfqpoint{2.393804in}{1.530627in}}{\pgfqpoint{2.393804in}{1.522391in}}%
\pgfpathcurveto{\pgfqpoint{2.393804in}{1.514155in}}{\pgfqpoint{2.397076in}{1.506255in}}{\pgfqpoint{2.402900in}{1.500431in}}%
\pgfpathcurveto{\pgfqpoint{2.408724in}{1.494607in}}{\pgfqpoint{2.416624in}{1.491335in}}{\pgfqpoint{2.424860in}{1.491335in}}%
\pgfpathclose%
\pgfusepath{stroke,fill}%
\end{pgfscope}%
\begin{pgfscope}%
\pgfpathrectangle{\pgfqpoint{0.100000in}{0.220728in}}{\pgfqpoint{3.696000in}{3.696000in}}%
\pgfusepath{clip}%
\pgfsetbuttcap%
\pgfsetroundjoin%
\definecolor{currentfill}{rgb}{0.121569,0.466667,0.705882}%
\pgfsetfillcolor{currentfill}%
\pgfsetfillopacity{0.981276}%
\pgfsetlinewidth{1.003750pt}%
\definecolor{currentstroke}{rgb}{0.121569,0.466667,0.705882}%
\pgfsetstrokecolor{currentstroke}%
\pgfsetstrokeopacity{0.981276}%
\pgfsetdash{}{0pt}%
\pgfpathmoveto{\pgfqpoint{2.418160in}{1.481913in}}%
\pgfpathcurveto{\pgfqpoint{2.426397in}{1.481913in}}{\pgfqpoint{2.434297in}{1.485185in}}{\pgfqpoint{2.440121in}{1.491009in}}%
\pgfpathcurveto{\pgfqpoint{2.445945in}{1.496833in}}{\pgfqpoint{2.449217in}{1.504733in}}{\pgfqpoint{2.449217in}{1.512969in}}%
\pgfpathcurveto{\pgfqpoint{2.449217in}{1.521206in}}{\pgfqpoint{2.445945in}{1.529106in}}{\pgfqpoint{2.440121in}{1.534930in}}%
\pgfpathcurveto{\pgfqpoint{2.434297in}{1.540754in}}{\pgfqpoint{2.426397in}{1.544026in}}{\pgfqpoint{2.418160in}{1.544026in}}%
\pgfpathcurveto{\pgfqpoint{2.409924in}{1.544026in}}{\pgfqpoint{2.402024in}{1.540754in}}{\pgfqpoint{2.396200in}{1.534930in}}%
\pgfpathcurveto{\pgfqpoint{2.390376in}{1.529106in}}{\pgfqpoint{2.387104in}{1.521206in}}{\pgfqpoint{2.387104in}{1.512969in}}%
\pgfpathcurveto{\pgfqpoint{2.387104in}{1.504733in}}{\pgfqpoint{2.390376in}{1.496833in}}{\pgfqpoint{2.396200in}{1.491009in}}%
\pgfpathcurveto{\pgfqpoint{2.402024in}{1.485185in}}{\pgfqpoint{2.409924in}{1.481913in}}{\pgfqpoint{2.418160in}{1.481913in}}%
\pgfpathclose%
\pgfusepath{stroke,fill}%
\end{pgfscope}%
\begin{pgfscope}%
\pgfpathrectangle{\pgfqpoint{0.100000in}{0.220728in}}{\pgfqpoint{3.696000in}{3.696000in}}%
\pgfusepath{clip}%
\pgfsetbuttcap%
\pgfsetroundjoin%
\definecolor{currentfill}{rgb}{0.121569,0.466667,0.705882}%
\pgfsetfillcolor{currentfill}%
\pgfsetfillopacity{0.982941}%
\pgfsetlinewidth{1.003750pt}%
\definecolor{currentstroke}{rgb}{0.121569,0.466667,0.705882}%
\pgfsetstrokecolor{currentstroke}%
\pgfsetstrokeopacity{0.982941}%
\pgfsetdash{}{0pt}%
\pgfpathmoveto{\pgfqpoint{2.269808in}{1.458130in}}%
\pgfpathcurveto{\pgfqpoint{2.278044in}{1.458130in}}{\pgfqpoint{2.285944in}{1.461402in}}{\pgfqpoint{2.291768in}{1.467226in}}%
\pgfpathcurveto{\pgfqpoint{2.297592in}{1.473050in}}{\pgfqpoint{2.300864in}{1.480950in}}{\pgfqpoint{2.300864in}{1.489186in}}%
\pgfpathcurveto{\pgfqpoint{2.300864in}{1.497422in}}{\pgfqpoint{2.297592in}{1.505322in}}{\pgfqpoint{2.291768in}{1.511146in}}%
\pgfpathcurveto{\pgfqpoint{2.285944in}{1.516970in}}{\pgfqpoint{2.278044in}{1.520243in}}{\pgfqpoint{2.269808in}{1.520243in}}%
\pgfpathcurveto{\pgfqpoint{2.261572in}{1.520243in}}{\pgfqpoint{2.253672in}{1.516970in}}{\pgfqpoint{2.247848in}{1.511146in}}%
\pgfpathcurveto{\pgfqpoint{2.242024in}{1.505322in}}{\pgfqpoint{2.238751in}{1.497422in}}{\pgfqpoint{2.238751in}{1.489186in}}%
\pgfpathcurveto{\pgfqpoint{2.238751in}{1.480950in}}{\pgfqpoint{2.242024in}{1.473050in}}{\pgfqpoint{2.247848in}{1.467226in}}%
\pgfpathcurveto{\pgfqpoint{2.253672in}{1.461402in}}{\pgfqpoint{2.261572in}{1.458130in}}{\pgfqpoint{2.269808in}{1.458130in}}%
\pgfpathclose%
\pgfusepath{stroke,fill}%
\end{pgfscope}%
\begin{pgfscope}%
\pgfpathrectangle{\pgfqpoint{0.100000in}{0.220728in}}{\pgfqpoint{3.696000in}{3.696000in}}%
\pgfusepath{clip}%
\pgfsetbuttcap%
\pgfsetroundjoin%
\definecolor{currentfill}{rgb}{0.121569,0.466667,0.705882}%
\pgfsetfillcolor{currentfill}%
\pgfsetfillopacity{0.983968}%
\pgfsetlinewidth{1.003750pt}%
\definecolor{currentstroke}{rgb}{0.121569,0.466667,0.705882}%
\pgfsetstrokecolor{currentstroke}%
\pgfsetstrokeopacity{0.983968}%
\pgfsetdash{}{0pt}%
\pgfpathmoveto{\pgfqpoint{2.411094in}{1.470002in}}%
\pgfpathcurveto{\pgfqpoint{2.419330in}{1.470002in}}{\pgfqpoint{2.427230in}{1.473274in}}{\pgfqpoint{2.433054in}{1.479098in}}%
\pgfpathcurveto{\pgfqpoint{2.438878in}{1.484922in}}{\pgfqpoint{2.442150in}{1.492822in}}{\pgfqpoint{2.442150in}{1.501058in}}%
\pgfpathcurveto{\pgfqpoint{2.442150in}{1.509295in}}{\pgfqpoint{2.438878in}{1.517195in}}{\pgfqpoint{2.433054in}{1.523019in}}%
\pgfpathcurveto{\pgfqpoint{2.427230in}{1.528843in}}{\pgfqpoint{2.419330in}{1.532115in}}{\pgfqpoint{2.411094in}{1.532115in}}%
\pgfpathcurveto{\pgfqpoint{2.402857in}{1.532115in}}{\pgfqpoint{2.394957in}{1.528843in}}{\pgfqpoint{2.389133in}{1.523019in}}%
\pgfpathcurveto{\pgfqpoint{2.383309in}{1.517195in}}{\pgfqpoint{2.380037in}{1.509295in}}{\pgfqpoint{2.380037in}{1.501058in}}%
\pgfpathcurveto{\pgfqpoint{2.380037in}{1.492822in}}{\pgfqpoint{2.383309in}{1.484922in}}{\pgfqpoint{2.389133in}{1.479098in}}%
\pgfpathcurveto{\pgfqpoint{2.394957in}{1.473274in}}{\pgfqpoint{2.402857in}{1.470002in}}{\pgfqpoint{2.411094in}{1.470002in}}%
\pgfpathclose%
\pgfusepath{stroke,fill}%
\end{pgfscope}%
\begin{pgfscope}%
\pgfpathrectangle{\pgfqpoint{0.100000in}{0.220728in}}{\pgfqpoint{3.696000in}{3.696000in}}%
\pgfusepath{clip}%
\pgfsetbuttcap%
\pgfsetroundjoin%
\definecolor{currentfill}{rgb}{0.121569,0.466667,0.705882}%
\pgfsetfillcolor{currentfill}%
\pgfsetfillopacity{0.985302}%
\pgfsetlinewidth{1.003750pt}%
\definecolor{currentstroke}{rgb}{0.121569,0.466667,0.705882}%
\pgfsetstrokecolor{currentstroke}%
\pgfsetstrokeopacity{0.985302}%
\pgfsetdash{}{0pt}%
\pgfpathmoveto{\pgfqpoint{2.406849in}{1.463267in}}%
\pgfpathcurveto{\pgfqpoint{2.415085in}{1.463267in}}{\pgfqpoint{2.422985in}{1.466539in}}{\pgfqpoint{2.428809in}{1.472363in}}%
\pgfpathcurveto{\pgfqpoint{2.434633in}{1.478187in}}{\pgfqpoint{2.437905in}{1.486087in}}{\pgfqpoint{2.437905in}{1.494323in}}%
\pgfpathcurveto{\pgfqpoint{2.437905in}{1.502560in}}{\pgfqpoint{2.434633in}{1.510460in}}{\pgfqpoint{2.428809in}{1.516284in}}%
\pgfpathcurveto{\pgfqpoint{2.422985in}{1.522108in}}{\pgfqpoint{2.415085in}{1.525380in}}{\pgfqpoint{2.406849in}{1.525380in}}%
\pgfpathcurveto{\pgfqpoint{2.398613in}{1.525380in}}{\pgfqpoint{2.390713in}{1.522108in}}{\pgfqpoint{2.384889in}{1.516284in}}%
\pgfpathcurveto{\pgfqpoint{2.379065in}{1.510460in}}{\pgfqpoint{2.375792in}{1.502560in}}{\pgfqpoint{2.375792in}{1.494323in}}%
\pgfpathcurveto{\pgfqpoint{2.375792in}{1.486087in}}{\pgfqpoint{2.379065in}{1.478187in}}{\pgfqpoint{2.384889in}{1.472363in}}%
\pgfpathcurveto{\pgfqpoint{2.390713in}{1.466539in}}{\pgfqpoint{2.398613in}{1.463267in}}{\pgfqpoint{2.406849in}{1.463267in}}%
\pgfpathclose%
\pgfusepath{stroke,fill}%
\end{pgfscope}%
\begin{pgfscope}%
\pgfpathrectangle{\pgfqpoint{0.100000in}{0.220728in}}{\pgfqpoint{3.696000in}{3.696000in}}%
\pgfusepath{clip}%
\pgfsetbuttcap%
\pgfsetroundjoin%
\definecolor{currentfill}{rgb}{0.121569,0.466667,0.705882}%
\pgfsetfillcolor{currentfill}%
\pgfsetfillopacity{0.985426}%
\pgfsetlinewidth{1.003750pt}%
\definecolor{currentstroke}{rgb}{0.121569,0.466667,0.705882}%
\pgfsetstrokecolor{currentstroke}%
\pgfsetstrokeopacity{0.985426}%
\pgfsetdash{}{0pt}%
\pgfpathmoveto{\pgfqpoint{2.282246in}{1.454459in}}%
\pgfpathcurveto{\pgfqpoint{2.290482in}{1.454459in}}{\pgfqpoint{2.298382in}{1.457732in}}{\pgfqpoint{2.304206in}{1.463556in}}%
\pgfpathcurveto{\pgfqpoint{2.310030in}{1.469380in}}{\pgfqpoint{2.313302in}{1.477280in}}{\pgfqpoint{2.313302in}{1.485516in}}%
\pgfpathcurveto{\pgfqpoint{2.313302in}{1.493752in}}{\pgfqpoint{2.310030in}{1.501652in}}{\pgfqpoint{2.304206in}{1.507476in}}%
\pgfpathcurveto{\pgfqpoint{2.298382in}{1.513300in}}{\pgfqpoint{2.290482in}{1.516572in}}{\pgfqpoint{2.282246in}{1.516572in}}%
\pgfpathcurveto{\pgfqpoint{2.274010in}{1.516572in}}{\pgfqpoint{2.266110in}{1.513300in}}{\pgfqpoint{2.260286in}{1.507476in}}%
\pgfpathcurveto{\pgfqpoint{2.254462in}{1.501652in}}{\pgfqpoint{2.251189in}{1.493752in}}{\pgfqpoint{2.251189in}{1.485516in}}%
\pgfpathcurveto{\pgfqpoint{2.251189in}{1.477280in}}{\pgfqpoint{2.254462in}{1.469380in}}{\pgfqpoint{2.260286in}{1.463556in}}%
\pgfpathcurveto{\pgfqpoint{2.266110in}{1.457732in}}{\pgfqpoint{2.274010in}{1.454459in}}{\pgfqpoint{2.282246in}{1.454459in}}%
\pgfpathclose%
\pgfusepath{stroke,fill}%
\end{pgfscope}%
\begin{pgfscope}%
\pgfpathrectangle{\pgfqpoint{0.100000in}{0.220728in}}{\pgfqpoint{3.696000in}{3.696000in}}%
\pgfusepath{clip}%
\pgfsetbuttcap%
\pgfsetroundjoin%
\definecolor{currentfill}{rgb}{0.121569,0.466667,0.705882}%
\pgfsetfillcolor{currentfill}%
\pgfsetfillopacity{0.986068}%
\pgfsetlinewidth{1.003750pt}%
\definecolor{currentstroke}{rgb}{0.121569,0.466667,0.705882}%
\pgfsetstrokecolor{currentstroke}%
\pgfsetstrokeopacity{0.986068}%
\pgfsetdash{}{0pt}%
\pgfpathmoveto{\pgfqpoint{2.404636in}{1.459533in}}%
\pgfpathcurveto{\pgfqpoint{2.412872in}{1.459533in}}{\pgfqpoint{2.420772in}{1.462806in}}{\pgfqpoint{2.426596in}{1.468630in}}%
\pgfpathcurveto{\pgfqpoint{2.432420in}{1.474454in}}{\pgfqpoint{2.435692in}{1.482354in}}{\pgfqpoint{2.435692in}{1.490590in}}%
\pgfpathcurveto{\pgfqpoint{2.435692in}{1.498826in}}{\pgfqpoint{2.432420in}{1.506726in}}{\pgfqpoint{2.426596in}{1.512550in}}%
\pgfpathcurveto{\pgfqpoint{2.420772in}{1.518374in}}{\pgfqpoint{2.412872in}{1.521646in}}{\pgfqpoint{2.404636in}{1.521646in}}%
\pgfpathcurveto{\pgfqpoint{2.396399in}{1.521646in}}{\pgfqpoint{2.388499in}{1.518374in}}{\pgfqpoint{2.382675in}{1.512550in}}%
\pgfpathcurveto{\pgfqpoint{2.376852in}{1.506726in}}{\pgfqpoint{2.373579in}{1.498826in}}{\pgfqpoint{2.373579in}{1.490590in}}%
\pgfpathcurveto{\pgfqpoint{2.373579in}{1.482354in}}{\pgfqpoint{2.376852in}{1.474454in}}{\pgfqpoint{2.382675in}{1.468630in}}%
\pgfpathcurveto{\pgfqpoint{2.388499in}{1.462806in}}{\pgfqpoint{2.396399in}{1.459533in}}{\pgfqpoint{2.404636in}{1.459533in}}%
\pgfpathclose%
\pgfusepath{stroke,fill}%
\end{pgfscope}%
\begin{pgfscope}%
\pgfpathrectangle{\pgfqpoint{0.100000in}{0.220728in}}{\pgfqpoint{3.696000in}{3.696000in}}%
\pgfusepath{clip}%
\pgfsetbuttcap%
\pgfsetroundjoin%
\definecolor{currentfill}{rgb}{0.121569,0.466667,0.705882}%
\pgfsetfillcolor{currentfill}%
\pgfsetfillopacity{0.986468}%
\pgfsetlinewidth{1.003750pt}%
\definecolor{currentstroke}{rgb}{0.121569,0.466667,0.705882}%
\pgfsetstrokecolor{currentstroke}%
\pgfsetstrokeopacity{0.986468}%
\pgfsetdash{}{0pt}%
\pgfpathmoveto{\pgfqpoint{2.403318in}{1.457513in}}%
\pgfpathcurveto{\pgfqpoint{2.411554in}{1.457513in}}{\pgfqpoint{2.419454in}{1.460786in}}{\pgfqpoint{2.425278in}{1.466609in}}%
\pgfpathcurveto{\pgfqpoint{2.431102in}{1.472433in}}{\pgfqpoint{2.434374in}{1.480333in}}{\pgfqpoint{2.434374in}{1.488570in}}%
\pgfpathcurveto{\pgfqpoint{2.434374in}{1.496806in}}{\pgfqpoint{2.431102in}{1.504706in}}{\pgfqpoint{2.425278in}{1.510530in}}%
\pgfpathcurveto{\pgfqpoint{2.419454in}{1.516354in}}{\pgfqpoint{2.411554in}{1.519626in}}{\pgfqpoint{2.403318in}{1.519626in}}%
\pgfpathcurveto{\pgfqpoint{2.395081in}{1.519626in}}{\pgfqpoint{2.387181in}{1.516354in}}{\pgfqpoint{2.381357in}{1.510530in}}%
\pgfpathcurveto{\pgfqpoint{2.375533in}{1.504706in}}{\pgfqpoint{2.372261in}{1.496806in}}{\pgfqpoint{2.372261in}{1.488570in}}%
\pgfpathcurveto{\pgfqpoint{2.372261in}{1.480333in}}{\pgfqpoint{2.375533in}{1.472433in}}{\pgfqpoint{2.381357in}{1.466609in}}%
\pgfpathcurveto{\pgfqpoint{2.387181in}{1.460786in}}{\pgfqpoint{2.395081in}{1.457513in}}{\pgfqpoint{2.403318in}{1.457513in}}%
\pgfpathclose%
\pgfusepath{stroke,fill}%
\end{pgfscope}%
\begin{pgfscope}%
\pgfpathrectangle{\pgfqpoint{0.100000in}{0.220728in}}{\pgfqpoint{3.696000in}{3.696000in}}%
\pgfusepath{clip}%
\pgfsetbuttcap%
\pgfsetroundjoin%
\definecolor{currentfill}{rgb}{0.121569,0.466667,0.705882}%
\pgfsetfillcolor{currentfill}%
\pgfsetfillopacity{0.986702}%
\pgfsetlinewidth{1.003750pt}%
\definecolor{currentstroke}{rgb}{0.121569,0.466667,0.705882}%
\pgfsetstrokecolor{currentstroke}%
\pgfsetstrokeopacity{0.986702}%
\pgfsetdash{}{0pt}%
\pgfpathmoveto{\pgfqpoint{2.402559in}{1.456525in}}%
\pgfpathcurveto{\pgfqpoint{2.410795in}{1.456525in}}{\pgfqpoint{2.418696in}{1.459797in}}{\pgfqpoint{2.424519in}{1.465621in}}%
\pgfpathcurveto{\pgfqpoint{2.430343in}{1.471445in}}{\pgfqpoint{2.433616in}{1.479345in}}{\pgfqpoint{2.433616in}{1.487582in}}%
\pgfpathcurveto{\pgfqpoint{2.433616in}{1.495818in}}{\pgfqpoint{2.430343in}{1.503718in}}{\pgfqpoint{2.424519in}{1.509542in}}%
\pgfpathcurveto{\pgfqpoint{2.418696in}{1.515366in}}{\pgfqpoint{2.410795in}{1.518638in}}{\pgfqpoint{2.402559in}{1.518638in}}%
\pgfpathcurveto{\pgfqpoint{2.394323in}{1.518638in}}{\pgfqpoint{2.386423in}{1.515366in}}{\pgfqpoint{2.380599in}{1.509542in}}%
\pgfpathcurveto{\pgfqpoint{2.374775in}{1.503718in}}{\pgfqpoint{2.371503in}{1.495818in}}{\pgfqpoint{2.371503in}{1.487582in}}%
\pgfpathcurveto{\pgfqpoint{2.371503in}{1.479345in}}{\pgfqpoint{2.374775in}{1.471445in}}{\pgfqpoint{2.380599in}{1.465621in}}%
\pgfpathcurveto{\pgfqpoint{2.386423in}{1.459797in}}{\pgfqpoint{2.394323in}{1.456525in}}{\pgfqpoint{2.402559in}{1.456525in}}%
\pgfpathclose%
\pgfusepath{stroke,fill}%
\end{pgfscope}%
\begin{pgfscope}%
\pgfpathrectangle{\pgfqpoint{0.100000in}{0.220728in}}{\pgfqpoint{3.696000in}{3.696000in}}%
\pgfusepath{clip}%
\pgfsetbuttcap%
\pgfsetroundjoin%
\definecolor{currentfill}{rgb}{0.121569,0.466667,0.705882}%
\pgfsetfillcolor{currentfill}%
\pgfsetfillopacity{0.986817}%
\pgfsetlinewidth{1.003750pt}%
\definecolor{currentstroke}{rgb}{0.121569,0.466667,0.705882}%
\pgfsetstrokecolor{currentstroke}%
\pgfsetstrokeopacity{0.986817}%
\pgfsetdash{}{0pt}%
\pgfpathmoveto{\pgfqpoint{2.402209in}{1.455823in}}%
\pgfpathcurveto{\pgfqpoint{2.410445in}{1.455823in}}{\pgfqpoint{2.418345in}{1.459096in}}{\pgfqpoint{2.424169in}{1.464920in}}%
\pgfpathcurveto{\pgfqpoint{2.429993in}{1.470744in}}{\pgfqpoint{2.433265in}{1.478644in}}{\pgfqpoint{2.433265in}{1.486880in}}%
\pgfpathcurveto{\pgfqpoint{2.433265in}{1.495116in}}{\pgfqpoint{2.429993in}{1.503016in}}{\pgfqpoint{2.424169in}{1.508840in}}%
\pgfpathcurveto{\pgfqpoint{2.418345in}{1.514664in}}{\pgfqpoint{2.410445in}{1.517936in}}{\pgfqpoint{2.402209in}{1.517936in}}%
\pgfpathcurveto{\pgfqpoint{2.393972in}{1.517936in}}{\pgfqpoint{2.386072in}{1.514664in}}{\pgfqpoint{2.380248in}{1.508840in}}%
\pgfpathcurveto{\pgfqpoint{2.374424in}{1.503016in}}{\pgfqpoint{2.371152in}{1.495116in}}{\pgfqpoint{2.371152in}{1.486880in}}%
\pgfpathcurveto{\pgfqpoint{2.371152in}{1.478644in}}{\pgfqpoint{2.374424in}{1.470744in}}{\pgfqpoint{2.380248in}{1.464920in}}%
\pgfpathcurveto{\pgfqpoint{2.386072in}{1.459096in}}{\pgfqpoint{2.393972in}{1.455823in}}{\pgfqpoint{2.402209in}{1.455823in}}%
\pgfpathclose%
\pgfusepath{stroke,fill}%
\end{pgfscope}%
\begin{pgfscope}%
\pgfpathrectangle{\pgfqpoint{0.100000in}{0.220728in}}{\pgfqpoint{3.696000in}{3.696000in}}%
\pgfusepath{clip}%
\pgfsetbuttcap%
\pgfsetroundjoin%
\definecolor{currentfill}{rgb}{0.121569,0.466667,0.705882}%
\pgfsetfillcolor{currentfill}%
\pgfsetfillopacity{0.987250}%
\pgfsetlinewidth{1.003750pt}%
\definecolor{currentstroke}{rgb}{0.121569,0.466667,0.705882}%
\pgfsetstrokecolor{currentstroke}%
\pgfsetstrokeopacity{0.987250}%
\pgfsetdash{}{0pt}%
\pgfpathmoveto{\pgfqpoint{2.289558in}{1.447590in}}%
\pgfpathcurveto{\pgfqpoint{2.297794in}{1.447590in}}{\pgfqpoint{2.305694in}{1.450862in}}{\pgfqpoint{2.311518in}{1.456686in}}%
\pgfpathcurveto{\pgfqpoint{2.317342in}{1.462510in}}{\pgfqpoint{2.320614in}{1.470410in}}{\pgfqpoint{2.320614in}{1.478646in}}%
\pgfpathcurveto{\pgfqpoint{2.320614in}{1.486882in}}{\pgfqpoint{2.317342in}{1.494782in}}{\pgfqpoint{2.311518in}{1.500606in}}%
\pgfpathcurveto{\pgfqpoint{2.305694in}{1.506430in}}{\pgfqpoint{2.297794in}{1.509703in}}{\pgfqpoint{2.289558in}{1.509703in}}%
\pgfpathcurveto{\pgfqpoint{2.281322in}{1.509703in}}{\pgfqpoint{2.273422in}{1.506430in}}{\pgfqpoint{2.267598in}{1.500606in}}%
\pgfpathcurveto{\pgfqpoint{2.261774in}{1.494782in}}{\pgfqpoint{2.258501in}{1.486882in}}{\pgfqpoint{2.258501in}{1.478646in}}%
\pgfpathcurveto{\pgfqpoint{2.258501in}{1.470410in}}{\pgfqpoint{2.261774in}{1.462510in}}{\pgfqpoint{2.267598in}{1.456686in}}%
\pgfpathcurveto{\pgfqpoint{2.273422in}{1.450862in}}{\pgfqpoint{2.281322in}{1.447590in}}{\pgfqpoint{2.289558in}{1.447590in}}%
\pgfpathclose%
\pgfusepath{stroke,fill}%
\end{pgfscope}%
\begin{pgfscope}%
\pgfpathrectangle{\pgfqpoint{0.100000in}{0.220728in}}{\pgfqpoint{3.696000in}{3.696000in}}%
\pgfusepath{clip}%
\pgfsetbuttcap%
\pgfsetroundjoin%
\definecolor{currentfill}{rgb}{0.121569,0.466667,0.705882}%
\pgfsetfillcolor{currentfill}%
\pgfsetfillopacity{0.987763}%
\pgfsetlinewidth{1.003750pt}%
\definecolor{currentstroke}{rgb}{0.121569,0.466667,0.705882}%
\pgfsetstrokecolor{currentstroke}%
\pgfsetstrokeopacity{0.987763}%
\pgfsetdash{}{0pt}%
\pgfpathmoveto{\pgfqpoint{2.291376in}{1.446091in}}%
\pgfpathcurveto{\pgfqpoint{2.299613in}{1.446091in}}{\pgfqpoint{2.307513in}{1.449364in}}{\pgfqpoint{2.313337in}{1.455188in}}%
\pgfpathcurveto{\pgfqpoint{2.319160in}{1.461012in}}{\pgfqpoint{2.322433in}{1.468912in}}{\pgfqpoint{2.322433in}{1.477148in}}%
\pgfpathcurveto{\pgfqpoint{2.322433in}{1.485384in}}{\pgfqpoint{2.319160in}{1.493284in}}{\pgfqpoint{2.313337in}{1.499108in}}%
\pgfpathcurveto{\pgfqpoint{2.307513in}{1.504932in}}{\pgfqpoint{2.299613in}{1.508204in}}{\pgfqpoint{2.291376in}{1.508204in}}%
\pgfpathcurveto{\pgfqpoint{2.283140in}{1.508204in}}{\pgfqpoint{2.275240in}{1.504932in}}{\pgfqpoint{2.269416in}{1.499108in}}%
\pgfpathcurveto{\pgfqpoint{2.263592in}{1.493284in}}{\pgfqpoint{2.260320in}{1.485384in}}{\pgfqpoint{2.260320in}{1.477148in}}%
\pgfpathcurveto{\pgfqpoint{2.260320in}{1.468912in}}{\pgfqpoint{2.263592in}{1.461012in}}{\pgfqpoint{2.269416in}{1.455188in}}%
\pgfpathcurveto{\pgfqpoint{2.275240in}{1.449364in}}{\pgfqpoint{2.283140in}{1.446091in}}{\pgfqpoint{2.291376in}{1.446091in}}%
\pgfpathclose%
\pgfusepath{stroke,fill}%
\end{pgfscope}%
\begin{pgfscope}%
\pgfpathrectangle{\pgfqpoint{0.100000in}{0.220728in}}{\pgfqpoint{3.696000in}{3.696000in}}%
\pgfusepath{clip}%
\pgfsetbuttcap%
\pgfsetroundjoin%
\definecolor{currentfill}{rgb}{0.121569,0.466667,0.705882}%
\pgfsetfillcolor{currentfill}%
\pgfsetfillopacity{0.987804}%
\pgfsetlinewidth{1.003750pt}%
\definecolor{currentstroke}{rgb}{0.121569,0.466667,0.705882}%
\pgfsetstrokecolor{currentstroke}%
\pgfsetstrokeopacity{0.987804}%
\pgfsetdash{}{0pt}%
\pgfpathmoveto{\pgfqpoint{2.398849in}{1.451234in}}%
\pgfpathcurveto{\pgfqpoint{2.407085in}{1.451234in}}{\pgfqpoint{2.414985in}{1.454506in}}{\pgfqpoint{2.420809in}{1.460330in}}%
\pgfpathcurveto{\pgfqpoint{2.426633in}{1.466154in}}{\pgfqpoint{2.429905in}{1.474054in}}{\pgfqpoint{2.429905in}{1.482291in}}%
\pgfpathcurveto{\pgfqpoint{2.429905in}{1.490527in}}{\pgfqpoint{2.426633in}{1.498427in}}{\pgfqpoint{2.420809in}{1.504251in}}%
\pgfpathcurveto{\pgfqpoint{2.414985in}{1.510075in}}{\pgfqpoint{2.407085in}{1.513347in}}{\pgfqpoint{2.398849in}{1.513347in}}%
\pgfpathcurveto{\pgfqpoint{2.390613in}{1.513347in}}{\pgfqpoint{2.382713in}{1.510075in}}{\pgfqpoint{2.376889in}{1.504251in}}%
\pgfpathcurveto{\pgfqpoint{2.371065in}{1.498427in}}{\pgfqpoint{2.367792in}{1.490527in}}{\pgfqpoint{2.367792in}{1.482291in}}%
\pgfpathcurveto{\pgfqpoint{2.367792in}{1.474054in}}{\pgfqpoint{2.371065in}{1.466154in}}{\pgfqpoint{2.376889in}{1.460330in}}%
\pgfpathcurveto{\pgfqpoint{2.382713in}{1.454506in}}{\pgfqpoint{2.390613in}{1.451234in}}{\pgfqpoint{2.398849in}{1.451234in}}%
\pgfpathclose%
\pgfusepath{stroke,fill}%
\end{pgfscope}%
\begin{pgfscope}%
\pgfpathrectangle{\pgfqpoint{0.100000in}{0.220728in}}{\pgfqpoint{3.696000in}{3.696000in}}%
\pgfusepath{clip}%
\pgfsetbuttcap%
\pgfsetroundjoin%
\definecolor{currentfill}{rgb}{0.121569,0.466667,0.705882}%
\pgfsetfillcolor{currentfill}%
\pgfsetfillopacity{0.988380}%
\pgfsetlinewidth{1.003750pt}%
\definecolor{currentstroke}{rgb}{0.121569,0.466667,0.705882}%
\pgfsetstrokecolor{currentstroke}%
\pgfsetstrokeopacity{0.988380}%
\pgfsetdash{}{0pt}%
\pgfpathmoveto{\pgfqpoint{2.397337in}{1.448397in}}%
\pgfpathcurveto{\pgfqpoint{2.405573in}{1.448397in}}{\pgfqpoint{2.413473in}{1.451670in}}{\pgfqpoint{2.419297in}{1.457494in}}%
\pgfpathcurveto{\pgfqpoint{2.425121in}{1.463318in}}{\pgfqpoint{2.428393in}{1.471218in}}{\pgfqpoint{2.428393in}{1.479454in}}%
\pgfpathcurveto{\pgfqpoint{2.428393in}{1.487690in}}{\pgfqpoint{2.425121in}{1.495590in}}{\pgfqpoint{2.419297in}{1.501414in}}%
\pgfpathcurveto{\pgfqpoint{2.413473in}{1.507238in}}{\pgfqpoint{2.405573in}{1.510510in}}{\pgfqpoint{2.397337in}{1.510510in}}%
\pgfpathcurveto{\pgfqpoint{2.389100in}{1.510510in}}{\pgfqpoint{2.381200in}{1.507238in}}{\pgfqpoint{2.375376in}{1.501414in}}%
\pgfpathcurveto{\pgfqpoint{2.369552in}{1.495590in}}{\pgfqpoint{2.366280in}{1.487690in}}{\pgfqpoint{2.366280in}{1.479454in}}%
\pgfpathcurveto{\pgfqpoint{2.366280in}{1.471218in}}{\pgfqpoint{2.369552in}{1.463318in}}{\pgfqpoint{2.375376in}{1.457494in}}%
\pgfpathcurveto{\pgfqpoint{2.381200in}{1.451670in}}{\pgfqpoint{2.389100in}{1.448397in}}{\pgfqpoint{2.397337in}{1.448397in}}%
\pgfpathclose%
\pgfusepath{stroke,fill}%
\end{pgfscope}%
\begin{pgfscope}%
\pgfpathrectangle{\pgfqpoint{0.100000in}{0.220728in}}{\pgfqpoint{3.696000in}{3.696000in}}%
\pgfusepath{clip}%
\pgfsetbuttcap%
\pgfsetroundjoin%
\definecolor{currentfill}{rgb}{0.121569,0.466667,0.705882}%
\pgfsetfillcolor{currentfill}%
\pgfsetfillopacity{0.988655}%
\pgfsetlinewidth{1.003750pt}%
\definecolor{currentstroke}{rgb}{0.121569,0.466667,0.705882}%
\pgfsetstrokecolor{currentstroke}%
\pgfsetstrokeopacity{0.988655}%
\pgfsetdash{}{0pt}%
\pgfpathmoveto{\pgfqpoint{2.293931in}{1.441910in}}%
\pgfpathcurveto{\pgfqpoint{2.302167in}{1.441910in}}{\pgfqpoint{2.310067in}{1.445183in}}{\pgfqpoint{2.315891in}{1.451006in}}%
\pgfpathcurveto{\pgfqpoint{2.321715in}{1.456830in}}{\pgfqpoint{2.324987in}{1.464730in}}{\pgfqpoint{2.324987in}{1.472967in}}%
\pgfpathcurveto{\pgfqpoint{2.324987in}{1.481203in}}{\pgfqpoint{2.321715in}{1.489103in}}{\pgfqpoint{2.315891in}{1.494927in}}%
\pgfpathcurveto{\pgfqpoint{2.310067in}{1.500751in}}{\pgfqpoint{2.302167in}{1.504023in}}{\pgfqpoint{2.293931in}{1.504023in}}%
\pgfpathcurveto{\pgfqpoint{2.285695in}{1.504023in}}{\pgfqpoint{2.277795in}{1.500751in}}{\pgfqpoint{2.271971in}{1.494927in}}%
\pgfpathcurveto{\pgfqpoint{2.266147in}{1.489103in}}{\pgfqpoint{2.262874in}{1.481203in}}{\pgfqpoint{2.262874in}{1.472967in}}%
\pgfpathcurveto{\pgfqpoint{2.262874in}{1.464730in}}{\pgfqpoint{2.266147in}{1.456830in}}{\pgfqpoint{2.271971in}{1.451006in}}%
\pgfpathcurveto{\pgfqpoint{2.277795in}{1.445183in}}{\pgfqpoint{2.285695in}{1.441910in}}{\pgfqpoint{2.293931in}{1.441910in}}%
\pgfpathclose%
\pgfusepath{stroke,fill}%
\end{pgfscope}%
\begin{pgfscope}%
\pgfpathrectangle{\pgfqpoint{0.100000in}{0.220728in}}{\pgfqpoint{3.696000in}{3.696000in}}%
\pgfusepath{clip}%
\pgfsetbuttcap%
\pgfsetroundjoin%
\definecolor{currentfill}{rgb}{0.121569,0.466667,0.705882}%
\pgfsetfillcolor{currentfill}%
\pgfsetfillopacity{0.989558}%
\pgfsetlinewidth{1.003750pt}%
\definecolor{currentstroke}{rgb}{0.121569,0.466667,0.705882}%
\pgfsetstrokecolor{currentstroke}%
\pgfsetstrokeopacity{0.989558}%
\pgfsetdash{}{0pt}%
\pgfpathmoveto{\pgfqpoint{2.393332in}{1.442929in}}%
\pgfpathcurveto{\pgfqpoint{2.401569in}{1.442929in}}{\pgfqpoint{2.409469in}{1.446201in}}{\pgfqpoint{2.415293in}{1.452025in}}%
\pgfpathcurveto{\pgfqpoint{2.421117in}{1.457849in}}{\pgfqpoint{2.424389in}{1.465749in}}{\pgfqpoint{2.424389in}{1.473985in}}%
\pgfpathcurveto{\pgfqpoint{2.424389in}{1.482222in}}{\pgfqpoint{2.421117in}{1.490122in}}{\pgfqpoint{2.415293in}{1.495946in}}%
\pgfpathcurveto{\pgfqpoint{2.409469in}{1.501770in}}{\pgfqpoint{2.401569in}{1.505042in}}{\pgfqpoint{2.393332in}{1.505042in}}%
\pgfpathcurveto{\pgfqpoint{2.385096in}{1.505042in}}{\pgfqpoint{2.377196in}{1.501770in}}{\pgfqpoint{2.371372in}{1.495946in}}%
\pgfpathcurveto{\pgfqpoint{2.365548in}{1.490122in}}{\pgfqpoint{2.362276in}{1.482222in}}{\pgfqpoint{2.362276in}{1.473985in}}%
\pgfpathcurveto{\pgfqpoint{2.362276in}{1.465749in}}{\pgfqpoint{2.365548in}{1.457849in}}{\pgfqpoint{2.371372in}{1.452025in}}%
\pgfpathcurveto{\pgfqpoint{2.377196in}{1.446201in}}{\pgfqpoint{2.385096in}{1.442929in}}{\pgfqpoint{2.393332in}{1.442929in}}%
\pgfpathclose%
\pgfusepath{stroke,fill}%
\end{pgfscope}%
\begin{pgfscope}%
\pgfpathrectangle{\pgfqpoint{0.100000in}{0.220728in}}{\pgfqpoint{3.696000in}{3.696000in}}%
\pgfusepath{clip}%
\pgfsetbuttcap%
\pgfsetroundjoin%
\definecolor{currentfill}{rgb}{0.121569,0.466667,0.705882}%
\pgfsetfillcolor{currentfill}%
\pgfsetfillopacity{0.990186}%
\pgfsetlinewidth{1.003750pt}%
\definecolor{currentstroke}{rgb}{0.121569,0.466667,0.705882}%
\pgfsetstrokecolor{currentstroke}%
\pgfsetstrokeopacity{0.990186}%
\pgfsetdash{}{0pt}%
\pgfpathmoveto{\pgfqpoint{2.297471in}{1.432695in}}%
\pgfpathcurveto{\pgfqpoint{2.305708in}{1.432695in}}{\pgfqpoint{2.313608in}{1.435967in}}{\pgfqpoint{2.319432in}{1.441791in}}%
\pgfpathcurveto{\pgfqpoint{2.325256in}{1.447615in}}{\pgfqpoint{2.328528in}{1.455515in}}{\pgfqpoint{2.328528in}{1.463751in}}%
\pgfpathcurveto{\pgfqpoint{2.328528in}{1.471987in}}{\pgfqpoint{2.325256in}{1.479888in}}{\pgfqpoint{2.319432in}{1.485711in}}%
\pgfpathcurveto{\pgfqpoint{2.313608in}{1.491535in}}{\pgfqpoint{2.305708in}{1.494808in}}{\pgfqpoint{2.297471in}{1.494808in}}%
\pgfpathcurveto{\pgfqpoint{2.289235in}{1.494808in}}{\pgfqpoint{2.281335in}{1.491535in}}{\pgfqpoint{2.275511in}{1.485711in}}%
\pgfpathcurveto{\pgfqpoint{2.269687in}{1.479888in}}{\pgfqpoint{2.266415in}{1.471987in}}{\pgfqpoint{2.266415in}{1.463751in}}%
\pgfpathcurveto{\pgfqpoint{2.266415in}{1.455515in}}{\pgfqpoint{2.269687in}{1.447615in}}{\pgfqpoint{2.275511in}{1.441791in}}%
\pgfpathcurveto{\pgfqpoint{2.281335in}{1.435967in}}{\pgfqpoint{2.289235in}{1.432695in}}{\pgfqpoint{2.297471in}{1.432695in}}%
\pgfpathclose%
\pgfusepath{stroke,fill}%
\end{pgfscope}%
\begin{pgfscope}%
\pgfpathrectangle{\pgfqpoint{0.100000in}{0.220728in}}{\pgfqpoint{3.696000in}{3.696000in}}%
\pgfusepath{clip}%
\pgfsetbuttcap%
\pgfsetroundjoin%
\definecolor{currentfill}{rgb}{0.121569,0.466667,0.705882}%
\pgfsetfillcolor{currentfill}%
\pgfsetfillopacity{0.990230}%
\pgfsetlinewidth{1.003750pt}%
\definecolor{currentstroke}{rgb}{0.121569,0.466667,0.705882}%
\pgfsetstrokecolor{currentstroke}%
\pgfsetstrokeopacity{0.990230}%
\pgfsetdash{}{0pt}%
\pgfpathmoveto{\pgfqpoint{2.391362in}{1.439678in}}%
\pgfpathcurveto{\pgfqpoint{2.399598in}{1.439678in}}{\pgfqpoint{2.407498in}{1.442950in}}{\pgfqpoint{2.413322in}{1.448774in}}%
\pgfpathcurveto{\pgfqpoint{2.419146in}{1.454598in}}{\pgfqpoint{2.422418in}{1.462498in}}{\pgfqpoint{2.422418in}{1.470734in}}%
\pgfpathcurveto{\pgfqpoint{2.422418in}{1.478970in}}{\pgfqpoint{2.419146in}{1.486870in}}{\pgfqpoint{2.413322in}{1.492694in}}%
\pgfpathcurveto{\pgfqpoint{2.407498in}{1.498518in}}{\pgfqpoint{2.399598in}{1.501791in}}{\pgfqpoint{2.391362in}{1.501791in}}%
\pgfpathcurveto{\pgfqpoint{2.383125in}{1.501791in}}{\pgfqpoint{2.375225in}{1.498518in}}{\pgfqpoint{2.369401in}{1.492694in}}%
\pgfpathcurveto{\pgfqpoint{2.363578in}{1.486870in}}{\pgfqpoint{2.360305in}{1.478970in}}{\pgfqpoint{2.360305in}{1.470734in}}%
\pgfpathcurveto{\pgfqpoint{2.360305in}{1.462498in}}{\pgfqpoint{2.363578in}{1.454598in}}{\pgfqpoint{2.369401in}{1.448774in}}%
\pgfpathcurveto{\pgfqpoint{2.375225in}{1.442950in}}{\pgfqpoint{2.383125in}{1.439678in}}{\pgfqpoint{2.391362in}{1.439678in}}%
\pgfpathclose%
\pgfusepath{stroke,fill}%
\end{pgfscope}%
\begin{pgfscope}%
\pgfpathrectangle{\pgfqpoint{0.100000in}{0.220728in}}{\pgfqpoint{3.696000in}{3.696000in}}%
\pgfusepath{clip}%
\pgfsetbuttcap%
\pgfsetroundjoin%
\definecolor{currentfill}{rgb}{0.121569,0.466667,0.705882}%
\pgfsetfillcolor{currentfill}%
\pgfsetfillopacity{0.990554}%
\pgfsetlinewidth{1.003750pt}%
\definecolor{currentstroke}{rgb}{0.121569,0.466667,0.705882}%
\pgfsetstrokecolor{currentstroke}%
\pgfsetstrokeopacity{0.990554}%
\pgfsetdash{}{0pt}%
\pgfpathmoveto{\pgfqpoint{2.390191in}{1.437795in}}%
\pgfpathcurveto{\pgfqpoint{2.398427in}{1.437795in}}{\pgfqpoint{2.406327in}{1.441068in}}{\pgfqpoint{2.412151in}{1.446892in}}%
\pgfpathcurveto{\pgfqpoint{2.417975in}{1.452716in}}{\pgfqpoint{2.421247in}{1.460616in}}{\pgfqpoint{2.421247in}{1.468852in}}%
\pgfpathcurveto{\pgfqpoint{2.421247in}{1.477088in}}{\pgfqpoint{2.417975in}{1.484988in}}{\pgfqpoint{2.412151in}{1.490812in}}%
\pgfpathcurveto{\pgfqpoint{2.406327in}{1.496636in}}{\pgfqpoint{2.398427in}{1.499908in}}{\pgfqpoint{2.390191in}{1.499908in}}%
\pgfpathcurveto{\pgfqpoint{2.381954in}{1.499908in}}{\pgfqpoint{2.374054in}{1.496636in}}{\pgfqpoint{2.368230in}{1.490812in}}%
\pgfpathcurveto{\pgfqpoint{2.362406in}{1.484988in}}{\pgfqpoint{2.359134in}{1.477088in}}{\pgfqpoint{2.359134in}{1.468852in}}%
\pgfpathcurveto{\pgfqpoint{2.359134in}{1.460616in}}{\pgfqpoint{2.362406in}{1.452716in}}{\pgfqpoint{2.368230in}{1.446892in}}%
\pgfpathcurveto{\pgfqpoint{2.374054in}{1.441068in}}{\pgfqpoint{2.381954in}{1.437795in}}{\pgfqpoint{2.390191in}{1.437795in}}%
\pgfpathclose%
\pgfusepath{stroke,fill}%
\end{pgfscope}%
\begin{pgfscope}%
\pgfpathrectangle{\pgfqpoint{0.100000in}{0.220728in}}{\pgfqpoint{3.696000in}{3.696000in}}%
\pgfusepath{clip}%
\pgfsetbuttcap%
\pgfsetroundjoin%
\definecolor{currentfill}{rgb}{0.121569,0.466667,0.705882}%
\pgfsetfillcolor{currentfill}%
\pgfsetfillopacity{0.990761}%
\pgfsetlinewidth{1.003750pt}%
\definecolor{currentstroke}{rgb}{0.121569,0.466667,0.705882}%
\pgfsetstrokecolor{currentstroke}%
\pgfsetstrokeopacity{0.990761}%
\pgfsetdash{}{0pt}%
\pgfpathmoveto{\pgfqpoint{2.389517in}{1.436934in}}%
\pgfpathcurveto{\pgfqpoint{2.397754in}{1.436934in}}{\pgfqpoint{2.405654in}{1.440207in}}{\pgfqpoint{2.411478in}{1.446031in}}%
\pgfpathcurveto{\pgfqpoint{2.417302in}{1.451855in}}{\pgfqpoint{2.420574in}{1.459755in}}{\pgfqpoint{2.420574in}{1.467991in}}%
\pgfpathcurveto{\pgfqpoint{2.420574in}{1.476227in}}{\pgfqpoint{2.417302in}{1.484127in}}{\pgfqpoint{2.411478in}{1.489951in}}%
\pgfpathcurveto{\pgfqpoint{2.405654in}{1.495775in}}{\pgfqpoint{2.397754in}{1.499047in}}{\pgfqpoint{2.389517in}{1.499047in}}%
\pgfpathcurveto{\pgfqpoint{2.381281in}{1.499047in}}{\pgfqpoint{2.373381in}{1.495775in}}{\pgfqpoint{2.367557in}{1.489951in}}%
\pgfpathcurveto{\pgfqpoint{2.361733in}{1.484127in}}{\pgfqpoint{2.358461in}{1.476227in}}{\pgfqpoint{2.358461in}{1.467991in}}%
\pgfpathcurveto{\pgfqpoint{2.358461in}{1.459755in}}{\pgfqpoint{2.361733in}{1.451855in}}{\pgfqpoint{2.367557in}{1.446031in}}%
\pgfpathcurveto{\pgfqpoint{2.373381in}{1.440207in}}{\pgfqpoint{2.381281in}{1.436934in}}{\pgfqpoint{2.389517in}{1.436934in}}%
\pgfpathclose%
\pgfusepath{stroke,fill}%
\end{pgfscope}%
\begin{pgfscope}%
\pgfpathrectangle{\pgfqpoint{0.100000in}{0.220728in}}{\pgfqpoint{3.696000in}{3.696000in}}%
\pgfusepath{clip}%
\pgfsetbuttcap%
\pgfsetroundjoin%
\definecolor{currentfill}{rgb}{0.121569,0.466667,0.705882}%
\pgfsetfillcolor{currentfill}%
\pgfsetfillopacity{0.990860}%
\pgfsetlinewidth{1.003750pt}%
\definecolor{currentstroke}{rgb}{0.121569,0.466667,0.705882}%
\pgfsetstrokecolor{currentstroke}%
\pgfsetstrokeopacity{0.990860}%
\pgfsetdash{}{0pt}%
\pgfpathmoveto{\pgfqpoint{2.299212in}{1.428944in}}%
\pgfpathcurveto{\pgfqpoint{2.307449in}{1.428944in}}{\pgfqpoint{2.315349in}{1.432216in}}{\pgfqpoint{2.321173in}{1.438040in}}%
\pgfpathcurveto{\pgfqpoint{2.326996in}{1.443864in}}{\pgfqpoint{2.330269in}{1.451764in}}{\pgfqpoint{2.330269in}{1.460000in}}%
\pgfpathcurveto{\pgfqpoint{2.330269in}{1.468237in}}{\pgfqpoint{2.326996in}{1.476137in}}{\pgfqpoint{2.321173in}{1.481961in}}%
\pgfpathcurveto{\pgfqpoint{2.315349in}{1.487785in}}{\pgfqpoint{2.307449in}{1.491057in}}{\pgfqpoint{2.299212in}{1.491057in}}%
\pgfpathcurveto{\pgfqpoint{2.290976in}{1.491057in}}{\pgfqpoint{2.283076in}{1.487785in}}{\pgfqpoint{2.277252in}{1.481961in}}%
\pgfpathcurveto{\pgfqpoint{2.271428in}{1.476137in}}{\pgfqpoint{2.268156in}{1.468237in}}{\pgfqpoint{2.268156in}{1.460000in}}%
\pgfpathcurveto{\pgfqpoint{2.268156in}{1.451764in}}{\pgfqpoint{2.271428in}{1.443864in}}{\pgfqpoint{2.277252in}{1.438040in}}%
\pgfpathcurveto{\pgfqpoint{2.283076in}{1.432216in}}{\pgfqpoint{2.290976in}{1.428944in}}{\pgfqpoint{2.299212in}{1.428944in}}%
\pgfpathclose%
\pgfusepath{stroke,fill}%
\end{pgfscope}%
\begin{pgfscope}%
\pgfpathrectangle{\pgfqpoint{0.100000in}{0.220728in}}{\pgfqpoint{3.696000in}{3.696000in}}%
\pgfusepath{clip}%
\pgfsetbuttcap%
\pgfsetroundjoin%
\definecolor{currentfill}{rgb}{0.121569,0.466667,0.705882}%
\pgfsetfillcolor{currentfill}%
\pgfsetfillopacity{0.990877}%
\pgfsetlinewidth{1.003750pt}%
\definecolor{currentstroke}{rgb}{0.121569,0.466667,0.705882}%
\pgfsetstrokecolor{currentstroke}%
\pgfsetstrokeopacity{0.990877}%
\pgfsetdash{}{0pt}%
\pgfpathmoveto{\pgfqpoint{2.389188in}{1.436411in}}%
\pgfpathcurveto{\pgfqpoint{2.397424in}{1.436411in}}{\pgfqpoint{2.405324in}{1.439684in}}{\pgfqpoint{2.411148in}{1.445508in}}%
\pgfpathcurveto{\pgfqpoint{2.416972in}{1.451331in}}{\pgfqpoint{2.420245in}{1.459232in}}{\pgfqpoint{2.420245in}{1.467468in}}%
\pgfpathcurveto{\pgfqpoint{2.420245in}{1.475704in}}{\pgfqpoint{2.416972in}{1.483604in}}{\pgfqpoint{2.411148in}{1.489428in}}%
\pgfpathcurveto{\pgfqpoint{2.405324in}{1.495252in}}{\pgfqpoint{2.397424in}{1.498524in}}{\pgfqpoint{2.389188in}{1.498524in}}%
\pgfpathcurveto{\pgfqpoint{2.380952in}{1.498524in}}{\pgfqpoint{2.373052in}{1.495252in}}{\pgfqpoint{2.367228in}{1.489428in}}%
\pgfpathcurveto{\pgfqpoint{2.361404in}{1.483604in}}{\pgfqpoint{2.358132in}{1.475704in}}{\pgfqpoint{2.358132in}{1.467468in}}%
\pgfpathcurveto{\pgfqpoint{2.358132in}{1.459232in}}{\pgfqpoint{2.361404in}{1.451331in}}{\pgfqpoint{2.367228in}{1.445508in}}%
\pgfpathcurveto{\pgfqpoint{2.373052in}{1.439684in}}{\pgfqpoint{2.380952in}{1.436411in}}{\pgfqpoint{2.389188in}{1.436411in}}%
\pgfpathclose%
\pgfusepath{stroke,fill}%
\end{pgfscope}%
\begin{pgfscope}%
\pgfpathrectangle{\pgfqpoint{0.100000in}{0.220728in}}{\pgfqpoint{3.696000in}{3.696000in}}%
\pgfusepath{clip}%
\pgfsetbuttcap%
\pgfsetroundjoin%
\definecolor{currentfill}{rgb}{0.121569,0.466667,0.705882}%
\pgfsetfillcolor{currentfill}%
\pgfsetfillopacity{0.990935}%
\pgfsetlinewidth{1.003750pt}%
\definecolor{currentstroke}{rgb}{0.121569,0.466667,0.705882}%
\pgfsetstrokecolor{currentstroke}%
\pgfsetstrokeopacity{0.990935}%
\pgfsetdash{}{0pt}%
\pgfpathmoveto{\pgfqpoint{2.388985in}{1.436131in}}%
\pgfpathcurveto{\pgfqpoint{2.397221in}{1.436131in}}{\pgfqpoint{2.405121in}{1.439404in}}{\pgfqpoint{2.410945in}{1.445227in}}%
\pgfpathcurveto{\pgfqpoint{2.416769in}{1.451051in}}{\pgfqpoint{2.420041in}{1.458951in}}{\pgfqpoint{2.420041in}{1.467188in}}%
\pgfpathcurveto{\pgfqpoint{2.420041in}{1.475424in}}{\pgfqpoint{2.416769in}{1.483324in}}{\pgfqpoint{2.410945in}{1.489148in}}%
\pgfpathcurveto{\pgfqpoint{2.405121in}{1.494972in}}{\pgfqpoint{2.397221in}{1.498244in}}{\pgfqpoint{2.388985in}{1.498244in}}%
\pgfpathcurveto{\pgfqpoint{2.380748in}{1.498244in}}{\pgfqpoint{2.372848in}{1.494972in}}{\pgfqpoint{2.367024in}{1.489148in}}%
\pgfpathcurveto{\pgfqpoint{2.361201in}{1.483324in}}{\pgfqpoint{2.357928in}{1.475424in}}{\pgfqpoint{2.357928in}{1.467188in}}%
\pgfpathcurveto{\pgfqpoint{2.357928in}{1.458951in}}{\pgfqpoint{2.361201in}{1.451051in}}{\pgfqpoint{2.367024in}{1.445227in}}%
\pgfpathcurveto{\pgfqpoint{2.372848in}{1.439404in}}{\pgfqpoint{2.380748in}{1.436131in}}{\pgfqpoint{2.388985in}{1.436131in}}%
\pgfpathclose%
\pgfusepath{stroke,fill}%
\end{pgfscope}%
\begin{pgfscope}%
\pgfpathrectangle{\pgfqpoint{0.100000in}{0.220728in}}{\pgfqpoint{3.696000in}{3.696000in}}%
\pgfusepath{clip}%
\pgfsetbuttcap%
\pgfsetroundjoin%
\definecolor{currentfill}{rgb}{0.121569,0.466667,0.705882}%
\pgfsetfillcolor{currentfill}%
\pgfsetfillopacity{0.990967}%
\pgfsetlinewidth{1.003750pt}%
\definecolor{currentstroke}{rgb}{0.121569,0.466667,0.705882}%
\pgfsetstrokecolor{currentstroke}%
\pgfsetstrokeopacity{0.990967}%
\pgfsetdash{}{0pt}%
\pgfpathmoveto{\pgfqpoint{2.388890in}{1.435953in}}%
\pgfpathcurveto{\pgfqpoint{2.397126in}{1.435953in}}{\pgfqpoint{2.405026in}{1.439225in}}{\pgfqpoint{2.410850in}{1.445049in}}%
\pgfpathcurveto{\pgfqpoint{2.416674in}{1.450873in}}{\pgfqpoint{2.419947in}{1.458773in}}{\pgfqpoint{2.419947in}{1.467009in}}%
\pgfpathcurveto{\pgfqpoint{2.419947in}{1.475245in}}{\pgfqpoint{2.416674in}{1.483146in}}{\pgfqpoint{2.410850in}{1.488969in}}%
\pgfpathcurveto{\pgfqpoint{2.405026in}{1.494793in}}{\pgfqpoint{2.397126in}{1.498066in}}{\pgfqpoint{2.388890in}{1.498066in}}%
\pgfpathcurveto{\pgfqpoint{2.380654in}{1.498066in}}{\pgfqpoint{2.372754in}{1.494793in}}{\pgfqpoint{2.366930in}{1.488969in}}%
\pgfpathcurveto{\pgfqpoint{2.361106in}{1.483146in}}{\pgfqpoint{2.357834in}{1.475245in}}{\pgfqpoint{2.357834in}{1.467009in}}%
\pgfpathcurveto{\pgfqpoint{2.357834in}{1.458773in}}{\pgfqpoint{2.361106in}{1.450873in}}{\pgfqpoint{2.366930in}{1.445049in}}%
\pgfpathcurveto{\pgfqpoint{2.372754in}{1.439225in}}{\pgfqpoint{2.380654in}{1.435953in}}{\pgfqpoint{2.388890in}{1.435953in}}%
\pgfpathclose%
\pgfusepath{stroke,fill}%
\end{pgfscope}%
\begin{pgfscope}%
\pgfpathrectangle{\pgfqpoint{0.100000in}{0.220728in}}{\pgfqpoint{3.696000in}{3.696000in}}%
\pgfusepath{clip}%
\pgfsetbuttcap%
\pgfsetroundjoin%
\definecolor{currentfill}{rgb}{0.121569,0.466667,0.705882}%
\pgfsetfillcolor{currentfill}%
\pgfsetfillopacity{0.991408}%
\pgfsetlinewidth{1.003750pt}%
\definecolor{currentstroke}{rgb}{0.121569,0.466667,0.705882}%
\pgfsetstrokecolor{currentstroke}%
\pgfsetstrokeopacity{0.991408}%
\pgfsetdash{}{0pt}%
\pgfpathmoveto{\pgfqpoint{2.387294in}{1.433680in}}%
\pgfpathcurveto{\pgfqpoint{2.395530in}{1.433680in}}{\pgfqpoint{2.403430in}{1.436952in}}{\pgfqpoint{2.409254in}{1.442776in}}%
\pgfpathcurveto{\pgfqpoint{2.415078in}{1.448600in}}{\pgfqpoint{2.418350in}{1.456500in}}{\pgfqpoint{2.418350in}{1.464736in}}%
\pgfpathcurveto{\pgfqpoint{2.418350in}{1.472972in}}{\pgfqpoint{2.415078in}{1.480873in}}{\pgfqpoint{2.409254in}{1.486696in}}%
\pgfpathcurveto{\pgfqpoint{2.403430in}{1.492520in}}{\pgfqpoint{2.395530in}{1.495793in}}{\pgfqpoint{2.387294in}{1.495793in}}%
\pgfpathcurveto{\pgfqpoint{2.379057in}{1.495793in}}{\pgfqpoint{2.371157in}{1.492520in}}{\pgfqpoint{2.365333in}{1.486696in}}%
\pgfpathcurveto{\pgfqpoint{2.359510in}{1.480873in}}{\pgfqpoint{2.356237in}{1.472972in}}{\pgfqpoint{2.356237in}{1.464736in}}%
\pgfpathcurveto{\pgfqpoint{2.356237in}{1.456500in}}{\pgfqpoint{2.359510in}{1.448600in}}{\pgfqpoint{2.365333in}{1.442776in}}%
\pgfpathcurveto{\pgfqpoint{2.371157in}{1.436952in}}{\pgfqpoint{2.379057in}{1.433680in}}{\pgfqpoint{2.387294in}{1.433680in}}%
\pgfpathclose%
\pgfusepath{stroke,fill}%
\end{pgfscope}%
\begin{pgfscope}%
\pgfpathrectangle{\pgfqpoint{0.100000in}{0.220728in}}{\pgfqpoint{3.696000in}{3.696000in}}%
\pgfusepath{clip}%
\pgfsetbuttcap%
\pgfsetroundjoin%
\definecolor{currentfill}{rgb}{0.121569,0.466667,0.705882}%
\pgfsetfillcolor{currentfill}%
\pgfsetfillopacity{0.991675}%
\pgfsetlinewidth{1.003750pt}%
\definecolor{currentstroke}{rgb}{0.121569,0.466667,0.705882}%
\pgfsetstrokecolor{currentstroke}%
\pgfsetstrokeopacity{0.991675}%
\pgfsetdash{}{0pt}%
\pgfpathmoveto{\pgfqpoint{2.386552in}{1.432363in}}%
\pgfpathcurveto{\pgfqpoint{2.394789in}{1.432363in}}{\pgfqpoint{2.402689in}{1.435635in}}{\pgfqpoint{2.408513in}{1.441459in}}%
\pgfpathcurveto{\pgfqpoint{2.414336in}{1.447283in}}{\pgfqpoint{2.417609in}{1.455183in}}{\pgfqpoint{2.417609in}{1.463419in}}%
\pgfpathcurveto{\pgfqpoint{2.417609in}{1.471655in}}{\pgfqpoint{2.414336in}{1.479555in}}{\pgfqpoint{2.408513in}{1.485379in}}%
\pgfpathcurveto{\pgfqpoint{2.402689in}{1.491203in}}{\pgfqpoint{2.394789in}{1.494476in}}{\pgfqpoint{2.386552in}{1.494476in}}%
\pgfpathcurveto{\pgfqpoint{2.378316in}{1.494476in}}{\pgfqpoint{2.370416in}{1.491203in}}{\pgfqpoint{2.364592in}{1.485379in}}%
\pgfpathcurveto{\pgfqpoint{2.358768in}{1.479555in}}{\pgfqpoint{2.355496in}{1.471655in}}{\pgfqpoint{2.355496in}{1.463419in}}%
\pgfpathcurveto{\pgfqpoint{2.355496in}{1.455183in}}{\pgfqpoint{2.358768in}{1.447283in}}{\pgfqpoint{2.364592in}{1.441459in}}%
\pgfpathcurveto{\pgfqpoint{2.370416in}{1.435635in}}{\pgfqpoint{2.378316in}{1.432363in}}{\pgfqpoint{2.386552in}{1.432363in}}%
\pgfpathclose%
\pgfusepath{stroke,fill}%
\end{pgfscope}%
\begin{pgfscope}%
\pgfpathrectangle{\pgfqpoint{0.100000in}{0.220728in}}{\pgfqpoint{3.696000in}{3.696000in}}%
\pgfusepath{clip}%
\pgfsetbuttcap%
\pgfsetroundjoin%
\definecolor{currentfill}{rgb}{0.121569,0.466667,0.705882}%
\pgfsetfillcolor{currentfill}%
\pgfsetfillopacity{0.991810}%
\pgfsetlinewidth{1.003750pt}%
\definecolor{currentstroke}{rgb}{0.121569,0.466667,0.705882}%
\pgfsetstrokecolor{currentstroke}%
\pgfsetstrokeopacity{0.991810}%
\pgfsetdash{}{0pt}%
\pgfpathmoveto{\pgfqpoint{2.386073in}{1.431683in}}%
\pgfpathcurveto{\pgfqpoint{2.394309in}{1.431683in}}{\pgfqpoint{2.402209in}{1.434955in}}{\pgfqpoint{2.408033in}{1.440779in}}%
\pgfpathcurveto{\pgfqpoint{2.413857in}{1.446603in}}{\pgfqpoint{2.417130in}{1.454503in}}{\pgfqpoint{2.417130in}{1.462739in}}%
\pgfpathcurveto{\pgfqpoint{2.417130in}{1.470976in}}{\pgfqpoint{2.413857in}{1.478876in}}{\pgfqpoint{2.408033in}{1.484700in}}%
\pgfpathcurveto{\pgfqpoint{2.402209in}{1.490524in}}{\pgfqpoint{2.394309in}{1.493796in}}{\pgfqpoint{2.386073in}{1.493796in}}%
\pgfpathcurveto{\pgfqpoint{2.377837in}{1.493796in}}{\pgfqpoint{2.369937in}{1.490524in}}{\pgfqpoint{2.364113in}{1.484700in}}%
\pgfpathcurveto{\pgfqpoint{2.358289in}{1.478876in}}{\pgfqpoint{2.355017in}{1.470976in}}{\pgfqpoint{2.355017in}{1.462739in}}%
\pgfpathcurveto{\pgfqpoint{2.355017in}{1.454503in}}{\pgfqpoint{2.358289in}{1.446603in}}{\pgfqpoint{2.364113in}{1.440779in}}%
\pgfpathcurveto{\pgfqpoint{2.369937in}{1.434955in}}{\pgfqpoint{2.377837in}{1.431683in}}{\pgfqpoint{2.386073in}{1.431683in}}%
\pgfpathclose%
\pgfusepath{stroke,fill}%
\end{pgfscope}%
\begin{pgfscope}%
\pgfpathrectangle{\pgfqpoint{0.100000in}{0.220728in}}{\pgfqpoint{3.696000in}{3.696000in}}%
\pgfusepath{clip}%
\pgfsetbuttcap%
\pgfsetroundjoin%
\definecolor{currentfill}{rgb}{0.121569,0.466667,0.705882}%
\pgfsetfillcolor{currentfill}%
\pgfsetfillopacity{0.991892}%
\pgfsetlinewidth{1.003750pt}%
\definecolor{currentstroke}{rgb}{0.121569,0.466667,0.705882}%
\pgfsetstrokecolor{currentstroke}%
\pgfsetstrokeopacity{0.991892}%
\pgfsetdash{}{0pt}%
\pgfpathmoveto{\pgfqpoint{2.385834in}{1.431310in}}%
\pgfpathcurveto{\pgfqpoint{2.394070in}{1.431310in}}{\pgfqpoint{2.401970in}{1.434583in}}{\pgfqpoint{2.407794in}{1.440407in}}%
\pgfpathcurveto{\pgfqpoint{2.413618in}{1.446231in}}{\pgfqpoint{2.416890in}{1.454131in}}{\pgfqpoint{2.416890in}{1.462367in}}%
\pgfpathcurveto{\pgfqpoint{2.416890in}{1.470603in}}{\pgfqpoint{2.413618in}{1.478503in}}{\pgfqpoint{2.407794in}{1.484327in}}%
\pgfpathcurveto{\pgfqpoint{2.401970in}{1.490151in}}{\pgfqpoint{2.394070in}{1.493423in}}{\pgfqpoint{2.385834in}{1.493423in}}%
\pgfpathcurveto{\pgfqpoint{2.377598in}{1.493423in}}{\pgfqpoint{2.369698in}{1.490151in}}{\pgfqpoint{2.363874in}{1.484327in}}%
\pgfpathcurveto{\pgfqpoint{2.358050in}{1.478503in}}{\pgfqpoint{2.354777in}{1.470603in}}{\pgfqpoint{2.354777in}{1.462367in}}%
\pgfpathcurveto{\pgfqpoint{2.354777in}{1.454131in}}{\pgfqpoint{2.358050in}{1.446231in}}{\pgfqpoint{2.363874in}{1.440407in}}%
\pgfpathcurveto{\pgfqpoint{2.369698in}{1.434583in}}{\pgfqpoint{2.377598in}{1.431310in}}{\pgfqpoint{2.385834in}{1.431310in}}%
\pgfpathclose%
\pgfusepath{stroke,fill}%
\end{pgfscope}%
\begin{pgfscope}%
\pgfpathrectangle{\pgfqpoint{0.100000in}{0.220728in}}{\pgfqpoint{3.696000in}{3.696000in}}%
\pgfusepath{clip}%
\pgfsetbuttcap%
\pgfsetroundjoin%
\definecolor{currentfill}{rgb}{0.121569,0.466667,0.705882}%
\pgfsetfillcolor{currentfill}%
\pgfsetfillopacity{0.991935}%
\pgfsetlinewidth{1.003750pt}%
\definecolor{currentstroke}{rgb}{0.121569,0.466667,0.705882}%
\pgfsetstrokecolor{currentstroke}%
\pgfsetstrokeopacity{0.991935}%
\pgfsetdash{}{0pt}%
\pgfpathmoveto{\pgfqpoint{2.385696in}{1.431101in}}%
\pgfpathcurveto{\pgfqpoint{2.393932in}{1.431101in}}{\pgfqpoint{2.401832in}{1.434373in}}{\pgfqpoint{2.407656in}{1.440197in}}%
\pgfpathcurveto{\pgfqpoint{2.413480in}{1.446021in}}{\pgfqpoint{2.416752in}{1.453921in}}{\pgfqpoint{2.416752in}{1.462158in}}%
\pgfpathcurveto{\pgfqpoint{2.416752in}{1.470394in}}{\pgfqpoint{2.413480in}{1.478294in}}{\pgfqpoint{2.407656in}{1.484118in}}%
\pgfpathcurveto{\pgfqpoint{2.401832in}{1.489942in}}{\pgfqpoint{2.393932in}{1.493214in}}{\pgfqpoint{2.385696in}{1.493214in}}%
\pgfpathcurveto{\pgfqpoint{2.377459in}{1.493214in}}{\pgfqpoint{2.369559in}{1.489942in}}{\pgfqpoint{2.363735in}{1.484118in}}%
\pgfpathcurveto{\pgfqpoint{2.357911in}{1.478294in}}{\pgfqpoint{2.354639in}{1.470394in}}{\pgfqpoint{2.354639in}{1.462158in}}%
\pgfpathcurveto{\pgfqpoint{2.354639in}{1.453921in}}{\pgfqpoint{2.357911in}{1.446021in}}{\pgfqpoint{2.363735in}{1.440197in}}%
\pgfpathcurveto{\pgfqpoint{2.369559in}{1.434373in}}{\pgfqpoint{2.377459in}{1.431101in}}{\pgfqpoint{2.385696in}{1.431101in}}%
\pgfpathclose%
\pgfusepath{stroke,fill}%
\end{pgfscope}%
\begin{pgfscope}%
\pgfpathrectangle{\pgfqpoint{0.100000in}{0.220728in}}{\pgfqpoint{3.696000in}{3.696000in}}%
\pgfusepath{clip}%
\pgfsetbuttcap%
\pgfsetroundjoin%
\definecolor{currentfill}{rgb}{0.121569,0.466667,0.705882}%
\pgfsetfillcolor{currentfill}%
\pgfsetfillopacity{0.991958}%
\pgfsetlinewidth{1.003750pt}%
\definecolor{currentstroke}{rgb}{0.121569,0.466667,0.705882}%
\pgfsetstrokecolor{currentstroke}%
\pgfsetstrokeopacity{0.991958}%
\pgfsetdash{}{0pt}%
\pgfpathmoveto{\pgfqpoint{2.385616in}{1.430993in}}%
\pgfpathcurveto{\pgfqpoint{2.393852in}{1.430993in}}{\pgfqpoint{2.401752in}{1.434265in}}{\pgfqpoint{2.407576in}{1.440089in}}%
\pgfpathcurveto{\pgfqpoint{2.413400in}{1.445913in}}{\pgfqpoint{2.416672in}{1.453813in}}{\pgfqpoint{2.416672in}{1.462050in}}%
\pgfpathcurveto{\pgfqpoint{2.416672in}{1.470286in}}{\pgfqpoint{2.413400in}{1.478186in}}{\pgfqpoint{2.407576in}{1.484010in}}%
\pgfpathcurveto{\pgfqpoint{2.401752in}{1.489834in}}{\pgfqpoint{2.393852in}{1.493106in}}{\pgfqpoint{2.385616in}{1.493106in}}%
\pgfpathcurveto{\pgfqpoint{2.377379in}{1.493106in}}{\pgfqpoint{2.369479in}{1.489834in}}{\pgfqpoint{2.363655in}{1.484010in}}%
\pgfpathcurveto{\pgfqpoint{2.357831in}{1.478186in}}{\pgfqpoint{2.354559in}{1.470286in}}{\pgfqpoint{2.354559in}{1.462050in}}%
\pgfpathcurveto{\pgfqpoint{2.354559in}{1.453813in}}{\pgfqpoint{2.357831in}{1.445913in}}{\pgfqpoint{2.363655in}{1.440089in}}%
\pgfpathcurveto{\pgfqpoint{2.369479in}{1.434265in}}{\pgfqpoint{2.377379in}{1.430993in}}{\pgfqpoint{2.385616in}{1.430993in}}%
\pgfpathclose%
\pgfusepath{stroke,fill}%
\end{pgfscope}%
\begin{pgfscope}%
\pgfpathrectangle{\pgfqpoint{0.100000in}{0.220728in}}{\pgfqpoint{3.696000in}{3.696000in}}%
\pgfusepath{clip}%
\pgfsetbuttcap%
\pgfsetroundjoin%
\definecolor{currentfill}{rgb}{0.121569,0.466667,0.705882}%
\pgfsetfillcolor{currentfill}%
\pgfsetfillopacity{0.992042}%
\pgfsetlinewidth{1.003750pt}%
\definecolor{currentstroke}{rgb}{0.121569,0.466667,0.705882}%
\pgfsetstrokecolor{currentstroke}%
\pgfsetstrokeopacity{0.992042}%
\pgfsetdash{}{0pt}%
\pgfpathmoveto{\pgfqpoint{2.302305in}{1.421818in}}%
\pgfpathcurveto{\pgfqpoint{2.310541in}{1.421818in}}{\pgfqpoint{2.318442in}{1.425090in}}{\pgfqpoint{2.324265in}{1.430914in}}%
\pgfpathcurveto{\pgfqpoint{2.330089in}{1.436738in}}{\pgfqpoint{2.333362in}{1.444638in}}{\pgfqpoint{2.333362in}{1.452874in}}%
\pgfpathcurveto{\pgfqpoint{2.333362in}{1.461110in}}{\pgfqpoint{2.330089in}{1.469011in}}{\pgfqpoint{2.324265in}{1.474834in}}%
\pgfpathcurveto{\pgfqpoint{2.318442in}{1.480658in}}{\pgfqpoint{2.310541in}{1.483931in}}{\pgfqpoint{2.302305in}{1.483931in}}%
\pgfpathcurveto{\pgfqpoint{2.294069in}{1.483931in}}{\pgfqpoint{2.286169in}{1.480658in}}{\pgfqpoint{2.280345in}{1.474834in}}%
\pgfpathcurveto{\pgfqpoint{2.274521in}{1.469011in}}{\pgfqpoint{2.271249in}{1.461110in}}{\pgfqpoint{2.271249in}{1.452874in}}%
\pgfpathcurveto{\pgfqpoint{2.271249in}{1.444638in}}{\pgfqpoint{2.274521in}{1.436738in}}{\pgfqpoint{2.280345in}{1.430914in}}%
\pgfpathcurveto{\pgfqpoint{2.286169in}{1.425090in}}{\pgfqpoint{2.294069in}{1.421818in}}{\pgfqpoint{2.302305in}{1.421818in}}%
\pgfpathclose%
\pgfusepath{stroke,fill}%
\end{pgfscope}%
\begin{pgfscope}%
\pgfpathrectangle{\pgfqpoint{0.100000in}{0.220728in}}{\pgfqpoint{3.696000in}{3.696000in}}%
\pgfusepath{clip}%
\pgfsetbuttcap%
\pgfsetroundjoin%
\definecolor{currentfill}{rgb}{0.121569,0.466667,0.705882}%
\pgfsetfillcolor{currentfill}%
\pgfsetfillopacity{0.992518}%
\pgfsetlinewidth{1.003750pt}%
\definecolor{currentstroke}{rgb}{0.121569,0.466667,0.705882}%
\pgfsetstrokecolor{currentstroke}%
\pgfsetstrokeopacity{0.992518}%
\pgfsetdash{}{0pt}%
\pgfpathmoveto{\pgfqpoint{2.384137in}{1.428185in}}%
\pgfpathcurveto{\pgfqpoint{2.392373in}{1.428185in}}{\pgfqpoint{2.400273in}{1.431457in}}{\pgfqpoint{2.406097in}{1.437281in}}%
\pgfpathcurveto{\pgfqpoint{2.411921in}{1.443105in}}{\pgfqpoint{2.415193in}{1.451005in}}{\pgfqpoint{2.415193in}{1.459241in}}%
\pgfpathcurveto{\pgfqpoint{2.415193in}{1.467478in}}{\pgfqpoint{2.411921in}{1.475378in}}{\pgfqpoint{2.406097in}{1.481202in}}%
\pgfpathcurveto{\pgfqpoint{2.400273in}{1.487026in}}{\pgfqpoint{2.392373in}{1.490298in}}{\pgfqpoint{2.384137in}{1.490298in}}%
\pgfpathcurveto{\pgfqpoint{2.375901in}{1.490298in}}{\pgfqpoint{2.368001in}{1.487026in}}{\pgfqpoint{2.362177in}{1.481202in}}%
\pgfpathcurveto{\pgfqpoint{2.356353in}{1.475378in}}{\pgfqpoint{2.353080in}{1.467478in}}{\pgfqpoint{2.353080in}{1.459241in}}%
\pgfpathcurveto{\pgfqpoint{2.353080in}{1.451005in}}{\pgfqpoint{2.356353in}{1.443105in}}{\pgfqpoint{2.362177in}{1.437281in}}%
\pgfpathcurveto{\pgfqpoint{2.368001in}{1.431457in}}{\pgfqpoint{2.375901in}{1.428185in}}{\pgfqpoint{2.384137in}{1.428185in}}%
\pgfpathclose%
\pgfusepath{stroke,fill}%
\end{pgfscope}%
\begin{pgfscope}%
\pgfpathrectangle{\pgfqpoint{0.100000in}{0.220728in}}{\pgfqpoint{3.696000in}{3.696000in}}%
\pgfusepath{clip}%
\pgfsetbuttcap%
\pgfsetroundjoin%
\definecolor{currentfill}{rgb}{0.121569,0.466667,0.705882}%
\pgfsetfillcolor{currentfill}%
\pgfsetfillopacity{0.993690}%
\pgfsetlinewidth{1.003750pt}%
\definecolor{currentstroke}{rgb}{0.121569,0.466667,0.705882}%
\pgfsetstrokecolor{currentstroke}%
\pgfsetstrokeopacity{0.993690}%
\pgfsetdash{}{0pt}%
\pgfpathmoveto{\pgfqpoint{2.380098in}{1.422269in}}%
\pgfpathcurveto{\pgfqpoint{2.388334in}{1.422269in}}{\pgfqpoint{2.396234in}{1.425541in}}{\pgfqpoint{2.402058in}{1.431365in}}%
\pgfpathcurveto{\pgfqpoint{2.407882in}{1.437189in}}{\pgfqpoint{2.411154in}{1.445089in}}{\pgfqpoint{2.411154in}{1.453326in}}%
\pgfpathcurveto{\pgfqpoint{2.411154in}{1.461562in}}{\pgfqpoint{2.407882in}{1.469462in}}{\pgfqpoint{2.402058in}{1.475286in}}%
\pgfpathcurveto{\pgfqpoint{2.396234in}{1.481110in}}{\pgfqpoint{2.388334in}{1.484382in}}{\pgfqpoint{2.380098in}{1.484382in}}%
\pgfpathcurveto{\pgfqpoint{2.371862in}{1.484382in}}{\pgfqpoint{2.363962in}{1.481110in}}{\pgfqpoint{2.358138in}{1.475286in}}%
\pgfpathcurveto{\pgfqpoint{2.352314in}{1.469462in}}{\pgfqpoint{2.349041in}{1.461562in}}{\pgfqpoint{2.349041in}{1.453326in}}%
\pgfpathcurveto{\pgfqpoint{2.349041in}{1.445089in}}{\pgfqpoint{2.352314in}{1.437189in}}{\pgfqpoint{2.358138in}{1.431365in}}%
\pgfpathcurveto{\pgfqpoint{2.363962in}{1.425541in}}{\pgfqpoint{2.371862in}{1.422269in}}{\pgfqpoint{2.380098in}{1.422269in}}%
\pgfpathclose%
\pgfusepath{stroke,fill}%
\end{pgfscope}%
\begin{pgfscope}%
\pgfpathrectangle{\pgfqpoint{0.100000in}{0.220728in}}{\pgfqpoint{3.696000in}{3.696000in}}%
\pgfusepath{clip}%
\pgfsetbuttcap%
\pgfsetroundjoin%
\definecolor{currentfill}{rgb}{0.121569,0.466667,0.705882}%
\pgfsetfillcolor{currentfill}%
\pgfsetfillopacity{0.994093}%
\pgfsetlinewidth{1.003750pt}%
\definecolor{currentstroke}{rgb}{0.121569,0.466667,0.705882}%
\pgfsetstrokecolor{currentstroke}%
\pgfsetstrokeopacity{0.994093}%
\pgfsetdash{}{0pt}%
\pgfpathmoveto{\pgfqpoint{2.308865in}{1.409550in}}%
\pgfpathcurveto{\pgfqpoint{2.317102in}{1.409550in}}{\pgfqpoint{2.325002in}{1.412822in}}{\pgfqpoint{2.330826in}{1.418646in}}%
\pgfpathcurveto{\pgfqpoint{2.336650in}{1.424470in}}{\pgfqpoint{2.339922in}{1.432370in}}{\pgfqpoint{2.339922in}{1.440606in}}%
\pgfpathcurveto{\pgfqpoint{2.339922in}{1.448843in}}{\pgfqpoint{2.336650in}{1.456743in}}{\pgfqpoint{2.330826in}{1.462567in}}%
\pgfpathcurveto{\pgfqpoint{2.325002in}{1.468391in}}{\pgfqpoint{2.317102in}{1.471663in}}{\pgfqpoint{2.308865in}{1.471663in}}%
\pgfpathcurveto{\pgfqpoint{2.300629in}{1.471663in}}{\pgfqpoint{2.292729in}{1.468391in}}{\pgfqpoint{2.286905in}{1.462567in}}%
\pgfpathcurveto{\pgfqpoint{2.281081in}{1.456743in}}{\pgfqpoint{2.277809in}{1.448843in}}{\pgfqpoint{2.277809in}{1.440606in}}%
\pgfpathcurveto{\pgfqpoint{2.277809in}{1.432370in}}{\pgfqpoint{2.281081in}{1.424470in}}{\pgfqpoint{2.286905in}{1.418646in}}%
\pgfpathcurveto{\pgfqpoint{2.292729in}{1.412822in}}{\pgfqpoint{2.300629in}{1.409550in}}{\pgfqpoint{2.308865in}{1.409550in}}%
\pgfpathclose%
\pgfusepath{stroke,fill}%
\end{pgfscope}%
\begin{pgfscope}%
\pgfpathrectangle{\pgfqpoint{0.100000in}{0.220728in}}{\pgfqpoint{3.696000in}{3.696000in}}%
\pgfusepath{clip}%
\pgfsetbuttcap%
\pgfsetroundjoin%
\definecolor{currentfill}{rgb}{0.121569,0.466667,0.705882}%
\pgfsetfillcolor{currentfill}%
\pgfsetfillopacity{0.995778}%
\pgfsetlinewidth{1.003750pt}%
\definecolor{currentstroke}{rgb}{0.121569,0.466667,0.705882}%
\pgfsetstrokecolor{currentstroke}%
\pgfsetstrokeopacity{0.995778}%
\pgfsetdash{}{0pt}%
\pgfpathmoveto{\pgfqpoint{2.374647in}{1.412424in}}%
\pgfpathcurveto{\pgfqpoint{2.382883in}{1.412424in}}{\pgfqpoint{2.390783in}{1.415697in}}{\pgfqpoint{2.396607in}{1.421520in}}%
\pgfpathcurveto{\pgfqpoint{2.402431in}{1.427344in}}{\pgfqpoint{2.405704in}{1.435244in}}{\pgfqpoint{2.405704in}{1.443481in}}%
\pgfpathcurveto{\pgfqpoint{2.405704in}{1.451717in}}{\pgfqpoint{2.402431in}{1.459617in}}{\pgfqpoint{2.396607in}{1.465441in}}%
\pgfpathcurveto{\pgfqpoint{2.390783in}{1.471265in}}{\pgfqpoint{2.382883in}{1.474537in}}{\pgfqpoint{2.374647in}{1.474537in}}%
\pgfpathcurveto{\pgfqpoint{2.366411in}{1.474537in}}{\pgfqpoint{2.358511in}{1.471265in}}{\pgfqpoint{2.352687in}{1.465441in}}%
\pgfpathcurveto{\pgfqpoint{2.346863in}{1.459617in}}{\pgfqpoint{2.343591in}{1.451717in}}{\pgfqpoint{2.343591in}{1.443481in}}%
\pgfpathcurveto{\pgfqpoint{2.343591in}{1.435244in}}{\pgfqpoint{2.346863in}{1.427344in}}{\pgfqpoint{2.352687in}{1.421520in}}%
\pgfpathcurveto{\pgfqpoint{2.358511in}{1.415697in}}{\pgfqpoint{2.366411in}{1.412424in}}{\pgfqpoint{2.374647in}{1.412424in}}%
\pgfpathclose%
\pgfusepath{stroke,fill}%
\end{pgfscope}%
\begin{pgfscope}%
\pgfpathrectangle{\pgfqpoint{0.100000in}{0.220728in}}{\pgfqpoint{3.696000in}{3.696000in}}%
\pgfusepath{clip}%
\pgfsetbuttcap%
\pgfsetroundjoin%
\definecolor{currentfill}{rgb}{0.121569,0.466667,0.705882}%
\pgfsetfillcolor{currentfill}%
\pgfsetfillopacity{0.996794}%
\pgfsetlinewidth{1.003750pt}%
\definecolor{currentstroke}{rgb}{0.121569,0.466667,0.705882}%
\pgfsetstrokecolor{currentstroke}%
\pgfsetstrokeopacity{0.996794}%
\pgfsetdash{}{0pt}%
\pgfpathmoveto{\pgfqpoint{2.371126in}{1.407042in}}%
\pgfpathcurveto{\pgfqpoint{2.379362in}{1.407042in}}{\pgfqpoint{2.387262in}{1.410314in}}{\pgfqpoint{2.393086in}{1.416138in}}%
\pgfpathcurveto{\pgfqpoint{2.398910in}{1.421962in}}{\pgfqpoint{2.402183in}{1.429862in}}{\pgfqpoint{2.402183in}{1.438098in}}%
\pgfpathcurveto{\pgfqpoint{2.402183in}{1.446335in}}{\pgfqpoint{2.398910in}{1.454235in}}{\pgfqpoint{2.393086in}{1.460059in}}%
\pgfpathcurveto{\pgfqpoint{2.387262in}{1.465882in}}{\pgfqpoint{2.379362in}{1.469155in}}{\pgfqpoint{2.371126in}{1.469155in}}%
\pgfpathcurveto{\pgfqpoint{2.362890in}{1.469155in}}{\pgfqpoint{2.354990in}{1.465882in}}{\pgfqpoint{2.349166in}{1.460059in}}%
\pgfpathcurveto{\pgfqpoint{2.343342in}{1.454235in}}{\pgfqpoint{2.340070in}{1.446335in}}{\pgfqpoint{2.340070in}{1.438098in}}%
\pgfpathcurveto{\pgfqpoint{2.340070in}{1.429862in}}{\pgfqpoint{2.343342in}{1.421962in}}{\pgfqpoint{2.349166in}{1.416138in}}%
\pgfpathcurveto{\pgfqpoint{2.354990in}{1.410314in}}{\pgfqpoint{2.362890in}{1.407042in}}{\pgfqpoint{2.371126in}{1.407042in}}%
\pgfpathclose%
\pgfusepath{stroke,fill}%
\end{pgfscope}%
\begin{pgfscope}%
\pgfpathrectangle{\pgfqpoint{0.100000in}{0.220728in}}{\pgfqpoint{3.696000in}{3.696000in}}%
\pgfusepath{clip}%
\pgfsetbuttcap%
\pgfsetroundjoin%
\definecolor{currentfill}{rgb}{0.121569,0.466667,0.705882}%
\pgfsetfillcolor{currentfill}%
\pgfsetfillopacity{0.997400}%
\pgfsetlinewidth{1.003750pt}%
\definecolor{currentstroke}{rgb}{0.121569,0.466667,0.705882}%
\pgfsetstrokecolor{currentstroke}%
\pgfsetstrokeopacity{0.997400}%
\pgfsetdash{}{0pt}%
\pgfpathmoveto{\pgfqpoint{2.369173in}{1.404328in}}%
\pgfpathcurveto{\pgfqpoint{2.377409in}{1.404328in}}{\pgfqpoint{2.385309in}{1.407600in}}{\pgfqpoint{2.391133in}{1.413424in}}%
\pgfpathcurveto{\pgfqpoint{2.396957in}{1.419248in}}{\pgfqpoint{2.400230in}{1.427148in}}{\pgfqpoint{2.400230in}{1.435385in}}%
\pgfpathcurveto{\pgfqpoint{2.400230in}{1.443621in}}{\pgfqpoint{2.396957in}{1.451521in}}{\pgfqpoint{2.391133in}{1.457345in}}%
\pgfpathcurveto{\pgfqpoint{2.385309in}{1.463169in}}{\pgfqpoint{2.377409in}{1.466441in}}{\pgfqpoint{2.369173in}{1.466441in}}%
\pgfpathcurveto{\pgfqpoint{2.360937in}{1.466441in}}{\pgfqpoint{2.353037in}{1.463169in}}{\pgfqpoint{2.347213in}{1.457345in}}%
\pgfpathcurveto{\pgfqpoint{2.341389in}{1.451521in}}{\pgfqpoint{2.338117in}{1.443621in}}{\pgfqpoint{2.338117in}{1.435385in}}%
\pgfpathcurveto{\pgfqpoint{2.338117in}{1.427148in}}{\pgfqpoint{2.341389in}{1.419248in}}{\pgfqpoint{2.347213in}{1.413424in}}%
\pgfpathcurveto{\pgfqpoint{2.353037in}{1.407600in}}{\pgfqpoint{2.360937in}{1.404328in}}{\pgfqpoint{2.369173in}{1.404328in}}%
\pgfpathclose%
\pgfusepath{stroke,fill}%
\end{pgfscope}%
\begin{pgfscope}%
\pgfpathrectangle{\pgfqpoint{0.100000in}{0.220728in}}{\pgfqpoint{3.696000in}{3.696000in}}%
\pgfusepath{clip}%
\pgfsetbuttcap%
\pgfsetroundjoin%
\definecolor{currentfill}{rgb}{0.121569,0.466667,0.705882}%
\pgfsetfillcolor{currentfill}%
\pgfsetfillopacity{0.997523}%
\pgfsetlinewidth{1.003750pt}%
\definecolor{currentstroke}{rgb}{0.121569,0.466667,0.705882}%
\pgfsetstrokecolor{currentstroke}%
\pgfsetstrokeopacity{0.997523}%
\pgfsetdash{}{0pt}%
\pgfpathmoveto{\pgfqpoint{2.324529in}{1.391990in}}%
\pgfpathcurveto{\pgfqpoint{2.332766in}{1.391990in}}{\pgfqpoint{2.340666in}{1.395262in}}{\pgfqpoint{2.346490in}{1.401086in}}%
\pgfpathcurveto{\pgfqpoint{2.352314in}{1.406910in}}{\pgfqpoint{2.355586in}{1.414810in}}{\pgfqpoint{2.355586in}{1.423047in}}%
\pgfpathcurveto{\pgfqpoint{2.355586in}{1.431283in}}{\pgfqpoint{2.352314in}{1.439183in}}{\pgfqpoint{2.346490in}{1.445007in}}%
\pgfpathcurveto{\pgfqpoint{2.340666in}{1.450831in}}{\pgfqpoint{2.332766in}{1.454103in}}{\pgfqpoint{2.324529in}{1.454103in}}%
\pgfpathcurveto{\pgfqpoint{2.316293in}{1.454103in}}{\pgfqpoint{2.308393in}{1.450831in}}{\pgfqpoint{2.302569in}{1.445007in}}%
\pgfpathcurveto{\pgfqpoint{2.296745in}{1.439183in}}{\pgfqpoint{2.293473in}{1.431283in}}{\pgfqpoint{2.293473in}{1.423047in}}%
\pgfpathcurveto{\pgfqpoint{2.293473in}{1.414810in}}{\pgfqpoint{2.296745in}{1.406910in}}{\pgfqpoint{2.302569in}{1.401086in}}%
\pgfpathcurveto{\pgfqpoint{2.308393in}{1.395262in}}{\pgfqpoint{2.316293in}{1.391990in}}{\pgfqpoint{2.324529in}{1.391990in}}%
\pgfpathclose%
\pgfusepath{stroke,fill}%
\end{pgfscope}%
\begin{pgfscope}%
\pgfpathrectangle{\pgfqpoint{0.100000in}{0.220728in}}{\pgfqpoint{3.696000in}{3.696000in}}%
\pgfusepath{clip}%
\pgfsetbuttcap%
\pgfsetroundjoin%
\definecolor{currentfill}{rgb}{0.121569,0.466667,0.705882}%
\pgfsetfillcolor{currentfill}%
\pgfsetfillopacity{0.997704}%
\pgfsetlinewidth{1.003750pt}%
\definecolor{currentstroke}{rgb}{0.121569,0.466667,0.705882}%
\pgfsetstrokecolor{currentstroke}%
\pgfsetstrokeopacity{0.997704}%
\pgfsetdash{}{0pt}%
\pgfpathmoveto{\pgfqpoint{2.368122in}{1.402668in}}%
\pgfpathcurveto{\pgfqpoint{2.376358in}{1.402668in}}{\pgfqpoint{2.384258in}{1.405941in}}{\pgfqpoint{2.390082in}{1.411764in}}%
\pgfpathcurveto{\pgfqpoint{2.395906in}{1.417588in}}{\pgfqpoint{2.399178in}{1.425488in}}{\pgfqpoint{2.399178in}{1.433725in}}%
\pgfpathcurveto{\pgfqpoint{2.399178in}{1.441961in}}{\pgfqpoint{2.395906in}{1.449861in}}{\pgfqpoint{2.390082in}{1.455685in}}%
\pgfpathcurveto{\pgfqpoint{2.384258in}{1.461509in}}{\pgfqpoint{2.376358in}{1.464781in}}{\pgfqpoint{2.368122in}{1.464781in}}%
\pgfpathcurveto{\pgfqpoint{2.359886in}{1.464781in}}{\pgfqpoint{2.351986in}{1.461509in}}{\pgfqpoint{2.346162in}{1.455685in}}%
\pgfpathcurveto{\pgfqpoint{2.340338in}{1.449861in}}{\pgfqpoint{2.337065in}{1.441961in}}{\pgfqpoint{2.337065in}{1.433725in}}%
\pgfpathcurveto{\pgfqpoint{2.337065in}{1.425488in}}{\pgfqpoint{2.340338in}{1.417588in}}{\pgfqpoint{2.346162in}{1.411764in}}%
\pgfpathcurveto{\pgfqpoint{2.351986in}{1.405941in}}{\pgfqpoint{2.359886in}{1.402668in}}{\pgfqpoint{2.368122in}{1.402668in}}%
\pgfpathclose%
\pgfusepath{stroke,fill}%
\end{pgfscope}%
\begin{pgfscope}%
\pgfpathrectangle{\pgfqpoint{0.100000in}{0.220728in}}{\pgfqpoint{3.696000in}{3.696000in}}%
\pgfusepath{clip}%
\pgfsetbuttcap%
\pgfsetroundjoin%
\definecolor{currentfill}{rgb}{0.121569,0.466667,0.705882}%
\pgfsetfillcolor{currentfill}%
\pgfsetfillopacity{0.997887}%
\pgfsetlinewidth{1.003750pt}%
\definecolor{currentstroke}{rgb}{0.121569,0.466667,0.705882}%
\pgfsetstrokecolor{currentstroke}%
\pgfsetstrokeopacity{0.997887}%
\pgfsetdash{}{0pt}%
\pgfpathmoveto{\pgfqpoint{2.367546in}{1.401816in}}%
\pgfpathcurveto{\pgfqpoint{2.375782in}{1.401816in}}{\pgfqpoint{2.383683in}{1.405089in}}{\pgfqpoint{2.389506in}{1.410912in}}%
\pgfpathcurveto{\pgfqpoint{2.395330in}{1.416736in}}{\pgfqpoint{2.398603in}{1.424636in}}{\pgfqpoint{2.398603in}{1.432873in}}%
\pgfpathcurveto{\pgfqpoint{2.398603in}{1.441109in}}{\pgfqpoint{2.395330in}{1.449009in}}{\pgfqpoint{2.389506in}{1.454833in}}%
\pgfpathcurveto{\pgfqpoint{2.383683in}{1.460657in}}{\pgfqpoint{2.375782in}{1.463929in}}{\pgfqpoint{2.367546in}{1.463929in}}%
\pgfpathcurveto{\pgfqpoint{2.359310in}{1.463929in}}{\pgfqpoint{2.351410in}{1.460657in}}{\pgfqpoint{2.345586in}{1.454833in}}%
\pgfpathcurveto{\pgfqpoint{2.339762in}{1.449009in}}{\pgfqpoint{2.336490in}{1.441109in}}{\pgfqpoint{2.336490in}{1.432873in}}%
\pgfpathcurveto{\pgfqpoint{2.336490in}{1.424636in}}{\pgfqpoint{2.339762in}{1.416736in}}{\pgfqpoint{2.345586in}{1.410912in}}%
\pgfpathcurveto{\pgfqpoint{2.351410in}{1.405089in}}{\pgfqpoint{2.359310in}{1.401816in}}{\pgfqpoint{2.367546in}{1.401816in}}%
\pgfpathclose%
\pgfusepath{stroke,fill}%
\end{pgfscope}%
\begin{pgfscope}%
\pgfpathrectangle{\pgfqpoint{0.100000in}{0.220728in}}{\pgfqpoint{3.696000in}{3.696000in}}%
\pgfusepath{clip}%
\pgfsetbuttcap%
\pgfsetroundjoin%
\definecolor{currentfill}{rgb}{0.121569,0.466667,0.705882}%
\pgfsetfillcolor{currentfill}%
\pgfsetfillopacity{0.997990}%
\pgfsetlinewidth{1.003750pt}%
\definecolor{currentstroke}{rgb}{0.121569,0.466667,0.705882}%
\pgfsetstrokecolor{currentstroke}%
\pgfsetstrokeopacity{0.997990}%
\pgfsetdash{}{0pt}%
\pgfpathmoveto{\pgfqpoint{2.367246in}{1.401335in}}%
\pgfpathcurveto{\pgfqpoint{2.375482in}{1.401335in}}{\pgfqpoint{2.383383in}{1.404607in}}{\pgfqpoint{2.389206in}{1.410431in}}%
\pgfpathcurveto{\pgfqpoint{2.395030in}{1.416255in}}{\pgfqpoint{2.398303in}{1.424155in}}{\pgfqpoint{2.398303in}{1.432391in}}%
\pgfpathcurveto{\pgfqpoint{2.398303in}{1.440627in}}{\pgfqpoint{2.395030in}{1.448527in}}{\pgfqpoint{2.389206in}{1.454351in}}%
\pgfpathcurveto{\pgfqpoint{2.383383in}{1.460175in}}{\pgfqpoint{2.375482in}{1.463448in}}{\pgfqpoint{2.367246in}{1.463448in}}%
\pgfpathcurveto{\pgfqpoint{2.359010in}{1.463448in}}{\pgfqpoint{2.351110in}{1.460175in}}{\pgfqpoint{2.345286in}{1.454351in}}%
\pgfpathcurveto{\pgfqpoint{2.339462in}{1.448527in}}{\pgfqpoint{2.336190in}{1.440627in}}{\pgfqpoint{2.336190in}{1.432391in}}%
\pgfpathcurveto{\pgfqpoint{2.336190in}{1.424155in}}{\pgfqpoint{2.339462in}{1.416255in}}{\pgfqpoint{2.345286in}{1.410431in}}%
\pgfpathcurveto{\pgfqpoint{2.351110in}{1.404607in}}{\pgfqpoint{2.359010in}{1.401335in}}{\pgfqpoint{2.367246in}{1.401335in}}%
\pgfpathclose%
\pgfusepath{stroke,fill}%
\end{pgfscope}%
\begin{pgfscope}%
\pgfpathrectangle{\pgfqpoint{0.100000in}{0.220728in}}{\pgfqpoint{3.696000in}{3.696000in}}%
\pgfusepath{clip}%
\pgfsetbuttcap%
\pgfsetroundjoin%
\definecolor{currentfill}{rgb}{0.121569,0.466667,0.705882}%
\pgfsetfillcolor{currentfill}%
\pgfsetfillopacity{0.998043}%
\pgfsetlinewidth{1.003750pt}%
\definecolor{currentstroke}{rgb}{0.121569,0.466667,0.705882}%
\pgfsetstrokecolor{currentstroke}%
\pgfsetstrokeopacity{0.998043}%
\pgfsetdash{}{0pt}%
\pgfpathmoveto{\pgfqpoint{2.367060in}{1.401088in}}%
\pgfpathcurveto{\pgfqpoint{2.375296in}{1.401088in}}{\pgfqpoint{2.383196in}{1.404361in}}{\pgfqpoint{2.389020in}{1.410184in}}%
\pgfpathcurveto{\pgfqpoint{2.394844in}{1.416008in}}{\pgfqpoint{2.398117in}{1.423908in}}{\pgfqpoint{2.398117in}{1.432145in}}%
\pgfpathcurveto{\pgfqpoint{2.398117in}{1.440381in}}{\pgfqpoint{2.394844in}{1.448281in}}{\pgfqpoint{2.389020in}{1.454105in}}%
\pgfpathcurveto{\pgfqpoint{2.383196in}{1.459929in}}{\pgfqpoint{2.375296in}{1.463201in}}{\pgfqpoint{2.367060in}{1.463201in}}%
\pgfpathcurveto{\pgfqpoint{2.358824in}{1.463201in}}{\pgfqpoint{2.350924in}{1.459929in}}{\pgfqpoint{2.345100in}{1.454105in}}%
\pgfpathcurveto{\pgfqpoint{2.339276in}{1.448281in}}{\pgfqpoint{2.336004in}{1.440381in}}{\pgfqpoint{2.336004in}{1.432145in}}%
\pgfpathcurveto{\pgfqpoint{2.336004in}{1.423908in}}{\pgfqpoint{2.339276in}{1.416008in}}{\pgfqpoint{2.345100in}{1.410184in}}%
\pgfpathcurveto{\pgfqpoint{2.350924in}{1.404361in}}{\pgfqpoint{2.358824in}{1.401088in}}{\pgfqpoint{2.367060in}{1.401088in}}%
\pgfpathclose%
\pgfusepath{stroke,fill}%
\end{pgfscope}%
\begin{pgfscope}%
\pgfpathrectangle{\pgfqpoint{0.100000in}{0.220728in}}{\pgfqpoint{3.696000in}{3.696000in}}%
\pgfusepath{clip}%
\pgfsetbuttcap%
\pgfsetroundjoin%
\definecolor{currentfill}{rgb}{0.121569,0.466667,0.705882}%
\pgfsetfillcolor{currentfill}%
\pgfsetfillopacity{0.998074}%
\pgfsetlinewidth{1.003750pt}%
\definecolor{currentstroke}{rgb}{0.121569,0.466667,0.705882}%
\pgfsetstrokecolor{currentstroke}%
\pgfsetstrokeopacity{0.998074}%
\pgfsetdash{}{0pt}%
\pgfpathmoveto{\pgfqpoint{2.366973in}{1.400938in}}%
\pgfpathcurveto{\pgfqpoint{2.375209in}{1.400938in}}{\pgfqpoint{2.383109in}{1.404210in}}{\pgfqpoint{2.388933in}{1.410034in}}%
\pgfpathcurveto{\pgfqpoint{2.394757in}{1.415858in}}{\pgfqpoint{2.398029in}{1.423758in}}{\pgfqpoint{2.398029in}{1.431994in}}%
\pgfpathcurveto{\pgfqpoint{2.398029in}{1.440230in}}{\pgfqpoint{2.394757in}{1.448131in}}{\pgfqpoint{2.388933in}{1.453954in}}%
\pgfpathcurveto{\pgfqpoint{2.383109in}{1.459778in}}{\pgfqpoint{2.375209in}{1.463051in}}{\pgfqpoint{2.366973in}{1.463051in}}%
\pgfpathcurveto{\pgfqpoint{2.358737in}{1.463051in}}{\pgfqpoint{2.350836in}{1.459778in}}{\pgfqpoint{2.345013in}{1.453954in}}%
\pgfpathcurveto{\pgfqpoint{2.339189in}{1.448131in}}{\pgfqpoint{2.335916in}{1.440230in}}{\pgfqpoint{2.335916in}{1.431994in}}%
\pgfpathcurveto{\pgfqpoint{2.335916in}{1.423758in}}{\pgfqpoint{2.339189in}{1.415858in}}{\pgfqpoint{2.345013in}{1.410034in}}%
\pgfpathcurveto{\pgfqpoint{2.350836in}{1.404210in}}{\pgfqpoint{2.358737in}{1.400938in}}{\pgfqpoint{2.366973in}{1.400938in}}%
\pgfpathclose%
\pgfusepath{stroke,fill}%
\end{pgfscope}%
\begin{pgfscope}%
\pgfpathrectangle{\pgfqpoint{0.100000in}{0.220728in}}{\pgfqpoint{3.696000in}{3.696000in}}%
\pgfusepath{clip}%
\pgfsetbuttcap%
\pgfsetroundjoin%
\definecolor{currentfill}{rgb}{0.121569,0.466667,0.705882}%
\pgfsetfillcolor{currentfill}%
\pgfsetfillopacity{0.998091}%
\pgfsetlinewidth{1.003750pt}%
\definecolor{currentstroke}{rgb}{0.121569,0.466667,0.705882}%
\pgfsetstrokecolor{currentstroke}%
\pgfsetstrokeopacity{0.998091}%
\pgfsetdash{}{0pt}%
\pgfpathmoveto{\pgfqpoint{2.366922in}{1.400858in}}%
\pgfpathcurveto{\pgfqpoint{2.375158in}{1.400858in}}{\pgfqpoint{2.383058in}{1.404131in}}{\pgfqpoint{2.388882in}{1.409954in}}%
\pgfpathcurveto{\pgfqpoint{2.394706in}{1.415778in}}{\pgfqpoint{2.397978in}{1.423678in}}{\pgfqpoint{2.397978in}{1.431915in}}%
\pgfpathcurveto{\pgfqpoint{2.397978in}{1.440151in}}{\pgfqpoint{2.394706in}{1.448051in}}{\pgfqpoint{2.388882in}{1.453875in}}%
\pgfpathcurveto{\pgfqpoint{2.383058in}{1.459699in}}{\pgfqpoint{2.375158in}{1.462971in}}{\pgfqpoint{2.366922in}{1.462971in}}%
\pgfpathcurveto{\pgfqpoint{2.358686in}{1.462971in}}{\pgfqpoint{2.350785in}{1.459699in}}{\pgfqpoint{2.344962in}{1.453875in}}%
\pgfpathcurveto{\pgfqpoint{2.339138in}{1.448051in}}{\pgfqpoint{2.335865in}{1.440151in}}{\pgfqpoint{2.335865in}{1.431915in}}%
\pgfpathcurveto{\pgfqpoint{2.335865in}{1.423678in}}{\pgfqpoint{2.339138in}{1.415778in}}{\pgfqpoint{2.344962in}{1.409954in}}%
\pgfpathcurveto{\pgfqpoint{2.350785in}{1.404131in}}{\pgfqpoint{2.358686in}{1.400858in}}{\pgfqpoint{2.366922in}{1.400858in}}%
\pgfpathclose%
\pgfusepath{stroke,fill}%
\end{pgfscope}%
\begin{pgfscope}%
\pgfpathrectangle{\pgfqpoint{0.100000in}{0.220728in}}{\pgfqpoint{3.696000in}{3.696000in}}%
\pgfusepath{clip}%
\pgfsetbuttcap%
\pgfsetroundjoin%
\definecolor{currentfill}{rgb}{0.121569,0.466667,0.705882}%
\pgfsetfillcolor{currentfill}%
\pgfsetfillopacity{0.998100}%
\pgfsetlinewidth{1.003750pt}%
\definecolor{currentstroke}{rgb}{0.121569,0.466667,0.705882}%
\pgfsetstrokecolor{currentstroke}%
\pgfsetstrokeopacity{0.998100}%
\pgfsetdash{}{0pt}%
\pgfpathmoveto{\pgfqpoint{2.366892in}{1.400813in}}%
\pgfpathcurveto{\pgfqpoint{2.375129in}{1.400813in}}{\pgfqpoint{2.383029in}{1.404086in}}{\pgfqpoint{2.388853in}{1.409910in}}%
\pgfpathcurveto{\pgfqpoint{2.394677in}{1.415734in}}{\pgfqpoint{2.397949in}{1.423634in}}{\pgfqpoint{2.397949in}{1.431870in}}%
\pgfpathcurveto{\pgfqpoint{2.397949in}{1.440106in}}{\pgfqpoint{2.394677in}{1.448006in}}{\pgfqpoint{2.388853in}{1.453830in}}%
\pgfpathcurveto{\pgfqpoint{2.383029in}{1.459654in}}{\pgfqpoint{2.375129in}{1.462926in}}{\pgfqpoint{2.366892in}{1.462926in}}%
\pgfpathcurveto{\pgfqpoint{2.358656in}{1.462926in}}{\pgfqpoint{2.350756in}{1.459654in}}{\pgfqpoint{2.344932in}{1.453830in}}%
\pgfpathcurveto{\pgfqpoint{2.339108in}{1.448006in}}{\pgfqpoint{2.335836in}{1.440106in}}{\pgfqpoint{2.335836in}{1.431870in}}%
\pgfpathcurveto{\pgfqpoint{2.335836in}{1.423634in}}{\pgfqpoint{2.339108in}{1.415734in}}{\pgfqpoint{2.344932in}{1.409910in}}%
\pgfpathcurveto{\pgfqpoint{2.350756in}{1.404086in}}{\pgfqpoint{2.358656in}{1.400813in}}{\pgfqpoint{2.366892in}{1.400813in}}%
\pgfpathclose%
\pgfusepath{stroke,fill}%
\end{pgfscope}%
\begin{pgfscope}%
\pgfpathrectangle{\pgfqpoint{0.100000in}{0.220728in}}{\pgfqpoint{3.696000in}{3.696000in}}%
\pgfusepath{clip}%
\pgfsetbuttcap%
\pgfsetroundjoin%
\definecolor{currentfill}{rgb}{0.121569,0.466667,0.705882}%
\pgfsetfillcolor{currentfill}%
\pgfsetfillopacity{0.998104}%
\pgfsetlinewidth{1.003750pt}%
\definecolor{currentstroke}{rgb}{0.121569,0.466667,0.705882}%
\pgfsetstrokecolor{currentstroke}%
\pgfsetstrokeopacity{0.998104}%
\pgfsetdash{}{0pt}%
\pgfpathmoveto{\pgfqpoint{2.366877in}{1.400789in}}%
\pgfpathcurveto{\pgfqpoint{2.375113in}{1.400789in}}{\pgfqpoint{2.383013in}{1.404061in}}{\pgfqpoint{2.388837in}{1.409885in}}%
\pgfpathcurveto{\pgfqpoint{2.394661in}{1.415709in}}{\pgfqpoint{2.397933in}{1.423609in}}{\pgfqpoint{2.397933in}{1.431845in}}%
\pgfpathcurveto{\pgfqpoint{2.397933in}{1.440082in}}{\pgfqpoint{2.394661in}{1.447982in}}{\pgfqpoint{2.388837in}{1.453806in}}%
\pgfpathcurveto{\pgfqpoint{2.383013in}{1.459630in}}{\pgfqpoint{2.375113in}{1.462902in}}{\pgfqpoint{2.366877in}{1.462902in}}%
\pgfpathcurveto{\pgfqpoint{2.358640in}{1.462902in}}{\pgfqpoint{2.350740in}{1.459630in}}{\pgfqpoint{2.344916in}{1.453806in}}%
\pgfpathcurveto{\pgfqpoint{2.339092in}{1.447982in}}{\pgfqpoint{2.335820in}{1.440082in}}{\pgfqpoint{2.335820in}{1.431845in}}%
\pgfpathcurveto{\pgfqpoint{2.335820in}{1.423609in}}{\pgfqpoint{2.339092in}{1.415709in}}{\pgfqpoint{2.344916in}{1.409885in}}%
\pgfpathcurveto{\pgfqpoint{2.350740in}{1.404061in}}{\pgfqpoint{2.358640in}{1.400789in}}{\pgfqpoint{2.366877in}{1.400789in}}%
\pgfpathclose%
\pgfusepath{stroke,fill}%
\end{pgfscope}%
\begin{pgfscope}%
\pgfpathrectangle{\pgfqpoint{0.100000in}{0.220728in}}{\pgfqpoint{3.696000in}{3.696000in}}%
\pgfusepath{clip}%
\pgfsetbuttcap%
\pgfsetroundjoin%
\definecolor{currentfill}{rgb}{0.121569,0.466667,0.705882}%
\pgfsetfillcolor{currentfill}%
\pgfsetfillopacity{0.998107}%
\pgfsetlinewidth{1.003750pt}%
\definecolor{currentstroke}{rgb}{0.121569,0.466667,0.705882}%
\pgfsetstrokecolor{currentstroke}%
\pgfsetstrokeopacity{0.998107}%
\pgfsetdash{}{0pt}%
\pgfpathmoveto{\pgfqpoint{2.366868in}{1.400775in}}%
\pgfpathcurveto{\pgfqpoint{2.375104in}{1.400775in}}{\pgfqpoint{2.383004in}{1.404047in}}{\pgfqpoint{2.388828in}{1.409871in}}%
\pgfpathcurveto{\pgfqpoint{2.394652in}{1.415695in}}{\pgfqpoint{2.397924in}{1.423595in}}{\pgfqpoint{2.397924in}{1.431831in}}%
\pgfpathcurveto{\pgfqpoint{2.397924in}{1.440068in}}{\pgfqpoint{2.394652in}{1.447968in}}{\pgfqpoint{2.388828in}{1.453792in}}%
\pgfpathcurveto{\pgfqpoint{2.383004in}{1.459616in}}{\pgfqpoint{2.375104in}{1.462888in}}{\pgfqpoint{2.366868in}{1.462888in}}%
\pgfpathcurveto{\pgfqpoint{2.358631in}{1.462888in}}{\pgfqpoint{2.350731in}{1.459616in}}{\pgfqpoint{2.344907in}{1.453792in}}%
\pgfpathcurveto{\pgfqpoint{2.339083in}{1.447968in}}{\pgfqpoint{2.335811in}{1.440068in}}{\pgfqpoint{2.335811in}{1.431831in}}%
\pgfpathcurveto{\pgfqpoint{2.335811in}{1.423595in}}{\pgfqpoint{2.339083in}{1.415695in}}{\pgfqpoint{2.344907in}{1.409871in}}%
\pgfpathcurveto{\pgfqpoint{2.350731in}{1.404047in}}{\pgfqpoint{2.358631in}{1.400775in}}{\pgfqpoint{2.366868in}{1.400775in}}%
\pgfpathclose%
\pgfusepath{stroke,fill}%
\end{pgfscope}%
\begin{pgfscope}%
\pgfpathrectangle{\pgfqpoint{0.100000in}{0.220728in}}{\pgfqpoint{3.696000in}{3.696000in}}%
\pgfusepath{clip}%
\pgfsetbuttcap%
\pgfsetroundjoin%
\definecolor{currentfill}{rgb}{0.121569,0.466667,0.705882}%
\pgfsetfillcolor{currentfill}%
\pgfsetfillopacity{0.998108}%
\pgfsetlinewidth{1.003750pt}%
\definecolor{currentstroke}{rgb}{0.121569,0.466667,0.705882}%
\pgfsetstrokecolor{currentstroke}%
\pgfsetstrokeopacity{0.998108}%
\pgfsetdash{}{0pt}%
\pgfpathmoveto{\pgfqpoint{2.366863in}{1.400767in}}%
\pgfpathcurveto{\pgfqpoint{2.375099in}{1.400767in}}{\pgfqpoint{2.382999in}{1.404040in}}{\pgfqpoint{2.388823in}{1.409864in}}%
\pgfpathcurveto{\pgfqpoint{2.394647in}{1.415688in}}{\pgfqpoint{2.397919in}{1.423588in}}{\pgfqpoint{2.397919in}{1.431824in}}%
\pgfpathcurveto{\pgfqpoint{2.397919in}{1.440060in}}{\pgfqpoint{2.394647in}{1.447960in}}{\pgfqpoint{2.388823in}{1.453784in}}%
\pgfpathcurveto{\pgfqpoint{2.382999in}{1.459608in}}{\pgfqpoint{2.375099in}{1.462880in}}{\pgfqpoint{2.366863in}{1.462880in}}%
\pgfpathcurveto{\pgfqpoint{2.358626in}{1.462880in}}{\pgfqpoint{2.350726in}{1.459608in}}{\pgfqpoint{2.344902in}{1.453784in}}%
\pgfpathcurveto{\pgfqpoint{2.339078in}{1.447960in}}{\pgfqpoint{2.335806in}{1.440060in}}{\pgfqpoint{2.335806in}{1.431824in}}%
\pgfpathcurveto{\pgfqpoint{2.335806in}{1.423588in}}{\pgfqpoint{2.339078in}{1.415688in}}{\pgfqpoint{2.344902in}{1.409864in}}%
\pgfpathcurveto{\pgfqpoint{2.350726in}{1.404040in}}{\pgfqpoint{2.358626in}{1.400767in}}{\pgfqpoint{2.366863in}{1.400767in}}%
\pgfpathclose%
\pgfusepath{stroke,fill}%
\end{pgfscope}%
\begin{pgfscope}%
\pgfpathrectangle{\pgfqpoint{0.100000in}{0.220728in}}{\pgfqpoint{3.696000in}{3.696000in}}%
\pgfusepath{clip}%
\pgfsetbuttcap%
\pgfsetroundjoin%
\definecolor{currentfill}{rgb}{0.121569,0.466667,0.705882}%
\pgfsetfillcolor{currentfill}%
\pgfsetfillopacity{0.998470}%
\pgfsetlinewidth{1.003750pt}%
\definecolor{currentstroke}{rgb}{0.121569,0.466667,0.705882}%
\pgfsetstrokecolor{currentstroke}%
\pgfsetstrokeopacity{0.998470}%
\pgfsetdash{}{0pt}%
\pgfpathmoveto{\pgfqpoint{2.365150in}{1.398716in}}%
\pgfpathcurveto{\pgfqpoint{2.373386in}{1.398716in}}{\pgfqpoint{2.381286in}{1.401988in}}{\pgfqpoint{2.387110in}{1.407812in}}%
\pgfpathcurveto{\pgfqpoint{2.392934in}{1.413636in}}{\pgfqpoint{2.396207in}{1.421536in}}{\pgfqpoint{2.396207in}{1.429773in}}%
\pgfpathcurveto{\pgfqpoint{2.396207in}{1.438009in}}{\pgfqpoint{2.392934in}{1.445909in}}{\pgfqpoint{2.387110in}{1.451733in}}%
\pgfpathcurveto{\pgfqpoint{2.381286in}{1.457557in}}{\pgfqpoint{2.373386in}{1.460829in}}{\pgfqpoint{2.365150in}{1.460829in}}%
\pgfpathcurveto{\pgfqpoint{2.356914in}{1.460829in}}{\pgfqpoint{2.349014in}{1.457557in}}{\pgfqpoint{2.343190in}{1.451733in}}%
\pgfpathcurveto{\pgfqpoint{2.337366in}{1.445909in}}{\pgfqpoint{2.334094in}{1.438009in}}{\pgfqpoint{2.334094in}{1.429773in}}%
\pgfpathcurveto{\pgfqpoint{2.334094in}{1.421536in}}{\pgfqpoint{2.337366in}{1.413636in}}{\pgfqpoint{2.343190in}{1.407812in}}%
\pgfpathcurveto{\pgfqpoint{2.349014in}{1.401988in}}{\pgfqpoint{2.356914in}{1.398716in}}{\pgfqpoint{2.365150in}{1.398716in}}%
\pgfpathclose%
\pgfusepath{stroke,fill}%
\end{pgfscope}%
\begin{pgfscope}%
\pgfpathrectangle{\pgfqpoint{0.100000in}{0.220728in}}{\pgfqpoint{3.696000in}{3.696000in}}%
\pgfusepath{clip}%
\pgfsetbuttcap%
\pgfsetroundjoin%
\definecolor{currentfill}{rgb}{0.121569,0.466667,0.705882}%
\pgfsetfillcolor{currentfill}%
\pgfsetfillopacity{0.999070}%
\pgfsetlinewidth{1.003750pt}%
\definecolor{currentstroke}{rgb}{0.121569,0.466667,0.705882}%
\pgfsetstrokecolor{currentstroke}%
\pgfsetstrokeopacity{0.999070}%
\pgfsetdash{}{0pt}%
\pgfpathmoveto{\pgfqpoint{2.361458in}{1.394390in}}%
\pgfpathcurveto{\pgfqpoint{2.369694in}{1.394390in}}{\pgfqpoint{2.377594in}{1.397663in}}{\pgfqpoint{2.383418in}{1.403487in}}%
\pgfpathcurveto{\pgfqpoint{2.389242in}{1.409311in}}{\pgfqpoint{2.392514in}{1.417211in}}{\pgfqpoint{2.392514in}{1.425447in}}%
\pgfpathcurveto{\pgfqpoint{2.392514in}{1.433683in}}{\pgfqpoint{2.389242in}{1.441583in}}{\pgfqpoint{2.383418in}{1.447407in}}%
\pgfpathcurveto{\pgfqpoint{2.377594in}{1.453231in}}{\pgfqpoint{2.369694in}{1.456503in}}{\pgfqpoint{2.361458in}{1.456503in}}%
\pgfpathcurveto{\pgfqpoint{2.353221in}{1.456503in}}{\pgfqpoint{2.345321in}{1.453231in}}{\pgfqpoint{2.339498in}{1.447407in}}%
\pgfpathcurveto{\pgfqpoint{2.333674in}{1.441583in}}{\pgfqpoint{2.330401in}{1.433683in}}{\pgfqpoint{2.330401in}{1.425447in}}%
\pgfpathcurveto{\pgfqpoint{2.330401in}{1.417211in}}{\pgfqpoint{2.333674in}{1.409311in}}{\pgfqpoint{2.339498in}{1.403487in}}%
\pgfpathcurveto{\pgfqpoint{2.345321in}{1.397663in}}{\pgfqpoint{2.353221in}{1.394390in}}{\pgfqpoint{2.361458in}{1.394390in}}%
\pgfpathclose%
\pgfusepath{stroke,fill}%
\end{pgfscope}%
\begin{pgfscope}%
\pgfpathrectangle{\pgfqpoint{0.100000in}{0.220728in}}{\pgfqpoint{3.696000in}{3.696000in}}%
\pgfusepath{clip}%
\pgfsetbuttcap%
\pgfsetroundjoin%
\definecolor{currentfill}{rgb}{0.121569,0.466667,0.705882}%
\pgfsetfillcolor{currentfill}%
\pgfsetfillopacity{0.999379}%
\pgfsetlinewidth{1.003750pt}%
\definecolor{currentstroke}{rgb}{0.121569,0.466667,0.705882}%
\pgfsetstrokecolor{currentstroke}%
\pgfsetstrokeopacity{0.999379}%
\pgfsetdash{}{0pt}%
\pgfpathmoveto{\pgfqpoint{2.339696in}{1.384634in}}%
\pgfpathcurveto{\pgfqpoint{2.347933in}{1.384634in}}{\pgfqpoint{2.355833in}{1.387907in}}{\pgfqpoint{2.361657in}{1.393731in}}%
\pgfpathcurveto{\pgfqpoint{2.367480in}{1.399555in}}{\pgfqpoint{2.370753in}{1.407455in}}{\pgfqpoint{2.370753in}{1.415691in}}%
\pgfpathcurveto{\pgfqpoint{2.370753in}{1.423927in}}{\pgfqpoint{2.367480in}{1.431827in}}{\pgfqpoint{2.361657in}{1.437651in}}%
\pgfpathcurveto{\pgfqpoint{2.355833in}{1.443475in}}{\pgfqpoint{2.347933in}{1.446747in}}{\pgfqpoint{2.339696in}{1.446747in}}%
\pgfpathcurveto{\pgfqpoint{2.331460in}{1.446747in}}{\pgfqpoint{2.323560in}{1.443475in}}{\pgfqpoint{2.317736in}{1.437651in}}%
\pgfpathcurveto{\pgfqpoint{2.311912in}{1.431827in}}{\pgfqpoint{2.308640in}{1.423927in}}{\pgfqpoint{2.308640in}{1.415691in}}%
\pgfpathcurveto{\pgfqpoint{2.308640in}{1.407455in}}{\pgfqpoint{2.311912in}{1.399555in}}{\pgfqpoint{2.317736in}{1.393731in}}%
\pgfpathcurveto{\pgfqpoint{2.323560in}{1.387907in}}{\pgfqpoint{2.331460in}{1.384634in}}{\pgfqpoint{2.339696in}{1.384634in}}%
\pgfpathclose%
\pgfusepath{stroke,fill}%
\end{pgfscope}%
\begin{pgfscope}%
\pgfpathrectangle{\pgfqpoint{0.100000in}{0.220728in}}{\pgfqpoint{3.696000in}{3.696000in}}%
\pgfusepath{clip}%
\pgfsetbuttcap%
\pgfsetroundjoin%
\definecolor{currentfill}{rgb}{0.121569,0.466667,0.705882}%
\pgfsetfillcolor{currentfill}%
\pgfsetfillopacity{0.999917}%
\pgfsetlinewidth{1.003750pt}%
\definecolor{currentstroke}{rgb}{0.121569,0.466667,0.705882}%
\pgfsetstrokecolor{currentstroke}%
\pgfsetstrokeopacity{0.999917}%
\pgfsetdash{}{0pt}%
\pgfpathmoveto{\pgfqpoint{2.354383in}{1.388350in}}%
\pgfpathcurveto{\pgfqpoint{2.362619in}{1.388350in}}{\pgfqpoint{2.370519in}{1.391622in}}{\pgfqpoint{2.376343in}{1.397446in}}%
\pgfpathcurveto{\pgfqpoint{2.382167in}{1.403270in}}{\pgfqpoint{2.385439in}{1.411170in}}{\pgfqpoint{2.385439in}{1.419406in}}%
\pgfpathcurveto{\pgfqpoint{2.385439in}{1.427642in}}{\pgfqpoint{2.382167in}{1.435542in}}{\pgfqpoint{2.376343in}{1.441366in}}%
\pgfpathcurveto{\pgfqpoint{2.370519in}{1.447190in}}{\pgfqpoint{2.362619in}{1.450463in}}{\pgfqpoint{2.354383in}{1.450463in}}%
\pgfpathcurveto{\pgfqpoint{2.346147in}{1.450463in}}{\pgfqpoint{2.338246in}{1.447190in}}{\pgfqpoint{2.332423in}{1.441366in}}%
\pgfpathcurveto{\pgfqpoint{2.326599in}{1.435542in}}{\pgfqpoint{2.323326in}{1.427642in}}{\pgfqpoint{2.323326in}{1.419406in}}%
\pgfpathcurveto{\pgfqpoint{2.323326in}{1.411170in}}{\pgfqpoint{2.326599in}{1.403270in}}{\pgfqpoint{2.332423in}{1.397446in}}%
\pgfpathcurveto{\pgfqpoint{2.338246in}{1.391622in}}{\pgfqpoint{2.346147in}{1.388350in}}{\pgfqpoint{2.354383in}{1.388350in}}%
\pgfpathclose%
\pgfusepath{stroke,fill}%
\end{pgfscope}%
\begin{pgfscope}%
\pgfpathrectangle{\pgfqpoint{0.100000in}{0.220728in}}{\pgfqpoint{3.696000in}{3.696000in}}%
\pgfusepath{clip}%
\pgfsetbuttcap%
\pgfsetroundjoin%
\definecolor{currentfill}{rgb}{0.121569,0.466667,0.705882}%
\pgfsetfillcolor{currentfill}%
\pgfsetlinewidth{1.003750pt}%
\definecolor{currentstroke}{rgb}{0.121569,0.466667,0.705882}%
\pgfsetstrokecolor{currentstroke}%
\pgfsetdash{}{0pt}%
\pgfpathmoveto{\pgfqpoint{2.349891in}{1.385917in}}%
\pgfpathcurveto{\pgfqpoint{2.358127in}{1.385917in}}{\pgfqpoint{2.366027in}{1.389189in}}{\pgfqpoint{2.371851in}{1.395013in}}%
\pgfpathcurveto{\pgfqpoint{2.377675in}{1.400837in}}{\pgfqpoint{2.380947in}{1.408737in}}{\pgfqpoint{2.380947in}{1.416973in}}%
\pgfpathcurveto{\pgfqpoint{2.380947in}{1.425209in}}{\pgfqpoint{2.377675in}{1.433109in}}{\pgfqpoint{2.371851in}{1.438933in}}%
\pgfpathcurveto{\pgfqpoint{2.366027in}{1.444757in}}{\pgfqpoint{2.358127in}{1.448030in}}{\pgfqpoint{2.349891in}{1.448030in}}%
\pgfpathcurveto{\pgfqpoint{2.341655in}{1.448030in}}{\pgfqpoint{2.333755in}{1.444757in}}{\pgfqpoint{2.327931in}{1.438933in}}%
\pgfpathcurveto{\pgfqpoint{2.322107in}{1.433109in}}{\pgfqpoint{2.318834in}{1.425209in}}{\pgfqpoint{2.318834in}{1.416973in}}%
\pgfpathcurveto{\pgfqpoint{2.318834in}{1.408737in}}{\pgfqpoint{2.322107in}{1.400837in}}{\pgfqpoint{2.327931in}{1.395013in}}%
\pgfpathcurveto{\pgfqpoint{2.333755in}{1.389189in}}{\pgfqpoint{2.341655in}{1.385917in}}{\pgfqpoint{2.349891in}{1.385917in}}%
\pgfpathclose%
\pgfusepath{stroke,fill}%
\end{pgfscope}%
\begin{pgfscope}%
\definecolor{textcolor}{rgb}{0.000000,0.000000,0.000000}%
\pgfsetstrokecolor{textcolor}%
\pgfsetfillcolor{textcolor}%
\pgftext[x=1.948000in,y=4.000061in,,base]{\color{textcolor}\sffamily\fontsize{12.000000}{14.400000}\selectfont FLAE}%
\end{pgfscope}%
\begin{pgfscope}%
\pgfpathrectangle{\pgfqpoint{0.100000in}{0.220728in}}{\pgfqpoint{3.696000in}{3.696000in}}%
\pgfusepath{clip}%
\pgfsetbuttcap%
\pgfsetroundjoin%
\definecolor{currentfill}{rgb}{1.000000,0.498039,0.054902}%
\pgfsetfillcolor{currentfill}%
\pgfsetfillopacity{0.300000}%
\pgfsetlinewidth{1.003750pt}%
\definecolor{currentstroke}{rgb}{1.000000,0.498039,0.054902}%
\pgfsetstrokecolor{currentstroke}%
\pgfsetstrokeopacity{0.300000}%
\pgfsetdash{}{0pt}%
\pgfpathmoveto{\pgfqpoint{1.654126in}{3.126024in}}%
\pgfpathcurveto{\pgfqpoint{1.662362in}{3.126024in}}{\pgfqpoint{1.670263in}{3.129297in}}{\pgfqpoint{1.676086in}{3.135121in}}%
\pgfpathcurveto{\pgfqpoint{1.681910in}{3.140945in}}{\pgfqpoint{1.685183in}{3.148845in}}{\pgfqpoint{1.685183in}{3.157081in}}%
\pgfpathcurveto{\pgfqpoint{1.685183in}{3.165317in}}{\pgfqpoint{1.681910in}{3.173217in}}{\pgfqpoint{1.676086in}{3.179041in}}%
\pgfpathcurveto{\pgfqpoint{1.670263in}{3.184865in}}{\pgfqpoint{1.662362in}{3.188137in}}{\pgfqpoint{1.654126in}{3.188137in}}%
\pgfpathcurveto{\pgfqpoint{1.645890in}{3.188137in}}{\pgfqpoint{1.637990in}{3.184865in}}{\pgfqpoint{1.632166in}{3.179041in}}%
\pgfpathcurveto{\pgfqpoint{1.626342in}{3.173217in}}{\pgfqpoint{1.623070in}{3.165317in}}{\pgfqpoint{1.623070in}{3.157081in}}%
\pgfpathcurveto{\pgfqpoint{1.623070in}{3.148845in}}{\pgfqpoint{1.626342in}{3.140945in}}{\pgfqpoint{1.632166in}{3.135121in}}%
\pgfpathcurveto{\pgfqpoint{1.637990in}{3.129297in}}{\pgfqpoint{1.645890in}{3.126024in}}{\pgfqpoint{1.654126in}{3.126024in}}%
\pgfpathclose%
\pgfusepath{stroke,fill}%
\end{pgfscope}%
\begin{pgfscope}%
\pgfpathrectangle{\pgfqpoint{0.100000in}{0.220728in}}{\pgfqpoint{3.696000in}{3.696000in}}%
\pgfusepath{clip}%
\pgfsetbuttcap%
\pgfsetroundjoin%
\definecolor{currentfill}{rgb}{1.000000,0.498039,0.054902}%
\pgfsetfillcolor{currentfill}%
\pgfsetfillopacity{0.587662}%
\pgfsetlinewidth{1.003750pt}%
\definecolor{currentstroke}{rgb}{1.000000,0.498039,0.054902}%
\pgfsetstrokecolor{currentstroke}%
\pgfsetstrokeopacity{0.587662}%
\pgfsetdash{}{0pt}%
\pgfpathmoveto{\pgfqpoint{0.854595in}{1.389376in}}%
\pgfpathcurveto{\pgfqpoint{0.862831in}{1.389376in}}{\pgfqpoint{0.870732in}{1.392648in}}{\pgfqpoint{0.876555in}{1.398472in}}%
\pgfpathcurveto{\pgfqpoint{0.882379in}{1.404296in}}{\pgfqpoint{0.885652in}{1.412196in}}{\pgfqpoint{0.885652in}{1.420432in}}%
\pgfpathcurveto{\pgfqpoint{0.885652in}{1.428669in}}{\pgfqpoint{0.882379in}{1.436569in}}{\pgfqpoint{0.876555in}{1.442393in}}%
\pgfpathcurveto{\pgfqpoint{0.870732in}{1.448217in}}{\pgfqpoint{0.862831in}{1.451489in}}{\pgfqpoint{0.854595in}{1.451489in}}%
\pgfpathcurveto{\pgfqpoint{0.846359in}{1.451489in}}{\pgfqpoint{0.838459in}{1.448217in}}{\pgfqpoint{0.832635in}{1.442393in}}%
\pgfpathcurveto{\pgfqpoint{0.826811in}{1.436569in}}{\pgfqpoint{0.823539in}{1.428669in}}{\pgfqpoint{0.823539in}{1.420432in}}%
\pgfpathcurveto{\pgfqpoint{0.823539in}{1.412196in}}{\pgfqpoint{0.826811in}{1.404296in}}{\pgfqpoint{0.832635in}{1.398472in}}%
\pgfpathcurveto{\pgfqpoint{0.838459in}{1.392648in}}{\pgfqpoint{0.846359in}{1.389376in}}{\pgfqpoint{0.854595in}{1.389376in}}%
\pgfpathclose%
\pgfusepath{stroke,fill}%
\end{pgfscope}%
\begin{pgfscope}%
\pgfpathrectangle{\pgfqpoint{0.100000in}{0.220728in}}{\pgfqpoint{3.696000in}{3.696000in}}%
\pgfusepath{clip}%
\pgfsetbuttcap%
\pgfsetroundjoin%
\definecolor{currentfill}{rgb}{1.000000,0.498039,0.054902}%
\pgfsetfillcolor{currentfill}%
\pgfsetfillopacity{0.695101}%
\pgfsetlinewidth{1.003750pt}%
\definecolor{currentstroke}{rgb}{1.000000,0.498039,0.054902}%
\pgfsetstrokecolor{currentstroke}%
\pgfsetstrokeopacity{0.695101}%
\pgfsetdash{}{0pt}%
\pgfpathmoveto{\pgfqpoint{0.754893in}{2.319267in}}%
\pgfpathcurveto{\pgfqpoint{0.763129in}{2.319267in}}{\pgfqpoint{0.771029in}{2.322540in}}{\pgfqpoint{0.776853in}{2.328364in}}%
\pgfpathcurveto{\pgfqpoint{0.782677in}{2.334188in}}{\pgfqpoint{0.785949in}{2.342088in}}{\pgfqpoint{0.785949in}{2.350324in}}%
\pgfpathcurveto{\pgfqpoint{0.785949in}{2.358560in}}{\pgfqpoint{0.782677in}{2.366460in}}{\pgfqpoint{0.776853in}{2.372284in}}%
\pgfpathcurveto{\pgfqpoint{0.771029in}{2.378108in}}{\pgfqpoint{0.763129in}{2.381380in}}{\pgfqpoint{0.754893in}{2.381380in}}%
\pgfpathcurveto{\pgfqpoint{0.746657in}{2.381380in}}{\pgfqpoint{0.738757in}{2.378108in}}{\pgfqpoint{0.732933in}{2.372284in}}%
\pgfpathcurveto{\pgfqpoint{0.727109in}{2.366460in}}{\pgfqpoint{0.723836in}{2.358560in}}{\pgfqpoint{0.723836in}{2.350324in}}%
\pgfpathcurveto{\pgfqpoint{0.723836in}{2.342088in}}{\pgfqpoint{0.727109in}{2.334188in}}{\pgfqpoint{0.732933in}{2.328364in}}%
\pgfpathcurveto{\pgfqpoint{0.738757in}{2.322540in}}{\pgfqpoint{0.746657in}{2.319267in}}{\pgfqpoint{0.754893in}{2.319267in}}%
\pgfpathclose%
\pgfusepath{stroke,fill}%
\end{pgfscope}%
\begin{pgfscope}%
\pgfpathrectangle{\pgfqpoint{0.100000in}{0.220728in}}{\pgfqpoint{3.696000in}{3.696000in}}%
\pgfusepath{clip}%
\pgfsetbuttcap%
\pgfsetroundjoin%
\definecolor{currentfill}{rgb}{1.000000,0.498039,0.054902}%
\pgfsetfillcolor{currentfill}%
\pgfsetfillopacity{0.697636}%
\pgfsetlinewidth{1.003750pt}%
\definecolor{currentstroke}{rgb}{1.000000,0.498039,0.054902}%
\pgfsetstrokecolor{currentstroke}%
\pgfsetstrokeopacity{0.697636}%
\pgfsetdash{}{0pt}%
\pgfpathmoveto{\pgfqpoint{3.311522in}{2.876114in}}%
\pgfpathcurveto{\pgfqpoint{3.319758in}{2.876114in}}{\pgfqpoint{3.327658in}{2.879386in}}{\pgfqpoint{3.333482in}{2.885210in}}%
\pgfpathcurveto{\pgfqpoint{3.339306in}{2.891034in}}{\pgfqpoint{3.342578in}{2.898934in}}{\pgfqpoint{3.342578in}{2.907170in}}%
\pgfpathcurveto{\pgfqpoint{3.342578in}{2.915406in}}{\pgfqpoint{3.339306in}{2.923307in}}{\pgfqpoint{3.333482in}{2.929130in}}%
\pgfpathcurveto{\pgfqpoint{3.327658in}{2.934954in}}{\pgfqpoint{3.319758in}{2.938227in}}{\pgfqpoint{3.311522in}{2.938227in}}%
\pgfpathcurveto{\pgfqpoint{3.303285in}{2.938227in}}{\pgfqpoint{3.295385in}{2.934954in}}{\pgfqpoint{3.289561in}{2.929130in}}%
\pgfpathcurveto{\pgfqpoint{3.283738in}{2.923307in}}{\pgfqpoint{3.280465in}{2.915406in}}{\pgfqpoint{3.280465in}{2.907170in}}%
\pgfpathcurveto{\pgfqpoint{3.280465in}{2.898934in}}{\pgfqpoint{3.283738in}{2.891034in}}{\pgfqpoint{3.289561in}{2.885210in}}%
\pgfpathcurveto{\pgfqpoint{3.295385in}{2.879386in}}{\pgfqpoint{3.303285in}{2.876114in}}{\pgfqpoint{3.311522in}{2.876114in}}%
\pgfpathclose%
\pgfusepath{stroke,fill}%
\end{pgfscope}%
\begin{pgfscope}%
\pgfpathrectangle{\pgfqpoint{0.100000in}{0.220728in}}{\pgfqpoint{3.696000in}{3.696000in}}%
\pgfusepath{clip}%
\pgfsetbuttcap%
\pgfsetroundjoin%
\definecolor{currentfill}{rgb}{1.000000,0.498039,0.054902}%
\pgfsetfillcolor{currentfill}%
\pgfsetlinewidth{1.003750pt}%
\definecolor{currentstroke}{rgb}{1.000000,0.498039,0.054902}%
\pgfsetstrokecolor{currentstroke}%
\pgfsetdash{}{0pt}%
\pgfpathmoveto{\pgfqpoint{2.349891in}{1.385917in}}%
\pgfpathcurveto{\pgfqpoint{2.358127in}{1.385917in}}{\pgfqpoint{2.366027in}{1.389189in}}{\pgfqpoint{2.371851in}{1.395013in}}%
\pgfpathcurveto{\pgfqpoint{2.377675in}{1.400837in}}{\pgfqpoint{2.380947in}{1.408737in}}{\pgfqpoint{2.380947in}{1.416973in}}%
\pgfpathcurveto{\pgfqpoint{2.380947in}{1.425209in}}{\pgfqpoint{2.377675in}{1.433109in}}{\pgfqpoint{2.371851in}{1.438933in}}%
\pgfpathcurveto{\pgfqpoint{2.366027in}{1.444757in}}{\pgfqpoint{2.358127in}{1.448030in}}{\pgfqpoint{2.349891in}{1.448030in}}%
\pgfpathcurveto{\pgfqpoint{2.341655in}{1.448030in}}{\pgfqpoint{2.333755in}{1.444757in}}{\pgfqpoint{2.327931in}{1.438933in}}%
\pgfpathcurveto{\pgfqpoint{2.322107in}{1.433109in}}{\pgfqpoint{2.318834in}{1.425209in}}{\pgfqpoint{2.318834in}{1.416973in}}%
\pgfpathcurveto{\pgfqpoint{2.318834in}{1.408737in}}{\pgfqpoint{2.322107in}{1.400837in}}{\pgfqpoint{2.327931in}{1.395013in}}%
\pgfpathcurveto{\pgfqpoint{2.333755in}{1.389189in}}{\pgfqpoint{2.341655in}{1.385917in}}{\pgfqpoint{2.349891in}{1.385917in}}%
\pgfpathclose%
\pgfusepath{stroke,fill}%
\end{pgfscope}%
\begin{pgfscope}%
\pgfsetbuttcap%
\pgfsetmiterjoin%
\definecolor{currentfill}{rgb}{1.000000,1.000000,1.000000}%
\pgfsetfillcolor{currentfill}%
\pgfsetfillopacity{0.800000}%
\pgfsetlinewidth{1.003750pt}%
\definecolor{currentstroke}{rgb}{0.800000,0.800000,0.800000}%
\pgfsetstrokecolor{currentstroke}%
\pgfsetstrokeopacity{0.800000}%
\pgfsetdash{}{0pt}%
\pgfpathmoveto{\pgfqpoint{1.958421in}{3.194045in}}%
\pgfpathlineto{\pgfqpoint{3.698778in}{3.194045in}}%
\pgfpathquadraticcurveto{\pgfqpoint{3.726556in}{3.194045in}}{\pgfqpoint{3.726556in}{3.221823in}}%
\pgfpathlineto{\pgfqpoint{3.726556in}{3.819506in}}%
\pgfpathquadraticcurveto{\pgfqpoint{3.726556in}{3.847284in}}{\pgfqpoint{3.698778in}{3.847284in}}%
\pgfpathlineto{\pgfqpoint{1.958421in}{3.847284in}}%
\pgfpathquadraticcurveto{\pgfqpoint{1.930644in}{3.847284in}}{\pgfqpoint{1.930644in}{3.819506in}}%
\pgfpathlineto{\pgfqpoint{1.930644in}{3.221823in}}%
\pgfpathquadraticcurveto{\pgfqpoint{1.930644in}{3.194045in}}{\pgfqpoint{1.958421in}{3.194045in}}%
\pgfpathclose%
\pgfusepath{stroke,fill}%
\end{pgfscope}%
\begin{pgfscope}%
\pgfsetrectcap%
\pgfsetroundjoin%
\pgfsetlinewidth{1.505625pt}%
\definecolor{currentstroke}{rgb}{0.121569,0.466667,0.705882}%
\pgfsetstrokecolor{currentstroke}%
\pgfsetdash{}{0pt}%
\pgfpathmoveto{\pgfqpoint{1.986199in}{3.734816in}}%
\pgfpathlineto{\pgfqpoint{2.263977in}{3.734816in}}%
\pgfusepath{stroke}%
\end{pgfscope}%
\begin{pgfscope}%
\definecolor{textcolor}{rgb}{0.000000,0.000000,0.000000}%
\pgfsetstrokecolor{textcolor}%
\pgfsetfillcolor{textcolor}%
\pgftext[x=2.375088in,y=3.686205in,left,base]{\color{textcolor}\sffamily\fontsize{10.000000}{12.000000}\selectfont Ground truth}%
\end{pgfscope}%
\begin{pgfscope}%
\pgfsetbuttcap%
\pgfsetroundjoin%
\definecolor{currentfill}{rgb}{0.121569,0.466667,0.705882}%
\pgfsetfillcolor{currentfill}%
\pgfsetlinewidth{1.003750pt}%
\definecolor{currentstroke}{rgb}{0.121569,0.466667,0.705882}%
\pgfsetstrokecolor{currentstroke}%
\pgfsetdash{}{0pt}%
\pgfsys@defobject{currentmarker}{\pgfqpoint{-0.031056in}{-0.031056in}}{\pgfqpoint{0.031056in}{0.031056in}}{%
\pgfpathmoveto{\pgfqpoint{0.000000in}{-0.031056in}}%
\pgfpathcurveto{\pgfqpoint{0.008236in}{-0.031056in}}{\pgfqpoint{0.016136in}{-0.027784in}}{\pgfqpoint{0.021960in}{-0.021960in}}%
\pgfpathcurveto{\pgfqpoint{0.027784in}{-0.016136in}}{\pgfqpoint{0.031056in}{-0.008236in}}{\pgfqpoint{0.031056in}{0.000000in}}%
\pgfpathcurveto{\pgfqpoint{0.031056in}{0.008236in}}{\pgfqpoint{0.027784in}{0.016136in}}{\pgfqpoint{0.021960in}{0.021960in}}%
\pgfpathcurveto{\pgfqpoint{0.016136in}{0.027784in}}{\pgfqpoint{0.008236in}{0.031056in}}{\pgfqpoint{0.000000in}{0.031056in}}%
\pgfpathcurveto{\pgfqpoint{-0.008236in}{0.031056in}}{\pgfqpoint{-0.016136in}{0.027784in}}{\pgfqpoint{-0.021960in}{0.021960in}}%
\pgfpathcurveto{\pgfqpoint{-0.027784in}{0.016136in}}{\pgfqpoint{-0.031056in}{0.008236in}}{\pgfqpoint{-0.031056in}{0.000000in}}%
\pgfpathcurveto{\pgfqpoint{-0.031056in}{-0.008236in}}{\pgfqpoint{-0.027784in}{-0.016136in}}{\pgfqpoint{-0.021960in}{-0.021960in}}%
\pgfpathcurveto{\pgfqpoint{-0.016136in}{-0.027784in}}{\pgfqpoint{-0.008236in}{-0.031056in}}{\pgfqpoint{0.000000in}{-0.031056in}}%
\pgfpathclose%
\pgfusepath{stroke,fill}%
}%
\begin{pgfscope}%
\pgfsys@transformshift{2.125088in}{3.518806in}%
\pgfsys@useobject{currentmarker}{}%
\end{pgfscope}%
\end{pgfscope}%
\begin{pgfscope}%
\definecolor{textcolor}{rgb}{0.000000,0.000000,0.000000}%
\pgfsetstrokecolor{textcolor}%
\pgfsetfillcolor{textcolor}%
\pgftext[x=2.375088in,y=3.482348in,left,base]{\color{textcolor}\sffamily\fontsize{10.000000}{12.000000}\selectfont Estimated position}%
\end{pgfscope}%
\begin{pgfscope}%
\pgfsetbuttcap%
\pgfsetroundjoin%
\definecolor{currentfill}{rgb}{1.000000,0.498039,0.054902}%
\pgfsetfillcolor{currentfill}%
\pgfsetlinewidth{1.003750pt}%
\definecolor{currentstroke}{rgb}{1.000000,0.498039,0.054902}%
\pgfsetstrokecolor{currentstroke}%
\pgfsetdash{}{0pt}%
\pgfsys@defobject{currentmarker}{\pgfqpoint{-0.031056in}{-0.031056in}}{\pgfqpoint{0.031056in}{0.031056in}}{%
\pgfpathmoveto{\pgfqpoint{0.000000in}{-0.031056in}}%
\pgfpathcurveto{\pgfqpoint{0.008236in}{-0.031056in}}{\pgfqpoint{0.016136in}{-0.027784in}}{\pgfqpoint{0.021960in}{-0.021960in}}%
\pgfpathcurveto{\pgfqpoint{0.027784in}{-0.016136in}}{\pgfqpoint{0.031056in}{-0.008236in}}{\pgfqpoint{0.031056in}{0.000000in}}%
\pgfpathcurveto{\pgfqpoint{0.031056in}{0.008236in}}{\pgfqpoint{0.027784in}{0.016136in}}{\pgfqpoint{0.021960in}{0.021960in}}%
\pgfpathcurveto{\pgfqpoint{0.016136in}{0.027784in}}{\pgfqpoint{0.008236in}{0.031056in}}{\pgfqpoint{0.000000in}{0.031056in}}%
\pgfpathcurveto{\pgfqpoint{-0.008236in}{0.031056in}}{\pgfqpoint{-0.016136in}{0.027784in}}{\pgfqpoint{-0.021960in}{0.021960in}}%
\pgfpathcurveto{\pgfqpoint{-0.027784in}{0.016136in}}{\pgfqpoint{-0.031056in}{0.008236in}}{\pgfqpoint{-0.031056in}{0.000000in}}%
\pgfpathcurveto{\pgfqpoint{-0.031056in}{-0.008236in}}{\pgfqpoint{-0.027784in}{-0.016136in}}{\pgfqpoint{-0.021960in}{-0.021960in}}%
\pgfpathcurveto{\pgfqpoint{-0.016136in}{-0.027784in}}{\pgfqpoint{-0.008236in}{-0.031056in}}{\pgfqpoint{0.000000in}{-0.031056in}}%
\pgfpathclose%
\pgfusepath{stroke,fill}%
}%
\begin{pgfscope}%
\pgfsys@transformshift{2.125088in}{3.314949in}%
\pgfsys@useobject{currentmarker}{}%
\end{pgfscope}%
\end{pgfscope}%
\begin{pgfscope}%
\definecolor{textcolor}{rgb}{0.000000,0.000000,0.000000}%
\pgfsetstrokecolor{textcolor}%
\pgfsetfillcolor{textcolor}%
\pgftext[x=2.375088in,y=3.278491in,left,base]{\color{textcolor}\sffamily\fontsize{10.000000}{12.000000}\selectfont Estimated turn}%
\end{pgfscope}%
\end{pgfpicture}%
\makeatother%
\endgroup%
}
%         \caption{FLAE's 3D position estimation had the lowest turn error for the 4-meter  side square experiment.}
%         \label{fig:square43D}
%     \end{subfigure}
%     \caption{Position estimation by the best performing algorithms in the 4-meter side square experiment.}
%     \label{fig:square4}
% \end{figure}

% \subsubsection{16 meter}

% For the 16-meter square experiment, the Mahony algorithm which had the lowest displacement error with an average of 1.93 meters (3.02\% of error margin), and FLAE with an average of 1.82 meters of turn error (2.84\% of error margin).

% \begin{figure}[!h]
%     \centering
%     \begin{table}[H]
    \begin{center}
        \resizebox{1\linewidth}{!}{
            \begin{tabular}[t]{lcccc}
                \hline
                Algorithm   & Displacement Error[$m$] & Displacement Error[\%] & Turn Error[$m$] & Turn Error[\%] \\
                \hline
                AngularRate & 21.40                   & 33.43                  & 27.65           & 43.20          \\            AQUA            & 9.62  & 15.03 & 10.47 & 16.35              \\            Complementary            & 6.52  & 10.19 & 9.07 & 14.18              \\            Davenport            & 9.11  & 14.23 & 9.60 & 15.00              \\            EKF            & 1.54  & 2.40 & 2.03 & 3.18              \\            FAMC            & 19.84  & 30.99 & 25.26 & 39.47              \\            FLAE            & 9.61  & 15.01 & 9.56 & 14.94              \\            Fourati            & 18.54  & 28.97 & 23.98 & 37.46              \\            Madgwick            & 9.38  & 14.66 & 7.58 & 11.84              \\            Mahony            & 1.73  & 2.71 & 2.07 & 3.23              \\            OLEQ            & 2.16  & 3.38 & 2.86 & 4.47              \\            QUEST            & 15.99  & 24.99 & 21.68 & 33.88              \\            ROLEQ            & 2.29  & 3.58 & 2.72 & 4.25              \\            SAAM            & 9.67  & 15.11 & 9.80 & 15.31              \\            Tilt            & 9.67  & 15.11 & 9.80 & 15.31              \\
                \hline
                Average     & 9.80                    & 15.32                  & 11.61           & 18.14
            \end{tabular}
        }
        \caption{Accelerometer Specifications. }
        \label{tab:accelerometer_specification}
    \end{center}
\end{table}
% \end{figure}

% \begin{figure}[!h]
%     \centering
%     \begin{subfigure}{0.49\textwidth}
%         \centering
%         \resizebox{1\linewidth}{!}{%% Creator: Matplotlib, PGF backend
%%
%% To include the figure in your LaTeX document, write
%%   \input{<filename>.pgf}
%%
%% Make sure the required packages are loaded in your preamble
%%   \usepackage{pgf}
%%
%% and, on pdftex
%%   \usepackage[utf8]{inputenc}\DeclareUnicodeCharacter{2212}{-}
%%
%% or, on luatex and xetex
%%   \usepackage{unicode-math}
%%
%% Figures using additional raster images can only be included by \input if
%% they are in the same directory as the main LaTeX file. For loading figures
%% from other directories you can use the `import` package
%%   \usepackage{import}
%%
%% and then include the figures with
%%   \import{<path to file>}{<filename>.pgf}
%%
%% Matplotlib used the following preamble
%%   \usepackage{fontspec}
%%
\begingroup%
\makeatletter%
\begin{pgfpicture}%
\pgfpathrectangle{\pgfpointorigin}{\pgfqpoint{5.737192in}{4.311000in}}%
\pgfusepath{use as bounding box, clip}%
\begin{pgfscope}%
\pgfsetbuttcap%
\pgfsetmiterjoin%
\definecolor{currentfill}{rgb}{1.000000,1.000000,1.000000}%
\pgfsetfillcolor{currentfill}%
\pgfsetlinewidth{0.000000pt}%
\definecolor{currentstroke}{rgb}{1.000000,1.000000,1.000000}%
\pgfsetstrokecolor{currentstroke}%
\pgfsetdash{}{0pt}%
\pgfpathmoveto{\pgfqpoint{0.000000in}{0.000000in}}%
\pgfpathlineto{\pgfqpoint{5.737192in}{0.000000in}}%
\pgfpathlineto{\pgfqpoint{5.737192in}{4.311000in}}%
\pgfpathlineto{\pgfqpoint{0.000000in}{4.311000in}}%
\pgfpathclose%
\pgfusepath{fill}%
\end{pgfscope}%
\begin{pgfscope}%
\pgfsetbuttcap%
\pgfsetmiterjoin%
\definecolor{currentfill}{rgb}{1.000000,1.000000,1.000000}%
\pgfsetfillcolor{currentfill}%
\pgfsetlinewidth{0.000000pt}%
\definecolor{currentstroke}{rgb}{0.000000,0.000000,0.000000}%
\pgfsetstrokecolor{currentstroke}%
\pgfsetstrokeopacity{0.000000}%
\pgfsetdash{}{0pt}%
\pgfpathmoveto{\pgfqpoint{0.677192in}{0.515000in}}%
\pgfpathlineto{\pgfqpoint{5.637192in}{0.515000in}}%
\pgfpathlineto{\pgfqpoint{5.637192in}{4.211000in}}%
\pgfpathlineto{\pgfqpoint{0.677192in}{4.211000in}}%
\pgfpathclose%
\pgfusepath{fill}%
\end{pgfscope}%
\begin{pgfscope}%
\pgfpathrectangle{\pgfqpoint{0.677192in}{0.515000in}}{\pgfqpoint{4.960000in}{3.696000in}}%
\pgfusepath{clip}%
\pgfsetbuttcap%
\pgfsetroundjoin%
\definecolor{currentfill}{rgb}{0.121569,0.466667,0.705882}%
\pgfsetfillcolor{currentfill}%
\pgfsetlinewidth{1.003750pt}%
\definecolor{currentstroke}{rgb}{0.121569,0.466667,0.705882}%
\pgfsetstrokecolor{currentstroke}%
\pgfsetdash{}{0pt}%
\pgfsys@defobject{currentmarker}{\pgfqpoint{-0.041667in}{-0.041667in}}{\pgfqpoint{0.041667in}{0.041667in}}{%
\pgfpathmoveto{\pgfqpoint{0.000000in}{-0.041667in}}%
\pgfpathcurveto{\pgfqpoint{0.011050in}{-0.041667in}}{\pgfqpoint{0.021649in}{-0.037276in}}{\pgfqpoint{0.029463in}{-0.029463in}}%
\pgfpathcurveto{\pgfqpoint{0.037276in}{-0.021649in}}{\pgfqpoint{0.041667in}{-0.011050in}}{\pgfqpoint{0.041667in}{0.000000in}}%
\pgfpathcurveto{\pgfqpoint{0.041667in}{0.011050in}}{\pgfqpoint{0.037276in}{0.021649in}}{\pgfqpoint{0.029463in}{0.029463in}}%
\pgfpathcurveto{\pgfqpoint{0.021649in}{0.037276in}}{\pgfqpoint{0.011050in}{0.041667in}}{\pgfqpoint{0.000000in}{0.041667in}}%
\pgfpathcurveto{\pgfqpoint{-0.011050in}{0.041667in}}{\pgfqpoint{-0.021649in}{0.037276in}}{\pgfqpoint{-0.029463in}{0.029463in}}%
\pgfpathcurveto{\pgfqpoint{-0.037276in}{0.021649in}}{\pgfqpoint{-0.041667in}{0.011050in}}{\pgfqpoint{-0.041667in}{0.000000in}}%
\pgfpathcurveto{\pgfqpoint{-0.041667in}{-0.011050in}}{\pgfqpoint{-0.037276in}{-0.021649in}}{\pgfqpoint{-0.029463in}{-0.029463in}}%
\pgfpathcurveto{\pgfqpoint{-0.021649in}{-0.037276in}}{\pgfqpoint{-0.011050in}{-0.041667in}}{\pgfqpoint{0.000000in}{-0.041667in}}%
\pgfpathclose%
\pgfusepath{stroke,fill}%
}%
\begin{pgfscope}%
\pgfsys@transformshift{1.485217in}{1.038526in}%
\pgfsys@useobject{currentmarker}{}%
\end{pgfscope}%
\begin{pgfscope}%
\pgfsys@transformshift{1.485474in}{1.035907in}%
\pgfsys@useobject{currentmarker}{}%
\end{pgfscope}%
\begin{pgfscope}%
\pgfsys@transformshift{1.486007in}{1.030796in}%
\pgfsys@useobject{currentmarker}{}%
\end{pgfscope}%
\begin{pgfscope}%
\pgfsys@transformshift{1.486630in}{1.023482in}%
\pgfsys@useobject{currentmarker}{}%
\end{pgfscope}%
\begin{pgfscope}%
\pgfsys@transformshift{1.486747in}{1.014531in}%
\pgfsys@useobject{currentmarker}{}%
\end{pgfscope}%
\begin{pgfscope}%
\pgfsys@transformshift{1.486731in}{1.009607in}%
\pgfsys@useobject{currentmarker}{}%
\end{pgfscope}%
\begin{pgfscope}%
\pgfsys@transformshift{1.486857in}{1.006902in}%
\pgfsys@useobject{currentmarker}{}%
\end{pgfscope}%
\begin{pgfscope}%
\pgfsys@transformshift{1.486883in}{1.005413in}%
\pgfsys@useobject{currentmarker}{}%
\end{pgfscope}%
\begin{pgfscope}%
\pgfsys@transformshift{1.486932in}{1.004595in}%
\pgfsys@useobject{currentmarker}{}%
\end{pgfscope}%
\begin{pgfscope}%
\pgfsys@transformshift{1.486964in}{1.004146in}%
\pgfsys@useobject{currentmarker}{}%
\end{pgfscope}%
\begin{pgfscope}%
\pgfsys@transformshift{1.486978in}{1.003898in}%
\pgfsys@useobject{currentmarker}{}%
\end{pgfscope}%
\begin{pgfscope}%
\pgfsys@transformshift{1.486978in}{1.003762in}%
\pgfsys@useobject{currentmarker}{}%
\end{pgfscope}%
\begin{pgfscope}%
\pgfsys@transformshift{1.486976in}{1.003687in}%
\pgfsys@useobject{currentmarker}{}%
\end{pgfscope}%
\begin{pgfscope}%
\pgfsys@transformshift{1.486978in}{1.003646in}%
\pgfsys@useobject{currentmarker}{}%
\end{pgfscope}%
\begin{pgfscope}%
\pgfsys@transformshift{1.486978in}{1.003623in}%
\pgfsys@useobject{currentmarker}{}%
\end{pgfscope}%
\begin{pgfscope}%
\pgfsys@transformshift{1.486978in}{1.003611in}%
\pgfsys@useobject{currentmarker}{}%
\end{pgfscope}%
\begin{pgfscope}%
\pgfsys@transformshift{1.486978in}{1.003604in}%
\pgfsys@useobject{currentmarker}{}%
\end{pgfscope}%
\begin{pgfscope}%
\pgfsys@transformshift{1.486978in}{1.003600in}%
\pgfsys@useobject{currentmarker}{}%
\end{pgfscope}%
\begin{pgfscope}%
\pgfsys@transformshift{1.486978in}{1.003598in}%
\pgfsys@useobject{currentmarker}{}%
\end{pgfscope}%
\begin{pgfscope}%
\pgfsys@transformshift{1.486978in}{1.003597in}%
\pgfsys@useobject{currentmarker}{}%
\end{pgfscope}%
\begin{pgfscope}%
\pgfsys@transformshift{1.486978in}{1.003596in}%
\pgfsys@useobject{currentmarker}{}%
\end{pgfscope}%
\begin{pgfscope}%
\pgfsys@transformshift{1.486978in}{1.003596in}%
\pgfsys@useobject{currentmarker}{}%
\end{pgfscope}%
\begin{pgfscope}%
\pgfsys@transformshift{1.486978in}{1.003596in}%
\pgfsys@useobject{currentmarker}{}%
\end{pgfscope}%
\begin{pgfscope}%
\pgfsys@transformshift{1.486978in}{1.003596in}%
\pgfsys@useobject{currentmarker}{}%
\end{pgfscope}%
\begin{pgfscope}%
\pgfsys@transformshift{1.486978in}{1.003596in}%
\pgfsys@useobject{currentmarker}{}%
\end{pgfscope}%
\begin{pgfscope}%
\pgfsys@transformshift{1.486978in}{1.003595in}%
\pgfsys@useobject{currentmarker}{}%
\end{pgfscope}%
\begin{pgfscope}%
\pgfsys@transformshift{1.486978in}{1.003595in}%
\pgfsys@useobject{currentmarker}{}%
\end{pgfscope}%
\begin{pgfscope}%
\pgfsys@transformshift{1.486978in}{1.003595in}%
\pgfsys@useobject{currentmarker}{}%
\end{pgfscope}%
\begin{pgfscope}%
\pgfsys@transformshift{1.486978in}{1.003595in}%
\pgfsys@useobject{currentmarker}{}%
\end{pgfscope}%
\begin{pgfscope}%
\pgfsys@transformshift{1.486978in}{1.003595in}%
\pgfsys@useobject{currentmarker}{}%
\end{pgfscope}%
\begin{pgfscope}%
\pgfsys@transformshift{1.486978in}{1.003595in}%
\pgfsys@useobject{currentmarker}{}%
\end{pgfscope}%
\begin{pgfscope}%
\pgfsys@transformshift{1.486978in}{1.003595in}%
\pgfsys@useobject{currentmarker}{}%
\end{pgfscope}%
\begin{pgfscope}%
\pgfsys@transformshift{1.486978in}{1.003595in}%
\pgfsys@useobject{currentmarker}{}%
\end{pgfscope}%
\begin{pgfscope}%
\pgfsys@transformshift{1.486978in}{1.003595in}%
\pgfsys@useobject{currentmarker}{}%
\end{pgfscope}%
\begin{pgfscope}%
\pgfsys@transformshift{1.486978in}{1.003595in}%
\pgfsys@useobject{currentmarker}{}%
\end{pgfscope}%
\begin{pgfscope}%
\pgfsys@transformshift{1.486978in}{1.003595in}%
\pgfsys@useobject{currentmarker}{}%
\end{pgfscope}%
\begin{pgfscope}%
\pgfsys@transformshift{1.486978in}{1.003595in}%
\pgfsys@useobject{currentmarker}{}%
\end{pgfscope}%
\begin{pgfscope}%
\pgfsys@transformshift{1.486978in}{1.003595in}%
\pgfsys@useobject{currentmarker}{}%
\end{pgfscope}%
\begin{pgfscope}%
\pgfsys@transformshift{1.486978in}{1.003595in}%
\pgfsys@useobject{currentmarker}{}%
\end{pgfscope}%
\begin{pgfscope}%
\pgfsys@transformshift{1.486978in}{1.003595in}%
\pgfsys@useobject{currentmarker}{}%
\end{pgfscope}%
\begin{pgfscope}%
\pgfsys@transformshift{1.486978in}{1.003595in}%
\pgfsys@useobject{currentmarker}{}%
\end{pgfscope}%
\begin{pgfscope}%
\pgfsys@transformshift{1.486978in}{1.003595in}%
\pgfsys@useobject{currentmarker}{}%
\end{pgfscope}%
\begin{pgfscope}%
\pgfsys@transformshift{1.486978in}{1.003595in}%
\pgfsys@useobject{currentmarker}{}%
\end{pgfscope}%
\begin{pgfscope}%
\pgfsys@transformshift{1.486978in}{1.003595in}%
\pgfsys@useobject{currentmarker}{}%
\end{pgfscope}%
\begin{pgfscope}%
\pgfsys@transformshift{1.486978in}{1.003595in}%
\pgfsys@useobject{currentmarker}{}%
\end{pgfscope}%
\begin{pgfscope}%
\pgfsys@transformshift{1.486978in}{1.003595in}%
\pgfsys@useobject{currentmarker}{}%
\end{pgfscope}%
\begin{pgfscope}%
\pgfsys@transformshift{1.486978in}{1.003595in}%
\pgfsys@useobject{currentmarker}{}%
\end{pgfscope}%
\begin{pgfscope}%
\pgfsys@transformshift{1.486978in}{1.003595in}%
\pgfsys@useobject{currentmarker}{}%
\end{pgfscope}%
\begin{pgfscope}%
\pgfsys@transformshift{1.486978in}{1.003595in}%
\pgfsys@useobject{currentmarker}{}%
\end{pgfscope}%
\begin{pgfscope}%
\pgfsys@transformshift{1.486978in}{1.003595in}%
\pgfsys@useobject{currentmarker}{}%
\end{pgfscope}%
\begin{pgfscope}%
\pgfsys@transformshift{1.486978in}{1.003595in}%
\pgfsys@useobject{currentmarker}{}%
\end{pgfscope}%
\begin{pgfscope}%
\pgfsys@transformshift{1.486978in}{1.003595in}%
\pgfsys@useobject{currentmarker}{}%
\end{pgfscope}%
\begin{pgfscope}%
\pgfsys@transformshift{1.486978in}{1.003595in}%
\pgfsys@useobject{currentmarker}{}%
\end{pgfscope}%
\begin{pgfscope}%
\pgfsys@transformshift{1.486978in}{1.003595in}%
\pgfsys@useobject{currentmarker}{}%
\end{pgfscope}%
\begin{pgfscope}%
\pgfsys@transformshift{1.486978in}{1.003595in}%
\pgfsys@useobject{currentmarker}{}%
\end{pgfscope}%
\begin{pgfscope}%
\pgfsys@transformshift{1.486978in}{1.003595in}%
\pgfsys@useobject{currentmarker}{}%
\end{pgfscope}%
\begin{pgfscope}%
\pgfsys@transformshift{1.486978in}{1.003595in}%
\pgfsys@useobject{currentmarker}{}%
\end{pgfscope}%
\begin{pgfscope}%
\pgfsys@transformshift{1.486978in}{1.003595in}%
\pgfsys@useobject{currentmarker}{}%
\end{pgfscope}%
\begin{pgfscope}%
\pgfsys@transformshift{1.486978in}{1.003595in}%
\pgfsys@useobject{currentmarker}{}%
\end{pgfscope}%
\begin{pgfscope}%
\pgfsys@transformshift{1.486978in}{1.003595in}%
\pgfsys@useobject{currentmarker}{}%
\end{pgfscope}%
\begin{pgfscope}%
\pgfsys@transformshift{1.486978in}{1.003595in}%
\pgfsys@useobject{currentmarker}{}%
\end{pgfscope}%
\begin{pgfscope}%
\pgfsys@transformshift{1.486978in}{1.003595in}%
\pgfsys@useobject{currentmarker}{}%
\end{pgfscope}%
\begin{pgfscope}%
\pgfsys@transformshift{1.486978in}{1.003595in}%
\pgfsys@useobject{currentmarker}{}%
\end{pgfscope}%
\begin{pgfscope}%
\pgfsys@transformshift{1.486978in}{1.003595in}%
\pgfsys@useobject{currentmarker}{}%
\end{pgfscope}%
\begin{pgfscope}%
\pgfsys@transformshift{1.486978in}{1.003595in}%
\pgfsys@useobject{currentmarker}{}%
\end{pgfscope}%
\begin{pgfscope}%
\pgfsys@transformshift{1.486978in}{1.003595in}%
\pgfsys@useobject{currentmarker}{}%
\end{pgfscope}%
\begin{pgfscope}%
\pgfsys@transformshift{1.486978in}{1.003595in}%
\pgfsys@useobject{currentmarker}{}%
\end{pgfscope}%
\begin{pgfscope}%
\pgfsys@transformshift{1.486978in}{1.003595in}%
\pgfsys@useobject{currentmarker}{}%
\end{pgfscope}%
\begin{pgfscope}%
\pgfsys@transformshift{1.486978in}{1.003595in}%
\pgfsys@useobject{currentmarker}{}%
\end{pgfscope}%
\begin{pgfscope}%
\pgfsys@transformshift{1.486978in}{1.003595in}%
\pgfsys@useobject{currentmarker}{}%
\end{pgfscope}%
\begin{pgfscope}%
\pgfsys@transformshift{1.486978in}{1.003595in}%
\pgfsys@useobject{currentmarker}{}%
\end{pgfscope}%
\begin{pgfscope}%
\pgfsys@transformshift{1.486978in}{1.003595in}%
\pgfsys@useobject{currentmarker}{}%
\end{pgfscope}%
\begin{pgfscope}%
\pgfsys@transformshift{1.486978in}{1.003595in}%
\pgfsys@useobject{currentmarker}{}%
\end{pgfscope}%
\begin{pgfscope}%
\pgfsys@transformshift{1.486978in}{1.003595in}%
\pgfsys@useobject{currentmarker}{}%
\end{pgfscope}%
\begin{pgfscope}%
\pgfsys@transformshift{1.486978in}{1.003595in}%
\pgfsys@useobject{currentmarker}{}%
\end{pgfscope}%
\begin{pgfscope}%
\pgfsys@transformshift{1.486978in}{1.003595in}%
\pgfsys@useobject{currentmarker}{}%
\end{pgfscope}%
\begin{pgfscope}%
\pgfsys@transformshift{1.486978in}{1.003595in}%
\pgfsys@useobject{currentmarker}{}%
\end{pgfscope}%
\begin{pgfscope}%
\pgfsys@transformshift{1.486978in}{1.003595in}%
\pgfsys@useobject{currentmarker}{}%
\end{pgfscope}%
\begin{pgfscope}%
\pgfsys@transformshift{1.486978in}{1.003595in}%
\pgfsys@useobject{currentmarker}{}%
\end{pgfscope}%
\begin{pgfscope}%
\pgfsys@transformshift{1.486978in}{1.003595in}%
\pgfsys@useobject{currentmarker}{}%
\end{pgfscope}%
\begin{pgfscope}%
\pgfsys@transformshift{1.486978in}{1.003595in}%
\pgfsys@useobject{currentmarker}{}%
\end{pgfscope}%
\begin{pgfscope}%
\pgfsys@transformshift{1.486978in}{1.003595in}%
\pgfsys@useobject{currentmarker}{}%
\end{pgfscope}%
\begin{pgfscope}%
\pgfsys@transformshift{1.486978in}{1.003595in}%
\pgfsys@useobject{currentmarker}{}%
\end{pgfscope}%
\begin{pgfscope}%
\pgfsys@transformshift{1.486978in}{1.003595in}%
\pgfsys@useobject{currentmarker}{}%
\end{pgfscope}%
\begin{pgfscope}%
\pgfsys@transformshift{1.486978in}{1.003595in}%
\pgfsys@useobject{currentmarker}{}%
\end{pgfscope}%
\begin{pgfscope}%
\pgfsys@transformshift{1.486978in}{1.003595in}%
\pgfsys@useobject{currentmarker}{}%
\end{pgfscope}%
\begin{pgfscope}%
\pgfsys@transformshift{1.486978in}{1.003595in}%
\pgfsys@useobject{currentmarker}{}%
\end{pgfscope}%
\begin{pgfscope}%
\pgfsys@transformshift{1.486978in}{1.003595in}%
\pgfsys@useobject{currentmarker}{}%
\end{pgfscope}%
\begin{pgfscope}%
\pgfsys@transformshift{1.486978in}{1.003595in}%
\pgfsys@useobject{currentmarker}{}%
\end{pgfscope}%
\begin{pgfscope}%
\pgfsys@transformshift{1.486978in}{1.003595in}%
\pgfsys@useobject{currentmarker}{}%
\end{pgfscope}%
\begin{pgfscope}%
\pgfsys@transformshift{1.486978in}{1.003595in}%
\pgfsys@useobject{currentmarker}{}%
\end{pgfscope}%
\begin{pgfscope}%
\pgfsys@transformshift{1.486978in}{1.003595in}%
\pgfsys@useobject{currentmarker}{}%
\end{pgfscope}%
\begin{pgfscope}%
\pgfsys@transformshift{1.486978in}{1.003595in}%
\pgfsys@useobject{currentmarker}{}%
\end{pgfscope}%
\begin{pgfscope}%
\pgfsys@transformshift{1.486978in}{1.003595in}%
\pgfsys@useobject{currentmarker}{}%
\end{pgfscope}%
\begin{pgfscope}%
\pgfsys@transformshift{1.486978in}{1.003595in}%
\pgfsys@useobject{currentmarker}{}%
\end{pgfscope}%
\begin{pgfscope}%
\pgfsys@transformshift{1.486978in}{1.003595in}%
\pgfsys@useobject{currentmarker}{}%
\end{pgfscope}%
\begin{pgfscope}%
\pgfsys@transformshift{1.486978in}{1.003595in}%
\pgfsys@useobject{currentmarker}{}%
\end{pgfscope}%
\begin{pgfscope}%
\pgfsys@transformshift{1.486978in}{1.003595in}%
\pgfsys@useobject{currentmarker}{}%
\end{pgfscope}%
\begin{pgfscope}%
\pgfsys@transformshift{1.486978in}{1.003595in}%
\pgfsys@useobject{currentmarker}{}%
\end{pgfscope}%
\begin{pgfscope}%
\pgfsys@transformshift{1.486978in}{1.003595in}%
\pgfsys@useobject{currentmarker}{}%
\end{pgfscope}%
\begin{pgfscope}%
\pgfsys@transformshift{1.486978in}{1.003595in}%
\pgfsys@useobject{currentmarker}{}%
\end{pgfscope}%
\begin{pgfscope}%
\pgfsys@transformshift{1.486978in}{1.003595in}%
\pgfsys@useobject{currentmarker}{}%
\end{pgfscope}%
\begin{pgfscope}%
\pgfsys@transformshift{1.486978in}{1.003595in}%
\pgfsys@useobject{currentmarker}{}%
\end{pgfscope}%
\begin{pgfscope}%
\pgfsys@transformshift{1.486978in}{1.003595in}%
\pgfsys@useobject{currentmarker}{}%
\end{pgfscope}%
\begin{pgfscope}%
\pgfsys@transformshift{1.486978in}{1.003595in}%
\pgfsys@useobject{currentmarker}{}%
\end{pgfscope}%
\begin{pgfscope}%
\pgfsys@transformshift{1.486978in}{1.003595in}%
\pgfsys@useobject{currentmarker}{}%
\end{pgfscope}%
\begin{pgfscope}%
\pgfsys@transformshift{1.486978in}{1.003595in}%
\pgfsys@useobject{currentmarker}{}%
\end{pgfscope}%
\begin{pgfscope}%
\pgfsys@transformshift{1.487726in}{1.003000in}%
\pgfsys@useobject{currentmarker}{}%
\end{pgfscope}%
\begin{pgfscope}%
\pgfsys@transformshift{1.490446in}{1.001814in}%
\pgfsys@useobject{currentmarker}{}%
\end{pgfscope}%
\begin{pgfscope}%
\pgfsys@transformshift{1.492078in}{1.001759in}%
\pgfsys@useobject{currentmarker}{}%
\end{pgfscope}%
\begin{pgfscope}%
\pgfsys@transformshift{1.494652in}{1.001845in}%
\pgfsys@useobject{currentmarker}{}%
\end{pgfscope}%
\begin{pgfscope}%
\pgfsys@transformshift{1.499275in}{1.003504in}%
\pgfsys@useobject{currentmarker}{}%
\end{pgfscope}%
\begin{pgfscope}%
\pgfsys@transformshift{1.501687in}{1.004720in}%
\pgfsys@useobject{currentmarker}{}%
\end{pgfscope}%
\begin{pgfscope}%
\pgfsys@transformshift{1.504063in}{1.007803in}%
\pgfsys@useobject{currentmarker}{}%
\end{pgfscope}%
\begin{pgfscope}%
\pgfsys@transformshift{1.506469in}{1.011993in}%
\pgfsys@useobject{currentmarker}{}%
\end{pgfscope}%
\begin{pgfscope}%
\pgfsys@transformshift{1.507925in}{1.018431in}%
\pgfsys@useobject{currentmarker}{}%
\end{pgfscope}%
\begin{pgfscope}%
\pgfsys@transformshift{1.509821in}{1.027682in}%
\pgfsys@useobject{currentmarker}{}%
\end{pgfscope}%
\begin{pgfscope}%
\pgfsys@transformshift{1.512657in}{1.038849in}%
\pgfsys@useobject{currentmarker}{}%
\end{pgfscope}%
\begin{pgfscope}%
\pgfsys@transformshift{1.511993in}{1.052419in}%
\pgfsys@useobject{currentmarker}{}%
\end{pgfscope}%
\begin{pgfscope}%
\pgfsys@transformshift{1.513947in}{1.067656in}%
\pgfsys@useobject{currentmarker}{}%
\end{pgfscope}%
\begin{pgfscope}%
\pgfsys@transformshift{1.514727in}{1.076069in}%
\pgfsys@useobject{currentmarker}{}%
\end{pgfscope}%
\begin{pgfscope}%
\pgfsys@transformshift{1.514907in}{1.080713in}%
\pgfsys@useobject{currentmarker}{}%
\end{pgfscope}%
\begin{pgfscope}%
\pgfsys@transformshift{1.515199in}{1.083252in}%
\pgfsys@useobject{currentmarker}{}%
\end{pgfscope}%
\begin{pgfscope}%
\pgfsys@transformshift{1.515371in}{1.084647in}%
\pgfsys@useobject{currentmarker}{}%
\end{pgfscope}%
\begin{pgfscope}%
\pgfsys@transformshift{1.515527in}{1.085404in}%
\pgfsys@useobject{currentmarker}{}%
\end{pgfscope}%
\begin{pgfscope}%
\pgfsys@transformshift{1.515573in}{1.085827in}%
\pgfsys@useobject{currentmarker}{}%
\end{pgfscope}%
\begin{pgfscope}%
\pgfsys@transformshift{1.515618in}{1.086056in}%
\pgfsys@useobject{currentmarker}{}%
\end{pgfscope}%
\begin{pgfscope}%
\pgfsys@transformshift{1.515636in}{1.086184in}%
\pgfsys@useobject{currentmarker}{}%
\end{pgfscope}%
\begin{pgfscope}%
\pgfsys@transformshift{1.515648in}{1.086253in}%
\pgfsys@useobject{currentmarker}{}%
\end{pgfscope}%
\begin{pgfscope}%
\pgfsys@transformshift{1.515732in}{1.087266in}%
\pgfsys@useobject{currentmarker}{}%
\end{pgfscope}%
\begin{pgfscope}%
\pgfsys@transformshift{1.515781in}{1.087822in}%
\pgfsys@useobject{currentmarker}{}%
\end{pgfscope}%
\begin{pgfscope}%
\pgfsys@transformshift{1.516032in}{1.089399in}%
\pgfsys@useobject{currentmarker}{}%
\end{pgfscope}%
\begin{pgfscope}%
\pgfsys@transformshift{1.516083in}{1.090276in}%
\pgfsys@useobject{currentmarker}{}%
\end{pgfscope}%
\begin{pgfscope}%
\pgfsys@transformshift{1.516110in}{1.090758in}%
\pgfsys@useobject{currentmarker}{}%
\end{pgfscope}%
\begin{pgfscope}%
\pgfsys@transformshift{1.516134in}{1.091023in}%
\pgfsys@useobject{currentmarker}{}%
\end{pgfscope}%
\begin{pgfscope}%
\pgfsys@transformshift{1.516206in}{1.092536in}%
\pgfsys@useobject{currentmarker}{}%
\end{pgfscope}%
\begin{pgfscope}%
\pgfsys@transformshift{1.516260in}{1.093368in}%
\pgfsys@useobject{currentmarker}{}%
\end{pgfscope}%
\begin{pgfscope}%
\pgfsys@transformshift{1.516297in}{1.095726in}%
\pgfsys@useobject{currentmarker}{}%
\end{pgfscope}%
\begin{pgfscope}%
\pgfsys@transformshift{1.516288in}{1.097024in}%
\pgfsys@useobject{currentmarker}{}%
\end{pgfscope}%
\begin{pgfscope}%
\pgfsys@transformshift{1.517808in}{1.101781in}%
\pgfsys@useobject{currentmarker}{}%
\end{pgfscope}%
\begin{pgfscope}%
\pgfsys@transformshift{1.515820in}{1.110683in}%
\pgfsys@useobject{currentmarker}{}%
\end{pgfscope}%
\begin{pgfscope}%
\pgfsys@transformshift{1.517476in}{1.115418in}%
\pgfsys@useobject{currentmarker}{}%
\end{pgfscope}%
\begin{pgfscope}%
\pgfsys@transformshift{1.516747in}{1.118079in}%
\pgfsys@useobject{currentmarker}{}%
\end{pgfscope}%
\begin{pgfscope}%
\pgfsys@transformshift{1.517986in}{1.122668in}%
\pgfsys@useobject{currentmarker}{}%
\end{pgfscope}%
\begin{pgfscope}%
\pgfsys@transformshift{1.517413in}{1.125219in}%
\pgfsys@useobject{currentmarker}{}%
\end{pgfscope}%
\begin{pgfscope}%
\pgfsys@transformshift{1.518477in}{1.129748in}%
\pgfsys@useobject{currentmarker}{}%
\end{pgfscope}%
\begin{pgfscope}%
\pgfsys@transformshift{1.516181in}{1.137008in}%
\pgfsys@useobject{currentmarker}{}%
\end{pgfscope}%
\begin{pgfscope}%
\pgfsys@transformshift{1.518281in}{1.146281in}%
\pgfsys@useobject{currentmarker}{}%
\end{pgfscope}%
\begin{pgfscope}%
\pgfsys@transformshift{1.516902in}{1.151326in}%
\pgfsys@useobject{currentmarker}{}%
\end{pgfscope}%
\begin{pgfscope}%
\pgfsys@transformshift{1.518940in}{1.159536in}%
\pgfsys@useobject{currentmarker}{}%
\end{pgfscope}%
\begin{pgfscope}%
\pgfsys@transformshift{1.515113in}{1.169894in}%
\pgfsys@useobject{currentmarker}{}%
\end{pgfscope}%
\begin{pgfscope}%
\pgfsys@transformshift{1.518640in}{1.183843in}%
\pgfsys@useobject{currentmarker}{}%
\end{pgfscope}%
\begin{pgfscope}%
\pgfsys@transformshift{1.513077in}{1.201438in}%
\pgfsys@useobject{currentmarker}{}%
\end{pgfscope}%
\begin{pgfscope}%
\pgfsys@transformshift{1.517955in}{1.223102in}%
\pgfsys@useobject{currentmarker}{}%
\end{pgfscope}%
\begin{pgfscope}%
\pgfsys@transformshift{1.509441in}{1.246308in}%
\pgfsys@useobject{currentmarker}{}%
\end{pgfscope}%
\begin{pgfscope}%
\pgfsys@transformshift{1.516952in}{1.273632in}%
\pgfsys@useobject{currentmarker}{}%
\end{pgfscope}%
\begin{pgfscope}%
\pgfsys@transformshift{1.512263in}{1.288496in}%
\pgfsys@useobject{currentmarker}{}%
\end{pgfscope}%
\begin{pgfscope}%
\pgfsys@transformshift{1.515524in}{1.306290in}%
\pgfsys@useobject{currentmarker}{}%
\end{pgfscope}%
\begin{pgfscope}%
\pgfsys@transformshift{1.512613in}{1.315805in}%
\pgfsys@useobject{currentmarker}{}%
\end{pgfscope}%
\begin{pgfscope}%
\pgfsys@transformshift{1.515015in}{1.328667in}%
\pgfsys@useobject{currentmarker}{}%
\end{pgfscope}%
\begin{pgfscope}%
\pgfsys@transformshift{1.510296in}{1.342205in}%
\pgfsys@useobject{currentmarker}{}%
\end{pgfscope}%
\begin{pgfscope}%
\pgfsys@transformshift{1.513605in}{1.359222in}%
\pgfsys@useobject{currentmarker}{}%
\end{pgfscope}%
\begin{pgfscope}%
\pgfsys@transformshift{1.510625in}{1.368279in}%
\pgfsys@useobject{currentmarker}{}%
\end{pgfscope}%
\begin{pgfscope}%
\pgfsys@transformshift{1.514069in}{1.380932in}%
\pgfsys@useobject{currentmarker}{}%
\end{pgfscope}%
\begin{pgfscope}%
\pgfsys@transformshift{1.511915in}{1.387815in}%
\pgfsys@useobject{currentmarker}{}%
\end{pgfscope}%
\begin{pgfscope}%
\pgfsys@transformshift{1.514732in}{1.398099in}%
\pgfsys@useobject{currentmarker}{}%
\end{pgfscope}%
\begin{pgfscope}%
\pgfsys@transformshift{1.513206in}{1.403762in}%
\pgfsys@useobject{currentmarker}{}%
\end{pgfscope}%
\begin{pgfscope}%
\pgfsys@transformshift{1.514075in}{1.406868in}%
\pgfsys@useobject{currentmarker}{}%
\end{pgfscope}%
\begin{pgfscope}%
\pgfsys@transformshift{1.513683in}{1.408599in}%
\pgfsys@useobject{currentmarker}{}%
\end{pgfscope}%
\begin{pgfscope}%
\pgfsys@transformshift{1.513870in}{1.409556in}%
\pgfsys@useobject{currentmarker}{}%
\end{pgfscope}%
\begin{pgfscope}%
\pgfsys@transformshift{1.513747in}{1.411643in}%
\pgfsys@useobject{currentmarker}{}%
\end{pgfscope}%
\begin{pgfscope}%
\pgfsys@transformshift{1.513689in}{1.415836in}%
\pgfsys@useobject{currentmarker}{}%
\end{pgfscope}%
\begin{pgfscope}%
\pgfsys@transformshift{1.514224in}{1.418080in}%
\pgfsys@useobject{currentmarker}{}%
\end{pgfscope}%
\begin{pgfscope}%
\pgfsys@transformshift{1.513857in}{1.419294in}%
\pgfsys@useobject{currentmarker}{}%
\end{pgfscope}%
\begin{pgfscope}%
\pgfsys@transformshift{1.515378in}{1.424250in}%
\pgfsys@useobject{currentmarker}{}%
\end{pgfscope}%
\begin{pgfscope}%
\pgfsys@transformshift{1.512947in}{1.432301in}%
\pgfsys@useobject{currentmarker}{}%
\end{pgfscope}%
\begin{pgfscope}%
\pgfsys@transformshift{1.516598in}{1.445464in}%
\pgfsys@useobject{currentmarker}{}%
\end{pgfscope}%
\begin{pgfscope}%
\pgfsys@transformshift{1.511112in}{1.461064in}%
\pgfsys@useobject{currentmarker}{}%
\end{pgfscope}%
\begin{pgfscope}%
\pgfsys@transformshift{1.515561in}{1.480847in}%
\pgfsys@useobject{currentmarker}{}%
\end{pgfscope}%
\begin{pgfscope}%
\pgfsys@transformshift{1.509536in}{1.502211in}%
\pgfsys@useobject{currentmarker}{}%
\end{pgfscope}%
\begin{pgfscope}%
\pgfsys@transformshift{1.514233in}{1.527217in}%
\pgfsys@useobject{currentmarker}{}%
\end{pgfscope}%
\begin{pgfscope}%
\pgfsys@transformshift{1.511291in}{1.540898in}%
\pgfsys@useobject{currentmarker}{}%
\end{pgfscope}%
\begin{pgfscope}%
\pgfsys@transformshift{1.512359in}{1.548520in}%
\pgfsys@useobject{currentmarker}{}%
\end{pgfscope}%
\begin{pgfscope}%
\pgfsys@transformshift{1.511389in}{1.557575in}%
\pgfsys@useobject{currentmarker}{}%
\end{pgfscope}%
\begin{pgfscope}%
\pgfsys@transformshift{1.512497in}{1.567991in}%
\pgfsys@useobject{currentmarker}{}%
\end{pgfscope}%
\begin{pgfscope}%
\pgfsys@transformshift{1.512949in}{1.579507in}%
\pgfsys@useobject{currentmarker}{}%
\end{pgfscope}%
\begin{pgfscope}%
\pgfsys@transformshift{1.510705in}{1.591747in}%
\pgfsys@useobject{currentmarker}{}%
\end{pgfscope}%
\begin{pgfscope}%
\pgfsys@transformshift{1.512334in}{1.598394in}%
\pgfsys@useobject{currentmarker}{}%
\end{pgfscope}%
\begin{pgfscope}%
\pgfsys@transformshift{1.511383in}{1.602037in}%
\pgfsys@useobject{currentmarker}{}%
\end{pgfscope}%
\begin{pgfscope}%
\pgfsys@transformshift{1.512475in}{1.606590in}%
\pgfsys@useobject{currentmarker}{}%
\end{pgfscope}%
\begin{pgfscope}%
\pgfsys@transformshift{1.510905in}{1.612351in}%
\pgfsys@useobject{currentmarker}{}%
\end{pgfscope}%
\begin{pgfscope}%
\pgfsys@transformshift{1.513688in}{1.620350in}%
\pgfsys@useobject{currentmarker}{}%
\end{pgfscope}%
\begin{pgfscope}%
\pgfsys@transformshift{1.510218in}{1.632702in}%
\pgfsys@useobject{currentmarker}{}%
\end{pgfscope}%
\begin{pgfscope}%
\pgfsys@transformshift{1.514552in}{1.649004in}%
\pgfsys@useobject{currentmarker}{}%
\end{pgfscope}%
\begin{pgfscope}%
\pgfsys@transformshift{1.508213in}{1.667515in}%
\pgfsys@useobject{currentmarker}{}%
\end{pgfscope}%
\begin{pgfscope}%
\pgfsys@transformshift{1.515002in}{1.689771in}%
\pgfsys@useobject{currentmarker}{}%
\end{pgfscope}%
\begin{pgfscope}%
\pgfsys@transformshift{1.508635in}{1.714208in}%
\pgfsys@useobject{currentmarker}{}%
\end{pgfscope}%
\begin{pgfscope}%
\pgfsys@transformshift{1.511725in}{1.727748in}%
\pgfsys@useobject{currentmarker}{}%
\end{pgfscope}%
\begin{pgfscope}%
\pgfsys@transformshift{1.509877in}{1.735160in}%
\pgfsys@useobject{currentmarker}{}%
\end{pgfscope}%
\begin{pgfscope}%
\pgfsys@transformshift{1.510801in}{1.744501in}%
\pgfsys@useobject{currentmarker}{}%
\end{pgfscope}%
\begin{pgfscope}%
\pgfsys@transformshift{1.511020in}{1.755011in}%
\pgfsys@useobject{currentmarker}{}%
\end{pgfscope}%
\begin{pgfscope}%
\pgfsys@transformshift{1.508969in}{1.760417in}%
\pgfsys@useobject{currentmarker}{}%
\end{pgfscope}%
\begin{pgfscope}%
\pgfsys@transformshift{1.511743in}{1.771139in}%
\pgfsys@useobject{currentmarker}{}%
\end{pgfscope}%
\begin{pgfscope}%
\pgfsys@transformshift{1.508397in}{1.785107in}%
\pgfsys@useobject{currentmarker}{}%
\end{pgfscope}%
\begin{pgfscope}%
\pgfsys@transformshift{1.513199in}{1.805043in}%
\pgfsys@useobject{currentmarker}{}%
\end{pgfscope}%
\begin{pgfscope}%
\pgfsys@transformshift{1.505265in}{1.825442in}%
\pgfsys@useobject{currentmarker}{}%
\end{pgfscope}%
\begin{pgfscope}%
\pgfsys@transformshift{1.508486in}{1.852491in}%
\pgfsys@useobject{currentmarker}{}%
\end{pgfscope}%
\begin{pgfscope}%
\pgfsys@transformshift{1.503774in}{1.866713in}%
\pgfsys@useobject{currentmarker}{}%
\end{pgfscope}%
\begin{pgfscope}%
\pgfsys@transformshift{1.506093in}{1.884724in}%
\pgfsys@useobject{currentmarker}{}%
\end{pgfscope}%
\begin{pgfscope}%
\pgfsys@transformshift{1.504091in}{1.894508in}%
\pgfsys@useobject{currentmarker}{}%
\end{pgfscope}%
\begin{pgfscope}%
\pgfsys@transformshift{1.504576in}{1.899980in}%
\pgfsys@useobject{currentmarker}{}%
\end{pgfscope}%
\begin{pgfscope}%
\pgfsys@transformshift{1.504112in}{1.906839in}%
\pgfsys@useobject{currentmarker}{}%
\end{pgfscope}%
\begin{pgfscope}%
\pgfsys@transformshift{1.502023in}{1.915266in}%
\pgfsys@useobject{currentmarker}{}%
\end{pgfscope}%
\begin{pgfscope}%
\pgfsys@transformshift{1.503444in}{1.919825in}%
\pgfsys@useobject{currentmarker}{}%
\end{pgfscope}%
\begin{pgfscope}%
\pgfsys@transformshift{1.501239in}{1.926590in}%
\pgfsys@useobject{currentmarker}{}%
\end{pgfscope}%
\begin{pgfscope}%
\pgfsys@transformshift{1.503273in}{1.935670in}%
\pgfsys@useobject{currentmarker}{}%
\end{pgfscope}%
\begin{pgfscope}%
\pgfsys@transformshift{1.500225in}{1.946908in}%
\pgfsys@useobject{currentmarker}{}%
\end{pgfscope}%
\begin{pgfscope}%
\pgfsys@transformshift{1.505143in}{1.961655in}%
\pgfsys@useobject{currentmarker}{}%
\end{pgfscope}%
\begin{pgfscope}%
\pgfsys@transformshift{1.499093in}{1.981952in}%
\pgfsys@useobject{currentmarker}{}%
\end{pgfscope}%
\begin{pgfscope}%
\pgfsys@transformshift{1.507127in}{2.007295in}%
\pgfsys@useobject{currentmarker}{}%
\end{pgfscope}%
\begin{pgfscope}%
\pgfsys@transformshift{1.498943in}{2.034694in}%
\pgfsys@useobject{currentmarker}{}%
\end{pgfscope}%
\begin{pgfscope}%
\pgfsys@transformshift{1.505158in}{2.065662in}%
\pgfsys@useobject{currentmarker}{}%
\end{pgfscope}%
\begin{pgfscope}%
\pgfsys@transformshift{1.500220in}{2.098108in}%
\pgfsys@useobject{currentmarker}{}%
\end{pgfscope}%
\begin{pgfscope}%
\pgfsys@transformshift{1.505455in}{2.132374in}%
\pgfsys@useobject{currentmarker}{}%
\end{pgfscope}%
\begin{pgfscope}%
\pgfsys@transformshift{1.511336in}{2.167469in}%
\pgfsys@useobject{currentmarker}{}%
\end{pgfscope}%
\begin{pgfscope}%
\pgfsys@transformshift{1.500418in}{2.205861in}%
\pgfsys@useobject{currentmarker}{}%
\end{pgfscope}%
\begin{pgfscope}%
\pgfsys@transformshift{1.515169in}{2.245295in}%
\pgfsys@useobject{currentmarker}{}%
\end{pgfscope}%
\begin{pgfscope}%
\pgfsys@transformshift{1.504674in}{2.290608in}%
\pgfsys@useobject{currentmarker}{}%
\end{pgfscope}%
\begin{pgfscope}%
\pgfsys@transformshift{1.520748in}{2.339326in}%
\pgfsys@useobject{currentmarker}{}%
\end{pgfscope}%
\begin{pgfscope}%
\pgfsys@transformshift{1.506915in}{2.393000in}%
\pgfsys@useobject{currentmarker}{}%
\end{pgfscope}%
\begin{pgfscope}%
\pgfsys@transformshift{1.519712in}{2.450010in}%
\pgfsys@useobject{currentmarker}{}%
\end{pgfscope}%
\begin{pgfscope}%
\pgfsys@transformshift{1.514430in}{2.481709in}%
\pgfsys@useobject{currentmarker}{}%
\end{pgfscope}%
\begin{pgfscope}%
\pgfsys@transformshift{1.517528in}{2.515570in}%
\pgfsys@useobject{currentmarker}{}%
\end{pgfscope}%
\begin{pgfscope}%
\pgfsys@transformshift{1.519295in}{2.534187in}%
\pgfsys@useobject{currentmarker}{}%
\end{pgfscope}%
\begin{pgfscope}%
\pgfsys@transformshift{1.514558in}{2.553928in}%
\pgfsys@useobject{currentmarker}{}%
\end{pgfscope}%
\begin{pgfscope}%
\pgfsys@transformshift{1.522101in}{2.574804in}%
\pgfsys@useobject{currentmarker}{}%
\end{pgfscope}%
\begin{pgfscope}%
\pgfsys@transformshift{1.516544in}{2.600375in}%
\pgfsys@useobject{currentmarker}{}%
\end{pgfscope}%
\begin{pgfscope}%
\pgfsys@transformshift{1.523563in}{2.629287in}%
\pgfsys@useobject{currentmarker}{}%
\end{pgfscope}%
\begin{pgfscope}%
\pgfsys@transformshift{1.515896in}{2.660598in}%
\pgfsys@useobject{currentmarker}{}%
\end{pgfscope}%
\begin{pgfscope}%
\pgfsys@transformshift{1.524860in}{2.694630in}%
\pgfsys@useobject{currentmarker}{}%
\end{pgfscope}%
\begin{pgfscope}%
\pgfsys@transformshift{1.520636in}{2.730608in}%
\pgfsys@useobject{currentmarker}{}%
\end{pgfscope}%
\begin{pgfscope}%
\pgfsys@transformshift{1.524729in}{2.768440in}%
\pgfsys@useobject{currentmarker}{}%
\end{pgfscope}%
\begin{pgfscope}%
\pgfsys@transformshift{1.528215in}{2.789076in}%
\pgfsys@useobject{currentmarker}{}%
\end{pgfscope}%
\begin{pgfscope}%
\pgfsys@transformshift{1.524859in}{2.800087in}%
\pgfsys@useobject{currentmarker}{}%
\end{pgfscope}%
\begin{pgfscope}%
\pgfsys@transformshift{1.529432in}{2.813863in}%
\pgfsys@useobject{currentmarker}{}%
\end{pgfscope}%
\begin{pgfscope}%
\pgfsys@transformshift{1.526199in}{2.830935in}%
\pgfsys@useobject{currentmarker}{}%
\end{pgfscope}%
\begin{pgfscope}%
\pgfsys@transformshift{1.532815in}{2.852270in}%
\pgfsys@useobject{currentmarker}{}%
\end{pgfscope}%
\begin{pgfscope}%
\pgfsys@transformshift{1.528144in}{2.875199in}%
\pgfsys@useobject{currentmarker}{}%
\end{pgfscope}%
\begin{pgfscope}%
\pgfsys@transformshift{1.535277in}{2.899792in}%
\pgfsys@useobject{currentmarker}{}%
\end{pgfscope}%
\begin{pgfscope}%
\pgfsys@transformshift{1.533569in}{2.926376in}%
\pgfsys@useobject{currentmarker}{}%
\end{pgfscope}%
\begin{pgfscope}%
\pgfsys@transformshift{1.535898in}{2.954707in}%
\pgfsys@useobject{currentmarker}{}%
\end{pgfscope}%
\begin{pgfscope}%
\pgfsys@transformshift{1.538628in}{2.984782in}%
\pgfsys@useobject{currentmarker}{}%
\end{pgfscope}%
\begin{pgfscope}%
\pgfsys@transformshift{1.532130in}{3.016544in}%
\pgfsys@useobject{currentmarker}{}%
\end{pgfscope}%
\begin{pgfscope}%
\pgfsys@transformshift{1.537052in}{3.033682in}%
\pgfsys@useobject{currentmarker}{}%
\end{pgfscope}%
\begin{pgfscope}%
\pgfsys@transformshift{1.531119in}{3.054295in}%
\pgfsys@useobject{currentmarker}{}%
\end{pgfscope}%
\begin{pgfscope}%
\pgfsys@transformshift{1.538201in}{3.077816in}%
\pgfsys@useobject{currentmarker}{}%
\end{pgfscope}%
\begin{pgfscope}%
\pgfsys@transformshift{1.531523in}{3.104730in}%
\pgfsys@useobject{currentmarker}{}%
\end{pgfscope}%
\begin{pgfscope}%
\pgfsys@transformshift{1.539559in}{3.132329in}%
\pgfsys@useobject{currentmarker}{}%
\end{pgfscope}%
\begin{pgfscope}%
\pgfsys@transformshift{1.534634in}{3.162528in}%
\pgfsys@useobject{currentmarker}{}%
\end{pgfscope}%
\begin{pgfscope}%
\pgfsys@transformshift{1.542839in}{3.194555in}%
\pgfsys@useobject{currentmarker}{}%
\end{pgfscope}%
\begin{pgfscope}%
\pgfsys@transformshift{1.540153in}{3.228387in}%
\pgfsys@useobject{currentmarker}{}%
\end{pgfscope}%
\begin{pgfscope}%
\pgfsys@transformshift{1.548372in}{3.263158in}%
\pgfsys@useobject{currentmarker}{}%
\end{pgfscope}%
\begin{pgfscope}%
\pgfsys@transformshift{1.550030in}{3.282739in}%
\pgfsys@useobject{currentmarker}{}%
\end{pgfscope}%
\begin{pgfscope}%
\pgfsys@transformshift{1.548948in}{3.304134in}%
\pgfsys@useobject{currentmarker}{}%
\end{pgfscope}%
\begin{pgfscope}%
\pgfsys@transformshift{1.556949in}{3.325641in}%
\pgfsys@useobject{currentmarker}{}%
\end{pgfscope}%
\begin{pgfscope}%
\pgfsys@transformshift{1.550878in}{3.352426in}%
\pgfsys@useobject{currentmarker}{}%
\end{pgfscope}%
\begin{pgfscope}%
\pgfsys@transformshift{1.560747in}{3.380021in}%
\pgfsys@useobject{currentmarker}{}%
\end{pgfscope}%
\begin{pgfscope}%
\pgfsys@transformshift{1.554025in}{3.412964in}%
\pgfsys@useobject{currentmarker}{}%
\end{pgfscope}%
\begin{pgfscope}%
\pgfsys@transformshift{1.565445in}{3.446894in}%
\pgfsys@useobject{currentmarker}{}%
\end{pgfscope}%
\begin{pgfscope}%
\pgfsys@transformshift{1.561383in}{3.484786in}%
\pgfsys@useobject{currentmarker}{}%
\end{pgfscope}%
\begin{pgfscope}%
\pgfsys@transformshift{1.569816in}{3.522981in}%
\pgfsys@useobject{currentmarker}{}%
\end{pgfscope}%
\begin{pgfscope}%
\pgfsys@transformshift{1.571073in}{3.544457in}%
\pgfsys@useobject{currentmarker}{}%
\end{pgfscope}%
\begin{pgfscope}%
\pgfsys@transformshift{1.570785in}{3.556285in}%
\pgfsys@useobject{currentmarker}{}%
\end{pgfscope}%
\begin{pgfscope}%
\pgfsys@transformshift{1.573307in}{3.562285in}%
\pgfsys@useobject{currentmarker}{}%
\end{pgfscope}%
\begin{pgfscope}%
\pgfsys@transformshift{1.571594in}{3.573927in}%
\pgfsys@useobject{currentmarker}{}%
\end{pgfscope}%
\begin{pgfscope}%
\pgfsys@transformshift{1.577104in}{3.586850in}%
\pgfsys@useobject{currentmarker}{}%
\end{pgfscope}%
\begin{pgfscope}%
\pgfsys@transformshift{1.574221in}{3.604644in}%
\pgfsys@useobject{currentmarker}{}%
\end{pgfscope}%
\begin{pgfscope}%
\pgfsys@transformshift{1.577661in}{3.613942in}%
\pgfsys@useobject{currentmarker}{}%
\end{pgfscope}%
\begin{pgfscope}%
\pgfsys@transformshift{1.577810in}{3.619393in}%
\pgfsys@useobject{currentmarker}{}%
\end{pgfscope}%
\begin{pgfscope}%
\pgfsys@transformshift{1.578817in}{3.626619in}%
\pgfsys@useobject{currentmarker}{}%
\end{pgfscope}%
\begin{pgfscope}%
\pgfsys@transformshift{1.580893in}{3.635856in}%
\pgfsys@useobject{currentmarker}{}%
\end{pgfscope}%
\begin{pgfscope}%
\pgfsys@transformshift{1.576646in}{3.648671in}%
\pgfsys@useobject{currentmarker}{}%
\end{pgfscope}%
\begin{pgfscope}%
\pgfsys@transformshift{1.582258in}{3.663785in}%
\pgfsys@useobject{currentmarker}{}%
\end{pgfscope}%
\begin{pgfscope}%
\pgfsys@transformshift{1.579339in}{3.682499in}%
\pgfsys@useobject{currentmarker}{}%
\end{pgfscope}%
\begin{pgfscope}%
\pgfsys@transformshift{1.582102in}{3.692543in}%
\pgfsys@useobject{currentmarker}{}%
\end{pgfscope}%
\begin{pgfscope}%
\pgfsys@transformshift{1.581984in}{3.698271in}%
\pgfsys@useobject{currentmarker}{}%
\end{pgfscope}%
\begin{pgfscope}%
\pgfsys@transformshift{1.581870in}{3.701420in}%
\pgfsys@useobject{currentmarker}{}%
\end{pgfscope}%
\begin{pgfscope}%
\pgfsys@transformshift{1.582510in}{3.703031in}%
\pgfsys@useobject{currentmarker}{}%
\end{pgfscope}%
\begin{pgfscope}%
\pgfsys@transformshift{1.581598in}{3.707609in}%
\pgfsys@useobject{currentmarker}{}%
\end{pgfscope}%
\begin{pgfscope}%
\pgfsys@transformshift{1.584394in}{3.715430in}%
\pgfsys@useobject{currentmarker}{}%
\end{pgfscope}%
\begin{pgfscope}%
\pgfsys@transformshift{1.583958in}{3.726391in}%
\pgfsys@useobject{currentmarker}{}%
\end{pgfscope}%
\begin{pgfscope}%
\pgfsys@transformshift{1.585726in}{3.732159in}%
\pgfsys@useobject{currentmarker}{}%
\end{pgfscope}%
\begin{pgfscope}%
\pgfsys@transformshift{1.586167in}{3.735448in}%
\pgfsys@useobject{currentmarker}{}%
\end{pgfscope}%
\begin{pgfscope}%
\pgfsys@transformshift{1.585791in}{3.737234in}%
\pgfsys@useobject{currentmarker}{}%
\end{pgfscope}%
\begin{pgfscope}%
\pgfsys@transformshift{1.587981in}{3.741938in}%
\pgfsys@useobject{currentmarker}{}%
\end{pgfscope}%
\begin{pgfscope}%
\pgfsys@transformshift{1.586827in}{3.749125in}%
\pgfsys@useobject{currentmarker}{}%
\end{pgfscope}%
\begin{pgfscope}%
\pgfsys@transformshift{1.590315in}{3.758820in}%
\pgfsys@useobject{currentmarker}{}%
\end{pgfscope}%
\begin{pgfscope}%
\pgfsys@transformshift{1.589313in}{3.770649in}%
\pgfsys@useobject{currentmarker}{}%
\end{pgfscope}%
\begin{pgfscope}%
\pgfsys@transformshift{1.591638in}{3.776751in}%
\pgfsys@useobject{currentmarker}{}%
\end{pgfscope}%
\begin{pgfscope}%
\pgfsys@transformshift{1.592168in}{3.780303in}%
\pgfsys@useobject{currentmarker}{}%
\end{pgfscope}%
\begin{pgfscope}%
\pgfsys@transformshift{1.591912in}{3.782261in}%
\pgfsys@useobject{currentmarker}{}%
\end{pgfscope}%
\begin{pgfscope}%
\pgfsys@transformshift{1.594419in}{3.787430in}%
\pgfsys@useobject{currentmarker}{}%
\end{pgfscope}%
\begin{pgfscope}%
\pgfsys@transformshift{1.594154in}{3.790579in}%
\pgfsys@useobject{currentmarker}{}%
\end{pgfscope}%
\begin{pgfscope}%
\pgfsys@transformshift{1.594723in}{3.792221in}%
\pgfsys@useobject{currentmarker}{}%
\end{pgfscope}%
\begin{pgfscope}%
\pgfsys@transformshift{1.594741in}{3.793177in}%
\pgfsys@useobject{currentmarker}{}%
\end{pgfscope}%
\begin{pgfscope}%
\pgfsys@transformshift{1.594849in}{3.793691in}%
\pgfsys@useobject{currentmarker}{}%
\end{pgfscope}%
\begin{pgfscope}%
\pgfsys@transformshift{1.594937in}{3.795350in}%
\pgfsys@useobject{currentmarker}{}%
\end{pgfscope}%
\begin{pgfscope}%
\pgfsys@transformshift{1.595057in}{3.796255in}%
\pgfsys@useobject{currentmarker}{}%
\end{pgfscope}%
\begin{pgfscope}%
\pgfsys@transformshift{1.594757in}{3.799313in}%
\pgfsys@useobject{currentmarker}{}%
\end{pgfscope}%
\begin{pgfscope}%
\pgfsys@transformshift{1.595937in}{3.803496in}%
\pgfsys@useobject{currentmarker}{}%
\end{pgfscope}%
\begin{pgfscope}%
\pgfsys@transformshift{1.595511in}{3.805848in}%
\pgfsys@useobject{currentmarker}{}%
\end{pgfscope}%
\begin{pgfscope}%
\pgfsys@transformshift{1.595868in}{3.807113in}%
\pgfsys@useobject{currentmarker}{}%
\end{pgfscope}%
\begin{pgfscope}%
\pgfsys@transformshift{1.595573in}{3.809696in}%
\pgfsys@useobject{currentmarker}{}%
\end{pgfscope}%
\begin{pgfscope}%
\pgfsys@transformshift{1.597394in}{3.813810in}%
\pgfsys@useobject{currentmarker}{}%
\end{pgfscope}%
\begin{pgfscope}%
\pgfsys@transformshift{1.597176in}{3.816274in}%
\pgfsys@useobject{currentmarker}{}%
\end{pgfscope}%
\begin{pgfscope}%
\pgfsys@transformshift{1.597605in}{3.817566in}%
\pgfsys@useobject{currentmarker}{}%
\end{pgfscope}%
\begin{pgfscope}%
\pgfsys@transformshift{1.597599in}{3.818314in}%
\pgfsys@useobject{currentmarker}{}%
\end{pgfscope}%
\begin{pgfscope}%
\pgfsys@transformshift{1.597722in}{3.818707in}%
\pgfsys@useobject{currentmarker}{}%
\end{pgfscope}%
\begin{pgfscope}%
\pgfsys@transformshift{1.597711in}{3.818933in}%
\pgfsys@useobject{currentmarker}{}%
\end{pgfscope}%
\begin{pgfscope}%
\pgfsys@transformshift{1.598181in}{3.820151in}%
\pgfsys@useobject{currentmarker}{}%
\end{pgfscope}%
\begin{pgfscope}%
\pgfsys@transformshift{1.597675in}{3.822312in}%
\pgfsys@useobject{currentmarker}{}%
\end{pgfscope}%
\begin{pgfscope}%
\pgfsys@transformshift{1.599485in}{3.827489in}%
\pgfsys@useobject{currentmarker}{}%
\end{pgfscope}%
\begin{pgfscope}%
\pgfsys@transformshift{1.598833in}{3.833865in}%
\pgfsys@useobject{currentmarker}{}%
\end{pgfscope}%
\begin{pgfscope}%
\pgfsys@transformshift{1.601486in}{3.841636in}%
\pgfsys@useobject{currentmarker}{}%
\end{pgfscope}%
\begin{pgfscope}%
\pgfsys@transformshift{1.601256in}{3.846147in}%
\pgfsys@useobject{currentmarker}{}%
\end{pgfscope}%
\begin{pgfscope}%
\pgfsys@transformshift{1.601293in}{3.848631in}%
\pgfsys@useobject{currentmarker}{}%
\end{pgfscope}%
\begin{pgfscope}%
\pgfsys@transformshift{1.602765in}{3.852179in}%
\pgfsys@useobject{currentmarker}{}%
\end{pgfscope}%
\begin{pgfscope}%
\pgfsys@transformshift{1.601815in}{3.858623in}%
\pgfsys@useobject{currentmarker}{}%
\end{pgfscope}%
\begin{pgfscope}%
\pgfsys@transformshift{1.602720in}{3.862090in}%
\pgfsys@useobject{currentmarker}{}%
\end{pgfscope}%
\begin{pgfscope}%
\pgfsys@transformshift{1.602812in}{3.864058in}%
\pgfsys@useobject{currentmarker}{}%
\end{pgfscope}%
\begin{pgfscope}%
\pgfsys@transformshift{1.602667in}{3.865132in}%
\pgfsys@useobject{currentmarker}{}%
\end{pgfscope}%
\begin{pgfscope}%
\pgfsys@transformshift{1.604270in}{3.870027in}%
\pgfsys@useobject{currentmarker}{}%
\end{pgfscope}%
\begin{pgfscope}%
\pgfsys@transformshift{1.604403in}{3.876273in}%
\pgfsys@useobject{currentmarker}{}%
\end{pgfscope}%
\begin{pgfscope}%
\pgfsys@transformshift{1.604241in}{3.883863in}%
\pgfsys@useobject{currentmarker}{}%
\end{pgfscope}%
\begin{pgfscope}%
\pgfsys@transformshift{1.607581in}{3.893501in}%
\pgfsys@useobject{currentmarker}{}%
\end{pgfscope}%
\begin{pgfscope}%
\pgfsys@transformshift{1.606353in}{3.904509in}%
\pgfsys@useobject{currentmarker}{}%
\end{pgfscope}%
\begin{pgfscope}%
\pgfsys@transformshift{1.608004in}{3.910373in}%
\pgfsys@useobject{currentmarker}{}%
\end{pgfscope}%
\begin{pgfscope}%
\pgfsys@transformshift{1.609074in}{3.917304in}%
\pgfsys@useobject{currentmarker}{}%
\end{pgfscope}%
\begin{pgfscope}%
\pgfsys@transformshift{1.607431in}{3.927702in}%
\pgfsys@useobject{currentmarker}{}%
\end{pgfscope}%
\begin{pgfscope}%
\pgfsys@transformshift{1.612545in}{3.940416in}%
\pgfsys@useobject{currentmarker}{}%
\end{pgfscope}%
\begin{pgfscope}%
\pgfsys@transformshift{1.611967in}{3.947931in}%
\pgfsys@useobject{currentmarker}{}%
\end{pgfscope}%
\begin{pgfscope}%
\pgfsys@transformshift{1.612393in}{3.956778in}%
\pgfsys@useobject{currentmarker}{}%
\end{pgfscope}%
\begin{pgfscope}%
\pgfsys@transformshift{1.614217in}{3.961295in}%
\pgfsys@useobject{currentmarker}{}%
\end{pgfscope}%
\begin{pgfscope}%
\pgfsys@transformshift{1.613576in}{3.968422in}%
\pgfsys@useobject{currentmarker}{}%
\end{pgfscope}%
\begin{pgfscope}%
\pgfsys@transformshift{1.614256in}{3.972298in}%
\pgfsys@useobject{currentmarker}{}%
\end{pgfscope}%
\begin{pgfscope}%
\pgfsys@transformshift{1.614817in}{3.974388in}%
\pgfsys@useobject{currentmarker}{}%
\end{pgfscope}%
\begin{pgfscope}%
\pgfsys@transformshift{1.614558in}{3.979491in}%
\pgfsys@useobject{currentmarker}{}%
\end{pgfscope}%
\begin{pgfscope}%
\pgfsys@transformshift{1.615624in}{3.982092in}%
\pgfsys@useobject{currentmarker}{}%
\end{pgfscope}%
\begin{pgfscope}%
\pgfsys@transformshift{1.615846in}{3.985791in}%
\pgfsys@useobject{currentmarker}{}%
\end{pgfscope}%
\begin{pgfscope}%
\pgfsys@transformshift{1.615857in}{3.987829in}%
\pgfsys@useobject{currentmarker}{}%
\end{pgfscope}%
\begin{pgfscope}%
\pgfsys@transformshift{1.617735in}{3.993047in}%
\pgfsys@useobject{currentmarker}{}%
\end{pgfscope}%
\begin{pgfscope}%
\pgfsys@transformshift{1.617555in}{3.996091in}%
\pgfsys@useobject{currentmarker}{}%
\end{pgfscope}%
\begin{pgfscope}%
\pgfsys@transformshift{1.617905in}{4.000685in}%
\pgfsys@useobject{currentmarker}{}%
\end{pgfscope}%
\begin{pgfscope}%
\pgfsys@transformshift{1.618776in}{4.003064in}%
\pgfsys@useobject{currentmarker}{}%
\end{pgfscope}%
\begin{pgfscope}%
\pgfsys@transformshift{1.618226in}{4.007983in}%
\pgfsys@useobject{currentmarker}{}%
\end{pgfscope}%
\begin{pgfscope}%
\pgfsys@transformshift{1.620010in}{4.014627in}%
\pgfsys@useobject{currentmarker}{}%
\end{pgfscope}%
\begin{pgfscope}%
\pgfsys@transformshift{1.619886in}{4.018409in}%
\pgfsys@useobject{currentmarker}{}%
\end{pgfscope}%
\begin{pgfscope}%
\pgfsys@transformshift{1.620356in}{4.020436in}%
\pgfsys@useobject{currentmarker}{}%
\end{pgfscope}%
\begin{pgfscope}%
\pgfsys@transformshift{1.620405in}{4.021580in}%
\pgfsys@useobject{currentmarker}{}%
\end{pgfscope}%
\begin{pgfscope}%
\pgfsys@transformshift{1.620741in}{4.023666in}%
\pgfsys@useobject{currentmarker}{}%
\end{pgfscope}%
\begin{pgfscope}%
\pgfsys@transformshift{1.620780in}{4.024827in}%
\pgfsys@useobject{currentmarker}{}%
\end{pgfscope}%
\begin{pgfscope}%
\pgfsys@transformshift{1.620801in}{4.025466in}%
\pgfsys@useobject{currentmarker}{}%
\end{pgfscope}%
\begin{pgfscope}%
\pgfsys@transformshift{1.620409in}{4.027240in}%
\pgfsys@useobject{currentmarker}{}%
\end{pgfscope}%
\begin{pgfscope}%
\pgfsys@transformshift{1.618640in}{4.031611in}%
\pgfsys@useobject{currentmarker}{}%
\end{pgfscope}%
\begin{pgfscope}%
\pgfsys@transformshift{1.614634in}{4.036323in}%
\pgfsys@useobject{currentmarker}{}%
\end{pgfscope}%
\begin{pgfscope}%
\pgfsys@transformshift{1.612391in}{4.038880in}%
\pgfsys@useobject{currentmarker}{}%
\end{pgfscope}%
\begin{pgfscope}%
\pgfsys@transformshift{1.607814in}{4.040310in}%
\pgfsys@useobject{currentmarker}{}%
\end{pgfscope}%
\begin{pgfscope}%
\pgfsys@transformshift{1.602128in}{4.042665in}%
\pgfsys@useobject{currentmarker}{}%
\end{pgfscope}%
\begin{pgfscope}%
\pgfsys@transformshift{1.598744in}{4.042701in}%
\pgfsys@useobject{currentmarker}{}%
\end{pgfscope}%
\begin{pgfscope}%
\pgfsys@transformshift{1.596906in}{4.043000in}%
\pgfsys@useobject{currentmarker}{}%
\end{pgfscope}%
\begin{pgfscope}%
\pgfsys@transformshift{1.595893in}{4.042851in}%
\pgfsys@useobject{currentmarker}{}%
\end{pgfscope}%
\begin{pgfscope}%
\pgfsys@transformshift{1.593989in}{4.042931in}%
\pgfsys@useobject{currentmarker}{}%
\end{pgfscope}%
\begin{pgfscope}%
\pgfsys@transformshift{1.592944in}{4.042854in}%
\pgfsys@useobject{currentmarker}{}%
\end{pgfscope}%
\begin{pgfscope}%
\pgfsys@transformshift{1.592368in}{4.042867in}%
\pgfsys@useobject{currentmarker}{}%
\end{pgfscope}%
\begin{pgfscope}%
\pgfsys@transformshift{1.592052in}{4.042843in}%
\pgfsys@useobject{currentmarker}{}%
\end{pgfscope}%
\begin{pgfscope}%
\pgfsys@transformshift{1.590747in}{4.042892in}%
\pgfsys@useobject{currentmarker}{}%
\end{pgfscope}%
\begin{pgfscope}%
\pgfsys@transformshift{1.585541in}{4.041851in}%
\pgfsys@useobject{currentmarker}{}%
\end{pgfscope}%
\begin{pgfscope}%
\pgfsys@transformshift{1.578299in}{4.042491in}%
\pgfsys@useobject{currentmarker}{}%
\end{pgfscope}%
\begin{pgfscope}%
\pgfsys@transformshift{1.568957in}{4.041101in}%
\pgfsys@useobject{currentmarker}{}%
\end{pgfscope}%
\begin{pgfscope}%
\pgfsys@transformshift{1.558229in}{4.040810in}%
\pgfsys@useobject{currentmarker}{}%
\end{pgfscope}%
\begin{pgfscope}%
\pgfsys@transformshift{1.552366in}{4.040132in}%
\pgfsys@useobject{currentmarker}{}%
\end{pgfscope}%
\begin{pgfscope}%
\pgfsys@transformshift{1.549122in}{4.040008in}%
\pgfsys@useobject{currentmarker}{}%
\end{pgfscope}%
\begin{pgfscope}%
\pgfsys@transformshift{1.544931in}{4.039446in}%
\pgfsys@useobject{currentmarker}{}%
\end{pgfscope}%
\begin{pgfscope}%
\pgfsys@transformshift{1.542606in}{4.039433in}%
\pgfsys@useobject{currentmarker}{}%
\end{pgfscope}%
\begin{pgfscope}%
\pgfsys@transformshift{1.541345in}{4.039218in}%
\pgfsys@useobject{currentmarker}{}%
\end{pgfscope}%
\begin{pgfscope}%
\pgfsys@transformshift{1.537862in}{4.039043in}%
\pgfsys@useobject{currentmarker}{}%
\end{pgfscope}%
\begin{pgfscope}%
\pgfsys@transformshift{1.535978in}{4.038684in}%
\pgfsys@useobject{currentmarker}{}%
\end{pgfscope}%
\begin{pgfscope}%
\pgfsys@transformshift{1.528144in}{4.038370in}%
\pgfsys@useobject{currentmarker}{}%
\end{pgfscope}%
\begin{pgfscope}%
\pgfsys@transformshift{1.519028in}{4.037738in}%
\pgfsys@useobject{currentmarker}{}%
\end{pgfscope}%
\begin{pgfscope}%
\pgfsys@transformshift{1.514176in}{4.039049in}%
\pgfsys@useobject{currentmarker}{}%
\end{pgfscope}%
\begin{pgfscope}%
\pgfsys@transformshift{1.507213in}{4.038947in}%
\pgfsys@useobject{currentmarker}{}%
\end{pgfscope}%
\begin{pgfscope}%
\pgfsys@transformshift{1.495665in}{4.040757in}%
\pgfsys@useobject{currentmarker}{}%
\end{pgfscope}%
\begin{pgfscope}%
\pgfsys@transformshift{1.489239in}{4.040546in}%
\pgfsys@useobject{currentmarker}{}%
\end{pgfscope}%
\begin{pgfscope}%
\pgfsys@transformshift{1.489554in}{4.040758in}%
\pgfsys@useobject{currentmarker}{}%
\end{pgfscope}%
\begin{pgfscope}%
\pgfsys@transformshift{1.500586in}{4.038745in}%
\pgfsys@useobject{currentmarker}{}%
\end{pgfscope}%
\begin{pgfscope}%
\pgfsys@transformshift{1.518699in}{4.037190in}%
\pgfsys@useobject{currentmarker}{}%
\end{pgfscope}%
\begin{pgfscope}%
\pgfsys@transformshift{1.538299in}{4.035649in}%
\pgfsys@useobject{currentmarker}{}%
\end{pgfscope}%
\begin{pgfscope}%
\pgfsys@transformshift{1.559705in}{4.037663in}%
\pgfsys@useobject{currentmarker}{}%
\end{pgfscope}%
\begin{pgfscope}%
\pgfsys@transformshift{1.582197in}{4.036814in}%
\pgfsys@useobject{currentmarker}{}%
\end{pgfscope}%
\begin{pgfscope}%
\pgfsys@transformshift{1.609137in}{4.039490in}%
\pgfsys@useobject{currentmarker}{}%
\end{pgfscope}%
\begin{pgfscope}%
\pgfsys@transformshift{1.623965in}{4.038140in}%
\pgfsys@useobject{currentmarker}{}%
\end{pgfscope}%
\begin{pgfscope}%
\pgfsys@transformshift{1.640856in}{4.037899in}%
\pgfsys@useobject{currentmarker}{}%
\end{pgfscope}%
\begin{pgfscope}%
\pgfsys@transformshift{1.659445in}{4.036506in}%
\pgfsys@useobject{currentmarker}{}%
\end{pgfscope}%
\begin{pgfscope}%
\pgfsys@transformshift{1.686053in}{4.035214in}%
\pgfsys@useobject{currentmarker}{}%
\end{pgfscope}%
\begin{pgfscope}%
\pgfsys@transformshift{1.715836in}{4.035828in}%
\pgfsys@useobject{currentmarker}{}%
\end{pgfscope}%
\begin{pgfscope}%
\pgfsys@transformshift{1.747377in}{4.034818in}%
\pgfsys@useobject{currentmarker}{}%
\end{pgfscope}%
\begin{pgfscope}%
\pgfsys@transformshift{1.779856in}{4.031787in}%
\pgfsys@useobject{currentmarker}{}%
\end{pgfscope}%
\begin{pgfscope}%
\pgfsys@transformshift{1.817289in}{4.031687in}%
\pgfsys@useobject{currentmarker}{}%
\end{pgfscope}%
\begin{pgfscope}%
\pgfsys@transformshift{1.837803in}{4.029953in}%
\pgfsys@useobject{currentmarker}{}%
\end{pgfscope}%
\begin{pgfscope}%
\pgfsys@transformshift{1.860034in}{4.028526in}%
\pgfsys@useobject{currentmarker}{}%
\end{pgfscope}%
\begin{pgfscope}%
\pgfsys@transformshift{1.872101in}{4.026400in}%
\pgfsys@useobject{currentmarker}{}%
\end{pgfscope}%
\begin{pgfscope}%
\pgfsys@transformshift{1.889526in}{4.024566in}%
\pgfsys@useobject{currentmarker}{}%
\end{pgfscope}%
\begin{pgfscope}%
\pgfsys@transformshift{1.909473in}{4.023820in}%
\pgfsys@useobject{currentmarker}{}%
\end{pgfscope}%
\begin{pgfscope}%
\pgfsys@transformshift{1.933053in}{4.023356in}%
\pgfsys@useobject{currentmarker}{}%
\end{pgfscope}%
\begin{pgfscope}%
\pgfsys@transformshift{1.958210in}{4.021943in}%
\pgfsys@useobject{currentmarker}{}%
\end{pgfscope}%
\begin{pgfscope}%
\pgfsys@transformshift{1.987860in}{4.020694in}%
\pgfsys@useobject{currentmarker}{}%
\end{pgfscope}%
\begin{pgfscope}%
\pgfsys@transformshift{2.018432in}{4.019739in}%
\pgfsys@useobject{currentmarker}{}%
\end{pgfscope}%
\begin{pgfscope}%
\pgfsys@transformshift{2.035218in}{4.018638in}%
\pgfsys@useobject{currentmarker}{}%
\end{pgfscope}%
\begin{pgfscope}%
\pgfsys@transformshift{2.052934in}{4.020194in}%
\pgfsys@useobject{currentmarker}{}%
\end{pgfscope}%
\begin{pgfscope}%
\pgfsys@transformshift{2.062693in}{4.019529in}%
\pgfsys@useobject{currentmarker}{}%
\end{pgfscope}%
\begin{pgfscope}%
\pgfsys@transformshift{2.077567in}{4.020655in}%
\pgfsys@useobject{currentmarker}{}%
\end{pgfscope}%
\begin{pgfscope}%
\pgfsys@transformshift{2.093571in}{4.018745in}%
\pgfsys@useobject{currentmarker}{}%
\end{pgfscope}%
\begin{pgfscope}%
\pgfsys@transformshift{2.111367in}{4.018502in}%
\pgfsys@useobject{currentmarker}{}%
\end{pgfscope}%
\begin{pgfscope}%
\pgfsys@transformshift{2.121115in}{4.017610in}%
\pgfsys@useobject{currentmarker}{}%
\end{pgfscope}%
\begin{pgfscope}%
\pgfsys@transformshift{2.135832in}{4.016305in}%
\pgfsys@useobject{currentmarker}{}%
\end{pgfscope}%
\begin{pgfscope}%
\pgfsys@transformshift{2.152296in}{4.016409in}%
\pgfsys@useobject{currentmarker}{}%
\end{pgfscope}%
\begin{pgfscope}%
\pgfsys@transformshift{2.169814in}{4.013700in}%
\pgfsys@useobject{currentmarker}{}%
\end{pgfscope}%
\begin{pgfscope}%
\pgfsys@transformshift{2.188832in}{4.012763in}%
\pgfsys@useobject{currentmarker}{}%
\end{pgfscope}%
\begin{pgfscope}%
\pgfsys@transformshift{2.212512in}{4.009315in}%
\pgfsys@useobject{currentmarker}{}%
\end{pgfscope}%
\begin{pgfscope}%
\pgfsys@transformshift{2.225595in}{4.010751in}%
\pgfsys@useobject{currentmarker}{}%
\end{pgfscope}%
\begin{pgfscope}%
\pgfsys@transformshift{2.240705in}{4.009679in}%
\pgfsys@useobject{currentmarker}{}%
\end{pgfscope}%
\begin{pgfscope}%
\pgfsys@transformshift{2.258857in}{4.010242in}%
\pgfsys@useobject{currentmarker}{}%
\end{pgfscope}%
\begin{pgfscope}%
\pgfsys@transformshift{2.283387in}{4.006014in}%
\pgfsys@useobject{currentmarker}{}%
\end{pgfscope}%
\begin{pgfscope}%
\pgfsys@transformshift{2.310928in}{4.008881in}%
\pgfsys@useobject{currentmarker}{}%
\end{pgfscope}%
\begin{pgfscope}%
\pgfsys@transformshift{2.340188in}{4.004499in}%
\pgfsys@useobject{currentmarker}{}%
\end{pgfscope}%
\begin{pgfscope}%
\pgfsys@transformshift{2.370619in}{4.002319in}%
\pgfsys@useobject{currentmarker}{}%
\end{pgfscope}%
\begin{pgfscope}%
\pgfsys@transformshift{2.404389in}{3.997996in}%
\pgfsys@useobject{currentmarker}{}%
\end{pgfscope}%
\begin{pgfscope}%
\pgfsys@transformshift{2.439534in}{3.994994in}%
\pgfsys@useobject{currentmarker}{}%
\end{pgfscope}%
\begin{pgfscope}%
\pgfsys@transformshift{2.458931in}{3.994658in}%
\pgfsys@useobject{currentmarker}{}%
\end{pgfscope}%
\begin{pgfscope}%
\pgfsys@transformshift{2.469591in}{3.995122in}%
\pgfsys@useobject{currentmarker}{}%
\end{pgfscope}%
\begin{pgfscope}%
\pgfsys@transformshift{2.481956in}{3.994209in}%
\pgfsys@useobject{currentmarker}{}%
\end{pgfscope}%
\begin{pgfscope}%
\pgfsys@transformshift{2.496308in}{3.994642in}%
\pgfsys@useobject{currentmarker}{}%
\end{pgfscope}%
\begin{pgfscope}%
\pgfsys@transformshift{2.504205in}{3.994659in}%
\pgfsys@useobject{currentmarker}{}%
\end{pgfscope}%
\begin{pgfscope}%
\pgfsys@transformshift{2.508529in}{3.994251in}%
\pgfsys@useobject{currentmarker}{}%
\end{pgfscope}%
\begin{pgfscope}%
\pgfsys@transformshift{2.514263in}{3.994903in}%
\pgfsys@useobject{currentmarker}{}%
\end{pgfscope}%
\begin{pgfscope}%
\pgfsys@transformshift{2.517423in}{3.994602in}%
\pgfsys@useobject{currentmarker}{}%
\end{pgfscope}%
\begin{pgfscope}%
\pgfsys@transformshift{2.525137in}{3.994655in}%
\pgfsys@useobject{currentmarker}{}%
\end{pgfscope}%
\begin{pgfscope}%
\pgfsys@transformshift{2.529368in}{3.994350in}%
\pgfsys@useobject{currentmarker}{}%
\end{pgfscope}%
\begin{pgfscope}%
\pgfsys@transformshift{2.531699in}{3.994233in}%
\pgfsys@useobject{currentmarker}{}%
\end{pgfscope}%
\begin{pgfscope}%
\pgfsys@transformshift{2.537545in}{3.993197in}%
\pgfsys@useobject{currentmarker}{}%
\end{pgfscope}%
\begin{pgfscope}%
\pgfsys@transformshift{2.546977in}{3.992015in}%
\pgfsys@useobject{currentmarker}{}%
\end{pgfscope}%
\begin{pgfscope}%
\pgfsys@transformshift{2.552204in}{3.991882in}%
\pgfsys@useobject{currentmarker}{}%
\end{pgfscope}%
\begin{pgfscope}%
\pgfsys@transformshift{2.560882in}{3.990827in}%
\pgfsys@useobject{currentmarker}{}%
\end{pgfscope}%
\begin{pgfscope}%
\pgfsys@transformshift{2.573181in}{3.990165in}%
\pgfsys@useobject{currentmarker}{}%
\end{pgfscope}%
\begin{pgfscope}%
\pgfsys@transformshift{2.587247in}{3.989238in}%
\pgfsys@useobject{currentmarker}{}%
\end{pgfscope}%
\begin{pgfscope}%
\pgfsys@transformshift{2.602427in}{3.987162in}%
\pgfsys@useobject{currentmarker}{}%
\end{pgfscope}%
\begin{pgfscope}%
\pgfsys@transformshift{2.619741in}{3.987284in}%
\pgfsys@useobject{currentmarker}{}%
\end{pgfscope}%
\begin{pgfscope}%
\pgfsys@transformshift{2.637881in}{3.986112in}%
\pgfsys@useobject{currentmarker}{}%
\end{pgfscope}%
\begin{pgfscope}%
\pgfsys@transformshift{2.659904in}{3.984381in}%
\pgfsys@useobject{currentmarker}{}%
\end{pgfscope}%
\begin{pgfscope}%
\pgfsys@transformshift{2.683146in}{3.982604in}%
\pgfsys@useobject{currentmarker}{}%
\end{pgfscope}%
\begin{pgfscope}%
\pgfsys@transformshift{2.710899in}{3.980369in}%
\pgfsys@useobject{currentmarker}{}%
\end{pgfscope}%
\begin{pgfscope}%
\pgfsys@transformshift{2.726167in}{3.979198in}%
\pgfsys@useobject{currentmarker}{}%
\end{pgfscope}%
\begin{pgfscope}%
\pgfsys@transformshift{2.734588in}{3.979358in}%
\pgfsys@useobject{currentmarker}{}%
\end{pgfscope}%
\begin{pgfscope}%
\pgfsys@transformshift{2.745279in}{3.978040in}%
\pgfsys@useobject{currentmarker}{}%
\end{pgfscope}%
\begin{pgfscope}%
\pgfsys@transformshift{2.751196in}{3.978338in}%
\pgfsys@useobject{currentmarker}{}%
\end{pgfscope}%
\begin{pgfscope}%
\pgfsys@transformshift{2.754399in}{3.977741in}%
\pgfsys@useobject{currentmarker}{}%
\end{pgfscope}%
\begin{pgfscope}%
\pgfsys@transformshift{2.761027in}{3.977353in}%
\pgfsys@useobject{currentmarker}{}%
\end{pgfscope}%
\begin{pgfscope}%
\pgfsys@transformshift{2.764671in}{3.977114in}%
\pgfsys@useobject{currentmarker}{}%
\end{pgfscope}%
\begin{pgfscope}%
\pgfsys@transformshift{2.770004in}{3.976775in}%
\pgfsys@useobject{currentmarker}{}%
\end{pgfscope}%
\begin{pgfscope}%
\pgfsys@transformshift{2.772942in}{3.976867in}%
\pgfsys@useobject{currentmarker}{}%
\end{pgfscope}%
\begin{pgfscope}%
\pgfsys@transformshift{2.774555in}{3.976972in}%
\pgfsys@useobject{currentmarker}{}%
\end{pgfscope}%
\begin{pgfscope}%
\pgfsys@transformshift{2.779321in}{3.976808in}%
\pgfsys@useobject{currentmarker}{}%
\end{pgfscope}%
\begin{pgfscope}%
\pgfsys@transformshift{2.788206in}{3.976154in}%
\pgfsys@useobject{currentmarker}{}%
\end{pgfscope}%
\begin{pgfscope}%
\pgfsys@transformshift{2.798532in}{3.975151in}%
\pgfsys@useobject{currentmarker}{}%
\end{pgfscope}%
\begin{pgfscope}%
\pgfsys@transformshift{2.809977in}{3.974012in}%
\pgfsys@useobject{currentmarker}{}%
\end{pgfscope}%
\begin{pgfscope}%
\pgfsys@transformshift{2.823083in}{3.975223in}%
\pgfsys@useobject{currentmarker}{}%
\end{pgfscope}%
\begin{pgfscope}%
\pgfsys@transformshift{2.830306in}{3.974743in}%
\pgfsys@useobject{currentmarker}{}%
\end{pgfscope}%
\begin{pgfscope}%
\pgfsys@transformshift{2.839017in}{3.974118in}%
\pgfsys@useobject{currentmarker}{}%
\end{pgfscope}%
\begin{pgfscope}%
\pgfsys@transformshift{2.852744in}{3.975283in}%
\pgfsys@useobject{currentmarker}{}%
\end{pgfscope}%
\begin{pgfscope}%
\pgfsys@transformshift{2.867641in}{3.973699in}%
\pgfsys@useobject{currentmarker}{}%
\end{pgfscope}%
\begin{pgfscope}%
\pgfsys@transformshift{2.875876in}{3.973995in}%
\pgfsys@useobject{currentmarker}{}%
\end{pgfscope}%
\begin{pgfscope}%
\pgfsys@transformshift{2.885978in}{3.972386in}%
\pgfsys@useobject{currentmarker}{}%
\end{pgfscope}%
\begin{pgfscope}%
\pgfsys@transformshift{2.891587in}{3.972825in}%
\pgfsys@useobject{currentmarker}{}%
\end{pgfscope}%
\begin{pgfscope}%
\pgfsys@transformshift{2.898510in}{3.971360in}%
\pgfsys@useobject{currentmarker}{}%
\end{pgfscope}%
\begin{pgfscope}%
\pgfsys@transformshift{2.902369in}{3.970858in}%
\pgfsys@useobject{currentmarker}{}%
\end{pgfscope}%
\begin{pgfscope}%
\pgfsys@transformshift{2.907943in}{3.970383in}%
\pgfsys@useobject{currentmarker}{}%
\end{pgfscope}%
\begin{pgfscope}%
\pgfsys@transformshift{2.915138in}{3.969824in}%
\pgfsys@useobject{currentmarker}{}%
\end{pgfscope}%
\begin{pgfscope}%
\pgfsys@transformshift{2.919107in}{3.969890in}%
\pgfsys@useobject{currentmarker}{}%
\end{pgfscope}%
\begin{pgfscope}%
\pgfsys@transformshift{2.924587in}{3.969397in}%
\pgfsys@useobject{currentmarker}{}%
\end{pgfscope}%
\begin{pgfscope}%
\pgfsys@transformshift{2.932792in}{3.968949in}%
\pgfsys@useobject{currentmarker}{}%
\end{pgfscope}%
\begin{pgfscope}%
\pgfsys@transformshift{2.946164in}{3.968433in}%
\pgfsys@useobject{currentmarker}{}%
\end{pgfscope}%
\begin{pgfscope}%
\pgfsys@transformshift{2.961575in}{3.967452in}%
\pgfsys@useobject{currentmarker}{}%
\end{pgfscope}%
\begin{pgfscope}%
\pgfsys@transformshift{2.978738in}{3.966477in}%
\pgfsys@useobject{currentmarker}{}%
\end{pgfscope}%
\begin{pgfscope}%
\pgfsys@transformshift{2.998921in}{3.966911in}%
\pgfsys@useobject{currentmarker}{}%
\end{pgfscope}%
\begin{pgfscope}%
\pgfsys@transformshift{3.021354in}{3.964090in}%
\pgfsys@useobject{currentmarker}{}%
\end{pgfscope}%
\begin{pgfscope}%
\pgfsys@transformshift{3.033783in}{3.963672in}%
\pgfsys@useobject{currentmarker}{}%
\end{pgfscope}%
\begin{pgfscope}%
\pgfsys@transformshift{3.047746in}{3.961773in}%
\pgfsys@useobject{currentmarker}{}%
\end{pgfscope}%
\begin{pgfscope}%
\pgfsys@transformshift{3.064326in}{3.961235in}%
\pgfsys@useobject{currentmarker}{}%
\end{pgfscope}%
\begin{pgfscope}%
\pgfsys@transformshift{3.083321in}{3.960409in}%
\pgfsys@useobject{currentmarker}{}%
\end{pgfscope}%
\begin{pgfscope}%
\pgfsys@transformshift{3.103991in}{3.957690in}%
\pgfsys@useobject{currentmarker}{}%
\end{pgfscope}%
\begin{pgfscope}%
\pgfsys@transformshift{3.128920in}{3.960913in}%
\pgfsys@useobject{currentmarker}{}%
\end{pgfscope}%
\begin{pgfscope}%
\pgfsys@transformshift{3.156442in}{3.956266in}%
\pgfsys@useobject{currentmarker}{}%
\end{pgfscope}%
\begin{pgfscope}%
\pgfsys@transformshift{3.171782in}{3.956875in}%
\pgfsys@useobject{currentmarker}{}%
\end{pgfscope}%
\begin{pgfscope}%
\pgfsys@transformshift{3.188495in}{3.954072in}%
\pgfsys@useobject{currentmarker}{}%
\end{pgfscope}%
\begin{pgfscope}%
\pgfsys@transformshift{3.209005in}{3.952752in}%
\pgfsys@useobject{currentmarker}{}%
\end{pgfscope}%
\begin{pgfscope}%
\pgfsys@transformshift{3.220279in}{3.951917in}%
\pgfsys@useobject{currentmarker}{}%
\end{pgfscope}%
\begin{pgfscope}%
\pgfsys@transformshift{3.233508in}{3.953438in}%
\pgfsys@useobject{currentmarker}{}%
\end{pgfscope}%
\begin{pgfscope}%
\pgfsys@transformshift{3.250479in}{3.951564in}%
\pgfsys@useobject{currentmarker}{}%
\end{pgfscope}%
\begin{pgfscope}%
\pgfsys@transformshift{3.271922in}{3.953819in}%
\pgfsys@useobject{currentmarker}{}%
\end{pgfscope}%
\begin{pgfscope}%
\pgfsys@transformshift{3.298250in}{3.947476in}%
\pgfsys@useobject{currentmarker}{}%
\end{pgfscope}%
\begin{pgfscope}%
\pgfsys@transformshift{3.326623in}{3.946663in}%
\pgfsys@useobject{currentmarker}{}%
\end{pgfscope}%
\begin{pgfscope}%
\pgfsys@transformshift{3.341900in}{3.943449in}%
\pgfsys@useobject{currentmarker}{}%
\end{pgfscope}%
\begin{pgfscope}%
\pgfsys@transformshift{3.358885in}{3.942162in}%
\pgfsys@useobject{currentmarker}{}%
\end{pgfscope}%
\begin{pgfscope}%
\pgfsys@transformshift{3.377587in}{3.940599in}%
\pgfsys@useobject{currentmarker}{}%
\end{pgfscope}%
\begin{pgfscope}%
\pgfsys@transformshift{3.387818in}{3.941967in}%
\pgfsys@useobject{currentmarker}{}%
\end{pgfscope}%
\begin{pgfscope}%
\pgfsys@transformshift{3.400948in}{3.940389in}%
\pgfsys@useobject{currentmarker}{}%
\end{pgfscope}%
\begin{pgfscope}%
\pgfsys@transformshift{3.418449in}{3.942238in}%
\pgfsys@useobject{currentmarker}{}%
\end{pgfscope}%
\begin{pgfscope}%
\pgfsys@transformshift{3.438408in}{3.940360in}%
\pgfsys@useobject{currentmarker}{}%
\end{pgfscope}%
\begin{pgfscope}%
\pgfsys@transformshift{3.449422in}{3.940867in}%
\pgfsys@useobject{currentmarker}{}%
\end{pgfscope}%
\begin{pgfscope}%
\pgfsys@transformshift{3.464110in}{3.940788in}%
\pgfsys@useobject{currentmarker}{}%
\end{pgfscope}%
\begin{pgfscope}%
\pgfsys@transformshift{3.481808in}{3.938511in}%
\pgfsys@useobject{currentmarker}{}%
\end{pgfscope}%
\begin{pgfscope}%
\pgfsys@transformshift{3.491593in}{3.939260in}%
\pgfsys@useobject{currentmarker}{}%
\end{pgfscope}%
\begin{pgfscope}%
\pgfsys@transformshift{3.505000in}{3.938512in}%
\pgfsys@useobject{currentmarker}{}%
\end{pgfscope}%
\begin{pgfscope}%
\pgfsys@transformshift{3.512381in}{3.938243in}%
\pgfsys@useobject{currentmarker}{}%
\end{pgfscope}%
\begin{pgfscope}%
\pgfsys@transformshift{3.521285in}{3.938489in}%
\pgfsys@useobject{currentmarker}{}%
\end{pgfscope}%
\begin{pgfscope}%
\pgfsys@transformshift{3.526179in}{3.938270in}%
\pgfsys@useobject{currentmarker}{}%
\end{pgfscope}%
\begin{pgfscope}%
\pgfsys@transformshift{3.533585in}{3.938900in}%
\pgfsys@useobject{currentmarker}{}%
\end{pgfscope}%
\begin{pgfscope}%
\pgfsys@transformshift{3.543153in}{3.938065in}%
\pgfsys@useobject{currentmarker}{}%
\end{pgfscope}%
\begin{pgfscope}%
\pgfsys@transformshift{3.548427in}{3.937783in}%
\pgfsys@useobject{currentmarker}{}%
\end{pgfscope}%
\begin{pgfscope}%
\pgfsys@transformshift{3.555720in}{3.937360in}%
\pgfsys@useobject{currentmarker}{}%
\end{pgfscope}%
\begin{pgfscope}%
\pgfsys@transformshift{3.559719in}{3.936976in}%
\pgfsys@useobject{currentmarker}{}%
\end{pgfscope}%
\begin{pgfscope}%
\pgfsys@transformshift{3.565775in}{3.936691in}%
\pgfsys@useobject{currentmarker}{}%
\end{pgfscope}%
\begin{pgfscope}%
\pgfsys@transformshift{3.572887in}{3.936293in}%
\pgfsys@useobject{currentmarker}{}%
\end{pgfscope}%
\begin{pgfscope}%
\pgfsys@transformshift{3.576765in}{3.935743in}%
\pgfsys@useobject{currentmarker}{}%
\end{pgfscope}%
\begin{pgfscope}%
\pgfsys@transformshift{3.578920in}{3.935718in}%
\pgfsys@useobject{currentmarker}{}%
\end{pgfscope}%
\begin{pgfscope}%
\pgfsys@transformshift{3.582920in}{3.935738in}%
\pgfsys@useobject{currentmarker}{}%
\end{pgfscope}%
\begin{pgfscope}%
\pgfsys@transformshift{3.585105in}{3.935490in}%
\pgfsys@useobject{currentmarker}{}%
\end{pgfscope}%
\begin{pgfscope}%
\pgfsys@transformshift{3.590567in}{3.936454in}%
\pgfsys@useobject{currentmarker}{}%
\end{pgfscope}%
\begin{pgfscope}%
\pgfsys@transformshift{3.596972in}{3.935186in}%
\pgfsys@useobject{currentmarker}{}%
\end{pgfscope}%
\begin{pgfscope}%
\pgfsys@transformshift{3.605427in}{3.935679in}%
\pgfsys@useobject{currentmarker}{}%
\end{pgfscope}%
\begin{pgfscope}%
\pgfsys@transformshift{3.615843in}{3.934146in}%
\pgfsys@useobject{currentmarker}{}%
\end{pgfscope}%
\begin{pgfscope}%
\pgfsys@transformshift{3.630051in}{3.934758in}%
\pgfsys@useobject{currentmarker}{}%
\end{pgfscope}%
\begin{pgfscope}%
\pgfsys@transformshift{3.645582in}{3.934161in}%
\pgfsys@useobject{currentmarker}{}%
\end{pgfscope}%
\begin{pgfscope}%
\pgfsys@transformshift{3.654092in}{3.934975in}%
\pgfsys@useobject{currentmarker}{}%
\end{pgfscope}%
\begin{pgfscope}%
\pgfsys@transformshift{3.658773in}{3.934536in}%
\pgfsys@useobject{currentmarker}{}%
\end{pgfscope}%
\begin{pgfscope}%
\pgfsys@transformshift{3.666351in}{3.935440in}%
\pgfsys@useobject{currentmarker}{}%
\end{pgfscope}%
\begin{pgfscope}%
\pgfsys@transformshift{3.676550in}{3.935026in}%
\pgfsys@useobject{currentmarker}{}%
\end{pgfscope}%
\begin{pgfscope}%
\pgfsys@transformshift{3.682163in}{3.935121in}%
\pgfsys@useobject{currentmarker}{}%
\end{pgfscope}%
\begin{pgfscope}%
\pgfsys@transformshift{3.690098in}{3.934521in}%
\pgfsys@useobject{currentmarker}{}%
\end{pgfscope}%
\begin{pgfscope}%
\pgfsys@transformshift{3.699439in}{3.934454in}%
\pgfsys@useobject{currentmarker}{}%
\end{pgfscope}%
\begin{pgfscope}%
\pgfsys@transformshift{3.710245in}{3.933594in}%
\pgfsys@useobject{currentmarker}{}%
\end{pgfscope}%
\begin{pgfscope}%
\pgfsys@transformshift{3.722742in}{3.934118in}%
\pgfsys@useobject{currentmarker}{}%
\end{pgfscope}%
\begin{pgfscope}%
\pgfsys@transformshift{3.739849in}{3.933960in}%
\pgfsys@useobject{currentmarker}{}%
\end{pgfscope}%
\begin{pgfscope}%
\pgfsys@transformshift{3.759880in}{3.933899in}%
\pgfsys@useobject{currentmarker}{}%
\end{pgfscope}%
\begin{pgfscope}%
\pgfsys@transformshift{3.781794in}{3.932514in}%
\pgfsys@useobject{currentmarker}{}%
\end{pgfscope}%
\begin{pgfscope}%
\pgfsys@transformshift{3.793869in}{3.932714in}%
\pgfsys@useobject{currentmarker}{}%
\end{pgfscope}%
\begin{pgfscope}%
\pgfsys@transformshift{3.807752in}{3.932245in}%
\pgfsys@useobject{currentmarker}{}%
\end{pgfscope}%
\begin{pgfscope}%
\pgfsys@transformshift{3.824396in}{3.933434in}%
\pgfsys@useobject{currentmarker}{}%
\end{pgfscope}%
\begin{pgfscope}%
\pgfsys@transformshift{3.842650in}{3.932290in}%
\pgfsys@useobject{currentmarker}{}%
\end{pgfscope}%
\begin{pgfscope}%
\pgfsys@transformshift{3.864343in}{3.931560in}%
\pgfsys@useobject{currentmarker}{}%
\end{pgfscope}%
\begin{pgfscope}%
\pgfsys@transformshift{3.887137in}{3.930640in}%
\pgfsys@useobject{currentmarker}{}%
\end{pgfscope}%
\begin{pgfscope}%
\pgfsys@transformshift{3.899681in}{3.930367in}%
\pgfsys@useobject{currentmarker}{}%
\end{pgfscope}%
\begin{pgfscope}%
\pgfsys@transformshift{3.915527in}{3.931585in}%
\pgfsys@useobject{currentmarker}{}%
\end{pgfscope}%
\begin{pgfscope}%
\pgfsys@transformshift{3.924206in}{3.932631in}%
\pgfsys@useobject{currentmarker}{}%
\end{pgfscope}%
\begin{pgfscope}%
\pgfsys@transformshift{3.935170in}{3.932667in}%
\pgfsys@useobject{currentmarker}{}%
\end{pgfscope}%
\begin{pgfscope}%
\pgfsys@transformshift{3.951069in}{3.932056in}%
\pgfsys@useobject{currentmarker}{}%
\end{pgfscope}%
\begin{pgfscope}%
\pgfsys@transformshift{3.959819in}{3.932200in}%
\pgfsys@useobject{currentmarker}{}%
\end{pgfscope}%
\begin{pgfscope}%
\pgfsys@transformshift{3.969942in}{3.931687in}%
\pgfsys@useobject{currentmarker}{}%
\end{pgfscope}%
\begin{pgfscope}%
\pgfsys@transformshift{3.975515in}{3.931839in}%
\pgfsys@useobject{currentmarker}{}%
\end{pgfscope}%
\begin{pgfscope}%
\pgfsys@transformshift{3.984837in}{3.931458in}%
\pgfsys@useobject{currentmarker}{}%
\end{pgfscope}%
\begin{pgfscope}%
\pgfsys@transformshift{3.997376in}{3.931472in}%
\pgfsys@useobject{currentmarker}{}%
\end{pgfscope}%
\begin{pgfscope}%
\pgfsys@transformshift{4.012095in}{3.932254in}%
\pgfsys@useobject{currentmarker}{}%
\end{pgfscope}%
\begin{pgfscope}%
\pgfsys@transformshift{4.029424in}{3.931566in}%
\pgfsys@useobject{currentmarker}{}%
\end{pgfscope}%
\begin{pgfscope}%
\pgfsys@transformshift{4.051019in}{3.932993in}%
\pgfsys@useobject{currentmarker}{}%
\end{pgfscope}%
\begin{pgfscope}%
\pgfsys@transformshift{4.074737in}{3.933459in}%
\pgfsys@useobject{currentmarker}{}%
\end{pgfscope}%
\begin{pgfscope}%
\pgfsys@transformshift{4.102194in}{3.933205in}%
\pgfsys@useobject{currentmarker}{}%
\end{pgfscope}%
\begin{pgfscope}%
\pgfsys@transformshift{4.130756in}{3.932768in}%
\pgfsys@useobject{currentmarker}{}%
\end{pgfscope}%
\begin{pgfscope}%
\pgfsys@transformshift{4.162616in}{3.932816in}%
\pgfsys@useobject{currentmarker}{}%
\end{pgfscope}%
\begin{pgfscope}%
\pgfsys@transformshift{4.196876in}{3.932685in}%
\pgfsys@useobject{currentmarker}{}%
\end{pgfscope}%
\begin{pgfscope}%
\pgfsys@transformshift{4.235230in}{3.930038in}%
\pgfsys@useobject{currentmarker}{}%
\end{pgfscope}%
\begin{pgfscope}%
\pgfsys@transformshift{4.256355in}{3.930961in}%
\pgfsys@useobject{currentmarker}{}%
\end{pgfscope}%
\begin{pgfscope}%
\pgfsys@transformshift{4.267974in}{3.930457in}%
\pgfsys@useobject{currentmarker}{}%
\end{pgfscope}%
\begin{pgfscope}%
\pgfsys@transformshift{4.282967in}{3.932423in}%
\pgfsys@useobject{currentmarker}{}%
\end{pgfscope}%
\begin{pgfscope}%
\pgfsys@transformshift{4.304048in}{3.930730in}%
\pgfsys@useobject{currentmarker}{}%
\end{pgfscope}%
\begin{pgfscope}%
\pgfsys@transformshift{4.326052in}{3.933159in}%
\pgfsys@useobject{currentmarker}{}%
\end{pgfscope}%
\begin{pgfscope}%
\pgfsys@transformshift{4.338165in}{3.934386in}%
\pgfsys@useobject{currentmarker}{}%
\end{pgfscope}%
\begin{pgfscope}%
\pgfsys@transformshift{4.352416in}{3.935590in}%
\pgfsys@useobject{currentmarker}{}%
\end{pgfscope}%
\begin{pgfscope}%
\pgfsys@transformshift{4.370266in}{3.935474in}%
\pgfsys@useobject{currentmarker}{}%
\end{pgfscope}%
\begin{pgfscope}%
\pgfsys@transformshift{4.390167in}{3.935773in}%
\pgfsys@useobject{currentmarker}{}%
\end{pgfscope}%
\begin{pgfscope}%
\pgfsys@transformshift{4.411710in}{3.935849in}%
\pgfsys@useobject{currentmarker}{}%
\end{pgfscope}%
\begin{pgfscope}%
\pgfsys@transformshift{4.434039in}{3.937739in}%
\pgfsys@useobject{currentmarker}{}%
\end{pgfscope}%
\begin{pgfscope}%
\pgfsys@transformshift{4.458334in}{3.936943in}%
\pgfsys@useobject{currentmarker}{}%
\end{pgfscope}%
\begin{pgfscope}%
\pgfsys@transformshift{4.484908in}{3.939017in}%
\pgfsys@useobject{currentmarker}{}%
\end{pgfscope}%
\begin{pgfscope}%
\pgfsys@transformshift{4.512423in}{3.935822in}%
\pgfsys@useobject{currentmarker}{}%
\end{pgfscope}%
\begin{pgfscope}%
\pgfsys@transformshift{4.542113in}{3.939773in}%
\pgfsys@useobject{currentmarker}{}%
\end{pgfscope}%
\begin{pgfscope}%
\pgfsys@transformshift{4.573286in}{3.936720in}%
\pgfsys@useobject{currentmarker}{}%
\end{pgfscope}%
\begin{pgfscope}%
\pgfsys@transformshift{4.590288in}{3.939498in}%
\pgfsys@useobject{currentmarker}{}%
\end{pgfscope}%
\begin{pgfscope}%
\pgfsys@transformshift{4.599717in}{3.938568in}%
\pgfsys@useobject{currentmarker}{}%
\end{pgfscope}%
\begin{pgfscope}%
\pgfsys@transformshift{4.611034in}{3.939761in}%
\pgfsys@useobject{currentmarker}{}%
\end{pgfscope}%
\begin{pgfscope}%
\pgfsys@transformshift{4.617255in}{3.939070in}%
\pgfsys@useobject{currentmarker}{}%
\end{pgfscope}%
\begin{pgfscope}%
\pgfsys@transformshift{4.625154in}{3.939365in}%
\pgfsys@useobject{currentmarker}{}%
\end{pgfscope}%
\begin{pgfscope}%
\pgfsys@transformshift{4.629456in}{3.938740in}%
\pgfsys@useobject{currentmarker}{}%
\end{pgfscope}%
\begin{pgfscope}%
\pgfsys@transformshift{4.631847in}{3.938720in}%
\pgfsys@useobject{currentmarker}{}%
\end{pgfscope}%
\begin{pgfscope}%
\pgfsys@transformshift{4.633067in}{3.938229in}%
\pgfsys@useobject{currentmarker}{}%
\end{pgfscope}%
\begin{pgfscope}%
\pgfsys@transformshift{4.636000in}{3.937986in}%
\pgfsys@useobject{currentmarker}{}%
\end{pgfscope}%
\begin{pgfscope}%
\pgfsys@transformshift{4.639973in}{3.935948in}%
\pgfsys@useobject{currentmarker}{}%
\end{pgfscope}%
\begin{pgfscope}%
\pgfsys@transformshift{4.642270in}{3.935078in}%
\pgfsys@useobject{currentmarker}{}%
\end{pgfscope}%
\begin{pgfscope}%
\pgfsys@transformshift{4.643129in}{3.934036in}%
\pgfsys@useobject{currentmarker}{}%
\end{pgfscope}%
\begin{pgfscope}%
\pgfsys@transformshift{4.643591in}{3.933454in}%
\pgfsys@useobject{currentmarker}{}%
\end{pgfscope}%
\begin{pgfscope}%
\pgfsys@transformshift{4.644037in}{3.930923in}%
\pgfsys@useobject{currentmarker}{}%
\end{pgfscope}%
\begin{pgfscope}%
\pgfsys@transformshift{4.643792in}{3.926801in}%
\pgfsys@useobject{currentmarker}{}%
\end{pgfscope}%
\begin{pgfscope}%
\pgfsys@transformshift{4.643669in}{3.919428in}%
\pgfsys@useobject{currentmarker}{}%
\end{pgfscope}%
\begin{pgfscope}%
\pgfsys@transformshift{4.643437in}{3.910868in}%
\pgfsys@useobject{currentmarker}{}%
\end{pgfscope}%
\begin{pgfscope}%
\pgfsys@transformshift{4.643305in}{3.906159in}%
\pgfsys@useobject{currentmarker}{}%
\end{pgfscope}%
\begin{pgfscope}%
\pgfsys@transformshift{4.643100in}{3.903577in}%
\pgfsys@useobject{currentmarker}{}%
\end{pgfscope}%
\begin{pgfscope}%
\pgfsys@transformshift{4.643006in}{3.902155in}%
\pgfsys@useobject{currentmarker}{}%
\end{pgfscope}%
\begin{pgfscope}%
\pgfsys@transformshift{4.642931in}{3.901375in}%
\pgfsys@useobject{currentmarker}{}%
\end{pgfscope}%
\begin{pgfscope}%
\pgfsys@transformshift{4.642943in}{3.900945in}%
\pgfsys@useobject{currentmarker}{}%
\end{pgfscope}%
\begin{pgfscope}%
\pgfsys@transformshift{4.642883in}{3.900715in}%
\pgfsys@useobject{currentmarker}{}%
\end{pgfscope}%
\begin{pgfscope}%
\pgfsys@transformshift{4.642899in}{3.900586in}%
\pgfsys@useobject{currentmarker}{}%
\end{pgfscope}%
\begin{pgfscope}%
\pgfsys@transformshift{4.642582in}{3.899442in}%
\pgfsys@useobject{currentmarker}{}%
\end{pgfscope}%
\begin{pgfscope}%
\pgfsys@transformshift{4.642963in}{3.895677in}%
\pgfsys@useobject{currentmarker}{}%
\end{pgfscope}%
\begin{pgfscope}%
\pgfsys@transformshift{4.642079in}{3.891118in}%
\pgfsys@useobject{currentmarker}{}%
\end{pgfscope}%
\begin{pgfscope}%
\pgfsys@transformshift{4.641871in}{3.888572in}%
\pgfsys@useobject{currentmarker}{}%
\end{pgfscope}%
\begin{pgfscope}%
\pgfsys@transformshift{4.642139in}{3.887193in}%
\pgfsys@useobject{currentmarker}{}%
\end{pgfscope}%
\begin{pgfscope}%
\pgfsys@transformshift{4.641432in}{3.883433in}%
\pgfsys@useobject{currentmarker}{}%
\end{pgfscope}%
\begin{pgfscope}%
\pgfsys@transformshift{4.642284in}{3.877985in}%
\pgfsys@useobject{currentmarker}{}%
\end{pgfscope}%
\begin{pgfscope}%
\pgfsys@transformshift{4.642366in}{3.871312in}%
\pgfsys@useobject{currentmarker}{}%
\end{pgfscope}%
\begin{pgfscope}%
\pgfsys@transformshift{4.641267in}{3.863648in}%
\pgfsys@useobject{currentmarker}{}%
\end{pgfscope}%
\begin{pgfscope}%
\pgfsys@transformshift{4.643636in}{3.852607in}%
\pgfsys@useobject{currentmarker}{}%
\end{pgfscope}%
\begin{pgfscope}%
\pgfsys@transformshift{4.642788in}{3.846454in}%
\pgfsys@useobject{currentmarker}{}%
\end{pgfscope}%
\begin{pgfscope}%
\pgfsys@transformshift{4.643147in}{3.838857in}%
\pgfsys@useobject{currentmarker}{}%
\end{pgfscope}%
\begin{pgfscope}%
\pgfsys@transformshift{4.644075in}{3.830389in}%
\pgfsys@useobject{currentmarker}{}%
\end{pgfscope}%
\begin{pgfscope}%
\pgfsys@transformshift{4.642607in}{3.819606in}%
\pgfsys@useobject{currentmarker}{}%
\end{pgfscope}%
\begin{pgfscope}%
\pgfsys@transformshift{4.643140in}{3.807803in}%
\pgfsys@useobject{currentmarker}{}%
\end{pgfscope}%
\begin{pgfscope}%
\pgfsys@transformshift{4.642911in}{3.794902in}%
\pgfsys@useobject{currentmarker}{}%
\end{pgfscope}%
\begin{pgfscope}%
\pgfsys@transformshift{4.640349in}{3.781175in}%
\pgfsys@useobject{currentmarker}{}%
\end{pgfscope}%
\begin{pgfscope}%
\pgfsys@transformshift{4.642422in}{3.764461in}%
\pgfsys@useobject{currentmarker}{}%
\end{pgfscope}%
\begin{pgfscope}%
\pgfsys@transformshift{4.641140in}{3.755287in}%
\pgfsys@useobject{currentmarker}{}%
\end{pgfscope}%
\begin{pgfscope}%
\pgfsys@transformshift{4.640962in}{3.750195in}%
\pgfsys@useobject{currentmarker}{}%
\end{pgfscope}%
\begin{pgfscope}%
\pgfsys@transformshift{4.641456in}{3.747437in}%
\pgfsys@useobject{currentmarker}{}%
\end{pgfscope}%
\begin{pgfscope}%
\pgfsys@transformshift{4.640750in}{3.740687in}%
\pgfsys@useobject{currentmarker}{}%
\end{pgfscope}%
\begin{pgfscope}%
\pgfsys@transformshift{4.640859in}{3.736956in}%
\pgfsys@useobject{currentmarker}{}%
\end{pgfscope}%
\begin{pgfscope}%
\pgfsys@transformshift{4.640992in}{3.734907in}%
\pgfsys@useobject{currentmarker}{}%
\end{pgfscope}%
\begin{pgfscope}%
\pgfsys@transformshift{4.640817in}{3.733792in}%
\pgfsys@useobject{currentmarker}{}%
\end{pgfscope}%
\begin{pgfscope}%
\pgfsys@transformshift{4.641323in}{3.729718in}%
\pgfsys@useobject{currentmarker}{}%
\end{pgfscope}%
\begin{pgfscope}%
\pgfsys@transformshift{4.640728in}{3.724792in}%
\pgfsys@useobject{currentmarker}{}%
\end{pgfscope}%
\begin{pgfscope}%
\pgfsys@transformshift{4.640274in}{3.722102in}%
\pgfsys@useobject{currentmarker}{}%
\end{pgfscope}%
\begin{pgfscope}%
\pgfsys@transformshift{4.641070in}{3.717927in}%
\pgfsys@useobject{currentmarker}{}%
\end{pgfscope}%
\begin{pgfscope}%
\pgfsys@transformshift{4.640019in}{3.710518in}%
\pgfsys@useobject{currentmarker}{}%
\end{pgfscope}%
\begin{pgfscope}%
\pgfsys@transformshift{4.640406in}{3.706420in}%
\pgfsys@useobject{currentmarker}{}%
\end{pgfscope}%
\begin{pgfscope}%
\pgfsys@transformshift{4.640549in}{3.704160in}%
\pgfsys@useobject{currentmarker}{}%
\end{pgfscope}%
\begin{pgfscope}%
\pgfsys@transformshift{4.640305in}{3.702939in}%
\pgfsys@useobject{currentmarker}{}%
\end{pgfscope}%
\begin{pgfscope}%
\pgfsys@transformshift{4.641155in}{3.699107in}%
\pgfsys@useobject{currentmarker}{}%
\end{pgfscope}%
\begin{pgfscope}%
\pgfsys@transformshift{4.640880in}{3.696965in}%
\pgfsys@useobject{currentmarker}{}%
\end{pgfscope}%
\begin{pgfscope}%
\pgfsys@transformshift{4.640914in}{3.695778in}%
\pgfsys@useobject{currentmarker}{}%
\end{pgfscope}%
\begin{pgfscope}%
\pgfsys@transformshift{4.641042in}{3.695138in}%
\pgfsys@useobject{currentmarker}{}%
\end{pgfscope}%
\begin{pgfscope}%
\pgfsys@transformshift{4.640204in}{3.690865in}%
\pgfsys@useobject{currentmarker}{}%
\end{pgfscope}%
\begin{pgfscope}%
\pgfsys@transformshift{4.640346in}{3.688474in}%
\pgfsys@useobject{currentmarker}{}%
\end{pgfscope}%
\begin{pgfscope}%
\pgfsys@transformshift{4.640350in}{3.685161in}%
\pgfsys@useobject{currentmarker}{}%
\end{pgfscope}%
\begin{pgfscope}%
\pgfsys@transformshift{4.640144in}{3.683351in}%
\pgfsys@useobject{currentmarker}{}%
\end{pgfscope}%
\begin{pgfscope}%
\pgfsys@transformshift{4.640822in}{3.678726in}%
\pgfsys@useobject{currentmarker}{}%
\end{pgfscope}%
\begin{pgfscope}%
\pgfsys@transformshift{4.640725in}{3.676158in}%
\pgfsys@useobject{currentmarker}{}%
\end{pgfscope}%
\begin{pgfscope}%
\pgfsys@transformshift{4.640571in}{3.674753in}%
\pgfsys@useobject{currentmarker}{}%
\end{pgfscope}%
\begin{pgfscope}%
\pgfsys@transformshift{4.641262in}{3.671774in}%
\pgfsys@useobject{currentmarker}{}%
\end{pgfscope}%
\begin{pgfscope}%
\pgfsys@transformshift{4.640389in}{3.666793in}%
\pgfsys@useobject{currentmarker}{}%
\end{pgfscope}%
\begin{pgfscope}%
\pgfsys@transformshift{4.640440in}{3.664013in}%
\pgfsys@useobject{currentmarker}{}%
\end{pgfscope}%
\begin{pgfscope}%
\pgfsys@transformshift{4.640670in}{3.662501in}%
\pgfsys@useobject{currentmarker}{}%
\end{pgfscope}%
\begin{pgfscope}%
\pgfsys@transformshift{4.639994in}{3.659321in}%
\pgfsys@useobject{currentmarker}{}%
\end{pgfscope}%
\begin{pgfscope}%
\pgfsys@transformshift{4.641154in}{3.653600in}%
\pgfsys@useobject{currentmarker}{}%
\end{pgfscope}%
\begin{pgfscope}%
\pgfsys@transformshift{4.640186in}{3.646420in}%
\pgfsys@useobject{currentmarker}{}%
\end{pgfscope}%
\begin{pgfscope}%
\pgfsys@transformshift{4.639821in}{3.642452in}%
\pgfsys@useobject{currentmarker}{}%
\end{pgfscope}%
\begin{pgfscope}%
\pgfsys@transformshift{4.640308in}{3.640315in}%
\pgfsys@useobject{currentmarker}{}%
\end{pgfscope}%
\begin{pgfscope}%
\pgfsys@transformshift{4.639463in}{3.635397in}%
\pgfsys@useobject{currentmarker}{}%
\end{pgfscope}%
\begin{pgfscope}%
\pgfsys@transformshift{4.639685in}{3.632661in}%
\pgfsys@useobject{currentmarker}{}%
\end{pgfscope}%
\begin{pgfscope}%
\pgfsys@transformshift{4.639834in}{3.631159in}%
\pgfsys@useobject{currentmarker}{}%
\end{pgfscope}%
\begin{pgfscope}%
\pgfsys@transformshift{4.639638in}{3.630353in}%
\pgfsys@useobject{currentmarker}{}%
\end{pgfscope}%
\begin{pgfscope}%
\pgfsys@transformshift{4.640479in}{3.625554in}%
\pgfsys@useobject{currentmarker}{}%
\end{pgfscope}%
\begin{pgfscope}%
\pgfsys@transformshift{4.640021in}{3.622914in}%
\pgfsys@useobject{currentmarker}{}%
\end{pgfscope}%
\begin{pgfscope}%
\pgfsys@transformshift{4.639823in}{3.621454in}%
\pgfsys@useobject{currentmarker}{}%
\end{pgfscope}%
\begin{pgfscope}%
\pgfsys@transformshift{4.639989in}{3.620661in}%
\pgfsys@useobject{currentmarker}{}%
\end{pgfscope}%
\begin{pgfscope}%
\pgfsys@transformshift{4.639507in}{3.617052in}%
\pgfsys@useobject{currentmarker}{}%
\end{pgfscope}%
\begin{pgfscope}%
\pgfsys@transformshift{4.639600in}{3.615052in}%
\pgfsys@useobject{currentmarker}{}%
\end{pgfscope}%
\begin{pgfscope}%
\pgfsys@transformshift{4.639753in}{3.613961in}%
\pgfsys@useobject{currentmarker}{}%
\end{pgfscope}%
\begin{pgfscope}%
\pgfsys@transformshift{4.639250in}{3.611490in}%
\pgfsys@useobject{currentmarker}{}%
\end{pgfscope}%
\begin{pgfscope}%
\pgfsys@transformshift{4.639551in}{3.607362in}%
\pgfsys@useobject{currentmarker}{}%
\end{pgfscope}%
\begin{pgfscope}%
\pgfsys@transformshift{4.640017in}{3.605135in}%
\pgfsys@useobject{currentmarker}{}%
\end{pgfscope}%
\begin{pgfscope}%
\pgfsys@transformshift{4.638698in}{3.600246in}%
\pgfsys@useobject{currentmarker}{}%
\end{pgfscope}%
\begin{pgfscope}%
\pgfsys@transformshift{4.639494in}{3.593956in}%
\pgfsys@useobject{currentmarker}{}%
\end{pgfscope}%
\begin{pgfscope}%
\pgfsys@transformshift{4.639510in}{3.590469in}%
\pgfsys@useobject{currentmarker}{}%
\end{pgfscope}%
\begin{pgfscope}%
\pgfsys@transformshift{4.639317in}{3.588560in}%
\pgfsys@useobject{currentmarker}{}%
\end{pgfscope}%
\begin{pgfscope}%
\pgfsys@transformshift{4.640874in}{3.582455in}%
\pgfsys@useobject{currentmarker}{}%
\end{pgfscope}%
\begin{pgfscope}%
\pgfsys@transformshift{4.640406in}{3.579021in}%
\pgfsys@useobject{currentmarker}{}%
\end{pgfscope}%
\begin{pgfscope}%
\pgfsys@transformshift{4.640879in}{3.574451in}%
\pgfsys@useobject{currentmarker}{}%
\end{pgfscope}%
\begin{pgfscope}%
\pgfsys@transformshift{4.641055in}{3.571930in}%
\pgfsys@useobject{currentmarker}{}%
\end{pgfscope}%
\begin{pgfscope}%
\pgfsys@transformshift{4.640616in}{3.568389in}%
\pgfsys@useobject{currentmarker}{}%
\end{pgfscope}%
\begin{pgfscope}%
\pgfsys@transformshift{4.641476in}{3.562181in}%
\pgfsys@useobject{currentmarker}{}%
\end{pgfscope}%
\begin{pgfscope}%
\pgfsys@transformshift{4.641560in}{3.558735in}%
\pgfsys@useobject{currentmarker}{}%
\end{pgfscope}%
\begin{pgfscope}%
\pgfsys@transformshift{4.641452in}{3.556842in}%
\pgfsys@useobject{currentmarker}{}%
\end{pgfscope}%
\begin{pgfscope}%
\pgfsys@transformshift{4.642319in}{3.552688in}%
\pgfsys@useobject{currentmarker}{}%
\end{pgfscope}%
\begin{pgfscope}%
\pgfsys@transformshift{4.641736in}{3.546423in}%
\pgfsys@useobject{currentmarker}{}%
\end{pgfscope}%
\begin{pgfscope}%
\pgfsys@transformshift{4.642299in}{3.543008in}%
\pgfsys@useobject{currentmarker}{}%
\end{pgfscope}%
\begin{pgfscope}%
\pgfsys@transformshift{4.642665in}{3.541140in}%
\pgfsys@useobject{currentmarker}{}%
\end{pgfscope}%
\begin{pgfscope}%
\pgfsys@transformshift{4.642056in}{3.538096in}%
\pgfsys@useobject{currentmarker}{}%
\end{pgfscope}%
\begin{pgfscope}%
\pgfsys@transformshift{4.643167in}{3.531826in}%
\pgfsys@useobject{currentmarker}{}%
\end{pgfscope}%
\begin{pgfscope}%
\pgfsys@transformshift{4.643165in}{3.524563in}%
\pgfsys@useobject{currentmarker}{}%
\end{pgfscope}%
\begin{pgfscope}%
\pgfsys@transformshift{4.642476in}{3.520628in}%
\pgfsys@useobject{currentmarker}{}%
\end{pgfscope}%
\begin{pgfscope}%
\pgfsys@transformshift{4.643946in}{3.513869in}%
\pgfsys@useobject{currentmarker}{}%
\end{pgfscope}%
\begin{pgfscope}%
\pgfsys@transformshift{4.643462in}{3.510096in}%
\pgfsys@useobject{currentmarker}{}%
\end{pgfscope}%
\begin{pgfscope}%
\pgfsys@transformshift{4.643368in}{3.508006in}%
\pgfsys@useobject{currentmarker}{}%
\end{pgfscope}%
\begin{pgfscope}%
\pgfsys@transformshift{4.643577in}{3.506874in}%
\pgfsys@useobject{currentmarker}{}%
\end{pgfscope}%
\begin{pgfscope}%
\pgfsys@transformshift{4.642812in}{3.501898in}%
\pgfsys@useobject{currentmarker}{}%
\end{pgfscope}%
\begin{pgfscope}%
\pgfsys@transformshift{4.643004in}{3.499135in}%
\pgfsys@useobject{currentmarker}{}%
\end{pgfscope}%
\begin{pgfscope}%
\pgfsys@transformshift{4.643076in}{3.497613in}%
\pgfsys@useobject{currentmarker}{}%
\end{pgfscope}%
\begin{pgfscope}%
\pgfsys@transformshift{4.642790in}{3.495126in}%
\pgfsys@useobject{currentmarker}{}%
\end{pgfscope}%
\begin{pgfscope}%
\pgfsys@transformshift{4.643265in}{3.491124in}%
\pgfsys@useobject{currentmarker}{}%
\end{pgfscope}%
\begin{pgfscope}%
\pgfsys@transformshift{4.644047in}{3.486121in}%
\pgfsys@useobject{currentmarker}{}%
\end{pgfscope}%
\begin{pgfscope}%
\pgfsys@transformshift{4.643410in}{3.478446in}%
\pgfsys@useobject{currentmarker}{}%
\end{pgfscope}%
\begin{pgfscope}%
\pgfsys@transformshift{4.642974in}{3.474232in}%
\pgfsys@useobject{currentmarker}{}%
\end{pgfscope}%
\begin{pgfscope}%
\pgfsys@transformshift{4.643973in}{3.469083in}%
\pgfsys@useobject{currentmarker}{}%
\end{pgfscope}%
\begin{pgfscope}%
\pgfsys@transformshift{4.642985in}{3.459806in}%
\pgfsys@useobject{currentmarker}{}%
\end{pgfscope}%
\begin{pgfscope}%
\pgfsys@transformshift{4.642424in}{3.454705in}%
\pgfsys@useobject{currentmarker}{}%
\end{pgfscope}%
\begin{pgfscope}%
\pgfsys@transformshift{4.642808in}{3.451908in}%
\pgfsys@useobject{currentmarker}{}%
\end{pgfscope}%
\begin{pgfscope}%
\pgfsys@transformshift{4.641294in}{3.444405in}%
\pgfsys@useobject{currentmarker}{}%
\end{pgfscope}%
\begin{pgfscope}%
\pgfsys@transformshift{4.641249in}{3.435426in}%
\pgfsys@useobject{currentmarker}{}%
\end{pgfscope}%
\begin{pgfscope}%
\pgfsys@transformshift{4.642000in}{3.430545in}%
\pgfsys@useobject{currentmarker}{}%
\end{pgfscope}%
\begin{pgfscope}%
\pgfsys@transformshift{4.640318in}{3.423981in}%
\pgfsys@useobject{currentmarker}{}%
\end{pgfscope}%
\begin{pgfscope}%
\pgfsys@transformshift{4.641746in}{3.413095in}%
\pgfsys@useobject{currentmarker}{}%
\end{pgfscope}%
\begin{pgfscope}%
\pgfsys@transformshift{4.638701in}{3.399966in}%
\pgfsys@useobject{currentmarker}{}%
\end{pgfscope}%
\begin{pgfscope}%
\pgfsys@transformshift{4.638015in}{3.385185in}%
\pgfsys@useobject{currentmarker}{}%
\end{pgfscope}%
\begin{pgfscope}%
\pgfsys@transformshift{4.639376in}{3.377161in}%
\pgfsys@useobject{currentmarker}{}%
\end{pgfscope}%
\begin{pgfscope}%
\pgfsys@transformshift{4.636732in}{3.364614in}%
\pgfsys@useobject{currentmarker}{}%
\end{pgfscope}%
\begin{pgfscope}%
\pgfsys@transformshift{4.638640in}{3.348897in}%
\pgfsys@useobject{currentmarker}{}%
\end{pgfscope}%
\begin{pgfscope}%
\pgfsys@transformshift{4.637900in}{3.340220in}%
\pgfsys@useobject{currentmarker}{}%
\end{pgfscope}%
\begin{pgfscope}%
\pgfsys@transformshift{4.637228in}{3.335478in}%
\pgfsys@useobject{currentmarker}{}%
\end{pgfscope}%
\begin{pgfscope}%
\pgfsys@transformshift{4.638847in}{3.327726in}%
\pgfsys@useobject{currentmarker}{}%
\end{pgfscope}%
\begin{pgfscope}%
\pgfsys@transformshift{4.637959in}{3.323461in}%
\pgfsys@useobject{currentmarker}{}%
\end{pgfscope}%
\begin{pgfscope}%
\pgfsys@transformshift{4.637871in}{3.317769in}%
\pgfsys@useobject{currentmarker}{}%
\end{pgfscope}%
\begin{pgfscope}%
\pgfsys@transformshift{4.638418in}{3.314686in}%
\pgfsys@useobject{currentmarker}{}%
\end{pgfscope}%
\begin{pgfscope}%
\pgfsys@transformshift{4.637453in}{3.309648in}%
\pgfsys@useobject{currentmarker}{}%
\end{pgfscope}%
\begin{pgfscope}%
\pgfsys@transformshift{4.639449in}{3.301339in}%
\pgfsys@useobject{currentmarker}{}%
\end{pgfscope}%
\begin{pgfscope}%
\pgfsys@transformshift{4.638572in}{3.296722in}%
\pgfsys@useobject{currentmarker}{}%
\end{pgfscope}%
\begin{pgfscope}%
\pgfsys@transformshift{4.638415in}{3.294142in}%
\pgfsys@useobject{currentmarker}{}%
\end{pgfscope}%
\begin{pgfscope}%
\pgfsys@transformshift{4.638719in}{3.292754in}%
\pgfsys@useobject{currentmarker}{}%
\end{pgfscope}%
\begin{pgfscope}%
\pgfsys@transformshift{4.638148in}{3.289476in}%
\pgfsys@useobject{currentmarker}{}%
\end{pgfscope}%
\begin{pgfscope}%
\pgfsys@transformshift{4.638676in}{3.284413in}%
\pgfsys@useobject{currentmarker}{}%
\end{pgfscope}%
\begin{pgfscope}%
\pgfsys@transformshift{4.637975in}{3.277127in}%
\pgfsys@useobject{currentmarker}{}%
\end{pgfscope}%
\begin{pgfscope}%
\pgfsys@transformshift{4.638429in}{3.268637in}%
\pgfsys@useobject{currentmarker}{}%
\end{pgfscope}%
\begin{pgfscope}%
\pgfsys@transformshift{4.638766in}{3.263973in}%
\pgfsys@useobject{currentmarker}{}%
\end{pgfscope}%
\begin{pgfscope}%
\pgfsys@transformshift{4.637787in}{3.258275in}%
\pgfsys@useobject{currentmarker}{}%
\end{pgfscope}%
\begin{pgfscope}%
\pgfsys@transformshift{4.639504in}{3.250023in}%
\pgfsys@useobject{currentmarker}{}%
\end{pgfscope}%
\begin{pgfscope}%
\pgfsys@transformshift{4.639088in}{3.245406in}%
\pgfsys@useobject{currentmarker}{}%
\end{pgfscope}%
\begin{pgfscope}%
\pgfsys@transformshift{4.639039in}{3.242857in}%
\pgfsys@useobject{currentmarker}{}%
\end{pgfscope}%
\begin{pgfscope}%
\pgfsys@transformshift{4.639459in}{3.241519in}%
\pgfsys@useobject{currentmarker}{}%
\end{pgfscope}%
\begin{pgfscope}%
\pgfsys@transformshift{4.638465in}{3.236641in}%
\pgfsys@useobject{currentmarker}{}%
\end{pgfscope}%
\begin{pgfscope}%
\pgfsys@transformshift{4.639142in}{3.230077in}%
\pgfsys@useobject{currentmarker}{}%
\end{pgfscope}%
\begin{pgfscope}%
\pgfsys@transformshift{4.639552in}{3.226470in}%
\pgfsys@useobject{currentmarker}{}%
\end{pgfscope}%
\begin{pgfscope}%
\pgfsys@transformshift{4.639385in}{3.224481in}%
\pgfsys@useobject{currentmarker}{}%
\end{pgfscope}%
\begin{pgfscope}%
\pgfsys@transformshift{4.640544in}{3.219210in}%
\pgfsys@useobject{currentmarker}{}%
\end{pgfscope}%
\begin{pgfscope}%
\pgfsys@transformshift{4.640158in}{3.216267in}%
\pgfsys@useobject{currentmarker}{}%
\end{pgfscope}%
\begin{pgfscope}%
\pgfsys@transformshift{4.640193in}{3.214635in}%
\pgfsys@useobject{currentmarker}{}%
\end{pgfscope}%
\begin{pgfscope}%
\pgfsys@transformshift{4.640662in}{3.212147in}%
\pgfsys@useobject{currentmarker}{}%
\end{pgfscope}%
\begin{pgfscope}%
\pgfsys@transformshift{4.639911in}{3.207166in}%
\pgfsys@useobject{currentmarker}{}%
\end{pgfscope}%
\begin{pgfscope}%
\pgfsys@transformshift{4.640976in}{3.200572in}%
\pgfsys@useobject{currentmarker}{}%
\end{pgfscope}%
\begin{pgfscope}%
\pgfsys@transformshift{4.640867in}{3.196900in}%
\pgfsys@useobject{currentmarker}{}%
\end{pgfscope}%
\begin{pgfscope}%
\pgfsys@transformshift{4.640639in}{3.194893in}%
\pgfsys@useobject{currentmarker}{}%
\end{pgfscope}%
\begin{pgfscope}%
\pgfsys@transformshift{4.640798in}{3.193793in}%
\pgfsys@useobject{currentmarker}{}%
\end{pgfscope}%
\begin{pgfscope}%
\pgfsys@transformshift{4.640455in}{3.191331in}%
\pgfsys@useobject{currentmarker}{}%
\end{pgfscope}%
\begin{pgfscope}%
\pgfsys@transformshift{4.640431in}{3.189964in}%
\pgfsys@useobject{currentmarker}{}%
\end{pgfscope}%
\begin{pgfscope}%
\pgfsys@transformshift{4.640462in}{3.189213in}%
\pgfsys@useobject{currentmarker}{}%
\end{pgfscope}%
\begin{pgfscope}%
\pgfsys@transformshift{4.640435in}{3.188800in}%
\pgfsys@useobject{currentmarker}{}%
\end{pgfscope}%
\begin{pgfscope}%
\pgfsys@transformshift{4.640450in}{3.188573in}%
\pgfsys@useobject{currentmarker}{}%
\end{pgfscope}%
\begin{pgfscope}%
\pgfsys@transformshift{4.640431in}{3.188449in}%
\pgfsys@useobject{currentmarker}{}%
\end{pgfscope}%
\begin{pgfscope}%
\pgfsys@transformshift{4.640432in}{3.188380in}%
\pgfsys@useobject{currentmarker}{}%
\end{pgfscope}%
\begin{pgfscope}%
\pgfsys@transformshift{4.640433in}{3.188343in}%
\pgfsys@useobject{currentmarker}{}%
\end{pgfscope}%
\begin{pgfscope}%
\pgfsys@transformshift{4.640431in}{3.188322in}%
\pgfsys@useobject{currentmarker}{}%
\end{pgfscope}%
\begin{pgfscope}%
\pgfsys@transformshift{4.640755in}{3.186473in}%
\pgfsys@useobject{currentmarker}{}%
\end{pgfscope}%
\begin{pgfscope}%
\pgfsys@transformshift{4.640656in}{3.185446in}%
\pgfsys@useobject{currentmarker}{}%
\end{pgfscope}%
\begin{pgfscope}%
\pgfsys@transformshift{4.640638in}{3.184879in}%
\pgfsys@useobject{currentmarker}{}%
\end{pgfscope}%
\begin{pgfscope}%
\pgfsys@transformshift{4.640704in}{3.184574in}%
\pgfsys@useobject{currentmarker}{}%
\end{pgfscope}%
\begin{pgfscope}%
\pgfsys@transformshift{4.640354in}{3.182042in}%
\pgfsys@useobject{currentmarker}{}%
\end{pgfscope}%
\begin{pgfscope}%
\pgfsys@transformshift{4.640432in}{3.180639in}%
\pgfsys@useobject{currentmarker}{}%
\end{pgfscope}%
\begin{pgfscope}%
\pgfsys@transformshift{4.640519in}{3.179871in}%
\pgfsys@useobject{currentmarker}{}%
\end{pgfscope}%
\begin{pgfscope}%
\pgfsys@transformshift{4.640123in}{3.177735in}%
\pgfsys@useobject{currentmarker}{}%
\end{pgfscope}%
\begin{pgfscope}%
\pgfsys@transformshift{4.640938in}{3.173754in}%
\pgfsys@useobject{currentmarker}{}%
\end{pgfscope}%
\begin{pgfscope}%
\pgfsys@transformshift{4.640752in}{3.171526in}%
\pgfsys@useobject{currentmarker}{}%
\end{pgfscope}%
\begin{pgfscope}%
\pgfsys@transformshift{4.640750in}{3.170297in}%
\pgfsys@useobject{currentmarker}{}%
\end{pgfscope}%
\begin{pgfscope}%
\pgfsys@transformshift{4.640900in}{3.169637in}%
\pgfsys@useobject{currentmarker}{}%
\end{pgfscope}%
\begin{pgfscope}%
\pgfsys@transformshift{4.640722in}{3.167050in}%
\pgfsys@useobject{currentmarker}{}%
\end{pgfscope}%
\begin{pgfscope}%
\pgfsys@transformshift{4.640796in}{3.165625in}%
\pgfsys@useobject{currentmarker}{}%
\end{pgfscope}%
\begin{pgfscope}%
\pgfsys@transformshift{4.640893in}{3.164846in}%
\pgfsys@useobject{currentmarker}{}%
\end{pgfscope}%
\begin{pgfscope}%
\pgfsys@transformshift{4.640828in}{3.164420in}%
\pgfsys@useobject{currentmarker}{}%
\end{pgfscope}%
\begin{pgfscope}%
\pgfsys@transformshift{4.641186in}{3.161227in}%
\pgfsys@useobject{currentmarker}{}%
\end{pgfscope}%
\begin{pgfscope}%
\pgfsys@transformshift{4.641248in}{3.159461in}%
\pgfsys@useobject{currentmarker}{}%
\end{pgfscope}%
\begin{pgfscope}%
\pgfsys@transformshift{4.641164in}{3.158493in}%
\pgfsys@useobject{currentmarker}{}%
\end{pgfscope}%
\begin{pgfscope}%
\pgfsys@transformshift{4.641700in}{3.155350in}%
\pgfsys@useobject{currentmarker}{}%
\end{pgfscope}%
\begin{pgfscope}%
\pgfsys@transformshift{4.641552in}{3.150850in}%
\pgfsys@useobject{currentmarker}{}%
\end{pgfscope}%
\begin{pgfscope}%
\pgfsys@transformshift{4.641609in}{3.148374in}%
\pgfsys@useobject{currentmarker}{}%
\end{pgfscope}%
\begin{pgfscope}%
\pgfsys@transformshift{4.641915in}{3.147047in}%
\pgfsys@useobject{currentmarker}{}%
\end{pgfscope}%
\begin{pgfscope}%
\pgfsys@transformshift{4.641544in}{3.143954in}%
\pgfsys@useobject{currentmarker}{}%
\end{pgfscope}%
\begin{pgfscope}%
\pgfsys@transformshift{4.641740in}{3.142251in}%
\pgfsys@useobject{currentmarker}{}%
\end{pgfscope}%
\begin{pgfscope}%
\pgfsys@transformshift{4.642287in}{3.139484in}%
\pgfsys@useobject{currentmarker}{}%
\end{pgfscope}%
\begin{pgfscope}%
\pgfsys@transformshift{4.641843in}{3.134219in}%
\pgfsys@useobject{currentmarker}{}%
\end{pgfscope}%
\begin{pgfscope}%
\pgfsys@transformshift{4.642264in}{3.131343in}%
\pgfsys@useobject{currentmarker}{}%
\end{pgfscope}%
\begin{pgfscope}%
\pgfsys@transformshift{4.642513in}{3.129764in}%
\pgfsys@useobject{currentmarker}{}%
\end{pgfscope}%
\begin{pgfscope}%
\pgfsys@transformshift{4.642415in}{3.128890in}%
\pgfsys@useobject{currentmarker}{}%
\end{pgfscope}%
\begin{pgfscope}%
\pgfsys@transformshift{4.642828in}{3.126307in}%
\pgfsys@useobject{currentmarker}{}%
\end{pgfscope}%
\begin{pgfscope}%
\pgfsys@transformshift{4.642869in}{3.124869in}%
\pgfsys@useobject{currentmarker}{}%
\end{pgfscope}%
\begin{pgfscope}%
\pgfsys@transformshift{4.642835in}{3.124079in}%
\pgfsys@useobject{currentmarker}{}%
\end{pgfscope}%
\begin{pgfscope}%
\pgfsys@transformshift{4.643554in}{3.121152in}%
\pgfsys@useobject{currentmarker}{}%
\end{pgfscope}%
\begin{pgfscope}%
\pgfsys@transformshift{4.643412in}{3.116858in}%
\pgfsys@useobject{currentmarker}{}%
\end{pgfscope}%
\begin{pgfscope}%
\pgfsys@transformshift{4.643589in}{3.111471in}%
\pgfsys@useobject{currentmarker}{}%
\end{pgfscope}%
\begin{pgfscope}%
\pgfsys@transformshift{4.645540in}{3.102852in}%
\pgfsys@useobject{currentmarker}{}%
\end{pgfscope}%
\begin{pgfscope}%
\pgfsys@transformshift{4.645526in}{3.097991in}%
\pgfsys@useobject{currentmarker}{}%
\end{pgfscope}%
\begin{pgfscope}%
\pgfsys@transformshift{4.645474in}{3.095318in}%
\pgfsys@useobject{currentmarker}{}%
\end{pgfscope}%
\begin{pgfscope}%
\pgfsys@transformshift{4.645837in}{3.093893in}%
\pgfsys@useobject{currentmarker}{}%
\end{pgfscope}%
\begin{pgfscope}%
\pgfsys@transformshift{4.645741in}{3.090756in}%
\pgfsys@useobject{currentmarker}{}%
\end{pgfscope}%
\begin{pgfscope}%
\pgfsys@transformshift{4.645878in}{3.089035in}%
\pgfsys@useobject{currentmarker}{}%
\end{pgfscope}%
\begin{pgfscope}%
\pgfsys@transformshift{4.646107in}{3.088113in}%
\pgfsys@useobject{currentmarker}{}%
\end{pgfscope}%
\begin{pgfscope}%
\pgfsys@transformshift{4.646023in}{3.084431in}%
\pgfsys@useobject{currentmarker}{}%
\end{pgfscope}%
\begin{pgfscope}%
\pgfsys@transformshift{4.646281in}{3.082422in}%
\pgfsys@useobject{currentmarker}{}%
\end{pgfscope}%
\begin{pgfscope}%
\pgfsys@transformshift{4.646442in}{3.081320in}%
\pgfsys@useobject{currentmarker}{}%
\end{pgfscope}%
\begin{pgfscope}%
\pgfsys@transformshift{4.646229in}{3.079245in}%
\pgfsys@useobject{currentmarker}{}%
\end{pgfscope}%
\begin{pgfscope}%
\pgfsys@transformshift{4.646876in}{3.075580in}%
\pgfsys@useobject{currentmarker}{}%
\end{pgfscope}%
\begin{pgfscope}%
\pgfsys@transformshift{4.647185in}{3.070642in}%
\pgfsys@useobject{currentmarker}{}%
\end{pgfscope}%
\begin{pgfscope}%
\pgfsys@transformshift{4.646687in}{3.064805in}%
\pgfsys@useobject{currentmarker}{}%
\end{pgfscope}%
\begin{pgfscope}%
\pgfsys@transformshift{4.648443in}{3.056696in}%
\pgfsys@useobject{currentmarker}{}%
\end{pgfscope}%
\begin{pgfscope}%
\pgfsys@transformshift{4.648886in}{3.047510in}%
\pgfsys@useobject{currentmarker}{}%
\end{pgfscope}%
\begin{pgfscope}%
\pgfsys@transformshift{4.648221in}{3.036979in}%
\pgfsys@useobject{currentmarker}{}%
\end{pgfscope}%
\begin{pgfscope}%
\pgfsys@transformshift{4.651634in}{3.024427in}%
\pgfsys@useobject{currentmarker}{}%
\end{pgfscope}%
\begin{pgfscope}%
\pgfsys@transformshift{4.651494in}{3.017274in}%
\pgfsys@useobject{currentmarker}{}%
\end{pgfscope}%
\begin{pgfscope}%
\pgfsys@transformshift{4.651699in}{3.013344in}%
\pgfsys@useobject{currentmarker}{}%
\end{pgfscope}%
\begin{pgfscope}%
\pgfsys@transformshift{4.652308in}{3.011267in}%
\pgfsys@useobject{currentmarker}{}%
\end{pgfscope}%
\begin{pgfscope}%
\pgfsys@transformshift{4.651844in}{3.006968in}%
\pgfsys@useobject{currentmarker}{}%
\end{pgfscope}%
\begin{pgfscope}%
\pgfsys@transformshift{4.652025in}{3.004597in}%
\pgfsys@useobject{currentmarker}{}%
\end{pgfscope}%
\begin{pgfscope}%
\pgfsys@transformshift{4.652227in}{3.003304in}%
\pgfsys@useobject{currentmarker}{}%
\end{pgfscope}%
\begin{pgfscope}%
\pgfsys@transformshift{4.652245in}{3.000254in}%
\pgfsys@useobject{currentmarker}{}%
\end{pgfscope}%
\begin{pgfscope}%
\pgfsys@transformshift{4.652388in}{2.998582in}%
\pgfsys@useobject{currentmarker}{}%
\end{pgfscope}%
\begin{pgfscope}%
\pgfsys@transformshift{4.652587in}{2.997681in}%
\pgfsys@useobject{currentmarker}{}%
\end{pgfscope}%
\begin{pgfscope}%
\pgfsys@transformshift{4.652442in}{2.995284in}%
\pgfsys@useobject{currentmarker}{}%
\end{pgfscope}%
\begin{pgfscope}%
\pgfsys@transformshift{4.652606in}{2.993974in}%
\pgfsys@useobject{currentmarker}{}%
\end{pgfscope}%
\begin{pgfscope}%
\pgfsys@transformshift{4.652736in}{2.993259in}%
\pgfsys@useobject{currentmarker}{}%
\end{pgfscope}%
\begin{pgfscope}%
\pgfsys@transformshift{4.652693in}{2.992862in}%
\pgfsys@useobject{currentmarker}{}%
\end{pgfscope}%
\begin{pgfscope}%
\pgfsys@transformshift{4.653238in}{2.990128in}%
\pgfsys@useobject{currentmarker}{}%
\end{pgfscope}%
\begin{pgfscope}%
\pgfsys@transformshift{4.653503in}{2.986054in}%
\pgfsys@useobject{currentmarker}{}%
\end{pgfscope}%
\begin{pgfscope}%
\pgfsys@transformshift{4.653884in}{2.981034in}%
\pgfsys@useobject{currentmarker}{}%
\end{pgfscope}%
\begin{pgfscope}%
\pgfsys@transformshift{4.656502in}{2.973913in}%
\pgfsys@useobject{currentmarker}{}%
\end{pgfscope}%
\begin{pgfscope}%
\pgfsys@transformshift{4.655384in}{2.965353in}%
\pgfsys@useobject{currentmarker}{}%
\end{pgfscope}%
\begin{pgfscope}%
\pgfsys@transformshift{4.655992in}{2.960643in}%
\pgfsys@useobject{currentmarker}{}%
\end{pgfscope}%
\begin{pgfscope}%
\pgfsys@transformshift{4.657581in}{2.954792in}%
\pgfsys@useobject{currentmarker}{}%
\end{pgfscope}%
\begin{pgfscope}%
\pgfsys@transformshift{4.656282in}{2.945703in}%
\pgfsys@useobject{currentmarker}{}%
\end{pgfscope}%
\begin{pgfscope}%
\pgfsys@transformshift{4.656934in}{2.940695in}%
\pgfsys@useobject{currentmarker}{}%
\end{pgfscope}%
\begin{pgfscope}%
\pgfsys@transformshift{4.657546in}{2.937986in}%
\pgfsys@useobject{currentmarker}{}%
\end{pgfscope}%
\begin{pgfscope}%
\pgfsys@transformshift{4.656957in}{2.934026in}%
\pgfsys@useobject{currentmarker}{}%
\end{pgfscope}%
\begin{pgfscope}%
\pgfsys@transformshift{4.658096in}{2.928227in}%
\pgfsys@useobject{currentmarker}{}%
\end{pgfscope}%
\begin{pgfscope}%
\pgfsys@transformshift{4.658319in}{2.924985in}%
\pgfsys@useobject{currentmarker}{}%
\end{pgfscope}%
\begin{pgfscope}%
\pgfsys@transformshift{4.658075in}{2.923214in}%
\pgfsys@useobject{currentmarker}{}%
\end{pgfscope}%
\begin{pgfscope}%
\pgfsys@transformshift{4.658712in}{2.918693in}%
\pgfsys@useobject{currentmarker}{}%
\end{pgfscope}%
\begin{pgfscope}%
\pgfsys@transformshift{4.658929in}{2.916191in}%
\pgfsys@useobject{currentmarker}{}%
\end{pgfscope}%
\begin{pgfscope}%
\pgfsys@transformshift{4.658873in}{2.914811in}%
\pgfsys@useobject{currentmarker}{}%
\end{pgfscope}%
\begin{pgfscope}%
\pgfsys@transformshift{4.659553in}{2.910185in}%
\pgfsys@useobject{currentmarker}{}%
\end{pgfscope}%
\begin{pgfscope}%
\pgfsys@transformshift{4.659941in}{2.904063in}%
\pgfsys@useobject{currentmarker}{}%
\end{pgfscope}%
\begin{pgfscope}%
\pgfsys@transformshift{4.659330in}{2.895208in}%
\pgfsys@useobject{currentmarker}{}%
\end{pgfscope}%
\begin{pgfscope}%
\pgfsys@transformshift{4.661493in}{2.883738in}%
\pgfsys@useobject{currentmarker}{}%
\end{pgfscope}%
\begin{pgfscope}%
\pgfsys@transformshift{4.662498in}{2.877398in}%
\pgfsys@useobject{currentmarker}{}%
\end{pgfscope}%
\begin{pgfscope}%
\pgfsys@transformshift{4.661756in}{2.869837in}%
\pgfsys@useobject{currentmarker}{}%
\end{pgfscope}%
\begin{pgfscope}%
\pgfsys@transformshift{4.664059in}{2.860359in}%
\pgfsys@useobject{currentmarker}{}%
\end{pgfscope}%
\begin{pgfscope}%
\pgfsys@transformshift{4.664518in}{2.849623in}%
\pgfsys@useobject{currentmarker}{}%
\end{pgfscope}%
\begin{pgfscope}%
\pgfsys@transformshift{4.664446in}{2.843713in}%
\pgfsys@useobject{currentmarker}{}%
\end{pgfscope}%
\begin{pgfscope}%
\pgfsys@transformshift{4.665953in}{2.835693in}%
\pgfsys@useobject{currentmarker}{}%
\end{pgfscope}%
\begin{pgfscope}%
\pgfsys@transformshift{4.666195in}{2.831212in}%
\pgfsys@useobject{currentmarker}{}%
\end{pgfscope}%
\begin{pgfscope}%
\pgfsys@transformshift{4.666136in}{2.825493in}%
\pgfsys@useobject{currentmarker}{}%
\end{pgfscope}%
\begin{pgfscope}%
\pgfsys@transformshift{4.667934in}{2.818350in}%
\pgfsys@useobject{currentmarker}{}%
\end{pgfscope}%
\begin{pgfscope}%
\pgfsys@transformshift{4.667864in}{2.814300in}%
\pgfsys@useobject{currentmarker}{}%
\end{pgfscope}%
\begin{pgfscope}%
\pgfsys@transformshift{4.667846in}{2.812072in}%
\pgfsys@useobject{currentmarker}{}%
\end{pgfscope}%
\begin{pgfscope}%
\pgfsys@transformshift{4.668122in}{2.810878in}%
\pgfsys@useobject{currentmarker}{}%
\end{pgfscope}%
\begin{pgfscope}%
\pgfsys@transformshift{4.668006in}{2.807733in}%
\pgfsys@useobject{currentmarker}{}%
\end{pgfscope}%
\begin{pgfscope}%
\pgfsys@transformshift{4.668053in}{2.806004in}%
\pgfsys@useobject{currentmarker}{}%
\end{pgfscope}%
\begin{pgfscope}%
\pgfsys@transformshift{4.668547in}{2.803409in}%
\pgfsys@useobject{currentmarker}{}%
\end{pgfscope}%
\begin{pgfscope}%
\pgfsys@transformshift{4.668268in}{2.799213in}%
\pgfsys@useobject{currentmarker}{}%
\end{pgfscope}%
\begin{pgfscope}%
\pgfsys@transformshift{4.668626in}{2.796928in}%
\pgfsys@useobject{currentmarker}{}%
\end{pgfscope}%
\begin{pgfscope}%
\pgfsys@transformshift{4.668809in}{2.795669in}%
\pgfsys@useobject{currentmarker}{}%
\end{pgfscope}%
\begin{pgfscope}%
\pgfsys@transformshift{4.668713in}{2.794976in}%
\pgfsys@useobject{currentmarker}{}%
\end{pgfscope}%
\begin{pgfscope}%
\pgfsys@transformshift{4.669200in}{2.792424in}%
\pgfsys@useobject{currentmarker}{}%
\end{pgfscope}%
\begin{pgfscope}%
\pgfsys@transformshift{4.669256in}{2.788753in}%
\pgfsys@useobject{currentmarker}{}%
\end{pgfscope}%
\begin{pgfscope}%
\pgfsys@transformshift{4.669091in}{2.784181in}%
\pgfsys@useobject{currentmarker}{}%
\end{pgfscope}%
\begin{pgfscope}%
\pgfsys@transformshift{4.670884in}{2.777110in}%
\pgfsys@useobject{currentmarker}{}%
\end{pgfscope}%
\begin{pgfscope}%
\pgfsys@transformshift{4.670579in}{2.768483in}%
\pgfsys@useobject{currentmarker}{}%
\end{pgfscope}%
\begin{pgfscope}%
\pgfsys@transformshift{4.669887in}{2.758958in}%
\pgfsys@useobject{currentmarker}{}%
\end{pgfscope}%
\begin{pgfscope}%
\pgfsys@transformshift{4.673319in}{2.747823in}%
\pgfsys@useobject{currentmarker}{}%
\end{pgfscope}%
\begin{pgfscope}%
\pgfsys@transformshift{4.673225in}{2.741415in}%
\pgfsys@useobject{currentmarker}{}%
\end{pgfscope}%
\begin{pgfscope}%
\pgfsys@transformshift{4.673272in}{2.737890in}%
\pgfsys@useobject{currentmarker}{}%
\end{pgfscope}%
\begin{pgfscope}%
\pgfsys@transformshift{4.673863in}{2.736044in}%
\pgfsys@useobject{currentmarker}{}%
\end{pgfscope}%
\begin{pgfscope}%
\pgfsys@transformshift{4.673290in}{2.732170in}%
\pgfsys@useobject{currentmarker}{}%
\end{pgfscope}%
\begin{pgfscope}%
\pgfsys@transformshift{4.673504in}{2.730026in}%
\pgfsys@useobject{currentmarker}{}%
\end{pgfscope}%
\begin{pgfscope}%
\pgfsys@transformshift{4.673777in}{2.728873in}%
\pgfsys@useobject{currentmarker}{}%
\end{pgfscope}%
\begin{pgfscope}%
\pgfsys@transformshift{4.673431in}{2.725414in}%
\pgfsys@useobject{currentmarker}{}%
\end{pgfscope}%
\begin{pgfscope}%
\pgfsys@transformshift{4.674283in}{2.719625in}%
\pgfsys@useobject{currentmarker}{}%
\end{pgfscope}%
\begin{pgfscope}%
\pgfsys@transformshift{4.675394in}{2.712379in}%
\pgfsys@useobject{currentmarker}{}%
\end{pgfscope}%
\begin{pgfscope}%
\pgfsys@transformshift{4.675119in}{2.703986in}%
\pgfsys@useobject{currentmarker}{}%
\end{pgfscope}%
\begin{pgfscope}%
\pgfsys@transformshift{4.678047in}{2.692449in}%
\pgfsys@useobject{currentmarker}{}%
\end{pgfscope}%
\begin{pgfscope}%
\pgfsys@transformshift{4.678172in}{2.685904in}%
\pgfsys@useobject{currentmarker}{}%
\end{pgfscope}%
\begin{pgfscope}%
\pgfsys@transformshift{4.678153in}{2.682304in}%
\pgfsys@useobject{currentmarker}{}%
\end{pgfscope}%
\begin{pgfscope}%
\pgfsys@transformshift{4.678757in}{2.680418in}%
\pgfsys@useobject{currentmarker}{}%
\end{pgfscope}%
\begin{pgfscope}%
\pgfsys@transformshift{4.678144in}{2.675314in}%
\pgfsys@useobject{currentmarker}{}%
\end{pgfscope}%
\begin{pgfscope}%
\pgfsys@transformshift{4.678550in}{2.672515in}%
\pgfsys@useobject{currentmarker}{}%
\end{pgfscope}%
\begin{pgfscope}%
\pgfsys@transformshift{4.679295in}{2.668603in}%
\pgfsys@useobject{currentmarker}{}%
\end{pgfscope}%
\begin{pgfscope}%
\pgfsys@transformshift{4.678937in}{2.666442in}%
\pgfsys@useobject{currentmarker}{}%
\end{pgfscope}%
\begin{pgfscope}%
\pgfsys@transformshift{4.679987in}{2.661993in}%
\pgfsys@useobject{currentmarker}{}%
\end{pgfscope}%
\begin{pgfscope}%
\pgfsys@transformshift{4.680304in}{2.656266in}%
\pgfsys@useobject{currentmarker}{}%
\end{pgfscope}%
\begin{pgfscope}%
\pgfsys@transformshift{4.680130in}{2.649628in}%
\pgfsys@useobject{currentmarker}{}%
\end{pgfscope}%
\begin{pgfscope}%
\pgfsys@transformshift{4.682731in}{2.641926in}%
\pgfsys@useobject{currentmarker}{}%
\end{pgfscope}%
\begin{pgfscope}%
\pgfsys@transformshift{4.681159in}{2.632329in}%
\pgfsys@useobject{currentmarker}{}%
\end{pgfscope}%
\begin{pgfscope}%
\pgfsys@transformshift{4.683344in}{2.620783in}%
\pgfsys@useobject{currentmarker}{}%
\end{pgfscope}%
\begin{pgfscope}%
\pgfsys@transformshift{4.684412in}{2.608132in}%
\pgfsys@useobject{currentmarker}{}%
\end{pgfscope}%
\begin{pgfscope}%
\pgfsys@transformshift{4.683663in}{2.601189in}%
\pgfsys@useobject{currentmarker}{}%
\end{pgfscope}%
\begin{pgfscope}%
\pgfsys@transformshift{4.687221in}{2.591080in}%
\pgfsys@useobject{currentmarker}{}%
\end{pgfscope}%
\begin{pgfscope}%
\pgfsys@transformshift{4.687138in}{2.585186in}%
\pgfsys@useobject{currentmarker}{}%
\end{pgfscope}%
\begin{pgfscope}%
\pgfsys@transformshift{4.687195in}{2.581945in}%
\pgfsys@useobject{currentmarker}{}%
\end{pgfscope}%
\begin{pgfscope}%
\pgfsys@transformshift{4.687664in}{2.580225in}%
\pgfsys@useobject{currentmarker}{}%
\end{pgfscope}%
\begin{pgfscope}%
\pgfsys@transformshift{4.687462in}{2.576164in}%
\pgfsys@useobject{currentmarker}{}%
\end{pgfscope}%
\begin{pgfscope}%
\pgfsys@transformshift{4.688411in}{2.570457in}%
\pgfsys@useobject{currentmarker}{}%
\end{pgfscope}%
\begin{pgfscope}%
\pgfsys@transformshift{4.689652in}{2.563704in}%
\pgfsys@useobject{currentmarker}{}%
\end{pgfscope}%
\begin{pgfscope}%
\pgfsys@transformshift{4.689287in}{2.555920in}%
\pgfsys@useobject{currentmarker}{}%
\end{pgfscope}%
\begin{pgfscope}%
\pgfsys@transformshift{4.691332in}{2.545828in}%
\pgfsys@useobject{currentmarker}{}%
\end{pgfscope}%
\begin{pgfscope}%
\pgfsys@transformshift{4.692124in}{2.534117in}%
\pgfsys@useobject{currentmarker}{}%
\end{pgfscope}%
\begin{pgfscope}%
\pgfsys@transformshift{4.692043in}{2.527662in}%
\pgfsys@useobject{currentmarker}{}%
\end{pgfscope}%
\begin{pgfscope}%
\pgfsys@transformshift{4.694518in}{2.520656in}%
\pgfsys@useobject{currentmarker}{}%
\end{pgfscope}%
\begin{pgfscope}%
\pgfsys@transformshift{4.694626in}{2.510903in}%
\pgfsys@useobject{currentmarker}{}%
\end{pgfscope}%
\begin{pgfscope}%
\pgfsys@transformshift{4.695395in}{2.505593in}%
\pgfsys@useobject{currentmarker}{}%
\end{pgfscope}%
\begin{pgfscope}%
\pgfsys@transformshift{4.696195in}{2.502753in}%
\pgfsys@useobject{currentmarker}{}%
\end{pgfscope}%
\begin{pgfscope}%
\pgfsys@transformshift{4.696005in}{2.501142in}%
\pgfsys@useobject{currentmarker}{}%
\end{pgfscope}%
\begin{pgfscope}%
\pgfsys@transformshift{4.696903in}{2.497422in}%
\pgfsys@useobject{currentmarker}{}%
\end{pgfscope}%
\begin{pgfscope}%
\pgfsys@transformshift{4.697113in}{2.495327in}%
\pgfsys@useobject{currentmarker}{}%
\end{pgfscope}%
\begin{pgfscope}%
\pgfsys@transformshift{4.696966in}{2.491553in}%
\pgfsys@useobject{currentmarker}{}%
\end{pgfscope}%
\begin{pgfscope}%
\pgfsys@transformshift{4.699309in}{2.484980in}%
\pgfsys@useobject{currentmarker}{}%
\end{pgfscope}%
\begin{pgfscope}%
\pgfsys@transformshift{4.698960in}{2.481158in}%
\pgfsys@useobject{currentmarker}{}%
\end{pgfscope}%
\begin{pgfscope}%
\pgfsys@transformshift{4.700073in}{2.475893in}%
\pgfsys@useobject{currentmarker}{}%
\end{pgfscope}%
\begin{pgfscope}%
\pgfsys@transformshift{4.700756in}{2.473013in}%
\pgfsys@useobject{currentmarker}{}%
\end{pgfscope}%
\begin{pgfscope}%
\pgfsys@transformshift{4.700042in}{2.468443in}%
\pgfsys@useobject{currentmarker}{}%
\end{pgfscope}%
\begin{pgfscope}%
\pgfsys@transformshift{4.701851in}{2.461993in}%
\pgfsys@useobject{currentmarker}{}%
\end{pgfscope}%
\begin{pgfscope}%
\pgfsys@transformshift{4.702886in}{2.454396in}%
\pgfsys@useobject{currentmarker}{}%
\end{pgfscope}%
\begin{pgfscope}%
\pgfsys@transformshift{4.702320in}{2.445201in}%
\pgfsys@useobject{currentmarker}{}%
\end{pgfscope}%
\begin{pgfscope}%
\pgfsys@transformshift{4.705782in}{2.432982in}%
\pgfsys@useobject{currentmarker}{}%
\end{pgfscope}%
\begin{pgfscope}%
\pgfsys@transformshift{4.706184in}{2.426009in}%
\pgfsys@useobject{currentmarker}{}%
\end{pgfscope}%
\begin{pgfscope}%
\pgfsys@transformshift{4.706067in}{2.422169in}%
\pgfsys@useobject{currentmarker}{}%
\end{pgfscope}%
\begin{pgfscope}%
\pgfsys@transformshift{4.707773in}{2.414672in}%
\pgfsys@useobject{currentmarker}{}%
\end{pgfscope}%
\begin{pgfscope}%
\pgfsys@transformshift{4.708132in}{2.410459in}%
\pgfsys@useobject{currentmarker}{}%
\end{pgfscope}%
\begin{pgfscope}%
\pgfsys@transformshift{4.707796in}{2.408158in}%
\pgfsys@useobject{currentmarker}{}%
\end{pgfscope}%
\begin{pgfscope}%
\pgfsys@transformshift{4.708785in}{2.401309in}%
\pgfsys@useobject{currentmarker}{}%
\end{pgfscope}%
\begin{pgfscope}%
\pgfsys@transformshift{4.712339in}{2.391416in}%
\pgfsys@useobject{currentmarker}{}%
\end{pgfscope}%
\begin{pgfscope}%
\pgfsys@transformshift{4.711319in}{2.375684in}%
\pgfsys@useobject{currentmarker}{}%
\end{pgfscope}%
\begin{pgfscope}%
\pgfsys@transformshift{4.711033in}{2.367017in}%
\pgfsys@useobject{currentmarker}{}%
\end{pgfscope}%
\begin{pgfscope}%
\pgfsys@transformshift{4.715093in}{2.353112in}%
\pgfsys@useobject{currentmarker}{}%
\end{pgfscope}%
\begin{pgfscope}%
\pgfsys@transformshift{4.715092in}{2.336471in}%
\pgfsys@useobject{currentmarker}{}%
\end{pgfscope}%
\begin{pgfscope}%
\pgfsys@transformshift{4.713195in}{2.327518in}%
\pgfsys@useobject{currentmarker}{}%
\end{pgfscope}%
\begin{pgfscope}%
\pgfsys@transformshift{4.717863in}{2.310193in}%
\pgfsys@useobject{currentmarker}{}%
\end{pgfscope}%
\begin{pgfscope}%
\pgfsys@transformshift{4.716578in}{2.300408in}%
\pgfsys@useobject{currentmarker}{}%
\end{pgfscope}%
\begin{pgfscope}%
\pgfsys@transformshift{4.714420in}{2.289369in}%
\pgfsys@useobject{currentmarker}{}%
\end{pgfscope}%
\begin{pgfscope}%
\pgfsys@transformshift{4.719775in}{2.274072in}%
\pgfsys@useobject{currentmarker}{}%
\end{pgfscope}%
\begin{pgfscope}%
\pgfsys@transformshift{4.716879in}{2.255176in}%
\pgfsys@useobject{currentmarker}{}%
\end{pgfscope}%
\begin{pgfscope}%
\pgfsys@transformshift{4.717323in}{2.244671in}%
\pgfsys@useobject{currentmarker}{}%
\end{pgfscope}%
\begin{pgfscope}%
\pgfsys@transformshift{4.719208in}{2.239204in}%
\pgfsys@useobject{currentmarker}{}%
\end{pgfscope}%
\begin{pgfscope}%
\pgfsys@transformshift{4.717281in}{2.227273in}%
\pgfsys@useobject{currentmarker}{}%
\end{pgfscope}%
\begin{pgfscope}%
\pgfsys@transformshift{4.720275in}{2.214137in}%
\pgfsys@useobject{currentmarker}{}%
\end{pgfscope}%
\begin{pgfscope}%
\pgfsys@transformshift{4.722862in}{2.199895in}%
\pgfsys@useobject{currentmarker}{}%
\end{pgfscope}%
\begin{pgfscope}%
\pgfsys@transformshift{4.721022in}{2.192150in}%
\pgfsys@useobject{currentmarker}{}%
\end{pgfscope}%
\begin{pgfscope}%
\pgfsys@transformshift{4.721379in}{2.179456in}%
\pgfsys@useobject{currentmarker}{}%
\end{pgfscope}%
\begin{pgfscope}%
\pgfsys@transformshift{4.723056in}{2.165786in}%
\pgfsys@useobject{currentmarker}{}%
\end{pgfscope}%
\begin{pgfscope}%
\pgfsys@transformshift{4.722630in}{2.158223in}%
\pgfsys@useobject{currentmarker}{}%
\end{pgfscope}%
\begin{pgfscope}%
\pgfsys@transformshift{4.724491in}{2.145583in}%
\pgfsys@useobject{currentmarker}{}%
\end{pgfscope}%
\begin{pgfscope}%
\pgfsys@transformshift{4.725636in}{2.131849in}%
\pgfsys@useobject{currentmarker}{}%
\end{pgfscope}%
\begin{pgfscope}%
\pgfsys@transformshift{4.723645in}{2.117283in}%
\pgfsys@useobject{currentmarker}{}%
\end{pgfscope}%
\begin{pgfscope}%
\pgfsys@transformshift{4.728384in}{2.096691in}%
\pgfsys@useobject{currentmarker}{}%
\end{pgfscope}%
\begin{pgfscope}%
\pgfsys@transformshift{4.728033in}{2.085074in}%
\pgfsys@useobject{currentmarker}{}%
\end{pgfscope}%
\begin{pgfscope}%
\pgfsys@transformshift{4.726726in}{2.072596in}%
\pgfsys@useobject{currentmarker}{}%
\end{pgfscope}%
\begin{pgfscope}%
\pgfsys@transformshift{4.733026in}{2.057066in}%
\pgfsys@useobject{currentmarker}{}%
\end{pgfscope}%
\begin{pgfscope}%
\pgfsys@transformshift{4.731883in}{2.039020in}%
\pgfsys@useobject{currentmarker}{}%
\end{pgfscope}%
\begin{pgfscope}%
\pgfsys@transformshift{4.732817in}{2.029118in}%
\pgfsys@useobject{currentmarker}{}%
\end{pgfscope}%
\begin{pgfscope}%
\pgfsys@transformshift{4.735116in}{2.024154in}%
\pgfsys@useobject{currentmarker}{}%
\end{pgfscope}%
\begin{pgfscope}%
\pgfsys@transformshift{4.733368in}{2.012875in}%
\pgfsys@useobject{currentmarker}{}%
\end{pgfscope}%
\begin{pgfscope}%
\pgfsys@transformshift{4.735523in}{2.000252in}%
\pgfsys@useobject{currentmarker}{}%
\end{pgfscope}%
\begin{pgfscope}%
\pgfsys@transformshift{4.737331in}{1.993444in}%
\pgfsys@useobject{currentmarker}{}%
\end{pgfscope}%
\begin{pgfscope}%
\pgfsys@transformshift{4.735365in}{1.983615in}%
\pgfsys@useobject{currentmarker}{}%
\end{pgfscope}%
\begin{pgfscope}%
\pgfsys@transformshift{4.739067in}{1.970059in}%
\pgfsys@useobject{currentmarker}{}%
\end{pgfscope}%
\begin{pgfscope}%
\pgfsys@transformshift{4.739228in}{1.962331in}%
\pgfsys@useobject{currentmarker}{}%
\end{pgfscope}%
\begin{pgfscope}%
\pgfsys@transformshift{4.738594in}{1.953750in}%
\pgfsys@useobject{currentmarker}{}%
\end{pgfscope}%
\begin{pgfscope}%
\pgfsys@transformshift{4.744171in}{1.939785in}%
\pgfsys@useobject{currentmarker}{}%
\end{pgfscope}%
\begin{pgfscope}%
\pgfsys@transformshift{4.743039in}{1.923756in}%
\pgfsys@useobject{currentmarker}{}%
\end{pgfscope}%
\begin{pgfscope}%
\pgfsys@transformshift{4.741904in}{1.906712in}%
\pgfsys@useobject{currentmarker}{}%
\end{pgfscope}%
\begin{pgfscope}%
\pgfsys@transformshift{4.745536in}{1.898047in}%
\pgfsys@useobject{currentmarker}{}%
\end{pgfscope}%
\begin{pgfscope}%
\pgfsys@transformshift{4.743790in}{1.882303in}%
\pgfsys@useobject{currentmarker}{}%
\end{pgfscope}%
\begin{pgfscope}%
\pgfsys@transformshift{4.747115in}{1.865642in}%
\pgfsys@useobject{currentmarker}{}%
\end{pgfscope}%
\begin{pgfscope}%
\pgfsys@transformshift{4.748077in}{1.856348in}%
\pgfsys@useobject{currentmarker}{}%
\end{pgfscope}%
\begin{pgfscope}%
\pgfsys@transformshift{4.747365in}{1.851259in}%
\pgfsys@useobject{currentmarker}{}%
\end{pgfscope}%
\begin{pgfscope}%
\pgfsys@transformshift{4.749800in}{1.841429in}%
\pgfsys@useobject{currentmarker}{}%
\end{pgfscope}%
\begin{pgfscope}%
\pgfsys@transformshift{4.749152in}{1.829956in}%
\pgfsys@useobject{currentmarker}{}%
\end{pgfscope}%
\begin{pgfscope}%
\pgfsys@transformshift{4.749640in}{1.816475in}%
\pgfsys@useobject{currentmarker}{}%
\end{pgfscope}%
\begin{pgfscope}%
\pgfsys@transformshift{4.752914in}{1.809816in}%
\pgfsys@useobject{currentmarker}{}%
\end{pgfscope}%
\begin{pgfscope}%
\pgfsys@transformshift{4.751373in}{1.796319in}%
\pgfsys@useobject{currentmarker}{}%
\end{pgfscope}%
\begin{pgfscope}%
\pgfsys@transformshift{4.752857in}{1.788996in}%
\pgfsys@useobject{currentmarker}{}%
\end{pgfscope}%
\begin{pgfscope}%
\pgfsys@transformshift{4.754492in}{1.785226in}%
\pgfsys@useobject{currentmarker}{}%
\end{pgfscope}%
\begin{pgfscope}%
\pgfsys@transformshift{4.754091in}{1.776397in}%
\pgfsys@useobject{currentmarker}{}%
\end{pgfscope}%
\begin{pgfscope}%
\pgfsys@transformshift{4.756222in}{1.766219in}%
\pgfsys@useobject{currentmarker}{}%
\end{pgfscope}%
\begin{pgfscope}%
\pgfsys@transformshift{4.757495in}{1.760644in}%
\pgfsys@useobject{currentmarker}{}%
\end{pgfscope}%
\begin{pgfscope}%
\pgfsys@transformshift{4.756672in}{1.751866in}%
\pgfsys@useobject{currentmarker}{}%
\end{pgfscope}%
\begin{pgfscope}%
\pgfsys@transformshift{4.760113in}{1.740467in}%
\pgfsys@useobject{currentmarker}{}%
\end{pgfscope}%
\begin{pgfscope}%
\pgfsys@transformshift{4.760219in}{1.733919in}%
\pgfsys@useobject{currentmarker}{}%
\end{pgfscope}%
\begin{pgfscope}%
\pgfsys@transformshift{4.760336in}{1.730319in}%
\pgfsys@useobject{currentmarker}{}%
\end{pgfscope}%
\begin{pgfscope}%
\pgfsys@transformshift{4.762558in}{1.724384in}%
\pgfsys@useobject{currentmarker}{}%
\end{pgfscope}%
\begin{pgfscope}%
\pgfsys@transformshift{4.760744in}{1.715310in}%
\pgfsys@useobject{currentmarker}{}%
\end{pgfscope}%
\begin{pgfscope}%
\pgfsys@transformshift{4.761606in}{1.703663in}%
\pgfsys@useobject{currentmarker}{}%
\end{pgfscope}%
\begin{pgfscope}%
\pgfsys@transformshift{4.764699in}{1.691343in}%
\pgfsys@useobject{currentmarker}{}%
\end{pgfscope}%
\begin{pgfscope}%
\pgfsys@transformshift{4.762251in}{1.671902in}%
\pgfsys@useobject{currentmarker}{}%
\end{pgfscope}%
\begin{pgfscope}%
\pgfsys@transformshift{4.764864in}{1.649555in}%
\pgfsys@useobject{currentmarker}{}%
\end{pgfscope}%
\begin{pgfscope}%
\pgfsys@transformshift{4.766078in}{1.624789in}%
\pgfsys@useobject{currentmarker}{}%
\end{pgfscope}%
\begin{pgfscope}%
\pgfsys@transformshift{4.764847in}{1.611207in}%
\pgfsys@useobject{currentmarker}{}%
\end{pgfscope}%
\begin{pgfscope}%
\pgfsys@transformshift{4.770157in}{1.596810in}%
\pgfsys@useobject{currentmarker}{}%
\end{pgfscope}%
\begin{pgfscope}%
\pgfsys@transformshift{4.768246in}{1.576781in}%
\pgfsys@useobject{currentmarker}{}%
\end{pgfscope}%
\begin{pgfscope}%
\pgfsys@transformshift{4.770107in}{1.554891in}%
\pgfsys@useobject{currentmarker}{}%
\end{pgfscope}%
\begin{pgfscope}%
\pgfsys@transformshift{4.776275in}{1.532235in}%
\pgfsys@useobject{currentmarker}{}%
\end{pgfscope}%
\begin{pgfscope}%
\pgfsys@transformshift{4.774906in}{1.519394in}%
\pgfsys@useobject{currentmarker}{}%
\end{pgfscope}%
\begin{pgfscope}%
\pgfsys@transformshift{4.778632in}{1.501289in}%
\pgfsys@useobject{currentmarker}{}%
\end{pgfscope}%
\begin{pgfscope}%
\pgfsys@transformshift{4.778944in}{1.491128in}%
\pgfsys@useobject{currentmarker}{}%
\end{pgfscope}%
\begin{pgfscope}%
\pgfsys@transformshift{4.779057in}{1.485537in}%
\pgfsys@useobject{currentmarker}{}%
\end{pgfscope}%
\begin{pgfscope}%
\pgfsys@transformshift{4.782193in}{1.478647in}%
\pgfsys@useobject{currentmarker}{}%
\end{pgfscope}%
\begin{pgfscope}%
\pgfsys@transformshift{4.781313in}{1.468243in}%
\pgfsys@useobject{currentmarker}{}%
\end{pgfscope}%
\begin{pgfscope}%
\pgfsys@transformshift{4.781896in}{1.462530in}%
\pgfsys@useobject{currentmarker}{}%
\end{pgfscope}%
\begin{pgfscope}%
\pgfsys@transformshift{4.784742in}{1.456242in}%
\pgfsys@useobject{currentmarker}{}%
\end{pgfscope}%
\begin{pgfscope}%
\pgfsys@transformshift{4.783471in}{1.442794in}%
\pgfsys@useobject{currentmarker}{}%
\end{pgfscope}%
\begin{pgfscope}%
\pgfsys@transformshift{4.786782in}{1.428493in}%
\pgfsys@useobject{currentmarker}{}%
\end{pgfscope}%
\begin{pgfscope}%
\pgfsys@transformshift{4.788161in}{1.420538in}%
\pgfsys@useobject{currentmarker}{}%
\end{pgfscope}%
\begin{pgfscope}%
\pgfsys@transformshift{4.786359in}{1.411444in}%
\pgfsys@useobject{currentmarker}{}%
\end{pgfscope}%
\begin{pgfscope}%
\pgfsys@transformshift{4.790564in}{1.397989in}%
\pgfsys@useobject{currentmarker}{}%
\end{pgfscope}%
\begin{pgfscope}%
\pgfsys@transformshift{4.790115in}{1.390249in}%
\pgfsys@useobject{currentmarker}{}%
\end{pgfscope}%
\begin{pgfscope}%
\pgfsys@transformshift{4.790101in}{1.385985in}%
\pgfsys@useobject{currentmarker}{}%
\end{pgfscope}%
\begin{pgfscope}%
\pgfsys@transformshift{4.791845in}{1.379836in}%
\pgfsys@useobject{currentmarker}{}%
\end{pgfscope}%
\begin{pgfscope}%
\pgfsys@transformshift{4.791488in}{1.370001in}%
\pgfsys@useobject{currentmarker}{}%
\end{pgfscope}%
\begin{pgfscope}%
\pgfsys@transformshift{4.789639in}{1.358926in}%
\pgfsys@useobject{currentmarker}{}%
\end{pgfscope}%
\begin{pgfscope}%
\pgfsys@transformshift{4.793318in}{1.347378in}%
\pgfsys@useobject{currentmarker}{}%
\end{pgfscope}%
\begin{pgfscope}%
\pgfsys@transformshift{4.793349in}{1.332860in}%
\pgfsys@useobject{currentmarker}{}%
\end{pgfscope}%
\begin{pgfscope}%
\pgfsys@transformshift{4.790962in}{1.317410in}%
\pgfsys@useobject{currentmarker}{}%
\end{pgfscope}%
\begin{pgfscope}%
\pgfsys@transformshift{4.798094in}{1.297268in}%
\pgfsys@useobject{currentmarker}{}%
\end{pgfscope}%
\begin{pgfscope}%
\pgfsys@transformshift{4.792316in}{1.271541in}%
\pgfsys@useobject{currentmarker}{}%
\end{pgfscope}%
\begin{pgfscope}%
\pgfsys@transformshift{4.802203in}{1.240317in}%
\pgfsys@useobject{currentmarker}{}%
\end{pgfscope}%
\begin{pgfscope}%
\pgfsys@transformshift{4.799231in}{1.204403in}%
\pgfsys@useobject{currentmarker}{}%
\end{pgfscope}%
\begin{pgfscope}%
\pgfsys@transformshift{4.805893in}{1.164412in}%
\pgfsys@useobject{currentmarker}{}%
\end{pgfscope}%
\begin{pgfscope}%
\pgfsys@transformshift{4.809838in}{1.142465in}%
\pgfsys@useobject{currentmarker}{}%
\end{pgfscope}%
\begin{pgfscope}%
\pgfsys@transformshift{4.806716in}{1.130605in}%
\pgfsys@useobject{currentmarker}{}%
\end{pgfscope}%
\begin{pgfscope}%
\pgfsys@transformshift{4.811039in}{1.118204in}%
\pgfsys@useobject{currentmarker}{}%
\end{pgfscope}%
\begin{pgfscope}%
\pgfsys@transformshift{4.808580in}{1.096187in}%
\pgfsys@useobject{currentmarker}{}%
\end{pgfscope}%
\begin{pgfscope}%
\pgfsys@transformshift{4.814390in}{1.072080in}%
\pgfsys@useobject{currentmarker}{}%
\end{pgfscope}%
\begin{pgfscope}%
\pgfsys@transformshift{4.811972in}{1.041961in}%
\pgfsys@useobject{currentmarker}{}%
\end{pgfscope}%
\begin{pgfscope}%
\pgfsys@transformshift{4.816321in}{1.010195in}%
\pgfsys@useobject{currentmarker}{}%
\end{pgfscope}%
\begin{pgfscope}%
\pgfsys@transformshift{4.817172in}{0.992582in}%
\pgfsys@useobject{currentmarker}{}%
\end{pgfscope}%
\begin{pgfscope}%
\pgfsys@transformshift{4.814448in}{0.983274in}%
\pgfsys@useobject{currentmarker}{}%
\end{pgfscope}%
\begin{pgfscope}%
\pgfsys@transformshift{4.816242in}{0.978250in}%
\pgfsys@useobject{currentmarker}{}%
\end{pgfscope}%
\begin{pgfscope}%
\pgfsys@transformshift{4.815222in}{0.964584in}%
\pgfsys@useobject{currentmarker}{}%
\end{pgfscope}%
\begin{pgfscope}%
\pgfsys@transformshift{4.821074in}{0.949478in}%
\pgfsys@useobject{currentmarker}{}%
\end{pgfscope}%
\begin{pgfscope}%
\pgfsys@transformshift{4.818412in}{0.928889in}%
\pgfsys@useobject{currentmarker}{}%
\end{pgfscope}%
\begin{pgfscope}%
\pgfsys@transformshift{4.825996in}{0.908582in}%
\pgfsys@useobject{currentmarker}{}%
\end{pgfscope}%
\begin{pgfscope}%
\pgfsys@transformshift{4.825658in}{0.885782in}%
\pgfsys@useobject{currentmarker}{}%
\end{pgfscope}%
\begin{pgfscope}%
\pgfsys@transformshift{4.826181in}{0.873252in}%
\pgfsys@useobject{currentmarker}{}%
\end{pgfscope}%
\begin{pgfscope}%
\pgfsys@transformshift{4.827229in}{0.866434in}%
\pgfsys@useobject{currentmarker}{}%
\end{pgfscope}%
\begin{pgfscope}%
\pgfsys@transformshift{4.828108in}{0.858521in}%
\pgfsys@useobject{currentmarker}{}%
\end{pgfscope}%
\begin{pgfscope}%
\pgfsys@transformshift{4.828687in}{0.854181in}%
\pgfsys@useobject{currentmarker}{}%
\end{pgfscope}%
\begin{pgfscope}%
\pgfsys@transformshift{4.828937in}{0.851786in}%
\pgfsys@useobject{currentmarker}{}%
\end{pgfscope}%
\begin{pgfscope}%
\pgfsys@transformshift{4.829073in}{0.850468in}%
\pgfsys@useobject{currentmarker}{}%
\end{pgfscope}%
\begin{pgfscope}%
\pgfsys@transformshift{4.829143in}{0.849743in}%
\pgfsys@useobject{currentmarker}{}%
\end{pgfscope}%
\begin{pgfscope}%
\pgfsys@transformshift{4.829156in}{0.849342in}%
\pgfsys@useobject{currentmarker}{}%
\end{pgfscope}%
\begin{pgfscope}%
\pgfsys@transformshift{4.829166in}{0.849122in}%
\pgfsys@useobject{currentmarker}{}%
\end{pgfscope}%
\begin{pgfscope}%
\pgfsys@transformshift{4.829130in}{0.849007in}%
\pgfsys@useobject{currentmarker}{}%
\end{pgfscope}%
\begin{pgfscope}%
\pgfsys@transformshift{4.829103in}{0.848945in}%
\pgfsys@useobject{currentmarker}{}%
\end{pgfscope}%
\begin{pgfscope}%
\pgfsys@transformshift{4.829081in}{0.848917in}%
\pgfsys@useobject{currentmarker}{}%
\end{pgfscope}%
\begin{pgfscope}%
\pgfsys@transformshift{4.825507in}{0.845525in}%
\pgfsys@useobject{currentmarker}{}%
\end{pgfscope}%
\begin{pgfscope}%
\pgfsys@transformshift{4.822957in}{0.844608in}%
\pgfsys@useobject{currentmarker}{}%
\end{pgfscope}%
\begin{pgfscope}%
\pgfsys@transformshift{4.819392in}{0.843913in}%
\pgfsys@useobject{currentmarker}{}%
\end{pgfscope}%
\begin{pgfscope}%
\pgfsys@transformshift{4.813675in}{0.843006in}%
\pgfsys@useobject{currentmarker}{}%
\end{pgfscope}%
\begin{pgfscope}%
\pgfsys@transformshift{4.810507in}{0.842694in}%
\pgfsys@useobject{currentmarker}{}%
\end{pgfscope}%
\begin{pgfscope}%
\pgfsys@transformshift{4.806439in}{0.842369in}%
\pgfsys@useobject{currentmarker}{}%
\end{pgfscope}%
\begin{pgfscope}%
\pgfsys@transformshift{4.804195in}{0.842374in}%
\pgfsys@useobject{currentmarker}{}%
\end{pgfscope}%
\begin{pgfscope}%
\pgfsys@transformshift{4.802969in}{0.842517in}%
\pgfsys@useobject{currentmarker}{}%
\end{pgfscope}%
\begin{pgfscope}%
\pgfsys@transformshift{4.802291in}{0.842561in}%
\pgfsys@useobject{currentmarker}{}%
\end{pgfscope}%
\begin{pgfscope}%
\pgfsys@transformshift{4.801918in}{0.842577in}%
\pgfsys@useobject{currentmarker}{}%
\end{pgfscope}%
\begin{pgfscope}%
\pgfsys@transformshift{4.801713in}{0.842582in}%
\pgfsys@useobject{currentmarker}{}%
\end{pgfscope}%
\begin{pgfscope}%
\pgfsys@transformshift{4.801600in}{0.842580in}%
\pgfsys@useobject{currentmarker}{}%
\end{pgfscope}%
\begin{pgfscope}%
\pgfsys@transformshift{4.797108in}{0.842592in}%
\pgfsys@useobject{currentmarker}{}%
\end{pgfscope}%
\begin{pgfscope}%
\pgfsys@transformshift{4.788419in}{0.842090in}%
\pgfsys@useobject{currentmarker}{}%
\end{pgfscope}%
\begin{pgfscope}%
\pgfsys@transformshift{4.777177in}{0.840873in}%
\pgfsys@useobject{currentmarker}{}%
\end{pgfscope}%
\begin{pgfscope}%
\pgfsys@transformshift{4.764469in}{0.839219in}%
\pgfsys@useobject{currentmarker}{}%
\end{pgfscope}%
\begin{pgfscope}%
\pgfsys@transformshift{4.750395in}{0.837300in}%
\pgfsys@useobject{currentmarker}{}%
\end{pgfscope}%
\begin{pgfscope}%
\pgfsys@transformshift{4.732159in}{0.833500in}%
\pgfsys@useobject{currentmarker}{}%
\end{pgfscope}%
\begin{pgfscope}%
\pgfsys@transformshift{4.708201in}{0.828975in}%
\pgfsys@useobject{currentmarker}{}%
\end{pgfscope}%
\begin{pgfscope}%
\pgfsys@transformshift{4.681832in}{0.827264in}%
\pgfsys@useobject{currentmarker}{}%
\end{pgfscope}%
\begin{pgfscope}%
\pgfsys@transformshift{4.667840in}{0.823336in}%
\pgfsys@useobject{currentmarker}{}%
\end{pgfscope}%
\begin{pgfscope}%
\pgfsys@transformshift{4.649313in}{0.820435in}%
\pgfsys@useobject{currentmarker}{}%
\end{pgfscope}%
\begin{pgfscope}%
\pgfsys@transformshift{4.624549in}{0.816201in}%
\pgfsys@useobject{currentmarker}{}%
\end{pgfscope}%
\begin{pgfscope}%
\pgfsys@transformshift{4.598853in}{0.809580in}%
\pgfsys@useobject{currentmarker}{}%
\end{pgfscope}%
\begin{pgfscope}%
\pgfsys@transformshift{4.571441in}{0.805209in}%
\pgfsys@useobject{currentmarker}{}%
\end{pgfscope}%
\begin{pgfscope}%
\pgfsys@transformshift{4.538668in}{0.798130in}%
\pgfsys@useobject{currentmarker}{}%
\end{pgfscope}%
\begin{pgfscope}%
\pgfsys@transformshift{4.501411in}{0.792120in}%
\pgfsys@useobject{currentmarker}{}%
\end{pgfscope}%
\begin{pgfscope}%
\pgfsys@transformshift{4.480696in}{0.790801in}%
\pgfsys@useobject{currentmarker}{}%
\end{pgfscope}%
\begin{pgfscope}%
\pgfsys@transformshift{4.469439in}{0.788903in}%
\pgfsys@useobject{currentmarker}{}%
\end{pgfscope}%
\begin{pgfscope}%
\pgfsys@transformshift{4.455407in}{0.788307in}%
\pgfsys@useobject{currentmarker}{}%
\end{pgfscope}%
\begin{pgfscope}%
\pgfsys@transformshift{4.435440in}{0.788075in}%
\pgfsys@useobject{currentmarker}{}%
\end{pgfscope}%
\begin{pgfscope}%
\pgfsys@transformshift{4.412284in}{0.787496in}%
\pgfsys@useobject{currentmarker}{}%
\end{pgfscope}%
\begin{pgfscope}%
\pgfsys@transformshift{4.399544in}{0.787469in}%
\pgfsys@useobject{currentmarker}{}%
\end{pgfscope}%
\begin{pgfscope}%
\pgfsys@transformshift{4.384902in}{0.787041in}%
\pgfsys@useobject{currentmarker}{}%
\end{pgfscope}%
\begin{pgfscope}%
\pgfsys@transformshift{4.367038in}{0.787270in}%
\pgfsys@useobject{currentmarker}{}%
\end{pgfscope}%
\begin{pgfscope}%
\pgfsys@transformshift{4.347534in}{0.785587in}%
\pgfsys@useobject{currentmarker}{}%
\end{pgfscope}%
\begin{pgfscope}%
\pgfsys@transformshift{4.326032in}{0.784734in}%
\pgfsys@useobject{currentmarker}{}%
\end{pgfscope}%
\begin{pgfscope}%
\pgfsys@transformshift{4.314204in}{0.785127in}%
\pgfsys@useobject{currentmarker}{}%
\end{pgfscope}%
\begin{pgfscope}%
\pgfsys@transformshift{4.297300in}{0.784706in}%
\pgfsys@useobject{currentmarker}{}%
\end{pgfscope}%
\begin{pgfscope}%
\pgfsys@transformshift{4.276628in}{0.785525in}%
\pgfsys@useobject{currentmarker}{}%
\end{pgfscope}%
\begin{pgfscope}%
\pgfsys@transformshift{4.254737in}{0.783047in}%
\pgfsys@useobject{currentmarker}{}%
\end{pgfscope}%
\begin{pgfscope}%
\pgfsys@transformshift{4.231818in}{0.782453in}%
\pgfsys@useobject{currentmarker}{}%
\end{pgfscope}%
\begin{pgfscope}%
\pgfsys@transformshift{4.205958in}{0.782251in}%
\pgfsys@useobject{currentmarker}{}%
\end{pgfscope}%
\begin{pgfscope}%
\pgfsys@transformshift{4.174519in}{0.779897in}%
\pgfsys@useobject{currentmarker}{}%
\end{pgfscope}%
\begin{pgfscope}%
\pgfsys@transformshift{4.140661in}{0.777449in}%
\pgfsys@useobject{currentmarker}{}%
\end{pgfscope}%
\begin{pgfscope}%
\pgfsys@transformshift{4.121997in}{0.777936in}%
\pgfsys@useobject{currentmarker}{}%
\end{pgfscope}%
\begin{pgfscope}%
\pgfsys@transformshift{4.111736in}{0.777541in}%
\pgfsys@useobject{currentmarker}{}%
\end{pgfscope}%
\begin{pgfscope}%
\pgfsys@transformshift{4.098286in}{0.777669in}%
\pgfsys@useobject{currentmarker}{}%
\end{pgfscope}%
\begin{pgfscope}%
\pgfsys@transformshift{4.083255in}{0.776336in}%
\pgfsys@useobject{currentmarker}{}%
\end{pgfscope}%
\begin{pgfscope}%
\pgfsys@transformshift{4.067277in}{0.775140in}%
\pgfsys@useobject{currentmarker}{}%
\end{pgfscope}%
\begin{pgfscope}%
\pgfsys@transformshift{4.049031in}{0.775296in}%
\pgfsys@useobject{currentmarker}{}%
\end{pgfscope}%
\begin{pgfscope}%
\pgfsys@transformshift{4.025673in}{0.774666in}%
\pgfsys@useobject{currentmarker}{}%
\end{pgfscope}%
\begin{pgfscope}%
\pgfsys@transformshift{3.998058in}{0.772106in}%
\pgfsys@useobject{currentmarker}{}%
\end{pgfscope}%
\begin{pgfscope}%
\pgfsys@transformshift{3.982806in}{0.771969in}%
\pgfsys@useobject{currentmarker}{}%
\end{pgfscope}%
\begin{pgfscope}%
\pgfsys@transformshift{3.966752in}{0.768929in}%
\pgfsys@useobject{currentmarker}{}%
\end{pgfscope}%
\begin{pgfscope}%
\pgfsys@transformshift{3.946582in}{0.767895in}%
\pgfsys@useobject{currentmarker}{}%
\end{pgfscope}%
\begin{pgfscope}%
\pgfsys@transformshift{3.922085in}{0.769271in}%
\pgfsys@useobject{currentmarker}{}%
\end{pgfscope}%
\begin{pgfscope}%
\pgfsys@transformshift{3.896013in}{0.771249in}%
\pgfsys@useobject{currentmarker}{}%
\end{pgfscope}%
\begin{pgfscope}%
\pgfsys@transformshift{3.867667in}{0.770902in}%
\pgfsys@useobject{currentmarker}{}%
\end{pgfscope}%
\begin{pgfscope}%
\pgfsys@transformshift{3.837159in}{0.771532in}%
\pgfsys@useobject{currentmarker}{}%
\end{pgfscope}%
\begin{pgfscope}%
\pgfsys@transformshift{3.803653in}{0.771509in}%
\pgfsys@useobject{currentmarker}{}%
\end{pgfscope}%
\begin{pgfscope}%
\pgfsys@transformshift{3.769215in}{0.770867in}%
\pgfsys@useobject{currentmarker}{}%
\end{pgfscope}%
\begin{pgfscope}%
\pgfsys@transformshift{3.733376in}{0.766225in}%
\pgfsys@useobject{currentmarker}{}%
\end{pgfscope}%
\begin{pgfscope}%
\pgfsys@transformshift{3.689757in}{0.765179in}%
\pgfsys@useobject{currentmarker}{}%
\end{pgfscope}%
\begin{pgfscope}%
\pgfsys@transformshift{3.642104in}{0.762837in}%
\pgfsys@useobject{currentmarker}{}%
\end{pgfscope}%
\begin{pgfscope}%
\pgfsys@transformshift{3.616119in}{0.759176in}%
\pgfsys@useobject{currentmarker}{}%
\end{pgfscope}%
\begin{pgfscope}%
\pgfsys@transformshift{3.586539in}{0.758001in}%
\pgfsys@useobject{currentmarker}{}%
\end{pgfscope}%
\begin{pgfscope}%
\pgfsys@transformshift{3.552329in}{0.757714in}%
\pgfsys@useobject{currentmarker}{}%
\end{pgfscope}%
\begin{pgfscope}%
\pgfsys@transformshift{3.513429in}{0.755497in}%
\pgfsys@useobject{currentmarker}{}%
\end{pgfscope}%
\begin{pgfscope}%
\pgfsys@transformshift{3.472627in}{0.754328in}%
\pgfsys@useobject{currentmarker}{}%
\end{pgfscope}%
\begin{pgfscope}%
\pgfsys@transformshift{3.425680in}{0.752537in}%
\pgfsys@useobject{currentmarker}{}%
\end{pgfscope}%
\begin{pgfscope}%
\pgfsys@transformshift{3.375317in}{0.752583in}%
\pgfsys@useobject{currentmarker}{}%
\end{pgfscope}%
\begin{pgfscope}%
\pgfsys@transformshift{3.323432in}{0.747491in}%
\pgfsys@useobject{currentmarker}{}%
\end{pgfscope}%
\begin{pgfscope}%
\pgfsys@transformshift{3.294881in}{0.744834in}%
\pgfsys@useobject{currentmarker}{}%
\end{pgfscope}%
\begin{pgfscope}%
\pgfsys@transformshift{3.263752in}{0.742908in}%
\pgfsys@useobject{currentmarker}{}%
\end{pgfscope}%
\begin{pgfscope}%
\pgfsys@transformshift{3.224584in}{0.742054in}%
\pgfsys@useobject{currentmarker}{}%
\end{pgfscope}%
\begin{pgfscope}%
\pgfsys@transformshift{3.182932in}{0.741444in}%
\pgfsys@useobject{currentmarker}{}%
\end{pgfscope}%
\begin{pgfscope}%
\pgfsys@transformshift{3.160031in}{0.740765in}%
\pgfsys@useobject{currentmarker}{}%
\end{pgfscope}%
\begin{pgfscope}%
\pgfsys@transformshift{3.147459in}{0.739901in}%
\pgfsys@useobject{currentmarker}{}%
\end{pgfscope}%
\begin{pgfscope}%
\pgfsys@transformshift{3.131157in}{0.739059in}%
\pgfsys@useobject{currentmarker}{}%
\end{pgfscope}%
\begin{pgfscope}%
\pgfsys@transformshift{3.110123in}{0.738368in}%
\pgfsys@useobject{currentmarker}{}%
\end{pgfscope}%
\begin{pgfscope}%
\pgfsys@transformshift{3.098678in}{0.736636in}%
\pgfsys@useobject{currentmarker}{}%
\end{pgfscope}%
\begin{pgfscope}%
\pgfsys@transformshift{3.085428in}{0.736506in}%
\pgfsys@useobject{currentmarker}{}%
\end{pgfscope}%
\begin{pgfscope}%
\pgfsys@transformshift{3.070020in}{0.735311in}%
\pgfsys@useobject{currentmarker}{}%
\end{pgfscope}%
\begin{pgfscope}%
\pgfsys@transformshift{3.051966in}{0.736771in}%
\pgfsys@useobject{currentmarker}{}%
\end{pgfscope}%
\begin{pgfscope}%
\pgfsys@transformshift{3.031994in}{0.736174in}%
\pgfsys@useobject{currentmarker}{}%
\end{pgfscope}%
\begin{pgfscope}%
\pgfsys@transformshift{3.021015in}{0.735695in}%
\pgfsys@useobject{currentmarker}{}%
\end{pgfscope}%
\begin{pgfscope}%
\pgfsys@transformshift{3.007962in}{0.734989in}%
\pgfsys@useobject{currentmarker}{}%
\end{pgfscope}%
\begin{pgfscope}%
\pgfsys@transformshift{2.990560in}{0.734597in}%
\pgfsys@useobject{currentmarker}{}%
\end{pgfscope}%
\begin{pgfscope}%
\pgfsys@transformshift{2.970036in}{0.732521in}%
\pgfsys@useobject{currentmarker}{}%
\end{pgfscope}%
\begin{pgfscope}%
\pgfsys@transformshift{2.948585in}{0.730896in}%
\pgfsys@useobject{currentmarker}{}%
\end{pgfscope}%
\begin{pgfscope}%
\pgfsys@transformshift{2.925562in}{0.727601in}%
\pgfsys@useobject{currentmarker}{}%
\end{pgfscope}%
\begin{pgfscope}%
\pgfsys@transformshift{2.897833in}{0.726302in}%
\pgfsys@useobject{currentmarker}{}%
\end{pgfscope}%
\begin{pgfscope}%
\pgfsys@transformshift{2.865552in}{0.726993in}%
\pgfsys@useobject{currentmarker}{}%
\end{pgfscope}%
\begin{pgfscope}%
\pgfsys@transformshift{2.830652in}{0.725397in}%
\pgfsys@useobject{currentmarker}{}%
\end{pgfscope}%
\begin{pgfscope}%
\pgfsys@transformshift{2.794463in}{0.723983in}%
\pgfsys@useobject{currentmarker}{}%
\end{pgfscope}%
\begin{pgfscope}%
\pgfsys@transformshift{2.754216in}{0.725042in}%
\pgfsys@useobject{currentmarker}{}%
\end{pgfscope}%
\begin{pgfscope}%
\pgfsys@transformshift{2.709575in}{0.724100in}%
\pgfsys@useobject{currentmarker}{}%
\end{pgfscope}%
\begin{pgfscope}%
\pgfsys@transformshift{2.662116in}{0.720564in}%
\pgfsys@useobject{currentmarker}{}%
\end{pgfscope}%
\begin{pgfscope}%
\pgfsys@transformshift{2.636243in}{0.716600in}%
\pgfsys@useobject{currentmarker}{}%
\end{pgfscope}%
\begin{pgfscope}%
\pgfsys@transformshift{2.602925in}{0.713699in}%
\pgfsys@useobject{currentmarker}{}%
\end{pgfscope}%
\begin{pgfscope}%
\pgfsys@transformshift{2.562861in}{0.712046in}%
\pgfsys@useobject{currentmarker}{}%
\end{pgfscope}%
\begin{pgfscope}%
\pgfsys@transformshift{2.518096in}{0.711101in}%
\pgfsys@useobject{currentmarker}{}%
\end{pgfscope}%
\begin{pgfscope}%
\pgfsys@transformshift{2.493498in}{0.709930in}%
\pgfsys@useobject{currentmarker}{}%
\end{pgfscope}%
\begin{pgfscope}%
\pgfsys@transformshift{2.467989in}{0.709928in}%
\pgfsys@useobject{currentmarker}{}%
\end{pgfscope}%
\begin{pgfscope}%
\pgfsys@transformshift{2.436954in}{0.708535in}%
\pgfsys@useobject{currentmarker}{}%
\end{pgfscope}%
\begin{pgfscope}%
\pgfsys@transformshift{2.401786in}{0.706877in}%
\pgfsys@useobject{currentmarker}{}%
\end{pgfscope}%
\begin{pgfscope}%
\pgfsys@transformshift{2.365847in}{0.701214in}%
\pgfsys@useobject{currentmarker}{}%
\end{pgfscope}%
\begin{pgfscope}%
\pgfsys@transformshift{2.327403in}{0.700783in}%
\pgfsys@useobject{currentmarker}{}%
\end{pgfscope}%
\begin{pgfscope}%
\pgfsys@transformshift{2.286354in}{0.695869in}%
\pgfsys@useobject{currentmarker}{}%
\end{pgfscope}%
\begin{pgfscope}%
\pgfsys@transformshift{2.238585in}{0.695409in}%
\pgfsys@useobject{currentmarker}{}%
\end{pgfscope}%
\begin{pgfscope}%
\pgfsys@transformshift{2.188618in}{0.698699in}%
\pgfsys@useobject{currentmarker}{}%
\end{pgfscope}%
\begin{pgfscope}%
\pgfsys@transformshift{2.161115in}{0.697229in}%
\pgfsys@useobject{currentmarker}{}%
\end{pgfscope}%
\begin{pgfscope}%
\pgfsys@transformshift{2.131537in}{0.698097in}%
\pgfsys@useobject{currentmarker}{}%
\end{pgfscope}%
\begin{pgfscope}%
\pgfsys@transformshift{2.093386in}{0.697992in}%
\pgfsys@useobject{currentmarker}{}%
\end{pgfscope}%
\begin{pgfscope}%
\pgfsys@transformshift{2.053667in}{0.698950in}%
\pgfsys@useobject{currentmarker}{}%
\end{pgfscope}%
\begin{pgfscope}%
\pgfsys@transformshift{2.012576in}{0.699640in}%
\pgfsys@useobject{currentmarker}{}%
\end{pgfscope}%
\begin{pgfscope}%
\pgfsys@transformshift{1.990046in}{0.697812in}%
\pgfsys@useobject{currentmarker}{}%
\end{pgfscope}%
\begin{pgfscope}%
\pgfsys@transformshift{1.962364in}{0.698017in}%
\pgfsys@useobject{currentmarker}{}%
\end{pgfscope}%
\begin{pgfscope}%
\pgfsys@transformshift{1.929216in}{0.698974in}%
\pgfsys@useobject{currentmarker}{}%
\end{pgfscope}%
\begin{pgfscope}%
\pgfsys@transformshift{1.894447in}{0.699051in}%
\pgfsys@useobject{currentmarker}{}%
\end{pgfscope}%
\begin{pgfscope}%
\pgfsys@transformshift{1.875329in}{0.699527in}%
\pgfsys@useobject{currentmarker}{}%
\end{pgfscope}%
\begin{pgfscope}%
\pgfsys@transformshift{1.854870in}{0.699586in}%
\pgfsys@useobject{currentmarker}{}%
\end{pgfscope}%
\begin{pgfscope}%
\pgfsys@transformshift{1.830421in}{0.699008in}%
\pgfsys@useobject{currentmarker}{}%
\end{pgfscope}%
\begin{pgfscope}%
\pgfsys@transformshift{1.801786in}{0.697227in}%
\pgfsys@useobject{currentmarker}{}%
\end{pgfscope}%
\begin{pgfscope}%
\pgfsys@transformshift{1.786016in}{0.696694in}%
\pgfsys@useobject{currentmarker}{}%
\end{pgfscope}%
\begin{pgfscope}%
\pgfsys@transformshift{1.765378in}{0.698062in}%
\pgfsys@useobject{currentmarker}{}%
\end{pgfscope}%
\begin{pgfscope}%
\pgfsys@transformshift{1.738363in}{0.697636in}%
\pgfsys@useobject{currentmarker}{}%
\end{pgfscope}%
\begin{pgfscope}%
\pgfsys@transformshift{1.707100in}{0.702280in}%
\pgfsys@useobject{currentmarker}{}%
\end{pgfscope}%
\begin{pgfscope}%
\pgfsys@transformshift{1.689727in}{0.702853in}%
\pgfsys@useobject{currentmarker}{}%
\end{pgfscope}%
\begin{pgfscope}%
\pgfsys@transformshift{1.670881in}{0.701071in}%
\pgfsys@useobject{currentmarker}{}%
\end{pgfscope}%
\begin{pgfscope}%
\pgfsys@transformshift{1.646632in}{0.702128in}%
\pgfsys@useobject{currentmarker}{}%
\end{pgfscope}%
\begin{pgfscope}%
\pgfsys@transformshift{1.619713in}{0.699723in}%
\pgfsys@useobject{currentmarker}{}%
\end{pgfscope}%
\begin{pgfscope}%
\pgfsys@transformshift{1.588298in}{0.699417in}%
\pgfsys@useobject{currentmarker}{}%
\end{pgfscope}%
\begin{pgfscope}%
\pgfsys@transformshift{1.571023in}{0.699827in}%
\pgfsys@useobject{currentmarker}{}%
\end{pgfscope}%
\begin{pgfscope}%
\pgfsys@transformshift{1.550753in}{0.698681in}%
\pgfsys@useobject{currentmarker}{}%
\end{pgfscope}%
\begin{pgfscope}%
\pgfsys@transformshift{1.528856in}{0.696608in}%
\pgfsys@useobject{currentmarker}{}%
\end{pgfscope}%
\begin{pgfscope}%
\pgfsys@transformshift{1.502010in}{0.695410in}%
\pgfsys@useobject{currentmarker}{}%
\end{pgfscope}%
\begin{pgfscope}%
\pgfsys@transformshift{1.496379in}{0.694711in}%
\pgfsys@useobject{currentmarker}{}%
\end{pgfscope}%
\begin{pgfscope}%
\pgfsys@transformshift{1.511744in}{0.694026in}%
\pgfsys@useobject{currentmarker}{}%
\end{pgfscope}%
\begin{pgfscope}%
\pgfsys@transformshift{1.520174in}{0.694734in}%
\pgfsys@useobject{currentmarker}{}%
\end{pgfscope}%
\begin{pgfscope}%
\pgfsys@transformshift{1.524786in}{0.695343in}%
\pgfsys@useobject{currentmarker}{}%
\end{pgfscope}%
\begin{pgfscope}%
\pgfsys@transformshift{1.530522in}{0.697403in}%
\pgfsys@useobject{currentmarker}{}%
\end{pgfscope}%
\begin{pgfscope}%
\pgfsys@transformshift{1.533777in}{0.698202in}%
\pgfsys@useobject{currentmarker}{}%
\end{pgfscope}%
\begin{pgfscope}%
\pgfsys@transformshift{1.542044in}{0.695286in}%
\pgfsys@useobject{currentmarker}{}%
\end{pgfscope}%
\begin{pgfscope}%
\pgfsys@transformshift{1.552941in}{0.693082in}%
\pgfsys@useobject{currentmarker}{}%
\end{pgfscope}%
\begin{pgfscope}%
\pgfsys@transformshift{1.566580in}{0.692126in}%
\pgfsys@useobject{currentmarker}{}%
\end{pgfscope}%
\begin{pgfscope}%
\pgfsys@transformshift{1.574031in}{0.691107in}%
\pgfsys@useobject{currentmarker}{}%
\end{pgfscope}%
\begin{pgfscope}%
\pgfsys@transformshift{1.581994in}{0.687979in}%
\pgfsys@useobject{currentmarker}{}%
\end{pgfscope}%
\begin{pgfscope}%
\pgfsys@transformshift{1.586123in}{0.685722in}%
\pgfsys@useobject{currentmarker}{}%
\end{pgfscope}%
\begin{pgfscope}%
\pgfsys@transformshift{1.591104in}{0.683000in}%
\pgfsys@useobject{currentmarker}{}%
\end{pgfscope}%
\end{pgfscope}%
\begin{pgfscope}%
\pgfsetbuttcap%
\pgfsetroundjoin%
\definecolor{currentfill}{rgb}{0.000000,0.000000,0.000000}%
\pgfsetfillcolor{currentfill}%
\pgfsetlinewidth{0.803000pt}%
\definecolor{currentstroke}{rgb}{0.000000,0.000000,0.000000}%
\pgfsetstrokecolor{currentstroke}%
\pgfsetdash{}{0pt}%
\pgfsys@defobject{currentmarker}{\pgfqpoint{0.000000in}{-0.048611in}}{\pgfqpoint{0.000000in}{0.000000in}}{%
\pgfpathmoveto{\pgfqpoint{0.000000in}{0.000000in}}%
\pgfpathlineto{\pgfqpoint{0.000000in}{-0.048611in}}%
\pgfusepath{stroke,fill}%
}%
\begin{pgfscope}%
\pgfsys@transformshift{1.485217in}{0.515000in}%
\pgfsys@useobject{currentmarker}{}%
\end{pgfscope}%
\end{pgfscope}%
\begin{pgfscope}%
\definecolor{textcolor}{rgb}{0.000000,0.000000,0.000000}%
\pgfsetstrokecolor{textcolor}%
\pgfsetfillcolor{textcolor}%
\pgftext[x=1.485217in,y=0.417777in,,top]{\color{textcolor}\rmfamily\fontsize{10.000000}{12.000000}\selectfont \(\displaystyle {0}\)}%
\end{pgfscope}%
\begin{pgfscope}%
\pgfsetbuttcap%
\pgfsetroundjoin%
\definecolor{currentfill}{rgb}{0.000000,0.000000,0.000000}%
\pgfsetfillcolor{currentfill}%
\pgfsetlinewidth{0.803000pt}%
\definecolor{currentstroke}{rgb}{0.000000,0.000000,0.000000}%
\pgfsetstrokecolor{currentstroke}%
\pgfsetdash{}{0pt}%
\pgfsys@defobject{currentmarker}{\pgfqpoint{0.000000in}{-0.048611in}}{\pgfqpoint{0.000000in}{0.000000in}}{%
\pgfpathmoveto{\pgfqpoint{0.000000in}{0.000000in}}%
\pgfpathlineto{\pgfqpoint{0.000000in}{-0.048611in}}%
\pgfusepath{stroke,fill}%
}%
\begin{pgfscope}%
\pgfsys@transformshift{2.333320in}{0.515000in}%
\pgfsys@useobject{currentmarker}{}%
\end{pgfscope}%
\end{pgfscope}%
\begin{pgfscope}%
\definecolor{textcolor}{rgb}{0.000000,0.000000,0.000000}%
\pgfsetstrokecolor{textcolor}%
\pgfsetfillcolor{textcolor}%
\pgftext[x=2.333320in,y=0.417777in,,top]{\color{textcolor}\rmfamily\fontsize{10.000000}{12.000000}\selectfont \(\displaystyle {5}\)}%
\end{pgfscope}%
\begin{pgfscope}%
\pgfsetbuttcap%
\pgfsetroundjoin%
\definecolor{currentfill}{rgb}{0.000000,0.000000,0.000000}%
\pgfsetfillcolor{currentfill}%
\pgfsetlinewidth{0.803000pt}%
\definecolor{currentstroke}{rgb}{0.000000,0.000000,0.000000}%
\pgfsetstrokecolor{currentstroke}%
\pgfsetdash{}{0pt}%
\pgfsys@defobject{currentmarker}{\pgfqpoint{0.000000in}{-0.048611in}}{\pgfqpoint{0.000000in}{0.000000in}}{%
\pgfpathmoveto{\pgfqpoint{0.000000in}{0.000000in}}%
\pgfpathlineto{\pgfqpoint{0.000000in}{-0.048611in}}%
\pgfusepath{stroke,fill}%
}%
\begin{pgfscope}%
\pgfsys@transformshift{3.181422in}{0.515000in}%
\pgfsys@useobject{currentmarker}{}%
\end{pgfscope}%
\end{pgfscope}%
\begin{pgfscope}%
\definecolor{textcolor}{rgb}{0.000000,0.000000,0.000000}%
\pgfsetstrokecolor{textcolor}%
\pgfsetfillcolor{textcolor}%
\pgftext[x=3.181422in,y=0.417777in,,top]{\color{textcolor}\rmfamily\fontsize{10.000000}{12.000000}\selectfont \(\displaystyle {10}\)}%
\end{pgfscope}%
\begin{pgfscope}%
\pgfsetbuttcap%
\pgfsetroundjoin%
\definecolor{currentfill}{rgb}{0.000000,0.000000,0.000000}%
\pgfsetfillcolor{currentfill}%
\pgfsetlinewidth{0.803000pt}%
\definecolor{currentstroke}{rgb}{0.000000,0.000000,0.000000}%
\pgfsetstrokecolor{currentstroke}%
\pgfsetdash{}{0pt}%
\pgfsys@defobject{currentmarker}{\pgfqpoint{0.000000in}{-0.048611in}}{\pgfqpoint{0.000000in}{0.000000in}}{%
\pgfpathmoveto{\pgfqpoint{0.000000in}{0.000000in}}%
\pgfpathlineto{\pgfqpoint{0.000000in}{-0.048611in}}%
\pgfusepath{stroke,fill}%
}%
\begin{pgfscope}%
\pgfsys@transformshift{4.029525in}{0.515000in}%
\pgfsys@useobject{currentmarker}{}%
\end{pgfscope}%
\end{pgfscope}%
\begin{pgfscope}%
\definecolor{textcolor}{rgb}{0.000000,0.000000,0.000000}%
\pgfsetstrokecolor{textcolor}%
\pgfsetfillcolor{textcolor}%
\pgftext[x=4.029525in,y=0.417777in,,top]{\color{textcolor}\rmfamily\fontsize{10.000000}{12.000000}\selectfont \(\displaystyle {15}\)}%
\end{pgfscope}%
\begin{pgfscope}%
\pgfsetbuttcap%
\pgfsetroundjoin%
\definecolor{currentfill}{rgb}{0.000000,0.000000,0.000000}%
\pgfsetfillcolor{currentfill}%
\pgfsetlinewidth{0.803000pt}%
\definecolor{currentstroke}{rgb}{0.000000,0.000000,0.000000}%
\pgfsetstrokecolor{currentstroke}%
\pgfsetdash{}{0pt}%
\pgfsys@defobject{currentmarker}{\pgfqpoint{0.000000in}{-0.048611in}}{\pgfqpoint{0.000000in}{0.000000in}}{%
\pgfpathmoveto{\pgfqpoint{0.000000in}{0.000000in}}%
\pgfpathlineto{\pgfqpoint{0.000000in}{-0.048611in}}%
\pgfusepath{stroke,fill}%
}%
\begin{pgfscope}%
\pgfsys@transformshift{4.877627in}{0.515000in}%
\pgfsys@useobject{currentmarker}{}%
\end{pgfscope}%
\end{pgfscope}%
\begin{pgfscope}%
\definecolor{textcolor}{rgb}{0.000000,0.000000,0.000000}%
\pgfsetstrokecolor{textcolor}%
\pgfsetfillcolor{textcolor}%
\pgftext[x=4.877627in,y=0.417777in,,top]{\color{textcolor}\rmfamily\fontsize{10.000000}{12.000000}\selectfont \(\displaystyle {20}\)}%
\end{pgfscope}%
\begin{pgfscope}%
\definecolor{textcolor}{rgb}{0.000000,0.000000,0.000000}%
\pgfsetstrokecolor{textcolor}%
\pgfsetfillcolor{textcolor}%
\pgftext[x=3.157192in,y=0.238889in,,top]{\color{textcolor}\rmfamily\fontsize{10.000000}{12.000000}\selectfont Position X [\(\displaystyle m\)]}%
\end{pgfscope}%
\begin{pgfscope}%
\pgfsetbuttcap%
\pgfsetroundjoin%
\definecolor{currentfill}{rgb}{0.000000,0.000000,0.000000}%
\pgfsetfillcolor{currentfill}%
\pgfsetlinewidth{0.803000pt}%
\definecolor{currentstroke}{rgb}{0.000000,0.000000,0.000000}%
\pgfsetstrokecolor{currentstroke}%
\pgfsetdash{}{0pt}%
\pgfsys@defobject{currentmarker}{\pgfqpoint{-0.048611in}{0.000000in}}{\pgfqpoint{-0.000000in}{0.000000in}}{%
\pgfpathmoveto{\pgfqpoint{-0.000000in}{0.000000in}}%
\pgfpathlineto{\pgfqpoint{-0.048611in}{0.000000in}}%
\pgfusepath{stroke,fill}%
}%
\begin{pgfscope}%
\pgfsys@transformshift{0.677192in}{0.614475in}%
\pgfsys@useobject{currentmarker}{}%
\end{pgfscope}%
\end{pgfscope}%
\begin{pgfscope}%
\definecolor{textcolor}{rgb}{0.000000,0.000000,0.000000}%
\pgfsetstrokecolor{textcolor}%
\pgfsetfillcolor{textcolor}%
\pgftext[x=0.294444in, y=0.566280in, left, base]{\color{textcolor}\rmfamily\fontsize{10.000000}{12.000000}\selectfont \(\displaystyle {−2.5}\)}%
\end{pgfscope}%
\begin{pgfscope}%
\pgfsetbuttcap%
\pgfsetroundjoin%
\definecolor{currentfill}{rgb}{0.000000,0.000000,0.000000}%
\pgfsetfillcolor{currentfill}%
\pgfsetlinewidth{0.803000pt}%
\definecolor{currentstroke}{rgb}{0.000000,0.000000,0.000000}%
\pgfsetstrokecolor{currentstroke}%
\pgfsetdash{}{0pt}%
\pgfsys@defobject{currentmarker}{\pgfqpoint{-0.048611in}{0.000000in}}{\pgfqpoint{-0.000000in}{0.000000in}}{%
\pgfpathmoveto{\pgfqpoint{-0.000000in}{0.000000in}}%
\pgfpathlineto{\pgfqpoint{-0.048611in}{0.000000in}}%
\pgfusepath{stroke,fill}%
}%
\begin{pgfscope}%
\pgfsys@transformshift{0.677192in}{1.038526in}%
\pgfsys@useobject{currentmarker}{}%
\end{pgfscope}%
\end{pgfscope}%
\begin{pgfscope}%
\definecolor{textcolor}{rgb}{0.000000,0.000000,0.000000}%
\pgfsetstrokecolor{textcolor}%
\pgfsetfillcolor{textcolor}%
\pgftext[x=0.402500in, y=0.990332in, left, base]{\color{textcolor}\rmfamily\fontsize{10.000000}{12.000000}\selectfont \(\displaystyle {0.0}\)}%
\end{pgfscope}%
\begin{pgfscope}%
\pgfsetbuttcap%
\pgfsetroundjoin%
\definecolor{currentfill}{rgb}{0.000000,0.000000,0.000000}%
\pgfsetfillcolor{currentfill}%
\pgfsetlinewidth{0.803000pt}%
\definecolor{currentstroke}{rgb}{0.000000,0.000000,0.000000}%
\pgfsetstrokecolor{currentstroke}%
\pgfsetdash{}{0pt}%
\pgfsys@defobject{currentmarker}{\pgfqpoint{-0.048611in}{0.000000in}}{\pgfqpoint{-0.000000in}{0.000000in}}{%
\pgfpathmoveto{\pgfqpoint{-0.000000in}{0.000000in}}%
\pgfpathlineto{\pgfqpoint{-0.048611in}{0.000000in}}%
\pgfusepath{stroke,fill}%
}%
\begin{pgfscope}%
\pgfsys@transformshift{0.677192in}{1.462577in}%
\pgfsys@useobject{currentmarker}{}%
\end{pgfscope}%
\end{pgfscope}%
\begin{pgfscope}%
\definecolor{textcolor}{rgb}{0.000000,0.000000,0.000000}%
\pgfsetstrokecolor{textcolor}%
\pgfsetfillcolor{textcolor}%
\pgftext[x=0.402500in, y=1.414383in, left, base]{\color{textcolor}\rmfamily\fontsize{10.000000}{12.000000}\selectfont \(\displaystyle {2.5}\)}%
\end{pgfscope}%
\begin{pgfscope}%
\pgfsetbuttcap%
\pgfsetroundjoin%
\definecolor{currentfill}{rgb}{0.000000,0.000000,0.000000}%
\pgfsetfillcolor{currentfill}%
\pgfsetlinewidth{0.803000pt}%
\definecolor{currentstroke}{rgb}{0.000000,0.000000,0.000000}%
\pgfsetstrokecolor{currentstroke}%
\pgfsetdash{}{0pt}%
\pgfsys@defobject{currentmarker}{\pgfqpoint{-0.048611in}{0.000000in}}{\pgfqpoint{-0.000000in}{0.000000in}}{%
\pgfpathmoveto{\pgfqpoint{-0.000000in}{0.000000in}}%
\pgfpathlineto{\pgfqpoint{-0.048611in}{0.000000in}}%
\pgfusepath{stroke,fill}%
}%
\begin{pgfscope}%
\pgfsys@transformshift{0.677192in}{1.886629in}%
\pgfsys@useobject{currentmarker}{}%
\end{pgfscope}%
\end{pgfscope}%
\begin{pgfscope}%
\definecolor{textcolor}{rgb}{0.000000,0.000000,0.000000}%
\pgfsetstrokecolor{textcolor}%
\pgfsetfillcolor{textcolor}%
\pgftext[x=0.402500in, y=1.838434in, left, base]{\color{textcolor}\rmfamily\fontsize{10.000000}{12.000000}\selectfont \(\displaystyle {5.0}\)}%
\end{pgfscope}%
\begin{pgfscope}%
\pgfsetbuttcap%
\pgfsetroundjoin%
\definecolor{currentfill}{rgb}{0.000000,0.000000,0.000000}%
\pgfsetfillcolor{currentfill}%
\pgfsetlinewidth{0.803000pt}%
\definecolor{currentstroke}{rgb}{0.000000,0.000000,0.000000}%
\pgfsetstrokecolor{currentstroke}%
\pgfsetdash{}{0pt}%
\pgfsys@defobject{currentmarker}{\pgfqpoint{-0.048611in}{0.000000in}}{\pgfqpoint{-0.000000in}{0.000000in}}{%
\pgfpathmoveto{\pgfqpoint{-0.000000in}{0.000000in}}%
\pgfpathlineto{\pgfqpoint{-0.048611in}{0.000000in}}%
\pgfusepath{stroke,fill}%
}%
\begin{pgfscope}%
\pgfsys@transformshift{0.677192in}{2.310680in}%
\pgfsys@useobject{currentmarker}{}%
\end{pgfscope}%
\end{pgfscope}%
\begin{pgfscope}%
\definecolor{textcolor}{rgb}{0.000000,0.000000,0.000000}%
\pgfsetstrokecolor{textcolor}%
\pgfsetfillcolor{textcolor}%
\pgftext[x=0.402500in, y=2.262485in, left, base]{\color{textcolor}\rmfamily\fontsize{10.000000}{12.000000}\selectfont \(\displaystyle {7.5}\)}%
\end{pgfscope}%
\begin{pgfscope}%
\pgfsetbuttcap%
\pgfsetroundjoin%
\definecolor{currentfill}{rgb}{0.000000,0.000000,0.000000}%
\pgfsetfillcolor{currentfill}%
\pgfsetlinewidth{0.803000pt}%
\definecolor{currentstroke}{rgb}{0.000000,0.000000,0.000000}%
\pgfsetstrokecolor{currentstroke}%
\pgfsetdash{}{0pt}%
\pgfsys@defobject{currentmarker}{\pgfqpoint{-0.048611in}{0.000000in}}{\pgfqpoint{-0.000000in}{0.000000in}}{%
\pgfpathmoveto{\pgfqpoint{-0.000000in}{0.000000in}}%
\pgfpathlineto{\pgfqpoint{-0.048611in}{0.000000in}}%
\pgfusepath{stroke,fill}%
}%
\begin{pgfscope}%
\pgfsys@transformshift{0.677192in}{2.734731in}%
\pgfsys@useobject{currentmarker}{}%
\end{pgfscope}%
\end{pgfscope}%
\begin{pgfscope}%
\definecolor{textcolor}{rgb}{0.000000,0.000000,0.000000}%
\pgfsetstrokecolor{textcolor}%
\pgfsetfillcolor{textcolor}%
\pgftext[x=0.333055in, y=2.686537in, left, base]{\color{textcolor}\rmfamily\fontsize{10.000000}{12.000000}\selectfont \(\displaystyle {10.0}\)}%
\end{pgfscope}%
\begin{pgfscope}%
\pgfsetbuttcap%
\pgfsetroundjoin%
\definecolor{currentfill}{rgb}{0.000000,0.000000,0.000000}%
\pgfsetfillcolor{currentfill}%
\pgfsetlinewidth{0.803000pt}%
\definecolor{currentstroke}{rgb}{0.000000,0.000000,0.000000}%
\pgfsetstrokecolor{currentstroke}%
\pgfsetdash{}{0pt}%
\pgfsys@defobject{currentmarker}{\pgfqpoint{-0.048611in}{0.000000in}}{\pgfqpoint{-0.000000in}{0.000000in}}{%
\pgfpathmoveto{\pgfqpoint{-0.000000in}{0.000000in}}%
\pgfpathlineto{\pgfqpoint{-0.048611in}{0.000000in}}%
\pgfusepath{stroke,fill}%
}%
\begin{pgfscope}%
\pgfsys@transformshift{0.677192in}{3.158782in}%
\pgfsys@useobject{currentmarker}{}%
\end{pgfscope}%
\end{pgfscope}%
\begin{pgfscope}%
\definecolor{textcolor}{rgb}{0.000000,0.000000,0.000000}%
\pgfsetstrokecolor{textcolor}%
\pgfsetfillcolor{textcolor}%
\pgftext[x=0.333055in, y=3.110588in, left, base]{\color{textcolor}\rmfamily\fontsize{10.000000}{12.000000}\selectfont \(\displaystyle {12.5}\)}%
\end{pgfscope}%
\begin{pgfscope}%
\pgfsetbuttcap%
\pgfsetroundjoin%
\definecolor{currentfill}{rgb}{0.000000,0.000000,0.000000}%
\pgfsetfillcolor{currentfill}%
\pgfsetlinewidth{0.803000pt}%
\definecolor{currentstroke}{rgb}{0.000000,0.000000,0.000000}%
\pgfsetstrokecolor{currentstroke}%
\pgfsetdash{}{0pt}%
\pgfsys@defobject{currentmarker}{\pgfqpoint{-0.048611in}{0.000000in}}{\pgfqpoint{-0.000000in}{0.000000in}}{%
\pgfpathmoveto{\pgfqpoint{-0.000000in}{0.000000in}}%
\pgfpathlineto{\pgfqpoint{-0.048611in}{0.000000in}}%
\pgfusepath{stroke,fill}%
}%
\begin{pgfscope}%
\pgfsys@transformshift{0.677192in}{3.582834in}%
\pgfsys@useobject{currentmarker}{}%
\end{pgfscope}%
\end{pgfscope}%
\begin{pgfscope}%
\definecolor{textcolor}{rgb}{0.000000,0.000000,0.000000}%
\pgfsetstrokecolor{textcolor}%
\pgfsetfillcolor{textcolor}%
\pgftext[x=0.333055in, y=3.534639in, left, base]{\color{textcolor}\rmfamily\fontsize{10.000000}{12.000000}\selectfont \(\displaystyle {15.0}\)}%
\end{pgfscope}%
\begin{pgfscope}%
\pgfsetbuttcap%
\pgfsetroundjoin%
\definecolor{currentfill}{rgb}{0.000000,0.000000,0.000000}%
\pgfsetfillcolor{currentfill}%
\pgfsetlinewidth{0.803000pt}%
\definecolor{currentstroke}{rgb}{0.000000,0.000000,0.000000}%
\pgfsetstrokecolor{currentstroke}%
\pgfsetdash{}{0pt}%
\pgfsys@defobject{currentmarker}{\pgfqpoint{-0.048611in}{0.000000in}}{\pgfqpoint{-0.000000in}{0.000000in}}{%
\pgfpathmoveto{\pgfqpoint{-0.000000in}{0.000000in}}%
\pgfpathlineto{\pgfqpoint{-0.048611in}{0.000000in}}%
\pgfusepath{stroke,fill}%
}%
\begin{pgfscope}%
\pgfsys@transformshift{0.677192in}{4.006885in}%
\pgfsys@useobject{currentmarker}{}%
\end{pgfscope}%
\end{pgfscope}%
\begin{pgfscope}%
\definecolor{textcolor}{rgb}{0.000000,0.000000,0.000000}%
\pgfsetstrokecolor{textcolor}%
\pgfsetfillcolor{textcolor}%
\pgftext[x=0.333055in, y=3.958691in, left, base]{\color{textcolor}\rmfamily\fontsize{10.000000}{12.000000}\selectfont \(\displaystyle {17.5}\)}%
\end{pgfscope}%
\begin{pgfscope}%
\definecolor{textcolor}{rgb}{0.000000,0.000000,0.000000}%
\pgfsetstrokecolor{textcolor}%
\pgfsetfillcolor{textcolor}%
\pgftext[x=0.238889in,y=2.363000in,,bottom,rotate=90.000000]{\color{textcolor}\rmfamily\fontsize{10.000000}{12.000000}\selectfont Position Y [\(\displaystyle m\)]}%
\end{pgfscope}%
\begin{pgfscope}%
\pgfpathrectangle{\pgfqpoint{0.677192in}{0.515000in}}{\pgfqpoint{4.960000in}{3.696000in}}%
\pgfusepath{clip}%
\pgfsetrectcap%
\pgfsetroundjoin%
\pgfsetlinewidth{1.505625pt}%
\definecolor{currentstroke}{rgb}{0.121569,0.466667,0.705882}%
\pgfsetstrokecolor{currentstroke}%
\pgfsetdash{}{0pt}%
\pgfpathmoveto{\pgfqpoint{1.485217in}{1.038526in}}%
\pgfpathlineto{\pgfqpoint{1.485217in}{3.752454in}}%
\pgfpathlineto{\pgfqpoint{4.199145in}{3.752454in}}%
\pgfpathlineto{\pgfqpoint{4.199145in}{1.038526in}}%
\pgfpathlineto{\pgfqpoint{1.485217in}{1.038526in}}%
\pgfusepath{stroke}%
\end{pgfscope}%
\begin{pgfscope}%
\pgfsetrectcap%
\pgfsetmiterjoin%
\pgfsetlinewidth{0.803000pt}%
\definecolor{currentstroke}{rgb}{0.000000,0.000000,0.000000}%
\pgfsetstrokecolor{currentstroke}%
\pgfsetdash{}{0pt}%
\pgfpathmoveto{\pgfqpoint{0.677192in}{0.515000in}}%
\pgfpathlineto{\pgfqpoint{0.677192in}{4.211000in}}%
\pgfusepath{stroke}%
\end{pgfscope}%
\begin{pgfscope}%
\pgfsetrectcap%
\pgfsetmiterjoin%
\pgfsetlinewidth{0.803000pt}%
\definecolor{currentstroke}{rgb}{0.000000,0.000000,0.000000}%
\pgfsetstrokecolor{currentstroke}%
\pgfsetdash{}{0pt}%
\pgfpathmoveto{\pgfqpoint{5.637192in}{0.515000in}}%
\pgfpathlineto{\pgfqpoint{5.637192in}{4.211000in}}%
\pgfusepath{stroke}%
\end{pgfscope}%
\begin{pgfscope}%
\pgfsetrectcap%
\pgfsetmiterjoin%
\pgfsetlinewidth{0.803000pt}%
\definecolor{currentstroke}{rgb}{0.000000,0.000000,0.000000}%
\pgfsetstrokecolor{currentstroke}%
\pgfsetdash{}{0pt}%
\pgfpathmoveto{\pgfqpoint{0.677192in}{0.515000in}}%
\pgfpathlineto{\pgfqpoint{5.637192in}{0.515000in}}%
\pgfusepath{stroke}%
\end{pgfscope}%
\begin{pgfscope}%
\pgfsetrectcap%
\pgfsetmiterjoin%
\pgfsetlinewidth{0.803000pt}%
\definecolor{currentstroke}{rgb}{0.000000,0.000000,0.000000}%
\pgfsetstrokecolor{currentstroke}%
\pgfsetdash{}{0pt}%
\pgfpathmoveto{\pgfqpoint{0.677192in}{4.211000in}}%
\pgfpathlineto{\pgfqpoint{5.637192in}{4.211000in}}%
\pgfusepath{stroke}%
\end{pgfscope}%
\begin{pgfscope}%
\pgfsetbuttcap%
\pgfsetmiterjoin%
\definecolor{currentfill}{rgb}{1.000000,1.000000,1.000000}%
\pgfsetfillcolor{currentfill}%
\pgfsetfillopacity{0.800000}%
\pgfsetlinewidth{1.003750pt}%
\definecolor{currentstroke}{rgb}{0.800000,0.800000,0.800000}%
\pgfsetstrokecolor{currentstroke}%
\pgfsetstrokeopacity{0.800000}%
\pgfsetdash{}{0pt}%
\pgfpathmoveto{\pgfqpoint{2.360247in}{2.148555in}}%
\pgfpathlineto{\pgfqpoint{3.954136in}{2.148555in}}%
\pgfpathquadraticcurveto{\pgfqpoint{3.981914in}{2.148555in}}{\pgfqpoint{3.981914in}{2.176333in}}%
\pgfpathlineto{\pgfqpoint{3.981914in}{2.549666in}}%
\pgfpathquadraticcurveto{\pgfqpoint{3.981914in}{2.577444in}}{\pgfqpoint{3.954136in}{2.577444in}}%
\pgfpathlineto{\pgfqpoint{2.360247in}{2.577444in}}%
\pgfpathquadraticcurveto{\pgfqpoint{2.332469in}{2.577444in}}{\pgfqpoint{2.332469in}{2.549666in}}%
\pgfpathlineto{\pgfqpoint{2.332469in}{2.176333in}}%
\pgfpathquadraticcurveto{\pgfqpoint{2.332469in}{2.148555in}}{\pgfqpoint{2.360247in}{2.148555in}}%
\pgfpathclose%
\pgfusepath{stroke,fill}%
\end{pgfscope}%
\begin{pgfscope}%
\pgfsetrectcap%
\pgfsetroundjoin%
\pgfsetlinewidth{1.505625pt}%
\definecolor{currentstroke}{rgb}{0.121569,0.466667,0.705882}%
\pgfsetstrokecolor{currentstroke}%
\pgfsetdash{}{0pt}%
\pgfpathmoveto{\pgfqpoint{2.388025in}{2.473277in}}%
\pgfpathlineto{\pgfqpoint{2.665803in}{2.473277in}}%
\pgfusepath{stroke}%
\end{pgfscope}%
\begin{pgfscope}%
\definecolor{textcolor}{rgb}{0.000000,0.000000,0.000000}%
\pgfsetstrokecolor{textcolor}%
\pgfsetfillcolor{textcolor}%
\pgftext[x=2.776914in,y=2.424666in,left,base]{\color{textcolor}\rmfamily\fontsize{10.000000}{12.000000}\selectfont Ground truth}%
\end{pgfscope}%
\begin{pgfscope}%
\pgfsetbuttcap%
\pgfsetroundjoin%
\definecolor{currentfill}{rgb}{0.121569,0.466667,0.705882}%
\pgfsetfillcolor{currentfill}%
\pgfsetlinewidth{1.003750pt}%
\definecolor{currentstroke}{rgb}{0.121569,0.466667,0.705882}%
\pgfsetstrokecolor{currentstroke}%
\pgfsetdash{}{0pt}%
\pgfsys@defobject{currentmarker}{\pgfqpoint{-0.041667in}{-0.041667in}}{\pgfqpoint{0.041667in}{0.041667in}}{%
\pgfpathmoveto{\pgfqpoint{0.000000in}{-0.041667in}}%
\pgfpathcurveto{\pgfqpoint{0.011050in}{-0.041667in}}{\pgfqpoint{0.021649in}{-0.037276in}}{\pgfqpoint{0.029463in}{-0.029463in}}%
\pgfpathcurveto{\pgfqpoint{0.037276in}{-0.021649in}}{\pgfqpoint{0.041667in}{-0.011050in}}{\pgfqpoint{0.041667in}{0.000000in}}%
\pgfpathcurveto{\pgfqpoint{0.041667in}{0.011050in}}{\pgfqpoint{0.037276in}{0.021649in}}{\pgfqpoint{0.029463in}{0.029463in}}%
\pgfpathcurveto{\pgfqpoint{0.021649in}{0.037276in}}{\pgfqpoint{0.011050in}{0.041667in}}{\pgfqpoint{0.000000in}{0.041667in}}%
\pgfpathcurveto{\pgfqpoint{-0.011050in}{0.041667in}}{\pgfqpoint{-0.021649in}{0.037276in}}{\pgfqpoint{-0.029463in}{0.029463in}}%
\pgfpathcurveto{\pgfqpoint{-0.037276in}{0.021649in}}{\pgfqpoint{-0.041667in}{0.011050in}}{\pgfqpoint{-0.041667in}{0.000000in}}%
\pgfpathcurveto{\pgfqpoint{-0.041667in}{-0.011050in}}{\pgfqpoint{-0.037276in}{-0.021649in}}{\pgfqpoint{-0.029463in}{-0.029463in}}%
\pgfpathcurveto{\pgfqpoint{-0.021649in}{-0.037276in}}{\pgfqpoint{-0.011050in}{-0.041667in}}{\pgfqpoint{0.000000in}{-0.041667in}}%
\pgfpathclose%
\pgfusepath{stroke,fill}%
}%
\begin{pgfscope}%
\pgfsys@transformshift{2.526914in}{2.267513in}%
\pgfsys@useobject{currentmarker}{}%
\end{pgfscope}%
\end{pgfscope}%
\begin{pgfscope}%
\definecolor{textcolor}{rgb}{0.000000,0.000000,0.000000}%
\pgfsetstrokecolor{textcolor}%
\pgfsetfillcolor{textcolor}%
\pgftext[x=2.776914in,y=2.231055in,left,base]{\color{textcolor}\rmfamily\fontsize{10.000000}{12.000000}\selectfont Estimated position}%
\end{pgfscope}%
\end{pgfpicture}%
\makeatother%
\endgroup%
}
%         \caption{Madgwick's 3D position estimation had the lowest displacement error for the 16-meter side square experiment.}
%         \label{fig:square162D}
%     \end{subfigure}
%     \begin{subfigure}{0.49\textwidth}
%         \centering
%         \resizebox{1\linewidth}{!}{%% Creator: Matplotlib, PGF backend
%%
%% To include the figure in your LaTeX document, write
%%   \input{<filename>.pgf}
%%
%% Make sure the required packages are loaded in your preamble
%%   \usepackage{pgf}
%%
%% and, on pdftex
%%   \usepackage[utf8]{inputenc}\DeclareUnicodeCharacter{2212}{-}
%%
%% or, on luatex and xetex
%%   \usepackage{unicode-math}
%%
%% Figures using additional raster images can only be included by \input if
%% they are in the same directory as the main LaTeX file. For loading figures
%% from other directories you can use the `import` package
%%   \usepackage{import}
%%
%% and then include the figures with
%%   \import{<path to file>}{<filename>.pgf}
%%
%% Matplotlib used the following preamble
%%   \usepackage{fontspec}
%%   \setmainfont{DejaVuSerif.ttf}[Path=C:/Users/Claudio/AppData/Local/Programs/Python/Python39/Lib/site-packages/matplotlib/mpl-data/fonts/ttf/]
%%   \setsansfont{DejaVuSans.ttf}[Path=C:/Users/Claudio/AppData/Local/Programs/Python/Python39/Lib/site-packages/matplotlib/mpl-data/fonts/ttf/]
%%   \setmonofont{DejaVuSansMono.ttf}[Path=C:/Users/Claudio/AppData/Local/Programs/Python/Python39/Lib/site-packages/matplotlib/mpl-data/fonts/ttf/]
%%
\begingroup%
\makeatletter%
\begin{pgfpicture}%
\pgfpathrectangle{\pgfpointorigin}{\pgfqpoint{4.342069in}{4.226689in}}%
\pgfusepath{use as bounding box, clip}%
\begin{pgfscope}%
\pgfsetbuttcap%
\pgfsetmiterjoin%
\definecolor{currentfill}{rgb}{1.000000,1.000000,1.000000}%
\pgfsetfillcolor{currentfill}%
\pgfsetlinewidth{0.000000pt}%
\definecolor{currentstroke}{rgb}{1.000000,1.000000,1.000000}%
\pgfsetstrokecolor{currentstroke}%
\pgfsetdash{}{0pt}%
\pgfpathmoveto{\pgfqpoint{0.000000in}{0.000000in}}%
\pgfpathlineto{\pgfqpoint{4.342069in}{0.000000in}}%
\pgfpathlineto{\pgfqpoint{4.342069in}{4.226689in}}%
\pgfpathlineto{\pgfqpoint{0.000000in}{4.226689in}}%
\pgfpathclose%
\pgfusepath{fill}%
\end{pgfscope}%
\begin{pgfscope}%
\pgfsetbuttcap%
\pgfsetmiterjoin%
\definecolor{currentfill}{rgb}{1.000000,1.000000,1.000000}%
\pgfsetfillcolor{currentfill}%
\pgfsetlinewidth{0.000000pt}%
\definecolor{currentstroke}{rgb}{0.000000,0.000000,0.000000}%
\pgfsetstrokecolor{currentstroke}%
\pgfsetstrokeopacity{0.000000}%
\pgfsetdash{}{0pt}%
\pgfpathmoveto{\pgfqpoint{0.100000in}{0.220728in}}%
\pgfpathlineto{\pgfqpoint{3.796000in}{0.220728in}}%
\pgfpathlineto{\pgfqpoint{3.796000in}{3.916728in}}%
\pgfpathlineto{\pgfqpoint{0.100000in}{3.916728in}}%
\pgfpathclose%
\pgfusepath{fill}%
\end{pgfscope}%
\begin{pgfscope}%
\pgfsetbuttcap%
\pgfsetmiterjoin%
\definecolor{currentfill}{rgb}{0.950000,0.950000,0.950000}%
\pgfsetfillcolor{currentfill}%
\pgfsetfillopacity{0.500000}%
\pgfsetlinewidth{1.003750pt}%
\definecolor{currentstroke}{rgb}{0.950000,0.950000,0.950000}%
\pgfsetstrokecolor{currentstroke}%
\pgfsetstrokeopacity{0.500000}%
\pgfsetdash{}{0pt}%
\pgfpathmoveto{\pgfqpoint{0.379073in}{1.132043in}}%
\pgfpathlineto{\pgfqpoint{1.599613in}{2.155124in}}%
\pgfpathlineto{\pgfqpoint{1.582647in}{3.630589in}}%
\pgfpathlineto{\pgfqpoint{0.303698in}{2.697271in}}%
\pgfusepath{stroke,fill}%
\end{pgfscope}%
\begin{pgfscope}%
\pgfsetbuttcap%
\pgfsetmiterjoin%
\definecolor{currentfill}{rgb}{0.900000,0.900000,0.900000}%
\pgfsetfillcolor{currentfill}%
\pgfsetfillopacity{0.500000}%
\pgfsetlinewidth{1.003750pt}%
\definecolor{currentstroke}{rgb}{0.900000,0.900000,0.900000}%
\pgfsetstrokecolor{currentstroke}%
\pgfsetstrokeopacity{0.500000}%
\pgfsetdash{}{0pt}%
\pgfpathmoveto{\pgfqpoint{1.599613in}{2.155124in}}%
\pgfpathlineto{\pgfqpoint{3.558144in}{1.585856in}}%
\pgfpathlineto{\pgfqpoint{3.628038in}{3.112142in}}%
\pgfpathlineto{\pgfqpoint{1.582647in}{3.630589in}}%
\pgfusepath{stroke,fill}%
\end{pgfscope}%
\begin{pgfscope}%
\pgfsetbuttcap%
\pgfsetmiterjoin%
\definecolor{currentfill}{rgb}{0.925000,0.925000,0.925000}%
\pgfsetfillcolor{currentfill}%
\pgfsetfillopacity{0.500000}%
\pgfsetlinewidth{1.003750pt}%
\definecolor{currentstroke}{rgb}{0.925000,0.925000,0.925000}%
\pgfsetstrokecolor{currentstroke}%
\pgfsetstrokeopacity{0.500000}%
\pgfsetdash{}{0pt}%
\pgfpathmoveto{\pgfqpoint{0.379073in}{1.132043in}}%
\pgfpathlineto{\pgfqpoint{2.455212in}{0.453976in}}%
\pgfpathlineto{\pgfqpoint{3.558144in}{1.585856in}}%
\pgfpathlineto{\pgfqpoint{1.599613in}{2.155124in}}%
\pgfusepath{stroke,fill}%
\end{pgfscope}%
\begin{pgfscope}%
\pgfsetrectcap%
\pgfsetroundjoin%
\pgfsetlinewidth{0.803000pt}%
\definecolor{currentstroke}{rgb}{0.000000,0.000000,0.000000}%
\pgfsetstrokecolor{currentstroke}%
\pgfsetdash{}{0pt}%
\pgfpathmoveto{\pgfqpoint{0.379073in}{1.132043in}}%
\pgfpathlineto{\pgfqpoint{2.455212in}{0.453976in}}%
\pgfusepath{stroke}%
\end{pgfscope}%
\begin{pgfscope}%
\definecolor{textcolor}{rgb}{0.000000,0.000000,0.000000}%
\pgfsetstrokecolor{textcolor}%
\pgfsetfillcolor{textcolor}%
\pgftext[x=0.697927in, y=0.423808in, left, base,rotate=341.912962]{\color{textcolor}\sffamily\fontsize{10.000000}{12.000000}\selectfont Position X [\(\displaystyle m\)]}%
\end{pgfscope}%
\begin{pgfscope}%
\pgfsetbuttcap%
\pgfsetroundjoin%
\pgfsetlinewidth{0.803000pt}%
\definecolor{currentstroke}{rgb}{0.690196,0.690196,0.690196}%
\pgfsetstrokecolor{currentstroke}%
\pgfsetdash{}{0pt}%
\pgfpathmoveto{\pgfqpoint{0.697629in}{1.028003in}}%
\pgfpathlineto{\pgfqpoint{1.901248in}{2.067451in}}%
\pgfpathlineto{\pgfqpoint{1.897097in}{3.550885in}}%
\pgfusepath{stroke}%
\end{pgfscope}%
\begin{pgfscope}%
\pgfsetbuttcap%
\pgfsetroundjoin%
\pgfsetlinewidth{0.803000pt}%
\definecolor{currentstroke}{rgb}{0.690196,0.690196,0.690196}%
\pgfsetstrokecolor{currentstroke}%
\pgfsetdash{}{0pt}%
\pgfpathmoveto{\pgfqpoint{1.125744in}{0.888181in}}%
\pgfpathlineto{\pgfqpoint{2.305979in}{1.949811in}}%
\pgfpathlineto{\pgfqpoint{2.319344in}{3.443858in}}%
\pgfusepath{stroke}%
\end{pgfscope}%
\begin{pgfscope}%
\pgfsetbuttcap%
\pgfsetroundjoin%
\pgfsetlinewidth{0.803000pt}%
\definecolor{currentstroke}{rgb}{0.690196,0.690196,0.690196}%
\pgfsetstrokecolor{currentstroke}%
\pgfsetdash{}{0pt}%
\pgfpathmoveto{\pgfqpoint{1.563447in}{0.745227in}}%
\pgfpathlineto{\pgfqpoint{2.719014in}{1.829758in}}%
\pgfpathlineto{\pgfqpoint{2.750634in}{3.334538in}}%
\pgfusepath{stroke}%
\end{pgfscope}%
\begin{pgfscope}%
\pgfsetbuttcap%
\pgfsetroundjoin%
\pgfsetlinewidth{0.803000pt}%
\definecolor{currentstroke}{rgb}{0.690196,0.690196,0.690196}%
\pgfsetstrokecolor{currentstroke}%
\pgfsetdash{}{0pt}%
\pgfpathmoveto{\pgfqpoint{2.011065in}{0.599035in}}%
\pgfpathlineto{\pgfqpoint{3.140610in}{1.707217in}}%
\pgfpathlineto{\pgfqpoint{3.191260in}{3.222853in}}%
\pgfusepath{stroke}%
\end{pgfscope}%
\begin{pgfscope}%
\pgfsetrectcap%
\pgfsetroundjoin%
\pgfsetlinewidth{0.803000pt}%
\definecolor{currentstroke}{rgb}{0.000000,0.000000,0.000000}%
\pgfsetstrokecolor{currentstroke}%
\pgfsetdash{}{0pt}%
\pgfpathmoveto{\pgfqpoint{0.708114in}{1.037058in}}%
\pgfpathlineto{\pgfqpoint{0.676613in}{1.009854in}}%
\pgfusepath{stroke}%
\end{pgfscope}%
\begin{pgfscope}%
\definecolor{textcolor}{rgb}{0.000000,0.000000,0.000000}%
\pgfsetstrokecolor{textcolor}%
\pgfsetfillcolor{textcolor}%
\pgftext[x=0.593253in,y=0.808465in,,top]{\color{textcolor}\sffamily\fontsize{10.000000}{12.000000}\selectfont 0}%
\end{pgfscope}%
\begin{pgfscope}%
\pgfsetrectcap%
\pgfsetroundjoin%
\pgfsetlinewidth{0.803000pt}%
\definecolor{currentstroke}{rgb}{0.000000,0.000000,0.000000}%
\pgfsetstrokecolor{currentstroke}%
\pgfsetdash{}{0pt}%
\pgfpathmoveto{\pgfqpoint{1.136034in}{0.897437in}}%
\pgfpathlineto{\pgfqpoint{1.105117in}{0.869627in}}%
\pgfusepath{stroke}%
\end{pgfscope}%
\begin{pgfscope}%
\definecolor{textcolor}{rgb}{0.000000,0.000000,0.000000}%
\pgfsetstrokecolor{textcolor}%
\pgfsetfillcolor{textcolor}%
\pgftext[x=1.021825in,y=0.665675in,,top]{\color{textcolor}\sffamily\fontsize{10.000000}{12.000000}\selectfont 5}%
\end{pgfscope}%
\begin{pgfscope}%
\pgfsetrectcap%
\pgfsetroundjoin%
\pgfsetlinewidth{0.803000pt}%
\definecolor{currentstroke}{rgb}{0.000000,0.000000,0.000000}%
\pgfsetstrokecolor{currentstroke}%
\pgfsetdash{}{0pt}%
\pgfpathmoveto{\pgfqpoint{1.573532in}{0.754692in}}%
\pgfpathlineto{\pgfqpoint{1.543232in}{0.726255in}}%
\pgfusepath{stroke}%
\end{pgfscope}%
\begin{pgfscope}%
\definecolor{textcolor}{rgb}{0.000000,0.000000,0.000000}%
\pgfsetstrokecolor{textcolor}%
\pgfsetfillcolor{textcolor}%
\pgftext[x=1.460032in,y=0.519673in,,top]{\color{textcolor}\sffamily\fontsize{10.000000}{12.000000}\selectfont 10}%
\end{pgfscope}%
\begin{pgfscope}%
\pgfsetrectcap%
\pgfsetroundjoin%
\pgfsetlinewidth{0.803000pt}%
\definecolor{currentstroke}{rgb}{0.000000,0.000000,0.000000}%
\pgfsetstrokecolor{currentstroke}%
\pgfsetdash{}{0pt}%
\pgfpathmoveto{\pgfqpoint{2.020933in}{0.608715in}}%
\pgfpathlineto{\pgfqpoint{1.991287in}{0.579630in}}%
\pgfusepath{stroke}%
\end{pgfscope}%
\begin{pgfscope}%
\definecolor{textcolor}{rgb}{0.000000,0.000000,0.000000}%
\pgfsetstrokecolor{textcolor}%
\pgfsetfillcolor{textcolor}%
\pgftext[x=1.908203in,y=0.370352in,,top]{\color{textcolor}\sffamily\fontsize{10.000000}{12.000000}\selectfont 15}%
\end{pgfscope}%
\begin{pgfscope}%
\pgfsetrectcap%
\pgfsetroundjoin%
\pgfsetlinewidth{0.803000pt}%
\definecolor{currentstroke}{rgb}{0.000000,0.000000,0.000000}%
\pgfsetstrokecolor{currentstroke}%
\pgfsetdash{}{0pt}%
\pgfpathmoveto{\pgfqpoint{3.558144in}{1.585856in}}%
\pgfpathlineto{\pgfqpoint{2.455212in}{0.453976in}}%
\pgfusepath{stroke}%
\end{pgfscope}%
\begin{pgfscope}%
\definecolor{textcolor}{rgb}{0.000000,0.000000,0.000000}%
\pgfsetstrokecolor{textcolor}%
\pgfsetfillcolor{textcolor}%
\pgftext[x=3.103916in, y=0.291339in, left, base,rotate=45.742112]{\color{textcolor}\sffamily\fontsize{10.000000}{12.000000}\selectfont Position Y [\(\displaystyle m\)]}%
\end{pgfscope}%
\begin{pgfscope}%
\pgfsetbuttcap%
\pgfsetroundjoin%
\pgfsetlinewidth{0.803000pt}%
\definecolor{currentstroke}{rgb}{0.690196,0.690196,0.690196}%
\pgfsetstrokecolor{currentstroke}%
\pgfsetdash{}{0pt}%
\pgfpathmoveto{\pgfqpoint{0.464545in}{2.814650in}}%
\pgfpathlineto{\pgfqpoint{0.532066in}{1.260285in}}%
\pgfpathlineto{\pgfqpoint{2.593998in}{0.596405in}}%
\pgfusepath{stroke}%
\end{pgfscope}%
\begin{pgfscope}%
\pgfsetbuttcap%
\pgfsetroundjoin%
\pgfsetlinewidth{0.803000pt}%
\definecolor{currentstroke}{rgb}{0.690196,0.690196,0.690196}%
\pgfsetstrokecolor{currentstroke}%
\pgfsetdash{}{0pt}%
\pgfpathmoveto{\pgfqpoint{0.775141in}{3.041309in}}%
\pgfpathlineto{\pgfqpoint{0.827908in}{1.508266in}}%
\pgfpathlineto{\pgfqpoint{2.861931in}{0.871370in}}%
\pgfusepath{stroke}%
\end{pgfscope}%
\begin{pgfscope}%
\pgfsetbuttcap%
\pgfsetroundjoin%
\pgfsetlinewidth{0.803000pt}%
\definecolor{currentstroke}{rgb}{0.690196,0.690196,0.690196}%
\pgfsetstrokecolor{currentstroke}%
\pgfsetdash{}{0pt}%
\pgfpathmoveto{\pgfqpoint{1.073013in}{3.258682in}}%
\pgfpathlineto{\pgfqpoint{1.112143in}{1.746517in}}%
\pgfpathlineto{\pgfqpoint{3.118813in}{1.134994in}}%
\pgfusepath{stroke}%
\end{pgfscope}%
\begin{pgfscope}%
\pgfsetbuttcap%
\pgfsetroundjoin%
\pgfsetlinewidth{0.803000pt}%
\definecolor{currentstroke}{rgb}{0.690196,0.690196,0.690196}%
\pgfsetstrokecolor{currentstroke}%
\pgfsetdash{}{0pt}%
\pgfpathmoveto{\pgfqpoint{1.358927in}{3.467328in}}%
\pgfpathlineto{\pgfqpoint{1.385441in}{1.975600in}}%
\pgfpathlineto{\pgfqpoint{3.365312in}{1.387963in}}%
\pgfusepath{stroke}%
\end{pgfscope}%
\begin{pgfscope}%
\pgfsetrectcap%
\pgfsetroundjoin%
\pgfsetlinewidth{0.803000pt}%
\definecolor{currentstroke}{rgb}{0.000000,0.000000,0.000000}%
\pgfsetstrokecolor{currentstroke}%
\pgfsetdash{}{0pt}%
\pgfpathmoveto{\pgfqpoint{2.576626in}{0.601998in}}%
\pgfpathlineto{\pgfqpoint{2.628786in}{0.585204in}}%
\pgfusepath{stroke}%
\end{pgfscope}%
\begin{pgfscope}%
\definecolor{textcolor}{rgb}{0.000000,0.000000,0.000000}%
\pgfsetstrokecolor{textcolor}%
\pgfsetfillcolor{textcolor}%
\pgftext[x=2.772058in,y=0.410920in,,top]{\color{textcolor}\sffamily\fontsize{10.000000}{12.000000}\selectfont 0}%
\end{pgfscope}%
\begin{pgfscope}%
\pgfsetrectcap%
\pgfsetroundjoin%
\pgfsetlinewidth{0.803000pt}%
\definecolor{currentstroke}{rgb}{0.000000,0.000000,0.000000}%
\pgfsetstrokecolor{currentstroke}%
\pgfsetdash{}{0pt}%
\pgfpathmoveto{\pgfqpoint{2.844813in}{0.876731in}}%
\pgfpathlineto{\pgfqpoint{2.896211in}{0.860637in}}%
\pgfusepath{stroke}%
\end{pgfscope}%
\begin{pgfscope}%
\definecolor{textcolor}{rgb}{0.000000,0.000000,0.000000}%
\pgfsetstrokecolor{textcolor}%
\pgfsetfillcolor{textcolor}%
\pgftext[x=3.036396in,y=0.689954in,,top]{\color{textcolor}\sffamily\fontsize{10.000000}{12.000000}\selectfont 5}%
\end{pgfscope}%
\begin{pgfscope}%
\pgfsetrectcap%
\pgfsetroundjoin%
\pgfsetlinewidth{0.803000pt}%
\definecolor{currentstroke}{rgb}{0.000000,0.000000,0.000000}%
\pgfsetstrokecolor{currentstroke}%
\pgfsetdash{}{0pt}%
\pgfpathmoveto{\pgfqpoint{3.101942in}{1.140135in}}%
\pgfpathlineto{\pgfqpoint{3.152596in}{1.124699in}}%
\pgfusepath{stroke}%
\end{pgfscope}%
\begin{pgfscope}%
\definecolor{textcolor}{rgb}{0.000000,0.000000,0.000000}%
\pgfsetstrokecolor{textcolor}%
\pgfsetfillcolor{textcolor}%
\pgftext[x=3.289824in,y=0.957472in,,top]{\color{textcolor}\sffamily\fontsize{10.000000}{12.000000}\selectfont 10}%
\end{pgfscope}%
\begin{pgfscope}%
\pgfsetrectcap%
\pgfsetroundjoin%
\pgfsetlinewidth{0.803000pt}%
\definecolor{currentstroke}{rgb}{0.000000,0.000000,0.000000}%
\pgfsetstrokecolor{currentstroke}%
\pgfsetdash{}{0pt}%
\pgfpathmoveto{\pgfqpoint{3.348683in}{1.392899in}}%
\pgfpathlineto{\pgfqpoint{3.398611in}{1.378080in}}%
\pgfusepath{stroke}%
\end{pgfscope}%
\begin{pgfscope}%
\definecolor{textcolor}{rgb}{0.000000,0.000000,0.000000}%
\pgfsetstrokecolor{textcolor}%
\pgfsetfillcolor{textcolor}%
\pgftext[x=3.533003in,y=1.214172in,,top]{\color{textcolor}\sffamily\fontsize{10.000000}{12.000000}\selectfont 15}%
\end{pgfscope}%
\begin{pgfscope}%
\pgfsetrectcap%
\pgfsetroundjoin%
\pgfsetlinewidth{0.803000pt}%
\definecolor{currentstroke}{rgb}{0.000000,0.000000,0.000000}%
\pgfsetstrokecolor{currentstroke}%
\pgfsetdash{}{0pt}%
\pgfpathmoveto{\pgfqpoint{3.558144in}{1.585856in}}%
\pgfpathlineto{\pgfqpoint{3.628038in}{3.112142in}}%
\pgfusepath{stroke}%
\end{pgfscope}%
\begin{pgfscope}%
\definecolor{textcolor}{rgb}{0.000000,0.000000,0.000000}%
\pgfsetstrokecolor{textcolor}%
\pgfsetfillcolor{textcolor}%
\pgftext[x=4.169544in, y=1.928890in, left, base,rotate=87.378092]{\color{textcolor}\sffamily\fontsize{10.000000}{12.000000}\selectfont Position Z [\(\displaystyle m\)]}%
\end{pgfscope}%
\begin{pgfscope}%
\pgfsetbuttcap%
\pgfsetroundjoin%
\pgfsetlinewidth{0.803000pt}%
\definecolor{currentstroke}{rgb}{0.690196,0.690196,0.690196}%
\pgfsetstrokecolor{currentstroke}%
\pgfsetdash{}{0pt}%
\pgfpathmoveto{\pgfqpoint{3.562758in}{1.686598in}}%
\pgfpathlineto{\pgfqpoint{1.598491in}{2.252704in}}%
\pgfpathlineto{\pgfqpoint{0.374106in}{1.235193in}}%
\pgfusepath{stroke}%
\end{pgfscope}%
\begin{pgfscope}%
\pgfsetbuttcap%
\pgfsetroundjoin%
\pgfsetlinewidth{0.803000pt}%
\definecolor{currentstroke}{rgb}{0.690196,0.690196,0.690196}%
\pgfsetstrokecolor{currentstroke}%
\pgfsetdash{}{0pt}%
\pgfpathmoveto{\pgfqpoint{3.571660in}{1.881000in}}%
\pgfpathlineto{\pgfqpoint{1.596327in}{2.440926in}}%
\pgfpathlineto{\pgfqpoint{0.364518in}{1.434305in}}%
\pgfusepath{stroke}%
\end{pgfscope}%
\begin{pgfscope}%
\pgfsetbuttcap%
\pgfsetroundjoin%
\pgfsetlinewidth{0.803000pt}%
\definecolor{currentstroke}{rgb}{0.690196,0.690196,0.690196}%
\pgfsetstrokecolor{currentstroke}%
\pgfsetdash{}{0pt}%
\pgfpathmoveto{\pgfqpoint{3.580664in}{2.077626in}}%
\pgfpathlineto{\pgfqpoint{1.594139in}{2.631199in}}%
\pgfpathlineto{\pgfqpoint{0.354815in}{1.635783in}}%
\pgfusepath{stroke}%
\end{pgfscope}%
\begin{pgfscope}%
\pgfsetbuttcap%
\pgfsetroundjoin%
\pgfsetlinewidth{0.803000pt}%
\definecolor{currentstroke}{rgb}{0.690196,0.690196,0.690196}%
\pgfsetstrokecolor{currentstroke}%
\pgfsetdash{}{0pt}%
\pgfpathmoveto{\pgfqpoint{3.589772in}{2.276515in}}%
\pgfpathlineto{\pgfqpoint{1.591927in}{2.823557in}}%
\pgfpathlineto{\pgfqpoint{0.344997in}{1.839669in}}%
\pgfusepath{stroke}%
\end{pgfscope}%
\begin{pgfscope}%
\pgfsetbuttcap%
\pgfsetroundjoin%
\pgfsetlinewidth{0.803000pt}%
\definecolor{currentstroke}{rgb}{0.690196,0.690196,0.690196}%
\pgfsetstrokecolor{currentstroke}%
\pgfsetdash{}{0pt}%
\pgfpathmoveto{\pgfqpoint{3.598985in}{2.477706in}}%
\pgfpathlineto{\pgfqpoint{1.589691in}{3.018034in}}%
\pgfpathlineto{\pgfqpoint{0.335060in}{2.046006in}}%
\pgfusepath{stroke}%
\end{pgfscope}%
\begin{pgfscope}%
\pgfsetbuttcap%
\pgfsetroundjoin%
\pgfsetlinewidth{0.803000pt}%
\definecolor{currentstroke}{rgb}{0.690196,0.690196,0.690196}%
\pgfsetstrokecolor{currentstroke}%
\pgfsetdash{}{0pt}%
\pgfpathmoveto{\pgfqpoint{3.608305in}{2.681240in}}%
\pgfpathlineto{\pgfqpoint{1.587430in}{3.214665in}}%
\pgfpathlineto{\pgfqpoint{0.325004in}{2.254839in}}%
\pgfusepath{stroke}%
\end{pgfscope}%
\begin{pgfscope}%
\pgfsetbuttcap%
\pgfsetroundjoin%
\pgfsetlinewidth{0.803000pt}%
\definecolor{currentstroke}{rgb}{0.690196,0.690196,0.690196}%
\pgfsetstrokecolor{currentstroke}%
\pgfsetdash{}{0pt}%
\pgfpathmoveto{\pgfqpoint{3.617735in}{2.887157in}}%
\pgfpathlineto{\pgfqpoint{1.585143in}{3.413486in}}%
\pgfpathlineto{\pgfqpoint{0.314825in}{2.466213in}}%
\pgfusepath{stroke}%
\end{pgfscope}%
\begin{pgfscope}%
\pgfsetrectcap%
\pgfsetroundjoin%
\pgfsetlinewidth{0.803000pt}%
\definecolor{currentstroke}{rgb}{0.000000,0.000000,0.000000}%
\pgfsetstrokecolor{currentstroke}%
\pgfsetdash{}{0pt}%
\pgfpathmoveto{\pgfqpoint{3.546270in}{1.691350in}}%
\pgfpathlineto{\pgfqpoint{3.595773in}{1.677083in}}%
\pgfusepath{stroke}%
\end{pgfscope}%
\begin{pgfscope}%
\definecolor{textcolor}{rgb}{0.000000,0.000000,0.000000}%
\pgfsetstrokecolor{textcolor}%
\pgfsetfillcolor{textcolor}%
\pgftext[x=3.816944in,y=1.722582in,,top]{\color{textcolor}\sffamily\fontsize{10.000000}{12.000000}\selectfont 0.0}%
\end{pgfscope}%
\begin{pgfscope}%
\pgfsetrectcap%
\pgfsetroundjoin%
\pgfsetlinewidth{0.803000pt}%
\definecolor{currentstroke}{rgb}{0.000000,0.000000,0.000000}%
\pgfsetstrokecolor{currentstroke}%
\pgfsetdash{}{0pt}%
\pgfpathmoveto{\pgfqpoint{3.555075in}{1.885701in}}%
\pgfpathlineto{\pgfqpoint{3.604870in}{1.871586in}}%
\pgfusepath{stroke}%
\end{pgfscope}%
\begin{pgfscope}%
\definecolor{textcolor}{rgb}{0.000000,0.000000,0.000000}%
\pgfsetstrokecolor{textcolor}%
\pgfsetfillcolor{textcolor}%
\pgftext[x=3.827261in,y=1.916600in,,top]{\color{textcolor}\sffamily\fontsize{10.000000}{12.000000}\selectfont 0.1}%
\end{pgfscope}%
\begin{pgfscope}%
\pgfsetrectcap%
\pgfsetroundjoin%
\pgfsetlinewidth{0.803000pt}%
\definecolor{currentstroke}{rgb}{0.000000,0.000000,0.000000}%
\pgfsetstrokecolor{currentstroke}%
\pgfsetdash{}{0pt}%
\pgfpathmoveto{\pgfqpoint{3.563980in}{2.082275in}}%
\pgfpathlineto{\pgfqpoint{3.614072in}{2.068316in}}%
\pgfusepath{stroke}%
\end{pgfscope}%
\begin{pgfscope}%
\definecolor{textcolor}{rgb}{0.000000,0.000000,0.000000}%
\pgfsetstrokecolor{textcolor}%
\pgfsetfillcolor{textcolor}%
\pgftext[x=3.837696in,y=2.112831in,,top]{\color{textcolor}\sffamily\fontsize{10.000000}{12.000000}\selectfont 0.2}%
\end{pgfscope}%
\begin{pgfscope}%
\pgfsetrectcap%
\pgfsetroundjoin%
\pgfsetlinewidth{0.803000pt}%
\definecolor{currentstroke}{rgb}{0.000000,0.000000,0.000000}%
\pgfsetstrokecolor{currentstroke}%
\pgfsetdash{}{0pt}%
\pgfpathmoveto{\pgfqpoint{3.572988in}{2.281110in}}%
\pgfpathlineto{\pgfqpoint{3.623379in}{2.267312in}}%
\pgfusepath{stroke}%
\end{pgfscope}%
\begin{pgfscope}%
\definecolor{textcolor}{rgb}{0.000000,0.000000,0.000000}%
\pgfsetstrokecolor{textcolor}%
\pgfsetfillcolor{textcolor}%
\pgftext[x=3.848250in,y=2.311314in,,top]{\color{textcolor}\sffamily\fontsize{10.000000}{12.000000}\selectfont 0.3}%
\end{pgfscope}%
\begin{pgfscope}%
\pgfsetrectcap%
\pgfsetroundjoin%
\pgfsetlinewidth{0.803000pt}%
\definecolor{currentstroke}{rgb}{0.000000,0.000000,0.000000}%
\pgfsetstrokecolor{currentstroke}%
\pgfsetdash{}{0pt}%
\pgfpathmoveto{\pgfqpoint{3.582101in}{2.482246in}}%
\pgfpathlineto{\pgfqpoint{3.632794in}{2.468614in}}%
\pgfusepath{stroke}%
\end{pgfscope}%
\begin{pgfscope}%
\definecolor{textcolor}{rgb}{0.000000,0.000000,0.000000}%
\pgfsetstrokecolor{textcolor}%
\pgfsetfillcolor{textcolor}%
\pgftext[x=3.858927in,y=2.512087in,,top]{\color{textcolor}\sffamily\fontsize{10.000000}{12.000000}\selectfont 0.4}%
\end{pgfscope}%
\begin{pgfscope}%
\pgfsetrectcap%
\pgfsetroundjoin%
\pgfsetlinewidth{0.803000pt}%
\definecolor{currentstroke}{rgb}{0.000000,0.000000,0.000000}%
\pgfsetstrokecolor{currentstroke}%
\pgfsetdash{}{0pt}%
\pgfpathmoveto{\pgfqpoint{3.591319in}{2.685723in}}%
\pgfpathlineto{\pgfqpoint{3.642319in}{2.672261in}}%
\pgfusepath{stroke}%
\end{pgfscope}%
\begin{pgfscope}%
\definecolor{textcolor}{rgb}{0.000000,0.000000,0.000000}%
\pgfsetstrokecolor{textcolor}%
\pgfsetfillcolor{textcolor}%
\pgftext[x=3.869727in,y=2.715190in,,top]{\color{textcolor}\sffamily\fontsize{10.000000}{12.000000}\selectfont 0.5}%
\end{pgfscope}%
\begin{pgfscope}%
\pgfsetrectcap%
\pgfsetroundjoin%
\pgfsetlinewidth{0.803000pt}%
\definecolor{currentstroke}{rgb}{0.000000,0.000000,0.000000}%
\pgfsetstrokecolor{currentstroke}%
\pgfsetdash{}{0pt}%
\pgfpathmoveto{\pgfqpoint{3.600645in}{2.891582in}}%
\pgfpathlineto{\pgfqpoint{3.651956in}{2.878295in}}%
\pgfusepath{stroke}%
\end{pgfscope}%
\begin{pgfscope}%
\definecolor{textcolor}{rgb}{0.000000,0.000000,0.000000}%
\pgfsetstrokecolor{textcolor}%
\pgfsetfillcolor{textcolor}%
\pgftext[x=3.880653in,y=2.920665in,,top]{\color{textcolor}\sffamily\fontsize{10.000000}{12.000000}\selectfont 0.6}%
\end{pgfscope}%
\begin{pgfscope}%
\pgfpathrectangle{\pgfqpoint{0.100000in}{0.220728in}}{\pgfqpoint{3.696000in}{3.696000in}}%
\pgfusepath{clip}%
\pgfsetrectcap%
\pgfsetroundjoin%
\pgfsetlinewidth{1.505625pt}%
\definecolor{currentstroke}{rgb}{1.000000,0.000000,0.000000}%
\pgfsetstrokecolor{currentstroke}%
\pgfsetdash{}{0pt}%
\pgfpathmoveto{\pgfqpoint{0.845071in}{1.261386in}}%
\pgfpathlineto{\pgfqpoint{0.845071in}{1.261386in}}%
\pgfusepath{stroke}%
\end{pgfscope}%
\begin{pgfscope}%
\pgfpathrectangle{\pgfqpoint{0.100000in}{0.220728in}}{\pgfqpoint{3.696000in}{3.696000in}}%
\pgfusepath{clip}%
\pgfsetrectcap%
\pgfsetroundjoin%
\pgfsetlinewidth{1.505625pt}%
\definecolor{currentstroke}{rgb}{1.000000,0.000000,0.000000}%
\pgfsetstrokecolor{currentstroke}%
\pgfsetdash{}{0pt}%
\pgfpathmoveto{\pgfqpoint{1.776593in}{3.293739in}}%
\pgfpathlineto{\pgfqpoint{1.742067in}{2.029460in}}%
\pgfusepath{stroke}%
\end{pgfscope}%
\begin{pgfscope}%
\pgfpathrectangle{\pgfqpoint{0.100000in}{0.220728in}}{\pgfqpoint{3.696000in}{3.696000in}}%
\pgfusepath{clip}%
\pgfsetrectcap%
\pgfsetroundjoin%
\pgfsetlinewidth{1.505625pt}%
\definecolor{currentstroke}{rgb}{1.000000,0.000000,0.000000}%
\pgfsetstrokecolor{currentstroke}%
\pgfsetdash{}{0pt}%
\pgfpathmoveto{\pgfqpoint{3.401463in}{2.969095in}}%
\pgfpathlineto{\pgfqpoint{3.081581in}{1.637273in}}%
\pgfusepath{stroke}%
\end{pgfscope}%
\begin{pgfscope}%
\pgfpathrectangle{\pgfqpoint{0.100000in}{0.220728in}}{\pgfqpoint{3.696000in}{3.696000in}}%
\pgfusepath{clip}%
\pgfsetrectcap%
\pgfsetroundjoin%
\pgfsetlinewidth{1.505625pt}%
\definecolor{currentstroke}{rgb}{1.000000,0.000000,0.000000}%
\pgfsetstrokecolor{currentstroke}%
\pgfsetdash{}{0pt}%
\pgfpathmoveto{\pgfqpoint{2.384544in}{0.749136in}}%
\pgfpathlineto{\pgfqpoint{2.243952in}{0.814661in}}%
\pgfusepath{stroke}%
\end{pgfscope}%
\begin{pgfscope}%
\pgfpathrectangle{\pgfqpoint{0.100000in}{0.220728in}}{\pgfqpoint{3.696000in}{3.696000in}}%
\pgfusepath{clip}%
\pgfsetrectcap%
\pgfsetroundjoin%
\pgfsetlinewidth{1.505625pt}%
\definecolor{currentstroke}{rgb}{1.000000,0.000000,0.000000}%
\pgfsetstrokecolor{currentstroke}%
\pgfsetdash{}{0pt}%
\pgfpathmoveto{\pgfqpoint{0.592749in}{1.489290in}}%
\pgfpathlineto{\pgfqpoint{0.845071in}{1.261386in}}%
\pgfusepath{stroke}%
\end{pgfscope}%
\begin{pgfscope}%
\pgfpathrectangle{\pgfqpoint{0.100000in}{0.220728in}}{\pgfqpoint{3.696000in}{3.696000in}}%
\pgfusepath{clip}%
\pgfsetrectcap%
\pgfsetroundjoin%
\pgfsetlinewidth{1.505625pt}%
\definecolor{currentstroke}{rgb}{0.121569,0.466667,0.705882}%
\pgfsetstrokecolor{currentstroke}%
\pgfsetdash{}{0pt}%
\pgfpathmoveto{\pgfqpoint{0.845071in}{1.261386in}}%
\pgfpathlineto{\pgfqpoint{1.742067in}{2.029460in}}%
\pgfpathlineto{\pgfqpoint{3.081581in}{1.637273in}}%
\pgfpathlineto{\pgfqpoint{2.243952in}{0.814661in}}%
\pgfpathlineto{\pgfqpoint{0.845071in}{1.261386in}}%
\pgfusepath{stroke}%
\end{pgfscope}%
\begin{pgfscope}%
\pgfpathrectangle{\pgfqpoint{0.100000in}{0.220728in}}{\pgfqpoint{3.696000in}{3.696000in}}%
\pgfusepath{clip}%
\pgfsetbuttcap%
\pgfsetroundjoin%
\definecolor{currentfill}{rgb}{1.000000,0.498039,0.054902}%
\pgfsetfillcolor{currentfill}%
\pgfsetfillopacity{0.300000}%
\pgfsetlinewidth{1.003750pt}%
\definecolor{currentstroke}{rgb}{1.000000,0.498039,0.054902}%
\pgfsetstrokecolor{currentstroke}%
\pgfsetstrokeopacity{0.300000}%
\pgfsetdash{}{0pt}%
\pgfpathmoveto{\pgfqpoint{1.776593in}{3.262683in}}%
\pgfpathcurveto{\pgfqpoint{1.784830in}{3.262683in}}{\pgfqpoint{1.792730in}{3.265955in}}{\pgfqpoint{1.798554in}{3.271779in}}%
\pgfpathcurveto{\pgfqpoint{1.804378in}{3.277603in}}{\pgfqpoint{1.807650in}{3.285503in}}{\pgfqpoint{1.807650in}{3.293739in}}%
\pgfpathcurveto{\pgfqpoint{1.807650in}{3.301976in}}{\pgfqpoint{1.804378in}{3.309876in}}{\pgfqpoint{1.798554in}{3.315700in}}%
\pgfpathcurveto{\pgfqpoint{1.792730in}{3.321523in}}{\pgfqpoint{1.784830in}{3.324796in}}{\pgfqpoint{1.776593in}{3.324796in}}%
\pgfpathcurveto{\pgfqpoint{1.768357in}{3.324796in}}{\pgfqpoint{1.760457in}{3.321523in}}{\pgfqpoint{1.754633in}{3.315700in}}%
\pgfpathcurveto{\pgfqpoint{1.748809in}{3.309876in}}{\pgfqpoint{1.745537in}{3.301976in}}{\pgfqpoint{1.745537in}{3.293739in}}%
\pgfpathcurveto{\pgfqpoint{1.745537in}{3.285503in}}{\pgfqpoint{1.748809in}{3.277603in}}{\pgfqpoint{1.754633in}{3.271779in}}%
\pgfpathcurveto{\pgfqpoint{1.760457in}{3.265955in}}{\pgfqpoint{1.768357in}{3.262683in}}{\pgfqpoint{1.776593in}{3.262683in}}%
\pgfpathclose%
\pgfusepath{stroke,fill}%
\end{pgfscope}%
\begin{pgfscope}%
\pgfpathrectangle{\pgfqpoint{0.100000in}{0.220728in}}{\pgfqpoint{3.696000in}{3.696000in}}%
\pgfusepath{clip}%
\pgfsetbuttcap%
\pgfsetroundjoin%
\definecolor{currentfill}{rgb}{1.000000,0.498039,0.054902}%
\pgfsetfillcolor{currentfill}%
\pgfsetfillopacity{0.611573}%
\pgfsetlinewidth{1.003750pt}%
\definecolor{currentstroke}{rgb}{1.000000,0.498039,0.054902}%
\pgfsetstrokecolor{currentstroke}%
\pgfsetstrokeopacity{0.611573}%
\pgfsetdash{}{0pt}%
\pgfpathmoveto{\pgfqpoint{0.845071in}{1.230330in}}%
\pgfpathcurveto{\pgfqpoint{0.853307in}{1.230330in}}{\pgfqpoint{0.861207in}{1.233602in}}{\pgfqpoint{0.867031in}{1.239426in}}%
\pgfpathcurveto{\pgfqpoint{0.872855in}{1.245250in}}{\pgfqpoint{0.876128in}{1.253150in}}{\pgfqpoint{0.876128in}{1.261386in}}%
\pgfpathcurveto{\pgfqpoint{0.876128in}{1.269622in}}{\pgfqpoint{0.872855in}{1.277522in}}{\pgfqpoint{0.867031in}{1.283346in}}%
\pgfpathcurveto{\pgfqpoint{0.861207in}{1.289170in}}{\pgfqpoint{0.853307in}{1.292443in}}{\pgfqpoint{0.845071in}{1.292443in}}%
\pgfpathcurveto{\pgfqpoint{0.836835in}{1.292443in}}{\pgfqpoint{0.828935in}{1.289170in}}{\pgfqpoint{0.823111in}{1.283346in}}%
\pgfpathcurveto{\pgfqpoint{0.817287in}{1.277522in}}{\pgfqpoint{0.814015in}{1.269622in}}{\pgfqpoint{0.814015in}{1.261386in}}%
\pgfpathcurveto{\pgfqpoint{0.814015in}{1.253150in}}{\pgfqpoint{0.817287in}{1.245250in}}{\pgfqpoint{0.823111in}{1.239426in}}%
\pgfpathcurveto{\pgfqpoint{0.828935in}{1.233602in}}{\pgfqpoint{0.836835in}{1.230330in}}{\pgfqpoint{0.845071in}{1.230330in}}%
\pgfpathclose%
\pgfusepath{stroke,fill}%
\end{pgfscope}%
\begin{pgfscope}%
\pgfpathrectangle{\pgfqpoint{0.100000in}{0.220728in}}{\pgfqpoint{3.696000in}{3.696000in}}%
\pgfusepath{clip}%
\pgfsetbuttcap%
\pgfsetroundjoin%
\definecolor{currentfill}{rgb}{1.000000,0.498039,0.054902}%
\pgfsetfillcolor{currentfill}%
\pgfsetfillopacity{0.651420}%
\pgfsetlinewidth{1.003750pt}%
\definecolor{currentstroke}{rgb}{1.000000,0.498039,0.054902}%
\pgfsetstrokecolor{currentstroke}%
\pgfsetstrokeopacity{0.651420}%
\pgfsetdash{}{0pt}%
\pgfpathmoveto{\pgfqpoint{0.592749in}{1.458234in}}%
\pgfpathcurveto{\pgfqpoint{0.600985in}{1.458234in}}{\pgfqpoint{0.608886in}{1.461506in}}{\pgfqpoint{0.614709in}{1.467330in}}%
\pgfpathcurveto{\pgfqpoint{0.620533in}{1.473154in}}{\pgfqpoint{0.623806in}{1.481054in}}{\pgfqpoint{0.623806in}{1.489290in}}%
\pgfpathcurveto{\pgfqpoint{0.623806in}{1.497526in}}{\pgfqpoint{0.620533in}{1.505426in}}{\pgfqpoint{0.614709in}{1.511250in}}%
\pgfpathcurveto{\pgfqpoint{0.608886in}{1.517074in}}{\pgfqpoint{0.600985in}{1.520347in}}{\pgfqpoint{0.592749in}{1.520347in}}%
\pgfpathcurveto{\pgfqpoint{0.584513in}{1.520347in}}{\pgfqpoint{0.576613in}{1.517074in}}{\pgfqpoint{0.570789in}{1.511250in}}%
\pgfpathcurveto{\pgfqpoint{0.564965in}{1.505426in}}{\pgfqpoint{0.561693in}{1.497526in}}{\pgfqpoint{0.561693in}{1.489290in}}%
\pgfpathcurveto{\pgfqpoint{0.561693in}{1.481054in}}{\pgfqpoint{0.564965in}{1.473154in}}{\pgfqpoint{0.570789in}{1.467330in}}%
\pgfpathcurveto{\pgfqpoint{0.576613in}{1.461506in}}{\pgfqpoint{0.584513in}{1.458234in}}{\pgfqpoint{0.592749in}{1.458234in}}%
\pgfpathclose%
\pgfusepath{stroke,fill}%
\end{pgfscope}%
\begin{pgfscope}%
\pgfpathrectangle{\pgfqpoint{0.100000in}{0.220728in}}{\pgfqpoint{3.696000in}{3.696000in}}%
\pgfusepath{clip}%
\pgfsetbuttcap%
\pgfsetroundjoin%
\definecolor{currentfill}{rgb}{1.000000,0.498039,0.054902}%
\pgfsetfillcolor{currentfill}%
\pgfsetfillopacity{0.669975}%
\pgfsetlinewidth{1.003750pt}%
\definecolor{currentstroke}{rgb}{1.000000,0.498039,0.054902}%
\pgfsetstrokecolor{currentstroke}%
\pgfsetstrokeopacity{0.669975}%
\pgfsetdash{}{0pt}%
\pgfpathmoveto{\pgfqpoint{3.401463in}{2.938039in}}%
\pgfpathcurveto{\pgfqpoint{3.409700in}{2.938039in}}{\pgfqpoint{3.417600in}{2.941311in}}{\pgfqpoint{3.423424in}{2.947135in}}%
\pgfpathcurveto{\pgfqpoint{3.429248in}{2.952959in}}{\pgfqpoint{3.432520in}{2.960859in}}{\pgfqpoint{3.432520in}{2.969095in}}%
\pgfpathcurveto{\pgfqpoint{3.432520in}{2.977332in}}{\pgfqpoint{3.429248in}{2.985232in}}{\pgfqpoint{3.423424in}{2.991056in}}%
\pgfpathcurveto{\pgfqpoint{3.417600in}{2.996880in}}{\pgfqpoint{3.409700in}{3.000152in}}{\pgfqpoint{3.401463in}{3.000152in}}%
\pgfpathcurveto{\pgfqpoint{3.393227in}{3.000152in}}{\pgfqpoint{3.385327in}{2.996880in}}{\pgfqpoint{3.379503in}{2.991056in}}%
\pgfpathcurveto{\pgfqpoint{3.373679in}{2.985232in}}{\pgfqpoint{3.370407in}{2.977332in}}{\pgfqpoint{3.370407in}{2.969095in}}%
\pgfpathcurveto{\pgfqpoint{3.370407in}{2.960859in}}{\pgfqpoint{3.373679in}{2.952959in}}{\pgfqpoint{3.379503in}{2.947135in}}%
\pgfpathcurveto{\pgfqpoint{3.385327in}{2.941311in}}{\pgfqpoint{3.393227in}{2.938039in}}{\pgfqpoint{3.401463in}{2.938039in}}%
\pgfpathclose%
\pgfusepath{stroke,fill}%
\end{pgfscope}%
\begin{pgfscope}%
\pgfpathrectangle{\pgfqpoint{0.100000in}{0.220728in}}{\pgfqpoint{3.696000in}{3.696000in}}%
\pgfusepath{clip}%
\pgfsetbuttcap%
\pgfsetroundjoin%
\definecolor{currentfill}{rgb}{1.000000,0.498039,0.054902}%
\pgfsetfillcolor{currentfill}%
\pgfsetlinewidth{1.003750pt}%
\definecolor{currentstroke}{rgb}{1.000000,0.498039,0.054902}%
\pgfsetstrokecolor{currentstroke}%
\pgfsetdash{}{0pt}%
\pgfpathmoveto{\pgfqpoint{2.384544in}{0.718080in}}%
\pgfpathcurveto{\pgfqpoint{2.392781in}{0.718080in}}{\pgfqpoint{2.400681in}{0.721352in}}{\pgfqpoint{2.406505in}{0.727176in}}%
\pgfpathcurveto{\pgfqpoint{2.412329in}{0.733000in}}{\pgfqpoint{2.415601in}{0.740900in}}{\pgfqpoint{2.415601in}{0.749136in}}%
\pgfpathcurveto{\pgfqpoint{2.415601in}{0.757372in}}{\pgfqpoint{2.412329in}{0.765272in}}{\pgfqpoint{2.406505in}{0.771096in}}%
\pgfpathcurveto{\pgfqpoint{2.400681in}{0.776920in}}{\pgfqpoint{2.392781in}{0.780193in}}{\pgfqpoint{2.384544in}{0.780193in}}%
\pgfpathcurveto{\pgfqpoint{2.376308in}{0.780193in}}{\pgfqpoint{2.368408in}{0.776920in}}{\pgfqpoint{2.362584in}{0.771096in}}%
\pgfpathcurveto{\pgfqpoint{2.356760in}{0.765272in}}{\pgfqpoint{2.353488in}{0.757372in}}{\pgfqpoint{2.353488in}{0.749136in}}%
\pgfpathcurveto{\pgfqpoint{2.353488in}{0.740900in}}{\pgfqpoint{2.356760in}{0.733000in}}{\pgfqpoint{2.362584in}{0.727176in}}%
\pgfpathcurveto{\pgfqpoint{2.368408in}{0.721352in}}{\pgfqpoint{2.376308in}{0.718080in}}{\pgfqpoint{2.384544in}{0.718080in}}%
\pgfpathclose%
\pgfusepath{stroke,fill}%
\end{pgfscope}%
\begin{pgfscope}%
\definecolor{textcolor}{rgb}{0.000000,0.000000,0.000000}%
\pgfsetstrokecolor{textcolor}%
\pgfsetfillcolor{textcolor}%
\pgftext[x=1.948000in,y=4.000061in,,base]{\color{textcolor}\sffamily\fontsize{12.000000}{14.400000}\selectfont FLAE}%
\end{pgfscope}%
\begin{pgfscope}%
\pgfpathrectangle{\pgfqpoint{0.100000in}{0.220728in}}{\pgfqpoint{3.696000in}{3.696000in}}%
\pgfusepath{clip}%
\pgfsetbuttcap%
\pgfsetroundjoin%
\definecolor{currentfill}{rgb}{0.121569,0.466667,0.705882}%
\pgfsetfillcolor{currentfill}%
\pgfsetfillopacity{0.300000}%
\pgfsetlinewidth{1.003750pt}%
\definecolor{currentstroke}{rgb}{0.121569,0.466667,0.705882}%
\pgfsetstrokecolor{currentstroke}%
\pgfsetstrokeopacity{0.300000}%
\pgfsetdash{}{0pt}%
\pgfpathmoveto{\pgfqpoint{1.773800in}{3.261306in}}%
\pgfpathcurveto{\pgfqpoint{1.782036in}{3.261306in}}{\pgfqpoint{1.789936in}{3.264579in}}{\pgfqpoint{1.795760in}{3.270402in}}%
\pgfpathcurveto{\pgfqpoint{1.801584in}{3.276226in}}{\pgfqpoint{1.804856in}{3.284126in}}{\pgfqpoint{1.804856in}{3.292363in}}%
\pgfpathcurveto{\pgfqpoint{1.804856in}{3.300599in}}{\pgfqpoint{1.801584in}{3.308499in}}{\pgfqpoint{1.795760in}{3.314323in}}%
\pgfpathcurveto{\pgfqpoint{1.789936in}{3.320147in}}{\pgfqpoint{1.782036in}{3.323419in}}{\pgfqpoint{1.773800in}{3.323419in}}%
\pgfpathcurveto{\pgfqpoint{1.765563in}{3.323419in}}{\pgfqpoint{1.757663in}{3.320147in}}{\pgfqpoint{1.751839in}{3.314323in}}%
\pgfpathcurveto{\pgfqpoint{1.746015in}{3.308499in}}{\pgfqpoint{1.742743in}{3.300599in}}{\pgfqpoint{1.742743in}{3.292363in}}%
\pgfpathcurveto{\pgfqpoint{1.742743in}{3.284126in}}{\pgfqpoint{1.746015in}{3.276226in}}{\pgfqpoint{1.751839in}{3.270402in}}%
\pgfpathcurveto{\pgfqpoint{1.757663in}{3.264579in}}{\pgfqpoint{1.765563in}{3.261306in}}{\pgfqpoint{1.773800in}{3.261306in}}%
\pgfpathclose%
\pgfusepath{stroke,fill}%
\end{pgfscope}%
\begin{pgfscope}%
\pgfpathrectangle{\pgfqpoint{0.100000in}{0.220728in}}{\pgfqpoint{3.696000in}{3.696000in}}%
\pgfusepath{clip}%
\pgfsetbuttcap%
\pgfsetroundjoin%
\definecolor{currentfill}{rgb}{0.121569,0.466667,0.705882}%
\pgfsetfillcolor{currentfill}%
\pgfsetfillopacity{0.300139}%
\pgfsetlinewidth{1.003750pt}%
\definecolor{currentstroke}{rgb}{0.121569,0.466667,0.705882}%
\pgfsetstrokecolor{currentstroke}%
\pgfsetstrokeopacity{0.300139}%
\pgfsetdash{}{0pt}%
\pgfpathmoveto{\pgfqpoint{1.771942in}{3.260125in}}%
\pgfpathcurveto{\pgfqpoint{1.780178in}{3.260125in}}{\pgfqpoint{1.788079in}{3.263397in}}{\pgfqpoint{1.793902in}{3.269221in}}%
\pgfpathcurveto{\pgfqpoint{1.799726in}{3.275045in}}{\pgfqpoint{1.802999in}{3.282945in}}{\pgfqpoint{1.802999in}{3.291181in}}%
\pgfpathcurveto{\pgfqpoint{1.802999in}{3.299418in}}{\pgfqpoint{1.799726in}{3.307318in}}{\pgfqpoint{1.793902in}{3.313142in}}%
\pgfpathcurveto{\pgfqpoint{1.788079in}{3.318966in}}{\pgfqpoint{1.780178in}{3.322238in}}{\pgfqpoint{1.771942in}{3.322238in}}%
\pgfpathcurveto{\pgfqpoint{1.763706in}{3.322238in}}{\pgfqpoint{1.755806in}{3.318966in}}{\pgfqpoint{1.749982in}{3.313142in}}%
\pgfpathcurveto{\pgfqpoint{1.744158in}{3.307318in}}{\pgfqpoint{1.740886in}{3.299418in}}{\pgfqpoint{1.740886in}{3.291181in}}%
\pgfpathcurveto{\pgfqpoint{1.740886in}{3.282945in}}{\pgfqpoint{1.744158in}{3.275045in}}{\pgfqpoint{1.749982in}{3.269221in}}%
\pgfpathcurveto{\pgfqpoint{1.755806in}{3.263397in}}{\pgfqpoint{1.763706in}{3.260125in}}{\pgfqpoint{1.771942in}{3.260125in}}%
\pgfpathclose%
\pgfusepath{stroke,fill}%
\end{pgfscope}%
\begin{pgfscope}%
\pgfpathrectangle{\pgfqpoint{0.100000in}{0.220728in}}{\pgfqpoint{3.696000in}{3.696000in}}%
\pgfusepath{clip}%
\pgfsetbuttcap%
\pgfsetroundjoin%
\definecolor{currentfill}{rgb}{0.121569,0.466667,0.705882}%
\pgfsetfillcolor{currentfill}%
\pgfsetfillopacity{0.300195}%
\pgfsetlinewidth{1.003750pt}%
\definecolor{currentstroke}{rgb}{0.121569,0.466667,0.705882}%
\pgfsetstrokecolor{currentstroke}%
\pgfsetstrokeopacity{0.300195}%
\pgfsetdash{}{0pt}%
\pgfpathmoveto{\pgfqpoint{1.776593in}{3.262683in}}%
\pgfpathcurveto{\pgfqpoint{1.784830in}{3.262683in}}{\pgfqpoint{1.792730in}{3.265955in}}{\pgfqpoint{1.798554in}{3.271779in}}%
\pgfpathcurveto{\pgfqpoint{1.804378in}{3.277603in}}{\pgfqpoint{1.807650in}{3.285503in}}{\pgfqpoint{1.807650in}{3.293739in}}%
\pgfpathcurveto{\pgfqpoint{1.807650in}{3.301976in}}{\pgfqpoint{1.804378in}{3.309876in}}{\pgfqpoint{1.798554in}{3.315700in}}%
\pgfpathcurveto{\pgfqpoint{1.792730in}{3.321523in}}{\pgfqpoint{1.784830in}{3.324796in}}{\pgfqpoint{1.776593in}{3.324796in}}%
\pgfpathcurveto{\pgfqpoint{1.768357in}{3.324796in}}{\pgfqpoint{1.760457in}{3.321523in}}{\pgfqpoint{1.754633in}{3.315700in}}%
\pgfpathcurveto{\pgfqpoint{1.748809in}{3.309876in}}{\pgfqpoint{1.745537in}{3.301976in}}{\pgfqpoint{1.745537in}{3.293739in}}%
\pgfpathcurveto{\pgfqpoint{1.745537in}{3.285503in}}{\pgfqpoint{1.748809in}{3.277603in}}{\pgfqpoint{1.754633in}{3.271779in}}%
\pgfpathcurveto{\pgfqpoint{1.760457in}{3.265955in}}{\pgfqpoint{1.768357in}{3.262683in}}{\pgfqpoint{1.776593in}{3.262683in}}%
\pgfpathclose%
\pgfusepath{stroke,fill}%
\end{pgfscope}%
\begin{pgfscope}%
\pgfpathrectangle{\pgfqpoint{0.100000in}{0.220728in}}{\pgfqpoint{3.696000in}{3.696000in}}%
\pgfusepath{clip}%
\pgfsetbuttcap%
\pgfsetroundjoin%
\definecolor{currentfill}{rgb}{0.121569,0.466667,0.705882}%
\pgfsetfillcolor{currentfill}%
\pgfsetfillopacity{0.300231}%
\pgfsetlinewidth{1.003750pt}%
\definecolor{currentstroke}{rgb}{0.121569,0.466667,0.705882}%
\pgfsetstrokecolor{currentstroke}%
\pgfsetstrokeopacity{0.300231}%
\pgfsetdash{}{0pt}%
\pgfpathmoveto{\pgfqpoint{1.768640in}{3.256927in}}%
\pgfpathcurveto{\pgfqpoint{1.776876in}{3.256927in}}{\pgfqpoint{1.784776in}{3.260200in}}{\pgfqpoint{1.790600in}{3.266024in}}%
\pgfpathcurveto{\pgfqpoint{1.796424in}{3.271848in}}{\pgfqpoint{1.799696in}{3.279748in}}{\pgfqpoint{1.799696in}{3.287984in}}%
\pgfpathcurveto{\pgfqpoint{1.799696in}{3.296220in}}{\pgfqpoint{1.796424in}{3.304120in}}{\pgfqpoint{1.790600in}{3.309944in}}%
\pgfpathcurveto{\pgfqpoint{1.784776in}{3.315768in}}{\pgfqpoint{1.776876in}{3.319040in}}{\pgfqpoint{1.768640in}{3.319040in}}%
\pgfpathcurveto{\pgfqpoint{1.760403in}{3.319040in}}{\pgfqpoint{1.752503in}{3.315768in}}{\pgfqpoint{1.746679in}{3.309944in}}%
\pgfpathcurveto{\pgfqpoint{1.740855in}{3.304120in}}{\pgfqpoint{1.737583in}{3.296220in}}{\pgfqpoint{1.737583in}{3.287984in}}%
\pgfpathcurveto{\pgfqpoint{1.737583in}{3.279748in}}{\pgfqpoint{1.740855in}{3.271848in}}{\pgfqpoint{1.746679in}{3.266024in}}%
\pgfpathcurveto{\pgfqpoint{1.752503in}{3.260200in}}{\pgfqpoint{1.760403in}{3.256927in}}{\pgfqpoint{1.768640in}{3.256927in}}%
\pgfpathclose%
\pgfusepath{stroke,fill}%
\end{pgfscope}%
\begin{pgfscope}%
\pgfpathrectangle{\pgfqpoint{0.100000in}{0.220728in}}{\pgfqpoint{3.696000in}{3.696000in}}%
\pgfusepath{clip}%
\pgfsetbuttcap%
\pgfsetroundjoin%
\definecolor{currentfill}{rgb}{0.121569,0.466667,0.705882}%
\pgfsetfillcolor{currentfill}%
\pgfsetfillopacity{0.300519}%
\pgfsetlinewidth{1.003750pt}%
\definecolor{currentstroke}{rgb}{0.121569,0.466667,0.705882}%
\pgfsetstrokecolor{currentstroke}%
\pgfsetstrokeopacity{0.300519}%
\pgfsetdash{}{0pt}%
\pgfpathmoveto{\pgfqpoint{1.766478in}{3.254266in}}%
\pgfpathcurveto{\pgfqpoint{1.774714in}{3.254266in}}{\pgfqpoint{1.782614in}{3.257538in}}{\pgfqpoint{1.788438in}{3.263362in}}%
\pgfpathcurveto{\pgfqpoint{1.794262in}{3.269186in}}{\pgfqpoint{1.797534in}{3.277086in}}{\pgfqpoint{1.797534in}{3.285322in}}%
\pgfpathcurveto{\pgfqpoint{1.797534in}{3.293559in}}{\pgfqpoint{1.794262in}{3.301459in}}{\pgfqpoint{1.788438in}{3.307283in}}%
\pgfpathcurveto{\pgfqpoint{1.782614in}{3.313107in}}{\pgfqpoint{1.774714in}{3.316379in}}{\pgfqpoint{1.766478in}{3.316379in}}%
\pgfpathcurveto{\pgfqpoint{1.758242in}{3.316379in}}{\pgfqpoint{1.750342in}{3.313107in}}{\pgfqpoint{1.744518in}{3.307283in}}%
\pgfpathcurveto{\pgfqpoint{1.738694in}{3.301459in}}{\pgfqpoint{1.735421in}{3.293559in}}{\pgfqpoint{1.735421in}{3.285322in}}%
\pgfpathcurveto{\pgfqpoint{1.735421in}{3.277086in}}{\pgfqpoint{1.738694in}{3.269186in}}{\pgfqpoint{1.744518in}{3.263362in}}%
\pgfpathcurveto{\pgfqpoint{1.750342in}{3.257538in}}{\pgfqpoint{1.758242in}{3.254266in}}{\pgfqpoint{1.766478in}{3.254266in}}%
\pgfpathclose%
\pgfusepath{stroke,fill}%
\end{pgfscope}%
\begin{pgfscope}%
\pgfpathrectangle{\pgfqpoint{0.100000in}{0.220728in}}{\pgfqpoint{3.696000in}{3.696000in}}%
\pgfusepath{clip}%
\pgfsetbuttcap%
\pgfsetroundjoin%
\definecolor{currentfill}{rgb}{0.121569,0.466667,0.705882}%
\pgfsetfillcolor{currentfill}%
\pgfsetfillopacity{0.300656}%
\pgfsetlinewidth{1.003750pt}%
\definecolor{currentstroke}{rgb}{0.121569,0.466667,0.705882}%
\pgfsetstrokecolor{currentstroke}%
\pgfsetstrokeopacity{0.300656}%
\pgfsetdash{}{0pt}%
\pgfpathmoveto{\pgfqpoint{1.765828in}{3.253008in}}%
\pgfpathcurveto{\pgfqpoint{1.774064in}{3.253008in}}{\pgfqpoint{1.781964in}{3.256281in}}{\pgfqpoint{1.787788in}{3.262104in}}%
\pgfpathcurveto{\pgfqpoint{1.793612in}{3.267928in}}{\pgfqpoint{1.796884in}{3.275828in}}{\pgfqpoint{1.796884in}{3.284065in}}%
\pgfpathcurveto{\pgfqpoint{1.796884in}{3.292301in}}{\pgfqpoint{1.793612in}{3.300201in}}{\pgfqpoint{1.787788in}{3.306025in}}%
\pgfpathcurveto{\pgfqpoint{1.781964in}{3.311849in}}{\pgfqpoint{1.774064in}{3.315121in}}{\pgfqpoint{1.765828in}{3.315121in}}%
\pgfpathcurveto{\pgfqpoint{1.757592in}{3.315121in}}{\pgfqpoint{1.749692in}{3.311849in}}{\pgfqpoint{1.743868in}{3.306025in}}%
\pgfpathcurveto{\pgfqpoint{1.738044in}{3.300201in}}{\pgfqpoint{1.734771in}{3.292301in}}{\pgfqpoint{1.734771in}{3.284065in}}%
\pgfpathcurveto{\pgfqpoint{1.734771in}{3.275828in}}{\pgfqpoint{1.738044in}{3.267928in}}{\pgfqpoint{1.743868in}{3.262104in}}%
\pgfpathcurveto{\pgfqpoint{1.749692in}{3.256281in}}{\pgfqpoint{1.757592in}{3.253008in}}{\pgfqpoint{1.765828in}{3.253008in}}%
\pgfpathclose%
\pgfusepath{stroke,fill}%
\end{pgfscope}%
\begin{pgfscope}%
\pgfpathrectangle{\pgfqpoint{0.100000in}{0.220728in}}{\pgfqpoint{3.696000in}{3.696000in}}%
\pgfusepath{clip}%
\pgfsetbuttcap%
\pgfsetroundjoin%
\definecolor{currentfill}{rgb}{0.121569,0.466667,0.705882}%
\pgfsetfillcolor{currentfill}%
\pgfsetfillopacity{0.300717}%
\pgfsetlinewidth{1.003750pt}%
\definecolor{currentstroke}{rgb}{0.121569,0.466667,0.705882}%
\pgfsetstrokecolor{currentstroke}%
\pgfsetstrokeopacity{0.300717}%
\pgfsetdash{}{0pt}%
\pgfpathmoveto{\pgfqpoint{1.765642in}{3.252571in}}%
\pgfpathcurveto{\pgfqpoint{1.773878in}{3.252571in}}{\pgfqpoint{1.781778in}{3.255843in}}{\pgfqpoint{1.787602in}{3.261667in}}%
\pgfpathcurveto{\pgfqpoint{1.793426in}{3.267491in}}{\pgfqpoint{1.796698in}{3.275391in}}{\pgfqpoint{1.796698in}{3.283627in}}%
\pgfpathcurveto{\pgfqpoint{1.796698in}{3.291864in}}{\pgfqpoint{1.793426in}{3.299764in}}{\pgfqpoint{1.787602in}{3.305588in}}%
\pgfpathcurveto{\pgfqpoint{1.781778in}{3.311412in}}{\pgfqpoint{1.773878in}{3.314684in}}{\pgfqpoint{1.765642in}{3.314684in}}%
\pgfpathcurveto{\pgfqpoint{1.757405in}{3.314684in}}{\pgfqpoint{1.749505in}{3.311412in}}{\pgfqpoint{1.743681in}{3.305588in}}%
\pgfpathcurveto{\pgfqpoint{1.737857in}{3.299764in}}{\pgfqpoint{1.734585in}{3.291864in}}{\pgfqpoint{1.734585in}{3.283627in}}%
\pgfpathcurveto{\pgfqpoint{1.734585in}{3.275391in}}{\pgfqpoint{1.737857in}{3.267491in}}{\pgfqpoint{1.743681in}{3.261667in}}%
\pgfpathcurveto{\pgfqpoint{1.749505in}{3.255843in}}{\pgfqpoint{1.757405in}{3.252571in}}{\pgfqpoint{1.765642in}{3.252571in}}%
\pgfpathclose%
\pgfusepath{stroke,fill}%
\end{pgfscope}%
\begin{pgfscope}%
\pgfpathrectangle{\pgfqpoint{0.100000in}{0.220728in}}{\pgfqpoint{3.696000in}{3.696000in}}%
\pgfusepath{clip}%
\pgfsetbuttcap%
\pgfsetroundjoin%
\definecolor{currentfill}{rgb}{0.121569,0.466667,0.705882}%
\pgfsetfillcolor{currentfill}%
\pgfsetfillopacity{0.300727}%
\pgfsetlinewidth{1.003750pt}%
\definecolor{currentstroke}{rgb}{0.121569,0.466667,0.705882}%
\pgfsetstrokecolor{currentstroke}%
\pgfsetstrokeopacity{0.300727}%
\pgfsetdash{}{0pt}%
\pgfpathmoveto{\pgfqpoint{1.780057in}{3.263735in}}%
\pgfpathcurveto{\pgfqpoint{1.788293in}{3.263735in}}{\pgfqpoint{1.796193in}{3.267008in}}{\pgfqpoint{1.802017in}{3.272832in}}%
\pgfpathcurveto{\pgfqpoint{1.807841in}{3.278655in}}{\pgfqpoint{1.811113in}{3.286556in}}{\pgfqpoint{1.811113in}{3.294792in}}%
\pgfpathcurveto{\pgfqpoint{1.811113in}{3.303028in}}{\pgfqpoint{1.807841in}{3.310928in}}{\pgfqpoint{1.802017in}{3.316752in}}%
\pgfpathcurveto{\pgfqpoint{1.796193in}{3.322576in}}{\pgfqpoint{1.788293in}{3.325848in}}{\pgfqpoint{1.780057in}{3.325848in}}%
\pgfpathcurveto{\pgfqpoint{1.771821in}{3.325848in}}{\pgfqpoint{1.763921in}{3.322576in}}{\pgfqpoint{1.758097in}{3.316752in}}%
\pgfpathcurveto{\pgfqpoint{1.752273in}{3.310928in}}{\pgfqpoint{1.749000in}{3.303028in}}{\pgfqpoint{1.749000in}{3.294792in}}%
\pgfpathcurveto{\pgfqpoint{1.749000in}{3.286556in}}{\pgfqpoint{1.752273in}{3.278655in}}{\pgfqpoint{1.758097in}{3.272832in}}%
\pgfpathcurveto{\pgfqpoint{1.763921in}{3.267008in}}{\pgfqpoint{1.771821in}{3.263735in}}{\pgfqpoint{1.780057in}{3.263735in}}%
\pgfpathclose%
\pgfusepath{stroke,fill}%
\end{pgfscope}%
\begin{pgfscope}%
\pgfpathrectangle{\pgfqpoint{0.100000in}{0.220728in}}{\pgfqpoint{3.696000in}{3.696000in}}%
\pgfusepath{clip}%
\pgfsetbuttcap%
\pgfsetroundjoin%
\definecolor{currentfill}{rgb}{0.121569,0.466667,0.705882}%
\pgfsetfillcolor{currentfill}%
\pgfsetfillopacity{0.300839}%
\pgfsetlinewidth{1.003750pt}%
\definecolor{currentstroke}{rgb}{0.121569,0.466667,0.705882}%
\pgfsetstrokecolor{currentstroke}%
\pgfsetstrokeopacity{0.300839}%
\pgfsetdash{}{0pt}%
\pgfpathmoveto{\pgfqpoint{1.765314in}{3.251811in}}%
\pgfpathcurveto{\pgfqpoint{1.773551in}{3.251811in}}{\pgfqpoint{1.781451in}{3.255083in}}{\pgfqpoint{1.787275in}{3.260907in}}%
\pgfpathcurveto{\pgfqpoint{1.793098in}{3.266731in}}{\pgfqpoint{1.796371in}{3.274631in}}{\pgfqpoint{1.796371in}{3.282867in}}%
\pgfpathcurveto{\pgfqpoint{1.796371in}{3.291103in}}{\pgfqpoint{1.793098in}{3.299004in}}{\pgfqpoint{1.787275in}{3.304827in}}%
\pgfpathcurveto{\pgfqpoint{1.781451in}{3.310651in}}{\pgfqpoint{1.773551in}{3.313924in}}{\pgfqpoint{1.765314in}{3.313924in}}%
\pgfpathcurveto{\pgfqpoint{1.757078in}{3.313924in}}{\pgfqpoint{1.749178in}{3.310651in}}{\pgfqpoint{1.743354in}{3.304827in}}%
\pgfpathcurveto{\pgfqpoint{1.737530in}{3.299004in}}{\pgfqpoint{1.734258in}{3.291103in}}{\pgfqpoint{1.734258in}{3.282867in}}%
\pgfpathcurveto{\pgfqpoint{1.734258in}{3.274631in}}{\pgfqpoint{1.737530in}{3.266731in}}{\pgfqpoint{1.743354in}{3.260907in}}%
\pgfpathcurveto{\pgfqpoint{1.749178in}{3.255083in}}{\pgfqpoint{1.757078in}{3.251811in}}{\pgfqpoint{1.765314in}{3.251811in}}%
\pgfpathclose%
\pgfusepath{stroke,fill}%
\end{pgfscope}%
\begin{pgfscope}%
\pgfpathrectangle{\pgfqpoint{0.100000in}{0.220728in}}{\pgfqpoint{3.696000in}{3.696000in}}%
\pgfusepath{clip}%
\pgfsetbuttcap%
\pgfsetroundjoin%
\definecolor{currentfill}{rgb}{0.121569,0.466667,0.705882}%
\pgfsetfillcolor{currentfill}%
\pgfsetfillopacity{0.301043}%
\pgfsetlinewidth{1.003750pt}%
\definecolor{currentstroke}{rgb}{0.121569,0.466667,0.705882}%
\pgfsetstrokecolor{currentstroke}%
\pgfsetstrokeopacity{0.301043}%
\pgfsetdash{}{0pt}%
\pgfpathmoveto{\pgfqpoint{1.781863in}{3.263805in}}%
\pgfpathcurveto{\pgfqpoint{1.790099in}{3.263805in}}{\pgfqpoint{1.797999in}{3.267077in}}{\pgfqpoint{1.803823in}{3.272901in}}%
\pgfpathcurveto{\pgfqpoint{1.809647in}{3.278725in}}{\pgfqpoint{1.812919in}{3.286625in}}{\pgfqpoint{1.812919in}{3.294861in}}%
\pgfpathcurveto{\pgfqpoint{1.812919in}{3.303098in}}{\pgfqpoint{1.809647in}{3.310998in}}{\pgfqpoint{1.803823in}{3.316822in}}%
\pgfpathcurveto{\pgfqpoint{1.797999in}{3.322645in}}{\pgfqpoint{1.790099in}{3.325918in}}{\pgfqpoint{1.781863in}{3.325918in}}%
\pgfpathcurveto{\pgfqpoint{1.773627in}{3.325918in}}{\pgfqpoint{1.765727in}{3.322645in}}{\pgfqpoint{1.759903in}{3.316822in}}%
\pgfpathcurveto{\pgfqpoint{1.754079in}{3.310998in}}{\pgfqpoint{1.750806in}{3.303098in}}{\pgfqpoint{1.750806in}{3.294861in}}%
\pgfpathcurveto{\pgfqpoint{1.750806in}{3.286625in}}{\pgfqpoint{1.754079in}{3.278725in}}{\pgfqpoint{1.759903in}{3.272901in}}%
\pgfpathcurveto{\pgfqpoint{1.765727in}{3.267077in}}{\pgfqpoint{1.773627in}{3.263805in}}{\pgfqpoint{1.781863in}{3.263805in}}%
\pgfpathclose%
\pgfusepath{stroke,fill}%
\end{pgfscope}%
\begin{pgfscope}%
\pgfpathrectangle{\pgfqpoint{0.100000in}{0.220728in}}{\pgfqpoint{3.696000in}{3.696000in}}%
\pgfusepath{clip}%
\pgfsetbuttcap%
\pgfsetroundjoin%
\definecolor{currentfill}{rgb}{0.121569,0.466667,0.705882}%
\pgfsetfillcolor{currentfill}%
\pgfsetfillopacity{0.301076}%
\pgfsetlinewidth{1.003750pt}%
\definecolor{currentstroke}{rgb}{0.121569,0.466667,0.705882}%
\pgfsetstrokecolor{currentstroke}%
\pgfsetstrokeopacity{0.301076}%
\pgfsetdash{}{0pt}%
\pgfpathmoveto{\pgfqpoint{1.764731in}{3.250488in}}%
\pgfpathcurveto{\pgfqpoint{1.772967in}{3.250488in}}{\pgfqpoint{1.780867in}{3.253761in}}{\pgfqpoint{1.786691in}{3.259585in}}%
\pgfpathcurveto{\pgfqpoint{1.792515in}{3.265409in}}{\pgfqpoint{1.795787in}{3.273309in}}{\pgfqpoint{1.795787in}{3.281545in}}%
\pgfpathcurveto{\pgfqpoint{1.795787in}{3.289781in}}{\pgfqpoint{1.792515in}{3.297681in}}{\pgfqpoint{1.786691in}{3.303505in}}%
\pgfpathcurveto{\pgfqpoint{1.780867in}{3.309329in}}{\pgfqpoint{1.772967in}{3.312601in}}{\pgfqpoint{1.764731in}{3.312601in}}%
\pgfpathcurveto{\pgfqpoint{1.756494in}{3.312601in}}{\pgfqpoint{1.748594in}{3.309329in}}{\pgfqpoint{1.742770in}{3.303505in}}%
\pgfpathcurveto{\pgfqpoint{1.736946in}{3.297681in}}{\pgfqpoint{1.733674in}{3.289781in}}{\pgfqpoint{1.733674in}{3.281545in}}%
\pgfpathcurveto{\pgfqpoint{1.733674in}{3.273309in}}{\pgfqpoint{1.736946in}{3.265409in}}{\pgfqpoint{1.742770in}{3.259585in}}%
\pgfpathcurveto{\pgfqpoint{1.748594in}{3.253761in}}{\pgfqpoint{1.756494in}{3.250488in}}{\pgfqpoint{1.764731in}{3.250488in}}%
\pgfpathclose%
\pgfusepath{stroke,fill}%
\end{pgfscope}%
\begin{pgfscope}%
\pgfpathrectangle{\pgfqpoint{0.100000in}{0.220728in}}{\pgfqpoint{3.696000in}{3.696000in}}%
\pgfusepath{clip}%
\pgfsetbuttcap%
\pgfsetroundjoin%
\definecolor{currentfill}{rgb}{0.121569,0.466667,0.705882}%
\pgfsetfillcolor{currentfill}%
\pgfsetfillopacity{0.301119}%
\pgfsetlinewidth{1.003750pt}%
\definecolor{currentstroke}{rgb}{0.121569,0.466667,0.705882}%
\pgfsetstrokecolor{currentstroke}%
\pgfsetstrokeopacity{0.301119}%
\pgfsetdash{}{0pt}%
\pgfpathmoveto{\pgfqpoint{1.782915in}{3.263759in}}%
\pgfpathcurveto{\pgfqpoint{1.791151in}{3.263759in}}{\pgfqpoint{1.799051in}{3.267031in}}{\pgfqpoint{1.804875in}{3.272855in}}%
\pgfpathcurveto{\pgfqpoint{1.810699in}{3.278679in}}{\pgfqpoint{1.813971in}{3.286579in}}{\pgfqpoint{1.813971in}{3.294815in}}%
\pgfpathcurveto{\pgfqpoint{1.813971in}{3.303052in}}{\pgfqpoint{1.810699in}{3.310952in}}{\pgfqpoint{1.804875in}{3.316776in}}%
\pgfpathcurveto{\pgfqpoint{1.799051in}{3.322600in}}{\pgfqpoint{1.791151in}{3.325872in}}{\pgfqpoint{1.782915in}{3.325872in}}%
\pgfpathcurveto{\pgfqpoint{1.774679in}{3.325872in}}{\pgfqpoint{1.766779in}{3.322600in}}{\pgfqpoint{1.760955in}{3.316776in}}%
\pgfpathcurveto{\pgfqpoint{1.755131in}{3.310952in}}{\pgfqpoint{1.751858in}{3.303052in}}{\pgfqpoint{1.751858in}{3.294815in}}%
\pgfpathcurveto{\pgfqpoint{1.751858in}{3.286579in}}{\pgfqpoint{1.755131in}{3.278679in}}{\pgfqpoint{1.760955in}{3.272855in}}%
\pgfpathcurveto{\pgfqpoint{1.766779in}{3.267031in}}{\pgfqpoint{1.774679in}{3.263759in}}{\pgfqpoint{1.782915in}{3.263759in}}%
\pgfpathclose%
\pgfusepath{stroke,fill}%
\end{pgfscope}%
\begin{pgfscope}%
\pgfpathrectangle{\pgfqpoint{0.100000in}{0.220728in}}{\pgfqpoint{3.696000in}{3.696000in}}%
\pgfusepath{clip}%
\pgfsetbuttcap%
\pgfsetroundjoin%
\definecolor{currentfill}{rgb}{0.121569,0.466667,0.705882}%
\pgfsetfillcolor{currentfill}%
\pgfsetfillopacity{0.301205}%
\pgfsetlinewidth{1.003750pt}%
\definecolor{currentstroke}{rgb}{0.121569,0.466667,0.705882}%
\pgfsetstrokecolor{currentstroke}%
\pgfsetstrokeopacity{0.301205}%
\pgfsetdash{}{0pt}%
\pgfpathmoveto{\pgfqpoint{1.764447in}{3.249746in}}%
\pgfpathcurveto{\pgfqpoint{1.772684in}{3.249746in}}{\pgfqpoint{1.780584in}{3.253018in}}{\pgfqpoint{1.786408in}{3.258842in}}%
\pgfpathcurveto{\pgfqpoint{1.792231in}{3.264666in}}{\pgfqpoint{1.795504in}{3.272566in}}{\pgfqpoint{1.795504in}{3.280803in}}%
\pgfpathcurveto{\pgfqpoint{1.795504in}{3.289039in}}{\pgfqpoint{1.792231in}{3.296939in}}{\pgfqpoint{1.786408in}{3.302763in}}%
\pgfpathcurveto{\pgfqpoint{1.780584in}{3.308587in}}{\pgfqpoint{1.772684in}{3.311859in}}{\pgfqpoint{1.764447in}{3.311859in}}%
\pgfpathcurveto{\pgfqpoint{1.756211in}{3.311859in}}{\pgfqpoint{1.748311in}{3.308587in}}{\pgfqpoint{1.742487in}{3.302763in}}%
\pgfpathcurveto{\pgfqpoint{1.736663in}{3.296939in}}{\pgfqpoint{1.733391in}{3.289039in}}{\pgfqpoint{1.733391in}{3.280803in}}%
\pgfpathcurveto{\pgfqpoint{1.733391in}{3.272566in}}{\pgfqpoint{1.736663in}{3.264666in}}{\pgfqpoint{1.742487in}{3.258842in}}%
\pgfpathcurveto{\pgfqpoint{1.748311in}{3.253018in}}{\pgfqpoint{1.756211in}{3.249746in}}{\pgfqpoint{1.764447in}{3.249746in}}%
\pgfpathclose%
\pgfusepath{stroke,fill}%
\end{pgfscope}%
\begin{pgfscope}%
\pgfpathrectangle{\pgfqpoint{0.100000in}{0.220728in}}{\pgfqpoint{3.696000in}{3.696000in}}%
\pgfusepath{clip}%
\pgfsetbuttcap%
\pgfsetroundjoin%
\definecolor{currentfill}{rgb}{0.121569,0.466667,0.705882}%
\pgfsetfillcolor{currentfill}%
\pgfsetfillopacity{0.301219}%
\pgfsetlinewidth{1.003750pt}%
\definecolor{currentstroke}{rgb}{0.121569,0.466667,0.705882}%
\pgfsetstrokecolor{currentstroke}%
\pgfsetstrokeopacity{0.301219}%
\pgfsetdash{}{0pt}%
\pgfpathmoveto{\pgfqpoint{1.783439in}{3.263702in}}%
\pgfpathcurveto{\pgfqpoint{1.791675in}{3.263702in}}{\pgfqpoint{1.799575in}{3.266974in}}{\pgfqpoint{1.805399in}{3.272798in}}%
\pgfpathcurveto{\pgfqpoint{1.811223in}{3.278622in}}{\pgfqpoint{1.814495in}{3.286522in}}{\pgfqpoint{1.814495in}{3.294758in}}%
\pgfpathcurveto{\pgfqpoint{1.814495in}{3.302994in}}{\pgfqpoint{1.811223in}{3.310894in}}{\pgfqpoint{1.805399in}{3.316718in}}%
\pgfpathcurveto{\pgfqpoint{1.799575in}{3.322542in}}{\pgfqpoint{1.791675in}{3.325815in}}{\pgfqpoint{1.783439in}{3.325815in}}%
\pgfpathcurveto{\pgfqpoint{1.775203in}{3.325815in}}{\pgfqpoint{1.767303in}{3.322542in}}{\pgfqpoint{1.761479in}{3.316718in}}%
\pgfpathcurveto{\pgfqpoint{1.755655in}{3.310894in}}{\pgfqpoint{1.752382in}{3.302994in}}{\pgfqpoint{1.752382in}{3.294758in}}%
\pgfpathcurveto{\pgfqpoint{1.752382in}{3.286522in}}{\pgfqpoint{1.755655in}{3.278622in}}{\pgfqpoint{1.761479in}{3.272798in}}%
\pgfpathcurveto{\pgfqpoint{1.767303in}{3.266974in}}{\pgfqpoint{1.775203in}{3.263702in}}{\pgfqpoint{1.783439in}{3.263702in}}%
\pgfpathclose%
\pgfusepath{stroke,fill}%
\end{pgfscope}%
\begin{pgfscope}%
\pgfpathrectangle{\pgfqpoint{0.100000in}{0.220728in}}{\pgfqpoint{3.696000in}{3.696000in}}%
\pgfusepath{clip}%
\pgfsetbuttcap%
\pgfsetroundjoin%
\definecolor{currentfill}{rgb}{0.121569,0.466667,0.705882}%
\pgfsetfillcolor{currentfill}%
\pgfsetfillopacity{0.301424}%
\pgfsetlinewidth{1.003750pt}%
\definecolor{currentstroke}{rgb}{0.121569,0.466667,0.705882}%
\pgfsetstrokecolor{currentstroke}%
\pgfsetstrokeopacity{0.301424}%
\pgfsetdash{}{0pt}%
\pgfpathmoveto{\pgfqpoint{1.763793in}{3.248463in}}%
\pgfpathcurveto{\pgfqpoint{1.772030in}{3.248463in}}{\pgfqpoint{1.779930in}{3.251735in}}{\pgfqpoint{1.785754in}{3.257559in}}%
\pgfpathcurveto{\pgfqpoint{1.791578in}{3.263383in}}{\pgfqpoint{1.794850in}{3.271283in}}{\pgfqpoint{1.794850in}{3.279519in}}%
\pgfpathcurveto{\pgfqpoint{1.794850in}{3.287756in}}{\pgfqpoint{1.791578in}{3.295656in}}{\pgfqpoint{1.785754in}{3.301480in}}%
\pgfpathcurveto{\pgfqpoint{1.779930in}{3.307303in}}{\pgfqpoint{1.772030in}{3.310576in}}{\pgfqpoint{1.763793in}{3.310576in}}%
\pgfpathcurveto{\pgfqpoint{1.755557in}{3.310576in}}{\pgfqpoint{1.747657in}{3.307303in}}{\pgfqpoint{1.741833in}{3.301480in}}%
\pgfpathcurveto{\pgfqpoint{1.736009in}{3.295656in}}{\pgfqpoint{1.732737in}{3.287756in}}{\pgfqpoint{1.732737in}{3.279519in}}%
\pgfpathcurveto{\pgfqpoint{1.732737in}{3.271283in}}{\pgfqpoint{1.736009in}{3.263383in}}{\pgfqpoint{1.741833in}{3.257559in}}%
\pgfpathcurveto{\pgfqpoint{1.747657in}{3.251735in}}{\pgfqpoint{1.755557in}{3.248463in}}{\pgfqpoint{1.763793in}{3.248463in}}%
\pgfpathclose%
\pgfusepath{stroke,fill}%
\end{pgfscope}%
\begin{pgfscope}%
\pgfpathrectangle{\pgfqpoint{0.100000in}{0.220728in}}{\pgfqpoint{3.696000in}{3.696000in}}%
\pgfusepath{clip}%
\pgfsetbuttcap%
\pgfsetroundjoin%
\definecolor{currentfill}{rgb}{0.121569,0.466667,0.705882}%
\pgfsetfillcolor{currentfill}%
\pgfsetfillopacity{0.301468}%
\pgfsetlinewidth{1.003750pt}%
\definecolor{currentstroke}{rgb}{0.121569,0.466667,0.705882}%
\pgfsetstrokecolor{currentstroke}%
\pgfsetstrokeopacity{0.301468}%
\pgfsetdash{}{0pt}%
\pgfpathmoveto{\pgfqpoint{1.784323in}{3.263562in}}%
\pgfpathcurveto{\pgfqpoint{1.792560in}{3.263562in}}{\pgfqpoint{1.800460in}{3.266834in}}{\pgfqpoint{1.806283in}{3.272658in}}%
\pgfpathcurveto{\pgfqpoint{1.812107in}{3.278482in}}{\pgfqpoint{1.815380in}{3.286382in}}{\pgfqpoint{1.815380in}{3.294618in}}%
\pgfpathcurveto{\pgfqpoint{1.815380in}{3.302854in}}{\pgfqpoint{1.812107in}{3.310754in}}{\pgfqpoint{1.806283in}{3.316578in}}%
\pgfpathcurveto{\pgfqpoint{1.800460in}{3.322402in}}{\pgfqpoint{1.792560in}{3.325675in}}{\pgfqpoint{1.784323in}{3.325675in}}%
\pgfpathcurveto{\pgfqpoint{1.776087in}{3.325675in}}{\pgfqpoint{1.768187in}{3.322402in}}{\pgfqpoint{1.762363in}{3.316578in}}%
\pgfpathcurveto{\pgfqpoint{1.756539in}{3.310754in}}{\pgfqpoint{1.753267in}{3.302854in}}{\pgfqpoint{1.753267in}{3.294618in}}%
\pgfpathcurveto{\pgfqpoint{1.753267in}{3.286382in}}{\pgfqpoint{1.756539in}{3.278482in}}{\pgfqpoint{1.762363in}{3.272658in}}%
\pgfpathcurveto{\pgfqpoint{1.768187in}{3.266834in}}{\pgfqpoint{1.776087in}{3.263562in}}{\pgfqpoint{1.784323in}{3.263562in}}%
\pgfpathclose%
\pgfusepath{stroke,fill}%
\end{pgfscope}%
\begin{pgfscope}%
\pgfpathrectangle{\pgfqpoint{0.100000in}{0.220728in}}{\pgfqpoint{3.696000in}{3.696000in}}%
\pgfusepath{clip}%
\pgfsetbuttcap%
\pgfsetroundjoin%
\definecolor{currentfill}{rgb}{0.121569,0.466667,0.705882}%
\pgfsetfillcolor{currentfill}%
\pgfsetfillopacity{0.301536}%
\pgfsetlinewidth{1.003750pt}%
\definecolor{currentstroke}{rgb}{0.121569,0.466667,0.705882}%
\pgfsetstrokecolor{currentstroke}%
\pgfsetstrokeopacity{0.301536}%
\pgfsetdash{}{0pt}%
\pgfpathmoveto{\pgfqpoint{1.784869in}{3.263383in}}%
\pgfpathcurveto{\pgfqpoint{1.793105in}{3.263383in}}{\pgfqpoint{1.801005in}{3.266655in}}{\pgfqpoint{1.806829in}{3.272479in}}%
\pgfpathcurveto{\pgfqpoint{1.812653in}{3.278303in}}{\pgfqpoint{1.815926in}{3.286203in}}{\pgfqpoint{1.815926in}{3.294440in}}%
\pgfpathcurveto{\pgfqpoint{1.815926in}{3.302676in}}{\pgfqpoint{1.812653in}{3.310576in}}{\pgfqpoint{1.806829in}{3.316400in}}%
\pgfpathcurveto{\pgfqpoint{1.801005in}{3.322224in}}{\pgfqpoint{1.793105in}{3.325496in}}{\pgfqpoint{1.784869in}{3.325496in}}%
\pgfpathcurveto{\pgfqpoint{1.776633in}{3.325496in}}{\pgfqpoint{1.768733in}{3.322224in}}{\pgfqpoint{1.762909in}{3.316400in}}%
\pgfpathcurveto{\pgfqpoint{1.757085in}{3.310576in}}{\pgfqpoint{1.753813in}{3.302676in}}{\pgfqpoint{1.753813in}{3.294440in}}%
\pgfpathcurveto{\pgfqpoint{1.753813in}{3.286203in}}{\pgfqpoint{1.757085in}{3.278303in}}{\pgfqpoint{1.762909in}{3.272479in}}%
\pgfpathcurveto{\pgfqpoint{1.768733in}{3.266655in}}{\pgfqpoint{1.776633in}{3.263383in}}{\pgfqpoint{1.784869in}{3.263383in}}%
\pgfpathclose%
\pgfusepath{stroke,fill}%
\end{pgfscope}%
\begin{pgfscope}%
\pgfpathrectangle{\pgfqpoint{0.100000in}{0.220728in}}{\pgfqpoint{3.696000in}{3.696000in}}%
\pgfusepath{clip}%
\pgfsetbuttcap%
\pgfsetroundjoin%
\definecolor{currentfill}{rgb}{0.121569,0.466667,0.705882}%
\pgfsetfillcolor{currentfill}%
\pgfsetfillopacity{0.301608}%
\pgfsetlinewidth{1.003750pt}%
\definecolor{currentstroke}{rgb}{0.121569,0.466667,0.705882}%
\pgfsetstrokecolor{currentstroke}%
\pgfsetstrokeopacity{0.301608}%
\pgfsetdash{}{0pt}%
\pgfpathmoveto{\pgfqpoint{1.785142in}{3.263343in}}%
\pgfpathcurveto{\pgfqpoint{1.793378in}{3.263343in}}{\pgfqpoint{1.801278in}{3.266616in}}{\pgfqpoint{1.807102in}{3.272440in}}%
\pgfpathcurveto{\pgfqpoint{1.812926in}{3.278264in}}{\pgfqpoint{1.816198in}{3.286164in}}{\pgfqpoint{1.816198in}{3.294400in}}%
\pgfpathcurveto{\pgfqpoint{1.816198in}{3.302636in}}{\pgfqpoint{1.812926in}{3.310536in}}{\pgfqpoint{1.807102in}{3.316360in}}%
\pgfpathcurveto{\pgfqpoint{1.801278in}{3.322184in}}{\pgfqpoint{1.793378in}{3.325456in}}{\pgfqpoint{1.785142in}{3.325456in}}%
\pgfpathcurveto{\pgfqpoint{1.776905in}{3.325456in}}{\pgfqpoint{1.769005in}{3.322184in}}{\pgfqpoint{1.763181in}{3.316360in}}%
\pgfpathcurveto{\pgfqpoint{1.757357in}{3.310536in}}{\pgfqpoint{1.754085in}{3.302636in}}{\pgfqpoint{1.754085in}{3.294400in}}%
\pgfpathcurveto{\pgfqpoint{1.754085in}{3.286164in}}{\pgfqpoint{1.757357in}{3.278264in}}{\pgfqpoint{1.763181in}{3.272440in}}%
\pgfpathcurveto{\pgfqpoint{1.769005in}{3.266616in}}{\pgfqpoint{1.776905in}{3.263343in}}{\pgfqpoint{1.785142in}{3.263343in}}%
\pgfpathclose%
\pgfusepath{stroke,fill}%
\end{pgfscope}%
\begin{pgfscope}%
\pgfpathrectangle{\pgfqpoint{0.100000in}{0.220728in}}{\pgfqpoint{3.696000in}{3.696000in}}%
\pgfusepath{clip}%
\pgfsetbuttcap%
\pgfsetroundjoin%
\definecolor{currentfill}{rgb}{0.121569,0.466667,0.705882}%
\pgfsetfillcolor{currentfill}%
\pgfsetfillopacity{0.301636}%
\pgfsetlinewidth{1.003750pt}%
\definecolor{currentstroke}{rgb}{0.121569,0.466667,0.705882}%
\pgfsetstrokecolor{currentstroke}%
\pgfsetstrokeopacity{0.301636}%
\pgfsetdash{}{0pt}%
\pgfpathmoveto{\pgfqpoint{1.785306in}{3.263317in}}%
\pgfpathcurveto{\pgfqpoint{1.793542in}{3.263317in}}{\pgfqpoint{1.801442in}{3.266590in}}{\pgfqpoint{1.807266in}{3.272414in}}%
\pgfpathcurveto{\pgfqpoint{1.813090in}{3.278237in}}{\pgfqpoint{1.816362in}{3.286138in}}{\pgfqpoint{1.816362in}{3.294374in}}%
\pgfpathcurveto{\pgfqpoint{1.816362in}{3.302610in}}{\pgfqpoint{1.813090in}{3.310510in}}{\pgfqpoint{1.807266in}{3.316334in}}%
\pgfpathcurveto{\pgfqpoint{1.801442in}{3.322158in}}{\pgfqpoint{1.793542in}{3.325430in}}{\pgfqpoint{1.785306in}{3.325430in}}%
\pgfpathcurveto{\pgfqpoint{1.777069in}{3.325430in}}{\pgfqpoint{1.769169in}{3.322158in}}{\pgfqpoint{1.763345in}{3.316334in}}%
\pgfpathcurveto{\pgfqpoint{1.757521in}{3.310510in}}{\pgfqpoint{1.754249in}{3.302610in}}{\pgfqpoint{1.754249in}{3.294374in}}%
\pgfpathcurveto{\pgfqpoint{1.754249in}{3.286138in}}{\pgfqpoint{1.757521in}{3.278237in}}{\pgfqpoint{1.763345in}{3.272414in}}%
\pgfpathcurveto{\pgfqpoint{1.769169in}{3.266590in}}{\pgfqpoint{1.777069in}{3.263317in}}{\pgfqpoint{1.785306in}{3.263317in}}%
\pgfpathclose%
\pgfusepath{stroke,fill}%
\end{pgfscope}%
\begin{pgfscope}%
\pgfpathrectangle{\pgfqpoint{0.100000in}{0.220728in}}{\pgfqpoint{3.696000in}{3.696000in}}%
\pgfusepath{clip}%
\pgfsetbuttcap%
\pgfsetroundjoin%
\definecolor{currentfill}{rgb}{0.121569,0.466667,0.705882}%
\pgfsetfillcolor{currentfill}%
\pgfsetfillopacity{0.301781}%
\pgfsetlinewidth{1.003750pt}%
\definecolor{currentstroke}{rgb}{0.121569,0.466667,0.705882}%
\pgfsetstrokecolor{currentstroke}%
\pgfsetstrokeopacity{0.301781}%
\pgfsetdash{}{0pt}%
\pgfpathmoveto{\pgfqpoint{1.785945in}{3.263214in}}%
\pgfpathcurveto{\pgfqpoint{1.794182in}{3.263214in}}{\pgfqpoint{1.802082in}{3.266486in}}{\pgfqpoint{1.807906in}{3.272310in}}%
\pgfpathcurveto{\pgfqpoint{1.813730in}{3.278134in}}{\pgfqpoint{1.817002in}{3.286034in}}{\pgfqpoint{1.817002in}{3.294270in}}%
\pgfpathcurveto{\pgfqpoint{1.817002in}{3.302507in}}{\pgfqpoint{1.813730in}{3.310407in}}{\pgfqpoint{1.807906in}{3.316231in}}%
\pgfpathcurveto{\pgfqpoint{1.802082in}{3.322055in}}{\pgfqpoint{1.794182in}{3.325327in}}{\pgfqpoint{1.785945in}{3.325327in}}%
\pgfpathcurveto{\pgfqpoint{1.777709in}{3.325327in}}{\pgfqpoint{1.769809in}{3.322055in}}{\pgfqpoint{1.763985in}{3.316231in}}%
\pgfpathcurveto{\pgfqpoint{1.758161in}{3.310407in}}{\pgfqpoint{1.754889in}{3.302507in}}{\pgfqpoint{1.754889in}{3.294270in}}%
\pgfpathcurveto{\pgfqpoint{1.754889in}{3.286034in}}{\pgfqpoint{1.758161in}{3.278134in}}{\pgfqpoint{1.763985in}{3.272310in}}%
\pgfpathcurveto{\pgfqpoint{1.769809in}{3.266486in}}{\pgfqpoint{1.777709in}{3.263214in}}{\pgfqpoint{1.785945in}{3.263214in}}%
\pgfpathclose%
\pgfusepath{stroke,fill}%
\end{pgfscope}%
\begin{pgfscope}%
\pgfpathrectangle{\pgfqpoint{0.100000in}{0.220728in}}{\pgfqpoint{3.696000in}{3.696000in}}%
\pgfusepath{clip}%
\pgfsetbuttcap%
\pgfsetroundjoin%
\definecolor{currentfill}{rgb}{0.121569,0.466667,0.705882}%
\pgfsetfillcolor{currentfill}%
\pgfsetfillopacity{0.301842}%
\pgfsetlinewidth{1.003750pt}%
\definecolor{currentstroke}{rgb}{0.121569,0.466667,0.705882}%
\pgfsetstrokecolor{currentstroke}%
\pgfsetstrokeopacity{0.301842}%
\pgfsetdash{}{0pt}%
\pgfpathmoveto{\pgfqpoint{1.762747in}{3.246064in}}%
\pgfpathcurveto{\pgfqpoint{1.770983in}{3.246064in}}{\pgfqpoint{1.778883in}{3.249336in}}{\pgfqpoint{1.784707in}{3.255160in}}%
\pgfpathcurveto{\pgfqpoint{1.790531in}{3.260984in}}{\pgfqpoint{1.793804in}{3.268884in}}{\pgfqpoint{1.793804in}{3.277120in}}%
\pgfpathcurveto{\pgfqpoint{1.793804in}{3.285356in}}{\pgfqpoint{1.790531in}{3.293256in}}{\pgfqpoint{1.784707in}{3.299080in}}%
\pgfpathcurveto{\pgfqpoint{1.778883in}{3.304904in}}{\pgfqpoint{1.770983in}{3.308177in}}{\pgfqpoint{1.762747in}{3.308177in}}%
\pgfpathcurveto{\pgfqpoint{1.754511in}{3.308177in}}{\pgfqpoint{1.746611in}{3.304904in}}{\pgfqpoint{1.740787in}{3.299080in}}%
\pgfpathcurveto{\pgfqpoint{1.734963in}{3.293256in}}{\pgfqpoint{1.731691in}{3.285356in}}{\pgfqpoint{1.731691in}{3.277120in}}%
\pgfpathcurveto{\pgfqpoint{1.731691in}{3.268884in}}{\pgfqpoint{1.734963in}{3.260984in}}{\pgfqpoint{1.740787in}{3.255160in}}%
\pgfpathcurveto{\pgfqpoint{1.746611in}{3.249336in}}{\pgfqpoint{1.754511in}{3.246064in}}{\pgfqpoint{1.762747in}{3.246064in}}%
\pgfpathclose%
\pgfusepath{stroke,fill}%
\end{pgfscope}%
\begin{pgfscope}%
\pgfpathrectangle{\pgfqpoint{0.100000in}{0.220728in}}{\pgfqpoint{3.696000in}{3.696000in}}%
\pgfusepath{clip}%
\pgfsetbuttcap%
\pgfsetroundjoin%
\definecolor{currentfill}{rgb}{0.121569,0.466667,0.705882}%
\pgfsetfillcolor{currentfill}%
\pgfsetfillopacity{0.302372}%
\pgfsetlinewidth{1.003750pt}%
\definecolor{currentstroke}{rgb}{0.121569,0.466667,0.705882}%
\pgfsetstrokecolor{currentstroke}%
\pgfsetstrokeopacity{0.302372}%
\pgfsetdash{}{0pt}%
\pgfpathmoveto{\pgfqpoint{1.797600in}{3.262603in}}%
\pgfpathcurveto{\pgfqpoint{1.805836in}{3.262603in}}{\pgfqpoint{1.813736in}{3.265875in}}{\pgfqpoint{1.819560in}{3.271699in}}%
\pgfpathcurveto{\pgfqpoint{1.825384in}{3.277523in}}{\pgfqpoint{1.828657in}{3.285423in}}{\pgfqpoint{1.828657in}{3.293659in}}%
\pgfpathcurveto{\pgfqpoint{1.828657in}{3.301896in}}{\pgfqpoint{1.825384in}{3.309796in}}{\pgfqpoint{1.819560in}{3.315620in}}%
\pgfpathcurveto{\pgfqpoint{1.813736in}{3.321443in}}{\pgfqpoint{1.805836in}{3.324716in}}{\pgfqpoint{1.797600in}{3.324716in}}%
\pgfpathcurveto{\pgfqpoint{1.789364in}{3.324716in}}{\pgfqpoint{1.781464in}{3.321443in}}{\pgfqpoint{1.775640in}{3.315620in}}%
\pgfpathcurveto{\pgfqpoint{1.769816in}{3.309796in}}{\pgfqpoint{1.766544in}{3.301896in}}{\pgfqpoint{1.766544in}{3.293659in}}%
\pgfpathcurveto{\pgfqpoint{1.766544in}{3.285423in}}{\pgfqpoint{1.769816in}{3.277523in}}{\pgfqpoint{1.775640in}{3.271699in}}%
\pgfpathcurveto{\pgfqpoint{1.781464in}{3.265875in}}{\pgfqpoint{1.789364in}{3.262603in}}{\pgfqpoint{1.797600in}{3.262603in}}%
\pgfpathclose%
\pgfusepath{stroke,fill}%
\end{pgfscope}%
\begin{pgfscope}%
\pgfpathrectangle{\pgfqpoint{0.100000in}{0.220728in}}{\pgfqpoint{3.696000in}{3.696000in}}%
\pgfusepath{clip}%
\pgfsetbuttcap%
\pgfsetroundjoin%
\definecolor{currentfill}{rgb}{0.121569,0.466667,0.705882}%
\pgfsetfillcolor{currentfill}%
\pgfsetfillopacity{0.302396}%
\pgfsetlinewidth{1.003750pt}%
\definecolor{currentstroke}{rgb}{0.121569,0.466667,0.705882}%
\pgfsetstrokecolor{currentstroke}%
\pgfsetstrokeopacity{0.302396}%
\pgfsetdash{}{0pt}%
\pgfpathmoveto{\pgfqpoint{1.788728in}{3.263623in}}%
\pgfpathcurveto{\pgfqpoint{1.796965in}{3.263623in}}{\pgfqpoint{1.804865in}{3.266895in}}{\pgfqpoint{1.810689in}{3.272719in}}%
\pgfpathcurveto{\pgfqpoint{1.816513in}{3.278543in}}{\pgfqpoint{1.819785in}{3.286443in}}{\pgfqpoint{1.819785in}{3.294679in}}%
\pgfpathcurveto{\pgfqpoint{1.819785in}{3.302915in}}{\pgfqpoint{1.816513in}{3.310815in}}{\pgfqpoint{1.810689in}{3.316639in}}%
\pgfpathcurveto{\pgfqpoint{1.804865in}{3.322463in}}{\pgfqpoint{1.796965in}{3.325736in}}{\pgfqpoint{1.788728in}{3.325736in}}%
\pgfpathcurveto{\pgfqpoint{1.780492in}{3.325736in}}{\pgfqpoint{1.772592in}{3.322463in}}{\pgfqpoint{1.766768in}{3.316639in}}%
\pgfpathcurveto{\pgfqpoint{1.760944in}{3.310815in}}{\pgfqpoint{1.757672in}{3.302915in}}{\pgfqpoint{1.757672in}{3.294679in}}%
\pgfpathcurveto{\pgfqpoint{1.757672in}{3.286443in}}{\pgfqpoint{1.760944in}{3.278543in}}{\pgfqpoint{1.766768in}{3.272719in}}%
\pgfpathcurveto{\pgfqpoint{1.772592in}{3.266895in}}{\pgfqpoint{1.780492in}{3.263623in}}{\pgfqpoint{1.788728in}{3.263623in}}%
\pgfpathclose%
\pgfusepath{stroke,fill}%
\end{pgfscope}%
\begin{pgfscope}%
\pgfpathrectangle{\pgfqpoint{0.100000in}{0.220728in}}{\pgfqpoint{3.696000in}{3.696000in}}%
\pgfusepath{clip}%
\pgfsetbuttcap%
\pgfsetroundjoin%
\definecolor{currentfill}{rgb}{0.121569,0.466667,0.705882}%
\pgfsetfillcolor{currentfill}%
\pgfsetfillopacity{0.302573}%
\pgfsetlinewidth{1.003750pt}%
\definecolor{currentstroke}{rgb}{0.121569,0.466667,0.705882}%
\pgfsetstrokecolor{currentstroke}%
\pgfsetstrokeopacity{0.302573}%
\pgfsetdash{}{0pt}%
\pgfpathmoveto{\pgfqpoint{1.760652in}{3.241776in}}%
\pgfpathcurveto{\pgfqpoint{1.768888in}{3.241776in}}{\pgfqpoint{1.776788in}{3.245048in}}{\pgfqpoint{1.782612in}{3.250872in}}%
\pgfpathcurveto{\pgfqpoint{1.788436in}{3.256696in}}{\pgfqpoint{1.791708in}{3.264596in}}{\pgfqpoint{1.791708in}{3.272832in}}%
\pgfpathcurveto{\pgfqpoint{1.791708in}{3.281068in}}{\pgfqpoint{1.788436in}{3.288969in}}{\pgfqpoint{1.782612in}{3.294792in}}%
\pgfpathcurveto{\pgfqpoint{1.776788in}{3.300616in}}{\pgfqpoint{1.768888in}{3.303889in}}{\pgfqpoint{1.760652in}{3.303889in}}%
\pgfpathcurveto{\pgfqpoint{1.752416in}{3.303889in}}{\pgfqpoint{1.744516in}{3.300616in}}{\pgfqpoint{1.738692in}{3.294792in}}%
\pgfpathcurveto{\pgfqpoint{1.732868in}{3.288969in}}{\pgfqpoint{1.729595in}{3.281068in}}{\pgfqpoint{1.729595in}{3.272832in}}%
\pgfpathcurveto{\pgfqpoint{1.729595in}{3.264596in}}{\pgfqpoint{1.732868in}{3.256696in}}{\pgfqpoint{1.738692in}{3.250872in}}%
\pgfpathcurveto{\pgfqpoint{1.744516in}{3.245048in}}{\pgfqpoint{1.752416in}{3.241776in}}{\pgfqpoint{1.760652in}{3.241776in}}%
\pgfpathclose%
\pgfusepath{stroke,fill}%
\end{pgfscope}%
\begin{pgfscope}%
\pgfpathrectangle{\pgfqpoint{0.100000in}{0.220728in}}{\pgfqpoint{3.696000in}{3.696000in}}%
\pgfusepath{clip}%
\pgfsetbuttcap%
\pgfsetroundjoin%
\definecolor{currentfill}{rgb}{0.121569,0.466667,0.705882}%
\pgfsetfillcolor{currentfill}%
\pgfsetfillopacity{0.303018}%
\pgfsetlinewidth{1.003750pt}%
\definecolor{currentstroke}{rgb}{0.121569,0.466667,0.705882}%
\pgfsetstrokecolor{currentstroke}%
\pgfsetstrokeopacity{0.303018}%
\pgfsetdash{}{0pt}%
\pgfpathmoveto{\pgfqpoint{1.758933in}{3.238590in}}%
\pgfpathcurveto{\pgfqpoint{1.767169in}{3.238590in}}{\pgfqpoint{1.775069in}{3.241862in}}{\pgfqpoint{1.780893in}{3.247686in}}%
\pgfpathcurveto{\pgfqpoint{1.786717in}{3.253510in}}{\pgfqpoint{1.789990in}{3.261410in}}{\pgfqpoint{1.789990in}{3.269646in}}%
\pgfpathcurveto{\pgfqpoint{1.789990in}{3.277883in}}{\pgfqpoint{1.786717in}{3.285783in}}{\pgfqpoint{1.780893in}{3.291607in}}%
\pgfpathcurveto{\pgfqpoint{1.775069in}{3.297431in}}{\pgfqpoint{1.767169in}{3.300703in}}{\pgfqpoint{1.758933in}{3.300703in}}%
\pgfpathcurveto{\pgfqpoint{1.750697in}{3.300703in}}{\pgfqpoint{1.742797in}{3.297431in}}{\pgfqpoint{1.736973in}{3.291607in}}%
\pgfpathcurveto{\pgfqpoint{1.731149in}{3.285783in}}{\pgfqpoint{1.727877in}{3.277883in}}{\pgfqpoint{1.727877in}{3.269646in}}%
\pgfpathcurveto{\pgfqpoint{1.727877in}{3.261410in}}{\pgfqpoint{1.731149in}{3.253510in}}{\pgfqpoint{1.736973in}{3.247686in}}%
\pgfpathcurveto{\pgfqpoint{1.742797in}{3.241862in}}{\pgfqpoint{1.750697in}{3.238590in}}{\pgfqpoint{1.758933in}{3.238590in}}%
\pgfpathclose%
\pgfusepath{stroke,fill}%
\end{pgfscope}%
\begin{pgfscope}%
\pgfpathrectangle{\pgfqpoint{0.100000in}{0.220728in}}{\pgfqpoint{3.696000in}{3.696000in}}%
\pgfusepath{clip}%
\pgfsetbuttcap%
\pgfsetroundjoin%
\definecolor{currentfill}{rgb}{0.121569,0.466667,0.705882}%
\pgfsetfillcolor{currentfill}%
\pgfsetfillopacity{0.303162}%
\pgfsetlinewidth{1.003750pt}%
\definecolor{currentstroke}{rgb}{0.121569,0.466667,0.705882}%
\pgfsetstrokecolor{currentstroke}%
\pgfsetstrokeopacity{0.303162}%
\pgfsetdash{}{0pt}%
\pgfpathmoveto{\pgfqpoint{1.792210in}{3.262587in}}%
\pgfpathcurveto{\pgfqpoint{1.800446in}{3.262587in}}{\pgfqpoint{1.808346in}{3.265859in}}{\pgfqpoint{1.814170in}{3.271683in}}%
\pgfpathcurveto{\pgfqpoint{1.819994in}{3.277507in}}{\pgfqpoint{1.823266in}{3.285407in}}{\pgfqpoint{1.823266in}{3.293643in}}%
\pgfpathcurveto{\pgfqpoint{1.823266in}{3.301880in}}{\pgfqpoint{1.819994in}{3.309780in}}{\pgfqpoint{1.814170in}{3.315604in}}%
\pgfpathcurveto{\pgfqpoint{1.808346in}{3.321428in}}{\pgfqpoint{1.800446in}{3.324700in}}{\pgfqpoint{1.792210in}{3.324700in}}%
\pgfpathcurveto{\pgfqpoint{1.783973in}{3.324700in}}{\pgfqpoint{1.776073in}{3.321428in}}{\pgfqpoint{1.770249in}{3.315604in}}%
\pgfpathcurveto{\pgfqpoint{1.764426in}{3.309780in}}{\pgfqpoint{1.761153in}{3.301880in}}{\pgfqpoint{1.761153in}{3.293643in}}%
\pgfpathcurveto{\pgfqpoint{1.761153in}{3.285407in}}{\pgfqpoint{1.764426in}{3.277507in}}{\pgfqpoint{1.770249in}{3.271683in}}%
\pgfpathcurveto{\pgfqpoint{1.776073in}{3.265859in}}{\pgfqpoint{1.783973in}{3.262587in}}{\pgfqpoint{1.792210in}{3.262587in}}%
\pgfpathclose%
\pgfusepath{stroke,fill}%
\end{pgfscope}%
\begin{pgfscope}%
\pgfpathrectangle{\pgfqpoint{0.100000in}{0.220728in}}{\pgfqpoint{3.696000in}{3.696000in}}%
\pgfusepath{clip}%
\pgfsetbuttcap%
\pgfsetroundjoin%
\definecolor{currentfill}{rgb}{0.121569,0.466667,0.705882}%
\pgfsetfillcolor{currentfill}%
\pgfsetfillopacity{0.303297}%
\pgfsetlinewidth{1.003750pt}%
\definecolor{currentstroke}{rgb}{0.121569,0.466667,0.705882}%
\pgfsetstrokecolor{currentstroke}%
\pgfsetstrokeopacity{0.303297}%
\pgfsetdash{}{0pt}%
\pgfpathmoveto{\pgfqpoint{1.758650in}{3.236745in}}%
\pgfpathcurveto{\pgfqpoint{1.766886in}{3.236745in}}{\pgfqpoint{1.774787in}{3.240018in}}{\pgfqpoint{1.780610in}{3.245842in}}%
\pgfpathcurveto{\pgfqpoint{1.786434in}{3.251666in}}{\pgfqpoint{1.789707in}{3.259566in}}{\pgfqpoint{1.789707in}{3.267802in}}%
\pgfpathcurveto{\pgfqpoint{1.789707in}{3.276038in}}{\pgfqpoint{1.786434in}{3.283938in}}{\pgfqpoint{1.780610in}{3.289762in}}%
\pgfpathcurveto{\pgfqpoint{1.774787in}{3.295586in}}{\pgfqpoint{1.766886in}{3.298858in}}{\pgfqpoint{1.758650in}{3.298858in}}%
\pgfpathcurveto{\pgfqpoint{1.750414in}{3.298858in}}{\pgfqpoint{1.742514in}{3.295586in}}{\pgfqpoint{1.736690in}{3.289762in}}%
\pgfpathcurveto{\pgfqpoint{1.730866in}{3.283938in}}{\pgfqpoint{1.727594in}{3.276038in}}{\pgfqpoint{1.727594in}{3.267802in}}%
\pgfpathcurveto{\pgfqpoint{1.727594in}{3.259566in}}{\pgfqpoint{1.730866in}{3.251666in}}{\pgfqpoint{1.736690in}{3.245842in}}%
\pgfpathcurveto{\pgfqpoint{1.742514in}{3.240018in}}{\pgfqpoint{1.750414in}{3.236745in}}{\pgfqpoint{1.758650in}{3.236745in}}%
\pgfpathclose%
\pgfusepath{stroke,fill}%
\end{pgfscope}%
\begin{pgfscope}%
\pgfpathrectangle{\pgfqpoint{0.100000in}{0.220728in}}{\pgfqpoint{3.696000in}{3.696000in}}%
\pgfusepath{clip}%
\pgfsetbuttcap%
\pgfsetroundjoin%
\definecolor{currentfill}{rgb}{0.121569,0.466667,0.705882}%
\pgfsetfillcolor{currentfill}%
\pgfsetfillopacity{0.303699}%
\pgfsetlinewidth{1.003750pt}%
\definecolor{currentstroke}{rgb}{0.121569,0.466667,0.705882}%
\pgfsetstrokecolor{currentstroke}%
\pgfsetstrokeopacity{0.303699}%
\pgfsetdash{}{0pt}%
\pgfpathmoveto{\pgfqpoint{1.756899in}{3.233950in}}%
\pgfpathcurveto{\pgfqpoint{1.765135in}{3.233950in}}{\pgfqpoint{1.773035in}{3.237222in}}{\pgfqpoint{1.778859in}{3.243046in}}%
\pgfpathcurveto{\pgfqpoint{1.784683in}{3.248870in}}{\pgfqpoint{1.787955in}{3.256770in}}{\pgfqpoint{1.787955in}{3.265006in}}%
\pgfpathcurveto{\pgfqpoint{1.787955in}{3.273243in}}{\pgfqpoint{1.784683in}{3.281143in}}{\pgfqpoint{1.778859in}{3.286967in}}%
\pgfpathcurveto{\pgfqpoint{1.773035in}{3.292790in}}{\pgfqpoint{1.765135in}{3.296063in}}{\pgfqpoint{1.756899in}{3.296063in}}%
\pgfpathcurveto{\pgfqpoint{1.748662in}{3.296063in}}{\pgfqpoint{1.740762in}{3.292790in}}{\pgfqpoint{1.734938in}{3.286967in}}%
\pgfpathcurveto{\pgfqpoint{1.729115in}{3.281143in}}{\pgfqpoint{1.725842in}{3.273243in}}{\pgfqpoint{1.725842in}{3.265006in}}%
\pgfpathcurveto{\pgfqpoint{1.725842in}{3.256770in}}{\pgfqpoint{1.729115in}{3.248870in}}{\pgfqpoint{1.734938in}{3.243046in}}%
\pgfpathcurveto{\pgfqpoint{1.740762in}{3.237222in}}{\pgfqpoint{1.748662in}{3.233950in}}{\pgfqpoint{1.756899in}{3.233950in}}%
\pgfpathclose%
\pgfusepath{stroke,fill}%
\end{pgfscope}%
\begin{pgfscope}%
\pgfpathrectangle{\pgfqpoint{0.100000in}{0.220728in}}{\pgfqpoint{3.696000in}{3.696000in}}%
\pgfusepath{clip}%
\pgfsetbuttcap%
\pgfsetroundjoin%
\definecolor{currentfill}{rgb}{0.121569,0.466667,0.705882}%
\pgfsetfillcolor{currentfill}%
\pgfsetfillopacity{0.304040}%
\pgfsetlinewidth{1.003750pt}%
\definecolor{currentstroke}{rgb}{0.121569,0.466667,0.705882}%
\pgfsetstrokecolor{currentstroke}%
\pgfsetstrokeopacity{0.304040}%
\pgfsetdash{}{0pt}%
\pgfpathmoveto{\pgfqpoint{1.756041in}{3.232044in}}%
\pgfpathcurveto{\pgfqpoint{1.764277in}{3.232044in}}{\pgfqpoint{1.772177in}{3.235316in}}{\pgfqpoint{1.778001in}{3.241140in}}%
\pgfpathcurveto{\pgfqpoint{1.783825in}{3.246964in}}{\pgfqpoint{1.787097in}{3.254864in}}{\pgfqpoint{1.787097in}{3.263100in}}%
\pgfpathcurveto{\pgfqpoint{1.787097in}{3.271336in}}{\pgfqpoint{1.783825in}{3.279236in}}{\pgfqpoint{1.778001in}{3.285060in}}%
\pgfpathcurveto{\pgfqpoint{1.772177in}{3.290884in}}{\pgfqpoint{1.764277in}{3.294157in}}{\pgfqpoint{1.756041in}{3.294157in}}%
\pgfpathcurveto{\pgfqpoint{1.747804in}{3.294157in}}{\pgfqpoint{1.739904in}{3.290884in}}{\pgfqpoint{1.734080in}{3.285060in}}%
\pgfpathcurveto{\pgfqpoint{1.728256in}{3.279236in}}{\pgfqpoint{1.724984in}{3.271336in}}{\pgfqpoint{1.724984in}{3.263100in}}%
\pgfpathcurveto{\pgfqpoint{1.724984in}{3.254864in}}{\pgfqpoint{1.728256in}{3.246964in}}{\pgfqpoint{1.734080in}{3.241140in}}%
\pgfpathcurveto{\pgfqpoint{1.739904in}{3.235316in}}{\pgfqpoint{1.747804in}{3.232044in}}{\pgfqpoint{1.756041in}{3.232044in}}%
\pgfpathclose%
\pgfusepath{stroke,fill}%
\end{pgfscope}%
\begin{pgfscope}%
\pgfpathrectangle{\pgfqpoint{0.100000in}{0.220728in}}{\pgfqpoint{3.696000in}{3.696000in}}%
\pgfusepath{clip}%
\pgfsetbuttcap%
\pgfsetroundjoin%
\definecolor{currentfill}{rgb}{0.121569,0.466667,0.705882}%
\pgfsetfillcolor{currentfill}%
\pgfsetfillopacity{0.304289}%
\pgfsetlinewidth{1.003750pt}%
\definecolor{currentstroke}{rgb}{0.121569,0.466667,0.705882}%
\pgfsetstrokecolor{currentstroke}%
\pgfsetstrokeopacity{0.304289}%
\pgfsetdash{}{0pt}%
\pgfpathmoveto{\pgfqpoint{1.802354in}{3.263482in}}%
\pgfpathcurveto{\pgfqpoint{1.810590in}{3.263482in}}{\pgfqpoint{1.818490in}{3.266754in}}{\pgfqpoint{1.824314in}{3.272578in}}%
\pgfpathcurveto{\pgfqpoint{1.830138in}{3.278402in}}{\pgfqpoint{1.833410in}{3.286302in}}{\pgfqpoint{1.833410in}{3.294538in}}%
\pgfpathcurveto{\pgfqpoint{1.833410in}{3.302775in}}{\pgfqpoint{1.830138in}{3.310675in}}{\pgfqpoint{1.824314in}{3.316499in}}%
\pgfpathcurveto{\pgfqpoint{1.818490in}{3.322323in}}{\pgfqpoint{1.810590in}{3.325595in}}{\pgfqpoint{1.802354in}{3.325595in}}%
\pgfpathcurveto{\pgfqpoint{1.794117in}{3.325595in}}{\pgfqpoint{1.786217in}{3.322323in}}{\pgfqpoint{1.780393in}{3.316499in}}%
\pgfpathcurveto{\pgfqpoint{1.774569in}{3.310675in}}{\pgfqpoint{1.771297in}{3.302775in}}{\pgfqpoint{1.771297in}{3.294538in}}%
\pgfpathcurveto{\pgfqpoint{1.771297in}{3.286302in}}{\pgfqpoint{1.774569in}{3.278402in}}{\pgfqpoint{1.780393in}{3.272578in}}%
\pgfpathcurveto{\pgfqpoint{1.786217in}{3.266754in}}{\pgfqpoint{1.794117in}{3.263482in}}{\pgfqpoint{1.802354in}{3.263482in}}%
\pgfpathclose%
\pgfusepath{stroke,fill}%
\end{pgfscope}%
\begin{pgfscope}%
\pgfpathrectangle{\pgfqpoint{0.100000in}{0.220728in}}{\pgfqpoint{3.696000in}{3.696000in}}%
\pgfusepath{clip}%
\pgfsetbuttcap%
\pgfsetroundjoin%
\definecolor{currentfill}{rgb}{0.121569,0.466667,0.705882}%
\pgfsetfillcolor{currentfill}%
\pgfsetfillopacity{0.304707}%
\pgfsetlinewidth{1.003750pt}%
\definecolor{currentstroke}{rgb}{0.121569,0.466667,0.705882}%
\pgfsetstrokecolor{currentstroke}%
\pgfsetstrokeopacity{0.304707}%
\pgfsetdash{}{0pt}%
\pgfpathmoveto{\pgfqpoint{1.755326in}{3.228275in}}%
\pgfpathcurveto{\pgfqpoint{1.763563in}{3.228275in}}{\pgfqpoint{1.771463in}{3.231547in}}{\pgfqpoint{1.777287in}{3.237371in}}%
\pgfpathcurveto{\pgfqpoint{1.783110in}{3.243195in}}{\pgfqpoint{1.786383in}{3.251095in}}{\pgfqpoint{1.786383in}{3.259331in}}%
\pgfpathcurveto{\pgfqpoint{1.786383in}{3.267568in}}{\pgfqpoint{1.783110in}{3.275468in}}{\pgfqpoint{1.777287in}{3.281292in}}%
\pgfpathcurveto{\pgfqpoint{1.771463in}{3.287116in}}{\pgfqpoint{1.763563in}{3.290388in}}{\pgfqpoint{1.755326in}{3.290388in}}%
\pgfpathcurveto{\pgfqpoint{1.747090in}{3.290388in}}{\pgfqpoint{1.739190in}{3.287116in}}{\pgfqpoint{1.733366in}{3.281292in}}%
\pgfpathcurveto{\pgfqpoint{1.727542in}{3.275468in}}{\pgfqpoint{1.724270in}{3.267568in}}{\pgfqpoint{1.724270in}{3.259331in}}%
\pgfpathcurveto{\pgfqpoint{1.724270in}{3.251095in}}{\pgfqpoint{1.727542in}{3.243195in}}{\pgfqpoint{1.733366in}{3.237371in}}%
\pgfpathcurveto{\pgfqpoint{1.739190in}{3.231547in}}{\pgfqpoint{1.747090in}{3.228275in}}{\pgfqpoint{1.755326in}{3.228275in}}%
\pgfpathclose%
\pgfusepath{stroke,fill}%
\end{pgfscope}%
\begin{pgfscope}%
\pgfpathrectangle{\pgfqpoint{0.100000in}{0.220728in}}{\pgfqpoint{3.696000in}{3.696000in}}%
\pgfusepath{clip}%
\pgfsetbuttcap%
\pgfsetroundjoin%
\definecolor{currentfill}{rgb}{0.121569,0.466667,0.705882}%
\pgfsetfillcolor{currentfill}%
\pgfsetfillopacity{0.304875}%
\pgfsetlinewidth{1.003750pt}%
\definecolor{currentstroke}{rgb}{0.121569,0.466667,0.705882}%
\pgfsetstrokecolor{currentstroke}%
\pgfsetstrokeopacity{0.304875}%
\pgfsetdash{}{0pt}%
\pgfpathmoveto{\pgfqpoint{1.754496in}{3.227094in}}%
\pgfpathcurveto{\pgfqpoint{1.762732in}{3.227094in}}{\pgfqpoint{1.770632in}{3.230366in}}{\pgfqpoint{1.776456in}{3.236190in}}%
\pgfpathcurveto{\pgfqpoint{1.782280in}{3.242014in}}{\pgfqpoint{1.785552in}{3.249914in}}{\pgfqpoint{1.785552in}{3.258151in}}%
\pgfpathcurveto{\pgfqpoint{1.785552in}{3.266387in}}{\pgfqpoint{1.782280in}{3.274287in}}{\pgfqpoint{1.776456in}{3.280111in}}%
\pgfpathcurveto{\pgfqpoint{1.770632in}{3.285935in}}{\pgfqpoint{1.762732in}{3.289207in}}{\pgfqpoint{1.754496in}{3.289207in}}%
\pgfpathcurveto{\pgfqpoint{1.746259in}{3.289207in}}{\pgfqpoint{1.738359in}{3.285935in}}{\pgfqpoint{1.732535in}{3.280111in}}%
\pgfpathcurveto{\pgfqpoint{1.726711in}{3.274287in}}{\pgfqpoint{1.723439in}{3.266387in}}{\pgfqpoint{1.723439in}{3.258151in}}%
\pgfpathcurveto{\pgfqpoint{1.723439in}{3.249914in}}{\pgfqpoint{1.726711in}{3.242014in}}{\pgfqpoint{1.732535in}{3.236190in}}%
\pgfpathcurveto{\pgfqpoint{1.738359in}{3.230366in}}{\pgfqpoint{1.746259in}{3.227094in}}{\pgfqpoint{1.754496in}{3.227094in}}%
\pgfpathclose%
\pgfusepath{stroke,fill}%
\end{pgfscope}%
\begin{pgfscope}%
\pgfpathrectangle{\pgfqpoint{0.100000in}{0.220728in}}{\pgfqpoint{3.696000in}{3.696000in}}%
\pgfusepath{clip}%
\pgfsetbuttcap%
\pgfsetroundjoin%
\definecolor{currentfill}{rgb}{0.121569,0.466667,0.705882}%
\pgfsetfillcolor{currentfill}%
\pgfsetfillopacity{0.305292}%
\pgfsetlinewidth{1.003750pt}%
\definecolor{currentstroke}{rgb}{0.121569,0.466667,0.705882}%
\pgfsetstrokecolor{currentstroke}%
\pgfsetstrokeopacity{0.305292}%
\pgfsetdash{}{0pt}%
\pgfpathmoveto{\pgfqpoint{1.804919in}{3.263602in}}%
\pgfpathcurveto{\pgfqpoint{1.813155in}{3.263602in}}{\pgfqpoint{1.821055in}{3.266874in}}{\pgfqpoint{1.826879in}{3.272698in}}%
\pgfpathcurveto{\pgfqpoint{1.832703in}{3.278522in}}{\pgfqpoint{1.835975in}{3.286422in}}{\pgfqpoint{1.835975in}{3.294658in}}%
\pgfpathcurveto{\pgfqpoint{1.835975in}{3.302895in}}{\pgfqpoint{1.832703in}{3.310795in}}{\pgfqpoint{1.826879in}{3.316619in}}%
\pgfpathcurveto{\pgfqpoint{1.821055in}{3.322442in}}{\pgfqpoint{1.813155in}{3.325715in}}{\pgfqpoint{1.804919in}{3.325715in}}%
\pgfpathcurveto{\pgfqpoint{1.796683in}{3.325715in}}{\pgfqpoint{1.788783in}{3.322442in}}{\pgfqpoint{1.782959in}{3.316619in}}%
\pgfpathcurveto{\pgfqpoint{1.777135in}{3.310795in}}{\pgfqpoint{1.773862in}{3.302895in}}{\pgfqpoint{1.773862in}{3.294658in}}%
\pgfpathcurveto{\pgfqpoint{1.773862in}{3.286422in}}{\pgfqpoint{1.777135in}{3.278522in}}{\pgfqpoint{1.782959in}{3.272698in}}%
\pgfpathcurveto{\pgfqpoint{1.788783in}{3.266874in}}{\pgfqpoint{1.796683in}{3.263602in}}{\pgfqpoint{1.804919in}{3.263602in}}%
\pgfpathclose%
\pgfusepath{stroke,fill}%
\end{pgfscope}%
\begin{pgfscope}%
\pgfpathrectangle{\pgfqpoint{0.100000in}{0.220728in}}{\pgfqpoint{3.696000in}{3.696000in}}%
\pgfusepath{clip}%
\pgfsetbuttcap%
\pgfsetroundjoin%
\definecolor{currentfill}{rgb}{0.121569,0.466667,0.705882}%
\pgfsetfillcolor{currentfill}%
\pgfsetfillopacity{0.305343}%
\pgfsetlinewidth{1.003750pt}%
\definecolor{currentstroke}{rgb}{0.121569,0.466667,0.705882}%
\pgfsetstrokecolor{currentstroke}%
\pgfsetstrokeopacity{0.305343}%
\pgfsetdash{}{0pt}%
\pgfpathmoveto{\pgfqpoint{1.754013in}{3.224672in}}%
\pgfpathcurveto{\pgfqpoint{1.762249in}{3.224672in}}{\pgfqpoint{1.770149in}{3.227944in}}{\pgfqpoint{1.775973in}{3.233768in}}%
\pgfpathcurveto{\pgfqpoint{1.781797in}{3.239592in}}{\pgfqpoint{1.785069in}{3.247492in}}{\pgfqpoint{1.785069in}{3.255728in}}%
\pgfpathcurveto{\pgfqpoint{1.785069in}{3.263965in}}{\pgfqpoint{1.781797in}{3.271865in}}{\pgfqpoint{1.775973in}{3.277689in}}%
\pgfpathcurveto{\pgfqpoint{1.770149in}{3.283513in}}{\pgfqpoint{1.762249in}{3.286785in}}{\pgfqpoint{1.754013in}{3.286785in}}%
\pgfpathcurveto{\pgfqpoint{1.745776in}{3.286785in}}{\pgfqpoint{1.737876in}{3.283513in}}{\pgfqpoint{1.732052in}{3.277689in}}%
\pgfpathcurveto{\pgfqpoint{1.726228in}{3.271865in}}{\pgfqpoint{1.722956in}{3.263965in}}{\pgfqpoint{1.722956in}{3.255728in}}%
\pgfpathcurveto{\pgfqpoint{1.722956in}{3.247492in}}{\pgfqpoint{1.726228in}{3.239592in}}{\pgfqpoint{1.732052in}{3.233768in}}%
\pgfpathcurveto{\pgfqpoint{1.737876in}{3.227944in}}{\pgfqpoint{1.745776in}{3.224672in}}{\pgfqpoint{1.754013in}{3.224672in}}%
\pgfpathclose%
\pgfusepath{stroke,fill}%
\end{pgfscope}%
\begin{pgfscope}%
\pgfpathrectangle{\pgfqpoint{0.100000in}{0.220728in}}{\pgfqpoint{3.696000in}{3.696000in}}%
\pgfusepath{clip}%
\pgfsetbuttcap%
\pgfsetroundjoin%
\definecolor{currentfill}{rgb}{0.121569,0.466667,0.705882}%
\pgfsetfillcolor{currentfill}%
\pgfsetfillopacity{0.305664}%
\pgfsetlinewidth{1.003750pt}%
\definecolor{currentstroke}{rgb}{0.121569,0.466667,0.705882}%
\pgfsetstrokecolor{currentstroke}%
\pgfsetstrokeopacity{0.305664}%
\pgfsetdash{}{0pt}%
\pgfpathmoveto{\pgfqpoint{1.753399in}{3.222806in}}%
\pgfpathcurveto{\pgfqpoint{1.761635in}{3.222806in}}{\pgfqpoint{1.769535in}{3.226078in}}{\pgfqpoint{1.775359in}{3.231902in}}%
\pgfpathcurveto{\pgfqpoint{1.781183in}{3.237726in}}{\pgfqpoint{1.784455in}{3.245626in}}{\pgfqpoint{1.784455in}{3.253862in}}%
\pgfpathcurveto{\pgfqpoint{1.784455in}{3.262099in}}{\pgfqpoint{1.781183in}{3.269999in}}{\pgfqpoint{1.775359in}{3.275823in}}%
\pgfpathcurveto{\pgfqpoint{1.769535in}{3.281646in}}{\pgfqpoint{1.761635in}{3.284919in}}{\pgfqpoint{1.753399in}{3.284919in}}%
\pgfpathcurveto{\pgfqpoint{1.745163in}{3.284919in}}{\pgfqpoint{1.737263in}{3.281646in}}{\pgfqpoint{1.731439in}{3.275823in}}%
\pgfpathcurveto{\pgfqpoint{1.725615in}{3.269999in}}{\pgfqpoint{1.722342in}{3.262099in}}{\pgfqpoint{1.722342in}{3.253862in}}%
\pgfpathcurveto{\pgfqpoint{1.722342in}{3.245626in}}{\pgfqpoint{1.725615in}{3.237726in}}{\pgfqpoint{1.731439in}{3.231902in}}%
\pgfpathcurveto{\pgfqpoint{1.737263in}{3.226078in}}{\pgfqpoint{1.745163in}{3.222806in}}{\pgfqpoint{1.753399in}{3.222806in}}%
\pgfpathclose%
\pgfusepath{stroke,fill}%
\end{pgfscope}%
\begin{pgfscope}%
\pgfpathrectangle{\pgfqpoint{0.100000in}{0.220728in}}{\pgfqpoint{3.696000in}{3.696000in}}%
\pgfusepath{clip}%
\pgfsetbuttcap%
\pgfsetroundjoin%
\definecolor{currentfill}{rgb}{0.121569,0.466667,0.705882}%
\pgfsetfillcolor{currentfill}%
\pgfsetfillopacity{0.305720}%
\pgfsetlinewidth{1.003750pt}%
\definecolor{currentstroke}{rgb}{0.121569,0.466667,0.705882}%
\pgfsetstrokecolor{currentstroke}%
\pgfsetstrokeopacity{0.305720}%
\pgfsetdash{}{0pt}%
\pgfpathmoveto{\pgfqpoint{1.806495in}{3.263591in}}%
\pgfpathcurveto{\pgfqpoint{1.814732in}{3.263591in}}{\pgfqpoint{1.822632in}{3.266863in}}{\pgfqpoint{1.828456in}{3.272687in}}%
\pgfpathcurveto{\pgfqpoint{1.834279in}{3.278511in}}{\pgfqpoint{1.837552in}{3.286411in}}{\pgfqpoint{1.837552in}{3.294648in}}%
\pgfpathcurveto{\pgfqpoint{1.837552in}{3.302884in}}{\pgfqpoint{1.834279in}{3.310784in}}{\pgfqpoint{1.828456in}{3.316608in}}%
\pgfpathcurveto{\pgfqpoint{1.822632in}{3.322432in}}{\pgfqpoint{1.814732in}{3.325704in}}{\pgfqpoint{1.806495in}{3.325704in}}%
\pgfpathcurveto{\pgfqpoint{1.798259in}{3.325704in}}{\pgfqpoint{1.790359in}{3.322432in}}{\pgfqpoint{1.784535in}{3.316608in}}%
\pgfpathcurveto{\pgfqpoint{1.778711in}{3.310784in}}{\pgfqpoint{1.775439in}{3.302884in}}{\pgfqpoint{1.775439in}{3.294648in}}%
\pgfpathcurveto{\pgfqpoint{1.775439in}{3.286411in}}{\pgfqpoint{1.778711in}{3.278511in}}{\pgfqpoint{1.784535in}{3.272687in}}%
\pgfpathcurveto{\pgfqpoint{1.790359in}{3.266863in}}{\pgfqpoint{1.798259in}{3.263591in}}{\pgfqpoint{1.806495in}{3.263591in}}%
\pgfpathclose%
\pgfusepath{stroke,fill}%
\end{pgfscope}%
\begin{pgfscope}%
\pgfpathrectangle{\pgfqpoint{0.100000in}{0.220728in}}{\pgfqpoint{3.696000in}{3.696000in}}%
\pgfusepath{clip}%
\pgfsetbuttcap%
\pgfsetroundjoin%
\definecolor{currentfill}{rgb}{0.121569,0.466667,0.705882}%
\pgfsetfillcolor{currentfill}%
\pgfsetfillopacity{0.306146}%
\pgfsetlinewidth{1.003750pt}%
\definecolor{currentstroke}{rgb}{0.121569,0.466667,0.705882}%
\pgfsetstrokecolor{currentstroke}%
\pgfsetstrokeopacity{0.306146}%
\pgfsetdash{}{0pt}%
\pgfpathmoveto{\pgfqpoint{1.751617in}{3.219618in}}%
\pgfpathcurveto{\pgfqpoint{1.759853in}{3.219618in}}{\pgfqpoint{1.767753in}{3.222891in}}{\pgfqpoint{1.773577in}{3.228715in}}%
\pgfpathcurveto{\pgfqpoint{1.779401in}{3.234539in}}{\pgfqpoint{1.782674in}{3.242439in}}{\pgfqpoint{1.782674in}{3.250675in}}%
\pgfpathcurveto{\pgfqpoint{1.782674in}{3.258911in}}{\pgfqpoint{1.779401in}{3.266811in}}{\pgfqpoint{1.773577in}{3.272635in}}%
\pgfpathcurveto{\pgfqpoint{1.767753in}{3.278459in}}{\pgfqpoint{1.759853in}{3.281731in}}{\pgfqpoint{1.751617in}{3.281731in}}%
\pgfpathcurveto{\pgfqpoint{1.743381in}{3.281731in}}{\pgfqpoint{1.735481in}{3.278459in}}{\pgfqpoint{1.729657in}{3.272635in}}%
\pgfpathcurveto{\pgfqpoint{1.723833in}{3.266811in}}{\pgfqpoint{1.720561in}{3.258911in}}{\pgfqpoint{1.720561in}{3.250675in}}%
\pgfpathcurveto{\pgfqpoint{1.720561in}{3.242439in}}{\pgfqpoint{1.723833in}{3.234539in}}{\pgfqpoint{1.729657in}{3.228715in}}%
\pgfpathcurveto{\pgfqpoint{1.735481in}{3.222891in}}{\pgfqpoint{1.743381in}{3.219618in}}{\pgfqpoint{1.751617in}{3.219618in}}%
\pgfpathclose%
\pgfusepath{stroke,fill}%
\end{pgfscope}%
\begin{pgfscope}%
\pgfpathrectangle{\pgfqpoint{0.100000in}{0.220728in}}{\pgfqpoint{3.696000in}{3.696000in}}%
\pgfusepath{clip}%
\pgfsetbuttcap%
\pgfsetroundjoin%
\definecolor{currentfill}{rgb}{0.121569,0.466667,0.705882}%
\pgfsetfillcolor{currentfill}%
\pgfsetfillopacity{0.306402}%
\pgfsetlinewidth{1.003750pt}%
\definecolor{currentstroke}{rgb}{0.121569,0.466667,0.705882}%
\pgfsetstrokecolor{currentstroke}%
\pgfsetstrokeopacity{0.306402}%
\pgfsetdash{}{0pt}%
\pgfpathmoveto{\pgfqpoint{1.751333in}{3.218129in}}%
\pgfpathcurveto{\pgfqpoint{1.759570in}{3.218129in}}{\pgfqpoint{1.767470in}{3.221402in}}{\pgfqpoint{1.773294in}{3.227226in}}%
\pgfpathcurveto{\pgfqpoint{1.779118in}{3.233049in}}{\pgfqpoint{1.782390in}{3.240950in}}{\pgfqpoint{1.782390in}{3.249186in}}%
\pgfpathcurveto{\pgfqpoint{1.782390in}{3.257422in}}{\pgfqpoint{1.779118in}{3.265322in}}{\pgfqpoint{1.773294in}{3.271146in}}%
\pgfpathcurveto{\pgfqpoint{1.767470in}{3.276970in}}{\pgfqpoint{1.759570in}{3.280242in}}{\pgfqpoint{1.751333in}{3.280242in}}%
\pgfpathcurveto{\pgfqpoint{1.743097in}{3.280242in}}{\pgfqpoint{1.735197in}{3.276970in}}{\pgfqpoint{1.729373in}{3.271146in}}%
\pgfpathcurveto{\pgfqpoint{1.723549in}{3.265322in}}{\pgfqpoint{1.720277in}{3.257422in}}{\pgfqpoint{1.720277in}{3.249186in}}%
\pgfpathcurveto{\pgfqpoint{1.720277in}{3.240950in}}{\pgfqpoint{1.723549in}{3.233049in}}{\pgfqpoint{1.729373in}{3.227226in}}%
\pgfpathcurveto{\pgfqpoint{1.735197in}{3.221402in}}{\pgfqpoint{1.743097in}{3.218129in}}{\pgfqpoint{1.751333in}{3.218129in}}%
\pgfpathclose%
\pgfusepath{stroke,fill}%
\end{pgfscope}%
\begin{pgfscope}%
\pgfpathrectangle{\pgfqpoint{0.100000in}{0.220728in}}{\pgfqpoint{3.696000in}{3.696000in}}%
\pgfusepath{clip}%
\pgfsetbuttcap%
\pgfsetroundjoin%
\definecolor{currentfill}{rgb}{0.121569,0.466667,0.705882}%
\pgfsetfillcolor{currentfill}%
\pgfsetfillopacity{0.306503}%
\pgfsetlinewidth{1.003750pt}%
\definecolor{currentstroke}{rgb}{0.121569,0.466667,0.705882}%
\pgfsetstrokecolor{currentstroke}%
\pgfsetstrokeopacity{0.306503}%
\pgfsetdash{}{0pt}%
\pgfpathmoveto{\pgfqpoint{1.808201in}{3.263653in}}%
\pgfpathcurveto{\pgfqpoint{1.816437in}{3.263653in}}{\pgfqpoint{1.824337in}{3.266925in}}{\pgfqpoint{1.830161in}{3.272749in}}%
\pgfpathcurveto{\pgfqpoint{1.835985in}{3.278573in}}{\pgfqpoint{1.839257in}{3.286473in}}{\pgfqpoint{1.839257in}{3.294709in}}%
\pgfpathcurveto{\pgfqpoint{1.839257in}{3.302945in}}{\pgfqpoint{1.835985in}{3.310845in}}{\pgfqpoint{1.830161in}{3.316669in}}%
\pgfpathcurveto{\pgfqpoint{1.824337in}{3.322493in}}{\pgfqpoint{1.816437in}{3.325766in}}{\pgfqpoint{1.808201in}{3.325766in}}%
\pgfpathcurveto{\pgfqpoint{1.799965in}{3.325766in}}{\pgfqpoint{1.792065in}{3.322493in}}{\pgfqpoint{1.786241in}{3.316669in}}%
\pgfpathcurveto{\pgfqpoint{1.780417in}{3.310845in}}{\pgfqpoint{1.777144in}{3.302945in}}{\pgfqpoint{1.777144in}{3.294709in}}%
\pgfpathcurveto{\pgfqpoint{1.777144in}{3.286473in}}{\pgfqpoint{1.780417in}{3.278573in}}{\pgfqpoint{1.786241in}{3.272749in}}%
\pgfpathcurveto{\pgfqpoint{1.792065in}{3.266925in}}{\pgfqpoint{1.799965in}{3.263653in}}{\pgfqpoint{1.808201in}{3.263653in}}%
\pgfpathclose%
\pgfusepath{stroke,fill}%
\end{pgfscope}%
\begin{pgfscope}%
\pgfpathrectangle{\pgfqpoint{0.100000in}{0.220728in}}{\pgfqpoint{3.696000in}{3.696000in}}%
\pgfusepath{clip}%
\pgfsetbuttcap%
\pgfsetroundjoin%
\definecolor{currentfill}{rgb}{0.121569,0.466667,0.705882}%
\pgfsetfillcolor{currentfill}%
\pgfsetfillopacity{0.306788}%
\pgfsetlinewidth{1.003750pt}%
\definecolor{currentstroke}{rgb}{0.121569,0.466667,0.705882}%
\pgfsetstrokecolor{currentstroke}%
\pgfsetstrokeopacity{0.306788}%
\pgfsetdash{}{0pt}%
\pgfpathmoveto{\pgfqpoint{1.809343in}{3.263590in}}%
\pgfpathcurveto{\pgfqpoint{1.817579in}{3.263590in}}{\pgfqpoint{1.825479in}{3.266862in}}{\pgfqpoint{1.831303in}{3.272686in}}%
\pgfpathcurveto{\pgfqpoint{1.837127in}{3.278510in}}{\pgfqpoint{1.840399in}{3.286410in}}{\pgfqpoint{1.840399in}{3.294647in}}%
\pgfpathcurveto{\pgfqpoint{1.840399in}{3.302883in}}{\pgfqpoint{1.837127in}{3.310783in}}{\pgfqpoint{1.831303in}{3.316607in}}%
\pgfpathcurveto{\pgfqpoint{1.825479in}{3.322431in}}{\pgfqpoint{1.817579in}{3.325703in}}{\pgfqpoint{1.809343in}{3.325703in}}%
\pgfpathcurveto{\pgfqpoint{1.801106in}{3.325703in}}{\pgfqpoint{1.793206in}{3.322431in}}{\pgfqpoint{1.787382in}{3.316607in}}%
\pgfpathcurveto{\pgfqpoint{1.781558in}{3.310783in}}{\pgfqpoint{1.778286in}{3.302883in}}{\pgfqpoint{1.778286in}{3.294647in}}%
\pgfpathcurveto{\pgfqpoint{1.778286in}{3.286410in}}{\pgfqpoint{1.781558in}{3.278510in}}{\pgfqpoint{1.787382in}{3.272686in}}%
\pgfpathcurveto{\pgfqpoint{1.793206in}{3.266862in}}{\pgfqpoint{1.801106in}{3.263590in}}{\pgfqpoint{1.809343in}{3.263590in}}%
\pgfpathclose%
\pgfusepath{stroke,fill}%
\end{pgfscope}%
\begin{pgfscope}%
\pgfpathrectangle{\pgfqpoint{0.100000in}{0.220728in}}{\pgfqpoint{3.696000in}{3.696000in}}%
\pgfusepath{clip}%
\pgfsetbuttcap%
\pgfsetroundjoin%
\definecolor{currentfill}{rgb}{0.121569,0.466667,0.705882}%
\pgfsetfillcolor{currentfill}%
\pgfsetfillopacity{0.306795}%
\pgfsetlinewidth{1.003750pt}%
\definecolor{currentstroke}{rgb}{0.121569,0.466667,0.705882}%
\pgfsetstrokecolor{currentstroke}%
\pgfsetstrokeopacity{0.306795}%
\pgfsetdash{}{0pt}%
\pgfpathmoveto{\pgfqpoint{1.749924in}{3.215824in}}%
\pgfpathcurveto{\pgfqpoint{1.758160in}{3.215824in}}{\pgfqpoint{1.766060in}{3.219096in}}{\pgfqpoint{1.771884in}{3.224920in}}%
\pgfpathcurveto{\pgfqpoint{1.777708in}{3.230744in}}{\pgfqpoint{1.780981in}{3.238644in}}{\pgfqpoint{1.780981in}{3.246880in}}%
\pgfpathcurveto{\pgfqpoint{1.780981in}{3.255117in}}{\pgfqpoint{1.777708in}{3.263017in}}{\pgfqpoint{1.771884in}{3.268841in}}%
\pgfpathcurveto{\pgfqpoint{1.766060in}{3.274665in}}{\pgfqpoint{1.758160in}{3.277937in}}{\pgfqpoint{1.749924in}{3.277937in}}%
\pgfpathcurveto{\pgfqpoint{1.741688in}{3.277937in}}{\pgfqpoint{1.733788in}{3.274665in}}{\pgfqpoint{1.727964in}{3.268841in}}%
\pgfpathcurveto{\pgfqpoint{1.722140in}{3.263017in}}{\pgfqpoint{1.718868in}{3.255117in}}{\pgfqpoint{1.718868in}{3.246880in}}%
\pgfpathcurveto{\pgfqpoint{1.718868in}{3.238644in}}{\pgfqpoint{1.722140in}{3.230744in}}{\pgfqpoint{1.727964in}{3.224920in}}%
\pgfpathcurveto{\pgfqpoint{1.733788in}{3.219096in}}{\pgfqpoint{1.741688in}{3.215824in}}{\pgfqpoint{1.749924in}{3.215824in}}%
\pgfpathclose%
\pgfusepath{stroke,fill}%
\end{pgfscope}%
\begin{pgfscope}%
\pgfpathrectangle{\pgfqpoint{0.100000in}{0.220728in}}{\pgfqpoint{3.696000in}{3.696000in}}%
\pgfusepath{clip}%
\pgfsetbuttcap%
\pgfsetroundjoin%
\definecolor{currentfill}{rgb}{0.121569,0.466667,0.705882}%
\pgfsetfillcolor{currentfill}%
\pgfsetfillopacity{0.306981}%
\pgfsetlinewidth{1.003750pt}%
\definecolor{currentstroke}{rgb}{0.121569,0.466667,0.705882}%
\pgfsetstrokecolor{currentstroke}%
\pgfsetstrokeopacity{0.306981}%
\pgfsetdash{}{0pt}%
\pgfpathmoveto{\pgfqpoint{1.809913in}{3.263546in}}%
\pgfpathcurveto{\pgfqpoint{1.818150in}{3.263546in}}{\pgfqpoint{1.826050in}{3.266819in}}{\pgfqpoint{1.831873in}{3.272643in}}%
\pgfpathcurveto{\pgfqpoint{1.837697in}{3.278466in}}{\pgfqpoint{1.840970in}{3.286367in}}{\pgfqpoint{1.840970in}{3.294603in}}%
\pgfpathcurveto{\pgfqpoint{1.840970in}{3.302839in}}{\pgfqpoint{1.837697in}{3.310739in}}{\pgfqpoint{1.831873in}{3.316563in}}%
\pgfpathcurveto{\pgfqpoint{1.826050in}{3.322387in}}{\pgfqpoint{1.818150in}{3.325659in}}{\pgfqpoint{1.809913in}{3.325659in}}%
\pgfpathcurveto{\pgfqpoint{1.801677in}{3.325659in}}{\pgfqpoint{1.793777in}{3.322387in}}{\pgfqpoint{1.787953in}{3.316563in}}%
\pgfpathcurveto{\pgfqpoint{1.782129in}{3.310739in}}{\pgfqpoint{1.778857in}{3.302839in}}{\pgfqpoint{1.778857in}{3.294603in}}%
\pgfpathcurveto{\pgfqpoint{1.778857in}{3.286367in}}{\pgfqpoint{1.782129in}{3.278466in}}{\pgfqpoint{1.787953in}{3.272643in}}%
\pgfpathcurveto{\pgfqpoint{1.793777in}{3.266819in}}{\pgfqpoint{1.801677in}{3.263546in}}{\pgfqpoint{1.809913in}{3.263546in}}%
\pgfpathclose%
\pgfusepath{stroke,fill}%
\end{pgfscope}%
\begin{pgfscope}%
\pgfpathrectangle{\pgfqpoint{0.100000in}{0.220728in}}{\pgfqpoint{3.696000in}{3.696000in}}%
\pgfusepath{clip}%
\pgfsetbuttcap%
\pgfsetroundjoin%
\definecolor{currentfill}{rgb}{0.121569,0.466667,0.705882}%
\pgfsetfillcolor{currentfill}%
\pgfsetfillopacity{0.307500}%
\pgfsetlinewidth{1.003750pt}%
\definecolor{currentstroke}{rgb}{0.121569,0.466667,0.705882}%
\pgfsetstrokecolor{currentstroke}%
\pgfsetstrokeopacity{0.307500}%
\pgfsetdash{}{0pt}%
\pgfpathmoveto{\pgfqpoint{1.747736in}{3.211115in}}%
\pgfpathcurveto{\pgfqpoint{1.755973in}{3.211115in}}{\pgfqpoint{1.763873in}{3.214387in}}{\pgfqpoint{1.769697in}{3.220211in}}%
\pgfpathcurveto{\pgfqpoint{1.775521in}{3.226035in}}{\pgfqpoint{1.778793in}{3.233935in}}{\pgfqpoint{1.778793in}{3.242172in}}%
\pgfpathcurveto{\pgfqpoint{1.778793in}{3.250408in}}{\pgfqpoint{1.775521in}{3.258308in}}{\pgfqpoint{1.769697in}{3.264132in}}%
\pgfpathcurveto{\pgfqpoint{1.763873in}{3.269956in}}{\pgfqpoint{1.755973in}{3.273228in}}{\pgfqpoint{1.747736in}{3.273228in}}%
\pgfpathcurveto{\pgfqpoint{1.739500in}{3.273228in}}{\pgfqpoint{1.731600in}{3.269956in}}{\pgfqpoint{1.725776in}{3.264132in}}%
\pgfpathcurveto{\pgfqpoint{1.719952in}{3.258308in}}{\pgfqpoint{1.716680in}{3.250408in}}{\pgfqpoint{1.716680in}{3.242172in}}%
\pgfpathcurveto{\pgfqpoint{1.716680in}{3.233935in}}{\pgfqpoint{1.719952in}{3.226035in}}{\pgfqpoint{1.725776in}{3.220211in}}%
\pgfpathcurveto{\pgfqpoint{1.731600in}{3.214387in}}{\pgfqpoint{1.739500in}{3.211115in}}{\pgfqpoint{1.747736in}{3.211115in}}%
\pgfpathclose%
\pgfusepath{stroke,fill}%
\end{pgfscope}%
\begin{pgfscope}%
\pgfpathrectangle{\pgfqpoint{0.100000in}{0.220728in}}{\pgfqpoint{3.696000in}{3.696000in}}%
\pgfusepath{clip}%
\pgfsetbuttcap%
\pgfsetroundjoin%
\definecolor{currentfill}{rgb}{0.121569,0.466667,0.705882}%
\pgfsetfillcolor{currentfill}%
\pgfsetfillopacity{0.307561}%
\pgfsetlinewidth{1.003750pt}%
\definecolor{currentstroke}{rgb}{0.121569,0.466667,0.705882}%
\pgfsetstrokecolor{currentstroke}%
\pgfsetstrokeopacity{0.307561}%
\pgfsetdash{}{0pt}%
\pgfpathmoveto{\pgfqpoint{1.811436in}{3.263574in}}%
\pgfpathcurveto{\pgfqpoint{1.819672in}{3.263574in}}{\pgfqpoint{1.827573in}{3.266846in}}{\pgfqpoint{1.833396in}{3.272670in}}%
\pgfpathcurveto{\pgfqpoint{1.839220in}{3.278494in}}{\pgfqpoint{1.842493in}{3.286394in}}{\pgfqpoint{1.842493in}{3.294630in}}%
\pgfpathcurveto{\pgfqpoint{1.842493in}{3.302866in}}{\pgfqpoint{1.839220in}{3.310767in}}{\pgfqpoint{1.833396in}{3.316590in}}%
\pgfpathcurveto{\pgfqpoint{1.827573in}{3.322414in}}{\pgfqpoint{1.819672in}{3.325687in}}{\pgfqpoint{1.811436in}{3.325687in}}%
\pgfpathcurveto{\pgfqpoint{1.803200in}{3.325687in}}{\pgfqpoint{1.795300in}{3.322414in}}{\pgfqpoint{1.789476in}{3.316590in}}%
\pgfpathcurveto{\pgfqpoint{1.783652in}{3.310767in}}{\pgfqpoint{1.780380in}{3.302866in}}{\pgfqpoint{1.780380in}{3.294630in}}%
\pgfpathcurveto{\pgfqpoint{1.780380in}{3.286394in}}{\pgfqpoint{1.783652in}{3.278494in}}{\pgfqpoint{1.789476in}{3.272670in}}%
\pgfpathcurveto{\pgfqpoint{1.795300in}{3.266846in}}{\pgfqpoint{1.803200in}{3.263574in}}{\pgfqpoint{1.811436in}{3.263574in}}%
\pgfpathclose%
\pgfusepath{stroke,fill}%
\end{pgfscope}%
\begin{pgfscope}%
\pgfpathrectangle{\pgfqpoint{0.100000in}{0.220728in}}{\pgfqpoint{3.696000in}{3.696000in}}%
\pgfusepath{clip}%
\pgfsetbuttcap%
\pgfsetroundjoin%
\definecolor{currentfill}{rgb}{0.121569,0.466667,0.705882}%
\pgfsetfillcolor{currentfill}%
\pgfsetfillopacity{0.307843}%
\pgfsetlinewidth{1.003750pt}%
\definecolor{currentstroke}{rgb}{0.121569,0.466667,0.705882}%
\pgfsetstrokecolor{currentstroke}%
\pgfsetstrokeopacity{0.307843}%
\pgfsetdash{}{0pt}%
\pgfpathmoveto{\pgfqpoint{1.812378in}{3.263734in}}%
\pgfpathcurveto{\pgfqpoint{1.820614in}{3.263734in}}{\pgfqpoint{1.828514in}{3.267006in}}{\pgfqpoint{1.834338in}{3.272830in}}%
\pgfpathcurveto{\pgfqpoint{1.840162in}{3.278654in}}{\pgfqpoint{1.843434in}{3.286554in}}{\pgfqpoint{1.843434in}{3.294790in}}%
\pgfpathcurveto{\pgfqpoint{1.843434in}{3.303026in}}{\pgfqpoint{1.840162in}{3.310927in}}{\pgfqpoint{1.834338in}{3.316750in}}%
\pgfpathcurveto{\pgfqpoint{1.828514in}{3.322574in}}{\pgfqpoint{1.820614in}{3.325847in}}{\pgfqpoint{1.812378in}{3.325847in}}%
\pgfpathcurveto{\pgfqpoint{1.804141in}{3.325847in}}{\pgfqpoint{1.796241in}{3.322574in}}{\pgfqpoint{1.790417in}{3.316750in}}%
\pgfpathcurveto{\pgfqpoint{1.784593in}{3.310927in}}{\pgfqpoint{1.781321in}{3.303026in}}{\pgfqpoint{1.781321in}{3.294790in}}%
\pgfpathcurveto{\pgfqpoint{1.781321in}{3.286554in}}{\pgfqpoint{1.784593in}{3.278654in}}{\pgfqpoint{1.790417in}{3.272830in}}%
\pgfpathcurveto{\pgfqpoint{1.796241in}{3.267006in}}{\pgfqpoint{1.804141in}{3.263734in}}{\pgfqpoint{1.812378in}{3.263734in}}%
\pgfpathclose%
\pgfusepath{stroke,fill}%
\end{pgfscope}%
\begin{pgfscope}%
\pgfpathrectangle{\pgfqpoint{0.100000in}{0.220728in}}{\pgfqpoint{3.696000in}{3.696000in}}%
\pgfusepath{clip}%
\pgfsetbuttcap%
\pgfsetroundjoin%
\definecolor{currentfill}{rgb}{0.121569,0.466667,0.705882}%
\pgfsetfillcolor{currentfill}%
\pgfsetfillopacity{0.308039}%
\pgfsetlinewidth{1.003750pt}%
\definecolor{currentstroke}{rgb}{0.121569,0.466667,0.705882}%
\pgfsetstrokecolor{currentstroke}%
\pgfsetstrokeopacity{0.308039}%
\pgfsetdash{}{0pt}%
\pgfpathmoveto{\pgfqpoint{1.747356in}{3.207541in}}%
\pgfpathcurveto{\pgfqpoint{1.755593in}{3.207541in}}{\pgfqpoint{1.763493in}{3.210813in}}{\pgfqpoint{1.769317in}{3.216637in}}%
\pgfpathcurveto{\pgfqpoint{1.775141in}{3.222461in}}{\pgfqpoint{1.778413in}{3.230361in}}{\pgfqpoint{1.778413in}{3.238597in}}%
\pgfpathcurveto{\pgfqpoint{1.778413in}{3.246834in}}{\pgfqpoint{1.775141in}{3.254734in}}{\pgfqpoint{1.769317in}{3.260557in}}%
\pgfpathcurveto{\pgfqpoint{1.763493in}{3.266381in}}{\pgfqpoint{1.755593in}{3.269654in}}{\pgfqpoint{1.747356in}{3.269654in}}%
\pgfpathcurveto{\pgfqpoint{1.739120in}{3.269654in}}{\pgfqpoint{1.731220in}{3.266381in}}{\pgfqpoint{1.725396in}{3.260557in}}%
\pgfpathcurveto{\pgfqpoint{1.719572in}{3.254734in}}{\pgfqpoint{1.716300in}{3.246834in}}{\pgfqpoint{1.716300in}{3.238597in}}%
\pgfpathcurveto{\pgfqpoint{1.716300in}{3.230361in}}{\pgfqpoint{1.719572in}{3.222461in}}{\pgfqpoint{1.725396in}{3.216637in}}%
\pgfpathcurveto{\pgfqpoint{1.731220in}{3.210813in}}{\pgfqpoint{1.739120in}{3.207541in}}{\pgfqpoint{1.747356in}{3.207541in}}%
\pgfpathclose%
\pgfusepath{stroke,fill}%
\end{pgfscope}%
\begin{pgfscope}%
\pgfpathrectangle{\pgfqpoint{0.100000in}{0.220728in}}{\pgfqpoint{3.696000in}{3.696000in}}%
\pgfusepath{clip}%
\pgfsetbuttcap%
\pgfsetroundjoin%
\definecolor{currentfill}{rgb}{0.121569,0.466667,0.705882}%
\pgfsetfillcolor{currentfill}%
\pgfsetfillopacity{0.308723}%
\pgfsetlinewidth{1.003750pt}%
\definecolor{currentstroke}{rgb}{0.121569,0.466667,0.705882}%
\pgfsetstrokecolor{currentstroke}%
\pgfsetstrokeopacity{0.308723}%
\pgfsetdash{}{0pt}%
\pgfpathmoveto{\pgfqpoint{1.743674in}{3.202316in}}%
\pgfpathcurveto{\pgfqpoint{1.751910in}{3.202316in}}{\pgfqpoint{1.759810in}{3.205588in}}{\pgfqpoint{1.765634in}{3.211412in}}%
\pgfpathcurveto{\pgfqpoint{1.771458in}{3.217236in}}{\pgfqpoint{1.774730in}{3.225136in}}{\pgfqpoint{1.774730in}{3.233372in}}%
\pgfpathcurveto{\pgfqpoint{1.774730in}{3.241609in}}{\pgfqpoint{1.771458in}{3.249509in}}{\pgfqpoint{1.765634in}{3.255333in}}%
\pgfpathcurveto{\pgfqpoint{1.759810in}{3.261157in}}{\pgfqpoint{1.751910in}{3.264429in}}{\pgfqpoint{1.743674in}{3.264429in}}%
\pgfpathcurveto{\pgfqpoint{1.735438in}{3.264429in}}{\pgfqpoint{1.727538in}{3.261157in}}{\pgfqpoint{1.721714in}{3.255333in}}%
\pgfpathcurveto{\pgfqpoint{1.715890in}{3.249509in}}{\pgfqpoint{1.712617in}{3.241609in}}{\pgfqpoint{1.712617in}{3.233372in}}%
\pgfpathcurveto{\pgfqpoint{1.712617in}{3.225136in}}{\pgfqpoint{1.715890in}{3.217236in}}{\pgfqpoint{1.721714in}{3.211412in}}%
\pgfpathcurveto{\pgfqpoint{1.727538in}{3.205588in}}{\pgfqpoint{1.735438in}{3.202316in}}{\pgfqpoint{1.743674in}{3.202316in}}%
\pgfpathclose%
\pgfusepath{stroke,fill}%
\end{pgfscope}%
\begin{pgfscope}%
\pgfpathrectangle{\pgfqpoint{0.100000in}{0.220728in}}{\pgfqpoint{3.696000in}{3.696000in}}%
\pgfusepath{clip}%
\pgfsetbuttcap%
\pgfsetroundjoin%
\definecolor{currentfill}{rgb}{0.121569,0.466667,0.705882}%
\pgfsetfillcolor{currentfill}%
\pgfsetfillopacity{0.308736}%
\pgfsetlinewidth{1.003750pt}%
\definecolor{currentstroke}{rgb}{0.121569,0.466667,0.705882}%
\pgfsetstrokecolor{currentstroke}%
\pgfsetstrokeopacity{0.308736}%
\pgfsetdash{}{0pt}%
\pgfpathmoveto{\pgfqpoint{1.816178in}{3.263044in}}%
\pgfpathcurveto{\pgfqpoint{1.824414in}{3.263044in}}{\pgfqpoint{1.832314in}{3.266317in}}{\pgfqpoint{1.838138in}{3.272141in}}%
\pgfpathcurveto{\pgfqpoint{1.843962in}{3.277965in}}{\pgfqpoint{1.847234in}{3.285865in}}{\pgfqpoint{1.847234in}{3.294101in}}%
\pgfpathcurveto{\pgfqpoint{1.847234in}{3.302337in}}{\pgfqpoint{1.843962in}{3.310237in}}{\pgfqpoint{1.838138in}{3.316061in}}%
\pgfpathcurveto{\pgfqpoint{1.832314in}{3.321885in}}{\pgfqpoint{1.824414in}{3.325157in}}{\pgfqpoint{1.816178in}{3.325157in}}%
\pgfpathcurveto{\pgfqpoint{1.807941in}{3.325157in}}{\pgfqpoint{1.800041in}{3.321885in}}{\pgfqpoint{1.794217in}{3.316061in}}%
\pgfpathcurveto{\pgfqpoint{1.788393in}{3.310237in}}{\pgfqpoint{1.785121in}{3.302337in}}{\pgfqpoint{1.785121in}{3.294101in}}%
\pgfpathcurveto{\pgfqpoint{1.785121in}{3.285865in}}{\pgfqpoint{1.788393in}{3.277965in}}{\pgfqpoint{1.794217in}{3.272141in}}%
\pgfpathcurveto{\pgfqpoint{1.800041in}{3.266317in}}{\pgfqpoint{1.807941in}{3.263044in}}{\pgfqpoint{1.816178in}{3.263044in}}%
\pgfpathclose%
\pgfusepath{stroke,fill}%
\end{pgfscope}%
\begin{pgfscope}%
\pgfpathrectangle{\pgfqpoint{0.100000in}{0.220728in}}{\pgfqpoint{3.696000in}{3.696000in}}%
\pgfusepath{clip}%
\pgfsetbuttcap%
\pgfsetroundjoin%
\definecolor{currentfill}{rgb}{0.121569,0.466667,0.705882}%
\pgfsetfillcolor{currentfill}%
\pgfsetfillopacity{0.309564}%
\pgfsetlinewidth{1.003750pt}%
\definecolor{currentstroke}{rgb}{0.121569,0.466667,0.705882}%
\pgfsetstrokecolor{currentstroke}%
\pgfsetstrokeopacity{0.309564}%
\pgfsetdash{}{0pt}%
\pgfpathmoveto{\pgfqpoint{1.741729in}{3.197436in}}%
\pgfpathcurveto{\pgfqpoint{1.749966in}{3.197436in}}{\pgfqpoint{1.757866in}{3.200708in}}{\pgfqpoint{1.763690in}{3.206532in}}%
\pgfpathcurveto{\pgfqpoint{1.769513in}{3.212356in}}{\pgfqpoint{1.772786in}{3.220256in}}{\pgfqpoint{1.772786in}{3.228492in}}%
\pgfpathcurveto{\pgfqpoint{1.772786in}{3.236728in}}{\pgfqpoint{1.769513in}{3.244628in}}{\pgfqpoint{1.763690in}{3.250452in}}%
\pgfpathcurveto{\pgfqpoint{1.757866in}{3.256276in}}{\pgfqpoint{1.749966in}{3.259549in}}{\pgfqpoint{1.741729in}{3.259549in}}%
\pgfpathcurveto{\pgfqpoint{1.733493in}{3.259549in}}{\pgfqpoint{1.725593in}{3.256276in}}{\pgfqpoint{1.719769in}{3.250452in}}%
\pgfpathcurveto{\pgfqpoint{1.713945in}{3.244628in}}{\pgfqpoint{1.710673in}{3.236728in}}{\pgfqpoint{1.710673in}{3.228492in}}%
\pgfpathcurveto{\pgfqpoint{1.710673in}{3.220256in}}{\pgfqpoint{1.713945in}{3.212356in}}{\pgfqpoint{1.719769in}{3.206532in}}%
\pgfpathcurveto{\pgfqpoint{1.725593in}{3.200708in}}{\pgfqpoint{1.733493in}{3.197436in}}{\pgfqpoint{1.741729in}{3.197436in}}%
\pgfpathclose%
\pgfusepath{stroke,fill}%
\end{pgfscope}%
\begin{pgfscope}%
\pgfpathrectangle{\pgfqpoint{0.100000in}{0.220728in}}{\pgfqpoint{3.696000in}{3.696000in}}%
\pgfusepath{clip}%
\pgfsetbuttcap%
\pgfsetroundjoin%
\definecolor{currentfill}{rgb}{0.121569,0.466667,0.705882}%
\pgfsetfillcolor{currentfill}%
\pgfsetfillopacity{0.309904}%
\pgfsetlinewidth{1.003750pt}%
\definecolor{currentstroke}{rgb}{0.121569,0.466667,0.705882}%
\pgfsetstrokecolor{currentstroke}%
\pgfsetstrokeopacity{0.309904}%
\pgfsetdash{}{0pt}%
\pgfpathmoveto{\pgfqpoint{1.820442in}{3.262289in}}%
\pgfpathcurveto{\pgfqpoint{1.828678in}{3.262289in}}{\pgfqpoint{1.836578in}{3.265562in}}{\pgfqpoint{1.842402in}{3.271386in}}%
\pgfpathcurveto{\pgfqpoint{1.848226in}{3.277209in}}{\pgfqpoint{1.851499in}{3.285110in}}{\pgfqpoint{1.851499in}{3.293346in}}%
\pgfpathcurveto{\pgfqpoint{1.851499in}{3.301582in}}{\pgfqpoint{1.848226in}{3.309482in}}{\pgfqpoint{1.842402in}{3.315306in}}%
\pgfpathcurveto{\pgfqpoint{1.836578in}{3.321130in}}{\pgfqpoint{1.828678in}{3.324402in}}{\pgfqpoint{1.820442in}{3.324402in}}%
\pgfpathcurveto{\pgfqpoint{1.812206in}{3.324402in}}{\pgfqpoint{1.804306in}{3.321130in}}{\pgfqpoint{1.798482in}{3.315306in}}%
\pgfpathcurveto{\pgfqpoint{1.792658in}{3.309482in}}{\pgfqpoint{1.789386in}{3.301582in}}{\pgfqpoint{1.789386in}{3.293346in}}%
\pgfpathcurveto{\pgfqpoint{1.789386in}{3.285110in}}{\pgfqpoint{1.792658in}{3.277209in}}{\pgfqpoint{1.798482in}{3.271386in}}%
\pgfpathcurveto{\pgfqpoint{1.804306in}{3.265562in}}{\pgfqpoint{1.812206in}{3.262289in}}{\pgfqpoint{1.820442in}{3.262289in}}%
\pgfpathclose%
\pgfusepath{stroke,fill}%
\end{pgfscope}%
\begin{pgfscope}%
\pgfpathrectangle{\pgfqpoint{0.100000in}{0.220728in}}{\pgfqpoint{3.696000in}{3.696000in}}%
\pgfusepath{clip}%
\pgfsetbuttcap%
\pgfsetroundjoin%
\definecolor{currentfill}{rgb}{0.121569,0.466667,0.705882}%
\pgfsetfillcolor{currentfill}%
\pgfsetfillopacity{0.310321}%
\pgfsetlinewidth{1.003750pt}%
\definecolor{currentstroke}{rgb}{0.121569,0.466667,0.705882}%
\pgfsetstrokecolor{currentstroke}%
\pgfsetstrokeopacity{0.310321}%
\pgfsetdash{}{0pt}%
\pgfpathmoveto{\pgfqpoint{1.822806in}{3.260916in}}%
\pgfpathcurveto{\pgfqpoint{1.831043in}{3.260916in}}{\pgfqpoint{1.838943in}{3.264188in}}{\pgfqpoint{1.844767in}{3.270012in}}%
\pgfpathcurveto{\pgfqpoint{1.850591in}{3.275836in}}{\pgfqpoint{1.853863in}{3.283736in}}{\pgfqpoint{1.853863in}{3.291972in}}%
\pgfpathcurveto{\pgfqpoint{1.853863in}{3.300208in}}{\pgfqpoint{1.850591in}{3.308108in}}{\pgfqpoint{1.844767in}{3.313932in}}%
\pgfpathcurveto{\pgfqpoint{1.838943in}{3.319756in}}{\pgfqpoint{1.831043in}{3.323029in}}{\pgfqpoint{1.822806in}{3.323029in}}%
\pgfpathcurveto{\pgfqpoint{1.814570in}{3.323029in}}{\pgfqpoint{1.806670in}{3.319756in}}{\pgfqpoint{1.800846in}{3.313932in}}%
\pgfpathcurveto{\pgfqpoint{1.795022in}{3.308108in}}{\pgfqpoint{1.791750in}{3.300208in}}{\pgfqpoint{1.791750in}{3.291972in}}%
\pgfpathcurveto{\pgfqpoint{1.791750in}{3.283736in}}{\pgfqpoint{1.795022in}{3.275836in}}{\pgfqpoint{1.800846in}{3.270012in}}%
\pgfpathcurveto{\pgfqpoint{1.806670in}{3.264188in}}{\pgfqpoint{1.814570in}{3.260916in}}{\pgfqpoint{1.822806in}{3.260916in}}%
\pgfpathclose%
\pgfusepath{stroke,fill}%
\end{pgfscope}%
\begin{pgfscope}%
\pgfpathrectangle{\pgfqpoint{0.100000in}{0.220728in}}{\pgfqpoint{3.696000in}{3.696000in}}%
\pgfusepath{clip}%
\pgfsetbuttcap%
\pgfsetroundjoin%
\definecolor{currentfill}{rgb}{0.121569,0.466667,0.705882}%
\pgfsetfillcolor{currentfill}%
\pgfsetfillopacity{0.311140}%
\pgfsetlinewidth{1.003750pt}%
\definecolor{currentstroke}{rgb}{0.121569,0.466667,0.705882}%
\pgfsetstrokecolor{currentstroke}%
\pgfsetstrokeopacity{0.311140}%
\pgfsetdash{}{0pt}%
\pgfpathmoveto{\pgfqpoint{1.739613in}{3.187898in}}%
\pgfpathcurveto{\pgfqpoint{1.747849in}{3.187898in}}{\pgfqpoint{1.755749in}{3.191171in}}{\pgfqpoint{1.761573in}{3.196995in}}%
\pgfpathcurveto{\pgfqpoint{1.767397in}{3.202819in}}{\pgfqpoint{1.770669in}{3.210719in}}{\pgfqpoint{1.770669in}{3.218955in}}%
\pgfpathcurveto{\pgfqpoint{1.770669in}{3.227191in}}{\pgfqpoint{1.767397in}{3.235091in}}{\pgfqpoint{1.761573in}{3.240915in}}%
\pgfpathcurveto{\pgfqpoint{1.755749in}{3.246739in}}{\pgfqpoint{1.747849in}{3.250011in}}{\pgfqpoint{1.739613in}{3.250011in}}%
\pgfpathcurveto{\pgfqpoint{1.731376in}{3.250011in}}{\pgfqpoint{1.723476in}{3.246739in}}{\pgfqpoint{1.717652in}{3.240915in}}%
\pgfpathcurveto{\pgfqpoint{1.711829in}{3.235091in}}{\pgfqpoint{1.708556in}{3.227191in}}{\pgfqpoint{1.708556in}{3.218955in}}%
\pgfpathcurveto{\pgfqpoint{1.708556in}{3.210719in}}{\pgfqpoint{1.711829in}{3.202819in}}{\pgfqpoint{1.717652in}{3.196995in}}%
\pgfpathcurveto{\pgfqpoint{1.723476in}{3.191171in}}{\pgfqpoint{1.731376in}{3.187898in}}{\pgfqpoint{1.739613in}{3.187898in}}%
\pgfpathclose%
\pgfusepath{stroke,fill}%
\end{pgfscope}%
\begin{pgfscope}%
\pgfpathrectangle{\pgfqpoint{0.100000in}{0.220728in}}{\pgfqpoint{3.696000in}{3.696000in}}%
\pgfusepath{clip}%
\pgfsetbuttcap%
\pgfsetroundjoin%
\definecolor{currentfill}{rgb}{0.121569,0.466667,0.705882}%
\pgfsetfillcolor{currentfill}%
\pgfsetfillopacity{0.311524}%
\pgfsetlinewidth{1.003750pt}%
\definecolor{currentstroke}{rgb}{0.121569,0.466667,0.705882}%
\pgfsetstrokecolor{currentstroke}%
\pgfsetstrokeopacity{0.311524}%
\pgfsetdash{}{0pt}%
\pgfpathmoveto{\pgfqpoint{1.825829in}{3.261117in}}%
\pgfpathcurveto{\pgfqpoint{1.834065in}{3.261117in}}{\pgfqpoint{1.841966in}{3.264389in}}{\pgfqpoint{1.847789in}{3.270213in}}%
\pgfpathcurveto{\pgfqpoint{1.853613in}{3.276037in}}{\pgfqpoint{1.856886in}{3.283937in}}{\pgfqpoint{1.856886in}{3.292173in}}%
\pgfpathcurveto{\pgfqpoint{1.856886in}{3.300409in}}{\pgfqpoint{1.853613in}{3.308309in}}{\pgfqpoint{1.847789in}{3.314133in}}%
\pgfpathcurveto{\pgfqpoint{1.841966in}{3.319957in}}{\pgfqpoint{1.834065in}{3.323230in}}{\pgfqpoint{1.825829in}{3.323230in}}%
\pgfpathcurveto{\pgfqpoint{1.817593in}{3.323230in}}{\pgfqpoint{1.809693in}{3.319957in}}{\pgfqpoint{1.803869in}{3.314133in}}%
\pgfpathcurveto{\pgfqpoint{1.798045in}{3.308309in}}{\pgfqpoint{1.794773in}{3.300409in}}{\pgfqpoint{1.794773in}{3.292173in}}%
\pgfpathcurveto{\pgfqpoint{1.794773in}{3.283937in}}{\pgfqpoint{1.798045in}{3.276037in}}{\pgfqpoint{1.803869in}{3.270213in}}%
\pgfpathcurveto{\pgfqpoint{1.809693in}{3.264389in}}{\pgfqpoint{1.817593in}{3.261117in}}{\pgfqpoint{1.825829in}{3.261117in}}%
\pgfpathclose%
\pgfusepath{stroke,fill}%
\end{pgfscope}%
\begin{pgfscope}%
\pgfpathrectangle{\pgfqpoint{0.100000in}{0.220728in}}{\pgfqpoint{3.696000in}{3.696000in}}%
\pgfusepath{clip}%
\pgfsetbuttcap%
\pgfsetroundjoin%
\definecolor{currentfill}{rgb}{0.121569,0.466667,0.705882}%
\pgfsetfillcolor{currentfill}%
\pgfsetfillopacity{0.311978}%
\pgfsetlinewidth{1.003750pt}%
\definecolor{currentstroke}{rgb}{0.121569,0.466667,0.705882}%
\pgfsetstrokecolor{currentstroke}%
\pgfsetstrokeopacity{0.311978}%
\pgfsetdash{}{0pt}%
\pgfpathmoveto{\pgfqpoint{1.735365in}{3.181543in}}%
\pgfpathcurveto{\pgfqpoint{1.743601in}{3.181543in}}{\pgfqpoint{1.751501in}{3.184815in}}{\pgfqpoint{1.757325in}{3.190639in}}%
\pgfpathcurveto{\pgfqpoint{1.763149in}{3.196463in}}{\pgfqpoint{1.766422in}{3.204363in}}{\pgfqpoint{1.766422in}{3.212599in}}%
\pgfpathcurveto{\pgfqpoint{1.766422in}{3.220836in}}{\pgfqpoint{1.763149in}{3.228736in}}{\pgfqpoint{1.757325in}{3.234560in}}%
\pgfpathcurveto{\pgfqpoint{1.751501in}{3.240383in}}{\pgfqpoint{1.743601in}{3.243656in}}{\pgfqpoint{1.735365in}{3.243656in}}%
\pgfpathcurveto{\pgfqpoint{1.727129in}{3.243656in}}{\pgfqpoint{1.719229in}{3.240383in}}{\pgfqpoint{1.713405in}{3.234560in}}%
\pgfpathcurveto{\pgfqpoint{1.707581in}{3.228736in}}{\pgfqpoint{1.704309in}{3.220836in}}{\pgfqpoint{1.704309in}{3.212599in}}%
\pgfpathcurveto{\pgfqpoint{1.704309in}{3.204363in}}{\pgfqpoint{1.707581in}{3.196463in}}{\pgfqpoint{1.713405in}{3.190639in}}%
\pgfpathcurveto{\pgfqpoint{1.719229in}{3.184815in}}{\pgfqpoint{1.727129in}{3.181543in}}{\pgfqpoint{1.735365in}{3.181543in}}%
\pgfpathclose%
\pgfusepath{stroke,fill}%
\end{pgfscope}%
\begin{pgfscope}%
\pgfpathrectangle{\pgfqpoint{0.100000in}{0.220728in}}{\pgfqpoint{3.696000in}{3.696000in}}%
\pgfusepath{clip}%
\pgfsetbuttcap%
\pgfsetroundjoin%
\definecolor{currentfill}{rgb}{0.121569,0.466667,0.705882}%
\pgfsetfillcolor{currentfill}%
\pgfsetfillopacity{0.312794}%
\pgfsetlinewidth{1.003750pt}%
\definecolor{currentstroke}{rgb}{0.121569,0.466667,0.705882}%
\pgfsetstrokecolor{currentstroke}%
\pgfsetstrokeopacity{0.312794}%
\pgfsetdash{}{0pt}%
\pgfpathmoveto{\pgfqpoint{1.734205in}{3.176729in}}%
\pgfpathcurveto{\pgfqpoint{1.742441in}{3.176729in}}{\pgfqpoint{1.750341in}{3.180001in}}{\pgfqpoint{1.756165in}{3.185825in}}%
\pgfpathcurveto{\pgfqpoint{1.761989in}{3.191649in}}{\pgfqpoint{1.765261in}{3.199549in}}{\pgfqpoint{1.765261in}{3.207785in}}%
\pgfpathcurveto{\pgfqpoint{1.765261in}{3.216022in}}{\pgfqpoint{1.761989in}{3.223922in}}{\pgfqpoint{1.756165in}{3.229746in}}%
\pgfpathcurveto{\pgfqpoint{1.750341in}{3.235570in}}{\pgfqpoint{1.742441in}{3.238842in}}{\pgfqpoint{1.734205in}{3.238842in}}%
\pgfpathcurveto{\pgfqpoint{1.725969in}{3.238842in}}{\pgfqpoint{1.718069in}{3.235570in}}{\pgfqpoint{1.712245in}{3.229746in}}%
\pgfpathcurveto{\pgfqpoint{1.706421in}{3.223922in}}{\pgfqpoint{1.703148in}{3.216022in}}{\pgfqpoint{1.703148in}{3.207785in}}%
\pgfpathcurveto{\pgfqpoint{1.703148in}{3.199549in}}{\pgfqpoint{1.706421in}{3.191649in}}{\pgfqpoint{1.712245in}{3.185825in}}%
\pgfpathcurveto{\pgfqpoint{1.718069in}{3.180001in}}{\pgfqpoint{1.725969in}{3.176729in}}{\pgfqpoint{1.734205in}{3.176729in}}%
\pgfpathclose%
\pgfusepath{stroke,fill}%
\end{pgfscope}%
\begin{pgfscope}%
\pgfpathrectangle{\pgfqpoint{0.100000in}{0.220728in}}{\pgfqpoint{3.696000in}{3.696000in}}%
\pgfusepath{clip}%
\pgfsetbuttcap%
\pgfsetroundjoin%
\definecolor{currentfill}{rgb}{0.121569,0.466667,0.705882}%
\pgfsetfillcolor{currentfill}%
\pgfsetfillopacity{0.313498}%
\pgfsetlinewidth{1.003750pt}%
\definecolor{currentstroke}{rgb}{0.121569,0.466667,0.705882}%
\pgfsetstrokecolor{currentstroke}%
\pgfsetstrokeopacity{0.313498}%
\pgfsetdash{}{0pt}%
\pgfpathmoveto{\pgfqpoint{1.732448in}{3.173054in}}%
\pgfpathcurveto{\pgfqpoint{1.740684in}{3.173054in}}{\pgfqpoint{1.748584in}{3.176326in}}{\pgfqpoint{1.754408in}{3.182150in}}%
\pgfpathcurveto{\pgfqpoint{1.760232in}{3.187974in}}{\pgfqpoint{1.763504in}{3.195874in}}{\pgfqpoint{1.763504in}{3.204111in}}%
\pgfpathcurveto{\pgfqpoint{1.763504in}{3.212347in}}{\pgfqpoint{1.760232in}{3.220247in}}{\pgfqpoint{1.754408in}{3.226071in}}%
\pgfpathcurveto{\pgfqpoint{1.748584in}{3.231895in}}{\pgfqpoint{1.740684in}{3.235167in}}{\pgfqpoint{1.732448in}{3.235167in}}%
\pgfpathcurveto{\pgfqpoint{1.724211in}{3.235167in}}{\pgfqpoint{1.716311in}{3.231895in}}{\pgfqpoint{1.710487in}{3.226071in}}%
\pgfpathcurveto{\pgfqpoint{1.704663in}{3.220247in}}{\pgfqpoint{1.701391in}{3.212347in}}{\pgfqpoint{1.701391in}{3.204111in}}%
\pgfpathcurveto{\pgfqpoint{1.701391in}{3.195874in}}{\pgfqpoint{1.704663in}{3.187974in}}{\pgfqpoint{1.710487in}{3.182150in}}%
\pgfpathcurveto{\pgfqpoint{1.716311in}{3.176326in}}{\pgfqpoint{1.724211in}{3.173054in}}{\pgfqpoint{1.732448in}{3.173054in}}%
\pgfpathclose%
\pgfusepath{stroke,fill}%
\end{pgfscope}%
\begin{pgfscope}%
\pgfpathrectangle{\pgfqpoint{0.100000in}{0.220728in}}{\pgfqpoint{3.696000in}{3.696000in}}%
\pgfusepath{clip}%
\pgfsetbuttcap%
\pgfsetroundjoin%
\definecolor{currentfill}{rgb}{0.121569,0.466667,0.705882}%
\pgfsetfillcolor{currentfill}%
\pgfsetfillopacity{0.314504}%
\pgfsetlinewidth{1.003750pt}%
\definecolor{currentstroke}{rgb}{0.121569,0.466667,0.705882}%
\pgfsetstrokecolor{currentstroke}%
\pgfsetstrokeopacity{0.314504}%
\pgfsetdash{}{0pt}%
\pgfpathmoveto{\pgfqpoint{1.826017in}{3.260078in}}%
\pgfpathcurveto{\pgfqpoint{1.834253in}{3.260078in}}{\pgfqpoint{1.842153in}{3.263351in}}{\pgfqpoint{1.847977in}{3.269175in}}%
\pgfpathcurveto{\pgfqpoint{1.853801in}{3.274998in}}{\pgfqpoint{1.857073in}{3.282899in}}{\pgfqpoint{1.857073in}{3.291135in}}%
\pgfpathcurveto{\pgfqpoint{1.857073in}{3.299371in}}{\pgfqpoint{1.853801in}{3.307271in}}{\pgfqpoint{1.847977in}{3.313095in}}%
\pgfpathcurveto{\pgfqpoint{1.842153in}{3.318919in}}{\pgfqpoint{1.834253in}{3.322191in}}{\pgfqpoint{1.826017in}{3.322191in}}%
\pgfpathcurveto{\pgfqpoint{1.817780in}{3.322191in}}{\pgfqpoint{1.809880in}{3.318919in}}{\pgfqpoint{1.804056in}{3.313095in}}%
\pgfpathcurveto{\pgfqpoint{1.798232in}{3.307271in}}{\pgfqpoint{1.794960in}{3.299371in}}{\pgfqpoint{1.794960in}{3.291135in}}%
\pgfpathcurveto{\pgfqpoint{1.794960in}{3.282899in}}{\pgfqpoint{1.798232in}{3.274998in}}{\pgfqpoint{1.804056in}{3.269175in}}%
\pgfpathcurveto{\pgfqpoint{1.809880in}{3.263351in}}{\pgfqpoint{1.817780in}{3.260078in}}{\pgfqpoint{1.826017in}{3.260078in}}%
\pgfpathclose%
\pgfusepath{stroke,fill}%
\end{pgfscope}%
\begin{pgfscope}%
\pgfpathrectangle{\pgfqpoint{0.100000in}{0.220728in}}{\pgfqpoint{3.696000in}{3.696000in}}%
\pgfusepath{clip}%
\pgfsetbuttcap%
\pgfsetroundjoin%
\definecolor{currentfill}{rgb}{0.121569,0.466667,0.705882}%
\pgfsetfillcolor{currentfill}%
\pgfsetfillopacity{0.314548}%
\pgfsetlinewidth{1.003750pt}%
\definecolor{currentstroke}{rgb}{0.121569,0.466667,0.705882}%
\pgfsetstrokecolor{currentstroke}%
\pgfsetstrokeopacity{0.314548}%
\pgfsetdash{}{0pt}%
\pgfpathmoveto{\pgfqpoint{1.728638in}{3.166071in}}%
\pgfpathcurveto{\pgfqpoint{1.736875in}{3.166071in}}{\pgfqpoint{1.744775in}{3.169343in}}{\pgfqpoint{1.750599in}{3.175167in}}%
\pgfpathcurveto{\pgfqpoint{1.756422in}{3.180991in}}{\pgfqpoint{1.759695in}{3.188891in}}{\pgfqpoint{1.759695in}{3.197127in}}%
\pgfpathcurveto{\pgfqpoint{1.759695in}{3.205363in}}{\pgfqpoint{1.756422in}{3.213263in}}{\pgfqpoint{1.750599in}{3.219087in}}%
\pgfpathcurveto{\pgfqpoint{1.744775in}{3.224911in}}{\pgfqpoint{1.736875in}{3.228184in}}{\pgfqpoint{1.728638in}{3.228184in}}%
\pgfpathcurveto{\pgfqpoint{1.720402in}{3.228184in}}{\pgfqpoint{1.712502in}{3.224911in}}{\pgfqpoint{1.706678in}{3.219087in}}%
\pgfpathcurveto{\pgfqpoint{1.700854in}{3.213263in}}{\pgfqpoint{1.697582in}{3.205363in}}{\pgfqpoint{1.697582in}{3.197127in}}%
\pgfpathcurveto{\pgfqpoint{1.697582in}{3.188891in}}{\pgfqpoint{1.700854in}{3.180991in}}{\pgfqpoint{1.706678in}{3.175167in}}%
\pgfpathcurveto{\pgfqpoint{1.712502in}{3.169343in}}{\pgfqpoint{1.720402in}{3.166071in}}{\pgfqpoint{1.728638in}{3.166071in}}%
\pgfpathclose%
\pgfusepath{stroke,fill}%
\end{pgfscope}%
\begin{pgfscope}%
\pgfpathrectangle{\pgfqpoint{0.100000in}{0.220728in}}{\pgfqpoint{3.696000in}{3.696000in}}%
\pgfusepath{clip}%
\pgfsetbuttcap%
\pgfsetroundjoin%
\definecolor{currentfill}{rgb}{0.121569,0.466667,0.705882}%
\pgfsetfillcolor{currentfill}%
\pgfsetfillopacity{0.315151}%
\pgfsetlinewidth{1.003750pt}%
\definecolor{currentstroke}{rgb}{0.121569,0.466667,0.705882}%
\pgfsetstrokecolor{currentstroke}%
\pgfsetstrokeopacity{0.315151}%
\pgfsetdash{}{0pt}%
\pgfpathmoveto{\pgfqpoint{1.829179in}{3.259267in}}%
\pgfpathcurveto{\pgfqpoint{1.837415in}{3.259267in}}{\pgfqpoint{1.845315in}{3.262539in}}{\pgfqpoint{1.851139in}{3.268363in}}%
\pgfpathcurveto{\pgfqpoint{1.856963in}{3.274187in}}{\pgfqpoint{1.860235in}{3.282087in}}{\pgfqpoint{1.860235in}{3.290323in}}%
\pgfpathcurveto{\pgfqpoint{1.860235in}{3.298559in}}{\pgfqpoint{1.856963in}{3.306460in}}{\pgfqpoint{1.851139in}{3.312283in}}%
\pgfpathcurveto{\pgfqpoint{1.845315in}{3.318107in}}{\pgfqpoint{1.837415in}{3.321380in}}{\pgfqpoint{1.829179in}{3.321380in}}%
\pgfpathcurveto{\pgfqpoint{1.820942in}{3.321380in}}{\pgfqpoint{1.813042in}{3.318107in}}{\pgfqpoint{1.807218in}{3.312283in}}%
\pgfpathcurveto{\pgfqpoint{1.801394in}{3.306460in}}{\pgfqpoint{1.798122in}{3.298559in}}{\pgfqpoint{1.798122in}{3.290323in}}%
\pgfpathcurveto{\pgfqpoint{1.798122in}{3.282087in}}{\pgfqpoint{1.801394in}{3.274187in}}{\pgfqpoint{1.807218in}{3.268363in}}%
\pgfpathcurveto{\pgfqpoint{1.813042in}{3.262539in}}{\pgfqpoint{1.820942in}{3.259267in}}{\pgfqpoint{1.829179in}{3.259267in}}%
\pgfpathclose%
\pgfusepath{stroke,fill}%
\end{pgfscope}%
\begin{pgfscope}%
\pgfpathrectangle{\pgfqpoint{0.100000in}{0.220728in}}{\pgfqpoint{3.696000in}{3.696000in}}%
\pgfusepath{clip}%
\pgfsetbuttcap%
\pgfsetroundjoin%
\definecolor{currentfill}{rgb}{0.121569,0.466667,0.705882}%
\pgfsetfillcolor{currentfill}%
\pgfsetfillopacity{0.315689}%
\pgfsetlinewidth{1.003750pt}%
\definecolor{currentstroke}{rgb}{0.121569,0.466667,0.705882}%
\pgfsetstrokecolor{currentstroke}%
\pgfsetstrokeopacity{0.315689}%
\pgfsetdash{}{0pt}%
\pgfpathmoveto{\pgfqpoint{1.727562in}{3.158665in}}%
\pgfpathcurveto{\pgfqpoint{1.735798in}{3.158665in}}{\pgfqpoint{1.743698in}{3.161937in}}{\pgfqpoint{1.749522in}{3.167761in}}%
\pgfpathcurveto{\pgfqpoint{1.755346in}{3.173585in}}{\pgfqpoint{1.758618in}{3.181485in}}{\pgfqpoint{1.758618in}{3.189721in}}%
\pgfpathcurveto{\pgfqpoint{1.758618in}{3.197958in}}{\pgfqpoint{1.755346in}{3.205858in}}{\pgfqpoint{1.749522in}{3.211682in}}%
\pgfpathcurveto{\pgfqpoint{1.743698in}{3.217505in}}{\pgfqpoint{1.735798in}{3.220778in}}{\pgfqpoint{1.727562in}{3.220778in}}%
\pgfpathcurveto{\pgfqpoint{1.719325in}{3.220778in}}{\pgfqpoint{1.711425in}{3.217505in}}{\pgfqpoint{1.705601in}{3.211682in}}%
\pgfpathcurveto{\pgfqpoint{1.699777in}{3.205858in}}{\pgfqpoint{1.696505in}{3.197958in}}{\pgfqpoint{1.696505in}{3.189721in}}%
\pgfpathcurveto{\pgfqpoint{1.696505in}{3.181485in}}{\pgfqpoint{1.699777in}{3.173585in}}{\pgfqpoint{1.705601in}{3.167761in}}%
\pgfpathcurveto{\pgfqpoint{1.711425in}{3.161937in}}{\pgfqpoint{1.719325in}{3.158665in}}{\pgfqpoint{1.727562in}{3.158665in}}%
\pgfpathclose%
\pgfusepath{stroke,fill}%
\end{pgfscope}%
\begin{pgfscope}%
\pgfpathrectangle{\pgfqpoint{0.100000in}{0.220728in}}{\pgfqpoint{3.696000in}{3.696000in}}%
\pgfusepath{clip}%
\pgfsetbuttcap%
\pgfsetroundjoin%
\definecolor{currentfill}{rgb}{0.121569,0.466667,0.705882}%
\pgfsetfillcolor{currentfill}%
\pgfsetfillopacity{0.316271}%
\pgfsetlinewidth{1.003750pt}%
\definecolor{currentstroke}{rgb}{0.121569,0.466667,0.705882}%
\pgfsetstrokecolor{currentstroke}%
\pgfsetstrokeopacity{0.316271}%
\pgfsetdash{}{0pt}%
\pgfpathmoveto{\pgfqpoint{1.724530in}{3.153916in}}%
\pgfpathcurveto{\pgfqpoint{1.732767in}{3.153916in}}{\pgfqpoint{1.740667in}{3.157188in}}{\pgfqpoint{1.746491in}{3.163012in}}%
\pgfpathcurveto{\pgfqpoint{1.752314in}{3.168836in}}{\pgfqpoint{1.755587in}{3.176736in}}{\pgfqpoint{1.755587in}{3.184972in}}%
\pgfpathcurveto{\pgfqpoint{1.755587in}{3.193209in}}{\pgfqpoint{1.752314in}{3.201109in}}{\pgfqpoint{1.746491in}{3.206933in}}%
\pgfpathcurveto{\pgfqpoint{1.740667in}{3.212757in}}{\pgfqpoint{1.732767in}{3.216029in}}{\pgfqpoint{1.724530in}{3.216029in}}%
\pgfpathcurveto{\pgfqpoint{1.716294in}{3.216029in}}{\pgfqpoint{1.708394in}{3.212757in}}{\pgfqpoint{1.702570in}{3.206933in}}%
\pgfpathcurveto{\pgfqpoint{1.696746in}{3.201109in}}{\pgfqpoint{1.693474in}{3.193209in}}{\pgfqpoint{1.693474in}{3.184972in}}%
\pgfpathcurveto{\pgfqpoint{1.693474in}{3.176736in}}{\pgfqpoint{1.696746in}{3.168836in}}{\pgfqpoint{1.702570in}{3.163012in}}%
\pgfpathcurveto{\pgfqpoint{1.708394in}{3.157188in}}{\pgfqpoint{1.716294in}{3.153916in}}{\pgfqpoint{1.724530in}{3.153916in}}%
\pgfpathclose%
\pgfusepath{stroke,fill}%
\end{pgfscope}%
\begin{pgfscope}%
\pgfpathrectangle{\pgfqpoint{0.100000in}{0.220728in}}{\pgfqpoint{3.696000in}{3.696000in}}%
\pgfusepath{clip}%
\pgfsetbuttcap%
\pgfsetroundjoin%
\definecolor{currentfill}{rgb}{0.121569,0.466667,0.705882}%
\pgfsetfillcolor{currentfill}%
\pgfsetfillopacity{0.316428}%
\pgfsetlinewidth{1.003750pt}%
\definecolor{currentstroke}{rgb}{0.121569,0.466667,0.705882}%
\pgfsetstrokecolor{currentstroke}%
\pgfsetstrokeopacity{0.316428}%
\pgfsetdash{}{0pt}%
\pgfpathmoveto{\pgfqpoint{1.832916in}{3.259021in}}%
\pgfpathcurveto{\pgfqpoint{1.841152in}{3.259021in}}{\pgfqpoint{1.849053in}{3.262293in}}{\pgfqpoint{1.854876in}{3.268117in}}%
\pgfpathcurveto{\pgfqpoint{1.860700in}{3.273941in}}{\pgfqpoint{1.863973in}{3.281841in}}{\pgfqpoint{1.863973in}{3.290077in}}%
\pgfpathcurveto{\pgfqpoint{1.863973in}{3.298313in}}{\pgfqpoint{1.860700in}{3.306214in}}{\pgfqpoint{1.854876in}{3.312037in}}%
\pgfpathcurveto{\pgfqpoint{1.849053in}{3.317861in}}{\pgfqpoint{1.841152in}{3.321134in}}{\pgfqpoint{1.832916in}{3.321134in}}%
\pgfpathcurveto{\pgfqpoint{1.824680in}{3.321134in}}{\pgfqpoint{1.816780in}{3.317861in}}{\pgfqpoint{1.810956in}{3.312037in}}%
\pgfpathcurveto{\pgfqpoint{1.805132in}{3.306214in}}{\pgfqpoint{1.801860in}{3.298313in}}{\pgfqpoint{1.801860in}{3.290077in}}%
\pgfpathcurveto{\pgfqpoint{1.801860in}{3.281841in}}{\pgfqpoint{1.805132in}{3.273941in}}{\pgfqpoint{1.810956in}{3.268117in}}%
\pgfpathcurveto{\pgfqpoint{1.816780in}{3.262293in}}{\pgfqpoint{1.824680in}{3.259021in}}{\pgfqpoint{1.832916in}{3.259021in}}%
\pgfpathclose%
\pgfusepath{stroke,fill}%
\end{pgfscope}%
\begin{pgfscope}%
\pgfpathrectangle{\pgfqpoint{0.100000in}{0.220728in}}{\pgfqpoint{3.696000in}{3.696000in}}%
\pgfusepath{clip}%
\pgfsetbuttcap%
\pgfsetroundjoin%
\definecolor{currentfill}{rgb}{0.121569,0.466667,0.705882}%
\pgfsetfillcolor{currentfill}%
\pgfsetfillopacity{0.316956}%
\pgfsetlinewidth{1.003750pt}%
\definecolor{currentstroke}{rgb}{0.121569,0.466667,0.705882}%
\pgfsetstrokecolor{currentstroke}%
\pgfsetstrokeopacity{0.316956}%
\pgfsetdash{}{0pt}%
\pgfpathmoveto{\pgfqpoint{1.723144in}{3.149632in}}%
\pgfpathcurveto{\pgfqpoint{1.731380in}{3.149632in}}{\pgfqpoint{1.739281in}{3.152905in}}{\pgfqpoint{1.745104in}{3.158729in}}%
\pgfpathcurveto{\pgfqpoint{1.750928in}{3.164552in}}{\pgfqpoint{1.754201in}{3.172453in}}{\pgfqpoint{1.754201in}{3.180689in}}%
\pgfpathcurveto{\pgfqpoint{1.754201in}{3.188925in}}{\pgfqpoint{1.750928in}{3.196825in}}{\pgfqpoint{1.745104in}{3.202649in}}%
\pgfpathcurveto{\pgfqpoint{1.739281in}{3.208473in}}{\pgfqpoint{1.731380in}{3.211745in}}{\pgfqpoint{1.723144in}{3.211745in}}%
\pgfpathcurveto{\pgfqpoint{1.714908in}{3.211745in}}{\pgfqpoint{1.707008in}{3.208473in}}{\pgfqpoint{1.701184in}{3.202649in}}%
\pgfpathcurveto{\pgfqpoint{1.695360in}{3.196825in}}{\pgfqpoint{1.692088in}{3.188925in}}{\pgfqpoint{1.692088in}{3.180689in}}%
\pgfpathcurveto{\pgfqpoint{1.692088in}{3.172453in}}{\pgfqpoint{1.695360in}{3.164552in}}{\pgfqpoint{1.701184in}{3.158729in}}%
\pgfpathcurveto{\pgfqpoint{1.707008in}{3.152905in}}{\pgfqpoint{1.714908in}{3.149632in}}{\pgfqpoint{1.723144in}{3.149632in}}%
\pgfpathclose%
\pgfusepath{stroke,fill}%
\end{pgfscope}%
\begin{pgfscope}%
\pgfpathrectangle{\pgfqpoint{0.100000in}{0.220728in}}{\pgfqpoint{3.696000in}{3.696000in}}%
\pgfusepath{clip}%
\pgfsetbuttcap%
\pgfsetroundjoin%
\definecolor{currentfill}{rgb}{0.121569,0.466667,0.705882}%
\pgfsetfillcolor{currentfill}%
\pgfsetfillopacity{0.317535}%
\pgfsetlinewidth{1.003750pt}%
\definecolor{currentstroke}{rgb}{0.121569,0.466667,0.705882}%
\pgfsetstrokecolor{currentstroke}%
\pgfsetstrokeopacity{0.317535}%
\pgfsetdash{}{0pt}%
\pgfpathmoveto{\pgfqpoint{1.722237in}{3.146024in}}%
\pgfpathcurveto{\pgfqpoint{1.730473in}{3.146024in}}{\pgfqpoint{1.738373in}{3.149297in}}{\pgfqpoint{1.744197in}{3.155120in}}%
\pgfpathcurveto{\pgfqpoint{1.750021in}{3.160944in}}{\pgfqpoint{1.753293in}{3.168844in}}{\pgfqpoint{1.753293in}{3.177081in}}%
\pgfpathcurveto{\pgfqpoint{1.753293in}{3.185317in}}{\pgfqpoint{1.750021in}{3.193217in}}{\pgfqpoint{1.744197in}{3.199041in}}%
\pgfpathcurveto{\pgfqpoint{1.738373in}{3.204865in}}{\pgfqpoint{1.730473in}{3.208137in}}{\pgfqpoint{1.722237in}{3.208137in}}%
\pgfpathcurveto{\pgfqpoint{1.714000in}{3.208137in}}{\pgfqpoint{1.706100in}{3.204865in}}{\pgfqpoint{1.700276in}{3.199041in}}%
\pgfpathcurveto{\pgfqpoint{1.694452in}{3.193217in}}{\pgfqpoint{1.691180in}{3.185317in}}{\pgfqpoint{1.691180in}{3.177081in}}%
\pgfpathcurveto{\pgfqpoint{1.691180in}{3.168844in}}{\pgfqpoint{1.694452in}{3.160944in}}{\pgfqpoint{1.700276in}{3.155120in}}%
\pgfpathcurveto{\pgfqpoint{1.706100in}{3.149297in}}{\pgfqpoint{1.714000in}{3.146024in}}{\pgfqpoint{1.722237in}{3.146024in}}%
\pgfpathclose%
\pgfusepath{stroke,fill}%
\end{pgfscope}%
\begin{pgfscope}%
\pgfpathrectangle{\pgfqpoint{0.100000in}{0.220728in}}{\pgfqpoint{3.696000in}{3.696000in}}%
\pgfusepath{clip}%
\pgfsetbuttcap%
\pgfsetroundjoin%
\definecolor{currentfill}{rgb}{0.121569,0.466667,0.705882}%
\pgfsetfillcolor{currentfill}%
\pgfsetfillopacity{0.317608}%
\pgfsetlinewidth{1.003750pt}%
\definecolor{currentstroke}{rgb}{0.121569,0.466667,0.705882}%
\pgfsetstrokecolor{currentstroke}%
\pgfsetstrokeopacity{0.317608}%
\pgfsetdash{}{0pt}%
\pgfpathmoveto{\pgfqpoint{1.721782in}{3.145346in}}%
\pgfpathcurveto{\pgfqpoint{1.730018in}{3.145346in}}{\pgfqpoint{1.737918in}{3.148618in}}{\pgfqpoint{1.743742in}{3.154442in}}%
\pgfpathcurveto{\pgfqpoint{1.749566in}{3.160266in}}{\pgfqpoint{1.752838in}{3.168166in}}{\pgfqpoint{1.752838in}{3.176402in}}%
\pgfpathcurveto{\pgfqpoint{1.752838in}{3.184639in}}{\pgfqpoint{1.749566in}{3.192539in}}{\pgfqpoint{1.743742in}{3.198363in}}%
\pgfpathcurveto{\pgfqpoint{1.737918in}{3.204187in}}{\pgfqpoint{1.730018in}{3.207459in}}{\pgfqpoint{1.721782in}{3.207459in}}%
\pgfpathcurveto{\pgfqpoint{1.713546in}{3.207459in}}{\pgfqpoint{1.705645in}{3.204187in}}{\pgfqpoint{1.699822in}{3.198363in}}%
\pgfpathcurveto{\pgfqpoint{1.693998in}{3.192539in}}{\pgfqpoint{1.690725in}{3.184639in}}{\pgfqpoint{1.690725in}{3.176402in}}%
\pgfpathcurveto{\pgfqpoint{1.690725in}{3.168166in}}{\pgfqpoint{1.693998in}{3.160266in}}{\pgfqpoint{1.699822in}{3.154442in}}%
\pgfpathcurveto{\pgfqpoint{1.705645in}{3.148618in}}{\pgfqpoint{1.713546in}{3.145346in}}{\pgfqpoint{1.721782in}{3.145346in}}%
\pgfpathclose%
\pgfusepath{stroke,fill}%
\end{pgfscope}%
\begin{pgfscope}%
\pgfpathrectangle{\pgfqpoint{0.100000in}{0.220728in}}{\pgfqpoint{3.696000in}{3.696000in}}%
\pgfusepath{clip}%
\pgfsetbuttcap%
\pgfsetroundjoin%
\definecolor{currentfill}{rgb}{0.121569,0.466667,0.705882}%
\pgfsetfillcolor{currentfill}%
\pgfsetfillopacity{0.317750}%
\pgfsetlinewidth{1.003750pt}%
\definecolor{currentstroke}{rgb}{0.121569,0.466667,0.705882}%
\pgfsetstrokecolor{currentstroke}%
\pgfsetstrokeopacity{0.317750}%
\pgfsetdash{}{0pt}%
\pgfpathmoveto{\pgfqpoint{1.838595in}{3.259056in}}%
\pgfpathcurveto{\pgfqpoint{1.846832in}{3.259056in}}{\pgfqpoint{1.854732in}{3.262329in}}{\pgfqpoint{1.860556in}{3.268153in}}%
\pgfpathcurveto{\pgfqpoint{1.866380in}{3.273977in}}{\pgfqpoint{1.869652in}{3.281877in}}{\pgfqpoint{1.869652in}{3.290113in}}%
\pgfpathcurveto{\pgfqpoint{1.869652in}{3.298349in}}{\pgfqpoint{1.866380in}{3.306249in}}{\pgfqpoint{1.860556in}{3.312073in}}%
\pgfpathcurveto{\pgfqpoint{1.854732in}{3.317897in}}{\pgfqpoint{1.846832in}{3.321169in}}{\pgfqpoint{1.838595in}{3.321169in}}%
\pgfpathcurveto{\pgfqpoint{1.830359in}{3.321169in}}{\pgfqpoint{1.822459in}{3.317897in}}{\pgfqpoint{1.816635in}{3.312073in}}%
\pgfpathcurveto{\pgfqpoint{1.810811in}{3.306249in}}{\pgfqpoint{1.807539in}{3.298349in}}{\pgfqpoint{1.807539in}{3.290113in}}%
\pgfpathcurveto{\pgfqpoint{1.807539in}{3.281877in}}{\pgfqpoint{1.810811in}{3.273977in}}{\pgfqpoint{1.816635in}{3.268153in}}%
\pgfpathcurveto{\pgfqpoint{1.822459in}{3.262329in}}{\pgfqpoint{1.830359in}{3.259056in}}{\pgfqpoint{1.838595in}{3.259056in}}%
\pgfpathclose%
\pgfusepath{stroke,fill}%
\end{pgfscope}%
\begin{pgfscope}%
\pgfpathrectangle{\pgfqpoint{0.100000in}{0.220728in}}{\pgfqpoint{3.696000in}{3.696000in}}%
\pgfusepath{clip}%
\pgfsetbuttcap%
\pgfsetroundjoin%
\definecolor{currentfill}{rgb}{0.121569,0.466667,0.705882}%
\pgfsetfillcolor{currentfill}%
\pgfsetfillopacity{0.317836}%
\pgfsetlinewidth{1.003750pt}%
\definecolor{currentstroke}{rgb}{0.121569,0.466667,0.705882}%
\pgfsetstrokecolor{currentstroke}%
\pgfsetstrokeopacity{0.317836}%
\pgfsetdash{}{0pt}%
\pgfpathmoveto{\pgfqpoint{1.721352in}{3.144037in}}%
\pgfpathcurveto{\pgfqpoint{1.729588in}{3.144037in}}{\pgfqpoint{1.737488in}{3.147310in}}{\pgfqpoint{1.743312in}{3.153134in}}%
\pgfpathcurveto{\pgfqpoint{1.749136in}{3.158958in}}{\pgfqpoint{1.752408in}{3.166858in}}{\pgfqpoint{1.752408in}{3.175094in}}%
\pgfpathcurveto{\pgfqpoint{1.752408in}{3.183330in}}{\pgfqpoint{1.749136in}{3.191230in}}{\pgfqpoint{1.743312in}{3.197054in}}%
\pgfpathcurveto{\pgfqpoint{1.737488in}{3.202878in}}{\pgfqpoint{1.729588in}{3.206150in}}{\pgfqpoint{1.721352in}{3.206150in}}%
\pgfpathcurveto{\pgfqpoint{1.713116in}{3.206150in}}{\pgfqpoint{1.705216in}{3.202878in}}{\pgfqpoint{1.699392in}{3.197054in}}%
\pgfpathcurveto{\pgfqpoint{1.693568in}{3.191230in}}{\pgfqpoint{1.690295in}{3.183330in}}{\pgfqpoint{1.690295in}{3.175094in}}%
\pgfpathcurveto{\pgfqpoint{1.690295in}{3.166858in}}{\pgfqpoint{1.693568in}{3.158958in}}{\pgfqpoint{1.699392in}{3.153134in}}%
\pgfpathcurveto{\pgfqpoint{1.705216in}{3.147310in}}{\pgfqpoint{1.713116in}{3.144037in}}{\pgfqpoint{1.721352in}{3.144037in}}%
\pgfpathclose%
\pgfusepath{stroke,fill}%
\end{pgfscope}%
\begin{pgfscope}%
\pgfpathrectangle{\pgfqpoint{0.100000in}{0.220728in}}{\pgfqpoint{3.696000in}{3.696000in}}%
\pgfusepath{clip}%
\pgfsetbuttcap%
\pgfsetroundjoin%
\definecolor{currentfill}{rgb}{0.121569,0.466667,0.705882}%
\pgfsetfillcolor{currentfill}%
\pgfsetfillopacity{0.318251}%
\pgfsetlinewidth{1.003750pt}%
\definecolor{currentstroke}{rgb}{0.121569,0.466667,0.705882}%
\pgfsetstrokecolor{currentstroke}%
\pgfsetstrokeopacity{0.318251}%
\pgfsetdash{}{0pt}%
\pgfpathmoveto{\pgfqpoint{1.720423in}{3.141780in}}%
\pgfpathcurveto{\pgfqpoint{1.728659in}{3.141780in}}{\pgfqpoint{1.736559in}{3.145052in}}{\pgfqpoint{1.742383in}{3.150876in}}%
\pgfpathcurveto{\pgfqpoint{1.748207in}{3.156700in}}{\pgfqpoint{1.751479in}{3.164600in}}{\pgfqpoint{1.751479in}{3.172836in}}%
\pgfpathcurveto{\pgfqpoint{1.751479in}{3.181072in}}{\pgfqpoint{1.748207in}{3.188972in}}{\pgfqpoint{1.742383in}{3.194796in}}%
\pgfpathcurveto{\pgfqpoint{1.736559in}{3.200620in}}{\pgfqpoint{1.728659in}{3.203893in}}{\pgfqpoint{1.720423in}{3.203893in}}%
\pgfpathcurveto{\pgfqpoint{1.712187in}{3.203893in}}{\pgfqpoint{1.704287in}{3.200620in}}{\pgfqpoint{1.698463in}{3.194796in}}%
\pgfpathcurveto{\pgfqpoint{1.692639in}{3.188972in}}{\pgfqpoint{1.689366in}{3.181072in}}{\pgfqpoint{1.689366in}{3.172836in}}%
\pgfpathcurveto{\pgfqpoint{1.689366in}{3.164600in}}{\pgfqpoint{1.692639in}{3.156700in}}{\pgfqpoint{1.698463in}{3.150876in}}%
\pgfpathcurveto{\pgfqpoint{1.704287in}{3.145052in}}{\pgfqpoint{1.712187in}{3.141780in}}{\pgfqpoint{1.720423in}{3.141780in}}%
\pgfpathclose%
\pgfusepath{stroke,fill}%
\end{pgfscope}%
\begin{pgfscope}%
\pgfpathrectangle{\pgfqpoint{0.100000in}{0.220728in}}{\pgfqpoint{3.696000in}{3.696000in}}%
\pgfusepath{clip}%
\pgfsetbuttcap%
\pgfsetroundjoin%
\definecolor{currentfill}{rgb}{0.121569,0.466667,0.705882}%
\pgfsetfillcolor{currentfill}%
\pgfsetfillopacity{0.318824}%
\pgfsetlinewidth{1.003750pt}%
\definecolor{currentstroke}{rgb}{0.121569,0.466667,0.705882}%
\pgfsetstrokecolor{currentstroke}%
\pgfsetstrokeopacity{0.318824}%
\pgfsetdash{}{0pt}%
\pgfpathmoveto{\pgfqpoint{1.717928in}{3.137839in}}%
\pgfpathcurveto{\pgfqpoint{1.726164in}{3.137839in}}{\pgfqpoint{1.734064in}{3.141111in}}{\pgfqpoint{1.739888in}{3.146935in}}%
\pgfpathcurveto{\pgfqpoint{1.745712in}{3.152759in}}{\pgfqpoint{1.748985in}{3.160659in}}{\pgfqpoint{1.748985in}{3.168895in}}%
\pgfpathcurveto{\pgfqpoint{1.748985in}{3.177131in}}{\pgfqpoint{1.745712in}{3.185031in}}{\pgfqpoint{1.739888in}{3.190855in}}%
\pgfpathcurveto{\pgfqpoint{1.734064in}{3.196679in}}{\pgfqpoint{1.726164in}{3.199952in}}{\pgfqpoint{1.717928in}{3.199952in}}%
\pgfpathcurveto{\pgfqpoint{1.709692in}{3.199952in}}{\pgfqpoint{1.701792in}{3.196679in}}{\pgfqpoint{1.695968in}{3.190855in}}%
\pgfpathcurveto{\pgfqpoint{1.690144in}{3.185031in}}{\pgfqpoint{1.686872in}{3.177131in}}{\pgfqpoint{1.686872in}{3.168895in}}%
\pgfpathcurveto{\pgfqpoint{1.686872in}{3.160659in}}{\pgfqpoint{1.690144in}{3.152759in}}{\pgfqpoint{1.695968in}{3.146935in}}%
\pgfpathcurveto{\pgfqpoint{1.701792in}{3.141111in}}{\pgfqpoint{1.709692in}{3.137839in}}{\pgfqpoint{1.717928in}{3.137839in}}%
\pgfpathclose%
\pgfusepath{stroke,fill}%
\end{pgfscope}%
\begin{pgfscope}%
\pgfpathrectangle{\pgfqpoint{0.100000in}{0.220728in}}{\pgfqpoint{3.696000in}{3.696000in}}%
\pgfusepath{clip}%
\pgfsetbuttcap%
\pgfsetroundjoin%
\definecolor{currentfill}{rgb}{0.121569,0.466667,0.705882}%
\pgfsetfillcolor{currentfill}%
\pgfsetfillopacity{0.319234}%
\pgfsetlinewidth{1.003750pt}%
\definecolor{currentstroke}{rgb}{0.121569,0.466667,0.705882}%
\pgfsetstrokecolor{currentstroke}%
\pgfsetstrokeopacity{0.319234}%
\pgfsetdash{}{0pt}%
\pgfpathmoveto{\pgfqpoint{1.717592in}{3.134948in}}%
\pgfpathcurveto{\pgfqpoint{1.725828in}{3.134948in}}{\pgfqpoint{1.733728in}{3.138221in}}{\pgfqpoint{1.739552in}{3.144045in}}%
\pgfpathcurveto{\pgfqpoint{1.745376in}{3.149869in}}{\pgfqpoint{1.748648in}{3.157769in}}{\pgfqpoint{1.748648in}{3.166005in}}%
\pgfpathcurveto{\pgfqpoint{1.748648in}{3.174241in}}{\pgfqpoint{1.745376in}{3.182141in}}{\pgfqpoint{1.739552in}{3.187965in}}%
\pgfpathcurveto{\pgfqpoint{1.733728in}{3.193789in}}{\pgfqpoint{1.725828in}{3.197061in}}{\pgfqpoint{1.717592in}{3.197061in}}%
\pgfpathcurveto{\pgfqpoint{1.709355in}{3.197061in}}{\pgfqpoint{1.701455in}{3.193789in}}{\pgfqpoint{1.695631in}{3.187965in}}%
\pgfpathcurveto{\pgfqpoint{1.689807in}{3.182141in}}{\pgfqpoint{1.686535in}{3.174241in}}{\pgfqpoint{1.686535in}{3.166005in}}%
\pgfpathcurveto{\pgfqpoint{1.686535in}{3.157769in}}{\pgfqpoint{1.689807in}{3.149869in}}{\pgfqpoint{1.695631in}{3.144045in}}%
\pgfpathcurveto{\pgfqpoint{1.701455in}{3.138221in}}{\pgfqpoint{1.709355in}{3.134948in}}{\pgfqpoint{1.717592in}{3.134948in}}%
\pgfpathclose%
\pgfusepath{stroke,fill}%
\end{pgfscope}%
\begin{pgfscope}%
\pgfpathrectangle{\pgfqpoint{0.100000in}{0.220728in}}{\pgfqpoint{3.696000in}{3.696000in}}%
\pgfusepath{clip}%
\pgfsetbuttcap%
\pgfsetroundjoin%
\definecolor{currentfill}{rgb}{0.121569,0.466667,0.705882}%
\pgfsetfillcolor{currentfill}%
\pgfsetfillopacity{0.319420}%
\pgfsetlinewidth{1.003750pt}%
\definecolor{currentstroke}{rgb}{0.121569,0.466667,0.705882}%
\pgfsetstrokecolor{currentstroke}%
\pgfsetstrokeopacity{0.319420}%
\pgfsetdash{}{0pt}%
\pgfpathmoveto{\pgfqpoint{1.716551in}{3.133462in}}%
\pgfpathcurveto{\pgfqpoint{1.724787in}{3.133462in}}{\pgfqpoint{1.732687in}{3.136734in}}{\pgfqpoint{1.738511in}{3.142558in}}%
\pgfpathcurveto{\pgfqpoint{1.744335in}{3.148382in}}{\pgfqpoint{1.747608in}{3.156282in}}{\pgfqpoint{1.747608in}{3.164518in}}%
\pgfpathcurveto{\pgfqpoint{1.747608in}{3.172755in}}{\pgfqpoint{1.744335in}{3.180655in}}{\pgfqpoint{1.738511in}{3.186479in}}%
\pgfpathcurveto{\pgfqpoint{1.732687in}{3.192302in}}{\pgfqpoint{1.724787in}{3.195575in}}{\pgfqpoint{1.716551in}{3.195575in}}%
\pgfpathcurveto{\pgfqpoint{1.708315in}{3.195575in}}{\pgfqpoint{1.700415in}{3.192302in}}{\pgfqpoint{1.694591in}{3.186479in}}%
\pgfpathcurveto{\pgfqpoint{1.688767in}{3.180655in}}{\pgfqpoint{1.685495in}{3.172755in}}{\pgfqpoint{1.685495in}{3.164518in}}%
\pgfpathcurveto{\pgfqpoint{1.685495in}{3.156282in}}{\pgfqpoint{1.688767in}{3.148382in}}{\pgfqpoint{1.694591in}{3.142558in}}%
\pgfpathcurveto{\pgfqpoint{1.700415in}{3.136734in}}{\pgfqpoint{1.708315in}{3.133462in}}{\pgfqpoint{1.716551in}{3.133462in}}%
\pgfpathclose%
\pgfusepath{stroke,fill}%
\end{pgfscope}%
\begin{pgfscope}%
\pgfpathrectangle{\pgfqpoint{0.100000in}{0.220728in}}{\pgfqpoint{3.696000in}{3.696000in}}%
\pgfusepath{clip}%
\pgfsetbuttcap%
\pgfsetroundjoin%
\definecolor{currentfill}{rgb}{0.121569,0.466667,0.705882}%
\pgfsetfillcolor{currentfill}%
\pgfsetfillopacity{0.319920}%
\pgfsetlinewidth{1.003750pt}%
\definecolor{currentstroke}{rgb}{0.121569,0.466667,0.705882}%
\pgfsetstrokecolor{currentstroke}%
\pgfsetstrokeopacity{0.319920}%
\pgfsetdash{}{0pt}%
\pgfpathmoveto{\pgfqpoint{1.715377in}{3.130511in}}%
\pgfpathcurveto{\pgfqpoint{1.723613in}{3.130511in}}{\pgfqpoint{1.731513in}{3.133783in}}{\pgfqpoint{1.737337in}{3.139607in}}%
\pgfpathcurveto{\pgfqpoint{1.743161in}{3.145431in}}{\pgfqpoint{1.746433in}{3.153331in}}{\pgfqpoint{1.746433in}{3.161567in}}%
\pgfpathcurveto{\pgfqpoint{1.746433in}{3.169804in}}{\pgfqpoint{1.743161in}{3.177704in}}{\pgfqpoint{1.737337in}{3.183528in}}%
\pgfpathcurveto{\pgfqpoint{1.731513in}{3.189352in}}{\pgfqpoint{1.723613in}{3.192624in}}{\pgfqpoint{1.715377in}{3.192624in}}%
\pgfpathcurveto{\pgfqpoint{1.707141in}{3.192624in}}{\pgfqpoint{1.699241in}{3.189352in}}{\pgfqpoint{1.693417in}{3.183528in}}%
\pgfpathcurveto{\pgfqpoint{1.687593in}{3.177704in}}{\pgfqpoint{1.684320in}{3.169804in}}{\pgfqpoint{1.684320in}{3.161567in}}%
\pgfpathcurveto{\pgfqpoint{1.684320in}{3.153331in}}{\pgfqpoint{1.687593in}{3.145431in}}{\pgfqpoint{1.693417in}{3.139607in}}%
\pgfpathcurveto{\pgfqpoint{1.699241in}{3.133783in}}{\pgfqpoint{1.707141in}{3.130511in}}{\pgfqpoint{1.715377in}{3.130511in}}%
\pgfpathclose%
\pgfusepath{stroke,fill}%
\end{pgfscope}%
\begin{pgfscope}%
\pgfpathrectangle{\pgfqpoint{0.100000in}{0.220728in}}{\pgfqpoint{3.696000in}{3.696000in}}%
\pgfusepath{clip}%
\pgfsetbuttcap%
\pgfsetroundjoin%
\definecolor{currentfill}{rgb}{0.121569,0.466667,0.705882}%
\pgfsetfillcolor{currentfill}%
\pgfsetfillopacity{0.320754}%
\pgfsetlinewidth{1.003750pt}%
\definecolor{currentstroke}{rgb}{0.121569,0.466667,0.705882}%
\pgfsetstrokecolor{currentstroke}%
\pgfsetstrokeopacity{0.320754}%
\pgfsetdash{}{0pt}%
\pgfpathmoveto{\pgfqpoint{1.846310in}{3.258468in}}%
\pgfpathcurveto{\pgfqpoint{1.854546in}{3.258468in}}{\pgfqpoint{1.862446in}{3.261740in}}{\pgfqpoint{1.868270in}{3.267564in}}%
\pgfpathcurveto{\pgfqpoint{1.874094in}{3.273388in}}{\pgfqpoint{1.877367in}{3.281288in}}{\pgfqpoint{1.877367in}{3.289524in}}%
\pgfpathcurveto{\pgfqpoint{1.877367in}{3.297760in}}{\pgfqpoint{1.874094in}{3.305660in}}{\pgfqpoint{1.868270in}{3.311484in}}%
\pgfpathcurveto{\pgfqpoint{1.862446in}{3.317308in}}{\pgfqpoint{1.854546in}{3.320581in}}{\pgfqpoint{1.846310in}{3.320581in}}%
\pgfpathcurveto{\pgfqpoint{1.838074in}{3.320581in}}{\pgfqpoint{1.830174in}{3.317308in}}{\pgfqpoint{1.824350in}{3.311484in}}%
\pgfpathcurveto{\pgfqpoint{1.818526in}{3.305660in}}{\pgfqpoint{1.815254in}{3.297760in}}{\pgfqpoint{1.815254in}{3.289524in}}%
\pgfpathcurveto{\pgfqpoint{1.815254in}{3.281288in}}{\pgfqpoint{1.818526in}{3.273388in}}{\pgfqpoint{1.824350in}{3.267564in}}%
\pgfpathcurveto{\pgfqpoint{1.830174in}{3.261740in}}{\pgfqpoint{1.838074in}{3.258468in}}{\pgfqpoint{1.846310in}{3.258468in}}%
\pgfpathclose%
\pgfusepath{stroke,fill}%
\end{pgfscope}%
\begin{pgfscope}%
\pgfpathrectangle{\pgfqpoint{0.100000in}{0.220728in}}{\pgfqpoint{3.696000in}{3.696000in}}%
\pgfusepath{clip}%
\pgfsetbuttcap%
\pgfsetroundjoin%
\definecolor{currentfill}{rgb}{0.121569,0.466667,0.705882}%
\pgfsetfillcolor{currentfill}%
\pgfsetfillopacity{0.320836}%
\pgfsetlinewidth{1.003750pt}%
\definecolor{currentstroke}{rgb}{0.121569,0.466667,0.705882}%
\pgfsetstrokecolor{currentstroke}%
\pgfsetstrokeopacity{0.320836}%
\pgfsetdash{}{0pt}%
\pgfpathmoveto{\pgfqpoint{1.713875in}{3.124748in}}%
\pgfpathcurveto{\pgfqpoint{1.722111in}{3.124748in}}{\pgfqpoint{1.730011in}{3.128021in}}{\pgfqpoint{1.735835in}{3.133845in}}%
\pgfpathcurveto{\pgfqpoint{1.741659in}{3.139669in}}{\pgfqpoint{1.744931in}{3.147569in}}{\pgfqpoint{1.744931in}{3.155805in}}%
\pgfpathcurveto{\pgfqpoint{1.744931in}{3.164041in}}{\pgfqpoint{1.741659in}{3.171941in}}{\pgfqpoint{1.735835in}{3.177765in}}%
\pgfpathcurveto{\pgfqpoint{1.730011in}{3.183589in}}{\pgfqpoint{1.722111in}{3.186861in}}{\pgfqpoint{1.713875in}{3.186861in}}%
\pgfpathcurveto{\pgfqpoint{1.705639in}{3.186861in}}{\pgfqpoint{1.697739in}{3.183589in}}{\pgfqpoint{1.691915in}{3.177765in}}%
\pgfpathcurveto{\pgfqpoint{1.686091in}{3.171941in}}{\pgfqpoint{1.682818in}{3.164041in}}{\pgfqpoint{1.682818in}{3.155805in}}%
\pgfpathcurveto{\pgfqpoint{1.682818in}{3.147569in}}{\pgfqpoint{1.686091in}{3.139669in}}{\pgfqpoint{1.691915in}{3.133845in}}%
\pgfpathcurveto{\pgfqpoint{1.697739in}{3.128021in}}{\pgfqpoint{1.705639in}{3.124748in}}{\pgfqpoint{1.713875in}{3.124748in}}%
\pgfpathclose%
\pgfusepath{stroke,fill}%
\end{pgfscope}%
\begin{pgfscope}%
\pgfpathrectangle{\pgfqpoint{0.100000in}{0.220728in}}{\pgfqpoint{3.696000in}{3.696000in}}%
\pgfusepath{clip}%
\pgfsetbuttcap%
\pgfsetroundjoin%
\definecolor{currentfill}{rgb}{0.121569,0.466667,0.705882}%
\pgfsetfillcolor{currentfill}%
\pgfsetfillopacity{0.321412}%
\pgfsetlinewidth{1.003750pt}%
\definecolor{currentstroke}{rgb}{0.121569,0.466667,0.705882}%
\pgfsetstrokecolor{currentstroke}%
\pgfsetstrokeopacity{0.321412}%
\pgfsetdash{}{0pt}%
\pgfpathmoveto{\pgfqpoint{1.711421in}{3.120919in}}%
\pgfpathcurveto{\pgfqpoint{1.719657in}{3.120919in}}{\pgfqpoint{1.727557in}{3.124191in}}{\pgfqpoint{1.733381in}{3.130015in}}%
\pgfpathcurveto{\pgfqpoint{1.739205in}{3.135839in}}{\pgfqpoint{1.742477in}{3.143739in}}{\pgfqpoint{1.742477in}{3.151975in}}%
\pgfpathcurveto{\pgfqpoint{1.742477in}{3.160212in}}{\pgfqpoint{1.739205in}{3.168112in}}{\pgfqpoint{1.733381in}{3.173936in}}%
\pgfpathcurveto{\pgfqpoint{1.727557in}{3.179760in}}{\pgfqpoint{1.719657in}{3.183032in}}{\pgfqpoint{1.711421in}{3.183032in}}%
\pgfpathcurveto{\pgfqpoint{1.703184in}{3.183032in}}{\pgfqpoint{1.695284in}{3.179760in}}{\pgfqpoint{1.689460in}{3.173936in}}%
\pgfpathcurveto{\pgfqpoint{1.683637in}{3.168112in}}{\pgfqpoint{1.680364in}{3.160212in}}{\pgfqpoint{1.680364in}{3.151975in}}%
\pgfpathcurveto{\pgfqpoint{1.680364in}{3.143739in}}{\pgfqpoint{1.683637in}{3.135839in}}{\pgfqpoint{1.689460in}{3.130015in}}%
\pgfpathcurveto{\pgfqpoint{1.695284in}{3.124191in}}{\pgfqpoint{1.703184in}{3.120919in}}{\pgfqpoint{1.711421in}{3.120919in}}%
\pgfpathclose%
\pgfusepath{stroke,fill}%
\end{pgfscope}%
\begin{pgfscope}%
\pgfpathrectangle{\pgfqpoint{0.100000in}{0.220728in}}{\pgfqpoint{3.696000in}{3.696000in}}%
\pgfusepath{clip}%
\pgfsetbuttcap%
\pgfsetroundjoin%
\definecolor{currentfill}{rgb}{0.121569,0.466667,0.705882}%
\pgfsetfillcolor{currentfill}%
\pgfsetfillopacity{0.322022}%
\pgfsetlinewidth{1.003750pt}%
\definecolor{currentstroke}{rgb}{0.121569,0.466667,0.705882}%
\pgfsetstrokecolor{currentstroke}%
\pgfsetstrokeopacity{0.322022}%
\pgfsetdash{}{0pt}%
\pgfpathmoveto{\pgfqpoint{1.710692in}{3.116952in}}%
\pgfpathcurveto{\pgfqpoint{1.718928in}{3.116952in}}{\pgfqpoint{1.726828in}{3.120224in}}{\pgfqpoint{1.732652in}{3.126048in}}%
\pgfpathcurveto{\pgfqpoint{1.738476in}{3.131872in}}{\pgfqpoint{1.741749in}{3.139772in}}{\pgfqpoint{1.741749in}{3.148009in}}%
\pgfpathcurveto{\pgfqpoint{1.741749in}{3.156245in}}{\pgfqpoint{1.738476in}{3.164145in}}{\pgfqpoint{1.732652in}{3.169969in}}%
\pgfpathcurveto{\pgfqpoint{1.726828in}{3.175793in}}{\pgfqpoint{1.718928in}{3.179065in}}{\pgfqpoint{1.710692in}{3.179065in}}%
\pgfpathcurveto{\pgfqpoint{1.702456in}{3.179065in}}{\pgfqpoint{1.694556in}{3.175793in}}{\pgfqpoint{1.688732in}{3.169969in}}%
\pgfpathcurveto{\pgfqpoint{1.682908in}{3.164145in}}{\pgfqpoint{1.679636in}{3.156245in}}{\pgfqpoint{1.679636in}{3.148009in}}%
\pgfpathcurveto{\pgfqpoint{1.679636in}{3.139772in}}{\pgfqpoint{1.682908in}{3.131872in}}{\pgfqpoint{1.688732in}{3.126048in}}%
\pgfpathcurveto{\pgfqpoint{1.694556in}{3.120224in}}{\pgfqpoint{1.702456in}{3.116952in}}{\pgfqpoint{1.710692in}{3.116952in}}%
\pgfpathclose%
\pgfusepath{stroke,fill}%
\end{pgfscope}%
\begin{pgfscope}%
\pgfpathrectangle{\pgfqpoint{0.100000in}{0.220728in}}{\pgfqpoint{3.696000in}{3.696000in}}%
\pgfusepath{clip}%
\pgfsetbuttcap%
\pgfsetroundjoin%
\definecolor{currentfill}{rgb}{0.121569,0.466667,0.705882}%
\pgfsetfillcolor{currentfill}%
\pgfsetfillopacity{0.322209}%
\pgfsetlinewidth{1.003750pt}%
\definecolor{currentstroke}{rgb}{0.121569,0.466667,0.705882}%
\pgfsetstrokecolor{currentstroke}%
\pgfsetstrokeopacity{0.322209}%
\pgfsetdash{}{0pt}%
\pgfpathmoveto{\pgfqpoint{1.709796in}{3.115647in}}%
\pgfpathcurveto{\pgfqpoint{1.718033in}{3.115647in}}{\pgfqpoint{1.725933in}{3.118919in}}{\pgfqpoint{1.731757in}{3.124743in}}%
\pgfpathcurveto{\pgfqpoint{1.737581in}{3.130567in}}{\pgfqpoint{1.740853in}{3.138467in}}{\pgfqpoint{1.740853in}{3.146703in}}%
\pgfpathcurveto{\pgfqpoint{1.740853in}{3.154940in}}{\pgfqpoint{1.737581in}{3.162840in}}{\pgfqpoint{1.731757in}{3.168664in}}%
\pgfpathcurveto{\pgfqpoint{1.725933in}{3.174488in}}{\pgfqpoint{1.718033in}{3.177760in}}{\pgfqpoint{1.709796in}{3.177760in}}%
\pgfpathcurveto{\pgfqpoint{1.701560in}{3.177760in}}{\pgfqpoint{1.693660in}{3.174488in}}{\pgfqpoint{1.687836in}{3.168664in}}%
\pgfpathcurveto{\pgfqpoint{1.682012in}{3.162840in}}{\pgfqpoint{1.678740in}{3.154940in}}{\pgfqpoint{1.678740in}{3.146703in}}%
\pgfpathcurveto{\pgfqpoint{1.678740in}{3.138467in}}{\pgfqpoint{1.682012in}{3.130567in}}{\pgfqpoint{1.687836in}{3.124743in}}%
\pgfpathcurveto{\pgfqpoint{1.693660in}{3.118919in}}{\pgfqpoint{1.701560in}{3.115647in}}{\pgfqpoint{1.709796in}{3.115647in}}%
\pgfpathclose%
\pgfusepath{stroke,fill}%
\end{pgfscope}%
\begin{pgfscope}%
\pgfpathrectangle{\pgfqpoint{0.100000in}{0.220728in}}{\pgfqpoint{3.696000in}{3.696000in}}%
\pgfusepath{clip}%
\pgfsetbuttcap%
\pgfsetroundjoin%
\definecolor{currentfill}{rgb}{0.121569,0.466667,0.705882}%
\pgfsetfillcolor{currentfill}%
\pgfsetfillopacity{0.322362}%
\pgfsetlinewidth{1.003750pt}%
\definecolor{currentstroke}{rgb}{0.121569,0.466667,0.705882}%
\pgfsetstrokecolor{currentstroke}%
\pgfsetstrokeopacity{0.322362}%
\pgfsetdash{}{0pt}%
\pgfpathmoveto{\pgfqpoint{1.709486in}{3.114808in}}%
\pgfpathcurveto{\pgfqpoint{1.717723in}{3.114808in}}{\pgfqpoint{1.725623in}{3.118080in}}{\pgfqpoint{1.731447in}{3.123904in}}%
\pgfpathcurveto{\pgfqpoint{1.737271in}{3.129728in}}{\pgfqpoint{1.740543in}{3.137628in}}{\pgfqpoint{1.740543in}{3.145865in}}%
\pgfpathcurveto{\pgfqpoint{1.740543in}{3.154101in}}{\pgfqpoint{1.737271in}{3.162001in}}{\pgfqpoint{1.731447in}{3.167825in}}%
\pgfpathcurveto{\pgfqpoint{1.725623in}{3.173649in}}{\pgfqpoint{1.717723in}{3.176921in}}{\pgfqpoint{1.709486in}{3.176921in}}%
\pgfpathcurveto{\pgfqpoint{1.701250in}{3.176921in}}{\pgfqpoint{1.693350in}{3.173649in}}{\pgfqpoint{1.687526in}{3.167825in}}%
\pgfpathcurveto{\pgfqpoint{1.681702in}{3.162001in}}{\pgfqpoint{1.678430in}{3.154101in}}{\pgfqpoint{1.678430in}{3.145865in}}%
\pgfpathcurveto{\pgfqpoint{1.678430in}{3.137628in}}{\pgfqpoint{1.681702in}{3.129728in}}{\pgfqpoint{1.687526in}{3.123904in}}%
\pgfpathcurveto{\pgfqpoint{1.693350in}{3.118080in}}{\pgfqpoint{1.701250in}{3.114808in}}{\pgfqpoint{1.709486in}{3.114808in}}%
\pgfpathclose%
\pgfusepath{stroke,fill}%
\end{pgfscope}%
\begin{pgfscope}%
\pgfpathrectangle{\pgfqpoint{0.100000in}{0.220728in}}{\pgfqpoint{3.696000in}{3.696000in}}%
\pgfusepath{clip}%
\pgfsetbuttcap%
\pgfsetroundjoin%
\definecolor{currentfill}{rgb}{0.121569,0.466667,0.705882}%
\pgfsetfillcolor{currentfill}%
\pgfsetfillopacity{0.322388}%
\pgfsetlinewidth{1.003750pt}%
\definecolor{currentstroke}{rgb}{0.121569,0.466667,0.705882}%
\pgfsetstrokecolor{currentstroke}%
\pgfsetstrokeopacity{0.322388}%
\pgfsetdash{}{0pt}%
\pgfpathmoveto{\pgfqpoint{1.709435in}{3.114654in}}%
\pgfpathcurveto{\pgfqpoint{1.717671in}{3.114654in}}{\pgfqpoint{1.725572in}{3.117927in}}{\pgfqpoint{1.731395in}{3.123751in}}%
\pgfpathcurveto{\pgfqpoint{1.737219in}{3.129575in}}{\pgfqpoint{1.740492in}{3.137475in}}{\pgfqpoint{1.740492in}{3.145711in}}%
\pgfpathcurveto{\pgfqpoint{1.740492in}{3.153947in}}{\pgfqpoint{1.737219in}{3.161847in}}{\pgfqpoint{1.731395in}{3.167671in}}%
\pgfpathcurveto{\pgfqpoint{1.725572in}{3.173495in}}{\pgfqpoint{1.717671in}{3.176767in}}{\pgfqpoint{1.709435in}{3.176767in}}%
\pgfpathcurveto{\pgfqpoint{1.701199in}{3.176767in}}{\pgfqpoint{1.693299in}{3.173495in}}{\pgfqpoint{1.687475in}{3.167671in}}%
\pgfpathcurveto{\pgfqpoint{1.681651in}{3.161847in}}{\pgfqpoint{1.678379in}{3.153947in}}{\pgfqpoint{1.678379in}{3.145711in}}%
\pgfpathcurveto{\pgfqpoint{1.678379in}{3.137475in}}{\pgfqpoint{1.681651in}{3.129575in}}{\pgfqpoint{1.687475in}{3.123751in}}%
\pgfpathcurveto{\pgfqpoint{1.693299in}{3.117927in}}{\pgfqpoint{1.701199in}{3.114654in}}{\pgfqpoint{1.709435in}{3.114654in}}%
\pgfpathclose%
\pgfusepath{stroke,fill}%
\end{pgfscope}%
\begin{pgfscope}%
\pgfpathrectangle{\pgfqpoint{0.100000in}{0.220728in}}{\pgfqpoint{3.696000in}{3.696000in}}%
\pgfusepath{clip}%
\pgfsetbuttcap%
\pgfsetroundjoin%
\definecolor{currentfill}{rgb}{0.121569,0.466667,0.705882}%
\pgfsetfillcolor{currentfill}%
\pgfsetfillopacity{0.322433}%
\pgfsetlinewidth{1.003750pt}%
\definecolor{currentstroke}{rgb}{0.121569,0.466667,0.705882}%
\pgfsetstrokecolor{currentstroke}%
\pgfsetstrokeopacity{0.322433}%
\pgfsetdash{}{0pt}%
\pgfpathmoveto{\pgfqpoint{1.709319in}{3.114391in}}%
\pgfpathcurveto{\pgfqpoint{1.717555in}{3.114391in}}{\pgfqpoint{1.725455in}{3.117664in}}{\pgfqpoint{1.731279in}{3.123488in}}%
\pgfpathcurveto{\pgfqpoint{1.737103in}{3.129311in}}{\pgfqpoint{1.740375in}{3.137212in}}{\pgfqpoint{1.740375in}{3.145448in}}%
\pgfpathcurveto{\pgfqpoint{1.740375in}{3.153684in}}{\pgfqpoint{1.737103in}{3.161584in}}{\pgfqpoint{1.731279in}{3.167408in}}%
\pgfpathcurveto{\pgfqpoint{1.725455in}{3.173232in}}{\pgfqpoint{1.717555in}{3.176504in}}{\pgfqpoint{1.709319in}{3.176504in}}%
\pgfpathcurveto{\pgfqpoint{1.701082in}{3.176504in}}{\pgfqpoint{1.693182in}{3.173232in}}{\pgfqpoint{1.687359in}{3.167408in}}%
\pgfpathcurveto{\pgfqpoint{1.681535in}{3.161584in}}{\pgfqpoint{1.678262in}{3.153684in}}{\pgfqpoint{1.678262in}{3.145448in}}%
\pgfpathcurveto{\pgfqpoint{1.678262in}{3.137212in}}{\pgfqpoint{1.681535in}{3.129311in}}{\pgfqpoint{1.687359in}{3.123488in}}%
\pgfpathcurveto{\pgfqpoint{1.693182in}{3.117664in}}{\pgfqpoint{1.701082in}{3.114391in}}{\pgfqpoint{1.709319in}{3.114391in}}%
\pgfpathclose%
\pgfusepath{stroke,fill}%
\end{pgfscope}%
\begin{pgfscope}%
\pgfpathrectangle{\pgfqpoint{0.100000in}{0.220728in}}{\pgfqpoint{3.696000in}{3.696000in}}%
\pgfusepath{clip}%
\pgfsetbuttcap%
\pgfsetroundjoin%
\definecolor{currentfill}{rgb}{0.121569,0.466667,0.705882}%
\pgfsetfillcolor{currentfill}%
\pgfsetfillopacity{0.322512}%
\pgfsetlinewidth{1.003750pt}%
\definecolor{currentstroke}{rgb}{0.121569,0.466667,0.705882}%
\pgfsetstrokecolor{currentstroke}%
\pgfsetstrokeopacity{0.322512}%
\pgfsetdash{}{0pt}%
\pgfpathmoveto{\pgfqpoint{1.709120in}{3.113891in}}%
\pgfpathcurveto{\pgfqpoint{1.717356in}{3.113891in}}{\pgfqpoint{1.725256in}{3.117163in}}{\pgfqpoint{1.731080in}{3.122987in}}%
\pgfpathcurveto{\pgfqpoint{1.736904in}{3.128811in}}{\pgfqpoint{1.740176in}{3.136711in}}{\pgfqpoint{1.740176in}{3.144947in}}%
\pgfpathcurveto{\pgfqpoint{1.740176in}{3.153184in}}{\pgfqpoint{1.736904in}{3.161084in}}{\pgfqpoint{1.731080in}{3.166908in}}%
\pgfpathcurveto{\pgfqpoint{1.725256in}{3.172732in}}{\pgfqpoint{1.717356in}{3.176004in}}{\pgfqpoint{1.709120in}{3.176004in}}%
\pgfpathcurveto{\pgfqpoint{1.700883in}{3.176004in}}{\pgfqpoint{1.692983in}{3.172732in}}{\pgfqpoint{1.687159in}{3.166908in}}%
\pgfpathcurveto{\pgfqpoint{1.681335in}{3.161084in}}{\pgfqpoint{1.678063in}{3.153184in}}{\pgfqpoint{1.678063in}{3.144947in}}%
\pgfpathcurveto{\pgfqpoint{1.678063in}{3.136711in}}{\pgfqpoint{1.681335in}{3.128811in}}{\pgfqpoint{1.687159in}{3.122987in}}%
\pgfpathcurveto{\pgfqpoint{1.692983in}{3.117163in}}{\pgfqpoint{1.700883in}{3.113891in}}{\pgfqpoint{1.709120in}{3.113891in}}%
\pgfpathclose%
\pgfusepath{stroke,fill}%
\end{pgfscope}%
\begin{pgfscope}%
\pgfpathrectangle{\pgfqpoint{0.100000in}{0.220728in}}{\pgfqpoint{3.696000in}{3.696000in}}%
\pgfusepath{clip}%
\pgfsetbuttcap%
\pgfsetroundjoin%
\definecolor{currentfill}{rgb}{0.121569,0.466667,0.705882}%
\pgfsetfillcolor{currentfill}%
\pgfsetfillopacity{0.322611}%
\pgfsetlinewidth{1.003750pt}%
\definecolor{currentstroke}{rgb}{0.121569,0.466667,0.705882}%
\pgfsetstrokecolor{currentstroke}%
\pgfsetstrokeopacity{0.322611}%
\pgfsetdash{}{0pt}%
\pgfpathmoveto{\pgfqpoint{1.856269in}{3.256420in}}%
\pgfpathcurveto{\pgfqpoint{1.864505in}{3.256420in}}{\pgfqpoint{1.872405in}{3.259692in}}{\pgfqpoint{1.878229in}{3.265516in}}%
\pgfpathcurveto{\pgfqpoint{1.884053in}{3.271340in}}{\pgfqpoint{1.887325in}{3.279240in}}{\pgfqpoint{1.887325in}{3.287476in}}%
\pgfpathcurveto{\pgfqpoint{1.887325in}{3.295713in}}{\pgfqpoint{1.884053in}{3.303613in}}{\pgfqpoint{1.878229in}{3.309437in}}%
\pgfpathcurveto{\pgfqpoint{1.872405in}{3.315261in}}{\pgfqpoint{1.864505in}{3.318533in}}{\pgfqpoint{1.856269in}{3.318533in}}%
\pgfpathcurveto{\pgfqpoint{1.848032in}{3.318533in}}{\pgfqpoint{1.840132in}{3.315261in}}{\pgfqpoint{1.834308in}{3.309437in}}%
\pgfpathcurveto{\pgfqpoint{1.828484in}{3.303613in}}{\pgfqpoint{1.825212in}{3.295713in}}{\pgfqpoint{1.825212in}{3.287476in}}%
\pgfpathcurveto{\pgfqpoint{1.825212in}{3.279240in}}{\pgfqpoint{1.828484in}{3.271340in}}{\pgfqpoint{1.834308in}{3.265516in}}%
\pgfpathcurveto{\pgfqpoint{1.840132in}{3.259692in}}{\pgfqpoint{1.848032in}{3.256420in}}{\pgfqpoint{1.856269in}{3.256420in}}%
\pgfpathclose%
\pgfusepath{stroke,fill}%
\end{pgfscope}%
\begin{pgfscope}%
\pgfpathrectangle{\pgfqpoint{0.100000in}{0.220728in}}{\pgfqpoint{3.696000in}{3.696000in}}%
\pgfusepath{clip}%
\pgfsetbuttcap%
\pgfsetroundjoin%
\definecolor{currentfill}{rgb}{0.121569,0.466667,0.705882}%
\pgfsetfillcolor{currentfill}%
\pgfsetfillopacity{0.322668}%
\pgfsetlinewidth{1.003750pt}%
\definecolor{currentstroke}{rgb}{0.121569,0.466667,0.705882}%
\pgfsetstrokecolor{currentstroke}%
\pgfsetstrokeopacity{0.322668}%
\pgfsetdash{}{0pt}%
\pgfpathmoveto{\pgfqpoint{1.708751in}{3.113036in}}%
\pgfpathcurveto{\pgfqpoint{1.716987in}{3.113036in}}{\pgfqpoint{1.724887in}{3.116308in}}{\pgfqpoint{1.730711in}{3.122132in}}%
\pgfpathcurveto{\pgfqpoint{1.736535in}{3.127956in}}{\pgfqpoint{1.739807in}{3.135856in}}{\pgfqpoint{1.739807in}{3.144092in}}%
\pgfpathcurveto{\pgfqpoint{1.739807in}{3.152328in}}{\pgfqpoint{1.736535in}{3.160228in}}{\pgfqpoint{1.730711in}{3.166052in}}%
\pgfpathcurveto{\pgfqpoint{1.724887in}{3.171876in}}{\pgfqpoint{1.716987in}{3.175149in}}{\pgfqpoint{1.708751in}{3.175149in}}%
\pgfpathcurveto{\pgfqpoint{1.700514in}{3.175149in}}{\pgfqpoint{1.692614in}{3.171876in}}{\pgfqpoint{1.686790in}{3.166052in}}%
\pgfpathcurveto{\pgfqpoint{1.680966in}{3.160228in}}{\pgfqpoint{1.677694in}{3.152328in}}{\pgfqpoint{1.677694in}{3.144092in}}%
\pgfpathcurveto{\pgfqpoint{1.677694in}{3.135856in}}{\pgfqpoint{1.680966in}{3.127956in}}{\pgfqpoint{1.686790in}{3.122132in}}%
\pgfpathcurveto{\pgfqpoint{1.692614in}{3.116308in}}{\pgfqpoint{1.700514in}{3.113036in}}{\pgfqpoint{1.708751in}{3.113036in}}%
\pgfpathclose%
\pgfusepath{stroke,fill}%
\end{pgfscope}%
\begin{pgfscope}%
\pgfpathrectangle{\pgfqpoint{0.100000in}{0.220728in}}{\pgfqpoint{3.696000in}{3.696000in}}%
\pgfusepath{clip}%
\pgfsetbuttcap%
\pgfsetroundjoin%
\definecolor{currentfill}{rgb}{0.121569,0.466667,0.705882}%
\pgfsetfillcolor{currentfill}%
\pgfsetfillopacity{0.322907}%
\pgfsetlinewidth{1.003750pt}%
\definecolor{currentstroke}{rgb}{0.121569,0.466667,0.705882}%
\pgfsetstrokecolor{currentstroke}%
\pgfsetstrokeopacity{0.322907}%
\pgfsetdash{}{0pt}%
\pgfpathmoveto{\pgfqpoint{1.707988in}{3.111380in}}%
\pgfpathcurveto{\pgfqpoint{1.716224in}{3.111380in}}{\pgfqpoint{1.724124in}{3.114653in}}{\pgfqpoint{1.729948in}{3.120477in}}%
\pgfpathcurveto{\pgfqpoint{1.735772in}{3.126300in}}{\pgfqpoint{1.739044in}{3.134201in}}{\pgfqpoint{1.739044in}{3.142437in}}%
\pgfpathcurveto{\pgfqpoint{1.739044in}{3.150673in}}{\pgfqpoint{1.735772in}{3.158573in}}{\pgfqpoint{1.729948in}{3.164397in}}%
\pgfpathcurveto{\pgfqpoint{1.724124in}{3.170221in}}{\pgfqpoint{1.716224in}{3.173493in}}{\pgfqpoint{1.707988in}{3.173493in}}%
\pgfpathcurveto{\pgfqpoint{1.699752in}{3.173493in}}{\pgfqpoint{1.691852in}{3.170221in}}{\pgfqpoint{1.686028in}{3.164397in}}%
\pgfpathcurveto{\pgfqpoint{1.680204in}{3.158573in}}{\pgfqpoint{1.676931in}{3.150673in}}{\pgfqpoint{1.676931in}{3.142437in}}%
\pgfpathcurveto{\pgfqpoint{1.676931in}{3.134201in}}{\pgfqpoint{1.680204in}{3.126300in}}{\pgfqpoint{1.686028in}{3.120477in}}%
\pgfpathcurveto{\pgfqpoint{1.691852in}{3.114653in}}{\pgfqpoint{1.699752in}{3.111380in}}{\pgfqpoint{1.707988in}{3.111380in}}%
\pgfpathclose%
\pgfusepath{stroke,fill}%
\end{pgfscope}%
\begin{pgfscope}%
\pgfpathrectangle{\pgfqpoint{0.100000in}{0.220728in}}{\pgfqpoint{3.696000in}{3.696000in}}%
\pgfusepath{clip}%
\pgfsetbuttcap%
\pgfsetroundjoin%
\definecolor{currentfill}{rgb}{0.121569,0.466667,0.705882}%
\pgfsetfillcolor{currentfill}%
\pgfsetfillopacity{0.323404}%
\pgfsetlinewidth{1.003750pt}%
\definecolor{currentstroke}{rgb}{0.121569,0.466667,0.705882}%
\pgfsetstrokecolor{currentstroke}%
\pgfsetstrokeopacity{0.323404}%
\pgfsetdash{}{0pt}%
\pgfpathmoveto{\pgfqpoint{1.707579in}{3.108068in}}%
\pgfpathcurveto{\pgfqpoint{1.715816in}{3.108068in}}{\pgfqpoint{1.723716in}{3.111340in}}{\pgfqpoint{1.729540in}{3.117164in}}%
\pgfpathcurveto{\pgfqpoint{1.735363in}{3.122988in}}{\pgfqpoint{1.738636in}{3.130888in}}{\pgfqpoint{1.738636in}{3.139124in}}%
\pgfpathcurveto{\pgfqpoint{1.738636in}{3.147361in}}{\pgfqpoint{1.735363in}{3.155261in}}{\pgfqpoint{1.729540in}{3.161085in}}%
\pgfpathcurveto{\pgfqpoint{1.723716in}{3.166909in}}{\pgfqpoint{1.715816in}{3.170181in}}{\pgfqpoint{1.707579in}{3.170181in}}%
\pgfpathcurveto{\pgfqpoint{1.699343in}{3.170181in}}{\pgfqpoint{1.691443in}{3.166909in}}{\pgfqpoint{1.685619in}{3.161085in}}%
\pgfpathcurveto{\pgfqpoint{1.679795in}{3.155261in}}{\pgfqpoint{1.676523in}{3.147361in}}{\pgfqpoint{1.676523in}{3.139124in}}%
\pgfpathcurveto{\pgfqpoint{1.676523in}{3.130888in}}{\pgfqpoint{1.679795in}{3.122988in}}{\pgfqpoint{1.685619in}{3.117164in}}%
\pgfpathcurveto{\pgfqpoint{1.691443in}{3.111340in}}{\pgfqpoint{1.699343in}{3.108068in}}{\pgfqpoint{1.707579in}{3.108068in}}%
\pgfpathclose%
\pgfusepath{stroke,fill}%
\end{pgfscope}%
\begin{pgfscope}%
\pgfpathrectangle{\pgfqpoint{0.100000in}{0.220728in}}{\pgfqpoint{3.696000in}{3.696000in}}%
\pgfusepath{clip}%
\pgfsetbuttcap%
\pgfsetroundjoin%
\definecolor{currentfill}{rgb}{0.121569,0.466667,0.705882}%
\pgfsetfillcolor{currentfill}%
\pgfsetfillopacity{0.323615}%
\pgfsetlinewidth{1.003750pt}%
\definecolor{currentstroke}{rgb}{0.121569,0.466667,0.705882}%
\pgfsetstrokecolor{currentstroke}%
\pgfsetstrokeopacity{0.323615}%
\pgfsetdash{}{0pt}%
\pgfpathmoveto{\pgfqpoint{1.706551in}{3.106467in}}%
\pgfpathcurveto{\pgfqpoint{1.714787in}{3.106467in}}{\pgfqpoint{1.722687in}{3.109739in}}{\pgfqpoint{1.728511in}{3.115563in}}%
\pgfpathcurveto{\pgfqpoint{1.734335in}{3.121387in}}{\pgfqpoint{1.737607in}{3.129287in}}{\pgfqpoint{1.737607in}{3.137523in}}%
\pgfpathcurveto{\pgfqpoint{1.737607in}{3.145759in}}{\pgfqpoint{1.734335in}{3.153659in}}{\pgfqpoint{1.728511in}{3.159483in}}%
\pgfpathcurveto{\pgfqpoint{1.722687in}{3.165307in}}{\pgfqpoint{1.714787in}{3.168580in}}{\pgfqpoint{1.706551in}{3.168580in}}%
\pgfpathcurveto{\pgfqpoint{1.698315in}{3.168580in}}{\pgfqpoint{1.690414in}{3.165307in}}{\pgfqpoint{1.684591in}{3.159483in}}%
\pgfpathcurveto{\pgfqpoint{1.678767in}{3.153659in}}{\pgfqpoint{1.675494in}{3.145759in}}{\pgfqpoint{1.675494in}{3.137523in}}%
\pgfpathcurveto{\pgfqpoint{1.675494in}{3.129287in}}{\pgfqpoint{1.678767in}{3.121387in}}{\pgfqpoint{1.684591in}{3.115563in}}%
\pgfpathcurveto{\pgfqpoint{1.690414in}{3.109739in}}{\pgfqpoint{1.698315in}{3.106467in}}{\pgfqpoint{1.706551in}{3.106467in}}%
\pgfpathclose%
\pgfusepath{stroke,fill}%
\end{pgfscope}%
\begin{pgfscope}%
\pgfpathrectangle{\pgfqpoint{0.100000in}{0.220728in}}{\pgfqpoint{3.696000in}{3.696000in}}%
\pgfusepath{clip}%
\pgfsetbuttcap%
\pgfsetroundjoin%
\definecolor{currentfill}{rgb}{0.121569,0.466667,0.705882}%
\pgfsetfillcolor{currentfill}%
\pgfsetfillopacity{0.323775}%
\pgfsetlinewidth{1.003750pt}%
\definecolor{currentstroke}{rgb}{0.121569,0.466667,0.705882}%
\pgfsetstrokecolor{currentstroke}%
\pgfsetstrokeopacity{0.323775}%
\pgfsetdash{}{0pt}%
\pgfpathmoveto{\pgfqpoint{1.706292in}{3.105610in}}%
\pgfpathcurveto{\pgfqpoint{1.714529in}{3.105610in}}{\pgfqpoint{1.722429in}{3.108883in}}{\pgfqpoint{1.728252in}{3.114707in}}%
\pgfpathcurveto{\pgfqpoint{1.734076in}{3.120531in}}{\pgfqpoint{1.737349in}{3.128431in}}{\pgfqpoint{1.737349in}{3.136667in}}%
\pgfpathcurveto{\pgfqpoint{1.737349in}{3.144903in}}{\pgfqpoint{1.734076in}{3.152803in}}{\pgfqpoint{1.728252in}{3.158627in}}%
\pgfpathcurveto{\pgfqpoint{1.722429in}{3.164451in}}{\pgfqpoint{1.714529in}{3.167723in}}{\pgfqpoint{1.706292in}{3.167723in}}%
\pgfpathcurveto{\pgfqpoint{1.698056in}{3.167723in}}{\pgfqpoint{1.690156in}{3.164451in}}{\pgfqpoint{1.684332in}{3.158627in}}%
\pgfpathcurveto{\pgfqpoint{1.678508in}{3.152803in}}{\pgfqpoint{1.675236in}{3.144903in}}{\pgfqpoint{1.675236in}{3.136667in}}%
\pgfpathcurveto{\pgfqpoint{1.675236in}{3.128431in}}{\pgfqpoint{1.678508in}{3.120531in}}{\pgfqpoint{1.684332in}{3.114707in}}%
\pgfpathcurveto{\pgfqpoint{1.690156in}{3.108883in}}{\pgfqpoint{1.698056in}{3.105610in}}{\pgfqpoint{1.706292in}{3.105610in}}%
\pgfpathclose%
\pgfusepath{stroke,fill}%
\end{pgfscope}%
\begin{pgfscope}%
\pgfpathrectangle{\pgfqpoint{0.100000in}{0.220728in}}{\pgfqpoint{3.696000in}{3.696000in}}%
\pgfusepath{clip}%
\pgfsetbuttcap%
\pgfsetroundjoin%
\definecolor{currentfill}{rgb}{0.121569,0.466667,0.705882}%
\pgfsetfillcolor{currentfill}%
\pgfsetfillopacity{0.323982}%
\pgfsetlinewidth{1.003750pt}%
\definecolor{currentstroke}{rgb}{0.121569,0.466667,0.705882}%
\pgfsetstrokecolor{currentstroke}%
\pgfsetstrokeopacity{0.323982}%
\pgfsetdash{}{0pt}%
\pgfpathmoveto{\pgfqpoint{1.705375in}{3.104150in}}%
\pgfpathcurveto{\pgfqpoint{1.713612in}{3.104150in}}{\pgfqpoint{1.721512in}{3.107422in}}{\pgfqpoint{1.727336in}{3.113246in}}%
\pgfpathcurveto{\pgfqpoint{1.733160in}{3.119070in}}{\pgfqpoint{1.736432in}{3.126970in}}{\pgfqpoint{1.736432in}{3.135206in}}%
\pgfpathcurveto{\pgfqpoint{1.736432in}{3.143442in}}{\pgfqpoint{1.733160in}{3.151342in}}{\pgfqpoint{1.727336in}{3.157166in}}%
\pgfpathcurveto{\pgfqpoint{1.721512in}{3.162990in}}{\pgfqpoint{1.713612in}{3.166263in}}{\pgfqpoint{1.705375in}{3.166263in}}%
\pgfpathcurveto{\pgfqpoint{1.697139in}{3.166263in}}{\pgfqpoint{1.689239in}{3.162990in}}{\pgfqpoint{1.683415in}{3.157166in}}%
\pgfpathcurveto{\pgfqpoint{1.677591in}{3.151342in}}{\pgfqpoint{1.674319in}{3.143442in}}{\pgfqpoint{1.674319in}{3.135206in}}%
\pgfpathcurveto{\pgfqpoint{1.674319in}{3.126970in}}{\pgfqpoint{1.677591in}{3.119070in}}{\pgfqpoint{1.683415in}{3.113246in}}%
\pgfpathcurveto{\pgfqpoint{1.689239in}{3.107422in}}{\pgfqpoint{1.697139in}{3.104150in}}{\pgfqpoint{1.705375in}{3.104150in}}%
\pgfpathclose%
\pgfusepath{stroke,fill}%
\end{pgfscope}%
\begin{pgfscope}%
\pgfpathrectangle{\pgfqpoint{0.100000in}{0.220728in}}{\pgfqpoint{3.696000in}{3.696000in}}%
\pgfusepath{clip}%
\pgfsetbuttcap%
\pgfsetroundjoin%
\definecolor{currentfill}{rgb}{0.121569,0.466667,0.705882}%
\pgfsetfillcolor{currentfill}%
\pgfsetfillopacity{0.324410}%
\pgfsetlinewidth{1.003750pt}%
\definecolor{currentstroke}{rgb}{0.121569,0.466667,0.705882}%
\pgfsetstrokecolor{currentstroke}%
\pgfsetstrokeopacity{0.324410}%
\pgfsetdash{}{0pt}%
\pgfpathmoveto{\pgfqpoint{1.867144in}{3.253060in}}%
\pgfpathcurveto{\pgfqpoint{1.875381in}{3.253060in}}{\pgfqpoint{1.883281in}{3.256333in}}{\pgfqpoint{1.889105in}{3.262157in}}%
\pgfpathcurveto{\pgfqpoint{1.894928in}{3.267981in}}{\pgfqpoint{1.898201in}{3.275881in}}{\pgfqpoint{1.898201in}{3.284117in}}%
\pgfpathcurveto{\pgfqpoint{1.898201in}{3.292353in}}{\pgfqpoint{1.894928in}{3.300253in}}{\pgfqpoint{1.889105in}{3.306077in}}%
\pgfpathcurveto{\pgfqpoint{1.883281in}{3.311901in}}{\pgfqpoint{1.875381in}{3.315173in}}{\pgfqpoint{1.867144in}{3.315173in}}%
\pgfpathcurveto{\pgfqpoint{1.858908in}{3.315173in}}{\pgfqpoint{1.851008in}{3.311901in}}{\pgfqpoint{1.845184in}{3.306077in}}%
\pgfpathcurveto{\pgfqpoint{1.839360in}{3.300253in}}{\pgfqpoint{1.836088in}{3.292353in}}{\pgfqpoint{1.836088in}{3.284117in}}%
\pgfpathcurveto{\pgfqpoint{1.836088in}{3.275881in}}{\pgfqpoint{1.839360in}{3.267981in}}{\pgfqpoint{1.845184in}{3.262157in}}%
\pgfpathcurveto{\pgfqpoint{1.851008in}{3.256333in}}{\pgfqpoint{1.858908in}{3.253060in}}{\pgfqpoint{1.867144in}{3.253060in}}%
\pgfpathclose%
\pgfusepath{stroke,fill}%
\end{pgfscope}%
\begin{pgfscope}%
\pgfpathrectangle{\pgfqpoint{0.100000in}{0.220728in}}{\pgfqpoint{3.696000in}{3.696000in}}%
\pgfusepath{clip}%
\pgfsetbuttcap%
\pgfsetroundjoin%
\definecolor{currentfill}{rgb}{0.121569,0.466667,0.705882}%
\pgfsetfillcolor{currentfill}%
\pgfsetfillopacity{0.324507}%
\pgfsetlinewidth{1.003750pt}%
\definecolor{currentstroke}{rgb}{0.121569,0.466667,0.705882}%
\pgfsetstrokecolor{currentstroke}%
\pgfsetstrokeopacity{0.324507}%
\pgfsetdash{}{0pt}%
\pgfpathmoveto{\pgfqpoint{1.704445in}{3.101346in}}%
\pgfpathcurveto{\pgfqpoint{1.712681in}{3.101346in}}{\pgfqpoint{1.720581in}{3.104619in}}{\pgfqpoint{1.726405in}{3.110443in}}%
\pgfpathcurveto{\pgfqpoint{1.732229in}{3.116266in}}{\pgfqpoint{1.735501in}{3.124166in}}{\pgfqpoint{1.735501in}{3.132403in}}%
\pgfpathcurveto{\pgfqpoint{1.735501in}{3.140639in}}{\pgfqpoint{1.732229in}{3.148539in}}{\pgfqpoint{1.726405in}{3.154363in}}%
\pgfpathcurveto{\pgfqpoint{1.720581in}{3.160187in}}{\pgfqpoint{1.712681in}{3.163459in}}{\pgfqpoint{1.704445in}{3.163459in}}%
\pgfpathcurveto{\pgfqpoint{1.696208in}{3.163459in}}{\pgfqpoint{1.688308in}{3.160187in}}{\pgfqpoint{1.682484in}{3.154363in}}%
\pgfpathcurveto{\pgfqpoint{1.676660in}{3.148539in}}{\pgfqpoint{1.673388in}{3.140639in}}{\pgfqpoint{1.673388in}{3.132403in}}%
\pgfpathcurveto{\pgfqpoint{1.673388in}{3.124166in}}{\pgfqpoint{1.676660in}{3.116266in}}{\pgfqpoint{1.682484in}{3.110443in}}%
\pgfpathcurveto{\pgfqpoint{1.688308in}{3.104619in}}{\pgfqpoint{1.696208in}{3.101346in}}{\pgfqpoint{1.704445in}{3.101346in}}%
\pgfpathclose%
\pgfusepath{stroke,fill}%
\end{pgfscope}%
\begin{pgfscope}%
\pgfpathrectangle{\pgfqpoint{0.100000in}{0.220728in}}{\pgfqpoint{3.696000in}{3.696000in}}%
\pgfusepath{clip}%
\pgfsetbuttcap%
\pgfsetroundjoin%
\definecolor{currentfill}{rgb}{0.121569,0.466667,0.705882}%
\pgfsetfillcolor{currentfill}%
\pgfsetfillopacity{0.324789}%
\pgfsetlinewidth{1.003750pt}%
\definecolor{currentstroke}{rgb}{0.121569,0.466667,0.705882}%
\pgfsetstrokecolor{currentstroke}%
\pgfsetstrokeopacity{0.324789}%
\pgfsetdash{}{0pt}%
\pgfpathmoveto{\pgfqpoint{1.703282in}{3.099511in}}%
\pgfpathcurveto{\pgfqpoint{1.711518in}{3.099511in}}{\pgfqpoint{1.719418in}{3.102783in}}{\pgfqpoint{1.725242in}{3.108607in}}%
\pgfpathcurveto{\pgfqpoint{1.731066in}{3.114431in}}{\pgfqpoint{1.734339in}{3.122331in}}{\pgfqpoint{1.734339in}{3.130567in}}%
\pgfpathcurveto{\pgfqpoint{1.734339in}{3.138804in}}{\pgfqpoint{1.731066in}{3.146704in}}{\pgfqpoint{1.725242in}{3.152528in}}%
\pgfpathcurveto{\pgfqpoint{1.719418in}{3.158352in}}{\pgfqpoint{1.711518in}{3.161624in}}{\pgfqpoint{1.703282in}{3.161624in}}%
\pgfpathcurveto{\pgfqpoint{1.695046in}{3.161624in}}{\pgfqpoint{1.687146in}{3.158352in}}{\pgfqpoint{1.681322in}{3.152528in}}%
\pgfpathcurveto{\pgfqpoint{1.675498in}{3.146704in}}{\pgfqpoint{1.672226in}{3.138804in}}{\pgfqpoint{1.672226in}{3.130567in}}%
\pgfpathcurveto{\pgfqpoint{1.672226in}{3.122331in}}{\pgfqpoint{1.675498in}{3.114431in}}{\pgfqpoint{1.681322in}{3.108607in}}%
\pgfpathcurveto{\pgfqpoint{1.687146in}{3.102783in}}{\pgfqpoint{1.695046in}{3.099511in}}{\pgfqpoint{1.703282in}{3.099511in}}%
\pgfpathclose%
\pgfusepath{stroke,fill}%
\end{pgfscope}%
\begin{pgfscope}%
\pgfpathrectangle{\pgfqpoint{0.100000in}{0.220728in}}{\pgfqpoint{3.696000in}{3.696000in}}%
\pgfusepath{clip}%
\pgfsetbuttcap%
\pgfsetroundjoin%
\definecolor{currentfill}{rgb}{0.121569,0.466667,0.705882}%
\pgfsetfillcolor{currentfill}%
\pgfsetfillopacity{0.324881}%
\pgfsetlinewidth{1.003750pt}%
\definecolor{currentstroke}{rgb}{0.121569,0.466667,0.705882}%
\pgfsetstrokecolor{currentstroke}%
\pgfsetstrokeopacity{0.324881}%
\pgfsetdash{}{0pt}%
\pgfpathmoveto{\pgfqpoint{1.703000in}{3.098914in}}%
\pgfpathcurveto{\pgfqpoint{1.711237in}{3.098914in}}{\pgfqpoint{1.719137in}{3.102186in}}{\pgfqpoint{1.724961in}{3.108010in}}%
\pgfpathcurveto{\pgfqpoint{1.730784in}{3.113834in}}{\pgfqpoint{1.734057in}{3.121734in}}{\pgfqpoint{1.734057in}{3.129970in}}%
\pgfpathcurveto{\pgfqpoint{1.734057in}{3.138207in}}{\pgfqpoint{1.730784in}{3.146107in}}{\pgfqpoint{1.724961in}{3.151931in}}%
\pgfpathcurveto{\pgfqpoint{1.719137in}{3.157755in}}{\pgfqpoint{1.711237in}{3.161027in}}{\pgfqpoint{1.703000in}{3.161027in}}%
\pgfpathcurveto{\pgfqpoint{1.694764in}{3.161027in}}{\pgfqpoint{1.686864in}{3.157755in}}{\pgfqpoint{1.681040in}{3.151931in}}%
\pgfpathcurveto{\pgfqpoint{1.675216in}{3.146107in}}{\pgfqpoint{1.671944in}{3.138207in}}{\pgfqpoint{1.671944in}{3.129970in}}%
\pgfpathcurveto{\pgfqpoint{1.671944in}{3.121734in}}{\pgfqpoint{1.675216in}{3.113834in}}{\pgfqpoint{1.681040in}{3.108010in}}%
\pgfpathcurveto{\pgfqpoint{1.686864in}{3.102186in}}{\pgfqpoint{1.694764in}{3.098914in}}{\pgfqpoint{1.703000in}{3.098914in}}%
\pgfpathclose%
\pgfusepath{stroke,fill}%
\end{pgfscope}%
\begin{pgfscope}%
\pgfpathrectangle{\pgfqpoint{0.100000in}{0.220728in}}{\pgfqpoint{3.696000in}{3.696000in}}%
\pgfusepath{clip}%
\pgfsetbuttcap%
\pgfsetroundjoin%
\definecolor{currentfill}{rgb}{0.121569,0.466667,0.705882}%
\pgfsetfillcolor{currentfill}%
\pgfsetfillopacity{0.325065}%
\pgfsetlinewidth{1.003750pt}%
\definecolor{currentstroke}{rgb}{0.121569,0.466667,0.705882}%
\pgfsetstrokecolor{currentstroke}%
\pgfsetstrokeopacity{0.325065}%
\pgfsetdash{}{0pt}%
\pgfpathmoveto{\pgfqpoint{1.702606in}{3.097800in}}%
\pgfpathcurveto{\pgfqpoint{1.710842in}{3.097800in}}{\pgfqpoint{1.718742in}{3.101073in}}{\pgfqpoint{1.724566in}{3.106897in}}%
\pgfpathcurveto{\pgfqpoint{1.730390in}{3.112720in}}{\pgfqpoint{1.733662in}{3.120621in}}{\pgfqpoint{1.733662in}{3.128857in}}%
\pgfpathcurveto{\pgfqpoint{1.733662in}{3.137093in}}{\pgfqpoint{1.730390in}{3.144993in}}{\pgfqpoint{1.724566in}{3.150817in}}%
\pgfpathcurveto{\pgfqpoint{1.718742in}{3.156641in}}{\pgfqpoint{1.710842in}{3.159913in}}{\pgfqpoint{1.702606in}{3.159913in}}%
\pgfpathcurveto{\pgfqpoint{1.694369in}{3.159913in}}{\pgfqpoint{1.686469in}{3.156641in}}{\pgfqpoint{1.680645in}{3.150817in}}%
\pgfpathcurveto{\pgfqpoint{1.674821in}{3.144993in}}{\pgfqpoint{1.671549in}{3.137093in}}{\pgfqpoint{1.671549in}{3.128857in}}%
\pgfpathcurveto{\pgfqpoint{1.671549in}{3.120621in}}{\pgfqpoint{1.674821in}{3.112720in}}{\pgfqpoint{1.680645in}{3.106897in}}%
\pgfpathcurveto{\pgfqpoint{1.686469in}{3.101073in}}{\pgfqpoint{1.694369in}{3.097800in}}{\pgfqpoint{1.702606in}{3.097800in}}%
\pgfpathclose%
\pgfusepath{stroke,fill}%
\end{pgfscope}%
\begin{pgfscope}%
\pgfpathrectangle{\pgfqpoint{0.100000in}{0.220728in}}{\pgfqpoint{3.696000in}{3.696000in}}%
\pgfusepath{clip}%
\pgfsetbuttcap%
\pgfsetroundjoin%
\definecolor{currentfill}{rgb}{0.121569,0.466667,0.705882}%
\pgfsetfillcolor{currentfill}%
\pgfsetfillopacity{0.325119}%
\pgfsetlinewidth{1.003750pt}%
\definecolor{currentstroke}{rgb}{0.121569,0.466667,0.705882}%
\pgfsetstrokecolor{currentstroke}%
\pgfsetstrokeopacity{0.325119}%
\pgfsetdash{}{0pt}%
\pgfpathmoveto{\pgfqpoint{1.702451in}{3.097452in}}%
\pgfpathcurveto{\pgfqpoint{1.710687in}{3.097452in}}{\pgfqpoint{1.718587in}{3.100725in}}{\pgfqpoint{1.724411in}{3.106549in}}%
\pgfpathcurveto{\pgfqpoint{1.730235in}{3.112373in}}{\pgfqpoint{1.733507in}{3.120273in}}{\pgfqpoint{1.733507in}{3.128509in}}%
\pgfpathcurveto{\pgfqpoint{1.733507in}{3.136745in}}{\pgfqpoint{1.730235in}{3.144645in}}{\pgfqpoint{1.724411in}{3.150469in}}%
\pgfpathcurveto{\pgfqpoint{1.718587in}{3.156293in}}{\pgfqpoint{1.710687in}{3.159565in}}{\pgfqpoint{1.702451in}{3.159565in}}%
\pgfpathcurveto{\pgfqpoint{1.694215in}{3.159565in}}{\pgfqpoint{1.686315in}{3.156293in}}{\pgfqpoint{1.680491in}{3.150469in}}%
\pgfpathcurveto{\pgfqpoint{1.674667in}{3.144645in}}{\pgfqpoint{1.671394in}{3.136745in}}{\pgfqpoint{1.671394in}{3.128509in}}%
\pgfpathcurveto{\pgfqpoint{1.671394in}{3.120273in}}{\pgfqpoint{1.674667in}{3.112373in}}{\pgfqpoint{1.680491in}{3.106549in}}%
\pgfpathcurveto{\pgfqpoint{1.686315in}{3.100725in}}{\pgfqpoint{1.694215in}{3.097452in}}{\pgfqpoint{1.702451in}{3.097452in}}%
\pgfpathclose%
\pgfusepath{stroke,fill}%
\end{pgfscope}%
\begin{pgfscope}%
\pgfpathrectangle{\pgfqpoint{0.100000in}{0.220728in}}{\pgfqpoint{3.696000in}{3.696000in}}%
\pgfusepath{clip}%
\pgfsetbuttcap%
\pgfsetroundjoin%
\definecolor{currentfill}{rgb}{0.121569,0.466667,0.705882}%
\pgfsetfillcolor{currentfill}%
\pgfsetfillopacity{0.325228}%
\pgfsetlinewidth{1.003750pt}%
\definecolor{currentstroke}{rgb}{0.121569,0.466667,0.705882}%
\pgfsetstrokecolor{currentstroke}%
\pgfsetstrokeopacity{0.325228}%
\pgfsetdash{}{0pt}%
\pgfpathmoveto{\pgfqpoint{1.702215in}{3.096830in}}%
\pgfpathcurveto{\pgfqpoint{1.710452in}{3.096830in}}{\pgfqpoint{1.718352in}{3.100102in}}{\pgfqpoint{1.724176in}{3.105926in}}%
\pgfpathcurveto{\pgfqpoint{1.729999in}{3.111750in}}{\pgfqpoint{1.733272in}{3.119650in}}{\pgfqpoint{1.733272in}{3.127886in}}%
\pgfpathcurveto{\pgfqpoint{1.733272in}{3.136123in}}{\pgfqpoint{1.729999in}{3.144023in}}{\pgfqpoint{1.724176in}{3.149847in}}%
\pgfpathcurveto{\pgfqpoint{1.718352in}{3.155670in}}{\pgfqpoint{1.710452in}{3.158943in}}{\pgfqpoint{1.702215in}{3.158943in}}%
\pgfpathcurveto{\pgfqpoint{1.693979in}{3.158943in}}{\pgfqpoint{1.686079in}{3.155670in}}{\pgfqpoint{1.680255in}{3.149847in}}%
\pgfpathcurveto{\pgfqpoint{1.674431in}{3.144023in}}{\pgfqpoint{1.671159in}{3.136123in}}{\pgfqpoint{1.671159in}{3.127886in}}%
\pgfpathcurveto{\pgfqpoint{1.671159in}{3.119650in}}{\pgfqpoint{1.674431in}{3.111750in}}{\pgfqpoint{1.680255in}{3.105926in}}%
\pgfpathcurveto{\pgfqpoint{1.686079in}{3.100102in}}{\pgfqpoint{1.693979in}{3.096830in}}{\pgfqpoint{1.702215in}{3.096830in}}%
\pgfpathclose%
\pgfusepath{stroke,fill}%
\end{pgfscope}%
\begin{pgfscope}%
\pgfpathrectangle{\pgfqpoint{0.100000in}{0.220728in}}{\pgfqpoint{3.696000in}{3.696000in}}%
\pgfusepath{clip}%
\pgfsetbuttcap%
\pgfsetroundjoin%
\definecolor{currentfill}{rgb}{0.121569,0.466667,0.705882}%
\pgfsetfillcolor{currentfill}%
\pgfsetfillopacity{0.325428}%
\pgfsetlinewidth{1.003750pt}%
\definecolor{currentstroke}{rgb}{0.121569,0.466667,0.705882}%
\pgfsetstrokecolor{currentstroke}%
\pgfsetstrokeopacity{0.325428}%
\pgfsetdash{}{0pt}%
\pgfpathmoveto{\pgfqpoint{1.701848in}{3.095663in}}%
\pgfpathcurveto{\pgfqpoint{1.710085in}{3.095663in}}{\pgfqpoint{1.717985in}{3.098935in}}{\pgfqpoint{1.723809in}{3.104759in}}%
\pgfpathcurveto{\pgfqpoint{1.729633in}{3.110583in}}{\pgfqpoint{1.732905in}{3.118483in}}{\pgfqpoint{1.732905in}{3.126720in}}%
\pgfpathcurveto{\pgfqpoint{1.732905in}{3.134956in}}{\pgfqpoint{1.729633in}{3.142856in}}{\pgfqpoint{1.723809in}{3.148680in}}%
\pgfpathcurveto{\pgfqpoint{1.717985in}{3.154504in}}{\pgfqpoint{1.710085in}{3.157776in}}{\pgfqpoint{1.701848in}{3.157776in}}%
\pgfpathcurveto{\pgfqpoint{1.693612in}{3.157776in}}{\pgfqpoint{1.685712in}{3.154504in}}{\pgfqpoint{1.679888in}{3.148680in}}%
\pgfpathcurveto{\pgfqpoint{1.674064in}{3.142856in}}{\pgfqpoint{1.670792in}{3.134956in}}{\pgfqpoint{1.670792in}{3.126720in}}%
\pgfpathcurveto{\pgfqpoint{1.670792in}{3.118483in}}{\pgfqpoint{1.674064in}{3.110583in}}{\pgfqpoint{1.679888in}{3.104759in}}%
\pgfpathcurveto{\pgfqpoint{1.685712in}{3.098935in}}{\pgfqpoint{1.693612in}{3.095663in}}{\pgfqpoint{1.701848in}{3.095663in}}%
\pgfpathclose%
\pgfusepath{stroke,fill}%
\end{pgfscope}%
\begin{pgfscope}%
\pgfpathrectangle{\pgfqpoint{0.100000in}{0.220728in}}{\pgfqpoint{3.696000in}{3.696000in}}%
\pgfusepath{clip}%
\pgfsetbuttcap%
\pgfsetroundjoin%
\definecolor{currentfill}{rgb}{0.121569,0.466667,0.705882}%
\pgfsetfillcolor{currentfill}%
\pgfsetfillopacity{0.325764}%
\pgfsetlinewidth{1.003750pt}%
\definecolor{currentstroke}{rgb}{0.121569,0.466667,0.705882}%
\pgfsetstrokecolor{currentstroke}%
\pgfsetstrokeopacity{0.325764}%
\pgfsetdash{}{0pt}%
\pgfpathmoveto{\pgfqpoint{1.700833in}{3.093724in}}%
\pgfpathcurveto{\pgfqpoint{1.709069in}{3.093724in}}{\pgfqpoint{1.716970in}{3.096997in}}{\pgfqpoint{1.722793in}{3.102821in}}%
\pgfpathcurveto{\pgfqpoint{1.728617in}{3.108644in}}{\pgfqpoint{1.731890in}{3.116544in}}{\pgfqpoint{1.731890in}{3.124781in}}%
\pgfpathcurveto{\pgfqpoint{1.731890in}{3.133017in}}{\pgfqpoint{1.728617in}{3.140917in}}{\pgfqpoint{1.722793in}{3.146741in}}%
\pgfpathcurveto{\pgfqpoint{1.716970in}{3.152565in}}{\pgfqpoint{1.709069in}{3.155837in}}{\pgfqpoint{1.700833in}{3.155837in}}%
\pgfpathcurveto{\pgfqpoint{1.692597in}{3.155837in}}{\pgfqpoint{1.684697in}{3.152565in}}{\pgfqpoint{1.678873in}{3.146741in}}%
\pgfpathcurveto{\pgfqpoint{1.673049in}{3.140917in}}{\pgfqpoint{1.669777in}{3.133017in}}{\pgfqpoint{1.669777in}{3.124781in}}%
\pgfpathcurveto{\pgfqpoint{1.669777in}{3.116544in}}{\pgfqpoint{1.673049in}{3.108644in}}{\pgfqpoint{1.678873in}{3.102821in}}%
\pgfpathcurveto{\pgfqpoint{1.684697in}{3.096997in}}{\pgfqpoint{1.692597in}{3.093724in}}{\pgfqpoint{1.700833in}{3.093724in}}%
\pgfpathclose%
\pgfusepath{stroke,fill}%
\end{pgfscope}%
\begin{pgfscope}%
\pgfpathrectangle{\pgfqpoint{0.100000in}{0.220728in}}{\pgfqpoint{3.696000in}{3.696000in}}%
\pgfusepath{clip}%
\pgfsetbuttcap%
\pgfsetroundjoin%
\definecolor{currentfill}{rgb}{0.121569,0.466667,0.705882}%
\pgfsetfillcolor{currentfill}%
\pgfsetfillopacity{0.326405}%
\pgfsetlinewidth{1.003750pt}%
\definecolor{currentstroke}{rgb}{0.121569,0.466667,0.705882}%
\pgfsetstrokecolor{currentstroke}%
\pgfsetstrokeopacity{0.326405}%
\pgfsetdash{}{0pt}%
\pgfpathmoveto{\pgfqpoint{1.700578in}{3.089498in}}%
\pgfpathcurveto{\pgfqpoint{1.708815in}{3.089498in}}{\pgfqpoint{1.716715in}{3.092770in}}{\pgfqpoint{1.722539in}{3.098594in}}%
\pgfpathcurveto{\pgfqpoint{1.728363in}{3.104418in}}{\pgfqpoint{1.731635in}{3.112318in}}{\pgfqpoint{1.731635in}{3.120555in}}%
\pgfpathcurveto{\pgfqpoint{1.731635in}{3.128791in}}{\pgfqpoint{1.728363in}{3.136691in}}{\pgfqpoint{1.722539in}{3.142515in}}%
\pgfpathcurveto{\pgfqpoint{1.716715in}{3.148339in}}{\pgfqpoint{1.708815in}{3.151611in}}{\pgfqpoint{1.700578in}{3.151611in}}%
\pgfpathcurveto{\pgfqpoint{1.692342in}{3.151611in}}{\pgfqpoint{1.684442in}{3.148339in}}{\pgfqpoint{1.678618in}{3.142515in}}%
\pgfpathcurveto{\pgfqpoint{1.672794in}{3.136691in}}{\pgfqpoint{1.669522in}{3.128791in}}{\pgfqpoint{1.669522in}{3.120555in}}%
\pgfpathcurveto{\pgfqpoint{1.669522in}{3.112318in}}{\pgfqpoint{1.672794in}{3.104418in}}{\pgfqpoint{1.678618in}{3.098594in}}%
\pgfpathcurveto{\pgfqpoint{1.684442in}{3.092770in}}{\pgfqpoint{1.692342in}{3.089498in}}{\pgfqpoint{1.700578in}{3.089498in}}%
\pgfpathclose%
\pgfusepath{stroke,fill}%
\end{pgfscope}%
\begin{pgfscope}%
\pgfpathrectangle{\pgfqpoint{0.100000in}{0.220728in}}{\pgfqpoint{3.696000in}{3.696000in}}%
\pgfusepath{clip}%
\pgfsetbuttcap%
\pgfsetroundjoin%
\definecolor{currentfill}{rgb}{0.121569,0.466667,0.705882}%
\pgfsetfillcolor{currentfill}%
\pgfsetfillopacity{0.326588}%
\pgfsetlinewidth{1.003750pt}%
\definecolor{currentstroke}{rgb}{0.121569,0.466667,0.705882}%
\pgfsetstrokecolor{currentstroke}%
\pgfsetstrokeopacity{0.326588}%
\pgfsetdash{}{0pt}%
\pgfpathmoveto{\pgfqpoint{1.699755in}{3.088453in}}%
\pgfpathcurveto{\pgfqpoint{1.707991in}{3.088453in}}{\pgfqpoint{1.715892in}{3.091726in}}{\pgfqpoint{1.721715in}{3.097550in}}%
\pgfpathcurveto{\pgfqpoint{1.727539in}{3.103373in}}{\pgfqpoint{1.730812in}{3.111274in}}{\pgfqpoint{1.730812in}{3.119510in}}%
\pgfpathcurveto{\pgfqpoint{1.730812in}{3.127746in}}{\pgfqpoint{1.727539in}{3.135646in}}{\pgfqpoint{1.721715in}{3.141470in}}%
\pgfpathcurveto{\pgfqpoint{1.715892in}{3.147294in}}{\pgfqpoint{1.707991in}{3.150566in}}{\pgfqpoint{1.699755in}{3.150566in}}%
\pgfpathcurveto{\pgfqpoint{1.691519in}{3.150566in}}{\pgfqpoint{1.683619in}{3.147294in}}{\pgfqpoint{1.677795in}{3.141470in}}%
\pgfpathcurveto{\pgfqpoint{1.671971in}{3.135646in}}{\pgfqpoint{1.668699in}{3.127746in}}{\pgfqpoint{1.668699in}{3.119510in}}%
\pgfpathcurveto{\pgfqpoint{1.668699in}{3.111274in}}{\pgfqpoint{1.671971in}{3.103373in}}{\pgfqpoint{1.677795in}{3.097550in}}%
\pgfpathcurveto{\pgfqpoint{1.683619in}{3.091726in}}{\pgfqpoint{1.691519in}{3.088453in}}{\pgfqpoint{1.699755in}{3.088453in}}%
\pgfpathclose%
\pgfusepath{stroke,fill}%
\end{pgfscope}%
\begin{pgfscope}%
\pgfpathrectangle{\pgfqpoint{0.100000in}{0.220728in}}{\pgfqpoint{3.696000in}{3.696000in}}%
\pgfusepath{clip}%
\pgfsetbuttcap%
\pgfsetroundjoin%
\definecolor{currentfill}{rgb}{0.121569,0.466667,0.705882}%
\pgfsetfillcolor{currentfill}%
\pgfsetfillopacity{0.327018}%
\pgfsetlinewidth{1.003750pt}%
\definecolor{currentstroke}{rgb}{0.121569,0.466667,0.705882}%
\pgfsetstrokecolor{currentstroke}%
\pgfsetstrokeopacity{0.327018}%
\pgfsetdash{}{0pt}%
\pgfpathmoveto{\pgfqpoint{1.699175in}{3.086019in}}%
\pgfpathcurveto{\pgfqpoint{1.707411in}{3.086019in}}{\pgfqpoint{1.715311in}{3.089291in}}{\pgfqpoint{1.721135in}{3.095115in}}%
\pgfpathcurveto{\pgfqpoint{1.726959in}{3.100939in}}{\pgfqpoint{1.730231in}{3.108839in}}{\pgfqpoint{1.730231in}{3.117075in}}%
\pgfpathcurveto{\pgfqpoint{1.730231in}{3.125312in}}{\pgfqpoint{1.726959in}{3.133212in}}{\pgfqpoint{1.721135in}{3.139036in}}%
\pgfpathcurveto{\pgfqpoint{1.715311in}{3.144860in}}{\pgfqpoint{1.707411in}{3.148132in}}{\pgfqpoint{1.699175in}{3.148132in}}%
\pgfpathcurveto{\pgfqpoint{1.690938in}{3.148132in}}{\pgfqpoint{1.683038in}{3.144860in}}{\pgfqpoint{1.677214in}{3.139036in}}%
\pgfpathcurveto{\pgfqpoint{1.671390in}{3.133212in}}{\pgfqpoint{1.668118in}{3.125312in}}{\pgfqpoint{1.668118in}{3.117075in}}%
\pgfpathcurveto{\pgfqpoint{1.668118in}{3.108839in}}{\pgfqpoint{1.671390in}{3.100939in}}{\pgfqpoint{1.677214in}{3.095115in}}%
\pgfpathcurveto{\pgfqpoint{1.683038in}{3.089291in}}{\pgfqpoint{1.690938in}{3.086019in}}{\pgfqpoint{1.699175in}{3.086019in}}%
\pgfpathclose%
\pgfusepath{stroke,fill}%
\end{pgfscope}%
\begin{pgfscope}%
\pgfpathrectangle{\pgfqpoint{0.100000in}{0.220728in}}{\pgfqpoint{3.696000in}{3.696000in}}%
\pgfusepath{clip}%
\pgfsetbuttcap%
\pgfsetroundjoin%
\definecolor{currentfill}{rgb}{0.121569,0.466667,0.705882}%
\pgfsetfillcolor{currentfill}%
\pgfsetfillopacity{0.327492}%
\pgfsetlinewidth{1.003750pt}%
\definecolor{currentstroke}{rgb}{0.121569,0.466667,0.705882}%
\pgfsetstrokecolor{currentstroke}%
\pgfsetstrokeopacity{0.327492}%
\pgfsetdash{}{0pt}%
\pgfpathmoveto{\pgfqpoint{1.877554in}{3.251656in}}%
\pgfpathcurveto{\pgfqpoint{1.885790in}{3.251656in}}{\pgfqpoint{1.893690in}{3.254928in}}{\pgfqpoint{1.899514in}{3.260752in}}%
\pgfpathcurveto{\pgfqpoint{1.905338in}{3.266576in}}{\pgfqpoint{1.908610in}{3.274476in}}{\pgfqpoint{1.908610in}{3.282712in}}%
\pgfpathcurveto{\pgfqpoint{1.908610in}{3.290949in}}{\pgfqpoint{1.905338in}{3.298849in}}{\pgfqpoint{1.899514in}{3.304672in}}%
\pgfpathcurveto{\pgfqpoint{1.893690in}{3.310496in}}{\pgfqpoint{1.885790in}{3.313769in}}{\pgfqpoint{1.877554in}{3.313769in}}%
\pgfpathcurveto{\pgfqpoint{1.869317in}{3.313769in}}{\pgfqpoint{1.861417in}{3.310496in}}{\pgfqpoint{1.855594in}{3.304672in}}%
\pgfpathcurveto{\pgfqpoint{1.849770in}{3.298849in}}{\pgfqpoint{1.846497in}{3.290949in}}{\pgfqpoint{1.846497in}{3.282712in}}%
\pgfpathcurveto{\pgfqpoint{1.846497in}{3.274476in}}{\pgfqpoint{1.849770in}{3.266576in}}{\pgfqpoint{1.855594in}{3.260752in}}%
\pgfpathcurveto{\pgfqpoint{1.861417in}{3.254928in}}{\pgfqpoint{1.869317in}{3.251656in}}{\pgfqpoint{1.877554in}{3.251656in}}%
\pgfpathclose%
\pgfusepath{stroke,fill}%
\end{pgfscope}%
\begin{pgfscope}%
\pgfpathrectangle{\pgfqpoint{0.100000in}{0.220728in}}{\pgfqpoint{3.696000in}{3.696000in}}%
\pgfusepath{clip}%
\pgfsetbuttcap%
\pgfsetroundjoin%
\definecolor{currentfill}{rgb}{0.121569,0.466667,0.705882}%
\pgfsetfillcolor{currentfill}%
\pgfsetfillopacity{0.327774}%
\pgfsetlinewidth{1.003750pt}%
\definecolor{currentstroke}{rgb}{0.121569,0.466667,0.705882}%
\pgfsetstrokecolor{currentstroke}%
\pgfsetstrokeopacity{0.327774}%
\pgfsetdash{}{0pt}%
\pgfpathmoveto{\pgfqpoint{1.697729in}{3.081692in}}%
\pgfpathcurveto{\pgfqpoint{1.705965in}{3.081692in}}{\pgfqpoint{1.713865in}{3.084964in}}{\pgfqpoint{1.719689in}{3.090788in}}%
\pgfpathcurveto{\pgfqpoint{1.725513in}{3.096612in}}{\pgfqpoint{1.728785in}{3.104512in}}{\pgfqpoint{1.728785in}{3.112748in}}%
\pgfpathcurveto{\pgfqpoint{1.728785in}{3.120984in}}{\pgfqpoint{1.725513in}{3.128884in}}{\pgfqpoint{1.719689in}{3.134708in}}%
\pgfpathcurveto{\pgfqpoint{1.713865in}{3.140532in}}{\pgfqpoint{1.705965in}{3.143805in}}{\pgfqpoint{1.697729in}{3.143805in}}%
\pgfpathcurveto{\pgfqpoint{1.689493in}{3.143805in}}{\pgfqpoint{1.681593in}{3.140532in}}{\pgfqpoint{1.675769in}{3.134708in}}%
\pgfpathcurveto{\pgfqpoint{1.669945in}{3.128884in}}{\pgfqpoint{1.666672in}{3.120984in}}{\pgfqpoint{1.666672in}{3.112748in}}%
\pgfpathcurveto{\pgfqpoint{1.666672in}{3.104512in}}{\pgfqpoint{1.669945in}{3.096612in}}{\pgfqpoint{1.675769in}{3.090788in}}%
\pgfpathcurveto{\pgfqpoint{1.681593in}{3.084964in}}{\pgfqpoint{1.689493in}{3.081692in}}{\pgfqpoint{1.697729in}{3.081692in}}%
\pgfpathclose%
\pgfusepath{stroke,fill}%
\end{pgfscope}%
\begin{pgfscope}%
\pgfpathrectangle{\pgfqpoint{0.100000in}{0.220728in}}{\pgfqpoint{3.696000in}{3.696000in}}%
\pgfusepath{clip}%
\pgfsetbuttcap%
\pgfsetroundjoin%
\definecolor{currentfill}{rgb}{0.121569,0.466667,0.705882}%
\pgfsetfillcolor{currentfill}%
\pgfsetfillopacity{0.329047}%
\pgfsetlinewidth{1.003750pt}%
\definecolor{currentstroke}{rgb}{0.121569,0.466667,0.705882}%
\pgfsetstrokecolor{currentstroke}%
\pgfsetstrokeopacity{0.329047}%
\pgfsetdash{}{0pt}%
\pgfpathmoveto{\pgfqpoint{1.694212in}{3.074123in}}%
\pgfpathcurveto{\pgfqpoint{1.702448in}{3.074123in}}{\pgfqpoint{1.710348in}{3.077395in}}{\pgfqpoint{1.716172in}{3.083219in}}%
\pgfpathcurveto{\pgfqpoint{1.721996in}{3.089043in}}{\pgfqpoint{1.725268in}{3.096943in}}{\pgfqpoint{1.725268in}{3.105179in}}%
\pgfpathcurveto{\pgfqpoint{1.725268in}{3.113416in}}{\pgfqpoint{1.721996in}{3.121316in}}{\pgfqpoint{1.716172in}{3.127140in}}%
\pgfpathcurveto{\pgfqpoint{1.710348in}{3.132964in}}{\pgfqpoint{1.702448in}{3.136236in}}{\pgfqpoint{1.694212in}{3.136236in}}%
\pgfpathcurveto{\pgfqpoint{1.685975in}{3.136236in}}{\pgfqpoint{1.678075in}{3.132964in}}{\pgfqpoint{1.672251in}{3.127140in}}%
\pgfpathcurveto{\pgfqpoint{1.666427in}{3.121316in}}{\pgfqpoint{1.663155in}{3.113416in}}{\pgfqpoint{1.663155in}{3.105179in}}%
\pgfpathcurveto{\pgfqpoint{1.663155in}{3.096943in}}{\pgfqpoint{1.666427in}{3.089043in}}{\pgfqpoint{1.672251in}{3.083219in}}%
\pgfpathcurveto{\pgfqpoint{1.678075in}{3.077395in}}{\pgfqpoint{1.685975in}{3.074123in}}{\pgfqpoint{1.694212in}{3.074123in}}%
\pgfpathclose%
\pgfusepath{stroke,fill}%
\end{pgfscope}%
\begin{pgfscope}%
\pgfpathrectangle{\pgfqpoint{0.100000in}{0.220728in}}{\pgfqpoint{3.696000in}{3.696000in}}%
\pgfusepath{clip}%
\pgfsetbuttcap%
\pgfsetroundjoin%
\definecolor{currentfill}{rgb}{0.121569,0.466667,0.705882}%
\pgfsetfillcolor{currentfill}%
\pgfsetfillopacity{0.329814}%
\pgfsetlinewidth{1.003750pt}%
\definecolor{currentstroke}{rgb}{0.121569,0.466667,0.705882}%
\pgfsetstrokecolor{currentstroke}%
\pgfsetstrokeopacity{0.329814}%
\pgfsetdash{}{0pt}%
\pgfpathmoveto{\pgfqpoint{1.891653in}{3.249230in}}%
\pgfpathcurveto{\pgfqpoint{1.899889in}{3.249230in}}{\pgfqpoint{1.907789in}{3.252502in}}{\pgfqpoint{1.913613in}{3.258326in}}%
\pgfpathcurveto{\pgfqpoint{1.919437in}{3.264150in}}{\pgfqpoint{1.922709in}{3.272050in}}{\pgfqpoint{1.922709in}{3.280286in}}%
\pgfpathcurveto{\pgfqpoint{1.922709in}{3.288522in}}{\pgfqpoint{1.919437in}{3.296422in}}{\pgfqpoint{1.913613in}{3.302246in}}%
\pgfpathcurveto{\pgfqpoint{1.907789in}{3.308070in}}{\pgfqpoint{1.899889in}{3.311343in}}{\pgfqpoint{1.891653in}{3.311343in}}%
\pgfpathcurveto{\pgfqpoint{1.883417in}{3.311343in}}{\pgfqpoint{1.875517in}{3.308070in}}{\pgfqpoint{1.869693in}{3.302246in}}%
\pgfpathcurveto{\pgfqpoint{1.863869in}{3.296422in}}{\pgfqpoint{1.860596in}{3.288522in}}{\pgfqpoint{1.860596in}{3.280286in}}%
\pgfpathcurveto{\pgfqpoint{1.860596in}{3.272050in}}{\pgfqpoint{1.863869in}{3.264150in}}{\pgfqpoint{1.869693in}{3.258326in}}%
\pgfpathcurveto{\pgfqpoint{1.875517in}{3.252502in}}{\pgfqpoint{1.883417in}{3.249230in}}{\pgfqpoint{1.891653in}{3.249230in}}%
\pgfpathclose%
\pgfusepath{stroke,fill}%
\end{pgfscope}%
\begin{pgfscope}%
\pgfpathrectangle{\pgfqpoint{0.100000in}{0.220728in}}{\pgfqpoint{3.696000in}{3.696000in}}%
\pgfusepath{clip}%
\pgfsetbuttcap%
\pgfsetroundjoin%
\definecolor{currentfill}{rgb}{0.121569,0.466667,0.705882}%
\pgfsetfillcolor{currentfill}%
\pgfsetfillopacity{0.330296}%
\pgfsetlinewidth{1.003750pt}%
\definecolor{currentstroke}{rgb}{0.121569,0.466667,0.705882}%
\pgfsetstrokecolor{currentstroke}%
\pgfsetstrokeopacity{0.330296}%
\pgfsetdash{}{0pt}%
\pgfpathmoveto{\pgfqpoint{1.692149in}{3.067392in}}%
\pgfpathcurveto{\pgfqpoint{1.700385in}{3.067392in}}{\pgfqpoint{1.708285in}{3.070665in}}{\pgfqpoint{1.714109in}{3.076489in}}%
\pgfpathcurveto{\pgfqpoint{1.719933in}{3.082312in}}{\pgfqpoint{1.723206in}{3.090213in}}{\pgfqpoint{1.723206in}{3.098449in}}%
\pgfpathcurveto{\pgfqpoint{1.723206in}{3.106685in}}{\pgfqpoint{1.719933in}{3.114585in}}{\pgfqpoint{1.714109in}{3.120409in}}%
\pgfpathcurveto{\pgfqpoint{1.708285in}{3.126233in}}{\pgfqpoint{1.700385in}{3.129505in}}{\pgfqpoint{1.692149in}{3.129505in}}%
\pgfpathcurveto{\pgfqpoint{1.683913in}{3.129505in}}{\pgfqpoint{1.676013in}{3.126233in}}{\pgfqpoint{1.670189in}{3.120409in}}%
\pgfpathcurveto{\pgfqpoint{1.664365in}{3.114585in}}{\pgfqpoint{1.661093in}{3.106685in}}{\pgfqpoint{1.661093in}{3.098449in}}%
\pgfpathcurveto{\pgfqpoint{1.661093in}{3.090213in}}{\pgfqpoint{1.664365in}{3.082312in}}{\pgfqpoint{1.670189in}{3.076489in}}%
\pgfpathcurveto{\pgfqpoint{1.676013in}{3.070665in}}{\pgfqpoint{1.683913in}{3.067392in}}{\pgfqpoint{1.692149in}{3.067392in}}%
\pgfpathclose%
\pgfusepath{stroke,fill}%
\end{pgfscope}%
\begin{pgfscope}%
\pgfpathrectangle{\pgfqpoint{0.100000in}{0.220728in}}{\pgfqpoint{3.696000in}{3.696000in}}%
\pgfusepath{clip}%
\pgfsetbuttcap%
\pgfsetroundjoin%
\definecolor{currentfill}{rgb}{0.121569,0.466667,0.705882}%
\pgfsetfillcolor{currentfill}%
\pgfsetfillopacity{0.331009}%
\pgfsetlinewidth{1.003750pt}%
\definecolor{currentstroke}{rgb}{0.121569,0.466667,0.705882}%
\pgfsetstrokecolor{currentstroke}%
\pgfsetstrokeopacity{0.331009}%
\pgfsetdash{}{0pt}%
\pgfpathmoveto{\pgfqpoint{1.689274in}{3.063419in}}%
\pgfpathcurveto{\pgfqpoint{1.697510in}{3.063419in}}{\pgfqpoint{1.705410in}{3.066691in}}{\pgfqpoint{1.711234in}{3.072515in}}%
\pgfpathcurveto{\pgfqpoint{1.717058in}{3.078339in}}{\pgfqpoint{1.720331in}{3.086239in}}{\pgfqpoint{1.720331in}{3.094475in}}%
\pgfpathcurveto{\pgfqpoint{1.720331in}{3.102711in}}{\pgfqpoint{1.717058in}{3.110611in}}{\pgfqpoint{1.711234in}{3.116435in}}%
\pgfpathcurveto{\pgfqpoint{1.705410in}{3.122259in}}{\pgfqpoint{1.697510in}{3.125532in}}{\pgfqpoint{1.689274in}{3.125532in}}%
\pgfpathcurveto{\pgfqpoint{1.681038in}{3.125532in}}{\pgfqpoint{1.673138in}{3.122259in}}{\pgfqpoint{1.667314in}{3.116435in}}%
\pgfpathcurveto{\pgfqpoint{1.661490in}{3.110611in}}{\pgfqpoint{1.658218in}{3.102711in}}{\pgfqpoint{1.658218in}{3.094475in}}%
\pgfpathcurveto{\pgfqpoint{1.658218in}{3.086239in}}{\pgfqpoint{1.661490in}{3.078339in}}{\pgfqpoint{1.667314in}{3.072515in}}%
\pgfpathcurveto{\pgfqpoint{1.673138in}{3.066691in}}{\pgfqpoint{1.681038in}{3.063419in}}{\pgfqpoint{1.689274in}{3.063419in}}%
\pgfpathclose%
\pgfusepath{stroke,fill}%
\end{pgfscope}%
\begin{pgfscope}%
\pgfpathrectangle{\pgfqpoint{0.100000in}{0.220728in}}{\pgfqpoint{3.696000in}{3.696000in}}%
\pgfusepath{clip}%
\pgfsetbuttcap%
\pgfsetroundjoin%
\definecolor{currentfill}{rgb}{0.121569,0.466667,0.705882}%
\pgfsetfillcolor{currentfill}%
\pgfsetfillopacity{0.331570}%
\pgfsetlinewidth{1.003750pt}%
\definecolor{currentstroke}{rgb}{0.121569,0.466667,0.705882}%
\pgfsetstrokecolor{currentstroke}%
\pgfsetstrokeopacity{0.331570}%
\pgfsetdash{}{0pt}%
\pgfpathmoveto{\pgfqpoint{1.688913in}{3.059516in}}%
\pgfpathcurveto{\pgfqpoint{1.697150in}{3.059516in}}{\pgfqpoint{1.705050in}{3.062789in}}{\pgfqpoint{1.710874in}{3.068613in}}%
\pgfpathcurveto{\pgfqpoint{1.716697in}{3.074436in}}{\pgfqpoint{1.719970in}{3.082336in}}{\pgfqpoint{1.719970in}{3.090573in}}%
\pgfpathcurveto{\pgfqpoint{1.719970in}{3.098809in}}{\pgfqpoint{1.716697in}{3.106709in}}{\pgfqpoint{1.710874in}{3.112533in}}%
\pgfpathcurveto{\pgfqpoint{1.705050in}{3.118357in}}{\pgfqpoint{1.697150in}{3.121629in}}{\pgfqpoint{1.688913in}{3.121629in}}%
\pgfpathcurveto{\pgfqpoint{1.680677in}{3.121629in}}{\pgfqpoint{1.672777in}{3.118357in}}{\pgfqpoint{1.666953in}{3.112533in}}%
\pgfpathcurveto{\pgfqpoint{1.661129in}{3.106709in}}{\pgfqpoint{1.657857in}{3.098809in}}{\pgfqpoint{1.657857in}{3.090573in}}%
\pgfpathcurveto{\pgfqpoint{1.657857in}{3.082336in}}{\pgfqpoint{1.661129in}{3.074436in}}{\pgfqpoint{1.666953in}{3.068613in}}%
\pgfpathcurveto{\pgfqpoint{1.672777in}{3.062789in}}{\pgfqpoint{1.680677in}{3.059516in}}{\pgfqpoint{1.688913in}{3.059516in}}%
\pgfpathclose%
\pgfusepath{stroke,fill}%
\end{pgfscope}%
\begin{pgfscope}%
\pgfpathrectangle{\pgfqpoint{0.100000in}{0.220728in}}{\pgfqpoint{3.696000in}{3.696000in}}%
\pgfusepath{clip}%
\pgfsetbuttcap%
\pgfsetroundjoin%
\definecolor{currentfill}{rgb}{0.121569,0.466667,0.705882}%
\pgfsetfillcolor{currentfill}%
\pgfsetfillopacity{0.331703}%
\pgfsetlinewidth{1.003750pt}%
\definecolor{currentstroke}{rgb}{0.121569,0.466667,0.705882}%
\pgfsetstrokecolor{currentstroke}%
\pgfsetstrokeopacity{0.331703}%
\pgfsetdash{}{0pt}%
\pgfpathmoveto{\pgfqpoint{1.688076in}{3.058558in}}%
\pgfpathcurveto{\pgfqpoint{1.696312in}{3.058558in}}{\pgfqpoint{1.704212in}{3.061830in}}{\pgfqpoint{1.710036in}{3.067654in}}%
\pgfpathcurveto{\pgfqpoint{1.715860in}{3.073478in}}{\pgfqpoint{1.719133in}{3.081378in}}{\pgfqpoint{1.719133in}{3.089614in}}%
\pgfpathcurveto{\pgfqpoint{1.719133in}{3.097851in}}{\pgfqpoint{1.715860in}{3.105751in}}{\pgfqpoint{1.710036in}{3.111575in}}%
\pgfpathcurveto{\pgfqpoint{1.704212in}{3.117399in}}{\pgfqpoint{1.696312in}{3.120671in}}{\pgfqpoint{1.688076in}{3.120671in}}%
\pgfpathcurveto{\pgfqpoint{1.679840in}{3.120671in}}{\pgfqpoint{1.671940in}{3.117399in}}{\pgfqpoint{1.666116in}{3.111575in}}%
\pgfpathcurveto{\pgfqpoint{1.660292in}{3.105751in}}{\pgfqpoint{1.657020in}{3.097851in}}{\pgfqpoint{1.657020in}{3.089614in}}%
\pgfpathcurveto{\pgfqpoint{1.657020in}{3.081378in}}{\pgfqpoint{1.660292in}{3.073478in}}{\pgfqpoint{1.666116in}{3.067654in}}%
\pgfpathcurveto{\pgfqpoint{1.671940in}{3.061830in}}{\pgfqpoint{1.679840in}{3.058558in}}{\pgfqpoint{1.688076in}{3.058558in}}%
\pgfpathclose%
\pgfusepath{stroke,fill}%
\end{pgfscope}%
\begin{pgfscope}%
\pgfpathrectangle{\pgfqpoint{0.100000in}{0.220728in}}{\pgfqpoint{3.696000in}{3.696000in}}%
\pgfusepath{clip}%
\pgfsetbuttcap%
\pgfsetroundjoin%
\definecolor{currentfill}{rgb}{0.121569,0.466667,0.705882}%
\pgfsetfillcolor{currentfill}%
\pgfsetfillopacity{0.331718}%
\pgfsetlinewidth{1.003750pt}%
\definecolor{currentstroke}{rgb}{0.121569,0.466667,0.705882}%
\pgfsetstrokecolor{currentstroke}%
\pgfsetstrokeopacity{0.331718}%
\pgfsetdash{}{0pt}%
\pgfpathmoveto{\pgfqpoint{1.898925in}{3.248858in}}%
\pgfpathcurveto{\pgfqpoint{1.907161in}{3.248858in}}{\pgfqpoint{1.915061in}{3.252130in}}{\pgfqpoint{1.920885in}{3.257954in}}%
\pgfpathcurveto{\pgfqpoint{1.926709in}{3.263778in}}{\pgfqpoint{1.929982in}{3.271678in}}{\pgfqpoint{1.929982in}{3.279915in}}%
\pgfpathcurveto{\pgfqpoint{1.929982in}{3.288151in}}{\pgfqpoint{1.926709in}{3.296051in}}{\pgfqpoint{1.920885in}{3.301875in}}%
\pgfpathcurveto{\pgfqpoint{1.915061in}{3.307699in}}{\pgfqpoint{1.907161in}{3.310971in}}{\pgfqpoint{1.898925in}{3.310971in}}%
\pgfpathcurveto{\pgfqpoint{1.890689in}{3.310971in}}{\pgfqpoint{1.882789in}{3.307699in}}{\pgfqpoint{1.876965in}{3.301875in}}%
\pgfpathcurveto{\pgfqpoint{1.871141in}{3.296051in}}{\pgfqpoint{1.867869in}{3.288151in}}{\pgfqpoint{1.867869in}{3.279915in}}%
\pgfpathcurveto{\pgfqpoint{1.867869in}{3.271678in}}{\pgfqpoint{1.871141in}{3.263778in}}{\pgfqpoint{1.876965in}{3.257954in}}%
\pgfpathcurveto{\pgfqpoint{1.882789in}{3.252130in}}{\pgfqpoint{1.890689in}{3.248858in}}{\pgfqpoint{1.898925in}{3.248858in}}%
\pgfpathclose%
\pgfusepath{stroke,fill}%
\end{pgfscope}%
\begin{pgfscope}%
\pgfpathrectangle{\pgfqpoint{0.100000in}{0.220728in}}{\pgfqpoint{3.696000in}{3.696000in}}%
\pgfusepath{clip}%
\pgfsetbuttcap%
\pgfsetroundjoin%
\definecolor{currentfill}{rgb}{0.121569,0.466667,0.705882}%
\pgfsetfillcolor{currentfill}%
\pgfsetfillopacity{0.332095}%
\pgfsetlinewidth{1.003750pt}%
\definecolor{currentstroke}{rgb}{0.121569,0.466667,0.705882}%
\pgfsetstrokecolor{currentstroke}%
\pgfsetstrokeopacity{0.332095}%
\pgfsetdash{}{0pt}%
\pgfpathmoveto{\pgfqpoint{1.687536in}{3.056282in}}%
\pgfpathcurveto{\pgfqpoint{1.695772in}{3.056282in}}{\pgfqpoint{1.703672in}{3.059554in}}{\pgfqpoint{1.709496in}{3.065378in}}%
\pgfpathcurveto{\pgfqpoint{1.715320in}{3.071202in}}{\pgfqpoint{1.718592in}{3.079102in}}{\pgfqpoint{1.718592in}{3.087338in}}%
\pgfpathcurveto{\pgfqpoint{1.718592in}{3.095574in}}{\pgfqpoint{1.715320in}{3.103474in}}{\pgfqpoint{1.709496in}{3.109298in}}%
\pgfpathcurveto{\pgfqpoint{1.703672in}{3.115122in}}{\pgfqpoint{1.695772in}{3.118395in}}{\pgfqpoint{1.687536in}{3.118395in}}%
\pgfpathcurveto{\pgfqpoint{1.679299in}{3.118395in}}{\pgfqpoint{1.671399in}{3.115122in}}{\pgfqpoint{1.665575in}{3.109298in}}%
\pgfpathcurveto{\pgfqpoint{1.659752in}{3.103474in}}{\pgfqpoint{1.656479in}{3.095574in}}{\pgfqpoint{1.656479in}{3.087338in}}%
\pgfpathcurveto{\pgfqpoint{1.656479in}{3.079102in}}{\pgfqpoint{1.659752in}{3.071202in}}{\pgfqpoint{1.665575in}{3.065378in}}%
\pgfpathcurveto{\pgfqpoint{1.671399in}{3.059554in}}{\pgfqpoint{1.679299in}{3.056282in}}{\pgfqpoint{1.687536in}{3.056282in}}%
\pgfpathclose%
\pgfusepath{stroke,fill}%
\end{pgfscope}%
\begin{pgfscope}%
\pgfpathrectangle{\pgfqpoint{0.100000in}{0.220728in}}{\pgfqpoint{3.696000in}{3.696000in}}%
\pgfusepath{clip}%
\pgfsetbuttcap%
\pgfsetroundjoin%
\definecolor{currentfill}{rgb}{0.121569,0.466667,0.705882}%
\pgfsetfillcolor{currentfill}%
\pgfsetfillopacity{0.332806}%
\pgfsetlinewidth{1.003750pt}%
\definecolor{currentstroke}{rgb}{0.121569,0.466667,0.705882}%
\pgfsetstrokecolor{currentstroke}%
\pgfsetstrokeopacity{0.332806}%
\pgfsetdash{}{0pt}%
\pgfpathmoveto{\pgfqpoint{1.686221in}{3.052311in}}%
\pgfpathcurveto{\pgfqpoint{1.694457in}{3.052311in}}{\pgfqpoint{1.702357in}{3.055584in}}{\pgfqpoint{1.708181in}{3.061408in}}%
\pgfpathcurveto{\pgfqpoint{1.714005in}{3.067232in}}{\pgfqpoint{1.717278in}{3.075132in}}{\pgfqpoint{1.717278in}{3.083368in}}%
\pgfpathcurveto{\pgfqpoint{1.717278in}{3.091604in}}{\pgfqpoint{1.714005in}{3.099504in}}{\pgfqpoint{1.708181in}{3.105328in}}%
\pgfpathcurveto{\pgfqpoint{1.702357in}{3.111152in}}{\pgfqpoint{1.694457in}{3.114424in}}{\pgfqpoint{1.686221in}{3.114424in}}%
\pgfpathcurveto{\pgfqpoint{1.677985in}{3.114424in}}{\pgfqpoint{1.670085in}{3.111152in}}{\pgfqpoint{1.664261in}{3.105328in}}%
\pgfpathcurveto{\pgfqpoint{1.658437in}{3.099504in}}{\pgfqpoint{1.655165in}{3.091604in}}{\pgfqpoint{1.655165in}{3.083368in}}%
\pgfpathcurveto{\pgfqpoint{1.655165in}{3.075132in}}{\pgfqpoint{1.658437in}{3.067232in}}{\pgfqpoint{1.664261in}{3.061408in}}%
\pgfpathcurveto{\pgfqpoint{1.670085in}{3.055584in}}{\pgfqpoint{1.677985in}{3.052311in}}{\pgfqpoint{1.686221in}{3.052311in}}%
\pgfpathclose%
\pgfusepath{stroke,fill}%
\end{pgfscope}%
\begin{pgfscope}%
\pgfpathrectangle{\pgfqpoint{0.100000in}{0.220728in}}{\pgfqpoint{3.696000in}{3.696000in}}%
\pgfusepath{clip}%
\pgfsetbuttcap%
\pgfsetroundjoin%
\definecolor{currentfill}{rgb}{0.121569,0.466667,0.705882}%
\pgfsetfillcolor{currentfill}%
\pgfsetfillopacity{0.333861}%
\pgfsetlinewidth{1.003750pt}%
\definecolor{currentstroke}{rgb}{0.121569,0.466667,0.705882}%
\pgfsetstrokecolor{currentstroke}%
\pgfsetstrokeopacity{0.333861}%
\pgfsetdash{}{0pt}%
\pgfpathmoveto{\pgfqpoint{1.682133in}{3.045835in}}%
\pgfpathcurveto{\pgfqpoint{1.690369in}{3.045835in}}{\pgfqpoint{1.698269in}{3.049107in}}{\pgfqpoint{1.704093in}{3.054931in}}%
\pgfpathcurveto{\pgfqpoint{1.709917in}{3.060755in}}{\pgfqpoint{1.713189in}{3.068655in}}{\pgfqpoint{1.713189in}{3.076891in}}%
\pgfpathcurveto{\pgfqpoint{1.713189in}{3.085128in}}{\pgfqpoint{1.709917in}{3.093028in}}{\pgfqpoint{1.704093in}{3.098852in}}%
\pgfpathcurveto{\pgfqpoint{1.698269in}{3.104676in}}{\pgfqpoint{1.690369in}{3.107948in}}{\pgfqpoint{1.682133in}{3.107948in}}%
\pgfpathcurveto{\pgfqpoint{1.673897in}{3.107948in}}{\pgfqpoint{1.665996in}{3.104676in}}{\pgfqpoint{1.660173in}{3.098852in}}%
\pgfpathcurveto{\pgfqpoint{1.654349in}{3.093028in}}{\pgfqpoint{1.651076in}{3.085128in}}{\pgfqpoint{1.651076in}{3.076891in}}%
\pgfpathcurveto{\pgfqpoint{1.651076in}{3.068655in}}{\pgfqpoint{1.654349in}{3.060755in}}{\pgfqpoint{1.660173in}{3.054931in}}%
\pgfpathcurveto{\pgfqpoint{1.665996in}{3.049107in}}{\pgfqpoint{1.673897in}{3.045835in}}{\pgfqpoint{1.682133in}{3.045835in}}%
\pgfpathclose%
\pgfusepath{stroke,fill}%
\end{pgfscope}%
\begin{pgfscope}%
\pgfpathrectangle{\pgfqpoint{0.100000in}{0.220728in}}{\pgfqpoint{3.696000in}{3.696000in}}%
\pgfusepath{clip}%
\pgfsetbuttcap%
\pgfsetroundjoin%
\definecolor{currentfill}{rgb}{0.121569,0.466667,0.705882}%
\pgfsetfillcolor{currentfill}%
\pgfsetfillopacity{0.334331}%
\pgfsetlinewidth{1.003750pt}%
\definecolor{currentstroke}{rgb}{0.121569,0.466667,0.705882}%
\pgfsetstrokecolor{currentstroke}%
\pgfsetstrokeopacity{0.334331}%
\pgfsetdash{}{0pt}%
\pgfpathmoveto{\pgfqpoint{1.906692in}{3.248944in}}%
\pgfpathcurveto{\pgfqpoint{1.914929in}{3.248944in}}{\pgfqpoint{1.922829in}{3.252217in}}{\pgfqpoint{1.928653in}{3.258040in}}%
\pgfpathcurveto{\pgfqpoint{1.934477in}{3.263864in}}{\pgfqpoint{1.937749in}{3.271764in}}{\pgfqpoint{1.937749in}{3.280001in}}%
\pgfpathcurveto{\pgfqpoint{1.937749in}{3.288237in}}{\pgfqpoint{1.934477in}{3.296137in}}{\pgfqpoint{1.928653in}{3.301961in}}%
\pgfpathcurveto{\pgfqpoint{1.922829in}{3.307785in}}{\pgfqpoint{1.914929in}{3.311057in}}{\pgfqpoint{1.906692in}{3.311057in}}%
\pgfpathcurveto{\pgfqpoint{1.898456in}{3.311057in}}{\pgfqpoint{1.890556in}{3.307785in}}{\pgfqpoint{1.884732in}{3.301961in}}%
\pgfpathcurveto{\pgfqpoint{1.878908in}{3.296137in}}{\pgfqpoint{1.875636in}{3.288237in}}{\pgfqpoint{1.875636in}{3.280001in}}%
\pgfpathcurveto{\pgfqpoint{1.875636in}{3.271764in}}{\pgfqpoint{1.878908in}{3.263864in}}{\pgfqpoint{1.884732in}{3.258040in}}%
\pgfpathcurveto{\pgfqpoint{1.890556in}{3.252217in}}{\pgfqpoint{1.898456in}{3.248944in}}{\pgfqpoint{1.906692in}{3.248944in}}%
\pgfpathclose%
\pgfusepath{stroke,fill}%
\end{pgfscope}%
\begin{pgfscope}%
\pgfpathrectangle{\pgfqpoint{0.100000in}{0.220728in}}{\pgfqpoint{3.696000in}{3.696000in}}%
\pgfusepath{clip}%
\pgfsetbuttcap%
\pgfsetroundjoin%
\definecolor{currentfill}{rgb}{0.121569,0.466667,0.705882}%
\pgfsetfillcolor{currentfill}%
\pgfsetfillopacity{0.334836}%
\pgfsetlinewidth{1.003750pt}%
\definecolor{currentstroke}{rgb}{0.121569,0.466667,0.705882}%
\pgfsetstrokecolor{currentstroke}%
\pgfsetstrokeopacity{0.334836}%
\pgfsetdash{}{0pt}%
\pgfpathmoveto{\pgfqpoint{1.681508in}{3.039923in}}%
\pgfpathcurveto{\pgfqpoint{1.689744in}{3.039923in}}{\pgfqpoint{1.697644in}{3.043196in}}{\pgfqpoint{1.703468in}{3.049019in}}%
\pgfpathcurveto{\pgfqpoint{1.709292in}{3.054843in}}{\pgfqpoint{1.712564in}{3.062743in}}{\pgfqpoint{1.712564in}{3.070980in}}%
\pgfpathcurveto{\pgfqpoint{1.712564in}{3.079216in}}{\pgfqpoint{1.709292in}{3.087116in}}{\pgfqpoint{1.703468in}{3.092940in}}%
\pgfpathcurveto{\pgfqpoint{1.697644in}{3.098764in}}{\pgfqpoint{1.689744in}{3.102036in}}{\pgfqpoint{1.681508in}{3.102036in}}%
\pgfpathcurveto{\pgfqpoint{1.673272in}{3.102036in}}{\pgfqpoint{1.665371in}{3.098764in}}{\pgfqpoint{1.659548in}{3.092940in}}%
\pgfpathcurveto{\pgfqpoint{1.653724in}{3.087116in}}{\pgfqpoint{1.650451in}{3.079216in}}{\pgfqpoint{1.650451in}{3.070980in}}%
\pgfpathcurveto{\pgfqpoint{1.650451in}{3.062743in}}{\pgfqpoint{1.653724in}{3.054843in}}{\pgfqpoint{1.659548in}{3.049019in}}%
\pgfpathcurveto{\pgfqpoint{1.665371in}{3.043196in}}{\pgfqpoint{1.673272in}{3.039923in}}{\pgfqpoint{1.681508in}{3.039923in}}%
\pgfpathclose%
\pgfusepath{stroke,fill}%
\end{pgfscope}%
\begin{pgfscope}%
\pgfpathrectangle{\pgfqpoint{0.100000in}{0.220728in}}{\pgfqpoint{3.696000in}{3.696000in}}%
\pgfusepath{clip}%
\pgfsetbuttcap%
\pgfsetroundjoin%
\definecolor{currentfill}{rgb}{0.121569,0.466667,0.705882}%
\pgfsetfillcolor{currentfill}%
\pgfsetfillopacity{0.335235}%
\pgfsetlinewidth{1.003750pt}%
\definecolor{currentstroke}{rgb}{0.121569,0.466667,0.705882}%
\pgfsetstrokecolor{currentstroke}%
\pgfsetstrokeopacity{0.335235}%
\pgfsetdash{}{0pt}%
\pgfpathmoveto{\pgfqpoint{1.679590in}{3.037247in}}%
\pgfpathcurveto{\pgfqpoint{1.687827in}{3.037247in}}{\pgfqpoint{1.695727in}{3.040520in}}{\pgfqpoint{1.701550in}{3.046344in}}%
\pgfpathcurveto{\pgfqpoint{1.707374in}{3.052168in}}{\pgfqpoint{1.710647in}{3.060068in}}{\pgfqpoint{1.710647in}{3.068304in}}%
\pgfpathcurveto{\pgfqpoint{1.710647in}{3.076540in}}{\pgfqpoint{1.707374in}{3.084440in}}{\pgfqpoint{1.701550in}{3.090264in}}%
\pgfpathcurveto{\pgfqpoint{1.695727in}{3.096088in}}{\pgfqpoint{1.687827in}{3.099360in}}{\pgfqpoint{1.679590in}{3.099360in}}%
\pgfpathcurveto{\pgfqpoint{1.671354in}{3.099360in}}{\pgfqpoint{1.663454in}{3.096088in}}{\pgfqpoint{1.657630in}{3.090264in}}%
\pgfpathcurveto{\pgfqpoint{1.651806in}{3.084440in}}{\pgfqpoint{1.648534in}{3.076540in}}{\pgfqpoint{1.648534in}{3.068304in}}%
\pgfpathcurveto{\pgfqpoint{1.648534in}{3.060068in}}{\pgfqpoint{1.651806in}{3.052168in}}{\pgfqpoint{1.657630in}{3.046344in}}%
\pgfpathcurveto{\pgfqpoint{1.663454in}{3.040520in}}{\pgfqpoint{1.671354in}{3.037247in}}{\pgfqpoint{1.679590in}{3.037247in}}%
\pgfpathclose%
\pgfusepath{stroke,fill}%
\end{pgfscope}%
\begin{pgfscope}%
\pgfpathrectangle{\pgfqpoint{0.100000in}{0.220728in}}{\pgfqpoint{3.696000in}{3.696000in}}%
\pgfusepath{clip}%
\pgfsetbuttcap%
\pgfsetroundjoin%
\definecolor{currentfill}{rgb}{0.121569,0.466667,0.705882}%
\pgfsetfillcolor{currentfill}%
\pgfsetfillopacity{0.335448}%
\pgfsetlinewidth{1.003750pt}%
\definecolor{currentstroke}{rgb}{0.121569,0.466667,0.705882}%
\pgfsetstrokecolor{currentstroke}%
\pgfsetstrokeopacity{0.335448}%
\pgfsetdash{}{0pt}%
\pgfpathmoveto{\pgfqpoint{1.679341in}{3.036079in}}%
\pgfpathcurveto{\pgfqpoint{1.687578in}{3.036079in}}{\pgfqpoint{1.695478in}{3.039351in}}{\pgfqpoint{1.701302in}{3.045175in}}%
\pgfpathcurveto{\pgfqpoint{1.707126in}{3.050999in}}{\pgfqpoint{1.710398in}{3.058899in}}{\pgfqpoint{1.710398in}{3.067135in}}%
\pgfpathcurveto{\pgfqpoint{1.710398in}{3.075372in}}{\pgfqpoint{1.707126in}{3.083272in}}{\pgfqpoint{1.701302in}{3.089096in}}%
\pgfpathcurveto{\pgfqpoint{1.695478in}{3.094920in}}{\pgfqpoint{1.687578in}{3.098192in}}{\pgfqpoint{1.679341in}{3.098192in}}%
\pgfpathcurveto{\pgfqpoint{1.671105in}{3.098192in}}{\pgfqpoint{1.663205in}{3.094920in}}{\pgfqpoint{1.657381in}{3.089096in}}%
\pgfpathcurveto{\pgfqpoint{1.651557in}{3.083272in}}{\pgfqpoint{1.648285in}{3.075372in}}{\pgfqpoint{1.648285in}{3.067135in}}%
\pgfpathcurveto{\pgfqpoint{1.648285in}{3.058899in}}{\pgfqpoint{1.651557in}{3.050999in}}{\pgfqpoint{1.657381in}{3.045175in}}%
\pgfpathcurveto{\pgfqpoint{1.663205in}{3.039351in}}{\pgfqpoint{1.671105in}{3.036079in}}{\pgfqpoint{1.679341in}{3.036079in}}%
\pgfpathclose%
\pgfusepath{stroke,fill}%
\end{pgfscope}%
\begin{pgfscope}%
\pgfpathrectangle{\pgfqpoint{0.100000in}{0.220728in}}{\pgfqpoint{3.696000in}{3.696000in}}%
\pgfusepath{clip}%
\pgfsetbuttcap%
\pgfsetroundjoin%
\definecolor{currentfill}{rgb}{0.121569,0.466667,0.705882}%
\pgfsetfillcolor{currentfill}%
\pgfsetfillopacity{0.335697}%
\pgfsetlinewidth{1.003750pt}%
\definecolor{currentstroke}{rgb}{0.121569,0.466667,0.705882}%
\pgfsetstrokecolor{currentstroke}%
\pgfsetstrokeopacity{0.335697}%
\pgfsetdash{}{0pt}%
\pgfpathmoveto{\pgfqpoint{1.678008in}{3.034259in}}%
\pgfpathcurveto{\pgfqpoint{1.686244in}{3.034259in}}{\pgfqpoint{1.694144in}{3.037531in}}{\pgfqpoint{1.699968in}{3.043355in}}%
\pgfpathcurveto{\pgfqpoint{1.705792in}{3.049179in}}{\pgfqpoint{1.709065in}{3.057079in}}{\pgfqpoint{1.709065in}{3.065316in}}%
\pgfpathcurveto{\pgfqpoint{1.709065in}{3.073552in}}{\pgfqpoint{1.705792in}{3.081452in}}{\pgfqpoint{1.699968in}{3.087276in}}%
\pgfpathcurveto{\pgfqpoint{1.694144in}{3.093100in}}{\pgfqpoint{1.686244in}{3.096372in}}{\pgfqpoint{1.678008in}{3.096372in}}%
\pgfpathcurveto{\pgfqpoint{1.669772in}{3.096372in}}{\pgfqpoint{1.661872in}{3.093100in}}{\pgfqpoint{1.656048in}{3.087276in}}%
\pgfpathcurveto{\pgfqpoint{1.650224in}{3.081452in}}{\pgfqpoint{1.646952in}{3.073552in}}{\pgfqpoint{1.646952in}{3.065316in}}%
\pgfpathcurveto{\pgfqpoint{1.646952in}{3.057079in}}{\pgfqpoint{1.650224in}{3.049179in}}{\pgfqpoint{1.656048in}{3.043355in}}%
\pgfpathcurveto{\pgfqpoint{1.661872in}{3.037531in}}{\pgfqpoint{1.669772in}{3.034259in}}{\pgfqpoint{1.678008in}{3.034259in}}%
\pgfpathclose%
\pgfusepath{stroke,fill}%
\end{pgfscope}%
\begin{pgfscope}%
\pgfpathrectangle{\pgfqpoint{0.100000in}{0.220728in}}{\pgfqpoint{3.696000in}{3.696000in}}%
\pgfusepath{clip}%
\pgfsetbuttcap%
\pgfsetroundjoin%
\definecolor{currentfill}{rgb}{0.121569,0.466667,0.705882}%
\pgfsetfillcolor{currentfill}%
\pgfsetfillopacity{0.336198}%
\pgfsetlinewidth{1.003750pt}%
\definecolor{currentstroke}{rgb}{0.121569,0.466667,0.705882}%
\pgfsetstrokecolor{currentstroke}%
\pgfsetstrokeopacity{0.336198}%
\pgfsetdash{}{0pt}%
\pgfpathmoveto{\pgfqpoint{1.916328in}{3.248093in}}%
\pgfpathcurveto{\pgfqpoint{1.924565in}{3.248093in}}{\pgfqpoint{1.932465in}{3.251366in}}{\pgfqpoint{1.938289in}{3.257189in}}%
\pgfpathcurveto{\pgfqpoint{1.944112in}{3.263013in}}{\pgfqpoint{1.947385in}{3.270913in}}{\pgfqpoint{1.947385in}{3.279150in}}%
\pgfpathcurveto{\pgfqpoint{1.947385in}{3.287386in}}{\pgfqpoint{1.944112in}{3.295286in}}{\pgfqpoint{1.938289in}{3.301110in}}%
\pgfpathcurveto{\pgfqpoint{1.932465in}{3.306934in}}{\pgfqpoint{1.924565in}{3.310206in}}{\pgfqpoint{1.916328in}{3.310206in}}%
\pgfpathcurveto{\pgfqpoint{1.908092in}{3.310206in}}{\pgfqpoint{1.900192in}{3.306934in}}{\pgfqpoint{1.894368in}{3.301110in}}%
\pgfpathcurveto{\pgfqpoint{1.888544in}{3.295286in}}{\pgfqpoint{1.885272in}{3.287386in}}{\pgfqpoint{1.885272in}{3.279150in}}%
\pgfpathcurveto{\pgfqpoint{1.885272in}{3.270913in}}{\pgfqpoint{1.888544in}{3.263013in}}{\pgfqpoint{1.894368in}{3.257189in}}%
\pgfpathcurveto{\pgfqpoint{1.900192in}{3.251366in}}{\pgfqpoint{1.908092in}{3.248093in}}{\pgfqpoint{1.916328in}{3.248093in}}%
\pgfpathclose%
\pgfusepath{stroke,fill}%
\end{pgfscope}%
\begin{pgfscope}%
\pgfpathrectangle{\pgfqpoint{0.100000in}{0.220728in}}{\pgfqpoint{3.696000in}{3.696000in}}%
\pgfusepath{clip}%
\pgfsetbuttcap%
\pgfsetroundjoin%
\definecolor{currentfill}{rgb}{0.121569,0.466667,0.705882}%
\pgfsetfillcolor{currentfill}%
\pgfsetfillopacity{0.336394}%
\pgfsetlinewidth{1.003750pt}%
\definecolor{currentstroke}{rgb}{0.121569,0.466667,0.705882}%
\pgfsetstrokecolor{currentstroke}%
\pgfsetstrokeopacity{0.336394}%
\pgfsetdash{}{0pt}%
\pgfpathmoveto{\pgfqpoint{1.676618in}{3.030676in}}%
\pgfpathcurveto{\pgfqpoint{1.684854in}{3.030676in}}{\pgfqpoint{1.692755in}{3.033949in}}{\pgfqpoint{1.698578in}{3.039773in}}%
\pgfpathcurveto{\pgfqpoint{1.704402in}{3.045597in}}{\pgfqpoint{1.707675in}{3.053497in}}{\pgfqpoint{1.707675in}{3.061733in}}%
\pgfpathcurveto{\pgfqpoint{1.707675in}{3.069969in}}{\pgfqpoint{1.704402in}{3.077869in}}{\pgfqpoint{1.698578in}{3.083693in}}%
\pgfpathcurveto{\pgfqpoint{1.692755in}{3.089517in}}{\pgfqpoint{1.684854in}{3.092789in}}{\pgfqpoint{1.676618in}{3.092789in}}%
\pgfpathcurveto{\pgfqpoint{1.668382in}{3.092789in}}{\pgfqpoint{1.660482in}{3.089517in}}{\pgfqpoint{1.654658in}{3.083693in}}%
\pgfpathcurveto{\pgfqpoint{1.648834in}{3.077869in}}{\pgfqpoint{1.645562in}{3.069969in}}{\pgfqpoint{1.645562in}{3.061733in}}%
\pgfpathcurveto{\pgfqpoint{1.645562in}{3.053497in}}{\pgfqpoint{1.648834in}{3.045597in}}{\pgfqpoint{1.654658in}{3.039773in}}%
\pgfpathcurveto{\pgfqpoint{1.660482in}{3.033949in}}{\pgfqpoint{1.668382in}{3.030676in}}{\pgfqpoint{1.676618in}{3.030676in}}%
\pgfpathclose%
\pgfusepath{stroke,fill}%
\end{pgfscope}%
\begin{pgfscope}%
\pgfpathrectangle{\pgfqpoint{0.100000in}{0.220728in}}{\pgfqpoint{3.696000in}{3.696000in}}%
\pgfusepath{clip}%
\pgfsetbuttcap%
\pgfsetroundjoin%
\definecolor{currentfill}{rgb}{0.121569,0.466667,0.705882}%
\pgfsetfillcolor{currentfill}%
\pgfsetfillopacity{0.337661}%
\pgfsetlinewidth{1.003750pt}%
\definecolor{currentstroke}{rgb}{0.121569,0.466667,0.705882}%
\pgfsetstrokecolor{currentstroke}%
\pgfsetstrokeopacity{0.337661}%
\pgfsetdash{}{0pt}%
\pgfpathmoveto{\pgfqpoint{1.674006in}{3.024224in}}%
\pgfpathcurveto{\pgfqpoint{1.682243in}{3.024224in}}{\pgfqpoint{1.690143in}{3.027496in}}{\pgfqpoint{1.695967in}{3.033320in}}%
\pgfpathcurveto{\pgfqpoint{1.701791in}{3.039144in}}{\pgfqpoint{1.705063in}{3.047044in}}{\pgfqpoint{1.705063in}{3.055281in}}%
\pgfpathcurveto{\pgfqpoint{1.705063in}{3.063517in}}{\pgfqpoint{1.701791in}{3.071417in}}{\pgfqpoint{1.695967in}{3.077241in}}%
\pgfpathcurveto{\pgfqpoint{1.690143in}{3.083065in}}{\pgfqpoint{1.682243in}{3.086337in}}{\pgfqpoint{1.674006in}{3.086337in}}%
\pgfpathcurveto{\pgfqpoint{1.665770in}{3.086337in}}{\pgfqpoint{1.657870in}{3.083065in}}{\pgfqpoint{1.652046in}{3.077241in}}%
\pgfpathcurveto{\pgfqpoint{1.646222in}{3.071417in}}{\pgfqpoint{1.642950in}{3.063517in}}{\pgfqpoint{1.642950in}{3.055281in}}%
\pgfpathcurveto{\pgfqpoint{1.642950in}{3.047044in}}{\pgfqpoint{1.646222in}{3.039144in}}{\pgfqpoint{1.652046in}{3.033320in}}%
\pgfpathcurveto{\pgfqpoint{1.657870in}{3.027496in}}{\pgfqpoint{1.665770in}{3.024224in}}{\pgfqpoint{1.674006in}{3.024224in}}%
\pgfpathclose%
\pgfusepath{stroke,fill}%
\end{pgfscope}%
\begin{pgfscope}%
\pgfpathrectangle{\pgfqpoint{0.100000in}{0.220728in}}{\pgfqpoint{3.696000in}{3.696000in}}%
\pgfusepath{clip}%
\pgfsetbuttcap%
\pgfsetroundjoin%
\definecolor{currentfill}{rgb}{0.121569,0.466667,0.705882}%
\pgfsetfillcolor{currentfill}%
\pgfsetfillopacity{0.339383}%
\pgfsetlinewidth{1.003750pt}%
\definecolor{currentstroke}{rgb}{0.121569,0.466667,0.705882}%
\pgfsetstrokecolor{currentstroke}%
\pgfsetstrokeopacity{0.339383}%
\pgfsetdash{}{0pt}%
\pgfpathmoveto{\pgfqpoint{1.666725in}{3.012926in}}%
\pgfpathcurveto{\pgfqpoint{1.674961in}{3.012926in}}{\pgfqpoint{1.682861in}{3.016198in}}{\pgfqpoint{1.688685in}{3.022022in}}%
\pgfpathcurveto{\pgfqpoint{1.694509in}{3.027846in}}{\pgfqpoint{1.697781in}{3.035746in}}{\pgfqpoint{1.697781in}{3.043982in}}%
\pgfpathcurveto{\pgfqpoint{1.697781in}{3.052219in}}{\pgfqpoint{1.694509in}{3.060119in}}{\pgfqpoint{1.688685in}{3.065943in}}%
\pgfpathcurveto{\pgfqpoint{1.682861in}{3.071766in}}{\pgfqpoint{1.674961in}{3.075039in}}{\pgfqpoint{1.666725in}{3.075039in}}%
\pgfpathcurveto{\pgfqpoint{1.658488in}{3.075039in}}{\pgfqpoint{1.650588in}{3.071766in}}{\pgfqpoint{1.644764in}{3.065943in}}%
\pgfpathcurveto{\pgfqpoint{1.638940in}{3.060119in}}{\pgfqpoint{1.635668in}{3.052219in}}{\pgfqpoint{1.635668in}{3.043982in}}%
\pgfpathcurveto{\pgfqpoint{1.635668in}{3.035746in}}{\pgfqpoint{1.638940in}{3.027846in}}{\pgfqpoint{1.644764in}{3.022022in}}%
\pgfpathcurveto{\pgfqpoint{1.650588in}{3.016198in}}{\pgfqpoint{1.658488in}{3.012926in}}{\pgfqpoint{1.666725in}{3.012926in}}%
\pgfpathclose%
\pgfusepath{stroke,fill}%
\end{pgfscope}%
\begin{pgfscope}%
\pgfpathrectangle{\pgfqpoint{0.100000in}{0.220728in}}{\pgfqpoint{3.696000in}{3.696000in}}%
\pgfusepath{clip}%
\pgfsetbuttcap%
\pgfsetroundjoin%
\definecolor{currentfill}{rgb}{0.121569,0.466667,0.705882}%
\pgfsetfillcolor{currentfill}%
\pgfsetfillopacity{0.340070}%
\pgfsetlinewidth{1.003750pt}%
\definecolor{currentstroke}{rgb}{0.121569,0.466667,0.705882}%
\pgfsetstrokecolor{currentstroke}%
\pgfsetstrokeopacity{0.340070}%
\pgfsetdash{}{0pt}%
\pgfpathmoveto{\pgfqpoint{1.928731in}{3.247499in}}%
\pgfpathcurveto{\pgfqpoint{1.936968in}{3.247499in}}{\pgfqpoint{1.944868in}{3.250771in}}{\pgfqpoint{1.950692in}{3.256595in}}%
\pgfpathcurveto{\pgfqpoint{1.956516in}{3.262419in}}{\pgfqpoint{1.959788in}{3.270319in}}{\pgfqpoint{1.959788in}{3.278555in}}%
\pgfpathcurveto{\pgfqpoint{1.959788in}{3.286792in}}{\pgfqpoint{1.956516in}{3.294692in}}{\pgfqpoint{1.950692in}{3.300516in}}%
\pgfpathcurveto{\pgfqpoint{1.944868in}{3.306340in}}{\pgfqpoint{1.936968in}{3.309612in}}{\pgfqpoint{1.928731in}{3.309612in}}%
\pgfpathcurveto{\pgfqpoint{1.920495in}{3.309612in}}{\pgfqpoint{1.912595in}{3.306340in}}{\pgfqpoint{1.906771in}{3.300516in}}%
\pgfpathcurveto{\pgfqpoint{1.900947in}{3.294692in}}{\pgfqpoint{1.897675in}{3.286792in}}{\pgfqpoint{1.897675in}{3.278555in}}%
\pgfpathcurveto{\pgfqpoint{1.897675in}{3.270319in}}{\pgfqpoint{1.900947in}{3.262419in}}{\pgfqpoint{1.906771in}{3.256595in}}%
\pgfpathcurveto{\pgfqpoint{1.912595in}{3.250771in}}{\pgfqpoint{1.920495in}{3.247499in}}{\pgfqpoint{1.928731in}{3.247499in}}%
\pgfpathclose%
\pgfusepath{stroke,fill}%
\end{pgfscope}%
\begin{pgfscope}%
\pgfpathrectangle{\pgfqpoint{0.100000in}{0.220728in}}{\pgfqpoint{3.696000in}{3.696000in}}%
\pgfusepath{clip}%
\pgfsetbuttcap%
\pgfsetroundjoin%
\definecolor{currentfill}{rgb}{0.121569,0.466667,0.705882}%
\pgfsetfillcolor{currentfill}%
\pgfsetfillopacity{0.341130}%
\pgfsetlinewidth{1.003750pt}%
\definecolor{currentstroke}{rgb}{0.121569,0.466667,0.705882}%
\pgfsetstrokecolor{currentstroke}%
\pgfsetstrokeopacity{0.341130}%
\pgfsetdash{}{0pt}%
\pgfpathmoveto{\pgfqpoint{1.665544in}{3.000778in}}%
\pgfpathcurveto{\pgfqpoint{1.673780in}{3.000778in}}{\pgfqpoint{1.681680in}{3.004050in}}{\pgfqpoint{1.687504in}{3.009874in}}%
\pgfpathcurveto{\pgfqpoint{1.693328in}{3.015698in}}{\pgfqpoint{1.696601in}{3.023598in}}{\pgfqpoint{1.696601in}{3.031835in}}%
\pgfpathcurveto{\pgfqpoint{1.696601in}{3.040071in}}{\pgfqpoint{1.693328in}{3.047971in}}{\pgfqpoint{1.687504in}{3.053795in}}%
\pgfpathcurveto{\pgfqpoint{1.681680in}{3.059619in}}{\pgfqpoint{1.673780in}{3.062891in}}{\pgfqpoint{1.665544in}{3.062891in}}%
\pgfpathcurveto{\pgfqpoint{1.657308in}{3.062891in}}{\pgfqpoint{1.649408in}{3.059619in}}{\pgfqpoint{1.643584in}{3.053795in}}%
\pgfpathcurveto{\pgfqpoint{1.637760in}{3.047971in}}{\pgfqpoint{1.634488in}{3.040071in}}{\pgfqpoint{1.634488in}{3.031835in}}%
\pgfpathcurveto{\pgfqpoint{1.634488in}{3.023598in}}{\pgfqpoint{1.637760in}{3.015698in}}{\pgfqpoint{1.643584in}{3.009874in}}%
\pgfpathcurveto{\pgfqpoint{1.649408in}{3.004050in}}{\pgfqpoint{1.657308in}{3.000778in}}{\pgfqpoint{1.665544in}{3.000778in}}%
\pgfpathclose%
\pgfusepath{stroke,fill}%
\end{pgfscope}%
\begin{pgfscope}%
\pgfpathrectangle{\pgfqpoint{0.100000in}{0.220728in}}{\pgfqpoint{3.696000in}{3.696000in}}%
\pgfusepath{clip}%
\pgfsetbuttcap%
\pgfsetroundjoin%
\definecolor{currentfill}{rgb}{0.121569,0.466667,0.705882}%
\pgfsetfillcolor{currentfill}%
\pgfsetfillopacity{0.342084}%
\pgfsetlinewidth{1.003750pt}%
\definecolor{currentstroke}{rgb}{0.121569,0.466667,0.705882}%
\pgfsetstrokecolor{currentstroke}%
\pgfsetstrokeopacity{0.342084}%
\pgfsetdash{}{0pt}%
\pgfpathmoveto{\pgfqpoint{1.659573in}{2.993006in}}%
\pgfpathcurveto{\pgfqpoint{1.667809in}{2.993006in}}{\pgfqpoint{1.675709in}{2.996279in}}{\pgfqpoint{1.681533in}{3.002103in}}%
\pgfpathcurveto{\pgfqpoint{1.687357in}{3.007927in}}{\pgfqpoint{1.690629in}{3.015827in}}{\pgfqpoint{1.690629in}{3.024063in}}%
\pgfpathcurveto{\pgfqpoint{1.690629in}{3.032299in}}{\pgfqpoint{1.687357in}{3.040199in}}{\pgfqpoint{1.681533in}{3.046023in}}%
\pgfpathcurveto{\pgfqpoint{1.675709in}{3.051847in}}{\pgfqpoint{1.667809in}{3.055119in}}{\pgfqpoint{1.659573in}{3.055119in}}%
\pgfpathcurveto{\pgfqpoint{1.651337in}{3.055119in}}{\pgfqpoint{1.643437in}{3.051847in}}{\pgfqpoint{1.637613in}{3.046023in}}%
\pgfpathcurveto{\pgfqpoint{1.631789in}{3.040199in}}{\pgfqpoint{1.628516in}{3.032299in}}{\pgfqpoint{1.628516in}{3.024063in}}%
\pgfpathcurveto{\pgfqpoint{1.628516in}{3.015827in}}{\pgfqpoint{1.631789in}{3.007927in}}{\pgfqpoint{1.637613in}{3.002103in}}%
\pgfpathcurveto{\pgfqpoint{1.643437in}{2.996279in}}{\pgfqpoint{1.651337in}{2.993006in}}{\pgfqpoint{1.659573in}{2.993006in}}%
\pgfpathclose%
\pgfusepath{stroke,fill}%
\end{pgfscope}%
\begin{pgfscope}%
\pgfpathrectangle{\pgfqpoint{0.100000in}{0.220728in}}{\pgfqpoint{3.696000in}{3.696000in}}%
\pgfusepath{clip}%
\pgfsetbuttcap%
\pgfsetroundjoin%
\definecolor{currentfill}{rgb}{0.121569,0.466667,0.705882}%
\pgfsetfillcolor{currentfill}%
\pgfsetfillopacity{0.343196}%
\pgfsetlinewidth{1.003750pt}%
\definecolor{currentstroke}{rgb}{0.121569,0.466667,0.705882}%
\pgfsetstrokecolor{currentstroke}%
\pgfsetstrokeopacity{0.343196}%
\pgfsetdash{}{0pt}%
\pgfpathmoveto{\pgfqpoint{1.657646in}{2.986640in}}%
\pgfpathcurveto{\pgfqpoint{1.665882in}{2.986640in}}{\pgfqpoint{1.673782in}{2.989912in}}{\pgfqpoint{1.679606in}{2.995736in}}%
\pgfpathcurveto{\pgfqpoint{1.685430in}{3.001560in}}{\pgfqpoint{1.688702in}{3.009460in}}{\pgfqpoint{1.688702in}{3.017696in}}%
\pgfpathcurveto{\pgfqpoint{1.688702in}{3.025932in}}{\pgfqpoint{1.685430in}{3.033832in}}{\pgfqpoint{1.679606in}{3.039656in}}%
\pgfpathcurveto{\pgfqpoint{1.673782in}{3.045480in}}{\pgfqpoint{1.665882in}{3.048753in}}{\pgfqpoint{1.657646in}{3.048753in}}%
\pgfpathcurveto{\pgfqpoint{1.649409in}{3.048753in}}{\pgfqpoint{1.641509in}{3.045480in}}{\pgfqpoint{1.635685in}{3.039656in}}%
\pgfpathcurveto{\pgfqpoint{1.629861in}{3.033832in}}{\pgfqpoint{1.626589in}{3.025932in}}{\pgfqpoint{1.626589in}{3.017696in}}%
\pgfpathcurveto{\pgfqpoint{1.626589in}{3.009460in}}{\pgfqpoint{1.629861in}{3.001560in}}{\pgfqpoint{1.635685in}{2.995736in}}%
\pgfpathcurveto{\pgfqpoint{1.641509in}{2.989912in}}{\pgfqpoint{1.649409in}{2.986640in}}{\pgfqpoint{1.657646in}{2.986640in}}%
\pgfpathclose%
\pgfusepath{stroke,fill}%
\end{pgfscope}%
\begin{pgfscope}%
\pgfpathrectangle{\pgfqpoint{0.100000in}{0.220728in}}{\pgfqpoint{3.696000in}{3.696000in}}%
\pgfusepath{clip}%
\pgfsetbuttcap%
\pgfsetroundjoin%
\definecolor{currentfill}{rgb}{0.121569,0.466667,0.705882}%
\pgfsetfillcolor{currentfill}%
\pgfsetfillopacity{0.343379}%
\pgfsetlinewidth{1.003750pt}%
\definecolor{currentstroke}{rgb}{0.121569,0.466667,0.705882}%
\pgfsetstrokecolor{currentstroke}%
\pgfsetstrokeopacity{0.343379}%
\pgfsetdash{}{0pt}%
\pgfpathmoveto{\pgfqpoint{1.943685in}{3.245758in}}%
\pgfpathcurveto{\pgfqpoint{1.951921in}{3.245758in}}{\pgfqpoint{1.959821in}{3.249031in}}{\pgfqpoint{1.965645in}{3.254855in}}%
\pgfpathcurveto{\pgfqpoint{1.971469in}{3.260679in}}{\pgfqpoint{1.974741in}{3.268579in}}{\pgfqpoint{1.974741in}{3.276815in}}%
\pgfpathcurveto{\pgfqpoint{1.974741in}{3.285051in}}{\pgfqpoint{1.971469in}{3.292951in}}{\pgfqpoint{1.965645in}{3.298775in}}%
\pgfpathcurveto{\pgfqpoint{1.959821in}{3.304599in}}{\pgfqpoint{1.951921in}{3.307871in}}{\pgfqpoint{1.943685in}{3.307871in}}%
\pgfpathcurveto{\pgfqpoint{1.935448in}{3.307871in}}{\pgfqpoint{1.927548in}{3.304599in}}{\pgfqpoint{1.921724in}{3.298775in}}%
\pgfpathcurveto{\pgfqpoint{1.915900in}{3.292951in}}{\pgfqpoint{1.912628in}{3.285051in}}{\pgfqpoint{1.912628in}{3.276815in}}%
\pgfpathcurveto{\pgfqpoint{1.912628in}{3.268579in}}{\pgfqpoint{1.915900in}{3.260679in}}{\pgfqpoint{1.921724in}{3.254855in}}%
\pgfpathcurveto{\pgfqpoint{1.927548in}{3.249031in}}{\pgfqpoint{1.935448in}{3.245758in}}{\pgfqpoint{1.943685in}{3.245758in}}%
\pgfpathclose%
\pgfusepath{stroke,fill}%
\end{pgfscope}%
\begin{pgfscope}%
\pgfpathrectangle{\pgfqpoint{0.100000in}{0.220728in}}{\pgfqpoint{3.696000in}{3.696000in}}%
\pgfusepath{clip}%
\pgfsetbuttcap%
\pgfsetroundjoin%
\definecolor{currentfill}{rgb}{0.121569,0.466667,0.705882}%
\pgfsetfillcolor{currentfill}%
\pgfsetfillopacity{0.343978}%
\pgfsetlinewidth{1.003750pt}%
\definecolor{currentstroke}{rgb}{0.121569,0.466667,0.705882}%
\pgfsetstrokecolor{currentstroke}%
\pgfsetstrokeopacity{0.343978}%
\pgfsetdash{}{0pt}%
\pgfpathmoveto{\pgfqpoint{1.655386in}{2.982054in}}%
\pgfpathcurveto{\pgfqpoint{1.663623in}{2.982054in}}{\pgfqpoint{1.671523in}{2.985327in}}{\pgfqpoint{1.677347in}{2.991151in}}%
\pgfpathcurveto{\pgfqpoint{1.683171in}{2.996975in}}{\pgfqpoint{1.686443in}{3.004875in}}{\pgfqpoint{1.686443in}{3.013111in}}%
\pgfpathcurveto{\pgfqpoint{1.686443in}{3.021347in}}{\pgfqpoint{1.683171in}{3.029247in}}{\pgfqpoint{1.677347in}{3.035071in}}%
\pgfpathcurveto{\pgfqpoint{1.671523in}{3.040895in}}{\pgfqpoint{1.663623in}{3.044167in}}{\pgfqpoint{1.655386in}{3.044167in}}%
\pgfpathcurveto{\pgfqpoint{1.647150in}{3.044167in}}{\pgfqpoint{1.639250in}{3.040895in}}{\pgfqpoint{1.633426in}{3.035071in}}%
\pgfpathcurveto{\pgfqpoint{1.627602in}{3.029247in}}{\pgfqpoint{1.624330in}{3.021347in}}{\pgfqpoint{1.624330in}{3.013111in}}%
\pgfpathcurveto{\pgfqpoint{1.624330in}{3.004875in}}{\pgfqpoint{1.627602in}{2.996975in}}{\pgfqpoint{1.633426in}{2.991151in}}%
\pgfpathcurveto{\pgfqpoint{1.639250in}{2.985327in}}{\pgfqpoint{1.647150in}{2.982054in}}{\pgfqpoint{1.655386in}{2.982054in}}%
\pgfpathclose%
\pgfusepath{stroke,fill}%
\end{pgfscope}%
\begin{pgfscope}%
\pgfpathrectangle{\pgfqpoint{0.100000in}{0.220728in}}{\pgfqpoint{3.696000in}{3.696000in}}%
\pgfusepath{clip}%
\pgfsetbuttcap%
\pgfsetroundjoin%
\definecolor{currentfill}{rgb}{0.121569,0.466667,0.705882}%
\pgfsetfillcolor{currentfill}%
\pgfsetfillopacity{0.344139}%
\pgfsetlinewidth{1.003750pt}%
\definecolor{currentstroke}{rgb}{0.121569,0.466667,0.705882}%
\pgfsetstrokecolor{currentstroke}%
\pgfsetstrokeopacity{0.344139}%
\pgfsetdash{}{0pt}%
\pgfpathmoveto{\pgfqpoint{1.961842in}{3.244541in}}%
\pgfpathcurveto{\pgfqpoint{1.970078in}{3.244541in}}{\pgfqpoint{1.977978in}{3.247814in}}{\pgfqpoint{1.983802in}{3.253638in}}%
\pgfpathcurveto{\pgfqpoint{1.989626in}{3.259462in}}{\pgfqpoint{1.992898in}{3.267362in}}{\pgfqpoint{1.992898in}{3.275598in}}%
\pgfpathcurveto{\pgfqpoint{1.992898in}{3.283834in}}{\pgfqpoint{1.989626in}{3.291734in}}{\pgfqpoint{1.983802in}{3.297558in}}%
\pgfpathcurveto{\pgfqpoint{1.977978in}{3.303382in}}{\pgfqpoint{1.970078in}{3.306654in}}{\pgfqpoint{1.961842in}{3.306654in}}%
\pgfpathcurveto{\pgfqpoint{1.953605in}{3.306654in}}{\pgfqpoint{1.945705in}{3.303382in}}{\pgfqpoint{1.939881in}{3.297558in}}%
\pgfpathcurveto{\pgfqpoint{1.934057in}{3.291734in}}{\pgfqpoint{1.930785in}{3.283834in}}{\pgfqpoint{1.930785in}{3.275598in}}%
\pgfpathcurveto{\pgfqpoint{1.930785in}{3.267362in}}{\pgfqpoint{1.934057in}{3.259462in}}{\pgfqpoint{1.939881in}{3.253638in}}%
\pgfpathcurveto{\pgfqpoint{1.945705in}{3.247814in}}{\pgfqpoint{1.953605in}{3.244541in}}{\pgfqpoint{1.961842in}{3.244541in}}%
\pgfpathclose%
\pgfusepath{stroke,fill}%
\end{pgfscope}%
\begin{pgfscope}%
\pgfpathrectangle{\pgfqpoint{0.100000in}{0.220728in}}{\pgfqpoint{3.696000in}{3.696000in}}%
\pgfusepath{clip}%
\pgfsetbuttcap%
\pgfsetroundjoin%
\definecolor{currentfill}{rgb}{0.121569,0.466667,0.705882}%
\pgfsetfillcolor{currentfill}%
\pgfsetfillopacity{0.344614}%
\pgfsetlinewidth{1.003750pt}%
\definecolor{currentstroke}{rgb}{0.121569,0.466667,0.705882}%
\pgfsetstrokecolor{currentstroke}%
\pgfsetstrokeopacity{0.344614}%
\pgfsetdash{}{0pt}%
\pgfpathmoveto{\pgfqpoint{1.654163in}{2.978439in}}%
\pgfpathcurveto{\pgfqpoint{1.662400in}{2.978439in}}{\pgfqpoint{1.670300in}{2.981711in}}{\pgfqpoint{1.676124in}{2.987535in}}%
\pgfpathcurveto{\pgfqpoint{1.681948in}{2.993359in}}{\pgfqpoint{1.685220in}{3.001259in}}{\pgfqpoint{1.685220in}{3.009495in}}%
\pgfpathcurveto{\pgfqpoint{1.685220in}{3.017731in}}{\pgfqpoint{1.681948in}{3.025631in}}{\pgfqpoint{1.676124in}{3.031455in}}%
\pgfpathcurveto{\pgfqpoint{1.670300in}{3.037279in}}{\pgfqpoint{1.662400in}{3.040552in}}{\pgfqpoint{1.654163in}{3.040552in}}%
\pgfpathcurveto{\pgfqpoint{1.645927in}{3.040552in}}{\pgfqpoint{1.638027in}{3.037279in}}{\pgfqpoint{1.632203in}{3.031455in}}%
\pgfpathcurveto{\pgfqpoint{1.626379in}{3.025631in}}{\pgfqpoint{1.623107in}{3.017731in}}{\pgfqpoint{1.623107in}{3.009495in}}%
\pgfpathcurveto{\pgfqpoint{1.623107in}{3.001259in}}{\pgfqpoint{1.626379in}{2.993359in}}{\pgfqpoint{1.632203in}{2.987535in}}%
\pgfpathcurveto{\pgfqpoint{1.638027in}{2.981711in}}{\pgfqpoint{1.645927in}{2.978439in}}{\pgfqpoint{1.654163in}{2.978439in}}%
\pgfpathclose%
\pgfusepath{stroke,fill}%
\end{pgfscope}%
\begin{pgfscope}%
\pgfpathrectangle{\pgfqpoint{0.100000in}{0.220728in}}{\pgfqpoint{3.696000in}{3.696000in}}%
\pgfusepath{clip}%
\pgfsetbuttcap%
\pgfsetroundjoin%
\definecolor{currentfill}{rgb}{0.121569,0.466667,0.705882}%
\pgfsetfillcolor{currentfill}%
\pgfsetfillopacity{0.345827}%
\pgfsetlinewidth{1.003750pt}%
\definecolor{currentstroke}{rgb}{0.121569,0.466667,0.705882}%
\pgfsetstrokecolor{currentstroke}%
\pgfsetstrokeopacity{0.345827}%
\pgfsetdash{}{0pt}%
\pgfpathmoveto{\pgfqpoint{1.651875in}{2.972151in}}%
\pgfpathcurveto{\pgfqpoint{1.660111in}{2.972151in}}{\pgfqpoint{1.668011in}{2.975423in}}{\pgfqpoint{1.673835in}{2.981247in}}%
\pgfpathcurveto{\pgfqpoint{1.679659in}{2.987071in}}{\pgfqpoint{1.682931in}{2.994971in}}{\pgfqpoint{1.682931in}{3.003207in}}%
\pgfpathcurveto{\pgfqpoint{1.682931in}{3.011444in}}{\pgfqpoint{1.679659in}{3.019344in}}{\pgfqpoint{1.673835in}{3.025168in}}%
\pgfpathcurveto{\pgfqpoint{1.668011in}{3.030992in}}{\pgfqpoint{1.660111in}{3.034264in}}{\pgfqpoint{1.651875in}{3.034264in}}%
\pgfpathcurveto{\pgfqpoint{1.643638in}{3.034264in}}{\pgfqpoint{1.635738in}{3.030992in}}{\pgfqpoint{1.629914in}{3.025168in}}%
\pgfpathcurveto{\pgfqpoint{1.624090in}{3.019344in}}{\pgfqpoint{1.620818in}{3.011444in}}{\pgfqpoint{1.620818in}{3.003207in}}%
\pgfpathcurveto{\pgfqpoint{1.620818in}{2.994971in}}{\pgfqpoint{1.624090in}{2.987071in}}{\pgfqpoint{1.629914in}{2.981247in}}%
\pgfpathcurveto{\pgfqpoint{1.635738in}{2.975423in}}{\pgfqpoint{1.643638in}{2.972151in}}{\pgfqpoint{1.651875in}{2.972151in}}%
\pgfpathclose%
\pgfusepath{stroke,fill}%
\end{pgfscope}%
\begin{pgfscope}%
\pgfpathrectangle{\pgfqpoint{0.100000in}{0.220728in}}{\pgfqpoint{3.696000in}{3.696000in}}%
\pgfusepath{clip}%
\pgfsetbuttcap%
\pgfsetroundjoin%
\definecolor{currentfill}{rgb}{0.121569,0.466667,0.705882}%
\pgfsetfillcolor{currentfill}%
\pgfsetfillopacity{0.347585}%
\pgfsetlinewidth{1.003750pt}%
\definecolor{currentstroke}{rgb}{0.121569,0.466667,0.705882}%
\pgfsetstrokecolor{currentstroke}%
\pgfsetstrokeopacity{0.347585}%
\pgfsetdash{}{0pt}%
\pgfpathmoveto{\pgfqpoint{1.645314in}{2.961294in}}%
\pgfpathcurveto{\pgfqpoint{1.653550in}{2.961294in}}{\pgfqpoint{1.661450in}{2.964566in}}{\pgfqpoint{1.667274in}{2.970390in}}%
\pgfpathcurveto{\pgfqpoint{1.673098in}{2.976214in}}{\pgfqpoint{1.676371in}{2.984114in}}{\pgfqpoint{1.676371in}{2.992350in}}%
\pgfpathcurveto{\pgfqpoint{1.676371in}{3.000586in}}{\pgfqpoint{1.673098in}{3.008486in}}{\pgfqpoint{1.667274in}{3.014310in}}%
\pgfpathcurveto{\pgfqpoint{1.661450in}{3.020134in}}{\pgfqpoint{1.653550in}{3.023407in}}{\pgfqpoint{1.645314in}{3.023407in}}%
\pgfpathcurveto{\pgfqpoint{1.637078in}{3.023407in}}{\pgfqpoint{1.629178in}{3.020134in}}{\pgfqpoint{1.623354in}{3.014310in}}%
\pgfpathcurveto{\pgfqpoint{1.617530in}{3.008486in}}{\pgfqpoint{1.614258in}{3.000586in}}{\pgfqpoint{1.614258in}{2.992350in}}%
\pgfpathcurveto{\pgfqpoint{1.614258in}{2.984114in}}{\pgfqpoint{1.617530in}{2.976214in}}{\pgfqpoint{1.623354in}{2.970390in}}%
\pgfpathcurveto{\pgfqpoint{1.629178in}{2.964566in}}{\pgfqpoint{1.637078in}{2.961294in}}{\pgfqpoint{1.645314in}{2.961294in}}%
\pgfpathclose%
\pgfusepath{stroke,fill}%
\end{pgfscope}%
\begin{pgfscope}%
\pgfpathrectangle{\pgfqpoint{0.100000in}{0.220728in}}{\pgfqpoint{3.696000in}{3.696000in}}%
\pgfusepath{clip}%
\pgfsetbuttcap%
\pgfsetroundjoin%
\definecolor{currentfill}{rgb}{0.121569,0.466667,0.705882}%
\pgfsetfillcolor{currentfill}%
\pgfsetfillopacity{0.349077}%
\pgfsetlinewidth{1.003750pt}%
\definecolor{currentstroke}{rgb}{0.121569,0.466667,0.705882}%
\pgfsetstrokecolor{currentstroke}%
\pgfsetstrokeopacity{0.349077}%
\pgfsetdash{}{0pt}%
\pgfpathmoveto{\pgfqpoint{1.644025in}{2.950570in}}%
\pgfpathcurveto{\pgfqpoint{1.652261in}{2.950570in}}{\pgfqpoint{1.660161in}{2.953843in}}{\pgfqpoint{1.665985in}{2.959666in}}%
\pgfpathcurveto{\pgfqpoint{1.671809in}{2.965490in}}{\pgfqpoint{1.675082in}{2.973390in}}{\pgfqpoint{1.675082in}{2.981627in}}%
\pgfpathcurveto{\pgfqpoint{1.675082in}{2.989863in}}{\pgfqpoint{1.671809in}{2.997763in}}{\pgfqpoint{1.665985in}{3.003587in}}%
\pgfpathcurveto{\pgfqpoint{1.660161in}{3.009411in}}{\pgfqpoint{1.652261in}{3.012683in}}{\pgfqpoint{1.644025in}{3.012683in}}%
\pgfpathcurveto{\pgfqpoint{1.635789in}{3.012683in}}{\pgfqpoint{1.627889in}{3.009411in}}{\pgfqpoint{1.622065in}{3.003587in}}%
\pgfpathcurveto{\pgfqpoint{1.616241in}{2.997763in}}{\pgfqpoint{1.612969in}{2.989863in}}{\pgfqpoint{1.612969in}{2.981627in}}%
\pgfpathcurveto{\pgfqpoint{1.612969in}{2.973390in}}{\pgfqpoint{1.616241in}{2.965490in}}{\pgfqpoint{1.622065in}{2.959666in}}%
\pgfpathcurveto{\pgfqpoint{1.627889in}{2.953843in}}{\pgfqpoint{1.635789in}{2.950570in}}{\pgfqpoint{1.644025in}{2.950570in}}%
\pgfpathclose%
\pgfusepath{stroke,fill}%
\end{pgfscope}%
\begin{pgfscope}%
\pgfpathrectangle{\pgfqpoint{0.100000in}{0.220728in}}{\pgfqpoint{3.696000in}{3.696000in}}%
\pgfusepath{clip}%
\pgfsetbuttcap%
\pgfsetroundjoin%
\definecolor{currentfill}{rgb}{0.121569,0.466667,0.705882}%
\pgfsetfillcolor{currentfill}%
\pgfsetfillopacity{0.349958}%
\pgfsetlinewidth{1.003750pt}%
\definecolor{currentstroke}{rgb}{0.121569,0.466667,0.705882}%
\pgfsetstrokecolor{currentstroke}%
\pgfsetstrokeopacity{0.349958}%
\pgfsetdash{}{0pt}%
\pgfpathmoveto{\pgfqpoint{1.638861in}{2.943882in}}%
\pgfpathcurveto{\pgfqpoint{1.647097in}{2.943882in}}{\pgfqpoint{1.654998in}{2.947154in}}{\pgfqpoint{1.660821in}{2.952978in}}%
\pgfpathcurveto{\pgfqpoint{1.666645in}{2.958802in}}{\pgfqpoint{1.669918in}{2.966702in}}{\pgfqpoint{1.669918in}{2.974938in}}%
\pgfpathcurveto{\pgfqpoint{1.669918in}{2.983175in}}{\pgfqpoint{1.666645in}{2.991075in}}{\pgfqpoint{1.660821in}{2.996898in}}%
\pgfpathcurveto{\pgfqpoint{1.654998in}{3.002722in}}{\pgfqpoint{1.647097in}{3.005995in}}{\pgfqpoint{1.638861in}{3.005995in}}%
\pgfpathcurveto{\pgfqpoint{1.630625in}{3.005995in}}{\pgfqpoint{1.622725in}{3.002722in}}{\pgfqpoint{1.616901in}{2.996898in}}%
\pgfpathcurveto{\pgfqpoint{1.611077in}{2.991075in}}{\pgfqpoint{1.607805in}{2.983175in}}{\pgfqpoint{1.607805in}{2.974938in}}%
\pgfpathcurveto{\pgfqpoint{1.607805in}{2.966702in}}{\pgfqpoint{1.611077in}{2.958802in}}{\pgfqpoint{1.616901in}{2.952978in}}%
\pgfpathcurveto{\pgfqpoint{1.622725in}{2.947154in}}{\pgfqpoint{1.630625in}{2.943882in}}{\pgfqpoint{1.638861in}{2.943882in}}%
\pgfpathclose%
\pgfusepath{stroke,fill}%
\end{pgfscope}%
\begin{pgfscope}%
\pgfpathrectangle{\pgfqpoint{0.100000in}{0.220728in}}{\pgfqpoint{3.696000in}{3.696000in}}%
\pgfusepath{clip}%
\pgfsetbuttcap%
\pgfsetroundjoin%
\definecolor{currentfill}{rgb}{0.121569,0.466667,0.705882}%
\pgfsetfillcolor{currentfill}%
\pgfsetfillopacity{0.349992}%
\pgfsetlinewidth{1.003750pt}%
\definecolor{currentstroke}{rgb}{0.121569,0.466667,0.705882}%
\pgfsetstrokecolor{currentstroke}%
\pgfsetstrokeopacity{0.349992}%
\pgfsetdash{}{0pt}%
\pgfpathmoveto{\pgfqpoint{1.975723in}{3.245126in}}%
\pgfpathcurveto{\pgfqpoint{1.983960in}{3.245126in}}{\pgfqpoint{1.991860in}{3.248398in}}{\pgfqpoint{1.997684in}{3.254222in}}%
\pgfpathcurveto{\pgfqpoint{2.003507in}{3.260046in}}{\pgfqpoint{2.006780in}{3.267946in}}{\pgfqpoint{2.006780in}{3.276182in}}%
\pgfpathcurveto{\pgfqpoint{2.006780in}{3.284418in}}{\pgfqpoint{2.003507in}{3.292319in}}{\pgfqpoint{1.997684in}{3.298142in}}%
\pgfpathcurveto{\pgfqpoint{1.991860in}{3.303966in}}{\pgfqpoint{1.983960in}{3.307239in}}{\pgfqpoint{1.975723in}{3.307239in}}%
\pgfpathcurveto{\pgfqpoint{1.967487in}{3.307239in}}{\pgfqpoint{1.959587in}{3.303966in}}{\pgfqpoint{1.953763in}{3.298142in}}%
\pgfpathcurveto{\pgfqpoint{1.947939in}{3.292319in}}{\pgfqpoint{1.944667in}{3.284418in}}{\pgfqpoint{1.944667in}{3.276182in}}%
\pgfpathcurveto{\pgfqpoint{1.944667in}{3.267946in}}{\pgfqpoint{1.947939in}{3.260046in}}{\pgfqpoint{1.953763in}{3.254222in}}%
\pgfpathcurveto{\pgfqpoint{1.959587in}{3.248398in}}{\pgfqpoint{1.967487in}{3.245126in}}{\pgfqpoint{1.975723in}{3.245126in}}%
\pgfpathclose%
\pgfusepath{stroke,fill}%
\end{pgfscope}%
\begin{pgfscope}%
\pgfpathrectangle{\pgfqpoint{0.100000in}{0.220728in}}{\pgfqpoint{3.696000in}{3.696000in}}%
\pgfusepath{clip}%
\pgfsetbuttcap%
\pgfsetroundjoin%
\definecolor{currentfill}{rgb}{0.121569,0.466667,0.705882}%
\pgfsetfillcolor{currentfill}%
\pgfsetfillopacity{0.350775}%
\pgfsetlinewidth{1.003750pt}%
\definecolor{currentstroke}{rgb}{0.121569,0.466667,0.705882}%
\pgfsetstrokecolor{currentstroke}%
\pgfsetstrokeopacity{0.350775}%
\pgfsetdash{}{0pt}%
\pgfpathmoveto{\pgfqpoint{1.637940in}{2.939536in}}%
\pgfpathcurveto{\pgfqpoint{1.646176in}{2.939536in}}{\pgfqpoint{1.654076in}{2.942808in}}{\pgfqpoint{1.659900in}{2.948632in}}%
\pgfpathcurveto{\pgfqpoint{1.665724in}{2.954456in}}{\pgfqpoint{1.668996in}{2.962356in}}{\pgfqpoint{1.668996in}{2.970592in}}%
\pgfpathcurveto{\pgfqpoint{1.668996in}{2.978829in}}{\pgfqpoint{1.665724in}{2.986729in}}{\pgfqpoint{1.659900in}{2.992553in}}%
\pgfpathcurveto{\pgfqpoint{1.654076in}{2.998377in}}{\pgfqpoint{1.646176in}{3.001649in}}{\pgfqpoint{1.637940in}{3.001649in}}%
\pgfpathcurveto{\pgfqpoint{1.629704in}{3.001649in}}{\pgfqpoint{1.621804in}{2.998377in}}{\pgfqpoint{1.615980in}{2.992553in}}%
\pgfpathcurveto{\pgfqpoint{1.610156in}{2.986729in}}{\pgfqpoint{1.606883in}{2.978829in}}{\pgfqpoint{1.606883in}{2.970592in}}%
\pgfpathcurveto{\pgfqpoint{1.606883in}{2.962356in}}{\pgfqpoint{1.610156in}{2.954456in}}{\pgfqpoint{1.615980in}{2.948632in}}%
\pgfpathcurveto{\pgfqpoint{1.621804in}{2.942808in}}{\pgfqpoint{1.629704in}{2.939536in}}{\pgfqpoint{1.637940in}{2.939536in}}%
\pgfpathclose%
\pgfusepath{stroke,fill}%
\end{pgfscope}%
\begin{pgfscope}%
\pgfpathrectangle{\pgfqpoint{0.100000in}{0.220728in}}{\pgfqpoint{3.696000in}{3.696000in}}%
\pgfusepath{clip}%
\pgfsetbuttcap%
\pgfsetroundjoin%
\definecolor{currentfill}{rgb}{0.121569,0.466667,0.705882}%
\pgfsetfillcolor{currentfill}%
\pgfsetfillopacity{0.352018}%
\pgfsetlinewidth{1.003750pt}%
\definecolor{currentstroke}{rgb}{0.121569,0.466667,0.705882}%
\pgfsetstrokecolor{currentstroke}%
\pgfsetstrokeopacity{0.352018}%
\pgfsetdash{}{0pt}%
\pgfpathmoveto{\pgfqpoint{1.633573in}{2.932862in}}%
\pgfpathcurveto{\pgfqpoint{1.641809in}{2.932862in}}{\pgfqpoint{1.649709in}{2.936134in}}{\pgfqpoint{1.655533in}{2.941958in}}%
\pgfpathcurveto{\pgfqpoint{1.661357in}{2.947782in}}{\pgfqpoint{1.664629in}{2.955682in}}{\pgfqpoint{1.664629in}{2.963918in}}%
\pgfpathcurveto{\pgfqpoint{1.664629in}{2.972155in}}{\pgfqpoint{1.661357in}{2.980055in}}{\pgfqpoint{1.655533in}{2.985879in}}%
\pgfpathcurveto{\pgfqpoint{1.649709in}{2.991703in}}{\pgfqpoint{1.641809in}{2.994975in}}{\pgfqpoint{1.633573in}{2.994975in}}%
\pgfpathcurveto{\pgfqpoint{1.625337in}{2.994975in}}{\pgfqpoint{1.617437in}{2.991703in}}{\pgfqpoint{1.611613in}{2.985879in}}%
\pgfpathcurveto{\pgfqpoint{1.605789in}{2.980055in}}{\pgfqpoint{1.602516in}{2.972155in}}{\pgfqpoint{1.602516in}{2.963918in}}%
\pgfpathcurveto{\pgfqpoint{1.602516in}{2.955682in}}{\pgfqpoint{1.605789in}{2.947782in}}{\pgfqpoint{1.611613in}{2.941958in}}%
\pgfpathcurveto{\pgfqpoint{1.617437in}{2.936134in}}{\pgfqpoint{1.625337in}{2.932862in}}{\pgfqpoint{1.633573in}{2.932862in}}%
\pgfpathclose%
\pgfusepath{stroke,fill}%
\end{pgfscope}%
\begin{pgfscope}%
\pgfpathrectangle{\pgfqpoint{0.100000in}{0.220728in}}{\pgfqpoint{3.696000in}{3.696000in}}%
\pgfusepath{clip}%
\pgfsetbuttcap%
\pgfsetroundjoin%
\definecolor{currentfill}{rgb}{0.121569,0.466667,0.705882}%
\pgfsetfillcolor{currentfill}%
\pgfsetfillopacity{0.352025}%
\pgfsetlinewidth{1.003750pt}%
\definecolor{currentstroke}{rgb}{0.121569,0.466667,0.705882}%
\pgfsetstrokecolor{currentstroke}%
\pgfsetstrokeopacity{0.352025}%
\pgfsetdash{}{0pt}%
\pgfpathmoveto{\pgfqpoint{1.996067in}{3.240046in}}%
\pgfpathcurveto{\pgfqpoint{2.004303in}{3.240046in}}{\pgfqpoint{2.012203in}{3.243318in}}{\pgfqpoint{2.018027in}{3.249142in}}%
\pgfpathcurveto{\pgfqpoint{2.023851in}{3.254966in}}{\pgfqpoint{2.027123in}{3.262866in}}{\pgfqpoint{2.027123in}{3.271102in}}%
\pgfpathcurveto{\pgfqpoint{2.027123in}{3.279339in}}{\pgfqpoint{2.023851in}{3.287239in}}{\pgfqpoint{2.018027in}{3.293063in}}%
\pgfpathcurveto{\pgfqpoint{2.012203in}{3.298887in}}{\pgfqpoint{2.004303in}{3.302159in}}{\pgfqpoint{1.996067in}{3.302159in}}%
\pgfpathcurveto{\pgfqpoint{1.987830in}{3.302159in}}{\pgfqpoint{1.979930in}{3.298887in}}{\pgfqpoint{1.974106in}{3.293063in}}%
\pgfpathcurveto{\pgfqpoint{1.968283in}{3.287239in}}{\pgfqpoint{1.965010in}{3.279339in}}{\pgfqpoint{1.965010in}{3.271102in}}%
\pgfpathcurveto{\pgfqpoint{1.965010in}{3.262866in}}{\pgfqpoint{1.968283in}{3.254966in}}{\pgfqpoint{1.974106in}{3.249142in}}%
\pgfpathcurveto{\pgfqpoint{1.979930in}{3.243318in}}{\pgfqpoint{1.987830in}{3.240046in}}{\pgfqpoint{1.996067in}{3.240046in}}%
\pgfpathclose%
\pgfusepath{stroke,fill}%
\end{pgfscope}%
\begin{pgfscope}%
\pgfpathrectangle{\pgfqpoint{0.100000in}{0.220728in}}{\pgfqpoint{3.696000in}{3.696000in}}%
\pgfusepath{clip}%
\pgfsetbuttcap%
\pgfsetroundjoin%
\definecolor{currentfill}{rgb}{0.121569,0.466667,0.705882}%
\pgfsetfillcolor{currentfill}%
\pgfsetfillopacity{0.354598}%
\pgfsetlinewidth{1.003750pt}%
\definecolor{currentstroke}{rgb}{0.121569,0.466667,0.705882}%
\pgfsetstrokecolor{currentstroke}%
\pgfsetstrokeopacity{0.354598}%
\pgfsetdash{}{0pt}%
\pgfpathmoveto{\pgfqpoint{1.627944in}{2.919443in}}%
\pgfpathcurveto{\pgfqpoint{1.636180in}{2.919443in}}{\pgfqpoint{1.644080in}{2.922716in}}{\pgfqpoint{1.649904in}{2.928540in}}%
\pgfpathcurveto{\pgfqpoint{1.655728in}{2.934364in}}{\pgfqpoint{1.659000in}{2.942264in}}{\pgfqpoint{1.659000in}{2.950500in}}%
\pgfpathcurveto{\pgfqpoint{1.659000in}{2.958736in}}{\pgfqpoint{1.655728in}{2.966636in}}{\pgfqpoint{1.649904in}{2.972460in}}%
\pgfpathcurveto{\pgfqpoint{1.644080in}{2.978284in}}{\pgfqpoint{1.636180in}{2.981556in}}{\pgfqpoint{1.627944in}{2.981556in}}%
\pgfpathcurveto{\pgfqpoint{1.619707in}{2.981556in}}{\pgfqpoint{1.611807in}{2.978284in}}{\pgfqpoint{1.605984in}{2.972460in}}%
\pgfpathcurveto{\pgfqpoint{1.600160in}{2.966636in}}{\pgfqpoint{1.596887in}{2.958736in}}{\pgfqpoint{1.596887in}{2.950500in}}%
\pgfpathcurveto{\pgfqpoint{1.596887in}{2.942264in}}{\pgfqpoint{1.600160in}{2.934364in}}{\pgfqpoint{1.605984in}{2.928540in}}%
\pgfpathcurveto{\pgfqpoint{1.611807in}{2.922716in}}{\pgfqpoint{1.619707in}{2.919443in}}{\pgfqpoint{1.627944in}{2.919443in}}%
\pgfpathclose%
\pgfusepath{stroke,fill}%
\end{pgfscope}%
\begin{pgfscope}%
\pgfpathrectangle{\pgfqpoint{0.100000in}{0.220728in}}{\pgfqpoint{3.696000in}{3.696000in}}%
\pgfusepath{clip}%
\pgfsetbuttcap%
\pgfsetroundjoin%
\definecolor{currentfill}{rgb}{0.121569,0.466667,0.705882}%
\pgfsetfillcolor{currentfill}%
\pgfsetfillopacity{0.354869}%
\pgfsetlinewidth{1.003750pt}%
\definecolor{currentstroke}{rgb}{0.121569,0.466667,0.705882}%
\pgfsetstrokecolor{currentstroke}%
\pgfsetstrokeopacity{0.354869}%
\pgfsetdash{}{0pt}%
\pgfpathmoveto{\pgfqpoint{2.005360in}{3.237935in}}%
\pgfpathcurveto{\pgfqpoint{2.013596in}{3.237935in}}{\pgfqpoint{2.021496in}{3.241207in}}{\pgfqpoint{2.027320in}{3.247031in}}%
\pgfpathcurveto{\pgfqpoint{2.033144in}{3.252855in}}{\pgfqpoint{2.036417in}{3.260755in}}{\pgfqpoint{2.036417in}{3.268992in}}%
\pgfpathcurveto{\pgfqpoint{2.036417in}{3.277228in}}{\pgfqpoint{2.033144in}{3.285128in}}{\pgfqpoint{2.027320in}{3.290952in}}%
\pgfpathcurveto{\pgfqpoint{2.021496in}{3.296776in}}{\pgfqpoint{2.013596in}{3.300048in}}{\pgfqpoint{2.005360in}{3.300048in}}%
\pgfpathcurveto{\pgfqpoint{1.997124in}{3.300048in}}{\pgfqpoint{1.989224in}{3.296776in}}{\pgfqpoint{1.983400in}{3.290952in}}%
\pgfpathcurveto{\pgfqpoint{1.977576in}{3.285128in}}{\pgfqpoint{1.974304in}{3.277228in}}{\pgfqpoint{1.974304in}{3.268992in}}%
\pgfpathcurveto{\pgfqpoint{1.974304in}{3.260755in}}{\pgfqpoint{1.977576in}{3.252855in}}{\pgfqpoint{1.983400in}{3.247031in}}%
\pgfpathcurveto{\pgfqpoint{1.989224in}{3.241207in}}{\pgfqpoint{1.997124in}{3.237935in}}{\pgfqpoint{2.005360in}{3.237935in}}%
\pgfpathclose%
\pgfusepath{stroke,fill}%
\end{pgfscope}%
\begin{pgfscope}%
\pgfpathrectangle{\pgfqpoint{0.100000in}{0.220728in}}{\pgfqpoint{3.696000in}{3.696000in}}%
\pgfusepath{clip}%
\pgfsetbuttcap%
\pgfsetroundjoin%
\definecolor{currentfill}{rgb}{0.121569,0.466667,0.705882}%
\pgfsetfillcolor{currentfill}%
\pgfsetfillopacity{0.357555}%
\pgfsetlinewidth{1.003750pt}%
\definecolor{currentstroke}{rgb}{0.121569,0.466667,0.705882}%
\pgfsetstrokecolor{currentstroke}%
\pgfsetstrokeopacity{0.357555}%
\pgfsetdash{}{0pt}%
\pgfpathmoveto{\pgfqpoint{2.016257in}{3.236382in}}%
\pgfpathcurveto{\pgfqpoint{2.024493in}{3.236382in}}{\pgfqpoint{2.032393in}{3.239654in}}{\pgfqpoint{2.038217in}{3.245478in}}%
\pgfpathcurveto{\pgfqpoint{2.044041in}{3.251302in}}{\pgfqpoint{2.047313in}{3.259202in}}{\pgfqpoint{2.047313in}{3.267438in}}%
\pgfpathcurveto{\pgfqpoint{2.047313in}{3.275674in}}{\pgfqpoint{2.044041in}{3.283575in}}{\pgfqpoint{2.038217in}{3.289398in}}%
\pgfpathcurveto{\pgfqpoint{2.032393in}{3.295222in}}{\pgfqpoint{2.024493in}{3.298495in}}{\pgfqpoint{2.016257in}{3.298495in}}%
\pgfpathcurveto{\pgfqpoint{2.008020in}{3.298495in}}{\pgfqpoint{2.000120in}{3.295222in}}{\pgfqpoint{1.994296in}{3.289398in}}%
\pgfpathcurveto{\pgfqpoint{1.988472in}{3.283575in}}{\pgfqpoint{1.985200in}{3.275674in}}{\pgfqpoint{1.985200in}{3.267438in}}%
\pgfpathcurveto{\pgfqpoint{1.985200in}{3.259202in}}{\pgfqpoint{1.988472in}{3.251302in}}{\pgfqpoint{1.994296in}{3.245478in}}%
\pgfpathcurveto{\pgfqpoint{2.000120in}{3.239654in}}{\pgfqpoint{2.008020in}{3.236382in}}{\pgfqpoint{2.016257in}{3.236382in}}%
\pgfpathclose%
\pgfusepath{stroke,fill}%
\end{pgfscope}%
\begin{pgfscope}%
\pgfpathrectangle{\pgfqpoint{0.100000in}{0.220728in}}{\pgfqpoint{3.696000in}{3.696000in}}%
\pgfusepath{clip}%
\pgfsetbuttcap%
\pgfsetroundjoin%
\definecolor{currentfill}{rgb}{0.121569,0.466667,0.705882}%
\pgfsetfillcolor{currentfill}%
\pgfsetfillopacity{0.359233}%
\pgfsetlinewidth{1.003750pt}%
\definecolor{currentstroke}{rgb}{0.121569,0.466667,0.705882}%
\pgfsetstrokecolor{currentstroke}%
\pgfsetstrokeopacity{0.359233}%
\pgfsetdash{}{0pt}%
\pgfpathmoveto{\pgfqpoint{2.022034in}{3.235726in}}%
\pgfpathcurveto{\pgfqpoint{2.030270in}{3.235726in}}{\pgfqpoint{2.038171in}{3.238998in}}{\pgfqpoint{2.043994in}{3.244822in}}%
\pgfpathcurveto{\pgfqpoint{2.049818in}{3.250646in}}{\pgfqpoint{2.053091in}{3.258546in}}{\pgfqpoint{2.053091in}{3.266782in}}%
\pgfpathcurveto{\pgfqpoint{2.053091in}{3.275019in}}{\pgfqpoint{2.049818in}{3.282919in}}{\pgfqpoint{2.043994in}{3.288743in}}%
\pgfpathcurveto{\pgfqpoint{2.038171in}{3.294567in}}{\pgfqpoint{2.030270in}{3.297839in}}{\pgfqpoint{2.022034in}{3.297839in}}%
\pgfpathcurveto{\pgfqpoint{2.013798in}{3.297839in}}{\pgfqpoint{2.005898in}{3.294567in}}{\pgfqpoint{2.000074in}{3.288743in}}%
\pgfpathcurveto{\pgfqpoint{1.994250in}{3.282919in}}{\pgfqpoint{1.990978in}{3.275019in}}{\pgfqpoint{1.990978in}{3.266782in}}%
\pgfpathcurveto{\pgfqpoint{1.990978in}{3.258546in}}{\pgfqpoint{1.994250in}{3.250646in}}{\pgfqpoint{2.000074in}{3.244822in}}%
\pgfpathcurveto{\pgfqpoint{2.005898in}{3.238998in}}{\pgfqpoint{2.013798in}{3.235726in}}{\pgfqpoint{2.022034in}{3.235726in}}%
\pgfpathclose%
\pgfusepath{stroke,fill}%
\end{pgfscope}%
\begin{pgfscope}%
\pgfpathrectangle{\pgfqpoint{0.100000in}{0.220728in}}{\pgfqpoint{3.696000in}{3.696000in}}%
\pgfusepath{clip}%
\pgfsetbuttcap%
\pgfsetroundjoin%
\definecolor{currentfill}{rgb}{0.121569,0.466667,0.705882}%
\pgfsetfillcolor{currentfill}%
\pgfsetfillopacity{0.359281}%
\pgfsetlinewidth{1.003750pt}%
\definecolor{currentstroke}{rgb}{0.121569,0.466667,0.705882}%
\pgfsetstrokecolor{currentstroke}%
\pgfsetstrokeopacity{0.359281}%
\pgfsetdash{}{0pt}%
\pgfpathmoveto{\pgfqpoint{1.618357in}{2.894448in}}%
\pgfpathcurveto{\pgfqpoint{1.626593in}{2.894448in}}{\pgfqpoint{1.634493in}{2.897720in}}{\pgfqpoint{1.640317in}{2.903544in}}%
\pgfpathcurveto{\pgfqpoint{1.646141in}{2.909368in}}{\pgfqpoint{1.649414in}{2.917268in}}{\pgfqpoint{1.649414in}{2.925504in}}%
\pgfpathcurveto{\pgfqpoint{1.649414in}{2.933740in}}{\pgfqpoint{1.646141in}{2.941640in}}{\pgfqpoint{1.640317in}{2.947464in}}%
\pgfpathcurveto{\pgfqpoint{1.634493in}{2.953288in}}{\pgfqpoint{1.626593in}{2.956561in}}{\pgfqpoint{1.618357in}{2.956561in}}%
\pgfpathcurveto{\pgfqpoint{1.610121in}{2.956561in}}{\pgfqpoint{1.602221in}{2.953288in}}{\pgfqpoint{1.596397in}{2.947464in}}%
\pgfpathcurveto{\pgfqpoint{1.590573in}{2.941640in}}{\pgfqpoint{1.587301in}{2.933740in}}{\pgfqpoint{1.587301in}{2.925504in}}%
\pgfpathcurveto{\pgfqpoint{1.587301in}{2.917268in}}{\pgfqpoint{1.590573in}{2.909368in}}{\pgfqpoint{1.596397in}{2.903544in}}%
\pgfpathcurveto{\pgfqpoint{1.602221in}{2.897720in}}{\pgfqpoint{1.610121in}{2.894448in}}{\pgfqpoint{1.618357in}{2.894448in}}%
\pgfpathclose%
\pgfusepath{stroke,fill}%
\end{pgfscope}%
\begin{pgfscope}%
\pgfpathrectangle{\pgfqpoint{0.100000in}{0.220728in}}{\pgfqpoint{3.696000in}{3.696000in}}%
\pgfusepath{clip}%
\pgfsetbuttcap%
\pgfsetroundjoin%
\definecolor{currentfill}{rgb}{0.121569,0.466667,0.705882}%
\pgfsetfillcolor{currentfill}%
\pgfsetfillopacity{0.361961}%
\pgfsetlinewidth{1.003750pt}%
\definecolor{currentstroke}{rgb}{0.121569,0.466667,0.705882}%
\pgfsetstrokecolor{currentstroke}%
\pgfsetstrokeopacity{0.361961}%
\pgfsetdash{}{0pt}%
\pgfpathmoveto{\pgfqpoint{2.029510in}{3.234128in}}%
\pgfpathcurveto{\pgfqpoint{2.037746in}{3.234128in}}{\pgfqpoint{2.045646in}{3.237400in}}{\pgfqpoint{2.051470in}{3.243224in}}%
\pgfpathcurveto{\pgfqpoint{2.057294in}{3.249048in}}{\pgfqpoint{2.060566in}{3.256948in}}{\pgfqpoint{2.060566in}{3.265184in}}%
\pgfpathcurveto{\pgfqpoint{2.060566in}{3.273421in}}{\pgfqpoint{2.057294in}{3.281321in}}{\pgfqpoint{2.051470in}{3.287145in}}%
\pgfpathcurveto{\pgfqpoint{2.045646in}{3.292969in}}{\pgfqpoint{2.037746in}{3.296241in}}{\pgfqpoint{2.029510in}{3.296241in}}%
\pgfpathcurveto{\pgfqpoint{2.021274in}{3.296241in}}{\pgfqpoint{2.013373in}{3.292969in}}{\pgfqpoint{2.007550in}{3.287145in}}%
\pgfpathcurveto{\pgfqpoint{2.001726in}{3.281321in}}{\pgfqpoint{1.998453in}{3.273421in}}{\pgfqpoint{1.998453in}{3.265184in}}%
\pgfpathcurveto{\pgfqpoint{1.998453in}{3.256948in}}{\pgfqpoint{2.001726in}{3.249048in}}{\pgfqpoint{2.007550in}{3.243224in}}%
\pgfpathcurveto{\pgfqpoint{2.013373in}{3.237400in}}{\pgfqpoint{2.021274in}{3.234128in}}{\pgfqpoint{2.029510in}{3.234128in}}%
\pgfpathclose%
\pgfusepath{stroke,fill}%
\end{pgfscope}%
\begin{pgfscope}%
\pgfpathrectangle{\pgfqpoint{0.100000in}{0.220728in}}{\pgfqpoint{3.696000in}{3.696000in}}%
\pgfusepath{clip}%
\pgfsetbuttcap%
\pgfsetroundjoin%
\definecolor{currentfill}{rgb}{0.121569,0.466667,0.705882}%
\pgfsetfillcolor{currentfill}%
\pgfsetfillopacity{0.362990}%
\pgfsetlinewidth{1.003750pt}%
\definecolor{currentstroke}{rgb}{0.121569,0.466667,0.705882}%
\pgfsetstrokecolor{currentstroke}%
\pgfsetstrokeopacity{0.362990}%
\pgfsetdash{}{0pt}%
\pgfpathmoveto{\pgfqpoint{1.604059in}{2.871895in}}%
\pgfpathcurveto{\pgfqpoint{1.612295in}{2.871895in}}{\pgfqpoint{1.620195in}{2.875167in}}{\pgfqpoint{1.626019in}{2.880991in}}%
\pgfpathcurveto{\pgfqpoint{1.631843in}{2.886815in}}{\pgfqpoint{1.635115in}{2.894715in}}{\pgfqpoint{1.635115in}{2.902952in}}%
\pgfpathcurveto{\pgfqpoint{1.635115in}{2.911188in}}{\pgfqpoint{1.631843in}{2.919088in}}{\pgfqpoint{1.626019in}{2.924912in}}%
\pgfpathcurveto{\pgfqpoint{1.620195in}{2.930736in}}{\pgfqpoint{1.612295in}{2.934008in}}{\pgfqpoint{1.604059in}{2.934008in}}%
\pgfpathcurveto{\pgfqpoint{1.595822in}{2.934008in}}{\pgfqpoint{1.587922in}{2.930736in}}{\pgfqpoint{1.582098in}{2.924912in}}%
\pgfpathcurveto{\pgfqpoint{1.576275in}{2.919088in}}{\pgfqpoint{1.573002in}{2.911188in}}{\pgfqpoint{1.573002in}{2.902952in}}%
\pgfpathcurveto{\pgfqpoint{1.573002in}{2.894715in}}{\pgfqpoint{1.576275in}{2.886815in}}{\pgfqpoint{1.582098in}{2.880991in}}%
\pgfpathcurveto{\pgfqpoint{1.587922in}{2.875167in}}{\pgfqpoint{1.595822in}{2.871895in}}{\pgfqpoint{1.604059in}{2.871895in}}%
\pgfpathclose%
\pgfusepath{stroke,fill}%
\end{pgfscope}%
\begin{pgfscope}%
\pgfpathrectangle{\pgfqpoint{0.100000in}{0.220728in}}{\pgfqpoint{3.696000in}{3.696000in}}%
\pgfusepath{clip}%
\pgfsetbuttcap%
\pgfsetroundjoin%
\definecolor{currentfill}{rgb}{0.121569,0.466667,0.705882}%
\pgfsetfillcolor{currentfill}%
\pgfsetfillopacity{0.363685}%
\pgfsetlinewidth{1.003750pt}%
\definecolor{currentstroke}{rgb}{0.121569,0.466667,0.705882}%
\pgfsetstrokecolor{currentstroke}%
\pgfsetstrokeopacity{0.363685}%
\pgfsetdash{}{0pt}%
\pgfpathmoveto{\pgfqpoint{2.039783in}{3.231508in}}%
\pgfpathcurveto{\pgfqpoint{2.048019in}{3.231508in}}{\pgfqpoint{2.055919in}{3.234781in}}{\pgfqpoint{2.061743in}{3.240605in}}%
\pgfpathcurveto{\pgfqpoint{2.067567in}{3.246429in}}{\pgfqpoint{2.070839in}{3.254329in}}{\pgfqpoint{2.070839in}{3.262565in}}%
\pgfpathcurveto{\pgfqpoint{2.070839in}{3.270801in}}{\pgfqpoint{2.067567in}{3.278701in}}{\pgfqpoint{2.061743in}{3.284525in}}%
\pgfpathcurveto{\pgfqpoint{2.055919in}{3.290349in}}{\pgfqpoint{2.048019in}{3.293621in}}{\pgfqpoint{2.039783in}{3.293621in}}%
\pgfpathcurveto{\pgfqpoint{2.031546in}{3.293621in}}{\pgfqpoint{2.023646in}{3.290349in}}{\pgfqpoint{2.017822in}{3.284525in}}%
\pgfpathcurveto{\pgfqpoint{2.011998in}{3.278701in}}{\pgfqpoint{2.008726in}{3.270801in}}{\pgfqpoint{2.008726in}{3.262565in}}%
\pgfpathcurveto{\pgfqpoint{2.008726in}{3.254329in}}{\pgfqpoint{2.011998in}{3.246429in}}{\pgfqpoint{2.017822in}{3.240605in}}%
\pgfpathcurveto{\pgfqpoint{2.023646in}{3.234781in}}{\pgfqpoint{2.031546in}{3.231508in}}{\pgfqpoint{2.039783in}{3.231508in}}%
\pgfpathclose%
\pgfusepath{stroke,fill}%
\end{pgfscope}%
\begin{pgfscope}%
\pgfpathrectangle{\pgfqpoint{0.100000in}{0.220728in}}{\pgfqpoint{3.696000in}{3.696000in}}%
\pgfusepath{clip}%
\pgfsetbuttcap%
\pgfsetroundjoin%
\definecolor{currentfill}{rgb}{0.121569,0.466667,0.705882}%
\pgfsetfillcolor{currentfill}%
\pgfsetfillopacity{0.366241}%
\pgfsetlinewidth{1.003750pt}%
\definecolor{currentstroke}{rgb}{0.121569,0.466667,0.705882}%
\pgfsetstrokecolor{currentstroke}%
\pgfsetstrokeopacity{0.366241}%
\pgfsetdash{}{0pt}%
\pgfpathmoveto{\pgfqpoint{2.051497in}{3.229094in}}%
\pgfpathcurveto{\pgfqpoint{2.059733in}{3.229094in}}{\pgfqpoint{2.067633in}{3.232366in}}{\pgfqpoint{2.073457in}{3.238190in}}%
\pgfpathcurveto{\pgfqpoint{2.079281in}{3.244014in}}{\pgfqpoint{2.082553in}{3.251914in}}{\pgfqpoint{2.082553in}{3.260150in}}%
\pgfpathcurveto{\pgfqpoint{2.082553in}{3.268386in}}{\pgfqpoint{2.079281in}{3.276286in}}{\pgfqpoint{2.073457in}{3.282110in}}%
\pgfpathcurveto{\pgfqpoint{2.067633in}{3.287934in}}{\pgfqpoint{2.059733in}{3.291206in}}{\pgfqpoint{2.051497in}{3.291206in}}%
\pgfpathcurveto{\pgfqpoint{2.043261in}{3.291206in}}{\pgfqpoint{2.035361in}{3.287934in}}{\pgfqpoint{2.029537in}{3.282110in}}%
\pgfpathcurveto{\pgfqpoint{2.023713in}{3.276286in}}{\pgfqpoint{2.020440in}{3.268386in}}{\pgfqpoint{2.020440in}{3.260150in}}%
\pgfpathcurveto{\pgfqpoint{2.020440in}{3.251914in}}{\pgfqpoint{2.023713in}{3.244014in}}{\pgfqpoint{2.029537in}{3.238190in}}%
\pgfpathcurveto{\pgfqpoint{2.035361in}{3.232366in}}{\pgfqpoint{2.043261in}{3.229094in}}{\pgfqpoint{2.051497in}{3.229094in}}%
\pgfpathclose%
\pgfusepath{stroke,fill}%
\end{pgfscope}%
\begin{pgfscope}%
\pgfpathrectangle{\pgfqpoint{0.100000in}{0.220728in}}{\pgfqpoint{3.696000in}{3.696000in}}%
\pgfusepath{clip}%
\pgfsetbuttcap%
\pgfsetroundjoin%
\definecolor{currentfill}{rgb}{0.121569,0.466667,0.705882}%
\pgfsetfillcolor{currentfill}%
\pgfsetfillopacity{0.367385}%
\pgfsetlinewidth{1.003750pt}%
\definecolor{currentstroke}{rgb}{0.121569,0.466667,0.705882}%
\pgfsetstrokecolor{currentstroke}%
\pgfsetstrokeopacity{0.367385}%
\pgfsetdash{}{0pt}%
\pgfpathmoveto{\pgfqpoint{1.598985in}{2.847392in}}%
\pgfpathcurveto{\pgfqpoint{1.607222in}{2.847392in}}{\pgfqpoint{1.615122in}{2.850665in}}{\pgfqpoint{1.620946in}{2.856489in}}%
\pgfpathcurveto{\pgfqpoint{1.626770in}{2.862313in}}{\pgfqpoint{1.630042in}{2.870213in}}{\pgfqpoint{1.630042in}{2.878449in}}%
\pgfpathcurveto{\pgfqpoint{1.630042in}{2.886685in}}{\pgfqpoint{1.626770in}{2.894585in}}{\pgfqpoint{1.620946in}{2.900409in}}%
\pgfpathcurveto{\pgfqpoint{1.615122in}{2.906233in}}{\pgfqpoint{1.607222in}{2.909505in}}{\pgfqpoint{1.598985in}{2.909505in}}%
\pgfpathcurveto{\pgfqpoint{1.590749in}{2.909505in}}{\pgfqpoint{1.582849in}{2.906233in}}{\pgfqpoint{1.577025in}{2.900409in}}%
\pgfpathcurveto{\pgfqpoint{1.571201in}{2.894585in}}{\pgfqpoint{1.567929in}{2.886685in}}{\pgfqpoint{1.567929in}{2.878449in}}%
\pgfpathcurveto{\pgfqpoint{1.567929in}{2.870213in}}{\pgfqpoint{1.571201in}{2.862313in}}{\pgfqpoint{1.577025in}{2.856489in}}%
\pgfpathcurveto{\pgfqpoint{1.582849in}{2.850665in}}{\pgfqpoint{1.590749in}{2.847392in}}{\pgfqpoint{1.598985in}{2.847392in}}%
\pgfpathclose%
\pgfusepath{stroke,fill}%
\end{pgfscope}%
\begin{pgfscope}%
\pgfpathrectangle{\pgfqpoint{0.100000in}{0.220728in}}{\pgfqpoint{3.696000in}{3.696000in}}%
\pgfusepath{clip}%
\pgfsetbuttcap%
\pgfsetroundjoin%
\definecolor{currentfill}{rgb}{0.121569,0.466667,0.705882}%
\pgfsetfillcolor{currentfill}%
\pgfsetfillopacity{0.370152}%
\pgfsetlinewidth{1.003750pt}%
\definecolor{currentstroke}{rgb}{0.121569,0.466667,0.705882}%
\pgfsetstrokecolor{currentstroke}%
\pgfsetstrokeopacity{0.370152}%
\pgfsetdash{}{0pt}%
\pgfpathmoveto{\pgfqpoint{1.584961in}{2.827644in}}%
\pgfpathcurveto{\pgfqpoint{1.593197in}{2.827644in}}{\pgfqpoint{1.601097in}{2.830916in}}{\pgfqpoint{1.606921in}{2.836740in}}%
\pgfpathcurveto{\pgfqpoint{1.612745in}{2.842564in}}{\pgfqpoint{1.616017in}{2.850464in}}{\pgfqpoint{1.616017in}{2.858700in}}%
\pgfpathcurveto{\pgfqpoint{1.616017in}{2.866937in}}{\pgfqpoint{1.612745in}{2.874837in}}{\pgfqpoint{1.606921in}{2.880661in}}%
\pgfpathcurveto{\pgfqpoint{1.601097in}{2.886485in}}{\pgfqpoint{1.593197in}{2.889757in}}{\pgfqpoint{1.584961in}{2.889757in}}%
\pgfpathcurveto{\pgfqpoint{1.576724in}{2.889757in}}{\pgfqpoint{1.568824in}{2.886485in}}{\pgfqpoint{1.563000in}{2.880661in}}%
\pgfpathcurveto{\pgfqpoint{1.557176in}{2.874837in}}{\pgfqpoint{1.553904in}{2.866937in}}{\pgfqpoint{1.553904in}{2.858700in}}%
\pgfpathcurveto{\pgfqpoint{1.553904in}{2.850464in}}{\pgfqpoint{1.557176in}{2.842564in}}{\pgfqpoint{1.563000in}{2.836740in}}%
\pgfpathcurveto{\pgfqpoint{1.568824in}{2.830916in}}{\pgfqpoint{1.576724in}{2.827644in}}{\pgfqpoint{1.584961in}{2.827644in}}%
\pgfpathclose%
\pgfusepath{stroke,fill}%
\end{pgfscope}%
\begin{pgfscope}%
\pgfpathrectangle{\pgfqpoint{0.100000in}{0.220728in}}{\pgfqpoint{3.696000in}{3.696000in}}%
\pgfusepath{clip}%
\pgfsetbuttcap%
\pgfsetroundjoin%
\definecolor{currentfill}{rgb}{0.121569,0.466667,0.705882}%
\pgfsetfillcolor{currentfill}%
\pgfsetfillopacity{0.370205}%
\pgfsetlinewidth{1.003750pt}%
\definecolor{currentstroke}{rgb}{0.121569,0.466667,0.705882}%
\pgfsetstrokecolor{currentstroke}%
\pgfsetstrokeopacity{0.370205}%
\pgfsetdash{}{0pt}%
\pgfpathmoveto{\pgfqpoint{2.062911in}{3.228542in}}%
\pgfpathcurveto{\pgfqpoint{2.071147in}{3.228542in}}{\pgfqpoint{2.079047in}{3.231815in}}{\pgfqpoint{2.084871in}{3.237639in}}%
\pgfpathcurveto{\pgfqpoint{2.090695in}{3.243463in}}{\pgfqpoint{2.093967in}{3.251363in}}{\pgfqpoint{2.093967in}{3.259599in}}%
\pgfpathcurveto{\pgfqpoint{2.093967in}{3.267835in}}{\pgfqpoint{2.090695in}{3.275735in}}{\pgfqpoint{2.084871in}{3.281559in}}%
\pgfpathcurveto{\pgfqpoint{2.079047in}{3.287383in}}{\pgfqpoint{2.071147in}{3.290655in}}{\pgfqpoint{2.062911in}{3.290655in}}%
\pgfpathcurveto{\pgfqpoint{2.054675in}{3.290655in}}{\pgfqpoint{2.046774in}{3.287383in}}{\pgfqpoint{2.040951in}{3.281559in}}%
\pgfpathcurveto{\pgfqpoint{2.035127in}{3.275735in}}{\pgfqpoint{2.031854in}{3.267835in}}{\pgfqpoint{2.031854in}{3.259599in}}%
\pgfpathcurveto{\pgfqpoint{2.031854in}{3.251363in}}{\pgfqpoint{2.035127in}{3.243463in}}{\pgfqpoint{2.040951in}{3.237639in}}%
\pgfpathcurveto{\pgfqpoint{2.046774in}{3.231815in}}{\pgfqpoint{2.054675in}{3.228542in}}{\pgfqpoint{2.062911in}{3.228542in}}%
\pgfpathclose%
\pgfusepath{stroke,fill}%
\end{pgfscope}%
\begin{pgfscope}%
\pgfpathrectangle{\pgfqpoint{0.100000in}{0.220728in}}{\pgfqpoint{3.696000in}{3.696000in}}%
\pgfusepath{clip}%
\pgfsetbuttcap%
\pgfsetroundjoin%
\definecolor{currentfill}{rgb}{0.121569,0.466667,0.705882}%
\pgfsetfillcolor{currentfill}%
\pgfsetfillopacity{0.373283}%
\pgfsetlinewidth{1.003750pt}%
\definecolor{currentstroke}{rgb}{0.121569,0.466667,0.705882}%
\pgfsetstrokecolor{currentstroke}%
\pgfsetstrokeopacity{0.373283}%
\pgfsetdash{}{0pt}%
\pgfpathmoveto{\pgfqpoint{1.582709in}{2.805114in}}%
\pgfpathcurveto{\pgfqpoint{1.590945in}{2.805114in}}{\pgfqpoint{1.598846in}{2.808386in}}{\pgfqpoint{1.604669in}{2.814210in}}%
\pgfpathcurveto{\pgfqpoint{1.610493in}{2.820034in}}{\pgfqpoint{1.613766in}{2.827934in}}{\pgfqpoint{1.613766in}{2.836170in}}%
\pgfpathcurveto{\pgfqpoint{1.613766in}{2.844406in}}{\pgfqpoint{1.610493in}{2.852307in}}{\pgfqpoint{1.604669in}{2.858130in}}%
\pgfpathcurveto{\pgfqpoint{1.598846in}{2.863954in}}{\pgfqpoint{1.590945in}{2.867227in}}{\pgfqpoint{1.582709in}{2.867227in}}%
\pgfpathcurveto{\pgfqpoint{1.574473in}{2.867227in}}{\pgfqpoint{1.566573in}{2.863954in}}{\pgfqpoint{1.560749in}{2.858130in}}%
\pgfpathcurveto{\pgfqpoint{1.554925in}{2.852307in}}{\pgfqpoint{1.551653in}{2.844406in}}{\pgfqpoint{1.551653in}{2.836170in}}%
\pgfpathcurveto{\pgfqpoint{1.551653in}{2.827934in}}{\pgfqpoint{1.554925in}{2.820034in}}{\pgfqpoint{1.560749in}{2.814210in}}%
\pgfpathcurveto{\pgfqpoint{1.566573in}{2.808386in}}{\pgfqpoint{1.574473in}{2.805114in}}{\pgfqpoint{1.582709in}{2.805114in}}%
\pgfpathclose%
\pgfusepath{stroke,fill}%
\end{pgfscope}%
\begin{pgfscope}%
\pgfpathrectangle{\pgfqpoint{0.100000in}{0.220728in}}{\pgfqpoint{3.696000in}{3.696000in}}%
\pgfusepath{clip}%
\pgfsetbuttcap%
\pgfsetroundjoin%
\definecolor{currentfill}{rgb}{0.121569,0.466667,0.705882}%
\pgfsetfillcolor{currentfill}%
\pgfsetfillopacity{0.373923}%
\pgfsetlinewidth{1.003750pt}%
\definecolor{currentstroke}{rgb}{0.121569,0.466667,0.705882}%
\pgfsetstrokecolor{currentstroke}%
\pgfsetstrokeopacity{0.373923}%
\pgfsetdash{}{0pt}%
\pgfpathmoveto{\pgfqpoint{2.077734in}{3.227875in}}%
\pgfpathcurveto{\pgfqpoint{2.085971in}{3.227875in}}{\pgfqpoint{2.093871in}{3.231147in}}{\pgfqpoint{2.099695in}{3.236971in}}%
\pgfpathcurveto{\pgfqpoint{2.105519in}{3.242795in}}{\pgfqpoint{2.108791in}{3.250695in}}{\pgfqpoint{2.108791in}{3.258932in}}%
\pgfpathcurveto{\pgfqpoint{2.108791in}{3.267168in}}{\pgfqpoint{2.105519in}{3.275068in}}{\pgfqpoint{2.099695in}{3.280892in}}%
\pgfpathcurveto{\pgfqpoint{2.093871in}{3.286716in}}{\pgfqpoint{2.085971in}{3.289988in}}{\pgfqpoint{2.077734in}{3.289988in}}%
\pgfpathcurveto{\pgfqpoint{2.069498in}{3.289988in}}{\pgfqpoint{2.061598in}{3.286716in}}{\pgfqpoint{2.055774in}{3.280892in}}%
\pgfpathcurveto{\pgfqpoint{2.049950in}{3.275068in}}{\pgfqpoint{2.046678in}{3.267168in}}{\pgfqpoint{2.046678in}{3.258932in}}%
\pgfpathcurveto{\pgfqpoint{2.046678in}{3.250695in}}{\pgfqpoint{2.049950in}{3.242795in}}{\pgfqpoint{2.055774in}{3.236971in}}%
\pgfpathcurveto{\pgfqpoint{2.061598in}{3.231147in}}{\pgfqpoint{2.069498in}{3.227875in}}{\pgfqpoint{2.077734in}{3.227875in}}%
\pgfpathclose%
\pgfusepath{stroke,fill}%
\end{pgfscope}%
\begin{pgfscope}%
\pgfpathrectangle{\pgfqpoint{0.100000in}{0.220728in}}{\pgfqpoint{3.696000in}{3.696000in}}%
\pgfusepath{clip}%
\pgfsetbuttcap%
\pgfsetroundjoin%
\definecolor{currentfill}{rgb}{0.121569,0.466667,0.705882}%
\pgfsetfillcolor{currentfill}%
\pgfsetfillopacity{0.375026}%
\pgfsetlinewidth{1.003750pt}%
\definecolor{currentstroke}{rgb}{0.121569,0.466667,0.705882}%
\pgfsetstrokecolor{currentstroke}%
\pgfsetstrokeopacity{0.375026}%
\pgfsetdash{}{0pt}%
\pgfpathmoveto{\pgfqpoint{1.569784in}{2.789959in}}%
\pgfpathcurveto{\pgfqpoint{1.578020in}{2.789959in}}{\pgfqpoint{1.585920in}{2.793231in}}{\pgfqpoint{1.591744in}{2.799055in}}%
\pgfpathcurveto{\pgfqpoint{1.597568in}{2.804879in}}{\pgfqpoint{1.600840in}{2.812779in}}{\pgfqpoint{1.600840in}{2.821016in}}%
\pgfpathcurveto{\pgfqpoint{1.600840in}{2.829252in}}{\pgfqpoint{1.597568in}{2.837152in}}{\pgfqpoint{1.591744in}{2.842976in}}%
\pgfpathcurveto{\pgfqpoint{1.585920in}{2.848800in}}{\pgfqpoint{1.578020in}{2.852072in}}{\pgfqpoint{1.569784in}{2.852072in}}%
\pgfpathcurveto{\pgfqpoint{1.561547in}{2.852072in}}{\pgfqpoint{1.553647in}{2.848800in}}{\pgfqpoint{1.547823in}{2.842976in}}%
\pgfpathcurveto{\pgfqpoint{1.541999in}{2.837152in}}{\pgfqpoint{1.538727in}{2.829252in}}{\pgfqpoint{1.538727in}{2.821016in}}%
\pgfpathcurveto{\pgfqpoint{1.538727in}{2.812779in}}{\pgfqpoint{1.541999in}{2.804879in}}{\pgfqpoint{1.547823in}{2.799055in}}%
\pgfpathcurveto{\pgfqpoint{1.553647in}{2.793231in}}{\pgfqpoint{1.561547in}{2.789959in}}{\pgfqpoint{1.569784in}{2.789959in}}%
\pgfpathclose%
\pgfusepath{stroke,fill}%
\end{pgfscope}%
\begin{pgfscope}%
\pgfpathrectangle{\pgfqpoint{0.100000in}{0.220728in}}{\pgfqpoint{3.696000in}{3.696000in}}%
\pgfusepath{clip}%
\pgfsetbuttcap%
\pgfsetroundjoin%
\definecolor{currentfill}{rgb}{0.121569,0.466667,0.705882}%
\pgfsetfillcolor{currentfill}%
\pgfsetfillopacity{0.377738}%
\pgfsetlinewidth{1.003750pt}%
\definecolor{currentstroke}{rgb}{0.121569,0.466667,0.705882}%
\pgfsetstrokecolor{currentstroke}%
\pgfsetstrokeopacity{0.377738}%
\pgfsetdash{}{0pt}%
\pgfpathmoveto{\pgfqpoint{1.565440in}{2.774171in}}%
\pgfpathcurveto{\pgfqpoint{1.573676in}{2.774171in}}{\pgfqpoint{1.581576in}{2.777444in}}{\pgfqpoint{1.587400in}{2.783268in}}%
\pgfpathcurveto{\pgfqpoint{1.593224in}{2.789091in}}{\pgfqpoint{1.596497in}{2.796992in}}{\pgfqpoint{1.596497in}{2.805228in}}%
\pgfpathcurveto{\pgfqpoint{1.596497in}{2.813464in}}{\pgfqpoint{1.593224in}{2.821364in}}{\pgfqpoint{1.587400in}{2.827188in}}%
\pgfpathcurveto{\pgfqpoint{1.581576in}{2.833012in}}{\pgfqpoint{1.573676in}{2.836284in}}{\pgfqpoint{1.565440in}{2.836284in}}%
\pgfpathcurveto{\pgfqpoint{1.557204in}{2.836284in}}{\pgfqpoint{1.549304in}{2.833012in}}{\pgfqpoint{1.543480in}{2.827188in}}%
\pgfpathcurveto{\pgfqpoint{1.537656in}{2.821364in}}{\pgfqpoint{1.534384in}{2.813464in}}{\pgfqpoint{1.534384in}{2.805228in}}%
\pgfpathcurveto{\pgfqpoint{1.534384in}{2.796992in}}{\pgfqpoint{1.537656in}{2.789091in}}{\pgfqpoint{1.543480in}{2.783268in}}%
\pgfpathcurveto{\pgfqpoint{1.549304in}{2.777444in}}{\pgfqpoint{1.557204in}{2.774171in}}{\pgfqpoint{1.565440in}{2.774171in}}%
\pgfpathclose%
\pgfusepath{stroke,fill}%
\end{pgfscope}%
\begin{pgfscope}%
\pgfpathrectangle{\pgfqpoint{0.100000in}{0.220728in}}{\pgfqpoint{3.696000in}{3.696000in}}%
\pgfusepath{clip}%
\pgfsetbuttcap%
\pgfsetroundjoin%
\definecolor{currentfill}{rgb}{0.121569,0.466667,0.705882}%
\pgfsetfillcolor{currentfill}%
\pgfsetfillopacity{0.377743}%
\pgfsetlinewidth{1.003750pt}%
\definecolor{currentstroke}{rgb}{0.121569,0.466667,0.705882}%
\pgfsetstrokecolor{currentstroke}%
\pgfsetstrokeopacity{0.377743}%
\pgfsetdash{}{0pt}%
\pgfpathmoveto{\pgfqpoint{2.092427in}{3.225157in}}%
\pgfpathcurveto{\pgfqpoint{2.100663in}{3.225157in}}{\pgfqpoint{2.108563in}{3.228429in}}{\pgfqpoint{2.114387in}{3.234253in}}%
\pgfpathcurveto{\pgfqpoint{2.120211in}{3.240077in}}{\pgfqpoint{2.123483in}{3.247977in}}{\pgfqpoint{2.123483in}{3.256213in}}%
\pgfpathcurveto{\pgfqpoint{2.123483in}{3.264449in}}{\pgfqpoint{2.120211in}{3.272349in}}{\pgfqpoint{2.114387in}{3.278173in}}%
\pgfpathcurveto{\pgfqpoint{2.108563in}{3.283997in}}{\pgfqpoint{2.100663in}{3.287270in}}{\pgfqpoint{2.092427in}{3.287270in}}%
\pgfpathcurveto{\pgfqpoint{2.084190in}{3.287270in}}{\pgfqpoint{2.076290in}{3.283997in}}{\pgfqpoint{2.070466in}{3.278173in}}%
\pgfpathcurveto{\pgfqpoint{2.064642in}{3.272349in}}{\pgfqpoint{2.061370in}{3.264449in}}{\pgfqpoint{2.061370in}{3.256213in}}%
\pgfpathcurveto{\pgfqpoint{2.061370in}{3.247977in}}{\pgfqpoint{2.064642in}{3.240077in}}{\pgfqpoint{2.070466in}{3.234253in}}%
\pgfpathcurveto{\pgfqpoint{2.076290in}{3.228429in}}{\pgfqpoint{2.084190in}{3.225157in}}{\pgfqpoint{2.092427in}{3.225157in}}%
\pgfpathclose%
\pgfusepath{stroke,fill}%
\end{pgfscope}%
\begin{pgfscope}%
\pgfpathrectangle{\pgfqpoint{0.100000in}{0.220728in}}{\pgfqpoint{3.696000in}{3.696000in}}%
\pgfusepath{clip}%
\pgfsetbuttcap%
\pgfsetroundjoin%
\definecolor{currentfill}{rgb}{0.121569,0.466667,0.705882}%
\pgfsetfillcolor{currentfill}%
\pgfsetfillopacity{0.379876}%
\pgfsetlinewidth{1.003750pt}%
\definecolor{currentstroke}{rgb}{0.121569,0.466667,0.705882}%
\pgfsetstrokecolor{currentstroke}%
\pgfsetstrokeopacity{0.379876}%
\pgfsetdash{}{0pt}%
\pgfpathmoveto{\pgfqpoint{1.558296in}{2.760355in}}%
\pgfpathcurveto{\pgfqpoint{1.566532in}{2.760355in}}{\pgfqpoint{1.574432in}{2.763627in}}{\pgfqpoint{1.580256in}{2.769451in}}%
\pgfpathcurveto{\pgfqpoint{1.586080in}{2.775275in}}{\pgfqpoint{1.589352in}{2.783175in}}{\pgfqpoint{1.589352in}{2.791411in}}%
\pgfpathcurveto{\pgfqpoint{1.589352in}{2.799648in}}{\pgfqpoint{1.586080in}{2.807548in}}{\pgfqpoint{1.580256in}{2.813372in}}%
\pgfpathcurveto{\pgfqpoint{1.574432in}{2.819196in}}{\pgfqpoint{1.566532in}{2.822468in}}{\pgfqpoint{1.558296in}{2.822468in}}%
\pgfpathcurveto{\pgfqpoint{1.550060in}{2.822468in}}{\pgfqpoint{1.542160in}{2.819196in}}{\pgfqpoint{1.536336in}{2.813372in}}%
\pgfpathcurveto{\pgfqpoint{1.530512in}{2.807548in}}{\pgfqpoint{1.527239in}{2.799648in}}{\pgfqpoint{1.527239in}{2.791411in}}%
\pgfpathcurveto{\pgfqpoint{1.527239in}{2.783175in}}{\pgfqpoint{1.530512in}{2.775275in}}{\pgfqpoint{1.536336in}{2.769451in}}%
\pgfpathcurveto{\pgfqpoint{1.542160in}{2.763627in}}{\pgfqpoint{1.550060in}{2.760355in}}{\pgfqpoint{1.558296in}{2.760355in}}%
\pgfpathclose%
\pgfusepath{stroke,fill}%
\end{pgfscope}%
\begin{pgfscope}%
\pgfpathrectangle{\pgfqpoint{0.100000in}{0.220728in}}{\pgfqpoint{3.696000in}{3.696000in}}%
\pgfusepath{clip}%
\pgfsetbuttcap%
\pgfsetroundjoin%
\definecolor{currentfill}{rgb}{0.121569,0.466667,0.705882}%
\pgfsetfillcolor{currentfill}%
\pgfsetfillopacity{0.380092}%
\pgfsetlinewidth{1.003750pt}%
\definecolor{currentstroke}{rgb}{0.121569,0.466667,0.705882}%
\pgfsetstrokecolor{currentstroke}%
\pgfsetstrokeopacity{0.380092}%
\pgfsetdash{}{0pt}%
\pgfpathmoveto{\pgfqpoint{2.100427in}{3.224472in}}%
\pgfpathcurveto{\pgfqpoint{2.108664in}{3.224472in}}{\pgfqpoint{2.116564in}{3.227744in}}{\pgfqpoint{2.122387in}{3.233568in}}%
\pgfpathcurveto{\pgfqpoint{2.128211in}{3.239392in}}{\pgfqpoint{2.131484in}{3.247292in}}{\pgfqpoint{2.131484in}{3.255528in}}%
\pgfpathcurveto{\pgfqpoint{2.131484in}{3.263764in}}{\pgfqpoint{2.128211in}{3.271664in}}{\pgfqpoint{2.122387in}{3.277488in}}%
\pgfpathcurveto{\pgfqpoint{2.116564in}{3.283312in}}{\pgfqpoint{2.108664in}{3.286585in}}{\pgfqpoint{2.100427in}{3.286585in}}%
\pgfpathcurveto{\pgfqpoint{2.092191in}{3.286585in}}{\pgfqpoint{2.084291in}{3.283312in}}{\pgfqpoint{2.078467in}{3.277488in}}%
\pgfpathcurveto{\pgfqpoint{2.072643in}{3.271664in}}{\pgfqpoint{2.069371in}{3.263764in}}{\pgfqpoint{2.069371in}{3.255528in}}%
\pgfpathcurveto{\pgfqpoint{2.069371in}{3.247292in}}{\pgfqpoint{2.072643in}{3.239392in}}{\pgfqpoint{2.078467in}{3.233568in}}%
\pgfpathcurveto{\pgfqpoint{2.084291in}{3.227744in}}{\pgfqpoint{2.092191in}{3.224472in}}{\pgfqpoint{2.100427in}{3.224472in}}%
\pgfpathclose%
\pgfusepath{stroke,fill}%
\end{pgfscope}%
\begin{pgfscope}%
\pgfpathrectangle{\pgfqpoint{0.100000in}{0.220728in}}{\pgfqpoint{3.696000in}{3.696000in}}%
\pgfusepath{clip}%
\pgfsetbuttcap%
\pgfsetroundjoin%
\definecolor{currentfill}{rgb}{0.121569,0.466667,0.705882}%
\pgfsetfillcolor{currentfill}%
\pgfsetfillopacity{0.381333}%
\pgfsetlinewidth{1.003750pt}%
\definecolor{currentstroke}{rgb}{0.121569,0.466667,0.705882}%
\pgfsetstrokecolor{currentstroke}%
\pgfsetstrokeopacity{0.381333}%
\pgfsetdash{}{0pt}%
\pgfpathmoveto{\pgfqpoint{2.109878in}{3.221927in}}%
\pgfpathcurveto{\pgfqpoint{2.118114in}{3.221927in}}{\pgfqpoint{2.126014in}{3.225199in}}{\pgfqpoint{2.131838in}{3.231023in}}%
\pgfpathcurveto{\pgfqpoint{2.137662in}{3.236847in}}{\pgfqpoint{2.140934in}{3.244747in}}{\pgfqpoint{2.140934in}{3.252983in}}%
\pgfpathcurveto{\pgfqpoint{2.140934in}{3.261219in}}{\pgfqpoint{2.137662in}{3.269120in}}{\pgfqpoint{2.131838in}{3.274943in}}%
\pgfpathcurveto{\pgfqpoint{2.126014in}{3.280767in}}{\pgfqpoint{2.118114in}{3.284040in}}{\pgfqpoint{2.109878in}{3.284040in}}%
\pgfpathcurveto{\pgfqpoint{2.101641in}{3.284040in}}{\pgfqpoint{2.093741in}{3.280767in}}{\pgfqpoint{2.087917in}{3.274943in}}%
\pgfpathcurveto{\pgfqpoint{2.082093in}{3.269120in}}{\pgfqpoint{2.078821in}{3.261219in}}{\pgfqpoint{2.078821in}{3.252983in}}%
\pgfpathcurveto{\pgfqpoint{2.078821in}{3.244747in}}{\pgfqpoint{2.082093in}{3.236847in}}{\pgfqpoint{2.087917in}{3.231023in}}%
\pgfpathcurveto{\pgfqpoint{2.093741in}{3.225199in}}{\pgfqpoint{2.101641in}{3.221927in}}{\pgfqpoint{2.109878in}{3.221927in}}%
\pgfpathclose%
\pgfusepath{stroke,fill}%
\end{pgfscope}%
\begin{pgfscope}%
\pgfpathrectangle{\pgfqpoint{0.100000in}{0.220728in}}{\pgfqpoint{3.696000in}{3.696000in}}%
\pgfusepath{clip}%
\pgfsetbuttcap%
\pgfsetroundjoin%
\definecolor{currentfill}{rgb}{0.121569,0.466667,0.705882}%
\pgfsetfillcolor{currentfill}%
\pgfsetfillopacity{0.382228}%
\pgfsetlinewidth{1.003750pt}%
\definecolor{currentstroke}{rgb}{0.121569,0.466667,0.705882}%
\pgfsetstrokecolor{currentstroke}%
\pgfsetstrokeopacity{0.382228}%
\pgfsetdash{}{0pt}%
\pgfpathmoveto{\pgfqpoint{1.553662in}{2.747494in}}%
\pgfpathcurveto{\pgfqpoint{1.561899in}{2.747494in}}{\pgfqpoint{1.569799in}{2.750766in}}{\pgfqpoint{1.575623in}{2.756590in}}%
\pgfpathcurveto{\pgfqpoint{1.581447in}{2.762414in}}{\pgfqpoint{1.584719in}{2.770314in}}{\pgfqpoint{1.584719in}{2.778550in}}%
\pgfpathcurveto{\pgfqpoint{1.584719in}{2.786787in}}{\pgfqpoint{1.581447in}{2.794687in}}{\pgfqpoint{1.575623in}{2.800510in}}%
\pgfpathcurveto{\pgfqpoint{1.569799in}{2.806334in}}{\pgfqpoint{1.561899in}{2.809607in}}{\pgfqpoint{1.553662in}{2.809607in}}%
\pgfpathcurveto{\pgfqpoint{1.545426in}{2.809607in}}{\pgfqpoint{1.537526in}{2.806334in}}{\pgfqpoint{1.531702in}{2.800510in}}%
\pgfpathcurveto{\pgfqpoint{1.525878in}{2.794687in}}{\pgfqpoint{1.522606in}{2.786787in}}{\pgfqpoint{1.522606in}{2.778550in}}%
\pgfpathcurveto{\pgfqpoint{1.522606in}{2.770314in}}{\pgfqpoint{1.525878in}{2.762414in}}{\pgfqpoint{1.531702in}{2.756590in}}%
\pgfpathcurveto{\pgfqpoint{1.537526in}{2.750766in}}{\pgfqpoint{1.545426in}{2.747494in}}{\pgfqpoint{1.553662in}{2.747494in}}%
\pgfpathclose%
\pgfusepath{stroke,fill}%
\end{pgfscope}%
\begin{pgfscope}%
\pgfpathrectangle{\pgfqpoint{0.100000in}{0.220728in}}{\pgfqpoint{3.696000in}{3.696000in}}%
\pgfusepath{clip}%
\pgfsetbuttcap%
\pgfsetroundjoin%
\definecolor{currentfill}{rgb}{0.121569,0.466667,0.705882}%
\pgfsetfillcolor{currentfill}%
\pgfsetfillopacity{0.383045}%
\pgfsetlinewidth{1.003750pt}%
\definecolor{currentstroke}{rgb}{0.121569,0.466667,0.705882}%
\pgfsetstrokecolor{currentstroke}%
\pgfsetstrokeopacity{0.383045}%
\pgfsetdash{}{0pt}%
\pgfpathmoveto{\pgfqpoint{2.114131in}{3.221942in}}%
\pgfpathcurveto{\pgfqpoint{2.122367in}{3.221942in}}{\pgfqpoint{2.130267in}{3.225214in}}{\pgfqpoint{2.136091in}{3.231038in}}%
\pgfpathcurveto{\pgfqpoint{2.141915in}{3.236862in}}{\pgfqpoint{2.145187in}{3.244762in}}{\pgfqpoint{2.145187in}{3.252999in}}%
\pgfpathcurveto{\pgfqpoint{2.145187in}{3.261235in}}{\pgfqpoint{2.141915in}{3.269135in}}{\pgfqpoint{2.136091in}{3.274959in}}%
\pgfpathcurveto{\pgfqpoint{2.130267in}{3.280783in}}{\pgfqpoint{2.122367in}{3.284055in}}{\pgfqpoint{2.114131in}{3.284055in}}%
\pgfpathcurveto{\pgfqpoint{2.105895in}{3.284055in}}{\pgfqpoint{2.097995in}{3.280783in}}{\pgfqpoint{2.092171in}{3.274959in}}%
\pgfpathcurveto{\pgfqpoint{2.086347in}{3.269135in}}{\pgfqpoint{2.083074in}{3.261235in}}{\pgfqpoint{2.083074in}{3.252999in}}%
\pgfpathcurveto{\pgfqpoint{2.083074in}{3.244762in}}{\pgfqpoint{2.086347in}{3.236862in}}{\pgfqpoint{2.092171in}{3.231038in}}%
\pgfpathcurveto{\pgfqpoint{2.097995in}{3.225214in}}{\pgfqpoint{2.105895in}{3.221942in}}{\pgfqpoint{2.114131in}{3.221942in}}%
\pgfpathclose%
\pgfusepath{stroke,fill}%
\end{pgfscope}%
\begin{pgfscope}%
\pgfpathrectangle{\pgfqpoint{0.100000in}{0.220728in}}{\pgfqpoint{3.696000in}{3.696000in}}%
\pgfusepath{clip}%
\pgfsetbuttcap%
\pgfsetroundjoin%
\definecolor{currentfill}{rgb}{0.121569,0.466667,0.705882}%
\pgfsetfillcolor{currentfill}%
\pgfsetfillopacity{0.384396}%
\pgfsetlinewidth{1.003750pt}%
\definecolor{currentstroke}{rgb}{0.121569,0.466667,0.705882}%
\pgfsetstrokecolor{currentstroke}%
\pgfsetstrokeopacity{0.384396}%
\pgfsetdash{}{0pt}%
\pgfpathmoveto{\pgfqpoint{2.121897in}{3.220441in}}%
\pgfpathcurveto{\pgfqpoint{2.130134in}{3.220441in}}{\pgfqpoint{2.138034in}{3.223714in}}{\pgfqpoint{2.143857in}{3.229538in}}%
\pgfpathcurveto{\pgfqpoint{2.149681in}{3.235362in}}{\pgfqpoint{2.152954in}{3.243262in}}{\pgfqpoint{2.152954in}{3.251498in}}%
\pgfpathcurveto{\pgfqpoint{2.152954in}{3.259734in}}{\pgfqpoint{2.149681in}{3.267634in}}{\pgfqpoint{2.143857in}{3.273458in}}%
\pgfpathcurveto{\pgfqpoint{2.138034in}{3.279282in}}{\pgfqpoint{2.130134in}{3.282554in}}{\pgfqpoint{2.121897in}{3.282554in}}%
\pgfpathcurveto{\pgfqpoint{2.113661in}{3.282554in}}{\pgfqpoint{2.105761in}{3.279282in}}{\pgfqpoint{2.099937in}{3.273458in}}%
\pgfpathcurveto{\pgfqpoint{2.094113in}{3.267634in}}{\pgfqpoint{2.090841in}{3.259734in}}{\pgfqpoint{2.090841in}{3.251498in}}%
\pgfpathcurveto{\pgfqpoint{2.090841in}{3.243262in}}{\pgfqpoint{2.094113in}{3.235362in}}{\pgfqpoint{2.099937in}{3.229538in}}%
\pgfpathcurveto{\pgfqpoint{2.105761in}{3.223714in}}{\pgfqpoint{2.113661in}{3.220441in}}{\pgfqpoint{2.121897in}{3.220441in}}%
\pgfpathclose%
\pgfusepath{stroke,fill}%
\end{pgfscope}%
\begin{pgfscope}%
\pgfpathrectangle{\pgfqpoint{0.100000in}{0.220728in}}{\pgfqpoint{3.696000in}{3.696000in}}%
\pgfusepath{clip}%
\pgfsetbuttcap%
\pgfsetroundjoin%
\definecolor{currentfill}{rgb}{0.121569,0.466667,0.705882}%
\pgfsetfillcolor{currentfill}%
\pgfsetfillopacity{0.386261}%
\pgfsetlinewidth{1.003750pt}%
\definecolor{currentstroke}{rgb}{0.121569,0.466667,0.705882}%
\pgfsetstrokecolor{currentstroke}%
\pgfsetstrokeopacity{0.386261}%
\pgfsetdash{}{0pt}%
\pgfpathmoveto{\pgfqpoint{1.543920in}{2.724082in}}%
\pgfpathcurveto{\pgfqpoint{1.552157in}{2.724082in}}{\pgfqpoint{1.560057in}{2.727354in}}{\pgfqpoint{1.565881in}{2.733178in}}%
\pgfpathcurveto{\pgfqpoint{1.571705in}{2.739002in}}{\pgfqpoint{1.574977in}{2.746902in}}{\pgfqpoint{1.574977in}{2.755139in}}%
\pgfpathcurveto{\pgfqpoint{1.574977in}{2.763375in}}{\pgfqpoint{1.571705in}{2.771275in}}{\pgfqpoint{1.565881in}{2.777099in}}%
\pgfpathcurveto{\pgfqpoint{1.560057in}{2.782923in}}{\pgfqpoint{1.552157in}{2.786195in}}{\pgfqpoint{1.543920in}{2.786195in}}%
\pgfpathcurveto{\pgfqpoint{1.535684in}{2.786195in}}{\pgfqpoint{1.527784in}{2.782923in}}{\pgfqpoint{1.521960in}{2.777099in}}%
\pgfpathcurveto{\pgfqpoint{1.516136in}{2.771275in}}{\pgfqpoint{1.512864in}{2.763375in}}{\pgfqpoint{1.512864in}{2.755139in}}%
\pgfpathcurveto{\pgfqpoint{1.512864in}{2.746902in}}{\pgfqpoint{1.516136in}{2.739002in}}{\pgfqpoint{1.521960in}{2.733178in}}%
\pgfpathcurveto{\pgfqpoint{1.527784in}{2.727354in}}{\pgfqpoint{1.535684in}{2.724082in}}{\pgfqpoint{1.543920in}{2.724082in}}%
\pgfpathclose%
\pgfusepath{stroke,fill}%
\end{pgfscope}%
\begin{pgfscope}%
\pgfpathrectangle{\pgfqpoint{0.100000in}{0.220728in}}{\pgfqpoint{3.696000in}{3.696000in}}%
\pgfusepath{clip}%
\pgfsetbuttcap%
\pgfsetroundjoin%
\definecolor{currentfill}{rgb}{0.121569,0.466667,0.705882}%
\pgfsetfillcolor{currentfill}%
\pgfsetfillopacity{0.386437}%
\pgfsetlinewidth{1.003750pt}%
\definecolor{currentstroke}{rgb}{0.121569,0.466667,0.705882}%
\pgfsetstrokecolor{currentstroke}%
\pgfsetstrokeopacity{0.386437}%
\pgfsetdash{}{0pt}%
\pgfpathmoveto{\pgfqpoint{2.129772in}{3.219465in}}%
\pgfpathcurveto{\pgfqpoint{2.138008in}{3.219465in}}{\pgfqpoint{2.145908in}{3.222737in}}{\pgfqpoint{2.151732in}{3.228561in}}%
\pgfpathcurveto{\pgfqpoint{2.157556in}{3.234385in}}{\pgfqpoint{2.160828in}{3.242285in}}{\pgfqpoint{2.160828in}{3.250522in}}%
\pgfpathcurveto{\pgfqpoint{2.160828in}{3.258758in}}{\pgfqpoint{2.157556in}{3.266658in}}{\pgfqpoint{2.151732in}{3.272482in}}%
\pgfpathcurveto{\pgfqpoint{2.145908in}{3.278306in}}{\pgfqpoint{2.138008in}{3.281578in}}{\pgfqpoint{2.129772in}{3.281578in}}%
\pgfpathcurveto{\pgfqpoint{2.121536in}{3.281578in}}{\pgfqpoint{2.113636in}{3.278306in}}{\pgfqpoint{2.107812in}{3.272482in}}%
\pgfpathcurveto{\pgfqpoint{2.101988in}{3.266658in}}{\pgfqpoint{2.098715in}{3.258758in}}{\pgfqpoint{2.098715in}{3.250522in}}%
\pgfpathcurveto{\pgfqpoint{2.098715in}{3.242285in}}{\pgfqpoint{2.101988in}{3.234385in}}{\pgfqpoint{2.107812in}{3.228561in}}%
\pgfpathcurveto{\pgfqpoint{2.113636in}{3.222737in}}{\pgfqpoint{2.121536in}{3.219465in}}{\pgfqpoint{2.129772in}{3.219465in}}%
\pgfpathclose%
\pgfusepath{stroke,fill}%
\end{pgfscope}%
\begin{pgfscope}%
\pgfpathrectangle{\pgfqpoint{0.100000in}{0.220728in}}{\pgfqpoint{3.696000in}{3.696000in}}%
\pgfusepath{clip}%
\pgfsetbuttcap%
\pgfsetroundjoin%
\definecolor{currentfill}{rgb}{0.121569,0.466667,0.705882}%
\pgfsetfillcolor{currentfill}%
\pgfsetfillopacity{0.387898}%
\pgfsetlinewidth{1.003750pt}%
\definecolor{currentstroke}{rgb}{0.121569,0.466667,0.705882}%
\pgfsetstrokecolor{currentstroke}%
\pgfsetstrokeopacity{0.387898}%
\pgfsetdash{}{0pt}%
\pgfpathmoveto{\pgfqpoint{2.142876in}{3.218199in}}%
\pgfpathcurveto{\pgfqpoint{2.151112in}{3.218199in}}{\pgfqpoint{2.159012in}{3.221471in}}{\pgfqpoint{2.164836in}{3.227295in}}%
\pgfpathcurveto{\pgfqpoint{2.170660in}{3.233119in}}{\pgfqpoint{2.173932in}{3.241019in}}{\pgfqpoint{2.173932in}{3.249255in}}%
\pgfpathcurveto{\pgfqpoint{2.173932in}{3.257492in}}{\pgfqpoint{2.170660in}{3.265392in}}{\pgfqpoint{2.164836in}{3.271216in}}%
\pgfpathcurveto{\pgfqpoint{2.159012in}{3.277039in}}{\pgfqpoint{2.151112in}{3.280312in}}{\pgfqpoint{2.142876in}{3.280312in}}%
\pgfpathcurveto{\pgfqpoint{2.134639in}{3.280312in}}{\pgfqpoint{2.126739in}{3.277039in}}{\pgfqpoint{2.120915in}{3.271216in}}%
\pgfpathcurveto{\pgfqpoint{2.115091in}{3.265392in}}{\pgfqpoint{2.111819in}{3.257492in}}{\pgfqpoint{2.111819in}{3.249255in}}%
\pgfpathcurveto{\pgfqpoint{2.111819in}{3.241019in}}{\pgfqpoint{2.115091in}{3.233119in}}{\pgfqpoint{2.120915in}{3.227295in}}%
\pgfpathcurveto{\pgfqpoint{2.126739in}{3.221471in}}{\pgfqpoint{2.134639in}{3.218199in}}{\pgfqpoint{2.142876in}{3.218199in}}%
\pgfpathclose%
\pgfusepath{stroke,fill}%
\end{pgfscope}%
\begin{pgfscope}%
\pgfpathrectangle{\pgfqpoint{0.100000in}{0.220728in}}{\pgfqpoint{3.696000in}{3.696000in}}%
\pgfusepath{clip}%
\pgfsetbuttcap%
\pgfsetroundjoin%
\definecolor{currentfill}{rgb}{0.121569,0.466667,0.705882}%
\pgfsetfillcolor{currentfill}%
\pgfsetfillopacity{0.389370}%
\pgfsetlinewidth{1.003750pt}%
\definecolor{currentstroke}{rgb}{0.121569,0.466667,0.705882}%
\pgfsetstrokecolor{currentstroke}%
\pgfsetstrokeopacity{0.389370}%
\pgfsetdash{}{0pt}%
\pgfpathmoveto{\pgfqpoint{2.137613in}{3.218901in}}%
\pgfpathcurveto{\pgfqpoint{2.145850in}{3.218901in}}{\pgfqpoint{2.153750in}{3.222173in}}{\pgfqpoint{2.159574in}{3.227997in}}%
\pgfpathcurveto{\pgfqpoint{2.165398in}{3.233821in}}{\pgfqpoint{2.168670in}{3.241721in}}{\pgfqpoint{2.168670in}{3.249957in}}%
\pgfpathcurveto{\pgfqpoint{2.168670in}{3.258194in}}{\pgfqpoint{2.165398in}{3.266094in}}{\pgfqpoint{2.159574in}{3.271918in}}%
\pgfpathcurveto{\pgfqpoint{2.153750in}{3.277742in}}{\pgfqpoint{2.145850in}{3.281014in}}{\pgfqpoint{2.137613in}{3.281014in}}%
\pgfpathcurveto{\pgfqpoint{2.129377in}{3.281014in}}{\pgfqpoint{2.121477in}{3.277742in}}{\pgfqpoint{2.115653in}{3.271918in}}%
\pgfpathcurveto{\pgfqpoint{2.109829in}{3.266094in}}{\pgfqpoint{2.106557in}{3.258194in}}{\pgfqpoint{2.106557in}{3.249957in}}%
\pgfpathcurveto{\pgfqpoint{2.106557in}{3.241721in}}{\pgfqpoint{2.109829in}{3.233821in}}{\pgfqpoint{2.115653in}{3.227997in}}%
\pgfpathcurveto{\pgfqpoint{2.121477in}{3.222173in}}{\pgfqpoint{2.129377in}{3.218901in}}{\pgfqpoint{2.137613in}{3.218901in}}%
\pgfpathclose%
\pgfusepath{stroke,fill}%
\end{pgfscope}%
\begin{pgfscope}%
\pgfpathrectangle{\pgfqpoint{0.100000in}{0.220728in}}{\pgfqpoint{3.696000in}{3.696000in}}%
\pgfusepath{clip}%
\pgfsetbuttcap%
\pgfsetroundjoin%
\definecolor{currentfill}{rgb}{0.121569,0.466667,0.705882}%
\pgfsetfillcolor{currentfill}%
\pgfsetfillopacity{0.389706}%
\pgfsetlinewidth{1.003750pt}%
\definecolor{currentstroke}{rgb}{0.121569,0.466667,0.705882}%
\pgfsetstrokecolor{currentstroke}%
\pgfsetstrokeopacity{0.389706}%
\pgfsetdash{}{0pt}%
\pgfpathmoveto{\pgfqpoint{1.532457in}{2.702509in}}%
\pgfpathcurveto{\pgfqpoint{1.540694in}{2.702509in}}{\pgfqpoint{1.548594in}{2.705782in}}{\pgfqpoint{1.554418in}{2.711606in}}%
\pgfpathcurveto{\pgfqpoint{1.560242in}{2.717430in}}{\pgfqpoint{1.563514in}{2.725330in}}{\pgfqpoint{1.563514in}{2.733566in}}%
\pgfpathcurveto{\pgfqpoint{1.563514in}{2.741802in}}{\pgfqpoint{1.560242in}{2.749702in}}{\pgfqpoint{1.554418in}{2.755526in}}%
\pgfpathcurveto{\pgfqpoint{1.548594in}{2.761350in}}{\pgfqpoint{1.540694in}{2.764622in}}{\pgfqpoint{1.532457in}{2.764622in}}%
\pgfpathcurveto{\pgfqpoint{1.524221in}{2.764622in}}{\pgfqpoint{1.516321in}{2.761350in}}{\pgfqpoint{1.510497in}{2.755526in}}%
\pgfpathcurveto{\pgfqpoint{1.504673in}{2.749702in}}{\pgfqpoint{1.501401in}{2.741802in}}{\pgfqpoint{1.501401in}{2.733566in}}%
\pgfpathcurveto{\pgfqpoint{1.501401in}{2.725330in}}{\pgfqpoint{1.504673in}{2.717430in}}{\pgfqpoint{1.510497in}{2.711606in}}%
\pgfpathcurveto{\pgfqpoint{1.516321in}{2.705782in}}{\pgfqpoint{1.524221in}{2.702509in}}{\pgfqpoint{1.532457in}{2.702509in}}%
\pgfpathclose%
\pgfusepath{stroke,fill}%
\end{pgfscope}%
\begin{pgfscope}%
\pgfpathrectangle{\pgfqpoint{0.100000in}{0.220728in}}{\pgfqpoint{3.696000in}{3.696000in}}%
\pgfusepath{clip}%
\pgfsetbuttcap%
\pgfsetroundjoin%
\definecolor{currentfill}{rgb}{0.121569,0.466667,0.705882}%
\pgfsetfillcolor{currentfill}%
\pgfsetfillopacity{0.390363}%
\pgfsetlinewidth{1.003750pt}%
\definecolor{currentstroke}{rgb}{0.121569,0.466667,0.705882}%
\pgfsetstrokecolor{currentstroke}%
\pgfsetstrokeopacity{0.390363}%
\pgfsetdash{}{0pt}%
\pgfpathmoveto{\pgfqpoint{2.149088in}{3.217146in}}%
\pgfpathcurveto{\pgfqpoint{2.157325in}{3.217146in}}{\pgfqpoint{2.165225in}{3.220418in}}{\pgfqpoint{2.171049in}{3.226242in}}%
\pgfpathcurveto{\pgfqpoint{2.176872in}{3.232066in}}{\pgfqpoint{2.180145in}{3.239966in}}{\pgfqpoint{2.180145in}{3.248202in}}%
\pgfpathcurveto{\pgfqpoint{2.180145in}{3.256438in}}{\pgfqpoint{2.176872in}{3.264339in}}{\pgfqpoint{2.171049in}{3.270162in}}%
\pgfpathcurveto{\pgfqpoint{2.165225in}{3.275986in}}{\pgfqpoint{2.157325in}{3.279259in}}{\pgfqpoint{2.149088in}{3.279259in}}%
\pgfpathcurveto{\pgfqpoint{2.140852in}{3.279259in}}{\pgfqpoint{2.132952in}{3.275986in}}{\pgfqpoint{2.127128in}{3.270162in}}%
\pgfpathcurveto{\pgfqpoint{2.121304in}{3.264339in}}{\pgfqpoint{2.118032in}{3.256438in}}{\pgfqpoint{2.118032in}{3.248202in}}%
\pgfpathcurveto{\pgfqpoint{2.118032in}{3.239966in}}{\pgfqpoint{2.121304in}{3.232066in}}{\pgfqpoint{2.127128in}{3.226242in}}%
\pgfpathcurveto{\pgfqpoint{2.132952in}{3.220418in}}{\pgfqpoint{2.140852in}{3.217146in}}{\pgfqpoint{2.149088in}{3.217146in}}%
\pgfpathclose%
\pgfusepath{stroke,fill}%
\end{pgfscope}%
\begin{pgfscope}%
\pgfpathrectangle{\pgfqpoint{0.100000in}{0.220728in}}{\pgfqpoint{3.696000in}{3.696000in}}%
\pgfusepath{clip}%
\pgfsetbuttcap%
\pgfsetroundjoin%
\definecolor{currentfill}{rgb}{0.121569,0.466667,0.705882}%
\pgfsetfillcolor{currentfill}%
\pgfsetfillopacity{0.393286}%
\pgfsetlinewidth{1.003750pt}%
\definecolor{currentstroke}{rgb}{0.121569,0.466667,0.705882}%
\pgfsetstrokecolor{currentstroke}%
\pgfsetstrokeopacity{0.393286}%
\pgfsetdash{}{0pt}%
\pgfpathmoveto{\pgfqpoint{1.525996in}{2.678306in}}%
\pgfpathcurveto{\pgfqpoint{1.534232in}{2.678306in}}{\pgfqpoint{1.542132in}{2.681579in}}{\pgfqpoint{1.547956in}{2.687403in}}%
\pgfpathcurveto{\pgfqpoint{1.553780in}{2.693227in}}{\pgfqpoint{1.557053in}{2.701127in}}{\pgfqpoint{1.557053in}{2.709363in}}%
\pgfpathcurveto{\pgfqpoint{1.557053in}{2.717599in}}{\pgfqpoint{1.553780in}{2.725499in}}{\pgfqpoint{1.547956in}{2.731323in}}%
\pgfpathcurveto{\pgfqpoint{1.542132in}{2.737147in}}{\pgfqpoint{1.534232in}{2.740419in}}{\pgfqpoint{1.525996in}{2.740419in}}%
\pgfpathcurveto{\pgfqpoint{1.517760in}{2.740419in}}{\pgfqpoint{1.509860in}{2.737147in}}{\pgfqpoint{1.504036in}{2.731323in}}%
\pgfpathcurveto{\pgfqpoint{1.498212in}{2.725499in}}{\pgfqpoint{1.494940in}{2.717599in}}{\pgfqpoint{1.494940in}{2.709363in}}%
\pgfpathcurveto{\pgfqpoint{1.494940in}{2.701127in}}{\pgfqpoint{1.498212in}{2.693227in}}{\pgfqpoint{1.504036in}{2.687403in}}%
\pgfpathcurveto{\pgfqpoint{1.509860in}{2.681579in}}{\pgfqpoint{1.517760in}{2.678306in}}{\pgfqpoint{1.525996in}{2.678306in}}%
\pgfpathclose%
\pgfusepath{stroke,fill}%
\end{pgfscope}%
\begin{pgfscope}%
\pgfpathrectangle{\pgfqpoint{0.100000in}{0.220728in}}{\pgfqpoint{3.696000in}{3.696000in}}%
\pgfusepath{clip}%
\pgfsetbuttcap%
\pgfsetroundjoin%
\definecolor{currentfill}{rgb}{0.121569,0.466667,0.705882}%
\pgfsetfillcolor{currentfill}%
\pgfsetfillopacity{0.393672}%
\pgfsetlinewidth{1.003750pt}%
\definecolor{currentstroke}{rgb}{0.121569,0.466667,0.705882}%
\pgfsetstrokecolor{currentstroke}%
\pgfsetstrokeopacity{0.393672}%
\pgfsetdash{}{0pt}%
\pgfpathmoveto{\pgfqpoint{2.153295in}{3.213958in}}%
\pgfpathcurveto{\pgfqpoint{2.161531in}{3.213958in}}{\pgfqpoint{2.169431in}{3.217231in}}{\pgfqpoint{2.175255in}{3.223055in}}%
\pgfpathcurveto{\pgfqpoint{2.181079in}{3.228878in}}{\pgfqpoint{2.184351in}{3.236779in}}{\pgfqpoint{2.184351in}{3.245015in}}%
\pgfpathcurveto{\pgfqpoint{2.184351in}{3.253251in}}{\pgfqpoint{2.181079in}{3.261151in}}{\pgfqpoint{2.175255in}{3.266975in}}%
\pgfpathcurveto{\pgfqpoint{2.169431in}{3.272799in}}{\pgfqpoint{2.161531in}{3.276071in}}{\pgfqpoint{2.153295in}{3.276071in}}%
\pgfpathcurveto{\pgfqpoint{2.145059in}{3.276071in}}{\pgfqpoint{2.137159in}{3.272799in}}{\pgfqpoint{2.131335in}{3.266975in}}%
\pgfpathcurveto{\pgfqpoint{2.125511in}{3.261151in}}{\pgfqpoint{2.122238in}{3.253251in}}{\pgfqpoint{2.122238in}{3.245015in}}%
\pgfpathcurveto{\pgfqpoint{2.122238in}{3.236779in}}{\pgfqpoint{2.125511in}{3.228878in}}{\pgfqpoint{2.131335in}{3.223055in}}%
\pgfpathcurveto{\pgfqpoint{2.137159in}{3.217231in}}{\pgfqpoint{2.145059in}{3.213958in}}{\pgfqpoint{2.153295in}{3.213958in}}%
\pgfpathclose%
\pgfusepath{stroke,fill}%
\end{pgfscope}%
\begin{pgfscope}%
\pgfpathrectangle{\pgfqpoint{0.100000in}{0.220728in}}{\pgfqpoint{3.696000in}{3.696000in}}%
\pgfusepath{clip}%
\pgfsetbuttcap%
\pgfsetroundjoin%
\definecolor{currentfill}{rgb}{0.121569,0.466667,0.705882}%
\pgfsetfillcolor{currentfill}%
\pgfsetfillopacity{0.396063}%
\pgfsetlinewidth{1.003750pt}%
\definecolor{currentstroke}{rgb}{0.121569,0.466667,0.705882}%
\pgfsetstrokecolor{currentstroke}%
\pgfsetstrokeopacity{0.396063}%
\pgfsetdash{}{0pt}%
\pgfpathmoveto{\pgfqpoint{1.514494in}{2.659003in}}%
\pgfpathcurveto{\pgfqpoint{1.522731in}{2.659003in}}{\pgfqpoint{1.530631in}{2.662275in}}{\pgfqpoint{1.536455in}{2.668099in}}%
\pgfpathcurveto{\pgfqpoint{1.542279in}{2.673923in}}{\pgfqpoint{1.545551in}{2.681823in}}{\pgfqpoint{1.545551in}{2.690059in}}%
\pgfpathcurveto{\pgfqpoint{1.545551in}{2.698296in}}{\pgfqpoint{1.542279in}{2.706196in}}{\pgfqpoint{1.536455in}{2.712020in}}%
\pgfpathcurveto{\pgfqpoint{1.530631in}{2.717844in}}{\pgfqpoint{1.522731in}{2.721116in}}{\pgfqpoint{1.514494in}{2.721116in}}%
\pgfpathcurveto{\pgfqpoint{1.506258in}{2.721116in}}{\pgfqpoint{1.498358in}{2.717844in}}{\pgfqpoint{1.492534in}{2.712020in}}%
\pgfpathcurveto{\pgfqpoint{1.486710in}{2.706196in}}{\pgfqpoint{1.483438in}{2.698296in}}{\pgfqpoint{1.483438in}{2.690059in}}%
\pgfpathcurveto{\pgfqpoint{1.483438in}{2.681823in}}{\pgfqpoint{1.486710in}{2.673923in}}{\pgfqpoint{1.492534in}{2.668099in}}%
\pgfpathcurveto{\pgfqpoint{1.498358in}{2.662275in}}{\pgfqpoint{1.506258in}{2.659003in}}{\pgfqpoint{1.514494in}{2.659003in}}%
\pgfpathclose%
\pgfusepath{stroke,fill}%
\end{pgfscope}%
\begin{pgfscope}%
\pgfpathrectangle{\pgfqpoint{0.100000in}{0.220728in}}{\pgfqpoint{3.696000in}{3.696000in}}%
\pgfusepath{clip}%
\pgfsetbuttcap%
\pgfsetroundjoin%
\definecolor{currentfill}{rgb}{0.121569,0.466667,0.705882}%
\pgfsetfillcolor{currentfill}%
\pgfsetfillopacity{0.396284}%
\pgfsetlinewidth{1.003750pt}%
\definecolor{currentstroke}{rgb}{0.121569,0.466667,0.705882}%
\pgfsetstrokecolor{currentstroke}%
\pgfsetstrokeopacity{0.396284}%
\pgfsetdash{}{0pt}%
\pgfpathmoveto{\pgfqpoint{2.161269in}{3.212413in}}%
\pgfpathcurveto{\pgfqpoint{2.169506in}{3.212413in}}{\pgfqpoint{2.177406in}{3.215685in}}{\pgfqpoint{2.183230in}{3.221509in}}%
\pgfpathcurveto{\pgfqpoint{2.189054in}{3.227333in}}{\pgfqpoint{2.192326in}{3.235233in}}{\pgfqpoint{2.192326in}{3.243469in}}%
\pgfpathcurveto{\pgfqpoint{2.192326in}{3.251706in}}{\pgfqpoint{2.189054in}{3.259606in}}{\pgfqpoint{2.183230in}{3.265429in}}%
\pgfpathcurveto{\pgfqpoint{2.177406in}{3.271253in}}{\pgfqpoint{2.169506in}{3.274526in}}{\pgfqpoint{2.161269in}{3.274526in}}%
\pgfpathcurveto{\pgfqpoint{2.153033in}{3.274526in}}{\pgfqpoint{2.145133in}{3.271253in}}{\pgfqpoint{2.139309in}{3.265429in}}%
\pgfpathcurveto{\pgfqpoint{2.133485in}{3.259606in}}{\pgfqpoint{2.130213in}{3.251706in}}{\pgfqpoint{2.130213in}{3.243469in}}%
\pgfpathcurveto{\pgfqpoint{2.130213in}{3.235233in}}{\pgfqpoint{2.133485in}{3.227333in}}{\pgfqpoint{2.139309in}{3.221509in}}%
\pgfpathcurveto{\pgfqpoint{2.145133in}{3.215685in}}{\pgfqpoint{2.153033in}{3.212413in}}{\pgfqpoint{2.161269in}{3.212413in}}%
\pgfpathclose%
\pgfusepath{stroke,fill}%
\end{pgfscope}%
\begin{pgfscope}%
\pgfpathrectangle{\pgfqpoint{0.100000in}{0.220728in}}{\pgfqpoint{3.696000in}{3.696000in}}%
\pgfusepath{clip}%
\pgfsetbuttcap%
\pgfsetroundjoin%
\definecolor{currentfill}{rgb}{0.121569,0.466667,0.705882}%
\pgfsetfillcolor{currentfill}%
\pgfsetfillopacity{0.399041}%
\pgfsetlinewidth{1.003750pt}%
\definecolor{currentstroke}{rgb}{0.121569,0.466667,0.705882}%
\pgfsetstrokecolor{currentstroke}%
\pgfsetstrokeopacity{0.399041}%
\pgfsetdash{}{0pt}%
\pgfpathmoveto{\pgfqpoint{1.507883in}{2.637934in}}%
\pgfpathcurveto{\pgfqpoint{1.516120in}{2.637934in}}{\pgfqpoint{1.524020in}{2.641207in}}{\pgfqpoint{1.529844in}{2.647031in}}%
\pgfpathcurveto{\pgfqpoint{1.535668in}{2.652855in}}{\pgfqpoint{1.538940in}{2.660755in}}{\pgfqpoint{1.538940in}{2.668991in}}%
\pgfpathcurveto{\pgfqpoint{1.538940in}{2.677227in}}{\pgfqpoint{1.535668in}{2.685127in}}{\pgfqpoint{1.529844in}{2.690951in}}%
\pgfpathcurveto{\pgfqpoint{1.524020in}{2.696775in}}{\pgfqpoint{1.516120in}{2.700047in}}{\pgfqpoint{1.507883in}{2.700047in}}%
\pgfpathcurveto{\pgfqpoint{1.499647in}{2.700047in}}{\pgfqpoint{1.491747in}{2.696775in}}{\pgfqpoint{1.485923in}{2.690951in}}%
\pgfpathcurveto{\pgfqpoint{1.480099in}{2.685127in}}{\pgfqpoint{1.476827in}{2.677227in}}{\pgfqpoint{1.476827in}{2.668991in}}%
\pgfpathcurveto{\pgfqpoint{1.476827in}{2.660755in}}{\pgfqpoint{1.480099in}{2.652855in}}{\pgfqpoint{1.485923in}{2.647031in}}%
\pgfpathcurveto{\pgfqpoint{1.491747in}{2.641207in}}{\pgfqpoint{1.499647in}{2.637934in}}{\pgfqpoint{1.507883in}{2.637934in}}%
\pgfpathclose%
\pgfusepath{stroke,fill}%
\end{pgfscope}%
\begin{pgfscope}%
\pgfpathrectangle{\pgfqpoint{0.100000in}{0.220728in}}{\pgfqpoint{3.696000in}{3.696000in}}%
\pgfusepath{clip}%
\pgfsetbuttcap%
\pgfsetroundjoin%
\definecolor{currentfill}{rgb}{0.121569,0.466667,0.705882}%
\pgfsetfillcolor{currentfill}%
\pgfsetfillopacity{0.399127}%
\pgfsetlinewidth{1.003750pt}%
\definecolor{currentstroke}{rgb}{0.121569,0.466667,0.705882}%
\pgfsetstrokecolor{currentstroke}%
\pgfsetstrokeopacity{0.399127}%
\pgfsetdash{}{0pt}%
\pgfpathmoveto{\pgfqpoint{2.169989in}{3.211297in}}%
\pgfpathcurveto{\pgfqpoint{2.178225in}{3.211297in}}{\pgfqpoint{2.186125in}{3.214569in}}{\pgfqpoint{2.191949in}{3.220393in}}%
\pgfpathcurveto{\pgfqpoint{2.197773in}{3.226217in}}{\pgfqpoint{2.201045in}{3.234117in}}{\pgfqpoint{2.201045in}{3.242353in}}%
\pgfpathcurveto{\pgfqpoint{2.201045in}{3.250590in}}{\pgfqpoint{2.197773in}{3.258490in}}{\pgfqpoint{2.191949in}{3.264314in}}%
\pgfpathcurveto{\pgfqpoint{2.186125in}{3.270138in}}{\pgfqpoint{2.178225in}{3.273410in}}{\pgfqpoint{2.169989in}{3.273410in}}%
\pgfpathcurveto{\pgfqpoint{2.161753in}{3.273410in}}{\pgfqpoint{2.153853in}{3.270138in}}{\pgfqpoint{2.148029in}{3.264314in}}%
\pgfpathcurveto{\pgfqpoint{2.142205in}{3.258490in}}{\pgfqpoint{2.138932in}{3.250590in}}{\pgfqpoint{2.138932in}{3.242353in}}%
\pgfpathcurveto{\pgfqpoint{2.138932in}{3.234117in}}{\pgfqpoint{2.142205in}{3.226217in}}{\pgfqpoint{2.148029in}{3.220393in}}%
\pgfpathcurveto{\pgfqpoint{2.153853in}{3.214569in}}{\pgfqpoint{2.161753in}{3.211297in}}{\pgfqpoint{2.169989in}{3.211297in}}%
\pgfpathclose%
\pgfusepath{stroke,fill}%
\end{pgfscope}%
\begin{pgfscope}%
\pgfpathrectangle{\pgfqpoint{0.100000in}{0.220728in}}{\pgfqpoint{3.696000in}{3.696000in}}%
\pgfusepath{clip}%
\pgfsetbuttcap%
\pgfsetroundjoin%
\definecolor{currentfill}{rgb}{0.121569,0.466667,0.705882}%
\pgfsetfillcolor{currentfill}%
\pgfsetfillopacity{0.401200}%
\pgfsetlinewidth{1.003750pt}%
\definecolor{currentstroke}{rgb}{0.121569,0.466667,0.705882}%
\pgfsetstrokecolor{currentstroke}%
\pgfsetstrokeopacity{0.401200}%
\pgfsetdash{}{0pt}%
\pgfpathmoveto{\pgfqpoint{1.496059in}{2.621300in}}%
\pgfpathcurveto{\pgfqpoint{1.504295in}{2.621300in}}{\pgfqpoint{1.512195in}{2.624572in}}{\pgfqpoint{1.518019in}{2.630396in}}%
\pgfpathcurveto{\pgfqpoint{1.523843in}{2.636220in}}{\pgfqpoint{1.527115in}{2.644120in}}{\pgfqpoint{1.527115in}{2.652356in}}%
\pgfpathcurveto{\pgfqpoint{1.527115in}{2.660593in}}{\pgfqpoint{1.523843in}{2.668493in}}{\pgfqpoint{1.518019in}{2.674317in}}%
\pgfpathcurveto{\pgfqpoint{1.512195in}{2.680141in}}{\pgfqpoint{1.504295in}{2.683413in}}{\pgfqpoint{1.496059in}{2.683413in}}%
\pgfpathcurveto{\pgfqpoint{1.487823in}{2.683413in}}{\pgfqpoint{1.479922in}{2.680141in}}{\pgfqpoint{1.474099in}{2.674317in}}%
\pgfpathcurveto{\pgfqpoint{1.468275in}{2.668493in}}{\pgfqpoint{1.465002in}{2.660593in}}{\pgfqpoint{1.465002in}{2.652356in}}%
\pgfpathcurveto{\pgfqpoint{1.465002in}{2.644120in}}{\pgfqpoint{1.468275in}{2.636220in}}{\pgfqpoint{1.474099in}{2.630396in}}%
\pgfpathcurveto{\pgfqpoint{1.479922in}{2.624572in}}{\pgfqpoint{1.487823in}{2.621300in}}{\pgfqpoint{1.496059in}{2.621300in}}%
\pgfpathclose%
\pgfusepath{stroke,fill}%
\end{pgfscope}%
\begin{pgfscope}%
\pgfpathrectangle{\pgfqpoint{0.100000in}{0.220728in}}{\pgfqpoint{3.696000in}{3.696000in}}%
\pgfusepath{clip}%
\pgfsetbuttcap%
\pgfsetroundjoin%
\definecolor{currentfill}{rgb}{0.121569,0.466667,0.705882}%
\pgfsetfillcolor{currentfill}%
\pgfsetfillopacity{0.401278}%
\pgfsetlinewidth{1.003750pt}%
\definecolor{currentstroke}{rgb}{0.121569,0.466667,0.705882}%
\pgfsetstrokecolor{currentstroke}%
\pgfsetstrokeopacity{0.401278}%
\pgfsetdash{}{0pt}%
\pgfpathmoveto{\pgfqpoint{2.182484in}{3.208757in}}%
\pgfpathcurveto{\pgfqpoint{2.190720in}{3.208757in}}{\pgfqpoint{2.198620in}{3.212029in}}{\pgfqpoint{2.204444in}{3.217853in}}%
\pgfpathcurveto{\pgfqpoint{2.210268in}{3.223677in}}{\pgfqpoint{2.213541in}{3.231577in}}{\pgfqpoint{2.213541in}{3.239813in}}%
\pgfpathcurveto{\pgfqpoint{2.213541in}{3.248049in}}{\pgfqpoint{2.210268in}{3.255949in}}{\pgfqpoint{2.204444in}{3.261773in}}%
\pgfpathcurveto{\pgfqpoint{2.198620in}{3.267597in}}{\pgfqpoint{2.190720in}{3.270870in}}{\pgfqpoint{2.182484in}{3.270870in}}%
\pgfpathcurveto{\pgfqpoint{2.174248in}{3.270870in}}{\pgfqpoint{2.166348in}{3.267597in}}{\pgfqpoint{2.160524in}{3.261773in}}%
\pgfpathcurveto{\pgfqpoint{2.154700in}{3.255949in}}{\pgfqpoint{2.151428in}{3.248049in}}{\pgfqpoint{2.151428in}{3.239813in}}%
\pgfpathcurveto{\pgfqpoint{2.151428in}{3.231577in}}{\pgfqpoint{2.154700in}{3.223677in}}{\pgfqpoint{2.160524in}{3.217853in}}%
\pgfpathcurveto{\pgfqpoint{2.166348in}{3.212029in}}{\pgfqpoint{2.174248in}{3.208757in}}{\pgfqpoint{2.182484in}{3.208757in}}%
\pgfpathclose%
\pgfusepath{stroke,fill}%
\end{pgfscope}%
\begin{pgfscope}%
\pgfpathrectangle{\pgfqpoint{0.100000in}{0.220728in}}{\pgfqpoint{3.696000in}{3.696000in}}%
\pgfusepath{clip}%
\pgfsetbuttcap%
\pgfsetroundjoin%
\definecolor{currentfill}{rgb}{0.121569,0.466667,0.705882}%
\pgfsetfillcolor{currentfill}%
\pgfsetfillopacity{0.401941}%
\pgfsetlinewidth{1.003750pt}%
\definecolor{currentstroke}{rgb}{0.121569,0.466667,0.705882}%
\pgfsetstrokecolor{currentstroke}%
\pgfsetstrokeopacity{0.401941}%
\pgfsetdash{}{0pt}%
\pgfpathmoveto{\pgfqpoint{2.189304in}{3.204913in}}%
\pgfpathcurveto{\pgfqpoint{2.197540in}{3.204913in}}{\pgfqpoint{2.205440in}{3.208185in}}{\pgfqpoint{2.211264in}{3.214009in}}%
\pgfpathcurveto{\pgfqpoint{2.217088in}{3.219833in}}{\pgfqpoint{2.220360in}{3.227733in}}{\pgfqpoint{2.220360in}{3.235969in}}%
\pgfpathcurveto{\pgfqpoint{2.220360in}{3.244205in}}{\pgfqpoint{2.217088in}{3.252105in}}{\pgfqpoint{2.211264in}{3.257929in}}%
\pgfpathcurveto{\pgfqpoint{2.205440in}{3.263753in}}{\pgfqpoint{2.197540in}{3.267026in}}{\pgfqpoint{2.189304in}{3.267026in}}%
\pgfpathcurveto{\pgfqpoint{2.181067in}{3.267026in}}{\pgfqpoint{2.173167in}{3.263753in}}{\pgfqpoint{2.167343in}{3.257929in}}%
\pgfpathcurveto{\pgfqpoint{2.161520in}{3.252105in}}{\pgfqpoint{2.158247in}{3.244205in}}{\pgfqpoint{2.158247in}{3.235969in}}%
\pgfpathcurveto{\pgfqpoint{2.158247in}{3.227733in}}{\pgfqpoint{2.161520in}{3.219833in}}{\pgfqpoint{2.167343in}{3.214009in}}%
\pgfpathcurveto{\pgfqpoint{2.173167in}{3.208185in}}{\pgfqpoint{2.181067in}{3.204913in}}{\pgfqpoint{2.189304in}{3.204913in}}%
\pgfpathclose%
\pgfusepath{stroke,fill}%
\end{pgfscope}%
\begin{pgfscope}%
\pgfpathrectangle{\pgfqpoint{0.100000in}{0.220728in}}{\pgfqpoint{3.696000in}{3.696000in}}%
\pgfusepath{clip}%
\pgfsetbuttcap%
\pgfsetroundjoin%
\definecolor{currentfill}{rgb}{0.121569,0.466667,0.705882}%
\pgfsetfillcolor{currentfill}%
\pgfsetfillopacity{0.403248}%
\pgfsetlinewidth{1.003750pt}%
\definecolor{currentstroke}{rgb}{0.121569,0.466667,0.705882}%
\pgfsetstrokecolor{currentstroke}%
\pgfsetstrokeopacity{0.403248}%
\pgfsetdash{}{0pt}%
\pgfpathmoveto{\pgfqpoint{2.197114in}{3.202661in}}%
\pgfpathcurveto{\pgfqpoint{2.205351in}{3.202661in}}{\pgfqpoint{2.213251in}{3.205933in}}{\pgfqpoint{2.219075in}{3.211757in}}%
\pgfpathcurveto{\pgfqpoint{2.224898in}{3.217581in}}{\pgfqpoint{2.228171in}{3.225481in}}{\pgfqpoint{2.228171in}{3.233717in}}%
\pgfpathcurveto{\pgfqpoint{2.228171in}{3.241953in}}{\pgfqpoint{2.224898in}{3.249853in}}{\pgfqpoint{2.219075in}{3.255677in}}%
\pgfpathcurveto{\pgfqpoint{2.213251in}{3.261501in}}{\pgfqpoint{2.205351in}{3.264774in}}{\pgfqpoint{2.197114in}{3.264774in}}%
\pgfpathcurveto{\pgfqpoint{2.188878in}{3.264774in}}{\pgfqpoint{2.180978in}{3.261501in}}{\pgfqpoint{2.175154in}{3.255677in}}%
\pgfpathcurveto{\pgfqpoint{2.169330in}{3.249853in}}{\pgfqpoint{2.166058in}{3.241953in}}{\pgfqpoint{2.166058in}{3.233717in}}%
\pgfpathcurveto{\pgfqpoint{2.166058in}{3.225481in}}{\pgfqpoint{2.169330in}{3.217581in}}{\pgfqpoint{2.175154in}{3.211757in}}%
\pgfpathcurveto{\pgfqpoint{2.180978in}{3.205933in}}{\pgfqpoint{2.188878in}{3.202661in}}{\pgfqpoint{2.197114in}{3.202661in}}%
\pgfpathclose%
\pgfusepath{stroke,fill}%
\end{pgfscope}%
\begin{pgfscope}%
\pgfpathrectangle{\pgfqpoint{0.100000in}{0.220728in}}{\pgfqpoint{3.696000in}{3.696000in}}%
\pgfusepath{clip}%
\pgfsetbuttcap%
\pgfsetroundjoin%
\definecolor{currentfill}{rgb}{0.121569,0.466667,0.705882}%
\pgfsetfillcolor{currentfill}%
\pgfsetfillopacity{0.403885}%
\pgfsetlinewidth{1.003750pt}%
\definecolor{currentstroke}{rgb}{0.121569,0.466667,0.705882}%
\pgfsetstrokecolor{currentstroke}%
\pgfsetstrokeopacity{0.403885}%
\pgfsetdash{}{0pt}%
\pgfpathmoveto{\pgfqpoint{1.493641in}{2.602513in}}%
\pgfpathcurveto{\pgfqpoint{1.501877in}{2.602513in}}{\pgfqpoint{1.509777in}{2.605785in}}{\pgfqpoint{1.515601in}{2.611609in}}%
\pgfpathcurveto{\pgfqpoint{1.521425in}{2.617433in}}{\pgfqpoint{1.524698in}{2.625333in}}{\pgfqpoint{1.524698in}{2.633569in}}%
\pgfpathcurveto{\pgfqpoint{1.524698in}{2.641805in}}{\pgfqpoint{1.521425in}{2.649705in}}{\pgfqpoint{1.515601in}{2.655529in}}%
\pgfpathcurveto{\pgfqpoint{1.509777in}{2.661353in}}{\pgfqpoint{1.501877in}{2.664626in}}{\pgfqpoint{1.493641in}{2.664626in}}%
\pgfpathcurveto{\pgfqpoint{1.485405in}{2.664626in}}{\pgfqpoint{1.477505in}{2.661353in}}{\pgfqpoint{1.471681in}{2.655529in}}%
\pgfpathcurveto{\pgfqpoint{1.465857in}{2.649705in}}{\pgfqpoint{1.462585in}{2.641805in}}{\pgfqpoint{1.462585in}{2.633569in}}%
\pgfpathcurveto{\pgfqpoint{1.462585in}{2.625333in}}{\pgfqpoint{1.465857in}{2.617433in}}{\pgfqpoint{1.471681in}{2.611609in}}%
\pgfpathcurveto{\pgfqpoint{1.477505in}{2.605785in}}{\pgfqpoint{1.485405in}{2.602513in}}{\pgfqpoint{1.493641in}{2.602513in}}%
\pgfpathclose%
\pgfusepath{stroke,fill}%
\end{pgfscope}%
\begin{pgfscope}%
\pgfpathrectangle{\pgfqpoint{0.100000in}{0.220728in}}{\pgfqpoint{3.696000in}{3.696000in}}%
\pgfusepath{clip}%
\pgfsetbuttcap%
\pgfsetroundjoin%
\definecolor{currentfill}{rgb}{0.121569,0.466667,0.705882}%
\pgfsetfillcolor{currentfill}%
\pgfsetfillopacity{0.404958}%
\pgfsetlinewidth{1.003750pt}%
\definecolor{currentstroke}{rgb}{0.121569,0.466667,0.705882}%
\pgfsetstrokecolor{currentstroke}%
\pgfsetstrokeopacity{0.404958}%
\pgfsetdash{}{0pt}%
\pgfpathmoveto{\pgfqpoint{1.483139in}{2.590327in}}%
\pgfpathcurveto{\pgfqpoint{1.491376in}{2.590327in}}{\pgfqpoint{1.499276in}{2.593600in}}{\pgfqpoint{1.505100in}{2.599424in}}%
\pgfpathcurveto{\pgfqpoint{1.510923in}{2.605248in}}{\pgfqpoint{1.514196in}{2.613148in}}{\pgfqpoint{1.514196in}{2.621384in}}%
\pgfpathcurveto{\pgfqpoint{1.514196in}{2.629620in}}{\pgfqpoint{1.510923in}{2.637520in}}{\pgfqpoint{1.505100in}{2.643344in}}%
\pgfpathcurveto{\pgfqpoint{1.499276in}{2.649168in}}{\pgfqpoint{1.491376in}{2.652440in}}{\pgfqpoint{1.483139in}{2.652440in}}%
\pgfpathcurveto{\pgfqpoint{1.474903in}{2.652440in}}{\pgfqpoint{1.467003in}{2.649168in}}{\pgfqpoint{1.461179in}{2.643344in}}%
\pgfpathcurveto{\pgfqpoint{1.455355in}{2.637520in}}{\pgfqpoint{1.452083in}{2.629620in}}{\pgfqpoint{1.452083in}{2.621384in}}%
\pgfpathcurveto{\pgfqpoint{1.452083in}{2.613148in}}{\pgfqpoint{1.455355in}{2.605248in}}{\pgfqpoint{1.461179in}{2.599424in}}%
\pgfpathcurveto{\pgfqpoint{1.467003in}{2.593600in}}{\pgfqpoint{1.474903in}{2.590327in}}{\pgfqpoint{1.483139in}{2.590327in}}%
\pgfpathclose%
\pgfusepath{stroke,fill}%
\end{pgfscope}%
\begin{pgfscope}%
\pgfpathrectangle{\pgfqpoint{0.100000in}{0.220728in}}{\pgfqpoint{3.696000in}{3.696000in}}%
\pgfusepath{clip}%
\pgfsetbuttcap%
\pgfsetroundjoin%
\definecolor{currentfill}{rgb}{0.121569,0.466667,0.705882}%
\pgfsetfillcolor{currentfill}%
\pgfsetfillopacity{0.405095}%
\pgfsetlinewidth{1.003750pt}%
\definecolor{currentstroke}{rgb}{0.121569,0.466667,0.705882}%
\pgfsetstrokecolor{currentstroke}%
\pgfsetstrokeopacity{0.405095}%
\pgfsetdash{}{0pt}%
\pgfpathmoveto{\pgfqpoint{2.206427in}{3.200938in}}%
\pgfpathcurveto{\pgfqpoint{2.214663in}{3.200938in}}{\pgfqpoint{2.222563in}{3.204210in}}{\pgfqpoint{2.228387in}{3.210034in}}%
\pgfpathcurveto{\pgfqpoint{2.234211in}{3.215858in}}{\pgfqpoint{2.237483in}{3.223758in}}{\pgfqpoint{2.237483in}{3.231994in}}%
\pgfpathcurveto{\pgfqpoint{2.237483in}{3.240230in}}{\pgfqpoint{2.234211in}{3.248130in}}{\pgfqpoint{2.228387in}{3.253954in}}%
\pgfpathcurveto{\pgfqpoint{2.222563in}{3.259778in}}{\pgfqpoint{2.214663in}{3.263051in}}{\pgfqpoint{2.206427in}{3.263051in}}%
\pgfpathcurveto{\pgfqpoint{2.198190in}{3.263051in}}{\pgfqpoint{2.190290in}{3.259778in}}{\pgfqpoint{2.184466in}{3.253954in}}%
\pgfpathcurveto{\pgfqpoint{2.178642in}{3.248130in}}{\pgfqpoint{2.175370in}{3.240230in}}{\pgfqpoint{2.175370in}{3.231994in}}%
\pgfpathcurveto{\pgfqpoint{2.175370in}{3.223758in}}{\pgfqpoint{2.178642in}{3.215858in}}{\pgfqpoint{2.184466in}{3.210034in}}%
\pgfpathcurveto{\pgfqpoint{2.190290in}{3.204210in}}{\pgfqpoint{2.198190in}{3.200938in}}{\pgfqpoint{2.206427in}{3.200938in}}%
\pgfpathclose%
\pgfusepath{stroke,fill}%
\end{pgfscope}%
\begin{pgfscope}%
\pgfpathrectangle{\pgfqpoint{0.100000in}{0.220728in}}{\pgfqpoint{3.696000in}{3.696000in}}%
\pgfusepath{clip}%
\pgfsetbuttcap%
\pgfsetroundjoin%
\definecolor{currentfill}{rgb}{0.121569,0.466667,0.705882}%
\pgfsetfillcolor{currentfill}%
\pgfsetfillopacity{0.407029}%
\pgfsetlinewidth{1.003750pt}%
\definecolor{currentstroke}{rgb}{0.121569,0.466667,0.705882}%
\pgfsetstrokecolor{currentstroke}%
\pgfsetstrokeopacity{0.407029}%
\pgfsetdash{}{0pt}%
\pgfpathmoveto{\pgfqpoint{1.480991in}{2.577269in}}%
\pgfpathcurveto{\pgfqpoint{1.489228in}{2.577269in}}{\pgfqpoint{1.497128in}{2.580542in}}{\pgfqpoint{1.502952in}{2.586365in}}%
\pgfpathcurveto{\pgfqpoint{1.508776in}{2.592189in}}{\pgfqpoint{1.512048in}{2.600089in}}{\pgfqpoint{1.512048in}{2.608326in}}%
\pgfpathcurveto{\pgfqpoint{1.512048in}{2.616562in}}{\pgfqpoint{1.508776in}{2.624462in}}{\pgfqpoint{1.502952in}{2.630286in}}%
\pgfpathcurveto{\pgfqpoint{1.497128in}{2.636110in}}{\pgfqpoint{1.489228in}{2.639382in}}{\pgfqpoint{1.480991in}{2.639382in}}%
\pgfpathcurveto{\pgfqpoint{1.472755in}{2.639382in}}{\pgfqpoint{1.464855in}{2.636110in}}{\pgfqpoint{1.459031in}{2.630286in}}%
\pgfpathcurveto{\pgfqpoint{1.453207in}{2.624462in}}{\pgfqpoint{1.449935in}{2.616562in}}{\pgfqpoint{1.449935in}{2.608326in}}%
\pgfpathcurveto{\pgfqpoint{1.449935in}{2.600089in}}{\pgfqpoint{1.453207in}{2.592189in}}{\pgfqpoint{1.459031in}{2.586365in}}%
\pgfpathcurveto{\pgfqpoint{1.464855in}{2.580542in}}{\pgfqpoint{1.472755in}{2.577269in}}{\pgfqpoint{1.480991in}{2.577269in}}%
\pgfpathclose%
\pgfusepath{stroke,fill}%
\end{pgfscope}%
\begin{pgfscope}%
\pgfpathrectangle{\pgfqpoint{0.100000in}{0.220728in}}{\pgfqpoint{3.696000in}{3.696000in}}%
\pgfusepath{clip}%
\pgfsetbuttcap%
\pgfsetroundjoin%
\definecolor{currentfill}{rgb}{0.121569,0.466667,0.705882}%
\pgfsetfillcolor{currentfill}%
\pgfsetfillopacity{0.408262}%
\pgfsetlinewidth{1.003750pt}%
\definecolor{currentstroke}{rgb}{0.121569,0.466667,0.705882}%
\pgfsetstrokecolor{currentstroke}%
\pgfsetstrokeopacity{0.408262}%
\pgfsetdash{}{0pt}%
\pgfpathmoveto{\pgfqpoint{2.233668in}{3.192694in}}%
\pgfpathcurveto{\pgfqpoint{2.241904in}{3.192694in}}{\pgfqpoint{2.249804in}{3.195967in}}{\pgfqpoint{2.255628in}{3.201791in}}%
\pgfpathcurveto{\pgfqpoint{2.261452in}{3.207615in}}{\pgfqpoint{2.264724in}{3.215515in}}{\pgfqpoint{2.264724in}{3.223751in}}%
\pgfpathcurveto{\pgfqpoint{2.264724in}{3.231987in}}{\pgfqpoint{2.261452in}{3.239887in}}{\pgfqpoint{2.255628in}{3.245711in}}%
\pgfpathcurveto{\pgfqpoint{2.249804in}{3.251535in}}{\pgfqpoint{2.241904in}{3.254807in}}{\pgfqpoint{2.233668in}{3.254807in}}%
\pgfpathcurveto{\pgfqpoint{2.225431in}{3.254807in}}{\pgfqpoint{2.217531in}{3.251535in}}{\pgfqpoint{2.211707in}{3.245711in}}%
\pgfpathcurveto{\pgfqpoint{2.205883in}{3.239887in}}{\pgfqpoint{2.202611in}{3.231987in}}{\pgfqpoint{2.202611in}{3.223751in}}%
\pgfpathcurveto{\pgfqpoint{2.202611in}{3.215515in}}{\pgfqpoint{2.205883in}{3.207615in}}{\pgfqpoint{2.211707in}{3.201791in}}%
\pgfpathcurveto{\pgfqpoint{2.217531in}{3.195967in}}{\pgfqpoint{2.225431in}{3.192694in}}{\pgfqpoint{2.233668in}{3.192694in}}%
\pgfpathclose%
\pgfusepath{stroke,fill}%
\end{pgfscope}%
\begin{pgfscope}%
\pgfpathrectangle{\pgfqpoint{0.100000in}{0.220728in}}{\pgfqpoint{3.696000in}{3.696000in}}%
\pgfusepath{clip}%
\pgfsetbuttcap%
\pgfsetroundjoin%
\definecolor{currentfill}{rgb}{0.121569,0.466667,0.705882}%
\pgfsetfillcolor{currentfill}%
\pgfsetfillopacity{0.408909}%
\pgfsetlinewidth{1.003750pt}%
\definecolor{currentstroke}{rgb}{0.121569,0.466667,0.705882}%
\pgfsetstrokecolor{currentstroke}%
\pgfsetstrokeopacity{0.408909}%
\pgfsetdash{}{0pt}%
\pgfpathmoveto{\pgfqpoint{2.217584in}{3.199138in}}%
\pgfpathcurveto{\pgfqpoint{2.225820in}{3.199138in}}{\pgfqpoint{2.233720in}{3.202411in}}{\pgfqpoint{2.239544in}{3.208235in}}%
\pgfpathcurveto{\pgfqpoint{2.245368in}{3.214059in}}{\pgfqpoint{2.248641in}{3.221959in}}{\pgfqpoint{2.248641in}{3.230195in}}%
\pgfpathcurveto{\pgfqpoint{2.248641in}{3.238431in}}{\pgfqpoint{2.245368in}{3.246331in}}{\pgfqpoint{2.239544in}{3.252155in}}%
\pgfpathcurveto{\pgfqpoint{2.233720in}{3.257979in}}{\pgfqpoint{2.225820in}{3.261251in}}{\pgfqpoint{2.217584in}{3.261251in}}%
\pgfpathcurveto{\pgfqpoint{2.209348in}{3.261251in}}{\pgfqpoint{2.201448in}{3.257979in}}{\pgfqpoint{2.195624in}{3.252155in}}%
\pgfpathcurveto{\pgfqpoint{2.189800in}{3.246331in}}{\pgfqpoint{2.186528in}{3.238431in}}{\pgfqpoint{2.186528in}{3.230195in}}%
\pgfpathcurveto{\pgfqpoint{2.186528in}{3.221959in}}{\pgfqpoint{2.189800in}{3.214059in}}{\pgfqpoint{2.195624in}{3.208235in}}%
\pgfpathcurveto{\pgfqpoint{2.201448in}{3.202411in}}{\pgfqpoint{2.209348in}{3.199138in}}{\pgfqpoint{2.217584in}{3.199138in}}%
\pgfpathclose%
\pgfusepath{stroke,fill}%
\end{pgfscope}%
\begin{pgfscope}%
\pgfpathrectangle{\pgfqpoint{0.100000in}{0.220728in}}{\pgfqpoint{3.696000in}{3.696000in}}%
\pgfusepath{clip}%
\pgfsetbuttcap%
\pgfsetroundjoin%
\definecolor{currentfill}{rgb}{0.121569,0.466667,0.705882}%
\pgfsetfillcolor{currentfill}%
\pgfsetfillopacity{0.409276}%
\pgfsetlinewidth{1.003750pt}%
\definecolor{currentstroke}{rgb}{0.121569,0.466667,0.705882}%
\pgfsetstrokecolor{currentstroke}%
\pgfsetstrokeopacity{0.409276}%
\pgfsetdash{}{0pt}%
\pgfpathmoveto{\pgfqpoint{1.465754in}{2.559684in}}%
\pgfpathcurveto{\pgfqpoint{1.473990in}{2.559684in}}{\pgfqpoint{1.481890in}{2.562957in}}{\pgfqpoint{1.487714in}{2.568781in}}%
\pgfpathcurveto{\pgfqpoint{1.493538in}{2.574605in}}{\pgfqpoint{1.496810in}{2.582505in}}{\pgfqpoint{1.496810in}{2.590741in}}%
\pgfpathcurveto{\pgfqpoint{1.496810in}{2.598977in}}{\pgfqpoint{1.493538in}{2.606877in}}{\pgfqpoint{1.487714in}{2.612701in}}%
\pgfpathcurveto{\pgfqpoint{1.481890in}{2.618525in}}{\pgfqpoint{1.473990in}{2.621797in}}{\pgfqpoint{1.465754in}{2.621797in}}%
\pgfpathcurveto{\pgfqpoint{1.457518in}{2.621797in}}{\pgfqpoint{1.449617in}{2.618525in}}{\pgfqpoint{1.443794in}{2.612701in}}%
\pgfpathcurveto{\pgfqpoint{1.437970in}{2.606877in}}{\pgfqpoint{1.434697in}{2.598977in}}{\pgfqpoint{1.434697in}{2.590741in}}%
\pgfpathcurveto{\pgfqpoint{1.434697in}{2.582505in}}{\pgfqpoint{1.437970in}{2.574605in}}{\pgfqpoint{1.443794in}{2.568781in}}%
\pgfpathcurveto{\pgfqpoint{1.449617in}{2.562957in}}{\pgfqpoint{1.457518in}{2.559684in}}{\pgfqpoint{1.465754in}{2.559684in}}%
\pgfpathclose%
\pgfusepath{stroke,fill}%
\end{pgfscope}%
\begin{pgfscope}%
\pgfpathrectangle{\pgfqpoint{0.100000in}{0.220728in}}{\pgfqpoint{3.696000in}{3.696000in}}%
\pgfusepath{clip}%
\pgfsetbuttcap%
\pgfsetroundjoin%
\definecolor{currentfill}{rgb}{0.121569,0.466667,0.705882}%
\pgfsetfillcolor{currentfill}%
\pgfsetfillopacity{0.410538}%
\pgfsetlinewidth{1.003750pt}%
\definecolor{currentstroke}{rgb}{0.121569,0.466667,0.705882}%
\pgfsetstrokecolor{currentstroke}%
\pgfsetstrokeopacity{0.410538}%
\pgfsetdash{}{0pt}%
\pgfpathmoveto{\pgfqpoint{2.248931in}{3.187014in}}%
\pgfpathcurveto{\pgfqpoint{2.257167in}{3.187014in}}{\pgfqpoint{2.265067in}{3.190286in}}{\pgfqpoint{2.270891in}{3.196110in}}%
\pgfpathcurveto{\pgfqpoint{2.276715in}{3.201934in}}{\pgfqpoint{2.279988in}{3.209834in}}{\pgfqpoint{2.279988in}{3.218070in}}%
\pgfpathcurveto{\pgfqpoint{2.279988in}{3.226306in}}{\pgfqpoint{2.276715in}{3.234206in}}{\pgfqpoint{2.270891in}{3.240030in}}%
\pgfpathcurveto{\pgfqpoint{2.265067in}{3.245854in}}{\pgfqpoint{2.257167in}{3.249127in}}{\pgfqpoint{2.248931in}{3.249127in}}%
\pgfpathcurveto{\pgfqpoint{2.240695in}{3.249127in}}{\pgfqpoint{2.232795in}{3.245854in}}{\pgfqpoint{2.226971in}{3.240030in}}%
\pgfpathcurveto{\pgfqpoint{2.221147in}{3.234206in}}{\pgfqpoint{2.217875in}{3.226306in}}{\pgfqpoint{2.217875in}{3.218070in}}%
\pgfpathcurveto{\pgfqpoint{2.217875in}{3.209834in}}{\pgfqpoint{2.221147in}{3.201934in}}{\pgfqpoint{2.226971in}{3.196110in}}%
\pgfpathcurveto{\pgfqpoint{2.232795in}{3.190286in}}{\pgfqpoint{2.240695in}{3.187014in}}{\pgfqpoint{2.248931in}{3.187014in}}%
\pgfpathclose%
\pgfusepath{stroke,fill}%
\end{pgfscope}%
\begin{pgfscope}%
\pgfpathrectangle{\pgfqpoint{0.100000in}{0.220728in}}{\pgfqpoint{3.696000in}{3.696000in}}%
\pgfusepath{clip}%
\pgfsetbuttcap%
\pgfsetroundjoin%
\definecolor{currentfill}{rgb}{0.121569,0.466667,0.705882}%
\pgfsetfillcolor{currentfill}%
\pgfsetfillopacity{0.412794}%
\pgfsetlinewidth{1.003750pt}%
\definecolor{currentstroke}{rgb}{0.121569,0.466667,0.705882}%
\pgfsetstrokecolor{currentstroke}%
\pgfsetstrokeopacity{0.412794}%
\pgfsetdash{}{0pt}%
\pgfpathmoveto{\pgfqpoint{1.457774in}{2.539848in}}%
\pgfpathcurveto{\pgfqpoint{1.466010in}{2.539848in}}{\pgfqpoint{1.473910in}{2.543120in}}{\pgfqpoint{1.479734in}{2.548944in}}%
\pgfpathcurveto{\pgfqpoint{1.485558in}{2.554768in}}{\pgfqpoint{1.488831in}{2.562668in}}{\pgfqpoint{1.488831in}{2.570904in}}%
\pgfpathcurveto{\pgfqpoint{1.488831in}{2.579140in}}{\pgfqpoint{1.485558in}{2.587040in}}{\pgfqpoint{1.479734in}{2.592864in}}%
\pgfpathcurveto{\pgfqpoint{1.473910in}{2.598688in}}{\pgfqpoint{1.466010in}{2.601961in}}{\pgfqpoint{1.457774in}{2.601961in}}%
\pgfpathcurveto{\pgfqpoint{1.449538in}{2.601961in}}{\pgfqpoint{1.441638in}{2.598688in}}{\pgfqpoint{1.435814in}{2.592864in}}%
\pgfpathcurveto{\pgfqpoint{1.429990in}{2.587040in}}{\pgfqpoint{1.426718in}{2.579140in}}{\pgfqpoint{1.426718in}{2.570904in}}%
\pgfpathcurveto{\pgfqpoint{1.426718in}{2.562668in}}{\pgfqpoint{1.429990in}{2.554768in}}{\pgfqpoint{1.435814in}{2.548944in}}%
\pgfpathcurveto{\pgfqpoint{1.441638in}{2.543120in}}{\pgfqpoint{1.449538in}{2.539848in}}{\pgfqpoint{1.457774in}{2.539848in}}%
\pgfpathclose%
\pgfusepath{stroke,fill}%
\end{pgfscope}%
\begin{pgfscope}%
\pgfpathrectangle{\pgfqpoint{0.100000in}{0.220728in}}{\pgfqpoint{3.696000in}{3.696000in}}%
\pgfusepath{clip}%
\pgfsetbuttcap%
\pgfsetroundjoin%
\definecolor{currentfill}{rgb}{0.121569,0.466667,0.705882}%
\pgfsetfillcolor{currentfill}%
\pgfsetfillopacity{0.415710}%
\pgfsetlinewidth{1.003750pt}%
\definecolor{currentstroke}{rgb}{0.121569,0.466667,0.705882}%
\pgfsetstrokecolor{currentstroke}%
\pgfsetstrokeopacity{0.415710}%
\pgfsetdash{}{0pt}%
\pgfpathmoveto{\pgfqpoint{1.447444in}{2.522597in}}%
\pgfpathcurveto{\pgfqpoint{1.455680in}{2.522597in}}{\pgfqpoint{1.463580in}{2.525870in}}{\pgfqpoint{1.469404in}{2.531694in}}%
\pgfpathcurveto{\pgfqpoint{1.475228in}{2.537517in}}{\pgfqpoint{1.478500in}{2.545418in}}{\pgfqpoint{1.478500in}{2.553654in}}%
\pgfpathcurveto{\pgfqpoint{1.478500in}{2.561890in}}{\pgfqpoint{1.475228in}{2.569790in}}{\pgfqpoint{1.469404in}{2.575614in}}%
\pgfpathcurveto{\pgfqpoint{1.463580in}{2.581438in}}{\pgfqpoint{1.455680in}{2.584710in}}{\pgfqpoint{1.447444in}{2.584710in}}%
\pgfpathcurveto{\pgfqpoint{1.439208in}{2.584710in}}{\pgfqpoint{1.431308in}{2.581438in}}{\pgfqpoint{1.425484in}{2.575614in}}%
\pgfpathcurveto{\pgfqpoint{1.419660in}{2.569790in}}{\pgfqpoint{1.416388in}{2.561890in}}{\pgfqpoint{1.416388in}{2.553654in}}%
\pgfpathcurveto{\pgfqpoint{1.416388in}{2.545418in}}{\pgfqpoint{1.419660in}{2.537517in}}{\pgfqpoint{1.425484in}{2.531694in}}%
\pgfpathcurveto{\pgfqpoint{1.431308in}{2.525870in}}{\pgfqpoint{1.439208in}{2.522597in}}{\pgfqpoint{1.447444in}{2.522597in}}%
\pgfpathclose%
\pgfusepath{stroke,fill}%
\end{pgfscope}%
\begin{pgfscope}%
\pgfpathrectangle{\pgfqpoint{0.100000in}{0.220728in}}{\pgfqpoint{3.696000in}{3.696000in}}%
\pgfusepath{clip}%
\pgfsetbuttcap%
\pgfsetroundjoin%
\definecolor{currentfill}{rgb}{0.121569,0.466667,0.705882}%
\pgfsetfillcolor{currentfill}%
\pgfsetfillopacity{0.415925}%
\pgfsetlinewidth{1.003750pt}%
\definecolor{currentstroke}{rgb}{0.121569,0.466667,0.705882}%
\pgfsetstrokecolor{currentstroke}%
\pgfsetstrokeopacity{0.415925}%
\pgfsetdash{}{0pt}%
\pgfpathmoveto{\pgfqpoint{2.262169in}{3.186658in}}%
\pgfpathcurveto{\pgfqpoint{2.270406in}{3.186658in}}{\pgfqpoint{2.278306in}{3.189930in}}{\pgfqpoint{2.284130in}{3.195754in}}%
\pgfpathcurveto{\pgfqpoint{2.289954in}{3.201578in}}{\pgfqpoint{2.293226in}{3.209478in}}{\pgfqpoint{2.293226in}{3.217714in}}%
\pgfpathcurveto{\pgfqpoint{2.293226in}{3.225950in}}{\pgfqpoint{2.289954in}{3.233851in}}{\pgfqpoint{2.284130in}{3.239674in}}%
\pgfpathcurveto{\pgfqpoint{2.278306in}{3.245498in}}{\pgfqpoint{2.270406in}{3.248771in}}{\pgfqpoint{2.262169in}{3.248771in}}%
\pgfpathcurveto{\pgfqpoint{2.253933in}{3.248771in}}{\pgfqpoint{2.246033in}{3.245498in}}{\pgfqpoint{2.240209in}{3.239674in}}%
\pgfpathcurveto{\pgfqpoint{2.234385in}{3.233851in}}{\pgfqpoint{2.231113in}{3.225950in}}{\pgfqpoint{2.231113in}{3.217714in}}%
\pgfpathcurveto{\pgfqpoint{2.231113in}{3.209478in}}{\pgfqpoint{2.234385in}{3.201578in}}{\pgfqpoint{2.240209in}{3.195754in}}%
\pgfpathcurveto{\pgfqpoint{2.246033in}{3.189930in}}{\pgfqpoint{2.253933in}{3.186658in}}{\pgfqpoint{2.262169in}{3.186658in}}%
\pgfpathclose%
\pgfusepath{stroke,fill}%
\end{pgfscope}%
\begin{pgfscope}%
\pgfpathrectangle{\pgfqpoint{0.100000in}{0.220728in}}{\pgfqpoint{3.696000in}{3.696000in}}%
\pgfusepath{clip}%
\pgfsetbuttcap%
\pgfsetroundjoin%
\definecolor{currentfill}{rgb}{0.121569,0.466667,0.705882}%
\pgfsetfillcolor{currentfill}%
\pgfsetfillopacity{0.418154}%
\pgfsetlinewidth{1.003750pt}%
\definecolor{currentstroke}{rgb}{0.121569,0.466667,0.705882}%
\pgfsetstrokecolor{currentstroke}%
\pgfsetstrokeopacity{0.418154}%
\pgfsetdash{}{0pt}%
\pgfpathmoveto{\pgfqpoint{2.280608in}{3.182052in}}%
\pgfpathcurveto{\pgfqpoint{2.288844in}{3.182052in}}{\pgfqpoint{2.296744in}{3.185325in}}{\pgfqpoint{2.302568in}{3.191149in}}%
\pgfpathcurveto{\pgfqpoint{2.308392in}{3.196973in}}{\pgfqpoint{2.311664in}{3.204873in}}{\pgfqpoint{2.311664in}{3.213109in}}%
\pgfpathcurveto{\pgfqpoint{2.311664in}{3.221345in}}{\pgfqpoint{2.308392in}{3.229245in}}{\pgfqpoint{2.302568in}{3.235069in}}%
\pgfpathcurveto{\pgfqpoint{2.296744in}{3.240893in}}{\pgfqpoint{2.288844in}{3.244165in}}{\pgfqpoint{2.280608in}{3.244165in}}%
\pgfpathcurveto{\pgfqpoint{2.272372in}{3.244165in}}{\pgfqpoint{2.264472in}{3.240893in}}{\pgfqpoint{2.258648in}{3.235069in}}%
\pgfpathcurveto{\pgfqpoint{2.252824in}{3.229245in}}{\pgfqpoint{2.249551in}{3.221345in}}{\pgfqpoint{2.249551in}{3.213109in}}%
\pgfpathcurveto{\pgfqpoint{2.249551in}{3.204873in}}{\pgfqpoint{2.252824in}{3.196973in}}{\pgfqpoint{2.258648in}{3.191149in}}%
\pgfpathcurveto{\pgfqpoint{2.264472in}{3.185325in}}{\pgfqpoint{2.272372in}{3.182052in}}{\pgfqpoint{2.280608in}{3.182052in}}%
\pgfpathclose%
\pgfusepath{stroke,fill}%
\end{pgfscope}%
\begin{pgfscope}%
\pgfpathrectangle{\pgfqpoint{0.100000in}{0.220728in}}{\pgfqpoint{3.696000in}{3.696000in}}%
\pgfusepath{clip}%
\pgfsetbuttcap%
\pgfsetroundjoin%
\definecolor{currentfill}{rgb}{0.121569,0.466667,0.705882}%
\pgfsetfillcolor{currentfill}%
\pgfsetfillopacity{0.418782}%
\pgfsetlinewidth{1.003750pt}%
\definecolor{currentstroke}{rgb}{0.121569,0.466667,0.705882}%
\pgfsetstrokecolor{currentstroke}%
\pgfsetstrokeopacity{0.418782}%
\pgfsetdash{}{0pt}%
\pgfpathmoveto{\pgfqpoint{1.439858in}{2.505394in}}%
\pgfpathcurveto{\pgfqpoint{1.448094in}{2.505394in}}{\pgfqpoint{1.455994in}{2.508666in}}{\pgfqpoint{1.461818in}{2.514490in}}%
\pgfpathcurveto{\pgfqpoint{1.467642in}{2.520314in}}{\pgfqpoint{1.470914in}{2.528214in}}{\pgfqpoint{1.470914in}{2.536450in}}%
\pgfpathcurveto{\pgfqpoint{1.470914in}{2.544686in}}{\pgfqpoint{1.467642in}{2.552587in}}{\pgfqpoint{1.461818in}{2.558410in}}%
\pgfpathcurveto{\pgfqpoint{1.455994in}{2.564234in}}{\pgfqpoint{1.448094in}{2.567507in}}{\pgfqpoint{1.439858in}{2.567507in}}%
\pgfpathcurveto{\pgfqpoint{1.431622in}{2.567507in}}{\pgfqpoint{1.423722in}{2.564234in}}{\pgfqpoint{1.417898in}{2.558410in}}%
\pgfpathcurveto{\pgfqpoint{1.412074in}{2.552587in}}{\pgfqpoint{1.408801in}{2.544686in}}{\pgfqpoint{1.408801in}{2.536450in}}%
\pgfpathcurveto{\pgfqpoint{1.408801in}{2.528214in}}{\pgfqpoint{1.412074in}{2.520314in}}{\pgfqpoint{1.417898in}{2.514490in}}%
\pgfpathcurveto{\pgfqpoint{1.423722in}{2.508666in}}{\pgfqpoint{1.431622in}{2.505394in}}{\pgfqpoint{1.439858in}{2.505394in}}%
\pgfpathclose%
\pgfusepath{stroke,fill}%
\end{pgfscope}%
\begin{pgfscope}%
\pgfpathrectangle{\pgfqpoint{0.100000in}{0.220728in}}{\pgfqpoint{3.696000in}{3.696000in}}%
\pgfusepath{clip}%
\pgfsetbuttcap%
\pgfsetroundjoin%
\definecolor{currentfill}{rgb}{0.121569,0.466667,0.705882}%
\pgfsetfillcolor{currentfill}%
\pgfsetfillopacity{0.421805}%
\pgfsetlinewidth{1.003750pt}%
\definecolor{currentstroke}{rgb}{0.121569,0.466667,0.705882}%
\pgfsetstrokecolor{currentstroke}%
\pgfsetstrokeopacity{0.421805}%
\pgfsetdash{}{0pt}%
\pgfpathmoveto{\pgfqpoint{1.433786in}{2.488124in}}%
\pgfpathcurveto{\pgfqpoint{1.442022in}{2.488124in}}{\pgfqpoint{1.449922in}{2.491396in}}{\pgfqpoint{1.455746in}{2.497220in}}%
\pgfpathcurveto{\pgfqpoint{1.461570in}{2.503044in}}{\pgfqpoint{1.464843in}{2.510944in}}{\pgfqpoint{1.464843in}{2.519180in}}%
\pgfpathcurveto{\pgfqpoint{1.464843in}{2.527417in}}{\pgfqpoint{1.461570in}{2.535317in}}{\pgfqpoint{1.455746in}{2.541141in}}%
\pgfpathcurveto{\pgfqpoint{1.449922in}{2.546965in}}{\pgfqpoint{1.442022in}{2.550237in}}{\pgfqpoint{1.433786in}{2.550237in}}%
\pgfpathcurveto{\pgfqpoint{1.425550in}{2.550237in}}{\pgfqpoint{1.417650in}{2.546965in}}{\pgfqpoint{1.411826in}{2.541141in}}%
\pgfpathcurveto{\pgfqpoint{1.406002in}{2.535317in}}{\pgfqpoint{1.402730in}{2.527417in}}{\pgfqpoint{1.402730in}{2.519180in}}%
\pgfpathcurveto{\pgfqpoint{1.402730in}{2.510944in}}{\pgfqpoint{1.406002in}{2.503044in}}{\pgfqpoint{1.411826in}{2.497220in}}%
\pgfpathcurveto{\pgfqpoint{1.417650in}{2.491396in}}{\pgfqpoint{1.425550in}{2.488124in}}{\pgfqpoint{1.433786in}{2.488124in}}%
\pgfpathclose%
\pgfusepath{stroke,fill}%
\end{pgfscope}%
\begin{pgfscope}%
\pgfpathrectangle{\pgfqpoint{0.100000in}{0.220728in}}{\pgfqpoint{3.696000in}{3.696000in}}%
\pgfusepath{clip}%
\pgfsetbuttcap%
\pgfsetroundjoin%
\definecolor{currentfill}{rgb}{0.121569,0.466667,0.705882}%
\pgfsetfillcolor{currentfill}%
\pgfsetfillopacity{0.422588}%
\pgfsetlinewidth{1.003750pt}%
\definecolor{currentstroke}{rgb}{0.121569,0.466667,0.705882}%
\pgfsetstrokecolor{currentstroke}%
\pgfsetstrokeopacity{0.422588}%
\pgfsetdash{}{0pt}%
\pgfpathmoveto{\pgfqpoint{2.296318in}{3.174820in}}%
\pgfpathcurveto{\pgfqpoint{2.304554in}{3.174820in}}{\pgfqpoint{2.312454in}{3.178092in}}{\pgfqpoint{2.318278in}{3.183916in}}%
\pgfpathcurveto{\pgfqpoint{2.324102in}{3.189740in}}{\pgfqpoint{2.327374in}{3.197640in}}{\pgfqpoint{2.327374in}{3.205876in}}%
\pgfpathcurveto{\pgfqpoint{2.327374in}{3.214112in}}{\pgfqpoint{2.324102in}{3.222012in}}{\pgfqpoint{2.318278in}{3.227836in}}%
\pgfpathcurveto{\pgfqpoint{2.312454in}{3.233660in}}{\pgfqpoint{2.304554in}{3.236933in}}{\pgfqpoint{2.296318in}{3.236933in}}%
\pgfpathcurveto{\pgfqpoint{2.288082in}{3.236933in}}{\pgfqpoint{2.280182in}{3.233660in}}{\pgfqpoint{2.274358in}{3.227836in}}%
\pgfpathcurveto{\pgfqpoint{2.268534in}{3.222012in}}{\pgfqpoint{2.265261in}{3.214112in}}{\pgfqpoint{2.265261in}{3.205876in}}%
\pgfpathcurveto{\pgfqpoint{2.265261in}{3.197640in}}{\pgfqpoint{2.268534in}{3.189740in}}{\pgfqpoint{2.274358in}{3.183916in}}%
\pgfpathcurveto{\pgfqpoint{2.280182in}{3.178092in}}{\pgfqpoint{2.288082in}{3.174820in}}{\pgfqpoint{2.296318in}{3.174820in}}%
\pgfpathclose%
\pgfusepath{stroke,fill}%
\end{pgfscope}%
\begin{pgfscope}%
\pgfpathrectangle{\pgfqpoint{0.100000in}{0.220728in}}{\pgfqpoint{3.696000in}{3.696000in}}%
\pgfusepath{clip}%
\pgfsetbuttcap%
\pgfsetroundjoin%
\definecolor{currentfill}{rgb}{0.121569,0.466667,0.705882}%
\pgfsetfillcolor{currentfill}%
\pgfsetfillopacity{0.423778}%
\pgfsetlinewidth{1.003750pt}%
\definecolor{currentstroke}{rgb}{0.121569,0.466667,0.705882}%
\pgfsetstrokecolor{currentstroke}%
\pgfsetstrokeopacity{0.423778}%
\pgfsetdash{}{0pt}%
\pgfpathmoveto{\pgfqpoint{1.423890in}{2.474353in}}%
\pgfpathcurveto{\pgfqpoint{1.432126in}{2.474353in}}{\pgfqpoint{1.440026in}{2.477625in}}{\pgfqpoint{1.445850in}{2.483449in}}%
\pgfpathcurveto{\pgfqpoint{1.451674in}{2.489273in}}{\pgfqpoint{1.454947in}{2.497173in}}{\pgfqpoint{1.454947in}{2.505409in}}%
\pgfpathcurveto{\pgfqpoint{1.454947in}{2.513646in}}{\pgfqpoint{1.451674in}{2.521546in}}{\pgfqpoint{1.445850in}{2.527370in}}%
\pgfpathcurveto{\pgfqpoint{1.440026in}{2.533194in}}{\pgfqpoint{1.432126in}{2.536466in}}{\pgfqpoint{1.423890in}{2.536466in}}%
\pgfpathcurveto{\pgfqpoint{1.415654in}{2.536466in}}{\pgfqpoint{1.407754in}{2.533194in}}{\pgfqpoint{1.401930in}{2.527370in}}%
\pgfpathcurveto{\pgfqpoint{1.396106in}{2.521546in}}{\pgfqpoint{1.392834in}{2.513646in}}{\pgfqpoint{1.392834in}{2.505409in}}%
\pgfpathcurveto{\pgfqpoint{1.392834in}{2.497173in}}{\pgfqpoint{1.396106in}{2.489273in}}{\pgfqpoint{1.401930in}{2.483449in}}%
\pgfpathcurveto{\pgfqpoint{1.407754in}{2.477625in}}{\pgfqpoint{1.415654in}{2.474353in}}{\pgfqpoint{1.423890in}{2.474353in}}%
\pgfpathclose%
\pgfusepath{stroke,fill}%
\end{pgfscope}%
\begin{pgfscope}%
\pgfpathrectangle{\pgfqpoint{0.100000in}{0.220728in}}{\pgfqpoint{3.696000in}{3.696000in}}%
\pgfusepath{clip}%
\pgfsetbuttcap%
\pgfsetroundjoin%
\definecolor{currentfill}{rgb}{0.121569,0.466667,0.705882}%
\pgfsetfillcolor{currentfill}%
\pgfsetfillopacity{0.424030}%
\pgfsetlinewidth{1.003750pt}%
\definecolor{currentstroke}{rgb}{0.121569,0.466667,0.705882}%
\pgfsetstrokecolor{currentstroke}%
\pgfsetstrokeopacity{0.424030}%
\pgfsetdash{}{0pt}%
\pgfpathmoveto{\pgfqpoint{2.311928in}{3.169467in}}%
\pgfpathcurveto{\pgfqpoint{2.320165in}{3.169467in}}{\pgfqpoint{2.328065in}{3.172739in}}{\pgfqpoint{2.333889in}{3.178563in}}%
\pgfpathcurveto{\pgfqpoint{2.339713in}{3.184387in}}{\pgfqpoint{2.342985in}{3.192287in}}{\pgfqpoint{2.342985in}{3.200523in}}%
\pgfpathcurveto{\pgfqpoint{2.342985in}{3.208759in}}{\pgfqpoint{2.339713in}{3.216660in}}{\pgfqpoint{2.333889in}{3.222483in}}%
\pgfpathcurveto{\pgfqpoint{2.328065in}{3.228307in}}{\pgfqpoint{2.320165in}{3.231580in}}{\pgfqpoint{2.311928in}{3.231580in}}%
\pgfpathcurveto{\pgfqpoint{2.303692in}{3.231580in}}{\pgfqpoint{2.295792in}{3.228307in}}{\pgfqpoint{2.289968in}{3.222483in}}%
\pgfpathcurveto{\pgfqpoint{2.284144in}{3.216660in}}{\pgfqpoint{2.280872in}{3.208759in}}{\pgfqpoint{2.280872in}{3.200523in}}%
\pgfpathcurveto{\pgfqpoint{2.280872in}{3.192287in}}{\pgfqpoint{2.284144in}{3.184387in}}{\pgfqpoint{2.289968in}{3.178563in}}%
\pgfpathcurveto{\pgfqpoint{2.295792in}{3.172739in}}{\pgfqpoint{2.303692in}{3.169467in}}{\pgfqpoint{2.311928in}{3.169467in}}%
\pgfpathclose%
\pgfusepath{stroke,fill}%
\end{pgfscope}%
\begin{pgfscope}%
\pgfpathrectangle{\pgfqpoint{0.100000in}{0.220728in}}{\pgfqpoint{3.696000in}{3.696000in}}%
\pgfusepath{clip}%
\pgfsetbuttcap%
\pgfsetroundjoin%
\definecolor{currentfill}{rgb}{0.121569,0.466667,0.705882}%
\pgfsetfillcolor{currentfill}%
\pgfsetfillopacity{0.424666}%
\pgfsetlinewidth{1.003750pt}%
\definecolor{currentstroke}{rgb}{0.121569,0.466667,0.705882}%
\pgfsetstrokecolor{currentstroke}%
\pgfsetstrokeopacity{0.424666}%
\pgfsetdash{}{0pt}%
\pgfpathmoveto{\pgfqpoint{2.305628in}{3.171097in}}%
\pgfpathcurveto{\pgfqpoint{2.313864in}{3.171097in}}{\pgfqpoint{2.321764in}{3.174369in}}{\pgfqpoint{2.327588in}{3.180193in}}%
\pgfpathcurveto{\pgfqpoint{2.333412in}{3.186017in}}{\pgfqpoint{2.336684in}{3.193917in}}{\pgfqpoint{2.336684in}{3.202153in}}%
\pgfpathcurveto{\pgfqpoint{2.336684in}{3.210390in}}{\pgfqpoint{2.333412in}{3.218290in}}{\pgfqpoint{2.327588in}{3.224114in}}%
\pgfpathcurveto{\pgfqpoint{2.321764in}{3.229937in}}{\pgfqpoint{2.313864in}{3.233210in}}{\pgfqpoint{2.305628in}{3.233210in}}%
\pgfpathcurveto{\pgfqpoint{2.297392in}{3.233210in}}{\pgfqpoint{2.289492in}{3.229937in}}{\pgfqpoint{2.283668in}{3.224114in}}%
\pgfpathcurveto{\pgfqpoint{2.277844in}{3.218290in}}{\pgfqpoint{2.274571in}{3.210390in}}{\pgfqpoint{2.274571in}{3.202153in}}%
\pgfpathcurveto{\pgfqpoint{2.274571in}{3.193917in}}{\pgfqpoint{2.277844in}{3.186017in}}{\pgfqpoint{2.283668in}{3.180193in}}%
\pgfpathcurveto{\pgfqpoint{2.289492in}{3.174369in}}{\pgfqpoint{2.297392in}{3.171097in}}{\pgfqpoint{2.305628in}{3.171097in}}%
\pgfpathclose%
\pgfusepath{stroke,fill}%
\end{pgfscope}%
\begin{pgfscope}%
\pgfpathrectangle{\pgfqpoint{0.100000in}{0.220728in}}{\pgfqpoint{3.696000in}{3.696000in}}%
\pgfusepath{clip}%
\pgfsetbuttcap%
\pgfsetroundjoin%
\definecolor{currentfill}{rgb}{0.121569,0.466667,0.705882}%
\pgfsetfillcolor{currentfill}%
\pgfsetfillopacity{0.425745}%
\pgfsetlinewidth{1.003750pt}%
\definecolor{currentstroke}{rgb}{0.121569,0.466667,0.705882}%
\pgfsetstrokecolor{currentstroke}%
\pgfsetstrokeopacity{0.425745}%
\pgfsetdash{}{0pt}%
\pgfpathmoveto{\pgfqpoint{2.317784in}{3.168443in}}%
\pgfpathcurveto{\pgfqpoint{2.326020in}{3.168443in}}{\pgfqpoint{2.333920in}{3.171715in}}{\pgfqpoint{2.339744in}{3.177539in}}%
\pgfpathcurveto{\pgfqpoint{2.345568in}{3.183363in}}{\pgfqpoint{2.348840in}{3.191263in}}{\pgfqpoint{2.348840in}{3.199499in}}%
\pgfpathcurveto{\pgfqpoint{2.348840in}{3.207735in}}{\pgfqpoint{2.345568in}{3.215635in}}{\pgfqpoint{2.339744in}{3.221459in}}%
\pgfpathcurveto{\pgfqpoint{2.333920in}{3.227283in}}{\pgfqpoint{2.326020in}{3.230555in}}{\pgfqpoint{2.317784in}{3.230555in}}%
\pgfpathcurveto{\pgfqpoint{2.309548in}{3.230555in}}{\pgfqpoint{2.301648in}{3.227283in}}{\pgfqpoint{2.295824in}{3.221459in}}%
\pgfpathcurveto{\pgfqpoint{2.290000in}{3.215635in}}{\pgfqpoint{2.286727in}{3.207735in}}{\pgfqpoint{2.286727in}{3.199499in}}%
\pgfpathcurveto{\pgfqpoint{2.286727in}{3.191263in}}{\pgfqpoint{2.290000in}{3.183363in}}{\pgfqpoint{2.295824in}{3.177539in}}%
\pgfpathcurveto{\pgfqpoint{2.301648in}{3.171715in}}{\pgfqpoint{2.309548in}{3.168443in}}{\pgfqpoint{2.317784in}{3.168443in}}%
\pgfpathclose%
\pgfusepath{stroke,fill}%
\end{pgfscope}%
\begin{pgfscope}%
\pgfpathrectangle{\pgfqpoint{0.100000in}{0.220728in}}{\pgfqpoint{3.696000in}{3.696000in}}%
\pgfusepath{clip}%
\pgfsetbuttcap%
\pgfsetroundjoin%
\definecolor{currentfill}{rgb}{0.121569,0.466667,0.705882}%
\pgfsetfillcolor{currentfill}%
\pgfsetfillopacity{0.426502}%
\pgfsetlinewidth{1.003750pt}%
\definecolor{currentstroke}{rgb}{0.121569,0.466667,0.705882}%
\pgfsetstrokecolor{currentstroke}%
\pgfsetstrokeopacity{0.426502}%
\pgfsetdash{}{0pt}%
\pgfpathmoveto{\pgfqpoint{1.421638in}{2.458326in}}%
\pgfpathcurveto{\pgfqpoint{1.429874in}{2.458326in}}{\pgfqpoint{1.437774in}{2.461598in}}{\pgfqpoint{1.443598in}{2.467422in}}%
\pgfpathcurveto{\pgfqpoint{1.449422in}{2.473246in}}{\pgfqpoint{1.452694in}{2.481146in}}{\pgfqpoint{1.452694in}{2.489383in}}%
\pgfpathcurveto{\pgfqpoint{1.452694in}{2.497619in}}{\pgfqpoint{1.449422in}{2.505519in}}{\pgfqpoint{1.443598in}{2.511343in}}%
\pgfpathcurveto{\pgfqpoint{1.437774in}{2.517167in}}{\pgfqpoint{1.429874in}{2.520439in}}{\pgfqpoint{1.421638in}{2.520439in}}%
\pgfpathcurveto{\pgfqpoint{1.413401in}{2.520439in}}{\pgfqpoint{1.405501in}{2.517167in}}{\pgfqpoint{1.399677in}{2.511343in}}%
\pgfpathcurveto{\pgfqpoint{1.393853in}{2.505519in}}{\pgfqpoint{1.390581in}{2.497619in}}{\pgfqpoint{1.390581in}{2.489383in}}%
\pgfpathcurveto{\pgfqpoint{1.390581in}{2.481146in}}{\pgfqpoint{1.393853in}{2.473246in}}{\pgfqpoint{1.399677in}{2.467422in}}%
\pgfpathcurveto{\pgfqpoint{1.405501in}{2.461598in}}{\pgfqpoint{1.413401in}{2.458326in}}{\pgfqpoint{1.421638in}{2.458326in}}%
\pgfpathclose%
\pgfusepath{stroke,fill}%
\end{pgfscope}%
\begin{pgfscope}%
\pgfpathrectangle{\pgfqpoint{0.100000in}{0.220728in}}{\pgfqpoint{3.696000in}{3.696000in}}%
\pgfusepath{clip}%
\pgfsetbuttcap%
\pgfsetroundjoin%
\definecolor{currentfill}{rgb}{0.121569,0.466667,0.705882}%
\pgfsetfillcolor{currentfill}%
\pgfsetfillopacity{0.427108}%
\pgfsetlinewidth{1.003750pt}%
\definecolor{currentstroke}{rgb}{0.121569,0.466667,0.705882}%
\pgfsetstrokecolor{currentstroke}%
\pgfsetstrokeopacity{0.427108}%
\pgfsetdash{}{0pt}%
\pgfpathmoveto{\pgfqpoint{2.325079in}{3.166234in}}%
\pgfpathcurveto{\pgfqpoint{2.333315in}{3.166234in}}{\pgfqpoint{2.341215in}{3.169506in}}{\pgfqpoint{2.347039in}{3.175330in}}%
\pgfpathcurveto{\pgfqpoint{2.352863in}{3.181154in}}{\pgfqpoint{2.356135in}{3.189054in}}{\pgfqpoint{2.356135in}{3.197290in}}%
\pgfpathcurveto{\pgfqpoint{2.356135in}{3.205526in}}{\pgfqpoint{2.352863in}{3.213426in}}{\pgfqpoint{2.347039in}{3.219250in}}%
\pgfpathcurveto{\pgfqpoint{2.341215in}{3.225074in}}{\pgfqpoint{2.333315in}{3.228347in}}{\pgfqpoint{2.325079in}{3.228347in}}%
\pgfpathcurveto{\pgfqpoint{2.316842in}{3.228347in}}{\pgfqpoint{2.308942in}{3.225074in}}{\pgfqpoint{2.303118in}{3.219250in}}%
\pgfpathcurveto{\pgfqpoint{2.297295in}{3.213426in}}{\pgfqpoint{2.294022in}{3.205526in}}{\pgfqpoint{2.294022in}{3.197290in}}%
\pgfpathcurveto{\pgfqpoint{2.294022in}{3.189054in}}{\pgfqpoint{2.297295in}{3.181154in}}{\pgfqpoint{2.303118in}{3.175330in}}%
\pgfpathcurveto{\pgfqpoint{2.308942in}{3.169506in}}{\pgfqpoint{2.316842in}{3.166234in}}{\pgfqpoint{2.325079in}{3.166234in}}%
\pgfpathclose%
\pgfusepath{stroke,fill}%
\end{pgfscope}%
\begin{pgfscope}%
\pgfpathrectangle{\pgfqpoint{0.100000in}{0.220728in}}{\pgfqpoint{3.696000in}{3.696000in}}%
\pgfusepath{clip}%
\pgfsetbuttcap%
\pgfsetroundjoin%
\definecolor{currentfill}{rgb}{0.121569,0.466667,0.705882}%
\pgfsetfillcolor{currentfill}%
\pgfsetfillopacity{0.427698}%
\pgfsetlinewidth{1.003750pt}%
\definecolor{currentstroke}{rgb}{0.121569,0.466667,0.705882}%
\pgfsetstrokecolor{currentstroke}%
\pgfsetstrokeopacity{0.427698}%
\pgfsetdash{}{0pt}%
\pgfpathmoveto{\pgfqpoint{1.413759in}{2.448061in}}%
\pgfpathcurveto{\pgfqpoint{1.421995in}{2.448061in}}{\pgfqpoint{1.429896in}{2.451334in}}{\pgfqpoint{1.435719in}{2.457158in}}%
\pgfpathcurveto{\pgfqpoint{1.441543in}{2.462982in}}{\pgfqpoint{1.444816in}{2.470882in}}{\pgfqpoint{1.444816in}{2.479118in}}%
\pgfpathcurveto{\pgfqpoint{1.444816in}{2.487354in}}{\pgfqpoint{1.441543in}{2.495254in}}{\pgfqpoint{1.435719in}{2.501078in}}%
\pgfpathcurveto{\pgfqpoint{1.429896in}{2.506902in}}{\pgfqpoint{1.421995in}{2.510174in}}{\pgfqpoint{1.413759in}{2.510174in}}%
\pgfpathcurveto{\pgfqpoint{1.405523in}{2.510174in}}{\pgfqpoint{1.397623in}{2.506902in}}{\pgfqpoint{1.391799in}{2.501078in}}%
\pgfpathcurveto{\pgfqpoint{1.385975in}{2.495254in}}{\pgfqpoint{1.382703in}{2.487354in}}{\pgfqpoint{1.382703in}{2.479118in}}%
\pgfpathcurveto{\pgfqpoint{1.382703in}{2.470882in}}{\pgfqpoint{1.385975in}{2.462982in}}{\pgfqpoint{1.391799in}{2.457158in}}%
\pgfpathcurveto{\pgfqpoint{1.397623in}{2.451334in}}{\pgfqpoint{1.405523in}{2.448061in}}{\pgfqpoint{1.413759in}{2.448061in}}%
\pgfpathclose%
\pgfusepath{stroke,fill}%
\end{pgfscope}%
\begin{pgfscope}%
\pgfpathrectangle{\pgfqpoint{0.100000in}{0.220728in}}{\pgfqpoint{3.696000in}{3.696000in}}%
\pgfusepath{clip}%
\pgfsetbuttcap%
\pgfsetroundjoin%
\definecolor{currentfill}{rgb}{0.121569,0.466667,0.705882}%
\pgfsetfillcolor{currentfill}%
\pgfsetfillopacity{0.427859}%
\pgfsetlinewidth{1.003750pt}%
\definecolor{currentstroke}{rgb}{0.121569,0.466667,0.705882}%
\pgfsetstrokecolor{currentstroke}%
\pgfsetstrokeopacity{0.427859}%
\pgfsetdash{}{0pt}%
\pgfpathmoveto{\pgfqpoint{2.328996in}{3.164695in}}%
\pgfpathcurveto{\pgfqpoint{2.337232in}{3.164695in}}{\pgfqpoint{2.345133in}{3.167967in}}{\pgfqpoint{2.350956in}{3.173791in}}%
\pgfpathcurveto{\pgfqpoint{2.356780in}{3.179615in}}{\pgfqpoint{2.360053in}{3.187515in}}{\pgfqpoint{2.360053in}{3.195751in}}%
\pgfpathcurveto{\pgfqpoint{2.360053in}{3.203988in}}{\pgfqpoint{2.356780in}{3.211888in}}{\pgfqpoint{2.350956in}{3.217712in}}%
\pgfpathcurveto{\pgfqpoint{2.345133in}{3.223536in}}{\pgfqpoint{2.337232in}{3.226808in}}{\pgfqpoint{2.328996in}{3.226808in}}%
\pgfpathcurveto{\pgfqpoint{2.320760in}{3.226808in}}{\pgfqpoint{2.312860in}{3.223536in}}{\pgfqpoint{2.307036in}{3.217712in}}%
\pgfpathcurveto{\pgfqpoint{2.301212in}{3.211888in}}{\pgfqpoint{2.297940in}{3.203988in}}{\pgfqpoint{2.297940in}{3.195751in}}%
\pgfpathcurveto{\pgfqpoint{2.297940in}{3.187515in}}{\pgfqpoint{2.301212in}{3.179615in}}{\pgfqpoint{2.307036in}{3.173791in}}%
\pgfpathcurveto{\pgfqpoint{2.312860in}{3.167967in}}{\pgfqpoint{2.320760in}{3.164695in}}{\pgfqpoint{2.328996in}{3.164695in}}%
\pgfpathclose%
\pgfusepath{stroke,fill}%
\end{pgfscope}%
\begin{pgfscope}%
\pgfpathrectangle{\pgfqpoint{0.100000in}{0.220728in}}{\pgfqpoint{3.696000in}{3.696000in}}%
\pgfusepath{clip}%
\pgfsetbuttcap%
\pgfsetroundjoin%
\definecolor{currentfill}{rgb}{0.121569,0.466667,0.705882}%
\pgfsetfillcolor{currentfill}%
\pgfsetfillopacity{0.428314}%
\pgfsetlinewidth{1.003750pt}%
\definecolor{currentstroke}{rgb}{0.121569,0.466667,0.705882}%
\pgfsetstrokecolor{currentstroke}%
\pgfsetstrokeopacity{0.428314}%
\pgfsetdash{}{0pt}%
\pgfpathmoveto{\pgfqpoint{2.331213in}{3.164240in}}%
\pgfpathcurveto{\pgfqpoint{2.339449in}{3.164240in}}{\pgfqpoint{2.347349in}{3.167512in}}{\pgfqpoint{2.353173in}{3.173336in}}%
\pgfpathcurveto{\pgfqpoint{2.358997in}{3.179160in}}{\pgfqpoint{2.362269in}{3.187060in}}{\pgfqpoint{2.362269in}{3.195296in}}%
\pgfpathcurveto{\pgfqpoint{2.362269in}{3.203533in}}{\pgfqpoint{2.358997in}{3.211433in}}{\pgfqpoint{2.353173in}{3.217257in}}%
\pgfpathcurveto{\pgfqpoint{2.347349in}{3.223080in}}{\pgfqpoint{2.339449in}{3.226353in}}{\pgfqpoint{2.331213in}{3.226353in}}%
\pgfpathcurveto{\pgfqpoint{2.322977in}{3.226353in}}{\pgfqpoint{2.315077in}{3.223080in}}{\pgfqpoint{2.309253in}{3.217257in}}%
\pgfpathcurveto{\pgfqpoint{2.303429in}{3.211433in}}{\pgfqpoint{2.300156in}{3.203533in}}{\pgfqpoint{2.300156in}{3.195296in}}%
\pgfpathcurveto{\pgfqpoint{2.300156in}{3.187060in}}{\pgfqpoint{2.303429in}{3.179160in}}{\pgfqpoint{2.309253in}{3.173336in}}%
\pgfpathcurveto{\pgfqpoint{2.315077in}{3.167512in}}{\pgfqpoint{2.322977in}{3.164240in}}{\pgfqpoint{2.331213in}{3.164240in}}%
\pgfpathclose%
\pgfusepath{stroke,fill}%
\end{pgfscope}%
\begin{pgfscope}%
\pgfpathrectangle{\pgfqpoint{0.100000in}{0.220728in}}{\pgfqpoint{3.696000in}{3.696000in}}%
\pgfusepath{clip}%
\pgfsetbuttcap%
\pgfsetroundjoin%
\definecolor{currentfill}{rgb}{0.121569,0.466667,0.705882}%
\pgfsetfillcolor{currentfill}%
\pgfsetfillopacity{0.428696}%
\pgfsetlinewidth{1.003750pt}%
\definecolor{currentstroke}{rgb}{0.121569,0.466667,0.705882}%
\pgfsetstrokecolor{currentstroke}%
\pgfsetstrokeopacity{0.428696}%
\pgfsetdash{}{0pt}%
\pgfpathmoveto{\pgfqpoint{2.334163in}{3.162714in}}%
\pgfpathcurveto{\pgfqpoint{2.342399in}{3.162714in}}{\pgfqpoint{2.350299in}{3.165986in}}{\pgfqpoint{2.356123in}{3.171810in}}%
\pgfpathcurveto{\pgfqpoint{2.361947in}{3.177634in}}{\pgfqpoint{2.365220in}{3.185534in}}{\pgfqpoint{2.365220in}{3.193770in}}%
\pgfpathcurveto{\pgfqpoint{2.365220in}{3.202006in}}{\pgfqpoint{2.361947in}{3.209906in}}{\pgfqpoint{2.356123in}{3.215730in}}%
\pgfpathcurveto{\pgfqpoint{2.350299in}{3.221554in}}{\pgfqpoint{2.342399in}{3.224827in}}{\pgfqpoint{2.334163in}{3.224827in}}%
\pgfpathcurveto{\pgfqpoint{2.325927in}{3.224827in}}{\pgfqpoint{2.318027in}{3.221554in}}{\pgfqpoint{2.312203in}{3.215730in}}%
\pgfpathcurveto{\pgfqpoint{2.306379in}{3.209906in}}{\pgfqpoint{2.303107in}{3.202006in}}{\pgfqpoint{2.303107in}{3.193770in}}%
\pgfpathcurveto{\pgfqpoint{2.303107in}{3.185534in}}{\pgfqpoint{2.306379in}{3.177634in}}{\pgfqpoint{2.312203in}{3.171810in}}%
\pgfpathcurveto{\pgfqpoint{2.318027in}{3.165986in}}{\pgfqpoint{2.325927in}{3.162714in}}{\pgfqpoint{2.334163in}{3.162714in}}%
\pgfpathclose%
\pgfusepath{stroke,fill}%
\end{pgfscope}%
\begin{pgfscope}%
\pgfpathrectangle{\pgfqpoint{0.100000in}{0.220728in}}{\pgfqpoint{3.696000in}{3.696000in}}%
\pgfusepath{clip}%
\pgfsetbuttcap%
\pgfsetroundjoin%
\definecolor{currentfill}{rgb}{0.121569,0.466667,0.705882}%
\pgfsetfillcolor{currentfill}%
\pgfsetfillopacity{0.429170}%
\pgfsetlinewidth{1.003750pt}%
\definecolor{currentstroke}{rgb}{0.121569,0.466667,0.705882}%
\pgfsetstrokecolor{currentstroke}%
\pgfsetstrokeopacity{0.429170}%
\pgfsetdash{}{0pt}%
\pgfpathmoveto{\pgfqpoint{2.335601in}{3.162427in}}%
\pgfpathcurveto{\pgfqpoint{2.343837in}{3.162427in}}{\pgfqpoint{2.351737in}{3.165699in}}{\pgfqpoint{2.357561in}{3.171523in}}%
\pgfpathcurveto{\pgfqpoint{2.363385in}{3.177347in}}{\pgfqpoint{2.366657in}{3.185247in}}{\pgfqpoint{2.366657in}{3.193483in}}%
\pgfpathcurveto{\pgfqpoint{2.366657in}{3.201719in}}{\pgfqpoint{2.363385in}{3.209620in}}{\pgfqpoint{2.357561in}{3.215443in}}%
\pgfpathcurveto{\pgfqpoint{2.351737in}{3.221267in}}{\pgfqpoint{2.343837in}{3.224540in}}{\pgfqpoint{2.335601in}{3.224540in}}%
\pgfpathcurveto{\pgfqpoint{2.327364in}{3.224540in}}{\pgfqpoint{2.319464in}{3.221267in}}{\pgfqpoint{2.313640in}{3.215443in}}%
\pgfpathcurveto{\pgfqpoint{2.307816in}{3.209620in}}{\pgfqpoint{2.304544in}{3.201719in}}{\pgfqpoint{2.304544in}{3.193483in}}%
\pgfpathcurveto{\pgfqpoint{2.304544in}{3.185247in}}{\pgfqpoint{2.307816in}{3.177347in}}{\pgfqpoint{2.313640in}{3.171523in}}%
\pgfpathcurveto{\pgfqpoint{2.319464in}{3.165699in}}{\pgfqpoint{2.327364in}{3.162427in}}{\pgfqpoint{2.335601in}{3.162427in}}%
\pgfpathclose%
\pgfusepath{stroke,fill}%
\end{pgfscope}%
\begin{pgfscope}%
\pgfpathrectangle{\pgfqpoint{0.100000in}{0.220728in}}{\pgfqpoint{3.696000in}{3.696000in}}%
\pgfusepath{clip}%
\pgfsetbuttcap%
\pgfsetroundjoin%
\definecolor{currentfill}{rgb}{0.121569,0.466667,0.705882}%
\pgfsetfillcolor{currentfill}%
\pgfsetfillopacity{0.429461}%
\pgfsetlinewidth{1.003750pt}%
\definecolor{currentstroke}{rgb}{0.121569,0.466667,0.705882}%
\pgfsetstrokecolor{currentstroke}%
\pgfsetstrokeopacity{0.429461}%
\pgfsetdash{}{0pt}%
\pgfpathmoveto{\pgfqpoint{1.411287in}{2.437893in}}%
\pgfpathcurveto{\pgfqpoint{1.419524in}{2.437893in}}{\pgfqpoint{1.427424in}{2.441165in}}{\pgfqpoint{1.433248in}{2.446989in}}%
\pgfpathcurveto{\pgfqpoint{1.439072in}{2.452813in}}{\pgfqpoint{1.442344in}{2.460713in}}{\pgfqpoint{1.442344in}{2.468949in}}%
\pgfpathcurveto{\pgfqpoint{1.442344in}{2.477185in}}{\pgfqpoint{1.439072in}{2.485086in}}{\pgfqpoint{1.433248in}{2.490909in}}%
\pgfpathcurveto{\pgfqpoint{1.427424in}{2.496733in}}{\pgfqpoint{1.419524in}{2.500006in}}{\pgfqpoint{1.411287in}{2.500006in}}%
\pgfpathcurveto{\pgfqpoint{1.403051in}{2.500006in}}{\pgfqpoint{1.395151in}{2.496733in}}{\pgfqpoint{1.389327in}{2.490909in}}%
\pgfpathcurveto{\pgfqpoint{1.383503in}{2.485086in}}{\pgfqpoint{1.380231in}{2.477185in}}{\pgfqpoint{1.380231in}{2.468949in}}%
\pgfpathcurveto{\pgfqpoint{1.380231in}{2.460713in}}{\pgfqpoint{1.383503in}{2.452813in}}{\pgfqpoint{1.389327in}{2.446989in}}%
\pgfpathcurveto{\pgfqpoint{1.395151in}{2.441165in}}{\pgfqpoint{1.403051in}{2.437893in}}{\pgfqpoint{1.411287in}{2.437893in}}%
\pgfpathclose%
\pgfusepath{stroke,fill}%
\end{pgfscope}%
\begin{pgfscope}%
\pgfpathrectangle{\pgfqpoint{0.100000in}{0.220728in}}{\pgfqpoint{3.696000in}{3.696000in}}%
\pgfusepath{clip}%
\pgfsetbuttcap%
\pgfsetroundjoin%
\definecolor{currentfill}{rgb}{0.121569,0.466667,0.705882}%
\pgfsetfillcolor{currentfill}%
\pgfsetfillopacity{0.429884}%
\pgfsetlinewidth{1.003750pt}%
\definecolor{currentstroke}{rgb}{0.121569,0.466667,0.705882}%
\pgfsetstrokecolor{currentstroke}%
\pgfsetstrokeopacity{0.429884}%
\pgfsetdash{}{0pt}%
\pgfpathmoveto{\pgfqpoint{2.339520in}{3.161151in}}%
\pgfpathcurveto{\pgfqpoint{2.347756in}{3.161151in}}{\pgfqpoint{2.355656in}{3.164424in}}{\pgfqpoint{2.361480in}{3.170248in}}%
\pgfpathcurveto{\pgfqpoint{2.367304in}{3.176072in}}{\pgfqpoint{2.370577in}{3.183972in}}{\pgfqpoint{2.370577in}{3.192208in}}%
\pgfpathcurveto{\pgfqpoint{2.370577in}{3.200444in}}{\pgfqpoint{2.367304in}{3.208344in}}{\pgfqpoint{2.361480in}{3.214168in}}%
\pgfpathcurveto{\pgfqpoint{2.355656in}{3.219992in}}{\pgfqpoint{2.347756in}{3.223264in}}{\pgfqpoint{2.339520in}{3.223264in}}%
\pgfpathcurveto{\pgfqpoint{2.331284in}{3.223264in}}{\pgfqpoint{2.323384in}{3.219992in}}{\pgfqpoint{2.317560in}{3.214168in}}%
\pgfpathcurveto{\pgfqpoint{2.311736in}{3.208344in}}{\pgfqpoint{2.308464in}{3.200444in}}{\pgfqpoint{2.308464in}{3.192208in}}%
\pgfpathcurveto{\pgfqpoint{2.308464in}{3.183972in}}{\pgfqpoint{2.311736in}{3.176072in}}{\pgfqpoint{2.317560in}{3.170248in}}%
\pgfpathcurveto{\pgfqpoint{2.323384in}{3.164424in}}{\pgfqpoint{2.331284in}{3.161151in}}{\pgfqpoint{2.339520in}{3.161151in}}%
\pgfpathclose%
\pgfusepath{stroke,fill}%
\end{pgfscope}%
\begin{pgfscope}%
\pgfpathrectangle{\pgfqpoint{0.100000in}{0.220728in}}{\pgfqpoint{3.696000in}{3.696000in}}%
\pgfusepath{clip}%
\pgfsetbuttcap%
\pgfsetroundjoin%
\definecolor{currentfill}{rgb}{0.121569,0.466667,0.705882}%
\pgfsetfillcolor{currentfill}%
\pgfsetfillopacity{0.430340}%
\pgfsetlinewidth{1.003750pt}%
\definecolor{currentstroke}{rgb}{0.121569,0.466667,0.705882}%
\pgfsetstrokecolor{currentstroke}%
\pgfsetstrokeopacity{0.430340}%
\pgfsetdash{}{0pt}%
\pgfpathmoveto{\pgfqpoint{1.406301in}{2.430940in}}%
\pgfpathcurveto{\pgfqpoint{1.414538in}{2.430940in}}{\pgfqpoint{1.422438in}{2.434212in}}{\pgfqpoint{1.428262in}{2.440036in}}%
\pgfpathcurveto{\pgfqpoint{1.434086in}{2.445860in}}{\pgfqpoint{1.437358in}{2.453760in}}{\pgfqpoint{1.437358in}{2.461996in}}%
\pgfpathcurveto{\pgfqpoint{1.437358in}{2.470232in}}{\pgfqpoint{1.434086in}{2.478132in}}{\pgfqpoint{1.428262in}{2.483956in}}%
\pgfpathcurveto{\pgfqpoint{1.422438in}{2.489780in}}{\pgfqpoint{1.414538in}{2.493053in}}{\pgfqpoint{1.406301in}{2.493053in}}%
\pgfpathcurveto{\pgfqpoint{1.398065in}{2.493053in}}{\pgfqpoint{1.390165in}{2.489780in}}{\pgfqpoint{1.384341in}{2.483956in}}%
\pgfpathcurveto{\pgfqpoint{1.378517in}{2.478132in}}{\pgfqpoint{1.375245in}{2.470232in}}{\pgfqpoint{1.375245in}{2.461996in}}%
\pgfpathcurveto{\pgfqpoint{1.375245in}{2.453760in}}{\pgfqpoint{1.378517in}{2.445860in}}{\pgfqpoint{1.384341in}{2.440036in}}%
\pgfpathcurveto{\pgfqpoint{1.390165in}{2.434212in}}{\pgfqpoint{1.398065in}{2.430940in}}{\pgfqpoint{1.406301in}{2.430940in}}%
\pgfpathclose%
\pgfusepath{stroke,fill}%
\end{pgfscope}%
\begin{pgfscope}%
\pgfpathrectangle{\pgfqpoint{0.100000in}{0.220728in}}{\pgfqpoint{3.696000in}{3.696000in}}%
\pgfusepath{clip}%
\pgfsetbuttcap%
\pgfsetroundjoin%
\definecolor{currentfill}{rgb}{0.121569,0.466667,0.705882}%
\pgfsetfillcolor{currentfill}%
\pgfsetfillopacity{0.430430}%
\pgfsetlinewidth{1.003750pt}%
\definecolor{currentstroke}{rgb}{0.121569,0.466667,0.705882}%
\pgfsetstrokecolor{currentstroke}%
\pgfsetstrokeopacity{0.430430}%
\pgfsetdash{}{0pt}%
\pgfpathmoveto{\pgfqpoint{2.341463in}{3.160447in}}%
\pgfpathcurveto{\pgfqpoint{2.349700in}{3.160447in}}{\pgfqpoint{2.357600in}{3.163720in}}{\pgfqpoint{2.363424in}{3.169544in}}%
\pgfpathcurveto{\pgfqpoint{2.369247in}{3.175368in}}{\pgfqpoint{2.372520in}{3.183268in}}{\pgfqpoint{2.372520in}{3.191504in}}%
\pgfpathcurveto{\pgfqpoint{2.372520in}{3.199740in}}{\pgfqpoint{2.369247in}{3.207640in}}{\pgfqpoint{2.363424in}{3.213464in}}%
\pgfpathcurveto{\pgfqpoint{2.357600in}{3.219288in}}{\pgfqpoint{2.349700in}{3.222560in}}{\pgfqpoint{2.341463in}{3.222560in}}%
\pgfpathcurveto{\pgfqpoint{2.333227in}{3.222560in}}{\pgfqpoint{2.325327in}{3.219288in}}{\pgfqpoint{2.319503in}{3.213464in}}%
\pgfpathcurveto{\pgfqpoint{2.313679in}{3.207640in}}{\pgfqpoint{2.310407in}{3.199740in}}{\pgfqpoint{2.310407in}{3.191504in}}%
\pgfpathcurveto{\pgfqpoint{2.310407in}{3.183268in}}{\pgfqpoint{2.313679in}{3.175368in}}{\pgfqpoint{2.319503in}{3.169544in}}%
\pgfpathcurveto{\pgfqpoint{2.325327in}{3.163720in}}{\pgfqpoint{2.333227in}{3.160447in}}{\pgfqpoint{2.341463in}{3.160447in}}%
\pgfpathclose%
\pgfusepath{stroke,fill}%
\end{pgfscope}%
\begin{pgfscope}%
\pgfpathrectangle{\pgfqpoint{0.100000in}{0.220728in}}{\pgfqpoint{3.696000in}{3.696000in}}%
\pgfusepath{clip}%
\pgfsetbuttcap%
\pgfsetroundjoin%
\definecolor{currentfill}{rgb}{0.121569,0.466667,0.705882}%
\pgfsetfillcolor{currentfill}%
\pgfsetfillopacity{0.430725}%
\pgfsetlinewidth{1.003750pt}%
\definecolor{currentstroke}{rgb}{0.121569,0.466667,0.705882}%
\pgfsetstrokecolor{currentstroke}%
\pgfsetstrokeopacity{0.430725}%
\pgfsetdash{}{0pt}%
\pgfpathmoveto{\pgfqpoint{2.342592in}{3.160210in}}%
\pgfpathcurveto{\pgfqpoint{2.350828in}{3.160210in}}{\pgfqpoint{2.358728in}{3.163483in}}{\pgfqpoint{2.364552in}{3.169307in}}%
\pgfpathcurveto{\pgfqpoint{2.370376in}{3.175130in}}{\pgfqpoint{2.373649in}{3.183031in}}{\pgfqpoint{2.373649in}{3.191267in}}%
\pgfpathcurveto{\pgfqpoint{2.373649in}{3.199503in}}{\pgfqpoint{2.370376in}{3.207403in}}{\pgfqpoint{2.364552in}{3.213227in}}%
\pgfpathcurveto{\pgfqpoint{2.358728in}{3.219051in}}{\pgfqpoint{2.350828in}{3.222323in}}{\pgfqpoint{2.342592in}{3.222323in}}%
\pgfpathcurveto{\pgfqpoint{2.334356in}{3.222323in}}{\pgfqpoint{2.326456in}{3.219051in}}{\pgfqpoint{2.320632in}{3.213227in}}%
\pgfpathcurveto{\pgfqpoint{2.314808in}{3.207403in}}{\pgfqpoint{2.311536in}{3.199503in}}{\pgfqpoint{2.311536in}{3.191267in}}%
\pgfpathcurveto{\pgfqpoint{2.311536in}{3.183031in}}{\pgfqpoint{2.314808in}{3.175130in}}{\pgfqpoint{2.320632in}{3.169307in}}%
\pgfpathcurveto{\pgfqpoint{2.326456in}{3.163483in}}{\pgfqpoint{2.334356in}{3.160210in}}{\pgfqpoint{2.342592in}{3.160210in}}%
\pgfpathclose%
\pgfusepath{stroke,fill}%
\end{pgfscope}%
\begin{pgfscope}%
\pgfpathrectangle{\pgfqpoint{0.100000in}{0.220728in}}{\pgfqpoint{3.696000in}{3.696000in}}%
\pgfusepath{clip}%
\pgfsetbuttcap%
\pgfsetroundjoin%
\definecolor{currentfill}{rgb}{0.121569,0.466667,0.705882}%
\pgfsetfillcolor{currentfill}%
\pgfsetfillopacity{0.431506}%
\pgfsetlinewidth{1.003750pt}%
\definecolor{currentstroke}{rgb}{0.121569,0.466667,0.705882}%
\pgfsetstrokecolor{currentstroke}%
\pgfsetstrokeopacity{0.431506}%
\pgfsetdash{}{0pt}%
\pgfpathmoveto{\pgfqpoint{2.345450in}{3.159685in}}%
\pgfpathcurveto{\pgfqpoint{2.353686in}{3.159685in}}{\pgfqpoint{2.361586in}{3.162957in}}{\pgfqpoint{2.367410in}{3.168781in}}%
\pgfpathcurveto{\pgfqpoint{2.373234in}{3.174605in}}{\pgfqpoint{2.376506in}{3.182505in}}{\pgfqpoint{2.376506in}{3.190741in}}%
\pgfpathcurveto{\pgfqpoint{2.376506in}{3.198978in}}{\pgfqpoint{2.373234in}{3.206878in}}{\pgfqpoint{2.367410in}{3.212702in}}%
\pgfpathcurveto{\pgfqpoint{2.361586in}{3.218526in}}{\pgfqpoint{2.353686in}{3.221798in}}{\pgfqpoint{2.345450in}{3.221798in}}%
\pgfpathcurveto{\pgfqpoint{2.337214in}{3.221798in}}{\pgfqpoint{2.329314in}{3.218526in}}{\pgfqpoint{2.323490in}{3.212702in}}%
\pgfpathcurveto{\pgfqpoint{2.317666in}{3.206878in}}{\pgfqpoint{2.314393in}{3.198978in}}{\pgfqpoint{2.314393in}{3.190741in}}%
\pgfpathcurveto{\pgfqpoint{2.314393in}{3.182505in}}{\pgfqpoint{2.317666in}{3.174605in}}{\pgfqpoint{2.323490in}{3.168781in}}%
\pgfpathcurveto{\pgfqpoint{2.329314in}{3.162957in}}{\pgfqpoint{2.337214in}{3.159685in}}{\pgfqpoint{2.345450in}{3.159685in}}%
\pgfpathclose%
\pgfusepath{stroke,fill}%
\end{pgfscope}%
\begin{pgfscope}%
\pgfpathrectangle{\pgfqpoint{0.100000in}{0.220728in}}{\pgfqpoint{3.696000in}{3.696000in}}%
\pgfusepath{clip}%
\pgfsetbuttcap%
\pgfsetroundjoin%
\definecolor{currentfill}{rgb}{0.121569,0.466667,0.705882}%
\pgfsetfillcolor{currentfill}%
\pgfsetfillopacity{0.432820}%
\pgfsetlinewidth{1.003750pt}%
\definecolor{currentstroke}{rgb}{0.121569,0.466667,0.705882}%
\pgfsetstrokecolor{currentstroke}%
\pgfsetstrokeopacity{0.432820}%
\pgfsetdash{}{0pt}%
\pgfpathmoveto{\pgfqpoint{1.401766in}{2.416501in}}%
\pgfpathcurveto{\pgfqpoint{1.410002in}{2.416501in}}{\pgfqpoint{1.417903in}{2.419773in}}{\pgfqpoint{1.423726in}{2.425597in}}%
\pgfpathcurveto{\pgfqpoint{1.429550in}{2.431421in}}{\pgfqpoint{1.432823in}{2.439321in}}{\pgfqpoint{1.432823in}{2.447557in}}%
\pgfpathcurveto{\pgfqpoint{1.432823in}{2.455794in}}{\pgfqpoint{1.429550in}{2.463694in}}{\pgfqpoint{1.423726in}{2.469518in}}%
\pgfpathcurveto{\pgfqpoint{1.417903in}{2.475342in}}{\pgfqpoint{1.410002in}{2.478614in}}{\pgfqpoint{1.401766in}{2.478614in}}%
\pgfpathcurveto{\pgfqpoint{1.393530in}{2.478614in}}{\pgfqpoint{1.385630in}{2.475342in}}{\pgfqpoint{1.379806in}{2.469518in}}%
\pgfpathcurveto{\pgfqpoint{1.373982in}{2.463694in}}{\pgfqpoint{1.370710in}{2.455794in}}{\pgfqpoint{1.370710in}{2.447557in}}%
\pgfpathcurveto{\pgfqpoint{1.370710in}{2.439321in}}{\pgfqpoint{1.373982in}{2.431421in}}{\pgfqpoint{1.379806in}{2.425597in}}%
\pgfpathcurveto{\pgfqpoint{1.385630in}{2.419773in}}{\pgfqpoint{1.393530in}{2.416501in}}{\pgfqpoint{1.401766in}{2.416501in}}%
\pgfpathclose%
\pgfusepath{stroke,fill}%
\end{pgfscope}%
\begin{pgfscope}%
\pgfpathrectangle{\pgfqpoint{0.100000in}{0.220728in}}{\pgfqpoint{3.696000in}{3.696000in}}%
\pgfusepath{clip}%
\pgfsetbuttcap%
\pgfsetroundjoin%
\definecolor{currentfill}{rgb}{0.121569,0.466667,0.705882}%
\pgfsetfillcolor{currentfill}%
\pgfsetfillopacity{0.433170}%
\pgfsetlinewidth{1.003750pt}%
\definecolor{currentstroke}{rgb}{0.121569,0.466667,0.705882}%
\pgfsetstrokecolor{currentstroke}%
\pgfsetstrokeopacity{0.433170}%
\pgfsetdash{}{0pt}%
\pgfpathmoveto{\pgfqpoint{2.349402in}{3.159020in}}%
\pgfpathcurveto{\pgfqpoint{2.357639in}{3.159020in}}{\pgfqpoint{2.365539in}{3.162292in}}{\pgfqpoint{2.371363in}{3.168116in}}%
\pgfpathcurveto{\pgfqpoint{2.377187in}{3.173940in}}{\pgfqpoint{2.380459in}{3.181840in}}{\pgfqpoint{2.380459in}{3.190077in}}%
\pgfpathcurveto{\pgfqpoint{2.380459in}{3.198313in}}{\pgfqpoint{2.377187in}{3.206213in}}{\pgfqpoint{2.371363in}{3.212037in}}%
\pgfpathcurveto{\pgfqpoint{2.365539in}{3.217861in}}{\pgfqpoint{2.357639in}{3.221133in}}{\pgfqpoint{2.349402in}{3.221133in}}%
\pgfpathcurveto{\pgfqpoint{2.341166in}{3.221133in}}{\pgfqpoint{2.333266in}{3.217861in}}{\pgfqpoint{2.327442in}{3.212037in}}%
\pgfpathcurveto{\pgfqpoint{2.321618in}{3.206213in}}{\pgfqpoint{2.318346in}{3.198313in}}{\pgfqpoint{2.318346in}{3.190077in}}%
\pgfpathcurveto{\pgfqpoint{2.318346in}{3.181840in}}{\pgfqpoint{2.321618in}{3.173940in}}{\pgfqpoint{2.327442in}{3.168116in}}%
\pgfpathcurveto{\pgfqpoint{2.333266in}{3.162292in}}{\pgfqpoint{2.341166in}{3.159020in}}{\pgfqpoint{2.349402in}{3.159020in}}%
\pgfpathclose%
\pgfusepath{stroke,fill}%
\end{pgfscope}%
\begin{pgfscope}%
\pgfpathrectangle{\pgfqpoint{0.100000in}{0.220728in}}{\pgfqpoint{3.696000in}{3.696000in}}%
\pgfusepath{clip}%
\pgfsetbuttcap%
\pgfsetroundjoin%
\definecolor{currentfill}{rgb}{0.121569,0.466667,0.705882}%
\pgfsetfillcolor{currentfill}%
\pgfsetfillopacity{0.433617}%
\pgfsetlinewidth{1.003750pt}%
\definecolor{currentstroke}{rgb}{0.121569,0.466667,0.705882}%
\pgfsetstrokecolor{currentstroke}%
\pgfsetstrokeopacity{0.433617}%
\pgfsetdash{}{0pt}%
\pgfpathmoveto{\pgfqpoint{2.352163in}{3.158385in}}%
\pgfpathcurveto{\pgfqpoint{2.360399in}{3.158385in}}{\pgfqpoint{2.368299in}{3.161658in}}{\pgfqpoint{2.374123in}{3.167481in}}%
\pgfpathcurveto{\pgfqpoint{2.379947in}{3.173305in}}{\pgfqpoint{2.383219in}{3.181205in}}{\pgfqpoint{2.383219in}{3.189442in}}%
\pgfpathcurveto{\pgfqpoint{2.383219in}{3.197678in}}{\pgfqpoint{2.379947in}{3.205578in}}{\pgfqpoint{2.374123in}{3.211402in}}%
\pgfpathcurveto{\pgfqpoint{2.368299in}{3.217226in}}{\pgfqpoint{2.360399in}{3.220498in}}{\pgfqpoint{2.352163in}{3.220498in}}%
\pgfpathcurveto{\pgfqpoint{2.343926in}{3.220498in}}{\pgfqpoint{2.336026in}{3.217226in}}{\pgfqpoint{2.330202in}{3.211402in}}%
\pgfpathcurveto{\pgfqpoint{2.324378in}{3.205578in}}{\pgfqpoint{2.321106in}{3.197678in}}{\pgfqpoint{2.321106in}{3.189442in}}%
\pgfpathcurveto{\pgfqpoint{2.321106in}{3.181205in}}{\pgfqpoint{2.324378in}{3.173305in}}{\pgfqpoint{2.330202in}{3.167481in}}%
\pgfpathcurveto{\pgfqpoint{2.336026in}{3.161658in}}{\pgfqpoint{2.343926in}{3.158385in}}{\pgfqpoint{2.352163in}{3.158385in}}%
\pgfpathclose%
\pgfusepath{stroke,fill}%
\end{pgfscope}%
\begin{pgfscope}%
\pgfpathrectangle{\pgfqpoint{0.100000in}{0.220728in}}{\pgfqpoint{3.696000in}{3.696000in}}%
\pgfusepath{clip}%
\pgfsetbuttcap%
\pgfsetroundjoin%
\definecolor{currentfill}{rgb}{0.121569,0.466667,0.705882}%
\pgfsetfillcolor{currentfill}%
\pgfsetfillopacity{0.434656}%
\pgfsetlinewidth{1.003750pt}%
\definecolor{currentstroke}{rgb}{0.121569,0.466667,0.705882}%
\pgfsetstrokecolor{currentstroke}%
\pgfsetstrokeopacity{0.434656}%
\pgfsetdash{}{0pt}%
\pgfpathmoveto{\pgfqpoint{2.356436in}{3.157328in}}%
\pgfpathcurveto{\pgfqpoint{2.364672in}{3.157328in}}{\pgfqpoint{2.372572in}{3.160600in}}{\pgfqpoint{2.378396in}{3.166424in}}%
\pgfpathcurveto{\pgfqpoint{2.384220in}{3.172248in}}{\pgfqpoint{2.387492in}{3.180148in}}{\pgfqpoint{2.387492in}{3.188385in}}%
\pgfpathcurveto{\pgfqpoint{2.387492in}{3.196621in}}{\pgfqpoint{2.384220in}{3.204521in}}{\pgfqpoint{2.378396in}{3.210345in}}%
\pgfpathcurveto{\pgfqpoint{2.372572in}{3.216169in}}{\pgfqpoint{2.364672in}{3.219441in}}{\pgfqpoint{2.356436in}{3.219441in}}%
\pgfpathcurveto{\pgfqpoint{2.348199in}{3.219441in}}{\pgfqpoint{2.340299in}{3.216169in}}{\pgfqpoint{2.334475in}{3.210345in}}%
\pgfpathcurveto{\pgfqpoint{2.328652in}{3.204521in}}{\pgfqpoint{2.325379in}{3.196621in}}{\pgfqpoint{2.325379in}{3.188385in}}%
\pgfpathcurveto{\pgfqpoint{2.325379in}{3.180148in}}{\pgfqpoint{2.328652in}{3.172248in}}{\pgfqpoint{2.334475in}{3.166424in}}%
\pgfpathcurveto{\pgfqpoint{2.340299in}{3.160600in}}{\pgfqpoint{2.348199in}{3.157328in}}{\pgfqpoint{2.356436in}{3.157328in}}%
\pgfpathclose%
\pgfusepath{stroke,fill}%
\end{pgfscope}%
\begin{pgfscope}%
\pgfpathrectangle{\pgfqpoint{0.100000in}{0.220728in}}{\pgfqpoint{3.696000in}{3.696000in}}%
\pgfusepath{clip}%
\pgfsetbuttcap%
\pgfsetroundjoin%
\definecolor{currentfill}{rgb}{0.121569,0.466667,0.705882}%
\pgfsetfillcolor{currentfill}%
\pgfsetfillopacity{0.436239}%
\pgfsetlinewidth{1.003750pt}%
\definecolor{currentstroke}{rgb}{0.121569,0.466667,0.705882}%
\pgfsetstrokecolor{currentstroke}%
\pgfsetstrokeopacity{0.436239}%
\pgfsetdash{}{0pt}%
\pgfpathmoveto{\pgfqpoint{2.362173in}{3.155511in}}%
\pgfpathcurveto{\pgfqpoint{2.370409in}{3.155511in}}{\pgfqpoint{2.378309in}{3.158783in}}{\pgfqpoint{2.384133in}{3.164607in}}%
\pgfpathcurveto{\pgfqpoint{2.389957in}{3.170431in}}{\pgfqpoint{2.393229in}{3.178331in}}{\pgfqpoint{2.393229in}{3.186568in}}%
\pgfpathcurveto{\pgfqpoint{2.393229in}{3.194804in}}{\pgfqpoint{2.389957in}{3.202704in}}{\pgfqpoint{2.384133in}{3.208528in}}%
\pgfpathcurveto{\pgfqpoint{2.378309in}{3.214352in}}{\pgfqpoint{2.370409in}{3.217624in}}{\pgfqpoint{2.362173in}{3.217624in}}%
\pgfpathcurveto{\pgfqpoint{2.353936in}{3.217624in}}{\pgfqpoint{2.346036in}{3.214352in}}{\pgfqpoint{2.340212in}{3.208528in}}%
\pgfpathcurveto{\pgfqpoint{2.334388in}{3.202704in}}{\pgfqpoint{2.331116in}{3.194804in}}{\pgfqpoint{2.331116in}{3.186568in}}%
\pgfpathcurveto{\pgfqpoint{2.331116in}{3.178331in}}{\pgfqpoint{2.334388in}{3.170431in}}{\pgfqpoint{2.340212in}{3.164607in}}%
\pgfpathcurveto{\pgfqpoint{2.346036in}{3.158783in}}{\pgfqpoint{2.353936in}{3.155511in}}{\pgfqpoint{2.362173in}{3.155511in}}%
\pgfpathclose%
\pgfusepath{stroke,fill}%
\end{pgfscope}%
\begin{pgfscope}%
\pgfpathrectangle{\pgfqpoint{0.100000in}{0.220728in}}{\pgfqpoint{3.696000in}{3.696000in}}%
\pgfusepath{clip}%
\pgfsetbuttcap%
\pgfsetroundjoin%
\definecolor{currentfill}{rgb}{0.121569,0.466667,0.705882}%
\pgfsetfillcolor{currentfill}%
\pgfsetfillopacity{0.436602}%
\pgfsetlinewidth{1.003750pt}%
\definecolor{currentstroke}{rgb}{0.121569,0.466667,0.705882}%
\pgfsetstrokecolor{currentstroke}%
\pgfsetstrokeopacity{0.436602}%
\pgfsetdash{}{0pt}%
\pgfpathmoveto{\pgfqpoint{1.387479in}{2.393276in}}%
\pgfpathcurveto{\pgfqpoint{1.395715in}{2.393276in}}{\pgfqpoint{1.403615in}{2.396548in}}{\pgfqpoint{1.409439in}{2.402372in}}%
\pgfpathcurveto{\pgfqpoint{1.415263in}{2.408196in}}{\pgfqpoint{1.418535in}{2.416096in}}{\pgfqpoint{1.418535in}{2.424332in}}%
\pgfpathcurveto{\pgfqpoint{1.418535in}{2.432568in}}{\pgfqpoint{1.415263in}{2.440468in}}{\pgfqpoint{1.409439in}{2.446292in}}%
\pgfpathcurveto{\pgfqpoint{1.403615in}{2.452116in}}{\pgfqpoint{1.395715in}{2.455389in}}{\pgfqpoint{1.387479in}{2.455389in}}%
\pgfpathcurveto{\pgfqpoint{1.379243in}{2.455389in}}{\pgfqpoint{1.371343in}{2.452116in}}{\pgfqpoint{1.365519in}{2.446292in}}%
\pgfpathcurveto{\pgfqpoint{1.359695in}{2.440468in}}{\pgfqpoint{1.356422in}{2.432568in}}{\pgfqpoint{1.356422in}{2.424332in}}%
\pgfpathcurveto{\pgfqpoint{1.356422in}{2.416096in}}{\pgfqpoint{1.359695in}{2.408196in}}{\pgfqpoint{1.365519in}{2.402372in}}%
\pgfpathcurveto{\pgfqpoint{1.371343in}{2.396548in}}{\pgfqpoint{1.379243in}{2.393276in}}{\pgfqpoint{1.387479in}{2.393276in}}%
\pgfpathclose%
\pgfusepath{stroke,fill}%
\end{pgfscope}%
\begin{pgfscope}%
\pgfpathrectangle{\pgfqpoint{0.100000in}{0.220728in}}{\pgfqpoint{3.696000in}{3.696000in}}%
\pgfusepath{clip}%
\pgfsetbuttcap%
\pgfsetroundjoin%
\definecolor{currentfill}{rgb}{0.121569,0.466667,0.705882}%
\pgfsetfillcolor{currentfill}%
\pgfsetfillopacity{0.437649}%
\pgfsetlinewidth{1.003750pt}%
\definecolor{currentstroke}{rgb}{0.121569,0.466667,0.705882}%
\pgfsetstrokecolor{currentstroke}%
\pgfsetstrokeopacity{0.437649}%
\pgfsetdash{}{0pt}%
\pgfpathmoveto{\pgfqpoint{2.369242in}{3.153247in}}%
\pgfpathcurveto{\pgfqpoint{2.377478in}{3.153247in}}{\pgfqpoint{2.385378in}{3.156519in}}{\pgfqpoint{2.391202in}{3.162343in}}%
\pgfpathcurveto{\pgfqpoint{2.397026in}{3.168167in}}{\pgfqpoint{2.400299in}{3.176067in}}{\pgfqpoint{2.400299in}{3.184304in}}%
\pgfpathcurveto{\pgfqpoint{2.400299in}{3.192540in}}{\pgfqpoint{2.397026in}{3.200440in}}{\pgfqpoint{2.391202in}{3.206264in}}%
\pgfpathcurveto{\pgfqpoint{2.385378in}{3.212088in}}{\pgfqpoint{2.377478in}{3.215360in}}{\pgfqpoint{2.369242in}{3.215360in}}%
\pgfpathcurveto{\pgfqpoint{2.361006in}{3.215360in}}{\pgfqpoint{2.353106in}{3.212088in}}{\pgfqpoint{2.347282in}{3.206264in}}%
\pgfpathcurveto{\pgfqpoint{2.341458in}{3.200440in}}{\pgfqpoint{2.338186in}{3.192540in}}{\pgfqpoint{2.338186in}{3.184304in}}%
\pgfpathcurveto{\pgfqpoint{2.338186in}{3.176067in}}{\pgfqpoint{2.341458in}{3.168167in}}{\pgfqpoint{2.347282in}{3.162343in}}%
\pgfpathcurveto{\pgfqpoint{2.353106in}{3.156519in}}{\pgfqpoint{2.361006in}{3.153247in}}{\pgfqpoint{2.369242in}{3.153247in}}%
\pgfpathclose%
\pgfusepath{stroke,fill}%
\end{pgfscope}%
\begin{pgfscope}%
\pgfpathrectangle{\pgfqpoint{0.100000in}{0.220728in}}{\pgfqpoint{3.696000in}{3.696000in}}%
\pgfusepath{clip}%
\pgfsetbuttcap%
\pgfsetroundjoin%
\definecolor{currentfill}{rgb}{0.121569,0.466667,0.705882}%
\pgfsetfillcolor{currentfill}%
\pgfsetfillopacity{0.440022}%
\pgfsetlinewidth{1.003750pt}%
\definecolor{currentstroke}{rgb}{0.121569,0.466667,0.705882}%
\pgfsetstrokecolor{currentstroke}%
\pgfsetstrokeopacity{0.440022}%
\pgfsetdash{}{0pt}%
\pgfpathmoveto{\pgfqpoint{2.376158in}{3.152088in}}%
\pgfpathcurveto{\pgfqpoint{2.384394in}{3.152088in}}{\pgfqpoint{2.392294in}{3.155360in}}{\pgfqpoint{2.398118in}{3.161184in}}%
\pgfpathcurveto{\pgfqpoint{2.403942in}{3.167008in}}{\pgfqpoint{2.407215in}{3.174908in}}{\pgfqpoint{2.407215in}{3.183144in}}%
\pgfpathcurveto{\pgfqpoint{2.407215in}{3.191380in}}{\pgfqpoint{2.403942in}{3.199280in}}{\pgfqpoint{2.398118in}{3.205104in}}%
\pgfpathcurveto{\pgfqpoint{2.392294in}{3.210928in}}{\pgfqpoint{2.384394in}{3.214201in}}{\pgfqpoint{2.376158in}{3.214201in}}%
\pgfpathcurveto{\pgfqpoint{2.367922in}{3.214201in}}{\pgfqpoint{2.360022in}{3.210928in}}{\pgfqpoint{2.354198in}{3.205104in}}%
\pgfpathcurveto{\pgfqpoint{2.348374in}{3.199280in}}{\pgfqpoint{2.345102in}{3.191380in}}{\pgfqpoint{2.345102in}{3.183144in}}%
\pgfpathcurveto{\pgfqpoint{2.345102in}{3.174908in}}{\pgfqpoint{2.348374in}{3.167008in}}{\pgfqpoint{2.354198in}{3.161184in}}%
\pgfpathcurveto{\pgfqpoint{2.360022in}{3.155360in}}{\pgfqpoint{2.367922in}{3.152088in}}{\pgfqpoint{2.376158in}{3.152088in}}%
\pgfpathclose%
\pgfusepath{stroke,fill}%
\end{pgfscope}%
\begin{pgfscope}%
\pgfpathrectangle{\pgfqpoint{0.100000in}{0.220728in}}{\pgfqpoint{3.696000in}{3.696000in}}%
\pgfusepath{clip}%
\pgfsetbuttcap%
\pgfsetroundjoin%
\definecolor{currentfill}{rgb}{0.121569,0.466667,0.705882}%
\pgfsetfillcolor{currentfill}%
\pgfsetfillopacity{0.440623}%
\pgfsetlinewidth{1.003750pt}%
\definecolor{currentstroke}{rgb}{0.121569,0.466667,0.705882}%
\pgfsetstrokecolor{currentstroke}%
\pgfsetstrokeopacity{0.440623}%
\pgfsetdash{}{0pt}%
\pgfpathmoveto{\pgfqpoint{1.376013in}{2.370119in}}%
\pgfpathcurveto{\pgfqpoint{1.384249in}{2.370119in}}{\pgfqpoint{1.392149in}{2.373391in}}{\pgfqpoint{1.397973in}{2.379215in}}%
\pgfpathcurveto{\pgfqpoint{1.403797in}{2.385039in}}{\pgfqpoint{1.407069in}{2.392939in}}{\pgfqpoint{1.407069in}{2.401175in}}%
\pgfpathcurveto{\pgfqpoint{1.407069in}{2.409411in}}{\pgfqpoint{1.403797in}{2.417311in}}{\pgfqpoint{1.397973in}{2.423135in}}%
\pgfpathcurveto{\pgfqpoint{1.392149in}{2.428959in}}{\pgfqpoint{1.384249in}{2.432232in}}{\pgfqpoint{1.376013in}{2.432232in}}%
\pgfpathcurveto{\pgfqpoint{1.367776in}{2.432232in}}{\pgfqpoint{1.359876in}{2.428959in}}{\pgfqpoint{1.354052in}{2.423135in}}%
\pgfpathcurveto{\pgfqpoint{1.348229in}{2.417311in}}{\pgfqpoint{1.344956in}{2.409411in}}{\pgfqpoint{1.344956in}{2.401175in}}%
\pgfpathcurveto{\pgfqpoint{1.344956in}{2.392939in}}{\pgfqpoint{1.348229in}{2.385039in}}{\pgfqpoint{1.354052in}{2.379215in}}%
\pgfpathcurveto{\pgfqpoint{1.359876in}{2.373391in}}{\pgfqpoint{1.367776in}{2.370119in}}{\pgfqpoint{1.376013in}{2.370119in}}%
\pgfpathclose%
\pgfusepath{stroke,fill}%
\end{pgfscope}%
\begin{pgfscope}%
\pgfpathrectangle{\pgfqpoint{0.100000in}{0.220728in}}{\pgfqpoint{3.696000in}{3.696000in}}%
\pgfusepath{clip}%
\pgfsetbuttcap%
\pgfsetroundjoin%
\definecolor{currentfill}{rgb}{0.121569,0.466667,0.705882}%
\pgfsetfillcolor{currentfill}%
\pgfsetfillopacity{0.441437}%
\pgfsetlinewidth{1.003750pt}%
\definecolor{currentstroke}{rgb}{0.121569,0.466667,0.705882}%
\pgfsetstrokecolor{currentstroke}%
\pgfsetstrokeopacity{0.441437}%
\pgfsetdash{}{0pt}%
\pgfpathmoveto{\pgfqpoint{2.385028in}{3.148604in}}%
\pgfpathcurveto{\pgfqpoint{2.393265in}{3.148604in}}{\pgfqpoint{2.401165in}{3.151876in}}{\pgfqpoint{2.406989in}{3.157700in}}%
\pgfpathcurveto{\pgfqpoint{2.412813in}{3.163524in}}{\pgfqpoint{2.416085in}{3.171424in}}{\pgfqpoint{2.416085in}{3.179660in}}%
\pgfpathcurveto{\pgfqpoint{2.416085in}{3.187897in}}{\pgfqpoint{2.412813in}{3.195797in}}{\pgfqpoint{2.406989in}{3.201621in}}%
\pgfpathcurveto{\pgfqpoint{2.401165in}{3.207445in}}{\pgfqpoint{2.393265in}{3.210717in}}{\pgfqpoint{2.385028in}{3.210717in}}%
\pgfpathcurveto{\pgfqpoint{2.376792in}{3.210717in}}{\pgfqpoint{2.368892in}{3.207445in}}{\pgfqpoint{2.363068in}{3.201621in}}%
\pgfpathcurveto{\pgfqpoint{2.357244in}{3.195797in}}{\pgfqpoint{2.353972in}{3.187897in}}{\pgfqpoint{2.353972in}{3.179660in}}%
\pgfpathcurveto{\pgfqpoint{2.353972in}{3.171424in}}{\pgfqpoint{2.357244in}{3.163524in}}{\pgfqpoint{2.363068in}{3.157700in}}%
\pgfpathcurveto{\pgfqpoint{2.368892in}{3.151876in}}{\pgfqpoint{2.376792in}{3.148604in}}{\pgfqpoint{2.385028in}{3.148604in}}%
\pgfpathclose%
\pgfusepath{stroke,fill}%
\end{pgfscope}%
\begin{pgfscope}%
\pgfpathrectangle{\pgfqpoint{0.100000in}{0.220728in}}{\pgfqpoint{3.696000in}{3.696000in}}%
\pgfusepath{clip}%
\pgfsetbuttcap%
\pgfsetroundjoin%
\definecolor{currentfill}{rgb}{0.121569,0.466667,0.705882}%
\pgfsetfillcolor{currentfill}%
\pgfsetfillopacity{0.443535}%
\pgfsetlinewidth{1.003750pt}%
\definecolor{currentstroke}{rgb}{0.121569,0.466667,0.705882}%
\pgfsetstrokecolor{currentstroke}%
\pgfsetstrokeopacity{0.443535}%
\pgfsetdash{}{0pt}%
\pgfpathmoveto{\pgfqpoint{2.393823in}{3.145788in}}%
\pgfpathcurveto{\pgfqpoint{2.402059in}{3.145788in}}{\pgfqpoint{2.409959in}{3.149060in}}{\pgfqpoint{2.415783in}{3.154884in}}%
\pgfpathcurveto{\pgfqpoint{2.421607in}{3.160708in}}{\pgfqpoint{2.424879in}{3.168608in}}{\pgfqpoint{2.424879in}{3.176844in}}%
\pgfpathcurveto{\pgfqpoint{2.424879in}{3.185080in}}{\pgfqpoint{2.421607in}{3.192981in}}{\pgfqpoint{2.415783in}{3.198804in}}%
\pgfpathcurveto{\pgfqpoint{2.409959in}{3.204628in}}{\pgfqpoint{2.402059in}{3.207901in}}{\pgfqpoint{2.393823in}{3.207901in}}%
\pgfpathcurveto{\pgfqpoint{2.385586in}{3.207901in}}{\pgfqpoint{2.377686in}{3.204628in}}{\pgfqpoint{2.371862in}{3.198804in}}%
\pgfpathcurveto{\pgfqpoint{2.366038in}{3.192981in}}{\pgfqpoint{2.362766in}{3.185080in}}{\pgfqpoint{2.362766in}{3.176844in}}%
\pgfpathcurveto{\pgfqpoint{2.362766in}{3.168608in}}{\pgfqpoint{2.366038in}{3.160708in}}{\pgfqpoint{2.371862in}{3.154884in}}%
\pgfpathcurveto{\pgfqpoint{2.377686in}{3.149060in}}{\pgfqpoint{2.385586in}{3.145788in}}{\pgfqpoint{2.393823in}{3.145788in}}%
\pgfpathclose%
\pgfusepath{stroke,fill}%
\end{pgfscope}%
\begin{pgfscope}%
\pgfpathrectangle{\pgfqpoint{0.100000in}{0.220728in}}{\pgfqpoint{3.696000in}{3.696000in}}%
\pgfusepath{clip}%
\pgfsetbuttcap%
\pgfsetroundjoin%
\definecolor{currentfill}{rgb}{0.121569,0.466667,0.705882}%
\pgfsetfillcolor{currentfill}%
\pgfsetfillopacity{0.444759}%
\pgfsetlinewidth{1.003750pt}%
\definecolor{currentstroke}{rgb}{0.121569,0.466667,0.705882}%
\pgfsetstrokecolor{currentstroke}%
\pgfsetstrokeopacity{0.444759}%
\pgfsetdash{}{0pt}%
\pgfpathmoveto{\pgfqpoint{1.367825in}{2.345931in}}%
\pgfpathcurveto{\pgfqpoint{1.376062in}{2.345931in}}{\pgfqpoint{1.383962in}{2.349203in}}{\pgfqpoint{1.389786in}{2.355027in}}%
\pgfpathcurveto{\pgfqpoint{1.395610in}{2.360851in}}{\pgfqpoint{1.398882in}{2.368751in}}{\pgfqpoint{1.398882in}{2.376987in}}%
\pgfpathcurveto{\pgfqpoint{1.398882in}{2.385223in}}{\pgfqpoint{1.395610in}{2.393123in}}{\pgfqpoint{1.389786in}{2.398947in}}%
\pgfpathcurveto{\pgfqpoint{1.383962in}{2.404771in}}{\pgfqpoint{1.376062in}{2.408044in}}{\pgfqpoint{1.367825in}{2.408044in}}%
\pgfpathcurveto{\pgfqpoint{1.359589in}{2.408044in}}{\pgfqpoint{1.351689in}{2.404771in}}{\pgfqpoint{1.345865in}{2.398947in}}%
\pgfpathcurveto{\pgfqpoint{1.340041in}{2.393123in}}{\pgfqpoint{1.336769in}{2.385223in}}{\pgfqpoint{1.336769in}{2.376987in}}%
\pgfpathcurveto{\pgfqpoint{1.336769in}{2.368751in}}{\pgfqpoint{1.340041in}{2.360851in}}{\pgfqpoint{1.345865in}{2.355027in}}%
\pgfpathcurveto{\pgfqpoint{1.351689in}{2.349203in}}{\pgfqpoint{1.359589in}{2.345931in}}{\pgfqpoint{1.367825in}{2.345931in}}%
\pgfpathclose%
\pgfusepath{stroke,fill}%
\end{pgfscope}%
\begin{pgfscope}%
\pgfpathrectangle{\pgfqpoint{0.100000in}{0.220728in}}{\pgfqpoint{3.696000in}{3.696000in}}%
\pgfusepath{clip}%
\pgfsetbuttcap%
\pgfsetroundjoin%
\definecolor{currentfill}{rgb}{0.121569,0.466667,0.705882}%
\pgfsetfillcolor{currentfill}%
\pgfsetfillopacity{0.446278}%
\pgfsetlinewidth{1.003750pt}%
\definecolor{currentstroke}{rgb}{0.121569,0.466667,0.705882}%
\pgfsetstrokecolor{currentstroke}%
\pgfsetstrokeopacity{0.446278}%
\pgfsetdash{}{0pt}%
\pgfpathmoveto{\pgfqpoint{2.404820in}{3.144115in}}%
\pgfpathcurveto{\pgfqpoint{2.413057in}{3.144115in}}{\pgfqpoint{2.420957in}{3.147388in}}{\pgfqpoint{2.426781in}{3.153212in}}%
\pgfpathcurveto{\pgfqpoint{2.432604in}{3.159036in}}{\pgfqpoint{2.435877in}{3.166936in}}{\pgfqpoint{2.435877in}{3.175172in}}%
\pgfpathcurveto{\pgfqpoint{2.435877in}{3.183408in}}{\pgfqpoint{2.432604in}{3.191308in}}{\pgfqpoint{2.426781in}{3.197132in}}%
\pgfpathcurveto{\pgfqpoint{2.420957in}{3.202956in}}{\pgfqpoint{2.413057in}{3.206228in}}{\pgfqpoint{2.404820in}{3.206228in}}%
\pgfpathcurveto{\pgfqpoint{2.396584in}{3.206228in}}{\pgfqpoint{2.388684in}{3.202956in}}{\pgfqpoint{2.382860in}{3.197132in}}%
\pgfpathcurveto{\pgfqpoint{2.377036in}{3.191308in}}{\pgfqpoint{2.373764in}{3.183408in}}{\pgfqpoint{2.373764in}{3.175172in}}%
\pgfpathcurveto{\pgfqpoint{2.373764in}{3.166936in}}{\pgfqpoint{2.377036in}{3.159036in}}{\pgfqpoint{2.382860in}{3.153212in}}%
\pgfpathcurveto{\pgfqpoint{2.388684in}{3.147388in}}{\pgfqpoint{2.396584in}{3.144115in}}{\pgfqpoint{2.404820in}{3.144115in}}%
\pgfpathclose%
\pgfusepath{stroke,fill}%
\end{pgfscope}%
\begin{pgfscope}%
\pgfpathrectangle{\pgfqpoint{0.100000in}{0.220728in}}{\pgfqpoint{3.696000in}{3.696000in}}%
\pgfusepath{clip}%
\pgfsetbuttcap%
\pgfsetroundjoin%
\definecolor{currentfill}{rgb}{0.121569,0.466667,0.705882}%
\pgfsetfillcolor{currentfill}%
\pgfsetfillopacity{0.447232}%
\pgfsetlinewidth{1.003750pt}%
\definecolor{currentstroke}{rgb}{0.121569,0.466667,0.705882}%
\pgfsetstrokecolor{currentstroke}%
\pgfsetstrokeopacity{0.447232}%
\pgfsetdash{}{0pt}%
\pgfpathmoveto{\pgfqpoint{1.353544in}{2.326967in}}%
\pgfpathcurveto{\pgfqpoint{1.361780in}{2.326967in}}{\pgfqpoint{1.369680in}{2.330240in}}{\pgfqpoint{1.375504in}{2.336063in}}%
\pgfpathcurveto{\pgfqpoint{1.381328in}{2.341887in}}{\pgfqpoint{1.384600in}{2.349787in}}{\pgfqpoint{1.384600in}{2.358024in}}%
\pgfpathcurveto{\pgfqpoint{1.384600in}{2.366260in}}{\pgfqpoint{1.381328in}{2.374160in}}{\pgfqpoint{1.375504in}{2.379984in}}%
\pgfpathcurveto{\pgfqpoint{1.369680in}{2.385808in}}{\pgfqpoint{1.361780in}{2.389080in}}{\pgfqpoint{1.353544in}{2.389080in}}%
\pgfpathcurveto{\pgfqpoint{1.345307in}{2.389080in}}{\pgfqpoint{1.337407in}{2.385808in}}{\pgfqpoint{1.331583in}{2.379984in}}%
\pgfpathcurveto{\pgfqpoint{1.325760in}{2.374160in}}{\pgfqpoint{1.322487in}{2.366260in}}{\pgfqpoint{1.322487in}{2.358024in}}%
\pgfpathcurveto{\pgfqpoint{1.322487in}{2.349787in}}{\pgfqpoint{1.325760in}{2.341887in}}{\pgfqpoint{1.331583in}{2.336063in}}%
\pgfpathcurveto{\pgfqpoint{1.337407in}{2.330240in}}{\pgfqpoint{1.345307in}{2.326967in}}{\pgfqpoint{1.353544in}{2.326967in}}%
\pgfpathclose%
\pgfusepath{stroke,fill}%
\end{pgfscope}%
\begin{pgfscope}%
\pgfpathrectangle{\pgfqpoint{0.100000in}{0.220728in}}{\pgfqpoint{3.696000in}{3.696000in}}%
\pgfusepath{clip}%
\pgfsetbuttcap%
\pgfsetroundjoin%
\definecolor{currentfill}{rgb}{0.121569,0.466667,0.705882}%
\pgfsetfillcolor{currentfill}%
\pgfsetfillopacity{0.448547}%
\pgfsetlinewidth{1.003750pt}%
\definecolor{currentstroke}{rgb}{0.121569,0.466667,0.705882}%
\pgfsetstrokecolor{currentstroke}%
\pgfsetstrokeopacity{0.448547}%
\pgfsetdash{}{0pt}%
\pgfpathmoveto{\pgfqpoint{2.416605in}{3.140308in}}%
\pgfpathcurveto{\pgfqpoint{2.424842in}{3.140308in}}{\pgfqpoint{2.432742in}{3.143580in}}{\pgfqpoint{2.438566in}{3.149404in}}%
\pgfpathcurveto{\pgfqpoint{2.444390in}{3.155228in}}{\pgfqpoint{2.447662in}{3.163128in}}{\pgfqpoint{2.447662in}{3.171364in}}%
\pgfpathcurveto{\pgfqpoint{2.447662in}{3.179600in}}{\pgfqpoint{2.444390in}{3.187500in}}{\pgfqpoint{2.438566in}{3.193324in}}%
\pgfpathcurveto{\pgfqpoint{2.432742in}{3.199148in}}{\pgfqpoint{2.424842in}{3.202421in}}{\pgfqpoint{2.416605in}{3.202421in}}%
\pgfpathcurveto{\pgfqpoint{2.408369in}{3.202421in}}{\pgfqpoint{2.400469in}{3.199148in}}{\pgfqpoint{2.394645in}{3.193324in}}%
\pgfpathcurveto{\pgfqpoint{2.388821in}{3.187500in}}{\pgfqpoint{2.385549in}{3.179600in}}{\pgfqpoint{2.385549in}{3.171364in}}%
\pgfpathcurveto{\pgfqpoint{2.385549in}{3.163128in}}{\pgfqpoint{2.388821in}{3.155228in}}{\pgfqpoint{2.394645in}{3.149404in}}%
\pgfpathcurveto{\pgfqpoint{2.400469in}{3.143580in}}{\pgfqpoint{2.408369in}{3.140308in}}{\pgfqpoint{2.416605in}{3.140308in}}%
\pgfpathclose%
\pgfusepath{stroke,fill}%
\end{pgfscope}%
\begin{pgfscope}%
\pgfpathrectangle{\pgfqpoint{0.100000in}{0.220728in}}{\pgfqpoint{3.696000in}{3.696000in}}%
\pgfusepath{clip}%
\pgfsetbuttcap%
\pgfsetroundjoin%
\definecolor{currentfill}{rgb}{0.121569,0.466667,0.705882}%
\pgfsetfillcolor{currentfill}%
\pgfsetfillopacity{0.450714}%
\pgfsetlinewidth{1.003750pt}%
\definecolor{currentstroke}{rgb}{0.121569,0.466667,0.705882}%
\pgfsetstrokecolor{currentstroke}%
\pgfsetstrokeopacity{0.450714}%
\pgfsetdash{}{0pt}%
\pgfpathmoveto{\pgfqpoint{1.349960in}{2.304890in}}%
\pgfpathcurveto{\pgfqpoint{1.358196in}{2.304890in}}{\pgfqpoint{1.366096in}{2.308162in}}{\pgfqpoint{1.371920in}{2.313986in}}%
\pgfpathcurveto{\pgfqpoint{1.377744in}{2.319810in}}{\pgfqpoint{1.381017in}{2.327710in}}{\pgfqpoint{1.381017in}{2.335946in}}%
\pgfpathcurveto{\pgfqpoint{1.381017in}{2.344182in}}{\pgfqpoint{1.377744in}{2.352082in}}{\pgfqpoint{1.371920in}{2.357906in}}%
\pgfpathcurveto{\pgfqpoint{1.366096in}{2.363730in}}{\pgfqpoint{1.358196in}{2.367003in}}{\pgfqpoint{1.349960in}{2.367003in}}%
\pgfpathcurveto{\pgfqpoint{1.341724in}{2.367003in}}{\pgfqpoint{1.333824in}{2.363730in}}{\pgfqpoint{1.328000in}{2.357906in}}%
\pgfpathcurveto{\pgfqpoint{1.322176in}{2.352082in}}{\pgfqpoint{1.318904in}{2.344182in}}{\pgfqpoint{1.318904in}{2.335946in}}%
\pgfpathcurveto{\pgfqpoint{1.318904in}{2.327710in}}{\pgfqpoint{1.322176in}{2.319810in}}{\pgfqpoint{1.328000in}{2.313986in}}%
\pgfpathcurveto{\pgfqpoint{1.333824in}{2.308162in}}{\pgfqpoint{1.341724in}{2.304890in}}{\pgfqpoint{1.349960in}{2.304890in}}%
\pgfpathclose%
\pgfusepath{stroke,fill}%
\end{pgfscope}%
\begin{pgfscope}%
\pgfpathrectangle{\pgfqpoint{0.100000in}{0.220728in}}{\pgfqpoint{3.696000in}{3.696000in}}%
\pgfusepath{clip}%
\pgfsetbuttcap%
\pgfsetroundjoin%
\definecolor{currentfill}{rgb}{0.121569,0.466667,0.705882}%
\pgfsetfillcolor{currentfill}%
\pgfsetfillopacity{0.452164}%
\pgfsetlinewidth{1.003750pt}%
\definecolor{currentstroke}{rgb}{0.121569,0.466667,0.705882}%
\pgfsetstrokecolor{currentstroke}%
\pgfsetstrokeopacity{0.452164}%
\pgfsetdash{}{0pt}%
\pgfpathmoveto{\pgfqpoint{1.336983in}{2.290592in}}%
\pgfpathcurveto{\pgfqpoint{1.345220in}{2.290592in}}{\pgfqpoint{1.353120in}{2.293864in}}{\pgfqpoint{1.358944in}{2.299688in}}%
\pgfpathcurveto{\pgfqpoint{1.364768in}{2.305512in}}{\pgfqpoint{1.368040in}{2.313412in}}{\pgfqpoint{1.368040in}{2.321648in}}%
\pgfpathcurveto{\pgfqpoint{1.368040in}{2.329885in}}{\pgfqpoint{1.364768in}{2.337785in}}{\pgfqpoint{1.358944in}{2.343609in}}%
\pgfpathcurveto{\pgfqpoint{1.353120in}{2.349433in}}{\pgfqpoint{1.345220in}{2.352705in}}{\pgfqpoint{1.336983in}{2.352705in}}%
\pgfpathcurveto{\pgfqpoint{1.328747in}{2.352705in}}{\pgfqpoint{1.320847in}{2.349433in}}{\pgfqpoint{1.315023in}{2.343609in}}%
\pgfpathcurveto{\pgfqpoint{1.309199in}{2.337785in}}{\pgfqpoint{1.305927in}{2.329885in}}{\pgfqpoint{1.305927in}{2.321648in}}%
\pgfpathcurveto{\pgfqpoint{1.305927in}{2.313412in}}{\pgfqpoint{1.309199in}{2.305512in}}{\pgfqpoint{1.315023in}{2.299688in}}%
\pgfpathcurveto{\pgfqpoint{1.320847in}{2.293864in}}{\pgfqpoint{1.328747in}{2.290592in}}{\pgfqpoint{1.336983in}{2.290592in}}%
\pgfpathclose%
\pgfusepath{stroke,fill}%
\end{pgfscope}%
\begin{pgfscope}%
\pgfpathrectangle{\pgfqpoint{0.100000in}{0.220728in}}{\pgfqpoint{3.696000in}{3.696000in}}%
\pgfusepath{clip}%
\pgfsetbuttcap%
\pgfsetroundjoin%
\definecolor{currentfill}{rgb}{0.121569,0.466667,0.705882}%
\pgfsetfillcolor{currentfill}%
\pgfsetfillopacity{0.452193}%
\pgfsetlinewidth{1.003750pt}%
\definecolor{currentstroke}{rgb}{0.121569,0.466667,0.705882}%
\pgfsetstrokecolor{currentstroke}%
\pgfsetstrokeopacity{0.452193}%
\pgfsetdash{}{0pt}%
\pgfpathmoveto{\pgfqpoint{2.429387in}{3.135773in}}%
\pgfpathcurveto{\pgfqpoint{2.437623in}{3.135773in}}{\pgfqpoint{2.445523in}{3.139045in}}{\pgfqpoint{2.451347in}{3.144869in}}%
\pgfpathcurveto{\pgfqpoint{2.457171in}{3.150693in}}{\pgfqpoint{2.460443in}{3.158593in}}{\pgfqpoint{2.460443in}{3.166830in}}%
\pgfpathcurveto{\pgfqpoint{2.460443in}{3.175066in}}{\pgfqpoint{2.457171in}{3.182966in}}{\pgfqpoint{2.451347in}{3.188790in}}%
\pgfpathcurveto{\pgfqpoint{2.445523in}{3.194614in}}{\pgfqpoint{2.437623in}{3.197886in}}{\pgfqpoint{2.429387in}{3.197886in}}%
\pgfpathcurveto{\pgfqpoint{2.421151in}{3.197886in}}{\pgfqpoint{2.413251in}{3.194614in}}{\pgfqpoint{2.407427in}{3.188790in}}%
\pgfpathcurveto{\pgfqpoint{2.401603in}{3.182966in}}{\pgfqpoint{2.398330in}{3.175066in}}{\pgfqpoint{2.398330in}{3.166830in}}%
\pgfpathcurveto{\pgfqpoint{2.398330in}{3.158593in}}{\pgfqpoint{2.401603in}{3.150693in}}{\pgfqpoint{2.407427in}{3.144869in}}%
\pgfpathcurveto{\pgfqpoint{2.413251in}{3.139045in}}{\pgfqpoint{2.421151in}{3.135773in}}{\pgfqpoint{2.429387in}{3.135773in}}%
\pgfpathclose%
\pgfusepath{stroke,fill}%
\end{pgfscope}%
\begin{pgfscope}%
\pgfpathrectangle{\pgfqpoint{0.100000in}{0.220728in}}{\pgfqpoint{3.696000in}{3.696000in}}%
\pgfusepath{clip}%
\pgfsetbuttcap%
\pgfsetroundjoin%
\definecolor{currentfill}{rgb}{0.121569,0.466667,0.705882}%
\pgfsetfillcolor{currentfill}%
\pgfsetfillopacity{0.454237}%
\pgfsetlinewidth{1.003750pt}%
\definecolor{currentstroke}{rgb}{0.121569,0.466667,0.705882}%
\pgfsetstrokecolor{currentstroke}%
\pgfsetstrokeopacity{0.454237}%
\pgfsetdash{}{0pt}%
\pgfpathmoveto{\pgfqpoint{2.436605in}{3.133930in}}%
\pgfpathcurveto{\pgfqpoint{2.444842in}{3.133930in}}{\pgfqpoint{2.452742in}{3.137202in}}{\pgfqpoint{2.458566in}{3.143026in}}%
\pgfpathcurveto{\pgfqpoint{2.464390in}{3.148850in}}{\pgfqpoint{2.467662in}{3.156750in}}{\pgfqpoint{2.467662in}{3.164987in}}%
\pgfpathcurveto{\pgfqpoint{2.467662in}{3.173223in}}{\pgfqpoint{2.464390in}{3.181123in}}{\pgfqpoint{2.458566in}{3.186947in}}%
\pgfpathcurveto{\pgfqpoint{2.452742in}{3.192771in}}{\pgfqpoint{2.444842in}{3.196043in}}{\pgfqpoint{2.436605in}{3.196043in}}%
\pgfpathcurveto{\pgfqpoint{2.428369in}{3.196043in}}{\pgfqpoint{2.420469in}{3.192771in}}{\pgfqpoint{2.414645in}{3.186947in}}%
\pgfpathcurveto{\pgfqpoint{2.408821in}{3.181123in}}{\pgfqpoint{2.405549in}{3.173223in}}{\pgfqpoint{2.405549in}{3.164987in}}%
\pgfpathcurveto{\pgfqpoint{2.405549in}{3.156750in}}{\pgfqpoint{2.408821in}{3.148850in}}{\pgfqpoint{2.414645in}{3.143026in}}%
\pgfpathcurveto{\pgfqpoint{2.420469in}{3.137202in}}{\pgfqpoint{2.428369in}{3.133930in}}{\pgfqpoint{2.436605in}{3.133930in}}%
\pgfpathclose%
\pgfusepath{stroke,fill}%
\end{pgfscope}%
\begin{pgfscope}%
\pgfpathrectangle{\pgfqpoint{0.100000in}{0.220728in}}{\pgfqpoint{3.696000in}{3.696000in}}%
\pgfusepath{clip}%
\pgfsetbuttcap%
\pgfsetroundjoin%
\definecolor{currentfill}{rgb}{0.121569,0.466667,0.705882}%
\pgfsetfillcolor{currentfill}%
\pgfsetfillopacity{0.454670}%
\pgfsetlinewidth{1.003750pt}%
\definecolor{currentstroke}{rgb}{0.121569,0.466667,0.705882}%
\pgfsetstrokecolor{currentstroke}%
\pgfsetstrokeopacity{0.454670}%
\pgfsetdash{}{0pt}%
\pgfpathmoveto{\pgfqpoint{1.334143in}{2.273745in}}%
\pgfpathcurveto{\pgfqpoint{1.342379in}{2.273745in}}{\pgfqpoint{1.350279in}{2.277017in}}{\pgfqpoint{1.356103in}{2.282841in}}%
\pgfpathcurveto{\pgfqpoint{1.361927in}{2.288665in}}{\pgfqpoint{1.365199in}{2.296565in}}{\pgfqpoint{1.365199in}{2.304802in}}%
\pgfpathcurveto{\pgfqpoint{1.365199in}{2.313038in}}{\pgfqpoint{1.361927in}{2.320938in}}{\pgfqpoint{1.356103in}{2.326762in}}%
\pgfpathcurveto{\pgfqpoint{1.350279in}{2.332586in}}{\pgfqpoint{1.342379in}{2.335858in}}{\pgfqpoint{1.334143in}{2.335858in}}%
\pgfpathcurveto{\pgfqpoint{1.325907in}{2.335858in}}{\pgfqpoint{1.318007in}{2.332586in}}{\pgfqpoint{1.312183in}{2.326762in}}%
\pgfpathcurveto{\pgfqpoint{1.306359in}{2.320938in}}{\pgfqpoint{1.303086in}{2.313038in}}{\pgfqpoint{1.303086in}{2.304802in}}%
\pgfpathcurveto{\pgfqpoint{1.303086in}{2.296565in}}{\pgfqpoint{1.306359in}{2.288665in}}{\pgfqpoint{1.312183in}{2.282841in}}%
\pgfpathcurveto{\pgfqpoint{1.318007in}{2.277017in}}{\pgfqpoint{1.325907in}{2.273745in}}{\pgfqpoint{1.334143in}{2.273745in}}%
\pgfpathclose%
\pgfusepath{stroke,fill}%
\end{pgfscope}%
\begin{pgfscope}%
\pgfpathrectangle{\pgfqpoint{0.100000in}{0.220728in}}{\pgfqpoint{3.696000in}{3.696000in}}%
\pgfusepath{clip}%
\pgfsetbuttcap%
\pgfsetroundjoin%
\definecolor{currentfill}{rgb}{0.121569,0.466667,0.705882}%
\pgfsetfillcolor{currentfill}%
\pgfsetfillopacity{0.455009}%
\pgfsetlinewidth{1.003750pt}%
\definecolor{currentstroke}{rgb}{0.121569,0.466667,0.705882}%
\pgfsetstrokecolor{currentstroke}%
\pgfsetstrokeopacity{0.455009}%
\pgfsetdash{}{0pt}%
\pgfpathmoveto{\pgfqpoint{2.440924in}{3.132534in}}%
\pgfpathcurveto{\pgfqpoint{2.449160in}{3.132534in}}{\pgfqpoint{2.457060in}{3.135807in}}{\pgfqpoint{2.462884in}{3.141631in}}%
\pgfpathcurveto{\pgfqpoint{2.468708in}{3.147455in}}{\pgfqpoint{2.471980in}{3.155355in}}{\pgfqpoint{2.471980in}{3.163591in}}%
\pgfpathcurveto{\pgfqpoint{2.471980in}{3.171827in}}{\pgfqpoint{2.468708in}{3.179727in}}{\pgfqpoint{2.462884in}{3.185551in}}%
\pgfpathcurveto{\pgfqpoint{2.457060in}{3.191375in}}{\pgfqpoint{2.449160in}{3.194647in}}{\pgfqpoint{2.440924in}{3.194647in}}%
\pgfpathcurveto{\pgfqpoint{2.432687in}{3.194647in}}{\pgfqpoint{2.424787in}{3.191375in}}{\pgfqpoint{2.418963in}{3.185551in}}%
\pgfpathcurveto{\pgfqpoint{2.413139in}{3.179727in}}{\pgfqpoint{2.409867in}{3.171827in}}{\pgfqpoint{2.409867in}{3.163591in}}%
\pgfpathcurveto{\pgfqpoint{2.409867in}{3.155355in}}{\pgfqpoint{2.413139in}{3.147455in}}{\pgfqpoint{2.418963in}{3.141631in}}%
\pgfpathcurveto{\pgfqpoint{2.424787in}{3.135807in}}{\pgfqpoint{2.432687in}{3.132534in}}{\pgfqpoint{2.440924in}{3.132534in}}%
\pgfpathclose%
\pgfusepath{stroke,fill}%
\end{pgfscope}%
\begin{pgfscope}%
\pgfpathrectangle{\pgfqpoint{0.100000in}{0.220728in}}{\pgfqpoint{3.696000in}{3.696000in}}%
\pgfusepath{clip}%
\pgfsetbuttcap%
\pgfsetroundjoin%
\definecolor{currentfill}{rgb}{0.121569,0.466667,0.705882}%
\pgfsetfillcolor{currentfill}%
\pgfsetfillopacity{0.456410}%
\pgfsetlinewidth{1.003750pt}%
\definecolor{currentstroke}{rgb}{0.121569,0.466667,0.705882}%
\pgfsetstrokecolor{currentstroke}%
\pgfsetstrokeopacity{0.456410}%
\pgfsetdash{}{0pt}%
\pgfpathmoveto{\pgfqpoint{1.325486in}{2.261797in}}%
\pgfpathcurveto{\pgfqpoint{1.333722in}{2.261797in}}{\pgfqpoint{1.341622in}{2.265070in}}{\pgfqpoint{1.347446in}{2.270894in}}%
\pgfpathcurveto{\pgfqpoint{1.353270in}{2.276718in}}{\pgfqpoint{1.356543in}{2.284618in}}{\pgfqpoint{1.356543in}{2.292854in}}%
\pgfpathcurveto{\pgfqpoint{1.356543in}{2.301090in}}{\pgfqpoint{1.353270in}{2.308990in}}{\pgfqpoint{1.347446in}{2.314814in}}%
\pgfpathcurveto{\pgfqpoint{1.341622in}{2.320638in}}{\pgfqpoint{1.333722in}{2.323910in}}{\pgfqpoint{1.325486in}{2.323910in}}%
\pgfpathcurveto{\pgfqpoint{1.317250in}{2.323910in}}{\pgfqpoint{1.309350in}{2.320638in}}{\pgfqpoint{1.303526in}{2.314814in}}%
\pgfpathcurveto{\pgfqpoint{1.297702in}{2.308990in}}{\pgfqpoint{1.294430in}{2.301090in}}{\pgfqpoint{1.294430in}{2.292854in}}%
\pgfpathcurveto{\pgfqpoint{1.294430in}{2.284618in}}{\pgfqpoint{1.297702in}{2.276718in}}{\pgfqpoint{1.303526in}{2.270894in}}%
\pgfpathcurveto{\pgfqpoint{1.309350in}{2.265070in}}{\pgfqpoint{1.317250in}{2.261797in}}{\pgfqpoint{1.325486in}{2.261797in}}%
\pgfpathclose%
\pgfusepath{stroke,fill}%
\end{pgfscope}%
\begin{pgfscope}%
\pgfpathrectangle{\pgfqpoint{0.100000in}{0.220728in}}{\pgfqpoint{3.696000in}{3.696000in}}%
\pgfusepath{clip}%
\pgfsetbuttcap%
\pgfsetroundjoin%
\definecolor{currentfill}{rgb}{0.121569,0.466667,0.705882}%
\pgfsetfillcolor{currentfill}%
\pgfsetfillopacity{0.456715}%
\pgfsetlinewidth{1.003750pt}%
\definecolor{currentstroke}{rgb}{0.121569,0.466667,0.705882}%
\pgfsetstrokecolor{currentstroke}%
\pgfsetstrokeopacity{0.456715}%
\pgfsetdash{}{0pt}%
\pgfpathmoveto{\pgfqpoint{2.445808in}{3.131834in}}%
\pgfpathcurveto{\pgfqpoint{2.454044in}{3.131834in}}{\pgfqpoint{2.461944in}{3.135107in}}{\pgfqpoint{2.467768in}{3.140930in}}%
\pgfpathcurveto{\pgfqpoint{2.473592in}{3.146754in}}{\pgfqpoint{2.476864in}{3.154654in}}{\pgfqpoint{2.476864in}{3.162891in}}%
\pgfpathcurveto{\pgfqpoint{2.476864in}{3.171127in}}{\pgfqpoint{2.473592in}{3.179027in}}{\pgfqpoint{2.467768in}{3.184851in}}%
\pgfpathcurveto{\pgfqpoint{2.461944in}{3.190675in}}{\pgfqpoint{2.454044in}{3.193947in}}{\pgfqpoint{2.445808in}{3.193947in}}%
\pgfpathcurveto{\pgfqpoint{2.437571in}{3.193947in}}{\pgfqpoint{2.429671in}{3.190675in}}{\pgfqpoint{2.423847in}{3.184851in}}%
\pgfpathcurveto{\pgfqpoint{2.418023in}{3.179027in}}{\pgfqpoint{2.414751in}{3.171127in}}{\pgfqpoint{2.414751in}{3.162891in}}%
\pgfpathcurveto{\pgfqpoint{2.414751in}{3.154654in}}{\pgfqpoint{2.418023in}{3.146754in}}{\pgfqpoint{2.423847in}{3.140930in}}%
\pgfpathcurveto{\pgfqpoint{2.429671in}{3.135107in}}{\pgfqpoint{2.437571in}{3.131834in}}{\pgfqpoint{2.445808in}{3.131834in}}%
\pgfpathclose%
\pgfusepath{stroke,fill}%
\end{pgfscope}%
\begin{pgfscope}%
\pgfpathrectangle{\pgfqpoint{0.100000in}{0.220728in}}{\pgfqpoint{3.696000in}{3.696000in}}%
\pgfusepath{clip}%
\pgfsetbuttcap%
\pgfsetroundjoin%
\definecolor{currentfill}{rgb}{0.121569,0.466667,0.705882}%
\pgfsetfillcolor{currentfill}%
\pgfsetfillopacity{0.457278}%
\pgfsetlinewidth{1.003750pt}%
\definecolor{currentstroke}{rgb}{0.121569,0.466667,0.705882}%
\pgfsetstrokecolor{currentstroke}%
\pgfsetstrokeopacity{0.457278}%
\pgfsetdash{}{0pt}%
\pgfpathmoveto{\pgfqpoint{2.448801in}{3.130775in}}%
\pgfpathcurveto{\pgfqpoint{2.457037in}{3.130775in}}{\pgfqpoint{2.464937in}{3.134048in}}{\pgfqpoint{2.470761in}{3.139872in}}%
\pgfpathcurveto{\pgfqpoint{2.476585in}{3.145696in}}{\pgfqpoint{2.479857in}{3.153596in}}{\pgfqpoint{2.479857in}{3.161832in}}%
\pgfpathcurveto{\pgfqpoint{2.479857in}{3.170068in}}{\pgfqpoint{2.476585in}{3.177968in}}{\pgfqpoint{2.470761in}{3.183792in}}%
\pgfpathcurveto{\pgfqpoint{2.464937in}{3.189616in}}{\pgfqpoint{2.457037in}{3.192888in}}{\pgfqpoint{2.448801in}{3.192888in}}%
\pgfpathcurveto{\pgfqpoint{2.440565in}{3.192888in}}{\pgfqpoint{2.432665in}{3.189616in}}{\pgfqpoint{2.426841in}{3.183792in}}%
\pgfpathcurveto{\pgfqpoint{2.421017in}{3.177968in}}{\pgfqpoint{2.417744in}{3.170068in}}{\pgfqpoint{2.417744in}{3.161832in}}%
\pgfpathcurveto{\pgfqpoint{2.417744in}{3.153596in}}{\pgfqpoint{2.421017in}{3.145696in}}{\pgfqpoint{2.426841in}{3.139872in}}%
\pgfpathcurveto{\pgfqpoint{2.432665in}{3.134048in}}{\pgfqpoint{2.440565in}{3.130775in}}{\pgfqpoint{2.448801in}{3.130775in}}%
\pgfpathclose%
\pgfusepath{stroke,fill}%
\end{pgfscope}%
\begin{pgfscope}%
\pgfpathrectangle{\pgfqpoint{0.100000in}{0.220728in}}{\pgfqpoint{3.696000in}{3.696000in}}%
\pgfusepath{clip}%
\pgfsetbuttcap%
\pgfsetroundjoin%
\definecolor{currentfill}{rgb}{0.121569,0.466667,0.705882}%
\pgfsetfillcolor{currentfill}%
\pgfsetfillopacity{0.457714}%
\pgfsetlinewidth{1.003750pt}%
\definecolor{currentstroke}{rgb}{0.121569,0.466667,0.705882}%
\pgfsetstrokecolor{currentstroke}%
\pgfsetstrokeopacity{0.457714}%
\pgfsetdash{}{0pt}%
\pgfpathmoveto{\pgfqpoint{2.450350in}{3.130417in}}%
\pgfpathcurveto{\pgfqpoint{2.458586in}{3.130417in}}{\pgfqpoint{2.466486in}{3.133689in}}{\pgfqpoint{2.472310in}{3.139513in}}%
\pgfpathcurveto{\pgfqpoint{2.478134in}{3.145337in}}{\pgfqpoint{2.481406in}{3.153237in}}{\pgfqpoint{2.481406in}{3.161473in}}%
\pgfpathcurveto{\pgfqpoint{2.481406in}{3.169710in}}{\pgfqpoint{2.478134in}{3.177610in}}{\pgfqpoint{2.472310in}{3.183434in}}%
\pgfpathcurveto{\pgfqpoint{2.466486in}{3.189258in}}{\pgfqpoint{2.458586in}{3.192530in}}{\pgfqpoint{2.450350in}{3.192530in}}%
\pgfpathcurveto{\pgfqpoint{2.442114in}{3.192530in}}{\pgfqpoint{2.434213in}{3.189258in}}{\pgfqpoint{2.428390in}{3.183434in}}%
\pgfpathcurveto{\pgfqpoint{2.422566in}{3.177610in}}{\pgfqpoint{2.419293in}{3.169710in}}{\pgfqpoint{2.419293in}{3.161473in}}%
\pgfpathcurveto{\pgfqpoint{2.419293in}{3.153237in}}{\pgfqpoint{2.422566in}{3.145337in}}{\pgfqpoint{2.428390in}{3.139513in}}%
\pgfpathcurveto{\pgfqpoint{2.434213in}{3.133689in}}{\pgfqpoint{2.442114in}{3.130417in}}{\pgfqpoint{2.450350in}{3.130417in}}%
\pgfpathclose%
\pgfusepath{stroke,fill}%
\end{pgfscope}%
\begin{pgfscope}%
\pgfpathrectangle{\pgfqpoint{0.100000in}{0.220728in}}{\pgfqpoint{3.696000in}{3.696000in}}%
\pgfusepath{clip}%
\pgfsetbuttcap%
\pgfsetroundjoin%
\definecolor{currentfill}{rgb}{0.121569,0.466667,0.705882}%
\pgfsetfillcolor{currentfill}%
\pgfsetfillopacity{0.458574}%
\pgfsetlinewidth{1.003750pt}%
\definecolor{currentstroke}{rgb}{0.121569,0.466667,0.705882}%
\pgfsetstrokecolor{currentstroke}%
\pgfsetstrokeopacity{0.458574}%
\pgfsetdash{}{0pt}%
\pgfpathmoveto{\pgfqpoint{1.320295in}{2.249359in}}%
\pgfpathcurveto{\pgfqpoint{1.328531in}{2.249359in}}{\pgfqpoint{1.336431in}{2.252631in}}{\pgfqpoint{1.342255in}{2.258455in}}%
\pgfpathcurveto{\pgfqpoint{1.348079in}{2.264279in}}{\pgfqpoint{1.351351in}{2.272179in}}{\pgfqpoint{1.351351in}{2.280415in}}%
\pgfpathcurveto{\pgfqpoint{1.351351in}{2.288651in}}{\pgfqpoint{1.348079in}{2.296551in}}{\pgfqpoint{1.342255in}{2.302375in}}%
\pgfpathcurveto{\pgfqpoint{1.336431in}{2.308199in}}{\pgfqpoint{1.328531in}{2.311472in}}{\pgfqpoint{1.320295in}{2.311472in}}%
\pgfpathcurveto{\pgfqpoint{1.312058in}{2.311472in}}{\pgfqpoint{1.304158in}{2.308199in}}{\pgfqpoint{1.298334in}{2.302375in}}%
\pgfpathcurveto{\pgfqpoint{1.292511in}{2.296551in}}{\pgfqpoint{1.289238in}{2.288651in}}{\pgfqpoint{1.289238in}{2.280415in}}%
\pgfpathcurveto{\pgfqpoint{1.289238in}{2.272179in}}{\pgfqpoint{1.292511in}{2.264279in}}{\pgfqpoint{1.298334in}{2.258455in}}%
\pgfpathcurveto{\pgfqpoint{1.304158in}{2.252631in}}{\pgfqpoint{1.312058in}{2.249359in}}{\pgfqpoint{1.320295in}{2.249359in}}%
\pgfpathclose%
\pgfusepath{stroke,fill}%
\end{pgfscope}%
\begin{pgfscope}%
\pgfpathrectangle{\pgfqpoint{0.100000in}{0.220728in}}{\pgfqpoint{3.696000in}{3.696000in}}%
\pgfusepath{clip}%
\pgfsetbuttcap%
\pgfsetroundjoin%
\definecolor{currentfill}{rgb}{0.121569,0.466667,0.705882}%
\pgfsetfillcolor{currentfill}%
\pgfsetfillopacity{0.458784}%
\pgfsetlinewidth{1.003750pt}%
\definecolor{currentstroke}{rgb}{0.121569,0.466667,0.705882}%
\pgfsetstrokecolor{currentstroke}%
\pgfsetstrokeopacity{0.458784}%
\pgfsetdash{}{0pt}%
\pgfpathmoveto{\pgfqpoint{2.453294in}{3.129896in}}%
\pgfpathcurveto{\pgfqpoint{2.461530in}{3.129896in}}{\pgfqpoint{2.469430in}{3.133168in}}{\pgfqpoint{2.475254in}{3.138992in}}%
\pgfpathcurveto{\pgfqpoint{2.481078in}{3.144816in}}{\pgfqpoint{2.484351in}{3.152716in}}{\pgfqpoint{2.484351in}{3.160952in}}%
\pgfpathcurveto{\pgfqpoint{2.484351in}{3.169188in}}{\pgfqpoint{2.481078in}{3.177088in}}{\pgfqpoint{2.475254in}{3.182912in}}%
\pgfpathcurveto{\pgfqpoint{2.469430in}{3.188736in}}{\pgfqpoint{2.461530in}{3.192009in}}{\pgfqpoint{2.453294in}{3.192009in}}%
\pgfpathcurveto{\pgfqpoint{2.445058in}{3.192009in}}{\pgfqpoint{2.437158in}{3.188736in}}{\pgfqpoint{2.431334in}{3.182912in}}%
\pgfpathcurveto{\pgfqpoint{2.425510in}{3.177088in}}{\pgfqpoint{2.422238in}{3.169188in}}{\pgfqpoint{2.422238in}{3.160952in}}%
\pgfpathcurveto{\pgfqpoint{2.422238in}{3.152716in}}{\pgfqpoint{2.425510in}{3.144816in}}{\pgfqpoint{2.431334in}{3.138992in}}%
\pgfpathcurveto{\pgfqpoint{2.437158in}{3.133168in}}{\pgfqpoint{2.445058in}{3.129896in}}{\pgfqpoint{2.453294in}{3.129896in}}%
\pgfpathclose%
\pgfusepath{stroke,fill}%
\end{pgfscope}%
\begin{pgfscope}%
\pgfpathrectangle{\pgfqpoint{0.100000in}{0.220728in}}{\pgfqpoint{3.696000in}{3.696000in}}%
\pgfusepath{clip}%
\pgfsetbuttcap%
\pgfsetroundjoin%
\definecolor{currentfill}{rgb}{0.121569,0.466667,0.705882}%
\pgfsetfillcolor{currentfill}%
\pgfsetfillopacity{0.459173}%
\pgfsetlinewidth{1.003750pt}%
\definecolor{currentstroke}{rgb}{0.121569,0.466667,0.705882}%
\pgfsetstrokecolor{currentstroke}%
\pgfsetstrokeopacity{0.459173}%
\pgfsetdash{}{0pt}%
\pgfpathmoveto{\pgfqpoint{2.455110in}{3.129315in}}%
\pgfpathcurveto{\pgfqpoint{2.463347in}{3.129315in}}{\pgfqpoint{2.471247in}{3.132587in}}{\pgfqpoint{2.477071in}{3.138411in}}%
\pgfpathcurveto{\pgfqpoint{2.482895in}{3.144235in}}{\pgfqpoint{2.486167in}{3.152135in}}{\pgfqpoint{2.486167in}{3.160372in}}%
\pgfpathcurveto{\pgfqpoint{2.486167in}{3.168608in}}{\pgfqpoint{2.482895in}{3.176508in}}{\pgfqpoint{2.477071in}{3.182332in}}%
\pgfpathcurveto{\pgfqpoint{2.471247in}{3.188156in}}{\pgfqpoint{2.463347in}{3.191428in}}{\pgfqpoint{2.455110in}{3.191428in}}%
\pgfpathcurveto{\pgfqpoint{2.446874in}{3.191428in}}{\pgfqpoint{2.438974in}{3.188156in}}{\pgfqpoint{2.433150in}{3.182332in}}%
\pgfpathcurveto{\pgfqpoint{2.427326in}{3.176508in}}{\pgfqpoint{2.424054in}{3.168608in}}{\pgfqpoint{2.424054in}{3.160372in}}%
\pgfpathcurveto{\pgfqpoint{2.424054in}{3.152135in}}{\pgfqpoint{2.427326in}{3.144235in}}{\pgfqpoint{2.433150in}{3.138411in}}%
\pgfpathcurveto{\pgfqpoint{2.438974in}{3.132587in}}{\pgfqpoint{2.446874in}{3.129315in}}{\pgfqpoint{2.455110in}{3.129315in}}%
\pgfpathclose%
\pgfusepath{stroke,fill}%
\end{pgfscope}%
\begin{pgfscope}%
\pgfpathrectangle{\pgfqpoint{0.100000in}{0.220728in}}{\pgfqpoint{3.696000in}{3.696000in}}%
\pgfusepath{clip}%
\pgfsetbuttcap%
\pgfsetroundjoin%
\definecolor{currentfill}{rgb}{0.121569,0.466667,0.705882}%
\pgfsetfillcolor{currentfill}%
\pgfsetfillopacity{0.459871}%
\pgfsetlinewidth{1.003750pt}%
\definecolor{currentstroke}{rgb}{0.121569,0.466667,0.705882}%
\pgfsetstrokecolor{currentstroke}%
\pgfsetstrokeopacity{0.459871}%
\pgfsetdash{}{0pt}%
\pgfpathmoveto{\pgfqpoint{2.457587in}{3.128478in}}%
\pgfpathcurveto{\pgfqpoint{2.465823in}{3.128478in}}{\pgfqpoint{2.473723in}{3.131750in}}{\pgfqpoint{2.479547in}{3.137574in}}%
\pgfpathcurveto{\pgfqpoint{2.485371in}{3.143398in}}{\pgfqpoint{2.488643in}{3.151298in}}{\pgfqpoint{2.488643in}{3.159534in}}%
\pgfpathcurveto{\pgfqpoint{2.488643in}{3.167770in}}{\pgfqpoint{2.485371in}{3.175670in}}{\pgfqpoint{2.479547in}{3.181494in}}%
\pgfpathcurveto{\pgfqpoint{2.473723in}{3.187318in}}{\pgfqpoint{2.465823in}{3.190591in}}{\pgfqpoint{2.457587in}{3.190591in}}%
\pgfpathcurveto{\pgfqpoint{2.449350in}{3.190591in}}{\pgfqpoint{2.441450in}{3.187318in}}{\pgfqpoint{2.435626in}{3.181494in}}%
\pgfpathcurveto{\pgfqpoint{2.429802in}{3.175670in}}{\pgfqpoint{2.426530in}{3.167770in}}{\pgfqpoint{2.426530in}{3.159534in}}%
\pgfpathcurveto{\pgfqpoint{2.426530in}{3.151298in}}{\pgfqpoint{2.429802in}{3.143398in}}{\pgfqpoint{2.435626in}{3.137574in}}%
\pgfpathcurveto{\pgfqpoint{2.441450in}{3.131750in}}{\pgfqpoint{2.449350in}{3.128478in}}{\pgfqpoint{2.457587in}{3.128478in}}%
\pgfpathclose%
\pgfusepath{stroke,fill}%
\end{pgfscope}%
\begin{pgfscope}%
\pgfpathrectangle{\pgfqpoint{0.100000in}{0.220728in}}{\pgfqpoint{3.696000in}{3.696000in}}%
\pgfusepath{clip}%
\pgfsetbuttcap%
\pgfsetroundjoin%
\definecolor{currentfill}{rgb}{0.121569,0.466667,0.705882}%
\pgfsetfillcolor{currentfill}%
\pgfsetfillopacity{0.459969}%
\pgfsetlinewidth{1.003750pt}%
\definecolor{currentstroke}{rgb}{0.121569,0.466667,0.705882}%
\pgfsetstrokecolor{currentstroke}%
\pgfsetstrokeopacity{0.459969}%
\pgfsetdash{}{0pt}%
\pgfpathmoveto{\pgfqpoint{2.459202in}{3.127740in}}%
\pgfpathcurveto{\pgfqpoint{2.467439in}{3.127740in}}{\pgfqpoint{2.475339in}{3.131012in}}{\pgfqpoint{2.481162in}{3.136836in}}%
\pgfpathcurveto{\pgfqpoint{2.486986in}{3.142660in}}{\pgfqpoint{2.490259in}{3.150560in}}{\pgfqpoint{2.490259in}{3.158796in}}%
\pgfpathcurveto{\pgfqpoint{2.490259in}{3.167032in}}{\pgfqpoint{2.486986in}{3.174933in}}{\pgfqpoint{2.481162in}{3.180756in}}%
\pgfpathcurveto{\pgfqpoint{2.475339in}{3.186580in}}{\pgfqpoint{2.467439in}{3.189853in}}{\pgfqpoint{2.459202in}{3.189853in}}%
\pgfpathcurveto{\pgfqpoint{2.450966in}{3.189853in}}{\pgfqpoint{2.443066in}{3.186580in}}{\pgfqpoint{2.437242in}{3.180756in}}%
\pgfpathcurveto{\pgfqpoint{2.431418in}{3.174933in}}{\pgfqpoint{2.428146in}{3.167032in}}{\pgfqpoint{2.428146in}{3.158796in}}%
\pgfpathcurveto{\pgfqpoint{2.428146in}{3.150560in}}{\pgfqpoint{2.431418in}{3.142660in}}{\pgfqpoint{2.437242in}{3.136836in}}%
\pgfpathcurveto{\pgfqpoint{2.443066in}{3.131012in}}{\pgfqpoint{2.450966in}{3.127740in}}{\pgfqpoint{2.459202in}{3.127740in}}%
\pgfpathclose%
\pgfusepath{stroke,fill}%
\end{pgfscope}%
\begin{pgfscope}%
\pgfpathrectangle{\pgfqpoint{0.100000in}{0.220728in}}{\pgfqpoint{3.696000in}{3.696000in}}%
\pgfusepath{clip}%
\pgfsetbuttcap%
\pgfsetroundjoin%
\definecolor{currentfill}{rgb}{0.121569,0.466667,0.705882}%
\pgfsetfillcolor{currentfill}%
\pgfsetfillopacity{0.460109}%
\pgfsetlinewidth{1.003750pt}%
\definecolor{currentstroke}{rgb}{0.121569,0.466667,0.705882}%
\pgfsetstrokecolor{currentstroke}%
\pgfsetstrokeopacity{0.460109}%
\pgfsetdash{}{0pt}%
\pgfpathmoveto{\pgfqpoint{2.460031in}{3.127436in}}%
\pgfpathcurveto{\pgfqpoint{2.468268in}{3.127436in}}{\pgfqpoint{2.476168in}{3.130709in}}{\pgfqpoint{2.481992in}{3.136532in}}%
\pgfpathcurveto{\pgfqpoint{2.487816in}{3.142356in}}{\pgfqpoint{2.491088in}{3.150256in}}{\pgfqpoint{2.491088in}{3.158493in}}%
\pgfpathcurveto{\pgfqpoint{2.491088in}{3.166729in}}{\pgfqpoint{2.487816in}{3.174629in}}{\pgfqpoint{2.481992in}{3.180453in}}%
\pgfpathcurveto{\pgfqpoint{2.476168in}{3.186277in}}{\pgfqpoint{2.468268in}{3.189549in}}{\pgfqpoint{2.460031in}{3.189549in}}%
\pgfpathcurveto{\pgfqpoint{2.451795in}{3.189549in}}{\pgfqpoint{2.443895in}{3.186277in}}{\pgfqpoint{2.438071in}{3.180453in}}%
\pgfpathcurveto{\pgfqpoint{2.432247in}{3.174629in}}{\pgfqpoint{2.428975in}{3.166729in}}{\pgfqpoint{2.428975in}{3.158493in}}%
\pgfpathcurveto{\pgfqpoint{2.428975in}{3.150256in}}{\pgfqpoint{2.432247in}{3.142356in}}{\pgfqpoint{2.438071in}{3.136532in}}%
\pgfpathcurveto{\pgfqpoint{2.443895in}{3.130709in}}{\pgfqpoint{2.451795in}{3.127436in}}{\pgfqpoint{2.460031in}{3.127436in}}%
\pgfpathclose%
\pgfusepath{stroke,fill}%
\end{pgfscope}%
\begin{pgfscope}%
\pgfpathrectangle{\pgfqpoint{0.100000in}{0.220728in}}{\pgfqpoint{3.696000in}{3.696000in}}%
\pgfusepath{clip}%
\pgfsetbuttcap%
\pgfsetroundjoin%
\definecolor{currentfill}{rgb}{0.121569,0.466667,0.705882}%
\pgfsetfillcolor{currentfill}%
\pgfsetfillopacity{0.460786}%
\pgfsetlinewidth{1.003750pt}%
\definecolor{currentstroke}{rgb}{0.121569,0.466667,0.705882}%
\pgfsetstrokecolor{currentstroke}%
\pgfsetstrokeopacity{0.460786}%
\pgfsetdash{}{0pt}%
\pgfpathmoveto{\pgfqpoint{2.462410in}{3.127402in}}%
\pgfpathcurveto{\pgfqpoint{2.470647in}{3.127402in}}{\pgfqpoint{2.478547in}{3.130674in}}{\pgfqpoint{2.484371in}{3.136498in}}%
\pgfpathcurveto{\pgfqpoint{2.490195in}{3.142322in}}{\pgfqpoint{2.493467in}{3.150222in}}{\pgfqpoint{2.493467in}{3.158458in}}%
\pgfpathcurveto{\pgfqpoint{2.493467in}{3.166695in}}{\pgfqpoint{2.490195in}{3.174595in}}{\pgfqpoint{2.484371in}{3.180419in}}%
\pgfpathcurveto{\pgfqpoint{2.478547in}{3.186243in}}{\pgfqpoint{2.470647in}{3.189515in}}{\pgfqpoint{2.462410in}{3.189515in}}%
\pgfpathcurveto{\pgfqpoint{2.454174in}{3.189515in}}{\pgfqpoint{2.446274in}{3.186243in}}{\pgfqpoint{2.440450in}{3.180419in}}%
\pgfpathcurveto{\pgfqpoint{2.434626in}{3.174595in}}{\pgfqpoint{2.431354in}{3.166695in}}{\pgfqpoint{2.431354in}{3.158458in}}%
\pgfpathcurveto{\pgfqpoint{2.431354in}{3.150222in}}{\pgfqpoint{2.434626in}{3.142322in}}{\pgfqpoint{2.440450in}{3.136498in}}%
\pgfpathcurveto{\pgfqpoint{2.446274in}{3.130674in}}{\pgfqpoint{2.454174in}{3.127402in}}{\pgfqpoint{2.462410in}{3.127402in}}%
\pgfpathclose%
\pgfusepath{stroke,fill}%
\end{pgfscope}%
\begin{pgfscope}%
\pgfpathrectangle{\pgfqpoint{0.100000in}{0.220728in}}{\pgfqpoint{3.696000in}{3.696000in}}%
\pgfusepath{clip}%
\pgfsetbuttcap%
\pgfsetroundjoin%
\definecolor{currentfill}{rgb}{0.121569,0.466667,0.705882}%
\pgfsetfillcolor{currentfill}%
\pgfsetfillopacity{0.461670}%
\pgfsetlinewidth{1.003750pt}%
\definecolor{currentstroke}{rgb}{0.121569,0.466667,0.705882}%
\pgfsetstrokecolor{currentstroke}%
\pgfsetstrokeopacity{0.461670}%
\pgfsetdash{}{0pt}%
\pgfpathmoveto{\pgfqpoint{2.467098in}{3.126622in}}%
\pgfpathcurveto{\pgfqpoint{2.475334in}{3.126622in}}{\pgfqpoint{2.483234in}{3.129895in}}{\pgfqpoint{2.489058in}{3.135718in}}%
\pgfpathcurveto{\pgfqpoint{2.494882in}{3.141542in}}{\pgfqpoint{2.498154in}{3.149442in}}{\pgfqpoint{2.498154in}{3.157679in}}%
\pgfpathcurveto{\pgfqpoint{2.498154in}{3.165915in}}{\pgfqpoint{2.494882in}{3.173815in}}{\pgfqpoint{2.489058in}{3.179639in}}%
\pgfpathcurveto{\pgfqpoint{2.483234in}{3.185463in}}{\pgfqpoint{2.475334in}{3.188735in}}{\pgfqpoint{2.467098in}{3.188735in}}%
\pgfpathcurveto{\pgfqpoint{2.458862in}{3.188735in}}{\pgfqpoint{2.450962in}{3.185463in}}{\pgfqpoint{2.445138in}{3.179639in}}%
\pgfpathcurveto{\pgfqpoint{2.439314in}{3.173815in}}{\pgfqpoint{2.436041in}{3.165915in}}{\pgfqpoint{2.436041in}{3.157679in}}%
\pgfpathcurveto{\pgfqpoint{2.436041in}{3.149442in}}{\pgfqpoint{2.439314in}{3.141542in}}{\pgfqpoint{2.445138in}{3.135718in}}%
\pgfpathcurveto{\pgfqpoint{2.450962in}{3.129895in}}{\pgfqpoint{2.458862in}{3.126622in}}{\pgfqpoint{2.467098in}{3.126622in}}%
\pgfpathclose%
\pgfusepath{stroke,fill}%
\end{pgfscope}%
\begin{pgfscope}%
\pgfpathrectangle{\pgfqpoint{0.100000in}{0.220728in}}{\pgfqpoint{3.696000in}{3.696000in}}%
\pgfusepath{clip}%
\pgfsetbuttcap%
\pgfsetroundjoin%
\definecolor{currentfill}{rgb}{0.121569,0.466667,0.705882}%
\pgfsetfillcolor{currentfill}%
\pgfsetfillopacity{0.462089}%
\pgfsetlinewidth{1.003750pt}%
\definecolor{currentstroke}{rgb}{0.121569,0.466667,0.705882}%
\pgfsetstrokecolor{currentstroke}%
\pgfsetstrokeopacity{0.462089}%
\pgfsetdash{}{0pt}%
\pgfpathmoveto{\pgfqpoint{1.308233in}{2.227850in}}%
\pgfpathcurveto{\pgfqpoint{1.316469in}{2.227850in}}{\pgfqpoint{1.324369in}{2.231122in}}{\pgfqpoint{1.330193in}{2.236946in}}%
\pgfpathcurveto{\pgfqpoint{1.336017in}{2.242770in}}{\pgfqpoint{1.339290in}{2.250670in}}{\pgfqpoint{1.339290in}{2.258906in}}%
\pgfpathcurveto{\pgfqpoint{1.339290in}{2.267142in}}{\pgfqpoint{1.336017in}{2.275043in}}{\pgfqpoint{1.330193in}{2.280866in}}%
\pgfpathcurveto{\pgfqpoint{1.324369in}{2.286690in}}{\pgfqpoint{1.316469in}{2.289963in}}{\pgfqpoint{1.308233in}{2.289963in}}%
\pgfpathcurveto{\pgfqpoint{1.299997in}{2.289963in}}{\pgfqpoint{1.292097in}{2.286690in}}{\pgfqpoint{1.286273in}{2.280866in}}%
\pgfpathcurveto{\pgfqpoint{1.280449in}{2.275043in}}{\pgfqpoint{1.277177in}{2.267142in}}{\pgfqpoint{1.277177in}{2.258906in}}%
\pgfpathcurveto{\pgfqpoint{1.277177in}{2.250670in}}{\pgfqpoint{1.280449in}{2.242770in}}{\pgfqpoint{1.286273in}{2.236946in}}%
\pgfpathcurveto{\pgfqpoint{1.292097in}{2.231122in}}{\pgfqpoint{1.299997in}{2.227850in}}{\pgfqpoint{1.308233in}{2.227850in}}%
\pgfpathclose%
\pgfusepath{stroke,fill}%
\end{pgfscope}%
\begin{pgfscope}%
\pgfpathrectangle{\pgfqpoint{0.100000in}{0.220728in}}{\pgfqpoint{3.696000in}{3.696000in}}%
\pgfusepath{clip}%
\pgfsetbuttcap%
\pgfsetroundjoin%
\definecolor{currentfill}{rgb}{0.121569,0.466667,0.705882}%
\pgfsetfillcolor{currentfill}%
\pgfsetfillopacity{0.463079}%
\pgfsetlinewidth{1.003750pt}%
\definecolor{currentstroke}{rgb}{0.121569,0.466667,0.705882}%
\pgfsetstrokecolor{currentstroke}%
\pgfsetstrokeopacity{0.463079}%
\pgfsetdash{}{0pt}%
\pgfpathmoveto{\pgfqpoint{2.471777in}{3.124873in}}%
\pgfpathcurveto{\pgfqpoint{2.480014in}{3.124873in}}{\pgfqpoint{2.487914in}{3.128145in}}{\pgfqpoint{2.493738in}{3.133969in}}%
\pgfpathcurveto{\pgfqpoint{2.499561in}{3.139793in}}{\pgfqpoint{2.502834in}{3.147693in}}{\pgfqpoint{2.502834in}{3.155930in}}%
\pgfpathcurveto{\pgfqpoint{2.502834in}{3.164166in}}{\pgfqpoint{2.499561in}{3.172066in}}{\pgfqpoint{2.493738in}{3.177890in}}%
\pgfpathcurveto{\pgfqpoint{2.487914in}{3.183714in}}{\pgfqpoint{2.480014in}{3.186986in}}{\pgfqpoint{2.471777in}{3.186986in}}%
\pgfpathcurveto{\pgfqpoint{2.463541in}{3.186986in}}{\pgfqpoint{2.455641in}{3.183714in}}{\pgfqpoint{2.449817in}{3.177890in}}%
\pgfpathcurveto{\pgfqpoint{2.443993in}{3.172066in}}{\pgfqpoint{2.440721in}{3.164166in}}{\pgfqpoint{2.440721in}{3.155930in}}%
\pgfpathcurveto{\pgfqpoint{2.440721in}{3.147693in}}{\pgfqpoint{2.443993in}{3.139793in}}{\pgfqpoint{2.449817in}{3.133969in}}%
\pgfpathcurveto{\pgfqpoint{2.455641in}{3.128145in}}{\pgfqpoint{2.463541in}{3.124873in}}{\pgfqpoint{2.471777in}{3.124873in}}%
\pgfpathclose%
\pgfusepath{stroke,fill}%
\end{pgfscope}%
\begin{pgfscope}%
\pgfpathrectangle{\pgfqpoint{0.100000in}{0.220728in}}{\pgfqpoint{3.696000in}{3.696000in}}%
\pgfusepath{clip}%
\pgfsetbuttcap%
\pgfsetroundjoin%
\definecolor{currentfill}{rgb}{0.121569,0.466667,0.705882}%
\pgfsetfillcolor{currentfill}%
\pgfsetfillopacity{0.464306}%
\pgfsetlinewidth{1.003750pt}%
\definecolor{currentstroke}{rgb}{0.121569,0.466667,0.705882}%
\pgfsetstrokecolor{currentstroke}%
\pgfsetstrokeopacity{0.464306}%
\pgfsetdash{}{0pt}%
\pgfpathmoveto{\pgfqpoint{2.477389in}{3.122694in}}%
\pgfpathcurveto{\pgfqpoint{2.485625in}{3.122694in}}{\pgfqpoint{2.493525in}{3.125966in}}{\pgfqpoint{2.499349in}{3.131790in}}%
\pgfpathcurveto{\pgfqpoint{2.505173in}{3.137614in}}{\pgfqpoint{2.508445in}{3.145514in}}{\pgfqpoint{2.508445in}{3.153750in}}%
\pgfpathcurveto{\pgfqpoint{2.508445in}{3.161986in}}{\pgfqpoint{2.505173in}{3.169886in}}{\pgfqpoint{2.499349in}{3.175710in}}%
\pgfpathcurveto{\pgfqpoint{2.493525in}{3.181534in}}{\pgfqpoint{2.485625in}{3.184807in}}{\pgfqpoint{2.477389in}{3.184807in}}%
\pgfpathcurveto{\pgfqpoint{2.469152in}{3.184807in}}{\pgfqpoint{2.461252in}{3.181534in}}{\pgfqpoint{2.455428in}{3.175710in}}%
\pgfpathcurveto{\pgfqpoint{2.449604in}{3.169886in}}{\pgfqpoint{2.446332in}{3.161986in}}{\pgfqpoint{2.446332in}{3.153750in}}%
\pgfpathcurveto{\pgfqpoint{2.446332in}{3.145514in}}{\pgfqpoint{2.449604in}{3.137614in}}{\pgfqpoint{2.455428in}{3.131790in}}%
\pgfpathcurveto{\pgfqpoint{2.461252in}{3.125966in}}{\pgfqpoint{2.469152in}{3.122694in}}{\pgfqpoint{2.477389in}{3.122694in}}%
\pgfpathclose%
\pgfusepath{stroke,fill}%
\end{pgfscope}%
\begin{pgfscope}%
\pgfpathrectangle{\pgfqpoint{0.100000in}{0.220728in}}{\pgfqpoint{3.696000in}{3.696000in}}%
\pgfusepath{clip}%
\pgfsetbuttcap%
\pgfsetroundjoin%
\definecolor{currentfill}{rgb}{0.121569,0.466667,0.705882}%
\pgfsetfillcolor{currentfill}%
\pgfsetfillopacity{0.465188}%
\pgfsetlinewidth{1.003750pt}%
\definecolor{currentstroke}{rgb}{0.121569,0.466667,0.705882}%
\pgfsetstrokecolor{currentstroke}%
\pgfsetstrokeopacity{0.465188}%
\pgfsetdash{}{0pt}%
\pgfpathmoveto{\pgfqpoint{1.295830in}{2.208006in}}%
\pgfpathcurveto{\pgfqpoint{1.304067in}{2.208006in}}{\pgfqpoint{1.311967in}{2.211279in}}{\pgfqpoint{1.317791in}{2.217103in}}%
\pgfpathcurveto{\pgfqpoint{1.323615in}{2.222926in}}{\pgfqpoint{1.326887in}{2.230827in}}{\pgfqpoint{1.326887in}{2.239063in}}%
\pgfpathcurveto{\pgfqpoint{1.326887in}{2.247299in}}{\pgfqpoint{1.323615in}{2.255199in}}{\pgfqpoint{1.317791in}{2.261023in}}%
\pgfpathcurveto{\pgfqpoint{1.311967in}{2.266847in}}{\pgfqpoint{1.304067in}{2.270119in}}{\pgfqpoint{1.295830in}{2.270119in}}%
\pgfpathcurveto{\pgfqpoint{1.287594in}{2.270119in}}{\pgfqpoint{1.279694in}{2.266847in}}{\pgfqpoint{1.273870in}{2.261023in}}%
\pgfpathcurveto{\pgfqpoint{1.268046in}{2.255199in}}{\pgfqpoint{1.264774in}{2.247299in}}{\pgfqpoint{1.264774in}{2.239063in}}%
\pgfpathcurveto{\pgfqpoint{1.264774in}{2.230827in}}{\pgfqpoint{1.268046in}{2.222926in}}{\pgfqpoint{1.273870in}{2.217103in}}%
\pgfpathcurveto{\pgfqpoint{1.279694in}{2.211279in}}{\pgfqpoint{1.287594in}{2.208006in}}{\pgfqpoint{1.295830in}{2.208006in}}%
\pgfpathclose%
\pgfusepath{stroke,fill}%
\end{pgfscope}%
\begin{pgfscope}%
\pgfpathrectangle{\pgfqpoint{0.100000in}{0.220728in}}{\pgfqpoint{3.696000in}{3.696000in}}%
\pgfusepath{clip}%
\pgfsetbuttcap%
\pgfsetroundjoin%
\definecolor{currentfill}{rgb}{0.121569,0.466667,0.705882}%
\pgfsetfillcolor{currentfill}%
\pgfsetfillopacity{0.465349}%
\pgfsetlinewidth{1.003750pt}%
\definecolor{currentstroke}{rgb}{0.121569,0.466667,0.705882}%
\pgfsetstrokecolor{currentstroke}%
\pgfsetstrokeopacity{0.465349}%
\pgfsetdash{}{0pt}%
\pgfpathmoveto{\pgfqpoint{2.483870in}{3.118879in}}%
\pgfpathcurveto{\pgfqpoint{2.492106in}{3.118879in}}{\pgfqpoint{2.500006in}{3.122151in}}{\pgfqpoint{2.505830in}{3.127975in}}%
\pgfpathcurveto{\pgfqpoint{2.511654in}{3.133799in}}{\pgfqpoint{2.514926in}{3.141699in}}{\pgfqpoint{2.514926in}{3.149935in}}%
\pgfpathcurveto{\pgfqpoint{2.514926in}{3.158171in}}{\pgfqpoint{2.511654in}{3.166071in}}{\pgfqpoint{2.505830in}{3.171895in}}%
\pgfpathcurveto{\pgfqpoint{2.500006in}{3.177719in}}{\pgfqpoint{2.492106in}{3.180992in}}{\pgfqpoint{2.483870in}{3.180992in}}%
\pgfpathcurveto{\pgfqpoint{2.475633in}{3.180992in}}{\pgfqpoint{2.467733in}{3.177719in}}{\pgfqpoint{2.461909in}{3.171895in}}%
\pgfpathcurveto{\pgfqpoint{2.456085in}{3.166071in}}{\pgfqpoint{2.452813in}{3.158171in}}{\pgfqpoint{2.452813in}{3.149935in}}%
\pgfpathcurveto{\pgfqpoint{2.452813in}{3.141699in}}{\pgfqpoint{2.456085in}{3.133799in}}{\pgfqpoint{2.461909in}{3.127975in}}%
\pgfpathcurveto{\pgfqpoint{2.467733in}{3.122151in}}{\pgfqpoint{2.475633in}{3.118879in}}{\pgfqpoint{2.483870in}{3.118879in}}%
\pgfpathclose%
\pgfusepath{stroke,fill}%
\end{pgfscope}%
\begin{pgfscope}%
\pgfpathrectangle{\pgfqpoint{0.100000in}{0.220728in}}{\pgfqpoint{3.696000in}{3.696000in}}%
\pgfusepath{clip}%
\pgfsetbuttcap%
\pgfsetroundjoin%
\definecolor{currentfill}{rgb}{0.121569,0.466667,0.705882}%
\pgfsetfillcolor{currentfill}%
\pgfsetfillopacity{0.466558}%
\pgfsetlinewidth{1.003750pt}%
\definecolor{currentstroke}{rgb}{0.121569,0.466667,0.705882}%
\pgfsetstrokecolor{currentstroke}%
\pgfsetstrokeopacity{0.466558}%
\pgfsetdash{}{0pt}%
\pgfpathmoveto{\pgfqpoint{2.487084in}{3.118468in}}%
\pgfpathcurveto{\pgfqpoint{2.495321in}{3.118468in}}{\pgfqpoint{2.503221in}{3.121740in}}{\pgfqpoint{2.509045in}{3.127564in}}%
\pgfpathcurveto{\pgfqpoint{2.514869in}{3.133388in}}{\pgfqpoint{2.518141in}{3.141288in}}{\pgfqpoint{2.518141in}{3.149524in}}%
\pgfpathcurveto{\pgfqpoint{2.518141in}{3.157760in}}{\pgfqpoint{2.514869in}{3.165661in}}{\pgfqpoint{2.509045in}{3.171484in}}%
\pgfpathcurveto{\pgfqpoint{2.503221in}{3.177308in}}{\pgfqpoint{2.495321in}{3.180581in}}{\pgfqpoint{2.487084in}{3.180581in}}%
\pgfpathcurveto{\pgfqpoint{2.478848in}{3.180581in}}{\pgfqpoint{2.470948in}{3.177308in}}{\pgfqpoint{2.465124in}{3.171484in}}%
\pgfpathcurveto{\pgfqpoint{2.459300in}{3.165661in}}{\pgfqpoint{2.456028in}{3.157760in}}{\pgfqpoint{2.456028in}{3.149524in}}%
\pgfpathcurveto{\pgfqpoint{2.456028in}{3.141288in}}{\pgfqpoint{2.459300in}{3.133388in}}{\pgfqpoint{2.465124in}{3.127564in}}%
\pgfpathcurveto{\pgfqpoint{2.470948in}{3.121740in}}{\pgfqpoint{2.478848in}{3.118468in}}{\pgfqpoint{2.487084in}{3.118468in}}%
\pgfpathclose%
\pgfusepath{stroke,fill}%
\end{pgfscope}%
\begin{pgfscope}%
\pgfpathrectangle{\pgfqpoint{0.100000in}{0.220728in}}{\pgfqpoint{3.696000in}{3.696000in}}%
\pgfusepath{clip}%
\pgfsetbuttcap%
\pgfsetroundjoin%
\definecolor{currentfill}{rgb}{0.121569,0.466667,0.705882}%
\pgfsetfillcolor{currentfill}%
\pgfsetfillopacity{0.467639}%
\pgfsetlinewidth{1.003750pt}%
\definecolor{currentstroke}{rgb}{0.121569,0.466667,0.705882}%
\pgfsetstrokecolor{currentstroke}%
\pgfsetstrokeopacity{0.467639}%
\pgfsetdash{}{0pt}%
\pgfpathmoveto{\pgfqpoint{2.491515in}{3.117957in}}%
\pgfpathcurveto{\pgfqpoint{2.499752in}{3.117957in}}{\pgfqpoint{2.507652in}{3.121229in}}{\pgfqpoint{2.513476in}{3.127053in}}%
\pgfpathcurveto{\pgfqpoint{2.519299in}{3.132877in}}{\pgfqpoint{2.522572in}{3.140777in}}{\pgfqpoint{2.522572in}{3.149014in}}%
\pgfpathcurveto{\pgfqpoint{2.522572in}{3.157250in}}{\pgfqpoint{2.519299in}{3.165150in}}{\pgfqpoint{2.513476in}{3.170974in}}%
\pgfpathcurveto{\pgfqpoint{2.507652in}{3.176798in}}{\pgfqpoint{2.499752in}{3.180070in}}{\pgfqpoint{2.491515in}{3.180070in}}%
\pgfpathcurveto{\pgfqpoint{2.483279in}{3.180070in}}{\pgfqpoint{2.475379in}{3.176798in}}{\pgfqpoint{2.469555in}{3.170974in}}%
\pgfpathcurveto{\pgfqpoint{2.463731in}{3.165150in}}{\pgfqpoint{2.460459in}{3.157250in}}{\pgfqpoint{2.460459in}{3.149014in}}%
\pgfpathcurveto{\pgfqpoint{2.460459in}{3.140777in}}{\pgfqpoint{2.463731in}{3.132877in}}{\pgfqpoint{2.469555in}{3.127053in}}%
\pgfpathcurveto{\pgfqpoint{2.475379in}{3.121229in}}{\pgfqpoint{2.483279in}{3.117957in}}{\pgfqpoint{2.491515in}{3.117957in}}%
\pgfpathclose%
\pgfusepath{stroke,fill}%
\end{pgfscope}%
\begin{pgfscope}%
\pgfpathrectangle{\pgfqpoint{0.100000in}{0.220728in}}{\pgfqpoint{3.696000in}{3.696000in}}%
\pgfusepath{clip}%
\pgfsetbuttcap%
\pgfsetroundjoin%
\definecolor{currentfill}{rgb}{0.121569,0.466667,0.705882}%
\pgfsetfillcolor{currentfill}%
\pgfsetfillopacity{0.469837}%
\pgfsetlinewidth{1.003750pt}%
\definecolor{currentstroke}{rgb}{0.121569,0.466667,0.705882}%
\pgfsetstrokecolor{currentstroke}%
\pgfsetstrokeopacity{0.469837}%
\pgfsetdash{}{0pt}%
\pgfpathmoveto{\pgfqpoint{2.495843in}{3.113426in}}%
\pgfpathcurveto{\pgfqpoint{2.504080in}{3.113426in}}{\pgfqpoint{2.511980in}{3.116699in}}{\pgfqpoint{2.517803in}{3.122523in}}%
\pgfpathcurveto{\pgfqpoint{2.523627in}{3.128346in}}{\pgfqpoint{2.526900in}{3.136247in}}{\pgfqpoint{2.526900in}{3.144483in}}%
\pgfpathcurveto{\pgfqpoint{2.526900in}{3.152719in}}{\pgfqpoint{2.523627in}{3.160619in}}{\pgfqpoint{2.517803in}{3.166443in}}%
\pgfpathcurveto{\pgfqpoint{2.511980in}{3.172267in}}{\pgfqpoint{2.504080in}{3.175539in}}{\pgfqpoint{2.495843in}{3.175539in}}%
\pgfpathcurveto{\pgfqpoint{2.487607in}{3.175539in}}{\pgfqpoint{2.479707in}{3.172267in}}{\pgfqpoint{2.473883in}{3.166443in}}%
\pgfpathcurveto{\pgfqpoint{2.468059in}{3.160619in}}{\pgfqpoint{2.464787in}{3.152719in}}{\pgfqpoint{2.464787in}{3.144483in}}%
\pgfpathcurveto{\pgfqpoint{2.464787in}{3.136247in}}{\pgfqpoint{2.468059in}{3.128346in}}{\pgfqpoint{2.473883in}{3.122523in}}%
\pgfpathcurveto{\pgfqpoint{2.479707in}{3.116699in}}{\pgfqpoint{2.487607in}{3.113426in}}{\pgfqpoint{2.495843in}{3.113426in}}%
\pgfpathclose%
\pgfusepath{stroke,fill}%
\end{pgfscope}%
\begin{pgfscope}%
\pgfpathrectangle{\pgfqpoint{0.100000in}{0.220728in}}{\pgfqpoint{3.696000in}{3.696000in}}%
\pgfusepath{clip}%
\pgfsetbuttcap%
\pgfsetroundjoin%
\definecolor{currentfill}{rgb}{0.121569,0.466667,0.705882}%
\pgfsetfillcolor{currentfill}%
\pgfsetfillopacity{0.471806}%
\pgfsetlinewidth{1.003750pt}%
\definecolor{currentstroke}{rgb}{0.121569,0.466667,0.705882}%
\pgfsetstrokecolor{currentstroke}%
\pgfsetstrokeopacity{0.471806}%
\pgfsetdash{}{0pt}%
\pgfpathmoveto{\pgfqpoint{2.502892in}{3.111327in}}%
\pgfpathcurveto{\pgfqpoint{2.511128in}{3.111327in}}{\pgfqpoint{2.519028in}{3.114599in}}{\pgfqpoint{2.524852in}{3.120423in}}%
\pgfpathcurveto{\pgfqpoint{2.530676in}{3.126247in}}{\pgfqpoint{2.533948in}{3.134147in}}{\pgfqpoint{2.533948in}{3.142383in}}%
\pgfpathcurveto{\pgfqpoint{2.533948in}{3.150619in}}{\pgfqpoint{2.530676in}{3.158519in}}{\pgfqpoint{2.524852in}{3.164343in}}%
\pgfpathcurveto{\pgfqpoint{2.519028in}{3.170167in}}{\pgfqpoint{2.511128in}{3.173440in}}{\pgfqpoint{2.502892in}{3.173440in}}%
\pgfpathcurveto{\pgfqpoint{2.494655in}{3.173440in}}{\pgfqpoint{2.486755in}{3.170167in}}{\pgfqpoint{2.480931in}{3.164343in}}%
\pgfpathcurveto{\pgfqpoint{2.475108in}{3.158519in}}{\pgfqpoint{2.471835in}{3.150619in}}{\pgfqpoint{2.471835in}{3.142383in}}%
\pgfpathcurveto{\pgfqpoint{2.471835in}{3.134147in}}{\pgfqpoint{2.475108in}{3.126247in}}{\pgfqpoint{2.480931in}{3.120423in}}%
\pgfpathcurveto{\pgfqpoint{2.486755in}{3.114599in}}{\pgfqpoint{2.494655in}{3.111327in}}{\pgfqpoint{2.502892in}{3.111327in}}%
\pgfpathclose%
\pgfusepath{stroke,fill}%
\end{pgfscope}%
\begin{pgfscope}%
\pgfpathrectangle{\pgfqpoint{0.100000in}{0.220728in}}{\pgfqpoint{3.696000in}{3.696000in}}%
\pgfusepath{clip}%
\pgfsetbuttcap%
\pgfsetroundjoin%
\definecolor{currentfill}{rgb}{0.121569,0.466667,0.705882}%
\pgfsetfillcolor{currentfill}%
\pgfsetfillopacity{0.472001}%
\pgfsetlinewidth{1.003750pt}%
\definecolor{currentstroke}{rgb}{0.121569,0.466667,0.705882}%
\pgfsetstrokecolor{currentstroke}%
\pgfsetstrokeopacity{0.472001}%
\pgfsetdash{}{0pt}%
\pgfpathmoveto{\pgfqpoint{1.282704in}{2.166788in}}%
\pgfpathcurveto{\pgfqpoint{1.290940in}{2.166788in}}{\pgfqpoint{1.298841in}{2.170060in}}{\pgfqpoint{1.304664in}{2.175884in}}%
\pgfpathcurveto{\pgfqpoint{1.310488in}{2.181708in}}{\pgfqpoint{1.313761in}{2.189608in}}{\pgfqpoint{1.313761in}{2.197844in}}%
\pgfpathcurveto{\pgfqpoint{1.313761in}{2.206080in}}{\pgfqpoint{1.310488in}{2.213980in}}{\pgfqpoint{1.304664in}{2.219804in}}%
\pgfpathcurveto{\pgfqpoint{1.298841in}{2.225628in}}{\pgfqpoint{1.290940in}{2.228901in}}{\pgfqpoint{1.282704in}{2.228901in}}%
\pgfpathcurveto{\pgfqpoint{1.274468in}{2.228901in}}{\pgfqpoint{1.266568in}{2.225628in}}{\pgfqpoint{1.260744in}{2.219804in}}%
\pgfpathcurveto{\pgfqpoint{1.254920in}{2.213980in}}{\pgfqpoint{1.251648in}{2.206080in}}{\pgfqpoint{1.251648in}{2.197844in}}%
\pgfpathcurveto{\pgfqpoint{1.251648in}{2.189608in}}{\pgfqpoint{1.254920in}{2.181708in}}{\pgfqpoint{1.260744in}{2.175884in}}%
\pgfpathcurveto{\pgfqpoint{1.266568in}{2.170060in}}{\pgfqpoint{1.274468in}{2.166788in}}{\pgfqpoint{1.282704in}{2.166788in}}%
\pgfpathclose%
\pgfusepath{stroke,fill}%
\end{pgfscope}%
\begin{pgfscope}%
\pgfpathrectangle{\pgfqpoint{0.100000in}{0.220728in}}{\pgfqpoint{3.696000in}{3.696000in}}%
\pgfusepath{clip}%
\pgfsetbuttcap%
\pgfsetroundjoin%
\definecolor{currentfill}{rgb}{0.121569,0.466667,0.705882}%
\pgfsetfillcolor{currentfill}%
\pgfsetfillopacity{0.472588}%
\pgfsetlinewidth{1.003750pt}%
\definecolor{currentstroke}{rgb}{0.121569,0.466667,0.705882}%
\pgfsetstrokecolor{currentstroke}%
\pgfsetstrokeopacity{0.472588}%
\pgfsetdash{}{0pt}%
\pgfpathmoveto{\pgfqpoint{2.507119in}{3.110013in}}%
\pgfpathcurveto{\pgfqpoint{2.515356in}{3.110013in}}{\pgfqpoint{2.523256in}{3.113285in}}{\pgfqpoint{2.529080in}{3.119109in}}%
\pgfpathcurveto{\pgfqpoint{2.534904in}{3.124933in}}{\pgfqpoint{2.538176in}{3.132833in}}{\pgfqpoint{2.538176in}{3.141069in}}%
\pgfpathcurveto{\pgfqpoint{2.538176in}{3.149306in}}{\pgfqpoint{2.534904in}{3.157206in}}{\pgfqpoint{2.529080in}{3.163030in}}%
\pgfpathcurveto{\pgfqpoint{2.523256in}{3.168854in}}{\pgfqpoint{2.515356in}{3.172126in}}{\pgfqpoint{2.507119in}{3.172126in}}%
\pgfpathcurveto{\pgfqpoint{2.498883in}{3.172126in}}{\pgfqpoint{2.490983in}{3.168854in}}{\pgfqpoint{2.485159in}{3.163030in}}%
\pgfpathcurveto{\pgfqpoint{2.479335in}{3.157206in}}{\pgfqpoint{2.476063in}{3.149306in}}{\pgfqpoint{2.476063in}{3.141069in}}%
\pgfpathcurveto{\pgfqpoint{2.476063in}{3.132833in}}{\pgfqpoint{2.479335in}{3.124933in}}{\pgfqpoint{2.485159in}{3.119109in}}%
\pgfpathcurveto{\pgfqpoint{2.490983in}{3.113285in}}{\pgfqpoint{2.498883in}{3.110013in}}{\pgfqpoint{2.507119in}{3.110013in}}%
\pgfpathclose%
\pgfusepath{stroke,fill}%
\end{pgfscope}%
\begin{pgfscope}%
\pgfpathrectangle{\pgfqpoint{0.100000in}{0.220728in}}{\pgfqpoint{3.696000in}{3.696000in}}%
\pgfusepath{clip}%
\pgfsetbuttcap%
\pgfsetroundjoin%
\definecolor{currentfill}{rgb}{0.121569,0.466667,0.705882}%
\pgfsetfillcolor{currentfill}%
\pgfsetfillopacity{0.474222}%
\pgfsetlinewidth{1.003750pt}%
\definecolor{currentstroke}{rgb}{0.121569,0.466667,0.705882}%
\pgfsetstrokecolor{currentstroke}%
\pgfsetstrokeopacity{0.474222}%
\pgfsetdash{}{0pt}%
\pgfpathmoveto{\pgfqpoint{2.511651in}{3.109051in}}%
\pgfpathcurveto{\pgfqpoint{2.519887in}{3.109051in}}{\pgfqpoint{2.527787in}{3.112323in}}{\pgfqpoint{2.533611in}{3.118147in}}%
\pgfpathcurveto{\pgfqpoint{2.539435in}{3.123971in}}{\pgfqpoint{2.542707in}{3.131871in}}{\pgfqpoint{2.542707in}{3.140107in}}%
\pgfpathcurveto{\pgfqpoint{2.542707in}{3.148344in}}{\pgfqpoint{2.539435in}{3.156244in}}{\pgfqpoint{2.533611in}{3.162067in}}%
\pgfpathcurveto{\pgfqpoint{2.527787in}{3.167891in}}{\pgfqpoint{2.519887in}{3.171164in}}{\pgfqpoint{2.511651in}{3.171164in}}%
\pgfpathcurveto{\pgfqpoint{2.503414in}{3.171164in}}{\pgfqpoint{2.495514in}{3.167891in}}{\pgfqpoint{2.489690in}{3.162067in}}%
\pgfpathcurveto{\pgfqpoint{2.483866in}{3.156244in}}{\pgfqpoint{2.480594in}{3.148344in}}{\pgfqpoint{2.480594in}{3.140107in}}%
\pgfpathcurveto{\pgfqpoint{2.480594in}{3.131871in}}{\pgfqpoint{2.483866in}{3.123971in}}{\pgfqpoint{2.489690in}{3.118147in}}%
\pgfpathcurveto{\pgfqpoint{2.495514in}{3.112323in}}{\pgfqpoint{2.503414in}{3.109051in}}{\pgfqpoint{2.511651in}{3.109051in}}%
\pgfpathclose%
\pgfusepath{stroke,fill}%
\end{pgfscope}%
\begin{pgfscope}%
\pgfpathrectangle{\pgfqpoint{0.100000in}{0.220728in}}{\pgfqpoint{3.696000in}{3.696000in}}%
\pgfusepath{clip}%
\pgfsetbuttcap%
\pgfsetroundjoin%
\definecolor{currentfill}{rgb}{0.121569,0.466667,0.705882}%
\pgfsetfillcolor{currentfill}%
\pgfsetfillopacity{0.474655}%
\pgfsetlinewidth{1.003750pt}%
\definecolor{currentstroke}{rgb}{0.121569,0.466667,0.705882}%
\pgfsetstrokecolor{currentstroke}%
\pgfsetstrokeopacity{0.474655}%
\pgfsetdash{}{0pt}%
\pgfpathmoveto{\pgfqpoint{2.514571in}{3.107857in}}%
\pgfpathcurveto{\pgfqpoint{2.522807in}{3.107857in}}{\pgfqpoint{2.530707in}{3.111129in}}{\pgfqpoint{2.536531in}{3.116953in}}%
\pgfpathcurveto{\pgfqpoint{2.542355in}{3.122777in}}{\pgfqpoint{2.545627in}{3.130677in}}{\pgfqpoint{2.545627in}{3.138914in}}%
\pgfpathcurveto{\pgfqpoint{2.545627in}{3.147150in}}{\pgfqpoint{2.542355in}{3.155050in}}{\pgfqpoint{2.536531in}{3.160874in}}%
\pgfpathcurveto{\pgfqpoint{2.530707in}{3.166698in}}{\pgfqpoint{2.522807in}{3.169970in}}{\pgfqpoint{2.514571in}{3.169970in}}%
\pgfpathcurveto{\pgfqpoint{2.506334in}{3.169970in}}{\pgfqpoint{2.498434in}{3.166698in}}{\pgfqpoint{2.492610in}{3.160874in}}%
\pgfpathcurveto{\pgfqpoint{2.486787in}{3.155050in}}{\pgfqpoint{2.483514in}{3.147150in}}{\pgfqpoint{2.483514in}{3.138914in}}%
\pgfpathcurveto{\pgfqpoint{2.483514in}{3.130677in}}{\pgfqpoint{2.486787in}{3.122777in}}{\pgfqpoint{2.492610in}{3.116953in}}%
\pgfpathcurveto{\pgfqpoint{2.498434in}{3.111129in}}{\pgfqpoint{2.506334in}{3.107857in}}{\pgfqpoint{2.514571in}{3.107857in}}%
\pgfpathclose%
\pgfusepath{stroke,fill}%
\end{pgfscope}%
\begin{pgfscope}%
\pgfpathrectangle{\pgfqpoint{0.100000in}{0.220728in}}{\pgfqpoint{3.696000in}{3.696000in}}%
\pgfusepath{clip}%
\pgfsetbuttcap%
\pgfsetroundjoin%
\definecolor{currentfill}{rgb}{0.121569,0.466667,0.705882}%
\pgfsetfillcolor{currentfill}%
\pgfsetfillopacity{0.475629}%
\pgfsetlinewidth{1.003750pt}%
\definecolor{currentstroke}{rgb}{0.121569,0.466667,0.705882}%
\pgfsetstrokecolor{currentstroke}%
\pgfsetstrokeopacity{0.475629}%
\pgfsetdash{}{0pt}%
\pgfpathmoveto{\pgfqpoint{2.517916in}{3.107076in}}%
\pgfpathcurveto{\pgfqpoint{2.526153in}{3.107076in}}{\pgfqpoint{2.534053in}{3.110349in}}{\pgfqpoint{2.539876in}{3.116173in}}%
\pgfpathcurveto{\pgfqpoint{2.545700in}{3.121997in}}{\pgfqpoint{2.548973in}{3.129897in}}{\pgfqpoint{2.548973in}{3.138133in}}%
\pgfpathcurveto{\pgfqpoint{2.548973in}{3.146369in}}{\pgfqpoint{2.545700in}{3.154269in}}{\pgfqpoint{2.539876in}{3.160093in}}%
\pgfpathcurveto{\pgfqpoint{2.534053in}{3.165917in}}{\pgfqpoint{2.526153in}{3.169189in}}{\pgfqpoint{2.517916in}{3.169189in}}%
\pgfpathcurveto{\pgfqpoint{2.509680in}{3.169189in}}{\pgfqpoint{2.501780in}{3.165917in}}{\pgfqpoint{2.495956in}{3.160093in}}%
\pgfpathcurveto{\pgfqpoint{2.490132in}{3.154269in}}{\pgfqpoint{2.486860in}{3.146369in}}{\pgfqpoint{2.486860in}{3.138133in}}%
\pgfpathcurveto{\pgfqpoint{2.486860in}{3.129897in}}{\pgfqpoint{2.490132in}{3.121997in}}{\pgfqpoint{2.495956in}{3.116173in}}%
\pgfpathcurveto{\pgfqpoint{2.501780in}{3.110349in}}{\pgfqpoint{2.509680in}{3.107076in}}{\pgfqpoint{2.517916in}{3.107076in}}%
\pgfpathclose%
\pgfusepath{stroke,fill}%
\end{pgfscope}%
\begin{pgfscope}%
\pgfpathrectangle{\pgfqpoint{0.100000in}{0.220728in}}{\pgfqpoint{3.696000in}{3.696000in}}%
\pgfusepath{clip}%
\pgfsetbuttcap%
\pgfsetroundjoin%
\definecolor{currentfill}{rgb}{0.121569,0.466667,0.705882}%
\pgfsetfillcolor{currentfill}%
\pgfsetfillopacity{0.476340}%
\pgfsetlinewidth{1.003750pt}%
\definecolor{currentstroke}{rgb}{0.121569,0.466667,0.705882}%
\pgfsetstrokecolor{currentstroke}%
\pgfsetstrokeopacity{0.476340}%
\pgfsetdash{}{0pt}%
\pgfpathmoveto{\pgfqpoint{2.519609in}{3.107007in}}%
\pgfpathcurveto{\pgfqpoint{2.527846in}{3.107007in}}{\pgfqpoint{2.535746in}{3.110279in}}{\pgfqpoint{2.541570in}{3.116103in}}%
\pgfpathcurveto{\pgfqpoint{2.547394in}{3.121927in}}{\pgfqpoint{2.550666in}{3.129827in}}{\pgfqpoint{2.550666in}{3.138063in}}%
\pgfpathcurveto{\pgfqpoint{2.550666in}{3.146300in}}{\pgfqpoint{2.547394in}{3.154200in}}{\pgfqpoint{2.541570in}{3.160023in}}%
\pgfpathcurveto{\pgfqpoint{2.535746in}{3.165847in}}{\pgfqpoint{2.527846in}{3.169120in}}{\pgfqpoint{2.519609in}{3.169120in}}%
\pgfpathcurveto{\pgfqpoint{2.511373in}{3.169120in}}{\pgfqpoint{2.503473in}{3.165847in}}{\pgfqpoint{2.497649in}{3.160023in}}%
\pgfpathcurveto{\pgfqpoint{2.491825in}{3.154200in}}{\pgfqpoint{2.488553in}{3.146300in}}{\pgfqpoint{2.488553in}{3.138063in}}%
\pgfpathcurveto{\pgfqpoint{2.488553in}{3.129827in}}{\pgfqpoint{2.491825in}{3.121927in}}{\pgfqpoint{2.497649in}{3.116103in}}%
\pgfpathcurveto{\pgfqpoint{2.503473in}{3.110279in}}{\pgfqpoint{2.511373in}{3.107007in}}{\pgfqpoint{2.519609in}{3.107007in}}%
\pgfpathclose%
\pgfusepath{stroke,fill}%
\end{pgfscope}%
\begin{pgfscope}%
\pgfpathrectangle{\pgfqpoint{0.100000in}{0.220728in}}{\pgfqpoint{3.696000in}{3.696000in}}%
\pgfusepath{clip}%
\pgfsetbuttcap%
\pgfsetroundjoin%
\definecolor{currentfill}{rgb}{0.121569,0.466667,0.705882}%
\pgfsetfillcolor{currentfill}%
\pgfsetfillopacity{0.476764}%
\pgfsetlinewidth{1.003750pt}%
\definecolor{currentstroke}{rgb}{0.121569,0.466667,0.705882}%
\pgfsetstrokecolor{currentstroke}%
\pgfsetstrokeopacity{0.476764}%
\pgfsetdash{}{0pt}%
\pgfpathmoveto{\pgfqpoint{2.522462in}{3.105604in}}%
\pgfpathcurveto{\pgfqpoint{2.530699in}{3.105604in}}{\pgfqpoint{2.538599in}{3.108876in}}{\pgfqpoint{2.544423in}{3.114700in}}%
\pgfpathcurveto{\pgfqpoint{2.550246in}{3.120524in}}{\pgfqpoint{2.553519in}{3.128424in}}{\pgfqpoint{2.553519in}{3.136661in}}%
\pgfpathcurveto{\pgfqpoint{2.553519in}{3.144897in}}{\pgfqpoint{2.550246in}{3.152797in}}{\pgfqpoint{2.544423in}{3.158621in}}%
\pgfpathcurveto{\pgfqpoint{2.538599in}{3.164445in}}{\pgfqpoint{2.530699in}{3.167717in}}{\pgfqpoint{2.522462in}{3.167717in}}%
\pgfpathcurveto{\pgfqpoint{2.514226in}{3.167717in}}{\pgfqpoint{2.506326in}{3.164445in}}{\pgfqpoint{2.500502in}{3.158621in}}%
\pgfpathcurveto{\pgfqpoint{2.494678in}{3.152797in}}{\pgfqpoint{2.491406in}{3.144897in}}{\pgfqpoint{2.491406in}{3.136661in}}%
\pgfpathcurveto{\pgfqpoint{2.491406in}{3.128424in}}{\pgfqpoint{2.494678in}{3.120524in}}{\pgfqpoint{2.500502in}{3.114700in}}%
\pgfpathcurveto{\pgfqpoint{2.506326in}{3.108876in}}{\pgfqpoint{2.514226in}{3.105604in}}{\pgfqpoint{2.522462in}{3.105604in}}%
\pgfpathclose%
\pgfusepath{stroke,fill}%
\end{pgfscope}%
\begin{pgfscope}%
\pgfpathrectangle{\pgfqpoint{0.100000in}{0.220728in}}{\pgfqpoint{3.696000in}{3.696000in}}%
\pgfusepath{clip}%
\pgfsetbuttcap%
\pgfsetroundjoin%
\definecolor{currentfill}{rgb}{0.121569,0.466667,0.705882}%
\pgfsetfillcolor{currentfill}%
\pgfsetfillopacity{0.476800}%
\pgfsetlinewidth{1.003750pt}%
\definecolor{currentstroke}{rgb}{0.121569,0.466667,0.705882}%
\pgfsetstrokecolor{currentstroke}%
\pgfsetstrokeopacity{0.476800}%
\pgfsetdash{}{0pt}%
\pgfpathmoveto{\pgfqpoint{1.259171in}{2.133767in}}%
\pgfpathcurveto{\pgfqpoint{1.267408in}{2.133767in}}{\pgfqpoint{1.275308in}{2.137039in}}{\pgfqpoint{1.281132in}{2.142863in}}%
\pgfpathcurveto{\pgfqpoint{1.286955in}{2.148687in}}{\pgfqpoint{1.290228in}{2.156587in}}{\pgfqpoint{1.290228in}{2.164824in}}%
\pgfpathcurveto{\pgfqpoint{1.290228in}{2.173060in}}{\pgfqpoint{1.286955in}{2.180960in}}{\pgfqpoint{1.281132in}{2.186784in}}%
\pgfpathcurveto{\pgfqpoint{1.275308in}{2.192608in}}{\pgfqpoint{1.267408in}{2.195880in}}{\pgfqpoint{1.259171in}{2.195880in}}%
\pgfpathcurveto{\pgfqpoint{1.250935in}{2.195880in}}{\pgfqpoint{1.243035in}{2.192608in}}{\pgfqpoint{1.237211in}{2.186784in}}%
\pgfpathcurveto{\pgfqpoint{1.231387in}{2.180960in}}{\pgfqpoint{1.228115in}{2.173060in}}{\pgfqpoint{1.228115in}{2.164824in}}%
\pgfpathcurveto{\pgfqpoint{1.228115in}{2.156587in}}{\pgfqpoint{1.231387in}{2.148687in}}{\pgfqpoint{1.237211in}{2.142863in}}%
\pgfpathcurveto{\pgfqpoint{1.243035in}{2.137039in}}{\pgfqpoint{1.250935in}{2.133767in}}{\pgfqpoint{1.259171in}{2.133767in}}%
\pgfpathclose%
\pgfusepath{stroke,fill}%
\end{pgfscope}%
\begin{pgfscope}%
\pgfpathrectangle{\pgfqpoint{0.100000in}{0.220728in}}{\pgfqpoint{3.696000in}{3.696000in}}%
\pgfusepath{clip}%
\pgfsetbuttcap%
\pgfsetroundjoin%
\definecolor{currentfill}{rgb}{0.121569,0.466667,0.705882}%
\pgfsetfillcolor{currentfill}%
\pgfsetfillopacity{0.477789}%
\pgfsetlinewidth{1.003750pt}%
\definecolor{currentstroke}{rgb}{0.121569,0.466667,0.705882}%
\pgfsetstrokecolor{currentstroke}%
\pgfsetstrokeopacity{0.477789}%
\pgfsetdash{}{0pt}%
\pgfpathmoveto{\pgfqpoint{2.525590in}{3.104223in}}%
\pgfpathcurveto{\pgfqpoint{2.533826in}{3.104223in}}{\pgfqpoint{2.541726in}{3.107496in}}{\pgfqpoint{2.547550in}{3.113320in}}%
\pgfpathcurveto{\pgfqpoint{2.553374in}{3.119144in}}{\pgfqpoint{2.556647in}{3.127044in}}{\pgfqpoint{2.556647in}{3.135280in}}%
\pgfpathcurveto{\pgfqpoint{2.556647in}{3.143516in}}{\pgfqpoint{2.553374in}{3.151416in}}{\pgfqpoint{2.547550in}{3.157240in}}%
\pgfpathcurveto{\pgfqpoint{2.541726in}{3.163064in}}{\pgfqpoint{2.533826in}{3.166336in}}{\pgfqpoint{2.525590in}{3.166336in}}%
\pgfpathcurveto{\pgfqpoint{2.517354in}{3.166336in}}{\pgfqpoint{2.509454in}{3.163064in}}{\pgfqpoint{2.503630in}{3.157240in}}%
\pgfpathcurveto{\pgfqpoint{2.497806in}{3.151416in}}{\pgfqpoint{2.494534in}{3.143516in}}{\pgfqpoint{2.494534in}{3.135280in}}%
\pgfpathcurveto{\pgfqpoint{2.494534in}{3.127044in}}{\pgfqpoint{2.497806in}{3.119144in}}{\pgfqpoint{2.503630in}{3.113320in}}%
\pgfpathcurveto{\pgfqpoint{2.509454in}{3.107496in}}{\pgfqpoint{2.517354in}{3.104223in}}{\pgfqpoint{2.525590in}{3.104223in}}%
\pgfpathclose%
\pgfusepath{stroke,fill}%
\end{pgfscope}%
\begin{pgfscope}%
\pgfpathrectangle{\pgfqpoint{0.100000in}{0.220728in}}{\pgfqpoint{3.696000in}{3.696000in}}%
\pgfusepath{clip}%
\pgfsetbuttcap%
\pgfsetroundjoin%
\definecolor{currentfill}{rgb}{0.121569,0.466667,0.705882}%
\pgfsetfillcolor{currentfill}%
\pgfsetfillopacity{0.477921}%
\pgfsetlinewidth{1.003750pt}%
\definecolor{currentstroke}{rgb}{0.121569,0.466667,0.705882}%
\pgfsetstrokecolor{currentstroke}%
\pgfsetstrokeopacity{0.477921}%
\pgfsetdash{}{0pt}%
\pgfpathmoveto{\pgfqpoint{2.527846in}{3.103603in}}%
\pgfpathcurveto{\pgfqpoint{2.536082in}{3.103603in}}{\pgfqpoint{2.543982in}{3.106876in}}{\pgfqpoint{2.549806in}{3.112700in}}%
\pgfpathcurveto{\pgfqpoint{2.555630in}{3.118524in}}{\pgfqpoint{2.558902in}{3.126424in}}{\pgfqpoint{2.558902in}{3.134660in}}%
\pgfpathcurveto{\pgfqpoint{2.558902in}{3.142896in}}{\pgfqpoint{2.555630in}{3.150796in}}{\pgfqpoint{2.549806in}{3.156620in}}%
\pgfpathcurveto{\pgfqpoint{2.543982in}{3.162444in}}{\pgfqpoint{2.536082in}{3.165716in}}{\pgfqpoint{2.527846in}{3.165716in}}%
\pgfpathcurveto{\pgfqpoint{2.519609in}{3.165716in}}{\pgfqpoint{2.511709in}{3.162444in}}{\pgfqpoint{2.505885in}{3.156620in}}%
\pgfpathcurveto{\pgfqpoint{2.500061in}{3.150796in}}{\pgfqpoint{2.496789in}{3.142896in}}{\pgfqpoint{2.496789in}{3.134660in}}%
\pgfpathcurveto{\pgfqpoint{2.496789in}{3.126424in}}{\pgfqpoint{2.500061in}{3.118524in}}{\pgfqpoint{2.505885in}{3.112700in}}%
\pgfpathcurveto{\pgfqpoint{2.511709in}{3.106876in}}{\pgfqpoint{2.519609in}{3.103603in}}{\pgfqpoint{2.527846in}{3.103603in}}%
\pgfpathclose%
\pgfusepath{stroke,fill}%
\end{pgfscope}%
\begin{pgfscope}%
\pgfpathrectangle{\pgfqpoint{0.100000in}{0.220728in}}{\pgfqpoint{3.696000in}{3.696000in}}%
\pgfusepath{clip}%
\pgfsetbuttcap%
\pgfsetroundjoin%
\definecolor{currentfill}{rgb}{0.121569,0.466667,0.705882}%
\pgfsetfillcolor{currentfill}%
\pgfsetfillopacity{0.478537}%
\pgfsetlinewidth{1.003750pt}%
\definecolor{currentstroke}{rgb}{0.121569,0.466667,0.705882}%
\pgfsetstrokecolor{currentstroke}%
\pgfsetstrokeopacity{0.478537}%
\pgfsetdash{}{0pt}%
\pgfpathmoveto{\pgfqpoint{2.530666in}{3.103083in}}%
\pgfpathcurveto{\pgfqpoint{2.538903in}{3.103083in}}{\pgfqpoint{2.546803in}{3.106355in}}{\pgfqpoint{2.552627in}{3.112179in}}%
\pgfpathcurveto{\pgfqpoint{2.558451in}{3.118003in}}{\pgfqpoint{2.561723in}{3.125903in}}{\pgfqpoint{2.561723in}{3.134139in}}%
\pgfpathcurveto{\pgfqpoint{2.561723in}{3.142376in}}{\pgfqpoint{2.558451in}{3.150276in}}{\pgfqpoint{2.552627in}{3.156099in}}%
\pgfpathcurveto{\pgfqpoint{2.546803in}{3.161923in}}{\pgfqpoint{2.538903in}{3.165196in}}{\pgfqpoint{2.530666in}{3.165196in}}%
\pgfpathcurveto{\pgfqpoint{2.522430in}{3.165196in}}{\pgfqpoint{2.514530in}{3.161923in}}{\pgfqpoint{2.508706in}{3.156099in}}%
\pgfpathcurveto{\pgfqpoint{2.502882in}{3.150276in}}{\pgfqpoint{2.499610in}{3.142376in}}{\pgfqpoint{2.499610in}{3.134139in}}%
\pgfpathcurveto{\pgfqpoint{2.499610in}{3.125903in}}{\pgfqpoint{2.502882in}{3.118003in}}{\pgfqpoint{2.508706in}{3.112179in}}%
\pgfpathcurveto{\pgfqpoint{2.514530in}{3.106355in}}{\pgfqpoint{2.522430in}{3.103083in}}{\pgfqpoint{2.530666in}{3.103083in}}%
\pgfpathclose%
\pgfusepath{stroke,fill}%
\end{pgfscope}%
\begin{pgfscope}%
\pgfpathrectangle{\pgfqpoint{0.100000in}{0.220728in}}{\pgfqpoint{3.696000in}{3.696000in}}%
\pgfusepath{clip}%
\pgfsetbuttcap%
\pgfsetroundjoin%
\definecolor{currentfill}{rgb}{0.121569,0.466667,0.705882}%
\pgfsetfillcolor{currentfill}%
\pgfsetfillopacity{0.479763}%
\pgfsetlinewidth{1.003750pt}%
\definecolor{currentstroke}{rgb}{0.121569,0.466667,0.705882}%
\pgfsetstrokecolor{currentstroke}%
\pgfsetstrokeopacity{0.479763}%
\pgfsetdash{}{0pt}%
\pgfpathmoveto{\pgfqpoint{2.534573in}{3.102613in}}%
\pgfpathcurveto{\pgfqpoint{2.542809in}{3.102613in}}{\pgfqpoint{2.550709in}{3.105885in}}{\pgfqpoint{2.556533in}{3.111709in}}%
\pgfpathcurveto{\pgfqpoint{2.562357in}{3.117533in}}{\pgfqpoint{2.565630in}{3.125433in}}{\pgfqpoint{2.565630in}{3.133669in}}%
\pgfpathcurveto{\pgfqpoint{2.565630in}{3.141906in}}{\pgfqpoint{2.562357in}{3.149806in}}{\pgfqpoint{2.556533in}{3.155630in}}%
\pgfpathcurveto{\pgfqpoint{2.550709in}{3.161454in}}{\pgfqpoint{2.542809in}{3.164726in}}{\pgfqpoint{2.534573in}{3.164726in}}%
\pgfpathcurveto{\pgfqpoint{2.526337in}{3.164726in}}{\pgfqpoint{2.518437in}{3.161454in}}{\pgfqpoint{2.512613in}{3.155630in}}%
\pgfpathcurveto{\pgfqpoint{2.506789in}{3.149806in}}{\pgfqpoint{2.503517in}{3.141906in}}{\pgfqpoint{2.503517in}{3.133669in}}%
\pgfpathcurveto{\pgfqpoint{2.503517in}{3.125433in}}{\pgfqpoint{2.506789in}{3.117533in}}{\pgfqpoint{2.512613in}{3.111709in}}%
\pgfpathcurveto{\pgfqpoint{2.518437in}{3.105885in}}{\pgfqpoint{2.526337in}{3.102613in}}{\pgfqpoint{2.534573in}{3.102613in}}%
\pgfpathclose%
\pgfusepath{stroke,fill}%
\end{pgfscope}%
\begin{pgfscope}%
\pgfpathrectangle{\pgfqpoint{0.100000in}{0.220728in}}{\pgfqpoint{3.696000in}{3.696000in}}%
\pgfusepath{clip}%
\pgfsetbuttcap%
\pgfsetroundjoin%
\definecolor{currentfill}{rgb}{0.121569,0.466667,0.705882}%
\pgfsetfillcolor{currentfill}%
\pgfsetfillopacity{0.480461}%
\pgfsetlinewidth{1.003750pt}%
\definecolor{currentstroke}{rgb}{0.121569,0.466667,0.705882}%
\pgfsetstrokecolor{currentstroke}%
\pgfsetstrokeopacity{0.480461}%
\pgfsetdash{}{0pt}%
\pgfpathmoveto{\pgfqpoint{2.542063in}{3.100834in}}%
\pgfpathcurveto{\pgfqpoint{2.550300in}{3.100834in}}{\pgfqpoint{2.558200in}{3.104106in}}{\pgfqpoint{2.564024in}{3.109930in}}%
\pgfpathcurveto{\pgfqpoint{2.569848in}{3.115754in}}{\pgfqpoint{2.573120in}{3.123654in}}{\pgfqpoint{2.573120in}{3.131890in}}%
\pgfpathcurveto{\pgfqpoint{2.573120in}{3.140126in}}{\pgfqpoint{2.569848in}{3.148026in}}{\pgfqpoint{2.564024in}{3.153850in}}%
\pgfpathcurveto{\pgfqpoint{2.558200in}{3.159674in}}{\pgfqpoint{2.550300in}{3.162947in}}{\pgfqpoint{2.542063in}{3.162947in}}%
\pgfpathcurveto{\pgfqpoint{2.533827in}{3.162947in}}{\pgfqpoint{2.525927in}{3.159674in}}{\pgfqpoint{2.520103in}{3.153850in}}%
\pgfpathcurveto{\pgfqpoint{2.514279in}{3.148026in}}{\pgfqpoint{2.511007in}{3.140126in}}{\pgfqpoint{2.511007in}{3.131890in}}%
\pgfpathcurveto{\pgfqpoint{2.511007in}{3.123654in}}{\pgfqpoint{2.514279in}{3.115754in}}{\pgfqpoint{2.520103in}{3.109930in}}%
\pgfpathcurveto{\pgfqpoint{2.525927in}{3.104106in}}{\pgfqpoint{2.533827in}{3.100834in}}{\pgfqpoint{2.542063in}{3.100834in}}%
\pgfpathclose%
\pgfusepath{stroke,fill}%
\end{pgfscope}%
\begin{pgfscope}%
\pgfpathrectangle{\pgfqpoint{0.100000in}{0.220728in}}{\pgfqpoint{3.696000in}{3.696000in}}%
\pgfusepath{clip}%
\pgfsetbuttcap%
\pgfsetroundjoin%
\definecolor{currentfill}{rgb}{0.121569,0.466667,0.705882}%
\pgfsetfillcolor{currentfill}%
\pgfsetfillopacity{0.482546}%
\pgfsetlinewidth{1.003750pt}%
\definecolor{currentstroke}{rgb}{0.121569,0.466667,0.705882}%
\pgfsetstrokecolor{currentstroke}%
\pgfsetstrokeopacity{0.482546}%
\pgfsetdash{}{0pt}%
\pgfpathmoveto{\pgfqpoint{2.549383in}{3.098958in}}%
\pgfpathcurveto{\pgfqpoint{2.557619in}{3.098958in}}{\pgfqpoint{2.565519in}{3.102230in}}{\pgfqpoint{2.571343in}{3.108054in}}%
\pgfpathcurveto{\pgfqpoint{2.577167in}{3.113878in}}{\pgfqpoint{2.580439in}{3.121778in}}{\pgfqpoint{2.580439in}{3.130014in}}%
\pgfpathcurveto{\pgfqpoint{2.580439in}{3.138251in}}{\pgfqpoint{2.577167in}{3.146151in}}{\pgfqpoint{2.571343in}{3.151975in}}%
\pgfpathcurveto{\pgfqpoint{2.565519in}{3.157799in}}{\pgfqpoint{2.557619in}{3.161071in}}{\pgfqpoint{2.549383in}{3.161071in}}%
\pgfpathcurveto{\pgfqpoint{2.541146in}{3.161071in}}{\pgfqpoint{2.533246in}{3.157799in}}{\pgfqpoint{2.527422in}{3.151975in}}%
\pgfpathcurveto{\pgfqpoint{2.521598in}{3.146151in}}{\pgfqpoint{2.518326in}{3.138251in}}{\pgfqpoint{2.518326in}{3.130014in}}%
\pgfpathcurveto{\pgfqpoint{2.518326in}{3.121778in}}{\pgfqpoint{2.521598in}{3.113878in}}{\pgfqpoint{2.527422in}{3.108054in}}%
\pgfpathcurveto{\pgfqpoint{2.533246in}{3.102230in}}{\pgfqpoint{2.541146in}{3.098958in}}{\pgfqpoint{2.549383in}{3.098958in}}%
\pgfpathclose%
\pgfusepath{stroke,fill}%
\end{pgfscope}%
\begin{pgfscope}%
\pgfpathrectangle{\pgfqpoint{0.100000in}{0.220728in}}{\pgfqpoint{3.696000in}{3.696000in}}%
\pgfusepath{clip}%
\pgfsetbuttcap%
\pgfsetroundjoin%
\definecolor{currentfill}{rgb}{0.121569,0.466667,0.705882}%
\pgfsetfillcolor{currentfill}%
\pgfsetfillopacity{0.482783}%
\pgfsetlinewidth{1.003750pt}%
\definecolor{currentstroke}{rgb}{0.121569,0.466667,0.705882}%
\pgfsetstrokecolor{currentstroke}%
\pgfsetstrokeopacity{0.482783}%
\pgfsetdash{}{0pt}%
\pgfpathmoveto{\pgfqpoint{1.253838in}{2.094947in}}%
\pgfpathcurveto{\pgfqpoint{1.262075in}{2.094947in}}{\pgfqpoint{1.269975in}{2.098219in}}{\pgfqpoint{1.275799in}{2.104043in}}%
\pgfpathcurveto{\pgfqpoint{1.281622in}{2.109867in}}{\pgfqpoint{1.284895in}{2.117767in}}{\pgfqpoint{1.284895in}{2.126003in}}%
\pgfpathcurveto{\pgfqpoint{1.284895in}{2.134239in}}{\pgfqpoint{1.281622in}{2.142140in}}{\pgfqpoint{1.275799in}{2.147963in}}%
\pgfpathcurveto{\pgfqpoint{1.269975in}{2.153787in}}{\pgfqpoint{1.262075in}{2.157060in}}{\pgfqpoint{1.253838in}{2.157060in}}%
\pgfpathcurveto{\pgfqpoint{1.245602in}{2.157060in}}{\pgfqpoint{1.237702in}{2.153787in}}{\pgfqpoint{1.231878in}{2.147963in}}%
\pgfpathcurveto{\pgfqpoint{1.226054in}{2.142140in}}{\pgfqpoint{1.222782in}{2.134239in}}{\pgfqpoint{1.222782in}{2.126003in}}%
\pgfpathcurveto{\pgfqpoint{1.222782in}{2.117767in}}{\pgfqpoint{1.226054in}{2.109867in}}{\pgfqpoint{1.231878in}{2.104043in}}%
\pgfpathcurveto{\pgfqpoint{1.237702in}{2.098219in}}{\pgfqpoint{1.245602in}{2.094947in}}{\pgfqpoint{1.253838in}{2.094947in}}%
\pgfpathclose%
\pgfusepath{stroke,fill}%
\end{pgfscope}%
\begin{pgfscope}%
\pgfpathrectangle{\pgfqpoint{0.100000in}{0.220728in}}{\pgfqpoint{3.696000in}{3.696000in}}%
\pgfusepath{clip}%
\pgfsetbuttcap%
\pgfsetroundjoin%
\definecolor{currentfill}{rgb}{0.121569,0.466667,0.705882}%
\pgfsetfillcolor{currentfill}%
\pgfsetfillopacity{0.484695}%
\pgfsetlinewidth{1.003750pt}%
\definecolor{currentstroke}{rgb}{0.121569,0.466667,0.705882}%
\pgfsetstrokecolor{currentstroke}%
\pgfsetstrokeopacity{0.484695}%
\pgfsetdash{}{0pt}%
\pgfpathmoveto{\pgfqpoint{2.557895in}{3.097213in}}%
\pgfpathcurveto{\pgfqpoint{2.566131in}{3.097213in}}{\pgfqpoint{2.574031in}{3.100486in}}{\pgfqpoint{2.579855in}{3.106310in}}%
\pgfpathcurveto{\pgfqpoint{2.585679in}{3.112133in}}{\pgfqpoint{2.588951in}{3.120033in}}{\pgfqpoint{2.588951in}{3.128270in}}%
\pgfpathcurveto{\pgfqpoint{2.588951in}{3.136506in}}{\pgfqpoint{2.585679in}{3.144406in}}{\pgfqpoint{2.579855in}{3.150230in}}%
\pgfpathcurveto{\pgfqpoint{2.574031in}{3.156054in}}{\pgfqpoint{2.566131in}{3.159326in}}{\pgfqpoint{2.557895in}{3.159326in}}%
\pgfpathcurveto{\pgfqpoint{2.549658in}{3.159326in}}{\pgfqpoint{2.541758in}{3.156054in}}{\pgfqpoint{2.535934in}{3.150230in}}%
\pgfpathcurveto{\pgfqpoint{2.530110in}{3.144406in}}{\pgfqpoint{2.526838in}{3.136506in}}{\pgfqpoint{2.526838in}{3.128270in}}%
\pgfpathcurveto{\pgfqpoint{2.526838in}{3.120033in}}{\pgfqpoint{2.530110in}{3.112133in}}{\pgfqpoint{2.535934in}{3.106310in}}%
\pgfpathcurveto{\pgfqpoint{2.541758in}{3.100486in}}{\pgfqpoint{2.549658in}{3.097213in}}{\pgfqpoint{2.557895in}{3.097213in}}%
\pgfpathclose%
\pgfusepath{stroke,fill}%
\end{pgfscope}%
\begin{pgfscope}%
\pgfpathrectangle{\pgfqpoint{0.100000in}{0.220728in}}{\pgfqpoint{3.696000in}{3.696000in}}%
\pgfusepath{clip}%
\pgfsetbuttcap%
\pgfsetroundjoin%
\definecolor{currentfill}{rgb}{0.121569,0.466667,0.705882}%
\pgfsetfillcolor{currentfill}%
\pgfsetfillopacity{0.486225}%
\pgfsetlinewidth{1.003750pt}%
\definecolor{currentstroke}{rgb}{0.121569,0.466667,0.705882}%
\pgfsetstrokecolor{currentstroke}%
\pgfsetstrokeopacity{0.486225}%
\pgfsetdash{}{0pt}%
\pgfpathmoveto{\pgfqpoint{1.231820in}{2.069578in}}%
\pgfpathcurveto{\pgfqpoint{1.240056in}{2.069578in}}{\pgfqpoint{1.247956in}{2.072851in}}{\pgfqpoint{1.253780in}{2.078675in}}%
\pgfpathcurveto{\pgfqpoint{1.259604in}{2.084499in}}{\pgfqpoint{1.262876in}{2.092399in}}{\pgfqpoint{1.262876in}{2.100635in}}%
\pgfpathcurveto{\pgfqpoint{1.262876in}{2.108871in}}{\pgfqpoint{1.259604in}{2.116771in}}{\pgfqpoint{1.253780in}{2.122595in}}%
\pgfpathcurveto{\pgfqpoint{1.247956in}{2.128419in}}{\pgfqpoint{1.240056in}{2.131691in}}{\pgfqpoint{1.231820in}{2.131691in}}%
\pgfpathcurveto{\pgfqpoint{1.223583in}{2.131691in}}{\pgfqpoint{1.215683in}{2.128419in}}{\pgfqpoint{1.209859in}{2.122595in}}%
\pgfpathcurveto{\pgfqpoint{1.204035in}{2.116771in}}{\pgfqpoint{1.200763in}{2.108871in}}{\pgfqpoint{1.200763in}{2.100635in}}%
\pgfpathcurveto{\pgfqpoint{1.200763in}{2.092399in}}{\pgfqpoint{1.204035in}{2.084499in}}{\pgfqpoint{1.209859in}{2.078675in}}%
\pgfpathcurveto{\pgfqpoint{1.215683in}{2.072851in}}{\pgfqpoint{1.223583in}{2.069578in}}{\pgfqpoint{1.231820in}{2.069578in}}%
\pgfpathclose%
\pgfusepath{stroke,fill}%
\end{pgfscope}%
\begin{pgfscope}%
\pgfpathrectangle{\pgfqpoint{0.100000in}{0.220728in}}{\pgfqpoint{3.696000in}{3.696000in}}%
\pgfusepath{clip}%
\pgfsetbuttcap%
\pgfsetroundjoin%
\definecolor{currentfill}{rgb}{0.121569,0.466667,0.705882}%
\pgfsetfillcolor{currentfill}%
\pgfsetfillopacity{0.486453}%
\pgfsetlinewidth{1.003750pt}%
\definecolor{currentstroke}{rgb}{0.121569,0.466667,0.705882}%
\pgfsetstrokecolor{currentstroke}%
\pgfsetstrokeopacity{0.486453}%
\pgfsetdash{}{0pt}%
\pgfpathmoveto{\pgfqpoint{2.568452in}{3.093935in}}%
\pgfpathcurveto{\pgfqpoint{2.576688in}{3.093935in}}{\pgfqpoint{2.584589in}{3.097207in}}{\pgfqpoint{2.590412in}{3.103031in}}%
\pgfpathcurveto{\pgfqpoint{2.596236in}{3.108855in}}{\pgfqpoint{2.599509in}{3.116755in}}{\pgfqpoint{2.599509in}{3.124991in}}%
\pgfpathcurveto{\pgfqpoint{2.599509in}{3.133227in}}{\pgfqpoint{2.596236in}{3.141127in}}{\pgfqpoint{2.590412in}{3.146951in}}%
\pgfpathcurveto{\pgfqpoint{2.584589in}{3.152775in}}{\pgfqpoint{2.576688in}{3.156048in}}{\pgfqpoint{2.568452in}{3.156048in}}%
\pgfpathcurveto{\pgfqpoint{2.560216in}{3.156048in}}{\pgfqpoint{2.552316in}{3.152775in}}{\pgfqpoint{2.546492in}{3.146951in}}%
\pgfpathcurveto{\pgfqpoint{2.540668in}{3.141127in}}{\pgfqpoint{2.537396in}{3.133227in}}{\pgfqpoint{2.537396in}{3.124991in}}%
\pgfpathcurveto{\pgfqpoint{2.537396in}{3.116755in}}{\pgfqpoint{2.540668in}{3.108855in}}{\pgfqpoint{2.546492in}{3.103031in}}%
\pgfpathcurveto{\pgfqpoint{2.552316in}{3.097207in}}{\pgfqpoint{2.560216in}{3.093935in}}{\pgfqpoint{2.568452in}{3.093935in}}%
\pgfpathclose%
\pgfusepath{stroke,fill}%
\end{pgfscope}%
\begin{pgfscope}%
\pgfpathrectangle{\pgfqpoint{0.100000in}{0.220728in}}{\pgfqpoint{3.696000in}{3.696000in}}%
\pgfusepath{clip}%
\pgfsetbuttcap%
\pgfsetroundjoin%
\definecolor{currentfill}{rgb}{0.121569,0.466667,0.705882}%
\pgfsetfillcolor{currentfill}%
\pgfsetfillopacity{0.490561}%
\pgfsetlinewidth{1.003750pt}%
\definecolor{currentstroke}{rgb}{0.121569,0.466667,0.705882}%
\pgfsetstrokecolor{currentstroke}%
\pgfsetstrokeopacity{0.490561}%
\pgfsetdash{}{0pt}%
\pgfpathmoveto{\pgfqpoint{1.227804in}{2.035156in}}%
\pgfpathcurveto{\pgfqpoint{1.236040in}{2.035156in}}{\pgfqpoint{1.243940in}{2.038429in}}{\pgfqpoint{1.249764in}{2.044253in}}%
\pgfpathcurveto{\pgfqpoint{1.255588in}{2.050077in}}{\pgfqpoint{1.258860in}{2.057977in}}{\pgfqpoint{1.258860in}{2.066213in}}%
\pgfpathcurveto{\pgfqpoint{1.258860in}{2.074449in}}{\pgfqpoint{1.255588in}{2.082349in}}{\pgfqpoint{1.249764in}{2.088173in}}%
\pgfpathcurveto{\pgfqpoint{1.243940in}{2.093997in}}{\pgfqpoint{1.236040in}{2.097269in}}{\pgfqpoint{1.227804in}{2.097269in}}%
\pgfpathcurveto{\pgfqpoint{1.219567in}{2.097269in}}{\pgfqpoint{1.211667in}{2.093997in}}{\pgfqpoint{1.205843in}{2.088173in}}%
\pgfpathcurveto{\pgfqpoint{1.200019in}{2.082349in}}{\pgfqpoint{1.196747in}{2.074449in}}{\pgfqpoint{1.196747in}{2.066213in}}%
\pgfpathcurveto{\pgfqpoint{1.196747in}{2.057977in}}{\pgfqpoint{1.200019in}{2.050077in}}{\pgfqpoint{1.205843in}{2.044253in}}%
\pgfpathcurveto{\pgfqpoint{1.211667in}{2.038429in}}{\pgfqpoint{1.219567in}{2.035156in}}{\pgfqpoint{1.227804in}{2.035156in}}%
\pgfpathclose%
\pgfusepath{stroke,fill}%
\end{pgfscope}%
\begin{pgfscope}%
\pgfpathrectangle{\pgfqpoint{0.100000in}{0.220728in}}{\pgfqpoint{3.696000in}{3.696000in}}%
\pgfusepath{clip}%
\pgfsetbuttcap%
\pgfsetroundjoin%
\definecolor{currentfill}{rgb}{0.121569,0.466667,0.705882}%
\pgfsetfillcolor{currentfill}%
\pgfsetfillopacity{0.490593}%
\pgfsetlinewidth{1.003750pt}%
\definecolor{currentstroke}{rgb}{0.121569,0.466667,0.705882}%
\pgfsetstrokecolor{currentstroke}%
\pgfsetstrokeopacity{0.490593}%
\pgfsetdash{}{0pt}%
\pgfpathmoveto{\pgfqpoint{2.578594in}{3.094132in}}%
\pgfpathcurveto{\pgfqpoint{2.586830in}{3.094132in}}{\pgfqpoint{2.594730in}{3.097404in}}{\pgfqpoint{2.600554in}{3.103228in}}%
\pgfpathcurveto{\pgfqpoint{2.606378in}{3.109052in}}{\pgfqpoint{2.609650in}{3.116952in}}{\pgfqpoint{2.609650in}{3.125188in}}%
\pgfpathcurveto{\pgfqpoint{2.609650in}{3.133425in}}{\pgfqpoint{2.606378in}{3.141325in}}{\pgfqpoint{2.600554in}{3.147149in}}%
\pgfpathcurveto{\pgfqpoint{2.594730in}{3.152973in}}{\pgfqpoint{2.586830in}{3.156245in}}{\pgfqpoint{2.578594in}{3.156245in}}%
\pgfpathcurveto{\pgfqpoint{2.570357in}{3.156245in}}{\pgfqpoint{2.562457in}{3.152973in}}{\pgfqpoint{2.556633in}{3.147149in}}%
\pgfpathcurveto{\pgfqpoint{2.550809in}{3.141325in}}{\pgfqpoint{2.547537in}{3.133425in}}{\pgfqpoint{2.547537in}{3.125188in}}%
\pgfpathcurveto{\pgfqpoint{2.547537in}{3.116952in}}{\pgfqpoint{2.550809in}{3.109052in}}{\pgfqpoint{2.556633in}{3.103228in}}%
\pgfpathcurveto{\pgfqpoint{2.562457in}{3.097404in}}{\pgfqpoint{2.570357in}{3.094132in}}{\pgfqpoint{2.578594in}{3.094132in}}%
\pgfpathclose%
\pgfusepath{stroke,fill}%
\end{pgfscope}%
\begin{pgfscope}%
\pgfpathrectangle{\pgfqpoint{0.100000in}{0.220728in}}{\pgfqpoint{3.696000in}{3.696000in}}%
\pgfusepath{clip}%
\pgfsetbuttcap%
\pgfsetroundjoin%
\definecolor{currentfill}{rgb}{0.121569,0.466667,0.705882}%
\pgfsetfillcolor{currentfill}%
\pgfsetfillopacity{0.492088}%
\pgfsetlinewidth{1.003750pt}%
\definecolor{currentstroke}{rgb}{0.121569,0.466667,0.705882}%
\pgfsetstrokecolor{currentstroke}%
\pgfsetstrokeopacity{0.492088}%
\pgfsetdash{}{0pt}%
\pgfpathmoveto{\pgfqpoint{2.584887in}{3.093012in}}%
\pgfpathcurveto{\pgfqpoint{2.593123in}{3.093012in}}{\pgfqpoint{2.601023in}{3.096284in}}{\pgfqpoint{2.606847in}{3.102108in}}%
\pgfpathcurveto{\pgfqpoint{2.612671in}{3.107932in}}{\pgfqpoint{2.615943in}{3.115832in}}{\pgfqpoint{2.615943in}{3.124068in}}%
\pgfpathcurveto{\pgfqpoint{2.615943in}{3.132304in}}{\pgfqpoint{2.612671in}{3.140204in}}{\pgfqpoint{2.606847in}{3.146028in}}%
\pgfpathcurveto{\pgfqpoint{2.601023in}{3.151852in}}{\pgfqpoint{2.593123in}{3.155125in}}{\pgfqpoint{2.584887in}{3.155125in}}%
\pgfpathcurveto{\pgfqpoint{2.576650in}{3.155125in}}{\pgfqpoint{2.568750in}{3.151852in}}{\pgfqpoint{2.562926in}{3.146028in}}%
\pgfpathcurveto{\pgfqpoint{2.557103in}{3.140204in}}{\pgfqpoint{2.553830in}{3.132304in}}{\pgfqpoint{2.553830in}{3.124068in}}%
\pgfpathcurveto{\pgfqpoint{2.553830in}{3.115832in}}{\pgfqpoint{2.557103in}{3.107932in}}{\pgfqpoint{2.562926in}{3.102108in}}%
\pgfpathcurveto{\pgfqpoint{2.568750in}{3.096284in}}{\pgfqpoint{2.576650in}{3.093012in}}{\pgfqpoint{2.584887in}{3.093012in}}%
\pgfpathclose%
\pgfusepath{stroke,fill}%
\end{pgfscope}%
\begin{pgfscope}%
\pgfpathrectangle{\pgfqpoint{0.100000in}{0.220728in}}{\pgfqpoint{3.696000in}{3.696000in}}%
\pgfusepath{clip}%
\pgfsetbuttcap%
\pgfsetroundjoin%
\definecolor{currentfill}{rgb}{0.121569,0.466667,0.705882}%
\pgfsetfillcolor{currentfill}%
\pgfsetfillopacity{0.493103}%
\pgfsetlinewidth{1.003750pt}%
\definecolor{currentstroke}{rgb}{0.121569,0.466667,0.705882}%
\pgfsetstrokecolor{currentstroke}%
\pgfsetstrokeopacity{0.493103}%
\pgfsetdash{}{0pt}%
\pgfpathmoveto{\pgfqpoint{1.208654in}{2.012272in}}%
\pgfpathcurveto{\pgfqpoint{1.216890in}{2.012272in}}{\pgfqpoint{1.224791in}{2.015544in}}{\pgfqpoint{1.230614in}{2.021368in}}%
\pgfpathcurveto{\pgfqpoint{1.236438in}{2.027192in}}{\pgfqpoint{1.239711in}{2.035092in}}{\pgfqpoint{1.239711in}{2.043329in}}%
\pgfpathcurveto{\pgfqpoint{1.239711in}{2.051565in}}{\pgfqpoint{1.236438in}{2.059465in}}{\pgfqpoint{1.230614in}{2.065289in}}%
\pgfpathcurveto{\pgfqpoint{1.224791in}{2.071113in}}{\pgfqpoint{1.216890in}{2.074385in}}{\pgfqpoint{1.208654in}{2.074385in}}%
\pgfpathcurveto{\pgfqpoint{1.200418in}{2.074385in}}{\pgfqpoint{1.192518in}{2.071113in}}{\pgfqpoint{1.186694in}{2.065289in}}%
\pgfpathcurveto{\pgfqpoint{1.180870in}{2.059465in}}{\pgfqpoint{1.177598in}{2.051565in}}{\pgfqpoint{1.177598in}{2.043329in}}%
\pgfpathcurveto{\pgfqpoint{1.177598in}{2.035092in}}{\pgfqpoint{1.180870in}{2.027192in}}{\pgfqpoint{1.186694in}{2.021368in}}%
\pgfpathcurveto{\pgfqpoint{1.192518in}{2.015544in}}{\pgfqpoint{1.200418in}{2.012272in}}{\pgfqpoint{1.208654in}{2.012272in}}%
\pgfpathclose%
\pgfusepath{stroke,fill}%
\end{pgfscope}%
\begin{pgfscope}%
\pgfpathrectangle{\pgfqpoint{0.100000in}{0.220728in}}{\pgfqpoint{3.696000in}{3.696000in}}%
\pgfusepath{clip}%
\pgfsetbuttcap%
\pgfsetroundjoin%
\definecolor{currentfill}{rgb}{0.121569,0.466667,0.705882}%
\pgfsetfillcolor{currentfill}%
\pgfsetfillopacity{0.494209}%
\pgfsetlinewidth{1.003750pt}%
\definecolor{currentstroke}{rgb}{0.121569,0.466667,0.705882}%
\pgfsetstrokecolor{currentstroke}%
\pgfsetstrokeopacity{0.494209}%
\pgfsetdash{}{0pt}%
\pgfpathmoveto{\pgfqpoint{2.591314in}{3.091462in}}%
\pgfpathcurveto{\pgfqpoint{2.599551in}{3.091462in}}{\pgfqpoint{2.607451in}{3.094734in}}{\pgfqpoint{2.613275in}{3.100558in}}%
\pgfpathcurveto{\pgfqpoint{2.619099in}{3.106382in}}{\pgfqpoint{2.622371in}{3.114282in}}{\pgfqpoint{2.622371in}{3.122518in}}%
\pgfpathcurveto{\pgfqpoint{2.622371in}{3.130754in}}{\pgfqpoint{2.619099in}{3.138654in}}{\pgfqpoint{2.613275in}{3.144478in}}%
\pgfpathcurveto{\pgfqpoint{2.607451in}{3.150302in}}{\pgfqpoint{2.599551in}{3.153575in}}{\pgfqpoint{2.591314in}{3.153575in}}%
\pgfpathcurveto{\pgfqpoint{2.583078in}{3.153575in}}{\pgfqpoint{2.575178in}{3.150302in}}{\pgfqpoint{2.569354in}{3.144478in}}%
\pgfpathcurveto{\pgfqpoint{2.563530in}{3.138654in}}{\pgfqpoint{2.560258in}{3.130754in}}{\pgfqpoint{2.560258in}{3.122518in}}%
\pgfpathcurveto{\pgfqpoint{2.560258in}{3.114282in}}{\pgfqpoint{2.563530in}{3.106382in}}{\pgfqpoint{2.569354in}{3.100558in}}%
\pgfpathcurveto{\pgfqpoint{2.575178in}{3.094734in}}{\pgfqpoint{2.583078in}{3.091462in}}{\pgfqpoint{2.591314in}{3.091462in}}%
\pgfpathclose%
\pgfusepath{stroke,fill}%
\end{pgfscope}%
\begin{pgfscope}%
\pgfpathrectangle{\pgfqpoint{0.100000in}{0.220728in}}{\pgfqpoint{3.696000in}{3.696000in}}%
\pgfusepath{clip}%
\pgfsetbuttcap%
\pgfsetroundjoin%
\definecolor{currentfill}{rgb}{0.121569,0.466667,0.705882}%
\pgfsetfillcolor{currentfill}%
\pgfsetfillopacity{0.496378}%
\pgfsetlinewidth{1.003750pt}%
\definecolor{currentstroke}{rgb}{0.121569,0.466667,0.705882}%
\pgfsetstrokecolor{currentstroke}%
\pgfsetstrokeopacity{0.496378}%
\pgfsetdash{}{0pt}%
\pgfpathmoveto{\pgfqpoint{2.599399in}{3.089681in}}%
\pgfpathcurveto{\pgfqpoint{2.607635in}{3.089681in}}{\pgfqpoint{2.615535in}{3.092954in}}{\pgfqpoint{2.621359in}{3.098777in}}%
\pgfpathcurveto{\pgfqpoint{2.627183in}{3.104601in}}{\pgfqpoint{2.630456in}{3.112501in}}{\pgfqpoint{2.630456in}{3.120738in}}%
\pgfpathcurveto{\pgfqpoint{2.630456in}{3.128974in}}{\pgfqpoint{2.627183in}{3.136874in}}{\pgfqpoint{2.621359in}{3.142698in}}%
\pgfpathcurveto{\pgfqpoint{2.615535in}{3.148522in}}{\pgfqpoint{2.607635in}{3.151794in}}{\pgfqpoint{2.599399in}{3.151794in}}%
\pgfpathcurveto{\pgfqpoint{2.591163in}{3.151794in}}{\pgfqpoint{2.583263in}{3.148522in}}{\pgfqpoint{2.577439in}{3.142698in}}%
\pgfpathcurveto{\pgfqpoint{2.571615in}{3.136874in}}{\pgfqpoint{2.568343in}{3.128974in}}{\pgfqpoint{2.568343in}{3.120738in}}%
\pgfpathcurveto{\pgfqpoint{2.568343in}{3.112501in}}{\pgfqpoint{2.571615in}{3.104601in}}{\pgfqpoint{2.577439in}{3.098777in}}%
\pgfpathcurveto{\pgfqpoint{2.583263in}{3.092954in}}{\pgfqpoint{2.591163in}{3.089681in}}{\pgfqpoint{2.599399in}{3.089681in}}%
\pgfpathclose%
\pgfusepath{stroke,fill}%
\end{pgfscope}%
\begin{pgfscope}%
\pgfpathrectangle{\pgfqpoint{0.100000in}{0.220728in}}{\pgfqpoint{3.696000in}{3.696000in}}%
\pgfusepath{clip}%
\pgfsetbuttcap%
\pgfsetroundjoin%
\definecolor{currentfill}{rgb}{0.121569,0.466667,0.705882}%
\pgfsetfillcolor{currentfill}%
\pgfsetfillopacity{0.497409}%
\pgfsetlinewidth{1.003750pt}%
\definecolor{currentstroke}{rgb}{0.121569,0.466667,0.705882}%
\pgfsetstrokecolor{currentstroke}%
\pgfsetstrokeopacity{0.497409}%
\pgfsetdash{}{0pt}%
\pgfpathmoveto{\pgfqpoint{1.201961in}{1.986763in}}%
\pgfpathcurveto{\pgfqpoint{1.210198in}{1.986763in}}{\pgfqpoint{1.218098in}{1.990035in}}{\pgfqpoint{1.223922in}{1.995859in}}%
\pgfpathcurveto{\pgfqpoint{1.229746in}{2.001683in}}{\pgfqpoint{1.233018in}{2.009583in}}{\pgfqpoint{1.233018in}{2.017819in}}%
\pgfpathcurveto{\pgfqpoint{1.233018in}{2.026056in}}{\pgfqpoint{1.229746in}{2.033956in}}{\pgfqpoint{1.223922in}{2.039780in}}%
\pgfpathcurveto{\pgfqpoint{1.218098in}{2.045604in}}{\pgfqpoint{1.210198in}{2.048876in}}{\pgfqpoint{1.201961in}{2.048876in}}%
\pgfpathcurveto{\pgfqpoint{1.193725in}{2.048876in}}{\pgfqpoint{1.185825in}{2.045604in}}{\pgfqpoint{1.180001in}{2.039780in}}%
\pgfpathcurveto{\pgfqpoint{1.174177in}{2.033956in}}{\pgfqpoint{1.170905in}{2.026056in}}{\pgfqpoint{1.170905in}{2.017819in}}%
\pgfpathcurveto{\pgfqpoint{1.170905in}{2.009583in}}{\pgfqpoint{1.174177in}{2.001683in}}{\pgfqpoint{1.180001in}{1.995859in}}%
\pgfpathcurveto{\pgfqpoint{1.185825in}{1.990035in}}{\pgfqpoint{1.193725in}{1.986763in}}{\pgfqpoint{1.201961in}{1.986763in}}%
\pgfpathclose%
\pgfusepath{stroke,fill}%
\end{pgfscope}%
\begin{pgfscope}%
\pgfpathrectangle{\pgfqpoint{0.100000in}{0.220728in}}{\pgfqpoint{3.696000in}{3.696000in}}%
\pgfusepath{clip}%
\pgfsetbuttcap%
\pgfsetroundjoin%
\definecolor{currentfill}{rgb}{0.121569,0.466667,0.705882}%
\pgfsetfillcolor{currentfill}%
\pgfsetfillopacity{0.498265}%
\pgfsetlinewidth{1.003750pt}%
\definecolor{currentstroke}{rgb}{0.121569,0.466667,0.705882}%
\pgfsetstrokecolor{currentstroke}%
\pgfsetstrokeopacity{0.498265}%
\pgfsetdash{}{0pt}%
\pgfpathmoveto{\pgfqpoint{2.609169in}{3.086791in}}%
\pgfpathcurveto{\pgfqpoint{2.617405in}{3.086791in}}{\pgfqpoint{2.625305in}{3.090063in}}{\pgfqpoint{2.631129in}{3.095887in}}%
\pgfpathcurveto{\pgfqpoint{2.636953in}{3.101711in}}{\pgfqpoint{2.640225in}{3.109611in}}{\pgfqpoint{2.640225in}{3.117848in}}%
\pgfpathcurveto{\pgfqpoint{2.640225in}{3.126084in}}{\pgfqpoint{2.636953in}{3.133984in}}{\pgfqpoint{2.631129in}{3.139808in}}%
\pgfpathcurveto{\pgfqpoint{2.625305in}{3.145632in}}{\pgfqpoint{2.617405in}{3.148904in}}{\pgfqpoint{2.609169in}{3.148904in}}%
\pgfpathcurveto{\pgfqpoint{2.600933in}{3.148904in}}{\pgfqpoint{2.593033in}{3.145632in}}{\pgfqpoint{2.587209in}{3.139808in}}%
\pgfpathcurveto{\pgfqpoint{2.581385in}{3.133984in}}{\pgfqpoint{2.578112in}{3.126084in}}{\pgfqpoint{2.578112in}{3.117848in}}%
\pgfpathcurveto{\pgfqpoint{2.578112in}{3.109611in}}{\pgfqpoint{2.581385in}{3.101711in}}{\pgfqpoint{2.587209in}{3.095887in}}%
\pgfpathcurveto{\pgfqpoint{2.593033in}{3.090063in}}{\pgfqpoint{2.600933in}{3.086791in}}{\pgfqpoint{2.609169in}{3.086791in}}%
\pgfpathclose%
\pgfusepath{stroke,fill}%
\end{pgfscope}%
\begin{pgfscope}%
\pgfpathrectangle{\pgfqpoint{0.100000in}{0.220728in}}{\pgfqpoint{3.696000in}{3.696000in}}%
\pgfusepath{clip}%
\pgfsetbuttcap%
\pgfsetroundjoin%
\definecolor{currentfill}{rgb}{0.121569,0.466667,0.705882}%
\pgfsetfillcolor{currentfill}%
\pgfsetfillopacity{0.500748}%
\pgfsetlinewidth{1.003750pt}%
\definecolor{currentstroke}{rgb}{0.121569,0.466667,0.705882}%
\pgfsetstrokecolor{currentstroke}%
\pgfsetstrokeopacity{0.500748}%
\pgfsetdash{}{0pt}%
\pgfpathmoveto{\pgfqpoint{1.189086in}{1.964140in}}%
\pgfpathcurveto{\pgfqpoint{1.197322in}{1.964140in}}{\pgfqpoint{1.205222in}{1.967412in}}{\pgfqpoint{1.211046in}{1.973236in}}%
\pgfpathcurveto{\pgfqpoint{1.216870in}{1.979060in}}{\pgfqpoint{1.220142in}{1.986960in}}{\pgfqpoint{1.220142in}{1.995197in}}%
\pgfpathcurveto{\pgfqpoint{1.220142in}{2.003433in}}{\pgfqpoint{1.216870in}{2.011333in}}{\pgfqpoint{1.211046in}{2.017157in}}%
\pgfpathcurveto{\pgfqpoint{1.205222in}{2.022981in}}{\pgfqpoint{1.197322in}{2.026253in}}{\pgfqpoint{1.189086in}{2.026253in}}%
\pgfpathcurveto{\pgfqpoint{1.180849in}{2.026253in}}{\pgfqpoint{1.172949in}{2.022981in}}{\pgfqpoint{1.167125in}{2.017157in}}%
\pgfpathcurveto{\pgfqpoint{1.161301in}{2.011333in}}{\pgfqpoint{1.158029in}{2.003433in}}{\pgfqpoint{1.158029in}{1.995197in}}%
\pgfpathcurveto{\pgfqpoint{1.158029in}{1.986960in}}{\pgfqpoint{1.161301in}{1.979060in}}{\pgfqpoint{1.167125in}{1.973236in}}%
\pgfpathcurveto{\pgfqpoint{1.172949in}{1.967412in}}{\pgfqpoint{1.180849in}{1.964140in}}{\pgfqpoint{1.189086in}{1.964140in}}%
\pgfpathclose%
\pgfusepath{stroke,fill}%
\end{pgfscope}%
\begin{pgfscope}%
\pgfpathrectangle{\pgfqpoint{0.100000in}{0.220728in}}{\pgfqpoint{3.696000in}{3.696000in}}%
\pgfusepath{clip}%
\pgfsetbuttcap%
\pgfsetroundjoin%
\definecolor{currentfill}{rgb}{0.121569,0.466667,0.705882}%
\pgfsetfillcolor{currentfill}%
\pgfsetfillopacity{0.501274}%
\pgfsetlinewidth{1.003750pt}%
\definecolor{currentstroke}{rgb}{0.121569,0.466667,0.705882}%
\pgfsetstrokecolor{currentstroke}%
\pgfsetstrokeopacity{0.501274}%
\pgfsetdash{}{0pt}%
\pgfpathmoveto{\pgfqpoint{2.619262in}{3.085464in}}%
\pgfpathcurveto{\pgfqpoint{2.627498in}{3.085464in}}{\pgfqpoint{2.635398in}{3.088737in}}{\pgfqpoint{2.641222in}{3.094561in}}%
\pgfpathcurveto{\pgfqpoint{2.647046in}{3.100385in}}{\pgfqpoint{2.650318in}{3.108285in}}{\pgfqpoint{2.650318in}{3.116521in}}%
\pgfpathcurveto{\pgfqpoint{2.650318in}{3.124757in}}{\pgfqpoint{2.647046in}{3.132657in}}{\pgfqpoint{2.641222in}{3.138481in}}%
\pgfpathcurveto{\pgfqpoint{2.635398in}{3.144305in}}{\pgfqpoint{2.627498in}{3.147577in}}{\pgfqpoint{2.619262in}{3.147577in}}%
\pgfpathcurveto{\pgfqpoint{2.611025in}{3.147577in}}{\pgfqpoint{2.603125in}{3.144305in}}{\pgfqpoint{2.597302in}{3.138481in}}%
\pgfpathcurveto{\pgfqpoint{2.591478in}{3.132657in}}{\pgfqpoint{2.588205in}{3.124757in}}{\pgfqpoint{2.588205in}{3.116521in}}%
\pgfpathcurveto{\pgfqpoint{2.588205in}{3.108285in}}{\pgfqpoint{2.591478in}{3.100385in}}{\pgfqpoint{2.597302in}{3.094561in}}%
\pgfpathcurveto{\pgfqpoint{2.603125in}{3.088737in}}{\pgfqpoint{2.611025in}{3.085464in}}{\pgfqpoint{2.619262in}{3.085464in}}%
\pgfpathclose%
\pgfusepath{stroke,fill}%
\end{pgfscope}%
\begin{pgfscope}%
\pgfpathrectangle{\pgfqpoint{0.100000in}{0.220728in}}{\pgfqpoint{3.696000in}{3.696000in}}%
\pgfusepath{clip}%
\pgfsetbuttcap%
\pgfsetroundjoin%
\definecolor{currentfill}{rgb}{0.121569,0.466667,0.705882}%
\pgfsetfillcolor{currentfill}%
\pgfsetfillopacity{0.502762}%
\pgfsetlinewidth{1.003750pt}%
\definecolor{currentstroke}{rgb}{0.121569,0.466667,0.705882}%
\pgfsetstrokecolor{currentstroke}%
\pgfsetstrokeopacity{0.502762}%
\pgfsetdash{}{0pt}%
\pgfpathmoveto{\pgfqpoint{2.632764in}{3.079789in}}%
\pgfpathcurveto{\pgfqpoint{2.641001in}{3.079789in}}{\pgfqpoint{2.648901in}{3.083062in}}{\pgfqpoint{2.654725in}{3.088886in}}%
\pgfpathcurveto{\pgfqpoint{2.660549in}{3.094710in}}{\pgfqpoint{2.663821in}{3.102610in}}{\pgfqpoint{2.663821in}{3.110846in}}%
\pgfpathcurveto{\pgfqpoint{2.663821in}{3.119082in}}{\pgfqpoint{2.660549in}{3.126982in}}{\pgfqpoint{2.654725in}{3.132806in}}%
\pgfpathcurveto{\pgfqpoint{2.648901in}{3.138630in}}{\pgfqpoint{2.641001in}{3.141902in}}{\pgfqpoint{2.632764in}{3.141902in}}%
\pgfpathcurveto{\pgfqpoint{2.624528in}{3.141902in}}{\pgfqpoint{2.616628in}{3.138630in}}{\pgfqpoint{2.610804in}{3.132806in}}%
\pgfpathcurveto{\pgfqpoint{2.604980in}{3.126982in}}{\pgfqpoint{2.601708in}{3.119082in}}{\pgfqpoint{2.601708in}{3.110846in}}%
\pgfpathcurveto{\pgfqpoint{2.601708in}{3.102610in}}{\pgfqpoint{2.604980in}{3.094710in}}{\pgfqpoint{2.610804in}{3.088886in}}%
\pgfpathcurveto{\pgfqpoint{2.616628in}{3.083062in}}{\pgfqpoint{2.624528in}{3.079789in}}{\pgfqpoint{2.632764in}{3.079789in}}%
\pgfpathclose%
\pgfusepath{stroke,fill}%
\end{pgfscope}%
\begin{pgfscope}%
\pgfpathrectangle{\pgfqpoint{0.100000in}{0.220728in}}{\pgfqpoint{3.696000in}{3.696000in}}%
\pgfusepath{clip}%
\pgfsetbuttcap%
\pgfsetroundjoin%
\definecolor{currentfill}{rgb}{0.121569,0.466667,0.705882}%
\pgfsetfillcolor{currentfill}%
\pgfsetfillopacity{0.504483}%
\pgfsetlinewidth{1.003750pt}%
\definecolor{currentstroke}{rgb}{0.121569,0.466667,0.705882}%
\pgfsetstrokecolor{currentstroke}%
\pgfsetstrokeopacity{0.504483}%
\pgfsetdash{}{0pt}%
\pgfpathmoveto{\pgfqpoint{1.178241in}{1.943272in}}%
\pgfpathcurveto{\pgfqpoint{1.186478in}{1.943272in}}{\pgfqpoint{1.194378in}{1.946545in}}{\pgfqpoint{1.200202in}{1.952369in}}%
\pgfpathcurveto{\pgfqpoint{1.206026in}{1.958193in}}{\pgfqpoint{1.209298in}{1.966093in}}{\pgfqpoint{1.209298in}{1.974329in}}%
\pgfpathcurveto{\pgfqpoint{1.209298in}{1.982565in}}{\pgfqpoint{1.206026in}{1.990465in}}{\pgfqpoint{1.200202in}{1.996289in}}%
\pgfpathcurveto{\pgfqpoint{1.194378in}{2.002113in}}{\pgfqpoint{1.186478in}{2.005385in}}{\pgfqpoint{1.178241in}{2.005385in}}%
\pgfpathcurveto{\pgfqpoint{1.170005in}{2.005385in}}{\pgfqpoint{1.162105in}{2.002113in}}{\pgfqpoint{1.156281in}{1.996289in}}%
\pgfpathcurveto{\pgfqpoint{1.150457in}{1.990465in}}{\pgfqpoint{1.147185in}{1.982565in}}{\pgfqpoint{1.147185in}{1.974329in}}%
\pgfpathcurveto{\pgfqpoint{1.147185in}{1.966093in}}{\pgfqpoint{1.150457in}{1.958193in}}{\pgfqpoint{1.156281in}{1.952369in}}%
\pgfpathcurveto{\pgfqpoint{1.162105in}{1.946545in}}{\pgfqpoint{1.170005in}{1.943272in}}{\pgfqpoint{1.178241in}{1.943272in}}%
\pgfpathclose%
\pgfusepath{stroke,fill}%
\end{pgfscope}%
\begin{pgfscope}%
\pgfpathrectangle{\pgfqpoint{0.100000in}{0.220728in}}{\pgfqpoint{3.696000in}{3.696000in}}%
\pgfusepath{clip}%
\pgfsetbuttcap%
\pgfsetroundjoin%
\definecolor{currentfill}{rgb}{0.121569,0.466667,0.705882}%
\pgfsetfillcolor{currentfill}%
\pgfsetfillopacity{0.506158}%
\pgfsetlinewidth{1.003750pt}%
\definecolor{currentstroke}{rgb}{0.121569,0.466667,0.705882}%
\pgfsetstrokecolor{currentstroke}%
\pgfsetstrokeopacity{0.506158}%
\pgfsetdash{}{0pt}%
\pgfpathmoveto{\pgfqpoint{2.646837in}{3.077063in}}%
\pgfpathcurveto{\pgfqpoint{2.655074in}{3.077063in}}{\pgfqpoint{2.662974in}{3.080335in}}{\pgfqpoint{2.668798in}{3.086159in}}%
\pgfpathcurveto{\pgfqpoint{2.674622in}{3.091983in}}{\pgfqpoint{2.677894in}{3.099883in}}{\pgfqpoint{2.677894in}{3.108120in}}%
\pgfpathcurveto{\pgfqpoint{2.677894in}{3.116356in}}{\pgfqpoint{2.674622in}{3.124256in}}{\pgfqpoint{2.668798in}{3.130080in}}%
\pgfpathcurveto{\pgfqpoint{2.662974in}{3.135904in}}{\pgfqpoint{2.655074in}{3.139176in}}{\pgfqpoint{2.646837in}{3.139176in}}%
\pgfpathcurveto{\pgfqpoint{2.638601in}{3.139176in}}{\pgfqpoint{2.630701in}{3.135904in}}{\pgfqpoint{2.624877in}{3.130080in}}%
\pgfpathcurveto{\pgfqpoint{2.619053in}{3.124256in}}{\pgfqpoint{2.615781in}{3.116356in}}{\pgfqpoint{2.615781in}{3.108120in}}%
\pgfpathcurveto{\pgfqpoint{2.615781in}{3.099883in}}{\pgfqpoint{2.619053in}{3.091983in}}{\pgfqpoint{2.624877in}{3.086159in}}%
\pgfpathcurveto{\pgfqpoint{2.630701in}{3.080335in}}{\pgfqpoint{2.638601in}{3.077063in}}{\pgfqpoint{2.646837in}{3.077063in}}%
\pgfpathclose%
\pgfusepath{stroke,fill}%
\end{pgfscope}%
\begin{pgfscope}%
\pgfpathrectangle{\pgfqpoint{0.100000in}{0.220728in}}{\pgfqpoint{3.696000in}{3.696000in}}%
\pgfusepath{clip}%
\pgfsetbuttcap%
\pgfsetroundjoin%
\definecolor{currentfill}{rgb}{0.121569,0.466667,0.705882}%
\pgfsetfillcolor{currentfill}%
\pgfsetfillopacity{0.508070}%
\pgfsetlinewidth{1.003750pt}%
\definecolor{currentstroke}{rgb}{0.121569,0.466667,0.705882}%
\pgfsetstrokecolor{currentstroke}%
\pgfsetstrokeopacity{0.508070}%
\pgfsetdash{}{0pt}%
\pgfpathmoveto{\pgfqpoint{1.169077in}{1.921794in}}%
\pgfpathcurveto{\pgfqpoint{1.177313in}{1.921794in}}{\pgfqpoint{1.185213in}{1.925067in}}{\pgfqpoint{1.191037in}{1.930891in}}%
\pgfpathcurveto{\pgfqpoint{1.196861in}{1.936715in}}{\pgfqpoint{1.200133in}{1.944615in}}{\pgfqpoint{1.200133in}{1.952851in}}%
\pgfpathcurveto{\pgfqpoint{1.200133in}{1.961087in}}{\pgfqpoint{1.196861in}{1.968987in}}{\pgfqpoint{1.191037in}{1.974811in}}%
\pgfpathcurveto{\pgfqpoint{1.185213in}{1.980635in}}{\pgfqpoint{1.177313in}{1.983907in}}{\pgfqpoint{1.169077in}{1.983907in}}%
\pgfpathcurveto{\pgfqpoint{1.160841in}{1.983907in}}{\pgfqpoint{1.152941in}{1.980635in}}{\pgfqpoint{1.147117in}{1.974811in}}%
\pgfpathcurveto{\pgfqpoint{1.141293in}{1.968987in}}{\pgfqpoint{1.138020in}{1.961087in}}{\pgfqpoint{1.138020in}{1.952851in}}%
\pgfpathcurveto{\pgfqpoint{1.138020in}{1.944615in}}{\pgfqpoint{1.141293in}{1.936715in}}{\pgfqpoint{1.147117in}{1.930891in}}%
\pgfpathcurveto{\pgfqpoint{1.152941in}{1.925067in}}{\pgfqpoint{1.160841in}{1.921794in}}{\pgfqpoint{1.169077in}{1.921794in}}%
\pgfpathclose%
\pgfusepath{stroke,fill}%
\end{pgfscope}%
\begin{pgfscope}%
\pgfpathrectangle{\pgfqpoint{0.100000in}{0.220728in}}{\pgfqpoint{3.696000in}{3.696000in}}%
\pgfusepath{clip}%
\pgfsetbuttcap%
\pgfsetroundjoin%
\definecolor{currentfill}{rgb}{0.121569,0.466667,0.705882}%
\pgfsetfillcolor{currentfill}%
\pgfsetfillopacity{0.508161}%
\pgfsetlinewidth{1.003750pt}%
\definecolor{currentstroke}{rgb}{0.121569,0.466667,0.705882}%
\pgfsetstrokecolor{currentstroke}%
\pgfsetstrokeopacity{0.508161}%
\pgfsetdash{}{0pt}%
\pgfpathmoveto{\pgfqpoint{2.654252in}{3.075105in}}%
\pgfpathcurveto{\pgfqpoint{2.662488in}{3.075105in}}{\pgfqpoint{2.670388in}{3.078378in}}{\pgfqpoint{2.676212in}{3.084202in}}%
\pgfpathcurveto{\pgfqpoint{2.682036in}{3.090026in}}{\pgfqpoint{2.685309in}{3.097926in}}{\pgfqpoint{2.685309in}{3.106162in}}%
\pgfpathcurveto{\pgfqpoint{2.685309in}{3.114398in}}{\pgfqpoint{2.682036in}{3.122298in}}{\pgfqpoint{2.676212in}{3.128122in}}%
\pgfpathcurveto{\pgfqpoint{2.670388in}{3.133946in}}{\pgfqpoint{2.662488in}{3.137218in}}{\pgfqpoint{2.654252in}{3.137218in}}%
\pgfpathcurveto{\pgfqpoint{2.646016in}{3.137218in}}{\pgfqpoint{2.638116in}{3.133946in}}{\pgfqpoint{2.632292in}{3.128122in}}%
\pgfpathcurveto{\pgfqpoint{2.626468in}{3.122298in}}{\pgfqpoint{2.623196in}{3.114398in}}{\pgfqpoint{2.623196in}{3.106162in}}%
\pgfpathcurveto{\pgfqpoint{2.623196in}{3.097926in}}{\pgfqpoint{2.626468in}{3.090026in}}{\pgfqpoint{2.632292in}{3.084202in}}%
\pgfpathcurveto{\pgfqpoint{2.638116in}{3.078378in}}{\pgfqpoint{2.646016in}{3.075105in}}{\pgfqpoint{2.654252in}{3.075105in}}%
\pgfpathclose%
\pgfusepath{stroke,fill}%
\end{pgfscope}%
\begin{pgfscope}%
\pgfpathrectangle{\pgfqpoint{0.100000in}{0.220728in}}{\pgfqpoint{3.696000in}{3.696000in}}%
\pgfusepath{clip}%
\pgfsetbuttcap%
\pgfsetroundjoin%
\definecolor{currentfill}{rgb}{0.121569,0.466667,0.705882}%
\pgfsetfillcolor{currentfill}%
\pgfsetfillopacity{0.509848}%
\pgfsetlinewidth{1.003750pt}%
\definecolor{currentstroke}{rgb}{0.121569,0.466667,0.705882}%
\pgfsetstrokecolor{currentstroke}%
\pgfsetstrokeopacity{0.509848}%
\pgfsetdash{}{0pt}%
\pgfpathmoveto{\pgfqpoint{2.663042in}{3.072732in}}%
\pgfpathcurveto{\pgfqpoint{2.671278in}{3.072732in}}{\pgfqpoint{2.679178in}{3.076004in}}{\pgfqpoint{2.685002in}{3.081828in}}%
\pgfpathcurveto{\pgfqpoint{2.690826in}{3.087652in}}{\pgfqpoint{2.694098in}{3.095552in}}{\pgfqpoint{2.694098in}{3.103788in}}%
\pgfpathcurveto{\pgfqpoint{2.694098in}{3.112024in}}{\pgfqpoint{2.690826in}{3.119924in}}{\pgfqpoint{2.685002in}{3.125748in}}%
\pgfpathcurveto{\pgfqpoint{2.679178in}{3.131572in}}{\pgfqpoint{2.671278in}{3.134845in}}{\pgfqpoint{2.663042in}{3.134845in}}%
\pgfpathcurveto{\pgfqpoint{2.654806in}{3.134845in}}{\pgfqpoint{2.646906in}{3.131572in}}{\pgfqpoint{2.641082in}{3.125748in}}%
\pgfpathcurveto{\pgfqpoint{2.635258in}{3.119924in}}{\pgfqpoint{2.631985in}{3.112024in}}{\pgfqpoint{2.631985in}{3.103788in}}%
\pgfpathcurveto{\pgfqpoint{2.631985in}{3.095552in}}{\pgfqpoint{2.635258in}{3.087652in}}{\pgfqpoint{2.641082in}{3.081828in}}%
\pgfpathcurveto{\pgfqpoint{2.646906in}{3.076004in}}{\pgfqpoint{2.654806in}{3.072732in}}{\pgfqpoint{2.663042in}{3.072732in}}%
\pgfpathclose%
\pgfusepath{stroke,fill}%
\end{pgfscope}%
\begin{pgfscope}%
\pgfpathrectangle{\pgfqpoint{0.100000in}{0.220728in}}{\pgfqpoint{3.696000in}{3.696000in}}%
\pgfusepath{clip}%
\pgfsetbuttcap%
\pgfsetroundjoin%
\definecolor{currentfill}{rgb}{0.121569,0.466667,0.705882}%
\pgfsetfillcolor{currentfill}%
\pgfsetfillopacity{0.510816}%
\pgfsetlinewidth{1.003750pt}%
\definecolor{currentstroke}{rgb}{0.121569,0.466667,0.705882}%
\pgfsetstrokecolor{currentstroke}%
\pgfsetstrokeopacity{0.510816}%
\pgfsetdash{}{0pt}%
\pgfpathmoveto{\pgfqpoint{1.157138in}{1.905085in}}%
\pgfpathcurveto{\pgfqpoint{1.165374in}{1.905085in}}{\pgfqpoint{1.173274in}{1.908357in}}{\pgfqpoint{1.179098in}{1.914181in}}%
\pgfpathcurveto{\pgfqpoint{1.184922in}{1.920005in}}{\pgfqpoint{1.188194in}{1.927905in}}{\pgfqpoint{1.188194in}{1.936141in}}%
\pgfpathcurveto{\pgfqpoint{1.188194in}{1.944378in}}{\pgfqpoint{1.184922in}{1.952278in}}{\pgfqpoint{1.179098in}{1.958102in}}%
\pgfpathcurveto{\pgfqpoint{1.173274in}{1.963926in}}{\pgfqpoint{1.165374in}{1.967198in}}{\pgfqpoint{1.157138in}{1.967198in}}%
\pgfpathcurveto{\pgfqpoint{1.148902in}{1.967198in}}{\pgfqpoint{1.141001in}{1.963926in}}{\pgfqpoint{1.135178in}{1.958102in}}%
\pgfpathcurveto{\pgfqpoint{1.129354in}{1.952278in}}{\pgfqpoint{1.126081in}{1.944378in}}{\pgfqpoint{1.126081in}{1.936141in}}%
\pgfpathcurveto{\pgfqpoint{1.126081in}{1.927905in}}{\pgfqpoint{1.129354in}{1.920005in}}{\pgfqpoint{1.135178in}{1.914181in}}%
\pgfpathcurveto{\pgfqpoint{1.141001in}{1.908357in}}{\pgfqpoint{1.148902in}{1.905085in}}{\pgfqpoint{1.157138in}{1.905085in}}%
\pgfpathclose%
\pgfusepath{stroke,fill}%
\end{pgfscope}%
\begin{pgfscope}%
\pgfpathrectangle{\pgfqpoint{0.100000in}{0.220728in}}{\pgfqpoint{3.696000in}{3.696000in}}%
\pgfusepath{clip}%
\pgfsetbuttcap%
\pgfsetroundjoin%
\definecolor{currentfill}{rgb}{0.121569,0.466667,0.705882}%
\pgfsetfillcolor{currentfill}%
\pgfsetfillopacity{0.512776}%
\pgfsetlinewidth{1.003750pt}%
\definecolor{currentstroke}{rgb}{0.121569,0.466667,0.705882}%
\pgfsetstrokecolor{currentstroke}%
\pgfsetstrokeopacity{0.512776}%
\pgfsetdash{}{0pt}%
\pgfpathmoveto{\pgfqpoint{2.672914in}{3.070910in}}%
\pgfpathcurveto{\pgfqpoint{2.681150in}{3.070910in}}{\pgfqpoint{2.689050in}{3.074182in}}{\pgfqpoint{2.694874in}{3.080006in}}%
\pgfpathcurveto{\pgfqpoint{2.700698in}{3.085830in}}{\pgfqpoint{2.703970in}{3.093730in}}{\pgfqpoint{2.703970in}{3.101966in}}%
\pgfpathcurveto{\pgfqpoint{2.703970in}{3.110203in}}{\pgfqpoint{2.700698in}{3.118103in}}{\pgfqpoint{2.694874in}{3.123927in}}%
\pgfpathcurveto{\pgfqpoint{2.689050in}{3.129751in}}{\pgfqpoint{2.681150in}{3.133023in}}{\pgfqpoint{2.672914in}{3.133023in}}%
\pgfpathcurveto{\pgfqpoint{2.664678in}{3.133023in}}{\pgfqpoint{2.656777in}{3.129751in}}{\pgfqpoint{2.650954in}{3.123927in}}%
\pgfpathcurveto{\pgfqpoint{2.645130in}{3.118103in}}{\pgfqpoint{2.641857in}{3.110203in}}{\pgfqpoint{2.641857in}{3.101966in}}%
\pgfpathcurveto{\pgfqpoint{2.641857in}{3.093730in}}{\pgfqpoint{2.645130in}{3.085830in}}{\pgfqpoint{2.650954in}{3.080006in}}%
\pgfpathcurveto{\pgfqpoint{2.656777in}{3.074182in}}{\pgfqpoint{2.664678in}{3.070910in}}{\pgfqpoint{2.672914in}{3.070910in}}%
\pgfpathclose%
\pgfusepath{stroke,fill}%
\end{pgfscope}%
\begin{pgfscope}%
\pgfpathrectangle{\pgfqpoint{0.100000in}{0.220728in}}{\pgfqpoint{3.696000in}{3.696000in}}%
\pgfusepath{clip}%
\pgfsetbuttcap%
\pgfsetroundjoin%
\definecolor{currentfill}{rgb}{0.121569,0.466667,0.705882}%
\pgfsetfillcolor{currentfill}%
\pgfsetfillopacity{0.513892}%
\pgfsetlinewidth{1.003750pt}%
\definecolor{currentstroke}{rgb}{0.121569,0.466667,0.705882}%
\pgfsetstrokecolor{currentstroke}%
\pgfsetstrokeopacity{0.513892}%
\pgfsetdash{}{0pt}%
\pgfpathmoveto{\pgfqpoint{2.678746in}{3.069153in}}%
\pgfpathcurveto{\pgfqpoint{2.686982in}{3.069153in}}{\pgfqpoint{2.694882in}{3.072425in}}{\pgfqpoint{2.700706in}{3.078249in}}%
\pgfpathcurveto{\pgfqpoint{2.706530in}{3.084073in}}{\pgfqpoint{2.709802in}{3.091973in}}{\pgfqpoint{2.709802in}{3.100209in}}%
\pgfpathcurveto{\pgfqpoint{2.709802in}{3.108446in}}{\pgfqpoint{2.706530in}{3.116346in}}{\pgfqpoint{2.700706in}{3.122170in}}%
\pgfpathcurveto{\pgfqpoint{2.694882in}{3.127994in}}{\pgfqpoint{2.686982in}{3.131266in}}{\pgfqpoint{2.678746in}{3.131266in}}%
\pgfpathcurveto{\pgfqpoint{2.670510in}{3.131266in}}{\pgfqpoint{2.662610in}{3.127994in}}{\pgfqpoint{2.656786in}{3.122170in}}%
\pgfpathcurveto{\pgfqpoint{2.650962in}{3.116346in}}{\pgfqpoint{2.647689in}{3.108446in}}{\pgfqpoint{2.647689in}{3.100209in}}%
\pgfpathcurveto{\pgfqpoint{2.647689in}{3.091973in}}{\pgfqpoint{2.650962in}{3.084073in}}{\pgfqpoint{2.656786in}{3.078249in}}%
\pgfpathcurveto{\pgfqpoint{2.662610in}{3.072425in}}{\pgfqpoint{2.670510in}{3.069153in}}{\pgfqpoint{2.678746in}{3.069153in}}%
\pgfpathclose%
\pgfusepath{stroke,fill}%
\end{pgfscope}%
\begin{pgfscope}%
\pgfpathrectangle{\pgfqpoint{0.100000in}{0.220728in}}{\pgfqpoint{3.696000in}{3.696000in}}%
\pgfusepath{clip}%
\pgfsetbuttcap%
\pgfsetroundjoin%
\definecolor{currentfill}{rgb}{0.121569,0.466667,0.705882}%
\pgfsetfillcolor{currentfill}%
\pgfsetfillopacity{0.513953}%
\pgfsetlinewidth{1.003750pt}%
\definecolor{currentstroke}{rgb}{0.121569,0.466667,0.705882}%
\pgfsetstrokecolor{currentstroke}%
\pgfsetstrokeopacity{0.513953}%
\pgfsetdash{}{0pt}%
\pgfpathmoveto{\pgfqpoint{1.152084in}{1.885574in}}%
\pgfpathcurveto{\pgfqpoint{1.160320in}{1.885574in}}{\pgfqpoint{1.168220in}{1.888847in}}{\pgfqpoint{1.174044in}{1.894670in}}%
\pgfpathcurveto{\pgfqpoint{1.179868in}{1.900494in}}{\pgfqpoint{1.183141in}{1.908394in}}{\pgfqpoint{1.183141in}{1.916631in}}%
\pgfpathcurveto{\pgfqpoint{1.183141in}{1.924867in}}{\pgfqpoint{1.179868in}{1.932767in}}{\pgfqpoint{1.174044in}{1.938591in}}%
\pgfpathcurveto{\pgfqpoint{1.168220in}{1.944415in}}{\pgfqpoint{1.160320in}{1.947687in}}{\pgfqpoint{1.152084in}{1.947687in}}%
\pgfpathcurveto{\pgfqpoint{1.143848in}{1.947687in}}{\pgfqpoint{1.135948in}{1.944415in}}{\pgfqpoint{1.130124in}{1.938591in}}%
\pgfpathcurveto{\pgfqpoint{1.124300in}{1.932767in}}{\pgfqpoint{1.121028in}{1.924867in}}{\pgfqpoint{1.121028in}{1.916631in}}%
\pgfpathcurveto{\pgfqpoint{1.121028in}{1.908394in}}{\pgfqpoint{1.124300in}{1.900494in}}{\pgfqpoint{1.130124in}{1.894670in}}%
\pgfpathcurveto{\pgfqpoint{1.135948in}{1.888847in}}{\pgfqpoint{1.143848in}{1.885574in}}{\pgfqpoint{1.152084in}{1.885574in}}%
\pgfpathclose%
\pgfusepath{stroke,fill}%
\end{pgfscope}%
\begin{pgfscope}%
\pgfpathrectangle{\pgfqpoint{0.100000in}{0.220728in}}{\pgfqpoint{3.696000in}{3.696000in}}%
\pgfusepath{clip}%
\pgfsetbuttcap%
\pgfsetroundjoin%
\definecolor{currentfill}{rgb}{0.121569,0.466667,0.705882}%
\pgfsetfillcolor{currentfill}%
\pgfsetfillopacity{0.514335}%
\pgfsetlinewidth{1.003750pt}%
\definecolor{currentstroke}{rgb}{0.121569,0.466667,0.705882}%
\pgfsetstrokecolor{currentstroke}%
\pgfsetstrokeopacity{0.514335}%
\pgfsetdash{}{0pt}%
\pgfpathmoveto{\pgfqpoint{2.686224in}{3.066284in}}%
\pgfpathcurveto{\pgfqpoint{2.694461in}{3.066284in}}{\pgfqpoint{2.702361in}{3.069557in}}{\pgfqpoint{2.708185in}{3.075381in}}%
\pgfpathcurveto{\pgfqpoint{2.714009in}{3.081204in}}{\pgfqpoint{2.717281in}{3.089104in}}{\pgfqpoint{2.717281in}{3.097341in}}%
\pgfpathcurveto{\pgfqpoint{2.717281in}{3.105577in}}{\pgfqpoint{2.714009in}{3.113477in}}{\pgfqpoint{2.708185in}{3.119301in}}%
\pgfpathcurveto{\pgfqpoint{2.702361in}{3.125125in}}{\pgfqpoint{2.694461in}{3.128397in}}{\pgfqpoint{2.686224in}{3.128397in}}%
\pgfpathcurveto{\pgfqpoint{2.677988in}{3.128397in}}{\pgfqpoint{2.670088in}{3.125125in}}{\pgfqpoint{2.664264in}{3.119301in}}%
\pgfpathcurveto{\pgfqpoint{2.658440in}{3.113477in}}{\pgfqpoint{2.655168in}{3.105577in}}{\pgfqpoint{2.655168in}{3.097341in}}%
\pgfpathcurveto{\pgfqpoint{2.655168in}{3.089104in}}{\pgfqpoint{2.658440in}{3.081204in}}{\pgfqpoint{2.664264in}{3.075381in}}%
\pgfpathcurveto{\pgfqpoint{2.670088in}{3.069557in}}{\pgfqpoint{2.677988in}{3.066284in}}{\pgfqpoint{2.686224in}{3.066284in}}%
\pgfpathclose%
\pgfusepath{stroke,fill}%
\end{pgfscope}%
\begin{pgfscope}%
\pgfpathrectangle{\pgfqpoint{0.100000in}{0.220728in}}{\pgfqpoint{3.696000in}{3.696000in}}%
\pgfusepath{clip}%
\pgfsetbuttcap%
\pgfsetroundjoin%
\definecolor{currentfill}{rgb}{0.121569,0.466667,0.705882}%
\pgfsetfillcolor{currentfill}%
\pgfsetfillopacity{0.515507}%
\pgfsetlinewidth{1.003750pt}%
\definecolor{currentstroke}{rgb}{0.121569,0.466667,0.705882}%
\pgfsetstrokecolor{currentstroke}%
\pgfsetstrokeopacity{0.515507}%
\pgfsetdash{}{0pt}%
\pgfpathmoveto{\pgfqpoint{1.141991in}{1.873876in}}%
\pgfpathcurveto{\pgfqpoint{1.150227in}{1.873876in}}{\pgfqpoint{1.158127in}{1.877148in}}{\pgfqpoint{1.163951in}{1.882972in}}%
\pgfpathcurveto{\pgfqpoint{1.169775in}{1.888796in}}{\pgfqpoint{1.173048in}{1.896696in}}{\pgfqpoint{1.173048in}{1.904932in}}%
\pgfpathcurveto{\pgfqpoint{1.173048in}{1.913169in}}{\pgfqpoint{1.169775in}{1.921069in}}{\pgfqpoint{1.163951in}{1.926893in}}%
\pgfpathcurveto{\pgfqpoint{1.158127in}{1.932716in}}{\pgfqpoint{1.150227in}{1.935989in}}{\pgfqpoint{1.141991in}{1.935989in}}%
\pgfpathcurveto{\pgfqpoint{1.133755in}{1.935989in}}{\pgfqpoint{1.125855in}{1.932716in}}{\pgfqpoint{1.120031in}{1.926893in}}%
\pgfpathcurveto{\pgfqpoint{1.114207in}{1.921069in}}{\pgfqpoint{1.110935in}{1.913169in}}{\pgfqpoint{1.110935in}{1.904932in}}%
\pgfpathcurveto{\pgfqpoint{1.110935in}{1.896696in}}{\pgfqpoint{1.114207in}{1.888796in}}{\pgfqpoint{1.120031in}{1.882972in}}%
\pgfpathcurveto{\pgfqpoint{1.125855in}{1.877148in}}{\pgfqpoint{1.133755in}{1.873876in}}{\pgfqpoint{1.141991in}{1.873876in}}%
\pgfpathclose%
\pgfusepath{stroke,fill}%
\end{pgfscope}%
\begin{pgfscope}%
\pgfpathrectangle{\pgfqpoint{0.100000in}{0.220728in}}{\pgfqpoint{3.696000in}{3.696000in}}%
\pgfusepath{clip}%
\pgfsetbuttcap%
\pgfsetroundjoin%
\definecolor{currentfill}{rgb}{0.121569,0.466667,0.705882}%
\pgfsetfillcolor{currentfill}%
\pgfsetfillopacity{0.517129}%
\pgfsetlinewidth{1.003750pt}%
\definecolor{currentstroke}{rgb}{0.121569,0.466667,0.705882}%
\pgfsetstrokecolor{currentstroke}%
\pgfsetstrokeopacity{0.517129}%
\pgfsetdash{}{0pt}%
\pgfpathmoveto{\pgfqpoint{2.694141in}{3.065441in}}%
\pgfpathcurveto{\pgfqpoint{2.702377in}{3.065441in}}{\pgfqpoint{2.710277in}{3.068714in}}{\pgfqpoint{2.716101in}{3.074538in}}%
\pgfpathcurveto{\pgfqpoint{2.721925in}{3.080361in}}{\pgfqpoint{2.725197in}{3.088262in}}{\pgfqpoint{2.725197in}{3.096498in}}%
\pgfpathcurveto{\pgfqpoint{2.725197in}{3.104734in}}{\pgfqpoint{2.721925in}{3.112634in}}{\pgfqpoint{2.716101in}{3.118458in}}%
\pgfpathcurveto{\pgfqpoint{2.710277in}{3.124282in}}{\pgfqpoint{2.702377in}{3.127554in}}{\pgfqpoint{2.694141in}{3.127554in}}%
\pgfpathcurveto{\pgfqpoint{2.685904in}{3.127554in}}{\pgfqpoint{2.678004in}{3.124282in}}{\pgfqpoint{2.672180in}{3.118458in}}%
\pgfpathcurveto{\pgfqpoint{2.666356in}{3.112634in}}{\pgfqpoint{2.663084in}{3.104734in}}{\pgfqpoint{2.663084in}{3.096498in}}%
\pgfpathcurveto{\pgfqpoint{2.663084in}{3.088262in}}{\pgfqpoint{2.666356in}{3.080361in}}{\pgfqpoint{2.672180in}{3.074538in}}%
\pgfpathcurveto{\pgfqpoint{2.678004in}{3.068714in}}{\pgfqpoint{2.685904in}{3.065441in}}{\pgfqpoint{2.694141in}{3.065441in}}%
\pgfpathclose%
\pgfusepath{stroke,fill}%
\end{pgfscope}%
\begin{pgfscope}%
\pgfpathrectangle{\pgfqpoint{0.100000in}{0.220728in}}{\pgfqpoint{3.696000in}{3.696000in}}%
\pgfusepath{clip}%
\pgfsetbuttcap%
\pgfsetroundjoin%
\definecolor{currentfill}{rgb}{0.121569,0.466667,0.705882}%
\pgfsetfillcolor{currentfill}%
\pgfsetfillopacity{0.517280}%
\pgfsetlinewidth{1.003750pt}%
\definecolor{currentstroke}{rgb}{0.121569,0.466667,0.705882}%
\pgfsetstrokecolor{currentstroke}%
\pgfsetstrokeopacity{0.517280}%
\pgfsetdash{}{0pt}%
\pgfpathmoveto{\pgfqpoint{1.139972in}{1.861851in}}%
\pgfpathcurveto{\pgfqpoint{1.148209in}{1.861851in}}{\pgfqpoint{1.156109in}{1.865124in}}{\pgfqpoint{1.161933in}{1.870948in}}%
\pgfpathcurveto{\pgfqpoint{1.167757in}{1.876772in}}{\pgfqpoint{1.171029in}{1.884672in}}{\pgfqpoint{1.171029in}{1.892908in}}%
\pgfpathcurveto{\pgfqpoint{1.171029in}{1.901144in}}{\pgfqpoint{1.167757in}{1.909044in}}{\pgfqpoint{1.161933in}{1.914868in}}%
\pgfpathcurveto{\pgfqpoint{1.156109in}{1.920692in}}{\pgfqpoint{1.148209in}{1.923964in}}{\pgfqpoint{1.139972in}{1.923964in}}%
\pgfpathcurveto{\pgfqpoint{1.131736in}{1.923964in}}{\pgfqpoint{1.123836in}{1.920692in}}{\pgfqpoint{1.118012in}{1.914868in}}%
\pgfpathcurveto{\pgfqpoint{1.112188in}{1.909044in}}{\pgfqpoint{1.108916in}{1.901144in}}{\pgfqpoint{1.108916in}{1.892908in}}%
\pgfpathcurveto{\pgfqpoint{1.108916in}{1.884672in}}{\pgfqpoint{1.112188in}{1.876772in}}{\pgfqpoint{1.118012in}{1.870948in}}%
\pgfpathcurveto{\pgfqpoint{1.123836in}{1.865124in}}{\pgfqpoint{1.131736in}{1.861851in}}{\pgfqpoint{1.139972in}{1.861851in}}%
\pgfpathclose%
\pgfusepath{stroke,fill}%
\end{pgfscope}%
\begin{pgfscope}%
\pgfpathrectangle{\pgfqpoint{0.100000in}{0.220728in}}{\pgfqpoint{3.696000in}{3.696000in}}%
\pgfusepath{clip}%
\pgfsetbuttcap%
\pgfsetroundjoin%
\definecolor{currentfill}{rgb}{0.121569,0.466667,0.705882}%
\pgfsetfillcolor{currentfill}%
\pgfsetfillopacity{0.518036}%
\pgfsetlinewidth{1.003750pt}%
\definecolor{currentstroke}{rgb}{0.121569,0.466667,0.705882}%
\pgfsetstrokecolor{currentstroke}%
\pgfsetstrokeopacity{0.518036}%
\pgfsetdash{}{0pt}%
\pgfpathmoveto{\pgfqpoint{1.134141in}{1.855773in}}%
\pgfpathcurveto{\pgfqpoint{1.142377in}{1.855773in}}{\pgfqpoint{1.150277in}{1.859045in}}{\pgfqpoint{1.156101in}{1.864869in}}%
\pgfpathcurveto{\pgfqpoint{1.161925in}{1.870693in}}{\pgfqpoint{1.165198in}{1.878593in}}{\pgfqpoint{1.165198in}{1.886830in}}%
\pgfpathcurveto{\pgfqpoint{1.165198in}{1.895066in}}{\pgfqpoint{1.161925in}{1.902966in}}{\pgfqpoint{1.156101in}{1.908790in}}%
\pgfpathcurveto{\pgfqpoint{1.150277in}{1.914614in}}{\pgfqpoint{1.142377in}{1.917886in}}{\pgfqpoint{1.134141in}{1.917886in}}%
\pgfpathcurveto{\pgfqpoint{1.125905in}{1.917886in}}{\pgfqpoint{1.118005in}{1.914614in}}{\pgfqpoint{1.112181in}{1.908790in}}%
\pgfpathcurveto{\pgfqpoint{1.106357in}{1.902966in}}{\pgfqpoint{1.103085in}{1.895066in}}{\pgfqpoint{1.103085in}{1.886830in}}%
\pgfpathcurveto{\pgfqpoint{1.103085in}{1.878593in}}{\pgfqpoint{1.106357in}{1.870693in}}{\pgfqpoint{1.112181in}{1.864869in}}%
\pgfpathcurveto{\pgfqpoint{1.118005in}{1.859045in}}{\pgfqpoint{1.125905in}{1.855773in}}{\pgfqpoint{1.134141in}{1.855773in}}%
\pgfpathclose%
\pgfusepath{stroke,fill}%
\end{pgfscope}%
\begin{pgfscope}%
\pgfpathrectangle{\pgfqpoint{0.100000in}{0.220728in}}{\pgfqpoint{3.696000in}{3.696000in}}%
\pgfusepath{clip}%
\pgfsetbuttcap%
\pgfsetroundjoin%
\definecolor{currentfill}{rgb}{0.121569,0.466667,0.705882}%
\pgfsetfillcolor{currentfill}%
\pgfsetfillopacity{0.518833}%
\pgfsetlinewidth{1.003750pt}%
\definecolor{currentstroke}{rgb}{0.121569,0.466667,0.705882}%
\pgfsetstrokecolor{currentstroke}%
\pgfsetstrokeopacity{0.518833}%
\pgfsetdash{}{0pt}%
\pgfpathmoveto{\pgfqpoint{2.705700in}{3.061996in}}%
\pgfpathcurveto{\pgfqpoint{2.713936in}{3.061996in}}{\pgfqpoint{2.721836in}{3.065268in}}{\pgfqpoint{2.727660in}{3.071092in}}%
\pgfpathcurveto{\pgfqpoint{2.733484in}{3.076916in}}{\pgfqpoint{2.736757in}{3.084816in}}{\pgfqpoint{2.736757in}{3.093053in}}%
\pgfpathcurveto{\pgfqpoint{2.736757in}{3.101289in}}{\pgfqpoint{2.733484in}{3.109189in}}{\pgfqpoint{2.727660in}{3.115013in}}%
\pgfpathcurveto{\pgfqpoint{2.721836in}{3.120837in}}{\pgfqpoint{2.713936in}{3.124109in}}{\pgfqpoint{2.705700in}{3.124109in}}%
\pgfpathcurveto{\pgfqpoint{2.697464in}{3.124109in}}{\pgfqpoint{2.689564in}{3.120837in}}{\pgfqpoint{2.683740in}{3.115013in}}%
\pgfpathcurveto{\pgfqpoint{2.677916in}{3.109189in}}{\pgfqpoint{2.674644in}{3.101289in}}{\pgfqpoint{2.674644in}{3.093053in}}%
\pgfpathcurveto{\pgfqpoint{2.674644in}{3.084816in}}{\pgfqpoint{2.677916in}{3.076916in}}{\pgfqpoint{2.683740in}{3.071092in}}%
\pgfpathcurveto{\pgfqpoint{2.689564in}{3.065268in}}{\pgfqpoint{2.697464in}{3.061996in}}{\pgfqpoint{2.705700in}{3.061996in}}%
\pgfpathclose%
\pgfusepath{stroke,fill}%
\end{pgfscope}%
\begin{pgfscope}%
\pgfpathrectangle{\pgfqpoint{0.100000in}{0.220728in}}{\pgfqpoint{3.696000in}{3.696000in}}%
\pgfusepath{clip}%
\pgfsetbuttcap%
\pgfsetroundjoin%
\definecolor{currentfill}{rgb}{0.121569,0.466667,0.705882}%
\pgfsetfillcolor{currentfill}%
\pgfsetfillopacity{0.519114}%
\pgfsetlinewidth{1.003750pt}%
\definecolor{currentstroke}{rgb}{0.121569,0.466667,0.705882}%
\pgfsetstrokecolor{currentstroke}%
\pgfsetstrokeopacity{0.519114}%
\pgfsetdash{}{0pt}%
\pgfpathmoveto{\pgfqpoint{1.132793in}{1.848680in}}%
\pgfpathcurveto{\pgfqpoint{1.141029in}{1.848680in}}{\pgfqpoint{1.148929in}{1.851953in}}{\pgfqpoint{1.154753in}{1.857777in}}%
\pgfpathcurveto{\pgfqpoint{1.160577in}{1.863601in}}{\pgfqpoint{1.163849in}{1.871501in}}{\pgfqpoint{1.163849in}{1.879737in}}%
\pgfpathcurveto{\pgfqpoint{1.163849in}{1.887973in}}{\pgfqpoint{1.160577in}{1.895873in}}{\pgfqpoint{1.154753in}{1.901697in}}%
\pgfpathcurveto{\pgfqpoint{1.148929in}{1.907521in}}{\pgfqpoint{1.141029in}{1.910793in}}{\pgfqpoint{1.132793in}{1.910793in}}%
\pgfpathcurveto{\pgfqpoint{1.124557in}{1.910793in}}{\pgfqpoint{1.116657in}{1.907521in}}{\pgfqpoint{1.110833in}{1.901697in}}%
\pgfpathcurveto{\pgfqpoint{1.105009in}{1.895873in}}{\pgfqpoint{1.101736in}{1.887973in}}{\pgfqpoint{1.101736in}{1.879737in}}%
\pgfpathcurveto{\pgfqpoint{1.101736in}{1.871501in}}{\pgfqpoint{1.105009in}{1.863601in}}{\pgfqpoint{1.110833in}{1.857777in}}%
\pgfpathcurveto{\pgfqpoint{1.116657in}{1.851953in}}{\pgfqpoint{1.124557in}{1.848680in}}{\pgfqpoint{1.132793in}{1.848680in}}%
\pgfpathclose%
\pgfusepath{stroke,fill}%
\end{pgfscope}%
\begin{pgfscope}%
\pgfpathrectangle{\pgfqpoint{0.100000in}{0.220728in}}{\pgfqpoint{3.696000in}{3.696000in}}%
\pgfusepath{clip}%
\pgfsetbuttcap%
\pgfsetroundjoin%
\definecolor{currentfill}{rgb}{0.121569,0.466667,0.705882}%
\pgfsetfillcolor{currentfill}%
\pgfsetfillopacity{0.519524}%
\pgfsetlinewidth{1.003750pt}%
\definecolor{currentstroke}{rgb}{0.121569,0.466667,0.705882}%
\pgfsetstrokecolor{currentstroke}%
\pgfsetstrokeopacity{0.519524}%
\pgfsetdash{}{0pt}%
\pgfpathmoveto{\pgfqpoint{1.129225in}{1.844774in}}%
\pgfpathcurveto{\pgfqpoint{1.137461in}{1.844774in}}{\pgfqpoint{1.145361in}{1.848046in}}{\pgfqpoint{1.151185in}{1.853870in}}%
\pgfpathcurveto{\pgfqpoint{1.157009in}{1.859694in}}{\pgfqpoint{1.160281in}{1.867594in}}{\pgfqpoint{1.160281in}{1.875830in}}%
\pgfpathcurveto{\pgfqpoint{1.160281in}{1.884067in}}{\pgfqpoint{1.157009in}{1.891967in}}{\pgfqpoint{1.151185in}{1.897791in}}%
\pgfpathcurveto{\pgfqpoint{1.145361in}{1.903615in}}{\pgfqpoint{1.137461in}{1.906887in}}{\pgfqpoint{1.129225in}{1.906887in}}%
\pgfpathcurveto{\pgfqpoint{1.120988in}{1.906887in}}{\pgfqpoint{1.113088in}{1.903615in}}{\pgfqpoint{1.107264in}{1.897791in}}%
\pgfpathcurveto{\pgfqpoint{1.101441in}{1.891967in}}{\pgfqpoint{1.098168in}{1.884067in}}{\pgfqpoint{1.098168in}{1.875830in}}%
\pgfpathcurveto{\pgfqpoint{1.098168in}{1.867594in}}{\pgfqpoint{1.101441in}{1.859694in}}{\pgfqpoint{1.107264in}{1.853870in}}%
\pgfpathcurveto{\pgfqpoint{1.113088in}{1.848046in}}{\pgfqpoint{1.120988in}{1.844774in}}{\pgfqpoint{1.129225in}{1.844774in}}%
\pgfpathclose%
\pgfusepath{stroke,fill}%
\end{pgfscope}%
\begin{pgfscope}%
\pgfpathrectangle{\pgfqpoint{0.100000in}{0.220728in}}{\pgfqpoint{3.696000in}{3.696000in}}%
\pgfusepath{clip}%
\pgfsetbuttcap%
\pgfsetroundjoin%
\definecolor{currentfill}{rgb}{0.121569,0.466667,0.705882}%
\pgfsetfillcolor{currentfill}%
\pgfsetfillopacity{0.520092}%
\pgfsetlinewidth{1.003750pt}%
\definecolor{currentstroke}{rgb}{0.121569,0.466667,0.705882}%
\pgfsetstrokecolor{currentstroke}%
\pgfsetstrokeopacity{0.520092}%
\pgfsetdash{}{0pt}%
\pgfpathmoveto{\pgfqpoint{1.128219in}{1.841327in}}%
\pgfpathcurveto{\pgfqpoint{1.136455in}{1.841327in}}{\pgfqpoint{1.144355in}{1.844599in}}{\pgfqpoint{1.150179in}{1.850423in}}%
\pgfpathcurveto{\pgfqpoint{1.156003in}{1.856247in}}{\pgfqpoint{1.159276in}{1.864147in}}{\pgfqpoint{1.159276in}{1.872384in}}%
\pgfpathcurveto{\pgfqpoint{1.159276in}{1.880620in}}{\pgfqpoint{1.156003in}{1.888520in}}{\pgfqpoint{1.150179in}{1.894344in}}%
\pgfpathcurveto{\pgfqpoint{1.144355in}{1.900168in}}{\pgfqpoint{1.136455in}{1.903440in}}{\pgfqpoint{1.128219in}{1.903440in}}%
\pgfpathcurveto{\pgfqpoint{1.119983in}{1.903440in}}{\pgfqpoint{1.112083in}{1.900168in}}{\pgfqpoint{1.106259in}{1.894344in}}%
\pgfpathcurveto{\pgfqpoint{1.100435in}{1.888520in}}{\pgfqpoint{1.097163in}{1.880620in}}{\pgfqpoint{1.097163in}{1.872384in}}%
\pgfpathcurveto{\pgfqpoint{1.097163in}{1.864147in}}{\pgfqpoint{1.100435in}{1.856247in}}{\pgfqpoint{1.106259in}{1.850423in}}%
\pgfpathcurveto{\pgfqpoint{1.112083in}{1.844599in}}{\pgfqpoint{1.119983in}{1.841327in}}{\pgfqpoint{1.128219in}{1.841327in}}%
\pgfpathclose%
\pgfusepath{stroke,fill}%
\end{pgfscope}%
\begin{pgfscope}%
\pgfpathrectangle{\pgfqpoint{0.100000in}{0.220728in}}{\pgfqpoint{3.696000in}{3.696000in}}%
\pgfusepath{clip}%
\pgfsetbuttcap%
\pgfsetroundjoin%
\definecolor{currentfill}{rgb}{0.121569,0.466667,0.705882}%
\pgfsetfillcolor{currentfill}%
\pgfsetfillopacity{0.520733}%
\pgfsetlinewidth{1.003750pt}%
\definecolor{currentstroke}{rgb}{0.121569,0.466667,0.705882}%
\pgfsetstrokecolor{currentstroke}%
\pgfsetstrokeopacity{0.520733}%
\pgfsetdash{}{0pt}%
\pgfpathmoveto{\pgfqpoint{1.123985in}{1.836770in}}%
\pgfpathcurveto{\pgfqpoint{1.132221in}{1.836770in}}{\pgfqpoint{1.140121in}{1.840042in}}{\pgfqpoint{1.145945in}{1.845866in}}%
\pgfpathcurveto{\pgfqpoint{1.151769in}{1.851690in}}{\pgfqpoint{1.155042in}{1.859590in}}{\pgfqpoint{1.155042in}{1.867826in}}%
\pgfpathcurveto{\pgfqpoint{1.155042in}{1.876062in}}{\pgfqpoint{1.151769in}{1.883962in}}{\pgfqpoint{1.145945in}{1.889786in}}%
\pgfpathcurveto{\pgfqpoint{1.140121in}{1.895610in}}{\pgfqpoint{1.132221in}{1.898883in}}{\pgfqpoint{1.123985in}{1.898883in}}%
\pgfpathcurveto{\pgfqpoint{1.115749in}{1.898883in}}{\pgfqpoint{1.107849in}{1.895610in}}{\pgfqpoint{1.102025in}{1.889786in}}%
\pgfpathcurveto{\pgfqpoint{1.096201in}{1.883962in}}{\pgfqpoint{1.092929in}{1.876062in}}{\pgfqpoint{1.092929in}{1.867826in}}%
\pgfpathcurveto{\pgfqpoint{1.092929in}{1.859590in}}{\pgfqpoint{1.096201in}{1.851690in}}{\pgfqpoint{1.102025in}{1.845866in}}%
\pgfpathcurveto{\pgfqpoint{1.107849in}{1.840042in}}{\pgfqpoint{1.115749in}{1.836770in}}{\pgfqpoint{1.123985in}{1.836770in}}%
\pgfpathclose%
\pgfusepath{stroke,fill}%
\end{pgfscope}%
\begin{pgfscope}%
\pgfpathrectangle{\pgfqpoint{0.100000in}{0.220728in}}{\pgfqpoint{3.696000in}{3.696000in}}%
\pgfusepath{clip}%
\pgfsetbuttcap%
\pgfsetroundjoin%
\definecolor{currentfill}{rgb}{0.121569,0.466667,0.705882}%
\pgfsetfillcolor{currentfill}%
\pgfsetfillopacity{0.521502}%
\pgfsetlinewidth{1.003750pt}%
\definecolor{currentstroke}{rgb}{0.121569,0.466667,0.705882}%
\pgfsetstrokecolor{currentstroke}%
\pgfsetstrokeopacity{0.521502}%
\pgfsetdash{}{0pt}%
\pgfpathmoveto{\pgfqpoint{1.121761in}{1.832245in}}%
\pgfpathcurveto{\pgfqpoint{1.129997in}{1.832245in}}{\pgfqpoint{1.137897in}{1.835517in}}{\pgfqpoint{1.143721in}{1.841341in}}%
\pgfpathcurveto{\pgfqpoint{1.149545in}{1.847165in}}{\pgfqpoint{1.152817in}{1.855065in}}{\pgfqpoint{1.152817in}{1.863301in}}%
\pgfpathcurveto{\pgfqpoint{1.152817in}{1.871538in}}{\pgfqpoint{1.149545in}{1.879438in}}{\pgfqpoint{1.143721in}{1.885262in}}%
\pgfpathcurveto{\pgfqpoint{1.137897in}{1.891086in}}{\pgfqpoint{1.129997in}{1.894358in}}{\pgfqpoint{1.121761in}{1.894358in}}%
\pgfpathcurveto{\pgfqpoint{1.113524in}{1.894358in}}{\pgfqpoint{1.105624in}{1.891086in}}{\pgfqpoint{1.099800in}{1.885262in}}%
\pgfpathcurveto{\pgfqpoint{1.093977in}{1.879438in}}{\pgfqpoint{1.090704in}{1.871538in}}{\pgfqpoint{1.090704in}{1.863301in}}%
\pgfpathcurveto{\pgfqpoint{1.090704in}{1.855065in}}{\pgfqpoint{1.093977in}{1.847165in}}{\pgfqpoint{1.099800in}{1.841341in}}%
\pgfpathcurveto{\pgfqpoint{1.105624in}{1.835517in}}{\pgfqpoint{1.113524in}{1.832245in}}{\pgfqpoint{1.121761in}{1.832245in}}%
\pgfpathclose%
\pgfusepath{stroke,fill}%
\end{pgfscope}%
\begin{pgfscope}%
\pgfpathrectangle{\pgfqpoint{0.100000in}{0.220728in}}{\pgfqpoint{3.696000in}{3.696000in}}%
\pgfusepath{clip}%
\pgfsetbuttcap%
\pgfsetroundjoin%
\definecolor{currentfill}{rgb}{0.121569,0.466667,0.705882}%
\pgfsetfillcolor{currentfill}%
\pgfsetfillopacity{0.522091}%
\pgfsetlinewidth{1.003750pt}%
\definecolor{currentstroke}{rgb}{0.121569,0.466667,0.705882}%
\pgfsetstrokecolor{currentstroke}%
\pgfsetstrokeopacity{0.522091}%
\pgfsetdash{}{0pt}%
\pgfpathmoveto{\pgfqpoint{1.119797in}{1.828749in}}%
\pgfpathcurveto{\pgfqpoint{1.128033in}{1.828749in}}{\pgfqpoint{1.135933in}{1.832022in}}{\pgfqpoint{1.141757in}{1.837846in}}%
\pgfpathcurveto{\pgfqpoint{1.147581in}{1.843670in}}{\pgfqpoint{1.150853in}{1.851570in}}{\pgfqpoint{1.150853in}{1.859806in}}%
\pgfpathcurveto{\pgfqpoint{1.150853in}{1.868042in}}{\pgfqpoint{1.147581in}{1.875942in}}{\pgfqpoint{1.141757in}{1.881766in}}%
\pgfpathcurveto{\pgfqpoint{1.135933in}{1.887590in}}{\pgfqpoint{1.128033in}{1.890862in}}{\pgfqpoint{1.119797in}{1.890862in}}%
\pgfpathcurveto{\pgfqpoint{1.111560in}{1.890862in}}{\pgfqpoint{1.103660in}{1.887590in}}{\pgfqpoint{1.097836in}{1.881766in}}%
\pgfpathcurveto{\pgfqpoint{1.092013in}{1.875942in}}{\pgfqpoint{1.088740in}{1.868042in}}{\pgfqpoint{1.088740in}{1.859806in}}%
\pgfpathcurveto{\pgfqpoint{1.088740in}{1.851570in}}{\pgfqpoint{1.092013in}{1.843670in}}{\pgfqpoint{1.097836in}{1.837846in}}%
\pgfpathcurveto{\pgfqpoint{1.103660in}{1.832022in}}{\pgfqpoint{1.111560in}{1.828749in}}{\pgfqpoint{1.119797in}{1.828749in}}%
\pgfpathclose%
\pgfusepath{stroke,fill}%
\end{pgfscope}%
\begin{pgfscope}%
\pgfpathrectangle{\pgfqpoint{0.100000in}{0.220728in}}{\pgfqpoint{3.696000in}{3.696000in}}%
\pgfusepath{clip}%
\pgfsetbuttcap%
\pgfsetroundjoin%
\definecolor{currentfill}{rgb}{0.121569,0.466667,0.705882}%
\pgfsetfillcolor{currentfill}%
\pgfsetfillopacity{0.522585}%
\pgfsetlinewidth{1.003750pt}%
\definecolor{currentstroke}{rgb}{0.121569,0.466667,0.705882}%
\pgfsetstrokecolor{currentstroke}%
\pgfsetstrokeopacity{0.522585}%
\pgfsetdash{}{0pt}%
\pgfpathmoveto{\pgfqpoint{2.719126in}{3.060236in}}%
\pgfpathcurveto{\pgfqpoint{2.727363in}{3.060236in}}{\pgfqpoint{2.735263in}{3.063509in}}{\pgfqpoint{2.741087in}{3.069333in}}%
\pgfpathcurveto{\pgfqpoint{2.746911in}{3.075157in}}{\pgfqpoint{2.750183in}{3.083057in}}{\pgfqpoint{2.750183in}{3.091293in}}%
\pgfpathcurveto{\pgfqpoint{2.750183in}{3.099529in}}{\pgfqpoint{2.746911in}{3.107429in}}{\pgfqpoint{2.741087in}{3.113253in}}%
\pgfpathcurveto{\pgfqpoint{2.735263in}{3.119077in}}{\pgfqpoint{2.727363in}{3.122349in}}{\pgfqpoint{2.719126in}{3.122349in}}%
\pgfpathcurveto{\pgfqpoint{2.710890in}{3.122349in}}{\pgfqpoint{2.702990in}{3.119077in}}{\pgfqpoint{2.697166in}{3.113253in}}%
\pgfpathcurveto{\pgfqpoint{2.691342in}{3.107429in}}{\pgfqpoint{2.688070in}{3.099529in}}{\pgfqpoint{2.688070in}{3.091293in}}%
\pgfpathcurveto{\pgfqpoint{2.688070in}{3.083057in}}{\pgfqpoint{2.691342in}{3.075157in}}{\pgfqpoint{2.697166in}{3.069333in}}%
\pgfpathcurveto{\pgfqpoint{2.702990in}{3.063509in}}{\pgfqpoint{2.710890in}{3.060236in}}{\pgfqpoint{2.719126in}{3.060236in}}%
\pgfpathclose%
\pgfusepath{stroke,fill}%
\end{pgfscope}%
\begin{pgfscope}%
\pgfpathrectangle{\pgfqpoint{0.100000in}{0.220728in}}{\pgfqpoint{3.696000in}{3.696000in}}%
\pgfusepath{clip}%
\pgfsetbuttcap%
\pgfsetroundjoin%
\definecolor{currentfill}{rgb}{0.121569,0.466667,0.705882}%
\pgfsetfillcolor{currentfill}%
\pgfsetfillopacity{0.523056}%
\pgfsetlinewidth{1.003750pt}%
\definecolor{currentstroke}{rgb}{0.121569,0.466667,0.705882}%
\pgfsetstrokecolor{currentstroke}%
\pgfsetstrokeopacity{0.523056}%
\pgfsetdash{}{0pt}%
\pgfpathmoveto{\pgfqpoint{1.115834in}{1.822537in}}%
\pgfpathcurveto{\pgfqpoint{1.124070in}{1.822537in}}{\pgfqpoint{1.131970in}{1.825810in}}{\pgfqpoint{1.137794in}{1.831633in}}%
\pgfpathcurveto{\pgfqpoint{1.143618in}{1.837457in}}{\pgfqpoint{1.146890in}{1.845357in}}{\pgfqpoint{1.146890in}{1.853594in}}%
\pgfpathcurveto{\pgfqpoint{1.146890in}{1.861830in}}{\pgfqpoint{1.143618in}{1.869730in}}{\pgfqpoint{1.137794in}{1.875554in}}%
\pgfpathcurveto{\pgfqpoint{1.131970in}{1.881378in}}{\pgfqpoint{1.124070in}{1.884650in}}{\pgfqpoint{1.115834in}{1.884650in}}%
\pgfpathcurveto{\pgfqpoint{1.107597in}{1.884650in}}{\pgfqpoint{1.099697in}{1.881378in}}{\pgfqpoint{1.093873in}{1.875554in}}%
\pgfpathcurveto{\pgfqpoint{1.088049in}{1.869730in}}{\pgfqpoint{1.084777in}{1.861830in}}{\pgfqpoint{1.084777in}{1.853594in}}%
\pgfpathcurveto{\pgfqpoint{1.084777in}{1.845357in}}{\pgfqpoint{1.088049in}{1.837457in}}{\pgfqpoint{1.093873in}{1.831633in}}%
\pgfpathcurveto{\pgfqpoint{1.099697in}{1.825810in}}{\pgfqpoint{1.107597in}{1.822537in}}{\pgfqpoint{1.115834in}{1.822537in}}%
\pgfpathclose%
\pgfusepath{stroke,fill}%
\end{pgfscope}%
\begin{pgfscope}%
\pgfpathrectangle{\pgfqpoint{0.100000in}{0.220728in}}{\pgfqpoint{3.696000in}{3.696000in}}%
\pgfusepath{clip}%
\pgfsetbuttcap%
\pgfsetroundjoin%
\definecolor{currentfill}{rgb}{0.121569,0.466667,0.705882}%
\pgfsetfillcolor{currentfill}%
\pgfsetfillopacity{0.525178}%
\pgfsetlinewidth{1.003750pt}%
\definecolor{currentstroke}{rgb}{0.121569,0.466667,0.705882}%
\pgfsetstrokecolor{currentstroke}%
\pgfsetstrokeopacity{0.525178}%
\pgfsetdash{}{0pt}%
\pgfpathmoveto{\pgfqpoint{1.109421in}{1.811563in}}%
\pgfpathcurveto{\pgfqpoint{1.117658in}{1.811563in}}{\pgfqpoint{1.125558in}{1.814836in}}{\pgfqpoint{1.131382in}{1.820659in}}%
\pgfpathcurveto{\pgfqpoint{1.137205in}{1.826483in}}{\pgfqpoint{1.140478in}{1.834383in}}{\pgfqpoint{1.140478in}{1.842620in}}%
\pgfpathcurveto{\pgfqpoint{1.140478in}{1.850856in}}{\pgfqpoint{1.137205in}{1.858756in}}{\pgfqpoint{1.131382in}{1.864580in}}%
\pgfpathcurveto{\pgfqpoint{1.125558in}{1.870404in}}{\pgfqpoint{1.117658in}{1.873676in}}{\pgfqpoint{1.109421in}{1.873676in}}%
\pgfpathcurveto{\pgfqpoint{1.101185in}{1.873676in}}{\pgfqpoint{1.093285in}{1.870404in}}{\pgfqpoint{1.087461in}{1.864580in}}%
\pgfpathcurveto{\pgfqpoint{1.081637in}{1.858756in}}{\pgfqpoint{1.078365in}{1.850856in}}{\pgfqpoint{1.078365in}{1.842620in}}%
\pgfpathcurveto{\pgfqpoint{1.078365in}{1.834383in}}{\pgfqpoint{1.081637in}{1.826483in}}{\pgfqpoint{1.087461in}{1.820659in}}%
\pgfpathcurveto{\pgfqpoint{1.093285in}{1.814836in}}{\pgfqpoint{1.101185in}{1.811563in}}{\pgfqpoint{1.109421in}{1.811563in}}%
\pgfpathclose%
\pgfusepath{stroke,fill}%
\end{pgfscope}%
\begin{pgfscope}%
\pgfpathrectangle{\pgfqpoint{0.100000in}{0.220728in}}{\pgfqpoint{3.696000in}{3.696000in}}%
\pgfusepath{clip}%
\pgfsetbuttcap%
\pgfsetroundjoin%
\definecolor{currentfill}{rgb}{0.121569,0.466667,0.705882}%
\pgfsetfillcolor{currentfill}%
\pgfsetfillopacity{0.526430}%
\pgfsetlinewidth{1.003750pt}%
\definecolor{currentstroke}{rgb}{0.121569,0.466667,0.705882}%
\pgfsetstrokecolor{currentstroke}%
\pgfsetstrokeopacity{0.526430}%
\pgfsetdash{}{0pt}%
\pgfpathmoveto{\pgfqpoint{1.102606in}{1.803069in}}%
\pgfpathcurveto{\pgfqpoint{1.110842in}{1.803069in}}{\pgfqpoint{1.118742in}{1.806341in}}{\pgfqpoint{1.124566in}{1.812165in}}%
\pgfpathcurveto{\pgfqpoint{1.130390in}{1.817989in}}{\pgfqpoint{1.133662in}{1.825889in}}{\pgfqpoint{1.133662in}{1.834125in}}%
\pgfpathcurveto{\pgfqpoint{1.133662in}{1.842362in}}{\pgfqpoint{1.130390in}{1.850262in}}{\pgfqpoint{1.124566in}{1.856086in}}%
\pgfpathcurveto{\pgfqpoint{1.118742in}{1.861909in}}{\pgfqpoint{1.110842in}{1.865182in}}{\pgfqpoint{1.102606in}{1.865182in}}%
\pgfpathcurveto{\pgfqpoint{1.094370in}{1.865182in}}{\pgfqpoint{1.086469in}{1.861909in}}{\pgfqpoint{1.080646in}{1.856086in}}%
\pgfpathcurveto{\pgfqpoint{1.074822in}{1.850262in}}{\pgfqpoint{1.071549in}{1.842362in}}{\pgfqpoint{1.071549in}{1.834125in}}%
\pgfpathcurveto{\pgfqpoint{1.071549in}{1.825889in}}{\pgfqpoint{1.074822in}{1.817989in}}{\pgfqpoint{1.080646in}{1.812165in}}%
\pgfpathcurveto{\pgfqpoint{1.086469in}{1.806341in}}{\pgfqpoint{1.094370in}{1.803069in}}{\pgfqpoint{1.102606in}{1.803069in}}%
\pgfpathclose%
\pgfusepath{stroke,fill}%
\end{pgfscope}%
\begin{pgfscope}%
\pgfpathrectangle{\pgfqpoint{0.100000in}{0.220728in}}{\pgfqpoint{3.696000in}{3.696000in}}%
\pgfusepath{clip}%
\pgfsetbuttcap%
\pgfsetroundjoin%
\definecolor{currentfill}{rgb}{0.121569,0.466667,0.705882}%
\pgfsetfillcolor{currentfill}%
\pgfsetfillopacity{0.526682}%
\pgfsetlinewidth{1.003750pt}%
\definecolor{currentstroke}{rgb}{0.121569,0.466667,0.705882}%
\pgfsetstrokecolor{currentstroke}%
\pgfsetstrokeopacity{0.526682}%
\pgfsetdash{}{0pt}%
\pgfpathmoveto{\pgfqpoint{2.732589in}{3.057269in}}%
\pgfpathcurveto{\pgfqpoint{2.740825in}{3.057269in}}{\pgfqpoint{2.748725in}{3.060542in}}{\pgfqpoint{2.754549in}{3.066366in}}%
\pgfpathcurveto{\pgfqpoint{2.760373in}{3.072190in}}{\pgfqpoint{2.763645in}{3.080090in}}{\pgfqpoint{2.763645in}{3.088326in}}%
\pgfpathcurveto{\pgfqpoint{2.763645in}{3.096562in}}{\pgfqpoint{2.760373in}{3.104462in}}{\pgfqpoint{2.754549in}{3.110286in}}%
\pgfpathcurveto{\pgfqpoint{2.748725in}{3.116110in}}{\pgfqpoint{2.740825in}{3.119382in}}{\pgfqpoint{2.732589in}{3.119382in}}%
\pgfpathcurveto{\pgfqpoint{2.724353in}{3.119382in}}{\pgfqpoint{2.716453in}{3.116110in}}{\pgfqpoint{2.710629in}{3.110286in}}%
\pgfpathcurveto{\pgfqpoint{2.704805in}{3.104462in}}{\pgfqpoint{2.701532in}{3.096562in}}{\pgfqpoint{2.701532in}{3.088326in}}%
\pgfpathcurveto{\pgfqpoint{2.701532in}{3.080090in}}{\pgfqpoint{2.704805in}{3.072190in}}{\pgfqpoint{2.710629in}{3.066366in}}%
\pgfpathcurveto{\pgfqpoint{2.716453in}{3.060542in}}{\pgfqpoint{2.724353in}{3.057269in}}{\pgfqpoint{2.732589in}{3.057269in}}%
\pgfpathclose%
\pgfusepath{stroke,fill}%
\end{pgfscope}%
\begin{pgfscope}%
\pgfpathrectangle{\pgfqpoint{0.100000in}{0.220728in}}{\pgfqpoint{3.696000in}{3.696000in}}%
\pgfusepath{clip}%
\pgfsetbuttcap%
\pgfsetroundjoin%
\definecolor{currentfill}{rgb}{0.121569,0.466667,0.705882}%
\pgfsetfillcolor{currentfill}%
\pgfsetfillopacity{0.528778}%
\pgfsetlinewidth{1.003750pt}%
\definecolor{currentstroke}{rgb}{0.121569,0.466667,0.705882}%
\pgfsetstrokecolor{currentstroke}%
\pgfsetstrokeopacity{0.528778}%
\pgfsetdash{}{0pt}%
\pgfpathmoveto{\pgfqpoint{2.740224in}{3.055622in}}%
\pgfpathcurveto{\pgfqpoint{2.748460in}{3.055622in}}{\pgfqpoint{2.756360in}{3.058895in}}{\pgfqpoint{2.762184in}{3.064719in}}%
\pgfpathcurveto{\pgfqpoint{2.768008in}{3.070543in}}{\pgfqpoint{2.771280in}{3.078443in}}{\pgfqpoint{2.771280in}{3.086679in}}%
\pgfpathcurveto{\pgfqpoint{2.771280in}{3.094915in}}{\pgfqpoint{2.768008in}{3.102815in}}{\pgfqpoint{2.762184in}{3.108639in}}%
\pgfpathcurveto{\pgfqpoint{2.756360in}{3.114463in}}{\pgfqpoint{2.748460in}{3.117735in}}{\pgfqpoint{2.740224in}{3.117735in}}%
\pgfpathcurveto{\pgfqpoint{2.731987in}{3.117735in}}{\pgfqpoint{2.724087in}{3.114463in}}{\pgfqpoint{2.718263in}{3.108639in}}%
\pgfpathcurveto{\pgfqpoint{2.712439in}{3.102815in}}{\pgfqpoint{2.709167in}{3.094915in}}{\pgfqpoint{2.709167in}{3.086679in}}%
\pgfpathcurveto{\pgfqpoint{2.709167in}{3.078443in}}{\pgfqpoint{2.712439in}{3.070543in}}{\pgfqpoint{2.718263in}{3.064719in}}%
\pgfpathcurveto{\pgfqpoint{2.724087in}{3.058895in}}{\pgfqpoint{2.731987in}{3.055622in}}{\pgfqpoint{2.740224in}{3.055622in}}%
\pgfpathclose%
\pgfusepath{stroke,fill}%
\end{pgfscope}%
\begin{pgfscope}%
\pgfpathrectangle{\pgfqpoint{0.100000in}{0.220728in}}{\pgfqpoint{3.696000in}{3.696000in}}%
\pgfusepath{clip}%
\pgfsetbuttcap%
\pgfsetroundjoin%
\definecolor{currentfill}{rgb}{0.121569,0.466667,0.705882}%
\pgfsetfillcolor{currentfill}%
\pgfsetfillopacity{0.529756}%
\pgfsetlinewidth{1.003750pt}%
\definecolor{currentstroke}{rgb}{0.121569,0.466667,0.705882}%
\pgfsetstrokecolor{currentstroke}%
\pgfsetstrokeopacity{0.529756}%
\pgfsetdash{}{0pt}%
\pgfpathmoveto{\pgfqpoint{1.094678in}{1.785292in}}%
\pgfpathcurveto{\pgfqpoint{1.102914in}{1.785292in}}{\pgfqpoint{1.110814in}{1.788565in}}{\pgfqpoint{1.116638in}{1.794388in}}%
\pgfpathcurveto{\pgfqpoint{1.122462in}{1.800212in}}{\pgfqpoint{1.125735in}{1.808112in}}{\pgfqpoint{1.125735in}{1.816349in}}%
\pgfpathcurveto{\pgfqpoint{1.125735in}{1.824585in}}{\pgfqpoint{1.122462in}{1.832485in}}{\pgfqpoint{1.116638in}{1.838309in}}%
\pgfpathcurveto{\pgfqpoint{1.110814in}{1.844133in}}{\pgfqpoint{1.102914in}{1.847405in}}{\pgfqpoint{1.094678in}{1.847405in}}%
\pgfpathcurveto{\pgfqpoint{1.086442in}{1.847405in}}{\pgfqpoint{1.078542in}{1.844133in}}{\pgfqpoint{1.072718in}{1.838309in}}%
\pgfpathcurveto{\pgfqpoint{1.066894in}{1.832485in}}{\pgfqpoint{1.063622in}{1.824585in}}{\pgfqpoint{1.063622in}{1.816349in}}%
\pgfpathcurveto{\pgfqpoint{1.063622in}{1.808112in}}{\pgfqpoint{1.066894in}{1.800212in}}{\pgfqpoint{1.072718in}{1.794388in}}%
\pgfpathcurveto{\pgfqpoint{1.078542in}{1.788565in}}{\pgfqpoint{1.086442in}{1.785292in}}{\pgfqpoint{1.094678in}{1.785292in}}%
\pgfpathclose%
\pgfusepath{stroke,fill}%
\end{pgfscope}%
\begin{pgfscope}%
\pgfpathrectangle{\pgfqpoint{0.100000in}{0.220728in}}{\pgfqpoint{3.696000in}{3.696000in}}%
\pgfusepath{clip}%
\pgfsetbuttcap%
\pgfsetroundjoin%
\definecolor{currentfill}{rgb}{0.121569,0.466667,0.705882}%
\pgfsetfillcolor{currentfill}%
\pgfsetfillopacity{0.530918}%
\pgfsetlinewidth{1.003750pt}%
\definecolor{currentstroke}{rgb}{0.121569,0.466667,0.705882}%
\pgfsetstrokecolor{currentstroke}%
\pgfsetstrokeopacity{0.530918}%
\pgfsetdash{}{0pt}%
\pgfpathmoveto{\pgfqpoint{2.748643in}{3.053477in}}%
\pgfpathcurveto{\pgfqpoint{2.756880in}{3.053477in}}{\pgfqpoint{2.764780in}{3.056750in}}{\pgfqpoint{2.770604in}{3.062574in}}%
\pgfpathcurveto{\pgfqpoint{2.776428in}{3.068397in}}{\pgfqpoint{2.779700in}{3.076298in}}{\pgfqpoint{2.779700in}{3.084534in}}%
\pgfpathcurveto{\pgfqpoint{2.779700in}{3.092770in}}{\pgfqpoint{2.776428in}{3.100670in}}{\pgfqpoint{2.770604in}{3.106494in}}%
\pgfpathcurveto{\pgfqpoint{2.764780in}{3.112318in}}{\pgfqpoint{2.756880in}{3.115590in}}{\pgfqpoint{2.748643in}{3.115590in}}%
\pgfpathcurveto{\pgfqpoint{2.740407in}{3.115590in}}{\pgfqpoint{2.732507in}{3.112318in}}{\pgfqpoint{2.726683in}{3.106494in}}%
\pgfpathcurveto{\pgfqpoint{2.720859in}{3.100670in}}{\pgfqpoint{2.717587in}{3.092770in}}{\pgfqpoint{2.717587in}{3.084534in}}%
\pgfpathcurveto{\pgfqpoint{2.717587in}{3.076298in}}{\pgfqpoint{2.720859in}{3.068397in}}{\pgfqpoint{2.726683in}{3.062574in}}%
\pgfpathcurveto{\pgfqpoint{2.732507in}{3.056750in}}{\pgfqpoint{2.740407in}{3.053477in}}{\pgfqpoint{2.748643in}{3.053477in}}%
\pgfpathclose%
\pgfusepath{stroke,fill}%
\end{pgfscope}%
\begin{pgfscope}%
\pgfpathrectangle{\pgfqpoint{0.100000in}{0.220728in}}{\pgfqpoint{3.696000in}{3.696000in}}%
\pgfusepath{clip}%
\pgfsetbuttcap%
\pgfsetroundjoin%
\definecolor{currentfill}{rgb}{0.121569,0.466667,0.705882}%
\pgfsetfillcolor{currentfill}%
\pgfsetfillopacity{0.531269}%
\pgfsetlinewidth{1.003750pt}%
\definecolor{currentstroke}{rgb}{0.121569,0.466667,0.705882}%
\pgfsetstrokecolor{currentstroke}%
\pgfsetstrokeopacity{0.531269}%
\pgfsetdash{}{0pt}%
\pgfpathmoveto{\pgfqpoint{1.084636in}{1.771801in}}%
\pgfpathcurveto{\pgfqpoint{1.092872in}{1.771801in}}{\pgfqpoint{1.100772in}{1.775073in}}{\pgfqpoint{1.106596in}{1.780897in}}%
\pgfpathcurveto{\pgfqpoint{1.112420in}{1.786721in}}{\pgfqpoint{1.115692in}{1.794621in}}{\pgfqpoint{1.115692in}{1.802857in}}%
\pgfpathcurveto{\pgfqpoint{1.115692in}{1.811093in}}{\pgfqpoint{1.112420in}{1.818994in}}{\pgfqpoint{1.106596in}{1.824817in}}%
\pgfpathcurveto{\pgfqpoint{1.100772in}{1.830641in}}{\pgfqpoint{1.092872in}{1.833914in}}{\pgfqpoint{1.084636in}{1.833914in}}%
\pgfpathcurveto{\pgfqpoint{1.076399in}{1.833914in}}{\pgfqpoint{1.068499in}{1.830641in}}{\pgfqpoint{1.062675in}{1.824817in}}%
\pgfpathcurveto{\pgfqpoint{1.056851in}{1.818994in}}{\pgfqpoint{1.053579in}{1.811093in}}{\pgfqpoint{1.053579in}{1.802857in}}%
\pgfpathcurveto{\pgfqpoint{1.053579in}{1.794621in}}{\pgfqpoint{1.056851in}{1.786721in}}{\pgfqpoint{1.062675in}{1.780897in}}%
\pgfpathcurveto{\pgfqpoint{1.068499in}{1.775073in}}{\pgfqpoint{1.076399in}{1.771801in}}{\pgfqpoint{1.084636in}{1.771801in}}%
\pgfpathclose%
\pgfusepath{stroke,fill}%
\end{pgfscope}%
\begin{pgfscope}%
\pgfpathrectangle{\pgfqpoint{0.100000in}{0.220728in}}{\pgfqpoint{3.696000in}{3.696000in}}%
\pgfusepath{clip}%
\pgfsetbuttcap%
\pgfsetroundjoin%
\definecolor{currentfill}{rgb}{0.121569,0.466667,0.705882}%
\pgfsetfillcolor{currentfill}%
\pgfsetfillopacity{0.533379}%
\pgfsetlinewidth{1.003750pt}%
\definecolor{currentstroke}{rgb}{0.121569,0.466667,0.705882}%
\pgfsetstrokecolor{currentstroke}%
\pgfsetstrokeopacity{0.533379}%
\pgfsetdash{}{0pt}%
\pgfpathmoveto{\pgfqpoint{2.757479in}{3.050269in}}%
\pgfpathcurveto{\pgfqpoint{2.765715in}{3.050269in}}{\pgfqpoint{2.773615in}{3.053542in}}{\pgfqpoint{2.779439in}{3.059366in}}%
\pgfpathcurveto{\pgfqpoint{2.785263in}{3.065189in}}{\pgfqpoint{2.788535in}{3.073090in}}{\pgfqpoint{2.788535in}{3.081326in}}%
\pgfpathcurveto{\pgfqpoint{2.788535in}{3.089562in}}{\pgfqpoint{2.785263in}{3.097462in}}{\pgfqpoint{2.779439in}{3.103286in}}%
\pgfpathcurveto{\pgfqpoint{2.773615in}{3.109110in}}{\pgfqpoint{2.765715in}{3.112382in}}{\pgfqpoint{2.757479in}{3.112382in}}%
\pgfpathcurveto{\pgfqpoint{2.749242in}{3.112382in}}{\pgfqpoint{2.741342in}{3.109110in}}{\pgfqpoint{2.735518in}{3.103286in}}%
\pgfpathcurveto{\pgfqpoint{2.729695in}{3.097462in}}{\pgfqpoint{2.726422in}{3.089562in}}{\pgfqpoint{2.726422in}{3.081326in}}%
\pgfpathcurveto{\pgfqpoint{2.726422in}{3.073090in}}{\pgfqpoint{2.729695in}{3.065189in}}{\pgfqpoint{2.735518in}{3.059366in}}%
\pgfpathcurveto{\pgfqpoint{2.741342in}{3.053542in}}{\pgfqpoint{2.749242in}{3.050269in}}{\pgfqpoint{2.757479in}{3.050269in}}%
\pgfpathclose%
\pgfusepath{stroke,fill}%
\end{pgfscope}%
\begin{pgfscope}%
\pgfpathrectangle{\pgfqpoint{0.100000in}{0.220728in}}{\pgfqpoint{3.696000in}{3.696000in}}%
\pgfusepath{clip}%
\pgfsetbuttcap%
\pgfsetroundjoin%
\definecolor{currentfill}{rgb}{0.121569,0.466667,0.705882}%
\pgfsetfillcolor{currentfill}%
\pgfsetfillopacity{0.533594}%
\pgfsetlinewidth{1.003750pt}%
\definecolor{currentstroke}{rgb}{0.121569,0.466667,0.705882}%
\pgfsetstrokecolor{currentstroke}%
\pgfsetstrokeopacity{0.533594}%
\pgfsetdash{}{0pt}%
\pgfpathmoveto{\pgfqpoint{1.082354in}{1.755697in}}%
\pgfpathcurveto{\pgfqpoint{1.090590in}{1.755697in}}{\pgfqpoint{1.098490in}{1.758969in}}{\pgfqpoint{1.104314in}{1.764793in}}%
\pgfpathcurveto{\pgfqpoint{1.110138in}{1.770617in}}{\pgfqpoint{1.113411in}{1.778517in}}{\pgfqpoint{1.113411in}{1.786753in}}%
\pgfpathcurveto{\pgfqpoint{1.113411in}{1.794990in}}{\pgfqpoint{1.110138in}{1.802890in}}{\pgfqpoint{1.104314in}{1.808714in}}%
\pgfpathcurveto{\pgfqpoint{1.098490in}{1.814538in}}{\pgfqpoint{1.090590in}{1.817810in}}{\pgfqpoint{1.082354in}{1.817810in}}%
\pgfpathcurveto{\pgfqpoint{1.074118in}{1.817810in}}{\pgfqpoint{1.066218in}{1.814538in}}{\pgfqpoint{1.060394in}{1.808714in}}%
\pgfpathcurveto{\pgfqpoint{1.054570in}{1.802890in}}{\pgfqpoint{1.051298in}{1.794990in}}{\pgfqpoint{1.051298in}{1.786753in}}%
\pgfpathcurveto{\pgfqpoint{1.051298in}{1.778517in}}{\pgfqpoint{1.054570in}{1.770617in}}{\pgfqpoint{1.060394in}{1.764793in}}%
\pgfpathcurveto{\pgfqpoint{1.066218in}{1.758969in}}{\pgfqpoint{1.074118in}{1.755697in}}{\pgfqpoint{1.082354in}{1.755697in}}%
\pgfpathclose%
\pgfusepath{stroke,fill}%
\end{pgfscope}%
\begin{pgfscope}%
\pgfpathrectangle{\pgfqpoint{0.100000in}{0.220728in}}{\pgfqpoint{3.696000in}{3.696000in}}%
\pgfusepath{clip}%
\pgfsetbuttcap%
\pgfsetroundjoin%
\definecolor{currentfill}{rgb}{0.121569,0.466667,0.705882}%
\pgfsetfillcolor{currentfill}%
\pgfsetfillopacity{0.534134}%
\pgfsetlinewidth{1.003750pt}%
\definecolor{currentstroke}{rgb}{0.121569,0.466667,0.705882}%
\pgfsetstrokecolor{currentstroke}%
\pgfsetstrokeopacity{0.534134}%
\pgfsetdash{}{0pt}%
\pgfpathmoveto{\pgfqpoint{1.074721in}{1.748034in}}%
\pgfpathcurveto{\pgfqpoint{1.082958in}{1.748034in}}{\pgfqpoint{1.090858in}{1.751306in}}{\pgfqpoint{1.096681in}{1.757130in}}%
\pgfpathcurveto{\pgfqpoint{1.102505in}{1.762954in}}{\pgfqpoint{1.105778in}{1.770854in}}{\pgfqpoint{1.105778in}{1.779091in}}%
\pgfpathcurveto{\pgfqpoint{1.105778in}{1.787327in}}{\pgfqpoint{1.102505in}{1.795227in}}{\pgfqpoint{1.096681in}{1.801051in}}%
\pgfpathcurveto{\pgfqpoint{1.090858in}{1.806875in}}{\pgfqpoint{1.082958in}{1.810147in}}{\pgfqpoint{1.074721in}{1.810147in}}%
\pgfpathcurveto{\pgfqpoint{1.066485in}{1.810147in}}{\pgfqpoint{1.058585in}{1.806875in}}{\pgfqpoint{1.052761in}{1.801051in}}%
\pgfpathcurveto{\pgfqpoint{1.046937in}{1.795227in}}{\pgfqpoint{1.043665in}{1.787327in}}{\pgfqpoint{1.043665in}{1.779091in}}%
\pgfpathcurveto{\pgfqpoint{1.043665in}{1.770854in}}{\pgfqpoint{1.046937in}{1.762954in}}{\pgfqpoint{1.052761in}{1.757130in}}%
\pgfpathcurveto{\pgfqpoint{1.058585in}{1.751306in}}{\pgfqpoint{1.066485in}{1.748034in}}{\pgfqpoint{1.074721in}{1.748034in}}%
\pgfpathclose%
\pgfusepath{stroke,fill}%
\end{pgfscope}%
\begin{pgfscope}%
\pgfpathrectangle{\pgfqpoint{0.100000in}{0.220728in}}{\pgfqpoint{3.696000in}{3.696000in}}%
\pgfusepath{clip}%
\pgfsetbuttcap%
\pgfsetroundjoin%
\definecolor{currentfill}{rgb}{0.121569,0.466667,0.705882}%
\pgfsetfillcolor{currentfill}%
\pgfsetfillopacity{0.534671}%
\pgfsetlinewidth{1.003750pt}%
\definecolor{currentstroke}{rgb}{0.121569,0.466667,0.705882}%
\pgfsetstrokecolor{currentstroke}%
\pgfsetstrokeopacity{0.534671}%
\pgfsetdash{}{0pt}%
\pgfpathmoveto{\pgfqpoint{2.751697in}{3.050836in}}%
\pgfpathcurveto{\pgfqpoint{2.759933in}{3.050836in}}{\pgfqpoint{2.767833in}{3.054109in}}{\pgfqpoint{2.773657in}{3.059933in}}%
\pgfpathcurveto{\pgfqpoint{2.779481in}{3.065757in}}{\pgfqpoint{2.782753in}{3.073657in}}{\pgfqpoint{2.782753in}{3.081893in}}%
\pgfpathcurveto{\pgfqpoint{2.782753in}{3.090129in}}{\pgfqpoint{2.779481in}{3.098029in}}{\pgfqpoint{2.773657in}{3.103853in}}%
\pgfpathcurveto{\pgfqpoint{2.767833in}{3.109677in}}{\pgfqpoint{2.759933in}{3.112949in}}{\pgfqpoint{2.751697in}{3.112949in}}%
\pgfpathcurveto{\pgfqpoint{2.743461in}{3.112949in}}{\pgfqpoint{2.735561in}{3.109677in}}{\pgfqpoint{2.729737in}{3.103853in}}%
\pgfpathcurveto{\pgfqpoint{2.723913in}{3.098029in}}{\pgfqpoint{2.720640in}{3.090129in}}{\pgfqpoint{2.720640in}{3.081893in}}%
\pgfpathcurveto{\pgfqpoint{2.720640in}{3.073657in}}{\pgfqpoint{2.723913in}{3.065757in}}{\pgfqpoint{2.729737in}{3.059933in}}%
\pgfpathcurveto{\pgfqpoint{2.735561in}{3.054109in}}{\pgfqpoint{2.743461in}{3.050836in}}{\pgfqpoint{2.751697in}{3.050836in}}%
\pgfpathclose%
\pgfusepath{stroke,fill}%
\end{pgfscope}%
\begin{pgfscope}%
\pgfpathrectangle{\pgfqpoint{0.100000in}{0.220728in}}{\pgfqpoint{3.696000in}{3.696000in}}%
\pgfusepath{clip}%
\pgfsetbuttcap%
\pgfsetroundjoin%
\definecolor{currentfill}{rgb}{0.121569,0.466667,0.705882}%
\pgfsetfillcolor{currentfill}%
\pgfsetfillopacity{0.535372}%
\pgfsetlinewidth{1.003750pt}%
\definecolor{currentstroke}{rgb}{0.121569,0.466667,0.705882}%
\pgfsetstrokecolor{currentstroke}%
\pgfsetstrokeopacity{0.535372}%
\pgfsetdash{}{0pt}%
\pgfpathmoveto{\pgfqpoint{1.072904in}{1.739418in}}%
\pgfpathcurveto{\pgfqpoint{1.081140in}{1.739418in}}{\pgfqpoint{1.089040in}{1.742690in}}{\pgfqpoint{1.094864in}{1.748514in}}%
\pgfpathcurveto{\pgfqpoint{1.100688in}{1.754338in}}{\pgfqpoint{1.103960in}{1.762238in}}{\pgfqpoint{1.103960in}{1.770474in}}%
\pgfpathcurveto{\pgfqpoint{1.103960in}{1.778711in}}{\pgfqpoint{1.100688in}{1.786611in}}{\pgfqpoint{1.094864in}{1.792435in}}%
\pgfpathcurveto{\pgfqpoint{1.089040in}{1.798258in}}{\pgfqpoint{1.081140in}{1.801531in}}{\pgfqpoint{1.072904in}{1.801531in}}%
\pgfpathcurveto{\pgfqpoint{1.064667in}{1.801531in}}{\pgfqpoint{1.056767in}{1.798258in}}{\pgfqpoint{1.050943in}{1.792435in}}%
\pgfpathcurveto{\pgfqpoint{1.045119in}{1.786611in}}{\pgfqpoint{1.041847in}{1.778711in}}{\pgfqpoint{1.041847in}{1.770474in}}%
\pgfpathcurveto{\pgfqpoint{1.041847in}{1.762238in}}{\pgfqpoint{1.045119in}{1.754338in}}{\pgfqpoint{1.050943in}{1.748514in}}%
\pgfpathcurveto{\pgfqpoint{1.056767in}{1.742690in}}{\pgfqpoint{1.064667in}{1.739418in}}{\pgfqpoint{1.072904in}{1.739418in}}%
\pgfpathclose%
\pgfusepath{stroke,fill}%
\end{pgfscope}%
\begin{pgfscope}%
\pgfpathrectangle{\pgfqpoint{0.100000in}{0.220728in}}{\pgfqpoint{3.696000in}{3.696000in}}%
\pgfusepath{clip}%
\pgfsetbuttcap%
\pgfsetroundjoin%
\definecolor{currentfill}{rgb}{0.121569,0.466667,0.705882}%
\pgfsetfillcolor{currentfill}%
\pgfsetfillopacity{0.535722}%
\pgfsetlinewidth{1.003750pt}%
\definecolor{currentstroke}{rgb}{0.121569,0.466667,0.705882}%
\pgfsetstrokecolor{currentstroke}%
\pgfsetstrokeopacity{0.535722}%
\pgfsetdash{}{0pt}%
\pgfpathmoveto{\pgfqpoint{1.070057in}{1.736123in}}%
\pgfpathcurveto{\pgfqpoint{1.078293in}{1.736123in}}{\pgfqpoint{1.086193in}{1.739396in}}{\pgfqpoint{1.092017in}{1.745220in}}%
\pgfpathcurveto{\pgfqpoint{1.097841in}{1.751043in}}{\pgfqpoint{1.101113in}{1.758943in}}{\pgfqpoint{1.101113in}{1.767180in}}%
\pgfpathcurveto{\pgfqpoint{1.101113in}{1.775416in}}{\pgfqpoint{1.097841in}{1.783316in}}{\pgfqpoint{1.092017in}{1.789140in}}%
\pgfpathcurveto{\pgfqpoint{1.086193in}{1.794964in}}{\pgfqpoint{1.078293in}{1.798236in}}{\pgfqpoint{1.070057in}{1.798236in}}%
\pgfpathcurveto{\pgfqpoint{1.061820in}{1.798236in}}{\pgfqpoint{1.053920in}{1.794964in}}{\pgfqpoint{1.048096in}{1.789140in}}%
\pgfpathcurveto{\pgfqpoint{1.042272in}{1.783316in}}{\pgfqpoint{1.039000in}{1.775416in}}{\pgfqpoint{1.039000in}{1.767180in}}%
\pgfpathcurveto{\pgfqpoint{1.039000in}{1.758943in}}{\pgfqpoint{1.042272in}{1.751043in}}{\pgfqpoint{1.048096in}{1.745220in}}%
\pgfpathcurveto{\pgfqpoint{1.053920in}{1.739396in}}{\pgfqpoint{1.061820in}{1.736123in}}{\pgfqpoint{1.070057in}{1.736123in}}%
\pgfpathclose%
\pgfusepath{stroke,fill}%
\end{pgfscope}%
\begin{pgfscope}%
\pgfpathrectangle{\pgfqpoint{0.100000in}{0.220728in}}{\pgfqpoint{3.696000in}{3.696000in}}%
\pgfusepath{clip}%
\pgfsetbuttcap%
\pgfsetroundjoin%
\definecolor{currentfill}{rgb}{0.121569,0.466667,0.705882}%
\pgfsetfillcolor{currentfill}%
\pgfsetfillopacity{0.536749}%
\pgfsetlinewidth{1.003750pt}%
\definecolor{currentstroke}{rgb}{0.121569,0.466667,0.705882}%
\pgfsetstrokecolor{currentstroke}%
\pgfsetstrokeopacity{0.536749}%
\pgfsetdash{}{0pt}%
\pgfpathmoveto{\pgfqpoint{2.758148in}{3.050563in}}%
\pgfpathcurveto{\pgfqpoint{2.766385in}{3.050563in}}{\pgfqpoint{2.774285in}{3.053835in}}{\pgfqpoint{2.780109in}{3.059659in}}%
\pgfpathcurveto{\pgfqpoint{2.785933in}{3.065483in}}{\pgfqpoint{2.789205in}{3.073383in}}{\pgfqpoint{2.789205in}{3.081619in}}%
\pgfpathcurveto{\pgfqpoint{2.789205in}{3.089856in}}{\pgfqpoint{2.785933in}{3.097756in}}{\pgfqpoint{2.780109in}{3.103580in}}%
\pgfpathcurveto{\pgfqpoint{2.774285in}{3.109404in}}{\pgfqpoint{2.766385in}{3.112676in}}{\pgfqpoint{2.758148in}{3.112676in}}%
\pgfpathcurveto{\pgfqpoint{2.749912in}{3.112676in}}{\pgfqpoint{2.742012in}{3.109404in}}{\pgfqpoint{2.736188in}{3.103580in}}%
\pgfpathcurveto{\pgfqpoint{2.730364in}{3.097756in}}{\pgfqpoint{2.727092in}{3.089856in}}{\pgfqpoint{2.727092in}{3.081619in}}%
\pgfpathcurveto{\pgfqpoint{2.727092in}{3.073383in}}{\pgfqpoint{2.730364in}{3.065483in}}{\pgfqpoint{2.736188in}{3.059659in}}%
\pgfpathcurveto{\pgfqpoint{2.742012in}{3.053835in}}{\pgfqpoint{2.749912in}{3.050563in}}{\pgfqpoint{2.758148in}{3.050563in}}%
\pgfpathclose%
\pgfusepath{stroke,fill}%
\end{pgfscope}%
\begin{pgfscope}%
\pgfpathrectangle{\pgfqpoint{0.100000in}{0.220728in}}{\pgfqpoint{3.696000in}{3.696000in}}%
\pgfusepath{clip}%
\pgfsetbuttcap%
\pgfsetroundjoin%
\definecolor{currentfill}{rgb}{0.121569,0.466667,0.705882}%
\pgfsetfillcolor{currentfill}%
\pgfsetfillopacity{0.536962}%
\pgfsetlinewidth{1.003750pt}%
\definecolor{currentstroke}{rgb}{0.121569,0.466667,0.705882}%
\pgfsetstrokecolor{currentstroke}%
\pgfsetstrokeopacity{0.536962}%
\pgfsetdash{}{0pt}%
\pgfpathmoveto{\pgfqpoint{1.067092in}{1.728969in}}%
\pgfpathcurveto{\pgfqpoint{1.075328in}{1.728969in}}{\pgfqpoint{1.083228in}{1.732241in}}{\pgfqpoint{1.089052in}{1.738065in}}%
\pgfpathcurveto{\pgfqpoint{1.094876in}{1.743889in}}{\pgfqpoint{1.098149in}{1.751789in}}{\pgfqpoint{1.098149in}{1.760026in}}%
\pgfpathcurveto{\pgfqpoint{1.098149in}{1.768262in}}{\pgfqpoint{1.094876in}{1.776162in}}{\pgfqpoint{1.089052in}{1.781986in}}%
\pgfpathcurveto{\pgfqpoint{1.083228in}{1.787810in}}{\pgfqpoint{1.075328in}{1.791082in}}{\pgfqpoint{1.067092in}{1.791082in}}%
\pgfpathcurveto{\pgfqpoint{1.058856in}{1.791082in}}{\pgfqpoint{1.050956in}{1.787810in}}{\pgfqpoint{1.045132in}{1.781986in}}%
\pgfpathcurveto{\pgfqpoint{1.039308in}{1.776162in}}{\pgfqpoint{1.036036in}{1.768262in}}{\pgfqpoint{1.036036in}{1.760026in}}%
\pgfpathcurveto{\pgfqpoint{1.036036in}{1.751789in}}{\pgfqpoint{1.039308in}{1.743889in}}{\pgfqpoint{1.045132in}{1.738065in}}%
\pgfpathcurveto{\pgfqpoint{1.050956in}{1.732241in}}{\pgfqpoint{1.058856in}{1.728969in}}{\pgfqpoint{1.067092in}{1.728969in}}%
\pgfpathclose%
\pgfusepath{stroke,fill}%
\end{pgfscope}%
\begin{pgfscope}%
\pgfpathrectangle{\pgfqpoint{0.100000in}{0.220728in}}{\pgfqpoint{3.696000in}{3.696000in}}%
\pgfusepath{clip}%
\pgfsetbuttcap%
\pgfsetroundjoin%
\definecolor{currentfill}{rgb}{0.121569,0.466667,0.705882}%
\pgfsetfillcolor{currentfill}%
\pgfsetfillopacity{0.537870}%
\pgfsetlinewidth{1.003750pt}%
\definecolor{currentstroke}{rgb}{0.121569,0.466667,0.705882}%
\pgfsetstrokecolor{currentstroke}%
\pgfsetstrokeopacity{0.537870}%
\pgfsetdash{}{0pt}%
\pgfpathmoveto{\pgfqpoint{1.063071in}{1.723592in}}%
\pgfpathcurveto{\pgfqpoint{1.071307in}{1.723592in}}{\pgfqpoint{1.079207in}{1.726865in}}{\pgfqpoint{1.085031in}{1.732688in}}%
\pgfpathcurveto{\pgfqpoint{1.090855in}{1.738512in}}{\pgfqpoint{1.094127in}{1.746412in}}{\pgfqpoint{1.094127in}{1.754649in}}%
\pgfpathcurveto{\pgfqpoint{1.094127in}{1.762885in}}{\pgfqpoint{1.090855in}{1.770785in}}{\pgfqpoint{1.085031in}{1.776609in}}%
\pgfpathcurveto{\pgfqpoint{1.079207in}{1.782433in}}{\pgfqpoint{1.071307in}{1.785705in}}{\pgfqpoint{1.063071in}{1.785705in}}%
\pgfpathcurveto{\pgfqpoint{1.054834in}{1.785705in}}{\pgfqpoint{1.046934in}{1.782433in}}{\pgfqpoint{1.041110in}{1.776609in}}%
\pgfpathcurveto{\pgfqpoint{1.035286in}{1.770785in}}{\pgfqpoint{1.032014in}{1.762885in}}{\pgfqpoint{1.032014in}{1.754649in}}%
\pgfpathcurveto{\pgfqpoint{1.032014in}{1.746412in}}{\pgfqpoint{1.035286in}{1.738512in}}{\pgfqpoint{1.041110in}{1.732688in}}%
\pgfpathcurveto{\pgfqpoint{1.046934in}{1.726865in}}{\pgfqpoint{1.054834in}{1.723592in}}{\pgfqpoint{1.063071in}{1.723592in}}%
\pgfpathclose%
\pgfusepath{stroke,fill}%
\end{pgfscope}%
\begin{pgfscope}%
\pgfpathrectangle{\pgfqpoint{0.100000in}{0.220728in}}{\pgfqpoint{3.696000in}{3.696000in}}%
\pgfusepath{clip}%
\pgfsetbuttcap%
\pgfsetroundjoin%
\definecolor{currentfill}{rgb}{0.121569,0.466667,0.705882}%
\pgfsetfillcolor{currentfill}%
\pgfsetfillopacity{0.538651}%
\pgfsetlinewidth{1.003750pt}%
\definecolor{currentstroke}{rgb}{0.121569,0.466667,0.705882}%
\pgfsetstrokecolor{currentstroke}%
\pgfsetstrokeopacity{0.538651}%
\pgfsetdash{}{0pt}%
\pgfpathmoveto{\pgfqpoint{1.060308in}{1.718572in}}%
\pgfpathcurveto{\pgfqpoint{1.068545in}{1.718572in}}{\pgfqpoint{1.076445in}{1.721844in}}{\pgfqpoint{1.082269in}{1.727668in}}%
\pgfpathcurveto{\pgfqpoint{1.088093in}{1.733492in}}{\pgfqpoint{1.091365in}{1.741392in}}{\pgfqpoint{1.091365in}{1.749628in}}%
\pgfpathcurveto{\pgfqpoint{1.091365in}{1.757864in}}{\pgfqpoint{1.088093in}{1.765764in}}{\pgfqpoint{1.082269in}{1.771588in}}%
\pgfpathcurveto{\pgfqpoint{1.076445in}{1.777412in}}{\pgfqpoint{1.068545in}{1.780685in}}{\pgfqpoint{1.060308in}{1.780685in}}%
\pgfpathcurveto{\pgfqpoint{1.052072in}{1.780685in}}{\pgfqpoint{1.044172in}{1.777412in}}{\pgfqpoint{1.038348in}{1.771588in}}%
\pgfpathcurveto{\pgfqpoint{1.032524in}{1.765764in}}{\pgfqpoint{1.029252in}{1.757864in}}{\pgfqpoint{1.029252in}{1.749628in}}%
\pgfpathcurveto{\pgfqpoint{1.029252in}{1.741392in}}{\pgfqpoint{1.032524in}{1.733492in}}{\pgfqpoint{1.038348in}{1.727668in}}%
\pgfpathcurveto{\pgfqpoint{1.044172in}{1.721844in}}{\pgfqpoint{1.052072in}{1.718572in}}{\pgfqpoint{1.060308in}{1.718572in}}%
\pgfpathclose%
\pgfusepath{stroke,fill}%
\end{pgfscope}%
\begin{pgfscope}%
\pgfpathrectangle{\pgfqpoint{0.100000in}{0.220728in}}{\pgfqpoint{3.696000in}{3.696000in}}%
\pgfusepath{clip}%
\pgfsetbuttcap%
\pgfsetroundjoin%
\definecolor{currentfill}{rgb}{0.121569,0.466667,0.705882}%
\pgfsetfillcolor{currentfill}%
\pgfsetfillopacity{0.538948}%
\pgfsetlinewidth{1.003750pt}%
\definecolor{currentstroke}{rgb}{0.121569,0.466667,0.705882}%
\pgfsetstrokecolor{currentstroke}%
\pgfsetstrokeopacity{0.538948}%
\pgfsetdash{}{0pt}%
\pgfpathmoveto{\pgfqpoint{2.766932in}{3.048495in}}%
\pgfpathcurveto{\pgfqpoint{2.775168in}{3.048495in}}{\pgfqpoint{2.783068in}{3.051767in}}{\pgfqpoint{2.788892in}{3.057591in}}%
\pgfpathcurveto{\pgfqpoint{2.794716in}{3.063415in}}{\pgfqpoint{2.797988in}{3.071315in}}{\pgfqpoint{2.797988in}{3.079551in}}%
\pgfpathcurveto{\pgfqpoint{2.797988in}{3.087788in}}{\pgfqpoint{2.794716in}{3.095688in}}{\pgfqpoint{2.788892in}{3.101512in}}%
\pgfpathcurveto{\pgfqpoint{2.783068in}{3.107335in}}{\pgfqpoint{2.775168in}{3.110608in}}{\pgfqpoint{2.766932in}{3.110608in}}%
\pgfpathcurveto{\pgfqpoint{2.758696in}{3.110608in}}{\pgfqpoint{2.750795in}{3.107335in}}{\pgfqpoint{2.744972in}{3.101512in}}%
\pgfpathcurveto{\pgfqpoint{2.739148in}{3.095688in}}{\pgfqpoint{2.735875in}{3.087788in}}{\pgfqpoint{2.735875in}{3.079551in}}%
\pgfpathcurveto{\pgfqpoint{2.735875in}{3.071315in}}{\pgfqpoint{2.739148in}{3.063415in}}{\pgfqpoint{2.744972in}{3.057591in}}%
\pgfpathcurveto{\pgfqpoint{2.750795in}{3.051767in}}{\pgfqpoint{2.758696in}{3.048495in}}{\pgfqpoint{2.766932in}{3.048495in}}%
\pgfpathclose%
\pgfusepath{stroke,fill}%
\end{pgfscope}%
\begin{pgfscope}%
\pgfpathrectangle{\pgfqpoint{0.100000in}{0.220728in}}{\pgfqpoint{3.696000in}{3.696000in}}%
\pgfusepath{clip}%
\pgfsetbuttcap%
\pgfsetroundjoin%
\definecolor{currentfill}{rgb}{0.121569,0.466667,0.705882}%
\pgfsetfillcolor{currentfill}%
\pgfsetfillopacity{0.540255}%
\pgfsetlinewidth{1.003750pt}%
\definecolor{currentstroke}{rgb}{0.121569,0.466667,0.705882}%
\pgfsetstrokecolor{currentstroke}%
\pgfsetstrokeopacity{0.540255}%
\pgfsetdash{}{0pt}%
\pgfpathmoveto{\pgfqpoint{1.055997in}{1.709345in}}%
\pgfpathcurveto{\pgfqpoint{1.064233in}{1.709345in}}{\pgfqpoint{1.072133in}{1.712618in}}{\pgfqpoint{1.077957in}{1.718441in}}%
\pgfpathcurveto{\pgfqpoint{1.083781in}{1.724265in}}{\pgfqpoint{1.087054in}{1.732165in}}{\pgfqpoint{1.087054in}{1.740402in}}%
\pgfpathcurveto{\pgfqpoint{1.087054in}{1.748638in}}{\pgfqpoint{1.083781in}{1.756538in}}{\pgfqpoint{1.077957in}{1.762362in}}%
\pgfpathcurveto{\pgfqpoint{1.072133in}{1.768186in}}{\pgfqpoint{1.064233in}{1.771458in}}{\pgfqpoint{1.055997in}{1.771458in}}%
\pgfpathcurveto{\pgfqpoint{1.047761in}{1.771458in}}{\pgfqpoint{1.039861in}{1.768186in}}{\pgfqpoint{1.034037in}{1.762362in}}%
\pgfpathcurveto{\pgfqpoint{1.028213in}{1.756538in}}{\pgfqpoint{1.024941in}{1.748638in}}{\pgfqpoint{1.024941in}{1.740402in}}%
\pgfpathcurveto{\pgfqpoint{1.024941in}{1.732165in}}{\pgfqpoint{1.028213in}{1.724265in}}{\pgfqpoint{1.034037in}{1.718441in}}%
\pgfpathcurveto{\pgfqpoint{1.039861in}{1.712618in}}{\pgfqpoint{1.047761in}{1.709345in}}{\pgfqpoint{1.055997in}{1.709345in}}%
\pgfpathclose%
\pgfusepath{stroke,fill}%
\end{pgfscope}%
\begin{pgfscope}%
\pgfpathrectangle{\pgfqpoint{0.100000in}{0.220728in}}{\pgfqpoint{3.696000in}{3.696000in}}%
\pgfusepath{clip}%
\pgfsetbuttcap%
\pgfsetroundjoin%
\definecolor{currentfill}{rgb}{0.121569,0.466667,0.705882}%
\pgfsetfillcolor{currentfill}%
\pgfsetfillopacity{0.541501}%
\pgfsetlinewidth{1.003750pt}%
\definecolor{currentstroke}{rgb}{0.121569,0.466667,0.705882}%
\pgfsetstrokecolor{currentstroke}%
\pgfsetstrokeopacity{0.541501}%
\pgfsetdash{}{0pt}%
\pgfpathmoveto{\pgfqpoint{2.776929in}{3.046282in}}%
\pgfpathcurveto{\pgfqpoint{2.785165in}{3.046282in}}{\pgfqpoint{2.793065in}{3.049554in}}{\pgfqpoint{2.798889in}{3.055378in}}%
\pgfpathcurveto{\pgfqpoint{2.804713in}{3.061202in}}{\pgfqpoint{2.807986in}{3.069102in}}{\pgfqpoint{2.807986in}{3.077339in}}%
\pgfpathcurveto{\pgfqpoint{2.807986in}{3.085575in}}{\pgfqpoint{2.804713in}{3.093475in}}{\pgfqpoint{2.798889in}{3.099299in}}%
\pgfpathcurveto{\pgfqpoint{2.793065in}{3.105123in}}{\pgfqpoint{2.785165in}{3.108395in}}{\pgfqpoint{2.776929in}{3.108395in}}%
\pgfpathcurveto{\pgfqpoint{2.768693in}{3.108395in}}{\pgfqpoint{2.760793in}{3.105123in}}{\pgfqpoint{2.754969in}{3.099299in}}%
\pgfpathcurveto{\pgfqpoint{2.749145in}{3.093475in}}{\pgfqpoint{2.745873in}{3.085575in}}{\pgfqpoint{2.745873in}{3.077339in}}%
\pgfpathcurveto{\pgfqpoint{2.745873in}{3.069102in}}{\pgfqpoint{2.749145in}{3.061202in}}{\pgfqpoint{2.754969in}{3.055378in}}%
\pgfpathcurveto{\pgfqpoint{2.760793in}{3.049554in}}{\pgfqpoint{2.768693in}{3.046282in}}{\pgfqpoint{2.776929in}{3.046282in}}%
\pgfpathclose%
\pgfusepath{stroke,fill}%
\end{pgfscope}%
\begin{pgfscope}%
\pgfpathrectangle{\pgfqpoint{0.100000in}{0.220728in}}{\pgfqpoint{3.696000in}{3.696000in}}%
\pgfusepath{clip}%
\pgfsetbuttcap%
\pgfsetroundjoin%
\definecolor{currentfill}{rgb}{0.121569,0.466667,0.705882}%
\pgfsetfillcolor{currentfill}%
\pgfsetfillopacity{0.542313}%
\pgfsetlinewidth{1.003750pt}%
\definecolor{currentstroke}{rgb}{0.121569,0.466667,0.705882}%
\pgfsetstrokecolor{currentstroke}%
\pgfsetstrokeopacity{0.542313}%
\pgfsetdash{}{0pt}%
\pgfpathmoveto{\pgfqpoint{2.791540in}{3.042352in}}%
\pgfpathcurveto{\pgfqpoint{2.799776in}{3.042352in}}{\pgfqpoint{2.807676in}{3.045624in}}{\pgfqpoint{2.813500in}{3.051448in}}%
\pgfpathcurveto{\pgfqpoint{2.819324in}{3.057272in}}{\pgfqpoint{2.822596in}{3.065172in}}{\pgfqpoint{2.822596in}{3.073408in}}%
\pgfpathcurveto{\pgfqpoint{2.822596in}{3.081644in}}{\pgfqpoint{2.819324in}{3.089544in}}{\pgfqpoint{2.813500in}{3.095368in}}%
\pgfpathcurveto{\pgfqpoint{2.807676in}{3.101192in}}{\pgfqpoint{2.799776in}{3.104465in}}{\pgfqpoint{2.791540in}{3.104465in}}%
\pgfpathcurveto{\pgfqpoint{2.783303in}{3.104465in}}{\pgfqpoint{2.775403in}{3.101192in}}{\pgfqpoint{2.769579in}{3.095368in}}%
\pgfpathcurveto{\pgfqpoint{2.763755in}{3.089544in}}{\pgfqpoint{2.760483in}{3.081644in}}{\pgfqpoint{2.760483in}{3.073408in}}%
\pgfpathcurveto{\pgfqpoint{2.760483in}{3.065172in}}{\pgfqpoint{2.763755in}{3.057272in}}{\pgfqpoint{2.769579in}{3.051448in}}%
\pgfpathcurveto{\pgfqpoint{2.775403in}{3.045624in}}{\pgfqpoint{2.783303in}{3.042352in}}{\pgfqpoint{2.791540in}{3.042352in}}%
\pgfpathclose%
\pgfusepath{stroke,fill}%
\end{pgfscope}%
\begin{pgfscope}%
\pgfpathrectangle{\pgfqpoint{0.100000in}{0.220728in}}{\pgfqpoint{3.696000in}{3.696000in}}%
\pgfusepath{clip}%
\pgfsetbuttcap%
\pgfsetroundjoin%
\definecolor{currentfill}{rgb}{0.121569,0.466667,0.705882}%
\pgfsetfillcolor{currentfill}%
\pgfsetfillopacity{0.542485}%
\pgfsetlinewidth{1.003750pt}%
\definecolor{currentstroke}{rgb}{0.121569,0.466667,0.705882}%
\pgfsetstrokecolor{currentstroke}%
\pgfsetstrokeopacity{0.542485}%
\pgfsetdash{}{0pt}%
\pgfpathmoveto{\pgfqpoint{2.782866in}{3.044932in}}%
\pgfpathcurveto{\pgfqpoint{2.791102in}{3.044932in}}{\pgfqpoint{2.799002in}{3.048204in}}{\pgfqpoint{2.804826in}{3.054028in}}%
\pgfpathcurveto{\pgfqpoint{2.810650in}{3.059852in}}{\pgfqpoint{2.813922in}{3.067752in}}{\pgfqpoint{2.813922in}{3.075989in}}%
\pgfpathcurveto{\pgfqpoint{2.813922in}{3.084225in}}{\pgfqpoint{2.810650in}{3.092125in}}{\pgfqpoint{2.804826in}{3.097949in}}%
\pgfpathcurveto{\pgfqpoint{2.799002in}{3.103773in}}{\pgfqpoint{2.791102in}{3.107045in}}{\pgfqpoint{2.782866in}{3.107045in}}%
\pgfpathcurveto{\pgfqpoint{2.774630in}{3.107045in}}{\pgfqpoint{2.766730in}{3.103773in}}{\pgfqpoint{2.760906in}{3.097949in}}%
\pgfpathcurveto{\pgfqpoint{2.755082in}{3.092125in}}{\pgfqpoint{2.751809in}{3.084225in}}{\pgfqpoint{2.751809in}{3.075989in}}%
\pgfpathcurveto{\pgfqpoint{2.751809in}{3.067752in}}{\pgfqpoint{2.755082in}{3.059852in}}{\pgfqpoint{2.760906in}{3.054028in}}%
\pgfpathcurveto{\pgfqpoint{2.766730in}{3.048204in}}{\pgfqpoint{2.774630in}{3.044932in}}{\pgfqpoint{2.782866in}{3.044932in}}%
\pgfpathclose%
\pgfusepath{stroke,fill}%
\end{pgfscope}%
\begin{pgfscope}%
\pgfpathrectangle{\pgfqpoint{0.100000in}{0.220728in}}{\pgfqpoint{3.696000in}{3.696000in}}%
\pgfusepath{clip}%
\pgfsetbuttcap%
\pgfsetroundjoin%
\definecolor{currentfill}{rgb}{0.121569,0.466667,0.705882}%
\pgfsetfillcolor{currentfill}%
\pgfsetfillopacity{0.542597}%
\pgfsetlinewidth{1.003750pt}%
\definecolor{currentstroke}{rgb}{0.121569,0.466667,0.705882}%
\pgfsetstrokecolor{currentstroke}%
\pgfsetstrokeopacity{0.542597}%
\pgfsetdash{}{0pt}%
\pgfpathmoveto{\pgfqpoint{1.046422in}{1.692392in}}%
\pgfpathcurveto{\pgfqpoint{1.054658in}{1.692392in}}{\pgfqpoint{1.062559in}{1.695665in}}{\pgfqpoint{1.068382in}{1.701489in}}%
\pgfpathcurveto{\pgfqpoint{1.074206in}{1.707313in}}{\pgfqpoint{1.077479in}{1.715213in}}{\pgfqpoint{1.077479in}{1.723449in}}%
\pgfpathcurveto{\pgfqpoint{1.077479in}{1.731685in}}{\pgfqpoint{1.074206in}{1.739585in}}{\pgfqpoint{1.068382in}{1.745409in}}%
\pgfpathcurveto{\pgfqpoint{1.062559in}{1.751233in}}{\pgfqpoint{1.054658in}{1.754505in}}{\pgfqpoint{1.046422in}{1.754505in}}%
\pgfpathcurveto{\pgfqpoint{1.038186in}{1.754505in}}{\pgfqpoint{1.030286in}{1.751233in}}{\pgfqpoint{1.024462in}{1.745409in}}%
\pgfpathcurveto{\pgfqpoint{1.018638in}{1.739585in}}{\pgfqpoint{1.015366in}{1.731685in}}{\pgfqpoint{1.015366in}{1.723449in}}%
\pgfpathcurveto{\pgfqpoint{1.015366in}{1.715213in}}{\pgfqpoint{1.018638in}{1.707313in}}{\pgfqpoint{1.024462in}{1.701489in}}%
\pgfpathcurveto{\pgfqpoint{1.030286in}{1.695665in}}{\pgfqpoint{1.038186in}{1.692392in}}{\pgfqpoint{1.046422in}{1.692392in}}%
\pgfpathclose%
\pgfusepath{stroke,fill}%
\end{pgfscope}%
\begin{pgfscope}%
\pgfpathrectangle{\pgfqpoint{0.100000in}{0.220728in}}{\pgfqpoint{3.696000in}{3.696000in}}%
\pgfusepath{clip}%
\pgfsetbuttcap%
\pgfsetroundjoin%
\definecolor{currentfill}{rgb}{0.121569,0.466667,0.705882}%
\pgfsetfillcolor{currentfill}%
\pgfsetfillopacity{0.545383}%
\pgfsetlinewidth{1.003750pt}%
\definecolor{currentstroke}{rgb}{0.121569,0.466667,0.705882}%
\pgfsetstrokecolor{currentstroke}%
\pgfsetstrokeopacity{0.545383}%
\pgfsetdash{}{0pt}%
\pgfpathmoveto{\pgfqpoint{2.800093in}{3.042550in}}%
\pgfpathcurveto{\pgfqpoint{2.808329in}{3.042550in}}{\pgfqpoint{2.816229in}{3.045823in}}{\pgfqpoint{2.822053in}{3.051646in}}%
\pgfpathcurveto{\pgfqpoint{2.827877in}{3.057470in}}{\pgfqpoint{2.831149in}{3.065370in}}{\pgfqpoint{2.831149in}{3.073607in}}%
\pgfpathcurveto{\pgfqpoint{2.831149in}{3.081843in}}{\pgfqpoint{2.827877in}{3.089743in}}{\pgfqpoint{2.822053in}{3.095567in}}%
\pgfpathcurveto{\pgfqpoint{2.816229in}{3.101391in}}{\pgfqpoint{2.808329in}{3.104663in}}{\pgfqpoint{2.800093in}{3.104663in}}%
\pgfpathcurveto{\pgfqpoint{2.791857in}{3.104663in}}{\pgfqpoint{2.783956in}{3.101391in}}{\pgfqpoint{2.778133in}{3.095567in}}%
\pgfpathcurveto{\pgfqpoint{2.772309in}{3.089743in}}{\pgfqpoint{2.769036in}{3.081843in}}{\pgfqpoint{2.769036in}{3.073607in}}%
\pgfpathcurveto{\pgfqpoint{2.769036in}{3.065370in}}{\pgfqpoint{2.772309in}{3.057470in}}{\pgfqpoint{2.778133in}{3.051646in}}%
\pgfpathcurveto{\pgfqpoint{2.783956in}{3.045823in}}{\pgfqpoint{2.791857in}{3.042550in}}{\pgfqpoint{2.800093in}{3.042550in}}%
\pgfpathclose%
\pgfusepath{stroke,fill}%
\end{pgfscope}%
\begin{pgfscope}%
\pgfpathrectangle{\pgfqpoint{0.100000in}{0.220728in}}{\pgfqpoint{3.696000in}{3.696000in}}%
\pgfusepath{clip}%
\pgfsetbuttcap%
\pgfsetroundjoin%
\definecolor{currentfill}{rgb}{0.121569,0.466667,0.705882}%
\pgfsetfillcolor{currentfill}%
\pgfsetfillopacity{0.545394}%
\pgfsetlinewidth{1.003750pt}%
\definecolor{currentstroke}{rgb}{0.121569,0.466667,0.705882}%
\pgfsetstrokecolor{currentstroke}%
\pgfsetstrokeopacity{0.545394}%
\pgfsetdash{}{0pt}%
\pgfpathmoveto{\pgfqpoint{1.042242in}{1.675168in}}%
\pgfpathcurveto{\pgfqpoint{1.050478in}{1.675168in}}{\pgfqpoint{1.058378in}{1.678441in}}{\pgfqpoint{1.064202in}{1.684265in}}%
\pgfpathcurveto{\pgfqpoint{1.070026in}{1.690088in}}{\pgfqpoint{1.073298in}{1.697989in}}{\pgfqpoint{1.073298in}{1.706225in}}%
\pgfpathcurveto{\pgfqpoint{1.073298in}{1.714461in}}{\pgfqpoint{1.070026in}{1.722361in}}{\pgfqpoint{1.064202in}{1.728185in}}%
\pgfpathcurveto{\pgfqpoint{1.058378in}{1.734009in}}{\pgfqpoint{1.050478in}{1.737281in}}{\pgfqpoint{1.042242in}{1.737281in}}%
\pgfpathcurveto{\pgfqpoint{1.034006in}{1.737281in}}{\pgfqpoint{1.026106in}{1.734009in}}{\pgfqpoint{1.020282in}{1.728185in}}%
\pgfpathcurveto{\pgfqpoint{1.014458in}{1.722361in}}{\pgfqpoint{1.011185in}{1.714461in}}{\pgfqpoint{1.011185in}{1.706225in}}%
\pgfpathcurveto{\pgfqpoint{1.011185in}{1.697989in}}{\pgfqpoint{1.014458in}{1.690088in}}{\pgfqpoint{1.020282in}{1.684265in}}%
\pgfpathcurveto{\pgfqpoint{1.026106in}{1.678441in}}{\pgfqpoint{1.034006in}{1.675168in}}{\pgfqpoint{1.042242in}{1.675168in}}%
\pgfpathclose%
\pgfusepath{stroke,fill}%
\end{pgfscope}%
\begin{pgfscope}%
\pgfpathrectangle{\pgfqpoint{0.100000in}{0.220728in}}{\pgfqpoint{3.696000in}{3.696000in}}%
\pgfusepath{clip}%
\pgfsetbuttcap%
\pgfsetroundjoin%
\definecolor{currentfill}{rgb}{0.121569,0.466667,0.705882}%
\pgfsetfillcolor{currentfill}%
\pgfsetfillopacity{0.546507}%
\pgfsetlinewidth{1.003750pt}%
\definecolor{currentstroke}{rgb}{0.121569,0.466667,0.705882}%
\pgfsetstrokecolor{currentstroke}%
\pgfsetstrokeopacity{0.546507}%
\pgfsetdash{}{0pt}%
\pgfpathmoveto{\pgfqpoint{2.805029in}{3.041064in}}%
\pgfpathcurveto{\pgfqpoint{2.813265in}{3.041064in}}{\pgfqpoint{2.821165in}{3.044336in}}{\pgfqpoint{2.826989in}{3.050160in}}%
\pgfpathcurveto{\pgfqpoint{2.832813in}{3.055984in}}{\pgfqpoint{2.836085in}{3.063884in}}{\pgfqpoint{2.836085in}{3.072120in}}%
\pgfpathcurveto{\pgfqpoint{2.836085in}{3.080357in}}{\pgfqpoint{2.832813in}{3.088257in}}{\pgfqpoint{2.826989in}{3.094081in}}%
\pgfpathcurveto{\pgfqpoint{2.821165in}{3.099905in}}{\pgfqpoint{2.813265in}{3.103177in}}{\pgfqpoint{2.805029in}{3.103177in}}%
\pgfpathcurveto{\pgfqpoint{2.796793in}{3.103177in}}{\pgfqpoint{2.788893in}{3.099905in}}{\pgfqpoint{2.783069in}{3.094081in}}%
\pgfpathcurveto{\pgfqpoint{2.777245in}{3.088257in}}{\pgfqpoint{2.773972in}{3.080357in}}{\pgfqpoint{2.773972in}{3.072120in}}%
\pgfpathcurveto{\pgfqpoint{2.773972in}{3.063884in}}{\pgfqpoint{2.777245in}{3.055984in}}{\pgfqpoint{2.783069in}{3.050160in}}%
\pgfpathcurveto{\pgfqpoint{2.788893in}{3.044336in}}{\pgfqpoint{2.796793in}{3.041064in}}{\pgfqpoint{2.805029in}{3.041064in}}%
\pgfpathclose%
\pgfusepath{stroke,fill}%
\end{pgfscope}%
\begin{pgfscope}%
\pgfpathrectangle{\pgfqpoint{0.100000in}{0.220728in}}{\pgfqpoint{3.696000in}{3.696000in}}%
\pgfusepath{clip}%
\pgfsetbuttcap%
\pgfsetroundjoin%
\definecolor{currentfill}{rgb}{0.121569,0.466667,0.705882}%
\pgfsetfillcolor{currentfill}%
\pgfsetfillopacity{0.546668}%
\pgfsetlinewidth{1.003750pt}%
\definecolor{currentstroke}{rgb}{0.121569,0.466667,0.705882}%
\pgfsetstrokecolor{currentstroke}%
\pgfsetstrokeopacity{0.546668}%
\pgfsetdash{}{0pt}%
\pgfpathmoveto{\pgfqpoint{1.033105in}{1.663046in}}%
\pgfpathcurveto{\pgfqpoint{1.041341in}{1.663046in}}{\pgfqpoint{1.049241in}{1.666318in}}{\pgfqpoint{1.055065in}{1.672142in}}%
\pgfpathcurveto{\pgfqpoint{1.060889in}{1.677966in}}{\pgfqpoint{1.064161in}{1.685866in}}{\pgfqpoint{1.064161in}{1.694102in}}%
\pgfpathcurveto{\pgfqpoint{1.064161in}{1.702339in}}{\pgfqpoint{1.060889in}{1.710239in}}{\pgfqpoint{1.055065in}{1.716063in}}%
\pgfpathcurveto{\pgfqpoint{1.049241in}{1.721887in}}{\pgfqpoint{1.041341in}{1.725159in}}{\pgfqpoint{1.033105in}{1.725159in}}%
\pgfpathcurveto{\pgfqpoint{1.024869in}{1.725159in}}{\pgfqpoint{1.016969in}{1.721887in}}{\pgfqpoint{1.011145in}{1.716063in}}%
\pgfpathcurveto{\pgfqpoint{1.005321in}{1.710239in}}{\pgfqpoint{1.002048in}{1.702339in}}{\pgfqpoint{1.002048in}{1.694102in}}%
\pgfpathcurveto{\pgfqpoint{1.002048in}{1.685866in}}{\pgfqpoint{1.005321in}{1.677966in}}{\pgfqpoint{1.011145in}{1.672142in}}%
\pgfpathcurveto{\pgfqpoint{1.016969in}{1.666318in}}{\pgfqpoint{1.024869in}{1.663046in}}{\pgfqpoint{1.033105in}{1.663046in}}%
\pgfpathclose%
\pgfusepath{stroke,fill}%
\end{pgfscope}%
\begin{pgfscope}%
\pgfpathrectangle{\pgfqpoint{0.100000in}{0.220728in}}{\pgfqpoint{3.696000in}{3.696000in}}%
\pgfusepath{clip}%
\pgfsetbuttcap%
\pgfsetroundjoin%
\definecolor{currentfill}{rgb}{0.121569,0.466667,0.705882}%
\pgfsetfillcolor{currentfill}%
\pgfsetfillopacity{0.547961}%
\pgfsetlinewidth{1.003750pt}%
\definecolor{currentstroke}{rgb}{0.121569,0.466667,0.705882}%
\pgfsetstrokecolor{currentstroke}%
\pgfsetstrokeopacity{0.547961}%
\pgfsetdash{}{0pt}%
\pgfpathmoveto{\pgfqpoint{2.811944in}{3.039212in}}%
\pgfpathcurveto{\pgfqpoint{2.820180in}{3.039212in}}{\pgfqpoint{2.828080in}{3.042484in}}{\pgfqpoint{2.833904in}{3.048308in}}%
\pgfpathcurveto{\pgfqpoint{2.839728in}{3.054132in}}{\pgfqpoint{2.843001in}{3.062032in}}{\pgfqpoint{2.843001in}{3.070268in}}%
\pgfpathcurveto{\pgfqpoint{2.843001in}{3.078505in}}{\pgfqpoint{2.839728in}{3.086405in}}{\pgfqpoint{2.833904in}{3.092229in}}%
\pgfpathcurveto{\pgfqpoint{2.828080in}{3.098052in}}{\pgfqpoint{2.820180in}{3.101325in}}{\pgfqpoint{2.811944in}{3.101325in}}%
\pgfpathcurveto{\pgfqpoint{2.803708in}{3.101325in}}{\pgfqpoint{2.795808in}{3.098052in}}{\pgfqpoint{2.789984in}{3.092229in}}%
\pgfpathcurveto{\pgfqpoint{2.784160in}{3.086405in}}{\pgfqpoint{2.780888in}{3.078505in}}{\pgfqpoint{2.780888in}{3.070268in}}%
\pgfpathcurveto{\pgfqpoint{2.780888in}{3.062032in}}{\pgfqpoint{2.784160in}{3.054132in}}{\pgfqpoint{2.789984in}{3.048308in}}%
\pgfpathcurveto{\pgfqpoint{2.795808in}{3.042484in}}{\pgfqpoint{2.803708in}{3.039212in}}{\pgfqpoint{2.811944in}{3.039212in}}%
\pgfpathclose%
\pgfusepath{stroke,fill}%
\end{pgfscope}%
\begin{pgfscope}%
\pgfpathrectangle{\pgfqpoint{0.100000in}{0.220728in}}{\pgfqpoint{3.696000in}{3.696000in}}%
\pgfusepath{clip}%
\pgfsetbuttcap%
\pgfsetroundjoin%
\definecolor{currentfill}{rgb}{0.121569,0.466667,0.705882}%
\pgfsetfillcolor{currentfill}%
\pgfsetfillopacity{0.548648}%
\pgfsetlinewidth{1.003750pt}%
\definecolor{currentstroke}{rgb}{0.121569,0.466667,0.705882}%
\pgfsetstrokecolor{currentstroke}%
\pgfsetstrokeopacity{0.548648}%
\pgfsetdash{}{0pt}%
\pgfpathmoveto{\pgfqpoint{1.031401in}{1.649968in}}%
\pgfpathcurveto{\pgfqpoint{1.039637in}{1.649968in}}{\pgfqpoint{1.047537in}{1.653240in}}{\pgfqpoint{1.053361in}{1.659064in}}%
\pgfpathcurveto{\pgfqpoint{1.059185in}{1.664888in}}{\pgfqpoint{1.062457in}{1.672788in}}{\pgfqpoint{1.062457in}{1.681025in}}%
\pgfpathcurveto{\pgfqpoint{1.062457in}{1.689261in}}{\pgfqpoint{1.059185in}{1.697161in}}{\pgfqpoint{1.053361in}{1.702985in}}%
\pgfpathcurveto{\pgfqpoint{1.047537in}{1.708809in}}{\pgfqpoint{1.039637in}{1.712081in}}{\pgfqpoint{1.031401in}{1.712081in}}%
\pgfpathcurveto{\pgfqpoint{1.023164in}{1.712081in}}{\pgfqpoint{1.015264in}{1.708809in}}{\pgfqpoint{1.009440in}{1.702985in}}%
\pgfpathcurveto{\pgfqpoint{1.003616in}{1.697161in}}{\pgfqpoint{1.000344in}{1.689261in}}{\pgfqpoint{1.000344in}{1.681025in}}%
\pgfpathcurveto{\pgfqpoint{1.000344in}{1.672788in}}{\pgfqpoint{1.003616in}{1.664888in}}{\pgfqpoint{1.009440in}{1.659064in}}%
\pgfpathcurveto{\pgfqpoint{1.015264in}{1.653240in}}{\pgfqpoint{1.023164in}{1.649968in}}{\pgfqpoint{1.031401in}{1.649968in}}%
\pgfpathclose%
\pgfusepath{stroke,fill}%
\end{pgfscope}%
\begin{pgfscope}%
\pgfpathrectangle{\pgfqpoint{0.100000in}{0.220728in}}{\pgfqpoint{3.696000in}{3.696000in}}%
\pgfusepath{clip}%
\pgfsetbuttcap%
\pgfsetroundjoin%
\definecolor{currentfill}{rgb}{0.121569,0.466667,0.705882}%
\pgfsetfillcolor{currentfill}%
\pgfsetfillopacity{0.549104}%
\pgfsetlinewidth{1.003750pt}%
\definecolor{currentstroke}{rgb}{0.121569,0.466667,0.705882}%
\pgfsetstrokecolor{currentstroke}%
\pgfsetstrokeopacity{0.549104}%
\pgfsetdash{}{0pt}%
\pgfpathmoveto{\pgfqpoint{2.815556in}{3.038957in}}%
\pgfpathcurveto{\pgfqpoint{2.823792in}{3.038957in}}{\pgfqpoint{2.831692in}{3.042230in}}{\pgfqpoint{2.837516in}{3.048053in}}%
\pgfpathcurveto{\pgfqpoint{2.843340in}{3.053877in}}{\pgfqpoint{2.846612in}{3.061777in}}{\pgfqpoint{2.846612in}{3.070014in}}%
\pgfpathcurveto{\pgfqpoint{2.846612in}{3.078250in}}{\pgfqpoint{2.843340in}{3.086150in}}{\pgfqpoint{2.837516in}{3.091974in}}%
\pgfpathcurveto{\pgfqpoint{2.831692in}{3.097798in}}{\pgfqpoint{2.823792in}{3.101070in}}{\pgfqpoint{2.815556in}{3.101070in}}%
\pgfpathcurveto{\pgfqpoint{2.807320in}{3.101070in}}{\pgfqpoint{2.799420in}{3.097798in}}{\pgfqpoint{2.793596in}{3.091974in}}%
\pgfpathcurveto{\pgfqpoint{2.787772in}{3.086150in}}{\pgfqpoint{2.784499in}{3.078250in}}{\pgfqpoint{2.784499in}{3.070014in}}%
\pgfpathcurveto{\pgfqpoint{2.784499in}{3.061777in}}{\pgfqpoint{2.787772in}{3.053877in}}{\pgfqpoint{2.793596in}{3.048053in}}%
\pgfpathcurveto{\pgfqpoint{2.799420in}{3.042230in}}{\pgfqpoint{2.807320in}{3.038957in}}{\pgfqpoint{2.815556in}{3.038957in}}%
\pgfpathclose%
\pgfusepath{stroke,fill}%
\end{pgfscope}%
\begin{pgfscope}%
\pgfpathrectangle{\pgfqpoint{0.100000in}{0.220728in}}{\pgfqpoint{3.696000in}{3.696000in}}%
\pgfusepath{clip}%
\pgfsetbuttcap%
\pgfsetroundjoin%
\definecolor{currentfill}{rgb}{0.121569,0.466667,0.705882}%
\pgfsetfillcolor{currentfill}%
\pgfsetfillopacity{0.549635}%
\pgfsetlinewidth{1.003750pt}%
\definecolor{currentstroke}{rgb}{0.121569,0.466667,0.705882}%
\pgfsetstrokecolor{currentstroke}%
\pgfsetstrokeopacity{0.549635}%
\pgfsetdash{}{0pt}%
\pgfpathmoveto{\pgfqpoint{1.025263in}{1.642931in}}%
\pgfpathcurveto{\pgfqpoint{1.033499in}{1.642931in}}{\pgfqpoint{1.041399in}{1.646203in}}{\pgfqpoint{1.047223in}{1.652027in}}%
\pgfpathcurveto{\pgfqpoint{1.053047in}{1.657851in}}{\pgfqpoint{1.056320in}{1.665751in}}{\pgfqpoint{1.056320in}{1.673987in}}%
\pgfpathcurveto{\pgfqpoint{1.056320in}{1.682224in}}{\pgfqpoint{1.053047in}{1.690124in}}{\pgfqpoint{1.047223in}{1.695948in}}%
\pgfpathcurveto{\pgfqpoint{1.041399in}{1.701772in}}{\pgfqpoint{1.033499in}{1.705044in}}{\pgfqpoint{1.025263in}{1.705044in}}%
\pgfpathcurveto{\pgfqpoint{1.017027in}{1.705044in}}{\pgfqpoint{1.009127in}{1.701772in}}{\pgfqpoint{1.003303in}{1.695948in}}%
\pgfpathcurveto{\pgfqpoint{0.997479in}{1.690124in}}{\pgfqpoint{0.994207in}{1.682224in}}{\pgfqpoint{0.994207in}{1.673987in}}%
\pgfpathcurveto{\pgfqpoint{0.994207in}{1.665751in}}{\pgfqpoint{0.997479in}{1.657851in}}{\pgfqpoint{1.003303in}{1.652027in}}%
\pgfpathcurveto{\pgfqpoint{1.009127in}{1.646203in}}{\pgfqpoint{1.017027in}{1.642931in}}{\pgfqpoint{1.025263in}{1.642931in}}%
\pgfpathclose%
\pgfusepath{stroke,fill}%
\end{pgfscope}%
\begin{pgfscope}%
\pgfpathrectangle{\pgfqpoint{0.100000in}{0.220728in}}{\pgfqpoint{3.696000in}{3.696000in}}%
\pgfusepath{clip}%
\pgfsetbuttcap%
\pgfsetroundjoin%
\definecolor{currentfill}{rgb}{0.121569,0.466667,0.705882}%
\pgfsetfillcolor{currentfill}%
\pgfsetfillopacity{0.550100}%
\pgfsetlinewidth{1.003750pt}%
\definecolor{currentstroke}{rgb}{0.121569,0.466667,0.705882}%
\pgfsetstrokecolor{currentstroke}%
\pgfsetstrokeopacity{0.550100}%
\pgfsetdash{}{0pt}%
\pgfpathmoveto{\pgfqpoint{2.820212in}{3.038081in}}%
\pgfpathcurveto{\pgfqpoint{2.828448in}{3.038081in}}{\pgfqpoint{2.836348in}{3.041353in}}{\pgfqpoint{2.842172in}{3.047177in}}%
\pgfpathcurveto{\pgfqpoint{2.847996in}{3.053001in}}{\pgfqpoint{2.851268in}{3.060901in}}{\pgfqpoint{2.851268in}{3.069138in}}%
\pgfpathcurveto{\pgfqpoint{2.851268in}{3.077374in}}{\pgfqpoint{2.847996in}{3.085274in}}{\pgfqpoint{2.842172in}{3.091098in}}%
\pgfpathcurveto{\pgfqpoint{2.836348in}{3.096922in}}{\pgfqpoint{2.828448in}{3.100194in}}{\pgfqpoint{2.820212in}{3.100194in}}%
\pgfpathcurveto{\pgfqpoint{2.811976in}{3.100194in}}{\pgfqpoint{2.804075in}{3.096922in}}{\pgfqpoint{2.798252in}{3.091098in}}%
\pgfpathcurveto{\pgfqpoint{2.792428in}{3.085274in}}{\pgfqpoint{2.789155in}{3.077374in}}{\pgfqpoint{2.789155in}{3.069138in}}%
\pgfpathcurveto{\pgfqpoint{2.789155in}{3.060901in}}{\pgfqpoint{2.792428in}{3.053001in}}{\pgfqpoint{2.798252in}{3.047177in}}%
\pgfpathcurveto{\pgfqpoint{2.804075in}{3.041353in}}{\pgfqpoint{2.811976in}{3.038081in}}{\pgfqpoint{2.820212in}{3.038081in}}%
\pgfpathclose%
\pgfusepath{stroke,fill}%
\end{pgfscope}%
\begin{pgfscope}%
\pgfpathrectangle{\pgfqpoint{0.100000in}{0.220728in}}{\pgfqpoint{3.696000in}{3.696000in}}%
\pgfusepath{clip}%
\pgfsetbuttcap%
\pgfsetroundjoin%
\definecolor{currentfill}{rgb}{0.121569,0.466667,0.705882}%
\pgfsetfillcolor{currentfill}%
\pgfsetfillopacity{0.550629}%
\pgfsetlinewidth{1.003750pt}%
\definecolor{currentstroke}{rgb}{0.121569,0.466667,0.705882}%
\pgfsetstrokecolor{currentstroke}%
\pgfsetstrokeopacity{0.550629}%
\pgfsetdash{}{0pt}%
\pgfpathmoveto{\pgfqpoint{1.024002in}{1.636451in}}%
\pgfpathcurveto{\pgfqpoint{1.032238in}{1.636451in}}{\pgfqpoint{1.040138in}{1.639723in}}{\pgfqpoint{1.045962in}{1.645547in}}%
\pgfpathcurveto{\pgfqpoint{1.051786in}{1.651371in}}{\pgfqpoint{1.055058in}{1.659271in}}{\pgfqpoint{1.055058in}{1.667508in}}%
\pgfpathcurveto{\pgfqpoint{1.055058in}{1.675744in}}{\pgfqpoint{1.051786in}{1.683644in}}{\pgfqpoint{1.045962in}{1.689468in}}%
\pgfpathcurveto{\pgfqpoint{1.040138in}{1.695292in}}{\pgfqpoint{1.032238in}{1.698564in}}{\pgfqpoint{1.024002in}{1.698564in}}%
\pgfpathcurveto{\pgfqpoint{1.015765in}{1.698564in}}{\pgfqpoint{1.007865in}{1.695292in}}{\pgfqpoint{1.002041in}{1.689468in}}%
\pgfpathcurveto{\pgfqpoint{0.996218in}{1.683644in}}{\pgfqpoint{0.992945in}{1.675744in}}{\pgfqpoint{0.992945in}{1.667508in}}%
\pgfpathcurveto{\pgfqpoint{0.992945in}{1.659271in}}{\pgfqpoint{0.996218in}{1.651371in}}{\pgfqpoint{1.002041in}{1.645547in}}%
\pgfpathcurveto{\pgfqpoint{1.007865in}{1.639723in}}{\pgfqpoint{1.015765in}{1.636451in}}{\pgfqpoint{1.024002in}{1.636451in}}%
\pgfpathclose%
\pgfusepath{stroke,fill}%
\end{pgfscope}%
\begin{pgfscope}%
\pgfpathrectangle{\pgfqpoint{0.100000in}{0.220728in}}{\pgfqpoint{3.696000in}{3.696000in}}%
\pgfusepath{clip}%
\pgfsetbuttcap%
\pgfsetroundjoin%
\definecolor{currentfill}{rgb}{0.121569,0.466667,0.705882}%
\pgfsetfillcolor{currentfill}%
\pgfsetfillopacity{0.550701}%
\pgfsetlinewidth{1.003750pt}%
\definecolor{currentstroke}{rgb}{0.121569,0.466667,0.705882}%
\pgfsetstrokecolor{currentstroke}%
\pgfsetstrokeopacity{0.550701}%
\pgfsetdash{}{0pt}%
\pgfpathmoveto{\pgfqpoint{2.822725in}{3.037647in}}%
\pgfpathcurveto{\pgfqpoint{2.830961in}{3.037647in}}{\pgfqpoint{2.838861in}{3.040919in}}{\pgfqpoint{2.844685in}{3.046743in}}%
\pgfpathcurveto{\pgfqpoint{2.850509in}{3.052567in}}{\pgfqpoint{2.853782in}{3.060467in}}{\pgfqpoint{2.853782in}{3.068703in}}%
\pgfpathcurveto{\pgfqpoint{2.853782in}{3.076940in}}{\pgfqpoint{2.850509in}{3.084840in}}{\pgfqpoint{2.844685in}{3.090664in}}%
\pgfpathcurveto{\pgfqpoint{2.838861in}{3.096488in}}{\pgfqpoint{2.830961in}{3.099760in}}{\pgfqpoint{2.822725in}{3.099760in}}%
\pgfpathcurveto{\pgfqpoint{2.814489in}{3.099760in}}{\pgfqpoint{2.806589in}{3.096488in}}{\pgfqpoint{2.800765in}{3.090664in}}%
\pgfpathcurveto{\pgfqpoint{2.794941in}{3.084840in}}{\pgfqpoint{2.791669in}{3.076940in}}{\pgfqpoint{2.791669in}{3.068703in}}%
\pgfpathcurveto{\pgfqpoint{2.791669in}{3.060467in}}{\pgfqpoint{2.794941in}{3.052567in}}{\pgfqpoint{2.800765in}{3.046743in}}%
\pgfpathcurveto{\pgfqpoint{2.806589in}{3.040919in}}{\pgfqpoint{2.814489in}{3.037647in}}{\pgfqpoint{2.822725in}{3.037647in}}%
\pgfpathclose%
\pgfusepath{stroke,fill}%
\end{pgfscope}%
\begin{pgfscope}%
\pgfpathrectangle{\pgfqpoint{0.100000in}{0.220728in}}{\pgfqpoint{3.696000in}{3.696000in}}%
\pgfusepath{clip}%
\pgfsetbuttcap%
\pgfsetroundjoin%
\definecolor{currentfill}{rgb}{0.121569,0.466667,0.705882}%
\pgfsetfillcolor{currentfill}%
\pgfsetfillopacity{0.551101}%
\pgfsetlinewidth{1.003750pt}%
\definecolor{currentstroke}{rgb}{0.121569,0.466667,0.705882}%
\pgfsetstrokecolor{currentstroke}%
\pgfsetstrokeopacity{0.551101}%
\pgfsetdash{}{0pt}%
\pgfpathmoveto{\pgfqpoint{1.021290in}{1.632893in}}%
\pgfpathcurveto{\pgfqpoint{1.029526in}{1.632893in}}{\pgfqpoint{1.037426in}{1.636165in}}{\pgfqpoint{1.043250in}{1.641989in}}%
\pgfpathcurveto{\pgfqpoint{1.049074in}{1.647813in}}{\pgfqpoint{1.052346in}{1.655713in}}{\pgfqpoint{1.052346in}{1.663949in}}%
\pgfpathcurveto{\pgfqpoint{1.052346in}{1.672186in}}{\pgfqpoint{1.049074in}{1.680086in}}{\pgfqpoint{1.043250in}{1.685910in}}%
\pgfpathcurveto{\pgfqpoint{1.037426in}{1.691734in}}{\pgfqpoint{1.029526in}{1.695006in}}{\pgfqpoint{1.021290in}{1.695006in}}%
\pgfpathcurveto{\pgfqpoint{1.013053in}{1.695006in}}{\pgfqpoint{1.005153in}{1.691734in}}{\pgfqpoint{0.999329in}{1.685910in}}%
\pgfpathcurveto{\pgfqpoint{0.993505in}{1.680086in}}{\pgfqpoint{0.990233in}{1.672186in}}{\pgfqpoint{0.990233in}{1.663949in}}%
\pgfpathcurveto{\pgfqpoint{0.990233in}{1.655713in}}{\pgfqpoint{0.993505in}{1.647813in}}{\pgfqpoint{0.999329in}{1.641989in}}%
\pgfpathcurveto{\pgfqpoint{1.005153in}{1.636165in}}{\pgfqpoint{1.013053in}{1.632893in}}{\pgfqpoint{1.021290in}{1.632893in}}%
\pgfpathclose%
\pgfusepath{stroke,fill}%
\end{pgfscope}%
\begin{pgfscope}%
\pgfpathrectangle{\pgfqpoint{0.100000in}{0.220728in}}{\pgfqpoint{3.696000in}{3.696000in}}%
\pgfusepath{clip}%
\pgfsetbuttcap%
\pgfsetroundjoin%
\definecolor{currentfill}{rgb}{0.121569,0.466667,0.705882}%
\pgfsetfillcolor{currentfill}%
\pgfsetfillopacity{0.551414}%
\pgfsetlinewidth{1.003750pt}%
\definecolor{currentstroke}{rgb}{0.121569,0.466667,0.705882}%
\pgfsetstrokecolor{currentstroke}%
\pgfsetstrokeopacity{0.551414}%
\pgfsetdash{}{0pt}%
\pgfpathmoveto{\pgfqpoint{2.826601in}{3.036413in}}%
\pgfpathcurveto{\pgfqpoint{2.834838in}{3.036413in}}{\pgfqpoint{2.842738in}{3.039686in}}{\pgfqpoint{2.848562in}{3.045510in}}%
\pgfpathcurveto{\pgfqpoint{2.854386in}{3.051333in}}{\pgfqpoint{2.857658in}{3.059234in}}{\pgfqpoint{2.857658in}{3.067470in}}%
\pgfpathcurveto{\pgfqpoint{2.857658in}{3.075706in}}{\pgfqpoint{2.854386in}{3.083606in}}{\pgfqpoint{2.848562in}{3.089430in}}%
\pgfpathcurveto{\pgfqpoint{2.842738in}{3.095254in}}{\pgfqpoint{2.834838in}{3.098526in}}{\pgfqpoint{2.826601in}{3.098526in}}%
\pgfpathcurveto{\pgfqpoint{2.818365in}{3.098526in}}{\pgfqpoint{2.810465in}{3.095254in}}{\pgfqpoint{2.804641in}{3.089430in}}%
\pgfpathcurveto{\pgfqpoint{2.798817in}{3.083606in}}{\pgfqpoint{2.795545in}{3.075706in}}{\pgfqpoint{2.795545in}{3.067470in}}%
\pgfpathcurveto{\pgfqpoint{2.795545in}{3.059234in}}{\pgfqpoint{2.798817in}{3.051333in}}{\pgfqpoint{2.804641in}{3.045510in}}%
\pgfpathcurveto{\pgfqpoint{2.810465in}{3.039686in}}{\pgfqpoint{2.818365in}{3.036413in}}{\pgfqpoint{2.826601in}{3.036413in}}%
\pgfpathclose%
\pgfusepath{stroke,fill}%
\end{pgfscope}%
\begin{pgfscope}%
\pgfpathrectangle{\pgfqpoint{0.100000in}{0.220728in}}{\pgfqpoint{3.696000in}{3.696000in}}%
\pgfusepath{clip}%
\pgfsetbuttcap%
\pgfsetroundjoin%
\definecolor{currentfill}{rgb}{0.121569,0.466667,0.705882}%
\pgfsetfillcolor{currentfill}%
\pgfsetfillopacity{0.551669}%
\pgfsetlinewidth{1.003750pt}%
\definecolor{currentstroke}{rgb}{0.121569,0.466667,0.705882}%
\pgfsetstrokecolor{currentstroke}%
\pgfsetstrokeopacity{0.551669}%
\pgfsetdash{}{0pt}%
\pgfpathmoveto{\pgfqpoint{1.020570in}{1.629391in}}%
\pgfpathcurveto{\pgfqpoint{1.028806in}{1.629391in}}{\pgfqpoint{1.036706in}{1.632663in}}{\pgfqpoint{1.042530in}{1.638487in}}%
\pgfpathcurveto{\pgfqpoint{1.048354in}{1.644311in}}{\pgfqpoint{1.051626in}{1.652211in}}{\pgfqpoint{1.051626in}{1.660447in}}%
\pgfpathcurveto{\pgfqpoint{1.051626in}{1.668683in}}{\pgfqpoint{1.048354in}{1.676583in}}{\pgfqpoint{1.042530in}{1.682407in}}%
\pgfpathcurveto{\pgfqpoint{1.036706in}{1.688231in}}{\pgfqpoint{1.028806in}{1.691504in}}{\pgfqpoint{1.020570in}{1.691504in}}%
\pgfpathcurveto{\pgfqpoint{1.012333in}{1.691504in}}{\pgfqpoint{1.004433in}{1.688231in}}{\pgfqpoint{0.998609in}{1.682407in}}%
\pgfpathcurveto{\pgfqpoint{0.992785in}{1.676583in}}{\pgfqpoint{0.989513in}{1.668683in}}{\pgfqpoint{0.989513in}{1.660447in}}%
\pgfpathcurveto{\pgfqpoint{0.989513in}{1.652211in}}{\pgfqpoint{0.992785in}{1.644311in}}{\pgfqpoint{0.998609in}{1.638487in}}%
\pgfpathcurveto{\pgfqpoint{1.004433in}{1.632663in}}{\pgfqpoint{1.012333in}{1.629391in}}{\pgfqpoint{1.020570in}{1.629391in}}%
\pgfpathclose%
\pgfusepath{stroke,fill}%
\end{pgfscope}%
\begin{pgfscope}%
\pgfpathrectangle{\pgfqpoint{0.100000in}{0.220728in}}{\pgfqpoint{3.696000in}{3.696000in}}%
\pgfusepath{clip}%
\pgfsetbuttcap%
\pgfsetroundjoin%
\definecolor{currentfill}{rgb}{0.121569,0.466667,0.705882}%
\pgfsetfillcolor{currentfill}%
\pgfsetfillopacity{0.551949}%
\pgfsetlinewidth{1.003750pt}%
\definecolor{currentstroke}{rgb}{0.121569,0.466667,0.705882}%
\pgfsetstrokecolor{currentstroke}%
\pgfsetstrokeopacity{0.551949}%
\pgfsetdash{}{0pt}%
\pgfpathmoveto{\pgfqpoint{1.018761in}{1.627303in}}%
\pgfpathcurveto{\pgfqpoint{1.026997in}{1.627303in}}{\pgfqpoint{1.034897in}{1.630575in}}{\pgfqpoint{1.040721in}{1.636399in}}%
\pgfpathcurveto{\pgfqpoint{1.046545in}{1.642223in}}{\pgfqpoint{1.049818in}{1.650123in}}{\pgfqpoint{1.049818in}{1.658359in}}%
\pgfpathcurveto{\pgfqpoint{1.049818in}{1.666596in}}{\pgfqpoint{1.046545in}{1.674496in}}{\pgfqpoint{1.040721in}{1.680320in}}%
\pgfpathcurveto{\pgfqpoint{1.034897in}{1.686144in}}{\pgfqpoint{1.026997in}{1.689416in}}{\pgfqpoint{1.018761in}{1.689416in}}%
\pgfpathcurveto{\pgfqpoint{1.010525in}{1.689416in}}{\pgfqpoint{1.002625in}{1.686144in}}{\pgfqpoint{0.996801in}{1.680320in}}%
\pgfpathcurveto{\pgfqpoint{0.990977in}{1.674496in}}{\pgfqpoint{0.987705in}{1.666596in}}{\pgfqpoint{0.987705in}{1.658359in}}%
\pgfpathcurveto{\pgfqpoint{0.987705in}{1.650123in}}{\pgfqpoint{0.990977in}{1.642223in}}{\pgfqpoint{0.996801in}{1.636399in}}%
\pgfpathcurveto{\pgfqpoint{1.002625in}{1.630575in}}{\pgfqpoint{1.010525in}{1.627303in}}{\pgfqpoint{1.018761in}{1.627303in}}%
\pgfpathclose%
\pgfusepath{stroke,fill}%
\end{pgfscope}%
\begin{pgfscope}%
\pgfpathrectangle{\pgfqpoint{0.100000in}{0.220728in}}{\pgfqpoint{3.696000in}{3.696000in}}%
\pgfusepath{clip}%
\pgfsetbuttcap%
\pgfsetroundjoin%
\definecolor{currentfill}{rgb}{0.121569,0.466667,0.705882}%
\pgfsetfillcolor{currentfill}%
\pgfsetfillopacity{0.552336}%
\pgfsetlinewidth{1.003750pt}%
\definecolor{currentstroke}{rgb}{0.121569,0.466667,0.705882}%
\pgfsetstrokecolor{currentstroke}%
\pgfsetstrokeopacity{0.552336}%
\pgfsetdash{}{0pt}%
\pgfpathmoveto{\pgfqpoint{2.831843in}{3.035722in}}%
\pgfpathcurveto{\pgfqpoint{2.840079in}{3.035722in}}{\pgfqpoint{2.847979in}{3.038994in}}{\pgfqpoint{2.853803in}{3.044818in}}%
\pgfpathcurveto{\pgfqpoint{2.859627in}{3.050642in}}{\pgfqpoint{2.862899in}{3.058542in}}{\pgfqpoint{2.862899in}{3.066779in}}%
\pgfpathcurveto{\pgfqpoint{2.862899in}{3.075015in}}{\pgfqpoint{2.859627in}{3.082915in}}{\pgfqpoint{2.853803in}{3.088739in}}%
\pgfpathcurveto{\pgfqpoint{2.847979in}{3.094563in}}{\pgfqpoint{2.840079in}{3.097835in}}{\pgfqpoint{2.831843in}{3.097835in}}%
\pgfpathcurveto{\pgfqpoint{2.823607in}{3.097835in}}{\pgfqpoint{2.815707in}{3.094563in}}{\pgfqpoint{2.809883in}{3.088739in}}%
\pgfpathcurveto{\pgfqpoint{2.804059in}{3.082915in}}{\pgfqpoint{2.800786in}{3.075015in}}{\pgfqpoint{2.800786in}{3.066779in}}%
\pgfpathcurveto{\pgfqpoint{2.800786in}{3.058542in}}{\pgfqpoint{2.804059in}{3.050642in}}{\pgfqpoint{2.809883in}{3.044818in}}%
\pgfpathcurveto{\pgfqpoint{2.815707in}{3.038994in}}{\pgfqpoint{2.823607in}{3.035722in}}{\pgfqpoint{2.831843in}{3.035722in}}%
\pgfpathclose%
\pgfusepath{stroke,fill}%
\end{pgfscope}%
\begin{pgfscope}%
\pgfpathrectangle{\pgfqpoint{0.100000in}{0.220728in}}{\pgfqpoint{3.696000in}{3.696000in}}%
\pgfusepath{clip}%
\pgfsetbuttcap%
\pgfsetroundjoin%
\definecolor{currentfill}{rgb}{0.121569,0.466667,0.705882}%
\pgfsetfillcolor{currentfill}%
\pgfsetfillopacity{0.552827}%
\pgfsetlinewidth{1.003750pt}%
\definecolor{currentstroke}{rgb}{0.121569,0.466667,0.705882}%
\pgfsetstrokecolor{currentstroke}%
\pgfsetstrokeopacity{0.552827}%
\pgfsetdash{}{0pt}%
\pgfpathmoveto{\pgfqpoint{1.017189in}{1.622622in}}%
\pgfpathcurveto{\pgfqpoint{1.025425in}{1.622622in}}{\pgfqpoint{1.033325in}{1.625894in}}{\pgfqpoint{1.039149in}{1.631718in}}%
\pgfpathcurveto{\pgfqpoint{1.044973in}{1.637542in}}{\pgfqpoint{1.048245in}{1.645442in}}{\pgfqpoint{1.048245in}{1.653678in}}%
\pgfpathcurveto{\pgfqpoint{1.048245in}{1.661914in}}{\pgfqpoint{1.044973in}{1.669814in}}{\pgfqpoint{1.039149in}{1.675638in}}%
\pgfpathcurveto{\pgfqpoint{1.033325in}{1.681462in}}{\pgfqpoint{1.025425in}{1.684735in}}{\pgfqpoint{1.017189in}{1.684735in}}%
\pgfpathcurveto{\pgfqpoint{1.008953in}{1.684735in}}{\pgfqpoint{1.001053in}{1.681462in}}{\pgfqpoint{0.995229in}{1.675638in}}%
\pgfpathcurveto{\pgfqpoint{0.989405in}{1.669814in}}{\pgfqpoint{0.986132in}{1.661914in}}{\pgfqpoint{0.986132in}{1.653678in}}%
\pgfpathcurveto{\pgfqpoint{0.986132in}{1.645442in}}{\pgfqpoint{0.989405in}{1.637542in}}{\pgfqpoint{0.995229in}{1.631718in}}%
\pgfpathcurveto{\pgfqpoint{1.001053in}{1.625894in}}{\pgfqpoint{1.008953in}{1.622622in}}{\pgfqpoint{1.017189in}{1.622622in}}%
\pgfpathclose%
\pgfusepath{stroke,fill}%
\end{pgfscope}%
\begin{pgfscope}%
\pgfpathrectangle{\pgfqpoint{0.100000in}{0.220728in}}{\pgfqpoint{3.696000in}{3.696000in}}%
\pgfusepath{clip}%
\pgfsetbuttcap%
\pgfsetroundjoin%
\definecolor{currentfill}{rgb}{0.121569,0.466667,0.705882}%
\pgfsetfillcolor{currentfill}%
\pgfsetfillopacity{0.553160}%
\pgfsetlinewidth{1.003750pt}%
\definecolor{currentstroke}{rgb}{0.121569,0.466667,0.705882}%
\pgfsetstrokecolor{currentstroke}%
\pgfsetstrokeopacity{0.553160}%
\pgfsetdash{}{0pt}%
\pgfpathmoveto{\pgfqpoint{2.834441in}{3.035595in}}%
\pgfpathcurveto{\pgfqpoint{2.842677in}{3.035595in}}{\pgfqpoint{2.850577in}{3.038867in}}{\pgfqpoint{2.856401in}{3.044691in}}%
\pgfpathcurveto{\pgfqpoint{2.862225in}{3.050515in}}{\pgfqpoint{2.865497in}{3.058415in}}{\pgfqpoint{2.865497in}{3.066651in}}%
\pgfpathcurveto{\pgfqpoint{2.865497in}{3.074888in}}{\pgfqpoint{2.862225in}{3.082788in}}{\pgfqpoint{2.856401in}{3.088612in}}%
\pgfpathcurveto{\pgfqpoint{2.850577in}{3.094436in}}{\pgfqpoint{2.842677in}{3.097708in}}{\pgfqpoint{2.834441in}{3.097708in}}%
\pgfpathcurveto{\pgfqpoint{2.826205in}{3.097708in}}{\pgfqpoint{2.818305in}{3.094436in}}{\pgfqpoint{2.812481in}{3.088612in}}%
\pgfpathcurveto{\pgfqpoint{2.806657in}{3.082788in}}{\pgfqpoint{2.803384in}{3.074888in}}{\pgfqpoint{2.803384in}{3.066651in}}%
\pgfpathcurveto{\pgfqpoint{2.803384in}{3.058415in}}{\pgfqpoint{2.806657in}{3.050515in}}{\pgfqpoint{2.812481in}{3.044691in}}%
\pgfpathcurveto{\pgfqpoint{2.818305in}{3.038867in}}{\pgfqpoint{2.826205in}{3.035595in}}{\pgfqpoint{2.834441in}{3.035595in}}%
\pgfpathclose%
\pgfusepath{stroke,fill}%
\end{pgfscope}%
\begin{pgfscope}%
\pgfpathrectangle{\pgfqpoint{0.100000in}{0.220728in}}{\pgfqpoint{3.696000in}{3.696000in}}%
\pgfusepath{clip}%
\pgfsetbuttcap%
\pgfsetroundjoin%
\definecolor{currentfill}{rgb}{0.121569,0.466667,0.705882}%
\pgfsetfillcolor{currentfill}%
\pgfsetfillopacity{0.553854}%
\pgfsetlinewidth{1.003750pt}%
\definecolor{currentstroke}{rgb}{0.121569,0.466667,0.705882}%
\pgfsetstrokecolor{currentstroke}%
\pgfsetstrokeopacity{0.553854}%
\pgfsetdash{}{0pt}%
\pgfpathmoveto{\pgfqpoint{1.011471in}{1.615500in}}%
\pgfpathcurveto{\pgfqpoint{1.019708in}{1.615500in}}{\pgfqpoint{1.027608in}{1.618773in}}{\pgfqpoint{1.033431in}{1.624596in}}%
\pgfpathcurveto{\pgfqpoint{1.039255in}{1.630420in}}{\pgfqpoint{1.042528in}{1.638320in}}{\pgfqpoint{1.042528in}{1.646557in}}%
\pgfpathcurveto{\pgfqpoint{1.042528in}{1.654793in}}{\pgfqpoint{1.039255in}{1.662693in}}{\pgfqpoint{1.033431in}{1.668517in}}%
\pgfpathcurveto{\pgfqpoint{1.027608in}{1.674341in}}{\pgfqpoint{1.019708in}{1.677613in}}{\pgfqpoint{1.011471in}{1.677613in}}%
\pgfpathcurveto{\pgfqpoint{1.003235in}{1.677613in}}{\pgfqpoint{0.995335in}{1.674341in}}{\pgfqpoint{0.989511in}{1.668517in}}%
\pgfpathcurveto{\pgfqpoint{0.983687in}{1.662693in}}{\pgfqpoint{0.980415in}{1.654793in}}{\pgfqpoint{0.980415in}{1.646557in}}%
\pgfpathcurveto{\pgfqpoint{0.980415in}{1.638320in}}{\pgfqpoint{0.983687in}{1.630420in}}{\pgfqpoint{0.989511in}{1.624596in}}%
\pgfpathcurveto{\pgfqpoint{0.995335in}{1.618773in}}{\pgfqpoint{1.003235in}{1.615500in}}{\pgfqpoint{1.011471in}{1.615500in}}%
\pgfpathclose%
\pgfusepath{stroke,fill}%
\end{pgfscope}%
\begin{pgfscope}%
\pgfpathrectangle{\pgfqpoint{0.100000in}{0.220728in}}{\pgfqpoint{3.696000in}{3.696000in}}%
\pgfusepath{clip}%
\pgfsetbuttcap%
\pgfsetroundjoin%
\definecolor{currentfill}{rgb}{0.121569,0.466667,0.705882}%
\pgfsetfillcolor{currentfill}%
\pgfsetfillopacity{0.554039}%
\pgfsetlinewidth{1.003750pt}%
\definecolor{currentstroke}{rgb}{0.121569,0.466667,0.705882}%
\pgfsetstrokecolor{currentstroke}%
\pgfsetstrokeopacity{0.554039}%
\pgfsetdash{}{0pt}%
\pgfpathmoveto{\pgfqpoint{2.838212in}{3.034944in}}%
\pgfpathcurveto{\pgfqpoint{2.846449in}{3.034944in}}{\pgfqpoint{2.854349in}{3.038216in}}{\pgfqpoint{2.860173in}{3.044040in}}%
\pgfpathcurveto{\pgfqpoint{2.865997in}{3.049864in}}{\pgfqpoint{2.869269in}{3.057764in}}{\pgfqpoint{2.869269in}{3.066000in}}%
\pgfpathcurveto{\pgfqpoint{2.869269in}{3.074237in}}{\pgfqpoint{2.865997in}{3.082137in}}{\pgfqpoint{2.860173in}{3.087961in}}%
\pgfpathcurveto{\pgfqpoint{2.854349in}{3.093784in}}{\pgfqpoint{2.846449in}{3.097057in}}{\pgfqpoint{2.838212in}{3.097057in}}%
\pgfpathcurveto{\pgfqpoint{2.829976in}{3.097057in}}{\pgfqpoint{2.822076in}{3.093784in}}{\pgfqpoint{2.816252in}{3.087961in}}%
\pgfpathcurveto{\pgfqpoint{2.810428in}{3.082137in}}{\pgfqpoint{2.807156in}{3.074237in}}{\pgfqpoint{2.807156in}{3.066000in}}%
\pgfpathcurveto{\pgfqpoint{2.807156in}{3.057764in}}{\pgfqpoint{2.810428in}{3.049864in}}{\pgfqpoint{2.816252in}{3.044040in}}%
\pgfpathcurveto{\pgfqpoint{2.822076in}{3.038216in}}{\pgfqpoint{2.829976in}{3.034944in}}{\pgfqpoint{2.838212in}{3.034944in}}%
\pgfpathclose%
\pgfusepath{stroke,fill}%
\end{pgfscope}%
\begin{pgfscope}%
\pgfpathrectangle{\pgfqpoint{0.100000in}{0.220728in}}{\pgfqpoint{3.696000in}{3.696000in}}%
\pgfusepath{clip}%
\pgfsetbuttcap%
\pgfsetroundjoin%
\definecolor{currentfill}{rgb}{0.121569,0.466667,0.705882}%
\pgfsetfillcolor{currentfill}%
\pgfsetfillopacity{0.554632}%
\pgfsetlinewidth{1.003750pt}%
\definecolor{currentstroke}{rgb}{0.121569,0.466667,0.705882}%
\pgfsetstrokecolor{currentstroke}%
\pgfsetstrokeopacity{0.554632}%
\pgfsetdash{}{0pt}%
\pgfpathmoveto{\pgfqpoint{2.840191in}{3.034718in}}%
\pgfpathcurveto{\pgfqpoint{2.848427in}{3.034718in}}{\pgfqpoint{2.856327in}{3.037990in}}{\pgfqpoint{2.862151in}{3.043814in}}%
\pgfpathcurveto{\pgfqpoint{2.867975in}{3.049638in}}{\pgfqpoint{2.871248in}{3.057538in}}{\pgfqpoint{2.871248in}{3.065774in}}%
\pgfpathcurveto{\pgfqpoint{2.871248in}{3.074011in}}{\pgfqpoint{2.867975in}{3.081911in}}{\pgfqpoint{2.862151in}{3.087735in}}%
\pgfpathcurveto{\pgfqpoint{2.856327in}{3.093558in}}{\pgfqpoint{2.848427in}{3.096831in}}{\pgfqpoint{2.840191in}{3.096831in}}%
\pgfpathcurveto{\pgfqpoint{2.831955in}{3.096831in}}{\pgfqpoint{2.824055in}{3.093558in}}{\pgfqpoint{2.818231in}{3.087735in}}%
\pgfpathcurveto{\pgfqpoint{2.812407in}{3.081911in}}{\pgfqpoint{2.809135in}{3.074011in}}{\pgfqpoint{2.809135in}{3.065774in}}%
\pgfpathcurveto{\pgfqpoint{2.809135in}{3.057538in}}{\pgfqpoint{2.812407in}{3.049638in}}{\pgfqpoint{2.818231in}{3.043814in}}%
\pgfpathcurveto{\pgfqpoint{2.824055in}{3.037990in}}{\pgfqpoint{2.831955in}{3.034718in}}{\pgfqpoint{2.840191in}{3.034718in}}%
\pgfpathclose%
\pgfusepath{stroke,fill}%
\end{pgfscope}%
\begin{pgfscope}%
\pgfpathrectangle{\pgfqpoint{0.100000in}{0.220728in}}{\pgfqpoint{3.696000in}{3.696000in}}%
\pgfusepath{clip}%
\pgfsetbuttcap%
\pgfsetroundjoin%
\definecolor{currentfill}{rgb}{0.121569,0.466667,0.705882}%
\pgfsetfillcolor{currentfill}%
\pgfsetfillopacity{0.555249}%
\pgfsetlinewidth{1.003750pt}%
\definecolor{currentstroke}{rgb}{0.121569,0.466667,0.705882}%
\pgfsetstrokecolor{currentstroke}%
\pgfsetstrokeopacity{0.555249}%
\pgfsetdash{}{0pt}%
\pgfpathmoveto{\pgfqpoint{1.008548in}{1.607463in}}%
\pgfpathcurveto{\pgfqpoint{1.016784in}{1.607463in}}{\pgfqpoint{1.024684in}{1.610736in}}{\pgfqpoint{1.030508in}{1.616560in}}%
\pgfpathcurveto{\pgfqpoint{1.036332in}{1.622383in}}{\pgfqpoint{1.039605in}{1.630284in}}{\pgfqpoint{1.039605in}{1.638520in}}%
\pgfpathcurveto{\pgfqpoint{1.039605in}{1.646756in}}{\pgfqpoint{1.036332in}{1.654656in}}{\pgfqpoint{1.030508in}{1.660480in}}%
\pgfpathcurveto{\pgfqpoint{1.024684in}{1.666304in}}{\pgfqpoint{1.016784in}{1.669576in}}{\pgfqpoint{1.008548in}{1.669576in}}%
\pgfpathcurveto{\pgfqpoint{1.000312in}{1.669576in}}{\pgfqpoint{0.992412in}{1.666304in}}{\pgfqpoint{0.986588in}{1.660480in}}%
\pgfpathcurveto{\pgfqpoint{0.980764in}{1.654656in}}{\pgfqpoint{0.977492in}{1.646756in}}{\pgfqpoint{0.977492in}{1.638520in}}%
\pgfpathcurveto{\pgfqpoint{0.977492in}{1.630284in}}{\pgfqpoint{0.980764in}{1.622383in}}{\pgfqpoint{0.986588in}{1.616560in}}%
\pgfpathcurveto{\pgfqpoint{0.992412in}{1.610736in}}{\pgfqpoint{1.000312in}{1.607463in}}{\pgfqpoint{1.008548in}{1.607463in}}%
\pgfpathclose%
\pgfusepath{stroke,fill}%
\end{pgfscope}%
\begin{pgfscope}%
\pgfpathrectangle{\pgfqpoint{0.100000in}{0.220728in}}{\pgfqpoint{3.696000in}{3.696000in}}%
\pgfusepath{clip}%
\pgfsetbuttcap%
\pgfsetroundjoin%
\definecolor{currentfill}{rgb}{0.121569,0.466667,0.705882}%
\pgfsetfillcolor{currentfill}%
\pgfsetfillopacity{0.555384}%
\pgfsetlinewidth{1.003750pt}%
\definecolor{currentstroke}{rgb}{0.121569,0.466667,0.705882}%
\pgfsetstrokecolor{currentstroke}%
\pgfsetstrokeopacity{0.555384}%
\pgfsetdash{}{0pt}%
\pgfpathmoveto{\pgfqpoint{2.843256in}{3.034046in}}%
\pgfpathcurveto{\pgfqpoint{2.851492in}{3.034046in}}{\pgfqpoint{2.859392in}{3.037318in}}{\pgfqpoint{2.865216in}{3.043142in}}%
\pgfpathcurveto{\pgfqpoint{2.871040in}{3.048966in}}{\pgfqpoint{2.874312in}{3.056866in}}{\pgfqpoint{2.874312in}{3.065102in}}%
\pgfpathcurveto{\pgfqpoint{2.874312in}{3.073338in}}{\pgfqpoint{2.871040in}{3.081238in}}{\pgfqpoint{2.865216in}{3.087062in}}%
\pgfpathcurveto{\pgfqpoint{2.859392in}{3.092886in}}{\pgfqpoint{2.851492in}{3.096159in}}{\pgfqpoint{2.843256in}{3.096159in}}%
\pgfpathcurveto{\pgfqpoint{2.835019in}{3.096159in}}{\pgfqpoint{2.827119in}{3.092886in}}{\pgfqpoint{2.821295in}{3.087062in}}%
\pgfpathcurveto{\pgfqpoint{2.815472in}{3.081238in}}{\pgfqpoint{2.812199in}{3.073338in}}{\pgfqpoint{2.812199in}{3.065102in}}%
\pgfpathcurveto{\pgfqpoint{2.812199in}{3.056866in}}{\pgfqpoint{2.815472in}{3.048966in}}{\pgfqpoint{2.821295in}{3.043142in}}%
\pgfpathcurveto{\pgfqpoint{2.827119in}{3.037318in}}{\pgfqpoint{2.835019in}{3.034046in}}{\pgfqpoint{2.843256in}{3.034046in}}%
\pgfpathclose%
\pgfusepath{stroke,fill}%
\end{pgfscope}%
\begin{pgfscope}%
\pgfpathrectangle{\pgfqpoint{0.100000in}{0.220728in}}{\pgfqpoint{3.696000in}{3.696000in}}%
\pgfusepath{clip}%
\pgfsetbuttcap%
\pgfsetroundjoin%
\definecolor{currentfill}{rgb}{0.121569,0.466667,0.705882}%
\pgfsetfillcolor{currentfill}%
\pgfsetfillopacity{0.556168}%
\pgfsetlinewidth{1.003750pt}%
\definecolor{currentstroke}{rgb}{0.121569,0.466667,0.705882}%
\pgfsetstrokecolor{currentstroke}%
\pgfsetstrokeopacity{0.556168}%
\pgfsetdash{}{0pt}%
\pgfpathmoveto{\pgfqpoint{2.846949in}{3.033168in}}%
\pgfpathcurveto{\pgfqpoint{2.855185in}{3.033168in}}{\pgfqpoint{2.863085in}{3.036440in}}{\pgfqpoint{2.868909in}{3.042264in}}%
\pgfpathcurveto{\pgfqpoint{2.874733in}{3.048088in}}{\pgfqpoint{2.878005in}{3.055988in}}{\pgfqpoint{2.878005in}{3.064224in}}%
\pgfpathcurveto{\pgfqpoint{2.878005in}{3.072460in}}{\pgfqpoint{2.874733in}{3.080360in}}{\pgfqpoint{2.868909in}{3.086184in}}%
\pgfpathcurveto{\pgfqpoint{2.863085in}{3.092008in}}{\pgfqpoint{2.855185in}{3.095281in}}{\pgfqpoint{2.846949in}{3.095281in}}%
\pgfpathcurveto{\pgfqpoint{2.838712in}{3.095281in}}{\pgfqpoint{2.830812in}{3.092008in}}{\pgfqpoint{2.824988in}{3.086184in}}%
\pgfpathcurveto{\pgfqpoint{2.819164in}{3.080360in}}{\pgfqpoint{2.815892in}{3.072460in}}{\pgfqpoint{2.815892in}{3.064224in}}%
\pgfpathcurveto{\pgfqpoint{2.815892in}{3.055988in}}{\pgfqpoint{2.819164in}{3.048088in}}{\pgfqpoint{2.824988in}{3.042264in}}%
\pgfpathcurveto{\pgfqpoint{2.830812in}{3.036440in}}{\pgfqpoint{2.838712in}{3.033168in}}{\pgfqpoint{2.846949in}{3.033168in}}%
\pgfpathclose%
\pgfusepath{stroke,fill}%
\end{pgfscope}%
\begin{pgfscope}%
\pgfpathrectangle{\pgfqpoint{0.100000in}{0.220728in}}{\pgfqpoint{3.696000in}{3.696000in}}%
\pgfusepath{clip}%
\pgfsetbuttcap%
\pgfsetroundjoin%
\definecolor{currentfill}{rgb}{0.121569,0.466667,0.705882}%
\pgfsetfillcolor{currentfill}%
\pgfsetfillopacity{0.556418}%
\pgfsetlinewidth{1.003750pt}%
\definecolor{currentstroke}{rgb}{0.121569,0.466667,0.705882}%
\pgfsetstrokecolor{currentstroke}%
\pgfsetstrokeopacity{0.556418}%
\pgfsetdash{}{0pt}%
\pgfpathmoveto{\pgfqpoint{1.004583in}{1.601144in}}%
\pgfpathcurveto{\pgfqpoint{1.012819in}{1.601144in}}{\pgfqpoint{1.020719in}{1.604417in}}{\pgfqpoint{1.026543in}{1.610241in}}%
\pgfpathcurveto{\pgfqpoint{1.032367in}{1.616064in}}{\pgfqpoint{1.035639in}{1.623965in}}{\pgfqpoint{1.035639in}{1.632201in}}%
\pgfpathcurveto{\pgfqpoint{1.035639in}{1.640437in}}{\pgfqpoint{1.032367in}{1.648337in}}{\pgfqpoint{1.026543in}{1.654161in}}%
\pgfpathcurveto{\pgfqpoint{1.020719in}{1.659985in}}{\pgfqpoint{1.012819in}{1.663257in}}{\pgfqpoint{1.004583in}{1.663257in}}%
\pgfpathcurveto{\pgfqpoint{0.996346in}{1.663257in}}{\pgfqpoint{0.988446in}{1.659985in}}{\pgfqpoint{0.982622in}{1.654161in}}%
\pgfpathcurveto{\pgfqpoint{0.976798in}{1.648337in}}{\pgfqpoint{0.973526in}{1.640437in}}{\pgfqpoint{0.973526in}{1.632201in}}%
\pgfpathcurveto{\pgfqpoint{0.973526in}{1.623965in}}{\pgfqpoint{0.976798in}{1.616064in}}{\pgfqpoint{0.982622in}{1.610241in}}%
\pgfpathcurveto{\pgfqpoint{0.988446in}{1.604417in}}{\pgfqpoint{0.996346in}{1.601144in}}{\pgfqpoint{1.004583in}{1.601144in}}%
\pgfpathclose%
\pgfusepath{stroke,fill}%
\end{pgfscope}%
\begin{pgfscope}%
\pgfpathrectangle{\pgfqpoint{0.100000in}{0.220728in}}{\pgfqpoint{3.696000in}{3.696000in}}%
\pgfusepath{clip}%
\pgfsetbuttcap%
\pgfsetroundjoin%
\definecolor{currentfill}{rgb}{0.121569,0.466667,0.705882}%
\pgfsetfillcolor{currentfill}%
\pgfsetfillopacity{0.556807}%
\pgfsetlinewidth{1.003750pt}%
\definecolor{currentstroke}{rgb}{0.121569,0.466667,0.705882}%
\pgfsetstrokecolor{currentstroke}%
\pgfsetstrokeopacity{0.556807}%
\pgfsetdash{}{0pt}%
\pgfpathmoveto{\pgfqpoint{2.848802in}{3.032979in}}%
\pgfpathcurveto{\pgfqpoint{2.857039in}{3.032979in}}{\pgfqpoint{2.864939in}{3.036252in}}{\pgfqpoint{2.870763in}{3.042076in}}%
\pgfpathcurveto{\pgfqpoint{2.876587in}{3.047900in}}{\pgfqpoint{2.879859in}{3.055800in}}{\pgfqpoint{2.879859in}{3.064036in}}%
\pgfpathcurveto{\pgfqpoint{2.879859in}{3.072272in}}{\pgfqpoint{2.876587in}{3.080172in}}{\pgfqpoint{2.870763in}{3.085996in}}%
\pgfpathcurveto{\pgfqpoint{2.864939in}{3.091820in}}{\pgfqpoint{2.857039in}{3.095092in}}{\pgfqpoint{2.848802in}{3.095092in}}%
\pgfpathcurveto{\pgfqpoint{2.840566in}{3.095092in}}{\pgfqpoint{2.832666in}{3.091820in}}{\pgfqpoint{2.826842in}{3.085996in}}%
\pgfpathcurveto{\pgfqpoint{2.821018in}{3.080172in}}{\pgfqpoint{2.817746in}{3.072272in}}{\pgfqpoint{2.817746in}{3.064036in}}%
\pgfpathcurveto{\pgfqpoint{2.817746in}{3.055800in}}{\pgfqpoint{2.821018in}{3.047900in}}{\pgfqpoint{2.826842in}{3.042076in}}%
\pgfpathcurveto{\pgfqpoint{2.832666in}{3.036252in}}{\pgfqpoint{2.840566in}{3.032979in}}{\pgfqpoint{2.848802in}{3.032979in}}%
\pgfpathclose%
\pgfusepath{stroke,fill}%
\end{pgfscope}%
\begin{pgfscope}%
\pgfpathrectangle{\pgfqpoint{0.100000in}{0.220728in}}{\pgfqpoint{3.696000in}{3.696000in}}%
\pgfusepath{clip}%
\pgfsetbuttcap%
\pgfsetroundjoin%
\definecolor{currentfill}{rgb}{0.121569,0.466667,0.705882}%
\pgfsetfillcolor{currentfill}%
\pgfsetfillopacity{0.557065}%
\pgfsetlinewidth{1.003750pt}%
\definecolor{currentstroke}{rgb}{0.121569,0.466667,0.705882}%
\pgfsetstrokecolor{currentstroke}%
\pgfsetstrokeopacity{0.557065}%
\pgfsetdash{}{0pt}%
\pgfpathmoveto{\pgfqpoint{2.849881in}{3.032669in}}%
\pgfpathcurveto{\pgfqpoint{2.858117in}{3.032669in}}{\pgfqpoint{2.866017in}{3.035941in}}{\pgfqpoint{2.871841in}{3.041765in}}%
\pgfpathcurveto{\pgfqpoint{2.877665in}{3.047589in}}{\pgfqpoint{2.880938in}{3.055489in}}{\pgfqpoint{2.880938in}{3.063726in}}%
\pgfpathcurveto{\pgfqpoint{2.880938in}{3.071962in}}{\pgfqpoint{2.877665in}{3.079862in}}{\pgfqpoint{2.871841in}{3.085686in}}%
\pgfpathcurveto{\pgfqpoint{2.866017in}{3.091510in}}{\pgfqpoint{2.858117in}{3.094782in}}{\pgfqpoint{2.849881in}{3.094782in}}%
\pgfpathcurveto{\pgfqpoint{2.841645in}{3.094782in}}{\pgfqpoint{2.833745in}{3.091510in}}{\pgfqpoint{2.827921in}{3.085686in}}%
\pgfpathcurveto{\pgfqpoint{2.822097in}{3.079862in}}{\pgfqpoint{2.818825in}{3.071962in}}{\pgfqpoint{2.818825in}{3.063726in}}%
\pgfpathcurveto{\pgfqpoint{2.818825in}{3.055489in}}{\pgfqpoint{2.822097in}{3.047589in}}{\pgfqpoint{2.827921in}{3.041765in}}%
\pgfpathcurveto{\pgfqpoint{2.833745in}{3.035941in}}{\pgfqpoint{2.841645in}{3.032669in}}{\pgfqpoint{2.849881in}{3.032669in}}%
\pgfpathclose%
\pgfusepath{stroke,fill}%
\end{pgfscope}%
\begin{pgfscope}%
\pgfpathrectangle{\pgfqpoint{0.100000in}{0.220728in}}{\pgfqpoint{3.696000in}{3.696000in}}%
\pgfusepath{clip}%
\pgfsetbuttcap%
\pgfsetroundjoin%
\definecolor{currentfill}{rgb}{0.121569,0.466667,0.705882}%
\pgfsetfillcolor{currentfill}%
\pgfsetfillopacity{0.557254}%
\pgfsetlinewidth{1.003750pt}%
\definecolor{currentstroke}{rgb}{0.121569,0.466667,0.705882}%
\pgfsetstrokecolor{currentstroke}%
\pgfsetstrokeopacity{0.557254}%
\pgfsetdash{}{0pt}%
\pgfpathmoveto{\pgfqpoint{2.852119in}{3.031882in}}%
\pgfpathcurveto{\pgfqpoint{2.860355in}{3.031882in}}{\pgfqpoint{2.868255in}{3.035155in}}{\pgfqpoint{2.874079in}{3.040979in}}%
\pgfpathcurveto{\pgfqpoint{2.879903in}{3.046803in}}{\pgfqpoint{2.883175in}{3.054703in}}{\pgfqpoint{2.883175in}{3.062939in}}%
\pgfpathcurveto{\pgfqpoint{2.883175in}{3.071175in}}{\pgfqpoint{2.879903in}{3.079075in}}{\pgfqpoint{2.874079in}{3.084899in}}%
\pgfpathcurveto{\pgfqpoint{2.868255in}{3.090723in}}{\pgfqpoint{2.860355in}{3.093995in}}{\pgfqpoint{2.852119in}{3.093995in}}%
\pgfpathcurveto{\pgfqpoint{2.843882in}{3.093995in}}{\pgfqpoint{2.835982in}{3.090723in}}{\pgfqpoint{2.830158in}{3.084899in}}%
\pgfpathcurveto{\pgfqpoint{2.824334in}{3.079075in}}{\pgfqpoint{2.821062in}{3.071175in}}{\pgfqpoint{2.821062in}{3.062939in}}%
\pgfpathcurveto{\pgfqpoint{2.821062in}{3.054703in}}{\pgfqpoint{2.824334in}{3.046803in}}{\pgfqpoint{2.830158in}{3.040979in}}%
\pgfpathcurveto{\pgfqpoint{2.835982in}{3.035155in}}{\pgfqpoint{2.843882in}{3.031882in}}{\pgfqpoint{2.852119in}{3.031882in}}%
\pgfpathclose%
\pgfusepath{stroke,fill}%
\end{pgfscope}%
\begin{pgfscope}%
\pgfpathrectangle{\pgfqpoint{0.100000in}{0.220728in}}{\pgfqpoint{3.696000in}{3.696000in}}%
\pgfusepath{clip}%
\pgfsetbuttcap%
\pgfsetroundjoin%
\definecolor{currentfill}{rgb}{0.121569,0.466667,0.705882}%
\pgfsetfillcolor{currentfill}%
\pgfsetfillopacity{0.557404}%
\pgfsetlinewidth{1.003750pt}%
\definecolor{currentstroke}{rgb}{0.121569,0.466667,0.705882}%
\pgfsetstrokecolor{currentstroke}%
\pgfsetstrokeopacity{0.557404}%
\pgfsetdash{}{0pt}%
\pgfpathmoveto{\pgfqpoint{1.001578in}{1.594961in}}%
\pgfpathcurveto{\pgfqpoint{1.009814in}{1.594961in}}{\pgfqpoint{1.017714in}{1.598233in}}{\pgfqpoint{1.023538in}{1.604057in}}%
\pgfpathcurveto{\pgfqpoint{1.029362in}{1.609881in}}{\pgfqpoint{1.032635in}{1.617781in}}{\pgfqpoint{1.032635in}{1.626017in}}%
\pgfpathcurveto{\pgfqpoint{1.032635in}{1.634254in}}{\pgfqpoint{1.029362in}{1.642154in}}{\pgfqpoint{1.023538in}{1.647978in}}%
\pgfpathcurveto{\pgfqpoint{1.017714in}{1.653802in}}{\pgfqpoint{1.009814in}{1.657074in}}{\pgfqpoint{1.001578in}{1.657074in}}%
\pgfpathcurveto{\pgfqpoint{0.993342in}{1.657074in}}{\pgfqpoint{0.985442in}{1.653802in}}{\pgfqpoint{0.979618in}{1.647978in}}%
\pgfpathcurveto{\pgfqpoint{0.973794in}{1.642154in}}{\pgfqpoint{0.970522in}{1.634254in}}{\pgfqpoint{0.970522in}{1.626017in}}%
\pgfpathcurveto{\pgfqpoint{0.970522in}{1.617781in}}{\pgfqpoint{0.973794in}{1.609881in}}{\pgfqpoint{0.979618in}{1.604057in}}%
\pgfpathcurveto{\pgfqpoint{0.985442in}{1.598233in}}{\pgfqpoint{0.993342in}{1.594961in}}{\pgfqpoint{1.001578in}{1.594961in}}%
\pgfpathclose%
\pgfusepath{stroke,fill}%
\end{pgfscope}%
\begin{pgfscope}%
\pgfpathrectangle{\pgfqpoint{0.100000in}{0.220728in}}{\pgfqpoint{3.696000in}{3.696000in}}%
\pgfusepath{clip}%
\pgfsetbuttcap%
\pgfsetroundjoin%
\definecolor{currentfill}{rgb}{0.121569,0.466667,0.705882}%
\pgfsetfillcolor{currentfill}%
\pgfsetfillopacity{0.557566}%
\pgfsetlinewidth{1.003750pt}%
\definecolor{currentstroke}{rgb}{0.121569,0.466667,0.705882}%
\pgfsetstrokecolor{currentstroke}%
\pgfsetstrokeopacity{0.557566}%
\pgfsetdash{}{0pt}%
\pgfpathmoveto{\pgfqpoint{2.853214in}{3.031740in}}%
\pgfpathcurveto{\pgfqpoint{2.861451in}{3.031740in}}{\pgfqpoint{2.869351in}{3.035012in}}{\pgfqpoint{2.875175in}{3.040836in}}%
\pgfpathcurveto{\pgfqpoint{2.880998in}{3.046660in}}{\pgfqpoint{2.884271in}{3.054560in}}{\pgfqpoint{2.884271in}{3.062796in}}%
\pgfpathcurveto{\pgfqpoint{2.884271in}{3.071033in}}{\pgfqpoint{2.880998in}{3.078933in}}{\pgfqpoint{2.875175in}{3.084757in}}%
\pgfpathcurveto{\pgfqpoint{2.869351in}{3.090580in}}{\pgfqpoint{2.861451in}{3.093853in}}{\pgfqpoint{2.853214in}{3.093853in}}%
\pgfpathcurveto{\pgfqpoint{2.844978in}{3.093853in}}{\pgfqpoint{2.837078in}{3.090580in}}{\pgfqpoint{2.831254in}{3.084757in}}%
\pgfpathcurveto{\pgfqpoint{2.825430in}{3.078933in}}{\pgfqpoint{2.822158in}{3.071033in}}{\pgfqpoint{2.822158in}{3.062796in}}%
\pgfpathcurveto{\pgfqpoint{2.822158in}{3.054560in}}{\pgfqpoint{2.825430in}{3.046660in}}{\pgfqpoint{2.831254in}{3.040836in}}%
\pgfpathcurveto{\pgfqpoint{2.837078in}{3.035012in}}{\pgfqpoint{2.844978in}{3.031740in}}{\pgfqpoint{2.853214in}{3.031740in}}%
\pgfpathclose%
\pgfusepath{stroke,fill}%
\end{pgfscope}%
\begin{pgfscope}%
\pgfpathrectangle{\pgfqpoint{0.100000in}{0.220728in}}{\pgfqpoint{3.696000in}{3.696000in}}%
\pgfusepath{clip}%
\pgfsetbuttcap%
\pgfsetroundjoin%
\definecolor{currentfill}{rgb}{0.121569,0.466667,0.705882}%
\pgfsetfillcolor{currentfill}%
\pgfsetfillopacity{0.558229}%
\pgfsetlinewidth{1.003750pt}%
\definecolor{currentstroke}{rgb}{0.121569,0.466667,0.705882}%
\pgfsetstrokecolor{currentstroke}%
\pgfsetstrokeopacity{0.558229}%
\pgfsetdash{}{0pt}%
\pgfpathmoveto{\pgfqpoint{2.855693in}{3.030141in}}%
\pgfpathcurveto{\pgfqpoint{2.863930in}{3.030141in}}{\pgfqpoint{2.871830in}{3.033413in}}{\pgfqpoint{2.877654in}{3.039237in}}%
\pgfpathcurveto{\pgfqpoint{2.883477in}{3.045061in}}{\pgfqpoint{2.886750in}{3.052961in}}{\pgfqpoint{2.886750in}{3.061198in}}%
\pgfpathcurveto{\pgfqpoint{2.886750in}{3.069434in}}{\pgfqpoint{2.883477in}{3.077334in}}{\pgfqpoint{2.877654in}{3.083158in}}%
\pgfpathcurveto{\pgfqpoint{2.871830in}{3.088982in}}{\pgfqpoint{2.863930in}{3.092254in}}{\pgfqpoint{2.855693in}{3.092254in}}%
\pgfpathcurveto{\pgfqpoint{2.847457in}{3.092254in}}{\pgfqpoint{2.839557in}{3.088982in}}{\pgfqpoint{2.833733in}{3.083158in}}%
\pgfpathcurveto{\pgfqpoint{2.827909in}{3.077334in}}{\pgfqpoint{2.824637in}{3.069434in}}{\pgfqpoint{2.824637in}{3.061198in}}%
\pgfpathcurveto{\pgfqpoint{2.824637in}{3.052961in}}{\pgfqpoint{2.827909in}{3.045061in}}{\pgfqpoint{2.833733in}{3.039237in}}%
\pgfpathcurveto{\pgfqpoint{2.839557in}{3.033413in}}{\pgfqpoint{2.847457in}{3.030141in}}{\pgfqpoint{2.855693in}{3.030141in}}%
\pgfpathclose%
\pgfusepath{stroke,fill}%
\end{pgfscope}%
\begin{pgfscope}%
\pgfpathrectangle{\pgfqpoint{0.100000in}{0.220728in}}{\pgfqpoint{3.696000in}{3.696000in}}%
\pgfusepath{clip}%
\pgfsetbuttcap%
\pgfsetroundjoin%
\definecolor{currentfill}{rgb}{0.121569,0.466667,0.705882}%
\pgfsetfillcolor{currentfill}%
\pgfsetfillopacity{0.558273}%
\pgfsetlinewidth{1.003750pt}%
\definecolor{currentstroke}{rgb}{0.121569,0.466667,0.705882}%
\pgfsetstrokecolor{currentstroke}%
\pgfsetstrokeopacity{0.558273}%
\pgfsetdash{}{0pt}%
\pgfpathmoveto{\pgfqpoint{0.999133in}{1.589776in}}%
\pgfpathcurveto{\pgfqpoint{1.007370in}{1.589776in}}{\pgfqpoint{1.015270in}{1.593048in}}{\pgfqpoint{1.021094in}{1.598872in}}%
\pgfpathcurveto{\pgfqpoint{1.026918in}{1.604696in}}{\pgfqpoint{1.030190in}{1.612596in}}{\pgfqpoint{1.030190in}{1.620832in}}%
\pgfpathcurveto{\pgfqpoint{1.030190in}{1.629068in}}{\pgfqpoint{1.026918in}{1.636968in}}{\pgfqpoint{1.021094in}{1.642792in}}%
\pgfpathcurveto{\pgfqpoint{1.015270in}{1.648616in}}{\pgfqpoint{1.007370in}{1.651889in}}{\pgfqpoint{0.999133in}{1.651889in}}%
\pgfpathcurveto{\pgfqpoint{0.990897in}{1.651889in}}{\pgfqpoint{0.982997in}{1.648616in}}{\pgfqpoint{0.977173in}{1.642792in}}%
\pgfpathcurveto{\pgfqpoint{0.971349in}{1.636968in}}{\pgfqpoint{0.968077in}{1.629068in}}{\pgfqpoint{0.968077in}{1.620832in}}%
\pgfpathcurveto{\pgfqpoint{0.968077in}{1.612596in}}{\pgfqpoint{0.971349in}{1.604696in}}{\pgfqpoint{0.977173in}{1.598872in}}%
\pgfpathcurveto{\pgfqpoint{0.982997in}{1.593048in}}{\pgfqpoint{0.990897in}{1.589776in}}{\pgfqpoint{0.999133in}{1.589776in}}%
\pgfpathclose%
\pgfusepath{stroke,fill}%
\end{pgfscope}%
\begin{pgfscope}%
\pgfpathrectangle{\pgfqpoint{0.100000in}{0.220728in}}{\pgfqpoint{3.696000in}{3.696000in}}%
\pgfusepath{clip}%
\pgfsetbuttcap%
\pgfsetroundjoin%
\definecolor{currentfill}{rgb}{0.121569,0.466667,0.705882}%
\pgfsetfillcolor{currentfill}%
\pgfsetfillopacity{0.559145}%
\pgfsetlinewidth{1.003750pt}%
\definecolor{currentstroke}{rgb}{0.121569,0.466667,0.705882}%
\pgfsetstrokecolor{currentstroke}%
\pgfsetstrokeopacity{0.559145}%
\pgfsetdash{}{0pt}%
\pgfpathmoveto{\pgfqpoint{2.858942in}{3.029670in}}%
\pgfpathcurveto{\pgfqpoint{2.867178in}{3.029670in}}{\pgfqpoint{2.875078in}{3.032942in}}{\pgfqpoint{2.880902in}{3.038766in}}%
\pgfpathcurveto{\pgfqpoint{2.886726in}{3.044590in}}{\pgfqpoint{2.889999in}{3.052490in}}{\pgfqpoint{2.889999in}{3.060727in}}%
\pgfpathcurveto{\pgfqpoint{2.889999in}{3.068963in}}{\pgfqpoint{2.886726in}{3.076863in}}{\pgfqpoint{2.880902in}{3.082687in}}%
\pgfpathcurveto{\pgfqpoint{2.875078in}{3.088511in}}{\pgfqpoint{2.867178in}{3.091783in}}{\pgfqpoint{2.858942in}{3.091783in}}%
\pgfpathcurveto{\pgfqpoint{2.850706in}{3.091783in}}{\pgfqpoint{2.842806in}{3.088511in}}{\pgfqpoint{2.836982in}{3.082687in}}%
\pgfpathcurveto{\pgfqpoint{2.831158in}{3.076863in}}{\pgfqpoint{2.827886in}{3.068963in}}{\pgfqpoint{2.827886in}{3.060727in}}%
\pgfpathcurveto{\pgfqpoint{2.827886in}{3.052490in}}{\pgfqpoint{2.831158in}{3.044590in}}{\pgfqpoint{2.836982in}{3.038766in}}%
\pgfpathcurveto{\pgfqpoint{2.842806in}{3.032942in}}{\pgfqpoint{2.850706in}{3.029670in}}{\pgfqpoint{2.858942in}{3.029670in}}%
\pgfpathclose%
\pgfusepath{stroke,fill}%
\end{pgfscope}%
\begin{pgfscope}%
\pgfpathrectangle{\pgfqpoint{0.100000in}{0.220728in}}{\pgfqpoint{3.696000in}{3.696000in}}%
\pgfusepath{clip}%
\pgfsetbuttcap%
\pgfsetroundjoin%
\definecolor{currentfill}{rgb}{0.121569,0.466667,0.705882}%
\pgfsetfillcolor{currentfill}%
\pgfsetfillopacity{0.559785}%
\pgfsetlinewidth{1.003750pt}%
\definecolor{currentstroke}{rgb}{0.121569,0.466667,0.705882}%
\pgfsetstrokecolor{currentstroke}%
\pgfsetstrokeopacity{0.559785}%
\pgfsetdash{}{0pt}%
\pgfpathmoveto{\pgfqpoint{0.993781in}{1.581198in}}%
\pgfpathcurveto{\pgfqpoint{1.002017in}{1.581198in}}{\pgfqpoint{1.009917in}{1.584470in}}{\pgfqpoint{1.015741in}{1.590294in}}%
\pgfpathcurveto{\pgfqpoint{1.021565in}{1.596118in}}{\pgfqpoint{1.024837in}{1.604018in}}{\pgfqpoint{1.024837in}{1.612254in}}%
\pgfpathcurveto{\pgfqpoint{1.024837in}{1.620490in}}{\pgfqpoint{1.021565in}{1.628390in}}{\pgfqpoint{1.015741in}{1.634214in}}%
\pgfpathcurveto{\pgfqpoint{1.009917in}{1.640038in}}{\pgfqpoint{1.002017in}{1.643311in}}{\pgfqpoint{0.993781in}{1.643311in}}%
\pgfpathcurveto{\pgfqpoint{0.985545in}{1.643311in}}{\pgfqpoint{0.977645in}{1.640038in}}{\pgfqpoint{0.971821in}{1.634214in}}%
\pgfpathcurveto{\pgfqpoint{0.965997in}{1.628390in}}{\pgfqpoint{0.962724in}{1.620490in}}{\pgfqpoint{0.962724in}{1.612254in}}%
\pgfpathcurveto{\pgfqpoint{0.962724in}{1.604018in}}{\pgfqpoint{0.965997in}{1.596118in}}{\pgfqpoint{0.971821in}{1.590294in}}%
\pgfpathcurveto{\pgfqpoint{0.977645in}{1.584470in}}{\pgfqpoint{0.985545in}{1.581198in}}{\pgfqpoint{0.993781in}{1.581198in}}%
\pgfpathclose%
\pgfusepath{stroke,fill}%
\end{pgfscope}%
\begin{pgfscope}%
\pgfpathrectangle{\pgfqpoint{0.100000in}{0.220728in}}{\pgfqpoint{3.696000in}{3.696000in}}%
\pgfusepath{clip}%
\pgfsetbuttcap%
\pgfsetroundjoin%
\definecolor{currentfill}{rgb}{0.121569,0.466667,0.705882}%
\pgfsetfillcolor{currentfill}%
\pgfsetfillopacity{0.560271}%
\pgfsetlinewidth{1.003750pt}%
\definecolor{currentstroke}{rgb}{0.121569,0.466667,0.705882}%
\pgfsetstrokecolor{currentstroke}%
\pgfsetstrokeopacity{0.560271}%
\pgfsetdash{}{0pt}%
\pgfpathmoveto{\pgfqpoint{2.863140in}{3.028753in}}%
\pgfpathcurveto{\pgfqpoint{2.871376in}{3.028753in}}{\pgfqpoint{2.879276in}{3.032025in}}{\pgfqpoint{2.885100in}{3.037849in}}%
\pgfpathcurveto{\pgfqpoint{2.890924in}{3.043673in}}{\pgfqpoint{2.894196in}{3.051573in}}{\pgfqpoint{2.894196in}{3.059809in}}%
\pgfpathcurveto{\pgfqpoint{2.894196in}{3.068045in}}{\pgfqpoint{2.890924in}{3.075945in}}{\pgfqpoint{2.885100in}{3.081769in}}%
\pgfpathcurveto{\pgfqpoint{2.879276in}{3.087593in}}{\pgfqpoint{2.871376in}{3.090866in}}{\pgfqpoint{2.863140in}{3.090866in}}%
\pgfpathcurveto{\pgfqpoint{2.854904in}{3.090866in}}{\pgfqpoint{2.847004in}{3.087593in}}{\pgfqpoint{2.841180in}{3.081769in}}%
\pgfpathcurveto{\pgfqpoint{2.835356in}{3.075945in}}{\pgfqpoint{2.832083in}{3.068045in}}{\pgfqpoint{2.832083in}{3.059809in}}%
\pgfpathcurveto{\pgfqpoint{2.832083in}{3.051573in}}{\pgfqpoint{2.835356in}{3.043673in}}{\pgfqpoint{2.841180in}{3.037849in}}%
\pgfpathcurveto{\pgfqpoint{2.847004in}{3.032025in}}{\pgfqpoint{2.854904in}{3.028753in}}{\pgfqpoint{2.863140in}{3.028753in}}%
\pgfpathclose%
\pgfusepath{stroke,fill}%
\end{pgfscope}%
\begin{pgfscope}%
\pgfpathrectangle{\pgfqpoint{0.100000in}{0.220728in}}{\pgfqpoint{3.696000in}{3.696000in}}%
\pgfusepath{clip}%
\pgfsetbuttcap%
\pgfsetroundjoin%
\definecolor{currentfill}{rgb}{0.121569,0.466667,0.705882}%
\pgfsetfillcolor{currentfill}%
\pgfsetfillopacity{0.561177}%
\pgfsetlinewidth{1.003750pt}%
\definecolor{currentstroke}{rgb}{0.121569,0.466667,0.705882}%
\pgfsetstrokecolor{currentstroke}%
\pgfsetstrokeopacity{0.561177}%
\pgfsetdash{}{0pt}%
\pgfpathmoveto{\pgfqpoint{2.868846in}{3.027375in}}%
\pgfpathcurveto{\pgfqpoint{2.877083in}{3.027375in}}{\pgfqpoint{2.884983in}{3.030647in}}{\pgfqpoint{2.890807in}{3.036471in}}%
\pgfpathcurveto{\pgfqpoint{2.896631in}{3.042295in}}{\pgfqpoint{2.899903in}{3.050195in}}{\pgfqpoint{2.899903in}{3.058431in}}%
\pgfpathcurveto{\pgfqpoint{2.899903in}{3.066668in}}{\pgfqpoint{2.896631in}{3.074568in}}{\pgfqpoint{2.890807in}{3.080392in}}%
\pgfpathcurveto{\pgfqpoint{2.884983in}{3.086216in}}{\pgfqpoint{2.877083in}{3.089488in}}{\pgfqpoint{2.868846in}{3.089488in}}%
\pgfpathcurveto{\pgfqpoint{2.860610in}{3.089488in}}{\pgfqpoint{2.852710in}{3.086216in}}{\pgfqpoint{2.846886in}{3.080392in}}%
\pgfpathcurveto{\pgfqpoint{2.841062in}{3.074568in}}{\pgfqpoint{2.837790in}{3.066668in}}{\pgfqpoint{2.837790in}{3.058431in}}%
\pgfpathcurveto{\pgfqpoint{2.837790in}{3.050195in}}{\pgfqpoint{2.841062in}{3.042295in}}{\pgfqpoint{2.846886in}{3.036471in}}%
\pgfpathcurveto{\pgfqpoint{2.852710in}{3.030647in}}{\pgfqpoint{2.860610in}{3.027375in}}{\pgfqpoint{2.868846in}{3.027375in}}%
\pgfpathclose%
\pgfusepath{stroke,fill}%
\end{pgfscope}%
\begin{pgfscope}%
\pgfpathrectangle{\pgfqpoint{0.100000in}{0.220728in}}{\pgfqpoint{3.696000in}{3.696000in}}%
\pgfusepath{clip}%
\pgfsetbuttcap%
\pgfsetroundjoin%
\definecolor{currentfill}{rgb}{0.121569,0.466667,0.705882}%
\pgfsetfillcolor{currentfill}%
\pgfsetfillopacity{0.562732}%
\pgfsetlinewidth{1.003750pt}%
\definecolor{currentstroke}{rgb}{0.121569,0.466667,0.705882}%
\pgfsetstrokecolor{currentstroke}%
\pgfsetstrokeopacity{0.562732}%
\pgfsetdash{}{0pt}%
\pgfpathmoveto{\pgfqpoint{0.986548in}{1.563496in}}%
\pgfpathcurveto{\pgfqpoint{0.994784in}{1.563496in}}{\pgfqpoint{1.002684in}{1.566768in}}{\pgfqpoint{1.008508in}{1.572592in}}%
\pgfpathcurveto{\pgfqpoint{1.014332in}{1.578416in}}{\pgfqpoint{1.017604in}{1.586316in}}{\pgfqpoint{1.017604in}{1.594553in}}%
\pgfpathcurveto{\pgfqpoint{1.017604in}{1.602789in}}{\pgfqpoint{1.014332in}{1.610689in}}{\pgfqpoint{1.008508in}{1.616513in}}%
\pgfpathcurveto{\pgfqpoint{1.002684in}{1.622337in}}{\pgfqpoint{0.994784in}{1.625609in}}{\pgfqpoint{0.986548in}{1.625609in}}%
\pgfpathcurveto{\pgfqpoint{0.978311in}{1.625609in}}{\pgfqpoint{0.970411in}{1.622337in}}{\pgfqpoint{0.964587in}{1.616513in}}%
\pgfpathcurveto{\pgfqpoint{0.958763in}{1.610689in}}{\pgfqpoint{0.955491in}{1.602789in}}{\pgfqpoint{0.955491in}{1.594553in}}%
\pgfpathcurveto{\pgfqpoint{0.955491in}{1.586316in}}{\pgfqpoint{0.958763in}{1.578416in}}{\pgfqpoint{0.964587in}{1.572592in}}%
\pgfpathcurveto{\pgfqpoint{0.970411in}{1.566768in}}{\pgfqpoint{0.978311in}{1.563496in}}{\pgfqpoint{0.986548in}{1.563496in}}%
\pgfpathclose%
\pgfusepath{stroke,fill}%
\end{pgfscope}%
\begin{pgfscope}%
\pgfpathrectangle{\pgfqpoint{0.100000in}{0.220728in}}{\pgfqpoint{3.696000in}{3.696000in}}%
\pgfusepath{clip}%
\pgfsetbuttcap%
\pgfsetroundjoin%
\definecolor{currentfill}{rgb}{0.121569,0.466667,0.705882}%
\pgfsetfillcolor{currentfill}%
\pgfsetfillopacity{0.562767}%
\pgfsetlinewidth{1.003750pt}%
\definecolor{currentstroke}{rgb}{0.121569,0.466667,0.705882}%
\pgfsetstrokecolor{currentstroke}%
\pgfsetstrokeopacity{0.562767}%
\pgfsetdash{}{0pt}%
\pgfpathmoveto{\pgfqpoint{2.876266in}{3.025828in}}%
\pgfpathcurveto{\pgfqpoint{2.884502in}{3.025828in}}{\pgfqpoint{2.892402in}{3.029100in}}{\pgfqpoint{2.898226in}{3.034924in}}%
\pgfpathcurveto{\pgfqpoint{2.904050in}{3.040748in}}{\pgfqpoint{2.907322in}{3.048648in}}{\pgfqpoint{2.907322in}{3.056885in}}%
\pgfpathcurveto{\pgfqpoint{2.907322in}{3.065121in}}{\pgfqpoint{2.904050in}{3.073021in}}{\pgfqpoint{2.898226in}{3.078845in}}%
\pgfpathcurveto{\pgfqpoint{2.892402in}{3.084669in}}{\pgfqpoint{2.884502in}{3.087941in}}{\pgfqpoint{2.876266in}{3.087941in}}%
\pgfpathcurveto{\pgfqpoint{2.868030in}{3.087941in}}{\pgfqpoint{2.860130in}{3.084669in}}{\pgfqpoint{2.854306in}{3.078845in}}%
\pgfpathcurveto{\pgfqpoint{2.848482in}{3.073021in}}{\pgfqpoint{2.845209in}{3.065121in}}{\pgfqpoint{2.845209in}{3.056885in}}%
\pgfpathcurveto{\pgfqpoint{2.845209in}{3.048648in}}{\pgfqpoint{2.848482in}{3.040748in}}{\pgfqpoint{2.854306in}{3.034924in}}%
\pgfpathcurveto{\pgfqpoint{2.860130in}{3.029100in}}{\pgfqpoint{2.868030in}{3.025828in}}{\pgfqpoint{2.876266in}{3.025828in}}%
\pgfpathclose%
\pgfusepath{stroke,fill}%
\end{pgfscope}%
\begin{pgfscope}%
\pgfpathrectangle{\pgfqpoint{0.100000in}{0.220728in}}{\pgfqpoint{3.696000in}{3.696000in}}%
\pgfusepath{clip}%
\pgfsetbuttcap%
\pgfsetroundjoin%
\definecolor{currentfill}{rgb}{0.121569,0.466667,0.705882}%
\pgfsetfillcolor{currentfill}%
\pgfsetfillopacity{0.564638}%
\pgfsetlinewidth{1.003750pt}%
\definecolor{currentstroke}{rgb}{0.121569,0.466667,0.705882}%
\pgfsetstrokecolor{currentstroke}%
\pgfsetstrokeopacity{0.564638}%
\pgfsetdash{}{0pt}%
\pgfpathmoveto{\pgfqpoint{2.884276in}{3.024301in}}%
\pgfpathcurveto{\pgfqpoint{2.892512in}{3.024301in}}{\pgfqpoint{2.900412in}{3.027573in}}{\pgfqpoint{2.906236in}{3.033397in}}%
\pgfpathcurveto{\pgfqpoint{2.912060in}{3.039221in}}{\pgfqpoint{2.915332in}{3.047121in}}{\pgfqpoint{2.915332in}{3.055357in}}%
\pgfpathcurveto{\pgfqpoint{2.915332in}{3.063593in}}{\pgfqpoint{2.912060in}{3.071494in}}{\pgfqpoint{2.906236in}{3.077317in}}%
\pgfpathcurveto{\pgfqpoint{2.900412in}{3.083141in}}{\pgfqpoint{2.892512in}{3.086414in}}{\pgfqpoint{2.884276in}{3.086414in}}%
\pgfpathcurveto{\pgfqpoint{2.876039in}{3.086414in}}{\pgfqpoint{2.868139in}{3.083141in}}{\pgfqpoint{2.862315in}{3.077317in}}%
\pgfpathcurveto{\pgfqpoint{2.856491in}{3.071494in}}{\pgfqpoint{2.853219in}{3.063593in}}{\pgfqpoint{2.853219in}{3.055357in}}%
\pgfpathcurveto{\pgfqpoint{2.853219in}{3.047121in}}{\pgfqpoint{2.856491in}{3.039221in}}{\pgfqpoint{2.862315in}{3.033397in}}%
\pgfpathcurveto{\pgfqpoint{2.868139in}{3.027573in}}{\pgfqpoint{2.876039in}{3.024301in}}{\pgfqpoint{2.884276in}{3.024301in}}%
\pgfpathclose%
\pgfusepath{stroke,fill}%
\end{pgfscope}%
\begin{pgfscope}%
\pgfpathrectangle{\pgfqpoint{0.100000in}{0.220728in}}{\pgfqpoint{3.696000in}{3.696000in}}%
\pgfusepath{clip}%
\pgfsetbuttcap%
\pgfsetroundjoin%
\definecolor{currentfill}{rgb}{0.121569,0.466667,0.705882}%
\pgfsetfillcolor{currentfill}%
\pgfsetfillopacity{0.564665}%
\pgfsetlinewidth{1.003750pt}%
\definecolor{currentstroke}{rgb}{0.121569,0.466667,0.705882}%
\pgfsetstrokecolor{currentstroke}%
\pgfsetstrokeopacity{0.564665}%
\pgfsetdash{}{0pt}%
\pgfpathmoveto{\pgfqpoint{0.977185in}{1.549449in}}%
\pgfpathcurveto{\pgfqpoint{0.985421in}{1.549449in}}{\pgfqpoint{0.993321in}{1.552721in}}{\pgfqpoint{0.999145in}{1.558545in}}%
\pgfpathcurveto{\pgfqpoint{1.004969in}{1.564369in}}{\pgfqpoint{1.008241in}{1.572269in}}{\pgfqpoint{1.008241in}{1.580505in}}%
\pgfpathcurveto{\pgfqpoint{1.008241in}{1.588741in}}{\pgfqpoint{1.004969in}{1.596641in}}{\pgfqpoint{0.999145in}{1.602465in}}%
\pgfpathcurveto{\pgfqpoint{0.993321in}{1.608289in}}{\pgfqpoint{0.985421in}{1.611562in}}{\pgfqpoint{0.977185in}{1.611562in}}%
\pgfpathcurveto{\pgfqpoint{0.968948in}{1.611562in}}{\pgfqpoint{0.961048in}{1.608289in}}{\pgfqpoint{0.955224in}{1.602465in}}%
\pgfpathcurveto{\pgfqpoint{0.949401in}{1.596641in}}{\pgfqpoint{0.946128in}{1.588741in}}{\pgfqpoint{0.946128in}{1.580505in}}%
\pgfpathcurveto{\pgfqpoint{0.946128in}{1.572269in}}{\pgfqpoint{0.949401in}{1.564369in}}{\pgfqpoint{0.955224in}{1.558545in}}%
\pgfpathcurveto{\pgfqpoint{0.961048in}{1.552721in}}{\pgfqpoint{0.968948in}{1.549449in}}{\pgfqpoint{0.977185in}{1.549449in}}%
\pgfpathclose%
\pgfusepath{stroke,fill}%
\end{pgfscope}%
\begin{pgfscope}%
\pgfpathrectangle{\pgfqpoint{0.100000in}{0.220728in}}{\pgfqpoint{3.696000in}{3.696000in}}%
\pgfusepath{clip}%
\pgfsetbuttcap%
\pgfsetroundjoin%
\definecolor{currentfill}{rgb}{0.121569,0.466667,0.705882}%
\pgfsetfillcolor{currentfill}%
\pgfsetfillopacity{0.565166}%
\pgfsetlinewidth{1.003750pt}%
\definecolor{currentstroke}{rgb}{0.121569,0.466667,0.705882}%
\pgfsetstrokecolor{currentstroke}%
\pgfsetstrokeopacity{0.565166}%
\pgfsetdash{}{0pt}%
\pgfpathmoveto{\pgfqpoint{2.889203in}{3.023847in}}%
\pgfpathcurveto{\pgfqpoint{2.897439in}{3.023847in}}{\pgfqpoint{2.905339in}{3.027119in}}{\pgfqpoint{2.911163in}{3.032943in}}%
\pgfpathcurveto{\pgfqpoint{2.916987in}{3.038767in}}{\pgfqpoint{2.920259in}{3.046667in}}{\pgfqpoint{2.920259in}{3.054903in}}%
\pgfpathcurveto{\pgfqpoint{2.920259in}{3.063139in}}{\pgfqpoint{2.916987in}{3.071039in}}{\pgfqpoint{2.911163in}{3.076863in}}%
\pgfpathcurveto{\pgfqpoint{2.905339in}{3.082687in}}{\pgfqpoint{2.897439in}{3.085960in}}{\pgfqpoint{2.889203in}{3.085960in}}%
\pgfpathcurveto{\pgfqpoint{2.880967in}{3.085960in}}{\pgfqpoint{2.873067in}{3.082687in}}{\pgfqpoint{2.867243in}{3.076863in}}%
\pgfpathcurveto{\pgfqpoint{2.861419in}{3.071039in}}{\pgfqpoint{2.858146in}{3.063139in}}{\pgfqpoint{2.858146in}{3.054903in}}%
\pgfpathcurveto{\pgfqpoint{2.858146in}{3.046667in}}{\pgfqpoint{2.861419in}{3.038767in}}{\pgfqpoint{2.867243in}{3.032943in}}%
\pgfpathcurveto{\pgfqpoint{2.873067in}{3.027119in}}{\pgfqpoint{2.880967in}{3.023847in}}{\pgfqpoint{2.889203in}{3.023847in}}%
\pgfpathclose%
\pgfusepath{stroke,fill}%
\end{pgfscope}%
\begin{pgfscope}%
\pgfpathrectangle{\pgfqpoint{0.100000in}{0.220728in}}{\pgfqpoint{3.696000in}{3.696000in}}%
\pgfusepath{clip}%
\pgfsetbuttcap%
\pgfsetroundjoin%
\definecolor{currentfill}{rgb}{0.121569,0.466667,0.705882}%
\pgfsetfillcolor{currentfill}%
\pgfsetfillopacity{0.565870}%
\pgfsetlinewidth{1.003750pt}%
\definecolor{currentstroke}{rgb}{0.121569,0.466667,0.705882}%
\pgfsetstrokecolor{currentstroke}%
\pgfsetstrokeopacity{0.565870}%
\pgfsetdash{}{0pt}%
\pgfpathmoveto{\pgfqpoint{2.891610in}{3.023867in}}%
\pgfpathcurveto{\pgfqpoint{2.899846in}{3.023867in}}{\pgfqpoint{2.907746in}{3.027140in}}{\pgfqpoint{2.913570in}{3.032964in}}%
\pgfpathcurveto{\pgfqpoint{2.919394in}{3.038787in}}{\pgfqpoint{2.922666in}{3.046688in}}{\pgfqpoint{2.922666in}{3.054924in}}%
\pgfpathcurveto{\pgfqpoint{2.922666in}{3.063160in}}{\pgfqpoint{2.919394in}{3.071060in}}{\pgfqpoint{2.913570in}{3.076884in}}%
\pgfpathcurveto{\pgfqpoint{2.907746in}{3.082708in}}{\pgfqpoint{2.899846in}{3.085980in}}{\pgfqpoint{2.891610in}{3.085980in}}%
\pgfpathcurveto{\pgfqpoint{2.883373in}{3.085980in}}{\pgfqpoint{2.875473in}{3.082708in}}{\pgfqpoint{2.869649in}{3.076884in}}%
\pgfpathcurveto{\pgfqpoint{2.863825in}{3.071060in}}{\pgfqpoint{2.860553in}{3.063160in}}{\pgfqpoint{2.860553in}{3.054924in}}%
\pgfpathcurveto{\pgfqpoint{2.860553in}{3.046688in}}{\pgfqpoint{2.863825in}{3.038787in}}{\pgfqpoint{2.869649in}{3.032964in}}%
\pgfpathcurveto{\pgfqpoint{2.875473in}{3.027140in}}{\pgfqpoint{2.883373in}{3.023867in}}{\pgfqpoint{2.891610in}{3.023867in}}%
\pgfpathclose%
\pgfusepath{stroke,fill}%
\end{pgfscope}%
\begin{pgfscope}%
\pgfpathrectangle{\pgfqpoint{0.100000in}{0.220728in}}{\pgfqpoint{3.696000in}{3.696000in}}%
\pgfusepath{clip}%
\pgfsetbuttcap%
\pgfsetroundjoin%
\definecolor{currentfill}{rgb}{0.121569,0.466667,0.705882}%
\pgfsetfillcolor{currentfill}%
\pgfsetfillopacity{0.567072}%
\pgfsetlinewidth{1.003750pt}%
\definecolor{currentstroke}{rgb}{0.121569,0.466667,0.705882}%
\pgfsetstrokecolor{currentstroke}%
\pgfsetstrokeopacity{0.567072}%
\pgfsetdash{}{0pt}%
\pgfpathmoveto{\pgfqpoint{0.971987in}{1.535077in}}%
\pgfpathcurveto{\pgfqpoint{0.980223in}{1.535077in}}{\pgfqpoint{0.988124in}{1.538349in}}{\pgfqpoint{0.993947in}{1.544173in}}%
\pgfpathcurveto{\pgfqpoint{0.999771in}{1.549997in}}{\pgfqpoint{1.003044in}{1.557897in}}{\pgfqpoint{1.003044in}{1.566133in}}%
\pgfpathcurveto{\pgfqpoint{1.003044in}{1.574370in}}{\pgfqpoint{0.999771in}{1.582270in}}{\pgfqpoint{0.993947in}{1.588094in}}%
\pgfpathcurveto{\pgfqpoint{0.988124in}{1.593918in}}{\pgfqpoint{0.980223in}{1.597190in}}{\pgfqpoint{0.971987in}{1.597190in}}%
\pgfpathcurveto{\pgfqpoint{0.963751in}{1.597190in}}{\pgfqpoint{0.955851in}{1.593918in}}{\pgfqpoint{0.950027in}{1.588094in}}%
\pgfpathcurveto{\pgfqpoint{0.944203in}{1.582270in}}{\pgfqpoint{0.940931in}{1.574370in}}{\pgfqpoint{0.940931in}{1.566133in}}%
\pgfpathcurveto{\pgfqpoint{0.940931in}{1.557897in}}{\pgfqpoint{0.944203in}{1.549997in}}{\pgfqpoint{0.950027in}{1.544173in}}%
\pgfpathcurveto{\pgfqpoint{0.955851in}{1.538349in}}{\pgfqpoint{0.963751in}{1.535077in}}{\pgfqpoint{0.971987in}{1.535077in}}%
\pgfpathclose%
\pgfusepath{stroke,fill}%
\end{pgfscope}%
\begin{pgfscope}%
\pgfpathrectangle{\pgfqpoint{0.100000in}{0.220728in}}{\pgfqpoint{3.696000in}{3.696000in}}%
\pgfusepath{clip}%
\pgfsetbuttcap%
\pgfsetroundjoin%
\definecolor{currentfill}{rgb}{0.121569,0.466667,0.705882}%
\pgfsetfillcolor{currentfill}%
\pgfsetfillopacity{0.567109}%
\pgfsetlinewidth{1.003750pt}%
\definecolor{currentstroke}{rgb}{0.121569,0.466667,0.705882}%
\pgfsetstrokecolor{currentstroke}%
\pgfsetstrokeopacity{0.567109}%
\pgfsetdash{}{0pt}%
\pgfpathmoveto{\pgfqpoint{2.899900in}{3.021189in}}%
\pgfpathcurveto{\pgfqpoint{2.908137in}{3.021189in}}{\pgfqpoint{2.916037in}{3.024461in}}{\pgfqpoint{2.921860in}{3.030285in}}%
\pgfpathcurveto{\pgfqpoint{2.927684in}{3.036109in}}{\pgfqpoint{2.930957in}{3.044009in}}{\pgfqpoint{2.930957in}{3.052245in}}%
\pgfpathcurveto{\pgfqpoint{2.930957in}{3.060482in}}{\pgfqpoint{2.927684in}{3.068382in}}{\pgfqpoint{2.921860in}{3.074205in}}%
\pgfpathcurveto{\pgfqpoint{2.916037in}{3.080029in}}{\pgfqpoint{2.908137in}{3.083302in}}{\pgfqpoint{2.899900in}{3.083302in}}%
\pgfpathcurveto{\pgfqpoint{2.891664in}{3.083302in}}{\pgfqpoint{2.883764in}{3.080029in}}{\pgfqpoint{2.877940in}{3.074205in}}%
\pgfpathcurveto{\pgfqpoint{2.872116in}{3.068382in}}{\pgfqpoint{2.868844in}{3.060482in}}{\pgfqpoint{2.868844in}{3.052245in}}%
\pgfpathcurveto{\pgfqpoint{2.868844in}{3.044009in}}{\pgfqpoint{2.872116in}{3.036109in}}{\pgfqpoint{2.877940in}{3.030285in}}%
\pgfpathcurveto{\pgfqpoint{2.883764in}{3.024461in}}{\pgfqpoint{2.891664in}{3.021189in}}{\pgfqpoint{2.899900in}{3.021189in}}%
\pgfpathclose%
\pgfusepath{stroke,fill}%
\end{pgfscope}%
\begin{pgfscope}%
\pgfpathrectangle{\pgfqpoint{0.100000in}{0.220728in}}{\pgfqpoint{3.696000in}{3.696000in}}%
\pgfusepath{clip}%
\pgfsetbuttcap%
\pgfsetroundjoin%
\definecolor{currentfill}{rgb}{0.121569,0.466667,0.705882}%
\pgfsetfillcolor{currentfill}%
\pgfsetfillopacity{0.567549}%
\pgfsetlinewidth{1.003750pt}%
\definecolor{currentstroke}{rgb}{0.121569,0.466667,0.705882}%
\pgfsetstrokecolor{currentstroke}%
\pgfsetstrokeopacity{0.567549}%
\pgfsetdash{}{0pt}%
\pgfpathmoveto{\pgfqpoint{2.893800in}{3.022716in}}%
\pgfpathcurveto{\pgfqpoint{2.902037in}{3.022716in}}{\pgfqpoint{2.909937in}{3.025988in}}{\pgfqpoint{2.915761in}{3.031812in}}%
\pgfpathcurveto{\pgfqpoint{2.921584in}{3.037636in}}{\pgfqpoint{2.924857in}{3.045536in}}{\pgfqpoint{2.924857in}{3.053772in}}%
\pgfpathcurveto{\pgfqpoint{2.924857in}{3.062009in}}{\pgfqpoint{2.921584in}{3.069909in}}{\pgfqpoint{2.915761in}{3.075733in}}%
\pgfpathcurveto{\pgfqpoint{2.909937in}{3.081556in}}{\pgfqpoint{2.902037in}{3.084829in}}{\pgfqpoint{2.893800in}{3.084829in}}%
\pgfpathcurveto{\pgfqpoint{2.885564in}{3.084829in}}{\pgfqpoint{2.877664in}{3.081556in}}{\pgfqpoint{2.871840in}{3.075733in}}%
\pgfpathcurveto{\pgfqpoint{2.866016in}{3.069909in}}{\pgfqpoint{2.862744in}{3.062009in}}{\pgfqpoint{2.862744in}{3.053772in}}%
\pgfpathcurveto{\pgfqpoint{2.862744in}{3.045536in}}{\pgfqpoint{2.866016in}{3.037636in}}{\pgfqpoint{2.871840in}{3.031812in}}%
\pgfpathcurveto{\pgfqpoint{2.877664in}{3.025988in}}{\pgfqpoint{2.885564in}{3.022716in}}{\pgfqpoint{2.893800in}{3.022716in}}%
\pgfpathclose%
\pgfusepath{stroke,fill}%
\end{pgfscope}%
\begin{pgfscope}%
\pgfpathrectangle{\pgfqpoint{0.100000in}{0.220728in}}{\pgfqpoint{3.696000in}{3.696000in}}%
\pgfusepath{clip}%
\pgfsetbuttcap%
\pgfsetroundjoin%
\definecolor{currentfill}{rgb}{0.121569,0.466667,0.705882}%
\pgfsetfillcolor{currentfill}%
\pgfsetfillopacity{0.567914}%
\pgfsetlinewidth{1.003750pt}%
\definecolor{currentstroke}{rgb}{0.121569,0.466667,0.705882}%
\pgfsetstrokecolor{currentstroke}%
\pgfsetstrokeopacity{0.567914}%
\pgfsetdash{}{0pt}%
\pgfpathmoveto{\pgfqpoint{2.902750in}{3.020993in}}%
\pgfpathcurveto{\pgfqpoint{2.910987in}{3.020993in}}{\pgfqpoint{2.918887in}{3.024265in}}{\pgfqpoint{2.924711in}{3.030089in}}%
\pgfpathcurveto{\pgfqpoint{2.930535in}{3.035913in}}{\pgfqpoint{2.933807in}{3.043813in}}{\pgfqpoint{2.933807in}{3.052049in}}%
\pgfpathcurveto{\pgfqpoint{2.933807in}{3.060285in}}{\pgfqpoint{2.930535in}{3.068185in}}{\pgfqpoint{2.924711in}{3.074009in}}%
\pgfpathcurveto{\pgfqpoint{2.918887in}{3.079833in}}{\pgfqpoint{2.910987in}{3.083106in}}{\pgfqpoint{2.902750in}{3.083106in}}%
\pgfpathcurveto{\pgfqpoint{2.894514in}{3.083106in}}{\pgfqpoint{2.886614in}{3.079833in}}{\pgfqpoint{2.880790in}{3.074009in}}%
\pgfpathcurveto{\pgfqpoint{2.874966in}{3.068185in}}{\pgfqpoint{2.871694in}{3.060285in}}{\pgfqpoint{2.871694in}{3.052049in}}%
\pgfpathcurveto{\pgfqpoint{2.871694in}{3.043813in}}{\pgfqpoint{2.874966in}{3.035913in}}{\pgfqpoint{2.880790in}{3.030089in}}%
\pgfpathcurveto{\pgfqpoint{2.886614in}{3.024265in}}{\pgfqpoint{2.894514in}{3.020993in}}{\pgfqpoint{2.902750in}{3.020993in}}%
\pgfpathclose%
\pgfusepath{stroke,fill}%
\end{pgfscope}%
\begin{pgfscope}%
\pgfpathrectangle{\pgfqpoint{0.100000in}{0.220728in}}{\pgfqpoint{3.696000in}{3.696000in}}%
\pgfusepath{clip}%
\pgfsetbuttcap%
\pgfsetroundjoin%
\definecolor{currentfill}{rgb}{0.121569,0.466667,0.705882}%
\pgfsetfillcolor{currentfill}%
\pgfsetfillopacity{0.568318}%
\pgfsetlinewidth{1.003750pt}%
\definecolor{currentstroke}{rgb}{0.121569,0.466667,0.705882}%
\pgfsetstrokecolor{currentstroke}%
\pgfsetstrokeopacity{0.568318}%
\pgfsetdash{}{0pt}%
\pgfpathmoveto{\pgfqpoint{0.964076in}{1.525781in}}%
\pgfpathcurveto{\pgfqpoint{0.972313in}{1.525781in}}{\pgfqpoint{0.980213in}{1.529054in}}{\pgfqpoint{0.986036in}{1.534878in}}%
\pgfpathcurveto{\pgfqpoint{0.991860in}{1.540702in}}{\pgfqpoint{0.995133in}{1.548602in}}{\pgfqpoint{0.995133in}{1.556838in}}%
\pgfpathcurveto{\pgfqpoint{0.995133in}{1.565074in}}{\pgfqpoint{0.991860in}{1.572974in}}{\pgfqpoint{0.986036in}{1.578798in}}%
\pgfpathcurveto{\pgfqpoint{0.980213in}{1.584622in}}{\pgfqpoint{0.972313in}{1.587894in}}{\pgfqpoint{0.964076in}{1.587894in}}%
\pgfpathcurveto{\pgfqpoint{0.955840in}{1.587894in}}{\pgfqpoint{0.947940in}{1.584622in}}{\pgfqpoint{0.942116in}{1.578798in}}%
\pgfpathcurveto{\pgfqpoint{0.936292in}{1.572974in}}{\pgfqpoint{0.933020in}{1.565074in}}{\pgfqpoint{0.933020in}{1.556838in}}%
\pgfpathcurveto{\pgfqpoint{0.933020in}{1.548602in}}{\pgfqpoint{0.936292in}{1.540702in}}{\pgfqpoint{0.942116in}{1.534878in}}%
\pgfpathcurveto{\pgfqpoint{0.947940in}{1.529054in}}{\pgfqpoint{0.955840in}{1.525781in}}{\pgfqpoint{0.964076in}{1.525781in}}%
\pgfpathclose%
\pgfusepath{stroke,fill}%
\end{pgfscope}%
\begin{pgfscope}%
\pgfpathrectangle{\pgfqpoint{0.100000in}{0.220728in}}{\pgfqpoint{3.696000in}{3.696000in}}%
\pgfusepath{clip}%
\pgfsetbuttcap%
\pgfsetroundjoin%
\definecolor{currentfill}{rgb}{0.121569,0.466667,0.705882}%
\pgfsetfillcolor{currentfill}%
\pgfsetfillopacity{0.568401}%
\pgfsetlinewidth{1.003750pt}%
\definecolor{currentstroke}{rgb}{0.121569,0.466667,0.705882}%
\pgfsetstrokecolor{currentstroke}%
\pgfsetstrokeopacity{0.568401}%
\pgfsetdash{}{0pt}%
\pgfpathmoveto{\pgfqpoint{2.907323in}{3.020449in}}%
\pgfpathcurveto{\pgfqpoint{2.915560in}{3.020449in}}{\pgfqpoint{2.923460in}{3.023721in}}{\pgfqpoint{2.929284in}{3.029545in}}%
\pgfpathcurveto{\pgfqpoint{2.935107in}{3.035369in}}{\pgfqpoint{2.938380in}{3.043269in}}{\pgfqpoint{2.938380in}{3.051505in}}%
\pgfpathcurveto{\pgfqpoint{2.938380in}{3.059742in}}{\pgfqpoint{2.935107in}{3.067642in}}{\pgfqpoint{2.929284in}{3.073466in}}%
\pgfpathcurveto{\pgfqpoint{2.923460in}{3.079290in}}{\pgfqpoint{2.915560in}{3.082562in}}{\pgfqpoint{2.907323in}{3.082562in}}%
\pgfpathcurveto{\pgfqpoint{2.899087in}{3.082562in}}{\pgfqpoint{2.891187in}{3.079290in}}{\pgfqpoint{2.885363in}{3.073466in}}%
\pgfpathcurveto{\pgfqpoint{2.879539in}{3.067642in}}{\pgfqpoint{2.876267in}{3.059742in}}{\pgfqpoint{2.876267in}{3.051505in}}%
\pgfpathcurveto{\pgfqpoint{2.876267in}{3.043269in}}{\pgfqpoint{2.879539in}{3.035369in}}{\pgfqpoint{2.885363in}{3.029545in}}%
\pgfpathcurveto{\pgfqpoint{2.891187in}{3.023721in}}{\pgfqpoint{2.899087in}{3.020449in}}{\pgfqpoint{2.907323in}{3.020449in}}%
\pgfpathclose%
\pgfusepath{stroke,fill}%
\end{pgfscope}%
\begin{pgfscope}%
\pgfpathrectangle{\pgfqpoint{0.100000in}{0.220728in}}{\pgfqpoint{3.696000in}{3.696000in}}%
\pgfusepath{clip}%
\pgfsetbuttcap%
\pgfsetroundjoin%
\definecolor{currentfill}{rgb}{0.121569,0.466667,0.705882}%
\pgfsetfillcolor{currentfill}%
\pgfsetfillopacity{0.569803}%
\pgfsetlinewidth{1.003750pt}%
\definecolor{currentstroke}{rgb}{0.121569,0.466667,0.705882}%
\pgfsetstrokecolor{currentstroke}%
\pgfsetstrokeopacity{0.569803}%
\pgfsetdash{}{0pt}%
\pgfpathmoveto{\pgfqpoint{2.911918in}{3.019926in}}%
\pgfpathcurveto{\pgfqpoint{2.920154in}{3.019926in}}{\pgfqpoint{2.928055in}{3.023198in}}{\pgfqpoint{2.933878in}{3.029022in}}%
\pgfpathcurveto{\pgfqpoint{2.939702in}{3.034846in}}{\pgfqpoint{2.942975in}{3.042746in}}{\pgfqpoint{2.942975in}{3.050983in}}%
\pgfpathcurveto{\pgfqpoint{2.942975in}{3.059219in}}{\pgfqpoint{2.939702in}{3.067119in}}{\pgfqpoint{2.933878in}{3.072943in}}%
\pgfpathcurveto{\pgfqpoint{2.928055in}{3.078767in}}{\pgfqpoint{2.920154in}{3.082039in}}{\pgfqpoint{2.911918in}{3.082039in}}%
\pgfpathcurveto{\pgfqpoint{2.903682in}{3.082039in}}{\pgfqpoint{2.895782in}{3.078767in}}{\pgfqpoint{2.889958in}{3.072943in}}%
\pgfpathcurveto{\pgfqpoint{2.884134in}{3.067119in}}{\pgfqpoint{2.880862in}{3.059219in}}{\pgfqpoint{2.880862in}{3.050983in}}%
\pgfpathcurveto{\pgfqpoint{2.880862in}{3.042746in}}{\pgfqpoint{2.884134in}{3.034846in}}{\pgfqpoint{2.889958in}{3.029022in}}%
\pgfpathcurveto{\pgfqpoint{2.895782in}{3.023198in}}{\pgfqpoint{2.903682in}{3.019926in}}{\pgfqpoint{2.911918in}{3.019926in}}%
\pgfpathclose%
\pgfusepath{stroke,fill}%
\end{pgfscope}%
\begin{pgfscope}%
\pgfpathrectangle{\pgfqpoint{0.100000in}{0.220728in}}{\pgfqpoint{3.696000in}{3.696000in}}%
\pgfusepath{clip}%
\pgfsetbuttcap%
\pgfsetroundjoin%
\definecolor{currentfill}{rgb}{0.121569,0.466667,0.705882}%
\pgfsetfillcolor{currentfill}%
\pgfsetfillopacity{0.569911}%
\pgfsetlinewidth{1.003750pt}%
\definecolor{currentstroke}{rgb}{0.121569,0.466667,0.705882}%
\pgfsetstrokecolor{currentstroke}%
\pgfsetstrokeopacity{0.569911}%
\pgfsetdash{}{0pt}%
\pgfpathmoveto{\pgfqpoint{0.962810in}{1.515048in}}%
\pgfpathcurveto{\pgfqpoint{0.971046in}{1.515048in}}{\pgfqpoint{0.978946in}{1.518321in}}{\pgfqpoint{0.984770in}{1.524145in}}%
\pgfpathcurveto{\pgfqpoint{0.990594in}{1.529968in}}{\pgfqpoint{0.993866in}{1.537869in}}{\pgfqpoint{0.993866in}{1.546105in}}%
\pgfpathcurveto{\pgfqpoint{0.993866in}{1.554341in}}{\pgfqpoint{0.990594in}{1.562241in}}{\pgfqpoint{0.984770in}{1.568065in}}%
\pgfpathcurveto{\pgfqpoint{0.978946in}{1.573889in}}{\pgfqpoint{0.971046in}{1.577161in}}{\pgfqpoint{0.962810in}{1.577161in}}%
\pgfpathcurveto{\pgfqpoint{0.954574in}{1.577161in}}{\pgfqpoint{0.946674in}{1.573889in}}{\pgfqpoint{0.940850in}{1.568065in}}%
\pgfpathcurveto{\pgfqpoint{0.935026in}{1.562241in}}{\pgfqpoint{0.931753in}{1.554341in}}{\pgfqpoint{0.931753in}{1.546105in}}%
\pgfpathcurveto{\pgfqpoint{0.931753in}{1.537869in}}{\pgfqpoint{0.935026in}{1.529968in}}{\pgfqpoint{0.940850in}{1.524145in}}%
\pgfpathcurveto{\pgfqpoint{0.946674in}{1.518321in}}{\pgfqpoint{0.954574in}{1.515048in}}{\pgfqpoint{0.962810in}{1.515048in}}%
\pgfpathclose%
\pgfusepath{stroke,fill}%
\end{pgfscope}%
\begin{pgfscope}%
\pgfpathrectangle{\pgfqpoint{0.100000in}{0.220728in}}{\pgfqpoint{3.696000in}{3.696000in}}%
\pgfusepath{clip}%
\pgfsetbuttcap%
\pgfsetroundjoin%
\definecolor{currentfill}{rgb}{0.121569,0.466667,0.705882}%
\pgfsetfillcolor{currentfill}%
\pgfsetfillopacity{0.570466}%
\pgfsetlinewidth{1.003750pt}%
\definecolor{currentstroke}{rgb}{0.121569,0.466667,0.705882}%
\pgfsetstrokecolor{currentstroke}%
\pgfsetstrokeopacity{0.570466}%
\pgfsetdash{}{0pt}%
\pgfpathmoveto{\pgfqpoint{0.958621in}{1.510442in}}%
\pgfpathcurveto{\pgfqpoint{0.966857in}{1.510442in}}{\pgfqpoint{0.974757in}{1.513714in}}{\pgfqpoint{0.980581in}{1.519538in}}%
\pgfpathcurveto{\pgfqpoint{0.986405in}{1.525362in}}{\pgfqpoint{0.989678in}{1.533262in}}{\pgfqpoint{0.989678in}{1.541498in}}%
\pgfpathcurveto{\pgfqpoint{0.989678in}{1.549735in}}{\pgfqpoint{0.986405in}{1.557635in}}{\pgfqpoint{0.980581in}{1.563458in}}%
\pgfpathcurveto{\pgfqpoint{0.974757in}{1.569282in}}{\pgfqpoint{0.966857in}{1.572555in}}{\pgfqpoint{0.958621in}{1.572555in}}%
\pgfpathcurveto{\pgfqpoint{0.950385in}{1.572555in}}{\pgfqpoint{0.942485in}{1.569282in}}{\pgfqpoint{0.936661in}{1.563458in}}%
\pgfpathcurveto{\pgfqpoint{0.930837in}{1.557635in}}{\pgfqpoint{0.927565in}{1.549735in}}{\pgfqpoint{0.927565in}{1.541498in}}%
\pgfpathcurveto{\pgfqpoint{0.927565in}{1.533262in}}{\pgfqpoint{0.930837in}{1.525362in}}{\pgfqpoint{0.936661in}{1.519538in}}%
\pgfpathcurveto{\pgfqpoint{0.942485in}{1.513714in}}{\pgfqpoint{0.950385in}{1.510442in}}{\pgfqpoint{0.958621in}{1.510442in}}%
\pgfpathclose%
\pgfusepath{stroke,fill}%
\end{pgfscope}%
\begin{pgfscope}%
\pgfpathrectangle{\pgfqpoint{0.100000in}{0.220728in}}{\pgfqpoint{3.696000in}{3.696000in}}%
\pgfusepath{clip}%
\pgfsetbuttcap%
\pgfsetroundjoin%
\definecolor{currentfill}{rgb}{0.121569,0.466667,0.705882}%
\pgfsetfillcolor{currentfill}%
\pgfsetfillopacity{0.571086}%
\pgfsetlinewidth{1.003750pt}%
\definecolor{currentstroke}{rgb}{0.121569,0.466667,0.705882}%
\pgfsetstrokecolor{currentstroke}%
\pgfsetstrokeopacity{0.571086}%
\pgfsetdash{}{0pt}%
\pgfpathmoveto{\pgfqpoint{0.957736in}{1.506528in}}%
\pgfpathcurveto{\pgfqpoint{0.965972in}{1.506528in}}{\pgfqpoint{0.973872in}{1.509800in}}{\pgfqpoint{0.979696in}{1.515624in}}%
\pgfpathcurveto{\pgfqpoint{0.985520in}{1.521448in}}{\pgfqpoint{0.988792in}{1.529348in}}{\pgfqpoint{0.988792in}{1.537584in}}%
\pgfpathcurveto{\pgfqpoint{0.988792in}{1.545821in}}{\pgfqpoint{0.985520in}{1.553721in}}{\pgfqpoint{0.979696in}{1.559545in}}%
\pgfpathcurveto{\pgfqpoint{0.973872in}{1.565369in}}{\pgfqpoint{0.965972in}{1.568641in}}{\pgfqpoint{0.957736in}{1.568641in}}%
\pgfpathcurveto{\pgfqpoint{0.949499in}{1.568641in}}{\pgfqpoint{0.941599in}{1.565369in}}{\pgfqpoint{0.935775in}{1.559545in}}%
\pgfpathcurveto{\pgfqpoint{0.929951in}{1.553721in}}{\pgfqpoint{0.926679in}{1.545821in}}{\pgfqpoint{0.926679in}{1.537584in}}%
\pgfpathcurveto{\pgfqpoint{0.926679in}{1.529348in}}{\pgfqpoint{0.929951in}{1.521448in}}{\pgfqpoint{0.935775in}{1.515624in}}%
\pgfpathcurveto{\pgfqpoint{0.941599in}{1.509800in}}{\pgfqpoint{0.949499in}{1.506528in}}{\pgfqpoint{0.957736in}{1.506528in}}%
\pgfpathclose%
\pgfusepath{stroke,fill}%
\end{pgfscope}%
\begin{pgfscope}%
\pgfpathrectangle{\pgfqpoint{0.100000in}{0.220728in}}{\pgfqpoint{3.696000in}{3.696000in}}%
\pgfusepath{clip}%
\pgfsetbuttcap%
\pgfsetroundjoin%
\definecolor{currentfill}{rgb}{0.121569,0.466667,0.705882}%
\pgfsetfillcolor{currentfill}%
\pgfsetfillopacity{0.571172}%
\pgfsetlinewidth{1.003750pt}%
\definecolor{currentstroke}{rgb}{0.121569,0.466667,0.705882}%
\pgfsetstrokecolor{currentstroke}%
\pgfsetstrokeopacity{0.571172}%
\pgfsetdash{}{0pt}%
\pgfpathmoveto{\pgfqpoint{0.957130in}{1.505775in}}%
\pgfpathcurveto{\pgfqpoint{0.965366in}{1.505775in}}{\pgfqpoint{0.973266in}{1.509047in}}{\pgfqpoint{0.979090in}{1.514871in}}%
\pgfpathcurveto{\pgfqpoint{0.984914in}{1.520695in}}{\pgfqpoint{0.988187in}{1.528595in}}{\pgfqpoint{0.988187in}{1.536831in}}%
\pgfpathcurveto{\pgfqpoint{0.988187in}{1.545067in}}{\pgfqpoint{0.984914in}{1.552967in}}{\pgfqpoint{0.979090in}{1.558791in}}%
\pgfpathcurveto{\pgfqpoint{0.973266in}{1.564615in}}{\pgfqpoint{0.965366in}{1.567888in}}{\pgfqpoint{0.957130in}{1.567888in}}%
\pgfpathcurveto{\pgfqpoint{0.948894in}{1.567888in}}{\pgfqpoint{0.940994in}{1.564615in}}{\pgfqpoint{0.935170in}{1.558791in}}%
\pgfpathcurveto{\pgfqpoint{0.929346in}{1.552967in}}{\pgfqpoint{0.926074in}{1.545067in}}{\pgfqpoint{0.926074in}{1.536831in}}%
\pgfpathcurveto{\pgfqpoint{0.926074in}{1.528595in}}{\pgfqpoint{0.929346in}{1.520695in}}{\pgfqpoint{0.935170in}{1.514871in}}%
\pgfpathcurveto{\pgfqpoint{0.940994in}{1.509047in}}{\pgfqpoint{0.948894in}{1.505775in}}{\pgfqpoint{0.957130in}{1.505775in}}%
\pgfpathclose%
\pgfusepath{stroke,fill}%
\end{pgfscope}%
\begin{pgfscope}%
\pgfpathrectangle{\pgfqpoint{0.100000in}{0.220728in}}{\pgfqpoint{3.696000in}{3.696000in}}%
\pgfusepath{clip}%
\pgfsetbuttcap%
\pgfsetroundjoin%
\definecolor{currentfill}{rgb}{0.121569,0.466667,0.705882}%
\pgfsetfillcolor{currentfill}%
\pgfsetfillopacity{0.571295}%
\pgfsetlinewidth{1.003750pt}%
\definecolor{currentstroke}{rgb}{0.121569,0.466667,0.705882}%
\pgfsetstrokecolor{currentstroke}%
\pgfsetstrokeopacity{0.571295}%
\pgfsetdash{}{0pt}%
\pgfpathmoveto{\pgfqpoint{2.917503in}{3.019544in}}%
\pgfpathcurveto{\pgfqpoint{2.925740in}{3.019544in}}{\pgfqpoint{2.933640in}{3.022816in}}{\pgfqpoint{2.939464in}{3.028640in}}%
\pgfpathcurveto{\pgfqpoint{2.945287in}{3.034464in}}{\pgfqpoint{2.948560in}{3.042364in}}{\pgfqpoint{2.948560in}{3.050601in}}%
\pgfpathcurveto{\pgfqpoint{2.948560in}{3.058837in}}{\pgfqpoint{2.945287in}{3.066737in}}{\pgfqpoint{2.939464in}{3.072561in}}%
\pgfpathcurveto{\pgfqpoint{2.933640in}{3.078385in}}{\pgfqpoint{2.925740in}{3.081657in}}{\pgfqpoint{2.917503in}{3.081657in}}%
\pgfpathcurveto{\pgfqpoint{2.909267in}{3.081657in}}{\pgfqpoint{2.901367in}{3.078385in}}{\pgfqpoint{2.895543in}{3.072561in}}%
\pgfpathcurveto{\pgfqpoint{2.889719in}{3.066737in}}{\pgfqpoint{2.886447in}{3.058837in}}{\pgfqpoint{2.886447in}{3.050601in}}%
\pgfpathcurveto{\pgfqpoint{2.886447in}{3.042364in}}{\pgfqpoint{2.889719in}{3.034464in}}{\pgfqpoint{2.895543in}{3.028640in}}%
\pgfpathcurveto{\pgfqpoint{2.901367in}{3.022816in}}{\pgfqpoint{2.909267in}{3.019544in}}{\pgfqpoint{2.917503in}{3.019544in}}%
\pgfpathclose%
\pgfusepath{stroke,fill}%
\end{pgfscope}%
\begin{pgfscope}%
\pgfpathrectangle{\pgfqpoint{0.100000in}{0.220728in}}{\pgfqpoint{3.696000in}{3.696000in}}%
\pgfusepath{clip}%
\pgfsetbuttcap%
\pgfsetroundjoin%
\definecolor{currentfill}{rgb}{0.121569,0.466667,0.705882}%
\pgfsetfillcolor{currentfill}%
\pgfsetfillopacity{0.571449}%
\pgfsetlinewidth{1.003750pt}%
\definecolor{currentstroke}{rgb}{0.121569,0.466667,0.705882}%
\pgfsetstrokecolor{currentstroke}%
\pgfsetstrokeopacity{0.571449}%
\pgfsetdash{}{0pt}%
\pgfpathmoveto{\pgfqpoint{0.956618in}{1.504096in}}%
\pgfpathcurveto{\pgfqpoint{0.964854in}{1.504096in}}{\pgfqpoint{0.972754in}{1.507369in}}{\pgfqpoint{0.978578in}{1.513193in}}%
\pgfpathcurveto{\pgfqpoint{0.984402in}{1.519017in}}{\pgfqpoint{0.987674in}{1.526917in}}{\pgfqpoint{0.987674in}{1.535153in}}%
\pgfpathcurveto{\pgfqpoint{0.987674in}{1.543389in}}{\pgfqpoint{0.984402in}{1.551289in}}{\pgfqpoint{0.978578in}{1.557113in}}%
\pgfpathcurveto{\pgfqpoint{0.972754in}{1.562937in}}{\pgfqpoint{0.964854in}{1.566209in}}{\pgfqpoint{0.956618in}{1.566209in}}%
\pgfpathcurveto{\pgfqpoint{0.948381in}{1.566209in}}{\pgfqpoint{0.940481in}{1.562937in}}{\pgfqpoint{0.934657in}{1.557113in}}%
\pgfpathcurveto{\pgfqpoint{0.928833in}{1.551289in}}{\pgfqpoint{0.925561in}{1.543389in}}{\pgfqpoint{0.925561in}{1.535153in}}%
\pgfpathcurveto{\pgfqpoint{0.925561in}{1.526917in}}{\pgfqpoint{0.928833in}{1.519017in}}{\pgfqpoint{0.934657in}{1.513193in}}%
\pgfpathcurveto{\pgfqpoint{0.940481in}{1.507369in}}{\pgfqpoint{0.948381in}{1.504096in}}{\pgfqpoint{0.956618in}{1.504096in}}%
\pgfpathclose%
\pgfusepath{stroke,fill}%
\end{pgfscope}%
\begin{pgfscope}%
\pgfpathrectangle{\pgfqpoint{0.100000in}{0.220728in}}{\pgfqpoint{3.696000in}{3.696000in}}%
\pgfusepath{clip}%
\pgfsetbuttcap%
\pgfsetroundjoin%
\definecolor{currentfill}{rgb}{0.121569,0.466667,0.705882}%
\pgfsetfillcolor{currentfill}%
\pgfsetfillopacity{0.571849}%
\pgfsetlinewidth{1.003750pt}%
\definecolor{currentstroke}{rgb}{0.121569,0.466667,0.705882}%
\pgfsetstrokecolor{currentstroke}%
\pgfsetstrokeopacity{0.571849}%
\pgfsetdash{}{0pt}%
\pgfpathmoveto{\pgfqpoint{0.954785in}{1.501678in}}%
\pgfpathcurveto{\pgfqpoint{0.963022in}{1.501678in}}{\pgfqpoint{0.970922in}{1.504950in}}{\pgfqpoint{0.976746in}{1.510774in}}%
\pgfpathcurveto{\pgfqpoint{0.982570in}{1.516598in}}{\pgfqpoint{0.985842in}{1.524498in}}{\pgfqpoint{0.985842in}{1.532734in}}%
\pgfpathcurveto{\pgfqpoint{0.985842in}{1.540970in}}{\pgfqpoint{0.982570in}{1.548871in}}{\pgfqpoint{0.976746in}{1.554694in}}%
\pgfpathcurveto{\pgfqpoint{0.970922in}{1.560518in}}{\pgfqpoint{0.963022in}{1.563791in}}{\pgfqpoint{0.954785in}{1.563791in}}%
\pgfpathcurveto{\pgfqpoint{0.946549in}{1.563791in}}{\pgfqpoint{0.938649in}{1.560518in}}{\pgfqpoint{0.932825in}{1.554694in}}%
\pgfpathcurveto{\pgfqpoint{0.927001in}{1.548871in}}{\pgfqpoint{0.923729in}{1.540970in}}{\pgfqpoint{0.923729in}{1.532734in}}%
\pgfpathcurveto{\pgfqpoint{0.923729in}{1.524498in}}{\pgfqpoint{0.927001in}{1.516598in}}{\pgfqpoint{0.932825in}{1.510774in}}%
\pgfpathcurveto{\pgfqpoint{0.938649in}{1.504950in}}{\pgfqpoint{0.946549in}{1.501678in}}{\pgfqpoint{0.954785in}{1.501678in}}%
\pgfpathclose%
\pgfusepath{stroke,fill}%
\end{pgfscope}%
\begin{pgfscope}%
\pgfpathrectangle{\pgfqpoint{0.100000in}{0.220728in}}{\pgfqpoint{3.696000in}{3.696000in}}%
\pgfusepath{clip}%
\pgfsetbuttcap%
\pgfsetroundjoin%
\definecolor{currentfill}{rgb}{0.121569,0.466667,0.705882}%
\pgfsetfillcolor{currentfill}%
\pgfsetfillopacity{0.572101}%
\pgfsetlinewidth{1.003750pt}%
\definecolor{currentstroke}{rgb}{0.121569,0.466667,0.705882}%
\pgfsetstrokecolor{currentstroke}%
\pgfsetstrokeopacity{0.572101}%
\pgfsetdash{}{0pt}%
\pgfpathmoveto{\pgfqpoint{0.954187in}{1.500257in}}%
\pgfpathcurveto{\pgfqpoint{0.962423in}{1.500257in}}{\pgfqpoint{0.970323in}{1.503529in}}{\pgfqpoint{0.976147in}{1.509353in}}%
\pgfpathcurveto{\pgfqpoint{0.981971in}{1.515177in}}{\pgfqpoint{0.985243in}{1.523077in}}{\pgfqpoint{0.985243in}{1.531314in}}%
\pgfpathcurveto{\pgfqpoint{0.985243in}{1.539550in}}{\pgfqpoint{0.981971in}{1.547450in}}{\pgfqpoint{0.976147in}{1.553274in}}%
\pgfpathcurveto{\pgfqpoint{0.970323in}{1.559098in}}{\pgfqpoint{0.962423in}{1.562370in}}{\pgfqpoint{0.954187in}{1.562370in}}%
\pgfpathcurveto{\pgfqpoint{0.945951in}{1.562370in}}{\pgfqpoint{0.938051in}{1.559098in}}{\pgfqpoint{0.932227in}{1.553274in}}%
\pgfpathcurveto{\pgfqpoint{0.926403in}{1.547450in}}{\pgfqpoint{0.923130in}{1.539550in}}{\pgfqpoint{0.923130in}{1.531314in}}%
\pgfpathcurveto{\pgfqpoint{0.923130in}{1.523077in}}{\pgfqpoint{0.926403in}{1.515177in}}{\pgfqpoint{0.932227in}{1.509353in}}%
\pgfpathcurveto{\pgfqpoint{0.938051in}{1.503529in}}{\pgfqpoint{0.945951in}{1.500257in}}{\pgfqpoint{0.954187in}{1.500257in}}%
\pgfpathclose%
\pgfusepath{stroke,fill}%
\end{pgfscope}%
\begin{pgfscope}%
\pgfpathrectangle{\pgfqpoint{0.100000in}{0.220728in}}{\pgfqpoint{3.696000in}{3.696000in}}%
\pgfusepath{clip}%
\pgfsetbuttcap%
\pgfsetroundjoin%
\definecolor{currentfill}{rgb}{0.121569,0.466667,0.705882}%
\pgfsetfillcolor{currentfill}%
\pgfsetfillopacity{0.572212}%
\pgfsetlinewidth{1.003750pt}%
\definecolor{currentstroke}{rgb}{0.121569,0.466667,0.705882}%
\pgfsetstrokecolor{currentstroke}%
\pgfsetstrokeopacity{0.572212}%
\pgfsetdash{}{0pt}%
\pgfpathmoveto{\pgfqpoint{0.953850in}{1.499630in}}%
\pgfpathcurveto{\pgfqpoint{0.962086in}{1.499630in}}{\pgfqpoint{0.969986in}{1.502903in}}{\pgfqpoint{0.975810in}{1.508726in}}%
\pgfpathcurveto{\pgfqpoint{0.981634in}{1.514550in}}{\pgfqpoint{0.984906in}{1.522450in}}{\pgfqpoint{0.984906in}{1.530687in}}%
\pgfpathcurveto{\pgfqpoint{0.984906in}{1.538923in}}{\pgfqpoint{0.981634in}{1.546823in}}{\pgfqpoint{0.975810in}{1.552647in}}%
\pgfpathcurveto{\pgfqpoint{0.969986in}{1.558471in}}{\pgfqpoint{0.962086in}{1.561743in}}{\pgfqpoint{0.953850in}{1.561743in}}%
\pgfpathcurveto{\pgfqpoint{0.945613in}{1.561743in}}{\pgfqpoint{0.937713in}{1.558471in}}{\pgfqpoint{0.931889in}{1.552647in}}%
\pgfpathcurveto{\pgfqpoint{0.926065in}{1.546823in}}{\pgfqpoint{0.922793in}{1.538923in}}{\pgfqpoint{0.922793in}{1.530687in}}%
\pgfpathcurveto{\pgfqpoint{0.922793in}{1.522450in}}{\pgfqpoint{0.926065in}{1.514550in}}{\pgfqpoint{0.931889in}{1.508726in}}%
\pgfpathcurveto{\pgfqpoint{0.937713in}{1.502903in}}{\pgfqpoint{0.945613in}{1.499630in}}{\pgfqpoint{0.953850in}{1.499630in}}%
\pgfpathclose%
\pgfusepath{stroke,fill}%
\end{pgfscope}%
\begin{pgfscope}%
\pgfpathrectangle{\pgfqpoint{0.100000in}{0.220728in}}{\pgfqpoint{3.696000in}{3.696000in}}%
\pgfusepath{clip}%
\pgfsetbuttcap%
\pgfsetroundjoin%
\definecolor{currentfill}{rgb}{0.121569,0.466667,0.705882}%
\pgfsetfillcolor{currentfill}%
\pgfsetfillopacity{0.572405}%
\pgfsetlinewidth{1.003750pt}%
\definecolor{currentstroke}{rgb}{0.121569,0.466667,0.705882}%
\pgfsetstrokecolor{currentstroke}%
\pgfsetstrokeopacity{0.572405}%
\pgfsetdash{}{0pt}%
\pgfpathmoveto{\pgfqpoint{0.953227in}{1.498466in}}%
\pgfpathcurveto{\pgfqpoint{0.961463in}{1.498466in}}{\pgfqpoint{0.969363in}{1.501738in}}{\pgfqpoint{0.975187in}{1.507562in}}%
\pgfpathcurveto{\pgfqpoint{0.981011in}{1.513386in}}{\pgfqpoint{0.984283in}{1.521286in}}{\pgfqpoint{0.984283in}{1.529523in}}%
\pgfpathcurveto{\pgfqpoint{0.984283in}{1.537759in}}{\pgfqpoint{0.981011in}{1.545659in}}{\pgfqpoint{0.975187in}{1.551483in}}%
\pgfpathcurveto{\pgfqpoint{0.969363in}{1.557307in}}{\pgfqpoint{0.961463in}{1.560579in}}{\pgfqpoint{0.953227in}{1.560579in}}%
\pgfpathcurveto{\pgfqpoint{0.944990in}{1.560579in}}{\pgfqpoint{0.937090in}{1.557307in}}{\pgfqpoint{0.931266in}{1.551483in}}%
\pgfpathcurveto{\pgfqpoint{0.925442in}{1.545659in}}{\pgfqpoint{0.922170in}{1.537759in}}{\pgfqpoint{0.922170in}{1.529523in}}%
\pgfpathcurveto{\pgfqpoint{0.922170in}{1.521286in}}{\pgfqpoint{0.925442in}{1.513386in}}{\pgfqpoint{0.931266in}{1.507562in}}%
\pgfpathcurveto{\pgfqpoint{0.937090in}{1.501738in}}{\pgfqpoint{0.944990in}{1.498466in}}{\pgfqpoint{0.953227in}{1.498466in}}%
\pgfpathclose%
\pgfusepath{stroke,fill}%
\end{pgfscope}%
\begin{pgfscope}%
\pgfpathrectangle{\pgfqpoint{0.100000in}{0.220728in}}{\pgfqpoint{3.696000in}{3.696000in}}%
\pgfusepath{clip}%
\pgfsetbuttcap%
\pgfsetroundjoin%
\definecolor{currentfill}{rgb}{0.121569,0.466667,0.705882}%
\pgfsetfillcolor{currentfill}%
\pgfsetfillopacity{0.572575}%
\pgfsetlinewidth{1.003750pt}%
\definecolor{currentstroke}{rgb}{0.121569,0.466667,0.705882}%
\pgfsetstrokecolor{currentstroke}%
\pgfsetstrokeopacity{0.572575}%
\pgfsetdash{}{0pt}%
\pgfpathmoveto{\pgfqpoint{2.924182in}{3.018207in}}%
\pgfpathcurveto{\pgfqpoint{2.932419in}{3.018207in}}{\pgfqpoint{2.940319in}{3.021479in}}{\pgfqpoint{2.946143in}{3.027303in}}%
\pgfpathcurveto{\pgfqpoint{2.951967in}{3.033127in}}{\pgfqpoint{2.955239in}{3.041027in}}{\pgfqpoint{2.955239in}{3.049263in}}%
\pgfpathcurveto{\pgfqpoint{2.955239in}{3.057500in}}{\pgfqpoint{2.951967in}{3.065400in}}{\pgfqpoint{2.946143in}{3.071224in}}%
\pgfpathcurveto{\pgfqpoint{2.940319in}{3.077047in}}{\pgfqpoint{2.932419in}{3.080320in}}{\pgfqpoint{2.924182in}{3.080320in}}%
\pgfpathcurveto{\pgfqpoint{2.915946in}{3.080320in}}{\pgfqpoint{2.908046in}{3.077047in}}{\pgfqpoint{2.902222in}{3.071224in}}%
\pgfpathcurveto{\pgfqpoint{2.896398in}{3.065400in}}{\pgfqpoint{2.893126in}{3.057500in}}{\pgfqpoint{2.893126in}{3.049263in}}%
\pgfpathcurveto{\pgfqpoint{2.893126in}{3.041027in}}{\pgfqpoint{2.896398in}{3.033127in}}{\pgfqpoint{2.902222in}{3.027303in}}%
\pgfpathcurveto{\pgfqpoint{2.908046in}{3.021479in}}{\pgfqpoint{2.915946in}{3.018207in}}{\pgfqpoint{2.924182in}{3.018207in}}%
\pgfpathclose%
\pgfusepath{stroke,fill}%
\end{pgfscope}%
\begin{pgfscope}%
\pgfpathrectangle{\pgfqpoint{0.100000in}{0.220728in}}{\pgfqpoint{3.696000in}{3.696000in}}%
\pgfusepath{clip}%
\pgfsetbuttcap%
\pgfsetroundjoin%
\definecolor{currentfill}{rgb}{0.121569,0.466667,0.705882}%
\pgfsetfillcolor{currentfill}%
\pgfsetfillopacity{0.572810}%
\pgfsetlinewidth{1.003750pt}%
\definecolor{currentstroke}{rgb}{0.121569,0.466667,0.705882}%
\pgfsetstrokecolor{currentstroke}%
\pgfsetstrokeopacity{0.572810}%
\pgfsetdash{}{0pt}%
\pgfpathmoveto{\pgfqpoint{0.952204in}{1.496431in}}%
\pgfpathcurveto{\pgfqpoint{0.960440in}{1.496431in}}{\pgfqpoint{0.968340in}{1.499703in}}{\pgfqpoint{0.974164in}{1.505527in}}%
\pgfpathcurveto{\pgfqpoint{0.979988in}{1.511351in}}{\pgfqpoint{0.983261in}{1.519251in}}{\pgfqpoint{0.983261in}{1.527487in}}%
\pgfpathcurveto{\pgfqpoint{0.983261in}{1.535723in}}{\pgfqpoint{0.979988in}{1.543623in}}{\pgfqpoint{0.974164in}{1.549447in}}%
\pgfpathcurveto{\pgfqpoint{0.968340in}{1.555271in}}{\pgfqpoint{0.960440in}{1.558544in}}{\pgfqpoint{0.952204in}{1.558544in}}%
\pgfpathcurveto{\pgfqpoint{0.943968in}{1.558544in}}{\pgfqpoint{0.936068in}{1.555271in}}{\pgfqpoint{0.930244in}{1.549447in}}%
\pgfpathcurveto{\pgfqpoint{0.924420in}{1.543623in}}{\pgfqpoint{0.921148in}{1.535723in}}{\pgfqpoint{0.921148in}{1.527487in}}%
\pgfpathcurveto{\pgfqpoint{0.921148in}{1.519251in}}{\pgfqpoint{0.924420in}{1.511351in}}{\pgfqpoint{0.930244in}{1.505527in}}%
\pgfpathcurveto{\pgfqpoint{0.936068in}{1.499703in}}{\pgfqpoint{0.943968in}{1.496431in}}{\pgfqpoint{0.952204in}{1.496431in}}%
\pgfpathclose%
\pgfusepath{stroke,fill}%
\end{pgfscope}%
\begin{pgfscope}%
\pgfpathrectangle{\pgfqpoint{0.100000in}{0.220728in}}{\pgfqpoint{3.696000in}{3.696000in}}%
\pgfusepath{clip}%
\pgfsetbuttcap%
\pgfsetroundjoin%
\definecolor{currentfill}{rgb}{0.121569,0.466667,0.705882}%
\pgfsetfillcolor{currentfill}%
\pgfsetfillopacity{0.573439}%
\pgfsetlinewidth{1.003750pt}%
\definecolor{currentstroke}{rgb}{0.121569,0.466667,0.705882}%
\pgfsetstrokecolor{currentstroke}%
\pgfsetstrokeopacity{0.573439}%
\pgfsetdash{}{0pt}%
\pgfpathmoveto{\pgfqpoint{0.949964in}{1.492784in}}%
\pgfpathcurveto{\pgfqpoint{0.958200in}{1.492784in}}{\pgfqpoint{0.966100in}{1.496056in}}{\pgfqpoint{0.971924in}{1.501880in}}%
\pgfpathcurveto{\pgfqpoint{0.977748in}{1.507704in}}{\pgfqpoint{0.981020in}{1.515604in}}{\pgfqpoint{0.981020in}{1.523841in}}%
\pgfpathcurveto{\pgfqpoint{0.981020in}{1.532077in}}{\pgfqpoint{0.977748in}{1.539977in}}{\pgfqpoint{0.971924in}{1.545801in}}%
\pgfpathcurveto{\pgfqpoint{0.966100in}{1.551625in}}{\pgfqpoint{0.958200in}{1.554897in}}{\pgfqpoint{0.949964in}{1.554897in}}%
\pgfpathcurveto{\pgfqpoint{0.941727in}{1.554897in}}{\pgfqpoint{0.933827in}{1.551625in}}{\pgfqpoint{0.928004in}{1.545801in}}%
\pgfpathcurveto{\pgfqpoint{0.922180in}{1.539977in}}{\pgfqpoint{0.918907in}{1.532077in}}{\pgfqpoint{0.918907in}{1.523841in}}%
\pgfpathcurveto{\pgfqpoint{0.918907in}{1.515604in}}{\pgfqpoint{0.922180in}{1.507704in}}{\pgfqpoint{0.928004in}{1.501880in}}%
\pgfpathcurveto{\pgfqpoint{0.933827in}{1.496056in}}{\pgfqpoint{0.941727in}{1.492784in}}{\pgfqpoint{0.949964in}{1.492784in}}%
\pgfpathclose%
\pgfusepath{stroke,fill}%
\end{pgfscope}%
\begin{pgfscope}%
\pgfpathrectangle{\pgfqpoint{0.100000in}{0.220728in}}{\pgfqpoint{3.696000in}{3.696000in}}%
\pgfusepath{clip}%
\pgfsetbuttcap%
\pgfsetroundjoin%
\definecolor{currentfill}{rgb}{0.121569,0.466667,0.705882}%
\pgfsetfillcolor{currentfill}%
\pgfsetfillopacity{0.574586}%
\pgfsetlinewidth{1.003750pt}%
\definecolor{currentstroke}{rgb}{0.121569,0.466667,0.705882}%
\pgfsetstrokecolor{currentstroke}%
\pgfsetstrokeopacity{0.574586}%
\pgfsetdash{}{0pt}%
\pgfpathmoveto{\pgfqpoint{2.933012in}{3.016288in}}%
\pgfpathcurveto{\pgfqpoint{2.941248in}{3.016288in}}{\pgfqpoint{2.949148in}{3.019560in}}{\pgfqpoint{2.954972in}{3.025384in}}%
\pgfpathcurveto{\pgfqpoint{2.960796in}{3.031208in}}{\pgfqpoint{2.964068in}{3.039108in}}{\pgfqpoint{2.964068in}{3.047344in}}%
\pgfpathcurveto{\pgfqpoint{2.964068in}{3.055580in}}{\pgfqpoint{2.960796in}{3.063480in}}{\pgfqpoint{2.954972in}{3.069304in}}%
\pgfpathcurveto{\pgfqpoint{2.949148in}{3.075128in}}{\pgfqpoint{2.941248in}{3.078401in}}{\pgfqpoint{2.933012in}{3.078401in}}%
\pgfpathcurveto{\pgfqpoint{2.924775in}{3.078401in}}{\pgfqpoint{2.916875in}{3.075128in}}{\pgfqpoint{2.911051in}{3.069304in}}%
\pgfpathcurveto{\pgfqpoint{2.905227in}{3.063480in}}{\pgfqpoint{2.901955in}{3.055580in}}{\pgfqpoint{2.901955in}{3.047344in}}%
\pgfpathcurveto{\pgfqpoint{2.901955in}{3.039108in}}{\pgfqpoint{2.905227in}{3.031208in}}{\pgfqpoint{2.911051in}{3.025384in}}%
\pgfpathcurveto{\pgfqpoint{2.916875in}{3.019560in}}{\pgfqpoint{2.924775in}{3.016288in}}{\pgfqpoint{2.933012in}{3.016288in}}%
\pgfpathclose%
\pgfusepath{stroke,fill}%
\end{pgfscope}%
\begin{pgfscope}%
\pgfpathrectangle{\pgfqpoint{0.100000in}{0.220728in}}{\pgfqpoint{3.696000in}{3.696000in}}%
\pgfusepath{clip}%
\pgfsetbuttcap%
\pgfsetroundjoin%
\definecolor{currentfill}{rgb}{0.121569,0.466667,0.705882}%
\pgfsetfillcolor{currentfill}%
\pgfsetfillopacity{0.574851}%
\pgfsetlinewidth{1.003750pt}%
\definecolor{currentstroke}{rgb}{0.121569,0.466667,0.705882}%
\pgfsetstrokecolor{currentstroke}%
\pgfsetstrokeopacity{0.574851}%
\pgfsetdash{}{0pt}%
\pgfpathmoveto{\pgfqpoint{0.947297in}{1.485730in}}%
\pgfpathcurveto{\pgfqpoint{0.955533in}{1.485730in}}{\pgfqpoint{0.963433in}{1.489003in}}{\pgfqpoint{0.969257in}{1.494826in}}%
\pgfpathcurveto{\pgfqpoint{0.975081in}{1.500650in}}{\pgfqpoint{0.978353in}{1.508550in}}{\pgfqpoint{0.978353in}{1.516787in}}%
\pgfpathcurveto{\pgfqpoint{0.978353in}{1.525023in}}{\pgfqpoint{0.975081in}{1.532923in}}{\pgfqpoint{0.969257in}{1.538747in}}%
\pgfpathcurveto{\pgfqpoint{0.963433in}{1.544571in}}{\pgfqpoint{0.955533in}{1.547843in}}{\pgfqpoint{0.947297in}{1.547843in}}%
\pgfpathcurveto{\pgfqpoint{0.939061in}{1.547843in}}{\pgfqpoint{0.931161in}{1.544571in}}{\pgfqpoint{0.925337in}{1.538747in}}%
\pgfpathcurveto{\pgfqpoint{0.919513in}{1.532923in}}{\pgfqpoint{0.916240in}{1.525023in}}{\pgfqpoint{0.916240in}{1.516787in}}%
\pgfpathcurveto{\pgfqpoint{0.916240in}{1.508550in}}{\pgfqpoint{0.919513in}{1.500650in}}{\pgfqpoint{0.925337in}{1.494826in}}%
\pgfpathcurveto{\pgfqpoint{0.931161in}{1.489003in}}{\pgfqpoint{0.939061in}{1.485730in}}{\pgfqpoint{0.947297in}{1.485730in}}%
\pgfpathclose%
\pgfusepath{stroke,fill}%
\end{pgfscope}%
\begin{pgfscope}%
\pgfpathrectangle{\pgfqpoint{0.100000in}{0.220728in}}{\pgfqpoint{3.696000in}{3.696000in}}%
\pgfusepath{clip}%
\pgfsetbuttcap%
\pgfsetroundjoin%
\definecolor{currentfill}{rgb}{0.121569,0.466667,0.705882}%
\pgfsetfillcolor{currentfill}%
\pgfsetfillopacity{0.575533}%
\pgfsetlinewidth{1.003750pt}%
\definecolor{currentstroke}{rgb}{0.121569,0.466667,0.705882}%
\pgfsetstrokecolor{currentstroke}%
\pgfsetstrokeopacity{0.575533}%
\pgfsetdash{}{0pt}%
\pgfpathmoveto{\pgfqpoint{0.944201in}{1.481438in}}%
\pgfpathcurveto{\pgfqpoint{0.952437in}{1.481438in}}{\pgfqpoint{0.960338in}{1.484710in}}{\pgfqpoint{0.966161in}{1.490534in}}%
\pgfpathcurveto{\pgfqpoint{0.971985in}{1.496358in}}{\pgfqpoint{0.975258in}{1.504258in}}{\pgfqpoint{0.975258in}{1.512495in}}%
\pgfpathcurveto{\pgfqpoint{0.975258in}{1.520731in}}{\pgfqpoint{0.971985in}{1.528631in}}{\pgfqpoint{0.966161in}{1.534455in}}%
\pgfpathcurveto{\pgfqpoint{0.960338in}{1.540279in}}{\pgfqpoint{0.952437in}{1.543551in}}{\pgfqpoint{0.944201in}{1.543551in}}%
\pgfpathcurveto{\pgfqpoint{0.935965in}{1.543551in}}{\pgfqpoint{0.928065in}{1.540279in}}{\pgfqpoint{0.922241in}{1.534455in}}%
\pgfpathcurveto{\pgfqpoint{0.916417in}{1.528631in}}{\pgfqpoint{0.913145in}{1.520731in}}{\pgfqpoint{0.913145in}{1.512495in}}%
\pgfpathcurveto{\pgfqpoint{0.913145in}{1.504258in}}{\pgfqpoint{0.916417in}{1.496358in}}{\pgfqpoint{0.922241in}{1.490534in}}%
\pgfpathcurveto{\pgfqpoint{0.928065in}{1.484710in}}{\pgfqpoint{0.935965in}{1.481438in}}{\pgfqpoint{0.944201in}{1.481438in}}%
\pgfpathclose%
\pgfusepath{stroke,fill}%
\end{pgfscope}%
\begin{pgfscope}%
\pgfpathrectangle{\pgfqpoint{0.100000in}{0.220728in}}{\pgfqpoint{3.696000in}{3.696000in}}%
\pgfusepath{clip}%
\pgfsetbuttcap%
\pgfsetroundjoin%
\definecolor{currentfill}{rgb}{0.121569,0.466667,0.705882}%
\pgfsetfillcolor{currentfill}%
\pgfsetfillopacity{0.577181}%
\pgfsetlinewidth{1.003750pt}%
\definecolor{currentstroke}{rgb}{0.121569,0.466667,0.705882}%
\pgfsetstrokecolor{currentstroke}%
\pgfsetstrokeopacity{0.577181}%
\pgfsetdash{}{0pt}%
\pgfpathmoveto{\pgfqpoint{2.943488in}{3.015413in}}%
\pgfpathcurveto{\pgfqpoint{2.951724in}{3.015413in}}{\pgfqpoint{2.959624in}{3.018686in}}{\pgfqpoint{2.965448in}{3.024510in}}%
\pgfpathcurveto{\pgfqpoint{2.971272in}{3.030334in}}{\pgfqpoint{2.974544in}{3.038234in}}{\pgfqpoint{2.974544in}{3.046470in}}%
\pgfpathcurveto{\pgfqpoint{2.974544in}{3.054706in}}{\pgfqpoint{2.971272in}{3.062606in}}{\pgfqpoint{2.965448in}{3.068430in}}%
\pgfpathcurveto{\pgfqpoint{2.959624in}{3.074254in}}{\pgfqpoint{2.951724in}{3.077526in}}{\pgfqpoint{2.943488in}{3.077526in}}%
\pgfpathcurveto{\pgfqpoint{2.935251in}{3.077526in}}{\pgfqpoint{2.927351in}{3.074254in}}{\pgfqpoint{2.921527in}{3.068430in}}%
\pgfpathcurveto{\pgfqpoint{2.915703in}{3.062606in}}{\pgfqpoint{2.912431in}{3.054706in}}{\pgfqpoint{2.912431in}{3.046470in}}%
\pgfpathcurveto{\pgfqpoint{2.912431in}{3.038234in}}{\pgfqpoint{2.915703in}{3.030334in}}{\pgfqpoint{2.921527in}{3.024510in}}%
\pgfpathcurveto{\pgfqpoint{2.927351in}{3.018686in}}{\pgfqpoint{2.935251in}{3.015413in}}{\pgfqpoint{2.943488in}{3.015413in}}%
\pgfpathclose%
\pgfusepath{stroke,fill}%
\end{pgfscope}%
\begin{pgfscope}%
\pgfpathrectangle{\pgfqpoint{0.100000in}{0.220728in}}{\pgfqpoint{3.696000in}{3.696000in}}%
\pgfusepath{clip}%
\pgfsetbuttcap%
\pgfsetroundjoin%
\definecolor{currentfill}{rgb}{0.121569,0.466667,0.705882}%
\pgfsetfillcolor{currentfill}%
\pgfsetfillopacity{0.577203}%
\pgfsetlinewidth{1.003750pt}%
\definecolor{currentstroke}{rgb}{0.121569,0.466667,0.705882}%
\pgfsetstrokecolor{currentstroke}%
\pgfsetstrokeopacity{0.577203}%
\pgfsetdash{}{0pt}%
\pgfpathmoveto{\pgfqpoint{0.941129in}{1.472349in}}%
\pgfpathcurveto{\pgfqpoint{0.949365in}{1.472349in}}{\pgfqpoint{0.957265in}{1.475622in}}{\pgfqpoint{0.963089in}{1.481446in}}%
\pgfpathcurveto{\pgfqpoint{0.968913in}{1.487269in}}{\pgfqpoint{0.972185in}{1.495170in}}{\pgfqpoint{0.972185in}{1.503406in}}%
\pgfpathcurveto{\pgfqpoint{0.972185in}{1.511642in}}{\pgfqpoint{0.968913in}{1.519542in}}{\pgfqpoint{0.963089in}{1.525366in}}%
\pgfpathcurveto{\pgfqpoint{0.957265in}{1.531190in}}{\pgfqpoint{0.949365in}{1.534462in}}{\pgfqpoint{0.941129in}{1.534462in}}%
\pgfpathcurveto{\pgfqpoint{0.932892in}{1.534462in}}{\pgfqpoint{0.924992in}{1.531190in}}{\pgfqpoint{0.919168in}{1.525366in}}%
\pgfpathcurveto{\pgfqpoint{0.913344in}{1.519542in}}{\pgfqpoint{0.910072in}{1.511642in}}{\pgfqpoint{0.910072in}{1.503406in}}%
\pgfpathcurveto{\pgfqpoint{0.910072in}{1.495170in}}{\pgfqpoint{0.913344in}{1.487269in}}{\pgfqpoint{0.919168in}{1.481446in}}%
\pgfpathcurveto{\pgfqpoint{0.924992in}{1.475622in}}{\pgfqpoint{0.932892in}{1.472349in}}{\pgfqpoint{0.941129in}{1.472349in}}%
\pgfpathclose%
\pgfusepath{stroke,fill}%
\end{pgfscope}%
\begin{pgfscope}%
\pgfpathrectangle{\pgfqpoint{0.100000in}{0.220728in}}{\pgfqpoint{3.696000in}{3.696000in}}%
\pgfusepath{clip}%
\pgfsetbuttcap%
\pgfsetroundjoin%
\definecolor{currentfill}{rgb}{0.121569,0.466667,0.705882}%
\pgfsetfillcolor{currentfill}%
\pgfsetfillopacity{0.578078}%
\pgfsetlinewidth{1.003750pt}%
\definecolor{currentstroke}{rgb}{0.121569,0.466667,0.705882}%
\pgfsetstrokecolor{currentstroke}%
\pgfsetstrokeopacity{0.578078}%
\pgfsetdash{}{0pt}%
\pgfpathmoveto{\pgfqpoint{0.937013in}{1.466610in}}%
\pgfpathcurveto{\pgfqpoint{0.945249in}{1.466610in}}{\pgfqpoint{0.953149in}{1.469882in}}{\pgfqpoint{0.958973in}{1.475706in}}%
\pgfpathcurveto{\pgfqpoint{0.964797in}{1.481530in}}{\pgfqpoint{0.968069in}{1.489430in}}{\pgfqpoint{0.968069in}{1.497667in}}%
\pgfpathcurveto{\pgfqpoint{0.968069in}{1.505903in}}{\pgfqpoint{0.964797in}{1.513803in}}{\pgfqpoint{0.958973in}{1.519627in}}%
\pgfpathcurveto{\pgfqpoint{0.953149in}{1.525451in}}{\pgfqpoint{0.945249in}{1.528723in}}{\pgfqpoint{0.937013in}{1.528723in}}%
\pgfpathcurveto{\pgfqpoint{0.928776in}{1.528723in}}{\pgfqpoint{0.920876in}{1.525451in}}{\pgfqpoint{0.915052in}{1.519627in}}%
\pgfpathcurveto{\pgfqpoint{0.909229in}{1.513803in}}{\pgfqpoint{0.905956in}{1.505903in}}{\pgfqpoint{0.905956in}{1.497667in}}%
\pgfpathcurveto{\pgfqpoint{0.905956in}{1.489430in}}{\pgfqpoint{0.909229in}{1.481530in}}{\pgfqpoint{0.915052in}{1.475706in}}%
\pgfpathcurveto{\pgfqpoint{0.920876in}{1.469882in}}{\pgfqpoint{0.928776in}{1.466610in}}{\pgfqpoint{0.937013in}{1.466610in}}%
\pgfpathclose%
\pgfusepath{stroke,fill}%
\end{pgfscope}%
\begin{pgfscope}%
\pgfpathrectangle{\pgfqpoint{0.100000in}{0.220728in}}{\pgfqpoint{3.696000in}{3.696000in}}%
\pgfusepath{clip}%
\pgfsetbuttcap%
\pgfsetroundjoin%
\definecolor{currentfill}{rgb}{0.121569,0.466667,0.705882}%
\pgfsetfillcolor{currentfill}%
\pgfsetfillopacity{0.578679}%
\pgfsetlinewidth{1.003750pt}%
\definecolor{currentstroke}{rgb}{0.121569,0.466667,0.705882}%
\pgfsetstrokecolor{currentstroke}%
\pgfsetstrokeopacity{0.578679}%
\pgfsetdash{}{0pt}%
\pgfpathmoveto{\pgfqpoint{2.955977in}{3.013598in}}%
\pgfpathcurveto{\pgfqpoint{2.964214in}{3.013598in}}{\pgfqpoint{2.972114in}{3.016870in}}{\pgfqpoint{2.977938in}{3.022694in}}%
\pgfpathcurveto{\pgfqpoint{2.983761in}{3.028518in}}{\pgfqpoint{2.987034in}{3.036418in}}{\pgfqpoint{2.987034in}{3.044655in}}%
\pgfpathcurveto{\pgfqpoint{2.987034in}{3.052891in}}{\pgfqpoint{2.983761in}{3.060791in}}{\pgfqpoint{2.977938in}{3.066615in}}%
\pgfpathcurveto{\pgfqpoint{2.972114in}{3.072439in}}{\pgfqpoint{2.964214in}{3.075711in}}{\pgfqpoint{2.955977in}{3.075711in}}%
\pgfpathcurveto{\pgfqpoint{2.947741in}{3.075711in}}{\pgfqpoint{2.939841in}{3.072439in}}{\pgfqpoint{2.934017in}{3.066615in}}%
\pgfpathcurveto{\pgfqpoint{2.928193in}{3.060791in}}{\pgfqpoint{2.924921in}{3.052891in}}{\pgfqpoint{2.924921in}{3.044655in}}%
\pgfpathcurveto{\pgfqpoint{2.924921in}{3.036418in}}{\pgfqpoint{2.928193in}{3.028518in}}{\pgfqpoint{2.934017in}{3.022694in}}%
\pgfpathcurveto{\pgfqpoint{2.939841in}{3.016870in}}{\pgfqpoint{2.947741in}{3.013598in}}{\pgfqpoint{2.955977in}{3.013598in}}%
\pgfpathclose%
\pgfusepath{stroke,fill}%
\end{pgfscope}%
\begin{pgfscope}%
\pgfpathrectangle{\pgfqpoint{0.100000in}{0.220728in}}{\pgfqpoint{3.696000in}{3.696000in}}%
\pgfusepath{clip}%
\pgfsetbuttcap%
\pgfsetroundjoin%
\definecolor{currentfill}{rgb}{0.121569,0.466667,0.705882}%
\pgfsetfillcolor{currentfill}%
\pgfsetfillopacity{0.580185}%
\pgfsetlinewidth{1.003750pt}%
\definecolor{currentstroke}{rgb}{0.121569,0.466667,0.705882}%
\pgfsetstrokecolor{currentstroke}%
\pgfsetstrokeopacity{0.580185}%
\pgfsetdash{}{0pt}%
\pgfpathmoveto{\pgfqpoint{0.932219in}{1.454716in}}%
\pgfpathcurveto{\pgfqpoint{0.940456in}{1.454716in}}{\pgfqpoint{0.948356in}{1.457988in}}{\pgfqpoint{0.954180in}{1.463812in}}%
\pgfpathcurveto{\pgfqpoint{0.960004in}{1.469636in}}{\pgfqpoint{0.963276in}{1.477536in}}{\pgfqpoint{0.963276in}{1.485772in}}%
\pgfpathcurveto{\pgfqpoint{0.963276in}{1.494008in}}{\pgfqpoint{0.960004in}{1.501908in}}{\pgfqpoint{0.954180in}{1.507732in}}%
\pgfpathcurveto{\pgfqpoint{0.948356in}{1.513556in}}{\pgfqpoint{0.940456in}{1.516829in}}{\pgfqpoint{0.932219in}{1.516829in}}%
\pgfpathcurveto{\pgfqpoint{0.923983in}{1.516829in}}{\pgfqpoint{0.916083in}{1.513556in}}{\pgfqpoint{0.910259in}{1.507732in}}%
\pgfpathcurveto{\pgfqpoint{0.904435in}{1.501908in}}{\pgfqpoint{0.901163in}{1.494008in}}{\pgfqpoint{0.901163in}{1.485772in}}%
\pgfpathcurveto{\pgfqpoint{0.901163in}{1.477536in}}{\pgfqpoint{0.904435in}{1.469636in}}{\pgfqpoint{0.910259in}{1.463812in}}%
\pgfpathcurveto{\pgfqpoint{0.916083in}{1.457988in}}{\pgfqpoint{0.923983in}{1.454716in}}{\pgfqpoint{0.932219in}{1.454716in}}%
\pgfpathclose%
\pgfusepath{stroke,fill}%
\end{pgfscope}%
\begin{pgfscope}%
\pgfpathrectangle{\pgfqpoint{0.100000in}{0.220728in}}{\pgfqpoint{3.696000in}{3.696000in}}%
\pgfusepath{clip}%
\pgfsetbuttcap%
\pgfsetroundjoin%
\definecolor{currentfill}{rgb}{0.121569,0.466667,0.705882}%
\pgfsetfillcolor{currentfill}%
\pgfsetfillopacity{0.580231}%
\pgfsetlinewidth{1.003750pt}%
\definecolor{currentstroke}{rgb}{0.121569,0.466667,0.705882}%
\pgfsetstrokecolor{currentstroke}%
\pgfsetstrokeopacity{0.580231}%
\pgfsetdash{}{0pt}%
\pgfpathmoveto{\pgfqpoint{2.961956in}{3.011931in}}%
\pgfpathcurveto{\pgfqpoint{2.970192in}{3.011931in}}{\pgfqpoint{2.978092in}{3.015204in}}{\pgfqpoint{2.983916in}{3.021028in}}%
\pgfpathcurveto{\pgfqpoint{2.989740in}{3.026851in}}{\pgfqpoint{2.993012in}{3.034752in}}{\pgfqpoint{2.993012in}{3.042988in}}%
\pgfpathcurveto{\pgfqpoint{2.993012in}{3.051224in}}{\pgfqpoint{2.989740in}{3.059124in}}{\pgfqpoint{2.983916in}{3.064948in}}%
\pgfpathcurveto{\pgfqpoint{2.978092in}{3.070772in}}{\pgfqpoint{2.970192in}{3.074044in}}{\pgfqpoint{2.961956in}{3.074044in}}%
\pgfpathcurveto{\pgfqpoint{2.953719in}{3.074044in}}{\pgfqpoint{2.945819in}{3.070772in}}{\pgfqpoint{2.939995in}{3.064948in}}%
\pgfpathcurveto{\pgfqpoint{2.934171in}{3.059124in}}{\pgfqpoint{2.930899in}{3.051224in}}{\pgfqpoint{2.930899in}{3.042988in}}%
\pgfpathcurveto{\pgfqpoint{2.930899in}{3.034752in}}{\pgfqpoint{2.934171in}{3.026851in}}{\pgfqpoint{2.939995in}{3.021028in}}%
\pgfpathcurveto{\pgfqpoint{2.945819in}{3.015204in}}{\pgfqpoint{2.953719in}{3.011931in}}{\pgfqpoint{2.961956in}{3.011931in}}%
\pgfpathclose%
\pgfusepath{stroke,fill}%
\end{pgfscope}%
\begin{pgfscope}%
\pgfpathrectangle{\pgfqpoint{0.100000in}{0.220728in}}{\pgfqpoint{3.696000in}{3.696000in}}%
\pgfusepath{clip}%
\pgfsetbuttcap%
\pgfsetroundjoin%
\definecolor{currentfill}{rgb}{0.121569,0.466667,0.705882}%
\pgfsetfillcolor{currentfill}%
\pgfsetfillopacity{0.580908}%
\pgfsetlinewidth{1.003750pt}%
\definecolor{currentstroke}{rgb}{0.121569,0.466667,0.705882}%
\pgfsetstrokecolor{currentstroke}%
\pgfsetstrokeopacity{0.580908}%
\pgfsetdash{}{0pt}%
\pgfpathmoveto{\pgfqpoint{2.970002in}{3.010648in}}%
\pgfpathcurveto{\pgfqpoint{2.978238in}{3.010648in}}{\pgfqpoint{2.986138in}{3.013921in}}{\pgfqpoint{2.991962in}{3.019745in}}%
\pgfpathcurveto{\pgfqpoint{2.997786in}{3.025569in}}{\pgfqpoint{3.001059in}{3.033469in}}{\pgfqpoint{3.001059in}{3.041705in}}%
\pgfpathcurveto{\pgfqpoint{3.001059in}{3.049941in}}{\pgfqpoint{2.997786in}{3.057841in}}{\pgfqpoint{2.991962in}{3.063665in}}%
\pgfpathcurveto{\pgfqpoint{2.986138in}{3.069489in}}{\pgfqpoint{2.978238in}{3.072761in}}{\pgfqpoint{2.970002in}{3.072761in}}%
\pgfpathcurveto{\pgfqpoint{2.961766in}{3.072761in}}{\pgfqpoint{2.953866in}{3.069489in}}{\pgfqpoint{2.948042in}{3.063665in}}%
\pgfpathcurveto{\pgfqpoint{2.942218in}{3.057841in}}{\pgfqpoint{2.938946in}{3.049941in}}{\pgfqpoint{2.938946in}{3.041705in}}%
\pgfpathcurveto{\pgfqpoint{2.938946in}{3.033469in}}{\pgfqpoint{2.942218in}{3.025569in}}{\pgfqpoint{2.948042in}{3.019745in}}%
\pgfpathcurveto{\pgfqpoint{2.953866in}{3.013921in}}{\pgfqpoint{2.961766in}{3.010648in}}{\pgfqpoint{2.970002in}{3.010648in}}%
\pgfpathclose%
\pgfusepath{stroke,fill}%
\end{pgfscope}%
\begin{pgfscope}%
\pgfpathrectangle{\pgfqpoint{0.100000in}{0.220728in}}{\pgfqpoint{3.696000in}{3.696000in}}%
\pgfusepath{clip}%
\pgfsetbuttcap%
\pgfsetroundjoin%
\definecolor{currentfill}{rgb}{0.121569,0.466667,0.705882}%
\pgfsetfillcolor{currentfill}%
\pgfsetfillopacity{0.581331}%
\pgfsetlinewidth{1.003750pt}%
\definecolor{currentstroke}{rgb}{0.121569,0.466667,0.705882}%
\pgfsetstrokecolor{currentstroke}%
\pgfsetstrokeopacity{0.581331}%
\pgfsetdash{}{0pt}%
\pgfpathmoveto{\pgfqpoint{0.926005in}{1.445467in}}%
\pgfpathcurveto{\pgfqpoint{0.934242in}{1.445467in}}{\pgfqpoint{0.942142in}{1.448740in}}{\pgfqpoint{0.947966in}{1.454563in}}%
\pgfpathcurveto{\pgfqpoint{0.953789in}{1.460387in}}{\pgfqpoint{0.957062in}{1.468287in}}{\pgfqpoint{0.957062in}{1.476524in}}%
\pgfpathcurveto{\pgfqpoint{0.957062in}{1.484760in}}{\pgfqpoint{0.953789in}{1.492660in}}{\pgfqpoint{0.947966in}{1.498484in}}%
\pgfpathcurveto{\pgfqpoint{0.942142in}{1.504308in}}{\pgfqpoint{0.934242in}{1.507580in}}{\pgfqpoint{0.926005in}{1.507580in}}%
\pgfpathcurveto{\pgfqpoint{0.917769in}{1.507580in}}{\pgfqpoint{0.909869in}{1.504308in}}{\pgfqpoint{0.904045in}{1.498484in}}%
\pgfpathcurveto{\pgfqpoint{0.898221in}{1.492660in}}{\pgfqpoint{0.894949in}{1.484760in}}{\pgfqpoint{0.894949in}{1.476524in}}%
\pgfpathcurveto{\pgfqpoint{0.894949in}{1.468287in}}{\pgfqpoint{0.898221in}{1.460387in}}{\pgfqpoint{0.904045in}{1.454563in}}%
\pgfpathcurveto{\pgfqpoint{0.909869in}{1.448740in}}{\pgfqpoint{0.917769in}{1.445467in}}{\pgfqpoint{0.926005in}{1.445467in}}%
\pgfpathclose%
\pgfusepath{stroke,fill}%
\end{pgfscope}%
\begin{pgfscope}%
\pgfpathrectangle{\pgfqpoint{0.100000in}{0.220728in}}{\pgfqpoint{3.696000in}{3.696000in}}%
\pgfusepath{clip}%
\pgfsetbuttcap%
\pgfsetroundjoin%
\definecolor{currentfill}{rgb}{0.121569,0.466667,0.705882}%
\pgfsetfillcolor{currentfill}%
\pgfsetfillopacity{0.581926}%
\pgfsetlinewidth{1.003750pt}%
\definecolor{currentstroke}{rgb}{0.121569,0.466667,0.705882}%
\pgfsetstrokecolor{currentstroke}%
\pgfsetstrokeopacity{0.581926}%
\pgfsetdash{}{0pt}%
\pgfpathmoveto{\pgfqpoint{2.979431in}{3.008459in}}%
\pgfpathcurveto{\pgfqpoint{2.987668in}{3.008459in}}{\pgfqpoint{2.995568in}{3.011731in}}{\pgfqpoint{3.001392in}{3.017555in}}%
\pgfpathcurveto{\pgfqpoint{3.007215in}{3.023379in}}{\pgfqpoint{3.010488in}{3.031279in}}{\pgfqpoint{3.010488in}{3.039515in}}%
\pgfpathcurveto{\pgfqpoint{3.010488in}{3.047752in}}{\pgfqpoint{3.007215in}{3.055652in}}{\pgfqpoint{3.001392in}{3.061476in}}%
\pgfpathcurveto{\pgfqpoint{2.995568in}{3.067300in}}{\pgfqpoint{2.987668in}{3.070572in}}{\pgfqpoint{2.979431in}{3.070572in}}%
\pgfpathcurveto{\pgfqpoint{2.971195in}{3.070572in}}{\pgfqpoint{2.963295in}{3.067300in}}{\pgfqpoint{2.957471in}{3.061476in}}%
\pgfpathcurveto{\pgfqpoint{2.951647in}{3.055652in}}{\pgfqpoint{2.948375in}{3.047752in}}{\pgfqpoint{2.948375in}{3.039515in}}%
\pgfpathcurveto{\pgfqpoint{2.948375in}{3.031279in}}{\pgfqpoint{2.951647in}{3.023379in}}{\pgfqpoint{2.957471in}{3.017555in}}%
\pgfpathcurveto{\pgfqpoint{2.963295in}{3.011731in}}{\pgfqpoint{2.971195in}{3.008459in}}{\pgfqpoint{2.979431in}{3.008459in}}%
\pgfpathclose%
\pgfusepath{stroke,fill}%
\end{pgfscope}%
\begin{pgfscope}%
\pgfpathrectangle{\pgfqpoint{0.100000in}{0.220728in}}{\pgfqpoint{3.696000in}{3.696000in}}%
\pgfusepath{clip}%
\pgfsetbuttcap%
\pgfsetroundjoin%
\definecolor{currentfill}{rgb}{0.121569,0.466667,0.705882}%
\pgfsetfillcolor{currentfill}%
\pgfsetfillopacity{0.582935}%
\pgfsetlinewidth{1.003750pt}%
\definecolor{currentstroke}{rgb}{0.121569,0.466667,0.705882}%
\pgfsetstrokecolor{currentstroke}%
\pgfsetstrokeopacity{0.582935}%
\pgfsetdash{}{0pt}%
\pgfpathmoveto{\pgfqpoint{0.921821in}{1.437060in}}%
\pgfpathcurveto{\pgfqpoint{0.930058in}{1.437060in}}{\pgfqpoint{0.937958in}{1.440332in}}{\pgfqpoint{0.943782in}{1.446156in}}%
\pgfpathcurveto{\pgfqpoint{0.949605in}{1.451980in}}{\pgfqpoint{0.952878in}{1.459880in}}{\pgfqpoint{0.952878in}{1.468116in}}%
\pgfpathcurveto{\pgfqpoint{0.952878in}{1.476352in}}{\pgfqpoint{0.949605in}{1.484252in}}{\pgfqpoint{0.943782in}{1.490076in}}%
\pgfpathcurveto{\pgfqpoint{0.937958in}{1.495900in}}{\pgfqpoint{0.930058in}{1.499173in}}{\pgfqpoint{0.921821in}{1.499173in}}%
\pgfpathcurveto{\pgfqpoint{0.913585in}{1.499173in}}{\pgfqpoint{0.905685in}{1.495900in}}{\pgfqpoint{0.899861in}{1.490076in}}%
\pgfpathcurveto{\pgfqpoint{0.894037in}{1.484252in}}{\pgfqpoint{0.890765in}{1.476352in}}{\pgfqpoint{0.890765in}{1.468116in}}%
\pgfpathcurveto{\pgfqpoint{0.890765in}{1.459880in}}{\pgfqpoint{0.894037in}{1.451980in}}{\pgfqpoint{0.899861in}{1.446156in}}%
\pgfpathcurveto{\pgfqpoint{0.905685in}{1.440332in}}{\pgfqpoint{0.913585in}{1.437060in}}{\pgfqpoint{0.921821in}{1.437060in}}%
\pgfpathclose%
\pgfusepath{stroke,fill}%
\end{pgfscope}%
\begin{pgfscope}%
\pgfpathrectangle{\pgfqpoint{0.100000in}{0.220728in}}{\pgfqpoint{3.696000in}{3.696000in}}%
\pgfusepath{clip}%
\pgfsetbuttcap%
\pgfsetroundjoin%
\definecolor{currentfill}{rgb}{0.121569,0.466667,0.705882}%
\pgfsetfillcolor{currentfill}%
\pgfsetfillopacity{0.583807}%
\pgfsetlinewidth{1.003750pt}%
\definecolor{currentstroke}{rgb}{0.121569,0.466667,0.705882}%
\pgfsetstrokecolor{currentstroke}%
\pgfsetstrokeopacity{0.583807}%
\pgfsetdash{}{0pt}%
\pgfpathmoveto{\pgfqpoint{0.917628in}{1.430705in}}%
\pgfpathcurveto{\pgfqpoint{0.925865in}{1.430705in}}{\pgfqpoint{0.933765in}{1.433977in}}{\pgfqpoint{0.939589in}{1.439801in}}%
\pgfpathcurveto{\pgfqpoint{0.945412in}{1.445625in}}{\pgfqpoint{0.948685in}{1.453525in}}{\pgfqpoint{0.948685in}{1.461761in}}%
\pgfpathcurveto{\pgfqpoint{0.948685in}{1.469998in}}{\pgfqpoint{0.945412in}{1.477898in}}{\pgfqpoint{0.939589in}{1.483722in}}%
\pgfpathcurveto{\pgfqpoint{0.933765in}{1.489546in}}{\pgfqpoint{0.925865in}{1.492818in}}{\pgfqpoint{0.917628in}{1.492818in}}%
\pgfpathcurveto{\pgfqpoint{0.909392in}{1.492818in}}{\pgfqpoint{0.901492in}{1.489546in}}{\pgfqpoint{0.895668in}{1.483722in}}%
\pgfpathcurveto{\pgfqpoint{0.889844in}{1.477898in}}{\pgfqpoint{0.886572in}{1.469998in}}{\pgfqpoint{0.886572in}{1.461761in}}%
\pgfpathcurveto{\pgfqpoint{0.886572in}{1.453525in}}{\pgfqpoint{0.889844in}{1.445625in}}{\pgfqpoint{0.895668in}{1.439801in}}%
\pgfpathcurveto{\pgfqpoint{0.901492in}{1.433977in}}{\pgfqpoint{0.909392in}{1.430705in}}{\pgfqpoint{0.917628in}{1.430705in}}%
\pgfpathclose%
\pgfusepath{stroke,fill}%
\end{pgfscope}%
\begin{pgfscope}%
\pgfpathrectangle{\pgfqpoint{0.100000in}{0.220728in}}{\pgfqpoint{3.696000in}{3.696000in}}%
\pgfusepath{clip}%
\pgfsetbuttcap%
\pgfsetroundjoin%
\definecolor{currentfill}{rgb}{0.121569,0.466667,0.705882}%
\pgfsetfillcolor{currentfill}%
\pgfsetfillopacity{0.584748}%
\pgfsetlinewidth{1.003750pt}%
\definecolor{currentstroke}{rgb}{0.121569,0.466667,0.705882}%
\pgfsetstrokecolor{currentstroke}%
\pgfsetstrokeopacity{0.584748}%
\pgfsetdash{}{0pt}%
\pgfpathmoveto{\pgfqpoint{2.988497in}{3.007771in}}%
\pgfpathcurveto{\pgfqpoint{2.996734in}{3.007771in}}{\pgfqpoint{3.004634in}{3.011043in}}{\pgfqpoint{3.010458in}{3.016867in}}%
\pgfpathcurveto{\pgfqpoint{3.016282in}{3.022691in}}{\pgfqpoint{3.019554in}{3.030591in}}{\pgfqpoint{3.019554in}{3.038828in}}%
\pgfpathcurveto{\pgfqpoint{3.019554in}{3.047064in}}{\pgfqpoint{3.016282in}{3.054964in}}{\pgfqpoint{3.010458in}{3.060788in}}%
\pgfpathcurveto{\pgfqpoint{3.004634in}{3.066612in}}{\pgfqpoint{2.996734in}{3.069884in}}{\pgfqpoint{2.988497in}{3.069884in}}%
\pgfpathcurveto{\pgfqpoint{2.980261in}{3.069884in}}{\pgfqpoint{2.972361in}{3.066612in}}{\pgfqpoint{2.966537in}{3.060788in}}%
\pgfpathcurveto{\pgfqpoint{2.960713in}{3.054964in}}{\pgfqpoint{2.957441in}{3.047064in}}{\pgfqpoint{2.957441in}{3.038828in}}%
\pgfpathcurveto{\pgfqpoint{2.957441in}{3.030591in}}{\pgfqpoint{2.960713in}{3.022691in}}{\pgfqpoint{2.966537in}{3.016867in}}%
\pgfpathcurveto{\pgfqpoint{2.972361in}{3.011043in}}{\pgfqpoint{2.980261in}{3.007771in}}{\pgfqpoint{2.988497in}{3.007771in}}%
\pgfpathclose%
\pgfusepath{stroke,fill}%
\end{pgfscope}%
\begin{pgfscope}%
\pgfpathrectangle{\pgfqpoint{0.100000in}{0.220728in}}{\pgfqpoint{3.696000in}{3.696000in}}%
\pgfusepath{clip}%
\pgfsetbuttcap%
\pgfsetroundjoin%
\definecolor{currentfill}{rgb}{0.121569,0.466667,0.705882}%
\pgfsetfillcolor{currentfill}%
\pgfsetfillopacity{0.585897}%
\pgfsetlinewidth{1.003750pt}%
\definecolor{currentstroke}{rgb}{0.121569,0.466667,0.705882}%
\pgfsetstrokecolor{currentstroke}%
\pgfsetstrokeopacity{0.585897}%
\pgfsetdash{}{0pt}%
\pgfpathmoveto{\pgfqpoint{0.911658in}{1.418765in}}%
\pgfpathcurveto{\pgfqpoint{0.919894in}{1.418765in}}{\pgfqpoint{0.927794in}{1.422038in}}{\pgfqpoint{0.933618in}{1.427861in}}%
\pgfpathcurveto{\pgfqpoint{0.939442in}{1.433685in}}{\pgfqpoint{0.942714in}{1.441585in}}{\pgfqpoint{0.942714in}{1.449822in}}%
\pgfpathcurveto{\pgfqpoint{0.942714in}{1.458058in}}{\pgfqpoint{0.939442in}{1.465958in}}{\pgfqpoint{0.933618in}{1.471782in}}%
\pgfpathcurveto{\pgfqpoint{0.927794in}{1.477606in}}{\pgfqpoint{0.919894in}{1.480878in}}{\pgfqpoint{0.911658in}{1.480878in}}%
\pgfpathcurveto{\pgfqpoint{0.903421in}{1.480878in}}{\pgfqpoint{0.895521in}{1.477606in}}{\pgfqpoint{0.889697in}{1.471782in}}%
\pgfpathcurveto{\pgfqpoint{0.883873in}{1.465958in}}{\pgfqpoint{0.880601in}{1.458058in}}{\pgfqpoint{0.880601in}{1.449822in}}%
\pgfpathcurveto{\pgfqpoint{0.880601in}{1.441585in}}{\pgfqpoint{0.883873in}{1.433685in}}{\pgfqpoint{0.889697in}{1.427861in}}%
\pgfpathcurveto{\pgfqpoint{0.895521in}{1.422038in}}{\pgfqpoint{0.903421in}{1.418765in}}{\pgfqpoint{0.911658in}{1.418765in}}%
\pgfpathclose%
\pgfusepath{stroke,fill}%
\end{pgfscope}%
\begin{pgfscope}%
\pgfpathrectangle{\pgfqpoint{0.100000in}{0.220728in}}{\pgfqpoint{3.696000in}{3.696000in}}%
\pgfusepath{clip}%
\pgfsetbuttcap%
\pgfsetroundjoin%
\definecolor{currentfill}{rgb}{0.121569,0.466667,0.705882}%
\pgfsetfillcolor{currentfill}%
\pgfsetfillopacity{0.586890}%
\pgfsetlinewidth{1.003750pt}%
\definecolor{currentstroke}{rgb}{0.121569,0.466667,0.705882}%
\pgfsetstrokecolor{currentstroke}%
\pgfsetstrokeopacity{0.586890}%
\pgfsetdash{}{0pt}%
\pgfpathmoveto{\pgfqpoint{3.000279in}{3.005710in}}%
\pgfpathcurveto{\pgfqpoint{3.008515in}{3.005710in}}{\pgfqpoint{3.016415in}{3.008982in}}{\pgfqpoint{3.022239in}{3.014806in}}%
\pgfpathcurveto{\pgfqpoint{3.028063in}{3.020630in}}{\pgfqpoint{3.031335in}{3.028530in}}{\pgfqpoint{3.031335in}{3.036766in}}%
\pgfpathcurveto{\pgfqpoint{3.031335in}{3.045002in}}{\pgfqpoint{3.028063in}{3.052903in}}{\pgfqpoint{3.022239in}{3.058726in}}%
\pgfpathcurveto{\pgfqpoint{3.016415in}{3.064550in}}{\pgfqpoint{3.008515in}{3.067823in}}{\pgfqpoint{3.000279in}{3.067823in}}%
\pgfpathcurveto{\pgfqpoint{2.992043in}{3.067823in}}{\pgfqpoint{2.984143in}{3.064550in}}{\pgfqpoint{2.978319in}{3.058726in}}%
\pgfpathcurveto{\pgfqpoint{2.972495in}{3.052903in}}{\pgfqpoint{2.969222in}{3.045002in}}{\pgfqpoint{2.969222in}{3.036766in}}%
\pgfpathcurveto{\pgfqpoint{2.969222in}{3.028530in}}{\pgfqpoint{2.972495in}{3.020630in}}{\pgfqpoint{2.978319in}{3.014806in}}%
\pgfpathcurveto{\pgfqpoint{2.984143in}{3.008982in}}{\pgfqpoint{2.992043in}{3.005710in}}{\pgfqpoint{3.000279in}{3.005710in}}%
\pgfpathclose%
\pgfusepath{stroke,fill}%
\end{pgfscope}%
\begin{pgfscope}%
\pgfpathrectangle{\pgfqpoint{0.100000in}{0.220728in}}{\pgfqpoint{3.696000in}{3.696000in}}%
\pgfusepath{clip}%
\pgfsetbuttcap%
\pgfsetroundjoin%
\definecolor{currentfill}{rgb}{0.121569,0.466667,0.705882}%
\pgfsetfillcolor{currentfill}%
\pgfsetfillopacity{0.587446}%
\pgfsetlinewidth{1.003750pt}%
\definecolor{currentstroke}{rgb}{0.121569,0.466667,0.705882}%
\pgfsetstrokecolor{currentstroke}%
\pgfsetstrokeopacity{0.587446}%
\pgfsetdash{}{0pt}%
\pgfpathmoveto{\pgfqpoint{0.905400in}{1.408976in}}%
\pgfpathcurveto{\pgfqpoint{0.913636in}{1.408976in}}{\pgfqpoint{0.921536in}{1.412248in}}{\pgfqpoint{0.927360in}{1.418072in}}%
\pgfpathcurveto{\pgfqpoint{0.933184in}{1.423896in}}{\pgfqpoint{0.936456in}{1.431796in}}{\pgfqpoint{0.936456in}{1.440032in}}%
\pgfpathcurveto{\pgfqpoint{0.936456in}{1.448268in}}{\pgfqpoint{0.933184in}{1.456168in}}{\pgfqpoint{0.927360in}{1.461992in}}%
\pgfpathcurveto{\pgfqpoint{0.921536in}{1.467816in}}{\pgfqpoint{0.913636in}{1.471089in}}{\pgfqpoint{0.905400in}{1.471089in}}%
\pgfpathcurveto{\pgfqpoint{0.897163in}{1.471089in}}{\pgfqpoint{0.889263in}{1.467816in}}{\pgfqpoint{0.883439in}{1.461992in}}%
\pgfpathcurveto{\pgfqpoint{0.877615in}{1.456168in}}{\pgfqpoint{0.874343in}{1.448268in}}{\pgfqpoint{0.874343in}{1.440032in}}%
\pgfpathcurveto{\pgfqpoint{0.874343in}{1.431796in}}{\pgfqpoint{0.877615in}{1.423896in}}{\pgfqpoint{0.883439in}{1.418072in}}%
\pgfpathcurveto{\pgfqpoint{0.889263in}{1.412248in}}{\pgfqpoint{0.897163in}{1.408976in}}{\pgfqpoint{0.905400in}{1.408976in}}%
\pgfpathclose%
\pgfusepath{stroke,fill}%
\end{pgfscope}%
\begin{pgfscope}%
\pgfpathrectangle{\pgfqpoint{0.100000in}{0.220728in}}{\pgfqpoint{3.696000in}{3.696000in}}%
\pgfusepath{clip}%
\pgfsetbuttcap%
\pgfsetroundjoin%
\definecolor{currentfill}{rgb}{0.121569,0.466667,0.705882}%
\pgfsetfillcolor{currentfill}%
\pgfsetfillopacity{0.589486}%
\pgfsetlinewidth{1.003750pt}%
\definecolor{currentstroke}{rgb}{0.121569,0.466667,0.705882}%
\pgfsetstrokecolor{currentstroke}%
\pgfsetstrokeopacity{0.589486}%
\pgfsetdash{}{0pt}%
\pgfpathmoveto{\pgfqpoint{3.011837in}{3.001935in}}%
\pgfpathcurveto{\pgfqpoint{3.020073in}{3.001935in}}{\pgfqpoint{3.027973in}{3.005207in}}{\pgfqpoint{3.033797in}{3.011031in}}%
\pgfpathcurveto{\pgfqpoint{3.039621in}{3.016855in}}{\pgfqpoint{3.042893in}{3.024755in}}{\pgfqpoint{3.042893in}{3.032991in}}%
\pgfpathcurveto{\pgfqpoint{3.042893in}{3.041227in}}{\pgfqpoint{3.039621in}{3.049127in}}{\pgfqpoint{3.033797in}{3.054951in}}%
\pgfpathcurveto{\pgfqpoint{3.027973in}{3.060775in}}{\pgfqpoint{3.020073in}{3.064048in}}{\pgfqpoint{3.011837in}{3.064048in}}%
\pgfpathcurveto{\pgfqpoint{3.003600in}{3.064048in}}{\pgfqpoint{2.995700in}{3.060775in}}{\pgfqpoint{2.989876in}{3.054951in}}%
\pgfpathcurveto{\pgfqpoint{2.984052in}{3.049127in}}{\pgfqpoint{2.980780in}{3.041227in}}{\pgfqpoint{2.980780in}{3.032991in}}%
\pgfpathcurveto{\pgfqpoint{2.980780in}{3.024755in}}{\pgfqpoint{2.984052in}{3.016855in}}{\pgfqpoint{2.989876in}{3.011031in}}%
\pgfpathcurveto{\pgfqpoint{2.995700in}{3.005207in}}{\pgfqpoint{3.003600in}{3.001935in}}{\pgfqpoint{3.011837in}{3.001935in}}%
\pgfpathclose%
\pgfusepath{stroke,fill}%
\end{pgfscope}%
\begin{pgfscope}%
\pgfpathrectangle{\pgfqpoint{0.100000in}{0.220728in}}{\pgfqpoint{3.696000in}{3.696000in}}%
\pgfusepath{clip}%
\pgfsetbuttcap%
\pgfsetroundjoin%
\definecolor{currentfill}{rgb}{0.121569,0.466667,0.705882}%
\pgfsetfillcolor{currentfill}%
\pgfsetfillopacity{0.591155}%
\pgfsetlinewidth{1.003750pt}%
\definecolor{currentstroke}{rgb}{0.121569,0.466667,0.705882}%
\pgfsetstrokecolor{currentstroke}%
\pgfsetstrokeopacity{0.591155}%
\pgfsetdash{}{0pt}%
\pgfpathmoveto{\pgfqpoint{0.900443in}{1.388740in}}%
\pgfpathcurveto{\pgfqpoint{0.908680in}{1.388740in}}{\pgfqpoint{0.916580in}{1.392012in}}{\pgfqpoint{0.922404in}{1.397836in}}%
\pgfpathcurveto{\pgfqpoint{0.928228in}{1.403660in}}{\pgfqpoint{0.931500in}{1.411560in}}{\pgfqpoint{0.931500in}{1.419797in}}%
\pgfpathcurveto{\pgfqpoint{0.931500in}{1.428033in}}{\pgfqpoint{0.928228in}{1.435933in}}{\pgfqpoint{0.922404in}{1.441757in}}%
\pgfpathcurveto{\pgfqpoint{0.916580in}{1.447581in}}{\pgfqpoint{0.908680in}{1.450853in}}{\pgfqpoint{0.900443in}{1.450853in}}%
\pgfpathcurveto{\pgfqpoint{0.892207in}{1.450853in}}{\pgfqpoint{0.884307in}{1.447581in}}{\pgfqpoint{0.878483in}{1.441757in}}%
\pgfpathcurveto{\pgfqpoint{0.872659in}{1.435933in}}{\pgfqpoint{0.869387in}{1.428033in}}{\pgfqpoint{0.869387in}{1.419797in}}%
\pgfpathcurveto{\pgfqpoint{0.869387in}{1.411560in}}{\pgfqpoint{0.872659in}{1.403660in}}{\pgfqpoint{0.878483in}{1.397836in}}%
\pgfpathcurveto{\pgfqpoint{0.884307in}{1.392012in}}{\pgfqpoint{0.892207in}{1.388740in}}{\pgfqpoint{0.900443in}{1.388740in}}%
\pgfpathclose%
\pgfusepath{stroke,fill}%
\end{pgfscope}%
\begin{pgfscope}%
\pgfpathrectangle{\pgfqpoint{0.100000in}{0.220728in}}{\pgfqpoint{3.696000in}{3.696000in}}%
\pgfusepath{clip}%
\pgfsetbuttcap%
\pgfsetroundjoin%
\definecolor{currentfill}{rgb}{0.121569,0.466667,0.705882}%
\pgfsetfillcolor{currentfill}%
\pgfsetfillopacity{0.591297}%
\pgfsetlinewidth{1.003750pt}%
\definecolor{currentstroke}{rgb}{0.121569,0.466667,0.705882}%
\pgfsetstrokecolor{currentstroke}%
\pgfsetstrokeopacity{0.591297}%
\pgfsetdash{}{0pt}%
\pgfpathmoveto{\pgfqpoint{3.018033in}{3.000862in}}%
\pgfpathcurveto{\pgfqpoint{3.026270in}{3.000862in}}{\pgfqpoint{3.034170in}{3.004134in}}{\pgfqpoint{3.039994in}{3.009958in}}%
\pgfpathcurveto{\pgfqpoint{3.045818in}{3.015782in}}{\pgfqpoint{3.049090in}{3.023682in}}{\pgfqpoint{3.049090in}{3.031918in}}%
\pgfpathcurveto{\pgfqpoint{3.049090in}{3.040155in}}{\pgfqpoint{3.045818in}{3.048055in}}{\pgfqpoint{3.039994in}{3.053879in}}%
\pgfpathcurveto{\pgfqpoint{3.034170in}{3.059702in}}{\pgfqpoint{3.026270in}{3.062975in}}{\pgfqpoint{3.018033in}{3.062975in}}%
\pgfpathcurveto{\pgfqpoint{3.009797in}{3.062975in}}{\pgfqpoint{3.001897in}{3.059702in}}{\pgfqpoint{2.996073in}{3.053879in}}%
\pgfpathcurveto{\pgfqpoint{2.990249in}{3.048055in}}{\pgfqpoint{2.986977in}{3.040155in}}{\pgfqpoint{2.986977in}{3.031918in}}%
\pgfpathcurveto{\pgfqpoint{2.986977in}{3.023682in}}{\pgfqpoint{2.990249in}{3.015782in}}{\pgfqpoint{2.996073in}{3.009958in}}%
\pgfpathcurveto{\pgfqpoint{3.001897in}{3.004134in}}{\pgfqpoint{3.009797in}{3.000862in}}{\pgfqpoint{3.018033in}{3.000862in}}%
\pgfpathclose%
\pgfusepath{stroke,fill}%
\end{pgfscope}%
\begin{pgfscope}%
\pgfpathrectangle{\pgfqpoint{0.100000in}{0.220728in}}{\pgfqpoint{3.696000in}{3.696000in}}%
\pgfusepath{clip}%
\pgfsetbuttcap%
\pgfsetroundjoin%
\definecolor{currentfill}{rgb}{0.121569,0.466667,0.705882}%
\pgfsetfillcolor{currentfill}%
\pgfsetfillopacity{0.592163}%
\pgfsetlinewidth{1.003750pt}%
\definecolor{currentstroke}{rgb}{0.121569,0.466667,0.705882}%
\pgfsetstrokecolor{currentstroke}%
\pgfsetstrokeopacity{0.592163}%
\pgfsetdash{}{0pt}%
\pgfpathmoveto{\pgfqpoint{3.027257in}{2.999696in}}%
\pgfpathcurveto{\pgfqpoint{3.035493in}{2.999696in}}{\pgfqpoint{3.043393in}{3.002969in}}{\pgfqpoint{3.049217in}{3.008793in}}%
\pgfpathcurveto{\pgfqpoint{3.055041in}{3.014617in}}{\pgfqpoint{3.058313in}{3.022517in}}{\pgfqpoint{3.058313in}{3.030753in}}%
\pgfpathcurveto{\pgfqpoint{3.058313in}{3.038989in}}{\pgfqpoint{3.055041in}{3.046889in}}{\pgfqpoint{3.049217in}{3.052713in}}%
\pgfpathcurveto{\pgfqpoint{3.043393in}{3.058537in}}{\pgfqpoint{3.035493in}{3.061809in}}{\pgfqpoint{3.027257in}{3.061809in}}%
\pgfpathcurveto{\pgfqpoint{3.019021in}{3.061809in}}{\pgfqpoint{3.011121in}{3.058537in}}{\pgfqpoint{3.005297in}{3.052713in}}%
\pgfpathcurveto{\pgfqpoint{2.999473in}{3.046889in}}{\pgfqpoint{2.996200in}{3.038989in}}{\pgfqpoint{2.996200in}{3.030753in}}%
\pgfpathcurveto{\pgfqpoint{2.996200in}{3.022517in}}{\pgfqpoint{2.999473in}{3.014617in}}{\pgfqpoint{3.005297in}{3.008793in}}%
\pgfpathcurveto{\pgfqpoint{3.011121in}{3.002969in}}{\pgfqpoint{3.019021in}{2.999696in}}{\pgfqpoint{3.027257in}{2.999696in}}%
\pgfpathclose%
\pgfusepath{stroke,fill}%
\end{pgfscope}%
\begin{pgfscope}%
\pgfpathrectangle{\pgfqpoint{0.100000in}{0.220728in}}{\pgfqpoint{3.696000in}{3.696000in}}%
\pgfusepath{clip}%
\pgfsetbuttcap%
\pgfsetroundjoin%
\definecolor{currentfill}{rgb}{0.121569,0.466667,0.705882}%
\pgfsetfillcolor{currentfill}%
\pgfsetfillopacity{0.592704}%
\pgfsetlinewidth{1.003750pt}%
\definecolor{currentstroke}{rgb}{0.121569,0.466667,0.705882}%
\pgfsetstrokecolor{currentstroke}%
\pgfsetstrokeopacity{0.592704}%
\pgfsetdash{}{0pt}%
\pgfpathmoveto{\pgfqpoint{0.887904in}{1.375319in}}%
\pgfpathcurveto{\pgfqpoint{0.896141in}{1.375319in}}{\pgfqpoint{0.904041in}{1.378591in}}{\pgfqpoint{0.909865in}{1.384415in}}%
\pgfpathcurveto{\pgfqpoint{0.915689in}{1.390239in}}{\pgfqpoint{0.918961in}{1.398139in}}{\pgfqpoint{0.918961in}{1.406376in}}%
\pgfpathcurveto{\pgfqpoint{0.918961in}{1.414612in}}{\pgfqpoint{0.915689in}{1.422512in}}{\pgfqpoint{0.909865in}{1.428336in}}%
\pgfpathcurveto{\pgfqpoint{0.904041in}{1.434160in}}{\pgfqpoint{0.896141in}{1.437432in}}{\pgfqpoint{0.887904in}{1.437432in}}%
\pgfpathcurveto{\pgfqpoint{0.879668in}{1.437432in}}{\pgfqpoint{0.871768in}{1.434160in}}{\pgfqpoint{0.865944in}{1.428336in}}%
\pgfpathcurveto{\pgfqpoint{0.860120in}{1.422512in}}{\pgfqpoint{0.856848in}{1.414612in}}{\pgfqpoint{0.856848in}{1.406376in}}%
\pgfpathcurveto{\pgfqpoint{0.856848in}{1.398139in}}{\pgfqpoint{0.860120in}{1.390239in}}{\pgfqpoint{0.865944in}{1.384415in}}%
\pgfpathcurveto{\pgfqpoint{0.871768in}{1.378591in}}{\pgfqpoint{0.879668in}{1.375319in}}{\pgfqpoint{0.887904in}{1.375319in}}%
\pgfpathclose%
\pgfusepath{stroke,fill}%
\end{pgfscope}%
\begin{pgfscope}%
\pgfpathrectangle{\pgfqpoint{0.100000in}{0.220728in}}{\pgfqpoint{3.696000in}{3.696000in}}%
\pgfusepath{clip}%
\pgfsetbuttcap%
\pgfsetroundjoin%
\definecolor{currentfill}{rgb}{0.121569,0.466667,0.705882}%
\pgfsetfillcolor{currentfill}%
\pgfsetfillopacity{0.592989}%
\pgfsetlinewidth{1.003750pt}%
\definecolor{currentstroke}{rgb}{0.121569,0.466667,0.705882}%
\pgfsetstrokecolor{currentstroke}%
\pgfsetstrokeopacity{0.592989}%
\pgfsetdash{}{0pt}%
\pgfpathmoveto{\pgfqpoint{3.031922in}{2.998385in}}%
\pgfpathcurveto{\pgfqpoint{3.040158in}{2.998385in}}{\pgfqpoint{3.048058in}{3.001657in}}{\pgfqpoint{3.053882in}{3.007481in}}%
\pgfpathcurveto{\pgfqpoint{3.059706in}{3.013305in}}{\pgfqpoint{3.062978in}{3.021205in}}{\pgfqpoint{3.062978in}{3.029441in}}%
\pgfpathcurveto{\pgfqpoint{3.062978in}{3.037678in}}{\pgfqpoint{3.059706in}{3.045578in}}{\pgfqpoint{3.053882in}{3.051402in}}%
\pgfpathcurveto{\pgfqpoint{3.048058in}{3.057225in}}{\pgfqpoint{3.040158in}{3.060498in}}{\pgfqpoint{3.031922in}{3.060498in}}%
\pgfpathcurveto{\pgfqpoint{3.023686in}{3.060498in}}{\pgfqpoint{3.015786in}{3.057225in}}{\pgfqpoint{3.009962in}{3.051402in}}%
\pgfpathcurveto{\pgfqpoint{3.004138in}{3.045578in}}{\pgfqpoint{3.000865in}{3.037678in}}{\pgfqpoint{3.000865in}{3.029441in}}%
\pgfpathcurveto{\pgfqpoint{3.000865in}{3.021205in}}{\pgfqpoint{3.004138in}{3.013305in}}{\pgfqpoint{3.009962in}{3.007481in}}%
\pgfpathcurveto{\pgfqpoint{3.015786in}{3.001657in}}{\pgfqpoint{3.023686in}{2.998385in}}{\pgfqpoint{3.031922in}{2.998385in}}%
\pgfpathclose%
\pgfusepath{stroke,fill}%
\end{pgfscope}%
\begin{pgfscope}%
\pgfpathrectangle{\pgfqpoint{0.100000in}{0.220728in}}{\pgfqpoint{3.696000in}{3.696000in}}%
\pgfusepath{clip}%
\pgfsetbuttcap%
\pgfsetroundjoin%
\definecolor{currentfill}{rgb}{0.121569,0.466667,0.705882}%
\pgfsetfillcolor{currentfill}%
\pgfsetfillopacity{0.594191}%
\pgfsetlinewidth{1.003750pt}%
\definecolor{currentstroke}{rgb}{0.121569,0.466667,0.705882}%
\pgfsetstrokecolor{currentstroke}%
\pgfsetstrokeopacity{0.594191}%
\pgfsetdash{}{0pt}%
\pgfpathmoveto{\pgfqpoint{3.037869in}{2.997729in}}%
\pgfpathcurveto{\pgfqpoint{3.046106in}{2.997729in}}{\pgfqpoint{3.054006in}{3.001001in}}{\pgfqpoint{3.059830in}{3.006825in}}%
\pgfpathcurveto{\pgfqpoint{3.065654in}{3.012649in}}{\pgfqpoint{3.068926in}{3.020549in}}{\pgfqpoint{3.068926in}{3.028785in}}%
\pgfpathcurveto{\pgfqpoint{3.068926in}{3.037022in}}{\pgfqpoint{3.065654in}{3.044922in}}{\pgfqpoint{3.059830in}{3.050746in}}%
\pgfpathcurveto{\pgfqpoint{3.054006in}{3.056570in}}{\pgfqpoint{3.046106in}{3.059842in}}{\pgfqpoint{3.037869in}{3.059842in}}%
\pgfpathcurveto{\pgfqpoint{3.029633in}{3.059842in}}{\pgfqpoint{3.021733in}{3.056570in}}{\pgfqpoint{3.015909in}{3.050746in}}%
\pgfpathcurveto{\pgfqpoint{3.010085in}{3.044922in}}{\pgfqpoint{3.006813in}{3.037022in}}{\pgfqpoint{3.006813in}{3.028785in}}%
\pgfpathcurveto{\pgfqpoint{3.006813in}{3.020549in}}{\pgfqpoint{3.010085in}{3.012649in}}{\pgfqpoint{3.015909in}{3.006825in}}%
\pgfpathcurveto{\pgfqpoint{3.021733in}{3.001001in}}{\pgfqpoint{3.029633in}{2.997729in}}{\pgfqpoint{3.037869in}{2.997729in}}%
\pgfpathclose%
\pgfusepath{stroke,fill}%
\end{pgfscope}%
\begin{pgfscope}%
\pgfpathrectangle{\pgfqpoint{0.100000in}{0.220728in}}{\pgfqpoint{3.696000in}{3.696000in}}%
\pgfusepath{clip}%
\pgfsetbuttcap%
\pgfsetroundjoin%
\definecolor{currentfill}{rgb}{0.121569,0.466667,0.705882}%
\pgfsetfillcolor{currentfill}%
\pgfsetfillopacity{0.595149}%
\pgfsetlinewidth{1.003750pt}%
\definecolor{currentstroke}{rgb}{0.121569,0.466667,0.705882}%
\pgfsetstrokecolor{currentstroke}%
\pgfsetstrokeopacity{0.595149}%
\pgfsetdash{}{0pt}%
\pgfpathmoveto{\pgfqpoint{0.884125in}{1.357543in}}%
\pgfpathcurveto{\pgfqpoint{0.892361in}{1.357543in}}{\pgfqpoint{0.900261in}{1.360816in}}{\pgfqpoint{0.906085in}{1.366640in}}%
\pgfpathcurveto{\pgfqpoint{0.911909in}{1.372463in}}{\pgfqpoint{0.915181in}{1.380364in}}{\pgfqpoint{0.915181in}{1.388600in}}%
\pgfpathcurveto{\pgfqpoint{0.915181in}{1.396836in}}{\pgfqpoint{0.911909in}{1.404736in}}{\pgfqpoint{0.906085in}{1.410560in}}%
\pgfpathcurveto{\pgfqpoint{0.900261in}{1.416384in}}{\pgfqpoint{0.892361in}{1.419656in}}{\pgfqpoint{0.884125in}{1.419656in}}%
\pgfpathcurveto{\pgfqpoint{0.875888in}{1.419656in}}{\pgfqpoint{0.867988in}{1.416384in}}{\pgfqpoint{0.862164in}{1.410560in}}%
\pgfpathcurveto{\pgfqpoint{0.856340in}{1.404736in}}{\pgfqpoint{0.853068in}{1.396836in}}{\pgfqpoint{0.853068in}{1.388600in}}%
\pgfpathcurveto{\pgfqpoint{0.853068in}{1.380364in}}{\pgfqpoint{0.856340in}{1.372463in}}{\pgfqpoint{0.862164in}{1.366640in}}%
\pgfpathcurveto{\pgfqpoint{0.867988in}{1.360816in}}{\pgfqpoint{0.875888in}{1.357543in}}{\pgfqpoint{0.884125in}{1.357543in}}%
\pgfpathclose%
\pgfusepath{stroke,fill}%
\end{pgfscope}%
\begin{pgfscope}%
\pgfpathrectangle{\pgfqpoint{0.100000in}{0.220728in}}{\pgfqpoint{3.696000in}{3.696000in}}%
\pgfusepath{clip}%
\pgfsetbuttcap%
\pgfsetroundjoin%
\definecolor{currentfill}{rgb}{0.121569,0.466667,0.705882}%
\pgfsetfillcolor{currentfill}%
\pgfsetfillopacity{0.595871}%
\pgfsetlinewidth{1.003750pt}%
\definecolor{currentstroke}{rgb}{0.121569,0.466667,0.705882}%
\pgfsetstrokecolor{currentstroke}%
\pgfsetstrokeopacity{0.595871}%
\pgfsetdash{}{0pt}%
\pgfpathmoveto{\pgfqpoint{3.046550in}{2.996714in}}%
\pgfpathcurveto{\pgfqpoint{3.054786in}{2.996714in}}{\pgfqpoint{3.062686in}{2.999986in}}{\pgfqpoint{3.068510in}{3.005810in}}%
\pgfpathcurveto{\pgfqpoint{3.074334in}{3.011634in}}{\pgfqpoint{3.077606in}{3.019534in}}{\pgfqpoint{3.077606in}{3.027770in}}%
\pgfpathcurveto{\pgfqpoint{3.077606in}{3.036007in}}{\pgfqpoint{3.074334in}{3.043907in}}{\pgfqpoint{3.068510in}{3.049730in}}%
\pgfpathcurveto{\pgfqpoint{3.062686in}{3.055554in}}{\pgfqpoint{3.054786in}{3.058827in}}{\pgfqpoint{3.046550in}{3.058827in}}%
\pgfpathcurveto{\pgfqpoint{3.038313in}{3.058827in}}{\pgfqpoint{3.030413in}{3.055554in}}{\pgfqpoint{3.024589in}{3.049730in}}%
\pgfpathcurveto{\pgfqpoint{3.018765in}{3.043907in}}{\pgfqpoint{3.015493in}{3.036007in}}{\pgfqpoint{3.015493in}{3.027770in}}%
\pgfpathcurveto{\pgfqpoint{3.015493in}{3.019534in}}{\pgfqpoint{3.018765in}{3.011634in}}{\pgfqpoint{3.024589in}{3.005810in}}%
\pgfpathcurveto{\pgfqpoint{3.030413in}{2.999986in}}{\pgfqpoint{3.038313in}{2.996714in}}{\pgfqpoint{3.046550in}{2.996714in}}%
\pgfpathclose%
\pgfusepath{stroke,fill}%
\end{pgfscope}%
\begin{pgfscope}%
\pgfpathrectangle{\pgfqpoint{0.100000in}{0.220728in}}{\pgfqpoint{3.696000in}{3.696000in}}%
\pgfusepath{clip}%
\pgfsetbuttcap%
\pgfsetroundjoin%
\definecolor{currentfill}{rgb}{0.121569,0.466667,0.705882}%
\pgfsetfillcolor{currentfill}%
\pgfsetfillopacity{0.596282}%
\pgfsetlinewidth{1.003750pt}%
\definecolor{currentstroke}{rgb}{0.121569,0.466667,0.705882}%
\pgfsetstrokecolor{currentstroke}%
\pgfsetstrokeopacity{0.596282}%
\pgfsetdash{}{0pt}%
\pgfpathmoveto{\pgfqpoint{0.874681in}{1.347650in}}%
\pgfpathcurveto{\pgfqpoint{0.882917in}{1.347650in}}{\pgfqpoint{0.890817in}{1.350922in}}{\pgfqpoint{0.896641in}{1.356746in}}%
\pgfpathcurveto{\pgfqpoint{0.902465in}{1.362570in}}{\pgfqpoint{0.905737in}{1.370470in}}{\pgfqpoint{0.905737in}{1.378706in}}%
\pgfpathcurveto{\pgfqpoint{0.905737in}{1.386942in}}{\pgfqpoint{0.902465in}{1.394842in}}{\pgfqpoint{0.896641in}{1.400666in}}%
\pgfpathcurveto{\pgfqpoint{0.890817in}{1.406490in}}{\pgfqpoint{0.882917in}{1.409763in}}{\pgfqpoint{0.874681in}{1.409763in}}%
\pgfpathcurveto{\pgfqpoint{0.866444in}{1.409763in}}{\pgfqpoint{0.858544in}{1.406490in}}{\pgfqpoint{0.852720in}{1.400666in}}%
\pgfpathcurveto{\pgfqpoint{0.846896in}{1.394842in}}{\pgfqpoint{0.843624in}{1.386942in}}{\pgfqpoint{0.843624in}{1.378706in}}%
\pgfpathcurveto{\pgfqpoint{0.843624in}{1.370470in}}{\pgfqpoint{0.846896in}{1.362570in}}{\pgfqpoint{0.852720in}{1.356746in}}%
\pgfpathcurveto{\pgfqpoint{0.858544in}{1.350922in}}{\pgfqpoint{0.866444in}{1.347650in}}{\pgfqpoint{0.874681in}{1.347650in}}%
\pgfpathclose%
\pgfusepath{stroke,fill}%
\end{pgfscope}%
\begin{pgfscope}%
\pgfpathrectangle{\pgfqpoint{0.100000in}{0.220728in}}{\pgfqpoint{3.696000in}{3.696000in}}%
\pgfusepath{clip}%
\pgfsetbuttcap%
\pgfsetroundjoin%
\definecolor{currentfill}{rgb}{0.121569,0.466667,0.705882}%
\pgfsetfillcolor{currentfill}%
\pgfsetfillopacity{0.596916}%
\pgfsetlinewidth{1.003750pt}%
\definecolor{currentstroke}{rgb}{0.121569,0.466667,0.705882}%
\pgfsetstrokecolor{currentstroke}%
\pgfsetstrokeopacity{0.596916}%
\pgfsetdash{}{0pt}%
\pgfpathmoveto{\pgfqpoint{3.050964in}{2.995359in}}%
\pgfpathcurveto{\pgfqpoint{3.059200in}{2.995359in}}{\pgfqpoint{3.067101in}{2.998631in}}{\pgfqpoint{3.072924in}{3.004455in}}%
\pgfpathcurveto{\pgfqpoint{3.078748in}{3.010279in}}{\pgfqpoint{3.082021in}{3.018179in}}{\pgfqpoint{3.082021in}{3.026415in}}%
\pgfpathcurveto{\pgfqpoint{3.082021in}{3.034652in}}{\pgfqpoint{3.078748in}{3.042552in}}{\pgfqpoint{3.072924in}{3.048376in}}%
\pgfpathcurveto{\pgfqpoint{3.067101in}{3.054199in}}{\pgfqpoint{3.059200in}{3.057472in}}{\pgfqpoint{3.050964in}{3.057472in}}%
\pgfpathcurveto{\pgfqpoint{3.042728in}{3.057472in}}{\pgfqpoint{3.034828in}{3.054199in}}{\pgfqpoint{3.029004in}{3.048376in}}%
\pgfpathcurveto{\pgfqpoint{3.023180in}{3.042552in}}{\pgfqpoint{3.019908in}{3.034652in}}{\pgfqpoint{3.019908in}{3.026415in}}%
\pgfpathcurveto{\pgfqpoint{3.019908in}{3.018179in}}{\pgfqpoint{3.023180in}{3.010279in}}{\pgfqpoint{3.029004in}{3.004455in}}%
\pgfpathcurveto{\pgfqpoint{3.034828in}{2.998631in}}{\pgfqpoint{3.042728in}{2.995359in}}{\pgfqpoint{3.050964in}{2.995359in}}%
\pgfpathclose%
\pgfusepath{stroke,fill}%
\end{pgfscope}%
\begin{pgfscope}%
\pgfpathrectangle{\pgfqpoint{0.100000in}{0.220728in}}{\pgfqpoint{3.696000in}{3.696000in}}%
\pgfusepath{clip}%
\pgfsetbuttcap%
\pgfsetroundjoin%
\definecolor{currentfill}{rgb}{0.121569,0.466667,0.705882}%
\pgfsetfillcolor{currentfill}%
\pgfsetfillopacity{0.598119}%
\pgfsetlinewidth{1.003750pt}%
\definecolor{currentstroke}{rgb}{0.121569,0.466667,0.705882}%
\pgfsetstrokecolor{currentstroke}%
\pgfsetstrokeopacity{0.598119}%
\pgfsetdash{}{0pt}%
\pgfpathmoveto{\pgfqpoint{0.872846in}{1.336943in}}%
\pgfpathcurveto{\pgfqpoint{0.881082in}{1.336943in}}{\pgfqpoint{0.888982in}{1.340216in}}{\pgfqpoint{0.894806in}{1.346040in}}%
\pgfpathcurveto{\pgfqpoint{0.900630in}{1.351864in}}{\pgfqpoint{0.903902in}{1.359764in}}{\pgfqpoint{0.903902in}{1.368000in}}%
\pgfpathcurveto{\pgfqpoint{0.903902in}{1.376236in}}{\pgfqpoint{0.900630in}{1.384136in}}{\pgfqpoint{0.894806in}{1.389960in}}%
\pgfpathcurveto{\pgfqpoint{0.888982in}{1.395784in}}{\pgfqpoint{0.881082in}{1.399056in}}{\pgfqpoint{0.872846in}{1.399056in}}%
\pgfpathcurveto{\pgfqpoint{0.864610in}{1.399056in}}{\pgfqpoint{0.856710in}{1.395784in}}{\pgfqpoint{0.850886in}{1.389960in}}%
\pgfpathcurveto{\pgfqpoint{0.845062in}{1.384136in}}{\pgfqpoint{0.841789in}{1.376236in}}{\pgfqpoint{0.841789in}{1.368000in}}%
\pgfpathcurveto{\pgfqpoint{0.841789in}{1.359764in}}{\pgfqpoint{0.845062in}{1.351864in}}{\pgfqpoint{0.850886in}{1.346040in}}%
\pgfpathcurveto{\pgfqpoint{0.856710in}{1.340216in}}{\pgfqpoint{0.864610in}{1.336943in}}{\pgfqpoint{0.872846in}{1.336943in}}%
\pgfpathclose%
\pgfusepath{stroke,fill}%
\end{pgfscope}%
\begin{pgfscope}%
\pgfpathrectangle{\pgfqpoint{0.100000in}{0.220728in}}{\pgfqpoint{3.696000in}{3.696000in}}%
\pgfusepath{clip}%
\pgfsetbuttcap%
\pgfsetroundjoin%
\definecolor{currentfill}{rgb}{0.121569,0.466667,0.705882}%
\pgfsetfillcolor{currentfill}%
\pgfsetfillopacity{0.598233}%
\pgfsetlinewidth{1.003750pt}%
\definecolor{currentstroke}{rgb}{0.121569,0.466667,0.705882}%
\pgfsetstrokecolor{currentstroke}%
\pgfsetstrokeopacity{0.598233}%
\pgfsetdash{}{0pt}%
\pgfpathmoveto{\pgfqpoint{3.056151in}{2.994422in}}%
\pgfpathcurveto{\pgfqpoint{3.064388in}{2.994422in}}{\pgfqpoint{3.072288in}{2.997695in}}{\pgfqpoint{3.078112in}{3.003519in}}%
\pgfpathcurveto{\pgfqpoint{3.083936in}{3.009343in}}{\pgfqpoint{3.087208in}{3.017243in}}{\pgfqpoint{3.087208in}{3.025479in}}%
\pgfpathcurveto{\pgfqpoint{3.087208in}{3.033715in}}{\pgfqpoint{3.083936in}{3.041615in}}{\pgfqpoint{3.078112in}{3.047439in}}%
\pgfpathcurveto{\pgfqpoint{3.072288in}{3.053263in}}{\pgfqpoint{3.064388in}{3.056535in}}{\pgfqpoint{3.056151in}{3.056535in}}%
\pgfpathcurveto{\pgfqpoint{3.047915in}{3.056535in}}{\pgfqpoint{3.040015in}{3.053263in}}{\pgfqpoint{3.034191in}{3.047439in}}%
\pgfpathcurveto{\pgfqpoint{3.028367in}{3.041615in}}{\pgfqpoint{3.025095in}{3.033715in}}{\pgfqpoint{3.025095in}{3.025479in}}%
\pgfpathcurveto{\pgfqpoint{3.025095in}{3.017243in}}{\pgfqpoint{3.028367in}{3.009343in}}{\pgfqpoint{3.034191in}{3.003519in}}%
\pgfpathcurveto{\pgfqpoint{3.040015in}{2.997695in}}{\pgfqpoint{3.047915in}{2.994422in}}{\pgfqpoint{3.056151in}{2.994422in}}%
\pgfpathclose%
\pgfusepath{stroke,fill}%
\end{pgfscope}%
\begin{pgfscope}%
\pgfpathrectangle{\pgfqpoint{0.100000in}{0.220728in}}{\pgfqpoint{3.696000in}{3.696000in}}%
\pgfusepath{clip}%
\pgfsetbuttcap%
\pgfsetroundjoin%
\definecolor{currentfill}{rgb}{0.121569,0.466667,0.705882}%
\pgfsetfillcolor{currentfill}%
\pgfsetfillopacity{0.598805}%
\pgfsetlinewidth{1.003750pt}%
\definecolor{currentstroke}{rgb}{0.121569,0.466667,0.705882}%
\pgfsetstrokecolor{currentstroke}%
\pgfsetstrokeopacity{0.598805}%
\pgfsetdash{}{0pt}%
\pgfpathmoveto{\pgfqpoint{0.867366in}{1.330559in}}%
\pgfpathcurveto{\pgfqpoint{0.875603in}{1.330559in}}{\pgfqpoint{0.883503in}{1.333831in}}{\pgfqpoint{0.889327in}{1.339655in}}%
\pgfpathcurveto{\pgfqpoint{0.895151in}{1.345479in}}{\pgfqpoint{0.898423in}{1.353379in}}{\pgfqpoint{0.898423in}{1.361615in}}%
\pgfpathcurveto{\pgfqpoint{0.898423in}{1.369851in}}{\pgfqpoint{0.895151in}{1.377751in}}{\pgfqpoint{0.889327in}{1.383575in}}%
\pgfpathcurveto{\pgfqpoint{0.883503in}{1.389399in}}{\pgfqpoint{0.875603in}{1.392672in}}{\pgfqpoint{0.867366in}{1.392672in}}%
\pgfpathcurveto{\pgfqpoint{0.859130in}{1.392672in}}{\pgfqpoint{0.851230in}{1.389399in}}{\pgfqpoint{0.845406in}{1.383575in}}%
\pgfpathcurveto{\pgfqpoint{0.839582in}{1.377751in}}{\pgfqpoint{0.836310in}{1.369851in}}{\pgfqpoint{0.836310in}{1.361615in}}%
\pgfpathcurveto{\pgfqpoint{0.836310in}{1.353379in}}{\pgfqpoint{0.839582in}{1.345479in}}{\pgfqpoint{0.845406in}{1.339655in}}%
\pgfpathcurveto{\pgfqpoint{0.851230in}{1.333831in}}{\pgfqpoint{0.859130in}{1.330559in}}{\pgfqpoint{0.867366in}{1.330559in}}%
\pgfpathclose%
\pgfusepath{stroke,fill}%
\end{pgfscope}%
\begin{pgfscope}%
\pgfpathrectangle{\pgfqpoint{0.100000in}{0.220728in}}{\pgfqpoint{3.696000in}{3.696000in}}%
\pgfusepath{clip}%
\pgfsetbuttcap%
\pgfsetroundjoin%
\definecolor{currentfill}{rgb}{0.121569,0.466667,0.705882}%
\pgfsetfillcolor{currentfill}%
\pgfsetfillopacity{0.598918}%
\pgfsetlinewidth{1.003750pt}%
\definecolor{currentstroke}{rgb}{0.121569,0.466667,0.705882}%
\pgfsetstrokecolor{currentstroke}%
\pgfsetstrokeopacity{0.598918}%
\pgfsetdash{}{0pt}%
\pgfpathmoveto{\pgfqpoint{3.059078in}{2.993992in}}%
\pgfpathcurveto{\pgfqpoint{3.067314in}{2.993992in}}{\pgfqpoint{3.075214in}{2.997265in}}{\pgfqpoint{3.081038in}{3.003089in}}%
\pgfpathcurveto{\pgfqpoint{3.086862in}{3.008912in}}{\pgfqpoint{3.090135in}{3.016813in}}{\pgfqpoint{3.090135in}{3.025049in}}%
\pgfpathcurveto{\pgfqpoint{3.090135in}{3.033285in}}{\pgfqpoint{3.086862in}{3.041185in}}{\pgfqpoint{3.081038in}{3.047009in}}%
\pgfpathcurveto{\pgfqpoint{3.075214in}{3.052833in}}{\pgfqpoint{3.067314in}{3.056105in}}{\pgfqpoint{3.059078in}{3.056105in}}%
\pgfpathcurveto{\pgfqpoint{3.050842in}{3.056105in}}{\pgfqpoint{3.042942in}{3.052833in}}{\pgfqpoint{3.037118in}{3.047009in}}%
\pgfpathcurveto{\pgfqpoint{3.031294in}{3.041185in}}{\pgfqpoint{3.028022in}{3.033285in}}{\pgfqpoint{3.028022in}{3.025049in}}%
\pgfpathcurveto{\pgfqpoint{3.028022in}{3.016813in}}{\pgfqpoint{3.031294in}{3.008912in}}{\pgfqpoint{3.037118in}{3.003089in}}%
\pgfpathcurveto{\pgfqpoint{3.042942in}{2.997265in}}{\pgfqpoint{3.050842in}{2.993992in}}{\pgfqpoint{3.059078in}{2.993992in}}%
\pgfpathclose%
\pgfusepath{stroke,fill}%
\end{pgfscope}%
\begin{pgfscope}%
\pgfpathrectangle{\pgfqpoint{0.100000in}{0.220728in}}{\pgfqpoint{3.696000in}{3.696000in}}%
\pgfusepath{clip}%
\pgfsetbuttcap%
\pgfsetroundjoin%
\definecolor{currentfill}{rgb}{0.121569,0.466667,0.705882}%
\pgfsetfillcolor{currentfill}%
\pgfsetfillopacity{0.599894}%
\pgfsetlinewidth{1.003750pt}%
\definecolor{currentstroke}{rgb}{0.121569,0.466667,0.705882}%
\pgfsetstrokecolor{currentstroke}%
\pgfsetstrokeopacity{0.599894}%
\pgfsetdash{}{0pt}%
\pgfpathmoveto{\pgfqpoint{0.865697in}{1.324499in}}%
\pgfpathcurveto{\pgfqpoint{0.873934in}{1.324499in}}{\pgfqpoint{0.881834in}{1.327771in}}{\pgfqpoint{0.887658in}{1.333595in}}%
\pgfpathcurveto{\pgfqpoint{0.893482in}{1.339419in}}{\pgfqpoint{0.896754in}{1.347319in}}{\pgfqpoint{0.896754in}{1.355555in}}%
\pgfpathcurveto{\pgfqpoint{0.896754in}{1.363791in}}{\pgfqpoint{0.893482in}{1.371691in}}{\pgfqpoint{0.887658in}{1.377515in}}%
\pgfpathcurveto{\pgfqpoint{0.881834in}{1.383339in}}{\pgfqpoint{0.873934in}{1.386612in}}{\pgfqpoint{0.865697in}{1.386612in}}%
\pgfpathcurveto{\pgfqpoint{0.857461in}{1.386612in}}{\pgfqpoint{0.849561in}{1.383339in}}{\pgfqpoint{0.843737in}{1.377515in}}%
\pgfpathcurveto{\pgfqpoint{0.837913in}{1.371691in}}{\pgfqpoint{0.834641in}{1.363791in}}{\pgfqpoint{0.834641in}{1.355555in}}%
\pgfpathcurveto{\pgfqpoint{0.834641in}{1.347319in}}{\pgfqpoint{0.837913in}{1.339419in}}{\pgfqpoint{0.843737in}{1.333595in}}%
\pgfpathcurveto{\pgfqpoint{0.849561in}{1.327771in}}{\pgfqpoint{0.857461in}{1.324499in}}{\pgfqpoint{0.865697in}{1.324499in}}%
\pgfpathclose%
\pgfusepath{stroke,fill}%
\end{pgfscope}%
\begin{pgfscope}%
\pgfpathrectangle{\pgfqpoint{0.100000in}{0.220728in}}{\pgfqpoint{3.696000in}{3.696000in}}%
\pgfusepath{clip}%
\pgfsetbuttcap%
\pgfsetroundjoin%
\definecolor{currentfill}{rgb}{0.121569,0.466667,0.705882}%
\pgfsetfillcolor{currentfill}%
\pgfsetfillopacity{0.600011}%
\pgfsetlinewidth{1.003750pt}%
\definecolor{currentstroke}{rgb}{0.121569,0.466667,0.705882}%
\pgfsetstrokecolor{currentstroke}%
\pgfsetstrokeopacity{0.600011}%
\pgfsetdash{}{0pt}%
\pgfpathmoveto{\pgfqpoint{3.064006in}{2.993176in}}%
\pgfpathcurveto{\pgfqpoint{3.072242in}{2.993176in}}{\pgfqpoint{3.080142in}{2.996448in}}{\pgfqpoint{3.085966in}{3.002272in}}%
\pgfpathcurveto{\pgfqpoint{3.091790in}{3.008096in}}{\pgfqpoint{3.095063in}{3.015996in}}{\pgfqpoint{3.095063in}{3.024233in}}%
\pgfpathcurveto{\pgfqpoint{3.095063in}{3.032469in}}{\pgfqpoint{3.091790in}{3.040369in}}{\pgfqpoint{3.085966in}{3.046193in}}%
\pgfpathcurveto{\pgfqpoint{3.080142in}{3.052017in}}{\pgfqpoint{3.072242in}{3.055289in}}{\pgfqpoint{3.064006in}{3.055289in}}%
\pgfpathcurveto{\pgfqpoint{3.055770in}{3.055289in}}{\pgfqpoint{3.047870in}{3.052017in}}{\pgfqpoint{3.042046in}{3.046193in}}%
\pgfpathcurveto{\pgfqpoint{3.036222in}{3.040369in}}{\pgfqpoint{3.032950in}{3.032469in}}{\pgfqpoint{3.032950in}{3.024233in}}%
\pgfpathcurveto{\pgfqpoint{3.032950in}{3.015996in}}{\pgfqpoint{3.036222in}{3.008096in}}{\pgfqpoint{3.042046in}{3.002272in}}%
\pgfpathcurveto{\pgfqpoint{3.047870in}{2.996448in}}{\pgfqpoint{3.055770in}{2.993176in}}{\pgfqpoint{3.064006in}{2.993176in}}%
\pgfpathclose%
\pgfusepath{stroke,fill}%
\end{pgfscope}%
\begin{pgfscope}%
\pgfpathrectangle{\pgfqpoint{0.100000in}{0.220728in}}{\pgfqpoint{3.696000in}{3.696000in}}%
\pgfusepath{clip}%
\pgfsetbuttcap%
\pgfsetroundjoin%
\definecolor{currentfill}{rgb}{0.121569,0.466667,0.705882}%
\pgfsetfillcolor{currentfill}%
\pgfsetfillopacity{0.600322}%
\pgfsetlinewidth{1.003750pt}%
\definecolor{currentstroke}{rgb}{0.121569,0.466667,0.705882}%
\pgfsetstrokecolor{currentstroke}%
\pgfsetstrokeopacity{0.600322}%
\pgfsetdash{}{0pt}%
\pgfpathmoveto{\pgfqpoint{0.863323in}{1.321333in}}%
\pgfpathcurveto{\pgfqpoint{0.871560in}{1.321333in}}{\pgfqpoint{0.879460in}{1.324605in}}{\pgfqpoint{0.885283in}{1.330429in}}%
\pgfpathcurveto{\pgfqpoint{0.891107in}{1.336253in}}{\pgfqpoint{0.894380in}{1.344153in}}{\pgfqpoint{0.894380in}{1.352389in}}%
\pgfpathcurveto{\pgfqpoint{0.894380in}{1.360626in}}{\pgfqpoint{0.891107in}{1.368526in}}{\pgfqpoint{0.885283in}{1.374350in}}%
\pgfpathcurveto{\pgfqpoint{0.879460in}{1.380173in}}{\pgfqpoint{0.871560in}{1.383446in}}{\pgfqpoint{0.863323in}{1.383446in}}%
\pgfpathcurveto{\pgfqpoint{0.855087in}{1.383446in}}{\pgfqpoint{0.847187in}{1.380173in}}{\pgfqpoint{0.841363in}{1.374350in}}%
\pgfpathcurveto{\pgfqpoint{0.835539in}{1.368526in}}{\pgfqpoint{0.832267in}{1.360626in}}{\pgfqpoint{0.832267in}{1.352389in}}%
\pgfpathcurveto{\pgfqpoint{0.832267in}{1.344153in}}{\pgfqpoint{0.835539in}{1.336253in}}{\pgfqpoint{0.841363in}{1.330429in}}%
\pgfpathcurveto{\pgfqpoint{0.847187in}{1.324605in}}{\pgfqpoint{0.855087in}{1.321333in}}{\pgfqpoint{0.863323in}{1.321333in}}%
\pgfpathclose%
\pgfusepath{stroke,fill}%
\end{pgfscope}%
\begin{pgfscope}%
\pgfpathrectangle{\pgfqpoint{0.100000in}{0.220728in}}{\pgfqpoint{3.696000in}{3.696000in}}%
\pgfusepath{clip}%
\pgfsetbuttcap%
\pgfsetroundjoin%
\definecolor{currentfill}{rgb}{0.121569,0.466667,0.705882}%
\pgfsetfillcolor{currentfill}%
\pgfsetfillopacity{0.601352}%
\pgfsetlinewidth{1.003750pt}%
\definecolor{currentstroke}{rgb}{0.121569,0.466667,0.705882}%
\pgfsetstrokecolor{currentstroke}%
\pgfsetstrokeopacity{0.601352}%
\pgfsetdash{}{0pt}%
\pgfpathmoveto{\pgfqpoint{3.070712in}{2.991878in}}%
\pgfpathcurveto{\pgfqpoint{3.078949in}{2.991878in}}{\pgfqpoint{3.086849in}{2.995151in}}{\pgfqpoint{3.092673in}{3.000975in}}%
\pgfpathcurveto{\pgfqpoint{3.098496in}{3.006799in}}{\pgfqpoint{3.101769in}{3.014699in}}{\pgfqpoint{3.101769in}{3.022935in}}%
\pgfpathcurveto{\pgfqpoint{3.101769in}{3.031171in}}{\pgfqpoint{3.098496in}{3.039071in}}{\pgfqpoint{3.092673in}{3.044895in}}%
\pgfpathcurveto{\pgfqpoint{3.086849in}{3.050719in}}{\pgfqpoint{3.078949in}{3.053991in}}{\pgfqpoint{3.070712in}{3.053991in}}%
\pgfpathcurveto{\pgfqpoint{3.062476in}{3.053991in}}{\pgfqpoint{3.054576in}{3.050719in}}{\pgfqpoint{3.048752in}{3.044895in}}%
\pgfpathcurveto{\pgfqpoint{3.042928in}{3.039071in}}{\pgfqpoint{3.039656in}{3.031171in}}{\pgfqpoint{3.039656in}{3.022935in}}%
\pgfpathcurveto{\pgfqpoint{3.039656in}{3.014699in}}{\pgfqpoint{3.042928in}{3.006799in}}{\pgfqpoint{3.048752in}{3.000975in}}%
\pgfpathcurveto{\pgfqpoint{3.054576in}{2.995151in}}{\pgfqpoint{3.062476in}{2.991878in}}{\pgfqpoint{3.070712in}{2.991878in}}%
\pgfpathclose%
\pgfusepath{stroke,fill}%
\end{pgfscope}%
\begin{pgfscope}%
\pgfpathrectangle{\pgfqpoint{0.100000in}{0.220728in}}{\pgfqpoint{3.696000in}{3.696000in}}%
\pgfusepath{clip}%
\pgfsetbuttcap%
\pgfsetroundjoin%
\definecolor{currentfill}{rgb}{0.121569,0.466667,0.705882}%
\pgfsetfillcolor{currentfill}%
\pgfsetfillopacity{0.601518}%
\pgfsetlinewidth{1.003750pt}%
\definecolor{currentstroke}{rgb}{0.121569,0.466667,0.705882}%
\pgfsetstrokecolor{currentstroke}%
\pgfsetstrokeopacity{0.601518}%
\pgfsetdash{}{0pt}%
\pgfpathmoveto{\pgfqpoint{0.860923in}{1.314718in}}%
\pgfpathcurveto{\pgfqpoint{0.869160in}{1.314718in}}{\pgfqpoint{0.877060in}{1.317990in}}{\pgfqpoint{0.882884in}{1.323814in}}%
\pgfpathcurveto{\pgfqpoint{0.888708in}{1.329638in}}{\pgfqpoint{0.891980in}{1.337538in}}{\pgfqpoint{0.891980in}{1.345774in}}%
\pgfpathcurveto{\pgfqpoint{0.891980in}{1.354011in}}{\pgfqpoint{0.888708in}{1.361911in}}{\pgfqpoint{0.882884in}{1.367735in}}%
\pgfpathcurveto{\pgfqpoint{0.877060in}{1.373558in}}{\pgfqpoint{0.869160in}{1.376831in}}{\pgfqpoint{0.860923in}{1.376831in}}%
\pgfpathcurveto{\pgfqpoint{0.852687in}{1.376831in}}{\pgfqpoint{0.844787in}{1.373558in}}{\pgfqpoint{0.838963in}{1.367735in}}%
\pgfpathcurveto{\pgfqpoint{0.833139in}{1.361911in}}{\pgfqpoint{0.829867in}{1.354011in}}{\pgfqpoint{0.829867in}{1.345774in}}%
\pgfpathcurveto{\pgfqpoint{0.829867in}{1.337538in}}{\pgfqpoint{0.833139in}{1.329638in}}{\pgfqpoint{0.838963in}{1.323814in}}%
\pgfpathcurveto{\pgfqpoint{0.844787in}{1.317990in}}{\pgfqpoint{0.852687in}{1.314718in}}{\pgfqpoint{0.860923in}{1.314718in}}%
\pgfpathclose%
\pgfusepath{stroke,fill}%
\end{pgfscope}%
\begin{pgfscope}%
\pgfpathrectangle{\pgfqpoint{0.100000in}{0.220728in}}{\pgfqpoint{3.696000in}{3.696000in}}%
\pgfusepath{clip}%
\pgfsetbuttcap%
\pgfsetroundjoin%
\definecolor{currentfill}{rgb}{0.121569,0.466667,0.705882}%
\pgfsetfillcolor{currentfill}%
\pgfsetfillopacity{0.602101}%
\pgfsetlinewidth{1.003750pt}%
\definecolor{currentstroke}{rgb}{0.121569,0.466667,0.705882}%
\pgfsetstrokecolor{currentstroke}%
\pgfsetstrokeopacity{0.602101}%
\pgfsetdash{}{0pt}%
\pgfpathmoveto{\pgfqpoint{0.857407in}{1.310084in}}%
\pgfpathcurveto{\pgfqpoint{0.865643in}{1.310084in}}{\pgfqpoint{0.873543in}{1.313357in}}{\pgfqpoint{0.879367in}{1.319181in}}%
\pgfpathcurveto{\pgfqpoint{0.885191in}{1.325005in}}{\pgfqpoint{0.888463in}{1.332905in}}{\pgfqpoint{0.888463in}{1.341141in}}%
\pgfpathcurveto{\pgfqpoint{0.888463in}{1.349377in}}{\pgfqpoint{0.885191in}{1.357277in}}{\pgfqpoint{0.879367in}{1.363101in}}%
\pgfpathcurveto{\pgfqpoint{0.873543in}{1.368925in}}{\pgfqpoint{0.865643in}{1.372197in}}{\pgfqpoint{0.857407in}{1.372197in}}%
\pgfpathcurveto{\pgfqpoint{0.849170in}{1.372197in}}{\pgfqpoint{0.841270in}{1.368925in}}{\pgfqpoint{0.835446in}{1.363101in}}%
\pgfpathcurveto{\pgfqpoint{0.829622in}{1.357277in}}{\pgfqpoint{0.826350in}{1.349377in}}{\pgfqpoint{0.826350in}{1.341141in}}%
\pgfpathcurveto{\pgfqpoint{0.826350in}{1.332905in}}{\pgfqpoint{0.829622in}{1.325005in}}{\pgfqpoint{0.835446in}{1.319181in}}%
\pgfpathcurveto{\pgfqpoint{0.841270in}{1.313357in}}{\pgfqpoint{0.849170in}{1.310084in}}{\pgfqpoint{0.857407in}{1.310084in}}%
\pgfpathclose%
\pgfusepath{stroke,fill}%
\end{pgfscope}%
\begin{pgfscope}%
\pgfpathrectangle{\pgfqpoint{0.100000in}{0.220728in}}{\pgfqpoint{3.696000in}{3.696000in}}%
\pgfusepath{clip}%
\pgfsetbuttcap%
\pgfsetroundjoin%
\definecolor{currentfill}{rgb}{0.121569,0.466667,0.705882}%
\pgfsetfillcolor{currentfill}%
\pgfsetfillopacity{0.602250}%
\pgfsetlinewidth{1.003750pt}%
\definecolor{currentstroke}{rgb}{0.121569,0.466667,0.705882}%
\pgfsetstrokecolor{currentstroke}%
\pgfsetstrokeopacity{0.602250}%
\pgfsetdash{}{0pt}%
\pgfpathmoveto{\pgfqpoint{3.079067in}{2.989890in}}%
\pgfpathcurveto{\pgfqpoint{3.087303in}{2.989890in}}{\pgfqpoint{3.095203in}{2.993162in}}{\pgfqpoint{3.101027in}{2.998986in}}%
\pgfpathcurveto{\pgfqpoint{3.106851in}{3.004810in}}{\pgfqpoint{3.110123in}{3.012710in}}{\pgfqpoint{3.110123in}{3.020947in}}%
\pgfpathcurveto{\pgfqpoint{3.110123in}{3.029183in}}{\pgfqpoint{3.106851in}{3.037083in}}{\pgfqpoint{3.101027in}{3.042907in}}%
\pgfpathcurveto{\pgfqpoint{3.095203in}{3.048731in}}{\pgfqpoint{3.087303in}{3.052003in}}{\pgfqpoint{3.079067in}{3.052003in}}%
\pgfpathcurveto{\pgfqpoint{3.070830in}{3.052003in}}{\pgfqpoint{3.062930in}{3.048731in}}{\pgfqpoint{3.057106in}{3.042907in}}%
\pgfpathcurveto{\pgfqpoint{3.051282in}{3.037083in}}{\pgfqpoint{3.048010in}{3.029183in}}{\pgfqpoint{3.048010in}{3.020947in}}%
\pgfpathcurveto{\pgfqpoint{3.048010in}{3.012710in}}{\pgfqpoint{3.051282in}{3.004810in}}{\pgfqpoint{3.057106in}{2.998986in}}%
\pgfpathcurveto{\pgfqpoint{3.062930in}{2.993162in}}{\pgfqpoint{3.070830in}{2.989890in}}{\pgfqpoint{3.079067in}{2.989890in}}%
\pgfpathclose%
\pgfusepath{stroke,fill}%
\end{pgfscope}%
\begin{pgfscope}%
\pgfpathrectangle{\pgfqpoint{0.100000in}{0.220728in}}{\pgfqpoint{3.696000in}{3.696000in}}%
\pgfusepath{clip}%
\pgfsetbuttcap%
\pgfsetroundjoin%
\definecolor{currentfill}{rgb}{0.121569,0.466667,0.705882}%
\pgfsetfillcolor{currentfill}%
\pgfsetfillopacity{0.602713}%
\pgfsetlinewidth{1.003750pt}%
\definecolor{currentstroke}{rgb}{0.121569,0.466667,0.705882}%
\pgfsetstrokecolor{currentstroke}%
\pgfsetstrokeopacity{0.602713}%
\pgfsetdash{}{0pt}%
\pgfpathmoveto{\pgfqpoint{0.856433in}{1.306826in}}%
\pgfpathcurveto{\pgfqpoint{0.864669in}{1.306826in}}{\pgfqpoint{0.872569in}{1.310098in}}{\pgfqpoint{0.878393in}{1.315922in}}%
\pgfpathcurveto{\pgfqpoint{0.884217in}{1.321746in}}{\pgfqpoint{0.887489in}{1.329646in}}{\pgfqpoint{0.887489in}{1.337882in}}%
\pgfpathcurveto{\pgfqpoint{0.887489in}{1.346118in}}{\pgfqpoint{0.884217in}{1.354018in}}{\pgfqpoint{0.878393in}{1.359842in}}%
\pgfpathcurveto{\pgfqpoint{0.872569in}{1.365666in}}{\pgfqpoint{0.864669in}{1.368939in}}{\pgfqpoint{0.856433in}{1.368939in}}%
\pgfpathcurveto{\pgfqpoint{0.848197in}{1.368939in}}{\pgfqpoint{0.840297in}{1.365666in}}{\pgfqpoint{0.834473in}{1.359842in}}%
\pgfpathcurveto{\pgfqpoint{0.828649in}{1.354018in}}{\pgfqpoint{0.825376in}{1.346118in}}{\pgfqpoint{0.825376in}{1.337882in}}%
\pgfpathcurveto{\pgfqpoint{0.825376in}{1.329646in}}{\pgfqpoint{0.828649in}{1.321746in}}{\pgfqpoint{0.834473in}{1.315922in}}%
\pgfpathcurveto{\pgfqpoint{0.840297in}{1.310098in}}{\pgfqpoint{0.848197in}{1.306826in}}{\pgfqpoint{0.856433in}{1.306826in}}%
\pgfpathclose%
\pgfusepath{stroke,fill}%
\end{pgfscope}%
\begin{pgfscope}%
\pgfpathrectangle{\pgfqpoint{0.100000in}{0.220728in}}{\pgfqpoint{3.696000in}{3.696000in}}%
\pgfusepath{clip}%
\pgfsetbuttcap%
\pgfsetroundjoin%
\definecolor{currentfill}{rgb}{0.121569,0.466667,0.705882}%
\pgfsetfillcolor{currentfill}%
\pgfsetfillopacity{0.602915}%
\pgfsetlinewidth{1.003750pt}%
\definecolor{currentstroke}{rgb}{0.121569,0.466667,0.705882}%
\pgfsetstrokecolor{currentstroke}%
\pgfsetstrokeopacity{0.602915}%
\pgfsetdash{}{0pt}%
\pgfpathmoveto{\pgfqpoint{0.855221in}{1.305253in}}%
\pgfpathcurveto{\pgfqpoint{0.863457in}{1.305253in}}{\pgfqpoint{0.871357in}{1.308525in}}{\pgfqpoint{0.877181in}{1.314349in}}%
\pgfpathcurveto{\pgfqpoint{0.883005in}{1.320173in}}{\pgfqpoint{0.886278in}{1.328073in}}{\pgfqpoint{0.886278in}{1.336310in}}%
\pgfpathcurveto{\pgfqpoint{0.886278in}{1.344546in}}{\pgfqpoint{0.883005in}{1.352446in}}{\pgfqpoint{0.877181in}{1.358270in}}%
\pgfpathcurveto{\pgfqpoint{0.871357in}{1.364094in}}{\pgfqpoint{0.863457in}{1.367366in}}{\pgfqpoint{0.855221in}{1.367366in}}%
\pgfpathcurveto{\pgfqpoint{0.846985in}{1.367366in}}{\pgfqpoint{0.839085in}{1.364094in}}{\pgfqpoint{0.833261in}{1.358270in}}%
\pgfpathcurveto{\pgfqpoint{0.827437in}{1.352446in}}{\pgfqpoint{0.824165in}{1.344546in}}{\pgfqpoint{0.824165in}{1.336310in}}%
\pgfpathcurveto{\pgfqpoint{0.824165in}{1.328073in}}{\pgfqpoint{0.827437in}{1.320173in}}{\pgfqpoint{0.833261in}{1.314349in}}%
\pgfpathcurveto{\pgfqpoint{0.839085in}{1.308525in}}{\pgfqpoint{0.846985in}{1.305253in}}{\pgfqpoint{0.855221in}{1.305253in}}%
\pgfpathclose%
\pgfusepath{stroke,fill}%
\end{pgfscope}%
\begin{pgfscope}%
\pgfpathrectangle{\pgfqpoint{0.100000in}{0.220728in}}{\pgfqpoint{3.696000in}{3.696000in}}%
\pgfusepath{clip}%
\pgfsetbuttcap%
\pgfsetroundjoin%
\definecolor{currentfill}{rgb}{0.121569,0.466667,0.705882}%
\pgfsetfillcolor{currentfill}%
\pgfsetfillopacity{0.603547}%
\pgfsetlinewidth{1.003750pt}%
\definecolor{currentstroke}{rgb}{0.121569,0.466667,0.705882}%
\pgfsetstrokecolor{currentstroke}%
\pgfsetstrokeopacity{0.603547}%
\pgfsetdash{}{0pt}%
\pgfpathmoveto{\pgfqpoint{0.854333in}{1.301896in}}%
\pgfpathcurveto{\pgfqpoint{0.862569in}{1.301896in}}{\pgfqpoint{0.870469in}{1.305168in}}{\pgfqpoint{0.876293in}{1.310992in}}%
\pgfpathcurveto{\pgfqpoint{0.882117in}{1.316816in}}{\pgfqpoint{0.885389in}{1.324716in}}{\pgfqpoint{0.885389in}{1.332952in}}%
\pgfpathcurveto{\pgfqpoint{0.885389in}{1.341188in}}{\pgfqpoint{0.882117in}{1.349088in}}{\pgfqpoint{0.876293in}{1.354912in}}%
\pgfpathcurveto{\pgfqpoint{0.870469in}{1.360736in}}{\pgfqpoint{0.862569in}{1.364009in}}{\pgfqpoint{0.854333in}{1.364009in}}%
\pgfpathcurveto{\pgfqpoint{0.846097in}{1.364009in}}{\pgfqpoint{0.838197in}{1.360736in}}{\pgfqpoint{0.832373in}{1.354912in}}%
\pgfpathcurveto{\pgfqpoint{0.826549in}{1.349088in}}{\pgfqpoint{0.823276in}{1.341188in}}{\pgfqpoint{0.823276in}{1.332952in}}%
\pgfpathcurveto{\pgfqpoint{0.823276in}{1.324716in}}{\pgfqpoint{0.826549in}{1.316816in}}{\pgfqpoint{0.832373in}{1.310992in}}%
\pgfpathcurveto{\pgfqpoint{0.838197in}{1.305168in}}{\pgfqpoint{0.846097in}{1.301896in}}{\pgfqpoint{0.854333in}{1.301896in}}%
\pgfpathclose%
\pgfusepath{stroke,fill}%
\end{pgfscope}%
\begin{pgfscope}%
\pgfpathrectangle{\pgfqpoint{0.100000in}{0.220728in}}{\pgfqpoint{3.696000in}{3.696000in}}%
\pgfusepath{clip}%
\pgfsetbuttcap%
\pgfsetroundjoin%
\definecolor{currentfill}{rgb}{0.121569,0.466667,0.705882}%
\pgfsetfillcolor{currentfill}%
\pgfsetfillopacity{0.603806}%
\pgfsetlinewidth{1.003750pt}%
\definecolor{currentstroke}{rgb}{0.121569,0.466667,0.705882}%
\pgfsetstrokecolor{currentstroke}%
\pgfsetstrokeopacity{0.603806}%
\pgfsetdash{}{0pt}%
\pgfpathmoveto{\pgfqpoint{0.853196in}{1.300135in}}%
\pgfpathcurveto{\pgfqpoint{0.861432in}{1.300135in}}{\pgfqpoint{0.869332in}{1.303407in}}{\pgfqpoint{0.875156in}{1.309231in}}%
\pgfpathcurveto{\pgfqpoint{0.880980in}{1.315055in}}{\pgfqpoint{0.884252in}{1.322955in}}{\pgfqpoint{0.884252in}{1.331191in}}%
\pgfpathcurveto{\pgfqpoint{0.884252in}{1.339428in}}{\pgfqpoint{0.880980in}{1.347328in}}{\pgfqpoint{0.875156in}{1.353152in}}%
\pgfpathcurveto{\pgfqpoint{0.869332in}{1.358976in}}{\pgfqpoint{0.861432in}{1.362248in}}{\pgfqpoint{0.853196in}{1.362248in}}%
\pgfpathcurveto{\pgfqpoint{0.844960in}{1.362248in}}{\pgfqpoint{0.837060in}{1.358976in}}{\pgfqpoint{0.831236in}{1.353152in}}%
\pgfpathcurveto{\pgfqpoint{0.825412in}{1.347328in}}{\pgfqpoint{0.822139in}{1.339428in}}{\pgfqpoint{0.822139in}{1.331191in}}%
\pgfpathcurveto{\pgfqpoint{0.822139in}{1.322955in}}{\pgfqpoint{0.825412in}{1.315055in}}{\pgfqpoint{0.831236in}{1.309231in}}%
\pgfpathcurveto{\pgfqpoint{0.837060in}{1.303407in}}{\pgfqpoint{0.844960in}{1.300135in}}{\pgfqpoint{0.853196in}{1.300135in}}%
\pgfpathclose%
\pgfusepath{stroke,fill}%
\end{pgfscope}%
\begin{pgfscope}%
\pgfpathrectangle{\pgfqpoint{0.100000in}{0.220728in}}{\pgfqpoint{3.696000in}{3.696000in}}%
\pgfusepath{clip}%
\pgfsetbuttcap%
\pgfsetroundjoin%
\definecolor{currentfill}{rgb}{0.121569,0.466667,0.705882}%
\pgfsetfillcolor{currentfill}%
\pgfsetfillopacity{0.604396}%
\pgfsetlinewidth{1.003750pt}%
\definecolor{currentstroke}{rgb}{0.121569,0.466667,0.705882}%
\pgfsetstrokecolor{currentstroke}%
\pgfsetstrokeopacity{0.604396}%
\pgfsetdash{}{0pt}%
\pgfpathmoveto{\pgfqpoint{0.852610in}{1.296175in}}%
\pgfpathcurveto{\pgfqpoint{0.860846in}{1.296175in}}{\pgfqpoint{0.868746in}{1.299448in}}{\pgfqpoint{0.874570in}{1.305271in}}%
\pgfpathcurveto{\pgfqpoint{0.880394in}{1.311095in}}{\pgfqpoint{0.883666in}{1.318995in}}{\pgfqpoint{0.883666in}{1.327232in}}%
\pgfpathcurveto{\pgfqpoint{0.883666in}{1.335468in}}{\pgfqpoint{0.880394in}{1.343368in}}{\pgfqpoint{0.874570in}{1.349192in}}%
\pgfpathcurveto{\pgfqpoint{0.868746in}{1.355016in}}{\pgfqpoint{0.860846in}{1.358288in}}{\pgfqpoint{0.852610in}{1.358288in}}%
\pgfpathcurveto{\pgfqpoint{0.844373in}{1.358288in}}{\pgfqpoint{0.836473in}{1.355016in}}{\pgfqpoint{0.830649in}{1.349192in}}%
\pgfpathcurveto{\pgfqpoint{0.824825in}{1.343368in}}{\pgfqpoint{0.821553in}{1.335468in}}{\pgfqpoint{0.821553in}{1.327232in}}%
\pgfpathcurveto{\pgfqpoint{0.821553in}{1.318995in}}{\pgfqpoint{0.824825in}{1.311095in}}{\pgfqpoint{0.830649in}{1.305271in}}%
\pgfpathcurveto{\pgfqpoint{0.836473in}{1.299448in}}{\pgfqpoint{0.844373in}{1.296175in}}{\pgfqpoint{0.852610in}{1.296175in}}%
\pgfpathclose%
\pgfusepath{stroke,fill}%
\end{pgfscope}%
\begin{pgfscope}%
\pgfpathrectangle{\pgfqpoint{0.100000in}{0.220728in}}{\pgfqpoint{3.696000in}{3.696000in}}%
\pgfusepath{clip}%
\pgfsetbuttcap%
\pgfsetroundjoin%
\definecolor{currentfill}{rgb}{0.121569,0.466667,0.705882}%
\pgfsetfillcolor{currentfill}%
\pgfsetfillopacity{0.604861}%
\pgfsetlinewidth{1.003750pt}%
\definecolor{currentstroke}{rgb}{0.121569,0.466667,0.705882}%
\pgfsetstrokecolor{currentstroke}%
\pgfsetstrokeopacity{0.604861}%
\pgfsetdash{}{0pt}%
\pgfpathmoveto{\pgfqpoint{3.087994in}{2.989761in}}%
\pgfpathcurveto{\pgfqpoint{3.096230in}{2.989761in}}{\pgfqpoint{3.104130in}{2.993033in}}{\pgfqpoint{3.109954in}{2.998857in}}%
\pgfpathcurveto{\pgfqpoint{3.115778in}{3.004681in}}{\pgfqpoint{3.119050in}{3.012581in}}{\pgfqpoint{3.119050in}{3.020817in}}%
\pgfpathcurveto{\pgfqpoint{3.119050in}{3.029054in}}{\pgfqpoint{3.115778in}{3.036954in}}{\pgfqpoint{3.109954in}{3.042778in}}%
\pgfpathcurveto{\pgfqpoint{3.104130in}{3.048602in}}{\pgfqpoint{3.096230in}{3.051874in}}{\pgfqpoint{3.087994in}{3.051874in}}%
\pgfpathcurveto{\pgfqpoint{3.079758in}{3.051874in}}{\pgfqpoint{3.071858in}{3.048602in}}{\pgfqpoint{3.066034in}{3.042778in}}%
\pgfpathcurveto{\pgfqpoint{3.060210in}{3.036954in}}{\pgfqpoint{3.056937in}{3.029054in}}{\pgfqpoint{3.056937in}{3.020817in}}%
\pgfpathcurveto{\pgfqpoint{3.056937in}{3.012581in}}{\pgfqpoint{3.060210in}{3.004681in}}{\pgfqpoint{3.066034in}{2.998857in}}%
\pgfpathcurveto{\pgfqpoint{3.071858in}{2.993033in}}{\pgfqpoint{3.079758in}{2.989761in}}{\pgfqpoint{3.087994in}{2.989761in}}%
\pgfpathclose%
\pgfusepath{stroke,fill}%
\end{pgfscope}%
\begin{pgfscope}%
\pgfpathrectangle{\pgfqpoint{0.100000in}{0.220728in}}{\pgfqpoint{3.696000in}{3.696000in}}%
\pgfusepath{clip}%
\pgfsetbuttcap%
\pgfsetroundjoin%
\definecolor{currentfill}{rgb}{0.121569,0.466667,0.705882}%
\pgfsetfillcolor{currentfill}%
\pgfsetfillopacity{0.604898}%
\pgfsetlinewidth{1.003750pt}%
\definecolor{currentstroke}{rgb}{0.121569,0.466667,0.705882}%
\pgfsetstrokecolor{currentstroke}%
\pgfsetstrokeopacity{0.604898}%
\pgfsetdash{}{0pt}%
\pgfpathmoveto{\pgfqpoint{0.848291in}{1.290123in}}%
\pgfpathcurveto{\pgfqpoint{0.856528in}{1.290123in}}{\pgfqpoint{0.864428in}{1.293396in}}{\pgfqpoint{0.870252in}{1.299220in}}%
\pgfpathcurveto{\pgfqpoint{0.876076in}{1.305044in}}{\pgfqpoint{0.879348in}{1.312944in}}{\pgfqpoint{0.879348in}{1.321180in}}%
\pgfpathcurveto{\pgfqpoint{0.879348in}{1.329416in}}{\pgfqpoint{0.876076in}{1.337316in}}{\pgfqpoint{0.870252in}{1.343140in}}%
\pgfpathcurveto{\pgfqpoint{0.864428in}{1.348964in}}{\pgfqpoint{0.856528in}{1.352236in}}{\pgfqpoint{0.848291in}{1.352236in}}%
\pgfpathcurveto{\pgfqpoint{0.840055in}{1.352236in}}{\pgfqpoint{0.832155in}{1.348964in}}{\pgfqpoint{0.826331in}{1.343140in}}%
\pgfpathcurveto{\pgfqpoint{0.820507in}{1.337316in}}{\pgfqpoint{0.817235in}{1.329416in}}{\pgfqpoint{0.817235in}{1.321180in}}%
\pgfpathcurveto{\pgfqpoint{0.817235in}{1.312944in}}{\pgfqpoint{0.820507in}{1.305044in}}{\pgfqpoint{0.826331in}{1.299220in}}%
\pgfpathcurveto{\pgfqpoint{0.832155in}{1.293396in}}{\pgfqpoint{0.840055in}{1.290123in}}{\pgfqpoint{0.848291in}{1.290123in}}%
\pgfpathclose%
\pgfusepath{stroke,fill}%
\end{pgfscope}%
\begin{pgfscope}%
\pgfpathrectangle{\pgfqpoint{0.100000in}{0.220728in}}{\pgfqpoint{3.696000in}{3.696000in}}%
\pgfusepath{clip}%
\pgfsetbuttcap%
\pgfsetroundjoin%
\definecolor{currentfill}{rgb}{0.121569,0.466667,0.705882}%
\pgfsetfillcolor{currentfill}%
\pgfsetfillopacity{0.605446}%
\pgfsetlinewidth{1.003750pt}%
\definecolor{currentstroke}{rgb}{0.121569,0.466667,0.705882}%
\pgfsetstrokecolor{currentstroke}%
\pgfsetstrokeopacity{0.605446}%
\pgfsetdash{}{0pt}%
\pgfpathmoveto{\pgfqpoint{0.847395in}{1.286116in}}%
\pgfpathcurveto{\pgfqpoint{0.855631in}{1.286116in}}{\pgfqpoint{0.863531in}{1.289389in}}{\pgfqpoint{0.869355in}{1.295213in}}%
\pgfpathcurveto{\pgfqpoint{0.875179in}{1.301036in}}{\pgfqpoint{0.878451in}{1.308937in}}{\pgfqpoint{0.878451in}{1.317173in}}%
\pgfpathcurveto{\pgfqpoint{0.878451in}{1.325409in}}{\pgfqpoint{0.875179in}{1.333309in}}{\pgfqpoint{0.869355in}{1.339133in}}%
\pgfpathcurveto{\pgfqpoint{0.863531in}{1.344957in}}{\pgfqpoint{0.855631in}{1.348229in}}{\pgfqpoint{0.847395in}{1.348229in}}%
\pgfpathcurveto{\pgfqpoint{0.839158in}{1.348229in}}{\pgfqpoint{0.831258in}{1.344957in}}{\pgfqpoint{0.825434in}{1.339133in}}%
\pgfpathcurveto{\pgfqpoint{0.819610in}{1.333309in}}{\pgfqpoint{0.816338in}{1.325409in}}{\pgfqpoint{0.816338in}{1.317173in}}%
\pgfpathcurveto{\pgfqpoint{0.816338in}{1.308937in}}{\pgfqpoint{0.819610in}{1.301036in}}{\pgfqpoint{0.825434in}{1.295213in}}%
\pgfpathcurveto{\pgfqpoint{0.831258in}{1.289389in}}{\pgfqpoint{0.839158in}{1.286116in}}{\pgfqpoint{0.847395in}{1.286116in}}%
\pgfpathclose%
\pgfusepath{stroke,fill}%
\end{pgfscope}%
\begin{pgfscope}%
\pgfpathrectangle{\pgfqpoint{0.100000in}{0.220728in}}{\pgfqpoint{3.696000in}{3.696000in}}%
\pgfusepath{clip}%
\pgfsetbuttcap%
\pgfsetroundjoin%
\definecolor{currentfill}{rgb}{0.121569,0.466667,0.705882}%
\pgfsetfillcolor{currentfill}%
\pgfsetfillopacity{0.605588}%
\pgfsetlinewidth{1.003750pt}%
\definecolor{currentstroke}{rgb}{0.121569,0.466667,0.705882}%
\pgfsetstrokecolor{currentstroke}%
\pgfsetstrokeopacity{0.605588}%
\pgfsetdash{}{0pt}%
\pgfpathmoveto{\pgfqpoint{0.846964in}{1.285221in}}%
\pgfpathcurveto{\pgfqpoint{0.855200in}{1.285221in}}{\pgfqpoint{0.863100in}{1.288493in}}{\pgfqpoint{0.868924in}{1.294317in}}%
\pgfpathcurveto{\pgfqpoint{0.874748in}{1.300141in}}{\pgfqpoint{0.878020in}{1.308041in}}{\pgfqpoint{0.878020in}{1.316277in}}%
\pgfpathcurveto{\pgfqpoint{0.878020in}{1.324514in}}{\pgfqpoint{0.874748in}{1.332414in}}{\pgfqpoint{0.868924in}{1.338238in}}%
\pgfpathcurveto{\pgfqpoint{0.863100in}{1.344062in}}{\pgfqpoint{0.855200in}{1.347334in}}{\pgfqpoint{0.846964in}{1.347334in}}%
\pgfpathcurveto{\pgfqpoint{0.838727in}{1.347334in}}{\pgfqpoint{0.830827in}{1.344062in}}{\pgfqpoint{0.825003in}{1.338238in}}%
\pgfpathcurveto{\pgfqpoint{0.819180in}{1.332414in}}{\pgfqpoint{0.815907in}{1.324514in}}{\pgfqpoint{0.815907in}{1.316277in}}%
\pgfpathcurveto{\pgfqpoint{0.815907in}{1.308041in}}{\pgfqpoint{0.819180in}{1.300141in}}{\pgfqpoint{0.825003in}{1.294317in}}%
\pgfpathcurveto{\pgfqpoint{0.830827in}{1.288493in}}{\pgfqpoint{0.838727in}{1.285221in}}{\pgfqpoint{0.846964in}{1.285221in}}%
\pgfpathclose%
\pgfusepath{stroke,fill}%
\end{pgfscope}%
\begin{pgfscope}%
\pgfpathrectangle{\pgfqpoint{0.100000in}{0.220728in}}{\pgfqpoint{3.696000in}{3.696000in}}%
\pgfusepath{clip}%
\pgfsetbuttcap%
\pgfsetroundjoin%
\definecolor{currentfill}{rgb}{0.121569,0.466667,0.705882}%
\pgfsetfillcolor{currentfill}%
\pgfsetfillopacity{0.605864}%
\pgfsetlinewidth{1.003750pt}%
\definecolor{currentstroke}{rgb}{0.121569,0.466667,0.705882}%
\pgfsetstrokecolor{currentstroke}%
\pgfsetstrokeopacity{0.605864}%
\pgfsetdash{}{0pt}%
\pgfpathmoveto{\pgfqpoint{0.846124in}{1.283730in}}%
\pgfpathcurveto{\pgfqpoint{0.854361in}{1.283730in}}{\pgfqpoint{0.862261in}{1.287002in}}{\pgfqpoint{0.868085in}{1.292826in}}%
\pgfpathcurveto{\pgfqpoint{0.873909in}{1.298650in}}{\pgfqpoint{0.877181in}{1.306550in}}{\pgfqpoint{0.877181in}{1.314787in}}%
\pgfpathcurveto{\pgfqpoint{0.877181in}{1.323023in}}{\pgfqpoint{0.873909in}{1.330923in}}{\pgfqpoint{0.868085in}{1.336747in}}%
\pgfpathcurveto{\pgfqpoint{0.862261in}{1.342571in}}{\pgfqpoint{0.854361in}{1.345843in}}{\pgfqpoint{0.846124in}{1.345843in}}%
\pgfpathcurveto{\pgfqpoint{0.837888in}{1.345843in}}{\pgfqpoint{0.829988in}{1.342571in}}{\pgfqpoint{0.824164in}{1.336747in}}%
\pgfpathcurveto{\pgfqpoint{0.818340in}{1.330923in}}{\pgfqpoint{0.815068in}{1.323023in}}{\pgfqpoint{0.815068in}{1.314787in}}%
\pgfpathcurveto{\pgfqpoint{0.815068in}{1.306550in}}{\pgfqpoint{0.818340in}{1.298650in}}{\pgfqpoint{0.824164in}{1.292826in}}%
\pgfpathcurveto{\pgfqpoint{0.829988in}{1.287002in}}{\pgfqpoint{0.837888in}{1.283730in}}{\pgfqpoint{0.846124in}{1.283730in}}%
\pgfpathclose%
\pgfusepath{stroke,fill}%
\end{pgfscope}%
\begin{pgfscope}%
\pgfpathrectangle{\pgfqpoint{0.100000in}{0.220728in}}{\pgfqpoint{3.696000in}{3.696000in}}%
\pgfusepath{clip}%
\pgfsetbuttcap%
\pgfsetroundjoin%
\definecolor{currentfill}{rgb}{0.121569,0.466667,0.705882}%
\pgfsetfillcolor{currentfill}%
\pgfsetfillopacity{0.605955}%
\pgfsetlinewidth{1.003750pt}%
\definecolor{currentstroke}{rgb}{0.121569,0.466667,0.705882}%
\pgfsetstrokecolor{currentstroke}%
\pgfsetstrokeopacity{0.605955}%
\pgfsetdash{}{0pt}%
\pgfpathmoveto{\pgfqpoint{0.845850in}{1.283154in}}%
\pgfpathcurveto{\pgfqpoint{0.854086in}{1.283154in}}{\pgfqpoint{0.861986in}{1.286426in}}{\pgfqpoint{0.867810in}{1.292250in}}%
\pgfpathcurveto{\pgfqpoint{0.873634in}{1.298074in}}{\pgfqpoint{0.876907in}{1.305974in}}{\pgfqpoint{0.876907in}{1.314210in}}%
\pgfpathcurveto{\pgfqpoint{0.876907in}{1.322446in}}{\pgfqpoint{0.873634in}{1.330346in}}{\pgfqpoint{0.867810in}{1.336170in}}%
\pgfpathcurveto{\pgfqpoint{0.861986in}{1.341994in}}{\pgfqpoint{0.854086in}{1.345267in}}{\pgfqpoint{0.845850in}{1.345267in}}%
\pgfpathcurveto{\pgfqpoint{0.837614in}{1.345267in}}{\pgfqpoint{0.829714in}{1.341994in}}{\pgfqpoint{0.823890in}{1.336170in}}%
\pgfpathcurveto{\pgfqpoint{0.818066in}{1.330346in}}{\pgfqpoint{0.814794in}{1.322446in}}{\pgfqpoint{0.814794in}{1.314210in}}%
\pgfpathcurveto{\pgfqpoint{0.814794in}{1.305974in}}{\pgfqpoint{0.818066in}{1.298074in}}{\pgfqpoint{0.823890in}{1.292250in}}%
\pgfpathcurveto{\pgfqpoint{0.829714in}{1.286426in}}{\pgfqpoint{0.837614in}{1.283154in}}{\pgfqpoint{0.845850in}{1.283154in}}%
\pgfpathclose%
\pgfusepath{stroke,fill}%
\end{pgfscope}%
\begin{pgfscope}%
\pgfpathrectangle{\pgfqpoint{0.100000in}{0.220728in}}{\pgfqpoint{3.696000in}{3.696000in}}%
\pgfusepath{clip}%
\pgfsetbuttcap%
\pgfsetroundjoin%
\definecolor{currentfill}{rgb}{0.121569,0.466667,0.705882}%
\pgfsetfillcolor{currentfill}%
\pgfsetfillopacity{0.606128}%
\pgfsetlinewidth{1.003750pt}%
\definecolor{currentstroke}{rgb}{0.121569,0.466667,0.705882}%
\pgfsetstrokecolor{currentstroke}%
\pgfsetstrokeopacity{0.606128}%
\pgfsetdash{}{0pt}%
\pgfpathmoveto{\pgfqpoint{0.845396in}{1.282088in}}%
\pgfpathcurveto{\pgfqpoint{0.853632in}{1.282088in}}{\pgfqpoint{0.861532in}{1.285360in}}{\pgfqpoint{0.867356in}{1.291184in}}%
\pgfpathcurveto{\pgfqpoint{0.873180in}{1.297008in}}{\pgfqpoint{0.876452in}{1.304908in}}{\pgfqpoint{0.876452in}{1.313144in}}%
\pgfpathcurveto{\pgfqpoint{0.876452in}{1.321381in}}{\pgfqpoint{0.873180in}{1.329281in}}{\pgfqpoint{0.867356in}{1.335105in}}%
\pgfpathcurveto{\pgfqpoint{0.861532in}{1.340929in}}{\pgfqpoint{0.853632in}{1.344201in}}{\pgfqpoint{0.845396in}{1.344201in}}%
\pgfpathcurveto{\pgfqpoint{0.837160in}{1.344201in}}{\pgfqpoint{0.829260in}{1.340929in}}{\pgfqpoint{0.823436in}{1.335105in}}%
\pgfpathcurveto{\pgfqpoint{0.817612in}{1.329281in}}{\pgfqpoint{0.814339in}{1.321381in}}{\pgfqpoint{0.814339in}{1.313144in}}%
\pgfpathcurveto{\pgfqpoint{0.814339in}{1.304908in}}{\pgfqpoint{0.817612in}{1.297008in}}{\pgfqpoint{0.823436in}{1.291184in}}%
\pgfpathcurveto{\pgfqpoint{0.829260in}{1.285360in}}{\pgfqpoint{0.837160in}{1.282088in}}{\pgfqpoint{0.845396in}{1.282088in}}%
\pgfpathclose%
\pgfusepath{stroke,fill}%
\end{pgfscope}%
\begin{pgfscope}%
\pgfpathrectangle{\pgfqpoint{0.100000in}{0.220728in}}{\pgfqpoint{3.696000in}{3.696000in}}%
\pgfusepath{clip}%
\pgfsetbuttcap%
\pgfsetroundjoin%
\definecolor{currentfill}{rgb}{0.121569,0.466667,0.705882}%
\pgfsetfillcolor{currentfill}%
\pgfsetfillopacity{0.606156}%
\pgfsetlinewidth{1.003750pt}%
\definecolor{currentstroke}{rgb}{0.121569,0.466667,0.705882}%
\pgfsetstrokecolor{currentstroke}%
\pgfsetstrokeopacity{0.606156}%
\pgfsetdash{}{0pt}%
\pgfpathmoveto{\pgfqpoint{0.845300in}{1.281908in}}%
\pgfpathcurveto{\pgfqpoint{0.853537in}{1.281908in}}{\pgfqpoint{0.861437in}{1.285180in}}{\pgfqpoint{0.867261in}{1.291004in}}%
\pgfpathcurveto{\pgfqpoint{0.873085in}{1.296828in}}{\pgfqpoint{0.876357in}{1.304728in}}{\pgfqpoint{0.876357in}{1.312964in}}%
\pgfpathcurveto{\pgfqpoint{0.876357in}{1.321201in}}{\pgfqpoint{0.873085in}{1.329101in}}{\pgfqpoint{0.867261in}{1.334925in}}%
\pgfpathcurveto{\pgfqpoint{0.861437in}{1.340748in}}{\pgfqpoint{0.853537in}{1.344021in}}{\pgfqpoint{0.845300in}{1.344021in}}%
\pgfpathcurveto{\pgfqpoint{0.837064in}{1.344021in}}{\pgfqpoint{0.829164in}{1.340748in}}{\pgfqpoint{0.823340in}{1.334925in}}%
\pgfpathcurveto{\pgfqpoint{0.817516in}{1.329101in}}{\pgfqpoint{0.814244in}{1.321201in}}{\pgfqpoint{0.814244in}{1.312964in}}%
\pgfpathcurveto{\pgfqpoint{0.814244in}{1.304728in}}{\pgfqpoint{0.817516in}{1.296828in}}{\pgfqpoint{0.823340in}{1.291004in}}%
\pgfpathcurveto{\pgfqpoint{0.829164in}{1.285180in}}{\pgfqpoint{0.837064in}{1.281908in}}{\pgfqpoint{0.845300in}{1.281908in}}%
\pgfpathclose%
\pgfusepath{stroke,fill}%
\end{pgfscope}%
\begin{pgfscope}%
\pgfpathrectangle{\pgfqpoint{0.100000in}{0.220728in}}{\pgfqpoint{3.696000in}{3.696000in}}%
\pgfusepath{clip}%
\pgfsetbuttcap%
\pgfsetroundjoin%
\definecolor{currentfill}{rgb}{0.121569,0.466667,0.705882}%
\pgfsetfillcolor{currentfill}%
\pgfsetfillopacity{0.606211}%
\pgfsetlinewidth{1.003750pt}%
\definecolor{currentstroke}{rgb}{0.121569,0.466667,0.705882}%
\pgfsetstrokecolor{currentstroke}%
\pgfsetstrokeopacity{0.606211}%
\pgfsetdash{}{0pt}%
\pgfpathmoveto{\pgfqpoint{0.845142in}{1.281580in}}%
\pgfpathcurveto{\pgfqpoint{0.853378in}{1.281580in}}{\pgfqpoint{0.861278in}{1.284852in}}{\pgfqpoint{0.867102in}{1.290676in}}%
\pgfpathcurveto{\pgfqpoint{0.872926in}{1.296500in}}{\pgfqpoint{0.876199in}{1.304400in}}{\pgfqpoint{0.876199in}{1.312637in}}%
\pgfpathcurveto{\pgfqpoint{0.876199in}{1.320873in}}{\pgfqpoint{0.872926in}{1.328773in}}{\pgfqpoint{0.867102in}{1.334597in}}%
\pgfpathcurveto{\pgfqpoint{0.861278in}{1.340421in}}{\pgfqpoint{0.853378in}{1.343693in}}{\pgfqpoint{0.845142in}{1.343693in}}%
\pgfpathcurveto{\pgfqpoint{0.836906in}{1.343693in}}{\pgfqpoint{0.829006in}{1.340421in}}{\pgfqpoint{0.823182in}{1.334597in}}%
\pgfpathcurveto{\pgfqpoint{0.817358in}{1.328773in}}{\pgfqpoint{0.814086in}{1.320873in}}{\pgfqpoint{0.814086in}{1.312637in}}%
\pgfpathcurveto{\pgfqpoint{0.814086in}{1.304400in}}{\pgfqpoint{0.817358in}{1.296500in}}{\pgfqpoint{0.823182in}{1.290676in}}%
\pgfpathcurveto{\pgfqpoint{0.829006in}{1.284852in}}{\pgfqpoint{0.836906in}{1.281580in}}{\pgfqpoint{0.845142in}{1.281580in}}%
\pgfpathclose%
\pgfusepath{stroke,fill}%
\end{pgfscope}%
\begin{pgfscope}%
\pgfpathrectangle{\pgfqpoint{0.100000in}{0.220728in}}{\pgfqpoint{3.696000in}{3.696000in}}%
\pgfusepath{clip}%
\pgfsetbuttcap%
\pgfsetroundjoin%
\definecolor{currentfill}{rgb}{0.121569,0.466667,0.705882}%
\pgfsetfillcolor{currentfill}%
\pgfsetfillopacity{0.606314}%
\pgfsetlinewidth{1.003750pt}%
\definecolor{currentstroke}{rgb}{0.121569,0.466667,0.705882}%
\pgfsetstrokecolor{currentstroke}%
\pgfsetstrokeopacity{0.606314}%
\pgfsetdash{}{0pt}%
\pgfpathmoveto{\pgfqpoint{0.844921in}{1.280939in}}%
\pgfpathcurveto{\pgfqpoint{0.853157in}{1.280939in}}{\pgfqpoint{0.861057in}{1.284212in}}{\pgfqpoint{0.866881in}{1.290036in}}%
\pgfpathcurveto{\pgfqpoint{0.872705in}{1.295859in}}{\pgfqpoint{0.875978in}{1.303760in}}{\pgfqpoint{0.875978in}{1.311996in}}%
\pgfpathcurveto{\pgfqpoint{0.875978in}{1.320232in}}{\pgfqpoint{0.872705in}{1.328132in}}{\pgfqpoint{0.866881in}{1.333956in}}%
\pgfpathcurveto{\pgfqpoint{0.861057in}{1.339780in}}{\pgfqpoint{0.853157in}{1.343052in}}{\pgfqpoint{0.844921in}{1.343052in}}%
\pgfpathcurveto{\pgfqpoint{0.836685in}{1.343052in}}{\pgfqpoint{0.828785in}{1.339780in}}{\pgfqpoint{0.822961in}{1.333956in}}%
\pgfpathcurveto{\pgfqpoint{0.817137in}{1.328132in}}{\pgfqpoint{0.813865in}{1.320232in}}{\pgfqpoint{0.813865in}{1.311996in}}%
\pgfpathcurveto{\pgfqpoint{0.813865in}{1.303760in}}{\pgfqpoint{0.817137in}{1.295859in}}{\pgfqpoint{0.822961in}{1.290036in}}%
\pgfpathcurveto{\pgfqpoint{0.828785in}{1.284212in}}{\pgfqpoint{0.836685in}{1.280939in}}{\pgfqpoint{0.844921in}{1.280939in}}%
\pgfpathclose%
\pgfusepath{stroke,fill}%
\end{pgfscope}%
\begin{pgfscope}%
\pgfpathrectangle{\pgfqpoint{0.100000in}{0.220728in}}{\pgfqpoint{3.696000in}{3.696000in}}%
\pgfusepath{clip}%
\pgfsetbuttcap%
\pgfsetroundjoin%
\definecolor{currentfill}{rgb}{0.121569,0.466667,0.705882}%
\pgfsetfillcolor{currentfill}%
\pgfsetfillopacity{0.606490}%
\pgfsetlinewidth{1.003750pt}%
\definecolor{currentstroke}{rgb}{0.121569,0.466667,0.705882}%
\pgfsetstrokecolor{currentstroke}%
\pgfsetstrokeopacity{0.606490}%
\pgfsetdash{}{0pt}%
\pgfpathmoveto{\pgfqpoint{0.844373in}{1.279864in}}%
\pgfpathcurveto{\pgfqpoint{0.852610in}{1.279864in}}{\pgfqpoint{0.860510in}{1.283136in}}{\pgfqpoint{0.866334in}{1.288960in}}%
\pgfpathcurveto{\pgfqpoint{0.872157in}{1.294784in}}{\pgfqpoint{0.875430in}{1.302684in}}{\pgfqpoint{0.875430in}{1.310920in}}%
\pgfpathcurveto{\pgfqpoint{0.875430in}{1.319157in}}{\pgfqpoint{0.872157in}{1.327057in}}{\pgfqpoint{0.866334in}{1.332881in}}%
\pgfpathcurveto{\pgfqpoint{0.860510in}{1.338705in}}{\pgfqpoint{0.852610in}{1.341977in}}{\pgfqpoint{0.844373in}{1.341977in}}%
\pgfpathcurveto{\pgfqpoint{0.836137in}{1.341977in}}{\pgfqpoint{0.828237in}{1.338705in}}{\pgfqpoint{0.822413in}{1.332881in}}%
\pgfpathcurveto{\pgfqpoint{0.816589in}{1.327057in}}{\pgfqpoint{0.813317in}{1.319157in}}{\pgfqpoint{0.813317in}{1.310920in}}%
\pgfpathcurveto{\pgfqpoint{0.813317in}{1.302684in}}{\pgfqpoint{0.816589in}{1.294784in}}{\pgfqpoint{0.822413in}{1.288960in}}%
\pgfpathcurveto{\pgfqpoint{0.828237in}{1.283136in}}{\pgfqpoint{0.836137in}{1.279864in}}{\pgfqpoint{0.844373in}{1.279864in}}%
\pgfpathclose%
\pgfusepath{stroke,fill}%
\end{pgfscope}%
\begin{pgfscope}%
\pgfpathrectangle{\pgfqpoint{0.100000in}{0.220728in}}{\pgfqpoint{3.696000in}{3.696000in}}%
\pgfusepath{clip}%
\pgfsetbuttcap%
\pgfsetroundjoin%
\definecolor{currentfill}{rgb}{0.121569,0.466667,0.705882}%
\pgfsetfillcolor{currentfill}%
\pgfsetfillopacity{0.606552}%
\pgfsetlinewidth{1.003750pt}%
\definecolor{currentstroke}{rgb}{0.121569,0.466667,0.705882}%
\pgfsetstrokecolor{currentstroke}%
\pgfsetstrokeopacity{0.606552}%
\pgfsetdash{}{0pt}%
\pgfpathmoveto{\pgfqpoint{0.844200in}{1.279467in}}%
\pgfpathcurveto{\pgfqpoint{0.852437in}{1.279467in}}{\pgfqpoint{0.860337in}{1.282740in}}{\pgfqpoint{0.866161in}{1.288563in}}%
\pgfpathcurveto{\pgfqpoint{0.871984in}{1.294387in}}{\pgfqpoint{0.875257in}{1.302287in}}{\pgfqpoint{0.875257in}{1.310524in}}%
\pgfpathcurveto{\pgfqpoint{0.875257in}{1.318760in}}{\pgfqpoint{0.871984in}{1.326660in}}{\pgfqpoint{0.866161in}{1.332484in}}%
\pgfpathcurveto{\pgfqpoint{0.860337in}{1.338308in}}{\pgfqpoint{0.852437in}{1.341580in}}{\pgfqpoint{0.844200in}{1.341580in}}%
\pgfpathcurveto{\pgfqpoint{0.835964in}{1.341580in}}{\pgfqpoint{0.828064in}{1.338308in}}{\pgfqpoint{0.822240in}{1.332484in}}%
\pgfpathcurveto{\pgfqpoint{0.816416in}{1.326660in}}{\pgfqpoint{0.813144in}{1.318760in}}{\pgfqpoint{0.813144in}{1.310524in}}%
\pgfpathcurveto{\pgfqpoint{0.813144in}{1.302287in}}{\pgfqpoint{0.816416in}{1.294387in}}{\pgfqpoint{0.822240in}{1.288563in}}%
\pgfpathcurveto{\pgfqpoint{0.828064in}{1.282740in}}{\pgfqpoint{0.835964in}{1.279467in}}{\pgfqpoint{0.844200in}{1.279467in}}%
\pgfpathclose%
\pgfusepath{stroke,fill}%
\end{pgfscope}%
\begin{pgfscope}%
\pgfpathrectangle{\pgfqpoint{0.100000in}{0.220728in}}{\pgfqpoint{3.696000in}{3.696000in}}%
\pgfusepath{clip}%
\pgfsetbuttcap%
\pgfsetroundjoin%
\definecolor{currentfill}{rgb}{0.121569,0.466667,0.705882}%
\pgfsetfillcolor{currentfill}%
\pgfsetfillopacity{0.606658}%
\pgfsetlinewidth{1.003750pt}%
\definecolor{currentstroke}{rgb}{0.121569,0.466667,0.705882}%
\pgfsetstrokecolor{currentstroke}%
\pgfsetstrokeopacity{0.606658}%
\pgfsetdash{}{0pt}%
\pgfpathmoveto{\pgfqpoint{0.843912in}{1.278698in}}%
\pgfpathcurveto{\pgfqpoint{0.852148in}{1.278698in}}{\pgfqpoint{0.860048in}{1.281970in}}{\pgfqpoint{0.865872in}{1.287794in}}%
\pgfpathcurveto{\pgfqpoint{0.871696in}{1.293618in}}{\pgfqpoint{0.874969in}{1.301518in}}{\pgfqpoint{0.874969in}{1.309754in}}%
\pgfpathcurveto{\pgfqpoint{0.874969in}{1.317991in}}{\pgfqpoint{0.871696in}{1.325891in}}{\pgfqpoint{0.865872in}{1.331715in}}%
\pgfpathcurveto{\pgfqpoint{0.860048in}{1.337538in}}{\pgfqpoint{0.852148in}{1.340811in}}{\pgfqpoint{0.843912in}{1.340811in}}%
\pgfpathcurveto{\pgfqpoint{0.835676in}{1.340811in}}{\pgfqpoint{0.827776in}{1.337538in}}{\pgfqpoint{0.821952in}{1.331715in}}%
\pgfpathcurveto{\pgfqpoint{0.816128in}{1.325891in}}{\pgfqpoint{0.812856in}{1.317991in}}{\pgfqpoint{0.812856in}{1.309754in}}%
\pgfpathcurveto{\pgfqpoint{0.812856in}{1.301518in}}{\pgfqpoint{0.816128in}{1.293618in}}{\pgfqpoint{0.821952in}{1.287794in}}%
\pgfpathcurveto{\pgfqpoint{0.827776in}{1.281970in}}{\pgfqpoint{0.835676in}{1.278698in}}{\pgfqpoint{0.843912in}{1.278698in}}%
\pgfpathclose%
\pgfusepath{stroke,fill}%
\end{pgfscope}%
\begin{pgfscope}%
\pgfpathrectangle{\pgfqpoint{0.100000in}{0.220728in}}{\pgfqpoint{3.696000in}{3.696000in}}%
\pgfusepath{clip}%
\pgfsetbuttcap%
\pgfsetroundjoin%
\definecolor{currentfill}{rgb}{0.121569,0.466667,0.705882}%
\pgfsetfillcolor{currentfill}%
\pgfsetfillopacity{0.606666}%
\pgfsetlinewidth{1.003750pt}%
\definecolor{currentstroke}{rgb}{0.121569,0.466667,0.705882}%
\pgfsetstrokecolor{currentstroke}%
\pgfsetstrokeopacity{0.606666}%
\pgfsetdash{}{0pt}%
\pgfpathmoveto{\pgfqpoint{0.843893in}{1.278646in}}%
\pgfpathcurveto{\pgfqpoint{0.852129in}{1.278646in}}{\pgfqpoint{0.860029in}{1.281918in}}{\pgfqpoint{0.865853in}{1.287742in}}%
\pgfpathcurveto{\pgfqpoint{0.871677in}{1.293566in}}{\pgfqpoint{0.874949in}{1.301466in}}{\pgfqpoint{0.874949in}{1.309702in}}%
\pgfpathcurveto{\pgfqpoint{0.874949in}{1.317938in}}{\pgfqpoint{0.871677in}{1.325838in}}{\pgfqpoint{0.865853in}{1.331662in}}%
\pgfpathcurveto{\pgfqpoint{0.860029in}{1.337486in}}{\pgfqpoint{0.852129in}{1.340759in}}{\pgfqpoint{0.843893in}{1.340759in}}%
\pgfpathcurveto{\pgfqpoint{0.835656in}{1.340759in}}{\pgfqpoint{0.827756in}{1.337486in}}{\pgfqpoint{0.821932in}{1.331662in}}%
\pgfpathcurveto{\pgfqpoint{0.816109in}{1.325838in}}{\pgfqpoint{0.812836in}{1.317938in}}{\pgfqpoint{0.812836in}{1.309702in}}%
\pgfpathcurveto{\pgfqpoint{0.812836in}{1.301466in}}{\pgfqpoint{0.816109in}{1.293566in}}{\pgfqpoint{0.821932in}{1.287742in}}%
\pgfpathcurveto{\pgfqpoint{0.827756in}{1.281918in}}{\pgfqpoint{0.835656in}{1.278646in}}{\pgfqpoint{0.843893in}{1.278646in}}%
\pgfpathclose%
\pgfusepath{stroke,fill}%
\end{pgfscope}%
\begin{pgfscope}%
\pgfpathrectangle{\pgfqpoint{0.100000in}{0.220728in}}{\pgfqpoint{3.696000in}{3.696000in}}%
\pgfusepath{clip}%
\pgfsetbuttcap%
\pgfsetroundjoin%
\definecolor{currentfill}{rgb}{0.121569,0.466667,0.705882}%
\pgfsetfillcolor{currentfill}%
\pgfsetfillopacity{0.606680}%
\pgfsetlinewidth{1.003750pt}%
\definecolor{currentstroke}{rgb}{0.121569,0.466667,0.705882}%
\pgfsetstrokecolor{currentstroke}%
\pgfsetstrokeopacity{0.606680}%
\pgfsetdash{}{0pt}%
\pgfpathmoveto{\pgfqpoint{0.843857in}{1.278550in}}%
\pgfpathcurveto{\pgfqpoint{0.852093in}{1.278550in}}{\pgfqpoint{0.859993in}{1.281823in}}{\pgfqpoint{0.865817in}{1.287647in}}%
\pgfpathcurveto{\pgfqpoint{0.871641in}{1.293470in}}{\pgfqpoint{0.874914in}{1.301370in}}{\pgfqpoint{0.874914in}{1.309607in}}%
\pgfpathcurveto{\pgfqpoint{0.874914in}{1.317843in}}{\pgfqpoint{0.871641in}{1.325743in}}{\pgfqpoint{0.865817in}{1.331567in}}%
\pgfpathcurveto{\pgfqpoint{0.859993in}{1.337391in}}{\pgfqpoint{0.852093in}{1.340663in}}{\pgfqpoint{0.843857in}{1.340663in}}%
\pgfpathcurveto{\pgfqpoint{0.835621in}{1.340663in}}{\pgfqpoint{0.827721in}{1.337391in}}{\pgfqpoint{0.821897in}{1.331567in}}%
\pgfpathcurveto{\pgfqpoint{0.816073in}{1.325743in}}{\pgfqpoint{0.812801in}{1.317843in}}{\pgfqpoint{0.812801in}{1.309607in}}%
\pgfpathcurveto{\pgfqpoint{0.812801in}{1.301370in}}{\pgfqpoint{0.816073in}{1.293470in}}{\pgfqpoint{0.821897in}{1.287647in}}%
\pgfpathcurveto{\pgfqpoint{0.827721in}{1.281823in}}{\pgfqpoint{0.835621in}{1.278550in}}{\pgfqpoint{0.843857in}{1.278550in}}%
\pgfpathclose%
\pgfusepath{stroke,fill}%
\end{pgfscope}%
\begin{pgfscope}%
\pgfpathrectangle{\pgfqpoint{0.100000in}{0.220728in}}{\pgfqpoint{3.696000in}{3.696000in}}%
\pgfusepath{clip}%
\pgfsetbuttcap%
\pgfsetroundjoin%
\definecolor{currentfill}{rgb}{0.121569,0.466667,0.705882}%
\pgfsetfillcolor{currentfill}%
\pgfsetfillopacity{0.606704}%
\pgfsetlinewidth{1.003750pt}%
\definecolor{currentstroke}{rgb}{0.121569,0.466667,0.705882}%
\pgfsetstrokecolor{currentstroke}%
\pgfsetstrokeopacity{0.606704}%
\pgfsetdash{}{0pt}%
\pgfpathmoveto{\pgfqpoint{0.843787in}{1.278377in}}%
\pgfpathcurveto{\pgfqpoint{0.852023in}{1.278377in}}{\pgfqpoint{0.859923in}{1.281650in}}{\pgfqpoint{0.865747in}{1.287474in}}%
\pgfpathcurveto{\pgfqpoint{0.871571in}{1.293298in}}{\pgfqpoint{0.874843in}{1.301198in}}{\pgfqpoint{0.874843in}{1.309434in}}%
\pgfpathcurveto{\pgfqpoint{0.874843in}{1.317670in}}{\pgfqpoint{0.871571in}{1.325570in}}{\pgfqpoint{0.865747in}{1.331394in}}%
\pgfpathcurveto{\pgfqpoint{0.859923in}{1.337218in}}{\pgfqpoint{0.852023in}{1.340490in}}{\pgfqpoint{0.843787in}{1.340490in}}%
\pgfpathcurveto{\pgfqpoint{0.835551in}{1.340490in}}{\pgfqpoint{0.827650in}{1.337218in}}{\pgfqpoint{0.821827in}{1.331394in}}%
\pgfpathcurveto{\pgfqpoint{0.816003in}{1.325570in}}{\pgfqpoint{0.812730in}{1.317670in}}{\pgfqpoint{0.812730in}{1.309434in}}%
\pgfpathcurveto{\pgfqpoint{0.812730in}{1.301198in}}{\pgfqpoint{0.816003in}{1.293298in}}{\pgfqpoint{0.821827in}{1.287474in}}%
\pgfpathcurveto{\pgfqpoint{0.827650in}{1.281650in}}{\pgfqpoint{0.835551in}{1.278377in}}{\pgfqpoint{0.843787in}{1.278377in}}%
\pgfpathclose%
\pgfusepath{stroke,fill}%
\end{pgfscope}%
\begin{pgfscope}%
\pgfpathrectangle{\pgfqpoint{0.100000in}{0.220728in}}{\pgfqpoint{3.696000in}{3.696000in}}%
\pgfusepath{clip}%
\pgfsetbuttcap%
\pgfsetroundjoin%
\definecolor{currentfill}{rgb}{0.121569,0.466667,0.705882}%
\pgfsetfillcolor{currentfill}%
\pgfsetfillopacity{0.606752}%
\pgfsetlinewidth{1.003750pt}%
\definecolor{currentstroke}{rgb}{0.121569,0.466667,0.705882}%
\pgfsetstrokecolor{currentstroke}%
\pgfsetstrokeopacity{0.606752}%
\pgfsetdash{}{0pt}%
\pgfpathmoveto{\pgfqpoint{0.843674in}{1.278060in}}%
\pgfpathcurveto{\pgfqpoint{0.851910in}{1.278060in}}{\pgfqpoint{0.859810in}{1.281333in}}{\pgfqpoint{0.865634in}{1.287156in}}%
\pgfpathcurveto{\pgfqpoint{0.871458in}{1.292980in}}{\pgfqpoint{0.874731in}{1.300880in}}{\pgfqpoint{0.874731in}{1.309117in}}%
\pgfpathcurveto{\pgfqpoint{0.874731in}{1.317353in}}{\pgfqpoint{0.871458in}{1.325253in}}{\pgfqpoint{0.865634in}{1.331077in}}%
\pgfpathcurveto{\pgfqpoint{0.859810in}{1.336901in}}{\pgfqpoint{0.851910in}{1.340173in}}{\pgfqpoint{0.843674in}{1.340173in}}%
\pgfpathcurveto{\pgfqpoint{0.835438in}{1.340173in}}{\pgfqpoint{0.827538in}{1.336901in}}{\pgfqpoint{0.821714in}{1.331077in}}%
\pgfpathcurveto{\pgfqpoint{0.815890in}{1.325253in}}{\pgfqpoint{0.812618in}{1.317353in}}{\pgfqpoint{0.812618in}{1.309117in}}%
\pgfpathcurveto{\pgfqpoint{0.812618in}{1.300880in}}{\pgfqpoint{0.815890in}{1.292980in}}{\pgfqpoint{0.821714in}{1.287156in}}%
\pgfpathcurveto{\pgfqpoint{0.827538in}{1.281333in}}{\pgfqpoint{0.835438in}{1.278060in}}{\pgfqpoint{0.843674in}{1.278060in}}%
\pgfpathclose%
\pgfusepath{stroke,fill}%
\end{pgfscope}%
\begin{pgfscope}%
\pgfpathrectangle{\pgfqpoint{0.100000in}{0.220728in}}{\pgfqpoint{3.696000in}{3.696000in}}%
\pgfusepath{clip}%
\pgfsetbuttcap%
\pgfsetroundjoin%
\definecolor{currentfill}{rgb}{0.121569,0.466667,0.705882}%
\pgfsetfillcolor{currentfill}%
\pgfsetfillopacity{0.606840}%
\pgfsetlinewidth{1.003750pt}%
\definecolor{currentstroke}{rgb}{0.121569,0.466667,0.705882}%
\pgfsetstrokecolor{currentstroke}%
\pgfsetstrokeopacity{0.606840}%
\pgfsetdash{}{0pt}%
\pgfpathmoveto{\pgfqpoint{0.843464in}{1.277498in}}%
\pgfpathcurveto{\pgfqpoint{0.851700in}{1.277498in}}{\pgfqpoint{0.859600in}{1.280771in}}{\pgfqpoint{0.865424in}{1.286594in}}%
\pgfpathcurveto{\pgfqpoint{0.871248in}{1.292418in}}{\pgfqpoint{0.874520in}{1.300318in}}{\pgfqpoint{0.874520in}{1.308555in}}%
\pgfpathcurveto{\pgfqpoint{0.874520in}{1.316791in}}{\pgfqpoint{0.871248in}{1.324691in}}{\pgfqpoint{0.865424in}{1.330515in}}%
\pgfpathcurveto{\pgfqpoint{0.859600in}{1.336339in}}{\pgfqpoint{0.851700in}{1.339611in}}{\pgfqpoint{0.843464in}{1.339611in}}%
\pgfpathcurveto{\pgfqpoint{0.835227in}{1.339611in}}{\pgfqpoint{0.827327in}{1.336339in}}{\pgfqpoint{0.821503in}{1.330515in}}%
\pgfpathcurveto{\pgfqpoint{0.815679in}{1.324691in}}{\pgfqpoint{0.812407in}{1.316791in}}{\pgfqpoint{0.812407in}{1.308555in}}%
\pgfpathcurveto{\pgfqpoint{0.812407in}{1.300318in}}{\pgfqpoint{0.815679in}{1.292418in}}{\pgfqpoint{0.821503in}{1.286594in}}%
\pgfpathcurveto{\pgfqpoint{0.827327in}{1.280771in}}{\pgfqpoint{0.835227in}{1.277498in}}{\pgfqpoint{0.843464in}{1.277498in}}%
\pgfpathclose%
\pgfusepath{stroke,fill}%
\end{pgfscope}%
\begin{pgfscope}%
\pgfpathrectangle{\pgfqpoint{0.100000in}{0.220728in}}{\pgfqpoint{3.696000in}{3.696000in}}%
\pgfusepath{clip}%
\pgfsetbuttcap%
\pgfsetroundjoin%
\definecolor{currentfill}{rgb}{0.121569,0.466667,0.705882}%
\pgfsetfillcolor{currentfill}%
\pgfsetfillopacity{0.606995}%
\pgfsetlinewidth{1.003750pt}%
\definecolor{currentstroke}{rgb}{0.121569,0.466667,0.705882}%
\pgfsetstrokecolor{currentstroke}%
\pgfsetstrokeopacity{0.606995}%
\pgfsetdash{}{0pt}%
\pgfpathmoveto{\pgfqpoint{0.843054in}{1.276475in}}%
\pgfpathcurveto{\pgfqpoint{0.851290in}{1.276475in}}{\pgfqpoint{0.859190in}{1.279748in}}{\pgfqpoint{0.865014in}{1.285572in}}%
\pgfpathcurveto{\pgfqpoint{0.870838in}{1.291395in}}{\pgfqpoint{0.874110in}{1.299296in}}{\pgfqpoint{0.874110in}{1.307532in}}%
\pgfpathcurveto{\pgfqpoint{0.874110in}{1.315768in}}{\pgfqpoint{0.870838in}{1.323668in}}{\pgfqpoint{0.865014in}{1.329492in}}%
\pgfpathcurveto{\pgfqpoint{0.859190in}{1.335316in}}{\pgfqpoint{0.851290in}{1.338588in}}{\pgfqpoint{0.843054in}{1.338588in}}%
\pgfpathcurveto{\pgfqpoint{0.834817in}{1.338588in}}{\pgfqpoint{0.826917in}{1.335316in}}{\pgfqpoint{0.821093in}{1.329492in}}%
\pgfpathcurveto{\pgfqpoint{0.815270in}{1.323668in}}{\pgfqpoint{0.811997in}{1.315768in}}{\pgfqpoint{0.811997in}{1.307532in}}%
\pgfpathcurveto{\pgfqpoint{0.811997in}{1.299296in}}{\pgfqpoint{0.815270in}{1.291395in}}{\pgfqpoint{0.821093in}{1.285572in}}%
\pgfpathcurveto{\pgfqpoint{0.826917in}{1.279748in}}{\pgfqpoint{0.834817in}{1.276475in}}{\pgfqpoint{0.843054in}{1.276475in}}%
\pgfpathclose%
\pgfusepath{stroke,fill}%
\end{pgfscope}%
\begin{pgfscope}%
\pgfpathrectangle{\pgfqpoint{0.100000in}{0.220728in}}{\pgfqpoint{3.696000in}{3.696000in}}%
\pgfusepath{clip}%
\pgfsetbuttcap%
\pgfsetroundjoin%
\definecolor{currentfill}{rgb}{0.121569,0.466667,0.705882}%
\pgfsetfillcolor{currentfill}%
\pgfsetfillopacity{0.607296}%
\pgfsetlinewidth{1.003750pt}%
\definecolor{currentstroke}{rgb}{0.121569,0.466667,0.705882}%
\pgfsetstrokecolor{currentstroke}%
\pgfsetstrokeopacity{0.607296}%
\pgfsetdash{}{0pt}%
\pgfpathmoveto{\pgfqpoint{0.842226in}{1.274771in}}%
\pgfpathcurveto{\pgfqpoint{0.850463in}{1.274771in}}{\pgfqpoint{0.858363in}{1.278043in}}{\pgfqpoint{0.864187in}{1.283867in}}%
\pgfpathcurveto{\pgfqpoint{0.870010in}{1.289691in}}{\pgfqpoint{0.873283in}{1.297591in}}{\pgfqpoint{0.873283in}{1.305827in}}%
\pgfpathcurveto{\pgfqpoint{0.873283in}{1.314064in}}{\pgfqpoint{0.870010in}{1.321964in}}{\pgfqpoint{0.864187in}{1.327788in}}%
\pgfpathcurveto{\pgfqpoint{0.858363in}{1.333612in}}{\pgfqpoint{0.850463in}{1.336884in}}{\pgfqpoint{0.842226in}{1.336884in}}%
\pgfpathcurveto{\pgfqpoint{0.833990in}{1.336884in}}{\pgfqpoint{0.826090in}{1.333612in}}{\pgfqpoint{0.820266in}{1.327788in}}%
\pgfpathcurveto{\pgfqpoint{0.814442in}{1.321964in}}{\pgfqpoint{0.811170in}{1.314064in}}{\pgfqpoint{0.811170in}{1.305827in}}%
\pgfpathcurveto{\pgfqpoint{0.811170in}{1.297591in}}{\pgfqpoint{0.814442in}{1.289691in}}{\pgfqpoint{0.820266in}{1.283867in}}%
\pgfpathcurveto{\pgfqpoint{0.826090in}{1.278043in}}{\pgfqpoint{0.833990in}{1.274771in}}{\pgfqpoint{0.842226in}{1.274771in}}%
\pgfpathclose%
\pgfusepath{stroke,fill}%
\end{pgfscope}%
\begin{pgfscope}%
\pgfpathrectangle{\pgfqpoint{0.100000in}{0.220728in}}{\pgfqpoint{3.696000in}{3.696000in}}%
\pgfusepath{clip}%
\pgfsetbuttcap%
\pgfsetroundjoin%
\definecolor{currentfill}{rgb}{0.121569,0.466667,0.705882}%
\pgfsetfillcolor{currentfill}%
\pgfsetfillopacity{0.607998}%
\pgfsetlinewidth{1.003750pt}%
\definecolor{currentstroke}{rgb}{0.121569,0.466667,0.705882}%
\pgfsetstrokecolor{currentstroke}%
\pgfsetstrokeopacity{0.607998}%
\pgfsetdash{}{0pt}%
\pgfpathmoveto{\pgfqpoint{0.840618in}{1.272392in}}%
\pgfpathcurveto{\pgfqpoint{0.848854in}{1.272392in}}{\pgfqpoint{0.856754in}{1.275665in}}{\pgfqpoint{0.862578in}{1.281489in}}%
\pgfpathcurveto{\pgfqpoint{0.868402in}{1.287313in}}{\pgfqpoint{0.871674in}{1.295213in}}{\pgfqpoint{0.871674in}{1.303449in}}%
\pgfpathcurveto{\pgfqpoint{0.871674in}{1.311685in}}{\pgfqpoint{0.868402in}{1.319585in}}{\pgfqpoint{0.862578in}{1.325409in}}%
\pgfpathcurveto{\pgfqpoint{0.856754in}{1.331233in}}{\pgfqpoint{0.848854in}{1.334505in}}{\pgfqpoint{0.840618in}{1.334505in}}%
\pgfpathcurveto{\pgfqpoint{0.832381in}{1.334505in}}{\pgfqpoint{0.824481in}{1.331233in}}{\pgfqpoint{0.818657in}{1.325409in}}%
\pgfpathcurveto{\pgfqpoint{0.812833in}{1.319585in}}{\pgfqpoint{0.809561in}{1.311685in}}{\pgfqpoint{0.809561in}{1.303449in}}%
\pgfpathcurveto{\pgfqpoint{0.809561in}{1.295213in}}{\pgfqpoint{0.812833in}{1.287313in}}{\pgfqpoint{0.818657in}{1.281489in}}%
\pgfpathcurveto{\pgfqpoint{0.824481in}{1.275665in}}{\pgfqpoint{0.832381in}{1.272392in}}{\pgfqpoint{0.840618in}{1.272392in}}%
\pgfpathclose%
\pgfusepath{stroke,fill}%
\end{pgfscope}%
\begin{pgfscope}%
\pgfpathrectangle{\pgfqpoint{0.100000in}{0.220728in}}{\pgfqpoint{3.696000in}{3.696000in}}%
\pgfusepath{clip}%
\pgfsetbuttcap%
\pgfsetroundjoin%
\definecolor{currentfill}{rgb}{0.121569,0.466667,0.705882}%
\pgfsetfillcolor{currentfill}%
\pgfsetfillopacity{0.609138}%
\pgfsetlinewidth{1.003750pt}%
\definecolor{currentstroke}{rgb}{0.121569,0.466667,0.705882}%
\pgfsetstrokecolor{currentstroke}%
\pgfsetstrokeopacity{0.609138}%
\pgfsetdash{}{0pt}%
\pgfpathmoveto{\pgfqpoint{0.838435in}{1.266830in}}%
\pgfpathcurveto{\pgfqpoint{0.846671in}{1.266830in}}{\pgfqpoint{0.854571in}{1.270102in}}{\pgfqpoint{0.860395in}{1.275926in}}%
\pgfpathcurveto{\pgfqpoint{0.866219in}{1.281750in}}{\pgfqpoint{0.869491in}{1.289650in}}{\pgfqpoint{0.869491in}{1.297887in}}%
\pgfpathcurveto{\pgfqpoint{0.869491in}{1.306123in}}{\pgfqpoint{0.866219in}{1.314023in}}{\pgfqpoint{0.860395in}{1.319847in}}%
\pgfpathcurveto{\pgfqpoint{0.854571in}{1.325671in}}{\pgfqpoint{0.846671in}{1.328943in}}{\pgfqpoint{0.838435in}{1.328943in}}%
\pgfpathcurveto{\pgfqpoint{0.830198in}{1.328943in}}{\pgfqpoint{0.822298in}{1.325671in}}{\pgfqpoint{0.816474in}{1.319847in}}%
\pgfpathcurveto{\pgfqpoint{0.810651in}{1.314023in}}{\pgfqpoint{0.807378in}{1.306123in}}{\pgfqpoint{0.807378in}{1.297887in}}%
\pgfpathcurveto{\pgfqpoint{0.807378in}{1.289650in}}{\pgfqpoint{0.810651in}{1.281750in}}{\pgfqpoint{0.816474in}{1.275926in}}%
\pgfpathcurveto{\pgfqpoint{0.822298in}{1.270102in}}{\pgfqpoint{0.830198in}{1.266830in}}{\pgfqpoint{0.838435in}{1.266830in}}%
\pgfpathclose%
\pgfusepath{stroke,fill}%
\end{pgfscope}%
\begin{pgfscope}%
\pgfpathrectangle{\pgfqpoint{0.100000in}{0.220728in}}{\pgfqpoint{3.696000in}{3.696000in}}%
\pgfusepath{clip}%
\pgfsetbuttcap%
\pgfsetroundjoin%
\definecolor{currentfill}{rgb}{0.121569,0.466667,0.705882}%
\pgfsetfillcolor{currentfill}%
\pgfsetfillopacity{0.610256}%
\pgfsetlinewidth{1.003750pt}%
\definecolor{currentstroke}{rgb}{0.121569,0.466667,0.705882}%
\pgfsetstrokecolor{currentstroke}%
\pgfsetstrokeopacity{0.610256}%
\pgfsetdash{}{0pt}%
\pgfpathmoveto{\pgfqpoint{3.092393in}{2.986958in}}%
\pgfpathcurveto{\pgfqpoint{3.100629in}{2.986958in}}{\pgfqpoint{3.108529in}{2.990231in}}{\pgfqpoint{3.114353in}{2.996055in}}%
\pgfpathcurveto{\pgfqpoint{3.120177in}{3.001879in}}{\pgfqpoint{3.123449in}{3.009779in}}{\pgfqpoint{3.123449in}{3.018015in}}%
\pgfpathcurveto{\pgfqpoint{3.123449in}{3.026251in}}{\pgfqpoint{3.120177in}{3.034151in}}{\pgfqpoint{3.114353in}{3.039975in}}%
\pgfpathcurveto{\pgfqpoint{3.108529in}{3.045799in}}{\pgfqpoint{3.100629in}{3.049071in}}{\pgfqpoint{3.092393in}{3.049071in}}%
\pgfpathcurveto{\pgfqpoint{3.084157in}{3.049071in}}{\pgfqpoint{3.076257in}{3.045799in}}{\pgfqpoint{3.070433in}{3.039975in}}%
\pgfpathcurveto{\pgfqpoint{3.064609in}{3.034151in}}{\pgfqpoint{3.061336in}{3.026251in}}{\pgfqpoint{3.061336in}{3.018015in}}%
\pgfpathcurveto{\pgfqpoint{3.061336in}{3.009779in}}{\pgfqpoint{3.064609in}{3.001879in}}{\pgfqpoint{3.070433in}{2.996055in}}%
\pgfpathcurveto{\pgfqpoint{3.076257in}{2.990231in}}{\pgfqpoint{3.084157in}{2.986958in}}{\pgfqpoint{3.092393in}{2.986958in}}%
\pgfpathclose%
\pgfusepath{stroke,fill}%
\end{pgfscope}%
\begin{pgfscope}%
\pgfpathrectangle{\pgfqpoint{0.100000in}{0.220728in}}{\pgfqpoint{3.696000in}{3.696000in}}%
\pgfusepath{clip}%
\pgfsetbuttcap%
\pgfsetroundjoin%
\definecolor{currentfill}{rgb}{0.121569,0.466667,0.705882}%
\pgfsetfillcolor{currentfill}%
\pgfsetfillopacity{0.610458}%
\pgfsetlinewidth{1.003750pt}%
\definecolor{currentstroke}{rgb}{0.121569,0.466667,0.705882}%
\pgfsetstrokecolor{currentstroke}%
\pgfsetstrokeopacity{0.610458}%
\pgfsetdash{}{0pt}%
\pgfpathmoveto{\pgfqpoint{0.832914in}{1.255331in}}%
\pgfpathcurveto{\pgfqpoint{0.841150in}{1.255331in}}{\pgfqpoint{0.849050in}{1.258604in}}{\pgfqpoint{0.854874in}{1.264428in}}%
\pgfpathcurveto{\pgfqpoint{0.860698in}{1.270251in}}{\pgfqpoint{0.863970in}{1.278151in}}{\pgfqpoint{0.863970in}{1.286388in}}%
\pgfpathcurveto{\pgfqpoint{0.863970in}{1.294624in}}{\pgfqpoint{0.860698in}{1.302524in}}{\pgfqpoint{0.854874in}{1.308348in}}%
\pgfpathcurveto{\pgfqpoint{0.849050in}{1.314172in}}{\pgfqpoint{0.841150in}{1.317444in}}{\pgfqpoint{0.832914in}{1.317444in}}%
\pgfpathcurveto{\pgfqpoint{0.824677in}{1.317444in}}{\pgfqpoint{0.816777in}{1.314172in}}{\pgfqpoint{0.810953in}{1.308348in}}%
\pgfpathcurveto{\pgfqpoint{0.805130in}{1.302524in}}{\pgfqpoint{0.801857in}{1.294624in}}{\pgfqpoint{0.801857in}{1.286388in}}%
\pgfpathcurveto{\pgfqpoint{0.801857in}{1.278151in}}{\pgfqpoint{0.805130in}{1.270251in}}{\pgfqpoint{0.810953in}{1.264428in}}%
\pgfpathcurveto{\pgfqpoint{0.816777in}{1.258604in}}{\pgfqpoint{0.824677in}{1.255331in}}{\pgfqpoint{0.832914in}{1.255331in}}%
\pgfpathclose%
\pgfusepath{stroke,fill}%
\end{pgfscope}%
\begin{pgfscope}%
\pgfpathrectangle{\pgfqpoint{0.100000in}{0.220728in}}{\pgfqpoint{3.696000in}{3.696000in}}%
\pgfusepath{clip}%
\pgfsetbuttcap%
\pgfsetroundjoin%
\definecolor{currentfill}{rgb}{0.121569,0.466667,0.705882}%
\pgfsetfillcolor{currentfill}%
\pgfsetfillopacity{0.611621}%
\pgfsetlinewidth{1.003750pt}%
\definecolor{currentstroke}{rgb}{0.121569,0.466667,0.705882}%
\pgfsetstrokecolor{currentstroke}%
\pgfsetstrokeopacity{0.611621}%
\pgfsetdash{}{0pt}%
\pgfpathmoveto{\pgfqpoint{3.105696in}{2.982641in}}%
\pgfpathcurveto{\pgfqpoint{3.113932in}{2.982641in}}{\pgfqpoint{3.121832in}{2.985914in}}{\pgfqpoint{3.127656in}{2.991738in}}%
\pgfpathcurveto{\pgfqpoint{3.133480in}{2.997562in}}{\pgfqpoint{3.136752in}{3.005462in}}{\pgfqpoint{3.136752in}{3.013698in}}%
\pgfpathcurveto{\pgfqpoint{3.136752in}{3.021934in}}{\pgfqpoint{3.133480in}{3.029834in}}{\pgfqpoint{3.127656in}{3.035658in}}%
\pgfpathcurveto{\pgfqpoint{3.121832in}{3.041482in}}{\pgfqpoint{3.113932in}{3.044754in}}{\pgfqpoint{3.105696in}{3.044754in}}%
\pgfpathcurveto{\pgfqpoint{3.097460in}{3.044754in}}{\pgfqpoint{3.089560in}{3.041482in}}{\pgfqpoint{3.083736in}{3.035658in}}%
\pgfpathcurveto{\pgfqpoint{3.077912in}{3.029834in}}{\pgfqpoint{3.074639in}{3.021934in}}{\pgfqpoint{3.074639in}{3.013698in}}%
\pgfpathcurveto{\pgfqpoint{3.074639in}{3.005462in}}{\pgfqpoint{3.077912in}{2.997562in}}{\pgfqpoint{3.083736in}{2.991738in}}%
\pgfpathcurveto{\pgfqpoint{3.089560in}{2.985914in}}{\pgfqpoint{3.097460in}{2.982641in}}{\pgfqpoint{3.105696in}{2.982641in}}%
\pgfpathclose%
\pgfusepath{stroke,fill}%
\end{pgfscope}%
\begin{pgfscope}%
\pgfpathrectangle{\pgfqpoint{0.100000in}{0.220728in}}{\pgfqpoint{3.696000in}{3.696000in}}%
\pgfusepath{clip}%
\pgfsetbuttcap%
\pgfsetroundjoin%
\definecolor{currentfill}{rgb}{0.121569,0.466667,0.705882}%
\pgfsetfillcolor{currentfill}%
\pgfsetfillopacity{0.611647}%
\pgfsetlinewidth{1.003750pt}%
\definecolor{currentstroke}{rgb}{0.121569,0.466667,0.705882}%
\pgfsetstrokecolor{currentstroke}%
\pgfsetstrokeopacity{0.611647}%
\pgfsetdash{}{0pt}%
\pgfpathmoveto{\pgfqpoint{0.845071in}{1.230330in}}%
\pgfpathcurveto{\pgfqpoint{0.853307in}{1.230330in}}{\pgfqpoint{0.861207in}{1.233602in}}{\pgfqpoint{0.867031in}{1.239426in}}%
\pgfpathcurveto{\pgfqpoint{0.872855in}{1.245250in}}{\pgfqpoint{0.876128in}{1.253150in}}{\pgfqpoint{0.876128in}{1.261386in}}%
\pgfpathcurveto{\pgfqpoint{0.876128in}{1.269622in}}{\pgfqpoint{0.872855in}{1.277522in}}{\pgfqpoint{0.867031in}{1.283346in}}%
\pgfpathcurveto{\pgfqpoint{0.861207in}{1.289170in}}{\pgfqpoint{0.853307in}{1.292443in}}{\pgfqpoint{0.845071in}{1.292443in}}%
\pgfpathcurveto{\pgfqpoint{0.836835in}{1.292443in}}{\pgfqpoint{0.828935in}{1.289170in}}{\pgfqpoint{0.823111in}{1.283346in}}%
\pgfpathcurveto{\pgfqpoint{0.817287in}{1.277522in}}{\pgfqpoint{0.814015in}{1.269622in}}{\pgfqpoint{0.814015in}{1.261386in}}%
\pgfpathcurveto{\pgfqpoint{0.814015in}{1.253150in}}{\pgfqpoint{0.817287in}{1.245250in}}{\pgfqpoint{0.823111in}{1.239426in}}%
\pgfpathcurveto{\pgfqpoint{0.828935in}{1.233602in}}{\pgfqpoint{0.836835in}{1.230330in}}{\pgfqpoint{0.845071in}{1.230330in}}%
\pgfpathclose%
\pgfusepath{stroke,fill}%
\end{pgfscope}%
\begin{pgfscope}%
\pgfpathrectangle{\pgfqpoint{0.100000in}{0.220728in}}{\pgfqpoint{3.696000in}{3.696000in}}%
\pgfusepath{clip}%
\pgfsetbuttcap%
\pgfsetroundjoin%
\definecolor{currentfill}{rgb}{0.121569,0.466667,0.705882}%
\pgfsetfillcolor{currentfill}%
\pgfsetfillopacity{0.611953}%
\pgfsetlinewidth{1.003750pt}%
\definecolor{currentstroke}{rgb}{0.121569,0.466667,0.705882}%
\pgfsetstrokecolor{currentstroke}%
\pgfsetstrokeopacity{0.611953}%
\pgfsetdash{}{0pt}%
\pgfpathmoveto{\pgfqpoint{0.829351in}{1.245086in}}%
\pgfpathcurveto{\pgfqpoint{0.837588in}{1.245086in}}{\pgfqpoint{0.845488in}{1.248358in}}{\pgfqpoint{0.851312in}{1.254182in}}%
\pgfpathcurveto{\pgfqpoint{0.857135in}{1.260006in}}{\pgfqpoint{0.860408in}{1.267906in}}{\pgfqpoint{0.860408in}{1.276142in}}%
\pgfpathcurveto{\pgfqpoint{0.860408in}{1.284378in}}{\pgfqpoint{0.857135in}{1.292278in}}{\pgfqpoint{0.851312in}{1.298102in}}%
\pgfpathcurveto{\pgfqpoint{0.845488in}{1.303926in}}{\pgfqpoint{0.837588in}{1.307199in}}{\pgfqpoint{0.829351in}{1.307199in}}%
\pgfpathcurveto{\pgfqpoint{0.821115in}{1.307199in}}{\pgfqpoint{0.813215in}{1.303926in}}{\pgfqpoint{0.807391in}{1.298102in}}%
\pgfpathcurveto{\pgfqpoint{0.801567in}{1.292278in}}{\pgfqpoint{0.798295in}{1.284378in}}{\pgfqpoint{0.798295in}{1.276142in}}%
\pgfpathcurveto{\pgfqpoint{0.798295in}{1.267906in}}{\pgfqpoint{0.801567in}{1.260006in}}{\pgfqpoint{0.807391in}{1.254182in}}%
\pgfpathcurveto{\pgfqpoint{0.813215in}{1.248358in}}{\pgfqpoint{0.821115in}{1.245086in}}{\pgfqpoint{0.829351in}{1.245086in}}%
\pgfpathclose%
\pgfusepath{stroke,fill}%
\end{pgfscope}%
\begin{pgfscope}%
\pgfpathrectangle{\pgfqpoint{0.100000in}{0.220728in}}{\pgfqpoint{3.696000in}{3.696000in}}%
\pgfusepath{clip}%
\pgfsetbuttcap%
\pgfsetroundjoin%
\definecolor{currentfill}{rgb}{0.121569,0.466667,0.705882}%
\pgfsetfillcolor{currentfill}%
\pgfsetfillopacity{0.612004}%
\pgfsetlinewidth{1.003750pt}%
\definecolor{currentstroke}{rgb}{0.121569,0.466667,0.705882}%
\pgfsetstrokecolor{currentstroke}%
\pgfsetstrokeopacity{0.612004}%
\pgfsetdash{}{0pt}%
\pgfpathmoveto{\pgfqpoint{0.843997in}{1.229043in}}%
\pgfpathcurveto{\pgfqpoint{0.852233in}{1.229043in}}{\pgfqpoint{0.860133in}{1.232315in}}{\pgfqpoint{0.865957in}{1.238139in}}%
\pgfpathcurveto{\pgfqpoint{0.871781in}{1.243963in}}{\pgfqpoint{0.875054in}{1.251863in}}{\pgfqpoint{0.875054in}{1.260100in}}%
\pgfpathcurveto{\pgfqpoint{0.875054in}{1.268336in}}{\pgfqpoint{0.871781in}{1.276236in}}{\pgfqpoint{0.865957in}{1.282060in}}%
\pgfpathcurveto{\pgfqpoint{0.860133in}{1.287884in}}{\pgfqpoint{0.852233in}{1.291156in}}{\pgfqpoint{0.843997in}{1.291156in}}%
\pgfpathcurveto{\pgfqpoint{0.835761in}{1.291156in}}{\pgfqpoint{0.827861in}{1.287884in}}{\pgfqpoint{0.822037in}{1.282060in}}%
\pgfpathcurveto{\pgfqpoint{0.816213in}{1.276236in}}{\pgfqpoint{0.812941in}{1.268336in}}{\pgfqpoint{0.812941in}{1.260100in}}%
\pgfpathcurveto{\pgfqpoint{0.812941in}{1.251863in}}{\pgfqpoint{0.816213in}{1.243963in}}{\pgfqpoint{0.822037in}{1.238139in}}%
\pgfpathcurveto{\pgfqpoint{0.827861in}{1.232315in}}{\pgfqpoint{0.835761in}{1.229043in}}{\pgfqpoint{0.843997in}{1.229043in}}%
\pgfpathclose%
\pgfusepath{stroke,fill}%
\end{pgfscope}%
\begin{pgfscope}%
\pgfpathrectangle{\pgfqpoint{0.100000in}{0.220728in}}{\pgfqpoint{3.696000in}{3.696000in}}%
\pgfusepath{clip}%
\pgfsetbuttcap%
\pgfsetroundjoin%
\definecolor{currentfill}{rgb}{0.121569,0.466667,0.705882}%
\pgfsetfillcolor{currentfill}%
\pgfsetfillopacity{0.612753}%
\pgfsetlinewidth{1.003750pt}%
\definecolor{currentstroke}{rgb}{0.121569,0.466667,0.705882}%
\pgfsetstrokecolor{currentstroke}%
\pgfsetstrokeopacity{0.612753}%
\pgfsetdash{}{0pt}%
\pgfpathmoveto{\pgfqpoint{0.842015in}{1.226557in}}%
\pgfpathcurveto{\pgfqpoint{0.850251in}{1.226557in}}{\pgfqpoint{0.858151in}{1.229829in}}{\pgfqpoint{0.863975in}{1.235653in}}%
\pgfpathcurveto{\pgfqpoint{0.869799in}{1.241477in}}{\pgfqpoint{0.873071in}{1.249377in}}{\pgfqpoint{0.873071in}{1.257613in}}%
\pgfpathcurveto{\pgfqpoint{0.873071in}{1.265850in}}{\pgfqpoint{0.869799in}{1.273750in}}{\pgfqpoint{0.863975in}{1.279574in}}%
\pgfpathcurveto{\pgfqpoint{0.858151in}{1.285398in}}{\pgfqpoint{0.850251in}{1.288670in}}{\pgfqpoint{0.842015in}{1.288670in}}%
\pgfpathcurveto{\pgfqpoint{0.833779in}{1.288670in}}{\pgfqpoint{0.825878in}{1.285398in}}{\pgfqpoint{0.820055in}{1.279574in}}%
\pgfpathcurveto{\pgfqpoint{0.814231in}{1.273750in}}{\pgfqpoint{0.810958in}{1.265850in}}{\pgfqpoint{0.810958in}{1.257613in}}%
\pgfpathcurveto{\pgfqpoint{0.810958in}{1.249377in}}{\pgfqpoint{0.814231in}{1.241477in}}{\pgfqpoint{0.820055in}{1.235653in}}%
\pgfpathcurveto{\pgfqpoint{0.825878in}{1.229829in}}{\pgfqpoint{0.833779in}{1.226557in}}{\pgfqpoint{0.842015in}{1.226557in}}%
\pgfpathclose%
\pgfusepath{stroke,fill}%
\end{pgfscope}%
\begin{pgfscope}%
\pgfpathrectangle{\pgfqpoint{0.100000in}{0.220728in}}{\pgfqpoint{3.696000in}{3.696000in}}%
\pgfusepath{clip}%
\pgfsetbuttcap%
\pgfsetroundjoin%
\definecolor{currentfill}{rgb}{0.121569,0.466667,0.705882}%
\pgfsetfillcolor{currentfill}%
\pgfsetfillopacity{0.613139}%
\pgfsetlinewidth{1.003750pt}%
\definecolor{currentstroke}{rgb}{0.121569,0.466667,0.705882}%
\pgfsetstrokecolor{currentstroke}%
\pgfsetstrokeopacity{0.613139}%
\pgfsetdash{}{0pt}%
\pgfpathmoveto{\pgfqpoint{0.826359in}{1.236053in}}%
\pgfpathcurveto{\pgfqpoint{0.834596in}{1.236053in}}{\pgfqpoint{0.842496in}{1.239325in}}{\pgfqpoint{0.848320in}{1.245149in}}%
\pgfpathcurveto{\pgfqpoint{0.854143in}{1.250973in}}{\pgfqpoint{0.857416in}{1.258873in}}{\pgfqpoint{0.857416in}{1.267109in}}%
\pgfpathcurveto{\pgfqpoint{0.857416in}{1.275345in}}{\pgfqpoint{0.854143in}{1.283246in}}{\pgfqpoint{0.848320in}{1.289069in}}%
\pgfpathcurveto{\pgfqpoint{0.842496in}{1.294893in}}{\pgfqpoint{0.834596in}{1.298166in}}{\pgfqpoint{0.826359in}{1.298166in}}%
\pgfpathcurveto{\pgfqpoint{0.818123in}{1.298166in}}{\pgfqpoint{0.810223in}{1.294893in}}{\pgfqpoint{0.804399in}{1.289069in}}%
\pgfpathcurveto{\pgfqpoint{0.798575in}{1.283246in}}{\pgfqpoint{0.795303in}{1.275345in}}{\pgfqpoint{0.795303in}{1.267109in}}%
\pgfpathcurveto{\pgfqpoint{0.795303in}{1.258873in}}{\pgfqpoint{0.798575in}{1.250973in}}{\pgfqpoint{0.804399in}{1.245149in}}%
\pgfpathcurveto{\pgfqpoint{0.810223in}{1.239325in}}{\pgfqpoint{0.818123in}{1.236053in}}{\pgfqpoint{0.826359in}{1.236053in}}%
\pgfpathclose%
\pgfusepath{stroke,fill}%
\end{pgfscope}%
\begin{pgfscope}%
\pgfpathrectangle{\pgfqpoint{0.100000in}{0.220728in}}{\pgfqpoint{3.696000in}{3.696000in}}%
\pgfusepath{clip}%
\pgfsetbuttcap%
\pgfsetroundjoin%
\definecolor{currentfill}{rgb}{0.121569,0.466667,0.705882}%
\pgfsetfillcolor{currentfill}%
\pgfsetfillopacity{0.613744}%
\pgfsetlinewidth{1.003750pt}%
\definecolor{currentstroke}{rgb}{0.121569,0.466667,0.705882}%
\pgfsetstrokecolor{currentstroke}%
\pgfsetstrokeopacity{0.613744}%
\pgfsetdash{}{0pt}%
\pgfpathmoveto{\pgfqpoint{0.838954in}{1.223048in}}%
\pgfpathcurveto{\pgfqpoint{0.847190in}{1.223048in}}{\pgfqpoint{0.855090in}{1.226320in}}{\pgfqpoint{0.860914in}{1.232144in}}%
\pgfpathcurveto{\pgfqpoint{0.866738in}{1.237968in}}{\pgfqpoint{0.870010in}{1.245868in}}{\pgfqpoint{0.870010in}{1.254105in}}%
\pgfpathcurveto{\pgfqpoint{0.870010in}{1.262341in}}{\pgfqpoint{0.866738in}{1.270241in}}{\pgfqpoint{0.860914in}{1.276065in}}%
\pgfpathcurveto{\pgfqpoint{0.855090in}{1.281889in}}{\pgfqpoint{0.847190in}{1.285161in}}{\pgfqpoint{0.838954in}{1.285161in}}%
\pgfpathcurveto{\pgfqpoint{0.830717in}{1.285161in}}{\pgfqpoint{0.822817in}{1.281889in}}{\pgfqpoint{0.816993in}{1.276065in}}%
\pgfpathcurveto{\pgfqpoint{0.811169in}{1.270241in}}{\pgfqpoint{0.807897in}{1.262341in}}{\pgfqpoint{0.807897in}{1.254105in}}%
\pgfpathcurveto{\pgfqpoint{0.807897in}{1.245868in}}{\pgfqpoint{0.811169in}{1.237968in}}{\pgfqpoint{0.816993in}{1.232144in}}%
\pgfpathcurveto{\pgfqpoint{0.822817in}{1.226320in}}{\pgfqpoint{0.830717in}{1.223048in}}{\pgfqpoint{0.838954in}{1.223048in}}%
\pgfpathclose%
\pgfusepath{stroke,fill}%
\end{pgfscope}%
\begin{pgfscope}%
\pgfpathrectangle{\pgfqpoint{0.100000in}{0.220728in}}{\pgfqpoint{3.696000in}{3.696000in}}%
\pgfusepath{clip}%
\pgfsetbuttcap%
\pgfsetroundjoin%
\definecolor{currentfill}{rgb}{0.121569,0.466667,0.705882}%
\pgfsetfillcolor{currentfill}%
\pgfsetfillopacity{0.614011}%
\pgfsetlinewidth{1.003750pt}%
\definecolor{currentstroke}{rgb}{0.121569,0.466667,0.705882}%
\pgfsetstrokecolor{currentstroke}%
\pgfsetstrokeopacity{0.614011}%
\pgfsetdash{}{0pt}%
\pgfpathmoveto{\pgfqpoint{0.823482in}{1.228622in}}%
\pgfpathcurveto{\pgfqpoint{0.831718in}{1.228622in}}{\pgfqpoint{0.839618in}{1.231894in}}{\pgfqpoint{0.845442in}{1.237718in}}%
\pgfpathcurveto{\pgfqpoint{0.851266in}{1.243542in}}{\pgfqpoint{0.854538in}{1.251442in}}{\pgfqpoint{0.854538in}{1.259679in}}%
\pgfpathcurveto{\pgfqpoint{0.854538in}{1.267915in}}{\pgfqpoint{0.851266in}{1.275815in}}{\pgfqpoint{0.845442in}{1.281639in}}%
\pgfpathcurveto{\pgfqpoint{0.839618in}{1.287463in}}{\pgfqpoint{0.831718in}{1.290735in}}{\pgfqpoint{0.823482in}{1.290735in}}%
\pgfpathcurveto{\pgfqpoint{0.815245in}{1.290735in}}{\pgfqpoint{0.807345in}{1.287463in}}{\pgfqpoint{0.801521in}{1.281639in}}%
\pgfpathcurveto{\pgfqpoint{0.795698in}{1.275815in}}{\pgfqpoint{0.792425in}{1.267915in}}{\pgfqpoint{0.792425in}{1.259679in}}%
\pgfpathcurveto{\pgfqpoint{0.792425in}{1.251442in}}{\pgfqpoint{0.795698in}{1.243542in}}{\pgfqpoint{0.801521in}{1.237718in}}%
\pgfpathcurveto{\pgfqpoint{0.807345in}{1.231894in}}{\pgfqpoint{0.815245in}{1.228622in}}{\pgfqpoint{0.823482in}{1.228622in}}%
\pgfpathclose%
\pgfusepath{stroke,fill}%
\end{pgfscope}%
\begin{pgfscope}%
\pgfpathrectangle{\pgfqpoint{0.100000in}{0.220728in}}{\pgfqpoint{3.696000in}{3.696000in}}%
\pgfusepath{clip}%
\pgfsetbuttcap%
\pgfsetroundjoin%
\definecolor{currentfill}{rgb}{0.121569,0.466667,0.705882}%
\pgfsetfillcolor{currentfill}%
\pgfsetfillopacity{0.614792}%
\pgfsetlinewidth{1.003750pt}%
\definecolor{currentstroke}{rgb}{0.121569,0.466667,0.705882}%
\pgfsetstrokecolor{currentstroke}%
\pgfsetstrokeopacity{0.614792}%
\pgfsetdash{}{0pt}%
\pgfpathmoveto{\pgfqpoint{0.822684in}{1.223416in}}%
\pgfpathcurveto{\pgfqpoint{0.830920in}{1.223416in}}{\pgfqpoint{0.838821in}{1.226688in}}{\pgfqpoint{0.844644in}{1.232512in}}%
\pgfpathcurveto{\pgfqpoint{0.850468in}{1.238336in}}{\pgfqpoint{0.853741in}{1.246236in}}{\pgfqpoint{0.853741in}{1.254472in}}%
\pgfpathcurveto{\pgfqpoint{0.853741in}{1.262709in}}{\pgfqpoint{0.850468in}{1.270609in}}{\pgfqpoint{0.844644in}{1.276433in}}%
\pgfpathcurveto{\pgfqpoint{0.838821in}{1.282257in}}{\pgfqpoint{0.830920in}{1.285529in}}{\pgfqpoint{0.822684in}{1.285529in}}%
\pgfpathcurveto{\pgfqpoint{0.814448in}{1.285529in}}{\pgfqpoint{0.806548in}{1.282257in}}{\pgfqpoint{0.800724in}{1.276433in}}%
\pgfpathcurveto{\pgfqpoint{0.794900in}{1.270609in}}{\pgfqpoint{0.791628in}{1.262709in}}{\pgfqpoint{0.791628in}{1.254472in}}%
\pgfpathcurveto{\pgfqpoint{0.791628in}{1.246236in}}{\pgfqpoint{0.794900in}{1.238336in}}{\pgfqpoint{0.800724in}{1.232512in}}%
\pgfpathcurveto{\pgfqpoint{0.806548in}{1.226688in}}{\pgfqpoint{0.814448in}{1.223416in}}{\pgfqpoint{0.822684in}{1.223416in}}%
\pgfpathclose%
\pgfusepath{stroke,fill}%
\end{pgfscope}%
\begin{pgfscope}%
\pgfpathrectangle{\pgfqpoint{0.100000in}{0.220728in}}{\pgfqpoint{3.696000in}{3.696000in}}%
\pgfusepath{clip}%
\pgfsetbuttcap%
\pgfsetroundjoin%
\definecolor{currentfill}{rgb}{0.121569,0.466667,0.705882}%
\pgfsetfillcolor{currentfill}%
\pgfsetfillopacity{0.615003}%
\pgfsetlinewidth{1.003750pt}%
\definecolor{currentstroke}{rgb}{0.121569,0.466667,0.705882}%
\pgfsetstrokecolor{currentstroke}%
\pgfsetstrokeopacity{0.615003}%
\pgfsetdash{}{0pt}%
\pgfpathmoveto{\pgfqpoint{0.835204in}{1.218986in}}%
\pgfpathcurveto{\pgfqpoint{0.843440in}{1.218986in}}{\pgfqpoint{0.851340in}{1.222258in}}{\pgfqpoint{0.857164in}{1.228082in}}%
\pgfpathcurveto{\pgfqpoint{0.862988in}{1.233906in}}{\pgfqpoint{0.866261in}{1.241806in}}{\pgfqpoint{0.866261in}{1.250042in}}%
\pgfpathcurveto{\pgfqpoint{0.866261in}{1.258279in}}{\pgfqpoint{0.862988in}{1.266179in}}{\pgfqpoint{0.857164in}{1.272003in}}%
\pgfpathcurveto{\pgfqpoint{0.851340in}{1.277827in}}{\pgfqpoint{0.843440in}{1.281099in}}{\pgfqpoint{0.835204in}{1.281099in}}%
\pgfpathcurveto{\pgfqpoint{0.826968in}{1.281099in}}{\pgfqpoint{0.819068in}{1.277827in}}{\pgfqpoint{0.813244in}{1.272003in}}%
\pgfpathcurveto{\pgfqpoint{0.807420in}{1.266179in}}{\pgfqpoint{0.804148in}{1.258279in}}{\pgfqpoint{0.804148in}{1.250042in}}%
\pgfpathcurveto{\pgfqpoint{0.804148in}{1.241806in}}{\pgfqpoint{0.807420in}{1.233906in}}{\pgfqpoint{0.813244in}{1.228082in}}%
\pgfpathcurveto{\pgfqpoint{0.819068in}{1.222258in}}{\pgfqpoint{0.826968in}{1.218986in}}{\pgfqpoint{0.835204in}{1.218986in}}%
\pgfpathclose%
\pgfusepath{stroke,fill}%
\end{pgfscope}%
\begin{pgfscope}%
\pgfpathrectangle{\pgfqpoint{0.100000in}{0.220728in}}{\pgfqpoint{3.696000in}{3.696000in}}%
\pgfusepath{clip}%
\pgfsetbuttcap%
\pgfsetroundjoin%
\definecolor{currentfill}{rgb}{0.121569,0.466667,0.705882}%
\pgfsetfillcolor{currentfill}%
\pgfsetfillopacity{0.615377}%
\pgfsetlinewidth{1.003750pt}%
\definecolor{currentstroke}{rgb}{0.121569,0.466667,0.705882}%
\pgfsetstrokecolor{currentstroke}%
\pgfsetstrokeopacity{0.615377}%
\pgfsetdash{}{0pt}%
\pgfpathmoveto{\pgfqpoint{0.822321in}{1.219635in}}%
\pgfpathcurveto{\pgfqpoint{0.830557in}{1.219635in}}{\pgfqpoint{0.838457in}{1.222908in}}{\pgfqpoint{0.844281in}{1.228732in}}%
\pgfpathcurveto{\pgfqpoint{0.850105in}{1.234556in}}{\pgfqpoint{0.853377in}{1.242456in}}{\pgfqpoint{0.853377in}{1.250692in}}%
\pgfpathcurveto{\pgfqpoint{0.853377in}{1.258928in}}{\pgfqpoint{0.850105in}{1.266828in}}{\pgfqpoint{0.844281in}{1.272652in}}%
\pgfpathcurveto{\pgfqpoint{0.838457in}{1.278476in}}{\pgfqpoint{0.830557in}{1.281748in}}{\pgfqpoint{0.822321in}{1.281748in}}%
\pgfpathcurveto{\pgfqpoint{0.814085in}{1.281748in}}{\pgfqpoint{0.806185in}{1.278476in}}{\pgfqpoint{0.800361in}{1.272652in}}%
\pgfpathcurveto{\pgfqpoint{0.794537in}{1.266828in}}{\pgfqpoint{0.791264in}{1.258928in}}{\pgfqpoint{0.791264in}{1.250692in}}%
\pgfpathcurveto{\pgfqpoint{0.791264in}{1.242456in}}{\pgfqpoint{0.794537in}{1.234556in}}{\pgfqpoint{0.800361in}{1.228732in}}%
\pgfpathcurveto{\pgfqpoint{0.806185in}{1.222908in}}{\pgfqpoint{0.814085in}{1.219635in}}{\pgfqpoint{0.822321in}{1.219635in}}%
\pgfpathclose%
\pgfusepath{stroke,fill}%
\end{pgfscope}%
\begin{pgfscope}%
\pgfpathrectangle{\pgfqpoint{0.100000in}{0.220728in}}{\pgfqpoint{3.696000in}{3.696000in}}%
\pgfusepath{clip}%
\pgfsetbuttcap%
\pgfsetroundjoin%
\definecolor{currentfill}{rgb}{0.121569,0.466667,0.705882}%
\pgfsetfillcolor{currentfill}%
\pgfsetfillopacity{0.615738}%
\pgfsetlinewidth{1.003750pt}%
\definecolor{currentstroke}{rgb}{0.121569,0.466667,0.705882}%
\pgfsetstrokecolor{currentstroke}%
\pgfsetstrokeopacity{0.615738}%
\pgfsetdash{}{0pt}%
\pgfpathmoveto{\pgfqpoint{0.833339in}{1.216617in}}%
\pgfpathcurveto{\pgfqpoint{0.841575in}{1.216617in}}{\pgfqpoint{0.849475in}{1.219889in}}{\pgfqpoint{0.855299in}{1.225713in}}%
\pgfpathcurveto{\pgfqpoint{0.861123in}{1.231537in}}{\pgfqpoint{0.864395in}{1.239437in}}{\pgfqpoint{0.864395in}{1.247674in}}%
\pgfpathcurveto{\pgfqpoint{0.864395in}{1.255910in}}{\pgfqpoint{0.861123in}{1.263810in}}{\pgfqpoint{0.855299in}{1.269634in}}%
\pgfpathcurveto{\pgfqpoint{0.849475in}{1.275458in}}{\pgfqpoint{0.841575in}{1.278730in}}{\pgfqpoint{0.833339in}{1.278730in}}%
\pgfpathcurveto{\pgfqpoint{0.825102in}{1.278730in}}{\pgfqpoint{0.817202in}{1.275458in}}{\pgfqpoint{0.811379in}{1.269634in}}%
\pgfpathcurveto{\pgfqpoint{0.805555in}{1.263810in}}{\pgfqpoint{0.802282in}{1.255910in}}{\pgfqpoint{0.802282in}{1.247674in}}%
\pgfpathcurveto{\pgfqpoint{0.802282in}{1.239437in}}{\pgfqpoint{0.805555in}{1.231537in}}{\pgfqpoint{0.811379in}{1.225713in}}%
\pgfpathcurveto{\pgfqpoint{0.817202in}{1.219889in}}{\pgfqpoint{0.825102in}{1.216617in}}{\pgfqpoint{0.833339in}{1.216617in}}%
\pgfpathclose%
\pgfusepath{stroke,fill}%
\end{pgfscope}%
\begin{pgfscope}%
\pgfpathrectangle{\pgfqpoint{0.100000in}{0.220728in}}{\pgfqpoint{3.696000in}{3.696000in}}%
\pgfusepath{clip}%
\pgfsetbuttcap%
\pgfsetroundjoin%
\definecolor{currentfill}{rgb}{0.121569,0.466667,0.705882}%
\pgfsetfillcolor{currentfill}%
\pgfsetfillopacity{0.615748}%
\pgfsetlinewidth{1.003750pt}%
\definecolor{currentstroke}{rgb}{0.121569,0.466667,0.705882}%
\pgfsetstrokecolor{currentstroke}%
\pgfsetstrokeopacity{0.615748}%
\pgfsetdash{}{0pt}%
\pgfpathmoveto{\pgfqpoint{0.823174in}{1.216724in}}%
\pgfpathcurveto{\pgfqpoint{0.831411in}{1.216724in}}{\pgfqpoint{0.839311in}{1.219996in}}{\pgfqpoint{0.845135in}{1.225820in}}%
\pgfpathcurveto{\pgfqpoint{0.850958in}{1.231644in}}{\pgfqpoint{0.854231in}{1.239544in}}{\pgfqpoint{0.854231in}{1.247780in}}%
\pgfpathcurveto{\pgfqpoint{0.854231in}{1.256016in}}{\pgfqpoint{0.850958in}{1.263917in}}{\pgfqpoint{0.845135in}{1.269740in}}%
\pgfpathcurveto{\pgfqpoint{0.839311in}{1.275564in}}{\pgfqpoint{0.831411in}{1.278837in}}{\pgfqpoint{0.823174in}{1.278837in}}%
\pgfpathcurveto{\pgfqpoint{0.814938in}{1.278837in}}{\pgfqpoint{0.807038in}{1.275564in}}{\pgfqpoint{0.801214in}{1.269740in}}%
\pgfpathcurveto{\pgfqpoint{0.795390in}{1.263917in}}{\pgfqpoint{0.792118in}{1.256016in}}{\pgfqpoint{0.792118in}{1.247780in}}%
\pgfpathcurveto{\pgfqpoint{0.792118in}{1.239544in}}{\pgfqpoint{0.795390in}{1.231644in}}{\pgfqpoint{0.801214in}{1.225820in}}%
\pgfpathcurveto{\pgfqpoint{0.807038in}{1.219996in}}{\pgfqpoint{0.814938in}{1.216724in}}{\pgfqpoint{0.823174in}{1.216724in}}%
\pgfpathclose%
\pgfusepath{stroke,fill}%
\end{pgfscope}%
\begin{pgfscope}%
\pgfpathrectangle{\pgfqpoint{0.100000in}{0.220728in}}{\pgfqpoint{3.696000in}{3.696000in}}%
\pgfusepath{clip}%
\pgfsetbuttcap%
\pgfsetroundjoin%
\definecolor{currentfill}{rgb}{0.121569,0.466667,0.705882}%
\pgfsetfillcolor{currentfill}%
\pgfsetfillopacity{0.615816}%
\pgfsetlinewidth{1.003750pt}%
\definecolor{currentstroke}{rgb}{0.121569,0.466667,0.705882}%
\pgfsetstrokecolor{currentstroke}%
\pgfsetstrokeopacity{0.615816}%
\pgfsetdash{}{0pt}%
\pgfpathmoveto{\pgfqpoint{3.119517in}{2.981589in}}%
\pgfpathcurveto{\pgfqpoint{3.127753in}{2.981589in}}{\pgfqpoint{3.135653in}{2.984861in}}{\pgfqpoint{3.141477in}{2.990685in}}%
\pgfpathcurveto{\pgfqpoint{3.147301in}{2.996509in}}{\pgfqpoint{3.150574in}{3.004409in}}{\pgfqpoint{3.150574in}{3.012645in}}%
\pgfpathcurveto{\pgfqpoint{3.150574in}{3.020882in}}{\pgfqpoint{3.147301in}{3.028782in}}{\pgfqpoint{3.141477in}{3.034606in}}%
\pgfpathcurveto{\pgfqpoint{3.135653in}{3.040429in}}{\pgfqpoint{3.127753in}{3.043702in}}{\pgfqpoint{3.119517in}{3.043702in}}%
\pgfpathcurveto{\pgfqpoint{3.111281in}{3.043702in}}{\pgfqpoint{3.103381in}{3.040429in}}{\pgfqpoint{3.097557in}{3.034606in}}%
\pgfpathcurveto{\pgfqpoint{3.091733in}{3.028782in}}{\pgfqpoint{3.088461in}{3.020882in}}{\pgfqpoint{3.088461in}{3.012645in}}%
\pgfpathcurveto{\pgfqpoint{3.088461in}{3.004409in}}{\pgfqpoint{3.091733in}{2.996509in}}{\pgfqpoint{3.097557in}{2.990685in}}%
\pgfpathcurveto{\pgfqpoint{3.103381in}{2.984861in}}{\pgfqpoint{3.111281in}{2.981589in}}{\pgfqpoint{3.119517in}{2.981589in}}%
\pgfpathclose%
\pgfusepath{stroke,fill}%
\end{pgfscope}%
\begin{pgfscope}%
\pgfpathrectangle{\pgfqpoint{0.100000in}{0.220728in}}{\pgfqpoint{3.696000in}{3.696000in}}%
\pgfusepath{clip}%
\pgfsetbuttcap%
\pgfsetroundjoin%
\definecolor{currentfill}{rgb}{0.121569,0.466667,0.705882}%
\pgfsetfillcolor{currentfill}%
\pgfsetfillopacity{0.615967}%
\pgfsetlinewidth{1.003750pt}%
\definecolor{currentstroke}{rgb}{0.121569,0.466667,0.705882}%
\pgfsetstrokecolor{currentstroke}%
\pgfsetstrokeopacity{0.615967}%
\pgfsetdash{}{0pt}%
\pgfpathmoveto{\pgfqpoint{0.824119in}{1.215019in}}%
\pgfpathcurveto{\pgfqpoint{0.832355in}{1.215019in}}{\pgfqpoint{0.840255in}{1.218291in}}{\pgfqpoint{0.846079in}{1.224115in}}%
\pgfpathcurveto{\pgfqpoint{0.851903in}{1.229939in}}{\pgfqpoint{0.855176in}{1.237839in}}{\pgfqpoint{0.855176in}{1.246076in}}%
\pgfpathcurveto{\pgfqpoint{0.855176in}{1.254312in}}{\pgfqpoint{0.851903in}{1.262212in}}{\pgfqpoint{0.846079in}{1.268036in}}%
\pgfpathcurveto{\pgfqpoint{0.840255in}{1.273860in}}{\pgfqpoint{0.832355in}{1.277132in}}{\pgfqpoint{0.824119in}{1.277132in}}%
\pgfpathcurveto{\pgfqpoint{0.815883in}{1.277132in}}{\pgfqpoint{0.807983in}{1.273860in}}{\pgfqpoint{0.802159in}{1.268036in}}%
\pgfpathcurveto{\pgfqpoint{0.796335in}{1.262212in}}{\pgfqpoint{0.793063in}{1.254312in}}{\pgfqpoint{0.793063in}{1.246076in}}%
\pgfpathcurveto{\pgfqpoint{0.793063in}{1.237839in}}{\pgfqpoint{0.796335in}{1.229939in}}{\pgfqpoint{0.802159in}{1.224115in}}%
\pgfpathcurveto{\pgfqpoint{0.807983in}{1.218291in}}{\pgfqpoint{0.815883in}{1.215019in}}{\pgfqpoint{0.824119in}{1.215019in}}%
\pgfpathclose%
\pgfusepath{stroke,fill}%
\end{pgfscope}%
\begin{pgfscope}%
\pgfpathrectangle{\pgfqpoint{0.100000in}{0.220728in}}{\pgfqpoint{3.696000in}{3.696000in}}%
\pgfusepath{clip}%
\pgfsetbuttcap%
\pgfsetroundjoin%
\definecolor{currentfill}{rgb}{0.121569,0.466667,0.705882}%
\pgfsetfillcolor{currentfill}%
\pgfsetfillopacity{0.616135}%
\pgfsetlinewidth{1.003750pt}%
\definecolor{currentstroke}{rgb}{0.121569,0.466667,0.705882}%
\pgfsetstrokecolor{currentstroke}%
\pgfsetstrokeopacity{0.616135}%
\pgfsetdash{}{0pt}%
\pgfpathmoveto{\pgfqpoint{0.832336in}{1.215254in}}%
\pgfpathcurveto{\pgfqpoint{0.840572in}{1.215254in}}{\pgfqpoint{0.848472in}{1.218526in}}{\pgfqpoint{0.854296in}{1.224350in}}%
\pgfpathcurveto{\pgfqpoint{0.860120in}{1.230174in}}{\pgfqpoint{0.863393in}{1.238074in}}{\pgfqpoint{0.863393in}{1.246310in}}%
\pgfpathcurveto{\pgfqpoint{0.863393in}{1.254547in}}{\pgfqpoint{0.860120in}{1.262447in}}{\pgfqpoint{0.854296in}{1.268271in}}%
\pgfpathcurveto{\pgfqpoint{0.848472in}{1.274094in}}{\pgfqpoint{0.840572in}{1.277367in}}{\pgfqpoint{0.832336in}{1.277367in}}%
\pgfpathcurveto{\pgfqpoint{0.824100in}{1.277367in}}{\pgfqpoint{0.816200in}{1.274094in}}{\pgfqpoint{0.810376in}{1.268271in}}%
\pgfpathcurveto{\pgfqpoint{0.804552in}{1.262447in}}{\pgfqpoint{0.801280in}{1.254547in}}{\pgfqpoint{0.801280in}{1.246310in}}%
\pgfpathcurveto{\pgfqpoint{0.801280in}{1.238074in}}{\pgfqpoint{0.804552in}{1.230174in}}{\pgfqpoint{0.810376in}{1.224350in}}%
\pgfpathcurveto{\pgfqpoint{0.816200in}{1.218526in}}{\pgfqpoint{0.824100in}{1.215254in}}{\pgfqpoint{0.832336in}{1.215254in}}%
\pgfpathclose%
\pgfusepath{stroke,fill}%
\end{pgfscope}%
\begin{pgfscope}%
\pgfpathrectangle{\pgfqpoint{0.100000in}{0.220728in}}{\pgfqpoint{3.696000in}{3.696000in}}%
\pgfusepath{clip}%
\pgfsetbuttcap%
\pgfsetroundjoin%
\definecolor{currentfill}{rgb}{0.121569,0.466667,0.705882}%
\pgfsetfillcolor{currentfill}%
\pgfsetfillopacity{0.616350}%
\pgfsetlinewidth{1.003750pt}%
\definecolor{currentstroke}{rgb}{0.121569,0.466667,0.705882}%
\pgfsetstrokecolor{currentstroke}%
\pgfsetstrokeopacity{0.616350}%
\pgfsetdash{}{0pt}%
\pgfpathmoveto{\pgfqpoint{0.831751in}{1.214535in}}%
\pgfpathcurveto{\pgfqpoint{0.839988in}{1.214535in}}{\pgfqpoint{0.847888in}{1.217808in}}{\pgfqpoint{0.853712in}{1.223631in}}%
\pgfpathcurveto{\pgfqpoint{0.859536in}{1.229455in}}{\pgfqpoint{0.862808in}{1.237355in}}{\pgfqpoint{0.862808in}{1.245592in}}%
\pgfpathcurveto{\pgfqpoint{0.862808in}{1.253828in}}{\pgfqpoint{0.859536in}{1.261728in}}{\pgfqpoint{0.853712in}{1.267552in}}%
\pgfpathcurveto{\pgfqpoint{0.847888in}{1.273376in}}{\pgfqpoint{0.839988in}{1.276648in}}{\pgfqpoint{0.831751in}{1.276648in}}%
\pgfpathcurveto{\pgfqpoint{0.823515in}{1.276648in}}{\pgfqpoint{0.815615in}{1.273376in}}{\pgfqpoint{0.809791in}{1.267552in}}%
\pgfpathcurveto{\pgfqpoint{0.803967in}{1.261728in}}{\pgfqpoint{0.800695in}{1.253828in}}{\pgfqpoint{0.800695in}{1.245592in}}%
\pgfpathcurveto{\pgfqpoint{0.800695in}{1.237355in}}{\pgfqpoint{0.803967in}{1.229455in}}{\pgfqpoint{0.809791in}{1.223631in}}%
\pgfpathcurveto{\pgfqpoint{0.815615in}{1.217808in}}{\pgfqpoint{0.823515in}{1.214535in}}{\pgfqpoint{0.831751in}{1.214535in}}%
\pgfpathclose%
\pgfusepath{stroke,fill}%
\end{pgfscope}%
\begin{pgfscope}%
\pgfpathrectangle{\pgfqpoint{0.100000in}{0.220728in}}{\pgfqpoint{3.696000in}{3.696000in}}%
\pgfusepath{clip}%
\pgfsetbuttcap%
\pgfsetroundjoin%
\definecolor{currentfill}{rgb}{0.121569,0.466667,0.705882}%
\pgfsetfillcolor{currentfill}%
\pgfsetfillopacity{0.616379}%
\pgfsetlinewidth{1.003750pt}%
\definecolor{currentstroke}{rgb}{0.121569,0.466667,0.705882}%
\pgfsetstrokecolor{currentstroke}%
\pgfsetstrokeopacity{0.616379}%
\pgfsetdash{}{0pt}%
\pgfpathmoveto{\pgfqpoint{0.826405in}{1.213225in}}%
\pgfpathcurveto{\pgfqpoint{0.834641in}{1.213225in}}{\pgfqpoint{0.842541in}{1.216497in}}{\pgfqpoint{0.848365in}{1.222321in}}%
\pgfpathcurveto{\pgfqpoint{0.854189in}{1.228145in}}{\pgfqpoint{0.857461in}{1.236045in}}{\pgfqpoint{0.857461in}{1.244282in}}%
\pgfpathcurveto{\pgfqpoint{0.857461in}{1.252518in}}{\pgfqpoint{0.854189in}{1.260418in}}{\pgfqpoint{0.848365in}{1.266242in}}%
\pgfpathcurveto{\pgfqpoint{0.842541in}{1.272066in}}{\pgfqpoint{0.834641in}{1.275338in}}{\pgfqpoint{0.826405in}{1.275338in}}%
\pgfpathcurveto{\pgfqpoint{0.818169in}{1.275338in}}{\pgfqpoint{0.810269in}{1.272066in}}{\pgfqpoint{0.804445in}{1.266242in}}%
\pgfpathcurveto{\pgfqpoint{0.798621in}{1.260418in}}{\pgfqpoint{0.795348in}{1.252518in}}{\pgfqpoint{0.795348in}{1.244282in}}%
\pgfpathcurveto{\pgfqpoint{0.795348in}{1.236045in}}{\pgfqpoint{0.798621in}{1.228145in}}{\pgfqpoint{0.804445in}{1.222321in}}%
\pgfpathcurveto{\pgfqpoint{0.810269in}{1.216497in}}{\pgfqpoint{0.818169in}{1.213225in}}{\pgfqpoint{0.826405in}{1.213225in}}%
\pgfpathclose%
\pgfusepath{stroke,fill}%
\end{pgfscope}%
\begin{pgfscope}%
\pgfpathrectangle{\pgfqpoint{0.100000in}{0.220728in}}{\pgfqpoint{3.696000in}{3.696000in}}%
\pgfusepath{clip}%
\pgfsetbuttcap%
\pgfsetroundjoin%
\definecolor{currentfill}{rgb}{0.121569,0.466667,0.705882}%
\pgfsetfillcolor{currentfill}%
\pgfsetfillopacity{0.616465}%
\pgfsetlinewidth{1.003750pt}%
\definecolor{currentstroke}{rgb}{0.121569,0.466667,0.705882}%
\pgfsetstrokecolor{currentstroke}%
\pgfsetstrokeopacity{0.616465}%
\pgfsetdash{}{0pt}%
\pgfpathmoveto{\pgfqpoint{0.831442in}{1.214114in}}%
\pgfpathcurveto{\pgfqpoint{0.839679in}{1.214114in}}{\pgfqpoint{0.847579in}{1.217386in}}{\pgfqpoint{0.853403in}{1.223210in}}%
\pgfpathcurveto{\pgfqpoint{0.859227in}{1.229034in}}{\pgfqpoint{0.862499in}{1.236934in}}{\pgfqpoint{0.862499in}{1.245171in}}%
\pgfpathcurveto{\pgfqpoint{0.862499in}{1.253407in}}{\pgfqpoint{0.859227in}{1.261307in}}{\pgfqpoint{0.853403in}{1.267131in}}%
\pgfpathcurveto{\pgfqpoint{0.847579in}{1.272955in}}{\pgfqpoint{0.839679in}{1.276227in}}{\pgfqpoint{0.831442in}{1.276227in}}%
\pgfpathcurveto{\pgfqpoint{0.823206in}{1.276227in}}{\pgfqpoint{0.815306in}{1.272955in}}{\pgfqpoint{0.809482in}{1.267131in}}%
\pgfpathcurveto{\pgfqpoint{0.803658in}{1.261307in}}{\pgfqpoint{0.800386in}{1.253407in}}{\pgfqpoint{0.800386in}{1.245171in}}%
\pgfpathcurveto{\pgfqpoint{0.800386in}{1.236934in}}{\pgfqpoint{0.803658in}{1.229034in}}{\pgfqpoint{0.809482in}{1.223210in}}%
\pgfpathcurveto{\pgfqpoint{0.815306in}{1.217386in}}{\pgfqpoint{0.823206in}{1.214114in}}{\pgfqpoint{0.831442in}{1.214114in}}%
\pgfpathclose%
\pgfusepath{stroke,fill}%
\end{pgfscope}%
\begin{pgfscope}%
\pgfpathrectangle{\pgfqpoint{0.100000in}{0.220728in}}{\pgfqpoint{3.696000in}{3.696000in}}%
\pgfusepath{clip}%
\pgfsetbuttcap%
\pgfsetroundjoin%
\definecolor{currentfill}{rgb}{0.121569,0.466667,0.705882}%
\pgfsetfillcolor{currentfill}%
\pgfsetfillopacity{0.616524}%
\pgfsetlinewidth{1.003750pt}%
\definecolor{currentstroke}{rgb}{0.121569,0.466667,0.705882}%
\pgfsetstrokecolor{currentstroke}%
\pgfsetstrokeopacity{0.616524}%
\pgfsetdash{}{0pt}%
\pgfpathmoveto{\pgfqpoint{0.831268in}{1.213870in}}%
\pgfpathcurveto{\pgfqpoint{0.839504in}{1.213870in}}{\pgfqpoint{0.847404in}{1.217143in}}{\pgfqpoint{0.853228in}{1.222967in}}%
\pgfpathcurveto{\pgfqpoint{0.859052in}{1.228790in}}{\pgfqpoint{0.862324in}{1.236691in}}{\pgfqpoint{0.862324in}{1.244927in}}%
\pgfpathcurveto{\pgfqpoint{0.862324in}{1.253163in}}{\pgfqpoint{0.859052in}{1.261063in}}{\pgfqpoint{0.853228in}{1.266887in}}%
\pgfpathcurveto{\pgfqpoint{0.847404in}{1.272711in}}{\pgfqpoint{0.839504in}{1.275983in}}{\pgfqpoint{0.831268in}{1.275983in}}%
\pgfpathcurveto{\pgfqpoint{0.823031in}{1.275983in}}{\pgfqpoint{0.815131in}{1.272711in}}{\pgfqpoint{0.809307in}{1.266887in}}%
\pgfpathcurveto{\pgfqpoint{0.803483in}{1.261063in}}{\pgfqpoint{0.800211in}{1.253163in}}{\pgfqpoint{0.800211in}{1.244927in}}%
\pgfpathcurveto{\pgfqpoint{0.800211in}{1.236691in}}{\pgfqpoint{0.803483in}{1.228790in}}{\pgfqpoint{0.809307in}{1.222967in}}%
\pgfpathcurveto{\pgfqpoint{0.815131in}{1.217143in}}{\pgfqpoint{0.823031in}{1.213870in}}{\pgfqpoint{0.831268in}{1.213870in}}%
\pgfpathclose%
\pgfusepath{stroke,fill}%
\end{pgfscope}%
\begin{pgfscope}%
\pgfpathrectangle{\pgfqpoint{0.100000in}{0.220728in}}{\pgfqpoint{3.696000in}{3.696000in}}%
\pgfusepath{clip}%
\pgfsetbuttcap%
\pgfsetroundjoin%
\definecolor{currentfill}{rgb}{0.121569,0.466667,0.705882}%
\pgfsetfillcolor{currentfill}%
\pgfsetfillopacity{0.616556}%
\pgfsetlinewidth{1.003750pt}%
\definecolor{currentstroke}{rgb}{0.121569,0.466667,0.705882}%
\pgfsetstrokecolor{currentstroke}%
\pgfsetstrokeopacity{0.616556}%
\pgfsetdash{}{0pt}%
\pgfpathmoveto{\pgfqpoint{0.831169in}{1.213738in}}%
\pgfpathcurveto{\pgfqpoint{0.839405in}{1.213738in}}{\pgfqpoint{0.847305in}{1.217011in}}{\pgfqpoint{0.853129in}{1.222834in}}%
\pgfpathcurveto{\pgfqpoint{0.858953in}{1.228658in}}{\pgfqpoint{0.862225in}{1.236558in}}{\pgfqpoint{0.862225in}{1.244795in}}%
\pgfpathcurveto{\pgfqpoint{0.862225in}{1.253031in}}{\pgfqpoint{0.858953in}{1.260931in}}{\pgfqpoint{0.853129in}{1.266755in}}%
\pgfpathcurveto{\pgfqpoint{0.847305in}{1.272579in}}{\pgfqpoint{0.839405in}{1.275851in}}{\pgfqpoint{0.831169in}{1.275851in}}%
\pgfpathcurveto{\pgfqpoint{0.822932in}{1.275851in}}{\pgfqpoint{0.815032in}{1.272579in}}{\pgfqpoint{0.809208in}{1.266755in}}%
\pgfpathcurveto{\pgfqpoint{0.803384in}{1.260931in}}{\pgfqpoint{0.800112in}{1.253031in}}{\pgfqpoint{0.800112in}{1.244795in}}%
\pgfpathcurveto{\pgfqpoint{0.800112in}{1.236558in}}{\pgfqpoint{0.803384in}{1.228658in}}{\pgfqpoint{0.809208in}{1.222834in}}%
\pgfpathcurveto{\pgfqpoint{0.815032in}{1.217011in}}{\pgfqpoint{0.822932in}{1.213738in}}{\pgfqpoint{0.831169in}{1.213738in}}%
\pgfpathclose%
\pgfusepath{stroke,fill}%
\end{pgfscope}%
\begin{pgfscope}%
\pgfpathrectangle{\pgfqpoint{0.100000in}{0.220728in}}{\pgfqpoint{3.696000in}{3.696000in}}%
\pgfusepath{clip}%
\pgfsetbuttcap%
\pgfsetroundjoin%
\definecolor{currentfill}{rgb}{0.121569,0.466667,0.705882}%
\pgfsetfillcolor{currentfill}%
\pgfsetfillopacity{0.616564}%
\pgfsetlinewidth{1.003750pt}%
\definecolor{currentstroke}{rgb}{0.121569,0.466667,0.705882}%
\pgfsetstrokecolor{currentstroke}%
\pgfsetstrokeopacity{0.616564}%
\pgfsetdash{}{0pt}%
\pgfpathmoveto{\pgfqpoint{0.827725in}{1.212535in}}%
\pgfpathcurveto{\pgfqpoint{0.835961in}{1.212535in}}{\pgfqpoint{0.843862in}{1.215808in}}{\pgfqpoint{0.849685in}{1.221632in}}%
\pgfpathcurveto{\pgfqpoint{0.855509in}{1.227456in}}{\pgfqpoint{0.858782in}{1.235356in}}{\pgfqpoint{0.858782in}{1.243592in}}%
\pgfpathcurveto{\pgfqpoint{0.858782in}{1.251828in}}{\pgfqpoint{0.855509in}{1.259728in}}{\pgfqpoint{0.849685in}{1.265552in}}%
\pgfpathcurveto{\pgfqpoint{0.843862in}{1.271376in}}{\pgfqpoint{0.835961in}{1.274648in}}{\pgfqpoint{0.827725in}{1.274648in}}%
\pgfpathcurveto{\pgfqpoint{0.819489in}{1.274648in}}{\pgfqpoint{0.811589in}{1.271376in}}{\pgfqpoint{0.805765in}{1.265552in}}%
\pgfpathcurveto{\pgfqpoint{0.799941in}{1.259728in}}{\pgfqpoint{0.796669in}{1.251828in}}{\pgfqpoint{0.796669in}{1.243592in}}%
\pgfpathcurveto{\pgfqpoint{0.796669in}{1.235356in}}{\pgfqpoint{0.799941in}{1.227456in}}{\pgfqpoint{0.805765in}{1.221632in}}%
\pgfpathcurveto{\pgfqpoint{0.811589in}{1.215808in}}{\pgfqpoint{0.819489in}{1.212535in}}{\pgfqpoint{0.827725in}{1.212535in}}%
\pgfpathclose%
\pgfusepath{stroke,fill}%
\end{pgfscope}%
\begin{pgfscope}%
\pgfpathrectangle{\pgfqpoint{0.100000in}{0.220728in}}{\pgfqpoint{3.696000in}{3.696000in}}%
\pgfusepath{clip}%
\pgfsetbuttcap%
\pgfsetroundjoin%
\definecolor{currentfill}{rgb}{0.121569,0.466667,0.705882}%
\pgfsetfillcolor{currentfill}%
\pgfsetfillopacity{0.616574}%
\pgfsetlinewidth{1.003750pt}%
\definecolor{currentstroke}{rgb}{0.121569,0.466667,0.705882}%
\pgfsetstrokecolor{currentstroke}%
\pgfsetstrokeopacity{0.616574}%
\pgfsetdash{}{0pt}%
\pgfpathmoveto{\pgfqpoint{0.831115in}{1.213668in}}%
\pgfpathcurveto{\pgfqpoint{0.839351in}{1.213668in}}{\pgfqpoint{0.847251in}{1.216941in}}{\pgfqpoint{0.853075in}{1.222765in}}%
\pgfpathcurveto{\pgfqpoint{0.858899in}{1.228589in}}{\pgfqpoint{0.862171in}{1.236489in}}{\pgfqpoint{0.862171in}{1.244725in}}%
\pgfpathcurveto{\pgfqpoint{0.862171in}{1.252961in}}{\pgfqpoint{0.858899in}{1.260861in}}{\pgfqpoint{0.853075in}{1.266685in}}%
\pgfpathcurveto{\pgfqpoint{0.847251in}{1.272509in}}{\pgfqpoint{0.839351in}{1.275781in}}{\pgfqpoint{0.831115in}{1.275781in}}%
\pgfpathcurveto{\pgfqpoint{0.822879in}{1.275781in}}{\pgfqpoint{0.814979in}{1.272509in}}{\pgfqpoint{0.809155in}{1.266685in}}%
\pgfpathcurveto{\pgfqpoint{0.803331in}{1.260861in}}{\pgfqpoint{0.800058in}{1.252961in}}{\pgfqpoint{0.800058in}{1.244725in}}%
\pgfpathcurveto{\pgfqpoint{0.800058in}{1.236489in}}{\pgfqpoint{0.803331in}{1.228589in}}{\pgfqpoint{0.809155in}{1.222765in}}%
\pgfpathcurveto{\pgfqpoint{0.814979in}{1.216941in}}{\pgfqpoint{0.822879in}{1.213668in}}{\pgfqpoint{0.831115in}{1.213668in}}%
\pgfpathclose%
\pgfusepath{stroke,fill}%
\end{pgfscope}%
\begin{pgfscope}%
\pgfpathrectangle{\pgfqpoint{0.100000in}{0.220728in}}{\pgfqpoint{3.696000in}{3.696000in}}%
\pgfusepath{clip}%
\pgfsetbuttcap%
\pgfsetroundjoin%
\definecolor{currentfill}{rgb}{0.121569,0.466667,0.705882}%
\pgfsetfillcolor{currentfill}%
\pgfsetfillopacity{0.616582}%
\pgfsetlinewidth{1.003750pt}%
\definecolor{currentstroke}{rgb}{0.121569,0.466667,0.705882}%
\pgfsetstrokecolor{currentstroke}%
\pgfsetstrokeopacity{0.616582}%
\pgfsetdash{}{0pt}%
\pgfpathmoveto{\pgfqpoint{0.830483in}{1.213220in}}%
\pgfpathcurveto{\pgfqpoint{0.838719in}{1.213220in}}{\pgfqpoint{0.846619in}{1.216492in}}{\pgfqpoint{0.852443in}{1.222316in}}%
\pgfpathcurveto{\pgfqpoint{0.858267in}{1.228140in}}{\pgfqpoint{0.861539in}{1.236040in}}{\pgfqpoint{0.861539in}{1.244276in}}%
\pgfpathcurveto{\pgfqpoint{0.861539in}{1.252512in}}{\pgfqpoint{0.858267in}{1.260413in}}{\pgfqpoint{0.852443in}{1.266236in}}%
\pgfpathcurveto{\pgfqpoint{0.846619in}{1.272060in}}{\pgfqpoint{0.838719in}{1.275333in}}{\pgfqpoint{0.830483in}{1.275333in}}%
\pgfpathcurveto{\pgfqpoint{0.822247in}{1.275333in}}{\pgfqpoint{0.814347in}{1.272060in}}{\pgfqpoint{0.808523in}{1.266236in}}%
\pgfpathcurveto{\pgfqpoint{0.802699in}{1.260413in}}{\pgfqpoint{0.799426in}{1.252512in}}{\pgfqpoint{0.799426in}{1.244276in}}%
\pgfpathcurveto{\pgfqpoint{0.799426in}{1.236040in}}{\pgfqpoint{0.802699in}{1.228140in}}{\pgfqpoint{0.808523in}{1.222316in}}%
\pgfpathcurveto{\pgfqpoint{0.814347in}{1.216492in}}{\pgfqpoint{0.822247in}{1.213220in}}{\pgfqpoint{0.830483in}{1.213220in}}%
\pgfpathclose%
\pgfusepath{stroke,fill}%
\end{pgfscope}%
\begin{pgfscope}%
\pgfpathrectangle{\pgfqpoint{0.100000in}{0.220728in}}{\pgfqpoint{3.696000in}{3.696000in}}%
\pgfusepath{clip}%
\pgfsetbuttcap%
\pgfsetroundjoin%
\definecolor{currentfill}{rgb}{0.121569,0.466667,0.705882}%
\pgfsetfillcolor{currentfill}%
\pgfsetfillopacity{0.616585}%
\pgfsetlinewidth{1.003750pt}%
\definecolor{currentstroke}{rgb}{0.121569,0.466667,0.705882}%
\pgfsetstrokecolor{currentstroke}%
\pgfsetstrokeopacity{0.616585}%
\pgfsetdash{}{0pt}%
\pgfpathmoveto{\pgfqpoint{0.831087in}{1.213628in}}%
\pgfpathcurveto{\pgfqpoint{0.839323in}{1.213628in}}{\pgfqpoint{0.847223in}{1.216901in}}{\pgfqpoint{0.853047in}{1.222725in}}%
\pgfpathcurveto{\pgfqpoint{0.858871in}{1.228549in}}{\pgfqpoint{0.862143in}{1.236449in}}{\pgfqpoint{0.862143in}{1.244685in}}%
\pgfpathcurveto{\pgfqpoint{0.862143in}{1.252921in}}{\pgfqpoint{0.858871in}{1.260821in}}{\pgfqpoint{0.853047in}{1.266645in}}%
\pgfpathcurveto{\pgfqpoint{0.847223in}{1.272469in}}{\pgfqpoint{0.839323in}{1.275741in}}{\pgfqpoint{0.831087in}{1.275741in}}%
\pgfpathcurveto{\pgfqpoint{0.822850in}{1.275741in}}{\pgfqpoint{0.814950in}{1.272469in}}{\pgfqpoint{0.809126in}{1.266645in}}%
\pgfpathcurveto{\pgfqpoint{0.803302in}{1.260821in}}{\pgfqpoint{0.800030in}{1.252921in}}{\pgfqpoint{0.800030in}{1.244685in}}%
\pgfpathcurveto{\pgfqpoint{0.800030in}{1.236449in}}{\pgfqpoint{0.803302in}{1.228549in}}{\pgfqpoint{0.809126in}{1.222725in}}%
\pgfpathcurveto{\pgfqpoint{0.814950in}{1.216901in}}{\pgfqpoint{0.822850in}{1.213628in}}{\pgfqpoint{0.831087in}{1.213628in}}%
\pgfpathclose%
\pgfusepath{stroke,fill}%
\end{pgfscope}%
\begin{pgfscope}%
\pgfpathrectangle{\pgfqpoint{0.100000in}{0.220728in}}{\pgfqpoint{3.696000in}{3.696000in}}%
\pgfusepath{clip}%
\pgfsetbuttcap%
\pgfsetroundjoin%
\definecolor{currentfill}{rgb}{0.121569,0.466667,0.705882}%
\pgfsetfillcolor{currentfill}%
\pgfsetfillopacity{0.616589}%
\pgfsetlinewidth{1.003750pt}%
\definecolor{currentstroke}{rgb}{0.121569,0.466667,0.705882}%
\pgfsetstrokecolor{currentstroke}%
\pgfsetstrokeopacity{0.616589}%
\pgfsetdash{}{0pt}%
\pgfpathmoveto{\pgfqpoint{0.831071in}{1.213603in}}%
\pgfpathcurveto{\pgfqpoint{0.839308in}{1.213603in}}{\pgfqpoint{0.847208in}{1.216875in}}{\pgfqpoint{0.853032in}{1.222699in}}%
\pgfpathcurveto{\pgfqpoint{0.858855in}{1.228523in}}{\pgfqpoint{0.862128in}{1.236423in}}{\pgfqpoint{0.862128in}{1.244660in}}%
\pgfpathcurveto{\pgfqpoint{0.862128in}{1.252896in}}{\pgfqpoint{0.858855in}{1.260796in}}{\pgfqpoint{0.853032in}{1.266620in}}%
\pgfpathcurveto{\pgfqpoint{0.847208in}{1.272444in}}{\pgfqpoint{0.839308in}{1.275716in}}{\pgfqpoint{0.831071in}{1.275716in}}%
\pgfpathcurveto{\pgfqpoint{0.822835in}{1.275716in}}{\pgfqpoint{0.814935in}{1.272444in}}{\pgfqpoint{0.809111in}{1.266620in}}%
\pgfpathcurveto{\pgfqpoint{0.803287in}{1.260796in}}{\pgfqpoint{0.800015in}{1.252896in}}{\pgfqpoint{0.800015in}{1.244660in}}%
\pgfpathcurveto{\pgfqpoint{0.800015in}{1.236423in}}{\pgfqpoint{0.803287in}{1.228523in}}{\pgfqpoint{0.809111in}{1.222699in}}%
\pgfpathcurveto{\pgfqpoint{0.814935in}{1.216875in}}{\pgfqpoint{0.822835in}{1.213603in}}{\pgfqpoint{0.831071in}{1.213603in}}%
\pgfpathclose%
\pgfusepath{stroke,fill}%
\end{pgfscope}%
\begin{pgfscope}%
\pgfpathrectangle{\pgfqpoint{0.100000in}{0.220728in}}{\pgfqpoint{3.696000in}{3.696000in}}%
\pgfusepath{clip}%
\pgfsetbuttcap%
\pgfsetroundjoin%
\definecolor{currentfill}{rgb}{0.121569,0.466667,0.705882}%
\pgfsetfillcolor{currentfill}%
\pgfsetfillopacity{0.616592}%
\pgfsetlinewidth{1.003750pt}%
\definecolor{currentstroke}{rgb}{0.121569,0.466667,0.705882}%
\pgfsetstrokecolor{currentstroke}%
\pgfsetstrokeopacity{0.616592}%
\pgfsetdash{}{0pt}%
\pgfpathmoveto{\pgfqpoint{0.831063in}{1.213590in}}%
\pgfpathcurveto{\pgfqpoint{0.839299in}{1.213590in}}{\pgfqpoint{0.847199in}{1.216862in}}{\pgfqpoint{0.853023in}{1.222686in}}%
\pgfpathcurveto{\pgfqpoint{0.858847in}{1.228510in}}{\pgfqpoint{0.862119in}{1.236410in}}{\pgfqpoint{0.862119in}{1.244646in}}%
\pgfpathcurveto{\pgfqpoint{0.862119in}{1.252883in}}{\pgfqpoint{0.858847in}{1.260783in}}{\pgfqpoint{0.853023in}{1.266607in}}%
\pgfpathcurveto{\pgfqpoint{0.847199in}{1.272431in}}{\pgfqpoint{0.839299in}{1.275703in}}{\pgfqpoint{0.831063in}{1.275703in}}%
\pgfpathcurveto{\pgfqpoint{0.822826in}{1.275703in}}{\pgfqpoint{0.814926in}{1.272431in}}{\pgfqpoint{0.809103in}{1.266607in}}%
\pgfpathcurveto{\pgfqpoint{0.803279in}{1.260783in}}{\pgfqpoint{0.800006in}{1.252883in}}{\pgfqpoint{0.800006in}{1.244646in}}%
\pgfpathcurveto{\pgfqpoint{0.800006in}{1.236410in}}{\pgfqpoint{0.803279in}{1.228510in}}{\pgfqpoint{0.809103in}{1.222686in}}%
\pgfpathcurveto{\pgfqpoint{0.814926in}{1.216862in}}{\pgfqpoint{0.822826in}{1.213590in}}{\pgfqpoint{0.831063in}{1.213590in}}%
\pgfpathclose%
\pgfusepath{stroke,fill}%
\end{pgfscope}%
\begin{pgfscope}%
\pgfpathrectangle{\pgfqpoint{0.100000in}{0.220728in}}{\pgfqpoint{3.696000in}{3.696000in}}%
\pgfusepath{clip}%
\pgfsetbuttcap%
\pgfsetroundjoin%
\definecolor{currentfill}{rgb}{0.121569,0.466667,0.705882}%
\pgfsetfillcolor{currentfill}%
\pgfsetfillopacity{0.616594}%
\pgfsetlinewidth{1.003750pt}%
\definecolor{currentstroke}{rgb}{0.121569,0.466667,0.705882}%
\pgfsetstrokecolor{currentstroke}%
\pgfsetstrokeopacity{0.616594}%
\pgfsetdash{}{0pt}%
\pgfpathmoveto{\pgfqpoint{0.831058in}{1.213583in}}%
\pgfpathcurveto{\pgfqpoint{0.839294in}{1.213583in}}{\pgfqpoint{0.847194in}{1.216855in}}{\pgfqpoint{0.853018in}{1.222679in}}%
\pgfpathcurveto{\pgfqpoint{0.858842in}{1.228503in}}{\pgfqpoint{0.862114in}{1.236403in}}{\pgfqpoint{0.862114in}{1.244639in}}%
\pgfpathcurveto{\pgfqpoint{0.862114in}{1.252876in}}{\pgfqpoint{0.858842in}{1.260776in}}{\pgfqpoint{0.853018in}{1.266600in}}%
\pgfpathcurveto{\pgfqpoint{0.847194in}{1.272424in}}{\pgfqpoint{0.839294in}{1.275696in}}{\pgfqpoint{0.831058in}{1.275696in}}%
\pgfpathcurveto{\pgfqpoint{0.822822in}{1.275696in}}{\pgfqpoint{0.814922in}{1.272424in}}{\pgfqpoint{0.809098in}{1.266600in}}%
\pgfpathcurveto{\pgfqpoint{0.803274in}{1.260776in}}{\pgfqpoint{0.800001in}{1.252876in}}{\pgfqpoint{0.800001in}{1.244639in}}%
\pgfpathcurveto{\pgfqpoint{0.800001in}{1.236403in}}{\pgfqpoint{0.803274in}{1.228503in}}{\pgfqpoint{0.809098in}{1.222679in}}%
\pgfpathcurveto{\pgfqpoint{0.814922in}{1.216855in}}{\pgfqpoint{0.822822in}{1.213583in}}{\pgfqpoint{0.831058in}{1.213583in}}%
\pgfpathclose%
\pgfusepath{stroke,fill}%
\end{pgfscope}%
\begin{pgfscope}%
\pgfpathrectangle{\pgfqpoint{0.100000in}{0.220728in}}{\pgfqpoint{3.696000in}{3.696000in}}%
\pgfusepath{clip}%
\pgfsetbuttcap%
\pgfsetroundjoin%
\definecolor{currentfill}{rgb}{0.121569,0.466667,0.705882}%
\pgfsetfillcolor{currentfill}%
\pgfsetfillopacity{0.616595}%
\pgfsetlinewidth{1.003750pt}%
\definecolor{currentstroke}{rgb}{0.121569,0.466667,0.705882}%
\pgfsetstrokecolor{currentstroke}%
\pgfsetstrokeopacity{0.616595}%
\pgfsetdash{}{0pt}%
\pgfpathmoveto{\pgfqpoint{0.831055in}{1.213579in}}%
\pgfpathcurveto{\pgfqpoint{0.839292in}{1.213579in}}{\pgfqpoint{0.847192in}{1.216851in}}{\pgfqpoint{0.853016in}{1.222675in}}%
\pgfpathcurveto{\pgfqpoint{0.858840in}{1.228499in}}{\pgfqpoint{0.862112in}{1.236399in}}{\pgfqpoint{0.862112in}{1.244636in}}%
\pgfpathcurveto{\pgfqpoint{0.862112in}{1.252872in}}{\pgfqpoint{0.858840in}{1.260772in}}{\pgfqpoint{0.853016in}{1.266596in}}%
\pgfpathcurveto{\pgfqpoint{0.847192in}{1.272420in}}{\pgfqpoint{0.839292in}{1.275692in}}{\pgfqpoint{0.831055in}{1.275692in}}%
\pgfpathcurveto{\pgfqpoint{0.822819in}{1.275692in}}{\pgfqpoint{0.814919in}{1.272420in}}{\pgfqpoint{0.809095in}{1.266596in}}%
\pgfpathcurveto{\pgfqpoint{0.803271in}{1.260772in}}{\pgfqpoint{0.799999in}{1.252872in}}{\pgfqpoint{0.799999in}{1.244636in}}%
\pgfpathcurveto{\pgfqpoint{0.799999in}{1.236399in}}{\pgfqpoint{0.803271in}{1.228499in}}{\pgfqpoint{0.809095in}{1.222675in}}%
\pgfpathcurveto{\pgfqpoint{0.814919in}{1.216851in}}{\pgfqpoint{0.822819in}{1.213579in}}{\pgfqpoint{0.831055in}{1.213579in}}%
\pgfpathclose%
\pgfusepath{stroke,fill}%
\end{pgfscope}%
\begin{pgfscope}%
\pgfpathrectangle{\pgfqpoint{0.100000in}{0.220728in}}{\pgfqpoint{3.696000in}{3.696000in}}%
\pgfusepath{clip}%
\pgfsetbuttcap%
\pgfsetroundjoin%
\definecolor{currentfill}{rgb}{0.121569,0.466667,0.705882}%
\pgfsetfillcolor{currentfill}%
\pgfsetfillopacity{0.616595}%
\pgfsetlinewidth{1.003750pt}%
\definecolor{currentstroke}{rgb}{0.121569,0.466667,0.705882}%
\pgfsetstrokecolor{currentstroke}%
\pgfsetstrokeopacity{0.616595}%
\pgfsetdash{}{0pt}%
\pgfpathmoveto{\pgfqpoint{0.831054in}{1.213577in}}%
\pgfpathcurveto{\pgfqpoint{0.839290in}{1.213577in}}{\pgfqpoint{0.847190in}{1.216849in}}{\pgfqpoint{0.853014in}{1.222673in}}%
\pgfpathcurveto{\pgfqpoint{0.858838in}{1.228497in}}{\pgfqpoint{0.862110in}{1.236397in}}{\pgfqpoint{0.862110in}{1.244633in}}%
\pgfpathcurveto{\pgfqpoint{0.862110in}{1.252870in}}{\pgfqpoint{0.858838in}{1.260770in}}{\pgfqpoint{0.853014in}{1.266594in}}%
\pgfpathcurveto{\pgfqpoint{0.847190in}{1.272418in}}{\pgfqpoint{0.839290in}{1.275690in}}{\pgfqpoint{0.831054in}{1.275690in}}%
\pgfpathcurveto{\pgfqpoint{0.822818in}{1.275690in}}{\pgfqpoint{0.814918in}{1.272418in}}{\pgfqpoint{0.809094in}{1.266594in}}%
\pgfpathcurveto{\pgfqpoint{0.803270in}{1.260770in}}{\pgfqpoint{0.799997in}{1.252870in}}{\pgfqpoint{0.799997in}{1.244633in}}%
\pgfpathcurveto{\pgfqpoint{0.799997in}{1.236397in}}{\pgfqpoint{0.803270in}{1.228497in}}{\pgfqpoint{0.809094in}{1.222673in}}%
\pgfpathcurveto{\pgfqpoint{0.814918in}{1.216849in}}{\pgfqpoint{0.822818in}{1.213577in}}{\pgfqpoint{0.831054in}{1.213577in}}%
\pgfpathclose%
\pgfusepath{stroke,fill}%
\end{pgfscope}%
\begin{pgfscope}%
\pgfpathrectangle{\pgfqpoint{0.100000in}{0.220728in}}{\pgfqpoint{3.696000in}{3.696000in}}%
\pgfusepath{clip}%
\pgfsetbuttcap%
\pgfsetroundjoin%
\definecolor{currentfill}{rgb}{0.121569,0.466667,0.705882}%
\pgfsetfillcolor{currentfill}%
\pgfsetfillopacity{0.616595}%
\pgfsetlinewidth{1.003750pt}%
\definecolor{currentstroke}{rgb}{0.121569,0.466667,0.705882}%
\pgfsetstrokecolor{currentstroke}%
\pgfsetstrokeopacity{0.616595}%
\pgfsetdash{}{0pt}%
\pgfpathmoveto{\pgfqpoint{0.831053in}{1.213576in}}%
\pgfpathcurveto{\pgfqpoint{0.839289in}{1.213576in}}{\pgfqpoint{0.847190in}{1.216848in}}{\pgfqpoint{0.853013in}{1.222672in}}%
\pgfpathcurveto{\pgfqpoint{0.858837in}{1.228496in}}{\pgfqpoint{0.862110in}{1.236396in}}{\pgfqpoint{0.862110in}{1.244632in}}%
\pgfpathcurveto{\pgfqpoint{0.862110in}{1.252868in}}{\pgfqpoint{0.858837in}{1.260768in}}{\pgfqpoint{0.853013in}{1.266592in}}%
\pgfpathcurveto{\pgfqpoint{0.847190in}{1.272416in}}{\pgfqpoint{0.839289in}{1.275689in}}{\pgfqpoint{0.831053in}{1.275689in}}%
\pgfpathcurveto{\pgfqpoint{0.822817in}{1.275689in}}{\pgfqpoint{0.814917in}{1.272416in}}{\pgfqpoint{0.809093in}{1.266592in}}%
\pgfpathcurveto{\pgfqpoint{0.803269in}{1.260768in}}{\pgfqpoint{0.799997in}{1.252868in}}{\pgfqpoint{0.799997in}{1.244632in}}%
\pgfpathcurveto{\pgfqpoint{0.799997in}{1.236396in}}{\pgfqpoint{0.803269in}{1.228496in}}{\pgfqpoint{0.809093in}{1.222672in}}%
\pgfpathcurveto{\pgfqpoint{0.814917in}{1.216848in}}{\pgfqpoint{0.822817in}{1.213576in}}{\pgfqpoint{0.831053in}{1.213576in}}%
\pgfpathclose%
\pgfusepath{stroke,fill}%
\end{pgfscope}%
\begin{pgfscope}%
\pgfpathrectangle{\pgfqpoint{0.100000in}{0.220728in}}{\pgfqpoint{3.696000in}{3.696000in}}%
\pgfusepath{clip}%
\pgfsetbuttcap%
\pgfsetroundjoin%
\definecolor{currentfill}{rgb}{0.121569,0.466667,0.705882}%
\pgfsetfillcolor{currentfill}%
\pgfsetfillopacity{0.616596}%
\pgfsetlinewidth{1.003750pt}%
\definecolor{currentstroke}{rgb}{0.121569,0.466667,0.705882}%
\pgfsetstrokecolor{currentstroke}%
\pgfsetstrokeopacity{0.616596}%
\pgfsetdash{}{0pt}%
\pgfpathmoveto{\pgfqpoint{0.831053in}{1.213575in}}%
\pgfpathcurveto{\pgfqpoint{0.839289in}{1.213575in}}{\pgfqpoint{0.847189in}{1.216847in}}{\pgfqpoint{0.853013in}{1.222671in}}%
\pgfpathcurveto{\pgfqpoint{0.858837in}{1.228495in}}{\pgfqpoint{0.862109in}{1.236395in}}{\pgfqpoint{0.862109in}{1.244631in}}%
\pgfpathcurveto{\pgfqpoint{0.862109in}{1.252868in}}{\pgfqpoint{0.858837in}{1.260768in}}{\pgfqpoint{0.853013in}{1.266592in}}%
\pgfpathcurveto{\pgfqpoint{0.847189in}{1.272416in}}{\pgfqpoint{0.839289in}{1.275688in}}{\pgfqpoint{0.831053in}{1.275688in}}%
\pgfpathcurveto{\pgfqpoint{0.822816in}{1.275688in}}{\pgfqpoint{0.814916in}{1.272416in}}{\pgfqpoint{0.809092in}{1.266592in}}%
\pgfpathcurveto{\pgfqpoint{0.803269in}{1.260768in}}{\pgfqpoint{0.799996in}{1.252868in}}{\pgfqpoint{0.799996in}{1.244631in}}%
\pgfpathcurveto{\pgfqpoint{0.799996in}{1.236395in}}{\pgfqpoint{0.803269in}{1.228495in}}{\pgfqpoint{0.809092in}{1.222671in}}%
\pgfpathcurveto{\pgfqpoint{0.814916in}{1.216847in}}{\pgfqpoint{0.822816in}{1.213575in}}{\pgfqpoint{0.831053in}{1.213575in}}%
\pgfpathclose%
\pgfusepath{stroke,fill}%
\end{pgfscope}%
\begin{pgfscope}%
\pgfpathrectangle{\pgfqpoint{0.100000in}{0.220728in}}{\pgfqpoint{3.696000in}{3.696000in}}%
\pgfusepath{clip}%
\pgfsetbuttcap%
\pgfsetroundjoin%
\definecolor{currentfill}{rgb}{0.121569,0.466667,0.705882}%
\pgfsetfillcolor{currentfill}%
\pgfsetfillopacity{0.616596}%
\pgfsetlinewidth{1.003750pt}%
\definecolor{currentstroke}{rgb}{0.121569,0.466667,0.705882}%
\pgfsetstrokecolor{currentstroke}%
\pgfsetstrokeopacity{0.616596}%
\pgfsetdash{}{0pt}%
\pgfpathmoveto{\pgfqpoint{0.831053in}{1.213575in}}%
\pgfpathcurveto{\pgfqpoint{0.839289in}{1.213575in}}{\pgfqpoint{0.847189in}{1.216847in}}{\pgfqpoint{0.853013in}{1.222671in}}%
\pgfpathcurveto{\pgfqpoint{0.858837in}{1.228495in}}{\pgfqpoint{0.862109in}{1.236395in}}{\pgfqpoint{0.862109in}{1.244631in}}%
\pgfpathcurveto{\pgfqpoint{0.862109in}{1.252867in}}{\pgfqpoint{0.858837in}{1.260767in}}{\pgfqpoint{0.853013in}{1.266591in}}%
\pgfpathcurveto{\pgfqpoint{0.847189in}{1.272415in}}{\pgfqpoint{0.839289in}{1.275688in}}{\pgfqpoint{0.831053in}{1.275688in}}%
\pgfpathcurveto{\pgfqpoint{0.822816in}{1.275688in}}{\pgfqpoint{0.814916in}{1.272415in}}{\pgfqpoint{0.809092in}{1.266591in}}%
\pgfpathcurveto{\pgfqpoint{0.803268in}{1.260767in}}{\pgfqpoint{0.799996in}{1.252867in}}{\pgfqpoint{0.799996in}{1.244631in}}%
\pgfpathcurveto{\pgfqpoint{0.799996in}{1.236395in}}{\pgfqpoint{0.803268in}{1.228495in}}{\pgfqpoint{0.809092in}{1.222671in}}%
\pgfpathcurveto{\pgfqpoint{0.814916in}{1.216847in}}{\pgfqpoint{0.822816in}{1.213575in}}{\pgfqpoint{0.831053in}{1.213575in}}%
\pgfpathclose%
\pgfusepath{stroke,fill}%
\end{pgfscope}%
\begin{pgfscope}%
\pgfpathrectangle{\pgfqpoint{0.100000in}{0.220728in}}{\pgfqpoint{3.696000in}{3.696000in}}%
\pgfusepath{clip}%
\pgfsetbuttcap%
\pgfsetroundjoin%
\definecolor{currentfill}{rgb}{0.121569,0.466667,0.705882}%
\pgfsetfillcolor{currentfill}%
\pgfsetfillopacity{0.616596}%
\pgfsetlinewidth{1.003750pt}%
\definecolor{currentstroke}{rgb}{0.121569,0.466667,0.705882}%
\pgfsetstrokecolor{currentstroke}%
\pgfsetstrokeopacity{0.616596}%
\pgfsetdash{}{0pt}%
\pgfpathmoveto{\pgfqpoint{0.831052in}{1.213574in}}%
\pgfpathcurveto{\pgfqpoint{0.839289in}{1.213574in}}{\pgfqpoint{0.847189in}{1.216847in}}{\pgfqpoint{0.853013in}{1.222671in}}%
\pgfpathcurveto{\pgfqpoint{0.858837in}{1.228495in}}{\pgfqpoint{0.862109in}{1.236395in}}{\pgfqpoint{0.862109in}{1.244631in}}%
\pgfpathcurveto{\pgfqpoint{0.862109in}{1.252867in}}{\pgfqpoint{0.858837in}{1.260767in}}{\pgfqpoint{0.853013in}{1.266591in}}%
\pgfpathcurveto{\pgfqpoint{0.847189in}{1.272415in}}{\pgfqpoint{0.839289in}{1.275687in}}{\pgfqpoint{0.831052in}{1.275687in}}%
\pgfpathcurveto{\pgfqpoint{0.822816in}{1.275687in}}{\pgfqpoint{0.814916in}{1.272415in}}{\pgfqpoint{0.809092in}{1.266591in}}%
\pgfpathcurveto{\pgfqpoint{0.803268in}{1.260767in}}{\pgfqpoint{0.799996in}{1.252867in}}{\pgfqpoint{0.799996in}{1.244631in}}%
\pgfpathcurveto{\pgfqpoint{0.799996in}{1.236395in}}{\pgfqpoint{0.803268in}{1.228495in}}{\pgfqpoint{0.809092in}{1.222671in}}%
\pgfpathcurveto{\pgfqpoint{0.814916in}{1.216847in}}{\pgfqpoint{0.822816in}{1.213574in}}{\pgfqpoint{0.831052in}{1.213574in}}%
\pgfpathclose%
\pgfusepath{stroke,fill}%
\end{pgfscope}%
\begin{pgfscope}%
\pgfpathrectangle{\pgfqpoint{0.100000in}{0.220728in}}{\pgfqpoint{3.696000in}{3.696000in}}%
\pgfusepath{clip}%
\pgfsetbuttcap%
\pgfsetroundjoin%
\definecolor{currentfill}{rgb}{0.121569,0.466667,0.705882}%
\pgfsetfillcolor{currentfill}%
\pgfsetfillopacity{0.616596}%
\pgfsetlinewidth{1.003750pt}%
\definecolor{currentstroke}{rgb}{0.121569,0.466667,0.705882}%
\pgfsetstrokecolor{currentstroke}%
\pgfsetstrokeopacity{0.616596}%
\pgfsetdash{}{0pt}%
\pgfpathmoveto{\pgfqpoint{0.831052in}{1.213574in}}%
\pgfpathcurveto{\pgfqpoint{0.839289in}{1.213574in}}{\pgfqpoint{0.847189in}{1.216847in}}{\pgfqpoint{0.853013in}{1.222671in}}%
\pgfpathcurveto{\pgfqpoint{0.858836in}{1.228494in}}{\pgfqpoint{0.862109in}{1.236394in}}{\pgfqpoint{0.862109in}{1.244631in}}%
\pgfpathcurveto{\pgfqpoint{0.862109in}{1.252867in}}{\pgfqpoint{0.858836in}{1.260767in}}{\pgfqpoint{0.853013in}{1.266591in}}%
\pgfpathcurveto{\pgfqpoint{0.847189in}{1.272415in}}{\pgfqpoint{0.839289in}{1.275687in}}{\pgfqpoint{0.831052in}{1.275687in}}%
\pgfpathcurveto{\pgfqpoint{0.822816in}{1.275687in}}{\pgfqpoint{0.814916in}{1.272415in}}{\pgfqpoint{0.809092in}{1.266591in}}%
\pgfpathcurveto{\pgfqpoint{0.803268in}{1.260767in}}{\pgfqpoint{0.799996in}{1.252867in}}{\pgfqpoint{0.799996in}{1.244631in}}%
\pgfpathcurveto{\pgfqpoint{0.799996in}{1.236394in}}{\pgfqpoint{0.803268in}{1.228494in}}{\pgfqpoint{0.809092in}{1.222671in}}%
\pgfpathcurveto{\pgfqpoint{0.814916in}{1.216847in}}{\pgfqpoint{0.822816in}{1.213574in}}{\pgfqpoint{0.831052in}{1.213574in}}%
\pgfpathclose%
\pgfusepath{stroke,fill}%
\end{pgfscope}%
\begin{pgfscope}%
\pgfpathrectangle{\pgfqpoint{0.100000in}{0.220728in}}{\pgfqpoint{3.696000in}{3.696000in}}%
\pgfusepath{clip}%
\pgfsetbuttcap%
\pgfsetroundjoin%
\definecolor{currentfill}{rgb}{0.121569,0.466667,0.705882}%
\pgfsetfillcolor{currentfill}%
\pgfsetfillopacity{0.616596}%
\pgfsetlinewidth{1.003750pt}%
\definecolor{currentstroke}{rgb}{0.121569,0.466667,0.705882}%
\pgfsetstrokecolor{currentstroke}%
\pgfsetstrokeopacity{0.616596}%
\pgfsetdash{}{0pt}%
\pgfpathmoveto{\pgfqpoint{0.831052in}{1.213574in}}%
\pgfpathcurveto{\pgfqpoint{0.839289in}{1.213574in}}{\pgfqpoint{0.847189in}{1.216847in}}{\pgfqpoint{0.853013in}{1.222670in}}%
\pgfpathcurveto{\pgfqpoint{0.858836in}{1.228494in}}{\pgfqpoint{0.862109in}{1.236394in}}{\pgfqpoint{0.862109in}{1.244631in}}%
\pgfpathcurveto{\pgfqpoint{0.862109in}{1.252867in}}{\pgfqpoint{0.858836in}{1.260767in}}{\pgfqpoint{0.853013in}{1.266591in}}%
\pgfpathcurveto{\pgfqpoint{0.847189in}{1.272415in}}{\pgfqpoint{0.839289in}{1.275687in}}{\pgfqpoint{0.831052in}{1.275687in}}%
\pgfpathcurveto{\pgfqpoint{0.822816in}{1.275687in}}{\pgfqpoint{0.814916in}{1.272415in}}{\pgfqpoint{0.809092in}{1.266591in}}%
\pgfpathcurveto{\pgfqpoint{0.803268in}{1.260767in}}{\pgfqpoint{0.799996in}{1.252867in}}{\pgfqpoint{0.799996in}{1.244631in}}%
\pgfpathcurveto{\pgfqpoint{0.799996in}{1.236394in}}{\pgfqpoint{0.803268in}{1.228494in}}{\pgfqpoint{0.809092in}{1.222670in}}%
\pgfpathcurveto{\pgfqpoint{0.814916in}{1.216847in}}{\pgfqpoint{0.822816in}{1.213574in}}{\pgfqpoint{0.831052in}{1.213574in}}%
\pgfpathclose%
\pgfusepath{stroke,fill}%
\end{pgfscope}%
\begin{pgfscope}%
\pgfpathrectangle{\pgfqpoint{0.100000in}{0.220728in}}{\pgfqpoint{3.696000in}{3.696000in}}%
\pgfusepath{clip}%
\pgfsetbuttcap%
\pgfsetroundjoin%
\definecolor{currentfill}{rgb}{0.121569,0.466667,0.705882}%
\pgfsetfillcolor{currentfill}%
\pgfsetfillopacity{0.616596}%
\pgfsetlinewidth{1.003750pt}%
\definecolor{currentstroke}{rgb}{0.121569,0.466667,0.705882}%
\pgfsetstrokecolor{currentstroke}%
\pgfsetstrokeopacity{0.616596}%
\pgfsetdash{}{0pt}%
\pgfpathmoveto{\pgfqpoint{0.831052in}{1.213574in}}%
\pgfpathcurveto{\pgfqpoint{0.839289in}{1.213574in}}{\pgfqpoint{0.847189in}{1.216846in}}{\pgfqpoint{0.853012in}{1.222670in}}%
\pgfpathcurveto{\pgfqpoint{0.858836in}{1.228494in}}{\pgfqpoint{0.862109in}{1.236394in}}{\pgfqpoint{0.862109in}{1.244631in}}%
\pgfpathcurveto{\pgfqpoint{0.862109in}{1.252867in}}{\pgfqpoint{0.858836in}{1.260767in}}{\pgfqpoint{0.853012in}{1.266591in}}%
\pgfpathcurveto{\pgfqpoint{0.847189in}{1.272415in}}{\pgfqpoint{0.839289in}{1.275687in}}{\pgfqpoint{0.831052in}{1.275687in}}%
\pgfpathcurveto{\pgfqpoint{0.822816in}{1.275687in}}{\pgfqpoint{0.814916in}{1.272415in}}{\pgfqpoint{0.809092in}{1.266591in}}%
\pgfpathcurveto{\pgfqpoint{0.803268in}{1.260767in}}{\pgfqpoint{0.799996in}{1.252867in}}{\pgfqpoint{0.799996in}{1.244631in}}%
\pgfpathcurveto{\pgfqpoint{0.799996in}{1.236394in}}{\pgfqpoint{0.803268in}{1.228494in}}{\pgfqpoint{0.809092in}{1.222670in}}%
\pgfpathcurveto{\pgfqpoint{0.814916in}{1.216846in}}{\pgfqpoint{0.822816in}{1.213574in}}{\pgfqpoint{0.831052in}{1.213574in}}%
\pgfpathclose%
\pgfusepath{stroke,fill}%
\end{pgfscope}%
\begin{pgfscope}%
\pgfpathrectangle{\pgfqpoint{0.100000in}{0.220728in}}{\pgfqpoint{3.696000in}{3.696000in}}%
\pgfusepath{clip}%
\pgfsetbuttcap%
\pgfsetroundjoin%
\definecolor{currentfill}{rgb}{0.121569,0.466667,0.705882}%
\pgfsetfillcolor{currentfill}%
\pgfsetfillopacity{0.616596}%
\pgfsetlinewidth{1.003750pt}%
\definecolor{currentstroke}{rgb}{0.121569,0.466667,0.705882}%
\pgfsetstrokecolor{currentstroke}%
\pgfsetstrokeopacity{0.616596}%
\pgfsetdash{}{0pt}%
\pgfpathmoveto{\pgfqpoint{0.831052in}{1.213574in}}%
\pgfpathcurveto{\pgfqpoint{0.839289in}{1.213574in}}{\pgfqpoint{0.847189in}{1.216846in}}{\pgfqpoint{0.853012in}{1.222670in}}%
\pgfpathcurveto{\pgfqpoint{0.858836in}{1.228494in}}{\pgfqpoint{0.862109in}{1.236394in}}{\pgfqpoint{0.862109in}{1.244631in}}%
\pgfpathcurveto{\pgfqpoint{0.862109in}{1.252867in}}{\pgfqpoint{0.858836in}{1.260767in}}{\pgfqpoint{0.853012in}{1.266591in}}%
\pgfpathcurveto{\pgfqpoint{0.847189in}{1.272415in}}{\pgfqpoint{0.839289in}{1.275687in}}{\pgfqpoint{0.831052in}{1.275687in}}%
\pgfpathcurveto{\pgfqpoint{0.822816in}{1.275687in}}{\pgfqpoint{0.814916in}{1.272415in}}{\pgfqpoint{0.809092in}{1.266591in}}%
\pgfpathcurveto{\pgfqpoint{0.803268in}{1.260767in}}{\pgfqpoint{0.799996in}{1.252867in}}{\pgfqpoint{0.799996in}{1.244631in}}%
\pgfpathcurveto{\pgfqpoint{0.799996in}{1.236394in}}{\pgfqpoint{0.803268in}{1.228494in}}{\pgfqpoint{0.809092in}{1.222670in}}%
\pgfpathcurveto{\pgfqpoint{0.814916in}{1.216846in}}{\pgfqpoint{0.822816in}{1.213574in}}{\pgfqpoint{0.831052in}{1.213574in}}%
\pgfpathclose%
\pgfusepath{stroke,fill}%
\end{pgfscope}%
\begin{pgfscope}%
\pgfpathrectangle{\pgfqpoint{0.100000in}{0.220728in}}{\pgfqpoint{3.696000in}{3.696000in}}%
\pgfusepath{clip}%
\pgfsetbuttcap%
\pgfsetroundjoin%
\definecolor{currentfill}{rgb}{0.121569,0.466667,0.705882}%
\pgfsetfillcolor{currentfill}%
\pgfsetfillopacity{0.616596}%
\pgfsetlinewidth{1.003750pt}%
\definecolor{currentstroke}{rgb}{0.121569,0.466667,0.705882}%
\pgfsetstrokecolor{currentstroke}%
\pgfsetstrokeopacity{0.616596}%
\pgfsetdash{}{0pt}%
\pgfpathmoveto{\pgfqpoint{0.831052in}{1.213574in}}%
\pgfpathcurveto{\pgfqpoint{0.839288in}{1.213574in}}{\pgfqpoint{0.847189in}{1.216846in}}{\pgfqpoint{0.853012in}{1.222670in}}%
\pgfpathcurveto{\pgfqpoint{0.858836in}{1.228494in}}{\pgfqpoint{0.862109in}{1.236394in}}{\pgfqpoint{0.862109in}{1.244631in}}%
\pgfpathcurveto{\pgfqpoint{0.862109in}{1.252867in}}{\pgfqpoint{0.858836in}{1.260767in}}{\pgfqpoint{0.853012in}{1.266591in}}%
\pgfpathcurveto{\pgfqpoint{0.847189in}{1.272415in}}{\pgfqpoint{0.839288in}{1.275687in}}{\pgfqpoint{0.831052in}{1.275687in}}%
\pgfpathcurveto{\pgfqpoint{0.822816in}{1.275687in}}{\pgfqpoint{0.814916in}{1.272415in}}{\pgfqpoint{0.809092in}{1.266591in}}%
\pgfpathcurveto{\pgfqpoint{0.803268in}{1.260767in}}{\pgfqpoint{0.799996in}{1.252867in}}{\pgfqpoint{0.799996in}{1.244631in}}%
\pgfpathcurveto{\pgfqpoint{0.799996in}{1.236394in}}{\pgfqpoint{0.803268in}{1.228494in}}{\pgfqpoint{0.809092in}{1.222670in}}%
\pgfpathcurveto{\pgfqpoint{0.814916in}{1.216846in}}{\pgfqpoint{0.822816in}{1.213574in}}{\pgfqpoint{0.831052in}{1.213574in}}%
\pgfpathclose%
\pgfusepath{stroke,fill}%
\end{pgfscope}%
\begin{pgfscope}%
\pgfpathrectangle{\pgfqpoint{0.100000in}{0.220728in}}{\pgfqpoint{3.696000in}{3.696000in}}%
\pgfusepath{clip}%
\pgfsetbuttcap%
\pgfsetroundjoin%
\definecolor{currentfill}{rgb}{0.121569,0.466667,0.705882}%
\pgfsetfillcolor{currentfill}%
\pgfsetfillopacity{0.616596}%
\pgfsetlinewidth{1.003750pt}%
\definecolor{currentstroke}{rgb}{0.121569,0.466667,0.705882}%
\pgfsetstrokecolor{currentstroke}%
\pgfsetstrokeopacity{0.616596}%
\pgfsetdash{}{0pt}%
\pgfpathmoveto{\pgfqpoint{0.831052in}{1.213574in}}%
\pgfpathcurveto{\pgfqpoint{0.839288in}{1.213574in}}{\pgfqpoint{0.847189in}{1.216846in}}{\pgfqpoint{0.853012in}{1.222670in}}%
\pgfpathcurveto{\pgfqpoint{0.858836in}{1.228494in}}{\pgfqpoint{0.862109in}{1.236394in}}{\pgfqpoint{0.862109in}{1.244631in}}%
\pgfpathcurveto{\pgfqpoint{0.862109in}{1.252867in}}{\pgfqpoint{0.858836in}{1.260767in}}{\pgfqpoint{0.853012in}{1.266591in}}%
\pgfpathcurveto{\pgfqpoint{0.847189in}{1.272415in}}{\pgfqpoint{0.839288in}{1.275687in}}{\pgfqpoint{0.831052in}{1.275687in}}%
\pgfpathcurveto{\pgfqpoint{0.822816in}{1.275687in}}{\pgfqpoint{0.814916in}{1.272415in}}{\pgfqpoint{0.809092in}{1.266591in}}%
\pgfpathcurveto{\pgfqpoint{0.803268in}{1.260767in}}{\pgfqpoint{0.799996in}{1.252867in}}{\pgfqpoint{0.799996in}{1.244631in}}%
\pgfpathcurveto{\pgfqpoint{0.799996in}{1.236394in}}{\pgfqpoint{0.803268in}{1.228494in}}{\pgfqpoint{0.809092in}{1.222670in}}%
\pgfpathcurveto{\pgfqpoint{0.814916in}{1.216846in}}{\pgfqpoint{0.822816in}{1.213574in}}{\pgfqpoint{0.831052in}{1.213574in}}%
\pgfpathclose%
\pgfusepath{stroke,fill}%
\end{pgfscope}%
\begin{pgfscope}%
\pgfpathrectangle{\pgfqpoint{0.100000in}{0.220728in}}{\pgfqpoint{3.696000in}{3.696000in}}%
\pgfusepath{clip}%
\pgfsetbuttcap%
\pgfsetroundjoin%
\definecolor{currentfill}{rgb}{0.121569,0.466667,0.705882}%
\pgfsetfillcolor{currentfill}%
\pgfsetfillopacity{0.616596}%
\pgfsetlinewidth{1.003750pt}%
\definecolor{currentstroke}{rgb}{0.121569,0.466667,0.705882}%
\pgfsetstrokecolor{currentstroke}%
\pgfsetstrokeopacity{0.616596}%
\pgfsetdash{}{0pt}%
\pgfpathmoveto{\pgfqpoint{0.831052in}{1.213574in}}%
\pgfpathcurveto{\pgfqpoint{0.839288in}{1.213574in}}{\pgfqpoint{0.847189in}{1.216846in}}{\pgfqpoint{0.853012in}{1.222670in}}%
\pgfpathcurveto{\pgfqpoint{0.858836in}{1.228494in}}{\pgfqpoint{0.862109in}{1.236394in}}{\pgfqpoint{0.862109in}{1.244631in}}%
\pgfpathcurveto{\pgfqpoint{0.862109in}{1.252867in}}{\pgfqpoint{0.858836in}{1.260767in}}{\pgfqpoint{0.853012in}{1.266591in}}%
\pgfpathcurveto{\pgfqpoint{0.847189in}{1.272415in}}{\pgfqpoint{0.839288in}{1.275687in}}{\pgfqpoint{0.831052in}{1.275687in}}%
\pgfpathcurveto{\pgfqpoint{0.822816in}{1.275687in}}{\pgfqpoint{0.814916in}{1.272415in}}{\pgfqpoint{0.809092in}{1.266591in}}%
\pgfpathcurveto{\pgfqpoint{0.803268in}{1.260767in}}{\pgfqpoint{0.799996in}{1.252867in}}{\pgfqpoint{0.799996in}{1.244631in}}%
\pgfpathcurveto{\pgfqpoint{0.799996in}{1.236394in}}{\pgfqpoint{0.803268in}{1.228494in}}{\pgfqpoint{0.809092in}{1.222670in}}%
\pgfpathcurveto{\pgfqpoint{0.814916in}{1.216846in}}{\pgfqpoint{0.822816in}{1.213574in}}{\pgfqpoint{0.831052in}{1.213574in}}%
\pgfpathclose%
\pgfusepath{stroke,fill}%
\end{pgfscope}%
\begin{pgfscope}%
\pgfpathrectangle{\pgfqpoint{0.100000in}{0.220728in}}{\pgfqpoint{3.696000in}{3.696000in}}%
\pgfusepath{clip}%
\pgfsetbuttcap%
\pgfsetroundjoin%
\definecolor{currentfill}{rgb}{0.121569,0.466667,0.705882}%
\pgfsetfillcolor{currentfill}%
\pgfsetfillopacity{0.616596}%
\pgfsetlinewidth{1.003750pt}%
\definecolor{currentstroke}{rgb}{0.121569,0.466667,0.705882}%
\pgfsetstrokecolor{currentstroke}%
\pgfsetstrokeopacity{0.616596}%
\pgfsetdash{}{0pt}%
\pgfpathmoveto{\pgfqpoint{0.831052in}{1.213574in}}%
\pgfpathcurveto{\pgfqpoint{0.839288in}{1.213574in}}{\pgfqpoint{0.847189in}{1.216846in}}{\pgfqpoint{0.853012in}{1.222670in}}%
\pgfpathcurveto{\pgfqpoint{0.858836in}{1.228494in}}{\pgfqpoint{0.862109in}{1.236394in}}{\pgfqpoint{0.862109in}{1.244631in}}%
\pgfpathcurveto{\pgfqpoint{0.862109in}{1.252867in}}{\pgfqpoint{0.858836in}{1.260767in}}{\pgfqpoint{0.853012in}{1.266591in}}%
\pgfpathcurveto{\pgfqpoint{0.847189in}{1.272415in}}{\pgfqpoint{0.839288in}{1.275687in}}{\pgfqpoint{0.831052in}{1.275687in}}%
\pgfpathcurveto{\pgfqpoint{0.822816in}{1.275687in}}{\pgfqpoint{0.814916in}{1.272415in}}{\pgfqpoint{0.809092in}{1.266591in}}%
\pgfpathcurveto{\pgfqpoint{0.803268in}{1.260767in}}{\pgfqpoint{0.799996in}{1.252867in}}{\pgfqpoint{0.799996in}{1.244631in}}%
\pgfpathcurveto{\pgfqpoint{0.799996in}{1.236394in}}{\pgfqpoint{0.803268in}{1.228494in}}{\pgfqpoint{0.809092in}{1.222670in}}%
\pgfpathcurveto{\pgfqpoint{0.814916in}{1.216846in}}{\pgfqpoint{0.822816in}{1.213574in}}{\pgfqpoint{0.831052in}{1.213574in}}%
\pgfpathclose%
\pgfusepath{stroke,fill}%
\end{pgfscope}%
\begin{pgfscope}%
\pgfpathrectangle{\pgfqpoint{0.100000in}{0.220728in}}{\pgfqpoint{3.696000in}{3.696000in}}%
\pgfusepath{clip}%
\pgfsetbuttcap%
\pgfsetroundjoin%
\definecolor{currentfill}{rgb}{0.121569,0.466667,0.705882}%
\pgfsetfillcolor{currentfill}%
\pgfsetfillopacity{0.616596}%
\pgfsetlinewidth{1.003750pt}%
\definecolor{currentstroke}{rgb}{0.121569,0.466667,0.705882}%
\pgfsetstrokecolor{currentstroke}%
\pgfsetstrokeopacity{0.616596}%
\pgfsetdash{}{0pt}%
\pgfpathmoveto{\pgfqpoint{0.831052in}{1.213574in}}%
\pgfpathcurveto{\pgfqpoint{0.839288in}{1.213574in}}{\pgfqpoint{0.847189in}{1.216846in}}{\pgfqpoint{0.853012in}{1.222670in}}%
\pgfpathcurveto{\pgfqpoint{0.858836in}{1.228494in}}{\pgfqpoint{0.862109in}{1.236394in}}{\pgfqpoint{0.862109in}{1.244631in}}%
\pgfpathcurveto{\pgfqpoint{0.862109in}{1.252867in}}{\pgfqpoint{0.858836in}{1.260767in}}{\pgfqpoint{0.853012in}{1.266591in}}%
\pgfpathcurveto{\pgfqpoint{0.847189in}{1.272415in}}{\pgfqpoint{0.839288in}{1.275687in}}{\pgfqpoint{0.831052in}{1.275687in}}%
\pgfpathcurveto{\pgfqpoint{0.822816in}{1.275687in}}{\pgfqpoint{0.814916in}{1.272415in}}{\pgfqpoint{0.809092in}{1.266591in}}%
\pgfpathcurveto{\pgfqpoint{0.803268in}{1.260767in}}{\pgfqpoint{0.799996in}{1.252867in}}{\pgfqpoint{0.799996in}{1.244631in}}%
\pgfpathcurveto{\pgfqpoint{0.799996in}{1.236394in}}{\pgfqpoint{0.803268in}{1.228494in}}{\pgfqpoint{0.809092in}{1.222670in}}%
\pgfpathcurveto{\pgfqpoint{0.814916in}{1.216846in}}{\pgfqpoint{0.822816in}{1.213574in}}{\pgfqpoint{0.831052in}{1.213574in}}%
\pgfpathclose%
\pgfusepath{stroke,fill}%
\end{pgfscope}%
\begin{pgfscope}%
\pgfpathrectangle{\pgfqpoint{0.100000in}{0.220728in}}{\pgfqpoint{3.696000in}{3.696000in}}%
\pgfusepath{clip}%
\pgfsetbuttcap%
\pgfsetroundjoin%
\definecolor{currentfill}{rgb}{0.121569,0.466667,0.705882}%
\pgfsetfillcolor{currentfill}%
\pgfsetfillopacity{0.616596}%
\pgfsetlinewidth{1.003750pt}%
\definecolor{currentstroke}{rgb}{0.121569,0.466667,0.705882}%
\pgfsetstrokecolor{currentstroke}%
\pgfsetstrokeopacity{0.616596}%
\pgfsetdash{}{0pt}%
\pgfpathmoveto{\pgfqpoint{0.831052in}{1.213574in}}%
\pgfpathcurveto{\pgfqpoint{0.839288in}{1.213574in}}{\pgfqpoint{0.847189in}{1.216846in}}{\pgfqpoint{0.853012in}{1.222670in}}%
\pgfpathcurveto{\pgfqpoint{0.858836in}{1.228494in}}{\pgfqpoint{0.862109in}{1.236394in}}{\pgfqpoint{0.862109in}{1.244631in}}%
\pgfpathcurveto{\pgfqpoint{0.862109in}{1.252867in}}{\pgfqpoint{0.858836in}{1.260767in}}{\pgfqpoint{0.853012in}{1.266591in}}%
\pgfpathcurveto{\pgfqpoint{0.847189in}{1.272415in}}{\pgfqpoint{0.839288in}{1.275687in}}{\pgfqpoint{0.831052in}{1.275687in}}%
\pgfpathcurveto{\pgfqpoint{0.822816in}{1.275687in}}{\pgfqpoint{0.814916in}{1.272415in}}{\pgfqpoint{0.809092in}{1.266591in}}%
\pgfpathcurveto{\pgfqpoint{0.803268in}{1.260767in}}{\pgfqpoint{0.799996in}{1.252867in}}{\pgfqpoint{0.799996in}{1.244631in}}%
\pgfpathcurveto{\pgfqpoint{0.799996in}{1.236394in}}{\pgfqpoint{0.803268in}{1.228494in}}{\pgfqpoint{0.809092in}{1.222670in}}%
\pgfpathcurveto{\pgfqpoint{0.814916in}{1.216846in}}{\pgfqpoint{0.822816in}{1.213574in}}{\pgfqpoint{0.831052in}{1.213574in}}%
\pgfpathclose%
\pgfusepath{stroke,fill}%
\end{pgfscope}%
\begin{pgfscope}%
\pgfpathrectangle{\pgfqpoint{0.100000in}{0.220728in}}{\pgfqpoint{3.696000in}{3.696000in}}%
\pgfusepath{clip}%
\pgfsetbuttcap%
\pgfsetroundjoin%
\definecolor{currentfill}{rgb}{0.121569,0.466667,0.705882}%
\pgfsetfillcolor{currentfill}%
\pgfsetfillopacity{0.616596}%
\pgfsetlinewidth{1.003750pt}%
\definecolor{currentstroke}{rgb}{0.121569,0.466667,0.705882}%
\pgfsetstrokecolor{currentstroke}%
\pgfsetstrokeopacity{0.616596}%
\pgfsetdash{}{0pt}%
\pgfpathmoveto{\pgfqpoint{0.831052in}{1.213574in}}%
\pgfpathcurveto{\pgfqpoint{0.839288in}{1.213574in}}{\pgfqpoint{0.847189in}{1.216846in}}{\pgfqpoint{0.853012in}{1.222670in}}%
\pgfpathcurveto{\pgfqpoint{0.858836in}{1.228494in}}{\pgfqpoint{0.862109in}{1.236394in}}{\pgfqpoint{0.862109in}{1.244631in}}%
\pgfpathcurveto{\pgfqpoint{0.862109in}{1.252867in}}{\pgfqpoint{0.858836in}{1.260767in}}{\pgfqpoint{0.853012in}{1.266591in}}%
\pgfpathcurveto{\pgfqpoint{0.847189in}{1.272415in}}{\pgfqpoint{0.839288in}{1.275687in}}{\pgfqpoint{0.831052in}{1.275687in}}%
\pgfpathcurveto{\pgfqpoint{0.822816in}{1.275687in}}{\pgfqpoint{0.814916in}{1.272415in}}{\pgfqpoint{0.809092in}{1.266591in}}%
\pgfpathcurveto{\pgfqpoint{0.803268in}{1.260767in}}{\pgfqpoint{0.799996in}{1.252867in}}{\pgfqpoint{0.799996in}{1.244631in}}%
\pgfpathcurveto{\pgfqpoint{0.799996in}{1.236394in}}{\pgfqpoint{0.803268in}{1.228494in}}{\pgfqpoint{0.809092in}{1.222670in}}%
\pgfpathcurveto{\pgfqpoint{0.814916in}{1.216846in}}{\pgfqpoint{0.822816in}{1.213574in}}{\pgfqpoint{0.831052in}{1.213574in}}%
\pgfpathclose%
\pgfusepath{stroke,fill}%
\end{pgfscope}%
\begin{pgfscope}%
\pgfpathrectangle{\pgfqpoint{0.100000in}{0.220728in}}{\pgfqpoint{3.696000in}{3.696000in}}%
\pgfusepath{clip}%
\pgfsetbuttcap%
\pgfsetroundjoin%
\definecolor{currentfill}{rgb}{0.121569,0.466667,0.705882}%
\pgfsetfillcolor{currentfill}%
\pgfsetfillopacity{0.616596}%
\pgfsetlinewidth{1.003750pt}%
\definecolor{currentstroke}{rgb}{0.121569,0.466667,0.705882}%
\pgfsetstrokecolor{currentstroke}%
\pgfsetstrokeopacity{0.616596}%
\pgfsetdash{}{0pt}%
\pgfpathmoveto{\pgfqpoint{0.831052in}{1.213574in}}%
\pgfpathcurveto{\pgfqpoint{0.839288in}{1.213574in}}{\pgfqpoint{0.847189in}{1.216846in}}{\pgfqpoint{0.853012in}{1.222670in}}%
\pgfpathcurveto{\pgfqpoint{0.858836in}{1.228494in}}{\pgfqpoint{0.862109in}{1.236394in}}{\pgfqpoint{0.862109in}{1.244631in}}%
\pgfpathcurveto{\pgfqpoint{0.862109in}{1.252867in}}{\pgfqpoint{0.858836in}{1.260767in}}{\pgfqpoint{0.853012in}{1.266591in}}%
\pgfpathcurveto{\pgfqpoint{0.847189in}{1.272415in}}{\pgfqpoint{0.839288in}{1.275687in}}{\pgfqpoint{0.831052in}{1.275687in}}%
\pgfpathcurveto{\pgfqpoint{0.822816in}{1.275687in}}{\pgfqpoint{0.814916in}{1.272415in}}{\pgfqpoint{0.809092in}{1.266591in}}%
\pgfpathcurveto{\pgfqpoint{0.803268in}{1.260767in}}{\pgfqpoint{0.799996in}{1.252867in}}{\pgfqpoint{0.799996in}{1.244631in}}%
\pgfpathcurveto{\pgfqpoint{0.799996in}{1.236394in}}{\pgfqpoint{0.803268in}{1.228494in}}{\pgfqpoint{0.809092in}{1.222670in}}%
\pgfpathcurveto{\pgfqpoint{0.814916in}{1.216846in}}{\pgfqpoint{0.822816in}{1.213574in}}{\pgfqpoint{0.831052in}{1.213574in}}%
\pgfpathclose%
\pgfusepath{stroke,fill}%
\end{pgfscope}%
\begin{pgfscope}%
\pgfpathrectangle{\pgfqpoint{0.100000in}{0.220728in}}{\pgfqpoint{3.696000in}{3.696000in}}%
\pgfusepath{clip}%
\pgfsetbuttcap%
\pgfsetroundjoin%
\definecolor{currentfill}{rgb}{0.121569,0.466667,0.705882}%
\pgfsetfillcolor{currentfill}%
\pgfsetfillopacity{0.616596}%
\pgfsetlinewidth{1.003750pt}%
\definecolor{currentstroke}{rgb}{0.121569,0.466667,0.705882}%
\pgfsetstrokecolor{currentstroke}%
\pgfsetstrokeopacity{0.616596}%
\pgfsetdash{}{0pt}%
\pgfpathmoveto{\pgfqpoint{0.831052in}{1.213574in}}%
\pgfpathcurveto{\pgfqpoint{0.839288in}{1.213574in}}{\pgfqpoint{0.847189in}{1.216846in}}{\pgfqpoint{0.853012in}{1.222670in}}%
\pgfpathcurveto{\pgfqpoint{0.858836in}{1.228494in}}{\pgfqpoint{0.862109in}{1.236394in}}{\pgfqpoint{0.862109in}{1.244631in}}%
\pgfpathcurveto{\pgfqpoint{0.862109in}{1.252867in}}{\pgfqpoint{0.858836in}{1.260767in}}{\pgfqpoint{0.853012in}{1.266591in}}%
\pgfpathcurveto{\pgfqpoint{0.847189in}{1.272415in}}{\pgfqpoint{0.839288in}{1.275687in}}{\pgfqpoint{0.831052in}{1.275687in}}%
\pgfpathcurveto{\pgfqpoint{0.822816in}{1.275687in}}{\pgfqpoint{0.814916in}{1.272415in}}{\pgfqpoint{0.809092in}{1.266591in}}%
\pgfpathcurveto{\pgfqpoint{0.803268in}{1.260767in}}{\pgfqpoint{0.799996in}{1.252867in}}{\pgfqpoint{0.799996in}{1.244631in}}%
\pgfpathcurveto{\pgfqpoint{0.799996in}{1.236394in}}{\pgfqpoint{0.803268in}{1.228494in}}{\pgfqpoint{0.809092in}{1.222670in}}%
\pgfpathcurveto{\pgfqpoint{0.814916in}{1.216846in}}{\pgfqpoint{0.822816in}{1.213574in}}{\pgfqpoint{0.831052in}{1.213574in}}%
\pgfpathclose%
\pgfusepath{stroke,fill}%
\end{pgfscope}%
\begin{pgfscope}%
\pgfpathrectangle{\pgfqpoint{0.100000in}{0.220728in}}{\pgfqpoint{3.696000in}{3.696000in}}%
\pgfusepath{clip}%
\pgfsetbuttcap%
\pgfsetroundjoin%
\definecolor{currentfill}{rgb}{0.121569,0.466667,0.705882}%
\pgfsetfillcolor{currentfill}%
\pgfsetfillopacity{0.616596}%
\pgfsetlinewidth{1.003750pt}%
\definecolor{currentstroke}{rgb}{0.121569,0.466667,0.705882}%
\pgfsetstrokecolor{currentstroke}%
\pgfsetstrokeopacity{0.616596}%
\pgfsetdash{}{0pt}%
\pgfpathmoveto{\pgfqpoint{0.831052in}{1.213574in}}%
\pgfpathcurveto{\pgfqpoint{0.839288in}{1.213574in}}{\pgfqpoint{0.847189in}{1.216846in}}{\pgfqpoint{0.853012in}{1.222670in}}%
\pgfpathcurveto{\pgfqpoint{0.858836in}{1.228494in}}{\pgfqpoint{0.862109in}{1.236394in}}{\pgfqpoint{0.862109in}{1.244631in}}%
\pgfpathcurveto{\pgfqpoint{0.862109in}{1.252867in}}{\pgfqpoint{0.858836in}{1.260767in}}{\pgfqpoint{0.853012in}{1.266591in}}%
\pgfpathcurveto{\pgfqpoint{0.847189in}{1.272415in}}{\pgfqpoint{0.839288in}{1.275687in}}{\pgfqpoint{0.831052in}{1.275687in}}%
\pgfpathcurveto{\pgfqpoint{0.822816in}{1.275687in}}{\pgfqpoint{0.814916in}{1.272415in}}{\pgfqpoint{0.809092in}{1.266591in}}%
\pgfpathcurveto{\pgfqpoint{0.803268in}{1.260767in}}{\pgfqpoint{0.799996in}{1.252867in}}{\pgfqpoint{0.799996in}{1.244631in}}%
\pgfpathcurveto{\pgfqpoint{0.799996in}{1.236394in}}{\pgfqpoint{0.803268in}{1.228494in}}{\pgfqpoint{0.809092in}{1.222670in}}%
\pgfpathcurveto{\pgfqpoint{0.814916in}{1.216846in}}{\pgfqpoint{0.822816in}{1.213574in}}{\pgfqpoint{0.831052in}{1.213574in}}%
\pgfpathclose%
\pgfusepath{stroke,fill}%
\end{pgfscope}%
\begin{pgfscope}%
\pgfpathrectangle{\pgfqpoint{0.100000in}{0.220728in}}{\pgfqpoint{3.696000in}{3.696000in}}%
\pgfusepath{clip}%
\pgfsetbuttcap%
\pgfsetroundjoin%
\definecolor{currentfill}{rgb}{0.121569,0.466667,0.705882}%
\pgfsetfillcolor{currentfill}%
\pgfsetfillopacity{0.616596}%
\pgfsetlinewidth{1.003750pt}%
\definecolor{currentstroke}{rgb}{0.121569,0.466667,0.705882}%
\pgfsetstrokecolor{currentstroke}%
\pgfsetstrokeopacity{0.616596}%
\pgfsetdash{}{0pt}%
\pgfpathmoveto{\pgfqpoint{0.831052in}{1.213574in}}%
\pgfpathcurveto{\pgfqpoint{0.839288in}{1.213574in}}{\pgfqpoint{0.847189in}{1.216846in}}{\pgfqpoint{0.853012in}{1.222670in}}%
\pgfpathcurveto{\pgfqpoint{0.858836in}{1.228494in}}{\pgfqpoint{0.862109in}{1.236394in}}{\pgfqpoint{0.862109in}{1.244631in}}%
\pgfpathcurveto{\pgfqpoint{0.862109in}{1.252867in}}{\pgfqpoint{0.858836in}{1.260767in}}{\pgfqpoint{0.853012in}{1.266591in}}%
\pgfpathcurveto{\pgfqpoint{0.847189in}{1.272415in}}{\pgfqpoint{0.839288in}{1.275687in}}{\pgfqpoint{0.831052in}{1.275687in}}%
\pgfpathcurveto{\pgfqpoint{0.822816in}{1.275687in}}{\pgfqpoint{0.814916in}{1.272415in}}{\pgfqpoint{0.809092in}{1.266591in}}%
\pgfpathcurveto{\pgfqpoint{0.803268in}{1.260767in}}{\pgfqpoint{0.799996in}{1.252867in}}{\pgfqpoint{0.799996in}{1.244631in}}%
\pgfpathcurveto{\pgfqpoint{0.799996in}{1.236394in}}{\pgfqpoint{0.803268in}{1.228494in}}{\pgfqpoint{0.809092in}{1.222670in}}%
\pgfpathcurveto{\pgfqpoint{0.814916in}{1.216846in}}{\pgfqpoint{0.822816in}{1.213574in}}{\pgfqpoint{0.831052in}{1.213574in}}%
\pgfpathclose%
\pgfusepath{stroke,fill}%
\end{pgfscope}%
\begin{pgfscope}%
\pgfpathrectangle{\pgfqpoint{0.100000in}{0.220728in}}{\pgfqpoint{3.696000in}{3.696000in}}%
\pgfusepath{clip}%
\pgfsetbuttcap%
\pgfsetroundjoin%
\definecolor{currentfill}{rgb}{0.121569,0.466667,0.705882}%
\pgfsetfillcolor{currentfill}%
\pgfsetfillopacity{0.616596}%
\pgfsetlinewidth{1.003750pt}%
\definecolor{currentstroke}{rgb}{0.121569,0.466667,0.705882}%
\pgfsetstrokecolor{currentstroke}%
\pgfsetstrokeopacity{0.616596}%
\pgfsetdash{}{0pt}%
\pgfpathmoveto{\pgfqpoint{0.831052in}{1.213574in}}%
\pgfpathcurveto{\pgfqpoint{0.839288in}{1.213574in}}{\pgfqpoint{0.847189in}{1.216846in}}{\pgfqpoint{0.853012in}{1.222670in}}%
\pgfpathcurveto{\pgfqpoint{0.858836in}{1.228494in}}{\pgfqpoint{0.862109in}{1.236394in}}{\pgfqpoint{0.862109in}{1.244631in}}%
\pgfpathcurveto{\pgfqpoint{0.862109in}{1.252867in}}{\pgfqpoint{0.858836in}{1.260767in}}{\pgfqpoint{0.853012in}{1.266591in}}%
\pgfpathcurveto{\pgfqpoint{0.847189in}{1.272415in}}{\pgfqpoint{0.839288in}{1.275687in}}{\pgfqpoint{0.831052in}{1.275687in}}%
\pgfpathcurveto{\pgfqpoint{0.822816in}{1.275687in}}{\pgfqpoint{0.814916in}{1.272415in}}{\pgfqpoint{0.809092in}{1.266591in}}%
\pgfpathcurveto{\pgfqpoint{0.803268in}{1.260767in}}{\pgfqpoint{0.799996in}{1.252867in}}{\pgfqpoint{0.799996in}{1.244631in}}%
\pgfpathcurveto{\pgfqpoint{0.799996in}{1.236394in}}{\pgfqpoint{0.803268in}{1.228494in}}{\pgfqpoint{0.809092in}{1.222670in}}%
\pgfpathcurveto{\pgfqpoint{0.814916in}{1.216846in}}{\pgfqpoint{0.822816in}{1.213574in}}{\pgfqpoint{0.831052in}{1.213574in}}%
\pgfpathclose%
\pgfusepath{stroke,fill}%
\end{pgfscope}%
\begin{pgfscope}%
\pgfpathrectangle{\pgfqpoint{0.100000in}{0.220728in}}{\pgfqpoint{3.696000in}{3.696000in}}%
\pgfusepath{clip}%
\pgfsetbuttcap%
\pgfsetroundjoin%
\definecolor{currentfill}{rgb}{0.121569,0.466667,0.705882}%
\pgfsetfillcolor{currentfill}%
\pgfsetfillopacity{0.616596}%
\pgfsetlinewidth{1.003750pt}%
\definecolor{currentstroke}{rgb}{0.121569,0.466667,0.705882}%
\pgfsetstrokecolor{currentstroke}%
\pgfsetstrokeopacity{0.616596}%
\pgfsetdash{}{0pt}%
\pgfpathmoveto{\pgfqpoint{0.831052in}{1.213574in}}%
\pgfpathcurveto{\pgfqpoint{0.839288in}{1.213574in}}{\pgfqpoint{0.847189in}{1.216846in}}{\pgfqpoint{0.853012in}{1.222670in}}%
\pgfpathcurveto{\pgfqpoint{0.858836in}{1.228494in}}{\pgfqpoint{0.862109in}{1.236394in}}{\pgfqpoint{0.862109in}{1.244631in}}%
\pgfpathcurveto{\pgfqpoint{0.862109in}{1.252867in}}{\pgfqpoint{0.858836in}{1.260767in}}{\pgfqpoint{0.853012in}{1.266591in}}%
\pgfpathcurveto{\pgfqpoint{0.847189in}{1.272415in}}{\pgfqpoint{0.839288in}{1.275687in}}{\pgfqpoint{0.831052in}{1.275687in}}%
\pgfpathcurveto{\pgfqpoint{0.822816in}{1.275687in}}{\pgfqpoint{0.814916in}{1.272415in}}{\pgfqpoint{0.809092in}{1.266591in}}%
\pgfpathcurveto{\pgfqpoint{0.803268in}{1.260767in}}{\pgfqpoint{0.799996in}{1.252867in}}{\pgfqpoint{0.799996in}{1.244631in}}%
\pgfpathcurveto{\pgfqpoint{0.799996in}{1.236394in}}{\pgfqpoint{0.803268in}{1.228494in}}{\pgfqpoint{0.809092in}{1.222670in}}%
\pgfpathcurveto{\pgfqpoint{0.814916in}{1.216846in}}{\pgfqpoint{0.822816in}{1.213574in}}{\pgfqpoint{0.831052in}{1.213574in}}%
\pgfpathclose%
\pgfusepath{stroke,fill}%
\end{pgfscope}%
\begin{pgfscope}%
\pgfpathrectangle{\pgfqpoint{0.100000in}{0.220728in}}{\pgfqpoint{3.696000in}{3.696000in}}%
\pgfusepath{clip}%
\pgfsetbuttcap%
\pgfsetroundjoin%
\definecolor{currentfill}{rgb}{0.121569,0.466667,0.705882}%
\pgfsetfillcolor{currentfill}%
\pgfsetfillopacity{0.616596}%
\pgfsetlinewidth{1.003750pt}%
\definecolor{currentstroke}{rgb}{0.121569,0.466667,0.705882}%
\pgfsetstrokecolor{currentstroke}%
\pgfsetstrokeopacity{0.616596}%
\pgfsetdash{}{0pt}%
\pgfpathmoveto{\pgfqpoint{0.831052in}{1.213574in}}%
\pgfpathcurveto{\pgfqpoint{0.839288in}{1.213574in}}{\pgfqpoint{0.847189in}{1.216846in}}{\pgfqpoint{0.853012in}{1.222670in}}%
\pgfpathcurveto{\pgfqpoint{0.858836in}{1.228494in}}{\pgfqpoint{0.862109in}{1.236394in}}{\pgfqpoint{0.862109in}{1.244631in}}%
\pgfpathcurveto{\pgfqpoint{0.862109in}{1.252867in}}{\pgfqpoint{0.858836in}{1.260767in}}{\pgfqpoint{0.853012in}{1.266591in}}%
\pgfpathcurveto{\pgfqpoint{0.847189in}{1.272415in}}{\pgfqpoint{0.839288in}{1.275687in}}{\pgfqpoint{0.831052in}{1.275687in}}%
\pgfpathcurveto{\pgfqpoint{0.822816in}{1.275687in}}{\pgfqpoint{0.814916in}{1.272415in}}{\pgfqpoint{0.809092in}{1.266591in}}%
\pgfpathcurveto{\pgfqpoint{0.803268in}{1.260767in}}{\pgfqpoint{0.799996in}{1.252867in}}{\pgfqpoint{0.799996in}{1.244631in}}%
\pgfpathcurveto{\pgfqpoint{0.799996in}{1.236394in}}{\pgfqpoint{0.803268in}{1.228494in}}{\pgfqpoint{0.809092in}{1.222670in}}%
\pgfpathcurveto{\pgfqpoint{0.814916in}{1.216846in}}{\pgfqpoint{0.822816in}{1.213574in}}{\pgfqpoint{0.831052in}{1.213574in}}%
\pgfpathclose%
\pgfusepath{stroke,fill}%
\end{pgfscope}%
\begin{pgfscope}%
\pgfpathrectangle{\pgfqpoint{0.100000in}{0.220728in}}{\pgfqpoint{3.696000in}{3.696000in}}%
\pgfusepath{clip}%
\pgfsetbuttcap%
\pgfsetroundjoin%
\definecolor{currentfill}{rgb}{0.121569,0.466667,0.705882}%
\pgfsetfillcolor{currentfill}%
\pgfsetfillopacity{0.616596}%
\pgfsetlinewidth{1.003750pt}%
\definecolor{currentstroke}{rgb}{0.121569,0.466667,0.705882}%
\pgfsetstrokecolor{currentstroke}%
\pgfsetstrokeopacity{0.616596}%
\pgfsetdash{}{0pt}%
\pgfpathmoveto{\pgfqpoint{0.831052in}{1.213574in}}%
\pgfpathcurveto{\pgfqpoint{0.839288in}{1.213574in}}{\pgfqpoint{0.847189in}{1.216846in}}{\pgfqpoint{0.853012in}{1.222670in}}%
\pgfpathcurveto{\pgfqpoint{0.858836in}{1.228494in}}{\pgfqpoint{0.862109in}{1.236394in}}{\pgfqpoint{0.862109in}{1.244631in}}%
\pgfpathcurveto{\pgfqpoint{0.862109in}{1.252867in}}{\pgfqpoint{0.858836in}{1.260767in}}{\pgfqpoint{0.853012in}{1.266591in}}%
\pgfpathcurveto{\pgfqpoint{0.847189in}{1.272415in}}{\pgfqpoint{0.839288in}{1.275687in}}{\pgfqpoint{0.831052in}{1.275687in}}%
\pgfpathcurveto{\pgfqpoint{0.822816in}{1.275687in}}{\pgfqpoint{0.814916in}{1.272415in}}{\pgfqpoint{0.809092in}{1.266591in}}%
\pgfpathcurveto{\pgfqpoint{0.803268in}{1.260767in}}{\pgfqpoint{0.799996in}{1.252867in}}{\pgfqpoint{0.799996in}{1.244631in}}%
\pgfpathcurveto{\pgfqpoint{0.799996in}{1.236394in}}{\pgfqpoint{0.803268in}{1.228494in}}{\pgfqpoint{0.809092in}{1.222670in}}%
\pgfpathcurveto{\pgfqpoint{0.814916in}{1.216846in}}{\pgfqpoint{0.822816in}{1.213574in}}{\pgfqpoint{0.831052in}{1.213574in}}%
\pgfpathclose%
\pgfusepath{stroke,fill}%
\end{pgfscope}%
\begin{pgfscope}%
\pgfpathrectangle{\pgfqpoint{0.100000in}{0.220728in}}{\pgfqpoint{3.696000in}{3.696000in}}%
\pgfusepath{clip}%
\pgfsetbuttcap%
\pgfsetroundjoin%
\definecolor{currentfill}{rgb}{0.121569,0.466667,0.705882}%
\pgfsetfillcolor{currentfill}%
\pgfsetfillopacity{0.616596}%
\pgfsetlinewidth{1.003750pt}%
\definecolor{currentstroke}{rgb}{0.121569,0.466667,0.705882}%
\pgfsetstrokecolor{currentstroke}%
\pgfsetstrokeopacity{0.616596}%
\pgfsetdash{}{0pt}%
\pgfpathmoveto{\pgfqpoint{0.831052in}{1.213574in}}%
\pgfpathcurveto{\pgfqpoint{0.839288in}{1.213574in}}{\pgfqpoint{0.847189in}{1.216846in}}{\pgfqpoint{0.853012in}{1.222670in}}%
\pgfpathcurveto{\pgfqpoint{0.858836in}{1.228494in}}{\pgfqpoint{0.862109in}{1.236394in}}{\pgfqpoint{0.862109in}{1.244631in}}%
\pgfpathcurveto{\pgfqpoint{0.862109in}{1.252867in}}{\pgfqpoint{0.858836in}{1.260767in}}{\pgfqpoint{0.853012in}{1.266591in}}%
\pgfpathcurveto{\pgfqpoint{0.847189in}{1.272415in}}{\pgfqpoint{0.839288in}{1.275687in}}{\pgfqpoint{0.831052in}{1.275687in}}%
\pgfpathcurveto{\pgfqpoint{0.822816in}{1.275687in}}{\pgfqpoint{0.814916in}{1.272415in}}{\pgfqpoint{0.809092in}{1.266591in}}%
\pgfpathcurveto{\pgfqpoint{0.803268in}{1.260767in}}{\pgfqpoint{0.799996in}{1.252867in}}{\pgfqpoint{0.799996in}{1.244631in}}%
\pgfpathcurveto{\pgfqpoint{0.799996in}{1.236394in}}{\pgfqpoint{0.803268in}{1.228494in}}{\pgfqpoint{0.809092in}{1.222670in}}%
\pgfpathcurveto{\pgfqpoint{0.814916in}{1.216846in}}{\pgfqpoint{0.822816in}{1.213574in}}{\pgfqpoint{0.831052in}{1.213574in}}%
\pgfpathclose%
\pgfusepath{stroke,fill}%
\end{pgfscope}%
\begin{pgfscope}%
\pgfpathrectangle{\pgfqpoint{0.100000in}{0.220728in}}{\pgfqpoint{3.696000in}{3.696000in}}%
\pgfusepath{clip}%
\pgfsetbuttcap%
\pgfsetroundjoin%
\definecolor{currentfill}{rgb}{0.121569,0.466667,0.705882}%
\pgfsetfillcolor{currentfill}%
\pgfsetfillopacity{0.616596}%
\pgfsetlinewidth{1.003750pt}%
\definecolor{currentstroke}{rgb}{0.121569,0.466667,0.705882}%
\pgfsetstrokecolor{currentstroke}%
\pgfsetstrokeopacity{0.616596}%
\pgfsetdash{}{0pt}%
\pgfpathmoveto{\pgfqpoint{0.831052in}{1.213574in}}%
\pgfpathcurveto{\pgfqpoint{0.839288in}{1.213574in}}{\pgfqpoint{0.847189in}{1.216846in}}{\pgfqpoint{0.853012in}{1.222670in}}%
\pgfpathcurveto{\pgfqpoint{0.858836in}{1.228494in}}{\pgfqpoint{0.862109in}{1.236394in}}{\pgfqpoint{0.862109in}{1.244631in}}%
\pgfpathcurveto{\pgfqpoint{0.862109in}{1.252867in}}{\pgfqpoint{0.858836in}{1.260767in}}{\pgfqpoint{0.853012in}{1.266591in}}%
\pgfpathcurveto{\pgfqpoint{0.847189in}{1.272415in}}{\pgfqpoint{0.839288in}{1.275687in}}{\pgfqpoint{0.831052in}{1.275687in}}%
\pgfpathcurveto{\pgfqpoint{0.822816in}{1.275687in}}{\pgfqpoint{0.814916in}{1.272415in}}{\pgfqpoint{0.809092in}{1.266591in}}%
\pgfpathcurveto{\pgfqpoint{0.803268in}{1.260767in}}{\pgfqpoint{0.799996in}{1.252867in}}{\pgfqpoint{0.799996in}{1.244631in}}%
\pgfpathcurveto{\pgfqpoint{0.799996in}{1.236394in}}{\pgfqpoint{0.803268in}{1.228494in}}{\pgfqpoint{0.809092in}{1.222670in}}%
\pgfpathcurveto{\pgfqpoint{0.814916in}{1.216846in}}{\pgfqpoint{0.822816in}{1.213574in}}{\pgfqpoint{0.831052in}{1.213574in}}%
\pgfpathclose%
\pgfusepath{stroke,fill}%
\end{pgfscope}%
\begin{pgfscope}%
\pgfpathrectangle{\pgfqpoint{0.100000in}{0.220728in}}{\pgfqpoint{3.696000in}{3.696000in}}%
\pgfusepath{clip}%
\pgfsetbuttcap%
\pgfsetroundjoin%
\definecolor{currentfill}{rgb}{0.121569,0.466667,0.705882}%
\pgfsetfillcolor{currentfill}%
\pgfsetfillopacity{0.616596}%
\pgfsetlinewidth{1.003750pt}%
\definecolor{currentstroke}{rgb}{0.121569,0.466667,0.705882}%
\pgfsetstrokecolor{currentstroke}%
\pgfsetstrokeopacity{0.616596}%
\pgfsetdash{}{0pt}%
\pgfpathmoveto{\pgfqpoint{0.831052in}{1.213574in}}%
\pgfpathcurveto{\pgfqpoint{0.839288in}{1.213574in}}{\pgfqpoint{0.847189in}{1.216846in}}{\pgfqpoint{0.853012in}{1.222670in}}%
\pgfpathcurveto{\pgfqpoint{0.858836in}{1.228494in}}{\pgfqpoint{0.862109in}{1.236394in}}{\pgfqpoint{0.862109in}{1.244631in}}%
\pgfpathcurveto{\pgfqpoint{0.862109in}{1.252867in}}{\pgfqpoint{0.858836in}{1.260767in}}{\pgfqpoint{0.853012in}{1.266591in}}%
\pgfpathcurveto{\pgfqpoint{0.847189in}{1.272415in}}{\pgfqpoint{0.839288in}{1.275687in}}{\pgfqpoint{0.831052in}{1.275687in}}%
\pgfpathcurveto{\pgfqpoint{0.822816in}{1.275687in}}{\pgfqpoint{0.814916in}{1.272415in}}{\pgfqpoint{0.809092in}{1.266591in}}%
\pgfpathcurveto{\pgfqpoint{0.803268in}{1.260767in}}{\pgfqpoint{0.799996in}{1.252867in}}{\pgfqpoint{0.799996in}{1.244631in}}%
\pgfpathcurveto{\pgfqpoint{0.799996in}{1.236394in}}{\pgfqpoint{0.803268in}{1.228494in}}{\pgfqpoint{0.809092in}{1.222670in}}%
\pgfpathcurveto{\pgfqpoint{0.814916in}{1.216846in}}{\pgfqpoint{0.822816in}{1.213574in}}{\pgfqpoint{0.831052in}{1.213574in}}%
\pgfpathclose%
\pgfusepath{stroke,fill}%
\end{pgfscope}%
\begin{pgfscope}%
\pgfpathrectangle{\pgfqpoint{0.100000in}{0.220728in}}{\pgfqpoint{3.696000in}{3.696000in}}%
\pgfusepath{clip}%
\pgfsetbuttcap%
\pgfsetroundjoin%
\definecolor{currentfill}{rgb}{0.121569,0.466667,0.705882}%
\pgfsetfillcolor{currentfill}%
\pgfsetfillopacity{0.616596}%
\pgfsetlinewidth{1.003750pt}%
\definecolor{currentstroke}{rgb}{0.121569,0.466667,0.705882}%
\pgfsetstrokecolor{currentstroke}%
\pgfsetstrokeopacity{0.616596}%
\pgfsetdash{}{0pt}%
\pgfpathmoveto{\pgfqpoint{0.831052in}{1.213574in}}%
\pgfpathcurveto{\pgfqpoint{0.839288in}{1.213574in}}{\pgfqpoint{0.847189in}{1.216846in}}{\pgfqpoint{0.853012in}{1.222670in}}%
\pgfpathcurveto{\pgfqpoint{0.858836in}{1.228494in}}{\pgfqpoint{0.862109in}{1.236394in}}{\pgfqpoint{0.862109in}{1.244631in}}%
\pgfpathcurveto{\pgfqpoint{0.862109in}{1.252867in}}{\pgfqpoint{0.858836in}{1.260767in}}{\pgfqpoint{0.853012in}{1.266591in}}%
\pgfpathcurveto{\pgfqpoint{0.847189in}{1.272415in}}{\pgfqpoint{0.839288in}{1.275687in}}{\pgfqpoint{0.831052in}{1.275687in}}%
\pgfpathcurveto{\pgfqpoint{0.822816in}{1.275687in}}{\pgfqpoint{0.814916in}{1.272415in}}{\pgfqpoint{0.809092in}{1.266591in}}%
\pgfpathcurveto{\pgfqpoint{0.803268in}{1.260767in}}{\pgfqpoint{0.799996in}{1.252867in}}{\pgfqpoint{0.799996in}{1.244631in}}%
\pgfpathcurveto{\pgfqpoint{0.799996in}{1.236394in}}{\pgfqpoint{0.803268in}{1.228494in}}{\pgfqpoint{0.809092in}{1.222670in}}%
\pgfpathcurveto{\pgfqpoint{0.814916in}{1.216846in}}{\pgfqpoint{0.822816in}{1.213574in}}{\pgfqpoint{0.831052in}{1.213574in}}%
\pgfpathclose%
\pgfusepath{stroke,fill}%
\end{pgfscope}%
\begin{pgfscope}%
\pgfpathrectangle{\pgfqpoint{0.100000in}{0.220728in}}{\pgfqpoint{3.696000in}{3.696000in}}%
\pgfusepath{clip}%
\pgfsetbuttcap%
\pgfsetroundjoin%
\definecolor{currentfill}{rgb}{0.121569,0.466667,0.705882}%
\pgfsetfillcolor{currentfill}%
\pgfsetfillopacity{0.616596}%
\pgfsetlinewidth{1.003750pt}%
\definecolor{currentstroke}{rgb}{0.121569,0.466667,0.705882}%
\pgfsetstrokecolor{currentstroke}%
\pgfsetstrokeopacity{0.616596}%
\pgfsetdash{}{0pt}%
\pgfpathmoveto{\pgfqpoint{0.831052in}{1.213574in}}%
\pgfpathcurveto{\pgfqpoint{0.839288in}{1.213574in}}{\pgfqpoint{0.847189in}{1.216846in}}{\pgfqpoint{0.853012in}{1.222670in}}%
\pgfpathcurveto{\pgfqpoint{0.858836in}{1.228494in}}{\pgfqpoint{0.862109in}{1.236394in}}{\pgfqpoint{0.862109in}{1.244631in}}%
\pgfpathcurveto{\pgfqpoint{0.862109in}{1.252867in}}{\pgfqpoint{0.858836in}{1.260767in}}{\pgfqpoint{0.853012in}{1.266591in}}%
\pgfpathcurveto{\pgfqpoint{0.847189in}{1.272415in}}{\pgfqpoint{0.839288in}{1.275687in}}{\pgfqpoint{0.831052in}{1.275687in}}%
\pgfpathcurveto{\pgfqpoint{0.822816in}{1.275687in}}{\pgfqpoint{0.814916in}{1.272415in}}{\pgfqpoint{0.809092in}{1.266591in}}%
\pgfpathcurveto{\pgfqpoint{0.803268in}{1.260767in}}{\pgfqpoint{0.799996in}{1.252867in}}{\pgfqpoint{0.799996in}{1.244631in}}%
\pgfpathcurveto{\pgfqpoint{0.799996in}{1.236394in}}{\pgfqpoint{0.803268in}{1.228494in}}{\pgfqpoint{0.809092in}{1.222670in}}%
\pgfpathcurveto{\pgfqpoint{0.814916in}{1.216846in}}{\pgfqpoint{0.822816in}{1.213574in}}{\pgfqpoint{0.831052in}{1.213574in}}%
\pgfpathclose%
\pgfusepath{stroke,fill}%
\end{pgfscope}%
\begin{pgfscope}%
\pgfpathrectangle{\pgfqpoint{0.100000in}{0.220728in}}{\pgfqpoint{3.696000in}{3.696000in}}%
\pgfusepath{clip}%
\pgfsetbuttcap%
\pgfsetroundjoin%
\definecolor{currentfill}{rgb}{0.121569,0.466667,0.705882}%
\pgfsetfillcolor{currentfill}%
\pgfsetfillopacity{0.616596}%
\pgfsetlinewidth{1.003750pt}%
\definecolor{currentstroke}{rgb}{0.121569,0.466667,0.705882}%
\pgfsetstrokecolor{currentstroke}%
\pgfsetstrokeopacity{0.616596}%
\pgfsetdash{}{0pt}%
\pgfpathmoveto{\pgfqpoint{0.831052in}{1.213574in}}%
\pgfpathcurveto{\pgfqpoint{0.839288in}{1.213574in}}{\pgfqpoint{0.847189in}{1.216846in}}{\pgfqpoint{0.853012in}{1.222670in}}%
\pgfpathcurveto{\pgfqpoint{0.858836in}{1.228494in}}{\pgfqpoint{0.862109in}{1.236394in}}{\pgfqpoint{0.862109in}{1.244631in}}%
\pgfpathcurveto{\pgfqpoint{0.862109in}{1.252867in}}{\pgfqpoint{0.858836in}{1.260767in}}{\pgfqpoint{0.853012in}{1.266591in}}%
\pgfpathcurveto{\pgfqpoint{0.847189in}{1.272415in}}{\pgfqpoint{0.839288in}{1.275687in}}{\pgfqpoint{0.831052in}{1.275687in}}%
\pgfpathcurveto{\pgfqpoint{0.822816in}{1.275687in}}{\pgfqpoint{0.814916in}{1.272415in}}{\pgfqpoint{0.809092in}{1.266591in}}%
\pgfpathcurveto{\pgfqpoint{0.803268in}{1.260767in}}{\pgfqpoint{0.799996in}{1.252867in}}{\pgfqpoint{0.799996in}{1.244631in}}%
\pgfpathcurveto{\pgfqpoint{0.799996in}{1.236394in}}{\pgfqpoint{0.803268in}{1.228494in}}{\pgfqpoint{0.809092in}{1.222670in}}%
\pgfpathcurveto{\pgfqpoint{0.814916in}{1.216846in}}{\pgfqpoint{0.822816in}{1.213574in}}{\pgfqpoint{0.831052in}{1.213574in}}%
\pgfpathclose%
\pgfusepath{stroke,fill}%
\end{pgfscope}%
\begin{pgfscope}%
\pgfpathrectangle{\pgfqpoint{0.100000in}{0.220728in}}{\pgfqpoint{3.696000in}{3.696000in}}%
\pgfusepath{clip}%
\pgfsetbuttcap%
\pgfsetroundjoin%
\definecolor{currentfill}{rgb}{0.121569,0.466667,0.705882}%
\pgfsetfillcolor{currentfill}%
\pgfsetfillopacity{0.616596}%
\pgfsetlinewidth{1.003750pt}%
\definecolor{currentstroke}{rgb}{0.121569,0.466667,0.705882}%
\pgfsetstrokecolor{currentstroke}%
\pgfsetstrokeopacity{0.616596}%
\pgfsetdash{}{0pt}%
\pgfpathmoveto{\pgfqpoint{0.831052in}{1.213574in}}%
\pgfpathcurveto{\pgfqpoint{0.839288in}{1.213574in}}{\pgfqpoint{0.847189in}{1.216846in}}{\pgfqpoint{0.853012in}{1.222670in}}%
\pgfpathcurveto{\pgfqpoint{0.858836in}{1.228494in}}{\pgfqpoint{0.862109in}{1.236394in}}{\pgfqpoint{0.862109in}{1.244631in}}%
\pgfpathcurveto{\pgfqpoint{0.862109in}{1.252867in}}{\pgfqpoint{0.858836in}{1.260767in}}{\pgfqpoint{0.853012in}{1.266591in}}%
\pgfpathcurveto{\pgfqpoint{0.847189in}{1.272415in}}{\pgfqpoint{0.839288in}{1.275687in}}{\pgfqpoint{0.831052in}{1.275687in}}%
\pgfpathcurveto{\pgfqpoint{0.822816in}{1.275687in}}{\pgfqpoint{0.814916in}{1.272415in}}{\pgfqpoint{0.809092in}{1.266591in}}%
\pgfpathcurveto{\pgfqpoint{0.803268in}{1.260767in}}{\pgfqpoint{0.799996in}{1.252867in}}{\pgfqpoint{0.799996in}{1.244631in}}%
\pgfpathcurveto{\pgfqpoint{0.799996in}{1.236394in}}{\pgfqpoint{0.803268in}{1.228494in}}{\pgfqpoint{0.809092in}{1.222670in}}%
\pgfpathcurveto{\pgfqpoint{0.814916in}{1.216846in}}{\pgfqpoint{0.822816in}{1.213574in}}{\pgfqpoint{0.831052in}{1.213574in}}%
\pgfpathclose%
\pgfusepath{stroke,fill}%
\end{pgfscope}%
\begin{pgfscope}%
\pgfpathrectangle{\pgfqpoint{0.100000in}{0.220728in}}{\pgfqpoint{3.696000in}{3.696000in}}%
\pgfusepath{clip}%
\pgfsetbuttcap%
\pgfsetroundjoin%
\definecolor{currentfill}{rgb}{0.121569,0.466667,0.705882}%
\pgfsetfillcolor{currentfill}%
\pgfsetfillopacity{0.616596}%
\pgfsetlinewidth{1.003750pt}%
\definecolor{currentstroke}{rgb}{0.121569,0.466667,0.705882}%
\pgfsetstrokecolor{currentstroke}%
\pgfsetstrokeopacity{0.616596}%
\pgfsetdash{}{0pt}%
\pgfpathmoveto{\pgfqpoint{0.831052in}{1.213574in}}%
\pgfpathcurveto{\pgfqpoint{0.839288in}{1.213574in}}{\pgfqpoint{0.847189in}{1.216846in}}{\pgfqpoint{0.853012in}{1.222670in}}%
\pgfpathcurveto{\pgfqpoint{0.858836in}{1.228494in}}{\pgfqpoint{0.862109in}{1.236394in}}{\pgfqpoint{0.862109in}{1.244631in}}%
\pgfpathcurveto{\pgfqpoint{0.862109in}{1.252867in}}{\pgfqpoint{0.858836in}{1.260767in}}{\pgfqpoint{0.853012in}{1.266591in}}%
\pgfpathcurveto{\pgfqpoint{0.847189in}{1.272415in}}{\pgfqpoint{0.839288in}{1.275687in}}{\pgfqpoint{0.831052in}{1.275687in}}%
\pgfpathcurveto{\pgfqpoint{0.822816in}{1.275687in}}{\pgfqpoint{0.814916in}{1.272415in}}{\pgfqpoint{0.809092in}{1.266591in}}%
\pgfpathcurveto{\pgfqpoint{0.803268in}{1.260767in}}{\pgfqpoint{0.799996in}{1.252867in}}{\pgfqpoint{0.799996in}{1.244631in}}%
\pgfpathcurveto{\pgfqpoint{0.799996in}{1.236394in}}{\pgfqpoint{0.803268in}{1.228494in}}{\pgfqpoint{0.809092in}{1.222670in}}%
\pgfpathcurveto{\pgfqpoint{0.814916in}{1.216846in}}{\pgfqpoint{0.822816in}{1.213574in}}{\pgfqpoint{0.831052in}{1.213574in}}%
\pgfpathclose%
\pgfusepath{stroke,fill}%
\end{pgfscope}%
\begin{pgfscope}%
\pgfpathrectangle{\pgfqpoint{0.100000in}{0.220728in}}{\pgfqpoint{3.696000in}{3.696000in}}%
\pgfusepath{clip}%
\pgfsetbuttcap%
\pgfsetroundjoin%
\definecolor{currentfill}{rgb}{0.121569,0.466667,0.705882}%
\pgfsetfillcolor{currentfill}%
\pgfsetfillopacity{0.616596}%
\pgfsetlinewidth{1.003750pt}%
\definecolor{currentstroke}{rgb}{0.121569,0.466667,0.705882}%
\pgfsetstrokecolor{currentstroke}%
\pgfsetstrokeopacity{0.616596}%
\pgfsetdash{}{0pt}%
\pgfpathmoveto{\pgfqpoint{0.831052in}{1.213574in}}%
\pgfpathcurveto{\pgfqpoint{0.839288in}{1.213574in}}{\pgfqpoint{0.847189in}{1.216846in}}{\pgfqpoint{0.853012in}{1.222670in}}%
\pgfpathcurveto{\pgfqpoint{0.858836in}{1.228494in}}{\pgfqpoint{0.862109in}{1.236394in}}{\pgfqpoint{0.862109in}{1.244631in}}%
\pgfpathcurveto{\pgfqpoint{0.862109in}{1.252867in}}{\pgfqpoint{0.858836in}{1.260767in}}{\pgfqpoint{0.853012in}{1.266591in}}%
\pgfpathcurveto{\pgfqpoint{0.847189in}{1.272415in}}{\pgfqpoint{0.839288in}{1.275687in}}{\pgfqpoint{0.831052in}{1.275687in}}%
\pgfpathcurveto{\pgfqpoint{0.822816in}{1.275687in}}{\pgfqpoint{0.814916in}{1.272415in}}{\pgfqpoint{0.809092in}{1.266591in}}%
\pgfpathcurveto{\pgfqpoint{0.803268in}{1.260767in}}{\pgfqpoint{0.799996in}{1.252867in}}{\pgfqpoint{0.799996in}{1.244631in}}%
\pgfpathcurveto{\pgfqpoint{0.799996in}{1.236394in}}{\pgfqpoint{0.803268in}{1.228494in}}{\pgfqpoint{0.809092in}{1.222670in}}%
\pgfpathcurveto{\pgfqpoint{0.814916in}{1.216846in}}{\pgfqpoint{0.822816in}{1.213574in}}{\pgfqpoint{0.831052in}{1.213574in}}%
\pgfpathclose%
\pgfusepath{stroke,fill}%
\end{pgfscope}%
\begin{pgfscope}%
\pgfpathrectangle{\pgfqpoint{0.100000in}{0.220728in}}{\pgfqpoint{3.696000in}{3.696000in}}%
\pgfusepath{clip}%
\pgfsetbuttcap%
\pgfsetroundjoin%
\definecolor{currentfill}{rgb}{0.121569,0.466667,0.705882}%
\pgfsetfillcolor{currentfill}%
\pgfsetfillopacity{0.616596}%
\pgfsetlinewidth{1.003750pt}%
\definecolor{currentstroke}{rgb}{0.121569,0.466667,0.705882}%
\pgfsetstrokecolor{currentstroke}%
\pgfsetstrokeopacity{0.616596}%
\pgfsetdash{}{0pt}%
\pgfpathmoveto{\pgfqpoint{0.831052in}{1.213574in}}%
\pgfpathcurveto{\pgfqpoint{0.839288in}{1.213574in}}{\pgfqpoint{0.847189in}{1.216846in}}{\pgfqpoint{0.853012in}{1.222670in}}%
\pgfpathcurveto{\pgfqpoint{0.858836in}{1.228494in}}{\pgfqpoint{0.862109in}{1.236394in}}{\pgfqpoint{0.862109in}{1.244631in}}%
\pgfpathcurveto{\pgfqpoint{0.862109in}{1.252867in}}{\pgfqpoint{0.858836in}{1.260767in}}{\pgfqpoint{0.853012in}{1.266591in}}%
\pgfpathcurveto{\pgfqpoint{0.847189in}{1.272415in}}{\pgfqpoint{0.839288in}{1.275687in}}{\pgfqpoint{0.831052in}{1.275687in}}%
\pgfpathcurveto{\pgfqpoint{0.822816in}{1.275687in}}{\pgfqpoint{0.814916in}{1.272415in}}{\pgfqpoint{0.809092in}{1.266591in}}%
\pgfpathcurveto{\pgfqpoint{0.803268in}{1.260767in}}{\pgfqpoint{0.799996in}{1.252867in}}{\pgfqpoint{0.799996in}{1.244631in}}%
\pgfpathcurveto{\pgfqpoint{0.799996in}{1.236394in}}{\pgfqpoint{0.803268in}{1.228494in}}{\pgfqpoint{0.809092in}{1.222670in}}%
\pgfpathcurveto{\pgfqpoint{0.814916in}{1.216846in}}{\pgfqpoint{0.822816in}{1.213574in}}{\pgfqpoint{0.831052in}{1.213574in}}%
\pgfpathclose%
\pgfusepath{stroke,fill}%
\end{pgfscope}%
\begin{pgfscope}%
\pgfpathrectangle{\pgfqpoint{0.100000in}{0.220728in}}{\pgfqpoint{3.696000in}{3.696000in}}%
\pgfusepath{clip}%
\pgfsetbuttcap%
\pgfsetroundjoin%
\definecolor{currentfill}{rgb}{0.121569,0.466667,0.705882}%
\pgfsetfillcolor{currentfill}%
\pgfsetfillopacity{0.616596}%
\pgfsetlinewidth{1.003750pt}%
\definecolor{currentstroke}{rgb}{0.121569,0.466667,0.705882}%
\pgfsetstrokecolor{currentstroke}%
\pgfsetstrokeopacity{0.616596}%
\pgfsetdash{}{0pt}%
\pgfpathmoveto{\pgfqpoint{0.831052in}{1.213574in}}%
\pgfpathcurveto{\pgfqpoint{0.839288in}{1.213574in}}{\pgfqpoint{0.847189in}{1.216846in}}{\pgfqpoint{0.853012in}{1.222670in}}%
\pgfpathcurveto{\pgfqpoint{0.858836in}{1.228494in}}{\pgfqpoint{0.862109in}{1.236394in}}{\pgfqpoint{0.862109in}{1.244631in}}%
\pgfpathcurveto{\pgfqpoint{0.862109in}{1.252867in}}{\pgfqpoint{0.858836in}{1.260767in}}{\pgfqpoint{0.853012in}{1.266591in}}%
\pgfpathcurveto{\pgfqpoint{0.847189in}{1.272415in}}{\pgfqpoint{0.839288in}{1.275687in}}{\pgfqpoint{0.831052in}{1.275687in}}%
\pgfpathcurveto{\pgfqpoint{0.822816in}{1.275687in}}{\pgfqpoint{0.814916in}{1.272415in}}{\pgfqpoint{0.809092in}{1.266591in}}%
\pgfpathcurveto{\pgfqpoint{0.803268in}{1.260767in}}{\pgfqpoint{0.799996in}{1.252867in}}{\pgfqpoint{0.799996in}{1.244631in}}%
\pgfpathcurveto{\pgfqpoint{0.799996in}{1.236394in}}{\pgfqpoint{0.803268in}{1.228494in}}{\pgfqpoint{0.809092in}{1.222670in}}%
\pgfpathcurveto{\pgfqpoint{0.814916in}{1.216846in}}{\pgfqpoint{0.822816in}{1.213574in}}{\pgfqpoint{0.831052in}{1.213574in}}%
\pgfpathclose%
\pgfusepath{stroke,fill}%
\end{pgfscope}%
\begin{pgfscope}%
\pgfpathrectangle{\pgfqpoint{0.100000in}{0.220728in}}{\pgfqpoint{3.696000in}{3.696000in}}%
\pgfusepath{clip}%
\pgfsetbuttcap%
\pgfsetroundjoin%
\definecolor{currentfill}{rgb}{0.121569,0.466667,0.705882}%
\pgfsetfillcolor{currentfill}%
\pgfsetfillopacity{0.616596}%
\pgfsetlinewidth{1.003750pt}%
\definecolor{currentstroke}{rgb}{0.121569,0.466667,0.705882}%
\pgfsetstrokecolor{currentstroke}%
\pgfsetstrokeopacity{0.616596}%
\pgfsetdash{}{0pt}%
\pgfpathmoveto{\pgfqpoint{0.831052in}{1.213574in}}%
\pgfpathcurveto{\pgfqpoint{0.839288in}{1.213574in}}{\pgfqpoint{0.847189in}{1.216846in}}{\pgfqpoint{0.853012in}{1.222670in}}%
\pgfpathcurveto{\pgfqpoint{0.858836in}{1.228494in}}{\pgfqpoint{0.862109in}{1.236394in}}{\pgfqpoint{0.862109in}{1.244631in}}%
\pgfpathcurveto{\pgfqpoint{0.862109in}{1.252867in}}{\pgfqpoint{0.858836in}{1.260767in}}{\pgfqpoint{0.853012in}{1.266591in}}%
\pgfpathcurveto{\pgfqpoint{0.847189in}{1.272415in}}{\pgfqpoint{0.839288in}{1.275687in}}{\pgfqpoint{0.831052in}{1.275687in}}%
\pgfpathcurveto{\pgfqpoint{0.822816in}{1.275687in}}{\pgfqpoint{0.814916in}{1.272415in}}{\pgfqpoint{0.809092in}{1.266591in}}%
\pgfpathcurveto{\pgfqpoint{0.803268in}{1.260767in}}{\pgfqpoint{0.799996in}{1.252867in}}{\pgfqpoint{0.799996in}{1.244631in}}%
\pgfpathcurveto{\pgfqpoint{0.799996in}{1.236394in}}{\pgfqpoint{0.803268in}{1.228494in}}{\pgfqpoint{0.809092in}{1.222670in}}%
\pgfpathcurveto{\pgfqpoint{0.814916in}{1.216846in}}{\pgfqpoint{0.822816in}{1.213574in}}{\pgfqpoint{0.831052in}{1.213574in}}%
\pgfpathclose%
\pgfusepath{stroke,fill}%
\end{pgfscope}%
\begin{pgfscope}%
\pgfpathrectangle{\pgfqpoint{0.100000in}{0.220728in}}{\pgfqpoint{3.696000in}{3.696000in}}%
\pgfusepath{clip}%
\pgfsetbuttcap%
\pgfsetroundjoin%
\definecolor{currentfill}{rgb}{0.121569,0.466667,0.705882}%
\pgfsetfillcolor{currentfill}%
\pgfsetfillopacity{0.616596}%
\pgfsetlinewidth{1.003750pt}%
\definecolor{currentstroke}{rgb}{0.121569,0.466667,0.705882}%
\pgfsetstrokecolor{currentstroke}%
\pgfsetstrokeopacity{0.616596}%
\pgfsetdash{}{0pt}%
\pgfpathmoveto{\pgfqpoint{0.831052in}{1.213574in}}%
\pgfpathcurveto{\pgfqpoint{0.839288in}{1.213574in}}{\pgfqpoint{0.847189in}{1.216846in}}{\pgfqpoint{0.853012in}{1.222670in}}%
\pgfpathcurveto{\pgfqpoint{0.858836in}{1.228494in}}{\pgfqpoint{0.862109in}{1.236394in}}{\pgfqpoint{0.862109in}{1.244631in}}%
\pgfpathcurveto{\pgfqpoint{0.862109in}{1.252867in}}{\pgfqpoint{0.858836in}{1.260767in}}{\pgfqpoint{0.853012in}{1.266591in}}%
\pgfpathcurveto{\pgfqpoint{0.847189in}{1.272415in}}{\pgfqpoint{0.839288in}{1.275687in}}{\pgfqpoint{0.831052in}{1.275687in}}%
\pgfpathcurveto{\pgfqpoint{0.822816in}{1.275687in}}{\pgfqpoint{0.814916in}{1.272415in}}{\pgfqpoint{0.809092in}{1.266591in}}%
\pgfpathcurveto{\pgfqpoint{0.803268in}{1.260767in}}{\pgfqpoint{0.799996in}{1.252867in}}{\pgfqpoint{0.799996in}{1.244631in}}%
\pgfpathcurveto{\pgfqpoint{0.799996in}{1.236394in}}{\pgfqpoint{0.803268in}{1.228494in}}{\pgfqpoint{0.809092in}{1.222670in}}%
\pgfpathcurveto{\pgfqpoint{0.814916in}{1.216846in}}{\pgfqpoint{0.822816in}{1.213574in}}{\pgfqpoint{0.831052in}{1.213574in}}%
\pgfpathclose%
\pgfusepath{stroke,fill}%
\end{pgfscope}%
\begin{pgfscope}%
\pgfpathrectangle{\pgfqpoint{0.100000in}{0.220728in}}{\pgfqpoint{3.696000in}{3.696000in}}%
\pgfusepath{clip}%
\pgfsetbuttcap%
\pgfsetroundjoin%
\definecolor{currentfill}{rgb}{0.121569,0.466667,0.705882}%
\pgfsetfillcolor{currentfill}%
\pgfsetfillopacity{0.616596}%
\pgfsetlinewidth{1.003750pt}%
\definecolor{currentstroke}{rgb}{0.121569,0.466667,0.705882}%
\pgfsetstrokecolor{currentstroke}%
\pgfsetstrokeopacity{0.616596}%
\pgfsetdash{}{0pt}%
\pgfpathmoveto{\pgfqpoint{0.831052in}{1.213574in}}%
\pgfpathcurveto{\pgfqpoint{0.839288in}{1.213574in}}{\pgfqpoint{0.847189in}{1.216846in}}{\pgfqpoint{0.853012in}{1.222670in}}%
\pgfpathcurveto{\pgfqpoint{0.858836in}{1.228494in}}{\pgfqpoint{0.862109in}{1.236394in}}{\pgfqpoint{0.862109in}{1.244631in}}%
\pgfpathcurveto{\pgfqpoint{0.862109in}{1.252867in}}{\pgfqpoint{0.858836in}{1.260767in}}{\pgfqpoint{0.853012in}{1.266591in}}%
\pgfpathcurveto{\pgfqpoint{0.847189in}{1.272415in}}{\pgfqpoint{0.839288in}{1.275687in}}{\pgfqpoint{0.831052in}{1.275687in}}%
\pgfpathcurveto{\pgfqpoint{0.822816in}{1.275687in}}{\pgfqpoint{0.814916in}{1.272415in}}{\pgfqpoint{0.809092in}{1.266591in}}%
\pgfpathcurveto{\pgfqpoint{0.803268in}{1.260767in}}{\pgfqpoint{0.799996in}{1.252867in}}{\pgfqpoint{0.799996in}{1.244631in}}%
\pgfpathcurveto{\pgfqpoint{0.799996in}{1.236394in}}{\pgfqpoint{0.803268in}{1.228494in}}{\pgfqpoint{0.809092in}{1.222670in}}%
\pgfpathcurveto{\pgfqpoint{0.814916in}{1.216846in}}{\pgfqpoint{0.822816in}{1.213574in}}{\pgfqpoint{0.831052in}{1.213574in}}%
\pgfpathclose%
\pgfusepath{stroke,fill}%
\end{pgfscope}%
\begin{pgfscope}%
\pgfpathrectangle{\pgfqpoint{0.100000in}{0.220728in}}{\pgfqpoint{3.696000in}{3.696000in}}%
\pgfusepath{clip}%
\pgfsetbuttcap%
\pgfsetroundjoin%
\definecolor{currentfill}{rgb}{0.121569,0.466667,0.705882}%
\pgfsetfillcolor{currentfill}%
\pgfsetfillopacity{0.616596}%
\pgfsetlinewidth{1.003750pt}%
\definecolor{currentstroke}{rgb}{0.121569,0.466667,0.705882}%
\pgfsetstrokecolor{currentstroke}%
\pgfsetstrokeopacity{0.616596}%
\pgfsetdash{}{0pt}%
\pgfpathmoveto{\pgfqpoint{0.831052in}{1.213574in}}%
\pgfpathcurveto{\pgfqpoint{0.839288in}{1.213574in}}{\pgfqpoint{0.847189in}{1.216846in}}{\pgfqpoint{0.853012in}{1.222670in}}%
\pgfpathcurveto{\pgfqpoint{0.858836in}{1.228494in}}{\pgfqpoint{0.862109in}{1.236394in}}{\pgfqpoint{0.862109in}{1.244631in}}%
\pgfpathcurveto{\pgfqpoint{0.862109in}{1.252867in}}{\pgfqpoint{0.858836in}{1.260767in}}{\pgfqpoint{0.853012in}{1.266591in}}%
\pgfpathcurveto{\pgfqpoint{0.847189in}{1.272415in}}{\pgfqpoint{0.839288in}{1.275687in}}{\pgfqpoint{0.831052in}{1.275687in}}%
\pgfpathcurveto{\pgfqpoint{0.822816in}{1.275687in}}{\pgfqpoint{0.814916in}{1.272415in}}{\pgfqpoint{0.809092in}{1.266591in}}%
\pgfpathcurveto{\pgfqpoint{0.803268in}{1.260767in}}{\pgfqpoint{0.799996in}{1.252867in}}{\pgfqpoint{0.799996in}{1.244631in}}%
\pgfpathcurveto{\pgfqpoint{0.799996in}{1.236394in}}{\pgfqpoint{0.803268in}{1.228494in}}{\pgfqpoint{0.809092in}{1.222670in}}%
\pgfpathcurveto{\pgfqpoint{0.814916in}{1.216846in}}{\pgfqpoint{0.822816in}{1.213574in}}{\pgfqpoint{0.831052in}{1.213574in}}%
\pgfpathclose%
\pgfusepath{stroke,fill}%
\end{pgfscope}%
\begin{pgfscope}%
\pgfpathrectangle{\pgfqpoint{0.100000in}{0.220728in}}{\pgfqpoint{3.696000in}{3.696000in}}%
\pgfusepath{clip}%
\pgfsetbuttcap%
\pgfsetroundjoin%
\definecolor{currentfill}{rgb}{0.121569,0.466667,0.705882}%
\pgfsetfillcolor{currentfill}%
\pgfsetfillopacity{0.616596}%
\pgfsetlinewidth{1.003750pt}%
\definecolor{currentstroke}{rgb}{0.121569,0.466667,0.705882}%
\pgfsetstrokecolor{currentstroke}%
\pgfsetstrokeopacity{0.616596}%
\pgfsetdash{}{0pt}%
\pgfpathmoveto{\pgfqpoint{0.831052in}{1.213574in}}%
\pgfpathcurveto{\pgfqpoint{0.839288in}{1.213574in}}{\pgfqpoint{0.847189in}{1.216846in}}{\pgfqpoint{0.853012in}{1.222670in}}%
\pgfpathcurveto{\pgfqpoint{0.858836in}{1.228494in}}{\pgfqpoint{0.862109in}{1.236394in}}{\pgfqpoint{0.862109in}{1.244631in}}%
\pgfpathcurveto{\pgfqpoint{0.862109in}{1.252867in}}{\pgfqpoint{0.858836in}{1.260767in}}{\pgfqpoint{0.853012in}{1.266591in}}%
\pgfpathcurveto{\pgfqpoint{0.847189in}{1.272415in}}{\pgfqpoint{0.839288in}{1.275687in}}{\pgfqpoint{0.831052in}{1.275687in}}%
\pgfpathcurveto{\pgfqpoint{0.822816in}{1.275687in}}{\pgfqpoint{0.814916in}{1.272415in}}{\pgfqpoint{0.809092in}{1.266591in}}%
\pgfpathcurveto{\pgfqpoint{0.803268in}{1.260767in}}{\pgfqpoint{0.799996in}{1.252867in}}{\pgfqpoint{0.799996in}{1.244631in}}%
\pgfpathcurveto{\pgfqpoint{0.799996in}{1.236394in}}{\pgfqpoint{0.803268in}{1.228494in}}{\pgfqpoint{0.809092in}{1.222670in}}%
\pgfpathcurveto{\pgfqpoint{0.814916in}{1.216846in}}{\pgfqpoint{0.822816in}{1.213574in}}{\pgfqpoint{0.831052in}{1.213574in}}%
\pgfpathclose%
\pgfusepath{stroke,fill}%
\end{pgfscope}%
\begin{pgfscope}%
\pgfpathrectangle{\pgfqpoint{0.100000in}{0.220728in}}{\pgfqpoint{3.696000in}{3.696000in}}%
\pgfusepath{clip}%
\pgfsetbuttcap%
\pgfsetroundjoin%
\definecolor{currentfill}{rgb}{0.121569,0.466667,0.705882}%
\pgfsetfillcolor{currentfill}%
\pgfsetfillopacity{0.616596}%
\pgfsetlinewidth{1.003750pt}%
\definecolor{currentstroke}{rgb}{0.121569,0.466667,0.705882}%
\pgfsetstrokecolor{currentstroke}%
\pgfsetstrokeopacity{0.616596}%
\pgfsetdash{}{0pt}%
\pgfpathmoveto{\pgfqpoint{0.831052in}{1.213574in}}%
\pgfpathcurveto{\pgfqpoint{0.839288in}{1.213574in}}{\pgfqpoint{0.847189in}{1.216846in}}{\pgfqpoint{0.853012in}{1.222670in}}%
\pgfpathcurveto{\pgfqpoint{0.858836in}{1.228494in}}{\pgfqpoint{0.862109in}{1.236394in}}{\pgfqpoint{0.862109in}{1.244631in}}%
\pgfpathcurveto{\pgfqpoint{0.862109in}{1.252867in}}{\pgfqpoint{0.858836in}{1.260767in}}{\pgfqpoint{0.853012in}{1.266591in}}%
\pgfpathcurveto{\pgfqpoint{0.847189in}{1.272415in}}{\pgfqpoint{0.839288in}{1.275687in}}{\pgfqpoint{0.831052in}{1.275687in}}%
\pgfpathcurveto{\pgfqpoint{0.822816in}{1.275687in}}{\pgfqpoint{0.814916in}{1.272415in}}{\pgfqpoint{0.809092in}{1.266591in}}%
\pgfpathcurveto{\pgfqpoint{0.803268in}{1.260767in}}{\pgfqpoint{0.799996in}{1.252867in}}{\pgfqpoint{0.799996in}{1.244631in}}%
\pgfpathcurveto{\pgfqpoint{0.799996in}{1.236394in}}{\pgfqpoint{0.803268in}{1.228494in}}{\pgfqpoint{0.809092in}{1.222670in}}%
\pgfpathcurveto{\pgfqpoint{0.814916in}{1.216846in}}{\pgfqpoint{0.822816in}{1.213574in}}{\pgfqpoint{0.831052in}{1.213574in}}%
\pgfpathclose%
\pgfusepath{stroke,fill}%
\end{pgfscope}%
\begin{pgfscope}%
\pgfpathrectangle{\pgfqpoint{0.100000in}{0.220728in}}{\pgfqpoint{3.696000in}{3.696000in}}%
\pgfusepath{clip}%
\pgfsetbuttcap%
\pgfsetroundjoin%
\definecolor{currentfill}{rgb}{0.121569,0.466667,0.705882}%
\pgfsetfillcolor{currentfill}%
\pgfsetfillopacity{0.616596}%
\pgfsetlinewidth{1.003750pt}%
\definecolor{currentstroke}{rgb}{0.121569,0.466667,0.705882}%
\pgfsetstrokecolor{currentstroke}%
\pgfsetstrokeopacity{0.616596}%
\pgfsetdash{}{0pt}%
\pgfpathmoveto{\pgfqpoint{0.831052in}{1.213574in}}%
\pgfpathcurveto{\pgfqpoint{0.839288in}{1.213574in}}{\pgfqpoint{0.847189in}{1.216846in}}{\pgfqpoint{0.853012in}{1.222670in}}%
\pgfpathcurveto{\pgfqpoint{0.858836in}{1.228494in}}{\pgfqpoint{0.862109in}{1.236394in}}{\pgfqpoint{0.862109in}{1.244631in}}%
\pgfpathcurveto{\pgfqpoint{0.862109in}{1.252867in}}{\pgfqpoint{0.858836in}{1.260767in}}{\pgfqpoint{0.853012in}{1.266591in}}%
\pgfpathcurveto{\pgfqpoint{0.847189in}{1.272415in}}{\pgfqpoint{0.839288in}{1.275687in}}{\pgfqpoint{0.831052in}{1.275687in}}%
\pgfpathcurveto{\pgfqpoint{0.822816in}{1.275687in}}{\pgfqpoint{0.814916in}{1.272415in}}{\pgfqpoint{0.809092in}{1.266591in}}%
\pgfpathcurveto{\pgfqpoint{0.803268in}{1.260767in}}{\pgfqpoint{0.799996in}{1.252867in}}{\pgfqpoint{0.799996in}{1.244631in}}%
\pgfpathcurveto{\pgfqpoint{0.799996in}{1.236394in}}{\pgfqpoint{0.803268in}{1.228494in}}{\pgfqpoint{0.809092in}{1.222670in}}%
\pgfpathcurveto{\pgfqpoint{0.814916in}{1.216846in}}{\pgfqpoint{0.822816in}{1.213574in}}{\pgfqpoint{0.831052in}{1.213574in}}%
\pgfpathclose%
\pgfusepath{stroke,fill}%
\end{pgfscope}%
\begin{pgfscope}%
\pgfpathrectangle{\pgfqpoint{0.100000in}{0.220728in}}{\pgfqpoint{3.696000in}{3.696000in}}%
\pgfusepath{clip}%
\pgfsetbuttcap%
\pgfsetroundjoin%
\definecolor{currentfill}{rgb}{0.121569,0.466667,0.705882}%
\pgfsetfillcolor{currentfill}%
\pgfsetfillopacity{0.616596}%
\pgfsetlinewidth{1.003750pt}%
\definecolor{currentstroke}{rgb}{0.121569,0.466667,0.705882}%
\pgfsetstrokecolor{currentstroke}%
\pgfsetstrokeopacity{0.616596}%
\pgfsetdash{}{0pt}%
\pgfpathmoveto{\pgfqpoint{0.831052in}{1.213574in}}%
\pgfpathcurveto{\pgfqpoint{0.839288in}{1.213574in}}{\pgfqpoint{0.847189in}{1.216846in}}{\pgfqpoint{0.853012in}{1.222670in}}%
\pgfpathcurveto{\pgfqpoint{0.858836in}{1.228494in}}{\pgfqpoint{0.862109in}{1.236394in}}{\pgfqpoint{0.862109in}{1.244631in}}%
\pgfpathcurveto{\pgfqpoint{0.862109in}{1.252867in}}{\pgfqpoint{0.858836in}{1.260767in}}{\pgfqpoint{0.853012in}{1.266591in}}%
\pgfpathcurveto{\pgfqpoint{0.847189in}{1.272415in}}{\pgfqpoint{0.839288in}{1.275687in}}{\pgfqpoint{0.831052in}{1.275687in}}%
\pgfpathcurveto{\pgfqpoint{0.822816in}{1.275687in}}{\pgfqpoint{0.814916in}{1.272415in}}{\pgfqpoint{0.809092in}{1.266591in}}%
\pgfpathcurveto{\pgfqpoint{0.803268in}{1.260767in}}{\pgfqpoint{0.799996in}{1.252867in}}{\pgfqpoint{0.799996in}{1.244631in}}%
\pgfpathcurveto{\pgfqpoint{0.799996in}{1.236394in}}{\pgfqpoint{0.803268in}{1.228494in}}{\pgfqpoint{0.809092in}{1.222670in}}%
\pgfpathcurveto{\pgfqpoint{0.814916in}{1.216846in}}{\pgfqpoint{0.822816in}{1.213574in}}{\pgfqpoint{0.831052in}{1.213574in}}%
\pgfpathclose%
\pgfusepath{stroke,fill}%
\end{pgfscope}%
\begin{pgfscope}%
\pgfpathrectangle{\pgfqpoint{0.100000in}{0.220728in}}{\pgfqpoint{3.696000in}{3.696000in}}%
\pgfusepath{clip}%
\pgfsetbuttcap%
\pgfsetroundjoin%
\definecolor{currentfill}{rgb}{0.121569,0.466667,0.705882}%
\pgfsetfillcolor{currentfill}%
\pgfsetfillopacity{0.616596}%
\pgfsetlinewidth{1.003750pt}%
\definecolor{currentstroke}{rgb}{0.121569,0.466667,0.705882}%
\pgfsetstrokecolor{currentstroke}%
\pgfsetstrokeopacity{0.616596}%
\pgfsetdash{}{0pt}%
\pgfpathmoveto{\pgfqpoint{0.831052in}{1.213574in}}%
\pgfpathcurveto{\pgfqpoint{0.839288in}{1.213574in}}{\pgfqpoint{0.847189in}{1.216846in}}{\pgfqpoint{0.853012in}{1.222670in}}%
\pgfpathcurveto{\pgfqpoint{0.858836in}{1.228494in}}{\pgfqpoint{0.862109in}{1.236394in}}{\pgfqpoint{0.862109in}{1.244631in}}%
\pgfpathcurveto{\pgfqpoint{0.862109in}{1.252867in}}{\pgfqpoint{0.858836in}{1.260767in}}{\pgfqpoint{0.853012in}{1.266591in}}%
\pgfpathcurveto{\pgfqpoint{0.847189in}{1.272415in}}{\pgfqpoint{0.839288in}{1.275687in}}{\pgfqpoint{0.831052in}{1.275687in}}%
\pgfpathcurveto{\pgfqpoint{0.822816in}{1.275687in}}{\pgfqpoint{0.814916in}{1.272415in}}{\pgfqpoint{0.809092in}{1.266591in}}%
\pgfpathcurveto{\pgfqpoint{0.803268in}{1.260767in}}{\pgfqpoint{0.799996in}{1.252867in}}{\pgfqpoint{0.799996in}{1.244631in}}%
\pgfpathcurveto{\pgfqpoint{0.799996in}{1.236394in}}{\pgfqpoint{0.803268in}{1.228494in}}{\pgfqpoint{0.809092in}{1.222670in}}%
\pgfpathcurveto{\pgfqpoint{0.814916in}{1.216846in}}{\pgfqpoint{0.822816in}{1.213574in}}{\pgfqpoint{0.831052in}{1.213574in}}%
\pgfpathclose%
\pgfusepath{stroke,fill}%
\end{pgfscope}%
\begin{pgfscope}%
\pgfpathrectangle{\pgfqpoint{0.100000in}{0.220728in}}{\pgfqpoint{3.696000in}{3.696000in}}%
\pgfusepath{clip}%
\pgfsetbuttcap%
\pgfsetroundjoin%
\definecolor{currentfill}{rgb}{0.121569,0.466667,0.705882}%
\pgfsetfillcolor{currentfill}%
\pgfsetfillopacity{0.616596}%
\pgfsetlinewidth{1.003750pt}%
\definecolor{currentstroke}{rgb}{0.121569,0.466667,0.705882}%
\pgfsetstrokecolor{currentstroke}%
\pgfsetstrokeopacity{0.616596}%
\pgfsetdash{}{0pt}%
\pgfpathmoveto{\pgfqpoint{0.831052in}{1.213574in}}%
\pgfpathcurveto{\pgfqpoint{0.839288in}{1.213574in}}{\pgfqpoint{0.847189in}{1.216846in}}{\pgfqpoint{0.853012in}{1.222670in}}%
\pgfpathcurveto{\pgfqpoint{0.858836in}{1.228494in}}{\pgfqpoint{0.862109in}{1.236394in}}{\pgfqpoint{0.862109in}{1.244631in}}%
\pgfpathcurveto{\pgfqpoint{0.862109in}{1.252867in}}{\pgfqpoint{0.858836in}{1.260767in}}{\pgfqpoint{0.853012in}{1.266591in}}%
\pgfpathcurveto{\pgfqpoint{0.847189in}{1.272415in}}{\pgfqpoint{0.839288in}{1.275687in}}{\pgfqpoint{0.831052in}{1.275687in}}%
\pgfpathcurveto{\pgfqpoint{0.822816in}{1.275687in}}{\pgfqpoint{0.814916in}{1.272415in}}{\pgfqpoint{0.809092in}{1.266591in}}%
\pgfpathcurveto{\pgfqpoint{0.803268in}{1.260767in}}{\pgfqpoint{0.799996in}{1.252867in}}{\pgfqpoint{0.799996in}{1.244631in}}%
\pgfpathcurveto{\pgfqpoint{0.799996in}{1.236394in}}{\pgfqpoint{0.803268in}{1.228494in}}{\pgfqpoint{0.809092in}{1.222670in}}%
\pgfpathcurveto{\pgfqpoint{0.814916in}{1.216846in}}{\pgfqpoint{0.822816in}{1.213574in}}{\pgfqpoint{0.831052in}{1.213574in}}%
\pgfpathclose%
\pgfusepath{stroke,fill}%
\end{pgfscope}%
\begin{pgfscope}%
\pgfpathrectangle{\pgfqpoint{0.100000in}{0.220728in}}{\pgfqpoint{3.696000in}{3.696000in}}%
\pgfusepath{clip}%
\pgfsetbuttcap%
\pgfsetroundjoin%
\definecolor{currentfill}{rgb}{0.121569,0.466667,0.705882}%
\pgfsetfillcolor{currentfill}%
\pgfsetfillopacity{0.616596}%
\pgfsetlinewidth{1.003750pt}%
\definecolor{currentstroke}{rgb}{0.121569,0.466667,0.705882}%
\pgfsetstrokecolor{currentstroke}%
\pgfsetstrokeopacity{0.616596}%
\pgfsetdash{}{0pt}%
\pgfpathmoveto{\pgfqpoint{0.831052in}{1.213574in}}%
\pgfpathcurveto{\pgfqpoint{0.839288in}{1.213574in}}{\pgfqpoint{0.847189in}{1.216846in}}{\pgfqpoint{0.853012in}{1.222670in}}%
\pgfpathcurveto{\pgfqpoint{0.858836in}{1.228494in}}{\pgfqpoint{0.862109in}{1.236394in}}{\pgfqpoint{0.862109in}{1.244631in}}%
\pgfpathcurveto{\pgfqpoint{0.862109in}{1.252867in}}{\pgfqpoint{0.858836in}{1.260767in}}{\pgfqpoint{0.853012in}{1.266591in}}%
\pgfpathcurveto{\pgfqpoint{0.847189in}{1.272415in}}{\pgfqpoint{0.839288in}{1.275687in}}{\pgfqpoint{0.831052in}{1.275687in}}%
\pgfpathcurveto{\pgfqpoint{0.822816in}{1.275687in}}{\pgfqpoint{0.814916in}{1.272415in}}{\pgfqpoint{0.809092in}{1.266591in}}%
\pgfpathcurveto{\pgfqpoint{0.803268in}{1.260767in}}{\pgfqpoint{0.799996in}{1.252867in}}{\pgfqpoint{0.799996in}{1.244631in}}%
\pgfpathcurveto{\pgfqpoint{0.799996in}{1.236394in}}{\pgfqpoint{0.803268in}{1.228494in}}{\pgfqpoint{0.809092in}{1.222670in}}%
\pgfpathcurveto{\pgfqpoint{0.814916in}{1.216846in}}{\pgfqpoint{0.822816in}{1.213574in}}{\pgfqpoint{0.831052in}{1.213574in}}%
\pgfpathclose%
\pgfusepath{stroke,fill}%
\end{pgfscope}%
\begin{pgfscope}%
\pgfpathrectangle{\pgfqpoint{0.100000in}{0.220728in}}{\pgfqpoint{3.696000in}{3.696000in}}%
\pgfusepath{clip}%
\pgfsetbuttcap%
\pgfsetroundjoin%
\definecolor{currentfill}{rgb}{0.121569,0.466667,0.705882}%
\pgfsetfillcolor{currentfill}%
\pgfsetfillopacity{0.616596}%
\pgfsetlinewidth{1.003750pt}%
\definecolor{currentstroke}{rgb}{0.121569,0.466667,0.705882}%
\pgfsetstrokecolor{currentstroke}%
\pgfsetstrokeopacity{0.616596}%
\pgfsetdash{}{0pt}%
\pgfpathmoveto{\pgfqpoint{0.831052in}{1.213574in}}%
\pgfpathcurveto{\pgfqpoint{0.839288in}{1.213574in}}{\pgfqpoint{0.847189in}{1.216846in}}{\pgfqpoint{0.853012in}{1.222670in}}%
\pgfpathcurveto{\pgfqpoint{0.858836in}{1.228494in}}{\pgfqpoint{0.862109in}{1.236394in}}{\pgfqpoint{0.862109in}{1.244631in}}%
\pgfpathcurveto{\pgfqpoint{0.862109in}{1.252867in}}{\pgfqpoint{0.858836in}{1.260767in}}{\pgfqpoint{0.853012in}{1.266591in}}%
\pgfpathcurveto{\pgfqpoint{0.847189in}{1.272415in}}{\pgfqpoint{0.839288in}{1.275687in}}{\pgfqpoint{0.831052in}{1.275687in}}%
\pgfpathcurveto{\pgfqpoint{0.822816in}{1.275687in}}{\pgfqpoint{0.814916in}{1.272415in}}{\pgfqpoint{0.809092in}{1.266591in}}%
\pgfpathcurveto{\pgfqpoint{0.803268in}{1.260767in}}{\pgfqpoint{0.799996in}{1.252867in}}{\pgfqpoint{0.799996in}{1.244631in}}%
\pgfpathcurveto{\pgfqpoint{0.799996in}{1.236394in}}{\pgfqpoint{0.803268in}{1.228494in}}{\pgfqpoint{0.809092in}{1.222670in}}%
\pgfpathcurveto{\pgfqpoint{0.814916in}{1.216846in}}{\pgfqpoint{0.822816in}{1.213574in}}{\pgfqpoint{0.831052in}{1.213574in}}%
\pgfpathclose%
\pgfusepath{stroke,fill}%
\end{pgfscope}%
\begin{pgfscope}%
\pgfpathrectangle{\pgfqpoint{0.100000in}{0.220728in}}{\pgfqpoint{3.696000in}{3.696000in}}%
\pgfusepath{clip}%
\pgfsetbuttcap%
\pgfsetroundjoin%
\definecolor{currentfill}{rgb}{0.121569,0.466667,0.705882}%
\pgfsetfillcolor{currentfill}%
\pgfsetfillopacity{0.616596}%
\pgfsetlinewidth{1.003750pt}%
\definecolor{currentstroke}{rgb}{0.121569,0.466667,0.705882}%
\pgfsetstrokecolor{currentstroke}%
\pgfsetstrokeopacity{0.616596}%
\pgfsetdash{}{0pt}%
\pgfpathmoveto{\pgfqpoint{0.831052in}{1.213574in}}%
\pgfpathcurveto{\pgfqpoint{0.839288in}{1.213574in}}{\pgfqpoint{0.847189in}{1.216846in}}{\pgfqpoint{0.853012in}{1.222670in}}%
\pgfpathcurveto{\pgfqpoint{0.858836in}{1.228494in}}{\pgfqpoint{0.862109in}{1.236394in}}{\pgfqpoint{0.862109in}{1.244631in}}%
\pgfpathcurveto{\pgfqpoint{0.862109in}{1.252867in}}{\pgfqpoint{0.858836in}{1.260767in}}{\pgfqpoint{0.853012in}{1.266591in}}%
\pgfpathcurveto{\pgfqpoint{0.847189in}{1.272415in}}{\pgfqpoint{0.839288in}{1.275687in}}{\pgfqpoint{0.831052in}{1.275687in}}%
\pgfpathcurveto{\pgfqpoint{0.822816in}{1.275687in}}{\pgfqpoint{0.814916in}{1.272415in}}{\pgfqpoint{0.809092in}{1.266591in}}%
\pgfpathcurveto{\pgfqpoint{0.803268in}{1.260767in}}{\pgfqpoint{0.799996in}{1.252867in}}{\pgfqpoint{0.799996in}{1.244631in}}%
\pgfpathcurveto{\pgfqpoint{0.799996in}{1.236394in}}{\pgfqpoint{0.803268in}{1.228494in}}{\pgfqpoint{0.809092in}{1.222670in}}%
\pgfpathcurveto{\pgfqpoint{0.814916in}{1.216846in}}{\pgfqpoint{0.822816in}{1.213574in}}{\pgfqpoint{0.831052in}{1.213574in}}%
\pgfpathclose%
\pgfusepath{stroke,fill}%
\end{pgfscope}%
\begin{pgfscope}%
\pgfpathrectangle{\pgfqpoint{0.100000in}{0.220728in}}{\pgfqpoint{3.696000in}{3.696000in}}%
\pgfusepath{clip}%
\pgfsetbuttcap%
\pgfsetroundjoin%
\definecolor{currentfill}{rgb}{0.121569,0.466667,0.705882}%
\pgfsetfillcolor{currentfill}%
\pgfsetfillopacity{0.616596}%
\pgfsetlinewidth{1.003750pt}%
\definecolor{currentstroke}{rgb}{0.121569,0.466667,0.705882}%
\pgfsetstrokecolor{currentstroke}%
\pgfsetstrokeopacity{0.616596}%
\pgfsetdash{}{0pt}%
\pgfpathmoveto{\pgfqpoint{0.831052in}{1.213574in}}%
\pgfpathcurveto{\pgfqpoint{0.839288in}{1.213574in}}{\pgfqpoint{0.847189in}{1.216846in}}{\pgfqpoint{0.853012in}{1.222670in}}%
\pgfpathcurveto{\pgfqpoint{0.858836in}{1.228494in}}{\pgfqpoint{0.862109in}{1.236394in}}{\pgfqpoint{0.862109in}{1.244631in}}%
\pgfpathcurveto{\pgfqpoint{0.862109in}{1.252867in}}{\pgfqpoint{0.858836in}{1.260767in}}{\pgfqpoint{0.853012in}{1.266591in}}%
\pgfpathcurveto{\pgfqpoint{0.847189in}{1.272415in}}{\pgfqpoint{0.839288in}{1.275687in}}{\pgfqpoint{0.831052in}{1.275687in}}%
\pgfpathcurveto{\pgfqpoint{0.822816in}{1.275687in}}{\pgfqpoint{0.814916in}{1.272415in}}{\pgfqpoint{0.809092in}{1.266591in}}%
\pgfpathcurveto{\pgfqpoint{0.803268in}{1.260767in}}{\pgfqpoint{0.799996in}{1.252867in}}{\pgfqpoint{0.799996in}{1.244631in}}%
\pgfpathcurveto{\pgfqpoint{0.799996in}{1.236394in}}{\pgfqpoint{0.803268in}{1.228494in}}{\pgfqpoint{0.809092in}{1.222670in}}%
\pgfpathcurveto{\pgfqpoint{0.814916in}{1.216846in}}{\pgfqpoint{0.822816in}{1.213574in}}{\pgfqpoint{0.831052in}{1.213574in}}%
\pgfpathclose%
\pgfusepath{stroke,fill}%
\end{pgfscope}%
\begin{pgfscope}%
\pgfpathrectangle{\pgfqpoint{0.100000in}{0.220728in}}{\pgfqpoint{3.696000in}{3.696000in}}%
\pgfusepath{clip}%
\pgfsetbuttcap%
\pgfsetroundjoin%
\definecolor{currentfill}{rgb}{0.121569,0.466667,0.705882}%
\pgfsetfillcolor{currentfill}%
\pgfsetfillopacity{0.616596}%
\pgfsetlinewidth{1.003750pt}%
\definecolor{currentstroke}{rgb}{0.121569,0.466667,0.705882}%
\pgfsetstrokecolor{currentstroke}%
\pgfsetstrokeopacity{0.616596}%
\pgfsetdash{}{0pt}%
\pgfpathmoveto{\pgfqpoint{0.831052in}{1.213574in}}%
\pgfpathcurveto{\pgfqpoint{0.839288in}{1.213574in}}{\pgfqpoint{0.847189in}{1.216846in}}{\pgfqpoint{0.853012in}{1.222670in}}%
\pgfpathcurveto{\pgfqpoint{0.858836in}{1.228494in}}{\pgfqpoint{0.862109in}{1.236394in}}{\pgfqpoint{0.862109in}{1.244631in}}%
\pgfpathcurveto{\pgfqpoint{0.862109in}{1.252867in}}{\pgfqpoint{0.858836in}{1.260767in}}{\pgfqpoint{0.853012in}{1.266591in}}%
\pgfpathcurveto{\pgfqpoint{0.847189in}{1.272415in}}{\pgfqpoint{0.839288in}{1.275687in}}{\pgfqpoint{0.831052in}{1.275687in}}%
\pgfpathcurveto{\pgfqpoint{0.822816in}{1.275687in}}{\pgfqpoint{0.814916in}{1.272415in}}{\pgfqpoint{0.809092in}{1.266591in}}%
\pgfpathcurveto{\pgfqpoint{0.803268in}{1.260767in}}{\pgfqpoint{0.799996in}{1.252867in}}{\pgfqpoint{0.799996in}{1.244631in}}%
\pgfpathcurveto{\pgfqpoint{0.799996in}{1.236394in}}{\pgfqpoint{0.803268in}{1.228494in}}{\pgfqpoint{0.809092in}{1.222670in}}%
\pgfpathcurveto{\pgfqpoint{0.814916in}{1.216846in}}{\pgfqpoint{0.822816in}{1.213574in}}{\pgfqpoint{0.831052in}{1.213574in}}%
\pgfpathclose%
\pgfusepath{stroke,fill}%
\end{pgfscope}%
\begin{pgfscope}%
\pgfpathrectangle{\pgfqpoint{0.100000in}{0.220728in}}{\pgfqpoint{3.696000in}{3.696000in}}%
\pgfusepath{clip}%
\pgfsetbuttcap%
\pgfsetroundjoin%
\definecolor{currentfill}{rgb}{0.121569,0.466667,0.705882}%
\pgfsetfillcolor{currentfill}%
\pgfsetfillopacity{0.616596}%
\pgfsetlinewidth{1.003750pt}%
\definecolor{currentstroke}{rgb}{0.121569,0.466667,0.705882}%
\pgfsetstrokecolor{currentstroke}%
\pgfsetstrokeopacity{0.616596}%
\pgfsetdash{}{0pt}%
\pgfpathmoveto{\pgfqpoint{0.831052in}{1.213574in}}%
\pgfpathcurveto{\pgfqpoint{0.839288in}{1.213574in}}{\pgfqpoint{0.847189in}{1.216846in}}{\pgfqpoint{0.853012in}{1.222670in}}%
\pgfpathcurveto{\pgfqpoint{0.858836in}{1.228494in}}{\pgfqpoint{0.862109in}{1.236394in}}{\pgfqpoint{0.862109in}{1.244631in}}%
\pgfpathcurveto{\pgfqpoint{0.862109in}{1.252867in}}{\pgfqpoint{0.858836in}{1.260767in}}{\pgfqpoint{0.853012in}{1.266591in}}%
\pgfpathcurveto{\pgfqpoint{0.847189in}{1.272415in}}{\pgfqpoint{0.839288in}{1.275687in}}{\pgfqpoint{0.831052in}{1.275687in}}%
\pgfpathcurveto{\pgfqpoint{0.822816in}{1.275687in}}{\pgfqpoint{0.814916in}{1.272415in}}{\pgfqpoint{0.809092in}{1.266591in}}%
\pgfpathcurveto{\pgfqpoint{0.803268in}{1.260767in}}{\pgfqpoint{0.799996in}{1.252867in}}{\pgfqpoint{0.799996in}{1.244631in}}%
\pgfpathcurveto{\pgfqpoint{0.799996in}{1.236394in}}{\pgfqpoint{0.803268in}{1.228494in}}{\pgfqpoint{0.809092in}{1.222670in}}%
\pgfpathcurveto{\pgfqpoint{0.814916in}{1.216846in}}{\pgfqpoint{0.822816in}{1.213574in}}{\pgfqpoint{0.831052in}{1.213574in}}%
\pgfpathclose%
\pgfusepath{stroke,fill}%
\end{pgfscope}%
\begin{pgfscope}%
\pgfpathrectangle{\pgfqpoint{0.100000in}{0.220728in}}{\pgfqpoint{3.696000in}{3.696000in}}%
\pgfusepath{clip}%
\pgfsetbuttcap%
\pgfsetroundjoin%
\definecolor{currentfill}{rgb}{0.121569,0.466667,0.705882}%
\pgfsetfillcolor{currentfill}%
\pgfsetfillopacity{0.616596}%
\pgfsetlinewidth{1.003750pt}%
\definecolor{currentstroke}{rgb}{0.121569,0.466667,0.705882}%
\pgfsetstrokecolor{currentstroke}%
\pgfsetstrokeopacity{0.616596}%
\pgfsetdash{}{0pt}%
\pgfpathmoveto{\pgfqpoint{0.831052in}{1.213574in}}%
\pgfpathcurveto{\pgfqpoint{0.839288in}{1.213574in}}{\pgfqpoint{0.847189in}{1.216846in}}{\pgfqpoint{0.853012in}{1.222670in}}%
\pgfpathcurveto{\pgfqpoint{0.858836in}{1.228494in}}{\pgfqpoint{0.862109in}{1.236394in}}{\pgfqpoint{0.862109in}{1.244631in}}%
\pgfpathcurveto{\pgfqpoint{0.862109in}{1.252867in}}{\pgfqpoint{0.858836in}{1.260767in}}{\pgfqpoint{0.853012in}{1.266591in}}%
\pgfpathcurveto{\pgfqpoint{0.847189in}{1.272415in}}{\pgfqpoint{0.839288in}{1.275687in}}{\pgfqpoint{0.831052in}{1.275687in}}%
\pgfpathcurveto{\pgfqpoint{0.822816in}{1.275687in}}{\pgfqpoint{0.814916in}{1.272415in}}{\pgfqpoint{0.809092in}{1.266591in}}%
\pgfpathcurveto{\pgfqpoint{0.803268in}{1.260767in}}{\pgfqpoint{0.799996in}{1.252867in}}{\pgfqpoint{0.799996in}{1.244631in}}%
\pgfpathcurveto{\pgfqpoint{0.799996in}{1.236394in}}{\pgfqpoint{0.803268in}{1.228494in}}{\pgfqpoint{0.809092in}{1.222670in}}%
\pgfpathcurveto{\pgfqpoint{0.814916in}{1.216846in}}{\pgfqpoint{0.822816in}{1.213574in}}{\pgfqpoint{0.831052in}{1.213574in}}%
\pgfpathclose%
\pgfusepath{stroke,fill}%
\end{pgfscope}%
\begin{pgfscope}%
\pgfpathrectangle{\pgfqpoint{0.100000in}{0.220728in}}{\pgfqpoint{3.696000in}{3.696000in}}%
\pgfusepath{clip}%
\pgfsetbuttcap%
\pgfsetroundjoin%
\definecolor{currentfill}{rgb}{0.121569,0.466667,0.705882}%
\pgfsetfillcolor{currentfill}%
\pgfsetfillopacity{0.616596}%
\pgfsetlinewidth{1.003750pt}%
\definecolor{currentstroke}{rgb}{0.121569,0.466667,0.705882}%
\pgfsetstrokecolor{currentstroke}%
\pgfsetstrokeopacity{0.616596}%
\pgfsetdash{}{0pt}%
\pgfpathmoveto{\pgfqpoint{0.831052in}{1.213574in}}%
\pgfpathcurveto{\pgfqpoint{0.839288in}{1.213574in}}{\pgfqpoint{0.847189in}{1.216846in}}{\pgfqpoint{0.853012in}{1.222670in}}%
\pgfpathcurveto{\pgfqpoint{0.858836in}{1.228494in}}{\pgfqpoint{0.862109in}{1.236394in}}{\pgfqpoint{0.862109in}{1.244631in}}%
\pgfpathcurveto{\pgfqpoint{0.862109in}{1.252867in}}{\pgfqpoint{0.858836in}{1.260767in}}{\pgfqpoint{0.853012in}{1.266591in}}%
\pgfpathcurveto{\pgfqpoint{0.847189in}{1.272415in}}{\pgfqpoint{0.839288in}{1.275687in}}{\pgfqpoint{0.831052in}{1.275687in}}%
\pgfpathcurveto{\pgfqpoint{0.822816in}{1.275687in}}{\pgfqpoint{0.814916in}{1.272415in}}{\pgfqpoint{0.809092in}{1.266591in}}%
\pgfpathcurveto{\pgfqpoint{0.803268in}{1.260767in}}{\pgfqpoint{0.799996in}{1.252867in}}{\pgfqpoint{0.799996in}{1.244631in}}%
\pgfpathcurveto{\pgfqpoint{0.799996in}{1.236394in}}{\pgfqpoint{0.803268in}{1.228494in}}{\pgfqpoint{0.809092in}{1.222670in}}%
\pgfpathcurveto{\pgfqpoint{0.814916in}{1.216846in}}{\pgfqpoint{0.822816in}{1.213574in}}{\pgfqpoint{0.831052in}{1.213574in}}%
\pgfpathclose%
\pgfusepath{stroke,fill}%
\end{pgfscope}%
\begin{pgfscope}%
\pgfpathrectangle{\pgfqpoint{0.100000in}{0.220728in}}{\pgfqpoint{3.696000in}{3.696000in}}%
\pgfusepath{clip}%
\pgfsetbuttcap%
\pgfsetroundjoin%
\definecolor{currentfill}{rgb}{0.121569,0.466667,0.705882}%
\pgfsetfillcolor{currentfill}%
\pgfsetfillopacity{0.616596}%
\pgfsetlinewidth{1.003750pt}%
\definecolor{currentstroke}{rgb}{0.121569,0.466667,0.705882}%
\pgfsetstrokecolor{currentstroke}%
\pgfsetstrokeopacity{0.616596}%
\pgfsetdash{}{0pt}%
\pgfpathmoveto{\pgfqpoint{0.831052in}{1.213574in}}%
\pgfpathcurveto{\pgfqpoint{0.839288in}{1.213574in}}{\pgfqpoint{0.847189in}{1.216846in}}{\pgfqpoint{0.853012in}{1.222670in}}%
\pgfpathcurveto{\pgfqpoint{0.858836in}{1.228494in}}{\pgfqpoint{0.862109in}{1.236394in}}{\pgfqpoint{0.862109in}{1.244631in}}%
\pgfpathcurveto{\pgfqpoint{0.862109in}{1.252867in}}{\pgfqpoint{0.858836in}{1.260767in}}{\pgfqpoint{0.853012in}{1.266591in}}%
\pgfpathcurveto{\pgfqpoint{0.847189in}{1.272415in}}{\pgfqpoint{0.839288in}{1.275687in}}{\pgfqpoint{0.831052in}{1.275687in}}%
\pgfpathcurveto{\pgfqpoint{0.822816in}{1.275687in}}{\pgfqpoint{0.814916in}{1.272415in}}{\pgfqpoint{0.809092in}{1.266591in}}%
\pgfpathcurveto{\pgfqpoint{0.803268in}{1.260767in}}{\pgfqpoint{0.799996in}{1.252867in}}{\pgfqpoint{0.799996in}{1.244631in}}%
\pgfpathcurveto{\pgfqpoint{0.799996in}{1.236394in}}{\pgfqpoint{0.803268in}{1.228494in}}{\pgfqpoint{0.809092in}{1.222670in}}%
\pgfpathcurveto{\pgfqpoint{0.814916in}{1.216846in}}{\pgfqpoint{0.822816in}{1.213574in}}{\pgfqpoint{0.831052in}{1.213574in}}%
\pgfpathclose%
\pgfusepath{stroke,fill}%
\end{pgfscope}%
\begin{pgfscope}%
\pgfpathrectangle{\pgfqpoint{0.100000in}{0.220728in}}{\pgfqpoint{3.696000in}{3.696000in}}%
\pgfusepath{clip}%
\pgfsetbuttcap%
\pgfsetroundjoin%
\definecolor{currentfill}{rgb}{0.121569,0.466667,0.705882}%
\pgfsetfillcolor{currentfill}%
\pgfsetfillopacity{0.616596}%
\pgfsetlinewidth{1.003750pt}%
\definecolor{currentstroke}{rgb}{0.121569,0.466667,0.705882}%
\pgfsetstrokecolor{currentstroke}%
\pgfsetstrokeopacity{0.616596}%
\pgfsetdash{}{0pt}%
\pgfpathmoveto{\pgfqpoint{0.831052in}{1.213574in}}%
\pgfpathcurveto{\pgfqpoint{0.839288in}{1.213574in}}{\pgfqpoint{0.847189in}{1.216846in}}{\pgfqpoint{0.853012in}{1.222670in}}%
\pgfpathcurveto{\pgfqpoint{0.858836in}{1.228494in}}{\pgfqpoint{0.862109in}{1.236394in}}{\pgfqpoint{0.862109in}{1.244631in}}%
\pgfpathcurveto{\pgfqpoint{0.862109in}{1.252867in}}{\pgfqpoint{0.858836in}{1.260767in}}{\pgfqpoint{0.853012in}{1.266591in}}%
\pgfpathcurveto{\pgfqpoint{0.847189in}{1.272415in}}{\pgfqpoint{0.839288in}{1.275687in}}{\pgfqpoint{0.831052in}{1.275687in}}%
\pgfpathcurveto{\pgfqpoint{0.822816in}{1.275687in}}{\pgfqpoint{0.814916in}{1.272415in}}{\pgfqpoint{0.809092in}{1.266591in}}%
\pgfpathcurveto{\pgfqpoint{0.803268in}{1.260767in}}{\pgfqpoint{0.799996in}{1.252867in}}{\pgfqpoint{0.799996in}{1.244631in}}%
\pgfpathcurveto{\pgfqpoint{0.799996in}{1.236394in}}{\pgfqpoint{0.803268in}{1.228494in}}{\pgfqpoint{0.809092in}{1.222670in}}%
\pgfpathcurveto{\pgfqpoint{0.814916in}{1.216846in}}{\pgfqpoint{0.822816in}{1.213574in}}{\pgfqpoint{0.831052in}{1.213574in}}%
\pgfpathclose%
\pgfusepath{stroke,fill}%
\end{pgfscope}%
\begin{pgfscope}%
\pgfpathrectangle{\pgfqpoint{0.100000in}{0.220728in}}{\pgfqpoint{3.696000in}{3.696000in}}%
\pgfusepath{clip}%
\pgfsetbuttcap%
\pgfsetroundjoin%
\definecolor{currentfill}{rgb}{0.121569,0.466667,0.705882}%
\pgfsetfillcolor{currentfill}%
\pgfsetfillopacity{0.616596}%
\pgfsetlinewidth{1.003750pt}%
\definecolor{currentstroke}{rgb}{0.121569,0.466667,0.705882}%
\pgfsetstrokecolor{currentstroke}%
\pgfsetstrokeopacity{0.616596}%
\pgfsetdash{}{0pt}%
\pgfpathmoveto{\pgfqpoint{0.831052in}{1.213574in}}%
\pgfpathcurveto{\pgfqpoint{0.839288in}{1.213574in}}{\pgfqpoint{0.847189in}{1.216846in}}{\pgfqpoint{0.853012in}{1.222670in}}%
\pgfpathcurveto{\pgfqpoint{0.858836in}{1.228494in}}{\pgfqpoint{0.862109in}{1.236394in}}{\pgfqpoint{0.862109in}{1.244631in}}%
\pgfpathcurveto{\pgfqpoint{0.862109in}{1.252867in}}{\pgfqpoint{0.858836in}{1.260767in}}{\pgfqpoint{0.853012in}{1.266591in}}%
\pgfpathcurveto{\pgfqpoint{0.847189in}{1.272415in}}{\pgfqpoint{0.839288in}{1.275687in}}{\pgfqpoint{0.831052in}{1.275687in}}%
\pgfpathcurveto{\pgfqpoint{0.822816in}{1.275687in}}{\pgfqpoint{0.814916in}{1.272415in}}{\pgfqpoint{0.809092in}{1.266591in}}%
\pgfpathcurveto{\pgfqpoint{0.803268in}{1.260767in}}{\pgfqpoint{0.799996in}{1.252867in}}{\pgfqpoint{0.799996in}{1.244631in}}%
\pgfpathcurveto{\pgfqpoint{0.799996in}{1.236394in}}{\pgfqpoint{0.803268in}{1.228494in}}{\pgfqpoint{0.809092in}{1.222670in}}%
\pgfpathcurveto{\pgfqpoint{0.814916in}{1.216846in}}{\pgfqpoint{0.822816in}{1.213574in}}{\pgfqpoint{0.831052in}{1.213574in}}%
\pgfpathclose%
\pgfusepath{stroke,fill}%
\end{pgfscope}%
\begin{pgfscope}%
\pgfpathrectangle{\pgfqpoint{0.100000in}{0.220728in}}{\pgfqpoint{3.696000in}{3.696000in}}%
\pgfusepath{clip}%
\pgfsetbuttcap%
\pgfsetroundjoin%
\definecolor{currentfill}{rgb}{0.121569,0.466667,0.705882}%
\pgfsetfillcolor{currentfill}%
\pgfsetfillopacity{0.616596}%
\pgfsetlinewidth{1.003750pt}%
\definecolor{currentstroke}{rgb}{0.121569,0.466667,0.705882}%
\pgfsetstrokecolor{currentstroke}%
\pgfsetstrokeopacity{0.616596}%
\pgfsetdash{}{0pt}%
\pgfpathmoveto{\pgfqpoint{0.831052in}{1.213574in}}%
\pgfpathcurveto{\pgfqpoint{0.839288in}{1.213574in}}{\pgfqpoint{0.847189in}{1.216846in}}{\pgfqpoint{0.853012in}{1.222670in}}%
\pgfpathcurveto{\pgfqpoint{0.858836in}{1.228494in}}{\pgfqpoint{0.862109in}{1.236394in}}{\pgfqpoint{0.862109in}{1.244631in}}%
\pgfpathcurveto{\pgfqpoint{0.862109in}{1.252867in}}{\pgfqpoint{0.858836in}{1.260767in}}{\pgfqpoint{0.853012in}{1.266591in}}%
\pgfpathcurveto{\pgfqpoint{0.847189in}{1.272415in}}{\pgfqpoint{0.839288in}{1.275687in}}{\pgfqpoint{0.831052in}{1.275687in}}%
\pgfpathcurveto{\pgfqpoint{0.822816in}{1.275687in}}{\pgfqpoint{0.814916in}{1.272415in}}{\pgfqpoint{0.809092in}{1.266591in}}%
\pgfpathcurveto{\pgfqpoint{0.803268in}{1.260767in}}{\pgfqpoint{0.799996in}{1.252867in}}{\pgfqpoint{0.799996in}{1.244631in}}%
\pgfpathcurveto{\pgfqpoint{0.799996in}{1.236394in}}{\pgfqpoint{0.803268in}{1.228494in}}{\pgfqpoint{0.809092in}{1.222670in}}%
\pgfpathcurveto{\pgfqpoint{0.814916in}{1.216846in}}{\pgfqpoint{0.822816in}{1.213574in}}{\pgfqpoint{0.831052in}{1.213574in}}%
\pgfpathclose%
\pgfusepath{stroke,fill}%
\end{pgfscope}%
\begin{pgfscope}%
\pgfpathrectangle{\pgfqpoint{0.100000in}{0.220728in}}{\pgfqpoint{3.696000in}{3.696000in}}%
\pgfusepath{clip}%
\pgfsetbuttcap%
\pgfsetroundjoin%
\definecolor{currentfill}{rgb}{0.121569,0.466667,0.705882}%
\pgfsetfillcolor{currentfill}%
\pgfsetfillopacity{0.616596}%
\pgfsetlinewidth{1.003750pt}%
\definecolor{currentstroke}{rgb}{0.121569,0.466667,0.705882}%
\pgfsetstrokecolor{currentstroke}%
\pgfsetstrokeopacity{0.616596}%
\pgfsetdash{}{0pt}%
\pgfpathmoveto{\pgfqpoint{0.831052in}{1.213574in}}%
\pgfpathcurveto{\pgfqpoint{0.839288in}{1.213574in}}{\pgfqpoint{0.847189in}{1.216846in}}{\pgfqpoint{0.853012in}{1.222670in}}%
\pgfpathcurveto{\pgfqpoint{0.858836in}{1.228494in}}{\pgfqpoint{0.862109in}{1.236394in}}{\pgfqpoint{0.862109in}{1.244631in}}%
\pgfpathcurveto{\pgfqpoint{0.862109in}{1.252867in}}{\pgfqpoint{0.858836in}{1.260767in}}{\pgfqpoint{0.853012in}{1.266591in}}%
\pgfpathcurveto{\pgfqpoint{0.847189in}{1.272415in}}{\pgfqpoint{0.839288in}{1.275687in}}{\pgfqpoint{0.831052in}{1.275687in}}%
\pgfpathcurveto{\pgfqpoint{0.822816in}{1.275687in}}{\pgfqpoint{0.814916in}{1.272415in}}{\pgfqpoint{0.809092in}{1.266591in}}%
\pgfpathcurveto{\pgfqpoint{0.803268in}{1.260767in}}{\pgfqpoint{0.799996in}{1.252867in}}{\pgfqpoint{0.799996in}{1.244631in}}%
\pgfpathcurveto{\pgfqpoint{0.799996in}{1.236394in}}{\pgfqpoint{0.803268in}{1.228494in}}{\pgfqpoint{0.809092in}{1.222670in}}%
\pgfpathcurveto{\pgfqpoint{0.814916in}{1.216846in}}{\pgfqpoint{0.822816in}{1.213574in}}{\pgfqpoint{0.831052in}{1.213574in}}%
\pgfpathclose%
\pgfusepath{stroke,fill}%
\end{pgfscope}%
\begin{pgfscope}%
\pgfpathrectangle{\pgfqpoint{0.100000in}{0.220728in}}{\pgfqpoint{3.696000in}{3.696000in}}%
\pgfusepath{clip}%
\pgfsetbuttcap%
\pgfsetroundjoin%
\definecolor{currentfill}{rgb}{0.121569,0.466667,0.705882}%
\pgfsetfillcolor{currentfill}%
\pgfsetfillopacity{0.616596}%
\pgfsetlinewidth{1.003750pt}%
\definecolor{currentstroke}{rgb}{0.121569,0.466667,0.705882}%
\pgfsetstrokecolor{currentstroke}%
\pgfsetstrokeopacity{0.616596}%
\pgfsetdash{}{0pt}%
\pgfpathmoveto{\pgfqpoint{0.831052in}{1.213574in}}%
\pgfpathcurveto{\pgfqpoint{0.839288in}{1.213574in}}{\pgfqpoint{0.847189in}{1.216846in}}{\pgfqpoint{0.853012in}{1.222670in}}%
\pgfpathcurveto{\pgfqpoint{0.858836in}{1.228494in}}{\pgfqpoint{0.862109in}{1.236394in}}{\pgfqpoint{0.862109in}{1.244631in}}%
\pgfpathcurveto{\pgfqpoint{0.862109in}{1.252867in}}{\pgfqpoint{0.858836in}{1.260767in}}{\pgfqpoint{0.853012in}{1.266591in}}%
\pgfpathcurveto{\pgfqpoint{0.847189in}{1.272415in}}{\pgfqpoint{0.839288in}{1.275687in}}{\pgfqpoint{0.831052in}{1.275687in}}%
\pgfpathcurveto{\pgfqpoint{0.822816in}{1.275687in}}{\pgfqpoint{0.814916in}{1.272415in}}{\pgfqpoint{0.809092in}{1.266591in}}%
\pgfpathcurveto{\pgfqpoint{0.803268in}{1.260767in}}{\pgfqpoint{0.799996in}{1.252867in}}{\pgfqpoint{0.799996in}{1.244631in}}%
\pgfpathcurveto{\pgfqpoint{0.799996in}{1.236394in}}{\pgfqpoint{0.803268in}{1.228494in}}{\pgfqpoint{0.809092in}{1.222670in}}%
\pgfpathcurveto{\pgfqpoint{0.814916in}{1.216846in}}{\pgfqpoint{0.822816in}{1.213574in}}{\pgfqpoint{0.831052in}{1.213574in}}%
\pgfpathclose%
\pgfusepath{stroke,fill}%
\end{pgfscope}%
\begin{pgfscope}%
\pgfpathrectangle{\pgfqpoint{0.100000in}{0.220728in}}{\pgfqpoint{3.696000in}{3.696000in}}%
\pgfusepath{clip}%
\pgfsetbuttcap%
\pgfsetroundjoin%
\definecolor{currentfill}{rgb}{0.121569,0.466667,0.705882}%
\pgfsetfillcolor{currentfill}%
\pgfsetfillopacity{0.616596}%
\pgfsetlinewidth{1.003750pt}%
\definecolor{currentstroke}{rgb}{0.121569,0.466667,0.705882}%
\pgfsetstrokecolor{currentstroke}%
\pgfsetstrokeopacity{0.616596}%
\pgfsetdash{}{0pt}%
\pgfpathmoveto{\pgfqpoint{0.831052in}{1.213574in}}%
\pgfpathcurveto{\pgfqpoint{0.839288in}{1.213574in}}{\pgfqpoint{0.847189in}{1.216846in}}{\pgfqpoint{0.853012in}{1.222670in}}%
\pgfpathcurveto{\pgfqpoint{0.858836in}{1.228494in}}{\pgfqpoint{0.862109in}{1.236394in}}{\pgfqpoint{0.862109in}{1.244631in}}%
\pgfpathcurveto{\pgfqpoint{0.862109in}{1.252867in}}{\pgfqpoint{0.858836in}{1.260767in}}{\pgfqpoint{0.853012in}{1.266591in}}%
\pgfpathcurveto{\pgfqpoint{0.847189in}{1.272415in}}{\pgfqpoint{0.839288in}{1.275687in}}{\pgfqpoint{0.831052in}{1.275687in}}%
\pgfpathcurveto{\pgfqpoint{0.822816in}{1.275687in}}{\pgfqpoint{0.814916in}{1.272415in}}{\pgfqpoint{0.809092in}{1.266591in}}%
\pgfpathcurveto{\pgfqpoint{0.803268in}{1.260767in}}{\pgfqpoint{0.799996in}{1.252867in}}{\pgfqpoint{0.799996in}{1.244631in}}%
\pgfpathcurveto{\pgfqpoint{0.799996in}{1.236394in}}{\pgfqpoint{0.803268in}{1.228494in}}{\pgfqpoint{0.809092in}{1.222670in}}%
\pgfpathcurveto{\pgfqpoint{0.814916in}{1.216846in}}{\pgfqpoint{0.822816in}{1.213574in}}{\pgfqpoint{0.831052in}{1.213574in}}%
\pgfpathclose%
\pgfusepath{stroke,fill}%
\end{pgfscope}%
\begin{pgfscope}%
\pgfpathrectangle{\pgfqpoint{0.100000in}{0.220728in}}{\pgfqpoint{3.696000in}{3.696000in}}%
\pgfusepath{clip}%
\pgfsetbuttcap%
\pgfsetroundjoin%
\definecolor{currentfill}{rgb}{0.121569,0.466667,0.705882}%
\pgfsetfillcolor{currentfill}%
\pgfsetfillopacity{0.616596}%
\pgfsetlinewidth{1.003750pt}%
\definecolor{currentstroke}{rgb}{0.121569,0.466667,0.705882}%
\pgfsetstrokecolor{currentstroke}%
\pgfsetstrokeopacity{0.616596}%
\pgfsetdash{}{0pt}%
\pgfpathmoveto{\pgfqpoint{0.831052in}{1.213574in}}%
\pgfpathcurveto{\pgfqpoint{0.839288in}{1.213574in}}{\pgfqpoint{0.847189in}{1.216846in}}{\pgfqpoint{0.853012in}{1.222670in}}%
\pgfpathcurveto{\pgfqpoint{0.858836in}{1.228494in}}{\pgfqpoint{0.862109in}{1.236394in}}{\pgfqpoint{0.862109in}{1.244631in}}%
\pgfpathcurveto{\pgfqpoint{0.862109in}{1.252867in}}{\pgfqpoint{0.858836in}{1.260767in}}{\pgfqpoint{0.853012in}{1.266591in}}%
\pgfpathcurveto{\pgfqpoint{0.847189in}{1.272415in}}{\pgfqpoint{0.839288in}{1.275687in}}{\pgfqpoint{0.831052in}{1.275687in}}%
\pgfpathcurveto{\pgfqpoint{0.822816in}{1.275687in}}{\pgfqpoint{0.814916in}{1.272415in}}{\pgfqpoint{0.809092in}{1.266591in}}%
\pgfpathcurveto{\pgfqpoint{0.803268in}{1.260767in}}{\pgfqpoint{0.799996in}{1.252867in}}{\pgfqpoint{0.799996in}{1.244631in}}%
\pgfpathcurveto{\pgfqpoint{0.799996in}{1.236394in}}{\pgfqpoint{0.803268in}{1.228494in}}{\pgfqpoint{0.809092in}{1.222670in}}%
\pgfpathcurveto{\pgfqpoint{0.814916in}{1.216846in}}{\pgfqpoint{0.822816in}{1.213574in}}{\pgfqpoint{0.831052in}{1.213574in}}%
\pgfpathclose%
\pgfusepath{stroke,fill}%
\end{pgfscope}%
\begin{pgfscope}%
\pgfpathrectangle{\pgfqpoint{0.100000in}{0.220728in}}{\pgfqpoint{3.696000in}{3.696000in}}%
\pgfusepath{clip}%
\pgfsetbuttcap%
\pgfsetroundjoin%
\definecolor{currentfill}{rgb}{0.121569,0.466667,0.705882}%
\pgfsetfillcolor{currentfill}%
\pgfsetfillopacity{0.616596}%
\pgfsetlinewidth{1.003750pt}%
\definecolor{currentstroke}{rgb}{0.121569,0.466667,0.705882}%
\pgfsetstrokecolor{currentstroke}%
\pgfsetstrokeopacity{0.616596}%
\pgfsetdash{}{0pt}%
\pgfpathmoveto{\pgfqpoint{0.831052in}{1.213574in}}%
\pgfpathcurveto{\pgfqpoint{0.839288in}{1.213574in}}{\pgfqpoint{0.847189in}{1.216846in}}{\pgfqpoint{0.853012in}{1.222670in}}%
\pgfpathcurveto{\pgfqpoint{0.858836in}{1.228494in}}{\pgfqpoint{0.862109in}{1.236394in}}{\pgfqpoint{0.862109in}{1.244631in}}%
\pgfpathcurveto{\pgfqpoint{0.862109in}{1.252867in}}{\pgfqpoint{0.858836in}{1.260767in}}{\pgfqpoint{0.853012in}{1.266591in}}%
\pgfpathcurveto{\pgfqpoint{0.847189in}{1.272415in}}{\pgfqpoint{0.839288in}{1.275687in}}{\pgfqpoint{0.831052in}{1.275687in}}%
\pgfpathcurveto{\pgfqpoint{0.822816in}{1.275687in}}{\pgfqpoint{0.814916in}{1.272415in}}{\pgfqpoint{0.809092in}{1.266591in}}%
\pgfpathcurveto{\pgfqpoint{0.803268in}{1.260767in}}{\pgfqpoint{0.799996in}{1.252867in}}{\pgfqpoint{0.799996in}{1.244631in}}%
\pgfpathcurveto{\pgfqpoint{0.799996in}{1.236394in}}{\pgfqpoint{0.803268in}{1.228494in}}{\pgfqpoint{0.809092in}{1.222670in}}%
\pgfpathcurveto{\pgfqpoint{0.814916in}{1.216846in}}{\pgfqpoint{0.822816in}{1.213574in}}{\pgfqpoint{0.831052in}{1.213574in}}%
\pgfpathclose%
\pgfusepath{stroke,fill}%
\end{pgfscope}%
\begin{pgfscope}%
\pgfpathrectangle{\pgfqpoint{0.100000in}{0.220728in}}{\pgfqpoint{3.696000in}{3.696000in}}%
\pgfusepath{clip}%
\pgfsetbuttcap%
\pgfsetroundjoin%
\definecolor{currentfill}{rgb}{0.121569,0.466667,0.705882}%
\pgfsetfillcolor{currentfill}%
\pgfsetfillopacity{0.616596}%
\pgfsetlinewidth{1.003750pt}%
\definecolor{currentstroke}{rgb}{0.121569,0.466667,0.705882}%
\pgfsetstrokecolor{currentstroke}%
\pgfsetstrokeopacity{0.616596}%
\pgfsetdash{}{0pt}%
\pgfpathmoveto{\pgfqpoint{0.831052in}{1.213574in}}%
\pgfpathcurveto{\pgfqpoint{0.839288in}{1.213574in}}{\pgfqpoint{0.847189in}{1.216846in}}{\pgfqpoint{0.853012in}{1.222670in}}%
\pgfpathcurveto{\pgfqpoint{0.858836in}{1.228494in}}{\pgfqpoint{0.862109in}{1.236394in}}{\pgfqpoint{0.862109in}{1.244631in}}%
\pgfpathcurveto{\pgfqpoint{0.862109in}{1.252867in}}{\pgfqpoint{0.858836in}{1.260767in}}{\pgfqpoint{0.853012in}{1.266591in}}%
\pgfpathcurveto{\pgfqpoint{0.847189in}{1.272415in}}{\pgfqpoint{0.839288in}{1.275687in}}{\pgfqpoint{0.831052in}{1.275687in}}%
\pgfpathcurveto{\pgfqpoint{0.822816in}{1.275687in}}{\pgfqpoint{0.814916in}{1.272415in}}{\pgfqpoint{0.809092in}{1.266591in}}%
\pgfpathcurveto{\pgfqpoint{0.803268in}{1.260767in}}{\pgfqpoint{0.799996in}{1.252867in}}{\pgfqpoint{0.799996in}{1.244631in}}%
\pgfpathcurveto{\pgfqpoint{0.799996in}{1.236394in}}{\pgfqpoint{0.803268in}{1.228494in}}{\pgfqpoint{0.809092in}{1.222670in}}%
\pgfpathcurveto{\pgfqpoint{0.814916in}{1.216846in}}{\pgfqpoint{0.822816in}{1.213574in}}{\pgfqpoint{0.831052in}{1.213574in}}%
\pgfpathclose%
\pgfusepath{stroke,fill}%
\end{pgfscope}%
\begin{pgfscope}%
\pgfpathrectangle{\pgfqpoint{0.100000in}{0.220728in}}{\pgfqpoint{3.696000in}{3.696000in}}%
\pgfusepath{clip}%
\pgfsetbuttcap%
\pgfsetroundjoin%
\definecolor{currentfill}{rgb}{0.121569,0.466667,0.705882}%
\pgfsetfillcolor{currentfill}%
\pgfsetfillopacity{0.616596}%
\pgfsetlinewidth{1.003750pt}%
\definecolor{currentstroke}{rgb}{0.121569,0.466667,0.705882}%
\pgfsetstrokecolor{currentstroke}%
\pgfsetstrokeopacity{0.616596}%
\pgfsetdash{}{0pt}%
\pgfpathmoveto{\pgfqpoint{0.831052in}{1.213574in}}%
\pgfpathcurveto{\pgfqpoint{0.839288in}{1.213574in}}{\pgfqpoint{0.847189in}{1.216846in}}{\pgfqpoint{0.853012in}{1.222670in}}%
\pgfpathcurveto{\pgfqpoint{0.858836in}{1.228494in}}{\pgfqpoint{0.862109in}{1.236394in}}{\pgfqpoint{0.862109in}{1.244631in}}%
\pgfpathcurveto{\pgfqpoint{0.862109in}{1.252867in}}{\pgfqpoint{0.858836in}{1.260767in}}{\pgfqpoint{0.853012in}{1.266591in}}%
\pgfpathcurveto{\pgfqpoint{0.847189in}{1.272415in}}{\pgfqpoint{0.839288in}{1.275687in}}{\pgfqpoint{0.831052in}{1.275687in}}%
\pgfpathcurveto{\pgfqpoint{0.822816in}{1.275687in}}{\pgfqpoint{0.814916in}{1.272415in}}{\pgfqpoint{0.809092in}{1.266591in}}%
\pgfpathcurveto{\pgfqpoint{0.803268in}{1.260767in}}{\pgfqpoint{0.799996in}{1.252867in}}{\pgfqpoint{0.799996in}{1.244631in}}%
\pgfpathcurveto{\pgfqpoint{0.799996in}{1.236394in}}{\pgfqpoint{0.803268in}{1.228494in}}{\pgfqpoint{0.809092in}{1.222670in}}%
\pgfpathcurveto{\pgfqpoint{0.814916in}{1.216846in}}{\pgfqpoint{0.822816in}{1.213574in}}{\pgfqpoint{0.831052in}{1.213574in}}%
\pgfpathclose%
\pgfusepath{stroke,fill}%
\end{pgfscope}%
\begin{pgfscope}%
\pgfpathrectangle{\pgfqpoint{0.100000in}{0.220728in}}{\pgfqpoint{3.696000in}{3.696000in}}%
\pgfusepath{clip}%
\pgfsetbuttcap%
\pgfsetroundjoin%
\definecolor{currentfill}{rgb}{0.121569,0.466667,0.705882}%
\pgfsetfillcolor{currentfill}%
\pgfsetfillopacity{0.616596}%
\pgfsetlinewidth{1.003750pt}%
\definecolor{currentstroke}{rgb}{0.121569,0.466667,0.705882}%
\pgfsetstrokecolor{currentstroke}%
\pgfsetstrokeopacity{0.616596}%
\pgfsetdash{}{0pt}%
\pgfpathmoveto{\pgfqpoint{0.831052in}{1.213574in}}%
\pgfpathcurveto{\pgfqpoint{0.839288in}{1.213574in}}{\pgfqpoint{0.847189in}{1.216846in}}{\pgfqpoint{0.853012in}{1.222670in}}%
\pgfpathcurveto{\pgfqpoint{0.858836in}{1.228494in}}{\pgfqpoint{0.862109in}{1.236394in}}{\pgfqpoint{0.862109in}{1.244631in}}%
\pgfpathcurveto{\pgfqpoint{0.862109in}{1.252867in}}{\pgfqpoint{0.858836in}{1.260767in}}{\pgfqpoint{0.853012in}{1.266591in}}%
\pgfpathcurveto{\pgfqpoint{0.847189in}{1.272415in}}{\pgfqpoint{0.839288in}{1.275687in}}{\pgfqpoint{0.831052in}{1.275687in}}%
\pgfpathcurveto{\pgfqpoint{0.822816in}{1.275687in}}{\pgfqpoint{0.814916in}{1.272415in}}{\pgfqpoint{0.809092in}{1.266591in}}%
\pgfpathcurveto{\pgfqpoint{0.803268in}{1.260767in}}{\pgfqpoint{0.799996in}{1.252867in}}{\pgfqpoint{0.799996in}{1.244631in}}%
\pgfpathcurveto{\pgfqpoint{0.799996in}{1.236394in}}{\pgfqpoint{0.803268in}{1.228494in}}{\pgfqpoint{0.809092in}{1.222670in}}%
\pgfpathcurveto{\pgfqpoint{0.814916in}{1.216846in}}{\pgfqpoint{0.822816in}{1.213574in}}{\pgfqpoint{0.831052in}{1.213574in}}%
\pgfpathclose%
\pgfusepath{stroke,fill}%
\end{pgfscope}%
\begin{pgfscope}%
\pgfpathrectangle{\pgfqpoint{0.100000in}{0.220728in}}{\pgfqpoint{3.696000in}{3.696000in}}%
\pgfusepath{clip}%
\pgfsetbuttcap%
\pgfsetroundjoin%
\definecolor{currentfill}{rgb}{0.121569,0.466667,0.705882}%
\pgfsetfillcolor{currentfill}%
\pgfsetfillopacity{0.616596}%
\pgfsetlinewidth{1.003750pt}%
\definecolor{currentstroke}{rgb}{0.121569,0.466667,0.705882}%
\pgfsetstrokecolor{currentstroke}%
\pgfsetstrokeopacity{0.616596}%
\pgfsetdash{}{0pt}%
\pgfpathmoveto{\pgfqpoint{0.831052in}{1.213574in}}%
\pgfpathcurveto{\pgfqpoint{0.839288in}{1.213574in}}{\pgfqpoint{0.847189in}{1.216846in}}{\pgfqpoint{0.853012in}{1.222670in}}%
\pgfpathcurveto{\pgfqpoint{0.858836in}{1.228494in}}{\pgfqpoint{0.862109in}{1.236394in}}{\pgfqpoint{0.862109in}{1.244631in}}%
\pgfpathcurveto{\pgfqpoint{0.862109in}{1.252867in}}{\pgfqpoint{0.858836in}{1.260767in}}{\pgfqpoint{0.853012in}{1.266591in}}%
\pgfpathcurveto{\pgfqpoint{0.847189in}{1.272415in}}{\pgfqpoint{0.839288in}{1.275687in}}{\pgfqpoint{0.831052in}{1.275687in}}%
\pgfpathcurveto{\pgfqpoint{0.822816in}{1.275687in}}{\pgfqpoint{0.814916in}{1.272415in}}{\pgfqpoint{0.809092in}{1.266591in}}%
\pgfpathcurveto{\pgfqpoint{0.803268in}{1.260767in}}{\pgfqpoint{0.799996in}{1.252867in}}{\pgfqpoint{0.799996in}{1.244631in}}%
\pgfpathcurveto{\pgfqpoint{0.799996in}{1.236394in}}{\pgfqpoint{0.803268in}{1.228494in}}{\pgfqpoint{0.809092in}{1.222670in}}%
\pgfpathcurveto{\pgfqpoint{0.814916in}{1.216846in}}{\pgfqpoint{0.822816in}{1.213574in}}{\pgfqpoint{0.831052in}{1.213574in}}%
\pgfpathclose%
\pgfusepath{stroke,fill}%
\end{pgfscope}%
\begin{pgfscope}%
\pgfpathrectangle{\pgfqpoint{0.100000in}{0.220728in}}{\pgfqpoint{3.696000in}{3.696000in}}%
\pgfusepath{clip}%
\pgfsetbuttcap%
\pgfsetroundjoin%
\definecolor{currentfill}{rgb}{0.121569,0.466667,0.705882}%
\pgfsetfillcolor{currentfill}%
\pgfsetfillopacity{0.616596}%
\pgfsetlinewidth{1.003750pt}%
\definecolor{currentstroke}{rgb}{0.121569,0.466667,0.705882}%
\pgfsetstrokecolor{currentstroke}%
\pgfsetstrokeopacity{0.616596}%
\pgfsetdash{}{0pt}%
\pgfpathmoveto{\pgfqpoint{0.831052in}{1.213574in}}%
\pgfpathcurveto{\pgfqpoint{0.839288in}{1.213574in}}{\pgfqpoint{0.847189in}{1.216846in}}{\pgfqpoint{0.853012in}{1.222670in}}%
\pgfpathcurveto{\pgfqpoint{0.858836in}{1.228494in}}{\pgfqpoint{0.862109in}{1.236394in}}{\pgfqpoint{0.862109in}{1.244631in}}%
\pgfpathcurveto{\pgfqpoint{0.862109in}{1.252867in}}{\pgfqpoint{0.858836in}{1.260767in}}{\pgfqpoint{0.853012in}{1.266591in}}%
\pgfpathcurveto{\pgfqpoint{0.847189in}{1.272415in}}{\pgfqpoint{0.839288in}{1.275687in}}{\pgfqpoint{0.831052in}{1.275687in}}%
\pgfpathcurveto{\pgfqpoint{0.822816in}{1.275687in}}{\pgfqpoint{0.814916in}{1.272415in}}{\pgfqpoint{0.809092in}{1.266591in}}%
\pgfpathcurveto{\pgfqpoint{0.803268in}{1.260767in}}{\pgfqpoint{0.799996in}{1.252867in}}{\pgfqpoint{0.799996in}{1.244631in}}%
\pgfpathcurveto{\pgfqpoint{0.799996in}{1.236394in}}{\pgfqpoint{0.803268in}{1.228494in}}{\pgfqpoint{0.809092in}{1.222670in}}%
\pgfpathcurveto{\pgfqpoint{0.814916in}{1.216846in}}{\pgfqpoint{0.822816in}{1.213574in}}{\pgfqpoint{0.831052in}{1.213574in}}%
\pgfpathclose%
\pgfusepath{stroke,fill}%
\end{pgfscope}%
\begin{pgfscope}%
\pgfpathrectangle{\pgfqpoint{0.100000in}{0.220728in}}{\pgfqpoint{3.696000in}{3.696000in}}%
\pgfusepath{clip}%
\pgfsetbuttcap%
\pgfsetroundjoin%
\definecolor{currentfill}{rgb}{0.121569,0.466667,0.705882}%
\pgfsetfillcolor{currentfill}%
\pgfsetfillopacity{0.616596}%
\pgfsetlinewidth{1.003750pt}%
\definecolor{currentstroke}{rgb}{0.121569,0.466667,0.705882}%
\pgfsetstrokecolor{currentstroke}%
\pgfsetstrokeopacity{0.616596}%
\pgfsetdash{}{0pt}%
\pgfpathmoveto{\pgfqpoint{0.831052in}{1.213574in}}%
\pgfpathcurveto{\pgfqpoint{0.839288in}{1.213574in}}{\pgfqpoint{0.847189in}{1.216846in}}{\pgfqpoint{0.853012in}{1.222670in}}%
\pgfpathcurveto{\pgfqpoint{0.858836in}{1.228494in}}{\pgfqpoint{0.862109in}{1.236394in}}{\pgfqpoint{0.862109in}{1.244631in}}%
\pgfpathcurveto{\pgfqpoint{0.862109in}{1.252867in}}{\pgfqpoint{0.858836in}{1.260767in}}{\pgfqpoint{0.853012in}{1.266591in}}%
\pgfpathcurveto{\pgfqpoint{0.847189in}{1.272415in}}{\pgfqpoint{0.839288in}{1.275687in}}{\pgfqpoint{0.831052in}{1.275687in}}%
\pgfpathcurveto{\pgfqpoint{0.822816in}{1.275687in}}{\pgfqpoint{0.814916in}{1.272415in}}{\pgfqpoint{0.809092in}{1.266591in}}%
\pgfpathcurveto{\pgfqpoint{0.803268in}{1.260767in}}{\pgfqpoint{0.799996in}{1.252867in}}{\pgfqpoint{0.799996in}{1.244631in}}%
\pgfpathcurveto{\pgfqpoint{0.799996in}{1.236394in}}{\pgfqpoint{0.803268in}{1.228494in}}{\pgfqpoint{0.809092in}{1.222670in}}%
\pgfpathcurveto{\pgfqpoint{0.814916in}{1.216846in}}{\pgfqpoint{0.822816in}{1.213574in}}{\pgfqpoint{0.831052in}{1.213574in}}%
\pgfpathclose%
\pgfusepath{stroke,fill}%
\end{pgfscope}%
\begin{pgfscope}%
\pgfpathrectangle{\pgfqpoint{0.100000in}{0.220728in}}{\pgfqpoint{3.696000in}{3.696000in}}%
\pgfusepath{clip}%
\pgfsetbuttcap%
\pgfsetroundjoin%
\definecolor{currentfill}{rgb}{0.121569,0.466667,0.705882}%
\pgfsetfillcolor{currentfill}%
\pgfsetfillopacity{0.616596}%
\pgfsetlinewidth{1.003750pt}%
\definecolor{currentstroke}{rgb}{0.121569,0.466667,0.705882}%
\pgfsetstrokecolor{currentstroke}%
\pgfsetstrokeopacity{0.616596}%
\pgfsetdash{}{0pt}%
\pgfpathmoveto{\pgfqpoint{0.831052in}{1.213574in}}%
\pgfpathcurveto{\pgfqpoint{0.839288in}{1.213574in}}{\pgfqpoint{0.847189in}{1.216846in}}{\pgfqpoint{0.853012in}{1.222670in}}%
\pgfpathcurveto{\pgfqpoint{0.858836in}{1.228494in}}{\pgfqpoint{0.862109in}{1.236394in}}{\pgfqpoint{0.862109in}{1.244631in}}%
\pgfpathcurveto{\pgfqpoint{0.862109in}{1.252867in}}{\pgfqpoint{0.858836in}{1.260767in}}{\pgfqpoint{0.853012in}{1.266591in}}%
\pgfpathcurveto{\pgfqpoint{0.847189in}{1.272415in}}{\pgfqpoint{0.839288in}{1.275687in}}{\pgfqpoint{0.831052in}{1.275687in}}%
\pgfpathcurveto{\pgfqpoint{0.822816in}{1.275687in}}{\pgfqpoint{0.814916in}{1.272415in}}{\pgfqpoint{0.809092in}{1.266591in}}%
\pgfpathcurveto{\pgfqpoint{0.803268in}{1.260767in}}{\pgfqpoint{0.799996in}{1.252867in}}{\pgfqpoint{0.799996in}{1.244631in}}%
\pgfpathcurveto{\pgfqpoint{0.799996in}{1.236394in}}{\pgfqpoint{0.803268in}{1.228494in}}{\pgfqpoint{0.809092in}{1.222670in}}%
\pgfpathcurveto{\pgfqpoint{0.814916in}{1.216846in}}{\pgfqpoint{0.822816in}{1.213574in}}{\pgfqpoint{0.831052in}{1.213574in}}%
\pgfpathclose%
\pgfusepath{stroke,fill}%
\end{pgfscope}%
\begin{pgfscope}%
\pgfpathrectangle{\pgfqpoint{0.100000in}{0.220728in}}{\pgfqpoint{3.696000in}{3.696000in}}%
\pgfusepath{clip}%
\pgfsetbuttcap%
\pgfsetroundjoin%
\definecolor{currentfill}{rgb}{0.121569,0.466667,0.705882}%
\pgfsetfillcolor{currentfill}%
\pgfsetfillopacity{0.616596}%
\pgfsetlinewidth{1.003750pt}%
\definecolor{currentstroke}{rgb}{0.121569,0.466667,0.705882}%
\pgfsetstrokecolor{currentstroke}%
\pgfsetstrokeopacity{0.616596}%
\pgfsetdash{}{0pt}%
\pgfpathmoveto{\pgfqpoint{0.831052in}{1.213574in}}%
\pgfpathcurveto{\pgfqpoint{0.839288in}{1.213574in}}{\pgfqpoint{0.847189in}{1.216846in}}{\pgfqpoint{0.853012in}{1.222670in}}%
\pgfpathcurveto{\pgfqpoint{0.858836in}{1.228494in}}{\pgfqpoint{0.862109in}{1.236394in}}{\pgfqpoint{0.862109in}{1.244631in}}%
\pgfpathcurveto{\pgfqpoint{0.862109in}{1.252867in}}{\pgfqpoint{0.858836in}{1.260767in}}{\pgfqpoint{0.853012in}{1.266591in}}%
\pgfpathcurveto{\pgfqpoint{0.847189in}{1.272415in}}{\pgfqpoint{0.839288in}{1.275687in}}{\pgfqpoint{0.831052in}{1.275687in}}%
\pgfpathcurveto{\pgfqpoint{0.822816in}{1.275687in}}{\pgfqpoint{0.814916in}{1.272415in}}{\pgfqpoint{0.809092in}{1.266591in}}%
\pgfpathcurveto{\pgfqpoint{0.803268in}{1.260767in}}{\pgfqpoint{0.799996in}{1.252867in}}{\pgfqpoint{0.799996in}{1.244631in}}%
\pgfpathcurveto{\pgfqpoint{0.799996in}{1.236394in}}{\pgfqpoint{0.803268in}{1.228494in}}{\pgfqpoint{0.809092in}{1.222670in}}%
\pgfpathcurveto{\pgfqpoint{0.814916in}{1.216846in}}{\pgfqpoint{0.822816in}{1.213574in}}{\pgfqpoint{0.831052in}{1.213574in}}%
\pgfpathclose%
\pgfusepath{stroke,fill}%
\end{pgfscope}%
\begin{pgfscope}%
\pgfpathrectangle{\pgfqpoint{0.100000in}{0.220728in}}{\pgfqpoint{3.696000in}{3.696000in}}%
\pgfusepath{clip}%
\pgfsetbuttcap%
\pgfsetroundjoin%
\definecolor{currentfill}{rgb}{0.121569,0.466667,0.705882}%
\pgfsetfillcolor{currentfill}%
\pgfsetfillopacity{0.616596}%
\pgfsetlinewidth{1.003750pt}%
\definecolor{currentstroke}{rgb}{0.121569,0.466667,0.705882}%
\pgfsetstrokecolor{currentstroke}%
\pgfsetstrokeopacity{0.616596}%
\pgfsetdash{}{0pt}%
\pgfpathmoveto{\pgfqpoint{0.831052in}{1.213574in}}%
\pgfpathcurveto{\pgfqpoint{0.839288in}{1.213574in}}{\pgfqpoint{0.847189in}{1.216846in}}{\pgfqpoint{0.853012in}{1.222670in}}%
\pgfpathcurveto{\pgfqpoint{0.858836in}{1.228494in}}{\pgfqpoint{0.862109in}{1.236394in}}{\pgfqpoint{0.862109in}{1.244631in}}%
\pgfpathcurveto{\pgfqpoint{0.862109in}{1.252867in}}{\pgfqpoint{0.858836in}{1.260767in}}{\pgfqpoint{0.853012in}{1.266591in}}%
\pgfpathcurveto{\pgfqpoint{0.847189in}{1.272415in}}{\pgfqpoint{0.839288in}{1.275687in}}{\pgfqpoint{0.831052in}{1.275687in}}%
\pgfpathcurveto{\pgfqpoint{0.822816in}{1.275687in}}{\pgfqpoint{0.814916in}{1.272415in}}{\pgfqpoint{0.809092in}{1.266591in}}%
\pgfpathcurveto{\pgfqpoint{0.803268in}{1.260767in}}{\pgfqpoint{0.799996in}{1.252867in}}{\pgfqpoint{0.799996in}{1.244631in}}%
\pgfpathcurveto{\pgfqpoint{0.799996in}{1.236394in}}{\pgfqpoint{0.803268in}{1.228494in}}{\pgfqpoint{0.809092in}{1.222670in}}%
\pgfpathcurveto{\pgfqpoint{0.814916in}{1.216846in}}{\pgfqpoint{0.822816in}{1.213574in}}{\pgfqpoint{0.831052in}{1.213574in}}%
\pgfpathclose%
\pgfusepath{stroke,fill}%
\end{pgfscope}%
\begin{pgfscope}%
\pgfpathrectangle{\pgfqpoint{0.100000in}{0.220728in}}{\pgfqpoint{3.696000in}{3.696000in}}%
\pgfusepath{clip}%
\pgfsetbuttcap%
\pgfsetroundjoin%
\definecolor{currentfill}{rgb}{0.121569,0.466667,0.705882}%
\pgfsetfillcolor{currentfill}%
\pgfsetfillopacity{0.616596}%
\pgfsetlinewidth{1.003750pt}%
\definecolor{currentstroke}{rgb}{0.121569,0.466667,0.705882}%
\pgfsetstrokecolor{currentstroke}%
\pgfsetstrokeopacity{0.616596}%
\pgfsetdash{}{0pt}%
\pgfpathmoveto{\pgfqpoint{0.831052in}{1.213574in}}%
\pgfpathcurveto{\pgfqpoint{0.839288in}{1.213574in}}{\pgfqpoint{0.847189in}{1.216846in}}{\pgfqpoint{0.853012in}{1.222670in}}%
\pgfpathcurveto{\pgfqpoint{0.858836in}{1.228494in}}{\pgfqpoint{0.862109in}{1.236394in}}{\pgfqpoint{0.862109in}{1.244631in}}%
\pgfpathcurveto{\pgfqpoint{0.862109in}{1.252867in}}{\pgfqpoint{0.858836in}{1.260767in}}{\pgfqpoint{0.853012in}{1.266591in}}%
\pgfpathcurveto{\pgfqpoint{0.847189in}{1.272415in}}{\pgfqpoint{0.839288in}{1.275687in}}{\pgfqpoint{0.831052in}{1.275687in}}%
\pgfpathcurveto{\pgfqpoint{0.822816in}{1.275687in}}{\pgfqpoint{0.814916in}{1.272415in}}{\pgfqpoint{0.809092in}{1.266591in}}%
\pgfpathcurveto{\pgfqpoint{0.803268in}{1.260767in}}{\pgfqpoint{0.799996in}{1.252867in}}{\pgfqpoint{0.799996in}{1.244631in}}%
\pgfpathcurveto{\pgfqpoint{0.799996in}{1.236394in}}{\pgfqpoint{0.803268in}{1.228494in}}{\pgfqpoint{0.809092in}{1.222670in}}%
\pgfpathcurveto{\pgfqpoint{0.814916in}{1.216846in}}{\pgfqpoint{0.822816in}{1.213574in}}{\pgfqpoint{0.831052in}{1.213574in}}%
\pgfpathclose%
\pgfusepath{stroke,fill}%
\end{pgfscope}%
\begin{pgfscope}%
\pgfpathrectangle{\pgfqpoint{0.100000in}{0.220728in}}{\pgfqpoint{3.696000in}{3.696000in}}%
\pgfusepath{clip}%
\pgfsetbuttcap%
\pgfsetroundjoin%
\definecolor{currentfill}{rgb}{0.121569,0.466667,0.705882}%
\pgfsetfillcolor{currentfill}%
\pgfsetfillopacity{0.616596}%
\pgfsetlinewidth{1.003750pt}%
\definecolor{currentstroke}{rgb}{0.121569,0.466667,0.705882}%
\pgfsetstrokecolor{currentstroke}%
\pgfsetstrokeopacity{0.616596}%
\pgfsetdash{}{0pt}%
\pgfpathmoveto{\pgfqpoint{0.831052in}{1.213574in}}%
\pgfpathcurveto{\pgfqpoint{0.839288in}{1.213574in}}{\pgfqpoint{0.847189in}{1.216846in}}{\pgfqpoint{0.853012in}{1.222670in}}%
\pgfpathcurveto{\pgfqpoint{0.858836in}{1.228494in}}{\pgfqpoint{0.862109in}{1.236394in}}{\pgfqpoint{0.862109in}{1.244631in}}%
\pgfpathcurveto{\pgfqpoint{0.862109in}{1.252867in}}{\pgfqpoint{0.858836in}{1.260767in}}{\pgfqpoint{0.853012in}{1.266591in}}%
\pgfpathcurveto{\pgfqpoint{0.847189in}{1.272415in}}{\pgfqpoint{0.839288in}{1.275687in}}{\pgfqpoint{0.831052in}{1.275687in}}%
\pgfpathcurveto{\pgfqpoint{0.822816in}{1.275687in}}{\pgfqpoint{0.814916in}{1.272415in}}{\pgfqpoint{0.809092in}{1.266591in}}%
\pgfpathcurveto{\pgfqpoint{0.803268in}{1.260767in}}{\pgfqpoint{0.799996in}{1.252867in}}{\pgfqpoint{0.799996in}{1.244631in}}%
\pgfpathcurveto{\pgfqpoint{0.799996in}{1.236394in}}{\pgfqpoint{0.803268in}{1.228494in}}{\pgfqpoint{0.809092in}{1.222670in}}%
\pgfpathcurveto{\pgfqpoint{0.814916in}{1.216846in}}{\pgfqpoint{0.822816in}{1.213574in}}{\pgfqpoint{0.831052in}{1.213574in}}%
\pgfpathclose%
\pgfusepath{stroke,fill}%
\end{pgfscope}%
\begin{pgfscope}%
\pgfpathrectangle{\pgfqpoint{0.100000in}{0.220728in}}{\pgfqpoint{3.696000in}{3.696000in}}%
\pgfusepath{clip}%
\pgfsetbuttcap%
\pgfsetroundjoin%
\definecolor{currentfill}{rgb}{0.121569,0.466667,0.705882}%
\pgfsetfillcolor{currentfill}%
\pgfsetfillopacity{0.616596}%
\pgfsetlinewidth{1.003750pt}%
\definecolor{currentstroke}{rgb}{0.121569,0.466667,0.705882}%
\pgfsetstrokecolor{currentstroke}%
\pgfsetstrokeopacity{0.616596}%
\pgfsetdash{}{0pt}%
\pgfpathmoveto{\pgfqpoint{0.831052in}{1.213574in}}%
\pgfpathcurveto{\pgfqpoint{0.839288in}{1.213574in}}{\pgfqpoint{0.847189in}{1.216846in}}{\pgfqpoint{0.853012in}{1.222670in}}%
\pgfpathcurveto{\pgfqpoint{0.858836in}{1.228494in}}{\pgfqpoint{0.862109in}{1.236394in}}{\pgfqpoint{0.862109in}{1.244631in}}%
\pgfpathcurveto{\pgfqpoint{0.862109in}{1.252867in}}{\pgfqpoint{0.858836in}{1.260767in}}{\pgfqpoint{0.853012in}{1.266591in}}%
\pgfpathcurveto{\pgfqpoint{0.847189in}{1.272415in}}{\pgfqpoint{0.839288in}{1.275687in}}{\pgfqpoint{0.831052in}{1.275687in}}%
\pgfpathcurveto{\pgfqpoint{0.822816in}{1.275687in}}{\pgfqpoint{0.814916in}{1.272415in}}{\pgfqpoint{0.809092in}{1.266591in}}%
\pgfpathcurveto{\pgfqpoint{0.803268in}{1.260767in}}{\pgfqpoint{0.799996in}{1.252867in}}{\pgfqpoint{0.799996in}{1.244631in}}%
\pgfpathcurveto{\pgfqpoint{0.799996in}{1.236394in}}{\pgfqpoint{0.803268in}{1.228494in}}{\pgfqpoint{0.809092in}{1.222670in}}%
\pgfpathcurveto{\pgfqpoint{0.814916in}{1.216846in}}{\pgfqpoint{0.822816in}{1.213574in}}{\pgfqpoint{0.831052in}{1.213574in}}%
\pgfpathclose%
\pgfusepath{stroke,fill}%
\end{pgfscope}%
\begin{pgfscope}%
\pgfpathrectangle{\pgfqpoint{0.100000in}{0.220728in}}{\pgfqpoint{3.696000in}{3.696000in}}%
\pgfusepath{clip}%
\pgfsetbuttcap%
\pgfsetroundjoin%
\definecolor{currentfill}{rgb}{0.121569,0.466667,0.705882}%
\pgfsetfillcolor{currentfill}%
\pgfsetfillopacity{0.616596}%
\pgfsetlinewidth{1.003750pt}%
\definecolor{currentstroke}{rgb}{0.121569,0.466667,0.705882}%
\pgfsetstrokecolor{currentstroke}%
\pgfsetstrokeopacity{0.616596}%
\pgfsetdash{}{0pt}%
\pgfpathmoveto{\pgfqpoint{0.831052in}{1.213574in}}%
\pgfpathcurveto{\pgfqpoint{0.839288in}{1.213574in}}{\pgfqpoint{0.847189in}{1.216846in}}{\pgfqpoint{0.853012in}{1.222670in}}%
\pgfpathcurveto{\pgfqpoint{0.858836in}{1.228494in}}{\pgfqpoint{0.862109in}{1.236394in}}{\pgfqpoint{0.862109in}{1.244631in}}%
\pgfpathcurveto{\pgfqpoint{0.862109in}{1.252867in}}{\pgfqpoint{0.858836in}{1.260767in}}{\pgfqpoint{0.853012in}{1.266591in}}%
\pgfpathcurveto{\pgfqpoint{0.847189in}{1.272415in}}{\pgfqpoint{0.839288in}{1.275687in}}{\pgfqpoint{0.831052in}{1.275687in}}%
\pgfpathcurveto{\pgfqpoint{0.822816in}{1.275687in}}{\pgfqpoint{0.814916in}{1.272415in}}{\pgfqpoint{0.809092in}{1.266591in}}%
\pgfpathcurveto{\pgfqpoint{0.803268in}{1.260767in}}{\pgfqpoint{0.799996in}{1.252867in}}{\pgfqpoint{0.799996in}{1.244631in}}%
\pgfpathcurveto{\pgfqpoint{0.799996in}{1.236394in}}{\pgfqpoint{0.803268in}{1.228494in}}{\pgfqpoint{0.809092in}{1.222670in}}%
\pgfpathcurveto{\pgfqpoint{0.814916in}{1.216846in}}{\pgfqpoint{0.822816in}{1.213574in}}{\pgfqpoint{0.831052in}{1.213574in}}%
\pgfpathclose%
\pgfusepath{stroke,fill}%
\end{pgfscope}%
\begin{pgfscope}%
\pgfpathrectangle{\pgfqpoint{0.100000in}{0.220728in}}{\pgfqpoint{3.696000in}{3.696000in}}%
\pgfusepath{clip}%
\pgfsetbuttcap%
\pgfsetroundjoin%
\definecolor{currentfill}{rgb}{0.121569,0.466667,0.705882}%
\pgfsetfillcolor{currentfill}%
\pgfsetfillopacity{0.616596}%
\pgfsetlinewidth{1.003750pt}%
\definecolor{currentstroke}{rgb}{0.121569,0.466667,0.705882}%
\pgfsetstrokecolor{currentstroke}%
\pgfsetstrokeopacity{0.616596}%
\pgfsetdash{}{0pt}%
\pgfpathmoveto{\pgfqpoint{0.831052in}{1.213574in}}%
\pgfpathcurveto{\pgfqpoint{0.839288in}{1.213574in}}{\pgfqpoint{0.847189in}{1.216846in}}{\pgfqpoint{0.853012in}{1.222670in}}%
\pgfpathcurveto{\pgfqpoint{0.858836in}{1.228494in}}{\pgfqpoint{0.862109in}{1.236394in}}{\pgfqpoint{0.862109in}{1.244631in}}%
\pgfpathcurveto{\pgfqpoint{0.862109in}{1.252867in}}{\pgfqpoint{0.858836in}{1.260767in}}{\pgfqpoint{0.853012in}{1.266591in}}%
\pgfpathcurveto{\pgfqpoint{0.847189in}{1.272415in}}{\pgfqpoint{0.839288in}{1.275687in}}{\pgfqpoint{0.831052in}{1.275687in}}%
\pgfpathcurveto{\pgfqpoint{0.822816in}{1.275687in}}{\pgfqpoint{0.814916in}{1.272415in}}{\pgfqpoint{0.809092in}{1.266591in}}%
\pgfpathcurveto{\pgfqpoint{0.803268in}{1.260767in}}{\pgfqpoint{0.799996in}{1.252867in}}{\pgfqpoint{0.799996in}{1.244631in}}%
\pgfpathcurveto{\pgfqpoint{0.799996in}{1.236394in}}{\pgfqpoint{0.803268in}{1.228494in}}{\pgfqpoint{0.809092in}{1.222670in}}%
\pgfpathcurveto{\pgfqpoint{0.814916in}{1.216846in}}{\pgfqpoint{0.822816in}{1.213574in}}{\pgfqpoint{0.831052in}{1.213574in}}%
\pgfpathclose%
\pgfusepath{stroke,fill}%
\end{pgfscope}%
\begin{pgfscope}%
\pgfpathrectangle{\pgfqpoint{0.100000in}{0.220728in}}{\pgfqpoint{3.696000in}{3.696000in}}%
\pgfusepath{clip}%
\pgfsetbuttcap%
\pgfsetroundjoin%
\definecolor{currentfill}{rgb}{0.121569,0.466667,0.705882}%
\pgfsetfillcolor{currentfill}%
\pgfsetfillopacity{0.616596}%
\pgfsetlinewidth{1.003750pt}%
\definecolor{currentstroke}{rgb}{0.121569,0.466667,0.705882}%
\pgfsetstrokecolor{currentstroke}%
\pgfsetstrokeopacity{0.616596}%
\pgfsetdash{}{0pt}%
\pgfpathmoveto{\pgfqpoint{0.831052in}{1.213574in}}%
\pgfpathcurveto{\pgfqpoint{0.839288in}{1.213574in}}{\pgfqpoint{0.847189in}{1.216846in}}{\pgfqpoint{0.853012in}{1.222670in}}%
\pgfpathcurveto{\pgfqpoint{0.858836in}{1.228494in}}{\pgfqpoint{0.862109in}{1.236394in}}{\pgfqpoint{0.862109in}{1.244631in}}%
\pgfpathcurveto{\pgfqpoint{0.862109in}{1.252867in}}{\pgfqpoint{0.858836in}{1.260767in}}{\pgfqpoint{0.853012in}{1.266591in}}%
\pgfpathcurveto{\pgfqpoint{0.847189in}{1.272415in}}{\pgfqpoint{0.839288in}{1.275687in}}{\pgfqpoint{0.831052in}{1.275687in}}%
\pgfpathcurveto{\pgfqpoint{0.822816in}{1.275687in}}{\pgfqpoint{0.814916in}{1.272415in}}{\pgfqpoint{0.809092in}{1.266591in}}%
\pgfpathcurveto{\pgfqpoint{0.803268in}{1.260767in}}{\pgfqpoint{0.799996in}{1.252867in}}{\pgfqpoint{0.799996in}{1.244631in}}%
\pgfpathcurveto{\pgfqpoint{0.799996in}{1.236394in}}{\pgfqpoint{0.803268in}{1.228494in}}{\pgfqpoint{0.809092in}{1.222670in}}%
\pgfpathcurveto{\pgfqpoint{0.814916in}{1.216846in}}{\pgfqpoint{0.822816in}{1.213574in}}{\pgfqpoint{0.831052in}{1.213574in}}%
\pgfpathclose%
\pgfusepath{stroke,fill}%
\end{pgfscope}%
\begin{pgfscope}%
\pgfpathrectangle{\pgfqpoint{0.100000in}{0.220728in}}{\pgfqpoint{3.696000in}{3.696000in}}%
\pgfusepath{clip}%
\pgfsetbuttcap%
\pgfsetroundjoin%
\definecolor{currentfill}{rgb}{0.121569,0.466667,0.705882}%
\pgfsetfillcolor{currentfill}%
\pgfsetfillopacity{0.616596}%
\pgfsetlinewidth{1.003750pt}%
\definecolor{currentstroke}{rgb}{0.121569,0.466667,0.705882}%
\pgfsetstrokecolor{currentstroke}%
\pgfsetstrokeopacity{0.616596}%
\pgfsetdash{}{0pt}%
\pgfpathmoveto{\pgfqpoint{0.831052in}{1.213574in}}%
\pgfpathcurveto{\pgfqpoint{0.839288in}{1.213574in}}{\pgfqpoint{0.847189in}{1.216846in}}{\pgfqpoint{0.853012in}{1.222670in}}%
\pgfpathcurveto{\pgfqpoint{0.858836in}{1.228494in}}{\pgfqpoint{0.862109in}{1.236394in}}{\pgfqpoint{0.862109in}{1.244631in}}%
\pgfpathcurveto{\pgfqpoint{0.862109in}{1.252867in}}{\pgfqpoint{0.858836in}{1.260767in}}{\pgfqpoint{0.853012in}{1.266591in}}%
\pgfpathcurveto{\pgfqpoint{0.847189in}{1.272415in}}{\pgfqpoint{0.839288in}{1.275687in}}{\pgfqpoint{0.831052in}{1.275687in}}%
\pgfpathcurveto{\pgfqpoint{0.822816in}{1.275687in}}{\pgfqpoint{0.814916in}{1.272415in}}{\pgfqpoint{0.809092in}{1.266591in}}%
\pgfpathcurveto{\pgfqpoint{0.803268in}{1.260767in}}{\pgfqpoint{0.799996in}{1.252867in}}{\pgfqpoint{0.799996in}{1.244631in}}%
\pgfpathcurveto{\pgfqpoint{0.799996in}{1.236394in}}{\pgfqpoint{0.803268in}{1.228494in}}{\pgfqpoint{0.809092in}{1.222670in}}%
\pgfpathcurveto{\pgfqpoint{0.814916in}{1.216846in}}{\pgfqpoint{0.822816in}{1.213574in}}{\pgfqpoint{0.831052in}{1.213574in}}%
\pgfpathclose%
\pgfusepath{stroke,fill}%
\end{pgfscope}%
\begin{pgfscope}%
\pgfpathrectangle{\pgfqpoint{0.100000in}{0.220728in}}{\pgfqpoint{3.696000in}{3.696000in}}%
\pgfusepath{clip}%
\pgfsetbuttcap%
\pgfsetroundjoin%
\definecolor{currentfill}{rgb}{0.121569,0.466667,0.705882}%
\pgfsetfillcolor{currentfill}%
\pgfsetfillopacity{0.616596}%
\pgfsetlinewidth{1.003750pt}%
\definecolor{currentstroke}{rgb}{0.121569,0.466667,0.705882}%
\pgfsetstrokecolor{currentstroke}%
\pgfsetstrokeopacity{0.616596}%
\pgfsetdash{}{0pt}%
\pgfpathmoveto{\pgfqpoint{0.831052in}{1.213574in}}%
\pgfpathcurveto{\pgfqpoint{0.839288in}{1.213574in}}{\pgfqpoint{0.847189in}{1.216846in}}{\pgfqpoint{0.853012in}{1.222670in}}%
\pgfpathcurveto{\pgfqpoint{0.858836in}{1.228494in}}{\pgfqpoint{0.862109in}{1.236394in}}{\pgfqpoint{0.862109in}{1.244631in}}%
\pgfpathcurveto{\pgfqpoint{0.862109in}{1.252867in}}{\pgfqpoint{0.858836in}{1.260767in}}{\pgfqpoint{0.853012in}{1.266591in}}%
\pgfpathcurveto{\pgfqpoint{0.847189in}{1.272415in}}{\pgfqpoint{0.839288in}{1.275687in}}{\pgfqpoint{0.831052in}{1.275687in}}%
\pgfpathcurveto{\pgfqpoint{0.822816in}{1.275687in}}{\pgfqpoint{0.814916in}{1.272415in}}{\pgfqpoint{0.809092in}{1.266591in}}%
\pgfpathcurveto{\pgfqpoint{0.803268in}{1.260767in}}{\pgfqpoint{0.799996in}{1.252867in}}{\pgfqpoint{0.799996in}{1.244631in}}%
\pgfpathcurveto{\pgfqpoint{0.799996in}{1.236394in}}{\pgfqpoint{0.803268in}{1.228494in}}{\pgfqpoint{0.809092in}{1.222670in}}%
\pgfpathcurveto{\pgfqpoint{0.814916in}{1.216846in}}{\pgfqpoint{0.822816in}{1.213574in}}{\pgfqpoint{0.831052in}{1.213574in}}%
\pgfpathclose%
\pgfusepath{stroke,fill}%
\end{pgfscope}%
\begin{pgfscope}%
\pgfpathrectangle{\pgfqpoint{0.100000in}{0.220728in}}{\pgfqpoint{3.696000in}{3.696000in}}%
\pgfusepath{clip}%
\pgfsetbuttcap%
\pgfsetroundjoin%
\definecolor{currentfill}{rgb}{0.121569,0.466667,0.705882}%
\pgfsetfillcolor{currentfill}%
\pgfsetfillopacity{0.616596}%
\pgfsetlinewidth{1.003750pt}%
\definecolor{currentstroke}{rgb}{0.121569,0.466667,0.705882}%
\pgfsetstrokecolor{currentstroke}%
\pgfsetstrokeopacity{0.616596}%
\pgfsetdash{}{0pt}%
\pgfpathmoveto{\pgfqpoint{0.831052in}{1.213574in}}%
\pgfpathcurveto{\pgfqpoint{0.839288in}{1.213574in}}{\pgfqpoint{0.847189in}{1.216846in}}{\pgfqpoint{0.853012in}{1.222670in}}%
\pgfpathcurveto{\pgfqpoint{0.858836in}{1.228494in}}{\pgfqpoint{0.862109in}{1.236394in}}{\pgfqpoint{0.862109in}{1.244631in}}%
\pgfpathcurveto{\pgfqpoint{0.862109in}{1.252867in}}{\pgfqpoint{0.858836in}{1.260767in}}{\pgfqpoint{0.853012in}{1.266591in}}%
\pgfpathcurveto{\pgfqpoint{0.847189in}{1.272415in}}{\pgfqpoint{0.839288in}{1.275687in}}{\pgfqpoint{0.831052in}{1.275687in}}%
\pgfpathcurveto{\pgfqpoint{0.822816in}{1.275687in}}{\pgfqpoint{0.814916in}{1.272415in}}{\pgfqpoint{0.809092in}{1.266591in}}%
\pgfpathcurveto{\pgfqpoint{0.803268in}{1.260767in}}{\pgfqpoint{0.799996in}{1.252867in}}{\pgfqpoint{0.799996in}{1.244631in}}%
\pgfpathcurveto{\pgfqpoint{0.799996in}{1.236394in}}{\pgfqpoint{0.803268in}{1.228494in}}{\pgfqpoint{0.809092in}{1.222670in}}%
\pgfpathcurveto{\pgfqpoint{0.814916in}{1.216846in}}{\pgfqpoint{0.822816in}{1.213574in}}{\pgfqpoint{0.831052in}{1.213574in}}%
\pgfpathclose%
\pgfusepath{stroke,fill}%
\end{pgfscope}%
\begin{pgfscope}%
\pgfpathrectangle{\pgfqpoint{0.100000in}{0.220728in}}{\pgfqpoint{3.696000in}{3.696000in}}%
\pgfusepath{clip}%
\pgfsetbuttcap%
\pgfsetroundjoin%
\definecolor{currentfill}{rgb}{0.121569,0.466667,0.705882}%
\pgfsetfillcolor{currentfill}%
\pgfsetfillopacity{0.616596}%
\pgfsetlinewidth{1.003750pt}%
\definecolor{currentstroke}{rgb}{0.121569,0.466667,0.705882}%
\pgfsetstrokecolor{currentstroke}%
\pgfsetstrokeopacity{0.616596}%
\pgfsetdash{}{0pt}%
\pgfpathmoveto{\pgfqpoint{0.831052in}{1.213574in}}%
\pgfpathcurveto{\pgfqpoint{0.839288in}{1.213574in}}{\pgfqpoint{0.847189in}{1.216846in}}{\pgfqpoint{0.853012in}{1.222670in}}%
\pgfpathcurveto{\pgfqpoint{0.858836in}{1.228494in}}{\pgfqpoint{0.862109in}{1.236394in}}{\pgfqpoint{0.862109in}{1.244631in}}%
\pgfpathcurveto{\pgfqpoint{0.862109in}{1.252867in}}{\pgfqpoint{0.858836in}{1.260767in}}{\pgfqpoint{0.853012in}{1.266591in}}%
\pgfpathcurveto{\pgfqpoint{0.847189in}{1.272415in}}{\pgfqpoint{0.839288in}{1.275687in}}{\pgfqpoint{0.831052in}{1.275687in}}%
\pgfpathcurveto{\pgfqpoint{0.822816in}{1.275687in}}{\pgfqpoint{0.814916in}{1.272415in}}{\pgfqpoint{0.809092in}{1.266591in}}%
\pgfpathcurveto{\pgfqpoint{0.803268in}{1.260767in}}{\pgfqpoint{0.799996in}{1.252867in}}{\pgfqpoint{0.799996in}{1.244631in}}%
\pgfpathcurveto{\pgfqpoint{0.799996in}{1.236394in}}{\pgfqpoint{0.803268in}{1.228494in}}{\pgfqpoint{0.809092in}{1.222670in}}%
\pgfpathcurveto{\pgfqpoint{0.814916in}{1.216846in}}{\pgfqpoint{0.822816in}{1.213574in}}{\pgfqpoint{0.831052in}{1.213574in}}%
\pgfpathclose%
\pgfusepath{stroke,fill}%
\end{pgfscope}%
\begin{pgfscope}%
\pgfpathrectangle{\pgfqpoint{0.100000in}{0.220728in}}{\pgfqpoint{3.696000in}{3.696000in}}%
\pgfusepath{clip}%
\pgfsetbuttcap%
\pgfsetroundjoin%
\definecolor{currentfill}{rgb}{0.121569,0.466667,0.705882}%
\pgfsetfillcolor{currentfill}%
\pgfsetfillopacity{0.616596}%
\pgfsetlinewidth{1.003750pt}%
\definecolor{currentstroke}{rgb}{0.121569,0.466667,0.705882}%
\pgfsetstrokecolor{currentstroke}%
\pgfsetstrokeopacity{0.616596}%
\pgfsetdash{}{0pt}%
\pgfpathmoveto{\pgfqpoint{0.831052in}{1.213574in}}%
\pgfpathcurveto{\pgfqpoint{0.839288in}{1.213574in}}{\pgfqpoint{0.847189in}{1.216846in}}{\pgfqpoint{0.853012in}{1.222670in}}%
\pgfpathcurveto{\pgfqpoint{0.858836in}{1.228494in}}{\pgfqpoint{0.862109in}{1.236394in}}{\pgfqpoint{0.862109in}{1.244631in}}%
\pgfpathcurveto{\pgfqpoint{0.862109in}{1.252867in}}{\pgfqpoint{0.858836in}{1.260767in}}{\pgfqpoint{0.853012in}{1.266591in}}%
\pgfpathcurveto{\pgfqpoint{0.847189in}{1.272415in}}{\pgfqpoint{0.839288in}{1.275687in}}{\pgfqpoint{0.831052in}{1.275687in}}%
\pgfpathcurveto{\pgfqpoint{0.822816in}{1.275687in}}{\pgfqpoint{0.814916in}{1.272415in}}{\pgfqpoint{0.809092in}{1.266591in}}%
\pgfpathcurveto{\pgfqpoint{0.803268in}{1.260767in}}{\pgfqpoint{0.799996in}{1.252867in}}{\pgfqpoint{0.799996in}{1.244631in}}%
\pgfpathcurveto{\pgfqpoint{0.799996in}{1.236394in}}{\pgfqpoint{0.803268in}{1.228494in}}{\pgfqpoint{0.809092in}{1.222670in}}%
\pgfpathcurveto{\pgfqpoint{0.814916in}{1.216846in}}{\pgfqpoint{0.822816in}{1.213574in}}{\pgfqpoint{0.831052in}{1.213574in}}%
\pgfpathclose%
\pgfusepath{stroke,fill}%
\end{pgfscope}%
\begin{pgfscope}%
\pgfpathrectangle{\pgfqpoint{0.100000in}{0.220728in}}{\pgfqpoint{3.696000in}{3.696000in}}%
\pgfusepath{clip}%
\pgfsetbuttcap%
\pgfsetroundjoin%
\definecolor{currentfill}{rgb}{0.121569,0.466667,0.705882}%
\pgfsetfillcolor{currentfill}%
\pgfsetfillopacity{0.616596}%
\pgfsetlinewidth{1.003750pt}%
\definecolor{currentstroke}{rgb}{0.121569,0.466667,0.705882}%
\pgfsetstrokecolor{currentstroke}%
\pgfsetstrokeopacity{0.616596}%
\pgfsetdash{}{0pt}%
\pgfpathmoveto{\pgfqpoint{0.831052in}{1.213574in}}%
\pgfpathcurveto{\pgfqpoint{0.839288in}{1.213574in}}{\pgfqpoint{0.847189in}{1.216846in}}{\pgfqpoint{0.853012in}{1.222670in}}%
\pgfpathcurveto{\pgfqpoint{0.858836in}{1.228494in}}{\pgfqpoint{0.862109in}{1.236394in}}{\pgfqpoint{0.862109in}{1.244631in}}%
\pgfpathcurveto{\pgfqpoint{0.862109in}{1.252867in}}{\pgfqpoint{0.858836in}{1.260767in}}{\pgfqpoint{0.853012in}{1.266591in}}%
\pgfpathcurveto{\pgfqpoint{0.847189in}{1.272415in}}{\pgfqpoint{0.839288in}{1.275687in}}{\pgfqpoint{0.831052in}{1.275687in}}%
\pgfpathcurveto{\pgfqpoint{0.822816in}{1.275687in}}{\pgfqpoint{0.814916in}{1.272415in}}{\pgfqpoint{0.809092in}{1.266591in}}%
\pgfpathcurveto{\pgfqpoint{0.803268in}{1.260767in}}{\pgfqpoint{0.799996in}{1.252867in}}{\pgfqpoint{0.799996in}{1.244631in}}%
\pgfpathcurveto{\pgfqpoint{0.799996in}{1.236394in}}{\pgfqpoint{0.803268in}{1.228494in}}{\pgfqpoint{0.809092in}{1.222670in}}%
\pgfpathcurveto{\pgfqpoint{0.814916in}{1.216846in}}{\pgfqpoint{0.822816in}{1.213574in}}{\pgfqpoint{0.831052in}{1.213574in}}%
\pgfpathclose%
\pgfusepath{stroke,fill}%
\end{pgfscope}%
\begin{pgfscope}%
\pgfpathrectangle{\pgfqpoint{0.100000in}{0.220728in}}{\pgfqpoint{3.696000in}{3.696000in}}%
\pgfusepath{clip}%
\pgfsetbuttcap%
\pgfsetroundjoin%
\definecolor{currentfill}{rgb}{0.121569,0.466667,0.705882}%
\pgfsetfillcolor{currentfill}%
\pgfsetfillopacity{0.616596}%
\pgfsetlinewidth{1.003750pt}%
\definecolor{currentstroke}{rgb}{0.121569,0.466667,0.705882}%
\pgfsetstrokecolor{currentstroke}%
\pgfsetstrokeopacity{0.616596}%
\pgfsetdash{}{0pt}%
\pgfpathmoveto{\pgfqpoint{0.831052in}{1.213574in}}%
\pgfpathcurveto{\pgfqpoint{0.839288in}{1.213574in}}{\pgfqpoint{0.847189in}{1.216846in}}{\pgfqpoint{0.853012in}{1.222670in}}%
\pgfpathcurveto{\pgfqpoint{0.858836in}{1.228494in}}{\pgfqpoint{0.862109in}{1.236394in}}{\pgfqpoint{0.862109in}{1.244631in}}%
\pgfpathcurveto{\pgfqpoint{0.862109in}{1.252867in}}{\pgfqpoint{0.858836in}{1.260767in}}{\pgfqpoint{0.853012in}{1.266591in}}%
\pgfpathcurveto{\pgfqpoint{0.847189in}{1.272415in}}{\pgfqpoint{0.839288in}{1.275687in}}{\pgfqpoint{0.831052in}{1.275687in}}%
\pgfpathcurveto{\pgfqpoint{0.822816in}{1.275687in}}{\pgfqpoint{0.814916in}{1.272415in}}{\pgfqpoint{0.809092in}{1.266591in}}%
\pgfpathcurveto{\pgfqpoint{0.803268in}{1.260767in}}{\pgfqpoint{0.799996in}{1.252867in}}{\pgfqpoint{0.799996in}{1.244631in}}%
\pgfpathcurveto{\pgfqpoint{0.799996in}{1.236394in}}{\pgfqpoint{0.803268in}{1.228494in}}{\pgfqpoint{0.809092in}{1.222670in}}%
\pgfpathcurveto{\pgfqpoint{0.814916in}{1.216846in}}{\pgfqpoint{0.822816in}{1.213574in}}{\pgfqpoint{0.831052in}{1.213574in}}%
\pgfpathclose%
\pgfusepath{stroke,fill}%
\end{pgfscope}%
\begin{pgfscope}%
\pgfpathrectangle{\pgfqpoint{0.100000in}{0.220728in}}{\pgfqpoint{3.696000in}{3.696000in}}%
\pgfusepath{clip}%
\pgfsetbuttcap%
\pgfsetroundjoin%
\definecolor{currentfill}{rgb}{0.121569,0.466667,0.705882}%
\pgfsetfillcolor{currentfill}%
\pgfsetfillopacity{0.616596}%
\pgfsetlinewidth{1.003750pt}%
\definecolor{currentstroke}{rgb}{0.121569,0.466667,0.705882}%
\pgfsetstrokecolor{currentstroke}%
\pgfsetstrokeopacity{0.616596}%
\pgfsetdash{}{0pt}%
\pgfpathmoveto{\pgfqpoint{0.831052in}{1.213574in}}%
\pgfpathcurveto{\pgfqpoint{0.839288in}{1.213574in}}{\pgfqpoint{0.847189in}{1.216846in}}{\pgfqpoint{0.853012in}{1.222670in}}%
\pgfpathcurveto{\pgfqpoint{0.858836in}{1.228494in}}{\pgfqpoint{0.862109in}{1.236394in}}{\pgfqpoint{0.862109in}{1.244631in}}%
\pgfpathcurveto{\pgfqpoint{0.862109in}{1.252867in}}{\pgfqpoint{0.858836in}{1.260767in}}{\pgfqpoint{0.853012in}{1.266591in}}%
\pgfpathcurveto{\pgfqpoint{0.847189in}{1.272415in}}{\pgfqpoint{0.839288in}{1.275687in}}{\pgfqpoint{0.831052in}{1.275687in}}%
\pgfpathcurveto{\pgfqpoint{0.822816in}{1.275687in}}{\pgfqpoint{0.814916in}{1.272415in}}{\pgfqpoint{0.809092in}{1.266591in}}%
\pgfpathcurveto{\pgfqpoint{0.803268in}{1.260767in}}{\pgfqpoint{0.799996in}{1.252867in}}{\pgfqpoint{0.799996in}{1.244631in}}%
\pgfpathcurveto{\pgfqpoint{0.799996in}{1.236394in}}{\pgfqpoint{0.803268in}{1.228494in}}{\pgfqpoint{0.809092in}{1.222670in}}%
\pgfpathcurveto{\pgfqpoint{0.814916in}{1.216846in}}{\pgfqpoint{0.822816in}{1.213574in}}{\pgfqpoint{0.831052in}{1.213574in}}%
\pgfpathclose%
\pgfusepath{stroke,fill}%
\end{pgfscope}%
\begin{pgfscope}%
\pgfpathrectangle{\pgfqpoint{0.100000in}{0.220728in}}{\pgfqpoint{3.696000in}{3.696000in}}%
\pgfusepath{clip}%
\pgfsetbuttcap%
\pgfsetroundjoin%
\definecolor{currentfill}{rgb}{0.121569,0.466667,0.705882}%
\pgfsetfillcolor{currentfill}%
\pgfsetfillopacity{0.616596}%
\pgfsetlinewidth{1.003750pt}%
\definecolor{currentstroke}{rgb}{0.121569,0.466667,0.705882}%
\pgfsetstrokecolor{currentstroke}%
\pgfsetstrokeopacity{0.616596}%
\pgfsetdash{}{0pt}%
\pgfpathmoveto{\pgfqpoint{0.831052in}{1.213574in}}%
\pgfpathcurveto{\pgfqpoint{0.839288in}{1.213574in}}{\pgfqpoint{0.847189in}{1.216846in}}{\pgfqpoint{0.853012in}{1.222670in}}%
\pgfpathcurveto{\pgfqpoint{0.858836in}{1.228494in}}{\pgfqpoint{0.862109in}{1.236394in}}{\pgfqpoint{0.862109in}{1.244631in}}%
\pgfpathcurveto{\pgfqpoint{0.862109in}{1.252867in}}{\pgfqpoint{0.858836in}{1.260767in}}{\pgfqpoint{0.853012in}{1.266591in}}%
\pgfpathcurveto{\pgfqpoint{0.847189in}{1.272415in}}{\pgfqpoint{0.839288in}{1.275687in}}{\pgfqpoint{0.831052in}{1.275687in}}%
\pgfpathcurveto{\pgfqpoint{0.822816in}{1.275687in}}{\pgfqpoint{0.814916in}{1.272415in}}{\pgfqpoint{0.809092in}{1.266591in}}%
\pgfpathcurveto{\pgfqpoint{0.803268in}{1.260767in}}{\pgfqpoint{0.799996in}{1.252867in}}{\pgfqpoint{0.799996in}{1.244631in}}%
\pgfpathcurveto{\pgfqpoint{0.799996in}{1.236394in}}{\pgfqpoint{0.803268in}{1.228494in}}{\pgfqpoint{0.809092in}{1.222670in}}%
\pgfpathcurveto{\pgfqpoint{0.814916in}{1.216846in}}{\pgfqpoint{0.822816in}{1.213574in}}{\pgfqpoint{0.831052in}{1.213574in}}%
\pgfpathclose%
\pgfusepath{stroke,fill}%
\end{pgfscope}%
\begin{pgfscope}%
\pgfpathrectangle{\pgfqpoint{0.100000in}{0.220728in}}{\pgfqpoint{3.696000in}{3.696000in}}%
\pgfusepath{clip}%
\pgfsetbuttcap%
\pgfsetroundjoin%
\definecolor{currentfill}{rgb}{0.121569,0.466667,0.705882}%
\pgfsetfillcolor{currentfill}%
\pgfsetfillopacity{0.616596}%
\pgfsetlinewidth{1.003750pt}%
\definecolor{currentstroke}{rgb}{0.121569,0.466667,0.705882}%
\pgfsetstrokecolor{currentstroke}%
\pgfsetstrokeopacity{0.616596}%
\pgfsetdash{}{0pt}%
\pgfpathmoveto{\pgfqpoint{0.831052in}{1.213574in}}%
\pgfpathcurveto{\pgfqpoint{0.839288in}{1.213574in}}{\pgfqpoint{0.847189in}{1.216846in}}{\pgfqpoint{0.853012in}{1.222670in}}%
\pgfpathcurveto{\pgfqpoint{0.858836in}{1.228494in}}{\pgfqpoint{0.862109in}{1.236394in}}{\pgfqpoint{0.862109in}{1.244631in}}%
\pgfpathcurveto{\pgfqpoint{0.862109in}{1.252867in}}{\pgfqpoint{0.858836in}{1.260767in}}{\pgfqpoint{0.853012in}{1.266591in}}%
\pgfpathcurveto{\pgfqpoint{0.847189in}{1.272415in}}{\pgfqpoint{0.839288in}{1.275687in}}{\pgfqpoint{0.831052in}{1.275687in}}%
\pgfpathcurveto{\pgfqpoint{0.822816in}{1.275687in}}{\pgfqpoint{0.814916in}{1.272415in}}{\pgfqpoint{0.809092in}{1.266591in}}%
\pgfpathcurveto{\pgfqpoint{0.803268in}{1.260767in}}{\pgfqpoint{0.799996in}{1.252867in}}{\pgfqpoint{0.799996in}{1.244631in}}%
\pgfpathcurveto{\pgfqpoint{0.799996in}{1.236394in}}{\pgfqpoint{0.803268in}{1.228494in}}{\pgfqpoint{0.809092in}{1.222670in}}%
\pgfpathcurveto{\pgfqpoint{0.814916in}{1.216846in}}{\pgfqpoint{0.822816in}{1.213574in}}{\pgfqpoint{0.831052in}{1.213574in}}%
\pgfpathclose%
\pgfusepath{stroke,fill}%
\end{pgfscope}%
\begin{pgfscope}%
\pgfpathrectangle{\pgfqpoint{0.100000in}{0.220728in}}{\pgfqpoint{3.696000in}{3.696000in}}%
\pgfusepath{clip}%
\pgfsetbuttcap%
\pgfsetroundjoin%
\definecolor{currentfill}{rgb}{0.121569,0.466667,0.705882}%
\pgfsetfillcolor{currentfill}%
\pgfsetfillopacity{0.616596}%
\pgfsetlinewidth{1.003750pt}%
\definecolor{currentstroke}{rgb}{0.121569,0.466667,0.705882}%
\pgfsetstrokecolor{currentstroke}%
\pgfsetstrokeopacity{0.616596}%
\pgfsetdash{}{0pt}%
\pgfpathmoveto{\pgfqpoint{0.831052in}{1.213574in}}%
\pgfpathcurveto{\pgfqpoint{0.839288in}{1.213574in}}{\pgfqpoint{0.847189in}{1.216846in}}{\pgfqpoint{0.853012in}{1.222670in}}%
\pgfpathcurveto{\pgfqpoint{0.858836in}{1.228494in}}{\pgfqpoint{0.862109in}{1.236394in}}{\pgfqpoint{0.862109in}{1.244631in}}%
\pgfpathcurveto{\pgfqpoint{0.862109in}{1.252867in}}{\pgfqpoint{0.858836in}{1.260767in}}{\pgfqpoint{0.853012in}{1.266591in}}%
\pgfpathcurveto{\pgfqpoint{0.847189in}{1.272415in}}{\pgfqpoint{0.839288in}{1.275687in}}{\pgfqpoint{0.831052in}{1.275687in}}%
\pgfpathcurveto{\pgfqpoint{0.822816in}{1.275687in}}{\pgfqpoint{0.814916in}{1.272415in}}{\pgfqpoint{0.809092in}{1.266591in}}%
\pgfpathcurveto{\pgfqpoint{0.803268in}{1.260767in}}{\pgfqpoint{0.799996in}{1.252867in}}{\pgfqpoint{0.799996in}{1.244631in}}%
\pgfpathcurveto{\pgfqpoint{0.799996in}{1.236394in}}{\pgfqpoint{0.803268in}{1.228494in}}{\pgfqpoint{0.809092in}{1.222670in}}%
\pgfpathcurveto{\pgfqpoint{0.814916in}{1.216846in}}{\pgfqpoint{0.822816in}{1.213574in}}{\pgfqpoint{0.831052in}{1.213574in}}%
\pgfpathclose%
\pgfusepath{stroke,fill}%
\end{pgfscope}%
\begin{pgfscope}%
\pgfpathrectangle{\pgfqpoint{0.100000in}{0.220728in}}{\pgfqpoint{3.696000in}{3.696000in}}%
\pgfusepath{clip}%
\pgfsetbuttcap%
\pgfsetroundjoin%
\definecolor{currentfill}{rgb}{0.121569,0.466667,0.705882}%
\pgfsetfillcolor{currentfill}%
\pgfsetfillopacity{0.616596}%
\pgfsetlinewidth{1.003750pt}%
\definecolor{currentstroke}{rgb}{0.121569,0.466667,0.705882}%
\pgfsetstrokecolor{currentstroke}%
\pgfsetstrokeopacity{0.616596}%
\pgfsetdash{}{0pt}%
\pgfpathmoveto{\pgfqpoint{0.831052in}{1.213574in}}%
\pgfpathcurveto{\pgfqpoint{0.839288in}{1.213574in}}{\pgfqpoint{0.847189in}{1.216846in}}{\pgfqpoint{0.853012in}{1.222670in}}%
\pgfpathcurveto{\pgfqpoint{0.858836in}{1.228494in}}{\pgfqpoint{0.862109in}{1.236394in}}{\pgfqpoint{0.862109in}{1.244631in}}%
\pgfpathcurveto{\pgfqpoint{0.862109in}{1.252867in}}{\pgfqpoint{0.858836in}{1.260767in}}{\pgfqpoint{0.853012in}{1.266591in}}%
\pgfpathcurveto{\pgfqpoint{0.847189in}{1.272415in}}{\pgfqpoint{0.839288in}{1.275687in}}{\pgfqpoint{0.831052in}{1.275687in}}%
\pgfpathcurveto{\pgfqpoint{0.822816in}{1.275687in}}{\pgfqpoint{0.814916in}{1.272415in}}{\pgfqpoint{0.809092in}{1.266591in}}%
\pgfpathcurveto{\pgfqpoint{0.803268in}{1.260767in}}{\pgfqpoint{0.799996in}{1.252867in}}{\pgfqpoint{0.799996in}{1.244631in}}%
\pgfpathcurveto{\pgfqpoint{0.799996in}{1.236394in}}{\pgfqpoint{0.803268in}{1.228494in}}{\pgfqpoint{0.809092in}{1.222670in}}%
\pgfpathcurveto{\pgfqpoint{0.814916in}{1.216846in}}{\pgfqpoint{0.822816in}{1.213574in}}{\pgfqpoint{0.831052in}{1.213574in}}%
\pgfpathclose%
\pgfusepath{stroke,fill}%
\end{pgfscope}%
\begin{pgfscope}%
\pgfpathrectangle{\pgfqpoint{0.100000in}{0.220728in}}{\pgfqpoint{3.696000in}{3.696000in}}%
\pgfusepath{clip}%
\pgfsetbuttcap%
\pgfsetroundjoin%
\definecolor{currentfill}{rgb}{0.121569,0.466667,0.705882}%
\pgfsetfillcolor{currentfill}%
\pgfsetfillopacity{0.616596}%
\pgfsetlinewidth{1.003750pt}%
\definecolor{currentstroke}{rgb}{0.121569,0.466667,0.705882}%
\pgfsetstrokecolor{currentstroke}%
\pgfsetstrokeopacity{0.616596}%
\pgfsetdash{}{0pt}%
\pgfpathmoveto{\pgfqpoint{0.831052in}{1.213574in}}%
\pgfpathcurveto{\pgfqpoint{0.839288in}{1.213574in}}{\pgfqpoint{0.847189in}{1.216846in}}{\pgfqpoint{0.853012in}{1.222670in}}%
\pgfpathcurveto{\pgfqpoint{0.858836in}{1.228494in}}{\pgfqpoint{0.862109in}{1.236394in}}{\pgfqpoint{0.862109in}{1.244631in}}%
\pgfpathcurveto{\pgfqpoint{0.862109in}{1.252867in}}{\pgfqpoint{0.858836in}{1.260767in}}{\pgfqpoint{0.853012in}{1.266591in}}%
\pgfpathcurveto{\pgfqpoint{0.847189in}{1.272415in}}{\pgfqpoint{0.839288in}{1.275687in}}{\pgfqpoint{0.831052in}{1.275687in}}%
\pgfpathcurveto{\pgfqpoint{0.822816in}{1.275687in}}{\pgfqpoint{0.814916in}{1.272415in}}{\pgfqpoint{0.809092in}{1.266591in}}%
\pgfpathcurveto{\pgfqpoint{0.803268in}{1.260767in}}{\pgfqpoint{0.799996in}{1.252867in}}{\pgfqpoint{0.799996in}{1.244631in}}%
\pgfpathcurveto{\pgfqpoint{0.799996in}{1.236394in}}{\pgfqpoint{0.803268in}{1.228494in}}{\pgfqpoint{0.809092in}{1.222670in}}%
\pgfpathcurveto{\pgfqpoint{0.814916in}{1.216846in}}{\pgfqpoint{0.822816in}{1.213574in}}{\pgfqpoint{0.831052in}{1.213574in}}%
\pgfpathclose%
\pgfusepath{stroke,fill}%
\end{pgfscope}%
\begin{pgfscope}%
\pgfpathrectangle{\pgfqpoint{0.100000in}{0.220728in}}{\pgfqpoint{3.696000in}{3.696000in}}%
\pgfusepath{clip}%
\pgfsetbuttcap%
\pgfsetroundjoin%
\definecolor{currentfill}{rgb}{0.121569,0.466667,0.705882}%
\pgfsetfillcolor{currentfill}%
\pgfsetfillopacity{0.616636}%
\pgfsetlinewidth{1.003750pt}%
\definecolor{currentstroke}{rgb}{0.121569,0.466667,0.705882}%
\pgfsetstrokecolor{currentstroke}%
\pgfsetstrokeopacity{0.616636}%
\pgfsetdash{}{0pt}%
\pgfpathmoveto{\pgfqpoint{0.828691in}{1.212634in}}%
\pgfpathcurveto{\pgfqpoint{0.836927in}{1.212634in}}{\pgfqpoint{0.844827in}{1.215907in}}{\pgfqpoint{0.850651in}{1.221731in}}%
\pgfpathcurveto{\pgfqpoint{0.856475in}{1.227554in}}{\pgfqpoint{0.859748in}{1.235454in}}{\pgfqpoint{0.859748in}{1.243691in}}%
\pgfpathcurveto{\pgfqpoint{0.859748in}{1.251927in}}{\pgfqpoint{0.856475in}{1.259827in}}{\pgfqpoint{0.850651in}{1.265651in}}%
\pgfpathcurveto{\pgfqpoint{0.844827in}{1.271475in}}{\pgfqpoint{0.836927in}{1.274747in}}{\pgfqpoint{0.828691in}{1.274747in}}%
\pgfpathcurveto{\pgfqpoint{0.820455in}{1.274747in}}{\pgfqpoint{0.812555in}{1.271475in}}{\pgfqpoint{0.806731in}{1.265651in}}%
\pgfpathcurveto{\pgfqpoint{0.800907in}{1.259827in}}{\pgfqpoint{0.797635in}{1.251927in}}{\pgfqpoint{0.797635in}{1.243691in}}%
\pgfpathcurveto{\pgfqpoint{0.797635in}{1.235454in}}{\pgfqpoint{0.800907in}{1.227554in}}{\pgfqpoint{0.806731in}{1.221731in}}%
\pgfpathcurveto{\pgfqpoint{0.812555in}{1.215907in}}{\pgfqpoint{0.820455in}{1.212634in}}{\pgfqpoint{0.828691in}{1.212634in}}%
\pgfpathclose%
\pgfusepath{stroke,fill}%
\end{pgfscope}%
\begin{pgfscope}%
\pgfpathrectangle{\pgfqpoint{0.100000in}{0.220728in}}{\pgfqpoint{3.696000in}{3.696000in}}%
\pgfusepath{clip}%
\pgfsetbuttcap%
\pgfsetroundjoin%
\definecolor{currentfill}{rgb}{0.121569,0.466667,0.705882}%
\pgfsetfillcolor{currentfill}%
\pgfsetfillopacity{0.618213}%
\pgfsetlinewidth{1.003750pt}%
\definecolor{currentstroke}{rgb}{0.121569,0.466667,0.705882}%
\pgfsetstrokecolor{currentstroke}%
\pgfsetstrokeopacity{0.618213}%
\pgfsetdash{}{0pt}%
\pgfpathmoveto{\pgfqpoint{3.135636in}{2.979479in}}%
\pgfpathcurveto{\pgfqpoint{3.143873in}{2.979479in}}{\pgfqpoint{3.151773in}{2.982751in}}{\pgfqpoint{3.157597in}{2.988575in}}%
\pgfpathcurveto{\pgfqpoint{3.163421in}{2.994399in}}{\pgfqpoint{3.166693in}{3.002299in}}{\pgfqpoint{3.166693in}{3.010536in}}%
\pgfpathcurveto{\pgfqpoint{3.166693in}{3.018772in}}{\pgfqpoint{3.163421in}{3.026672in}}{\pgfqpoint{3.157597in}{3.032496in}}%
\pgfpathcurveto{\pgfqpoint{3.151773in}{3.038320in}}{\pgfqpoint{3.143873in}{3.041592in}}{\pgfqpoint{3.135636in}{3.041592in}}%
\pgfpathcurveto{\pgfqpoint{3.127400in}{3.041592in}}{\pgfqpoint{3.119500in}{3.038320in}}{\pgfqpoint{3.113676in}{3.032496in}}%
\pgfpathcurveto{\pgfqpoint{3.107852in}{3.026672in}}{\pgfqpoint{3.104580in}{3.018772in}}{\pgfqpoint{3.104580in}{3.010536in}}%
\pgfpathcurveto{\pgfqpoint{3.104580in}{3.002299in}}{\pgfqpoint{3.107852in}{2.994399in}}{\pgfqpoint{3.113676in}{2.988575in}}%
\pgfpathcurveto{\pgfqpoint{3.119500in}{2.982751in}}{\pgfqpoint{3.127400in}{2.979479in}}{\pgfqpoint{3.135636in}{2.979479in}}%
\pgfpathclose%
\pgfusepath{stroke,fill}%
\end{pgfscope}%
\begin{pgfscope}%
\pgfpathrectangle{\pgfqpoint{0.100000in}{0.220728in}}{\pgfqpoint{3.696000in}{3.696000in}}%
\pgfusepath{clip}%
\pgfsetbuttcap%
\pgfsetroundjoin%
\definecolor{currentfill}{rgb}{0.121569,0.466667,0.705882}%
\pgfsetfillcolor{currentfill}%
\pgfsetfillopacity{0.622365}%
\pgfsetlinewidth{1.003750pt}%
\definecolor{currentstroke}{rgb}{0.121569,0.466667,0.705882}%
\pgfsetstrokecolor{currentstroke}%
\pgfsetstrokeopacity{0.622365}%
\pgfsetdash{}{0pt}%
\pgfpathmoveto{\pgfqpoint{3.152059in}{2.976570in}}%
\pgfpathcurveto{\pgfqpoint{3.160296in}{2.976570in}}{\pgfqpoint{3.168196in}{2.979843in}}{\pgfqpoint{3.174019in}{2.985667in}}%
\pgfpathcurveto{\pgfqpoint{3.179843in}{2.991491in}}{\pgfqpoint{3.183116in}{2.999391in}}{\pgfqpoint{3.183116in}{3.007627in}}%
\pgfpathcurveto{\pgfqpoint{3.183116in}{3.015863in}}{\pgfqpoint{3.179843in}{3.023763in}}{\pgfqpoint{3.174019in}{3.029587in}}%
\pgfpathcurveto{\pgfqpoint{3.168196in}{3.035411in}}{\pgfqpoint{3.160296in}{3.038683in}}{\pgfqpoint{3.152059in}{3.038683in}}%
\pgfpathcurveto{\pgfqpoint{3.143823in}{3.038683in}}{\pgfqpoint{3.135923in}{3.035411in}}{\pgfqpoint{3.130099in}{3.029587in}}%
\pgfpathcurveto{\pgfqpoint{3.124275in}{3.023763in}}{\pgfqpoint{3.121003in}{3.015863in}}{\pgfqpoint{3.121003in}{3.007627in}}%
\pgfpathcurveto{\pgfqpoint{3.121003in}{2.999391in}}{\pgfqpoint{3.124275in}{2.991491in}}{\pgfqpoint{3.130099in}{2.985667in}}%
\pgfpathcurveto{\pgfqpoint{3.135923in}{2.979843in}}{\pgfqpoint{3.143823in}{2.976570in}}{\pgfqpoint{3.152059in}{2.976570in}}%
\pgfpathclose%
\pgfusepath{stroke,fill}%
\end{pgfscope}%
\begin{pgfscope}%
\pgfpathrectangle{\pgfqpoint{0.100000in}{0.220728in}}{\pgfqpoint{3.696000in}{3.696000in}}%
\pgfusepath{clip}%
\pgfsetbuttcap%
\pgfsetroundjoin%
\definecolor{currentfill}{rgb}{0.121569,0.466667,0.705882}%
\pgfsetfillcolor{currentfill}%
\pgfsetfillopacity{0.622827}%
\pgfsetlinewidth{1.003750pt}%
\definecolor{currentstroke}{rgb}{0.121569,0.466667,0.705882}%
\pgfsetstrokecolor{currentstroke}%
\pgfsetstrokeopacity{0.622827}%
\pgfsetdash{}{0pt}%
\pgfpathmoveto{\pgfqpoint{3.172542in}{2.973651in}}%
\pgfpathcurveto{\pgfqpoint{3.180778in}{2.973651in}}{\pgfqpoint{3.188678in}{2.976924in}}{\pgfqpoint{3.194502in}{2.982748in}}%
\pgfpathcurveto{\pgfqpoint{3.200326in}{2.988572in}}{\pgfqpoint{3.203598in}{2.996472in}}{\pgfqpoint{3.203598in}{3.004708in}}%
\pgfpathcurveto{\pgfqpoint{3.203598in}{3.012944in}}{\pgfqpoint{3.200326in}{3.020844in}}{\pgfqpoint{3.194502in}{3.026668in}}%
\pgfpathcurveto{\pgfqpoint{3.188678in}{3.032492in}}{\pgfqpoint{3.180778in}{3.035764in}}{\pgfqpoint{3.172542in}{3.035764in}}%
\pgfpathcurveto{\pgfqpoint{3.164305in}{3.035764in}}{\pgfqpoint{3.156405in}{3.032492in}}{\pgfqpoint{3.150581in}{3.026668in}}%
\pgfpathcurveto{\pgfqpoint{3.144757in}{3.020844in}}{\pgfqpoint{3.141485in}{3.012944in}}{\pgfqpoint{3.141485in}{3.004708in}}%
\pgfpathcurveto{\pgfqpoint{3.141485in}{2.996472in}}{\pgfqpoint{3.144757in}{2.988572in}}{\pgfqpoint{3.150581in}{2.982748in}}%
\pgfpathcurveto{\pgfqpoint{3.156405in}{2.976924in}}{\pgfqpoint{3.164305in}{2.973651in}}{\pgfqpoint{3.172542in}{2.973651in}}%
\pgfpathclose%
\pgfusepath{stroke,fill}%
\end{pgfscope}%
\begin{pgfscope}%
\pgfpathrectangle{\pgfqpoint{0.100000in}{0.220728in}}{\pgfqpoint{3.696000in}{3.696000in}}%
\pgfusepath{clip}%
\pgfsetbuttcap%
\pgfsetroundjoin%
\definecolor{currentfill}{rgb}{0.121569,0.466667,0.705882}%
\pgfsetfillcolor{currentfill}%
\pgfsetfillopacity{0.627913}%
\pgfsetlinewidth{1.003750pt}%
\definecolor{currentstroke}{rgb}{0.121569,0.466667,0.705882}%
\pgfsetstrokecolor{currentstroke}%
\pgfsetstrokeopacity{0.627913}%
\pgfsetdash{}{0pt}%
\pgfpathmoveto{\pgfqpoint{3.193063in}{2.972624in}}%
\pgfpathcurveto{\pgfqpoint{3.201300in}{2.972624in}}{\pgfqpoint{3.209200in}{2.975897in}}{\pgfqpoint{3.215023in}{2.981721in}}%
\pgfpathcurveto{\pgfqpoint{3.220847in}{2.987545in}}{\pgfqpoint{3.224120in}{2.995445in}}{\pgfqpoint{3.224120in}{3.003681in}}%
\pgfpathcurveto{\pgfqpoint{3.224120in}{3.011917in}}{\pgfqpoint{3.220847in}{3.019817in}}{\pgfqpoint{3.215023in}{3.025641in}}%
\pgfpathcurveto{\pgfqpoint{3.209200in}{3.031465in}}{\pgfqpoint{3.201300in}{3.034737in}}{\pgfqpoint{3.193063in}{3.034737in}}%
\pgfpathcurveto{\pgfqpoint{3.184827in}{3.034737in}}{\pgfqpoint{3.176927in}{3.031465in}}{\pgfqpoint{3.171103in}{3.025641in}}%
\pgfpathcurveto{\pgfqpoint{3.165279in}{3.019817in}}{\pgfqpoint{3.162007in}{3.011917in}}{\pgfqpoint{3.162007in}{3.003681in}}%
\pgfpathcurveto{\pgfqpoint{3.162007in}{2.995445in}}{\pgfqpoint{3.165279in}{2.987545in}}{\pgfqpoint{3.171103in}{2.981721in}}%
\pgfpathcurveto{\pgfqpoint{3.176927in}{2.975897in}}{\pgfqpoint{3.184827in}{2.972624in}}{\pgfqpoint{3.193063in}{2.972624in}}%
\pgfpathclose%
\pgfusepath{stroke,fill}%
\end{pgfscope}%
\begin{pgfscope}%
\pgfpathrectangle{\pgfqpoint{0.100000in}{0.220728in}}{\pgfqpoint{3.696000in}{3.696000in}}%
\pgfusepath{clip}%
\pgfsetbuttcap%
\pgfsetroundjoin%
\definecolor{currentfill}{rgb}{0.121569,0.466667,0.705882}%
\pgfsetfillcolor{currentfill}%
\pgfsetfillopacity{0.630219}%
\pgfsetlinewidth{1.003750pt}%
\definecolor{currentstroke}{rgb}{0.121569,0.466667,0.705882}%
\pgfsetstrokecolor{currentstroke}%
\pgfsetstrokeopacity{0.630219}%
\pgfsetdash{}{0pt}%
\pgfpathmoveto{\pgfqpoint{3.204397in}{2.970358in}}%
\pgfpathcurveto{\pgfqpoint{3.212633in}{2.970358in}}{\pgfqpoint{3.220533in}{2.973631in}}{\pgfqpoint{3.226357in}{2.979455in}}%
\pgfpathcurveto{\pgfqpoint{3.232181in}{2.985279in}}{\pgfqpoint{3.235453in}{2.993179in}}{\pgfqpoint{3.235453in}{3.001415in}}%
\pgfpathcurveto{\pgfqpoint{3.235453in}{3.009651in}}{\pgfqpoint{3.232181in}{3.017551in}}{\pgfqpoint{3.226357in}{3.023375in}}%
\pgfpathcurveto{\pgfqpoint{3.220533in}{3.029199in}}{\pgfqpoint{3.212633in}{3.032471in}}{\pgfqpoint{3.204397in}{3.032471in}}%
\pgfpathcurveto{\pgfqpoint{3.196161in}{3.032471in}}{\pgfqpoint{3.188261in}{3.029199in}}{\pgfqpoint{3.182437in}{3.023375in}}%
\pgfpathcurveto{\pgfqpoint{3.176613in}{3.017551in}}{\pgfqpoint{3.173340in}{3.009651in}}{\pgfqpoint{3.173340in}{3.001415in}}%
\pgfpathcurveto{\pgfqpoint{3.173340in}{2.993179in}}{\pgfqpoint{3.176613in}{2.985279in}}{\pgfqpoint{3.182437in}{2.979455in}}%
\pgfpathcurveto{\pgfqpoint{3.188261in}{2.973631in}}{\pgfqpoint{3.196161in}{2.970358in}}{\pgfqpoint{3.204397in}{2.970358in}}%
\pgfpathclose%
\pgfusepath{stroke,fill}%
\end{pgfscope}%
\begin{pgfscope}%
\pgfpathrectangle{\pgfqpoint{0.100000in}{0.220728in}}{\pgfqpoint{3.696000in}{3.696000in}}%
\pgfusepath{clip}%
\pgfsetbuttcap%
\pgfsetroundjoin%
\definecolor{currentfill}{rgb}{0.121569,0.466667,0.705882}%
\pgfsetfillcolor{currentfill}%
\pgfsetfillopacity{0.630608}%
\pgfsetlinewidth{1.003750pt}%
\definecolor{currentstroke}{rgb}{0.121569,0.466667,0.705882}%
\pgfsetstrokecolor{currentstroke}%
\pgfsetstrokeopacity{0.630608}%
\pgfsetdash{}{0pt}%
\pgfpathmoveto{\pgfqpoint{3.211279in}{2.969322in}}%
\pgfpathcurveto{\pgfqpoint{3.219516in}{2.969322in}}{\pgfqpoint{3.227416in}{2.972595in}}{\pgfqpoint{3.233240in}{2.978418in}}%
\pgfpathcurveto{\pgfqpoint{3.239064in}{2.984242in}}{\pgfqpoint{3.242336in}{2.992142in}}{\pgfqpoint{3.242336in}{3.000379in}}%
\pgfpathcurveto{\pgfqpoint{3.242336in}{3.008615in}}{\pgfqpoint{3.239064in}{3.016515in}}{\pgfqpoint{3.233240in}{3.022339in}}%
\pgfpathcurveto{\pgfqpoint{3.227416in}{3.028163in}}{\pgfqpoint{3.219516in}{3.031435in}}{\pgfqpoint{3.211279in}{3.031435in}}%
\pgfpathcurveto{\pgfqpoint{3.203043in}{3.031435in}}{\pgfqpoint{3.195143in}{3.028163in}}{\pgfqpoint{3.189319in}{3.022339in}}%
\pgfpathcurveto{\pgfqpoint{3.183495in}{3.016515in}}{\pgfqpoint{3.180223in}{3.008615in}}{\pgfqpoint{3.180223in}{3.000379in}}%
\pgfpathcurveto{\pgfqpoint{3.180223in}{2.992142in}}{\pgfqpoint{3.183495in}{2.984242in}}{\pgfqpoint{3.189319in}{2.978418in}}%
\pgfpathcurveto{\pgfqpoint{3.195143in}{2.972595in}}{\pgfqpoint{3.203043in}{2.969322in}}{\pgfqpoint{3.211279in}{2.969322in}}%
\pgfpathclose%
\pgfusepath{stroke,fill}%
\end{pgfscope}%
\begin{pgfscope}%
\pgfpathrectangle{\pgfqpoint{0.100000in}{0.220728in}}{\pgfqpoint{3.696000in}{3.696000in}}%
\pgfusepath{clip}%
\pgfsetbuttcap%
\pgfsetroundjoin%
\definecolor{currentfill}{rgb}{0.121569,0.466667,0.705882}%
\pgfsetfillcolor{currentfill}%
\pgfsetfillopacity{0.631466}%
\pgfsetlinewidth{1.003750pt}%
\definecolor{currentstroke}{rgb}{0.121569,0.466667,0.705882}%
\pgfsetstrokecolor{currentstroke}%
\pgfsetstrokeopacity{0.631466}%
\pgfsetdash{}{0pt}%
\pgfpathmoveto{\pgfqpoint{3.219945in}{2.967290in}}%
\pgfpathcurveto{\pgfqpoint{3.228181in}{2.967290in}}{\pgfqpoint{3.236081in}{2.970562in}}{\pgfqpoint{3.241905in}{2.976386in}}%
\pgfpathcurveto{\pgfqpoint{3.247729in}{2.982210in}}{\pgfqpoint{3.251001in}{2.990110in}}{\pgfqpoint{3.251001in}{2.998346in}}%
\pgfpathcurveto{\pgfqpoint{3.251001in}{3.006582in}}{\pgfqpoint{3.247729in}{3.014482in}}{\pgfqpoint{3.241905in}{3.020306in}}%
\pgfpathcurveto{\pgfqpoint{3.236081in}{3.026130in}}{\pgfqpoint{3.228181in}{3.029403in}}{\pgfqpoint{3.219945in}{3.029403in}}%
\pgfpathcurveto{\pgfqpoint{3.211708in}{3.029403in}}{\pgfqpoint{3.203808in}{3.026130in}}{\pgfqpoint{3.197984in}{3.020306in}}%
\pgfpathcurveto{\pgfqpoint{3.192161in}{3.014482in}}{\pgfqpoint{3.188888in}{3.006582in}}{\pgfqpoint{3.188888in}{2.998346in}}%
\pgfpathcurveto{\pgfqpoint{3.188888in}{2.990110in}}{\pgfqpoint{3.192161in}{2.982210in}}{\pgfqpoint{3.197984in}{2.976386in}}%
\pgfpathcurveto{\pgfqpoint{3.203808in}{2.970562in}}{\pgfqpoint{3.211708in}{2.967290in}}{\pgfqpoint{3.219945in}{2.967290in}}%
\pgfpathclose%
\pgfusepath{stroke,fill}%
\end{pgfscope}%
\begin{pgfscope}%
\pgfpathrectangle{\pgfqpoint{0.100000in}{0.220728in}}{\pgfqpoint{3.696000in}{3.696000in}}%
\pgfusepath{clip}%
\pgfsetbuttcap%
\pgfsetroundjoin%
\definecolor{currentfill}{rgb}{0.121569,0.466667,0.705882}%
\pgfsetfillcolor{currentfill}%
\pgfsetfillopacity{0.634207}%
\pgfsetlinewidth{1.003750pt}%
\definecolor{currentstroke}{rgb}{0.121569,0.466667,0.705882}%
\pgfsetstrokecolor{currentstroke}%
\pgfsetstrokeopacity{0.634207}%
\pgfsetdash{}{0pt}%
\pgfpathmoveto{\pgfqpoint{3.231114in}{2.965999in}}%
\pgfpathcurveto{\pgfqpoint{3.239350in}{2.965999in}}{\pgfqpoint{3.247250in}{2.969271in}}{\pgfqpoint{3.253074in}{2.975095in}}%
\pgfpathcurveto{\pgfqpoint{3.258898in}{2.980919in}}{\pgfqpoint{3.262170in}{2.988819in}}{\pgfqpoint{3.262170in}{2.997055in}}%
\pgfpathcurveto{\pgfqpoint{3.262170in}{3.005292in}}{\pgfqpoint{3.258898in}{3.013192in}}{\pgfqpoint{3.253074in}{3.019016in}}%
\pgfpathcurveto{\pgfqpoint{3.247250in}{3.024840in}}{\pgfqpoint{3.239350in}{3.028112in}}{\pgfqpoint{3.231114in}{3.028112in}}%
\pgfpathcurveto{\pgfqpoint{3.222878in}{3.028112in}}{\pgfqpoint{3.214977in}{3.024840in}}{\pgfqpoint{3.209154in}{3.019016in}}%
\pgfpathcurveto{\pgfqpoint{3.203330in}{3.013192in}}{\pgfqpoint{3.200057in}{3.005292in}}{\pgfqpoint{3.200057in}{2.997055in}}%
\pgfpathcurveto{\pgfqpoint{3.200057in}{2.988819in}}{\pgfqpoint{3.203330in}{2.980919in}}{\pgfqpoint{3.209154in}{2.975095in}}%
\pgfpathcurveto{\pgfqpoint{3.214977in}{2.969271in}}{\pgfqpoint{3.222878in}{2.965999in}}{\pgfqpoint{3.231114in}{2.965999in}}%
\pgfpathclose%
\pgfusepath{stroke,fill}%
\end{pgfscope}%
\begin{pgfscope}%
\pgfpathrectangle{\pgfqpoint{0.100000in}{0.220728in}}{\pgfqpoint{3.696000in}{3.696000in}}%
\pgfusepath{clip}%
\pgfsetbuttcap%
\pgfsetroundjoin%
\definecolor{currentfill}{rgb}{0.121569,0.466667,0.705882}%
\pgfsetfillcolor{currentfill}%
\pgfsetfillopacity{0.635547}%
\pgfsetlinewidth{1.003750pt}%
\definecolor{currentstroke}{rgb}{0.121569,0.466667,0.705882}%
\pgfsetstrokecolor{currentstroke}%
\pgfsetstrokeopacity{0.635547}%
\pgfsetdash{}{0pt}%
\pgfpathmoveto{\pgfqpoint{3.243553in}{2.962014in}}%
\pgfpathcurveto{\pgfqpoint{3.251789in}{2.962014in}}{\pgfqpoint{3.259689in}{2.965287in}}{\pgfqpoint{3.265513in}{2.971111in}}%
\pgfpathcurveto{\pgfqpoint{3.271337in}{2.976934in}}{\pgfqpoint{3.274609in}{2.984835in}}{\pgfqpoint{3.274609in}{2.993071in}}%
\pgfpathcurveto{\pgfqpoint{3.274609in}{3.001307in}}{\pgfqpoint{3.271337in}{3.009207in}}{\pgfqpoint{3.265513in}{3.015031in}}%
\pgfpathcurveto{\pgfqpoint{3.259689in}{3.020855in}}{\pgfqpoint{3.251789in}{3.024127in}}{\pgfqpoint{3.243553in}{3.024127in}}%
\pgfpathcurveto{\pgfqpoint{3.235316in}{3.024127in}}{\pgfqpoint{3.227416in}{3.020855in}}{\pgfqpoint{3.221593in}{3.015031in}}%
\pgfpathcurveto{\pgfqpoint{3.215769in}{3.009207in}}{\pgfqpoint{3.212496in}{3.001307in}}{\pgfqpoint{3.212496in}{2.993071in}}%
\pgfpathcurveto{\pgfqpoint{3.212496in}{2.984835in}}{\pgfqpoint{3.215769in}{2.976934in}}{\pgfqpoint{3.221593in}{2.971111in}}%
\pgfpathcurveto{\pgfqpoint{3.227416in}{2.965287in}}{\pgfqpoint{3.235316in}{2.962014in}}{\pgfqpoint{3.243553in}{2.962014in}}%
\pgfpathclose%
\pgfusepath{stroke,fill}%
\end{pgfscope}%
\begin{pgfscope}%
\pgfpathrectangle{\pgfqpoint{0.100000in}{0.220728in}}{\pgfqpoint{3.696000in}{3.696000in}}%
\pgfusepath{clip}%
\pgfsetbuttcap%
\pgfsetroundjoin%
\definecolor{currentfill}{rgb}{0.121569,0.466667,0.705882}%
\pgfsetfillcolor{currentfill}%
\pgfsetfillopacity{0.637182}%
\pgfsetlinewidth{1.003750pt}%
\definecolor{currentstroke}{rgb}{0.121569,0.466667,0.705882}%
\pgfsetstrokecolor{currentstroke}%
\pgfsetstrokeopacity{0.637182}%
\pgfsetdash{}{0pt}%
\pgfpathmoveto{\pgfqpoint{3.249056in}{2.958910in}}%
\pgfpathcurveto{\pgfqpoint{3.257292in}{2.958910in}}{\pgfqpoint{3.265192in}{2.962182in}}{\pgfqpoint{3.271016in}{2.968006in}}%
\pgfpathcurveto{\pgfqpoint{3.276840in}{2.973830in}}{\pgfqpoint{3.280112in}{2.981730in}}{\pgfqpoint{3.280112in}{2.989967in}}%
\pgfpathcurveto{\pgfqpoint{3.280112in}{2.998203in}}{\pgfqpoint{3.276840in}{3.006103in}}{\pgfqpoint{3.271016in}{3.011927in}}%
\pgfpathcurveto{\pgfqpoint{3.265192in}{3.017751in}}{\pgfqpoint{3.257292in}{3.021023in}}{\pgfqpoint{3.249056in}{3.021023in}}%
\pgfpathcurveto{\pgfqpoint{3.240819in}{3.021023in}}{\pgfqpoint{3.232919in}{3.017751in}}{\pgfqpoint{3.227095in}{3.011927in}}%
\pgfpathcurveto{\pgfqpoint{3.221271in}{3.006103in}}{\pgfqpoint{3.217999in}{2.998203in}}{\pgfqpoint{3.217999in}{2.989967in}}%
\pgfpathcurveto{\pgfqpoint{3.217999in}{2.981730in}}{\pgfqpoint{3.221271in}{2.973830in}}{\pgfqpoint{3.227095in}{2.968006in}}%
\pgfpathcurveto{\pgfqpoint{3.232919in}{2.962182in}}{\pgfqpoint{3.240819in}{2.958910in}}{\pgfqpoint{3.249056in}{2.958910in}}%
\pgfpathclose%
\pgfusepath{stroke,fill}%
\end{pgfscope}%
\begin{pgfscope}%
\pgfpathrectangle{\pgfqpoint{0.100000in}{0.220728in}}{\pgfqpoint{3.696000in}{3.696000in}}%
\pgfusepath{clip}%
\pgfsetbuttcap%
\pgfsetroundjoin%
\definecolor{currentfill}{rgb}{0.121569,0.466667,0.705882}%
\pgfsetfillcolor{currentfill}%
\pgfsetfillopacity{0.638421}%
\pgfsetlinewidth{1.003750pt}%
\definecolor{currentstroke}{rgb}{0.121569,0.466667,0.705882}%
\pgfsetstrokecolor{currentstroke}%
\pgfsetstrokeopacity{0.638421}%
\pgfsetdash{}{0pt}%
\pgfpathmoveto{\pgfqpoint{3.257237in}{2.958322in}}%
\pgfpathcurveto{\pgfqpoint{3.265473in}{2.958322in}}{\pgfqpoint{3.273373in}{2.961595in}}{\pgfqpoint{3.279197in}{2.967419in}}%
\pgfpathcurveto{\pgfqpoint{3.285021in}{2.973242in}}{\pgfqpoint{3.288294in}{2.981143in}}{\pgfqpoint{3.288294in}{2.989379in}}%
\pgfpathcurveto{\pgfqpoint{3.288294in}{2.997615in}}{\pgfqpoint{3.285021in}{3.005515in}}{\pgfqpoint{3.279197in}{3.011339in}}%
\pgfpathcurveto{\pgfqpoint{3.273373in}{3.017163in}}{\pgfqpoint{3.265473in}{3.020435in}}{\pgfqpoint{3.257237in}{3.020435in}}%
\pgfpathcurveto{\pgfqpoint{3.249001in}{3.020435in}}{\pgfqpoint{3.241101in}{3.017163in}}{\pgfqpoint{3.235277in}{3.011339in}}%
\pgfpathcurveto{\pgfqpoint{3.229453in}{3.005515in}}{\pgfqpoint{3.226181in}{2.997615in}}{\pgfqpoint{3.226181in}{2.989379in}}%
\pgfpathcurveto{\pgfqpoint{3.226181in}{2.981143in}}{\pgfqpoint{3.229453in}{2.973242in}}{\pgfqpoint{3.235277in}{2.967419in}}%
\pgfpathcurveto{\pgfqpoint{3.241101in}{2.961595in}}{\pgfqpoint{3.249001in}{2.958322in}}{\pgfqpoint{3.257237in}{2.958322in}}%
\pgfpathclose%
\pgfusepath{stroke,fill}%
\end{pgfscope}%
\begin{pgfscope}%
\pgfpathrectangle{\pgfqpoint{0.100000in}{0.220728in}}{\pgfqpoint{3.696000in}{3.696000in}}%
\pgfusepath{clip}%
\pgfsetbuttcap%
\pgfsetroundjoin%
\definecolor{currentfill}{rgb}{0.121569,0.466667,0.705882}%
\pgfsetfillcolor{currentfill}%
\pgfsetfillopacity{0.639143}%
\pgfsetlinewidth{1.003750pt}%
\definecolor{currentstroke}{rgb}{0.121569,0.466667,0.705882}%
\pgfsetstrokecolor{currentstroke}%
\pgfsetstrokeopacity{0.639143}%
\pgfsetdash{}{0pt}%
\pgfpathmoveto{\pgfqpoint{3.267955in}{2.958313in}}%
\pgfpathcurveto{\pgfqpoint{3.276191in}{2.958313in}}{\pgfqpoint{3.284091in}{2.961585in}}{\pgfqpoint{3.289915in}{2.967409in}}%
\pgfpathcurveto{\pgfqpoint{3.295739in}{2.973233in}}{\pgfqpoint{3.299011in}{2.981133in}}{\pgfqpoint{3.299011in}{2.989370in}}%
\pgfpathcurveto{\pgfqpoint{3.299011in}{2.997606in}}{\pgfqpoint{3.295739in}{3.005506in}}{\pgfqpoint{3.289915in}{3.011330in}}%
\pgfpathcurveto{\pgfqpoint{3.284091in}{3.017154in}}{\pgfqpoint{3.276191in}{3.020426in}}{\pgfqpoint{3.267955in}{3.020426in}}%
\pgfpathcurveto{\pgfqpoint{3.259718in}{3.020426in}}{\pgfqpoint{3.251818in}{3.017154in}}{\pgfqpoint{3.245994in}{3.011330in}}%
\pgfpathcurveto{\pgfqpoint{3.240170in}{3.005506in}}{\pgfqpoint{3.236898in}{2.997606in}}{\pgfqpoint{3.236898in}{2.989370in}}%
\pgfpathcurveto{\pgfqpoint{3.236898in}{2.981133in}}{\pgfqpoint{3.240170in}{2.973233in}}{\pgfqpoint{3.245994in}{2.967409in}}%
\pgfpathcurveto{\pgfqpoint{3.251818in}{2.961585in}}{\pgfqpoint{3.259718in}{2.958313in}}{\pgfqpoint{3.267955in}{2.958313in}}%
\pgfpathclose%
\pgfusepath{stroke,fill}%
\end{pgfscope}%
\begin{pgfscope}%
\pgfpathrectangle{\pgfqpoint{0.100000in}{0.220728in}}{\pgfqpoint{3.696000in}{3.696000in}}%
\pgfusepath{clip}%
\pgfsetbuttcap%
\pgfsetroundjoin%
\definecolor{currentfill}{rgb}{0.121569,0.466667,0.705882}%
\pgfsetfillcolor{currentfill}%
\pgfsetfillopacity{0.641564}%
\pgfsetlinewidth{1.003750pt}%
\definecolor{currentstroke}{rgb}{0.121569,0.466667,0.705882}%
\pgfsetstrokecolor{currentstroke}%
\pgfsetstrokeopacity{0.641564}%
\pgfsetdash{}{0pt}%
\pgfpathmoveto{\pgfqpoint{3.278795in}{2.957575in}}%
\pgfpathcurveto{\pgfqpoint{3.287031in}{2.957575in}}{\pgfqpoint{3.294931in}{2.960848in}}{\pgfqpoint{3.300755in}{2.966671in}}%
\pgfpathcurveto{\pgfqpoint{3.306579in}{2.972495in}}{\pgfqpoint{3.309851in}{2.980395in}}{\pgfqpoint{3.309851in}{2.988632in}}%
\pgfpathcurveto{\pgfqpoint{3.309851in}{2.996868in}}{\pgfqpoint{3.306579in}{3.004768in}}{\pgfqpoint{3.300755in}{3.010592in}}%
\pgfpathcurveto{\pgfqpoint{3.294931in}{3.016416in}}{\pgfqpoint{3.287031in}{3.019688in}}{\pgfqpoint{3.278795in}{3.019688in}}%
\pgfpathcurveto{\pgfqpoint{3.270558in}{3.019688in}}{\pgfqpoint{3.262658in}{3.016416in}}{\pgfqpoint{3.256834in}{3.010592in}}%
\pgfpathcurveto{\pgfqpoint{3.251010in}{3.004768in}}{\pgfqpoint{3.247738in}{2.996868in}}{\pgfqpoint{3.247738in}{2.988632in}}%
\pgfpathcurveto{\pgfqpoint{3.247738in}{2.980395in}}{\pgfqpoint{3.251010in}{2.972495in}}{\pgfqpoint{3.256834in}{2.966671in}}%
\pgfpathcurveto{\pgfqpoint{3.262658in}{2.960848in}}{\pgfqpoint{3.270558in}{2.957575in}}{\pgfqpoint{3.278795in}{2.957575in}}%
\pgfpathclose%
\pgfusepath{stroke,fill}%
\end{pgfscope}%
\begin{pgfscope}%
\pgfpathrectangle{\pgfqpoint{0.100000in}{0.220728in}}{\pgfqpoint{3.696000in}{3.696000in}}%
\pgfusepath{clip}%
\pgfsetbuttcap%
\pgfsetroundjoin%
\definecolor{currentfill}{rgb}{0.121569,0.466667,0.705882}%
\pgfsetfillcolor{currentfill}%
\pgfsetfillopacity{0.644500}%
\pgfsetlinewidth{1.003750pt}%
\definecolor{currentstroke}{rgb}{0.121569,0.466667,0.705882}%
\pgfsetstrokecolor{currentstroke}%
\pgfsetstrokeopacity{0.644500}%
\pgfsetdash{}{0pt}%
\pgfpathmoveto{\pgfqpoint{3.290112in}{2.956452in}}%
\pgfpathcurveto{\pgfqpoint{3.298348in}{2.956452in}}{\pgfqpoint{3.306248in}{2.959725in}}{\pgfqpoint{3.312072in}{2.965548in}}%
\pgfpathcurveto{\pgfqpoint{3.317896in}{2.971372in}}{\pgfqpoint{3.321168in}{2.979272in}}{\pgfqpoint{3.321168in}{2.987509in}}%
\pgfpathcurveto{\pgfqpoint{3.321168in}{2.995745in}}{\pgfqpoint{3.317896in}{3.003645in}}{\pgfqpoint{3.312072in}{3.009469in}}%
\pgfpathcurveto{\pgfqpoint{3.306248in}{3.015293in}}{\pgfqpoint{3.298348in}{3.018565in}}{\pgfqpoint{3.290112in}{3.018565in}}%
\pgfpathcurveto{\pgfqpoint{3.281875in}{3.018565in}}{\pgfqpoint{3.273975in}{3.015293in}}{\pgfqpoint{3.268151in}{3.009469in}}%
\pgfpathcurveto{\pgfqpoint{3.262327in}{3.003645in}}{\pgfqpoint{3.259055in}{2.995745in}}{\pgfqpoint{3.259055in}{2.987509in}}%
\pgfpathcurveto{\pgfqpoint{3.259055in}{2.979272in}}{\pgfqpoint{3.262327in}{2.971372in}}{\pgfqpoint{3.268151in}{2.965548in}}%
\pgfpathcurveto{\pgfqpoint{3.273975in}{2.959725in}}{\pgfqpoint{3.281875in}{2.956452in}}{\pgfqpoint{3.290112in}{2.956452in}}%
\pgfpathclose%
\pgfusepath{stroke,fill}%
\end{pgfscope}%
\begin{pgfscope}%
\pgfpathrectangle{\pgfqpoint{0.100000in}{0.220728in}}{\pgfqpoint{3.696000in}{3.696000in}}%
\pgfusepath{clip}%
\pgfsetbuttcap%
\pgfsetroundjoin%
\definecolor{currentfill}{rgb}{0.121569,0.466667,0.705882}%
\pgfsetfillcolor{currentfill}%
\pgfsetfillopacity{0.650908}%
\pgfsetlinewidth{1.003750pt}%
\definecolor{currentstroke}{rgb}{0.121569,0.466667,0.705882}%
\pgfsetstrokecolor{currentstroke}%
\pgfsetstrokeopacity{0.650908}%
\pgfsetdash{}{0pt}%
\pgfpathmoveto{\pgfqpoint{3.286572in}{2.955810in}}%
\pgfpathcurveto{\pgfqpoint{3.294808in}{2.955810in}}{\pgfqpoint{3.302708in}{2.959083in}}{\pgfqpoint{3.308532in}{2.964907in}}%
\pgfpathcurveto{\pgfqpoint{3.314356in}{2.970731in}}{\pgfqpoint{3.317629in}{2.978631in}}{\pgfqpoint{3.317629in}{2.986867in}}%
\pgfpathcurveto{\pgfqpoint{3.317629in}{2.995103in}}{\pgfqpoint{3.314356in}{3.003003in}}{\pgfqpoint{3.308532in}{3.008827in}}%
\pgfpathcurveto{\pgfqpoint{3.302708in}{3.014651in}}{\pgfqpoint{3.294808in}{3.017923in}}{\pgfqpoint{3.286572in}{3.017923in}}%
\pgfpathcurveto{\pgfqpoint{3.278336in}{3.017923in}}{\pgfqpoint{3.270436in}{3.014651in}}{\pgfqpoint{3.264612in}{3.008827in}}%
\pgfpathcurveto{\pgfqpoint{3.258788in}{3.003003in}}{\pgfqpoint{3.255516in}{2.995103in}}{\pgfqpoint{3.255516in}{2.986867in}}%
\pgfpathcurveto{\pgfqpoint{3.255516in}{2.978631in}}{\pgfqpoint{3.258788in}{2.970731in}}{\pgfqpoint{3.264612in}{2.964907in}}%
\pgfpathcurveto{\pgfqpoint{3.270436in}{2.959083in}}{\pgfqpoint{3.278336in}{2.955810in}}{\pgfqpoint{3.286572in}{2.955810in}}%
\pgfpathclose%
\pgfusepath{stroke,fill}%
\end{pgfscope}%
\begin{pgfscope}%
\pgfpathrectangle{\pgfqpoint{0.100000in}{0.220728in}}{\pgfqpoint{3.696000in}{3.696000in}}%
\pgfusepath{clip}%
\pgfsetbuttcap%
\pgfsetroundjoin%
\definecolor{currentfill}{rgb}{0.121569,0.466667,0.705882}%
\pgfsetfillcolor{currentfill}%
\pgfsetfillopacity{0.651477}%
\pgfsetlinewidth{1.003750pt}%
\definecolor{currentstroke}{rgb}{0.121569,0.466667,0.705882}%
\pgfsetstrokecolor{currentstroke}%
\pgfsetstrokeopacity{0.651477}%
\pgfsetdash{}{0pt}%
\pgfpathmoveto{\pgfqpoint{0.592749in}{1.458234in}}%
\pgfpathcurveto{\pgfqpoint{0.600985in}{1.458234in}}{\pgfqpoint{0.608886in}{1.461506in}}{\pgfqpoint{0.614709in}{1.467330in}}%
\pgfpathcurveto{\pgfqpoint{0.620533in}{1.473154in}}{\pgfqpoint{0.623806in}{1.481054in}}{\pgfqpoint{0.623806in}{1.489290in}}%
\pgfpathcurveto{\pgfqpoint{0.623806in}{1.497526in}}{\pgfqpoint{0.620533in}{1.505426in}}{\pgfqpoint{0.614709in}{1.511250in}}%
\pgfpathcurveto{\pgfqpoint{0.608886in}{1.517074in}}{\pgfqpoint{0.600985in}{1.520347in}}{\pgfqpoint{0.592749in}{1.520347in}}%
\pgfpathcurveto{\pgfqpoint{0.584513in}{1.520347in}}{\pgfqpoint{0.576613in}{1.517074in}}{\pgfqpoint{0.570789in}{1.511250in}}%
\pgfpathcurveto{\pgfqpoint{0.564965in}{1.505426in}}{\pgfqpoint{0.561693in}{1.497526in}}{\pgfqpoint{0.561693in}{1.489290in}}%
\pgfpathcurveto{\pgfqpoint{0.561693in}{1.481054in}}{\pgfqpoint{0.564965in}{1.473154in}}{\pgfqpoint{0.570789in}{1.467330in}}%
\pgfpathcurveto{\pgfqpoint{0.576613in}{1.461506in}}{\pgfqpoint{0.584513in}{1.458234in}}{\pgfqpoint{0.592749in}{1.458234in}}%
\pgfpathclose%
\pgfusepath{stroke,fill}%
\end{pgfscope}%
\begin{pgfscope}%
\pgfpathrectangle{\pgfqpoint{0.100000in}{0.220728in}}{\pgfqpoint{3.696000in}{3.696000in}}%
\pgfusepath{clip}%
\pgfsetbuttcap%
\pgfsetroundjoin%
\definecolor{currentfill}{rgb}{0.121569,0.466667,0.705882}%
\pgfsetfillcolor{currentfill}%
\pgfsetfillopacity{0.652157}%
\pgfsetlinewidth{1.003750pt}%
\definecolor{currentstroke}{rgb}{0.121569,0.466667,0.705882}%
\pgfsetstrokecolor{currentstroke}%
\pgfsetstrokeopacity{0.652157}%
\pgfsetdash{}{0pt}%
\pgfpathmoveto{\pgfqpoint{0.596003in}{1.459896in}}%
\pgfpathcurveto{\pgfqpoint{0.604240in}{1.459896in}}{\pgfqpoint{0.612140in}{1.463169in}}{\pgfqpoint{0.617964in}{1.468993in}}%
\pgfpathcurveto{\pgfqpoint{0.623787in}{1.474817in}}{\pgfqpoint{0.627060in}{1.482717in}}{\pgfqpoint{0.627060in}{1.490953in}}%
\pgfpathcurveto{\pgfqpoint{0.627060in}{1.499189in}}{\pgfqpoint{0.623787in}{1.507089in}}{\pgfqpoint{0.617964in}{1.512913in}}%
\pgfpathcurveto{\pgfqpoint{0.612140in}{1.518737in}}{\pgfqpoint{0.604240in}{1.522009in}}{\pgfqpoint{0.596003in}{1.522009in}}%
\pgfpathcurveto{\pgfqpoint{0.587767in}{1.522009in}}{\pgfqpoint{0.579867in}{1.518737in}}{\pgfqpoint{0.574043in}{1.512913in}}%
\pgfpathcurveto{\pgfqpoint{0.568219in}{1.507089in}}{\pgfqpoint{0.564947in}{1.499189in}}{\pgfqpoint{0.564947in}{1.490953in}}%
\pgfpathcurveto{\pgfqpoint{0.564947in}{1.482717in}}{\pgfqpoint{0.568219in}{1.474817in}}{\pgfqpoint{0.574043in}{1.468993in}}%
\pgfpathcurveto{\pgfqpoint{0.579867in}{1.463169in}}{\pgfqpoint{0.587767in}{1.459896in}}{\pgfqpoint{0.596003in}{1.459896in}}%
\pgfpathclose%
\pgfusepath{stroke,fill}%
\end{pgfscope}%
\begin{pgfscope}%
\pgfpathrectangle{\pgfqpoint{0.100000in}{0.220728in}}{\pgfqpoint{3.696000in}{3.696000in}}%
\pgfusepath{clip}%
\pgfsetbuttcap%
\pgfsetroundjoin%
\definecolor{currentfill}{rgb}{0.121569,0.466667,0.705882}%
\pgfsetfillcolor{currentfill}%
\pgfsetfillopacity{0.652814}%
\pgfsetlinewidth{1.003750pt}%
\definecolor{currentstroke}{rgb}{0.121569,0.466667,0.705882}%
\pgfsetstrokecolor{currentstroke}%
\pgfsetstrokeopacity{0.652814}%
\pgfsetdash{}{0pt}%
\pgfpathmoveto{\pgfqpoint{0.598648in}{1.461341in}}%
\pgfpathcurveto{\pgfqpoint{0.606885in}{1.461341in}}{\pgfqpoint{0.614785in}{1.464614in}}{\pgfqpoint{0.620609in}{1.470438in}}%
\pgfpathcurveto{\pgfqpoint{0.626432in}{1.476261in}}{\pgfqpoint{0.629705in}{1.484162in}}{\pgfqpoint{0.629705in}{1.492398in}}%
\pgfpathcurveto{\pgfqpoint{0.629705in}{1.500634in}}{\pgfqpoint{0.626432in}{1.508534in}}{\pgfqpoint{0.620609in}{1.514358in}}%
\pgfpathcurveto{\pgfqpoint{0.614785in}{1.520182in}}{\pgfqpoint{0.606885in}{1.523454in}}{\pgfqpoint{0.598648in}{1.523454in}}%
\pgfpathcurveto{\pgfqpoint{0.590412in}{1.523454in}}{\pgfqpoint{0.582512in}{1.520182in}}{\pgfqpoint{0.576688in}{1.514358in}}%
\pgfpathcurveto{\pgfqpoint{0.570864in}{1.508534in}}{\pgfqpoint{0.567592in}{1.500634in}}{\pgfqpoint{0.567592in}{1.492398in}}%
\pgfpathcurveto{\pgfqpoint{0.567592in}{1.484162in}}{\pgfqpoint{0.570864in}{1.476261in}}{\pgfqpoint{0.576688in}{1.470438in}}%
\pgfpathcurveto{\pgfqpoint{0.582512in}{1.464614in}}{\pgfqpoint{0.590412in}{1.461341in}}{\pgfqpoint{0.598648in}{1.461341in}}%
\pgfpathclose%
\pgfusepath{stroke,fill}%
\end{pgfscope}%
\begin{pgfscope}%
\pgfpathrectangle{\pgfqpoint{0.100000in}{0.220728in}}{\pgfqpoint{3.696000in}{3.696000in}}%
\pgfusepath{clip}%
\pgfsetbuttcap%
\pgfsetroundjoin%
\definecolor{currentfill}{rgb}{0.121569,0.466667,0.705882}%
\pgfsetfillcolor{currentfill}%
\pgfsetfillopacity{0.653733}%
\pgfsetlinewidth{1.003750pt}%
\definecolor{currentstroke}{rgb}{0.121569,0.466667,0.705882}%
\pgfsetstrokecolor{currentstroke}%
\pgfsetstrokeopacity{0.653733}%
\pgfsetdash{}{0pt}%
\pgfpathmoveto{\pgfqpoint{3.299959in}{2.954873in}}%
\pgfpathcurveto{\pgfqpoint{3.308195in}{2.954873in}}{\pgfqpoint{3.316095in}{2.958146in}}{\pgfqpoint{3.321919in}{2.963969in}}%
\pgfpathcurveto{\pgfqpoint{3.327743in}{2.969793in}}{\pgfqpoint{3.331015in}{2.977693in}}{\pgfqpoint{3.331015in}{2.985930in}}%
\pgfpathcurveto{\pgfqpoint{3.331015in}{2.994166in}}{\pgfqpoint{3.327743in}{3.002066in}}{\pgfqpoint{3.321919in}{3.007890in}}%
\pgfpathcurveto{\pgfqpoint{3.316095in}{3.013714in}}{\pgfqpoint{3.308195in}{3.016986in}}{\pgfqpoint{3.299959in}{3.016986in}}%
\pgfpathcurveto{\pgfqpoint{3.291723in}{3.016986in}}{\pgfqpoint{3.283823in}{3.013714in}}{\pgfqpoint{3.277999in}{3.007890in}}%
\pgfpathcurveto{\pgfqpoint{3.272175in}{3.002066in}}{\pgfqpoint{3.268902in}{2.994166in}}{\pgfqpoint{3.268902in}{2.985930in}}%
\pgfpathcurveto{\pgfqpoint{3.268902in}{2.977693in}}{\pgfqpoint{3.272175in}{2.969793in}}{\pgfqpoint{3.277999in}{2.963969in}}%
\pgfpathcurveto{\pgfqpoint{3.283823in}{2.958146in}}{\pgfqpoint{3.291723in}{2.954873in}}{\pgfqpoint{3.299959in}{2.954873in}}%
\pgfpathclose%
\pgfusepath{stroke,fill}%
\end{pgfscope}%
\begin{pgfscope}%
\pgfpathrectangle{\pgfqpoint{0.100000in}{0.220728in}}{\pgfqpoint{3.696000in}{3.696000in}}%
\pgfusepath{clip}%
\pgfsetbuttcap%
\pgfsetroundjoin%
\definecolor{currentfill}{rgb}{0.121569,0.466667,0.705882}%
\pgfsetfillcolor{currentfill}%
\pgfsetfillopacity{0.655168}%
\pgfsetlinewidth{1.003750pt}%
\definecolor{currentstroke}{rgb}{0.121569,0.466667,0.705882}%
\pgfsetstrokecolor{currentstroke}%
\pgfsetstrokeopacity{0.655168}%
\pgfsetdash{}{0pt}%
\pgfpathmoveto{\pgfqpoint{0.602904in}{1.466665in}}%
\pgfpathcurveto{\pgfqpoint{0.611141in}{1.466665in}}{\pgfqpoint{0.619041in}{1.469937in}}{\pgfqpoint{0.624865in}{1.475761in}}%
\pgfpathcurveto{\pgfqpoint{0.630689in}{1.481585in}}{\pgfqpoint{0.633961in}{1.489485in}}{\pgfqpoint{0.633961in}{1.497721in}}%
\pgfpathcurveto{\pgfqpoint{0.633961in}{1.505957in}}{\pgfqpoint{0.630689in}{1.513857in}}{\pgfqpoint{0.624865in}{1.519681in}}%
\pgfpathcurveto{\pgfqpoint{0.619041in}{1.525505in}}{\pgfqpoint{0.611141in}{1.528778in}}{\pgfqpoint{0.602904in}{1.528778in}}%
\pgfpathcurveto{\pgfqpoint{0.594668in}{1.528778in}}{\pgfqpoint{0.586768in}{1.525505in}}{\pgfqpoint{0.580944in}{1.519681in}}%
\pgfpathcurveto{\pgfqpoint{0.575120in}{1.513857in}}{\pgfqpoint{0.571848in}{1.505957in}}{\pgfqpoint{0.571848in}{1.497721in}}%
\pgfpathcurveto{\pgfqpoint{0.571848in}{1.489485in}}{\pgfqpoint{0.575120in}{1.481585in}}{\pgfqpoint{0.580944in}{1.475761in}}%
\pgfpathcurveto{\pgfqpoint{0.586768in}{1.469937in}}{\pgfqpoint{0.594668in}{1.466665in}}{\pgfqpoint{0.602904in}{1.466665in}}%
\pgfpathclose%
\pgfusepath{stroke,fill}%
\end{pgfscope}%
\begin{pgfscope}%
\pgfpathrectangle{\pgfqpoint{0.100000in}{0.220728in}}{\pgfqpoint{3.696000in}{3.696000in}}%
\pgfusepath{clip}%
\pgfsetbuttcap%
\pgfsetroundjoin%
\definecolor{currentfill}{rgb}{0.121569,0.466667,0.705882}%
\pgfsetfillcolor{currentfill}%
\pgfsetfillopacity{0.655329}%
\pgfsetlinewidth{1.003750pt}%
\definecolor{currentstroke}{rgb}{0.121569,0.466667,0.705882}%
\pgfsetstrokecolor{currentstroke}%
\pgfsetstrokeopacity{0.655329}%
\pgfsetdash{}{0pt}%
\pgfpathmoveto{\pgfqpoint{3.331233in}{2.953830in}}%
\pgfpathcurveto{\pgfqpoint{3.339469in}{2.953830in}}{\pgfqpoint{3.347369in}{2.957102in}}{\pgfqpoint{3.353193in}{2.962926in}}%
\pgfpathcurveto{\pgfqpoint{3.359017in}{2.968750in}}{\pgfqpoint{3.362289in}{2.976650in}}{\pgfqpoint{3.362289in}{2.984886in}}%
\pgfpathcurveto{\pgfqpoint{3.362289in}{2.993123in}}{\pgfqpoint{3.359017in}{3.001023in}}{\pgfqpoint{3.353193in}{3.006847in}}%
\pgfpathcurveto{\pgfqpoint{3.347369in}{3.012671in}}{\pgfqpoint{3.339469in}{3.015943in}}{\pgfqpoint{3.331233in}{3.015943in}}%
\pgfpathcurveto{\pgfqpoint{3.322997in}{3.015943in}}{\pgfqpoint{3.315097in}{3.012671in}}{\pgfqpoint{3.309273in}{3.006847in}}%
\pgfpathcurveto{\pgfqpoint{3.303449in}{3.001023in}}{\pgfqpoint{3.300176in}{2.993123in}}{\pgfqpoint{3.300176in}{2.984886in}}%
\pgfpathcurveto{\pgfqpoint{3.300176in}{2.976650in}}{\pgfqpoint{3.303449in}{2.968750in}}{\pgfqpoint{3.309273in}{2.962926in}}%
\pgfpathcurveto{\pgfqpoint{3.315097in}{2.957102in}}{\pgfqpoint{3.322997in}{2.953830in}}{\pgfqpoint{3.331233in}{2.953830in}}%
\pgfpathclose%
\pgfusepath{stroke,fill}%
\end{pgfscope}%
\begin{pgfscope}%
\pgfpathrectangle{\pgfqpoint{0.100000in}{0.220728in}}{\pgfqpoint{3.696000in}{3.696000in}}%
\pgfusepath{clip}%
\pgfsetbuttcap%
\pgfsetroundjoin%
\definecolor{currentfill}{rgb}{0.121569,0.466667,0.705882}%
\pgfsetfillcolor{currentfill}%
\pgfsetfillopacity{0.656693}%
\pgfsetlinewidth{1.003750pt}%
\definecolor{currentstroke}{rgb}{0.121569,0.466667,0.705882}%
\pgfsetstrokecolor{currentstroke}%
\pgfsetstrokeopacity{0.656693}%
\pgfsetdash{}{0pt}%
\pgfpathmoveto{\pgfqpoint{3.314645in}{2.953323in}}%
\pgfpathcurveto{\pgfqpoint{3.322881in}{2.953323in}}{\pgfqpoint{3.330781in}{2.956595in}}{\pgfqpoint{3.336605in}{2.962419in}}%
\pgfpathcurveto{\pgfqpoint{3.342429in}{2.968243in}}{\pgfqpoint{3.345701in}{2.976143in}}{\pgfqpoint{3.345701in}{2.984379in}}%
\pgfpathcurveto{\pgfqpoint{3.345701in}{2.992616in}}{\pgfqpoint{3.342429in}{3.000516in}}{\pgfqpoint{3.336605in}{3.006340in}}%
\pgfpathcurveto{\pgfqpoint{3.330781in}{3.012164in}}{\pgfqpoint{3.322881in}{3.015436in}}{\pgfqpoint{3.314645in}{3.015436in}}%
\pgfpathcurveto{\pgfqpoint{3.306408in}{3.015436in}}{\pgfqpoint{3.298508in}{3.012164in}}{\pgfqpoint{3.292684in}{3.006340in}}%
\pgfpathcurveto{\pgfqpoint{3.286860in}{3.000516in}}{\pgfqpoint{3.283588in}{2.992616in}}{\pgfqpoint{3.283588in}{2.984379in}}%
\pgfpathcurveto{\pgfqpoint{3.283588in}{2.976143in}}{\pgfqpoint{3.286860in}{2.968243in}}{\pgfqpoint{3.292684in}{2.962419in}}%
\pgfpathcurveto{\pgfqpoint{3.298508in}{2.956595in}}{\pgfqpoint{3.306408in}{2.953323in}}{\pgfqpoint{3.314645in}{2.953323in}}%
\pgfpathclose%
\pgfusepath{stroke,fill}%
\end{pgfscope}%
\begin{pgfscope}%
\pgfpathrectangle{\pgfqpoint{0.100000in}{0.220728in}}{\pgfqpoint{3.696000in}{3.696000in}}%
\pgfusepath{clip}%
\pgfsetbuttcap%
\pgfsetroundjoin%
\definecolor{currentfill}{rgb}{0.121569,0.466667,0.705882}%
\pgfsetfillcolor{currentfill}%
\pgfsetfillopacity{0.656735}%
\pgfsetlinewidth{1.003750pt}%
\definecolor{currentstroke}{rgb}{0.121569,0.466667,0.705882}%
\pgfsetstrokecolor{currentstroke}%
\pgfsetstrokeopacity{0.656735}%
\pgfsetdash{}{0pt}%
\pgfpathmoveto{\pgfqpoint{0.606727in}{1.469309in}}%
\pgfpathcurveto{\pgfqpoint{0.614963in}{1.469309in}}{\pgfqpoint{0.622864in}{1.472582in}}{\pgfqpoint{0.628687in}{1.478406in}}%
\pgfpathcurveto{\pgfqpoint{0.634511in}{1.484229in}}{\pgfqpoint{0.637784in}{1.492130in}}{\pgfqpoint{0.637784in}{1.500366in}}%
\pgfpathcurveto{\pgfqpoint{0.637784in}{1.508602in}}{\pgfqpoint{0.634511in}{1.516502in}}{\pgfqpoint{0.628687in}{1.522326in}}%
\pgfpathcurveto{\pgfqpoint{0.622864in}{1.528150in}}{\pgfqpoint{0.614963in}{1.531422in}}{\pgfqpoint{0.606727in}{1.531422in}}%
\pgfpathcurveto{\pgfqpoint{0.598491in}{1.531422in}}{\pgfqpoint{0.590591in}{1.528150in}}{\pgfqpoint{0.584767in}{1.522326in}}%
\pgfpathcurveto{\pgfqpoint{0.578943in}{1.516502in}}{\pgfqpoint{0.575671in}{1.508602in}}{\pgfqpoint{0.575671in}{1.500366in}}%
\pgfpathcurveto{\pgfqpoint{0.575671in}{1.492130in}}{\pgfqpoint{0.578943in}{1.484229in}}{\pgfqpoint{0.584767in}{1.478406in}}%
\pgfpathcurveto{\pgfqpoint{0.590591in}{1.472582in}}{\pgfqpoint{0.598491in}{1.469309in}}{\pgfqpoint{0.606727in}{1.469309in}}%
\pgfpathclose%
\pgfusepath{stroke,fill}%
\end{pgfscope}%
\begin{pgfscope}%
\pgfpathrectangle{\pgfqpoint{0.100000in}{0.220728in}}{\pgfqpoint{3.696000in}{3.696000in}}%
\pgfusepath{clip}%
\pgfsetbuttcap%
\pgfsetroundjoin%
\definecolor{currentfill}{rgb}{0.121569,0.466667,0.705882}%
\pgfsetfillcolor{currentfill}%
\pgfsetfillopacity{0.658276}%
\pgfsetlinewidth{1.003750pt}%
\definecolor{currentstroke}{rgb}{0.121569,0.466667,0.705882}%
\pgfsetstrokecolor{currentstroke}%
\pgfsetstrokeopacity{0.658276}%
\pgfsetdash{}{0pt}%
\pgfpathmoveto{\pgfqpoint{0.614205in}{1.470305in}}%
\pgfpathcurveto{\pgfqpoint{0.622441in}{1.470305in}}{\pgfqpoint{0.630341in}{1.473578in}}{\pgfqpoint{0.636165in}{1.479402in}}%
\pgfpathcurveto{\pgfqpoint{0.641989in}{1.485226in}}{\pgfqpoint{0.645261in}{1.493126in}}{\pgfqpoint{0.645261in}{1.501362in}}%
\pgfpathcurveto{\pgfqpoint{0.645261in}{1.509598in}}{\pgfqpoint{0.641989in}{1.517498in}}{\pgfqpoint{0.636165in}{1.523322in}}%
\pgfpathcurveto{\pgfqpoint{0.630341in}{1.529146in}}{\pgfqpoint{0.622441in}{1.532418in}}{\pgfqpoint{0.614205in}{1.532418in}}%
\pgfpathcurveto{\pgfqpoint{0.605969in}{1.532418in}}{\pgfqpoint{0.598069in}{1.529146in}}{\pgfqpoint{0.592245in}{1.523322in}}%
\pgfpathcurveto{\pgfqpoint{0.586421in}{1.517498in}}{\pgfqpoint{0.583148in}{1.509598in}}{\pgfqpoint{0.583148in}{1.501362in}}%
\pgfpathcurveto{\pgfqpoint{0.583148in}{1.493126in}}{\pgfqpoint{0.586421in}{1.485226in}}{\pgfqpoint{0.592245in}{1.479402in}}%
\pgfpathcurveto{\pgfqpoint{0.598069in}{1.473578in}}{\pgfqpoint{0.605969in}{1.470305in}}{\pgfqpoint{0.614205in}{1.470305in}}%
\pgfpathclose%
\pgfusepath{stroke,fill}%
\end{pgfscope}%
\begin{pgfscope}%
\pgfpathrectangle{\pgfqpoint{0.100000in}{0.220728in}}{\pgfqpoint{3.696000in}{3.696000in}}%
\pgfusepath{clip}%
\pgfsetbuttcap%
\pgfsetroundjoin%
\definecolor{currentfill}{rgb}{0.121569,0.466667,0.705882}%
\pgfsetfillcolor{currentfill}%
\pgfsetfillopacity{0.658318}%
\pgfsetlinewidth{1.003750pt}%
\definecolor{currentstroke}{rgb}{0.121569,0.466667,0.705882}%
\pgfsetstrokecolor{currentstroke}%
\pgfsetstrokeopacity{0.658318}%
\pgfsetdash{}{0pt}%
\pgfpathmoveto{\pgfqpoint{0.624870in}{1.460544in}}%
\pgfpathcurveto{\pgfqpoint{0.633106in}{1.460544in}}{\pgfqpoint{0.641006in}{1.463816in}}{\pgfqpoint{0.646830in}{1.469640in}}%
\pgfpathcurveto{\pgfqpoint{0.652654in}{1.475464in}}{\pgfqpoint{0.655926in}{1.483364in}}{\pgfqpoint{0.655926in}{1.491601in}}%
\pgfpathcurveto{\pgfqpoint{0.655926in}{1.499837in}}{\pgfqpoint{0.652654in}{1.507737in}}{\pgfqpoint{0.646830in}{1.513561in}}%
\pgfpathcurveto{\pgfqpoint{0.641006in}{1.519385in}}{\pgfqpoint{0.633106in}{1.522657in}}{\pgfqpoint{0.624870in}{1.522657in}}%
\pgfpathcurveto{\pgfqpoint{0.616633in}{1.522657in}}{\pgfqpoint{0.608733in}{1.519385in}}{\pgfqpoint{0.602909in}{1.513561in}}%
\pgfpathcurveto{\pgfqpoint{0.597085in}{1.507737in}}{\pgfqpoint{0.593813in}{1.499837in}}{\pgfqpoint{0.593813in}{1.491601in}}%
\pgfpathcurveto{\pgfqpoint{0.593813in}{1.483364in}}{\pgfqpoint{0.597085in}{1.475464in}}{\pgfqpoint{0.602909in}{1.469640in}}%
\pgfpathcurveto{\pgfqpoint{0.608733in}{1.463816in}}{\pgfqpoint{0.616633in}{1.460544in}}{\pgfqpoint{0.624870in}{1.460544in}}%
\pgfpathclose%
\pgfusepath{stroke,fill}%
\end{pgfscope}%
\begin{pgfscope}%
\pgfpathrectangle{\pgfqpoint{0.100000in}{0.220728in}}{\pgfqpoint{3.696000in}{3.696000in}}%
\pgfusepath{clip}%
\pgfsetbuttcap%
\pgfsetroundjoin%
\definecolor{currentfill}{rgb}{0.121569,0.466667,0.705882}%
\pgfsetfillcolor{currentfill}%
\pgfsetfillopacity{0.658524}%
\pgfsetlinewidth{1.003750pt}%
\definecolor{currentstroke}{rgb}{0.121569,0.466667,0.705882}%
\pgfsetstrokecolor{currentstroke}%
\pgfsetstrokeopacity{0.658524}%
\pgfsetdash{}{0pt}%
\pgfpathmoveto{\pgfqpoint{0.626311in}{1.458672in}}%
\pgfpathcurveto{\pgfqpoint{0.634548in}{1.458672in}}{\pgfqpoint{0.642448in}{1.461944in}}{\pgfqpoint{0.648272in}{1.467768in}}%
\pgfpathcurveto{\pgfqpoint{0.654096in}{1.473592in}}{\pgfqpoint{0.657368in}{1.481492in}}{\pgfqpoint{0.657368in}{1.489728in}}%
\pgfpathcurveto{\pgfqpoint{0.657368in}{1.497965in}}{\pgfqpoint{0.654096in}{1.505865in}}{\pgfqpoint{0.648272in}{1.511689in}}%
\pgfpathcurveto{\pgfqpoint{0.642448in}{1.517512in}}{\pgfqpoint{0.634548in}{1.520785in}}{\pgfqpoint{0.626311in}{1.520785in}}%
\pgfpathcurveto{\pgfqpoint{0.618075in}{1.520785in}}{\pgfqpoint{0.610175in}{1.517512in}}{\pgfqpoint{0.604351in}{1.511689in}}%
\pgfpathcurveto{\pgfqpoint{0.598527in}{1.505865in}}{\pgfqpoint{0.595255in}{1.497965in}}{\pgfqpoint{0.595255in}{1.489728in}}%
\pgfpathcurveto{\pgfqpoint{0.595255in}{1.481492in}}{\pgfqpoint{0.598527in}{1.473592in}}{\pgfqpoint{0.604351in}{1.467768in}}%
\pgfpathcurveto{\pgfqpoint{0.610175in}{1.461944in}}{\pgfqpoint{0.618075in}{1.458672in}}{\pgfqpoint{0.626311in}{1.458672in}}%
\pgfpathclose%
\pgfusepath{stroke,fill}%
\end{pgfscope}%
\begin{pgfscope}%
\pgfpathrectangle{\pgfqpoint{0.100000in}{0.220728in}}{\pgfqpoint{3.696000in}{3.696000in}}%
\pgfusepath{clip}%
\pgfsetbuttcap%
\pgfsetroundjoin%
\definecolor{currentfill}{rgb}{0.121569,0.466667,0.705882}%
\pgfsetfillcolor{currentfill}%
\pgfsetfillopacity{0.658525}%
\pgfsetlinewidth{1.003750pt}%
\definecolor{currentstroke}{rgb}{0.121569,0.466667,0.705882}%
\pgfsetstrokecolor{currentstroke}%
\pgfsetstrokeopacity{0.658525}%
\pgfsetdash{}{0pt}%
\pgfpathmoveto{\pgfqpoint{3.347257in}{2.949768in}}%
\pgfpathcurveto{\pgfqpoint{3.355493in}{2.949768in}}{\pgfqpoint{3.363393in}{2.953040in}}{\pgfqpoint{3.369217in}{2.958864in}}%
\pgfpathcurveto{\pgfqpoint{3.375041in}{2.964688in}}{\pgfqpoint{3.378314in}{2.972588in}}{\pgfqpoint{3.378314in}{2.980824in}}%
\pgfpathcurveto{\pgfqpoint{3.378314in}{2.989061in}}{\pgfqpoint{3.375041in}{2.996961in}}{\pgfqpoint{3.369217in}{3.002785in}}%
\pgfpathcurveto{\pgfqpoint{3.363393in}{3.008609in}}{\pgfqpoint{3.355493in}{3.011881in}}{\pgfqpoint{3.347257in}{3.011881in}}%
\pgfpathcurveto{\pgfqpoint{3.339021in}{3.011881in}}{\pgfqpoint{3.331121in}{3.008609in}}{\pgfqpoint{3.325297in}{3.002785in}}%
\pgfpathcurveto{\pgfqpoint{3.319473in}{2.996961in}}{\pgfqpoint{3.316201in}{2.989061in}}{\pgfqpoint{3.316201in}{2.980824in}}%
\pgfpathcurveto{\pgfqpoint{3.316201in}{2.972588in}}{\pgfqpoint{3.319473in}{2.964688in}}{\pgfqpoint{3.325297in}{2.958864in}}%
\pgfpathcurveto{\pgfqpoint{3.331121in}{2.953040in}}{\pgfqpoint{3.339021in}{2.949768in}}{\pgfqpoint{3.347257in}{2.949768in}}%
\pgfpathclose%
\pgfusepath{stroke,fill}%
\end{pgfscope}%
\begin{pgfscope}%
\pgfpathrectangle{\pgfqpoint{0.100000in}{0.220728in}}{\pgfqpoint{3.696000in}{3.696000in}}%
\pgfusepath{clip}%
\pgfsetbuttcap%
\pgfsetroundjoin%
\definecolor{currentfill}{rgb}{0.121569,0.466667,0.705882}%
\pgfsetfillcolor{currentfill}%
\pgfsetfillopacity{0.659146}%
\pgfsetlinewidth{1.003750pt}%
\definecolor{currentstroke}{rgb}{0.121569,0.466667,0.705882}%
\pgfsetstrokecolor{currentstroke}%
\pgfsetstrokeopacity{0.659146}%
\pgfsetdash{}{0pt}%
\pgfpathmoveto{\pgfqpoint{0.620330in}{1.469471in}}%
\pgfpathcurveto{\pgfqpoint{0.628567in}{1.469471in}}{\pgfqpoint{0.636467in}{1.472743in}}{\pgfqpoint{0.642291in}{1.478567in}}%
\pgfpathcurveto{\pgfqpoint{0.648115in}{1.484391in}}{\pgfqpoint{0.651387in}{1.492291in}}{\pgfqpoint{0.651387in}{1.500527in}}%
\pgfpathcurveto{\pgfqpoint{0.651387in}{1.508763in}}{\pgfqpoint{0.648115in}{1.516663in}}{\pgfqpoint{0.642291in}{1.522487in}}%
\pgfpathcurveto{\pgfqpoint{0.636467in}{1.528311in}}{\pgfqpoint{0.628567in}{1.531584in}}{\pgfqpoint{0.620330in}{1.531584in}}%
\pgfpathcurveto{\pgfqpoint{0.612094in}{1.531584in}}{\pgfqpoint{0.604194in}{1.528311in}}{\pgfqpoint{0.598370in}{1.522487in}}%
\pgfpathcurveto{\pgfqpoint{0.592546in}{1.516663in}}{\pgfqpoint{0.589274in}{1.508763in}}{\pgfqpoint{0.589274in}{1.500527in}}%
\pgfpathcurveto{\pgfqpoint{0.589274in}{1.492291in}}{\pgfqpoint{0.592546in}{1.484391in}}{\pgfqpoint{0.598370in}{1.478567in}}%
\pgfpathcurveto{\pgfqpoint{0.604194in}{1.472743in}}{\pgfqpoint{0.612094in}{1.469471in}}{\pgfqpoint{0.620330in}{1.469471in}}%
\pgfpathclose%
\pgfusepath{stroke,fill}%
\end{pgfscope}%
\begin{pgfscope}%
\pgfpathrectangle{\pgfqpoint{0.100000in}{0.220728in}}{\pgfqpoint{3.696000in}{3.696000in}}%
\pgfusepath{clip}%
\pgfsetbuttcap%
\pgfsetroundjoin%
\definecolor{currentfill}{rgb}{0.121569,0.466667,0.705882}%
\pgfsetfillcolor{currentfill}%
\pgfsetfillopacity{0.659834}%
\pgfsetlinewidth{1.003750pt}%
\definecolor{currentstroke}{rgb}{0.121569,0.466667,0.705882}%
\pgfsetstrokecolor{currentstroke}%
\pgfsetstrokeopacity{0.659834}%
\pgfsetdash{}{0pt}%
\pgfpathmoveto{\pgfqpoint{0.628511in}{1.458447in}}%
\pgfpathcurveto{\pgfqpoint{0.636747in}{1.458447in}}{\pgfqpoint{0.644648in}{1.461719in}}{\pgfqpoint{0.650471in}{1.467543in}}%
\pgfpathcurveto{\pgfqpoint{0.656295in}{1.473367in}}{\pgfqpoint{0.659568in}{1.481267in}}{\pgfqpoint{0.659568in}{1.489503in}}%
\pgfpathcurveto{\pgfqpoint{0.659568in}{1.497740in}}{\pgfqpoint{0.656295in}{1.505640in}}{\pgfqpoint{0.650471in}{1.511464in}}%
\pgfpathcurveto{\pgfqpoint{0.644648in}{1.517288in}}{\pgfqpoint{0.636747in}{1.520560in}}{\pgfqpoint{0.628511in}{1.520560in}}%
\pgfpathcurveto{\pgfqpoint{0.620275in}{1.520560in}}{\pgfqpoint{0.612375in}{1.517288in}}{\pgfqpoint{0.606551in}{1.511464in}}%
\pgfpathcurveto{\pgfqpoint{0.600727in}{1.505640in}}{\pgfqpoint{0.597455in}{1.497740in}}{\pgfqpoint{0.597455in}{1.489503in}}%
\pgfpathcurveto{\pgfqpoint{0.597455in}{1.481267in}}{\pgfqpoint{0.600727in}{1.473367in}}{\pgfqpoint{0.606551in}{1.467543in}}%
\pgfpathcurveto{\pgfqpoint{0.612375in}{1.461719in}}{\pgfqpoint{0.620275in}{1.458447in}}{\pgfqpoint{0.628511in}{1.458447in}}%
\pgfpathclose%
\pgfusepath{stroke,fill}%
\end{pgfscope}%
\begin{pgfscope}%
\pgfpathrectangle{\pgfqpoint{0.100000in}{0.220728in}}{\pgfqpoint{3.696000in}{3.696000in}}%
\pgfusepath{clip}%
\pgfsetbuttcap%
\pgfsetroundjoin%
\definecolor{currentfill}{rgb}{0.121569,0.466667,0.705882}%
\pgfsetfillcolor{currentfill}%
\pgfsetfillopacity{0.660905}%
\pgfsetlinewidth{1.003750pt}%
\definecolor{currentstroke}{rgb}{0.121569,0.466667,0.705882}%
\pgfsetstrokecolor{currentstroke}%
\pgfsetstrokeopacity{0.660905}%
\pgfsetdash{}{0pt}%
\pgfpathmoveto{\pgfqpoint{0.630037in}{1.458301in}}%
\pgfpathcurveto{\pgfqpoint{0.638273in}{1.458301in}}{\pgfqpoint{0.646173in}{1.461573in}}{\pgfqpoint{0.651997in}{1.467397in}}%
\pgfpathcurveto{\pgfqpoint{0.657821in}{1.473221in}}{\pgfqpoint{0.661093in}{1.481121in}}{\pgfqpoint{0.661093in}{1.489358in}}%
\pgfpathcurveto{\pgfqpoint{0.661093in}{1.497594in}}{\pgfqpoint{0.657821in}{1.505494in}}{\pgfqpoint{0.651997in}{1.511318in}}%
\pgfpathcurveto{\pgfqpoint{0.646173in}{1.517142in}}{\pgfqpoint{0.638273in}{1.520414in}}{\pgfqpoint{0.630037in}{1.520414in}}%
\pgfpathcurveto{\pgfqpoint{0.621800in}{1.520414in}}{\pgfqpoint{0.613900in}{1.517142in}}{\pgfqpoint{0.608076in}{1.511318in}}%
\pgfpathcurveto{\pgfqpoint{0.602252in}{1.505494in}}{\pgfqpoint{0.598980in}{1.497594in}}{\pgfqpoint{0.598980in}{1.489358in}}%
\pgfpathcurveto{\pgfqpoint{0.598980in}{1.481121in}}{\pgfqpoint{0.602252in}{1.473221in}}{\pgfqpoint{0.608076in}{1.467397in}}%
\pgfpathcurveto{\pgfqpoint{0.613900in}{1.461573in}}{\pgfqpoint{0.621800in}{1.458301in}}{\pgfqpoint{0.630037in}{1.458301in}}%
\pgfpathclose%
\pgfusepath{stroke,fill}%
\end{pgfscope}%
\begin{pgfscope}%
\pgfpathrectangle{\pgfqpoint{0.100000in}{0.220728in}}{\pgfqpoint{3.696000in}{3.696000in}}%
\pgfusepath{clip}%
\pgfsetbuttcap%
\pgfsetroundjoin%
\definecolor{currentfill}{rgb}{0.121569,0.466667,0.705882}%
\pgfsetfillcolor{currentfill}%
\pgfsetfillopacity{0.662348}%
\pgfsetlinewidth{1.003750pt}%
\definecolor{currentstroke}{rgb}{0.121569,0.466667,0.705882}%
\pgfsetstrokecolor{currentstroke}%
\pgfsetstrokeopacity{0.662348}%
\pgfsetdash{}{0pt}%
\pgfpathmoveto{\pgfqpoint{3.364369in}{2.948492in}}%
\pgfpathcurveto{\pgfqpoint{3.372606in}{2.948492in}}{\pgfqpoint{3.380506in}{2.951764in}}{\pgfqpoint{3.386330in}{2.957588in}}%
\pgfpathcurveto{\pgfqpoint{3.392154in}{2.963412in}}{\pgfqpoint{3.395426in}{2.971312in}}{\pgfqpoint{3.395426in}{2.979548in}}%
\pgfpathcurveto{\pgfqpoint{3.395426in}{2.987785in}}{\pgfqpoint{3.392154in}{2.995685in}}{\pgfqpoint{3.386330in}{3.001509in}}%
\pgfpathcurveto{\pgfqpoint{3.380506in}{3.007333in}}{\pgfqpoint{3.372606in}{3.010605in}}{\pgfqpoint{3.364369in}{3.010605in}}%
\pgfpathcurveto{\pgfqpoint{3.356133in}{3.010605in}}{\pgfqpoint{3.348233in}{3.007333in}}{\pgfqpoint{3.342409in}{3.001509in}}%
\pgfpathcurveto{\pgfqpoint{3.336585in}{2.995685in}}{\pgfqpoint{3.333313in}{2.987785in}}{\pgfqpoint{3.333313in}{2.979548in}}%
\pgfpathcurveto{\pgfqpoint{3.333313in}{2.971312in}}{\pgfqpoint{3.336585in}{2.963412in}}{\pgfqpoint{3.342409in}{2.957588in}}%
\pgfpathcurveto{\pgfqpoint{3.348233in}{2.951764in}}{\pgfqpoint{3.356133in}{2.948492in}}{\pgfqpoint{3.364369in}{2.948492in}}%
\pgfpathclose%
\pgfusepath{stroke,fill}%
\end{pgfscope}%
\begin{pgfscope}%
\pgfpathrectangle{\pgfqpoint{0.100000in}{0.220728in}}{\pgfqpoint{3.696000in}{3.696000in}}%
\pgfusepath{clip}%
\pgfsetbuttcap%
\pgfsetroundjoin%
\definecolor{currentfill}{rgb}{0.121569,0.466667,0.705882}%
\pgfsetfillcolor{currentfill}%
\pgfsetfillopacity{0.662484}%
\pgfsetlinewidth{1.003750pt}%
\definecolor{currentstroke}{rgb}{0.121569,0.466667,0.705882}%
\pgfsetstrokecolor{currentstroke}%
\pgfsetstrokeopacity{0.662484}%
\pgfsetdash{}{0pt}%
\pgfpathmoveto{\pgfqpoint{0.633219in}{1.457205in}}%
\pgfpathcurveto{\pgfqpoint{0.641456in}{1.457205in}}{\pgfqpoint{0.649356in}{1.460478in}}{\pgfqpoint{0.655180in}{1.466301in}}%
\pgfpathcurveto{\pgfqpoint{0.661003in}{1.472125in}}{\pgfqpoint{0.664276in}{1.480025in}}{\pgfqpoint{0.664276in}{1.488262in}}%
\pgfpathcurveto{\pgfqpoint{0.664276in}{1.496498in}}{\pgfqpoint{0.661003in}{1.504398in}}{\pgfqpoint{0.655180in}{1.510222in}}%
\pgfpathcurveto{\pgfqpoint{0.649356in}{1.516046in}}{\pgfqpoint{0.641456in}{1.519318in}}{\pgfqpoint{0.633219in}{1.519318in}}%
\pgfpathcurveto{\pgfqpoint{0.624983in}{1.519318in}}{\pgfqpoint{0.617083in}{1.516046in}}{\pgfqpoint{0.611259in}{1.510222in}}%
\pgfpathcurveto{\pgfqpoint{0.605435in}{1.504398in}}{\pgfqpoint{0.602163in}{1.496498in}}{\pgfqpoint{0.602163in}{1.488262in}}%
\pgfpathcurveto{\pgfqpoint{0.602163in}{1.480025in}}{\pgfqpoint{0.605435in}{1.472125in}}{\pgfqpoint{0.611259in}{1.466301in}}%
\pgfpathcurveto{\pgfqpoint{0.617083in}{1.460478in}}{\pgfqpoint{0.624983in}{1.457205in}}{\pgfqpoint{0.633219in}{1.457205in}}%
\pgfpathclose%
\pgfusepath{stroke,fill}%
\end{pgfscope}%
\begin{pgfscope}%
\pgfpathrectangle{\pgfqpoint{0.100000in}{0.220728in}}{\pgfqpoint{3.696000in}{3.696000in}}%
\pgfusepath{clip}%
\pgfsetbuttcap%
\pgfsetroundjoin%
\definecolor{currentfill}{rgb}{0.121569,0.466667,0.705882}%
\pgfsetfillcolor{currentfill}%
\pgfsetfillopacity{0.663134}%
\pgfsetlinewidth{1.003750pt}%
\definecolor{currentstroke}{rgb}{0.121569,0.466667,0.705882}%
\pgfsetstrokecolor{currentstroke}%
\pgfsetstrokeopacity{0.663134}%
\pgfsetdash{}{0pt}%
\pgfpathmoveto{\pgfqpoint{3.374320in}{2.945638in}}%
\pgfpathcurveto{\pgfqpoint{3.382556in}{2.945638in}}{\pgfqpoint{3.390456in}{2.948910in}}{\pgfqpoint{3.396280in}{2.954734in}}%
\pgfpathcurveto{\pgfqpoint{3.402104in}{2.960558in}}{\pgfqpoint{3.405376in}{2.968458in}}{\pgfqpoint{3.405376in}{2.976695in}}%
\pgfpathcurveto{\pgfqpoint{3.405376in}{2.984931in}}{\pgfqpoint{3.402104in}{2.992831in}}{\pgfqpoint{3.396280in}{2.998655in}}%
\pgfpathcurveto{\pgfqpoint{3.390456in}{3.004479in}}{\pgfqpoint{3.382556in}{3.007751in}}{\pgfqpoint{3.374320in}{3.007751in}}%
\pgfpathcurveto{\pgfqpoint{3.366084in}{3.007751in}}{\pgfqpoint{3.358184in}{3.004479in}}{\pgfqpoint{3.352360in}{2.998655in}}%
\pgfpathcurveto{\pgfqpoint{3.346536in}{2.992831in}}{\pgfqpoint{3.343263in}{2.984931in}}{\pgfqpoint{3.343263in}{2.976695in}}%
\pgfpathcurveto{\pgfqpoint{3.343263in}{2.968458in}}{\pgfqpoint{3.346536in}{2.960558in}}{\pgfqpoint{3.352360in}{2.954734in}}%
\pgfpathcurveto{\pgfqpoint{3.358184in}{2.948910in}}{\pgfqpoint{3.366084in}{2.945638in}}{\pgfqpoint{3.374320in}{2.945638in}}%
\pgfpathclose%
\pgfusepath{stroke,fill}%
\end{pgfscope}%
\begin{pgfscope}%
\pgfpathrectangle{\pgfqpoint{0.100000in}{0.220728in}}{\pgfqpoint{3.696000in}{3.696000in}}%
\pgfusepath{clip}%
\pgfsetbuttcap%
\pgfsetroundjoin%
\definecolor{currentfill}{rgb}{0.121569,0.466667,0.705882}%
\pgfsetfillcolor{currentfill}%
\pgfsetfillopacity{0.663279}%
\pgfsetlinewidth{1.003750pt}%
\definecolor{currentstroke}{rgb}{0.121569,0.466667,0.705882}%
\pgfsetstrokecolor{currentstroke}%
\pgfsetstrokeopacity{0.663279}%
\pgfsetdash{}{0pt}%
\pgfpathmoveto{\pgfqpoint{0.640977in}{1.451110in}}%
\pgfpathcurveto{\pgfqpoint{0.649214in}{1.451110in}}{\pgfqpoint{0.657114in}{1.454382in}}{\pgfqpoint{0.662938in}{1.460206in}}%
\pgfpathcurveto{\pgfqpoint{0.668762in}{1.466030in}}{\pgfqpoint{0.672034in}{1.473930in}}{\pgfqpoint{0.672034in}{1.482166in}}%
\pgfpathcurveto{\pgfqpoint{0.672034in}{1.490402in}}{\pgfqpoint{0.668762in}{1.498302in}}{\pgfqpoint{0.662938in}{1.504126in}}%
\pgfpathcurveto{\pgfqpoint{0.657114in}{1.509950in}}{\pgfqpoint{0.649214in}{1.513223in}}{\pgfqpoint{0.640977in}{1.513223in}}%
\pgfpathcurveto{\pgfqpoint{0.632741in}{1.513223in}}{\pgfqpoint{0.624841in}{1.509950in}}{\pgfqpoint{0.619017in}{1.504126in}}%
\pgfpathcurveto{\pgfqpoint{0.613193in}{1.498302in}}{\pgfqpoint{0.609921in}{1.490402in}}{\pgfqpoint{0.609921in}{1.482166in}}%
\pgfpathcurveto{\pgfqpoint{0.609921in}{1.473930in}}{\pgfqpoint{0.613193in}{1.466030in}}{\pgfqpoint{0.619017in}{1.460206in}}%
\pgfpathcurveto{\pgfqpoint{0.624841in}{1.454382in}}{\pgfqpoint{0.632741in}{1.451110in}}{\pgfqpoint{0.640977in}{1.451110in}}%
\pgfpathclose%
\pgfusepath{stroke,fill}%
\end{pgfscope}%
\begin{pgfscope}%
\pgfpathrectangle{\pgfqpoint{0.100000in}{0.220728in}}{\pgfqpoint{3.696000in}{3.696000in}}%
\pgfusepath{clip}%
\pgfsetbuttcap%
\pgfsetroundjoin%
\definecolor{currentfill}{rgb}{0.121569,0.466667,0.705882}%
\pgfsetfillcolor{currentfill}%
\pgfsetfillopacity{0.664347}%
\pgfsetlinewidth{1.003750pt}%
\definecolor{currentstroke}{rgb}{0.121569,0.466667,0.705882}%
\pgfsetstrokecolor{currentstroke}%
\pgfsetstrokeopacity{0.664347}%
\pgfsetdash{}{0pt}%
\pgfpathmoveto{\pgfqpoint{3.379283in}{2.944697in}}%
\pgfpathcurveto{\pgfqpoint{3.387520in}{2.944697in}}{\pgfqpoint{3.395420in}{2.947969in}}{\pgfqpoint{3.401244in}{2.953793in}}%
\pgfpathcurveto{\pgfqpoint{3.407068in}{2.959617in}}{\pgfqpoint{3.410340in}{2.967517in}}{\pgfqpoint{3.410340in}{2.975753in}}%
\pgfpathcurveto{\pgfqpoint{3.410340in}{2.983990in}}{\pgfqpoint{3.407068in}{2.991890in}}{\pgfqpoint{3.401244in}{2.997714in}}%
\pgfpathcurveto{\pgfqpoint{3.395420in}{3.003538in}}{\pgfqpoint{3.387520in}{3.006810in}}{\pgfqpoint{3.379283in}{3.006810in}}%
\pgfpathcurveto{\pgfqpoint{3.371047in}{3.006810in}}{\pgfqpoint{3.363147in}{3.003538in}}{\pgfqpoint{3.357323in}{2.997714in}}%
\pgfpathcurveto{\pgfqpoint{3.351499in}{2.991890in}}{\pgfqpoint{3.348227in}{2.983990in}}{\pgfqpoint{3.348227in}{2.975753in}}%
\pgfpathcurveto{\pgfqpoint{3.348227in}{2.967517in}}{\pgfqpoint{3.351499in}{2.959617in}}{\pgfqpoint{3.357323in}{2.953793in}}%
\pgfpathcurveto{\pgfqpoint{3.363147in}{2.947969in}}{\pgfqpoint{3.371047in}{2.944697in}}{\pgfqpoint{3.379283in}{2.944697in}}%
\pgfpathclose%
\pgfusepath{stroke,fill}%
\end{pgfscope}%
\begin{pgfscope}%
\pgfpathrectangle{\pgfqpoint{0.100000in}{0.220728in}}{\pgfqpoint{3.696000in}{3.696000in}}%
\pgfusepath{clip}%
\pgfsetbuttcap%
\pgfsetroundjoin%
\definecolor{currentfill}{rgb}{0.121569,0.466667,0.705882}%
\pgfsetfillcolor{currentfill}%
\pgfsetfillopacity{0.665603}%
\pgfsetlinewidth{1.003750pt}%
\definecolor{currentstroke}{rgb}{0.121569,0.466667,0.705882}%
\pgfsetstrokecolor{currentstroke}%
\pgfsetstrokeopacity{0.665603}%
\pgfsetdash{}{0pt}%
\pgfpathmoveto{\pgfqpoint{3.385348in}{2.943175in}}%
\pgfpathcurveto{\pgfqpoint{3.393585in}{2.943175in}}{\pgfqpoint{3.401485in}{2.946447in}}{\pgfqpoint{3.407309in}{2.952271in}}%
\pgfpathcurveto{\pgfqpoint{3.413132in}{2.958095in}}{\pgfqpoint{3.416405in}{2.965995in}}{\pgfqpoint{3.416405in}{2.974231in}}%
\pgfpathcurveto{\pgfqpoint{3.416405in}{2.982468in}}{\pgfqpoint{3.413132in}{2.990368in}}{\pgfqpoint{3.407309in}{2.996192in}}%
\pgfpathcurveto{\pgfqpoint{3.401485in}{3.002016in}}{\pgfqpoint{3.393585in}{3.005288in}}{\pgfqpoint{3.385348in}{3.005288in}}%
\pgfpathcurveto{\pgfqpoint{3.377112in}{3.005288in}}{\pgfqpoint{3.369212in}{3.002016in}}{\pgfqpoint{3.363388in}{2.996192in}}%
\pgfpathcurveto{\pgfqpoint{3.357564in}{2.990368in}}{\pgfqpoint{3.354292in}{2.982468in}}{\pgfqpoint{3.354292in}{2.974231in}}%
\pgfpathcurveto{\pgfqpoint{3.354292in}{2.965995in}}{\pgfqpoint{3.357564in}{2.958095in}}{\pgfqpoint{3.363388in}{2.952271in}}%
\pgfpathcurveto{\pgfqpoint{3.369212in}{2.946447in}}{\pgfqpoint{3.377112in}{2.943175in}}{\pgfqpoint{3.385348in}{2.943175in}}%
\pgfpathclose%
\pgfusepath{stroke,fill}%
\end{pgfscope}%
\begin{pgfscope}%
\pgfpathrectangle{\pgfqpoint{0.100000in}{0.220728in}}{\pgfqpoint{3.696000in}{3.696000in}}%
\pgfusepath{clip}%
\pgfsetbuttcap%
\pgfsetroundjoin%
\definecolor{currentfill}{rgb}{0.121569,0.466667,0.705882}%
\pgfsetfillcolor{currentfill}%
\pgfsetfillopacity{0.665731}%
\pgfsetlinewidth{1.003750pt}%
\definecolor{currentstroke}{rgb}{0.121569,0.466667,0.705882}%
\pgfsetstrokecolor{currentstroke}%
\pgfsetstrokeopacity{0.665731}%
\pgfsetdash{}{0pt}%
\pgfpathmoveto{\pgfqpoint{0.654998in}{1.444319in}}%
\pgfpathcurveto{\pgfqpoint{0.663234in}{1.444319in}}{\pgfqpoint{0.671134in}{1.447591in}}{\pgfqpoint{0.676958in}{1.453415in}}%
\pgfpathcurveto{\pgfqpoint{0.682782in}{1.459239in}}{\pgfqpoint{0.686054in}{1.467139in}}{\pgfqpoint{0.686054in}{1.475375in}}%
\pgfpathcurveto{\pgfqpoint{0.686054in}{1.483612in}}{\pgfqpoint{0.682782in}{1.491512in}}{\pgfqpoint{0.676958in}{1.497336in}}%
\pgfpathcurveto{\pgfqpoint{0.671134in}{1.503160in}}{\pgfqpoint{0.663234in}{1.506432in}}{\pgfqpoint{0.654998in}{1.506432in}}%
\pgfpathcurveto{\pgfqpoint{0.646762in}{1.506432in}}{\pgfqpoint{0.638861in}{1.503160in}}{\pgfqpoint{0.633038in}{1.497336in}}%
\pgfpathcurveto{\pgfqpoint{0.627214in}{1.491512in}}{\pgfqpoint{0.623941in}{1.483612in}}{\pgfqpoint{0.623941in}{1.475375in}}%
\pgfpathcurveto{\pgfqpoint{0.623941in}{1.467139in}}{\pgfqpoint{0.627214in}{1.459239in}}{\pgfqpoint{0.633038in}{1.453415in}}%
\pgfpathcurveto{\pgfqpoint{0.638861in}{1.447591in}}{\pgfqpoint{0.646762in}{1.444319in}}{\pgfqpoint{0.654998in}{1.444319in}}%
\pgfpathclose%
\pgfusepath{stroke,fill}%
\end{pgfscope}%
\begin{pgfscope}%
\pgfpathrectangle{\pgfqpoint{0.100000in}{0.220728in}}{\pgfqpoint{3.696000in}{3.696000in}}%
\pgfusepath{clip}%
\pgfsetbuttcap%
\pgfsetroundjoin%
\definecolor{currentfill}{rgb}{0.121569,0.466667,0.705882}%
\pgfsetfillcolor{currentfill}%
\pgfsetfillopacity{0.665892}%
\pgfsetlinewidth{1.003750pt}%
\definecolor{currentstroke}{rgb}{0.121569,0.466667,0.705882}%
\pgfsetstrokecolor{currentstroke}%
\pgfsetstrokeopacity{0.665892}%
\pgfsetdash{}{0pt}%
\pgfpathmoveto{\pgfqpoint{3.388991in}{2.942295in}}%
\pgfpathcurveto{\pgfqpoint{3.397227in}{2.942295in}}{\pgfqpoint{3.405127in}{2.945567in}}{\pgfqpoint{3.410951in}{2.951391in}}%
\pgfpathcurveto{\pgfqpoint{3.416775in}{2.957215in}}{\pgfqpoint{3.420047in}{2.965115in}}{\pgfqpoint{3.420047in}{2.973352in}}%
\pgfpathcurveto{\pgfqpoint{3.420047in}{2.981588in}}{\pgfqpoint{3.416775in}{2.989488in}}{\pgfqpoint{3.410951in}{2.995312in}}%
\pgfpathcurveto{\pgfqpoint{3.405127in}{3.001136in}}{\pgfqpoint{3.397227in}{3.004408in}}{\pgfqpoint{3.388991in}{3.004408in}}%
\pgfpathcurveto{\pgfqpoint{3.380754in}{3.004408in}}{\pgfqpoint{3.372854in}{3.001136in}}{\pgfqpoint{3.367030in}{2.995312in}}%
\pgfpathcurveto{\pgfqpoint{3.361207in}{2.989488in}}{\pgfqpoint{3.357934in}{2.981588in}}{\pgfqpoint{3.357934in}{2.973352in}}%
\pgfpathcurveto{\pgfqpoint{3.357934in}{2.965115in}}{\pgfqpoint{3.361207in}{2.957215in}}{\pgfqpoint{3.367030in}{2.951391in}}%
\pgfpathcurveto{\pgfqpoint{3.372854in}{2.945567in}}{\pgfqpoint{3.380754in}{2.942295in}}{\pgfqpoint{3.388991in}{2.942295in}}%
\pgfpathclose%
\pgfusepath{stroke,fill}%
\end{pgfscope}%
\begin{pgfscope}%
\pgfpathrectangle{\pgfqpoint{0.100000in}{0.220728in}}{\pgfqpoint{3.696000in}{3.696000in}}%
\pgfusepath{clip}%
\pgfsetbuttcap%
\pgfsetroundjoin%
\definecolor{currentfill}{rgb}{0.121569,0.466667,0.705882}%
\pgfsetfillcolor{currentfill}%
\pgfsetfillopacity{0.666988}%
\pgfsetlinewidth{1.003750pt}%
\definecolor{currentstroke}{rgb}{0.121569,0.466667,0.705882}%
\pgfsetstrokecolor{currentstroke}%
\pgfsetstrokeopacity{0.666988}%
\pgfsetdash{}{0pt}%
\pgfpathmoveto{\pgfqpoint{3.393066in}{2.941607in}}%
\pgfpathcurveto{\pgfqpoint{3.401303in}{2.941607in}}{\pgfqpoint{3.409203in}{2.944880in}}{\pgfqpoint{3.415027in}{2.950704in}}%
\pgfpathcurveto{\pgfqpoint{3.420851in}{2.956527in}}{\pgfqpoint{3.424123in}{2.964427in}}{\pgfqpoint{3.424123in}{2.972664in}}%
\pgfpathcurveto{\pgfqpoint{3.424123in}{2.980900in}}{\pgfqpoint{3.420851in}{2.988800in}}{\pgfqpoint{3.415027in}{2.994624in}}%
\pgfpathcurveto{\pgfqpoint{3.409203in}{3.000448in}}{\pgfqpoint{3.401303in}{3.003720in}}{\pgfqpoint{3.393066in}{3.003720in}}%
\pgfpathcurveto{\pgfqpoint{3.384830in}{3.003720in}}{\pgfqpoint{3.376930in}{3.000448in}}{\pgfqpoint{3.371106in}{2.994624in}}%
\pgfpathcurveto{\pgfqpoint{3.365282in}{2.988800in}}{\pgfqpoint{3.362010in}{2.980900in}}{\pgfqpoint{3.362010in}{2.972664in}}%
\pgfpathcurveto{\pgfqpoint{3.362010in}{2.964427in}}{\pgfqpoint{3.365282in}{2.956527in}}{\pgfqpoint{3.371106in}{2.950704in}}%
\pgfpathcurveto{\pgfqpoint{3.376930in}{2.944880in}}{\pgfqpoint{3.384830in}{2.941607in}}{\pgfqpoint{3.393066in}{2.941607in}}%
\pgfpathclose%
\pgfusepath{stroke,fill}%
\end{pgfscope}%
\begin{pgfscope}%
\pgfpathrectangle{\pgfqpoint{0.100000in}{0.220728in}}{\pgfqpoint{3.696000in}{3.696000in}}%
\pgfusepath{clip}%
\pgfsetbuttcap%
\pgfsetroundjoin%
\definecolor{currentfill}{rgb}{0.121569,0.466667,0.705882}%
\pgfsetfillcolor{currentfill}%
\pgfsetfillopacity{0.667142}%
\pgfsetlinewidth{1.003750pt}%
\definecolor{currentstroke}{rgb}{0.121569,0.466667,0.705882}%
\pgfsetstrokecolor{currentstroke}%
\pgfsetstrokeopacity{0.667142}%
\pgfsetdash{}{0pt}%
\pgfpathmoveto{\pgfqpoint{0.669318in}{1.436598in}}%
\pgfpathcurveto{\pgfqpoint{0.677554in}{1.436598in}}{\pgfqpoint{0.685454in}{1.439871in}}{\pgfqpoint{0.691278in}{1.445694in}}%
\pgfpathcurveto{\pgfqpoint{0.697102in}{1.451518in}}{\pgfqpoint{0.700375in}{1.459418in}}{\pgfqpoint{0.700375in}{1.467655in}}%
\pgfpathcurveto{\pgfqpoint{0.700375in}{1.475891in}}{\pgfqpoint{0.697102in}{1.483791in}}{\pgfqpoint{0.691278in}{1.489615in}}%
\pgfpathcurveto{\pgfqpoint{0.685454in}{1.495439in}}{\pgfqpoint{0.677554in}{1.498711in}}{\pgfqpoint{0.669318in}{1.498711in}}%
\pgfpathcurveto{\pgfqpoint{0.661082in}{1.498711in}}{\pgfqpoint{0.653182in}{1.495439in}}{\pgfqpoint{0.647358in}{1.489615in}}%
\pgfpathcurveto{\pgfqpoint{0.641534in}{1.483791in}}{\pgfqpoint{0.638262in}{1.475891in}}{\pgfqpoint{0.638262in}{1.467655in}}%
\pgfpathcurveto{\pgfqpoint{0.638262in}{1.459418in}}{\pgfqpoint{0.641534in}{1.451518in}}{\pgfqpoint{0.647358in}{1.445694in}}%
\pgfpathcurveto{\pgfqpoint{0.653182in}{1.439871in}}{\pgfqpoint{0.661082in}{1.436598in}}{\pgfqpoint{0.669318in}{1.436598in}}%
\pgfpathclose%
\pgfusepath{stroke,fill}%
\end{pgfscope}%
\begin{pgfscope}%
\pgfpathrectangle{\pgfqpoint{0.100000in}{0.220728in}}{\pgfqpoint{3.696000in}{3.696000in}}%
\pgfusepath{clip}%
\pgfsetbuttcap%
\pgfsetroundjoin%
\definecolor{currentfill}{rgb}{0.121569,0.466667,0.705882}%
\pgfsetfillcolor{currentfill}%
\pgfsetfillopacity{0.667564}%
\pgfsetlinewidth{1.003750pt}%
\definecolor{currentstroke}{rgb}{0.121569,0.466667,0.705882}%
\pgfsetstrokecolor{currentstroke}%
\pgfsetstrokeopacity{0.667564}%
\pgfsetdash{}{0pt}%
\pgfpathmoveto{\pgfqpoint{3.395350in}{2.941257in}}%
\pgfpathcurveto{\pgfqpoint{3.403586in}{2.941257in}}{\pgfqpoint{3.411486in}{2.944530in}}{\pgfqpoint{3.417310in}{2.950353in}}%
\pgfpathcurveto{\pgfqpoint{3.423134in}{2.956177in}}{\pgfqpoint{3.426407in}{2.964077in}}{\pgfqpoint{3.426407in}{2.972314in}}%
\pgfpathcurveto{\pgfqpoint{3.426407in}{2.980550in}}{\pgfqpoint{3.423134in}{2.988450in}}{\pgfqpoint{3.417310in}{2.994274in}}%
\pgfpathcurveto{\pgfqpoint{3.411486in}{3.000098in}}{\pgfqpoint{3.403586in}{3.003370in}}{\pgfqpoint{3.395350in}{3.003370in}}%
\pgfpathcurveto{\pgfqpoint{3.387114in}{3.003370in}}{\pgfqpoint{3.379214in}{3.000098in}}{\pgfqpoint{3.373390in}{2.994274in}}%
\pgfpathcurveto{\pgfqpoint{3.367566in}{2.988450in}}{\pgfqpoint{3.364294in}{2.980550in}}{\pgfqpoint{3.364294in}{2.972314in}}%
\pgfpathcurveto{\pgfqpoint{3.364294in}{2.964077in}}{\pgfqpoint{3.367566in}{2.956177in}}{\pgfqpoint{3.373390in}{2.950353in}}%
\pgfpathcurveto{\pgfqpoint{3.379214in}{2.944530in}}{\pgfqpoint{3.387114in}{2.941257in}}{\pgfqpoint{3.395350in}{2.941257in}}%
\pgfpathclose%
\pgfusepath{stroke,fill}%
\end{pgfscope}%
\begin{pgfscope}%
\pgfpathrectangle{\pgfqpoint{0.100000in}{0.220728in}}{\pgfqpoint{3.696000in}{3.696000in}}%
\pgfusepath{clip}%
\pgfsetbuttcap%
\pgfsetroundjoin%
\definecolor{currentfill}{rgb}{0.121569,0.466667,0.705882}%
\pgfsetfillcolor{currentfill}%
\pgfsetfillopacity{0.667922}%
\pgfsetlinewidth{1.003750pt}%
\definecolor{currentstroke}{rgb}{0.121569,0.466667,0.705882}%
\pgfsetstrokecolor{currentstroke}%
\pgfsetstrokeopacity{0.667922}%
\pgfsetdash{}{0pt}%
\pgfpathmoveto{\pgfqpoint{3.396547in}{2.941043in}}%
\pgfpathcurveto{\pgfqpoint{3.404783in}{2.941043in}}{\pgfqpoint{3.412683in}{2.944316in}}{\pgfqpoint{3.418507in}{2.950140in}}%
\pgfpathcurveto{\pgfqpoint{3.424331in}{2.955964in}}{\pgfqpoint{3.427603in}{2.963864in}}{\pgfqpoint{3.427603in}{2.972100in}}%
\pgfpathcurveto{\pgfqpoint{3.427603in}{2.980336in}}{\pgfqpoint{3.424331in}{2.988236in}}{\pgfqpoint{3.418507in}{2.994060in}}%
\pgfpathcurveto{\pgfqpoint{3.412683in}{2.999884in}}{\pgfqpoint{3.404783in}{3.003156in}}{\pgfqpoint{3.396547in}{3.003156in}}%
\pgfpathcurveto{\pgfqpoint{3.388311in}{3.003156in}}{\pgfqpoint{3.380410in}{2.999884in}}{\pgfqpoint{3.374587in}{2.994060in}}%
\pgfpathcurveto{\pgfqpoint{3.368763in}{2.988236in}}{\pgfqpoint{3.365490in}{2.980336in}}{\pgfqpoint{3.365490in}{2.972100in}}%
\pgfpathcurveto{\pgfqpoint{3.365490in}{2.963864in}}{\pgfqpoint{3.368763in}{2.955964in}}{\pgfqpoint{3.374587in}{2.950140in}}%
\pgfpathcurveto{\pgfqpoint{3.380410in}{2.944316in}}{\pgfqpoint{3.388311in}{2.941043in}}{\pgfqpoint{3.396547in}{2.941043in}}%
\pgfpathclose%
\pgfusepath{stroke,fill}%
\end{pgfscope}%
\begin{pgfscope}%
\pgfpathrectangle{\pgfqpoint{0.100000in}{0.220728in}}{\pgfqpoint{3.696000in}{3.696000in}}%
\pgfusepath{clip}%
\pgfsetbuttcap%
\pgfsetroundjoin%
\definecolor{currentfill}{rgb}{0.121569,0.466667,0.705882}%
\pgfsetfillcolor{currentfill}%
\pgfsetfillopacity{0.668276}%
\pgfsetlinewidth{1.003750pt}%
\definecolor{currentstroke}{rgb}{0.121569,0.466667,0.705882}%
\pgfsetstrokecolor{currentstroke}%
\pgfsetstrokeopacity{0.668276}%
\pgfsetdash{}{0pt}%
\pgfpathmoveto{\pgfqpoint{3.397097in}{2.941285in}}%
\pgfpathcurveto{\pgfqpoint{3.405333in}{2.941285in}}{\pgfqpoint{3.413233in}{2.944557in}}{\pgfqpoint{3.419057in}{2.950381in}}%
\pgfpathcurveto{\pgfqpoint{3.424881in}{2.956205in}}{\pgfqpoint{3.428153in}{2.964105in}}{\pgfqpoint{3.428153in}{2.972341in}}%
\pgfpathcurveto{\pgfqpoint{3.428153in}{2.980577in}}{\pgfqpoint{3.424881in}{2.988478in}}{\pgfqpoint{3.419057in}{2.994301in}}%
\pgfpathcurveto{\pgfqpoint{3.413233in}{3.000125in}}{\pgfqpoint{3.405333in}{3.003398in}}{\pgfqpoint{3.397097in}{3.003398in}}%
\pgfpathcurveto{\pgfqpoint{3.388860in}{3.003398in}}{\pgfqpoint{3.380960in}{3.000125in}}{\pgfqpoint{3.375136in}{2.994301in}}%
\pgfpathcurveto{\pgfqpoint{3.369312in}{2.988478in}}{\pgfqpoint{3.366040in}{2.980577in}}{\pgfqpoint{3.366040in}{2.972341in}}%
\pgfpathcurveto{\pgfqpoint{3.366040in}{2.964105in}}{\pgfqpoint{3.369312in}{2.956205in}}{\pgfqpoint{3.375136in}{2.950381in}}%
\pgfpathcurveto{\pgfqpoint{3.380960in}{2.944557in}}{\pgfqpoint{3.388860in}{2.941285in}}{\pgfqpoint{3.397097in}{2.941285in}}%
\pgfpathclose%
\pgfusepath{stroke,fill}%
\end{pgfscope}%
\begin{pgfscope}%
\pgfpathrectangle{\pgfqpoint{0.100000in}{0.220728in}}{\pgfqpoint{3.696000in}{3.696000in}}%
\pgfusepath{clip}%
\pgfsetbuttcap%
\pgfsetroundjoin%
\definecolor{currentfill}{rgb}{0.121569,0.466667,0.705882}%
\pgfsetfillcolor{currentfill}%
\pgfsetfillopacity{0.668667}%
\pgfsetlinewidth{1.003750pt}%
\definecolor{currentstroke}{rgb}{0.121569,0.466667,0.705882}%
\pgfsetstrokecolor{currentstroke}%
\pgfsetstrokeopacity{0.668667}%
\pgfsetdash{}{0pt}%
\pgfpathmoveto{\pgfqpoint{3.398464in}{2.940560in}}%
\pgfpathcurveto{\pgfqpoint{3.406700in}{2.940560in}}{\pgfqpoint{3.414600in}{2.943832in}}{\pgfqpoint{3.420424in}{2.949656in}}%
\pgfpathcurveto{\pgfqpoint{3.426248in}{2.955480in}}{\pgfqpoint{3.429520in}{2.963380in}}{\pgfqpoint{3.429520in}{2.971616in}}%
\pgfpathcurveto{\pgfqpoint{3.429520in}{2.979852in}}{\pgfqpoint{3.426248in}{2.987752in}}{\pgfqpoint{3.420424in}{2.993576in}}%
\pgfpathcurveto{\pgfqpoint{3.414600in}{2.999400in}}{\pgfqpoint{3.406700in}{3.002673in}}{\pgfqpoint{3.398464in}{3.002673in}}%
\pgfpathcurveto{\pgfqpoint{3.390227in}{3.002673in}}{\pgfqpoint{3.382327in}{2.999400in}}{\pgfqpoint{3.376503in}{2.993576in}}%
\pgfpathcurveto{\pgfqpoint{3.370679in}{2.987752in}}{\pgfqpoint{3.367407in}{2.979852in}}{\pgfqpoint{3.367407in}{2.971616in}}%
\pgfpathcurveto{\pgfqpoint{3.367407in}{2.963380in}}{\pgfqpoint{3.370679in}{2.955480in}}{\pgfqpoint{3.376503in}{2.949656in}}%
\pgfpathcurveto{\pgfqpoint{3.382327in}{2.943832in}}{\pgfqpoint{3.390227in}{2.940560in}}{\pgfqpoint{3.398464in}{2.940560in}}%
\pgfpathclose%
\pgfusepath{stroke,fill}%
\end{pgfscope}%
\begin{pgfscope}%
\pgfpathrectangle{\pgfqpoint{0.100000in}{0.220728in}}{\pgfqpoint{3.696000in}{3.696000in}}%
\pgfusepath{clip}%
\pgfsetbuttcap%
\pgfsetroundjoin%
\definecolor{currentfill}{rgb}{0.121569,0.466667,0.705882}%
\pgfsetfillcolor{currentfill}%
\pgfsetfillopacity{0.669075}%
\pgfsetlinewidth{1.003750pt}%
\definecolor{currentstroke}{rgb}{0.121569,0.466667,0.705882}%
\pgfsetstrokecolor{currentstroke}%
\pgfsetstrokeopacity{0.669075}%
\pgfsetdash{}{0pt}%
\pgfpathmoveto{\pgfqpoint{0.680479in}{1.431696in}}%
\pgfpathcurveto{\pgfqpoint{0.688716in}{1.431696in}}{\pgfqpoint{0.696616in}{1.434968in}}{\pgfqpoint{0.702440in}{1.440792in}}%
\pgfpathcurveto{\pgfqpoint{0.708263in}{1.446616in}}{\pgfqpoint{0.711536in}{1.454516in}}{\pgfqpoint{0.711536in}{1.462752in}}%
\pgfpathcurveto{\pgfqpoint{0.711536in}{1.470988in}}{\pgfqpoint{0.708263in}{1.478888in}}{\pgfqpoint{0.702440in}{1.484712in}}%
\pgfpathcurveto{\pgfqpoint{0.696616in}{1.490536in}}{\pgfqpoint{0.688716in}{1.493809in}}{\pgfqpoint{0.680479in}{1.493809in}}%
\pgfpathcurveto{\pgfqpoint{0.672243in}{1.493809in}}{\pgfqpoint{0.664343in}{1.490536in}}{\pgfqpoint{0.658519in}{1.484712in}}%
\pgfpathcurveto{\pgfqpoint{0.652695in}{1.478888in}}{\pgfqpoint{0.649423in}{1.470988in}}{\pgfqpoint{0.649423in}{1.462752in}}%
\pgfpathcurveto{\pgfqpoint{0.649423in}{1.454516in}}{\pgfqpoint{0.652695in}{1.446616in}}{\pgfqpoint{0.658519in}{1.440792in}}%
\pgfpathcurveto{\pgfqpoint{0.664343in}{1.434968in}}{\pgfqpoint{0.672243in}{1.431696in}}{\pgfqpoint{0.680479in}{1.431696in}}%
\pgfpathclose%
\pgfusepath{stroke,fill}%
\end{pgfscope}%
\begin{pgfscope}%
\pgfpathrectangle{\pgfqpoint{0.100000in}{0.220728in}}{\pgfqpoint{3.696000in}{3.696000in}}%
\pgfusepath{clip}%
\pgfsetbuttcap%
\pgfsetroundjoin%
\definecolor{currentfill}{rgb}{0.121569,0.466667,0.705882}%
\pgfsetfillcolor{currentfill}%
\pgfsetfillopacity{0.669351}%
\pgfsetlinewidth{1.003750pt}%
\definecolor{currentstroke}{rgb}{0.121569,0.466667,0.705882}%
\pgfsetstrokecolor{currentstroke}%
\pgfsetstrokeopacity{0.669351}%
\pgfsetdash{}{0pt}%
\pgfpathmoveto{\pgfqpoint{3.400310in}{2.939329in}}%
\pgfpathcurveto{\pgfqpoint{3.408546in}{2.939329in}}{\pgfqpoint{3.416446in}{2.942602in}}{\pgfqpoint{3.422270in}{2.948426in}}%
\pgfpathcurveto{\pgfqpoint{3.428094in}{2.954250in}}{\pgfqpoint{3.431366in}{2.962150in}}{\pgfqpoint{3.431366in}{2.970386in}}%
\pgfpathcurveto{\pgfqpoint{3.431366in}{2.978622in}}{\pgfqpoint{3.428094in}{2.986522in}}{\pgfqpoint{3.422270in}{2.992346in}}%
\pgfpathcurveto{\pgfqpoint{3.416446in}{2.998170in}}{\pgfqpoint{3.408546in}{3.001442in}}{\pgfqpoint{3.400310in}{3.001442in}}%
\pgfpathcurveto{\pgfqpoint{3.392074in}{3.001442in}}{\pgfqpoint{3.384174in}{2.998170in}}{\pgfqpoint{3.378350in}{2.992346in}}%
\pgfpathcurveto{\pgfqpoint{3.372526in}{2.986522in}}{\pgfqpoint{3.369253in}{2.978622in}}{\pgfqpoint{3.369253in}{2.970386in}}%
\pgfpathcurveto{\pgfqpoint{3.369253in}{2.962150in}}{\pgfqpoint{3.372526in}{2.954250in}}{\pgfqpoint{3.378350in}{2.948426in}}%
\pgfpathcurveto{\pgfqpoint{3.384174in}{2.942602in}}{\pgfqpoint{3.392074in}{2.939329in}}{\pgfqpoint{3.400310in}{2.939329in}}%
\pgfpathclose%
\pgfusepath{stroke,fill}%
\end{pgfscope}%
\begin{pgfscope}%
\pgfpathrectangle{\pgfqpoint{0.100000in}{0.220728in}}{\pgfqpoint{3.696000in}{3.696000in}}%
\pgfusepath{clip}%
\pgfsetbuttcap%
\pgfsetroundjoin%
\definecolor{currentfill}{rgb}{0.121569,0.466667,0.705882}%
\pgfsetfillcolor{currentfill}%
\pgfsetfillopacity{0.669738}%
\pgfsetlinewidth{1.003750pt}%
\definecolor{currentstroke}{rgb}{0.121569,0.466667,0.705882}%
\pgfsetstrokecolor{currentstroke}%
\pgfsetstrokeopacity{0.669738}%
\pgfsetdash{}{0pt}%
\pgfpathmoveto{\pgfqpoint{3.401171in}{2.938437in}}%
\pgfpathcurveto{\pgfqpoint{3.409407in}{2.938437in}}{\pgfqpoint{3.417307in}{2.941709in}}{\pgfqpoint{3.423131in}{2.947533in}}%
\pgfpathcurveto{\pgfqpoint{3.428955in}{2.953357in}}{\pgfqpoint{3.432227in}{2.961257in}}{\pgfqpoint{3.432227in}{2.969493in}}%
\pgfpathcurveto{\pgfqpoint{3.432227in}{2.977730in}}{\pgfqpoint{3.428955in}{2.985630in}}{\pgfqpoint{3.423131in}{2.991453in}}%
\pgfpathcurveto{\pgfqpoint{3.417307in}{2.997277in}}{\pgfqpoint{3.409407in}{3.000550in}}{\pgfqpoint{3.401171in}{3.000550in}}%
\pgfpathcurveto{\pgfqpoint{3.392935in}{3.000550in}}{\pgfqpoint{3.385035in}{2.997277in}}{\pgfqpoint{3.379211in}{2.991453in}}%
\pgfpathcurveto{\pgfqpoint{3.373387in}{2.985630in}}{\pgfqpoint{3.370114in}{2.977730in}}{\pgfqpoint{3.370114in}{2.969493in}}%
\pgfpathcurveto{\pgfqpoint{3.370114in}{2.961257in}}{\pgfqpoint{3.373387in}{2.953357in}}{\pgfqpoint{3.379211in}{2.947533in}}%
\pgfpathcurveto{\pgfqpoint{3.385035in}{2.941709in}}{\pgfqpoint{3.392935in}{2.938437in}}{\pgfqpoint{3.401171in}{2.938437in}}%
\pgfpathclose%
\pgfusepath{stroke,fill}%
\end{pgfscope}%
\begin{pgfscope}%
\pgfpathrectangle{\pgfqpoint{0.100000in}{0.220728in}}{\pgfqpoint{3.696000in}{3.696000in}}%
\pgfusepath{clip}%
\pgfsetbuttcap%
\pgfsetroundjoin%
\definecolor{currentfill}{rgb}{0.121569,0.466667,0.705882}%
\pgfsetfillcolor{currentfill}%
\pgfsetfillopacity{0.670025}%
\pgfsetlinewidth{1.003750pt}%
\definecolor{currentstroke}{rgb}{0.121569,0.466667,0.705882}%
\pgfsetstrokecolor{currentstroke}%
\pgfsetstrokeopacity{0.670025}%
\pgfsetdash{}{0pt}%
\pgfpathmoveto{\pgfqpoint{3.401463in}{2.938039in}}%
\pgfpathcurveto{\pgfqpoint{3.409700in}{2.938039in}}{\pgfqpoint{3.417600in}{2.941311in}}{\pgfqpoint{3.423424in}{2.947135in}}%
\pgfpathcurveto{\pgfqpoint{3.429248in}{2.952959in}}{\pgfqpoint{3.432520in}{2.960859in}}{\pgfqpoint{3.432520in}{2.969095in}}%
\pgfpathcurveto{\pgfqpoint{3.432520in}{2.977332in}}{\pgfqpoint{3.429248in}{2.985232in}}{\pgfqpoint{3.423424in}{2.991056in}}%
\pgfpathcurveto{\pgfqpoint{3.417600in}{2.996880in}}{\pgfqpoint{3.409700in}{3.000152in}}{\pgfqpoint{3.401463in}{3.000152in}}%
\pgfpathcurveto{\pgfqpoint{3.393227in}{3.000152in}}{\pgfqpoint{3.385327in}{2.996880in}}{\pgfqpoint{3.379503in}{2.991056in}}%
\pgfpathcurveto{\pgfqpoint{3.373679in}{2.985232in}}{\pgfqpoint{3.370407in}{2.977332in}}{\pgfqpoint{3.370407in}{2.969095in}}%
\pgfpathcurveto{\pgfqpoint{3.370407in}{2.960859in}}{\pgfqpoint{3.373679in}{2.952959in}}{\pgfqpoint{3.379503in}{2.947135in}}%
\pgfpathcurveto{\pgfqpoint{3.385327in}{2.941311in}}{\pgfqpoint{3.393227in}{2.938039in}}{\pgfqpoint{3.401463in}{2.938039in}}%
\pgfpathclose%
\pgfusepath{stroke,fill}%
\end{pgfscope}%
\begin{pgfscope}%
\pgfpathrectangle{\pgfqpoint{0.100000in}{0.220728in}}{\pgfqpoint{3.696000in}{3.696000in}}%
\pgfusepath{clip}%
\pgfsetbuttcap%
\pgfsetroundjoin%
\definecolor{currentfill}{rgb}{0.121569,0.466667,0.705882}%
\pgfsetfillcolor{currentfill}%
\pgfsetfillopacity{0.670113}%
\pgfsetlinewidth{1.003750pt}%
\definecolor{currentstroke}{rgb}{0.121569,0.466667,0.705882}%
\pgfsetstrokecolor{currentstroke}%
\pgfsetstrokeopacity{0.670113}%
\pgfsetdash{}{0pt}%
\pgfpathmoveto{\pgfqpoint{3.401565in}{2.937492in}}%
\pgfpathcurveto{\pgfqpoint{3.409801in}{2.937492in}}{\pgfqpoint{3.417701in}{2.940765in}}{\pgfqpoint{3.423525in}{2.946588in}}%
\pgfpathcurveto{\pgfqpoint{3.429349in}{2.952412in}}{\pgfqpoint{3.432621in}{2.960312in}}{\pgfqpoint{3.432621in}{2.968549in}}%
\pgfpathcurveto{\pgfqpoint{3.432621in}{2.976785in}}{\pgfqpoint{3.429349in}{2.984685in}}{\pgfqpoint{3.423525in}{2.990509in}}%
\pgfpathcurveto{\pgfqpoint{3.417701in}{2.996333in}}{\pgfqpoint{3.409801in}{2.999605in}}{\pgfqpoint{3.401565in}{2.999605in}}%
\pgfpathcurveto{\pgfqpoint{3.393329in}{2.999605in}}{\pgfqpoint{3.385429in}{2.996333in}}{\pgfqpoint{3.379605in}{2.990509in}}%
\pgfpathcurveto{\pgfqpoint{3.373781in}{2.984685in}}{\pgfqpoint{3.370508in}{2.976785in}}{\pgfqpoint{3.370508in}{2.968549in}}%
\pgfpathcurveto{\pgfqpoint{3.370508in}{2.960312in}}{\pgfqpoint{3.373781in}{2.952412in}}{\pgfqpoint{3.379605in}{2.946588in}}%
\pgfpathcurveto{\pgfqpoint{3.385429in}{2.940765in}}{\pgfqpoint{3.393329in}{2.937492in}}{\pgfqpoint{3.401565in}{2.937492in}}%
\pgfpathclose%
\pgfusepath{stroke,fill}%
\end{pgfscope}%
\begin{pgfscope}%
\pgfpathrectangle{\pgfqpoint{0.100000in}{0.220728in}}{\pgfqpoint{3.696000in}{3.696000in}}%
\pgfusepath{clip}%
\pgfsetbuttcap%
\pgfsetroundjoin%
\definecolor{currentfill}{rgb}{0.121569,0.466667,0.705882}%
\pgfsetfillcolor{currentfill}%
\pgfsetfillopacity{0.670406}%
\pgfsetlinewidth{1.003750pt}%
\definecolor{currentstroke}{rgb}{0.121569,0.466667,0.705882}%
\pgfsetstrokecolor{currentstroke}%
\pgfsetstrokeopacity{0.670406}%
\pgfsetdash{}{0pt}%
\pgfpathmoveto{\pgfqpoint{3.401466in}{2.935438in}}%
\pgfpathcurveto{\pgfqpoint{3.409702in}{2.935438in}}{\pgfqpoint{3.417603in}{2.938711in}}{\pgfqpoint{3.423426in}{2.944534in}}%
\pgfpathcurveto{\pgfqpoint{3.429250in}{2.950358in}}{\pgfqpoint{3.432523in}{2.958258in}}{\pgfqpoint{3.432523in}{2.966495in}}%
\pgfpathcurveto{\pgfqpoint{3.432523in}{2.974731in}}{\pgfqpoint{3.429250in}{2.982631in}}{\pgfqpoint{3.423426in}{2.988455in}}%
\pgfpathcurveto{\pgfqpoint{3.417603in}{2.994279in}}{\pgfqpoint{3.409702in}{2.997551in}}{\pgfqpoint{3.401466in}{2.997551in}}%
\pgfpathcurveto{\pgfqpoint{3.393230in}{2.997551in}}{\pgfqpoint{3.385330in}{2.994279in}}{\pgfqpoint{3.379506in}{2.988455in}}%
\pgfpathcurveto{\pgfqpoint{3.373682in}{2.982631in}}{\pgfqpoint{3.370410in}{2.974731in}}{\pgfqpoint{3.370410in}{2.966495in}}%
\pgfpathcurveto{\pgfqpoint{3.370410in}{2.958258in}}{\pgfqpoint{3.373682in}{2.950358in}}{\pgfqpoint{3.379506in}{2.944534in}}%
\pgfpathcurveto{\pgfqpoint{3.385330in}{2.938711in}}{\pgfqpoint{3.393230in}{2.935438in}}{\pgfqpoint{3.401466in}{2.935438in}}%
\pgfpathclose%
\pgfusepath{stroke,fill}%
\end{pgfscope}%
\begin{pgfscope}%
\pgfpathrectangle{\pgfqpoint{0.100000in}{0.220728in}}{\pgfqpoint{3.696000in}{3.696000in}}%
\pgfusepath{clip}%
\pgfsetbuttcap%
\pgfsetroundjoin%
\definecolor{currentfill}{rgb}{0.121569,0.466667,0.705882}%
\pgfsetfillcolor{currentfill}%
\pgfsetfillopacity{0.670869}%
\pgfsetlinewidth{1.003750pt}%
\definecolor{currentstroke}{rgb}{0.121569,0.466667,0.705882}%
\pgfsetstrokecolor{currentstroke}%
\pgfsetstrokeopacity{0.670869}%
\pgfsetdash{}{0pt}%
\pgfpathmoveto{\pgfqpoint{3.400263in}{2.932634in}}%
\pgfpathcurveto{\pgfqpoint{3.408499in}{2.932634in}}{\pgfqpoint{3.416399in}{2.935906in}}{\pgfqpoint{3.422223in}{2.941730in}}%
\pgfpathcurveto{\pgfqpoint{3.428047in}{2.947554in}}{\pgfqpoint{3.431319in}{2.955454in}}{\pgfqpoint{3.431319in}{2.963690in}}%
\pgfpathcurveto{\pgfqpoint{3.431319in}{2.971926in}}{\pgfqpoint{3.428047in}{2.979826in}}{\pgfqpoint{3.422223in}{2.985650in}}%
\pgfpathcurveto{\pgfqpoint{3.416399in}{2.991474in}}{\pgfqpoint{3.408499in}{2.994747in}}{\pgfqpoint{3.400263in}{2.994747in}}%
\pgfpathcurveto{\pgfqpoint{3.392027in}{2.994747in}}{\pgfqpoint{3.384127in}{2.991474in}}{\pgfqpoint{3.378303in}{2.985650in}}%
\pgfpathcurveto{\pgfqpoint{3.372479in}{2.979826in}}{\pgfqpoint{3.369206in}{2.971926in}}{\pgfqpoint{3.369206in}{2.963690in}}%
\pgfpathcurveto{\pgfqpoint{3.369206in}{2.955454in}}{\pgfqpoint{3.372479in}{2.947554in}}{\pgfqpoint{3.378303in}{2.941730in}}%
\pgfpathcurveto{\pgfqpoint{3.384127in}{2.935906in}}{\pgfqpoint{3.392027in}{2.932634in}}{\pgfqpoint{3.400263in}{2.932634in}}%
\pgfpathclose%
\pgfusepath{stroke,fill}%
\end{pgfscope}%
\begin{pgfscope}%
\pgfpathrectangle{\pgfqpoint{0.100000in}{0.220728in}}{\pgfqpoint{3.696000in}{3.696000in}}%
\pgfusepath{clip}%
\pgfsetbuttcap%
\pgfsetroundjoin%
\definecolor{currentfill}{rgb}{0.121569,0.466667,0.705882}%
\pgfsetfillcolor{currentfill}%
\pgfsetfillopacity{0.671231}%
\pgfsetlinewidth{1.003750pt}%
\definecolor{currentstroke}{rgb}{0.121569,0.466667,0.705882}%
\pgfsetstrokecolor{currentstroke}%
\pgfsetstrokeopacity{0.671231}%
\pgfsetdash{}{0pt}%
\pgfpathmoveto{\pgfqpoint{0.690190in}{1.426879in}}%
\pgfpathcurveto{\pgfqpoint{0.698426in}{1.426879in}}{\pgfqpoint{0.706326in}{1.430152in}}{\pgfqpoint{0.712150in}{1.435976in}}%
\pgfpathcurveto{\pgfqpoint{0.717974in}{1.441800in}}{\pgfqpoint{0.721247in}{1.449700in}}{\pgfqpoint{0.721247in}{1.457936in}}%
\pgfpathcurveto{\pgfqpoint{0.721247in}{1.466172in}}{\pgfqpoint{0.717974in}{1.474072in}}{\pgfqpoint{0.712150in}{1.479896in}}%
\pgfpathcurveto{\pgfqpoint{0.706326in}{1.485720in}}{\pgfqpoint{0.698426in}{1.488992in}}{\pgfqpoint{0.690190in}{1.488992in}}%
\pgfpathcurveto{\pgfqpoint{0.681954in}{1.488992in}}{\pgfqpoint{0.674054in}{1.485720in}}{\pgfqpoint{0.668230in}{1.479896in}}%
\pgfpathcurveto{\pgfqpoint{0.662406in}{1.474072in}}{\pgfqpoint{0.659134in}{1.466172in}}{\pgfqpoint{0.659134in}{1.457936in}}%
\pgfpathcurveto{\pgfqpoint{0.659134in}{1.449700in}}{\pgfqpoint{0.662406in}{1.441800in}}{\pgfqpoint{0.668230in}{1.435976in}}%
\pgfpathcurveto{\pgfqpoint{0.674054in}{1.430152in}}{\pgfqpoint{0.681954in}{1.426879in}}{\pgfqpoint{0.690190in}{1.426879in}}%
\pgfpathclose%
\pgfusepath{stroke,fill}%
\end{pgfscope}%
\begin{pgfscope}%
\pgfpathrectangle{\pgfqpoint{0.100000in}{0.220728in}}{\pgfqpoint{3.696000in}{3.696000in}}%
\pgfusepath{clip}%
\pgfsetbuttcap%
\pgfsetroundjoin%
\definecolor{currentfill}{rgb}{0.121569,0.466667,0.705882}%
\pgfsetfillcolor{currentfill}%
\pgfsetfillopacity{0.671478}%
\pgfsetlinewidth{1.003750pt}%
\definecolor{currentstroke}{rgb}{0.121569,0.466667,0.705882}%
\pgfsetstrokecolor{currentstroke}%
\pgfsetstrokeopacity{0.671478}%
\pgfsetdash{}{0pt}%
\pgfpathmoveto{\pgfqpoint{3.397538in}{2.927401in}}%
\pgfpathcurveto{\pgfqpoint{3.405774in}{2.927401in}}{\pgfqpoint{3.413674in}{2.930674in}}{\pgfqpoint{3.419498in}{2.936498in}}%
\pgfpathcurveto{\pgfqpoint{3.425322in}{2.942322in}}{\pgfqpoint{3.428594in}{2.950222in}}{\pgfqpoint{3.428594in}{2.958458in}}%
\pgfpathcurveto{\pgfqpoint{3.428594in}{2.966694in}}{\pgfqpoint{3.425322in}{2.974594in}}{\pgfqpoint{3.419498in}{2.980418in}}%
\pgfpathcurveto{\pgfqpoint{3.413674in}{2.986242in}}{\pgfqpoint{3.405774in}{2.989514in}}{\pgfqpoint{3.397538in}{2.989514in}}%
\pgfpathcurveto{\pgfqpoint{3.389301in}{2.989514in}}{\pgfqpoint{3.381401in}{2.986242in}}{\pgfqpoint{3.375577in}{2.980418in}}%
\pgfpathcurveto{\pgfqpoint{3.369753in}{2.974594in}}{\pgfqpoint{3.366481in}{2.966694in}}{\pgfqpoint{3.366481in}{2.958458in}}%
\pgfpathcurveto{\pgfqpoint{3.366481in}{2.950222in}}{\pgfqpoint{3.369753in}{2.942322in}}{\pgfqpoint{3.375577in}{2.936498in}}%
\pgfpathcurveto{\pgfqpoint{3.381401in}{2.930674in}}{\pgfqpoint{3.389301in}{2.927401in}}{\pgfqpoint{3.397538in}{2.927401in}}%
\pgfpathclose%
\pgfusepath{stroke,fill}%
\end{pgfscope}%
\begin{pgfscope}%
\pgfpathrectangle{\pgfqpoint{0.100000in}{0.220728in}}{\pgfqpoint{3.696000in}{3.696000in}}%
\pgfusepath{clip}%
\pgfsetbuttcap%
\pgfsetroundjoin%
\definecolor{currentfill}{rgb}{0.121569,0.466667,0.705882}%
\pgfsetfillcolor{currentfill}%
\pgfsetfillopacity{0.672272}%
\pgfsetlinewidth{1.003750pt}%
\definecolor{currentstroke}{rgb}{0.121569,0.466667,0.705882}%
\pgfsetstrokecolor{currentstroke}%
\pgfsetstrokeopacity{0.672272}%
\pgfsetdash{}{0pt}%
\pgfpathmoveto{\pgfqpoint{3.394075in}{2.922201in}}%
\pgfpathcurveto{\pgfqpoint{3.402312in}{2.922201in}}{\pgfqpoint{3.410212in}{2.925473in}}{\pgfqpoint{3.416035in}{2.931297in}}%
\pgfpathcurveto{\pgfqpoint{3.421859in}{2.937121in}}{\pgfqpoint{3.425132in}{2.945021in}}{\pgfqpoint{3.425132in}{2.953258in}}%
\pgfpathcurveto{\pgfqpoint{3.425132in}{2.961494in}}{\pgfqpoint{3.421859in}{2.969394in}}{\pgfqpoint{3.416035in}{2.975218in}}%
\pgfpathcurveto{\pgfqpoint{3.410212in}{2.981042in}}{\pgfqpoint{3.402312in}{2.984314in}}{\pgfqpoint{3.394075in}{2.984314in}}%
\pgfpathcurveto{\pgfqpoint{3.385839in}{2.984314in}}{\pgfqpoint{3.377939in}{2.981042in}}{\pgfqpoint{3.372115in}{2.975218in}}%
\pgfpathcurveto{\pgfqpoint{3.366291in}{2.969394in}}{\pgfqpoint{3.363019in}{2.961494in}}{\pgfqpoint{3.363019in}{2.953258in}}%
\pgfpathcurveto{\pgfqpoint{3.363019in}{2.945021in}}{\pgfqpoint{3.366291in}{2.937121in}}{\pgfqpoint{3.372115in}{2.931297in}}%
\pgfpathcurveto{\pgfqpoint{3.377939in}{2.925473in}}{\pgfqpoint{3.385839in}{2.922201in}}{\pgfqpoint{3.394075in}{2.922201in}}%
\pgfpathclose%
\pgfusepath{stroke,fill}%
\end{pgfscope}%
\begin{pgfscope}%
\pgfpathrectangle{\pgfqpoint{0.100000in}{0.220728in}}{\pgfqpoint{3.696000in}{3.696000in}}%
\pgfusepath{clip}%
\pgfsetbuttcap%
\pgfsetroundjoin%
\definecolor{currentfill}{rgb}{0.121569,0.466667,0.705882}%
\pgfsetfillcolor{currentfill}%
\pgfsetfillopacity{0.672734}%
\pgfsetlinewidth{1.003750pt}%
\definecolor{currentstroke}{rgb}{0.121569,0.466667,0.705882}%
\pgfsetstrokecolor{currentstroke}%
\pgfsetstrokeopacity{0.672734}%
\pgfsetdash{}{0pt}%
\pgfpathmoveto{\pgfqpoint{3.392438in}{2.919032in}}%
\pgfpathcurveto{\pgfqpoint{3.400674in}{2.919032in}}{\pgfqpoint{3.408574in}{2.922305in}}{\pgfqpoint{3.414398in}{2.928129in}}%
\pgfpathcurveto{\pgfqpoint{3.420222in}{2.933953in}}{\pgfqpoint{3.423494in}{2.941853in}}{\pgfqpoint{3.423494in}{2.950089in}}%
\pgfpathcurveto{\pgfqpoint{3.423494in}{2.958325in}}{\pgfqpoint{3.420222in}{2.966225in}}{\pgfqpoint{3.414398in}{2.972049in}}%
\pgfpathcurveto{\pgfqpoint{3.408574in}{2.977873in}}{\pgfqpoint{3.400674in}{2.981145in}}{\pgfqpoint{3.392438in}{2.981145in}}%
\pgfpathcurveto{\pgfqpoint{3.384202in}{2.981145in}}{\pgfqpoint{3.376302in}{2.977873in}}{\pgfqpoint{3.370478in}{2.972049in}}%
\pgfpathcurveto{\pgfqpoint{3.364654in}{2.966225in}}{\pgfqpoint{3.361381in}{2.958325in}}{\pgfqpoint{3.361381in}{2.950089in}}%
\pgfpathcurveto{\pgfqpoint{3.361381in}{2.941853in}}{\pgfqpoint{3.364654in}{2.933953in}}{\pgfqpoint{3.370478in}{2.928129in}}%
\pgfpathcurveto{\pgfqpoint{3.376302in}{2.922305in}}{\pgfqpoint{3.384202in}{2.919032in}}{\pgfqpoint{3.392438in}{2.919032in}}%
\pgfpathclose%
\pgfusepath{stroke,fill}%
\end{pgfscope}%
\begin{pgfscope}%
\pgfpathrectangle{\pgfqpoint{0.100000in}{0.220728in}}{\pgfqpoint{3.696000in}{3.696000in}}%
\pgfusepath{clip}%
\pgfsetbuttcap%
\pgfsetroundjoin%
\definecolor{currentfill}{rgb}{0.121569,0.466667,0.705882}%
\pgfsetfillcolor{currentfill}%
\pgfsetfillopacity{0.672977}%
\pgfsetlinewidth{1.003750pt}%
\definecolor{currentstroke}{rgb}{0.121569,0.466667,0.705882}%
\pgfsetstrokecolor{currentstroke}%
\pgfsetstrokeopacity{0.672977}%
\pgfsetdash{}{0pt}%
\pgfpathmoveto{\pgfqpoint{3.391414in}{2.917432in}}%
\pgfpathcurveto{\pgfqpoint{3.399650in}{2.917432in}}{\pgfqpoint{3.407550in}{2.920704in}}{\pgfqpoint{3.413374in}{2.926528in}}%
\pgfpathcurveto{\pgfqpoint{3.419198in}{2.932352in}}{\pgfqpoint{3.422470in}{2.940252in}}{\pgfqpoint{3.422470in}{2.948488in}}%
\pgfpathcurveto{\pgfqpoint{3.422470in}{2.956725in}}{\pgfqpoint{3.419198in}{2.964625in}}{\pgfqpoint{3.413374in}{2.970449in}}%
\pgfpathcurveto{\pgfqpoint{3.407550in}{2.976273in}}{\pgfqpoint{3.399650in}{2.979545in}}{\pgfqpoint{3.391414in}{2.979545in}}%
\pgfpathcurveto{\pgfqpoint{3.383178in}{2.979545in}}{\pgfqpoint{3.375278in}{2.976273in}}{\pgfqpoint{3.369454in}{2.970449in}}%
\pgfpathcurveto{\pgfqpoint{3.363630in}{2.964625in}}{\pgfqpoint{3.360357in}{2.956725in}}{\pgfqpoint{3.360357in}{2.948488in}}%
\pgfpathcurveto{\pgfqpoint{3.360357in}{2.940252in}}{\pgfqpoint{3.363630in}{2.932352in}}{\pgfqpoint{3.369454in}{2.926528in}}%
\pgfpathcurveto{\pgfqpoint{3.375278in}{2.920704in}}{\pgfqpoint{3.383178in}{2.917432in}}{\pgfqpoint{3.391414in}{2.917432in}}%
\pgfpathclose%
\pgfusepath{stroke,fill}%
\end{pgfscope}%
\begin{pgfscope}%
\pgfpathrectangle{\pgfqpoint{0.100000in}{0.220728in}}{\pgfqpoint{3.696000in}{3.696000in}}%
\pgfusepath{clip}%
\pgfsetbuttcap%
\pgfsetroundjoin%
\definecolor{currentfill}{rgb}{0.121569,0.466667,0.705882}%
\pgfsetfillcolor{currentfill}%
\pgfsetfillopacity{0.673117}%
\pgfsetlinewidth{1.003750pt}%
\definecolor{currentstroke}{rgb}{0.121569,0.466667,0.705882}%
\pgfsetstrokecolor{currentstroke}%
\pgfsetstrokeopacity{0.673117}%
\pgfsetdash{}{0pt}%
\pgfpathmoveto{\pgfqpoint{3.390887in}{2.916521in}}%
\pgfpathcurveto{\pgfqpoint{3.399123in}{2.916521in}}{\pgfqpoint{3.407023in}{2.919793in}}{\pgfqpoint{3.412847in}{2.925617in}}%
\pgfpathcurveto{\pgfqpoint{3.418671in}{2.931441in}}{\pgfqpoint{3.421943in}{2.939341in}}{\pgfqpoint{3.421943in}{2.947578in}}%
\pgfpathcurveto{\pgfqpoint{3.421943in}{2.955814in}}{\pgfqpoint{3.418671in}{2.963714in}}{\pgfqpoint{3.412847in}{2.969538in}}%
\pgfpathcurveto{\pgfqpoint{3.407023in}{2.975362in}}{\pgfqpoint{3.399123in}{2.978634in}}{\pgfqpoint{3.390887in}{2.978634in}}%
\pgfpathcurveto{\pgfqpoint{3.382650in}{2.978634in}}{\pgfqpoint{3.374750in}{2.975362in}}{\pgfqpoint{3.368926in}{2.969538in}}%
\pgfpathcurveto{\pgfqpoint{3.363102in}{2.963714in}}{\pgfqpoint{3.359830in}{2.955814in}}{\pgfqpoint{3.359830in}{2.947578in}}%
\pgfpathcurveto{\pgfqpoint{3.359830in}{2.939341in}}{\pgfqpoint{3.363102in}{2.931441in}}{\pgfqpoint{3.368926in}{2.925617in}}%
\pgfpathcurveto{\pgfqpoint{3.374750in}{2.919793in}}{\pgfqpoint{3.382650in}{2.916521in}}{\pgfqpoint{3.390887in}{2.916521in}}%
\pgfpathclose%
\pgfusepath{stroke,fill}%
\end{pgfscope}%
\begin{pgfscope}%
\pgfpathrectangle{\pgfqpoint{0.100000in}{0.220728in}}{\pgfqpoint{3.696000in}{3.696000in}}%
\pgfusepath{clip}%
\pgfsetbuttcap%
\pgfsetroundjoin%
\definecolor{currentfill}{rgb}{0.121569,0.466667,0.705882}%
\pgfsetfillcolor{currentfill}%
\pgfsetfillopacity{0.673209}%
\pgfsetlinewidth{1.003750pt}%
\definecolor{currentstroke}{rgb}{0.121569,0.466667,0.705882}%
\pgfsetstrokecolor{currentstroke}%
\pgfsetstrokeopacity{0.673209}%
\pgfsetdash{}{0pt}%
\pgfpathmoveto{\pgfqpoint{3.390607in}{2.916071in}}%
\pgfpathcurveto{\pgfqpoint{3.398844in}{2.916071in}}{\pgfqpoint{3.406744in}{2.919343in}}{\pgfqpoint{3.412568in}{2.925167in}}%
\pgfpathcurveto{\pgfqpoint{3.418392in}{2.930991in}}{\pgfqpoint{3.421664in}{2.938891in}}{\pgfqpoint{3.421664in}{2.947127in}}%
\pgfpathcurveto{\pgfqpoint{3.421664in}{2.955363in}}{\pgfqpoint{3.418392in}{2.963263in}}{\pgfqpoint{3.412568in}{2.969087in}}%
\pgfpathcurveto{\pgfqpoint{3.406744in}{2.974911in}}{\pgfqpoint{3.398844in}{2.978184in}}{\pgfqpoint{3.390607in}{2.978184in}}%
\pgfpathcurveto{\pgfqpoint{3.382371in}{2.978184in}}{\pgfqpoint{3.374471in}{2.974911in}}{\pgfqpoint{3.368647in}{2.969087in}}%
\pgfpathcurveto{\pgfqpoint{3.362823in}{2.963263in}}{\pgfqpoint{3.359551in}{2.955363in}}{\pgfqpoint{3.359551in}{2.947127in}}%
\pgfpathcurveto{\pgfqpoint{3.359551in}{2.938891in}}{\pgfqpoint{3.362823in}{2.930991in}}{\pgfqpoint{3.368647in}{2.925167in}}%
\pgfpathcurveto{\pgfqpoint{3.374471in}{2.919343in}}{\pgfqpoint{3.382371in}{2.916071in}}{\pgfqpoint{3.390607in}{2.916071in}}%
\pgfpathclose%
\pgfusepath{stroke,fill}%
\end{pgfscope}%
\begin{pgfscope}%
\pgfpathrectangle{\pgfqpoint{0.100000in}{0.220728in}}{\pgfqpoint{3.696000in}{3.696000in}}%
\pgfusepath{clip}%
\pgfsetbuttcap%
\pgfsetroundjoin%
\definecolor{currentfill}{rgb}{0.121569,0.466667,0.705882}%
\pgfsetfillcolor{currentfill}%
\pgfsetfillopacity{0.673258}%
\pgfsetlinewidth{1.003750pt}%
\definecolor{currentstroke}{rgb}{0.121569,0.466667,0.705882}%
\pgfsetstrokecolor{currentstroke}%
\pgfsetstrokeopacity{0.673258}%
\pgfsetdash{}{0pt}%
\pgfpathmoveto{\pgfqpoint{0.698559in}{1.423949in}}%
\pgfpathcurveto{\pgfqpoint{0.706795in}{1.423949in}}{\pgfqpoint{0.714695in}{1.427222in}}{\pgfqpoint{0.720519in}{1.433046in}}%
\pgfpathcurveto{\pgfqpoint{0.726343in}{1.438870in}}{\pgfqpoint{0.729616in}{1.446770in}}{\pgfqpoint{0.729616in}{1.455006in}}%
\pgfpathcurveto{\pgfqpoint{0.729616in}{1.463242in}}{\pgfqpoint{0.726343in}{1.471142in}}{\pgfqpoint{0.720519in}{1.476966in}}%
\pgfpathcurveto{\pgfqpoint{0.714695in}{1.482790in}}{\pgfqpoint{0.706795in}{1.486062in}}{\pgfqpoint{0.698559in}{1.486062in}}%
\pgfpathcurveto{\pgfqpoint{0.690323in}{1.486062in}}{\pgfqpoint{0.682423in}{1.482790in}}{\pgfqpoint{0.676599in}{1.476966in}}%
\pgfpathcurveto{\pgfqpoint{0.670775in}{1.471142in}}{\pgfqpoint{0.667503in}{1.463242in}}{\pgfqpoint{0.667503in}{1.455006in}}%
\pgfpathcurveto{\pgfqpoint{0.667503in}{1.446770in}}{\pgfqpoint{0.670775in}{1.438870in}}{\pgfqpoint{0.676599in}{1.433046in}}%
\pgfpathcurveto{\pgfqpoint{0.682423in}{1.427222in}}{\pgfqpoint{0.690323in}{1.423949in}}{\pgfqpoint{0.698559in}{1.423949in}}%
\pgfpathclose%
\pgfusepath{stroke,fill}%
\end{pgfscope}%
\begin{pgfscope}%
\pgfpathrectangle{\pgfqpoint{0.100000in}{0.220728in}}{\pgfqpoint{3.696000in}{3.696000in}}%
\pgfusepath{clip}%
\pgfsetbuttcap%
\pgfsetroundjoin%
\definecolor{currentfill}{rgb}{0.121569,0.466667,0.705882}%
\pgfsetfillcolor{currentfill}%
\pgfsetfillopacity{0.673259}%
\pgfsetlinewidth{1.003750pt}%
\definecolor{currentstroke}{rgb}{0.121569,0.466667,0.705882}%
\pgfsetstrokecolor{currentstroke}%
\pgfsetstrokeopacity{0.673259}%
\pgfsetdash{}{0pt}%
\pgfpathmoveto{\pgfqpoint{3.390453in}{2.915821in}}%
\pgfpathcurveto{\pgfqpoint{3.398690in}{2.915821in}}{\pgfqpoint{3.406590in}{2.919093in}}{\pgfqpoint{3.412414in}{2.924917in}}%
\pgfpathcurveto{\pgfqpoint{3.418238in}{2.930741in}}{\pgfqpoint{3.421510in}{2.938641in}}{\pgfqpoint{3.421510in}{2.946878in}}%
\pgfpathcurveto{\pgfqpoint{3.421510in}{2.955114in}}{\pgfqpoint{3.418238in}{2.963014in}}{\pgfqpoint{3.412414in}{2.968838in}}%
\pgfpathcurveto{\pgfqpoint{3.406590in}{2.974662in}}{\pgfqpoint{3.398690in}{2.977934in}}{\pgfqpoint{3.390453in}{2.977934in}}%
\pgfpathcurveto{\pgfqpoint{3.382217in}{2.977934in}}{\pgfqpoint{3.374317in}{2.974662in}}{\pgfqpoint{3.368493in}{2.968838in}}%
\pgfpathcurveto{\pgfqpoint{3.362669in}{2.963014in}}{\pgfqpoint{3.359397in}{2.955114in}}{\pgfqpoint{3.359397in}{2.946878in}}%
\pgfpathcurveto{\pgfqpoint{3.359397in}{2.938641in}}{\pgfqpoint{3.362669in}{2.930741in}}{\pgfqpoint{3.368493in}{2.924917in}}%
\pgfpathcurveto{\pgfqpoint{3.374317in}{2.919093in}}{\pgfqpoint{3.382217in}{2.915821in}}{\pgfqpoint{3.390453in}{2.915821in}}%
\pgfpathclose%
\pgfusepath{stroke,fill}%
\end{pgfscope}%
\begin{pgfscope}%
\pgfpathrectangle{\pgfqpoint{0.100000in}{0.220728in}}{\pgfqpoint{3.696000in}{3.696000in}}%
\pgfusepath{clip}%
\pgfsetbuttcap%
\pgfsetroundjoin%
\definecolor{currentfill}{rgb}{0.121569,0.466667,0.705882}%
\pgfsetfillcolor{currentfill}%
\pgfsetfillopacity{0.673281}%
\pgfsetlinewidth{1.003750pt}%
\definecolor{currentstroke}{rgb}{0.121569,0.466667,0.705882}%
\pgfsetstrokecolor{currentstroke}%
\pgfsetstrokeopacity{0.673281}%
\pgfsetdash{}{0pt}%
\pgfpathmoveto{\pgfqpoint{3.390351in}{2.915686in}}%
\pgfpathcurveto{\pgfqpoint{3.398587in}{2.915686in}}{\pgfqpoint{3.406487in}{2.918958in}}{\pgfqpoint{3.412311in}{2.924782in}}%
\pgfpathcurveto{\pgfqpoint{3.418135in}{2.930606in}}{\pgfqpoint{3.421407in}{2.938506in}}{\pgfqpoint{3.421407in}{2.946742in}}%
\pgfpathcurveto{\pgfqpoint{3.421407in}{2.954979in}}{\pgfqpoint{3.418135in}{2.962879in}}{\pgfqpoint{3.412311in}{2.968703in}}%
\pgfpathcurveto{\pgfqpoint{3.406487in}{2.974527in}}{\pgfqpoint{3.398587in}{2.977799in}}{\pgfqpoint{3.390351in}{2.977799in}}%
\pgfpathcurveto{\pgfqpoint{3.382115in}{2.977799in}}{\pgfqpoint{3.374215in}{2.974527in}}{\pgfqpoint{3.368391in}{2.968703in}}%
\pgfpathcurveto{\pgfqpoint{3.362567in}{2.962879in}}{\pgfqpoint{3.359294in}{2.954979in}}{\pgfqpoint{3.359294in}{2.946742in}}%
\pgfpathcurveto{\pgfqpoint{3.359294in}{2.938506in}}{\pgfqpoint{3.362567in}{2.930606in}}{\pgfqpoint{3.368391in}{2.924782in}}%
\pgfpathcurveto{\pgfqpoint{3.374215in}{2.918958in}}{\pgfqpoint{3.382115in}{2.915686in}}{\pgfqpoint{3.390351in}{2.915686in}}%
\pgfpathclose%
\pgfusepath{stroke,fill}%
\end{pgfscope}%
\begin{pgfscope}%
\pgfpathrectangle{\pgfqpoint{0.100000in}{0.220728in}}{\pgfqpoint{3.696000in}{3.696000in}}%
\pgfusepath{clip}%
\pgfsetbuttcap%
\pgfsetroundjoin%
\definecolor{currentfill}{rgb}{0.121569,0.466667,0.705882}%
\pgfsetfillcolor{currentfill}%
\pgfsetfillopacity{0.673297}%
\pgfsetlinewidth{1.003750pt}%
\definecolor{currentstroke}{rgb}{0.121569,0.466667,0.705882}%
\pgfsetstrokecolor{currentstroke}%
\pgfsetstrokeopacity{0.673297}%
\pgfsetdash{}{0pt}%
\pgfpathmoveto{\pgfqpoint{3.390315in}{2.915604in}}%
\pgfpathcurveto{\pgfqpoint{3.398551in}{2.915604in}}{\pgfqpoint{3.406451in}{2.918877in}}{\pgfqpoint{3.412275in}{2.924701in}}%
\pgfpathcurveto{\pgfqpoint{3.418099in}{2.930524in}}{\pgfqpoint{3.421371in}{2.938424in}}{\pgfqpoint{3.421371in}{2.946661in}}%
\pgfpathcurveto{\pgfqpoint{3.421371in}{2.954897in}}{\pgfqpoint{3.418099in}{2.962797in}}{\pgfqpoint{3.412275in}{2.968621in}}%
\pgfpathcurveto{\pgfqpoint{3.406451in}{2.974445in}}{\pgfqpoint{3.398551in}{2.977717in}}{\pgfqpoint{3.390315in}{2.977717in}}%
\pgfpathcurveto{\pgfqpoint{3.382079in}{2.977717in}}{\pgfqpoint{3.374178in}{2.974445in}}{\pgfqpoint{3.368355in}{2.968621in}}%
\pgfpathcurveto{\pgfqpoint{3.362531in}{2.962797in}}{\pgfqpoint{3.359258in}{2.954897in}}{\pgfqpoint{3.359258in}{2.946661in}}%
\pgfpathcurveto{\pgfqpoint{3.359258in}{2.938424in}}{\pgfqpoint{3.362531in}{2.930524in}}{\pgfqpoint{3.368355in}{2.924701in}}%
\pgfpathcurveto{\pgfqpoint{3.374178in}{2.918877in}}{\pgfqpoint{3.382079in}{2.915604in}}{\pgfqpoint{3.390315in}{2.915604in}}%
\pgfpathclose%
\pgfusepath{stroke,fill}%
\end{pgfscope}%
\begin{pgfscope}%
\pgfpathrectangle{\pgfqpoint{0.100000in}{0.220728in}}{\pgfqpoint{3.696000in}{3.696000in}}%
\pgfusepath{clip}%
\pgfsetbuttcap%
\pgfsetroundjoin%
\definecolor{currentfill}{rgb}{0.121569,0.466667,0.705882}%
\pgfsetfillcolor{currentfill}%
\pgfsetfillopacity{0.673401}%
\pgfsetlinewidth{1.003750pt}%
\definecolor{currentstroke}{rgb}{0.121569,0.466667,0.705882}%
\pgfsetstrokecolor{currentstroke}%
\pgfsetstrokeopacity{0.673401}%
\pgfsetdash{}{0pt}%
\pgfpathmoveto{\pgfqpoint{3.389807in}{2.914906in}}%
\pgfpathcurveto{\pgfqpoint{3.398043in}{2.914906in}}{\pgfqpoint{3.405943in}{2.918179in}}{\pgfqpoint{3.411767in}{2.924003in}}%
\pgfpathcurveto{\pgfqpoint{3.417591in}{2.929827in}}{\pgfqpoint{3.420863in}{2.937727in}}{\pgfqpoint{3.420863in}{2.945963in}}%
\pgfpathcurveto{\pgfqpoint{3.420863in}{2.954199in}}{\pgfqpoint{3.417591in}{2.962099in}}{\pgfqpoint{3.411767in}{2.967923in}}%
\pgfpathcurveto{\pgfqpoint{3.405943in}{2.973747in}}{\pgfqpoint{3.398043in}{2.977019in}}{\pgfqpoint{3.389807in}{2.977019in}}%
\pgfpathcurveto{\pgfqpoint{3.381571in}{2.977019in}}{\pgfqpoint{3.373671in}{2.973747in}}{\pgfqpoint{3.367847in}{2.967923in}}%
\pgfpathcurveto{\pgfqpoint{3.362023in}{2.962099in}}{\pgfqpoint{3.358750in}{2.954199in}}{\pgfqpoint{3.358750in}{2.945963in}}%
\pgfpathcurveto{\pgfqpoint{3.358750in}{2.937727in}}{\pgfqpoint{3.362023in}{2.929827in}}{\pgfqpoint{3.367847in}{2.924003in}}%
\pgfpathcurveto{\pgfqpoint{3.373671in}{2.918179in}}{\pgfqpoint{3.381571in}{2.914906in}}{\pgfqpoint{3.389807in}{2.914906in}}%
\pgfpathclose%
\pgfusepath{stroke,fill}%
\end{pgfscope}%
\begin{pgfscope}%
\pgfpathrectangle{\pgfqpoint{0.100000in}{0.220728in}}{\pgfqpoint{3.696000in}{3.696000in}}%
\pgfusepath{clip}%
\pgfsetbuttcap%
\pgfsetroundjoin%
\definecolor{currentfill}{rgb}{0.121569,0.466667,0.705882}%
\pgfsetfillcolor{currentfill}%
\pgfsetfillopacity{0.673914}%
\pgfsetlinewidth{1.003750pt}%
\definecolor{currentstroke}{rgb}{0.121569,0.466667,0.705882}%
\pgfsetstrokecolor{currentstroke}%
\pgfsetstrokeopacity{0.673914}%
\pgfsetdash{}{0pt}%
\pgfpathmoveto{\pgfqpoint{3.388657in}{2.912776in}}%
\pgfpathcurveto{\pgfqpoint{3.396894in}{2.912776in}}{\pgfqpoint{3.404794in}{2.916048in}}{\pgfqpoint{3.410618in}{2.921872in}}%
\pgfpathcurveto{\pgfqpoint{3.416441in}{2.927696in}}{\pgfqpoint{3.419714in}{2.935596in}}{\pgfqpoint{3.419714in}{2.943832in}}%
\pgfpathcurveto{\pgfqpoint{3.419714in}{2.952069in}}{\pgfqpoint{3.416441in}{2.959969in}}{\pgfqpoint{3.410618in}{2.965793in}}%
\pgfpathcurveto{\pgfqpoint{3.404794in}{2.971617in}}{\pgfqpoint{3.396894in}{2.974889in}}{\pgfqpoint{3.388657in}{2.974889in}}%
\pgfpathcurveto{\pgfqpoint{3.380421in}{2.974889in}}{\pgfqpoint{3.372521in}{2.971617in}}{\pgfqpoint{3.366697in}{2.965793in}}%
\pgfpathcurveto{\pgfqpoint{3.360873in}{2.959969in}}{\pgfqpoint{3.357601in}{2.952069in}}{\pgfqpoint{3.357601in}{2.943832in}}%
\pgfpathcurveto{\pgfqpoint{3.357601in}{2.935596in}}{\pgfqpoint{3.360873in}{2.927696in}}{\pgfqpoint{3.366697in}{2.921872in}}%
\pgfpathcurveto{\pgfqpoint{3.372521in}{2.916048in}}{\pgfqpoint{3.380421in}{2.912776in}}{\pgfqpoint{3.388657in}{2.912776in}}%
\pgfpathclose%
\pgfusepath{stroke,fill}%
\end{pgfscope}%
\begin{pgfscope}%
\pgfpathrectangle{\pgfqpoint{0.100000in}{0.220728in}}{\pgfqpoint{3.696000in}{3.696000in}}%
\pgfusepath{clip}%
\pgfsetbuttcap%
\pgfsetroundjoin%
\definecolor{currentfill}{rgb}{0.121569,0.466667,0.705882}%
\pgfsetfillcolor{currentfill}%
\pgfsetfillopacity{0.674380}%
\pgfsetlinewidth{1.003750pt}%
\definecolor{currentstroke}{rgb}{0.121569,0.466667,0.705882}%
\pgfsetstrokecolor{currentstroke}%
\pgfsetstrokeopacity{0.674380}%
\pgfsetdash{}{0pt}%
\pgfpathmoveto{\pgfqpoint{3.386943in}{2.909837in}}%
\pgfpathcurveto{\pgfqpoint{3.395180in}{2.909837in}}{\pgfqpoint{3.403080in}{2.913109in}}{\pgfqpoint{3.408904in}{2.918933in}}%
\pgfpathcurveto{\pgfqpoint{3.414727in}{2.924757in}}{\pgfqpoint{3.418000in}{2.932657in}}{\pgfqpoint{3.418000in}{2.940894in}}%
\pgfpathcurveto{\pgfqpoint{3.418000in}{2.949130in}}{\pgfqpoint{3.414727in}{2.957030in}}{\pgfqpoint{3.408904in}{2.962854in}}%
\pgfpathcurveto{\pgfqpoint{3.403080in}{2.968678in}}{\pgfqpoint{3.395180in}{2.971950in}}{\pgfqpoint{3.386943in}{2.971950in}}%
\pgfpathcurveto{\pgfqpoint{3.378707in}{2.971950in}}{\pgfqpoint{3.370807in}{2.968678in}}{\pgfqpoint{3.364983in}{2.962854in}}%
\pgfpathcurveto{\pgfqpoint{3.359159in}{2.957030in}}{\pgfqpoint{3.355887in}{2.949130in}}{\pgfqpoint{3.355887in}{2.940894in}}%
\pgfpathcurveto{\pgfqpoint{3.355887in}{2.932657in}}{\pgfqpoint{3.359159in}{2.924757in}}{\pgfqpoint{3.364983in}{2.918933in}}%
\pgfpathcurveto{\pgfqpoint{3.370807in}{2.913109in}}{\pgfqpoint{3.378707in}{2.909837in}}{\pgfqpoint{3.386943in}{2.909837in}}%
\pgfpathclose%
\pgfusepath{stroke,fill}%
\end{pgfscope}%
\begin{pgfscope}%
\pgfpathrectangle{\pgfqpoint{0.100000in}{0.220728in}}{\pgfqpoint{3.696000in}{3.696000in}}%
\pgfusepath{clip}%
\pgfsetbuttcap%
\pgfsetroundjoin%
\definecolor{currentfill}{rgb}{0.121569,0.466667,0.705882}%
\pgfsetfillcolor{currentfill}%
\pgfsetfillopacity{0.674587}%
\pgfsetlinewidth{1.003750pt}%
\definecolor{currentstroke}{rgb}{0.121569,0.466667,0.705882}%
\pgfsetstrokecolor{currentstroke}%
\pgfsetstrokeopacity{0.674587}%
\pgfsetdash{}{0pt}%
\pgfpathmoveto{\pgfqpoint{3.385816in}{2.908337in}}%
\pgfpathcurveto{\pgfqpoint{3.394052in}{2.908337in}}{\pgfqpoint{3.401952in}{2.911609in}}{\pgfqpoint{3.407776in}{2.917433in}}%
\pgfpathcurveto{\pgfqpoint{3.413600in}{2.923257in}}{\pgfqpoint{3.416872in}{2.931157in}}{\pgfqpoint{3.416872in}{2.939393in}}%
\pgfpathcurveto{\pgfqpoint{3.416872in}{2.947630in}}{\pgfqpoint{3.413600in}{2.955530in}}{\pgfqpoint{3.407776in}{2.961354in}}%
\pgfpathcurveto{\pgfqpoint{3.401952in}{2.967178in}}{\pgfqpoint{3.394052in}{2.970450in}}{\pgfqpoint{3.385816in}{2.970450in}}%
\pgfpathcurveto{\pgfqpoint{3.377579in}{2.970450in}}{\pgfqpoint{3.369679in}{2.967178in}}{\pgfqpoint{3.363855in}{2.961354in}}%
\pgfpathcurveto{\pgfqpoint{3.358031in}{2.955530in}}{\pgfqpoint{3.354759in}{2.947630in}}{\pgfqpoint{3.354759in}{2.939393in}}%
\pgfpathcurveto{\pgfqpoint{3.354759in}{2.931157in}}{\pgfqpoint{3.358031in}{2.923257in}}{\pgfqpoint{3.363855in}{2.917433in}}%
\pgfpathcurveto{\pgfqpoint{3.369679in}{2.911609in}}{\pgfqpoint{3.377579in}{2.908337in}}{\pgfqpoint{3.385816in}{2.908337in}}%
\pgfpathclose%
\pgfusepath{stroke,fill}%
\end{pgfscope}%
\begin{pgfscope}%
\pgfpathrectangle{\pgfqpoint{0.100000in}{0.220728in}}{\pgfqpoint{3.696000in}{3.696000in}}%
\pgfusepath{clip}%
\pgfsetbuttcap%
\pgfsetroundjoin%
\definecolor{currentfill}{rgb}{0.121569,0.466667,0.705882}%
\pgfsetfillcolor{currentfill}%
\pgfsetfillopacity{0.674767}%
\pgfsetlinewidth{1.003750pt}%
\definecolor{currentstroke}{rgb}{0.121569,0.466667,0.705882}%
\pgfsetstrokecolor{currentstroke}%
\pgfsetstrokeopacity{0.674767}%
\pgfsetdash{}{0pt}%
\pgfpathmoveto{\pgfqpoint{3.385444in}{2.907442in}}%
\pgfpathcurveto{\pgfqpoint{3.393680in}{2.907442in}}{\pgfqpoint{3.401580in}{2.910714in}}{\pgfqpoint{3.407404in}{2.916538in}}%
\pgfpathcurveto{\pgfqpoint{3.413228in}{2.922362in}}{\pgfqpoint{3.416500in}{2.930262in}}{\pgfqpoint{3.416500in}{2.938498in}}%
\pgfpathcurveto{\pgfqpoint{3.416500in}{2.946735in}}{\pgfqpoint{3.413228in}{2.954635in}}{\pgfqpoint{3.407404in}{2.960459in}}%
\pgfpathcurveto{\pgfqpoint{3.401580in}{2.966282in}}{\pgfqpoint{3.393680in}{2.969555in}}{\pgfqpoint{3.385444in}{2.969555in}}%
\pgfpathcurveto{\pgfqpoint{3.377207in}{2.969555in}}{\pgfqpoint{3.369307in}{2.966282in}}{\pgfqpoint{3.363483in}{2.960459in}}%
\pgfpathcurveto{\pgfqpoint{3.357659in}{2.954635in}}{\pgfqpoint{3.354387in}{2.946735in}}{\pgfqpoint{3.354387in}{2.938498in}}%
\pgfpathcurveto{\pgfqpoint{3.354387in}{2.930262in}}{\pgfqpoint{3.357659in}{2.922362in}}{\pgfqpoint{3.363483in}{2.916538in}}%
\pgfpathcurveto{\pgfqpoint{3.369307in}{2.910714in}}{\pgfqpoint{3.377207in}{2.907442in}}{\pgfqpoint{3.385444in}{2.907442in}}%
\pgfpathclose%
\pgfusepath{stroke,fill}%
\end{pgfscope}%
\begin{pgfscope}%
\pgfpathrectangle{\pgfqpoint{0.100000in}{0.220728in}}{\pgfqpoint{3.696000in}{3.696000in}}%
\pgfusepath{clip}%
\pgfsetbuttcap%
\pgfsetroundjoin%
\definecolor{currentfill}{rgb}{0.121569,0.466667,0.705882}%
\pgfsetfillcolor{currentfill}%
\pgfsetfillopacity{0.675103}%
\pgfsetlinewidth{1.003750pt}%
\definecolor{currentstroke}{rgb}{0.121569,0.466667,0.705882}%
\pgfsetstrokecolor{currentstroke}%
\pgfsetstrokeopacity{0.675103}%
\pgfsetdash{}{0pt}%
\pgfpathmoveto{\pgfqpoint{3.383819in}{2.905178in}}%
\pgfpathcurveto{\pgfqpoint{3.392055in}{2.905178in}}{\pgfqpoint{3.399955in}{2.908450in}}{\pgfqpoint{3.405779in}{2.914274in}}%
\pgfpathcurveto{\pgfqpoint{3.411603in}{2.920098in}}{\pgfqpoint{3.414875in}{2.927998in}}{\pgfqpoint{3.414875in}{2.936235in}}%
\pgfpathcurveto{\pgfqpoint{3.414875in}{2.944471in}}{\pgfqpoint{3.411603in}{2.952371in}}{\pgfqpoint{3.405779in}{2.958195in}}%
\pgfpathcurveto{\pgfqpoint{3.399955in}{2.964019in}}{\pgfqpoint{3.392055in}{2.967291in}}{\pgfqpoint{3.383819in}{2.967291in}}%
\pgfpathcurveto{\pgfqpoint{3.375583in}{2.967291in}}{\pgfqpoint{3.367683in}{2.964019in}}{\pgfqpoint{3.361859in}{2.958195in}}%
\pgfpathcurveto{\pgfqpoint{3.356035in}{2.952371in}}{\pgfqpoint{3.352762in}{2.944471in}}{\pgfqpoint{3.352762in}{2.936235in}}%
\pgfpathcurveto{\pgfqpoint{3.352762in}{2.927998in}}{\pgfqpoint{3.356035in}{2.920098in}}{\pgfqpoint{3.361859in}{2.914274in}}%
\pgfpathcurveto{\pgfqpoint{3.367683in}{2.908450in}}{\pgfqpoint{3.375583in}{2.905178in}}{\pgfqpoint{3.383819in}{2.905178in}}%
\pgfpathclose%
\pgfusepath{stroke,fill}%
\end{pgfscope}%
\begin{pgfscope}%
\pgfpathrectangle{\pgfqpoint{0.100000in}{0.220728in}}{\pgfqpoint{3.696000in}{3.696000in}}%
\pgfusepath{clip}%
\pgfsetbuttcap%
\pgfsetroundjoin%
\definecolor{currentfill}{rgb}{0.121569,0.466667,0.705882}%
\pgfsetfillcolor{currentfill}%
\pgfsetfillopacity{0.675776}%
\pgfsetlinewidth{1.003750pt}%
\definecolor{currentstroke}{rgb}{0.121569,0.466667,0.705882}%
\pgfsetstrokecolor{currentstroke}%
\pgfsetstrokeopacity{0.675776}%
\pgfsetdash{}{0pt}%
\pgfpathmoveto{\pgfqpoint{3.382103in}{2.901785in}}%
\pgfpathcurveto{\pgfqpoint{3.390339in}{2.901785in}}{\pgfqpoint{3.398239in}{2.905058in}}{\pgfqpoint{3.404063in}{2.910882in}}%
\pgfpathcurveto{\pgfqpoint{3.409887in}{2.916706in}}{\pgfqpoint{3.413159in}{2.924606in}}{\pgfqpoint{3.413159in}{2.932842in}}%
\pgfpathcurveto{\pgfqpoint{3.413159in}{2.941078in}}{\pgfqpoint{3.409887in}{2.948978in}}{\pgfqpoint{3.404063in}{2.954802in}}%
\pgfpathcurveto{\pgfqpoint{3.398239in}{2.960626in}}{\pgfqpoint{3.390339in}{2.963898in}}{\pgfqpoint{3.382103in}{2.963898in}}%
\pgfpathcurveto{\pgfqpoint{3.373867in}{2.963898in}}{\pgfqpoint{3.365967in}{2.960626in}}{\pgfqpoint{3.360143in}{2.954802in}}%
\pgfpathcurveto{\pgfqpoint{3.354319in}{2.948978in}}{\pgfqpoint{3.351046in}{2.941078in}}{\pgfqpoint{3.351046in}{2.932842in}}%
\pgfpathcurveto{\pgfqpoint{3.351046in}{2.924606in}}{\pgfqpoint{3.354319in}{2.916706in}}{\pgfqpoint{3.360143in}{2.910882in}}%
\pgfpathcurveto{\pgfqpoint{3.365967in}{2.905058in}}{\pgfqpoint{3.373867in}{2.901785in}}{\pgfqpoint{3.382103in}{2.901785in}}%
\pgfpathclose%
\pgfusepath{stroke,fill}%
\end{pgfscope}%
\begin{pgfscope}%
\pgfpathrectangle{\pgfqpoint{0.100000in}{0.220728in}}{\pgfqpoint{3.696000in}{3.696000in}}%
\pgfusepath{clip}%
\pgfsetbuttcap%
\pgfsetroundjoin%
\definecolor{currentfill}{rgb}{0.121569,0.466667,0.705882}%
\pgfsetfillcolor{currentfill}%
\pgfsetfillopacity{0.676303}%
\pgfsetlinewidth{1.003750pt}%
\definecolor{currentstroke}{rgb}{0.121569,0.466667,0.705882}%
\pgfsetstrokecolor{currentstroke}%
\pgfsetstrokeopacity{0.676303}%
\pgfsetdash{}{0pt}%
\pgfpathmoveto{\pgfqpoint{0.714164in}{1.416956in}}%
\pgfpathcurveto{\pgfqpoint{0.722400in}{1.416956in}}{\pgfqpoint{0.730300in}{1.420228in}}{\pgfqpoint{0.736124in}{1.426052in}}%
\pgfpathcurveto{\pgfqpoint{0.741948in}{1.431876in}}{\pgfqpoint{0.745220in}{1.439776in}}{\pgfqpoint{0.745220in}{1.448012in}}%
\pgfpathcurveto{\pgfqpoint{0.745220in}{1.456249in}}{\pgfqpoint{0.741948in}{1.464149in}}{\pgfqpoint{0.736124in}{1.469973in}}%
\pgfpathcurveto{\pgfqpoint{0.730300in}{1.475797in}}{\pgfqpoint{0.722400in}{1.479069in}}{\pgfqpoint{0.714164in}{1.479069in}}%
\pgfpathcurveto{\pgfqpoint{0.705928in}{1.479069in}}{\pgfqpoint{0.698028in}{1.475797in}}{\pgfqpoint{0.692204in}{1.469973in}}%
\pgfpathcurveto{\pgfqpoint{0.686380in}{1.464149in}}{\pgfqpoint{0.683107in}{1.456249in}}{\pgfqpoint{0.683107in}{1.448012in}}%
\pgfpathcurveto{\pgfqpoint{0.683107in}{1.439776in}}{\pgfqpoint{0.686380in}{1.431876in}}{\pgfqpoint{0.692204in}{1.426052in}}%
\pgfpathcurveto{\pgfqpoint{0.698028in}{1.420228in}}{\pgfqpoint{0.705928in}{1.416956in}}{\pgfqpoint{0.714164in}{1.416956in}}%
\pgfpathclose%
\pgfusepath{stroke,fill}%
\end{pgfscope}%
\begin{pgfscope}%
\pgfpathrectangle{\pgfqpoint{0.100000in}{0.220728in}}{\pgfqpoint{3.696000in}{3.696000in}}%
\pgfusepath{clip}%
\pgfsetbuttcap%
\pgfsetroundjoin%
\definecolor{currentfill}{rgb}{0.121569,0.466667,0.705882}%
\pgfsetfillcolor{currentfill}%
\pgfsetfillopacity{0.676680}%
\pgfsetlinewidth{1.003750pt}%
\definecolor{currentstroke}{rgb}{0.121569,0.466667,0.705882}%
\pgfsetstrokecolor{currentstroke}%
\pgfsetstrokeopacity{0.676680}%
\pgfsetdash{}{0pt}%
\pgfpathmoveto{\pgfqpoint{3.380300in}{2.897773in}}%
\pgfpathcurveto{\pgfqpoint{3.388537in}{2.897773in}}{\pgfqpoint{3.396437in}{2.901045in}}{\pgfqpoint{3.402261in}{2.906869in}}%
\pgfpathcurveto{\pgfqpoint{3.408084in}{2.912693in}}{\pgfqpoint{3.411357in}{2.920593in}}{\pgfqpoint{3.411357in}{2.928830in}}%
\pgfpathcurveto{\pgfqpoint{3.411357in}{2.937066in}}{\pgfqpoint{3.408084in}{2.944966in}}{\pgfqpoint{3.402261in}{2.950790in}}%
\pgfpathcurveto{\pgfqpoint{3.396437in}{2.956614in}}{\pgfqpoint{3.388537in}{2.959886in}}{\pgfqpoint{3.380300in}{2.959886in}}%
\pgfpathcurveto{\pgfqpoint{3.372064in}{2.959886in}}{\pgfqpoint{3.364164in}{2.956614in}}{\pgfqpoint{3.358340in}{2.950790in}}%
\pgfpathcurveto{\pgfqpoint{3.352516in}{2.944966in}}{\pgfqpoint{3.349244in}{2.937066in}}{\pgfqpoint{3.349244in}{2.928830in}}%
\pgfpathcurveto{\pgfqpoint{3.349244in}{2.920593in}}{\pgfqpoint{3.352516in}{2.912693in}}{\pgfqpoint{3.358340in}{2.906869in}}%
\pgfpathcurveto{\pgfqpoint{3.364164in}{2.901045in}}{\pgfqpoint{3.372064in}{2.897773in}}{\pgfqpoint{3.380300in}{2.897773in}}%
\pgfpathclose%
\pgfusepath{stroke,fill}%
\end{pgfscope}%
\begin{pgfscope}%
\pgfpathrectangle{\pgfqpoint{0.100000in}{0.220728in}}{\pgfqpoint{3.696000in}{3.696000in}}%
\pgfusepath{clip}%
\pgfsetbuttcap%
\pgfsetroundjoin%
\definecolor{currentfill}{rgb}{0.121569,0.466667,0.705882}%
\pgfsetfillcolor{currentfill}%
\pgfsetfillopacity{0.677533}%
\pgfsetlinewidth{1.003750pt}%
\definecolor{currentstroke}{rgb}{0.121569,0.466667,0.705882}%
\pgfsetstrokecolor{currentstroke}%
\pgfsetstrokeopacity{0.677533}%
\pgfsetdash{}{0pt}%
\pgfpathmoveto{\pgfqpoint{3.377210in}{2.893593in}}%
\pgfpathcurveto{\pgfqpoint{3.385446in}{2.893593in}}{\pgfqpoint{3.393346in}{2.896865in}}{\pgfqpoint{3.399170in}{2.902689in}}%
\pgfpathcurveto{\pgfqpoint{3.404994in}{2.908513in}}{\pgfqpoint{3.408266in}{2.916413in}}{\pgfqpoint{3.408266in}{2.924649in}}%
\pgfpathcurveto{\pgfqpoint{3.408266in}{2.932885in}}{\pgfqpoint{3.404994in}{2.940785in}}{\pgfqpoint{3.399170in}{2.946609in}}%
\pgfpathcurveto{\pgfqpoint{3.393346in}{2.952433in}}{\pgfqpoint{3.385446in}{2.955706in}}{\pgfqpoint{3.377210in}{2.955706in}}%
\pgfpathcurveto{\pgfqpoint{3.368973in}{2.955706in}}{\pgfqpoint{3.361073in}{2.952433in}}{\pgfqpoint{3.355249in}{2.946609in}}%
\pgfpathcurveto{\pgfqpoint{3.349425in}{2.940785in}}{\pgfqpoint{3.346153in}{2.932885in}}{\pgfqpoint{3.346153in}{2.924649in}}%
\pgfpathcurveto{\pgfqpoint{3.346153in}{2.916413in}}{\pgfqpoint{3.349425in}{2.908513in}}{\pgfqpoint{3.355249in}{2.902689in}}%
\pgfpathcurveto{\pgfqpoint{3.361073in}{2.896865in}}{\pgfqpoint{3.368973in}{2.893593in}}{\pgfqpoint{3.377210in}{2.893593in}}%
\pgfpathclose%
\pgfusepath{stroke,fill}%
\end{pgfscope}%
\begin{pgfscope}%
\pgfpathrectangle{\pgfqpoint{0.100000in}{0.220728in}}{\pgfqpoint{3.696000in}{3.696000in}}%
\pgfusepath{clip}%
\pgfsetbuttcap%
\pgfsetroundjoin%
\definecolor{currentfill}{rgb}{0.121569,0.466667,0.705882}%
\pgfsetfillcolor{currentfill}%
\pgfsetfillopacity{0.678805}%
\pgfsetlinewidth{1.003750pt}%
\definecolor{currentstroke}{rgb}{0.121569,0.466667,0.705882}%
\pgfsetstrokecolor{currentstroke}%
\pgfsetstrokeopacity{0.678805}%
\pgfsetdash{}{0pt}%
\pgfpathmoveto{\pgfqpoint{0.726761in}{1.407837in}}%
\pgfpathcurveto{\pgfqpoint{0.734998in}{1.407837in}}{\pgfqpoint{0.742898in}{1.411110in}}{\pgfqpoint{0.748722in}{1.416934in}}%
\pgfpathcurveto{\pgfqpoint{0.754546in}{1.422758in}}{\pgfqpoint{0.757818in}{1.430658in}}{\pgfqpoint{0.757818in}{1.438894in}}%
\pgfpathcurveto{\pgfqpoint{0.757818in}{1.447130in}}{\pgfqpoint{0.754546in}{1.455030in}}{\pgfqpoint{0.748722in}{1.460854in}}%
\pgfpathcurveto{\pgfqpoint{0.742898in}{1.466678in}}{\pgfqpoint{0.734998in}{1.469950in}}{\pgfqpoint{0.726761in}{1.469950in}}%
\pgfpathcurveto{\pgfqpoint{0.718525in}{1.469950in}}{\pgfqpoint{0.710625in}{1.466678in}}{\pgfqpoint{0.704801in}{1.460854in}}%
\pgfpathcurveto{\pgfqpoint{0.698977in}{1.455030in}}{\pgfqpoint{0.695705in}{1.447130in}}{\pgfqpoint{0.695705in}{1.438894in}}%
\pgfpathcurveto{\pgfqpoint{0.695705in}{1.430658in}}{\pgfqpoint{0.698977in}{1.422758in}}{\pgfqpoint{0.704801in}{1.416934in}}%
\pgfpathcurveto{\pgfqpoint{0.710625in}{1.411110in}}{\pgfqpoint{0.718525in}{1.407837in}}{\pgfqpoint{0.726761in}{1.407837in}}%
\pgfpathclose%
\pgfusepath{stroke,fill}%
\end{pgfscope}%
\begin{pgfscope}%
\pgfpathrectangle{\pgfqpoint{0.100000in}{0.220728in}}{\pgfqpoint{3.696000in}{3.696000in}}%
\pgfusepath{clip}%
\pgfsetbuttcap%
\pgfsetroundjoin%
\definecolor{currentfill}{rgb}{0.121569,0.466667,0.705882}%
\pgfsetfillcolor{currentfill}%
\pgfsetfillopacity{0.678874}%
\pgfsetlinewidth{1.003750pt}%
\definecolor{currentstroke}{rgb}{0.121569,0.466667,0.705882}%
\pgfsetstrokecolor{currentstroke}%
\pgfsetstrokeopacity{0.678874}%
\pgfsetdash{}{0pt}%
\pgfpathmoveto{\pgfqpoint{3.374261in}{2.885883in}}%
\pgfpathcurveto{\pgfqpoint{3.382498in}{2.885883in}}{\pgfqpoint{3.390398in}{2.889156in}}{\pgfqpoint{3.396222in}{2.894980in}}%
\pgfpathcurveto{\pgfqpoint{3.402046in}{2.900803in}}{\pgfqpoint{3.405318in}{2.908704in}}{\pgfqpoint{3.405318in}{2.916940in}}%
\pgfpathcurveto{\pgfqpoint{3.405318in}{2.925176in}}{\pgfqpoint{3.402046in}{2.933076in}}{\pgfqpoint{3.396222in}{2.938900in}}%
\pgfpathcurveto{\pgfqpoint{3.390398in}{2.944724in}}{\pgfqpoint{3.382498in}{2.947996in}}{\pgfqpoint{3.374261in}{2.947996in}}%
\pgfpathcurveto{\pgfqpoint{3.366025in}{2.947996in}}{\pgfqpoint{3.358125in}{2.944724in}}{\pgfqpoint{3.352301in}{2.938900in}}%
\pgfpathcurveto{\pgfqpoint{3.346477in}{2.933076in}}{\pgfqpoint{3.343205in}{2.925176in}}{\pgfqpoint{3.343205in}{2.916940in}}%
\pgfpathcurveto{\pgfqpoint{3.343205in}{2.908704in}}{\pgfqpoint{3.346477in}{2.900803in}}{\pgfqpoint{3.352301in}{2.894980in}}%
\pgfpathcurveto{\pgfqpoint{3.358125in}{2.889156in}}{\pgfqpoint{3.366025in}{2.885883in}}{\pgfqpoint{3.374261in}{2.885883in}}%
\pgfpathclose%
\pgfusepath{stroke,fill}%
\end{pgfscope}%
\begin{pgfscope}%
\pgfpathrectangle{\pgfqpoint{0.100000in}{0.220728in}}{\pgfqpoint{3.696000in}{3.696000in}}%
\pgfusepath{clip}%
\pgfsetbuttcap%
\pgfsetroundjoin%
\definecolor{currentfill}{rgb}{0.121569,0.466667,0.705882}%
\pgfsetfillcolor{currentfill}%
\pgfsetfillopacity{0.679515}%
\pgfsetlinewidth{1.003750pt}%
\definecolor{currentstroke}{rgb}{0.121569,0.466667,0.705882}%
\pgfsetstrokecolor{currentstroke}%
\pgfsetstrokeopacity{0.679515}%
\pgfsetdash{}{0pt}%
\pgfpathmoveto{\pgfqpoint{3.371947in}{2.882061in}}%
\pgfpathcurveto{\pgfqpoint{3.380183in}{2.882061in}}{\pgfqpoint{3.388083in}{2.885334in}}{\pgfqpoint{3.393907in}{2.891158in}}%
\pgfpathcurveto{\pgfqpoint{3.399731in}{2.896981in}}{\pgfqpoint{3.403003in}{2.904881in}}{\pgfqpoint{3.403003in}{2.913118in}}%
\pgfpathcurveto{\pgfqpoint{3.403003in}{2.921354in}}{\pgfqpoint{3.399731in}{2.929254in}}{\pgfqpoint{3.393907in}{2.935078in}}%
\pgfpathcurveto{\pgfqpoint{3.388083in}{2.940902in}}{\pgfqpoint{3.380183in}{2.944174in}}{\pgfqpoint{3.371947in}{2.944174in}}%
\pgfpathcurveto{\pgfqpoint{3.363710in}{2.944174in}}{\pgfqpoint{3.355810in}{2.940902in}}{\pgfqpoint{3.349986in}{2.935078in}}%
\pgfpathcurveto{\pgfqpoint{3.344162in}{2.929254in}}{\pgfqpoint{3.340890in}{2.921354in}}{\pgfqpoint{3.340890in}{2.913118in}}%
\pgfpathcurveto{\pgfqpoint{3.340890in}{2.904881in}}{\pgfqpoint{3.344162in}{2.896981in}}{\pgfqpoint{3.349986in}{2.891158in}}%
\pgfpathcurveto{\pgfqpoint{3.355810in}{2.885334in}}{\pgfqpoint{3.363710in}{2.882061in}}{\pgfqpoint{3.371947in}{2.882061in}}%
\pgfpathclose%
\pgfusepath{stroke,fill}%
\end{pgfscope}%
\begin{pgfscope}%
\pgfpathrectangle{\pgfqpoint{0.100000in}{0.220728in}}{\pgfqpoint{3.696000in}{3.696000in}}%
\pgfusepath{clip}%
\pgfsetbuttcap%
\pgfsetroundjoin%
\definecolor{currentfill}{rgb}{0.121569,0.466667,0.705882}%
\pgfsetfillcolor{currentfill}%
\pgfsetfillopacity{0.680416}%
\pgfsetlinewidth{1.003750pt}%
\definecolor{currentstroke}{rgb}{0.121569,0.466667,0.705882}%
\pgfsetstrokecolor{currentstroke}%
\pgfsetstrokeopacity{0.680416}%
\pgfsetdash{}{0pt}%
\pgfpathmoveto{\pgfqpoint{3.369409in}{2.877459in}}%
\pgfpathcurveto{\pgfqpoint{3.377645in}{2.877459in}}{\pgfqpoint{3.385545in}{2.880732in}}{\pgfqpoint{3.391369in}{2.886556in}}%
\pgfpathcurveto{\pgfqpoint{3.397193in}{2.892380in}}{\pgfqpoint{3.400465in}{2.900280in}}{\pgfqpoint{3.400465in}{2.908516in}}%
\pgfpathcurveto{\pgfqpoint{3.400465in}{2.916752in}}{\pgfqpoint{3.397193in}{2.924652in}}{\pgfqpoint{3.391369in}{2.930476in}}%
\pgfpathcurveto{\pgfqpoint{3.385545in}{2.936300in}}{\pgfqpoint{3.377645in}{2.939572in}}{\pgfqpoint{3.369409in}{2.939572in}}%
\pgfpathcurveto{\pgfqpoint{3.361173in}{2.939572in}}{\pgfqpoint{3.353273in}{2.936300in}}{\pgfqpoint{3.347449in}{2.930476in}}%
\pgfpathcurveto{\pgfqpoint{3.341625in}{2.924652in}}{\pgfqpoint{3.338352in}{2.916752in}}{\pgfqpoint{3.338352in}{2.908516in}}%
\pgfpathcurveto{\pgfqpoint{3.338352in}{2.900280in}}{\pgfqpoint{3.341625in}{2.892380in}}{\pgfqpoint{3.347449in}{2.886556in}}%
\pgfpathcurveto{\pgfqpoint{3.353273in}{2.880732in}}{\pgfqpoint{3.361173in}{2.877459in}}{\pgfqpoint{3.369409in}{2.877459in}}%
\pgfpathclose%
\pgfusepath{stroke,fill}%
\end{pgfscope}%
\begin{pgfscope}%
\pgfpathrectangle{\pgfqpoint{0.100000in}{0.220728in}}{\pgfqpoint{3.696000in}{3.696000in}}%
\pgfusepath{clip}%
\pgfsetbuttcap%
\pgfsetroundjoin%
\definecolor{currentfill}{rgb}{0.121569,0.466667,0.705882}%
\pgfsetfillcolor{currentfill}%
\pgfsetfillopacity{0.681083}%
\pgfsetlinewidth{1.003750pt}%
\definecolor{currentstroke}{rgb}{0.121569,0.466667,0.705882}%
\pgfsetstrokecolor{currentstroke}%
\pgfsetstrokeopacity{0.681083}%
\pgfsetdash{}{0pt}%
\pgfpathmoveto{\pgfqpoint{0.738996in}{1.402639in}}%
\pgfpathcurveto{\pgfqpoint{0.747233in}{1.402639in}}{\pgfqpoint{0.755133in}{1.405912in}}{\pgfqpoint{0.760957in}{1.411735in}}%
\pgfpathcurveto{\pgfqpoint{0.766781in}{1.417559in}}{\pgfqpoint{0.770053in}{1.425459in}}{\pgfqpoint{0.770053in}{1.433696in}}%
\pgfpathcurveto{\pgfqpoint{0.770053in}{1.441932in}}{\pgfqpoint{0.766781in}{1.449832in}}{\pgfqpoint{0.760957in}{1.455656in}}%
\pgfpathcurveto{\pgfqpoint{0.755133in}{1.461480in}}{\pgfqpoint{0.747233in}{1.464752in}}{\pgfqpoint{0.738996in}{1.464752in}}%
\pgfpathcurveto{\pgfqpoint{0.730760in}{1.464752in}}{\pgfqpoint{0.722860in}{1.461480in}}{\pgfqpoint{0.717036in}{1.455656in}}%
\pgfpathcurveto{\pgfqpoint{0.711212in}{1.449832in}}{\pgfqpoint{0.707940in}{1.441932in}}{\pgfqpoint{0.707940in}{1.433696in}}%
\pgfpathcurveto{\pgfqpoint{0.707940in}{1.425459in}}{\pgfqpoint{0.711212in}{1.417559in}}{\pgfqpoint{0.717036in}{1.411735in}}%
\pgfpathcurveto{\pgfqpoint{0.722860in}{1.405912in}}{\pgfqpoint{0.730760in}{1.402639in}}{\pgfqpoint{0.738996in}{1.402639in}}%
\pgfpathclose%
\pgfusepath{stroke,fill}%
\end{pgfscope}%
\begin{pgfscope}%
\pgfpathrectangle{\pgfqpoint{0.100000in}{0.220728in}}{\pgfqpoint{3.696000in}{3.696000in}}%
\pgfusepath{clip}%
\pgfsetbuttcap%
\pgfsetroundjoin%
\definecolor{currentfill}{rgb}{0.121569,0.466667,0.705882}%
\pgfsetfillcolor{currentfill}%
\pgfsetfillopacity{0.681401}%
\pgfsetlinewidth{1.003750pt}%
\definecolor{currentstroke}{rgb}{0.121569,0.466667,0.705882}%
\pgfsetstrokecolor{currentstroke}%
\pgfsetstrokeopacity{0.681401}%
\pgfsetdash{}{0pt}%
\pgfpathmoveto{\pgfqpoint{3.367115in}{2.871586in}}%
\pgfpathcurveto{\pgfqpoint{3.375352in}{2.871586in}}{\pgfqpoint{3.383252in}{2.874859in}}{\pgfqpoint{3.389076in}{2.880683in}}%
\pgfpathcurveto{\pgfqpoint{3.394899in}{2.886507in}}{\pgfqpoint{3.398172in}{2.894407in}}{\pgfqpoint{3.398172in}{2.902643in}}%
\pgfpathcurveto{\pgfqpoint{3.398172in}{2.910879in}}{\pgfqpoint{3.394899in}{2.918779in}}{\pgfqpoint{3.389076in}{2.924603in}}%
\pgfpathcurveto{\pgfqpoint{3.383252in}{2.930427in}}{\pgfqpoint{3.375352in}{2.933699in}}{\pgfqpoint{3.367115in}{2.933699in}}%
\pgfpathcurveto{\pgfqpoint{3.358879in}{2.933699in}}{\pgfqpoint{3.350979in}{2.930427in}}{\pgfqpoint{3.345155in}{2.924603in}}%
\pgfpathcurveto{\pgfqpoint{3.339331in}{2.918779in}}{\pgfqpoint{3.336059in}{2.910879in}}{\pgfqpoint{3.336059in}{2.902643in}}%
\pgfpathcurveto{\pgfqpoint{3.336059in}{2.894407in}}{\pgfqpoint{3.339331in}{2.886507in}}{\pgfqpoint{3.345155in}{2.880683in}}%
\pgfpathcurveto{\pgfqpoint{3.350979in}{2.874859in}}{\pgfqpoint{3.358879in}{2.871586in}}{\pgfqpoint{3.367115in}{2.871586in}}%
\pgfpathclose%
\pgfusepath{stroke,fill}%
\end{pgfscope}%
\begin{pgfscope}%
\pgfpathrectangle{\pgfqpoint{0.100000in}{0.220728in}}{\pgfqpoint{3.696000in}{3.696000in}}%
\pgfusepath{clip}%
\pgfsetbuttcap%
\pgfsetroundjoin%
\definecolor{currentfill}{rgb}{0.121569,0.466667,0.705882}%
\pgfsetfillcolor{currentfill}%
\pgfsetfillopacity{0.682268}%
\pgfsetlinewidth{1.003750pt}%
\definecolor{currentstroke}{rgb}{0.121569,0.466667,0.705882}%
\pgfsetstrokecolor{currentstroke}%
\pgfsetstrokeopacity{0.682268}%
\pgfsetdash{}{0pt}%
\pgfpathmoveto{\pgfqpoint{3.362381in}{2.864947in}}%
\pgfpathcurveto{\pgfqpoint{3.370618in}{2.864947in}}{\pgfqpoint{3.378518in}{2.868220in}}{\pgfqpoint{3.384342in}{2.874044in}}%
\pgfpathcurveto{\pgfqpoint{3.390166in}{2.879867in}}{\pgfqpoint{3.393438in}{2.887768in}}{\pgfqpoint{3.393438in}{2.896004in}}%
\pgfpathcurveto{\pgfqpoint{3.393438in}{2.904240in}}{\pgfqpoint{3.390166in}{2.912140in}}{\pgfqpoint{3.384342in}{2.917964in}}%
\pgfpathcurveto{\pgfqpoint{3.378518in}{2.923788in}}{\pgfqpoint{3.370618in}{2.927060in}}{\pgfqpoint{3.362381in}{2.927060in}}%
\pgfpathcurveto{\pgfqpoint{3.354145in}{2.927060in}}{\pgfqpoint{3.346245in}{2.923788in}}{\pgfqpoint{3.340421in}{2.917964in}}%
\pgfpathcurveto{\pgfqpoint{3.334597in}{2.912140in}}{\pgfqpoint{3.331325in}{2.904240in}}{\pgfqpoint{3.331325in}{2.896004in}}%
\pgfpathcurveto{\pgfqpoint{3.331325in}{2.887768in}}{\pgfqpoint{3.334597in}{2.879867in}}{\pgfqpoint{3.340421in}{2.874044in}}%
\pgfpathcurveto{\pgfqpoint{3.346245in}{2.868220in}}{\pgfqpoint{3.354145in}{2.864947in}}{\pgfqpoint{3.362381in}{2.864947in}}%
\pgfpathclose%
\pgfusepath{stroke,fill}%
\end{pgfscope}%
\begin{pgfscope}%
\pgfpathrectangle{\pgfqpoint{0.100000in}{0.220728in}}{\pgfqpoint{3.696000in}{3.696000in}}%
\pgfusepath{clip}%
\pgfsetbuttcap%
\pgfsetroundjoin%
\definecolor{currentfill}{rgb}{0.121569,0.466667,0.705882}%
\pgfsetfillcolor{currentfill}%
\pgfsetfillopacity{0.682435}%
\pgfsetlinewidth{1.003750pt}%
\definecolor{currentstroke}{rgb}{0.121569,0.466667,0.705882}%
\pgfsetstrokecolor{currentstroke}%
\pgfsetstrokeopacity{0.682435}%
\pgfsetdash{}{0pt}%
\pgfpathmoveto{\pgfqpoint{0.748318in}{1.395917in}}%
\pgfpathcurveto{\pgfqpoint{0.756554in}{1.395917in}}{\pgfqpoint{0.764454in}{1.399189in}}{\pgfqpoint{0.770278in}{1.405013in}}%
\pgfpathcurveto{\pgfqpoint{0.776102in}{1.410837in}}{\pgfqpoint{0.779375in}{1.418737in}}{\pgfqpoint{0.779375in}{1.426974in}}%
\pgfpathcurveto{\pgfqpoint{0.779375in}{1.435210in}}{\pgfqpoint{0.776102in}{1.443110in}}{\pgfqpoint{0.770278in}{1.448934in}}%
\pgfpathcurveto{\pgfqpoint{0.764454in}{1.454758in}}{\pgfqpoint{0.756554in}{1.458030in}}{\pgfqpoint{0.748318in}{1.458030in}}%
\pgfpathcurveto{\pgfqpoint{0.740082in}{1.458030in}}{\pgfqpoint{0.732182in}{1.454758in}}{\pgfqpoint{0.726358in}{1.448934in}}%
\pgfpathcurveto{\pgfqpoint{0.720534in}{1.443110in}}{\pgfqpoint{0.717262in}{1.435210in}}{\pgfqpoint{0.717262in}{1.426974in}}%
\pgfpathcurveto{\pgfqpoint{0.717262in}{1.418737in}}{\pgfqpoint{0.720534in}{1.410837in}}{\pgfqpoint{0.726358in}{1.405013in}}%
\pgfpathcurveto{\pgfqpoint{0.732182in}{1.399189in}}{\pgfqpoint{0.740082in}{1.395917in}}{\pgfqpoint{0.748318in}{1.395917in}}%
\pgfpathclose%
\pgfusepath{stroke,fill}%
\end{pgfscope}%
\begin{pgfscope}%
\pgfpathrectangle{\pgfqpoint{0.100000in}{0.220728in}}{\pgfqpoint{3.696000in}{3.696000in}}%
\pgfusepath{clip}%
\pgfsetbuttcap%
\pgfsetroundjoin%
\definecolor{currentfill}{rgb}{0.121569,0.466667,0.705882}%
\pgfsetfillcolor{currentfill}%
\pgfsetfillopacity{0.683809}%
\pgfsetlinewidth{1.003750pt}%
\definecolor{currentstroke}{rgb}{0.121569,0.466667,0.705882}%
\pgfsetstrokecolor{currentstroke}%
\pgfsetstrokeopacity{0.683809}%
\pgfsetdash{}{0pt}%
\pgfpathmoveto{\pgfqpoint{3.359135in}{2.857613in}}%
\pgfpathcurveto{\pgfqpoint{3.367371in}{2.857613in}}{\pgfqpoint{3.375271in}{2.860885in}}{\pgfqpoint{3.381095in}{2.866709in}}%
\pgfpathcurveto{\pgfqpoint{3.386919in}{2.872533in}}{\pgfqpoint{3.390191in}{2.880433in}}{\pgfqpoint{3.390191in}{2.888669in}}%
\pgfpathcurveto{\pgfqpoint{3.390191in}{2.896905in}}{\pgfqpoint{3.386919in}{2.904805in}}{\pgfqpoint{3.381095in}{2.910629in}}%
\pgfpathcurveto{\pgfqpoint{3.375271in}{2.916453in}}{\pgfqpoint{3.367371in}{2.919726in}}{\pgfqpoint{3.359135in}{2.919726in}}%
\pgfpathcurveto{\pgfqpoint{3.350899in}{2.919726in}}{\pgfqpoint{3.342999in}{2.916453in}}{\pgfqpoint{3.337175in}{2.910629in}}%
\pgfpathcurveto{\pgfqpoint{3.331351in}{2.904805in}}{\pgfqpoint{3.328078in}{2.896905in}}{\pgfqpoint{3.328078in}{2.888669in}}%
\pgfpathcurveto{\pgfqpoint{3.328078in}{2.880433in}}{\pgfqpoint{3.331351in}{2.872533in}}{\pgfqpoint{3.337175in}{2.866709in}}%
\pgfpathcurveto{\pgfqpoint{3.342999in}{2.860885in}}{\pgfqpoint{3.350899in}{2.857613in}}{\pgfqpoint{3.359135in}{2.857613in}}%
\pgfpathclose%
\pgfusepath{stroke,fill}%
\end{pgfscope}%
\begin{pgfscope}%
\pgfpathrectangle{\pgfqpoint{0.100000in}{0.220728in}}{\pgfqpoint{3.696000in}{3.696000in}}%
\pgfusepath{clip}%
\pgfsetbuttcap%
\pgfsetroundjoin%
\definecolor{currentfill}{rgb}{0.121569,0.466667,0.705882}%
\pgfsetfillcolor{currentfill}%
\pgfsetfillopacity{0.684544}%
\pgfsetlinewidth{1.003750pt}%
\definecolor{currentstroke}{rgb}{0.121569,0.466667,0.705882}%
\pgfsetstrokecolor{currentstroke}%
\pgfsetstrokeopacity{0.684544}%
\pgfsetdash{}{0pt}%
\pgfpathmoveto{\pgfqpoint{0.756239in}{1.391706in}}%
\pgfpathcurveto{\pgfqpoint{0.764476in}{1.391706in}}{\pgfqpoint{0.772376in}{1.394979in}}{\pgfqpoint{0.778200in}{1.400803in}}%
\pgfpathcurveto{\pgfqpoint{0.784024in}{1.406627in}}{\pgfqpoint{0.787296in}{1.414527in}}{\pgfqpoint{0.787296in}{1.422763in}}%
\pgfpathcurveto{\pgfqpoint{0.787296in}{1.430999in}}{\pgfqpoint{0.784024in}{1.438899in}}{\pgfqpoint{0.778200in}{1.444723in}}%
\pgfpathcurveto{\pgfqpoint{0.772376in}{1.450547in}}{\pgfqpoint{0.764476in}{1.453819in}}{\pgfqpoint{0.756239in}{1.453819in}}%
\pgfpathcurveto{\pgfqpoint{0.748003in}{1.453819in}}{\pgfqpoint{0.740103in}{1.450547in}}{\pgfqpoint{0.734279in}{1.444723in}}%
\pgfpathcurveto{\pgfqpoint{0.728455in}{1.438899in}}{\pgfqpoint{0.725183in}{1.430999in}}{\pgfqpoint{0.725183in}{1.422763in}}%
\pgfpathcurveto{\pgfqpoint{0.725183in}{1.414527in}}{\pgfqpoint{0.728455in}{1.406627in}}{\pgfqpoint{0.734279in}{1.400803in}}%
\pgfpathcurveto{\pgfqpoint{0.740103in}{1.394979in}}{\pgfqpoint{0.748003in}{1.391706in}}{\pgfqpoint{0.756239in}{1.391706in}}%
\pgfpathclose%
\pgfusepath{stroke,fill}%
\end{pgfscope}%
\begin{pgfscope}%
\pgfpathrectangle{\pgfqpoint{0.100000in}{0.220728in}}{\pgfqpoint{3.696000in}{3.696000in}}%
\pgfusepath{clip}%
\pgfsetbuttcap%
\pgfsetroundjoin%
\definecolor{currentfill}{rgb}{0.121569,0.466667,0.705882}%
\pgfsetfillcolor{currentfill}%
\pgfsetfillopacity{0.685474}%
\pgfsetlinewidth{1.003750pt}%
\definecolor{currentstroke}{rgb}{0.121569,0.466667,0.705882}%
\pgfsetstrokecolor{currentstroke}%
\pgfsetstrokeopacity{0.685474}%
\pgfsetdash{}{0pt}%
\pgfpathmoveto{\pgfqpoint{3.355055in}{2.850101in}}%
\pgfpathcurveto{\pgfqpoint{3.363291in}{2.850101in}}{\pgfqpoint{3.371191in}{2.853373in}}{\pgfqpoint{3.377015in}{2.859197in}}%
\pgfpathcurveto{\pgfqpoint{3.382839in}{2.865021in}}{\pgfqpoint{3.386112in}{2.872921in}}{\pgfqpoint{3.386112in}{2.881158in}}%
\pgfpathcurveto{\pgfqpoint{3.386112in}{2.889394in}}{\pgfqpoint{3.382839in}{2.897294in}}{\pgfqpoint{3.377015in}{2.903118in}}%
\pgfpathcurveto{\pgfqpoint{3.371191in}{2.908942in}}{\pgfqpoint{3.363291in}{2.912214in}}{\pgfqpoint{3.355055in}{2.912214in}}%
\pgfpathcurveto{\pgfqpoint{3.346819in}{2.912214in}}{\pgfqpoint{3.338919in}{2.908942in}}{\pgfqpoint{3.333095in}{2.903118in}}%
\pgfpathcurveto{\pgfqpoint{3.327271in}{2.897294in}}{\pgfqpoint{3.323999in}{2.889394in}}{\pgfqpoint{3.323999in}{2.881158in}}%
\pgfpathcurveto{\pgfqpoint{3.323999in}{2.872921in}}{\pgfqpoint{3.327271in}{2.865021in}}{\pgfqpoint{3.333095in}{2.859197in}}%
\pgfpathcurveto{\pgfqpoint{3.338919in}{2.853373in}}{\pgfqpoint{3.346819in}{2.850101in}}{\pgfqpoint{3.355055in}{2.850101in}}%
\pgfpathclose%
\pgfusepath{stroke,fill}%
\end{pgfscope}%
\begin{pgfscope}%
\pgfpathrectangle{\pgfqpoint{0.100000in}{0.220728in}}{\pgfqpoint{3.696000in}{3.696000in}}%
\pgfusepath{clip}%
\pgfsetbuttcap%
\pgfsetroundjoin%
\definecolor{currentfill}{rgb}{0.121569,0.466667,0.705882}%
\pgfsetfillcolor{currentfill}%
\pgfsetfillopacity{0.686719}%
\pgfsetlinewidth{1.003750pt}%
\definecolor{currentstroke}{rgb}{0.121569,0.466667,0.705882}%
\pgfsetstrokecolor{currentstroke}%
\pgfsetstrokeopacity{0.686719}%
\pgfsetdash{}{0pt}%
\pgfpathmoveto{\pgfqpoint{3.349222in}{2.841680in}}%
\pgfpathcurveto{\pgfqpoint{3.357458in}{2.841680in}}{\pgfqpoint{3.365358in}{2.844952in}}{\pgfqpoint{3.371182in}{2.850776in}}%
\pgfpathcurveto{\pgfqpoint{3.377006in}{2.856600in}}{\pgfqpoint{3.380278in}{2.864500in}}{\pgfqpoint{3.380278in}{2.872736in}}%
\pgfpathcurveto{\pgfqpoint{3.380278in}{2.880972in}}{\pgfqpoint{3.377006in}{2.888872in}}{\pgfqpoint{3.371182in}{2.894696in}}%
\pgfpathcurveto{\pgfqpoint{3.365358in}{2.900520in}}{\pgfqpoint{3.357458in}{2.903793in}}{\pgfqpoint{3.349222in}{2.903793in}}%
\pgfpathcurveto{\pgfqpoint{3.340985in}{2.903793in}}{\pgfqpoint{3.333085in}{2.900520in}}{\pgfqpoint{3.327261in}{2.894696in}}%
\pgfpathcurveto{\pgfqpoint{3.321437in}{2.888872in}}{\pgfqpoint{3.318165in}{2.880972in}}{\pgfqpoint{3.318165in}{2.872736in}}%
\pgfpathcurveto{\pgfqpoint{3.318165in}{2.864500in}}{\pgfqpoint{3.321437in}{2.856600in}}{\pgfqpoint{3.327261in}{2.850776in}}%
\pgfpathcurveto{\pgfqpoint{3.333085in}{2.844952in}}{\pgfqpoint{3.340985in}{2.841680in}}{\pgfqpoint{3.349222in}{2.841680in}}%
\pgfpathclose%
\pgfusepath{stroke,fill}%
\end{pgfscope}%
\begin{pgfscope}%
\pgfpathrectangle{\pgfqpoint{0.100000in}{0.220728in}}{\pgfqpoint{3.696000in}{3.696000in}}%
\pgfusepath{clip}%
\pgfsetbuttcap%
\pgfsetroundjoin%
\definecolor{currentfill}{rgb}{0.121569,0.466667,0.705882}%
\pgfsetfillcolor{currentfill}%
\pgfsetfillopacity{0.688649}%
\pgfsetlinewidth{1.003750pt}%
\definecolor{currentstroke}{rgb}{0.121569,0.466667,0.705882}%
\pgfsetstrokecolor{currentstroke}%
\pgfsetstrokeopacity{0.688649}%
\pgfsetdash{}{0pt}%
\pgfpathmoveto{\pgfqpoint{3.344383in}{2.830249in}}%
\pgfpathcurveto{\pgfqpoint{3.352619in}{2.830249in}}{\pgfqpoint{3.360519in}{2.833521in}}{\pgfqpoint{3.366343in}{2.839345in}}%
\pgfpathcurveto{\pgfqpoint{3.372167in}{2.845169in}}{\pgfqpoint{3.375439in}{2.853069in}}{\pgfqpoint{3.375439in}{2.861305in}}%
\pgfpathcurveto{\pgfqpoint{3.375439in}{2.869542in}}{\pgfqpoint{3.372167in}{2.877442in}}{\pgfqpoint{3.366343in}{2.883266in}}%
\pgfpathcurveto{\pgfqpoint{3.360519in}{2.889090in}}{\pgfqpoint{3.352619in}{2.892362in}}{\pgfqpoint{3.344383in}{2.892362in}}%
\pgfpathcurveto{\pgfqpoint{3.336147in}{2.892362in}}{\pgfqpoint{3.328247in}{2.889090in}}{\pgfqpoint{3.322423in}{2.883266in}}%
\pgfpathcurveto{\pgfqpoint{3.316599in}{2.877442in}}{\pgfqpoint{3.313326in}{2.869542in}}{\pgfqpoint{3.313326in}{2.861305in}}%
\pgfpathcurveto{\pgfqpoint{3.313326in}{2.853069in}}{\pgfqpoint{3.316599in}{2.845169in}}{\pgfqpoint{3.322423in}{2.839345in}}%
\pgfpathcurveto{\pgfqpoint{3.328247in}{2.833521in}}{\pgfqpoint{3.336147in}{2.830249in}}{\pgfqpoint{3.344383in}{2.830249in}}%
\pgfpathclose%
\pgfusepath{stroke,fill}%
\end{pgfscope}%
\begin{pgfscope}%
\pgfpathrectangle{\pgfqpoint{0.100000in}{0.220728in}}{\pgfqpoint{3.696000in}{3.696000in}}%
\pgfusepath{clip}%
\pgfsetbuttcap%
\pgfsetroundjoin%
\definecolor{currentfill}{rgb}{0.121569,0.466667,0.705882}%
\pgfsetfillcolor{currentfill}%
\pgfsetfillopacity{0.688788}%
\pgfsetlinewidth{1.003750pt}%
\definecolor{currentstroke}{rgb}{0.121569,0.466667,0.705882}%
\pgfsetstrokecolor{currentstroke}%
\pgfsetstrokeopacity{0.688788}%
\pgfsetdash{}{0pt}%
\pgfpathmoveto{\pgfqpoint{0.771002in}{1.386825in}}%
\pgfpathcurveto{\pgfqpoint{0.779239in}{1.386825in}}{\pgfqpoint{0.787139in}{1.390098in}}{\pgfqpoint{0.792963in}{1.395922in}}%
\pgfpathcurveto{\pgfqpoint{0.798787in}{1.401746in}}{\pgfqpoint{0.802059in}{1.409646in}}{\pgfqpoint{0.802059in}{1.417882in}}%
\pgfpathcurveto{\pgfqpoint{0.802059in}{1.426118in}}{\pgfqpoint{0.798787in}{1.434018in}}{\pgfqpoint{0.792963in}{1.439842in}}%
\pgfpathcurveto{\pgfqpoint{0.787139in}{1.445666in}}{\pgfqpoint{0.779239in}{1.448938in}}{\pgfqpoint{0.771002in}{1.448938in}}%
\pgfpathcurveto{\pgfqpoint{0.762766in}{1.448938in}}{\pgfqpoint{0.754866in}{1.445666in}}{\pgfqpoint{0.749042in}{1.439842in}}%
\pgfpathcurveto{\pgfqpoint{0.743218in}{1.434018in}}{\pgfqpoint{0.739946in}{1.426118in}}{\pgfqpoint{0.739946in}{1.417882in}}%
\pgfpathcurveto{\pgfqpoint{0.739946in}{1.409646in}}{\pgfqpoint{0.743218in}{1.401746in}}{\pgfqpoint{0.749042in}{1.395922in}}%
\pgfpathcurveto{\pgfqpoint{0.754866in}{1.390098in}}{\pgfqpoint{0.762766in}{1.386825in}}{\pgfqpoint{0.771002in}{1.386825in}}%
\pgfpathclose%
\pgfusepath{stroke,fill}%
\end{pgfscope}%
\begin{pgfscope}%
\pgfpathrectangle{\pgfqpoint{0.100000in}{0.220728in}}{\pgfqpoint{3.696000in}{3.696000in}}%
\pgfusepath{clip}%
\pgfsetbuttcap%
\pgfsetroundjoin%
\definecolor{currentfill}{rgb}{0.121569,0.466667,0.705882}%
\pgfsetfillcolor{currentfill}%
\pgfsetfillopacity{0.689498}%
\pgfsetlinewidth{1.003750pt}%
\definecolor{currentstroke}{rgb}{0.121569,0.466667,0.705882}%
\pgfsetstrokecolor{currentstroke}%
\pgfsetstrokeopacity{0.689498}%
\pgfsetdash{}{0pt}%
\pgfpathmoveto{\pgfqpoint{3.340778in}{2.824280in}}%
\pgfpathcurveto{\pgfqpoint{3.349014in}{2.824280in}}{\pgfqpoint{3.356914in}{2.827553in}}{\pgfqpoint{3.362738in}{2.833377in}}%
\pgfpathcurveto{\pgfqpoint{3.368562in}{2.839201in}}{\pgfqpoint{3.371835in}{2.847101in}}{\pgfqpoint{3.371835in}{2.855337in}}%
\pgfpathcurveto{\pgfqpoint{3.371835in}{2.863573in}}{\pgfqpoint{3.368562in}{2.871473in}}{\pgfqpoint{3.362738in}{2.877297in}}%
\pgfpathcurveto{\pgfqpoint{3.356914in}{2.883121in}}{\pgfqpoint{3.349014in}{2.886393in}}{\pgfqpoint{3.340778in}{2.886393in}}%
\pgfpathcurveto{\pgfqpoint{3.332542in}{2.886393in}}{\pgfqpoint{3.324642in}{2.883121in}}{\pgfqpoint{3.318818in}{2.877297in}}%
\pgfpathcurveto{\pgfqpoint{3.312994in}{2.871473in}}{\pgfqpoint{3.309722in}{2.863573in}}{\pgfqpoint{3.309722in}{2.855337in}}%
\pgfpathcurveto{\pgfqpoint{3.309722in}{2.847101in}}{\pgfqpoint{3.312994in}{2.839201in}}{\pgfqpoint{3.318818in}{2.833377in}}%
\pgfpathcurveto{\pgfqpoint{3.324642in}{2.827553in}}{\pgfqpoint{3.332542in}{2.824280in}}{\pgfqpoint{3.340778in}{2.824280in}}%
\pgfpathclose%
\pgfusepath{stroke,fill}%
\end{pgfscope}%
\begin{pgfscope}%
\pgfpathrectangle{\pgfqpoint{0.100000in}{0.220728in}}{\pgfqpoint{3.696000in}{3.696000in}}%
\pgfusepath{clip}%
\pgfsetbuttcap%
\pgfsetroundjoin%
\definecolor{currentfill}{rgb}{0.121569,0.466667,0.705882}%
\pgfsetfillcolor{currentfill}%
\pgfsetfillopacity{0.689997}%
\pgfsetlinewidth{1.003750pt}%
\definecolor{currentstroke}{rgb}{0.121569,0.466667,0.705882}%
\pgfsetstrokecolor{currentstroke}%
\pgfsetstrokeopacity{0.689997}%
\pgfsetdash{}{0pt}%
\pgfpathmoveto{\pgfqpoint{3.338810in}{2.821112in}}%
\pgfpathcurveto{\pgfqpoint{3.347046in}{2.821112in}}{\pgfqpoint{3.354946in}{2.824384in}}{\pgfqpoint{3.360770in}{2.830208in}}%
\pgfpathcurveto{\pgfqpoint{3.366594in}{2.836032in}}{\pgfqpoint{3.369866in}{2.843932in}}{\pgfqpoint{3.369866in}{2.852168in}}%
\pgfpathcurveto{\pgfqpoint{3.369866in}{2.860404in}}{\pgfqpoint{3.366594in}{2.868304in}}{\pgfqpoint{3.360770in}{2.874128in}}%
\pgfpathcurveto{\pgfqpoint{3.354946in}{2.879952in}}{\pgfqpoint{3.347046in}{2.883225in}}{\pgfqpoint{3.338810in}{2.883225in}}%
\pgfpathcurveto{\pgfqpoint{3.330573in}{2.883225in}}{\pgfqpoint{3.322673in}{2.879952in}}{\pgfqpoint{3.316849in}{2.874128in}}%
\pgfpathcurveto{\pgfqpoint{3.311025in}{2.868304in}}{\pgfqpoint{3.307753in}{2.860404in}}{\pgfqpoint{3.307753in}{2.852168in}}%
\pgfpathcurveto{\pgfqpoint{3.307753in}{2.843932in}}{\pgfqpoint{3.311025in}{2.836032in}}{\pgfqpoint{3.316849in}{2.830208in}}%
\pgfpathcurveto{\pgfqpoint{3.322673in}{2.824384in}}{\pgfqpoint{3.330573in}{2.821112in}}{\pgfqpoint{3.338810in}{2.821112in}}%
\pgfpathclose%
\pgfusepath{stroke,fill}%
\end{pgfscope}%
\begin{pgfscope}%
\pgfpathrectangle{\pgfqpoint{0.100000in}{0.220728in}}{\pgfqpoint{3.696000in}{3.696000in}}%
\pgfusepath{clip}%
\pgfsetbuttcap%
\pgfsetroundjoin%
\definecolor{currentfill}{rgb}{0.121569,0.466667,0.705882}%
\pgfsetfillcolor{currentfill}%
\pgfsetfillopacity{0.690285}%
\pgfsetlinewidth{1.003750pt}%
\definecolor{currentstroke}{rgb}{0.121569,0.466667,0.705882}%
\pgfsetstrokecolor{currentstroke}%
\pgfsetstrokeopacity{0.690285}%
\pgfsetdash{}{0pt}%
\pgfpathmoveto{\pgfqpoint{3.338056in}{2.819013in}}%
\pgfpathcurveto{\pgfqpoint{3.346293in}{2.819013in}}{\pgfqpoint{3.354193in}{2.822286in}}{\pgfqpoint{3.360017in}{2.828110in}}%
\pgfpathcurveto{\pgfqpoint{3.365841in}{2.833933in}}{\pgfqpoint{3.369113in}{2.841834in}}{\pgfqpoint{3.369113in}{2.850070in}}%
\pgfpathcurveto{\pgfqpoint{3.369113in}{2.858306in}}{\pgfqpoint{3.365841in}{2.866206in}}{\pgfqpoint{3.360017in}{2.872030in}}%
\pgfpathcurveto{\pgfqpoint{3.354193in}{2.877854in}}{\pgfqpoint{3.346293in}{2.881126in}}{\pgfqpoint{3.338056in}{2.881126in}}%
\pgfpathcurveto{\pgfqpoint{3.329820in}{2.881126in}}{\pgfqpoint{3.321920in}{2.877854in}}{\pgfqpoint{3.316096in}{2.872030in}}%
\pgfpathcurveto{\pgfqpoint{3.310272in}{2.866206in}}{\pgfqpoint{3.307000in}{2.858306in}}{\pgfqpoint{3.307000in}{2.850070in}}%
\pgfpathcurveto{\pgfqpoint{3.307000in}{2.841834in}}{\pgfqpoint{3.310272in}{2.833933in}}{\pgfqpoint{3.316096in}{2.828110in}}%
\pgfpathcurveto{\pgfqpoint{3.321920in}{2.822286in}}{\pgfqpoint{3.329820in}{2.819013in}}{\pgfqpoint{3.338056in}{2.819013in}}%
\pgfpathclose%
\pgfusepath{stroke,fill}%
\end{pgfscope}%
\begin{pgfscope}%
\pgfpathrectangle{\pgfqpoint{0.100000in}{0.220728in}}{\pgfqpoint{3.696000in}{3.696000in}}%
\pgfusepath{clip}%
\pgfsetbuttcap%
\pgfsetroundjoin%
\definecolor{currentfill}{rgb}{0.121569,0.466667,0.705882}%
\pgfsetfillcolor{currentfill}%
\pgfsetfillopacity{0.690805}%
\pgfsetlinewidth{1.003750pt}%
\definecolor{currentstroke}{rgb}{0.121569,0.466667,0.705882}%
\pgfsetstrokecolor{currentstroke}%
\pgfsetstrokeopacity{0.690805}%
\pgfsetdash{}{0pt}%
\pgfpathmoveto{\pgfqpoint{3.335168in}{2.814632in}}%
\pgfpathcurveto{\pgfqpoint{3.343404in}{2.814632in}}{\pgfqpoint{3.351304in}{2.817905in}}{\pgfqpoint{3.357128in}{2.823729in}}%
\pgfpathcurveto{\pgfqpoint{3.362952in}{2.829552in}}{\pgfqpoint{3.366224in}{2.837452in}}{\pgfqpoint{3.366224in}{2.845689in}}%
\pgfpathcurveto{\pgfqpoint{3.366224in}{2.853925in}}{\pgfqpoint{3.362952in}{2.861825in}}{\pgfqpoint{3.357128in}{2.867649in}}%
\pgfpathcurveto{\pgfqpoint{3.351304in}{2.873473in}}{\pgfqpoint{3.343404in}{2.876745in}}{\pgfqpoint{3.335168in}{2.876745in}}%
\pgfpathcurveto{\pgfqpoint{3.326931in}{2.876745in}}{\pgfqpoint{3.319031in}{2.873473in}}{\pgfqpoint{3.313207in}{2.867649in}}%
\pgfpathcurveto{\pgfqpoint{3.307383in}{2.861825in}}{\pgfqpoint{3.304111in}{2.853925in}}{\pgfqpoint{3.304111in}{2.845689in}}%
\pgfpathcurveto{\pgfqpoint{3.304111in}{2.837452in}}{\pgfqpoint{3.307383in}{2.829552in}}{\pgfqpoint{3.313207in}{2.823729in}}%
\pgfpathcurveto{\pgfqpoint{3.319031in}{2.817905in}}{\pgfqpoint{3.326931in}{2.814632in}}{\pgfqpoint{3.335168in}{2.814632in}}%
\pgfpathclose%
\pgfusepath{stroke,fill}%
\end{pgfscope}%
\begin{pgfscope}%
\pgfpathrectangle{\pgfqpoint{0.100000in}{0.220728in}}{\pgfqpoint{3.696000in}{3.696000in}}%
\pgfusepath{clip}%
\pgfsetbuttcap%
\pgfsetroundjoin%
\definecolor{currentfill}{rgb}{0.121569,0.466667,0.705882}%
\pgfsetfillcolor{currentfill}%
\pgfsetfillopacity{0.691207}%
\pgfsetlinewidth{1.003750pt}%
\definecolor{currentstroke}{rgb}{0.121569,0.466667,0.705882}%
\pgfsetstrokecolor{currentstroke}%
\pgfsetstrokeopacity{0.691207}%
\pgfsetdash{}{0pt}%
\pgfpathmoveto{\pgfqpoint{3.333972in}{2.812122in}}%
\pgfpathcurveto{\pgfqpoint{3.342209in}{2.812122in}}{\pgfqpoint{3.350109in}{2.815394in}}{\pgfqpoint{3.355933in}{2.821218in}}%
\pgfpathcurveto{\pgfqpoint{3.361757in}{2.827042in}}{\pgfqpoint{3.365029in}{2.834942in}}{\pgfqpoint{3.365029in}{2.843178in}}%
\pgfpathcurveto{\pgfqpoint{3.365029in}{2.851415in}}{\pgfqpoint{3.361757in}{2.859315in}}{\pgfqpoint{3.355933in}{2.865139in}}%
\pgfpathcurveto{\pgfqpoint{3.350109in}{2.870963in}}{\pgfqpoint{3.342209in}{2.874235in}}{\pgfqpoint{3.333972in}{2.874235in}}%
\pgfpathcurveto{\pgfqpoint{3.325736in}{2.874235in}}{\pgfqpoint{3.317836in}{2.870963in}}{\pgfqpoint{3.312012in}{2.865139in}}%
\pgfpathcurveto{\pgfqpoint{3.306188in}{2.859315in}}{\pgfqpoint{3.302916in}{2.851415in}}{\pgfqpoint{3.302916in}{2.843178in}}%
\pgfpathcurveto{\pgfqpoint{3.302916in}{2.834942in}}{\pgfqpoint{3.306188in}{2.827042in}}{\pgfqpoint{3.312012in}{2.821218in}}%
\pgfpathcurveto{\pgfqpoint{3.317836in}{2.815394in}}{\pgfqpoint{3.325736in}{2.812122in}}{\pgfqpoint{3.333972in}{2.812122in}}%
\pgfpathclose%
\pgfusepath{stroke,fill}%
\end{pgfscope}%
\begin{pgfscope}%
\pgfpathrectangle{\pgfqpoint{0.100000in}{0.220728in}}{\pgfqpoint{3.696000in}{3.696000in}}%
\pgfusepath{clip}%
\pgfsetbuttcap%
\pgfsetroundjoin%
\definecolor{currentfill}{rgb}{0.121569,0.466667,0.705882}%
\pgfsetfillcolor{currentfill}%
\pgfsetfillopacity{0.691262}%
\pgfsetlinewidth{1.003750pt}%
\definecolor{currentstroke}{rgb}{0.121569,0.466667,0.705882}%
\pgfsetstrokecolor{currentstroke}%
\pgfsetstrokeopacity{0.691262}%
\pgfsetdash{}{0pt}%
\pgfpathmoveto{\pgfqpoint{0.783661in}{1.377612in}}%
\pgfpathcurveto{\pgfqpoint{0.791897in}{1.377612in}}{\pgfqpoint{0.799797in}{1.380885in}}{\pgfqpoint{0.805621in}{1.386709in}}%
\pgfpathcurveto{\pgfqpoint{0.811445in}{1.392533in}}{\pgfqpoint{0.814718in}{1.400433in}}{\pgfqpoint{0.814718in}{1.408669in}}%
\pgfpathcurveto{\pgfqpoint{0.814718in}{1.416905in}}{\pgfqpoint{0.811445in}{1.424805in}}{\pgfqpoint{0.805621in}{1.430629in}}%
\pgfpathcurveto{\pgfqpoint{0.799797in}{1.436453in}}{\pgfqpoint{0.791897in}{1.439725in}}{\pgfqpoint{0.783661in}{1.439725in}}%
\pgfpathcurveto{\pgfqpoint{0.775425in}{1.439725in}}{\pgfqpoint{0.767525in}{1.436453in}}{\pgfqpoint{0.761701in}{1.430629in}}%
\pgfpathcurveto{\pgfqpoint{0.755877in}{1.424805in}}{\pgfqpoint{0.752605in}{1.416905in}}{\pgfqpoint{0.752605in}{1.408669in}}%
\pgfpathcurveto{\pgfqpoint{0.752605in}{1.400433in}}{\pgfqpoint{0.755877in}{1.392533in}}{\pgfqpoint{0.761701in}{1.386709in}}%
\pgfpathcurveto{\pgfqpoint{0.767525in}{1.380885in}}{\pgfqpoint{0.775425in}{1.377612in}}{\pgfqpoint{0.783661in}{1.377612in}}%
\pgfpathclose%
\pgfusepath{stroke,fill}%
\end{pgfscope}%
\begin{pgfscope}%
\pgfpathrectangle{\pgfqpoint{0.100000in}{0.220728in}}{\pgfqpoint{3.696000in}{3.696000in}}%
\pgfusepath{clip}%
\pgfsetbuttcap%
\pgfsetroundjoin%
\definecolor{currentfill}{rgb}{0.121569,0.466667,0.705882}%
\pgfsetfillcolor{currentfill}%
\pgfsetfillopacity{0.691462}%
\pgfsetlinewidth{1.003750pt}%
\definecolor{currentstroke}{rgb}{0.121569,0.466667,0.705882}%
\pgfsetstrokecolor{currentstroke}%
\pgfsetstrokeopacity{0.691462}%
\pgfsetdash{}{0pt}%
\pgfpathmoveto{\pgfqpoint{3.333421in}{2.810769in}}%
\pgfpathcurveto{\pgfqpoint{3.341658in}{2.810769in}}{\pgfqpoint{3.349558in}{2.814041in}}{\pgfqpoint{3.355382in}{2.819865in}}%
\pgfpathcurveto{\pgfqpoint{3.361206in}{2.825689in}}{\pgfqpoint{3.364478in}{2.833589in}}{\pgfqpoint{3.364478in}{2.841825in}}%
\pgfpathcurveto{\pgfqpoint{3.364478in}{2.850062in}}{\pgfqpoint{3.361206in}{2.857962in}}{\pgfqpoint{3.355382in}{2.863786in}}%
\pgfpathcurveto{\pgfqpoint{3.349558in}{2.869610in}}{\pgfqpoint{3.341658in}{2.872882in}}{\pgfqpoint{3.333421in}{2.872882in}}%
\pgfpathcurveto{\pgfqpoint{3.325185in}{2.872882in}}{\pgfqpoint{3.317285in}{2.869610in}}{\pgfqpoint{3.311461in}{2.863786in}}%
\pgfpathcurveto{\pgfqpoint{3.305637in}{2.857962in}}{\pgfqpoint{3.302365in}{2.850062in}}{\pgfqpoint{3.302365in}{2.841825in}}%
\pgfpathcurveto{\pgfqpoint{3.302365in}{2.833589in}}{\pgfqpoint{3.305637in}{2.825689in}}{\pgfqpoint{3.311461in}{2.819865in}}%
\pgfpathcurveto{\pgfqpoint{3.317285in}{2.814041in}}{\pgfqpoint{3.325185in}{2.810769in}}{\pgfqpoint{3.333421in}{2.810769in}}%
\pgfpathclose%
\pgfusepath{stroke,fill}%
\end{pgfscope}%
\begin{pgfscope}%
\pgfpathrectangle{\pgfqpoint{0.100000in}{0.220728in}}{\pgfqpoint{3.696000in}{3.696000in}}%
\pgfusepath{clip}%
\pgfsetbuttcap%
\pgfsetroundjoin%
\definecolor{currentfill}{rgb}{0.121569,0.466667,0.705882}%
\pgfsetfillcolor{currentfill}%
\pgfsetfillopacity{0.691542}%
\pgfsetlinewidth{1.003750pt}%
\definecolor{currentstroke}{rgb}{0.121569,0.466667,0.705882}%
\pgfsetstrokecolor{currentstroke}%
\pgfsetstrokeopacity{0.691542}%
\pgfsetdash{}{0pt}%
\pgfpathmoveto{\pgfqpoint{3.332931in}{2.810032in}}%
\pgfpathcurveto{\pgfqpoint{3.341167in}{2.810032in}}{\pgfqpoint{3.349067in}{2.813305in}}{\pgfqpoint{3.354891in}{2.819128in}}%
\pgfpathcurveto{\pgfqpoint{3.360715in}{2.824952in}}{\pgfqpoint{3.363987in}{2.832852in}}{\pgfqpoint{3.363987in}{2.841089in}}%
\pgfpathcurveto{\pgfqpoint{3.363987in}{2.849325in}}{\pgfqpoint{3.360715in}{2.857225in}}{\pgfqpoint{3.354891in}{2.863049in}}%
\pgfpathcurveto{\pgfqpoint{3.349067in}{2.868873in}}{\pgfqpoint{3.341167in}{2.872145in}}{\pgfqpoint{3.332931in}{2.872145in}}%
\pgfpathcurveto{\pgfqpoint{3.324694in}{2.872145in}}{\pgfqpoint{3.316794in}{2.868873in}}{\pgfqpoint{3.310970in}{2.863049in}}%
\pgfpathcurveto{\pgfqpoint{3.305147in}{2.857225in}}{\pgfqpoint{3.301874in}{2.849325in}}{\pgfqpoint{3.301874in}{2.841089in}}%
\pgfpathcurveto{\pgfqpoint{3.301874in}{2.832852in}}{\pgfqpoint{3.305147in}{2.824952in}}{\pgfqpoint{3.310970in}{2.819128in}}%
\pgfpathcurveto{\pgfqpoint{3.316794in}{2.813305in}}{\pgfqpoint{3.324694in}{2.810032in}}{\pgfqpoint{3.332931in}{2.810032in}}%
\pgfpathclose%
\pgfusepath{stroke,fill}%
\end{pgfscope}%
\begin{pgfscope}%
\pgfpathrectangle{\pgfqpoint{0.100000in}{0.220728in}}{\pgfqpoint{3.696000in}{3.696000in}}%
\pgfusepath{clip}%
\pgfsetbuttcap%
\pgfsetroundjoin%
\definecolor{currentfill}{rgb}{0.121569,0.466667,0.705882}%
\pgfsetfillcolor{currentfill}%
\pgfsetfillopacity{0.692026}%
\pgfsetlinewidth{1.003750pt}%
\definecolor{currentstroke}{rgb}{0.121569,0.466667,0.705882}%
\pgfsetstrokecolor{currentstroke}%
\pgfsetstrokeopacity{0.692026}%
\pgfsetdash{}{0pt}%
\pgfpathmoveto{\pgfqpoint{3.331774in}{2.807276in}}%
\pgfpathcurveto{\pgfqpoint{3.340010in}{2.807276in}}{\pgfqpoint{3.347910in}{2.810548in}}{\pgfqpoint{3.353734in}{2.816372in}}%
\pgfpathcurveto{\pgfqpoint{3.359558in}{2.822196in}}{\pgfqpoint{3.362830in}{2.830096in}}{\pgfqpoint{3.362830in}{2.838332in}}%
\pgfpathcurveto{\pgfqpoint{3.362830in}{2.846569in}}{\pgfqpoint{3.359558in}{2.854469in}}{\pgfqpoint{3.353734in}{2.860293in}}%
\pgfpathcurveto{\pgfqpoint{3.347910in}{2.866117in}}{\pgfqpoint{3.340010in}{2.869389in}}{\pgfqpoint{3.331774in}{2.869389in}}%
\pgfpathcurveto{\pgfqpoint{3.323537in}{2.869389in}}{\pgfqpoint{3.315637in}{2.866117in}}{\pgfqpoint{3.309814in}{2.860293in}}%
\pgfpathcurveto{\pgfqpoint{3.303990in}{2.854469in}}{\pgfqpoint{3.300717in}{2.846569in}}{\pgfqpoint{3.300717in}{2.838332in}}%
\pgfpathcurveto{\pgfqpoint{3.300717in}{2.830096in}}{\pgfqpoint{3.303990in}{2.822196in}}{\pgfqpoint{3.309814in}{2.816372in}}%
\pgfpathcurveto{\pgfqpoint{3.315637in}{2.810548in}}{\pgfqpoint{3.323537in}{2.807276in}}{\pgfqpoint{3.331774in}{2.807276in}}%
\pgfpathclose%
\pgfusepath{stroke,fill}%
\end{pgfscope}%
\begin{pgfscope}%
\pgfpathrectangle{\pgfqpoint{0.100000in}{0.220728in}}{\pgfqpoint{3.696000in}{3.696000in}}%
\pgfusepath{clip}%
\pgfsetbuttcap%
\pgfsetroundjoin%
\definecolor{currentfill}{rgb}{0.121569,0.466667,0.705882}%
\pgfsetfillcolor{currentfill}%
\pgfsetfillopacity{0.692579}%
\pgfsetlinewidth{1.003750pt}%
\definecolor{currentstroke}{rgb}{0.121569,0.466667,0.705882}%
\pgfsetstrokecolor{currentstroke}%
\pgfsetstrokeopacity{0.692579}%
\pgfsetdash{}{0pt}%
\pgfpathmoveto{\pgfqpoint{3.330003in}{2.804258in}}%
\pgfpathcurveto{\pgfqpoint{3.338239in}{2.804258in}}{\pgfqpoint{3.346139in}{2.807530in}}{\pgfqpoint{3.351963in}{2.813354in}}%
\pgfpathcurveto{\pgfqpoint{3.357787in}{2.819178in}}{\pgfqpoint{3.361059in}{2.827078in}}{\pgfqpoint{3.361059in}{2.835314in}}%
\pgfpathcurveto{\pgfqpoint{3.361059in}{2.843550in}}{\pgfqpoint{3.357787in}{2.851450in}}{\pgfqpoint{3.351963in}{2.857274in}}%
\pgfpathcurveto{\pgfqpoint{3.346139in}{2.863098in}}{\pgfqpoint{3.338239in}{2.866371in}}{\pgfqpoint{3.330003in}{2.866371in}}%
\pgfpathcurveto{\pgfqpoint{3.321767in}{2.866371in}}{\pgfqpoint{3.313866in}{2.863098in}}{\pgfqpoint{3.308043in}{2.857274in}}%
\pgfpathcurveto{\pgfqpoint{3.302219in}{2.851450in}}{\pgfqpoint{3.298946in}{2.843550in}}{\pgfqpoint{3.298946in}{2.835314in}}%
\pgfpathcurveto{\pgfqpoint{3.298946in}{2.827078in}}{\pgfqpoint{3.302219in}{2.819178in}}{\pgfqpoint{3.308043in}{2.813354in}}%
\pgfpathcurveto{\pgfqpoint{3.313866in}{2.807530in}}{\pgfqpoint{3.321767in}{2.804258in}}{\pgfqpoint{3.330003in}{2.804258in}}%
\pgfpathclose%
\pgfusepath{stroke,fill}%
\end{pgfscope}%
\begin{pgfscope}%
\pgfpathrectangle{\pgfqpoint{0.100000in}{0.220728in}}{\pgfqpoint{3.696000in}{3.696000in}}%
\pgfusepath{clip}%
\pgfsetbuttcap%
\pgfsetroundjoin%
\definecolor{currentfill}{rgb}{0.121569,0.466667,0.705882}%
\pgfsetfillcolor{currentfill}%
\pgfsetfillopacity{0.692739}%
\pgfsetlinewidth{1.003750pt}%
\definecolor{currentstroke}{rgb}{0.121569,0.466667,0.705882}%
\pgfsetstrokecolor{currentstroke}%
\pgfsetstrokeopacity{0.692739}%
\pgfsetdash{}{0pt}%
\pgfpathmoveto{\pgfqpoint{3.328806in}{2.802344in}}%
\pgfpathcurveto{\pgfqpoint{3.337042in}{2.802344in}}{\pgfqpoint{3.344942in}{2.805616in}}{\pgfqpoint{3.350766in}{2.811440in}}%
\pgfpathcurveto{\pgfqpoint{3.356590in}{2.817264in}}{\pgfqpoint{3.359862in}{2.825164in}}{\pgfqpoint{3.359862in}{2.833400in}}%
\pgfpathcurveto{\pgfqpoint{3.359862in}{2.841637in}}{\pgfqpoint{3.356590in}{2.849537in}}{\pgfqpoint{3.350766in}{2.855361in}}%
\pgfpathcurveto{\pgfqpoint{3.344942in}{2.861185in}}{\pgfqpoint{3.337042in}{2.864457in}}{\pgfqpoint{3.328806in}{2.864457in}}%
\pgfpathcurveto{\pgfqpoint{3.320569in}{2.864457in}}{\pgfqpoint{3.312669in}{2.861185in}}{\pgfqpoint{3.306845in}{2.855361in}}%
\pgfpathcurveto{\pgfqpoint{3.301021in}{2.849537in}}{\pgfqpoint{3.297749in}{2.841637in}}{\pgfqpoint{3.297749in}{2.833400in}}%
\pgfpathcurveto{\pgfqpoint{3.297749in}{2.825164in}}{\pgfqpoint{3.301021in}{2.817264in}}{\pgfqpoint{3.306845in}{2.811440in}}%
\pgfpathcurveto{\pgfqpoint{3.312669in}{2.805616in}}{\pgfqpoint{3.320569in}{2.802344in}}{\pgfqpoint{3.328806in}{2.802344in}}%
\pgfpathclose%
\pgfusepath{stroke,fill}%
\end{pgfscope}%
\begin{pgfscope}%
\pgfpathrectangle{\pgfqpoint{0.100000in}{0.220728in}}{\pgfqpoint{3.696000in}{3.696000in}}%
\pgfusepath{clip}%
\pgfsetbuttcap%
\pgfsetroundjoin%
\definecolor{currentfill}{rgb}{0.121569,0.466667,0.705882}%
\pgfsetfillcolor{currentfill}%
\pgfsetfillopacity{0.693224}%
\pgfsetlinewidth{1.003750pt}%
\definecolor{currentstroke}{rgb}{0.121569,0.466667,0.705882}%
\pgfsetstrokecolor{currentstroke}%
\pgfsetstrokeopacity{0.693224}%
\pgfsetdash{}{0pt}%
\pgfpathmoveto{\pgfqpoint{3.327804in}{2.799246in}}%
\pgfpathcurveto{\pgfqpoint{3.336041in}{2.799246in}}{\pgfqpoint{3.343941in}{2.802518in}}{\pgfqpoint{3.349765in}{2.808342in}}%
\pgfpathcurveto{\pgfqpoint{3.355589in}{2.814166in}}{\pgfqpoint{3.358861in}{2.822066in}}{\pgfqpoint{3.358861in}{2.830302in}}%
\pgfpathcurveto{\pgfqpoint{3.358861in}{2.838538in}}{\pgfqpoint{3.355589in}{2.846438in}}{\pgfqpoint{3.349765in}{2.852262in}}%
\pgfpathcurveto{\pgfqpoint{3.343941in}{2.858086in}}{\pgfqpoint{3.336041in}{2.861359in}}{\pgfqpoint{3.327804in}{2.861359in}}%
\pgfpathcurveto{\pgfqpoint{3.319568in}{2.861359in}}{\pgfqpoint{3.311668in}{2.858086in}}{\pgfqpoint{3.305844in}{2.852262in}}%
\pgfpathcurveto{\pgfqpoint{3.300020in}{2.846438in}}{\pgfqpoint{3.296748in}{2.838538in}}{\pgfqpoint{3.296748in}{2.830302in}}%
\pgfpathcurveto{\pgfqpoint{3.296748in}{2.822066in}}{\pgfqpoint{3.300020in}{2.814166in}}{\pgfqpoint{3.305844in}{2.808342in}}%
\pgfpathcurveto{\pgfqpoint{3.311668in}{2.802518in}}{\pgfqpoint{3.319568in}{2.799246in}}{\pgfqpoint{3.327804in}{2.799246in}}%
\pgfpathclose%
\pgfusepath{stroke,fill}%
\end{pgfscope}%
\begin{pgfscope}%
\pgfpathrectangle{\pgfqpoint{0.100000in}{0.220728in}}{\pgfqpoint{3.696000in}{3.696000in}}%
\pgfusepath{clip}%
\pgfsetbuttcap%
\pgfsetroundjoin%
\definecolor{currentfill}{rgb}{0.121569,0.466667,0.705882}%
\pgfsetfillcolor{currentfill}%
\pgfsetfillopacity{0.693554}%
\pgfsetlinewidth{1.003750pt}%
\definecolor{currentstroke}{rgb}{0.121569,0.466667,0.705882}%
\pgfsetstrokecolor{currentstroke}%
\pgfsetstrokeopacity{0.693554}%
\pgfsetdash{}{0pt}%
\pgfpathmoveto{\pgfqpoint{0.793728in}{1.373425in}}%
\pgfpathcurveto{\pgfqpoint{0.801964in}{1.373425in}}{\pgfqpoint{0.809864in}{1.376697in}}{\pgfqpoint{0.815688in}{1.382521in}}%
\pgfpathcurveto{\pgfqpoint{0.821512in}{1.388345in}}{\pgfqpoint{0.824784in}{1.396245in}}{\pgfqpoint{0.824784in}{1.404481in}}%
\pgfpathcurveto{\pgfqpoint{0.824784in}{1.412718in}}{\pgfqpoint{0.821512in}{1.420618in}}{\pgfqpoint{0.815688in}{1.426442in}}%
\pgfpathcurveto{\pgfqpoint{0.809864in}{1.432265in}}{\pgfqpoint{0.801964in}{1.435538in}}{\pgfqpoint{0.793728in}{1.435538in}}%
\pgfpathcurveto{\pgfqpoint{0.785491in}{1.435538in}}{\pgfqpoint{0.777591in}{1.432265in}}{\pgfqpoint{0.771767in}{1.426442in}}%
\pgfpathcurveto{\pgfqpoint{0.765943in}{1.420618in}}{\pgfqpoint{0.762671in}{1.412718in}}{\pgfqpoint{0.762671in}{1.404481in}}%
\pgfpathcurveto{\pgfqpoint{0.762671in}{1.396245in}}{\pgfqpoint{0.765943in}{1.388345in}}{\pgfqpoint{0.771767in}{1.382521in}}%
\pgfpathcurveto{\pgfqpoint{0.777591in}{1.376697in}}{\pgfqpoint{0.785491in}{1.373425in}}{\pgfqpoint{0.793728in}{1.373425in}}%
\pgfpathclose%
\pgfusepath{stroke,fill}%
\end{pgfscope}%
\begin{pgfscope}%
\pgfpathrectangle{\pgfqpoint{0.100000in}{0.220728in}}{\pgfqpoint{3.696000in}{3.696000in}}%
\pgfusepath{clip}%
\pgfsetbuttcap%
\pgfsetroundjoin%
\definecolor{currentfill}{rgb}{0.121569,0.466667,0.705882}%
\pgfsetfillcolor{currentfill}%
\pgfsetfillopacity{0.693860}%
\pgfsetlinewidth{1.003750pt}%
\definecolor{currentstroke}{rgb}{0.121569,0.466667,0.705882}%
\pgfsetstrokecolor{currentstroke}%
\pgfsetstrokeopacity{0.693860}%
\pgfsetdash{}{0pt}%
\pgfpathmoveto{\pgfqpoint{3.324617in}{2.794689in}}%
\pgfpathcurveto{\pgfqpoint{3.332854in}{2.794689in}}{\pgfqpoint{3.340754in}{2.797961in}}{\pgfqpoint{3.346578in}{2.803785in}}%
\pgfpathcurveto{\pgfqpoint{3.352401in}{2.809609in}}{\pgfqpoint{3.355674in}{2.817509in}}{\pgfqpoint{3.355674in}{2.825745in}}%
\pgfpathcurveto{\pgfqpoint{3.355674in}{2.833982in}}{\pgfqpoint{3.352401in}{2.841882in}}{\pgfqpoint{3.346578in}{2.847706in}}%
\pgfpathcurveto{\pgfqpoint{3.340754in}{2.853529in}}{\pgfqpoint{3.332854in}{2.856802in}}{\pgfqpoint{3.324617in}{2.856802in}}%
\pgfpathcurveto{\pgfqpoint{3.316381in}{2.856802in}}{\pgfqpoint{3.308481in}{2.853529in}}{\pgfqpoint{3.302657in}{2.847706in}}%
\pgfpathcurveto{\pgfqpoint{3.296833in}{2.841882in}}{\pgfqpoint{3.293561in}{2.833982in}}{\pgfqpoint{3.293561in}{2.825745in}}%
\pgfpathcurveto{\pgfqpoint{3.293561in}{2.817509in}}{\pgfqpoint{3.296833in}{2.809609in}}{\pgfqpoint{3.302657in}{2.803785in}}%
\pgfpathcurveto{\pgfqpoint{3.308481in}{2.797961in}}{\pgfqpoint{3.316381in}{2.794689in}}{\pgfqpoint{3.324617in}{2.794689in}}%
\pgfpathclose%
\pgfusepath{stroke,fill}%
\end{pgfscope}%
\begin{pgfscope}%
\pgfpathrectangle{\pgfqpoint{0.100000in}{0.220728in}}{\pgfqpoint{3.696000in}{3.696000in}}%
\pgfusepath{clip}%
\pgfsetbuttcap%
\pgfsetroundjoin%
\definecolor{currentfill}{rgb}{0.121569,0.466667,0.705882}%
\pgfsetfillcolor{currentfill}%
\pgfsetfillopacity{0.694343}%
\pgfsetlinewidth{1.003750pt}%
\definecolor{currentstroke}{rgb}{0.121569,0.466667,0.705882}%
\pgfsetstrokecolor{currentstroke}%
\pgfsetstrokeopacity{0.694343}%
\pgfsetdash{}{0pt}%
\pgfpathmoveto{\pgfqpoint{3.323232in}{2.792173in}}%
\pgfpathcurveto{\pgfqpoint{3.331468in}{2.792173in}}{\pgfqpoint{3.339368in}{2.795446in}}{\pgfqpoint{3.345192in}{2.801270in}}%
\pgfpathcurveto{\pgfqpoint{3.351016in}{2.807094in}}{\pgfqpoint{3.354288in}{2.814994in}}{\pgfqpoint{3.354288in}{2.823230in}}%
\pgfpathcurveto{\pgfqpoint{3.354288in}{2.831466in}}{\pgfqpoint{3.351016in}{2.839366in}}{\pgfqpoint{3.345192in}{2.845190in}}%
\pgfpathcurveto{\pgfqpoint{3.339368in}{2.851014in}}{\pgfqpoint{3.331468in}{2.854286in}}{\pgfqpoint{3.323232in}{2.854286in}}%
\pgfpathcurveto{\pgfqpoint{3.314996in}{2.854286in}}{\pgfqpoint{3.307095in}{2.851014in}}{\pgfqpoint{3.301272in}{2.845190in}}%
\pgfpathcurveto{\pgfqpoint{3.295448in}{2.839366in}}{\pgfqpoint{3.292175in}{2.831466in}}{\pgfqpoint{3.292175in}{2.823230in}}%
\pgfpathcurveto{\pgfqpoint{3.292175in}{2.814994in}}{\pgfqpoint{3.295448in}{2.807094in}}{\pgfqpoint{3.301272in}{2.801270in}}%
\pgfpathcurveto{\pgfqpoint{3.307095in}{2.795446in}}{\pgfqpoint{3.314996in}{2.792173in}}{\pgfqpoint{3.323232in}{2.792173in}}%
\pgfpathclose%
\pgfusepath{stroke,fill}%
\end{pgfscope}%
\begin{pgfscope}%
\pgfpathrectangle{\pgfqpoint{0.100000in}{0.220728in}}{\pgfqpoint{3.696000in}{3.696000in}}%
\pgfusepath{clip}%
\pgfsetbuttcap%
\pgfsetroundjoin%
\definecolor{currentfill}{rgb}{0.121569,0.466667,0.705882}%
\pgfsetfillcolor{currentfill}%
\pgfsetfillopacity{0.694664}%
\pgfsetlinewidth{1.003750pt}%
\definecolor{currentstroke}{rgb}{0.121569,0.466667,0.705882}%
\pgfsetstrokecolor{currentstroke}%
\pgfsetstrokeopacity{0.694664}%
\pgfsetdash{}{0pt}%
\pgfpathmoveto{\pgfqpoint{3.322643in}{2.790837in}}%
\pgfpathcurveto{\pgfqpoint{3.330879in}{2.790837in}}{\pgfqpoint{3.338779in}{2.794110in}}{\pgfqpoint{3.344603in}{2.799934in}}%
\pgfpathcurveto{\pgfqpoint{3.350427in}{2.805757in}}{\pgfqpoint{3.353699in}{2.813658in}}{\pgfqpoint{3.353699in}{2.821894in}}%
\pgfpathcurveto{\pgfqpoint{3.353699in}{2.830130in}}{\pgfqpoint{3.350427in}{2.838030in}}{\pgfqpoint{3.344603in}{2.843854in}}%
\pgfpathcurveto{\pgfqpoint{3.338779in}{2.849678in}}{\pgfqpoint{3.330879in}{2.852950in}}{\pgfqpoint{3.322643in}{2.852950in}}%
\pgfpathcurveto{\pgfqpoint{3.314406in}{2.852950in}}{\pgfqpoint{3.306506in}{2.849678in}}{\pgfqpoint{3.300682in}{2.843854in}}%
\pgfpathcurveto{\pgfqpoint{3.294858in}{2.838030in}}{\pgfqpoint{3.291586in}{2.830130in}}{\pgfqpoint{3.291586in}{2.821894in}}%
\pgfpathcurveto{\pgfqpoint{3.291586in}{2.813658in}}{\pgfqpoint{3.294858in}{2.805757in}}{\pgfqpoint{3.300682in}{2.799934in}}%
\pgfpathcurveto{\pgfqpoint{3.306506in}{2.794110in}}{\pgfqpoint{3.314406in}{2.790837in}}{\pgfqpoint{3.322643in}{2.790837in}}%
\pgfpathclose%
\pgfusepath{stroke,fill}%
\end{pgfscope}%
\begin{pgfscope}%
\pgfpathrectangle{\pgfqpoint{0.100000in}{0.220728in}}{\pgfqpoint{3.696000in}{3.696000in}}%
\pgfusepath{clip}%
\pgfsetbuttcap%
\pgfsetroundjoin%
\definecolor{currentfill}{rgb}{0.121569,0.466667,0.705882}%
\pgfsetfillcolor{currentfill}%
\pgfsetfillopacity{0.694792}%
\pgfsetlinewidth{1.003750pt}%
\definecolor{currentstroke}{rgb}{0.121569,0.466667,0.705882}%
\pgfsetstrokecolor{currentstroke}%
\pgfsetstrokeopacity{0.694792}%
\pgfsetdash{}{0pt}%
\pgfpathmoveto{\pgfqpoint{3.322140in}{2.790126in}}%
\pgfpathcurveto{\pgfqpoint{3.330377in}{2.790126in}}{\pgfqpoint{3.338277in}{2.793398in}}{\pgfqpoint{3.344100in}{2.799222in}}%
\pgfpathcurveto{\pgfqpoint{3.349924in}{2.805046in}}{\pgfqpoint{3.353197in}{2.812946in}}{\pgfqpoint{3.353197in}{2.821182in}}%
\pgfpathcurveto{\pgfqpoint{3.353197in}{2.829419in}}{\pgfqpoint{3.349924in}{2.837319in}}{\pgfqpoint{3.344100in}{2.843143in}}%
\pgfpathcurveto{\pgfqpoint{3.338277in}{2.848966in}}{\pgfqpoint{3.330377in}{2.852239in}}{\pgfqpoint{3.322140in}{2.852239in}}%
\pgfpathcurveto{\pgfqpoint{3.313904in}{2.852239in}}{\pgfqpoint{3.306004in}{2.848966in}}{\pgfqpoint{3.300180in}{2.843143in}}%
\pgfpathcurveto{\pgfqpoint{3.294356in}{2.837319in}}{\pgfqpoint{3.291084in}{2.829419in}}{\pgfqpoint{3.291084in}{2.821182in}}%
\pgfpathcurveto{\pgfqpoint{3.291084in}{2.812946in}}{\pgfqpoint{3.294356in}{2.805046in}}{\pgfqpoint{3.300180in}{2.799222in}}%
\pgfpathcurveto{\pgfqpoint{3.306004in}{2.793398in}}{\pgfqpoint{3.313904in}{2.790126in}}{\pgfqpoint{3.322140in}{2.790126in}}%
\pgfpathclose%
\pgfusepath{stroke,fill}%
\end{pgfscope}%
\begin{pgfscope}%
\pgfpathrectangle{\pgfqpoint{0.100000in}{0.220728in}}{\pgfqpoint{3.696000in}{3.696000in}}%
\pgfusepath{clip}%
\pgfsetbuttcap%
\pgfsetroundjoin%
\definecolor{currentfill}{rgb}{0.121569,0.466667,0.705882}%
\pgfsetfillcolor{currentfill}%
\pgfsetfillopacity{0.694894}%
\pgfsetlinewidth{1.003750pt}%
\definecolor{currentstroke}{rgb}{0.121569,0.466667,0.705882}%
\pgfsetstrokecolor{currentstroke}%
\pgfsetstrokeopacity{0.694894}%
\pgfsetdash{}{0pt}%
\pgfpathmoveto{\pgfqpoint{0.801730in}{1.369460in}}%
\pgfpathcurveto{\pgfqpoint{0.809966in}{1.369460in}}{\pgfqpoint{0.817866in}{1.372732in}}{\pgfqpoint{0.823690in}{1.378556in}}%
\pgfpathcurveto{\pgfqpoint{0.829514in}{1.384380in}}{\pgfqpoint{0.832786in}{1.392280in}}{\pgfqpoint{0.832786in}{1.400517in}}%
\pgfpathcurveto{\pgfqpoint{0.832786in}{1.408753in}}{\pgfqpoint{0.829514in}{1.416653in}}{\pgfqpoint{0.823690in}{1.422477in}}%
\pgfpathcurveto{\pgfqpoint{0.817866in}{1.428301in}}{\pgfqpoint{0.809966in}{1.431573in}}{\pgfqpoint{0.801730in}{1.431573in}}%
\pgfpathcurveto{\pgfqpoint{0.793494in}{1.431573in}}{\pgfqpoint{0.785594in}{1.428301in}}{\pgfqpoint{0.779770in}{1.422477in}}%
\pgfpathcurveto{\pgfqpoint{0.773946in}{1.416653in}}{\pgfqpoint{0.770673in}{1.408753in}}{\pgfqpoint{0.770673in}{1.400517in}}%
\pgfpathcurveto{\pgfqpoint{0.770673in}{1.392280in}}{\pgfqpoint{0.773946in}{1.384380in}}{\pgfqpoint{0.779770in}{1.378556in}}%
\pgfpathcurveto{\pgfqpoint{0.785594in}{1.372732in}}{\pgfqpoint{0.793494in}{1.369460in}}{\pgfqpoint{0.801730in}{1.369460in}}%
\pgfpathclose%
\pgfusepath{stroke,fill}%
\end{pgfscope}%
\begin{pgfscope}%
\pgfpathrectangle{\pgfqpoint{0.100000in}{0.220728in}}{\pgfqpoint{3.696000in}{3.696000in}}%
\pgfusepath{clip}%
\pgfsetbuttcap%
\pgfsetroundjoin%
\definecolor{currentfill}{rgb}{0.121569,0.466667,0.705882}%
\pgfsetfillcolor{currentfill}%
\pgfsetfillopacity{0.695319}%
\pgfsetlinewidth{1.003750pt}%
\definecolor{currentstroke}{rgb}{0.121569,0.466667,0.705882}%
\pgfsetstrokecolor{currentstroke}%
\pgfsetstrokeopacity{0.695319}%
\pgfsetdash{}{0pt}%
\pgfpathmoveto{\pgfqpoint{3.320978in}{2.787819in}}%
\pgfpathcurveto{\pgfqpoint{3.329214in}{2.787819in}}{\pgfqpoint{3.337114in}{2.791091in}}{\pgfqpoint{3.342938in}{2.796915in}}%
\pgfpathcurveto{\pgfqpoint{3.348762in}{2.802739in}}{\pgfqpoint{3.352034in}{2.810639in}}{\pgfqpoint{3.352034in}{2.818875in}}%
\pgfpathcurveto{\pgfqpoint{3.352034in}{2.827112in}}{\pgfqpoint{3.348762in}{2.835012in}}{\pgfqpoint{3.342938in}{2.840836in}}%
\pgfpathcurveto{\pgfqpoint{3.337114in}{2.846659in}}{\pgfqpoint{3.329214in}{2.849932in}}{\pgfqpoint{3.320978in}{2.849932in}}%
\pgfpathcurveto{\pgfqpoint{3.312742in}{2.849932in}}{\pgfqpoint{3.304841in}{2.846659in}}{\pgfqpoint{3.299018in}{2.840836in}}%
\pgfpathcurveto{\pgfqpoint{3.293194in}{2.835012in}}{\pgfqpoint{3.289921in}{2.827112in}}{\pgfqpoint{3.289921in}{2.818875in}}%
\pgfpathcurveto{\pgfqpoint{3.289921in}{2.810639in}}{\pgfqpoint{3.293194in}{2.802739in}}{\pgfqpoint{3.299018in}{2.796915in}}%
\pgfpathcurveto{\pgfqpoint{3.304841in}{2.791091in}}{\pgfqpoint{3.312742in}{2.787819in}}{\pgfqpoint{3.320978in}{2.787819in}}%
\pgfpathclose%
\pgfusepath{stroke,fill}%
\end{pgfscope}%
\begin{pgfscope}%
\pgfpathrectangle{\pgfqpoint{0.100000in}{0.220728in}}{\pgfqpoint{3.696000in}{3.696000in}}%
\pgfusepath{clip}%
\pgfsetbuttcap%
\pgfsetroundjoin%
\definecolor{currentfill}{rgb}{0.121569,0.466667,0.705882}%
\pgfsetfillcolor{currentfill}%
\pgfsetfillopacity{0.695598}%
\pgfsetlinewidth{1.003750pt}%
\definecolor{currentstroke}{rgb}{0.121569,0.466667,0.705882}%
\pgfsetstrokecolor{currentstroke}%
\pgfsetstrokeopacity{0.695598}%
\pgfsetdash{}{0pt}%
\pgfpathmoveto{\pgfqpoint{3.320248in}{2.786615in}}%
\pgfpathcurveto{\pgfqpoint{3.328484in}{2.786615in}}{\pgfqpoint{3.336384in}{2.789887in}}{\pgfqpoint{3.342208in}{2.795711in}}%
\pgfpathcurveto{\pgfqpoint{3.348032in}{2.801535in}}{\pgfqpoint{3.351304in}{2.809435in}}{\pgfqpoint{3.351304in}{2.817672in}}%
\pgfpathcurveto{\pgfqpoint{3.351304in}{2.825908in}}{\pgfqpoint{3.348032in}{2.833808in}}{\pgfqpoint{3.342208in}{2.839632in}}%
\pgfpathcurveto{\pgfqpoint{3.336384in}{2.845456in}}{\pgfqpoint{3.328484in}{2.848728in}}{\pgfqpoint{3.320248in}{2.848728in}}%
\pgfpathcurveto{\pgfqpoint{3.312012in}{2.848728in}}{\pgfqpoint{3.304112in}{2.845456in}}{\pgfqpoint{3.298288in}{2.839632in}}%
\pgfpathcurveto{\pgfqpoint{3.292464in}{2.833808in}}{\pgfqpoint{3.289191in}{2.825908in}}{\pgfqpoint{3.289191in}{2.817672in}}%
\pgfpathcurveto{\pgfqpoint{3.289191in}{2.809435in}}{\pgfqpoint{3.292464in}{2.801535in}}{\pgfqpoint{3.298288in}{2.795711in}}%
\pgfpathcurveto{\pgfqpoint{3.304112in}{2.789887in}}{\pgfqpoint{3.312012in}{2.786615in}}{\pgfqpoint{3.320248in}{2.786615in}}%
\pgfpathclose%
\pgfusepath{stroke,fill}%
\end{pgfscope}%
\begin{pgfscope}%
\pgfpathrectangle{\pgfqpoint{0.100000in}{0.220728in}}{\pgfqpoint{3.696000in}{3.696000in}}%
\pgfusepath{clip}%
\pgfsetbuttcap%
\pgfsetroundjoin%
\definecolor{currentfill}{rgb}{0.121569,0.466667,0.705882}%
\pgfsetfillcolor{currentfill}%
\pgfsetfillopacity{0.695737}%
\pgfsetlinewidth{1.003750pt}%
\definecolor{currentstroke}{rgb}{0.121569,0.466667,0.705882}%
\pgfsetstrokecolor{currentstroke}%
\pgfsetstrokeopacity{0.695737}%
\pgfsetdash{}{0pt}%
\pgfpathmoveto{\pgfqpoint{3.319802in}{2.785953in}}%
\pgfpathcurveto{\pgfqpoint{3.328039in}{2.785953in}}{\pgfqpoint{3.335939in}{2.789225in}}{\pgfqpoint{3.341762in}{2.795049in}}%
\pgfpathcurveto{\pgfqpoint{3.347586in}{2.800873in}}{\pgfqpoint{3.350859in}{2.808773in}}{\pgfqpoint{3.350859in}{2.817010in}}%
\pgfpathcurveto{\pgfqpoint{3.350859in}{2.825246in}}{\pgfqpoint{3.347586in}{2.833146in}}{\pgfqpoint{3.341762in}{2.838970in}}%
\pgfpathcurveto{\pgfqpoint{3.335939in}{2.844794in}}{\pgfqpoint{3.328039in}{2.848066in}}{\pgfqpoint{3.319802in}{2.848066in}}%
\pgfpathcurveto{\pgfqpoint{3.311566in}{2.848066in}}{\pgfqpoint{3.303666in}{2.844794in}}{\pgfqpoint{3.297842in}{2.838970in}}%
\pgfpathcurveto{\pgfqpoint{3.292018in}{2.833146in}}{\pgfqpoint{3.288746in}{2.825246in}}{\pgfqpoint{3.288746in}{2.817010in}}%
\pgfpathcurveto{\pgfqpoint{3.288746in}{2.808773in}}{\pgfqpoint{3.292018in}{2.800873in}}{\pgfqpoint{3.297842in}{2.795049in}}%
\pgfpathcurveto{\pgfqpoint{3.303666in}{2.789225in}}{\pgfqpoint{3.311566in}{2.785953in}}{\pgfqpoint{3.319802in}{2.785953in}}%
\pgfpathclose%
\pgfusepath{stroke,fill}%
\end{pgfscope}%
\begin{pgfscope}%
\pgfpathrectangle{\pgfqpoint{0.100000in}{0.220728in}}{\pgfqpoint{3.696000in}{3.696000in}}%
\pgfusepath{clip}%
\pgfsetbuttcap%
\pgfsetroundjoin%
\definecolor{currentfill}{rgb}{0.121569,0.466667,0.705882}%
\pgfsetfillcolor{currentfill}%
\pgfsetfillopacity{0.695815}%
\pgfsetlinewidth{1.003750pt}%
\definecolor{currentstroke}{rgb}{0.121569,0.466667,0.705882}%
\pgfsetstrokecolor{currentstroke}%
\pgfsetstrokeopacity{0.695815}%
\pgfsetdash{}{0pt}%
\pgfpathmoveto{\pgfqpoint{3.319649in}{2.785494in}}%
\pgfpathcurveto{\pgfqpoint{3.327885in}{2.785494in}}{\pgfqpoint{3.335785in}{2.788766in}}{\pgfqpoint{3.341609in}{2.794590in}}%
\pgfpathcurveto{\pgfqpoint{3.347433in}{2.800414in}}{\pgfqpoint{3.350705in}{2.808314in}}{\pgfqpoint{3.350705in}{2.816550in}}%
\pgfpathcurveto{\pgfqpoint{3.350705in}{2.824787in}}{\pgfqpoint{3.347433in}{2.832687in}}{\pgfqpoint{3.341609in}{2.838511in}}%
\pgfpathcurveto{\pgfqpoint{3.335785in}{2.844334in}}{\pgfqpoint{3.327885in}{2.847607in}}{\pgfqpoint{3.319649in}{2.847607in}}%
\pgfpathcurveto{\pgfqpoint{3.311412in}{2.847607in}}{\pgfqpoint{3.303512in}{2.844334in}}{\pgfqpoint{3.297688in}{2.838511in}}%
\pgfpathcurveto{\pgfqpoint{3.291864in}{2.832687in}}{\pgfqpoint{3.288592in}{2.824787in}}{\pgfqpoint{3.288592in}{2.816550in}}%
\pgfpathcurveto{\pgfqpoint{3.288592in}{2.808314in}}{\pgfqpoint{3.291864in}{2.800414in}}{\pgfqpoint{3.297688in}{2.794590in}}%
\pgfpathcurveto{\pgfqpoint{3.303512in}{2.788766in}}{\pgfqpoint{3.311412in}{2.785494in}}{\pgfqpoint{3.319649in}{2.785494in}}%
\pgfpathclose%
\pgfusepath{stroke,fill}%
\end{pgfscope}%
\begin{pgfscope}%
\pgfpathrectangle{\pgfqpoint{0.100000in}{0.220728in}}{\pgfqpoint{3.696000in}{3.696000in}}%
\pgfusepath{clip}%
\pgfsetbuttcap%
\pgfsetroundjoin%
\definecolor{currentfill}{rgb}{0.121569,0.466667,0.705882}%
\pgfsetfillcolor{currentfill}%
\pgfsetfillopacity{0.696230}%
\pgfsetlinewidth{1.003750pt}%
\definecolor{currentstroke}{rgb}{0.121569,0.466667,0.705882}%
\pgfsetstrokecolor{currentstroke}%
\pgfsetstrokeopacity{0.696230}%
\pgfsetdash{}{0pt}%
\pgfpathmoveto{\pgfqpoint{3.317885in}{2.782867in}}%
\pgfpathcurveto{\pgfqpoint{3.326121in}{2.782867in}}{\pgfqpoint{3.334021in}{2.786139in}}{\pgfqpoint{3.339845in}{2.791963in}}%
\pgfpathcurveto{\pgfqpoint{3.345669in}{2.797787in}}{\pgfqpoint{3.348941in}{2.805687in}}{\pgfqpoint{3.348941in}{2.813923in}}%
\pgfpathcurveto{\pgfqpoint{3.348941in}{2.822159in}}{\pgfqpoint{3.345669in}{2.830059in}}{\pgfqpoint{3.339845in}{2.835883in}}%
\pgfpathcurveto{\pgfqpoint{3.334021in}{2.841707in}}{\pgfqpoint{3.326121in}{2.844980in}}{\pgfqpoint{3.317885in}{2.844980in}}%
\pgfpathcurveto{\pgfqpoint{3.309648in}{2.844980in}}{\pgfqpoint{3.301748in}{2.841707in}}{\pgfqpoint{3.295924in}{2.835883in}}%
\pgfpathcurveto{\pgfqpoint{3.290100in}{2.830059in}}{\pgfqpoint{3.286828in}{2.822159in}}{\pgfqpoint{3.286828in}{2.813923in}}%
\pgfpathcurveto{\pgfqpoint{3.286828in}{2.805687in}}{\pgfqpoint{3.290100in}{2.797787in}}{\pgfqpoint{3.295924in}{2.791963in}}%
\pgfpathcurveto{\pgfqpoint{3.301748in}{2.786139in}}{\pgfqpoint{3.309648in}{2.782867in}}{\pgfqpoint{3.317885in}{2.782867in}}%
\pgfpathclose%
\pgfusepath{stroke,fill}%
\end{pgfscope}%
\begin{pgfscope}%
\pgfpathrectangle{\pgfqpoint{0.100000in}{0.220728in}}{\pgfqpoint{3.696000in}{3.696000in}}%
\pgfusepath{clip}%
\pgfsetbuttcap%
\pgfsetroundjoin%
\definecolor{currentfill}{rgb}{0.121569,0.466667,0.705882}%
\pgfsetfillcolor{currentfill}%
\pgfsetfillopacity{0.696489}%
\pgfsetlinewidth{1.003750pt}%
\definecolor{currentstroke}{rgb}{0.121569,0.466667,0.705882}%
\pgfsetstrokecolor{currentstroke}%
\pgfsetstrokeopacity{0.696489}%
\pgfsetdash{}{0pt}%
\pgfpathmoveto{\pgfqpoint{3.317080in}{2.781302in}}%
\pgfpathcurveto{\pgfqpoint{3.325316in}{2.781302in}}{\pgfqpoint{3.333216in}{2.784574in}}{\pgfqpoint{3.339040in}{2.790398in}}%
\pgfpathcurveto{\pgfqpoint{3.344864in}{2.796222in}}{\pgfqpoint{3.348136in}{2.804122in}}{\pgfqpoint{3.348136in}{2.812358in}}%
\pgfpathcurveto{\pgfqpoint{3.348136in}{2.820595in}}{\pgfqpoint{3.344864in}{2.828495in}}{\pgfqpoint{3.339040in}{2.834319in}}%
\pgfpathcurveto{\pgfqpoint{3.333216in}{2.840143in}}{\pgfqpoint{3.325316in}{2.843415in}}{\pgfqpoint{3.317080in}{2.843415in}}%
\pgfpathcurveto{\pgfqpoint{3.308843in}{2.843415in}}{\pgfqpoint{3.300943in}{2.840143in}}{\pgfqpoint{3.295119in}{2.834319in}}%
\pgfpathcurveto{\pgfqpoint{3.289296in}{2.828495in}}{\pgfqpoint{3.286023in}{2.820595in}}{\pgfqpoint{3.286023in}{2.812358in}}%
\pgfpathcurveto{\pgfqpoint{3.286023in}{2.804122in}}{\pgfqpoint{3.289296in}{2.796222in}}{\pgfqpoint{3.295119in}{2.790398in}}%
\pgfpathcurveto{\pgfqpoint{3.300943in}{2.784574in}}{\pgfqpoint{3.308843in}{2.781302in}}{\pgfqpoint{3.317080in}{2.781302in}}%
\pgfpathclose%
\pgfusepath{stroke,fill}%
\end{pgfscope}%
\begin{pgfscope}%
\pgfpathrectangle{\pgfqpoint{0.100000in}{0.220728in}}{\pgfqpoint{3.696000in}{3.696000in}}%
\pgfusepath{clip}%
\pgfsetbuttcap%
\pgfsetroundjoin%
\definecolor{currentfill}{rgb}{0.121569,0.466667,0.705882}%
\pgfsetfillcolor{currentfill}%
\pgfsetfillopacity{0.696861}%
\pgfsetlinewidth{1.003750pt}%
\definecolor{currentstroke}{rgb}{0.121569,0.466667,0.705882}%
\pgfsetstrokecolor{currentstroke}%
\pgfsetstrokeopacity{0.696861}%
\pgfsetdash{}{0pt}%
\pgfpathmoveto{\pgfqpoint{3.316064in}{2.779079in}}%
\pgfpathcurveto{\pgfqpoint{3.324301in}{2.779079in}}{\pgfqpoint{3.332201in}{2.782352in}}{\pgfqpoint{3.338024in}{2.788176in}}%
\pgfpathcurveto{\pgfqpoint{3.343848in}{2.793999in}}{\pgfqpoint{3.347121in}{2.801900in}}{\pgfqpoint{3.347121in}{2.810136in}}%
\pgfpathcurveto{\pgfqpoint{3.347121in}{2.818372in}}{\pgfqpoint{3.343848in}{2.826272in}}{\pgfqpoint{3.338024in}{2.832096in}}%
\pgfpathcurveto{\pgfqpoint{3.332201in}{2.837920in}}{\pgfqpoint{3.324301in}{2.841192in}}{\pgfqpoint{3.316064in}{2.841192in}}%
\pgfpathcurveto{\pgfqpoint{3.307828in}{2.841192in}}{\pgfqpoint{3.299928in}{2.837920in}}{\pgfqpoint{3.294104in}{2.832096in}}%
\pgfpathcurveto{\pgfqpoint{3.288280in}{2.826272in}}{\pgfqpoint{3.285008in}{2.818372in}}{\pgfqpoint{3.285008in}{2.810136in}}%
\pgfpathcurveto{\pgfqpoint{3.285008in}{2.801900in}}{\pgfqpoint{3.288280in}{2.793999in}}{\pgfqpoint{3.294104in}{2.788176in}}%
\pgfpathcurveto{\pgfqpoint{3.299928in}{2.782352in}}{\pgfqpoint{3.307828in}{2.779079in}}{\pgfqpoint{3.316064in}{2.779079in}}%
\pgfpathclose%
\pgfusepath{stroke,fill}%
\end{pgfscope}%
\begin{pgfscope}%
\pgfpathrectangle{\pgfqpoint{0.100000in}{0.220728in}}{\pgfqpoint{3.696000in}{3.696000in}}%
\pgfusepath{clip}%
\pgfsetbuttcap%
\pgfsetroundjoin%
\definecolor{currentfill}{rgb}{0.121569,0.466667,0.705882}%
\pgfsetfillcolor{currentfill}%
\pgfsetfillopacity{0.697042}%
\pgfsetlinewidth{1.003750pt}%
\definecolor{currentstroke}{rgb}{0.121569,0.466667,0.705882}%
\pgfsetstrokecolor{currentstroke}%
\pgfsetstrokeopacity{0.697042}%
\pgfsetdash{}{0pt}%
\pgfpathmoveto{\pgfqpoint{3.315302in}{2.778055in}}%
\pgfpathcurveto{\pgfqpoint{3.323538in}{2.778055in}}{\pgfqpoint{3.331438in}{2.781328in}}{\pgfqpoint{3.337262in}{2.787151in}}%
\pgfpathcurveto{\pgfqpoint{3.343086in}{2.792975in}}{\pgfqpoint{3.346358in}{2.800875in}}{\pgfqpoint{3.346358in}{2.809112in}}%
\pgfpathcurveto{\pgfqpoint{3.346358in}{2.817348in}}{\pgfqpoint{3.343086in}{2.825248in}}{\pgfqpoint{3.337262in}{2.831072in}}%
\pgfpathcurveto{\pgfqpoint{3.331438in}{2.836896in}}{\pgfqpoint{3.323538in}{2.840168in}}{\pgfqpoint{3.315302in}{2.840168in}}%
\pgfpathcurveto{\pgfqpoint{3.307065in}{2.840168in}}{\pgfqpoint{3.299165in}{2.836896in}}{\pgfqpoint{3.293341in}{2.831072in}}%
\pgfpathcurveto{\pgfqpoint{3.287518in}{2.825248in}}{\pgfqpoint{3.284245in}{2.817348in}}{\pgfqpoint{3.284245in}{2.809112in}}%
\pgfpathcurveto{\pgfqpoint{3.284245in}{2.800875in}}{\pgfqpoint{3.287518in}{2.792975in}}{\pgfqpoint{3.293341in}{2.787151in}}%
\pgfpathcurveto{\pgfqpoint{3.299165in}{2.781328in}}{\pgfqpoint{3.307065in}{2.778055in}}{\pgfqpoint{3.315302in}{2.778055in}}%
\pgfpathclose%
\pgfusepath{stroke,fill}%
\end{pgfscope}%
\begin{pgfscope}%
\pgfpathrectangle{\pgfqpoint{0.100000in}{0.220728in}}{\pgfqpoint{3.696000in}{3.696000in}}%
\pgfusepath{clip}%
\pgfsetbuttcap%
\pgfsetroundjoin%
\definecolor{currentfill}{rgb}{0.121569,0.466667,0.705882}%
\pgfsetfillcolor{currentfill}%
\pgfsetfillopacity{0.697203}%
\pgfsetlinewidth{1.003750pt}%
\definecolor{currentstroke}{rgb}{0.121569,0.466667,0.705882}%
\pgfsetstrokecolor{currentstroke}%
\pgfsetstrokeopacity{0.697203}%
\pgfsetdash{}{0pt}%
\pgfpathmoveto{\pgfqpoint{0.816284in}{1.361670in}}%
\pgfpathcurveto{\pgfqpoint{0.824520in}{1.361670in}}{\pgfqpoint{0.832421in}{1.364943in}}{\pgfqpoint{0.838244in}{1.370767in}}%
\pgfpathcurveto{\pgfqpoint{0.844068in}{1.376591in}}{\pgfqpoint{0.847341in}{1.384491in}}{\pgfqpoint{0.847341in}{1.392727in}}%
\pgfpathcurveto{\pgfqpoint{0.847341in}{1.400963in}}{\pgfqpoint{0.844068in}{1.408863in}}{\pgfqpoint{0.838244in}{1.414687in}}%
\pgfpathcurveto{\pgfqpoint{0.832421in}{1.420511in}}{\pgfqpoint{0.824520in}{1.423783in}}{\pgfqpoint{0.816284in}{1.423783in}}%
\pgfpathcurveto{\pgfqpoint{0.808048in}{1.423783in}}{\pgfqpoint{0.800148in}{1.420511in}}{\pgfqpoint{0.794324in}{1.414687in}}%
\pgfpathcurveto{\pgfqpoint{0.788500in}{1.408863in}}{\pgfqpoint{0.785228in}{1.400963in}}{\pgfqpoint{0.785228in}{1.392727in}}%
\pgfpathcurveto{\pgfqpoint{0.785228in}{1.384491in}}{\pgfqpoint{0.788500in}{1.376591in}}{\pgfqpoint{0.794324in}{1.370767in}}%
\pgfpathcurveto{\pgfqpoint{0.800148in}{1.364943in}}{\pgfqpoint{0.808048in}{1.361670in}}{\pgfqpoint{0.816284in}{1.361670in}}%
\pgfpathclose%
\pgfusepath{stroke,fill}%
\end{pgfscope}%
\begin{pgfscope}%
\pgfpathrectangle{\pgfqpoint{0.100000in}{0.220728in}}{\pgfqpoint{3.696000in}{3.696000in}}%
\pgfusepath{clip}%
\pgfsetbuttcap%
\pgfsetroundjoin%
\definecolor{currentfill}{rgb}{0.121569,0.466667,0.705882}%
\pgfsetfillcolor{currentfill}%
\pgfsetfillopacity{0.697587}%
\pgfsetlinewidth{1.003750pt}%
\definecolor{currentstroke}{rgb}{0.121569,0.466667,0.705882}%
\pgfsetstrokecolor{currentstroke}%
\pgfsetstrokeopacity{0.697587}%
\pgfsetdash{}{0pt}%
\pgfpathmoveto{\pgfqpoint{3.313984in}{2.774884in}}%
\pgfpathcurveto{\pgfqpoint{3.322220in}{2.774884in}}{\pgfqpoint{3.330121in}{2.778156in}}{\pgfqpoint{3.335944in}{2.783980in}}%
\pgfpathcurveto{\pgfqpoint{3.341768in}{2.789804in}}{\pgfqpoint{3.345041in}{2.797704in}}{\pgfqpoint{3.345041in}{2.805940in}}%
\pgfpathcurveto{\pgfqpoint{3.345041in}{2.814176in}}{\pgfqpoint{3.341768in}{2.822076in}}{\pgfqpoint{3.335944in}{2.827900in}}%
\pgfpathcurveto{\pgfqpoint{3.330121in}{2.833724in}}{\pgfqpoint{3.322220in}{2.836997in}}{\pgfqpoint{3.313984in}{2.836997in}}%
\pgfpathcurveto{\pgfqpoint{3.305748in}{2.836997in}}{\pgfqpoint{3.297848in}{2.833724in}}{\pgfqpoint{3.292024in}{2.827900in}}%
\pgfpathcurveto{\pgfqpoint{3.286200in}{2.822076in}}{\pgfqpoint{3.282928in}{2.814176in}}{\pgfqpoint{3.282928in}{2.805940in}}%
\pgfpathcurveto{\pgfqpoint{3.282928in}{2.797704in}}{\pgfqpoint{3.286200in}{2.789804in}}{\pgfqpoint{3.292024in}{2.783980in}}%
\pgfpathcurveto{\pgfqpoint{3.297848in}{2.778156in}}{\pgfqpoint{3.305748in}{2.774884in}}{\pgfqpoint{3.313984in}{2.774884in}}%
\pgfpathclose%
\pgfusepath{stroke,fill}%
\end{pgfscope}%
\begin{pgfscope}%
\pgfpathrectangle{\pgfqpoint{0.100000in}{0.220728in}}{\pgfqpoint{3.696000in}{3.696000in}}%
\pgfusepath{clip}%
\pgfsetbuttcap%
\pgfsetroundjoin%
\definecolor{currentfill}{rgb}{0.121569,0.466667,0.705882}%
\pgfsetfillcolor{currentfill}%
\pgfsetfillopacity{0.697878}%
\pgfsetlinewidth{1.003750pt}%
\definecolor{currentstroke}{rgb}{0.121569,0.466667,0.705882}%
\pgfsetstrokecolor{currentstroke}%
\pgfsetstrokeopacity{0.697878}%
\pgfsetdash{}{0pt}%
\pgfpathmoveto{\pgfqpoint{3.313079in}{2.773316in}}%
\pgfpathcurveto{\pgfqpoint{3.321315in}{2.773316in}}{\pgfqpoint{3.329215in}{2.776589in}}{\pgfqpoint{3.335039in}{2.782413in}}%
\pgfpathcurveto{\pgfqpoint{3.340863in}{2.788237in}}{\pgfqpoint{3.344135in}{2.796137in}}{\pgfqpoint{3.344135in}{2.804373in}}%
\pgfpathcurveto{\pgfqpoint{3.344135in}{2.812609in}}{\pgfqpoint{3.340863in}{2.820509in}}{\pgfqpoint{3.335039in}{2.826333in}}%
\pgfpathcurveto{\pgfqpoint{3.329215in}{2.832157in}}{\pgfqpoint{3.321315in}{2.835429in}}{\pgfqpoint{3.313079in}{2.835429in}}%
\pgfpathcurveto{\pgfqpoint{3.304843in}{2.835429in}}{\pgfqpoint{3.296943in}{2.832157in}}{\pgfqpoint{3.291119in}{2.826333in}}%
\pgfpathcurveto{\pgfqpoint{3.285295in}{2.820509in}}{\pgfqpoint{3.282022in}{2.812609in}}{\pgfqpoint{3.282022in}{2.804373in}}%
\pgfpathcurveto{\pgfqpoint{3.282022in}{2.796137in}}{\pgfqpoint{3.285295in}{2.788237in}}{\pgfqpoint{3.291119in}{2.782413in}}%
\pgfpathcurveto{\pgfqpoint{3.296943in}{2.776589in}}{\pgfqpoint{3.304843in}{2.773316in}}{\pgfqpoint{3.313079in}{2.773316in}}%
\pgfpathclose%
\pgfusepath{stroke,fill}%
\end{pgfscope}%
\begin{pgfscope}%
\pgfpathrectangle{\pgfqpoint{0.100000in}{0.220728in}}{\pgfqpoint{3.696000in}{3.696000in}}%
\pgfusepath{clip}%
\pgfsetbuttcap%
\pgfsetroundjoin%
\definecolor{currentfill}{rgb}{0.121569,0.466667,0.705882}%
\pgfsetfillcolor{currentfill}%
\pgfsetfillopacity{0.698044}%
\pgfsetlinewidth{1.003750pt}%
\definecolor{currentstroke}{rgb}{0.121569,0.466667,0.705882}%
\pgfsetstrokecolor{currentstroke}%
\pgfsetstrokeopacity{0.698044}%
\pgfsetdash{}{0pt}%
\pgfpathmoveto{\pgfqpoint{3.312528in}{2.772564in}}%
\pgfpathcurveto{\pgfqpoint{3.320764in}{2.772564in}}{\pgfqpoint{3.328665in}{2.775837in}}{\pgfqpoint{3.334488in}{2.781661in}}%
\pgfpathcurveto{\pgfqpoint{3.340312in}{2.787485in}}{\pgfqpoint{3.343585in}{2.795385in}}{\pgfqpoint{3.343585in}{2.803621in}}%
\pgfpathcurveto{\pgfqpoint{3.343585in}{2.811857in}}{\pgfqpoint{3.340312in}{2.819757in}}{\pgfqpoint{3.334488in}{2.825581in}}%
\pgfpathcurveto{\pgfqpoint{3.328665in}{2.831405in}}{\pgfqpoint{3.320764in}{2.834677in}}{\pgfqpoint{3.312528in}{2.834677in}}%
\pgfpathcurveto{\pgfqpoint{3.304292in}{2.834677in}}{\pgfqpoint{3.296392in}{2.831405in}}{\pgfqpoint{3.290568in}{2.825581in}}%
\pgfpathcurveto{\pgfqpoint{3.284744in}{2.819757in}}{\pgfqpoint{3.281472in}{2.811857in}}{\pgfqpoint{3.281472in}{2.803621in}}%
\pgfpathcurveto{\pgfqpoint{3.281472in}{2.795385in}}{\pgfqpoint{3.284744in}{2.787485in}}{\pgfqpoint{3.290568in}{2.781661in}}%
\pgfpathcurveto{\pgfqpoint{3.296392in}{2.775837in}}{\pgfqpoint{3.304292in}{2.772564in}}{\pgfqpoint{3.312528in}{2.772564in}}%
\pgfpathclose%
\pgfusepath{stroke,fill}%
\end{pgfscope}%
\begin{pgfscope}%
\pgfpathrectangle{\pgfqpoint{0.100000in}{0.220728in}}{\pgfqpoint{3.696000in}{3.696000in}}%
\pgfusepath{clip}%
\pgfsetbuttcap%
\pgfsetroundjoin%
\definecolor{currentfill}{rgb}{0.121569,0.466667,0.705882}%
\pgfsetfillcolor{currentfill}%
\pgfsetfillopacity{0.698349}%
\pgfsetlinewidth{1.003750pt}%
\definecolor{currentstroke}{rgb}{0.121569,0.466667,0.705882}%
\pgfsetstrokecolor{currentstroke}%
\pgfsetstrokeopacity{0.698349}%
\pgfsetdash{}{0pt}%
\pgfpathmoveto{\pgfqpoint{3.311618in}{2.770312in}}%
\pgfpathcurveto{\pgfqpoint{3.319854in}{2.770312in}}{\pgfqpoint{3.327754in}{2.773584in}}{\pgfqpoint{3.333578in}{2.779408in}}%
\pgfpathcurveto{\pgfqpoint{3.339402in}{2.785232in}}{\pgfqpoint{3.342675in}{2.793132in}}{\pgfqpoint{3.342675in}{2.801369in}}%
\pgfpathcurveto{\pgfqpoint{3.342675in}{2.809605in}}{\pgfqpoint{3.339402in}{2.817505in}}{\pgfqpoint{3.333578in}{2.823329in}}%
\pgfpathcurveto{\pgfqpoint{3.327754in}{2.829153in}}{\pgfqpoint{3.319854in}{2.832425in}}{\pgfqpoint{3.311618in}{2.832425in}}%
\pgfpathcurveto{\pgfqpoint{3.303382in}{2.832425in}}{\pgfqpoint{3.295482in}{2.829153in}}{\pgfqpoint{3.289658in}{2.823329in}}%
\pgfpathcurveto{\pgfqpoint{3.283834in}{2.817505in}}{\pgfqpoint{3.280562in}{2.809605in}}{\pgfqpoint{3.280562in}{2.801369in}}%
\pgfpathcurveto{\pgfqpoint{3.280562in}{2.793132in}}{\pgfqpoint{3.283834in}{2.785232in}}{\pgfqpoint{3.289658in}{2.779408in}}%
\pgfpathcurveto{\pgfqpoint{3.295482in}{2.773584in}}{\pgfqpoint{3.303382in}{2.770312in}}{\pgfqpoint{3.311618in}{2.770312in}}%
\pgfpathclose%
\pgfusepath{stroke,fill}%
\end{pgfscope}%
\begin{pgfscope}%
\pgfpathrectangle{\pgfqpoint{0.100000in}{0.220728in}}{\pgfqpoint{3.696000in}{3.696000in}}%
\pgfusepath{clip}%
\pgfsetbuttcap%
\pgfsetroundjoin%
\definecolor{currentfill}{rgb}{0.121569,0.466667,0.705882}%
\pgfsetfillcolor{currentfill}%
\pgfsetfillopacity{0.698828}%
\pgfsetlinewidth{1.003750pt}%
\definecolor{currentstroke}{rgb}{0.121569,0.466667,0.705882}%
\pgfsetstrokecolor{currentstroke}%
\pgfsetstrokeopacity{0.698828}%
\pgfsetdash{}{0pt}%
\pgfpathmoveto{\pgfqpoint{3.309615in}{2.767164in}}%
\pgfpathcurveto{\pgfqpoint{3.317851in}{2.767164in}}{\pgfqpoint{3.325752in}{2.770436in}}{\pgfqpoint{3.331575in}{2.776260in}}%
\pgfpathcurveto{\pgfqpoint{3.337399in}{2.782084in}}{\pgfqpoint{3.340672in}{2.789984in}}{\pgfqpoint{3.340672in}{2.798220in}}%
\pgfpathcurveto{\pgfqpoint{3.340672in}{2.806456in}}{\pgfqpoint{3.337399in}{2.814356in}}{\pgfqpoint{3.331575in}{2.820180in}}%
\pgfpathcurveto{\pgfqpoint{3.325752in}{2.826004in}}{\pgfqpoint{3.317851in}{2.829277in}}{\pgfqpoint{3.309615in}{2.829277in}}%
\pgfpathcurveto{\pgfqpoint{3.301379in}{2.829277in}}{\pgfqpoint{3.293479in}{2.826004in}}{\pgfqpoint{3.287655in}{2.820180in}}%
\pgfpathcurveto{\pgfqpoint{3.281831in}{2.814356in}}{\pgfqpoint{3.278559in}{2.806456in}}{\pgfqpoint{3.278559in}{2.798220in}}%
\pgfpathcurveto{\pgfqpoint{3.278559in}{2.789984in}}{\pgfqpoint{3.281831in}{2.782084in}}{\pgfqpoint{3.287655in}{2.776260in}}%
\pgfpathcurveto{\pgfqpoint{3.293479in}{2.770436in}}{\pgfqpoint{3.301379in}{2.767164in}}{\pgfqpoint{3.309615in}{2.767164in}}%
\pgfpathclose%
\pgfusepath{stroke,fill}%
\end{pgfscope}%
\begin{pgfscope}%
\pgfpathrectangle{\pgfqpoint{0.100000in}{0.220728in}}{\pgfqpoint{3.696000in}{3.696000in}}%
\pgfusepath{clip}%
\pgfsetbuttcap%
\pgfsetroundjoin%
\definecolor{currentfill}{rgb}{0.121569,0.466667,0.705882}%
\pgfsetfillcolor{currentfill}%
\pgfsetfillopacity{0.699130}%
\pgfsetlinewidth{1.003750pt}%
\definecolor{currentstroke}{rgb}{0.121569,0.466667,0.705882}%
\pgfsetstrokecolor{currentstroke}%
\pgfsetstrokeopacity{0.699130}%
\pgfsetdash{}{0pt}%
\pgfpathmoveto{\pgfqpoint{3.308587in}{2.765487in}}%
\pgfpathcurveto{\pgfqpoint{3.316823in}{2.765487in}}{\pgfqpoint{3.324723in}{2.768760in}}{\pgfqpoint{3.330547in}{2.774583in}}%
\pgfpathcurveto{\pgfqpoint{3.336371in}{2.780407in}}{\pgfqpoint{3.339643in}{2.788307in}}{\pgfqpoint{3.339643in}{2.796544in}}%
\pgfpathcurveto{\pgfqpoint{3.339643in}{2.804780in}}{\pgfqpoint{3.336371in}{2.812680in}}{\pgfqpoint{3.330547in}{2.818504in}}%
\pgfpathcurveto{\pgfqpoint{3.324723in}{2.824328in}}{\pgfqpoint{3.316823in}{2.827600in}}{\pgfqpoint{3.308587in}{2.827600in}}%
\pgfpathcurveto{\pgfqpoint{3.300350in}{2.827600in}}{\pgfqpoint{3.292450in}{2.824328in}}{\pgfqpoint{3.286626in}{2.818504in}}%
\pgfpathcurveto{\pgfqpoint{3.280802in}{2.812680in}}{\pgfqpoint{3.277530in}{2.804780in}}{\pgfqpoint{3.277530in}{2.796544in}}%
\pgfpathcurveto{\pgfqpoint{3.277530in}{2.788307in}}{\pgfqpoint{3.280802in}{2.780407in}}{\pgfqpoint{3.286626in}{2.774583in}}%
\pgfpathcurveto{\pgfqpoint{3.292450in}{2.768760in}}{\pgfqpoint{3.300350in}{2.765487in}}{\pgfqpoint{3.308587in}{2.765487in}}%
\pgfpathclose%
\pgfusepath{stroke,fill}%
\end{pgfscope}%
\begin{pgfscope}%
\pgfpathrectangle{\pgfqpoint{0.100000in}{0.220728in}}{\pgfqpoint{3.696000in}{3.696000in}}%
\pgfusepath{clip}%
\pgfsetbuttcap%
\pgfsetroundjoin%
\definecolor{currentfill}{rgb}{0.121569,0.466667,0.705882}%
\pgfsetfillcolor{currentfill}%
\pgfsetfillopacity{0.699309}%
\pgfsetlinewidth{1.003750pt}%
\definecolor{currentstroke}{rgb}{0.121569,0.466667,0.705882}%
\pgfsetstrokecolor{currentstroke}%
\pgfsetstrokeopacity{0.699309}%
\pgfsetdash{}{0pt}%
\pgfpathmoveto{\pgfqpoint{3.308131in}{2.764475in}}%
\pgfpathcurveto{\pgfqpoint{3.316367in}{2.764475in}}{\pgfqpoint{3.324267in}{2.767747in}}{\pgfqpoint{3.330091in}{2.773571in}}%
\pgfpathcurveto{\pgfqpoint{3.335915in}{2.779395in}}{\pgfqpoint{3.339187in}{2.787295in}}{\pgfqpoint{3.339187in}{2.795532in}}%
\pgfpathcurveto{\pgfqpoint{3.339187in}{2.803768in}}{\pgfqpoint{3.335915in}{2.811668in}}{\pgfqpoint{3.330091in}{2.817492in}}%
\pgfpathcurveto{\pgfqpoint{3.324267in}{2.823316in}}{\pgfqpoint{3.316367in}{2.826588in}}{\pgfqpoint{3.308131in}{2.826588in}}%
\pgfpathcurveto{\pgfqpoint{3.299894in}{2.826588in}}{\pgfqpoint{3.291994in}{2.823316in}}{\pgfqpoint{3.286170in}{2.817492in}}%
\pgfpathcurveto{\pgfqpoint{3.280346in}{2.811668in}}{\pgfqpoint{3.277074in}{2.803768in}}{\pgfqpoint{3.277074in}{2.795532in}}%
\pgfpathcurveto{\pgfqpoint{3.277074in}{2.787295in}}{\pgfqpoint{3.280346in}{2.779395in}}{\pgfqpoint{3.286170in}{2.773571in}}%
\pgfpathcurveto{\pgfqpoint{3.291994in}{2.767747in}}{\pgfqpoint{3.299894in}{2.764475in}}{\pgfqpoint{3.308131in}{2.764475in}}%
\pgfpathclose%
\pgfusepath{stroke,fill}%
\end{pgfscope}%
\begin{pgfscope}%
\pgfpathrectangle{\pgfqpoint{0.100000in}{0.220728in}}{\pgfqpoint{3.696000in}{3.696000in}}%
\pgfusepath{clip}%
\pgfsetbuttcap%
\pgfsetroundjoin%
\definecolor{currentfill}{rgb}{0.121569,0.466667,0.705882}%
\pgfsetfillcolor{currentfill}%
\pgfsetfillopacity{0.699616}%
\pgfsetlinewidth{1.003750pt}%
\definecolor{currentstroke}{rgb}{0.121569,0.466667,0.705882}%
\pgfsetstrokecolor{currentstroke}%
\pgfsetstrokeopacity{0.699616}%
\pgfsetdash{}{0pt}%
\pgfpathmoveto{\pgfqpoint{3.306785in}{2.762551in}}%
\pgfpathcurveto{\pgfqpoint{3.315021in}{2.762551in}}{\pgfqpoint{3.322921in}{2.765823in}}{\pgfqpoint{3.328745in}{2.771647in}}%
\pgfpathcurveto{\pgfqpoint{3.334569in}{2.777471in}}{\pgfqpoint{3.337841in}{2.785371in}}{\pgfqpoint{3.337841in}{2.793608in}}%
\pgfpathcurveto{\pgfqpoint{3.337841in}{2.801844in}}{\pgfqpoint{3.334569in}{2.809744in}}{\pgfqpoint{3.328745in}{2.815568in}}%
\pgfpathcurveto{\pgfqpoint{3.322921in}{2.821392in}}{\pgfqpoint{3.315021in}{2.824664in}}{\pgfqpoint{3.306785in}{2.824664in}}%
\pgfpathcurveto{\pgfqpoint{3.298548in}{2.824664in}}{\pgfqpoint{3.290648in}{2.821392in}}{\pgfqpoint{3.284824in}{2.815568in}}%
\pgfpathcurveto{\pgfqpoint{3.279000in}{2.809744in}}{\pgfqpoint{3.275728in}{2.801844in}}{\pgfqpoint{3.275728in}{2.793608in}}%
\pgfpathcurveto{\pgfqpoint{3.275728in}{2.785371in}}{\pgfqpoint{3.279000in}{2.777471in}}{\pgfqpoint{3.284824in}{2.771647in}}%
\pgfpathcurveto{\pgfqpoint{3.290648in}{2.765823in}}{\pgfqpoint{3.298548in}{2.762551in}}{\pgfqpoint{3.306785in}{2.762551in}}%
\pgfpathclose%
\pgfusepath{stroke,fill}%
\end{pgfscope}%
\begin{pgfscope}%
\pgfpathrectangle{\pgfqpoint{0.100000in}{0.220728in}}{\pgfqpoint{3.696000in}{3.696000in}}%
\pgfusepath{clip}%
\pgfsetbuttcap%
\pgfsetroundjoin%
\definecolor{currentfill}{rgb}{0.121569,0.466667,0.705882}%
\pgfsetfillcolor{currentfill}%
\pgfsetfillopacity{0.699737}%
\pgfsetlinewidth{1.003750pt}%
\definecolor{currentstroke}{rgb}{0.121569,0.466667,0.705882}%
\pgfsetstrokecolor{currentstroke}%
\pgfsetstrokeopacity{0.699737}%
\pgfsetdash{}{0pt}%
\pgfpathmoveto{\pgfqpoint{0.828118in}{1.355636in}}%
\pgfpathcurveto{\pgfqpoint{0.836354in}{1.355636in}}{\pgfqpoint{0.844254in}{1.358908in}}{\pgfqpoint{0.850078in}{1.364732in}}%
\pgfpathcurveto{\pgfqpoint{0.855902in}{1.370556in}}{\pgfqpoint{0.859174in}{1.378456in}}{\pgfqpoint{0.859174in}{1.386692in}}%
\pgfpathcurveto{\pgfqpoint{0.859174in}{1.394929in}}{\pgfqpoint{0.855902in}{1.402829in}}{\pgfqpoint{0.850078in}{1.408653in}}%
\pgfpathcurveto{\pgfqpoint{0.844254in}{1.414477in}}{\pgfqpoint{0.836354in}{1.417749in}}{\pgfqpoint{0.828118in}{1.417749in}}%
\pgfpathcurveto{\pgfqpoint{0.819881in}{1.417749in}}{\pgfqpoint{0.811981in}{1.414477in}}{\pgfqpoint{0.806157in}{1.408653in}}%
\pgfpathcurveto{\pgfqpoint{0.800334in}{1.402829in}}{\pgfqpoint{0.797061in}{1.394929in}}{\pgfqpoint{0.797061in}{1.386692in}}%
\pgfpathcurveto{\pgfqpoint{0.797061in}{1.378456in}}{\pgfqpoint{0.800334in}{1.370556in}}{\pgfqpoint{0.806157in}{1.364732in}}%
\pgfpathcurveto{\pgfqpoint{0.811981in}{1.358908in}}{\pgfqpoint{0.819881in}{1.355636in}}{\pgfqpoint{0.828118in}{1.355636in}}%
\pgfpathclose%
\pgfusepath{stroke,fill}%
\end{pgfscope}%
\begin{pgfscope}%
\pgfpathrectangle{\pgfqpoint{0.100000in}{0.220728in}}{\pgfqpoint{3.696000in}{3.696000in}}%
\pgfusepath{clip}%
\pgfsetbuttcap%
\pgfsetroundjoin%
\definecolor{currentfill}{rgb}{0.121569,0.466667,0.705882}%
\pgfsetfillcolor{currentfill}%
\pgfsetfillopacity{0.700334}%
\pgfsetlinewidth{1.003750pt}%
\definecolor{currentstroke}{rgb}{0.121569,0.466667,0.705882}%
\pgfsetstrokecolor{currentstroke}%
\pgfsetstrokeopacity{0.700334}%
\pgfsetdash{}{0pt}%
\pgfpathmoveto{\pgfqpoint{3.305391in}{2.758513in}}%
\pgfpathcurveto{\pgfqpoint{3.313627in}{2.758513in}}{\pgfqpoint{3.321527in}{2.761785in}}{\pgfqpoint{3.327351in}{2.767609in}}%
\pgfpathcurveto{\pgfqpoint{3.333175in}{2.773433in}}{\pgfqpoint{3.336447in}{2.781333in}}{\pgfqpoint{3.336447in}{2.789569in}}%
\pgfpathcurveto{\pgfqpoint{3.336447in}{2.797805in}}{\pgfqpoint{3.333175in}{2.805705in}}{\pgfqpoint{3.327351in}{2.811529in}}%
\pgfpathcurveto{\pgfqpoint{3.321527in}{2.817353in}}{\pgfqpoint{3.313627in}{2.820626in}}{\pgfqpoint{3.305391in}{2.820626in}}%
\pgfpathcurveto{\pgfqpoint{3.297154in}{2.820626in}}{\pgfqpoint{3.289254in}{2.817353in}}{\pgfqpoint{3.283430in}{2.811529in}}%
\pgfpathcurveto{\pgfqpoint{3.277607in}{2.805705in}}{\pgfqpoint{3.274334in}{2.797805in}}{\pgfqpoint{3.274334in}{2.789569in}}%
\pgfpathcurveto{\pgfqpoint{3.274334in}{2.781333in}}{\pgfqpoint{3.277607in}{2.773433in}}{\pgfqpoint{3.283430in}{2.767609in}}%
\pgfpathcurveto{\pgfqpoint{3.289254in}{2.761785in}}{\pgfqpoint{3.297154in}{2.758513in}}{\pgfqpoint{3.305391in}{2.758513in}}%
\pgfpathclose%
\pgfusepath{stroke,fill}%
\end{pgfscope}%
\begin{pgfscope}%
\pgfpathrectangle{\pgfqpoint{0.100000in}{0.220728in}}{\pgfqpoint{3.696000in}{3.696000in}}%
\pgfusepath{clip}%
\pgfsetbuttcap%
\pgfsetroundjoin%
\definecolor{currentfill}{rgb}{0.121569,0.466667,0.705882}%
\pgfsetfillcolor{currentfill}%
\pgfsetfillopacity{0.701064}%
\pgfsetlinewidth{1.003750pt}%
\definecolor{currentstroke}{rgb}{0.121569,0.466667,0.705882}%
\pgfsetstrokecolor{currentstroke}%
\pgfsetstrokeopacity{0.701064}%
\pgfsetdash{}{0pt}%
\pgfpathmoveto{\pgfqpoint{3.302593in}{2.754070in}}%
\pgfpathcurveto{\pgfqpoint{3.310829in}{2.754070in}}{\pgfqpoint{3.318729in}{2.757342in}}{\pgfqpoint{3.324553in}{2.763166in}}%
\pgfpathcurveto{\pgfqpoint{3.330377in}{2.768990in}}{\pgfqpoint{3.333650in}{2.776890in}}{\pgfqpoint{3.333650in}{2.785126in}}%
\pgfpathcurveto{\pgfqpoint{3.333650in}{2.793363in}}{\pgfqpoint{3.330377in}{2.801263in}}{\pgfqpoint{3.324553in}{2.807087in}}%
\pgfpathcurveto{\pgfqpoint{3.318729in}{2.812910in}}{\pgfqpoint{3.310829in}{2.816183in}}{\pgfqpoint{3.302593in}{2.816183in}}%
\pgfpathcurveto{\pgfqpoint{3.294357in}{2.816183in}}{\pgfqpoint{3.286457in}{2.812910in}}{\pgfqpoint{3.280633in}{2.807087in}}%
\pgfpathcurveto{\pgfqpoint{3.274809in}{2.801263in}}{\pgfqpoint{3.271537in}{2.793363in}}{\pgfqpoint{3.271537in}{2.785126in}}%
\pgfpathcurveto{\pgfqpoint{3.271537in}{2.776890in}}{\pgfqpoint{3.274809in}{2.768990in}}{\pgfqpoint{3.280633in}{2.763166in}}%
\pgfpathcurveto{\pgfqpoint{3.286457in}{2.757342in}}{\pgfqpoint{3.294357in}{2.754070in}}{\pgfqpoint{3.302593in}{2.754070in}}%
\pgfpathclose%
\pgfusepath{stroke,fill}%
\end{pgfscope}%
\begin{pgfscope}%
\pgfpathrectangle{\pgfqpoint{0.100000in}{0.220728in}}{\pgfqpoint{3.696000in}{3.696000in}}%
\pgfusepath{clip}%
\pgfsetbuttcap%
\pgfsetroundjoin%
\definecolor{currentfill}{rgb}{0.121569,0.466667,0.705882}%
\pgfsetfillcolor{currentfill}%
\pgfsetfillopacity{0.701462}%
\pgfsetlinewidth{1.003750pt}%
\definecolor{currentstroke}{rgb}{0.121569,0.466667,0.705882}%
\pgfsetstrokecolor{currentstroke}%
\pgfsetstrokeopacity{0.701462}%
\pgfsetdash{}{0pt}%
\pgfpathmoveto{\pgfqpoint{3.300995in}{2.751710in}}%
\pgfpathcurveto{\pgfqpoint{3.309231in}{2.751710in}}{\pgfqpoint{3.317131in}{2.754983in}}{\pgfqpoint{3.322955in}{2.760806in}}%
\pgfpathcurveto{\pgfqpoint{3.328779in}{2.766630in}}{\pgfqpoint{3.332052in}{2.774530in}}{\pgfqpoint{3.332052in}{2.782767in}}%
\pgfpathcurveto{\pgfqpoint{3.332052in}{2.791003in}}{\pgfqpoint{3.328779in}{2.798903in}}{\pgfqpoint{3.322955in}{2.804727in}}%
\pgfpathcurveto{\pgfqpoint{3.317131in}{2.810551in}}{\pgfqpoint{3.309231in}{2.813823in}}{\pgfqpoint{3.300995in}{2.813823in}}%
\pgfpathcurveto{\pgfqpoint{3.292759in}{2.813823in}}{\pgfqpoint{3.284859in}{2.810551in}}{\pgfqpoint{3.279035in}{2.804727in}}%
\pgfpathcurveto{\pgfqpoint{3.273211in}{2.798903in}}{\pgfqpoint{3.269939in}{2.791003in}}{\pgfqpoint{3.269939in}{2.782767in}}%
\pgfpathcurveto{\pgfqpoint{3.269939in}{2.774530in}}{\pgfqpoint{3.273211in}{2.766630in}}{\pgfqpoint{3.279035in}{2.760806in}}%
\pgfpathcurveto{\pgfqpoint{3.284859in}{2.754983in}}{\pgfqpoint{3.292759in}{2.751710in}}{\pgfqpoint{3.300995in}{2.751710in}}%
\pgfpathclose%
\pgfusepath{stroke,fill}%
\end{pgfscope}%
\begin{pgfscope}%
\pgfpathrectangle{\pgfqpoint{0.100000in}{0.220728in}}{\pgfqpoint{3.696000in}{3.696000in}}%
\pgfusepath{clip}%
\pgfsetbuttcap%
\pgfsetroundjoin%
\definecolor{currentfill}{rgb}{0.121569,0.466667,0.705882}%
\pgfsetfillcolor{currentfill}%
\pgfsetfillopacity{0.701710}%
\pgfsetlinewidth{1.003750pt}%
\definecolor{currentstroke}{rgb}{0.121569,0.466667,0.705882}%
\pgfsetstrokecolor{currentstroke}%
\pgfsetstrokeopacity{0.701710}%
\pgfsetdash{}{0pt}%
\pgfpathmoveto{\pgfqpoint{3.300395in}{2.750168in}}%
\pgfpathcurveto{\pgfqpoint{3.308632in}{2.750168in}}{\pgfqpoint{3.316532in}{2.753440in}}{\pgfqpoint{3.322356in}{2.759264in}}%
\pgfpathcurveto{\pgfqpoint{3.328180in}{2.765088in}}{\pgfqpoint{3.331452in}{2.772988in}}{\pgfqpoint{3.331452in}{2.781224in}}%
\pgfpathcurveto{\pgfqpoint{3.331452in}{2.789461in}}{\pgfqpoint{3.328180in}{2.797361in}}{\pgfqpoint{3.322356in}{2.803185in}}%
\pgfpathcurveto{\pgfqpoint{3.316532in}{2.809009in}}{\pgfqpoint{3.308632in}{2.812281in}}{\pgfqpoint{3.300395in}{2.812281in}}%
\pgfpathcurveto{\pgfqpoint{3.292159in}{2.812281in}}{\pgfqpoint{3.284259in}{2.809009in}}{\pgfqpoint{3.278435in}{2.803185in}}%
\pgfpathcurveto{\pgfqpoint{3.272611in}{2.797361in}}{\pgfqpoint{3.269339in}{2.789461in}}{\pgfqpoint{3.269339in}{2.781224in}}%
\pgfpathcurveto{\pgfqpoint{3.269339in}{2.772988in}}{\pgfqpoint{3.272611in}{2.765088in}}{\pgfqpoint{3.278435in}{2.759264in}}%
\pgfpathcurveto{\pgfqpoint{3.284259in}{2.753440in}}{\pgfqpoint{3.292159in}{2.750168in}}{\pgfqpoint{3.300395in}{2.750168in}}%
\pgfpathclose%
\pgfusepath{stroke,fill}%
\end{pgfscope}%
\begin{pgfscope}%
\pgfpathrectangle{\pgfqpoint{0.100000in}{0.220728in}}{\pgfqpoint{3.696000in}{3.696000in}}%
\pgfusepath{clip}%
\pgfsetbuttcap%
\pgfsetroundjoin%
\definecolor{currentfill}{rgb}{0.121569,0.466667,0.705882}%
\pgfsetfillcolor{currentfill}%
\pgfsetfillopacity{0.701979}%
\pgfsetlinewidth{1.003750pt}%
\definecolor{currentstroke}{rgb}{0.121569,0.466667,0.705882}%
\pgfsetstrokecolor{currentstroke}%
\pgfsetstrokeopacity{0.701979}%
\pgfsetdash{}{0pt}%
\pgfpathmoveto{\pgfqpoint{0.837952in}{1.350904in}}%
\pgfpathcurveto{\pgfqpoint{0.846188in}{1.350904in}}{\pgfqpoint{0.854088in}{1.354176in}}{\pgfqpoint{0.859912in}{1.360000in}}%
\pgfpathcurveto{\pgfqpoint{0.865736in}{1.365824in}}{\pgfqpoint{0.869008in}{1.373724in}}{\pgfqpoint{0.869008in}{1.381961in}}%
\pgfpathcurveto{\pgfqpoint{0.869008in}{1.390197in}}{\pgfqpoint{0.865736in}{1.398097in}}{\pgfqpoint{0.859912in}{1.403921in}}%
\pgfpathcurveto{\pgfqpoint{0.854088in}{1.409745in}}{\pgfqpoint{0.846188in}{1.413017in}}{\pgfqpoint{0.837952in}{1.413017in}}%
\pgfpathcurveto{\pgfqpoint{0.829716in}{1.413017in}}{\pgfqpoint{0.821816in}{1.409745in}}{\pgfqpoint{0.815992in}{1.403921in}}%
\pgfpathcurveto{\pgfqpoint{0.810168in}{1.398097in}}{\pgfqpoint{0.806895in}{1.390197in}}{\pgfqpoint{0.806895in}{1.381961in}}%
\pgfpathcurveto{\pgfqpoint{0.806895in}{1.373724in}}{\pgfqpoint{0.810168in}{1.365824in}}{\pgfqpoint{0.815992in}{1.360000in}}%
\pgfpathcurveto{\pgfqpoint{0.821816in}{1.354176in}}{\pgfqpoint{0.829716in}{1.350904in}}{\pgfqpoint{0.837952in}{1.350904in}}%
\pgfpathclose%
\pgfusepath{stroke,fill}%
\end{pgfscope}%
\begin{pgfscope}%
\pgfpathrectangle{\pgfqpoint{0.100000in}{0.220728in}}{\pgfqpoint{3.696000in}{3.696000in}}%
\pgfusepath{clip}%
\pgfsetbuttcap%
\pgfsetroundjoin%
\definecolor{currentfill}{rgb}{0.121569,0.466667,0.705882}%
\pgfsetfillcolor{currentfill}%
\pgfsetfillopacity{0.702196}%
\pgfsetlinewidth{1.003750pt}%
\definecolor{currentstroke}{rgb}{0.121569,0.466667,0.705882}%
\pgfsetstrokecolor{currentstroke}%
\pgfsetstrokeopacity{0.702196}%
\pgfsetdash{}{0pt}%
\pgfpathmoveto{\pgfqpoint{3.298414in}{2.747122in}}%
\pgfpathcurveto{\pgfqpoint{3.306650in}{2.747122in}}{\pgfqpoint{3.314550in}{2.750394in}}{\pgfqpoint{3.320374in}{2.756218in}}%
\pgfpathcurveto{\pgfqpoint{3.326198in}{2.762042in}}{\pgfqpoint{3.329470in}{2.769942in}}{\pgfqpoint{3.329470in}{2.778179in}}%
\pgfpathcurveto{\pgfqpoint{3.329470in}{2.786415in}}{\pgfqpoint{3.326198in}{2.794315in}}{\pgfqpoint{3.320374in}{2.800139in}}%
\pgfpathcurveto{\pgfqpoint{3.314550in}{2.805963in}}{\pgfqpoint{3.306650in}{2.809235in}}{\pgfqpoint{3.298414in}{2.809235in}}%
\pgfpathcurveto{\pgfqpoint{3.290177in}{2.809235in}}{\pgfqpoint{3.282277in}{2.805963in}}{\pgfqpoint{3.276454in}{2.800139in}}%
\pgfpathcurveto{\pgfqpoint{3.270630in}{2.794315in}}{\pgfqpoint{3.267357in}{2.786415in}}{\pgfqpoint{3.267357in}{2.778179in}}%
\pgfpathcurveto{\pgfqpoint{3.267357in}{2.769942in}}{\pgfqpoint{3.270630in}{2.762042in}}{\pgfqpoint{3.276454in}{2.756218in}}%
\pgfpathcurveto{\pgfqpoint{3.282277in}{2.750394in}}{\pgfqpoint{3.290177in}{2.747122in}}{\pgfqpoint{3.298414in}{2.747122in}}%
\pgfpathclose%
\pgfusepath{stroke,fill}%
\end{pgfscope}%
\begin{pgfscope}%
\pgfpathrectangle{\pgfqpoint{0.100000in}{0.220728in}}{\pgfqpoint{3.696000in}{3.696000in}}%
\pgfusepath{clip}%
\pgfsetbuttcap%
\pgfsetroundjoin%
\definecolor{currentfill}{rgb}{0.121569,0.466667,0.705882}%
\pgfsetfillcolor{currentfill}%
\pgfsetfillopacity{0.702485}%
\pgfsetlinewidth{1.003750pt}%
\definecolor{currentstroke}{rgb}{0.121569,0.466667,0.705882}%
\pgfsetstrokecolor{currentstroke}%
\pgfsetstrokeopacity{0.702485}%
\pgfsetdash{}{0pt}%
\pgfpathmoveto{\pgfqpoint{3.297491in}{2.745292in}}%
\pgfpathcurveto{\pgfqpoint{3.305727in}{2.745292in}}{\pgfqpoint{3.313627in}{2.748565in}}{\pgfqpoint{3.319451in}{2.754388in}}%
\pgfpathcurveto{\pgfqpoint{3.325275in}{2.760212in}}{\pgfqpoint{3.328547in}{2.768112in}}{\pgfqpoint{3.328547in}{2.776349in}}%
\pgfpathcurveto{\pgfqpoint{3.328547in}{2.784585in}}{\pgfqpoint{3.325275in}{2.792485in}}{\pgfqpoint{3.319451in}{2.798309in}}%
\pgfpathcurveto{\pgfqpoint{3.313627in}{2.804133in}}{\pgfqpoint{3.305727in}{2.807405in}}{\pgfqpoint{3.297491in}{2.807405in}}%
\pgfpathcurveto{\pgfqpoint{3.289255in}{2.807405in}}{\pgfqpoint{3.281354in}{2.804133in}}{\pgfqpoint{3.275531in}{2.798309in}}%
\pgfpathcurveto{\pgfqpoint{3.269707in}{2.792485in}}{\pgfqpoint{3.266434in}{2.784585in}}{\pgfqpoint{3.266434in}{2.776349in}}%
\pgfpathcurveto{\pgfqpoint{3.266434in}{2.768112in}}{\pgfqpoint{3.269707in}{2.760212in}}{\pgfqpoint{3.275531in}{2.754388in}}%
\pgfpathcurveto{\pgfqpoint{3.281354in}{2.748565in}}{\pgfqpoint{3.289255in}{2.745292in}}{\pgfqpoint{3.297491in}{2.745292in}}%
\pgfpathclose%
\pgfusepath{stroke,fill}%
\end{pgfscope}%
\begin{pgfscope}%
\pgfpathrectangle{\pgfqpoint{0.100000in}{0.220728in}}{\pgfqpoint{3.696000in}{3.696000in}}%
\pgfusepath{clip}%
\pgfsetbuttcap%
\pgfsetroundjoin%
\definecolor{currentfill}{rgb}{0.121569,0.466667,0.705882}%
\pgfsetfillcolor{currentfill}%
\pgfsetfillopacity{0.702680}%
\pgfsetlinewidth{1.003750pt}%
\definecolor{currentstroke}{rgb}{0.121569,0.466667,0.705882}%
\pgfsetstrokecolor{currentstroke}%
\pgfsetstrokeopacity{0.702680}%
\pgfsetdash{}{0pt}%
\pgfpathmoveto{\pgfqpoint{3.297064in}{2.744348in}}%
\pgfpathcurveto{\pgfqpoint{3.305300in}{2.744348in}}{\pgfqpoint{3.313200in}{2.747620in}}{\pgfqpoint{3.319024in}{2.753444in}}%
\pgfpathcurveto{\pgfqpoint{3.324848in}{2.759268in}}{\pgfqpoint{3.328120in}{2.767168in}}{\pgfqpoint{3.328120in}{2.775405in}}%
\pgfpathcurveto{\pgfqpoint{3.328120in}{2.783641in}}{\pgfqpoint{3.324848in}{2.791541in}}{\pgfqpoint{3.319024in}{2.797365in}}%
\pgfpathcurveto{\pgfqpoint{3.313200in}{2.803189in}}{\pgfqpoint{3.305300in}{2.806461in}}{\pgfqpoint{3.297064in}{2.806461in}}%
\pgfpathcurveto{\pgfqpoint{3.288827in}{2.806461in}}{\pgfqpoint{3.280927in}{2.803189in}}{\pgfqpoint{3.275103in}{2.797365in}}%
\pgfpathcurveto{\pgfqpoint{3.269279in}{2.791541in}}{\pgfqpoint{3.266007in}{2.783641in}}{\pgfqpoint{3.266007in}{2.775405in}}%
\pgfpathcurveto{\pgfqpoint{3.266007in}{2.767168in}}{\pgfqpoint{3.269279in}{2.759268in}}{\pgfqpoint{3.275103in}{2.753444in}}%
\pgfpathcurveto{\pgfqpoint{3.280927in}{2.747620in}}{\pgfqpoint{3.288827in}{2.744348in}}{\pgfqpoint{3.297064in}{2.744348in}}%
\pgfpathclose%
\pgfusepath{stroke,fill}%
\end{pgfscope}%
\begin{pgfscope}%
\pgfpathrectangle{\pgfqpoint{0.100000in}{0.220728in}}{\pgfqpoint{3.696000in}{3.696000in}}%
\pgfusepath{clip}%
\pgfsetbuttcap%
\pgfsetroundjoin%
\definecolor{currentfill}{rgb}{0.121569,0.466667,0.705882}%
\pgfsetfillcolor{currentfill}%
\pgfsetfillopacity{0.702760}%
\pgfsetlinewidth{1.003750pt}%
\definecolor{currentstroke}{rgb}{0.121569,0.466667,0.705882}%
\pgfsetstrokecolor{currentstroke}%
\pgfsetstrokeopacity{0.702760}%
\pgfsetdash{}{0pt}%
\pgfpathmoveto{\pgfqpoint{3.296726in}{2.743849in}}%
\pgfpathcurveto{\pgfqpoint{3.304962in}{2.743849in}}{\pgfqpoint{3.312862in}{2.747121in}}{\pgfqpoint{3.318686in}{2.752945in}}%
\pgfpathcurveto{\pgfqpoint{3.324510in}{2.758769in}}{\pgfqpoint{3.327783in}{2.766669in}}{\pgfqpoint{3.327783in}{2.774906in}}%
\pgfpathcurveto{\pgfqpoint{3.327783in}{2.783142in}}{\pgfqpoint{3.324510in}{2.791042in}}{\pgfqpoint{3.318686in}{2.796866in}}%
\pgfpathcurveto{\pgfqpoint{3.312862in}{2.802690in}}{\pgfqpoint{3.304962in}{2.805962in}}{\pgfqpoint{3.296726in}{2.805962in}}%
\pgfpathcurveto{\pgfqpoint{3.288490in}{2.805962in}}{\pgfqpoint{3.280590in}{2.802690in}}{\pgfqpoint{3.274766in}{2.796866in}}%
\pgfpathcurveto{\pgfqpoint{3.268942in}{2.791042in}}{\pgfqpoint{3.265670in}{2.783142in}}{\pgfqpoint{3.265670in}{2.774906in}}%
\pgfpathcurveto{\pgfqpoint{3.265670in}{2.766669in}}{\pgfqpoint{3.268942in}{2.758769in}}{\pgfqpoint{3.274766in}{2.752945in}}%
\pgfpathcurveto{\pgfqpoint{3.280590in}{2.747121in}}{\pgfqpoint{3.288490in}{2.743849in}}{\pgfqpoint{3.296726in}{2.743849in}}%
\pgfpathclose%
\pgfusepath{stroke,fill}%
\end{pgfscope}%
\begin{pgfscope}%
\pgfpathrectangle{\pgfqpoint{0.100000in}{0.220728in}}{\pgfqpoint{3.696000in}{3.696000in}}%
\pgfusepath{clip}%
\pgfsetbuttcap%
\pgfsetroundjoin%
\definecolor{currentfill}{rgb}{0.121569,0.466667,0.705882}%
\pgfsetfillcolor{currentfill}%
\pgfsetfillopacity{0.703303}%
\pgfsetlinewidth{1.003750pt}%
\definecolor{currentstroke}{rgb}{0.121569,0.466667,0.705882}%
\pgfsetstrokecolor{currentstroke}%
\pgfsetstrokeopacity{0.703303}%
\pgfsetdash{}{0pt}%
\pgfpathmoveto{\pgfqpoint{3.295447in}{2.740333in}}%
\pgfpathcurveto{\pgfqpoint{3.303683in}{2.740333in}}{\pgfqpoint{3.311583in}{2.743605in}}{\pgfqpoint{3.317407in}{2.749429in}}%
\pgfpathcurveto{\pgfqpoint{3.323231in}{2.755253in}}{\pgfqpoint{3.326504in}{2.763153in}}{\pgfqpoint{3.326504in}{2.771389in}}%
\pgfpathcurveto{\pgfqpoint{3.326504in}{2.779626in}}{\pgfqpoint{3.323231in}{2.787526in}}{\pgfqpoint{3.317407in}{2.793350in}}%
\pgfpathcurveto{\pgfqpoint{3.311583in}{2.799174in}}{\pgfqpoint{3.303683in}{2.802446in}}{\pgfqpoint{3.295447in}{2.802446in}}%
\pgfpathcurveto{\pgfqpoint{3.287211in}{2.802446in}}{\pgfqpoint{3.279311in}{2.799174in}}{\pgfqpoint{3.273487in}{2.793350in}}%
\pgfpathcurveto{\pgfqpoint{3.267663in}{2.787526in}}{\pgfqpoint{3.264391in}{2.779626in}}{\pgfqpoint{3.264391in}{2.771389in}}%
\pgfpathcurveto{\pgfqpoint{3.264391in}{2.763153in}}{\pgfqpoint{3.267663in}{2.755253in}}{\pgfqpoint{3.273487in}{2.749429in}}%
\pgfpathcurveto{\pgfqpoint{3.279311in}{2.743605in}}{\pgfqpoint{3.287211in}{2.740333in}}{\pgfqpoint{3.295447in}{2.740333in}}%
\pgfpathclose%
\pgfusepath{stroke,fill}%
\end{pgfscope}%
\begin{pgfscope}%
\pgfpathrectangle{\pgfqpoint{0.100000in}{0.220728in}}{\pgfqpoint{3.696000in}{3.696000in}}%
\pgfusepath{clip}%
\pgfsetbuttcap%
\pgfsetroundjoin%
\definecolor{currentfill}{rgb}{0.121569,0.466667,0.705882}%
\pgfsetfillcolor{currentfill}%
\pgfsetfillopacity{0.703615}%
\pgfsetlinewidth{1.003750pt}%
\definecolor{currentstroke}{rgb}{0.121569,0.466667,0.705882}%
\pgfsetstrokecolor{currentstroke}%
\pgfsetstrokeopacity{0.703615}%
\pgfsetdash{}{0pt}%
\pgfpathmoveto{\pgfqpoint{3.294480in}{2.738768in}}%
\pgfpathcurveto{\pgfqpoint{3.302716in}{2.738768in}}{\pgfqpoint{3.310616in}{2.742041in}}{\pgfqpoint{3.316440in}{2.747864in}}%
\pgfpathcurveto{\pgfqpoint{3.322264in}{2.753688in}}{\pgfqpoint{3.325536in}{2.761588in}}{\pgfqpoint{3.325536in}{2.769825in}}%
\pgfpathcurveto{\pgfqpoint{3.325536in}{2.778061in}}{\pgfqpoint{3.322264in}{2.785961in}}{\pgfqpoint{3.316440in}{2.791785in}}%
\pgfpathcurveto{\pgfqpoint{3.310616in}{2.797609in}}{\pgfqpoint{3.302716in}{2.800881in}}{\pgfqpoint{3.294480in}{2.800881in}}%
\pgfpathcurveto{\pgfqpoint{3.286243in}{2.800881in}}{\pgfqpoint{3.278343in}{2.797609in}}{\pgfqpoint{3.272519in}{2.791785in}}%
\pgfpathcurveto{\pgfqpoint{3.266695in}{2.785961in}}{\pgfqpoint{3.263423in}{2.778061in}}{\pgfqpoint{3.263423in}{2.769825in}}%
\pgfpathcurveto{\pgfqpoint{3.263423in}{2.761588in}}{\pgfqpoint{3.266695in}{2.753688in}}{\pgfqpoint{3.272519in}{2.747864in}}%
\pgfpathcurveto{\pgfqpoint{3.278343in}{2.742041in}}{\pgfqpoint{3.286243in}{2.738768in}}{\pgfqpoint{3.294480in}{2.738768in}}%
\pgfpathclose%
\pgfusepath{stroke,fill}%
\end{pgfscope}%
\begin{pgfscope}%
\pgfpathrectangle{\pgfqpoint{0.100000in}{0.220728in}}{\pgfqpoint{3.696000in}{3.696000in}}%
\pgfusepath{clip}%
\pgfsetbuttcap%
\pgfsetroundjoin%
\definecolor{currentfill}{rgb}{0.121569,0.466667,0.705882}%
\pgfsetfillcolor{currentfill}%
\pgfsetfillopacity{0.703723}%
\pgfsetlinewidth{1.003750pt}%
\definecolor{currentstroke}{rgb}{0.121569,0.466667,0.705882}%
\pgfsetstrokecolor{currentstroke}%
\pgfsetstrokeopacity{0.703723}%
\pgfsetdash{}{0pt}%
\pgfpathmoveto{\pgfqpoint{3.293835in}{2.737822in}}%
\pgfpathcurveto{\pgfqpoint{3.302071in}{2.737822in}}{\pgfqpoint{3.309971in}{2.741094in}}{\pgfqpoint{3.315795in}{2.746918in}}%
\pgfpathcurveto{\pgfqpoint{3.321619in}{2.752742in}}{\pgfqpoint{3.324891in}{2.760642in}}{\pgfqpoint{3.324891in}{2.768878in}}%
\pgfpathcurveto{\pgfqpoint{3.324891in}{2.777115in}}{\pgfqpoint{3.321619in}{2.785015in}}{\pgfqpoint{3.315795in}{2.790839in}}%
\pgfpathcurveto{\pgfqpoint{3.309971in}{2.796662in}}{\pgfqpoint{3.302071in}{2.799935in}}{\pgfqpoint{3.293835in}{2.799935in}}%
\pgfpathcurveto{\pgfqpoint{3.285598in}{2.799935in}}{\pgfqpoint{3.277698in}{2.796662in}}{\pgfqpoint{3.271874in}{2.790839in}}%
\pgfpathcurveto{\pgfqpoint{3.266051in}{2.785015in}}{\pgfqpoint{3.262778in}{2.777115in}}{\pgfqpoint{3.262778in}{2.768878in}}%
\pgfpathcurveto{\pgfqpoint{3.262778in}{2.760642in}}{\pgfqpoint{3.266051in}{2.752742in}}{\pgfqpoint{3.271874in}{2.746918in}}%
\pgfpathcurveto{\pgfqpoint{3.277698in}{2.741094in}}{\pgfqpoint{3.285598in}{2.737822in}}{\pgfqpoint{3.293835in}{2.737822in}}%
\pgfpathclose%
\pgfusepath{stroke,fill}%
\end{pgfscope}%
\begin{pgfscope}%
\pgfpathrectangle{\pgfqpoint{0.100000in}{0.220728in}}{\pgfqpoint{3.696000in}{3.696000in}}%
\pgfusepath{clip}%
\pgfsetbuttcap%
\pgfsetroundjoin%
\definecolor{currentfill}{rgb}{0.121569,0.466667,0.705882}%
\pgfsetfillcolor{currentfill}%
\pgfsetfillopacity{0.703814}%
\pgfsetlinewidth{1.003750pt}%
\definecolor{currentstroke}{rgb}{0.121569,0.466667,0.705882}%
\pgfsetstrokecolor{currentstroke}%
\pgfsetstrokeopacity{0.703814}%
\pgfsetdash{}{0pt}%
\pgfpathmoveto{\pgfqpoint{3.293622in}{2.737238in}}%
\pgfpathcurveto{\pgfqpoint{3.301859in}{2.737238in}}{\pgfqpoint{3.309759in}{2.740510in}}{\pgfqpoint{3.315583in}{2.746334in}}%
\pgfpathcurveto{\pgfqpoint{3.321407in}{2.752158in}}{\pgfqpoint{3.324679in}{2.760058in}}{\pgfqpoint{3.324679in}{2.768294in}}%
\pgfpathcurveto{\pgfqpoint{3.324679in}{2.776531in}}{\pgfqpoint{3.321407in}{2.784431in}}{\pgfqpoint{3.315583in}{2.790255in}}%
\pgfpathcurveto{\pgfqpoint{3.309759in}{2.796079in}}{\pgfqpoint{3.301859in}{2.799351in}}{\pgfqpoint{3.293622in}{2.799351in}}%
\pgfpathcurveto{\pgfqpoint{3.285386in}{2.799351in}}{\pgfqpoint{3.277486in}{2.796079in}}{\pgfqpoint{3.271662in}{2.790255in}}%
\pgfpathcurveto{\pgfqpoint{3.265838in}{2.784431in}}{\pgfqpoint{3.262566in}{2.776531in}}{\pgfqpoint{3.262566in}{2.768294in}}%
\pgfpathcurveto{\pgfqpoint{3.262566in}{2.760058in}}{\pgfqpoint{3.265838in}{2.752158in}}{\pgfqpoint{3.271662in}{2.746334in}}%
\pgfpathcurveto{\pgfqpoint{3.277486in}{2.740510in}}{\pgfqpoint{3.285386in}{2.737238in}}{\pgfqpoint{3.293622in}{2.737238in}}%
\pgfpathclose%
\pgfusepath{stroke,fill}%
\end{pgfscope}%
\begin{pgfscope}%
\pgfpathrectangle{\pgfqpoint{0.100000in}{0.220728in}}{\pgfqpoint{3.696000in}{3.696000in}}%
\pgfusepath{clip}%
\pgfsetbuttcap%
\pgfsetroundjoin%
\definecolor{currentfill}{rgb}{0.121569,0.466667,0.705882}%
\pgfsetfillcolor{currentfill}%
\pgfsetfillopacity{0.704005}%
\pgfsetlinewidth{1.003750pt}%
\definecolor{currentstroke}{rgb}{0.121569,0.466667,0.705882}%
\pgfsetstrokecolor{currentstroke}%
\pgfsetstrokeopacity{0.704005}%
\pgfsetdash{}{0pt}%
\pgfpathmoveto{\pgfqpoint{0.847406in}{1.346977in}}%
\pgfpathcurveto{\pgfqpoint{0.855642in}{1.346977in}}{\pgfqpoint{0.863542in}{1.350250in}}{\pgfqpoint{0.869366in}{1.356074in}}%
\pgfpathcurveto{\pgfqpoint{0.875190in}{1.361897in}}{\pgfqpoint{0.878462in}{1.369798in}}{\pgfqpoint{0.878462in}{1.378034in}}%
\pgfpathcurveto{\pgfqpoint{0.878462in}{1.386270in}}{\pgfqpoint{0.875190in}{1.394170in}}{\pgfqpoint{0.869366in}{1.399994in}}%
\pgfpathcurveto{\pgfqpoint{0.863542in}{1.405818in}}{\pgfqpoint{0.855642in}{1.409090in}}{\pgfqpoint{0.847406in}{1.409090in}}%
\pgfpathcurveto{\pgfqpoint{0.839169in}{1.409090in}}{\pgfqpoint{0.831269in}{1.405818in}}{\pgfqpoint{0.825445in}{1.399994in}}%
\pgfpathcurveto{\pgfqpoint{0.819622in}{1.394170in}}{\pgfqpoint{0.816349in}{1.386270in}}{\pgfqpoint{0.816349in}{1.378034in}}%
\pgfpathcurveto{\pgfqpoint{0.816349in}{1.369798in}}{\pgfqpoint{0.819622in}{1.361897in}}{\pgfqpoint{0.825445in}{1.356074in}}%
\pgfpathcurveto{\pgfqpoint{0.831269in}{1.350250in}}{\pgfqpoint{0.839169in}{1.346977in}}{\pgfqpoint{0.847406in}{1.346977in}}%
\pgfpathclose%
\pgfusepath{stroke,fill}%
\end{pgfscope}%
\begin{pgfscope}%
\pgfpathrectangle{\pgfqpoint{0.100000in}{0.220728in}}{\pgfqpoint{3.696000in}{3.696000in}}%
\pgfusepath{clip}%
\pgfsetbuttcap%
\pgfsetroundjoin%
\definecolor{currentfill}{rgb}{0.121569,0.466667,0.705882}%
\pgfsetfillcolor{currentfill}%
\pgfsetfillopacity{0.704097}%
\pgfsetlinewidth{1.003750pt}%
\definecolor{currentstroke}{rgb}{0.121569,0.466667,0.705882}%
\pgfsetstrokecolor{currentstroke}%
\pgfsetstrokeopacity{0.704097}%
\pgfsetdash{}{0pt}%
\pgfpathmoveto{\pgfqpoint{3.292011in}{2.735016in}}%
\pgfpathcurveto{\pgfqpoint{3.300247in}{2.735016in}}{\pgfqpoint{3.308147in}{2.738288in}}{\pgfqpoint{3.313971in}{2.744112in}}%
\pgfpathcurveto{\pgfqpoint{3.319795in}{2.749936in}}{\pgfqpoint{3.323067in}{2.757836in}}{\pgfqpoint{3.323067in}{2.766072in}}%
\pgfpathcurveto{\pgfqpoint{3.323067in}{2.774308in}}{\pgfqpoint{3.319795in}{2.782208in}}{\pgfqpoint{3.313971in}{2.788032in}}%
\pgfpathcurveto{\pgfqpoint{3.308147in}{2.793856in}}{\pgfqpoint{3.300247in}{2.797129in}}{\pgfqpoint{3.292011in}{2.797129in}}%
\pgfpathcurveto{\pgfqpoint{3.283775in}{2.797129in}}{\pgfqpoint{3.275875in}{2.793856in}}{\pgfqpoint{3.270051in}{2.788032in}}%
\pgfpathcurveto{\pgfqpoint{3.264227in}{2.782208in}}{\pgfqpoint{3.260954in}{2.774308in}}{\pgfqpoint{3.260954in}{2.766072in}}%
\pgfpathcurveto{\pgfqpoint{3.260954in}{2.757836in}}{\pgfqpoint{3.264227in}{2.749936in}}{\pgfqpoint{3.270051in}{2.744112in}}%
\pgfpathcurveto{\pgfqpoint{3.275875in}{2.738288in}}{\pgfqpoint{3.283775in}{2.735016in}}{\pgfqpoint{3.292011in}{2.735016in}}%
\pgfpathclose%
\pgfusepath{stroke,fill}%
\end{pgfscope}%
\begin{pgfscope}%
\pgfpathrectangle{\pgfqpoint{0.100000in}{0.220728in}}{\pgfqpoint{3.696000in}{3.696000in}}%
\pgfusepath{clip}%
\pgfsetbuttcap%
\pgfsetroundjoin%
\definecolor{currentfill}{rgb}{0.121569,0.466667,0.705882}%
\pgfsetfillcolor{currentfill}%
\pgfsetfillopacity{0.704372}%
\pgfsetlinewidth{1.003750pt}%
\definecolor{currentstroke}{rgb}{0.121569,0.466667,0.705882}%
\pgfsetstrokecolor{currentstroke}%
\pgfsetstrokeopacity{0.704372}%
\pgfsetdash{}{0pt}%
\pgfpathmoveto{\pgfqpoint{3.291462in}{2.733814in}}%
\pgfpathcurveto{\pgfqpoint{3.299698in}{2.733814in}}{\pgfqpoint{3.307598in}{2.737087in}}{\pgfqpoint{3.313422in}{2.742911in}}%
\pgfpathcurveto{\pgfqpoint{3.319246in}{2.748735in}}{\pgfqpoint{3.322518in}{2.756635in}}{\pgfqpoint{3.322518in}{2.764871in}}%
\pgfpathcurveto{\pgfqpoint{3.322518in}{2.773107in}}{\pgfqpoint{3.319246in}{2.781007in}}{\pgfqpoint{3.313422in}{2.786831in}}%
\pgfpathcurveto{\pgfqpoint{3.307598in}{2.792655in}}{\pgfqpoint{3.299698in}{2.795927in}}{\pgfqpoint{3.291462in}{2.795927in}}%
\pgfpathcurveto{\pgfqpoint{3.283225in}{2.795927in}}{\pgfqpoint{3.275325in}{2.792655in}}{\pgfqpoint{3.269501in}{2.786831in}}%
\pgfpathcurveto{\pgfqpoint{3.263677in}{2.781007in}}{\pgfqpoint{3.260405in}{2.773107in}}{\pgfqpoint{3.260405in}{2.764871in}}%
\pgfpathcurveto{\pgfqpoint{3.260405in}{2.756635in}}{\pgfqpoint{3.263677in}{2.748735in}}{\pgfqpoint{3.269501in}{2.742911in}}%
\pgfpathcurveto{\pgfqpoint{3.275325in}{2.737087in}}{\pgfqpoint{3.283225in}{2.733814in}}{\pgfqpoint{3.291462in}{2.733814in}}%
\pgfpathclose%
\pgfusepath{stroke,fill}%
\end{pgfscope}%
\begin{pgfscope}%
\pgfpathrectangle{\pgfqpoint{0.100000in}{0.220728in}}{\pgfqpoint{3.696000in}{3.696000in}}%
\pgfusepath{clip}%
\pgfsetbuttcap%
\pgfsetroundjoin%
\definecolor{currentfill}{rgb}{0.121569,0.466667,0.705882}%
\pgfsetfillcolor{currentfill}%
\pgfsetfillopacity{0.704507}%
\pgfsetlinewidth{1.003750pt}%
\definecolor{currentstroke}{rgb}{0.121569,0.466667,0.705882}%
\pgfsetstrokecolor{currentstroke}%
\pgfsetstrokeopacity{0.704507}%
\pgfsetdash{}{0pt}%
\pgfpathmoveto{\pgfqpoint{3.291134in}{2.733110in}}%
\pgfpathcurveto{\pgfqpoint{3.299370in}{2.733110in}}{\pgfqpoint{3.307270in}{2.736382in}}{\pgfqpoint{3.313094in}{2.742206in}}%
\pgfpathcurveto{\pgfqpoint{3.318918in}{2.748030in}}{\pgfqpoint{3.322191in}{2.755930in}}{\pgfqpoint{3.322191in}{2.764167in}}%
\pgfpathcurveto{\pgfqpoint{3.322191in}{2.772403in}}{\pgfqpoint{3.318918in}{2.780303in}}{\pgfqpoint{3.313094in}{2.786127in}}%
\pgfpathcurveto{\pgfqpoint{3.307270in}{2.791951in}}{\pgfqpoint{3.299370in}{2.795223in}}{\pgfqpoint{3.291134in}{2.795223in}}%
\pgfpathcurveto{\pgfqpoint{3.282898in}{2.795223in}}{\pgfqpoint{3.274998in}{2.791951in}}{\pgfqpoint{3.269174in}{2.786127in}}%
\pgfpathcurveto{\pgfqpoint{3.263350in}{2.780303in}}{\pgfqpoint{3.260078in}{2.772403in}}{\pgfqpoint{3.260078in}{2.764167in}}%
\pgfpathcurveto{\pgfqpoint{3.260078in}{2.755930in}}{\pgfqpoint{3.263350in}{2.748030in}}{\pgfqpoint{3.269174in}{2.742206in}}%
\pgfpathcurveto{\pgfqpoint{3.274998in}{2.736382in}}{\pgfqpoint{3.282898in}{2.733110in}}{\pgfqpoint{3.291134in}{2.733110in}}%
\pgfpathclose%
\pgfusepath{stroke,fill}%
\end{pgfscope}%
\begin{pgfscope}%
\pgfpathrectangle{\pgfqpoint{0.100000in}{0.220728in}}{\pgfqpoint{3.696000in}{3.696000in}}%
\pgfusepath{clip}%
\pgfsetbuttcap%
\pgfsetroundjoin%
\definecolor{currentfill}{rgb}{0.121569,0.466667,0.705882}%
\pgfsetfillcolor{currentfill}%
\pgfsetfillopacity{0.704758}%
\pgfsetlinewidth{1.003750pt}%
\definecolor{currentstroke}{rgb}{0.121569,0.466667,0.705882}%
\pgfsetstrokecolor{currentstroke}%
\pgfsetstrokeopacity{0.704758}%
\pgfsetdash{}{0pt}%
\pgfpathmoveto{\pgfqpoint{3.290123in}{2.731608in}}%
\pgfpathcurveto{\pgfqpoint{3.298359in}{2.731608in}}{\pgfqpoint{3.306259in}{2.734881in}}{\pgfqpoint{3.312083in}{2.740705in}}%
\pgfpathcurveto{\pgfqpoint{3.317907in}{2.746529in}}{\pgfqpoint{3.321180in}{2.754429in}}{\pgfqpoint{3.321180in}{2.762665in}}%
\pgfpathcurveto{\pgfqpoint{3.321180in}{2.770901in}}{\pgfqpoint{3.317907in}{2.778801in}}{\pgfqpoint{3.312083in}{2.784625in}}%
\pgfpathcurveto{\pgfqpoint{3.306259in}{2.790449in}}{\pgfqpoint{3.298359in}{2.793721in}}{\pgfqpoint{3.290123in}{2.793721in}}%
\pgfpathcurveto{\pgfqpoint{3.281887in}{2.793721in}}{\pgfqpoint{3.273987in}{2.790449in}}{\pgfqpoint{3.268163in}{2.784625in}}%
\pgfpathcurveto{\pgfqpoint{3.262339in}{2.778801in}}{\pgfqpoint{3.259067in}{2.770901in}}{\pgfqpoint{3.259067in}{2.762665in}}%
\pgfpathcurveto{\pgfqpoint{3.259067in}{2.754429in}}{\pgfqpoint{3.262339in}{2.746529in}}{\pgfqpoint{3.268163in}{2.740705in}}%
\pgfpathcurveto{\pgfqpoint{3.273987in}{2.734881in}}{\pgfqpoint{3.281887in}{2.731608in}}{\pgfqpoint{3.290123in}{2.731608in}}%
\pgfpathclose%
\pgfusepath{stroke,fill}%
\end{pgfscope}%
\begin{pgfscope}%
\pgfpathrectangle{\pgfqpoint{0.100000in}{0.220728in}}{\pgfqpoint{3.696000in}{3.696000in}}%
\pgfusepath{clip}%
\pgfsetbuttcap%
\pgfsetroundjoin%
\definecolor{currentfill}{rgb}{0.121569,0.466667,0.705882}%
\pgfsetfillcolor{currentfill}%
\pgfsetfillopacity{0.705246}%
\pgfsetlinewidth{1.003750pt}%
\definecolor{currentstroke}{rgb}{0.121569,0.466667,0.705882}%
\pgfsetstrokecolor{currentstroke}%
\pgfsetstrokeopacity{0.705246}%
\pgfsetdash{}{0pt}%
\pgfpathmoveto{\pgfqpoint{3.288512in}{2.729396in}}%
\pgfpathcurveto{\pgfqpoint{3.296749in}{2.729396in}}{\pgfqpoint{3.304649in}{2.732668in}}{\pgfqpoint{3.310473in}{2.738492in}}%
\pgfpathcurveto{\pgfqpoint{3.316297in}{2.744316in}}{\pgfqpoint{3.319569in}{2.752216in}}{\pgfqpoint{3.319569in}{2.760452in}}%
\pgfpathcurveto{\pgfqpoint{3.319569in}{2.768689in}}{\pgfqpoint{3.316297in}{2.776589in}}{\pgfqpoint{3.310473in}{2.782413in}}%
\pgfpathcurveto{\pgfqpoint{3.304649in}{2.788237in}}{\pgfqpoint{3.296749in}{2.791509in}}{\pgfqpoint{3.288512in}{2.791509in}}%
\pgfpathcurveto{\pgfqpoint{3.280276in}{2.791509in}}{\pgfqpoint{3.272376in}{2.788237in}}{\pgfqpoint{3.266552in}{2.782413in}}%
\pgfpathcurveto{\pgfqpoint{3.260728in}{2.776589in}}{\pgfqpoint{3.257456in}{2.768689in}}{\pgfqpoint{3.257456in}{2.760452in}}%
\pgfpathcurveto{\pgfqpoint{3.257456in}{2.752216in}}{\pgfqpoint{3.260728in}{2.744316in}}{\pgfqpoint{3.266552in}{2.738492in}}%
\pgfpathcurveto{\pgfqpoint{3.272376in}{2.732668in}}{\pgfqpoint{3.280276in}{2.729396in}}{\pgfqpoint{3.288512in}{2.729396in}}%
\pgfpathclose%
\pgfusepath{stroke,fill}%
\end{pgfscope}%
\begin{pgfscope}%
\pgfpathrectangle{\pgfqpoint{0.100000in}{0.220728in}}{\pgfqpoint{3.696000in}{3.696000in}}%
\pgfusepath{clip}%
\pgfsetbuttcap%
\pgfsetroundjoin%
\definecolor{currentfill}{rgb}{0.121569,0.466667,0.705882}%
\pgfsetfillcolor{currentfill}%
\pgfsetfillopacity{0.705531}%
\pgfsetlinewidth{1.003750pt}%
\definecolor{currentstroke}{rgb}{0.121569,0.466667,0.705882}%
\pgfsetstrokecolor{currentstroke}%
\pgfsetstrokeopacity{0.705531}%
\pgfsetdash{}{0pt}%
\pgfpathmoveto{\pgfqpoint{3.287984in}{2.727831in}}%
\pgfpathcurveto{\pgfqpoint{3.296220in}{2.727831in}}{\pgfqpoint{3.304120in}{2.731103in}}{\pgfqpoint{3.309944in}{2.736927in}}%
\pgfpathcurveto{\pgfqpoint{3.315768in}{2.742751in}}{\pgfqpoint{3.319040in}{2.750651in}}{\pgfqpoint{3.319040in}{2.758887in}}%
\pgfpathcurveto{\pgfqpoint{3.319040in}{2.767123in}}{\pgfqpoint{3.315768in}{2.775023in}}{\pgfqpoint{3.309944in}{2.780847in}}%
\pgfpathcurveto{\pgfqpoint{3.304120in}{2.786671in}}{\pgfqpoint{3.296220in}{2.789944in}}{\pgfqpoint{3.287984in}{2.789944in}}%
\pgfpathcurveto{\pgfqpoint{3.279747in}{2.789944in}}{\pgfqpoint{3.271847in}{2.786671in}}{\pgfqpoint{3.266023in}{2.780847in}}%
\pgfpathcurveto{\pgfqpoint{3.260200in}{2.775023in}}{\pgfqpoint{3.256927in}{2.767123in}}{\pgfqpoint{3.256927in}{2.758887in}}%
\pgfpathcurveto{\pgfqpoint{3.256927in}{2.750651in}}{\pgfqpoint{3.260200in}{2.742751in}}{\pgfqpoint{3.266023in}{2.736927in}}%
\pgfpathcurveto{\pgfqpoint{3.271847in}{2.731103in}}{\pgfqpoint{3.279747in}{2.727831in}}{\pgfqpoint{3.287984in}{2.727831in}}%
\pgfpathclose%
\pgfusepath{stroke,fill}%
\end{pgfscope}%
\begin{pgfscope}%
\pgfpathrectangle{\pgfqpoint{0.100000in}{0.220728in}}{\pgfqpoint{3.696000in}{3.696000in}}%
\pgfusepath{clip}%
\pgfsetbuttcap%
\pgfsetroundjoin%
\definecolor{currentfill}{rgb}{0.121569,0.466667,0.705882}%
\pgfsetfillcolor{currentfill}%
\pgfsetfillopacity{0.705940}%
\pgfsetlinewidth{1.003750pt}%
\definecolor{currentstroke}{rgb}{0.121569,0.466667,0.705882}%
\pgfsetstrokecolor{currentstroke}%
\pgfsetstrokeopacity{0.705940}%
\pgfsetdash{}{0pt}%
\pgfpathmoveto{\pgfqpoint{3.285857in}{2.724571in}}%
\pgfpathcurveto{\pgfqpoint{3.294094in}{2.724571in}}{\pgfqpoint{3.301994in}{2.727843in}}{\pgfqpoint{3.307817in}{2.733667in}}%
\pgfpathcurveto{\pgfqpoint{3.313641in}{2.739491in}}{\pgfqpoint{3.316914in}{2.747391in}}{\pgfqpoint{3.316914in}{2.755627in}}%
\pgfpathcurveto{\pgfqpoint{3.316914in}{2.763863in}}{\pgfqpoint{3.313641in}{2.771763in}}{\pgfqpoint{3.307817in}{2.777587in}}%
\pgfpathcurveto{\pgfqpoint{3.301994in}{2.783411in}}{\pgfqpoint{3.294094in}{2.786684in}}{\pgfqpoint{3.285857in}{2.786684in}}%
\pgfpathcurveto{\pgfqpoint{3.277621in}{2.786684in}}{\pgfqpoint{3.269721in}{2.783411in}}{\pgfqpoint{3.263897in}{2.777587in}}%
\pgfpathcurveto{\pgfqpoint{3.258073in}{2.771763in}}{\pgfqpoint{3.254801in}{2.763863in}}{\pgfqpoint{3.254801in}{2.755627in}}%
\pgfpathcurveto{\pgfqpoint{3.254801in}{2.747391in}}{\pgfqpoint{3.258073in}{2.739491in}}{\pgfqpoint{3.263897in}{2.733667in}}%
\pgfpathcurveto{\pgfqpoint{3.269721in}{2.727843in}}{\pgfqpoint{3.277621in}{2.724571in}}{\pgfqpoint{3.285857in}{2.724571in}}%
\pgfpathclose%
\pgfusepath{stroke,fill}%
\end{pgfscope}%
\begin{pgfscope}%
\pgfpathrectangle{\pgfqpoint{0.100000in}{0.220728in}}{\pgfqpoint{3.696000in}{3.696000in}}%
\pgfusepath{clip}%
\pgfsetbuttcap%
\pgfsetroundjoin%
\definecolor{currentfill}{rgb}{0.121569,0.466667,0.705882}%
\pgfsetfillcolor{currentfill}%
\pgfsetfillopacity{0.706703}%
\pgfsetlinewidth{1.003750pt}%
\definecolor{currentstroke}{rgb}{0.121569,0.466667,0.705882}%
\pgfsetstrokecolor{currentstroke}%
\pgfsetstrokeopacity{0.706703}%
\pgfsetdash{}{0pt}%
\pgfpathmoveto{\pgfqpoint{3.283906in}{2.720521in}}%
\pgfpathcurveto{\pgfqpoint{3.292142in}{2.720521in}}{\pgfqpoint{3.300042in}{2.723794in}}{\pgfqpoint{3.305866in}{2.729618in}}%
\pgfpathcurveto{\pgfqpoint{3.311690in}{2.735441in}}{\pgfqpoint{3.314962in}{2.743342in}}{\pgfqpoint{3.314962in}{2.751578in}}%
\pgfpathcurveto{\pgfqpoint{3.314962in}{2.759814in}}{\pgfqpoint{3.311690in}{2.767714in}}{\pgfqpoint{3.305866in}{2.773538in}}%
\pgfpathcurveto{\pgfqpoint{3.300042in}{2.779362in}}{\pgfqpoint{3.292142in}{2.782634in}}{\pgfqpoint{3.283906in}{2.782634in}}%
\pgfpathcurveto{\pgfqpoint{3.275670in}{2.782634in}}{\pgfqpoint{3.267770in}{2.779362in}}{\pgfqpoint{3.261946in}{2.773538in}}%
\pgfpathcurveto{\pgfqpoint{3.256122in}{2.767714in}}{\pgfqpoint{3.252849in}{2.759814in}}{\pgfqpoint{3.252849in}{2.751578in}}%
\pgfpathcurveto{\pgfqpoint{3.252849in}{2.743342in}}{\pgfqpoint{3.256122in}{2.735441in}}{\pgfqpoint{3.261946in}{2.729618in}}%
\pgfpathcurveto{\pgfqpoint{3.267770in}{2.723794in}}{\pgfqpoint{3.275670in}{2.720521in}}{\pgfqpoint{3.283906in}{2.720521in}}%
\pgfpathclose%
\pgfusepath{stroke,fill}%
\end{pgfscope}%
\begin{pgfscope}%
\pgfpathrectangle{\pgfqpoint{0.100000in}{0.220728in}}{\pgfqpoint{3.696000in}{3.696000in}}%
\pgfusepath{clip}%
\pgfsetbuttcap%
\pgfsetroundjoin%
\definecolor{currentfill}{rgb}{0.121569,0.466667,0.705882}%
\pgfsetfillcolor{currentfill}%
\pgfsetfillopacity{0.707125}%
\pgfsetlinewidth{1.003750pt}%
\definecolor{currentstroke}{rgb}{0.121569,0.466667,0.705882}%
\pgfsetstrokecolor{currentstroke}%
\pgfsetstrokeopacity{0.707125}%
\pgfsetdash{}{0pt}%
\pgfpathmoveto{\pgfqpoint{3.282855in}{2.718282in}}%
\pgfpathcurveto{\pgfqpoint{3.291091in}{2.718282in}}{\pgfqpoint{3.298991in}{2.721554in}}{\pgfqpoint{3.304815in}{2.727378in}}%
\pgfpathcurveto{\pgfqpoint{3.310639in}{2.733202in}}{\pgfqpoint{3.313911in}{2.741102in}}{\pgfqpoint{3.313911in}{2.749338in}}%
\pgfpathcurveto{\pgfqpoint{3.313911in}{2.757575in}}{\pgfqpoint{3.310639in}{2.765475in}}{\pgfqpoint{3.304815in}{2.771299in}}%
\pgfpathcurveto{\pgfqpoint{3.298991in}{2.777122in}}{\pgfqpoint{3.291091in}{2.780395in}}{\pgfqpoint{3.282855in}{2.780395in}}%
\pgfpathcurveto{\pgfqpoint{3.274619in}{2.780395in}}{\pgfqpoint{3.266719in}{2.777122in}}{\pgfqpoint{3.260895in}{2.771299in}}%
\pgfpathcurveto{\pgfqpoint{3.255071in}{2.765475in}}{\pgfqpoint{3.251798in}{2.757575in}}{\pgfqpoint{3.251798in}{2.749338in}}%
\pgfpathcurveto{\pgfqpoint{3.251798in}{2.741102in}}{\pgfqpoint{3.255071in}{2.733202in}}{\pgfqpoint{3.260895in}{2.727378in}}%
\pgfpathcurveto{\pgfqpoint{3.266719in}{2.721554in}}{\pgfqpoint{3.274619in}{2.718282in}}{\pgfqpoint{3.282855in}{2.718282in}}%
\pgfpathclose%
\pgfusepath{stroke,fill}%
\end{pgfscope}%
\begin{pgfscope}%
\pgfpathrectangle{\pgfqpoint{0.100000in}{0.220728in}}{\pgfqpoint{3.696000in}{3.696000in}}%
\pgfusepath{clip}%
\pgfsetbuttcap%
\pgfsetroundjoin%
\definecolor{currentfill}{rgb}{0.121569,0.466667,0.705882}%
\pgfsetfillcolor{currentfill}%
\pgfsetfillopacity{0.707348}%
\pgfsetlinewidth{1.003750pt}%
\definecolor{currentstroke}{rgb}{0.121569,0.466667,0.705882}%
\pgfsetstrokecolor{currentstroke}%
\pgfsetstrokeopacity{0.707348}%
\pgfsetdash{}{0pt}%
\pgfpathmoveto{\pgfqpoint{3.282065in}{2.717318in}}%
\pgfpathcurveto{\pgfqpoint{3.290301in}{2.717318in}}{\pgfqpoint{3.298201in}{2.720590in}}{\pgfqpoint{3.304025in}{2.726414in}}%
\pgfpathcurveto{\pgfqpoint{3.309849in}{2.732238in}}{\pgfqpoint{3.313121in}{2.740138in}}{\pgfqpoint{3.313121in}{2.748374in}}%
\pgfpathcurveto{\pgfqpoint{3.313121in}{2.756611in}}{\pgfqpoint{3.309849in}{2.764511in}}{\pgfqpoint{3.304025in}{2.770335in}}%
\pgfpathcurveto{\pgfqpoint{3.298201in}{2.776159in}}{\pgfqpoint{3.290301in}{2.779431in}}{\pgfqpoint{3.282065in}{2.779431in}}%
\pgfpathcurveto{\pgfqpoint{3.273828in}{2.779431in}}{\pgfqpoint{3.265928in}{2.776159in}}{\pgfqpoint{3.260104in}{2.770335in}}%
\pgfpathcurveto{\pgfqpoint{3.254280in}{2.764511in}}{\pgfqpoint{3.251008in}{2.756611in}}{\pgfqpoint{3.251008in}{2.748374in}}%
\pgfpathcurveto{\pgfqpoint{3.251008in}{2.740138in}}{\pgfqpoint{3.254280in}{2.732238in}}{\pgfqpoint{3.260104in}{2.726414in}}%
\pgfpathcurveto{\pgfqpoint{3.265928in}{2.720590in}}{\pgfqpoint{3.273828in}{2.717318in}}{\pgfqpoint{3.282065in}{2.717318in}}%
\pgfpathclose%
\pgfusepath{stroke,fill}%
\end{pgfscope}%
\begin{pgfscope}%
\pgfpathrectangle{\pgfqpoint{0.100000in}{0.220728in}}{\pgfqpoint{3.696000in}{3.696000in}}%
\pgfusepath{clip}%
\pgfsetbuttcap%
\pgfsetroundjoin%
\definecolor{currentfill}{rgb}{0.121569,0.466667,0.705882}%
\pgfsetfillcolor{currentfill}%
\pgfsetfillopacity{0.708009}%
\pgfsetlinewidth{1.003750pt}%
\definecolor{currentstroke}{rgb}{0.121569,0.466667,0.705882}%
\pgfsetstrokecolor{currentstroke}%
\pgfsetstrokeopacity{0.708009}%
\pgfsetdash{}{0pt}%
\pgfpathmoveto{\pgfqpoint{3.280414in}{2.712581in}}%
\pgfpathcurveto{\pgfqpoint{3.288650in}{2.712581in}}{\pgfqpoint{3.296550in}{2.715853in}}{\pgfqpoint{3.302374in}{2.721677in}}%
\pgfpathcurveto{\pgfqpoint{3.308198in}{2.727501in}}{\pgfqpoint{3.311471in}{2.735401in}}{\pgfqpoint{3.311471in}{2.743638in}}%
\pgfpathcurveto{\pgfqpoint{3.311471in}{2.751874in}}{\pgfqpoint{3.308198in}{2.759774in}}{\pgfqpoint{3.302374in}{2.765598in}}%
\pgfpathcurveto{\pgfqpoint{3.296550in}{2.771422in}}{\pgfqpoint{3.288650in}{2.774694in}}{\pgfqpoint{3.280414in}{2.774694in}}%
\pgfpathcurveto{\pgfqpoint{3.272178in}{2.774694in}}{\pgfqpoint{3.264278in}{2.771422in}}{\pgfqpoint{3.258454in}{2.765598in}}%
\pgfpathcurveto{\pgfqpoint{3.252630in}{2.759774in}}{\pgfqpoint{3.249358in}{2.751874in}}{\pgfqpoint{3.249358in}{2.743638in}}%
\pgfpathcurveto{\pgfqpoint{3.249358in}{2.735401in}}{\pgfqpoint{3.252630in}{2.727501in}}{\pgfqpoint{3.258454in}{2.721677in}}%
\pgfpathcurveto{\pgfqpoint{3.264278in}{2.715853in}}{\pgfqpoint{3.272178in}{2.712581in}}{\pgfqpoint{3.280414in}{2.712581in}}%
\pgfpathclose%
\pgfusepath{stroke,fill}%
\end{pgfscope}%
\begin{pgfscope}%
\pgfpathrectangle{\pgfqpoint{0.100000in}{0.220728in}}{\pgfqpoint{3.696000in}{3.696000in}}%
\pgfusepath{clip}%
\pgfsetbuttcap%
\pgfsetroundjoin%
\definecolor{currentfill}{rgb}{0.121569,0.466667,0.705882}%
\pgfsetfillcolor{currentfill}%
\pgfsetfillopacity{0.708197}%
\pgfsetlinewidth{1.003750pt}%
\definecolor{currentstroke}{rgb}{0.121569,0.466667,0.705882}%
\pgfsetstrokecolor{currentstroke}%
\pgfsetstrokeopacity{0.708197}%
\pgfsetdash{}{0pt}%
\pgfpathmoveto{\pgfqpoint{0.863920in}{1.339922in}}%
\pgfpathcurveto{\pgfqpoint{0.872156in}{1.339922in}}{\pgfqpoint{0.880056in}{1.343194in}}{\pgfqpoint{0.885880in}{1.349018in}}%
\pgfpathcurveto{\pgfqpoint{0.891704in}{1.354842in}}{\pgfqpoint{0.894976in}{1.362742in}}{\pgfqpoint{0.894976in}{1.370978in}}%
\pgfpathcurveto{\pgfqpoint{0.894976in}{1.379214in}}{\pgfqpoint{0.891704in}{1.387115in}}{\pgfqpoint{0.885880in}{1.392938in}}%
\pgfpathcurveto{\pgfqpoint{0.880056in}{1.398762in}}{\pgfqpoint{0.872156in}{1.402035in}}{\pgfqpoint{0.863920in}{1.402035in}}%
\pgfpathcurveto{\pgfqpoint{0.855683in}{1.402035in}}{\pgfqpoint{0.847783in}{1.398762in}}{\pgfqpoint{0.841959in}{1.392938in}}%
\pgfpathcurveto{\pgfqpoint{0.836135in}{1.387115in}}{\pgfqpoint{0.832863in}{1.379214in}}{\pgfqpoint{0.832863in}{1.370978in}}%
\pgfpathcurveto{\pgfqpoint{0.832863in}{1.362742in}}{\pgfqpoint{0.836135in}{1.354842in}}{\pgfqpoint{0.841959in}{1.349018in}}%
\pgfpathcurveto{\pgfqpoint{0.847783in}{1.343194in}}{\pgfqpoint{0.855683in}{1.339922in}}{\pgfqpoint{0.863920in}{1.339922in}}%
\pgfpathclose%
\pgfusepath{stroke,fill}%
\end{pgfscope}%
\begin{pgfscope}%
\pgfpathrectangle{\pgfqpoint{0.100000in}{0.220728in}}{\pgfqpoint{3.696000in}{3.696000in}}%
\pgfusepath{clip}%
\pgfsetbuttcap%
\pgfsetroundjoin%
\definecolor{currentfill}{rgb}{0.121569,0.466667,0.705882}%
\pgfsetfillcolor{currentfill}%
\pgfsetfillopacity{0.708312}%
\pgfsetlinewidth{1.003750pt}%
\definecolor{currentstroke}{rgb}{0.121569,0.466667,0.705882}%
\pgfsetstrokecolor{currentstroke}%
\pgfsetstrokeopacity{0.708312}%
\pgfsetdash{}{0pt}%
\pgfpathmoveto{\pgfqpoint{3.279021in}{2.710335in}}%
\pgfpathcurveto{\pgfqpoint{3.287257in}{2.710335in}}{\pgfqpoint{3.295157in}{2.713607in}}{\pgfqpoint{3.300981in}{2.719431in}}%
\pgfpathcurveto{\pgfqpoint{3.306805in}{2.725255in}}{\pgfqpoint{3.310077in}{2.733155in}}{\pgfqpoint{3.310077in}{2.741392in}}%
\pgfpathcurveto{\pgfqpoint{3.310077in}{2.749628in}}{\pgfqpoint{3.306805in}{2.757528in}}{\pgfqpoint{3.300981in}{2.763352in}}%
\pgfpathcurveto{\pgfqpoint{3.295157in}{2.769176in}}{\pgfqpoint{3.287257in}{2.772448in}}{\pgfqpoint{3.279021in}{2.772448in}}%
\pgfpathcurveto{\pgfqpoint{3.270784in}{2.772448in}}{\pgfqpoint{3.262884in}{2.769176in}}{\pgfqpoint{3.257060in}{2.763352in}}%
\pgfpathcurveto{\pgfqpoint{3.251237in}{2.757528in}}{\pgfqpoint{3.247964in}{2.749628in}}{\pgfqpoint{3.247964in}{2.741392in}}%
\pgfpathcurveto{\pgfqpoint{3.247964in}{2.733155in}}{\pgfqpoint{3.251237in}{2.725255in}}{\pgfqpoint{3.257060in}{2.719431in}}%
\pgfpathcurveto{\pgfqpoint{3.262884in}{2.713607in}}{\pgfqpoint{3.270784in}{2.710335in}}{\pgfqpoint{3.279021in}{2.710335in}}%
\pgfpathclose%
\pgfusepath{stroke,fill}%
\end{pgfscope}%
\begin{pgfscope}%
\pgfpathrectangle{\pgfqpoint{0.100000in}{0.220728in}}{\pgfqpoint{3.696000in}{3.696000in}}%
\pgfusepath{clip}%
\pgfsetbuttcap%
\pgfsetroundjoin%
\definecolor{currentfill}{rgb}{0.121569,0.466667,0.705882}%
\pgfsetfillcolor{currentfill}%
\pgfsetfillopacity{0.708864}%
\pgfsetlinewidth{1.003750pt}%
\definecolor{currentstroke}{rgb}{0.121569,0.466667,0.705882}%
\pgfsetstrokecolor{currentstroke}%
\pgfsetstrokeopacity{0.708864}%
\pgfsetdash{}{0pt}%
\pgfpathmoveto{\pgfqpoint{3.277535in}{2.707480in}}%
\pgfpathcurveto{\pgfqpoint{3.285771in}{2.707480in}}{\pgfqpoint{3.293671in}{2.710753in}}{\pgfqpoint{3.299495in}{2.716577in}}%
\pgfpathcurveto{\pgfqpoint{3.305319in}{2.722401in}}{\pgfqpoint{3.308591in}{2.730301in}}{\pgfqpoint{3.308591in}{2.738537in}}%
\pgfpathcurveto{\pgfqpoint{3.308591in}{2.746773in}}{\pgfqpoint{3.305319in}{2.754673in}}{\pgfqpoint{3.299495in}{2.760497in}}%
\pgfpathcurveto{\pgfqpoint{3.293671in}{2.766321in}}{\pgfqpoint{3.285771in}{2.769593in}}{\pgfqpoint{3.277535in}{2.769593in}}%
\pgfpathcurveto{\pgfqpoint{3.269299in}{2.769593in}}{\pgfqpoint{3.261399in}{2.766321in}}{\pgfqpoint{3.255575in}{2.760497in}}%
\pgfpathcurveto{\pgfqpoint{3.249751in}{2.754673in}}{\pgfqpoint{3.246478in}{2.746773in}}{\pgfqpoint{3.246478in}{2.738537in}}%
\pgfpathcurveto{\pgfqpoint{3.246478in}{2.730301in}}{\pgfqpoint{3.249751in}{2.722401in}}{\pgfqpoint{3.255575in}{2.716577in}}%
\pgfpathcurveto{\pgfqpoint{3.261399in}{2.710753in}}{\pgfqpoint{3.269299in}{2.707480in}}{\pgfqpoint{3.277535in}{2.707480in}}%
\pgfpathclose%
\pgfusepath{stroke,fill}%
\end{pgfscope}%
\begin{pgfscope}%
\pgfpathrectangle{\pgfqpoint{0.100000in}{0.220728in}}{\pgfqpoint{3.696000in}{3.696000in}}%
\pgfusepath{clip}%
\pgfsetbuttcap%
\pgfsetroundjoin%
\definecolor{currentfill}{rgb}{0.121569,0.466667,0.705882}%
\pgfsetfillcolor{currentfill}%
\pgfsetfillopacity{0.709205}%
\pgfsetlinewidth{1.003750pt}%
\definecolor{currentstroke}{rgb}{0.121569,0.466667,0.705882}%
\pgfsetstrokecolor{currentstroke}%
\pgfsetstrokeopacity{0.709205}%
\pgfsetdash{}{0pt}%
\pgfpathmoveto{\pgfqpoint{3.276919in}{2.705864in}}%
\pgfpathcurveto{\pgfqpoint{3.285155in}{2.705864in}}{\pgfqpoint{3.293056in}{2.709136in}}{\pgfqpoint{3.298879in}{2.714960in}}%
\pgfpathcurveto{\pgfqpoint{3.304703in}{2.720784in}}{\pgfqpoint{3.307976in}{2.728684in}}{\pgfqpoint{3.307976in}{2.736921in}}%
\pgfpathcurveto{\pgfqpoint{3.307976in}{2.745157in}}{\pgfqpoint{3.304703in}{2.753057in}}{\pgfqpoint{3.298879in}{2.758881in}}%
\pgfpathcurveto{\pgfqpoint{3.293056in}{2.764705in}}{\pgfqpoint{3.285155in}{2.767977in}}{\pgfqpoint{3.276919in}{2.767977in}}%
\pgfpathcurveto{\pgfqpoint{3.268683in}{2.767977in}}{\pgfqpoint{3.260783in}{2.764705in}}{\pgfqpoint{3.254959in}{2.758881in}}%
\pgfpathcurveto{\pgfqpoint{3.249135in}{2.753057in}}{\pgfqpoint{3.245863in}{2.745157in}}{\pgfqpoint{3.245863in}{2.736921in}}%
\pgfpathcurveto{\pgfqpoint{3.245863in}{2.728684in}}{\pgfqpoint{3.249135in}{2.720784in}}{\pgfqpoint{3.254959in}{2.714960in}}%
\pgfpathcurveto{\pgfqpoint{3.260783in}{2.709136in}}{\pgfqpoint{3.268683in}{2.705864in}}{\pgfqpoint{3.276919in}{2.705864in}}%
\pgfpathclose%
\pgfusepath{stroke,fill}%
\end{pgfscope}%
\begin{pgfscope}%
\pgfpathrectangle{\pgfqpoint{0.100000in}{0.220728in}}{\pgfqpoint{3.696000in}{3.696000in}}%
\pgfusepath{clip}%
\pgfsetbuttcap%
\pgfsetroundjoin%
\definecolor{currentfill}{rgb}{0.121569,0.466667,0.705882}%
\pgfsetfillcolor{currentfill}%
\pgfsetfillopacity{0.709521}%
\pgfsetlinewidth{1.003750pt}%
\definecolor{currentstroke}{rgb}{0.121569,0.466667,0.705882}%
\pgfsetstrokecolor{currentstroke}%
\pgfsetstrokeopacity{0.709521}%
\pgfsetdash{}{0pt}%
\pgfpathmoveto{\pgfqpoint{3.275375in}{2.703771in}}%
\pgfpathcurveto{\pgfqpoint{3.283612in}{2.703771in}}{\pgfqpoint{3.291512in}{2.707044in}}{\pgfqpoint{3.297336in}{2.712867in}}%
\pgfpathcurveto{\pgfqpoint{3.303160in}{2.718691in}}{\pgfqpoint{3.306432in}{2.726591in}}{\pgfqpoint{3.306432in}{2.734828in}}%
\pgfpathcurveto{\pgfqpoint{3.306432in}{2.743064in}}{\pgfqpoint{3.303160in}{2.750964in}}{\pgfqpoint{3.297336in}{2.756788in}}%
\pgfpathcurveto{\pgfqpoint{3.291512in}{2.762612in}}{\pgfqpoint{3.283612in}{2.765884in}}{\pgfqpoint{3.275375in}{2.765884in}}%
\pgfpathcurveto{\pgfqpoint{3.267139in}{2.765884in}}{\pgfqpoint{3.259239in}{2.762612in}}{\pgfqpoint{3.253415in}{2.756788in}}%
\pgfpathcurveto{\pgfqpoint{3.247591in}{2.750964in}}{\pgfqpoint{3.244319in}{2.743064in}}{\pgfqpoint{3.244319in}{2.734828in}}%
\pgfpathcurveto{\pgfqpoint{3.244319in}{2.726591in}}{\pgfqpoint{3.247591in}{2.718691in}}{\pgfqpoint{3.253415in}{2.712867in}}%
\pgfpathcurveto{\pgfqpoint{3.259239in}{2.707044in}}{\pgfqpoint{3.267139in}{2.703771in}}{\pgfqpoint{3.275375in}{2.703771in}}%
\pgfpathclose%
\pgfusepath{stroke,fill}%
\end{pgfscope}%
\begin{pgfscope}%
\pgfpathrectangle{\pgfqpoint{0.100000in}{0.220728in}}{\pgfqpoint{3.696000in}{3.696000in}}%
\pgfusepath{clip}%
\pgfsetbuttcap%
\pgfsetroundjoin%
\definecolor{currentfill}{rgb}{0.121569,0.466667,0.705882}%
\pgfsetfillcolor{currentfill}%
\pgfsetfillopacity{0.710252}%
\pgfsetlinewidth{1.003750pt}%
\definecolor{currentstroke}{rgb}{0.121569,0.466667,0.705882}%
\pgfsetstrokecolor{currentstroke}%
\pgfsetstrokeopacity{0.710252}%
\pgfsetdash{}{0pt}%
\pgfpathmoveto{\pgfqpoint{3.273864in}{2.699261in}}%
\pgfpathcurveto{\pgfqpoint{3.282101in}{2.699261in}}{\pgfqpoint{3.290001in}{2.702533in}}{\pgfqpoint{3.295825in}{2.708357in}}%
\pgfpathcurveto{\pgfqpoint{3.301649in}{2.714181in}}{\pgfqpoint{3.304921in}{2.722081in}}{\pgfqpoint{3.304921in}{2.730318in}}%
\pgfpathcurveto{\pgfqpoint{3.304921in}{2.738554in}}{\pgfqpoint{3.301649in}{2.746454in}}{\pgfqpoint{3.295825in}{2.752278in}}%
\pgfpathcurveto{\pgfqpoint{3.290001in}{2.758102in}}{\pgfqpoint{3.282101in}{2.761374in}}{\pgfqpoint{3.273864in}{2.761374in}}%
\pgfpathcurveto{\pgfqpoint{3.265628in}{2.761374in}}{\pgfqpoint{3.257728in}{2.758102in}}{\pgfqpoint{3.251904in}{2.752278in}}%
\pgfpathcurveto{\pgfqpoint{3.246080in}{2.746454in}}{\pgfqpoint{3.242808in}{2.738554in}}{\pgfqpoint{3.242808in}{2.730318in}}%
\pgfpathcurveto{\pgfqpoint{3.242808in}{2.722081in}}{\pgfqpoint{3.246080in}{2.714181in}}{\pgfqpoint{3.251904in}{2.708357in}}%
\pgfpathcurveto{\pgfqpoint{3.257728in}{2.702533in}}{\pgfqpoint{3.265628in}{2.699261in}}{\pgfqpoint{3.273864in}{2.699261in}}%
\pgfpathclose%
\pgfusepath{stroke,fill}%
\end{pgfscope}%
\begin{pgfscope}%
\pgfpathrectangle{\pgfqpoint{0.100000in}{0.220728in}}{\pgfqpoint{3.696000in}{3.696000in}}%
\pgfusepath{clip}%
\pgfsetbuttcap%
\pgfsetroundjoin%
\definecolor{currentfill}{rgb}{0.121569,0.466667,0.705882}%
\pgfsetfillcolor{currentfill}%
\pgfsetfillopacity{0.710688}%
\pgfsetlinewidth{1.003750pt}%
\definecolor{currentstroke}{rgb}{0.121569,0.466667,0.705882}%
\pgfsetstrokecolor{currentstroke}%
\pgfsetstrokeopacity{0.710688}%
\pgfsetdash{}{0pt}%
\pgfpathmoveto{\pgfqpoint{3.272711in}{2.697260in}}%
\pgfpathcurveto{\pgfqpoint{3.280947in}{2.697260in}}{\pgfqpoint{3.288847in}{2.700532in}}{\pgfqpoint{3.294671in}{2.706356in}}%
\pgfpathcurveto{\pgfqpoint{3.300495in}{2.712180in}}{\pgfqpoint{3.303767in}{2.720080in}}{\pgfqpoint{3.303767in}{2.728316in}}%
\pgfpathcurveto{\pgfqpoint{3.303767in}{2.736552in}}{\pgfqpoint{3.300495in}{2.744453in}}{\pgfqpoint{3.294671in}{2.750276in}}%
\pgfpathcurveto{\pgfqpoint{3.288847in}{2.756100in}}{\pgfqpoint{3.280947in}{2.759373in}}{\pgfqpoint{3.272711in}{2.759373in}}%
\pgfpathcurveto{\pgfqpoint{3.264474in}{2.759373in}}{\pgfqpoint{3.256574in}{2.756100in}}{\pgfqpoint{3.250750in}{2.750276in}}%
\pgfpathcurveto{\pgfqpoint{3.244926in}{2.744453in}}{\pgfqpoint{3.241654in}{2.736552in}}{\pgfqpoint{3.241654in}{2.728316in}}%
\pgfpathcurveto{\pgfqpoint{3.241654in}{2.720080in}}{\pgfqpoint{3.244926in}{2.712180in}}{\pgfqpoint{3.250750in}{2.706356in}}%
\pgfpathcurveto{\pgfqpoint{3.256574in}{2.700532in}}{\pgfqpoint{3.264474in}{2.697260in}}{\pgfqpoint{3.272711in}{2.697260in}}%
\pgfpathclose%
\pgfusepath{stroke,fill}%
\end{pgfscope}%
\begin{pgfscope}%
\pgfpathrectangle{\pgfqpoint{0.100000in}{0.220728in}}{\pgfqpoint{3.696000in}{3.696000in}}%
\pgfusepath{clip}%
\pgfsetbuttcap%
\pgfsetroundjoin%
\definecolor{currentfill}{rgb}{0.121569,0.466667,0.705882}%
\pgfsetfillcolor{currentfill}%
\pgfsetfillopacity{0.710941}%
\pgfsetlinewidth{1.003750pt}%
\definecolor{currentstroke}{rgb}{0.121569,0.466667,0.705882}%
\pgfsetstrokecolor{currentstroke}%
\pgfsetstrokeopacity{0.710941}%
\pgfsetdash{}{0pt}%
\pgfpathmoveto{\pgfqpoint{3.272004in}{2.696323in}}%
\pgfpathcurveto{\pgfqpoint{3.280240in}{2.696323in}}{\pgfqpoint{3.288140in}{2.699595in}}{\pgfqpoint{3.293964in}{2.705419in}}%
\pgfpathcurveto{\pgfqpoint{3.299788in}{2.711243in}}{\pgfqpoint{3.303060in}{2.719143in}}{\pgfqpoint{3.303060in}{2.727379in}}%
\pgfpathcurveto{\pgfqpoint{3.303060in}{2.735616in}}{\pgfqpoint{3.299788in}{2.743516in}}{\pgfqpoint{3.293964in}{2.749340in}}%
\pgfpathcurveto{\pgfqpoint{3.288140in}{2.755164in}}{\pgfqpoint{3.280240in}{2.758436in}}{\pgfqpoint{3.272004in}{2.758436in}}%
\pgfpathcurveto{\pgfqpoint{3.263767in}{2.758436in}}{\pgfqpoint{3.255867in}{2.755164in}}{\pgfqpoint{3.250043in}{2.749340in}}%
\pgfpathcurveto{\pgfqpoint{3.244219in}{2.743516in}}{\pgfqpoint{3.240947in}{2.735616in}}{\pgfqpoint{3.240947in}{2.727379in}}%
\pgfpathcurveto{\pgfqpoint{3.240947in}{2.719143in}}{\pgfqpoint{3.244219in}{2.711243in}}{\pgfqpoint{3.250043in}{2.705419in}}%
\pgfpathcurveto{\pgfqpoint{3.255867in}{2.699595in}}{\pgfqpoint{3.263767in}{2.696323in}}{\pgfqpoint{3.272004in}{2.696323in}}%
\pgfpathclose%
\pgfusepath{stroke,fill}%
\end{pgfscope}%
\begin{pgfscope}%
\pgfpathrectangle{\pgfqpoint{0.100000in}{0.220728in}}{\pgfqpoint{3.696000in}{3.696000in}}%
\pgfusepath{clip}%
\pgfsetbuttcap%
\pgfsetroundjoin%
\definecolor{currentfill}{rgb}{0.121569,0.466667,0.705882}%
\pgfsetfillcolor{currentfill}%
\pgfsetfillopacity{0.711017}%
\pgfsetlinewidth{1.003750pt}%
\definecolor{currentstroke}{rgb}{0.121569,0.466667,0.705882}%
\pgfsetstrokecolor{currentstroke}%
\pgfsetstrokeopacity{0.711017}%
\pgfsetdash{}{0pt}%
\pgfpathmoveto{\pgfqpoint{0.880277in}{1.329815in}}%
\pgfpathcurveto{\pgfqpoint{0.888513in}{1.329815in}}{\pgfqpoint{0.896413in}{1.333088in}}{\pgfqpoint{0.902237in}{1.338911in}}%
\pgfpathcurveto{\pgfqpoint{0.908061in}{1.344735in}}{\pgfqpoint{0.911333in}{1.352635in}}{\pgfqpoint{0.911333in}{1.360872in}}%
\pgfpathcurveto{\pgfqpoint{0.911333in}{1.369108in}}{\pgfqpoint{0.908061in}{1.377008in}}{\pgfqpoint{0.902237in}{1.382832in}}%
\pgfpathcurveto{\pgfqpoint{0.896413in}{1.388656in}}{\pgfqpoint{0.888513in}{1.391928in}}{\pgfqpoint{0.880277in}{1.391928in}}%
\pgfpathcurveto{\pgfqpoint{0.872040in}{1.391928in}}{\pgfqpoint{0.864140in}{1.388656in}}{\pgfqpoint{0.858316in}{1.382832in}}%
\pgfpathcurveto{\pgfqpoint{0.852492in}{1.377008in}}{\pgfqpoint{0.849220in}{1.369108in}}{\pgfqpoint{0.849220in}{1.360872in}}%
\pgfpathcurveto{\pgfqpoint{0.849220in}{1.352635in}}{\pgfqpoint{0.852492in}{1.344735in}}{\pgfqpoint{0.858316in}{1.338911in}}%
\pgfpathcurveto{\pgfqpoint{0.864140in}{1.333088in}}{\pgfqpoint{0.872040in}{1.329815in}}{\pgfqpoint{0.880277in}{1.329815in}}%
\pgfpathclose%
\pgfusepath{stroke,fill}%
\end{pgfscope}%
\begin{pgfscope}%
\pgfpathrectangle{\pgfqpoint{0.100000in}{0.220728in}}{\pgfqpoint{3.696000in}{3.696000in}}%
\pgfusepath{clip}%
\pgfsetbuttcap%
\pgfsetroundjoin%
\definecolor{currentfill}{rgb}{0.121569,0.466667,0.705882}%
\pgfsetfillcolor{currentfill}%
\pgfsetfillopacity{0.711468}%
\pgfsetlinewidth{1.003750pt}%
\definecolor{currentstroke}{rgb}{0.121569,0.466667,0.705882}%
\pgfsetstrokecolor{currentstroke}%
\pgfsetstrokeopacity{0.711468}%
\pgfsetdash{}{0pt}%
\pgfpathmoveto{\pgfqpoint{3.271012in}{2.693376in}}%
\pgfpathcurveto{\pgfqpoint{3.279248in}{2.693376in}}{\pgfqpoint{3.287148in}{2.696648in}}{\pgfqpoint{3.292972in}{2.702472in}}%
\pgfpathcurveto{\pgfqpoint{3.298796in}{2.708296in}}{\pgfqpoint{3.302068in}{2.716196in}}{\pgfqpoint{3.302068in}{2.724433in}}%
\pgfpathcurveto{\pgfqpoint{3.302068in}{2.732669in}}{\pgfqpoint{3.298796in}{2.740569in}}{\pgfqpoint{3.292972in}{2.746393in}}%
\pgfpathcurveto{\pgfqpoint{3.287148in}{2.752217in}}{\pgfqpoint{3.279248in}{2.755489in}}{\pgfqpoint{3.271012in}{2.755489in}}%
\pgfpathcurveto{\pgfqpoint{3.262775in}{2.755489in}}{\pgfqpoint{3.254875in}{2.752217in}}{\pgfqpoint{3.249051in}{2.746393in}}%
\pgfpathcurveto{\pgfqpoint{3.243227in}{2.740569in}}{\pgfqpoint{3.239955in}{2.732669in}}{\pgfqpoint{3.239955in}{2.724433in}}%
\pgfpathcurveto{\pgfqpoint{3.239955in}{2.716196in}}{\pgfqpoint{3.243227in}{2.708296in}}{\pgfqpoint{3.249051in}{2.702472in}}%
\pgfpathcurveto{\pgfqpoint{3.254875in}{2.696648in}}{\pgfqpoint{3.262775in}{2.693376in}}{\pgfqpoint{3.271012in}{2.693376in}}%
\pgfpathclose%
\pgfusepath{stroke,fill}%
\end{pgfscope}%
\begin{pgfscope}%
\pgfpathrectangle{\pgfqpoint{0.100000in}{0.220728in}}{\pgfqpoint{3.696000in}{3.696000in}}%
\pgfusepath{clip}%
\pgfsetbuttcap%
\pgfsetroundjoin%
\definecolor{currentfill}{rgb}{0.121569,0.466667,0.705882}%
\pgfsetfillcolor{currentfill}%
\pgfsetfillopacity{0.712110}%
\pgfsetlinewidth{1.003750pt}%
\definecolor{currentstroke}{rgb}{0.121569,0.466667,0.705882}%
\pgfsetstrokecolor{currentstroke}%
\pgfsetstrokeopacity{0.712110}%
\pgfsetdash{}{0pt}%
\pgfpathmoveto{\pgfqpoint{3.268394in}{2.689855in}}%
\pgfpathcurveto{\pgfqpoint{3.276630in}{2.689855in}}{\pgfqpoint{3.284530in}{2.693127in}}{\pgfqpoint{3.290354in}{2.698951in}}%
\pgfpathcurveto{\pgfqpoint{3.296178in}{2.704775in}}{\pgfqpoint{3.299450in}{2.712675in}}{\pgfqpoint{3.299450in}{2.720911in}}%
\pgfpathcurveto{\pgfqpoint{3.299450in}{2.729148in}}{\pgfqpoint{3.296178in}{2.737048in}}{\pgfqpoint{3.290354in}{2.742872in}}%
\pgfpathcurveto{\pgfqpoint{3.284530in}{2.748696in}}{\pgfqpoint{3.276630in}{2.751968in}}{\pgfqpoint{3.268394in}{2.751968in}}%
\pgfpathcurveto{\pgfqpoint{3.260157in}{2.751968in}}{\pgfqpoint{3.252257in}{2.748696in}}{\pgfqpoint{3.246433in}{2.742872in}}%
\pgfpathcurveto{\pgfqpoint{3.240609in}{2.737048in}}{\pgfqpoint{3.237337in}{2.729148in}}{\pgfqpoint{3.237337in}{2.720911in}}%
\pgfpathcurveto{\pgfqpoint{3.237337in}{2.712675in}}{\pgfqpoint{3.240609in}{2.704775in}}{\pgfqpoint{3.246433in}{2.698951in}}%
\pgfpathcurveto{\pgfqpoint{3.252257in}{2.693127in}}{\pgfqpoint{3.260157in}{2.689855in}}{\pgfqpoint{3.268394in}{2.689855in}}%
\pgfpathclose%
\pgfusepath{stroke,fill}%
\end{pgfscope}%
\begin{pgfscope}%
\pgfpathrectangle{\pgfqpoint{0.100000in}{0.220728in}}{\pgfqpoint{3.696000in}{3.696000in}}%
\pgfusepath{clip}%
\pgfsetbuttcap%
\pgfsetroundjoin%
\definecolor{currentfill}{rgb}{0.121569,0.466667,0.705882}%
\pgfsetfillcolor{currentfill}%
\pgfsetfillopacity{0.712636}%
\pgfsetlinewidth{1.003750pt}%
\definecolor{currentstroke}{rgb}{0.121569,0.466667,0.705882}%
\pgfsetstrokecolor{currentstroke}%
\pgfsetstrokeopacity{0.712636}%
\pgfsetdash{}{0pt}%
\pgfpathmoveto{\pgfqpoint{3.267340in}{2.688098in}}%
\pgfpathcurveto{\pgfqpoint{3.275576in}{2.688098in}}{\pgfqpoint{3.283476in}{2.691371in}}{\pgfqpoint{3.289300in}{2.697195in}}%
\pgfpathcurveto{\pgfqpoint{3.295124in}{2.703018in}}{\pgfqpoint{3.298396in}{2.710919in}}{\pgfqpoint{3.298396in}{2.719155in}}%
\pgfpathcurveto{\pgfqpoint{3.298396in}{2.727391in}}{\pgfqpoint{3.295124in}{2.735291in}}{\pgfqpoint{3.289300in}{2.741115in}}%
\pgfpathcurveto{\pgfqpoint{3.283476in}{2.746939in}}{\pgfqpoint{3.275576in}{2.750211in}}{\pgfqpoint{3.267340in}{2.750211in}}%
\pgfpathcurveto{\pgfqpoint{3.259104in}{2.750211in}}{\pgfqpoint{3.251204in}{2.746939in}}{\pgfqpoint{3.245380in}{2.741115in}}%
\pgfpathcurveto{\pgfqpoint{3.239556in}{2.735291in}}{\pgfqpoint{3.236283in}{2.727391in}}{\pgfqpoint{3.236283in}{2.719155in}}%
\pgfpathcurveto{\pgfqpoint{3.236283in}{2.710919in}}{\pgfqpoint{3.239556in}{2.703018in}}{\pgfqpoint{3.245380in}{2.697195in}}%
\pgfpathcurveto{\pgfqpoint{3.251204in}{2.691371in}}{\pgfqpoint{3.259104in}{2.688098in}}{\pgfqpoint{3.267340in}{2.688098in}}%
\pgfpathclose%
\pgfusepath{stroke,fill}%
\end{pgfscope}%
\begin{pgfscope}%
\pgfpathrectangle{\pgfqpoint{0.100000in}{0.220728in}}{\pgfqpoint{3.696000in}{3.696000in}}%
\pgfusepath{clip}%
\pgfsetbuttcap%
\pgfsetroundjoin%
\definecolor{currentfill}{rgb}{0.121569,0.466667,0.705882}%
\pgfsetfillcolor{currentfill}%
\pgfsetfillopacity{0.712907}%
\pgfsetlinewidth{1.003750pt}%
\definecolor{currentstroke}{rgb}{0.121569,0.466667,0.705882}%
\pgfsetstrokecolor{currentstroke}%
\pgfsetstrokeopacity{0.712907}%
\pgfsetdash{}{0pt}%
\pgfpathmoveto{\pgfqpoint{3.266922in}{2.686904in}}%
\pgfpathcurveto{\pgfqpoint{3.275158in}{2.686904in}}{\pgfqpoint{3.283058in}{2.690176in}}{\pgfqpoint{3.288882in}{2.696000in}}%
\pgfpathcurveto{\pgfqpoint{3.294706in}{2.701824in}}{\pgfqpoint{3.297978in}{2.709724in}}{\pgfqpoint{3.297978in}{2.717961in}}%
\pgfpathcurveto{\pgfqpoint{3.297978in}{2.726197in}}{\pgfqpoint{3.294706in}{2.734097in}}{\pgfqpoint{3.288882in}{2.739921in}}%
\pgfpathcurveto{\pgfqpoint{3.283058in}{2.745745in}}{\pgfqpoint{3.275158in}{2.749017in}}{\pgfqpoint{3.266922in}{2.749017in}}%
\pgfpathcurveto{\pgfqpoint{3.258685in}{2.749017in}}{\pgfqpoint{3.250785in}{2.745745in}}{\pgfqpoint{3.244962in}{2.739921in}}%
\pgfpathcurveto{\pgfqpoint{3.239138in}{2.734097in}}{\pgfqpoint{3.235865in}{2.726197in}}{\pgfqpoint{3.235865in}{2.717961in}}%
\pgfpathcurveto{\pgfqpoint{3.235865in}{2.709724in}}{\pgfqpoint{3.239138in}{2.701824in}}{\pgfqpoint{3.244962in}{2.696000in}}%
\pgfpathcurveto{\pgfqpoint{3.250785in}{2.690176in}}{\pgfqpoint{3.258685in}{2.686904in}}{\pgfqpoint{3.266922in}{2.686904in}}%
\pgfpathclose%
\pgfusepath{stroke,fill}%
\end{pgfscope}%
\begin{pgfscope}%
\pgfpathrectangle{\pgfqpoint{0.100000in}{0.220728in}}{\pgfqpoint{3.696000in}{3.696000in}}%
\pgfusepath{clip}%
\pgfsetbuttcap%
\pgfsetroundjoin%
\definecolor{currentfill}{rgb}{0.121569,0.466667,0.705882}%
\pgfsetfillcolor{currentfill}%
\pgfsetfillopacity{0.713216}%
\pgfsetlinewidth{1.003750pt}%
\definecolor{currentstroke}{rgb}{0.121569,0.466667,0.705882}%
\pgfsetstrokecolor{currentstroke}%
\pgfsetstrokeopacity{0.713216}%
\pgfsetdash{}{0pt}%
\pgfpathmoveto{\pgfqpoint{3.265685in}{2.685033in}}%
\pgfpathcurveto{\pgfqpoint{3.273921in}{2.685033in}}{\pgfqpoint{3.281821in}{2.688306in}}{\pgfqpoint{3.287645in}{2.694130in}}%
\pgfpathcurveto{\pgfqpoint{3.293469in}{2.699954in}}{\pgfqpoint{3.296742in}{2.707854in}}{\pgfqpoint{3.296742in}{2.716090in}}%
\pgfpathcurveto{\pgfqpoint{3.296742in}{2.724326in}}{\pgfqpoint{3.293469in}{2.732226in}}{\pgfqpoint{3.287645in}{2.738050in}}%
\pgfpathcurveto{\pgfqpoint{3.281821in}{2.743874in}}{\pgfqpoint{3.273921in}{2.747146in}}{\pgfqpoint{3.265685in}{2.747146in}}%
\pgfpathcurveto{\pgfqpoint{3.257449in}{2.747146in}}{\pgfqpoint{3.249549in}{2.743874in}}{\pgfqpoint{3.243725in}{2.738050in}}%
\pgfpathcurveto{\pgfqpoint{3.237901in}{2.732226in}}{\pgfqpoint{3.234629in}{2.724326in}}{\pgfqpoint{3.234629in}{2.716090in}}%
\pgfpathcurveto{\pgfqpoint{3.234629in}{2.707854in}}{\pgfqpoint{3.237901in}{2.699954in}}{\pgfqpoint{3.243725in}{2.694130in}}%
\pgfpathcurveto{\pgfqpoint{3.249549in}{2.688306in}}{\pgfqpoint{3.257449in}{2.685033in}}{\pgfqpoint{3.265685in}{2.685033in}}%
\pgfpathclose%
\pgfusepath{stroke,fill}%
\end{pgfscope}%
\begin{pgfscope}%
\pgfpathrectangle{\pgfqpoint{0.100000in}{0.220728in}}{\pgfqpoint{3.696000in}{3.696000in}}%
\pgfusepath{clip}%
\pgfsetbuttcap%
\pgfsetroundjoin%
\definecolor{currentfill}{rgb}{0.121569,0.466667,0.705882}%
\pgfsetfillcolor{currentfill}%
\pgfsetfillopacity{0.713837}%
\pgfsetlinewidth{1.003750pt}%
\definecolor{currentstroke}{rgb}{0.121569,0.466667,0.705882}%
\pgfsetstrokecolor{currentstroke}%
\pgfsetstrokeopacity{0.713837}%
\pgfsetdash{}{0pt}%
\pgfpathmoveto{\pgfqpoint{0.893794in}{1.322953in}}%
\pgfpathcurveto{\pgfqpoint{0.902031in}{1.322953in}}{\pgfqpoint{0.909931in}{1.326225in}}{\pgfqpoint{0.915755in}{1.332049in}}%
\pgfpathcurveto{\pgfqpoint{0.921579in}{1.337873in}}{\pgfqpoint{0.924851in}{1.345773in}}{\pgfqpoint{0.924851in}{1.354009in}}%
\pgfpathcurveto{\pgfqpoint{0.924851in}{1.362245in}}{\pgfqpoint{0.921579in}{1.370145in}}{\pgfqpoint{0.915755in}{1.375969in}}%
\pgfpathcurveto{\pgfqpoint{0.909931in}{1.381793in}}{\pgfqpoint{0.902031in}{1.385066in}}{\pgfqpoint{0.893794in}{1.385066in}}%
\pgfpathcurveto{\pgfqpoint{0.885558in}{1.385066in}}{\pgfqpoint{0.877658in}{1.381793in}}{\pgfqpoint{0.871834in}{1.375969in}}%
\pgfpathcurveto{\pgfqpoint{0.866010in}{1.370145in}}{\pgfqpoint{0.862738in}{1.362245in}}{\pgfqpoint{0.862738in}{1.354009in}}%
\pgfpathcurveto{\pgfqpoint{0.862738in}{1.345773in}}{\pgfqpoint{0.866010in}{1.337873in}}{\pgfqpoint{0.871834in}{1.332049in}}%
\pgfpathcurveto{\pgfqpoint{0.877658in}{1.326225in}}{\pgfqpoint{0.885558in}{1.322953in}}{\pgfqpoint{0.893794in}{1.322953in}}%
\pgfpathclose%
\pgfusepath{stroke,fill}%
\end{pgfscope}%
\begin{pgfscope}%
\pgfpathrectangle{\pgfqpoint{0.100000in}{0.220728in}}{\pgfqpoint{3.696000in}{3.696000in}}%
\pgfusepath{clip}%
\pgfsetbuttcap%
\pgfsetroundjoin%
\definecolor{currentfill}{rgb}{0.121569,0.466667,0.705882}%
\pgfsetfillcolor{currentfill}%
\pgfsetfillopacity{0.714094}%
\pgfsetlinewidth{1.003750pt}%
\definecolor{currentstroke}{rgb}{0.121569,0.466667,0.705882}%
\pgfsetstrokecolor{currentstroke}%
\pgfsetstrokeopacity{0.714094}%
\pgfsetdash{}{0pt}%
\pgfpathmoveto{\pgfqpoint{3.264342in}{2.680874in}}%
\pgfpathcurveto{\pgfqpoint{3.272579in}{2.680874in}}{\pgfqpoint{3.280479in}{2.684146in}}{\pgfqpoint{3.286303in}{2.689970in}}%
\pgfpathcurveto{\pgfqpoint{3.292127in}{2.695794in}}{\pgfqpoint{3.295399in}{2.703694in}}{\pgfqpoint{3.295399in}{2.711931in}}%
\pgfpathcurveto{\pgfqpoint{3.295399in}{2.720167in}}{\pgfqpoint{3.292127in}{2.728067in}}{\pgfqpoint{3.286303in}{2.733891in}}%
\pgfpathcurveto{\pgfqpoint{3.280479in}{2.739715in}}{\pgfqpoint{3.272579in}{2.742987in}}{\pgfqpoint{3.264342in}{2.742987in}}%
\pgfpathcurveto{\pgfqpoint{3.256106in}{2.742987in}}{\pgfqpoint{3.248206in}{2.739715in}}{\pgfqpoint{3.242382in}{2.733891in}}%
\pgfpathcurveto{\pgfqpoint{3.236558in}{2.728067in}}{\pgfqpoint{3.233286in}{2.720167in}}{\pgfqpoint{3.233286in}{2.711931in}}%
\pgfpathcurveto{\pgfqpoint{3.233286in}{2.703694in}}{\pgfqpoint{3.236558in}{2.695794in}}{\pgfqpoint{3.242382in}{2.689970in}}%
\pgfpathcurveto{\pgfqpoint{3.248206in}{2.684146in}}{\pgfqpoint{3.256106in}{2.680874in}}{\pgfqpoint{3.264342in}{2.680874in}}%
\pgfpathclose%
\pgfusepath{stroke,fill}%
\end{pgfscope}%
\begin{pgfscope}%
\pgfpathrectangle{\pgfqpoint{0.100000in}{0.220728in}}{\pgfqpoint{3.696000in}{3.696000in}}%
\pgfusepath{clip}%
\pgfsetbuttcap%
\pgfsetroundjoin%
\definecolor{currentfill}{rgb}{0.121569,0.466667,0.705882}%
\pgfsetfillcolor{currentfill}%
\pgfsetfillopacity{0.715027}%
\pgfsetlinewidth{1.003750pt}%
\definecolor{currentstroke}{rgb}{0.121569,0.466667,0.705882}%
\pgfsetstrokecolor{currentstroke}%
\pgfsetstrokeopacity{0.715027}%
\pgfsetdash{}{0pt}%
\pgfpathmoveto{\pgfqpoint{3.262108in}{2.676468in}}%
\pgfpathcurveto{\pgfqpoint{3.270345in}{2.676468in}}{\pgfqpoint{3.278245in}{2.679740in}}{\pgfqpoint{3.284069in}{2.685564in}}%
\pgfpathcurveto{\pgfqpoint{3.289893in}{2.691388in}}{\pgfqpoint{3.293165in}{2.699288in}}{\pgfqpoint{3.293165in}{2.707524in}}%
\pgfpathcurveto{\pgfqpoint{3.293165in}{2.715761in}}{\pgfqpoint{3.289893in}{2.723661in}}{\pgfqpoint{3.284069in}{2.729485in}}%
\pgfpathcurveto{\pgfqpoint{3.278245in}{2.735309in}}{\pgfqpoint{3.270345in}{2.738581in}}{\pgfqpoint{3.262108in}{2.738581in}}%
\pgfpathcurveto{\pgfqpoint{3.253872in}{2.738581in}}{\pgfqpoint{3.245972in}{2.735309in}}{\pgfqpoint{3.240148in}{2.729485in}}%
\pgfpathcurveto{\pgfqpoint{3.234324in}{2.723661in}}{\pgfqpoint{3.231052in}{2.715761in}}{\pgfqpoint{3.231052in}{2.707524in}}%
\pgfpathcurveto{\pgfqpoint{3.231052in}{2.699288in}}{\pgfqpoint{3.234324in}{2.691388in}}{\pgfqpoint{3.240148in}{2.685564in}}%
\pgfpathcurveto{\pgfqpoint{3.245972in}{2.679740in}}{\pgfqpoint{3.253872in}{2.676468in}}{\pgfqpoint{3.262108in}{2.676468in}}%
\pgfpathclose%
\pgfusepath{stroke,fill}%
\end{pgfscope}%
\begin{pgfscope}%
\pgfpathrectangle{\pgfqpoint{0.100000in}{0.220728in}}{\pgfqpoint{3.696000in}{3.696000in}}%
\pgfusepath{clip}%
\pgfsetbuttcap%
\pgfsetroundjoin%
\definecolor{currentfill}{rgb}{0.121569,0.466667,0.705882}%
\pgfsetfillcolor{currentfill}%
\pgfsetfillopacity{0.715291}%
\pgfsetlinewidth{1.003750pt}%
\definecolor{currentstroke}{rgb}{0.121569,0.466667,0.705882}%
\pgfsetstrokecolor{currentstroke}%
\pgfsetstrokeopacity{0.715291}%
\pgfsetdash{}{0pt}%
\pgfpathmoveto{\pgfqpoint{0.904858in}{1.313684in}}%
\pgfpathcurveto{\pgfqpoint{0.913095in}{1.313684in}}{\pgfqpoint{0.920995in}{1.316956in}}{\pgfqpoint{0.926819in}{1.322780in}}%
\pgfpathcurveto{\pgfqpoint{0.932642in}{1.328604in}}{\pgfqpoint{0.935915in}{1.336504in}}{\pgfqpoint{0.935915in}{1.344741in}}%
\pgfpathcurveto{\pgfqpoint{0.935915in}{1.352977in}}{\pgfqpoint{0.932642in}{1.360877in}}{\pgfqpoint{0.926819in}{1.366701in}}%
\pgfpathcurveto{\pgfqpoint{0.920995in}{1.372525in}}{\pgfqpoint{0.913095in}{1.375797in}}{\pgfqpoint{0.904858in}{1.375797in}}%
\pgfpathcurveto{\pgfqpoint{0.896622in}{1.375797in}}{\pgfqpoint{0.888722in}{1.372525in}}{\pgfqpoint{0.882898in}{1.366701in}}%
\pgfpathcurveto{\pgfqpoint{0.877074in}{1.360877in}}{\pgfqpoint{0.873802in}{1.352977in}}{\pgfqpoint{0.873802in}{1.344741in}}%
\pgfpathcurveto{\pgfqpoint{0.873802in}{1.336504in}}{\pgfqpoint{0.877074in}{1.328604in}}{\pgfqpoint{0.882898in}{1.322780in}}%
\pgfpathcurveto{\pgfqpoint{0.888722in}{1.316956in}}{\pgfqpoint{0.896622in}{1.313684in}}{\pgfqpoint{0.904858in}{1.313684in}}%
\pgfpathclose%
\pgfusepath{stroke,fill}%
\end{pgfscope}%
\begin{pgfscope}%
\pgfpathrectangle{\pgfqpoint{0.100000in}{0.220728in}}{\pgfqpoint{3.696000in}{3.696000in}}%
\pgfusepath{clip}%
\pgfsetbuttcap%
\pgfsetroundjoin%
\definecolor{currentfill}{rgb}{0.121569,0.466667,0.705882}%
\pgfsetfillcolor{currentfill}%
\pgfsetfillopacity{0.715499}%
\pgfsetlinewidth{1.003750pt}%
\definecolor{currentstroke}{rgb}{0.121569,0.466667,0.705882}%
\pgfsetstrokecolor{currentstroke}%
\pgfsetstrokeopacity{0.715499}%
\pgfsetdash{}{0pt}%
\pgfpathmoveto{\pgfqpoint{3.260493in}{2.674429in}}%
\pgfpathcurveto{\pgfqpoint{3.268729in}{2.674429in}}{\pgfqpoint{3.276629in}{2.677702in}}{\pgfqpoint{3.282453in}{2.683526in}}%
\pgfpathcurveto{\pgfqpoint{3.288277in}{2.689350in}}{\pgfqpoint{3.291549in}{2.697250in}}{\pgfqpoint{3.291549in}{2.705486in}}%
\pgfpathcurveto{\pgfqpoint{3.291549in}{2.713722in}}{\pgfqpoint{3.288277in}{2.721622in}}{\pgfqpoint{3.282453in}{2.727446in}}%
\pgfpathcurveto{\pgfqpoint{3.276629in}{2.733270in}}{\pgfqpoint{3.268729in}{2.736542in}}{\pgfqpoint{3.260493in}{2.736542in}}%
\pgfpathcurveto{\pgfqpoint{3.252256in}{2.736542in}}{\pgfqpoint{3.244356in}{2.733270in}}{\pgfqpoint{3.238532in}{2.727446in}}%
\pgfpathcurveto{\pgfqpoint{3.232708in}{2.721622in}}{\pgfqpoint{3.229436in}{2.713722in}}{\pgfqpoint{3.229436in}{2.705486in}}%
\pgfpathcurveto{\pgfqpoint{3.229436in}{2.697250in}}{\pgfqpoint{3.232708in}{2.689350in}}{\pgfqpoint{3.238532in}{2.683526in}}%
\pgfpathcurveto{\pgfqpoint{3.244356in}{2.677702in}}{\pgfqpoint{3.252256in}{2.674429in}}{\pgfqpoint{3.260493in}{2.674429in}}%
\pgfpathclose%
\pgfusepath{stroke,fill}%
\end{pgfscope}%
\begin{pgfscope}%
\pgfpathrectangle{\pgfqpoint{0.100000in}{0.220728in}}{\pgfqpoint{3.696000in}{3.696000in}}%
\pgfusepath{clip}%
\pgfsetbuttcap%
\pgfsetroundjoin%
\definecolor{currentfill}{rgb}{0.121569,0.466667,0.705882}%
\pgfsetfillcolor{currentfill}%
\pgfsetfillopacity{0.716332}%
\pgfsetlinewidth{1.003750pt}%
\definecolor{currentstroke}{rgb}{0.121569,0.466667,0.705882}%
\pgfsetstrokecolor{currentstroke}%
\pgfsetstrokeopacity{0.716332}%
\pgfsetdash{}{0pt}%
\pgfpathmoveto{\pgfqpoint{3.258673in}{2.669676in}}%
\pgfpathcurveto{\pgfqpoint{3.266909in}{2.669676in}}{\pgfqpoint{3.274809in}{2.672949in}}{\pgfqpoint{3.280633in}{2.678773in}}%
\pgfpathcurveto{\pgfqpoint{3.286457in}{2.684597in}}{\pgfqpoint{3.289730in}{2.692497in}}{\pgfqpoint{3.289730in}{2.700733in}}%
\pgfpathcurveto{\pgfqpoint{3.289730in}{2.708969in}}{\pgfqpoint{3.286457in}{2.716869in}}{\pgfqpoint{3.280633in}{2.722693in}}%
\pgfpathcurveto{\pgfqpoint{3.274809in}{2.728517in}}{\pgfqpoint{3.266909in}{2.731789in}}{\pgfqpoint{3.258673in}{2.731789in}}%
\pgfpathcurveto{\pgfqpoint{3.250437in}{2.731789in}}{\pgfqpoint{3.242537in}{2.728517in}}{\pgfqpoint{3.236713in}{2.722693in}}%
\pgfpathcurveto{\pgfqpoint{3.230889in}{2.716869in}}{\pgfqpoint{3.227617in}{2.708969in}}{\pgfqpoint{3.227617in}{2.700733in}}%
\pgfpathcurveto{\pgfqpoint{3.227617in}{2.692497in}}{\pgfqpoint{3.230889in}{2.684597in}}{\pgfqpoint{3.236713in}{2.678773in}}%
\pgfpathcurveto{\pgfqpoint{3.242537in}{2.672949in}}{\pgfqpoint{3.250437in}{2.669676in}}{\pgfqpoint{3.258673in}{2.669676in}}%
\pgfpathclose%
\pgfusepath{stroke,fill}%
\end{pgfscope}%
\begin{pgfscope}%
\pgfpathrectangle{\pgfqpoint{0.100000in}{0.220728in}}{\pgfqpoint{3.696000in}{3.696000in}}%
\pgfusepath{clip}%
\pgfsetbuttcap%
\pgfsetroundjoin%
\definecolor{currentfill}{rgb}{0.121569,0.466667,0.705882}%
\pgfsetfillcolor{currentfill}%
\pgfsetfillopacity{0.716808}%
\pgfsetlinewidth{1.003750pt}%
\definecolor{currentstroke}{rgb}{0.121569,0.466667,0.705882}%
\pgfsetstrokecolor{currentstroke}%
\pgfsetstrokeopacity{0.716808}%
\pgfsetdash{}{0pt}%
\pgfpathmoveto{\pgfqpoint{3.257264in}{2.667631in}}%
\pgfpathcurveto{\pgfqpoint{3.265501in}{2.667631in}}{\pgfqpoint{3.273401in}{2.670903in}}{\pgfqpoint{3.279225in}{2.676727in}}%
\pgfpathcurveto{\pgfqpoint{3.285049in}{2.682551in}}{\pgfqpoint{3.288321in}{2.690451in}}{\pgfqpoint{3.288321in}{2.698687in}}%
\pgfpathcurveto{\pgfqpoint{3.288321in}{2.706923in}}{\pgfqpoint{3.285049in}{2.714823in}}{\pgfqpoint{3.279225in}{2.720647in}}%
\pgfpathcurveto{\pgfqpoint{3.273401in}{2.726471in}}{\pgfqpoint{3.265501in}{2.729744in}}{\pgfqpoint{3.257264in}{2.729744in}}%
\pgfpathcurveto{\pgfqpoint{3.249028in}{2.729744in}}{\pgfqpoint{3.241128in}{2.726471in}}{\pgfqpoint{3.235304in}{2.720647in}}%
\pgfpathcurveto{\pgfqpoint{3.229480in}{2.714823in}}{\pgfqpoint{3.226208in}{2.706923in}}{\pgfqpoint{3.226208in}{2.698687in}}%
\pgfpathcurveto{\pgfqpoint{3.226208in}{2.690451in}}{\pgfqpoint{3.229480in}{2.682551in}}{\pgfqpoint{3.235304in}{2.676727in}}%
\pgfpathcurveto{\pgfqpoint{3.241128in}{2.670903in}}{\pgfqpoint{3.249028in}{2.667631in}}{\pgfqpoint{3.257264in}{2.667631in}}%
\pgfpathclose%
\pgfusepath{stroke,fill}%
\end{pgfscope}%
\begin{pgfscope}%
\pgfpathrectangle{\pgfqpoint{0.100000in}{0.220728in}}{\pgfqpoint{3.696000in}{3.696000in}}%
\pgfusepath{clip}%
\pgfsetbuttcap%
\pgfsetroundjoin%
\definecolor{currentfill}{rgb}{0.121569,0.466667,0.705882}%
\pgfsetfillcolor{currentfill}%
\pgfsetfillopacity{0.717074}%
\pgfsetlinewidth{1.003750pt}%
\definecolor{currentstroke}{rgb}{0.121569,0.466667,0.705882}%
\pgfsetstrokecolor{currentstroke}%
\pgfsetstrokeopacity{0.717074}%
\pgfsetdash{}{0pt}%
\pgfpathmoveto{\pgfqpoint{3.256491in}{2.666520in}}%
\pgfpathcurveto{\pgfqpoint{3.264727in}{2.666520in}}{\pgfqpoint{3.272627in}{2.669792in}}{\pgfqpoint{3.278451in}{2.675616in}}%
\pgfpathcurveto{\pgfqpoint{3.284275in}{2.681440in}}{\pgfqpoint{3.287547in}{2.689340in}}{\pgfqpoint{3.287547in}{2.697576in}}%
\pgfpathcurveto{\pgfqpoint{3.287547in}{2.705812in}}{\pgfqpoint{3.284275in}{2.713712in}}{\pgfqpoint{3.278451in}{2.719536in}}%
\pgfpathcurveto{\pgfqpoint{3.272627in}{2.725360in}}{\pgfqpoint{3.264727in}{2.728633in}}{\pgfqpoint{3.256491in}{2.728633in}}%
\pgfpathcurveto{\pgfqpoint{3.248255in}{2.728633in}}{\pgfqpoint{3.240355in}{2.725360in}}{\pgfqpoint{3.234531in}{2.719536in}}%
\pgfpathcurveto{\pgfqpoint{3.228707in}{2.713712in}}{\pgfqpoint{3.225434in}{2.705812in}}{\pgfqpoint{3.225434in}{2.697576in}}%
\pgfpathcurveto{\pgfqpoint{3.225434in}{2.689340in}}{\pgfqpoint{3.228707in}{2.681440in}}{\pgfqpoint{3.234531in}{2.675616in}}%
\pgfpathcurveto{\pgfqpoint{3.240355in}{2.669792in}}{\pgfqpoint{3.248255in}{2.666520in}}{\pgfqpoint{3.256491in}{2.666520in}}%
\pgfpathclose%
\pgfusepath{stroke,fill}%
\end{pgfscope}%
\begin{pgfscope}%
\pgfpathrectangle{\pgfqpoint{0.100000in}{0.220728in}}{\pgfqpoint{3.696000in}{3.696000in}}%
\pgfusepath{clip}%
\pgfsetbuttcap%
\pgfsetroundjoin%
\definecolor{currentfill}{rgb}{0.121569,0.466667,0.705882}%
\pgfsetfillcolor{currentfill}%
\pgfsetfillopacity{0.717212}%
\pgfsetlinewidth{1.003750pt}%
\definecolor{currentstroke}{rgb}{0.121569,0.466667,0.705882}%
\pgfsetstrokecolor{currentstroke}%
\pgfsetstrokeopacity{0.717212}%
\pgfsetdash{}{0pt}%
\pgfpathmoveto{\pgfqpoint{3.256190in}{2.665727in}}%
\pgfpathcurveto{\pgfqpoint{3.264426in}{2.665727in}}{\pgfqpoint{3.272326in}{2.668999in}}{\pgfqpoint{3.278150in}{2.674823in}}%
\pgfpathcurveto{\pgfqpoint{3.283974in}{2.680647in}}{\pgfqpoint{3.287246in}{2.688547in}}{\pgfqpoint{3.287246in}{2.696783in}}%
\pgfpathcurveto{\pgfqpoint{3.287246in}{2.705019in}}{\pgfqpoint{3.283974in}{2.712919in}}{\pgfqpoint{3.278150in}{2.718743in}}%
\pgfpathcurveto{\pgfqpoint{3.272326in}{2.724567in}}{\pgfqpoint{3.264426in}{2.727840in}}{\pgfqpoint{3.256190in}{2.727840in}}%
\pgfpathcurveto{\pgfqpoint{3.247953in}{2.727840in}}{\pgfqpoint{3.240053in}{2.724567in}}{\pgfqpoint{3.234230in}{2.718743in}}%
\pgfpathcurveto{\pgfqpoint{3.228406in}{2.712919in}}{\pgfqpoint{3.225133in}{2.705019in}}{\pgfqpoint{3.225133in}{2.696783in}}%
\pgfpathcurveto{\pgfqpoint{3.225133in}{2.688547in}}{\pgfqpoint{3.228406in}{2.680647in}}{\pgfqpoint{3.234230in}{2.674823in}}%
\pgfpathcurveto{\pgfqpoint{3.240053in}{2.668999in}}{\pgfqpoint{3.247953in}{2.665727in}}{\pgfqpoint{3.256190in}{2.665727in}}%
\pgfpathclose%
\pgfusepath{stroke,fill}%
\end{pgfscope}%
\begin{pgfscope}%
\pgfpathrectangle{\pgfqpoint{0.100000in}{0.220728in}}{\pgfqpoint{3.696000in}{3.696000in}}%
\pgfusepath{clip}%
\pgfsetbuttcap%
\pgfsetroundjoin%
\definecolor{currentfill}{rgb}{0.121569,0.466667,0.705882}%
\pgfsetfillcolor{currentfill}%
\pgfsetfillopacity{0.717662}%
\pgfsetlinewidth{1.003750pt}%
\definecolor{currentstroke}{rgb}{0.121569,0.466667,0.705882}%
\pgfsetstrokecolor{currentstroke}%
\pgfsetstrokeopacity{0.717662}%
\pgfsetdash{}{0pt}%
\pgfpathmoveto{\pgfqpoint{3.253969in}{2.662869in}}%
\pgfpathcurveto{\pgfqpoint{3.262205in}{2.662869in}}{\pgfqpoint{3.270105in}{2.666141in}}{\pgfqpoint{3.275929in}{2.671965in}}%
\pgfpathcurveto{\pgfqpoint{3.281753in}{2.677789in}}{\pgfqpoint{3.285025in}{2.685689in}}{\pgfqpoint{3.285025in}{2.693925in}}%
\pgfpathcurveto{\pgfqpoint{3.285025in}{2.702161in}}{\pgfqpoint{3.281753in}{2.710061in}}{\pgfqpoint{3.275929in}{2.715885in}}%
\pgfpathcurveto{\pgfqpoint{3.270105in}{2.721709in}}{\pgfqpoint{3.262205in}{2.724982in}}{\pgfqpoint{3.253969in}{2.724982in}}%
\pgfpathcurveto{\pgfqpoint{3.245732in}{2.724982in}}{\pgfqpoint{3.237832in}{2.721709in}}{\pgfqpoint{3.232009in}{2.715885in}}%
\pgfpathcurveto{\pgfqpoint{3.226185in}{2.710061in}}{\pgfqpoint{3.222912in}{2.702161in}}{\pgfqpoint{3.222912in}{2.693925in}}%
\pgfpathcurveto{\pgfqpoint{3.222912in}{2.685689in}}{\pgfqpoint{3.226185in}{2.677789in}}{\pgfqpoint{3.232009in}{2.671965in}}%
\pgfpathcurveto{\pgfqpoint{3.237832in}{2.666141in}}{\pgfqpoint{3.245732in}{2.662869in}}{\pgfqpoint{3.253969in}{2.662869in}}%
\pgfpathclose%
\pgfusepath{stroke,fill}%
\end{pgfscope}%
\begin{pgfscope}%
\pgfpathrectangle{\pgfqpoint{0.100000in}{0.220728in}}{\pgfqpoint{3.696000in}{3.696000in}}%
\pgfusepath{clip}%
\pgfsetbuttcap%
\pgfsetroundjoin%
\definecolor{currentfill}{rgb}{0.121569,0.466667,0.705882}%
\pgfsetfillcolor{currentfill}%
\pgfsetfillopacity{0.718004}%
\pgfsetlinewidth{1.003750pt}%
\definecolor{currentstroke}{rgb}{0.121569,0.466667,0.705882}%
\pgfsetstrokecolor{currentstroke}%
\pgfsetstrokeopacity{0.718004}%
\pgfsetdash{}{0pt}%
\pgfpathmoveto{\pgfqpoint{3.253130in}{2.661110in}}%
\pgfpathcurveto{\pgfqpoint{3.261367in}{2.661110in}}{\pgfqpoint{3.269267in}{2.664382in}}{\pgfqpoint{3.275091in}{2.670206in}}%
\pgfpathcurveto{\pgfqpoint{3.280915in}{2.676030in}}{\pgfqpoint{3.284187in}{2.683930in}}{\pgfqpoint{3.284187in}{2.692167in}}%
\pgfpathcurveto{\pgfqpoint{3.284187in}{2.700403in}}{\pgfqpoint{3.280915in}{2.708303in}}{\pgfqpoint{3.275091in}{2.714127in}}%
\pgfpathcurveto{\pgfqpoint{3.269267in}{2.719951in}}{\pgfqpoint{3.261367in}{2.723223in}}{\pgfqpoint{3.253130in}{2.723223in}}%
\pgfpathcurveto{\pgfqpoint{3.244894in}{2.723223in}}{\pgfqpoint{3.236994in}{2.719951in}}{\pgfqpoint{3.231170in}{2.714127in}}%
\pgfpathcurveto{\pgfqpoint{3.225346in}{2.708303in}}{\pgfqpoint{3.222074in}{2.700403in}}{\pgfqpoint{3.222074in}{2.692167in}}%
\pgfpathcurveto{\pgfqpoint{3.222074in}{2.683930in}}{\pgfqpoint{3.225346in}{2.676030in}}{\pgfqpoint{3.231170in}{2.670206in}}%
\pgfpathcurveto{\pgfqpoint{3.236994in}{2.664382in}}{\pgfqpoint{3.244894in}{2.661110in}}{\pgfqpoint{3.253130in}{2.661110in}}%
\pgfpathclose%
\pgfusepath{stroke,fill}%
\end{pgfscope}%
\begin{pgfscope}%
\pgfpathrectangle{\pgfqpoint{0.100000in}{0.220728in}}{\pgfqpoint{3.696000in}{3.696000in}}%
\pgfusepath{clip}%
\pgfsetbuttcap%
\pgfsetroundjoin%
\definecolor{currentfill}{rgb}{0.121569,0.466667,0.705882}%
\pgfsetfillcolor{currentfill}%
\pgfsetfillopacity{0.718177}%
\pgfsetlinewidth{1.003750pt}%
\definecolor{currentstroke}{rgb}{0.121569,0.466667,0.705882}%
\pgfsetstrokecolor{currentstroke}%
\pgfsetstrokeopacity{0.718177}%
\pgfsetdash{}{0pt}%
\pgfpathmoveto{\pgfqpoint{3.252655in}{2.660098in}}%
\pgfpathcurveto{\pgfqpoint{3.260891in}{2.660098in}}{\pgfqpoint{3.268791in}{2.663370in}}{\pgfqpoint{3.274615in}{2.669194in}}%
\pgfpathcurveto{\pgfqpoint{3.280439in}{2.675018in}}{\pgfqpoint{3.283711in}{2.682918in}}{\pgfqpoint{3.283711in}{2.691154in}}%
\pgfpathcurveto{\pgfqpoint{3.283711in}{2.699390in}}{\pgfqpoint{3.280439in}{2.707290in}}{\pgfqpoint{3.274615in}{2.713114in}}%
\pgfpathcurveto{\pgfqpoint{3.268791in}{2.718938in}}{\pgfqpoint{3.260891in}{2.722211in}}{\pgfqpoint{3.252655in}{2.722211in}}%
\pgfpathcurveto{\pgfqpoint{3.244419in}{2.722211in}}{\pgfqpoint{3.236519in}{2.718938in}}{\pgfqpoint{3.230695in}{2.713114in}}%
\pgfpathcurveto{\pgfqpoint{3.224871in}{2.707290in}}{\pgfqpoint{3.221598in}{2.699390in}}{\pgfqpoint{3.221598in}{2.691154in}}%
\pgfpathcurveto{\pgfqpoint{3.221598in}{2.682918in}}{\pgfqpoint{3.224871in}{2.675018in}}{\pgfqpoint{3.230695in}{2.669194in}}%
\pgfpathcurveto{\pgfqpoint{3.236519in}{2.663370in}}{\pgfqpoint{3.244419in}{2.660098in}}{\pgfqpoint{3.252655in}{2.660098in}}%
\pgfpathclose%
\pgfusepath{stroke,fill}%
\end{pgfscope}%
\begin{pgfscope}%
\pgfpathrectangle{\pgfqpoint{0.100000in}{0.220728in}}{\pgfqpoint{3.696000in}{3.696000in}}%
\pgfusepath{clip}%
\pgfsetbuttcap%
\pgfsetroundjoin%
\definecolor{currentfill}{rgb}{0.121569,0.466667,0.705882}%
\pgfsetfillcolor{currentfill}%
\pgfsetfillopacity{0.718405}%
\pgfsetlinewidth{1.003750pt}%
\definecolor{currentstroke}{rgb}{0.121569,0.466667,0.705882}%
\pgfsetstrokecolor{currentstroke}%
\pgfsetstrokeopacity{0.718405}%
\pgfsetdash{}{0pt}%
\pgfpathmoveto{\pgfqpoint{0.925771in}{1.301443in}}%
\pgfpathcurveto{\pgfqpoint{0.934007in}{1.301443in}}{\pgfqpoint{0.941907in}{1.304715in}}{\pgfqpoint{0.947731in}{1.310539in}}%
\pgfpathcurveto{\pgfqpoint{0.953555in}{1.316363in}}{\pgfqpoint{0.956827in}{1.324263in}}{\pgfqpoint{0.956827in}{1.332499in}}%
\pgfpathcurveto{\pgfqpoint{0.956827in}{1.340735in}}{\pgfqpoint{0.953555in}{1.348636in}}{\pgfqpoint{0.947731in}{1.354459in}}%
\pgfpathcurveto{\pgfqpoint{0.941907in}{1.360283in}}{\pgfqpoint{0.934007in}{1.363556in}}{\pgfqpoint{0.925771in}{1.363556in}}%
\pgfpathcurveto{\pgfqpoint{0.917534in}{1.363556in}}{\pgfqpoint{0.909634in}{1.360283in}}{\pgfqpoint{0.903810in}{1.354459in}}%
\pgfpathcurveto{\pgfqpoint{0.897986in}{1.348636in}}{\pgfqpoint{0.894714in}{1.340735in}}{\pgfqpoint{0.894714in}{1.332499in}}%
\pgfpathcurveto{\pgfqpoint{0.894714in}{1.324263in}}{\pgfqpoint{0.897986in}{1.316363in}}{\pgfqpoint{0.903810in}{1.310539in}}%
\pgfpathcurveto{\pgfqpoint{0.909634in}{1.304715in}}{\pgfqpoint{0.917534in}{1.301443in}}{\pgfqpoint{0.925771in}{1.301443in}}%
\pgfpathclose%
\pgfusepath{stroke,fill}%
\end{pgfscope}%
\begin{pgfscope}%
\pgfpathrectangle{\pgfqpoint{0.100000in}{0.220728in}}{\pgfqpoint{3.696000in}{3.696000in}}%
\pgfusepath{clip}%
\pgfsetbuttcap%
\pgfsetroundjoin%
\definecolor{currentfill}{rgb}{0.121569,0.466667,0.705882}%
\pgfsetfillcolor{currentfill}%
\pgfsetfillopacity{0.718471}%
\pgfsetlinewidth{1.003750pt}%
\definecolor{currentstroke}{rgb}{0.121569,0.466667,0.705882}%
\pgfsetstrokecolor{currentstroke}%
\pgfsetstrokeopacity{0.718471}%
\pgfsetdash{}{0pt}%
\pgfpathmoveto{\pgfqpoint{3.251652in}{2.658790in}}%
\pgfpathcurveto{\pgfqpoint{3.259888in}{2.658790in}}{\pgfqpoint{3.267788in}{2.662062in}}{\pgfqpoint{3.273612in}{2.667886in}}%
\pgfpathcurveto{\pgfqpoint{3.279436in}{2.673710in}}{\pgfqpoint{3.282708in}{2.681610in}}{\pgfqpoint{3.282708in}{2.689846in}}%
\pgfpathcurveto{\pgfqpoint{3.282708in}{2.698082in}}{\pgfqpoint{3.279436in}{2.705983in}}{\pgfqpoint{3.273612in}{2.711806in}}%
\pgfpathcurveto{\pgfqpoint{3.267788in}{2.717630in}}{\pgfqpoint{3.259888in}{2.720903in}}{\pgfqpoint{3.251652in}{2.720903in}}%
\pgfpathcurveto{\pgfqpoint{3.243415in}{2.720903in}}{\pgfqpoint{3.235515in}{2.717630in}}{\pgfqpoint{3.229691in}{2.711806in}}%
\pgfpathcurveto{\pgfqpoint{3.223867in}{2.705983in}}{\pgfqpoint{3.220595in}{2.698082in}}{\pgfqpoint{3.220595in}{2.689846in}}%
\pgfpathcurveto{\pgfqpoint{3.220595in}{2.681610in}}{\pgfqpoint{3.223867in}{2.673710in}}{\pgfqpoint{3.229691in}{2.667886in}}%
\pgfpathcurveto{\pgfqpoint{3.235515in}{2.662062in}}{\pgfqpoint{3.243415in}{2.658790in}}{\pgfqpoint{3.251652in}{2.658790in}}%
\pgfpathclose%
\pgfusepath{stroke,fill}%
\end{pgfscope}%
\begin{pgfscope}%
\pgfpathrectangle{\pgfqpoint{0.100000in}{0.220728in}}{\pgfqpoint{3.696000in}{3.696000in}}%
\pgfusepath{clip}%
\pgfsetbuttcap%
\pgfsetroundjoin%
\definecolor{currentfill}{rgb}{0.121569,0.466667,0.705882}%
\pgfsetfillcolor{currentfill}%
\pgfsetfillopacity{0.719004}%
\pgfsetlinewidth{1.003750pt}%
\definecolor{currentstroke}{rgb}{0.121569,0.466667,0.705882}%
\pgfsetstrokecolor{currentstroke}%
\pgfsetstrokeopacity{0.719004}%
\pgfsetdash{}{0pt}%
\pgfpathmoveto{\pgfqpoint{3.250301in}{2.656543in}}%
\pgfpathcurveto{\pgfqpoint{3.258537in}{2.656543in}}{\pgfqpoint{3.266437in}{2.659815in}}{\pgfqpoint{3.272261in}{2.665639in}}%
\pgfpathcurveto{\pgfqpoint{3.278085in}{2.671463in}}{\pgfqpoint{3.281357in}{2.679363in}}{\pgfqpoint{3.281357in}{2.687600in}}%
\pgfpathcurveto{\pgfqpoint{3.281357in}{2.695836in}}{\pgfqpoint{3.278085in}{2.703736in}}{\pgfqpoint{3.272261in}{2.709560in}}%
\pgfpathcurveto{\pgfqpoint{3.266437in}{2.715384in}}{\pgfqpoint{3.258537in}{2.718656in}}{\pgfqpoint{3.250301in}{2.718656in}}%
\pgfpathcurveto{\pgfqpoint{3.242064in}{2.718656in}}{\pgfqpoint{3.234164in}{2.715384in}}{\pgfqpoint{3.228340in}{2.709560in}}%
\pgfpathcurveto{\pgfqpoint{3.222516in}{2.703736in}}{\pgfqpoint{3.219244in}{2.695836in}}{\pgfqpoint{3.219244in}{2.687600in}}%
\pgfpathcurveto{\pgfqpoint{3.219244in}{2.679363in}}{\pgfqpoint{3.222516in}{2.671463in}}{\pgfqpoint{3.228340in}{2.665639in}}%
\pgfpathcurveto{\pgfqpoint{3.234164in}{2.659815in}}{\pgfqpoint{3.242064in}{2.656543in}}{\pgfqpoint{3.250301in}{2.656543in}}%
\pgfpathclose%
\pgfusepath{stroke,fill}%
\end{pgfscope}%
\begin{pgfscope}%
\pgfpathrectangle{\pgfqpoint{0.100000in}{0.220728in}}{\pgfqpoint{3.696000in}{3.696000in}}%
\pgfusepath{clip}%
\pgfsetbuttcap%
\pgfsetroundjoin%
\definecolor{currentfill}{rgb}{0.121569,0.466667,0.705882}%
\pgfsetfillcolor{currentfill}%
\pgfsetfillopacity{0.719639}%
\pgfsetlinewidth{1.003750pt}%
\definecolor{currentstroke}{rgb}{0.121569,0.466667,0.705882}%
\pgfsetstrokecolor{currentstroke}%
\pgfsetstrokeopacity{0.719639}%
\pgfsetdash{}{0pt}%
\pgfpathmoveto{\pgfqpoint{3.249195in}{2.652981in}}%
\pgfpathcurveto{\pgfqpoint{3.257431in}{2.652981in}}{\pgfqpoint{3.265331in}{2.656253in}}{\pgfqpoint{3.271155in}{2.662077in}}%
\pgfpathcurveto{\pgfqpoint{3.276979in}{2.667901in}}{\pgfqpoint{3.280252in}{2.675801in}}{\pgfqpoint{3.280252in}{2.684037in}}%
\pgfpathcurveto{\pgfqpoint{3.280252in}{2.692274in}}{\pgfqpoint{3.276979in}{2.700174in}}{\pgfqpoint{3.271155in}{2.705998in}}%
\pgfpathcurveto{\pgfqpoint{3.265331in}{2.711822in}}{\pgfqpoint{3.257431in}{2.715094in}}{\pgfqpoint{3.249195in}{2.715094in}}%
\pgfpathcurveto{\pgfqpoint{3.240959in}{2.715094in}}{\pgfqpoint{3.233059in}{2.711822in}}{\pgfqpoint{3.227235in}{2.705998in}}%
\pgfpathcurveto{\pgfqpoint{3.221411in}{2.700174in}}{\pgfqpoint{3.218139in}{2.692274in}}{\pgfqpoint{3.218139in}{2.684037in}}%
\pgfpathcurveto{\pgfqpoint{3.218139in}{2.675801in}}{\pgfqpoint{3.221411in}{2.667901in}}{\pgfqpoint{3.227235in}{2.662077in}}%
\pgfpathcurveto{\pgfqpoint{3.233059in}{2.656253in}}{\pgfqpoint{3.240959in}{2.652981in}}{\pgfqpoint{3.249195in}{2.652981in}}%
\pgfpathclose%
\pgfusepath{stroke,fill}%
\end{pgfscope}%
\begin{pgfscope}%
\pgfpathrectangle{\pgfqpoint{0.100000in}{0.220728in}}{\pgfqpoint{3.696000in}{3.696000in}}%
\pgfusepath{clip}%
\pgfsetbuttcap%
\pgfsetroundjoin%
\definecolor{currentfill}{rgb}{0.121569,0.466667,0.705882}%
\pgfsetfillcolor{currentfill}%
\pgfsetfillopacity{0.720492}%
\pgfsetlinewidth{1.003750pt}%
\definecolor{currentstroke}{rgb}{0.121569,0.466667,0.705882}%
\pgfsetstrokecolor{currentstroke}%
\pgfsetstrokeopacity{0.720492}%
\pgfsetdash{}{0pt}%
\pgfpathmoveto{\pgfqpoint{3.246124in}{2.648701in}}%
\pgfpathcurveto{\pgfqpoint{3.254361in}{2.648701in}}{\pgfqpoint{3.262261in}{2.651974in}}{\pgfqpoint{3.268085in}{2.657797in}}%
\pgfpathcurveto{\pgfqpoint{3.273909in}{2.663621in}}{\pgfqpoint{3.277181in}{2.671521in}}{\pgfqpoint{3.277181in}{2.679758in}}%
\pgfpathcurveto{\pgfqpoint{3.277181in}{2.687994in}}{\pgfqpoint{3.273909in}{2.695894in}}{\pgfqpoint{3.268085in}{2.701718in}}%
\pgfpathcurveto{\pgfqpoint{3.262261in}{2.707542in}}{\pgfqpoint{3.254361in}{2.710814in}}{\pgfqpoint{3.246124in}{2.710814in}}%
\pgfpathcurveto{\pgfqpoint{3.237888in}{2.710814in}}{\pgfqpoint{3.229988in}{2.707542in}}{\pgfqpoint{3.224164in}{2.701718in}}%
\pgfpathcurveto{\pgfqpoint{3.218340in}{2.695894in}}{\pgfqpoint{3.215068in}{2.687994in}}{\pgfqpoint{3.215068in}{2.679758in}}%
\pgfpathcurveto{\pgfqpoint{3.215068in}{2.671521in}}{\pgfqpoint{3.218340in}{2.663621in}}{\pgfqpoint{3.224164in}{2.657797in}}%
\pgfpathcurveto{\pgfqpoint{3.229988in}{2.651974in}}{\pgfqpoint{3.237888in}{2.648701in}}{\pgfqpoint{3.246124in}{2.648701in}}%
\pgfpathclose%
\pgfusepath{stroke,fill}%
\end{pgfscope}%
\begin{pgfscope}%
\pgfpathrectangle{\pgfqpoint{0.100000in}{0.220728in}}{\pgfqpoint{3.696000in}{3.696000in}}%
\pgfusepath{clip}%
\pgfsetbuttcap%
\pgfsetroundjoin%
\definecolor{currentfill}{rgb}{0.121569,0.466667,0.705882}%
\pgfsetfillcolor{currentfill}%
\pgfsetfillopacity{0.721073}%
\pgfsetlinewidth{1.003750pt}%
\definecolor{currentstroke}{rgb}{0.121569,0.466667,0.705882}%
\pgfsetstrokecolor{currentstroke}%
\pgfsetstrokeopacity{0.721073}%
\pgfsetdash{}{0pt}%
\pgfpathmoveto{\pgfqpoint{3.244624in}{2.646537in}}%
\pgfpathcurveto{\pgfqpoint{3.252860in}{2.646537in}}{\pgfqpoint{3.260760in}{2.649810in}}{\pgfqpoint{3.266584in}{2.655634in}}%
\pgfpathcurveto{\pgfqpoint{3.272408in}{2.661457in}}{\pgfqpoint{3.275681in}{2.669358in}}{\pgfqpoint{3.275681in}{2.677594in}}%
\pgfpathcurveto{\pgfqpoint{3.275681in}{2.685830in}}{\pgfqpoint{3.272408in}{2.693730in}}{\pgfqpoint{3.266584in}{2.699554in}}%
\pgfpathcurveto{\pgfqpoint{3.260760in}{2.705378in}}{\pgfqpoint{3.252860in}{2.708650in}}{\pgfqpoint{3.244624in}{2.708650in}}%
\pgfpathcurveto{\pgfqpoint{3.236388in}{2.708650in}}{\pgfqpoint{3.228488in}{2.705378in}}{\pgfqpoint{3.222664in}{2.699554in}}%
\pgfpathcurveto{\pgfqpoint{3.216840in}{2.693730in}}{\pgfqpoint{3.213568in}{2.685830in}}{\pgfqpoint{3.213568in}{2.677594in}}%
\pgfpathcurveto{\pgfqpoint{3.213568in}{2.669358in}}{\pgfqpoint{3.216840in}{2.661457in}}{\pgfqpoint{3.222664in}{2.655634in}}%
\pgfpathcurveto{\pgfqpoint{3.228488in}{2.649810in}}{\pgfqpoint{3.236388in}{2.646537in}}{\pgfqpoint{3.244624in}{2.646537in}}%
\pgfpathclose%
\pgfusepath{stroke,fill}%
\end{pgfscope}%
\begin{pgfscope}%
\pgfpathrectangle{\pgfqpoint{0.100000in}{0.220728in}}{\pgfqpoint{3.696000in}{3.696000in}}%
\pgfusepath{clip}%
\pgfsetbuttcap%
\pgfsetroundjoin%
\definecolor{currentfill}{rgb}{0.121569,0.466667,0.705882}%
\pgfsetfillcolor{currentfill}%
\pgfsetfillopacity{0.721611}%
\pgfsetlinewidth{1.003750pt}%
\definecolor{currentstroke}{rgb}{0.121569,0.466667,0.705882}%
\pgfsetstrokecolor{currentstroke}%
\pgfsetstrokeopacity{0.721611}%
\pgfsetdash{}{0pt}%
\pgfpathmoveto{\pgfqpoint{3.242935in}{2.642846in}}%
\pgfpathcurveto{\pgfqpoint{3.251171in}{2.642846in}}{\pgfqpoint{3.259072in}{2.646118in}}{\pgfqpoint{3.264895in}{2.651942in}}%
\pgfpathcurveto{\pgfqpoint{3.270719in}{2.657766in}}{\pgfqpoint{3.273992in}{2.665666in}}{\pgfqpoint{3.273992in}{2.673902in}}%
\pgfpathcurveto{\pgfqpoint{3.273992in}{2.682139in}}{\pgfqpoint{3.270719in}{2.690039in}}{\pgfqpoint{3.264895in}{2.695863in}}%
\pgfpathcurveto{\pgfqpoint{3.259072in}{2.701686in}}{\pgfqpoint{3.251171in}{2.704959in}}{\pgfqpoint{3.242935in}{2.704959in}}%
\pgfpathcurveto{\pgfqpoint{3.234699in}{2.704959in}}{\pgfqpoint{3.226799in}{2.701686in}}{\pgfqpoint{3.220975in}{2.695863in}}%
\pgfpathcurveto{\pgfqpoint{3.215151in}{2.690039in}}{\pgfqpoint{3.211879in}{2.682139in}}{\pgfqpoint{3.211879in}{2.673902in}}%
\pgfpathcurveto{\pgfqpoint{3.211879in}{2.665666in}}{\pgfqpoint{3.215151in}{2.657766in}}{\pgfqpoint{3.220975in}{2.651942in}}%
\pgfpathcurveto{\pgfqpoint{3.226799in}{2.646118in}}{\pgfqpoint{3.234699in}{2.642846in}}{\pgfqpoint{3.242935in}{2.642846in}}%
\pgfpathclose%
\pgfusepath{stroke,fill}%
\end{pgfscope}%
\begin{pgfscope}%
\pgfpathrectangle{\pgfqpoint{0.100000in}{0.220728in}}{\pgfqpoint{3.696000in}{3.696000in}}%
\pgfusepath{clip}%
\pgfsetbuttcap%
\pgfsetroundjoin%
\definecolor{currentfill}{rgb}{0.121569,0.466667,0.705882}%
\pgfsetfillcolor{currentfill}%
\pgfsetfillopacity{0.722390}%
\pgfsetlinewidth{1.003750pt}%
\definecolor{currentstroke}{rgb}{0.121569,0.466667,0.705882}%
\pgfsetstrokecolor{currentstroke}%
\pgfsetstrokeopacity{0.722390}%
\pgfsetdash{}{0pt}%
\pgfpathmoveto{\pgfqpoint{3.238698in}{2.637558in}}%
\pgfpathcurveto{\pgfqpoint{3.246935in}{2.637558in}}{\pgfqpoint{3.254835in}{2.640831in}}{\pgfqpoint{3.260659in}{2.646654in}}%
\pgfpathcurveto{\pgfqpoint{3.266482in}{2.652478in}}{\pgfqpoint{3.269755in}{2.660378in}}{\pgfqpoint{3.269755in}{2.668615in}}%
\pgfpathcurveto{\pgfqpoint{3.269755in}{2.676851in}}{\pgfqpoint{3.266482in}{2.684751in}}{\pgfqpoint{3.260659in}{2.690575in}}%
\pgfpathcurveto{\pgfqpoint{3.254835in}{2.696399in}}{\pgfqpoint{3.246935in}{2.699671in}}{\pgfqpoint{3.238698in}{2.699671in}}%
\pgfpathcurveto{\pgfqpoint{3.230462in}{2.699671in}}{\pgfqpoint{3.222562in}{2.696399in}}{\pgfqpoint{3.216738in}{2.690575in}}%
\pgfpathcurveto{\pgfqpoint{3.210914in}{2.684751in}}{\pgfqpoint{3.207642in}{2.676851in}}{\pgfqpoint{3.207642in}{2.668615in}}%
\pgfpathcurveto{\pgfqpoint{3.207642in}{2.660378in}}{\pgfqpoint{3.210914in}{2.652478in}}{\pgfqpoint{3.216738in}{2.646654in}}%
\pgfpathcurveto{\pgfqpoint{3.222562in}{2.640831in}}{\pgfqpoint{3.230462in}{2.637558in}}{\pgfqpoint{3.238698in}{2.637558in}}%
\pgfpathclose%
\pgfusepath{stroke,fill}%
\end{pgfscope}%
\begin{pgfscope}%
\pgfpathrectangle{\pgfqpoint{0.100000in}{0.220728in}}{\pgfqpoint{3.696000in}{3.696000in}}%
\pgfusepath{clip}%
\pgfsetbuttcap%
\pgfsetroundjoin%
\definecolor{currentfill}{rgb}{0.121569,0.466667,0.705882}%
\pgfsetfillcolor{currentfill}%
\pgfsetfillopacity{0.723010}%
\pgfsetlinewidth{1.003750pt}%
\definecolor{currentstroke}{rgb}{0.121569,0.466667,0.705882}%
\pgfsetstrokecolor{currentstroke}%
\pgfsetstrokeopacity{0.723010}%
\pgfsetdash{}{0pt}%
\pgfpathmoveto{\pgfqpoint{0.944952in}{1.293166in}}%
\pgfpathcurveto{\pgfqpoint{0.953188in}{1.293166in}}{\pgfqpoint{0.961088in}{1.296438in}}{\pgfqpoint{0.966912in}{1.302262in}}%
\pgfpathcurveto{\pgfqpoint{0.972736in}{1.308086in}}{\pgfqpoint{0.976008in}{1.315986in}}{\pgfqpoint{0.976008in}{1.324223in}}%
\pgfpathcurveto{\pgfqpoint{0.976008in}{1.332459in}}{\pgfqpoint{0.972736in}{1.340359in}}{\pgfqpoint{0.966912in}{1.346183in}}%
\pgfpathcurveto{\pgfqpoint{0.961088in}{1.352007in}}{\pgfqpoint{0.953188in}{1.355279in}}{\pgfqpoint{0.944952in}{1.355279in}}%
\pgfpathcurveto{\pgfqpoint{0.936715in}{1.355279in}}{\pgfqpoint{0.928815in}{1.352007in}}{\pgfqpoint{0.922991in}{1.346183in}}%
\pgfpathcurveto{\pgfqpoint{0.917167in}{1.340359in}}{\pgfqpoint{0.913895in}{1.332459in}}{\pgfqpoint{0.913895in}{1.324223in}}%
\pgfpathcurveto{\pgfqpoint{0.913895in}{1.315986in}}{\pgfqpoint{0.917167in}{1.308086in}}{\pgfqpoint{0.922991in}{1.302262in}}%
\pgfpathcurveto{\pgfqpoint{0.928815in}{1.296438in}}{\pgfqpoint{0.936715in}{1.293166in}}{\pgfqpoint{0.944952in}{1.293166in}}%
\pgfpathclose%
\pgfusepath{stroke,fill}%
\end{pgfscope}%
\begin{pgfscope}%
\pgfpathrectangle{\pgfqpoint{0.100000in}{0.220728in}}{\pgfqpoint{3.696000in}{3.696000in}}%
\pgfusepath{clip}%
\pgfsetbuttcap%
\pgfsetroundjoin%
\definecolor{currentfill}{rgb}{0.121569,0.466667,0.705882}%
\pgfsetfillcolor{currentfill}%
\pgfsetfillopacity{0.723068}%
\pgfsetlinewidth{1.003750pt}%
\definecolor{currentstroke}{rgb}{0.121569,0.466667,0.705882}%
\pgfsetstrokecolor{currentstroke}%
\pgfsetstrokeopacity{0.723068}%
\pgfsetdash{}{0pt}%
\pgfpathmoveto{\pgfqpoint{3.236682in}{2.635131in}}%
\pgfpathcurveto{\pgfqpoint{3.244918in}{2.635131in}}{\pgfqpoint{3.252819in}{2.638404in}}{\pgfqpoint{3.258642in}{2.644228in}}%
\pgfpathcurveto{\pgfqpoint{3.264466in}{2.650052in}}{\pgfqpoint{3.267739in}{2.657952in}}{\pgfqpoint{3.267739in}{2.666188in}}%
\pgfpathcurveto{\pgfqpoint{3.267739in}{2.674424in}}{\pgfqpoint{3.264466in}{2.682324in}}{\pgfqpoint{3.258642in}{2.688148in}}%
\pgfpathcurveto{\pgfqpoint{3.252819in}{2.693972in}}{\pgfqpoint{3.244918in}{2.697244in}}{\pgfqpoint{3.236682in}{2.697244in}}%
\pgfpathcurveto{\pgfqpoint{3.228446in}{2.697244in}}{\pgfqpoint{3.220546in}{2.693972in}}{\pgfqpoint{3.214722in}{2.688148in}}%
\pgfpathcurveto{\pgfqpoint{3.208898in}{2.682324in}}{\pgfqpoint{3.205626in}{2.674424in}}{\pgfqpoint{3.205626in}{2.666188in}}%
\pgfpathcurveto{\pgfqpoint{3.205626in}{2.657952in}}{\pgfqpoint{3.208898in}{2.650052in}}{\pgfqpoint{3.214722in}{2.644228in}}%
\pgfpathcurveto{\pgfqpoint{3.220546in}{2.638404in}}{\pgfqpoint{3.228446in}{2.635131in}}{\pgfqpoint{3.236682in}{2.635131in}}%
\pgfpathclose%
\pgfusepath{stroke,fill}%
\end{pgfscope}%
\begin{pgfscope}%
\pgfpathrectangle{\pgfqpoint{0.100000in}{0.220728in}}{\pgfqpoint{3.696000in}{3.696000in}}%
\pgfusepath{clip}%
\pgfsetbuttcap%
\pgfsetroundjoin%
\definecolor{currentfill}{rgb}{0.121569,0.466667,0.705882}%
\pgfsetfillcolor{currentfill}%
\pgfsetfillopacity{0.723327}%
\pgfsetlinewidth{1.003750pt}%
\definecolor{currentstroke}{rgb}{0.121569,0.466667,0.705882}%
\pgfsetstrokecolor{currentstroke}%
\pgfsetstrokeopacity{0.723327}%
\pgfsetdash{}{0pt}%
\pgfpathmoveto{\pgfqpoint{3.235763in}{2.633024in}}%
\pgfpathcurveto{\pgfqpoint{3.244000in}{2.633024in}}{\pgfqpoint{3.251900in}{2.636297in}}{\pgfqpoint{3.257724in}{2.642121in}}%
\pgfpathcurveto{\pgfqpoint{3.263547in}{2.647944in}}{\pgfqpoint{3.266820in}{2.655844in}}{\pgfqpoint{3.266820in}{2.664081in}}%
\pgfpathcurveto{\pgfqpoint{3.266820in}{2.672317in}}{\pgfqpoint{3.263547in}{2.680217in}}{\pgfqpoint{3.257724in}{2.686041in}}%
\pgfpathcurveto{\pgfqpoint{3.251900in}{2.691865in}}{\pgfqpoint{3.244000in}{2.695137in}}{\pgfqpoint{3.235763in}{2.695137in}}%
\pgfpathcurveto{\pgfqpoint{3.227527in}{2.695137in}}{\pgfqpoint{3.219627in}{2.691865in}}{\pgfqpoint{3.213803in}{2.686041in}}%
\pgfpathcurveto{\pgfqpoint{3.207979in}{2.680217in}}{\pgfqpoint{3.204707in}{2.672317in}}{\pgfqpoint{3.204707in}{2.664081in}}%
\pgfpathcurveto{\pgfqpoint{3.204707in}{2.655844in}}{\pgfqpoint{3.207979in}{2.647944in}}{\pgfqpoint{3.213803in}{2.642121in}}%
\pgfpathcurveto{\pgfqpoint{3.219627in}{2.636297in}}{\pgfqpoint{3.227527in}{2.633024in}}{\pgfqpoint{3.235763in}{2.633024in}}%
\pgfpathclose%
\pgfusepath{stroke,fill}%
\end{pgfscope}%
\begin{pgfscope}%
\pgfpathrectangle{\pgfqpoint{0.100000in}{0.220728in}}{\pgfqpoint{3.696000in}{3.696000in}}%
\pgfusepath{clip}%
\pgfsetbuttcap%
\pgfsetroundjoin%
\definecolor{currentfill}{rgb}{0.121569,0.466667,0.705882}%
\pgfsetfillcolor{currentfill}%
\pgfsetfillopacity{0.723830}%
\pgfsetlinewidth{1.003750pt}%
\definecolor{currentstroke}{rgb}{0.121569,0.466667,0.705882}%
\pgfsetstrokecolor{currentstroke}%
\pgfsetstrokeopacity{0.723830}%
\pgfsetdash{}{0pt}%
\pgfpathmoveto{\pgfqpoint{3.232125in}{2.628455in}}%
\pgfpathcurveto{\pgfqpoint{3.240361in}{2.628455in}}{\pgfqpoint{3.248261in}{2.631727in}}{\pgfqpoint{3.254085in}{2.637551in}}%
\pgfpathcurveto{\pgfqpoint{3.259909in}{2.643375in}}{\pgfqpoint{3.263182in}{2.651275in}}{\pgfqpoint{3.263182in}{2.659511in}}%
\pgfpathcurveto{\pgfqpoint{3.263182in}{2.667747in}}{\pgfqpoint{3.259909in}{2.675647in}}{\pgfqpoint{3.254085in}{2.681471in}}%
\pgfpathcurveto{\pgfqpoint{3.248261in}{2.687295in}}{\pgfqpoint{3.240361in}{2.690568in}}{\pgfqpoint{3.232125in}{2.690568in}}%
\pgfpathcurveto{\pgfqpoint{3.223889in}{2.690568in}}{\pgfqpoint{3.215989in}{2.687295in}}{\pgfqpoint{3.210165in}{2.681471in}}%
\pgfpathcurveto{\pgfqpoint{3.204341in}{2.675647in}}{\pgfqpoint{3.201069in}{2.667747in}}{\pgfqpoint{3.201069in}{2.659511in}}%
\pgfpathcurveto{\pgfqpoint{3.201069in}{2.651275in}}{\pgfqpoint{3.204341in}{2.643375in}}{\pgfqpoint{3.210165in}{2.637551in}}%
\pgfpathcurveto{\pgfqpoint{3.215989in}{2.631727in}}{\pgfqpoint{3.223889in}{2.628455in}}{\pgfqpoint{3.232125in}{2.628455in}}%
\pgfpathclose%
\pgfusepath{stroke,fill}%
\end{pgfscope}%
\begin{pgfscope}%
\pgfpathrectangle{\pgfqpoint{0.100000in}{0.220728in}}{\pgfqpoint{3.696000in}{3.696000in}}%
\pgfusepath{clip}%
\pgfsetbuttcap%
\pgfsetroundjoin%
\definecolor{currentfill}{rgb}{0.121569,0.466667,0.705882}%
\pgfsetfillcolor{currentfill}%
\pgfsetfillopacity{0.724916}%
\pgfsetlinewidth{1.003750pt}%
\definecolor{currentstroke}{rgb}{0.121569,0.466667,0.705882}%
\pgfsetstrokecolor{currentstroke}%
\pgfsetstrokeopacity{0.724916}%
\pgfsetdash{}{0pt}%
\pgfpathmoveto{\pgfqpoint{3.229157in}{2.622935in}}%
\pgfpathcurveto{\pgfqpoint{3.237394in}{2.622935in}}{\pgfqpoint{3.245294in}{2.626207in}}{\pgfqpoint{3.251118in}{2.632031in}}%
\pgfpathcurveto{\pgfqpoint{3.256942in}{2.637855in}}{\pgfqpoint{3.260214in}{2.645755in}}{\pgfqpoint{3.260214in}{2.653991in}}%
\pgfpathcurveto{\pgfqpoint{3.260214in}{2.662227in}}{\pgfqpoint{3.256942in}{2.670128in}}{\pgfqpoint{3.251118in}{2.675951in}}%
\pgfpathcurveto{\pgfqpoint{3.245294in}{2.681775in}}{\pgfqpoint{3.237394in}{2.685048in}}{\pgfqpoint{3.229157in}{2.685048in}}%
\pgfpathcurveto{\pgfqpoint{3.220921in}{2.685048in}}{\pgfqpoint{3.213021in}{2.681775in}}{\pgfqpoint{3.207197in}{2.675951in}}%
\pgfpathcurveto{\pgfqpoint{3.201373in}{2.670128in}}{\pgfqpoint{3.198101in}{2.662227in}}{\pgfqpoint{3.198101in}{2.653991in}}%
\pgfpathcurveto{\pgfqpoint{3.198101in}{2.645755in}}{\pgfqpoint{3.201373in}{2.637855in}}{\pgfqpoint{3.207197in}{2.632031in}}%
\pgfpathcurveto{\pgfqpoint{3.213021in}{2.626207in}}{\pgfqpoint{3.220921in}{2.622935in}}{\pgfqpoint{3.229157in}{2.622935in}}%
\pgfpathclose%
\pgfusepath{stroke,fill}%
\end{pgfscope}%
\begin{pgfscope}%
\pgfpathrectangle{\pgfqpoint{0.100000in}{0.220728in}}{\pgfqpoint{3.696000in}{3.696000in}}%
\pgfusepath{clip}%
\pgfsetbuttcap%
\pgfsetroundjoin%
\definecolor{currentfill}{rgb}{0.121569,0.466667,0.705882}%
\pgfsetfillcolor{currentfill}%
\pgfsetfillopacity{0.725470}%
\pgfsetlinewidth{1.003750pt}%
\definecolor{currentstroke}{rgb}{0.121569,0.466667,0.705882}%
\pgfsetstrokecolor{currentstroke}%
\pgfsetstrokeopacity{0.725470}%
\pgfsetdash{}{0pt}%
\pgfpathmoveto{\pgfqpoint{3.227585in}{2.619633in}}%
\pgfpathcurveto{\pgfqpoint{3.235821in}{2.619633in}}{\pgfqpoint{3.243721in}{2.622906in}}{\pgfqpoint{3.249545in}{2.628730in}}%
\pgfpathcurveto{\pgfqpoint{3.255369in}{2.634554in}}{\pgfqpoint{3.258641in}{2.642454in}}{\pgfqpoint{3.258641in}{2.650690in}}%
\pgfpathcurveto{\pgfqpoint{3.258641in}{2.658926in}}{\pgfqpoint{3.255369in}{2.666826in}}{\pgfqpoint{3.249545in}{2.672650in}}%
\pgfpathcurveto{\pgfqpoint{3.243721in}{2.678474in}}{\pgfqpoint{3.235821in}{2.681746in}}{\pgfqpoint{3.227585in}{2.681746in}}%
\pgfpathcurveto{\pgfqpoint{3.219348in}{2.681746in}}{\pgfqpoint{3.211448in}{2.678474in}}{\pgfqpoint{3.205624in}{2.672650in}}%
\pgfpathcurveto{\pgfqpoint{3.199800in}{2.666826in}}{\pgfqpoint{3.196528in}{2.658926in}}{\pgfqpoint{3.196528in}{2.650690in}}%
\pgfpathcurveto{\pgfqpoint{3.196528in}{2.642454in}}{\pgfqpoint{3.199800in}{2.634554in}}{\pgfqpoint{3.205624in}{2.628730in}}%
\pgfpathcurveto{\pgfqpoint{3.211448in}{2.622906in}}{\pgfqpoint{3.219348in}{2.619633in}}{\pgfqpoint{3.227585in}{2.619633in}}%
\pgfpathclose%
\pgfusepath{stroke,fill}%
\end{pgfscope}%
\begin{pgfscope}%
\pgfpathrectangle{\pgfqpoint{0.100000in}{0.220728in}}{\pgfqpoint{3.696000in}{3.696000in}}%
\pgfusepath{clip}%
\pgfsetbuttcap%
\pgfsetroundjoin%
\definecolor{currentfill}{rgb}{0.121569,0.466667,0.705882}%
\pgfsetfillcolor{currentfill}%
\pgfsetfillopacity{0.726144}%
\pgfsetlinewidth{1.003750pt}%
\definecolor{currentstroke}{rgb}{0.121569,0.466667,0.705882}%
\pgfsetstrokecolor{currentstroke}%
\pgfsetstrokeopacity{0.726144}%
\pgfsetdash{}{0pt}%
\pgfpathmoveto{\pgfqpoint{3.224631in}{2.615997in}}%
\pgfpathcurveto{\pgfqpoint{3.232867in}{2.615997in}}{\pgfqpoint{3.240767in}{2.619269in}}{\pgfqpoint{3.246591in}{2.625093in}}%
\pgfpathcurveto{\pgfqpoint{3.252415in}{2.630917in}}{\pgfqpoint{3.255687in}{2.638817in}}{\pgfqpoint{3.255687in}{2.647054in}}%
\pgfpathcurveto{\pgfqpoint{3.255687in}{2.655290in}}{\pgfqpoint{3.252415in}{2.663190in}}{\pgfqpoint{3.246591in}{2.669014in}}%
\pgfpathcurveto{\pgfqpoint{3.240767in}{2.674838in}}{\pgfqpoint{3.232867in}{2.678110in}}{\pgfqpoint{3.224631in}{2.678110in}}%
\pgfpathcurveto{\pgfqpoint{3.216395in}{2.678110in}}{\pgfqpoint{3.208495in}{2.674838in}}{\pgfqpoint{3.202671in}{2.669014in}}%
\pgfpathcurveto{\pgfqpoint{3.196847in}{2.663190in}}{\pgfqpoint{3.193574in}{2.655290in}}{\pgfqpoint{3.193574in}{2.647054in}}%
\pgfpathcurveto{\pgfqpoint{3.193574in}{2.638817in}}{\pgfqpoint{3.196847in}{2.630917in}}{\pgfqpoint{3.202671in}{2.625093in}}%
\pgfpathcurveto{\pgfqpoint{3.208495in}{2.619269in}}{\pgfqpoint{3.216395in}{2.615997in}}{\pgfqpoint{3.224631in}{2.615997in}}%
\pgfpathclose%
\pgfusepath{stroke,fill}%
\end{pgfscope}%
\begin{pgfscope}%
\pgfpathrectangle{\pgfqpoint{0.100000in}{0.220728in}}{\pgfqpoint{3.696000in}{3.696000in}}%
\pgfusepath{clip}%
\pgfsetbuttcap%
\pgfsetroundjoin%
\definecolor{currentfill}{rgb}{0.121569,0.466667,0.705882}%
\pgfsetfillcolor{currentfill}%
\pgfsetfillopacity{0.726632}%
\pgfsetlinewidth{1.003750pt}%
\definecolor{currentstroke}{rgb}{0.121569,0.466667,0.705882}%
\pgfsetstrokecolor{currentstroke}%
\pgfsetstrokeopacity{0.726632}%
\pgfsetdash{}{0pt}%
\pgfpathmoveto{\pgfqpoint{0.963341in}{1.281569in}}%
\pgfpathcurveto{\pgfqpoint{0.971577in}{1.281569in}}{\pgfqpoint{0.979478in}{1.284841in}}{\pgfqpoint{0.985301in}{1.290665in}}%
\pgfpathcurveto{\pgfqpoint{0.991125in}{1.296489in}}{\pgfqpoint{0.994398in}{1.304389in}}{\pgfqpoint{0.994398in}{1.312625in}}%
\pgfpathcurveto{\pgfqpoint{0.994398in}{1.320861in}}{\pgfqpoint{0.991125in}{1.328762in}}{\pgfqpoint{0.985301in}{1.334585in}}%
\pgfpathcurveto{\pgfqpoint{0.979478in}{1.340409in}}{\pgfqpoint{0.971577in}{1.343682in}}{\pgfqpoint{0.963341in}{1.343682in}}%
\pgfpathcurveto{\pgfqpoint{0.955105in}{1.343682in}}{\pgfqpoint{0.947205in}{1.340409in}}{\pgfqpoint{0.941381in}{1.334585in}}%
\pgfpathcurveto{\pgfqpoint{0.935557in}{1.328762in}}{\pgfqpoint{0.932285in}{1.320861in}}{\pgfqpoint{0.932285in}{1.312625in}}%
\pgfpathcurveto{\pgfqpoint{0.932285in}{1.304389in}}{\pgfqpoint{0.935557in}{1.296489in}}{\pgfqpoint{0.941381in}{1.290665in}}%
\pgfpathcurveto{\pgfqpoint{0.947205in}{1.284841in}}{\pgfqpoint{0.955105in}{1.281569in}}{\pgfqpoint{0.963341in}{1.281569in}}%
\pgfpathclose%
\pgfusepath{stroke,fill}%
\end{pgfscope}%
\begin{pgfscope}%
\pgfpathrectangle{\pgfqpoint{0.100000in}{0.220728in}}{\pgfqpoint{3.696000in}{3.696000in}}%
\pgfusepath{clip}%
\pgfsetbuttcap%
\pgfsetroundjoin%
\definecolor{currentfill}{rgb}{0.121569,0.466667,0.705882}%
\pgfsetfillcolor{currentfill}%
\pgfsetfillopacity{0.727134}%
\pgfsetlinewidth{1.003750pt}%
\definecolor{currentstroke}{rgb}{0.121569,0.466667,0.705882}%
\pgfsetstrokecolor{currentstroke}%
\pgfsetstrokeopacity{0.727134}%
\pgfsetdash{}{0pt}%
\pgfpathmoveto{\pgfqpoint{3.221480in}{2.607245in}}%
\pgfpathcurveto{\pgfqpoint{3.229716in}{2.607245in}}{\pgfqpoint{3.237616in}{2.610517in}}{\pgfqpoint{3.243440in}{2.616341in}}%
\pgfpathcurveto{\pgfqpoint{3.249264in}{2.622165in}}{\pgfqpoint{3.252536in}{2.630065in}}{\pgfqpoint{3.252536in}{2.638302in}}%
\pgfpathcurveto{\pgfqpoint{3.252536in}{2.646538in}}{\pgfqpoint{3.249264in}{2.654438in}}{\pgfqpoint{3.243440in}{2.660262in}}%
\pgfpathcurveto{\pgfqpoint{3.237616in}{2.666086in}}{\pgfqpoint{3.229716in}{2.669358in}}{\pgfqpoint{3.221480in}{2.669358in}}%
\pgfpathcurveto{\pgfqpoint{3.213243in}{2.669358in}}{\pgfqpoint{3.205343in}{2.666086in}}{\pgfqpoint{3.199519in}{2.660262in}}%
\pgfpathcurveto{\pgfqpoint{3.193695in}{2.654438in}}{\pgfqpoint{3.190423in}{2.646538in}}{\pgfqpoint{3.190423in}{2.638302in}}%
\pgfpathcurveto{\pgfqpoint{3.190423in}{2.630065in}}{\pgfqpoint{3.193695in}{2.622165in}}{\pgfqpoint{3.199519in}{2.616341in}}%
\pgfpathcurveto{\pgfqpoint{3.205343in}{2.610517in}}{\pgfqpoint{3.213243in}{2.607245in}}{\pgfqpoint{3.221480in}{2.607245in}}%
\pgfpathclose%
\pgfusepath{stroke,fill}%
\end{pgfscope}%
\begin{pgfscope}%
\pgfpathrectangle{\pgfqpoint{0.100000in}{0.220728in}}{\pgfqpoint{3.696000in}{3.696000in}}%
\pgfusepath{clip}%
\pgfsetbuttcap%
\pgfsetroundjoin%
\definecolor{currentfill}{rgb}{0.121569,0.466667,0.705882}%
\pgfsetfillcolor{currentfill}%
\pgfsetfillopacity{0.727938}%
\pgfsetlinewidth{1.003750pt}%
\definecolor{currentstroke}{rgb}{0.121569,0.466667,0.705882}%
\pgfsetstrokecolor{currentstroke}%
\pgfsetstrokeopacity{0.727938}%
\pgfsetdash{}{0pt}%
\pgfpathmoveto{\pgfqpoint{3.215183in}{2.598549in}}%
\pgfpathcurveto{\pgfqpoint{3.223419in}{2.598549in}}{\pgfqpoint{3.231319in}{2.601822in}}{\pgfqpoint{3.237143in}{2.607646in}}%
\pgfpathcurveto{\pgfqpoint{3.242967in}{2.613470in}}{\pgfqpoint{3.246239in}{2.621370in}}{\pgfqpoint{3.246239in}{2.629606in}}%
\pgfpathcurveto{\pgfqpoint{3.246239in}{2.637842in}}{\pgfqpoint{3.242967in}{2.645742in}}{\pgfqpoint{3.237143in}{2.651566in}}%
\pgfpathcurveto{\pgfqpoint{3.231319in}{2.657390in}}{\pgfqpoint{3.223419in}{2.660662in}}{\pgfqpoint{3.215183in}{2.660662in}}%
\pgfpathcurveto{\pgfqpoint{3.206947in}{2.660662in}}{\pgfqpoint{3.199047in}{2.657390in}}{\pgfqpoint{3.193223in}{2.651566in}}%
\pgfpathcurveto{\pgfqpoint{3.187399in}{2.645742in}}{\pgfqpoint{3.184126in}{2.637842in}}{\pgfqpoint{3.184126in}{2.629606in}}%
\pgfpathcurveto{\pgfqpoint{3.184126in}{2.621370in}}{\pgfqpoint{3.187399in}{2.613470in}}{\pgfqpoint{3.193223in}{2.607646in}}%
\pgfpathcurveto{\pgfqpoint{3.199047in}{2.601822in}}{\pgfqpoint{3.206947in}{2.598549in}}{\pgfqpoint{3.215183in}{2.598549in}}%
\pgfpathclose%
\pgfusepath{stroke,fill}%
\end{pgfscope}%
\begin{pgfscope}%
\pgfpathrectangle{\pgfqpoint{0.100000in}{0.220728in}}{\pgfqpoint{3.696000in}{3.696000in}}%
\pgfusepath{clip}%
\pgfsetbuttcap%
\pgfsetroundjoin%
\definecolor{currentfill}{rgb}{0.121569,0.466667,0.705882}%
\pgfsetfillcolor{currentfill}%
\pgfsetfillopacity{0.729514}%
\pgfsetlinewidth{1.003750pt}%
\definecolor{currentstroke}{rgb}{0.121569,0.466667,0.705882}%
\pgfsetstrokecolor{currentstroke}%
\pgfsetstrokeopacity{0.729514}%
\pgfsetdash{}{0pt}%
\pgfpathmoveto{\pgfqpoint{0.978419in}{1.275471in}}%
\pgfpathcurveto{\pgfqpoint{0.986656in}{1.275471in}}{\pgfqpoint{0.994556in}{1.278743in}}{\pgfqpoint{1.000380in}{1.284567in}}%
\pgfpathcurveto{\pgfqpoint{1.006204in}{1.290391in}}{\pgfqpoint{1.009476in}{1.298291in}}{\pgfqpoint{1.009476in}{1.306527in}}%
\pgfpathcurveto{\pgfqpoint{1.009476in}{1.314763in}}{\pgfqpoint{1.006204in}{1.322663in}}{\pgfqpoint{1.000380in}{1.328487in}}%
\pgfpathcurveto{\pgfqpoint{0.994556in}{1.334311in}}{\pgfqpoint{0.986656in}{1.337584in}}{\pgfqpoint{0.978419in}{1.337584in}}%
\pgfpathcurveto{\pgfqpoint{0.970183in}{1.337584in}}{\pgfqpoint{0.962283in}{1.334311in}}{\pgfqpoint{0.956459in}{1.328487in}}%
\pgfpathcurveto{\pgfqpoint{0.950635in}{1.322663in}}{\pgfqpoint{0.947363in}{1.314763in}}{\pgfqpoint{0.947363in}{1.306527in}}%
\pgfpathcurveto{\pgfqpoint{0.947363in}{1.298291in}}{\pgfqpoint{0.950635in}{1.290391in}}{\pgfqpoint{0.956459in}{1.284567in}}%
\pgfpathcurveto{\pgfqpoint{0.962283in}{1.278743in}}{\pgfqpoint{0.970183in}{1.275471in}}{\pgfqpoint{0.978419in}{1.275471in}}%
\pgfpathclose%
\pgfusepath{stroke,fill}%
\end{pgfscope}%
\begin{pgfscope}%
\pgfpathrectangle{\pgfqpoint{0.100000in}{0.220728in}}{\pgfqpoint{3.696000in}{3.696000in}}%
\pgfusepath{clip}%
\pgfsetbuttcap%
\pgfsetroundjoin%
\definecolor{currentfill}{rgb}{0.121569,0.466667,0.705882}%
\pgfsetfillcolor{currentfill}%
\pgfsetfillopacity{0.729693}%
\pgfsetlinewidth{1.003750pt}%
\definecolor{currentstroke}{rgb}{0.121569,0.466667,0.705882}%
\pgfsetstrokecolor{currentstroke}%
\pgfsetstrokeopacity{0.729693}%
\pgfsetdash{}{0pt}%
\pgfpathmoveto{\pgfqpoint{3.209723in}{2.590062in}}%
\pgfpathcurveto{\pgfqpoint{3.217960in}{2.590062in}}{\pgfqpoint{3.225860in}{2.593334in}}{\pgfqpoint{3.231684in}{2.599158in}}%
\pgfpathcurveto{\pgfqpoint{3.237508in}{2.604982in}}{\pgfqpoint{3.240780in}{2.612882in}}{\pgfqpoint{3.240780in}{2.621119in}}%
\pgfpathcurveto{\pgfqpoint{3.240780in}{2.629355in}}{\pgfqpoint{3.237508in}{2.637255in}}{\pgfqpoint{3.231684in}{2.643079in}}%
\pgfpathcurveto{\pgfqpoint{3.225860in}{2.648903in}}{\pgfqpoint{3.217960in}{2.652175in}}{\pgfqpoint{3.209723in}{2.652175in}}%
\pgfpathcurveto{\pgfqpoint{3.201487in}{2.652175in}}{\pgfqpoint{3.193587in}{2.648903in}}{\pgfqpoint{3.187763in}{2.643079in}}%
\pgfpathcurveto{\pgfqpoint{3.181939in}{2.637255in}}{\pgfqpoint{3.178667in}{2.629355in}}{\pgfqpoint{3.178667in}{2.621119in}}%
\pgfpathcurveto{\pgfqpoint{3.178667in}{2.612882in}}{\pgfqpoint{3.181939in}{2.604982in}}{\pgfqpoint{3.187763in}{2.599158in}}%
\pgfpathcurveto{\pgfqpoint{3.193587in}{2.593334in}}{\pgfqpoint{3.201487in}{2.590062in}}{\pgfqpoint{3.209723in}{2.590062in}}%
\pgfpathclose%
\pgfusepath{stroke,fill}%
\end{pgfscope}%
\begin{pgfscope}%
\pgfpathrectangle{\pgfqpoint{0.100000in}{0.220728in}}{\pgfqpoint{3.696000in}{3.696000in}}%
\pgfusepath{clip}%
\pgfsetbuttcap%
\pgfsetroundjoin%
\definecolor{currentfill}{rgb}{0.121569,0.466667,0.705882}%
\pgfsetfillcolor{currentfill}%
\pgfsetfillopacity{0.730598}%
\pgfsetlinewidth{1.003750pt}%
\definecolor{currentstroke}{rgb}{0.121569,0.466667,0.705882}%
\pgfsetstrokecolor{currentstroke}%
\pgfsetstrokeopacity{0.730598}%
\pgfsetdash{}{0pt}%
\pgfpathmoveto{\pgfqpoint{3.207330in}{2.584355in}}%
\pgfpathcurveto{\pgfqpoint{3.215566in}{2.584355in}}{\pgfqpoint{3.223466in}{2.587627in}}{\pgfqpoint{3.229290in}{2.593451in}}%
\pgfpathcurveto{\pgfqpoint{3.235114in}{2.599275in}}{\pgfqpoint{3.238386in}{2.607175in}}{\pgfqpoint{3.238386in}{2.615411in}}%
\pgfpathcurveto{\pgfqpoint{3.238386in}{2.623647in}}{\pgfqpoint{3.235114in}{2.631548in}}{\pgfqpoint{3.229290in}{2.637371in}}%
\pgfpathcurveto{\pgfqpoint{3.223466in}{2.643195in}}{\pgfqpoint{3.215566in}{2.646468in}}{\pgfqpoint{3.207330in}{2.646468in}}%
\pgfpathcurveto{\pgfqpoint{3.199093in}{2.646468in}}{\pgfqpoint{3.191193in}{2.643195in}}{\pgfqpoint{3.185369in}{2.637371in}}%
\pgfpathcurveto{\pgfqpoint{3.179545in}{2.631548in}}{\pgfqpoint{3.176273in}{2.623647in}}{\pgfqpoint{3.176273in}{2.615411in}}%
\pgfpathcurveto{\pgfqpoint{3.176273in}{2.607175in}}{\pgfqpoint{3.179545in}{2.599275in}}{\pgfqpoint{3.185369in}{2.593451in}}%
\pgfpathcurveto{\pgfqpoint{3.191193in}{2.587627in}}{\pgfqpoint{3.199093in}{2.584355in}}{\pgfqpoint{3.207330in}{2.584355in}}%
\pgfpathclose%
\pgfusepath{stroke,fill}%
\end{pgfscope}%
\begin{pgfscope}%
\pgfpathrectangle{\pgfqpoint{0.100000in}{0.220728in}}{\pgfqpoint{3.696000in}{3.696000in}}%
\pgfusepath{clip}%
\pgfsetbuttcap%
\pgfsetroundjoin%
\definecolor{currentfill}{rgb}{0.121569,0.466667,0.705882}%
\pgfsetfillcolor{currentfill}%
\pgfsetfillopacity{0.731469}%
\pgfsetlinewidth{1.003750pt}%
\definecolor{currentstroke}{rgb}{0.121569,0.466667,0.705882}%
\pgfsetstrokecolor{currentstroke}%
\pgfsetstrokeopacity{0.731469}%
\pgfsetdash{}{0pt}%
\pgfpathmoveto{\pgfqpoint{3.201284in}{2.576670in}}%
\pgfpathcurveto{\pgfqpoint{3.209520in}{2.576670in}}{\pgfqpoint{3.217420in}{2.579942in}}{\pgfqpoint{3.223244in}{2.585766in}}%
\pgfpathcurveto{\pgfqpoint{3.229068in}{2.591590in}}{\pgfqpoint{3.232341in}{2.599490in}}{\pgfqpoint{3.232341in}{2.607727in}}%
\pgfpathcurveto{\pgfqpoint{3.232341in}{2.615963in}}{\pgfqpoint{3.229068in}{2.623863in}}{\pgfqpoint{3.223244in}{2.629687in}}%
\pgfpathcurveto{\pgfqpoint{3.217420in}{2.635511in}}{\pgfqpoint{3.209520in}{2.638783in}}{\pgfqpoint{3.201284in}{2.638783in}}%
\pgfpathcurveto{\pgfqpoint{3.193048in}{2.638783in}}{\pgfqpoint{3.185148in}{2.635511in}}{\pgfqpoint{3.179324in}{2.629687in}}%
\pgfpathcurveto{\pgfqpoint{3.173500in}{2.623863in}}{\pgfqpoint{3.170228in}{2.615963in}}{\pgfqpoint{3.170228in}{2.607727in}}%
\pgfpathcurveto{\pgfqpoint{3.170228in}{2.599490in}}{\pgfqpoint{3.173500in}{2.591590in}}{\pgfqpoint{3.179324in}{2.585766in}}%
\pgfpathcurveto{\pgfqpoint{3.185148in}{2.579942in}}{\pgfqpoint{3.193048in}{2.576670in}}{\pgfqpoint{3.201284in}{2.576670in}}%
\pgfpathclose%
\pgfusepath{stroke,fill}%
\end{pgfscope}%
\begin{pgfscope}%
\pgfpathrectangle{\pgfqpoint{0.100000in}{0.220728in}}{\pgfqpoint{3.696000in}{3.696000in}}%
\pgfusepath{clip}%
\pgfsetbuttcap%
\pgfsetroundjoin%
\definecolor{currentfill}{rgb}{0.121569,0.466667,0.705882}%
\pgfsetfillcolor{currentfill}%
\pgfsetfillopacity{0.732592}%
\pgfsetlinewidth{1.003750pt}%
\definecolor{currentstroke}{rgb}{0.121569,0.466667,0.705882}%
\pgfsetstrokecolor{currentstroke}%
\pgfsetstrokeopacity{0.732592}%
\pgfsetdash{}{0pt}%
\pgfpathmoveto{\pgfqpoint{0.991361in}{1.268050in}}%
\pgfpathcurveto{\pgfqpoint{0.999598in}{1.268050in}}{\pgfqpoint{1.007498in}{1.271323in}}{\pgfqpoint{1.013322in}{1.277146in}}%
\pgfpathcurveto{\pgfqpoint{1.019145in}{1.282970in}}{\pgfqpoint{1.022418in}{1.290870in}}{\pgfqpoint{1.022418in}{1.299107in}}%
\pgfpathcurveto{\pgfqpoint{1.022418in}{1.307343in}}{\pgfqpoint{1.019145in}{1.315243in}}{\pgfqpoint{1.013322in}{1.321067in}}%
\pgfpathcurveto{\pgfqpoint{1.007498in}{1.326891in}}{\pgfqpoint{0.999598in}{1.330163in}}{\pgfqpoint{0.991361in}{1.330163in}}%
\pgfpathcurveto{\pgfqpoint{0.983125in}{1.330163in}}{\pgfqpoint{0.975225in}{1.326891in}}{\pgfqpoint{0.969401in}{1.321067in}}%
\pgfpathcurveto{\pgfqpoint{0.963577in}{1.315243in}}{\pgfqpoint{0.960305in}{1.307343in}}{\pgfqpoint{0.960305in}{1.299107in}}%
\pgfpathcurveto{\pgfqpoint{0.960305in}{1.290870in}}{\pgfqpoint{0.963577in}{1.282970in}}{\pgfqpoint{0.969401in}{1.277146in}}%
\pgfpathcurveto{\pgfqpoint{0.975225in}{1.271323in}}{\pgfqpoint{0.983125in}{1.268050in}}{\pgfqpoint{0.991361in}{1.268050in}}%
\pgfpathclose%
\pgfusepath{stroke,fill}%
\end{pgfscope}%
\begin{pgfscope}%
\pgfpathrectangle{\pgfqpoint{0.100000in}{0.220728in}}{\pgfqpoint{3.696000in}{3.696000in}}%
\pgfusepath{clip}%
\pgfsetbuttcap%
\pgfsetroundjoin%
\definecolor{currentfill}{rgb}{0.121569,0.466667,0.705882}%
\pgfsetfillcolor{currentfill}%
\pgfsetfillopacity{0.733552}%
\pgfsetlinewidth{1.003750pt}%
\definecolor{currentstroke}{rgb}{0.121569,0.466667,0.705882}%
\pgfsetstrokecolor{currentstroke}%
\pgfsetstrokeopacity{0.733552}%
\pgfsetdash{}{0pt}%
\pgfpathmoveto{\pgfqpoint{3.197052in}{2.566508in}}%
\pgfpathcurveto{\pgfqpoint{3.205288in}{2.566508in}}{\pgfqpoint{3.213188in}{2.569780in}}{\pgfqpoint{3.219012in}{2.575604in}}%
\pgfpathcurveto{\pgfqpoint{3.224836in}{2.581428in}}{\pgfqpoint{3.228108in}{2.589328in}}{\pgfqpoint{3.228108in}{2.597564in}}%
\pgfpathcurveto{\pgfqpoint{3.228108in}{2.605800in}}{\pgfqpoint{3.224836in}{2.613700in}}{\pgfqpoint{3.219012in}{2.619524in}}%
\pgfpathcurveto{\pgfqpoint{3.213188in}{2.625348in}}{\pgfqpoint{3.205288in}{2.628621in}}{\pgfqpoint{3.197052in}{2.628621in}}%
\pgfpathcurveto{\pgfqpoint{3.188816in}{2.628621in}}{\pgfqpoint{3.180916in}{2.625348in}}{\pgfqpoint{3.175092in}{2.619524in}}%
\pgfpathcurveto{\pgfqpoint{3.169268in}{2.613700in}}{\pgfqpoint{3.165995in}{2.605800in}}{\pgfqpoint{3.165995in}{2.597564in}}%
\pgfpathcurveto{\pgfqpoint{3.165995in}{2.589328in}}{\pgfqpoint{3.169268in}{2.581428in}}{\pgfqpoint{3.175092in}{2.575604in}}%
\pgfpathcurveto{\pgfqpoint{3.180916in}{2.569780in}}{\pgfqpoint{3.188816in}{2.566508in}}{\pgfqpoint{3.197052in}{2.566508in}}%
\pgfpathclose%
\pgfusepath{stroke,fill}%
\end{pgfscope}%
\begin{pgfscope}%
\pgfpathrectangle{\pgfqpoint{0.100000in}{0.220728in}}{\pgfqpoint{3.696000in}{3.696000in}}%
\pgfusepath{clip}%
\pgfsetbuttcap%
\pgfsetroundjoin%
\definecolor{currentfill}{rgb}{0.121569,0.466667,0.705882}%
\pgfsetfillcolor{currentfill}%
\pgfsetfillopacity{0.734578}%
\pgfsetlinewidth{1.003750pt}%
\definecolor{currentstroke}{rgb}{0.121569,0.466667,0.705882}%
\pgfsetstrokecolor{currentstroke}%
\pgfsetstrokeopacity{0.734578}%
\pgfsetdash{}{0pt}%
\pgfpathmoveto{\pgfqpoint{3.194284in}{2.560866in}}%
\pgfpathcurveto{\pgfqpoint{3.202520in}{2.560866in}}{\pgfqpoint{3.210420in}{2.564138in}}{\pgfqpoint{3.216244in}{2.569962in}}%
\pgfpathcurveto{\pgfqpoint{3.222068in}{2.575786in}}{\pgfqpoint{3.225341in}{2.583686in}}{\pgfqpoint{3.225341in}{2.591922in}}%
\pgfpathcurveto{\pgfqpoint{3.225341in}{2.600158in}}{\pgfqpoint{3.222068in}{2.608058in}}{\pgfqpoint{3.216244in}{2.613882in}}%
\pgfpathcurveto{\pgfqpoint{3.210420in}{2.619706in}}{\pgfqpoint{3.202520in}{2.622979in}}{\pgfqpoint{3.194284in}{2.622979in}}%
\pgfpathcurveto{\pgfqpoint{3.186048in}{2.622979in}}{\pgfqpoint{3.178148in}{2.619706in}}{\pgfqpoint{3.172324in}{2.613882in}}%
\pgfpathcurveto{\pgfqpoint{3.166500in}{2.608058in}}{\pgfqpoint{3.163228in}{2.600158in}}{\pgfqpoint{3.163228in}{2.591922in}}%
\pgfpathcurveto{\pgfqpoint{3.163228in}{2.583686in}}{\pgfqpoint{3.166500in}{2.575786in}}{\pgfqpoint{3.172324in}{2.569962in}}%
\pgfpathcurveto{\pgfqpoint{3.178148in}{2.564138in}}{\pgfqpoint{3.186048in}{2.560866in}}{\pgfqpoint{3.194284in}{2.560866in}}%
\pgfpathclose%
\pgfusepath{stroke,fill}%
\end{pgfscope}%
\begin{pgfscope}%
\pgfpathrectangle{\pgfqpoint{0.100000in}{0.220728in}}{\pgfqpoint{3.696000in}{3.696000in}}%
\pgfusepath{clip}%
\pgfsetbuttcap%
\pgfsetroundjoin%
\definecolor{currentfill}{rgb}{0.121569,0.466667,0.705882}%
\pgfsetfillcolor{currentfill}%
\pgfsetfillopacity{0.735070}%
\pgfsetlinewidth{1.003750pt}%
\definecolor{currentstroke}{rgb}{0.121569,0.466667,0.705882}%
\pgfsetstrokecolor{currentstroke}%
\pgfsetstrokeopacity{0.735070}%
\pgfsetdash{}{0pt}%
\pgfpathmoveto{\pgfqpoint{3.192151in}{2.558442in}}%
\pgfpathcurveto{\pgfqpoint{3.200387in}{2.558442in}}{\pgfqpoint{3.208287in}{2.561714in}}{\pgfqpoint{3.214111in}{2.567538in}}%
\pgfpathcurveto{\pgfqpoint{3.219935in}{2.573362in}}{\pgfqpoint{3.223207in}{2.581262in}}{\pgfqpoint{3.223207in}{2.589499in}}%
\pgfpathcurveto{\pgfqpoint{3.223207in}{2.597735in}}{\pgfqpoint{3.219935in}{2.605635in}}{\pgfqpoint{3.214111in}{2.611459in}}%
\pgfpathcurveto{\pgfqpoint{3.208287in}{2.617283in}}{\pgfqpoint{3.200387in}{2.620555in}}{\pgfqpoint{3.192151in}{2.620555in}}%
\pgfpathcurveto{\pgfqpoint{3.183914in}{2.620555in}}{\pgfqpoint{3.176014in}{2.617283in}}{\pgfqpoint{3.170190in}{2.611459in}}%
\pgfpathcurveto{\pgfqpoint{3.164366in}{2.605635in}}{\pgfqpoint{3.161094in}{2.597735in}}{\pgfqpoint{3.161094in}{2.589499in}}%
\pgfpathcurveto{\pgfqpoint{3.161094in}{2.581262in}}{\pgfqpoint{3.164366in}{2.573362in}}{\pgfqpoint{3.170190in}{2.567538in}}%
\pgfpathcurveto{\pgfqpoint{3.176014in}{2.561714in}}{\pgfqpoint{3.183914in}{2.558442in}}{\pgfqpoint{3.192151in}{2.558442in}}%
\pgfpathclose%
\pgfusepath{stroke,fill}%
\end{pgfscope}%
\begin{pgfscope}%
\pgfpathrectangle{\pgfqpoint{0.100000in}{0.220728in}}{\pgfqpoint{3.696000in}{3.696000in}}%
\pgfusepath{clip}%
\pgfsetbuttcap%
\pgfsetroundjoin%
\definecolor{currentfill}{rgb}{0.121569,0.466667,0.705882}%
\pgfsetfillcolor{currentfill}%
\pgfsetfillopacity{0.735852}%
\pgfsetlinewidth{1.003750pt}%
\definecolor{currentstroke}{rgb}{0.121569,0.466667,0.705882}%
\pgfsetstrokecolor{currentstroke}%
\pgfsetstrokeopacity{0.735852}%
\pgfsetdash{}{0pt}%
\pgfpathmoveto{\pgfqpoint{3.189758in}{2.552522in}}%
\pgfpathcurveto{\pgfqpoint{3.197995in}{2.552522in}}{\pgfqpoint{3.205895in}{2.555794in}}{\pgfqpoint{3.211718in}{2.561618in}}%
\pgfpathcurveto{\pgfqpoint{3.217542in}{2.567442in}}{\pgfqpoint{3.220815in}{2.575342in}}{\pgfqpoint{3.220815in}{2.583578in}}%
\pgfpathcurveto{\pgfqpoint{3.220815in}{2.591815in}}{\pgfqpoint{3.217542in}{2.599715in}}{\pgfqpoint{3.211718in}{2.605539in}}%
\pgfpathcurveto{\pgfqpoint{3.205895in}{2.611363in}}{\pgfqpoint{3.197995in}{2.614635in}}{\pgfqpoint{3.189758in}{2.614635in}}%
\pgfpathcurveto{\pgfqpoint{3.181522in}{2.614635in}}{\pgfqpoint{3.173622in}{2.611363in}}{\pgfqpoint{3.167798in}{2.605539in}}%
\pgfpathcurveto{\pgfqpoint{3.161974in}{2.599715in}}{\pgfqpoint{3.158702in}{2.591815in}}{\pgfqpoint{3.158702in}{2.583578in}}%
\pgfpathcurveto{\pgfqpoint{3.158702in}{2.575342in}}{\pgfqpoint{3.161974in}{2.567442in}}{\pgfqpoint{3.167798in}{2.561618in}}%
\pgfpathcurveto{\pgfqpoint{3.173622in}{2.555794in}}{\pgfqpoint{3.181522in}{2.552522in}}{\pgfqpoint{3.189758in}{2.552522in}}%
\pgfpathclose%
\pgfusepath{stroke,fill}%
\end{pgfscope}%
\begin{pgfscope}%
\pgfpathrectangle{\pgfqpoint{0.100000in}{0.220728in}}{\pgfqpoint{3.696000in}{3.696000in}}%
\pgfusepath{clip}%
\pgfsetbuttcap%
\pgfsetroundjoin%
\definecolor{currentfill}{rgb}{0.121569,0.466667,0.705882}%
\pgfsetfillcolor{currentfill}%
\pgfsetfillopacity{0.736235}%
\pgfsetlinewidth{1.003750pt}%
\definecolor{currentstroke}{rgb}{0.121569,0.466667,0.705882}%
\pgfsetstrokecolor{currentstroke}%
\pgfsetstrokeopacity{0.736235}%
\pgfsetdash{}{0pt}%
\pgfpathmoveto{\pgfqpoint{3.187983in}{2.549705in}}%
\pgfpathcurveto{\pgfqpoint{3.196219in}{2.549705in}}{\pgfqpoint{3.204120in}{2.552978in}}{\pgfqpoint{3.209943in}{2.558802in}}%
\pgfpathcurveto{\pgfqpoint{3.215767in}{2.564626in}}{\pgfqpoint{3.219040in}{2.572526in}}{\pgfqpoint{3.219040in}{2.580762in}}%
\pgfpathcurveto{\pgfqpoint{3.219040in}{2.588998in}}{\pgfqpoint{3.215767in}{2.596898in}}{\pgfqpoint{3.209943in}{2.602722in}}%
\pgfpathcurveto{\pgfqpoint{3.204120in}{2.608546in}}{\pgfqpoint{3.196219in}{2.611818in}}{\pgfqpoint{3.187983in}{2.611818in}}%
\pgfpathcurveto{\pgfqpoint{3.179747in}{2.611818in}}{\pgfqpoint{3.171847in}{2.608546in}}{\pgfqpoint{3.166023in}{2.602722in}}%
\pgfpathcurveto{\pgfqpoint{3.160199in}{2.596898in}}{\pgfqpoint{3.156927in}{2.588998in}}{\pgfqpoint{3.156927in}{2.580762in}}%
\pgfpathcurveto{\pgfqpoint{3.156927in}{2.572526in}}{\pgfqpoint{3.160199in}{2.564626in}}{\pgfqpoint{3.166023in}{2.558802in}}%
\pgfpathcurveto{\pgfqpoint{3.171847in}{2.552978in}}{\pgfqpoint{3.179747in}{2.549705in}}{\pgfqpoint{3.187983in}{2.549705in}}%
\pgfpathclose%
\pgfusepath{stroke,fill}%
\end{pgfscope}%
\begin{pgfscope}%
\pgfpathrectangle{\pgfqpoint{0.100000in}{0.220728in}}{\pgfqpoint{3.696000in}{3.696000in}}%
\pgfusepath{clip}%
\pgfsetbuttcap%
\pgfsetroundjoin%
\definecolor{currentfill}{rgb}{0.121569,0.466667,0.705882}%
\pgfsetfillcolor{currentfill}%
\pgfsetfillopacity{0.736865}%
\pgfsetlinewidth{1.003750pt}%
\definecolor{currentstroke}{rgb}{0.121569,0.466667,0.705882}%
\pgfsetstrokecolor{currentstroke}%
\pgfsetstrokeopacity{0.736865}%
\pgfsetdash{}{0pt}%
\pgfpathmoveto{\pgfqpoint{1.016815in}{1.254687in}}%
\pgfpathcurveto{\pgfqpoint{1.025051in}{1.254687in}}{\pgfqpoint{1.032951in}{1.257959in}}{\pgfqpoint{1.038775in}{1.263783in}}%
\pgfpathcurveto{\pgfqpoint{1.044599in}{1.269607in}}{\pgfqpoint{1.047871in}{1.277507in}}{\pgfqpoint{1.047871in}{1.285743in}}%
\pgfpathcurveto{\pgfqpoint{1.047871in}{1.293980in}}{\pgfqpoint{1.044599in}{1.301880in}}{\pgfqpoint{1.038775in}{1.307704in}}%
\pgfpathcurveto{\pgfqpoint{1.032951in}{1.313528in}}{\pgfqpoint{1.025051in}{1.316800in}}{\pgfqpoint{1.016815in}{1.316800in}}%
\pgfpathcurveto{\pgfqpoint{1.008579in}{1.316800in}}{\pgfqpoint{1.000678in}{1.313528in}}{\pgfqpoint{0.994855in}{1.307704in}}%
\pgfpathcurveto{\pgfqpoint{0.989031in}{1.301880in}}{\pgfqpoint{0.985758in}{1.293980in}}{\pgfqpoint{0.985758in}{1.285743in}}%
\pgfpathcurveto{\pgfqpoint{0.985758in}{1.277507in}}{\pgfqpoint{0.989031in}{1.269607in}}{\pgfqpoint{0.994855in}{1.263783in}}%
\pgfpathcurveto{\pgfqpoint{1.000678in}{1.257959in}}{\pgfqpoint{1.008579in}{1.254687in}}{\pgfqpoint{1.016815in}{1.254687in}}%
\pgfpathclose%
\pgfusepath{stroke,fill}%
\end{pgfscope}%
\begin{pgfscope}%
\pgfpathrectangle{\pgfqpoint{0.100000in}{0.220728in}}{\pgfqpoint{3.696000in}{3.696000in}}%
\pgfusepath{clip}%
\pgfsetbuttcap%
\pgfsetroundjoin%
\definecolor{currentfill}{rgb}{0.121569,0.466667,0.705882}%
\pgfsetfillcolor{currentfill}%
\pgfsetfillopacity{0.736920}%
\pgfsetlinewidth{1.003750pt}%
\definecolor{currentstroke}{rgb}{0.121569,0.466667,0.705882}%
\pgfsetstrokecolor{currentstroke}%
\pgfsetstrokeopacity{0.736920}%
\pgfsetdash{}{0pt}%
\pgfpathmoveto{\pgfqpoint{3.185937in}{2.546382in}}%
\pgfpathcurveto{\pgfqpoint{3.194174in}{2.546382in}}{\pgfqpoint{3.202074in}{2.549655in}}{\pgfqpoint{3.207898in}{2.555479in}}%
\pgfpathcurveto{\pgfqpoint{3.213722in}{2.561302in}}{\pgfqpoint{3.216994in}{2.569203in}}{\pgfqpoint{3.216994in}{2.577439in}}%
\pgfpathcurveto{\pgfqpoint{3.216994in}{2.585675in}}{\pgfqpoint{3.213722in}{2.593575in}}{\pgfqpoint{3.207898in}{2.599399in}}%
\pgfpathcurveto{\pgfqpoint{3.202074in}{2.605223in}}{\pgfqpoint{3.194174in}{2.608495in}}{\pgfqpoint{3.185937in}{2.608495in}}%
\pgfpathcurveto{\pgfqpoint{3.177701in}{2.608495in}}{\pgfqpoint{3.169801in}{2.605223in}}{\pgfqpoint{3.163977in}{2.599399in}}%
\pgfpathcurveto{\pgfqpoint{3.158153in}{2.593575in}}{\pgfqpoint{3.154881in}{2.585675in}}{\pgfqpoint{3.154881in}{2.577439in}}%
\pgfpathcurveto{\pgfqpoint{3.154881in}{2.569203in}}{\pgfqpoint{3.158153in}{2.561302in}}{\pgfqpoint{3.163977in}{2.555479in}}%
\pgfpathcurveto{\pgfqpoint{3.169801in}{2.549655in}}{\pgfqpoint{3.177701in}{2.546382in}}{\pgfqpoint{3.185937in}{2.546382in}}%
\pgfpathclose%
\pgfusepath{stroke,fill}%
\end{pgfscope}%
\begin{pgfscope}%
\pgfpathrectangle{\pgfqpoint{0.100000in}{0.220728in}}{\pgfqpoint{3.696000in}{3.696000in}}%
\pgfusepath{clip}%
\pgfsetbuttcap%
\pgfsetroundjoin%
\definecolor{currentfill}{rgb}{0.121569,0.466667,0.705882}%
\pgfsetfillcolor{currentfill}%
\pgfsetfillopacity{0.737280}%
\pgfsetlinewidth{1.003750pt}%
\definecolor{currentstroke}{rgb}{0.121569,0.466667,0.705882}%
\pgfsetstrokecolor{currentstroke}%
\pgfsetstrokeopacity{0.737280}%
\pgfsetdash{}{0pt}%
\pgfpathmoveto{\pgfqpoint{3.184958in}{2.544295in}}%
\pgfpathcurveto{\pgfqpoint{3.193195in}{2.544295in}}{\pgfqpoint{3.201095in}{2.547568in}}{\pgfqpoint{3.206919in}{2.553392in}}%
\pgfpathcurveto{\pgfqpoint{3.212743in}{2.559216in}}{\pgfqpoint{3.216015in}{2.567116in}}{\pgfqpoint{3.216015in}{2.575352in}}%
\pgfpathcurveto{\pgfqpoint{3.216015in}{2.583588in}}{\pgfqpoint{3.212743in}{2.591488in}}{\pgfqpoint{3.206919in}{2.597312in}}%
\pgfpathcurveto{\pgfqpoint{3.201095in}{2.603136in}}{\pgfqpoint{3.193195in}{2.606408in}}{\pgfqpoint{3.184958in}{2.606408in}}%
\pgfpathcurveto{\pgfqpoint{3.176722in}{2.606408in}}{\pgfqpoint{3.168822in}{2.603136in}}{\pgfqpoint{3.162998in}{2.597312in}}%
\pgfpathcurveto{\pgfqpoint{3.157174in}{2.591488in}}{\pgfqpoint{3.153902in}{2.583588in}}{\pgfqpoint{3.153902in}{2.575352in}}%
\pgfpathcurveto{\pgfqpoint{3.153902in}{2.567116in}}{\pgfqpoint{3.157174in}{2.559216in}}{\pgfqpoint{3.162998in}{2.553392in}}%
\pgfpathcurveto{\pgfqpoint{3.168822in}{2.547568in}}{\pgfqpoint{3.176722in}{2.544295in}}{\pgfqpoint{3.184958in}{2.544295in}}%
\pgfpathclose%
\pgfusepath{stroke,fill}%
\end{pgfscope}%
\begin{pgfscope}%
\pgfpathrectangle{\pgfqpoint{0.100000in}{0.220728in}}{\pgfqpoint{3.696000in}{3.696000in}}%
\pgfusepath{clip}%
\pgfsetbuttcap%
\pgfsetroundjoin%
\definecolor{currentfill}{rgb}{0.121569,0.466667,0.705882}%
\pgfsetfillcolor{currentfill}%
\pgfsetfillopacity{0.737761}%
\pgfsetlinewidth{1.003750pt}%
\definecolor{currentstroke}{rgb}{0.121569,0.466667,0.705882}%
\pgfsetstrokecolor{currentstroke}%
\pgfsetstrokeopacity{0.737761}%
\pgfsetdash{}{0pt}%
\pgfpathmoveto{\pgfqpoint{3.182681in}{2.541468in}}%
\pgfpathcurveto{\pgfqpoint{3.190918in}{2.541468in}}{\pgfqpoint{3.198818in}{2.544740in}}{\pgfqpoint{3.204642in}{2.550564in}}%
\pgfpathcurveto{\pgfqpoint{3.210466in}{2.556388in}}{\pgfqpoint{3.213738in}{2.564288in}}{\pgfqpoint{3.213738in}{2.572525in}}%
\pgfpathcurveto{\pgfqpoint{3.213738in}{2.580761in}}{\pgfqpoint{3.210466in}{2.588661in}}{\pgfqpoint{3.204642in}{2.594485in}}%
\pgfpathcurveto{\pgfqpoint{3.198818in}{2.600309in}}{\pgfqpoint{3.190918in}{2.603581in}}{\pgfqpoint{3.182681in}{2.603581in}}%
\pgfpathcurveto{\pgfqpoint{3.174445in}{2.603581in}}{\pgfqpoint{3.166545in}{2.600309in}}{\pgfqpoint{3.160721in}{2.594485in}}%
\pgfpathcurveto{\pgfqpoint{3.154897in}{2.588661in}}{\pgfqpoint{3.151625in}{2.580761in}}{\pgfqpoint{3.151625in}{2.572525in}}%
\pgfpathcurveto{\pgfqpoint{3.151625in}{2.564288in}}{\pgfqpoint{3.154897in}{2.556388in}}{\pgfqpoint{3.160721in}{2.550564in}}%
\pgfpathcurveto{\pgfqpoint{3.166545in}{2.544740in}}{\pgfqpoint{3.174445in}{2.541468in}}{\pgfqpoint{3.182681in}{2.541468in}}%
\pgfpathclose%
\pgfusepath{stroke,fill}%
\end{pgfscope}%
\begin{pgfscope}%
\pgfpathrectangle{\pgfqpoint{0.100000in}{0.220728in}}{\pgfqpoint{3.696000in}{3.696000in}}%
\pgfusepath{clip}%
\pgfsetbuttcap%
\pgfsetroundjoin%
\definecolor{currentfill}{rgb}{0.121569,0.466667,0.705882}%
\pgfsetfillcolor{currentfill}%
\pgfsetfillopacity{0.738643}%
\pgfsetlinewidth{1.003750pt}%
\definecolor{currentstroke}{rgb}{0.121569,0.466667,0.705882}%
\pgfsetstrokecolor{currentstroke}%
\pgfsetstrokeopacity{0.738643}%
\pgfsetdash{}{0pt}%
\pgfpathmoveto{\pgfqpoint{3.180709in}{2.534683in}}%
\pgfpathcurveto{\pgfqpoint{3.188945in}{2.534683in}}{\pgfqpoint{3.196846in}{2.537955in}}{\pgfqpoint{3.202669in}{2.543779in}}%
\pgfpathcurveto{\pgfqpoint{3.208493in}{2.549603in}}{\pgfqpoint{3.211766in}{2.557503in}}{\pgfqpoint{3.211766in}{2.565739in}}%
\pgfpathcurveto{\pgfqpoint{3.211766in}{2.573975in}}{\pgfqpoint{3.208493in}{2.581876in}}{\pgfqpoint{3.202669in}{2.587699in}}%
\pgfpathcurveto{\pgfqpoint{3.196846in}{2.593523in}}{\pgfqpoint{3.188945in}{2.596796in}}{\pgfqpoint{3.180709in}{2.596796in}}%
\pgfpathcurveto{\pgfqpoint{3.172473in}{2.596796in}}{\pgfqpoint{3.164573in}{2.593523in}}{\pgfqpoint{3.158749in}{2.587699in}}%
\pgfpathcurveto{\pgfqpoint{3.152925in}{2.581876in}}{\pgfqpoint{3.149653in}{2.573975in}}{\pgfqpoint{3.149653in}{2.565739in}}%
\pgfpathcurveto{\pgfqpoint{3.149653in}{2.557503in}}{\pgfqpoint{3.152925in}{2.549603in}}{\pgfqpoint{3.158749in}{2.543779in}}%
\pgfpathcurveto{\pgfqpoint{3.164573in}{2.537955in}}{\pgfqpoint{3.172473in}{2.534683in}}{\pgfqpoint{3.180709in}{2.534683in}}%
\pgfpathclose%
\pgfusepath{stroke,fill}%
\end{pgfscope}%
\begin{pgfscope}%
\pgfpathrectangle{\pgfqpoint{0.100000in}{0.220728in}}{\pgfqpoint{3.696000in}{3.696000in}}%
\pgfusepath{clip}%
\pgfsetbuttcap%
\pgfsetroundjoin%
\definecolor{currentfill}{rgb}{0.121569,0.466667,0.705882}%
\pgfsetfillcolor{currentfill}%
\pgfsetfillopacity{0.739114}%
\pgfsetlinewidth{1.003750pt}%
\definecolor{currentstroke}{rgb}{0.121569,0.466667,0.705882}%
\pgfsetstrokecolor{currentstroke}%
\pgfsetstrokeopacity{0.739114}%
\pgfsetdash{}{0pt}%
\pgfpathmoveto{\pgfqpoint{3.179031in}{2.531517in}}%
\pgfpathcurveto{\pgfqpoint{3.187267in}{2.531517in}}{\pgfqpoint{3.195167in}{2.534789in}}{\pgfqpoint{3.200991in}{2.540613in}}%
\pgfpathcurveto{\pgfqpoint{3.206815in}{2.546437in}}{\pgfqpoint{3.210087in}{2.554337in}}{\pgfqpoint{3.210087in}{2.562573in}}%
\pgfpathcurveto{\pgfqpoint{3.210087in}{2.570810in}}{\pgfqpoint{3.206815in}{2.578710in}}{\pgfqpoint{3.200991in}{2.584534in}}%
\pgfpathcurveto{\pgfqpoint{3.195167in}{2.590358in}}{\pgfqpoint{3.187267in}{2.593630in}}{\pgfqpoint{3.179031in}{2.593630in}}%
\pgfpathcurveto{\pgfqpoint{3.170795in}{2.593630in}}{\pgfqpoint{3.162894in}{2.590358in}}{\pgfqpoint{3.157071in}{2.584534in}}%
\pgfpathcurveto{\pgfqpoint{3.151247in}{2.578710in}}{\pgfqpoint{3.147974in}{2.570810in}}{\pgfqpoint{3.147974in}{2.562573in}}%
\pgfpathcurveto{\pgfqpoint{3.147974in}{2.554337in}}{\pgfqpoint{3.151247in}{2.546437in}}{\pgfqpoint{3.157071in}{2.540613in}}%
\pgfpathcurveto{\pgfqpoint{3.162894in}{2.534789in}}{\pgfqpoint{3.170795in}{2.531517in}}{\pgfqpoint{3.179031in}{2.531517in}}%
\pgfpathclose%
\pgfusepath{stroke,fill}%
\end{pgfscope}%
\begin{pgfscope}%
\pgfpathrectangle{\pgfqpoint{0.100000in}{0.220728in}}{\pgfqpoint{3.696000in}{3.696000in}}%
\pgfusepath{clip}%
\pgfsetbuttcap%
\pgfsetroundjoin%
\definecolor{currentfill}{rgb}{0.121569,0.466667,0.705882}%
\pgfsetfillcolor{currentfill}%
\pgfsetfillopacity{0.739444}%
\pgfsetlinewidth{1.003750pt}%
\definecolor{currentstroke}{rgb}{0.121569,0.466667,0.705882}%
\pgfsetstrokecolor{currentstroke}%
\pgfsetstrokeopacity{0.739444}%
\pgfsetdash{}{0pt}%
\pgfpathmoveto{\pgfqpoint{3.178006in}{2.530236in}}%
\pgfpathcurveto{\pgfqpoint{3.186242in}{2.530236in}}{\pgfqpoint{3.194142in}{2.533508in}}{\pgfqpoint{3.199966in}{2.539332in}}%
\pgfpathcurveto{\pgfqpoint{3.205790in}{2.545156in}}{\pgfqpoint{3.209063in}{2.553056in}}{\pgfqpoint{3.209063in}{2.561292in}}%
\pgfpathcurveto{\pgfqpoint{3.209063in}{2.569529in}}{\pgfqpoint{3.205790in}{2.577429in}}{\pgfqpoint{3.199966in}{2.583253in}}%
\pgfpathcurveto{\pgfqpoint{3.194142in}{2.589077in}}{\pgfqpoint{3.186242in}{2.592349in}}{\pgfqpoint{3.178006in}{2.592349in}}%
\pgfpathcurveto{\pgfqpoint{3.169770in}{2.592349in}}{\pgfqpoint{3.161870in}{2.589077in}}{\pgfqpoint{3.156046in}{2.583253in}}%
\pgfpathcurveto{\pgfqpoint{3.150222in}{2.577429in}}{\pgfqpoint{3.146950in}{2.569529in}}{\pgfqpoint{3.146950in}{2.561292in}}%
\pgfpathcurveto{\pgfqpoint{3.146950in}{2.553056in}}{\pgfqpoint{3.150222in}{2.545156in}}{\pgfqpoint{3.156046in}{2.539332in}}%
\pgfpathcurveto{\pgfqpoint{3.161870in}{2.533508in}}{\pgfqpoint{3.169770in}{2.530236in}}{\pgfqpoint{3.178006in}{2.530236in}}%
\pgfpathclose%
\pgfusepath{stroke,fill}%
\end{pgfscope}%
\begin{pgfscope}%
\pgfpathrectangle{\pgfqpoint{0.100000in}{0.220728in}}{\pgfqpoint{3.696000in}{3.696000in}}%
\pgfusepath{clip}%
\pgfsetbuttcap%
\pgfsetroundjoin%
\definecolor{currentfill}{rgb}{0.121569,0.466667,0.705882}%
\pgfsetfillcolor{currentfill}%
\pgfsetfillopacity{0.739601}%
\pgfsetlinewidth{1.003750pt}%
\definecolor{currentstroke}{rgb}{0.121569,0.466667,0.705882}%
\pgfsetstrokecolor{currentstroke}%
\pgfsetstrokeopacity{0.739601}%
\pgfsetdash{}{0pt}%
\pgfpathmoveto{\pgfqpoint{3.177519in}{2.529313in}}%
\pgfpathcurveto{\pgfqpoint{3.185755in}{2.529313in}}{\pgfqpoint{3.193655in}{2.532586in}}{\pgfqpoint{3.199479in}{2.538410in}}%
\pgfpathcurveto{\pgfqpoint{3.205303in}{2.544234in}}{\pgfqpoint{3.208575in}{2.552134in}}{\pgfqpoint{3.208575in}{2.560370in}}%
\pgfpathcurveto{\pgfqpoint{3.208575in}{2.568606in}}{\pgfqpoint{3.205303in}{2.576506in}}{\pgfqpoint{3.199479in}{2.582330in}}%
\pgfpathcurveto{\pgfqpoint{3.193655in}{2.588154in}}{\pgfqpoint{3.185755in}{2.591426in}}{\pgfqpoint{3.177519in}{2.591426in}}%
\pgfpathcurveto{\pgfqpoint{3.169282in}{2.591426in}}{\pgfqpoint{3.161382in}{2.588154in}}{\pgfqpoint{3.155558in}{2.582330in}}%
\pgfpathcurveto{\pgfqpoint{3.149735in}{2.576506in}}{\pgfqpoint{3.146462in}{2.568606in}}{\pgfqpoint{3.146462in}{2.560370in}}%
\pgfpathcurveto{\pgfqpoint{3.146462in}{2.552134in}}{\pgfqpoint{3.149735in}{2.544234in}}{\pgfqpoint{3.155558in}{2.538410in}}%
\pgfpathcurveto{\pgfqpoint{3.161382in}{2.532586in}}{\pgfqpoint{3.169282in}{2.529313in}}{\pgfqpoint{3.177519in}{2.529313in}}%
\pgfpathclose%
\pgfusepath{stroke,fill}%
\end{pgfscope}%
\begin{pgfscope}%
\pgfpathrectangle{\pgfqpoint{0.100000in}{0.220728in}}{\pgfqpoint{3.696000in}{3.696000in}}%
\pgfusepath{clip}%
\pgfsetbuttcap%
\pgfsetroundjoin%
\definecolor{currentfill}{rgb}{0.121569,0.466667,0.705882}%
\pgfsetfillcolor{currentfill}%
\pgfsetfillopacity{0.740050}%
\pgfsetlinewidth{1.003750pt}%
\definecolor{currentstroke}{rgb}{0.121569,0.466667,0.705882}%
\pgfsetstrokecolor{currentstroke}%
\pgfsetstrokeopacity{0.740050}%
\pgfsetdash{}{0pt}%
\pgfpathmoveto{\pgfqpoint{3.176220in}{2.527740in}}%
\pgfpathcurveto{\pgfqpoint{3.184456in}{2.527740in}}{\pgfqpoint{3.192356in}{2.531012in}}{\pgfqpoint{3.198180in}{2.536836in}}%
\pgfpathcurveto{\pgfqpoint{3.204004in}{2.542660in}}{\pgfqpoint{3.207276in}{2.550560in}}{\pgfqpoint{3.207276in}{2.558796in}}%
\pgfpathcurveto{\pgfqpoint{3.207276in}{2.567032in}}{\pgfqpoint{3.204004in}{2.574932in}}{\pgfqpoint{3.198180in}{2.580756in}}%
\pgfpathcurveto{\pgfqpoint{3.192356in}{2.586580in}}{\pgfqpoint{3.184456in}{2.589853in}}{\pgfqpoint{3.176220in}{2.589853in}}%
\pgfpathcurveto{\pgfqpoint{3.167984in}{2.589853in}}{\pgfqpoint{3.160084in}{2.586580in}}{\pgfqpoint{3.154260in}{2.580756in}}%
\pgfpathcurveto{\pgfqpoint{3.148436in}{2.574932in}}{\pgfqpoint{3.145163in}{2.567032in}}{\pgfqpoint{3.145163in}{2.558796in}}%
\pgfpathcurveto{\pgfqpoint{3.145163in}{2.550560in}}{\pgfqpoint{3.148436in}{2.542660in}}{\pgfqpoint{3.154260in}{2.536836in}}%
\pgfpathcurveto{\pgfqpoint{3.160084in}{2.531012in}}{\pgfqpoint{3.167984in}{2.527740in}}{\pgfqpoint{3.176220in}{2.527740in}}%
\pgfpathclose%
\pgfusepath{stroke,fill}%
\end{pgfscope}%
\begin{pgfscope}%
\pgfpathrectangle{\pgfqpoint{0.100000in}{0.220728in}}{\pgfqpoint{3.696000in}{3.696000in}}%
\pgfusepath{clip}%
\pgfsetbuttcap%
\pgfsetroundjoin%
\definecolor{currentfill}{rgb}{0.121569,0.466667,0.705882}%
\pgfsetfillcolor{currentfill}%
\pgfsetfillopacity{0.740500}%
\pgfsetlinewidth{1.003750pt}%
\definecolor{currentstroke}{rgb}{0.121569,0.466667,0.705882}%
\pgfsetstrokecolor{currentstroke}%
\pgfsetstrokeopacity{0.740500}%
\pgfsetdash{}{0pt}%
\pgfpathmoveto{\pgfqpoint{3.174632in}{2.523756in}}%
\pgfpathcurveto{\pgfqpoint{3.182868in}{2.523756in}}{\pgfqpoint{3.190768in}{2.527028in}}{\pgfqpoint{3.196592in}{2.532852in}}%
\pgfpathcurveto{\pgfqpoint{3.202416in}{2.538676in}}{\pgfqpoint{3.205689in}{2.546576in}}{\pgfqpoint{3.205689in}{2.554812in}}%
\pgfpathcurveto{\pgfqpoint{3.205689in}{2.563049in}}{\pgfqpoint{3.202416in}{2.570949in}}{\pgfqpoint{3.196592in}{2.576773in}}%
\pgfpathcurveto{\pgfqpoint{3.190768in}{2.582596in}}{\pgfqpoint{3.182868in}{2.585869in}}{\pgfqpoint{3.174632in}{2.585869in}}%
\pgfpathcurveto{\pgfqpoint{3.166396in}{2.585869in}}{\pgfqpoint{3.158496in}{2.582596in}}{\pgfqpoint{3.152672in}{2.576773in}}%
\pgfpathcurveto{\pgfqpoint{3.146848in}{2.570949in}}{\pgfqpoint{3.143576in}{2.563049in}}{\pgfqpoint{3.143576in}{2.554812in}}%
\pgfpathcurveto{\pgfqpoint{3.143576in}{2.546576in}}{\pgfqpoint{3.146848in}{2.538676in}}{\pgfqpoint{3.152672in}{2.532852in}}%
\pgfpathcurveto{\pgfqpoint{3.158496in}{2.527028in}}{\pgfqpoint{3.166396in}{2.523756in}}{\pgfqpoint{3.174632in}{2.523756in}}%
\pgfpathclose%
\pgfusepath{stroke,fill}%
\end{pgfscope}%
\begin{pgfscope}%
\pgfpathrectangle{\pgfqpoint{0.100000in}{0.220728in}}{\pgfqpoint{3.696000in}{3.696000in}}%
\pgfusepath{clip}%
\pgfsetbuttcap%
\pgfsetroundjoin%
\definecolor{currentfill}{rgb}{0.121569,0.466667,0.705882}%
\pgfsetfillcolor{currentfill}%
\pgfsetfillopacity{0.741215}%
\pgfsetlinewidth{1.003750pt}%
\definecolor{currentstroke}{rgb}{0.121569,0.466667,0.705882}%
\pgfsetstrokecolor{currentstroke}%
\pgfsetstrokeopacity{0.741215}%
\pgfsetdash{}{0pt}%
\pgfpathmoveto{\pgfqpoint{3.171507in}{2.519587in}}%
\pgfpathcurveto{\pgfqpoint{3.179744in}{2.519587in}}{\pgfqpoint{3.187644in}{2.522860in}}{\pgfqpoint{3.193468in}{2.528684in}}%
\pgfpathcurveto{\pgfqpoint{3.199291in}{2.534508in}}{\pgfqpoint{3.202564in}{2.542408in}}{\pgfqpoint{3.202564in}{2.550644in}}%
\pgfpathcurveto{\pgfqpoint{3.202564in}{2.558880in}}{\pgfqpoint{3.199291in}{2.566780in}}{\pgfqpoint{3.193468in}{2.572604in}}%
\pgfpathcurveto{\pgfqpoint{3.187644in}{2.578428in}}{\pgfqpoint{3.179744in}{2.581700in}}{\pgfqpoint{3.171507in}{2.581700in}}%
\pgfpathcurveto{\pgfqpoint{3.163271in}{2.581700in}}{\pgfqpoint{3.155371in}{2.578428in}}{\pgfqpoint{3.149547in}{2.572604in}}%
\pgfpathcurveto{\pgfqpoint{3.143723in}{2.566780in}}{\pgfqpoint{3.140451in}{2.558880in}}{\pgfqpoint{3.140451in}{2.550644in}}%
\pgfpathcurveto{\pgfqpoint{3.140451in}{2.542408in}}{\pgfqpoint{3.143723in}{2.534508in}}{\pgfqpoint{3.149547in}{2.528684in}}%
\pgfpathcurveto{\pgfqpoint{3.155371in}{2.522860in}}{\pgfqpoint{3.163271in}{2.519587in}}{\pgfqpoint{3.171507in}{2.519587in}}%
\pgfpathclose%
\pgfusepath{stroke,fill}%
\end{pgfscope}%
\begin{pgfscope}%
\pgfpathrectangle{\pgfqpoint{0.100000in}{0.220728in}}{\pgfqpoint{3.696000in}{3.696000in}}%
\pgfusepath{clip}%
\pgfsetbuttcap%
\pgfsetroundjoin%
\definecolor{currentfill}{rgb}{0.121569,0.466667,0.705882}%
\pgfsetfillcolor{currentfill}%
\pgfsetfillopacity{0.741262}%
\pgfsetlinewidth{1.003750pt}%
\definecolor{currentstroke}{rgb}{0.121569,0.466667,0.705882}%
\pgfsetstrokecolor{currentstroke}%
\pgfsetstrokeopacity{0.741262}%
\pgfsetdash{}{0pt}%
\pgfpathmoveto{\pgfqpoint{1.042010in}{1.246566in}}%
\pgfpathcurveto{\pgfqpoint{1.050246in}{1.246566in}}{\pgfqpoint{1.058146in}{1.249838in}}{\pgfqpoint{1.063970in}{1.255662in}}%
\pgfpathcurveto{\pgfqpoint{1.069794in}{1.261486in}}{\pgfqpoint{1.073066in}{1.269386in}}{\pgfqpoint{1.073066in}{1.277622in}}%
\pgfpathcurveto{\pgfqpoint{1.073066in}{1.285859in}}{\pgfqpoint{1.069794in}{1.293759in}}{\pgfqpoint{1.063970in}{1.299583in}}%
\pgfpathcurveto{\pgfqpoint{1.058146in}{1.305407in}}{\pgfqpoint{1.050246in}{1.308679in}}{\pgfqpoint{1.042010in}{1.308679in}}%
\pgfpathcurveto{\pgfqpoint{1.033773in}{1.308679in}}{\pgfqpoint{1.025873in}{1.305407in}}{\pgfqpoint{1.020049in}{1.299583in}}%
\pgfpathcurveto{\pgfqpoint{1.014225in}{1.293759in}}{\pgfqpoint{1.010953in}{1.285859in}}{\pgfqpoint{1.010953in}{1.277622in}}%
\pgfpathcurveto{\pgfqpoint{1.010953in}{1.269386in}}{\pgfqpoint{1.014225in}{1.261486in}}{\pgfqpoint{1.020049in}{1.255662in}}%
\pgfpathcurveto{\pgfqpoint{1.025873in}{1.249838in}}{\pgfqpoint{1.033773in}{1.246566in}}{\pgfqpoint{1.042010in}{1.246566in}}%
\pgfpathclose%
\pgfusepath{stroke,fill}%
\end{pgfscope}%
\begin{pgfscope}%
\pgfpathrectangle{\pgfqpoint{0.100000in}{0.220728in}}{\pgfqpoint{3.696000in}{3.696000in}}%
\pgfusepath{clip}%
\pgfsetbuttcap%
\pgfsetroundjoin%
\definecolor{currentfill}{rgb}{0.121569,0.466667,0.705882}%
\pgfsetfillcolor{currentfill}%
\pgfsetfillopacity{0.742376}%
\pgfsetlinewidth{1.003750pt}%
\definecolor{currentstroke}{rgb}{0.121569,0.466667,0.705882}%
\pgfsetstrokecolor{currentstroke}%
\pgfsetstrokeopacity{0.742376}%
\pgfsetdash{}{0pt}%
\pgfpathmoveto{\pgfqpoint{3.168780in}{2.514765in}}%
\pgfpathcurveto{\pgfqpoint{3.177016in}{2.514765in}}{\pgfqpoint{3.184916in}{2.518037in}}{\pgfqpoint{3.190740in}{2.523861in}}%
\pgfpathcurveto{\pgfqpoint{3.196564in}{2.529685in}}{\pgfqpoint{3.199836in}{2.537585in}}{\pgfqpoint{3.199836in}{2.545821in}}%
\pgfpathcurveto{\pgfqpoint{3.199836in}{2.554058in}}{\pgfqpoint{3.196564in}{2.561958in}}{\pgfqpoint{3.190740in}{2.567782in}}%
\pgfpathcurveto{\pgfqpoint{3.184916in}{2.573606in}}{\pgfqpoint{3.177016in}{2.576878in}}{\pgfqpoint{3.168780in}{2.576878in}}%
\pgfpathcurveto{\pgfqpoint{3.160543in}{2.576878in}}{\pgfqpoint{3.152643in}{2.573606in}}{\pgfqpoint{3.146819in}{2.567782in}}%
\pgfpathcurveto{\pgfqpoint{3.140995in}{2.561958in}}{\pgfqpoint{3.137723in}{2.554058in}}{\pgfqpoint{3.137723in}{2.545821in}}%
\pgfpathcurveto{\pgfqpoint{3.137723in}{2.537585in}}{\pgfqpoint{3.140995in}{2.529685in}}{\pgfqpoint{3.146819in}{2.523861in}}%
\pgfpathcurveto{\pgfqpoint{3.152643in}{2.518037in}}{\pgfqpoint{3.160543in}{2.514765in}}{\pgfqpoint{3.168780in}{2.514765in}}%
\pgfpathclose%
\pgfusepath{stroke,fill}%
\end{pgfscope}%
\begin{pgfscope}%
\pgfpathrectangle{\pgfqpoint{0.100000in}{0.220728in}}{\pgfqpoint{3.696000in}{3.696000in}}%
\pgfusepath{clip}%
\pgfsetbuttcap%
\pgfsetroundjoin%
\definecolor{currentfill}{rgb}{0.121569,0.466667,0.705882}%
\pgfsetfillcolor{currentfill}%
\pgfsetfillopacity{0.742950}%
\pgfsetlinewidth{1.003750pt}%
\definecolor{currentstroke}{rgb}{0.121569,0.466667,0.705882}%
\pgfsetstrokecolor{currentstroke}%
\pgfsetstrokeopacity{0.742950}%
\pgfsetdash{}{0pt}%
\pgfpathmoveto{\pgfqpoint{3.167421in}{2.511681in}}%
\pgfpathcurveto{\pgfqpoint{3.175657in}{2.511681in}}{\pgfqpoint{3.183557in}{2.514953in}}{\pgfqpoint{3.189381in}{2.520777in}}%
\pgfpathcurveto{\pgfqpoint{3.195205in}{2.526601in}}{\pgfqpoint{3.198477in}{2.534501in}}{\pgfqpoint{3.198477in}{2.542737in}}%
\pgfpathcurveto{\pgfqpoint{3.198477in}{2.550974in}}{\pgfqpoint{3.195205in}{2.558874in}}{\pgfqpoint{3.189381in}{2.564698in}}%
\pgfpathcurveto{\pgfqpoint{3.183557in}{2.570522in}}{\pgfqpoint{3.175657in}{2.573794in}}{\pgfqpoint{3.167421in}{2.573794in}}%
\pgfpathcurveto{\pgfqpoint{3.159185in}{2.573794in}}{\pgfqpoint{3.151285in}{2.570522in}}{\pgfqpoint{3.145461in}{2.564698in}}%
\pgfpathcurveto{\pgfqpoint{3.139637in}{2.558874in}}{\pgfqpoint{3.136364in}{2.550974in}}{\pgfqpoint{3.136364in}{2.542737in}}%
\pgfpathcurveto{\pgfqpoint{3.136364in}{2.534501in}}{\pgfqpoint{3.139637in}{2.526601in}}{\pgfqpoint{3.145461in}{2.520777in}}%
\pgfpathcurveto{\pgfqpoint{3.151285in}{2.514953in}}{\pgfqpoint{3.159185in}{2.511681in}}{\pgfqpoint{3.167421in}{2.511681in}}%
\pgfpathclose%
\pgfusepath{stroke,fill}%
\end{pgfscope}%
\begin{pgfscope}%
\pgfpathrectangle{\pgfqpoint{0.100000in}{0.220728in}}{\pgfqpoint{3.696000in}{3.696000in}}%
\pgfusepath{clip}%
\pgfsetbuttcap%
\pgfsetroundjoin%
\definecolor{currentfill}{rgb}{0.121569,0.466667,0.705882}%
\pgfsetfillcolor{currentfill}%
\pgfsetfillopacity{0.743453}%
\pgfsetlinewidth{1.003750pt}%
\definecolor{currentstroke}{rgb}{0.121569,0.466667,0.705882}%
\pgfsetstrokecolor{currentstroke}%
\pgfsetstrokeopacity{0.743453}%
\pgfsetdash{}{0pt}%
\pgfpathmoveto{\pgfqpoint{3.164790in}{2.508454in}}%
\pgfpathcurveto{\pgfqpoint{3.173026in}{2.508454in}}{\pgfqpoint{3.180926in}{2.511727in}}{\pgfqpoint{3.186750in}{2.517551in}}%
\pgfpathcurveto{\pgfqpoint{3.192574in}{2.523375in}}{\pgfqpoint{3.195846in}{2.531275in}}{\pgfqpoint{3.195846in}{2.539511in}}%
\pgfpathcurveto{\pgfqpoint{3.195846in}{2.547747in}}{\pgfqpoint{3.192574in}{2.555647in}}{\pgfqpoint{3.186750in}{2.561471in}}%
\pgfpathcurveto{\pgfqpoint{3.180926in}{2.567295in}}{\pgfqpoint{3.173026in}{2.570567in}}{\pgfqpoint{3.164790in}{2.570567in}}%
\pgfpathcurveto{\pgfqpoint{3.156554in}{2.570567in}}{\pgfqpoint{3.148654in}{2.567295in}}{\pgfqpoint{3.142830in}{2.561471in}}%
\pgfpathcurveto{\pgfqpoint{3.137006in}{2.555647in}}{\pgfqpoint{3.133733in}{2.547747in}}{\pgfqpoint{3.133733in}{2.539511in}}%
\pgfpathcurveto{\pgfqpoint{3.133733in}{2.531275in}}{\pgfqpoint{3.137006in}{2.523375in}}{\pgfqpoint{3.142830in}{2.517551in}}%
\pgfpathcurveto{\pgfqpoint{3.148654in}{2.511727in}}{\pgfqpoint{3.156554in}{2.508454in}}{\pgfqpoint{3.164790in}{2.508454in}}%
\pgfpathclose%
\pgfusepath{stroke,fill}%
\end{pgfscope}%
\begin{pgfscope}%
\pgfpathrectangle{\pgfqpoint{0.100000in}{0.220728in}}{\pgfqpoint{3.696000in}{3.696000in}}%
\pgfusepath{clip}%
\pgfsetbuttcap%
\pgfsetroundjoin%
\definecolor{currentfill}{rgb}{0.121569,0.466667,0.705882}%
\pgfsetfillcolor{currentfill}%
\pgfsetfillopacity{0.744462}%
\pgfsetlinewidth{1.003750pt}%
\definecolor{currentstroke}{rgb}{0.121569,0.466667,0.705882}%
\pgfsetstrokecolor{currentstroke}%
\pgfsetstrokeopacity{0.744462}%
\pgfsetdash{}{0pt}%
\pgfpathmoveto{\pgfqpoint{3.162855in}{2.502322in}}%
\pgfpathcurveto{\pgfqpoint{3.171092in}{2.502322in}}{\pgfqpoint{3.178992in}{2.505594in}}{\pgfqpoint{3.184816in}{2.511418in}}%
\pgfpathcurveto{\pgfqpoint{3.190640in}{2.517242in}}{\pgfqpoint{3.193912in}{2.525142in}}{\pgfqpoint{3.193912in}{2.533379in}}%
\pgfpathcurveto{\pgfqpoint{3.193912in}{2.541615in}}{\pgfqpoint{3.190640in}{2.549515in}}{\pgfqpoint{3.184816in}{2.555339in}}%
\pgfpathcurveto{\pgfqpoint{3.178992in}{2.561163in}}{\pgfqpoint{3.171092in}{2.564435in}}{\pgfqpoint{3.162855in}{2.564435in}}%
\pgfpathcurveto{\pgfqpoint{3.154619in}{2.564435in}}{\pgfqpoint{3.146719in}{2.561163in}}{\pgfqpoint{3.140895in}{2.555339in}}%
\pgfpathcurveto{\pgfqpoint{3.135071in}{2.549515in}}{\pgfqpoint{3.131799in}{2.541615in}}{\pgfqpoint{3.131799in}{2.533379in}}%
\pgfpathcurveto{\pgfqpoint{3.131799in}{2.525142in}}{\pgfqpoint{3.135071in}{2.517242in}}{\pgfqpoint{3.140895in}{2.511418in}}%
\pgfpathcurveto{\pgfqpoint{3.146719in}{2.505594in}}{\pgfqpoint{3.154619in}{2.502322in}}{\pgfqpoint{3.162855in}{2.502322in}}%
\pgfpathclose%
\pgfusepath{stroke,fill}%
\end{pgfscope}%
\begin{pgfscope}%
\pgfpathrectangle{\pgfqpoint{0.100000in}{0.220728in}}{\pgfqpoint{3.696000in}{3.696000in}}%
\pgfusepath{clip}%
\pgfsetbuttcap%
\pgfsetroundjoin%
\definecolor{currentfill}{rgb}{0.121569,0.466667,0.705882}%
\pgfsetfillcolor{currentfill}%
\pgfsetfillopacity{0.744992}%
\pgfsetlinewidth{1.003750pt}%
\definecolor{currentstroke}{rgb}{0.121569,0.466667,0.705882}%
\pgfsetstrokecolor{currentstroke}%
\pgfsetstrokeopacity{0.744992}%
\pgfsetdash{}{0pt}%
\pgfpathmoveto{\pgfqpoint{3.161169in}{2.499511in}}%
\pgfpathcurveto{\pgfqpoint{3.169405in}{2.499511in}}{\pgfqpoint{3.177305in}{2.502784in}}{\pgfqpoint{3.183129in}{2.508608in}}%
\pgfpathcurveto{\pgfqpoint{3.188953in}{2.514432in}}{\pgfqpoint{3.192225in}{2.522332in}}{\pgfqpoint{3.192225in}{2.530568in}}%
\pgfpathcurveto{\pgfqpoint{3.192225in}{2.538804in}}{\pgfqpoint{3.188953in}{2.546704in}}{\pgfqpoint{3.183129in}{2.552528in}}%
\pgfpathcurveto{\pgfqpoint{3.177305in}{2.558352in}}{\pgfqpoint{3.169405in}{2.561624in}}{\pgfqpoint{3.161169in}{2.561624in}}%
\pgfpathcurveto{\pgfqpoint{3.152932in}{2.561624in}}{\pgfqpoint{3.145032in}{2.558352in}}{\pgfqpoint{3.139208in}{2.552528in}}%
\pgfpathcurveto{\pgfqpoint{3.133385in}{2.546704in}}{\pgfqpoint{3.130112in}{2.538804in}}{\pgfqpoint{3.130112in}{2.530568in}}%
\pgfpathcurveto{\pgfqpoint{3.130112in}{2.522332in}}{\pgfqpoint{3.133385in}{2.514432in}}{\pgfqpoint{3.139208in}{2.508608in}}%
\pgfpathcurveto{\pgfqpoint{3.145032in}{2.502784in}}{\pgfqpoint{3.152932in}{2.499511in}}{\pgfqpoint{3.161169in}{2.499511in}}%
\pgfpathclose%
\pgfusepath{stroke,fill}%
\end{pgfscope}%
\begin{pgfscope}%
\pgfpathrectangle{\pgfqpoint{0.100000in}{0.220728in}}{\pgfqpoint{3.696000in}{3.696000in}}%
\pgfusepath{clip}%
\pgfsetbuttcap%
\pgfsetroundjoin%
\definecolor{currentfill}{rgb}{0.121569,0.466667,0.705882}%
\pgfsetfillcolor{currentfill}%
\pgfsetfillopacity{0.745266}%
\pgfsetlinewidth{1.003750pt}%
\definecolor{currentstroke}{rgb}{0.121569,0.466667,0.705882}%
\pgfsetstrokecolor{currentstroke}%
\pgfsetstrokeopacity{0.745266}%
\pgfsetdash{}{0pt}%
\pgfpathmoveto{\pgfqpoint{3.160170in}{2.498000in}}%
\pgfpathcurveto{\pgfqpoint{3.168407in}{2.498000in}}{\pgfqpoint{3.176307in}{2.501272in}}{\pgfqpoint{3.182131in}{2.507096in}}%
\pgfpathcurveto{\pgfqpoint{3.187955in}{2.512920in}}{\pgfqpoint{3.191227in}{2.520820in}}{\pgfqpoint{3.191227in}{2.529057in}}%
\pgfpathcurveto{\pgfqpoint{3.191227in}{2.537293in}}{\pgfqpoint{3.187955in}{2.545193in}}{\pgfqpoint{3.182131in}{2.551017in}}%
\pgfpathcurveto{\pgfqpoint{3.176307in}{2.556841in}}{\pgfqpoint{3.168407in}{2.560113in}}{\pgfqpoint{3.160170in}{2.560113in}}%
\pgfpathcurveto{\pgfqpoint{3.151934in}{2.560113in}}{\pgfqpoint{3.144034in}{2.556841in}}{\pgfqpoint{3.138210in}{2.551017in}}%
\pgfpathcurveto{\pgfqpoint{3.132386in}{2.545193in}}{\pgfqpoint{3.129114in}{2.537293in}}{\pgfqpoint{3.129114in}{2.529057in}}%
\pgfpathcurveto{\pgfqpoint{3.129114in}{2.520820in}}{\pgfqpoint{3.132386in}{2.512920in}}{\pgfqpoint{3.138210in}{2.507096in}}%
\pgfpathcurveto{\pgfqpoint{3.144034in}{2.501272in}}{\pgfqpoint{3.151934in}{2.498000in}}{\pgfqpoint{3.160170in}{2.498000in}}%
\pgfpathclose%
\pgfusepath{stroke,fill}%
\end{pgfscope}%
\begin{pgfscope}%
\pgfpathrectangle{\pgfqpoint{0.100000in}{0.220728in}}{\pgfqpoint{3.696000in}{3.696000in}}%
\pgfusepath{clip}%
\pgfsetbuttcap%
\pgfsetroundjoin%
\definecolor{currentfill}{rgb}{0.121569,0.466667,0.705882}%
\pgfsetfillcolor{currentfill}%
\pgfsetfillopacity{0.745288}%
\pgfsetlinewidth{1.003750pt}%
\definecolor{currentstroke}{rgb}{0.121569,0.466667,0.705882}%
\pgfsetstrokecolor{currentstroke}%
\pgfsetstrokeopacity{0.745288}%
\pgfsetdash{}{0pt}%
\pgfpathmoveto{\pgfqpoint{1.062559in}{1.235962in}}%
\pgfpathcurveto{\pgfqpoint{1.070795in}{1.235962in}}{\pgfqpoint{1.078695in}{1.239234in}}{\pgfqpoint{1.084519in}{1.245058in}}%
\pgfpathcurveto{\pgfqpoint{1.090343in}{1.250882in}}{\pgfqpoint{1.093615in}{1.258782in}}{\pgfqpoint{1.093615in}{1.267018in}}%
\pgfpathcurveto{\pgfqpoint{1.093615in}{1.275255in}}{\pgfqpoint{1.090343in}{1.283155in}}{\pgfqpoint{1.084519in}{1.288979in}}%
\pgfpathcurveto{\pgfqpoint{1.078695in}{1.294803in}}{\pgfqpoint{1.070795in}{1.298075in}}{\pgfqpoint{1.062559in}{1.298075in}}%
\pgfpathcurveto{\pgfqpoint{1.054322in}{1.298075in}}{\pgfqpoint{1.046422in}{1.294803in}}{\pgfqpoint{1.040598in}{1.288979in}}%
\pgfpathcurveto{\pgfqpoint{1.034774in}{1.283155in}}{\pgfqpoint{1.031502in}{1.275255in}}{\pgfqpoint{1.031502in}{1.267018in}}%
\pgfpathcurveto{\pgfqpoint{1.031502in}{1.258782in}}{\pgfqpoint{1.034774in}{1.250882in}}{\pgfqpoint{1.040598in}{1.245058in}}%
\pgfpathcurveto{\pgfqpoint{1.046422in}{1.239234in}}{\pgfqpoint{1.054322in}{1.235962in}}{\pgfqpoint{1.062559in}{1.235962in}}%
\pgfpathclose%
\pgfusepath{stroke,fill}%
\end{pgfscope}%
\begin{pgfscope}%
\pgfpathrectangle{\pgfqpoint{0.100000in}{0.220728in}}{\pgfqpoint{3.696000in}{3.696000in}}%
\pgfusepath{clip}%
\pgfsetbuttcap%
\pgfsetroundjoin%
\definecolor{currentfill}{rgb}{0.121569,0.466667,0.705882}%
\pgfsetfillcolor{currentfill}%
\pgfsetfillopacity{0.745438}%
\pgfsetlinewidth{1.003750pt}%
\definecolor{currentstroke}{rgb}{0.121569,0.466667,0.705882}%
\pgfsetstrokecolor{currentstroke}%
\pgfsetstrokeopacity{0.745438}%
\pgfsetdash{}{0pt}%
\pgfpathmoveto{\pgfqpoint{3.159850in}{2.496999in}}%
\pgfpathcurveto{\pgfqpoint{3.168086in}{2.496999in}}{\pgfqpoint{3.175986in}{2.500271in}}{\pgfqpoint{3.181810in}{2.506095in}}%
\pgfpathcurveto{\pgfqpoint{3.187634in}{2.511919in}}{\pgfqpoint{3.190906in}{2.519819in}}{\pgfqpoint{3.190906in}{2.528055in}}%
\pgfpathcurveto{\pgfqpoint{3.190906in}{2.536292in}}{\pgfqpoint{3.187634in}{2.544192in}}{\pgfqpoint{3.181810in}{2.550015in}}%
\pgfpathcurveto{\pgfqpoint{3.175986in}{2.555839in}}{\pgfqpoint{3.168086in}{2.559112in}}{\pgfqpoint{3.159850in}{2.559112in}}%
\pgfpathcurveto{\pgfqpoint{3.151613in}{2.559112in}}{\pgfqpoint{3.143713in}{2.555839in}}{\pgfqpoint{3.137889in}{2.550015in}}%
\pgfpathcurveto{\pgfqpoint{3.132065in}{2.544192in}}{\pgfqpoint{3.128793in}{2.536292in}}{\pgfqpoint{3.128793in}{2.528055in}}%
\pgfpathcurveto{\pgfqpoint{3.128793in}{2.519819in}}{\pgfqpoint{3.132065in}{2.511919in}}{\pgfqpoint{3.137889in}{2.506095in}}%
\pgfpathcurveto{\pgfqpoint{3.143713in}{2.500271in}}{\pgfqpoint{3.151613in}{2.496999in}}{\pgfqpoint{3.159850in}{2.496999in}}%
\pgfpathclose%
\pgfusepath{stroke,fill}%
\end{pgfscope}%
\begin{pgfscope}%
\pgfpathrectangle{\pgfqpoint{0.100000in}{0.220728in}}{\pgfqpoint{3.696000in}{3.696000in}}%
\pgfusepath{clip}%
\pgfsetbuttcap%
\pgfsetroundjoin%
\definecolor{currentfill}{rgb}{0.121569,0.466667,0.705882}%
\pgfsetfillcolor{currentfill}%
\pgfsetfillopacity{0.746029}%
\pgfsetlinewidth{1.003750pt}%
\definecolor{currentstroke}{rgb}{0.121569,0.466667,0.705882}%
\pgfsetstrokecolor{currentstroke}%
\pgfsetstrokeopacity{0.746029}%
\pgfsetdash{}{0pt}%
\pgfpathmoveto{\pgfqpoint{3.157986in}{2.494151in}}%
\pgfpathcurveto{\pgfqpoint{3.166223in}{2.494151in}}{\pgfqpoint{3.174123in}{2.497423in}}{\pgfqpoint{3.179947in}{2.503247in}}%
\pgfpathcurveto{\pgfqpoint{3.185771in}{2.509071in}}{\pgfqpoint{3.189043in}{2.516971in}}{\pgfqpoint{3.189043in}{2.525207in}}%
\pgfpathcurveto{\pgfqpoint{3.189043in}{2.533444in}}{\pgfqpoint{3.185771in}{2.541344in}}{\pgfqpoint{3.179947in}{2.547168in}}%
\pgfpathcurveto{\pgfqpoint{3.174123in}{2.552991in}}{\pgfqpoint{3.166223in}{2.556264in}}{\pgfqpoint{3.157986in}{2.556264in}}%
\pgfpathcurveto{\pgfqpoint{3.149750in}{2.556264in}}{\pgfqpoint{3.141850in}{2.552991in}}{\pgfqpoint{3.136026in}{2.547168in}}%
\pgfpathcurveto{\pgfqpoint{3.130202in}{2.541344in}}{\pgfqpoint{3.126930in}{2.533444in}}{\pgfqpoint{3.126930in}{2.525207in}}%
\pgfpathcurveto{\pgfqpoint{3.126930in}{2.516971in}}{\pgfqpoint{3.130202in}{2.509071in}}{\pgfqpoint{3.136026in}{2.503247in}}%
\pgfpathcurveto{\pgfqpoint{3.141850in}{2.497423in}}{\pgfqpoint{3.149750in}{2.494151in}}{\pgfqpoint{3.157986in}{2.494151in}}%
\pgfpathclose%
\pgfusepath{stroke,fill}%
\end{pgfscope}%
\begin{pgfscope}%
\pgfpathrectangle{\pgfqpoint{0.100000in}{0.220728in}}{\pgfqpoint{3.696000in}{3.696000in}}%
\pgfusepath{clip}%
\pgfsetbuttcap%
\pgfsetroundjoin%
\definecolor{currentfill}{rgb}{0.121569,0.466667,0.705882}%
\pgfsetfillcolor{currentfill}%
\pgfsetfillopacity{0.746887}%
\pgfsetlinewidth{1.003750pt}%
\definecolor{currentstroke}{rgb}{0.121569,0.466667,0.705882}%
\pgfsetstrokecolor{currentstroke}%
\pgfsetstrokeopacity{0.746887}%
\pgfsetdash{}{0pt}%
\pgfpathmoveto{\pgfqpoint{3.155944in}{2.490126in}}%
\pgfpathcurveto{\pgfqpoint{3.164181in}{2.490126in}}{\pgfqpoint{3.172081in}{2.493399in}}{\pgfqpoint{3.177905in}{2.499222in}}%
\pgfpathcurveto{\pgfqpoint{3.183729in}{2.505046in}}{\pgfqpoint{3.187001in}{2.512946in}}{\pgfqpoint{3.187001in}{2.521183in}}%
\pgfpathcurveto{\pgfqpoint{3.187001in}{2.529419in}}{\pgfqpoint{3.183729in}{2.537319in}}{\pgfqpoint{3.177905in}{2.543143in}}%
\pgfpathcurveto{\pgfqpoint{3.172081in}{2.548967in}}{\pgfqpoint{3.164181in}{2.552239in}}{\pgfqpoint{3.155944in}{2.552239in}}%
\pgfpathcurveto{\pgfqpoint{3.147708in}{2.552239in}}{\pgfqpoint{3.139808in}{2.548967in}}{\pgfqpoint{3.133984in}{2.543143in}}%
\pgfpathcurveto{\pgfqpoint{3.128160in}{2.537319in}}{\pgfqpoint{3.124888in}{2.529419in}}{\pgfqpoint{3.124888in}{2.521183in}}%
\pgfpathcurveto{\pgfqpoint{3.124888in}{2.512946in}}{\pgfqpoint{3.128160in}{2.505046in}}{\pgfqpoint{3.133984in}{2.499222in}}%
\pgfpathcurveto{\pgfqpoint{3.139808in}{2.493399in}}{\pgfqpoint{3.147708in}{2.490126in}}{\pgfqpoint{3.155944in}{2.490126in}}%
\pgfpathclose%
\pgfusepath{stroke,fill}%
\end{pgfscope}%
\begin{pgfscope}%
\pgfpathrectangle{\pgfqpoint{0.100000in}{0.220728in}}{\pgfqpoint{3.696000in}{3.696000in}}%
\pgfusepath{clip}%
\pgfsetbuttcap%
\pgfsetroundjoin%
\definecolor{currentfill}{rgb}{0.121569,0.466667,0.705882}%
\pgfsetfillcolor{currentfill}%
\pgfsetfillopacity{0.747330}%
\pgfsetlinewidth{1.003750pt}%
\definecolor{currentstroke}{rgb}{0.121569,0.466667,0.705882}%
\pgfsetstrokecolor{currentstroke}%
\pgfsetstrokeopacity{0.747330}%
\pgfsetdash{}{0pt}%
\pgfpathmoveto{\pgfqpoint{3.154753in}{2.487867in}}%
\pgfpathcurveto{\pgfqpoint{3.162989in}{2.487867in}}{\pgfqpoint{3.170889in}{2.491139in}}{\pgfqpoint{3.176713in}{2.496963in}}%
\pgfpathcurveto{\pgfqpoint{3.182537in}{2.502787in}}{\pgfqpoint{3.185809in}{2.510687in}}{\pgfqpoint{3.185809in}{2.518923in}}%
\pgfpathcurveto{\pgfqpoint{3.185809in}{2.527159in}}{\pgfqpoint{3.182537in}{2.535059in}}{\pgfqpoint{3.176713in}{2.540883in}}%
\pgfpathcurveto{\pgfqpoint{3.170889in}{2.546707in}}{\pgfqpoint{3.162989in}{2.549980in}}{\pgfqpoint{3.154753in}{2.549980in}}%
\pgfpathcurveto{\pgfqpoint{3.146516in}{2.549980in}}{\pgfqpoint{3.138616in}{2.546707in}}{\pgfqpoint{3.132792in}{2.540883in}}%
\pgfpathcurveto{\pgfqpoint{3.126968in}{2.535059in}}{\pgfqpoint{3.123696in}{2.527159in}}{\pgfqpoint{3.123696in}{2.518923in}}%
\pgfpathcurveto{\pgfqpoint{3.123696in}{2.510687in}}{\pgfqpoint{3.126968in}{2.502787in}}{\pgfqpoint{3.132792in}{2.496963in}}%
\pgfpathcurveto{\pgfqpoint{3.138616in}{2.491139in}}{\pgfqpoint{3.146516in}{2.487867in}}{\pgfqpoint{3.154753in}{2.487867in}}%
\pgfpathclose%
\pgfusepath{stroke,fill}%
\end{pgfscope}%
\begin{pgfscope}%
\pgfpathrectangle{\pgfqpoint{0.100000in}{0.220728in}}{\pgfqpoint{3.696000in}{3.696000in}}%
\pgfusepath{clip}%
\pgfsetbuttcap%
\pgfsetroundjoin%
\definecolor{currentfill}{rgb}{0.121569,0.466667,0.705882}%
\pgfsetfillcolor{currentfill}%
\pgfsetfillopacity{0.747608}%
\pgfsetlinewidth{1.003750pt}%
\definecolor{currentstroke}{rgb}{0.121569,0.466667,0.705882}%
\pgfsetstrokecolor{currentstroke}%
\pgfsetstrokeopacity{0.747608}%
\pgfsetdash{}{0pt}%
\pgfpathmoveto{\pgfqpoint{3.154001in}{2.486909in}}%
\pgfpathcurveto{\pgfqpoint{3.162237in}{2.486909in}}{\pgfqpoint{3.170137in}{2.490182in}}{\pgfqpoint{3.175961in}{2.496006in}}%
\pgfpathcurveto{\pgfqpoint{3.181785in}{2.501830in}}{\pgfqpoint{3.185058in}{2.509730in}}{\pgfqpoint{3.185058in}{2.517966in}}%
\pgfpathcurveto{\pgfqpoint{3.185058in}{2.526202in}}{\pgfqpoint{3.181785in}{2.534102in}}{\pgfqpoint{3.175961in}{2.539926in}}%
\pgfpathcurveto{\pgfqpoint{3.170137in}{2.545750in}}{\pgfqpoint{3.162237in}{2.549022in}}{\pgfqpoint{3.154001in}{2.549022in}}%
\pgfpathcurveto{\pgfqpoint{3.145765in}{2.549022in}}{\pgfqpoint{3.137865in}{2.545750in}}{\pgfqpoint{3.132041in}{2.539926in}}%
\pgfpathcurveto{\pgfqpoint{3.126217in}{2.534102in}}{\pgfqpoint{3.122945in}{2.526202in}}{\pgfqpoint{3.122945in}{2.517966in}}%
\pgfpathcurveto{\pgfqpoint{3.122945in}{2.509730in}}{\pgfqpoint{3.126217in}{2.501830in}}{\pgfqpoint{3.132041in}{2.496006in}}%
\pgfpathcurveto{\pgfqpoint{3.137865in}{2.490182in}}{\pgfqpoint{3.145765in}{2.486909in}}{\pgfqpoint{3.154001in}{2.486909in}}%
\pgfpathclose%
\pgfusepath{stroke,fill}%
\end{pgfscope}%
\begin{pgfscope}%
\pgfpathrectangle{\pgfqpoint{0.100000in}{0.220728in}}{\pgfqpoint{3.696000in}{3.696000in}}%
\pgfusepath{clip}%
\pgfsetbuttcap%
\pgfsetroundjoin%
\definecolor{currentfill}{rgb}{0.121569,0.466667,0.705882}%
\pgfsetfillcolor{currentfill}%
\pgfsetfillopacity{0.748264}%
\pgfsetlinewidth{1.003750pt}%
\definecolor{currentstroke}{rgb}{0.121569,0.466667,0.705882}%
\pgfsetstrokecolor{currentstroke}%
\pgfsetstrokeopacity{0.748264}%
\pgfsetdash{}{0pt}%
\pgfpathmoveto{\pgfqpoint{3.152558in}{2.483186in}}%
\pgfpathcurveto{\pgfqpoint{3.160794in}{2.483186in}}{\pgfqpoint{3.168694in}{2.486458in}}{\pgfqpoint{3.174518in}{2.492282in}}%
\pgfpathcurveto{\pgfqpoint{3.180342in}{2.498106in}}{\pgfqpoint{3.183614in}{2.506006in}}{\pgfqpoint{3.183614in}{2.514242in}}%
\pgfpathcurveto{\pgfqpoint{3.183614in}{2.522478in}}{\pgfqpoint{3.180342in}{2.530378in}}{\pgfqpoint{3.174518in}{2.536202in}}%
\pgfpathcurveto{\pgfqpoint{3.168694in}{2.542026in}}{\pgfqpoint{3.160794in}{2.545299in}}{\pgfqpoint{3.152558in}{2.545299in}}%
\pgfpathcurveto{\pgfqpoint{3.144321in}{2.545299in}}{\pgfqpoint{3.136421in}{2.542026in}}{\pgfqpoint{3.130597in}{2.536202in}}%
\pgfpathcurveto{\pgfqpoint{3.124773in}{2.530378in}}{\pgfqpoint{3.121501in}{2.522478in}}{\pgfqpoint{3.121501in}{2.514242in}}%
\pgfpathcurveto{\pgfqpoint{3.121501in}{2.506006in}}{\pgfqpoint{3.124773in}{2.498106in}}{\pgfqpoint{3.130597in}{2.492282in}}%
\pgfpathcurveto{\pgfqpoint{3.136421in}{2.486458in}}{\pgfqpoint{3.144321in}{2.483186in}}{\pgfqpoint{3.152558in}{2.483186in}}%
\pgfpathclose%
\pgfusepath{stroke,fill}%
\end{pgfscope}%
\begin{pgfscope}%
\pgfpathrectangle{\pgfqpoint{0.100000in}{0.220728in}}{\pgfqpoint{3.696000in}{3.696000in}}%
\pgfusepath{clip}%
\pgfsetbuttcap%
\pgfsetroundjoin%
\definecolor{currentfill}{rgb}{0.121569,0.466667,0.705882}%
\pgfsetfillcolor{currentfill}%
\pgfsetfillopacity{0.748634}%
\pgfsetlinewidth{1.003750pt}%
\definecolor{currentstroke}{rgb}{0.121569,0.466667,0.705882}%
\pgfsetstrokecolor{currentstroke}%
\pgfsetstrokeopacity{0.748634}%
\pgfsetdash{}{0pt}%
\pgfpathmoveto{\pgfqpoint{3.151542in}{2.481425in}}%
\pgfpathcurveto{\pgfqpoint{3.159778in}{2.481425in}}{\pgfqpoint{3.167678in}{2.484697in}}{\pgfqpoint{3.173502in}{2.490521in}}%
\pgfpathcurveto{\pgfqpoint{3.179326in}{2.496345in}}{\pgfqpoint{3.182599in}{2.504245in}}{\pgfqpoint{3.182599in}{2.512481in}}%
\pgfpathcurveto{\pgfqpoint{3.182599in}{2.520718in}}{\pgfqpoint{3.179326in}{2.528618in}}{\pgfqpoint{3.173502in}{2.534442in}}%
\pgfpathcurveto{\pgfqpoint{3.167678in}{2.540266in}}{\pgfqpoint{3.159778in}{2.543538in}}{\pgfqpoint{3.151542in}{2.543538in}}%
\pgfpathcurveto{\pgfqpoint{3.143306in}{2.543538in}}{\pgfqpoint{3.135406in}{2.540266in}}{\pgfqpoint{3.129582in}{2.534442in}}%
\pgfpathcurveto{\pgfqpoint{3.123758in}{2.528618in}}{\pgfqpoint{3.120486in}{2.520718in}}{\pgfqpoint{3.120486in}{2.512481in}}%
\pgfpathcurveto{\pgfqpoint{3.120486in}{2.504245in}}{\pgfqpoint{3.123758in}{2.496345in}}{\pgfqpoint{3.129582in}{2.490521in}}%
\pgfpathcurveto{\pgfqpoint{3.135406in}{2.484697in}}{\pgfqpoint{3.143306in}{2.481425in}}{\pgfqpoint{3.151542in}{2.481425in}}%
\pgfpathclose%
\pgfusepath{stroke,fill}%
\end{pgfscope}%
\begin{pgfscope}%
\pgfpathrectangle{\pgfqpoint{0.100000in}{0.220728in}}{\pgfqpoint{3.696000in}{3.696000in}}%
\pgfusepath{clip}%
\pgfsetbuttcap%
\pgfsetroundjoin%
\definecolor{currentfill}{rgb}{0.121569,0.466667,0.705882}%
\pgfsetfillcolor{currentfill}%
\pgfsetfillopacity{0.748834}%
\pgfsetlinewidth{1.003750pt}%
\definecolor{currentstroke}{rgb}{0.121569,0.466667,0.705882}%
\pgfsetstrokecolor{currentstroke}%
\pgfsetstrokeopacity{0.748834}%
\pgfsetdash{}{0pt}%
\pgfpathmoveto{\pgfqpoint{3.150928in}{2.480523in}}%
\pgfpathcurveto{\pgfqpoint{3.159165in}{2.480523in}}{\pgfqpoint{3.167065in}{2.483796in}}{\pgfqpoint{3.172889in}{2.489619in}}%
\pgfpathcurveto{\pgfqpoint{3.178712in}{2.495443in}}{\pgfqpoint{3.181985in}{2.503343in}}{\pgfqpoint{3.181985in}{2.511580in}}%
\pgfpathcurveto{\pgfqpoint{3.181985in}{2.519816in}}{\pgfqpoint{3.178712in}{2.527716in}}{\pgfqpoint{3.172889in}{2.533540in}}%
\pgfpathcurveto{\pgfqpoint{3.167065in}{2.539364in}}{\pgfqpoint{3.159165in}{2.542636in}}{\pgfqpoint{3.150928in}{2.542636in}}%
\pgfpathcurveto{\pgfqpoint{3.142692in}{2.542636in}}{\pgfqpoint{3.134792in}{2.539364in}}{\pgfqpoint{3.128968in}{2.533540in}}%
\pgfpathcurveto{\pgfqpoint{3.123144in}{2.527716in}}{\pgfqpoint{3.119872in}{2.519816in}}{\pgfqpoint{3.119872in}{2.511580in}}%
\pgfpathcurveto{\pgfqpoint{3.119872in}{2.503343in}}{\pgfqpoint{3.123144in}{2.495443in}}{\pgfqpoint{3.128968in}{2.489619in}}%
\pgfpathcurveto{\pgfqpoint{3.134792in}{2.483796in}}{\pgfqpoint{3.142692in}{2.480523in}}{\pgfqpoint{3.150928in}{2.480523in}}%
\pgfpathclose%
\pgfusepath{stroke,fill}%
\end{pgfscope}%
\begin{pgfscope}%
\pgfpathrectangle{\pgfqpoint{0.100000in}{0.220728in}}{\pgfqpoint{3.696000in}{3.696000in}}%
\pgfusepath{clip}%
\pgfsetbuttcap%
\pgfsetroundjoin%
\definecolor{currentfill}{rgb}{0.121569,0.466667,0.705882}%
\pgfsetfillcolor{currentfill}%
\pgfsetfillopacity{0.749170}%
\pgfsetlinewidth{1.003750pt}%
\definecolor{currentstroke}{rgb}{0.121569,0.466667,0.705882}%
\pgfsetstrokecolor{currentstroke}%
\pgfsetstrokeopacity{0.749170}%
\pgfsetdash{}{0pt}%
\pgfpathmoveto{\pgfqpoint{3.150260in}{2.478885in}}%
\pgfpathcurveto{\pgfqpoint{3.158497in}{2.478885in}}{\pgfqpoint{3.166397in}{2.482157in}}{\pgfqpoint{3.172221in}{2.487981in}}%
\pgfpathcurveto{\pgfqpoint{3.178045in}{2.493805in}}{\pgfqpoint{3.181317in}{2.501705in}}{\pgfqpoint{3.181317in}{2.509941in}}%
\pgfpathcurveto{\pgfqpoint{3.181317in}{2.518177in}}{\pgfqpoint{3.178045in}{2.526077in}}{\pgfqpoint{3.172221in}{2.531901in}}%
\pgfpathcurveto{\pgfqpoint{3.166397in}{2.537725in}}{\pgfqpoint{3.158497in}{2.540998in}}{\pgfqpoint{3.150260in}{2.540998in}}%
\pgfpathcurveto{\pgfqpoint{3.142024in}{2.540998in}}{\pgfqpoint{3.134124in}{2.537725in}}{\pgfqpoint{3.128300in}{2.531901in}}%
\pgfpathcurveto{\pgfqpoint{3.122476in}{2.526077in}}{\pgfqpoint{3.119204in}{2.518177in}}{\pgfqpoint{3.119204in}{2.509941in}}%
\pgfpathcurveto{\pgfqpoint{3.119204in}{2.501705in}}{\pgfqpoint{3.122476in}{2.493805in}}{\pgfqpoint{3.128300in}{2.487981in}}%
\pgfpathcurveto{\pgfqpoint{3.134124in}{2.482157in}}{\pgfqpoint{3.142024in}{2.478885in}}{\pgfqpoint{3.150260in}{2.478885in}}%
\pgfpathclose%
\pgfusepath{stroke,fill}%
\end{pgfscope}%
\begin{pgfscope}%
\pgfpathrectangle{\pgfqpoint{0.100000in}{0.220728in}}{\pgfqpoint{3.696000in}{3.696000in}}%
\pgfusepath{clip}%
\pgfsetbuttcap%
\pgfsetroundjoin%
\definecolor{currentfill}{rgb}{0.121569,0.466667,0.705882}%
\pgfsetfillcolor{currentfill}%
\pgfsetfillopacity{0.749706}%
\pgfsetlinewidth{1.003750pt}%
\definecolor{currentstroke}{rgb}{0.121569,0.466667,0.705882}%
\pgfsetstrokecolor{currentstroke}%
\pgfsetstrokeopacity{0.749706}%
\pgfsetdash{}{0pt}%
\pgfpathmoveto{\pgfqpoint{3.148227in}{2.475974in}}%
\pgfpathcurveto{\pgfqpoint{3.156463in}{2.475974in}}{\pgfqpoint{3.164363in}{2.479246in}}{\pgfqpoint{3.170187in}{2.485070in}}%
\pgfpathcurveto{\pgfqpoint{3.176011in}{2.490894in}}{\pgfqpoint{3.179283in}{2.498794in}}{\pgfqpoint{3.179283in}{2.507030in}}%
\pgfpathcurveto{\pgfqpoint{3.179283in}{2.515266in}}{\pgfqpoint{3.176011in}{2.523166in}}{\pgfqpoint{3.170187in}{2.528990in}}%
\pgfpathcurveto{\pgfqpoint{3.164363in}{2.534814in}}{\pgfqpoint{3.156463in}{2.538087in}}{\pgfqpoint{3.148227in}{2.538087in}}%
\pgfpathcurveto{\pgfqpoint{3.139991in}{2.538087in}}{\pgfqpoint{3.132090in}{2.534814in}}{\pgfqpoint{3.126267in}{2.528990in}}%
\pgfpathcurveto{\pgfqpoint{3.120443in}{2.523166in}}{\pgfqpoint{3.117170in}{2.515266in}}{\pgfqpoint{3.117170in}{2.507030in}}%
\pgfpathcurveto{\pgfqpoint{3.117170in}{2.498794in}}{\pgfqpoint{3.120443in}{2.490894in}}{\pgfqpoint{3.126267in}{2.485070in}}%
\pgfpathcurveto{\pgfqpoint{3.132090in}{2.479246in}}{\pgfqpoint{3.139991in}{2.475974in}}{\pgfqpoint{3.148227in}{2.475974in}}%
\pgfpathclose%
\pgfusepath{stroke,fill}%
\end{pgfscope}%
\begin{pgfscope}%
\pgfpathrectangle{\pgfqpoint{0.100000in}{0.220728in}}{\pgfqpoint{3.696000in}{3.696000in}}%
\pgfusepath{clip}%
\pgfsetbuttcap%
\pgfsetroundjoin%
\definecolor{currentfill}{rgb}{0.121569,0.466667,0.705882}%
\pgfsetfillcolor{currentfill}%
\pgfsetfillopacity{0.750451}%
\pgfsetlinewidth{1.003750pt}%
\definecolor{currentstroke}{rgb}{0.121569,0.466667,0.705882}%
\pgfsetstrokecolor{currentstroke}%
\pgfsetstrokeopacity{0.750451}%
\pgfsetdash{}{0pt}%
\pgfpathmoveto{\pgfqpoint{1.081876in}{1.232821in}}%
\pgfpathcurveto{\pgfqpoint{1.090112in}{1.232821in}}{\pgfqpoint{1.098012in}{1.236094in}}{\pgfqpoint{1.103836in}{1.241918in}}%
\pgfpathcurveto{\pgfqpoint{1.109660in}{1.247742in}}{\pgfqpoint{1.112932in}{1.255642in}}{\pgfqpoint{1.112932in}{1.263878in}}%
\pgfpathcurveto{\pgfqpoint{1.112932in}{1.272114in}}{\pgfqpoint{1.109660in}{1.280014in}}{\pgfqpoint{1.103836in}{1.285838in}}%
\pgfpathcurveto{\pgfqpoint{1.098012in}{1.291662in}}{\pgfqpoint{1.090112in}{1.294934in}}{\pgfqpoint{1.081876in}{1.294934in}}%
\pgfpathcurveto{\pgfqpoint{1.073639in}{1.294934in}}{\pgfqpoint{1.065739in}{1.291662in}}{\pgfqpoint{1.059915in}{1.285838in}}%
\pgfpathcurveto{\pgfqpoint{1.054092in}{1.280014in}}{\pgfqpoint{1.050819in}{1.272114in}}{\pgfqpoint{1.050819in}{1.263878in}}%
\pgfpathcurveto{\pgfqpoint{1.050819in}{1.255642in}}{\pgfqpoint{1.054092in}{1.247742in}}{\pgfqpoint{1.059915in}{1.241918in}}%
\pgfpathcurveto{\pgfqpoint{1.065739in}{1.236094in}}{\pgfqpoint{1.073639in}{1.232821in}}{\pgfqpoint{1.081876in}{1.232821in}}%
\pgfpathclose%
\pgfusepath{stroke,fill}%
\end{pgfscope}%
\begin{pgfscope}%
\pgfpathrectangle{\pgfqpoint{0.100000in}{0.220728in}}{\pgfqpoint{3.696000in}{3.696000in}}%
\pgfusepath{clip}%
\pgfsetbuttcap%
\pgfsetroundjoin%
\definecolor{currentfill}{rgb}{0.121569,0.466667,0.705882}%
\pgfsetfillcolor{currentfill}%
\pgfsetfillopacity{0.750647}%
\pgfsetlinewidth{1.003750pt}%
\definecolor{currentstroke}{rgb}{0.121569,0.466667,0.705882}%
\pgfsetstrokecolor{currentstroke}%
\pgfsetstrokeopacity{0.750647}%
\pgfsetdash{}{0pt}%
\pgfpathmoveto{\pgfqpoint{3.146583in}{2.471784in}}%
\pgfpathcurveto{\pgfqpoint{3.154819in}{2.471784in}}{\pgfqpoint{3.162719in}{2.475056in}}{\pgfqpoint{3.168543in}{2.480880in}}%
\pgfpathcurveto{\pgfqpoint{3.174367in}{2.486704in}}{\pgfqpoint{3.177639in}{2.494604in}}{\pgfqpoint{3.177639in}{2.502841in}}%
\pgfpathcurveto{\pgfqpoint{3.177639in}{2.511077in}}{\pgfqpoint{3.174367in}{2.518977in}}{\pgfqpoint{3.168543in}{2.524801in}}%
\pgfpathcurveto{\pgfqpoint{3.162719in}{2.530625in}}{\pgfqpoint{3.154819in}{2.533897in}}{\pgfqpoint{3.146583in}{2.533897in}}%
\pgfpathcurveto{\pgfqpoint{3.138347in}{2.533897in}}{\pgfqpoint{3.130446in}{2.530625in}}{\pgfqpoint{3.124623in}{2.524801in}}%
\pgfpathcurveto{\pgfqpoint{3.118799in}{2.518977in}}{\pgfqpoint{3.115526in}{2.511077in}}{\pgfqpoint{3.115526in}{2.502841in}}%
\pgfpathcurveto{\pgfqpoint{3.115526in}{2.494604in}}{\pgfqpoint{3.118799in}{2.486704in}}{\pgfqpoint{3.124623in}{2.480880in}}%
\pgfpathcurveto{\pgfqpoint{3.130446in}{2.475056in}}{\pgfqpoint{3.138347in}{2.471784in}}{\pgfqpoint{3.146583in}{2.471784in}}%
\pgfpathclose%
\pgfusepath{stroke,fill}%
\end{pgfscope}%
\begin{pgfscope}%
\pgfpathrectangle{\pgfqpoint{0.100000in}{0.220728in}}{\pgfqpoint{3.696000in}{3.696000in}}%
\pgfusepath{clip}%
\pgfsetbuttcap%
\pgfsetroundjoin%
\definecolor{currentfill}{rgb}{0.121569,0.466667,0.705882}%
\pgfsetfillcolor{currentfill}%
\pgfsetfillopacity{0.751106}%
\pgfsetlinewidth{1.003750pt}%
\definecolor{currentstroke}{rgb}{0.121569,0.466667,0.705882}%
\pgfsetstrokecolor{currentstroke}%
\pgfsetstrokeopacity{0.751106}%
\pgfsetdash{}{0pt}%
\pgfpathmoveto{\pgfqpoint{3.145396in}{2.469518in}}%
\pgfpathcurveto{\pgfqpoint{3.153633in}{2.469518in}}{\pgfqpoint{3.161533in}{2.472791in}}{\pgfqpoint{3.167357in}{2.478615in}}%
\pgfpathcurveto{\pgfqpoint{3.173181in}{2.484439in}}{\pgfqpoint{3.176453in}{2.492339in}}{\pgfqpoint{3.176453in}{2.500575in}}%
\pgfpathcurveto{\pgfqpoint{3.176453in}{2.508811in}}{\pgfqpoint{3.173181in}{2.516711in}}{\pgfqpoint{3.167357in}{2.522535in}}%
\pgfpathcurveto{\pgfqpoint{3.161533in}{2.528359in}}{\pgfqpoint{3.153633in}{2.531631in}}{\pgfqpoint{3.145396in}{2.531631in}}%
\pgfpathcurveto{\pgfqpoint{3.137160in}{2.531631in}}{\pgfqpoint{3.129260in}{2.528359in}}{\pgfqpoint{3.123436in}{2.522535in}}%
\pgfpathcurveto{\pgfqpoint{3.117612in}{2.516711in}}{\pgfqpoint{3.114340in}{2.508811in}}{\pgfqpoint{3.114340in}{2.500575in}}%
\pgfpathcurveto{\pgfqpoint{3.114340in}{2.492339in}}{\pgfqpoint{3.117612in}{2.484439in}}{\pgfqpoint{3.123436in}{2.478615in}}%
\pgfpathcurveto{\pgfqpoint{3.129260in}{2.472791in}}{\pgfqpoint{3.137160in}{2.469518in}}{\pgfqpoint{3.145396in}{2.469518in}}%
\pgfpathclose%
\pgfusepath{stroke,fill}%
\end{pgfscope}%
\begin{pgfscope}%
\pgfpathrectangle{\pgfqpoint{0.100000in}{0.220728in}}{\pgfqpoint{3.696000in}{3.696000in}}%
\pgfusepath{clip}%
\pgfsetbuttcap%
\pgfsetroundjoin%
\definecolor{currentfill}{rgb}{0.121569,0.466667,0.705882}%
\pgfsetfillcolor{currentfill}%
\pgfsetfillopacity{0.751343}%
\pgfsetlinewidth{1.003750pt}%
\definecolor{currentstroke}{rgb}{0.121569,0.466667,0.705882}%
\pgfsetstrokecolor{currentstroke}%
\pgfsetstrokeopacity{0.751343}%
\pgfsetdash{}{0pt}%
\pgfpathmoveto{\pgfqpoint{3.144633in}{2.468354in}}%
\pgfpathcurveto{\pgfqpoint{3.152870in}{2.468354in}}{\pgfqpoint{3.160770in}{2.471627in}}{\pgfqpoint{3.166594in}{2.477450in}}%
\pgfpathcurveto{\pgfqpoint{3.172418in}{2.483274in}}{\pgfqpoint{3.175690in}{2.491174in}}{\pgfqpoint{3.175690in}{2.499411in}}%
\pgfpathcurveto{\pgfqpoint{3.175690in}{2.507647in}}{\pgfqpoint{3.172418in}{2.515547in}}{\pgfqpoint{3.166594in}{2.521371in}}%
\pgfpathcurveto{\pgfqpoint{3.160770in}{2.527195in}}{\pgfqpoint{3.152870in}{2.530467in}}{\pgfqpoint{3.144633in}{2.530467in}}%
\pgfpathcurveto{\pgfqpoint{3.136397in}{2.530467in}}{\pgfqpoint{3.128497in}{2.527195in}}{\pgfqpoint{3.122673in}{2.521371in}}%
\pgfpathcurveto{\pgfqpoint{3.116849in}{2.515547in}}{\pgfqpoint{3.113577in}{2.507647in}}{\pgfqpoint{3.113577in}{2.499411in}}%
\pgfpathcurveto{\pgfqpoint{3.113577in}{2.491174in}}{\pgfqpoint{3.116849in}{2.483274in}}{\pgfqpoint{3.122673in}{2.477450in}}%
\pgfpathcurveto{\pgfqpoint{3.128497in}{2.471627in}}{\pgfqpoint{3.136397in}{2.468354in}}{\pgfqpoint{3.144633in}{2.468354in}}%
\pgfpathclose%
\pgfusepath{stroke,fill}%
\end{pgfscope}%
\begin{pgfscope}%
\pgfpathrectangle{\pgfqpoint{0.100000in}{0.220728in}}{\pgfqpoint{3.696000in}{3.696000in}}%
\pgfusepath{clip}%
\pgfsetbuttcap%
\pgfsetroundjoin%
\definecolor{currentfill}{rgb}{0.121569,0.466667,0.705882}%
\pgfsetfillcolor{currentfill}%
\pgfsetfillopacity{0.751476}%
\pgfsetlinewidth{1.003750pt}%
\definecolor{currentstroke}{rgb}{0.121569,0.466667,0.705882}%
\pgfsetstrokecolor{currentstroke}%
\pgfsetstrokeopacity{0.751476}%
\pgfsetdash{}{0pt}%
\pgfpathmoveto{\pgfqpoint{3.144312in}{2.467601in}}%
\pgfpathcurveto{\pgfqpoint{3.152549in}{2.467601in}}{\pgfqpoint{3.160449in}{2.470874in}}{\pgfqpoint{3.166273in}{2.476698in}}%
\pgfpathcurveto{\pgfqpoint{3.172097in}{2.482522in}}{\pgfqpoint{3.175369in}{2.490422in}}{\pgfqpoint{3.175369in}{2.498658in}}%
\pgfpathcurveto{\pgfqpoint{3.175369in}{2.506894in}}{\pgfqpoint{3.172097in}{2.514794in}}{\pgfqpoint{3.166273in}{2.520618in}}%
\pgfpathcurveto{\pgfqpoint{3.160449in}{2.526442in}}{\pgfqpoint{3.152549in}{2.529714in}}{\pgfqpoint{3.144312in}{2.529714in}}%
\pgfpathcurveto{\pgfqpoint{3.136076in}{2.529714in}}{\pgfqpoint{3.128176in}{2.526442in}}{\pgfqpoint{3.122352in}{2.520618in}}%
\pgfpathcurveto{\pgfqpoint{3.116528in}{2.514794in}}{\pgfqpoint{3.113256in}{2.506894in}}{\pgfqpoint{3.113256in}{2.498658in}}%
\pgfpathcurveto{\pgfqpoint{3.113256in}{2.490422in}}{\pgfqpoint{3.116528in}{2.482522in}}{\pgfqpoint{3.122352in}{2.476698in}}%
\pgfpathcurveto{\pgfqpoint{3.128176in}{2.470874in}}{\pgfqpoint{3.136076in}{2.467601in}}{\pgfqpoint{3.144312in}{2.467601in}}%
\pgfpathclose%
\pgfusepath{stroke,fill}%
\end{pgfscope}%
\begin{pgfscope}%
\pgfpathrectangle{\pgfqpoint{0.100000in}{0.220728in}}{\pgfqpoint{3.696000in}{3.696000in}}%
\pgfusepath{clip}%
\pgfsetbuttcap%
\pgfsetroundjoin%
\definecolor{currentfill}{rgb}{0.121569,0.466667,0.705882}%
\pgfsetfillcolor{currentfill}%
\pgfsetfillopacity{0.751721}%
\pgfsetlinewidth{1.003750pt}%
\definecolor{currentstroke}{rgb}{0.121569,0.466667,0.705882}%
\pgfsetstrokecolor{currentstroke}%
\pgfsetstrokeopacity{0.751721}%
\pgfsetdash{}{0pt}%
\pgfpathmoveto{\pgfqpoint{3.143301in}{2.466091in}}%
\pgfpathcurveto{\pgfqpoint{3.151537in}{2.466091in}}{\pgfqpoint{3.159437in}{2.469363in}}{\pgfqpoint{3.165261in}{2.475187in}}%
\pgfpathcurveto{\pgfqpoint{3.171085in}{2.481011in}}{\pgfqpoint{3.174358in}{2.488911in}}{\pgfqpoint{3.174358in}{2.497147in}}%
\pgfpathcurveto{\pgfqpoint{3.174358in}{2.505384in}}{\pgfqpoint{3.171085in}{2.513284in}}{\pgfqpoint{3.165261in}{2.519108in}}%
\pgfpathcurveto{\pgfqpoint{3.159437in}{2.524932in}}{\pgfqpoint{3.151537in}{2.528204in}}{\pgfqpoint{3.143301in}{2.528204in}}%
\pgfpathcurveto{\pgfqpoint{3.135065in}{2.528204in}}{\pgfqpoint{3.127165in}{2.524932in}}{\pgfqpoint{3.121341in}{2.519108in}}%
\pgfpathcurveto{\pgfqpoint{3.115517in}{2.513284in}}{\pgfqpoint{3.112245in}{2.505384in}}{\pgfqpoint{3.112245in}{2.497147in}}%
\pgfpathcurveto{\pgfqpoint{3.112245in}{2.488911in}}{\pgfqpoint{3.115517in}{2.481011in}}{\pgfqpoint{3.121341in}{2.475187in}}%
\pgfpathcurveto{\pgfqpoint{3.127165in}{2.469363in}}{\pgfqpoint{3.135065in}{2.466091in}}{\pgfqpoint{3.143301in}{2.466091in}}%
\pgfpathclose%
\pgfusepath{stroke,fill}%
\end{pgfscope}%
\begin{pgfscope}%
\pgfpathrectangle{\pgfqpoint{0.100000in}{0.220728in}}{\pgfqpoint{3.696000in}{3.696000in}}%
\pgfusepath{clip}%
\pgfsetbuttcap%
\pgfsetroundjoin%
\definecolor{currentfill}{rgb}{0.121569,0.466667,0.705882}%
\pgfsetfillcolor{currentfill}%
\pgfsetfillopacity{0.751886}%
\pgfsetlinewidth{1.003750pt}%
\definecolor{currentstroke}{rgb}{0.121569,0.466667,0.705882}%
\pgfsetstrokecolor{currentstroke}%
\pgfsetstrokeopacity{0.751886}%
\pgfsetdash{}{0pt}%
\pgfpathmoveto{\pgfqpoint{3.142870in}{2.465209in}}%
\pgfpathcurveto{\pgfqpoint{3.151106in}{2.465209in}}{\pgfqpoint{3.159006in}{2.468482in}}{\pgfqpoint{3.164830in}{2.474306in}}%
\pgfpathcurveto{\pgfqpoint{3.170654in}{2.480129in}}{\pgfqpoint{3.173926in}{2.488030in}}{\pgfqpoint{3.173926in}{2.496266in}}%
\pgfpathcurveto{\pgfqpoint{3.173926in}{2.504502in}}{\pgfqpoint{3.170654in}{2.512402in}}{\pgfqpoint{3.164830in}{2.518226in}}%
\pgfpathcurveto{\pgfqpoint{3.159006in}{2.524050in}}{\pgfqpoint{3.151106in}{2.527322in}}{\pgfqpoint{3.142870in}{2.527322in}}%
\pgfpathcurveto{\pgfqpoint{3.134633in}{2.527322in}}{\pgfqpoint{3.126733in}{2.524050in}}{\pgfqpoint{3.120909in}{2.518226in}}%
\pgfpathcurveto{\pgfqpoint{3.115085in}{2.512402in}}{\pgfqpoint{3.111813in}{2.504502in}}{\pgfqpoint{3.111813in}{2.496266in}}%
\pgfpathcurveto{\pgfqpoint{3.111813in}{2.488030in}}{\pgfqpoint{3.115085in}{2.480129in}}{\pgfqpoint{3.120909in}{2.474306in}}%
\pgfpathcurveto{\pgfqpoint{3.126733in}{2.468482in}}{\pgfqpoint{3.134633in}{2.465209in}}{\pgfqpoint{3.142870in}{2.465209in}}%
\pgfpathclose%
\pgfusepath{stroke,fill}%
\end{pgfscope}%
\begin{pgfscope}%
\pgfpathrectangle{\pgfqpoint{0.100000in}{0.220728in}}{\pgfqpoint{3.696000in}{3.696000in}}%
\pgfusepath{clip}%
\pgfsetbuttcap%
\pgfsetroundjoin%
\definecolor{currentfill}{rgb}{0.121569,0.466667,0.705882}%
\pgfsetfillcolor{currentfill}%
\pgfsetfillopacity{0.751977}%
\pgfsetlinewidth{1.003750pt}%
\definecolor{currentstroke}{rgb}{0.121569,0.466667,0.705882}%
\pgfsetstrokecolor{currentstroke}%
\pgfsetstrokeopacity{0.751977}%
\pgfsetdash{}{0pt}%
\pgfpathmoveto{\pgfqpoint{3.142605in}{2.464761in}}%
\pgfpathcurveto{\pgfqpoint{3.150842in}{2.464761in}}{\pgfqpoint{3.158742in}{2.468033in}}{\pgfqpoint{3.164566in}{2.473857in}}%
\pgfpathcurveto{\pgfqpoint{3.170389in}{2.479681in}}{\pgfqpoint{3.173662in}{2.487581in}}{\pgfqpoint{3.173662in}{2.495817in}}%
\pgfpathcurveto{\pgfqpoint{3.173662in}{2.504054in}}{\pgfqpoint{3.170389in}{2.511954in}}{\pgfqpoint{3.164566in}{2.517777in}}%
\pgfpathcurveto{\pgfqpoint{3.158742in}{2.523601in}}{\pgfqpoint{3.150842in}{2.526874in}}{\pgfqpoint{3.142605in}{2.526874in}}%
\pgfpathcurveto{\pgfqpoint{3.134369in}{2.526874in}}{\pgfqpoint{3.126469in}{2.523601in}}{\pgfqpoint{3.120645in}{2.517777in}}%
\pgfpathcurveto{\pgfqpoint{3.114821in}{2.511954in}}{\pgfqpoint{3.111549in}{2.504054in}}{\pgfqpoint{3.111549in}{2.495817in}}%
\pgfpathcurveto{\pgfqpoint{3.111549in}{2.487581in}}{\pgfqpoint{3.114821in}{2.479681in}}{\pgfqpoint{3.120645in}{2.473857in}}%
\pgfpathcurveto{\pgfqpoint{3.126469in}{2.468033in}}{\pgfqpoint{3.134369in}{2.464761in}}{\pgfqpoint{3.142605in}{2.464761in}}%
\pgfpathclose%
\pgfusepath{stroke,fill}%
\end{pgfscope}%
\begin{pgfscope}%
\pgfpathrectangle{\pgfqpoint{0.100000in}{0.220728in}}{\pgfqpoint{3.696000in}{3.696000in}}%
\pgfusepath{clip}%
\pgfsetbuttcap%
\pgfsetroundjoin%
\definecolor{currentfill}{rgb}{0.121569,0.466667,0.705882}%
\pgfsetfillcolor{currentfill}%
\pgfsetfillopacity{0.752030}%
\pgfsetlinewidth{1.003750pt}%
\definecolor{currentstroke}{rgb}{0.121569,0.466667,0.705882}%
\pgfsetstrokecolor{currentstroke}%
\pgfsetstrokeopacity{0.752030}%
\pgfsetdash{}{0pt}%
\pgfpathmoveto{\pgfqpoint{3.142445in}{2.464547in}}%
\pgfpathcurveto{\pgfqpoint{3.150682in}{2.464547in}}{\pgfqpoint{3.158582in}{2.467820in}}{\pgfqpoint{3.164406in}{2.473644in}}%
\pgfpathcurveto{\pgfqpoint{3.170229in}{2.479468in}}{\pgfqpoint{3.173502in}{2.487368in}}{\pgfqpoint{3.173502in}{2.495604in}}%
\pgfpathcurveto{\pgfqpoint{3.173502in}{2.503840in}}{\pgfqpoint{3.170229in}{2.511740in}}{\pgfqpoint{3.164406in}{2.517564in}}%
\pgfpathcurveto{\pgfqpoint{3.158582in}{2.523388in}}{\pgfqpoint{3.150682in}{2.526660in}}{\pgfqpoint{3.142445in}{2.526660in}}%
\pgfpathcurveto{\pgfqpoint{3.134209in}{2.526660in}}{\pgfqpoint{3.126309in}{2.523388in}}{\pgfqpoint{3.120485in}{2.517564in}}%
\pgfpathcurveto{\pgfqpoint{3.114661in}{2.511740in}}{\pgfqpoint{3.111389in}{2.503840in}}{\pgfqpoint{3.111389in}{2.495604in}}%
\pgfpathcurveto{\pgfqpoint{3.111389in}{2.487368in}}{\pgfqpoint{3.114661in}{2.479468in}}{\pgfqpoint{3.120485in}{2.473644in}}%
\pgfpathcurveto{\pgfqpoint{3.126309in}{2.467820in}}{\pgfqpoint{3.134209in}{2.464547in}}{\pgfqpoint{3.142445in}{2.464547in}}%
\pgfpathclose%
\pgfusepath{stroke,fill}%
\end{pgfscope}%
\begin{pgfscope}%
\pgfpathrectangle{\pgfqpoint{0.100000in}{0.220728in}}{\pgfqpoint{3.696000in}{3.696000in}}%
\pgfusepath{clip}%
\pgfsetbuttcap%
\pgfsetroundjoin%
\definecolor{currentfill}{rgb}{0.121569,0.466667,0.705882}%
\pgfsetfillcolor{currentfill}%
\pgfsetfillopacity{0.752060}%
\pgfsetlinewidth{1.003750pt}%
\definecolor{currentstroke}{rgb}{0.121569,0.466667,0.705882}%
\pgfsetstrokecolor{currentstroke}%
\pgfsetstrokeopacity{0.752060}%
\pgfsetdash{}{0pt}%
\pgfpathmoveto{\pgfqpoint{3.142385in}{2.464399in}}%
\pgfpathcurveto{\pgfqpoint{3.150621in}{2.464399in}}{\pgfqpoint{3.158521in}{2.467671in}}{\pgfqpoint{3.164345in}{2.473495in}}%
\pgfpathcurveto{\pgfqpoint{3.170169in}{2.479319in}}{\pgfqpoint{3.173441in}{2.487219in}}{\pgfqpoint{3.173441in}{2.495456in}}%
\pgfpathcurveto{\pgfqpoint{3.173441in}{2.503692in}}{\pgfqpoint{3.170169in}{2.511592in}}{\pgfqpoint{3.164345in}{2.517416in}}%
\pgfpathcurveto{\pgfqpoint{3.158521in}{2.523240in}}{\pgfqpoint{3.150621in}{2.526512in}}{\pgfqpoint{3.142385in}{2.526512in}}%
\pgfpathcurveto{\pgfqpoint{3.134149in}{2.526512in}}{\pgfqpoint{3.126249in}{2.523240in}}{\pgfqpoint{3.120425in}{2.517416in}}%
\pgfpathcurveto{\pgfqpoint{3.114601in}{2.511592in}}{\pgfqpoint{3.111328in}{2.503692in}}{\pgfqpoint{3.111328in}{2.495456in}}%
\pgfpathcurveto{\pgfqpoint{3.111328in}{2.487219in}}{\pgfqpoint{3.114601in}{2.479319in}}{\pgfqpoint{3.120425in}{2.473495in}}%
\pgfpathcurveto{\pgfqpoint{3.126249in}{2.467671in}}{\pgfqpoint{3.134149in}{2.464399in}}{\pgfqpoint{3.142385in}{2.464399in}}%
\pgfpathclose%
\pgfusepath{stroke,fill}%
\end{pgfscope}%
\begin{pgfscope}%
\pgfpathrectangle{\pgfqpoint{0.100000in}{0.220728in}}{\pgfqpoint{3.696000in}{3.696000in}}%
\pgfusepath{clip}%
\pgfsetbuttcap%
\pgfsetroundjoin%
\definecolor{currentfill}{rgb}{0.121569,0.466667,0.705882}%
\pgfsetfillcolor{currentfill}%
\pgfsetfillopacity{0.752073}%
\pgfsetlinewidth{1.003750pt}%
\definecolor{currentstroke}{rgb}{0.121569,0.466667,0.705882}%
\pgfsetstrokecolor{currentstroke}%
\pgfsetstrokeopacity{0.752073}%
\pgfsetdash{}{0pt}%
\pgfpathmoveto{\pgfqpoint{3.142335in}{2.464324in}}%
\pgfpathcurveto{\pgfqpoint{3.150571in}{2.464324in}}{\pgfqpoint{3.158471in}{2.467597in}}{\pgfqpoint{3.164295in}{2.473421in}}%
\pgfpathcurveto{\pgfqpoint{3.170119in}{2.479245in}}{\pgfqpoint{3.173391in}{2.487145in}}{\pgfqpoint{3.173391in}{2.495381in}}%
\pgfpathcurveto{\pgfqpoint{3.173391in}{2.503617in}}{\pgfqpoint{3.170119in}{2.511517in}}{\pgfqpoint{3.164295in}{2.517341in}}%
\pgfpathcurveto{\pgfqpoint{3.158471in}{2.523165in}}{\pgfqpoint{3.150571in}{2.526437in}}{\pgfqpoint{3.142335in}{2.526437in}}%
\pgfpathcurveto{\pgfqpoint{3.134099in}{2.526437in}}{\pgfqpoint{3.126199in}{2.523165in}}{\pgfqpoint{3.120375in}{2.517341in}}%
\pgfpathcurveto{\pgfqpoint{3.114551in}{2.511517in}}{\pgfqpoint{3.111278in}{2.503617in}}{\pgfqpoint{3.111278in}{2.495381in}}%
\pgfpathcurveto{\pgfqpoint{3.111278in}{2.487145in}}{\pgfqpoint{3.114551in}{2.479245in}}{\pgfqpoint{3.120375in}{2.473421in}}%
\pgfpathcurveto{\pgfqpoint{3.126199in}{2.467597in}}{\pgfqpoint{3.134099in}{2.464324in}}{\pgfqpoint{3.142335in}{2.464324in}}%
\pgfpathclose%
\pgfusepath{stroke,fill}%
\end{pgfscope}%
\begin{pgfscope}%
\pgfpathrectangle{\pgfqpoint{0.100000in}{0.220728in}}{\pgfqpoint{3.696000in}{3.696000in}}%
\pgfusepath{clip}%
\pgfsetbuttcap%
\pgfsetroundjoin%
\definecolor{currentfill}{rgb}{0.121569,0.466667,0.705882}%
\pgfsetfillcolor{currentfill}%
\pgfsetfillopacity{0.752081}%
\pgfsetlinewidth{1.003750pt}%
\definecolor{currentstroke}{rgb}{0.121569,0.466667,0.705882}%
\pgfsetstrokecolor{currentstroke}%
\pgfsetstrokeopacity{0.752081}%
\pgfsetdash{}{0pt}%
\pgfpathmoveto{\pgfqpoint{3.142311in}{2.464285in}}%
\pgfpathcurveto{\pgfqpoint{3.150547in}{2.464285in}}{\pgfqpoint{3.158447in}{2.467557in}}{\pgfqpoint{3.164271in}{2.473381in}}%
\pgfpathcurveto{\pgfqpoint{3.170095in}{2.479205in}}{\pgfqpoint{3.173367in}{2.487105in}}{\pgfqpoint{3.173367in}{2.495341in}}%
\pgfpathcurveto{\pgfqpoint{3.173367in}{2.503578in}}{\pgfqpoint{3.170095in}{2.511478in}}{\pgfqpoint{3.164271in}{2.517302in}}%
\pgfpathcurveto{\pgfqpoint{3.158447in}{2.523126in}}{\pgfqpoint{3.150547in}{2.526398in}}{\pgfqpoint{3.142311in}{2.526398in}}%
\pgfpathcurveto{\pgfqpoint{3.134074in}{2.526398in}}{\pgfqpoint{3.126174in}{2.523126in}}{\pgfqpoint{3.120350in}{2.517302in}}%
\pgfpathcurveto{\pgfqpoint{3.114526in}{2.511478in}}{\pgfqpoint{3.111254in}{2.503578in}}{\pgfqpoint{3.111254in}{2.495341in}}%
\pgfpathcurveto{\pgfqpoint{3.111254in}{2.487105in}}{\pgfqpoint{3.114526in}{2.479205in}}{\pgfqpoint{3.120350in}{2.473381in}}%
\pgfpathcurveto{\pgfqpoint{3.126174in}{2.467557in}}{\pgfqpoint{3.134074in}{2.464285in}}{\pgfqpoint{3.142311in}{2.464285in}}%
\pgfpathclose%
\pgfusepath{stroke,fill}%
\end{pgfscope}%
\begin{pgfscope}%
\pgfpathrectangle{\pgfqpoint{0.100000in}{0.220728in}}{\pgfqpoint{3.696000in}{3.696000in}}%
\pgfusepath{clip}%
\pgfsetbuttcap%
\pgfsetroundjoin%
\definecolor{currentfill}{rgb}{0.121569,0.466667,0.705882}%
\pgfsetfillcolor{currentfill}%
\pgfsetfillopacity{0.752086}%
\pgfsetlinewidth{1.003750pt}%
\definecolor{currentstroke}{rgb}{0.121569,0.466667,0.705882}%
\pgfsetstrokecolor{currentstroke}%
\pgfsetstrokeopacity{0.752086}%
\pgfsetdash{}{0pt}%
\pgfpathmoveto{\pgfqpoint{3.142298in}{2.464262in}}%
\pgfpathcurveto{\pgfqpoint{3.150534in}{2.464262in}}{\pgfqpoint{3.158434in}{2.467534in}}{\pgfqpoint{3.164258in}{2.473358in}}%
\pgfpathcurveto{\pgfqpoint{3.170082in}{2.479182in}}{\pgfqpoint{3.173354in}{2.487082in}}{\pgfqpoint{3.173354in}{2.495318in}}%
\pgfpathcurveto{\pgfqpoint{3.173354in}{2.503555in}}{\pgfqpoint{3.170082in}{2.511455in}}{\pgfqpoint{3.164258in}{2.517279in}}%
\pgfpathcurveto{\pgfqpoint{3.158434in}{2.523103in}}{\pgfqpoint{3.150534in}{2.526375in}}{\pgfqpoint{3.142298in}{2.526375in}}%
\pgfpathcurveto{\pgfqpoint{3.134061in}{2.526375in}}{\pgfqpoint{3.126161in}{2.523103in}}{\pgfqpoint{3.120337in}{2.517279in}}%
\pgfpathcurveto{\pgfqpoint{3.114513in}{2.511455in}}{\pgfqpoint{3.111241in}{2.503555in}}{\pgfqpoint{3.111241in}{2.495318in}}%
\pgfpathcurveto{\pgfqpoint{3.111241in}{2.487082in}}{\pgfqpoint{3.114513in}{2.479182in}}{\pgfqpoint{3.120337in}{2.473358in}}%
\pgfpathcurveto{\pgfqpoint{3.126161in}{2.467534in}}{\pgfqpoint{3.134061in}{2.464262in}}{\pgfqpoint{3.142298in}{2.464262in}}%
\pgfpathclose%
\pgfusepath{stroke,fill}%
\end{pgfscope}%
\begin{pgfscope}%
\pgfpathrectangle{\pgfqpoint{0.100000in}{0.220728in}}{\pgfqpoint{3.696000in}{3.696000in}}%
\pgfusepath{clip}%
\pgfsetbuttcap%
\pgfsetroundjoin%
\definecolor{currentfill}{rgb}{0.121569,0.466667,0.705882}%
\pgfsetfillcolor{currentfill}%
\pgfsetfillopacity{0.752088}%
\pgfsetlinewidth{1.003750pt}%
\definecolor{currentstroke}{rgb}{0.121569,0.466667,0.705882}%
\pgfsetstrokecolor{currentstroke}%
\pgfsetstrokeopacity{0.752088}%
\pgfsetdash{}{0pt}%
\pgfpathmoveto{\pgfqpoint{3.142290in}{2.464251in}}%
\pgfpathcurveto{\pgfqpoint{3.150526in}{2.464251in}}{\pgfqpoint{3.158426in}{2.467523in}}{\pgfqpoint{3.164250in}{2.473347in}}%
\pgfpathcurveto{\pgfqpoint{3.170074in}{2.479171in}}{\pgfqpoint{3.173346in}{2.487071in}}{\pgfqpoint{3.173346in}{2.495308in}}%
\pgfpathcurveto{\pgfqpoint{3.173346in}{2.503544in}}{\pgfqpoint{3.170074in}{2.511444in}}{\pgfqpoint{3.164250in}{2.517268in}}%
\pgfpathcurveto{\pgfqpoint{3.158426in}{2.523092in}}{\pgfqpoint{3.150526in}{2.526364in}}{\pgfqpoint{3.142290in}{2.526364in}}%
\pgfpathcurveto{\pgfqpoint{3.134053in}{2.526364in}}{\pgfqpoint{3.126153in}{2.523092in}}{\pgfqpoint{3.120329in}{2.517268in}}%
\pgfpathcurveto{\pgfqpoint{3.114505in}{2.511444in}}{\pgfqpoint{3.111233in}{2.503544in}}{\pgfqpoint{3.111233in}{2.495308in}}%
\pgfpathcurveto{\pgfqpoint{3.111233in}{2.487071in}}{\pgfqpoint{3.114505in}{2.479171in}}{\pgfqpoint{3.120329in}{2.473347in}}%
\pgfpathcurveto{\pgfqpoint{3.126153in}{2.467523in}}{\pgfqpoint{3.134053in}{2.464251in}}{\pgfqpoint{3.142290in}{2.464251in}}%
\pgfpathclose%
\pgfusepath{stroke,fill}%
\end{pgfscope}%
\begin{pgfscope}%
\pgfpathrectangle{\pgfqpoint{0.100000in}{0.220728in}}{\pgfqpoint{3.696000in}{3.696000in}}%
\pgfusepath{clip}%
\pgfsetbuttcap%
\pgfsetroundjoin%
\definecolor{currentfill}{rgb}{0.121569,0.466667,0.705882}%
\pgfsetfillcolor{currentfill}%
\pgfsetfillopacity{0.752311}%
\pgfsetlinewidth{1.003750pt}%
\definecolor{currentstroke}{rgb}{0.121569,0.466667,0.705882}%
\pgfsetstrokecolor{currentstroke}%
\pgfsetstrokeopacity{0.752311}%
\pgfsetdash{}{0pt}%
\pgfpathmoveto{\pgfqpoint{3.141739in}{2.462977in}}%
\pgfpathcurveto{\pgfqpoint{3.149975in}{2.462977in}}{\pgfqpoint{3.157875in}{2.466249in}}{\pgfqpoint{3.163699in}{2.472073in}}%
\pgfpathcurveto{\pgfqpoint{3.169523in}{2.477897in}}{\pgfqpoint{3.172795in}{2.485797in}}{\pgfqpoint{3.172795in}{2.494033in}}%
\pgfpathcurveto{\pgfqpoint{3.172795in}{2.502270in}}{\pgfqpoint{3.169523in}{2.510170in}}{\pgfqpoint{3.163699in}{2.515994in}}%
\pgfpathcurveto{\pgfqpoint{3.157875in}{2.521818in}}{\pgfqpoint{3.149975in}{2.525090in}}{\pgfqpoint{3.141739in}{2.525090in}}%
\pgfpathcurveto{\pgfqpoint{3.133502in}{2.525090in}}{\pgfqpoint{3.125602in}{2.521818in}}{\pgfqpoint{3.119778in}{2.515994in}}%
\pgfpathcurveto{\pgfqpoint{3.113954in}{2.510170in}}{\pgfqpoint{3.110682in}{2.502270in}}{\pgfqpoint{3.110682in}{2.494033in}}%
\pgfpathcurveto{\pgfqpoint{3.110682in}{2.485797in}}{\pgfqpoint{3.113954in}{2.477897in}}{\pgfqpoint{3.119778in}{2.472073in}}%
\pgfpathcurveto{\pgfqpoint{3.125602in}{2.466249in}}{\pgfqpoint{3.133502in}{2.462977in}}{\pgfqpoint{3.141739in}{2.462977in}}%
\pgfpathclose%
\pgfusepath{stroke,fill}%
\end{pgfscope}%
\begin{pgfscope}%
\pgfpathrectangle{\pgfqpoint{0.100000in}{0.220728in}}{\pgfqpoint{3.696000in}{3.696000in}}%
\pgfusepath{clip}%
\pgfsetbuttcap%
\pgfsetroundjoin%
\definecolor{currentfill}{rgb}{0.121569,0.466667,0.705882}%
\pgfsetfillcolor{currentfill}%
\pgfsetfillopacity{0.752420}%
\pgfsetlinewidth{1.003750pt}%
\definecolor{currentstroke}{rgb}{0.121569,0.466667,0.705882}%
\pgfsetstrokecolor{currentstroke}%
\pgfsetstrokeopacity{0.752420}%
\pgfsetdash{}{0pt}%
\pgfpathmoveto{\pgfqpoint{3.141359in}{2.462319in}}%
\pgfpathcurveto{\pgfqpoint{3.149595in}{2.462319in}}{\pgfqpoint{3.157495in}{2.465592in}}{\pgfqpoint{3.163319in}{2.471416in}}%
\pgfpathcurveto{\pgfqpoint{3.169143in}{2.477240in}}{\pgfqpoint{3.172415in}{2.485140in}}{\pgfqpoint{3.172415in}{2.493376in}}%
\pgfpathcurveto{\pgfqpoint{3.172415in}{2.501612in}}{\pgfqpoint{3.169143in}{2.509512in}}{\pgfqpoint{3.163319in}{2.515336in}}%
\pgfpathcurveto{\pgfqpoint{3.157495in}{2.521160in}}{\pgfqpoint{3.149595in}{2.524432in}}{\pgfqpoint{3.141359in}{2.524432in}}%
\pgfpathcurveto{\pgfqpoint{3.133122in}{2.524432in}}{\pgfqpoint{3.125222in}{2.521160in}}{\pgfqpoint{3.119398in}{2.515336in}}%
\pgfpathcurveto{\pgfqpoint{3.113575in}{2.509512in}}{\pgfqpoint{3.110302in}{2.501612in}}{\pgfqpoint{3.110302in}{2.493376in}}%
\pgfpathcurveto{\pgfqpoint{3.110302in}{2.485140in}}{\pgfqpoint{3.113575in}{2.477240in}}{\pgfqpoint{3.119398in}{2.471416in}}%
\pgfpathcurveto{\pgfqpoint{3.125222in}{2.465592in}}{\pgfqpoint{3.133122in}{2.462319in}}{\pgfqpoint{3.141359in}{2.462319in}}%
\pgfpathclose%
\pgfusepath{stroke,fill}%
\end{pgfscope}%
\begin{pgfscope}%
\pgfpathrectangle{\pgfqpoint{0.100000in}{0.220728in}}{\pgfqpoint{3.696000in}{3.696000in}}%
\pgfusepath{clip}%
\pgfsetbuttcap%
\pgfsetroundjoin%
\definecolor{currentfill}{rgb}{0.121569,0.466667,0.705882}%
\pgfsetfillcolor{currentfill}%
\pgfsetfillopacity{0.752486}%
\pgfsetlinewidth{1.003750pt}%
\definecolor{currentstroke}{rgb}{0.121569,0.466667,0.705882}%
\pgfsetstrokecolor{currentstroke}%
\pgfsetstrokeopacity{0.752486}%
\pgfsetdash{}{0pt}%
\pgfpathmoveto{\pgfqpoint{3.141141in}{2.461992in}}%
\pgfpathcurveto{\pgfqpoint{3.149378in}{2.461992in}}{\pgfqpoint{3.157278in}{2.465265in}}{\pgfqpoint{3.163102in}{2.471089in}}%
\pgfpathcurveto{\pgfqpoint{3.168926in}{2.476913in}}{\pgfqpoint{3.172198in}{2.484813in}}{\pgfqpoint{3.172198in}{2.493049in}}%
\pgfpathcurveto{\pgfqpoint{3.172198in}{2.501285in}}{\pgfqpoint{3.168926in}{2.509185in}}{\pgfqpoint{3.163102in}{2.515009in}}%
\pgfpathcurveto{\pgfqpoint{3.157278in}{2.520833in}}{\pgfqpoint{3.149378in}{2.524105in}}{\pgfqpoint{3.141141in}{2.524105in}}%
\pgfpathcurveto{\pgfqpoint{3.132905in}{2.524105in}}{\pgfqpoint{3.125005in}{2.520833in}}{\pgfqpoint{3.119181in}{2.515009in}}%
\pgfpathcurveto{\pgfqpoint{3.113357in}{2.509185in}}{\pgfqpoint{3.110085in}{2.501285in}}{\pgfqpoint{3.110085in}{2.493049in}}%
\pgfpathcurveto{\pgfqpoint{3.110085in}{2.484813in}}{\pgfqpoint{3.113357in}{2.476913in}}{\pgfqpoint{3.119181in}{2.471089in}}%
\pgfpathcurveto{\pgfqpoint{3.125005in}{2.465265in}}{\pgfqpoint{3.132905in}{2.461992in}}{\pgfqpoint{3.141141in}{2.461992in}}%
\pgfpathclose%
\pgfusepath{stroke,fill}%
\end{pgfscope}%
\begin{pgfscope}%
\pgfpathrectangle{\pgfqpoint{0.100000in}{0.220728in}}{\pgfqpoint{3.696000in}{3.696000in}}%
\pgfusepath{clip}%
\pgfsetbuttcap%
\pgfsetroundjoin%
\definecolor{currentfill}{rgb}{0.121569,0.466667,0.705882}%
\pgfsetfillcolor{currentfill}%
\pgfsetfillopacity{0.752522}%
\pgfsetlinewidth{1.003750pt}%
\definecolor{currentstroke}{rgb}{0.121569,0.466667,0.705882}%
\pgfsetstrokecolor{currentstroke}%
\pgfsetstrokeopacity{0.752522}%
\pgfsetdash{}{0pt}%
\pgfpathmoveto{\pgfqpoint{3.141056in}{2.461770in}}%
\pgfpathcurveto{\pgfqpoint{3.149292in}{2.461770in}}{\pgfqpoint{3.157192in}{2.465042in}}{\pgfqpoint{3.163016in}{2.470866in}}%
\pgfpathcurveto{\pgfqpoint{3.168840in}{2.476690in}}{\pgfqpoint{3.172112in}{2.484590in}}{\pgfqpoint{3.172112in}{2.492827in}}%
\pgfpathcurveto{\pgfqpoint{3.172112in}{2.501063in}}{\pgfqpoint{3.168840in}{2.508963in}}{\pgfqpoint{3.163016in}{2.514787in}}%
\pgfpathcurveto{\pgfqpoint{3.157192in}{2.520611in}}{\pgfqpoint{3.149292in}{2.523883in}}{\pgfqpoint{3.141056in}{2.523883in}}%
\pgfpathcurveto{\pgfqpoint{3.132820in}{2.523883in}}{\pgfqpoint{3.124920in}{2.520611in}}{\pgfqpoint{3.119096in}{2.514787in}}%
\pgfpathcurveto{\pgfqpoint{3.113272in}{2.508963in}}{\pgfqpoint{3.109999in}{2.501063in}}{\pgfqpoint{3.109999in}{2.492827in}}%
\pgfpathcurveto{\pgfqpoint{3.109999in}{2.484590in}}{\pgfqpoint{3.113272in}{2.476690in}}{\pgfqpoint{3.119096in}{2.470866in}}%
\pgfpathcurveto{\pgfqpoint{3.124920in}{2.465042in}}{\pgfqpoint{3.132820in}{2.461770in}}{\pgfqpoint{3.141056in}{2.461770in}}%
\pgfpathclose%
\pgfusepath{stroke,fill}%
\end{pgfscope}%
\begin{pgfscope}%
\pgfpathrectangle{\pgfqpoint{0.100000in}{0.220728in}}{\pgfqpoint{3.696000in}{3.696000in}}%
\pgfusepath{clip}%
\pgfsetbuttcap%
\pgfsetroundjoin%
\definecolor{currentfill}{rgb}{0.121569,0.466667,0.705882}%
\pgfsetfillcolor{currentfill}%
\pgfsetfillopacity{0.752800}%
\pgfsetlinewidth{1.003750pt}%
\definecolor{currentstroke}{rgb}{0.121569,0.466667,0.705882}%
\pgfsetstrokecolor{currentstroke}%
\pgfsetstrokeopacity{0.752800}%
\pgfsetdash{}{0pt}%
\pgfpathmoveto{\pgfqpoint{3.140081in}{2.460220in}}%
\pgfpathcurveto{\pgfqpoint{3.148317in}{2.460220in}}{\pgfqpoint{3.156218in}{2.463493in}}{\pgfqpoint{3.162041in}{2.469317in}}%
\pgfpathcurveto{\pgfqpoint{3.167865in}{2.475141in}}{\pgfqpoint{3.171138in}{2.483041in}}{\pgfqpoint{3.171138in}{2.491277in}}%
\pgfpathcurveto{\pgfqpoint{3.171138in}{2.499513in}}{\pgfqpoint{3.167865in}{2.507413in}}{\pgfqpoint{3.162041in}{2.513237in}}%
\pgfpathcurveto{\pgfqpoint{3.156218in}{2.519061in}}{\pgfqpoint{3.148317in}{2.522333in}}{\pgfqpoint{3.140081in}{2.522333in}}%
\pgfpathcurveto{\pgfqpoint{3.131845in}{2.522333in}}{\pgfqpoint{3.123945in}{2.519061in}}{\pgfqpoint{3.118121in}{2.513237in}}%
\pgfpathcurveto{\pgfqpoint{3.112297in}{2.507413in}}{\pgfqpoint{3.109025in}{2.499513in}}{\pgfqpoint{3.109025in}{2.491277in}}%
\pgfpathcurveto{\pgfqpoint{3.109025in}{2.483041in}}{\pgfqpoint{3.112297in}{2.475141in}}{\pgfqpoint{3.118121in}{2.469317in}}%
\pgfpathcurveto{\pgfqpoint{3.123945in}{2.463493in}}{\pgfqpoint{3.131845in}{2.460220in}}{\pgfqpoint{3.140081in}{2.460220in}}%
\pgfpathclose%
\pgfusepath{stroke,fill}%
\end{pgfscope}%
\begin{pgfscope}%
\pgfpathrectangle{\pgfqpoint{0.100000in}{0.220728in}}{\pgfqpoint{3.696000in}{3.696000in}}%
\pgfusepath{clip}%
\pgfsetbuttcap%
\pgfsetroundjoin%
\definecolor{currentfill}{rgb}{0.121569,0.466667,0.705882}%
\pgfsetfillcolor{currentfill}%
\pgfsetfillopacity{0.752984}%
\pgfsetlinewidth{1.003750pt}%
\definecolor{currentstroke}{rgb}{0.121569,0.466667,0.705882}%
\pgfsetstrokecolor{currentstroke}%
\pgfsetstrokeopacity{0.752984}%
\pgfsetdash{}{0pt}%
\pgfpathmoveto{\pgfqpoint{3.139626in}{2.459390in}}%
\pgfpathcurveto{\pgfqpoint{3.147862in}{2.459390in}}{\pgfqpoint{3.155762in}{2.462663in}}{\pgfqpoint{3.161586in}{2.468487in}}%
\pgfpathcurveto{\pgfqpoint{3.167410in}{2.474311in}}{\pgfqpoint{3.170683in}{2.482211in}}{\pgfqpoint{3.170683in}{2.490447in}}%
\pgfpathcurveto{\pgfqpoint{3.170683in}{2.498683in}}{\pgfqpoint{3.167410in}{2.506583in}}{\pgfqpoint{3.161586in}{2.512407in}}%
\pgfpathcurveto{\pgfqpoint{3.155762in}{2.518231in}}{\pgfqpoint{3.147862in}{2.521503in}}{\pgfqpoint{3.139626in}{2.521503in}}%
\pgfpathcurveto{\pgfqpoint{3.131390in}{2.521503in}}{\pgfqpoint{3.123490in}{2.518231in}}{\pgfqpoint{3.117666in}{2.512407in}}%
\pgfpathcurveto{\pgfqpoint{3.111842in}{2.506583in}}{\pgfqpoint{3.108570in}{2.498683in}}{\pgfqpoint{3.108570in}{2.490447in}}%
\pgfpathcurveto{\pgfqpoint{3.108570in}{2.482211in}}{\pgfqpoint{3.111842in}{2.474311in}}{\pgfqpoint{3.117666in}{2.468487in}}%
\pgfpathcurveto{\pgfqpoint{3.123490in}{2.462663in}}{\pgfqpoint{3.131390in}{2.459390in}}{\pgfqpoint{3.139626in}{2.459390in}}%
\pgfpathclose%
\pgfusepath{stroke,fill}%
\end{pgfscope}%
\begin{pgfscope}%
\pgfpathrectangle{\pgfqpoint{0.100000in}{0.220728in}}{\pgfqpoint{3.696000in}{3.696000in}}%
\pgfusepath{clip}%
\pgfsetbuttcap%
\pgfsetroundjoin%
\definecolor{currentfill}{rgb}{0.121569,0.466667,0.705882}%
\pgfsetfillcolor{currentfill}%
\pgfsetfillopacity{0.753080}%
\pgfsetlinewidth{1.003750pt}%
\definecolor{currentstroke}{rgb}{0.121569,0.466667,0.705882}%
\pgfsetstrokecolor{currentstroke}%
\pgfsetstrokeopacity{0.753080}%
\pgfsetdash{}{0pt}%
\pgfpathmoveto{\pgfqpoint{3.139408in}{2.458875in}}%
\pgfpathcurveto{\pgfqpoint{3.147644in}{2.458875in}}{\pgfqpoint{3.155544in}{2.462147in}}{\pgfqpoint{3.161368in}{2.467971in}}%
\pgfpathcurveto{\pgfqpoint{3.167192in}{2.473795in}}{\pgfqpoint{3.170464in}{2.481695in}}{\pgfqpoint{3.170464in}{2.489931in}}%
\pgfpathcurveto{\pgfqpoint{3.170464in}{2.498167in}}{\pgfqpoint{3.167192in}{2.506067in}}{\pgfqpoint{3.161368in}{2.511891in}}%
\pgfpathcurveto{\pgfqpoint{3.155544in}{2.517715in}}{\pgfqpoint{3.147644in}{2.520988in}}{\pgfqpoint{3.139408in}{2.520988in}}%
\pgfpathcurveto{\pgfqpoint{3.131172in}{2.520988in}}{\pgfqpoint{3.123272in}{2.517715in}}{\pgfqpoint{3.117448in}{2.511891in}}%
\pgfpathcurveto{\pgfqpoint{3.111624in}{2.506067in}}{\pgfqpoint{3.108351in}{2.498167in}}{\pgfqpoint{3.108351in}{2.489931in}}%
\pgfpathcurveto{\pgfqpoint{3.108351in}{2.481695in}}{\pgfqpoint{3.111624in}{2.473795in}}{\pgfqpoint{3.117448in}{2.467971in}}%
\pgfpathcurveto{\pgfqpoint{3.123272in}{2.462147in}}{\pgfqpoint{3.131172in}{2.458875in}}{\pgfqpoint{3.139408in}{2.458875in}}%
\pgfpathclose%
\pgfusepath{stroke,fill}%
\end{pgfscope}%
\begin{pgfscope}%
\pgfpathrectangle{\pgfqpoint{0.100000in}{0.220728in}}{\pgfqpoint{3.696000in}{3.696000in}}%
\pgfusepath{clip}%
\pgfsetbuttcap%
\pgfsetroundjoin%
\definecolor{currentfill}{rgb}{0.121569,0.466667,0.705882}%
\pgfsetfillcolor{currentfill}%
\pgfsetfillopacity{0.753161}%
\pgfsetlinewidth{1.003750pt}%
\definecolor{currentstroke}{rgb}{0.121569,0.466667,0.705882}%
\pgfsetstrokecolor{currentstroke}%
\pgfsetstrokeopacity{0.753161}%
\pgfsetdash{}{0pt}%
\pgfpathmoveto{\pgfqpoint{1.100470in}{1.221431in}}%
\pgfpathcurveto{\pgfqpoint{1.108706in}{1.221431in}}{\pgfqpoint{1.116606in}{1.224703in}}{\pgfqpoint{1.122430in}{1.230527in}}%
\pgfpathcurveto{\pgfqpoint{1.128254in}{1.236351in}}{\pgfqpoint{1.131526in}{1.244251in}}{\pgfqpoint{1.131526in}{1.252487in}}%
\pgfpathcurveto{\pgfqpoint{1.131526in}{1.260723in}}{\pgfqpoint{1.128254in}{1.268623in}}{\pgfqpoint{1.122430in}{1.274447in}}%
\pgfpathcurveto{\pgfqpoint{1.116606in}{1.280271in}}{\pgfqpoint{1.108706in}{1.283544in}}{\pgfqpoint{1.100470in}{1.283544in}}%
\pgfpathcurveto{\pgfqpoint{1.092234in}{1.283544in}}{\pgfqpoint{1.084334in}{1.280271in}}{\pgfqpoint{1.078510in}{1.274447in}}%
\pgfpathcurveto{\pgfqpoint{1.072686in}{1.268623in}}{\pgfqpoint{1.069413in}{1.260723in}}{\pgfqpoint{1.069413in}{1.252487in}}%
\pgfpathcurveto{\pgfqpoint{1.069413in}{1.244251in}}{\pgfqpoint{1.072686in}{1.236351in}}{\pgfqpoint{1.078510in}{1.230527in}}%
\pgfpathcurveto{\pgfqpoint{1.084334in}{1.224703in}}{\pgfqpoint{1.092234in}{1.221431in}}{\pgfqpoint{1.100470in}{1.221431in}}%
\pgfpathclose%
\pgfusepath{stroke,fill}%
\end{pgfscope}%
\begin{pgfscope}%
\pgfpathrectangle{\pgfqpoint{0.100000in}{0.220728in}}{\pgfqpoint{3.696000in}{3.696000in}}%
\pgfusepath{clip}%
\pgfsetbuttcap%
\pgfsetroundjoin%
\definecolor{currentfill}{rgb}{0.121569,0.466667,0.705882}%
\pgfsetfillcolor{currentfill}%
\pgfsetfillopacity{0.753296}%
\pgfsetlinewidth{1.003750pt}%
\definecolor{currentstroke}{rgb}{0.121569,0.466667,0.705882}%
\pgfsetstrokecolor{currentstroke}%
\pgfsetstrokeopacity{0.753296}%
\pgfsetdash{}{0pt}%
\pgfpathmoveto{\pgfqpoint{3.138497in}{2.457612in}}%
\pgfpathcurveto{\pgfqpoint{3.146733in}{2.457612in}}{\pgfqpoint{3.154633in}{2.460884in}}{\pgfqpoint{3.160457in}{2.466708in}}%
\pgfpathcurveto{\pgfqpoint{3.166281in}{2.472532in}}{\pgfqpoint{3.169553in}{2.480432in}}{\pgfqpoint{3.169553in}{2.488669in}}%
\pgfpathcurveto{\pgfqpoint{3.169553in}{2.496905in}}{\pgfqpoint{3.166281in}{2.504805in}}{\pgfqpoint{3.160457in}{2.510629in}}%
\pgfpathcurveto{\pgfqpoint{3.154633in}{2.516453in}}{\pgfqpoint{3.146733in}{2.519725in}}{\pgfqpoint{3.138497in}{2.519725in}}%
\pgfpathcurveto{\pgfqpoint{3.130261in}{2.519725in}}{\pgfqpoint{3.122361in}{2.516453in}}{\pgfqpoint{3.116537in}{2.510629in}}%
\pgfpathcurveto{\pgfqpoint{3.110713in}{2.504805in}}{\pgfqpoint{3.107440in}{2.496905in}}{\pgfqpoint{3.107440in}{2.488669in}}%
\pgfpathcurveto{\pgfqpoint{3.107440in}{2.480432in}}{\pgfqpoint{3.110713in}{2.472532in}}{\pgfqpoint{3.116537in}{2.466708in}}%
\pgfpathcurveto{\pgfqpoint{3.122361in}{2.460884in}}{\pgfqpoint{3.130261in}{2.457612in}}{\pgfqpoint{3.138497in}{2.457612in}}%
\pgfpathclose%
\pgfusepath{stroke,fill}%
\end{pgfscope}%
\begin{pgfscope}%
\pgfpathrectangle{\pgfqpoint{0.100000in}{0.220728in}}{\pgfqpoint{3.696000in}{3.696000in}}%
\pgfusepath{clip}%
\pgfsetbuttcap%
\pgfsetroundjoin%
\definecolor{currentfill}{rgb}{0.121569,0.466667,0.705882}%
\pgfsetfillcolor{currentfill}%
\pgfsetfillopacity{0.753756}%
\pgfsetlinewidth{1.003750pt}%
\definecolor{currentstroke}{rgb}{0.121569,0.466667,0.705882}%
\pgfsetstrokecolor{currentstroke}%
\pgfsetstrokeopacity{0.753756}%
\pgfsetdash{}{0pt}%
\pgfpathmoveto{\pgfqpoint{3.137339in}{2.454726in}}%
\pgfpathcurveto{\pgfqpoint{3.145576in}{2.454726in}}{\pgfqpoint{3.153476in}{2.457998in}}{\pgfqpoint{3.159300in}{2.463822in}}%
\pgfpathcurveto{\pgfqpoint{3.165124in}{2.469646in}}{\pgfqpoint{3.168396in}{2.477546in}}{\pgfqpoint{3.168396in}{2.485782in}}%
\pgfpathcurveto{\pgfqpoint{3.168396in}{2.494019in}}{\pgfqpoint{3.165124in}{2.501919in}}{\pgfqpoint{3.159300in}{2.507743in}}%
\pgfpathcurveto{\pgfqpoint{3.153476in}{2.513567in}}{\pgfqpoint{3.145576in}{2.516839in}}{\pgfqpoint{3.137339in}{2.516839in}}%
\pgfpathcurveto{\pgfqpoint{3.129103in}{2.516839in}}{\pgfqpoint{3.121203in}{2.513567in}}{\pgfqpoint{3.115379in}{2.507743in}}%
\pgfpathcurveto{\pgfqpoint{3.109555in}{2.501919in}}{\pgfqpoint{3.106283in}{2.494019in}}{\pgfqpoint{3.106283in}{2.485782in}}%
\pgfpathcurveto{\pgfqpoint{3.106283in}{2.477546in}}{\pgfqpoint{3.109555in}{2.469646in}}{\pgfqpoint{3.115379in}{2.463822in}}%
\pgfpathcurveto{\pgfqpoint{3.121203in}{2.457998in}}{\pgfqpoint{3.129103in}{2.454726in}}{\pgfqpoint{3.137339in}{2.454726in}}%
\pgfpathclose%
\pgfusepath{stroke,fill}%
\end{pgfscope}%
\begin{pgfscope}%
\pgfpathrectangle{\pgfqpoint{0.100000in}{0.220728in}}{\pgfqpoint{3.696000in}{3.696000in}}%
\pgfusepath{clip}%
\pgfsetbuttcap%
\pgfsetroundjoin%
\definecolor{currentfill}{rgb}{0.121569,0.466667,0.705882}%
\pgfsetfillcolor{currentfill}%
\pgfsetfillopacity{0.754030}%
\pgfsetlinewidth{1.003750pt}%
\definecolor{currentstroke}{rgb}{0.121569,0.466667,0.705882}%
\pgfsetstrokecolor{currentstroke}%
\pgfsetstrokeopacity{0.754030}%
\pgfsetdash{}{0pt}%
\pgfpathmoveto{\pgfqpoint{3.136634in}{2.453303in}}%
\pgfpathcurveto{\pgfqpoint{3.144870in}{2.453303in}}{\pgfqpoint{3.152770in}{2.456575in}}{\pgfqpoint{3.158594in}{2.462399in}}%
\pgfpathcurveto{\pgfqpoint{3.164418in}{2.468223in}}{\pgfqpoint{3.167690in}{2.476123in}}{\pgfqpoint{3.167690in}{2.484359in}}%
\pgfpathcurveto{\pgfqpoint{3.167690in}{2.492595in}}{\pgfqpoint{3.164418in}{2.500495in}}{\pgfqpoint{3.158594in}{2.506319in}}%
\pgfpathcurveto{\pgfqpoint{3.152770in}{2.512143in}}{\pgfqpoint{3.144870in}{2.515416in}}{\pgfqpoint{3.136634in}{2.515416in}}%
\pgfpathcurveto{\pgfqpoint{3.128397in}{2.515416in}}{\pgfqpoint{3.120497in}{2.512143in}}{\pgfqpoint{3.114673in}{2.506319in}}%
\pgfpathcurveto{\pgfqpoint{3.108850in}{2.500495in}}{\pgfqpoint{3.105577in}{2.492595in}}{\pgfqpoint{3.105577in}{2.484359in}}%
\pgfpathcurveto{\pgfqpoint{3.105577in}{2.476123in}}{\pgfqpoint{3.108850in}{2.468223in}}{\pgfqpoint{3.114673in}{2.462399in}}%
\pgfpathcurveto{\pgfqpoint{3.120497in}{2.456575in}}{\pgfqpoint{3.128397in}{2.453303in}}{\pgfqpoint{3.136634in}{2.453303in}}%
\pgfpathclose%
\pgfusepath{stroke,fill}%
\end{pgfscope}%
\begin{pgfscope}%
\pgfpathrectangle{\pgfqpoint{0.100000in}{0.220728in}}{\pgfqpoint{3.696000in}{3.696000in}}%
\pgfusepath{clip}%
\pgfsetbuttcap%
\pgfsetroundjoin%
\definecolor{currentfill}{rgb}{0.121569,0.466667,0.705882}%
\pgfsetfillcolor{currentfill}%
\pgfsetfillopacity{0.754184}%
\pgfsetlinewidth{1.003750pt}%
\definecolor{currentstroke}{rgb}{0.121569,0.466667,0.705882}%
\pgfsetstrokecolor{currentstroke}%
\pgfsetstrokeopacity{0.754184}%
\pgfsetdash{}{0pt}%
\pgfpathmoveto{\pgfqpoint{3.136159in}{2.452652in}}%
\pgfpathcurveto{\pgfqpoint{3.144395in}{2.452652in}}{\pgfqpoint{3.152295in}{2.455924in}}{\pgfqpoint{3.158119in}{2.461748in}}%
\pgfpathcurveto{\pgfqpoint{3.163943in}{2.467572in}}{\pgfqpoint{3.167215in}{2.475472in}}{\pgfqpoint{3.167215in}{2.483708in}}%
\pgfpathcurveto{\pgfqpoint{3.167215in}{2.491945in}}{\pgfqpoint{3.163943in}{2.499845in}}{\pgfqpoint{3.158119in}{2.505669in}}%
\pgfpathcurveto{\pgfqpoint{3.152295in}{2.511492in}}{\pgfqpoint{3.144395in}{2.514765in}}{\pgfqpoint{3.136159in}{2.514765in}}%
\pgfpathcurveto{\pgfqpoint{3.127923in}{2.514765in}}{\pgfqpoint{3.120023in}{2.511492in}}{\pgfqpoint{3.114199in}{2.505669in}}%
\pgfpathcurveto{\pgfqpoint{3.108375in}{2.499845in}}{\pgfqpoint{3.105102in}{2.491945in}}{\pgfqpoint{3.105102in}{2.483708in}}%
\pgfpathcurveto{\pgfqpoint{3.105102in}{2.475472in}}{\pgfqpoint{3.108375in}{2.467572in}}{\pgfqpoint{3.114199in}{2.461748in}}%
\pgfpathcurveto{\pgfqpoint{3.120023in}{2.455924in}}{\pgfqpoint{3.127923in}{2.452652in}}{\pgfqpoint{3.136159in}{2.452652in}}%
\pgfpathclose%
\pgfusepath{stroke,fill}%
\end{pgfscope}%
\begin{pgfscope}%
\pgfpathrectangle{\pgfqpoint{0.100000in}{0.220728in}}{\pgfqpoint{3.696000in}{3.696000in}}%
\pgfusepath{clip}%
\pgfsetbuttcap%
\pgfsetroundjoin%
\definecolor{currentfill}{rgb}{0.121569,0.466667,0.705882}%
\pgfsetfillcolor{currentfill}%
\pgfsetfillopacity{0.754268}%
\pgfsetlinewidth{1.003750pt}%
\definecolor{currentstroke}{rgb}{0.121569,0.466667,0.705882}%
\pgfsetstrokecolor{currentstroke}%
\pgfsetstrokeopacity{0.754268}%
\pgfsetdash{}{0pt}%
\pgfpathmoveto{\pgfqpoint{3.135989in}{2.452183in}}%
\pgfpathcurveto{\pgfqpoint{3.144226in}{2.452183in}}{\pgfqpoint{3.152126in}{2.455455in}}{\pgfqpoint{3.157950in}{2.461279in}}%
\pgfpathcurveto{\pgfqpoint{3.163773in}{2.467103in}}{\pgfqpoint{3.167046in}{2.475003in}}{\pgfqpoint{3.167046in}{2.483239in}}%
\pgfpathcurveto{\pgfqpoint{3.167046in}{2.491475in}}{\pgfqpoint{3.163773in}{2.499375in}}{\pgfqpoint{3.157950in}{2.505199in}}%
\pgfpathcurveto{\pgfqpoint{3.152126in}{2.511023in}}{\pgfqpoint{3.144226in}{2.514296in}}{\pgfqpoint{3.135989in}{2.514296in}}%
\pgfpathcurveto{\pgfqpoint{3.127753in}{2.514296in}}{\pgfqpoint{3.119853in}{2.511023in}}{\pgfqpoint{3.114029in}{2.505199in}}%
\pgfpathcurveto{\pgfqpoint{3.108205in}{2.499375in}}{\pgfqpoint{3.104933in}{2.491475in}}{\pgfqpoint{3.104933in}{2.483239in}}%
\pgfpathcurveto{\pgfqpoint{3.104933in}{2.475003in}}{\pgfqpoint{3.108205in}{2.467103in}}{\pgfqpoint{3.114029in}{2.461279in}}%
\pgfpathcurveto{\pgfqpoint{3.119853in}{2.455455in}}{\pgfqpoint{3.127753in}{2.452183in}}{\pgfqpoint{3.135989in}{2.452183in}}%
\pgfpathclose%
\pgfusepath{stroke,fill}%
\end{pgfscope}%
\begin{pgfscope}%
\pgfpathrectangle{\pgfqpoint{0.100000in}{0.220728in}}{\pgfqpoint{3.696000in}{3.696000in}}%
\pgfusepath{clip}%
\pgfsetbuttcap%
\pgfsetroundjoin%
\definecolor{currentfill}{rgb}{0.121569,0.466667,0.705882}%
\pgfsetfillcolor{currentfill}%
\pgfsetfillopacity{0.754543}%
\pgfsetlinewidth{1.003750pt}%
\definecolor{currentstroke}{rgb}{0.121569,0.466667,0.705882}%
\pgfsetstrokecolor{currentstroke}%
\pgfsetstrokeopacity{0.754543}%
\pgfsetdash{}{0pt}%
\pgfpathmoveto{\pgfqpoint{3.135018in}{2.450551in}}%
\pgfpathcurveto{\pgfqpoint{3.143254in}{2.450551in}}{\pgfqpoint{3.151154in}{2.453823in}}{\pgfqpoint{3.156978in}{2.459647in}}%
\pgfpathcurveto{\pgfqpoint{3.162802in}{2.465471in}}{\pgfqpoint{3.166074in}{2.473371in}}{\pgfqpoint{3.166074in}{2.481608in}}%
\pgfpathcurveto{\pgfqpoint{3.166074in}{2.489844in}}{\pgfqpoint{3.162802in}{2.497744in}}{\pgfqpoint{3.156978in}{2.503568in}}%
\pgfpathcurveto{\pgfqpoint{3.151154in}{2.509392in}}{\pgfqpoint{3.143254in}{2.512664in}}{\pgfqpoint{3.135018in}{2.512664in}}%
\pgfpathcurveto{\pgfqpoint{3.126781in}{2.512664in}}{\pgfqpoint{3.118881in}{2.509392in}}{\pgfqpoint{3.113057in}{2.503568in}}%
\pgfpathcurveto{\pgfqpoint{3.107233in}{2.497744in}}{\pgfqpoint{3.103961in}{2.489844in}}{\pgfqpoint{3.103961in}{2.481608in}}%
\pgfpathcurveto{\pgfqpoint{3.103961in}{2.473371in}}{\pgfqpoint{3.107233in}{2.465471in}}{\pgfqpoint{3.113057in}{2.459647in}}%
\pgfpathcurveto{\pgfqpoint{3.118881in}{2.453823in}}{\pgfqpoint{3.126781in}{2.450551in}}{\pgfqpoint{3.135018in}{2.450551in}}%
\pgfpathclose%
\pgfusepath{stroke,fill}%
\end{pgfscope}%
\begin{pgfscope}%
\pgfpathrectangle{\pgfqpoint{0.100000in}{0.220728in}}{\pgfqpoint{3.696000in}{3.696000in}}%
\pgfusepath{clip}%
\pgfsetbuttcap%
\pgfsetroundjoin%
\definecolor{currentfill}{rgb}{0.121569,0.466667,0.705882}%
\pgfsetfillcolor{currentfill}%
\pgfsetfillopacity{0.754750}%
\pgfsetlinewidth{1.003750pt}%
\definecolor{currentstroke}{rgb}{0.121569,0.466667,0.705882}%
\pgfsetstrokecolor{currentstroke}%
\pgfsetstrokeopacity{0.754750}%
\pgfsetdash{}{0pt}%
\pgfpathmoveto{\pgfqpoint{3.134611in}{2.449733in}}%
\pgfpathcurveto{\pgfqpoint{3.142847in}{2.449733in}}{\pgfqpoint{3.150747in}{2.453005in}}{\pgfqpoint{3.156571in}{2.458829in}}%
\pgfpathcurveto{\pgfqpoint{3.162395in}{2.464653in}}{\pgfqpoint{3.165667in}{2.472553in}}{\pgfqpoint{3.165667in}{2.480789in}}%
\pgfpathcurveto{\pgfqpoint{3.165667in}{2.489025in}}{\pgfqpoint{3.162395in}{2.496925in}}{\pgfqpoint{3.156571in}{2.502749in}}%
\pgfpathcurveto{\pgfqpoint{3.150747in}{2.508573in}}{\pgfqpoint{3.142847in}{2.511846in}}{\pgfqpoint{3.134611in}{2.511846in}}%
\pgfpathcurveto{\pgfqpoint{3.126374in}{2.511846in}}{\pgfqpoint{3.118474in}{2.508573in}}{\pgfqpoint{3.112650in}{2.502749in}}%
\pgfpathcurveto{\pgfqpoint{3.106826in}{2.496925in}}{\pgfqpoint{3.103554in}{2.489025in}}{\pgfqpoint{3.103554in}{2.480789in}}%
\pgfpathcurveto{\pgfqpoint{3.103554in}{2.472553in}}{\pgfqpoint{3.106826in}{2.464653in}}{\pgfqpoint{3.112650in}{2.458829in}}%
\pgfpathcurveto{\pgfqpoint{3.118474in}{2.453005in}}{\pgfqpoint{3.126374in}{2.449733in}}{\pgfqpoint{3.134611in}{2.449733in}}%
\pgfpathclose%
\pgfusepath{stroke,fill}%
\end{pgfscope}%
\begin{pgfscope}%
\pgfpathrectangle{\pgfqpoint{0.100000in}{0.220728in}}{\pgfqpoint{3.696000in}{3.696000in}}%
\pgfusepath{clip}%
\pgfsetbuttcap%
\pgfsetroundjoin%
\definecolor{currentfill}{rgb}{0.121569,0.466667,0.705882}%
\pgfsetfillcolor{currentfill}%
\pgfsetfillopacity{0.754846}%
\pgfsetlinewidth{1.003750pt}%
\definecolor{currentstroke}{rgb}{0.121569,0.466667,0.705882}%
\pgfsetstrokecolor{currentstroke}%
\pgfsetstrokeopacity{0.754846}%
\pgfsetdash{}{0pt}%
\pgfpathmoveto{\pgfqpoint{3.134381in}{2.449211in}}%
\pgfpathcurveto{\pgfqpoint{3.142618in}{2.449211in}}{\pgfqpoint{3.150518in}{2.452484in}}{\pgfqpoint{3.156342in}{2.458308in}}%
\pgfpathcurveto{\pgfqpoint{3.162166in}{2.464131in}}{\pgfqpoint{3.165438in}{2.472032in}}{\pgfqpoint{3.165438in}{2.480268in}}%
\pgfpathcurveto{\pgfqpoint{3.165438in}{2.488504in}}{\pgfqpoint{3.162166in}{2.496404in}}{\pgfqpoint{3.156342in}{2.502228in}}%
\pgfpathcurveto{\pgfqpoint{3.150518in}{2.508052in}}{\pgfqpoint{3.142618in}{2.511324in}}{\pgfqpoint{3.134381in}{2.511324in}}%
\pgfpathcurveto{\pgfqpoint{3.126145in}{2.511324in}}{\pgfqpoint{3.118245in}{2.508052in}}{\pgfqpoint{3.112421in}{2.502228in}}%
\pgfpathcurveto{\pgfqpoint{3.106597in}{2.496404in}}{\pgfqpoint{3.103325in}{2.488504in}}{\pgfqpoint{3.103325in}{2.480268in}}%
\pgfpathcurveto{\pgfqpoint{3.103325in}{2.472032in}}{\pgfqpoint{3.106597in}{2.464131in}}{\pgfqpoint{3.112421in}{2.458308in}}%
\pgfpathcurveto{\pgfqpoint{3.118245in}{2.452484in}}{\pgfqpoint{3.126145in}{2.449211in}}{\pgfqpoint{3.134381in}{2.449211in}}%
\pgfpathclose%
\pgfusepath{stroke,fill}%
\end{pgfscope}%
\begin{pgfscope}%
\pgfpathrectangle{\pgfqpoint{0.100000in}{0.220728in}}{\pgfqpoint{3.696000in}{3.696000in}}%
\pgfusepath{clip}%
\pgfsetbuttcap%
\pgfsetroundjoin%
\definecolor{currentfill}{rgb}{0.121569,0.466667,0.705882}%
\pgfsetfillcolor{currentfill}%
\pgfsetfillopacity{0.754891}%
\pgfsetlinewidth{1.003750pt}%
\definecolor{currentstroke}{rgb}{0.121569,0.466667,0.705882}%
\pgfsetstrokecolor{currentstroke}%
\pgfsetstrokeopacity{0.754891}%
\pgfsetdash{}{0pt}%
\pgfpathmoveto{\pgfqpoint{3.134203in}{2.448963in}}%
\pgfpathcurveto{\pgfqpoint{3.142440in}{2.448963in}}{\pgfqpoint{3.150340in}{2.452235in}}{\pgfqpoint{3.156164in}{2.458059in}}%
\pgfpathcurveto{\pgfqpoint{3.161988in}{2.463883in}}{\pgfqpoint{3.165260in}{2.471783in}}{\pgfqpoint{3.165260in}{2.480019in}}%
\pgfpathcurveto{\pgfqpoint{3.165260in}{2.488256in}}{\pgfqpoint{3.161988in}{2.496156in}}{\pgfqpoint{3.156164in}{2.501980in}}%
\pgfpathcurveto{\pgfqpoint{3.150340in}{2.507804in}}{\pgfqpoint{3.142440in}{2.511076in}}{\pgfqpoint{3.134203in}{2.511076in}}%
\pgfpathcurveto{\pgfqpoint{3.125967in}{2.511076in}}{\pgfqpoint{3.118067in}{2.507804in}}{\pgfqpoint{3.112243in}{2.501980in}}%
\pgfpathcurveto{\pgfqpoint{3.106419in}{2.496156in}}{\pgfqpoint{3.103147in}{2.488256in}}{\pgfqpoint{3.103147in}{2.480019in}}%
\pgfpathcurveto{\pgfqpoint{3.103147in}{2.471783in}}{\pgfqpoint{3.106419in}{2.463883in}}{\pgfqpoint{3.112243in}{2.458059in}}%
\pgfpathcurveto{\pgfqpoint{3.118067in}{2.452235in}}{\pgfqpoint{3.125967in}{2.448963in}}{\pgfqpoint{3.134203in}{2.448963in}}%
\pgfpathclose%
\pgfusepath{stroke,fill}%
\end{pgfscope}%
\begin{pgfscope}%
\pgfpathrectangle{\pgfqpoint{0.100000in}{0.220728in}}{\pgfqpoint{3.696000in}{3.696000in}}%
\pgfusepath{clip}%
\pgfsetbuttcap%
\pgfsetroundjoin%
\definecolor{currentfill}{rgb}{0.121569,0.466667,0.705882}%
\pgfsetfillcolor{currentfill}%
\pgfsetfillopacity{0.755303}%
\pgfsetlinewidth{1.003750pt}%
\definecolor{currentstroke}{rgb}{0.121569,0.466667,0.705882}%
\pgfsetstrokecolor{currentstroke}%
\pgfsetstrokeopacity{0.755303}%
\pgfsetdash{}{0pt}%
\pgfpathmoveto{\pgfqpoint{3.133335in}{2.446837in}}%
\pgfpathcurveto{\pgfqpoint{3.141572in}{2.446837in}}{\pgfqpoint{3.149472in}{2.450109in}}{\pgfqpoint{3.155296in}{2.455933in}}%
\pgfpathcurveto{\pgfqpoint{3.161119in}{2.461757in}}{\pgfqpoint{3.164392in}{2.469657in}}{\pgfqpoint{3.164392in}{2.477893in}}%
\pgfpathcurveto{\pgfqpoint{3.164392in}{2.486129in}}{\pgfqpoint{3.161119in}{2.494029in}}{\pgfqpoint{3.155296in}{2.499853in}}%
\pgfpathcurveto{\pgfqpoint{3.149472in}{2.505677in}}{\pgfqpoint{3.141572in}{2.508950in}}{\pgfqpoint{3.133335in}{2.508950in}}%
\pgfpathcurveto{\pgfqpoint{3.125099in}{2.508950in}}{\pgfqpoint{3.117199in}{2.505677in}}{\pgfqpoint{3.111375in}{2.499853in}}%
\pgfpathcurveto{\pgfqpoint{3.105551in}{2.494029in}}{\pgfqpoint{3.102279in}{2.486129in}}{\pgfqpoint{3.102279in}{2.477893in}}%
\pgfpathcurveto{\pgfqpoint{3.102279in}{2.469657in}}{\pgfqpoint{3.105551in}{2.461757in}}{\pgfqpoint{3.111375in}{2.455933in}}%
\pgfpathcurveto{\pgfqpoint{3.117199in}{2.450109in}}{\pgfqpoint{3.125099in}{2.446837in}}{\pgfqpoint{3.133335in}{2.446837in}}%
\pgfpathclose%
\pgfusepath{stroke,fill}%
\end{pgfscope}%
\begin{pgfscope}%
\pgfpathrectangle{\pgfqpoint{0.100000in}{0.220728in}}{\pgfqpoint{3.696000in}{3.696000in}}%
\pgfusepath{clip}%
\pgfsetbuttcap%
\pgfsetroundjoin%
\definecolor{currentfill}{rgb}{0.121569,0.466667,0.705882}%
\pgfsetfillcolor{currentfill}%
\pgfsetfillopacity{0.755346}%
\pgfsetlinewidth{1.003750pt}%
\definecolor{currentstroke}{rgb}{0.121569,0.466667,0.705882}%
\pgfsetstrokecolor{currentstroke}%
\pgfsetstrokeopacity{0.755346}%
\pgfsetdash{}{0pt}%
\pgfpathmoveto{\pgfqpoint{1.119045in}{1.210556in}}%
\pgfpathcurveto{\pgfqpoint{1.127282in}{1.210556in}}{\pgfqpoint{1.135182in}{1.213828in}}{\pgfqpoint{1.141006in}{1.219652in}}%
\pgfpathcurveto{\pgfqpoint{1.146830in}{1.225476in}}{\pgfqpoint{1.150102in}{1.233376in}}{\pgfqpoint{1.150102in}{1.241612in}}%
\pgfpathcurveto{\pgfqpoint{1.150102in}{1.249849in}}{\pgfqpoint{1.146830in}{1.257749in}}{\pgfqpoint{1.141006in}{1.263573in}}%
\pgfpathcurveto{\pgfqpoint{1.135182in}{1.269397in}}{\pgfqpoint{1.127282in}{1.272669in}}{\pgfqpoint{1.119045in}{1.272669in}}%
\pgfpathcurveto{\pgfqpoint{1.110809in}{1.272669in}}{\pgfqpoint{1.102909in}{1.269397in}}{\pgfqpoint{1.097085in}{1.263573in}}%
\pgfpathcurveto{\pgfqpoint{1.091261in}{1.257749in}}{\pgfqpoint{1.087989in}{1.249849in}}{\pgfqpoint{1.087989in}{1.241612in}}%
\pgfpathcurveto{\pgfqpoint{1.087989in}{1.233376in}}{\pgfqpoint{1.091261in}{1.225476in}}{\pgfqpoint{1.097085in}{1.219652in}}%
\pgfpathcurveto{\pgfqpoint{1.102909in}{1.213828in}}{\pgfqpoint{1.110809in}{1.210556in}}{\pgfqpoint{1.119045in}{1.210556in}}%
\pgfpathclose%
\pgfusepath{stroke,fill}%
\end{pgfscope}%
\begin{pgfscope}%
\pgfpathrectangle{\pgfqpoint{0.100000in}{0.220728in}}{\pgfqpoint{3.696000in}{3.696000in}}%
\pgfusepath{clip}%
\pgfsetbuttcap%
\pgfsetroundjoin%
\definecolor{currentfill}{rgb}{0.121569,0.466667,0.705882}%
\pgfsetfillcolor{currentfill}%
\pgfsetfillopacity{0.755514}%
\pgfsetlinewidth{1.003750pt}%
\definecolor{currentstroke}{rgb}{0.121569,0.466667,0.705882}%
\pgfsetstrokecolor{currentstroke}%
\pgfsetstrokeopacity{0.755514}%
\pgfsetdash{}{0pt}%
\pgfpathmoveto{\pgfqpoint{3.132767in}{2.445699in}}%
\pgfpathcurveto{\pgfqpoint{3.141004in}{2.445699in}}{\pgfqpoint{3.148904in}{2.448971in}}{\pgfqpoint{3.154728in}{2.454795in}}%
\pgfpathcurveto{\pgfqpoint{3.160551in}{2.460619in}}{\pgfqpoint{3.163824in}{2.468519in}}{\pgfqpoint{3.163824in}{2.476755in}}%
\pgfpathcurveto{\pgfqpoint{3.163824in}{2.484992in}}{\pgfqpoint{3.160551in}{2.492892in}}{\pgfqpoint{3.154728in}{2.498715in}}%
\pgfpathcurveto{\pgfqpoint{3.148904in}{2.504539in}}{\pgfqpoint{3.141004in}{2.507812in}}{\pgfqpoint{3.132767in}{2.507812in}}%
\pgfpathcurveto{\pgfqpoint{3.124531in}{2.507812in}}{\pgfqpoint{3.116631in}{2.504539in}}{\pgfqpoint{3.110807in}{2.498715in}}%
\pgfpathcurveto{\pgfqpoint{3.104983in}{2.492892in}}{\pgfqpoint{3.101711in}{2.484992in}}{\pgfqpoint{3.101711in}{2.476755in}}%
\pgfpathcurveto{\pgfqpoint{3.101711in}{2.468519in}}{\pgfqpoint{3.104983in}{2.460619in}}{\pgfqpoint{3.110807in}{2.454795in}}%
\pgfpathcurveto{\pgfqpoint{3.116631in}{2.448971in}}{\pgfqpoint{3.124531in}{2.445699in}}{\pgfqpoint{3.132767in}{2.445699in}}%
\pgfpathclose%
\pgfusepath{stroke,fill}%
\end{pgfscope}%
\begin{pgfscope}%
\pgfpathrectangle{\pgfqpoint{0.100000in}{0.220728in}}{\pgfqpoint{3.696000in}{3.696000in}}%
\pgfusepath{clip}%
\pgfsetbuttcap%
\pgfsetroundjoin%
\definecolor{currentfill}{rgb}{0.121569,0.466667,0.705882}%
\pgfsetfillcolor{currentfill}%
\pgfsetfillopacity{0.755633}%
\pgfsetlinewidth{1.003750pt}%
\definecolor{currentstroke}{rgb}{0.121569,0.466667,0.705882}%
\pgfsetstrokecolor{currentstroke}%
\pgfsetstrokeopacity{0.755633}%
\pgfsetdash{}{0pt}%
\pgfpathmoveto{\pgfqpoint{3.132394in}{2.445169in}}%
\pgfpathcurveto{\pgfqpoint{3.140631in}{2.445169in}}{\pgfqpoint{3.148531in}{2.448442in}}{\pgfqpoint{3.154355in}{2.454265in}}%
\pgfpathcurveto{\pgfqpoint{3.160179in}{2.460089in}}{\pgfqpoint{3.163451in}{2.467989in}}{\pgfqpoint{3.163451in}{2.476226in}}%
\pgfpathcurveto{\pgfqpoint{3.163451in}{2.484462in}}{\pgfqpoint{3.160179in}{2.492362in}}{\pgfqpoint{3.154355in}{2.498186in}}%
\pgfpathcurveto{\pgfqpoint{3.148531in}{2.504010in}}{\pgfqpoint{3.140631in}{2.507282in}}{\pgfqpoint{3.132394in}{2.507282in}}%
\pgfpathcurveto{\pgfqpoint{3.124158in}{2.507282in}}{\pgfqpoint{3.116258in}{2.504010in}}{\pgfqpoint{3.110434in}{2.498186in}}%
\pgfpathcurveto{\pgfqpoint{3.104610in}{2.492362in}}{\pgfqpoint{3.101338in}{2.484462in}}{\pgfqpoint{3.101338in}{2.476226in}}%
\pgfpathcurveto{\pgfqpoint{3.101338in}{2.467989in}}{\pgfqpoint{3.104610in}{2.460089in}}{\pgfqpoint{3.110434in}{2.454265in}}%
\pgfpathcurveto{\pgfqpoint{3.116258in}{2.448442in}}{\pgfqpoint{3.124158in}{2.445169in}}{\pgfqpoint{3.132394in}{2.445169in}}%
\pgfpathclose%
\pgfusepath{stroke,fill}%
\end{pgfscope}%
\begin{pgfscope}%
\pgfpathrectangle{\pgfqpoint{0.100000in}{0.220728in}}{\pgfqpoint{3.696000in}{3.696000in}}%
\pgfusepath{clip}%
\pgfsetbuttcap%
\pgfsetroundjoin%
\definecolor{currentfill}{rgb}{0.121569,0.466667,0.705882}%
\pgfsetfillcolor{currentfill}%
\pgfsetfillopacity{0.756058}%
\pgfsetlinewidth{1.003750pt}%
\definecolor{currentstroke}{rgb}{0.121569,0.466667,0.705882}%
\pgfsetstrokecolor{currentstroke}%
\pgfsetstrokeopacity{0.756058}%
\pgfsetdash{}{0pt}%
\pgfpathmoveto{\pgfqpoint{3.131577in}{2.443087in}}%
\pgfpathcurveto{\pgfqpoint{3.139813in}{2.443087in}}{\pgfqpoint{3.147713in}{2.446359in}}{\pgfqpoint{3.153537in}{2.452183in}}%
\pgfpathcurveto{\pgfqpoint{3.159361in}{2.458007in}}{\pgfqpoint{3.162633in}{2.465907in}}{\pgfqpoint{3.162633in}{2.474143in}}%
\pgfpathcurveto{\pgfqpoint{3.162633in}{2.482379in}}{\pgfqpoint{3.159361in}{2.490279in}}{\pgfqpoint{3.153537in}{2.496103in}}%
\pgfpathcurveto{\pgfqpoint{3.147713in}{2.501927in}}{\pgfqpoint{3.139813in}{2.505200in}}{\pgfqpoint{3.131577in}{2.505200in}}%
\pgfpathcurveto{\pgfqpoint{3.123341in}{2.505200in}}{\pgfqpoint{3.115441in}{2.501927in}}{\pgfqpoint{3.109617in}{2.496103in}}%
\pgfpathcurveto{\pgfqpoint{3.103793in}{2.490279in}}{\pgfqpoint{3.100520in}{2.482379in}}{\pgfqpoint{3.100520in}{2.474143in}}%
\pgfpathcurveto{\pgfqpoint{3.100520in}{2.465907in}}{\pgfqpoint{3.103793in}{2.458007in}}{\pgfqpoint{3.109617in}{2.452183in}}%
\pgfpathcurveto{\pgfqpoint{3.115441in}{2.446359in}}{\pgfqpoint{3.123341in}{2.443087in}}{\pgfqpoint{3.131577in}{2.443087in}}%
\pgfpathclose%
\pgfusepath{stroke,fill}%
\end{pgfscope}%
\begin{pgfscope}%
\pgfpathrectangle{\pgfqpoint{0.100000in}{0.220728in}}{\pgfqpoint{3.696000in}{3.696000in}}%
\pgfusepath{clip}%
\pgfsetbuttcap%
\pgfsetroundjoin%
\definecolor{currentfill}{rgb}{0.121569,0.466667,0.705882}%
\pgfsetfillcolor{currentfill}%
\pgfsetfillopacity{0.756621}%
\pgfsetlinewidth{1.003750pt}%
\definecolor{currentstroke}{rgb}{0.121569,0.466667,0.705882}%
\pgfsetstrokecolor{currentstroke}%
\pgfsetstrokeopacity{0.756621}%
\pgfsetdash{}{0pt}%
\pgfpathmoveto{\pgfqpoint{3.130094in}{2.440338in}}%
\pgfpathcurveto{\pgfqpoint{3.138330in}{2.440338in}}{\pgfqpoint{3.146231in}{2.443611in}}{\pgfqpoint{3.152054in}{2.449435in}}%
\pgfpathcurveto{\pgfqpoint{3.157878in}{2.455258in}}{\pgfqpoint{3.161151in}{2.463159in}}{\pgfqpoint{3.161151in}{2.471395in}}%
\pgfpathcurveto{\pgfqpoint{3.161151in}{2.479631in}}{\pgfqpoint{3.157878in}{2.487531in}}{\pgfqpoint{3.152054in}{2.493355in}}%
\pgfpathcurveto{\pgfqpoint{3.146231in}{2.499179in}}{\pgfqpoint{3.138330in}{2.502451in}}{\pgfqpoint{3.130094in}{2.502451in}}%
\pgfpathcurveto{\pgfqpoint{3.121858in}{2.502451in}}{\pgfqpoint{3.113958in}{2.499179in}}{\pgfqpoint{3.108134in}{2.493355in}}%
\pgfpathcurveto{\pgfqpoint{3.102310in}{2.487531in}}{\pgfqpoint{3.099038in}{2.479631in}}{\pgfqpoint{3.099038in}{2.471395in}}%
\pgfpathcurveto{\pgfqpoint{3.099038in}{2.463159in}}{\pgfqpoint{3.102310in}{2.455258in}}{\pgfqpoint{3.108134in}{2.449435in}}%
\pgfpathcurveto{\pgfqpoint{3.113958in}{2.443611in}}{\pgfqpoint{3.121858in}{2.440338in}}{\pgfqpoint{3.130094in}{2.440338in}}%
\pgfpathclose%
\pgfusepath{stroke,fill}%
\end{pgfscope}%
\begin{pgfscope}%
\pgfpathrectangle{\pgfqpoint{0.100000in}{0.220728in}}{\pgfqpoint{3.696000in}{3.696000in}}%
\pgfusepath{clip}%
\pgfsetbuttcap%
\pgfsetroundjoin%
\definecolor{currentfill}{rgb}{0.121569,0.466667,0.705882}%
\pgfsetfillcolor{currentfill}%
\pgfsetfillopacity{0.756953}%
\pgfsetlinewidth{1.003750pt}%
\definecolor{currentstroke}{rgb}{0.121569,0.466667,0.705882}%
\pgfsetstrokecolor{currentstroke}%
\pgfsetstrokeopacity{0.756953}%
\pgfsetdash{}{0pt}%
\pgfpathmoveto{\pgfqpoint{3.129200in}{2.439027in}}%
\pgfpathcurveto{\pgfqpoint{3.137436in}{2.439027in}}{\pgfqpoint{3.145336in}{2.442299in}}{\pgfqpoint{3.151160in}{2.448123in}}%
\pgfpathcurveto{\pgfqpoint{3.156984in}{2.453947in}}{\pgfqpoint{3.160256in}{2.461847in}}{\pgfqpoint{3.160256in}{2.470083in}}%
\pgfpathcurveto{\pgfqpoint{3.160256in}{2.478319in}}{\pgfqpoint{3.156984in}{2.486220in}}{\pgfqpoint{3.151160in}{2.492043in}}%
\pgfpathcurveto{\pgfqpoint{3.145336in}{2.497867in}}{\pgfqpoint{3.137436in}{2.501140in}}{\pgfqpoint{3.129200in}{2.501140in}}%
\pgfpathcurveto{\pgfqpoint{3.120963in}{2.501140in}}{\pgfqpoint{3.113063in}{2.497867in}}{\pgfqpoint{3.107239in}{2.492043in}}%
\pgfpathcurveto{\pgfqpoint{3.101416in}{2.486220in}}{\pgfqpoint{3.098143in}{2.478319in}}{\pgfqpoint{3.098143in}{2.470083in}}%
\pgfpathcurveto{\pgfqpoint{3.098143in}{2.461847in}}{\pgfqpoint{3.101416in}{2.453947in}}{\pgfqpoint{3.107239in}{2.448123in}}%
\pgfpathcurveto{\pgfqpoint{3.113063in}{2.442299in}}{\pgfqpoint{3.120963in}{2.439027in}}{\pgfqpoint{3.129200in}{2.439027in}}%
\pgfpathclose%
\pgfusepath{stroke,fill}%
\end{pgfscope}%
\begin{pgfscope}%
\pgfpathrectangle{\pgfqpoint{0.100000in}{0.220728in}}{\pgfqpoint{3.696000in}{3.696000in}}%
\pgfusepath{clip}%
\pgfsetbuttcap%
\pgfsetroundjoin%
\definecolor{currentfill}{rgb}{0.121569,0.466667,0.705882}%
\pgfsetfillcolor{currentfill}%
\pgfsetfillopacity{0.757120}%
\pgfsetlinewidth{1.003750pt}%
\definecolor{currentstroke}{rgb}{0.121569,0.466667,0.705882}%
\pgfsetstrokecolor{currentstroke}%
\pgfsetstrokeopacity{0.757120}%
\pgfsetdash{}{0pt}%
\pgfpathmoveto{\pgfqpoint{3.128866in}{2.438060in}}%
\pgfpathcurveto{\pgfqpoint{3.137103in}{2.438060in}}{\pgfqpoint{3.145003in}{2.441333in}}{\pgfqpoint{3.150827in}{2.447157in}}%
\pgfpathcurveto{\pgfqpoint{3.156650in}{2.452981in}}{\pgfqpoint{3.159923in}{2.460881in}}{\pgfqpoint{3.159923in}{2.469117in}}%
\pgfpathcurveto{\pgfqpoint{3.159923in}{2.477353in}}{\pgfqpoint{3.156650in}{2.485253in}}{\pgfqpoint{3.150827in}{2.491077in}}%
\pgfpathcurveto{\pgfqpoint{3.145003in}{2.496901in}}{\pgfqpoint{3.137103in}{2.500173in}}{\pgfqpoint{3.128866in}{2.500173in}}%
\pgfpathcurveto{\pgfqpoint{3.120630in}{2.500173in}}{\pgfqpoint{3.112730in}{2.496901in}}{\pgfqpoint{3.106906in}{2.491077in}}%
\pgfpathcurveto{\pgfqpoint{3.101082in}{2.485253in}}{\pgfqpoint{3.097810in}{2.477353in}}{\pgfqpoint{3.097810in}{2.469117in}}%
\pgfpathcurveto{\pgfqpoint{3.097810in}{2.460881in}}{\pgfqpoint{3.101082in}{2.452981in}}{\pgfqpoint{3.106906in}{2.447157in}}%
\pgfpathcurveto{\pgfqpoint{3.112730in}{2.441333in}}{\pgfqpoint{3.120630in}{2.438060in}}{\pgfqpoint{3.128866in}{2.438060in}}%
\pgfpathclose%
\pgfusepath{stroke,fill}%
\end{pgfscope}%
\begin{pgfscope}%
\pgfpathrectangle{\pgfqpoint{0.100000in}{0.220728in}}{\pgfqpoint{3.696000in}{3.696000in}}%
\pgfusepath{clip}%
\pgfsetbuttcap%
\pgfsetroundjoin%
\definecolor{currentfill}{rgb}{0.121569,0.466667,0.705882}%
\pgfsetfillcolor{currentfill}%
\pgfsetfillopacity{0.757457}%
\pgfsetlinewidth{1.003750pt}%
\definecolor{currentstroke}{rgb}{0.121569,0.466667,0.705882}%
\pgfsetstrokecolor{currentstroke}%
\pgfsetstrokeopacity{0.757457}%
\pgfsetdash{}{0pt}%
\pgfpathmoveto{\pgfqpoint{3.127750in}{2.436049in}}%
\pgfpathcurveto{\pgfqpoint{3.135987in}{2.436049in}}{\pgfqpoint{3.143887in}{2.439322in}}{\pgfqpoint{3.149711in}{2.445146in}}%
\pgfpathcurveto{\pgfqpoint{3.155535in}{2.450970in}}{\pgfqpoint{3.158807in}{2.458870in}}{\pgfqpoint{3.158807in}{2.467106in}}%
\pgfpathcurveto{\pgfqpoint{3.158807in}{2.475342in}}{\pgfqpoint{3.155535in}{2.483242in}}{\pgfqpoint{3.149711in}{2.489066in}}%
\pgfpathcurveto{\pgfqpoint{3.143887in}{2.494890in}}{\pgfqpoint{3.135987in}{2.498162in}}{\pgfqpoint{3.127750in}{2.498162in}}%
\pgfpathcurveto{\pgfqpoint{3.119514in}{2.498162in}}{\pgfqpoint{3.111614in}{2.494890in}}{\pgfqpoint{3.105790in}{2.489066in}}%
\pgfpathcurveto{\pgfqpoint{3.099966in}{2.483242in}}{\pgfqpoint{3.096694in}{2.475342in}}{\pgfqpoint{3.096694in}{2.467106in}}%
\pgfpathcurveto{\pgfqpoint{3.096694in}{2.458870in}}{\pgfqpoint{3.099966in}{2.450970in}}{\pgfqpoint{3.105790in}{2.445146in}}%
\pgfpathcurveto{\pgfqpoint{3.111614in}{2.439322in}}{\pgfqpoint{3.119514in}{2.436049in}}{\pgfqpoint{3.127750in}{2.436049in}}%
\pgfpathclose%
\pgfusepath{stroke,fill}%
\end{pgfscope}%
\begin{pgfscope}%
\pgfpathrectangle{\pgfqpoint{0.100000in}{0.220728in}}{\pgfqpoint{3.696000in}{3.696000in}}%
\pgfusepath{clip}%
\pgfsetbuttcap%
\pgfsetroundjoin%
\definecolor{currentfill}{rgb}{0.121569,0.466667,0.705882}%
\pgfsetfillcolor{currentfill}%
\pgfsetfillopacity{0.757671}%
\pgfsetlinewidth{1.003750pt}%
\definecolor{currentstroke}{rgb}{0.121569,0.466667,0.705882}%
\pgfsetstrokecolor{currentstroke}%
\pgfsetstrokeopacity{0.757671}%
\pgfsetdash{}{0pt}%
\pgfpathmoveto{\pgfqpoint{3.127142in}{2.435059in}}%
\pgfpathcurveto{\pgfqpoint{3.135379in}{2.435059in}}{\pgfqpoint{3.143279in}{2.438331in}}{\pgfqpoint{3.149103in}{2.444155in}}%
\pgfpathcurveto{\pgfqpoint{3.154927in}{2.449979in}}{\pgfqpoint{3.158199in}{2.457879in}}{\pgfqpoint{3.158199in}{2.466115in}}%
\pgfpathcurveto{\pgfqpoint{3.158199in}{2.474352in}}{\pgfqpoint{3.154927in}{2.482252in}}{\pgfqpoint{3.149103in}{2.488076in}}%
\pgfpathcurveto{\pgfqpoint{3.143279in}{2.493900in}}{\pgfqpoint{3.135379in}{2.497172in}}{\pgfqpoint{3.127142in}{2.497172in}}%
\pgfpathcurveto{\pgfqpoint{3.118906in}{2.497172in}}{\pgfqpoint{3.111006in}{2.493900in}}{\pgfqpoint{3.105182in}{2.488076in}}%
\pgfpathcurveto{\pgfqpoint{3.099358in}{2.482252in}}{\pgfqpoint{3.096086in}{2.474352in}}{\pgfqpoint{3.096086in}{2.466115in}}%
\pgfpathcurveto{\pgfqpoint{3.096086in}{2.457879in}}{\pgfqpoint{3.099358in}{2.449979in}}{\pgfqpoint{3.105182in}{2.444155in}}%
\pgfpathcurveto{\pgfqpoint{3.111006in}{2.438331in}}{\pgfqpoint{3.118906in}{2.435059in}}{\pgfqpoint{3.127142in}{2.435059in}}%
\pgfpathclose%
\pgfusepath{stroke,fill}%
\end{pgfscope}%
\begin{pgfscope}%
\pgfpathrectangle{\pgfqpoint{0.100000in}{0.220728in}}{\pgfqpoint{3.696000in}{3.696000in}}%
\pgfusepath{clip}%
\pgfsetbuttcap%
\pgfsetroundjoin%
\definecolor{currentfill}{rgb}{0.121569,0.466667,0.705882}%
\pgfsetfillcolor{currentfill}%
\pgfsetfillopacity{0.758041}%
\pgfsetlinewidth{1.003750pt}%
\definecolor{currentstroke}{rgb}{0.121569,0.466667,0.705882}%
\pgfsetstrokecolor{currentstroke}%
\pgfsetstrokeopacity{0.758041}%
\pgfsetdash{}{0pt}%
\pgfpathmoveto{\pgfqpoint{3.126505in}{2.433123in}}%
\pgfpathcurveto{\pgfqpoint{3.134741in}{2.433123in}}{\pgfqpoint{3.142641in}{2.436396in}}{\pgfqpoint{3.148465in}{2.442220in}}%
\pgfpathcurveto{\pgfqpoint{3.154289in}{2.448043in}}{\pgfqpoint{3.157562in}{2.455944in}}{\pgfqpoint{3.157562in}{2.464180in}}%
\pgfpathcurveto{\pgfqpoint{3.157562in}{2.472416in}}{\pgfqpoint{3.154289in}{2.480316in}}{\pgfqpoint{3.148465in}{2.486140in}}%
\pgfpathcurveto{\pgfqpoint{3.142641in}{2.491964in}}{\pgfqpoint{3.134741in}{2.495236in}}{\pgfqpoint{3.126505in}{2.495236in}}%
\pgfpathcurveto{\pgfqpoint{3.118269in}{2.495236in}}{\pgfqpoint{3.110369in}{2.491964in}}{\pgfqpoint{3.104545in}{2.486140in}}%
\pgfpathcurveto{\pgfqpoint{3.098721in}{2.480316in}}{\pgfqpoint{3.095449in}{2.472416in}}{\pgfqpoint{3.095449in}{2.464180in}}%
\pgfpathcurveto{\pgfqpoint{3.095449in}{2.455944in}}{\pgfqpoint{3.098721in}{2.448043in}}{\pgfqpoint{3.104545in}{2.442220in}}%
\pgfpathcurveto{\pgfqpoint{3.110369in}{2.436396in}}{\pgfqpoint{3.118269in}{2.433123in}}{\pgfqpoint{3.126505in}{2.433123in}}%
\pgfpathclose%
\pgfusepath{stroke,fill}%
\end{pgfscope}%
\begin{pgfscope}%
\pgfpathrectangle{\pgfqpoint{0.100000in}{0.220728in}}{\pgfqpoint{3.696000in}{3.696000in}}%
\pgfusepath{clip}%
\pgfsetbuttcap%
\pgfsetroundjoin%
\definecolor{currentfill}{rgb}{0.121569,0.466667,0.705882}%
\pgfsetfillcolor{currentfill}%
\pgfsetfillopacity{0.758614}%
\pgfsetlinewidth{1.003750pt}%
\definecolor{currentstroke}{rgb}{0.121569,0.466667,0.705882}%
\pgfsetstrokecolor{currentstroke}%
\pgfsetstrokeopacity{0.758614}%
\pgfsetdash{}{0pt}%
\pgfpathmoveto{\pgfqpoint{3.124524in}{2.429847in}}%
\pgfpathcurveto{\pgfqpoint{3.132760in}{2.429847in}}{\pgfqpoint{3.140660in}{2.433119in}}{\pgfqpoint{3.146484in}{2.438943in}}%
\pgfpathcurveto{\pgfqpoint{3.152308in}{2.444767in}}{\pgfqpoint{3.155580in}{2.452667in}}{\pgfqpoint{3.155580in}{2.460903in}}%
\pgfpathcurveto{\pgfqpoint{3.155580in}{2.469140in}}{\pgfqpoint{3.152308in}{2.477040in}}{\pgfqpoint{3.146484in}{2.482864in}}%
\pgfpathcurveto{\pgfqpoint{3.140660in}{2.488688in}}{\pgfqpoint{3.132760in}{2.491960in}}{\pgfqpoint{3.124524in}{2.491960in}}%
\pgfpathcurveto{\pgfqpoint{3.116288in}{2.491960in}}{\pgfqpoint{3.108387in}{2.488688in}}{\pgfqpoint{3.102564in}{2.482864in}}%
\pgfpathcurveto{\pgfqpoint{3.096740in}{2.477040in}}{\pgfqpoint{3.093467in}{2.469140in}}{\pgfqpoint{3.093467in}{2.460903in}}%
\pgfpathcurveto{\pgfqpoint{3.093467in}{2.452667in}}{\pgfqpoint{3.096740in}{2.444767in}}{\pgfqpoint{3.102564in}{2.438943in}}%
\pgfpathcurveto{\pgfqpoint{3.108387in}{2.433119in}}{\pgfqpoint{3.116288in}{2.429847in}}{\pgfqpoint{3.124524in}{2.429847in}}%
\pgfpathclose%
\pgfusepath{stroke,fill}%
\end{pgfscope}%
\begin{pgfscope}%
\pgfpathrectangle{\pgfqpoint{0.100000in}{0.220728in}}{\pgfqpoint{3.696000in}{3.696000in}}%
\pgfusepath{clip}%
\pgfsetbuttcap%
\pgfsetroundjoin%
\definecolor{currentfill}{rgb}{0.121569,0.466667,0.705882}%
\pgfsetfillcolor{currentfill}%
\pgfsetfillopacity{0.758802}%
\pgfsetlinewidth{1.003750pt}%
\definecolor{currentstroke}{rgb}{0.121569,0.466667,0.705882}%
\pgfsetstrokecolor{currentstroke}%
\pgfsetstrokeopacity{0.758802}%
\pgfsetdash{}{0pt}%
\pgfpathmoveto{\pgfqpoint{1.134358in}{1.203786in}}%
\pgfpathcurveto{\pgfqpoint{1.142594in}{1.203786in}}{\pgfqpoint{1.150494in}{1.207059in}}{\pgfqpoint{1.156318in}{1.212883in}}%
\pgfpathcurveto{\pgfqpoint{1.162142in}{1.218707in}}{\pgfqpoint{1.165414in}{1.226607in}}{\pgfqpoint{1.165414in}{1.234843in}}%
\pgfpathcurveto{\pgfqpoint{1.165414in}{1.243079in}}{\pgfqpoint{1.162142in}{1.250979in}}{\pgfqpoint{1.156318in}{1.256803in}}%
\pgfpathcurveto{\pgfqpoint{1.150494in}{1.262627in}}{\pgfqpoint{1.142594in}{1.265899in}}{\pgfqpoint{1.134358in}{1.265899in}}%
\pgfpathcurveto{\pgfqpoint{1.126122in}{1.265899in}}{\pgfqpoint{1.118222in}{1.262627in}}{\pgfqpoint{1.112398in}{1.256803in}}%
\pgfpathcurveto{\pgfqpoint{1.106574in}{1.250979in}}{\pgfqpoint{1.103301in}{1.243079in}}{\pgfqpoint{1.103301in}{1.234843in}}%
\pgfpathcurveto{\pgfqpoint{1.103301in}{1.226607in}}{\pgfqpoint{1.106574in}{1.218707in}}{\pgfqpoint{1.112398in}{1.212883in}}%
\pgfpathcurveto{\pgfqpoint{1.118222in}{1.207059in}}{\pgfqpoint{1.126122in}{1.203786in}}{\pgfqpoint{1.134358in}{1.203786in}}%
\pgfpathclose%
\pgfusepath{stroke,fill}%
\end{pgfscope}%
\begin{pgfscope}%
\pgfpathrectangle{\pgfqpoint{0.100000in}{0.220728in}}{\pgfqpoint{3.696000in}{3.696000in}}%
\pgfusepath{clip}%
\pgfsetbuttcap%
\pgfsetroundjoin%
\definecolor{currentfill}{rgb}{0.121569,0.466667,0.705882}%
\pgfsetfillcolor{currentfill}%
\pgfsetfillopacity{0.758992}%
\pgfsetlinewidth{1.003750pt}%
\definecolor{currentstroke}{rgb}{0.121569,0.466667,0.705882}%
\pgfsetstrokecolor{currentstroke}%
\pgfsetstrokeopacity{0.758992}%
\pgfsetdash{}{0pt}%
\pgfpathmoveto{\pgfqpoint{3.123733in}{2.427952in}}%
\pgfpathcurveto{\pgfqpoint{3.131969in}{2.427952in}}{\pgfqpoint{3.139870in}{2.431224in}}{\pgfqpoint{3.145693in}{2.437048in}}%
\pgfpathcurveto{\pgfqpoint{3.151517in}{2.442872in}}{\pgfqpoint{3.154790in}{2.450772in}}{\pgfqpoint{3.154790in}{2.459008in}}%
\pgfpathcurveto{\pgfqpoint{3.154790in}{2.467244in}}{\pgfqpoint{3.151517in}{2.475144in}}{\pgfqpoint{3.145693in}{2.480968in}}%
\pgfpathcurveto{\pgfqpoint{3.139870in}{2.486792in}}{\pgfqpoint{3.131969in}{2.490065in}}{\pgfqpoint{3.123733in}{2.490065in}}%
\pgfpathcurveto{\pgfqpoint{3.115497in}{2.490065in}}{\pgfqpoint{3.107597in}{2.486792in}}{\pgfqpoint{3.101773in}{2.480968in}}%
\pgfpathcurveto{\pgfqpoint{3.095949in}{2.475144in}}{\pgfqpoint{3.092677in}{2.467244in}}{\pgfqpoint{3.092677in}{2.459008in}}%
\pgfpathcurveto{\pgfqpoint{3.092677in}{2.450772in}}{\pgfqpoint{3.095949in}{2.442872in}}{\pgfqpoint{3.101773in}{2.437048in}}%
\pgfpathcurveto{\pgfqpoint{3.107597in}{2.431224in}}{\pgfqpoint{3.115497in}{2.427952in}}{\pgfqpoint{3.123733in}{2.427952in}}%
\pgfpathclose%
\pgfusepath{stroke,fill}%
\end{pgfscope}%
\begin{pgfscope}%
\pgfpathrectangle{\pgfqpoint{0.100000in}{0.220728in}}{\pgfqpoint{3.696000in}{3.696000in}}%
\pgfusepath{clip}%
\pgfsetbuttcap%
\pgfsetroundjoin%
\definecolor{currentfill}{rgb}{0.121569,0.466667,0.705882}%
\pgfsetfillcolor{currentfill}%
\pgfsetfillopacity{0.759211}%
\pgfsetlinewidth{1.003750pt}%
\definecolor{currentstroke}{rgb}{0.121569,0.466667,0.705882}%
\pgfsetstrokecolor{currentstroke}%
\pgfsetstrokeopacity{0.759211}%
\pgfsetdash{}{0pt}%
\pgfpathmoveto{\pgfqpoint{3.123330in}{2.426926in}}%
\pgfpathcurveto{\pgfqpoint{3.131566in}{2.426926in}}{\pgfqpoint{3.139466in}{2.430198in}}{\pgfqpoint{3.145290in}{2.436022in}}%
\pgfpathcurveto{\pgfqpoint{3.151114in}{2.441846in}}{\pgfqpoint{3.154386in}{2.449746in}}{\pgfqpoint{3.154386in}{2.457982in}}%
\pgfpathcurveto{\pgfqpoint{3.154386in}{2.466219in}}{\pgfqpoint{3.151114in}{2.474119in}}{\pgfqpoint{3.145290in}{2.479942in}}%
\pgfpathcurveto{\pgfqpoint{3.139466in}{2.485766in}}{\pgfqpoint{3.131566in}{2.489039in}}{\pgfqpoint{3.123330in}{2.489039in}}%
\pgfpathcurveto{\pgfqpoint{3.115093in}{2.489039in}}{\pgfqpoint{3.107193in}{2.485766in}}{\pgfqpoint{3.101369in}{2.479942in}}%
\pgfpathcurveto{\pgfqpoint{3.095546in}{2.474119in}}{\pgfqpoint{3.092273in}{2.466219in}}{\pgfqpoint{3.092273in}{2.457982in}}%
\pgfpathcurveto{\pgfqpoint{3.092273in}{2.449746in}}{\pgfqpoint{3.095546in}{2.441846in}}{\pgfqpoint{3.101369in}{2.436022in}}%
\pgfpathcurveto{\pgfqpoint{3.107193in}{2.430198in}}{\pgfqpoint{3.115093in}{2.426926in}}{\pgfqpoint{3.123330in}{2.426926in}}%
\pgfpathclose%
\pgfusepath{stroke,fill}%
\end{pgfscope}%
\begin{pgfscope}%
\pgfpathrectangle{\pgfqpoint{0.100000in}{0.220728in}}{\pgfqpoint{3.696000in}{3.696000in}}%
\pgfusepath{clip}%
\pgfsetbuttcap%
\pgfsetroundjoin%
\definecolor{currentfill}{rgb}{0.121569,0.466667,0.705882}%
\pgfsetfillcolor{currentfill}%
\pgfsetfillopacity{0.759311}%
\pgfsetlinewidth{1.003750pt}%
\definecolor{currentstroke}{rgb}{0.121569,0.466667,0.705882}%
\pgfsetstrokecolor{currentstroke}%
\pgfsetstrokeopacity{0.759311}%
\pgfsetdash{}{0pt}%
\pgfpathmoveto{\pgfqpoint{3.123002in}{2.426395in}}%
\pgfpathcurveto{\pgfqpoint{3.131238in}{2.426395in}}{\pgfqpoint{3.139138in}{2.429668in}}{\pgfqpoint{3.144962in}{2.435492in}}%
\pgfpathcurveto{\pgfqpoint{3.150786in}{2.441315in}}{\pgfqpoint{3.154058in}{2.449215in}}{\pgfqpoint{3.154058in}{2.457452in}}%
\pgfpathcurveto{\pgfqpoint{3.154058in}{2.465688in}}{\pgfqpoint{3.150786in}{2.473588in}}{\pgfqpoint{3.144962in}{2.479412in}}%
\pgfpathcurveto{\pgfqpoint{3.139138in}{2.485236in}}{\pgfqpoint{3.131238in}{2.488508in}}{\pgfqpoint{3.123002in}{2.488508in}}%
\pgfpathcurveto{\pgfqpoint{3.114765in}{2.488508in}}{\pgfqpoint{3.106865in}{2.485236in}}{\pgfqpoint{3.101042in}{2.479412in}}%
\pgfpathcurveto{\pgfqpoint{3.095218in}{2.473588in}}{\pgfqpoint{3.091945in}{2.465688in}}{\pgfqpoint{3.091945in}{2.457452in}}%
\pgfpathcurveto{\pgfqpoint{3.091945in}{2.449215in}}{\pgfqpoint{3.095218in}{2.441315in}}{\pgfqpoint{3.101042in}{2.435492in}}%
\pgfpathcurveto{\pgfqpoint{3.106865in}{2.429668in}}{\pgfqpoint{3.114765in}{2.426395in}}{\pgfqpoint{3.123002in}{2.426395in}}%
\pgfpathclose%
\pgfusepath{stroke,fill}%
\end{pgfscope}%
\begin{pgfscope}%
\pgfpathrectangle{\pgfqpoint{0.100000in}{0.220728in}}{\pgfqpoint{3.696000in}{3.696000in}}%
\pgfusepath{clip}%
\pgfsetbuttcap%
\pgfsetroundjoin%
\definecolor{currentfill}{rgb}{0.121569,0.466667,0.705882}%
\pgfsetfillcolor{currentfill}%
\pgfsetfillopacity{0.759654}%
\pgfsetlinewidth{1.003750pt}%
\definecolor{currentstroke}{rgb}{0.121569,0.466667,0.705882}%
\pgfsetstrokecolor{currentstroke}%
\pgfsetstrokeopacity{0.759654}%
\pgfsetdash{}{0pt}%
\pgfpathmoveto{\pgfqpoint{3.122327in}{2.424664in}}%
\pgfpathcurveto{\pgfqpoint{3.130563in}{2.424664in}}{\pgfqpoint{3.138463in}{2.427937in}}{\pgfqpoint{3.144287in}{2.433760in}}%
\pgfpathcurveto{\pgfqpoint{3.150111in}{2.439584in}}{\pgfqpoint{3.153384in}{2.447484in}}{\pgfqpoint{3.153384in}{2.455721in}}%
\pgfpathcurveto{\pgfqpoint{3.153384in}{2.463957in}}{\pgfqpoint{3.150111in}{2.471857in}}{\pgfqpoint{3.144287in}{2.477681in}}%
\pgfpathcurveto{\pgfqpoint{3.138463in}{2.483505in}}{\pgfqpoint{3.130563in}{2.486777in}}{\pgfqpoint{3.122327in}{2.486777in}}%
\pgfpathcurveto{\pgfqpoint{3.114091in}{2.486777in}}{\pgfqpoint{3.106191in}{2.483505in}}{\pgfqpoint{3.100367in}{2.477681in}}%
\pgfpathcurveto{\pgfqpoint{3.094543in}{2.471857in}}{\pgfqpoint{3.091271in}{2.463957in}}{\pgfqpoint{3.091271in}{2.455721in}}%
\pgfpathcurveto{\pgfqpoint{3.091271in}{2.447484in}}{\pgfqpoint{3.094543in}{2.439584in}}{\pgfqpoint{3.100367in}{2.433760in}}%
\pgfpathcurveto{\pgfqpoint{3.106191in}{2.427937in}}{\pgfqpoint{3.114091in}{2.424664in}}{\pgfqpoint{3.122327in}{2.424664in}}%
\pgfpathclose%
\pgfusepath{stroke,fill}%
\end{pgfscope}%
\begin{pgfscope}%
\pgfpathrectangle{\pgfqpoint{0.100000in}{0.220728in}}{\pgfqpoint{3.696000in}{3.696000in}}%
\pgfusepath{clip}%
\pgfsetbuttcap%
\pgfsetroundjoin%
\definecolor{currentfill}{rgb}{0.121569,0.466667,0.705882}%
\pgfsetfillcolor{currentfill}%
\pgfsetfillopacity{0.759842}%
\pgfsetlinewidth{1.003750pt}%
\definecolor{currentstroke}{rgb}{0.121569,0.466667,0.705882}%
\pgfsetstrokecolor{currentstroke}%
\pgfsetstrokeopacity{0.759842}%
\pgfsetdash{}{0pt}%
\pgfpathmoveto{\pgfqpoint{3.121910in}{2.423753in}}%
\pgfpathcurveto{\pgfqpoint{3.130146in}{2.423753in}}{\pgfqpoint{3.138046in}{2.427025in}}{\pgfqpoint{3.143870in}{2.432849in}}%
\pgfpathcurveto{\pgfqpoint{3.149694in}{2.438673in}}{\pgfqpoint{3.152966in}{2.446573in}}{\pgfqpoint{3.152966in}{2.454810in}}%
\pgfpathcurveto{\pgfqpoint{3.152966in}{2.463046in}}{\pgfqpoint{3.149694in}{2.470946in}}{\pgfqpoint{3.143870in}{2.476770in}}%
\pgfpathcurveto{\pgfqpoint{3.138046in}{2.482594in}}{\pgfqpoint{3.130146in}{2.485866in}}{\pgfqpoint{3.121910in}{2.485866in}}%
\pgfpathcurveto{\pgfqpoint{3.113674in}{2.485866in}}{\pgfqpoint{3.105774in}{2.482594in}}{\pgfqpoint{3.099950in}{2.476770in}}%
\pgfpathcurveto{\pgfqpoint{3.094126in}{2.470946in}}{\pgfqpoint{3.090853in}{2.463046in}}{\pgfqpoint{3.090853in}{2.454810in}}%
\pgfpathcurveto{\pgfqpoint{3.090853in}{2.446573in}}{\pgfqpoint{3.094126in}{2.438673in}}{\pgfqpoint{3.099950in}{2.432849in}}%
\pgfpathcurveto{\pgfqpoint{3.105774in}{2.427025in}}{\pgfqpoint{3.113674in}{2.423753in}}{\pgfqpoint{3.121910in}{2.423753in}}%
\pgfpathclose%
\pgfusepath{stroke,fill}%
\end{pgfscope}%
\begin{pgfscope}%
\pgfpathrectangle{\pgfqpoint{0.100000in}{0.220728in}}{\pgfqpoint{3.696000in}{3.696000in}}%
\pgfusepath{clip}%
\pgfsetbuttcap%
\pgfsetroundjoin%
\definecolor{currentfill}{rgb}{0.121569,0.466667,0.705882}%
\pgfsetfillcolor{currentfill}%
\pgfsetfillopacity{0.759928}%
\pgfsetlinewidth{1.003750pt}%
\definecolor{currentstroke}{rgb}{0.121569,0.466667,0.705882}%
\pgfsetstrokecolor{currentstroke}%
\pgfsetstrokeopacity{0.759928}%
\pgfsetdash{}{0pt}%
\pgfpathmoveto{\pgfqpoint{3.121600in}{2.423287in}}%
\pgfpathcurveto{\pgfqpoint{3.129836in}{2.423287in}}{\pgfqpoint{3.137736in}{2.426559in}}{\pgfqpoint{3.143560in}{2.432383in}}%
\pgfpathcurveto{\pgfqpoint{3.149384in}{2.438207in}}{\pgfqpoint{3.152656in}{2.446107in}}{\pgfqpoint{3.152656in}{2.454343in}}%
\pgfpathcurveto{\pgfqpoint{3.152656in}{2.462580in}}{\pgfqpoint{3.149384in}{2.470480in}}{\pgfqpoint{3.143560in}{2.476304in}}%
\pgfpathcurveto{\pgfqpoint{3.137736in}{2.482127in}}{\pgfqpoint{3.129836in}{2.485400in}}{\pgfqpoint{3.121600in}{2.485400in}}%
\pgfpathcurveto{\pgfqpoint{3.113363in}{2.485400in}}{\pgfqpoint{3.105463in}{2.482127in}}{\pgfqpoint{3.099639in}{2.476304in}}%
\pgfpathcurveto{\pgfqpoint{3.093815in}{2.470480in}}{\pgfqpoint{3.090543in}{2.462580in}}{\pgfqpoint{3.090543in}{2.454343in}}%
\pgfpathcurveto{\pgfqpoint{3.090543in}{2.446107in}}{\pgfqpoint{3.093815in}{2.438207in}}{\pgfqpoint{3.099639in}{2.432383in}}%
\pgfpathcurveto{\pgfqpoint{3.105463in}{2.426559in}}{\pgfqpoint{3.113363in}{2.423287in}}{\pgfqpoint{3.121600in}{2.423287in}}%
\pgfpathclose%
\pgfusepath{stroke,fill}%
\end{pgfscope}%
\begin{pgfscope}%
\pgfpathrectangle{\pgfqpoint{0.100000in}{0.220728in}}{\pgfqpoint{3.696000in}{3.696000in}}%
\pgfusepath{clip}%
\pgfsetbuttcap%
\pgfsetroundjoin%
\definecolor{currentfill}{rgb}{0.121569,0.466667,0.705882}%
\pgfsetfillcolor{currentfill}%
\pgfsetfillopacity{0.760308}%
\pgfsetlinewidth{1.003750pt}%
\definecolor{currentstroke}{rgb}{0.121569,0.466667,0.705882}%
\pgfsetstrokecolor{currentstroke}%
\pgfsetstrokeopacity{0.760308}%
\pgfsetdash{}{0pt}%
\pgfpathmoveto{\pgfqpoint{3.120862in}{2.421194in}}%
\pgfpathcurveto{\pgfqpoint{3.129098in}{2.421194in}}{\pgfqpoint{3.136998in}{2.424466in}}{\pgfqpoint{3.142822in}{2.430290in}}%
\pgfpathcurveto{\pgfqpoint{3.148646in}{2.436114in}}{\pgfqpoint{3.151918in}{2.444014in}}{\pgfqpoint{3.151918in}{2.452250in}}%
\pgfpathcurveto{\pgfqpoint{3.151918in}{2.460486in}}{\pgfqpoint{3.148646in}{2.468386in}}{\pgfqpoint{3.142822in}{2.474210in}}%
\pgfpathcurveto{\pgfqpoint{3.136998in}{2.480034in}}{\pgfqpoint{3.129098in}{2.483307in}}{\pgfqpoint{3.120862in}{2.483307in}}%
\pgfpathcurveto{\pgfqpoint{3.112625in}{2.483307in}}{\pgfqpoint{3.104725in}{2.480034in}}{\pgfqpoint{3.098901in}{2.474210in}}%
\pgfpathcurveto{\pgfqpoint{3.093077in}{2.468386in}}{\pgfqpoint{3.089805in}{2.460486in}}{\pgfqpoint{3.089805in}{2.452250in}}%
\pgfpathcurveto{\pgfqpoint{3.089805in}{2.444014in}}{\pgfqpoint{3.093077in}{2.436114in}}{\pgfqpoint{3.098901in}{2.430290in}}%
\pgfpathcurveto{\pgfqpoint{3.104725in}{2.424466in}}{\pgfqpoint{3.112625in}{2.421194in}}{\pgfqpoint{3.120862in}{2.421194in}}%
\pgfpathclose%
\pgfusepath{stroke,fill}%
\end{pgfscope}%
\begin{pgfscope}%
\pgfpathrectangle{\pgfqpoint{0.100000in}{0.220728in}}{\pgfqpoint{3.696000in}{3.696000in}}%
\pgfusepath{clip}%
\pgfsetbuttcap%
\pgfsetroundjoin%
\definecolor{currentfill}{rgb}{0.121569,0.466667,0.705882}%
\pgfsetfillcolor{currentfill}%
\pgfsetfillopacity{0.760851}%
\pgfsetlinewidth{1.003750pt}%
\definecolor{currentstroke}{rgb}{0.121569,0.466667,0.705882}%
\pgfsetstrokecolor{currentstroke}%
\pgfsetstrokeopacity{0.760851}%
\pgfsetdash{}{0pt}%
\pgfpathmoveto{\pgfqpoint{3.119637in}{2.418372in}}%
\pgfpathcurveto{\pgfqpoint{3.127873in}{2.418372in}}{\pgfqpoint{3.135773in}{2.421645in}}{\pgfqpoint{3.141597in}{2.427469in}}%
\pgfpathcurveto{\pgfqpoint{3.147421in}{2.433293in}}{\pgfqpoint{3.150694in}{2.441193in}}{\pgfqpoint{3.150694in}{2.449429in}}%
\pgfpathcurveto{\pgfqpoint{3.150694in}{2.457665in}}{\pgfqpoint{3.147421in}{2.465565in}}{\pgfqpoint{3.141597in}{2.471389in}}%
\pgfpathcurveto{\pgfqpoint{3.135773in}{2.477213in}}{\pgfqpoint{3.127873in}{2.480485in}}{\pgfqpoint{3.119637in}{2.480485in}}%
\pgfpathcurveto{\pgfqpoint{3.111401in}{2.480485in}}{\pgfqpoint{3.103501in}{2.477213in}}{\pgfqpoint{3.097677in}{2.471389in}}%
\pgfpathcurveto{\pgfqpoint{3.091853in}{2.465565in}}{\pgfqpoint{3.088581in}{2.457665in}}{\pgfqpoint{3.088581in}{2.449429in}}%
\pgfpathcurveto{\pgfqpoint{3.088581in}{2.441193in}}{\pgfqpoint{3.091853in}{2.433293in}}{\pgfqpoint{3.097677in}{2.427469in}}%
\pgfpathcurveto{\pgfqpoint{3.103501in}{2.421645in}}{\pgfqpoint{3.111401in}{2.418372in}}{\pgfqpoint{3.119637in}{2.418372in}}%
\pgfpathclose%
\pgfusepath{stroke,fill}%
\end{pgfscope}%
\begin{pgfscope}%
\pgfpathrectangle{\pgfqpoint{0.100000in}{0.220728in}}{\pgfqpoint{3.696000in}{3.696000in}}%
\pgfusepath{clip}%
\pgfsetbuttcap%
\pgfsetroundjoin%
\definecolor{currentfill}{rgb}{0.121569,0.466667,0.705882}%
\pgfsetfillcolor{currentfill}%
\pgfsetfillopacity{0.761477}%
\pgfsetlinewidth{1.003750pt}%
\definecolor{currentstroke}{rgb}{0.121569,0.466667,0.705882}%
\pgfsetstrokecolor{currentstroke}%
\pgfsetstrokeopacity{0.761477}%
\pgfsetdash{}{0pt}%
\pgfpathmoveto{\pgfqpoint{3.117592in}{2.415243in}}%
\pgfpathcurveto{\pgfqpoint{3.125828in}{2.415243in}}{\pgfqpoint{3.133729in}{2.418516in}}{\pgfqpoint{3.139552in}{2.424340in}}%
\pgfpathcurveto{\pgfqpoint{3.145376in}{2.430164in}}{\pgfqpoint{3.148649in}{2.438064in}}{\pgfqpoint{3.148649in}{2.446300in}}%
\pgfpathcurveto{\pgfqpoint{3.148649in}{2.454536in}}{\pgfqpoint{3.145376in}{2.462436in}}{\pgfqpoint{3.139552in}{2.468260in}}%
\pgfpathcurveto{\pgfqpoint{3.133729in}{2.474084in}}{\pgfqpoint{3.125828in}{2.477356in}}{\pgfqpoint{3.117592in}{2.477356in}}%
\pgfpathcurveto{\pgfqpoint{3.109356in}{2.477356in}}{\pgfqpoint{3.101456in}{2.474084in}}{\pgfqpoint{3.095632in}{2.468260in}}%
\pgfpathcurveto{\pgfqpoint{3.089808in}{2.462436in}}{\pgfqpoint{3.086536in}{2.454536in}}{\pgfqpoint{3.086536in}{2.446300in}}%
\pgfpathcurveto{\pgfqpoint{3.086536in}{2.438064in}}{\pgfqpoint{3.089808in}{2.430164in}}{\pgfqpoint{3.095632in}{2.424340in}}%
\pgfpathcurveto{\pgfqpoint{3.101456in}{2.418516in}}{\pgfqpoint{3.109356in}{2.415243in}}{\pgfqpoint{3.117592in}{2.415243in}}%
\pgfpathclose%
\pgfusepath{stroke,fill}%
\end{pgfscope}%
\begin{pgfscope}%
\pgfpathrectangle{\pgfqpoint{0.100000in}{0.220728in}}{\pgfqpoint{3.696000in}{3.696000in}}%
\pgfusepath{clip}%
\pgfsetbuttcap%
\pgfsetroundjoin%
\definecolor{currentfill}{rgb}{0.121569,0.466667,0.705882}%
\pgfsetfillcolor{currentfill}%
\pgfsetfillopacity{0.761584}%
\pgfsetlinewidth{1.003750pt}%
\definecolor{currentstroke}{rgb}{0.121569,0.466667,0.705882}%
\pgfsetstrokecolor{currentstroke}%
\pgfsetstrokeopacity{0.761584}%
\pgfsetdash{}{0pt}%
\pgfpathmoveto{\pgfqpoint{1.147272in}{1.199057in}}%
\pgfpathcurveto{\pgfqpoint{1.155508in}{1.199057in}}{\pgfqpoint{1.163408in}{1.202329in}}{\pgfqpoint{1.169232in}{1.208153in}}%
\pgfpathcurveto{\pgfqpoint{1.175056in}{1.213977in}}{\pgfqpoint{1.178329in}{1.221877in}}{\pgfqpoint{1.178329in}{1.230113in}}%
\pgfpathcurveto{\pgfqpoint{1.178329in}{1.238350in}}{\pgfqpoint{1.175056in}{1.246250in}}{\pgfqpoint{1.169232in}{1.252074in}}%
\pgfpathcurveto{\pgfqpoint{1.163408in}{1.257898in}}{\pgfqpoint{1.155508in}{1.261170in}}{\pgfqpoint{1.147272in}{1.261170in}}%
\pgfpathcurveto{\pgfqpoint{1.139036in}{1.261170in}}{\pgfqpoint{1.131136in}{1.257898in}}{\pgfqpoint{1.125312in}{1.252074in}}%
\pgfpathcurveto{\pgfqpoint{1.119488in}{1.246250in}}{\pgfqpoint{1.116216in}{1.238350in}}{\pgfqpoint{1.116216in}{1.230113in}}%
\pgfpathcurveto{\pgfqpoint{1.116216in}{1.221877in}}{\pgfqpoint{1.119488in}{1.213977in}}{\pgfqpoint{1.125312in}{1.208153in}}%
\pgfpathcurveto{\pgfqpoint{1.131136in}{1.202329in}}{\pgfqpoint{1.139036in}{1.199057in}}{\pgfqpoint{1.147272in}{1.199057in}}%
\pgfpathclose%
\pgfusepath{stroke,fill}%
\end{pgfscope}%
\begin{pgfscope}%
\pgfpathrectangle{\pgfqpoint{0.100000in}{0.220728in}}{\pgfqpoint{3.696000in}{3.696000in}}%
\pgfusepath{clip}%
\pgfsetbuttcap%
\pgfsetroundjoin%
\definecolor{currentfill}{rgb}{0.121569,0.466667,0.705882}%
\pgfsetfillcolor{currentfill}%
\pgfsetfillopacity{0.762564}%
\pgfsetlinewidth{1.003750pt}%
\definecolor{currentstroke}{rgb}{0.121569,0.466667,0.705882}%
\pgfsetstrokecolor{currentstroke}%
\pgfsetstrokeopacity{0.762564}%
\pgfsetdash{}{0pt}%
\pgfpathmoveto{\pgfqpoint{3.115687in}{2.408797in}}%
\pgfpathcurveto{\pgfqpoint{3.123923in}{2.408797in}}{\pgfqpoint{3.131823in}{2.412069in}}{\pgfqpoint{3.137647in}{2.417893in}}%
\pgfpathcurveto{\pgfqpoint{3.143471in}{2.423717in}}{\pgfqpoint{3.146743in}{2.431617in}}{\pgfqpoint{3.146743in}{2.439854in}}%
\pgfpathcurveto{\pgfqpoint{3.146743in}{2.448090in}}{\pgfqpoint{3.143471in}{2.455990in}}{\pgfqpoint{3.137647in}{2.461814in}}%
\pgfpathcurveto{\pgfqpoint{3.131823in}{2.467638in}}{\pgfqpoint{3.123923in}{2.470910in}}{\pgfqpoint{3.115687in}{2.470910in}}%
\pgfpathcurveto{\pgfqpoint{3.107450in}{2.470910in}}{\pgfqpoint{3.099550in}{2.467638in}}{\pgfqpoint{3.093726in}{2.461814in}}%
\pgfpathcurveto{\pgfqpoint{3.087902in}{2.455990in}}{\pgfqpoint{3.084630in}{2.448090in}}{\pgfqpoint{3.084630in}{2.439854in}}%
\pgfpathcurveto{\pgfqpoint{3.084630in}{2.431617in}}{\pgfqpoint{3.087902in}{2.423717in}}{\pgfqpoint{3.093726in}{2.417893in}}%
\pgfpathcurveto{\pgfqpoint{3.099550in}{2.412069in}}{\pgfqpoint{3.107450in}{2.408797in}}{\pgfqpoint{3.115687in}{2.408797in}}%
\pgfpathclose%
\pgfusepath{stroke,fill}%
\end{pgfscope}%
\begin{pgfscope}%
\pgfpathrectangle{\pgfqpoint{0.100000in}{0.220728in}}{\pgfqpoint{3.696000in}{3.696000in}}%
\pgfusepath{clip}%
\pgfsetbuttcap%
\pgfsetroundjoin%
\definecolor{currentfill}{rgb}{0.121569,0.466667,0.705882}%
\pgfsetfillcolor{currentfill}%
\pgfsetfillopacity{0.763150}%
\pgfsetlinewidth{1.003750pt}%
\definecolor{currentstroke}{rgb}{0.121569,0.466667,0.705882}%
\pgfsetstrokecolor{currentstroke}%
\pgfsetstrokeopacity{0.763150}%
\pgfsetdash{}{0pt}%
\pgfpathmoveto{\pgfqpoint{3.114050in}{2.405768in}}%
\pgfpathcurveto{\pgfqpoint{3.122287in}{2.405768in}}{\pgfqpoint{3.130187in}{2.409041in}}{\pgfqpoint{3.136011in}{2.414865in}}%
\pgfpathcurveto{\pgfqpoint{3.141835in}{2.420689in}}{\pgfqpoint{3.145107in}{2.428589in}}{\pgfqpoint{3.145107in}{2.436825in}}%
\pgfpathcurveto{\pgfqpoint{3.145107in}{2.445061in}}{\pgfqpoint{3.141835in}{2.452961in}}{\pgfqpoint{3.136011in}{2.458785in}}%
\pgfpathcurveto{\pgfqpoint{3.130187in}{2.464609in}}{\pgfqpoint{3.122287in}{2.467881in}}{\pgfqpoint{3.114050in}{2.467881in}}%
\pgfpathcurveto{\pgfqpoint{3.105814in}{2.467881in}}{\pgfqpoint{3.097914in}{2.464609in}}{\pgfqpoint{3.092090in}{2.458785in}}%
\pgfpathcurveto{\pgfqpoint{3.086266in}{2.452961in}}{\pgfqpoint{3.082994in}{2.445061in}}{\pgfqpoint{3.082994in}{2.436825in}}%
\pgfpathcurveto{\pgfqpoint{3.082994in}{2.428589in}}{\pgfqpoint{3.086266in}{2.420689in}}{\pgfqpoint{3.092090in}{2.414865in}}%
\pgfpathcurveto{\pgfqpoint{3.097914in}{2.409041in}}{\pgfqpoint{3.105814in}{2.405768in}}{\pgfqpoint{3.114050in}{2.405768in}}%
\pgfpathclose%
\pgfusepath{stroke,fill}%
\end{pgfscope}%
\begin{pgfscope}%
\pgfpathrectangle{\pgfqpoint{0.100000in}{0.220728in}}{\pgfqpoint{3.696000in}{3.696000in}}%
\pgfusepath{clip}%
\pgfsetbuttcap%
\pgfsetroundjoin%
\definecolor{currentfill}{rgb}{0.121569,0.466667,0.705882}%
\pgfsetfillcolor{currentfill}%
\pgfsetfillopacity{0.763437}%
\pgfsetlinewidth{1.003750pt}%
\definecolor{currentstroke}{rgb}{0.121569,0.466667,0.705882}%
\pgfsetstrokecolor{currentstroke}%
\pgfsetstrokeopacity{0.763437}%
\pgfsetdash{}{0pt}%
\pgfpathmoveto{\pgfqpoint{3.113028in}{2.404124in}}%
\pgfpathcurveto{\pgfqpoint{3.121264in}{2.404124in}}{\pgfqpoint{3.129164in}{2.407396in}}{\pgfqpoint{3.134988in}{2.413220in}}%
\pgfpathcurveto{\pgfqpoint{3.140812in}{2.419044in}}{\pgfqpoint{3.144084in}{2.426944in}}{\pgfqpoint{3.144084in}{2.435180in}}%
\pgfpathcurveto{\pgfqpoint{3.144084in}{2.443416in}}{\pgfqpoint{3.140812in}{2.451316in}}{\pgfqpoint{3.134988in}{2.457140in}}%
\pgfpathcurveto{\pgfqpoint{3.129164in}{2.462964in}}{\pgfqpoint{3.121264in}{2.466237in}}{\pgfqpoint{3.113028in}{2.466237in}}%
\pgfpathcurveto{\pgfqpoint{3.104792in}{2.466237in}}{\pgfqpoint{3.096892in}{2.462964in}}{\pgfqpoint{3.091068in}{2.457140in}}%
\pgfpathcurveto{\pgfqpoint{3.085244in}{2.451316in}}{\pgfqpoint{3.081971in}{2.443416in}}{\pgfqpoint{3.081971in}{2.435180in}}%
\pgfpathcurveto{\pgfqpoint{3.081971in}{2.426944in}}{\pgfqpoint{3.085244in}{2.419044in}}{\pgfqpoint{3.091068in}{2.413220in}}%
\pgfpathcurveto{\pgfqpoint{3.096892in}{2.407396in}}{\pgfqpoint{3.104792in}{2.404124in}}{\pgfqpoint{3.113028in}{2.404124in}}%
\pgfpathclose%
\pgfusepath{stroke,fill}%
\end{pgfscope}%
\begin{pgfscope}%
\pgfpathrectangle{\pgfqpoint{0.100000in}{0.220728in}}{\pgfqpoint{3.696000in}{3.696000in}}%
\pgfusepath{clip}%
\pgfsetbuttcap%
\pgfsetroundjoin%
\definecolor{currentfill}{rgb}{0.121569,0.466667,0.705882}%
\pgfsetfillcolor{currentfill}%
\pgfsetfillopacity{0.763622}%
\pgfsetlinewidth{1.003750pt}%
\definecolor{currentstroke}{rgb}{0.121569,0.466667,0.705882}%
\pgfsetstrokecolor{currentstroke}%
\pgfsetstrokeopacity{0.763622}%
\pgfsetdash{}{0pt}%
\pgfpathmoveto{\pgfqpoint{3.112755in}{2.403042in}}%
\pgfpathcurveto{\pgfqpoint{3.120991in}{2.403042in}}{\pgfqpoint{3.128891in}{2.406314in}}{\pgfqpoint{3.134715in}{2.412138in}}%
\pgfpathcurveto{\pgfqpoint{3.140539in}{2.417962in}}{\pgfqpoint{3.143811in}{2.425862in}}{\pgfqpoint{3.143811in}{2.434098in}}%
\pgfpathcurveto{\pgfqpoint{3.143811in}{2.442334in}}{\pgfqpoint{3.140539in}{2.450234in}}{\pgfqpoint{3.134715in}{2.456058in}}%
\pgfpathcurveto{\pgfqpoint{3.128891in}{2.461882in}}{\pgfqpoint{3.120991in}{2.465155in}}{\pgfqpoint{3.112755in}{2.465155in}}%
\pgfpathcurveto{\pgfqpoint{3.104519in}{2.465155in}}{\pgfqpoint{3.096619in}{2.461882in}}{\pgfqpoint{3.090795in}{2.456058in}}%
\pgfpathcurveto{\pgfqpoint{3.084971in}{2.450234in}}{\pgfqpoint{3.081698in}{2.442334in}}{\pgfqpoint{3.081698in}{2.434098in}}%
\pgfpathcurveto{\pgfqpoint{3.081698in}{2.425862in}}{\pgfqpoint{3.084971in}{2.417962in}}{\pgfqpoint{3.090795in}{2.412138in}}%
\pgfpathcurveto{\pgfqpoint{3.096619in}{2.406314in}}{\pgfqpoint{3.104519in}{2.403042in}}{\pgfqpoint{3.112755in}{2.403042in}}%
\pgfpathclose%
\pgfusepath{stroke,fill}%
\end{pgfscope}%
\begin{pgfscope}%
\pgfpathrectangle{\pgfqpoint{0.100000in}{0.220728in}}{\pgfqpoint{3.696000in}{3.696000in}}%
\pgfusepath{clip}%
\pgfsetbuttcap%
\pgfsetroundjoin%
\definecolor{currentfill}{rgb}{0.121569,0.466667,0.705882}%
\pgfsetfillcolor{currentfill}%
\pgfsetfillopacity{0.763978}%
\pgfsetlinewidth{1.003750pt}%
\definecolor{currentstroke}{rgb}{0.121569,0.466667,0.705882}%
\pgfsetstrokecolor{currentstroke}%
\pgfsetstrokeopacity{0.763978}%
\pgfsetdash{}{0pt}%
\pgfpathmoveto{\pgfqpoint{3.111629in}{2.401079in}}%
\pgfpathcurveto{\pgfqpoint{3.119866in}{2.401079in}}{\pgfqpoint{3.127766in}{2.404351in}}{\pgfqpoint{3.133590in}{2.410175in}}%
\pgfpathcurveto{\pgfqpoint{3.139413in}{2.415999in}}{\pgfqpoint{3.142686in}{2.423899in}}{\pgfqpoint{3.142686in}{2.432135in}}%
\pgfpathcurveto{\pgfqpoint{3.142686in}{2.440372in}}{\pgfqpoint{3.139413in}{2.448272in}}{\pgfqpoint{3.133590in}{2.454096in}}%
\pgfpathcurveto{\pgfqpoint{3.127766in}{2.459920in}}{\pgfqpoint{3.119866in}{2.463192in}}{\pgfqpoint{3.111629in}{2.463192in}}%
\pgfpathcurveto{\pgfqpoint{3.103393in}{2.463192in}}{\pgfqpoint{3.095493in}{2.459920in}}{\pgfqpoint{3.089669in}{2.454096in}}%
\pgfpathcurveto{\pgfqpoint{3.083845in}{2.448272in}}{\pgfqpoint{3.080573in}{2.440372in}}{\pgfqpoint{3.080573in}{2.432135in}}%
\pgfpathcurveto{\pgfqpoint{3.080573in}{2.423899in}}{\pgfqpoint{3.083845in}{2.415999in}}{\pgfqpoint{3.089669in}{2.410175in}}%
\pgfpathcurveto{\pgfqpoint{3.095493in}{2.404351in}}{\pgfqpoint{3.103393in}{2.401079in}}{\pgfqpoint{3.111629in}{2.401079in}}%
\pgfpathclose%
\pgfusepath{stroke,fill}%
\end{pgfscope}%
\begin{pgfscope}%
\pgfpathrectangle{\pgfqpoint{0.100000in}{0.220728in}}{\pgfqpoint{3.696000in}{3.696000in}}%
\pgfusepath{clip}%
\pgfsetbuttcap%
\pgfsetroundjoin%
\definecolor{currentfill}{rgb}{0.121569,0.466667,0.705882}%
\pgfsetfillcolor{currentfill}%
\pgfsetfillopacity{0.764205}%
\pgfsetlinewidth{1.003750pt}%
\definecolor{currentstroke}{rgb}{0.121569,0.466667,0.705882}%
\pgfsetstrokecolor{currentstroke}%
\pgfsetstrokeopacity{0.764205}%
\pgfsetdash{}{0pt}%
\pgfpathmoveto{\pgfqpoint{3.111036in}{2.400098in}}%
\pgfpathcurveto{\pgfqpoint{3.119273in}{2.400098in}}{\pgfqpoint{3.127173in}{2.403371in}}{\pgfqpoint{3.132997in}{2.409195in}}%
\pgfpathcurveto{\pgfqpoint{3.138821in}{2.415019in}}{\pgfqpoint{3.142093in}{2.422919in}}{\pgfqpoint{3.142093in}{2.431155in}}%
\pgfpathcurveto{\pgfqpoint{3.142093in}{2.439391in}}{\pgfqpoint{3.138821in}{2.447291in}}{\pgfqpoint{3.132997in}{2.453115in}}%
\pgfpathcurveto{\pgfqpoint{3.127173in}{2.458939in}}{\pgfqpoint{3.119273in}{2.462211in}}{\pgfqpoint{3.111036in}{2.462211in}}%
\pgfpathcurveto{\pgfqpoint{3.102800in}{2.462211in}}{\pgfqpoint{3.094900in}{2.458939in}}{\pgfqpoint{3.089076in}{2.453115in}}%
\pgfpathcurveto{\pgfqpoint{3.083252in}{2.447291in}}{\pgfqpoint{3.079980in}{2.439391in}}{\pgfqpoint{3.079980in}{2.431155in}}%
\pgfpathcurveto{\pgfqpoint{3.079980in}{2.422919in}}{\pgfqpoint{3.083252in}{2.415019in}}{\pgfqpoint{3.089076in}{2.409195in}}%
\pgfpathcurveto{\pgfqpoint{3.094900in}{2.403371in}}{\pgfqpoint{3.102800in}{2.400098in}}{\pgfqpoint{3.111036in}{2.400098in}}%
\pgfpathclose%
\pgfusepath{stroke,fill}%
\end{pgfscope}%
\begin{pgfscope}%
\pgfpathrectangle{\pgfqpoint{0.100000in}{0.220728in}}{\pgfqpoint{3.696000in}{3.696000in}}%
\pgfusepath{clip}%
\pgfsetbuttcap%
\pgfsetroundjoin%
\definecolor{currentfill}{rgb}{0.121569,0.466667,0.705882}%
\pgfsetfillcolor{currentfill}%
\pgfsetfillopacity{0.764323}%
\pgfsetlinewidth{1.003750pt}%
\definecolor{currentstroke}{rgb}{0.121569,0.466667,0.705882}%
\pgfsetstrokecolor{currentstroke}%
\pgfsetstrokeopacity{0.764323}%
\pgfsetdash{}{0pt}%
\pgfpathmoveto{\pgfqpoint{3.110833in}{2.399410in}}%
\pgfpathcurveto{\pgfqpoint{3.119069in}{2.399410in}}{\pgfqpoint{3.126969in}{2.402682in}}{\pgfqpoint{3.132793in}{2.408506in}}%
\pgfpathcurveto{\pgfqpoint{3.138617in}{2.414330in}}{\pgfqpoint{3.141890in}{2.422230in}}{\pgfqpoint{3.141890in}{2.430467in}}%
\pgfpathcurveto{\pgfqpoint{3.141890in}{2.438703in}}{\pgfqpoint{3.138617in}{2.446603in}}{\pgfqpoint{3.132793in}{2.452427in}}%
\pgfpathcurveto{\pgfqpoint{3.126969in}{2.458251in}}{\pgfqpoint{3.119069in}{2.461523in}}{\pgfqpoint{3.110833in}{2.461523in}}%
\pgfpathcurveto{\pgfqpoint{3.102597in}{2.461523in}}{\pgfqpoint{3.094697in}{2.458251in}}{\pgfqpoint{3.088873in}{2.452427in}}%
\pgfpathcurveto{\pgfqpoint{3.083049in}{2.446603in}}{\pgfqpoint{3.079777in}{2.438703in}}{\pgfqpoint{3.079777in}{2.430467in}}%
\pgfpathcurveto{\pgfqpoint{3.079777in}{2.422230in}}{\pgfqpoint{3.083049in}{2.414330in}}{\pgfqpoint{3.088873in}{2.408506in}}%
\pgfpathcurveto{\pgfqpoint{3.094697in}{2.402682in}}{\pgfqpoint{3.102597in}{2.399410in}}{\pgfqpoint{3.110833in}{2.399410in}}%
\pgfpathclose%
\pgfusepath{stroke,fill}%
\end{pgfscope}%
\begin{pgfscope}%
\pgfpathrectangle{\pgfqpoint{0.100000in}{0.220728in}}{\pgfqpoint{3.696000in}{3.696000in}}%
\pgfusepath{clip}%
\pgfsetbuttcap%
\pgfsetroundjoin%
\definecolor{currentfill}{rgb}{0.121569,0.466667,0.705882}%
\pgfsetfillcolor{currentfill}%
\pgfsetfillopacity{0.764750}%
\pgfsetlinewidth{1.003750pt}%
\definecolor{currentstroke}{rgb}{0.121569,0.466667,0.705882}%
\pgfsetstrokecolor{currentstroke}%
\pgfsetstrokeopacity{0.764750}%
\pgfsetdash{}{0pt}%
\pgfpathmoveto{\pgfqpoint{3.109484in}{2.397190in}}%
\pgfpathcurveto{\pgfqpoint{3.117720in}{2.397190in}}{\pgfqpoint{3.125621in}{2.400463in}}{\pgfqpoint{3.131444in}{2.406287in}}%
\pgfpathcurveto{\pgfqpoint{3.137268in}{2.412111in}}{\pgfqpoint{3.140541in}{2.420011in}}{\pgfqpoint{3.140541in}{2.428247in}}%
\pgfpathcurveto{\pgfqpoint{3.140541in}{2.436483in}}{\pgfqpoint{3.137268in}{2.444383in}}{\pgfqpoint{3.131444in}{2.450207in}}%
\pgfpathcurveto{\pgfqpoint{3.125621in}{2.456031in}}{\pgfqpoint{3.117720in}{2.459303in}}{\pgfqpoint{3.109484in}{2.459303in}}%
\pgfpathcurveto{\pgfqpoint{3.101248in}{2.459303in}}{\pgfqpoint{3.093348in}{2.456031in}}{\pgfqpoint{3.087524in}{2.450207in}}%
\pgfpathcurveto{\pgfqpoint{3.081700in}{2.444383in}}{\pgfqpoint{3.078428in}{2.436483in}}{\pgfqpoint{3.078428in}{2.428247in}}%
\pgfpathcurveto{\pgfqpoint{3.078428in}{2.420011in}}{\pgfqpoint{3.081700in}{2.412111in}}{\pgfqpoint{3.087524in}{2.406287in}}%
\pgfpathcurveto{\pgfqpoint{3.093348in}{2.400463in}}{\pgfqpoint{3.101248in}{2.397190in}}{\pgfqpoint{3.109484in}{2.397190in}}%
\pgfpathclose%
\pgfusepath{stroke,fill}%
\end{pgfscope}%
\begin{pgfscope}%
\pgfpathrectangle{\pgfqpoint{0.100000in}{0.220728in}}{\pgfqpoint{3.696000in}{3.696000in}}%
\pgfusepath{clip}%
\pgfsetbuttcap%
\pgfsetroundjoin%
\definecolor{currentfill}{rgb}{0.121569,0.466667,0.705882}%
\pgfsetfillcolor{currentfill}%
\pgfsetfillopacity{0.764767}%
\pgfsetlinewidth{1.003750pt}%
\definecolor{currentstroke}{rgb}{0.121569,0.466667,0.705882}%
\pgfsetstrokecolor{currentstroke}%
\pgfsetstrokeopacity{0.764767}%
\pgfsetdash{}{0pt}%
\pgfpathmoveto{\pgfqpoint{1.159641in}{1.196253in}}%
\pgfpathcurveto{\pgfqpoint{1.167877in}{1.196253in}}{\pgfqpoint{1.175778in}{1.199525in}}{\pgfqpoint{1.181601in}{1.205349in}}%
\pgfpathcurveto{\pgfqpoint{1.187425in}{1.211173in}}{\pgfqpoint{1.190698in}{1.219073in}}{\pgfqpoint{1.190698in}{1.227309in}}%
\pgfpathcurveto{\pgfqpoint{1.190698in}{1.235545in}}{\pgfqpoint{1.187425in}{1.243445in}}{\pgfqpoint{1.181601in}{1.249269in}}%
\pgfpathcurveto{\pgfqpoint{1.175778in}{1.255093in}}{\pgfqpoint{1.167877in}{1.258366in}}{\pgfqpoint{1.159641in}{1.258366in}}%
\pgfpathcurveto{\pgfqpoint{1.151405in}{1.258366in}}{\pgfqpoint{1.143505in}{1.255093in}}{\pgfqpoint{1.137681in}{1.249269in}}%
\pgfpathcurveto{\pgfqpoint{1.131857in}{1.243445in}}{\pgfqpoint{1.128585in}{1.235545in}}{\pgfqpoint{1.128585in}{1.227309in}}%
\pgfpathcurveto{\pgfqpoint{1.128585in}{1.219073in}}{\pgfqpoint{1.131857in}{1.211173in}}{\pgfqpoint{1.137681in}{1.205349in}}%
\pgfpathcurveto{\pgfqpoint{1.143505in}{1.199525in}}{\pgfqpoint{1.151405in}{1.196253in}}{\pgfqpoint{1.159641in}{1.196253in}}%
\pgfpathclose%
\pgfusepath{stroke,fill}%
\end{pgfscope}%
\begin{pgfscope}%
\pgfpathrectangle{\pgfqpoint{0.100000in}{0.220728in}}{\pgfqpoint{3.696000in}{3.696000in}}%
\pgfusepath{clip}%
\pgfsetbuttcap%
\pgfsetroundjoin%
\definecolor{currentfill}{rgb}{0.121569,0.466667,0.705882}%
\pgfsetfillcolor{currentfill}%
\pgfsetfillopacity{0.765032}%
\pgfsetlinewidth{1.003750pt}%
\definecolor{currentstroke}{rgb}{0.121569,0.466667,0.705882}%
\pgfsetstrokecolor{currentstroke}%
\pgfsetstrokeopacity{0.765032}%
\pgfsetdash{}{0pt}%
\pgfpathmoveto{\pgfqpoint{3.108912in}{2.395960in}}%
\pgfpathcurveto{\pgfqpoint{3.117148in}{2.395960in}}{\pgfqpoint{3.125048in}{2.399232in}}{\pgfqpoint{3.130872in}{2.405056in}}%
\pgfpathcurveto{\pgfqpoint{3.136696in}{2.410880in}}{\pgfqpoint{3.139968in}{2.418780in}}{\pgfqpoint{3.139968in}{2.427016in}}%
\pgfpathcurveto{\pgfqpoint{3.139968in}{2.435253in}}{\pgfqpoint{3.136696in}{2.443153in}}{\pgfqpoint{3.130872in}{2.448977in}}%
\pgfpathcurveto{\pgfqpoint{3.125048in}{2.454801in}}{\pgfqpoint{3.117148in}{2.458073in}}{\pgfqpoint{3.108912in}{2.458073in}}%
\pgfpathcurveto{\pgfqpoint{3.100675in}{2.458073in}}{\pgfqpoint{3.092775in}{2.454801in}}{\pgfqpoint{3.086951in}{2.448977in}}%
\pgfpathcurveto{\pgfqpoint{3.081128in}{2.443153in}}{\pgfqpoint{3.077855in}{2.435253in}}{\pgfqpoint{3.077855in}{2.427016in}}%
\pgfpathcurveto{\pgfqpoint{3.077855in}{2.418780in}}{\pgfqpoint{3.081128in}{2.410880in}}{\pgfqpoint{3.086951in}{2.405056in}}%
\pgfpathcurveto{\pgfqpoint{3.092775in}{2.399232in}}{\pgfqpoint{3.100675in}{2.395960in}}{\pgfqpoint{3.108912in}{2.395960in}}%
\pgfpathclose%
\pgfusepath{stroke,fill}%
\end{pgfscope}%
\begin{pgfscope}%
\pgfpathrectangle{\pgfqpoint{0.100000in}{0.220728in}}{\pgfqpoint{3.696000in}{3.696000in}}%
\pgfusepath{clip}%
\pgfsetbuttcap%
\pgfsetroundjoin%
\definecolor{currentfill}{rgb}{0.121569,0.466667,0.705882}%
\pgfsetfillcolor{currentfill}%
\pgfsetfillopacity{0.765177}%
\pgfsetlinewidth{1.003750pt}%
\definecolor{currentstroke}{rgb}{0.121569,0.466667,0.705882}%
\pgfsetstrokecolor{currentstroke}%
\pgfsetstrokeopacity{0.765177}%
\pgfsetdash{}{0pt}%
\pgfpathmoveto{\pgfqpoint{3.108640in}{2.395205in}}%
\pgfpathcurveto{\pgfqpoint{3.116876in}{2.395205in}}{\pgfqpoint{3.124776in}{2.398477in}}{\pgfqpoint{3.130600in}{2.404301in}}%
\pgfpathcurveto{\pgfqpoint{3.136424in}{2.410125in}}{\pgfqpoint{3.139696in}{2.418025in}}{\pgfqpoint{3.139696in}{2.426261in}}%
\pgfpathcurveto{\pgfqpoint{3.139696in}{2.434498in}}{\pgfqpoint{3.136424in}{2.442398in}}{\pgfqpoint{3.130600in}{2.448222in}}%
\pgfpathcurveto{\pgfqpoint{3.124776in}{2.454045in}}{\pgfqpoint{3.116876in}{2.457318in}}{\pgfqpoint{3.108640in}{2.457318in}}%
\pgfpathcurveto{\pgfqpoint{3.100404in}{2.457318in}}{\pgfqpoint{3.092504in}{2.454045in}}{\pgfqpoint{3.086680in}{2.448222in}}%
\pgfpathcurveto{\pgfqpoint{3.080856in}{2.442398in}}{\pgfqpoint{3.077583in}{2.434498in}}{\pgfqpoint{3.077583in}{2.426261in}}%
\pgfpathcurveto{\pgfqpoint{3.077583in}{2.418025in}}{\pgfqpoint{3.080856in}{2.410125in}}{\pgfqpoint{3.086680in}{2.404301in}}%
\pgfpathcurveto{\pgfqpoint{3.092504in}{2.398477in}}{\pgfqpoint{3.100404in}{2.395205in}}{\pgfqpoint{3.108640in}{2.395205in}}%
\pgfpathclose%
\pgfusepath{stroke,fill}%
\end{pgfscope}%
\begin{pgfscope}%
\pgfpathrectangle{\pgfqpoint{0.100000in}{0.220728in}}{\pgfqpoint{3.696000in}{3.696000in}}%
\pgfusepath{clip}%
\pgfsetbuttcap%
\pgfsetroundjoin%
\definecolor{currentfill}{rgb}{0.121569,0.466667,0.705882}%
\pgfsetfillcolor{currentfill}%
\pgfsetfillopacity{0.765381}%
\pgfsetlinewidth{1.003750pt}%
\definecolor{currentstroke}{rgb}{0.121569,0.466667,0.705882}%
\pgfsetstrokecolor{currentstroke}%
\pgfsetstrokeopacity{0.765381}%
\pgfsetdash{}{0pt}%
\pgfpathmoveto{\pgfqpoint{3.107830in}{2.393854in}}%
\pgfpathcurveto{\pgfqpoint{3.116066in}{2.393854in}}{\pgfqpoint{3.123966in}{2.397126in}}{\pgfqpoint{3.129790in}{2.402950in}}%
\pgfpathcurveto{\pgfqpoint{3.135614in}{2.408774in}}{\pgfqpoint{3.138886in}{2.416674in}}{\pgfqpoint{3.138886in}{2.424911in}}%
\pgfpathcurveto{\pgfqpoint{3.138886in}{2.433147in}}{\pgfqpoint{3.135614in}{2.441047in}}{\pgfqpoint{3.129790in}{2.446871in}}%
\pgfpathcurveto{\pgfqpoint{3.123966in}{2.452695in}}{\pgfqpoint{3.116066in}{2.455967in}}{\pgfqpoint{3.107830in}{2.455967in}}%
\pgfpathcurveto{\pgfqpoint{3.099593in}{2.455967in}}{\pgfqpoint{3.091693in}{2.452695in}}{\pgfqpoint{3.085869in}{2.446871in}}%
\pgfpathcurveto{\pgfqpoint{3.080046in}{2.441047in}}{\pgfqpoint{3.076773in}{2.433147in}}{\pgfqpoint{3.076773in}{2.424911in}}%
\pgfpathcurveto{\pgfqpoint{3.076773in}{2.416674in}}{\pgfqpoint{3.080046in}{2.408774in}}{\pgfqpoint{3.085869in}{2.402950in}}%
\pgfpathcurveto{\pgfqpoint{3.091693in}{2.397126in}}{\pgfqpoint{3.099593in}{2.393854in}}{\pgfqpoint{3.107830in}{2.393854in}}%
\pgfpathclose%
\pgfusepath{stroke,fill}%
\end{pgfscope}%
\begin{pgfscope}%
\pgfpathrectangle{\pgfqpoint{0.100000in}{0.220728in}}{\pgfqpoint{3.696000in}{3.696000in}}%
\pgfusepath{clip}%
\pgfsetbuttcap%
\pgfsetroundjoin%
\definecolor{currentfill}{rgb}{0.121569,0.466667,0.705882}%
\pgfsetfillcolor{currentfill}%
\pgfsetfillopacity{0.765856}%
\pgfsetlinewidth{1.003750pt}%
\definecolor{currentstroke}{rgb}{0.121569,0.466667,0.705882}%
\pgfsetstrokecolor{currentstroke}%
\pgfsetstrokeopacity{0.765856}%
\pgfsetdash{}{0pt}%
\pgfpathmoveto{\pgfqpoint{3.106851in}{2.391348in}}%
\pgfpathcurveto{\pgfqpoint{3.115087in}{2.391348in}}{\pgfqpoint{3.122987in}{2.394621in}}{\pgfqpoint{3.128811in}{2.400445in}}%
\pgfpathcurveto{\pgfqpoint{3.134635in}{2.406269in}}{\pgfqpoint{3.137907in}{2.414169in}}{\pgfqpoint{3.137907in}{2.422405in}}%
\pgfpathcurveto{\pgfqpoint{3.137907in}{2.430641in}}{\pgfqpoint{3.134635in}{2.438541in}}{\pgfqpoint{3.128811in}{2.444365in}}%
\pgfpathcurveto{\pgfqpoint{3.122987in}{2.450189in}}{\pgfqpoint{3.115087in}{2.453461in}}{\pgfqpoint{3.106851in}{2.453461in}}%
\pgfpathcurveto{\pgfqpoint{3.098614in}{2.453461in}}{\pgfqpoint{3.090714in}{2.450189in}}{\pgfqpoint{3.084890in}{2.444365in}}%
\pgfpathcurveto{\pgfqpoint{3.079066in}{2.438541in}}{\pgfqpoint{3.075794in}{2.430641in}}{\pgfqpoint{3.075794in}{2.422405in}}%
\pgfpathcurveto{\pgfqpoint{3.075794in}{2.414169in}}{\pgfqpoint{3.079066in}{2.406269in}}{\pgfqpoint{3.084890in}{2.400445in}}%
\pgfpathcurveto{\pgfqpoint{3.090714in}{2.394621in}}{\pgfqpoint{3.098614in}{2.391348in}}{\pgfqpoint{3.106851in}{2.391348in}}%
\pgfpathclose%
\pgfusepath{stroke,fill}%
\end{pgfscope}%
\begin{pgfscope}%
\pgfpathrectangle{\pgfqpoint{0.100000in}{0.220728in}}{\pgfqpoint{3.696000in}{3.696000in}}%
\pgfusepath{clip}%
\pgfsetbuttcap%
\pgfsetroundjoin%
\definecolor{currentfill}{rgb}{0.121569,0.466667,0.705882}%
\pgfsetfillcolor{currentfill}%
\pgfsetfillopacity{0.766472}%
\pgfsetlinewidth{1.003750pt}%
\definecolor{currentstroke}{rgb}{0.121569,0.466667,0.705882}%
\pgfsetstrokecolor{currentstroke}%
\pgfsetstrokeopacity{0.766472}%
\pgfsetdash{}{0pt}%
\pgfpathmoveto{\pgfqpoint{3.105420in}{2.388068in}}%
\pgfpathcurveto{\pgfqpoint{3.113656in}{2.388068in}}{\pgfqpoint{3.121556in}{2.391340in}}{\pgfqpoint{3.127380in}{2.397164in}}%
\pgfpathcurveto{\pgfqpoint{3.133204in}{2.402988in}}{\pgfqpoint{3.136477in}{2.410888in}}{\pgfqpoint{3.136477in}{2.419124in}}%
\pgfpathcurveto{\pgfqpoint{3.136477in}{2.427361in}}{\pgfqpoint{3.133204in}{2.435261in}}{\pgfqpoint{3.127380in}{2.441085in}}%
\pgfpathcurveto{\pgfqpoint{3.121556in}{2.446909in}}{\pgfqpoint{3.113656in}{2.450181in}}{\pgfqpoint{3.105420in}{2.450181in}}%
\pgfpathcurveto{\pgfqpoint{3.097184in}{2.450181in}}{\pgfqpoint{3.089284in}{2.446909in}}{\pgfqpoint{3.083460in}{2.441085in}}%
\pgfpathcurveto{\pgfqpoint{3.077636in}{2.435261in}}{\pgfqpoint{3.074364in}{2.427361in}}{\pgfqpoint{3.074364in}{2.419124in}}%
\pgfpathcurveto{\pgfqpoint{3.074364in}{2.410888in}}{\pgfqpoint{3.077636in}{2.402988in}}{\pgfqpoint{3.083460in}{2.397164in}}%
\pgfpathcurveto{\pgfqpoint{3.089284in}{2.391340in}}{\pgfqpoint{3.097184in}{2.388068in}}{\pgfqpoint{3.105420in}{2.388068in}}%
\pgfpathclose%
\pgfusepath{stroke,fill}%
\end{pgfscope}%
\begin{pgfscope}%
\pgfpathrectangle{\pgfqpoint{0.100000in}{0.220728in}}{\pgfqpoint{3.696000in}{3.696000in}}%
\pgfusepath{clip}%
\pgfsetbuttcap%
\pgfsetroundjoin%
\definecolor{currentfill}{rgb}{0.121569,0.466667,0.705882}%
\pgfsetfillcolor{currentfill}%
\pgfsetfillopacity{0.767078}%
\pgfsetlinewidth{1.003750pt}%
\definecolor{currentstroke}{rgb}{0.121569,0.466667,0.705882}%
\pgfsetstrokecolor{currentstroke}%
\pgfsetstrokeopacity{0.767078}%
\pgfsetdash{}{0pt}%
\pgfpathmoveto{\pgfqpoint{3.103116in}{2.384469in}}%
\pgfpathcurveto{\pgfqpoint{3.111352in}{2.384469in}}{\pgfqpoint{3.119252in}{2.387741in}}{\pgfqpoint{3.125076in}{2.393565in}}%
\pgfpathcurveto{\pgfqpoint{3.130900in}{2.399389in}}{\pgfqpoint{3.134172in}{2.407289in}}{\pgfqpoint{3.134172in}{2.415526in}}%
\pgfpathcurveto{\pgfqpoint{3.134172in}{2.423762in}}{\pgfqpoint{3.130900in}{2.431662in}}{\pgfqpoint{3.125076in}{2.437486in}}%
\pgfpathcurveto{\pgfqpoint{3.119252in}{2.443310in}}{\pgfqpoint{3.111352in}{2.446582in}}{\pgfqpoint{3.103116in}{2.446582in}}%
\pgfpathcurveto{\pgfqpoint{3.094880in}{2.446582in}}{\pgfqpoint{3.086979in}{2.443310in}}{\pgfqpoint{3.081156in}{2.437486in}}%
\pgfpathcurveto{\pgfqpoint{3.075332in}{2.431662in}}{\pgfqpoint{3.072059in}{2.423762in}}{\pgfqpoint{3.072059in}{2.415526in}}%
\pgfpathcurveto{\pgfqpoint{3.072059in}{2.407289in}}{\pgfqpoint{3.075332in}{2.399389in}}{\pgfqpoint{3.081156in}{2.393565in}}%
\pgfpathcurveto{\pgfqpoint{3.086979in}{2.387741in}}{\pgfqpoint{3.094880in}{2.384469in}}{\pgfqpoint{3.103116in}{2.384469in}}%
\pgfpathclose%
\pgfusepath{stroke,fill}%
\end{pgfscope}%
\begin{pgfscope}%
\pgfpathrectangle{\pgfqpoint{0.100000in}{0.220728in}}{\pgfqpoint{3.696000in}{3.696000in}}%
\pgfusepath{clip}%
\pgfsetbuttcap%
\pgfsetroundjoin%
\definecolor{currentfill}{rgb}{0.121569,0.466667,0.705882}%
\pgfsetfillcolor{currentfill}%
\pgfsetfillopacity{0.768179}%
\pgfsetlinewidth{1.003750pt}%
\definecolor{currentstroke}{rgb}{0.121569,0.466667,0.705882}%
\pgfsetstrokecolor{currentstroke}%
\pgfsetstrokeopacity{0.768179}%
\pgfsetdash{}{0pt}%
\pgfpathmoveto{\pgfqpoint{3.101138in}{2.378874in}}%
\pgfpathcurveto{\pgfqpoint{3.109375in}{2.378874in}}{\pgfqpoint{3.117275in}{2.382146in}}{\pgfqpoint{3.123099in}{2.387970in}}%
\pgfpathcurveto{\pgfqpoint{3.128923in}{2.393794in}}{\pgfqpoint{3.132195in}{2.401694in}}{\pgfqpoint{3.132195in}{2.409930in}}%
\pgfpathcurveto{\pgfqpoint{3.132195in}{2.418166in}}{\pgfqpoint{3.128923in}{2.426066in}}{\pgfqpoint{3.123099in}{2.431890in}}%
\pgfpathcurveto{\pgfqpoint{3.117275in}{2.437714in}}{\pgfqpoint{3.109375in}{2.440987in}}{\pgfqpoint{3.101138in}{2.440987in}}%
\pgfpathcurveto{\pgfqpoint{3.092902in}{2.440987in}}{\pgfqpoint{3.085002in}{2.437714in}}{\pgfqpoint{3.079178in}{2.431890in}}%
\pgfpathcurveto{\pgfqpoint{3.073354in}{2.426066in}}{\pgfqpoint{3.070082in}{2.418166in}}{\pgfqpoint{3.070082in}{2.409930in}}%
\pgfpathcurveto{\pgfqpoint{3.070082in}{2.401694in}}{\pgfqpoint{3.073354in}{2.393794in}}{\pgfqpoint{3.079178in}{2.387970in}}%
\pgfpathcurveto{\pgfqpoint{3.085002in}{2.382146in}}{\pgfqpoint{3.092902in}{2.378874in}}{\pgfqpoint{3.101138in}{2.378874in}}%
\pgfpathclose%
\pgfusepath{stroke,fill}%
\end{pgfscope}%
\begin{pgfscope}%
\pgfpathrectangle{\pgfqpoint{0.100000in}{0.220728in}}{\pgfqpoint{3.696000in}{3.696000in}}%
\pgfusepath{clip}%
\pgfsetbuttcap%
\pgfsetroundjoin%
\definecolor{currentfill}{rgb}{0.121569,0.466667,0.705882}%
\pgfsetfillcolor{currentfill}%
\pgfsetfillopacity{0.769339}%
\pgfsetlinewidth{1.003750pt}%
\definecolor{currentstroke}{rgb}{0.121569,0.466667,0.705882}%
\pgfsetstrokecolor{currentstroke}%
\pgfsetstrokeopacity{0.769339}%
\pgfsetdash{}{0pt}%
\pgfpathmoveto{\pgfqpoint{3.098333in}{2.372995in}}%
\pgfpathcurveto{\pgfqpoint{3.106569in}{2.372995in}}{\pgfqpoint{3.114469in}{2.376267in}}{\pgfqpoint{3.120293in}{2.382091in}}%
\pgfpathcurveto{\pgfqpoint{3.126117in}{2.387915in}}{\pgfqpoint{3.129389in}{2.395815in}}{\pgfqpoint{3.129389in}{2.404051in}}%
\pgfpathcurveto{\pgfqpoint{3.129389in}{2.412287in}}{\pgfqpoint{3.126117in}{2.420187in}}{\pgfqpoint{3.120293in}{2.426011in}}%
\pgfpathcurveto{\pgfqpoint{3.114469in}{2.431835in}}{\pgfqpoint{3.106569in}{2.435108in}}{\pgfqpoint{3.098333in}{2.435108in}}%
\pgfpathcurveto{\pgfqpoint{3.090097in}{2.435108in}}{\pgfqpoint{3.082196in}{2.431835in}}{\pgfqpoint{3.076373in}{2.426011in}}%
\pgfpathcurveto{\pgfqpoint{3.070549in}{2.420187in}}{\pgfqpoint{3.067276in}{2.412287in}}{\pgfqpoint{3.067276in}{2.404051in}}%
\pgfpathcurveto{\pgfqpoint{3.067276in}{2.395815in}}{\pgfqpoint{3.070549in}{2.387915in}}{\pgfqpoint{3.076373in}{2.382091in}}%
\pgfpathcurveto{\pgfqpoint{3.082196in}{2.376267in}}{\pgfqpoint{3.090097in}{2.372995in}}{\pgfqpoint{3.098333in}{2.372995in}}%
\pgfpathclose%
\pgfusepath{stroke,fill}%
\end{pgfscope}%
\begin{pgfscope}%
\pgfpathrectangle{\pgfqpoint{0.100000in}{0.220728in}}{\pgfqpoint{3.696000in}{3.696000in}}%
\pgfusepath{clip}%
\pgfsetbuttcap%
\pgfsetroundjoin%
\definecolor{currentfill}{rgb}{0.121569,0.466667,0.705882}%
\pgfsetfillcolor{currentfill}%
\pgfsetfillopacity{0.770327}%
\pgfsetlinewidth{1.003750pt}%
\definecolor{currentstroke}{rgb}{0.121569,0.466667,0.705882}%
\pgfsetstrokecolor{currentstroke}%
\pgfsetstrokeopacity{0.770327}%
\pgfsetdash{}{0pt}%
\pgfpathmoveto{\pgfqpoint{1.182361in}{1.190847in}}%
\pgfpathcurveto{\pgfqpoint{1.190597in}{1.190847in}}{\pgfqpoint{1.198497in}{1.194119in}}{\pgfqpoint{1.204321in}{1.199943in}}%
\pgfpathcurveto{\pgfqpoint{1.210145in}{1.205767in}}{\pgfqpoint{1.213418in}{1.213667in}}{\pgfqpoint{1.213418in}{1.221903in}}%
\pgfpathcurveto{\pgfqpoint{1.213418in}{1.230140in}}{\pgfqpoint{1.210145in}{1.238040in}}{\pgfqpoint{1.204321in}{1.243863in}}%
\pgfpathcurveto{\pgfqpoint{1.198497in}{1.249687in}}{\pgfqpoint{1.190597in}{1.252960in}}{\pgfqpoint{1.182361in}{1.252960in}}%
\pgfpathcurveto{\pgfqpoint{1.174125in}{1.252960in}}{\pgfqpoint{1.166225in}{1.249687in}}{\pgfqpoint{1.160401in}{1.243863in}}%
\pgfpathcurveto{\pgfqpoint{1.154577in}{1.238040in}}{\pgfqpoint{1.151305in}{1.230140in}}{\pgfqpoint{1.151305in}{1.221903in}}%
\pgfpathcurveto{\pgfqpoint{1.151305in}{1.213667in}}{\pgfqpoint{1.154577in}{1.205767in}}{\pgfqpoint{1.160401in}{1.199943in}}%
\pgfpathcurveto{\pgfqpoint{1.166225in}{1.194119in}}{\pgfqpoint{1.174125in}{1.190847in}}{\pgfqpoint{1.182361in}{1.190847in}}%
\pgfpathclose%
\pgfusepath{stroke,fill}%
\end{pgfscope}%
\begin{pgfscope}%
\pgfpathrectangle{\pgfqpoint{0.100000in}{0.220728in}}{\pgfqpoint{3.696000in}{3.696000in}}%
\pgfusepath{clip}%
\pgfsetbuttcap%
\pgfsetroundjoin%
\definecolor{currentfill}{rgb}{0.121569,0.466667,0.705882}%
\pgfsetfillcolor{currentfill}%
\pgfsetfillopacity{0.770517}%
\pgfsetlinewidth{1.003750pt}%
\definecolor{currentstroke}{rgb}{0.121569,0.466667,0.705882}%
\pgfsetstrokecolor{currentstroke}%
\pgfsetstrokeopacity{0.770517}%
\pgfsetdash{}{0pt}%
\pgfpathmoveto{\pgfqpoint{3.094440in}{2.366457in}}%
\pgfpathcurveto{\pgfqpoint{3.102676in}{2.366457in}}{\pgfqpoint{3.110576in}{2.369729in}}{\pgfqpoint{3.116400in}{2.375553in}}%
\pgfpathcurveto{\pgfqpoint{3.122224in}{2.381377in}}{\pgfqpoint{3.125496in}{2.389277in}}{\pgfqpoint{3.125496in}{2.397513in}}%
\pgfpathcurveto{\pgfqpoint{3.125496in}{2.405749in}}{\pgfqpoint{3.122224in}{2.413649in}}{\pgfqpoint{3.116400in}{2.419473in}}%
\pgfpathcurveto{\pgfqpoint{3.110576in}{2.425297in}}{\pgfqpoint{3.102676in}{2.428570in}}{\pgfqpoint{3.094440in}{2.428570in}}%
\pgfpathcurveto{\pgfqpoint{3.086204in}{2.428570in}}{\pgfqpoint{3.078304in}{2.425297in}}{\pgfqpoint{3.072480in}{2.419473in}}%
\pgfpathcurveto{\pgfqpoint{3.066656in}{2.413649in}}{\pgfqpoint{3.063383in}{2.405749in}}{\pgfqpoint{3.063383in}{2.397513in}}%
\pgfpathcurveto{\pgfqpoint{3.063383in}{2.389277in}}{\pgfqpoint{3.066656in}{2.381377in}}{\pgfqpoint{3.072480in}{2.375553in}}%
\pgfpathcurveto{\pgfqpoint{3.078304in}{2.369729in}}{\pgfqpoint{3.086204in}{2.366457in}}{\pgfqpoint{3.094440in}{2.366457in}}%
\pgfpathclose%
\pgfusepath{stroke,fill}%
\end{pgfscope}%
\begin{pgfscope}%
\pgfpathrectangle{\pgfqpoint{0.100000in}{0.220728in}}{\pgfqpoint{3.696000in}{3.696000in}}%
\pgfusepath{clip}%
\pgfsetbuttcap%
\pgfsetroundjoin%
\definecolor{currentfill}{rgb}{0.121569,0.466667,0.705882}%
\pgfsetfillcolor{currentfill}%
\pgfsetfillopacity{0.772108}%
\pgfsetlinewidth{1.003750pt}%
\definecolor{currentstroke}{rgb}{0.121569,0.466667,0.705882}%
\pgfsetstrokecolor{currentstroke}%
\pgfsetstrokeopacity{0.772108}%
\pgfsetdash{}{0pt}%
\pgfpathmoveto{\pgfqpoint{3.091143in}{2.357266in}}%
\pgfpathcurveto{\pgfqpoint{3.099379in}{2.357266in}}{\pgfqpoint{3.107280in}{2.360538in}}{\pgfqpoint{3.113103in}{2.366362in}}%
\pgfpathcurveto{\pgfqpoint{3.118927in}{2.372186in}}{\pgfqpoint{3.122200in}{2.380086in}}{\pgfqpoint{3.122200in}{2.388322in}}%
\pgfpathcurveto{\pgfqpoint{3.122200in}{2.396559in}}{\pgfqpoint{3.118927in}{2.404459in}}{\pgfqpoint{3.113103in}{2.410283in}}%
\pgfpathcurveto{\pgfqpoint{3.107280in}{2.416106in}}{\pgfqpoint{3.099379in}{2.419379in}}{\pgfqpoint{3.091143in}{2.419379in}}%
\pgfpathcurveto{\pgfqpoint{3.082907in}{2.419379in}}{\pgfqpoint{3.075007in}{2.416106in}}{\pgfqpoint{3.069183in}{2.410283in}}%
\pgfpathcurveto{\pgfqpoint{3.063359in}{2.404459in}}{\pgfqpoint{3.060087in}{2.396559in}}{\pgfqpoint{3.060087in}{2.388322in}}%
\pgfpathcurveto{\pgfqpoint{3.060087in}{2.380086in}}{\pgfqpoint{3.063359in}{2.372186in}}{\pgfqpoint{3.069183in}{2.366362in}}%
\pgfpathcurveto{\pgfqpoint{3.075007in}{2.360538in}}{\pgfqpoint{3.082907in}{2.357266in}}{\pgfqpoint{3.091143in}{2.357266in}}%
\pgfpathclose%
\pgfusepath{stroke,fill}%
\end{pgfscope}%
\begin{pgfscope}%
\pgfpathrectangle{\pgfqpoint{0.100000in}{0.220728in}}{\pgfqpoint{3.696000in}{3.696000in}}%
\pgfusepath{clip}%
\pgfsetbuttcap%
\pgfsetroundjoin%
\definecolor{currentfill}{rgb}{0.121569,0.466667,0.705882}%
\pgfsetfillcolor{currentfill}%
\pgfsetfillopacity{0.773011}%
\pgfsetlinewidth{1.003750pt}%
\definecolor{currentstroke}{rgb}{0.121569,0.466667,0.705882}%
\pgfsetstrokecolor{currentstroke}%
\pgfsetstrokeopacity{0.773011}%
\pgfsetdash{}{0pt}%
\pgfpathmoveto{\pgfqpoint{3.088846in}{2.352814in}}%
\pgfpathcurveto{\pgfqpoint{3.097082in}{2.352814in}}{\pgfqpoint{3.104982in}{2.356086in}}{\pgfqpoint{3.110806in}{2.361910in}}%
\pgfpathcurveto{\pgfqpoint{3.116630in}{2.367734in}}{\pgfqpoint{3.119903in}{2.375634in}}{\pgfqpoint{3.119903in}{2.383870in}}%
\pgfpathcurveto{\pgfqpoint{3.119903in}{2.392107in}}{\pgfqpoint{3.116630in}{2.400007in}}{\pgfqpoint{3.110806in}{2.405831in}}%
\pgfpathcurveto{\pgfqpoint{3.104982in}{2.411654in}}{\pgfqpoint{3.097082in}{2.414927in}}{\pgfqpoint{3.088846in}{2.414927in}}%
\pgfpathcurveto{\pgfqpoint{3.080610in}{2.414927in}}{\pgfqpoint{3.072710in}{2.411654in}}{\pgfqpoint{3.066886in}{2.405831in}}%
\pgfpathcurveto{\pgfqpoint{3.061062in}{2.400007in}}{\pgfqpoint{3.057790in}{2.392107in}}{\pgfqpoint{3.057790in}{2.383870in}}%
\pgfpathcurveto{\pgfqpoint{3.057790in}{2.375634in}}{\pgfqpoint{3.061062in}{2.367734in}}{\pgfqpoint{3.066886in}{2.361910in}}%
\pgfpathcurveto{\pgfqpoint{3.072710in}{2.356086in}}{\pgfqpoint{3.080610in}{2.352814in}}{\pgfqpoint{3.088846in}{2.352814in}}%
\pgfpathclose%
\pgfusepath{stroke,fill}%
\end{pgfscope}%
\begin{pgfscope}%
\pgfpathrectangle{\pgfqpoint{0.100000in}{0.220728in}}{\pgfqpoint{3.696000in}{3.696000in}}%
\pgfusepath{clip}%
\pgfsetbuttcap%
\pgfsetroundjoin%
\definecolor{currentfill}{rgb}{0.121569,0.466667,0.705882}%
\pgfsetfillcolor{currentfill}%
\pgfsetfillopacity{0.773495}%
\pgfsetlinewidth{1.003750pt}%
\definecolor{currentstroke}{rgb}{0.121569,0.466667,0.705882}%
\pgfsetstrokecolor{currentstroke}%
\pgfsetstrokeopacity{0.773495}%
\pgfsetdash{}{0pt}%
\pgfpathmoveto{\pgfqpoint{3.087433in}{2.350504in}}%
\pgfpathcurveto{\pgfqpoint{3.095669in}{2.350504in}}{\pgfqpoint{3.103569in}{2.353777in}}{\pgfqpoint{3.109393in}{2.359601in}}%
\pgfpathcurveto{\pgfqpoint{3.115217in}{2.365425in}}{\pgfqpoint{3.118489in}{2.373325in}}{\pgfqpoint{3.118489in}{2.381561in}}%
\pgfpathcurveto{\pgfqpoint{3.118489in}{2.389797in}}{\pgfqpoint{3.115217in}{2.397697in}}{\pgfqpoint{3.109393in}{2.403521in}}%
\pgfpathcurveto{\pgfqpoint{3.103569in}{2.409345in}}{\pgfqpoint{3.095669in}{2.412617in}}{\pgfqpoint{3.087433in}{2.412617in}}%
\pgfpathcurveto{\pgfqpoint{3.079197in}{2.412617in}}{\pgfqpoint{3.071296in}{2.409345in}}{\pgfqpoint{3.065473in}{2.403521in}}%
\pgfpathcurveto{\pgfqpoint{3.059649in}{2.397697in}}{\pgfqpoint{3.056376in}{2.389797in}}{\pgfqpoint{3.056376in}{2.381561in}}%
\pgfpathcurveto{\pgfqpoint{3.056376in}{2.373325in}}{\pgfqpoint{3.059649in}{2.365425in}}{\pgfqpoint{3.065473in}{2.359601in}}%
\pgfpathcurveto{\pgfqpoint{3.071296in}{2.353777in}}{\pgfqpoint{3.079197in}{2.350504in}}{\pgfqpoint{3.087433in}{2.350504in}}%
\pgfpathclose%
\pgfusepath{stroke,fill}%
\end{pgfscope}%
\begin{pgfscope}%
\pgfpathrectangle{\pgfqpoint{0.100000in}{0.220728in}}{\pgfqpoint{3.696000in}{3.696000in}}%
\pgfusepath{clip}%
\pgfsetbuttcap%
\pgfsetroundjoin%
\definecolor{currentfill}{rgb}{0.121569,0.466667,0.705882}%
\pgfsetfillcolor{currentfill}%
\pgfsetfillopacity{0.773752}%
\pgfsetlinewidth{1.003750pt}%
\definecolor{currentstroke}{rgb}{0.121569,0.466667,0.705882}%
\pgfsetstrokecolor{currentstroke}%
\pgfsetstrokeopacity{0.773752}%
\pgfsetdash{}{0pt}%
\pgfpathmoveto{\pgfqpoint{3.086956in}{2.348887in}}%
\pgfpathcurveto{\pgfqpoint{3.095192in}{2.348887in}}{\pgfqpoint{3.103093in}{2.352159in}}{\pgfqpoint{3.108916in}{2.357983in}}%
\pgfpathcurveto{\pgfqpoint{3.114740in}{2.363807in}}{\pgfqpoint{3.118013in}{2.371707in}}{\pgfqpoint{3.118013in}{2.379943in}}%
\pgfpathcurveto{\pgfqpoint{3.118013in}{2.388179in}}{\pgfqpoint{3.114740in}{2.396079in}}{\pgfqpoint{3.108916in}{2.401903in}}%
\pgfpathcurveto{\pgfqpoint{3.103093in}{2.407727in}}{\pgfqpoint{3.095192in}{2.411000in}}{\pgfqpoint{3.086956in}{2.411000in}}%
\pgfpathcurveto{\pgfqpoint{3.078720in}{2.411000in}}{\pgfqpoint{3.070820in}{2.407727in}}{\pgfqpoint{3.064996in}{2.401903in}}%
\pgfpathcurveto{\pgfqpoint{3.059172in}{2.396079in}}{\pgfqpoint{3.055900in}{2.388179in}}{\pgfqpoint{3.055900in}{2.379943in}}%
\pgfpathcurveto{\pgfqpoint{3.055900in}{2.371707in}}{\pgfqpoint{3.059172in}{2.363807in}}{\pgfqpoint{3.064996in}{2.357983in}}%
\pgfpathcurveto{\pgfqpoint{3.070820in}{2.352159in}}{\pgfqpoint{3.078720in}{2.348887in}}{\pgfqpoint{3.086956in}{2.348887in}}%
\pgfpathclose%
\pgfusepath{stroke,fill}%
\end{pgfscope}%
\begin{pgfscope}%
\pgfpathrectangle{\pgfqpoint{0.100000in}{0.220728in}}{\pgfqpoint{3.696000in}{3.696000in}}%
\pgfusepath{clip}%
\pgfsetbuttcap%
\pgfsetroundjoin%
\definecolor{currentfill}{rgb}{0.121569,0.466667,0.705882}%
\pgfsetfillcolor{currentfill}%
\pgfsetfillopacity{0.773960}%
\pgfsetlinewidth{1.003750pt}%
\definecolor{currentstroke}{rgb}{0.121569,0.466667,0.705882}%
\pgfsetstrokecolor{currentstroke}%
\pgfsetstrokeopacity{0.773960}%
\pgfsetdash{}{0pt}%
\pgfpathmoveto{\pgfqpoint{1.203164in}{1.181739in}}%
\pgfpathcurveto{\pgfqpoint{1.211400in}{1.181739in}}{\pgfqpoint{1.219300in}{1.185011in}}{\pgfqpoint{1.225124in}{1.190835in}}%
\pgfpathcurveto{\pgfqpoint{1.230948in}{1.196659in}}{\pgfqpoint{1.234221in}{1.204559in}}{\pgfqpoint{1.234221in}{1.212795in}}%
\pgfpathcurveto{\pgfqpoint{1.234221in}{1.221031in}}{\pgfqpoint{1.230948in}{1.228931in}}{\pgfqpoint{1.225124in}{1.234755in}}%
\pgfpathcurveto{\pgfqpoint{1.219300in}{1.240579in}}{\pgfqpoint{1.211400in}{1.243852in}}{\pgfqpoint{1.203164in}{1.243852in}}%
\pgfpathcurveto{\pgfqpoint{1.194928in}{1.243852in}}{\pgfqpoint{1.187028in}{1.240579in}}{\pgfqpoint{1.181204in}{1.234755in}}%
\pgfpathcurveto{\pgfqpoint{1.175380in}{1.228931in}}{\pgfqpoint{1.172108in}{1.221031in}}{\pgfqpoint{1.172108in}{1.212795in}}%
\pgfpathcurveto{\pgfqpoint{1.172108in}{1.204559in}}{\pgfqpoint{1.175380in}{1.196659in}}{\pgfqpoint{1.181204in}{1.190835in}}%
\pgfpathcurveto{\pgfqpoint{1.187028in}{1.185011in}}{\pgfqpoint{1.194928in}{1.181739in}}{\pgfqpoint{1.203164in}{1.181739in}}%
\pgfpathclose%
\pgfusepath{stroke,fill}%
\end{pgfscope}%
\begin{pgfscope}%
\pgfpathrectangle{\pgfqpoint{0.100000in}{0.220728in}}{\pgfqpoint{3.696000in}{3.696000in}}%
\pgfusepath{clip}%
\pgfsetbuttcap%
\pgfsetroundjoin%
\definecolor{currentfill}{rgb}{0.121569,0.466667,0.705882}%
\pgfsetfillcolor{currentfill}%
\pgfsetfillopacity{0.774237}%
\pgfsetlinewidth{1.003750pt}%
\definecolor{currentstroke}{rgb}{0.121569,0.466667,0.705882}%
\pgfsetstrokecolor{currentstroke}%
\pgfsetstrokeopacity{0.774237}%
\pgfsetdash{}{0pt}%
\pgfpathmoveto{\pgfqpoint{3.085463in}{2.346069in}}%
\pgfpathcurveto{\pgfqpoint{3.093699in}{2.346069in}}{\pgfqpoint{3.101599in}{2.349341in}}{\pgfqpoint{3.107423in}{2.355165in}}%
\pgfpathcurveto{\pgfqpoint{3.113247in}{2.360989in}}{\pgfqpoint{3.116519in}{2.368889in}}{\pgfqpoint{3.116519in}{2.377125in}}%
\pgfpathcurveto{\pgfqpoint{3.116519in}{2.385361in}}{\pgfqpoint{3.113247in}{2.393261in}}{\pgfqpoint{3.107423in}{2.399085in}}%
\pgfpathcurveto{\pgfqpoint{3.101599in}{2.404909in}}{\pgfqpoint{3.093699in}{2.408182in}}{\pgfqpoint{3.085463in}{2.408182in}}%
\pgfpathcurveto{\pgfqpoint{3.077226in}{2.408182in}}{\pgfqpoint{3.069326in}{2.404909in}}{\pgfqpoint{3.063502in}{2.399085in}}%
\pgfpathcurveto{\pgfqpoint{3.057678in}{2.393261in}}{\pgfqpoint{3.054406in}{2.385361in}}{\pgfqpoint{3.054406in}{2.377125in}}%
\pgfpathcurveto{\pgfqpoint{3.054406in}{2.368889in}}{\pgfqpoint{3.057678in}{2.360989in}}{\pgfqpoint{3.063502in}{2.355165in}}%
\pgfpathcurveto{\pgfqpoint{3.069326in}{2.349341in}}{\pgfqpoint{3.077226in}{2.346069in}}{\pgfqpoint{3.085463in}{2.346069in}}%
\pgfpathclose%
\pgfusepath{stroke,fill}%
\end{pgfscope}%
\begin{pgfscope}%
\pgfpathrectangle{\pgfqpoint{0.100000in}{0.220728in}}{\pgfqpoint{3.696000in}{3.696000in}}%
\pgfusepath{clip}%
\pgfsetbuttcap%
\pgfsetroundjoin%
\definecolor{currentfill}{rgb}{0.121569,0.466667,0.705882}%
\pgfsetfillcolor{currentfill}%
\pgfsetfillopacity{0.774507}%
\pgfsetlinewidth{1.003750pt}%
\definecolor{currentstroke}{rgb}{0.121569,0.466667,0.705882}%
\pgfsetstrokecolor{currentstroke}%
\pgfsetstrokeopacity{0.774507}%
\pgfsetdash{}{0pt}%
\pgfpathmoveto{\pgfqpoint{3.084640in}{2.344534in}}%
\pgfpathcurveto{\pgfqpoint{3.092876in}{2.344534in}}{\pgfqpoint{3.100777in}{2.347807in}}{\pgfqpoint{3.106600in}{2.353630in}}%
\pgfpathcurveto{\pgfqpoint{3.112424in}{2.359454in}}{\pgfqpoint{3.115697in}{2.367354in}}{\pgfqpoint{3.115697in}{2.375591in}}%
\pgfpathcurveto{\pgfqpoint{3.115697in}{2.383827in}}{\pgfqpoint{3.112424in}{2.391727in}}{\pgfqpoint{3.106600in}{2.397551in}}%
\pgfpathcurveto{\pgfqpoint{3.100777in}{2.403375in}}{\pgfqpoint{3.092876in}{2.406647in}}{\pgfqpoint{3.084640in}{2.406647in}}%
\pgfpathcurveto{\pgfqpoint{3.076404in}{2.406647in}}{\pgfqpoint{3.068504in}{2.403375in}}{\pgfqpoint{3.062680in}{2.397551in}}%
\pgfpathcurveto{\pgfqpoint{3.056856in}{2.391727in}}{\pgfqpoint{3.053584in}{2.383827in}}{\pgfqpoint{3.053584in}{2.375591in}}%
\pgfpathcurveto{\pgfqpoint{3.053584in}{2.367354in}}{\pgfqpoint{3.056856in}{2.359454in}}{\pgfqpoint{3.062680in}{2.353630in}}%
\pgfpathcurveto{\pgfqpoint{3.068504in}{2.347807in}}{\pgfqpoint{3.076404in}{2.344534in}}{\pgfqpoint{3.084640in}{2.344534in}}%
\pgfpathclose%
\pgfusepath{stroke,fill}%
\end{pgfscope}%
\begin{pgfscope}%
\pgfpathrectangle{\pgfqpoint{0.100000in}{0.220728in}}{\pgfqpoint{3.696000in}{3.696000in}}%
\pgfusepath{clip}%
\pgfsetbuttcap%
\pgfsetroundjoin%
\definecolor{currentfill}{rgb}{0.121569,0.466667,0.705882}%
\pgfsetfillcolor{currentfill}%
\pgfsetfillopacity{0.774680}%
\pgfsetlinewidth{1.003750pt}%
\definecolor{currentstroke}{rgb}{0.121569,0.466667,0.705882}%
\pgfsetstrokecolor{currentstroke}%
\pgfsetstrokeopacity{0.774680}%
\pgfsetdash{}{0pt}%
\pgfpathmoveto{\pgfqpoint{3.084303in}{2.343667in}}%
\pgfpathcurveto{\pgfqpoint{3.092539in}{2.343667in}}{\pgfqpoint{3.100439in}{2.346939in}}{\pgfqpoint{3.106263in}{2.352763in}}%
\pgfpathcurveto{\pgfqpoint{3.112087in}{2.358587in}}{\pgfqpoint{3.115359in}{2.366487in}}{\pgfqpoint{3.115359in}{2.374723in}}%
\pgfpathcurveto{\pgfqpoint{3.115359in}{2.382960in}}{\pgfqpoint{3.112087in}{2.390860in}}{\pgfqpoint{3.106263in}{2.396684in}}%
\pgfpathcurveto{\pgfqpoint{3.100439in}{2.402508in}}{\pgfqpoint{3.092539in}{2.405780in}}{\pgfqpoint{3.084303in}{2.405780in}}%
\pgfpathcurveto{\pgfqpoint{3.076067in}{2.405780in}}{\pgfqpoint{3.068167in}{2.402508in}}{\pgfqpoint{3.062343in}{2.396684in}}%
\pgfpathcurveto{\pgfqpoint{3.056519in}{2.390860in}}{\pgfqpoint{3.053246in}{2.382960in}}{\pgfqpoint{3.053246in}{2.374723in}}%
\pgfpathcurveto{\pgfqpoint{3.053246in}{2.366487in}}{\pgfqpoint{3.056519in}{2.358587in}}{\pgfqpoint{3.062343in}{2.352763in}}%
\pgfpathcurveto{\pgfqpoint{3.068167in}{2.346939in}}{\pgfqpoint{3.076067in}{2.343667in}}{\pgfqpoint{3.084303in}{2.343667in}}%
\pgfpathclose%
\pgfusepath{stroke,fill}%
\end{pgfscope}%
\begin{pgfscope}%
\pgfpathrectangle{\pgfqpoint{0.100000in}{0.220728in}}{\pgfqpoint{3.696000in}{3.696000in}}%
\pgfusepath{clip}%
\pgfsetbuttcap%
\pgfsetroundjoin%
\definecolor{currentfill}{rgb}{0.121569,0.466667,0.705882}%
\pgfsetfillcolor{currentfill}%
\pgfsetfillopacity{0.774989}%
\pgfsetlinewidth{1.003750pt}%
\definecolor{currentstroke}{rgb}{0.121569,0.466667,0.705882}%
\pgfsetstrokecolor{currentstroke}%
\pgfsetstrokeopacity{0.774989}%
\pgfsetdash{}{0pt}%
\pgfpathmoveto{\pgfqpoint{3.083153in}{2.341675in}}%
\pgfpathcurveto{\pgfqpoint{3.091390in}{2.341675in}}{\pgfqpoint{3.099290in}{2.344948in}}{\pgfqpoint{3.105113in}{2.350772in}}%
\pgfpathcurveto{\pgfqpoint{3.110937in}{2.356596in}}{\pgfqpoint{3.114210in}{2.364496in}}{\pgfqpoint{3.114210in}{2.372732in}}%
\pgfpathcurveto{\pgfqpoint{3.114210in}{2.380968in}}{\pgfqpoint{3.110937in}{2.388868in}}{\pgfqpoint{3.105113in}{2.394692in}}%
\pgfpathcurveto{\pgfqpoint{3.099290in}{2.400516in}}{\pgfqpoint{3.091390in}{2.403788in}}{\pgfqpoint{3.083153in}{2.403788in}}%
\pgfpathcurveto{\pgfqpoint{3.074917in}{2.403788in}}{\pgfqpoint{3.067017in}{2.400516in}}{\pgfqpoint{3.061193in}{2.394692in}}%
\pgfpathcurveto{\pgfqpoint{3.055369in}{2.388868in}}{\pgfqpoint{3.052097in}{2.380968in}}{\pgfqpoint{3.052097in}{2.372732in}}%
\pgfpathcurveto{\pgfqpoint{3.052097in}{2.364496in}}{\pgfqpoint{3.055369in}{2.356596in}}{\pgfqpoint{3.061193in}{2.350772in}}%
\pgfpathcurveto{\pgfqpoint{3.067017in}{2.344948in}}{\pgfqpoint{3.074917in}{2.341675in}}{\pgfqpoint{3.083153in}{2.341675in}}%
\pgfpathclose%
\pgfusepath{stroke,fill}%
\end{pgfscope}%
\begin{pgfscope}%
\pgfpathrectangle{\pgfqpoint{0.100000in}{0.220728in}}{\pgfqpoint{3.696000in}{3.696000in}}%
\pgfusepath{clip}%
\pgfsetbuttcap%
\pgfsetroundjoin%
\definecolor{currentfill}{rgb}{0.121569,0.466667,0.705882}%
\pgfsetfillcolor{currentfill}%
\pgfsetfillopacity{0.775208}%
\pgfsetlinewidth{1.003750pt}%
\definecolor{currentstroke}{rgb}{0.121569,0.466667,0.705882}%
\pgfsetstrokecolor{currentstroke}%
\pgfsetstrokeopacity{0.775208}%
\pgfsetdash{}{0pt}%
\pgfpathmoveto{\pgfqpoint{3.082662in}{2.340605in}}%
\pgfpathcurveto{\pgfqpoint{3.090898in}{2.340605in}}{\pgfqpoint{3.098798in}{2.343877in}}{\pgfqpoint{3.104622in}{2.349701in}}%
\pgfpathcurveto{\pgfqpoint{3.110446in}{2.355525in}}{\pgfqpoint{3.113718in}{2.363425in}}{\pgfqpoint{3.113718in}{2.371662in}}%
\pgfpathcurveto{\pgfqpoint{3.113718in}{2.379898in}}{\pgfqpoint{3.110446in}{2.387798in}}{\pgfqpoint{3.104622in}{2.393622in}}%
\pgfpathcurveto{\pgfqpoint{3.098798in}{2.399446in}}{\pgfqpoint{3.090898in}{2.402718in}}{\pgfqpoint{3.082662in}{2.402718in}}%
\pgfpathcurveto{\pgfqpoint{3.074425in}{2.402718in}}{\pgfqpoint{3.066525in}{2.399446in}}{\pgfqpoint{3.060701in}{2.393622in}}%
\pgfpathcurveto{\pgfqpoint{3.054877in}{2.387798in}}{\pgfqpoint{3.051605in}{2.379898in}}{\pgfqpoint{3.051605in}{2.371662in}}%
\pgfpathcurveto{\pgfqpoint{3.051605in}{2.363425in}}{\pgfqpoint{3.054877in}{2.355525in}}{\pgfqpoint{3.060701in}{2.349701in}}%
\pgfpathcurveto{\pgfqpoint{3.066525in}{2.343877in}}{\pgfqpoint{3.074425in}{2.340605in}}{\pgfqpoint{3.082662in}{2.340605in}}%
\pgfpathclose%
\pgfusepath{stroke,fill}%
\end{pgfscope}%
\begin{pgfscope}%
\pgfpathrectangle{\pgfqpoint{0.100000in}{0.220728in}}{\pgfqpoint{3.696000in}{3.696000in}}%
\pgfusepath{clip}%
\pgfsetbuttcap%
\pgfsetroundjoin%
\definecolor{currentfill}{rgb}{0.121569,0.466667,0.705882}%
\pgfsetfillcolor{currentfill}%
\pgfsetfillopacity{0.775322}%
\pgfsetlinewidth{1.003750pt}%
\definecolor{currentstroke}{rgb}{0.121569,0.466667,0.705882}%
\pgfsetstrokecolor{currentstroke}%
\pgfsetstrokeopacity{0.775322}%
\pgfsetdash{}{0pt}%
\pgfpathmoveto{\pgfqpoint{3.082430in}{2.339954in}}%
\pgfpathcurveto{\pgfqpoint{3.090666in}{2.339954in}}{\pgfqpoint{3.098566in}{2.343227in}}{\pgfqpoint{3.104390in}{2.349051in}}%
\pgfpathcurveto{\pgfqpoint{3.110214in}{2.354874in}}{\pgfqpoint{3.113487in}{2.362774in}}{\pgfqpoint{3.113487in}{2.371011in}}%
\pgfpathcurveto{\pgfqpoint{3.113487in}{2.379247in}}{\pgfqpoint{3.110214in}{2.387147in}}{\pgfqpoint{3.104390in}{2.392971in}}%
\pgfpathcurveto{\pgfqpoint{3.098566in}{2.398795in}}{\pgfqpoint{3.090666in}{2.402067in}}{\pgfqpoint{3.082430in}{2.402067in}}%
\pgfpathcurveto{\pgfqpoint{3.074194in}{2.402067in}}{\pgfqpoint{3.066294in}{2.398795in}}{\pgfqpoint{3.060470in}{2.392971in}}%
\pgfpathcurveto{\pgfqpoint{3.054646in}{2.387147in}}{\pgfqpoint{3.051374in}{2.379247in}}{\pgfqpoint{3.051374in}{2.371011in}}%
\pgfpathcurveto{\pgfqpoint{3.051374in}{2.362774in}}{\pgfqpoint{3.054646in}{2.354874in}}{\pgfqpoint{3.060470in}{2.349051in}}%
\pgfpathcurveto{\pgfqpoint{3.066294in}{2.343227in}}{\pgfqpoint{3.074194in}{2.339954in}}{\pgfqpoint{3.082430in}{2.339954in}}%
\pgfpathclose%
\pgfusepath{stroke,fill}%
\end{pgfscope}%
\begin{pgfscope}%
\pgfpathrectangle{\pgfqpoint{0.100000in}{0.220728in}}{\pgfqpoint{3.696000in}{3.696000in}}%
\pgfusepath{clip}%
\pgfsetbuttcap%
\pgfsetroundjoin%
\definecolor{currentfill}{rgb}{0.121569,0.466667,0.705882}%
\pgfsetfillcolor{currentfill}%
\pgfsetfillopacity{0.775573}%
\pgfsetlinewidth{1.003750pt}%
\definecolor{currentstroke}{rgb}{0.121569,0.466667,0.705882}%
\pgfsetstrokecolor{currentstroke}%
\pgfsetstrokeopacity{0.775573}%
\pgfsetdash{}{0pt}%
\pgfpathmoveto{\pgfqpoint{3.081543in}{2.338394in}}%
\pgfpathcurveto{\pgfqpoint{3.089779in}{2.338394in}}{\pgfqpoint{3.097679in}{2.341666in}}{\pgfqpoint{3.103503in}{2.347490in}}%
\pgfpathcurveto{\pgfqpoint{3.109327in}{2.353314in}}{\pgfqpoint{3.112599in}{2.361214in}}{\pgfqpoint{3.112599in}{2.369450in}}%
\pgfpathcurveto{\pgfqpoint{3.112599in}{2.377687in}}{\pgfqpoint{3.109327in}{2.385587in}}{\pgfqpoint{3.103503in}{2.391411in}}%
\pgfpathcurveto{\pgfqpoint{3.097679in}{2.397235in}}{\pgfqpoint{3.089779in}{2.400507in}}{\pgfqpoint{3.081543in}{2.400507in}}%
\pgfpathcurveto{\pgfqpoint{3.073306in}{2.400507in}}{\pgfqpoint{3.065406in}{2.397235in}}{\pgfqpoint{3.059582in}{2.391411in}}%
\pgfpathcurveto{\pgfqpoint{3.053758in}{2.385587in}}{\pgfqpoint{3.050486in}{2.377687in}}{\pgfqpoint{3.050486in}{2.369450in}}%
\pgfpathcurveto{\pgfqpoint{3.050486in}{2.361214in}}{\pgfqpoint{3.053758in}{2.353314in}}{\pgfqpoint{3.059582in}{2.347490in}}%
\pgfpathcurveto{\pgfqpoint{3.065406in}{2.341666in}}{\pgfqpoint{3.073306in}{2.338394in}}{\pgfqpoint{3.081543in}{2.338394in}}%
\pgfpathclose%
\pgfusepath{stroke,fill}%
\end{pgfscope}%
\begin{pgfscope}%
\pgfpathrectangle{\pgfqpoint{0.100000in}{0.220728in}}{\pgfqpoint{3.696000in}{3.696000in}}%
\pgfusepath{clip}%
\pgfsetbuttcap%
\pgfsetroundjoin%
\definecolor{currentfill}{rgb}{0.121569,0.466667,0.705882}%
\pgfsetfillcolor{currentfill}%
\pgfsetfillopacity{0.775749}%
\pgfsetlinewidth{1.003750pt}%
\definecolor{currentstroke}{rgb}{0.121569,0.466667,0.705882}%
\pgfsetstrokecolor{currentstroke}%
\pgfsetstrokeopacity{0.775749}%
\pgfsetdash{}{0pt}%
\pgfpathmoveto{\pgfqpoint{3.081187in}{2.337541in}}%
\pgfpathcurveto{\pgfqpoint{3.089423in}{2.337541in}}{\pgfqpoint{3.097323in}{2.340813in}}{\pgfqpoint{3.103147in}{2.346637in}}%
\pgfpathcurveto{\pgfqpoint{3.108971in}{2.352461in}}{\pgfqpoint{3.112243in}{2.360361in}}{\pgfqpoint{3.112243in}{2.368597in}}%
\pgfpathcurveto{\pgfqpoint{3.112243in}{2.376833in}}{\pgfqpoint{3.108971in}{2.384734in}}{\pgfqpoint{3.103147in}{2.390557in}}%
\pgfpathcurveto{\pgfqpoint{3.097323in}{2.396381in}}{\pgfqpoint{3.089423in}{2.399654in}}{\pgfqpoint{3.081187in}{2.399654in}}%
\pgfpathcurveto{\pgfqpoint{3.072950in}{2.399654in}}{\pgfqpoint{3.065050in}{2.396381in}}{\pgfqpoint{3.059226in}{2.390557in}}%
\pgfpathcurveto{\pgfqpoint{3.053402in}{2.384734in}}{\pgfqpoint{3.050130in}{2.376833in}}{\pgfqpoint{3.050130in}{2.368597in}}%
\pgfpathcurveto{\pgfqpoint{3.050130in}{2.360361in}}{\pgfqpoint{3.053402in}{2.352461in}}{\pgfqpoint{3.059226in}{2.346637in}}%
\pgfpathcurveto{\pgfqpoint{3.065050in}{2.340813in}}{\pgfqpoint{3.072950in}{2.337541in}}{\pgfqpoint{3.081187in}{2.337541in}}%
\pgfpathclose%
\pgfusepath{stroke,fill}%
\end{pgfscope}%
\begin{pgfscope}%
\pgfpathrectangle{\pgfqpoint{0.100000in}{0.220728in}}{\pgfqpoint{3.696000in}{3.696000in}}%
\pgfusepath{clip}%
\pgfsetbuttcap%
\pgfsetroundjoin%
\definecolor{currentfill}{rgb}{0.121569,0.466667,0.705882}%
\pgfsetfillcolor{currentfill}%
\pgfsetfillopacity{0.775850}%
\pgfsetlinewidth{1.003750pt}%
\definecolor{currentstroke}{rgb}{0.121569,0.466667,0.705882}%
\pgfsetstrokecolor{currentstroke}%
\pgfsetstrokeopacity{0.775850}%
\pgfsetdash{}{0pt}%
\pgfpathmoveto{\pgfqpoint{3.080982in}{2.337095in}}%
\pgfpathcurveto{\pgfqpoint{3.089218in}{2.337095in}}{\pgfqpoint{3.097118in}{2.340367in}}{\pgfqpoint{3.102942in}{2.346191in}}%
\pgfpathcurveto{\pgfqpoint{3.108766in}{2.352015in}}{\pgfqpoint{3.112038in}{2.359915in}}{\pgfqpoint{3.112038in}{2.368151in}}%
\pgfpathcurveto{\pgfqpoint{3.112038in}{2.376388in}}{\pgfqpoint{3.108766in}{2.384288in}}{\pgfqpoint{3.102942in}{2.390112in}}%
\pgfpathcurveto{\pgfqpoint{3.097118in}{2.395936in}}{\pgfqpoint{3.089218in}{2.399208in}}{\pgfqpoint{3.080982in}{2.399208in}}%
\pgfpathcurveto{\pgfqpoint{3.072745in}{2.399208in}}{\pgfqpoint{3.064845in}{2.395936in}}{\pgfqpoint{3.059022in}{2.390112in}}%
\pgfpathcurveto{\pgfqpoint{3.053198in}{2.384288in}}{\pgfqpoint{3.049925in}{2.376388in}}{\pgfqpoint{3.049925in}{2.368151in}}%
\pgfpathcurveto{\pgfqpoint{3.049925in}{2.359915in}}{\pgfqpoint{3.053198in}{2.352015in}}{\pgfqpoint{3.059022in}{2.346191in}}%
\pgfpathcurveto{\pgfqpoint{3.064845in}{2.340367in}}{\pgfqpoint{3.072745in}{2.337095in}}{\pgfqpoint{3.080982in}{2.337095in}}%
\pgfpathclose%
\pgfusepath{stroke,fill}%
\end{pgfscope}%
\begin{pgfscope}%
\pgfpathrectangle{\pgfqpoint{0.100000in}{0.220728in}}{\pgfqpoint{3.696000in}{3.696000in}}%
\pgfusepath{clip}%
\pgfsetbuttcap%
\pgfsetroundjoin%
\definecolor{currentfill}{rgb}{0.121569,0.466667,0.705882}%
\pgfsetfillcolor{currentfill}%
\pgfsetfillopacity{0.775897}%
\pgfsetlinewidth{1.003750pt}%
\definecolor{currentstroke}{rgb}{0.121569,0.466667,0.705882}%
\pgfsetstrokecolor{currentstroke}%
\pgfsetstrokeopacity{0.775897}%
\pgfsetdash{}{0pt}%
\pgfpathmoveto{\pgfqpoint{3.080834in}{2.336858in}}%
\pgfpathcurveto{\pgfqpoint{3.089071in}{2.336858in}}{\pgfqpoint{3.096971in}{2.340131in}}{\pgfqpoint{3.102795in}{2.345954in}}%
\pgfpathcurveto{\pgfqpoint{3.108618in}{2.351778in}}{\pgfqpoint{3.111891in}{2.359678in}}{\pgfqpoint{3.111891in}{2.367915in}}%
\pgfpathcurveto{\pgfqpoint{3.111891in}{2.376151in}}{\pgfqpoint{3.108618in}{2.384051in}}{\pgfqpoint{3.102795in}{2.389875in}}%
\pgfpathcurveto{\pgfqpoint{3.096971in}{2.395699in}}{\pgfqpoint{3.089071in}{2.398971in}}{\pgfqpoint{3.080834in}{2.398971in}}%
\pgfpathcurveto{\pgfqpoint{3.072598in}{2.398971in}}{\pgfqpoint{3.064698in}{2.395699in}}{\pgfqpoint{3.058874in}{2.389875in}}%
\pgfpathcurveto{\pgfqpoint{3.053050in}{2.384051in}}{\pgfqpoint{3.049778in}{2.376151in}}{\pgfqpoint{3.049778in}{2.367915in}}%
\pgfpathcurveto{\pgfqpoint{3.049778in}{2.359678in}}{\pgfqpoint{3.053050in}{2.351778in}}{\pgfqpoint{3.058874in}{2.345954in}}%
\pgfpathcurveto{\pgfqpoint{3.064698in}{2.340131in}}{\pgfqpoint{3.072598in}{2.336858in}}{\pgfqpoint{3.080834in}{2.336858in}}%
\pgfpathclose%
\pgfusepath{stroke,fill}%
\end{pgfscope}%
\begin{pgfscope}%
\pgfpathrectangle{\pgfqpoint{0.100000in}{0.220728in}}{\pgfqpoint{3.696000in}{3.696000in}}%
\pgfusepath{clip}%
\pgfsetbuttcap%
\pgfsetroundjoin%
\definecolor{currentfill}{rgb}{0.121569,0.466667,0.705882}%
\pgfsetfillcolor{currentfill}%
\pgfsetfillopacity{0.776259}%
\pgfsetlinewidth{1.003750pt}%
\definecolor{currentstroke}{rgb}{0.121569,0.466667,0.705882}%
\pgfsetstrokecolor{currentstroke}%
\pgfsetstrokeopacity{0.776259}%
\pgfsetdash{}{0pt}%
\pgfpathmoveto{\pgfqpoint{3.080148in}{2.334951in}}%
\pgfpathcurveto{\pgfqpoint{3.088385in}{2.334951in}}{\pgfqpoint{3.096285in}{2.338223in}}{\pgfqpoint{3.102109in}{2.344047in}}%
\pgfpathcurveto{\pgfqpoint{3.107933in}{2.349871in}}{\pgfqpoint{3.111205in}{2.357771in}}{\pgfqpoint{3.111205in}{2.366007in}}%
\pgfpathcurveto{\pgfqpoint{3.111205in}{2.374244in}}{\pgfqpoint{3.107933in}{2.382144in}}{\pgfqpoint{3.102109in}{2.387968in}}%
\pgfpathcurveto{\pgfqpoint{3.096285in}{2.393792in}}{\pgfqpoint{3.088385in}{2.397064in}}{\pgfqpoint{3.080148in}{2.397064in}}%
\pgfpathcurveto{\pgfqpoint{3.071912in}{2.397064in}}{\pgfqpoint{3.064012in}{2.393792in}}{\pgfqpoint{3.058188in}{2.387968in}}%
\pgfpathcurveto{\pgfqpoint{3.052364in}{2.382144in}}{\pgfqpoint{3.049092in}{2.374244in}}{\pgfqpoint{3.049092in}{2.366007in}}%
\pgfpathcurveto{\pgfqpoint{3.049092in}{2.357771in}}{\pgfqpoint{3.052364in}{2.349871in}}{\pgfqpoint{3.058188in}{2.344047in}}%
\pgfpathcurveto{\pgfqpoint{3.064012in}{2.338223in}}{\pgfqpoint{3.071912in}{2.334951in}}{\pgfqpoint{3.080148in}{2.334951in}}%
\pgfpathclose%
\pgfusepath{stroke,fill}%
\end{pgfscope}%
\begin{pgfscope}%
\pgfpathrectangle{\pgfqpoint{0.100000in}{0.220728in}}{\pgfqpoint{3.696000in}{3.696000in}}%
\pgfusepath{clip}%
\pgfsetbuttcap%
\pgfsetroundjoin%
\definecolor{currentfill}{rgb}{0.121569,0.466667,0.705882}%
\pgfsetfillcolor{currentfill}%
\pgfsetfillopacity{0.776744}%
\pgfsetlinewidth{1.003750pt}%
\definecolor{currentstroke}{rgb}{0.121569,0.466667,0.705882}%
\pgfsetstrokecolor{currentstroke}%
\pgfsetstrokeopacity{0.776744}%
\pgfsetdash{}{0pt}%
\pgfpathmoveto{\pgfqpoint{3.078865in}{2.332248in}}%
\pgfpathcurveto{\pgfqpoint{3.087101in}{2.332248in}}{\pgfqpoint{3.095001in}{2.335520in}}{\pgfqpoint{3.100825in}{2.341344in}}%
\pgfpathcurveto{\pgfqpoint{3.106649in}{2.347168in}}{\pgfqpoint{3.109921in}{2.355068in}}{\pgfqpoint{3.109921in}{2.363304in}}%
\pgfpathcurveto{\pgfqpoint{3.109921in}{2.371541in}}{\pgfqpoint{3.106649in}{2.379441in}}{\pgfqpoint{3.100825in}{2.385265in}}%
\pgfpathcurveto{\pgfqpoint{3.095001in}{2.391088in}}{\pgfqpoint{3.087101in}{2.394361in}}{\pgfqpoint{3.078865in}{2.394361in}}%
\pgfpathcurveto{\pgfqpoint{3.070629in}{2.394361in}}{\pgfqpoint{3.062729in}{2.391088in}}{\pgfqpoint{3.056905in}{2.385265in}}%
\pgfpathcurveto{\pgfqpoint{3.051081in}{2.379441in}}{\pgfqpoint{3.047808in}{2.371541in}}{\pgfqpoint{3.047808in}{2.363304in}}%
\pgfpathcurveto{\pgfqpoint{3.047808in}{2.355068in}}{\pgfqpoint{3.051081in}{2.347168in}}{\pgfqpoint{3.056905in}{2.341344in}}%
\pgfpathcurveto{\pgfqpoint{3.062729in}{2.335520in}}{\pgfqpoint{3.070629in}{2.332248in}}{\pgfqpoint{3.078865in}{2.332248in}}%
\pgfpathclose%
\pgfusepath{stroke,fill}%
\end{pgfscope}%
\begin{pgfscope}%
\pgfpathrectangle{\pgfqpoint{0.100000in}{0.220728in}}{\pgfqpoint{3.696000in}{3.696000in}}%
\pgfusepath{clip}%
\pgfsetbuttcap%
\pgfsetroundjoin%
\definecolor{currentfill}{rgb}{0.121569,0.466667,0.705882}%
\pgfsetfillcolor{currentfill}%
\pgfsetfillopacity{0.777412}%
\pgfsetlinewidth{1.003750pt}%
\definecolor{currentstroke}{rgb}{0.121569,0.466667,0.705882}%
\pgfsetstrokecolor{currentstroke}%
\pgfsetstrokeopacity{0.777412}%
\pgfsetdash{}{0pt}%
\pgfpathmoveto{\pgfqpoint{3.077020in}{2.329550in}}%
\pgfpathcurveto{\pgfqpoint{3.085256in}{2.329550in}}{\pgfqpoint{3.093156in}{2.332823in}}{\pgfqpoint{3.098980in}{2.338646in}}%
\pgfpathcurveto{\pgfqpoint{3.104804in}{2.344470in}}{\pgfqpoint{3.108076in}{2.352370in}}{\pgfqpoint{3.108076in}{2.360607in}}%
\pgfpathcurveto{\pgfqpoint{3.108076in}{2.368843in}}{\pgfqpoint{3.104804in}{2.376743in}}{\pgfqpoint{3.098980in}{2.382567in}}%
\pgfpathcurveto{\pgfqpoint{3.093156in}{2.388391in}}{\pgfqpoint{3.085256in}{2.391663in}}{\pgfqpoint{3.077020in}{2.391663in}}%
\pgfpathcurveto{\pgfqpoint{3.068783in}{2.391663in}}{\pgfqpoint{3.060883in}{2.388391in}}{\pgfqpoint{3.055059in}{2.382567in}}%
\pgfpathcurveto{\pgfqpoint{3.049235in}{2.376743in}}{\pgfqpoint{3.045963in}{2.368843in}}{\pgfqpoint{3.045963in}{2.360607in}}%
\pgfpathcurveto{\pgfqpoint{3.045963in}{2.352370in}}{\pgfqpoint{3.049235in}{2.344470in}}{\pgfqpoint{3.055059in}{2.338646in}}%
\pgfpathcurveto{\pgfqpoint{3.060883in}{2.332823in}}{\pgfqpoint{3.068783in}{2.329550in}}{\pgfqpoint{3.077020in}{2.329550in}}%
\pgfpathclose%
\pgfusepath{stroke,fill}%
\end{pgfscope}%
\begin{pgfscope}%
\pgfpathrectangle{\pgfqpoint{0.100000in}{0.220728in}}{\pgfqpoint{3.696000in}{3.696000in}}%
\pgfusepath{clip}%
\pgfsetbuttcap%
\pgfsetroundjoin%
\definecolor{currentfill}{rgb}{0.121569,0.466667,0.705882}%
\pgfsetfillcolor{currentfill}%
\pgfsetfillopacity{0.777533}%
\pgfsetlinewidth{1.003750pt}%
\definecolor{currentstroke}{rgb}{0.121569,0.466667,0.705882}%
\pgfsetstrokecolor{currentstroke}%
\pgfsetstrokeopacity{0.777533}%
\pgfsetdash{}{0pt}%
\pgfpathmoveto{\pgfqpoint{1.220763in}{1.177313in}}%
\pgfpathcurveto{\pgfqpoint{1.228999in}{1.177313in}}{\pgfqpoint{1.236899in}{1.180586in}}{\pgfqpoint{1.242723in}{1.186410in}}%
\pgfpathcurveto{\pgfqpoint{1.248547in}{1.192234in}}{\pgfqpoint{1.251820in}{1.200134in}}{\pgfqpoint{1.251820in}{1.208370in}}%
\pgfpathcurveto{\pgfqpoint{1.251820in}{1.216606in}}{\pgfqpoint{1.248547in}{1.224506in}}{\pgfqpoint{1.242723in}{1.230330in}}%
\pgfpathcurveto{\pgfqpoint{1.236899in}{1.236154in}}{\pgfqpoint{1.228999in}{1.239426in}}{\pgfqpoint{1.220763in}{1.239426in}}%
\pgfpathcurveto{\pgfqpoint{1.212527in}{1.239426in}}{\pgfqpoint{1.204627in}{1.236154in}}{\pgfqpoint{1.198803in}{1.230330in}}%
\pgfpathcurveto{\pgfqpoint{1.192979in}{1.224506in}}{\pgfqpoint{1.189707in}{1.216606in}}{\pgfqpoint{1.189707in}{1.208370in}}%
\pgfpathcurveto{\pgfqpoint{1.189707in}{1.200134in}}{\pgfqpoint{1.192979in}{1.192234in}}{\pgfqpoint{1.198803in}{1.186410in}}%
\pgfpathcurveto{\pgfqpoint{1.204627in}{1.180586in}}{\pgfqpoint{1.212527in}{1.177313in}}{\pgfqpoint{1.220763in}{1.177313in}}%
\pgfpathclose%
\pgfusepath{stroke,fill}%
\end{pgfscope}%
\begin{pgfscope}%
\pgfpathrectangle{\pgfqpoint{0.100000in}{0.220728in}}{\pgfqpoint{3.696000in}{3.696000in}}%
\pgfusepath{clip}%
\pgfsetbuttcap%
\pgfsetroundjoin%
\definecolor{currentfill}{rgb}{0.121569,0.466667,0.705882}%
\pgfsetfillcolor{currentfill}%
\pgfsetfillopacity{0.778257}%
\pgfsetlinewidth{1.003750pt}%
\definecolor{currentstroke}{rgb}{0.121569,0.466667,0.705882}%
\pgfsetstrokecolor{currentstroke}%
\pgfsetstrokeopacity{0.778257}%
\pgfsetdash{}{0pt}%
\pgfpathmoveto{\pgfqpoint{3.075530in}{2.323517in}}%
\pgfpathcurveto{\pgfqpoint{3.083767in}{2.323517in}}{\pgfqpoint{3.091667in}{2.326790in}}{\pgfqpoint{3.097491in}{2.332614in}}%
\pgfpathcurveto{\pgfqpoint{3.103315in}{2.338438in}}{\pgfqpoint{3.106587in}{2.346338in}}{\pgfqpoint{3.106587in}{2.354574in}}%
\pgfpathcurveto{\pgfqpoint{3.106587in}{2.362810in}}{\pgfqpoint{3.103315in}{2.370710in}}{\pgfqpoint{3.097491in}{2.376534in}}%
\pgfpathcurveto{\pgfqpoint{3.091667in}{2.382358in}}{\pgfqpoint{3.083767in}{2.385630in}}{\pgfqpoint{3.075530in}{2.385630in}}%
\pgfpathcurveto{\pgfqpoint{3.067294in}{2.385630in}}{\pgfqpoint{3.059394in}{2.382358in}}{\pgfqpoint{3.053570in}{2.376534in}}%
\pgfpathcurveto{\pgfqpoint{3.047746in}{2.370710in}}{\pgfqpoint{3.044474in}{2.362810in}}{\pgfqpoint{3.044474in}{2.354574in}}%
\pgfpathcurveto{\pgfqpoint{3.044474in}{2.346338in}}{\pgfqpoint{3.047746in}{2.338438in}}{\pgfqpoint{3.053570in}{2.332614in}}%
\pgfpathcurveto{\pgfqpoint{3.059394in}{2.326790in}}{\pgfqpoint{3.067294in}{2.323517in}}{\pgfqpoint{3.075530in}{2.323517in}}%
\pgfpathclose%
\pgfusepath{stroke,fill}%
\end{pgfscope}%
\begin{pgfscope}%
\pgfpathrectangle{\pgfqpoint{0.100000in}{0.220728in}}{\pgfqpoint{3.696000in}{3.696000in}}%
\pgfusepath{clip}%
\pgfsetbuttcap%
\pgfsetroundjoin%
\definecolor{currentfill}{rgb}{0.121569,0.466667,0.705882}%
\pgfsetfillcolor{currentfill}%
\pgfsetfillopacity{0.779141}%
\pgfsetlinewidth{1.003750pt}%
\definecolor{currentstroke}{rgb}{0.121569,0.466667,0.705882}%
\pgfsetstrokecolor{currentstroke}%
\pgfsetstrokeopacity{0.779141}%
\pgfsetdash{}{0pt}%
\pgfpathmoveto{\pgfqpoint{1.234312in}{1.167914in}}%
\pgfpathcurveto{\pgfqpoint{1.242548in}{1.167914in}}{\pgfqpoint{1.250448in}{1.171186in}}{\pgfqpoint{1.256272in}{1.177010in}}%
\pgfpathcurveto{\pgfqpoint{1.262096in}{1.182834in}}{\pgfqpoint{1.265368in}{1.190734in}}{\pgfqpoint{1.265368in}{1.198971in}}%
\pgfpathcurveto{\pgfqpoint{1.265368in}{1.207207in}}{\pgfqpoint{1.262096in}{1.215107in}}{\pgfqpoint{1.256272in}{1.220931in}}%
\pgfpathcurveto{\pgfqpoint{1.250448in}{1.226755in}}{\pgfqpoint{1.242548in}{1.230027in}}{\pgfqpoint{1.234312in}{1.230027in}}%
\pgfpathcurveto{\pgfqpoint{1.226075in}{1.230027in}}{\pgfqpoint{1.218175in}{1.226755in}}{\pgfqpoint{1.212351in}{1.220931in}}%
\pgfpathcurveto{\pgfqpoint{1.206528in}{1.215107in}}{\pgfqpoint{1.203255in}{1.207207in}}{\pgfqpoint{1.203255in}{1.198971in}}%
\pgfpathcurveto{\pgfqpoint{1.203255in}{1.190734in}}{\pgfqpoint{1.206528in}{1.182834in}}{\pgfqpoint{1.212351in}{1.177010in}}%
\pgfpathcurveto{\pgfqpoint{1.218175in}{1.171186in}}{\pgfqpoint{1.226075in}{1.167914in}}{\pgfqpoint{1.234312in}{1.167914in}}%
\pgfpathclose%
\pgfusepath{stroke,fill}%
\end{pgfscope}%
\begin{pgfscope}%
\pgfpathrectangle{\pgfqpoint{0.100000in}{0.220728in}}{\pgfqpoint{3.696000in}{3.696000in}}%
\pgfusepath{clip}%
\pgfsetbuttcap%
\pgfsetroundjoin%
\definecolor{currentfill}{rgb}{0.121569,0.466667,0.705882}%
\pgfsetfillcolor{currentfill}%
\pgfsetfillopacity{0.779378}%
\pgfsetlinewidth{1.003750pt}%
\definecolor{currentstroke}{rgb}{0.121569,0.466667,0.705882}%
\pgfsetstrokecolor{currentstroke}%
\pgfsetstrokeopacity{0.779378}%
\pgfsetdash{}{0pt}%
\pgfpathmoveto{\pgfqpoint{3.072763in}{2.318252in}}%
\pgfpathcurveto{\pgfqpoint{3.080999in}{2.318252in}}{\pgfqpoint{3.088899in}{2.321525in}}{\pgfqpoint{3.094723in}{2.327349in}}%
\pgfpathcurveto{\pgfqpoint{3.100547in}{2.333172in}}{\pgfqpoint{3.103819in}{2.341073in}}{\pgfqpoint{3.103819in}{2.349309in}}%
\pgfpathcurveto{\pgfqpoint{3.103819in}{2.357545in}}{\pgfqpoint{3.100547in}{2.365445in}}{\pgfqpoint{3.094723in}{2.371269in}}%
\pgfpathcurveto{\pgfqpoint{3.088899in}{2.377093in}}{\pgfqpoint{3.080999in}{2.380365in}}{\pgfqpoint{3.072763in}{2.380365in}}%
\pgfpathcurveto{\pgfqpoint{3.064526in}{2.380365in}}{\pgfqpoint{3.056626in}{2.377093in}}{\pgfqpoint{3.050802in}{2.371269in}}%
\pgfpathcurveto{\pgfqpoint{3.044978in}{2.365445in}}{\pgfqpoint{3.041706in}{2.357545in}}{\pgfqpoint{3.041706in}{2.349309in}}%
\pgfpathcurveto{\pgfqpoint{3.041706in}{2.341073in}}{\pgfqpoint{3.044978in}{2.333172in}}{\pgfqpoint{3.050802in}{2.327349in}}%
\pgfpathcurveto{\pgfqpoint{3.056626in}{2.321525in}}{\pgfqpoint{3.064526in}{2.318252in}}{\pgfqpoint{3.072763in}{2.318252in}}%
\pgfpathclose%
\pgfusepath{stroke,fill}%
\end{pgfscope}%
\begin{pgfscope}%
\pgfpathrectangle{\pgfqpoint{0.100000in}{0.220728in}}{\pgfqpoint{3.696000in}{3.696000in}}%
\pgfusepath{clip}%
\pgfsetbuttcap%
\pgfsetroundjoin%
\definecolor{currentfill}{rgb}{0.121569,0.466667,0.705882}%
\pgfsetfillcolor{currentfill}%
\pgfsetfillopacity{0.779924}%
\pgfsetlinewidth{1.003750pt}%
\definecolor{currentstroke}{rgb}{0.121569,0.466667,0.705882}%
\pgfsetstrokecolor{currentstroke}%
\pgfsetstrokeopacity{0.779924}%
\pgfsetdash{}{0pt}%
\pgfpathmoveto{\pgfqpoint{3.070990in}{2.315396in}}%
\pgfpathcurveto{\pgfqpoint{3.079226in}{2.315396in}}{\pgfqpoint{3.087126in}{2.318668in}}{\pgfqpoint{3.092950in}{2.324492in}}%
\pgfpathcurveto{\pgfqpoint{3.098774in}{2.330316in}}{\pgfqpoint{3.102047in}{2.338216in}}{\pgfqpoint{3.102047in}{2.346452in}}%
\pgfpathcurveto{\pgfqpoint{3.102047in}{2.354689in}}{\pgfqpoint{3.098774in}{2.362589in}}{\pgfqpoint{3.092950in}{2.368413in}}%
\pgfpathcurveto{\pgfqpoint{3.087126in}{2.374237in}}{\pgfqpoint{3.079226in}{2.377509in}}{\pgfqpoint{3.070990in}{2.377509in}}%
\pgfpathcurveto{\pgfqpoint{3.062754in}{2.377509in}}{\pgfqpoint{3.054854in}{2.374237in}}{\pgfqpoint{3.049030in}{2.368413in}}%
\pgfpathcurveto{\pgfqpoint{3.043206in}{2.362589in}}{\pgfqpoint{3.039934in}{2.354689in}}{\pgfqpoint{3.039934in}{2.346452in}}%
\pgfpathcurveto{\pgfqpoint{3.039934in}{2.338216in}}{\pgfqpoint{3.043206in}{2.330316in}}{\pgfqpoint{3.049030in}{2.324492in}}%
\pgfpathcurveto{\pgfqpoint{3.054854in}{2.318668in}}{\pgfqpoint{3.062754in}{2.315396in}}{\pgfqpoint{3.070990in}{2.315396in}}%
\pgfpathclose%
\pgfusepath{stroke,fill}%
\end{pgfscope}%
\begin{pgfscope}%
\pgfpathrectangle{\pgfqpoint{0.100000in}{0.220728in}}{\pgfqpoint{3.696000in}{3.696000in}}%
\pgfusepath{clip}%
\pgfsetbuttcap%
\pgfsetroundjoin%
\definecolor{currentfill}{rgb}{0.121569,0.466667,0.705882}%
\pgfsetfillcolor{currentfill}%
\pgfsetfillopacity{0.780721}%
\pgfsetlinewidth{1.003750pt}%
\definecolor{currentstroke}{rgb}{0.121569,0.466667,0.705882}%
\pgfsetstrokecolor{currentstroke}%
\pgfsetstrokeopacity{0.780721}%
\pgfsetdash{}{0pt}%
\pgfpathmoveto{\pgfqpoint{3.069741in}{2.311112in}}%
\pgfpathcurveto{\pgfqpoint{3.077978in}{2.311112in}}{\pgfqpoint{3.085878in}{2.314385in}}{\pgfqpoint{3.091702in}{2.320209in}}%
\pgfpathcurveto{\pgfqpoint{3.097526in}{2.326032in}}{\pgfqpoint{3.100798in}{2.333932in}}{\pgfqpoint{3.100798in}{2.342169in}}%
\pgfpathcurveto{\pgfqpoint{3.100798in}{2.350405in}}{\pgfqpoint{3.097526in}{2.358305in}}{\pgfqpoint{3.091702in}{2.364129in}}%
\pgfpathcurveto{\pgfqpoint{3.085878in}{2.369953in}}{\pgfqpoint{3.077978in}{2.373225in}}{\pgfqpoint{3.069741in}{2.373225in}}%
\pgfpathcurveto{\pgfqpoint{3.061505in}{2.373225in}}{\pgfqpoint{3.053605in}{2.369953in}}{\pgfqpoint{3.047781in}{2.364129in}}%
\pgfpathcurveto{\pgfqpoint{3.041957in}{2.358305in}}{\pgfqpoint{3.038685in}{2.350405in}}{\pgfqpoint{3.038685in}{2.342169in}}%
\pgfpathcurveto{\pgfqpoint{3.038685in}{2.333932in}}{\pgfqpoint{3.041957in}{2.326032in}}{\pgfqpoint{3.047781in}{2.320209in}}%
\pgfpathcurveto{\pgfqpoint{3.053605in}{2.314385in}}{\pgfqpoint{3.061505in}{2.311112in}}{\pgfqpoint{3.069741in}{2.311112in}}%
\pgfpathclose%
\pgfusepath{stroke,fill}%
\end{pgfscope}%
\begin{pgfscope}%
\pgfpathrectangle{\pgfqpoint{0.100000in}{0.220728in}}{\pgfqpoint{3.696000in}{3.696000in}}%
\pgfusepath{clip}%
\pgfsetbuttcap%
\pgfsetroundjoin%
\definecolor{currentfill}{rgb}{0.121569,0.466667,0.705882}%
\pgfsetfillcolor{currentfill}%
\pgfsetfillopacity{0.781746}%
\pgfsetlinewidth{1.003750pt}%
\definecolor{currentstroke}{rgb}{0.121569,0.466667,0.705882}%
\pgfsetstrokecolor{currentstroke}%
\pgfsetstrokeopacity{0.781746}%
\pgfsetdash{}{0pt}%
\pgfpathmoveto{\pgfqpoint{3.066509in}{2.305165in}}%
\pgfpathcurveto{\pgfqpoint{3.074745in}{2.305165in}}{\pgfqpoint{3.082645in}{2.308437in}}{\pgfqpoint{3.088469in}{2.314261in}}%
\pgfpathcurveto{\pgfqpoint{3.094293in}{2.320085in}}{\pgfqpoint{3.097565in}{2.327985in}}{\pgfqpoint{3.097565in}{2.336222in}}%
\pgfpathcurveto{\pgfqpoint{3.097565in}{2.344458in}}{\pgfqpoint{3.094293in}{2.352358in}}{\pgfqpoint{3.088469in}{2.358182in}}%
\pgfpathcurveto{\pgfqpoint{3.082645in}{2.364006in}}{\pgfqpoint{3.074745in}{2.367278in}}{\pgfqpoint{3.066509in}{2.367278in}}%
\pgfpathcurveto{\pgfqpoint{3.058272in}{2.367278in}}{\pgfqpoint{3.050372in}{2.364006in}}{\pgfqpoint{3.044548in}{2.358182in}}%
\pgfpathcurveto{\pgfqpoint{3.038724in}{2.352358in}}{\pgfqpoint{3.035452in}{2.344458in}}{\pgfqpoint{3.035452in}{2.336222in}}%
\pgfpathcurveto{\pgfqpoint{3.035452in}{2.327985in}}{\pgfqpoint{3.038724in}{2.320085in}}{\pgfqpoint{3.044548in}{2.314261in}}%
\pgfpathcurveto{\pgfqpoint{3.050372in}{2.308437in}}{\pgfqpoint{3.058272in}{2.305165in}}{\pgfqpoint{3.066509in}{2.305165in}}%
\pgfpathclose%
\pgfusepath{stroke,fill}%
\end{pgfscope}%
\begin{pgfscope}%
\pgfpathrectangle{\pgfqpoint{0.100000in}{0.220728in}}{\pgfqpoint{3.696000in}{3.696000in}}%
\pgfusepath{clip}%
\pgfsetbuttcap%
\pgfsetroundjoin%
\definecolor{currentfill}{rgb}{0.121569,0.466667,0.705882}%
\pgfsetfillcolor{currentfill}%
\pgfsetfillopacity{0.782334}%
\pgfsetlinewidth{1.003750pt}%
\definecolor{currentstroke}{rgb}{0.121569,0.466667,0.705882}%
\pgfsetstrokecolor{currentstroke}%
\pgfsetstrokeopacity{0.782334}%
\pgfsetdash{}{0pt}%
\pgfpathmoveto{\pgfqpoint{3.064713in}{2.302022in}}%
\pgfpathcurveto{\pgfqpoint{3.072949in}{2.302022in}}{\pgfqpoint{3.080849in}{2.305294in}}{\pgfqpoint{3.086673in}{2.311118in}}%
\pgfpathcurveto{\pgfqpoint{3.092497in}{2.316942in}}{\pgfqpoint{3.095769in}{2.324842in}}{\pgfqpoint{3.095769in}{2.333079in}}%
\pgfpathcurveto{\pgfqpoint{3.095769in}{2.341315in}}{\pgfqpoint{3.092497in}{2.349215in}}{\pgfqpoint{3.086673in}{2.355039in}}%
\pgfpathcurveto{\pgfqpoint{3.080849in}{2.360863in}}{\pgfqpoint{3.072949in}{2.364135in}}{\pgfqpoint{3.064713in}{2.364135in}}%
\pgfpathcurveto{\pgfqpoint{3.056477in}{2.364135in}}{\pgfqpoint{3.048577in}{2.360863in}}{\pgfqpoint{3.042753in}{2.355039in}}%
\pgfpathcurveto{\pgfqpoint{3.036929in}{2.349215in}}{\pgfqpoint{3.033656in}{2.341315in}}{\pgfqpoint{3.033656in}{2.333079in}}%
\pgfpathcurveto{\pgfqpoint{3.033656in}{2.324842in}}{\pgfqpoint{3.036929in}{2.316942in}}{\pgfqpoint{3.042753in}{2.311118in}}%
\pgfpathcurveto{\pgfqpoint{3.048577in}{2.305294in}}{\pgfqpoint{3.056477in}{2.302022in}}{\pgfqpoint{3.064713in}{2.302022in}}%
\pgfpathclose%
\pgfusepath{stroke,fill}%
\end{pgfscope}%
\begin{pgfscope}%
\pgfpathrectangle{\pgfqpoint{0.100000in}{0.220728in}}{\pgfqpoint{3.696000in}{3.696000in}}%
\pgfusepath{clip}%
\pgfsetbuttcap%
\pgfsetroundjoin%
\definecolor{currentfill}{rgb}{0.121569,0.466667,0.705882}%
\pgfsetfillcolor{currentfill}%
\pgfsetfillopacity{0.782380}%
\pgfsetlinewidth{1.003750pt}%
\definecolor{currentstroke}{rgb}{0.121569,0.466667,0.705882}%
\pgfsetstrokecolor{currentstroke}%
\pgfsetstrokeopacity{0.782380}%
\pgfsetdash{}{0pt}%
\pgfpathmoveto{\pgfqpoint{1.259436in}{1.153875in}}%
\pgfpathcurveto{\pgfqpoint{1.267673in}{1.153875in}}{\pgfqpoint{1.275573in}{1.157148in}}{\pgfqpoint{1.281397in}{1.162972in}}%
\pgfpathcurveto{\pgfqpoint{1.287221in}{1.168796in}}{\pgfqpoint{1.290493in}{1.176696in}}{\pgfqpoint{1.290493in}{1.184932in}}%
\pgfpathcurveto{\pgfqpoint{1.290493in}{1.193168in}}{\pgfqpoint{1.287221in}{1.201068in}}{\pgfqpoint{1.281397in}{1.206892in}}%
\pgfpathcurveto{\pgfqpoint{1.275573in}{1.212716in}}{\pgfqpoint{1.267673in}{1.215988in}}{\pgfqpoint{1.259436in}{1.215988in}}%
\pgfpathcurveto{\pgfqpoint{1.251200in}{1.215988in}}{\pgfqpoint{1.243300in}{1.212716in}}{\pgfqpoint{1.237476in}{1.206892in}}%
\pgfpathcurveto{\pgfqpoint{1.231652in}{1.201068in}}{\pgfqpoint{1.228380in}{1.193168in}}{\pgfqpoint{1.228380in}{1.184932in}}%
\pgfpathcurveto{\pgfqpoint{1.228380in}{1.176696in}}{\pgfqpoint{1.231652in}{1.168796in}}{\pgfqpoint{1.237476in}{1.162972in}}%
\pgfpathcurveto{\pgfqpoint{1.243300in}{1.157148in}}{\pgfqpoint{1.251200in}{1.153875in}}{\pgfqpoint{1.259436in}{1.153875in}}%
\pgfpathclose%
\pgfusepath{stroke,fill}%
\end{pgfscope}%
\begin{pgfscope}%
\pgfpathrectangle{\pgfqpoint{0.100000in}{0.220728in}}{\pgfqpoint{3.696000in}{3.696000in}}%
\pgfusepath{clip}%
\pgfsetbuttcap%
\pgfsetroundjoin%
\definecolor{currentfill}{rgb}{0.121569,0.466667,0.705882}%
\pgfsetfillcolor{currentfill}%
\pgfsetfillopacity{0.782691}%
\pgfsetlinewidth{1.003750pt}%
\definecolor{currentstroke}{rgb}{0.121569,0.466667,0.705882}%
\pgfsetstrokecolor{currentstroke}%
\pgfsetstrokeopacity{0.782691}%
\pgfsetdash{}{0pt}%
\pgfpathmoveto{\pgfqpoint{3.064026in}{2.300105in}}%
\pgfpathcurveto{\pgfqpoint{3.072262in}{2.300105in}}{\pgfqpoint{3.080162in}{2.303377in}}{\pgfqpoint{3.085986in}{2.309201in}}%
\pgfpathcurveto{\pgfqpoint{3.091810in}{2.315025in}}{\pgfqpoint{3.095082in}{2.322925in}}{\pgfqpoint{3.095082in}{2.331161in}}%
\pgfpathcurveto{\pgfqpoint{3.095082in}{2.339398in}}{\pgfqpoint{3.091810in}{2.347298in}}{\pgfqpoint{3.085986in}{2.353122in}}%
\pgfpathcurveto{\pgfqpoint{3.080162in}{2.358946in}}{\pgfqpoint{3.072262in}{2.362218in}}{\pgfqpoint{3.064026in}{2.362218in}}%
\pgfpathcurveto{\pgfqpoint{3.055789in}{2.362218in}}{\pgfqpoint{3.047889in}{2.358946in}}{\pgfqpoint{3.042065in}{2.353122in}}%
\pgfpathcurveto{\pgfqpoint{3.036241in}{2.347298in}}{\pgfqpoint{3.032969in}{2.339398in}}{\pgfqpoint{3.032969in}{2.331161in}}%
\pgfpathcurveto{\pgfqpoint{3.032969in}{2.322925in}}{\pgfqpoint{3.036241in}{2.315025in}}{\pgfqpoint{3.042065in}{2.309201in}}%
\pgfpathcurveto{\pgfqpoint{3.047889in}{2.303377in}}{\pgfqpoint{3.055789in}{2.300105in}}{\pgfqpoint{3.064026in}{2.300105in}}%
\pgfpathclose%
\pgfusepath{stroke,fill}%
\end{pgfscope}%
\begin{pgfscope}%
\pgfpathrectangle{\pgfqpoint{0.100000in}{0.220728in}}{\pgfqpoint{3.696000in}{3.696000in}}%
\pgfusepath{clip}%
\pgfsetbuttcap%
\pgfsetroundjoin%
\definecolor{currentfill}{rgb}{0.121569,0.466667,0.705882}%
\pgfsetfillcolor{currentfill}%
\pgfsetfillopacity{0.783129}%
\pgfsetlinewidth{1.003750pt}%
\definecolor{currentstroke}{rgb}{0.121569,0.466667,0.705882}%
\pgfsetstrokecolor{currentstroke}%
\pgfsetstrokeopacity{0.783129}%
\pgfsetdash{}{0pt}%
\pgfpathmoveto{\pgfqpoint{3.062574in}{2.297534in}}%
\pgfpathcurveto{\pgfqpoint{3.070810in}{2.297534in}}{\pgfqpoint{3.078710in}{2.300806in}}{\pgfqpoint{3.084534in}{2.306630in}}%
\pgfpathcurveto{\pgfqpoint{3.090358in}{2.312454in}}{\pgfqpoint{3.093630in}{2.320354in}}{\pgfqpoint{3.093630in}{2.328590in}}%
\pgfpathcurveto{\pgfqpoint{3.093630in}{2.336826in}}{\pgfqpoint{3.090358in}{2.344726in}}{\pgfqpoint{3.084534in}{2.350550in}}%
\pgfpathcurveto{\pgfqpoint{3.078710in}{2.356374in}}{\pgfqpoint{3.070810in}{2.359647in}}{\pgfqpoint{3.062574in}{2.359647in}}%
\pgfpathcurveto{\pgfqpoint{3.054337in}{2.359647in}}{\pgfqpoint{3.046437in}{2.356374in}}{\pgfqpoint{3.040613in}{2.350550in}}%
\pgfpathcurveto{\pgfqpoint{3.034790in}{2.344726in}}{\pgfqpoint{3.031517in}{2.336826in}}{\pgfqpoint{3.031517in}{2.328590in}}%
\pgfpathcurveto{\pgfqpoint{3.031517in}{2.320354in}}{\pgfqpoint{3.034790in}{2.312454in}}{\pgfqpoint{3.040613in}{2.306630in}}%
\pgfpathcurveto{\pgfqpoint{3.046437in}{2.300806in}}{\pgfqpoint{3.054337in}{2.297534in}}{\pgfqpoint{3.062574in}{2.297534in}}%
\pgfpathclose%
\pgfusepath{stroke,fill}%
\end{pgfscope}%
\begin{pgfscope}%
\pgfpathrectangle{\pgfqpoint{0.100000in}{0.220728in}}{\pgfqpoint{3.696000in}{3.696000in}}%
\pgfusepath{clip}%
\pgfsetbuttcap%
\pgfsetroundjoin%
\definecolor{currentfill}{rgb}{0.121569,0.466667,0.705882}%
\pgfsetfillcolor{currentfill}%
\pgfsetfillopacity{0.783883}%
\pgfsetlinewidth{1.003750pt}%
\definecolor{currentstroke}{rgb}{0.121569,0.466667,0.705882}%
\pgfsetstrokecolor{currentstroke}%
\pgfsetstrokeopacity{0.783883}%
\pgfsetdash{}{0pt}%
\pgfpathmoveto{\pgfqpoint{3.060814in}{2.293712in}}%
\pgfpathcurveto{\pgfqpoint{3.069050in}{2.293712in}}{\pgfqpoint{3.076950in}{2.296984in}}{\pgfqpoint{3.082774in}{2.302808in}}%
\pgfpathcurveto{\pgfqpoint{3.088598in}{2.308632in}}{\pgfqpoint{3.091870in}{2.316532in}}{\pgfqpoint{3.091870in}{2.324768in}}%
\pgfpathcurveto{\pgfqpoint{3.091870in}{2.333004in}}{\pgfqpoint{3.088598in}{2.340904in}}{\pgfqpoint{3.082774in}{2.346728in}}%
\pgfpathcurveto{\pgfqpoint{3.076950in}{2.352552in}}{\pgfqpoint{3.069050in}{2.355825in}}{\pgfqpoint{3.060814in}{2.355825in}}%
\pgfpathcurveto{\pgfqpoint{3.052578in}{2.355825in}}{\pgfqpoint{3.044677in}{2.352552in}}{\pgfqpoint{3.038854in}{2.346728in}}%
\pgfpathcurveto{\pgfqpoint{3.033030in}{2.340904in}}{\pgfqpoint{3.029757in}{2.333004in}}{\pgfqpoint{3.029757in}{2.324768in}}%
\pgfpathcurveto{\pgfqpoint{3.029757in}{2.316532in}}{\pgfqpoint{3.033030in}{2.308632in}}{\pgfqpoint{3.038854in}{2.302808in}}%
\pgfpathcurveto{\pgfqpoint{3.044677in}{2.296984in}}{\pgfqpoint{3.052578in}{2.293712in}}{\pgfqpoint{3.060814in}{2.293712in}}%
\pgfpathclose%
\pgfusepath{stroke,fill}%
\end{pgfscope}%
\begin{pgfscope}%
\pgfpathrectangle{\pgfqpoint{0.100000in}{0.220728in}}{\pgfqpoint{3.696000in}{3.696000in}}%
\pgfusepath{clip}%
\pgfsetbuttcap%
\pgfsetroundjoin%
\definecolor{currentfill}{rgb}{0.121569,0.466667,0.705882}%
\pgfsetfillcolor{currentfill}%
\pgfsetfillopacity{0.784316}%
\pgfsetlinewidth{1.003750pt}%
\definecolor{currentstroke}{rgb}{0.121569,0.466667,0.705882}%
\pgfsetstrokecolor{currentstroke}%
\pgfsetstrokeopacity{0.784316}%
\pgfsetdash{}{0pt}%
\pgfpathmoveto{\pgfqpoint{3.059969in}{2.291567in}}%
\pgfpathcurveto{\pgfqpoint{3.068205in}{2.291567in}}{\pgfqpoint{3.076105in}{2.294839in}}{\pgfqpoint{3.081929in}{2.300663in}}%
\pgfpathcurveto{\pgfqpoint{3.087753in}{2.306487in}}{\pgfqpoint{3.091025in}{2.314387in}}{\pgfqpoint{3.091025in}{2.322623in}}%
\pgfpathcurveto{\pgfqpoint{3.091025in}{2.330860in}}{\pgfqpoint{3.087753in}{2.338760in}}{\pgfqpoint{3.081929in}{2.344584in}}%
\pgfpathcurveto{\pgfqpoint{3.076105in}{2.350408in}}{\pgfqpoint{3.068205in}{2.353680in}}{\pgfqpoint{3.059969in}{2.353680in}}%
\pgfpathcurveto{\pgfqpoint{3.051732in}{2.353680in}}{\pgfqpoint{3.043832in}{2.350408in}}{\pgfqpoint{3.038008in}{2.344584in}}%
\pgfpathcurveto{\pgfqpoint{3.032185in}{2.338760in}}{\pgfqpoint{3.028912in}{2.330860in}}{\pgfqpoint{3.028912in}{2.322623in}}%
\pgfpathcurveto{\pgfqpoint{3.028912in}{2.314387in}}{\pgfqpoint{3.032185in}{2.306487in}}{\pgfqpoint{3.038008in}{2.300663in}}%
\pgfpathcurveto{\pgfqpoint{3.043832in}{2.294839in}}{\pgfqpoint{3.051732in}{2.291567in}}{\pgfqpoint{3.059969in}{2.291567in}}%
\pgfpathclose%
\pgfusepath{stroke,fill}%
\end{pgfscope}%
\begin{pgfscope}%
\pgfpathrectangle{\pgfqpoint{0.100000in}{0.220728in}}{\pgfqpoint{3.696000in}{3.696000in}}%
\pgfusepath{clip}%
\pgfsetbuttcap%
\pgfsetroundjoin%
\definecolor{currentfill}{rgb}{0.121569,0.466667,0.705882}%
\pgfsetfillcolor{currentfill}%
\pgfsetfillopacity{0.784504}%
\pgfsetlinewidth{1.003750pt}%
\definecolor{currentstroke}{rgb}{0.121569,0.466667,0.705882}%
\pgfsetstrokecolor{currentstroke}%
\pgfsetstrokeopacity{0.784504}%
\pgfsetdash{}{0pt}%
\pgfpathmoveto{\pgfqpoint{3.059269in}{2.290462in}}%
\pgfpathcurveto{\pgfqpoint{3.067506in}{2.290462in}}{\pgfqpoint{3.075406in}{2.293735in}}{\pgfqpoint{3.081230in}{2.299559in}}%
\pgfpathcurveto{\pgfqpoint{3.087053in}{2.305383in}}{\pgfqpoint{3.090326in}{2.313283in}}{\pgfqpoint{3.090326in}{2.321519in}}%
\pgfpathcurveto{\pgfqpoint{3.090326in}{2.329755in}}{\pgfqpoint{3.087053in}{2.337655in}}{\pgfqpoint{3.081230in}{2.343479in}}%
\pgfpathcurveto{\pgfqpoint{3.075406in}{2.349303in}}{\pgfqpoint{3.067506in}{2.352575in}}{\pgfqpoint{3.059269in}{2.352575in}}%
\pgfpathcurveto{\pgfqpoint{3.051033in}{2.352575in}}{\pgfqpoint{3.043133in}{2.349303in}}{\pgfqpoint{3.037309in}{2.343479in}}%
\pgfpathcurveto{\pgfqpoint{3.031485in}{2.337655in}}{\pgfqpoint{3.028213in}{2.329755in}}{\pgfqpoint{3.028213in}{2.321519in}}%
\pgfpathcurveto{\pgfqpoint{3.028213in}{2.313283in}}{\pgfqpoint{3.031485in}{2.305383in}}{\pgfqpoint{3.037309in}{2.299559in}}%
\pgfpathcurveto{\pgfqpoint{3.043133in}{2.293735in}}{\pgfqpoint{3.051033in}{2.290462in}}{\pgfqpoint{3.059269in}{2.290462in}}%
\pgfpathclose%
\pgfusepath{stroke,fill}%
\end{pgfscope}%
\begin{pgfscope}%
\pgfpathrectangle{\pgfqpoint{0.100000in}{0.220728in}}{\pgfqpoint{3.696000in}{3.696000in}}%
\pgfusepath{clip}%
\pgfsetbuttcap%
\pgfsetroundjoin%
\definecolor{currentfill}{rgb}{0.121569,0.466667,0.705882}%
\pgfsetfillcolor{currentfill}%
\pgfsetfillopacity{0.785026}%
\pgfsetlinewidth{1.003750pt}%
\definecolor{currentstroke}{rgb}{0.121569,0.466667,0.705882}%
\pgfsetstrokecolor{currentstroke}%
\pgfsetstrokeopacity{0.785026}%
\pgfsetdash{}{0pt}%
\pgfpathmoveto{\pgfqpoint{3.057948in}{2.287216in}}%
\pgfpathcurveto{\pgfqpoint{3.066184in}{2.287216in}}{\pgfqpoint{3.074084in}{2.290488in}}{\pgfqpoint{3.079908in}{2.296312in}}%
\pgfpathcurveto{\pgfqpoint{3.085732in}{2.302136in}}{\pgfqpoint{3.089004in}{2.310036in}}{\pgfqpoint{3.089004in}{2.318272in}}%
\pgfpathcurveto{\pgfqpoint{3.089004in}{2.326508in}}{\pgfqpoint{3.085732in}{2.334408in}}{\pgfqpoint{3.079908in}{2.340232in}}%
\pgfpathcurveto{\pgfqpoint{3.074084in}{2.346056in}}{\pgfqpoint{3.066184in}{2.349329in}}{\pgfqpoint{3.057948in}{2.349329in}}%
\pgfpathcurveto{\pgfqpoint{3.049711in}{2.349329in}}{\pgfqpoint{3.041811in}{2.346056in}}{\pgfqpoint{3.035987in}{2.340232in}}%
\pgfpathcurveto{\pgfqpoint{3.030164in}{2.334408in}}{\pgfqpoint{3.026891in}{2.326508in}}{\pgfqpoint{3.026891in}{2.318272in}}%
\pgfpathcurveto{\pgfqpoint{3.026891in}{2.310036in}}{\pgfqpoint{3.030164in}{2.302136in}}{\pgfqpoint{3.035987in}{2.296312in}}%
\pgfpathcurveto{\pgfqpoint{3.041811in}{2.290488in}}{\pgfqpoint{3.049711in}{2.287216in}}{\pgfqpoint{3.057948in}{2.287216in}}%
\pgfpathclose%
\pgfusepath{stroke,fill}%
\end{pgfscope}%
\begin{pgfscope}%
\pgfpathrectangle{\pgfqpoint{0.100000in}{0.220728in}}{\pgfqpoint{3.696000in}{3.696000in}}%
\pgfusepath{clip}%
\pgfsetbuttcap%
\pgfsetroundjoin%
\definecolor{currentfill}{rgb}{0.121569,0.466667,0.705882}%
\pgfsetfillcolor{currentfill}%
\pgfsetfillopacity{0.785334}%
\pgfsetlinewidth{1.003750pt}%
\definecolor{currentstroke}{rgb}{0.121569,0.466667,0.705882}%
\pgfsetstrokecolor{currentstroke}%
\pgfsetstrokeopacity{0.785334}%
\pgfsetdash{}{0pt}%
\pgfpathmoveto{\pgfqpoint{3.057159in}{2.285585in}}%
\pgfpathcurveto{\pgfqpoint{3.065396in}{2.285585in}}{\pgfqpoint{3.073296in}{2.288858in}}{\pgfqpoint{3.079120in}{2.294682in}}%
\pgfpathcurveto{\pgfqpoint{3.084944in}{2.300506in}}{\pgfqpoint{3.088216in}{2.308406in}}{\pgfqpoint{3.088216in}{2.316642in}}%
\pgfpathcurveto{\pgfqpoint{3.088216in}{2.324878in}}{\pgfqpoint{3.084944in}{2.332778in}}{\pgfqpoint{3.079120in}{2.338602in}}%
\pgfpathcurveto{\pgfqpoint{3.073296in}{2.344426in}}{\pgfqpoint{3.065396in}{2.347698in}}{\pgfqpoint{3.057159in}{2.347698in}}%
\pgfpathcurveto{\pgfqpoint{3.048923in}{2.347698in}}{\pgfqpoint{3.041023in}{2.344426in}}{\pgfqpoint{3.035199in}{2.338602in}}%
\pgfpathcurveto{\pgfqpoint{3.029375in}{2.332778in}}{\pgfqpoint{3.026103in}{2.324878in}}{\pgfqpoint{3.026103in}{2.316642in}}%
\pgfpathcurveto{\pgfqpoint{3.026103in}{2.308406in}}{\pgfqpoint{3.029375in}{2.300506in}}{\pgfqpoint{3.035199in}{2.294682in}}%
\pgfpathcurveto{\pgfqpoint{3.041023in}{2.288858in}}{\pgfqpoint{3.048923in}{2.285585in}}{\pgfqpoint{3.057159in}{2.285585in}}%
\pgfpathclose%
\pgfusepath{stroke,fill}%
\end{pgfscope}%
\begin{pgfscope}%
\pgfpathrectangle{\pgfqpoint{0.100000in}{0.220728in}}{\pgfqpoint{3.696000in}{3.696000in}}%
\pgfusepath{clip}%
\pgfsetbuttcap%
\pgfsetroundjoin%
\definecolor{currentfill}{rgb}{0.121569,0.466667,0.705882}%
\pgfsetfillcolor{currentfill}%
\pgfsetfillopacity{0.785505}%
\pgfsetlinewidth{1.003750pt}%
\definecolor{currentstroke}{rgb}{0.121569,0.466667,0.705882}%
\pgfsetstrokecolor{currentstroke}%
\pgfsetstrokeopacity{0.785505}%
\pgfsetdash{}{0pt}%
\pgfpathmoveto{\pgfqpoint{3.056649in}{2.284791in}}%
\pgfpathcurveto{\pgfqpoint{3.064885in}{2.284791in}}{\pgfqpoint{3.072785in}{2.288064in}}{\pgfqpoint{3.078609in}{2.293888in}}%
\pgfpathcurveto{\pgfqpoint{3.084433in}{2.299712in}}{\pgfqpoint{3.087705in}{2.307612in}}{\pgfqpoint{3.087705in}{2.315848in}}%
\pgfpathcurveto{\pgfqpoint{3.087705in}{2.324084in}}{\pgfqpoint{3.084433in}{2.331984in}}{\pgfqpoint{3.078609in}{2.337808in}}%
\pgfpathcurveto{\pgfqpoint{3.072785in}{2.343632in}}{\pgfqpoint{3.064885in}{2.346904in}}{\pgfqpoint{3.056649in}{2.346904in}}%
\pgfpathcurveto{\pgfqpoint{3.048413in}{2.346904in}}{\pgfqpoint{3.040513in}{2.343632in}}{\pgfqpoint{3.034689in}{2.337808in}}%
\pgfpathcurveto{\pgfqpoint{3.028865in}{2.331984in}}{\pgfqpoint{3.025592in}{2.324084in}}{\pgfqpoint{3.025592in}{2.315848in}}%
\pgfpathcurveto{\pgfqpoint{3.025592in}{2.307612in}}{\pgfqpoint{3.028865in}{2.299712in}}{\pgfqpoint{3.034689in}{2.293888in}}%
\pgfpathcurveto{\pgfqpoint{3.040513in}{2.288064in}}{\pgfqpoint{3.048413in}{2.284791in}}{\pgfqpoint{3.056649in}{2.284791in}}%
\pgfpathclose%
\pgfusepath{stroke,fill}%
\end{pgfscope}%
\begin{pgfscope}%
\pgfpathrectangle{\pgfqpoint{0.100000in}{0.220728in}}{\pgfqpoint{3.696000in}{3.696000in}}%
\pgfusepath{clip}%
\pgfsetbuttcap%
\pgfsetroundjoin%
\definecolor{currentfill}{rgb}{0.121569,0.466667,0.705882}%
\pgfsetfillcolor{currentfill}%
\pgfsetfillopacity{0.786104}%
\pgfsetlinewidth{1.003750pt}%
\definecolor{currentstroke}{rgb}{0.121569,0.466667,0.705882}%
\pgfsetstrokecolor{currentstroke}%
\pgfsetstrokeopacity{0.786104}%
\pgfsetdash{}{0pt}%
\pgfpathmoveto{\pgfqpoint{3.055534in}{2.281524in}}%
\pgfpathcurveto{\pgfqpoint{3.063770in}{2.281524in}}{\pgfqpoint{3.071670in}{2.284796in}}{\pgfqpoint{3.077494in}{2.290620in}}%
\pgfpathcurveto{\pgfqpoint{3.083318in}{2.296444in}}{\pgfqpoint{3.086590in}{2.304344in}}{\pgfqpoint{3.086590in}{2.312580in}}%
\pgfpathcurveto{\pgfqpoint{3.086590in}{2.320816in}}{\pgfqpoint{3.083318in}{2.328717in}}{\pgfqpoint{3.077494in}{2.334540in}}%
\pgfpathcurveto{\pgfqpoint{3.071670in}{2.340364in}}{\pgfqpoint{3.063770in}{2.343637in}}{\pgfqpoint{3.055534in}{2.343637in}}%
\pgfpathcurveto{\pgfqpoint{3.047297in}{2.343637in}}{\pgfqpoint{3.039397in}{2.340364in}}{\pgfqpoint{3.033573in}{2.334540in}}%
\pgfpathcurveto{\pgfqpoint{3.027750in}{2.328717in}}{\pgfqpoint{3.024477in}{2.320816in}}{\pgfqpoint{3.024477in}{2.312580in}}%
\pgfpathcurveto{\pgfqpoint{3.024477in}{2.304344in}}{\pgfqpoint{3.027750in}{2.296444in}}{\pgfqpoint{3.033573in}{2.290620in}}%
\pgfpathcurveto{\pgfqpoint{3.039397in}{2.284796in}}{\pgfqpoint{3.047297in}{2.281524in}}{\pgfqpoint{3.055534in}{2.281524in}}%
\pgfpathclose%
\pgfusepath{stroke,fill}%
\end{pgfscope}%
\begin{pgfscope}%
\pgfpathrectangle{\pgfqpoint{0.100000in}{0.220728in}}{\pgfqpoint{3.696000in}{3.696000in}}%
\pgfusepath{clip}%
\pgfsetbuttcap%
\pgfsetroundjoin%
\definecolor{currentfill}{rgb}{0.121569,0.466667,0.705882}%
\pgfsetfillcolor{currentfill}%
\pgfsetfillopacity{0.786839}%
\pgfsetlinewidth{1.003750pt}%
\definecolor{currentstroke}{rgb}{0.121569,0.466667,0.705882}%
\pgfsetstrokecolor{currentstroke}%
\pgfsetstrokeopacity{0.786839}%
\pgfsetdash{}{0pt}%
\pgfpathmoveto{\pgfqpoint{3.053425in}{2.277686in}}%
\pgfpathcurveto{\pgfqpoint{3.061661in}{2.277686in}}{\pgfqpoint{3.069561in}{2.280958in}}{\pgfqpoint{3.075385in}{2.286782in}}%
\pgfpathcurveto{\pgfqpoint{3.081209in}{2.292606in}}{\pgfqpoint{3.084481in}{2.300506in}}{\pgfqpoint{3.084481in}{2.308742in}}%
\pgfpathcurveto{\pgfqpoint{3.084481in}{2.316978in}}{\pgfqpoint{3.081209in}{2.324878in}}{\pgfqpoint{3.075385in}{2.330702in}}%
\pgfpathcurveto{\pgfqpoint{3.069561in}{2.336526in}}{\pgfqpoint{3.061661in}{2.339799in}}{\pgfqpoint{3.053425in}{2.339799in}}%
\pgfpathcurveto{\pgfqpoint{3.045188in}{2.339799in}}{\pgfqpoint{3.037288in}{2.336526in}}{\pgfqpoint{3.031464in}{2.330702in}}%
\pgfpathcurveto{\pgfqpoint{3.025640in}{2.324878in}}{\pgfqpoint{3.022368in}{2.316978in}}{\pgfqpoint{3.022368in}{2.308742in}}%
\pgfpathcurveto{\pgfqpoint{3.022368in}{2.300506in}}{\pgfqpoint{3.025640in}{2.292606in}}{\pgfqpoint{3.031464in}{2.286782in}}%
\pgfpathcurveto{\pgfqpoint{3.037288in}{2.280958in}}{\pgfqpoint{3.045188in}{2.277686in}}{\pgfqpoint{3.053425in}{2.277686in}}%
\pgfpathclose%
\pgfusepath{stroke,fill}%
\end{pgfscope}%
\begin{pgfscope}%
\pgfpathrectangle{\pgfqpoint{0.100000in}{0.220728in}}{\pgfqpoint{3.696000in}{3.696000in}}%
\pgfusepath{clip}%
\pgfsetbuttcap%
\pgfsetroundjoin%
\definecolor{currentfill}{rgb}{0.121569,0.466667,0.705882}%
\pgfsetfillcolor{currentfill}%
\pgfsetfillopacity{0.787839}%
\pgfsetlinewidth{1.003750pt}%
\definecolor{currentstroke}{rgb}{0.121569,0.466667,0.705882}%
\pgfsetstrokecolor{currentstroke}%
\pgfsetstrokeopacity{0.787839}%
\pgfsetdash{}{0pt}%
\pgfpathmoveto{\pgfqpoint{3.050152in}{2.272166in}}%
\pgfpathcurveto{\pgfqpoint{3.058388in}{2.272166in}}{\pgfqpoint{3.066289in}{2.275438in}}{\pgfqpoint{3.072112in}{2.281262in}}%
\pgfpathcurveto{\pgfqpoint{3.077936in}{2.287086in}}{\pgfqpoint{3.081209in}{2.294986in}}{\pgfqpoint{3.081209in}{2.303223in}}%
\pgfpathcurveto{\pgfqpoint{3.081209in}{2.311459in}}{\pgfqpoint{3.077936in}{2.319359in}}{\pgfqpoint{3.072112in}{2.325183in}}%
\pgfpathcurveto{\pgfqpoint{3.066289in}{2.331007in}}{\pgfqpoint{3.058388in}{2.334279in}}{\pgfqpoint{3.050152in}{2.334279in}}%
\pgfpathcurveto{\pgfqpoint{3.041916in}{2.334279in}}{\pgfqpoint{3.034016in}{2.331007in}}{\pgfqpoint{3.028192in}{2.325183in}}%
\pgfpathcurveto{\pgfqpoint{3.022368in}{2.319359in}}{\pgfqpoint{3.019096in}{2.311459in}}{\pgfqpoint{3.019096in}{2.303223in}}%
\pgfpathcurveto{\pgfqpoint{3.019096in}{2.294986in}}{\pgfqpoint{3.022368in}{2.287086in}}{\pgfqpoint{3.028192in}{2.281262in}}%
\pgfpathcurveto{\pgfqpoint{3.034016in}{2.275438in}}{\pgfqpoint{3.041916in}{2.272166in}}{\pgfqpoint{3.050152in}{2.272166in}}%
\pgfpathclose%
\pgfusepath{stroke,fill}%
\end{pgfscope}%
\begin{pgfscope}%
\pgfpathrectangle{\pgfqpoint{0.100000in}{0.220728in}}{\pgfqpoint{3.696000in}{3.696000in}}%
\pgfusepath{clip}%
\pgfsetbuttcap%
\pgfsetroundjoin%
\definecolor{currentfill}{rgb}{0.121569,0.466667,0.705882}%
\pgfsetfillcolor{currentfill}%
\pgfsetfillopacity{0.788197}%
\pgfsetlinewidth{1.003750pt}%
\definecolor{currentstroke}{rgb}{0.121569,0.466667,0.705882}%
\pgfsetstrokecolor{currentstroke}%
\pgfsetstrokeopacity{0.788197}%
\pgfsetdash{}{0pt}%
\pgfpathmoveto{\pgfqpoint{1.280863in}{1.145287in}}%
\pgfpathcurveto{\pgfqpoint{1.289099in}{1.145287in}}{\pgfqpoint{1.296999in}{1.148559in}}{\pgfqpoint{1.302823in}{1.154383in}}%
\pgfpathcurveto{\pgfqpoint{1.308647in}{1.160207in}}{\pgfqpoint{1.311920in}{1.168107in}}{\pgfqpoint{1.311920in}{1.176343in}}%
\pgfpathcurveto{\pgfqpoint{1.311920in}{1.184580in}}{\pgfqpoint{1.308647in}{1.192480in}}{\pgfqpoint{1.302823in}{1.198304in}}%
\pgfpathcurveto{\pgfqpoint{1.296999in}{1.204128in}}{\pgfqpoint{1.289099in}{1.207400in}}{\pgfqpoint{1.280863in}{1.207400in}}%
\pgfpathcurveto{\pgfqpoint{1.272627in}{1.207400in}}{\pgfqpoint{1.264727in}{1.204128in}}{\pgfqpoint{1.258903in}{1.198304in}}%
\pgfpathcurveto{\pgfqpoint{1.253079in}{1.192480in}}{\pgfqpoint{1.249807in}{1.184580in}}{\pgfqpoint{1.249807in}{1.176343in}}%
\pgfpathcurveto{\pgfqpoint{1.249807in}{1.168107in}}{\pgfqpoint{1.253079in}{1.160207in}}{\pgfqpoint{1.258903in}{1.154383in}}%
\pgfpathcurveto{\pgfqpoint{1.264727in}{1.148559in}}{\pgfqpoint{1.272627in}{1.145287in}}{\pgfqpoint{1.280863in}{1.145287in}}%
\pgfpathclose%
\pgfusepath{stroke,fill}%
\end{pgfscope}%
\begin{pgfscope}%
\pgfpathrectangle{\pgfqpoint{0.100000in}{0.220728in}}{\pgfqpoint{3.696000in}{3.696000in}}%
\pgfusepath{clip}%
\pgfsetbuttcap%
\pgfsetroundjoin%
\definecolor{currentfill}{rgb}{0.121569,0.466667,0.705882}%
\pgfsetfillcolor{currentfill}%
\pgfsetfillopacity{0.789331}%
\pgfsetlinewidth{1.003750pt}%
\definecolor{currentstroke}{rgb}{0.121569,0.466667,0.705882}%
\pgfsetstrokecolor{currentstroke}%
\pgfsetstrokeopacity{0.789331}%
\pgfsetdash{}{0pt}%
\pgfpathmoveto{\pgfqpoint{3.046561in}{2.264734in}}%
\pgfpathcurveto{\pgfqpoint{3.054797in}{2.264734in}}{\pgfqpoint{3.062697in}{2.268006in}}{\pgfqpoint{3.068521in}{2.273830in}}%
\pgfpathcurveto{\pgfqpoint{3.074345in}{2.279654in}}{\pgfqpoint{3.077618in}{2.287554in}}{\pgfqpoint{3.077618in}{2.295790in}}%
\pgfpathcurveto{\pgfqpoint{3.077618in}{2.304026in}}{\pgfqpoint{3.074345in}{2.311927in}}{\pgfqpoint{3.068521in}{2.317750in}}%
\pgfpathcurveto{\pgfqpoint{3.062697in}{2.323574in}}{\pgfqpoint{3.054797in}{2.326847in}}{\pgfqpoint{3.046561in}{2.326847in}}%
\pgfpathcurveto{\pgfqpoint{3.038325in}{2.326847in}}{\pgfqpoint{3.030425in}{2.323574in}}{\pgfqpoint{3.024601in}{2.317750in}}%
\pgfpathcurveto{\pgfqpoint{3.018777in}{2.311927in}}{\pgfqpoint{3.015505in}{2.304026in}}{\pgfqpoint{3.015505in}{2.295790in}}%
\pgfpathcurveto{\pgfqpoint{3.015505in}{2.287554in}}{\pgfqpoint{3.018777in}{2.279654in}}{\pgfqpoint{3.024601in}{2.273830in}}%
\pgfpathcurveto{\pgfqpoint{3.030425in}{2.268006in}}{\pgfqpoint{3.038325in}{2.264734in}}{\pgfqpoint{3.046561in}{2.264734in}}%
\pgfpathclose%
\pgfusepath{stroke,fill}%
\end{pgfscope}%
\begin{pgfscope}%
\pgfpathrectangle{\pgfqpoint{0.100000in}{0.220728in}}{\pgfqpoint{3.696000in}{3.696000in}}%
\pgfusepath{clip}%
\pgfsetbuttcap%
\pgfsetroundjoin%
\definecolor{currentfill}{rgb}{0.121569,0.466667,0.705882}%
\pgfsetfillcolor{currentfill}%
\pgfsetfillopacity{0.790195}%
\pgfsetlinewidth{1.003750pt}%
\definecolor{currentstroke}{rgb}{0.121569,0.466667,0.705882}%
\pgfsetstrokecolor{currentstroke}%
\pgfsetstrokeopacity{0.790195}%
\pgfsetdash{}{0pt}%
\pgfpathmoveto{\pgfqpoint{3.045110in}{2.260347in}}%
\pgfpathcurveto{\pgfqpoint{3.053346in}{2.260347in}}{\pgfqpoint{3.061246in}{2.263620in}}{\pgfqpoint{3.067070in}{2.269444in}}%
\pgfpathcurveto{\pgfqpoint{3.072894in}{2.275268in}}{\pgfqpoint{3.076166in}{2.283168in}}{\pgfqpoint{3.076166in}{2.291404in}}%
\pgfpathcurveto{\pgfqpoint{3.076166in}{2.299640in}}{\pgfqpoint{3.072894in}{2.307540in}}{\pgfqpoint{3.067070in}{2.313364in}}%
\pgfpathcurveto{\pgfqpoint{3.061246in}{2.319188in}}{\pgfqpoint{3.053346in}{2.322460in}}{\pgfqpoint{3.045110in}{2.322460in}}%
\pgfpathcurveto{\pgfqpoint{3.036873in}{2.322460in}}{\pgfqpoint{3.028973in}{2.319188in}}{\pgfqpoint{3.023149in}{2.313364in}}%
\pgfpathcurveto{\pgfqpoint{3.017326in}{2.307540in}}{\pgfqpoint{3.014053in}{2.299640in}}{\pgfqpoint{3.014053in}{2.291404in}}%
\pgfpathcurveto{\pgfqpoint{3.014053in}{2.283168in}}{\pgfqpoint{3.017326in}{2.275268in}}{\pgfqpoint{3.023149in}{2.269444in}}%
\pgfpathcurveto{\pgfqpoint{3.028973in}{2.263620in}}{\pgfqpoint{3.036873in}{2.260347in}}{\pgfqpoint{3.045110in}{2.260347in}}%
\pgfpathclose%
\pgfusepath{stroke,fill}%
\end{pgfscope}%
\begin{pgfscope}%
\pgfpathrectangle{\pgfqpoint{0.100000in}{0.220728in}}{\pgfqpoint{3.696000in}{3.696000in}}%
\pgfusepath{clip}%
\pgfsetbuttcap%
\pgfsetroundjoin%
\definecolor{currentfill}{rgb}{0.121569,0.466667,0.705882}%
\pgfsetfillcolor{currentfill}%
\pgfsetfillopacity{0.791077}%
\pgfsetlinewidth{1.003750pt}%
\definecolor{currentstroke}{rgb}{0.121569,0.466667,0.705882}%
\pgfsetstrokecolor{currentstroke}%
\pgfsetstrokeopacity{0.791077}%
\pgfsetdash{}{0pt}%
\pgfpathmoveto{\pgfqpoint{3.042266in}{2.255789in}}%
\pgfpathcurveto{\pgfqpoint{3.050502in}{2.255789in}}{\pgfqpoint{3.058402in}{2.259061in}}{\pgfqpoint{3.064226in}{2.264885in}}%
\pgfpathcurveto{\pgfqpoint{3.070050in}{2.270709in}}{\pgfqpoint{3.073322in}{2.278609in}}{\pgfqpoint{3.073322in}{2.286846in}}%
\pgfpathcurveto{\pgfqpoint{3.073322in}{2.295082in}}{\pgfqpoint{3.070050in}{2.302982in}}{\pgfqpoint{3.064226in}{2.308806in}}%
\pgfpathcurveto{\pgfqpoint{3.058402in}{2.314630in}}{\pgfqpoint{3.050502in}{2.317902in}}{\pgfqpoint{3.042266in}{2.317902in}}%
\pgfpathcurveto{\pgfqpoint{3.034029in}{2.317902in}}{\pgfqpoint{3.026129in}{2.314630in}}{\pgfqpoint{3.020305in}{2.308806in}}%
\pgfpathcurveto{\pgfqpoint{3.014481in}{2.302982in}}{\pgfqpoint{3.011209in}{2.295082in}}{\pgfqpoint{3.011209in}{2.286846in}}%
\pgfpathcurveto{\pgfqpoint{3.011209in}{2.278609in}}{\pgfqpoint{3.014481in}{2.270709in}}{\pgfqpoint{3.020305in}{2.264885in}}%
\pgfpathcurveto{\pgfqpoint{3.026129in}{2.259061in}}{\pgfqpoint{3.034029in}{2.255789in}}{\pgfqpoint{3.042266in}{2.255789in}}%
\pgfpathclose%
\pgfusepath{stroke,fill}%
\end{pgfscope}%
\begin{pgfscope}%
\pgfpathrectangle{\pgfqpoint{0.100000in}{0.220728in}}{\pgfqpoint{3.696000in}{3.696000in}}%
\pgfusepath{clip}%
\pgfsetbuttcap%
\pgfsetroundjoin%
\definecolor{currentfill}{rgb}{0.121569,0.466667,0.705882}%
\pgfsetfillcolor{currentfill}%
\pgfsetfillopacity{0.792343}%
\pgfsetlinewidth{1.003750pt}%
\definecolor{currentstroke}{rgb}{0.121569,0.466667,0.705882}%
\pgfsetstrokecolor{currentstroke}%
\pgfsetstrokeopacity{0.792343}%
\pgfsetdash{}{0pt}%
\pgfpathmoveto{\pgfqpoint{3.039836in}{2.249103in}}%
\pgfpathcurveto{\pgfqpoint{3.048073in}{2.249103in}}{\pgfqpoint{3.055973in}{2.252375in}}{\pgfqpoint{3.061796in}{2.258199in}}%
\pgfpathcurveto{\pgfqpoint{3.067620in}{2.264023in}}{\pgfqpoint{3.070893in}{2.271923in}}{\pgfqpoint{3.070893in}{2.280159in}}%
\pgfpathcurveto{\pgfqpoint{3.070893in}{2.288395in}}{\pgfqpoint{3.067620in}{2.296295in}}{\pgfqpoint{3.061796in}{2.302119in}}%
\pgfpathcurveto{\pgfqpoint{3.055973in}{2.307943in}}{\pgfqpoint{3.048073in}{2.311216in}}{\pgfqpoint{3.039836in}{2.311216in}}%
\pgfpathcurveto{\pgfqpoint{3.031600in}{2.311216in}}{\pgfqpoint{3.023700in}{2.307943in}}{\pgfqpoint{3.017876in}{2.302119in}}%
\pgfpathcurveto{\pgfqpoint{3.012052in}{2.296295in}}{\pgfqpoint{3.008780in}{2.288395in}}{\pgfqpoint{3.008780in}{2.280159in}}%
\pgfpathcurveto{\pgfqpoint{3.008780in}{2.271923in}}{\pgfqpoint{3.012052in}{2.264023in}}{\pgfqpoint{3.017876in}{2.258199in}}%
\pgfpathcurveto{\pgfqpoint{3.023700in}{2.252375in}}{\pgfqpoint{3.031600in}{2.249103in}}{\pgfqpoint{3.039836in}{2.249103in}}%
\pgfpathclose%
\pgfusepath{stroke,fill}%
\end{pgfscope}%
\begin{pgfscope}%
\pgfpathrectangle{\pgfqpoint{0.100000in}{0.220728in}}{\pgfqpoint{3.696000in}{3.696000in}}%
\pgfusepath{clip}%
\pgfsetbuttcap%
\pgfsetroundjoin%
\definecolor{currentfill}{rgb}{0.121569,0.466667,0.705882}%
\pgfsetfillcolor{currentfill}%
\pgfsetfillopacity{0.792395}%
\pgfsetlinewidth{1.003750pt}%
\definecolor{currentstroke}{rgb}{0.121569,0.466667,0.705882}%
\pgfsetstrokecolor{currentstroke}%
\pgfsetstrokeopacity{0.792395}%
\pgfsetdash{}{0pt}%
\pgfpathmoveto{\pgfqpoint{1.301483in}{1.137122in}}%
\pgfpathcurveto{\pgfqpoint{1.309720in}{1.137122in}}{\pgfqpoint{1.317620in}{1.140394in}}{\pgfqpoint{1.323444in}{1.146218in}}%
\pgfpathcurveto{\pgfqpoint{1.329268in}{1.152042in}}{\pgfqpoint{1.332540in}{1.159942in}}{\pgfqpoint{1.332540in}{1.168179in}}%
\pgfpathcurveto{\pgfqpoint{1.332540in}{1.176415in}}{\pgfqpoint{1.329268in}{1.184315in}}{\pgfqpoint{1.323444in}{1.190139in}}%
\pgfpathcurveto{\pgfqpoint{1.317620in}{1.195963in}}{\pgfqpoint{1.309720in}{1.199235in}}{\pgfqpoint{1.301483in}{1.199235in}}%
\pgfpathcurveto{\pgfqpoint{1.293247in}{1.199235in}}{\pgfqpoint{1.285347in}{1.195963in}}{\pgfqpoint{1.279523in}{1.190139in}}%
\pgfpathcurveto{\pgfqpoint{1.273699in}{1.184315in}}{\pgfqpoint{1.270427in}{1.176415in}}{\pgfqpoint{1.270427in}{1.168179in}}%
\pgfpathcurveto{\pgfqpoint{1.270427in}{1.159942in}}{\pgfqpoint{1.273699in}{1.152042in}}{\pgfqpoint{1.279523in}{1.146218in}}%
\pgfpathcurveto{\pgfqpoint{1.285347in}{1.140394in}}{\pgfqpoint{1.293247in}{1.137122in}}{\pgfqpoint{1.301483in}{1.137122in}}%
\pgfpathclose%
\pgfusepath{stroke,fill}%
\end{pgfscope}%
\begin{pgfscope}%
\pgfpathrectangle{\pgfqpoint{0.100000in}{0.220728in}}{\pgfqpoint{3.696000in}{3.696000in}}%
\pgfusepath{clip}%
\pgfsetbuttcap%
\pgfsetroundjoin%
\definecolor{currentfill}{rgb}{0.121569,0.466667,0.705882}%
\pgfsetfillcolor{currentfill}%
\pgfsetfillopacity{0.793808}%
\pgfsetlinewidth{1.003750pt}%
\definecolor{currentstroke}{rgb}{0.121569,0.466667,0.705882}%
\pgfsetstrokecolor{currentstroke}%
\pgfsetstrokeopacity{0.793808}%
\pgfsetdash{}{0pt}%
\pgfpathmoveto{\pgfqpoint{3.037076in}{2.242085in}}%
\pgfpathcurveto{\pgfqpoint{3.045312in}{2.242085in}}{\pgfqpoint{3.053212in}{2.245358in}}{\pgfqpoint{3.059036in}{2.251182in}}%
\pgfpathcurveto{\pgfqpoint{3.064860in}{2.257005in}}{\pgfqpoint{3.068132in}{2.264906in}}{\pgfqpoint{3.068132in}{2.273142in}}%
\pgfpathcurveto{\pgfqpoint{3.068132in}{2.281378in}}{\pgfqpoint{3.064860in}{2.289278in}}{\pgfqpoint{3.059036in}{2.295102in}}%
\pgfpathcurveto{\pgfqpoint{3.053212in}{2.300926in}}{\pgfqpoint{3.045312in}{2.304198in}}{\pgfqpoint{3.037076in}{2.304198in}}%
\pgfpathcurveto{\pgfqpoint{3.028840in}{2.304198in}}{\pgfqpoint{3.020940in}{2.300926in}}{\pgfqpoint{3.015116in}{2.295102in}}%
\pgfpathcurveto{\pgfqpoint{3.009292in}{2.289278in}}{\pgfqpoint{3.006019in}{2.281378in}}{\pgfqpoint{3.006019in}{2.273142in}}%
\pgfpathcurveto{\pgfqpoint{3.006019in}{2.264906in}}{\pgfqpoint{3.009292in}{2.257005in}}{\pgfqpoint{3.015116in}{2.251182in}}%
\pgfpathcurveto{\pgfqpoint{3.020940in}{2.245358in}}{\pgfqpoint{3.028840in}{2.242085in}}{\pgfqpoint{3.037076in}{2.242085in}}%
\pgfpathclose%
\pgfusepath{stroke,fill}%
\end{pgfscope}%
\begin{pgfscope}%
\pgfpathrectangle{\pgfqpoint{0.100000in}{0.220728in}}{\pgfqpoint{3.696000in}{3.696000in}}%
\pgfusepath{clip}%
\pgfsetbuttcap%
\pgfsetroundjoin%
\definecolor{currentfill}{rgb}{0.121569,0.466667,0.705882}%
\pgfsetfillcolor{currentfill}%
\pgfsetfillopacity{0.794486}%
\pgfsetlinewidth{1.003750pt}%
\definecolor{currentstroke}{rgb}{0.121569,0.466667,0.705882}%
\pgfsetstrokecolor{currentstroke}%
\pgfsetstrokeopacity{0.794486}%
\pgfsetdash{}{0pt}%
\pgfpathmoveto{\pgfqpoint{3.034846in}{2.238525in}}%
\pgfpathcurveto{\pgfqpoint{3.043082in}{2.238525in}}{\pgfqpoint{3.050982in}{2.241797in}}{\pgfqpoint{3.056806in}{2.247621in}}%
\pgfpathcurveto{\pgfqpoint{3.062630in}{2.253445in}}{\pgfqpoint{3.065903in}{2.261345in}}{\pgfqpoint{3.065903in}{2.269581in}}%
\pgfpathcurveto{\pgfqpoint{3.065903in}{2.277818in}}{\pgfqpoint{3.062630in}{2.285718in}}{\pgfqpoint{3.056806in}{2.291542in}}%
\pgfpathcurveto{\pgfqpoint{3.050982in}{2.297365in}}{\pgfqpoint{3.043082in}{2.300638in}}{\pgfqpoint{3.034846in}{2.300638in}}%
\pgfpathcurveto{\pgfqpoint{3.026610in}{2.300638in}}{\pgfqpoint{3.018710in}{2.297365in}}{\pgfqpoint{3.012886in}{2.291542in}}%
\pgfpathcurveto{\pgfqpoint{3.007062in}{2.285718in}}{\pgfqpoint{3.003790in}{2.277818in}}{\pgfqpoint{3.003790in}{2.269581in}}%
\pgfpathcurveto{\pgfqpoint{3.003790in}{2.261345in}}{\pgfqpoint{3.007062in}{2.253445in}}{\pgfqpoint{3.012886in}{2.247621in}}%
\pgfpathcurveto{\pgfqpoint{3.018710in}{2.241797in}}{\pgfqpoint{3.026610in}{2.238525in}}{\pgfqpoint{3.034846in}{2.238525in}}%
\pgfpathclose%
\pgfusepath{stroke,fill}%
\end{pgfscope}%
\begin{pgfscope}%
\pgfpathrectangle{\pgfqpoint{0.100000in}{0.220728in}}{\pgfqpoint{3.696000in}{3.696000in}}%
\pgfusepath{clip}%
\pgfsetbuttcap%
\pgfsetroundjoin%
\definecolor{currentfill}{rgb}{0.121569,0.466667,0.705882}%
\pgfsetfillcolor{currentfill}%
\pgfsetfillopacity{0.795618}%
\pgfsetlinewidth{1.003750pt}%
\definecolor{currentstroke}{rgb}{0.121569,0.466667,0.705882}%
\pgfsetstrokecolor{currentstroke}%
\pgfsetstrokeopacity{0.795618}%
\pgfsetdash{}{0pt}%
\pgfpathmoveto{\pgfqpoint{3.033006in}{2.233075in}}%
\pgfpathcurveto{\pgfqpoint{3.041243in}{2.233075in}}{\pgfqpoint{3.049143in}{2.236347in}}{\pgfqpoint{3.054966in}{2.242171in}}%
\pgfpathcurveto{\pgfqpoint{3.060790in}{2.247995in}}{\pgfqpoint{3.064063in}{2.255895in}}{\pgfqpoint{3.064063in}{2.264131in}}%
\pgfpathcurveto{\pgfqpoint{3.064063in}{2.272367in}}{\pgfqpoint{3.060790in}{2.280267in}}{\pgfqpoint{3.054966in}{2.286091in}}%
\pgfpathcurveto{\pgfqpoint{3.049143in}{2.291915in}}{\pgfqpoint{3.041243in}{2.295188in}}{\pgfqpoint{3.033006in}{2.295188in}}%
\pgfpathcurveto{\pgfqpoint{3.024770in}{2.295188in}}{\pgfqpoint{3.016870in}{2.291915in}}{\pgfqpoint{3.011046in}{2.286091in}}%
\pgfpathcurveto{\pgfqpoint{3.005222in}{2.280267in}}{\pgfqpoint{3.001950in}{2.272367in}}{\pgfqpoint{3.001950in}{2.264131in}}%
\pgfpathcurveto{\pgfqpoint{3.001950in}{2.255895in}}{\pgfqpoint{3.005222in}{2.247995in}}{\pgfqpoint{3.011046in}{2.242171in}}%
\pgfpathcurveto{\pgfqpoint{3.016870in}{2.236347in}}{\pgfqpoint{3.024770in}{2.233075in}}{\pgfqpoint{3.033006in}{2.233075in}}%
\pgfpathclose%
\pgfusepath{stroke,fill}%
\end{pgfscope}%
\begin{pgfscope}%
\pgfpathrectangle{\pgfqpoint{0.100000in}{0.220728in}}{\pgfqpoint{3.696000in}{3.696000in}}%
\pgfusepath{clip}%
\pgfsetbuttcap%
\pgfsetroundjoin%
\definecolor{currentfill}{rgb}{0.121569,0.466667,0.705882}%
\pgfsetfillcolor{currentfill}%
\pgfsetfillopacity{0.796200}%
\pgfsetlinewidth{1.003750pt}%
\definecolor{currentstroke}{rgb}{0.121569,0.466667,0.705882}%
\pgfsetstrokecolor{currentstroke}%
\pgfsetstrokeopacity{0.796200}%
\pgfsetdash{}{0pt}%
\pgfpathmoveto{\pgfqpoint{3.031623in}{2.230241in}}%
\pgfpathcurveto{\pgfqpoint{3.039859in}{2.230241in}}{\pgfqpoint{3.047760in}{2.233513in}}{\pgfqpoint{3.053583in}{2.239337in}}%
\pgfpathcurveto{\pgfqpoint{3.059407in}{2.245161in}}{\pgfqpoint{3.062680in}{2.253061in}}{\pgfqpoint{3.062680in}{2.261298in}}%
\pgfpathcurveto{\pgfqpoint{3.062680in}{2.269534in}}{\pgfqpoint{3.059407in}{2.277434in}}{\pgfqpoint{3.053583in}{2.283258in}}%
\pgfpathcurveto{\pgfqpoint{3.047760in}{2.289082in}}{\pgfqpoint{3.039859in}{2.292354in}}{\pgfqpoint{3.031623in}{2.292354in}}%
\pgfpathcurveto{\pgfqpoint{3.023387in}{2.292354in}}{\pgfqpoint{3.015487in}{2.289082in}}{\pgfqpoint{3.009663in}{2.283258in}}%
\pgfpathcurveto{\pgfqpoint{3.003839in}{2.277434in}}{\pgfqpoint{3.000567in}{2.269534in}}{\pgfqpoint{3.000567in}{2.261298in}}%
\pgfpathcurveto{\pgfqpoint{3.000567in}{2.253061in}}{\pgfqpoint{3.003839in}{2.245161in}}{\pgfqpoint{3.009663in}{2.239337in}}%
\pgfpathcurveto{\pgfqpoint{3.015487in}{2.233513in}}{\pgfqpoint{3.023387in}{2.230241in}}{\pgfqpoint{3.031623in}{2.230241in}}%
\pgfpathclose%
\pgfusepath{stroke,fill}%
\end{pgfscope}%
\begin{pgfscope}%
\pgfpathrectangle{\pgfqpoint{0.100000in}{0.220728in}}{\pgfqpoint{3.696000in}{3.696000in}}%
\pgfusepath{clip}%
\pgfsetbuttcap%
\pgfsetroundjoin%
\definecolor{currentfill}{rgb}{0.121569,0.466667,0.705882}%
\pgfsetfillcolor{currentfill}%
\pgfsetfillopacity{0.796615}%
\pgfsetlinewidth{1.003750pt}%
\definecolor{currentstroke}{rgb}{0.121569,0.466667,0.705882}%
\pgfsetstrokecolor{currentstroke}%
\pgfsetstrokeopacity{0.796615}%
\pgfsetdash{}{0pt}%
\pgfpathmoveto{\pgfqpoint{1.320250in}{1.132314in}}%
\pgfpathcurveto{\pgfqpoint{1.328486in}{1.132314in}}{\pgfqpoint{1.336386in}{1.135586in}}{\pgfqpoint{1.342210in}{1.141410in}}%
\pgfpathcurveto{\pgfqpoint{1.348034in}{1.147234in}}{\pgfqpoint{1.351307in}{1.155134in}}{\pgfqpoint{1.351307in}{1.163370in}}%
\pgfpathcurveto{\pgfqpoint{1.351307in}{1.171607in}}{\pgfqpoint{1.348034in}{1.179507in}}{\pgfqpoint{1.342210in}{1.185331in}}%
\pgfpathcurveto{\pgfqpoint{1.336386in}{1.191155in}}{\pgfqpoint{1.328486in}{1.194427in}}{\pgfqpoint{1.320250in}{1.194427in}}%
\pgfpathcurveto{\pgfqpoint{1.312014in}{1.194427in}}{\pgfqpoint{1.304114in}{1.191155in}}{\pgfqpoint{1.298290in}{1.185331in}}%
\pgfpathcurveto{\pgfqpoint{1.292466in}{1.179507in}}{\pgfqpoint{1.289194in}{1.171607in}}{\pgfqpoint{1.289194in}{1.163370in}}%
\pgfpathcurveto{\pgfqpoint{1.289194in}{1.155134in}}{\pgfqpoint{1.292466in}{1.147234in}}{\pgfqpoint{1.298290in}{1.141410in}}%
\pgfpathcurveto{\pgfqpoint{1.304114in}{1.135586in}}{\pgfqpoint{1.312014in}{1.132314in}}{\pgfqpoint{1.320250in}{1.132314in}}%
\pgfpathclose%
\pgfusepath{stroke,fill}%
\end{pgfscope}%
\begin{pgfscope}%
\pgfpathrectangle{\pgfqpoint{0.100000in}{0.220728in}}{\pgfqpoint{3.696000in}{3.696000in}}%
\pgfusepath{clip}%
\pgfsetbuttcap%
\pgfsetroundjoin%
\definecolor{currentfill}{rgb}{0.121569,0.466667,0.705882}%
\pgfsetfillcolor{currentfill}%
\pgfsetfillopacity{0.796952}%
\pgfsetlinewidth{1.003750pt}%
\definecolor{currentstroke}{rgb}{0.121569,0.466667,0.705882}%
\pgfsetstrokecolor{currentstroke}%
\pgfsetstrokeopacity{0.796952}%
\pgfsetdash{}{0pt}%
\pgfpathmoveto{\pgfqpoint{3.029589in}{2.227014in}}%
\pgfpathcurveto{\pgfqpoint{3.037825in}{2.227014in}}{\pgfqpoint{3.045725in}{2.230286in}}{\pgfqpoint{3.051549in}{2.236110in}}%
\pgfpathcurveto{\pgfqpoint{3.057373in}{2.241934in}}{\pgfqpoint{3.060645in}{2.249834in}}{\pgfqpoint{3.060645in}{2.258071in}}%
\pgfpathcurveto{\pgfqpoint{3.060645in}{2.266307in}}{\pgfqpoint{3.057373in}{2.274207in}}{\pgfqpoint{3.051549in}{2.280031in}}%
\pgfpathcurveto{\pgfqpoint{3.045725in}{2.285855in}}{\pgfqpoint{3.037825in}{2.289127in}}{\pgfqpoint{3.029589in}{2.289127in}}%
\pgfpathcurveto{\pgfqpoint{3.021352in}{2.289127in}}{\pgfqpoint{3.013452in}{2.285855in}}{\pgfqpoint{3.007628in}{2.280031in}}%
\pgfpathcurveto{\pgfqpoint{3.001804in}{2.274207in}}{\pgfqpoint{2.998532in}{2.266307in}}{\pgfqpoint{2.998532in}{2.258071in}}%
\pgfpathcurveto{\pgfqpoint{2.998532in}{2.249834in}}{\pgfqpoint{3.001804in}{2.241934in}}{\pgfqpoint{3.007628in}{2.236110in}}%
\pgfpathcurveto{\pgfqpoint{3.013452in}{2.230286in}}{\pgfqpoint{3.021352in}{2.227014in}}{\pgfqpoint{3.029589in}{2.227014in}}%
\pgfpathclose%
\pgfusepath{stroke,fill}%
\end{pgfscope}%
\begin{pgfscope}%
\pgfpathrectangle{\pgfqpoint{0.100000in}{0.220728in}}{\pgfqpoint{3.696000in}{3.696000in}}%
\pgfusepath{clip}%
\pgfsetbuttcap%
\pgfsetroundjoin%
\definecolor{currentfill}{rgb}{0.121569,0.466667,0.705882}%
\pgfsetfillcolor{currentfill}%
\pgfsetfillopacity{0.797851}%
\pgfsetlinewidth{1.003750pt}%
\definecolor{currentstroke}{rgb}{0.121569,0.466667,0.705882}%
\pgfsetstrokecolor{currentstroke}%
\pgfsetstrokeopacity{0.797851}%
\pgfsetdash{}{0pt}%
\pgfpathmoveto{\pgfqpoint{3.027736in}{2.221728in}}%
\pgfpathcurveto{\pgfqpoint{3.035972in}{2.221728in}}{\pgfqpoint{3.043872in}{2.225000in}}{\pgfqpoint{3.049696in}{2.230824in}}%
\pgfpathcurveto{\pgfqpoint{3.055520in}{2.236648in}}{\pgfqpoint{3.058793in}{2.244548in}}{\pgfqpoint{3.058793in}{2.252784in}}%
\pgfpathcurveto{\pgfqpoint{3.058793in}{2.261021in}}{\pgfqpoint{3.055520in}{2.268921in}}{\pgfqpoint{3.049696in}{2.274745in}}%
\pgfpathcurveto{\pgfqpoint{3.043872in}{2.280569in}}{\pgfqpoint{3.035972in}{2.283841in}}{\pgfqpoint{3.027736in}{2.283841in}}%
\pgfpathcurveto{\pgfqpoint{3.019500in}{2.283841in}}{\pgfqpoint{3.011600in}{2.280569in}}{\pgfqpoint{3.005776in}{2.274745in}}%
\pgfpathcurveto{\pgfqpoint{2.999952in}{2.268921in}}{\pgfqpoint{2.996680in}{2.261021in}}{\pgfqpoint{2.996680in}{2.252784in}}%
\pgfpathcurveto{\pgfqpoint{2.996680in}{2.244548in}}{\pgfqpoint{2.999952in}{2.236648in}}{\pgfqpoint{3.005776in}{2.230824in}}%
\pgfpathcurveto{\pgfqpoint{3.011600in}{2.225000in}}{\pgfqpoint{3.019500in}{2.221728in}}{\pgfqpoint{3.027736in}{2.221728in}}%
\pgfpathclose%
\pgfusepath{stroke,fill}%
\end{pgfscope}%
\begin{pgfscope}%
\pgfpathrectangle{\pgfqpoint{0.100000in}{0.220728in}}{\pgfqpoint{3.696000in}{3.696000in}}%
\pgfusepath{clip}%
\pgfsetbuttcap%
\pgfsetroundjoin%
\definecolor{currentfill}{rgb}{0.121569,0.466667,0.705882}%
\pgfsetfillcolor{currentfill}%
\pgfsetfillopacity{0.798361}%
\pgfsetlinewidth{1.003750pt}%
\definecolor{currentstroke}{rgb}{0.121569,0.466667,0.705882}%
\pgfsetstrokecolor{currentstroke}%
\pgfsetstrokeopacity{0.798361}%
\pgfsetdash{}{0pt}%
\pgfpathmoveto{\pgfqpoint{3.026417in}{2.219184in}}%
\pgfpathcurveto{\pgfqpoint{3.034653in}{2.219184in}}{\pgfqpoint{3.042553in}{2.222457in}}{\pgfqpoint{3.048377in}{2.228281in}}%
\pgfpathcurveto{\pgfqpoint{3.054201in}{2.234104in}}{\pgfqpoint{3.057473in}{2.242005in}}{\pgfqpoint{3.057473in}{2.250241in}}%
\pgfpathcurveto{\pgfqpoint{3.057473in}{2.258477in}}{\pgfqpoint{3.054201in}{2.266377in}}{\pgfqpoint{3.048377in}{2.272201in}}%
\pgfpathcurveto{\pgfqpoint{3.042553in}{2.278025in}}{\pgfqpoint{3.034653in}{2.281297in}}{\pgfqpoint{3.026417in}{2.281297in}}%
\pgfpathcurveto{\pgfqpoint{3.018180in}{2.281297in}}{\pgfqpoint{3.010280in}{2.278025in}}{\pgfqpoint{3.004456in}{2.272201in}}%
\pgfpathcurveto{\pgfqpoint{2.998633in}{2.266377in}}{\pgfqpoint{2.995360in}{2.258477in}}{\pgfqpoint{2.995360in}{2.250241in}}%
\pgfpathcurveto{\pgfqpoint{2.995360in}{2.242005in}}{\pgfqpoint{2.998633in}{2.234104in}}{\pgfqpoint{3.004456in}{2.228281in}}%
\pgfpathcurveto{\pgfqpoint{3.010280in}{2.222457in}}{\pgfqpoint{3.018180in}{2.219184in}}{\pgfqpoint{3.026417in}{2.219184in}}%
\pgfpathclose%
\pgfusepath{stroke,fill}%
\end{pgfscope}%
\begin{pgfscope}%
\pgfpathrectangle{\pgfqpoint{0.100000in}{0.220728in}}{\pgfqpoint{3.696000in}{3.696000in}}%
\pgfusepath{clip}%
\pgfsetbuttcap%
\pgfsetroundjoin%
\definecolor{currentfill}{rgb}{0.121569,0.466667,0.705882}%
\pgfsetfillcolor{currentfill}%
\pgfsetfillopacity{0.798624}%
\pgfsetlinewidth{1.003750pt}%
\definecolor{currentstroke}{rgb}{0.121569,0.466667,0.705882}%
\pgfsetstrokecolor{currentstroke}%
\pgfsetstrokeopacity{0.798624}%
\pgfsetdash{}{0pt}%
\pgfpathmoveto{\pgfqpoint{3.025633in}{2.217789in}}%
\pgfpathcurveto{\pgfqpoint{3.033869in}{2.217789in}}{\pgfqpoint{3.041769in}{2.221062in}}{\pgfqpoint{3.047593in}{2.226886in}}%
\pgfpathcurveto{\pgfqpoint{3.053417in}{2.232710in}}{\pgfqpoint{3.056689in}{2.240610in}}{\pgfqpoint{3.056689in}{2.248846in}}%
\pgfpathcurveto{\pgfqpoint{3.056689in}{2.257082in}}{\pgfqpoint{3.053417in}{2.264982in}}{\pgfqpoint{3.047593in}{2.270806in}}%
\pgfpathcurveto{\pgfqpoint{3.041769in}{2.276630in}}{\pgfqpoint{3.033869in}{2.279902in}}{\pgfqpoint{3.025633in}{2.279902in}}%
\pgfpathcurveto{\pgfqpoint{3.017397in}{2.279902in}}{\pgfqpoint{3.009497in}{2.276630in}}{\pgfqpoint{3.003673in}{2.270806in}}%
\pgfpathcurveto{\pgfqpoint{2.997849in}{2.264982in}}{\pgfqpoint{2.994576in}{2.257082in}}{\pgfqpoint{2.994576in}{2.248846in}}%
\pgfpathcurveto{\pgfqpoint{2.994576in}{2.240610in}}{\pgfqpoint{2.997849in}{2.232710in}}{\pgfqpoint{3.003673in}{2.226886in}}%
\pgfpathcurveto{\pgfqpoint{3.009497in}{2.221062in}}{\pgfqpoint{3.017397in}{2.217789in}}{\pgfqpoint{3.025633in}{2.217789in}}%
\pgfpathclose%
\pgfusepath{stroke,fill}%
\end{pgfscope}%
\begin{pgfscope}%
\pgfpathrectangle{\pgfqpoint{0.100000in}{0.220728in}}{\pgfqpoint{3.696000in}{3.696000in}}%
\pgfusepath{clip}%
\pgfsetbuttcap%
\pgfsetroundjoin%
\definecolor{currentfill}{rgb}{0.121569,0.466667,0.705882}%
\pgfsetfillcolor{currentfill}%
\pgfsetfillopacity{0.798778}%
\pgfsetlinewidth{1.003750pt}%
\definecolor{currentstroke}{rgb}{0.121569,0.466667,0.705882}%
\pgfsetstrokecolor{currentstroke}%
\pgfsetstrokeopacity{0.798778}%
\pgfsetdash{}{0pt}%
\pgfpathmoveto{\pgfqpoint{3.025318in}{2.216932in}}%
\pgfpathcurveto{\pgfqpoint{3.033555in}{2.216932in}}{\pgfqpoint{3.041455in}{2.220204in}}{\pgfqpoint{3.047279in}{2.226028in}}%
\pgfpathcurveto{\pgfqpoint{3.053102in}{2.231852in}}{\pgfqpoint{3.056375in}{2.239752in}}{\pgfqpoint{3.056375in}{2.247988in}}%
\pgfpathcurveto{\pgfqpoint{3.056375in}{2.256225in}}{\pgfqpoint{3.053102in}{2.264125in}}{\pgfqpoint{3.047279in}{2.269949in}}%
\pgfpathcurveto{\pgfqpoint{3.041455in}{2.275773in}}{\pgfqpoint{3.033555in}{2.279045in}}{\pgfqpoint{3.025318in}{2.279045in}}%
\pgfpathcurveto{\pgfqpoint{3.017082in}{2.279045in}}{\pgfqpoint{3.009182in}{2.275773in}}{\pgfqpoint{3.003358in}{2.269949in}}%
\pgfpathcurveto{\pgfqpoint{2.997534in}{2.264125in}}{\pgfqpoint{2.994262in}{2.256225in}}{\pgfqpoint{2.994262in}{2.247988in}}%
\pgfpathcurveto{\pgfqpoint{2.994262in}{2.239752in}}{\pgfqpoint{2.997534in}{2.231852in}}{\pgfqpoint{3.003358in}{2.226028in}}%
\pgfpathcurveto{\pgfqpoint{3.009182in}{2.220204in}}{\pgfqpoint{3.017082in}{2.216932in}}{\pgfqpoint{3.025318in}{2.216932in}}%
\pgfpathclose%
\pgfusepath{stroke,fill}%
\end{pgfscope}%
\begin{pgfscope}%
\pgfpathrectangle{\pgfqpoint{0.100000in}{0.220728in}}{\pgfqpoint{3.696000in}{3.696000in}}%
\pgfusepath{clip}%
\pgfsetbuttcap%
\pgfsetroundjoin%
\definecolor{currentfill}{rgb}{0.121569,0.466667,0.705882}%
\pgfsetfillcolor{currentfill}%
\pgfsetfillopacity{0.799133}%
\pgfsetlinewidth{1.003750pt}%
\definecolor{currentstroke}{rgb}{0.121569,0.466667,0.705882}%
\pgfsetstrokecolor{currentstroke}%
\pgfsetstrokeopacity{0.799133}%
\pgfsetdash{}{0pt}%
\pgfpathmoveto{\pgfqpoint{3.024183in}{2.214929in}}%
\pgfpathcurveto{\pgfqpoint{3.032420in}{2.214929in}}{\pgfqpoint{3.040320in}{2.218202in}}{\pgfqpoint{3.046144in}{2.224025in}}%
\pgfpathcurveto{\pgfqpoint{3.051968in}{2.229849in}}{\pgfqpoint{3.055240in}{2.237749in}}{\pgfqpoint{3.055240in}{2.245986in}}%
\pgfpathcurveto{\pgfqpoint{3.055240in}{2.254222in}}{\pgfqpoint{3.051968in}{2.262122in}}{\pgfqpoint{3.046144in}{2.267946in}}%
\pgfpathcurveto{\pgfqpoint{3.040320in}{2.273770in}}{\pgfqpoint{3.032420in}{2.277042in}}{\pgfqpoint{3.024183in}{2.277042in}}%
\pgfpathcurveto{\pgfqpoint{3.015947in}{2.277042in}}{\pgfqpoint{3.008047in}{2.273770in}}{\pgfqpoint{3.002223in}{2.267946in}}%
\pgfpathcurveto{\pgfqpoint{2.996399in}{2.262122in}}{\pgfqpoint{2.993127in}{2.254222in}}{\pgfqpoint{2.993127in}{2.245986in}}%
\pgfpathcurveto{\pgfqpoint{2.993127in}{2.237749in}}{\pgfqpoint{2.996399in}{2.229849in}}{\pgfqpoint{3.002223in}{2.224025in}}%
\pgfpathcurveto{\pgfqpoint{3.008047in}{2.218202in}}{\pgfqpoint{3.015947in}{2.214929in}}{\pgfqpoint{3.024183in}{2.214929in}}%
\pgfpathclose%
\pgfusepath{stroke,fill}%
\end{pgfscope}%
\begin{pgfscope}%
\pgfpathrectangle{\pgfqpoint{0.100000in}{0.220728in}}{\pgfqpoint{3.696000in}{3.696000in}}%
\pgfusepath{clip}%
\pgfsetbuttcap%
\pgfsetroundjoin%
\definecolor{currentfill}{rgb}{0.121569,0.466667,0.705882}%
\pgfsetfillcolor{currentfill}%
\pgfsetfillopacity{0.799347}%
\pgfsetlinewidth{1.003750pt}%
\definecolor{currentstroke}{rgb}{0.121569,0.466667,0.705882}%
\pgfsetstrokecolor{currentstroke}%
\pgfsetstrokeopacity{0.799347}%
\pgfsetdash{}{0pt}%
\pgfpathmoveto{\pgfqpoint{3.023621in}{2.213823in}}%
\pgfpathcurveto{\pgfqpoint{3.031858in}{2.213823in}}{\pgfqpoint{3.039758in}{2.217096in}}{\pgfqpoint{3.045582in}{2.222920in}}%
\pgfpathcurveto{\pgfqpoint{3.051406in}{2.228744in}}{\pgfqpoint{3.054678in}{2.236644in}}{\pgfqpoint{3.054678in}{2.244880in}}%
\pgfpathcurveto{\pgfqpoint{3.054678in}{2.253116in}}{\pgfqpoint{3.051406in}{2.261016in}}{\pgfqpoint{3.045582in}{2.266840in}}%
\pgfpathcurveto{\pgfqpoint{3.039758in}{2.272664in}}{\pgfqpoint{3.031858in}{2.275936in}}{\pgfqpoint{3.023621in}{2.275936in}}%
\pgfpathcurveto{\pgfqpoint{3.015385in}{2.275936in}}{\pgfqpoint{3.007485in}{2.272664in}}{\pgfqpoint{3.001661in}{2.266840in}}%
\pgfpathcurveto{\pgfqpoint{2.995837in}{2.261016in}}{\pgfqpoint{2.992565in}{2.253116in}}{\pgfqpoint{2.992565in}{2.244880in}}%
\pgfpathcurveto{\pgfqpoint{2.992565in}{2.236644in}}{\pgfqpoint{2.995837in}{2.228744in}}{\pgfqpoint{3.001661in}{2.222920in}}%
\pgfpathcurveto{\pgfqpoint{3.007485in}{2.217096in}}{\pgfqpoint{3.015385in}{2.213823in}}{\pgfqpoint{3.023621in}{2.213823in}}%
\pgfpathclose%
\pgfusepath{stroke,fill}%
\end{pgfscope}%
\begin{pgfscope}%
\pgfpathrectangle{\pgfqpoint{0.100000in}{0.220728in}}{\pgfqpoint{3.696000in}{3.696000in}}%
\pgfusepath{clip}%
\pgfsetbuttcap%
\pgfsetroundjoin%
\definecolor{currentfill}{rgb}{0.121569,0.466667,0.705882}%
\pgfsetfillcolor{currentfill}%
\pgfsetfillopacity{0.799687}%
\pgfsetlinewidth{1.003750pt}%
\definecolor{currentstroke}{rgb}{0.121569,0.466667,0.705882}%
\pgfsetstrokecolor{currentstroke}%
\pgfsetstrokeopacity{0.799687}%
\pgfsetdash{}{0pt}%
\pgfpathmoveto{\pgfqpoint{3.022907in}{2.212041in}}%
\pgfpathcurveto{\pgfqpoint{3.031143in}{2.212041in}}{\pgfqpoint{3.039043in}{2.215313in}}{\pgfqpoint{3.044867in}{2.221137in}}%
\pgfpathcurveto{\pgfqpoint{3.050691in}{2.226961in}}{\pgfqpoint{3.053963in}{2.234861in}}{\pgfqpoint{3.053963in}{2.243097in}}%
\pgfpathcurveto{\pgfqpoint{3.053963in}{2.251333in}}{\pgfqpoint{3.050691in}{2.259233in}}{\pgfqpoint{3.044867in}{2.265057in}}%
\pgfpathcurveto{\pgfqpoint{3.039043in}{2.270881in}}{\pgfqpoint{3.031143in}{2.274154in}}{\pgfqpoint{3.022907in}{2.274154in}}%
\pgfpathcurveto{\pgfqpoint{3.014670in}{2.274154in}}{\pgfqpoint{3.006770in}{2.270881in}}{\pgfqpoint{3.000947in}{2.265057in}}%
\pgfpathcurveto{\pgfqpoint{2.995123in}{2.259233in}}{\pgfqpoint{2.991850in}{2.251333in}}{\pgfqpoint{2.991850in}{2.243097in}}%
\pgfpathcurveto{\pgfqpoint{2.991850in}{2.234861in}}{\pgfqpoint{2.995123in}{2.226961in}}{\pgfqpoint{3.000947in}{2.221137in}}%
\pgfpathcurveto{\pgfqpoint{3.006770in}{2.215313in}}{\pgfqpoint{3.014670in}{2.212041in}}{\pgfqpoint{3.022907in}{2.212041in}}%
\pgfpathclose%
\pgfusepath{stroke,fill}%
\end{pgfscope}%
\begin{pgfscope}%
\pgfpathrectangle{\pgfqpoint{0.100000in}{0.220728in}}{\pgfqpoint{3.696000in}{3.696000in}}%
\pgfusepath{clip}%
\pgfsetbuttcap%
\pgfsetroundjoin%
\definecolor{currentfill}{rgb}{0.121569,0.466667,0.705882}%
\pgfsetfillcolor{currentfill}%
\pgfsetfillopacity{0.800097}%
\pgfsetlinewidth{1.003750pt}%
\definecolor{currentstroke}{rgb}{0.121569,0.466667,0.705882}%
\pgfsetstrokecolor{currentstroke}%
\pgfsetstrokeopacity{0.800097}%
\pgfsetdash{}{0pt}%
\pgfpathmoveto{\pgfqpoint{3.021308in}{2.209212in}}%
\pgfpathcurveto{\pgfqpoint{3.029544in}{2.209212in}}{\pgfqpoint{3.037444in}{2.212484in}}{\pgfqpoint{3.043268in}{2.218308in}}%
\pgfpathcurveto{\pgfqpoint{3.049092in}{2.224132in}}{\pgfqpoint{3.052364in}{2.232032in}}{\pgfqpoint{3.052364in}{2.240268in}}%
\pgfpathcurveto{\pgfqpoint{3.052364in}{2.248505in}}{\pgfqpoint{3.049092in}{2.256405in}}{\pgfqpoint{3.043268in}{2.262229in}}%
\pgfpathcurveto{\pgfqpoint{3.037444in}{2.268053in}}{\pgfqpoint{3.029544in}{2.271325in}}{\pgfqpoint{3.021308in}{2.271325in}}%
\pgfpathcurveto{\pgfqpoint{3.013072in}{2.271325in}}{\pgfqpoint{3.005171in}{2.268053in}}{\pgfqpoint{2.999348in}{2.262229in}}%
\pgfpathcurveto{\pgfqpoint{2.993524in}{2.256405in}}{\pgfqpoint{2.990251in}{2.248505in}}{\pgfqpoint{2.990251in}{2.240268in}}%
\pgfpathcurveto{\pgfqpoint{2.990251in}{2.232032in}}{\pgfqpoint{2.993524in}{2.224132in}}{\pgfqpoint{2.999348in}{2.218308in}}%
\pgfpathcurveto{\pgfqpoint{3.005171in}{2.212484in}}{\pgfqpoint{3.013072in}{2.209212in}}{\pgfqpoint{3.021308in}{2.209212in}}%
\pgfpathclose%
\pgfusepath{stroke,fill}%
\end{pgfscope}%
\begin{pgfscope}%
\pgfpathrectangle{\pgfqpoint{0.100000in}{0.220728in}}{\pgfqpoint{3.696000in}{3.696000in}}%
\pgfusepath{clip}%
\pgfsetbuttcap%
\pgfsetroundjoin%
\definecolor{currentfill}{rgb}{0.121569,0.466667,0.705882}%
\pgfsetfillcolor{currentfill}%
\pgfsetfillopacity{0.800425}%
\pgfsetlinewidth{1.003750pt}%
\definecolor{currentstroke}{rgb}{0.121569,0.466667,0.705882}%
\pgfsetstrokecolor{currentstroke}%
\pgfsetstrokeopacity{0.800425}%
\pgfsetdash{}{0pt}%
\pgfpathmoveto{\pgfqpoint{3.020660in}{2.207798in}}%
\pgfpathcurveto{\pgfqpoint{3.028897in}{2.207798in}}{\pgfqpoint{3.036797in}{2.211070in}}{\pgfqpoint{3.042621in}{2.216894in}}%
\pgfpathcurveto{\pgfqpoint{3.048445in}{2.222718in}}{\pgfqpoint{3.051717in}{2.230618in}}{\pgfqpoint{3.051717in}{2.238855in}}%
\pgfpathcurveto{\pgfqpoint{3.051717in}{2.247091in}}{\pgfqpoint{3.048445in}{2.254991in}}{\pgfqpoint{3.042621in}{2.260815in}}%
\pgfpathcurveto{\pgfqpoint{3.036797in}{2.266639in}}{\pgfqpoint{3.028897in}{2.269911in}}{\pgfqpoint{3.020660in}{2.269911in}}%
\pgfpathcurveto{\pgfqpoint{3.012424in}{2.269911in}}{\pgfqpoint{3.004524in}{2.266639in}}{\pgfqpoint{2.998700in}{2.260815in}}%
\pgfpathcurveto{\pgfqpoint{2.992876in}{2.254991in}}{\pgfqpoint{2.989604in}{2.247091in}}{\pgfqpoint{2.989604in}{2.238855in}}%
\pgfpathcurveto{\pgfqpoint{2.989604in}{2.230618in}}{\pgfqpoint{2.992876in}{2.222718in}}{\pgfqpoint{2.998700in}{2.216894in}}%
\pgfpathcurveto{\pgfqpoint{3.004524in}{2.211070in}}{\pgfqpoint{3.012424in}{2.207798in}}{\pgfqpoint{3.020660in}{2.207798in}}%
\pgfpathclose%
\pgfusepath{stroke,fill}%
\end{pgfscope}%
\begin{pgfscope}%
\pgfpathrectangle{\pgfqpoint{0.100000in}{0.220728in}}{\pgfqpoint{3.696000in}{3.696000in}}%
\pgfusepath{clip}%
\pgfsetbuttcap%
\pgfsetroundjoin%
\definecolor{currentfill}{rgb}{0.121569,0.466667,0.705882}%
\pgfsetfillcolor{currentfill}%
\pgfsetfillopacity{0.800601}%
\pgfsetlinewidth{1.003750pt}%
\definecolor{currentstroke}{rgb}{0.121569,0.466667,0.705882}%
\pgfsetstrokecolor{currentstroke}%
\pgfsetstrokeopacity{0.800601}%
\pgfsetdash{}{0pt}%
\pgfpathmoveto{\pgfqpoint{3.020336in}{2.206972in}}%
\pgfpathcurveto{\pgfqpoint{3.028572in}{2.206972in}}{\pgfqpoint{3.036472in}{2.210244in}}{\pgfqpoint{3.042296in}{2.216068in}}%
\pgfpathcurveto{\pgfqpoint{3.048120in}{2.221892in}}{\pgfqpoint{3.051392in}{2.229792in}}{\pgfqpoint{3.051392in}{2.238028in}}%
\pgfpathcurveto{\pgfqpoint{3.051392in}{2.246265in}}{\pgfqpoint{3.048120in}{2.254165in}}{\pgfqpoint{3.042296in}{2.259989in}}%
\pgfpathcurveto{\pgfqpoint{3.036472in}{2.265813in}}{\pgfqpoint{3.028572in}{2.269085in}}{\pgfqpoint{3.020336in}{2.269085in}}%
\pgfpathcurveto{\pgfqpoint{3.012099in}{2.269085in}}{\pgfqpoint{3.004199in}{2.265813in}}{\pgfqpoint{2.998375in}{2.259989in}}%
\pgfpathcurveto{\pgfqpoint{2.992552in}{2.254165in}}{\pgfqpoint{2.989279in}{2.246265in}}{\pgfqpoint{2.989279in}{2.238028in}}%
\pgfpathcurveto{\pgfqpoint{2.989279in}{2.229792in}}{\pgfqpoint{2.992552in}{2.221892in}}{\pgfqpoint{2.998375in}{2.216068in}}%
\pgfpathcurveto{\pgfqpoint{3.004199in}{2.210244in}}{\pgfqpoint{3.012099in}{2.206972in}}{\pgfqpoint{3.020336in}{2.206972in}}%
\pgfpathclose%
\pgfusepath{stroke,fill}%
\end{pgfscope}%
\begin{pgfscope}%
\pgfpathrectangle{\pgfqpoint{0.100000in}{0.220728in}}{\pgfqpoint{3.696000in}{3.696000in}}%
\pgfusepath{clip}%
\pgfsetbuttcap%
\pgfsetroundjoin%
\definecolor{currentfill}{rgb}{0.121569,0.466667,0.705882}%
\pgfsetfillcolor{currentfill}%
\pgfsetfillopacity{0.800682}%
\pgfsetlinewidth{1.003750pt}%
\definecolor{currentstroke}{rgb}{0.121569,0.466667,0.705882}%
\pgfsetstrokecolor{currentstroke}%
\pgfsetstrokeopacity{0.800682}%
\pgfsetdash{}{0pt}%
\pgfpathmoveto{\pgfqpoint{3.020064in}{2.206565in}}%
\pgfpathcurveto{\pgfqpoint{3.028301in}{2.206565in}}{\pgfqpoint{3.036201in}{2.209837in}}{\pgfqpoint{3.042025in}{2.215661in}}%
\pgfpathcurveto{\pgfqpoint{3.047848in}{2.221485in}}{\pgfqpoint{3.051121in}{2.229385in}}{\pgfqpoint{3.051121in}{2.237621in}}%
\pgfpathcurveto{\pgfqpoint{3.051121in}{2.245857in}}{\pgfqpoint{3.047848in}{2.253757in}}{\pgfqpoint{3.042025in}{2.259581in}}%
\pgfpathcurveto{\pgfqpoint{3.036201in}{2.265405in}}{\pgfqpoint{3.028301in}{2.268678in}}{\pgfqpoint{3.020064in}{2.268678in}}%
\pgfpathcurveto{\pgfqpoint{3.011828in}{2.268678in}}{\pgfqpoint{3.003928in}{2.265405in}}{\pgfqpoint{2.998104in}{2.259581in}}%
\pgfpathcurveto{\pgfqpoint{2.992280in}{2.253757in}}{\pgfqpoint{2.989008in}{2.245857in}}{\pgfqpoint{2.989008in}{2.237621in}}%
\pgfpathcurveto{\pgfqpoint{2.989008in}{2.229385in}}{\pgfqpoint{2.992280in}{2.221485in}}{\pgfqpoint{2.998104in}{2.215661in}}%
\pgfpathcurveto{\pgfqpoint{3.003928in}{2.209837in}}{\pgfqpoint{3.011828in}{2.206565in}}{\pgfqpoint{3.020064in}{2.206565in}}%
\pgfpathclose%
\pgfusepath{stroke,fill}%
\end{pgfscope}%
\begin{pgfscope}%
\pgfpathrectangle{\pgfqpoint{0.100000in}{0.220728in}}{\pgfqpoint{3.696000in}{3.696000in}}%
\pgfusepath{clip}%
\pgfsetbuttcap%
\pgfsetroundjoin%
\definecolor{currentfill}{rgb}{0.121569,0.466667,0.705882}%
\pgfsetfillcolor{currentfill}%
\pgfsetfillopacity{0.801009}%
\pgfsetlinewidth{1.003750pt}%
\definecolor{currentstroke}{rgb}{0.121569,0.466667,0.705882}%
\pgfsetstrokecolor{currentstroke}%
\pgfsetstrokeopacity{0.801009}%
\pgfsetdash{}{0pt}%
\pgfpathmoveto{\pgfqpoint{3.019370in}{2.204773in}}%
\pgfpathcurveto{\pgfqpoint{3.027606in}{2.204773in}}{\pgfqpoint{3.035506in}{2.208045in}}{\pgfqpoint{3.041330in}{2.213869in}}%
\pgfpathcurveto{\pgfqpoint{3.047154in}{2.219693in}}{\pgfqpoint{3.050426in}{2.227593in}}{\pgfqpoint{3.050426in}{2.235829in}}%
\pgfpathcurveto{\pgfqpoint{3.050426in}{2.244066in}}{\pgfqpoint{3.047154in}{2.251966in}}{\pgfqpoint{3.041330in}{2.257790in}}%
\pgfpathcurveto{\pgfqpoint{3.035506in}{2.263613in}}{\pgfqpoint{3.027606in}{2.266886in}}{\pgfqpoint{3.019370in}{2.266886in}}%
\pgfpathcurveto{\pgfqpoint{3.011134in}{2.266886in}}{\pgfqpoint{3.003234in}{2.263613in}}{\pgfqpoint{2.997410in}{2.257790in}}%
\pgfpathcurveto{\pgfqpoint{2.991586in}{2.251966in}}{\pgfqpoint{2.988313in}{2.244066in}}{\pgfqpoint{2.988313in}{2.235829in}}%
\pgfpathcurveto{\pgfqpoint{2.988313in}{2.227593in}}{\pgfqpoint{2.991586in}{2.219693in}}{\pgfqpoint{2.997410in}{2.213869in}}%
\pgfpathcurveto{\pgfqpoint{3.003234in}{2.208045in}}{\pgfqpoint{3.011134in}{2.204773in}}{\pgfqpoint{3.019370in}{2.204773in}}%
\pgfpathclose%
\pgfusepath{stroke,fill}%
\end{pgfscope}%
\begin{pgfscope}%
\pgfpathrectangle{\pgfqpoint{0.100000in}{0.220728in}}{\pgfqpoint{3.696000in}{3.696000in}}%
\pgfusepath{clip}%
\pgfsetbuttcap%
\pgfsetroundjoin%
\definecolor{currentfill}{rgb}{0.121569,0.466667,0.705882}%
\pgfsetfillcolor{currentfill}%
\pgfsetfillopacity{0.801040}%
\pgfsetlinewidth{1.003750pt}%
\definecolor{currentstroke}{rgb}{0.121569,0.466667,0.705882}%
\pgfsetstrokecolor{currentstroke}%
\pgfsetstrokeopacity{0.801040}%
\pgfsetdash{}{0pt}%
\pgfpathmoveto{\pgfqpoint{1.337713in}{1.126926in}}%
\pgfpathcurveto{\pgfqpoint{1.345949in}{1.126926in}}{\pgfqpoint{1.353849in}{1.130199in}}{\pgfqpoint{1.359673in}{1.136023in}}%
\pgfpathcurveto{\pgfqpoint{1.365497in}{1.141847in}}{\pgfqpoint{1.368769in}{1.149747in}}{\pgfqpoint{1.368769in}{1.157983in}}%
\pgfpathcurveto{\pgfqpoint{1.368769in}{1.166219in}}{\pgfqpoint{1.365497in}{1.174119in}}{\pgfqpoint{1.359673in}{1.179943in}}%
\pgfpathcurveto{\pgfqpoint{1.353849in}{1.185767in}}{\pgfqpoint{1.345949in}{1.189039in}}{\pgfqpoint{1.337713in}{1.189039in}}%
\pgfpathcurveto{\pgfqpoint{1.329477in}{1.189039in}}{\pgfqpoint{1.321576in}{1.185767in}}{\pgfqpoint{1.315753in}{1.179943in}}%
\pgfpathcurveto{\pgfqpoint{1.309929in}{1.174119in}}{\pgfqpoint{1.306656in}{1.166219in}}{\pgfqpoint{1.306656in}{1.157983in}}%
\pgfpathcurveto{\pgfqpoint{1.306656in}{1.149747in}}{\pgfqpoint{1.309929in}{1.141847in}}{\pgfqpoint{1.315753in}{1.136023in}}%
\pgfpathcurveto{\pgfqpoint{1.321576in}{1.130199in}}{\pgfqpoint{1.329477in}{1.126926in}}{\pgfqpoint{1.337713in}{1.126926in}}%
\pgfpathclose%
\pgfusepath{stroke,fill}%
\end{pgfscope}%
\begin{pgfscope}%
\pgfpathrectangle{\pgfqpoint{0.100000in}{0.220728in}}{\pgfqpoint{3.696000in}{3.696000in}}%
\pgfusepath{clip}%
\pgfsetbuttcap%
\pgfsetroundjoin%
\definecolor{currentfill}{rgb}{0.121569,0.466667,0.705882}%
\pgfsetfillcolor{currentfill}%
\pgfsetfillopacity{0.801499}%
\pgfsetlinewidth{1.003750pt}%
\definecolor{currentstroke}{rgb}{0.121569,0.466667,0.705882}%
\pgfsetstrokecolor{currentstroke}%
\pgfsetstrokeopacity{0.801499}%
\pgfsetdash{}{0pt}%
\pgfpathmoveto{\pgfqpoint{3.018252in}{2.202492in}}%
\pgfpathcurveto{\pgfqpoint{3.026488in}{2.202492in}}{\pgfqpoint{3.034389in}{2.205765in}}{\pgfqpoint{3.040212in}{2.211589in}}%
\pgfpathcurveto{\pgfqpoint{3.046036in}{2.217413in}}{\pgfqpoint{3.049309in}{2.225313in}}{\pgfqpoint{3.049309in}{2.233549in}}%
\pgfpathcurveto{\pgfqpoint{3.049309in}{2.241785in}}{\pgfqpoint{3.046036in}{2.249685in}}{\pgfqpoint{3.040212in}{2.255509in}}%
\pgfpathcurveto{\pgfqpoint{3.034389in}{2.261333in}}{\pgfqpoint{3.026488in}{2.264605in}}{\pgfqpoint{3.018252in}{2.264605in}}%
\pgfpathcurveto{\pgfqpoint{3.010016in}{2.264605in}}{\pgfqpoint{3.002116in}{2.261333in}}{\pgfqpoint{2.996292in}{2.255509in}}%
\pgfpathcurveto{\pgfqpoint{2.990468in}{2.249685in}}{\pgfqpoint{2.987196in}{2.241785in}}{\pgfqpoint{2.987196in}{2.233549in}}%
\pgfpathcurveto{\pgfqpoint{2.987196in}{2.225313in}}{\pgfqpoint{2.990468in}{2.217413in}}{\pgfqpoint{2.996292in}{2.211589in}}%
\pgfpathcurveto{\pgfqpoint{3.002116in}{2.205765in}}{\pgfqpoint{3.010016in}{2.202492in}}{\pgfqpoint{3.018252in}{2.202492in}}%
\pgfpathclose%
\pgfusepath{stroke,fill}%
\end{pgfscope}%
\begin{pgfscope}%
\pgfpathrectangle{\pgfqpoint{0.100000in}{0.220728in}}{\pgfqpoint{3.696000in}{3.696000in}}%
\pgfusepath{clip}%
\pgfsetbuttcap%
\pgfsetroundjoin%
\definecolor{currentfill}{rgb}{0.121569,0.466667,0.705882}%
\pgfsetfillcolor{currentfill}%
\pgfsetfillopacity{0.802029}%
\pgfsetlinewidth{1.003750pt}%
\definecolor{currentstroke}{rgb}{0.121569,0.466667,0.705882}%
\pgfsetstrokecolor{currentstroke}%
\pgfsetstrokeopacity{0.802029}%
\pgfsetdash{}{0pt}%
\pgfpathmoveto{\pgfqpoint{3.016482in}{2.199814in}}%
\pgfpathcurveto{\pgfqpoint{3.024718in}{2.199814in}}{\pgfqpoint{3.032618in}{2.203086in}}{\pgfqpoint{3.038442in}{2.208910in}}%
\pgfpathcurveto{\pgfqpoint{3.044266in}{2.214734in}}{\pgfqpoint{3.047539in}{2.222634in}}{\pgfqpoint{3.047539in}{2.230870in}}%
\pgfpathcurveto{\pgfqpoint{3.047539in}{2.239107in}}{\pgfqpoint{3.044266in}{2.247007in}}{\pgfqpoint{3.038442in}{2.252831in}}%
\pgfpathcurveto{\pgfqpoint{3.032618in}{2.258655in}}{\pgfqpoint{3.024718in}{2.261927in}}{\pgfqpoint{3.016482in}{2.261927in}}%
\pgfpathcurveto{\pgfqpoint{3.008246in}{2.261927in}}{\pgfqpoint{3.000346in}{2.258655in}}{\pgfqpoint{2.994522in}{2.252831in}}%
\pgfpathcurveto{\pgfqpoint{2.988698in}{2.247007in}}{\pgfqpoint{2.985426in}{2.239107in}}{\pgfqpoint{2.985426in}{2.230870in}}%
\pgfpathcurveto{\pgfqpoint{2.985426in}{2.222634in}}{\pgfqpoint{2.988698in}{2.214734in}}{\pgfqpoint{2.994522in}{2.208910in}}%
\pgfpathcurveto{\pgfqpoint{3.000346in}{2.203086in}}{\pgfqpoint{3.008246in}{2.199814in}}{\pgfqpoint{3.016482in}{2.199814in}}%
\pgfpathclose%
\pgfusepath{stroke,fill}%
\end{pgfscope}%
\begin{pgfscope}%
\pgfpathrectangle{\pgfqpoint{0.100000in}{0.220728in}}{\pgfqpoint{3.696000in}{3.696000in}}%
\pgfusepath{clip}%
\pgfsetbuttcap%
\pgfsetroundjoin%
\definecolor{currentfill}{rgb}{0.121569,0.466667,0.705882}%
\pgfsetfillcolor{currentfill}%
\pgfsetfillopacity{0.802937}%
\pgfsetlinewidth{1.003750pt}%
\definecolor{currentstroke}{rgb}{0.121569,0.466667,0.705882}%
\pgfsetstrokecolor{currentstroke}%
\pgfsetstrokeopacity{0.802937}%
\pgfsetdash{}{0pt}%
\pgfpathmoveto{\pgfqpoint{3.014743in}{2.194561in}}%
\pgfpathcurveto{\pgfqpoint{3.022980in}{2.194561in}}{\pgfqpoint{3.030880in}{2.197833in}}{\pgfqpoint{3.036703in}{2.203657in}}%
\pgfpathcurveto{\pgfqpoint{3.042527in}{2.209481in}}{\pgfqpoint{3.045800in}{2.217381in}}{\pgfqpoint{3.045800in}{2.225617in}}%
\pgfpathcurveto{\pgfqpoint{3.045800in}{2.233853in}}{\pgfqpoint{3.042527in}{2.241753in}}{\pgfqpoint{3.036703in}{2.247577in}}%
\pgfpathcurveto{\pgfqpoint{3.030880in}{2.253401in}}{\pgfqpoint{3.022980in}{2.256674in}}{\pgfqpoint{3.014743in}{2.256674in}}%
\pgfpathcurveto{\pgfqpoint{3.006507in}{2.256674in}}{\pgfqpoint{2.998607in}{2.253401in}}{\pgfqpoint{2.992783in}{2.247577in}}%
\pgfpathcurveto{\pgfqpoint{2.986959in}{2.241753in}}{\pgfqpoint{2.983687in}{2.233853in}}{\pgfqpoint{2.983687in}{2.225617in}}%
\pgfpathcurveto{\pgfqpoint{2.983687in}{2.217381in}}{\pgfqpoint{2.986959in}{2.209481in}}{\pgfqpoint{2.992783in}{2.203657in}}%
\pgfpathcurveto{\pgfqpoint{2.998607in}{2.197833in}}{\pgfqpoint{3.006507in}{2.194561in}}{\pgfqpoint{3.014743in}{2.194561in}}%
\pgfpathclose%
\pgfusepath{stroke,fill}%
\end{pgfscope}%
\begin{pgfscope}%
\pgfpathrectangle{\pgfqpoint{0.100000in}{0.220728in}}{\pgfqpoint{3.696000in}{3.696000in}}%
\pgfusepath{clip}%
\pgfsetbuttcap%
\pgfsetroundjoin%
\definecolor{currentfill}{rgb}{0.121569,0.466667,0.705882}%
\pgfsetfillcolor{currentfill}%
\pgfsetfillopacity{0.803684}%
\pgfsetlinewidth{1.003750pt}%
\definecolor{currentstroke}{rgb}{0.121569,0.466667,0.705882}%
\pgfsetstrokecolor{currentstroke}%
\pgfsetstrokeopacity{0.803684}%
\pgfsetdash{}{0pt}%
\pgfpathmoveto{\pgfqpoint{1.354627in}{1.118463in}}%
\pgfpathcurveto{\pgfqpoint{1.362864in}{1.118463in}}{\pgfqpoint{1.370764in}{1.121735in}}{\pgfqpoint{1.376588in}{1.127559in}}%
\pgfpathcurveto{\pgfqpoint{1.382412in}{1.133383in}}{\pgfqpoint{1.385684in}{1.141283in}}{\pgfqpoint{1.385684in}{1.149520in}}%
\pgfpathcurveto{\pgfqpoint{1.385684in}{1.157756in}}{\pgfqpoint{1.382412in}{1.165656in}}{\pgfqpoint{1.376588in}{1.171480in}}%
\pgfpathcurveto{\pgfqpoint{1.370764in}{1.177304in}}{\pgfqpoint{1.362864in}{1.180576in}}{\pgfqpoint{1.354627in}{1.180576in}}%
\pgfpathcurveto{\pgfqpoint{1.346391in}{1.180576in}}{\pgfqpoint{1.338491in}{1.177304in}}{\pgfqpoint{1.332667in}{1.171480in}}%
\pgfpathcurveto{\pgfqpoint{1.326843in}{1.165656in}}{\pgfqpoint{1.323571in}{1.157756in}}{\pgfqpoint{1.323571in}{1.149520in}}%
\pgfpathcurveto{\pgfqpoint{1.323571in}{1.141283in}}{\pgfqpoint{1.326843in}{1.133383in}}{\pgfqpoint{1.332667in}{1.127559in}}%
\pgfpathcurveto{\pgfqpoint{1.338491in}{1.121735in}}{\pgfqpoint{1.346391in}{1.118463in}}{\pgfqpoint{1.354627in}{1.118463in}}%
\pgfpathclose%
\pgfusepath{stroke,fill}%
\end{pgfscope}%
\begin{pgfscope}%
\pgfpathrectangle{\pgfqpoint{0.100000in}{0.220728in}}{\pgfqpoint{3.696000in}{3.696000in}}%
\pgfusepath{clip}%
\pgfsetbuttcap%
\pgfsetroundjoin%
\definecolor{currentfill}{rgb}{0.121569,0.466667,0.705882}%
\pgfsetfillcolor{currentfill}%
\pgfsetfillopacity{0.804055}%
\pgfsetlinewidth{1.003750pt}%
\definecolor{currentstroke}{rgb}{0.121569,0.466667,0.705882}%
\pgfsetstrokecolor{currentstroke}%
\pgfsetstrokeopacity{0.804055}%
\pgfsetdash{}{0pt}%
\pgfpathmoveto{\pgfqpoint{3.012186in}{2.188975in}}%
\pgfpathcurveto{\pgfqpoint{3.020422in}{2.188975in}}{\pgfqpoint{3.028322in}{2.192247in}}{\pgfqpoint{3.034146in}{2.198071in}}%
\pgfpathcurveto{\pgfqpoint{3.039970in}{2.203895in}}{\pgfqpoint{3.043242in}{2.211795in}}{\pgfqpoint{3.043242in}{2.220031in}}%
\pgfpathcurveto{\pgfqpoint{3.043242in}{2.228267in}}{\pgfqpoint{3.039970in}{2.236167in}}{\pgfqpoint{3.034146in}{2.241991in}}%
\pgfpathcurveto{\pgfqpoint{3.028322in}{2.247815in}}{\pgfqpoint{3.020422in}{2.251088in}}{\pgfqpoint{3.012186in}{2.251088in}}%
\pgfpathcurveto{\pgfqpoint{3.003949in}{2.251088in}}{\pgfqpoint{2.996049in}{2.247815in}}{\pgfqpoint{2.990225in}{2.241991in}}%
\pgfpathcurveto{\pgfqpoint{2.984401in}{2.236167in}}{\pgfqpoint{2.981129in}{2.228267in}}{\pgfqpoint{2.981129in}{2.220031in}}%
\pgfpathcurveto{\pgfqpoint{2.981129in}{2.211795in}}{\pgfqpoint{2.984401in}{2.203895in}}{\pgfqpoint{2.990225in}{2.198071in}}%
\pgfpathcurveto{\pgfqpoint{2.996049in}{2.192247in}}{\pgfqpoint{3.003949in}{2.188975in}}{\pgfqpoint{3.012186in}{2.188975in}}%
\pgfpathclose%
\pgfusepath{stroke,fill}%
\end{pgfscope}%
\begin{pgfscope}%
\pgfpathrectangle{\pgfqpoint{0.100000in}{0.220728in}}{\pgfqpoint{3.696000in}{3.696000in}}%
\pgfusepath{clip}%
\pgfsetbuttcap%
\pgfsetroundjoin%
\definecolor{currentfill}{rgb}{0.121569,0.466667,0.705882}%
\pgfsetfillcolor{currentfill}%
\pgfsetfillopacity{0.805033}%
\pgfsetlinewidth{1.003750pt}%
\definecolor{currentstroke}{rgb}{0.121569,0.466667,0.705882}%
\pgfsetstrokecolor{currentstroke}%
\pgfsetstrokeopacity{0.805033}%
\pgfsetdash{}{0pt}%
\pgfpathmoveto{\pgfqpoint{3.008293in}{2.183157in}}%
\pgfpathcurveto{\pgfqpoint{3.016530in}{2.183157in}}{\pgfqpoint{3.024430in}{2.186429in}}{\pgfqpoint{3.030254in}{2.192253in}}%
\pgfpathcurveto{\pgfqpoint{3.036078in}{2.198077in}}{\pgfqpoint{3.039350in}{2.205977in}}{\pgfqpoint{3.039350in}{2.214213in}}%
\pgfpathcurveto{\pgfqpoint{3.039350in}{2.222449in}}{\pgfqpoint{3.036078in}{2.230350in}}{\pgfqpoint{3.030254in}{2.236173in}}%
\pgfpathcurveto{\pgfqpoint{3.024430in}{2.241997in}}{\pgfqpoint{3.016530in}{2.245270in}}{\pgfqpoint{3.008293in}{2.245270in}}%
\pgfpathcurveto{\pgfqpoint{3.000057in}{2.245270in}}{\pgfqpoint{2.992157in}{2.241997in}}{\pgfqpoint{2.986333in}{2.236173in}}%
\pgfpathcurveto{\pgfqpoint{2.980509in}{2.230350in}}{\pgfqpoint{2.977237in}{2.222449in}}{\pgfqpoint{2.977237in}{2.214213in}}%
\pgfpathcurveto{\pgfqpoint{2.977237in}{2.205977in}}{\pgfqpoint{2.980509in}{2.198077in}}{\pgfqpoint{2.986333in}{2.192253in}}%
\pgfpathcurveto{\pgfqpoint{2.992157in}{2.186429in}}{\pgfqpoint{3.000057in}{2.183157in}}{\pgfqpoint{3.008293in}{2.183157in}}%
\pgfpathclose%
\pgfusepath{stroke,fill}%
\end{pgfscope}%
\begin{pgfscope}%
\pgfpathrectangle{\pgfqpoint{0.100000in}{0.220728in}}{\pgfqpoint{3.696000in}{3.696000in}}%
\pgfusepath{clip}%
\pgfsetbuttcap%
\pgfsetroundjoin%
\definecolor{currentfill}{rgb}{0.121569,0.466667,0.705882}%
\pgfsetfillcolor{currentfill}%
\pgfsetfillopacity{0.806413}%
\pgfsetlinewidth{1.003750pt}%
\definecolor{currentstroke}{rgb}{0.121569,0.466667,0.705882}%
\pgfsetstrokecolor{currentstroke}%
\pgfsetstrokeopacity{0.806413}%
\pgfsetdash{}{0pt}%
\pgfpathmoveto{\pgfqpoint{3.005370in}{2.174578in}}%
\pgfpathcurveto{\pgfqpoint{3.013606in}{2.174578in}}{\pgfqpoint{3.021506in}{2.177850in}}{\pgfqpoint{3.027330in}{2.183674in}}%
\pgfpathcurveto{\pgfqpoint{3.033154in}{2.189498in}}{\pgfqpoint{3.036426in}{2.197398in}}{\pgfqpoint{3.036426in}{2.205635in}}%
\pgfpathcurveto{\pgfqpoint{3.036426in}{2.213871in}}{\pgfqpoint{3.033154in}{2.221771in}}{\pgfqpoint{3.027330in}{2.227595in}}%
\pgfpathcurveto{\pgfqpoint{3.021506in}{2.233419in}}{\pgfqpoint{3.013606in}{2.236691in}}{\pgfqpoint{3.005370in}{2.236691in}}%
\pgfpathcurveto{\pgfqpoint{2.997134in}{2.236691in}}{\pgfqpoint{2.989234in}{2.233419in}}{\pgfqpoint{2.983410in}{2.227595in}}%
\pgfpathcurveto{\pgfqpoint{2.977586in}{2.221771in}}{\pgfqpoint{2.974313in}{2.213871in}}{\pgfqpoint{2.974313in}{2.205635in}}%
\pgfpathcurveto{\pgfqpoint{2.974313in}{2.197398in}}{\pgfqpoint{2.977586in}{2.189498in}}{\pgfqpoint{2.983410in}{2.183674in}}%
\pgfpathcurveto{\pgfqpoint{2.989234in}{2.177850in}}{\pgfqpoint{2.997134in}{2.174578in}}{\pgfqpoint{3.005370in}{2.174578in}}%
\pgfpathclose%
\pgfusepath{stroke,fill}%
\end{pgfscope}%
\begin{pgfscope}%
\pgfpathrectangle{\pgfqpoint{0.100000in}{0.220728in}}{\pgfqpoint{3.696000in}{3.696000in}}%
\pgfusepath{clip}%
\pgfsetbuttcap%
\pgfsetroundjoin%
\definecolor{currentfill}{rgb}{0.121569,0.466667,0.705882}%
\pgfsetfillcolor{currentfill}%
\pgfsetfillopacity{0.806921}%
\pgfsetlinewidth{1.003750pt}%
\definecolor{currentstroke}{rgb}{0.121569,0.466667,0.705882}%
\pgfsetstrokecolor{currentstroke}%
\pgfsetstrokeopacity{0.806921}%
\pgfsetdash{}{0pt}%
\pgfpathmoveto{\pgfqpoint{1.369270in}{1.115552in}}%
\pgfpathcurveto{\pgfqpoint{1.377506in}{1.115552in}}{\pgfqpoint{1.385406in}{1.118824in}}{\pgfqpoint{1.391230in}{1.124648in}}%
\pgfpathcurveto{\pgfqpoint{1.397054in}{1.130472in}}{\pgfqpoint{1.400326in}{1.138372in}}{\pgfqpoint{1.400326in}{1.146608in}}%
\pgfpathcurveto{\pgfqpoint{1.400326in}{1.154844in}}{\pgfqpoint{1.397054in}{1.162744in}}{\pgfqpoint{1.391230in}{1.168568in}}%
\pgfpathcurveto{\pgfqpoint{1.385406in}{1.174392in}}{\pgfqpoint{1.377506in}{1.177665in}}{\pgfqpoint{1.369270in}{1.177665in}}%
\pgfpathcurveto{\pgfqpoint{1.361033in}{1.177665in}}{\pgfqpoint{1.353133in}{1.174392in}}{\pgfqpoint{1.347309in}{1.168568in}}%
\pgfpathcurveto{\pgfqpoint{1.341486in}{1.162744in}}{\pgfqpoint{1.338213in}{1.154844in}}{\pgfqpoint{1.338213in}{1.146608in}}%
\pgfpathcurveto{\pgfqpoint{1.338213in}{1.138372in}}{\pgfqpoint{1.341486in}{1.130472in}}{\pgfqpoint{1.347309in}{1.124648in}}%
\pgfpathcurveto{\pgfqpoint{1.353133in}{1.118824in}}{\pgfqpoint{1.361033in}{1.115552in}}{\pgfqpoint{1.369270in}{1.115552in}}%
\pgfpathclose%
\pgfusepath{stroke,fill}%
\end{pgfscope}%
\begin{pgfscope}%
\pgfpathrectangle{\pgfqpoint{0.100000in}{0.220728in}}{\pgfqpoint{3.696000in}{3.696000in}}%
\pgfusepath{clip}%
\pgfsetbuttcap%
\pgfsetroundjoin%
\definecolor{currentfill}{rgb}{0.121569,0.466667,0.705882}%
\pgfsetfillcolor{currentfill}%
\pgfsetfillopacity{0.807144}%
\pgfsetlinewidth{1.003750pt}%
\definecolor{currentstroke}{rgb}{0.121569,0.466667,0.705882}%
\pgfsetstrokecolor{currentstroke}%
\pgfsetstrokeopacity{0.807144}%
\pgfsetdash{}{0pt}%
\pgfpathmoveto{\pgfqpoint{3.003225in}{2.170291in}}%
\pgfpathcurveto{\pgfqpoint{3.011461in}{2.170291in}}{\pgfqpoint{3.019361in}{2.173564in}}{\pgfqpoint{3.025185in}{2.179388in}}%
\pgfpathcurveto{\pgfqpoint{3.031009in}{2.185211in}}{\pgfqpoint{3.034281in}{2.193112in}}{\pgfqpoint{3.034281in}{2.201348in}}%
\pgfpathcurveto{\pgfqpoint{3.034281in}{2.209584in}}{\pgfqpoint{3.031009in}{2.217484in}}{\pgfqpoint{3.025185in}{2.223308in}}%
\pgfpathcurveto{\pgfqpoint{3.019361in}{2.229132in}}{\pgfqpoint{3.011461in}{2.232404in}}{\pgfqpoint{3.003225in}{2.232404in}}%
\pgfpathcurveto{\pgfqpoint{2.994988in}{2.232404in}}{\pgfqpoint{2.987088in}{2.229132in}}{\pgfqpoint{2.981264in}{2.223308in}}%
\pgfpathcurveto{\pgfqpoint{2.975441in}{2.217484in}}{\pgfqpoint{2.972168in}{2.209584in}}{\pgfqpoint{2.972168in}{2.201348in}}%
\pgfpathcurveto{\pgfqpoint{2.972168in}{2.193112in}}{\pgfqpoint{2.975441in}{2.185211in}}{\pgfqpoint{2.981264in}{2.179388in}}%
\pgfpathcurveto{\pgfqpoint{2.987088in}{2.173564in}}{\pgfqpoint{2.994988in}{2.170291in}}{\pgfqpoint{3.003225in}{2.170291in}}%
\pgfpathclose%
\pgfusepath{stroke,fill}%
\end{pgfscope}%
\begin{pgfscope}%
\pgfpathrectangle{\pgfqpoint{0.100000in}{0.220728in}}{\pgfqpoint{3.696000in}{3.696000in}}%
\pgfusepath{clip}%
\pgfsetbuttcap%
\pgfsetroundjoin%
\definecolor{currentfill}{rgb}{0.121569,0.466667,0.705882}%
\pgfsetfillcolor{currentfill}%
\pgfsetfillopacity{0.807600}%
\pgfsetlinewidth{1.003750pt}%
\definecolor{currentstroke}{rgb}{0.121569,0.466667,0.705882}%
\pgfsetstrokecolor{currentstroke}%
\pgfsetstrokeopacity{0.807600}%
\pgfsetdash{}{0pt}%
\pgfpathmoveto{\pgfqpoint{3.001945in}{2.168292in}}%
\pgfpathcurveto{\pgfqpoint{3.010182in}{2.168292in}}{\pgfqpoint{3.018082in}{2.171565in}}{\pgfqpoint{3.023906in}{2.177389in}}%
\pgfpathcurveto{\pgfqpoint{3.029730in}{2.183213in}}{\pgfqpoint{3.033002in}{2.191113in}}{\pgfqpoint{3.033002in}{2.199349in}}%
\pgfpathcurveto{\pgfqpoint{3.033002in}{2.207585in}}{\pgfqpoint{3.029730in}{2.215485in}}{\pgfqpoint{3.023906in}{2.221309in}}%
\pgfpathcurveto{\pgfqpoint{3.018082in}{2.227133in}}{\pgfqpoint{3.010182in}{2.230405in}}{\pgfqpoint{3.001945in}{2.230405in}}%
\pgfpathcurveto{\pgfqpoint{2.993709in}{2.230405in}}{\pgfqpoint{2.985809in}{2.227133in}}{\pgfqpoint{2.979985in}{2.221309in}}%
\pgfpathcurveto{\pgfqpoint{2.974161in}{2.215485in}}{\pgfqpoint{2.970889in}{2.207585in}}{\pgfqpoint{2.970889in}{2.199349in}}%
\pgfpathcurveto{\pgfqpoint{2.970889in}{2.191113in}}{\pgfqpoint{2.974161in}{2.183213in}}{\pgfqpoint{2.979985in}{2.177389in}}%
\pgfpathcurveto{\pgfqpoint{2.985809in}{2.171565in}}{\pgfqpoint{2.993709in}{2.168292in}}{\pgfqpoint{3.001945in}{2.168292in}}%
\pgfpathclose%
\pgfusepath{stroke,fill}%
\end{pgfscope}%
\begin{pgfscope}%
\pgfpathrectangle{\pgfqpoint{0.100000in}{0.220728in}}{\pgfqpoint{3.696000in}{3.696000in}}%
\pgfusepath{clip}%
\pgfsetbuttcap%
\pgfsetroundjoin%
\definecolor{currentfill}{rgb}{0.121569,0.466667,0.705882}%
\pgfsetfillcolor{currentfill}%
\pgfsetfillopacity{0.807830}%
\pgfsetlinewidth{1.003750pt}%
\definecolor{currentstroke}{rgb}{0.121569,0.466667,0.705882}%
\pgfsetstrokecolor{currentstroke}%
\pgfsetstrokeopacity{0.807830}%
\pgfsetdash{}{0pt}%
\pgfpathmoveto{\pgfqpoint{3.001474in}{2.166851in}}%
\pgfpathcurveto{\pgfqpoint{3.009711in}{2.166851in}}{\pgfqpoint{3.017611in}{2.170123in}}{\pgfqpoint{3.023435in}{2.175947in}}%
\pgfpathcurveto{\pgfqpoint{3.029259in}{2.181771in}}{\pgfqpoint{3.032531in}{2.189671in}}{\pgfqpoint{3.032531in}{2.197907in}}%
\pgfpathcurveto{\pgfqpoint{3.032531in}{2.206143in}}{\pgfqpoint{3.029259in}{2.214043in}}{\pgfqpoint{3.023435in}{2.219867in}}%
\pgfpathcurveto{\pgfqpoint{3.017611in}{2.225691in}}{\pgfqpoint{3.009711in}{2.228964in}}{\pgfqpoint{3.001474in}{2.228964in}}%
\pgfpathcurveto{\pgfqpoint{2.993238in}{2.228964in}}{\pgfqpoint{2.985338in}{2.225691in}}{\pgfqpoint{2.979514in}{2.219867in}}%
\pgfpathcurveto{\pgfqpoint{2.973690in}{2.214043in}}{\pgfqpoint{2.970418in}{2.206143in}}{\pgfqpoint{2.970418in}{2.197907in}}%
\pgfpathcurveto{\pgfqpoint{2.970418in}{2.189671in}}{\pgfqpoint{2.973690in}{2.181771in}}{\pgfqpoint{2.979514in}{2.175947in}}%
\pgfpathcurveto{\pgfqpoint{2.985338in}{2.170123in}}{\pgfqpoint{2.993238in}{2.166851in}}{\pgfqpoint{3.001474in}{2.166851in}}%
\pgfpathclose%
\pgfusepath{stroke,fill}%
\end{pgfscope}%
\begin{pgfscope}%
\pgfpathrectangle{\pgfqpoint{0.100000in}{0.220728in}}{\pgfqpoint{3.696000in}{3.696000in}}%
\pgfusepath{clip}%
\pgfsetbuttcap%
\pgfsetroundjoin%
\definecolor{currentfill}{rgb}{0.121569,0.466667,0.705882}%
\pgfsetfillcolor{currentfill}%
\pgfsetfillopacity{0.808282}%
\pgfsetlinewidth{1.003750pt}%
\definecolor{currentstroke}{rgb}{0.121569,0.466667,0.705882}%
\pgfsetstrokecolor{currentstroke}%
\pgfsetstrokeopacity{0.808282}%
\pgfsetdash{}{0pt}%
\pgfpathmoveto{\pgfqpoint{3.000068in}{2.164379in}}%
\pgfpathcurveto{\pgfqpoint{3.008304in}{2.164379in}}{\pgfqpoint{3.016204in}{2.167651in}}{\pgfqpoint{3.022028in}{2.173475in}}%
\pgfpathcurveto{\pgfqpoint{3.027852in}{2.179299in}}{\pgfqpoint{3.031124in}{2.187199in}}{\pgfqpoint{3.031124in}{2.195435in}}%
\pgfpathcurveto{\pgfqpoint{3.031124in}{2.203672in}}{\pgfqpoint{3.027852in}{2.211572in}}{\pgfqpoint{3.022028in}{2.217396in}}%
\pgfpathcurveto{\pgfqpoint{3.016204in}{2.223220in}}{\pgfqpoint{3.008304in}{2.226492in}}{\pgfqpoint{3.000068in}{2.226492in}}%
\pgfpathcurveto{\pgfqpoint{2.991831in}{2.226492in}}{\pgfqpoint{2.983931in}{2.223220in}}{\pgfqpoint{2.978107in}{2.217396in}}%
\pgfpathcurveto{\pgfqpoint{2.972284in}{2.211572in}}{\pgfqpoint{2.969011in}{2.203672in}}{\pgfqpoint{2.969011in}{2.195435in}}%
\pgfpathcurveto{\pgfqpoint{2.969011in}{2.187199in}}{\pgfqpoint{2.972284in}{2.179299in}}{\pgfqpoint{2.978107in}{2.173475in}}%
\pgfpathcurveto{\pgfqpoint{2.983931in}{2.167651in}}{\pgfqpoint{2.991831in}{2.164379in}}{\pgfqpoint{3.000068in}{2.164379in}}%
\pgfpathclose%
\pgfusepath{stroke,fill}%
\end{pgfscope}%
\begin{pgfscope}%
\pgfpathrectangle{\pgfqpoint{0.100000in}{0.220728in}}{\pgfqpoint{3.696000in}{3.696000in}}%
\pgfusepath{clip}%
\pgfsetbuttcap%
\pgfsetroundjoin%
\definecolor{currentfill}{rgb}{0.121569,0.466667,0.705882}%
\pgfsetfillcolor{currentfill}%
\pgfsetfillopacity{0.808522}%
\pgfsetlinewidth{1.003750pt}%
\definecolor{currentstroke}{rgb}{0.121569,0.466667,0.705882}%
\pgfsetstrokecolor{currentstroke}%
\pgfsetstrokeopacity{0.808522}%
\pgfsetdash{}{0pt}%
\pgfpathmoveto{\pgfqpoint{2.999317in}{2.162950in}}%
\pgfpathcurveto{\pgfqpoint{3.007553in}{2.162950in}}{\pgfqpoint{3.015453in}{2.166222in}}{\pgfqpoint{3.021277in}{2.172046in}}%
\pgfpathcurveto{\pgfqpoint{3.027101in}{2.177870in}}{\pgfqpoint{3.030374in}{2.185770in}}{\pgfqpoint{3.030374in}{2.194006in}}%
\pgfpathcurveto{\pgfqpoint{3.030374in}{2.202243in}}{\pgfqpoint{3.027101in}{2.210143in}}{\pgfqpoint{3.021277in}{2.215967in}}%
\pgfpathcurveto{\pgfqpoint{3.015453in}{2.221790in}}{\pgfqpoint{3.007553in}{2.225063in}}{\pgfqpoint{2.999317in}{2.225063in}}%
\pgfpathcurveto{\pgfqpoint{2.991081in}{2.225063in}}{\pgfqpoint{2.983181in}{2.221790in}}{\pgfqpoint{2.977357in}{2.215967in}}%
\pgfpathcurveto{\pgfqpoint{2.971533in}{2.210143in}}{\pgfqpoint{2.968261in}{2.202243in}}{\pgfqpoint{2.968261in}{2.194006in}}%
\pgfpathcurveto{\pgfqpoint{2.968261in}{2.185770in}}{\pgfqpoint{2.971533in}{2.177870in}}{\pgfqpoint{2.977357in}{2.172046in}}%
\pgfpathcurveto{\pgfqpoint{2.983181in}{2.166222in}}{\pgfqpoint{2.991081in}{2.162950in}}{\pgfqpoint{2.999317in}{2.162950in}}%
\pgfpathclose%
\pgfusepath{stroke,fill}%
\end{pgfscope}%
\begin{pgfscope}%
\pgfpathrectangle{\pgfqpoint{0.100000in}{0.220728in}}{\pgfqpoint{3.696000in}{3.696000in}}%
\pgfusepath{clip}%
\pgfsetbuttcap%
\pgfsetroundjoin%
\definecolor{currentfill}{rgb}{0.121569,0.466667,0.705882}%
\pgfsetfillcolor{currentfill}%
\pgfsetfillopacity{0.808634}%
\pgfsetlinewidth{1.003750pt}%
\definecolor{currentstroke}{rgb}{0.121569,0.466667,0.705882}%
\pgfsetstrokecolor{currentstroke}%
\pgfsetstrokeopacity{0.808634}%
\pgfsetdash{}{0pt}%
\pgfpathmoveto{\pgfqpoint{1.381528in}{1.108854in}}%
\pgfpathcurveto{\pgfqpoint{1.389764in}{1.108854in}}{\pgfqpoint{1.397664in}{1.112127in}}{\pgfqpoint{1.403488in}{1.117951in}}%
\pgfpathcurveto{\pgfqpoint{1.409312in}{1.123775in}}{\pgfqpoint{1.412584in}{1.131675in}}{\pgfqpoint{1.412584in}{1.139911in}}%
\pgfpathcurveto{\pgfqpoint{1.412584in}{1.148147in}}{\pgfqpoint{1.409312in}{1.156047in}}{\pgfqpoint{1.403488in}{1.161871in}}%
\pgfpathcurveto{\pgfqpoint{1.397664in}{1.167695in}}{\pgfqpoint{1.389764in}{1.170967in}}{\pgfqpoint{1.381528in}{1.170967in}}%
\pgfpathcurveto{\pgfqpoint{1.373292in}{1.170967in}}{\pgfqpoint{1.365392in}{1.167695in}}{\pgfqpoint{1.359568in}{1.161871in}}%
\pgfpathcurveto{\pgfqpoint{1.353744in}{1.156047in}}{\pgfqpoint{1.350471in}{1.148147in}}{\pgfqpoint{1.350471in}{1.139911in}}%
\pgfpathcurveto{\pgfqpoint{1.350471in}{1.131675in}}{\pgfqpoint{1.353744in}{1.123775in}}{\pgfqpoint{1.359568in}{1.117951in}}%
\pgfpathcurveto{\pgfqpoint{1.365392in}{1.112127in}}{\pgfqpoint{1.373292in}{1.108854in}}{\pgfqpoint{1.381528in}{1.108854in}}%
\pgfpathclose%
\pgfusepath{stroke,fill}%
\end{pgfscope}%
\begin{pgfscope}%
\pgfpathrectangle{\pgfqpoint{0.100000in}{0.220728in}}{\pgfqpoint{3.696000in}{3.696000in}}%
\pgfusepath{clip}%
\pgfsetbuttcap%
\pgfsetroundjoin%
\definecolor{currentfill}{rgb}{0.121569,0.466667,0.705882}%
\pgfsetfillcolor{currentfill}%
\pgfsetfillopacity{0.808666}%
\pgfsetlinewidth{1.003750pt}%
\definecolor{currentstroke}{rgb}{0.121569,0.466667,0.705882}%
\pgfsetstrokecolor{currentstroke}%
\pgfsetstrokeopacity{0.808666}%
\pgfsetdash{}{0pt}%
\pgfpathmoveto{\pgfqpoint{2.998988in}{2.162122in}}%
\pgfpathcurveto{\pgfqpoint{3.007224in}{2.162122in}}{\pgfqpoint{3.015124in}{2.165395in}}{\pgfqpoint{3.020948in}{2.171219in}}%
\pgfpathcurveto{\pgfqpoint{3.026772in}{2.177043in}}{\pgfqpoint{3.030044in}{2.184943in}}{\pgfqpoint{3.030044in}{2.193179in}}%
\pgfpathcurveto{\pgfqpoint{3.030044in}{2.201415in}}{\pgfqpoint{3.026772in}{2.209315in}}{\pgfqpoint{3.020948in}{2.215139in}}%
\pgfpathcurveto{\pgfqpoint{3.015124in}{2.220963in}}{\pgfqpoint{3.007224in}{2.224235in}}{\pgfqpoint{2.998988in}{2.224235in}}%
\pgfpathcurveto{\pgfqpoint{2.990752in}{2.224235in}}{\pgfqpoint{2.982852in}{2.220963in}}{\pgfqpoint{2.977028in}{2.215139in}}%
\pgfpathcurveto{\pgfqpoint{2.971204in}{2.209315in}}{\pgfqpoint{2.967931in}{2.201415in}}{\pgfqpoint{2.967931in}{2.193179in}}%
\pgfpathcurveto{\pgfqpoint{2.967931in}{2.184943in}}{\pgfqpoint{2.971204in}{2.177043in}}{\pgfqpoint{2.977028in}{2.171219in}}%
\pgfpathcurveto{\pgfqpoint{2.982852in}{2.165395in}}{\pgfqpoint{2.990752in}{2.162122in}}{\pgfqpoint{2.998988in}{2.162122in}}%
\pgfpathclose%
\pgfusepath{stroke,fill}%
\end{pgfscope}%
\begin{pgfscope}%
\pgfpathrectangle{\pgfqpoint{0.100000in}{0.220728in}}{\pgfqpoint{3.696000in}{3.696000in}}%
\pgfusepath{clip}%
\pgfsetbuttcap%
\pgfsetroundjoin%
\definecolor{currentfill}{rgb}{0.121569,0.466667,0.705882}%
\pgfsetfillcolor{currentfill}%
\pgfsetfillopacity{0.809043}%
\pgfsetlinewidth{1.003750pt}%
\definecolor{currentstroke}{rgb}{0.121569,0.466667,0.705882}%
\pgfsetstrokecolor{currentstroke}%
\pgfsetstrokeopacity{0.809043}%
\pgfsetdash{}{0pt}%
\pgfpathmoveto{\pgfqpoint{2.997711in}{2.159865in}}%
\pgfpathcurveto{\pgfqpoint{3.005948in}{2.159865in}}{\pgfqpoint{3.013848in}{2.163137in}}{\pgfqpoint{3.019672in}{2.168961in}}%
\pgfpathcurveto{\pgfqpoint{3.025496in}{2.174785in}}{\pgfqpoint{3.028768in}{2.182685in}}{\pgfqpoint{3.028768in}{2.190921in}}%
\pgfpathcurveto{\pgfqpoint{3.028768in}{2.199158in}}{\pgfqpoint{3.025496in}{2.207058in}}{\pgfqpoint{3.019672in}{2.212882in}}%
\pgfpathcurveto{\pgfqpoint{3.013848in}{2.218706in}}{\pgfqpoint{3.005948in}{2.221978in}}{\pgfqpoint{2.997711in}{2.221978in}}%
\pgfpathcurveto{\pgfqpoint{2.989475in}{2.221978in}}{\pgfqpoint{2.981575in}{2.218706in}}{\pgfqpoint{2.975751in}{2.212882in}}%
\pgfpathcurveto{\pgfqpoint{2.969927in}{2.207058in}}{\pgfqpoint{2.966655in}{2.199158in}}{\pgfqpoint{2.966655in}{2.190921in}}%
\pgfpathcurveto{\pgfqpoint{2.966655in}{2.182685in}}{\pgfqpoint{2.969927in}{2.174785in}}{\pgfqpoint{2.975751in}{2.168961in}}%
\pgfpathcurveto{\pgfqpoint{2.981575in}{2.163137in}}{\pgfqpoint{2.989475in}{2.159865in}}{\pgfqpoint{2.997711in}{2.159865in}}%
\pgfpathclose%
\pgfusepath{stroke,fill}%
\end{pgfscope}%
\begin{pgfscope}%
\pgfpathrectangle{\pgfqpoint{0.100000in}{0.220728in}}{\pgfqpoint{3.696000in}{3.696000in}}%
\pgfusepath{clip}%
\pgfsetbuttcap%
\pgfsetroundjoin%
\definecolor{currentfill}{rgb}{0.121569,0.466667,0.705882}%
\pgfsetfillcolor{currentfill}%
\pgfsetfillopacity{0.809751}%
\pgfsetlinewidth{1.003750pt}%
\definecolor{currentstroke}{rgb}{0.121569,0.466667,0.705882}%
\pgfsetstrokecolor{currentstroke}%
\pgfsetstrokeopacity{0.809751}%
\pgfsetdash{}{0pt}%
\pgfpathmoveto{\pgfqpoint{2.996343in}{2.155530in}}%
\pgfpathcurveto{\pgfqpoint{3.004580in}{2.155530in}}{\pgfqpoint{3.012480in}{2.158803in}}{\pgfqpoint{3.018304in}{2.164627in}}%
\pgfpathcurveto{\pgfqpoint{3.024128in}{2.170451in}}{\pgfqpoint{3.027400in}{2.178351in}}{\pgfqpoint{3.027400in}{2.186587in}}%
\pgfpathcurveto{\pgfqpoint{3.027400in}{2.194823in}}{\pgfqpoint{3.024128in}{2.202723in}}{\pgfqpoint{3.018304in}{2.208547in}}%
\pgfpathcurveto{\pgfqpoint{3.012480in}{2.214371in}}{\pgfqpoint{3.004580in}{2.217643in}}{\pgfqpoint{2.996343in}{2.217643in}}%
\pgfpathcurveto{\pgfqpoint{2.988107in}{2.217643in}}{\pgfqpoint{2.980207in}{2.214371in}}{\pgfqpoint{2.974383in}{2.208547in}}%
\pgfpathcurveto{\pgfqpoint{2.968559in}{2.202723in}}{\pgfqpoint{2.965287in}{2.194823in}}{\pgfqpoint{2.965287in}{2.186587in}}%
\pgfpathcurveto{\pgfqpoint{2.965287in}{2.178351in}}{\pgfqpoint{2.968559in}{2.170451in}}{\pgfqpoint{2.974383in}{2.164627in}}%
\pgfpathcurveto{\pgfqpoint{2.980207in}{2.158803in}}{\pgfqpoint{2.988107in}{2.155530in}}{\pgfqpoint{2.996343in}{2.155530in}}%
\pgfpathclose%
\pgfusepath{stroke,fill}%
\end{pgfscope}%
\begin{pgfscope}%
\pgfpathrectangle{\pgfqpoint{0.100000in}{0.220728in}}{\pgfqpoint{3.696000in}{3.696000in}}%
\pgfusepath{clip}%
\pgfsetbuttcap%
\pgfsetroundjoin%
\definecolor{currentfill}{rgb}{0.121569,0.466667,0.705882}%
\pgfsetfillcolor{currentfill}%
\pgfsetfillopacity{0.810339}%
\pgfsetlinewidth{1.003750pt}%
\definecolor{currentstroke}{rgb}{0.121569,0.466667,0.705882}%
\pgfsetstrokecolor{currentstroke}%
\pgfsetstrokeopacity{0.810339}%
\pgfsetdash{}{0pt}%
\pgfpathmoveto{\pgfqpoint{1.393316in}{1.104920in}}%
\pgfpathcurveto{\pgfqpoint{1.401552in}{1.104920in}}{\pgfqpoint{1.409452in}{1.108193in}}{\pgfqpoint{1.415276in}{1.114017in}}%
\pgfpathcurveto{\pgfqpoint{1.421100in}{1.119841in}}{\pgfqpoint{1.424372in}{1.127741in}}{\pgfqpoint{1.424372in}{1.135977in}}%
\pgfpathcurveto{\pgfqpoint{1.424372in}{1.144213in}}{\pgfqpoint{1.421100in}{1.152113in}}{\pgfqpoint{1.415276in}{1.157937in}}%
\pgfpathcurveto{\pgfqpoint{1.409452in}{1.163761in}}{\pgfqpoint{1.401552in}{1.167033in}}{\pgfqpoint{1.393316in}{1.167033in}}%
\pgfpathcurveto{\pgfqpoint{1.385079in}{1.167033in}}{\pgfqpoint{1.377179in}{1.163761in}}{\pgfqpoint{1.371355in}{1.157937in}}%
\pgfpathcurveto{\pgfqpoint{1.365531in}{1.152113in}}{\pgfqpoint{1.362259in}{1.144213in}}{\pgfqpoint{1.362259in}{1.135977in}}%
\pgfpathcurveto{\pgfqpoint{1.362259in}{1.127741in}}{\pgfqpoint{1.365531in}{1.119841in}}{\pgfqpoint{1.371355in}{1.114017in}}%
\pgfpathcurveto{\pgfqpoint{1.377179in}{1.108193in}}{\pgfqpoint{1.385079in}{1.104920in}}{\pgfqpoint{1.393316in}{1.104920in}}%
\pgfpathclose%
\pgfusepath{stroke,fill}%
\end{pgfscope}%
\begin{pgfscope}%
\pgfpathrectangle{\pgfqpoint{0.100000in}{0.220728in}}{\pgfqpoint{3.696000in}{3.696000in}}%
\pgfusepath{clip}%
\pgfsetbuttcap%
\pgfsetroundjoin%
\definecolor{currentfill}{rgb}{0.121569,0.466667,0.705882}%
\pgfsetfillcolor{currentfill}%
\pgfsetfillopacity{0.810609}%
\pgfsetlinewidth{1.003750pt}%
\definecolor{currentstroke}{rgb}{0.121569,0.466667,0.705882}%
\pgfsetstrokecolor{currentstroke}%
\pgfsetstrokeopacity{0.810609}%
\pgfsetdash{}{0pt}%
\pgfpathmoveto{\pgfqpoint{2.994032in}{2.150541in}}%
\pgfpathcurveto{\pgfqpoint{3.002268in}{2.150541in}}{\pgfqpoint{3.010168in}{2.153813in}}{\pgfqpoint{3.015992in}{2.159637in}}%
\pgfpathcurveto{\pgfqpoint{3.021816in}{2.165461in}}{\pgfqpoint{3.025088in}{2.173361in}}{\pgfqpoint{3.025088in}{2.181597in}}%
\pgfpathcurveto{\pgfqpoint{3.025088in}{2.189833in}}{\pgfqpoint{3.021816in}{2.197733in}}{\pgfqpoint{3.015992in}{2.203557in}}%
\pgfpathcurveto{\pgfqpoint{3.010168in}{2.209381in}}{\pgfqpoint{3.002268in}{2.212654in}}{\pgfqpoint{2.994032in}{2.212654in}}%
\pgfpathcurveto{\pgfqpoint{2.985795in}{2.212654in}}{\pgfqpoint{2.977895in}{2.209381in}}{\pgfqpoint{2.972071in}{2.203557in}}%
\pgfpathcurveto{\pgfqpoint{2.966247in}{2.197733in}}{\pgfqpoint{2.962975in}{2.189833in}}{\pgfqpoint{2.962975in}{2.181597in}}%
\pgfpathcurveto{\pgfqpoint{2.962975in}{2.173361in}}{\pgfqpoint{2.966247in}{2.165461in}}{\pgfqpoint{2.972071in}{2.159637in}}%
\pgfpathcurveto{\pgfqpoint{2.977895in}{2.153813in}}{\pgfqpoint{2.985795in}{2.150541in}}{\pgfqpoint{2.994032in}{2.150541in}}%
\pgfpathclose%
\pgfusepath{stroke,fill}%
\end{pgfscope}%
\begin{pgfscope}%
\pgfpathrectangle{\pgfqpoint{0.100000in}{0.220728in}}{\pgfqpoint{3.696000in}{3.696000in}}%
\pgfusepath{clip}%
\pgfsetbuttcap%
\pgfsetroundjoin%
\definecolor{currentfill}{rgb}{0.121569,0.466667,0.705882}%
\pgfsetfillcolor{currentfill}%
\pgfsetfillopacity{0.811593}%
\pgfsetlinewidth{1.003750pt}%
\definecolor{currentstroke}{rgb}{0.121569,0.466667,0.705882}%
\pgfsetstrokecolor{currentstroke}%
\pgfsetstrokeopacity{0.811593}%
\pgfsetdash{}{0pt}%
\pgfpathmoveto{\pgfqpoint{2.990970in}{2.145355in}}%
\pgfpathcurveto{\pgfqpoint{2.999207in}{2.145355in}}{\pgfqpoint{3.007107in}{2.148627in}}{\pgfqpoint{3.012931in}{2.154451in}}%
\pgfpathcurveto{\pgfqpoint{3.018754in}{2.160275in}}{\pgfqpoint{3.022027in}{2.168175in}}{\pgfqpoint{3.022027in}{2.176411in}}%
\pgfpathcurveto{\pgfqpoint{3.022027in}{2.184648in}}{\pgfqpoint{3.018754in}{2.192548in}}{\pgfqpoint{3.012931in}{2.198372in}}%
\pgfpathcurveto{\pgfqpoint{3.007107in}{2.204196in}}{\pgfqpoint{2.999207in}{2.207468in}}{\pgfqpoint{2.990970in}{2.207468in}}%
\pgfpathcurveto{\pgfqpoint{2.982734in}{2.207468in}}{\pgfqpoint{2.974834in}{2.204196in}}{\pgfqpoint{2.969010in}{2.198372in}}%
\pgfpathcurveto{\pgfqpoint{2.963186in}{2.192548in}}{\pgfqpoint{2.959914in}{2.184648in}}{\pgfqpoint{2.959914in}{2.176411in}}%
\pgfpathcurveto{\pgfqpoint{2.959914in}{2.168175in}}{\pgfqpoint{2.963186in}{2.160275in}}{\pgfqpoint{2.969010in}{2.154451in}}%
\pgfpathcurveto{\pgfqpoint{2.974834in}{2.148627in}}{\pgfqpoint{2.982734in}{2.145355in}}{\pgfqpoint{2.990970in}{2.145355in}}%
\pgfpathclose%
\pgfusepath{stroke,fill}%
\end{pgfscope}%
\begin{pgfscope}%
\pgfpathrectangle{\pgfqpoint{0.100000in}{0.220728in}}{\pgfqpoint{3.696000in}{3.696000in}}%
\pgfusepath{clip}%
\pgfsetbuttcap%
\pgfsetroundjoin%
\definecolor{currentfill}{rgb}{0.121569,0.466667,0.705882}%
\pgfsetfillcolor{currentfill}%
\pgfsetfillopacity{0.812795}%
\pgfsetlinewidth{1.003750pt}%
\definecolor{currentstroke}{rgb}{0.121569,0.466667,0.705882}%
\pgfsetstrokecolor{currentstroke}%
\pgfsetstrokeopacity{0.812795}%
\pgfsetdash{}{0pt}%
\pgfpathmoveto{\pgfqpoint{1.403786in}{1.101441in}}%
\pgfpathcurveto{\pgfqpoint{1.412022in}{1.101441in}}{\pgfqpoint{1.419922in}{1.104713in}}{\pgfqpoint{1.425746in}{1.110537in}}%
\pgfpathcurveto{\pgfqpoint{1.431570in}{1.116361in}}{\pgfqpoint{1.434842in}{1.124261in}}{\pgfqpoint{1.434842in}{1.132497in}}%
\pgfpathcurveto{\pgfqpoint{1.434842in}{1.140733in}}{\pgfqpoint{1.431570in}{1.148634in}}{\pgfqpoint{1.425746in}{1.154457in}}%
\pgfpathcurveto{\pgfqpoint{1.419922in}{1.160281in}}{\pgfqpoint{1.412022in}{1.163554in}}{\pgfqpoint{1.403786in}{1.163554in}}%
\pgfpathcurveto{\pgfqpoint{1.395550in}{1.163554in}}{\pgfqpoint{1.387650in}{1.160281in}}{\pgfqpoint{1.381826in}{1.154457in}}%
\pgfpathcurveto{\pgfqpoint{1.376002in}{1.148634in}}{\pgfqpoint{1.372729in}{1.140733in}}{\pgfqpoint{1.372729in}{1.132497in}}%
\pgfpathcurveto{\pgfqpoint{1.372729in}{1.124261in}}{\pgfqpoint{1.376002in}{1.116361in}}{\pgfqpoint{1.381826in}{1.110537in}}%
\pgfpathcurveto{\pgfqpoint{1.387650in}{1.104713in}}{\pgfqpoint{1.395550in}{1.101441in}}{\pgfqpoint{1.403786in}{1.101441in}}%
\pgfpathclose%
\pgfusepath{stroke,fill}%
\end{pgfscope}%
\begin{pgfscope}%
\pgfpathrectangle{\pgfqpoint{0.100000in}{0.220728in}}{\pgfqpoint{3.696000in}{3.696000in}}%
\pgfusepath{clip}%
\pgfsetbuttcap%
\pgfsetroundjoin%
\definecolor{currentfill}{rgb}{0.121569,0.466667,0.705882}%
\pgfsetfillcolor{currentfill}%
\pgfsetfillopacity{0.813103}%
\pgfsetlinewidth{1.003750pt}%
\definecolor{currentstroke}{rgb}{0.121569,0.466667,0.705882}%
\pgfsetstrokecolor{currentstroke}%
\pgfsetstrokeopacity{0.813103}%
\pgfsetdash{}{0pt}%
\pgfpathmoveto{\pgfqpoint{2.988474in}{2.136625in}}%
\pgfpathcurveto{\pgfqpoint{2.996711in}{2.136625in}}{\pgfqpoint{3.004611in}{2.139897in}}{\pgfqpoint{3.010435in}{2.145721in}}%
\pgfpathcurveto{\pgfqpoint{3.016259in}{2.151545in}}{\pgfqpoint{3.019531in}{2.159445in}}{\pgfqpoint{3.019531in}{2.167682in}}%
\pgfpathcurveto{\pgfqpoint{3.019531in}{2.175918in}}{\pgfqpoint{3.016259in}{2.183818in}}{\pgfqpoint{3.010435in}{2.189642in}}%
\pgfpathcurveto{\pgfqpoint{3.004611in}{2.195466in}}{\pgfqpoint{2.996711in}{2.198738in}}{\pgfqpoint{2.988474in}{2.198738in}}%
\pgfpathcurveto{\pgfqpoint{2.980238in}{2.198738in}}{\pgfqpoint{2.972338in}{2.195466in}}{\pgfqpoint{2.966514in}{2.189642in}}%
\pgfpathcurveto{\pgfqpoint{2.960690in}{2.183818in}}{\pgfqpoint{2.957418in}{2.175918in}}{\pgfqpoint{2.957418in}{2.167682in}}%
\pgfpathcurveto{\pgfqpoint{2.957418in}{2.159445in}}{\pgfqpoint{2.960690in}{2.151545in}}{\pgfqpoint{2.966514in}{2.145721in}}%
\pgfpathcurveto{\pgfqpoint{2.972338in}{2.139897in}}{\pgfqpoint{2.980238in}{2.136625in}}{\pgfqpoint{2.988474in}{2.136625in}}%
\pgfpathclose%
\pgfusepath{stroke,fill}%
\end{pgfscope}%
\begin{pgfscope}%
\pgfpathrectangle{\pgfqpoint{0.100000in}{0.220728in}}{\pgfqpoint{3.696000in}{3.696000in}}%
\pgfusepath{clip}%
\pgfsetbuttcap%
\pgfsetroundjoin%
\definecolor{currentfill}{rgb}{0.121569,0.466667,0.705882}%
\pgfsetfillcolor{currentfill}%
\pgfsetfillopacity{0.813830}%
\pgfsetlinewidth{1.003750pt}%
\definecolor{currentstroke}{rgb}{0.121569,0.466667,0.705882}%
\pgfsetstrokecolor{currentstroke}%
\pgfsetstrokeopacity{0.813830}%
\pgfsetdash{}{0pt}%
\pgfpathmoveto{\pgfqpoint{2.986211in}{2.132238in}}%
\pgfpathcurveto{\pgfqpoint{2.994448in}{2.132238in}}{\pgfqpoint{3.002348in}{2.135510in}}{\pgfqpoint{3.008172in}{2.141334in}}%
\pgfpathcurveto{\pgfqpoint{3.013996in}{2.147158in}}{\pgfqpoint{3.017268in}{2.155058in}}{\pgfqpoint{3.017268in}{2.163294in}}%
\pgfpathcurveto{\pgfqpoint{3.017268in}{2.171531in}}{\pgfqpoint{3.013996in}{2.179431in}}{\pgfqpoint{3.008172in}{2.185255in}}%
\pgfpathcurveto{\pgfqpoint{3.002348in}{2.191079in}}{\pgfqpoint{2.994448in}{2.194351in}}{\pgfqpoint{2.986211in}{2.194351in}}%
\pgfpathcurveto{\pgfqpoint{2.977975in}{2.194351in}}{\pgfqpoint{2.970075in}{2.191079in}}{\pgfqpoint{2.964251in}{2.185255in}}%
\pgfpathcurveto{\pgfqpoint{2.958427in}{2.179431in}}{\pgfqpoint{2.955155in}{2.171531in}}{\pgfqpoint{2.955155in}{2.163294in}}%
\pgfpathcurveto{\pgfqpoint{2.955155in}{2.155058in}}{\pgfqpoint{2.958427in}{2.147158in}}{\pgfqpoint{2.964251in}{2.141334in}}%
\pgfpathcurveto{\pgfqpoint{2.970075in}{2.135510in}}{\pgfqpoint{2.977975in}{2.132238in}}{\pgfqpoint{2.986211in}{2.132238in}}%
\pgfpathclose%
\pgfusepath{stroke,fill}%
\end{pgfscope}%
\begin{pgfscope}%
\pgfpathrectangle{\pgfqpoint{0.100000in}{0.220728in}}{\pgfqpoint{3.696000in}{3.696000in}}%
\pgfusepath{clip}%
\pgfsetbuttcap%
\pgfsetroundjoin%
\definecolor{currentfill}{rgb}{0.121569,0.466667,0.705882}%
\pgfsetfillcolor{currentfill}%
\pgfsetfillopacity{0.814277}%
\pgfsetlinewidth{1.003750pt}%
\definecolor{currentstroke}{rgb}{0.121569,0.466667,0.705882}%
\pgfsetstrokecolor{currentstroke}%
\pgfsetstrokeopacity{0.814277}%
\pgfsetdash{}{0pt}%
\pgfpathmoveto{\pgfqpoint{2.984956in}{2.130037in}}%
\pgfpathcurveto{\pgfqpoint{2.993192in}{2.130037in}}{\pgfqpoint{3.001092in}{2.133310in}}{\pgfqpoint{3.006916in}{2.139134in}}%
\pgfpathcurveto{\pgfqpoint{3.012740in}{2.144958in}}{\pgfqpoint{3.016012in}{2.152858in}}{\pgfqpoint{3.016012in}{2.161094in}}%
\pgfpathcurveto{\pgfqpoint{3.016012in}{2.169330in}}{\pgfqpoint{3.012740in}{2.177230in}}{\pgfqpoint{3.006916in}{2.183054in}}%
\pgfpathcurveto{\pgfqpoint{3.001092in}{2.188878in}}{\pgfqpoint{2.993192in}{2.192150in}}{\pgfqpoint{2.984956in}{2.192150in}}%
\pgfpathcurveto{\pgfqpoint{2.976720in}{2.192150in}}{\pgfqpoint{2.968820in}{2.188878in}}{\pgfqpoint{2.962996in}{2.183054in}}%
\pgfpathcurveto{\pgfqpoint{2.957172in}{2.177230in}}{\pgfqpoint{2.953899in}{2.169330in}}{\pgfqpoint{2.953899in}{2.161094in}}%
\pgfpathcurveto{\pgfqpoint{2.953899in}{2.152858in}}{\pgfqpoint{2.957172in}{2.144958in}}{\pgfqpoint{2.962996in}{2.139134in}}%
\pgfpathcurveto{\pgfqpoint{2.968820in}{2.133310in}}{\pgfqpoint{2.976720in}{2.130037in}}{\pgfqpoint{2.984956in}{2.130037in}}%
\pgfpathclose%
\pgfusepath{stroke,fill}%
\end{pgfscope}%
\begin{pgfscope}%
\pgfpathrectangle{\pgfqpoint{0.100000in}{0.220728in}}{\pgfqpoint{3.696000in}{3.696000in}}%
\pgfusepath{clip}%
\pgfsetbuttcap%
\pgfsetroundjoin%
\definecolor{currentfill}{rgb}{0.121569,0.466667,0.705882}%
\pgfsetfillcolor{currentfill}%
\pgfsetfillopacity{0.814519}%
\pgfsetlinewidth{1.003750pt}%
\definecolor{currentstroke}{rgb}{0.121569,0.466667,0.705882}%
\pgfsetstrokecolor{currentstroke}%
\pgfsetstrokeopacity{0.814519}%
\pgfsetdash{}{0pt}%
\pgfpathmoveto{\pgfqpoint{2.984482in}{2.128589in}}%
\pgfpathcurveto{\pgfqpoint{2.992718in}{2.128589in}}{\pgfqpoint{3.000618in}{2.131861in}}{\pgfqpoint{3.006442in}{2.137685in}}%
\pgfpathcurveto{\pgfqpoint{3.012266in}{2.143509in}}{\pgfqpoint{3.015539in}{2.151409in}}{\pgfqpoint{3.015539in}{2.159645in}}%
\pgfpathcurveto{\pgfqpoint{3.015539in}{2.167881in}}{\pgfqpoint{3.012266in}{2.175781in}}{\pgfqpoint{3.006442in}{2.181605in}}%
\pgfpathcurveto{\pgfqpoint{3.000618in}{2.187429in}}{\pgfqpoint{2.992718in}{2.190702in}}{\pgfqpoint{2.984482in}{2.190702in}}%
\pgfpathcurveto{\pgfqpoint{2.976246in}{2.190702in}}{\pgfqpoint{2.968346in}{2.187429in}}{\pgfqpoint{2.962522in}{2.181605in}}%
\pgfpathcurveto{\pgfqpoint{2.956698in}{2.175781in}}{\pgfqpoint{2.953426in}{2.167881in}}{\pgfqpoint{2.953426in}{2.159645in}}%
\pgfpathcurveto{\pgfqpoint{2.953426in}{2.151409in}}{\pgfqpoint{2.956698in}{2.143509in}}{\pgfqpoint{2.962522in}{2.137685in}}%
\pgfpathcurveto{\pgfqpoint{2.968346in}{2.131861in}}{\pgfqpoint{2.976246in}{2.128589in}}{\pgfqpoint{2.984482in}{2.128589in}}%
\pgfpathclose%
\pgfusepath{stroke,fill}%
\end{pgfscope}%
\begin{pgfscope}%
\pgfpathrectangle{\pgfqpoint{0.100000in}{0.220728in}}{\pgfqpoint{3.696000in}{3.696000in}}%
\pgfusepath{clip}%
\pgfsetbuttcap%
\pgfsetroundjoin%
\definecolor{currentfill}{rgb}{0.121569,0.466667,0.705882}%
\pgfsetfillcolor{currentfill}%
\pgfsetfillopacity{0.814850}%
\pgfsetlinewidth{1.003750pt}%
\definecolor{currentstroke}{rgb}{0.121569,0.466667,0.705882}%
\pgfsetstrokecolor{currentstroke}%
\pgfsetstrokeopacity{0.814850}%
\pgfsetdash{}{0pt}%
\pgfpathmoveto{\pgfqpoint{1.412753in}{1.098856in}}%
\pgfpathcurveto{\pgfqpoint{1.420989in}{1.098856in}}{\pgfqpoint{1.428889in}{1.102128in}}{\pgfqpoint{1.434713in}{1.107952in}}%
\pgfpathcurveto{\pgfqpoint{1.440537in}{1.113776in}}{\pgfqpoint{1.443810in}{1.121676in}}{\pgfqpoint{1.443810in}{1.129913in}}%
\pgfpathcurveto{\pgfqpoint{1.443810in}{1.138149in}}{\pgfqpoint{1.440537in}{1.146049in}}{\pgfqpoint{1.434713in}{1.151873in}}%
\pgfpathcurveto{\pgfqpoint{1.428889in}{1.157697in}}{\pgfqpoint{1.420989in}{1.160969in}}{\pgfqpoint{1.412753in}{1.160969in}}%
\pgfpathcurveto{\pgfqpoint{1.404517in}{1.160969in}}{\pgfqpoint{1.396617in}{1.157697in}}{\pgfqpoint{1.390793in}{1.151873in}}%
\pgfpathcurveto{\pgfqpoint{1.384969in}{1.146049in}}{\pgfqpoint{1.381697in}{1.138149in}}{\pgfqpoint{1.381697in}{1.129913in}}%
\pgfpathcurveto{\pgfqpoint{1.381697in}{1.121676in}}{\pgfqpoint{1.384969in}{1.113776in}}{\pgfqpoint{1.390793in}{1.107952in}}%
\pgfpathcurveto{\pgfqpoint{1.396617in}{1.102128in}}{\pgfqpoint{1.404517in}{1.098856in}}{\pgfqpoint{1.412753in}{1.098856in}}%
\pgfpathclose%
\pgfusepath{stroke,fill}%
\end{pgfscope}%
\begin{pgfscope}%
\pgfpathrectangle{\pgfqpoint{0.100000in}{0.220728in}}{\pgfqpoint{3.696000in}{3.696000in}}%
\pgfusepath{clip}%
\pgfsetbuttcap%
\pgfsetroundjoin%
\definecolor{currentfill}{rgb}{0.121569,0.466667,0.705882}%
\pgfsetfillcolor{currentfill}%
\pgfsetfillopacity{0.815024}%
\pgfsetlinewidth{1.003750pt}%
\definecolor{currentstroke}{rgb}{0.121569,0.466667,0.705882}%
\pgfsetstrokecolor{currentstroke}%
\pgfsetstrokeopacity{0.815024}%
\pgfsetdash{}{0pt}%
\pgfpathmoveto{\pgfqpoint{2.982523in}{2.125120in}}%
\pgfpathcurveto{\pgfqpoint{2.990759in}{2.125120in}}{\pgfqpoint{2.998659in}{2.128392in}}{\pgfqpoint{3.004483in}{2.134216in}}%
\pgfpathcurveto{\pgfqpoint{3.010307in}{2.140040in}}{\pgfqpoint{3.013579in}{2.147940in}}{\pgfqpoint{3.013579in}{2.156176in}}%
\pgfpathcurveto{\pgfqpoint{3.013579in}{2.164413in}}{\pgfqpoint{3.010307in}{2.172313in}}{\pgfqpoint{3.004483in}{2.178137in}}%
\pgfpathcurveto{\pgfqpoint{2.998659in}{2.183960in}}{\pgfqpoint{2.990759in}{2.187233in}}{\pgfqpoint{2.982523in}{2.187233in}}%
\pgfpathcurveto{\pgfqpoint{2.974287in}{2.187233in}}{\pgfqpoint{2.966387in}{2.183960in}}{\pgfqpoint{2.960563in}{2.178137in}}%
\pgfpathcurveto{\pgfqpoint{2.954739in}{2.172313in}}{\pgfqpoint{2.951466in}{2.164413in}}{\pgfqpoint{2.951466in}{2.156176in}}%
\pgfpathcurveto{\pgfqpoint{2.951466in}{2.147940in}}{\pgfqpoint{2.954739in}{2.140040in}}{\pgfqpoint{2.960563in}{2.134216in}}%
\pgfpathcurveto{\pgfqpoint{2.966387in}{2.128392in}}{\pgfqpoint{2.974287in}{2.125120in}}{\pgfqpoint{2.982523in}{2.125120in}}%
\pgfpathclose%
\pgfusepath{stroke,fill}%
\end{pgfscope}%
\begin{pgfscope}%
\pgfpathrectangle{\pgfqpoint{0.100000in}{0.220728in}}{\pgfqpoint{3.696000in}{3.696000in}}%
\pgfusepath{clip}%
\pgfsetbuttcap%
\pgfsetroundjoin%
\definecolor{currentfill}{rgb}{0.121569,0.466667,0.705882}%
\pgfsetfillcolor{currentfill}%
\pgfsetfillopacity{0.815363}%
\pgfsetlinewidth{1.003750pt}%
\definecolor{currentstroke}{rgb}{0.121569,0.466667,0.705882}%
\pgfsetstrokecolor{currentstroke}%
\pgfsetstrokeopacity{0.815363}%
\pgfsetdash{}{0pt}%
\pgfpathmoveto{\pgfqpoint{2.981654in}{2.123198in}}%
\pgfpathcurveto{\pgfqpoint{2.989890in}{2.123198in}}{\pgfqpoint{2.997790in}{2.126471in}}{\pgfqpoint{3.003614in}{2.132295in}}%
\pgfpathcurveto{\pgfqpoint{3.009438in}{2.138119in}}{\pgfqpoint{3.012711in}{2.146019in}}{\pgfqpoint{3.012711in}{2.154255in}}%
\pgfpathcurveto{\pgfqpoint{3.012711in}{2.162491in}}{\pgfqpoint{3.009438in}{2.170391in}}{\pgfqpoint{3.003614in}{2.176215in}}%
\pgfpathcurveto{\pgfqpoint{2.997790in}{2.182039in}}{\pgfqpoint{2.989890in}{2.185311in}}{\pgfqpoint{2.981654in}{2.185311in}}%
\pgfpathcurveto{\pgfqpoint{2.973418in}{2.185311in}}{\pgfqpoint{2.965518in}{2.182039in}}{\pgfqpoint{2.959694in}{2.176215in}}%
\pgfpathcurveto{\pgfqpoint{2.953870in}{2.170391in}}{\pgfqpoint{2.950598in}{2.162491in}}{\pgfqpoint{2.950598in}{2.154255in}}%
\pgfpathcurveto{\pgfqpoint{2.950598in}{2.146019in}}{\pgfqpoint{2.953870in}{2.138119in}}{\pgfqpoint{2.959694in}{2.132295in}}%
\pgfpathcurveto{\pgfqpoint{2.965518in}{2.126471in}}{\pgfqpoint{2.973418in}{2.123198in}}{\pgfqpoint{2.981654in}{2.123198in}}%
\pgfpathclose%
\pgfusepath{stroke,fill}%
\end{pgfscope}%
\begin{pgfscope}%
\pgfpathrectangle{\pgfqpoint{0.100000in}{0.220728in}}{\pgfqpoint{3.696000in}{3.696000in}}%
\pgfusepath{clip}%
\pgfsetbuttcap%
\pgfsetroundjoin%
\definecolor{currentfill}{rgb}{0.121569,0.466667,0.705882}%
\pgfsetfillcolor{currentfill}%
\pgfsetfillopacity{0.815861}%
\pgfsetlinewidth{1.003750pt}%
\definecolor{currentstroke}{rgb}{0.121569,0.466667,0.705882}%
\pgfsetstrokecolor{currentstroke}%
\pgfsetstrokeopacity{0.815861}%
\pgfsetdash{}{0pt}%
\pgfpathmoveto{\pgfqpoint{2.980589in}{2.120418in}}%
\pgfpathcurveto{\pgfqpoint{2.988825in}{2.120418in}}{\pgfqpoint{2.996726in}{2.123691in}}{\pgfqpoint{3.002549in}{2.129515in}}%
\pgfpathcurveto{\pgfqpoint{3.008373in}{2.135338in}}{\pgfqpoint{3.011646in}{2.143239in}}{\pgfqpoint{3.011646in}{2.151475in}}%
\pgfpathcurveto{\pgfqpoint{3.011646in}{2.159711in}}{\pgfqpoint{3.008373in}{2.167611in}}{\pgfqpoint{3.002549in}{2.173435in}}%
\pgfpathcurveto{\pgfqpoint{2.996726in}{2.179259in}}{\pgfqpoint{2.988825in}{2.182531in}}{\pgfqpoint{2.980589in}{2.182531in}}%
\pgfpathcurveto{\pgfqpoint{2.972353in}{2.182531in}}{\pgfqpoint{2.964453in}{2.179259in}}{\pgfqpoint{2.958629in}{2.173435in}}%
\pgfpathcurveto{\pgfqpoint{2.952805in}{2.167611in}}{\pgfqpoint{2.949533in}{2.159711in}}{\pgfqpoint{2.949533in}{2.151475in}}%
\pgfpathcurveto{\pgfqpoint{2.949533in}{2.143239in}}{\pgfqpoint{2.952805in}{2.135338in}}{\pgfqpoint{2.958629in}{2.129515in}}%
\pgfpathcurveto{\pgfqpoint{2.964453in}{2.123691in}}{\pgfqpoint{2.972353in}{2.120418in}}{\pgfqpoint{2.980589in}{2.120418in}}%
\pgfpathclose%
\pgfusepath{stroke,fill}%
\end{pgfscope}%
\begin{pgfscope}%
\pgfpathrectangle{\pgfqpoint{0.100000in}{0.220728in}}{\pgfqpoint{3.696000in}{3.696000in}}%
\pgfusepath{clip}%
\pgfsetbuttcap%
\pgfsetroundjoin%
\definecolor{currentfill}{rgb}{0.121569,0.466667,0.705882}%
\pgfsetfillcolor{currentfill}%
\pgfsetfillopacity{0.816083}%
\pgfsetlinewidth{1.003750pt}%
\definecolor{currentstroke}{rgb}{0.121569,0.466667,0.705882}%
\pgfsetstrokecolor{currentstroke}%
\pgfsetstrokeopacity{0.816083}%
\pgfsetdash{}{0pt}%
\pgfpathmoveto{\pgfqpoint{2.979741in}{2.118988in}}%
\pgfpathcurveto{\pgfqpoint{2.987978in}{2.118988in}}{\pgfqpoint{2.995878in}{2.122261in}}{\pgfqpoint{3.001702in}{2.128085in}}%
\pgfpathcurveto{\pgfqpoint{3.007525in}{2.133909in}}{\pgfqpoint{3.010798in}{2.141809in}}{\pgfqpoint{3.010798in}{2.150045in}}%
\pgfpathcurveto{\pgfqpoint{3.010798in}{2.158281in}}{\pgfqpoint{3.007525in}{2.166181in}}{\pgfqpoint{3.001702in}{2.172005in}}%
\pgfpathcurveto{\pgfqpoint{2.995878in}{2.177829in}}{\pgfqpoint{2.987978in}{2.181101in}}{\pgfqpoint{2.979741in}{2.181101in}}%
\pgfpathcurveto{\pgfqpoint{2.971505in}{2.181101in}}{\pgfqpoint{2.963605in}{2.177829in}}{\pgfqpoint{2.957781in}{2.172005in}}%
\pgfpathcurveto{\pgfqpoint{2.951957in}{2.166181in}}{\pgfqpoint{2.948685in}{2.158281in}}{\pgfqpoint{2.948685in}{2.150045in}}%
\pgfpathcurveto{\pgfqpoint{2.948685in}{2.141809in}}{\pgfqpoint{2.951957in}{2.133909in}}{\pgfqpoint{2.957781in}{2.128085in}}%
\pgfpathcurveto{\pgfqpoint{2.963605in}{2.122261in}}{\pgfqpoint{2.971505in}{2.118988in}}{\pgfqpoint{2.979741in}{2.118988in}}%
\pgfpathclose%
\pgfusepath{stroke,fill}%
\end{pgfscope}%
\begin{pgfscope}%
\pgfpathrectangle{\pgfqpoint{0.100000in}{0.220728in}}{\pgfqpoint{3.696000in}{3.696000in}}%
\pgfusepath{clip}%
\pgfsetbuttcap%
\pgfsetroundjoin%
\definecolor{currentfill}{rgb}{0.121569,0.466667,0.705882}%
\pgfsetfillcolor{currentfill}%
\pgfsetfillopacity{0.816414}%
\pgfsetlinewidth{1.003750pt}%
\definecolor{currentstroke}{rgb}{0.121569,0.466667,0.705882}%
\pgfsetstrokecolor{currentstroke}%
\pgfsetstrokeopacity{0.816414}%
\pgfsetdash{}{0pt}%
\pgfpathmoveto{\pgfqpoint{1.419394in}{1.096673in}}%
\pgfpathcurveto{\pgfqpoint{1.427630in}{1.096673in}}{\pgfqpoint{1.435530in}{1.099945in}}{\pgfqpoint{1.441354in}{1.105769in}}%
\pgfpathcurveto{\pgfqpoint{1.447178in}{1.111593in}}{\pgfqpoint{1.450450in}{1.119493in}}{\pgfqpoint{1.450450in}{1.127730in}}%
\pgfpathcurveto{\pgfqpoint{1.450450in}{1.135966in}}{\pgfqpoint{1.447178in}{1.143866in}}{\pgfqpoint{1.441354in}{1.149690in}}%
\pgfpathcurveto{\pgfqpoint{1.435530in}{1.155514in}}{\pgfqpoint{1.427630in}{1.158786in}}{\pgfqpoint{1.419394in}{1.158786in}}%
\pgfpathcurveto{\pgfqpoint{1.411158in}{1.158786in}}{\pgfqpoint{1.403258in}{1.155514in}}{\pgfqpoint{1.397434in}{1.149690in}}%
\pgfpathcurveto{\pgfqpoint{1.391610in}{1.143866in}}{\pgfqpoint{1.388337in}{1.135966in}}{\pgfqpoint{1.388337in}{1.127730in}}%
\pgfpathcurveto{\pgfqpoint{1.388337in}{1.119493in}}{\pgfqpoint{1.391610in}{1.111593in}}{\pgfqpoint{1.397434in}{1.105769in}}%
\pgfpathcurveto{\pgfqpoint{1.403258in}{1.099945in}}{\pgfqpoint{1.411158in}{1.096673in}}{\pgfqpoint{1.419394in}{1.096673in}}%
\pgfpathclose%
\pgfusepath{stroke,fill}%
\end{pgfscope}%
\begin{pgfscope}%
\pgfpathrectangle{\pgfqpoint{0.100000in}{0.220728in}}{\pgfqpoint{3.696000in}{3.696000in}}%
\pgfusepath{clip}%
\pgfsetbuttcap%
\pgfsetroundjoin%
\definecolor{currentfill}{rgb}{0.121569,0.466667,0.705882}%
\pgfsetfillcolor{currentfill}%
\pgfsetfillopacity{0.816661}%
\pgfsetlinewidth{1.003750pt}%
\definecolor{currentstroke}{rgb}{0.121569,0.466667,0.705882}%
\pgfsetstrokecolor{currentstroke}%
\pgfsetstrokeopacity{0.816661}%
\pgfsetdash{}{0pt}%
\pgfpathmoveto{\pgfqpoint{2.978510in}{2.115832in}}%
\pgfpathcurveto{\pgfqpoint{2.986747in}{2.115832in}}{\pgfqpoint{2.994647in}{2.119105in}}{\pgfqpoint{3.000471in}{2.124929in}}%
\pgfpathcurveto{\pgfqpoint{3.006295in}{2.130753in}}{\pgfqpoint{3.009567in}{2.138653in}}{\pgfqpoint{3.009567in}{2.146889in}}%
\pgfpathcurveto{\pgfqpoint{3.009567in}{2.155125in}}{\pgfqpoint{3.006295in}{2.163025in}}{\pgfqpoint{3.000471in}{2.168849in}}%
\pgfpathcurveto{\pgfqpoint{2.994647in}{2.174673in}}{\pgfqpoint{2.986747in}{2.177945in}}{\pgfqpoint{2.978510in}{2.177945in}}%
\pgfpathcurveto{\pgfqpoint{2.970274in}{2.177945in}}{\pgfqpoint{2.962374in}{2.174673in}}{\pgfqpoint{2.956550in}{2.168849in}}%
\pgfpathcurveto{\pgfqpoint{2.950726in}{2.163025in}}{\pgfqpoint{2.947454in}{2.155125in}}{\pgfqpoint{2.947454in}{2.146889in}}%
\pgfpathcurveto{\pgfqpoint{2.947454in}{2.138653in}}{\pgfqpoint{2.950726in}{2.130753in}}{\pgfqpoint{2.956550in}{2.124929in}}%
\pgfpathcurveto{\pgfqpoint{2.962374in}{2.119105in}}{\pgfqpoint{2.970274in}{2.115832in}}{\pgfqpoint{2.978510in}{2.115832in}}%
\pgfpathclose%
\pgfusepath{stroke,fill}%
\end{pgfscope}%
\begin{pgfscope}%
\pgfpathrectangle{\pgfqpoint{0.100000in}{0.220728in}}{\pgfqpoint{3.696000in}{3.696000in}}%
\pgfusepath{clip}%
\pgfsetbuttcap%
\pgfsetroundjoin%
\definecolor{currentfill}{rgb}{0.121569,0.466667,0.705882}%
\pgfsetfillcolor{currentfill}%
\pgfsetfillopacity{0.817382}%
\pgfsetlinewidth{1.003750pt}%
\definecolor{currentstroke}{rgb}{0.121569,0.466667,0.705882}%
\pgfsetstrokecolor{currentstroke}%
\pgfsetstrokeopacity{0.817382}%
\pgfsetdash{}{0pt}%
\pgfpathmoveto{\pgfqpoint{2.976699in}{2.112121in}}%
\pgfpathcurveto{\pgfqpoint{2.984935in}{2.112121in}}{\pgfqpoint{2.992836in}{2.115394in}}{\pgfqpoint{2.998659in}{2.121218in}}%
\pgfpathcurveto{\pgfqpoint{3.004483in}{2.127041in}}{\pgfqpoint{3.007756in}{2.134942in}}{\pgfqpoint{3.007756in}{2.143178in}}%
\pgfpathcurveto{\pgfqpoint{3.007756in}{2.151414in}}{\pgfqpoint{3.004483in}{2.159314in}}{\pgfqpoint{2.998659in}{2.165138in}}%
\pgfpathcurveto{\pgfqpoint{2.992836in}{2.170962in}}{\pgfqpoint{2.984935in}{2.174234in}}{\pgfqpoint{2.976699in}{2.174234in}}%
\pgfpathcurveto{\pgfqpoint{2.968463in}{2.174234in}}{\pgfqpoint{2.960563in}{2.170962in}}{\pgfqpoint{2.954739in}{2.165138in}}%
\pgfpathcurveto{\pgfqpoint{2.948915in}{2.159314in}}{\pgfqpoint{2.945643in}{2.151414in}}{\pgfqpoint{2.945643in}{2.143178in}}%
\pgfpathcurveto{\pgfqpoint{2.945643in}{2.134942in}}{\pgfqpoint{2.948915in}{2.127041in}}{\pgfqpoint{2.954739in}{2.121218in}}%
\pgfpathcurveto{\pgfqpoint{2.960563in}{2.115394in}}{\pgfqpoint{2.968463in}{2.112121in}}{\pgfqpoint{2.976699in}{2.112121in}}%
\pgfpathclose%
\pgfusepath{stroke,fill}%
\end{pgfscope}%
\begin{pgfscope}%
\pgfpathrectangle{\pgfqpoint{0.100000in}{0.220728in}}{\pgfqpoint{3.696000in}{3.696000in}}%
\pgfusepath{clip}%
\pgfsetbuttcap%
\pgfsetroundjoin%
\definecolor{currentfill}{rgb}{0.121569,0.466667,0.705882}%
\pgfsetfillcolor{currentfill}%
\pgfsetfillopacity{0.817466}%
\pgfsetlinewidth{1.003750pt}%
\definecolor{currentstroke}{rgb}{0.121569,0.466667,0.705882}%
\pgfsetstrokecolor{currentstroke}%
\pgfsetstrokeopacity{0.817466}%
\pgfsetdash{}{0pt}%
\pgfpathmoveto{\pgfqpoint{1.424925in}{1.093578in}}%
\pgfpathcurveto{\pgfqpoint{1.433161in}{1.093578in}}{\pgfqpoint{1.441061in}{1.096850in}}{\pgfqpoint{1.446885in}{1.102674in}}%
\pgfpathcurveto{\pgfqpoint{1.452709in}{1.108498in}}{\pgfqpoint{1.455982in}{1.116398in}}{\pgfqpoint{1.455982in}{1.124634in}}%
\pgfpathcurveto{\pgfqpoint{1.455982in}{1.132870in}}{\pgfqpoint{1.452709in}{1.140770in}}{\pgfqpoint{1.446885in}{1.146594in}}%
\pgfpathcurveto{\pgfqpoint{1.441061in}{1.152418in}}{\pgfqpoint{1.433161in}{1.155691in}}{\pgfqpoint{1.424925in}{1.155691in}}%
\pgfpathcurveto{\pgfqpoint{1.416689in}{1.155691in}}{\pgfqpoint{1.408789in}{1.152418in}}{\pgfqpoint{1.402965in}{1.146594in}}%
\pgfpathcurveto{\pgfqpoint{1.397141in}{1.140770in}}{\pgfqpoint{1.393869in}{1.132870in}}{\pgfqpoint{1.393869in}{1.124634in}}%
\pgfpathcurveto{\pgfqpoint{1.393869in}{1.116398in}}{\pgfqpoint{1.397141in}{1.108498in}}{\pgfqpoint{1.402965in}{1.102674in}}%
\pgfpathcurveto{\pgfqpoint{1.408789in}{1.096850in}}{\pgfqpoint{1.416689in}{1.093578in}}{\pgfqpoint{1.424925in}{1.093578in}}%
\pgfpathclose%
\pgfusepath{stroke,fill}%
\end{pgfscope}%
\begin{pgfscope}%
\pgfpathrectangle{\pgfqpoint{0.100000in}{0.220728in}}{\pgfqpoint{3.696000in}{3.696000in}}%
\pgfusepath{clip}%
\pgfsetbuttcap%
\pgfsetroundjoin%
\definecolor{currentfill}{rgb}{0.121569,0.466667,0.705882}%
\pgfsetfillcolor{currentfill}%
\pgfsetfillopacity{0.818222}%
\pgfsetlinewidth{1.003750pt}%
\definecolor{currentstroke}{rgb}{0.121569,0.466667,0.705882}%
\pgfsetstrokecolor{currentstroke}%
\pgfsetstrokeopacity{0.818222}%
\pgfsetdash{}{0pt}%
\pgfpathmoveto{\pgfqpoint{2.974272in}{2.108275in}}%
\pgfpathcurveto{\pgfqpoint{2.982508in}{2.108275in}}{\pgfqpoint{2.990408in}{2.111547in}}{\pgfqpoint{2.996232in}{2.117371in}}%
\pgfpathcurveto{\pgfqpoint{3.002056in}{2.123195in}}{\pgfqpoint{3.005328in}{2.131095in}}{\pgfqpoint{3.005328in}{2.139331in}}%
\pgfpathcurveto{\pgfqpoint{3.005328in}{2.147567in}}{\pgfqpoint{3.002056in}{2.155467in}}{\pgfqpoint{2.996232in}{2.161291in}}%
\pgfpathcurveto{\pgfqpoint{2.990408in}{2.167115in}}{\pgfqpoint{2.982508in}{2.170388in}}{\pgfqpoint{2.974272in}{2.170388in}}%
\pgfpathcurveto{\pgfqpoint{2.966035in}{2.170388in}}{\pgfqpoint{2.958135in}{2.167115in}}{\pgfqpoint{2.952311in}{2.161291in}}%
\pgfpathcurveto{\pgfqpoint{2.946487in}{2.155467in}}{\pgfqpoint{2.943215in}{2.147567in}}{\pgfqpoint{2.943215in}{2.139331in}}%
\pgfpathcurveto{\pgfqpoint{2.943215in}{2.131095in}}{\pgfqpoint{2.946487in}{2.123195in}}{\pgfqpoint{2.952311in}{2.117371in}}%
\pgfpathcurveto{\pgfqpoint{2.958135in}{2.111547in}}{\pgfqpoint{2.966035in}{2.108275in}}{\pgfqpoint{2.974272in}{2.108275in}}%
\pgfpathclose%
\pgfusepath{stroke,fill}%
\end{pgfscope}%
\begin{pgfscope}%
\pgfpathrectangle{\pgfqpoint{0.100000in}{0.220728in}}{\pgfqpoint{3.696000in}{3.696000in}}%
\pgfusepath{clip}%
\pgfsetbuttcap%
\pgfsetroundjoin%
\definecolor{currentfill}{rgb}{0.121569,0.466667,0.705882}%
\pgfsetfillcolor{currentfill}%
\pgfsetfillopacity{0.819104}%
\pgfsetlinewidth{1.003750pt}%
\definecolor{currentstroke}{rgb}{0.121569,0.466667,0.705882}%
\pgfsetstrokecolor{currentstroke}%
\pgfsetstrokeopacity{0.819104}%
\pgfsetdash{}{0pt}%
\pgfpathmoveto{\pgfqpoint{2.972229in}{2.101927in}}%
\pgfpathcurveto{\pgfqpoint{2.980465in}{2.101927in}}{\pgfqpoint{2.988365in}{2.105199in}}{\pgfqpoint{2.994189in}{2.111023in}}%
\pgfpathcurveto{\pgfqpoint{3.000013in}{2.116847in}}{\pgfqpoint{3.003285in}{2.124747in}}{\pgfqpoint{3.003285in}{2.132984in}}%
\pgfpathcurveto{\pgfqpoint{3.003285in}{2.141220in}}{\pgfqpoint{3.000013in}{2.149120in}}{\pgfqpoint{2.994189in}{2.154944in}}%
\pgfpathcurveto{\pgfqpoint{2.988365in}{2.160768in}}{\pgfqpoint{2.980465in}{2.164040in}}{\pgfqpoint{2.972229in}{2.164040in}}%
\pgfpathcurveto{\pgfqpoint{2.963993in}{2.164040in}}{\pgfqpoint{2.956093in}{2.160768in}}{\pgfqpoint{2.950269in}{2.154944in}}%
\pgfpathcurveto{\pgfqpoint{2.944445in}{2.149120in}}{\pgfqpoint{2.941172in}{2.141220in}}{\pgfqpoint{2.941172in}{2.132984in}}%
\pgfpathcurveto{\pgfqpoint{2.941172in}{2.124747in}}{\pgfqpoint{2.944445in}{2.116847in}}{\pgfqpoint{2.950269in}{2.111023in}}%
\pgfpathcurveto{\pgfqpoint{2.956093in}{2.105199in}}{\pgfqpoint{2.963993in}{2.101927in}}{\pgfqpoint{2.972229in}{2.101927in}}%
\pgfpathclose%
\pgfusepath{stroke,fill}%
\end{pgfscope}%
\begin{pgfscope}%
\pgfpathrectangle{\pgfqpoint{0.100000in}{0.220728in}}{\pgfqpoint{3.696000in}{3.696000in}}%
\pgfusepath{clip}%
\pgfsetbuttcap%
\pgfsetroundjoin%
\definecolor{currentfill}{rgb}{0.121569,0.466667,0.705882}%
\pgfsetfillcolor{currentfill}%
\pgfsetfillopacity{0.819602}%
\pgfsetlinewidth{1.003750pt}%
\definecolor{currentstroke}{rgb}{0.121569,0.466667,0.705882}%
\pgfsetstrokecolor{currentstroke}%
\pgfsetstrokeopacity{0.819602}%
\pgfsetdash{}{0pt}%
\pgfpathmoveto{\pgfqpoint{1.434686in}{1.087955in}}%
\pgfpathcurveto{\pgfqpoint{1.442922in}{1.087955in}}{\pgfqpoint{1.450822in}{1.091227in}}{\pgfqpoint{1.456646in}{1.097051in}}%
\pgfpathcurveto{\pgfqpoint{1.462470in}{1.102875in}}{\pgfqpoint{1.465742in}{1.110775in}}{\pgfqpoint{1.465742in}{1.119011in}}%
\pgfpathcurveto{\pgfqpoint{1.465742in}{1.127247in}}{\pgfqpoint{1.462470in}{1.135148in}}{\pgfqpoint{1.456646in}{1.140971in}}%
\pgfpathcurveto{\pgfqpoint{1.450822in}{1.146795in}}{\pgfqpoint{1.442922in}{1.150068in}}{\pgfqpoint{1.434686in}{1.150068in}}%
\pgfpathcurveto{\pgfqpoint{1.426450in}{1.150068in}}{\pgfqpoint{1.418549in}{1.146795in}}{\pgfqpoint{1.412726in}{1.140971in}}%
\pgfpathcurveto{\pgfqpoint{1.406902in}{1.135148in}}{\pgfqpoint{1.403629in}{1.127247in}}{\pgfqpoint{1.403629in}{1.119011in}}%
\pgfpathcurveto{\pgfqpoint{1.403629in}{1.110775in}}{\pgfqpoint{1.406902in}{1.102875in}}{\pgfqpoint{1.412726in}{1.097051in}}%
\pgfpathcurveto{\pgfqpoint{1.418549in}{1.091227in}}{\pgfqpoint{1.426450in}{1.087955in}}{\pgfqpoint{1.434686in}{1.087955in}}%
\pgfpathclose%
\pgfusepath{stroke,fill}%
\end{pgfscope}%
\begin{pgfscope}%
\pgfpathrectangle{\pgfqpoint{0.100000in}{0.220728in}}{\pgfqpoint{3.696000in}{3.696000in}}%
\pgfusepath{clip}%
\pgfsetbuttcap%
\pgfsetroundjoin%
\definecolor{currentfill}{rgb}{0.121569,0.466667,0.705882}%
\pgfsetfillcolor{currentfill}%
\pgfsetfillopacity{0.820159}%
\pgfsetlinewidth{1.003750pt}%
\definecolor{currentstroke}{rgb}{0.121569,0.466667,0.705882}%
\pgfsetstrokecolor{currentstroke}%
\pgfsetstrokeopacity{0.820159}%
\pgfsetdash{}{0pt}%
\pgfpathmoveto{\pgfqpoint{2.968615in}{2.095633in}}%
\pgfpathcurveto{\pgfqpoint{2.976851in}{2.095633in}}{\pgfqpoint{2.984751in}{2.098905in}}{\pgfqpoint{2.990575in}{2.104729in}}%
\pgfpathcurveto{\pgfqpoint{2.996399in}{2.110553in}}{\pgfqpoint{2.999671in}{2.118453in}}{\pgfqpoint{2.999671in}{2.126689in}}%
\pgfpathcurveto{\pgfqpoint{2.999671in}{2.134926in}}{\pgfqpoint{2.996399in}{2.142826in}}{\pgfqpoint{2.990575in}{2.148650in}}%
\pgfpathcurveto{\pgfqpoint{2.984751in}{2.154473in}}{\pgfqpoint{2.976851in}{2.157746in}}{\pgfqpoint{2.968615in}{2.157746in}}%
\pgfpathcurveto{\pgfqpoint{2.960379in}{2.157746in}}{\pgfqpoint{2.952479in}{2.154473in}}{\pgfqpoint{2.946655in}{2.148650in}}%
\pgfpathcurveto{\pgfqpoint{2.940831in}{2.142826in}}{\pgfqpoint{2.937558in}{2.134926in}}{\pgfqpoint{2.937558in}{2.126689in}}%
\pgfpathcurveto{\pgfqpoint{2.937558in}{2.118453in}}{\pgfqpoint{2.940831in}{2.110553in}}{\pgfqpoint{2.946655in}{2.104729in}}%
\pgfpathcurveto{\pgfqpoint{2.952479in}{2.098905in}}{\pgfqpoint{2.960379in}{2.095633in}}{\pgfqpoint{2.968615in}{2.095633in}}%
\pgfpathclose%
\pgfusepath{stroke,fill}%
\end{pgfscope}%
\begin{pgfscope}%
\pgfpathrectangle{\pgfqpoint{0.100000in}{0.220728in}}{\pgfqpoint{3.696000in}{3.696000in}}%
\pgfusepath{clip}%
\pgfsetbuttcap%
\pgfsetroundjoin%
\definecolor{currentfill}{rgb}{0.121569,0.466667,0.705882}%
\pgfsetfillcolor{currentfill}%
\pgfsetfillopacity{0.821441}%
\pgfsetlinewidth{1.003750pt}%
\definecolor{currentstroke}{rgb}{0.121569,0.466667,0.705882}%
\pgfsetstrokecolor{currentstroke}%
\pgfsetstrokeopacity{0.821441}%
\pgfsetdash{}{0pt}%
\pgfpathmoveto{\pgfqpoint{2.964794in}{2.087334in}}%
\pgfpathcurveto{\pgfqpoint{2.973030in}{2.087334in}}{\pgfqpoint{2.980930in}{2.090607in}}{\pgfqpoint{2.986754in}{2.096430in}}%
\pgfpathcurveto{\pgfqpoint{2.992578in}{2.102254in}}{\pgfqpoint{2.995850in}{2.110154in}}{\pgfqpoint{2.995850in}{2.118391in}}%
\pgfpathcurveto{\pgfqpoint{2.995850in}{2.126627in}}{\pgfqpoint{2.992578in}{2.134527in}}{\pgfqpoint{2.986754in}{2.140351in}}%
\pgfpathcurveto{\pgfqpoint{2.980930in}{2.146175in}}{\pgfqpoint{2.973030in}{2.149447in}}{\pgfqpoint{2.964794in}{2.149447in}}%
\pgfpathcurveto{\pgfqpoint{2.956557in}{2.149447in}}{\pgfqpoint{2.948657in}{2.146175in}}{\pgfqpoint{2.942833in}{2.140351in}}%
\pgfpathcurveto{\pgfqpoint{2.937009in}{2.134527in}}{\pgfqpoint{2.933737in}{2.126627in}}{\pgfqpoint{2.933737in}{2.118391in}}%
\pgfpathcurveto{\pgfqpoint{2.933737in}{2.110154in}}{\pgfqpoint{2.937009in}{2.102254in}}{\pgfqpoint{2.942833in}{2.096430in}}%
\pgfpathcurveto{\pgfqpoint{2.948657in}{2.090607in}}{\pgfqpoint{2.956557in}{2.087334in}}{\pgfqpoint{2.964794in}{2.087334in}}%
\pgfpathclose%
\pgfusepath{stroke,fill}%
\end{pgfscope}%
\begin{pgfscope}%
\pgfpathrectangle{\pgfqpoint{0.100000in}{0.220728in}}{\pgfqpoint{3.696000in}{3.696000in}}%
\pgfusepath{clip}%
\pgfsetbuttcap%
\pgfsetroundjoin%
\definecolor{currentfill}{rgb}{0.121569,0.466667,0.705882}%
\pgfsetfillcolor{currentfill}%
\pgfsetfillopacity{0.821820}%
\pgfsetlinewidth{1.003750pt}%
\definecolor{currentstroke}{rgb}{0.121569,0.466667,0.705882}%
\pgfsetstrokecolor{currentstroke}%
\pgfsetstrokeopacity{0.821820}%
\pgfsetdash{}{0pt}%
\pgfpathmoveto{\pgfqpoint{1.444091in}{1.085777in}}%
\pgfpathcurveto{\pgfqpoint{1.452327in}{1.085777in}}{\pgfqpoint{1.460227in}{1.089050in}}{\pgfqpoint{1.466051in}{1.094874in}}%
\pgfpathcurveto{\pgfqpoint{1.471875in}{1.100698in}}{\pgfqpoint{1.475147in}{1.108598in}}{\pgfqpoint{1.475147in}{1.116834in}}%
\pgfpathcurveto{\pgfqpoint{1.475147in}{1.125070in}}{\pgfqpoint{1.471875in}{1.132970in}}{\pgfqpoint{1.466051in}{1.138794in}}%
\pgfpathcurveto{\pgfqpoint{1.460227in}{1.144618in}}{\pgfqpoint{1.452327in}{1.147890in}}{\pgfqpoint{1.444091in}{1.147890in}}%
\pgfpathcurveto{\pgfqpoint{1.435854in}{1.147890in}}{\pgfqpoint{1.427954in}{1.144618in}}{\pgfqpoint{1.422130in}{1.138794in}}%
\pgfpathcurveto{\pgfqpoint{1.416306in}{1.132970in}}{\pgfqpoint{1.413034in}{1.125070in}}{\pgfqpoint{1.413034in}{1.116834in}}%
\pgfpathcurveto{\pgfqpoint{1.413034in}{1.108598in}}{\pgfqpoint{1.416306in}{1.100698in}}{\pgfqpoint{1.422130in}{1.094874in}}%
\pgfpathcurveto{\pgfqpoint{1.427954in}{1.089050in}}{\pgfqpoint{1.435854in}{1.085777in}}{\pgfqpoint{1.444091in}{1.085777in}}%
\pgfpathclose%
\pgfusepath{stroke,fill}%
\end{pgfscope}%
\begin{pgfscope}%
\pgfpathrectangle{\pgfqpoint{0.100000in}{0.220728in}}{\pgfqpoint{3.696000in}{3.696000in}}%
\pgfusepath{clip}%
\pgfsetbuttcap%
\pgfsetroundjoin%
\definecolor{currentfill}{rgb}{0.121569,0.466667,0.705882}%
\pgfsetfillcolor{currentfill}%
\pgfsetfillopacity{0.823132}%
\pgfsetlinewidth{1.003750pt}%
\definecolor{currentstroke}{rgb}{0.121569,0.466667,0.705882}%
\pgfsetstrokecolor{currentstroke}%
\pgfsetstrokeopacity{0.823132}%
\pgfsetdash{}{0pt}%
\pgfpathmoveto{\pgfqpoint{2.961324in}{2.078936in}}%
\pgfpathcurveto{\pgfqpoint{2.969560in}{2.078936in}}{\pgfqpoint{2.977461in}{2.082208in}}{\pgfqpoint{2.983284in}{2.088032in}}%
\pgfpathcurveto{\pgfqpoint{2.989108in}{2.093856in}}{\pgfqpoint{2.992381in}{2.101756in}}{\pgfqpoint{2.992381in}{2.109992in}}%
\pgfpathcurveto{\pgfqpoint{2.992381in}{2.118229in}}{\pgfqpoint{2.989108in}{2.126129in}}{\pgfqpoint{2.983284in}{2.131953in}}%
\pgfpathcurveto{\pgfqpoint{2.977461in}{2.137777in}}{\pgfqpoint{2.969560in}{2.141049in}}{\pgfqpoint{2.961324in}{2.141049in}}%
\pgfpathcurveto{\pgfqpoint{2.953088in}{2.141049in}}{\pgfqpoint{2.945188in}{2.137777in}}{\pgfqpoint{2.939364in}{2.131953in}}%
\pgfpathcurveto{\pgfqpoint{2.933540in}{2.126129in}}{\pgfqpoint{2.930268in}{2.118229in}}{\pgfqpoint{2.930268in}{2.109992in}}%
\pgfpathcurveto{\pgfqpoint{2.930268in}{2.101756in}}{\pgfqpoint{2.933540in}{2.093856in}}{\pgfqpoint{2.939364in}{2.088032in}}%
\pgfpathcurveto{\pgfqpoint{2.945188in}{2.082208in}}{\pgfqpoint{2.953088in}{2.078936in}}{\pgfqpoint{2.961324in}{2.078936in}}%
\pgfpathclose%
\pgfusepath{stroke,fill}%
\end{pgfscope}%
\begin{pgfscope}%
\pgfpathrectangle{\pgfqpoint{0.100000in}{0.220728in}}{\pgfqpoint{3.696000in}{3.696000in}}%
\pgfusepath{clip}%
\pgfsetbuttcap%
\pgfsetroundjoin%
\definecolor{currentfill}{rgb}{0.121569,0.466667,0.705882}%
\pgfsetfillcolor{currentfill}%
\pgfsetfillopacity{0.823431}%
\pgfsetlinewidth{1.003750pt}%
\definecolor{currentstroke}{rgb}{0.121569,0.466667,0.705882}%
\pgfsetstrokecolor{currentstroke}%
\pgfsetstrokeopacity{0.823431}%
\pgfsetdash{}{0pt}%
\pgfpathmoveto{\pgfqpoint{1.451968in}{1.082242in}}%
\pgfpathcurveto{\pgfqpoint{1.460204in}{1.082242in}}{\pgfqpoint{1.468104in}{1.085514in}}{\pgfqpoint{1.473928in}{1.091338in}}%
\pgfpathcurveto{\pgfqpoint{1.479752in}{1.097162in}}{\pgfqpoint{1.483025in}{1.105062in}}{\pgfqpoint{1.483025in}{1.113298in}}%
\pgfpathcurveto{\pgfqpoint{1.483025in}{1.121535in}}{\pgfqpoint{1.479752in}{1.129435in}}{\pgfqpoint{1.473928in}{1.135259in}}%
\pgfpathcurveto{\pgfqpoint{1.468104in}{1.141083in}}{\pgfqpoint{1.460204in}{1.144355in}}{\pgfqpoint{1.451968in}{1.144355in}}%
\pgfpathcurveto{\pgfqpoint{1.443732in}{1.144355in}}{\pgfqpoint{1.435832in}{1.141083in}}{\pgfqpoint{1.430008in}{1.135259in}}%
\pgfpathcurveto{\pgfqpoint{1.424184in}{1.129435in}}{\pgfqpoint{1.420912in}{1.121535in}}{\pgfqpoint{1.420912in}{1.113298in}}%
\pgfpathcurveto{\pgfqpoint{1.420912in}{1.105062in}}{\pgfqpoint{1.424184in}{1.097162in}}{\pgfqpoint{1.430008in}{1.091338in}}%
\pgfpathcurveto{\pgfqpoint{1.435832in}{1.085514in}}{\pgfqpoint{1.443732in}{1.082242in}}{\pgfqpoint{1.451968in}{1.082242in}}%
\pgfpathclose%
\pgfusepath{stroke,fill}%
\end{pgfscope}%
\begin{pgfscope}%
\pgfpathrectangle{\pgfqpoint{0.100000in}{0.220728in}}{\pgfqpoint{3.696000in}{3.696000in}}%
\pgfusepath{clip}%
\pgfsetbuttcap%
\pgfsetroundjoin%
\definecolor{currentfill}{rgb}{0.121569,0.466667,0.705882}%
\pgfsetfillcolor{currentfill}%
\pgfsetfillopacity{0.823783}%
\pgfsetlinewidth{1.003750pt}%
\definecolor{currentstroke}{rgb}{0.121569,0.466667,0.705882}%
\pgfsetstrokecolor{currentstroke}%
\pgfsetstrokeopacity{0.823783}%
\pgfsetdash{}{0pt}%
\pgfpathmoveto{\pgfqpoint{2.958499in}{2.074329in}}%
\pgfpathcurveto{\pgfqpoint{2.966735in}{2.074329in}}{\pgfqpoint{2.974635in}{2.077602in}}{\pgfqpoint{2.980459in}{2.083426in}}%
\pgfpathcurveto{\pgfqpoint{2.986283in}{2.089250in}}{\pgfqpoint{2.989555in}{2.097150in}}{\pgfqpoint{2.989555in}{2.105386in}}%
\pgfpathcurveto{\pgfqpoint{2.989555in}{2.113622in}}{\pgfqpoint{2.986283in}{2.121522in}}{\pgfqpoint{2.980459in}{2.127346in}}%
\pgfpathcurveto{\pgfqpoint{2.974635in}{2.133170in}}{\pgfqpoint{2.966735in}{2.136442in}}{\pgfqpoint{2.958499in}{2.136442in}}%
\pgfpathcurveto{\pgfqpoint{2.950262in}{2.136442in}}{\pgfqpoint{2.942362in}{2.133170in}}{\pgfqpoint{2.936538in}{2.127346in}}%
\pgfpathcurveto{\pgfqpoint{2.930714in}{2.121522in}}{\pgfqpoint{2.927442in}{2.113622in}}{\pgfqpoint{2.927442in}{2.105386in}}%
\pgfpathcurveto{\pgfqpoint{2.927442in}{2.097150in}}{\pgfqpoint{2.930714in}{2.089250in}}{\pgfqpoint{2.936538in}{2.083426in}}%
\pgfpathcurveto{\pgfqpoint{2.942362in}{2.077602in}}{\pgfqpoint{2.950262in}{2.074329in}}{\pgfqpoint{2.958499in}{2.074329in}}%
\pgfpathclose%
\pgfusepath{stroke,fill}%
\end{pgfscope}%
\begin{pgfscope}%
\pgfpathrectangle{\pgfqpoint{0.100000in}{0.220728in}}{\pgfqpoint{3.696000in}{3.696000in}}%
\pgfusepath{clip}%
\pgfsetbuttcap%
\pgfsetroundjoin%
\definecolor{currentfill}{rgb}{0.121569,0.466667,0.705882}%
\pgfsetfillcolor{currentfill}%
\pgfsetfillopacity{0.825135}%
\pgfsetlinewidth{1.003750pt}%
\definecolor{currentstroke}{rgb}{0.121569,0.466667,0.705882}%
\pgfsetstrokecolor{currentstroke}%
\pgfsetstrokeopacity{0.825135}%
\pgfsetdash{}{0pt}%
\pgfpathmoveto{\pgfqpoint{2.956200in}{2.066425in}}%
\pgfpathcurveto{\pgfqpoint{2.964437in}{2.066425in}}{\pgfqpoint{2.972337in}{2.069697in}}{\pgfqpoint{2.978161in}{2.075521in}}%
\pgfpathcurveto{\pgfqpoint{2.983984in}{2.081345in}}{\pgfqpoint{2.987257in}{2.089245in}}{\pgfqpoint{2.987257in}{2.097481in}}%
\pgfpathcurveto{\pgfqpoint{2.987257in}{2.105717in}}{\pgfqpoint{2.983984in}{2.113617in}}{\pgfqpoint{2.978161in}{2.119441in}}%
\pgfpathcurveto{\pgfqpoint{2.972337in}{2.125265in}}{\pgfqpoint{2.964437in}{2.128538in}}{\pgfqpoint{2.956200in}{2.128538in}}%
\pgfpathcurveto{\pgfqpoint{2.947964in}{2.128538in}}{\pgfqpoint{2.940064in}{2.125265in}}{\pgfqpoint{2.934240in}{2.119441in}}%
\pgfpathcurveto{\pgfqpoint{2.928416in}{2.113617in}}{\pgfqpoint{2.925144in}{2.105717in}}{\pgfqpoint{2.925144in}{2.097481in}}%
\pgfpathcurveto{\pgfqpoint{2.925144in}{2.089245in}}{\pgfqpoint{2.928416in}{2.081345in}}{\pgfqpoint{2.934240in}{2.075521in}}%
\pgfpathcurveto{\pgfqpoint{2.940064in}{2.069697in}}{\pgfqpoint{2.947964in}{2.066425in}}{\pgfqpoint{2.956200in}{2.066425in}}%
\pgfpathclose%
\pgfusepath{stroke,fill}%
\end{pgfscope}%
\begin{pgfscope}%
\pgfpathrectangle{\pgfqpoint{0.100000in}{0.220728in}}{\pgfqpoint{3.696000in}{3.696000in}}%
\pgfusepath{clip}%
\pgfsetbuttcap%
\pgfsetroundjoin%
\definecolor{currentfill}{rgb}{0.121569,0.466667,0.705882}%
\pgfsetfillcolor{currentfill}%
\pgfsetfillopacity{0.825245}%
\pgfsetlinewidth{1.003750pt}%
\definecolor{currentstroke}{rgb}{0.121569,0.466667,0.705882}%
\pgfsetstrokecolor{currentstroke}%
\pgfsetstrokeopacity{0.825245}%
\pgfsetdash{}{0pt}%
\pgfpathmoveto{\pgfqpoint{1.458724in}{1.080984in}}%
\pgfpathcurveto{\pgfqpoint{1.466960in}{1.080984in}}{\pgfqpoint{1.474860in}{1.084256in}}{\pgfqpoint{1.480684in}{1.090080in}}%
\pgfpathcurveto{\pgfqpoint{1.486508in}{1.095904in}}{\pgfqpoint{1.489780in}{1.103804in}}{\pgfqpoint{1.489780in}{1.112041in}}%
\pgfpathcurveto{\pgfqpoint{1.489780in}{1.120277in}}{\pgfqpoint{1.486508in}{1.128177in}}{\pgfqpoint{1.480684in}{1.134001in}}%
\pgfpathcurveto{\pgfqpoint{1.474860in}{1.139825in}}{\pgfqpoint{1.466960in}{1.143097in}}{\pgfqpoint{1.458724in}{1.143097in}}%
\pgfpathcurveto{\pgfqpoint{1.450487in}{1.143097in}}{\pgfqpoint{1.442587in}{1.139825in}}{\pgfqpoint{1.436763in}{1.134001in}}%
\pgfpathcurveto{\pgfqpoint{1.430939in}{1.128177in}}{\pgfqpoint{1.427667in}{1.120277in}}{\pgfqpoint{1.427667in}{1.112041in}}%
\pgfpathcurveto{\pgfqpoint{1.427667in}{1.103804in}}{\pgfqpoint{1.430939in}{1.095904in}}{\pgfqpoint{1.436763in}{1.090080in}}%
\pgfpathcurveto{\pgfqpoint{1.442587in}{1.084256in}}{\pgfqpoint{1.450487in}{1.080984in}}{\pgfqpoint{1.458724in}{1.080984in}}%
\pgfpathclose%
\pgfusepath{stroke,fill}%
\end{pgfscope}%
\begin{pgfscope}%
\pgfpathrectangle{\pgfqpoint{0.100000in}{0.220728in}}{\pgfqpoint{3.696000in}{3.696000in}}%
\pgfusepath{clip}%
\pgfsetbuttcap%
\pgfsetroundjoin%
\definecolor{currentfill}{rgb}{0.121569,0.466667,0.705882}%
\pgfsetfillcolor{currentfill}%
\pgfsetfillopacity{0.825821}%
\pgfsetlinewidth{1.003750pt}%
\definecolor{currentstroke}{rgb}{0.121569,0.466667,0.705882}%
\pgfsetstrokecolor{currentstroke}%
\pgfsetstrokeopacity{0.825821}%
\pgfsetdash{}{0pt}%
\pgfpathmoveto{\pgfqpoint{2.954294in}{2.062420in}}%
\pgfpathcurveto{\pgfqpoint{2.962530in}{2.062420in}}{\pgfqpoint{2.970430in}{2.065692in}}{\pgfqpoint{2.976254in}{2.071516in}}%
\pgfpathcurveto{\pgfqpoint{2.982078in}{2.077340in}}{\pgfqpoint{2.985351in}{2.085240in}}{\pgfqpoint{2.985351in}{2.093476in}}%
\pgfpathcurveto{\pgfqpoint{2.985351in}{2.101713in}}{\pgfqpoint{2.982078in}{2.109613in}}{\pgfqpoint{2.976254in}{2.115437in}}%
\pgfpathcurveto{\pgfqpoint{2.970430in}{2.121261in}}{\pgfqpoint{2.962530in}{2.124533in}}{\pgfqpoint{2.954294in}{2.124533in}}%
\pgfpathcurveto{\pgfqpoint{2.946058in}{2.124533in}}{\pgfqpoint{2.938158in}{2.121261in}}{\pgfqpoint{2.932334in}{2.115437in}}%
\pgfpathcurveto{\pgfqpoint{2.926510in}{2.109613in}}{\pgfqpoint{2.923238in}{2.101713in}}{\pgfqpoint{2.923238in}{2.093476in}}%
\pgfpathcurveto{\pgfqpoint{2.923238in}{2.085240in}}{\pgfqpoint{2.926510in}{2.077340in}}{\pgfqpoint{2.932334in}{2.071516in}}%
\pgfpathcurveto{\pgfqpoint{2.938158in}{2.065692in}}{\pgfqpoint{2.946058in}{2.062420in}}{\pgfqpoint{2.954294in}{2.062420in}}%
\pgfpathclose%
\pgfusepath{stroke,fill}%
\end{pgfscope}%
\begin{pgfscope}%
\pgfpathrectangle{\pgfqpoint{0.100000in}{0.220728in}}{\pgfqpoint{3.696000in}{3.696000in}}%
\pgfusepath{clip}%
\pgfsetbuttcap%
\pgfsetroundjoin%
\definecolor{currentfill}{rgb}{0.121569,0.466667,0.705882}%
\pgfsetfillcolor{currentfill}%
\pgfsetfillopacity{0.826014}%
\pgfsetlinewidth{1.003750pt}%
\definecolor{currentstroke}{rgb}{0.121569,0.466667,0.705882}%
\pgfsetstrokecolor{currentstroke}%
\pgfsetstrokeopacity{0.826014}%
\pgfsetdash{}{0pt}%
\pgfpathmoveto{\pgfqpoint{1.464704in}{1.076787in}}%
\pgfpathcurveto{\pgfqpoint{1.472940in}{1.076787in}}{\pgfqpoint{1.480840in}{1.080059in}}{\pgfqpoint{1.486664in}{1.085883in}}%
\pgfpathcurveto{\pgfqpoint{1.492488in}{1.091707in}}{\pgfqpoint{1.495760in}{1.099607in}}{\pgfqpoint{1.495760in}{1.107844in}}%
\pgfpathcurveto{\pgfqpoint{1.495760in}{1.116080in}}{\pgfqpoint{1.492488in}{1.123980in}}{\pgfqpoint{1.486664in}{1.129804in}}%
\pgfpathcurveto{\pgfqpoint{1.480840in}{1.135628in}}{\pgfqpoint{1.472940in}{1.138900in}}{\pgfqpoint{1.464704in}{1.138900in}}%
\pgfpathcurveto{\pgfqpoint{1.456468in}{1.138900in}}{\pgfqpoint{1.448568in}{1.135628in}}{\pgfqpoint{1.442744in}{1.129804in}}%
\pgfpathcurveto{\pgfqpoint{1.436920in}{1.123980in}}{\pgfqpoint{1.433647in}{1.116080in}}{\pgfqpoint{1.433647in}{1.107844in}}%
\pgfpathcurveto{\pgfqpoint{1.433647in}{1.099607in}}{\pgfqpoint{1.436920in}{1.091707in}}{\pgfqpoint{1.442744in}{1.085883in}}%
\pgfpathcurveto{\pgfqpoint{1.448568in}{1.080059in}}{\pgfqpoint{1.456468in}{1.076787in}}{\pgfqpoint{1.464704in}{1.076787in}}%
\pgfpathclose%
\pgfusepath{stroke,fill}%
\end{pgfscope}%
\begin{pgfscope}%
\pgfpathrectangle{\pgfqpoint{0.100000in}{0.220728in}}{\pgfqpoint{3.696000in}{3.696000in}}%
\pgfusepath{clip}%
\pgfsetbuttcap%
\pgfsetroundjoin%
\definecolor{currentfill}{rgb}{0.121569,0.466667,0.705882}%
\pgfsetfillcolor{currentfill}%
\pgfsetfillopacity{0.826213}%
\pgfsetlinewidth{1.003750pt}%
\definecolor{currentstroke}{rgb}{0.121569,0.466667,0.705882}%
\pgfsetstrokecolor{currentstroke}%
\pgfsetstrokeopacity{0.826213}%
\pgfsetdash{}{0pt}%
\pgfpathmoveto{\pgfqpoint{2.953238in}{2.060284in}}%
\pgfpathcurveto{\pgfqpoint{2.961474in}{2.060284in}}{\pgfqpoint{2.969374in}{2.063556in}}{\pgfqpoint{2.975198in}{2.069380in}}%
\pgfpathcurveto{\pgfqpoint{2.981022in}{2.075204in}}{\pgfqpoint{2.984294in}{2.083104in}}{\pgfqpoint{2.984294in}{2.091340in}}%
\pgfpathcurveto{\pgfqpoint{2.984294in}{2.099576in}}{\pgfqpoint{2.981022in}{2.107476in}}{\pgfqpoint{2.975198in}{2.113300in}}%
\pgfpathcurveto{\pgfqpoint{2.969374in}{2.119124in}}{\pgfqpoint{2.961474in}{2.122397in}}{\pgfqpoint{2.953238in}{2.122397in}}%
\pgfpathcurveto{\pgfqpoint{2.945001in}{2.122397in}}{\pgfqpoint{2.937101in}{2.119124in}}{\pgfqpoint{2.931277in}{2.113300in}}%
\pgfpathcurveto{\pgfqpoint{2.925453in}{2.107476in}}{\pgfqpoint{2.922181in}{2.099576in}}{\pgfqpoint{2.922181in}{2.091340in}}%
\pgfpathcurveto{\pgfqpoint{2.922181in}{2.083104in}}{\pgfqpoint{2.925453in}{2.075204in}}{\pgfqpoint{2.931277in}{2.069380in}}%
\pgfpathcurveto{\pgfqpoint{2.937101in}{2.063556in}}{\pgfqpoint{2.945001in}{2.060284in}}{\pgfqpoint{2.953238in}{2.060284in}}%
\pgfpathclose%
\pgfusepath{stroke,fill}%
\end{pgfscope}%
\begin{pgfscope}%
\pgfpathrectangle{\pgfqpoint{0.100000in}{0.220728in}}{\pgfqpoint{3.696000in}{3.696000in}}%
\pgfusepath{clip}%
\pgfsetbuttcap%
\pgfsetroundjoin%
\definecolor{currentfill}{rgb}{0.121569,0.466667,0.705882}%
\pgfsetfillcolor{currentfill}%
\pgfsetfillopacity{0.826428}%
\pgfsetlinewidth{1.003750pt}%
\definecolor{currentstroke}{rgb}{0.121569,0.466667,0.705882}%
\pgfsetstrokecolor{currentstroke}%
\pgfsetstrokeopacity{0.826428}%
\pgfsetdash{}{0pt}%
\pgfpathmoveto{\pgfqpoint{2.952745in}{2.059017in}}%
\pgfpathcurveto{\pgfqpoint{2.960981in}{2.059017in}}{\pgfqpoint{2.968881in}{2.062289in}}{\pgfqpoint{2.974705in}{2.068113in}}%
\pgfpathcurveto{\pgfqpoint{2.980529in}{2.073937in}}{\pgfqpoint{2.983801in}{2.081837in}}{\pgfqpoint{2.983801in}{2.090074in}}%
\pgfpathcurveto{\pgfqpoint{2.983801in}{2.098310in}}{\pgfqpoint{2.980529in}{2.106210in}}{\pgfqpoint{2.974705in}{2.112034in}}%
\pgfpathcurveto{\pgfqpoint{2.968881in}{2.117858in}}{\pgfqpoint{2.960981in}{2.121130in}}{\pgfqpoint{2.952745in}{2.121130in}}%
\pgfpathcurveto{\pgfqpoint{2.944508in}{2.121130in}}{\pgfqpoint{2.936608in}{2.117858in}}{\pgfqpoint{2.930784in}{2.112034in}}%
\pgfpathcurveto{\pgfqpoint{2.924960in}{2.106210in}}{\pgfqpoint{2.921688in}{2.098310in}}{\pgfqpoint{2.921688in}{2.090074in}}%
\pgfpathcurveto{\pgfqpoint{2.921688in}{2.081837in}}{\pgfqpoint{2.924960in}{2.073937in}}{\pgfqpoint{2.930784in}{2.068113in}}%
\pgfpathcurveto{\pgfqpoint{2.936608in}{2.062289in}}{\pgfqpoint{2.944508in}{2.059017in}}{\pgfqpoint{2.952745in}{2.059017in}}%
\pgfpathclose%
\pgfusepath{stroke,fill}%
\end{pgfscope}%
\begin{pgfscope}%
\pgfpathrectangle{\pgfqpoint{0.100000in}{0.220728in}}{\pgfqpoint{3.696000in}{3.696000in}}%
\pgfusepath{clip}%
\pgfsetbuttcap%
\pgfsetroundjoin%
\definecolor{currentfill}{rgb}{0.121569,0.466667,0.705882}%
\pgfsetfillcolor{currentfill}%
\pgfsetfillopacity{0.826834}%
\pgfsetlinewidth{1.003750pt}%
\definecolor{currentstroke}{rgb}{0.121569,0.466667,0.705882}%
\pgfsetstrokecolor{currentstroke}%
\pgfsetstrokeopacity{0.826834}%
\pgfsetdash{}{0pt}%
\pgfpathmoveto{\pgfqpoint{2.951182in}{2.056306in}}%
\pgfpathcurveto{\pgfqpoint{2.959418in}{2.056306in}}{\pgfqpoint{2.967318in}{2.059578in}}{\pgfqpoint{2.973142in}{2.065402in}}%
\pgfpathcurveto{\pgfqpoint{2.978966in}{2.071226in}}{\pgfqpoint{2.982238in}{2.079126in}}{\pgfqpoint{2.982238in}{2.087362in}}%
\pgfpathcurveto{\pgfqpoint{2.982238in}{2.095599in}}{\pgfqpoint{2.978966in}{2.103499in}}{\pgfqpoint{2.973142in}{2.109323in}}%
\pgfpathcurveto{\pgfqpoint{2.967318in}{2.115146in}}{\pgfqpoint{2.959418in}{2.118419in}}{\pgfqpoint{2.951182in}{2.118419in}}%
\pgfpathcurveto{\pgfqpoint{2.942945in}{2.118419in}}{\pgfqpoint{2.935045in}{2.115146in}}{\pgfqpoint{2.929221in}{2.109323in}}%
\pgfpathcurveto{\pgfqpoint{2.923398in}{2.103499in}}{\pgfqpoint{2.920125in}{2.095599in}}{\pgfqpoint{2.920125in}{2.087362in}}%
\pgfpathcurveto{\pgfqpoint{2.920125in}{2.079126in}}{\pgfqpoint{2.923398in}{2.071226in}}{\pgfqpoint{2.929221in}{2.065402in}}%
\pgfpathcurveto{\pgfqpoint{2.935045in}{2.059578in}}{\pgfqpoint{2.942945in}{2.056306in}}{\pgfqpoint{2.951182in}{2.056306in}}%
\pgfpathclose%
\pgfusepath{stroke,fill}%
\end{pgfscope}%
\begin{pgfscope}%
\pgfpathrectangle{\pgfqpoint{0.100000in}{0.220728in}}{\pgfqpoint{3.696000in}{3.696000in}}%
\pgfusepath{clip}%
\pgfsetbuttcap%
\pgfsetroundjoin%
\definecolor{currentfill}{rgb}{0.121569,0.466667,0.705882}%
\pgfsetfillcolor{currentfill}%
\pgfsetfillopacity{0.827521}%
\pgfsetlinewidth{1.003750pt}%
\definecolor{currentstroke}{rgb}{0.121569,0.466667,0.705882}%
\pgfsetstrokecolor{currentstroke}%
\pgfsetstrokeopacity{0.827521}%
\pgfsetdash{}{0pt}%
\pgfpathmoveto{\pgfqpoint{1.475828in}{1.070467in}}%
\pgfpathcurveto{\pgfqpoint{1.484064in}{1.070467in}}{\pgfqpoint{1.491965in}{1.073739in}}{\pgfqpoint{1.497788in}{1.079563in}}%
\pgfpathcurveto{\pgfqpoint{1.503612in}{1.085387in}}{\pgfqpoint{1.506885in}{1.093287in}}{\pgfqpoint{1.506885in}{1.101523in}}%
\pgfpathcurveto{\pgfqpoint{1.506885in}{1.109759in}}{\pgfqpoint{1.503612in}{1.117659in}}{\pgfqpoint{1.497788in}{1.123483in}}%
\pgfpathcurveto{\pgfqpoint{1.491965in}{1.129307in}}{\pgfqpoint{1.484064in}{1.132580in}}{\pgfqpoint{1.475828in}{1.132580in}}%
\pgfpathcurveto{\pgfqpoint{1.467592in}{1.132580in}}{\pgfqpoint{1.459692in}{1.129307in}}{\pgfqpoint{1.453868in}{1.123483in}}%
\pgfpathcurveto{\pgfqpoint{1.448044in}{1.117659in}}{\pgfqpoint{1.444772in}{1.109759in}}{\pgfqpoint{1.444772in}{1.101523in}}%
\pgfpathcurveto{\pgfqpoint{1.444772in}{1.093287in}}{\pgfqpoint{1.448044in}{1.085387in}}{\pgfqpoint{1.453868in}{1.079563in}}%
\pgfpathcurveto{\pgfqpoint{1.459692in}{1.073739in}}{\pgfqpoint{1.467592in}{1.070467in}}{\pgfqpoint{1.475828in}{1.070467in}}%
\pgfpathclose%
\pgfusepath{stroke,fill}%
\end{pgfscope}%
\begin{pgfscope}%
\pgfpathrectangle{\pgfqpoint{0.100000in}{0.220728in}}{\pgfqpoint{3.696000in}{3.696000in}}%
\pgfusepath{clip}%
\pgfsetbuttcap%
\pgfsetroundjoin%
\definecolor{currentfill}{rgb}{0.121569,0.466667,0.705882}%
\pgfsetfillcolor{currentfill}%
\pgfsetfillopacity{0.827561}%
\pgfsetlinewidth{1.003750pt}%
\definecolor{currentstroke}{rgb}{0.121569,0.466667,0.705882}%
\pgfsetstrokecolor{currentstroke}%
\pgfsetstrokeopacity{0.827561}%
\pgfsetdash{}{0pt}%
\pgfpathmoveto{\pgfqpoint{2.949881in}{2.052063in}}%
\pgfpathcurveto{\pgfqpoint{2.958118in}{2.052063in}}{\pgfqpoint{2.966018in}{2.055336in}}{\pgfqpoint{2.971842in}{2.061160in}}%
\pgfpathcurveto{\pgfqpoint{2.977665in}{2.066984in}}{\pgfqpoint{2.980938in}{2.074884in}}{\pgfqpoint{2.980938in}{2.083120in}}%
\pgfpathcurveto{\pgfqpoint{2.980938in}{2.091356in}}{\pgfqpoint{2.977665in}{2.099256in}}{\pgfqpoint{2.971842in}{2.105080in}}%
\pgfpathcurveto{\pgfqpoint{2.966018in}{2.110904in}}{\pgfqpoint{2.958118in}{2.114176in}}{\pgfqpoint{2.949881in}{2.114176in}}%
\pgfpathcurveto{\pgfqpoint{2.941645in}{2.114176in}}{\pgfqpoint{2.933745in}{2.110904in}}{\pgfqpoint{2.927921in}{2.105080in}}%
\pgfpathcurveto{\pgfqpoint{2.922097in}{2.099256in}}{\pgfqpoint{2.918825in}{2.091356in}}{\pgfqpoint{2.918825in}{2.083120in}}%
\pgfpathcurveto{\pgfqpoint{2.918825in}{2.074884in}}{\pgfqpoint{2.922097in}{2.066984in}}{\pgfqpoint{2.927921in}{2.061160in}}%
\pgfpathcurveto{\pgfqpoint{2.933745in}{2.055336in}}{\pgfqpoint{2.941645in}{2.052063in}}{\pgfqpoint{2.949881in}{2.052063in}}%
\pgfpathclose%
\pgfusepath{stroke,fill}%
\end{pgfscope}%
\begin{pgfscope}%
\pgfpathrectangle{\pgfqpoint{0.100000in}{0.220728in}}{\pgfqpoint{3.696000in}{3.696000in}}%
\pgfusepath{clip}%
\pgfsetbuttcap%
\pgfsetroundjoin%
\definecolor{currentfill}{rgb}{0.121569,0.466667,0.705882}%
\pgfsetfillcolor{currentfill}%
\pgfsetfillopacity{0.828347}%
\pgfsetlinewidth{1.003750pt}%
\definecolor{currentstroke}{rgb}{0.121569,0.466667,0.705882}%
\pgfsetstrokecolor{currentstroke}%
\pgfsetstrokeopacity{0.828347}%
\pgfsetdash{}{0pt}%
\pgfpathmoveto{\pgfqpoint{2.947792in}{2.047190in}}%
\pgfpathcurveto{\pgfqpoint{2.956028in}{2.047190in}}{\pgfqpoint{2.963928in}{2.050463in}}{\pgfqpoint{2.969752in}{2.056287in}}%
\pgfpathcurveto{\pgfqpoint{2.975576in}{2.062111in}}{\pgfqpoint{2.978848in}{2.070011in}}{\pgfqpoint{2.978848in}{2.078247in}}%
\pgfpathcurveto{\pgfqpoint{2.978848in}{2.086483in}}{\pgfqpoint{2.975576in}{2.094383in}}{\pgfqpoint{2.969752in}{2.100207in}}%
\pgfpathcurveto{\pgfqpoint{2.963928in}{2.106031in}}{\pgfqpoint{2.956028in}{2.109303in}}{\pgfqpoint{2.947792in}{2.109303in}}%
\pgfpathcurveto{\pgfqpoint{2.939556in}{2.109303in}}{\pgfqpoint{2.931656in}{2.106031in}}{\pgfqpoint{2.925832in}{2.100207in}}%
\pgfpathcurveto{\pgfqpoint{2.920008in}{2.094383in}}{\pgfqpoint{2.916735in}{2.086483in}}{\pgfqpoint{2.916735in}{2.078247in}}%
\pgfpathcurveto{\pgfqpoint{2.916735in}{2.070011in}}{\pgfqpoint{2.920008in}{2.062111in}}{\pgfqpoint{2.925832in}{2.056287in}}%
\pgfpathcurveto{\pgfqpoint{2.931656in}{2.050463in}}{\pgfqpoint{2.939556in}{2.047190in}}{\pgfqpoint{2.947792in}{2.047190in}}%
\pgfpathclose%
\pgfusepath{stroke,fill}%
\end{pgfscope}%
\begin{pgfscope}%
\pgfpathrectangle{\pgfqpoint{0.100000in}{0.220728in}}{\pgfqpoint{3.696000in}{3.696000in}}%
\pgfusepath{clip}%
\pgfsetbuttcap%
\pgfsetroundjoin%
\definecolor{currentfill}{rgb}{0.121569,0.466667,0.705882}%
\pgfsetfillcolor{currentfill}%
\pgfsetfillopacity{0.828903}%
\pgfsetlinewidth{1.003750pt}%
\definecolor{currentstroke}{rgb}{0.121569,0.466667,0.705882}%
\pgfsetstrokecolor{currentstroke}%
\pgfsetstrokeopacity{0.828903}%
\pgfsetdash{}{0pt}%
\pgfpathmoveto{\pgfqpoint{1.484534in}{1.066717in}}%
\pgfpathcurveto{\pgfqpoint{1.492770in}{1.066717in}}{\pgfqpoint{1.500670in}{1.069989in}}{\pgfqpoint{1.506494in}{1.075813in}}%
\pgfpathcurveto{\pgfqpoint{1.512318in}{1.081637in}}{\pgfqpoint{1.515590in}{1.089537in}}{\pgfqpoint{1.515590in}{1.097773in}}%
\pgfpathcurveto{\pgfqpoint{1.515590in}{1.106009in}}{\pgfqpoint{1.512318in}{1.113910in}}{\pgfqpoint{1.506494in}{1.119733in}}%
\pgfpathcurveto{\pgfqpoint{1.500670in}{1.125557in}}{\pgfqpoint{1.492770in}{1.128830in}}{\pgfqpoint{1.484534in}{1.128830in}}%
\pgfpathcurveto{\pgfqpoint{1.476297in}{1.128830in}}{\pgfqpoint{1.468397in}{1.125557in}}{\pgfqpoint{1.462573in}{1.119733in}}%
\pgfpathcurveto{\pgfqpoint{1.456749in}{1.113910in}}{\pgfqpoint{1.453477in}{1.106009in}}{\pgfqpoint{1.453477in}{1.097773in}}%
\pgfpathcurveto{\pgfqpoint{1.453477in}{1.089537in}}{\pgfqpoint{1.456749in}{1.081637in}}{\pgfqpoint{1.462573in}{1.075813in}}%
\pgfpathcurveto{\pgfqpoint{1.468397in}{1.069989in}}{\pgfqpoint{1.476297in}{1.066717in}}{\pgfqpoint{1.484534in}{1.066717in}}%
\pgfpathclose%
\pgfusepath{stroke,fill}%
\end{pgfscope}%
\begin{pgfscope}%
\pgfpathrectangle{\pgfqpoint{0.100000in}{0.220728in}}{\pgfqpoint{3.696000in}{3.696000in}}%
\pgfusepath{clip}%
\pgfsetbuttcap%
\pgfsetroundjoin%
\definecolor{currentfill}{rgb}{0.121569,0.466667,0.705882}%
\pgfsetfillcolor{currentfill}%
\pgfsetfillopacity{0.829353}%
\pgfsetlinewidth{1.003750pt}%
\definecolor{currentstroke}{rgb}{0.121569,0.466667,0.705882}%
\pgfsetstrokecolor{currentstroke}%
\pgfsetstrokeopacity{0.829353}%
\pgfsetdash{}{0pt}%
\pgfpathmoveto{\pgfqpoint{2.944972in}{2.042685in}}%
\pgfpathcurveto{\pgfqpoint{2.953208in}{2.042685in}}{\pgfqpoint{2.961108in}{2.045957in}}{\pgfqpoint{2.966932in}{2.051781in}}%
\pgfpathcurveto{\pgfqpoint{2.972756in}{2.057605in}}{\pgfqpoint{2.976028in}{2.065505in}}{\pgfqpoint{2.976028in}{2.073742in}}%
\pgfpathcurveto{\pgfqpoint{2.976028in}{2.081978in}}{\pgfqpoint{2.972756in}{2.089878in}}{\pgfqpoint{2.966932in}{2.095702in}}%
\pgfpathcurveto{\pgfqpoint{2.961108in}{2.101526in}}{\pgfqpoint{2.953208in}{2.104798in}}{\pgfqpoint{2.944972in}{2.104798in}}%
\pgfpathcurveto{\pgfqpoint{2.936735in}{2.104798in}}{\pgfqpoint{2.928835in}{2.101526in}}{\pgfqpoint{2.923011in}{2.095702in}}%
\pgfpathcurveto{\pgfqpoint{2.917187in}{2.089878in}}{\pgfqpoint{2.913915in}{2.081978in}}{\pgfqpoint{2.913915in}{2.073742in}}%
\pgfpathcurveto{\pgfqpoint{2.913915in}{2.065505in}}{\pgfqpoint{2.917187in}{2.057605in}}{\pgfqpoint{2.923011in}{2.051781in}}%
\pgfpathcurveto{\pgfqpoint{2.928835in}{2.045957in}}{\pgfqpoint{2.936735in}{2.042685in}}{\pgfqpoint{2.944972in}{2.042685in}}%
\pgfpathclose%
\pgfusepath{stroke,fill}%
\end{pgfscope}%
\begin{pgfscope}%
\pgfpathrectangle{\pgfqpoint{0.100000in}{0.220728in}}{\pgfqpoint{3.696000in}{3.696000in}}%
\pgfusepath{clip}%
\pgfsetbuttcap%
\pgfsetroundjoin%
\definecolor{currentfill}{rgb}{0.121569,0.466667,0.705882}%
\pgfsetfillcolor{currentfill}%
\pgfsetfillopacity{0.830147}%
\pgfsetlinewidth{1.003750pt}%
\definecolor{currentstroke}{rgb}{0.121569,0.466667,0.705882}%
\pgfsetstrokecolor{currentstroke}%
\pgfsetstrokeopacity{0.830147}%
\pgfsetdash{}{0pt}%
\pgfpathmoveto{\pgfqpoint{1.491044in}{1.063788in}}%
\pgfpathcurveto{\pgfqpoint{1.499280in}{1.063788in}}{\pgfqpoint{1.507180in}{1.067061in}}{\pgfqpoint{1.513004in}{1.072885in}}%
\pgfpathcurveto{\pgfqpoint{1.518828in}{1.078709in}}{\pgfqpoint{1.522100in}{1.086609in}}{\pgfqpoint{1.522100in}{1.094845in}}%
\pgfpathcurveto{\pgfqpoint{1.522100in}{1.103081in}}{\pgfqpoint{1.518828in}{1.110981in}}{\pgfqpoint{1.513004in}{1.116805in}}%
\pgfpathcurveto{\pgfqpoint{1.507180in}{1.122629in}}{\pgfqpoint{1.499280in}{1.125901in}}{\pgfqpoint{1.491044in}{1.125901in}}%
\pgfpathcurveto{\pgfqpoint{1.482808in}{1.125901in}}{\pgfqpoint{1.474908in}{1.122629in}}{\pgfqpoint{1.469084in}{1.116805in}}%
\pgfpathcurveto{\pgfqpoint{1.463260in}{1.110981in}}{\pgfqpoint{1.459987in}{1.103081in}}{\pgfqpoint{1.459987in}{1.094845in}}%
\pgfpathcurveto{\pgfqpoint{1.459987in}{1.086609in}}{\pgfqpoint{1.463260in}{1.078709in}}{\pgfqpoint{1.469084in}{1.072885in}}%
\pgfpathcurveto{\pgfqpoint{1.474908in}{1.067061in}}{\pgfqpoint{1.482808in}{1.063788in}}{\pgfqpoint{1.491044in}{1.063788in}}%
\pgfpathclose%
\pgfusepath{stroke,fill}%
\end{pgfscope}%
\begin{pgfscope}%
\pgfpathrectangle{\pgfqpoint{0.100000in}{0.220728in}}{\pgfqpoint{3.696000in}{3.696000in}}%
\pgfusepath{clip}%
\pgfsetbuttcap%
\pgfsetroundjoin%
\definecolor{currentfill}{rgb}{0.121569,0.466667,0.705882}%
\pgfsetfillcolor{currentfill}%
\pgfsetfillopacity{0.830652}%
\pgfsetlinewidth{1.003750pt}%
\definecolor{currentstroke}{rgb}{0.121569,0.466667,0.705882}%
\pgfsetstrokecolor{currentstroke}%
\pgfsetstrokeopacity{0.830652}%
\pgfsetdash{}{0pt}%
\pgfpathmoveto{\pgfqpoint{2.942548in}{2.035220in}}%
\pgfpathcurveto{\pgfqpoint{2.950785in}{2.035220in}}{\pgfqpoint{2.958685in}{2.038492in}}{\pgfqpoint{2.964509in}{2.044316in}}%
\pgfpathcurveto{\pgfqpoint{2.970333in}{2.050140in}}{\pgfqpoint{2.973605in}{2.058040in}}{\pgfqpoint{2.973605in}{2.066276in}}%
\pgfpathcurveto{\pgfqpoint{2.973605in}{2.074512in}}{\pgfqpoint{2.970333in}{2.082413in}}{\pgfqpoint{2.964509in}{2.088236in}}%
\pgfpathcurveto{\pgfqpoint{2.958685in}{2.094060in}}{\pgfqpoint{2.950785in}{2.097333in}}{\pgfqpoint{2.942548in}{2.097333in}}%
\pgfpathcurveto{\pgfqpoint{2.934312in}{2.097333in}}{\pgfqpoint{2.926412in}{2.094060in}}{\pgfqpoint{2.920588in}{2.088236in}}%
\pgfpathcurveto{\pgfqpoint{2.914764in}{2.082413in}}{\pgfqpoint{2.911492in}{2.074512in}}{\pgfqpoint{2.911492in}{2.066276in}}%
\pgfpathcurveto{\pgfqpoint{2.911492in}{2.058040in}}{\pgfqpoint{2.914764in}{2.050140in}}{\pgfqpoint{2.920588in}{2.044316in}}%
\pgfpathcurveto{\pgfqpoint{2.926412in}{2.038492in}}{\pgfqpoint{2.934312in}{2.035220in}}{\pgfqpoint{2.942548in}{2.035220in}}%
\pgfpathclose%
\pgfusepath{stroke,fill}%
\end{pgfscope}%
\begin{pgfscope}%
\pgfpathrectangle{\pgfqpoint{0.100000in}{0.220728in}}{\pgfqpoint{3.696000in}{3.696000in}}%
\pgfusepath{clip}%
\pgfsetbuttcap%
\pgfsetroundjoin%
\definecolor{currentfill}{rgb}{0.121569,0.466667,0.705882}%
\pgfsetfillcolor{currentfill}%
\pgfsetfillopacity{0.831954}%
\pgfsetlinewidth{1.003750pt}%
\definecolor{currentstroke}{rgb}{0.121569,0.466667,0.705882}%
\pgfsetstrokecolor{currentstroke}%
\pgfsetstrokeopacity{0.831954}%
\pgfsetdash{}{0pt}%
\pgfpathmoveto{\pgfqpoint{2.938543in}{2.027210in}}%
\pgfpathcurveto{\pgfqpoint{2.946779in}{2.027210in}}{\pgfqpoint{2.954679in}{2.030482in}}{\pgfqpoint{2.960503in}{2.036306in}}%
\pgfpathcurveto{\pgfqpoint{2.966327in}{2.042130in}}{\pgfqpoint{2.969600in}{2.050030in}}{\pgfqpoint{2.969600in}{2.058266in}}%
\pgfpathcurveto{\pgfqpoint{2.969600in}{2.066502in}}{\pgfqpoint{2.966327in}{2.074403in}}{\pgfqpoint{2.960503in}{2.080226in}}%
\pgfpathcurveto{\pgfqpoint{2.954679in}{2.086050in}}{\pgfqpoint{2.946779in}{2.089323in}}{\pgfqpoint{2.938543in}{2.089323in}}%
\pgfpathcurveto{\pgfqpoint{2.930307in}{2.089323in}}{\pgfqpoint{2.922407in}{2.086050in}}{\pgfqpoint{2.916583in}{2.080226in}}%
\pgfpathcurveto{\pgfqpoint{2.910759in}{2.074403in}}{\pgfqpoint{2.907487in}{2.066502in}}{\pgfqpoint{2.907487in}{2.058266in}}%
\pgfpathcurveto{\pgfqpoint{2.907487in}{2.050030in}}{\pgfqpoint{2.910759in}{2.042130in}}{\pgfqpoint{2.916583in}{2.036306in}}%
\pgfpathcurveto{\pgfqpoint{2.922407in}{2.030482in}}{\pgfqpoint{2.930307in}{2.027210in}}{\pgfqpoint{2.938543in}{2.027210in}}%
\pgfpathclose%
\pgfusepath{stroke,fill}%
\end{pgfscope}%
\begin{pgfscope}%
\pgfpathrectangle{\pgfqpoint{0.100000in}{0.220728in}}{\pgfqpoint{3.696000in}{3.696000in}}%
\pgfusepath{clip}%
\pgfsetbuttcap%
\pgfsetroundjoin%
\definecolor{currentfill}{rgb}{0.121569,0.466667,0.705882}%
\pgfsetfillcolor{currentfill}%
\pgfsetfillopacity{0.832752}%
\pgfsetlinewidth{1.003750pt}%
\definecolor{currentstroke}{rgb}{0.121569,0.466667,0.705882}%
\pgfsetstrokecolor{currentstroke}%
\pgfsetstrokeopacity{0.832752}%
\pgfsetdash{}{0pt}%
\pgfpathmoveto{\pgfqpoint{2.936195in}{2.023339in}}%
\pgfpathcurveto{\pgfqpoint{2.944432in}{2.023339in}}{\pgfqpoint{2.952332in}{2.026611in}}{\pgfqpoint{2.958156in}{2.032435in}}%
\pgfpathcurveto{\pgfqpoint{2.963980in}{2.038259in}}{\pgfqpoint{2.967252in}{2.046159in}}{\pgfqpoint{2.967252in}{2.054395in}}%
\pgfpathcurveto{\pgfqpoint{2.967252in}{2.062631in}}{\pgfqpoint{2.963980in}{2.070531in}}{\pgfqpoint{2.958156in}{2.076355in}}%
\pgfpathcurveto{\pgfqpoint{2.952332in}{2.082179in}}{\pgfqpoint{2.944432in}{2.085452in}}{\pgfqpoint{2.936195in}{2.085452in}}%
\pgfpathcurveto{\pgfqpoint{2.927959in}{2.085452in}}{\pgfqpoint{2.920059in}{2.082179in}}{\pgfqpoint{2.914235in}{2.076355in}}%
\pgfpathcurveto{\pgfqpoint{2.908411in}{2.070531in}}{\pgfqpoint{2.905139in}{2.062631in}}{\pgfqpoint{2.905139in}{2.054395in}}%
\pgfpathcurveto{\pgfqpoint{2.905139in}{2.046159in}}{\pgfqpoint{2.908411in}{2.038259in}}{\pgfqpoint{2.914235in}{2.032435in}}%
\pgfpathcurveto{\pgfqpoint{2.920059in}{2.026611in}}{\pgfqpoint{2.927959in}{2.023339in}}{\pgfqpoint{2.936195in}{2.023339in}}%
\pgfpathclose%
\pgfusepath{stroke,fill}%
\end{pgfscope}%
\begin{pgfscope}%
\pgfpathrectangle{\pgfqpoint{0.100000in}{0.220728in}}{\pgfqpoint{3.696000in}{3.696000in}}%
\pgfusepath{clip}%
\pgfsetbuttcap%
\pgfsetroundjoin%
\definecolor{currentfill}{rgb}{0.121569,0.466667,0.705882}%
\pgfsetfillcolor{currentfill}%
\pgfsetfillopacity{0.832777}%
\pgfsetlinewidth{1.003750pt}%
\definecolor{currentstroke}{rgb}{0.121569,0.466667,0.705882}%
\pgfsetstrokecolor{currentstroke}%
\pgfsetstrokeopacity{0.832777}%
\pgfsetdash{}{0pt}%
\pgfpathmoveto{\pgfqpoint{1.502940in}{1.060156in}}%
\pgfpathcurveto{\pgfqpoint{1.511177in}{1.060156in}}{\pgfqpoint{1.519077in}{1.063428in}}{\pgfqpoint{1.524901in}{1.069252in}}%
\pgfpathcurveto{\pgfqpoint{1.530725in}{1.075076in}}{\pgfqpoint{1.533997in}{1.082976in}}{\pgfqpoint{1.533997in}{1.091212in}}%
\pgfpathcurveto{\pgfqpoint{1.533997in}{1.099449in}}{\pgfqpoint{1.530725in}{1.107349in}}{\pgfqpoint{1.524901in}{1.113173in}}%
\pgfpathcurveto{\pgfqpoint{1.519077in}{1.118996in}}{\pgfqpoint{1.511177in}{1.122269in}}{\pgfqpoint{1.502940in}{1.122269in}}%
\pgfpathcurveto{\pgfqpoint{1.494704in}{1.122269in}}{\pgfqpoint{1.486804in}{1.118996in}}{\pgfqpoint{1.480980in}{1.113173in}}%
\pgfpathcurveto{\pgfqpoint{1.475156in}{1.107349in}}{\pgfqpoint{1.471884in}{1.099449in}}{\pgfqpoint{1.471884in}{1.091212in}}%
\pgfpathcurveto{\pgfqpoint{1.471884in}{1.082976in}}{\pgfqpoint{1.475156in}{1.075076in}}{\pgfqpoint{1.480980in}{1.069252in}}%
\pgfpathcurveto{\pgfqpoint{1.486804in}{1.063428in}}{\pgfqpoint{1.494704in}{1.060156in}}{\pgfqpoint{1.502940in}{1.060156in}}%
\pgfpathclose%
\pgfusepath{stroke,fill}%
\end{pgfscope}%
\begin{pgfscope}%
\pgfpathrectangle{\pgfqpoint{0.100000in}{0.220728in}}{\pgfqpoint{3.696000in}{3.696000in}}%
\pgfusepath{clip}%
\pgfsetbuttcap%
\pgfsetroundjoin%
\definecolor{currentfill}{rgb}{0.121569,0.466667,0.705882}%
\pgfsetfillcolor{currentfill}%
\pgfsetfillopacity{0.833657}%
\pgfsetlinewidth{1.003750pt}%
\definecolor{currentstroke}{rgb}{0.121569,0.466667,0.705882}%
\pgfsetstrokecolor{currentstroke}%
\pgfsetstrokeopacity{0.833657}%
\pgfsetdash{}{0pt}%
\pgfpathmoveto{\pgfqpoint{2.934557in}{2.017734in}}%
\pgfpathcurveto{\pgfqpoint{2.942793in}{2.017734in}}{\pgfqpoint{2.950693in}{2.021006in}}{\pgfqpoint{2.956517in}{2.026830in}}%
\pgfpathcurveto{\pgfqpoint{2.962341in}{2.032654in}}{\pgfqpoint{2.965613in}{2.040554in}}{\pgfqpoint{2.965613in}{2.048790in}}%
\pgfpathcurveto{\pgfqpoint{2.965613in}{2.057026in}}{\pgfqpoint{2.962341in}{2.064926in}}{\pgfqpoint{2.956517in}{2.070750in}}%
\pgfpathcurveto{\pgfqpoint{2.950693in}{2.076574in}}{\pgfqpoint{2.942793in}{2.079847in}}{\pgfqpoint{2.934557in}{2.079847in}}%
\pgfpathcurveto{\pgfqpoint{2.926320in}{2.079847in}}{\pgfqpoint{2.918420in}{2.076574in}}{\pgfqpoint{2.912596in}{2.070750in}}%
\pgfpathcurveto{\pgfqpoint{2.906773in}{2.064926in}}{\pgfqpoint{2.903500in}{2.057026in}}{\pgfqpoint{2.903500in}{2.048790in}}%
\pgfpathcurveto{\pgfqpoint{2.903500in}{2.040554in}}{\pgfqpoint{2.906773in}{2.032654in}}{\pgfqpoint{2.912596in}{2.026830in}}%
\pgfpathcurveto{\pgfqpoint{2.918420in}{2.021006in}}{\pgfqpoint{2.926320in}{2.017734in}}{\pgfqpoint{2.934557in}{2.017734in}}%
\pgfpathclose%
\pgfusepath{stroke,fill}%
\end{pgfscope}%
\begin{pgfscope}%
\pgfpathrectangle{\pgfqpoint{0.100000in}{0.220728in}}{\pgfqpoint{3.696000in}{3.696000in}}%
\pgfusepath{clip}%
\pgfsetbuttcap%
\pgfsetroundjoin%
\definecolor{currentfill}{rgb}{0.121569,0.466667,0.705882}%
\pgfsetfillcolor{currentfill}%
\pgfsetfillopacity{0.834722}%
\pgfsetlinewidth{1.003750pt}%
\definecolor{currentstroke}{rgb}{0.121569,0.466667,0.705882}%
\pgfsetstrokecolor{currentstroke}%
\pgfsetstrokeopacity{0.834722}%
\pgfsetdash{}{0pt}%
\pgfpathmoveto{\pgfqpoint{2.931122in}{2.011134in}}%
\pgfpathcurveto{\pgfqpoint{2.939359in}{2.011134in}}{\pgfqpoint{2.947259in}{2.014406in}}{\pgfqpoint{2.953083in}{2.020230in}}%
\pgfpathcurveto{\pgfqpoint{2.958907in}{2.026054in}}{\pgfqpoint{2.962179in}{2.033954in}}{\pgfqpoint{2.962179in}{2.042191in}}%
\pgfpathcurveto{\pgfqpoint{2.962179in}{2.050427in}}{\pgfqpoint{2.958907in}{2.058327in}}{\pgfqpoint{2.953083in}{2.064151in}}%
\pgfpathcurveto{\pgfqpoint{2.947259in}{2.069975in}}{\pgfqpoint{2.939359in}{2.073247in}}{\pgfqpoint{2.931122in}{2.073247in}}%
\pgfpathcurveto{\pgfqpoint{2.922886in}{2.073247in}}{\pgfqpoint{2.914986in}{2.069975in}}{\pgfqpoint{2.909162in}{2.064151in}}%
\pgfpathcurveto{\pgfqpoint{2.903338in}{2.058327in}}{\pgfqpoint{2.900066in}{2.050427in}}{\pgfqpoint{2.900066in}{2.042191in}}%
\pgfpathcurveto{\pgfqpoint{2.900066in}{2.033954in}}{\pgfqpoint{2.903338in}{2.026054in}}{\pgfqpoint{2.909162in}{2.020230in}}%
\pgfpathcurveto{\pgfqpoint{2.914986in}{2.014406in}}{\pgfqpoint{2.922886in}{2.011134in}}{\pgfqpoint{2.931122in}{2.011134in}}%
\pgfpathclose%
\pgfusepath{stroke,fill}%
\end{pgfscope}%
\begin{pgfscope}%
\pgfpathrectangle{\pgfqpoint{0.100000in}{0.220728in}}{\pgfqpoint{3.696000in}{3.696000in}}%
\pgfusepath{clip}%
\pgfsetbuttcap%
\pgfsetroundjoin%
\definecolor{currentfill}{rgb}{0.121569,0.466667,0.705882}%
\pgfsetfillcolor{currentfill}%
\pgfsetfillopacity{0.835391}%
\pgfsetlinewidth{1.003750pt}%
\definecolor{currentstroke}{rgb}{0.121569,0.466667,0.705882}%
\pgfsetstrokecolor{currentstroke}%
\pgfsetstrokeopacity{0.835391}%
\pgfsetdash{}{0pt}%
\pgfpathmoveto{\pgfqpoint{2.929569in}{2.007457in}}%
\pgfpathcurveto{\pgfqpoint{2.937806in}{2.007457in}}{\pgfqpoint{2.945706in}{2.010729in}}{\pgfqpoint{2.951530in}{2.016553in}}%
\pgfpathcurveto{\pgfqpoint{2.957353in}{2.022377in}}{\pgfqpoint{2.960626in}{2.030277in}}{\pgfqpoint{2.960626in}{2.038513in}}%
\pgfpathcurveto{\pgfqpoint{2.960626in}{2.046750in}}{\pgfqpoint{2.957353in}{2.054650in}}{\pgfqpoint{2.951530in}{2.060473in}}%
\pgfpathcurveto{\pgfqpoint{2.945706in}{2.066297in}}{\pgfqpoint{2.937806in}{2.069570in}}{\pgfqpoint{2.929569in}{2.069570in}}%
\pgfpathcurveto{\pgfqpoint{2.921333in}{2.069570in}}{\pgfqpoint{2.913433in}{2.066297in}}{\pgfqpoint{2.907609in}{2.060473in}}%
\pgfpathcurveto{\pgfqpoint{2.901785in}{2.054650in}}{\pgfqpoint{2.898513in}{2.046750in}}{\pgfqpoint{2.898513in}{2.038513in}}%
\pgfpathcurveto{\pgfqpoint{2.898513in}{2.030277in}}{\pgfqpoint{2.901785in}{2.022377in}}{\pgfqpoint{2.907609in}{2.016553in}}%
\pgfpathcurveto{\pgfqpoint{2.913433in}{2.010729in}}{\pgfqpoint{2.921333in}{2.007457in}}{\pgfqpoint{2.929569in}{2.007457in}}%
\pgfpathclose%
\pgfusepath{stroke,fill}%
\end{pgfscope}%
\begin{pgfscope}%
\pgfpathrectangle{\pgfqpoint{0.100000in}{0.220728in}}{\pgfqpoint{3.696000in}{3.696000in}}%
\pgfusepath{clip}%
\pgfsetbuttcap%
\pgfsetroundjoin%
\definecolor{currentfill}{rgb}{0.121569,0.466667,0.705882}%
\pgfsetfillcolor{currentfill}%
\pgfsetfillopacity{0.835772}%
\pgfsetlinewidth{1.003750pt}%
\definecolor{currentstroke}{rgb}{0.121569,0.466667,0.705882}%
\pgfsetstrokecolor{currentstroke}%
\pgfsetstrokeopacity{0.835772}%
\pgfsetdash{}{0pt}%
\pgfpathmoveto{\pgfqpoint{2.928861in}{2.005361in}}%
\pgfpathcurveto{\pgfqpoint{2.937097in}{2.005361in}}{\pgfqpoint{2.944997in}{2.008634in}}{\pgfqpoint{2.950821in}{2.014458in}}%
\pgfpathcurveto{\pgfqpoint{2.956645in}{2.020282in}}{\pgfqpoint{2.959917in}{2.028182in}}{\pgfqpoint{2.959917in}{2.036418in}}%
\pgfpathcurveto{\pgfqpoint{2.959917in}{2.044654in}}{\pgfqpoint{2.956645in}{2.052554in}}{\pgfqpoint{2.950821in}{2.058378in}}%
\pgfpathcurveto{\pgfqpoint{2.944997in}{2.064202in}}{\pgfqpoint{2.937097in}{2.067474in}}{\pgfqpoint{2.928861in}{2.067474in}}%
\pgfpathcurveto{\pgfqpoint{2.920625in}{2.067474in}}{\pgfqpoint{2.912724in}{2.064202in}}{\pgfqpoint{2.906901in}{2.058378in}}%
\pgfpathcurveto{\pgfqpoint{2.901077in}{2.052554in}}{\pgfqpoint{2.897804in}{2.044654in}}{\pgfqpoint{2.897804in}{2.036418in}}%
\pgfpathcurveto{\pgfqpoint{2.897804in}{2.028182in}}{\pgfqpoint{2.901077in}{2.020282in}}{\pgfqpoint{2.906901in}{2.014458in}}%
\pgfpathcurveto{\pgfqpoint{2.912724in}{2.008634in}}{\pgfqpoint{2.920625in}{2.005361in}}{\pgfqpoint{2.928861in}{2.005361in}}%
\pgfpathclose%
\pgfusepath{stroke,fill}%
\end{pgfscope}%
\begin{pgfscope}%
\pgfpathrectangle{\pgfqpoint{0.100000in}{0.220728in}}{\pgfqpoint{3.696000in}{3.696000in}}%
\pgfusepath{clip}%
\pgfsetbuttcap%
\pgfsetroundjoin%
\definecolor{currentfill}{rgb}{0.121569,0.466667,0.705882}%
\pgfsetfillcolor{currentfill}%
\pgfsetfillopacity{0.835949}%
\pgfsetlinewidth{1.003750pt}%
\definecolor{currentstroke}{rgb}{0.121569,0.466667,0.705882}%
\pgfsetstrokecolor{currentstroke}%
\pgfsetstrokeopacity{0.835949}%
\pgfsetdash{}{0pt}%
\pgfpathmoveto{\pgfqpoint{2.928246in}{2.004324in}}%
\pgfpathcurveto{\pgfqpoint{2.936482in}{2.004324in}}{\pgfqpoint{2.944382in}{2.007597in}}{\pgfqpoint{2.950206in}{2.013421in}}%
\pgfpathcurveto{\pgfqpoint{2.956030in}{2.019245in}}{\pgfqpoint{2.959302in}{2.027145in}}{\pgfqpoint{2.959302in}{2.035381in}}%
\pgfpathcurveto{\pgfqpoint{2.959302in}{2.043617in}}{\pgfqpoint{2.956030in}{2.051517in}}{\pgfqpoint{2.950206in}{2.057341in}}%
\pgfpathcurveto{\pgfqpoint{2.944382in}{2.063165in}}{\pgfqpoint{2.936482in}{2.066437in}}{\pgfqpoint{2.928246in}{2.066437in}}%
\pgfpathcurveto{\pgfqpoint{2.920009in}{2.066437in}}{\pgfqpoint{2.912109in}{2.063165in}}{\pgfqpoint{2.906285in}{2.057341in}}%
\pgfpathcurveto{\pgfqpoint{2.900462in}{2.051517in}}{\pgfqpoint{2.897189in}{2.043617in}}{\pgfqpoint{2.897189in}{2.035381in}}%
\pgfpathcurveto{\pgfqpoint{2.897189in}{2.027145in}}{\pgfqpoint{2.900462in}{2.019245in}}{\pgfqpoint{2.906285in}{2.013421in}}%
\pgfpathcurveto{\pgfqpoint{2.912109in}{2.007597in}}{\pgfqpoint{2.920009in}{2.004324in}}{\pgfqpoint{2.928246in}{2.004324in}}%
\pgfpathclose%
\pgfusepath{stroke,fill}%
\end{pgfscope}%
\begin{pgfscope}%
\pgfpathrectangle{\pgfqpoint{0.100000in}{0.220728in}}{\pgfqpoint{3.696000in}{3.696000in}}%
\pgfusepath{clip}%
\pgfsetbuttcap%
\pgfsetroundjoin%
\definecolor{currentfill}{rgb}{0.121569,0.466667,0.705882}%
\pgfsetfillcolor{currentfill}%
\pgfsetfillopacity{0.836425}%
\pgfsetlinewidth{1.003750pt}%
\definecolor{currentstroke}{rgb}{0.121569,0.466667,0.705882}%
\pgfsetstrokecolor{currentstroke}%
\pgfsetstrokeopacity{0.836425}%
\pgfsetdash{}{0pt}%
\pgfpathmoveto{\pgfqpoint{2.927342in}{2.001513in}}%
\pgfpathcurveto{\pgfqpoint{2.935579in}{2.001513in}}{\pgfqpoint{2.943479in}{2.004785in}}{\pgfqpoint{2.949303in}{2.010609in}}%
\pgfpathcurveto{\pgfqpoint{2.955127in}{2.016433in}}{\pgfqpoint{2.958399in}{2.024333in}}{\pgfqpoint{2.958399in}{2.032570in}}%
\pgfpathcurveto{\pgfqpoint{2.958399in}{2.040806in}}{\pgfqpoint{2.955127in}{2.048706in}}{\pgfqpoint{2.949303in}{2.054530in}}%
\pgfpathcurveto{\pgfqpoint{2.943479in}{2.060354in}}{\pgfqpoint{2.935579in}{2.063626in}}{\pgfqpoint{2.927342in}{2.063626in}}%
\pgfpathcurveto{\pgfqpoint{2.919106in}{2.063626in}}{\pgfqpoint{2.911206in}{2.060354in}}{\pgfqpoint{2.905382in}{2.054530in}}%
\pgfpathcurveto{\pgfqpoint{2.899558in}{2.048706in}}{\pgfqpoint{2.896286in}{2.040806in}}{\pgfqpoint{2.896286in}{2.032570in}}%
\pgfpathcurveto{\pgfqpoint{2.896286in}{2.024333in}}{\pgfqpoint{2.899558in}{2.016433in}}{\pgfqpoint{2.905382in}{2.010609in}}%
\pgfpathcurveto{\pgfqpoint{2.911206in}{2.004785in}}{\pgfqpoint{2.919106in}{2.001513in}}{\pgfqpoint{2.927342in}{2.001513in}}%
\pgfpathclose%
\pgfusepath{stroke,fill}%
\end{pgfscope}%
\begin{pgfscope}%
\pgfpathrectangle{\pgfqpoint{0.100000in}{0.220728in}}{\pgfqpoint{3.696000in}{3.696000in}}%
\pgfusepath{clip}%
\pgfsetbuttcap%
\pgfsetroundjoin%
\definecolor{currentfill}{rgb}{0.121569,0.466667,0.705882}%
\pgfsetfillcolor{currentfill}%
\pgfsetfillopacity{0.836667}%
\pgfsetlinewidth{1.003750pt}%
\definecolor{currentstroke}{rgb}{0.121569,0.466667,0.705882}%
\pgfsetstrokecolor{currentstroke}%
\pgfsetstrokeopacity{0.836667}%
\pgfsetdash{}{0pt}%
\pgfpathmoveto{\pgfqpoint{2.926701in}{2.000018in}}%
\pgfpathcurveto{\pgfqpoint{2.934937in}{2.000018in}}{\pgfqpoint{2.942837in}{2.003291in}}{\pgfqpoint{2.948661in}{2.009115in}}%
\pgfpathcurveto{\pgfqpoint{2.954485in}{2.014938in}}{\pgfqpoint{2.957757in}{2.022839in}}{\pgfqpoint{2.957757in}{2.031075in}}%
\pgfpathcurveto{\pgfqpoint{2.957757in}{2.039311in}}{\pgfqpoint{2.954485in}{2.047211in}}{\pgfqpoint{2.948661in}{2.053035in}}%
\pgfpathcurveto{\pgfqpoint{2.942837in}{2.058859in}}{\pgfqpoint{2.934937in}{2.062131in}}{\pgfqpoint{2.926701in}{2.062131in}}%
\pgfpathcurveto{\pgfqpoint{2.918464in}{2.062131in}}{\pgfqpoint{2.910564in}{2.058859in}}{\pgfqpoint{2.904740in}{2.053035in}}%
\pgfpathcurveto{\pgfqpoint{2.898916in}{2.047211in}}{\pgfqpoint{2.895644in}{2.039311in}}{\pgfqpoint{2.895644in}{2.031075in}}%
\pgfpathcurveto{\pgfqpoint{2.895644in}{2.022839in}}{\pgfqpoint{2.898916in}{2.014938in}}{\pgfqpoint{2.904740in}{2.009115in}}%
\pgfpathcurveto{\pgfqpoint{2.910564in}{2.003291in}}{\pgfqpoint{2.918464in}{2.000018in}}{\pgfqpoint{2.926701in}{2.000018in}}%
\pgfpathclose%
\pgfusepath{stroke,fill}%
\end{pgfscope}%
\begin{pgfscope}%
\pgfpathrectangle{\pgfqpoint{0.100000in}{0.220728in}}{\pgfqpoint{3.696000in}{3.696000in}}%
\pgfusepath{clip}%
\pgfsetbuttcap%
\pgfsetroundjoin%
\definecolor{currentfill}{rgb}{0.121569,0.466667,0.705882}%
\pgfsetfillcolor{currentfill}%
\pgfsetfillopacity{0.837131}%
\pgfsetlinewidth{1.003750pt}%
\definecolor{currentstroke}{rgb}{0.121569,0.466667,0.705882}%
\pgfsetstrokecolor{currentstroke}%
\pgfsetstrokeopacity{0.837131}%
\pgfsetdash{}{0pt}%
\pgfpathmoveto{\pgfqpoint{2.925324in}{1.997742in}}%
\pgfpathcurveto{\pgfqpoint{2.933560in}{1.997742in}}{\pgfqpoint{2.941460in}{2.001014in}}{\pgfqpoint{2.947284in}{2.006838in}}%
\pgfpathcurveto{\pgfqpoint{2.953108in}{2.012662in}}{\pgfqpoint{2.956381in}{2.020562in}}{\pgfqpoint{2.956381in}{2.028798in}}%
\pgfpathcurveto{\pgfqpoint{2.956381in}{2.037035in}}{\pgfqpoint{2.953108in}{2.044935in}}{\pgfqpoint{2.947284in}{2.050758in}}%
\pgfpathcurveto{\pgfqpoint{2.941460in}{2.056582in}}{\pgfqpoint{2.933560in}{2.059855in}}{\pgfqpoint{2.925324in}{2.059855in}}%
\pgfpathcurveto{\pgfqpoint{2.917088in}{2.059855in}}{\pgfqpoint{2.909188in}{2.056582in}}{\pgfqpoint{2.903364in}{2.050758in}}%
\pgfpathcurveto{\pgfqpoint{2.897540in}{2.044935in}}{\pgfqpoint{2.894268in}{2.037035in}}{\pgfqpoint{2.894268in}{2.028798in}}%
\pgfpathcurveto{\pgfqpoint{2.894268in}{2.020562in}}{\pgfqpoint{2.897540in}{2.012662in}}{\pgfqpoint{2.903364in}{2.006838in}}%
\pgfpathcurveto{\pgfqpoint{2.909188in}{2.001014in}}{\pgfqpoint{2.917088in}{1.997742in}}{\pgfqpoint{2.925324in}{1.997742in}}%
\pgfpathclose%
\pgfusepath{stroke,fill}%
\end{pgfscope}%
\begin{pgfscope}%
\pgfpathrectangle{\pgfqpoint{0.100000in}{0.220728in}}{\pgfqpoint{3.696000in}{3.696000in}}%
\pgfusepath{clip}%
\pgfsetbuttcap%
\pgfsetroundjoin%
\definecolor{currentfill}{rgb}{0.121569,0.466667,0.705882}%
\pgfsetfillcolor{currentfill}%
\pgfsetfillopacity{0.837973}%
\pgfsetlinewidth{1.003750pt}%
\definecolor{currentstroke}{rgb}{0.121569,0.466667,0.705882}%
\pgfsetstrokecolor{currentstroke}%
\pgfsetstrokeopacity{0.837973}%
\pgfsetdash{}{0pt}%
\pgfpathmoveto{\pgfqpoint{2.923829in}{1.992407in}}%
\pgfpathcurveto{\pgfqpoint{2.932065in}{1.992407in}}{\pgfqpoint{2.939965in}{1.995679in}}{\pgfqpoint{2.945789in}{2.001503in}}%
\pgfpathcurveto{\pgfqpoint{2.951613in}{2.007327in}}{\pgfqpoint{2.954885in}{2.015227in}}{\pgfqpoint{2.954885in}{2.023464in}}%
\pgfpathcurveto{\pgfqpoint{2.954885in}{2.031700in}}{\pgfqpoint{2.951613in}{2.039600in}}{\pgfqpoint{2.945789in}{2.045424in}}%
\pgfpathcurveto{\pgfqpoint{2.939965in}{2.051248in}}{\pgfqpoint{2.932065in}{2.054520in}}{\pgfqpoint{2.923829in}{2.054520in}}%
\pgfpathcurveto{\pgfqpoint{2.915592in}{2.054520in}}{\pgfqpoint{2.907692in}{2.051248in}}{\pgfqpoint{2.901868in}{2.045424in}}%
\pgfpathcurveto{\pgfqpoint{2.896044in}{2.039600in}}{\pgfqpoint{2.892772in}{2.031700in}}{\pgfqpoint{2.892772in}{2.023464in}}%
\pgfpathcurveto{\pgfqpoint{2.892772in}{2.015227in}}{\pgfqpoint{2.896044in}{2.007327in}}{\pgfqpoint{2.901868in}{2.001503in}}%
\pgfpathcurveto{\pgfqpoint{2.907692in}{1.995679in}}{\pgfqpoint{2.915592in}{1.992407in}}{\pgfqpoint{2.923829in}{1.992407in}}%
\pgfpathclose%
\pgfusepath{stroke,fill}%
\end{pgfscope}%
\begin{pgfscope}%
\pgfpathrectangle{\pgfqpoint{0.100000in}{0.220728in}}{\pgfqpoint{3.696000in}{3.696000in}}%
\pgfusepath{clip}%
\pgfsetbuttcap%
\pgfsetroundjoin%
\definecolor{currentfill}{rgb}{0.121569,0.466667,0.705882}%
\pgfsetfillcolor{currentfill}%
\pgfsetfillopacity{0.838130}%
\pgfsetlinewidth{1.003750pt}%
\definecolor{currentstroke}{rgb}{0.121569,0.466667,0.705882}%
\pgfsetstrokecolor{currentstroke}%
\pgfsetstrokeopacity{0.838130}%
\pgfsetdash{}{0pt}%
\pgfpathmoveto{\pgfqpoint{1.523940in}{1.053709in}}%
\pgfpathcurveto{\pgfqpoint{1.532176in}{1.053709in}}{\pgfqpoint{1.540076in}{1.056981in}}{\pgfqpoint{1.545900in}{1.062805in}}%
\pgfpathcurveto{\pgfqpoint{1.551724in}{1.068629in}}{\pgfqpoint{1.554996in}{1.076529in}}{\pgfqpoint{1.554996in}{1.084765in}}%
\pgfpathcurveto{\pgfqpoint{1.554996in}{1.093002in}}{\pgfqpoint{1.551724in}{1.100902in}}{\pgfqpoint{1.545900in}{1.106726in}}%
\pgfpathcurveto{\pgfqpoint{1.540076in}{1.112549in}}{\pgfqpoint{1.532176in}{1.115822in}}{\pgfqpoint{1.523940in}{1.115822in}}%
\pgfpathcurveto{\pgfqpoint{1.515703in}{1.115822in}}{\pgfqpoint{1.507803in}{1.112549in}}{\pgfqpoint{1.501979in}{1.106726in}}%
\pgfpathcurveto{\pgfqpoint{1.496155in}{1.100902in}}{\pgfqpoint{1.492883in}{1.093002in}}{\pgfqpoint{1.492883in}{1.084765in}}%
\pgfpathcurveto{\pgfqpoint{1.492883in}{1.076529in}}{\pgfqpoint{1.496155in}{1.068629in}}{\pgfqpoint{1.501979in}{1.062805in}}%
\pgfpathcurveto{\pgfqpoint{1.507803in}{1.056981in}}{\pgfqpoint{1.515703in}{1.053709in}}{\pgfqpoint{1.523940in}{1.053709in}}%
\pgfpathclose%
\pgfusepath{stroke,fill}%
\end{pgfscope}%
\begin{pgfscope}%
\pgfpathrectangle{\pgfqpoint{0.100000in}{0.220728in}}{\pgfqpoint{3.696000in}{3.696000in}}%
\pgfusepath{clip}%
\pgfsetbuttcap%
\pgfsetroundjoin%
\definecolor{currentfill}{rgb}{0.121569,0.466667,0.705882}%
\pgfsetfillcolor{currentfill}%
\pgfsetfillopacity{0.838398}%
\pgfsetlinewidth{1.003750pt}%
\definecolor{currentstroke}{rgb}{0.121569,0.466667,0.705882}%
\pgfsetstrokecolor{currentstroke}%
\pgfsetstrokeopacity{0.838398}%
\pgfsetdash{}{0pt}%
\pgfpathmoveto{\pgfqpoint{2.922500in}{1.989798in}}%
\pgfpathcurveto{\pgfqpoint{2.930736in}{1.989798in}}{\pgfqpoint{2.938637in}{1.993070in}}{\pgfqpoint{2.944460in}{1.998894in}}%
\pgfpathcurveto{\pgfqpoint{2.950284in}{2.004718in}}{\pgfqpoint{2.953557in}{2.012618in}}{\pgfqpoint{2.953557in}{2.020854in}}%
\pgfpathcurveto{\pgfqpoint{2.953557in}{2.029090in}}{\pgfqpoint{2.950284in}{2.036990in}}{\pgfqpoint{2.944460in}{2.042814in}}%
\pgfpathcurveto{\pgfqpoint{2.938637in}{2.048638in}}{\pgfqpoint{2.930736in}{2.051911in}}{\pgfqpoint{2.922500in}{2.051911in}}%
\pgfpathcurveto{\pgfqpoint{2.914264in}{2.051911in}}{\pgfqpoint{2.906364in}{2.048638in}}{\pgfqpoint{2.900540in}{2.042814in}}%
\pgfpathcurveto{\pgfqpoint{2.894716in}{2.036990in}}{\pgfqpoint{2.891444in}{2.029090in}}{\pgfqpoint{2.891444in}{2.020854in}}%
\pgfpathcurveto{\pgfqpoint{2.891444in}{2.012618in}}{\pgfqpoint{2.894716in}{2.004718in}}{\pgfqpoint{2.900540in}{1.998894in}}%
\pgfpathcurveto{\pgfqpoint{2.906364in}{1.993070in}}{\pgfqpoint{2.914264in}{1.989798in}}{\pgfqpoint{2.922500in}{1.989798in}}%
\pgfpathclose%
\pgfusepath{stroke,fill}%
\end{pgfscope}%
\begin{pgfscope}%
\pgfpathrectangle{\pgfqpoint{0.100000in}{0.220728in}}{\pgfqpoint{3.696000in}{3.696000in}}%
\pgfusepath{clip}%
\pgfsetbuttcap%
\pgfsetroundjoin%
\definecolor{currentfill}{rgb}{0.121569,0.466667,0.705882}%
\pgfsetfillcolor{currentfill}%
\pgfsetfillopacity{0.839016}%
\pgfsetlinewidth{1.003750pt}%
\definecolor{currentstroke}{rgb}{0.121569,0.466667,0.705882}%
\pgfsetstrokecolor{currentstroke}%
\pgfsetstrokeopacity{0.839016}%
\pgfsetdash{}{0pt}%
\pgfpathmoveto{\pgfqpoint{2.920782in}{1.986055in}}%
\pgfpathcurveto{\pgfqpoint{2.929018in}{1.986055in}}{\pgfqpoint{2.936918in}{1.989327in}}{\pgfqpoint{2.942742in}{1.995151in}}%
\pgfpathcurveto{\pgfqpoint{2.948566in}{2.000975in}}{\pgfqpoint{2.951839in}{2.008875in}}{\pgfqpoint{2.951839in}{2.017111in}}%
\pgfpathcurveto{\pgfqpoint{2.951839in}{2.025347in}}{\pgfqpoint{2.948566in}{2.033248in}}{\pgfqpoint{2.942742in}{2.039071in}}%
\pgfpathcurveto{\pgfqpoint{2.936918in}{2.044895in}}{\pgfqpoint{2.929018in}{2.048168in}}{\pgfqpoint{2.920782in}{2.048168in}}%
\pgfpathcurveto{\pgfqpoint{2.912546in}{2.048168in}}{\pgfqpoint{2.904646in}{2.044895in}}{\pgfqpoint{2.898822in}{2.039071in}}%
\pgfpathcurveto{\pgfqpoint{2.892998in}{2.033248in}}{\pgfqpoint{2.889726in}{2.025347in}}{\pgfqpoint{2.889726in}{2.017111in}}%
\pgfpathcurveto{\pgfqpoint{2.889726in}{2.008875in}}{\pgfqpoint{2.892998in}{2.000975in}}{\pgfqpoint{2.898822in}{1.995151in}}%
\pgfpathcurveto{\pgfqpoint{2.904646in}{1.989327in}}{\pgfqpoint{2.912546in}{1.986055in}}{\pgfqpoint{2.920782in}{1.986055in}}%
\pgfpathclose%
\pgfusepath{stroke,fill}%
\end{pgfscope}%
\begin{pgfscope}%
\pgfpathrectangle{\pgfqpoint{0.100000in}{0.220728in}}{\pgfqpoint{3.696000in}{3.696000in}}%
\pgfusepath{clip}%
\pgfsetbuttcap%
\pgfsetroundjoin%
\definecolor{currentfill}{rgb}{0.121569,0.466667,0.705882}%
\pgfsetfillcolor{currentfill}%
\pgfsetfillopacity{0.839396}%
\pgfsetlinewidth{1.003750pt}%
\definecolor{currentstroke}{rgb}{0.121569,0.466667,0.705882}%
\pgfsetstrokecolor{currentstroke}%
\pgfsetstrokeopacity{0.839396}%
\pgfsetdash{}{0pt}%
\pgfpathmoveto{\pgfqpoint{2.920115in}{1.983907in}}%
\pgfpathcurveto{\pgfqpoint{2.928351in}{1.983907in}}{\pgfqpoint{2.936251in}{1.987179in}}{\pgfqpoint{2.942075in}{1.993003in}}%
\pgfpathcurveto{\pgfqpoint{2.947899in}{1.998827in}}{\pgfqpoint{2.951171in}{2.006727in}}{\pgfqpoint{2.951171in}{2.014963in}}%
\pgfpathcurveto{\pgfqpoint{2.951171in}{2.023200in}}{\pgfqpoint{2.947899in}{2.031100in}}{\pgfqpoint{2.942075in}{2.036924in}}%
\pgfpathcurveto{\pgfqpoint{2.936251in}{2.042748in}}{\pgfqpoint{2.928351in}{2.046020in}}{\pgfqpoint{2.920115in}{2.046020in}}%
\pgfpathcurveto{\pgfqpoint{2.911878in}{2.046020in}}{\pgfqpoint{2.903978in}{2.042748in}}{\pgfqpoint{2.898154in}{2.036924in}}%
\pgfpathcurveto{\pgfqpoint{2.892330in}{2.031100in}}{\pgfqpoint{2.889058in}{2.023200in}}{\pgfqpoint{2.889058in}{2.014963in}}%
\pgfpathcurveto{\pgfqpoint{2.889058in}{2.006727in}}{\pgfqpoint{2.892330in}{1.998827in}}{\pgfqpoint{2.898154in}{1.993003in}}%
\pgfpathcurveto{\pgfqpoint{2.903978in}{1.987179in}}{\pgfqpoint{2.911878in}{1.983907in}}{\pgfqpoint{2.920115in}{1.983907in}}%
\pgfpathclose%
\pgfusepath{stroke,fill}%
\end{pgfscope}%
\begin{pgfscope}%
\pgfpathrectangle{\pgfqpoint{0.100000in}{0.220728in}}{\pgfqpoint{3.696000in}{3.696000in}}%
\pgfusepath{clip}%
\pgfsetbuttcap%
\pgfsetroundjoin%
\definecolor{currentfill}{rgb}{0.121569,0.466667,0.705882}%
\pgfsetfillcolor{currentfill}%
\pgfsetfillopacity{0.839858}%
\pgfsetlinewidth{1.003750pt}%
\definecolor{currentstroke}{rgb}{0.121569,0.466667,0.705882}%
\pgfsetstrokecolor{currentstroke}%
\pgfsetstrokeopacity{0.839858}%
\pgfsetdash{}{0pt}%
\pgfpathmoveto{\pgfqpoint{2.918361in}{1.980768in}}%
\pgfpathcurveto{\pgfqpoint{2.926598in}{1.980768in}}{\pgfqpoint{2.934498in}{1.984040in}}{\pgfqpoint{2.940322in}{1.989864in}}%
\pgfpathcurveto{\pgfqpoint{2.946146in}{1.995688in}}{\pgfqpoint{2.949418in}{2.003588in}}{\pgfqpoint{2.949418in}{2.011824in}}%
\pgfpathcurveto{\pgfqpoint{2.949418in}{2.020060in}}{\pgfqpoint{2.946146in}{2.027960in}}{\pgfqpoint{2.940322in}{2.033784in}}%
\pgfpathcurveto{\pgfqpoint{2.934498in}{2.039608in}}{\pgfqpoint{2.926598in}{2.042881in}}{\pgfqpoint{2.918361in}{2.042881in}}%
\pgfpathcurveto{\pgfqpoint{2.910125in}{2.042881in}}{\pgfqpoint{2.902225in}{2.039608in}}{\pgfqpoint{2.896401in}{2.033784in}}%
\pgfpathcurveto{\pgfqpoint{2.890577in}{2.027960in}}{\pgfqpoint{2.887305in}{2.020060in}}{\pgfqpoint{2.887305in}{2.011824in}}%
\pgfpathcurveto{\pgfqpoint{2.887305in}{2.003588in}}{\pgfqpoint{2.890577in}{1.995688in}}{\pgfqpoint{2.896401in}{1.989864in}}%
\pgfpathcurveto{\pgfqpoint{2.902225in}{1.984040in}}{\pgfqpoint{2.910125in}{1.980768in}}{\pgfqpoint{2.918361in}{1.980768in}}%
\pgfpathclose%
\pgfusepath{stroke,fill}%
\end{pgfscope}%
\begin{pgfscope}%
\pgfpathrectangle{\pgfqpoint{0.100000in}{0.220728in}}{\pgfqpoint{3.696000in}{3.696000in}}%
\pgfusepath{clip}%
\pgfsetbuttcap%
\pgfsetroundjoin%
\definecolor{currentfill}{rgb}{0.121569,0.466667,0.705882}%
\pgfsetfillcolor{currentfill}%
\pgfsetfillopacity{0.840758}%
\pgfsetlinewidth{1.003750pt}%
\definecolor{currentstroke}{rgb}{0.121569,0.466667,0.705882}%
\pgfsetstrokecolor{currentstroke}%
\pgfsetstrokeopacity{0.840758}%
\pgfsetdash{}{0pt}%
\pgfpathmoveto{\pgfqpoint{2.916763in}{1.976131in}}%
\pgfpathcurveto{\pgfqpoint{2.925000in}{1.976131in}}{\pgfqpoint{2.932900in}{1.979403in}}{\pgfqpoint{2.938724in}{1.985227in}}%
\pgfpathcurveto{\pgfqpoint{2.944547in}{1.991051in}}{\pgfqpoint{2.947820in}{1.998951in}}{\pgfqpoint{2.947820in}{2.007187in}}%
\pgfpathcurveto{\pgfqpoint{2.947820in}{2.015423in}}{\pgfqpoint{2.944547in}{2.023324in}}{\pgfqpoint{2.938724in}{2.029147in}}%
\pgfpathcurveto{\pgfqpoint{2.932900in}{2.034971in}}{\pgfqpoint{2.925000in}{2.038244in}}{\pgfqpoint{2.916763in}{2.038244in}}%
\pgfpathcurveto{\pgfqpoint{2.908527in}{2.038244in}}{\pgfqpoint{2.900627in}{2.034971in}}{\pgfqpoint{2.894803in}{2.029147in}}%
\pgfpathcurveto{\pgfqpoint{2.888979in}{2.023324in}}{\pgfqpoint{2.885707in}{2.015423in}}{\pgfqpoint{2.885707in}{2.007187in}}%
\pgfpathcurveto{\pgfqpoint{2.885707in}{1.998951in}}{\pgfqpoint{2.888979in}{1.991051in}}{\pgfqpoint{2.894803in}{1.985227in}}%
\pgfpathcurveto{\pgfqpoint{2.900627in}{1.979403in}}{\pgfqpoint{2.908527in}{1.976131in}}{\pgfqpoint{2.916763in}{1.976131in}}%
\pgfpathclose%
\pgfusepath{stroke,fill}%
\end{pgfscope}%
\begin{pgfscope}%
\pgfpathrectangle{\pgfqpoint{0.100000in}{0.220728in}}{\pgfqpoint{3.696000in}{3.696000in}}%
\pgfusepath{clip}%
\pgfsetbuttcap%
\pgfsetroundjoin%
\definecolor{currentfill}{rgb}{0.121569,0.466667,0.705882}%
\pgfsetfillcolor{currentfill}%
\pgfsetfillopacity{0.841475}%
\pgfsetlinewidth{1.003750pt}%
\definecolor{currentstroke}{rgb}{0.121569,0.466667,0.705882}%
\pgfsetstrokecolor{currentstroke}%
\pgfsetstrokeopacity{0.841475}%
\pgfsetdash{}{0pt}%
\pgfpathmoveto{\pgfqpoint{1.544363in}{1.042889in}}%
\pgfpathcurveto{\pgfqpoint{1.552600in}{1.042889in}}{\pgfqpoint{1.560500in}{1.046161in}}{\pgfqpoint{1.566324in}{1.051985in}}%
\pgfpathcurveto{\pgfqpoint{1.572148in}{1.057809in}}{\pgfqpoint{1.575420in}{1.065709in}}{\pgfqpoint{1.575420in}{1.073946in}}%
\pgfpathcurveto{\pgfqpoint{1.575420in}{1.082182in}}{\pgfqpoint{1.572148in}{1.090082in}}{\pgfqpoint{1.566324in}{1.095906in}}%
\pgfpathcurveto{\pgfqpoint{1.560500in}{1.101730in}}{\pgfqpoint{1.552600in}{1.105002in}}{\pgfqpoint{1.544363in}{1.105002in}}%
\pgfpathcurveto{\pgfqpoint{1.536127in}{1.105002in}}{\pgfqpoint{1.528227in}{1.101730in}}{\pgfqpoint{1.522403in}{1.095906in}}%
\pgfpathcurveto{\pgfqpoint{1.516579in}{1.090082in}}{\pgfqpoint{1.513307in}{1.082182in}}{\pgfqpoint{1.513307in}{1.073946in}}%
\pgfpathcurveto{\pgfqpoint{1.513307in}{1.065709in}}{\pgfqpoint{1.516579in}{1.057809in}}{\pgfqpoint{1.522403in}{1.051985in}}%
\pgfpathcurveto{\pgfqpoint{1.528227in}{1.046161in}}{\pgfqpoint{1.536127in}{1.042889in}}{\pgfqpoint{1.544363in}{1.042889in}}%
\pgfpathclose%
\pgfusepath{stroke,fill}%
\end{pgfscope}%
\begin{pgfscope}%
\pgfpathrectangle{\pgfqpoint{0.100000in}{0.220728in}}{\pgfqpoint{3.696000in}{3.696000in}}%
\pgfusepath{clip}%
\pgfsetbuttcap%
\pgfsetroundjoin%
\definecolor{currentfill}{rgb}{0.121569,0.466667,0.705882}%
\pgfsetfillcolor{currentfill}%
\pgfsetfillopacity{0.841730}%
\pgfsetlinewidth{1.003750pt}%
\definecolor{currentstroke}{rgb}{0.121569,0.466667,0.705882}%
\pgfsetstrokecolor{currentstroke}%
\pgfsetstrokeopacity{0.841730}%
\pgfsetdash{}{0pt}%
\pgfpathmoveto{\pgfqpoint{2.914528in}{1.970939in}}%
\pgfpathcurveto{\pgfqpoint{2.922764in}{1.970939in}}{\pgfqpoint{2.930665in}{1.974211in}}{\pgfqpoint{2.936488in}{1.980035in}}%
\pgfpathcurveto{\pgfqpoint{2.942312in}{1.985859in}}{\pgfqpoint{2.945585in}{1.993759in}}{\pgfqpoint{2.945585in}{2.001995in}}%
\pgfpathcurveto{\pgfqpoint{2.945585in}{2.010232in}}{\pgfqpoint{2.942312in}{2.018132in}}{\pgfqpoint{2.936488in}{2.023956in}}%
\pgfpathcurveto{\pgfqpoint{2.930665in}{2.029780in}}{\pgfqpoint{2.922764in}{2.033052in}}{\pgfqpoint{2.914528in}{2.033052in}}%
\pgfpathcurveto{\pgfqpoint{2.906292in}{2.033052in}}{\pgfqpoint{2.898392in}{2.029780in}}{\pgfqpoint{2.892568in}{2.023956in}}%
\pgfpathcurveto{\pgfqpoint{2.886744in}{2.018132in}}{\pgfqpoint{2.883472in}{2.010232in}}{\pgfqpoint{2.883472in}{2.001995in}}%
\pgfpathcurveto{\pgfqpoint{2.883472in}{1.993759in}}{\pgfqpoint{2.886744in}{1.985859in}}{\pgfqpoint{2.892568in}{1.980035in}}%
\pgfpathcurveto{\pgfqpoint{2.898392in}{1.974211in}}{\pgfqpoint{2.906292in}{1.970939in}}{\pgfqpoint{2.914528in}{1.970939in}}%
\pgfpathclose%
\pgfusepath{stroke,fill}%
\end{pgfscope}%
\begin{pgfscope}%
\pgfpathrectangle{\pgfqpoint{0.100000in}{0.220728in}}{\pgfqpoint{3.696000in}{3.696000in}}%
\pgfusepath{clip}%
\pgfsetbuttcap%
\pgfsetroundjoin%
\definecolor{currentfill}{rgb}{0.121569,0.466667,0.705882}%
\pgfsetfillcolor{currentfill}%
\pgfsetfillopacity{0.842802}%
\pgfsetlinewidth{1.003750pt}%
\definecolor{currentstroke}{rgb}{0.121569,0.466667,0.705882}%
\pgfsetstrokecolor{currentstroke}%
\pgfsetstrokeopacity{0.842802}%
\pgfsetdash{}{0pt}%
\pgfpathmoveto{\pgfqpoint{2.911281in}{1.964959in}}%
\pgfpathcurveto{\pgfqpoint{2.919517in}{1.964959in}}{\pgfqpoint{2.927417in}{1.968231in}}{\pgfqpoint{2.933241in}{1.974055in}}%
\pgfpathcurveto{\pgfqpoint{2.939065in}{1.979879in}}{\pgfqpoint{2.942337in}{1.987779in}}{\pgfqpoint{2.942337in}{1.996015in}}%
\pgfpathcurveto{\pgfqpoint{2.942337in}{2.004252in}}{\pgfqpoint{2.939065in}{2.012152in}}{\pgfqpoint{2.933241in}{2.017976in}}%
\pgfpathcurveto{\pgfqpoint{2.927417in}{2.023800in}}{\pgfqpoint{2.919517in}{2.027072in}}{\pgfqpoint{2.911281in}{2.027072in}}%
\pgfpathcurveto{\pgfqpoint{2.903044in}{2.027072in}}{\pgfqpoint{2.895144in}{2.023800in}}{\pgfqpoint{2.889320in}{2.017976in}}%
\pgfpathcurveto{\pgfqpoint{2.883496in}{2.012152in}}{\pgfqpoint{2.880224in}{2.004252in}}{\pgfqpoint{2.880224in}{1.996015in}}%
\pgfpathcurveto{\pgfqpoint{2.880224in}{1.987779in}}{\pgfqpoint{2.883496in}{1.979879in}}{\pgfqpoint{2.889320in}{1.974055in}}%
\pgfpathcurveto{\pgfqpoint{2.895144in}{1.968231in}}{\pgfqpoint{2.903044in}{1.964959in}}{\pgfqpoint{2.911281in}{1.964959in}}%
\pgfpathclose%
\pgfusepath{stroke,fill}%
\end{pgfscope}%
\begin{pgfscope}%
\pgfpathrectangle{\pgfqpoint{0.100000in}{0.220728in}}{\pgfqpoint{3.696000in}{3.696000in}}%
\pgfusepath{clip}%
\pgfsetbuttcap%
\pgfsetroundjoin%
\definecolor{currentfill}{rgb}{0.121569,0.466667,0.705882}%
\pgfsetfillcolor{currentfill}%
\pgfsetfillopacity{0.844470}%
\pgfsetlinewidth{1.003750pt}%
\definecolor{currentstroke}{rgb}{0.121569,0.466667,0.705882}%
\pgfsetstrokecolor{currentstroke}%
\pgfsetstrokeopacity{0.844470}%
\pgfsetdash{}{0pt}%
\pgfpathmoveto{\pgfqpoint{2.908478in}{1.955830in}}%
\pgfpathcurveto{\pgfqpoint{2.916714in}{1.955830in}}{\pgfqpoint{2.924614in}{1.959102in}}{\pgfqpoint{2.930438in}{1.964926in}}%
\pgfpathcurveto{\pgfqpoint{2.936262in}{1.970750in}}{\pgfqpoint{2.939534in}{1.978650in}}{\pgfqpoint{2.939534in}{1.986886in}}%
\pgfpathcurveto{\pgfqpoint{2.939534in}{1.995123in}}{\pgfqpoint{2.936262in}{2.003023in}}{\pgfqpoint{2.930438in}{2.008847in}}%
\pgfpathcurveto{\pgfqpoint{2.924614in}{2.014671in}}{\pgfqpoint{2.916714in}{2.017943in}}{\pgfqpoint{2.908478in}{2.017943in}}%
\pgfpathcurveto{\pgfqpoint{2.900241in}{2.017943in}}{\pgfqpoint{2.892341in}{2.014671in}}{\pgfqpoint{2.886517in}{2.008847in}}%
\pgfpathcurveto{\pgfqpoint{2.880694in}{2.003023in}}{\pgfqpoint{2.877421in}{1.995123in}}{\pgfqpoint{2.877421in}{1.986886in}}%
\pgfpathcurveto{\pgfqpoint{2.877421in}{1.978650in}}{\pgfqpoint{2.880694in}{1.970750in}}{\pgfqpoint{2.886517in}{1.964926in}}%
\pgfpathcurveto{\pgfqpoint{2.892341in}{1.959102in}}{\pgfqpoint{2.900241in}{1.955830in}}{\pgfqpoint{2.908478in}{1.955830in}}%
\pgfpathclose%
\pgfusepath{stroke,fill}%
\end{pgfscope}%
\begin{pgfscope}%
\pgfpathrectangle{\pgfqpoint{0.100000in}{0.220728in}}{\pgfqpoint{3.696000in}{3.696000in}}%
\pgfusepath{clip}%
\pgfsetbuttcap%
\pgfsetroundjoin%
\definecolor{currentfill}{rgb}{0.121569,0.466667,0.705882}%
\pgfsetfillcolor{currentfill}%
\pgfsetfillopacity{0.844544}%
\pgfsetlinewidth{1.003750pt}%
\definecolor{currentstroke}{rgb}{0.121569,0.466667,0.705882}%
\pgfsetstrokecolor{currentstroke}%
\pgfsetstrokeopacity{0.844544}%
\pgfsetdash{}{0pt}%
\pgfpathmoveto{\pgfqpoint{1.560663in}{1.036019in}}%
\pgfpathcurveto{\pgfqpoint{1.568899in}{1.036019in}}{\pgfqpoint{1.576799in}{1.039291in}}{\pgfqpoint{1.582623in}{1.045115in}}%
\pgfpathcurveto{\pgfqpoint{1.588447in}{1.050939in}}{\pgfqpoint{1.591719in}{1.058839in}}{\pgfqpoint{1.591719in}{1.067076in}}%
\pgfpathcurveto{\pgfqpoint{1.591719in}{1.075312in}}{\pgfqpoint{1.588447in}{1.083212in}}{\pgfqpoint{1.582623in}{1.089036in}}%
\pgfpathcurveto{\pgfqpoint{1.576799in}{1.094860in}}{\pgfqpoint{1.568899in}{1.098132in}}{\pgfqpoint{1.560663in}{1.098132in}}%
\pgfpathcurveto{\pgfqpoint{1.552427in}{1.098132in}}{\pgfqpoint{1.544527in}{1.094860in}}{\pgfqpoint{1.538703in}{1.089036in}}%
\pgfpathcurveto{\pgfqpoint{1.532879in}{1.083212in}}{\pgfqpoint{1.529606in}{1.075312in}}{\pgfqpoint{1.529606in}{1.067076in}}%
\pgfpathcurveto{\pgfqpoint{1.529606in}{1.058839in}}{\pgfqpoint{1.532879in}{1.050939in}}{\pgfqpoint{1.538703in}{1.045115in}}%
\pgfpathcurveto{\pgfqpoint{1.544527in}{1.039291in}}{\pgfqpoint{1.552427in}{1.036019in}}{\pgfqpoint{1.560663in}{1.036019in}}%
\pgfpathclose%
\pgfusepath{stroke,fill}%
\end{pgfscope}%
\begin{pgfscope}%
\pgfpathrectangle{\pgfqpoint{0.100000in}{0.220728in}}{\pgfqpoint{3.696000in}{3.696000in}}%
\pgfusepath{clip}%
\pgfsetbuttcap%
\pgfsetroundjoin%
\definecolor{currentfill}{rgb}{0.121569,0.466667,0.705882}%
\pgfsetfillcolor{currentfill}%
\pgfsetfillopacity{0.845204}%
\pgfsetlinewidth{1.003750pt}%
\definecolor{currentstroke}{rgb}{0.121569,0.466667,0.705882}%
\pgfsetstrokecolor{currentstroke}%
\pgfsetstrokeopacity{0.845204}%
\pgfsetdash{}{0pt}%
\pgfpathmoveto{\pgfqpoint{2.906002in}{1.950983in}}%
\pgfpathcurveto{\pgfqpoint{2.914238in}{1.950983in}}{\pgfqpoint{2.922138in}{1.954256in}}{\pgfqpoint{2.927962in}{1.960080in}}%
\pgfpathcurveto{\pgfqpoint{2.933786in}{1.965904in}}{\pgfqpoint{2.937058in}{1.973804in}}{\pgfqpoint{2.937058in}{1.982040in}}%
\pgfpathcurveto{\pgfqpoint{2.937058in}{1.990276in}}{\pgfqpoint{2.933786in}{1.998176in}}{\pgfqpoint{2.927962in}{2.004000in}}%
\pgfpathcurveto{\pgfqpoint{2.922138in}{2.009824in}}{\pgfqpoint{2.914238in}{2.013096in}}{\pgfqpoint{2.906002in}{2.013096in}}%
\pgfpathcurveto{\pgfqpoint{2.897765in}{2.013096in}}{\pgfqpoint{2.889865in}{2.009824in}}{\pgfqpoint{2.884041in}{2.004000in}}%
\pgfpathcurveto{\pgfqpoint{2.878218in}{1.998176in}}{\pgfqpoint{2.874945in}{1.990276in}}{\pgfqpoint{2.874945in}{1.982040in}}%
\pgfpathcurveto{\pgfqpoint{2.874945in}{1.973804in}}{\pgfqpoint{2.878218in}{1.965904in}}{\pgfqpoint{2.884041in}{1.960080in}}%
\pgfpathcurveto{\pgfqpoint{2.889865in}{1.954256in}}{\pgfqpoint{2.897765in}{1.950983in}}{\pgfqpoint{2.906002in}{1.950983in}}%
\pgfpathclose%
\pgfusepath{stroke,fill}%
\end{pgfscope}%
\begin{pgfscope}%
\pgfpathrectangle{\pgfqpoint{0.100000in}{0.220728in}}{\pgfqpoint{3.696000in}{3.696000in}}%
\pgfusepath{clip}%
\pgfsetbuttcap%
\pgfsetroundjoin%
\definecolor{currentfill}{rgb}{0.121569,0.466667,0.705882}%
\pgfsetfillcolor{currentfill}%
\pgfsetfillopacity{0.845655}%
\pgfsetlinewidth{1.003750pt}%
\definecolor{currentstroke}{rgb}{0.121569,0.466667,0.705882}%
\pgfsetstrokecolor{currentstroke}%
\pgfsetstrokeopacity{0.845655}%
\pgfsetdash{}{0pt}%
\pgfpathmoveto{\pgfqpoint{2.904504in}{1.948713in}}%
\pgfpathcurveto{\pgfqpoint{2.912740in}{1.948713in}}{\pgfqpoint{2.920640in}{1.951985in}}{\pgfqpoint{2.926464in}{1.957809in}}%
\pgfpathcurveto{\pgfqpoint{2.932288in}{1.963633in}}{\pgfqpoint{2.935560in}{1.971533in}}{\pgfqpoint{2.935560in}{1.979769in}}%
\pgfpathcurveto{\pgfqpoint{2.935560in}{1.988006in}}{\pgfqpoint{2.932288in}{1.995906in}}{\pgfqpoint{2.926464in}{2.001730in}}%
\pgfpathcurveto{\pgfqpoint{2.920640in}{2.007554in}}{\pgfqpoint{2.912740in}{2.010826in}}{\pgfqpoint{2.904504in}{2.010826in}}%
\pgfpathcurveto{\pgfqpoint{2.896267in}{2.010826in}}{\pgfqpoint{2.888367in}{2.007554in}}{\pgfqpoint{2.882543in}{2.001730in}}%
\pgfpathcurveto{\pgfqpoint{2.876719in}{1.995906in}}{\pgfqpoint{2.873447in}{1.988006in}}{\pgfqpoint{2.873447in}{1.979769in}}%
\pgfpathcurveto{\pgfqpoint{2.873447in}{1.971533in}}{\pgfqpoint{2.876719in}{1.963633in}}{\pgfqpoint{2.882543in}{1.957809in}}%
\pgfpathcurveto{\pgfqpoint{2.888367in}{1.951985in}}{\pgfqpoint{2.896267in}{1.948713in}}{\pgfqpoint{2.904504in}{1.948713in}}%
\pgfpathclose%
\pgfusepath{stroke,fill}%
\end{pgfscope}%
\begin{pgfscope}%
\pgfpathrectangle{\pgfqpoint{0.100000in}{0.220728in}}{\pgfqpoint{3.696000in}{3.696000in}}%
\pgfusepath{clip}%
\pgfsetbuttcap%
\pgfsetroundjoin%
\definecolor{currentfill}{rgb}{0.121569,0.466667,0.705882}%
\pgfsetfillcolor{currentfill}%
\pgfsetfillopacity{0.846157}%
\pgfsetlinewidth{1.003750pt}%
\definecolor{currentstroke}{rgb}{0.121569,0.466667,0.705882}%
\pgfsetstrokecolor{currentstroke}%
\pgfsetstrokeopacity{0.846157}%
\pgfsetdash{}{0pt}%
\pgfpathmoveto{\pgfqpoint{1.575949in}{1.025817in}}%
\pgfpathcurveto{\pgfqpoint{1.584186in}{1.025817in}}{\pgfqpoint{1.592086in}{1.029089in}}{\pgfqpoint{1.597910in}{1.034913in}}%
\pgfpathcurveto{\pgfqpoint{1.603734in}{1.040737in}}{\pgfqpoint{1.607006in}{1.048637in}}{\pgfqpoint{1.607006in}{1.056874in}}%
\pgfpathcurveto{\pgfqpoint{1.607006in}{1.065110in}}{\pgfqpoint{1.603734in}{1.073010in}}{\pgfqpoint{1.597910in}{1.078834in}}%
\pgfpathcurveto{\pgfqpoint{1.592086in}{1.084658in}}{\pgfqpoint{1.584186in}{1.087930in}}{\pgfqpoint{1.575949in}{1.087930in}}%
\pgfpathcurveto{\pgfqpoint{1.567713in}{1.087930in}}{\pgfqpoint{1.559813in}{1.084658in}}{\pgfqpoint{1.553989in}{1.078834in}}%
\pgfpathcurveto{\pgfqpoint{1.548165in}{1.073010in}}{\pgfqpoint{1.544893in}{1.065110in}}{\pgfqpoint{1.544893in}{1.056874in}}%
\pgfpathcurveto{\pgfqpoint{1.544893in}{1.048637in}}{\pgfqpoint{1.548165in}{1.040737in}}{\pgfqpoint{1.553989in}{1.034913in}}%
\pgfpathcurveto{\pgfqpoint{1.559813in}{1.029089in}}{\pgfqpoint{1.567713in}{1.025817in}}{\pgfqpoint{1.575949in}{1.025817in}}%
\pgfpathclose%
\pgfusepath{stroke,fill}%
\end{pgfscope}%
\begin{pgfscope}%
\pgfpathrectangle{\pgfqpoint{0.100000in}{0.220728in}}{\pgfqpoint{3.696000in}{3.696000in}}%
\pgfusepath{clip}%
\pgfsetbuttcap%
\pgfsetroundjoin%
\definecolor{currentfill}{rgb}{0.121569,0.466667,0.705882}%
\pgfsetfillcolor{currentfill}%
\pgfsetfillopacity{0.846575}%
\pgfsetlinewidth{1.003750pt}%
\definecolor{currentstroke}{rgb}{0.121569,0.466667,0.705882}%
\pgfsetstrokecolor{currentstroke}%
\pgfsetstrokeopacity{0.846575}%
\pgfsetdash{}{0pt}%
\pgfpathmoveto{\pgfqpoint{2.903244in}{1.942571in}}%
\pgfpathcurveto{\pgfqpoint{2.911480in}{1.942571in}}{\pgfqpoint{2.919380in}{1.945843in}}{\pgfqpoint{2.925204in}{1.951667in}}%
\pgfpathcurveto{\pgfqpoint{2.931028in}{1.957491in}}{\pgfqpoint{2.934300in}{1.965391in}}{\pgfqpoint{2.934300in}{1.973627in}}%
\pgfpathcurveto{\pgfqpoint{2.934300in}{1.981864in}}{\pgfqpoint{2.931028in}{1.989764in}}{\pgfqpoint{2.925204in}{1.995587in}}%
\pgfpathcurveto{\pgfqpoint{2.919380in}{2.001411in}}{\pgfqpoint{2.911480in}{2.004684in}}{\pgfqpoint{2.903244in}{2.004684in}}%
\pgfpathcurveto{\pgfqpoint{2.895008in}{2.004684in}}{\pgfqpoint{2.887108in}{2.001411in}}{\pgfqpoint{2.881284in}{1.995587in}}%
\pgfpathcurveto{\pgfqpoint{2.875460in}{1.989764in}}{\pgfqpoint{2.872187in}{1.981864in}}{\pgfqpoint{2.872187in}{1.973627in}}%
\pgfpathcurveto{\pgfqpoint{2.872187in}{1.965391in}}{\pgfqpoint{2.875460in}{1.957491in}}{\pgfqpoint{2.881284in}{1.951667in}}%
\pgfpathcurveto{\pgfqpoint{2.887108in}{1.945843in}}{\pgfqpoint{2.895008in}{1.942571in}}{\pgfqpoint{2.903244in}{1.942571in}}%
\pgfpathclose%
\pgfusepath{stroke,fill}%
\end{pgfscope}%
\begin{pgfscope}%
\pgfpathrectangle{\pgfqpoint{0.100000in}{0.220728in}}{\pgfqpoint{3.696000in}{3.696000in}}%
\pgfusepath{clip}%
\pgfsetbuttcap%
\pgfsetroundjoin%
\definecolor{currentfill}{rgb}{0.121569,0.466667,0.705882}%
\pgfsetfillcolor{currentfill}%
\pgfsetfillopacity{0.847132}%
\pgfsetlinewidth{1.003750pt}%
\definecolor{currentstroke}{rgb}{0.121569,0.466667,0.705882}%
\pgfsetstrokecolor{currentstroke}%
\pgfsetstrokeopacity{0.847132}%
\pgfsetdash{}{0pt}%
\pgfpathmoveto{\pgfqpoint{2.902191in}{1.939623in}}%
\pgfpathcurveto{\pgfqpoint{2.910427in}{1.939623in}}{\pgfqpoint{2.918327in}{1.942895in}}{\pgfqpoint{2.924151in}{1.948719in}}%
\pgfpathcurveto{\pgfqpoint{2.929975in}{1.954543in}}{\pgfqpoint{2.933247in}{1.962443in}}{\pgfqpoint{2.933247in}{1.970679in}}%
\pgfpathcurveto{\pgfqpoint{2.933247in}{1.978915in}}{\pgfqpoint{2.929975in}{1.986815in}}{\pgfqpoint{2.924151in}{1.992639in}}%
\pgfpathcurveto{\pgfqpoint{2.918327in}{1.998463in}}{\pgfqpoint{2.910427in}{2.001736in}}{\pgfqpoint{2.902191in}{2.001736in}}%
\pgfpathcurveto{\pgfqpoint{2.893955in}{2.001736in}}{\pgfqpoint{2.886055in}{1.998463in}}{\pgfqpoint{2.880231in}{1.992639in}}%
\pgfpathcurveto{\pgfqpoint{2.874407in}{1.986815in}}{\pgfqpoint{2.871135in}{1.978915in}}{\pgfqpoint{2.871135in}{1.970679in}}%
\pgfpathcurveto{\pgfqpoint{2.871135in}{1.962443in}}{\pgfqpoint{2.874407in}{1.954543in}}{\pgfqpoint{2.880231in}{1.948719in}}%
\pgfpathcurveto{\pgfqpoint{2.886055in}{1.942895in}}{\pgfqpoint{2.893955in}{1.939623in}}{\pgfqpoint{2.902191in}{1.939623in}}%
\pgfpathclose%
\pgfusepath{stroke,fill}%
\end{pgfscope}%
\begin{pgfscope}%
\pgfpathrectangle{\pgfqpoint{0.100000in}{0.220728in}}{\pgfqpoint{3.696000in}{3.696000in}}%
\pgfusepath{clip}%
\pgfsetbuttcap%
\pgfsetroundjoin%
\definecolor{currentfill}{rgb}{0.121569,0.466667,0.705882}%
\pgfsetfillcolor{currentfill}%
\pgfsetfillopacity{0.847253}%
\pgfsetlinewidth{1.003750pt}%
\definecolor{currentstroke}{rgb}{0.121569,0.466667,0.705882}%
\pgfsetstrokecolor{currentstroke}%
\pgfsetstrokeopacity{0.847253}%
\pgfsetdash{}{0pt}%
\pgfpathmoveto{\pgfqpoint{2.901129in}{1.937898in}}%
\pgfpathcurveto{\pgfqpoint{2.909365in}{1.937898in}}{\pgfqpoint{2.917265in}{1.941170in}}{\pgfqpoint{2.923089in}{1.946994in}}%
\pgfpathcurveto{\pgfqpoint{2.928913in}{1.952818in}}{\pgfqpoint{2.932186in}{1.960718in}}{\pgfqpoint{2.932186in}{1.968954in}}%
\pgfpathcurveto{\pgfqpoint{2.932186in}{1.977190in}}{\pgfqpoint{2.928913in}{1.985090in}}{\pgfqpoint{2.923089in}{1.990914in}}%
\pgfpathcurveto{\pgfqpoint{2.917265in}{1.996738in}}{\pgfqpoint{2.909365in}{2.000011in}}{\pgfqpoint{2.901129in}{2.000011in}}%
\pgfpathcurveto{\pgfqpoint{2.892893in}{2.000011in}}{\pgfqpoint{2.884993in}{1.996738in}}{\pgfqpoint{2.879169in}{1.990914in}}%
\pgfpathcurveto{\pgfqpoint{2.873345in}{1.985090in}}{\pgfqpoint{2.870073in}{1.977190in}}{\pgfqpoint{2.870073in}{1.968954in}}%
\pgfpathcurveto{\pgfqpoint{2.870073in}{1.960718in}}{\pgfqpoint{2.873345in}{1.952818in}}{\pgfqpoint{2.879169in}{1.946994in}}%
\pgfpathcurveto{\pgfqpoint{2.884993in}{1.941170in}}{\pgfqpoint{2.892893in}{1.937898in}}{\pgfqpoint{2.901129in}{1.937898in}}%
\pgfpathclose%
\pgfusepath{stroke,fill}%
\end{pgfscope}%
\begin{pgfscope}%
\pgfpathrectangle{\pgfqpoint{0.100000in}{0.220728in}}{\pgfqpoint{3.696000in}{3.696000in}}%
\pgfusepath{clip}%
\pgfsetbuttcap%
\pgfsetroundjoin%
\definecolor{currentfill}{rgb}{0.121569,0.466667,0.705882}%
\pgfsetfillcolor{currentfill}%
\pgfsetfillopacity{0.848087}%
\pgfsetlinewidth{1.003750pt}%
\definecolor{currentstroke}{rgb}{0.121569,0.466667,0.705882}%
\pgfsetstrokecolor{currentstroke}%
\pgfsetstrokeopacity{0.848087}%
\pgfsetdash{}{0pt}%
\pgfpathmoveto{\pgfqpoint{2.898725in}{1.933462in}}%
\pgfpathcurveto{\pgfqpoint{2.906961in}{1.933462in}}{\pgfqpoint{2.914861in}{1.936734in}}{\pgfqpoint{2.920685in}{1.942558in}}%
\pgfpathcurveto{\pgfqpoint{2.926509in}{1.948382in}}{\pgfqpoint{2.929782in}{1.956282in}}{\pgfqpoint{2.929782in}{1.964519in}}%
\pgfpathcurveto{\pgfqpoint{2.929782in}{1.972755in}}{\pgfqpoint{2.926509in}{1.980655in}}{\pgfqpoint{2.920685in}{1.986479in}}%
\pgfpathcurveto{\pgfqpoint{2.914861in}{1.992303in}}{\pgfqpoint{2.906961in}{1.995575in}}{\pgfqpoint{2.898725in}{1.995575in}}%
\pgfpathcurveto{\pgfqpoint{2.890489in}{1.995575in}}{\pgfqpoint{2.882589in}{1.992303in}}{\pgfqpoint{2.876765in}{1.986479in}}%
\pgfpathcurveto{\pgfqpoint{2.870941in}{1.980655in}}{\pgfqpoint{2.867669in}{1.972755in}}{\pgfqpoint{2.867669in}{1.964519in}}%
\pgfpathcurveto{\pgfqpoint{2.867669in}{1.956282in}}{\pgfqpoint{2.870941in}{1.948382in}}{\pgfqpoint{2.876765in}{1.942558in}}%
\pgfpathcurveto{\pgfqpoint{2.882589in}{1.936734in}}{\pgfqpoint{2.890489in}{1.933462in}}{\pgfqpoint{2.898725in}{1.933462in}}%
\pgfpathclose%
\pgfusepath{stroke,fill}%
\end{pgfscope}%
\begin{pgfscope}%
\pgfpathrectangle{\pgfqpoint{0.100000in}{0.220728in}}{\pgfqpoint{3.696000in}{3.696000in}}%
\pgfusepath{clip}%
\pgfsetbuttcap%
\pgfsetroundjoin%
\definecolor{currentfill}{rgb}{0.121569,0.466667,0.705882}%
\pgfsetfillcolor{currentfill}%
\pgfsetfillopacity{0.849164}%
\pgfsetlinewidth{1.003750pt}%
\definecolor{currentstroke}{rgb}{0.121569,0.466667,0.705882}%
\pgfsetstrokecolor{currentstroke}%
\pgfsetstrokeopacity{0.849164}%
\pgfsetdash{}{0pt}%
\pgfpathmoveto{\pgfqpoint{1.604188in}{1.009192in}}%
\pgfpathcurveto{\pgfqpoint{1.612424in}{1.009192in}}{\pgfqpoint{1.620324in}{1.012464in}}{\pgfqpoint{1.626148in}{1.018288in}}%
\pgfpathcurveto{\pgfqpoint{1.631972in}{1.024112in}}{\pgfqpoint{1.635244in}{1.032012in}}{\pgfqpoint{1.635244in}{1.040249in}}%
\pgfpathcurveto{\pgfqpoint{1.635244in}{1.048485in}}{\pgfqpoint{1.631972in}{1.056385in}}{\pgfqpoint{1.626148in}{1.062209in}}%
\pgfpathcurveto{\pgfqpoint{1.620324in}{1.068033in}}{\pgfqpoint{1.612424in}{1.071305in}}{\pgfqpoint{1.604188in}{1.071305in}}%
\pgfpathcurveto{\pgfqpoint{1.595952in}{1.071305in}}{\pgfqpoint{1.588052in}{1.068033in}}{\pgfqpoint{1.582228in}{1.062209in}}%
\pgfpathcurveto{\pgfqpoint{1.576404in}{1.056385in}}{\pgfqpoint{1.573131in}{1.048485in}}{\pgfqpoint{1.573131in}{1.040249in}}%
\pgfpathcurveto{\pgfqpoint{1.573131in}{1.032012in}}{\pgfqpoint{1.576404in}{1.024112in}}{\pgfqpoint{1.582228in}{1.018288in}}%
\pgfpathcurveto{\pgfqpoint{1.588052in}{1.012464in}}{\pgfqpoint{1.595952in}{1.009192in}}{\pgfqpoint{1.604188in}{1.009192in}}%
\pgfpathclose%
\pgfusepath{stroke,fill}%
\end{pgfscope}%
\begin{pgfscope}%
\pgfpathrectangle{\pgfqpoint{0.100000in}{0.220728in}}{\pgfqpoint{3.696000in}{3.696000in}}%
\pgfusepath{clip}%
\pgfsetbuttcap%
\pgfsetroundjoin%
\definecolor{currentfill}{rgb}{0.121569,0.466667,0.705882}%
\pgfsetfillcolor{currentfill}%
\pgfsetfillopacity{0.849295}%
\pgfsetlinewidth{1.003750pt}%
\definecolor{currentstroke}{rgb}{0.121569,0.466667,0.705882}%
\pgfsetstrokecolor{currentstroke}%
\pgfsetstrokeopacity{0.849295}%
\pgfsetdash{}{0pt}%
\pgfpathmoveto{\pgfqpoint{2.897199in}{1.924773in}}%
\pgfpathcurveto{\pgfqpoint{2.905436in}{1.924773in}}{\pgfqpoint{2.913336in}{1.928045in}}{\pgfqpoint{2.919160in}{1.933869in}}%
\pgfpathcurveto{\pgfqpoint{2.924983in}{1.939693in}}{\pgfqpoint{2.928256in}{1.947593in}}{\pgfqpoint{2.928256in}{1.955829in}}%
\pgfpathcurveto{\pgfqpoint{2.928256in}{1.964066in}}{\pgfqpoint{2.924983in}{1.971966in}}{\pgfqpoint{2.919160in}{1.977790in}}%
\pgfpathcurveto{\pgfqpoint{2.913336in}{1.983613in}}{\pgfqpoint{2.905436in}{1.986886in}}{\pgfqpoint{2.897199in}{1.986886in}}%
\pgfpathcurveto{\pgfqpoint{2.888963in}{1.986886in}}{\pgfqpoint{2.881063in}{1.983613in}}{\pgfqpoint{2.875239in}{1.977790in}}%
\pgfpathcurveto{\pgfqpoint{2.869415in}{1.971966in}}{\pgfqpoint{2.866143in}{1.964066in}}{\pgfqpoint{2.866143in}{1.955829in}}%
\pgfpathcurveto{\pgfqpoint{2.866143in}{1.947593in}}{\pgfqpoint{2.869415in}{1.939693in}}{\pgfqpoint{2.875239in}{1.933869in}}%
\pgfpathcurveto{\pgfqpoint{2.881063in}{1.928045in}}{\pgfqpoint{2.888963in}{1.924773in}}{\pgfqpoint{2.897199in}{1.924773in}}%
\pgfpathclose%
\pgfusepath{stroke,fill}%
\end{pgfscope}%
\begin{pgfscope}%
\pgfpathrectangle{\pgfqpoint{0.100000in}{0.220728in}}{\pgfqpoint{3.696000in}{3.696000in}}%
\pgfusepath{clip}%
\pgfsetbuttcap%
\pgfsetroundjoin%
\definecolor{currentfill}{rgb}{0.121569,0.466667,0.705882}%
\pgfsetfillcolor{currentfill}%
\pgfsetfillopacity{0.850489}%
\pgfsetlinewidth{1.003750pt}%
\definecolor{currentstroke}{rgb}{0.121569,0.466667,0.705882}%
\pgfsetstrokecolor{currentstroke}%
\pgfsetstrokeopacity{0.850489}%
\pgfsetdash{}{0pt}%
\pgfpathmoveto{\pgfqpoint{2.890801in}{1.913103in}}%
\pgfpathcurveto{\pgfqpoint{2.899037in}{1.913103in}}{\pgfqpoint{2.906938in}{1.916375in}}{\pgfqpoint{2.912761in}{1.922199in}}%
\pgfpathcurveto{\pgfqpoint{2.918585in}{1.928023in}}{\pgfqpoint{2.921858in}{1.935923in}}{\pgfqpoint{2.921858in}{1.944160in}}%
\pgfpathcurveto{\pgfqpoint{2.921858in}{1.952396in}}{\pgfqpoint{2.918585in}{1.960296in}}{\pgfqpoint{2.912761in}{1.966120in}}%
\pgfpathcurveto{\pgfqpoint{2.906938in}{1.971944in}}{\pgfqpoint{2.899037in}{1.975216in}}{\pgfqpoint{2.890801in}{1.975216in}}%
\pgfpathcurveto{\pgfqpoint{2.882565in}{1.975216in}}{\pgfqpoint{2.874665in}{1.971944in}}{\pgfqpoint{2.868841in}{1.966120in}}%
\pgfpathcurveto{\pgfqpoint{2.863017in}{1.960296in}}{\pgfqpoint{2.859745in}{1.952396in}}{\pgfqpoint{2.859745in}{1.944160in}}%
\pgfpathcurveto{\pgfqpoint{2.859745in}{1.935923in}}{\pgfqpoint{2.863017in}{1.928023in}}{\pgfqpoint{2.868841in}{1.922199in}}%
\pgfpathcurveto{\pgfqpoint{2.874665in}{1.916375in}}{\pgfqpoint{2.882565in}{1.913103in}}{\pgfqpoint{2.890801in}{1.913103in}}%
\pgfpathclose%
\pgfusepath{stroke,fill}%
\end{pgfscope}%
\begin{pgfscope}%
\pgfpathrectangle{\pgfqpoint{0.100000in}{0.220728in}}{\pgfqpoint{3.696000in}{3.696000in}}%
\pgfusepath{clip}%
\pgfsetbuttcap%
\pgfsetroundjoin%
\definecolor{currentfill}{rgb}{0.121569,0.466667,0.705882}%
\pgfsetfillcolor{currentfill}%
\pgfsetfillopacity{0.851240}%
\pgfsetlinewidth{1.003750pt}%
\definecolor{currentstroke}{rgb}{0.121569,0.466667,0.705882}%
\pgfsetstrokecolor{currentstroke}%
\pgfsetstrokeopacity{0.851240}%
\pgfsetdash{}{0pt}%
\pgfpathmoveto{\pgfqpoint{2.887161in}{1.907282in}}%
\pgfpathcurveto{\pgfqpoint{2.895397in}{1.907282in}}{\pgfqpoint{2.903298in}{1.910554in}}{\pgfqpoint{2.909121in}{1.916378in}}%
\pgfpathcurveto{\pgfqpoint{2.914945in}{1.922202in}}{\pgfqpoint{2.918218in}{1.930102in}}{\pgfqpoint{2.918218in}{1.938338in}}%
\pgfpathcurveto{\pgfqpoint{2.918218in}{1.946575in}}{\pgfqpoint{2.914945in}{1.954475in}}{\pgfqpoint{2.909121in}{1.960299in}}%
\pgfpathcurveto{\pgfqpoint{2.903298in}{1.966123in}}{\pgfqpoint{2.895397in}{1.969395in}}{\pgfqpoint{2.887161in}{1.969395in}}%
\pgfpathcurveto{\pgfqpoint{2.878925in}{1.969395in}}{\pgfqpoint{2.871025in}{1.966123in}}{\pgfqpoint{2.865201in}{1.960299in}}%
\pgfpathcurveto{\pgfqpoint{2.859377in}{1.954475in}}{\pgfqpoint{2.856105in}{1.946575in}}{\pgfqpoint{2.856105in}{1.938338in}}%
\pgfpathcurveto{\pgfqpoint{2.856105in}{1.930102in}}{\pgfqpoint{2.859377in}{1.922202in}}{\pgfqpoint{2.865201in}{1.916378in}}%
\pgfpathcurveto{\pgfqpoint{2.871025in}{1.910554in}}{\pgfqpoint{2.878925in}{1.907282in}}{\pgfqpoint{2.887161in}{1.907282in}}%
\pgfpathclose%
\pgfusepath{stroke,fill}%
\end{pgfscope}%
\begin{pgfscope}%
\pgfpathrectangle{\pgfqpoint{0.100000in}{0.220728in}}{\pgfqpoint{3.696000in}{3.696000in}}%
\pgfusepath{clip}%
\pgfsetbuttcap%
\pgfsetroundjoin%
\definecolor{currentfill}{rgb}{0.121569,0.466667,0.705882}%
\pgfsetfillcolor{currentfill}%
\pgfsetfillopacity{0.852697}%
\pgfsetlinewidth{1.003750pt}%
\definecolor{currentstroke}{rgb}{0.121569,0.466667,0.705882}%
\pgfsetstrokecolor{currentstroke}%
\pgfsetstrokeopacity{0.852697}%
\pgfsetdash{}{0pt}%
\pgfpathmoveto{\pgfqpoint{2.884954in}{1.894496in}}%
\pgfpathcurveto{\pgfqpoint{2.893190in}{1.894496in}}{\pgfqpoint{2.901090in}{1.897768in}}{\pgfqpoint{2.906914in}{1.903592in}}%
\pgfpathcurveto{\pgfqpoint{2.912738in}{1.909416in}}{\pgfqpoint{2.916010in}{1.917316in}}{\pgfqpoint{2.916010in}{1.925552in}}%
\pgfpathcurveto{\pgfqpoint{2.916010in}{1.933789in}}{\pgfqpoint{2.912738in}{1.941689in}}{\pgfqpoint{2.906914in}{1.947513in}}%
\pgfpathcurveto{\pgfqpoint{2.901090in}{1.953336in}}{\pgfqpoint{2.893190in}{1.956609in}}{\pgfqpoint{2.884954in}{1.956609in}}%
\pgfpathcurveto{\pgfqpoint{2.876718in}{1.956609in}}{\pgfqpoint{2.868818in}{1.953336in}}{\pgfqpoint{2.862994in}{1.947513in}}%
\pgfpathcurveto{\pgfqpoint{2.857170in}{1.941689in}}{\pgfqpoint{2.853897in}{1.933789in}}{\pgfqpoint{2.853897in}{1.925552in}}%
\pgfpathcurveto{\pgfqpoint{2.853897in}{1.917316in}}{\pgfqpoint{2.857170in}{1.909416in}}{\pgfqpoint{2.862994in}{1.903592in}}%
\pgfpathcurveto{\pgfqpoint{2.868818in}{1.897768in}}{\pgfqpoint{2.876718in}{1.894496in}}{\pgfqpoint{2.884954in}{1.894496in}}%
\pgfpathclose%
\pgfusepath{stroke,fill}%
\end{pgfscope}%
\begin{pgfscope}%
\pgfpathrectangle{\pgfqpoint{0.100000in}{0.220728in}}{\pgfqpoint{3.696000in}{3.696000in}}%
\pgfusepath{clip}%
\pgfsetbuttcap%
\pgfsetroundjoin%
\definecolor{currentfill}{rgb}{0.121569,0.466667,0.705882}%
\pgfsetfillcolor{currentfill}%
\pgfsetfillopacity{0.854345}%
\pgfsetlinewidth{1.003750pt}%
\definecolor{currentstroke}{rgb}{0.121569,0.466667,0.705882}%
\pgfsetstrokecolor{currentstroke}%
\pgfsetstrokeopacity{0.854345}%
\pgfsetdash{}{0pt}%
\pgfpathmoveto{\pgfqpoint{2.878530in}{1.883234in}}%
\pgfpathcurveto{\pgfqpoint{2.886766in}{1.883234in}}{\pgfqpoint{2.894666in}{1.886506in}}{\pgfqpoint{2.900490in}{1.892330in}}%
\pgfpathcurveto{\pgfqpoint{2.906314in}{1.898154in}}{\pgfqpoint{2.909586in}{1.906054in}}{\pgfqpoint{2.909586in}{1.914290in}}%
\pgfpathcurveto{\pgfqpoint{2.909586in}{1.922527in}}{\pgfqpoint{2.906314in}{1.930427in}}{\pgfqpoint{2.900490in}{1.936251in}}%
\pgfpathcurveto{\pgfqpoint{2.894666in}{1.942075in}}{\pgfqpoint{2.886766in}{1.945347in}}{\pgfqpoint{2.878530in}{1.945347in}}%
\pgfpathcurveto{\pgfqpoint{2.870293in}{1.945347in}}{\pgfqpoint{2.862393in}{1.942075in}}{\pgfqpoint{2.856569in}{1.936251in}}%
\pgfpathcurveto{\pgfqpoint{2.850746in}{1.930427in}}{\pgfqpoint{2.847473in}{1.922527in}}{\pgfqpoint{2.847473in}{1.914290in}}%
\pgfpathcurveto{\pgfqpoint{2.847473in}{1.906054in}}{\pgfqpoint{2.850746in}{1.898154in}}{\pgfqpoint{2.856569in}{1.892330in}}%
\pgfpathcurveto{\pgfqpoint{2.862393in}{1.886506in}}{\pgfqpoint{2.870293in}{1.883234in}}{\pgfqpoint{2.878530in}{1.883234in}}%
\pgfpathclose%
\pgfusepath{stroke,fill}%
\end{pgfscope}%
\begin{pgfscope}%
\pgfpathrectangle{\pgfqpoint{0.100000in}{0.220728in}}{\pgfqpoint{3.696000in}{3.696000in}}%
\pgfusepath{clip}%
\pgfsetbuttcap%
\pgfsetroundjoin%
\definecolor{currentfill}{rgb}{0.121569,0.466667,0.705882}%
\pgfsetfillcolor{currentfill}%
\pgfsetfillopacity{0.855144}%
\pgfsetlinewidth{1.003750pt}%
\definecolor{currentstroke}{rgb}{0.121569,0.466667,0.705882}%
\pgfsetstrokecolor{currentstroke}%
\pgfsetstrokeopacity{0.855144}%
\pgfsetdash{}{0pt}%
\pgfpathmoveto{\pgfqpoint{2.874446in}{1.877569in}}%
\pgfpathcurveto{\pgfqpoint{2.882683in}{1.877569in}}{\pgfqpoint{2.890583in}{1.880842in}}{\pgfqpoint{2.896407in}{1.886666in}}%
\pgfpathcurveto{\pgfqpoint{2.902231in}{1.892490in}}{\pgfqpoint{2.905503in}{1.900390in}}{\pgfqpoint{2.905503in}{1.908626in}}%
\pgfpathcurveto{\pgfqpoint{2.905503in}{1.916862in}}{\pgfqpoint{2.902231in}{1.924762in}}{\pgfqpoint{2.896407in}{1.930586in}}%
\pgfpathcurveto{\pgfqpoint{2.890583in}{1.936410in}}{\pgfqpoint{2.882683in}{1.939682in}}{\pgfqpoint{2.874446in}{1.939682in}}%
\pgfpathcurveto{\pgfqpoint{2.866210in}{1.939682in}}{\pgfqpoint{2.858310in}{1.936410in}}{\pgfqpoint{2.852486in}{1.930586in}}%
\pgfpathcurveto{\pgfqpoint{2.846662in}{1.924762in}}{\pgfqpoint{2.843390in}{1.916862in}}{\pgfqpoint{2.843390in}{1.908626in}}%
\pgfpathcurveto{\pgfqpoint{2.843390in}{1.900390in}}{\pgfqpoint{2.846662in}{1.892490in}}{\pgfqpoint{2.852486in}{1.886666in}}%
\pgfpathcurveto{\pgfqpoint{2.858310in}{1.880842in}}{\pgfqpoint{2.866210in}{1.877569in}}{\pgfqpoint{2.874446in}{1.877569in}}%
\pgfpathclose%
\pgfusepath{stroke,fill}%
\end{pgfscope}%
\begin{pgfscope}%
\pgfpathrectangle{\pgfqpoint{0.100000in}{0.220728in}}{\pgfqpoint{3.696000in}{3.696000in}}%
\pgfusepath{clip}%
\pgfsetbuttcap%
\pgfsetroundjoin%
\definecolor{currentfill}{rgb}{0.121569,0.466667,0.705882}%
\pgfsetfillcolor{currentfill}%
\pgfsetfillopacity{0.855324}%
\pgfsetlinewidth{1.003750pt}%
\definecolor{currentstroke}{rgb}{0.121569,0.466667,0.705882}%
\pgfsetstrokecolor{currentstroke}%
\pgfsetstrokeopacity{0.855324}%
\pgfsetdash{}{0pt}%
\pgfpathmoveto{\pgfqpoint{1.629904in}{1.000625in}}%
\pgfpathcurveto{\pgfqpoint{1.638140in}{1.000625in}}{\pgfqpoint{1.646040in}{1.003898in}}{\pgfqpoint{1.651864in}{1.009721in}}%
\pgfpathcurveto{\pgfqpoint{1.657688in}{1.015545in}}{\pgfqpoint{1.660960in}{1.023445in}}{\pgfqpoint{1.660960in}{1.031682in}}%
\pgfpathcurveto{\pgfqpoint{1.660960in}{1.039918in}}{\pgfqpoint{1.657688in}{1.047818in}}{\pgfqpoint{1.651864in}{1.053642in}}%
\pgfpathcurveto{\pgfqpoint{1.646040in}{1.059466in}}{\pgfqpoint{1.638140in}{1.062738in}}{\pgfqpoint{1.629904in}{1.062738in}}%
\pgfpathcurveto{\pgfqpoint{1.621668in}{1.062738in}}{\pgfqpoint{1.613767in}{1.059466in}}{\pgfqpoint{1.607944in}{1.053642in}}%
\pgfpathcurveto{\pgfqpoint{1.602120in}{1.047818in}}{\pgfqpoint{1.598847in}{1.039918in}}{\pgfqpoint{1.598847in}{1.031682in}}%
\pgfpathcurveto{\pgfqpoint{1.598847in}{1.023445in}}{\pgfqpoint{1.602120in}{1.015545in}}{\pgfqpoint{1.607944in}{1.009721in}}%
\pgfpathcurveto{\pgfqpoint{1.613767in}{1.003898in}}{\pgfqpoint{1.621668in}{1.000625in}}{\pgfqpoint{1.629904in}{1.000625in}}%
\pgfpathclose%
\pgfusepath{stroke,fill}%
\end{pgfscope}%
\begin{pgfscope}%
\pgfpathrectangle{\pgfqpoint{0.100000in}{0.220728in}}{\pgfqpoint{3.696000in}{3.696000in}}%
\pgfusepath{clip}%
\pgfsetbuttcap%
\pgfsetroundjoin%
\definecolor{currentfill}{rgb}{0.121569,0.466667,0.705882}%
\pgfsetfillcolor{currentfill}%
\pgfsetfillopacity{0.857114}%
\pgfsetlinewidth{1.003750pt}%
\definecolor{currentstroke}{rgb}{0.121569,0.466667,0.705882}%
\pgfsetstrokecolor{currentstroke}%
\pgfsetstrokeopacity{0.857114}%
\pgfsetdash{}{0pt}%
\pgfpathmoveto{\pgfqpoint{2.871061in}{1.862685in}}%
\pgfpathcurveto{\pgfqpoint{2.879297in}{1.862685in}}{\pgfqpoint{2.887197in}{1.865957in}}{\pgfqpoint{2.893021in}{1.871781in}}%
\pgfpathcurveto{\pgfqpoint{2.898845in}{1.877605in}}{\pgfqpoint{2.902117in}{1.885505in}}{\pgfqpoint{2.902117in}{1.893741in}}%
\pgfpathcurveto{\pgfqpoint{2.902117in}{1.901977in}}{\pgfqpoint{2.898845in}{1.909877in}}{\pgfqpoint{2.893021in}{1.915701in}}%
\pgfpathcurveto{\pgfqpoint{2.887197in}{1.921525in}}{\pgfqpoint{2.879297in}{1.924798in}}{\pgfqpoint{2.871061in}{1.924798in}}%
\pgfpathcurveto{\pgfqpoint{2.862824in}{1.924798in}}{\pgfqpoint{2.854924in}{1.921525in}}{\pgfqpoint{2.849100in}{1.915701in}}%
\pgfpathcurveto{\pgfqpoint{2.843277in}{1.909877in}}{\pgfqpoint{2.840004in}{1.901977in}}{\pgfqpoint{2.840004in}{1.893741in}}%
\pgfpathcurveto{\pgfqpoint{2.840004in}{1.885505in}}{\pgfqpoint{2.843277in}{1.877605in}}{\pgfqpoint{2.849100in}{1.871781in}}%
\pgfpathcurveto{\pgfqpoint{2.854924in}{1.865957in}}{\pgfqpoint{2.862824in}{1.862685in}}{\pgfqpoint{2.871061in}{1.862685in}}%
\pgfpathclose%
\pgfusepath{stroke,fill}%
\end{pgfscope}%
\begin{pgfscope}%
\pgfpathrectangle{\pgfqpoint{0.100000in}{0.220728in}}{\pgfqpoint{3.696000in}{3.696000in}}%
\pgfusepath{clip}%
\pgfsetbuttcap%
\pgfsetroundjoin%
\definecolor{currentfill}{rgb}{0.121569,0.466667,0.705882}%
\pgfsetfillcolor{currentfill}%
\pgfsetfillopacity{0.858075}%
\pgfsetlinewidth{1.003750pt}%
\definecolor{currentstroke}{rgb}{0.121569,0.466667,0.705882}%
\pgfsetstrokecolor{currentstroke}%
\pgfsetstrokeopacity{0.858075}%
\pgfsetdash{}{0pt}%
\pgfpathmoveto{\pgfqpoint{2.867361in}{1.855757in}}%
\pgfpathcurveto{\pgfqpoint{2.875597in}{1.855757in}}{\pgfqpoint{2.883497in}{1.859029in}}{\pgfqpoint{2.889321in}{1.864853in}}%
\pgfpathcurveto{\pgfqpoint{2.895145in}{1.870677in}}{\pgfqpoint{2.898417in}{1.878577in}}{\pgfqpoint{2.898417in}{1.886813in}}%
\pgfpathcurveto{\pgfqpoint{2.898417in}{1.895050in}}{\pgfqpoint{2.895145in}{1.902950in}}{\pgfqpoint{2.889321in}{1.908774in}}%
\pgfpathcurveto{\pgfqpoint{2.883497in}{1.914598in}}{\pgfqpoint{2.875597in}{1.917870in}}{\pgfqpoint{2.867361in}{1.917870in}}%
\pgfpathcurveto{\pgfqpoint{2.859125in}{1.917870in}}{\pgfqpoint{2.851225in}{1.914598in}}{\pgfqpoint{2.845401in}{1.908774in}}%
\pgfpathcurveto{\pgfqpoint{2.839577in}{1.902950in}}{\pgfqpoint{2.836304in}{1.895050in}}{\pgfqpoint{2.836304in}{1.886813in}}%
\pgfpathcurveto{\pgfqpoint{2.836304in}{1.878577in}}{\pgfqpoint{2.839577in}{1.870677in}}{\pgfqpoint{2.845401in}{1.864853in}}%
\pgfpathcurveto{\pgfqpoint{2.851225in}{1.859029in}}{\pgfqpoint{2.859125in}{1.855757in}}{\pgfqpoint{2.867361in}{1.855757in}}%
\pgfpathclose%
\pgfusepath{stroke,fill}%
\end{pgfscope}%
\begin{pgfscope}%
\pgfpathrectangle{\pgfqpoint{0.100000in}{0.220728in}}{\pgfqpoint{3.696000in}{3.696000in}}%
\pgfusepath{clip}%
\pgfsetbuttcap%
\pgfsetroundjoin%
\definecolor{currentfill}{rgb}{0.121569,0.466667,0.705882}%
\pgfsetfillcolor{currentfill}%
\pgfsetfillopacity{0.858833}%
\pgfsetlinewidth{1.003750pt}%
\definecolor{currentstroke}{rgb}{0.121569,0.466667,0.705882}%
\pgfsetstrokecolor{currentstroke}%
\pgfsetstrokeopacity{0.858833}%
\pgfsetdash{}{0pt}%
\pgfpathmoveto{\pgfqpoint{2.862009in}{1.848572in}}%
\pgfpathcurveto{\pgfqpoint{2.870245in}{1.848572in}}{\pgfqpoint{2.878145in}{1.851845in}}{\pgfqpoint{2.883969in}{1.857669in}}%
\pgfpathcurveto{\pgfqpoint{2.889793in}{1.863493in}}{\pgfqpoint{2.893065in}{1.871393in}}{\pgfqpoint{2.893065in}{1.879629in}}%
\pgfpathcurveto{\pgfqpoint{2.893065in}{1.887865in}}{\pgfqpoint{2.889793in}{1.895765in}}{\pgfqpoint{2.883969in}{1.901589in}}%
\pgfpathcurveto{\pgfqpoint{2.878145in}{1.907413in}}{\pgfqpoint{2.870245in}{1.910685in}}{\pgfqpoint{2.862009in}{1.910685in}}%
\pgfpathcurveto{\pgfqpoint{2.853773in}{1.910685in}}{\pgfqpoint{2.845873in}{1.907413in}}{\pgfqpoint{2.840049in}{1.901589in}}%
\pgfpathcurveto{\pgfqpoint{2.834225in}{1.895765in}}{\pgfqpoint{2.830952in}{1.887865in}}{\pgfqpoint{2.830952in}{1.879629in}}%
\pgfpathcurveto{\pgfqpoint{2.830952in}{1.871393in}}{\pgfqpoint{2.834225in}{1.863493in}}{\pgfqpoint{2.840049in}{1.857669in}}%
\pgfpathcurveto{\pgfqpoint{2.845873in}{1.851845in}}{\pgfqpoint{2.853773in}{1.848572in}}{\pgfqpoint{2.862009in}{1.848572in}}%
\pgfpathclose%
\pgfusepath{stroke,fill}%
\end{pgfscope}%
\begin{pgfscope}%
\pgfpathrectangle{\pgfqpoint{0.100000in}{0.220728in}}{\pgfqpoint{3.696000in}{3.696000in}}%
\pgfusepath{clip}%
\pgfsetbuttcap%
\pgfsetroundjoin%
\definecolor{currentfill}{rgb}{0.121569,0.466667,0.705882}%
\pgfsetfillcolor{currentfill}%
\pgfsetfillopacity{0.859831}%
\pgfsetlinewidth{1.003750pt}%
\definecolor{currentstroke}{rgb}{0.121569,0.466667,0.705882}%
\pgfsetstrokecolor{currentstroke}%
\pgfsetstrokeopacity{0.859831}%
\pgfsetdash{}{0pt}%
\pgfpathmoveto{\pgfqpoint{1.654282in}{0.988796in}}%
\pgfpathcurveto{\pgfqpoint{1.662519in}{0.988796in}}{\pgfqpoint{1.670419in}{0.992069in}}{\pgfqpoint{1.676243in}{0.997893in}}%
\pgfpathcurveto{\pgfqpoint{1.682067in}{1.003716in}}{\pgfqpoint{1.685339in}{1.011617in}}{\pgfqpoint{1.685339in}{1.019853in}}%
\pgfpathcurveto{\pgfqpoint{1.685339in}{1.028089in}}{\pgfqpoint{1.682067in}{1.035989in}}{\pgfqpoint{1.676243in}{1.041813in}}%
\pgfpathcurveto{\pgfqpoint{1.670419in}{1.047637in}}{\pgfqpoint{1.662519in}{1.050909in}}{\pgfqpoint{1.654282in}{1.050909in}}%
\pgfpathcurveto{\pgfqpoint{1.646046in}{1.050909in}}{\pgfqpoint{1.638146in}{1.047637in}}{\pgfqpoint{1.632322in}{1.041813in}}%
\pgfpathcurveto{\pgfqpoint{1.626498in}{1.035989in}}{\pgfqpoint{1.623226in}{1.028089in}}{\pgfqpoint{1.623226in}{1.019853in}}%
\pgfpathcurveto{\pgfqpoint{1.623226in}{1.011617in}}{\pgfqpoint{1.626498in}{1.003716in}}{\pgfqpoint{1.632322in}{0.997893in}}%
\pgfpathcurveto{\pgfqpoint{1.638146in}{0.992069in}}{\pgfqpoint{1.646046in}{0.988796in}}{\pgfqpoint{1.654282in}{0.988796in}}%
\pgfpathclose%
\pgfusepath{stroke,fill}%
\end{pgfscope}%
\begin{pgfscope}%
\pgfpathrectangle{\pgfqpoint{0.100000in}{0.220728in}}{\pgfqpoint{3.696000in}{3.696000in}}%
\pgfusepath{clip}%
\pgfsetbuttcap%
\pgfsetroundjoin%
\definecolor{currentfill}{rgb}{0.121569,0.466667,0.705882}%
\pgfsetfillcolor{currentfill}%
\pgfsetfillopacity{0.860375}%
\pgfsetlinewidth{1.003750pt}%
\definecolor{currentstroke}{rgb}{0.121569,0.466667,0.705882}%
\pgfsetstrokecolor{currentstroke}%
\pgfsetstrokeopacity{0.860375}%
\pgfsetdash{}{0pt}%
\pgfpathmoveto{\pgfqpoint{2.858543in}{1.834386in}}%
\pgfpathcurveto{\pgfqpoint{2.866779in}{1.834386in}}{\pgfqpoint{2.874679in}{1.837658in}}{\pgfqpoint{2.880503in}{1.843482in}}%
\pgfpathcurveto{\pgfqpoint{2.886327in}{1.849306in}}{\pgfqpoint{2.889599in}{1.857206in}}{\pgfqpoint{2.889599in}{1.865442in}}%
\pgfpathcurveto{\pgfqpoint{2.889599in}{1.873679in}}{\pgfqpoint{2.886327in}{1.881579in}}{\pgfqpoint{2.880503in}{1.887403in}}%
\pgfpathcurveto{\pgfqpoint{2.874679in}{1.893226in}}{\pgfqpoint{2.866779in}{1.896499in}}{\pgfqpoint{2.858543in}{1.896499in}}%
\pgfpathcurveto{\pgfqpoint{2.850306in}{1.896499in}}{\pgfqpoint{2.842406in}{1.893226in}}{\pgfqpoint{2.836582in}{1.887403in}}%
\pgfpathcurveto{\pgfqpoint{2.830759in}{1.881579in}}{\pgfqpoint{2.827486in}{1.873679in}}{\pgfqpoint{2.827486in}{1.865442in}}%
\pgfpathcurveto{\pgfqpoint{2.827486in}{1.857206in}}{\pgfqpoint{2.830759in}{1.849306in}}{\pgfqpoint{2.836582in}{1.843482in}}%
\pgfpathcurveto{\pgfqpoint{2.842406in}{1.837658in}}{\pgfqpoint{2.850306in}{1.834386in}}{\pgfqpoint{2.858543in}{1.834386in}}%
\pgfpathclose%
\pgfusepath{stroke,fill}%
\end{pgfscope}%
\begin{pgfscope}%
\pgfpathrectangle{\pgfqpoint{0.100000in}{0.220728in}}{\pgfqpoint{3.696000in}{3.696000in}}%
\pgfusepath{clip}%
\pgfsetbuttcap%
\pgfsetroundjoin%
\definecolor{currentfill}{rgb}{0.121569,0.466667,0.705882}%
\pgfsetfillcolor{currentfill}%
\pgfsetfillopacity{0.862204}%
\pgfsetlinewidth{1.003750pt}%
\definecolor{currentstroke}{rgb}{0.121569,0.466667,0.705882}%
\pgfsetstrokecolor{currentstroke}%
\pgfsetstrokeopacity{0.862204}%
\pgfsetdash{}{0pt}%
\pgfpathmoveto{\pgfqpoint{2.850374in}{1.822464in}}%
\pgfpathcurveto{\pgfqpoint{2.858611in}{1.822464in}}{\pgfqpoint{2.866511in}{1.825736in}}{\pgfqpoint{2.872335in}{1.831560in}}%
\pgfpathcurveto{\pgfqpoint{2.878159in}{1.837384in}}{\pgfqpoint{2.881431in}{1.845284in}}{\pgfqpoint{2.881431in}{1.853521in}}%
\pgfpathcurveto{\pgfqpoint{2.881431in}{1.861757in}}{\pgfqpoint{2.878159in}{1.869657in}}{\pgfqpoint{2.872335in}{1.875481in}}%
\pgfpathcurveto{\pgfqpoint{2.866511in}{1.881305in}}{\pgfqpoint{2.858611in}{1.884577in}}{\pgfqpoint{2.850374in}{1.884577in}}%
\pgfpathcurveto{\pgfqpoint{2.842138in}{1.884577in}}{\pgfqpoint{2.834238in}{1.881305in}}{\pgfqpoint{2.828414in}{1.875481in}}%
\pgfpathcurveto{\pgfqpoint{2.822590in}{1.869657in}}{\pgfqpoint{2.819318in}{1.861757in}}{\pgfqpoint{2.819318in}{1.853521in}}%
\pgfpathcurveto{\pgfqpoint{2.819318in}{1.845284in}}{\pgfqpoint{2.822590in}{1.837384in}}{\pgfqpoint{2.828414in}{1.831560in}}%
\pgfpathcurveto{\pgfqpoint{2.834238in}{1.825736in}}{\pgfqpoint{2.842138in}{1.822464in}}{\pgfqpoint{2.850374in}{1.822464in}}%
\pgfpathclose%
\pgfusepath{stroke,fill}%
\end{pgfscope}%
\begin{pgfscope}%
\pgfpathrectangle{\pgfqpoint{0.100000in}{0.220728in}}{\pgfqpoint{3.696000in}{3.696000in}}%
\pgfusepath{clip}%
\pgfsetbuttcap%
\pgfsetroundjoin%
\definecolor{currentfill}{rgb}{0.121569,0.466667,0.705882}%
\pgfsetfillcolor{currentfill}%
\pgfsetfillopacity{0.863142}%
\pgfsetlinewidth{1.003750pt}%
\definecolor{currentstroke}{rgb}{0.121569,0.466667,0.705882}%
\pgfsetstrokecolor{currentstroke}%
\pgfsetstrokeopacity{0.863142}%
\pgfsetdash{}{0pt}%
\pgfpathmoveto{\pgfqpoint{2.846044in}{1.815328in}}%
\pgfpathcurveto{\pgfqpoint{2.854280in}{1.815328in}}{\pgfqpoint{2.862180in}{1.818600in}}{\pgfqpoint{2.868004in}{1.824424in}}%
\pgfpathcurveto{\pgfqpoint{2.873828in}{1.830248in}}{\pgfqpoint{2.877100in}{1.838148in}}{\pgfqpoint{2.877100in}{1.846384in}}%
\pgfpathcurveto{\pgfqpoint{2.877100in}{1.854621in}}{\pgfqpoint{2.873828in}{1.862521in}}{\pgfqpoint{2.868004in}{1.868345in}}%
\pgfpathcurveto{\pgfqpoint{2.862180in}{1.874168in}}{\pgfqpoint{2.854280in}{1.877441in}}{\pgfqpoint{2.846044in}{1.877441in}}%
\pgfpathcurveto{\pgfqpoint{2.837807in}{1.877441in}}{\pgfqpoint{2.829907in}{1.874168in}}{\pgfqpoint{2.824083in}{1.868345in}}%
\pgfpathcurveto{\pgfqpoint{2.818259in}{1.862521in}}{\pgfqpoint{2.814987in}{1.854621in}}{\pgfqpoint{2.814987in}{1.846384in}}%
\pgfpathcurveto{\pgfqpoint{2.814987in}{1.838148in}}{\pgfqpoint{2.818259in}{1.830248in}}{\pgfqpoint{2.824083in}{1.824424in}}%
\pgfpathcurveto{\pgfqpoint{2.829907in}{1.818600in}}{\pgfqpoint{2.837807in}{1.815328in}}{\pgfqpoint{2.846044in}{1.815328in}}%
\pgfpathclose%
\pgfusepath{stroke,fill}%
\end{pgfscope}%
\begin{pgfscope}%
\pgfpathrectangle{\pgfqpoint{0.100000in}{0.220728in}}{\pgfqpoint{3.696000in}{3.696000in}}%
\pgfusepath{clip}%
\pgfsetbuttcap%
\pgfsetroundjoin%
\definecolor{currentfill}{rgb}{0.121569,0.466667,0.705882}%
\pgfsetfillcolor{currentfill}%
\pgfsetfillopacity{0.863747}%
\pgfsetlinewidth{1.003750pt}%
\definecolor{currentstroke}{rgb}{0.121569,0.466667,0.705882}%
\pgfsetstrokecolor{currentstroke}%
\pgfsetstrokeopacity{0.863747}%
\pgfsetdash{}{0pt}%
\pgfpathmoveto{\pgfqpoint{2.844743in}{1.810520in}}%
\pgfpathcurveto{\pgfqpoint{2.852980in}{1.810520in}}{\pgfqpoint{2.860880in}{1.813792in}}{\pgfqpoint{2.866704in}{1.819616in}}%
\pgfpathcurveto{\pgfqpoint{2.872528in}{1.825440in}}{\pgfqpoint{2.875800in}{1.833340in}}{\pgfqpoint{2.875800in}{1.841577in}}%
\pgfpathcurveto{\pgfqpoint{2.875800in}{1.849813in}}{\pgfqpoint{2.872528in}{1.857713in}}{\pgfqpoint{2.866704in}{1.863537in}}%
\pgfpathcurveto{\pgfqpoint{2.860880in}{1.869361in}}{\pgfqpoint{2.852980in}{1.872633in}}{\pgfqpoint{2.844743in}{1.872633in}}%
\pgfpathcurveto{\pgfqpoint{2.836507in}{1.872633in}}{\pgfqpoint{2.828607in}{1.869361in}}{\pgfqpoint{2.822783in}{1.863537in}}%
\pgfpathcurveto{\pgfqpoint{2.816959in}{1.857713in}}{\pgfqpoint{2.813687in}{1.849813in}}{\pgfqpoint{2.813687in}{1.841577in}}%
\pgfpathcurveto{\pgfqpoint{2.813687in}{1.833340in}}{\pgfqpoint{2.816959in}{1.825440in}}{\pgfqpoint{2.822783in}{1.819616in}}%
\pgfpathcurveto{\pgfqpoint{2.828607in}{1.813792in}}{\pgfqpoint{2.836507in}{1.810520in}}{\pgfqpoint{2.844743in}{1.810520in}}%
\pgfpathclose%
\pgfusepath{stroke,fill}%
\end{pgfscope}%
\begin{pgfscope}%
\pgfpathrectangle{\pgfqpoint{0.100000in}{0.220728in}}{\pgfqpoint{3.696000in}{3.696000in}}%
\pgfusepath{clip}%
\pgfsetbuttcap%
\pgfsetroundjoin%
\definecolor{currentfill}{rgb}{0.121569,0.466667,0.705882}%
\pgfsetfillcolor{currentfill}%
\pgfsetfillopacity{0.864452}%
\pgfsetlinewidth{1.003750pt}%
\definecolor{currentstroke}{rgb}{0.121569,0.466667,0.705882}%
\pgfsetstrokecolor{currentstroke}%
\pgfsetstrokeopacity{0.864452}%
\pgfsetdash{}{0pt}%
\pgfpathmoveto{\pgfqpoint{2.839321in}{1.801600in}}%
\pgfpathcurveto{\pgfqpoint{2.847557in}{1.801600in}}{\pgfqpoint{2.855457in}{1.804872in}}{\pgfqpoint{2.861281in}{1.810696in}}%
\pgfpathcurveto{\pgfqpoint{2.867105in}{1.816520in}}{\pgfqpoint{2.870377in}{1.824420in}}{\pgfqpoint{2.870377in}{1.832656in}}%
\pgfpathcurveto{\pgfqpoint{2.870377in}{1.840892in}}{\pgfqpoint{2.867105in}{1.848792in}}{\pgfqpoint{2.861281in}{1.854616in}}%
\pgfpathcurveto{\pgfqpoint{2.855457in}{1.860440in}}{\pgfqpoint{2.847557in}{1.863713in}}{\pgfqpoint{2.839321in}{1.863713in}}%
\pgfpathcurveto{\pgfqpoint{2.831084in}{1.863713in}}{\pgfqpoint{2.823184in}{1.860440in}}{\pgfqpoint{2.817360in}{1.854616in}}%
\pgfpathcurveto{\pgfqpoint{2.811536in}{1.848792in}}{\pgfqpoint{2.808264in}{1.840892in}}{\pgfqpoint{2.808264in}{1.832656in}}%
\pgfpathcurveto{\pgfqpoint{2.808264in}{1.824420in}}{\pgfqpoint{2.811536in}{1.816520in}}{\pgfqpoint{2.817360in}{1.810696in}}%
\pgfpathcurveto{\pgfqpoint{2.823184in}{1.804872in}}{\pgfqpoint{2.831084in}{1.801600in}}{\pgfqpoint{2.839321in}{1.801600in}}%
\pgfpathclose%
\pgfusepath{stroke,fill}%
\end{pgfscope}%
\begin{pgfscope}%
\pgfpathrectangle{\pgfqpoint{0.100000in}{0.220728in}}{\pgfqpoint{3.696000in}{3.696000in}}%
\pgfusepath{clip}%
\pgfsetbuttcap%
\pgfsetroundjoin%
\definecolor{currentfill}{rgb}{0.121569,0.466667,0.705882}%
\pgfsetfillcolor{currentfill}%
\pgfsetfillopacity{0.864880}%
\pgfsetlinewidth{1.003750pt}%
\definecolor{currentstroke}{rgb}{0.121569,0.466667,0.705882}%
\pgfsetstrokecolor{currentstroke}%
\pgfsetstrokeopacity{0.864880}%
\pgfsetdash{}{0pt}%
\pgfpathmoveto{\pgfqpoint{1.675882in}{0.984425in}}%
\pgfpathcurveto{\pgfqpoint{1.684118in}{0.984425in}}{\pgfqpoint{1.692018in}{0.987698in}}{\pgfqpoint{1.697842in}{0.993521in}}%
\pgfpathcurveto{\pgfqpoint{1.703666in}{0.999345in}}{\pgfqpoint{1.706939in}{1.007245in}}{\pgfqpoint{1.706939in}{1.015482in}}%
\pgfpathcurveto{\pgfqpoint{1.706939in}{1.023718in}}{\pgfqpoint{1.703666in}{1.031618in}}{\pgfqpoint{1.697842in}{1.037442in}}%
\pgfpathcurveto{\pgfqpoint{1.692018in}{1.043266in}}{\pgfqpoint{1.684118in}{1.046538in}}{\pgfqpoint{1.675882in}{1.046538in}}%
\pgfpathcurveto{\pgfqpoint{1.667646in}{1.046538in}}{\pgfqpoint{1.659746in}{1.043266in}}{\pgfqpoint{1.653922in}{1.037442in}}%
\pgfpathcurveto{\pgfqpoint{1.648098in}{1.031618in}}{\pgfqpoint{1.644826in}{1.023718in}}{\pgfqpoint{1.644826in}{1.015482in}}%
\pgfpathcurveto{\pgfqpoint{1.644826in}{1.007245in}}{\pgfqpoint{1.648098in}{0.999345in}}{\pgfqpoint{1.653922in}{0.993521in}}%
\pgfpathcurveto{\pgfqpoint{1.659746in}{0.987698in}}{\pgfqpoint{1.667646in}{0.984425in}}{\pgfqpoint{1.675882in}{0.984425in}}%
\pgfpathclose%
\pgfusepath{stroke,fill}%
\end{pgfscope}%
\begin{pgfscope}%
\pgfpathrectangle{\pgfqpoint{0.100000in}{0.220728in}}{\pgfqpoint{3.696000in}{3.696000in}}%
\pgfusepath{clip}%
\pgfsetbuttcap%
\pgfsetroundjoin%
\definecolor{currentfill}{rgb}{0.121569,0.466667,0.705882}%
\pgfsetfillcolor{currentfill}%
\pgfsetfillopacity{0.865855}%
\pgfsetlinewidth{1.003750pt}%
\definecolor{currentstroke}{rgb}{0.121569,0.466667,0.705882}%
\pgfsetstrokecolor{currentstroke}%
\pgfsetstrokeopacity{0.865855}%
\pgfsetdash{}{0pt}%
\pgfpathmoveto{\pgfqpoint{2.834834in}{1.791722in}}%
\pgfpathcurveto{\pgfqpoint{2.843071in}{1.791722in}}{\pgfqpoint{2.850971in}{1.794995in}}{\pgfqpoint{2.856795in}{1.800818in}}%
\pgfpathcurveto{\pgfqpoint{2.862619in}{1.806642in}}{\pgfqpoint{2.865891in}{1.814542in}}{\pgfqpoint{2.865891in}{1.822779in}}%
\pgfpathcurveto{\pgfqpoint{2.865891in}{1.831015in}}{\pgfqpoint{2.862619in}{1.838915in}}{\pgfqpoint{2.856795in}{1.844739in}}%
\pgfpathcurveto{\pgfqpoint{2.850971in}{1.850563in}}{\pgfqpoint{2.843071in}{1.853835in}}{\pgfqpoint{2.834834in}{1.853835in}}%
\pgfpathcurveto{\pgfqpoint{2.826598in}{1.853835in}}{\pgfqpoint{2.818698in}{1.850563in}}{\pgfqpoint{2.812874in}{1.844739in}}%
\pgfpathcurveto{\pgfqpoint{2.807050in}{1.838915in}}{\pgfqpoint{2.803778in}{1.831015in}}{\pgfqpoint{2.803778in}{1.822779in}}%
\pgfpathcurveto{\pgfqpoint{2.803778in}{1.814542in}}{\pgfqpoint{2.807050in}{1.806642in}}{\pgfqpoint{2.812874in}{1.800818in}}%
\pgfpathcurveto{\pgfqpoint{2.818698in}{1.794995in}}{\pgfqpoint{2.826598in}{1.791722in}}{\pgfqpoint{2.834834in}{1.791722in}}%
\pgfpathclose%
\pgfusepath{stroke,fill}%
\end{pgfscope}%
\begin{pgfscope}%
\pgfpathrectangle{\pgfqpoint{0.100000in}{0.220728in}}{\pgfqpoint{3.696000in}{3.696000in}}%
\pgfusepath{clip}%
\pgfsetbuttcap%
\pgfsetroundjoin%
\definecolor{currentfill}{rgb}{0.121569,0.466667,0.705882}%
\pgfsetfillcolor{currentfill}%
\pgfsetfillopacity{0.867293}%
\pgfsetlinewidth{1.003750pt}%
\definecolor{currentstroke}{rgb}{0.121569,0.466667,0.705882}%
\pgfsetstrokecolor{currentstroke}%
\pgfsetstrokeopacity{0.867293}%
\pgfsetdash{}{0pt}%
\pgfpathmoveto{\pgfqpoint{2.831201in}{1.779644in}}%
\pgfpathcurveto{\pgfqpoint{2.839437in}{1.779644in}}{\pgfqpoint{2.847337in}{1.782916in}}{\pgfqpoint{2.853161in}{1.788740in}}%
\pgfpathcurveto{\pgfqpoint{2.858985in}{1.794564in}}{\pgfqpoint{2.862258in}{1.802464in}}{\pgfqpoint{2.862258in}{1.810701in}}%
\pgfpathcurveto{\pgfqpoint{2.862258in}{1.818937in}}{\pgfqpoint{2.858985in}{1.826837in}}{\pgfqpoint{2.853161in}{1.832661in}}%
\pgfpathcurveto{\pgfqpoint{2.847337in}{1.838485in}}{\pgfqpoint{2.839437in}{1.841757in}}{\pgfqpoint{2.831201in}{1.841757in}}%
\pgfpathcurveto{\pgfqpoint{2.822965in}{1.841757in}}{\pgfqpoint{2.815065in}{1.838485in}}{\pgfqpoint{2.809241in}{1.832661in}}%
\pgfpathcurveto{\pgfqpoint{2.803417in}{1.826837in}}{\pgfqpoint{2.800145in}{1.818937in}}{\pgfqpoint{2.800145in}{1.810701in}}%
\pgfpathcurveto{\pgfqpoint{2.800145in}{1.802464in}}{\pgfqpoint{2.803417in}{1.794564in}}{\pgfqpoint{2.809241in}{1.788740in}}%
\pgfpathcurveto{\pgfqpoint{2.815065in}{1.782916in}}{\pgfqpoint{2.822965in}{1.779644in}}{\pgfqpoint{2.831201in}{1.779644in}}%
\pgfpathclose%
\pgfusepath{stroke,fill}%
\end{pgfscope}%
\begin{pgfscope}%
\pgfpathrectangle{\pgfqpoint{0.100000in}{0.220728in}}{\pgfqpoint{3.696000in}{3.696000in}}%
\pgfusepath{clip}%
\pgfsetbuttcap%
\pgfsetroundjoin%
\definecolor{currentfill}{rgb}{0.121569,0.466667,0.705882}%
\pgfsetfillcolor{currentfill}%
\pgfsetfillopacity{0.867727}%
\pgfsetlinewidth{1.003750pt}%
\definecolor{currentstroke}{rgb}{0.121569,0.466667,0.705882}%
\pgfsetstrokecolor{currentstroke}%
\pgfsetstrokeopacity{0.867727}%
\pgfsetdash{}{0pt}%
\pgfpathmoveto{\pgfqpoint{2.827552in}{1.773791in}}%
\pgfpathcurveto{\pgfqpoint{2.835789in}{1.773791in}}{\pgfqpoint{2.843689in}{1.777063in}}{\pgfqpoint{2.849513in}{1.782887in}}%
\pgfpathcurveto{\pgfqpoint{2.855337in}{1.788711in}}{\pgfqpoint{2.858609in}{1.796611in}}{\pgfqpoint{2.858609in}{1.804847in}}%
\pgfpathcurveto{\pgfqpoint{2.858609in}{1.813083in}}{\pgfqpoint{2.855337in}{1.820983in}}{\pgfqpoint{2.849513in}{1.826807in}}%
\pgfpathcurveto{\pgfqpoint{2.843689in}{1.832631in}}{\pgfqpoint{2.835789in}{1.835904in}}{\pgfqpoint{2.827552in}{1.835904in}}%
\pgfpathcurveto{\pgfqpoint{2.819316in}{1.835904in}}{\pgfqpoint{2.811416in}{1.832631in}}{\pgfqpoint{2.805592in}{1.826807in}}%
\pgfpathcurveto{\pgfqpoint{2.799768in}{1.820983in}}{\pgfqpoint{2.796496in}{1.813083in}}{\pgfqpoint{2.796496in}{1.804847in}}%
\pgfpathcurveto{\pgfqpoint{2.796496in}{1.796611in}}{\pgfqpoint{2.799768in}{1.788711in}}{\pgfqpoint{2.805592in}{1.782887in}}%
\pgfpathcurveto{\pgfqpoint{2.811416in}{1.777063in}}{\pgfqpoint{2.819316in}{1.773791in}}{\pgfqpoint{2.827552in}{1.773791in}}%
\pgfpathclose%
\pgfusepath{stroke,fill}%
\end{pgfscope}%
\begin{pgfscope}%
\pgfpathrectangle{\pgfqpoint{0.100000in}{0.220728in}}{\pgfqpoint{3.696000in}{3.696000in}}%
\pgfusepath{clip}%
\pgfsetbuttcap%
\pgfsetroundjoin%
\definecolor{currentfill}{rgb}{0.121569,0.466667,0.705882}%
\pgfsetfillcolor{currentfill}%
\pgfsetfillopacity{0.869194}%
\pgfsetlinewidth{1.003750pt}%
\definecolor{currentstroke}{rgb}{0.121569,0.466667,0.705882}%
\pgfsetstrokecolor{currentstroke}%
\pgfsetstrokeopacity{0.869194}%
\pgfsetdash{}{0pt}%
\pgfpathmoveto{\pgfqpoint{2.824086in}{1.764254in}}%
\pgfpathcurveto{\pgfqpoint{2.832322in}{1.764254in}}{\pgfqpoint{2.840222in}{1.767527in}}{\pgfqpoint{2.846046in}{1.773351in}}%
\pgfpathcurveto{\pgfqpoint{2.851870in}{1.779175in}}{\pgfqpoint{2.855143in}{1.787075in}}{\pgfqpoint{2.855143in}{1.795311in}}%
\pgfpathcurveto{\pgfqpoint{2.855143in}{1.803547in}}{\pgfqpoint{2.851870in}{1.811447in}}{\pgfqpoint{2.846046in}{1.817271in}}%
\pgfpathcurveto{\pgfqpoint{2.840222in}{1.823095in}}{\pgfqpoint{2.832322in}{1.826367in}}{\pgfqpoint{2.824086in}{1.826367in}}%
\pgfpathcurveto{\pgfqpoint{2.815850in}{1.826367in}}{\pgfqpoint{2.807950in}{1.823095in}}{\pgfqpoint{2.802126in}{1.817271in}}%
\pgfpathcurveto{\pgfqpoint{2.796302in}{1.811447in}}{\pgfqpoint{2.793030in}{1.803547in}}{\pgfqpoint{2.793030in}{1.795311in}}%
\pgfpathcurveto{\pgfqpoint{2.793030in}{1.787075in}}{\pgfqpoint{2.796302in}{1.779175in}}{\pgfqpoint{2.802126in}{1.773351in}}%
\pgfpathcurveto{\pgfqpoint{2.807950in}{1.767527in}}{\pgfqpoint{2.815850in}{1.764254in}}{\pgfqpoint{2.824086in}{1.764254in}}%
\pgfpathclose%
\pgfusepath{stroke,fill}%
\end{pgfscope}%
\begin{pgfscope}%
\pgfpathrectangle{\pgfqpoint{0.100000in}{0.220728in}}{\pgfqpoint{3.696000in}{3.696000in}}%
\pgfusepath{clip}%
\pgfsetbuttcap%
\pgfsetroundjoin%
\definecolor{currentfill}{rgb}{0.121569,0.466667,0.705882}%
\pgfsetfillcolor{currentfill}%
\pgfsetfillopacity{0.870090}%
\pgfsetlinewidth{1.003750pt}%
\definecolor{currentstroke}{rgb}{0.121569,0.466667,0.705882}%
\pgfsetstrokecolor{currentstroke}%
\pgfsetstrokeopacity{0.870090}%
\pgfsetdash{}{0pt}%
\pgfpathmoveto{\pgfqpoint{1.695629in}{0.978773in}}%
\pgfpathcurveto{\pgfqpoint{1.703865in}{0.978773in}}{\pgfqpoint{1.711765in}{0.982045in}}{\pgfqpoint{1.717589in}{0.987869in}}%
\pgfpathcurveto{\pgfqpoint{1.723413in}{0.993693in}}{\pgfqpoint{1.726685in}{1.001593in}}{\pgfqpoint{1.726685in}{1.009829in}}%
\pgfpathcurveto{\pgfqpoint{1.726685in}{1.018065in}}{\pgfqpoint{1.723413in}{1.025965in}}{\pgfqpoint{1.717589in}{1.031789in}}%
\pgfpathcurveto{\pgfqpoint{1.711765in}{1.037613in}}{\pgfqpoint{1.703865in}{1.040886in}}{\pgfqpoint{1.695629in}{1.040886in}}%
\pgfpathcurveto{\pgfqpoint{1.687393in}{1.040886in}}{\pgfqpoint{1.679493in}{1.037613in}}{\pgfqpoint{1.673669in}{1.031789in}}%
\pgfpathcurveto{\pgfqpoint{1.667845in}{1.025965in}}{\pgfqpoint{1.664572in}{1.018065in}}{\pgfqpoint{1.664572in}{1.009829in}}%
\pgfpathcurveto{\pgfqpoint{1.664572in}{1.001593in}}{\pgfqpoint{1.667845in}{0.993693in}}{\pgfqpoint{1.673669in}{0.987869in}}%
\pgfpathcurveto{\pgfqpoint{1.679493in}{0.982045in}}{\pgfqpoint{1.687393in}{0.978773in}}{\pgfqpoint{1.695629in}{0.978773in}}%
\pgfpathclose%
\pgfusepath{stroke,fill}%
\end{pgfscope}%
\begin{pgfscope}%
\pgfpathrectangle{\pgfqpoint{0.100000in}{0.220728in}}{\pgfqpoint{3.696000in}{3.696000in}}%
\pgfusepath{clip}%
\pgfsetbuttcap%
\pgfsetroundjoin%
\definecolor{currentfill}{rgb}{0.121569,0.466667,0.705882}%
\pgfsetfillcolor{currentfill}%
\pgfsetfillopacity{0.870334}%
\pgfsetlinewidth{1.003750pt}%
\definecolor{currentstroke}{rgb}{0.121569,0.466667,0.705882}%
\pgfsetstrokecolor{currentstroke}%
\pgfsetstrokeopacity{0.870334}%
\pgfsetdash{}{0pt}%
\pgfpathmoveto{\pgfqpoint{2.818895in}{1.753733in}}%
\pgfpathcurveto{\pgfqpoint{2.827132in}{1.753733in}}{\pgfqpoint{2.835032in}{1.757006in}}{\pgfqpoint{2.840856in}{1.762830in}}%
\pgfpathcurveto{\pgfqpoint{2.846680in}{1.768653in}}{\pgfqpoint{2.849952in}{1.776553in}}{\pgfqpoint{2.849952in}{1.784790in}}%
\pgfpathcurveto{\pgfqpoint{2.849952in}{1.793026in}}{\pgfqpoint{2.846680in}{1.800926in}}{\pgfqpoint{2.840856in}{1.806750in}}%
\pgfpathcurveto{\pgfqpoint{2.835032in}{1.812574in}}{\pgfqpoint{2.827132in}{1.815846in}}{\pgfqpoint{2.818895in}{1.815846in}}%
\pgfpathcurveto{\pgfqpoint{2.810659in}{1.815846in}}{\pgfqpoint{2.802759in}{1.812574in}}{\pgfqpoint{2.796935in}{1.806750in}}%
\pgfpathcurveto{\pgfqpoint{2.791111in}{1.800926in}}{\pgfqpoint{2.787839in}{1.793026in}}{\pgfqpoint{2.787839in}{1.784790in}}%
\pgfpathcurveto{\pgfqpoint{2.787839in}{1.776553in}}{\pgfqpoint{2.791111in}{1.768653in}}{\pgfqpoint{2.796935in}{1.762830in}}%
\pgfpathcurveto{\pgfqpoint{2.802759in}{1.757006in}}{\pgfqpoint{2.810659in}{1.753733in}}{\pgfqpoint{2.818895in}{1.753733in}}%
\pgfpathclose%
\pgfusepath{stroke,fill}%
\end{pgfscope}%
\begin{pgfscope}%
\pgfpathrectangle{\pgfqpoint{0.100000in}{0.220728in}}{\pgfqpoint{3.696000in}{3.696000in}}%
\pgfusepath{clip}%
\pgfsetbuttcap%
\pgfsetroundjoin%
\definecolor{currentfill}{rgb}{0.121569,0.466667,0.705882}%
\pgfsetfillcolor{currentfill}%
\pgfsetfillopacity{0.870902}%
\pgfsetlinewidth{1.003750pt}%
\definecolor{currentstroke}{rgb}{0.121569,0.466667,0.705882}%
\pgfsetstrokecolor{currentstroke}%
\pgfsetstrokeopacity{0.870902}%
\pgfsetdash{}{0pt}%
\pgfpathmoveto{\pgfqpoint{2.815422in}{1.748806in}}%
\pgfpathcurveto{\pgfqpoint{2.823658in}{1.748806in}}{\pgfqpoint{2.831558in}{1.752079in}}{\pgfqpoint{2.837382in}{1.757903in}}%
\pgfpathcurveto{\pgfqpoint{2.843206in}{1.763727in}}{\pgfqpoint{2.846478in}{1.771627in}}{\pgfqpoint{2.846478in}{1.779863in}}%
\pgfpathcurveto{\pgfqpoint{2.846478in}{1.788099in}}{\pgfqpoint{2.843206in}{1.795999in}}{\pgfqpoint{2.837382in}{1.801823in}}%
\pgfpathcurveto{\pgfqpoint{2.831558in}{1.807647in}}{\pgfqpoint{2.823658in}{1.810919in}}{\pgfqpoint{2.815422in}{1.810919in}}%
\pgfpathcurveto{\pgfqpoint{2.807186in}{1.810919in}}{\pgfqpoint{2.799286in}{1.807647in}}{\pgfqpoint{2.793462in}{1.801823in}}%
\pgfpathcurveto{\pgfqpoint{2.787638in}{1.795999in}}{\pgfqpoint{2.784365in}{1.788099in}}{\pgfqpoint{2.784365in}{1.779863in}}%
\pgfpathcurveto{\pgfqpoint{2.784365in}{1.771627in}}{\pgfqpoint{2.787638in}{1.763727in}}{\pgfqpoint{2.793462in}{1.757903in}}%
\pgfpathcurveto{\pgfqpoint{2.799286in}{1.752079in}}{\pgfqpoint{2.807186in}{1.748806in}}{\pgfqpoint{2.815422in}{1.748806in}}%
\pgfpathclose%
\pgfusepath{stroke,fill}%
\end{pgfscope}%
\begin{pgfscope}%
\pgfpathrectangle{\pgfqpoint{0.100000in}{0.220728in}}{\pgfqpoint{3.696000in}{3.696000in}}%
\pgfusepath{clip}%
\pgfsetbuttcap%
\pgfsetroundjoin%
\definecolor{currentfill}{rgb}{0.121569,0.466667,0.705882}%
\pgfsetfillcolor{currentfill}%
\pgfsetfillopacity{0.872350}%
\pgfsetlinewidth{1.003750pt}%
\definecolor{currentstroke}{rgb}{0.121569,0.466667,0.705882}%
\pgfsetstrokecolor{currentstroke}%
\pgfsetstrokeopacity{0.872350}%
\pgfsetdash{}{0pt}%
\pgfpathmoveto{\pgfqpoint{2.812394in}{1.738709in}}%
\pgfpathcurveto{\pgfqpoint{2.820630in}{1.738709in}}{\pgfqpoint{2.828530in}{1.741981in}}{\pgfqpoint{2.834354in}{1.747805in}}%
\pgfpathcurveto{\pgfqpoint{2.840178in}{1.753629in}}{\pgfqpoint{2.843450in}{1.761529in}}{\pgfqpoint{2.843450in}{1.769765in}}%
\pgfpathcurveto{\pgfqpoint{2.843450in}{1.778001in}}{\pgfqpoint{2.840178in}{1.785901in}}{\pgfqpoint{2.834354in}{1.791725in}}%
\pgfpathcurveto{\pgfqpoint{2.828530in}{1.797549in}}{\pgfqpoint{2.820630in}{1.800822in}}{\pgfqpoint{2.812394in}{1.800822in}}%
\pgfpathcurveto{\pgfqpoint{2.804158in}{1.800822in}}{\pgfqpoint{2.796258in}{1.797549in}}{\pgfqpoint{2.790434in}{1.791725in}}%
\pgfpathcurveto{\pgfqpoint{2.784610in}{1.785901in}}{\pgfqpoint{2.781337in}{1.778001in}}{\pgfqpoint{2.781337in}{1.769765in}}%
\pgfpathcurveto{\pgfqpoint{2.781337in}{1.761529in}}{\pgfqpoint{2.784610in}{1.753629in}}{\pgfqpoint{2.790434in}{1.747805in}}%
\pgfpathcurveto{\pgfqpoint{2.796258in}{1.741981in}}{\pgfqpoint{2.804158in}{1.738709in}}{\pgfqpoint{2.812394in}{1.738709in}}%
\pgfpathclose%
\pgfusepath{stroke,fill}%
\end{pgfscope}%
\begin{pgfscope}%
\pgfpathrectangle{\pgfqpoint{0.100000in}{0.220728in}}{\pgfqpoint{3.696000in}{3.696000in}}%
\pgfusepath{clip}%
\pgfsetbuttcap%
\pgfsetroundjoin%
\definecolor{currentfill}{rgb}{0.121569,0.466667,0.705882}%
\pgfsetfillcolor{currentfill}%
\pgfsetfillopacity{0.873549}%
\pgfsetlinewidth{1.003750pt}%
\definecolor{currentstroke}{rgb}{0.121569,0.466667,0.705882}%
\pgfsetstrokecolor{currentstroke}%
\pgfsetstrokeopacity{0.873549}%
\pgfsetdash{}{0pt}%
\pgfpathmoveto{\pgfqpoint{2.807622in}{1.727826in}}%
\pgfpathcurveto{\pgfqpoint{2.815859in}{1.727826in}}{\pgfqpoint{2.823759in}{1.731098in}}{\pgfqpoint{2.829583in}{1.736922in}}%
\pgfpathcurveto{\pgfqpoint{2.835407in}{1.742746in}}{\pgfqpoint{2.838679in}{1.750646in}}{\pgfqpoint{2.838679in}{1.758882in}}%
\pgfpathcurveto{\pgfqpoint{2.838679in}{1.767118in}}{\pgfqpoint{2.835407in}{1.775019in}}{\pgfqpoint{2.829583in}{1.780842in}}%
\pgfpathcurveto{\pgfqpoint{2.823759in}{1.786666in}}{\pgfqpoint{2.815859in}{1.789939in}}{\pgfqpoint{2.807622in}{1.789939in}}%
\pgfpathcurveto{\pgfqpoint{2.799386in}{1.789939in}}{\pgfqpoint{2.791486in}{1.786666in}}{\pgfqpoint{2.785662in}{1.780842in}}%
\pgfpathcurveto{\pgfqpoint{2.779838in}{1.775019in}}{\pgfqpoint{2.776566in}{1.767118in}}{\pgfqpoint{2.776566in}{1.758882in}}%
\pgfpathcurveto{\pgfqpoint{2.776566in}{1.750646in}}{\pgfqpoint{2.779838in}{1.742746in}}{\pgfqpoint{2.785662in}{1.736922in}}%
\pgfpathcurveto{\pgfqpoint{2.791486in}{1.731098in}}{\pgfqpoint{2.799386in}{1.727826in}}{\pgfqpoint{2.807622in}{1.727826in}}%
\pgfpathclose%
\pgfusepath{stroke,fill}%
\end{pgfscope}%
\begin{pgfscope}%
\pgfpathrectangle{\pgfqpoint{0.100000in}{0.220728in}}{\pgfqpoint{3.696000in}{3.696000in}}%
\pgfusepath{clip}%
\pgfsetbuttcap%
\pgfsetroundjoin%
\definecolor{currentfill}{rgb}{0.121569,0.466667,0.705882}%
\pgfsetfillcolor{currentfill}%
\pgfsetfillopacity{0.874104}%
\pgfsetlinewidth{1.003750pt}%
\definecolor{currentstroke}{rgb}{0.121569,0.466667,0.705882}%
\pgfsetstrokecolor{currentstroke}%
\pgfsetstrokeopacity{0.874104}%
\pgfsetdash{}{0pt}%
\pgfpathmoveto{\pgfqpoint{1.713253in}{0.972407in}}%
\pgfpathcurveto{\pgfqpoint{1.721489in}{0.972407in}}{\pgfqpoint{1.729389in}{0.975679in}}{\pgfqpoint{1.735213in}{0.981503in}}%
\pgfpathcurveto{\pgfqpoint{1.741037in}{0.987327in}}{\pgfqpoint{1.744309in}{0.995227in}}{\pgfqpoint{1.744309in}{1.003463in}}%
\pgfpathcurveto{\pgfqpoint{1.744309in}{1.011700in}}{\pgfqpoint{1.741037in}{1.019600in}}{\pgfqpoint{1.735213in}{1.025424in}}%
\pgfpathcurveto{\pgfqpoint{1.729389in}{1.031248in}}{\pgfqpoint{1.721489in}{1.034520in}}{\pgfqpoint{1.713253in}{1.034520in}}%
\pgfpathcurveto{\pgfqpoint{1.705017in}{1.034520in}}{\pgfqpoint{1.697117in}{1.031248in}}{\pgfqpoint{1.691293in}{1.025424in}}%
\pgfpathcurveto{\pgfqpoint{1.685469in}{1.019600in}}{\pgfqpoint{1.682196in}{1.011700in}}{\pgfqpoint{1.682196in}{1.003463in}}%
\pgfpathcurveto{\pgfqpoint{1.682196in}{0.995227in}}{\pgfqpoint{1.685469in}{0.987327in}}{\pgfqpoint{1.691293in}{0.981503in}}%
\pgfpathcurveto{\pgfqpoint{1.697117in}{0.975679in}}{\pgfqpoint{1.705017in}{0.972407in}}{\pgfqpoint{1.713253in}{0.972407in}}%
\pgfpathclose%
\pgfusepath{stroke,fill}%
\end{pgfscope}%
\begin{pgfscope}%
\pgfpathrectangle{\pgfqpoint{0.100000in}{0.220728in}}{\pgfqpoint{3.696000in}{3.696000in}}%
\pgfusepath{clip}%
\pgfsetbuttcap%
\pgfsetroundjoin%
\definecolor{currentfill}{rgb}{0.121569,0.466667,0.705882}%
\pgfsetfillcolor{currentfill}%
\pgfsetfillopacity{0.874718}%
\pgfsetlinewidth{1.003750pt}%
\definecolor{currentstroke}{rgb}{0.121569,0.466667,0.705882}%
\pgfsetstrokecolor{currentstroke}%
\pgfsetstrokeopacity{0.874718}%
\pgfsetdash{}{0pt}%
\pgfpathmoveto{\pgfqpoint{2.801161in}{1.717939in}}%
\pgfpathcurveto{\pgfqpoint{2.809398in}{1.717939in}}{\pgfqpoint{2.817298in}{1.721211in}}{\pgfqpoint{2.823122in}{1.727035in}}%
\pgfpathcurveto{\pgfqpoint{2.828946in}{1.732859in}}{\pgfqpoint{2.832218in}{1.740759in}}{\pgfqpoint{2.832218in}{1.748995in}}%
\pgfpathcurveto{\pgfqpoint{2.832218in}{1.757232in}}{\pgfqpoint{2.828946in}{1.765132in}}{\pgfqpoint{2.823122in}{1.770956in}}%
\pgfpathcurveto{\pgfqpoint{2.817298in}{1.776780in}}{\pgfqpoint{2.809398in}{1.780052in}}{\pgfqpoint{2.801161in}{1.780052in}}%
\pgfpathcurveto{\pgfqpoint{2.792925in}{1.780052in}}{\pgfqpoint{2.785025in}{1.776780in}}{\pgfqpoint{2.779201in}{1.770956in}}%
\pgfpathcurveto{\pgfqpoint{2.773377in}{1.765132in}}{\pgfqpoint{2.770105in}{1.757232in}}{\pgfqpoint{2.770105in}{1.748995in}}%
\pgfpathcurveto{\pgfqpoint{2.770105in}{1.740759in}}{\pgfqpoint{2.773377in}{1.732859in}}{\pgfqpoint{2.779201in}{1.727035in}}%
\pgfpathcurveto{\pgfqpoint{2.785025in}{1.721211in}}{\pgfqpoint{2.792925in}{1.717939in}}{\pgfqpoint{2.801161in}{1.717939in}}%
\pgfpathclose%
\pgfusepath{stroke,fill}%
\end{pgfscope}%
\begin{pgfscope}%
\pgfpathrectangle{\pgfqpoint{0.100000in}{0.220728in}}{\pgfqpoint{3.696000in}{3.696000in}}%
\pgfusepath{clip}%
\pgfsetbuttcap%
\pgfsetroundjoin%
\definecolor{currentfill}{rgb}{0.121569,0.466667,0.705882}%
\pgfsetfillcolor{currentfill}%
\pgfsetfillopacity{0.876669}%
\pgfsetlinewidth{1.003750pt}%
\definecolor{currentstroke}{rgb}{0.121569,0.466667,0.705882}%
\pgfsetstrokecolor{currentstroke}%
\pgfsetstrokeopacity{0.876669}%
\pgfsetdash{}{0pt}%
\pgfpathmoveto{\pgfqpoint{2.796504in}{1.699183in}}%
\pgfpathcurveto{\pgfqpoint{2.804741in}{1.699183in}}{\pgfqpoint{2.812641in}{1.702456in}}{\pgfqpoint{2.818465in}{1.708279in}}%
\pgfpathcurveto{\pgfqpoint{2.824289in}{1.714103in}}{\pgfqpoint{2.827561in}{1.722003in}}{\pgfqpoint{2.827561in}{1.730240in}}%
\pgfpathcurveto{\pgfqpoint{2.827561in}{1.738476in}}{\pgfqpoint{2.824289in}{1.746376in}}{\pgfqpoint{2.818465in}{1.752200in}}%
\pgfpathcurveto{\pgfqpoint{2.812641in}{1.758024in}}{\pgfqpoint{2.804741in}{1.761296in}}{\pgfqpoint{2.796504in}{1.761296in}}%
\pgfpathcurveto{\pgfqpoint{2.788268in}{1.761296in}}{\pgfqpoint{2.780368in}{1.758024in}}{\pgfqpoint{2.774544in}{1.752200in}}%
\pgfpathcurveto{\pgfqpoint{2.768720in}{1.746376in}}{\pgfqpoint{2.765448in}{1.738476in}}{\pgfqpoint{2.765448in}{1.730240in}}%
\pgfpathcurveto{\pgfqpoint{2.765448in}{1.722003in}}{\pgfqpoint{2.768720in}{1.714103in}}{\pgfqpoint{2.774544in}{1.708279in}}%
\pgfpathcurveto{\pgfqpoint{2.780368in}{1.702456in}}{\pgfqpoint{2.788268in}{1.699183in}}{\pgfqpoint{2.796504in}{1.699183in}}%
\pgfpathclose%
\pgfusepath{stroke,fill}%
\end{pgfscope}%
\begin{pgfscope}%
\pgfpathrectangle{\pgfqpoint{0.100000in}{0.220728in}}{\pgfqpoint{3.696000in}{3.696000in}}%
\pgfusepath{clip}%
\pgfsetbuttcap%
\pgfsetroundjoin%
\definecolor{currentfill}{rgb}{0.121569,0.466667,0.705882}%
\pgfsetfillcolor{currentfill}%
\pgfsetfillopacity{0.877511}%
\pgfsetlinewidth{1.003750pt}%
\definecolor{currentstroke}{rgb}{0.121569,0.466667,0.705882}%
\pgfsetstrokecolor{currentstroke}%
\pgfsetstrokeopacity{0.877511}%
\pgfsetdash{}{0pt}%
\pgfpathmoveto{\pgfqpoint{2.791881in}{1.690152in}}%
\pgfpathcurveto{\pgfqpoint{2.800117in}{1.690152in}}{\pgfqpoint{2.808018in}{1.693425in}}{\pgfqpoint{2.813841in}{1.699249in}}%
\pgfpathcurveto{\pgfqpoint{2.819665in}{1.705073in}}{\pgfqpoint{2.822938in}{1.712973in}}{\pgfqpoint{2.822938in}{1.721209in}}%
\pgfpathcurveto{\pgfqpoint{2.822938in}{1.729445in}}{\pgfqpoint{2.819665in}{1.737345in}}{\pgfqpoint{2.813841in}{1.743169in}}%
\pgfpathcurveto{\pgfqpoint{2.808018in}{1.748993in}}{\pgfqpoint{2.800117in}{1.752265in}}{\pgfqpoint{2.791881in}{1.752265in}}%
\pgfpathcurveto{\pgfqpoint{2.783645in}{1.752265in}}{\pgfqpoint{2.775745in}{1.748993in}}{\pgfqpoint{2.769921in}{1.743169in}}%
\pgfpathcurveto{\pgfqpoint{2.764097in}{1.737345in}}{\pgfqpoint{2.760825in}{1.729445in}}{\pgfqpoint{2.760825in}{1.721209in}}%
\pgfpathcurveto{\pgfqpoint{2.760825in}{1.712973in}}{\pgfqpoint{2.764097in}{1.705073in}}{\pgfqpoint{2.769921in}{1.699249in}}%
\pgfpathcurveto{\pgfqpoint{2.775745in}{1.693425in}}{\pgfqpoint{2.783645in}{1.690152in}}{\pgfqpoint{2.791881in}{1.690152in}}%
\pgfpathclose%
\pgfusepath{stroke,fill}%
\end{pgfscope}%
\begin{pgfscope}%
\pgfpathrectangle{\pgfqpoint{0.100000in}{0.220728in}}{\pgfqpoint{3.696000in}{3.696000in}}%
\pgfusepath{clip}%
\pgfsetbuttcap%
\pgfsetroundjoin%
\definecolor{currentfill}{rgb}{0.121569,0.466667,0.705882}%
\pgfsetfillcolor{currentfill}%
\pgfsetfillopacity{0.877529}%
\pgfsetlinewidth{1.003750pt}%
\definecolor{currentstroke}{rgb}{0.121569,0.466667,0.705882}%
\pgfsetstrokecolor{currentstroke}%
\pgfsetstrokeopacity{0.877529}%
\pgfsetdash{}{0pt}%
\pgfpathmoveto{\pgfqpoint{1.729261in}{0.969333in}}%
\pgfpathcurveto{\pgfqpoint{1.737498in}{0.969333in}}{\pgfqpoint{1.745398in}{0.972605in}}{\pgfqpoint{1.751222in}{0.978429in}}%
\pgfpathcurveto{\pgfqpoint{1.757046in}{0.984253in}}{\pgfqpoint{1.760318in}{0.992153in}}{\pgfqpoint{1.760318in}{1.000390in}}%
\pgfpathcurveto{\pgfqpoint{1.760318in}{1.008626in}}{\pgfqpoint{1.757046in}{1.016526in}}{\pgfqpoint{1.751222in}{1.022350in}}%
\pgfpathcurveto{\pgfqpoint{1.745398in}{1.028174in}}{\pgfqpoint{1.737498in}{1.031446in}}{\pgfqpoint{1.729261in}{1.031446in}}%
\pgfpathcurveto{\pgfqpoint{1.721025in}{1.031446in}}{\pgfqpoint{1.713125in}{1.028174in}}{\pgfqpoint{1.707301in}{1.022350in}}%
\pgfpathcurveto{\pgfqpoint{1.701477in}{1.016526in}}{\pgfqpoint{1.698205in}{1.008626in}}{\pgfqpoint{1.698205in}{1.000390in}}%
\pgfpathcurveto{\pgfqpoint{1.698205in}{0.992153in}}{\pgfqpoint{1.701477in}{0.984253in}}{\pgfqpoint{1.707301in}{0.978429in}}%
\pgfpathcurveto{\pgfqpoint{1.713125in}{0.972605in}}{\pgfqpoint{1.721025in}{0.969333in}}{\pgfqpoint{1.729261in}{0.969333in}}%
\pgfpathclose%
\pgfusepath{stroke,fill}%
\end{pgfscope}%
\begin{pgfscope}%
\pgfpathrectangle{\pgfqpoint{0.100000in}{0.220728in}}{\pgfqpoint{3.696000in}{3.696000in}}%
\pgfusepath{clip}%
\pgfsetbuttcap%
\pgfsetroundjoin%
\definecolor{currentfill}{rgb}{0.121569,0.466667,0.705882}%
\pgfsetfillcolor{currentfill}%
\pgfsetfillopacity{0.878588}%
\pgfsetlinewidth{1.003750pt}%
\definecolor{currentstroke}{rgb}{0.121569,0.466667,0.705882}%
\pgfsetstrokecolor{currentstroke}%
\pgfsetstrokeopacity{0.878588}%
\pgfsetdash{}{0pt}%
\pgfpathmoveto{\pgfqpoint{2.786520in}{1.681730in}}%
\pgfpathcurveto{\pgfqpoint{2.794756in}{1.681730in}}{\pgfqpoint{2.802657in}{1.685002in}}{\pgfqpoint{2.808480in}{1.690826in}}%
\pgfpathcurveto{\pgfqpoint{2.814304in}{1.696650in}}{\pgfqpoint{2.817577in}{1.704550in}}{\pgfqpoint{2.817577in}{1.712786in}}%
\pgfpathcurveto{\pgfqpoint{2.817577in}{1.721023in}}{\pgfqpoint{2.814304in}{1.728923in}}{\pgfqpoint{2.808480in}{1.734747in}}%
\pgfpathcurveto{\pgfqpoint{2.802657in}{1.740571in}}{\pgfqpoint{2.794756in}{1.743843in}}{\pgfqpoint{2.786520in}{1.743843in}}%
\pgfpathcurveto{\pgfqpoint{2.778284in}{1.743843in}}{\pgfqpoint{2.770384in}{1.740571in}}{\pgfqpoint{2.764560in}{1.734747in}}%
\pgfpathcurveto{\pgfqpoint{2.758736in}{1.728923in}}{\pgfqpoint{2.755464in}{1.721023in}}{\pgfqpoint{2.755464in}{1.712786in}}%
\pgfpathcurveto{\pgfqpoint{2.755464in}{1.704550in}}{\pgfqpoint{2.758736in}{1.696650in}}{\pgfqpoint{2.764560in}{1.690826in}}%
\pgfpathcurveto{\pgfqpoint{2.770384in}{1.685002in}}{\pgfqpoint{2.778284in}{1.681730in}}{\pgfqpoint{2.786520in}{1.681730in}}%
\pgfpathclose%
\pgfusepath{stroke,fill}%
\end{pgfscope}%
\begin{pgfscope}%
\pgfpathrectangle{\pgfqpoint{0.100000in}{0.220728in}}{\pgfqpoint{3.696000in}{3.696000in}}%
\pgfusepath{clip}%
\pgfsetbuttcap%
\pgfsetroundjoin%
\definecolor{currentfill}{rgb}{0.121569,0.466667,0.705882}%
\pgfsetfillcolor{currentfill}%
\pgfsetfillopacity{0.879802}%
\pgfsetlinewidth{1.003750pt}%
\definecolor{currentstroke}{rgb}{0.121569,0.466667,0.705882}%
\pgfsetstrokecolor{currentstroke}%
\pgfsetstrokeopacity{0.879802}%
\pgfsetdash{}{0pt}%
\pgfpathmoveto{\pgfqpoint{1.742954in}{0.961737in}}%
\pgfpathcurveto{\pgfqpoint{1.751190in}{0.961737in}}{\pgfqpoint{1.759090in}{0.965009in}}{\pgfqpoint{1.764914in}{0.970833in}}%
\pgfpathcurveto{\pgfqpoint{1.770738in}{0.976657in}}{\pgfqpoint{1.774010in}{0.984557in}}{\pgfqpoint{1.774010in}{0.992793in}}%
\pgfpathcurveto{\pgfqpoint{1.774010in}{1.001030in}}{\pgfqpoint{1.770738in}{1.008930in}}{\pgfqpoint{1.764914in}{1.014754in}}%
\pgfpathcurveto{\pgfqpoint{1.759090in}{1.020578in}}{\pgfqpoint{1.751190in}{1.023850in}}{\pgfqpoint{1.742954in}{1.023850in}}%
\pgfpathcurveto{\pgfqpoint{1.734718in}{1.023850in}}{\pgfqpoint{1.726818in}{1.020578in}}{\pgfqpoint{1.720994in}{1.014754in}}%
\pgfpathcurveto{\pgfqpoint{1.715170in}{1.008930in}}{\pgfqpoint{1.711897in}{1.001030in}}{\pgfqpoint{1.711897in}{0.992793in}}%
\pgfpathcurveto{\pgfqpoint{1.711897in}{0.984557in}}{\pgfqpoint{1.715170in}{0.976657in}}{\pgfqpoint{1.720994in}{0.970833in}}%
\pgfpathcurveto{\pgfqpoint{1.726818in}{0.965009in}}{\pgfqpoint{1.734718in}{0.961737in}}{\pgfqpoint{1.742954in}{0.961737in}}%
\pgfpathclose%
\pgfusepath{stroke,fill}%
\end{pgfscope}%
\begin{pgfscope}%
\pgfpathrectangle{\pgfqpoint{0.100000in}{0.220728in}}{\pgfqpoint{3.696000in}{3.696000in}}%
\pgfusepath{clip}%
\pgfsetbuttcap%
\pgfsetroundjoin%
\definecolor{currentfill}{rgb}{0.121569,0.466667,0.705882}%
\pgfsetfillcolor{currentfill}%
\pgfsetfillopacity{0.880199}%
\pgfsetlinewidth{1.003750pt}%
\definecolor{currentstroke}{rgb}{0.121569,0.466667,0.705882}%
\pgfsetstrokecolor{currentstroke}%
\pgfsetstrokeopacity{0.880199}%
\pgfsetdash{}{0pt}%
\pgfpathmoveto{\pgfqpoint{2.782990in}{1.666986in}}%
\pgfpathcurveto{\pgfqpoint{2.791226in}{1.666986in}}{\pgfqpoint{2.799127in}{1.670259in}}{\pgfqpoint{2.804950in}{1.676083in}}%
\pgfpathcurveto{\pgfqpoint{2.810774in}{1.681906in}}{\pgfqpoint{2.814047in}{1.689807in}}{\pgfqpoint{2.814047in}{1.698043in}}%
\pgfpathcurveto{\pgfqpoint{2.814047in}{1.706279in}}{\pgfqpoint{2.810774in}{1.714179in}}{\pgfqpoint{2.804950in}{1.720003in}}%
\pgfpathcurveto{\pgfqpoint{2.799127in}{1.725827in}}{\pgfqpoint{2.791226in}{1.729099in}}{\pgfqpoint{2.782990in}{1.729099in}}%
\pgfpathcurveto{\pgfqpoint{2.774754in}{1.729099in}}{\pgfqpoint{2.766854in}{1.725827in}}{\pgfqpoint{2.761030in}{1.720003in}}%
\pgfpathcurveto{\pgfqpoint{2.755206in}{1.714179in}}{\pgfqpoint{2.751934in}{1.706279in}}{\pgfqpoint{2.751934in}{1.698043in}}%
\pgfpathcurveto{\pgfqpoint{2.751934in}{1.689807in}}{\pgfqpoint{2.755206in}{1.681906in}}{\pgfqpoint{2.761030in}{1.676083in}}%
\pgfpathcurveto{\pgfqpoint{2.766854in}{1.670259in}}{\pgfqpoint{2.774754in}{1.666986in}}{\pgfqpoint{2.782990in}{1.666986in}}%
\pgfpathclose%
\pgfusepath{stroke,fill}%
\end{pgfscope}%
\begin{pgfscope}%
\pgfpathrectangle{\pgfqpoint{0.100000in}{0.220728in}}{\pgfqpoint{3.696000in}{3.696000in}}%
\pgfusepath{clip}%
\pgfsetbuttcap%
\pgfsetroundjoin%
\definecolor{currentfill}{rgb}{0.121569,0.466667,0.705882}%
\pgfsetfillcolor{currentfill}%
\pgfsetfillopacity{0.881758}%
\pgfsetlinewidth{1.003750pt}%
\definecolor{currentstroke}{rgb}{0.121569,0.466667,0.705882}%
\pgfsetstrokecolor{currentstroke}%
\pgfsetstrokeopacity{0.881758}%
\pgfsetdash{}{0pt}%
\pgfpathmoveto{\pgfqpoint{2.776483in}{1.652917in}}%
\pgfpathcurveto{\pgfqpoint{2.784720in}{1.652917in}}{\pgfqpoint{2.792620in}{1.656190in}}{\pgfqpoint{2.798444in}{1.662014in}}%
\pgfpathcurveto{\pgfqpoint{2.804267in}{1.667838in}}{\pgfqpoint{2.807540in}{1.675738in}}{\pgfqpoint{2.807540in}{1.683974in}}%
\pgfpathcurveto{\pgfqpoint{2.807540in}{1.692210in}}{\pgfqpoint{2.804267in}{1.700110in}}{\pgfqpoint{2.798444in}{1.705934in}}%
\pgfpathcurveto{\pgfqpoint{2.792620in}{1.711758in}}{\pgfqpoint{2.784720in}{1.715030in}}{\pgfqpoint{2.776483in}{1.715030in}}%
\pgfpathcurveto{\pgfqpoint{2.768247in}{1.715030in}}{\pgfqpoint{2.760347in}{1.711758in}}{\pgfqpoint{2.754523in}{1.705934in}}%
\pgfpathcurveto{\pgfqpoint{2.748699in}{1.700110in}}{\pgfqpoint{2.745427in}{1.692210in}}{\pgfqpoint{2.745427in}{1.683974in}}%
\pgfpathcurveto{\pgfqpoint{2.745427in}{1.675738in}}{\pgfqpoint{2.748699in}{1.667838in}}{\pgfqpoint{2.754523in}{1.662014in}}%
\pgfpathcurveto{\pgfqpoint{2.760347in}{1.656190in}}{\pgfqpoint{2.768247in}{1.652917in}}{\pgfqpoint{2.776483in}{1.652917in}}%
\pgfpathclose%
\pgfusepath{stroke,fill}%
\end{pgfscope}%
\begin{pgfscope}%
\pgfpathrectangle{\pgfqpoint{0.100000in}{0.220728in}}{\pgfqpoint{3.696000in}{3.696000in}}%
\pgfusepath{clip}%
\pgfsetbuttcap%
\pgfsetroundjoin%
\definecolor{currentfill}{rgb}{0.121569,0.466667,0.705882}%
\pgfsetfillcolor{currentfill}%
\pgfsetfillopacity{0.882789}%
\pgfsetlinewidth{1.003750pt}%
\definecolor{currentstroke}{rgb}{0.121569,0.466667,0.705882}%
\pgfsetstrokecolor{currentstroke}%
\pgfsetstrokeopacity{0.882789}%
\pgfsetdash{}{0pt}%
\pgfpathmoveto{\pgfqpoint{2.772662in}{1.646232in}}%
\pgfpathcurveto{\pgfqpoint{2.780898in}{1.646232in}}{\pgfqpoint{2.788798in}{1.649504in}}{\pgfqpoint{2.794622in}{1.655328in}}%
\pgfpathcurveto{\pgfqpoint{2.800446in}{1.661152in}}{\pgfqpoint{2.803718in}{1.669052in}}{\pgfqpoint{2.803718in}{1.677288in}}%
\pgfpathcurveto{\pgfqpoint{2.803718in}{1.685525in}}{\pgfqpoint{2.800446in}{1.693425in}}{\pgfqpoint{2.794622in}{1.699249in}}%
\pgfpathcurveto{\pgfqpoint{2.788798in}{1.705072in}}{\pgfqpoint{2.780898in}{1.708345in}}{\pgfqpoint{2.772662in}{1.708345in}}%
\pgfpathcurveto{\pgfqpoint{2.764425in}{1.708345in}}{\pgfqpoint{2.756525in}{1.705072in}}{\pgfqpoint{2.750701in}{1.699249in}}%
\pgfpathcurveto{\pgfqpoint{2.744877in}{1.693425in}}{\pgfqpoint{2.741605in}{1.685525in}}{\pgfqpoint{2.741605in}{1.677288in}}%
\pgfpathcurveto{\pgfqpoint{2.741605in}{1.669052in}}{\pgfqpoint{2.744877in}{1.661152in}}{\pgfqpoint{2.750701in}{1.655328in}}%
\pgfpathcurveto{\pgfqpoint{2.756525in}{1.649504in}}{\pgfqpoint{2.764425in}{1.646232in}}{\pgfqpoint{2.772662in}{1.646232in}}%
\pgfpathclose%
\pgfusepath{stroke,fill}%
\end{pgfscope}%
\begin{pgfscope}%
\pgfpathrectangle{\pgfqpoint{0.100000in}{0.220728in}}{\pgfqpoint{3.696000in}{3.696000in}}%
\pgfusepath{clip}%
\pgfsetbuttcap%
\pgfsetroundjoin%
\definecolor{currentfill}{rgb}{0.121569,0.466667,0.705882}%
\pgfsetfillcolor{currentfill}%
\pgfsetfillopacity{0.883283}%
\pgfsetlinewidth{1.003750pt}%
\definecolor{currentstroke}{rgb}{0.121569,0.466667,0.705882}%
\pgfsetstrokecolor{currentstroke}%
\pgfsetstrokeopacity{0.883283}%
\pgfsetdash{}{0pt}%
\pgfpathmoveto{\pgfqpoint{2.771695in}{1.641186in}}%
\pgfpathcurveto{\pgfqpoint{2.779931in}{1.641186in}}{\pgfqpoint{2.787831in}{1.644458in}}{\pgfqpoint{2.793655in}{1.650282in}}%
\pgfpathcurveto{\pgfqpoint{2.799479in}{1.656106in}}{\pgfqpoint{2.802751in}{1.664006in}}{\pgfqpoint{2.802751in}{1.672242in}}%
\pgfpathcurveto{\pgfqpoint{2.802751in}{1.680479in}}{\pgfqpoint{2.799479in}{1.688379in}}{\pgfqpoint{2.793655in}{1.694203in}}%
\pgfpathcurveto{\pgfqpoint{2.787831in}{1.700026in}}{\pgfqpoint{2.779931in}{1.703299in}}{\pgfqpoint{2.771695in}{1.703299in}}%
\pgfpathcurveto{\pgfqpoint{2.763458in}{1.703299in}}{\pgfqpoint{2.755558in}{1.700026in}}{\pgfqpoint{2.749734in}{1.694203in}}%
\pgfpathcurveto{\pgfqpoint{2.743910in}{1.688379in}}{\pgfqpoint{2.740638in}{1.680479in}}{\pgfqpoint{2.740638in}{1.672242in}}%
\pgfpathcurveto{\pgfqpoint{2.740638in}{1.664006in}}{\pgfqpoint{2.743910in}{1.656106in}}{\pgfqpoint{2.749734in}{1.650282in}}%
\pgfpathcurveto{\pgfqpoint{2.755558in}{1.644458in}}{\pgfqpoint{2.763458in}{1.641186in}}{\pgfqpoint{2.771695in}{1.641186in}}%
\pgfpathclose%
\pgfusepath{stroke,fill}%
\end{pgfscope}%
\begin{pgfscope}%
\pgfpathrectangle{\pgfqpoint{0.100000in}{0.220728in}}{\pgfqpoint{3.696000in}{3.696000in}}%
\pgfusepath{clip}%
\pgfsetbuttcap%
\pgfsetroundjoin%
\definecolor{currentfill}{rgb}{0.121569,0.466667,0.705882}%
\pgfsetfillcolor{currentfill}%
\pgfsetfillopacity{0.884325}%
\pgfsetlinewidth{1.003750pt}%
\definecolor{currentstroke}{rgb}{0.121569,0.466667,0.705882}%
\pgfsetstrokecolor{currentstroke}%
\pgfsetstrokeopacity{0.884325}%
\pgfsetdash{}{0pt}%
\pgfpathmoveto{\pgfqpoint{2.767093in}{1.633269in}}%
\pgfpathcurveto{\pgfqpoint{2.775329in}{1.633269in}}{\pgfqpoint{2.783229in}{1.636541in}}{\pgfqpoint{2.789053in}{1.642365in}}%
\pgfpathcurveto{\pgfqpoint{2.794877in}{1.648189in}}{\pgfqpoint{2.798149in}{1.656089in}}{\pgfqpoint{2.798149in}{1.664326in}}%
\pgfpathcurveto{\pgfqpoint{2.798149in}{1.672562in}}{\pgfqpoint{2.794877in}{1.680462in}}{\pgfqpoint{2.789053in}{1.686286in}}%
\pgfpathcurveto{\pgfqpoint{2.783229in}{1.692110in}}{\pgfqpoint{2.775329in}{1.695382in}}{\pgfqpoint{2.767093in}{1.695382in}}%
\pgfpathcurveto{\pgfqpoint{2.758856in}{1.695382in}}{\pgfqpoint{2.750956in}{1.692110in}}{\pgfqpoint{2.745133in}{1.686286in}}%
\pgfpathcurveto{\pgfqpoint{2.739309in}{1.680462in}}{\pgfqpoint{2.736036in}{1.672562in}}{\pgfqpoint{2.736036in}{1.664326in}}%
\pgfpathcurveto{\pgfqpoint{2.736036in}{1.656089in}}{\pgfqpoint{2.739309in}{1.648189in}}{\pgfqpoint{2.745133in}{1.642365in}}%
\pgfpathcurveto{\pgfqpoint{2.750956in}{1.636541in}}{\pgfqpoint{2.758856in}{1.633269in}}{\pgfqpoint{2.767093in}{1.633269in}}%
\pgfpathclose%
\pgfusepath{stroke,fill}%
\end{pgfscope}%
\begin{pgfscope}%
\pgfpathrectangle{\pgfqpoint{0.100000in}{0.220728in}}{\pgfqpoint{3.696000in}{3.696000in}}%
\pgfusepath{clip}%
\pgfsetbuttcap%
\pgfsetroundjoin%
\definecolor{currentfill}{rgb}{0.121569,0.466667,0.705882}%
\pgfsetfillcolor{currentfill}%
\pgfsetfillopacity{0.884383}%
\pgfsetlinewidth{1.003750pt}%
\definecolor{currentstroke}{rgb}{0.121569,0.466667,0.705882}%
\pgfsetstrokecolor{currentstroke}%
\pgfsetstrokeopacity{0.884383}%
\pgfsetdash{}{0pt}%
\pgfpathmoveto{\pgfqpoint{1.767535in}{0.948566in}}%
\pgfpathcurveto{\pgfqpoint{1.775771in}{0.948566in}}{\pgfqpoint{1.783671in}{0.951838in}}{\pgfqpoint{1.789495in}{0.957662in}}%
\pgfpathcurveto{\pgfqpoint{1.795319in}{0.963486in}}{\pgfqpoint{1.798591in}{0.971386in}}{\pgfqpoint{1.798591in}{0.979622in}}%
\pgfpathcurveto{\pgfqpoint{1.798591in}{0.987859in}}{\pgfqpoint{1.795319in}{0.995759in}}{\pgfqpoint{1.789495in}{1.001583in}}%
\pgfpathcurveto{\pgfqpoint{1.783671in}{1.007407in}}{\pgfqpoint{1.775771in}{1.010679in}}{\pgfqpoint{1.767535in}{1.010679in}}%
\pgfpathcurveto{\pgfqpoint{1.759298in}{1.010679in}}{\pgfqpoint{1.751398in}{1.007407in}}{\pgfqpoint{1.745574in}{1.001583in}}%
\pgfpathcurveto{\pgfqpoint{1.739750in}{0.995759in}}{\pgfqpoint{1.736478in}{0.987859in}}{\pgfqpoint{1.736478in}{0.979622in}}%
\pgfpathcurveto{\pgfqpoint{1.736478in}{0.971386in}}{\pgfqpoint{1.739750in}{0.963486in}}{\pgfqpoint{1.745574in}{0.957662in}}%
\pgfpathcurveto{\pgfqpoint{1.751398in}{0.951838in}}{\pgfqpoint{1.759298in}{0.948566in}}{\pgfqpoint{1.767535in}{0.948566in}}%
\pgfpathclose%
\pgfusepath{stroke,fill}%
\end{pgfscope}%
\begin{pgfscope}%
\pgfpathrectangle{\pgfqpoint{0.100000in}{0.220728in}}{\pgfqpoint{3.696000in}{3.696000in}}%
\pgfusepath{clip}%
\pgfsetbuttcap%
\pgfsetroundjoin%
\definecolor{currentfill}{rgb}{0.121569,0.466667,0.705882}%
\pgfsetfillcolor{currentfill}%
\pgfsetfillopacity{0.885394}%
\pgfsetlinewidth{1.003750pt}%
\definecolor{currentstroke}{rgb}{0.121569,0.466667,0.705882}%
\pgfsetstrokecolor{currentstroke}%
\pgfsetstrokeopacity{0.885394}%
\pgfsetdash{}{0pt}%
\pgfpathmoveto{\pgfqpoint{2.763249in}{1.622251in}}%
\pgfpathcurveto{\pgfqpoint{2.771485in}{1.622251in}}{\pgfqpoint{2.779385in}{1.625523in}}{\pgfqpoint{2.785209in}{1.631347in}}%
\pgfpathcurveto{\pgfqpoint{2.791033in}{1.637171in}}{\pgfqpoint{2.794306in}{1.645071in}}{\pgfqpoint{2.794306in}{1.653307in}}%
\pgfpathcurveto{\pgfqpoint{2.794306in}{1.661544in}}{\pgfqpoint{2.791033in}{1.669444in}}{\pgfqpoint{2.785209in}{1.675268in}}%
\pgfpathcurveto{\pgfqpoint{2.779385in}{1.681092in}}{\pgfqpoint{2.771485in}{1.684364in}}{\pgfqpoint{2.763249in}{1.684364in}}%
\pgfpathcurveto{\pgfqpoint{2.755013in}{1.684364in}}{\pgfqpoint{2.747113in}{1.681092in}}{\pgfqpoint{2.741289in}{1.675268in}}%
\pgfpathcurveto{\pgfqpoint{2.735465in}{1.669444in}}{\pgfqpoint{2.732193in}{1.661544in}}{\pgfqpoint{2.732193in}{1.653307in}}%
\pgfpathcurveto{\pgfqpoint{2.732193in}{1.645071in}}{\pgfqpoint{2.735465in}{1.637171in}}{\pgfqpoint{2.741289in}{1.631347in}}%
\pgfpathcurveto{\pgfqpoint{2.747113in}{1.625523in}}{\pgfqpoint{2.755013in}{1.622251in}}{\pgfqpoint{2.763249in}{1.622251in}}%
\pgfpathclose%
\pgfusepath{stroke,fill}%
\end{pgfscope}%
\begin{pgfscope}%
\pgfpathrectangle{\pgfqpoint{0.100000in}{0.220728in}}{\pgfqpoint{3.696000in}{3.696000in}}%
\pgfusepath{clip}%
\pgfsetbuttcap%
\pgfsetroundjoin%
\definecolor{currentfill}{rgb}{0.121569,0.466667,0.705882}%
\pgfsetfillcolor{currentfill}%
\pgfsetfillopacity{0.886239}%
\pgfsetlinewidth{1.003750pt}%
\definecolor{currentstroke}{rgb}{0.121569,0.466667,0.705882}%
\pgfsetstrokecolor{currentstroke}%
\pgfsetstrokeopacity{0.886239}%
\pgfsetdash{}{0pt}%
\pgfpathmoveto{\pgfqpoint{2.761620in}{1.616807in}}%
\pgfpathcurveto{\pgfqpoint{2.769856in}{1.616807in}}{\pgfqpoint{2.777756in}{1.620079in}}{\pgfqpoint{2.783580in}{1.625903in}}%
\pgfpathcurveto{\pgfqpoint{2.789404in}{1.631727in}}{\pgfqpoint{2.792676in}{1.639627in}}{\pgfqpoint{2.792676in}{1.647863in}}%
\pgfpathcurveto{\pgfqpoint{2.792676in}{1.656100in}}{\pgfqpoint{2.789404in}{1.664000in}}{\pgfqpoint{2.783580in}{1.669824in}}%
\pgfpathcurveto{\pgfqpoint{2.777756in}{1.675648in}}{\pgfqpoint{2.769856in}{1.678920in}}{\pgfqpoint{2.761620in}{1.678920in}}%
\pgfpathcurveto{\pgfqpoint{2.753383in}{1.678920in}}{\pgfqpoint{2.745483in}{1.675648in}}{\pgfqpoint{2.739659in}{1.669824in}}%
\pgfpathcurveto{\pgfqpoint{2.733835in}{1.664000in}}{\pgfqpoint{2.730563in}{1.656100in}}{\pgfqpoint{2.730563in}{1.647863in}}%
\pgfpathcurveto{\pgfqpoint{2.730563in}{1.639627in}}{\pgfqpoint{2.733835in}{1.631727in}}{\pgfqpoint{2.739659in}{1.625903in}}%
\pgfpathcurveto{\pgfqpoint{2.745483in}{1.620079in}}{\pgfqpoint{2.753383in}{1.616807in}}{\pgfqpoint{2.761620in}{1.616807in}}%
\pgfpathclose%
\pgfusepath{stroke,fill}%
\end{pgfscope}%
\begin{pgfscope}%
\pgfpathrectangle{\pgfqpoint{0.100000in}{0.220728in}}{\pgfqpoint{3.696000in}{3.696000in}}%
\pgfusepath{clip}%
\pgfsetbuttcap%
\pgfsetroundjoin%
\definecolor{currentfill}{rgb}{0.121569,0.466667,0.705882}%
\pgfsetfillcolor{currentfill}%
\pgfsetfillopacity{0.887044}%
\pgfsetlinewidth{1.003750pt}%
\definecolor{currentstroke}{rgb}{0.121569,0.466667,0.705882}%
\pgfsetstrokecolor{currentstroke}%
\pgfsetstrokeopacity{0.887044}%
\pgfsetdash{}{0pt}%
\pgfpathmoveto{\pgfqpoint{2.757482in}{1.609556in}}%
\pgfpathcurveto{\pgfqpoint{2.765718in}{1.609556in}}{\pgfqpoint{2.773618in}{1.612828in}}{\pgfqpoint{2.779442in}{1.618652in}}%
\pgfpathcurveto{\pgfqpoint{2.785266in}{1.624476in}}{\pgfqpoint{2.788538in}{1.632376in}}{\pgfqpoint{2.788538in}{1.640612in}}%
\pgfpathcurveto{\pgfqpoint{2.788538in}{1.648849in}}{\pgfqpoint{2.785266in}{1.656749in}}{\pgfqpoint{2.779442in}{1.662573in}}%
\pgfpathcurveto{\pgfqpoint{2.773618in}{1.668396in}}{\pgfqpoint{2.765718in}{1.671669in}}{\pgfqpoint{2.757482in}{1.671669in}}%
\pgfpathcurveto{\pgfqpoint{2.749245in}{1.671669in}}{\pgfqpoint{2.741345in}{1.668396in}}{\pgfqpoint{2.735521in}{1.662573in}}%
\pgfpathcurveto{\pgfqpoint{2.729698in}{1.656749in}}{\pgfqpoint{2.726425in}{1.648849in}}{\pgfqpoint{2.726425in}{1.640612in}}%
\pgfpathcurveto{\pgfqpoint{2.726425in}{1.632376in}}{\pgfqpoint{2.729698in}{1.624476in}}{\pgfqpoint{2.735521in}{1.618652in}}%
\pgfpathcurveto{\pgfqpoint{2.741345in}{1.612828in}}{\pgfqpoint{2.749245in}{1.609556in}}{\pgfqpoint{2.757482in}{1.609556in}}%
\pgfpathclose%
\pgfusepath{stroke,fill}%
\end{pgfscope}%
\begin{pgfscope}%
\pgfpathrectangle{\pgfqpoint{0.100000in}{0.220728in}}{\pgfqpoint{3.696000in}{3.696000in}}%
\pgfusepath{clip}%
\pgfsetbuttcap%
\pgfsetroundjoin%
\definecolor{currentfill}{rgb}{0.121569,0.466667,0.705882}%
\pgfsetfillcolor{currentfill}%
\pgfsetfillopacity{0.888661}%
\pgfsetlinewidth{1.003750pt}%
\definecolor{currentstroke}{rgb}{0.121569,0.466667,0.705882}%
\pgfsetstrokecolor{currentstroke}%
\pgfsetstrokeopacity{0.888661}%
\pgfsetdash{}{0pt}%
\pgfpathmoveto{\pgfqpoint{2.754168in}{1.598454in}}%
\pgfpathcurveto{\pgfqpoint{2.762405in}{1.598454in}}{\pgfqpoint{2.770305in}{1.601726in}}{\pgfqpoint{2.776129in}{1.607550in}}%
\pgfpathcurveto{\pgfqpoint{2.781953in}{1.613374in}}{\pgfqpoint{2.785225in}{1.621274in}}{\pgfqpoint{2.785225in}{1.629510in}}%
\pgfpathcurveto{\pgfqpoint{2.785225in}{1.637746in}}{\pgfqpoint{2.781953in}{1.645646in}}{\pgfqpoint{2.776129in}{1.651470in}}%
\pgfpathcurveto{\pgfqpoint{2.770305in}{1.657294in}}{\pgfqpoint{2.762405in}{1.660567in}}{\pgfqpoint{2.754168in}{1.660567in}}%
\pgfpathcurveto{\pgfqpoint{2.745932in}{1.660567in}}{\pgfqpoint{2.738032in}{1.657294in}}{\pgfqpoint{2.732208in}{1.651470in}}%
\pgfpathcurveto{\pgfqpoint{2.726384in}{1.645646in}}{\pgfqpoint{2.723112in}{1.637746in}}{\pgfqpoint{2.723112in}{1.629510in}}%
\pgfpathcurveto{\pgfqpoint{2.723112in}{1.621274in}}{\pgfqpoint{2.726384in}{1.613374in}}{\pgfqpoint{2.732208in}{1.607550in}}%
\pgfpathcurveto{\pgfqpoint{2.738032in}{1.601726in}}{\pgfqpoint{2.745932in}{1.598454in}}{\pgfqpoint{2.754168in}{1.598454in}}%
\pgfpathclose%
\pgfusepath{stroke,fill}%
\end{pgfscope}%
\begin{pgfscope}%
\pgfpathrectangle{\pgfqpoint{0.100000in}{0.220728in}}{\pgfqpoint{3.696000in}{3.696000in}}%
\pgfusepath{clip}%
\pgfsetbuttcap%
\pgfsetroundjoin%
\definecolor{currentfill}{rgb}{0.121569,0.466667,0.705882}%
\pgfsetfillcolor{currentfill}%
\pgfsetfillopacity{0.888822}%
\pgfsetlinewidth{1.003750pt}%
\definecolor{currentstroke}{rgb}{0.121569,0.466667,0.705882}%
\pgfsetstrokecolor{currentstroke}%
\pgfsetstrokeopacity{0.888822}%
\pgfsetdash{}{0pt}%
\pgfpathmoveto{\pgfqpoint{1.790688in}{0.939695in}}%
\pgfpathcurveto{\pgfqpoint{1.798924in}{0.939695in}}{\pgfqpoint{1.806824in}{0.942967in}}{\pgfqpoint{1.812648in}{0.948791in}}%
\pgfpathcurveto{\pgfqpoint{1.818472in}{0.954615in}}{\pgfqpoint{1.821744in}{0.962515in}}{\pgfqpoint{1.821744in}{0.970751in}}%
\pgfpathcurveto{\pgfqpoint{1.821744in}{0.978987in}}{\pgfqpoint{1.818472in}{0.986887in}}{\pgfqpoint{1.812648in}{0.992711in}}%
\pgfpathcurveto{\pgfqpoint{1.806824in}{0.998535in}}{\pgfqpoint{1.798924in}{1.001808in}}{\pgfqpoint{1.790688in}{1.001808in}}%
\pgfpathcurveto{\pgfqpoint{1.782452in}{1.001808in}}{\pgfqpoint{1.774552in}{0.998535in}}{\pgfqpoint{1.768728in}{0.992711in}}%
\pgfpathcurveto{\pgfqpoint{1.762904in}{0.986887in}}{\pgfqpoint{1.759631in}{0.978987in}}{\pgfqpoint{1.759631in}{0.970751in}}%
\pgfpathcurveto{\pgfqpoint{1.759631in}{0.962515in}}{\pgfqpoint{1.762904in}{0.954615in}}{\pgfqpoint{1.768728in}{0.948791in}}%
\pgfpathcurveto{\pgfqpoint{1.774552in}{0.942967in}}{\pgfqpoint{1.782452in}{0.939695in}}{\pgfqpoint{1.790688in}{0.939695in}}%
\pgfpathclose%
\pgfusepath{stroke,fill}%
\end{pgfscope}%
\begin{pgfscope}%
\pgfpathrectangle{\pgfqpoint{0.100000in}{0.220728in}}{\pgfqpoint{3.696000in}{3.696000in}}%
\pgfusepath{clip}%
\pgfsetbuttcap%
\pgfsetroundjoin%
\definecolor{currentfill}{rgb}{0.121569,0.466667,0.705882}%
\pgfsetfillcolor{currentfill}%
\pgfsetfillopacity{0.889517}%
\pgfsetlinewidth{1.003750pt}%
\definecolor{currentstroke}{rgb}{0.121569,0.466667,0.705882}%
\pgfsetstrokecolor{currentstroke}%
\pgfsetstrokeopacity{0.889517}%
\pgfsetdash{}{0pt}%
\pgfpathmoveto{\pgfqpoint{2.751882in}{1.592610in}}%
\pgfpathcurveto{\pgfqpoint{2.760118in}{1.592610in}}{\pgfqpoint{2.768018in}{1.595882in}}{\pgfqpoint{2.773842in}{1.601706in}}%
\pgfpathcurveto{\pgfqpoint{2.779666in}{1.607530in}}{\pgfqpoint{2.782938in}{1.615430in}}{\pgfqpoint{2.782938in}{1.623666in}}%
\pgfpathcurveto{\pgfqpoint{2.782938in}{1.631902in}}{\pgfqpoint{2.779666in}{1.639803in}}{\pgfqpoint{2.773842in}{1.645626in}}%
\pgfpathcurveto{\pgfqpoint{2.768018in}{1.651450in}}{\pgfqpoint{2.760118in}{1.654723in}}{\pgfqpoint{2.751882in}{1.654723in}}%
\pgfpathcurveto{\pgfqpoint{2.743645in}{1.654723in}}{\pgfqpoint{2.735745in}{1.651450in}}{\pgfqpoint{2.729921in}{1.645626in}}%
\pgfpathcurveto{\pgfqpoint{2.724098in}{1.639803in}}{\pgfqpoint{2.720825in}{1.631902in}}{\pgfqpoint{2.720825in}{1.623666in}}%
\pgfpathcurveto{\pgfqpoint{2.720825in}{1.615430in}}{\pgfqpoint{2.724098in}{1.607530in}}{\pgfqpoint{2.729921in}{1.601706in}}%
\pgfpathcurveto{\pgfqpoint{2.735745in}{1.595882in}}{\pgfqpoint{2.743645in}{1.592610in}}{\pgfqpoint{2.751882in}{1.592610in}}%
\pgfpathclose%
\pgfusepath{stroke,fill}%
\end{pgfscope}%
\begin{pgfscope}%
\pgfpathrectangle{\pgfqpoint{0.100000in}{0.220728in}}{\pgfqpoint{3.696000in}{3.696000in}}%
\pgfusepath{clip}%
\pgfsetbuttcap%
\pgfsetroundjoin%
\definecolor{currentfill}{rgb}{0.121569,0.466667,0.705882}%
\pgfsetfillcolor{currentfill}%
\pgfsetfillopacity{0.890391}%
\pgfsetlinewidth{1.003750pt}%
\definecolor{currentstroke}{rgb}{0.121569,0.466667,0.705882}%
\pgfsetstrokecolor{currentstroke}%
\pgfsetstrokeopacity{0.890391}%
\pgfsetdash{}{0pt}%
\pgfpathmoveto{\pgfqpoint{2.748101in}{1.587546in}}%
\pgfpathcurveto{\pgfqpoint{2.756337in}{1.587546in}}{\pgfqpoint{2.764237in}{1.590818in}}{\pgfqpoint{2.770061in}{1.596642in}}%
\pgfpathcurveto{\pgfqpoint{2.775885in}{1.602466in}}{\pgfqpoint{2.779158in}{1.610366in}}{\pgfqpoint{2.779158in}{1.618602in}}%
\pgfpathcurveto{\pgfqpoint{2.779158in}{1.626838in}}{\pgfqpoint{2.775885in}{1.634738in}}{\pgfqpoint{2.770061in}{1.640562in}}%
\pgfpathcurveto{\pgfqpoint{2.764237in}{1.646386in}}{\pgfqpoint{2.756337in}{1.649659in}}{\pgfqpoint{2.748101in}{1.649659in}}%
\pgfpathcurveto{\pgfqpoint{2.739865in}{1.649659in}}{\pgfqpoint{2.731965in}{1.646386in}}{\pgfqpoint{2.726141in}{1.640562in}}%
\pgfpathcurveto{\pgfqpoint{2.720317in}{1.634738in}}{\pgfqpoint{2.717045in}{1.626838in}}{\pgfqpoint{2.717045in}{1.618602in}}%
\pgfpathcurveto{\pgfqpoint{2.717045in}{1.610366in}}{\pgfqpoint{2.720317in}{1.602466in}}{\pgfqpoint{2.726141in}{1.596642in}}%
\pgfpathcurveto{\pgfqpoint{2.731965in}{1.590818in}}{\pgfqpoint{2.739865in}{1.587546in}}{\pgfqpoint{2.748101in}{1.587546in}}%
\pgfpathclose%
\pgfusepath{stroke,fill}%
\end{pgfscope}%
\begin{pgfscope}%
\pgfpathrectangle{\pgfqpoint{0.100000in}{0.220728in}}{\pgfqpoint{3.696000in}{3.696000in}}%
\pgfusepath{clip}%
\pgfsetbuttcap%
\pgfsetroundjoin%
\definecolor{currentfill}{rgb}{0.121569,0.466667,0.705882}%
\pgfsetfillcolor{currentfill}%
\pgfsetfillopacity{0.891476}%
\pgfsetlinewidth{1.003750pt}%
\definecolor{currentstroke}{rgb}{0.121569,0.466667,0.705882}%
\pgfsetstrokecolor{currentstroke}%
\pgfsetstrokeopacity{0.891476}%
\pgfsetdash{}{0pt}%
\pgfpathmoveto{\pgfqpoint{1.809624in}{0.927401in}}%
\pgfpathcurveto{\pgfqpoint{1.817860in}{0.927401in}}{\pgfqpoint{1.825760in}{0.930673in}}{\pgfqpoint{1.831584in}{0.936497in}}%
\pgfpathcurveto{\pgfqpoint{1.837408in}{0.942321in}}{\pgfqpoint{1.840680in}{0.950221in}}{\pgfqpoint{1.840680in}{0.958458in}}%
\pgfpathcurveto{\pgfqpoint{1.840680in}{0.966694in}}{\pgfqpoint{1.837408in}{0.974594in}}{\pgfqpoint{1.831584in}{0.980418in}}%
\pgfpathcurveto{\pgfqpoint{1.825760in}{0.986242in}}{\pgfqpoint{1.817860in}{0.989514in}}{\pgfqpoint{1.809624in}{0.989514in}}%
\pgfpathcurveto{\pgfqpoint{1.801388in}{0.989514in}}{\pgfqpoint{1.793488in}{0.986242in}}{\pgfqpoint{1.787664in}{0.980418in}}%
\pgfpathcurveto{\pgfqpoint{1.781840in}{0.974594in}}{\pgfqpoint{1.778567in}{0.966694in}}{\pgfqpoint{1.778567in}{0.958458in}}%
\pgfpathcurveto{\pgfqpoint{1.778567in}{0.950221in}}{\pgfqpoint{1.781840in}{0.942321in}}{\pgfqpoint{1.787664in}{0.936497in}}%
\pgfpathcurveto{\pgfqpoint{1.793488in}{0.930673in}}{\pgfqpoint{1.801388in}{0.927401in}}{\pgfqpoint{1.809624in}{0.927401in}}%
\pgfpathclose%
\pgfusepath{stroke,fill}%
\end{pgfscope}%
\begin{pgfscope}%
\pgfpathrectangle{\pgfqpoint{0.100000in}{0.220728in}}{\pgfqpoint{3.696000in}{3.696000in}}%
\pgfusepath{clip}%
\pgfsetbuttcap%
\pgfsetroundjoin%
\definecolor{currentfill}{rgb}{0.121569,0.466667,0.705882}%
\pgfsetfillcolor{currentfill}%
\pgfsetfillopacity{0.891776}%
\pgfsetlinewidth{1.003750pt}%
\definecolor{currentstroke}{rgb}{0.121569,0.466667,0.705882}%
\pgfsetstrokecolor{currentstroke}%
\pgfsetstrokeopacity{0.891776}%
\pgfsetdash{}{0pt}%
\pgfpathmoveto{\pgfqpoint{2.745246in}{1.573839in}}%
\pgfpathcurveto{\pgfqpoint{2.753482in}{1.573839in}}{\pgfqpoint{2.761382in}{1.577111in}}{\pgfqpoint{2.767206in}{1.582935in}}%
\pgfpathcurveto{\pgfqpoint{2.773030in}{1.588759in}}{\pgfqpoint{2.776302in}{1.596659in}}{\pgfqpoint{2.776302in}{1.604895in}}%
\pgfpathcurveto{\pgfqpoint{2.776302in}{1.613131in}}{\pgfqpoint{2.773030in}{1.621031in}}{\pgfqpoint{2.767206in}{1.626855in}}%
\pgfpathcurveto{\pgfqpoint{2.761382in}{1.632679in}}{\pgfqpoint{2.753482in}{1.635952in}}{\pgfqpoint{2.745246in}{1.635952in}}%
\pgfpathcurveto{\pgfqpoint{2.737009in}{1.635952in}}{\pgfqpoint{2.729109in}{1.632679in}}{\pgfqpoint{2.723285in}{1.626855in}}%
\pgfpathcurveto{\pgfqpoint{2.717461in}{1.621031in}}{\pgfqpoint{2.714189in}{1.613131in}}{\pgfqpoint{2.714189in}{1.604895in}}%
\pgfpathcurveto{\pgfqpoint{2.714189in}{1.596659in}}{\pgfqpoint{2.717461in}{1.588759in}}{\pgfqpoint{2.723285in}{1.582935in}}%
\pgfpathcurveto{\pgfqpoint{2.729109in}{1.577111in}}{\pgfqpoint{2.737009in}{1.573839in}}{\pgfqpoint{2.745246in}{1.573839in}}%
\pgfpathclose%
\pgfusepath{stroke,fill}%
\end{pgfscope}%
\begin{pgfscope}%
\pgfpathrectangle{\pgfqpoint{0.100000in}{0.220728in}}{\pgfqpoint{3.696000in}{3.696000in}}%
\pgfusepath{clip}%
\pgfsetbuttcap%
\pgfsetroundjoin%
\definecolor{currentfill}{rgb}{0.121569,0.466667,0.705882}%
\pgfsetfillcolor{currentfill}%
\pgfsetfillopacity{0.893347}%
\pgfsetlinewidth{1.003750pt}%
\definecolor{currentstroke}{rgb}{0.121569,0.466667,0.705882}%
\pgfsetstrokecolor{currentstroke}%
\pgfsetstrokeopacity{0.893347}%
\pgfsetdash{}{0pt}%
\pgfpathmoveto{\pgfqpoint{2.738916in}{1.562823in}}%
\pgfpathcurveto{\pgfqpoint{2.747153in}{1.562823in}}{\pgfqpoint{2.755053in}{1.566096in}}{\pgfqpoint{2.760877in}{1.571920in}}%
\pgfpathcurveto{\pgfqpoint{2.766700in}{1.577744in}}{\pgfqpoint{2.769973in}{1.585644in}}{\pgfqpoint{2.769973in}{1.593880in}}%
\pgfpathcurveto{\pgfqpoint{2.769973in}{1.602116in}}{\pgfqpoint{2.766700in}{1.610016in}}{\pgfqpoint{2.760877in}{1.615840in}}%
\pgfpathcurveto{\pgfqpoint{2.755053in}{1.621664in}}{\pgfqpoint{2.747153in}{1.624936in}}{\pgfqpoint{2.738916in}{1.624936in}}%
\pgfpathcurveto{\pgfqpoint{2.730680in}{1.624936in}}{\pgfqpoint{2.722780in}{1.621664in}}{\pgfqpoint{2.716956in}{1.615840in}}%
\pgfpathcurveto{\pgfqpoint{2.711132in}{1.610016in}}{\pgfqpoint{2.707860in}{1.602116in}}{\pgfqpoint{2.707860in}{1.593880in}}%
\pgfpathcurveto{\pgfqpoint{2.707860in}{1.585644in}}{\pgfqpoint{2.711132in}{1.577744in}}{\pgfqpoint{2.716956in}{1.571920in}}%
\pgfpathcurveto{\pgfqpoint{2.722780in}{1.566096in}}{\pgfqpoint{2.730680in}{1.562823in}}{\pgfqpoint{2.738916in}{1.562823in}}%
\pgfpathclose%
\pgfusepath{stroke,fill}%
\end{pgfscope}%
\begin{pgfscope}%
\pgfpathrectangle{\pgfqpoint{0.100000in}{0.220728in}}{\pgfqpoint{3.696000in}{3.696000in}}%
\pgfusepath{clip}%
\pgfsetbuttcap%
\pgfsetroundjoin%
\definecolor{currentfill}{rgb}{0.121569,0.466667,0.705882}%
\pgfsetfillcolor{currentfill}%
\pgfsetfillopacity{0.894765}%
\pgfsetlinewidth{1.003750pt}%
\definecolor{currentstroke}{rgb}{0.121569,0.466667,0.705882}%
\pgfsetstrokecolor{currentstroke}%
\pgfsetstrokeopacity{0.894765}%
\pgfsetdash{}{0pt}%
\pgfpathmoveto{\pgfqpoint{2.732383in}{1.549786in}}%
\pgfpathcurveto{\pgfqpoint{2.740619in}{1.549786in}}{\pgfqpoint{2.748519in}{1.553058in}}{\pgfqpoint{2.754343in}{1.558882in}}%
\pgfpathcurveto{\pgfqpoint{2.760167in}{1.564706in}}{\pgfqpoint{2.763439in}{1.572606in}}{\pgfqpoint{2.763439in}{1.580843in}}%
\pgfpathcurveto{\pgfqpoint{2.763439in}{1.589079in}}{\pgfqpoint{2.760167in}{1.596979in}}{\pgfqpoint{2.754343in}{1.602803in}}%
\pgfpathcurveto{\pgfqpoint{2.748519in}{1.608627in}}{\pgfqpoint{2.740619in}{1.611899in}}{\pgfqpoint{2.732383in}{1.611899in}}%
\pgfpathcurveto{\pgfqpoint{2.724146in}{1.611899in}}{\pgfqpoint{2.716246in}{1.608627in}}{\pgfqpoint{2.710422in}{1.602803in}}%
\pgfpathcurveto{\pgfqpoint{2.704598in}{1.596979in}}{\pgfqpoint{2.701326in}{1.589079in}}{\pgfqpoint{2.701326in}{1.580843in}}%
\pgfpathcurveto{\pgfqpoint{2.701326in}{1.572606in}}{\pgfqpoint{2.704598in}{1.564706in}}{\pgfqpoint{2.710422in}{1.558882in}}%
\pgfpathcurveto{\pgfqpoint{2.716246in}{1.553058in}}{\pgfqpoint{2.724146in}{1.549786in}}{\pgfqpoint{2.732383in}{1.549786in}}%
\pgfpathclose%
\pgfusepath{stroke,fill}%
\end{pgfscope}%
\begin{pgfscope}%
\pgfpathrectangle{\pgfqpoint{0.100000in}{0.220728in}}{\pgfqpoint{3.696000in}{3.696000in}}%
\pgfusepath{clip}%
\pgfsetbuttcap%
\pgfsetroundjoin%
\definecolor{currentfill}{rgb}{0.121569,0.466667,0.705882}%
\pgfsetfillcolor{currentfill}%
\pgfsetfillopacity{0.895305}%
\pgfsetlinewidth{1.003750pt}%
\definecolor{currentstroke}{rgb}{0.121569,0.466667,0.705882}%
\pgfsetstrokecolor{currentstroke}%
\pgfsetstrokeopacity{0.895305}%
\pgfsetdash{}{0pt}%
\pgfpathmoveto{\pgfqpoint{1.827553in}{0.920331in}}%
\pgfpathcurveto{\pgfqpoint{1.835790in}{0.920331in}}{\pgfqpoint{1.843690in}{0.923604in}}{\pgfqpoint{1.849514in}{0.929428in}}%
\pgfpathcurveto{\pgfqpoint{1.855338in}{0.935252in}}{\pgfqpoint{1.858610in}{0.943152in}}{\pgfqpoint{1.858610in}{0.951388in}}%
\pgfpathcurveto{\pgfqpoint{1.858610in}{0.959624in}}{\pgfqpoint{1.855338in}{0.967524in}}{\pgfqpoint{1.849514in}{0.973348in}}%
\pgfpathcurveto{\pgfqpoint{1.843690in}{0.979172in}}{\pgfqpoint{1.835790in}{0.982444in}}{\pgfqpoint{1.827553in}{0.982444in}}%
\pgfpathcurveto{\pgfqpoint{1.819317in}{0.982444in}}{\pgfqpoint{1.811417in}{0.979172in}}{\pgfqpoint{1.805593in}{0.973348in}}%
\pgfpathcurveto{\pgfqpoint{1.799769in}{0.967524in}}{\pgfqpoint{1.796497in}{0.959624in}}{\pgfqpoint{1.796497in}{0.951388in}}%
\pgfpathcurveto{\pgfqpoint{1.796497in}{0.943152in}}{\pgfqpoint{1.799769in}{0.935252in}}{\pgfqpoint{1.805593in}{0.929428in}}%
\pgfpathcurveto{\pgfqpoint{1.811417in}{0.923604in}}{\pgfqpoint{1.819317in}{0.920331in}}{\pgfqpoint{1.827553in}{0.920331in}}%
\pgfpathclose%
\pgfusepath{stroke,fill}%
\end{pgfscope}%
\begin{pgfscope}%
\pgfpathrectangle{\pgfqpoint{0.100000in}{0.220728in}}{\pgfqpoint{3.696000in}{3.696000in}}%
\pgfusepath{clip}%
\pgfsetbuttcap%
\pgfsetroundjoin%
\definecolor{currentfill}{rgb}{0.121569,0.466667,0.705882}%
\pgfsetfillcolor{currentfill}%
\pgfsetfillopacity{0.895656}%
\pgfsetlinewidth{1.003750pt}%
\definecolor{currentstroke}{rgb}{0.121569,0.466667,0.705882}%
\pgfsetstrokecolor{currentstroke}%
\pgfsetstrokeopacity{0.895656}%
\pgfsetdash{}{0pt}%
\pgfpathmoveto{\pgfqpoint{2.730038in}{1.541673in}}%
\pgfpathcurveto{\pgfqpoint{2.738275in}{1.541673in}}{\pgfqpoint{2.746175in}{1.544946in}}{\pgfqpoint{2.751999in}{1.550769in}}%
\pgfpathcurveto{\pgfqpoint{2.757823in}{1.556593in}}{\pgfqpoint{2.761095in}{1.564493in}}{\pgfqpoint{2.761095in}{1.572730in}}%
\pgfpathcurveto{\pgfqpoint{2.761095in}{1.580966in}}{\pgfqpoint{2.757823in}{1.588866in}}{\pgfqpoint{2.751999in}{1.594690in}}%
\pgfpathcurveto{\pgfqpoint{2.746175in}{1.600514in}}{\pgfqpoint{2.738275in}{1.603786in}}{\pgfqpoint{2.730038in}{1.603786in}}%
\pgfpathcurveto{\pgfqpoint{2.721802in}{1.603786in}}{\pgfqpoint{2.713902in}{1.600514in}}{\pgfqpoint{2.708078in}{1.594690in}}%
\pgfpathcurveto{\pgfqpoint{2.702254in}{1.588866in}}{\pgfqpoint{2.698982in}{1.580966in}}{\pgfqpoint{2.698982in}{1.572730in}}%
\pgfpathcurveto{\pgfqpoint{2.698982in}{1.564493in}}{\pgfqpoint{2.702254in}{1.556593in}}{\pgfqpoint{2.708078in}{1.550769in}}%
\pgfpathcurveto{\pgfqpoint{2.713902in}{1.544946in}}{\pgfqpoint{2.721802in}{1.541673in}}{\pgfqpoint{2.730038in}{1.541673in}}%
\pgfpathclose%
\pgfusepath{stroke,fill}%
\end{pgfscope}%
\begin{pgfscope}%
\pgfpathrectangle{\pgfqpoint{0.100000in}{0.220728in}}{\pgfqpoint{3.696000in}{3.696000in}}%
\pgfusepath{clip}%
\pgfsetbuttcap%
\pgfsetroundjoin%
\definecolor{currentfill}{rgb}{0.121569,0.466667,0.705882}%
\pgfsetfillcolor{currentfill}%
\pgfsetfillopacity{0.896640}%
\pgfsetlinewidth{1.003750pt}%
\definecolor{currentstroke}{rgb}{0.121569,0.466667,0.705882}%
\pgfsetstrokecolor{currentstroke}%
\pgfsetstrokeopacity{0.896640}%
\pgfsetdash{}{0pt}%
\pgfpathmoveto{\pgfqpoint{2.723124in}{1.529664in}}%
\pgfpathcurveto{\pgfqpoint{2.731360in}{1.529664in}}{\pgfqpoint{2.739260in}{1.532937in}}{\pgfqpoint{2.745084in}{1.538761in}}%
\pgfpathcurveto{\pgfqpoint{2.750908in}{1.544584in}}{\pgfqpoint{2.754180in}{1.552485in}}{\pgfqpoint{2.754180in}{1.560721in}}%
\pgfpathcurveto{\pgfqpoint{2.754180in}{1.568957in}}{\pgfqpoint{2.750908in}{1.576857in}}{\pgfqpoint{2.745084in}{1.582681in}}%
\pgfpathcurveto{\pgfqpoint{2.739260in}{1.588505in}}{\pgfqpoint{2.731360in}{1.591777in}}{\pgfqpoint{2.723124in}{1.591777in}}%
\pgfpathcurveto{\pgfqpoint{2.714888in}{1.591777in}}{\pgfqpoint{2.706988in}{1.588505in}}{\pgfqpoint{2.701164in}{1.582681in}}%
\pgfpathcurveto{\pgfqpoint{2.695340in}{1.576857in}}{\pgfqpoint{2.692067in}{1.568957in}}{\pgfqpoint{2.692067in}{1.560721in}}%
\pgfpathcurveto{\pgfqpoint{2.692067in}{1.552485in}}{\pgfqpoint{2.695340in}{1.544584in}}{\pgfqpoint{2.701164in}{1.538761in}}%
\pgfpathcurveto{\pgfqpoint{2.706988in}{1.532937in}}{\pgfqpoint{2.714888in}{1.529664in}}{\pgfqpoint{2.723124in}{1.529664in}}%
\pgfpathclose%
\pgfusepath{stroke,fill}%
\end{pgfscope}%
\begin{pgfscope}%
\pgfpathrectangle{\pgfqpoint{0.100000in}{0.220728in}}{\pgfqpoint{3.696000in}{3.696000in}}%
\pgfusepath{clip}%
\pgfsetbuttcap%
\pgfsetroundjoin%
\definecolor{currentfill}{rgb}{0.121569,0.466667,0.705882}%
\pgfsetfillcolor{currentfill}%
\pgfsetfillopacity{0.898505}%
\pgfsetlinewidth{1.003750pt}%
\definecolor{currentstroke}{rgb}{0.121569,0.466667,0.705882}%
\pgfsetstrokecolor{currentstroke}%
\pgfsetstrokeopacity{0.898505}%
\pgfsetdash{}{0pt}%
\pgfpathmoveto{\pgfqpoint{2.719309in}{1.515694in}}%
\pgfpathcurveto{\pgfqpoint{2.727545in}{1.515694in}}{\pgfqpoint{2.735445in}{1.518966in}}{\pgfqpoint{2.741269in}{1.524790in}}%
\pgfpathcurveto{\pgfqpoint{2.747093in}{1.530614in}}{\pgfqpoint{2.750365in}{1.538514in}}{\pgfqpoint{2.750365in}{1.546750in}}%
\pgfpathcurveto{\pgfqpoint{2.750365in}{1.554987in}}{\pgfqpoint{2.747093in}{1.562887in}}{\pgfqpoint{2.741269in}{1.568711in}}%
\pgfpathcurveto{\pgfqpoint{2.735445in}{1.574535in}}{\pgfqpoint{2.727545in}{1.577807in}}{\pgfqpoint{2.719309in}{1.577807in}}%
\pgfpathcurveto{\pgfqpoint{2.711072in}{1.577807in}}{\pgfqpoint{2.703172in}{1.574535in}}{\pgfqpoint{2.697348in}{1.568711in}}%
\pgfpathcurveto{\pgfqpoint{2.691524in}{1.562887in}}{\pgfqpoint{2.688252in}{1.554987in}}{\pgfqpoint{2.688252in}{1.546750in}}%
\pgfpathcurveto{\pgfqpoint{2.688252in}{1.538514in}}{\pgfqpoint{2.691524in}{1.530614in}}{\pgfqpoint{2.697348in}{1.524790in}}%
\pgfpathcurveto{\pgfqpoint{2.703172in}{1.518966in}}{\pgfqpoint{2.711072in}{1.515694in}}{\pgfqpoint{2.719309in}{1.515694in}}%
\pgfpathclose%
\pgfusepath{stroke,fill}%
\end{pgfscope}%
\begin{pgfscope}%
\pgfpathrectangle{\pgfqpoint{0.100000in}{0.220728in}}{\pgfqpoint{3.696000in}{3.696000in}}%
\pgfusepath{clip}%
\pgfsetbuttcap%
\pgfsetroundjoin%
\definecolor{currentfill}{rgb}{0.121569,0.466667,0.705882}%
\pgfsetfillcolor{currentfill}%
\pgfsetfillopacity{0.899465}%
\pgfsetlinewidth{1.003750pt}%
\definecolor{currentstroke}{rgb}{0.121569,0.466667,0.705882}%
\pgfsetstrokecolor{currentstroke}%
\pgfsetstrokeopacity{0.899465}%
\pgfsetdash{}{0pt}%
\pgfpathmoveto{\pgfqpoint{2.716352in}{1.508493in}}%
\pgfpathcurveto{\pgfqpoint{2.724589in}{1.508493in}}{\pgfqpoint{2.732489in}{1.511765in}}{\pgfqpoint{2.738313in}{1.517589in}}%
\pgfpathcurveto{\pgfqpoint{2.744137in}{1.523413in}}{\pgfqpoint{2.747409in}{1.531313in}}{\pgfqpoint{2.747409in}{1.539549in}}%
\pgfpathcurveto{\pgfqpoint{2.747409in}{1.547786in}}{\pgfqpoint{2.744137in}{1.555686in}}{\pgfqpoint{2.738313in}{1.561510in}}%
\pgfpathcurveto{\pgfqpoint{2.732489in}{1.567334in}}{\pgfqpoint{2.724589in}{1.570606in}}{\pgfqpoint{2.716352in}{1.570606in}}%
\pgfpathcurveto{\pgfqpoint{2.708116in}{1.570606in}}{\pgfqpoint{2.700216in}{1.567334in}}{\pgfqpoint{2.694392in}{1.561510in}}%
\pgfpathcurveto{\pgfqpoint{2.688568in}{1.555686in}}{\pgfqpoint{2.685296in}{1.547786in}}{\pgfqpoint{2.685296in}{1.539549in}}%
\pgfpathcurveto{\pgfqpoint{2.685296in}{1.531313in}}{\pgfqpoint{2.688568in}{1.523413in}}{\pgfqpoint{2.694392in}{1.517589in}}%
\pgfpathcurveto{\pgfqpoint{2.700216in}{1.511765in}}{\pgfqpoint{2.708116in}{1.508493in}}{\pgfqpoint{2.716352in}{1.508493in}}%
\pgfpathclose%
\pgfusepath{stroke,fill}%
\end{pgfscope}%
\begin{pgfscope}%
\pgfpathrectangle{\pgfqpoint{0.100000in}{0.220728in}}{\pgfqpoint{3.696000in}{3.696000in}}%
\pgfusepath{clip}%
\pgfsetbuttcap%
\pgfsetroundjoin%
\definecolor{currentfill}{rgb}{0.121569,0.466667,0.705882}%
\pgfsetfillcolor{currentfill}%
\pgfsetfillopacity{0.899797}%
\pgfsetlinewidth{1.003750pt}%
\definecolor{currentstroke}{rgb}{0.121569,0.466667,0.705882}%
\pgfsetstrokecolor{currentstroke}%
\pgfsetstrokeopacity{0.899797}%
\pgfsetdash{}{0pt}%
\pgfpathmoveto{\pgfqpoint{1.844379in}{0.914444in}}%
\pgfpathcurveto{\pgfqpoint{1.852615in}{0.914444in}}{\pgfqpoint{1.860515in}{0.917716in}}{\pgfqpoint{1.866339in}{0.923540in}}%
\pgfpathcurveto{\pgfqpoint{1.872163in}{0.929364in}}{\pgfqpoint{1.875435in}{0.937264in}}{\pgfqpoint{1.875435in}{0.945500in}}%
\pgfpathcurveto{\pgfqpoint{1.875435in}{0.953736in}}{\pgfqpoint{1.872163in}{0.961637in}}{\pgfqpoint{1.866339in}{0.967460in}}%
\pgfpathcurveto{\pgfqpoint{1.860515in}{0.973284in}}{\pgfqpoint{1.852615in}{0.976557in}}{\pgfqpoint{1.844379in}{0.976557in}}%
\pgfpathcurveto{\pgfqpoint{1.836143in}{0.976557in}}{\pgfqpoint{1.828243in}{0.973284in}}{\pgfqpoint{1.822419in}{0.967460in}}%
\pgfpathcurveto{\pgfqpoint{1.816595in}{0.961637in}}{\pgfqpoint{1.813322in}{0.953736in}}{\pgfqpoint{1.813322in}{0.945500in}}%
\pgfpathcurveto{\pgfqpoint{1.813322in}{0.937264in}}{\pgfqpoint{1.816595in}{0.929364in}}{\pgfqpoint{1.822419in}{0.923540in}}%
\pgfpathcurveto{\pgfqpoint{1.828243in}{0.917716in}}{\pgfqpoint{1.836143in}{0.914444in}}{\pgfqpoint{1.844379in}{0.914444in}}%
\pgfpathclose%
\pgfusepath{stroke,fill}%
\end{pgfscope}%
\begin{pgfscope}%
\pgfpathrectangle{\pgfqpoint{0.100000in}{0.220728in}}{\pgfqpoint{3.696000in}{3.696000in}}%
\pgfusepath{clip}%
\pgfsetbuttcap%
\pgfsetroundjoin%
\definecolor{currentfill}{rgb}{0.121569,0.466667,0.705882}%
\pgfsetfillcolor{currentfill}%
\pgfsetfillopacity{0.899919}%
\pgfsetlinewidth{1.003750pt}%
\definecolor{currentstroke}{rgb}{0.121569,0.466667,0.705882}%
\pgfsetstrokecolor{currentstroke}%
\pgfsetstrokeopacity{0.899919}%
\pgfsetdash{}{0pt}%
\pgfpathmoveto{\pgfqpoint{2.714174in}{1.505018in}}%
\pgfpathcurveto{\pgfqpoint{2.722410in}{1.505018in}}{\pgfqpoint{2.730310in}{1.508290in}}{\pgfqpoint{2.736134in}{1.514114in}}%
\pgfpathcurveto{\pgfqpoint{2.741958in}{1.519938in}}{\pgfqpoint{2.745230in}{1.527838in}}{\pgfqpoint{2.745230in}{1.536074in}}%
\pgfpathcurveto{\pgfqpoint{2.745230in}{1.544311in}}{\pgfqpoint{2.741958in}{1.552211in}}{\pgfqpoint{2.736134in}{1.558035in}}%
\pgfpathcurveto{\pgfqpoint{2.730310in}{1.563859in}}{\pgfqpoint{2.722410in}{1.567131in}}{\pgfqpoint{2.714174in}{1.567131in}}%
\pgfpathcurveto{\pgfqpoint{2.705937in}{1.567131in}}{\pgfqpoint{2.698037in}{1.563859in}}{\pgfqpoint{2.692213in}{1.558035in}}%
\pgfpathcurveto{\pgfqpoint{2.686389in}{1.552211in}}{\pgfqpoint{2.683117in}{1.544311in}}{\pgfqpoint{2.683117in}{1.536074in}}%
\pgfpathcurveto{\pgfqpoint{2.683117in}{1.527838in}}{\pgfqpoint{2.686389in}{1.519938in}}{\pgfqpoint{2.692213in}{1.514114in}}%
\pgfpathcurveto{\pgfqpoint{2.698037in}{1.508290in}}{\pgfqpoint{2.705937in}{1.505018in}}{\pgfqpoint{2.714174in}{1.505018in}}%
\pgfpathclose%
\pgfusepath{stroke,fill}%
\end{pgfscope}%
\begin{pgfscope}%
\pgfpathrectangle{\pgfqpoint{0.100000in}{0.220728in}}{\pgfqpoint{3.696000in}{3.696000in}}%
\pgfusepath{clip}%
\pgfsetbuttcap%
\pgfsetroundjoin%
\definecolor{currentfill}{rgb}{0.121569,0.466667,0.705882}%
\pgfsetfillcolor{currentfill}%
\pgfsetfillopacity{0.901026}%
\pgfsetlinewidth{1.003750pt}%
\definecolor{currentstroke}{rgb}{0.121569,0.466667,0.705882}%
\pgfsetstrokecolor{currentstroke}%
\pgfsetstrokeopacity{0.901026}%
\pgfsetdash{}{0pt}%
\pgfpathmoveto{\pgfqpoint{2.712017in}{1.496584in}}%
\pgfpathcurveto{\pgfqpoint{2.720253in}{1.496584in}}{\pgfqpoint{2.728153in}{1.499856in}}{\pgfqpoint{2.733977in}{1.505680in}}%
\pgfpathcurveto{\pgfqpoint{2.739801in}{1.511504in}}{\pgfqpoint{2.743074in}{1.519404in}}{\pgfqpoint{2.743074in}{1.527640in}}%
\pgfpathcurveto{\pgfqpoint{2.743074in}{1.535876in}}{\pgfqpoint{2.739801in}{1.543776in}}{\pgfqpoint{2.733977in}{1.549600in}}%
\pgfpathcurveto{\pgfqpoint{2.728153in}{1.555424in}}{\pgfqpoint{2.720253in}{1.558697in}}{\pgfqpoint{2.712017in}{1.558697in}}%
\pgfpathcurveto{\pgfqpoint{2.703781in}{1.558697in}}{\pgfqpoint{2.695881in}{1.555424in}}{\pgfqpoint{2.690057in}{1.549600in}}%
\pgfpathcurveto{\pgfqpoint{2.684233in}{1.543776in}}{\pgfqpoint{2.680961in}{1.535876in}}{\pgfqpoint{2.680961in}{1.527640in}}%
\pgfpathcurveto{\pgfqpoint{2.680961in}{1.519404in}}{\pgfqpoint{2.684233in}{1.511504in}}{\pgfqpoint{2.690057in}{1.505680in}}%
\pgfpathcurveto{\pgfqpoint{2.695881in}{1.499856in}}{\pgfqpoint{2.703781in}{1.496584in}}{\pgfqpoint{2.712017in}{1.496584in}}%
\pgfpathclose%
\pgfusepath{stroke,fill}%
\end{pgfscope}%
\begin{pgfscope}%
\pgfpathrectangle{\pgfqpoint{0.100000in}{0.220728in}}{\pgfqpoint{3.696000in}{3.696000in}}%
\pgfusepath{clip}%
\pgfsetbuttcap%
\pgfsetroundjoin%
\definecolor{currentfill}{rgb}{0.121569,0.466667,0.705882}%
\pgfsetfillcolor{currentfill}%
\pgfsetfillopacity{0.902014}%
\pgfsetlinewidth{1.003750pt}%
\definecolor{currentstroke}{rgb}{0.121569,0.466667,0.705882}%
\pgfsetstrokecolor{currentstroke}%
\pgfsetstrokeopacity{0.902014}%
\pgfsetdash{}{0pt}%
\pgfpathmoveto{\pgfqpoint{2.707323in}{1.488389in}}%
\pgfpathcurveto{\pgfqpoint{2.715560in}{1.488389in}}{\pgfqpoint{2.723460in}{1.491661in}}{\pgfqpoint{2.729284in}{1.497485in}}%
\pgfpathcurveto{\pgfqpoint{2.735108in}{1.503309in}}{\pgfqpoint{2.738380in}{1.511209in}}{\pgfqpoint{2.738380in}{1.519445in}}%
\pgfpathcurveto{\pgfqpoint{2.738380in}{1.527682in}}{\pgfqpoint{2.735108in}{1.535582in}}{\pgfqpoint{2.729284in}{1.541406in}}%
\pgfpathcurveto{\pgfqpoint{2.723460in}{1.547230in}}{\pgfqpoint{2.715560in}{1.550502in}}{\pgfqpoint{2.707323in}{1.550502in}}%
\pgfpathcurveto{\pgfqpoint{2.699087in}{1.550502in}}{\pgfqpoint{2.691187in}{1.547230in}}{\pgfqpoint{2.685363in}{1.541406in}}%
\pgfpathcurveto{\pgfqpoint{2.679539in}{1.535582in}}{\pgfqpoint{2.676267in}{1.527682in}}{\pgfqpoint{2.676267in}{1.519445in}}%
\pgfpathcurveto{\pgfqpoint{2.676267in}{1.511209in}}{\pgfqpoint{2.679539in}{1.503309in}}{\pgfqpoint{2.685363in}{1.497485in}}%
\pgfpathcurveto{\pgfqpoint{2.691187in}{1.491661in}}{\pgfqpoint{2.699087in}{1.488389in}}{\pgfqpoint{2.707323in}{1.488389in}}%
\pgfpathclose%
\pgfusepath{stroke,fill}%
\end{pgfscope}%
\begin{pgfscope}%
\pgfpathrectangle{\pgfqpoint{0.100000in}{0.220728in}}{\pgfqpoint{3.696000in}{3.696000in}}%
\pgfusepath{clip}%
\pgfsetbuttcap%
\pgfsetroundjoin%
\definecolor{currentfill}{rgb}{0.121569,0.466667,0.705882}%
\pgfsetfillcolor{currentfill}%
\pgfsetfillopacity{0.903389}%
\pgfsetlinewidth{1.003750pt}%
\definecolor{currentstroke}{rgb}{0.121569,0.466667,0.705882}%
\pgfsetstrokecolor{currentstroke}%
\pgfsetstrokeopacity{0.903389}%
\pgfsetdash{}{0pt}%
\pgfpathmoveto{\pgfqpoint{2.702370in}{1.478780in}}%
\pgfpathcurveto{\pgfqpoint{2.710606in}{1.478780in}}{\pgfqpoint{2.718506in}{1.482052in}}{\pgfqpoint{2.724330in}{1.487876in}}%
\pgfpathcurveto{\pgfqpoint{2.730154in}{1.493700in}}{\pgfqpoint{2.733427in}{1.501600in}}{\pgfqpoint{2.733427in}{1.509836in}}%
\pgfpathcurveto{\pgfqpoint{2.733427in}{1.518073in}}{\pgfqpoint{2.730154in}{1.525973in}}{\pgfqpoint{2.724330in}{1.531796in}}%
\pgfpathcurveto{\pgfqpoint{2.718506in}{1.537620in}}{\pgfqpoint{2.710606in}{1.540893in}}{\pgfqpoint{2.702370in}{1.540893in}}%
\pgfpathcurveto{\pgfqpoint{2.694134in}{1.540893in}}{\pgfqpoint{2.686234in}{1.537620in}}{\pgfqpoint{2.680410in}{1.531796in}}%
\pgfpathcurveto{\pgfqpoint{2.674586in}{1.525973in}}{\pgfqpoint{2.671314in}{1.518073in}}{\pgfqpoint{2.671314in}{1.509836in}}%
\pgfpathcurveto{\pgfqpoint{2.671314in}{1.501600in}}{\pgfqpoint{2.674586in}{1.493700in}}{\pgfqpoint{2.680410in}{1.487876in}}%
\pgfpathcurveto{\pgfqpoint{2.686234in}{1.482052in}}{\pgfqpoint{2.694134in}{1.478780in}}{\pgfqpoint{2.702370in}{1.478780in}}%
\pgfpathclose%
\pgfusepath{stroke,fill}%
\end{pgfscope}%
\begin{pgfscope}%
\pgfpathrectangle{\pgfqpoint{0.100000in}{0.220728in}}{\pgfqpoint{3.696000in}{3.696000in}}%
\pgfusepath{clip}%
\pgfsetbuttcap%
\pgfsetroundjoin%
\definecolor{currentfill}{rgb}{0.121569,0.466667,0.705882}%
\pgfsetfillcolor{currentfill}%
\pgfsetfillopacity{0.903463}%
\pgfsetlinewidth{1.003750pt}%
\definecolor{currentstroke}{rgb}{0.121569,0.466667,0.705882}%
\pgfsetstrokecolor{currentstroke}%
\pgfsetstrokeopacity{0.903463}%
\pgfsetdash{}{0pt}%
\pgfpathmoveto{\pgfqpoint{1.860310in}{0.909279in}}%
\pgfpathcurveto{\pgfqpoint{1.868546in}{0.909279in}}{\pgfqpoint{1.876446in}{0.912551in}}{\pgfqpoint{1.882270in}{0.918375in}}%
\pgfpathcurveto{\pgfqpoint{1.888094in}{0.924199in}}{\pgfqpoint{1.891366in}{0.932099in}}{\pgfqpoint{1.891366in}{0.940335in}}%
\pgfpathcurveto{\pgfqpoint{1.891366in}{0.948571in}}{\pgfqpoint{1.888094in}{0.956471in}}{\pgfqpoint{1.882270in}{0.962295in}}%
\pgfpathcurveto{\pgfqpoint{1.876446in}{0.968119in}}{\pgfqpoint{1.868546in}{0.971392in}}{\pgfqpoint{1.860310in}{0.971392in}}%
\pgfpathcurveto{\pgfqpoint{1.852073in}{0.971392in}}{\pgfqpoint{1.844173in}{0.968119in}}{\pgfqpoint{1.838349in}{0.962295in}}%
\pgfpathcurveto{\pgfqpoint{1.832525in}{0.956471in}}{\pgfqpoint{1.829253in}{0.948571in}}{\pgfqpoint{1.829253in}{0.940335in}}%
\pgfpathcurveto{\pgfqpoint{1.829253in}{0.932099in}}{\pgfqpoint{1.832525in}{0.924199in}}{\pgfqpoint{1.838349in}{0.918375in}}%
\pgfpathcurveto{\pgfqpoint{1.844173in}{0.912551in}}{\pgfqpoint{1.852073in}{0.909279in}}{\pgfqpoint{1.860310in}{0.909279in}}%
\pgfpathclose%
\pgfusepath{stroke,fill}%
\end{pgfscope}%
\begin{pgfscope}%
\pgfpathrectangle{\pgfqpoint{0.100000in}{0.220728in}}{\pgfqpoint{3.696000in}{3.696000in}}%
\pgfusepath{clip}%
\pgfsetbuttcap%
\pgfsetroundjoin%
\definecolor{currentfill}{rgb}{0.121569,0.466667,0.705882}%
\pgfsetfillcolor{currentfill}%
\pgfsetfillopacity{0.904077}%
\pgfsetlinewidth{1.003750pt}%
\definecolor{currentstroke}{rgb}{0.121569,0.466667,0.705882}%
\pgfsetstrokecolor{currentstroke}%
\pgfsetstrokeopacity{0.904077}%
\pgfsetdash{}{0pt}%
\pgfpathmoveto{\pgfqpoint{2.700776in}{1.472115in}}%
\pgfpathcurveto{\pgfqpoint{2.709012in}{1.472115in}}{\pgfqpoint{2.716912in}{1.475387in}}{\pgfqpoint{2.722736in}{1.481211in}}%
\pgfpathcurveto{\pgfqpoint{2.728560in}{1.487035in}}{\pgfqpoint{2.731833in}{1.494935in}}{\pgfqpoint{2.731833in}{1.503171in}}%
\pgfpathcurveto{\pgfqpoint{2.731833in}{1.511408in}}{\pgfqpoint{2.728560in}{1.519308in}}{\pgfqpoint{2.722736in}{1.525132in}}%
\pgfpathcurveto{\pgfqpoint{2.716912in}{1.530956in}}{\pgfqpoint{2.709012in}{1.534228in}}{\pgfqpoint{2.700776in}{1.534228in}}%
\pgfpathcurveto{\pgfqpoint{2.692540in}{1.534228in}}{\pgfqpoint{2.684640in}{1.530956in}}{\pgfqpoint{2.678816in}{1.525132in}}%
\pgfpathcurveto{\pgfqpoint{2.672992in}{1.519308in}}{\pgfqpoint{2.669720in}{1.511408in}}{\pgfqpoint{2.669720in}{1.503171in}}%
\pgfpathcurveto{\pgfqpoint{2.669720in}{1.494935in}}{\pgfqpoint{2.672992in}{1.487035in}}{\pgfqpoint{2.678816in}{1.481211in}}%
\pgfpathcurveto{\pgfqpoint{2.684640in}{1.475387in}}{\pgfqpoint{2.692540in}{1.472115in}}{\pgfqpoint{2.700776in}{1.472115in}}%
\pgfpathclose%
\pgfusepath{stroke,fill}%
\end{pgfscope}%
\begin{pgfscope}%
\pgfpathrectangle{\pgfqpoint{0.100000in}{0.220728in}}{\pgfqpoint{3.696000in}{3.696000in}}%
\pgfusepath{clip}%
\pgfsetbuttcap%
\pgfsetroundjoin%
\definecolor{currentfill}{rgb}{0.121569,0.466667,0.705882}%
\pgfsetfillcolor{currentfill}%
\pgfsetfillopacity{0.905493}%
\pgfsetlinewidth{1.003750pt}%
\definecolor{currentstroke}{rgb}{0.121569,0.466667,0.705882}%
\pgfsetstrokecolor{currentstroke}%
\pgfsetstrokeopacity{0.905493}%
\pgfsetdash{}{0pt}%
\pgfpathmoveto{\pgfqpoint{2.695646in}{1.462757in}}%
\pgfpathcurveto{\pgfqpoint{2.703882in}{1.462757in}}{\pgfqpoint{2.711782in}{1.466029in}}{\pgfqpoint{2.717606in}{1.471853in}}%
\pgfpathcurveto{\pgfqpoint{2.723430in}{1.477677in}}{\pgfqpoint{2.726703in}{1.485577in}}{\pgfqpoint{2.726703in}{1.493813in}}%
\pgfpathcurveto{\pgfqpoint{2.726703in}{1.502049in}}{\pgfqpoint{2.723430in}{1.509950in}}{\pgfqpoint{2.717606in}{1.515773in}}%
\pgfpathcurveto{\pgfqpoint{2.711782in}{1.521597in}}{\pgfqpoint{2.703882in}{1.524870in}}{\pgfqpoint{2.695646in}{1.524870in}}%
\pgfpathcurveto{\pgfqpoint{2.687410in}{1.524870in}}{\pgfqpoint{2.679510in}{1.521597in}}{\pgfqpoint{2.673686in}{1.515773in}}%
\pgfpathcurveto{\pgfqpoint{2.667862in}{1.509950in}}{\pgfqpoint{2.664590in}{1.502049in}}{\pgfqpoint{2.664590in}{1.493813in}}%
\pgfpathcurveto{\pgfqpoint{2.664590in}{1.485577in}}{\pgfqpoint{2.667862in}{1.477677in}}{\pgfqpoint{2.673686in}{1.471853in}}%
\pgfpathcurveto{\pgfqpoint{2.679510in}{1.466029in}}{\pgfqpoint{2.687410in}{1.462757in}}{\pgfqpoint{2.695646in}{1.462757in}}%
\pgfpathclose%
\pgfusepath{stroke,fill}%
\end{pgfscope}%
\begin{pgfscope}%
\pgfpathrectangle{\pgfqpoint{0.100000in}{0.220728in}}{\pgfqpoint{3.696000in}{3.696000in}}%
\pgfusepath{clip}%
\pgfsetbuttcap%
\pgfsetroundjoin%
\definecolor{currentfill}{rgb}{0.121569,0.466667,0.705882}%
\pgfsetfillcolor{currentfill}%
\pgfsetfillopacity{0.906276}%
\pgfsetlinewidth{1.003750pt}%
\definecolor{currentstroke}{rgb}{0.121569,0.466667,0.705882}%
\pgfsetstrokecolor{currentstroke}%
\pgfsetstrokeopacity{0.906276}%
\pgfsetdash{}{0pt}%
\pgfpathmoveto{\pgfqpoint{2.693169in}{1.457171in}}%
\pgfpathcurveto{\pgfqpoint{2.701405in}{1.457171in}}{\pgfqpoint{2.709305in}{1.460444in}}{\pgfqpoint{2.715129in}{1.466268in}}%
\pgfpathcurveto{\pgfqpoint{2.720953in}{1.472092in}}{\pgfqpoint{2.724225in}{1.479992in}}{\pgfqpoint{2.724225in}{1.488228in}}%
\pgfpathcurveto{\pgfqpoint{2.724225in}{1.496464in}}{\pgfqpoint{2.720953in}{1.504364in}}{\pgfqpoint{2.715129in}{1.510188in}}%
\pgfpathcurveto{\pgfqpoint{2.709305in}{1.516012in}}{\pgfqpoint{2.701405in}{1.519284in}}{\pgfqpoint{2.693169in}{1.519284in}}%
\pgfpathcurveto{\pgfqpoint{2.684933in}{1.519284in}}{\pgfqpoint{2.677033in}{1.516012in}}{\pgfqpoint{2.671209in}{1.510188in}}%
\pgfpathcurveto{\pgfqpoint{2.665385in}{1.504364in}}{\pgfqpoint{2.662112in}{1.496464in}}{\pgfqpoint{2.662112in}{1.488228in}}%
\pgfpathcurveto{\pgfqpoint{2.662112in}{1.479992in}}{\pgfqpoint{2.665385in}{1.472092in}}{\pgfqpoint{2.671209in}{1.466268in}}%
\pgfpathcurveto{\pgfqpoint{2.677033in}{1.460444in}}{\pgfqpoint{2.684933in}{1.457171in}}{\pgfqpoint{2.693169in}{1.457171in}}%
\pgfpathclose%
\pgfusepath{stroke,fill}%
\end{pgfscope}%
\begin{pgfscope}%
\pgfpathrectangle{\pgfqpoint{0.100000in}{0.220728in}}{\pgfqpoint{3.696000in}{3.696000in}}%
\pgfusepath{clip}%
\pgfsetbuttcap%
\pgfsetroundjoin%
\definecolor{currentfill}{rgb}{0.121569,0.466667,0.705882}%
\pgfsetfillcolor{currentfill}%
\pgfsetfillopacity{0.906666}%
\pgfsetlinewidth{1.003750pt}%
\definecolor{currentstroke}{rgb}{0.121569,0.466667,0.705882}%
\pgfsetstrokecolor{currentstroke}%
\pgfsetstrokeopacity{0.906666}%
\pgfsetdash{}{0pt}%
\pgfpathmoveto{\pgfqpoint{2.692330in}{1.453485in}}%
\pgfpathcurveto{\pgfqpoint{2.700566in}{1.453485in}}{\pgfqpoint{2.708466in}{1.456758in}}{\pgfqpoint{2.714290in}{1.462582in}}%
\pgfpathcurveto{\pgfqpoint{2.720114in}{1.468405in}}{\pgfqpoint{2.723386in}{1.476305in}}{\pgfqpoint{2.723386in}{1.484542in}}%
\pgfpathcurveto{\pgfqpoint{2.723386in}{1.492778in}}{\pgfqpoint{2.720114in}{1.500678in}}{\pgfqpoint{2.714290in}{1.506502in}}%
\pgfpathcurveto{\pgfqpoint{2.708466in}{1.512326in}}{\pgfqpoint{2.700566in}{1.515598in}}{\pgfqpoint{2.692330in}{1.515598in}}%
\pgfpathcurveto{\pgfqpoint{2.684094in}{1.515598in}}{\pgfqpoint{2.676194in}{1.512326in}}{\pgfqpoint{2.670370in}{1.506502in}}%
\pgfpathcurveto{\pgfqpoint{2.664546in}{1.500678in}}{\pgfqpoint{2.661273in}{1.492778in}}{\pgfqpoint{2.661273in}{1.484542in}}%
\pgfpathcurveto{\pgfqpoint{2.661273in}{1.476305in}}{\pgfqpoint{2.664546in}{1.468405in}}{\pgfqpoint{2.670370in}{1.462582in}}%
\pgfpathcurveto{\pgfqpoint{2.676194in}{1.456758in}}{\pgfqpoint{2.684094in}{1.453485in}}{\pgfqpoint{2.692330in}{1.453485in}}%
\pgfpathclose%
\pgfusepath{stroke,fill}%
\end{pgfscope}%
\begin{pgfscope}%
\pgfpathrectangle{\pgfqpoint{0.100000in}{0.220728in}}{\pgfqpoint{3.696000in}{3.696000in}}%
\pgfusepath{clip}%
\pgfsetbuttcap%
\pgfsetroundjoin%
\definecolor{currentfill}{rgb}{0.121569,0.466667,0.705882}%
\pgfsetfillcolor{currentfill}%
\pgfsetfillopacity{0.907619}%
\pgfsetlinewidth{1.003750pt}%
\definecolor{currentstroke}{rgb}{0.121569,0.466667,0.705882}%
\pgfsetstrokecolor{currentstroke}%
\pgfsetstrokeopacity{0.907619}%
\pgfsetdash{}{0pt}%
\pgfpathmoveto{\pgfqpoint{2.689111in}{1.447348in}}%
\pgfpathcurveto{\pgfqpoint{2.697347in}{1.447348in}}{\pgfqpoint{2.705247in}{1.450620in}}{\pgfqpoint{2.711071in}{1.456444in}}%
\pgfpathcurveto{\pgfqpoint{2.716895in}{1.462268in}}{\pgfqpoint{2.720167in}{1.470168in}}{\pgfqpoint{2.720167in}{1.478404in}}%
\pgfpathcurveto{\pgfqpoint{2.720167in}{1.486641in}}{\pgfqpoint{2.716895in}{1.494541in}}{\pgfqpoint{2.711071in}{1.500365in}}%
\pgfpathcurveto{\pgfqpoint{2.705247in}{1.506189in}}{\pgfqpoint{2.697347in}{1.509461in}}{\pgfqpoint{2.689111in}{1.509461in}}%
\pgfpathcurveto{\pgfqpoint{2.680874in}{1.509461in}}{\pgfqpoint{2.672974in}{1.506189in}}{\pgfqpoint{2.667150in}{1.500365in}}%
\pgfpathcurveto{\pgfqpoint{2.661327in}{1.494541in}}{\pgfqpoint{2.658054in}{1.486641in}}{\pgfqpoint{2.658054in}{1.478404in}}%
\pgfpathcurveto{\pgfqpoint{2.658054in}{1.470168in}}{\pgfqpoint{2.661327in}{1.462268in}}{\pgfqpoint{2.667150in}{1.456444in}}%
\pgfpathcurveto{\pgfqpoint{2.672974in}{1.450620in}}{\pgfqpoint{2.680874in}{1.447348in}}{\pgfqpoint{2.689111in}{1.447348in}}%
\pgfpathclose%
\pgfusepath{stroke,fill}%
\end{pgfscope}%
\begin{pgfscope}%
\pgfpathrectangle{\pgfqpoint{0.100000in}{0.220728in}}{\pgfqpoint{3.696000in}{3.696000in}}%
\pgfusepath{clip}%
\pgfsetbuttcap%
\pgfsetroundjoin%
\definecolor{currentfill}{rgb}{0.121569,0.466667,0.705882}%
\pgfsetfillcolor{currentfill}%
\pgfsetfillopacity{0.907873}%
\pgfsetlinewidth{1.003750pt}%
\definecolor{currentstroke}{rgb}{0.121569,0.466667,0.705882}%
\pgfsetstrokecolor{currentstroke}%
\pgfsetstrokeopacity{0.907873}%
\pgfsetdash{}{0pt}%
\pgfpathmoveto{\pgfqpoint{1.874070in}{0.905196in}}%
\pgfpathcurveto{\pgfqpoint{1.882306in}{0.905196in}}{\pgfqpoint{1.890206in}{0.908468in}}{\pgfqpoint{1.896030in}{0.914292in}}%
\pgfpathcurveto{\pgfqpoint{1.901854in}{0.920116in}}{\pgfqpoint{1.905126in}{0.928016in}}{\pgfqpoint{1.905126in}{0.936252in}}%
\pgfpathcurveto{\pgfqpoint{1.905126in}{0.944488in}}{\pgfqpoint{1.901854in}{0.952388in}}{\pgfqpoint{1.896030in}{0.958212in}}%
\pgfpathcurveto{\pgfqpoint{1.890206in}{0.964036in}}{\pgfqpoint{1.882306in}{0.967309in}}{\pgfqpoint{1.874070in}{0.967309in}}%
\pgfpathcurveto{\pgfqpoint{1.865834in}{0.967309in}}{\pgfqpoint{1.857934in}{0.964036in}}{\pgfqpoint{1.852110in}{0.958212in}}%
\pgfpathcurveto{\pgfqpoint{1.846286in}{0.952388in}}{\pgfqpoint{1.843013in}{0.944488in}}{\pgfqpoint{1.843013in}{0.936252in}}%
\pgfpathcurveto{\pgfqpoint{1.843013in}{0.928016in}}{\pgfqpoint{1.846286in}{0.920116in}}{\pgfqpoint{1.852110in}{0.914292in}}%
\pgfpathcurveto{\pgfqpoint{1.857934in}{0.908468in}}{\pgfqpoint{1.865834in}{0.905196in}}{\pgfqpoint{1.874070in}{0.905196in}}%
\pgfpathclose%
\pgfusepath{stroke,fill}%
\end{pgfscope}%
\begin{pgfscope}%
\pgfpathrectangle{\pgfqpoint{0.100000in}{0.220728in}}{\pgfqpoint{3.696000in}{3.696000in}}%
\pgfusepath{clip}%
\pgfsetbuttcap%
\pgfsetroundjoin%
\definecolor{currentfill}{rgb}{0.121569,0.466667,0.705882}%
\pgfsetfillcolor{currentfill}%
\pgfsetfillopacity{0.908708}%
\pgfsetlinewidth{1.003750pt}%
\definecolor{currentstroke}{rgb}{0.121569,0.466667,0.705882}%
\pgfsetstrokecolor{currentstroke}%
\pgfsetstrokeopacity{0.908708}%
\pgfsetdash{}{0pt}%
\pgfpathmoveto{\pgfqpoint{2.686440in}{1.438795in}}%
\pgfpathcurveto{\pgfqpoint{2.694676in}{1.438795in}}{\pgfqpoint{2.702576in}{1.442067in}}{\pgfqpoint{2.708400in}{1.447891in}}%
\pgfpathcurveto{\pgfqpoint{2.714224in}{1.453715in}}{\pgfqpoint{2.717497in}{1.461615in}}{\pgfqpoint{2.717497in}{1.469851in}}%
\pgfpathcurveto{\pgfqpoint{2.717497in}{1.478087in}}{\pgfqpoint{2.714224in}{1.485987in}}{\pgfqpoint{2.708400in}{1.491811in}}%
\pgfpathcurveto{\pgfqpoint{2.702576in}{1.497635in}}{\pgfqpoint{2.694676in}{1.500908in}}{\pgfqpoint{2.686440in}{1.500908in}}%
\pgfpathcurveto{\pgfqpoint{2.678204in}{1.500908in}}{\pgfqpoint{2.670304in}{1.497635in}}{\pgfqpoint{2.664480in}{1.491811in}}%
\pgfpathcurveto{\pgfqpoint{2.658656in}{1.485987in}}{\pgfqpoint{2.655384in}{1.478087in}}{\pgfqpoint{2.655384in}{1.469851in}}%
\pgfpathcurveto{\pgfqpoint{2.655384in}{1.461615in}}{\pgfqpoint{2.658656in}{1.453715in}}{\pgfqpoint{2.664480in}{1.447891in}}%
\pgfpathcurveto{\pgfqpoint{2.670304in}{1.442067in}}{\pgfqpoint{2.678204in}{1.438795in}}{\pgfqpoint{2.686440in}{1.438795in}}%
\pgfpathclose%
\pgfusepath{stroke,fill}%
\end{pgfscope}%
\begin{pgfscope}%
\pgfpathrectangle{\pgfqpoint{0.100000in}{0.220728in}}{\pgfqpoint{3.696000in}{3.696000in}}%
\pgfusepath{clip}%
\pgfsetbuttcap%
\pgfsetroundjoin%
\definecolor{currentfill}{rgb}{0.121569,0.466667,0.705882}%
\pgfsetfillcolor{currentfill}%
\pgfsetfillopacity{0.909333}%
\pgfsetlinewidth{1.003750pt}%
\definecolor{currentstroke}{rgb}{0.121569,0.466667,0.705882}%
\pgfsetstrokecolor{currentstroke}%
\pgfsetstrokeopacity{0.909333}%
\pgfsetdash{}{0pt}%
\pgfpathmoveto{\pgfqpoint{2.685082in}{1.434110in}}%
\pgfpathcurveto{\pgfqpoint{2.693318in}{1.434110in}}{\pgfqpoint{2.701218in}{1.437382in}}{\pgfqpoint{2.707042in}{1.443206in}}%
\pgfpathcurveto{\pgfqpoint{2.712866in}{1.449030in}}{\pgfqpoint{2.716139in}{1.456930in}}{\pgfqpoint{2.716139in}{1.465166in}}%
\pgfpathcurveto{\pgfqpoint{2.716139in}{1.473403in}}{\pgfqpoint{2.712866in}{1.481303in}}{\pgfqpoint{2.707042in}{1.487127in}}%
\pgfpathcurveto{\pgfqpoint{2.701218in}{1.492950in}}{\pgfqpoint{2.693318in}{1.496223in}}{\pgfqpoint{2.685082in}{1.496223in}}%
\pgfpathcurveto{\pgfqpoint{2.676846in}{1.496223in}}{\pgfqpoint{2.668946in}{1.492950in}}{\pgfqpoint{2.663122in}{1.487127in}}%
\pgfpathcurveto{\pgfqpoint{2.657298in}{1.481303in}}{\pgfqpoint{2.654026in}{1.473403in}}{\pgfqpoint{2.654026in}{1.465166in}}%
\pgfpathcurveto{\pgfqpoint{2.654026in}{1.456930in}}{\pgfqpoint{2.657298in}{1.449030in}}{\pgfqpoint{2.663122in}{1.443206in}}%
\pgfpathcurveto{\pgfqpoint{2.668946in}{1.437382in}}{\pgfqpoint{2.676846in}{1.434110in}}{\pgfqpoint{2.685082in}{1.434110in}}%
\pgfpathclose%
\pgfusepath{stroke,fill}%
\end{pgfscope}%
\begin{pgfscope}%
\pgfpathrectangle{\pgfqpoint{0.100000in}{0.220728in}}{\pgfqpoint{3.696000in}{3.696000in}}%
\pgfusepath{clip}%
\pgfsetbuttcap%
\pgfsetroundjoin%
\definecolor{currentfill}{rgb}{0.121569,0.466667,0.705882}%
\pgfsetfillcolor{currentfill}%
\pgfsetfillopacity{0.910293}%
\pgfsetlinewidth{1.003750pt}%
\definecolor{currentstroke}{rgb}{0.121569,0.466667,0.705882}%
\pgfsetstrokecolor{currentstroke}%
\pgfsetstrokeopacity{0.910293}%
\pgfsetdash{}{0pt}%
\pgfpathmoveto{\pgfqpoint{2.681417in}{1.428725in}}%
\pgfpathcurveto{\pgfqpoint{2.689654in}{1.428725in}}{\pgfqpoint{2.697554in}{1.431997in}}{\pgfqpoint{2.703378in}{1.437821in}}%
\pgfpathcurveto{\pgfqpoint{2.709201in}{1.443645in}}{\pgfqpoint{2.712474in}{1.451545in}}{\pgfqpoint{2.712474in}{1.459781in}}%
\pgfpathcurveto{\pgfqpoint{2.712474in}{1.468017in}}{\pgfqpoint{2.709201in}{1.475918in}}{\pgfqpoint{2.703378in}{1.481741in}}%
\pgfpathcurveto{\pgfqpoint{2.697554in}{1.487565in}}{\pgfqpoint{2.689654in}{1.490838in}}{\pgfqpoint{2.681417in}{1.490838in}}%
\pgfpathcurveto{\pgfqpoint{2.673181in}{1.490838in}}{\pgfqpoint{2.665281in}{1.487565in}}{\pgfqpoint{2.659457in}{1.481741in}}%
\pgfpathcurveto{\pgfqpoint{2.653633in}{1.475918in}}{\pgfqpoint{2.650361in}{1.468017in}}{\pgfqpoint{2.650361in}{1.459781in}}%
\pgfpathcurveto{\pgfqpoint{2.650361in}{1.451545in}}{\pgfqpoint{2.653633in}{1.443645in}}{\pgfqpoint{2.659457in}{1.437821in}}%
\pgfpathcurveto{\pgfqpoint{2.665281in}{1.431997in}}{\pgfqpoint{2.673181in}{1.428725in}}{\pgfqpoint{2.681417in}{1.428725in}}%
\pgfpathclose%
\pgfusepath{stroke,fill}%
\end{pgfscope}%
\begin{pgfscope}%
\pgfpathrectangle{\pgfqpoint{0.100000in}{0.220728in}}{\pgfqpoint{3.696000in}{3.696000in}}%
\pgfusepath{clip}%
\pgfsetbuttcap%
\pgfsetroundjoin%
\definecolor{currentfill}{rgb}{0.121569,0.466667,0.705882}%
\pgfsetfillcolor{currentfill}%
\pgfsetfillopacity{0.911545}%
\pgfsetlinewidth{1.003750pt}%
\definecolor{currentstroke}{rgb}{0.121569,0.466667,0.705882}%
\pgfsetstrokecolor{currentstroke}%
\pgfsetstrokeopacity{0.911545}%
\pgfsetdash{}{0pt}%
\pgfpathmoveto{\pgfqpoint{2.678641in}{1.418728in}}%
\pgfpathcurveto{\pgfqpoint{2.686877in}{1.418728in}}{\pgfqpoint{2.694777in}{1.422000in}}{\pgfqpoint{2.700601in}{1.427824in}}%
\pgfpathcurveto{\pgfqpoint{2.706425in}{1.433648in}}{\pgfqpoint{2.709697in}{1.441548in}}{\pgfqpoint{2.709697in}{1.449784in}}%
\pgfpathcurveto{\pgfqpoint{2.709697in}{1.458021in}}{\pgfqpoint{2.706425in}{1.465921in}}{\pgfqpoint{2.700601in}{1.471745in}}%
\pgfpathcurveto{\pgfqpoint{2.694777in}{1.477568in}}{\pgfqpoint{2.686877in}{1.480841in}}{\pgfqpoint{2.678641in}{1.480841in}}%
\pgfpathcurveto{\pgfqpoint{2.670405in}{1.480841in}}{\pgfqpoint{2.662505in}{1.477568in}}{\pgfqpoint{2.656681in}{1.471745in}}%
\pgfpathcurveto{\pgfqpoint{2.650857in}{1.465921in}}{\pgfqpoint{2.647584in}{1.458021in}}{\pgfqpoint{2.647584in}{1.449784in}}%
\pgfpathcurveto{\pgfqpoint{2.647584in}{1.441548in}}{\pgfqpoint{2.650857in}{1.433648in}}{\pgfqpoint{2.656681in}{1.427824in}}%
\pgfpathcurveto{\pgfqpoint{2.662505in}{1.422000in}}{\pgfqpoint{2.670405in}{1.418728in}}{\pgfqpoint{2.678641in}{1.418728in}}%
\pgfpathclose%
\pgfusepath{stroke,fill}%
\end{pgfscope}%
\begin{pgfscope}%
\pgfpathrectangle{\pgfqpoint{0.100000in}{0.220728in}}{\pgfqpoint{3.696000in}{3.696000in}}%
\pgfusepath{clip}%
\pgfsetbuttcap%
\pgfsetroundjoin%
\definecolor{currentfill}{rgb}{0.121569,0.466667,0.705882}%
\pgfsetfillcolor{currentfill}%
\pgfsetfillopacity{0.911776}%
\pgfsetlinewidth{1.003750pt}%
\definecolor{currentstroke}{rgb}{0.121569,0.466667,0.705882}%
\pgfsetstrokecolor{currentstroke}%
\pgfsetstrokeopacity{0.911776}%
\pgfsetdash{}{0pt}%
\pgfpathmoveto{\pgfqpoint{1.887392in}{0.902569in}}%
\pgfpathcurveto{\pgfqpoint{1.895628in}{0.902569in}}{\pgfqpoint{1.903528in}{0.905841in}}{\pgfqpoint{1.909352in}{0.911665in}}%
\pgfpathcurveto{\pgfqpoint{1.915176in}{0.917489in}}{\pgfqpoint{1.918448in}{0.925389in}}{\pgfqpoint{1.918448in}{0.933625in}}%
\pgfpathcurveto{\pgfqpoint{1.918448in}{0.941862in}}{\pgfqpoint{1.915176in}{0.949762in}}{\pgfqpoint{1.909352in}{0.955586in}}%
\pgfpathcurveto{\pgfqpoint{1.903528in}{0.961410in}}{\pgfqpoint{1.895628in}{0.964682in}}{\pgfqpoint{1.887392in}{0.964682in}}%
\pgfpathcurveto{\pgfqpoint{1.879155in}{0.964682in}}{\pgfqpoint{1.871255in}{0.961410in}}{\pgfqpoint{1.865431in}{0.955586in}}%
\pgfpathcurveto{\pgfqpoint{1.859607in}{0.949762in}}{\pgfqpoint{1.856335in}{0.941862in}}{\pgfqpoint{1.856335in}{0.933625in}}%
\pgfpathcurveto{\pgfqpoint{1.856335in}{0.925389in}}{\pgfqpoint{1.859607in}{0.917489in}}{\pgfqpoint{1.865431in}{0.911665in}}%
\pgfpathcurveto{\pgfqpoint{1.871255in}{0.905841in}}{\pgfqpoint{1.879155in}{0.902569in}}{\pgfqpoint{1.887392in}{0.902569in}}%
\pgfpathclose%
\pgfusepath{stroke,fill}%
\end{pgfscope}%
\begin{pgfscope}%
\pgfpathrectangle{\pgfqpoint{0.100000in}{0.220728in}}{\pgfqpoint{3.696000in}{3.696000in}}%
\pgfusepath{clip}%
\pgfsetbuttcap%
\pgfsetroundjoin%
\definecolor{currentfill}{rgb}{0.121569,0.466667,0.705882}%
\pgfsetfillcolor{currentfill}%
\pgfsetfillopacity{0.912217}%
\pgfsetlinewidth{1.003750pt}%
\definecolor{currentstroke}{rgb}{0.121569,0.466667,0.705882}%
\pgfsetstrokecolor{currentstroke}%
\pgfsetstrokeopacity{0.912217}%
\pgfsetdash{}{0pt}%
\pgfpathmoveto{\pgfqpoint{2.676586in}{1.413632in}}%
\pgfpathcurveto{\pgfqpoint{2.684822in}{1.413632in}}{\pgfqpoint{2.692722in}{1.416904in}}{\pgfqpoint{2.698546in}{1.422728in}}%
\pgfpathcurveto{\pgfqpoint{2.704370in}{1.428552in}}{\pgfqpoint{2.707642in}{1.436452in}}{\pgfqpoint{2.707642in}{1.444688in}}%
\pgfpathcurveto{\pgfqpoint{2.707642in}{1.452925in}}{\pgfqpoint{2.704370in}{1.460825in}}{\pgfqpoint{2.698546in}{1.466649in}}%
\pgfpathcurveto{\pgfqpoint{2.692722in}{1.472473in}}{\pgfqpoint{2.684822in}{1.475745in}}{\pgfqpoint{2.676586in}{1.475745in}}%
\pgfpathcurveto{\pgfqpoint{2.668349in}{1.475745in}}{\pgfqpoint{2.660449in}{1.472473in}}{\pgfqpoint{2.654625in}{1.466649in}}%
\pgfpathcurveto{\pgfqpoint{2.648801in}{1.460825in}}{\pgfqpoint{2.645529in}{1.452925in}}{\pgfqpoint{2.645529in}{1.444688in}}%
\pgfpathcurveto{\pgfqpoint{2.645529in}{1.436452in}}{\pgfqpoint{2.648801in}{1.428552in}}{\pgfqpoint{2.654625in}{1.422728in}}%
\pgfpathcurveto{\pgfqpoint{2.660449in}{1.416904in}}{\pgfqpoint{2.668349in}{1.413632in}}{\pgfqpoint{2.676586in}{1.413632in}}%
\pgfpathclose%
\pgfusepath{stroke,fill}%
\end{pgfscope}%
\begin{pgfscope}%
\pgfpathrectangle{\pgfqpoint{0.100000in}{0.220728in}}{\pgfqpoint{3.696000in}{3.696000in}}%
\pgfusepath{clip}%
\pgfsetbuttcap%
\pgfsetroundjoin%
\definecolor{currentfill}{rgb}{0.121569,0.466667,0.705882}%
\pgfsetfillcolor{currentfill}%
\pgfsetfillopacity{0.912658}%
\pgfsetlinewidth{1.003750pt}%
\definecolor{currentstroke}{rgb}{0.121569,0.466667,0.705882}%
\pgfsetstrokecolor{currentstroke}%
\pgfsetstrokeopacity{0.912658}%
\pgfsetdash{}{0pt}%
\pgfpathmoveto{\pgfqpoint{2.675180in}{1.411470in}}%
\pgfpathcurveto{\pgfqpoint{2.683416in}{1.411470in}}{\pgfqpoint{2.691316in}{1.414743in}}{\pgfqpoint{2.697140in}{1.420567in}}%
\pgfpathcurveto{\pgfqpoint{2.702964in}{1.426391in}}{\pgfqpoint{2.706236in}{1.434291in}}{\pgfqpoint{2.706236in}{1.442527in}}%
\pgfpathcurveto{\pgfqpoint{2.706236in}{1.450763in}}{\pgfqpoint{2.702964in}{1.458663in}}{\pgfqpoint{2.697140in}{1.464487in}}%
\pgfpathcurveto{\pgfqpoint{2.691316in}{1.470311in}}{\pgfqpoint{2.683416in}{1.473583in}}{\pgfqpoint{2.675180in}{1.473583in}}%
\pgfpathcurveto{\pgfqpoint{2.666943in}{1.473583in}}{\pgfqpoint{2.659043in}{1.470311in}}{\pgfqpoint{2.653220in}{1.464487in}}%
\pgfpathcurveto{\pgfqpoint{2.647396in}{1.458663in}}{\pgfqpoint{2.644123in}{1.450763in}}{\pgfqpoint{2.644123in}{1.442527in}}%
\pgfpathcurveto{\pgfqpoint{2.644123in}{1.434291in}}{\pgfqpoint{2.647396in}{1.426391in}}{\pgfqpoint{2.653220in}{1.420567in}}%
\pgfpathcurveto{\pgfqpoint{2.659043in}{1.414743in}}{\pgfqpoint{2.666943in}{1.411470in}}{\pgfqpoint{2.675180in}{1.411470in}}%
\pgfpathclose%
\pgfusepath{stroke,fill}%
\end{pgfscope}%
\begin{pgfscope}%
\pgfpathrectangle{\pgfqpoint{0.100000in}{0.220728in}}{\pgfqpoint{3.696000in}{3.696000in}}%
\pgfusepath{clip}%
\pgfsetbuttcap%
\pgfsetroundjoin%
\definecolor{currentfill}{rgb}{0.121569,0.466667,0.705882}%
\pgfsetfillcolor{currentfill}%
\pgfsetfillopacity{0.913311}%
\pgfsetlinewidth{1.003750pt}%
\definecolor{currentstroke}{rgb}{0.121569,0.466667,0.705882}%
\pgfsetstrokecolor{currentstroke}%
\pgfsetstrokeopacity{0.913311}%
\pgfsetdash{}{0pt}%
\pgfpathmoveto{\pgfqpoint{2.673838in}{1.406005in}}%
\pgfpathcurveto{\pgfqpoint{2.682074in}{1.406005in}}{\pgfqpoint{2.689974in}{1.409277in}}{\pgfqpoint{2.695798in}{1.415101in}}%
\pgfpathcurveto{\pgfqpoint{2.701622in}{1.420925in}}{\pgfqpoint{2.704894in}{1.428825in}}{\pgfqpoint{2.704894in}{1.437061in}}%
\pgfpathcurveto{\pgfqpoint{2.704894in}{1.445298in}}{\pgfqpoint{2.701622in}{1.453198in}}{\pgfqpoint{2.695798in}{1.459021in}}%
\pgfpathcurveto{\pgfqpoint{2.689974in}{1.464845in}}{\pgfqpoint{2.682074in}{1.468118in}}{\pgfqpoint{2.673838in}{1.468118in}}%
\pgfpathcurveto{\pgfqpoint{2.665602in}{1.468118in}}{\pgfqpoint{2.657702in}{1.464845in}}{\pgfqpoint{2.651878in}{1.459021in}}%
\pgfpathcurveto{\pgfqpoint{2.646054in}{1.453198in}}{\pgfqpoint{2.642781in}{1.445298in}}{\pgfqpoint{2.642781in}{1.437061in}}%
\pgfpathcurveto{\pgfqpoint{2.642781in}{1.428825in}}{\pgfqpoint{2.646054in}{1.420925in}}{\pgfqpoint{2.651878in}{1.415101in}}%
\pgfpathcurveto{\pgfqpoint{2.657702in}{1.409277in}}{\pgfqpoint{2.665602in}{1.406005in}}{\pgfqpoint{2.673838in}{1.406005in}}%
\pgfpathclose%
\pgfusepath{stroke,fill}%
\end{pgfscope}%
\begin{pgfscope}%
\pgfpathrectangle{\pgfqpoint{0.100000in}{0.220728in}}{\pgfqpoint{3.696000in}{3.696000in}}%
\pgfusepath{clip}%
\pgfsetbuttcap%
\pgfsetroundjoin%
\definecolor{currentfill}{rgb}{0.121569,0.466667,0.705882}%
\pgfsetfillcolor{currentfill}%
\pgfsetfillopacity{0.913746}%
\pgfsetlinewidth{1.003750pt}%
\definecolor{currentstroke}{rgb}{0.121569,0.466667,0.705882}%
\pgfsetstrokecolor{currentstroke}%
\pgfsetstrokeopacity{0.913746}%
\pgfsetdash{}{0pt}%
\pgfpathmoveto{\pgfqpoint{1.900572in}{0.895762in}}%
\pgfpathcurveto{\pgfqpoint{1.908808in}{0.895762in}}{\pgfqpoint{1.916708in}{0.899034in}}{\pgfqpoint{1.922532in}{0.904858in}}%
\pgfpathcurveto{\pgfqpoint{1.928356in}{0.910682in}}{\pgfqpoint{1.931629in}{0.918582in}}{\pgfqpoint{1.931629in}{0.926818in}}%
\pgfpathcurveto{\pgfqpoint{1.931629in}{0.935054in}}{\pgfqpoint{1.928356in}{0.942954in}}{\pgfqpoint{1.922532in}{0.948778in}}%
\pgfpathcurveto{\pgfqpoint{1.916708in}{0.954602in}}{\pgfqpoint{1.908808in}{0.957875in}}{\pgfqpoint{1.900572in}{0.957875in}}%
\pgfpathcurveto{\pgfqpoint{1.892336in}{0.957875in}}{\pgfqpoint{1.884436in}{0.954602in}}{\pgfqpoint{1.878612in}{0.948778in}}%
\pgfpathcurveto{\pgfqpoint{1.872788in}{0.942954in}}{\pgfqpoint{1.869516in}{0.935054in}}{\pgfqpoint{1.869516in}{0.926818in}}%
\pgfpathcurveto{\pgfqpoint{1.869516in}{0.918582in}}{\pgfqpoint{1.872788in}{0.910682in}}{\pgfqpoint{1.878612in}{0.904858in}}%
\pgfpathcurveto{\pgfqpoint{1.884436in}{0.899034in}}{\pgfqpoint{1.892336in}{0.895762in}}{\pgfqpoint{1.900572in}{0.895762in}}%
\pgfpathclose%
\pgfusepath{stroke,fill}%
\end{pgfscope}%
\begin{pgfscope}%
\pgfpathrectangle{\pgfqpoint{0.100000in}{0.220728in}}{\pgfqpoint{3.696000in}{3.696000in}}%
\pgfusepath{clip}%
\pgfsetbuttcap%
\pgfsetroundjoin%
\definecolor{currentfill}{rgb}{0.121569,0.466667,0.705882}%
\pgfsetfillcolor{currentfill}%
\pgfsetfillopacity{0.914116}%
\pgfsetlinewidth{1.003750pt}%
\definecolor{currentstroke}{rgb}{0.121569,0.466667,0.705882}%
\pgfsetstrokecolor{currentstroke}%
\pgfsetstrokeopacity{0.914116}%
\pgfsetdash{}{0pt}%
\pgfpathmoveto{\pgfqpoint{2.670201in}{1.399172in}}%
\pgfpathcurveto{\pgfqpoint{2.678437in}{1.399172in}}{\pgfqpoint{2.686337in}{1.402445in}}{\pgfqpoint{2.692161in}{1.408269in}}%
\pgfpathcurveto{\pgfqpoint{2.697985in}{1.414093in}}{\pgfqpoint{2.701257in}{1.421993in}}{\pgfqpoint{2.701257in}{1.430229in}}%
\pgfpathcurveto{\pgfqpoint{2.701257in}{1.438465in}}{\pgfqpoint{2.697985in}{1.446365in}}{\pgfqpoint{2.692161in}{1.452189in}}%
\pgfpathcurveto{\pgfqpoint{2.686337in}{1.458013in}}{\pgfqpoint{2.678437in}{1.461285in}}{\pgfqpoint{2.670201in}{1.461285in}}%
\pgfpathcurveto{\pgfqpoint{2.661965in}{1.461285in}}{\pgfqpoint{2.654065in}{1.458013in}}{\pgfqpoint{2.648241in}{1.452189in}}%
\pgfpathcurveto{\pgfqpoint{2.642417in}{1.446365in}}{\pgfqpoint{2.639144in}{1.438465in}}{\pgfqpoint{2.639144in}{1.430229in}}%
\pgfpathcurveto{\pgfqpoint{2.639144in}{1.421993in}}{\pgfqpoint{2.642417in}{1.414093in}}{\pgfqpoint{2.648241in}{1.408269in}}%
\pgfpathcurveto{\pgfqpoint{2.654065in}{1.402445in}}{\pgfqpoint{2.661965in}{1.399172in}}{\pgfqpoint{2.670201in}{1.399172in}}%
\pgfpathclose%
\pgfusepath{stroke,fill}%
\end{pgfscope}%
\begin{pgfscope}%
\pgfpathrectangle{\pgfqpoint{0.100000in}{0.220728in}}{\pgfqpoint{3.696000in}{3.696000in}}%
\pgfusepath{clip}%
\pgfsetbuttcap%
\pgfsetroundjoin%
\definecolor{currentfill}{rgb}{0.121569,0.466667,0.705882}%
\pgfsetfillcolor{currentfill}%
\pgfsetfillopacity{0.915131}%
\pgfsetlinewidth{1.003750pt}%
\definecolor{currentstroke}{rgb}{0.121569,0.466667,0.705882}%
\pgfsetstrokecolor{currentstroke}%
\pgfsetstrokeopacity{0.915131}%
\pgfsetdash{}{0pt}%
\pgfpathmoveto{\pgfqpoint{2.665884in}{1.390140in}}%
\pgfpathcurveto{\pgfqpoint{2.674120in}{1.390140in}}{\pgfqpoint{2.682020in}{1.393412in}}{\pgfqpoint{2.687844in}{1.399236in}}%
\pgfpathcurveto{\pgfqpoint{2.693668in}{1.405060in}}{\pgfqpoint{2.696940in}{1.412960in}}{\pgfqpoint{2.696940in}{1.421197in}}%
\pgfpathcurveto{\pgfqpoint{2.696940in}{1.429433in}}{\pgfqpoint{2.693668in}{1.437333in}}{\pgfqpoint{2.687844in}{1.443157in}}%
\pgfpathcurveto{\pgfqpoint{2.682020in}{1.448981in}}{\pgfqpoint{2.674120in}{1.452253in}}{\pgfqpoint{2.665884in}{1.452253in}}%
\pgfpathcurveto{\pgfqpoint{2.657648in}{1.452253in}}{\pgfqpoint{2.649748in}{1.448981in}}{\pgfqpoint{2.643924in}{1.443157in}}%
\pgfpathcurveto{\pgfqpoint{2.638100in}{1.437333in}}{\pgfqpoint{2.634827in}{1.429433in}}{\pgfqpoint{2.634827in}{1.421197in}}%
\pgfpathcurveto{\pgfqpoint{2.634827in}{1.412960in}}{\pgfqpoint{2.638100in}{1.405060in}}{\pgfqpoint{2.643924in}{1.399236in}}%
\pgfpathcurveto{\pgfqpoint{2.649748in}{1.393412in}}{\pgfqpoint{2.657648in}{1.390140in}}{\pgfqpoint{2.665884in}{1.390140in}}%
\pgfpathclose%
\pgfusepath{stroke,fill}%
\end{pgfscope}%
\begin{pgfscope}%
\pgfpathrectangle{\pgfqpoint{0.100000in}{0.220728in}}{\pgfqpoint{3.696000in}{3.696000in}}%
\pgfusepath{clip}%
\pgfsetbuttcap%
\pgfsetroundjoin%
\definecolor{currentfill}{rgb}{0.121569,0.466667,0.705882}%
\pgfsetfillcolor{currentfill}%
\pgfsetfillopacity{0.916219}%
\pgfsetlinewidth{1.003750pt}%
\definecolor{currentstroke}{rgb}{0.121569,0.466667,0.705882}%
\pgfsetstrokecolor{currentstroke}%
\pgfsetstrokeopacity{0.916219}%
\pgfsetdash{}{0pt}%
\pgfpathmoveto{\pgfqpoint{1.911550in}{0.893919in}}%
\pgfpathcurveto{\pgfqpoint{1.919787in}{0.893919in}}{\pgfqpoint{1.927687in}{0.897191in}}{\pgfqpoint{1.933511in}{0.903015in}}%
\pgfpathcurveto{\pgfqpoint{1.939335in}{0.908839in}}{\pgfqpoint{1.942607in}{0.916739in}}{\pgfqpoint{1.942607in}{0.924975in}}%
\pgfpathcurveto{\pgfqpoint{1.942607in}{0.933212in}}{\pgfqpoint{1.939335in}{0.941112in}}{\pgfqpoint{1.933511in}{0.946936in}}%
\pgfpathcurveto{\pgfqpoint{1.927687in}{0.952760in}}{\pgfqpoint{1.919787in}{0.956032in}}{\pgfqpoint{1.911550in}{0.956032in}}%
\pgfpathcurveto{\pgfqpoint{1.903314in}{0.956032in}}{\pgfqpoint{1.895414in}{0.952760in}}{\pgfqpoint{1.889590in}{0.946936in}}%
\pgfpathcurveto{\pgfqpoint{1.883766in}{0.941112in}}{\pgfqpoint{1.880494in}{0.933212in}}{\pgfqpoint{1.880494in}{0.924975in}}%
\pgfpathcurveto{\pgfqpoint{1.880494in}{0.916739in}}{\pgfqpoint{1.883766in}{0.908839in}}{\pgfqpoint{1.889590in}{0.903015in}}%
\pgfpathcurveto{\pgfqpoint{1.895414in}{0.897191in}}{\pgfqpoint{1.903314in}{0.893919in}}{\pgfqpoint{1.911550in}{0.893919in}}%
\pgfpathclose%
\pgfusepath{stroke,fill}%
\end{pgfscope}%
\begin{pgfscope}%
\pgfpathrectangle{\pgfqpoint{0.100000in}{0.220728in}}{\pgfqpoint{3.696000in}{3.696000in}}%
\pgfusepath{clip}%
\pgfsetbuttcap%
\pgfsetroundjoin%
\definecolor{currentfill}{rgb}{0.121569,0.466667,0.705882}%
\pgfsetfillcolor{currentfill}%
\pgfsetfillopacity{0.916515}%
\pgfsetlinewidth{1.003750pt}%
\definecolor{currentstroke}{rgb}{0.121569,0.466667,0.705882}%
\pgfsetstrokecolor{currentstroke}%
\pgfsetstrokeopacity{0.916515}%
\pgfsetdash{}{0pt}%
\pgfpathmoveto{\pgfqpoint{2.662560in}{1.379924in}}%
\pgfpathcurveto{\pgfqpoint{2.670796in}{1.379924in}}{\pgfqpoint{2.678696in}{1.383196in}}{\pgfqpoint{2.684520in}{1.389020in}}%
\pgfpathcurveto{\pgfqpoint{2.690344in}{1.394844in}}{\pgfqpoint{2.693616in}{1.402744in}}{\pgfqpoint{2.693616in}{1.410980in}}%
\pgfpathcurveto{\pgfqpoint{2.693616in}{1.419217in}}{\pgfqpoint{2.690344in}{1.427117in}}{\pgfqpoint{2.684520in}{1.432941in}}%
\pgfpathcurveto{\pgfqpoint{2.678696in}{1.438764in}}{\pgfqpoint{2.670796in}{1.442037in}}{\pgfqpoint{2.662560in}{1.442037in}}%
\pgfpathcurveto{\pgfqpoint{2.654324in}{1.442037in}}{\pgfqpoint{2.646424in}{1.438764in}}{\pgfqpoint{2.640600in}{1.432941in}}%
\pgfpathcurveto{\pgfqpoint{2.634776in}{1.427117in}}{\pgfqpoint{2.631503in}{1.419217in}}{\pgfqpoint{2.631503in}{1.410980in}}%
\pgfpathcurveto{\pgfqpoint{2.631503in}{1.402744in}}{\pgfqpoint{2.634776in}{1.394844in}}{\pgfqpoint{2.640600in}{1.389020in}}%
\pgfpathcurveto{\pgfqpoint{2.646424in}{1.383196in}}{\pgfqpoint{2.654324in}{1.379924in}}{\pgfqpoint{2.662560in}{1.379924in}}%
\pgfpathclose%
\pgfusepath{stroke,fill}%
\end{pgfscope}%
\begin{pgfscope}%
\pgfpathrectangle{\pgfqpoint{0.100000in}{0.220728in}}{\pgfqpoint{3.696000in}{3.696000in}}%
\pgfusepath{clip}%
\pgfsetbuttcap%
\pgfsetroundjoin%
\definecolor{currentfill}{rgb}{0.121569,0.466667,0.705882}%
\pgfsetfillcolor{currentfill}%
\pgfsetfillopacity{0.917558}%
\pgfsetlinewidth{1.003750pt}%
\definecolor{currentstroke}{rgb}{0.121569,0.466667,0.705882}%
\pgfsetstrokecolor{currentstroke}%
\pgfsetstrokeopacity{0.917558}%
\pgfsetdash{}{0pt}%
\pgfpathmoveto{\pgfqpoint{1.920146in}{0.888819in}}%
\pgfpathcurveto{\pgfqpoint{1.928382in}{0.888819in}}{\pgfqpoint{1.936282in}{0.892091in}}{\pgfqpoint{1.942106in}{0.897915in}}%
\pgfpathcurveto{\pgfqpoint{1.947930in}{0.903739in}}{\pgfqpoint{1.951202in}{0.911639in}}{\pgfqpoint{1.951202in}{0.919876in}}%
\pgfpathcurveto{\pgfqpoint{1.951202in}{0.928112in}}{\pgfqpoint{1.947930in}{0.936012in}}{\pgfqpoint{1.942106in}{0.941836in}}%
\pgfpathcurveto{\pgfqpoint{1.936282in}{0.947660in}}{\pgfqpoint{1.928382in}{0.950932in}}{\pgfqpoint{1.920146in}{0.950932in}}%
\pgfpathcurveto{\pgfqpoint{1.911910in}{0.950932in}}{\pgfqpoint{1.904010in}{0.947660in}}{\pgfqpoint{1.898186in}{0.941836in}}%
\pgfpathcurveto{\pgfqpoint{1.892362in}{0.936012in}}{\pgfqpoint{1.889089in}{0.928112in}}{\pgfqpoint{1.889089in}{0.919876in}}%
\pgfpathcurveto{\pgfqpoint{1.889089in}{0.911639in}}{\pgfqpoint{1.892362in}{0.903739in}}{\pgfqpoint{1.898186in}{0.897915in}}%
\pgfpathcurveto{\pgfqpoint{1.904010in}{0.892091in}}{\pgfqpoint{1.911910in}{0.888819in}}{\pgfqpoint{1.920146in}{0.888819in}}%
\pgfpathclose%
\pgfusepath{stroke,fill}%
\end{pgfscope}%
\begin{pgfscope}%
\pgfpathrectangle{\pgfqpoint{0.100000in}{0.220728in}}{\pgfqpoint{3.696000in}{3.696000in}}%
\pgfusepath{clip}%
\pgfsetbuttcap%
\pgfsetroundjoin%
\definecolor{currentfill}{rgb}{0.121569,0.466667,0.705882}%
\pgfsetfillcolor{currentfill}%
\pgfsetfillopacity{0.917649}%
\pgfsetlinewidth{1.003750pt}%
\definecolor{currentstroke}{rgb}{0.121569,0.466667,0.705882}%
\pgfsetstrokecolor{currentstroke}%
\pgfsetstrokeopacity{0.917649}%
\pgfsetdash{}{0pt}%
\pgfpathmoveto{\pgfqpoint{2.653078in}{1.366577in}}%
\pgfpathcurveto{\pgfqpoint{2.661314in}{1.366577in}}{\pgfqpoint{2.669215in}{1.369850in}}{\pgfqpoint{2.675038in}{1.375674in}}%
\pgfpathcurveto{\pgfqpoint{2.680862in}{1.381498in}}{\pgfqpoint{2.684135in}{1.389398in}}{\pgfqpoint{2.684135in}{1.397634in}}%
\pgfpathcurveto{\pgfqpoint{2.684135in}{1.405870in}}{\pgfqpoint{2.680862in}{1.413770in}}{\pgfqpoint{2.675038in}{1.419594in}}%
\pgfpathcurveto{\pgfqpoint{2.669215in}{1.425418in}}{\pgfqpoint{2.661314in}{1.428690in}}{\pgfqpoint{2.653078in}{1.428690in}}%
\pgfpathcurveto{\pgfqpoint{2.644842in}{1.428690in}}{\pgfqpoint{2.636942in}{1.425418in}}{\pgfqpoint{2.631118in}{1.419594in}}%
\pgfpathcurveto{\pgfqpoint{2.625294in}{1.413770in}}{\pgfqpoint{2.622022in}{1.405870in}}{\pgfqpoint{2.622022in}{1.397634in}}%
\pgfpathcurveto{\pgfqpoint{2.622022in}{1.389398in}}{\pgfqpoint{2.625294in}{1.381498in}}{\pgfqpoint{2.631118in}{1.375674in}}%
\pgfpathcurveto{\pgfqpoint{2.636942in}{1.369850in}}{\pgfqpoint{2.644842in}{1.366577in}}{\pgfqpoint{2.653078in}{1.366577in}}%
\pgfpathclose%
\pgfusepath{stroke,fill}%
\end{pgfscope}%
\begin{pgfscope}%
\pgfpathrectangle{\pgfqpoint{0.100000in}{0.220728in}}{\pgfqpoint{3.696000in}{3.696000in}}%
\pgfusepath{clip}%
\pgfsetbuttcap%
\pgfsetroundjoin%
\definecolor{currentfill}{rgb}{0.121569,0.466667,0.705882}%
\pgfsetfillcolor{currentfill}%
\pgfsetfillopacity{0.919290}%
\pgfsetlinewidth{1.003750pt}%
\definecolor{currentstroke}{rgb}{0.121569,0.466667,0.705882}%
\pgfsetstrokecolor{currentstroke}%
\pgfsetstrokeopacity{0.919290}%
\pgfsetdash{}{0pt}%
\pgfpathmoveto{\pgfqpoint{1.928341in}{0.886509in}}%
\pgfpathcurveto{\pgfqpoint{1.936577in}{0.886509in}}{\pgfqpoint{1.944477in}{0.889782in}}{\pgfqpoint{1.950301in}{0.895606in}}%
\pgfpathcurveto{\pgfqpoint{1.956125in}{0.901429in}}{\pgfqpoint{1.959398in}{0.909330in}}{\pgfqpoint{1.959398in}{0.917566in}}%
\pgfpathcurveto{\pgfqpoint{1.959398in}{0.925802in}}{\pgfqpoint{1.956125in}{0.933702in}}{\pgfqpoint{1.950301in}{0.939526in}}%
\pgfpathcurveto{\pgfqpoint{1.944477in}{0.945350in}}{\pgfqpoint{1.936577in}{0.948622in}}{\pgfqpoint{1.928341in}{0.948622in}}%
\pgfpathcurveto{\pgfqpoint{1.920105in}{0.948622in}}{\pgfqpoint{1.912205in}{0.945350in}}{\pgfqpoint{1.906381in}{0.939526in}}%
\pgfpathcurveto{\pgfqpoint{1.900557in}{0.933702in}}{\pgfqpoint{1.897285in}{0.925802in}}{\pgfqpoint{1.897285in}{0.917566in}}%
\pgfpathcurveto{\pgfqpoint{1.897285in}{0.909330in}}{\pgfqpoint{1.900557in}{0.901429in}}{\pgfqpoint{1.906381in}{0.895606in}}%
\pgfpathcurveto{\pgfqpoint{1.912205in}{0.889782in}}{\pgfqpoint{1.920105in}{0.886509in}}{\pgfqpoint{1.928341in}{0.886509in}}%
\pgfpathclose%
\pgfusepath{stroke,fill}%
\end{pgfscope}%
\begin{pgfscope}%
\pgfpathrectangle{\pgfqpoint{0.100000in}{0.220728in}}{\pgfqpoint{3.696000in}{3.696000in}}%
\pgfusepath{clip}%
\pgfsetbuttcap%
\pgfsetroundjoin%
\definecolor{currentfill}{rgb}{0.121569,0.466667,0.705882}%
\pgfsetfillcolor{currentfill}%
\pgfsetfillopacity{0.920120}%
\pgfsetlinewidth{1.003750pt}%
\definecolor{currentstroke}{rgb}{0.121569,0.466667,0.705882}%
\pgfsetstrokecolor{currentstroke}%
\pgfsetstrokeopacity{0.920120}%
\pgfsetdash{}{0pt}%
\pgfpathmoveto{\pgfqpoint{2.648431in}{1.347658in}}%
\pgfpathcurveto{\pgfqpoint{2.656667in}{1.347658in}}{\pgfqpoint{2.664567in}{1.350930in}}{\pgfqpoint{2.670391in}{1.356754in}}%
\pgfpathcurveto{\pgfqpoint{2.676215in}{1.362578in}}{\pgfqpoint{2.679487in}{1.370478in}}{\pgfqpoint{2.679487in}{1.378715in}}%
\pgfpathcurveto{\pgfqpoint{2.679487in}{1.386951in}}{\pgfqpoint{2.676215in}{1.394851in}}{\pgfqpoint{2.670391in}{1.400675in}}%
\pgfpathcurveto{\pgfqpoint{2.664567in}{1.406499in}}{\pgfqpoint{2.656667in}{1.409771in}}{\pgfqpoint{2.648431in}{1.409771in}}%
\pgfpathcurveto{\pgfqpoint{2.640194in}{1.409771in}}{\pgfqpoint{2.632294in}{1.406499in}}{\pgfqpoint{2.626470in}{1.400675in}}%
\pgfpathcurveto{\pgfqpoint{2.620646in}{1.394851in}}{\pgfqpoint{2.617374in}{1.386951in}}{\pgfqpoint{2.617374in}{1.378715in}}%
\pgfpathcurveto{\pgfqpoint{2.617374in}{1.370478in}}{\pgfqpoint{2.620646in}{1.362578in}}{\pgfqpoint{2.626470in}{1.356754in}}%
\pgfpathcurveto{\pgfqpoint{2.632294in}{1.350930in}}{\pgfqpoint{2.640194in}{1.347658in}}{\pgfqpoint{2.648431in}{1.347658in}}%
\pgfpathclose%
\pgfusepath{stroke,fill}%
\end{pgfscope}%
\begin{pgfscope}%
\pgfpathrectangle{\pgfqpoint{0.100000in}{0.220728in}}{\pgfqpoint{3.696000in}{3.696000in}}%
\pgfusepath{clip}%
\pgfsetbuttcap%
\pgfsetroundjoin%
\definecolor{currentfill}{rgb}{0.121569,0.466667,0.705882}%
\pgfsetfillcolor{currentfill}%
\pgfsetfillopacity{0.921840}%
\pgfsetlinewidth{1.003750pt}%
\definecolor{currentstroke}{rgb}{0.121569,0.466667,0.705882}%
\pgfsetstrokecolor{currentstroke}%
\pgfsetstrokeopacity{0.921840}%
\pgfsetdash{}{0pt}%
\pgfpathmoveto{\pgfqpoint{2.637930in}{1.328777in}}%
\pgfpathcurveto{\pgfqpoint{2.646166in}{1.328777in}}{\pgfqpoint{2.654067in}{1.332050in}}{\pgfqpoint{2.659890in}{1.337874in}}%
\pgfpathcurveto{\pgfqpoint{2.665714in}{1.343698in}}{\pgfqpoint{2.668987in}{1.351598in}}{\pgfqpoint{2.668987in}{1.359834in}}%
\pgfpathcurveto{\pgfqpoint{2.668987in}{1.368070in}}{\pgfqpoint{2.665714in}{1.375970in}}{\pgfqpoint{2.659890in}{1.381794in}}%
\pgfpathcurveto{\pgfqpoint{2.654067in}{1.387618in}}{\pgfqpoint{2.646166in}{1.390890in}}{\pgfqpoint{2.637930in}{1.390890in}}%
\pgfpathcurveto{\pgfqpoint{2.629694in}{1.390890in}}{\pgfqpoint{2.621794in}{1.387618in}}{\pgfqpoint{2.615970in}{1.381794in}}%
\pgfpathcurveto{\pgfqpoint{2.610146in}{1.375970in}}{\pgfqpoint{2.606874in}{1.368070in}}{\pgfqpoint{2.606874in}{1.359834in}}%
\pgfpathcurveto{\pgfqpoint{2.606874in}{1.351598in}}{\pgfqpoint{2.610146in}{1.343698in}}{\pgfqpoint{2.615970in}{1.337874in}}%
\pgfpathcurveto{\pgfqpoint{2.621794in}{1.332050in}}{\pgfqpoint{2.629694in}{1.328777in}}{\pgfqpoint{2.637930in}{1.328777in}}%
\pgfpathclose%
\pgfusepath{stroke,fill}%
\end{pgfscope}%
\begin{pgfscope}%
\pgfpathrectangle{\pgfqpoint{0.100000in}{0.220728in}}{\pgfqpoint{3.696000in}{3.696000in}}%
\pgfusepath{clip}%
\pgfsetbuttcap%
\pgfsetroundjoin%
\definecolor{currentfill}{rgb}{0.121569,0.466667,0.705882}%
\pgfsetfillcolor{currentfill}%
\pgfsetfillopacity{0.922093}%
\pgfsetlinewidth{1.003750pt}%
\definecolor{currentstroke}{rgb}{0.121569,0.466667,0.705882}%
\pgfsetstrokecolor{currentstroke}%
\pgfsetstrokeopacity{0.922093}%
\pgfsetdash{}{0pt}%
\pgfpathmoveto{\pgfqpoint{1.942716in}{0.879144in}}%
\pgfpathcurveto{\pgfqpoint{1.950952in}{0.879144in}}{\pgfqpoint{1.958852in}{0.882416in}}{\pgfqpoint{1.964676in}{0.888240in}}%
\pgfpathcurveto{\pgfqpoint{1.970500in}{0.894064in}}{\pgfqpoint{1.973772in}{0.901964in}}{\pgfqpoint{1.973772in}{0.910200in}}%
\pgfpathcurveto{\pgfqpoint{1.973772in}{0.918437in}}{\pgfqpoint{1.970500in}{0.926337in}}{\pgfqpoint{1.964676in}{0.932161in}}%
\pgfpathcurveto{\pgfqpoint{1.958852in}{0.937985in}}{\pgfqpoint{1.950952in}{0.941257in}}{\pgfqpoint{1.942716in}{0.941257in}}%
\pgfpathcurveto{\pgfqpoint{1.934480in}{0.941257in}}{\pgfqpoint{1.926580in}{0.937985in}}{\pgfqpoint{1.920756in}{0.932161in}}%
\pgfpathcurveto{\pgfqpoint{1.914932in}{0.926337in}}{\pgfqpoint{1.911659in}{0.918437in}}{\pgfqpoint{1.911659in}{0.910200in}}%
\pgfpathcurveto{\pgfqpoint{1.911659in}{0.901964in}}{\pgfqpoint{1.914932in}{0.894064in}}{\pgfqpoint{1.920756in}{0.888240in}}%
\pgfpathcurveto{\pgfqpoint{1.926580in}{0.882416in}}{\pgfqpoint{1.934480in}{0.879144in}}{\pgfqpoint{1.942716in}{0.879144in}}%
\pgfpathclose%
\pgfusepath{stroke,fill}%
\end{pgfscope}%
\begin{pgfscope}%
\pgfpathrectangle{\pgfqpoint{0.100000in}{0.220728in}}{\pgfqpoint{3.696000in}{3.696000in}}%
\pgfusepath{clip}%
\pgfsetbuttcap%
\pgfsetroundjoin%
\definecolor{currentfill}{rgb}{0.121569,0.466667,0.705882}%
\pgfsetfillcolor{currentfill}%
\pgfsetfillopacity{0.923345}%
\pgfsetlinewidth{1.003750pt}%
\definecolor{currentstroke}{rgb}{0.121569,0.466667,0.705882}%
\pgfsetstrokecolor{currentstroke}%
\pgfsetstrokeopacity{0.923345}%
\pgfsetdash{}{0pt}%
\pgfpathmoveto{\pgfqpoint{2.632611in}{1.319876in}}%
\pgfpathcurveto{\pgfqpoint{2.640847in}{1.319876in}}{\pgfqpoint{2.648747in}{1.323148in}}{\pgfqpoint{2.654571in}{1.328972in}}%
\pgfpathcurveto{\pgfqpoint{2.660395in}{1.334796in}}{\pgfqpoint{2.663667in}{1.342696in}}{\pgfqpoint{2.663667in}{1.350932in}}%
\pgfpathcurveto{\pgfqpoint{2.663667in}{1.359169in}}{\pgfqpoint{2.660395in}{1.367069in}}{\pgfqpoint{2.654571in}{1.372893in}}%
\pgfpathcurveto{\pgfqpoint{2.648747in}{1.378717in}}{\pgfqpoint{2.640847in}{1.381989in}}{\pgfqpoint{2.632611in}{1.381989in}}%
\pgfpathcurveto{\pgfqpoint{2.624375in}{1.381989in}}{\pgfqpoint{2.616475in}{1.378717in}}{\pgfqpoint{2.610651in}{1.372893in}}%
\pgfpathcurveto{\pgfqpoint{2.604827in}{1.367069in}}{\pgfqpoint{2.601554in}{1.359169in}}{\pgfqpoint{2.601554in}{1.350932in}}%
\pgfpathcurveto{\pgfqpoint{2.601554in}{1.342696in}}{\pgfqpoint{2.604827in}{1.334796in}}{\pgfqpoint{2.610651in}{1.328972in}}%
\pgfpathcurveto{\pgfqpoint{2.616475in}{1.323148in}}{\pgfqpoint{2.624375in}{1.319876in}}{\pgfqpoint{2.632611in}{1.319876in}}%
\pgfpathclose%
\pgfusepath{stroke,fill}%
\end{pgfscope}%
\begin{pgfscope}%
\pgfpathrectangle{\pgfqpoint{0.100000in}{0.220728in}}{\pgfqpoint{3.696000in}{3.696000in}}%
\pgfusepath{clip}%
\pgfsetbuttcap%
\pgfsetroundjoin%
\definecolor{currentfill}{rgb}{0.121569,0.466667,0.705882}%
\pgfsetfillcolor{currentfill}%
\pgfsetfillopacity{0.924186}%
\pgfsetlinewidth{1.003750pt}%
\definecolor{currentstroke}{rgb}{0.121569,0.466667,0.705882}%
\pgfsetstrokecolor{currentstroke}%
\pgfsetstrokeopacity{0.924186}%
\pgfsetdash{}{0pt}%
\pgfpathmoveto{\pgfqpoint{1.954784in}{0.871728in}}%
\pgfpathcurveto{\pgfqpoint{1.963020in}{0.871728in}}{\pgfqpoint{1.970920in}{0.875000in}}{\pgfqpoint{1.976744in}{0.880824in}}%
\pgfpathcurveto{\pgfqpoint{1.982568in}{0.886648in}}{\pgfqpoint{1.985840in}{0.894548in}}{\pgfqpoint{1.985840in}{0.902784in}}%
\pgfpathcurveto{\pgfqpoint{1.985840in}{0.911020in}}{\pgfqpoint{1.982568in}{0.918920in}}{\pgfqpoint{1.976744in}{0.924744in}}%
\pgfpathcurveto{\pgfqpoint{1.970920in}{0.930568in}}{\pgfqpoint{1.963020in}{0.933841in}}{\pgfqpoint{1.954784in}{0.933841in}}%
\pgfpathcurveto{\pgfqpoint{1.946548in}{0.933841in}}{\pgfqpoint{1.938648in}{0.930568in}}{\pgfqpoint{1.932824in}{0.924744in}}%
\pgfpathcurveto{\pgfqpoint{1.927000in}{0.918920in}}{\pgfqpoint{1.923727in}{0.911020in}}{\pgfqpoint{1.923727in}{0.902784in}}%
\pgfpathcurveto{\pgfqpoint{1.923727in}{0.894548in}}{\pgfqpoint{1.927000in}{0.886648in}}{\pgfqpoint{1.932824in}{0.880824in}}%
\pgfpathcurveto{\pgfqpoint{1.938648in}{0.875000in}}{\pgfqpoint{1.946548in}{0.871728in}}{\pgfqpoint{1.954784in}{0.871728in}}%
\pgfpathclose%
\pgfusepath{stroke,fill}%
\end{pgfscope}%
\begin{pgfscope}%
\pgfpathrectangle{\pgfqpoint{0.100000in}{0.220728in}}{\pgfqpoint{3.696000in}{3.696000in}}%
\pgfusepath{clip}%
\pgfsetbuttcap%
\pgfsetroundjoin%
\definecolor{currentfill}{rgb}{0.121569,0.466667,0.705882}%
\pgfsetfillcolor{currentfill}%
\pgfsetfillopacity{0.924701}%
\pgfsetlinewidth{1.003750pt}%
\definecolor{currentstroke}{rgb}{0.121569,0.466667,0.705882}%
\pgfsetstrokecolor{currentstroke}%
\pgfsetstrokeopacity{0.924701}%
\pgfsetdash{}{0pt}%
\pgfpathmoveto{\pgfqpoint{2.628870in}{1.306030in}}%
\pgfpathcurveto{\pgfqpoint{2.637107in}{1.306030in}}{\pgfqpoint{2.645007in}{1.309303in}}{\pgfqpoint{2.650831in}{1.315127in}}%
\pgfpathcurveto{\pgfqpoint{2.656655in}{1.320950in}}{\pgfqpoint{2.659927in}{1.328850in}}{\pgfqpoint{2.659927in}{1.337087in}}%
\pgfpathcurveto{\pgfqpoint{2.659927in}{1.345323in}}{\pgfqpoint{2.656655in}{1.353223in}}{\pgfqpoint{2.650831in}{1.359047in}}%
\pgfpathcurveto{\pgfqpoint{2.645007in}{1.364871in}}{\pgfqpoint{2.637107in}{1.368143in}}{\pgfqpoint{2.628870in}{1.368143in}}%
\pgfpathcurveto{\pgfqpoint{2.620634in}{1.368143in}}{\pgfqpoint{2.612734in}{1.364871in}}{\pgfqpoint{2.606910in}{1.359047in}}%
\pgfpathcurveto{\pgfqpoint{2.601086in}{1.353223in}}{\pgfqpoint{2.597814in}{1.345323in}}{\pgfqpoint{2.597814in}{1.337087in}}%
\pgfpathcurveto{\pgfqpoint{2.597814in}{1.328850in}}{\pgfqpoint{2.601086in}{1.320950in}}{\pgfqpoint{2.606910in}{1.315127in}}%
\pgfpathcurveto{\pgfqpoint{2.612734in}{1.309303in}}{\pgfqpoint{2.620634in}{1.306030in}}{\pgfqpoint{2.628870in}{1.306030in}}%
\pgfpathclose%
\pgfusepath{stroke,fill}%
\end{pgfscope}%
\begin{pgfscope}%
\pgfpathrectangle{\pgfqpoint{0.100000in}{0.220728in}}{\pgfqpoint{3.696000in}{3.696000in}}%
\pgfusepath{clip}%
\pgfsetbuttcap%
\pgfsetroundjoin%
\definecolor{currentfill}{rgb}{0.121569,0.466667,0.705882}%
\pgfsetfillcolor{currentfill}%
\pgfsetfillopacity{0.926592}%
\pgfsetlinewidth{1.003750pt}%
\definecolor{currentstroke}{rgb}{0.121569,0.466667,0.705882}%
\pgfsetstrokecolor{currentstroke}%
\pgfsetstrokeopacity{0.926592}%
\pgfsetdash{}{0pt}%
\pgfpathmoveto{\pgfqpoint{1.964367in}{0.869414in}}%
\pgfpathcurveto{\pgfqpoint{1.972603in}{0.869414in}}{\pgfqpoint{1.980504in}{0.872686in}}{\pgfqpoint{1.986327in}{0.878510in}}%
\pgfpathcurveto{\pgfqpoint{1.992151in}{0.884334in}}{\pgfqpoint{1.995424in}{0.892234in}}{\pgfqpoint{1.995424in}{0.900471in}}%
\pgfpathcurveto{\pgfqpoint{1.995424in}{0.908707in}}{\pgfqpoint{1.992151in}{0.916607in}}{\pgfqpoint{1.986327in}{0.922431in}}%
\pgfpathcurveto{\pgfqpoint{1.980504in}{0.928255in}}{\pgfqpoint{1.972603in}{0.931527in}}{\pgfqpoint{1.964367in}{0.931527in}}%
\pgfpathcurveto{\pgfqpoint{1.956131in}{0.931527in}}{\pgfqpoint{1.948231in}{0.928255in}}{\pgfqpoint{1.942407in}{0.922431in}}%
\pgfpathcurveto{\pgfqpoint{1.936583in}{0.916607in}}{\pgfqpoint{1.933311in}{0.908707in}}{\pgfqpoint{1.933311in}{0.900471in}}%
\pgfpathcurveto{\pgfqpoint{1.933311in}{0.892234in}}{\pgfqpoint{1.936583in}{0.884334in}}{\pgfqpoint{1.942407in}{0.878510in}}%
\pgfpathcurveto{\pgfqpoint{1.948231in}{0.872686in}}{\pgfqpoint{1.956131in}{0.869414in}}{\pgfqpoint{1.964367in}{0.869414in}}%
\pgfpathclose%
\pgfusepath{stroke,fill}%
\end{pgfscope}%
\begin{pgfscope}%
\pgfpathrectangle{\pgfqpoint{0.100000in}{0.220728in}}{\pgfqpoint{3.696000in}{3.696000in}}%
\pgfusepath{clip}%
\pgfsetbuttcap%
\pgfsetroundjoin%
\definecolor{currentfill}{rgb}{0.121569,0.466667,0.705882}%
\pgfsetfillcolor{currentfill}%
\pgfsetfillopacity{0.926735}%
\pgfsetlinewidth{1.003750pt}%
\definecolor{currentstroke}{rgb}{0.121569,0.466667,0.705882}%
\pgfsetstrokecolor{currentstroke}%
\pgfsetstrokeopacity{0.926735}%
\pgfsetdash{}{0pt}%
\pgfpathmoveto{\pgfqpoint{2.620654in}{1.292710in}}%
\pgfpathcurveto{\pgfqpoint{2.628891in}{1.292710in}}{\pgfqpoint{2.636791in}{1.295982in}}{\pgfqpoint{2.642615in}{1.301806in}}%
\pgfpathcurveto{\pgfqpoint{2.648439in}{1.307630in}}{\pgfqpoint{2.651711in}{1.315530in}}{\pgfqpoint{2.651711in}{1.323767in}}%
\pgfpathcurveto{\pgfqpoint{2.651711in}{1.332003in}}{\pgfqpoint{2.648439in}{1.339903in}}{\pgfqpoint{2.642615in}{1.345727in}}%
\pgfpathcurveto{\pgfqpoint{2.636791in}{1.351551in}}{\pgfqpoint{2.628891in}{1.354823in}}{\pgfqpoint{2.620654in}{1.354823in}}%
\pgfpathcurveto{\pgfqpoint{2.612418in}{1.354823in}}{\pgfqpoint{2.604518in}{1.351551in}}{\pgfqpoint{2.598694in}{1.345727in}}%
\pgfpathcurveto{\pgfqpoint{2.592870in}{1.339903in}}{\pgfqpoint{2.589598in}{1.332003in}}{\pgfqpoint{2.589598in}{1.323767in}}%
\pgfpathcurveto{\pgfqpoint{2.589598in}{1.315530in}}{\pgfqpoint{2.592870in}{1.307630in}}{\pgfqpoint{2.598694in}{1.301806in}}%
\pgfpathcurveto{\pgfqpoint{2.604518in}{1.295982in}}{\pgfqpoint{2.612418in}{1.292710in}}{\pgfqpoint{2.620654in}{1.292710in}}%
\pgfpathclose%
\pgfusepath{stroke,fill}%
\end{pgfscope}%
\begin{pgfscope}%
\pgfpathrectangle{\pgfqpoint{0.100000in}{0.220728in}}{\pgfqpoint{3.696000in}{3.696000in}}%
\pgfusepath{clip}%
\pgfsetbuttcap%
\pgfsetroundjoin%
\definecolor{currentfill}{rgb}{0.121569,0.466667,0.705882}%
\pgfsetfillcolor{currentfill}%
\pgfsetfillopacity{0.927810}%
\pgfsetlinewidth{1.003750pt}%
\definecolor{currentstroke}{rgb}{0.121569,0.466667,0.705882}%
\pgfsetstrokecolor{currentstroke}%
\pgfsetstrokeopacity{0.927810}%
\pgfsetdash{}{0pt}%
\pgfpathmoveto{\pgfqpoint{1.972930in}{0.864424in}}%
\pgfpathcurveto{\pgfqpoint{1.981166in}{0.864424in}}{\pgfqpoint{1.989066in}{0.867696in}}{\pgfqpoint{1.994890in}{0.873520in}}%
\pgfpathcurveto{\pgfqpoint{2.000714in}{0.879344in}}{\pgfqpoint{2.003986in}{0.887244in}}{\pgfqpoint{2.003986in}{0.895480in}}%
\pgfpathcurveto{\pgfqpoint{2.003986in}{0.903717in}}{\pgfqpoint{2.000714in}{0.911617in}}{\pgfqpoint{1.994890in}{0.917441in}}%
\pgfpathcurveto{\pgfqpoint{1.989066in}{0.923265in}}{\pgfqpoint{1.981166in}{0.926537in}}{\pgfqpoint{1.972930in}{0.926537in}}%
\pgfpathcurveto{\pgfqpoint{1.964693in}{0.926537in}}{\pgfqpoint{1.956793in}{0.923265in}}{\pgfqpoint{1.950969in}{0.917441in}}%
\pgfpathcurveto{\pgfqpoint{1.945145in}{0.911617in}}{\pgfqpoint{1.941873in}{0.903717in}}{\pgfqpoint{1.941873in}{0.895480in}}%
\pgfpathcurveto{\pgfqpoint{1.941873in}{0.887244in}}{\pgfqpoint{1.945145in}{0.879344in}}{\pgfqpoint{1.950969in}{0.873520in}}%
\pgfpathcurveto{\pgfqpoint{1.956793in}{0.867696in}}{\pgfqpoint{1.964693in}{0.864424in}}{\pgfqpoint{1.972930in}{0.864424in}}%
\pgfpathclose%
\pgfusepath{stroke,fill}%
\end{pgfscope}%
\begin{pgfscope}%
\pgfpathrectangle{\pgfqpoint{0.100000in}{0.220728in}}{\pgfqpoint{3.696000in}{3.696000in}}%
\pgfusepath{clip}%
\pgfsetbuttcap%
\pgfsetroundjoin%
\definecolor{currentfill}{rgb}{0.121569,0.466667,0.705882}%
\pgfsetfillcolor{currentfill}%
\pgfsetfillopacity{0.928735}%
\pgfsetlinewidth{1.003750pt}%
\definecolor{currentstroke}{rgb}{0.121569,0.466667,0.705882}%
\pgfsetstrokecolor{currentstroke}%
\pgfsetstrokeopacity{0.928735}%
\pgfsetdash{}{0pt}%
\pgfpathmoveto{\pgfqpoint{1.980997in}{0.858872in}}%
\pgfpathcurveto{\pgfqpoint{1.989233in}{0.858872in}}{\pgfqpoint{1.997133in}{0.862144in}}{\pgfqpoint{2.002957in}{0.867968in}}%
\pgfpathcurveto{\pgfqpoint{2.008781in}{0.873792in}}{\pgfqpoint{2.012053in}{0.881692in}}{\pgfqpoint{2.012053in}{0.889929in}}%
\pgfpathcurveto{\pgfqpoint{2.012053in}{0.898165in}}{\pgfqpoint{2.008781in}{0.906065in}}{\pgfqpoint{2.002957in}{0.911889in}}%
\pgfpathcurveto{\pgfqpoint{1.997133in}{0.917713in}}{\pgfqpoint{1.989233in}{0.920985in}}{\pgfqpoint{1.980997in}{0.920985in}}%
\pgfpathcurveto{\pgfqpoint{1.972760in}{0.920985in}}{\pgfqpoint{1.964860in}{0.917713in}}{\pgfqpoint{1.959037in}{0.911889in}}%
\pgfpathcurveto{\pgfqpoint{1.953213in}{0.906065in}}{\pgfqpoint{1.949940in}{0.898165in}}{\pgfqpoint{1.949940in}{0.889929in}}%
\pgfpathcurveto{\pgfqpoint{1.949940in}{0.881692in}}{\pgfqpoint{1.953213in}{0.873792in}}{\pgfqpoint{1.959037in}{0.867968in}}%
\pgfpathcurveto{\pgfqpoint{1.964860in}{0.862144in}}{\pgfqpoint{1.972760in}{0.858872in}}{\pgfqpoint{1.980997in}{0.858872in}}%
\pgfpathclose%
\pgfusepath{stroke,fill}%
\end{pgfscope}%
\begin{pgfscope}%
\pgfpathrectangle{\pgfqpoint{0.100000in}{0.220728in}}{\pgfqpoint{3.696000in}{3.696000in}}%
\pgfusepath{clip}%
\pgfsetbuttcap%
\pgfsetroundjoin%
\definecolor{currentfill}{rgb}{0.121569,0.466667,0.705882}%
\pgfsetfillcolor{currentfill}%
\pgfsetfillopacity{0.929110}%
\pgfsetlinewidth{1.003750pt}%
\definecolor{currentstroke}{rgb}{0.121569,0.466667,0.705882}%
\pgfsetstrokecolor{currentstroke}%
\pgfsetstrokeopacity{0.929110}%
\pgfsetdash{}{0pt}%
\pgfpathmoveto{\pgfqpoint{2.614363in}{1.275351in}}%
\pgfpathcurveto{\pgfqpoint{2.622600in}{1.275351in}}{\pgfqpoint{2.630500in}{1.278623in}}{\pgfqpoint{2.636324in}{1.284447in}}%
\pgfpathcurveto{\pgfqpoint{2.642147in}{1.290271in}}{\pgfqpoint{2.645420in}{1.298171in}}{\pgfqpoint{2.645420in}{1.306407in}}%
\pgfpathcurveto{\pgfqpoint{2.645420in}{1.314644in}}{\pgfqpoint{2.642147in}{1.322544in}}{\pgfqpoint{2.636324in}{1.328368in}}%
\pgfpathcurveto{\pgfqpoint{2.630500in}{1.334192in}}{\pgfqpoint{2.622600in}{1.337464in}}{\pgfqpoint{2.614363in}{1.337464in}}%
\pgfpathcurveto{\pgfqpoint{2.606127in}{1.337464in}}{\pgfqpoint{2.598227in}{1.334192in}}{\pgfqpoint{2.592403in}{1.328368in}}%
\pgfpathcurveto{\pgfqpoint{2.586579in}{1.322544in}}{\pgfqpoint{2.583307in}{1.314644in}}{\pgfqpoint{2.583307in}{1.306407in}}%
\pgfpathcurveto{\pgfqpoint{2.583307in}{1.298171in}}{\pgfqpoint{2.586579in}{1.290271in}}{\pgfqpoint{2.592403in}{1.284447in}}%
\pgfpathcurveto{\pgfqpoint{2.598227in}{1.278623in}}{\pgfqpoint{2.606127in}{1.275351in}}{\pgfqpoint{2.614363in}{1.275351in}}%
\pgfpathclose%
\pgfusepath{stroke,fill}%
\end{pgfscope}%
\begin{pgfscope}%
\pgfpathrectangle{\pgfqpoint{0.100000in}{0.220728in}}{\pgfqpoint{3.696000in}{3.696000in}}%
\pgfusepath{clip}%
\pgfsetbuttcap%
\pgfsetroundjoin%
\definecolor{currentfill}{rgb}{0.121569,0.466667,0.705882}%
\pgfsetfillcolor{currentfill}%
\pgfsetfillopacity{0.930249}%
\pgfsetlinewidth{1.003750pt}%
\definecolor{currentstroke}{rgb}{0.121569,0.466667,0.705882}%
\pgfsetstrokecolor{currentstroke}%
\pgfsetstrokeopacity{0.930249}%
\pgfsetdash{}{0pt}%
\pgfpathmoveto{\pgfqpoint{1.987886in}{0.855568in}}%
\pgfpathcurveto{\pgfqpoint{1.996123in}{0.855568in}}{\pgfqpoint{2.004023in}{0.858841in}}{\pgfqpoint{2.009847in}{0.864665in}}%
\pgfpathcurveto{\pgfqpoint{2.015671in}{0.870489in}}{\pgfqpoint{2.018943in}{0.878389in}}{\pgfqpoint{2.018943in}{0.886625in}}%
\pgfpathcurveto{\pgfqpoint{2.018943in}{0.894861in}}{\pgfqpoint{2.015671in}{0.902761in}}{\pgfqpoint{2.009847in}{0.908585in}}%
\pgfpathcurveto{\pgfqpoint{2.004023in}{0.914409in}}{\pgfqpoint{1.996123in}{0.917681in}}{\pgfqpoint{1.987886in}{0.917681in}}%
\pgfpathcurveto{\pgfqpoint{1.979650in}{0.917681in}}{\pgfqpoint{1.971750in}{0.914409in}}{\pgfqpoint{1.965926in}{0.908585in}}%
\pgfpathcurveto{\pgfqpoint{1.960102in}{0.902761in}}{\pgfqpoint{1.956830in}{0.894861in}}{\pgfqpoint{1.956830in}{0.886625in}}%
\pgfpathcurveto{\pgfqpoint{1.956830in}{0.878389in}}{\pgfqpoint{1.960102in}{0.870489in}}{\pgfqpoint{1.965926in}{0.864665in}}%
\pgfpathcurveto{\pgfqpoint{1.971750in}{0.858841in}}{\pgfqpoint{1.979650in}{0.855568in}}{\pgfqpoint{1.987886in}{0.855568in}}%
\pgfpathclose%
\pgfusepath{stroke,fill}%
\end{pgfscope}%
\begin{pgfscope}%
\pgfpathrectangle{\pgfqpoint{0.100000in}{0.220728in}}{\pgfqpoint{3.696000in}{3.696000in}}%
\pgfusepath{clip}%
\pgfsetbuttcap%
\pgfsetroundjoin%
\definecolor{currentfill}{rgb}{0.121569,0.466667,0.705882}%
\pgfsetfillcolor{currentfill}%
\pgfsetfillopacity{0.931169}%
\pgfsetlinewidth{1.003750pt}%
\definecolor{currentstroke}{rgb}{0.121569,0.466667,0.705882}%
\pgfsetstrokecolor{currentstroke}%
\pgfsetstrokeopacity{0.931169}%
\pgfsetdash{}{0pt}%
\pgfpathmoveto{\pgfqpoint{1.993140in}{0.852114in}}%
\pgfpathcurveto{\pgfqpoint{2.001376in}{0.852114in}}{\pgfqpoint{2.009276in}{0.855386in}}{\pgfqpoint{2.015100in}{0.861210in}}%
\pgfpathcurveto{\pgfqpoint{2.020924in}{0.867034in}}{\pgfqpoint{2.024197in}{0.874934in}}{\pgfqpoint{2.024197in}{0.883171in}}%
\pgfpathcurveto{\pgfqpoint{2.024197in}{0.891407in}}{\pgfqpoint{2.020924in}{0.899307in}}{\pgfqpoint{2.015100in}{0.905131in}}%
\pgfpathcurveto{\pgfqpoint{2.009276in}{0.910955in}}{\pgfqpoint{2.001376in}{0.914227in}}{\pgfqpoint{1.993140in}{0.914227in}}%
\pgfpathcurveto{\pgfqpoint{1.984904in}{0.914227in}}{\pgfqpoint{1.977004in}{0.910955in}}{\pgfqpoint{1.971180in}{0.905131in}}%
\pgfpathcurveto{\pgfqpoint{1.965356in}{0.899307in}}{\pgfqpoint{1.962084in}{0.891407in}}{\pgfqpoint{1.962084in}{0.883171in}}%
\pgfpathcurveto{\pgfqpoint{1.962084in}{0.874934in}}{\pgfqpoint{1.965356in}{0.867034in}}{\pgfqpoint{1.971180in}{0.861210in}}%
\pgfpathcurveto{\pgfqpoint{1.977004in}{0.855386in}}{\pgfqpoint{1.984904in}{0.852114in}}{\pgfqpoint{1.993140in}{0.852114in}}%
\pgfpathclose%
\pgfusepath{stroke,fill}%
\end{pgfscope}%
\begin{pgfscope}%
\pgfpathrectangle{\pgfqpoint{0.100000in}{0.220728in}}{\pgfqpoint{3.696000in}{3.696000in}}%
\pgfusepath{clip}%
\pgfsetbuttcap%
\pgfsetroundjoin%
\definecolor{currentfill}{rgb}{0.121569,0.466667,0.705882}%
\pgfsetfillcolor{currentfill}%
\pgfsetfillopacity{0.931864}%
\pgfsetlinewidth{1.003750pt}%
\definecolor{currentstroke}{rgb}{0.121569,0.466667,0.705882}%
\pgfsetstrokecolor{currentstroke}%
\pgfsetstrokeopacity{0.931864}%
\pgfsetdash{}{0pt}%
\pgfpathmoveto{\pgfqpoint{2.607356in}{1.257901in}}%
\pgfpathcurveto{\pgfqpoint{2.615592in}{1.257901in}}{\pgfqpoint{2.623492in}{1.261174in}}{\pgfqpoint{2.629316in}{1.266998in}}%
\pgfpathcurveto{\pgfqpoint{2.635140in}{1.272822in}}{\pgfqpoint{2.638413in}{1.280722in}}{\pgfqpoint{2.638413in}{1.288958in}}%
\pgfpathcurveto{\pgfqpoint{2.638413in}{1.297194in}}{\pgfqpoint{2.635140in}{1.305094in}}{\pgfqpoint{2.629316in}{1.310918in}}%
\pgfpathcurveto{\pgfqpoint{2.623492in}{1.316742in}}{\pgfqpoint{2.615592in}{1.320014in}}{\pgfqpoint{2.607356in}{1.320014in}}%
\pgfpathcurveto{\pgfqpoint{2.599120in}{1.320014in}}{\pgfqpoint{2.591220in}{1.316742in}}{\pgfqpoint{2.585396in}{1.310918in}}%
\pgfpathcurveto{\pgfqpoint{2.579572in}{1.305094in}}{\pgfqpoint{2.576300in}{1.297194in}}{\pgfqpoint{2.576300in}{1.288958in}}%
\pgfpathcurveto{\pgfqpoint{2.576300in}{1.280722in}}{\pgfqpoint{2.579572in}{1.272822in}}{\pgfqpoint{2.585396in}{1.266998in}}%
\pgfpathcurveto{\pgfqpoint{2.591220in}{1.261174in}}{\pgfqpoint{2.599120in}{1.257901in}}{\pgfqpoint{2.607356in}{1.257901in}}%
\pgfpathclose%
\pgfusepath{stroke,fill}%
\end{pgfscope}%
\begin{pgfscope}%
\pgfpathrectangle{\pgfqpoint{0.100000in}{0.220728in}}{\pgfqpoint{3.696000in}{3.696000in}}%
\pgfusepath{clip}%
\pgfsetbuttcap%
\pgfsetroundjoin%
\definecolor{currentfill}{rgb}{0.121569,0.466667,0.705882}%
\pgfsetfillcolor{currentfill}%
\pgfsetfillopacity{0.932868}%
\pgfsetlinewidth{1.003750pt}%
\definecolor{currentstroke}{rgb}{0.121569,0.466667,0.705882}%
\pgfsetstrokecolor{currentstroke}%
\pgfsetstrokeopacity{0.932868}%
\pgfsetdash{}{0pt}%
\pgfpathmoveto{\pgfqpoint{2.601619in}{1.248954in}}%
\pgfpathcurveto{\pgfqpoint{2.609856in}{1.248954in}}{\pgfqpoint{2.617756in}{1.252227in}}{\pgfqpoint{2.623580in}{1.258051in}}%
\pgfpathcurveto{\pgfqpoint{2.629404in}{1.263875in}}{\pgfqpoint{2.632676in}{1.271775in}}{\pgfqpoint{2.632676in}{1.280011in}}%
\pgfpathcurveto{\pgfqpoint{2.632676in}{1.288247in}}{\pgfqpoint{2.629404in}{1.296147in}}{\pgfqpoint{2.623580in}{1.301971in}}%
\pgfpathcurveto{\pgfqpoint{2.617756in}{1.307795in}}{\pgfqpoint{2.609856in}{1.311067in}}{\pgfqpoint{2.601619in}{1.311067in}}%
\pgfpathcurveto{\pgfqpoint{2.593383in}{1.311067in}}{\pgfqpoint{2.585483in}{1.307795in}}{\pgfqpoint{2.579659in}{1.301971in}}%
\pgfpathcurveto{\pgfqpoint{2.573835in}{1.296147in}}{\pgfqpoint{2.570563in}{1.288247in}}{\pgfqpoint{2.570563in}{1.280011in}}%
\pgfpathcurveto{\pgfqpoint{2.570563in}{1.271775in}}{\pgfqpoint{2.573835in}{1.263875in}}{\pgfqpoint{2.579659in}{1.258051in}}%
\pgfpathcurveto{\pgfqpoint{2.585483in}{1.252227in}}{\pgfqpoint{2.593383in}{1.248954in}}{\pgfqpoint{2.601619in}{1.248954in}}%
\pgfpathclose%
\pgfusepath{stroke,fill}%
\end{pgfscope}%
\begin{pgfscope}%
\pgfpathrectangle{\pgfqpoint{0.100000in}{0.220728in}}{\pgfqpoint{3.696000in}{3.696000in}}%
\pgfusepath{clip}%
\pgfsetbuttcap%
\pgfsetroundjoin%
\definecolor{currentfill}{rgb}{0.121569,0.466667,0.705882}%
\pgfsetfillcolor{currentfill}%
\pgfsetfillopacity{0.933597}%
\pgfsetlinewidth{1.003750pt}%
\definecolor{currentstroke}{rgb}{0.121569,0.466667,0.705882}%
\pgfsetstrokecolor{currentstroke}%
\pgfsetstrokeopacity{0.933597}%
\pgfsetdash{}{0pt}%
\pgfpathmoveto{\pgfqpoint{2.003030in}{0.849815in}}%
\pgfpathcurveto{\pgfqpoint{2.011267in}{0.849815in}}{\pgfqpoint{2.019167in}{0.853087in}}{\pgfqpoint{2.024991in}{0.858911in}}%
\pgfpathcurveto{\pgfqpoint{2.030814in}{0.864735in}}{\pgfqpoint{2.034087in}{0.872635in}}{\pgfqpoint{2.034087in}{0.880871in}}%
\pgfpathcurveto{\pgfqpoint{2.034087in}{0.889108in}}{\pgfqpoint{2.030814in}{0.897008in}}{\pgfqpoint{2.024991in}{0.902831in}}%
\pgfpathcurveto{\pgfqpoint{2.019167in}{0.908655in}}{\pgfqpoint{2.011267in}{0.911928in}}{\pgfqpoint{2.003030in}{0.911928in}}%
\pgfpathcurveto{\pgfqpoint{1.994794in}{0.911928in}}{\pgfqpoint{1.986894in}{0.908655in}}{\pgfqpoint{1.981070in}{0.902831in}}%
\pgfpathcurveto{\pgfqpoint{1.975246in}{0.897008in}}{\pgfqpoint{1.971974in}{0.889108in}}{\pgfqpoint{1.971974in}{0.880871in}}%
\pgfpathcurveto{\pgfqpoint{1.971974in}{0.872635in}}{\pgfqpoint{1.975246in}{0.864735in}}{\pgfqpoint{1.981070in}{0.858911in}}%
\pgfpathcurveto{\pgfqpoint{1.986894in}{0.853087in}}{\pgfqpoint{1.994794in}{0.849815in}}{\pgfqpoint{2.003030in}{0.849815in}}%
\pgfpathclose%
\pgfusepath{stroke,fill}%
\end{pgfscope}%
\begin{pgfscope}%
\pgfpathrectangle{\pgfqpoint{0.100000in}{0.220728in}}{\pgfqpoint{3.696000in}{3.696000in}}%
\pgfusepath{clip}%
\pgfsetbuttcap%
\pgfsetroundjoin%
\definecolor{currentfill}{rgb}{0.121569,0.466667,0.705882}%
\pgfsetfillcolor{currentfill}%
\pgfsetfillopacity{0.934807}%
\pgfsetlinewidth{1.003750pt}%
\definecolor{currentstroke}{rgb}{0.121569,0.466667,0.705882}%
\pgfsetstrokecolor{currentstroke}%
\pgfsetstrokeopacity{0.934807}%
\pgfsetdash{}{0pt}%
\pgfpathmoveto{\pgfqpoint{2.598010in}{1.232843in}}%
\pgfpathcurveto{\pgfqpoint{2.606246in}{1.232843in}}{\pgfqpoint{2.614146in}{1.236115in}}{\pgfqpoint{2.619970in}{1.241939in}}%
\pgfpathcurveto{\pgfqpoint{2.625794in}{1.247763in}}{\pgfqpoint{2.629066in}{1.255663in}}{\pgfqpoint{2.629066in}{1.263899in}}%
\pgfpathcurveto{\pgfqpoint{2.629066in}{1.272135in}}{\pgfqpoint{2.625794in}{1.280035in}}{\pgfqpoint{2.619970in}{1.285859in}}%
\pgfpathcurveto{\pgfqpoint{2.614146in}{1.291683in}}{\pgfqpoint{2.606246in}{1.294956in}}{\pgfqpoint{2.598010in}{1.294956in}}%
\pgfpathcurveto{\pgfqpoint{2.589774in}{1.294956in}}{\pgfqpoint{2.581874in}{1.291683in}}{\pgfqpoint{2.576050in}{1.285859in}}%
\pgfpathcurveto{\pgfqpoint{2.570226in}{1.280035in}}{\pgfqpoint{2.566953in}{1.272135in}}{\pgfqpoint{2.566953in}{1.263899in}}%
\pgfpathcurveto{\pgfqpoint{2.566953in}{1.255663in}}{\pgfqpoint{2.570226in}{1.247763in}}{\pgfqpoint{2.576050in}{1.241939in}}%
\pgfpathcurveto{\pgfqpoint{2.581874in}{1.236115in}}{\pgfqpoint{2.589774in}{1.232843in}}{\pgfqpoint{2.598010in}{1.232843in}}%
\pgfpathclose%
\pgfusepath{stroke,fill}%
\end{pgfscope}%
\begin{pgfscope}%
\pgfpathrectangle{\pgfqpoint{0.100000in}{0.220728in}}{\pgfqpoint{3.696000in}{3.696000in}}%
\pgfusepath{clip}%
\pgfsetbuttcap%
\pgfsetroundjoin%
\definecolor{currentfill}{rgb}{0.121569,0.466667,0.705882}%
\pgfsetfillcolor{currentfill}%
\pgfsetfillopacity{0.935595}%
\pgfsetlinewidth{1.003750pt}%
\definecolor{currentstroke}{rgb}{0.121569,0.466667,0.705882}%
\pgfsetstrokecolor{currentstroke}%
\pgfsetstrokeopacity{0.935595}%
\pgfsetdash{}{0pt}%
\pgfpathmoveto{\pgfqpoint{2.594021in}{1.224863in}}%
\pgfpathcurveto{\pgfqpoint{2.602257in}{1.224863in}}{\pgfqpoint{2.610157in}{1.228136in}}{\pgfqpoint{2.615981in}{1.233959in}}%
\pgfpathcurveto{\pgfqpoint{2.621805in}{1.239783in}}{\pgfqpoint{2.625077in}{1.247683in}}{\pgfqpoint{2.625077in}{1.255920in}}%
\pgfpathcurveto{\pgfqpoint{2.625077in}{1.264156in}}{\pgfqpoint{2.621805in}{1.272056in}}{\pgfqpoint{2.615981in}{1.277880in}}%
\pgfpathcurveto{\pgfqpoint{2.610157in}{1.283704in}}{\pgfqpoint{2.602257in}{1.286976in}}{\pgfqpoint{2.594021in}{1.286976in}}%
\pgfpathcurveto{\pgfqpoint{2.585784in}{1.286976in}}{\pgfqpoint{2.577884in}{1.283704in}}{\pgfqpoint{2.572060in}{1.277880in}}%
\pgfpathcurveto{\pgfqpoint{2.566236in}{1.272056in}}{\pgfqpoint{2.562964in}{1.264156in}}{\pgfqpoint{2.562964in}{1.255920in}}%
\pgfpathcurveto{\pgfqpoint{2.562964in}{1.247683in}}{\pgfqpoint{2.566236in}{1.239783in}}{\pgfqpoint{2.572060in}{1.233959in}}%
\pgfpathcurveto{\pgfqpoint{2.577884in}{1.228136in}}{\pgfqpoint{2.585784in}{1.224863in}}{\pgfqpoint{2.594021in}{1.224863in}}%
\pgfpathclose%
\pgfusepath{stroke,fill}%
\end{pgfscope}%
\begin{pgfscope}%
\pgfpathrectangle{\pgfqpoint{0.100000in}{0.220728in}}{\pgfqpoint{3.696000in}{3.696000in}}%
\pgfusepath{clip}%
\pgfsetbuttcap%
\pgfsetroundjoin%
\definecolor{currentfill}{rgb}{0.121569,0.466667,0.705882}%
\pgfsetfillcolor{currentfill}%
\pgfsetfillopacity{0.936156}%
\pgfsetlinewidth{1.003750pt}%
\definecolor{currentstroke}{rgb}{0.121569,0.466667,0.705882}%
\pgfsetstrokecolor{currentstroke}%
\pgfsetstrokeopacity{0.936156}%
\pgfsetdash{}{0pt}%
\pgfpathmoveto{\pgfqpoint{2.591780in}{1.221055in}}%
\pgfpathcurveto{\pgfqpoint{2.600016in}{1.221055in}}{\pgfqpoint{2.607916in}{1.224327in}}{\pgfqpoint{2.613740in}{1.230151in}}%
\pgfpathcurveto{\pgfqpoint{2.619564in}{1.235975in}}{\pgfqpoint{2.622836in}{1.243875in}}{\pgfqpoint{2.622836in}{1.252111in}}%
\pgfpathcurveto{\pgfqpoint{2.622836in}{1.260347in}}{\pgfqpoint{2.619564in}{1.268248in}}{\pgfqpoint{2.613740in}{1.274071in}}%
\pgfpathcurveto{\pgfqpoint{2.607916in}{1.279895in}}{\pgfqpoint{2.600016in}{1.283168in}}{\pgfqpoint{2.591780in}{1.283168in}}%
\pgfpathcurveto{\pgfqpoint{2.583544in}{1.283168in}}{\pgfqpoint{2.575644in}{1.279895in}}{\pgfqpoint{2.569820in}{1.274071in}}%
\pgfpathcurveto{\pgfqpoint{2.563996in}{1.268248in}}{\pgfqpoint{2.560723in}{1.260347in}}{\pgfqpoint{2.560723in}{1.252111in}}%
\pgfpathcurveto{\pgfqpoint{2.560723in}{1.243875in}}{\pgfqpoint{2.563996in}{1.235975in}}{\pgfqpoint{2.569820in}{1.230151in}}%
\pgfpathcurveto{\pgfqpoint{2.575644in}{1.224327in}}{\pgfqpoint{2.583544in}{1.221055in}}{\pgfqpoint{2.591780in}{1.221055in}}%
\pgfpathclose%
\pgfusepath{stroke,fill}%
\end{pgfscope}%
\begin{pgfscope}%
\pgfpathrectangle{\pgfqpoint{0.100000in}{0.220728in}}{\pgfqpoint{3.696000in}{3.696000in}}%
\pgfusepath{clip}%
\pgfsetbuttcap%
\pgfsetroundjoin%
\definecolor{currentfill}{rgb}{0.121569,0.466667,0.705882}%
\pgfsetfillcolor{currentfill}%
\pgfsetfillopacity{0.936857}%
\pgfsetlinewidth{1.003750pt}%
\definecolor{currentstroke}{rgb}{0.121569,0.466667,0.705882}%
\pgfsetstrokecolor{currentstroke}%
\pgfsetstrokeopacity{0.936857}%
\pgfsetdash{}{0pt}%
\pgfpathmoveto{\pgfqpoint{2.590517in}{1.213980in}}%
\pgfpathcurveto{\pgfqpoint{2.598754in}{1.213980in}}{\pgfqpoint{2.606654in}{1.217253in}}{\pgfqpoint{2.612478in}{1.223077in}}%
\pgfpathcurveto{\pgfqpoint{2.618301in}{1.228900in}}{\pgfqpoint{2.621574in}{1.236801in}}{\pgfqpoint{2.621574in}{1.245037in}}%
\pgfpathcurveto{\pgfqpoint{2.621574in}{1.253273in}}{\pgfqpoint{2.618301in}{1.261173in}}{\pgfqpoint{2.612478in}{1.266997in}}%
\pgfpathcurveto{\pgfqpoint{2.606654in}{1.272821in}}{\pgfqpoint{2.598754in}{1.276093in}}{\pgfqpoint{2.590517in}{1.276093in}}%
\pgfpathcurveto{\pgfqpoint{2.582281in}{1.276093in}}{\pgfqpoint{2.574381in}{1.272821in}}{\pgfqpoint{2.568557in}{1.266997in}}%
\pgfpathcurveto{\pgfqpoint{2.562733in}{1.261173in}}{\pgfqpoint{2.559461in}{1.253273in}}{\pgfqpoint{2.559461in}{1.245037in}}%
\pgfpathcurveto{\pgfqpoint{2.559461in}{1.236801in}}{\pgfqpoint{2.562733in}{1.228900in}}{\pgfqpoint{2.568557in}{1.223077in}}%
\pgfpathcurveto{\pgfqpoint{2.574381in}{1.217253in}}{\pgfqpoint{2.582281in}{1.213980in}}{\pgfqpoint{2.590517in}{1.213980in}}%
\pgfpathclose%
\pgfusepath{stroke,fill}%
\end{pgfscope}%
\begin{pgfscope}%
\pgfpathrectangle{\pgfqpoint{0.100000in}{0.220728in}}{\pgfqpoint{3.696000in}{3.696000in}}%
\pgfusepath{clip}%
\pgfsetbuttcap%
\pgfsetroundjoin%
\definecolor{currentfill}{rgb}{0.121569,0.466667,0.705882}%
\pgfsetfillcolor{currentfill}%
\pgfsetfillopacity{0.937003}%
\pgfsetlinewidth{1.003750pt}%
\definecolor{currentstroke}{rgb}{0.121569,0.466667,0.705882}%
\pgfsetstrokecolor{currentstroke}%
\pgfsetstrokeopacity{0.937003}%
\pgfsetdash{}{0pt}%
\pgfpathmoveto{\pgfqpoint{2.021207in}{0.842386in}}%
\pgfpathcurveto{\pgfqpoint{2.029443in}{0.842386in}}{\pgfqpoint{2.037344in}{0.845658in}}{\pgfqpoint{2.043167in}{0.851482in}}%
\pgfpathcurveto{\pgfqpoint{2.048991in}{0.857306in}}{\pgfqpoint{2.052264in}{0.865206in}}{\pgfqpoint{2.052264in}{0.873442in}}%
\pgfpathcurveto{\pgfqpoint{2.052264in}{0.881679in}}{\pgfqpoint{2.048991in}{0.889579in}}{\pgfqpoint{2.043167in}{0.895403in}}%
\pgfpathcurveto{\pgfqpoint{2.037344in}{0.901226in}}{\pgfqpoint{2.029443in}{0.904499in}}{\pgfqpoint{2.021207in}{0.904499in}}%
\pgfpathcurveto{\pgfqpoint{2.012971in}{0.904499in}}{\pgfqpoint{2.005071in}{0.901226in}}{\pgfqpoint{1.999247in}{0.895403in}}%
\pgfpathcurveto{\pgfqpoint{1.993423in}{0.889579in}}{\pgfqpoint{1.990151in}{0.881679in}}{\pgfqpoint{1.990151in}{0.873442in}}%
\pgfpathcurveto{\pgfqpoint{1.990151in}{0.865206in}}{\pgfqpoint{1.993423in}{0.857306in}}{\pgfqpoint{1.999247in}{0.851482in}}%
\pgfpathcurveto{\pgfqpoint{2.005071in}{0.845658in}}{\pgfqpoint{2.012971in}{0.842386in}}{\pgfqpoint{2.021207in}{0.842386in}}%
\pgfpathclose%
\pgfusepath{stroke,fill}%
\end{pgfscope}%
\begin{pgfscope}%
\pgfpathrectangle{\pgfqpoint{0.100000in}{0.220728in}}{\pgfqpoint{3.696000in}{3.696000in}}%
\pgfusepath{clip}%
\pgfsetbuttcap%
\pgfsetroundjoin%
\definecolor{currentfill}{rgb}{0.121569,0.466667,0.705882}%
\pgfsetfillcolor{currentfill}%
\pgfsetfillopacity{0.937865}%
\pgfsetlinewidth{1.003750pt}%
\definecolor{currentstroke}{rgb}{0.121569,0.466667,0.705882}%
\pgfsetstrokecolor{currentstroke}%
\pgfsetstrokeopacity{0.937865}%
\pgfsetdash{}{0pt}%
\pgfpathmoveto{\pgfqpoint{2.586702in}{1.206158in}}%
\pgfpathcurveto{\pgfqpoint{2.594939in}{1.206158in}}{\pgfqpoint{2.602839in}{1.209431in}}{\pgfqpoint{2.608663in}{1.215255in}}%
\pgfpathcurveto{\pgfqpoint{2.614486in}{1.221079in}}{\pgfqpoint{2.617759in}{1.228979in}}{\pgfqpoint{2.617759in}{1.237215in}}%
\pgfpathcurveto{\pgfqpoint{2.617759in}{1.245451in}}{\pgfqpoint{2.614486in}{1.253351in}}{\pgfqpoint{2.608663in}{1.259175in}}%
\pgfpathcurveto{\pgfqpoint{2.602839in}{1.264999in}}{\pgfqpoint{2.594939in}{1.268271in}}{\pgfqpoint{2.586702in}{1.268271in}}%
\pgfpathcurveto{\pgfqpoint{2.578466in}{1.268271in}}{\pgfqpoint{2.570566in}{1.264999in}}{\pgfqpoint{2.564742in}{1.259175in}}%
\pgfpathcurveto{\pgfqpoint{2.558918in}{1.253351in}}{\pgfqpoint{2.555646in}{1.245451in}}{\pgfqpoint{2.555646in}{1.237215in}}%
\pgfpathcurveto{\pgfqpoint{2.555646in}{1.228979in}}{\pgfqpoint{2.558918in}{1.221079in}}{\pgfqpoint{2.564742in}{1.215255in}}%
\pgfpathcurveto{\pgfqpoint{2.570566in}{1.209431in}}{\pgfqpoint{2.578466in}{1.206158in}}{\pgfqpoint{2.586702in}{1.206158in}}%
\pgfpathclose%
\pgfusepath{stroke,fill}%
\end{pgfscope}%
\begin{pgfscope}%
\pgfpathrectangle{\pgfqpoint{0.100000in}{0.220728in}}{\pgfqpoint{3.696000in}{3.696000in}}%
\pgfusepath{clip}%
\pgfsetbuttcap%
\pgfsetroundjoin%
\definecolor{currentfill}{rgb}{0.121569,0.466667,0.705882}%
\pgfsetfillcolor{currentfill}%
\pgfsetfillopacity{0.938474}%
\pgfsetlinewidth{1.003750pt}%
\definecolor{currentstroke}{rgb}{0.121569,0.466667,0.705882}%
\pgfsetstrokecolor{currentstroke}%
\pgfsetstrokeopacity{0.938474}%
\pgfsetdash{}{0pt}%
\pgfpathmoveto{\pgfqpoint{2.584776in}{1.201849in}}%
\pgfpathcurveto{\pgfqpoint{2.593012in}{1.201849in}}{\pgfqpoint{2.600912in}{1.205122in}}{\pgfqpoint{2.606736in}{1.210946in}}%
\pgfpathcurveto{\pgfqpoint{2.612560in}{1.216770in}}{\pgfqpoint{2.615832in}{1.224670in}}{\pgfqpoint{2.615832in}{1.232906in}}%
\pgfpathcurveto{\pgfqpoint{2.615832in}{1.241142in}}{\pgfqpoint{2.612560in}{1.249042in}}{\pgfqpoint{2.606736in}{1.254866in}}%
\pgfpathcurveto{\pgfqpoint{2.600912in}{1.260690in}}{\pgfqpoint{2.593012in}{1.263962in}}{\pgfqpoint{2.584776in}{1.263962in}}%
\pgfpathcurveto{\pgfqpoint{2.576539in}{1.263962in}}{\pgfqpoint{2.568639in}{1.260690in}}{\pgfqpoint{2.562815in}{1.254866in}}%
\pgfpathcurveto{\pgfqpoint{2.556991in}{1.249042in}}{\pgfqpoint{2.553719in}{1.241142in}}{\pgfqpoint{2.553719in}{1.232906in}}%
\pgfpathcurveto{\pgfqpoint{2.553719in}{1.224670in}}{\pgfqpoint{2.556991in}{1.216770in}}{\pgfqpoint{2.562815in}{1.210946in}}%
\pgfpathcurveto{\pgfqpoint{2.568639in}{1.205122in}}{\pgfqpoint{2.576539in}{1.201849in}}{\pgfqpoint{2.584776in}{1.201849in}}%
\pgfpathclose%
\pgfusepath{stroke,fill}%
\end{pgfscope}%
\begin{pgfscope}%
\pgfpathrectangle{\pgfqpoint{0.100000in}{0.220728in}}{\pgfqpoint{3.696000in}{3.696000in}}%
\pgfusepath{clip}%
\pgfsetbuttcap%
\pgfsetroundjoin%
\definecolor{currentfill}{rgb}{0.121569,0.466667,0.705882}%
\pgfsetfillcolor{currentfill}%
\pgfsetfillopacity{0.939181}%
\pgfsetlinewidth{1.003750pt}%
\definecolor{currentstroke}{rgb}{0.121569,0.466667,0.705882}%
\pgfsetstrokecolor{currentstroke}%
\pgfsetstrokeopacity{0.939181}%
\pgfsetdash{}{0pt}%
\pgfpathmoveto{\pgfqpoint{2.583290in}{1.195836in}}%
\pgfpathcurveto{\pgfqpoint{2.591527in}{1.195836in}}{\pgfqpoint{2.599427in}{1.199108in}}{\pgfqpoint{2.605251in}{1.204932in}}%
\pgfpathcurveto{\pgfqpoint{2.611074in}{1.210756in}}{\pgfqpoint{2.614347in}{1.218656in}}{\pgfqpoint{2.614347in}{1.226893in}}%
\pgfpathcurveto{\pgfqpoint{2.614347in}{1.235129in}}{\pgfqpoint{2.611074in}{1.243029in}}{\pgfqpoint{2.605251in}{1.248853in}}%
\pgfpathcurveto{\pgfqpoint{2.599427in}{1.254677in}}{\pgfqpoint{2.591527in}{1.257949in}}{\pgfqpoint{2.583290in}{1.257949in}}%
\pgfpathcurveto{\pgfqpoint{2.575054in}{1.257949in}}{\pgfqpoint{2.567154in}{1.254677in}}{\pgfqpoint{2.561330in}{1.248853in}}%
\pgfpathcurveto{\pgfqpoint{2.555506in}{1.243029in}}{\pgfqpoint{2.552234in}{1.235129in}}{\pgfqpoint{2.552234in}{1.226893in}}%
\pgfpathcurveto{\pgfqpoint{2.552234in}{1.218656in}}{\pgfqpoint{2.555506in}{1.210756in}}{\pgfqpoint{2.561330in}{1.204932in}}%
\pgfpathcurveto{\pgfqpoint{2.567154in}{1.199108in}}{\pgfqpoint{2.575054in}{1.195836in}}{\pgfqpoint{2.583290in}{1.195836in}}%
\pgfpathclose%
\pgfusepath{stroke,fill}%
\end{pgfscope}%
\begin{pgfscope}%
\pgfpathrectangle{\pgfqpoint{0.100000in}{0.220728in}}{\pgfqpoint{3.696000in}{3.696000in}}%
\pgfusepath{clip}%
\pgfsetbuttcap%
\pgfsetroundjoin%
\definecolor{currentfill}{rgb}{0.121569,0.466667,0.705882}%
\pgfsetfillcolor{currentfill}%
\pgfsetfillopacity{0.939855}%
\pgfsetlinewidth{1.003750pt}%
\definecolor{currentstroke}{rgb}{0.121569,0.466667,0.705882}%
\pgfsetstrokecolor{currentstroke}%
\pgfsetstrokeopacity{0.939855}%
\pgfsetdash{}{0pt}%
\pgfpathmoveto{\pgfqpoint{2.037928in}{0.833684in}}%
\pgfpathcurveto{\pgfqpoint{2.046164in}{0.833684in}}{\pgfqpoint{2.054064in}{0.836956in}}{\pgfqpoint{2.059888in}{0.842780in}}%
\pgfpathcurveto{\pgfqpoint{2.065712in}{0.848604in}}{\pgfqpoint{2.068984in}{0.856504in}}{\pgfqpoint{2.068984in}{0.864740in}}%
\pgfpathcurveto{\pgfqpoint{2.068984in}{0.872977in}}{\pgfqpoint{2.065712in}{0.880877in}}{\pgfqpoint{2.059888in}{0.886701in}}%
\pgfpathcurveto{\pgfqpoint{2.054064in}{0.892525in}}{\pgfqpoint{2.046164in}{0.895797in}}{\pgfqpoint{2.037928in}{0.895797in}}%
\pgfpathcurveto{\pgfqpoint{2.029691in}{0.895797in}}{\pgfqpoint{2.021791in}{0.892525in}}{\pgfqpoint{2.015967in}{0.886701in}}%
\pgfpathcurveto{\pgfqpoint{2.010143in}{0.880877in}}{\pgfqpoint{2.006871in}{0.872977in}}{\pgfqpoint{2.006871in}{0.864740in}}%
\pgfpathcurveto{\pgfqpoint{2.006871in}{0.856504in}}{\pgfqpoint{2.010143in}{0.848604in}}{\pgfqpoint{2.015967in}{0.842780in}}%
\pgfpathcurveto{\pgfqpoint{2.021791in}{0.836956in}}{\pgfqpoint{2.029691in}{0.833684in}}{\pgfqpoint{2.037928in}{0.833684in}}%
\pgfpathclose%
\pgfusepath{stroke,fill}%
\end{pgfscope}%
\begin{pgfscope}%
\pgfpathrectangle{\pgfqpoint{0.100000in}{0.220728in}}{\pgfqpoint{3.696000in}{3.696000in}}%
\pgfusepath{clip}%
\pgfsetbuttcap%
\pgfsetroundjoin%
\definecolor{currentfill}{rgb}{0.121569,0.466667,0.705882}%
\pgfsetfillcolor{currentfill}%
\pgfsetfillopacity{0.940585}%
\pgfsetlinewidth{1.003750pt}%
\definecolor{currentstroke}{rgb}{0.121569,0.466667,0.705882}%
\pgfsetstrokecolor{currentstroke}%
\pgfsetstrokeopacity{0.940585}%
\pgfsetdash{}{0pt}%
\pgfpathmoveto{\pgfqpoint{2.577912in}{1.186752in}}%
\pgfpathcurveto{\pgfqpoint{2.586148in}{1.186752in}}{\pgfqpoint{2.594048in}{1.190025in}}{\pgfqpoint{2.599872in}{1.195849in}}%
\pgfpathcurveto{\pgfqpoint{2.605696in}{1.201673in}}{\pgfqpoint{2.608968in}{1.209573in}}{\pgfqpoint{2.608968in}{1.217809in}}%
\pgfpathcurveto{\pgfqpoint{2.608968in}{1.226045in}}{\pgfqpoint{2.605696in}{1.233945in}}{\pgfqpoint{2.599872in}{1.239769in}}%
\pgfpathcurveto{\pgfqpoint{2.594048in}{1.245593in}}{\pgfqpoint{2.586148in}{1.248865in}}{\pgfqpoint{2.577912in}{1.248865in}}%
\pgfpathcurveto{\pgfqpoint{2.569675in}{1.248865in}}{\pgfqpoint{2.561775in}{1.245593in}}{\pgfqpoint{2.555951in}{1.239769in}}%
\pgfpathcurveto{\pgfqpoint{2.550128in}{1.233945in}}{\pgfqpoint{2.546855in}{1.226045in}}{\pgfqpoint{2.546855in}{1.217809in}}%
\pgfpathcurveto{\pgfqpoint{2.546855in}{1.209573in}}{\pgfqpoint{2.550128in}{1.201673in}}{\pgfqpoint{2.555951in}{1.195849in}}%
\pgfpathcurveto{\pgfqpoint{2.561775in}{1.190025in}}{\pgfqpoint{2.569675in}{1.186752in}}{\pgfqpoint{2.577912in}{1.186752in}}%
\pgfpathclose%
\pgfusepath{stroke,fill}%
\end{pgfscope}%
\begin{pgfscope}%
\pgfpathrectangle{\pgfqpoint{0.100000in}{0.220728in}}{\pgfqpoint{3.696000in}{3.696000in}}%
\pgfusepath{clip}%
\pgfsetbuttcap%
\pgfsetroundjoin%
\definecolor{currentfill}{rgb}{0.121569,0.466667,0.705882}%
\pgfsetfillcolor{currentfill}%
\pgfsetfillopacity{0.942093}%
\pgfsetlinewidth{1.003750pt}%
\definecolor{currentstroke}{rgb}{0.121569,0.466667,0.705882}%
\pgfsetstrokecolor{currentstroke}%
\pgfsetstrokeopacity{0.942093}%
\pgfsetdash{}{0pt}%
\pgfpathmoveto{\pgfqpoint{2.574395in}{1.174207in}}%
\pgfpathcurveto{\pgfqpoint{2.582631in}{1.174207in}}{\pgfqpoint{2.590531in}{1.177479in}}{\pgfqpoint{2.596355in}{1.183303in}}%
\pgfpathcurveto{\pgfqpoint{2.602179in}{1.189127in}}{\pgfqpoint{2.605452in}{1.197027in}}{\pgfqpoint{2.605452in}{1.205264in}}%
\pgfpathcurveto{\pgfqpoint{2.605452in}{1.213500in}}{\pgfqpoint{2.602179in}{1.221400in}}{\pgfqpoint{2.596355in}{1.227224in}}%
\pgfpathcurveto{\pgfqpoint{2.590531in}{1.233048in}}{\pgfqpoint{2.582631in}{1.236320in}}{\pgfqpoint{2.574395in}{1.236320in}}%
\pgfpathcurveto{\pgfqpoint{2.566159in}{1.236320in}}{\pgfqpoint{2.558259in}{1.233048in}}{\pgfqpoint{2.552435in}{1.227224in}}%
\pgfpathcurveto{\pgfqpoint{2.546611in}{1.221400in}}{\pgfqpoint{2.543339in}{1.213500in}}{\pgfqpoint{2.543339in}{1.205264in}}%
\pgfpathcurveto{\pgfqpoint{2.543339in}{1.197027in}}{\pgfqpoint{2.546611in}{1.189127in}}{\pgfqpoint{2.552435in}{1.183303in}}%
\pgfpathcurveto{\pgfqpoint{2.558259in}{1.177479in}}{\pgfqpoint{2.566159in}{1.174207in}}{\pgfqpoint{2.574395in}{1.174207in}}%
\pgfpathclose%
\pgfusepath{stroke,fill}%
\end{pgfscope}%
\begin{pgfscope}%
\pgfpathrectangle{\pgfqpoint{0.100000in}{0.220728in}}{\pgfqpoint{3.696000in}{3.696000in}}%
\pgfusepath{clip}%
\pgfsetbuttcap%
\pgfsetroundjoin%
\definecolor{currentfill}{rgb}{0.121569,0.466667,0.705882}%
\pgfsetfillcolor{currentfill}%
\pgfsetfillopacity{0.943098}%
\pgfsetlinewidth{1.003750pt}%
\definecolor{currentstroke}{rgb}{0.121569,0.466667,0.705882}%
\pgfsetstrokecolor{currentstroke}%
\pgfsetstrokeopacity{0.943098}%
\pgfsetdash{}{0pt}%
\pgfpathmoveto{\pgfqpoint{2.572396in}{1.168041in}}%
\pgfpathcurveto{\pgfqpoint{2.580633in}{1.168041in}}{\pgfqpoint{2.588533in}{1.171313in}}{\pgfqpoint{2.594357in}{1.177137in}}%
\pgfpathcurveto{\pgfqpoint{2.600180in}{1.182961in}}{\pgfqpoint{2.603453in}{1.190861in}}{\pgfqpoint{2.603453in}{1.199097in}}%
\pgfpathcurveto{\pgfqpoint{2.603453in}{1.207334in}}{\pgfqpoint{2.600180in}{1.215234in}}{\pgfqpoint{2.594357in}{1.221058in}}%
\pgfpathcurveto{\pgfqpoint{2.588533in}{1.226882in}}{\pgfqpoint{2.580633in}{1.230154in}}{\pgfqpoint{2.572396in}{1.230154in}}%
\pgfpathcurveto{\pgfqpoint{2.564160in}{1.230154in}}{\pgfqpoint{2.556260in}{1.226882in}}{\pgfqpoint{2.550436in}{1.221058in}}%
\pgfpathcurveto{\pgfqpoint{2.544612in}{1.215234in}}{\pgfqpoint{2.541340in}{1.207334in}}{\pgfqpoint{2.541340in}{1.199097in}}%
\pgfpathcurveto{\pgfqpoint{2.541340in}{1.190861in}}{\pgfqpoint{2.544612in}{1.182961in}}{\pgfqpoint{2.550436in}{1.177137in}}%
\pgfpathcurveto{\pgfqpoint{2.556260in}{1.171313in}}{\pgfqpoint{2.564160in}{1.168041in}}{\pgfqpoint{2.572396in}{1.168041in}}%
\pgfpathclose%
\pgfusepath{stroke,fill}%
\end{pgfscope}%
\begin{pgfscope}%
\pgfpathrectangle{\pgfqpoint{0.100000in}{0.220728in}}{\pgfqpoint{3.696000in}{3.696000in}}%
\pgfusepath{clip}%
\pgfsetbuttcap%
\pgfsetroundjoin%
\definecolor{currentfill}{rgb}{0.121569,0.466667,0.705882}%
\pgfsetfillcolor{currentfill}%
\pgfsetfillopacity{0.943549}%
\pgfsetlinewidth{1.003750pt}%
\definecolor{currentstroke}{rgb}{0.121569,0.466667,0.705882}%
\pgfsetstrokecolor{currentstroke}%
\pgfsetstrokeopacity{0.943549}%
\pgfsetdash{}{0pt}%
\pgfpathmoveto{\pgfqpoint{2.051286in}{0.830495in}}%
\pgfpathcurveto{\pgfqpoint{2.059522in}{0.830495in}}{\pgfqpoint{2.067422in}{0.833767in}}{\pgfqpoint{2.073246in}{0.839591in}}%
\pgfpathcurveto{\pgfqpoint{2.079070in}{0.845415in}}{\pgfqpoint{2.082342in}{0.853315in}}{\pgfqpoint{2.082342in}{0.861551in}}%
\pgfpathcurveto{\pgfqpoint{2.082342in}{0.869787in}}{\pgfqpoint{2.079070in}{0.877688in}}{\pgfqpoint{2.073246in}{0.883511in}}%
\pgfpathcurveto{\pgfqpoint{2.067422in}{0.889335in}}{\pgfqpoint{2.059522in}{0.892608in}}{\pgfqpoint{2.051286in}{0.892608in}}%
\pgfpathcurveto{\pgfqpoint{2.043049in}{0.892608in}}{\pgfqpoint{2.035149in}{0.889335in}}{\pgfqpoint{2.029325in}{0.883511in}}%
\pgfpathcurveto{\pgfqpoint{2.023501in}{0.877688in}}{\pgfqpoint{2.020229in}{0.869787in}}{\pgfqpoint{2.020229in}{0.861551in}}%
\pgfpathcurveto{\pgfqpoint{2.020229in}{0.853315in}}{\pgfqpoint{2.023501in}{0.845415in}}{\pgfqpoint{2.029325in}{0.839591in}}%
\pgfpathcurveto{\pgfqpoint{2.035149in}{0.833767in}}{\pgfqpoint{2.043049in}{0.830495in}}{\pgfqpoint{2.051286in}{0.830495in}}%
\pgfpathclose%
\pgfusepath{stroke,fill}%
\end{pgfscope}%
\begin{pgfscope}%
\pgfpathrectangle{\pgfqpoint{0.100000in}{0.220728in}}{\pgfqpoint{3.696000in}{3.696000in}}%
\pgfusepath{clip}%
\pgfsetbuttcap%
\pgfsetroundjoin%
\definecolor{currentfill}{rgb}{0.121569,0.466667,0.705882}%
\pgfsetfillcolor{currentfill}%
\pgfsetfillopacity{0.943827}%
\pgfsetlinewidth{1.003750pt}%
\definecolor{currentstroke}{rgb}{0.121569,0.466667,0.705882}%
\pgfsetstrokecolor{currentstroke}%
\pgfsetstrokeopacity{0.943827}%
\pgfsetdash{}{0pt}%
\pgfpathmoveto{\pgfqpoint{2.568268in}{1.161630in}}%
\pgfpathcurveto{\pgfqpoint{2.576504in}{1.161630in}}{\pgfqpoint{2.584404in}{1.164903in}}{\pgfqpoint{2.590228in}{1.170727in}}%
\pgfpathcurveto{\pgfqpoint{2.596052in}{1.176550in}}{\pgfqpoint{2.599325in}{1.184451in}}{\pgfqpoint{2.599325in}{1.192687in}}%
\pgfpathcurveto{\pgfqpoint{2.599325in}{1.200923in}}{\pgfqpoint{2.596052in}{1.208823in}}{\pgfqpoint{2.590228in}{1.214647in}}%
\pgfpathcurveto{\pgfqpoint{2.584404in}{1.220471in}}{\pgfqpoint{2.576504in}{1.223743in}}{\pgfqpoint{2.568268in}{1.223743in}}%
\pgfpathcurveto{\pgfqpoint{2.560032in}{1.223743in}}{\pgfqpoint{2.552132in}{1.220471in}}{\pgfqpoint{2.546308in}{1.214647in}}%
\pgfpathcurveto{\pgfqpoint{2.540484in}{1.208823in}}{\pgfqpoint{2.537212in}{1.200923in}}{\pgfqpoint{2.537212in}{1.192687in}}%
\pgfpathcurveto{\pgfqpoint{2.537212in}{1.184451in}}{\pgfqpoint{2.540484in}{1.176550in}}{\pgfqpoint{2.546308in}{1.170727in}}%
\pgfpathcurveto{\pgfqpoint{2.552132in}{1.164903in}}{\pgfqpoint{2.560032in}{1.161630in}}{\pgfqpoint{2.568268in}{1.161630in}}%
\pgfpathclose%
\pgfusepath{stroke,fill}%
\end{pgfscope}%
\begin{pgfscope}%
\pgfpathrectangle{\pgfqpoint{0.100000in}{0.220728in}}{\pgfqpoint{3.696000in}{3.696000in}}%
\pgfusepath{clip}%
\pgfsetbuttcap%
\pgfsetroundjoin%
\definecolor{currentfill}{rgb}{0.121569,0.466667,0.705882}%
\pgfsetfillcolor{currentfill}%
\pgfsetfillopacity{0.945545}%
\pgfsetlinewidth{1.003750pt}%
\definecolor{currentstroke}{rgb}{0.121569,0.466667,0.705882}%
\pgfsetstrokecolor{currentstroke}%
\pgfsetstrokeopacity{0.945545}%
\pgfsetdash{}{0pt}%
\pgfpathmoveto{\pgfqpoint{2.565259in}{1.150376in}}%
\pgfpathcurveto{\pgfqpoint{2.573496in}{1.150376in}}{\pgfqpoint{2.581396in}{1.153648in}}{\pgfqpoint{2.587220in}{1.159472in}}%
\pgfpathcurveto{\pgfqpoint{2.593044in}{1.165296in}}{\pgfqpoint{2.596316in}{1.173196in}}{\pgfqpoint{2.596316in}{1.181433in}}%
\pgfpathcurveto{\pgfqpoint{2.596316in}{1.189669in}}{\pgfqpoint{2.593044in}{1.197569in}}{\pgfqpoint{2.587220in}{1.203393in}}%
\pgfpathcurveto{\pgfqpoint{2.581396in}{1.209217in}}{\pgfqpoint{2.573496in}{1.212489in}}{\pgfqpoint{2.565259in}{1.212489in}}%
\pgfpathcurveto{\pgfqpoint{2.557023in}{1.212489in}}{\pgfqpoint{2.549123in}{1.209217in}}{\pgfqpoint{2.543299in}{1.203393in}}%
\pgfpathcurveto{\pgfqpoint{2.537475in}{1.197569in}}{\pgfqpoint{2.534203in}{1.189669in}}{\pgfqpoint{2.534203in}{1.181433in}}%
\pgfpathcurveto{\pgfqpoint{2.534203in}{1.173196in}}{\pgfqpoint{2.537475in}{1.165296in}}{\pgfqpoint{2.543299in}{1.159472in}}%
\pgfpathcurveto{\pgfqpoint{2.549123in}{1.153648in}}{\pgfqpoint{2.557023in}{1.150376in}}{\pgfqpoint{2.565259in}{1.150376in}}%
\pgfpathclose%
\pgfusepath{stroke,fill}%
\end{pgfscope}%
\begin{pgfscope}%
\pgfpathrectangle{\pgfqpoint{0.100000in}{0.220728in}}{\pgfqpoint{3.696000in}{3.696000in}}%
\pgfusepath{clip}%
\pgfsetbuttcap%
\pgfsetroundjoin%
\definecolor{currentfill}{rgb}{0.121569,0.466667,0.705882}%
\pgfsetfillcolor{currentfill}%
\pgfsetfillopacity{0.945715}%
\pgfsetlinewidth{1.003750pt}%
\definecolor{currentstroke}{rgb}{0.121569,0.466667,0.705882}%
\pgfsetstrokecolor{currentstroke}%
\pgfsetstrokeopacity{0.945715}%
\pgfsetdash{}{0pt}%
\pgfpathmoveto{\pgfqpoint{2.063747in}{0.825520in}}%
\pgfpathcurveto{\pgfqpoint{2.071984in}{0.825520in}}{\pgfqpoint{2.079884in}{0.828792in}}{\pgfqpoint{2.085708in}{0.834616in}}%
\pgfpathcurveto{\pgfqpoint{2.091532in}{0.840440in}}{\pgfqpoint{2.094804in}{0.848340in}}{\pgfqpoint{2.094804in}{0.856576in}}%
\pgfpathcurveto{\pgfqpoint{2.094804in}{0.864812in}}{\pgfqpoint{2.091532in}{0.872712in}}{\pgfqpoint{2.085708in}{0.878536in}}%
\pgfpathcurveto{\pgfqpoint{2.079884in}{0.884360in}}{\pgfqpoint{2.071984in}{0.887633in}}{\pgfqpoint{2.063747in}{0.887633in}}%
\pgfpathcurveto{\pgfqpoint{2.055511in}{0.887633in}}{\pgfqpoint{2.047611in}{0.884360in}}{\pgfqpoint{2.041787in}{0.878536in}}%
\pgfpathcurveto{\pgfqpoint{2.035963in}{0.872712in}}{\pgfqpoint{2.032691in}{0.864812in}}{\pgfqpoint{2.032691in}{0.856576in}}%
\pgfpathcurveto{\pgfqpoint{2.032691in}{0.848340in}}{\pgfqpoint{2.035963in}{0.840440in}}{\pgfqpoint{2.041787in}{0.834616in}}%
\pgfpathcurveto{\pgfqpoint{2.047611in}{0.828792in}}{\pgfqpoint{2.055511in}{0.825520in}}{\pgfqpoint{2.063747in}{0.825520in}}%
\pgfpathclose%
\pgfusepath{stroke,fill}%
\end{pgfscope}%
\begin{pgfscope}%
\pgfpathrectangle{\pgfqpoint{0.100000in}{0.220728in}}{\pgfqpoint{3.696000in}{3.696000in}}%
\pgfusepath{clip}%
\pgfsetbuttcap%
\pgfsetroundjoin%
\definecolor{currentfill}{rgb}{0.121569,0.466667,0.705882}%
\pgfsetfillcolor{currentfill}%
\pgfsetfillopacity{0.946337}%
\pgfsetlinewidth{1.003750pt}%
\definecolor{currentstroke}{rgb}{0.121569,0.466667,0.705882}%
\pgfsetstrokecolor{currentstroke}%
\pgfsetstrokeopacity{0.946337}%
\pgfsetdash{}{0pt}%
\pgfpathmoveto{\pgfqpoint{2.562994in}{1.144060in}}%
\pgfpathcurveto{\pgfqpoint{2.571230in}{1.144060in}}{\pgfqpoint{2.579130in}{1.147333in}}{\pgfqpoint{2.584954in}{1.153156in}}%
\pgfpathcurveto{\pgfqpoint{2.590778in}{1.158980in}}{\pgfqpoint{2.594050in}{1.166880in}}{\pgfqpoint{2.594050in}{1.175117in}}%
\pgfpathcurveto{\pgfqpoint{2.594050in}{1.183353in}}{\pgfqpoint{2.590778in}{1.191253in}}{\pgfqpoint{2.584954in}{1.197077in}}%
\pgfpathcurveto{\pgfqpoint{2.579130in}{1.202901in}}{\pgfqpoint{2.571230in}{1.206173in}}{\pgfqpoint{2.562994in}{1.206173in}}%
\pgfpathcurveto{\pgfqpoint{2.554757in}{1.206173in}}{\pgfqpoint{2.546857in}{1.202901in}}{\pgfqpoint{2.541033in}{1.197077in}}%
\pgfpathcurveto{\pgfqpoint{2.535209in}{1.191253in}}{\pgfqpoint{2.531937in}{1.183353in}}{\pgfqpoint{2.531937in}{1.175117in}}%
\pgfpathcurveto{\pgfqpoint{2.531937in}{1.166880in}}{\pgfqpoint{2.535209in}{1.158980in}}{\pgfqpoint{2.541033in}{1.153156in}}%
\pgfpathcurveto{\pgfqpoint{2.546857in}{1.147333in}}{\pgfqpoint{2.554757in}{1.144060in}}{\pgfqpoint{2.562994in}{1.144060in}}%
\pgfpathclose%
\pgfusepath{stroke,fill}%
\end{pgfscope}%
\begin{pgfscope}%
\pgfpathrectangle{\pgfqpoint{0.100000in}{0.220728in}}{\pgfqpoint{3.696000in}{3.696000in}}%
\pgfusepath{clip}%
\pgfsetbuttcap%
\pgfsetroundjoin%
\definecolor{currentfill}{rgb}{0.121569,0.466667,0.705882}%
\pgfsetfillcolor{currentfill}%
\pgfsetfillopacity{0.946703}%
\pgfsetlinewidth{1.003750pt}%
\definecolor{currentstroke}{rgb}{0.121569,0.466667,0.705882}%
\pgfsetstrokecolor{currentstroke}%
\pgfsetstrokeopacity{0.946703}%
\pgfsetdash{}{0pt}%
\pgfpathmoveto{\pgfqpoint{2.561139in}{1.141143in}}%
\pgfpathcurveto{\pgfqpoint{2.569375in}{1.141143in}}{\pgfqpoint{2.577276in}{1.144416in}}{\pgfqpoint{2.583099in}{1.150240in}}%
\pgfpathcurveto{\pgfqpoint{2.588923in}{1.156064in}}{\pgfqpoint{2.592196in}{1.163964in}}{\pgfqpoint{2.592196in}{1.172200in}}%
\pgfpathcurveto{\pgfqpoint{2.592196in}{1.180436in}}{\pgfqpoint{2.588923in}{1.188336in}}{\pgfqpoint{2.583099in}{1.194160in}}%
\pgfpathcurveto{\pgfqpoint{2.577276in}{1.199984in}}{\pgfqpoint{2.569375in}{1.203256in}}{\pgfqpoint{2.561139in}{1.203256in}}%
\pgfpathcurveto{\pgfqpoint{2.552903in}{1.203256in}}{\pgfqpoint{2.545003in}{1.199984in}}{\pgfqpoint{2.539179in}{1.194160in}}%
\pgfpathcurveto{\pgfqpoint{2.533355in}{1.188336in}}{\pgfqpoint{2.530083in}{1.180436in}}{\pgfqpoint{2.530083in}{1.172200in}}%
\pgfpathcurveto{\pgfqpoint{2.530083in}{1.163964in}}{\pgfqpoint{2.533355in}{1.156064in}}{\pgfqpoint{2.539179in}{1.150240in}}%
\pgfpathcurveto{\pgfqpoint{2.545003in}{1.144416in}}{\pgfqpoint{2.552903in}{1.141143in}}{\pgfqpoint{2.561139in}{1.141143in}}%
\pgfpathclose%
\pgfusepath{stroke,fill}%
\end{pgfscope}%
\begin{pgfscope}%
\pgfpathrectangle{\pgfqpoint{0.100000in}{0.220728in}}{\pgfqpoint{3.696000in}{3.696000in}}%
\pgfusepath{clip}%
\pgfsetbuttcap%
\pgfsetroundjoin%
\definecolor{currentfill}{rgb}{0.121569,0.466667,0.705882}%
\pgfsetfillcolor{currentfill}%
\pgfsetfillopacity{0.947389}%
\pgfsetlinewidth{1.003750pt}%
\definecolor{currentstroke}{rgb}{0.121569,0.466667,0.705882}%
\pgfsetstrokecolor{currentstroke}%
\pgfsetstrokeopacity{0.947389}%
\pgfsetdash{}{0pt}%
\pgfpathmoveto{\pgfqpoint{2.559888in}{1.135611in}}%
\pgfpathcurveto{\pgfqpoint{2.568124in}{1.135611in}}{\pgfqpoint{2.576024in}{1.138883in}}{\pgfqpoint{2.581848in}{1.144707in}}%
\pgfpathcurveto{\pgfqpoint{2.587672in}{1.150531in}}{\pgfqpoint{2.590944in}{1.158431in}}{\pgfqpoint{2.590944in}{1.166667in}}%
\pgfpathcurveto{\pgfqpoint{2.590944in}{1.174903in}}{\pgfqpoint{2.587672in}{1.182804in}}{\pgfqpoint{2.581848in}{1.188627in}}%
\pgfpathcurveto{\pgfqpoint{2.576024in}{1.194451in}}{\pgfqpoint{2.568124in}{1.197724in}}{\pgfqpoint{2.559888in}{1.197724in}}%
\pgfpathcurveto{\pgfqpoint{2.551651in}{1.197724in}}{\pgfqpoint{2.543751in}{1.194451in}}{\pgfqpoint{2.537928in}{1.188627in}}%
\pgfpathcurveto{\pgfqpoint{2.532104in}{1.182804in}}{\pgfqpoint{2.528831in}{1.174903in}}{\pgfqpoint{2.528831in}{1.166667in}}%
\pgfpathcurveto{\pgfqpoint{2.528831in}{1.158431in}}{\pgfqpoint{2.532104in}{1.150531in}}{\pgfqpoint{2.537928in}{1.144707in}}%
\pgfpathcurveto{\pgfqpoint{2.543751in}{1.138883in}}{\pgfqpoint{2.551651in}{1.135611in}}{\pgfqpoint{2.559888in}{1.135611in}}%
\pgfpathclose%
\pgfusepath{stroke,fill}%
\end{pgfscope}%
\begin{pgfscope}%
\pgfpathrectangle{\pgfqpoint{0.100000in}{0.220728in}}{\pgfqpoint{3.696000in}{3.696000in}}%
\pgfusepath{clip}%
\pgfsetbuttcap%
\pgfsetroundjoin%
\definecolor{currentfill}{rgb}{0.121569,0.466667,0.705882}%
\pgfsetfillcolor{currentfill}%
\pgfsetfillopacity{0.947983}%
\pgfsetlinewidth{1.003750pt}%
\definecolor{currentstroke}{rgb}{0.121569,0.466667,0.705882}%
\pgfsetstrokecolor{currentstroke}%
\pgfsetstrokeopacity{0.947983}%
\pgfsetdash{}{0pt}%
\pgfpathmoveto{\pgfqpoint{2.075019in}{0.819065in}}%
\pgfpathcurveto{\pgfqpoint{2.083256in}{0.819065in}}{\pgfqpoint{2.091156in}{0.822337in}}{\pgfqpoint{2.096980in}{0.828161in}}%
\pgfpathcurveto{\pgfqpoint{2.102803in}{0.833985in}}{\pgfqpoint{2.106076in}{0.841885in}}{\pgfqpoint{2.106076in}{0.850121in}}%
\pgfpathcurveto{\pgfqpoint{2.106076in}{0.858357in}}{\pgfqpoint{2.102803in}{0.866257in}}{\pgfqpoint{2.096980in}{0.872081in}}%
\pgfpathcurveto{\pgfqpoint{2.091156in}{0.877905in}}{\pgfqpoint{2.083256in}{0.881178in}}{\pgfqpoint{2.075019in}{0.881178in}}%
\pgfpathcurveto{\pgfqpoint{2.066783in}{0.881178in}}{\pgfqpoint{2.058883in}{0.877905in}}{\pgfqpoint{2.053059in}{0.872081in}}%
\pgfpathcurveto{\pgfqpoint{2.047235in}{0.866257in}}{\pgfqpoint{2.043963in}{0.858357in}}{\pgfqpoint{2.043963in}{0.850121in}}%
\pgfpathcurveto{\pgfqpoint{2.043963in}{0.841885in}}{\pgfqpoint{2.047235in}{0.833985in}}{\pgfqpoint{2.053059in}{0.828161in}}%
\pgfpathcurveto{\pgfqpoint{2.058883in}{0.822337in}}{\pgfqpoint{2.066783in}{0.819065in}}{\pgfqpoint{2.075019in}{0.819065in}}%
\pgfpathclose%
\pgfusepath{stroke,fill}%
\end{pgfscope}%
\begin{pgfscope}%
\pgfpathrectangle{\pgfqpoint{0.100000in}{0.220728in}}{\pgfqpoint{3.696000in}{3.696000in}}%
\pgfusepath{clip}%
\pgfsetbuttcap%
\pgfsetroundjoin%
\definecolor{currentfill}{rgb}{0.121569,0.466667,0.705882}%
\pgfsetfillcolor{currentfill}%
\pgfsetfillopacity{0.948086}%
\pgfsetlinewidth{1.003750pt}%
\definecolor{currentstroke}{rgb}{0.121569,0.466667,0.705882}%
\pgfsetstrokecolor{currentstroke}%
\pgfsetstrokeopacity{0.948086}%
\pgfsetdash{}{0pt}%
\pgfpathmoveto{\pgfqpoint{2.555566in}{1.128367in}}%
\pgfpathcurveto{\pgfqpoint{2.563802in}{1.128367in}}{\pgfqpoint{2.571702in}{1.131639in}}{\pgfqpoint{2.577526in}{1.137463in}}%
\pgfpathcurveto{\pgfqpoint{2.583350in}{1.143287in}}{\pgfqpoint{2.586622in}{1.151187in}}{\pgfqpoint{2.586622in}{1.159423in}}%
\pgfpathcurveto{\pgfqpoint{2.586622in}{1.167659in}}{\pgfqpoint{2.583350in}{1.175559in}}{\pgfqpoint{2.577526in}{1.181383in}}%
\pgfpathcurveto{\pgfqpoint{2.571702in}{1.187207in}}{\pgfqpoint{2.563802in}{1.190480in}}{\pgfqpoint{2.555566in}{1.190480in}}%
\pgfpathcurveto{\pgfqpoint{2.547329in}{1.190480in}}{\pgfqpoint{2.539429in}{1.187207in}}{\pgfqpoint{2.533605in}{1.181383in}}%
\pgfpathcurveto{\pgfqpoint{2.527781in}{1.175559in}}{\pgfqpoint{2.524509in}{1.167659in}}{\pgfqpoint{2.524509in}{1.159423in}}%
\pgfpathcurveto{\pgfqpoint{2.524509in}{1.151187in}}{\pgfqpoint{2.527781in}{1.143287in}}{\pgfqpoint{2.533605in}{1.137463in}}%
\pgfpathcurveto{\pgfqpoint{2.539429in}{1.131639in}}{\pgfqpoint{2.547329in}{1.128367in}}{\pgfqpoint{2.555566in}{1.128367in}}%
\pgfpathclose%
\pgfusepath{stroke,fill}%
\end{pgfscope}%
\begin{pgfscope}%
\pgfpathrectangle{\pgfqpoint{0.100000in}{0.220728in}}{\pgfqpoint{3.696000in}{3.696000in}}%
\pgfusepath{clip}%
\pgfsetbuttcap%
\pgfsetroundjoin%
\definecolor{currentfill}{rgb}{0.121569,0.466667,0.705882}%
\pgfsetfillcolor{currentfill}%
\pgfsetfillopacity{0.948836}%
\pgfsetlinewidth{1.003750pt}%
\definecolor{currentstroke}{rgb}{0.121569,0.466667,0.705882}%
\pgfsetstrokecolor{currentstroke}%
\pgfsetstrokeopacity{0.948836}%
\pgfsetdash{}{0pt}%
\pgfpathmoveto{\pgfqpoint{2.550357in}{1.120476in}}%
\pgfpathcurveto{\pgfqpoint{2.558593in}{1.120476in}}{\pgfqpoint{2.566493in}{1.123748in}}{\pgfqpoint{2.572317in}{1.129572in}}%
\pgfpathcurveto{\pgfqpoint{2.578141in}{1.135396in}}{\pgfqpoint{2.581414in}{1.143296in}}{\pgfqpoint{2.581414in}{1.151532in}}%
\pgfpathcurveto{\pgfqpoint{2.581414in}{1.159769in}}{\pgfqpoint{2.578141in}{1.167669in}}{\pgfqpoint{2.572317in}{1.173493in}}%
\pgfpathcurveto{\pgfqpoint{2.566493in}{1.179317in}}{\pgfqpoint{2.558593in}{1.182589in}}{\pgfqpoint{2.550357in}{1.182589in}}%
\pgfpathcurveto{\pgfqpoint{2.542121in}{1.182589in}}{\pgfqpoint{2.534221in}{1.179317in}}{\pgfqpoint{2.528397in}{1.173493in}}%
\pgfpathcurveto{\pgfqpoint{2.522573in}{1.167669in}}{\pgfqpoint{2.519301in}{1.159769in}}{\pgfqpoint{2.519301in}{1.151532in}}%
\pgfpathcurveto{\pgfqpoint{2.519301in}{1.143296in}}{\pgfqpoint{2.522573in}{1.135396in}}{\pgfqpoint{2.528397in}{1.129572in}}%
\pgfpathcurveto{\pgfqpoint{2.534221in}{1.123748in}}{\pgfqpoint{2.542121in}{1.120476in}}{\pgfqpoint{2.550357in}{1.120476in}}%
\pgfpathclose%
\pgfusepath{stroke,fill}%
\end{pgfscope}%
\begin{pgfscope}%
\pgfpathrectangle{\pgfqpoint{0.100000in}{0.220728in}}{\pgfqpoint{3.696000in}{3.696000in}}%
\pgfusepath{clip}%
\pgfsetbuttcap%
\pgfsetroundjoin%
\definecolor{currentfill}{rgb}{0.121569,0.466667,0.705882}%
\pgfsetfillcolor{currentfill}%
\pgfsetfillopacity{0.950227}%
\pgfsetlinewidth{1.003750pt}%
\definecolor{currentstroke}{rgb}{0.121569,0.466667,0.705882}%
\pgfsetstrokecolor{currentstroke}%
\pgfsetstrokeopacity{0.950227}%
\pgfsetdash{}{0pt}%
\pgfpathmoveto{\pgfqpoint{2.547568in}{1.110576in}}%
\pgfpathcurveto{\pgfqpoint{2.555804in}{1.110576in}}{\pgfqpoint{2.563704in}{1.113849in}}{\pgfqpoint{2.569528in}{1.119673in}}%
\pgfpathcurveto{\pgfqpoint{2.575352in}{1.125497in}}{\pgfqpoint{2.578625in}{1.133397in}}{\pgfqpoint{2.578625in}{1.141633in}}%
\pgfpathcurveto{\pgfqpoint{2.578625in}{1.149869in}}{\pgfqpoint{2.575352in}{1.157769in}}{\pgfqpoint{2.569528in}{1.163593in}}%
\pgfpathcurveto{\pgfqpoint{2.563704in}{1.169417in}}{\pgfqpoint{2.555804in}{1.172689in}}{\pgfqpoint{2.547568in}{1.172689in}}%
\pgfpathcurveto{\pgfqpoint{2.539332in}{1.172689in}}{\pgfqpoint{2.531432in}{1.169417in}}{\pgfqpoint{2.525608in}{1.163593in}}%
\pgfpathcurveto{\pgfqpoint{2.519784in}{1.157769in}}{\pgfqpoint{2.516512in}{1.149869in}}{\pgfqpoint{2.516512in}{1.141633in}}%
\pgfpathcurveto{\pgfqpoint{2.516512in}{1.133397in}}{\pgfqpoint{2.519784in}{1.125497in}}{\pgfqpoint{2.525608in}{1.119673in}}%
\pgfpathcurveto{\pgfqpoint{2.531432in}{1.113849in}}{\pgfqpoint{2.539332in}{1.110576in}}{\pgfqpoint{2.547568in}{1.110576in}}%
\pgfpathclose%
\pgfusepath{stroke,fill}%
\end{pgfscope}%
\begin{pgfscope}%
\pgfpathrectangle{\pgfqpoint{0.100000in}{0.220728in}}{\pgfqpoint{3.696000in}{3.696000in}}%
\pgfusepath{clip}%
\pgfsetbuttcap%
\pgfsetroundjoin%
\definecolor{currentfill}{rgb}{0.121569,0.466667,0.705882}%
\pgfsetfillcolor{currentfill}%
\pgfsetfillopacity{0.950474}%
\pgfsetlinewidth{1.003750pt}%
\definecolor{currentstroke}{rgb}{0.121569,0.466667,0.705882}%
\pgfsetstrokecolor{currentstroke}%
\pgfsetstrokeopacity{0.950474}%
\pgfsetdash{}{0pt}%
\pgfpathmoveto{\pgfqpoint{2.085826in}{0.815041in}}%
\pgfpathcurveto{\pgfqpoint{2.094062in}{0.815041in}}{\pgfqpoint{2.101962in}{0.818314in}}{\pgfqpoint{2.107786in}{0.824138in}}%
\pgfpathcurveto{\pgfqpoint{2.113610in}{0.829962in}}{\pgfqpoint{2.116883in}{0.837862in}}{\pgfqpoint{2.116883in}{0.846098in}}%
\pgfpathcurveto{\pgfqpoint{2.116883in}{0.854334in}}{\pgfqpoint{2.113610in}{0.862234in}}{\pgfqpoint{2.107786in}{0.868058in}}%
\pgfpathcurveto{\pgfqpoint{2.101962in}{0.873882in}}{\pgfqpoint{2.094062in}{0.877154in}}{\pgfqpoint{2.085826in}{0.877154in}}%
\pgfpathcurveto{\pgfqpoint{2.077590in}{0.877154in}}{\pgfqpoint{2.069690in}{0.873882in}}{\pgfqpoint{2.063866in}{0.868058in}}%
\pgfpathcurveto{\pgfqpoint{2.058042in}{0.862234in}}{\pgfqpoint{2.054770in}{0.854334in}}{\pgfqpoint{2.054770in}{0.846098in}}%
\pgfpathcurveto{\pgfqpoint{2.054770in}{0.837862in}}{\pgfqpoint{2.058042in}{0.829962in}}{\pgfqpoint{2.063866in}{0.824138in}}%
\pgfpathcurveto{\pgfqpoint{2.069690in}{0.818314in}}{\pgfqpoint{2.077590in}{0.815041in}}{\pgfqpoint{2.085826in}{0.815041in}}%
\pgfpathclose%
\pgfusepath{stroke,fill}%
\end{pgfscope}%
\begin{pgfscope}%
\pgfpathrectangle{\pgfqpoint{0.100000in}{0.220728in}}{\pgfqpoint{3.696000in}{3.696000in}}%
\pgfusepath{clip}%
\pgfsetbuttcap%
\pgfsetroundjoin%
\definecolor{currentfill}{rgb}{0.121569,0.466667,0.705882}%
\pgfsetfillcolor{currentfill}%
\pgfsetfillopacity{0.951763}%
\pgfsetlinewidth{1.003750pt}%
\definecolor{currentstroke}{rgb}{0.121569,0.466667,0.705882}%
\pgfsetstrokecolor{currentstroke}%
\pgfsetstrokeopacity{0.951763}%
\pgfsetdash{}{0pt}%
\pgfpathmoveto{\pgfqpoint{2.542875in}{1.099403in}}%
\pgfpathcurveto{\pgfqpoint{2.551111in}{1.099403in}}{\pgfqpoint{2.559011in}{1.102675in}}{\pgfqpoint{2.564835in}{1.108499in}}%
\pgfpathcurveto{\pgfqpoint{2.570659in}{1.114323in}}{\pgfqpoint{2.573931in}{1.122223in}}{\pgfqpoint{2.573931in}{1.130459in}}%
\pgfpathcurveto{\pgfqpoint{2.573931in}{1.138695in}}{\pgfqpoint{2.570659in}{1.146596in}}{\pgfqpoint{2.564835in}{1.152419in}}%
\pgfpathcurveto{\pgfqpoint{2.559011in}{1.158243in}}{\pgfqpoint{2.551111in}{1.161516in}}{\pgfqpoint{2.542875in}{1.161516in}}%
\pgfpathcurveto{\pgfqpoint{2.534639in}{1.161516in}}{\pgfqpoint{2.526739in}{1.158243in}}{\pgfqpoint{2.520915in}{1.152419in}}%
\pgfpathcurveto{\pgfqpoint{2.515091in}{1.146596in}}{\pgfqpoint{2.511818in}{1.138695in}}{\pgfqpoint{2.511818in}{1.130459in}}%
\pgfpathcurveto{\pgfqpoint{2.511818in}{1.122223in}}{\pgfqpoint{2.515091in}{1.114323in}}{\pgfqpoint{2.520915in}{1.108499in}}%
\pgfpathcurveto{\pgfqpoint{2.526739in}{1.102675in}}{\pgfqpoint{2.534639in}{1.099403in}}{\pgfqpoint{2.542875in}{1.099403in}}%
\pgfpathclose%
\pgfusepath{stroke,fill}%
\end{pgfscope}%
\begin{pgfscope}%
\pgfpathrectangle{\pgfqpoint{0.100000in}{0.220728in}}{\pgfqpoint{3.696000in}{3.696000in}}%
\pgfusepath{clip}%
\pgfsetbuttcap%
\pgfsetroundjoin%
\definecolor{currentfill}{rgb}{0.121569,0.466667,0.705882}%
\pgfsetfillcolor{currentfill}%
\pgfsetfillopacity{0.952184}%
\pgfsetlinewidth{1.003750pt}%
\definecolor{currentstroke}{rgb}{0.121569,0.466667,0.705882}%
\pgfsetstrokecolor{currentstroke}%
\pgfsetstrokeopacity{0.952184}%
\pgfsetdash{}{0pt}%
\pgfpathmoveto{\pgfqpoint{2.094437in}{0.809811in}}%
\pgfpathcurveto{\pgfqpoint{2.102673in}{0.809811in}}{\pgfqpoint{2.110573in}{0.813084in}}{\pgfqpoint{2.116397in}{0.818908in}}%
\pgfpathcurveto{\pgfqpoint{2.122221in}{0.824732in}}{\pgfqpoint{2.125493in}{0.832632in}}{\pgfqpoint{2.125493in}{0.840868in}}%
\pgfpathcurveto{\pgfqpoint{2.125493in}{0.849104in}}{\pgfqpoint{2.122221in}{0.857004in}}{\pgfqpoint{2.116397in}{0.862828in}}%
\pgfpathcurveto{\pgfqpoint{2.110573in}{0.868652in}}{\pgfqpoint{2.102673in}{0.871924in}}{\pgfqpoint{2.094437in}{0.871924in}}%
\pgfpathcurveto{\pgfqpoint{2.086201in}{0.871924in}}{\pgfqpoint{2.078301in}{0.868652in}}{\pgfqpoint{2.072477in}{0.862828in}}%
\pgfpathcurveto{\pgfqpoint{2.066653in}{0.857004in}}{\pgfqpoint{2.063380in}{0.849104in}}{\pgfqpoint{2.063380in}{0.840868in}}%
\pgfpathcurveto{\pgfqpoint{2.063380in}{0.832632in}}{\pgfqpoint{2.066653in}{0.824732in}}{\pgfqpoint{2.072477in}{0.818908in}}%
\pgfpathcurveto{\pgfqpoint{2.078301in}{0.813084in}}{\pgfqpoint{2.086201in}{0.809811in}}{\pgfqpoint{2.094437in}{0.809811in}}%
\pgfpathclose%
\pgfusepath{stroke,fill}%
\end{pgfscope}%
\begin{pgfscope}%
\pgfpathrectangle{\pgfqpoint{0.100000in}{0.220728in}}{\pgfqpoint{3.696000in}{3.696000in}}%
\pgfusepath{clip}%
\pgfsetbuttcap%
\pgfsetroundjoin%
\definecolor{currentfill}{rgb}{0.121569,0.466667,0.705882}%
\pgfsetfillcolor{currentfill}%
\pgfsetfillopacity{0.953127}%
\pgfsetlinewidth{1.003750pt}%
\definecolor{currentstroke}{rgb}{0.121569,0.466667,0.705882}%
\pgfsetstrokecolor{currentstroke}%
\pgfsetstrokeopacity{0.953127}%
\pgfsetdash{}{0pt}%
\pgfpathmoveto{\pgfqpoint{2.535905in}{1.089089in}}%
\pgfpathcurveto{\pgfqpoint{2.544141in}{1.089089in}}{\pgfqpoint{2.552041in}{1.092362in}}{\pgfqpoint{2.557865in}{1.098185in}}%
\pgfpathcurveto{\pgfqpoint{2.563689in}{1.104009in}}{\pgfqpoint{2.566961in}{1.111909in}}{\pgfqpoint{2.566961in}{1.120146in}}%
\pgfpathcurveto{\pgfqpoint{2.566961in}{1.128382in}}{\pgfqpoint{2.563689in}{1.136282in}}{\pgfqpoint{2.557865in}{1.142106in}}%
\pgfpathcurveto{\pgfqpoint{2.552041in}{1.147930in}}{\pgfqpoint{2.544141in}{1.151202in}}{\pgfqpoint{2.535905in}{1.151202in}}%
\pgfpathcurveto{\pgfqpoint{2.527668in}{1.151202in}}{\pgfqpoint{2.519768in}{1.147930in}}{\pgfqpoint{2.513944in}{1.142106in}}%
\pgfpathcurveto{\pgfqpoint{2.508120in}{1.136282in}}{\pgfqpoint{2.504848in}{1.128382in}}{\pgfqpoint{2.504848in}{1.120146in}}%
\pgfpathcurveto{\pgfqpoint{2.504848in}{1.111909in}}{\pgfqpoint{2.508120in}{1.104009in}}{\pgfqpoint{2.513944in}{1.098185in}}%
\pgfpathcurveto{\pgfqpoint{2.519768in}{1.092362in}}{\pgfqpoint{2.527668in}{1.089089in}}{\pgfqpoint{2.535905in}{1.089089in}}%
\pgfpathclose%
\pgfusepath{stroke,fill}%
\end{pgfscope}%
\begin{pgfscope}%
\pgfpathrectangle{\pgfqpoint{0.100000in}{0.220728in}}{\pgfqpoint{3.696000in}{3.696000in}}%
\pgfusepath{clip}%
\pgfsetbuttcap%
\pgfsetroundjoin%
\definecolor{currentfill}{rgb}{0.121569,0.466667,0.705882}%
\pgfsetfillcolor{currentfill}%
\pgfsetfillopacity{0.953705}%
\pgfsetlinewidth{1.003750pt}%
\definecolor{currentstroke}{rgb}{0.121569,0.466667,0.705882}%
\pgfsetstrokecolor{currentstroke}%
\pgfsetstrokeopacity{0.953705}%
\pgfsetdash{}{0pt}%
\pgfpathmoveto{\pgfqpoint{2.100503in}{0.807483in}}%
\pgfpathcurveto{\pgfqpoint{2.108739in}{0.807483in}}{\pgfqpoint{2.116639in}{0.810755in}}{\pgfqpoint{2.122463in}{0.816579in}}%
\pgfpathcurveto{\pgfqpoint{2.128287in}{0.822403in}}{\pgfqpoint{2.131559in}{0.830303in}}{\pgfqpoint{2.131559in}{0.838540in}}%
\pgfpathcurveto{\pgfqpoint{2.131559in}{0.846776in}}{\pgfqpoint{2.128287in}{0.854676in}}{\pgfqpoint{2.122463in}{0.860500in}}%
\pgfpathcurveto{\pgfqpoint{2.116639in}{0.866324in}}{\pgfqpoint{2.108739in}{0.869596in}}{\pgfqpoint{2.100503in}{0.869596in}}%
\pgfpathcurveto{\pgfqpoint{2.092267in}{0.869596in}}{\pgfqpoint{2.084366in}{0.866324in}}{\pgfqpoint{2.078543in}{0.860500in}}%
\pgfpathcurveto{\pgfqpoint{2.072719in}{0.854676in}}{\pgfqpoint{2.069446in}{0.846776in}}{\pgfqpoint{2.069446in}{0.838540in}}%
\pgfpathcurveto{\pgfqpoint{2.069446in}{0.830303in}}{\pgfqpoint{2.072719in}{0.822403in}}{\pgfqpoint{2.078543in}{0.816579in}}%
\pgfpathcurveto{\pgfqpoint{2.084366in}{0.810755in}}{\pgfqpoint{2.092267in}{0.807483in}}{\pgfqpoint{2.100503in}{0.807483in}}%
\pgfpathclose%
\pgfusepath{stroke,fill}%
\end{pgfscope}%
\begin{pgfscope}%
\pgfpathrectangle{\pgfqpoint{0.100000in}{0.220728in}}{\pgfqpoint{3.696000in}{3.696000in}}%
\pgfusepath{clip}%
\pgfsetbuttcap%
\pgfsetroundjoin%
\definecolor{currentfill}{rgb}{0.121569,0.466667,0.705882}%
\pgfsetfillcolor{currentfill}%
\pgfsetfillopacity{0.955402}%
\pgfsetlinewidth{1.003750pt}%
\definecolor{currentstroke}{rgb}{0.121569,0.466667,0.705882}%
\pgfsetstrokecolor{currentstroke}%
\pgfsetstrokeopacity{0.955402}%
\pgfsetdash{}{0pt}%
\pgfpathmoveto{\pgfqpoint{2.112230in}{0.801542in}}%
\pgfpathcurveto{\pgfqpoint{2.120466in}{0.801542in}}{\pgfqpoint{2.128366in}{0.804814in}}{\pgfqpoint{2.134190in}{0.810638in}}%
\pgfpathcurveto{\pgfqpoint{2.140014in}{0.816462in}}{\pgfqpoint{2.143286in}{0.824362in}}{\pgfqpoint{2.143286in}{0.832598in}}%
\pgfpathcurveto{\pgfqpoint{2.143286in}{0.840834in}}{\pgfqpoint{2.140014in}{0.848735in}}{\pgfqpoint{2.134190in}{0.854558in}}%
\pgfpathcurveto{\pgfqpoint{2.128366in}{0.860382in}}{\pgfqpoint{2.120466in}{0.863655in}}{\pgfqpoint{2.112230in}{0.863655in}}%
\pgfpathcurveto{\pgfqpoint{2.103994in}{0.863655in}}{\pgfqpoint{2.096093in}{0.860382in}}{\pgfqpoint{2.090270in}{0.854558in}}%
\pgfpathcurveto{\pgfqpoint{2.084446in}{0.848735in}}{\pgfqpoint{2.081173in}{0.840834in}}{\pgfqpoint{2.081173in}{0.832598in}}%
\pgfpathcurveto{\pgfqpoint{2.081173in}{0.824362in}}{\pgfqpoint{2.084446in}{0.816462in}}{\pgfqpoint{2.090270in}{0.810638in}}%
\pgfpathcurveto{\pgfqpoint{2.096093in}{0.804814in}}{\pgfqpoint{2.103994in}{0.801542in}}{\pgfqpoint{2.112230in}{0.801542in}}%
\pgfpathclose%
\pgfusepath{stroke,fill}%
\end{pgfscope}%
\begin{pgfscope}%
\pgfpathrectangle{\pgfqpoint{0.100000in}{0.220728in}}{\pgfqpoint{3.696000in}{3.696000in}}%
\pgfusepath{clip}%
\pgfsetbuttcap%
\pgfsetroundjoin%
\definecolor{currentfill}{rgb}{0.121569,0.466667,0.705882}%
\pgfsetfillcolor{currentfill}%
\pgfsetfillopacity{0.955677}%
\pgfsetlinewidth{1.003750pt}%
\definecolor{currentstroke}{rgb}{0.121569,0.466667,0.705882}%
\pgfsetstrokecolor{currentstroke}%
\pgfsetstrokeopacity{0.955677}%
\pgfsetdash{}{0pt}%
\pgfpathmoveto{\pgfqpoint{2.532597in}{1.071124in}}%
\pgfpathcurveto{\pgfqpoint{2.540834in}{1.071124in}}{\pgfqpoint{2.548734in}{1.074397in}}{\pgfqpoint{2.554558in}{1.080220in}}%
\pgfpathcurveto{\pgfqpoint{2.560382in}{1.086044in}}{\pgfqpoint{2.563654in}{1.093944in}}{\pgfqpoint{2.563654in}{1.102181in}}%
\pgfpathcurveto{\pgfqpoint{2.563654in}{1.110417in}}{\pgfqpoint{2.560382in}{1.118317in}}{\pgfqpoint{2.554558in}{1.124141in}}%
\pgfpathcurveto{\pgfqpoint{2.548734in}{1.129965in}}{\pgfqpoint{2.540834in}{1.133237in}}{\pgfqpoint{2.532597in}{1.133237in}}%
\pgfpathcurveto{\pgfqpoint{2.524361in}{1.133237in}}{\pgfqpoint{2.516461in}{1.129965in}}{\pgfqpoint{2.510637in}{1.124141in}}%
\pgfpathcurveto{\pgfqpoint{2.504813in}{1.118317in}}{\pgfqpoint{2.501541in}{1.110417in}}{\pgfqpoint{2.501541in}{1.102181in}}%
\pgfpathcurveto{\pgfqpoint{2.501541in}{1.093944in}}{\pgfqpoint{2.504813in}{1.086044in}}{\pgfqpoint{2.510637in}{1.080220in}}%
\pgfpathcurveto{\pgfqpoint{2.516461in}{1.074397in}}{\pgfqpoint{2.524361in}{1.071124in}}{\pgfqpoint{2.532597in}{1.071124in}}%
\pgfpathclose%
\pgfusepath{stroke,fill}%
\end{pgfscope}%
\begin{pgfscope}%
\pgfpathrectangle{\pgfqpoint{0.100000in}{0.220728in}}{\pgfqpoint{3.696000in}{3.696000in}}%
\pgfusepath{clip}%
\pgfsetbuttcap%
\pgfsetroundjoin%
\definecolor{currentfill}{rgb}{0.121569,0.466667,0.705882}%
\pgfsetfillcolor{currentfill}%
\pgfsetfillopacity{0.956262}%
\pgfsetlinewidth{1.003750pt}%
\definecolor{currentstroke}{rgb}{0.121569,0.466667,0.705882}%
\pgfsetstrokecolor{currentstroke}%
\pgfsetstrokeopacity{0.956262}%
\pgfsetdash{}{0pt}%
\pgfpathmoveto{\pgfqpoint{2.518211in}{1.053632in}}%
\pgfpathcurveto{\pgfqpoint{2.526447in}{1.053632in}}{\pgfqpoint{2.534347in}{1.056904in}}{\pgfqpoint{2.540171in}{1.062728in}}%
\pgfpathcurveto{\pgfqpoint{2.545995in}{1.068552in}}{\pgfqpoint{2.549267in}{1.076452in}}{\pgfqpoint{2.549267in}{1.084688in}}%
\pgfpathcurveto{\pgfqpoint{2.549267in}{1.092924in}}{\pgfqpoint{2.545995in}{1.100824in}}{\pgfqpoint{2.540171in}{1.106648in}}%
\pgfpathcurveto{\pgfqpoint{2.534347in}{1.112472in}}{\pgfqpoint{2.526447in}{1.115745in}}{\pgfqpoint{2.518211in}{1.115745in}}%
\pgfpathcurveto{\pgfqpoint{2.509974in}{1.115745in}}{\pgfqpoint{2.502074in}{1.112472in}}{\pgfqpoint{2.496250in}{1.106648in}}%
\pgfpathcurveto{\pgfqpoint{2.490426in}{1.100824in}}{\pgfqpoint{2.487154in}{1.092924in}}{\pgfqpoint{2.487154in}{1.084688in}}%
\pgfpathcurveto{\pgfqpoint{2.487154in}{1.076452in}}{\pgfqpoint{2.490426in}{1.068552in}}{\pgfqpoint{2.496250in}{1.062728in}}%
\pgfpathcurveto{\pgfqpoint{2.502074in}{1.056904in}}{\pgfqpoint{2.509974in}{1.053632in}}{\pgfqpoint{2.518211in}{1.053632in}}%
\pgfpathclose%
\pgfusepath{stroke,fill}%
\end{pgfscope}%
\begin{pgfscope}%
\pgfpathrectangle{\pgfqpoint{0.100000in}{0.220728in}}{\pgfqpoint{3.696000in}{3.696000in}}%
\pgfusepath{clip}%
\pgfsetbuttcap%
\pgfsetroundjoin%
\definecolor{currentfill}{rgb}{0.121569,0.466667,0.705882}%
\pgfsetfillcolor{currentfill}%
\pgfsetfillopacity{0.957525}%
\pgfsetlinewidth{1.003750pt}%
\definecolor{currentstroke}{rgb}{0.121569,0.466667,0.705882}%
\pgfsetstrokecolor{currentstroke}%
\pgfsetstrokeopacity{0.957525}%
\pgfsetdash{}{0pt}%
\pgfpathmoveto{\pgfqpoint{2.122367in}{0.796486in}}%
\pgfpathcurveto{\pgfqpoint{2.130603in}{0.796486in}}{\pgfqpoint{2.138503in}{0.799758in}}{\pgfqpoint{2.144327in}{0.805582in}}%
\pgfpathcurveto{\pgfqpoint{2.150151in}{0.811406in}}{\pgfqpoint{2.153424in}{0.819306in}}{\pgfqpoint{2.153424in}{0.827543in}}%
\pgfpathcurveto{\pgfqpoint{2.153424in}{0.835779in}}{\pgfqpoint{2.150151in}{0.843679in}}{\pgfqpoint{2.144327in}{0.849503in}}%
\pgfpathcurveto{\pgfqpoint{2.138503in}{0.855327in}}{\pgfqpoint{2.130603in}{0.858599in}}{\pgfqpoint{2.122367in}{0.858599in}}%
\pgfpathcurveto{\pgfqpoint{2.114131in}{0.858599in}}{\pgfqpoint{2.106231in}{0.855327in}}{\pgfqpoint{2.100407in}{0.849503in}}%
\pgfpathcurveto{\pgfqpoint{2.094583in}{0.843679in}}{\pgfqpoint{2.091311in}{0.835779in}}{\pgfqpoint{2.091311in}{0.827543in}}%
\pgfpathcurveto{\pgfqpoint{2.091311in}{0.819306in}}{\pgfqpoint{2.094583in}{0.811406in}}{\pgfqpoint{2.100407in}{0.805582in}}%
\pgfpathcurveto{\pgfqpoint{2.106231in}{0.799758in}}{\pgfqpoint{2.114131in}{0.796486in}}{\pgfqpoint{2.122367in}{0.796486in}}%
\pgfpathclose%
\pgfusepath{stroke,fill}%
\end{pgfscope}%
\begin{pgfscope}%
\pgfpathrectangle{\pgfqpoint{0.100000in}{0.220728in}}{\pgfqpoint{3.696000in}{3.696000in}}%
\pgfusepath{clip}%
\pgfsetbuttcap%
\pgfsetroundjoin%
\definecolor{currentfill}{rgb}{0.121569,0.466667,0.705882}%
\pgfsetfillcolor{currentfill}%
\pgfsetfillopacity{0.959635}%
\pgfsetlinewidth{1.003750pt}%
\definecolor{currentstroke}{rgb}{0.121569,0.466667,0.705882}%
\pgfsetstrokecolor{currentstroke}%
\pgfsetstrokeopacity{0.959635}%
\pgfsetdash{}{0pt}%
\pgfpathmoveto{\pgfqpoint{2.131781in}{0.793028in}}%
\pgfpathcurveto{\pgfqpoint{2.140017in}{0.793028in}}{\pgfqpoint{2.147917in}{0.796300in}}{\pgfqpoint{2.153741in}{0.802124in}}%
\pgfpathcurveto{\pgfqpoint{2.159565in}{0.807948in}}{\pgfqpoint{2.162837in}{0.815848in}}{\pgfqpoint{2.162837in}{0.824084in}}%
\pgfpathcurveto{\pgfqpoint{2.162837in}{0.832320in}}{\pgfqpoint{2.159565in}{0.840221in}}{\pgfqpoint{2.153741in}{0.846044in}}%
\pgfpathcurveto{\pgfqpoint{2.147917in}{0.851868in}}{\pgfqpoint{2.140017in}{0.855141in}}{\pgfqpoint{2.131781in}{0.855141in}}%
\pgfpathcurveto{\pgfqpoint{2.123545in}{0.855141in}}{\pgfqpoint{2.115645in}{0.851868in}}{\pgfqpoint{2.109821in}{0.846044in}}%
\pgfpathcurveto{\pgfqpoint{2.103997in}{0.840221in}}{\pgfqpoint{2.100724in}{0.832320in}}{\pgfqpoint{2.100724in}{0.824084in}}%
\pgfpathcurveto{\pgfqpoint{2.100724in}{0.815848in}}{\pgfqpoint{2.103997in}{0.807948in}}{\pgfqpoint{2.109821in}{0.802124in}}%
\pgfpathcurveto{\pgfqpoint{2.115645in}{0.796300in}}{\pgfqpoint{2.123545in}{0.793028in}}{\pgfqpoint{2.131781in}{0.793028in}}%
\pgfpathclose%
\pgfusepath{stroke,fill}%
\end{pgfscope}%
\begin{pgfscope}%
\pgfpathrectangle{\pgfqpoint{0.100000in}{0.220728in}}{\pgfqpoint{3.696000in}{3.696000in}}%
\pgfusepath{clip}%
\pgfsetbuttcap%
\pgfsetroundjoin%
\definecolor{currentfill}{rgb}{0.121569,0.466667,0.705882}%
\pgfsetfillcolor{currentfill}%
\pgfsetfillopacity{0.960016}%
\pgfsetlinewidth{1.003750pt}%
\definecolor{currentstroke}{rgb}{0.121569,0.466667,0.705882}%
\pgfsetstrokecolor{currentstroke}%
\pgfsetstrokeopacity{0.960016}%
\pgfsetdash{}{0pt}%
\pgfpathmoveto{\pgfqpoint{2.512584in}{1.025681in}}%
\pgfpathcurveto{\pgfqpoint{2.520820in}{1.025681in}}{\pgfqpoint{2.528720in}{1.028954in}}{\pgfqpoint{2.534544in}{1.034778in}}%
\pgfpathcurveto{\pgfqpoint{2.540368in}{1.040602in}}{\pgfqpoint{2.543641in}{1.048502in}}{\pgfqpoint{2.543641in}{1.056738in}}%
\pgfpathcurveto{\pgfqpoint{2.543641in}{1.064974in}}{\pgfqpoint{2.540368in}{1.072874in}}{\pgfqpoint{2.534544in}{1.078698in}}%
\pgfpathcurveto{\pgfqpoint{2.528720in}{1.084522in}}{\pgfqpoint{2.520820in}{1.087794in}}{\pgfqpoint{2.512584in}{1.087794in}}%
\pgfpathcurveto{\pgfqpoint{2.504348in}{1.087794in}}{\pgfqpoint{2.496448in}{1.084522in}}{\pgfqpoint{2.490624in}{1.078698in}}%
\pgfpathcurveto{\pgfqpoint{2.484800in}{1.072874in}}{\pgfqpoint{2.481528in}{1.064974in}}{\pgfqpoint{2.481528in}{1.056738in}}%
\pgfpathcurveto{\pgfqpoint{2.481528in}{1.048502in}}{\pgfqpoint{2.484800in}{1.040602in}}{\pgfqpoint{2.490624in}{1.034778in}}%
\pgfpathcurveto{\pgfqpoint{2.496448in}{1.028954in}}{\pgfqpoint{2.504348in}{1.025681in}}{\pgfqpoint{2.512584in}{1.025681in}}%
\pgfpathclose%
\pgfusepath{stroke,fill}%
\end{pgfscope}%
\begin{pgfscope}%
\pgfpathrectangle{\pgfqpoint{0.100000in}{0.220728in}}{\pgfqpoint{3.696000in}{3.696000in}}%
\pgfusepath{clip}%
\pgfsetbuttcap%
\pgfsetroundjoin%
\definecolor{currentfill}{rgb}{0.121569,0.466667,0.705882}%
\pgfsetfillcolor{currentfill}%
\pgfsetfillopacity{0.961248}%
\pgfsetlinewidth{1.003750pt}%
\definecolor{currentstroke}{rgb}{0.121569,0.466667,0.705882}%
\pgfsetstrokecolor{currentstroke}%
\pgfsetstrokeopacity{0.961248}%
\pgfsetdash{}{0pt}%
\pgfpathmoveto{\pgfqpoint{2.139386in}{0.789373in}}%
\pgfpathcurveto{\pgfqpoint{2.147622in}{0.789373in}}{\pgfqpoint{2.155522in}{0.792646in}}{\pgfqpoint{2.161346in}{0.798470in}}%
\pgfpathcurveto{\pgfqpoint{2.167170in}{0.804294in}}{\pgfqpoint{2.170443in}{0.812194in}}{\pgfqpoint{2.170443in}{0.820430in}}%
\pgfpathcurveto{\pgfqpoint{2.170443in}{0.828666in}}{\pgfqpoint{2.167170in}{0.836566in}}{\pgfqpoint{2.161346in}{0.842390in}}%
\pgfpathcurveto{\pgfqpoint{2.155522in}{0.848214in}}{\pgfqpoint{2.147622in}{0.851486in}}{\pgfqpoint{2.139386in}{0.851486in}}%
\pgfpathcurveto{\pgfqpoint{2.131150in}{0.851486in}}{\pgfqpoint{2.123250in}{0.848214in}}{\pgfqpoint{2.117426in}{0.842390in}}%
\pgfpathcurveto{\pgfqpoint{2.111602in}{0.836566in}}{\pgfqpoint{2.108330in}{0.828666in}}{\pgfqpoint{2.108330in}{0.820430in}}%
\pgfpathcurveto{\pgfqpoint{2.108330in}{0.812194in}}{\pgfqpoint{2.111602in}{0.804294in}}{\pgfqpoint{2.117426in}{0.798470in}}%
\pgfpathcurveto{\pgfqpoint{2.123250in}{0.792646in}}{\pgfqpoint{2.131150in}{0.789373in}}{\pgfqpoint{2.139386in}{0.789373in}}%
\pgfpathclose%
\pgfusepath{stroke,fill}%
\end{pgfscope}%
\begin{pgfscope}%
\pgfpathrectangle{\pgfqpoint{0.100000in}{0.220728in}}{\pgfqpoint{3.696000in}{3.696000in}}%
\pgfusepath{clip}%
\pgfsetbuttcap%
\pgfsetroundjoin%
\definecolor{currentfill}{rgb}{0.121569,0.466667,0.705882}%
\pgfsetfillcolor{currentfill}%
\pgfsetfillopacity{0.962575}%
\pgfsetlinewidth{1.003750pt}%
\definecolor{currentstroke}{rgb}{0.121569,0.466667,0.705882}%
\pgfsetstrokecolor{currentstroke}%
\pgfsetstrokeopacity{0.962575}%
\pgfsetdash{}{0pt}%
\pgfpathmoveto{\pgfqpoint{2.496863in}{0.998801in}}%
\pgfpathcurveto{\pgfqpoint{2.505099in}{0.998801in}}{\pgfqpoint{2.512999in}{1.002073in}}{\pgfqpoint{2.518823in}{1.007897in}}%
\pgfpathcurveto{\pgfqpoint{2.524647in}{1.013721in}}{\pgfqpoint{2.527920in}{1.021621in}}{\pgfqpoint{2.527920in}{1.029857in}}%
\pgfpathcurveto{\pgfqpoint{2.527920in}{1.038094in}}{\pgfqpoint{2.524647in}{1.045994in}}{\pgfqpoint{2.518823in}{1.051818in}}%
\pgfpathcurveto{\pgfqpoint{2.512999in}{1.057642in}}{\pgfqpoint{2.505099in}{1.060914in}}{\pgfqpoint{2.496863in}{1.060914in}}%
\pgfpathcurveto{\pgfqpoint{2.488627in}{1.060914in}}{\pgfqpoint{2.480727in}{1.057642in}}{\pgfqpoint{2.474903in}{1.051818in}}%
\pgfpathcurveto{\pgfqpoint{2.469079in}{1.045994in}}{\pgfqpoint{2.465807in}{1.038094in}}{\pgfqpoint{2.465807in}{1.029857in}}%
\pgfpathcurveto{\pgfqpoint{2.465807in}{1.021621in}}{\pgfqpoint{2.469079in}{1.013721in}}{\pgfqpoint{2.474903in}{1.007897in}}%
\pgfpathcurveto{\pgfqpoint{2.480727in}{1.002073in}}{\pgfqpoint{2.488627in}{0.998801in}}{\pgfqpoint{2.496863in}{0.998801in}}%
\pgfpathclose%
\pgfusepath{stroke,fill}%
\end{pgfscope}%
\begin{pgfscope}%
\pgfpathrectangle{\pgfqpoint{0.100000in}{0.220728in}}{\pgfqpoint{3.696000in}{3.696000in}}%
\pgfusepath{clip}%
\pgfsetbuttcap%
\pgfsetroundjoin%
\definecolor{currentfill}{rgb}{0.121569,0.466667,0.705882}%
\pgfsetfillcolor{currentfill}%
\pgfsetfillopacity{0.962954}%
\pgfsetlinewidth{1.003750pt}%
\definecolor{currentstroke}{rgb}{0.121569,0.466667,0.705882}%
\pgfsetstrokecolor{currentstroke}%
\pgfsetstrokeopacity{0.962954}%
\pgfsetdash{}{0pt}%
\pgfpathmoveto{\pgfqpoint{2.146047in}{0.787469in}}%
\pgfpathcurveto{\pgfqpoint{2.154284in}{0.787469in}}{\pgfqpoint{2.162184in}{0.790742in}}{\pgfqpoint{2.168008in}{0.796566in}}%
\pgfpathcurveto{\pgfqpoint{2.173832in}{0.802389in}}{\pgfqpoint{2.177104in}{0.810290in}}{\pgfqpoint{2.177104in}{0.818526in}}%
\pgfpathcurveto{\pgfqpoint{2.177104in}{0.826762in}}{\pgfqpoint{2.173832in}{0.834662in}}{\pgfqpoint{2.168008in}{0.840486in}}%
\pgfpathcurveto{\pgfqpoint{2.162184in}{0.846310in}}{\pgfqpoint{2.154284in}{0.849582in}}{\pgfqpoint{2.146047in}{0.849582in}}%
\pgfpathcurveto{\pgfqpoint{2.137811in}{0.849582in}}{\pgfqpoint{2.129911in}{0.846310in}}{\pgfqpoint{2.124087in}{0.840486in}}%
\pgfpathcurveto{\pgfqpoint{2.118263in}{0.834662in}}{\pgfqpoint{2.114991in}{0.826762in}}{\pgfqpoint{2.114991in}{0.818526in}}%
\pgfpathcurveto{\pgfqpoint{2.114991in}{0.810290in}}{\pgfqpoint{2.118263in}{0.802389in}}{\pgfqpoint{2.124087in}{0.796566in}}%
\pgfpathcurveto{\pgfqpoint{2.129911in}{0.790742in}}{\pgfqpoint{2.137811in}{0.787469in}}{\pgfqpoint{2.146047in}{0.787469in}}%
\pgfpathclose%
\pgfusepath{stroke,fill}%
\end{pgfscope}%
\begin{pgfscope}%
\pgfpathrectangle{\pgfqpoint{0.100000in}{0.220728in}}{\pgfqpoint{3.696000in}{3.696000in}}%
\pgfusepath{clip}%
\pgfsetbuttcap%
\pgfsetroundjoin%
\definecolor{currentfill}{rgb}{0.121569,0.466667,0.705882}%
\pgfsetfillcolor{currentfill}%
\pgfsetfillopacity{0.965760}%
\pgfsetlinewidth{1.003750pt}%
\definecolor{currentstroke}{rgb}{0.121569,0.466667,0.705882}%
\pgfsetstrokecolor{currentstroke}%
\pgfsetstrokeopacity{0.965760}%
\pgfsetdash{}{0pt}%
\pgfpathmoveto{\pgfqpoint{2.158130in}{0.782779in}}%
\pgfpathcurveto{\pgfqpoint{2.166366in}{0.782779in}}{\pgfqpoint{2.174266in}{0.786051in}}{\pgfqpoint{2.180090in}{0.791875in}}%
\pgfpathcurveto{\pgfqpoint{2.185914in}{0.797699in}}{\pgfqpoint{2.189187in}{0.805599in}}{\pgfqpoint{2.189187in}{0.813835in}}%
\pgfpathcurveto{\pgfqpoint{2.189187in}{0.822071in}}{\pgfqpoint{2.185914in}{0.829972in}}{\pgfqpoint{2.180090in}{0.835795in}}%
\pgfpathcurveto{\pgfqpoint{2.174266in}{0.841619in}}{\pgfqpoint{2.166366in}{0.844892in}}{\pgfqpoint{2.158130in}{0.844892in}}%
\pgfpathcurveto{\pgfqpoint{2.149894in}{0.844892in}}{\pgfqpoint{2.141994in}{0.841619in}}{\pgfqpoint{2.136170in}{0.835795in}}%
\pgfpathcurveto{\pgfqpoint{2.130346in}{0.829972in}}{\pgfqpoint{2.127074in}{0.822071in}}{\pgfqpoint{2.127074in}{0.813835in}}%
\pgfpathcurveto{\pgfqpoint{2.127074in}{0.805599in}}{\pgfqpoint{2.130346in}{0.797699in}}{\pgfqpoint{2.136170in}{0.791875in}}%
\pgfpathcurveto{\pgfqpoint{2.141994in}{0.786051in}}{\pgfqpoint{2.149894in}{0.782779in}}{\pgfqpoint{2.158130in}{0.782779in}}%
\pgfpathclose%
\pgfusepath{stroke,fill}%
\end{pgfscope}%
\begin{pgfscope}%
\pgfpathrectangle{\pgfqpoint{0.100000in}{0.220728in}}{\pgfqpoint{3.696000in}{3.696000in}}%
\pgfusepath{clip}%
\pgfsetbuttcap%
\pgfsetroundjoin%
\definecolor{currentfill}{rgb}{0.121569,0.466667,0.705882}%
\pgfsetfillcolor{currentfill}%
\pgfsetfillopacity{0.967193}%
\pgfsetlinewidth{1.003750pt}%
\definecolor{currentstroke}{rgb}{0.121569,0.466667,0.705882}%
\pgfsetstrokecolor{currentstroke}%
\pgfsetstrokeopacity{0.967193}%
\pgfsetdash{}{0pt}%
\pgfpathmoveto{\pgfqpoint{2.489239in}{0.964313in}}%
\pgfpathcurveto{\pgfqpoint{2.497476in}{0.964313in}}{\pgfqpoint{2.505376in}{0.967585in}}{\pgfqpoint{2.511200in}{0.973409in}}%
\pgfpathcurveto{\pgfqpoint{2.517024in}{0.979233in}}{\pgfqpoint{2.520296in}{0.987133in}}{\pgfqpoint{2.520296in}{0.995370in}}%
\pgfpathcurveto{\pgfqpoint{2.520296in}{1.003606in}}{\pgfqpoint{2.517024in}{1.011506in}}{\pgfqpoint{2.511200in}{1.017330in}}%
\pgfpathcurveto{\pgfqpoint{2.505376in}{1.023154in}}{\pgfqpoint{2.497476in}{1.026426in}}{\pgfqpoint{2.489239in}{1.026426in}}%
\pgfpathcurveto{\pgfqpoint{2.481003in}{1.026426in}}{\pgfqpoint{2.473103in}{1.023154in}}{\pgfqpoint{2.467279in}{1.017330in}}%
\pgfpathcurveto{\pgfqpoint{2.461455in}{1.011506in}}{\pgfqpoint{2.458183in}{1.003606in}}{\pgfqpoint{2.458183in}{0.995370in}}%
\pgfpathcurveto{\pgfqpoint{2.458183in}{0.987133in}}{\pgfqpoint{2.461455in}{0.979233in}}{\pgfqpoint{2.467279in}{0.973409in}}%
\pgfpathcurveto{\pgfqpoint{2.473103in}{0.967585in}}{\pgfqpoint{2.481003in}{0.964313in}}{\pgfqpoint{2.489239in}{0.964313in}}%
\pgfpathclose%
\pgfusepath{stroke,fill}%
\end{pgfscope}%
\begin{pgfscope}%
\pgfpathrectangle{\pgfqpoint{0.100000in}{0.220728in}}{\pgfqpoint{3.696000in}{3.696000in}}%
\pgfusepath{clip}%
\pgfsetbuttcap%
\pgfsetroundjoin%
\definecolor{currentfill}{rgb}{0.121569,0.466667,0.705882}%
\pgfsetfillcolor{currentfill}%
\pgfsetfillopacity{0.968039}%
\pgfsetlinewidth{1.003750pt}%
\definecolor{currentstroke}{rgb}{0.121569,0.466667,0.705882}%
\pgfsetstrokecolor{currentstroke}%
\pgfsetstrokeopacity{0.968039}%
\pgfsetdash{}{0pt}%
\pgfpathmoveto{\pgfqpoint{2.168659in}{0.778550in}}%
\pgfpathcurveto{\pgfqpoint{2.176895in}{0.778550in}}{\pgfqpoint{2.184795in}{0.781823in}}{\pgfqpoint{2.190619in}{0.787646in}}%
\pgfpathcurveto{\pgfqpoint{2.196443in}{0.793470in}}{\pgfqpoint{2.199716in}{0.801370in}}{\pgfqpoint{2.199716in}{0.809607in}}%
\pgfpathcurveto{\pgfqpoint{2.199716in}{0.817843in}}{\pgfqpoint{2.196443in}{0.825743in}}{\pgfqpoint{2.190619in}{0.831567in}}%
\pgfpathcurveto{\pgfqpoint{2.184795in}{0.837391in}}{\pgfqpoint{2.176895in}{0.840663in}}{\pgfqpoint{2.168659in}{0.840663in}}%
\pgfpathcurveto{\pgfqpoint{2.160423in}{0.840663in}}{\pgfqpoint{2.152523in}{0.837391in}}{\pgfqpoint{2.146699in}{0.831567in}}%
\pgfpathcurveto{\pgfqpoint{2.140875in}{0.825743in}}{\pgfqpoint{2.137603in}{0.817843in}}{\pgfqpoint{2.137603in}{0.809607in}}%
\pgfpathcurveto{\pgfqpoint{2.137603in}{0.801370in}}{\pgfqpoint{2.140875in}{0.793470in}}{\pgfqpoint{2.146699in}{0.787646in}}%
\pgfpathcurveto{\pgfqpoint{2.152523in}{0.781823in}}{\pgfqpoint{2.160423in}{0.778550in}}{\pgfqpoint{2.168659in}{0.778550in}}%
\pgfpathclose%
\pgfusepath{stroke,fill}%
\end{pgfscope}%
\begin{pgfscope}%
\pgfpathrectangle{\pgfqpoint{0.100000in}{0.220728in}}{\pgfqpoint{3.696000in}{3.696000in}}%
\pgfusepath{clip}%
\pgfsetbuttcap%
\pgfsetroundjoin%
\definecolor{currentfill}{rgb}{0.121569,0.466667,0.705882}%
\pgfsetfillcolor{currentfill}%
\pgfsetfillopacity{0.969131}%
\pgfsetlinewidth{1.003750pt}%
\definecolor{currentstroke}{rgb}{0.121569,0.466667,0.705882}%
\pgfsetstrokecolor{currentstroke}%
\pgfsetstrokeopacity{0.969131}%
\pgfsetdash{}{0pt}%
\pgfpathmoveto{\pgfqpoint{2.481690in}{0.945754in}}%
\pgfpathcurveto{\pgfqpoint{2.489926in}{0.945754in}}{\pgfqpoint{2.497826in}{0.949027in}}{\pgfqpoint{2.503650in}{0.954850in}}%
\pgfpathcurveto{\pgfqpoint{2.509474in}{0.960674in}}{\pgfqpoint{2.512746in}{0.968574in}}{\pgfqpoint{2.512746in}{0.976811in}}%
\pgfpathcurveto{\pgfqpoint{2.512746in}{0.985047in}}{\pgfqpoint{2.509474in}{0.992947in}}{\pgfqpoint{2.503650in}{0.998771in}}%
\pgfpathcurveto{\pgfqpoint{2.497826in}{1.004595in}}{\pgfqpoint{2.489926in}{1.007867in}}{\pgfqpoint{2.481690in}{1.007867in}}%
\pgfpathcurveto{\pgfqpoint{2.473454in}{1.007867in}}{\pgfqpoint{2.465554in}{1.004595in}}{\pgfqpoint{2.459730in}{0.998771in}}%
\pgfpathcurveto{\pgfqpoint{2.453906in}{0.992947in}}{\pgfqpoint{2.450633in}{0.985047in}}{\pgfqpoint{2.450633in}{0.976811in}}%
\pgfpathcurveto{\pgfqpoint{2.450633in}{0.968574in}}{\pgfqpoint{2.453906in}{0.960674in}}{\pgfqpoint{2.459730in}{0.954850in}}%
\pgfpathcurveto{\pgfqpoint{2.465554in}{0.949027in}}{\pgfqpoint{2.473454in}{0.945754in}}{\pgfqpoint{2.481690in}{0.945754in}}%
\pgfpathclose%
\pgfusepath{stroke,fill}%
\end{pgfscope}%
\begin{pgfscope}%
\pgfpathrectangle{\pgfqpoint{0.100000in}{0.220728in}}{\pgfqpoint{3.696000in}{3.696000in}}%
\pgfusepath{clip}%
\pgfsetbuttcap%
\pgfsetroundjoin%
\definecolor{currentfill}{rgb}{0.121569,0.466667,0.705882}%
\pgfsetfillcolor{currentfill}%
\pgfsetfillopacity{0.969822}%
\pgfsetlinewidth{1.003750pt}%
\definecolor{currentstroke}{rgb}{0.121569,0.466667,0.705882}%
\pgfsetstrokecolor{currentstroke}%
\pgfsetstrokeopacity{0.969822}%
\pgfsetdash{}{0pt}%
\pgfpathmoveto{\pgfqpoint{2.176273in}{0.776950in}}%
\pgfpathcurveto{\pgfqpoint{2.184509in}{0.776950in}}{\pgfqpoint{2.192410in}{0.780222in}}{\pgfqpoint{2.198233in}{0.786046in}}%
\pgfpathcurveto{\pgfqpoint{2.204057in}{0.791870in}}{\pgfqpoint{2.207330in}{0.799770in}}{\pgfqpoint{2.207330in}{0.808006in}}%
\pgfpathcurveto{\pgfqpoint{2.207330in}{0.816243in}}{\pgfqpoint{2.204057in}{0.824143in}}{\pgfqpoint{2.198233in}{0.829967in}}%
\pgfpathcurveto{\pgfqpoint{2.192410in}{0.835791in}}{\pgfqpoint{2.184509in}{0.839063in}}{\pgfqpoint{2.176273in}{0.839063in}}%
\pgfpathcurveto{\pgfqpoint{2.168037in}{0.839063in}}{\pgfqpoint{2.160137in}{0.835791in}}{\pgfqpoint{2.154313in}{0.829967in}}%
\pgfpathcurveto{\pgfqpoint{2.148489in}{0.824143in}}{\pgfqpoint{2.145217in}{0.816243in}}{\pgfqpoint{2.145217in}{0.808006in}}%
\pgfpathcurveto{\pgfqpoint{2.145217in}{0.799770in}}{\pgfqpoint{2.148489in}{0.791870in}}{\pgfqpoint{2.154313in}{0.786046in}}%
\pgfpathcurveto{\pgfqpoint{2.160137in}{0.780222in}}{\pgfqpoint{2.168037in}{0.776950in}}{\pgfqpoint{2.176273in}{0.776950in}}%
\pgfpathclose%
\pgfusepath{stroke,fill}%
\end{pgfscope}%
\begin{pgfscope}%
\pgfpathrectangle{\pgfqpoint{0.100000in}{0.220728in}}{\pgfqpoint{3.696000in}{3.696000in}}%
\pgfusepath{clip}%
\pgfsetbuttcap%
\pgfsetroundjoin%
\definecolor{currentfill}{rgb}{0.121569,0.466667,0.705882}%
\pgfsetfillcolor{currentfill}%
\pgfsetfillopacity{0.970319}%
\pgfsetlinewidth{1.003750pt}%
\definecolor{currentstroke}{rgb}{0.121569,0.466667,0.705882}%
\pgfsetstrokecolor{currentstroke}%
\pgfsetstrokeopacity{0.970319}%
\pgfsetdash{}{0pt}%
\pgfpathmoveto{\pgfqpoint{2.476641in}{0.937288in}}%
\pgfpathcurveto{\pgfqpoint{2.484877in}{0.937288in}}{\pgfqpoint{2.492777in}{0.940560in}}{\pgfqpoint{2.498601in}{0.946384in}}%
\pgfpathcurveto{\pgfqpoint{2.504425in}{0.952208in}}{\pgfqpoint{2.507697in}{0.960108in}}{\pgfqpoint{2.507697in}{0.968344in}}%
\pgfpathcurveto{\pgfqpoint{2.507697in}{0.976581in}}{\pgfqpoint{2.504425in}{0.984481in}}{\pgfqpoint{2.498601in}{0.990305in}}%
\pgfpathcurveto{\pgfqpoint{2.492777in}{0.996129in}}{\pgfqpoint{2.484877in}{0.999401in}}{\pgfqpoint{2.476641in}{0.999401in}}%
\pgfpathcurveto{\pgfqpoint{2.468404in}{0.999401in}}{\pgfqpoint{2.460504in}{0.996129in}}{\pgfqpoint{2.454680in}{0.990305in}}%
\pgfpathcurveto{\pgfqpoint{2.448856in}{0.984481in}}{\pgfqpoint{2.445584in}{0.976581in}}{\pgfqpoint{2.445584in}{0.968344in}}%
\pgfpathcurveto{\pgfqpoint{2.445584in}{0.960108in}}{\pgfqpoint{2.448856in}{0.952208in}}{\pgfqpoint{2.454680in}{0.946384in}}%
\pgfpathcurveto{\pgfqpoint{2.460504in}{0.940560in}}{\pgfqpoint{2.468404in}{0.937288in}}{\pgfqpoint{2.476641in}{0.937288in}}%
\pgfpathclose%
\pgfusepath{stroke,fill}%
\end{pgfscope}%
\begin{pgfscope}%
\pgfpathrectangle{\pgfqpoint{0.100000in}{0.220728in}}{\pgfqpoint{3.696000in}{3.696000in}}%
\pgfusepath{clip}%
\pgfsetbuttcap%
\pgfsetroundjoin%
\definecolor{currentfill}{rgb}{0.121569,0.466667,0.705882}%
\pgfsetfillcolor{currentfill}%
\pgfsetfillopacity{0.970577}%
\pgfsetlinewidth{1.003750pt}%
\definecolor{currentstroke}{rgb}{0.121569,0.466667,0.705882}%
\pgfsetstrokecolor{currentstroke}%
\pgfsetstrokeopacity{0.970577}%
\pgfsetdash{}{0pt}%
\pgfpathmoveto{\pgfqpoint{2.182529in}{0.773293in}}%
\pgfpathcurveto{\pgfqpoint{2.190765in}{0.773293in}}{\pgfqpoint{2.198665in}{0.776566in}}{\pgfqpoint{2.204489in}{0.782390in}}%
\pgfpathcurveto{\pgfqpoint{2.210313in}{0.788214in}}{\pgfqpoint{2.213586in}{0.796114in}}{\pgfqpoint{2.213586in}{0.804350in}}%
\pgfpathcurveto{\pgfqpoint{2.213586in}{0.812586in}}{\pgfqpoint{2.210313in}{0.820486in}}{\pgfqpoint{2.204489in}{0.826310in}}%
\pgfpathcurveto{\pgfqpoint{2.198665in}{0.832134in}}{\pgfqpoint{2.190765in}{0.835406in}}{\pgfqpoint{2.182529in}{0.835406in}}%
\pgfpathcurveto{\pgfqpoint{2.174293in}{0.835406in}}{\pgfqpoint{2.166393in}{0.832134in}}{\pgfqpoint{2.160569in}{0.826310in}}%
\pgfpathcurveto{\pgfqpoint{2.154745in}{0.820486in}}{\pgfqpoint{2.151473in}{0.812586in}}{\pgfqpoint{2.151473in}{0.804350in}}%
\pgfpathcurveto{\pgfqpoint{2.151473in}{0.796114in}}{\pgfqpoint{2.154745in}{0.788214in}}{\pgfqpoint{2.160569in}{0.782390in}}%
\pgfpathcurveto{\pgfqpoint{2.166393in}{0.776566in}}{\pgfqpoint{2.174293in}{0.773293in}}{\pgfqpoint{2.182529in}{0.773293in}}%
\pgfpathclose%
\pgfusepath{stroke,fill}%
\end{pgfscope}%
\begin{pgfscope}%
\pgfpathrectangle{\pgfqpoint{0.100000in}{0.220728in}}{\pgfqpoint{3.696000in}{3.696000in}}%
\pgfusepath{clip}%
\pgfsetbuttcap%
\pgfsetroundjoin%
\definecolor{currentfill}{rgb}{0.121569,0.466667,0.705882}%
\pgfsetfillcolor{currentfill}%
\pgfsetfillopacity{0.971778}%
\pgfsetlinewidth{1.003750pt}%
\definecolor{currentstroke}{rgb}{0.121569,0.466667,0.705882}%
\pgfsetstrokecolor{currentstroke}%
\pgfsetstrokeopacity{0.971778}%
\pgfsetdash{}{0pt}%
\pgfpathmoveto{\pgfqpoint{2.474032in}{0.926018in}}%
\pgfpathcurveto{\pgfqpoint{2.482269in}{0.926018in}}{\pgfqpoint{2.490169in}{0.929290in}}{\pgfqpoint{2.495993in}{0.935114in}}%
\pgfpathcurveto{\pgfqpoint{2.501816in}{0.940938in}}{\pgfqpoint{2.505089in}{0.948838in}}{\pgfqpoint{2.505089in}{0.957074in}}%
\pgfpathcurveto{\pgfqpoint{2.505089in}{0.965310in}}{\pgfqpoint{2.501816in}{0.973211in}}{\pgfqpoint{2.495993in}{0.979034in}}%
\pgfpathcurveto{\pgfqpoint{2.490169in}{0.984858in}}{\pgfqpoint{2.482269in}{0.988131in}}{\pgfqpoint{2.474032in}{0.988131in}}%
\pgfpathcurveto{\pgfqpoint{2.465796in}{0.988131in}}{\pgfqpoint{2.457896in}{0.984858in}}{\pgfqpoint{2.452072in}{0.979034in}}%
\pgfpathcurveto{\pgfqpoint{2.446248in}{0.973211in}}{\pgfqpoint{2.442976in}{0.965310in}}{\pgfqpoint{2.442976in}{0.957074in}}%
\pgfpathcurveto{\pgfqpoint{2.442976in}{0.948838in}}{\pgfqpoint{2.446248in}{0.940938in}}{\pgfqpoint{2.452072in}{0.935114in}}%
\pgfpathcurveto{\pgfqpoint{2.457896in}{0.929290in}}{\pgfqpoint{2.465796in}{0.926018in}}{\pgfqpoint{2.474032in}{0.926018in}}%
\pgfpathclose%
\pgfusepath{stroke,fill}%
\end{pgfscope}%
\begin{pgfscope}%
\pgfpathrectangle{\pgfqpoint{0.100000in}{0.220728in}}{\pgfqpoint{3.696000in}{3.696000in}}%
\pgfusepath{clip}%
\pgfsetbuttcap%
\pgfsetroundjoin%
\definecolor{currentfill}{rgb}{0.121569,0.466667,0.705882}%
\pgfsetfillcolor{currentfill}%
\pgfsetfillopacity{0.972198}%
\pgfsetlinewidth{1.003750pt}%
\definecolor{currentstroke}{rgb}{0.121569,0.466667,0.705882}%
\pgfsetstrokecolor{currentstroke}%
\pgfsetstrokeopacity{0.972198}%
\pgfsetdash{}{0pt}%
\pgfpathmoveto{\pgfqpoint{2.194246in}{0.768918in}}%
\pgfpathcurveto{\pgfqpoint{2.202482in}{0.768918in}}{\pgfqpoint{2.210382in}{0.772191in}}{\pgfqpoint{2.216206in}{0.778015in}}%
\pgfpathcurveto{\pgfqpoint{2.222030in}{0.783839in}}{\pgfqpoint{2.225302in}{0.791739in}}{\pgfqpoint{2.225302in}{0.799975in}}%
\pgfpathcurveto{\pgfqpoint{2.225302in}{0.808211in}}{\pgfqpoint{2.222030in}{0.816111in}}{\pgfqpoint{2.216206in}{0.821935in}}%
\pgfpathcurveto{\pgfqpoint{2.210382in}{0.827759in}}{\pgfqpoint{2.202482in}{0.831031in}}{\pgfqpoint{2.194246in}{0.831031in}}%
\pgfpathcurveto{\pgfqpoint{2.186009in}{0.831031in}}{\pgfqpoint{2.178109in}{0.827759in}}{\pgfqpoint{2.172285in}{0.821935in}}%
\pgfpathcurveto{\pgfqpoint{2.166461in}{0.816111in}}{\pgfqpoint{2.163189in}{0.808211in}}{\pgfqpoint{2.163189in}{0.799975in}}%
\pgfpathcurveto{\pgfqpoint{2.163189in}{0.791739in}}{\pgfqpoint{2.166461in}{0.783839in}}{\pgfqpoint{2.172285in}{0.778015in}}%
\pgfpathcurveto{\pgfqpoint{2.178109in}{0.772191in}}{\pgfqpoint{2.186009in}{0.768918in}}{\pgfqpoint{2.194246in}{0.768918in}}%
\pgfpathclose%
\pgfusepath{stroke,fill}%
\end{pgfscope}%
\begin{pgfscope}%
\pgfpathrectangle{\pgfqpoint{0.100000in}{0.220728in}}{\pgfqpoint{3.696000in}{3.696000in}}%
\pgfusepath{clip}%
\pgfsetbuttcap%
\pgfsetroundjoin%
\definecolor{currentfill}{rgb}{0.121569,0.466667,0.705882}%
\pgfsetfillcolor{currentfill}%
\pgfsetfillopacity{0.973394}%
\pgfsetlinewidth{1.003750pt}%
\definecolor{currentstroke}{rgb}{0.121569,0.466667,0.705882}%
\pgfsetstrokecolor{currentstroke}%
\pgfsetstrokeopacity{0.973394}%
\pgfsetdash{}{0pt}%
\pgfpathmoveto{\pgfqpoint{2.463508in}{0.911333in}}%
\pgfpathcurveto{\pgfqpoint{2.471744in}{0.911333in}}{\pgfqpoint{2.479644in}{0.914605in}}{\pgfqpoint{2.485468in}{0.920429in}}%
\pgfpathcurveto{\pgfqpoint{2.491292in}{0.926253in}}{\pgfqpoint{2.494564in}{0.934153in}}{\pgfqpoint{2.494564in}{0.942390in}}%
\pgfpathcurveto{\pgfqpoint{2.494564in}{0.950626in}}{\pgfqpoint{2.491292in}{0.958526in}}{\pgfqpoint{2.485468in}{0.964350in}}%
\pgfpathcurveto{\pgfqpoint{2.479644in}{0.970174in}}{\pgfqpoint{2.471744in}{0.973446in}}{\pgfqpoint{2.463508in}{0.973446in}}%
\pgfpathcurveto{\pgfqpoint{2.455271in}{0.973446in}}{\pgfqpoint{2.447371in}{0.970174in}}{\pgfqpoint{2.441548in}{0.964350in}}%
\pgfpathcurveto{\pgfqpoint{2.435724in}{0.958526in}}{\pgfqpoint{2.432451in}{0.950626in}}{\pgfqpoint{2.432451in}{0.942390in}}%
\pgfpathcurveto{\pgfqpoint{2.432451in}{0.934153in}}{\pgfqpoint{2.435724in}{0.926253in}}{\pgfqpoint{2.441548in}{0.920429in}}%
\pgfpathcurveto{\pgfqpoint{2.447371in}{0.914605in}}{\pgfqpoint{2.455271in}{0.911333in}}{\pgfqpoint{2.463508in}{0.911333in}}%
\pgfpathclose%
\pgfusepath{stroke,fill}%
\end{pgfscope}%
\begin{pgfscope}%
\pgfpathrectangle{\pgfqpoint{0.100000in}{0.220728in}}{\pgfqpoint{3.696000in}{3.696000in}}%
\pgfusepath{clip}%
\pgfsetbuttcap%
\pgfsetroundjoin%
\definecolor{currentfill}{rgb}{0.121569,0.466667,0.705882}%
\pgfsetfillcolor{currentfill}%
\pgfsetfillopacity{0.974034}%
\pgfsetlinewidth{1.003750pt}%
\definecolor{currentstroke}{rgb}{0.121569,0.466667,0.705882}%
\pgfsetstrokecolor{currentstroke}%
\pgfsetstrokeopacity{0.974034}%
\pgfsetdash{}{0pt}%
\pgfpathmoveto{\pgfqpoint{2.216657in}{0.762246in}}%
\pgfpathcurveto{\pgfqpoint{2.224893in}{0.762246in}}{\pgfqpoint{2.232793in}{0.765519in}}{\pgfqpoint{2.238617in}{0.771343in}}%
\pgfpathcurveto{\pgfqpoint{2.244441in}{0.777167in}}{\pgfqpoint{2.247713in}{0.785067in}}{\pgfqpoint{2.247713in}{0.793303in}}%
\pgfpathcurveto{\pgfqpoint{2.247713in}{0.801539in}}{\pgfqpoint{2.244441in}{0.809439in}}{\pgfqpoint{2.238617in}{0.815263in}}%
\pgfpathcurveto{\pgfqpoint{2.232793in}{0.821087in}}{\pgfqpoint{2.224893in}{0.824359in}}{\pgfqpoint{2.216657in}{0.824359in}}%
\pgfpathcurveto{\pgfqpoint{2.208421in}{0.824359in}}{\pgfqpoint{2.200521in}{0.821087in}}{\pgfqpoint{2.194697in}{0.815263in}}%
\pgfpathcurveto{\pgfqpoint{2.188873in}{0.809439in}}{\pgfqpoint{2.185600in}{0.801539in}}{\pgfqpoint{2.185600in}{0.793303in}}%
\pgfpathcurveto{\pgfqpoint{2.185600in}{0.785067in}}{\pgfqpoint{2.188873in}{0.777167in}}{\pgfqpoint{2.194697in}{0.771343in}}%
\pgfpathcurveto{\pgfqpoint{2.200521in}{0.765519in}}{\pgfqpoint{2.208421in}{0.762246in}}{\pgfqpoint{2.216657in}{0.762246in}}%
\pgfpathclose%
\pgfusepath{stroke,fill}%
\end{pgfscope}%
\begin{pgfscope}%
\pgfpathrectangle{\pgfqpoint{0.100000in}{0.220728in}}{\pgfqpoint{3.696000in}{3.696000in}}%
\pgfusepath{clip}%
\pgfsetbuttcap%
\pgfsetroundjoin%
\definecolor{currentfill}{rgb}{0.121569,0.466667,0.705882}%
\pgfsetfillcolor{currentfill}%
\pgfsetfillopacity{0.975953}%
\pgfsetlinewidth{1.003750pt}%
\definecolor{currentstroke}{rgb}{0.121569,0.466667,0.705882}%
\pgfsetstrokecolor{currentstroke}%
\pgfsetstrokeopacity{0.975953}%
\pgfsetdash{}{0pt}%
\pgfpathmoveto{\pgfqpoint{2.458328in}{0.889415in}}%
\pgfpathcurveto{\pgfqpoint{2.466565in}{0.889415in}}{\pgfqpoint{2.474465in}{0.892688in}}{\pgfqpoint{2.480289in}{0.898512in}}%
\pgfpathcurveto{\pgfqpoint{2.486112in}{0.904335in}}{\pgfqpoint{2.489385in}{0.912236in}}{\pgfqpoint{2.489385in}{0.920472in}}%
\pgfpathcurveto{\pgfqpoint{2.489385in}{0.928708in}}{\pgfqpoint{2.486112in}{0.936608in}}{\pgfqpoint{2.480289in}{0.942432in}}%
\pgfpathcurveto{\pgfqpoint{2.474465in}{0.948256in}}{\pgfqpoint{2.466565in}{0.951528in}}{\pgfqpoint{2.458328in}{0.951528in}}%
\pgfpathcurveto{\pgfqpoint{2.450092in}{0.951528in}}{\pgfqpoint{2.442192in}{0.948256in}}{\pgfqpoint{2.436368in}{0.942432in}}%
\pgfpathcurveto{\pgfqpoint{2.430544in}{0.936608in}}{\pgfqpoint{2.427272in}{0.928708in}}{\pgfqpoint{2.427272in}{0.920472in}}%
\pgfpathcurveto{\pgfqpoint{2.427272in}{0.912236in}}{\pgfqpoint{2.430544in}{0.904335in}}{\pgfqpoint{2.436368in}{0.898512in}}%
\pgfpathcurveto{\pgfqpoint{2.442192in}{0.892688in}}{\pgfqpoint{2.450092in}{0.889415in}}{\pgfqpoint{2.458328in}{0.889415in}}%
\pgfpathclose%
\pgfusepath{stroke,fill}%
\end{pgfscope}%
\begin{pgfscope}%
\pgfpathrectangle{\pgfqpoint{0.100000in}{0.220728in}}{\pgfqpoint{3.696000in}{3.696000in}}%
\pgfusepath{clip}%
\pgfsetbuttcap%
\pgfsetroundjoin%
\definecolor{currentfill}{rgb}{0.121569,0.466667,0.705882}%
\pgfsetfillcolor{currentfill}%
\pgfsetfillopacity{0.977543}%
\pgfsetlinewidth{1.003750pt}%
\definecolor{currentstroke}{rgb}{0.121569,0.466667,0.705882}%
\pgfsetstrokecolor{currentstroke}%
\pgfsetstrokeopacity{0.977543}%
\pgfsetdash{}{0pt}%
\pgfpathmoveto{\pgfqpoint{2.235427in}{0.757723in}}%
\pgfpathcurveto{\pgfqpoint{2.243663in}{0.757723in}}{\pgfqpoint{2.251563in}{0.760995in}}{\pgfqpoint{2.257387in}{0.766819in}}%
\pgfpathcurveto{\pgfqpoint{2.263211in}{0.772643in}}{\pgfqpoint{2.266484in}{0.780543in}}{\pgfqpoint{2.266484in}{0.788779in}}%
\pgfpathcurveto{\pgfqpoint{2.266484in}{0.797016in}}{\pgfqpoint{2.263211in}{0.804916in}}{\pgfqpoint{2.257387in}{0.810740in}}%
\pgfpathcurveto{\pgfqpoint{2.251563in}{0.816563in}}{\pgfqpoint{2.243663in}{0.819836in}}{\pgfqpoint{2.235427in}{0.819836in}}%
\pgfpathcurveto{\pgfqpoint{2.227191in}{0.819836in}}{\pgfqpoint{2.219291in}{0.816563in}}{\pgfqpoint{2.213467in}{0.810740in}}%
\pgfpathcurveto{\pgfqpoint{2.207643in}{0.804916in}}{\pgfqpoint{2.204371in}{0.797016in}}{\pgfqpoint{2.204371in}{0.788779in}}%
\pgfpathcurveto{\pgfqpoint{2.204371in}{0.780543in}}{\pgfqpoint{2.207643in}{0.772643in}}{\pgfqpoint{2.213467in}{0.766819in}}%
\pgfpathcurveto{\pgfqpoint{2.219291in}{0.760995in}}{\pgfqpoint{2.227191in}{0.757723in}}{\pgfqpoint{2.235427in}{0.757723in}}%
\pgfpathclose%
\pgfusepath{stroke,fill}%
\end{pgfscope}%
\begin{pgfscope}%
\pgfpathrectangle{\pgfqpoint{0.100000in}{0.220728in}}{\pgfqpoint{3.696000in}{3.696000in}}%
\pgfusepath{clip}%
\pgfsetbuttcap%
\pgfsetroundjoin%
\definecolor{currentfill}{rgb}{0.121569,0.466667,0.705882}%
\pgfsetfillcolor{currentfill}%
\pgfsetfillopacity{0.977944}%
\pgfsetlinewidth{1.003750pt}%
\definecolor{currentstroke}{rgb}{0.121569,0.466667,0.705882}%
\pgfsetstrokecolor{currentstroke}%
\pgfsetstrokeopacity{0.977944}%
\pgfsetdash{}{0pt}%
\pgfpathmoveto{\pgfqpoint{2.444419in}{0.867530in}}%
\pgfpathcurveto{\pgfqpoint{2.452655in}{0.867530in}}{\pgfqpoint{2.460555in}{0.870803in}}{\pgfqpoint{2.466379in}{0.876626in}}%
\pgfpathcurveto{\pgfqpoint{2.472203in}{0.882450in}}{\pgfqpoint{2.475475in}{0.890350in}}{\pgfqpoint{2.475475in}{0.898587in}}%
\pgfpathcurveto{\pgfqpoint{2.475475in}{0.906823in}}{\pgfqpoint{2.472203in}{0.914723in}}{\pgfqpoint{2.466379in}{0.920547in}}%
\pgfpathcurveto{\pgfqpoint{2.460555in}{0.926371in}}{\pgfqpoint{2.452655in}{0.929643in}}{\pgfqpoint{2.444419in}{0.929643in}}%
\pgfpathcurveto{\pgfqpoint{2.436183in}{0.929643in}}{\pgfqpoint{2.428283in}{0.926371in}}{\pgfqpoint{2.422459in}{0.920547in}}%
\pgfpathcurveto{\pgfqpoint{2.416635in}{0.914723in}}{\pgfqpoint{2.413362in}{0.906823in}}{\pgfqpoint{2.413362in}{0.898587in}}%
\pgfpathcurveto{\pgfqpoint{2.413362in}{0.890350in}}{\pgfqpoint{2.416635in}{0.882450in}}{\pgfqpoint{2.422459in}{0.876626in}}%
\pgfpathcurveto{\pgfqpoint{2.428283in}{0.870803in}}{\pgfqpoint{2.436183in}{0.867530in}}{\pgfqpoint{2.444419in}{0.867530in}}%
\pgfpathclose%
\pgfusepath{stroke,fill}%
\end{pgfscope}%
\begin{pgfscope}%
\pgfpathrectangle{\pgfqpoint{0.100000in}{0.220728in}}{\pgfqpoint{3.696000in}{3.696000in}}%
\pgfusepath{clip}%
\pgfsetbuttcap%
\pgfsetroundjoin%
\definecolor{currentfill}{rgb}{0.121569,0.466667,0.705882}%
\pgfsetfillcolor{currentfill}%
\pgfsetfillopacity{0.980935}%
\pgfsetlinewidth{1.003750pt}%
\definecolor{currentstroke}{rgb}{0.121569,0.466667,0.705882}%
\pgfsetstrokecolor{currentstroke}%
\pgfsetstrokeopacity{0.980935}%
\pgfsetdash{}{0pt}%
\pgfpathmoveto{\pgfqpoint{2.251351in}{0.757345in}}%
\pgfpathcurveto{\pgfqpoint{2.259587in}{0.757345in}}{\pgfqpoint{2.267487in}{0.760618in}}{\pgfqpoint{2.273311in}{0.766442in}}%
\pgfpathcurveto{\pgfqpoint{2.279135in}{0.772266in}}{\pgfqpoint{2.282408in}{0.780166in}}{\pgfqpoint{2.282408in}{0.788402in}}%
\pgfpathcurveto{\pgfqpoint{2.282408in}{0.796638in}}{\pgfqpoint{2.279135in}{0.804538in}}{\pgfqpoint{2.273311in}{0.810362in}}%
\pgfpathcurveto{\pgfqpoint{2.267487in}{0.816186in}}{\pgfqpoint{2.259587in}{0.819458in}}{\pgfqpoint{2.251351in}{0.819458in}}%
\pgfpathcurveto{\pgfqpoint{2.243115in}{0.819458in}}{\pgfqpoint{2.235215in}{0.816186in}}{\pgfqpoint{2.229391in}{0.810362in}}%
\pgfpathcurveto{\pgfqpoint{2.223567in}{0.804538in}}{\pgfqpoint{2.220295in}{0.796638in}}{\pgfqpoint{2.220295in}{0.788402in}}%
\pgfpathcurveto{\pgfqpoint{2.220295in}{0.780166in}}{\pgfqpoint{2.223567in}{0.772266in}}{\pgfqpoint{2.229391in}{0.766442in}}%
\pgfpathcurveto{\pgfqpoint{2.235215in}{0.760618in}}{\pgfqpoint{2.243115in}{0.757345in}}{\pgfqpoint{2.251351in}{0.757345in}}%
\pgfpathclose%
\pgfusepath{stroke,fill}%
\end{pgfscope}%
\begin{pgfscope}%
\pgfpathrectangle{\pgfqpoint{0.100000in}{0.220728in}}{\pgfqpoint{3.696000in}{3.696000in}}%
\pgfusepath{clip}%
\pgfsetbuttcap%
\pgfsetroundjoin%
\definecolor{currentfill}{rgb}{0.121569,0.466667,0.705882}%
\pgfsetfillcolor{currentfill}%
\pgfsetfillopacity{0.982406}%
\pgfsetlinewidth{1.003750pt}%
\definecolor{currentstroke}{rgb}{0.121569,0.466667,0.705882}%
\pgfsetstrokecolor{currentstroke}%
\pgfsetstrokeopacity{0.982406}%
\pgfsetdash{}{0pt}%
\pgfpathmoveto{\pgfqpoint{2.266525in}{0.750225in}}%
\pgfpathcurveto{\pgfqpoint{2.274762in}{0.750225in}}{\pgfqpoint{2.282662in}{0.753497in}}{\pgfqpoint{2.288485in}{0.759321in}}%
\pgfpathcurveto{\pgfqpoint{2.294309in}{0.765145in}}{\pgfqpoint{2.297582in}{0.773045in}}{\pgfqpoint{2.297582in}{0.781282in}}%
\pgfpathcurveto{\pgfqpoint{2.297582in}{0.789518in}}{\pgfqpoint{2.294309in}{0.797418in}}{\pgfqpoint{2.288485in}{0.803242in}}%
\pgfpathcurveto{\pgfqpoint{2.282662in}{0.809066in}}{\pgfqpoint{2.274762in}{0.812338in}}{\pgfqpoint{2.266525in}{0.812338in}}%
\pgfpathcurveto{\pgfqpoint{2.258289in}{0.812338in}}{\pgfqpoint{2.250389in}{0.809066in}}{\pgfqpoint{2.244565in}{0.803242in}}%
\pgfpathcurveto{\pgfqpoint{2.238741in}{0.797418in}}{\pgfqpoint{2.235469in}{0.789518in}}{\pgfqpoint{2.235469in}{0.781282in}}%
\pgfpathcurveto{\pgfqpoint{2.235469in}{0.773045in}}{\pgfqpoint{2.238741in}{0.765145in}}{\pgfqpoint{2.244565in}{0.759321in}}%
\pgfpathcurveto{\pgfqpoint{2.250389in}{0.753497in}}{\pgfqpoint{2.258289in}{0.750225in}}{\pgfqpoint{2.266525in}{0.750225in}}%
\pgfpathclose%
\pgfusepath{stroke,fill}%
\end{pgfscope}%
\begin{pgfscope}%
\pgfpathrectangle{\pgfqpoint{0.100000in}{0.220728in}}{\pgfqpoint{3.696000in}{3.696000in}}%
\pgfusepath{clip}%
\pgfsetbuttcap%
\pgfsetroundjoin%
\definecolor{currentfill}{rgb}{0.121569,0.466667,0.705882}%
\pgfsetfillcolor{currentfill}%
\pgfsetfillopacity{0.982609}%
\pgfsetlinewidth{1.003750pt}%
\definecolor{currentstroke}{rgb}{0.121569,0.466667,0.705882}%
\pgfsetstrokecolor{currentstroke}%
\pgfsetstrokeopacity{0.982609}%
\pgfsetdash{}{0pt}%
\pgfpathmoveto{\pgfqpoint{2.436953in}{0.844835in}}%
\pgfpathcurveto{\pgfqpoint{2.445189in}{0.844835in}}{\pgfqpoint{2.453089in}{0.848107in}}{\pgfqpoint{2.458913in}{0.853931in}}%
\pgfpathcurveto{\pgfqpoint{2.464737in}{0.859755in}}{\pgfqpoint{2.468009in}{0.867655in}}{\pgfqpoint{2.468009in}{0.875891in}}%
\pgfpathcurveto{\pgfqpoint{2.468009in}{0.884128in}}{\pgfqpoint{2.464737in}{0.892028in}}{\pgfqpoint{2.458913in}{0.897852in}}%
\pgfpathcurveto{\pgfqpoint{2.453089in}{0.903676in}}{\pgfqpoint{2.445189in}{0.906948in}}{\pgfqpoint{2.436953in}{0.906948in}}%
\pgfpathcurveto{\pgfqpoint{2.428716in}{0.906948in}}{\pgfqpoint{2.420816in}{0.903676in}}{\pgfqpoint{2.414992in}{0.897852in}}%
\pgfpathcurveto{\pgfqpoint{2.409168in}{0.892028in}}{\pgfqpoint{2.405896in}{0.884128in}}{\pgfqpoint{2.405896in}{0.875891in}}%
\pgfpathcurveto{\pgfqpoint{2.405896in}{0.867655in}}{\pgfqpoint{2.409168in}{0.859755in}}{\pgfqpoint{2.414992in}{0.853931in}}%
\pgfpathcurveto{\pgfqpoint{2.420816in}{0.848107in}}{\pgfqpoint{2.428716in}{0.844835in}}{\pgfqpoint{2.436953in}{0.844835in}}%
\pgfpathclose%
\pgfusepath{stroke,fill}%
\end{pgfscope}%
\begin{pgfscope}%
\pgfpathrectangle{\pgfqpoint{0.100000in}{0.220728in}}{\pgfqpoint{3.696000in}{3.696000in}}%
\pgfusepath{clip}%
\pgfsetbuttcap%
\pgfsetroundjoin%
\definecolor{currentfill}{rgb}{0.121569,0.466667,0.705882}%
\pgfsetfillcolor{currentfill}%
\pgfsetfillopacity{0.984737}%
\pgfsetlinewidth{1.003750pt}%
\definecolor{currentstroke}{rgb}{0.121569,0.466667,0.705882}%
\pgfsetstrokecolor{currentstroke}%
\pgfsetstrokeopacity{0.984737}%
\pgfsetdash{}{0pt}%
\pgfpathmoveto{\pgfqpoint{2.280647in}{0.745834in}}%
\pgfpathcurveto{\pgfqpoint{2.288883in}{0.745834in}}{\pgfqpoint{2.296783in}{0.749106in}}{\pgfqpoint{2.302607in}{0.754930in}}%
\pgfpathcurveto{\pgfqpoint{2.308431in}{0.760754in}}{\pgfqpoint{2.311703in}{0.768654in}}{\pgfqpoint{2.311703in}{0.776890in}}%
\pgfpathcurveto{\pgfqpoint{2.311703in}{0.785126in}}{\pgfqpoint{2.308431in}{0.793026in}}{\pgfqpoint{2.302607in}{0.798850in}}%
\pgfpathcurveto{\pgfqpoint{2.296783in}{0.804674in}}{\pgfqpoint{2.288883in}{0.807947in}}{\pgfqpoint{2.280647in}{0.807947in}}%
\pgfpathcurveto{\pgfqpoint{2.272411in}{0.807947in}}{\pgfqpoint{2.264511in}{0.804674in}}{\pgfqpoint{2.258687in}{0.798850in}}%
\pgfpathcurveto{\pgfqpoint{2.252863in}{0.793026in}}{\pgfqpoint{2.249590in}{0.785126in}}{\pgfqpoint{2.249590in}{0.776890in}}%
\pgfpathcurveto{\pgfqpoint{2.249590in}{0.768654in}}{\pgfqpoint{2.252863in}{0.760754in}}{\pgfqpoint{2.258687in}{0.754930in}}%
\pgfpathcurveto{\pgfqpoint{2.264511in}{0.749106in}}{\pgfqpoint{2.272411in}{0.745834in}}{\pgfqpoint{2.280647in}{0.745834in}}%
\pgfpathclose%
\pgfusepath{stroke,fill}%
\end{pgfscope}%
\begin{pgfscope}%
\pgfpathrectangle{\pgfqpoint{0.100000in}{0.220728in}}{\pgfqpoint{3.696000in}{3.696000in}}%
\pgfusepath{clip}%
\pgfsetbuttcap%
\pgfsetroundjoin%
\definecolor{currentfill}{rgb}{0.121569,0.466667,0.705882}%
\pgfsetfillcolor{currentfill}%
\pgfsetfillopacity{0.984917}%
\pgfsetlinewidth{1.003750pt}%
\definecolor{currentstroke}{rgb}{0.121569,0.466667,0.705882}%
\pgfsetstrokecolor{currentstroke}%
\pgfsetstrokeopacity{0.984917}%
\pgfsetdash{}{0pt}%
\pgfpathmoveto{\pgfqpoint{2.431455in}{0.832538in}}%
\pgfpathcurveto{\pgfqpoint{2.439692in}{0.832538in}}{\pgfqpoint{2.447592in}{0.835810in}}{\pgfqpoint{2.453416in}{0.841634in}}%
\pgfpathcurveto{\pgfqpoint{2.459240in}{0.847458in}}{\pgfqpoint{2.462512in}{0.855358in}}{\pgfqpoint{2.462512in}{0.863595in}}%
\pgfpathcurveto{\pgfqpoint{2.462512in}{0.871831in}}{\pgfqpoint{2.459240in}{0.879731in}}{\pgfqpoint{2.453416in}{0.885555in}}%
\pgfpathcurveto{\pgfqpoint{2.447592in}{0.891379in}}{\pgfqpoint{2.439692in}{0.894651in}}{\pgfqpoint{2.431455in}{0.894651in}}%
\pgfpathcurveto{\pgfqpoint{2.423219in}{0.894651in}}{\pgfqpoint{2.415319in}{0.891379in}}{\pgfqpoint{2.409495in}{0.885555in}}%
\pgfpathcurveto{\pgfqpoint{2.403671in}{0.879731in}}{\pgfqpoint{2.400399in}{0.871831in}}{\pgfqpoint{2.400399in}{0.863595in}}%
\pgfpathcurveto{\pgfqpoint{2.400399in}{0.855358in}}{\pgfqpoint{2.403671in}{0.847458in}}{\pgfqpoint{2.409495in}{0.841634in}}%
\pgfpathcurveto{\pgfqpoint{2.415319in}{0.835810in}}{\pgfqpoint{2.423219in}{0.832538in}}{\pgfqpoint{2.431455in}{0.832538in}}%
\pgfpathclose%
\pgfusepath{stroke,fill}%
\end{pgfscope}%
\begin{pgfscope}%
\pgfpathrectangle{\pgfqpoint{0.100000in}{0.220728in}}{\pgfqpoint{3.696000in}{3.696000in}}%
\pgfusepath{clip}%
\pgfsetbuttcap%
\pgfsetroundjoin%
\definecolor{currentfill}{rgb}{0.121569,0.466667,0.705882}%
\pgfsetfillcolor{currentfill}%
\pgfsetfillopacity{0.985421}%
\pgfsetlinewidth{1.003750pt}%
\definecolor{currentstroke}{rgb}{0.121569,0.466667,0.705882}%
\pgfsetstrokecolor{currentstroke}%
\pgfsetstrokeopacity{0.985421}%
\pgfsetdash{}{0pt}%
\pgfpathmoveto{\pgfqpoint{2.426965in}{0.825014in}}%
\pgfpathcurveto{\pgfqpoint{2.435201in}{0.825014in}}{\pgfqpoint{2.443101in}{0.828287in}}{\pgfqpoint{2.448925in}{0.834111in}}%
\pgfpathcurveto{\pgfqpoint{2.454749in}{0.839935in}}{\pgfqpoint{2.458021in}{0.847835in}}{\pgfqpoint{2.458021in}{0.856071in}}%
\pgfpathcurveto{\pgfqpoint{2.458021in}{0.864307in}}{\pgfqpoint{2.454749in}{0.872207in}}{\pgfqpoint{2.448925in}{0.878031in}}%
\pgfpathcurveto{\pgfqpoint{2.443101in}{0.883855in}}{\pgfqpoint{2.435201in}{0.887127in}}{\pgfqpoint{2.426965in}{0.887127in}}%
\pgfpathcurveto{\pgfqpoint{2.418729in}{0.887127in}}{\pgfqpoint{2.410829in}{0.883855in}}{\pgfqpoint{2.405005in}{0.878031in}}%
\pgfpathcurveto{\pgfqpoint{2.399181in}{0.872207in}}{\pgfqpoint{2.395908in}{0.864307in}}{\pgfqpoint{2.395908in}{0.856071in}}%
\pgfpathcurveto{\pgfqpoint{2.395908in}{0.847835in}}{\pgfqpoint{2.399181in}{0.839935in}}{\pgfqpoint{2.405005in}{0.834111in}}%
\pgfpathcurveto{\pgfqpoint{2.410829in}{0.828287in}}{\pgfqpoint{2.418729in}{0.825014in}}{\pgfqpoint{2.426965in}{0.825014in}}%
\pgfpathclose%
\pgfusepath{stroke,fill}%
\end{pgfscope}%
\begin{pgfscope}%
\pgfpathrectangle{\pgfqpoint{0.100000in}{0.220728in}}{\pgfqpoint{3.696000in}{3.696000in}}%
\pgfusepath{clip}%
\pgfsetbuttcap%
\pgfsetroundjoin%
\definecolor{currentfill}{rgb}{0.121569,0.466667,0.705882}%
\pgfsetfillcolor{currentfill}%
\pgfsetfillopacity{0.986001}%
\pgfsetlinewidth{1.003750pt}%
\definecolor{currentstroke}{rgb}{0.121569,0.466667,0.705882}%
\pgfsetstrokecolor{currentstroke}%
\pgfsetstrokeopacity{0.986001}%
\pgfsetdash{}{0pt}%
\pgfpathmoveto{\pgfqpoint{2.425804in}{0.820422in}}%
\pgfpathcurveto{\pgfqpoint{2.434040in}{0.820422in}}{\pgfqpoint{2.441940in}{0.823694in}}{\pgfqpoint{2.447764in}{0.829518in}}%
\pgfpathcurveto{\pgfqpoint{2.453588in}{0.835342in}}{\pgfqpoint{2.456860in}{0.843242in}}{\pgfqpoint{2.456860in}{0.851478in}}%
\pgfpathcurveto{\pgfqpoint{2.456860in}{0.859714in}}{\pgfqpoint{2.453588in}{0.867614in}}{\pgfqpoint{2.447764in}{0.873438in}}%
\pgfpathcurveto{\pgfqpoint{2.441940in}{0.879262in}}{\pgfqpoint{2.434040in}{0.882535in}}{\pgfqpoint{2.425804in}{0.882535in}}%
\pgfpathcurveto{\pgfqpoint{2.417568in}{0.882535in}}{\pgfqpoint{2.409668in}{0.879262in}}{\pgfqpoint{2.403844in}{0.873438in}}%
\pgfpathcurveto{\pgfqpoint{2.398020in}{0.867614in}}{\pgfqpoint{2.394747in}{0.859714in}}{\pgfqpoint{2.394747in}{0.851478in}}%
\pgfpathcurveto{\pgfqpoint{2.394747in}{0.843242in}}{\pgfqpoint{2.398020in}{0.835342in}}{\pgfqpoint{2.403844in}{0.829518in}}%
\pgfpathcurveto{\pgfqpoint{2.409668in}{0.823694in}}{\pgfqpoint{2.417568in}{0.820422in}}{\pgfqpoint{2.425804in}{0.820422in}}%
\pgfpathclose%
\pgfusepath{stroke,fill}%
\end{pgfscope}%
\begin{pgfscope}%
\pgfpathrectangle{\pgfqpoint{0.100000in}{0.220728in}}{\pgfqpoint{3.696000in}{3.696000in}}%
\pgfusepath{clip}%
\pgfsetbuttcap%
\pgfsetroundjoin%
\definecolor{currentfill}{rgb}{0.121569,0.466667,0.705882}%
\pgfsetfillcolor{currentfill}%
\pgfsetfillopacity{0.986671}%
\pgfsetlinewidth{1.003750pt}%
\definecolor{currentstroke}{rgb}{0.121569,0.466667,0.705882}%
\pgfsetstrokecolor{currentstroke}%
\pgfsetstrokeopacity{0.986671}%
\pgfsetdash{}{0pt}%
\pgfpathmoveto{\pgfqpoint{2.291404in}{0.744175in}}%
\pgfpathcurveto{\pgfqpoint{2.299641in}{0.744175in}}{\pgfqpoint{2.307541in}{0.747448in}}{\pgfqpoint{2.313365in}{0.753272in}}%
\pgfpathcurveto{\pgfqpoint{2.319189in}{0.759096in}}{\pgfqpoint{2.322461in}{0.766996in}}{\pgfqpoint{2.322461in}{0.775232in}}%
\pgfpathcurveto{\pgfqpoint{2.322461in}{0.783468in}}{\pgfqpoint{2.319189in}{0.791368in}}{\pgfqpoint{2.313365in}{0.797192in}}%
\pgfpathcurveto{\pgfqpoint{2.307541in}{0.803016in}}{\pgfqpoint{2.299641in}{0.806288in}}{\pgfqpoint{2.291404in}{0.806288in}}%
\pgfpathcurveto{\pgfqpoint{2.283168in}{0.806288in}}{\pgfqpoint{2.275268in}{0.803016in}}{\pgfqpoint{2.269444in}{0.797192in}}%
\pgfpathcurveto{\pgfqpoint{2.263620in}{0.791368in}}{\pgfqpoint{2.260348in}{0.783468in}}{\pgfqpoint{2.260348in}{0.775232in}}%
\pgfpathcurveto{\pgfqpoint{2.260348in}{0.766996in}}{\pgfqpoint{2.263620in}{0.759096in}}{\pgfqpoint{2.269444in}{0.753272in}}%
\pgfpathcurveto{\pgfqpoint{2.275268in}{0.747448in}}{\pgfqpoint{2.283168in}{0.744175in}}{\pgfqpoint{2.291404in}{0.744175in}}%
\pgfpathclose%
\pgfusepath{stroke,fill}%
\end{pgfscope}%
\begin{pgfscope}%
\pgfpathrectangle{\pgfqpoint{0.100000in}{0.220728in}}{\pgfqpoint{3.696000in}{3.696000in}}%
\pgfusepath{clip}%
\pgfsetbuttcap%
\pgfsetroundjoin%
\definecolor{currentfill}{rgb}{0.121569,0.466667,0.705882}%
\pgfsetfillcolor{currentfill}%
\pgfsetfillopacity{0.986721}%
\pgfsetlinewidth{1.003750pt}%
\definecolor{currentstroke}{rgb}{0.121569,0.466667,0.705882}%
\pgfsetstrokecolor{currentstroke}%
\pgfsetstrokeopacity{0.986721}%
\pgfsetdash{}{0pt}%
\pgfpathmoveto{\pgfqpoint{2.419192in}{0.810400in}}%
\pgfpathcurveto{\pgfqpoint{2.427429in}{0.810400in}}{\pgfqpoint{2.435329in}{0.813672in}}{\pgfqpoint{2.441153in}{0.819496in}}%
\pgfpathcurveto{\pgfqpoint{2.446977in}{0.825320in}}{\pgfqpoint{2.450249in}{0.833220in}}{\pgfqpoint{2.450249in}{0.841456in}}%
\pgfpathcurveto{\pgfqpoint{2.450249in}{0.849693in}}{\pgfqpoint{2.446977in}{0.857593in}}{\pgfqpoint{2.441153in}{0.863416in}}%
\pgfpathcurveto{\pgfqpoint{2.435329in}{0.869240in}}{\pgfqpoint{2.427429in}{0.872513in}}{\pgfqpoint{2.419192in}{0.872513in}}%
\pgfpathcurveto{\pgfqpoint{2.410956in}{0.872513in}}{\pgfqpoint{2.403056in}{0.869240in}}{\pgfqpoint{2.397232in}{0.863416in}}%
\pgfpathcurveto{\pgfqpoint{2.391408in}{0.857593in}}{\pgfqpoint{2.388136in}{0.849693in}}{\pgfqpoint{2.388136in}{0.841456in}}%
\pgfpathcurveto{\pgfqpoint{2.388136in}{0.833220in}}{\pgfqpoint{2.391408in}{0.825320in}}{\pgfqpoint{2.397232in}{0.819496in}}%
\pgfpathcurveto{\pgfqpoint{2.403056in}{0.813672in}}{\pgfqpoint{2.410956in}{0.810400in}}{\pgfqpoint{2.419192in}{0.810400in}}%
\pgfpathclose%
\pgfusepath{stroke,fill}%
\end{pgfscope}%
\begin{pgfscope}%
\pgfpathrectangle{\pgfqpoint{0.100000in}{0.220728in}}{\pgfqpoint{3.696000in}{3.696000in}}%
\pgfusepath{clip}%
\pgfsetbuttcap%
\pgfsetroundjoin%
\definecolor{currentfill}{rgb}{0.121569,0.466667,0.705882}%
\pgfsetfillcolor{currentfill}%
\pgfsetfillopacity{0.987393}%
\pgfsetlinewidth{1.003750pt}%
\definecolor{currentstroke}{rgb}{0.121569,0.466667,0.705882}%
\pgfsetstrokecolor{currentstroke}%
\pgfsetstrokeopacity{0.987393}%
\pgfsetdash{}{0pt}%
\pgfpathmoveto{\pgfqpoint{2.299419in}{0.738700in}}%
\pgfpathcurveto{\pgfqpoint{2.307655in}{0.738700in}}{\pgfqpoint{2.315555in}{0.741972in}}{\pgfqpoint{2.321379in}{0.747796in}}%
\pgfpathcurveto{\pgfqpoint{2.327203in}{0.753620in}}{\pgfqpoint{2.330476in}{0.761520in}}{\pgfqpoint{2.330476in}{0.769757in}}%
\pgfpathcurveto{\pgfqpoint{2.330476in}{0.777993in}}{\pgfqpoint{2.327203in}{0.785893in}}{\pgfqpoint{2.321379in}{0.791717in}}%
\pgfpathcurveto{\pgfqpoint{2.315555in}{0.797541in}}{\pgfqpoint{2.307655in}{0.800813in}}{\pgfqpoint{2.299419in}{0.800813in}}%
\pgfpathcurveto{\pgfqpoint{2.291183in}{0.800813in}}{\pgfqpoint{2.283283in}{0.797541in}}{\pgfqpoint{2.277459in}{0.791717in}}%
\pgfpathcurveto{\pgfqpoint{2.271635in}{0.785893in}}{\pgfqpoint{2.268363in}{0.777993in}}{\pgfqpoint{2.268363in}{0.769757in}}%
\pgfpathcurveto{\pgfqpoint{2.268363in}{0.761520in}}{\pgfqpoint{2.271635in}{0.753620in}}{\pgfqpoint{2.277459in}{0.747796in}}%
\pgfpathcurveto{\pgfqpoint{2.283283in}{0.741972in}}{\pgfqpoint{2.291183in}{0.738700in}}{\pgfqpoint{2.299419in}{0.738700in}}%
\pgfpathclose%
\pgfusepath{stroke,fill}%
\end{pgfscope}%
\begin{pgfscope}%
\pgfpathrectangle{\pgfqpoint{0.100000in}{0.220728in}}{\pgfqpoint{3.696000in}{3.696000in}}%
\pgfusepath{clip}%
\pgfsetbuttcap%
\pgfsetroundjoin%
\definecolor{currentfill}{rgb}{0.121569,0.466667,0.705882}%
\pgfsetfillcolor{currentfill}%
\pgfsetfillopacity{0.988482}%
\pgfsetlinewidth{1.003750pt}%
\definecolor{currentstroke}{rgb}{0.121569,0.466667,0.705882}%
\pgfsetstrokecolor{currentstroke}%
\pgfsetstrokeopacity{0.988482}%
\pgfsetdash{}{0pt}%
\pgfpathmoveto{\pgfqpoint{2.415689in}{0.796427in}}%
\pgfpathcurveto{\pgfqpoint{2.423926in}{0.796427in}}{\pgfqpoint{2.431826in}{0.799699in}}{\pgfqpoint{2.437650in}{0.805523in}}%
\pgfpathcurveto{\pgfqpoint{2.443474in}{0.811347in}}{\pgfqpoint{2.446746in}{0.819247in}}{\pgfqpoint{2.446746in}{0.827484in}}%
\pgfpathcurveto{\pgfqpoint{2.446746in}{0.835720in}}{\pgfqpoint{2.443474in}{0.843620in}}{\pgfqpoint{2.437650in}{0.849444in}}%
\pgfpathcurveto{\pgfqpoint{2.431826in}{0.855268in}}{\pgfqpoint{2.423926in}{0.858540in}}{\pgfqpoint{2.415689in}{0.858540in}}%
\pgfpathcurveto{\pgfqpoint{2.407453in}{0.858540in}}{\pgfqpoint{2.399553in}{0.855268in}}{\pgfqpoint{2.393729in}{0.849444in}}%
\pgfpathcurveto{\pgfqpoint{2.387905in}{0.843620in}}{\pgfqpoint{2.384633in}{0.835720in}}{\pgfqpoint{2.384633in}{0.827484in}}%
\pgfpathcurveto{\pgfqpoint{2.384633in}{0.819247in}}{\pgfqpoint{2.387905in}{0.811347in}}{\pgfqpoint{2.393729in}{0.805523in}}%
\pgfpathcurveto{\pgfqpoint{2.399553in}{0.799699in}}{\pgfqpoint{2.407453in}{0.796427in}}{\pgfqpoint{2.415689in}{0.796427in}}%
\pgfpathclose%
\pgfusepath{stroke,fill}%
\end{pgfscope}%
\begin{pgfscope}%
\pgfpathrectangle{\pgfqpoint{0.100000in}{0.220728in}}{\pgfqpoint{3.696000in}{3.696000in}}%
\pgfusepath{clip}%
\pgfsetbuttcap%
\pgfsetroundjoin%
\definecolor{currentfill}{rgb}{0.121569,0.466667,0.705882}%
\pgfsetfillcolor{currentfill}%
\pgfsetfillopacity{0.990119}%
\pgfsetlinewidth{1.003750pt}%
\definecolor{currentstroke}{rgb}{0.121569,0.466667,0.705882}%
\pgfsetstrokecolor{currentstroke}%
\pgfsetstrokeopacity{0.990119}%
\pgfsetdash{}{0pt}%
\pgfpathmoveto{\pgfqpoint{2.314435in}{0.735680in}}%
\pgfpathcurveto{\pgfqpoint{2.322671in}{0.735680in}}{\pgfqpoint{2.330572in}{0.738953in}}{\pgfqpoint{2.336395in}{0.744776in}}%
\pgfpathcurveto{\pgfqpoint{2.342219in}{0.750600in}}{\pgfqpoint{2.345492in}{0.758500in}}{\pgfqpoint{2.345492in}{0.766737in}}%
\pgfpathcurveto{\pgfqpoint{2.345492in}{0.774973in}}{\pgfqpoint{2.342219in}{0.782873in}}{\pgfqpoint{2.336395in}{0.788697in}}%
\pgfpathcurveto{\pgfqpoint{2.330572in}{0.794521in}}{\pgfqpoint{2.322671in}{0.797793in}}{\pgfqpoint{2.314435in}{0.797793in}}%
\pgfpathcurveto{\pgfqpoint{2.306199in}{0.797793in}}{\pgfqpoint{2.298299in}{0.794521in}}{\pgfqpoint{2.292475in}{0.788697in}}%
\pgfpathcurveto{\pgfqpoint{2.286651in}{0.782873in}}{\pgfqpoint{2.283379in}{0.774973in}}{\pgfqpoint{2.283379in}{0.766737in}}%
\pgfpathcurveto{\pgfqpoint{2.283379in}{0.758500in}}{\pgfqpoint{2.286651in}{0.750600in}}{\pgfqpoint{2.292475in}{0.744776in}}%
\pgfpathcurveto{\pgfqpoint{2.298299in}{0.738953in}}{\pgfqpoint{2.306199in}{0.735680in}}{\pgfqpoint{2.314435in}{0.735680in}}%
\pgfpathclose%
\pgfusepath{stroke,fill}%
\end{pgfscope}%
\begin{pgfscope}%
\pgfpathrectangle{\pgfqpoint{0.100000in}{0.220728in}}{\pgfqpoint{3.696000in}{3.696000in}}%
\pgfusepath{clip}%
\pgfsetbuttcap%
\pgfsetroundjoin%
\definecolor{currentfill}{rgb}{0.121569,0.466667,0.705882}%
\pgfsetfillcolor{currentfill}%
\pgfsetfillopacity{0.990167}%
\pgfsetlinewidth{1.003750pt}%
\definecolor{currentstroke}{rgb}{0.121569,0.466667,0.705882}%
\pgfsetstrokecolor{currentstroke}%
\pgfsetstrokeopacity{0.990167}%
\pgfsetdash{}{0pt}%
\pgfpathmoveto{\pgfqpoint{2.407575in}{0.779988in}}%
\pgfpathcurveto{\pgfqpoint{2.415812in}{0.779988in}}{\pgfqpoint{2.423712in}{0.783260in}}{\pgfqpoint{2.429536in}{0.789084in}}%
\pgfpathcurveto{\pgfqpoint{2.435360in}{0.794908in}}{\pgfqpoint{2.438632in}{0.802808in}}{\pgfqpoint{2.438632in}{0.811045in}}%
\pgfpathcurveto{\pgfqpoint{2.438632in}{0.819281in}}{\pgfqpoint{2.435360in}{0.827181in}}{\pgfqpoint{2.429536in}{0.833005in}}%
\pgfpathcurveto{\pgfqpoint{2.423712in}{0.838829in}}{\pgfqpoint{2.415812in}{0.842101in}}{\pgfqpoint{2.407575in}{0.842101in}}%
\pgfpathcurveto{\pgfqpoint{2.399339in}{0.842101in}}{\pgfqpoint{2.391439in}{0.838829in}}{\pgfqpoint{2.385615in}{0.833005in}}%
\pgfpathcurveto{\pgfqpoint{2.379791in}{0.827181in}}{\pgfqpoint{2.376519in}{0.819281in}}{\pgfqpoint{2.376519in}{0.811045in}}%
\pgfpathcurveto{\pgfqpoint{2.376519in}{0.802808in}}{\pgfqpoint{2.379791in}{0.794908in}}{\pgfqpoint{2.385615in}{0.789084in}}%
\pgfpathcurveto{\pgfqpoint{2.391439in}{0.783260in}}{\pgfqpoint{2.399339in}{0.779988in}}{\pgfqpoint{2.407575in}{0.779988in}}%
\pgfpathclose%
\pgfusepath{stroke,fill}%
\end{pgfscope}%
\begin{pgfscope}%
\pgfpathrectangle{\pgfqpoint{0.100000in}{0.220728in}}{\pgfqpoint{3.696000in}{3.696000in}}%
\pgfusepath{clip}%
\pgfsetbuttcap%
\pgfsetroundjoin%
\definecolor{currentfill}{rgb}{0.121569,0.466667,0.705882}%
\pgfsetfillcolor{currentfill}%
\pgfsetfillopacity{0.991498}%
\pgfsetlinewidth{1.003750pt}%
\definecolor{currentstroke}{rgb}{0.121569,0.466667,0.705882}%
\pgfsetstrokecolor{currentstroke}%
\pgfsetstrokeopacity{0.991498}%
\pgfsetdash{}{0pt}%
\pgfpathmoveto{\pgfqpoint{2.328924in}{0.731832in}}%
\pgfpathcurveto{\pgfqpoint{2.337160in}{0.731832in}}{\pgfqpoint{2.345060in}{0.735104in}}{\pgfqpoint{2.350884in}{0.740928in}}%
\pgfpathcurveto{\pgfqpoint{2.356708in}{0.746752in}}{\pgfqpoint{2.359981in}{0.754652in}}{\pgfqpoint{2.359981in}{0.762888in}}%
\pgfpathcurveto{\pgfqpoint{2.359981in}{0.771124in}}{\pgfqpoint{2.356708in}{0.779025in}}{\pgfqpoint{2.350884in}{0.784848in}}%
\pgfpathcurveto{\pgfqpoint{2.345060in}{0.790672in}}{\pgfqpoint{2.337160in}{0.793945in}}{\pgfqpoint{2.328924in}{0.793945in}}%
\pgfpathcurveto{\pgfqpoint{2.320688in}{0.793945in}}{\pgfqpoint{2.312788in}{0.790672in}}{\pgfqpoint{2.306964in}{0.784848in}}%
\pgfpathcurveto{\pgfqpoint{2.301140in}{0.779025in}}{\pgfqpoint{2.297868in}{0.771124in}}{\pgfqpoint{2.297868in}{0.762888in}}%
\pgfpathcurveto{\pgfqpoint{2.297868in}{0.754652in}}{\pgfqpoint{2.301140in}{0.746752in}}{\pgfqpoint{2.306964in}{0.740928in}}%
\pgfpathcurveto{\pgfqpoint{2.312788in}{0.735104in}}{\pgfqpoint{2.320688in}{0.731832in}}{\pgfqpoint{2.328924in}{0.731832in}}%
\pgfpathclose%
\pgfusepath{stroke,fill}%
\end{pgfscope}%
\begin{pgfscope}%
\pgfpathrectangle{\pgfqpoint{0.100000in}{0.220728in}}{\pgfqpoint{3.696000in}{3.696000in}}%
\pgfusepath{clip}%
\pgfsetbuttcap%
\pgfsetroundjoin%
\definecolor{currentfill}{rgb}{0.121569,0.466667,0.705882}%
\pgfsetfillcolor{currentfill}%
\pgfsetfillopacity{0.992687}%
\pgfsetlinewidth{1.003750pt}%
\definecolor{currentstroke}{rgb}{0.121569,0.466667,0.705882}%
\pgfsetstrokecolor{currentstroke}%
\pgfsetstrokeopacity{0.992687}%
\pgfsetdash{}{0pt}%
\pgfpathmoveto{\pgfqpoint{2.402374in}{0.762229in}}%
\pgfpathcurveto{\pgfqpoint{2.410611in}{0.762229in}}{\pgfqpoint{2.418511in}{0.765501in}}{\pgfqpoint{2.424335in}{0.771325in}}%
\pgfpathcurveto{\pgfqpoint{2.430158in}{0.777149in}}{\pgfqpoint{2.433431in}{0.785049in}}{\pgfqpoint{2.433431in}{0.793285in}}%
\pgfpathcurveto{\pgfqpoint{2.433431in}{0.801522in}}{\pgfqpoint{2.430158in}{0.809422in}}{\pgfqpoint{2.424335in}{0.815246in}}%
\pgfpathcurveto{\pgfqpoint{2.418511in}{0.821070in}}{\pgfqpoint{2.410611in}{0.824342in}}{\pgfqpoint{2.402374in}{0.824342in}}%
\pgfpathcurveto{\pgfqpoint{2.394138in}{0.824342in}}{\pgfqpoint{2.386238in}{0.821070in}}{\pgfqpoint{2.380414in}{0.815246in}}%
\pgfpathcurveto{\pgfqpoint{2.374590in}{0.809422in}}{\pgfqpoint{2.371318in}{0.801522in}}{\pgfqpoint{2.371318in}{0.793285in}}%
\pgfpathcurveto{\pgfqpoint{2.371318in}{0.785049in}}{\pgfqpoint{2.374590in}{0.777149in}}{\pgfqpoint{2.380414in}{0.771325in}}%
\pgfpathcurveto{\pgfqpoint{2.386238in}{0.765501in}}{\pgfqpoint{2.394138in}{0.762229in}}{\pgfqpoint{2.402374in}{0.762229in}}%
\pgfpathclose%
\pgfusepath{stroke,fill}%
\end{pgfscope}%
\begin{pgfscope}%
\pgfpathrectangle{\pgfqpoint{0.100000in}{0.220728in}}{\pgfqpoint{3.696000in}{3.696000in}}%
\pgfusepath{clip}%
\pgfsetbuttcap%
\pgfsetroundjoin%
\definecolor{currentfill}{rgb}{0.121569,0.466667,0.705882}%
\pgfsetfillcolor{currentfill}%
\pgfsetfillopacity{0.993306}%
\pgfsetlinewidth{1.003750pt}%
\definecolor{currentstroke}{rgb}{0.121569,0.466667,0.705882}%
\pgfsetstrokecolor{currentstroke}%
\pgfsetstrokeopacity{0.993306}%
\pgfsetdash{}{0pt}%
\pgfpathmoveto{\pgfqpoint{2.339377in}{0.728685in}}%
\pgfpathcurveto{\pgfqpoint{2.347614in}{0.728685in}}{\pgfqpoint{2.355514in}{0.731957in}}{\pgfqpoint{2.361338in}{0.737781in}}%
\pgfpathcurveto{\pgfqpoint{2.367162in}{0.743605in}}{\pgfqpoint{2.370434in}{0.751505in}}{\pgfqpoint{2.370434in}{0.759742in}}%
\pgfpathcurveto{\pgfqpoint{2.370434in}{0.767978in}}{\pgfqpoint{2.367162in}{0.775878in}}{\pgfqpoint{2.361338in}{0.781702in}}%
\pgfpathcurveto{\pgfqpoint{2.355514in}{0.787526in}}{\pgfqpoint{2.347614in}{0.790798in}}{\pgfqpoint{2.339377in}{0.790798in}}%
\pgfpathcurveto{\pgfqpoint{2.331141in}{0.790798in}}{\pgfqpoint{2.323241in}{0.787526in}}{\pgfqpoint{2.317417in}{0.781702in}}%
\pgfpathcurveto{\pgfqpoint{2.311593in}{0.775878in}}{\pgfqpoint{2.308321in}{0.767978in}}{\pgfqpoint{2.308321in}{0.759742in}}%
\pgfpathcurveto{\pgfqpoint{2.308321in}{0.751505in}}{\pgfqpoint{2.311593in}{0.743605in}}{\pgfqpoint{2.317417in}{0.737781in}}%
\pgfpathcurveto{\pgfqpoint{2.323241in}{0.731957in}}{\pgfqpoint{2.331141in}{0.728685in}}{\pgfqpoint{2.339377in}{0.728685in}}%
\pgfpathclose%
\pgfusepath{stroke,fill}%
\end{pgfscope}%
\begin{pgfscope}%
\pgfpathrectangle{\pgfqpoint{0.100000in}{0.220728in}}{\pgfqpoint{3.696000in}{3.696000in}}%
\pgfusepath{clip}%
\pgfsetbuttcap%
\pgfsetroundjoin%
\definecolor{currentfill}{rgb}{0.121569,0.466667,0.705882}%
\pgfsetfillcolor{currentfill}%
\pgfsetfillopacity{0.994921}%
\pgfsetlinewidth{1.003750pt}%
\definecolor{currentstroke}{rgb}{0.121569,0.466667,0.705882}%
\pgfsetstrokecolor{currentstroke}%
\pgfsetstrokeopacity{0.994921}%
\pgfsetdash{}{0pt}%
\pgfpathmoveto{\pgfqpoint{2.347361in}{0.727187in}}%
\pgfpathcurveto{\pgfqpoint{2.355597in}{0.727187in}}{\pgfqpoint{2.363497in}{0.730459in}}{\pgfqpoint{2.369321in}{0.736283in}}%
\pgfpathcurveto{\pgfqpoint{2.375145in}{0.742107in}}{\pgfqpoint{2.378418in}{0.750007in}}{\pgfqpoint{2.378418in}{0.758243in}}%
\pgfpathcurveto{\pgfqpoint{2.378418in}{0.766480in}}{\pgfqpoint{2.375145in}{0.774380in}}{\pgfqpoint{2.369321in}{0.780204in}}%
\pgfpathcurveto{\pgfqpoint{2.363497in}{0.786027in}}{\pgfqpoint{2.355597in}{0.789300in}}{\pgfqpoint{2.347361in}{0.789300in}}%
\pgfpathcurveto{\pgfqpoint{2.339125in}{0.789300in}}{\pgfqpoint{2.331225in}{0.786027in}}{\pgfqpoint{2.325401in}{0.780204in}}%
\pgfpathcurveto{\pgfqpoint{2.319577in}{0.774380in}}{\pgfqpoint{2.316305in}{0.766480in}}{\pgfqpoint{2.316305in}{0.758243in}}%
\pgfpathcurveto{\pgfqpoint{2.316305in}{0.750007in}}{\pgfqpoint{2.319577in}{0.742107in}}{\pgfqpoint{2.325401in}{0.736283in}}%
\pgfpathcurveto{\pgfqpoint{2.331225in}{0.730459in}}{\pgfqpoint{2.339125in}{0.727187in}}{\pgfqpoint{2.347361in}{0.727187in}}%
\pgfpathclose%
\pgfusepath{stroke,fill}%
\end{pgfscope}%
\begin{pgfscope}%
\pgfpathrectangle{\pgfqpoint{0.100000in}{0.220728in}}{\pgfqpoint{3.696000in}{3.696000in}}%
\pgfusepath{clip}%
\pgfsetbuttcap%
\pgfsetroundjoin%
\definecolor{currentfill}{rgb}{0.121569,0.466667,0.705882}%
\pgfsetfillcolor{currentfill}%
\pgfsetfillopacity{0.995805}%
\pgfsetlinewidth{1.003750pt}%
\definecolor{currentstroke}{rgb}{0.121569,0.466667,0.705882}%
\pgfsetstrokecolor{currentstroke}%
\pgfsetstrokeopacity{0.995805}%
\pgfsetdash{}{0pt}%
\pgfpathmoveto{\pgfqpoint{2.397285in}{0.745054in}}%
\pgfpathcurveto{\pgfqpoint{2.405522in}{0.745054in}}{\pgfqpoint{2.413422in}{0.748326in}}{\pgfqpoint{2.419246in}{0.754150in}}%
\pgfpathcurveto{\pgfqpoint{2.425070in}{0.759974in}}{\pgfqpoint{2.428342in}{0.767874in}}{\pgfqpoint{2.428342in}{0.776111in}}%
\pgfpathcurveto{\pgfqpoint{2.428342in}{0.784347in}}{\pgfqpoint{2.425070in}{0.792247in}}{\pgfqpoint{2.419246in}{0.798071in}}%
\pgfpathcurveto{\pgfqpoint{2.413422in}{0.803895in}}{\pgfqpoint{2.405522in}{0.807167in}}{\pgfqpoint{2.397285in}{0.807167in}}%
\pgfpathcurveto{\pgfqpoint{2.389049in}{0.807167in}}{\pgfqpoint{2.381149in}{0.803895in}}{\pgfqpoint{2.375325in}{0.798071in}}%
\pgfpathcurveto{\pgfqpoint{2.369501in}{0.792247in}}{\pgfqpoint{2.366229in}{0.784347in}}{\pgfqpoint{2.366229in}{0.776111in}}%
\pgfpathcurveto{\pgfqpoint{2.366229in}{0.767874in}}{\pgfqpoint{2.369501in}{0.759974in}}{\pgfqpoint{2.375325in}{0.754150in}}%
\pgfpathcurveto{\pgfqpoint{2.381149in}{0.748326in}}{\pgfqpoint{2.389049in}{0.745054in}}{\pgfqpoint{2.397285in}{0.745054in}}%
\pgfpathclose%
\pgfusepath{stroke,fill}%
\end{pgfscope}%
\begin{pgfscope}%
\pgfpathrectangle{\pgfqpoint{0.100000in}{0.220728in}}{\pgfqpoint{3.696000in}{3.696000in}}%
\pgfusepath{clip}%
\pgfsetbuttcap%
\pgfsetroundjoin%
\definecolor{currentfill}{rgb}{0.121569,0.466667,0.705882}%
\pgfsetfillcolor{currentfill}%
\pgfsetfillopacity{0.996083}%
\pgfsetlinewidth{1.003750pt}%
\definecolor{currentstroke}{rgb}{0.121569,0.466667,0.705882}%
\pgfsetstrokecolor{currentstroke}%
\pgfsetstrokeopacity{0.996083}%
\pgfsetdash{}{0pt}%
\pgfpathmoveto{\pgfqpoint{2.354481in}{0.724417in}}%
\pgfpathcurveto{\pgfqpoint{2.362717in}{0.724417in}}{\pgfqpoint{2.370617in}{0.727689in}}{\pgfqpoint{2.376441in}{0.733513in}}%
\pgfpathcurveto{\pgfqpoint{2.382265in}{0.739337in}}{\pgfqpoint{2.385537in}{0.747237in}}{\pgfqpoint{2.385537in}{0.755473in}}%
\pgfpathcurveto{\pgfqpoint{2.385537in}{0.763709in}}{\pgfqpoint{2.382265in}{0.771609in}}{\pgfqpoint{2.376441in}{0.777433in}}%
\pgfpathcurveto{\pgfqpoint{2.370617in}{0.783257in}}{\pgfqpoint{2.362717in}{0.786530in}}{\pgfqpoint{2.354481in}{0.786530in}}%
\pgfpathcurveto{\pgfqpoint{2.346244in}{0.786530in}}{\pgfqpoint{2.338344in}{0.783257in}}{\pgfqpoint{2.332520in}{0.777433in}}%
\pgfpathcurveto{\pgfqpoint{2.326696in}{0.771609in}}{\pgfqpoint{2.323424in}{0.763709in}}{\pgfqpoint{2.323424in}{0.755473in}}%
\pgfpathcurveto{\pgfqpoint{2.323424in}{0.747237in}}{\pgfqpoint{2.326696in}{0.739337in}}{\pgfqpoint{2.332520in}{0.733513in}}%
\pgfpathcurveto{\pgfqpoint{2.338344in}{0.727689in}}{\pgfqpoint{2.346244in}{0.724417in}}{\pgfqpoint{2.354481in}{0.724417in}}%
\pgfpathclose%
\pgfusepath{stroke,fill}%
\end{pgfscope}%
\begin{pgfscope}%
\pgfpathrectangle{\pgfqpoint{0.100000in}{0.220728in}}{\pgfqpoint{3.696000in}{3.696000in}}%
\pgfusepath{clip}%
\pgfsetbuttcap%
\pgfsetroundjoin%
\definecolor{currentfill}{rgb}{0.121569,0.466667,0.705882}%
\pgfsetfillcolor{currentfill}%
\pgfsetfillopacity{0.997068}%
\pgfsetlinewidth{1.003750pt}%
\definecolor{currentstroke}{rgb}{0.121569,0.466667,0.705882}%
\pgfsetstrokecolor{currentstroke}%
\pgfsetstrokeopacity{0.997068}%
\pgfsetdash{}{0pt}%
\pgfpathmoveto{\pgfqpoint{2.391881in}{0.736867in}}%
\pgfpathcurveto{\pgfqpoint{2.400117in}{0.736867in}}{\pgfqpoint{2.408017in}{0.740139in}}{\pgfqpoint{2.413841in}{0.745963in}}%
\pgfpathcurveto{\pgfqpoint{2.419665in}{0.751787in}}{\pgfqpoint{2.422937in}{0.759687in}}{\pgfqpoint{2.422937in}{0.767924in}}%
\pgfpathcurveto{\pgfqpoint{2.422937in}{0.776160in}}{\pgfqpoint{2.419665in}{0.784060in}}{\pgfqpoint{2.413841in}{0.789884in}}%
\pgfpathcurveto{\pgfqpoint{2.408017in}{0.795708in}}{\pgfqpoint{2.400117in}{0.798980in}}{\pgfqpoint{2.391881in}{0.798980in}}%
\pgfpathcurveto{\pgfqpoint{2.383645in}{0.798980in}}{\pgfqpoint{2.375744in}{0.795708in}}{\pgfqpoint{2.369921in}{0.789884in}}%
\pgfpathcurveto{\pgfqpoint{2.364097in}{0.784060in}}{\pgfqpoint{2.360824in}{0.776160in}}{\pgfqpoint{2.360824in}{0.767924in}}%
\pgfpathcurveto{\pgfqpoint{2.360824in}{0.759687in}}{\pgfqpoint{2.364097in}{0.751787in}}{\pgfqpoint{2.369921in}{0.745963in}}%
\pgfpathcurveto{\pgfqpoint{2.375744in}{0.740139in}}{\pgfqpoint{2.383645in}{0.736867in}}{\pgfqpoint{2.391881in}{0.736867in}}%
\pgfpathclose%
\pgfusepath{stroke,fill}%
\end{pgfscope}%
\begin{pgfscope}%
\pgfpathrectangle{\pgfqpoint{0.100000in}{0.220728in}}{\pgfqpoint{3.696000in}{3.696000in}}%
\pgfusepath{clip}%
\pgfsetbuttcap%
\pgfsetroundjoin%
\definecolor{currentfill}{rgb}{0.121569,0.466667,0.705882}%
\pgfsetfillcolor{currentfill}%
\pgfsetfillopacity{0.997085}%
\pgfsetlinewidth{1.003750pt}%
\definecolor{currentstroke}{rgb}{0.121569,0.466667,0.705882}%
\pgfsetstrokecolor{currentstroke}%
\pgfsetstrokeopacity{0.997085}%
\pgfsetdash{}{0pt}%
\pgfpathmoveto{\pgfqpoint{2.360623in}{0.721366in}}%
\pgfpathcurveto{\pgfqpoint{2.368859in}{0.721366in}}{\pgfqpoint{2.376759in}{0.724638in}}{\pgfqpoint{2.382583in}{0.730462in}}%
\pgfpathcurveto{\pgfqpoint{2.388407in}{0.736286in}}{\pgfqpoint{2.391679in}{0.744186in}}{\pgfqpoint{2.391679in}{0.752423in}}%
\pgfpathcurveto{\pgfqpoint{2.391679in}{0.760659in}}{\pgfqpoint{2.388407in}{0.768559in}}{\pgfqpoint{2.382583in}{0.774383in}}%
\pgfpathcurveto{\pgfqpoint{2.376759in}{0.780207in}}{\pgfqpoint{2.368859in}{0.783479in}}{\pgfqpoint{2.360623in}{0.783479in}}%
\pgfpathcurveto{\pgfqpoint{2.352387in}{0.783479in}}{\pgfqpoint{2.344487in}{0.780207in}}{\pgfqpoint{2.338663in}{0.774383in}}%
\pgfpathcurveto{\pgfqpoint{2.332839in}{0.768559in}}{\pgfqpoint{2.329566in}{0.760659in}}{\pgfqpoint{2.329566in}{0.752423in}}%
\pgfpathcurveto{\pgfqpoint{2.329566in}{0.744186in}}{\pgfqpoint{2.332839in}{0.736286in}}{\pgfqpoint{2.338663in}{0.730462in}}%
\pgfpathcurveto{\pgfqpoint{2.344487in}{0.724638in}}{\pgfqpoint{2.352387in}{0.721366in}}{\pgfqpoint{2.360623in}{0.721366in}}%
\pgfpathclose%
\pgfusepath{stroke,fill}%
\end{pgfscope}%
\begin{pgfscope}%
\pgfpathrectangle{\pgfqpoint{0.100000in}{0.220728in}}{\pgfqpoint{3.696000in}{3.696000in}}%
\pgfusepath{clip}%
\pgfsetbuttcap%
\pgfsetroundjoin%
\definecolor{currentfill}{rgb}{0.121569,0.466667,0.705882}%
\pgfsetfillcolor{currentfill}%
\pgfsetfillopacity{0.997897}%
\pgfsetlinewidth{1.003750pt}%
\definecolor{currentstroke}{rgb}{0.121569,0.466667,0.705882}%
\pgfsetstrokecolor{currentstroke}%
\pgfsetstrokeopacity{0.997897}%
\pgfsetdash{}{0pt}%
\pgfpathmoveto{\pgfqpoint{2.389768in}{0.731700in}}%
\pgfpathcurveto{\pgfqpoint{2.398005in}{0.731700in}}{\pgfqpoint{2.405905in}{0.734973in}}{\pgfqpoint{2.411729in}{0.740797in}}%
\pgfpathcurveto{\pgfqpoint{2.417552in}{0.746621in}}{\pgfqpoint{2.420825in}{0.754521in}}{\pgfqpoint{2.420825in}{0.762757in}}%
\pgfpathcurveto{\pgfqpoint{2.420825in}{0.770993in}}{\pgfqpoint{2.417552in}{0.778893in}}{\pgfqpoint{2.411729in}{0.784717in}}%
\pgfpathcurveto{\pgfqpoint{2.405905in}{0.790541in}}{\pgfqpoint{2.398005in}{0.793813in}}{\pgfqpoint{2.389768in}{0.793813in}}%
\pgfpathcurveto{\pgfqpoint{2.381532in}{0.793813in}}{\pgfqpoint{2.373632in}{0.790541in}}{\pgfqpoint{2.367808in}{0.784717in}}%
\pgfpathcurveto{\pgfqpoint{2.361984in}{0.778893in}}{\pgfqpoint{2.358712in}{0.770993in}}{\pgfqpoint{2.358712in}{0.762757in}}%
\pgfpathcurveto{\pgfqpoint{2.358712in}{0.754521in}}{\pgfqpoint{2.361984in}{0.746621in}}{\pgfqpoint{2.367808in}{0.740797in}}%
\pgfpathcurveto{\pgfqpoint{2.373632in}{0.734973in}}{\pgfqpoint{2.381532in}{0.731700in}}{\pgfqpoint{2.389768in}{0.731700in}}%
\pgfpathclose%
\pgfusepath{stroke,fill}%
\end{pgfscope}%
\begin{pgfscope}%
\pgfpathrectangle{\pgfqpoint{0.100000in}{0.220728in}}{\pgfqpoint{3.696000in}{3.696000in}}%
\pgfusepath{clip}%
\pgfsetbuttcap%
\pgfsetroundjoin%
\definecolor{currentfill}{rgb}{0.121569,0.466667,0.705882}%
\pgfsetfillcolor{currentfill}%
\pgfsetfillopacity{0.997929}%
\pgfsetlinewidth{1.003750pt}%
\definecolor{currentstroke}{rgb}{0.121569,0.466667,0.705882}%
\pgfsetstrokecolor{currentstroke}%
\pgfsetstrokeopacity{0.997929}%
\pgfsetdash{}{0pt}%
\pgfpathmoveto{\pgfqpoint{2.365174in}{0.718694in}}%
\pgfpathcurveto{\pgfqpoint{2.373410in}{0.718694in}}{\pgfqpoint{2.381310in}{0.721967in}}{\pgfqpoint{2.387134in}{0.727791in}}%
\pgfpathcurveto{\pgfqpoint{2.392958in}{0.733615in}}{\pgfqpoint{2.396231in}{0.741515in}}{\pgfqpoint{2.396231in}{0.749751in}}%
\pgfpathcurveto{\pgfqpoint{2.396231in}{0.757987in}}{\pgfqpoint{2.392958in}{0.765887in}}{\pgfqpoint{2.387134in}{0.771711in}}%
\pgfpathcurveto{\pgfqpoint{2.381310in}{0.777535in}}{\pgfqpoint{2.373410in}{0.780807in}}{\pgfqpoint{2.365174in}{0.780807in}}%
\pgfpathcurveto{\pgfqpoint{2.356938in}{0.780807in}}{\pgfqpoint{2.349038in}{0.777535in}}{\pgfqpoint{2.343214in}{0.771711in}}%
\pgfpathcurveto{\pgfqpoint{2.337390in}{0.765887in}}{\pgfqpoint{2.334118in}{0.757987in}}{\pgfqpoint{2.334118in}{0.749751in}}%
\pgfpathcurveto{\pgfqpoint{2.334118in}{0.741515in}}{\pgfqpoint{2.337390in}{0.733615in}}{\pgfqpoint{2.343214in}{0.727791in}}%
\pgfpathcurveto{\pgfqpoint{2.349038in}{0.721967in}}{\pgfqpoint{2.356938in}{0.718694in}}{\pgfqpoint{2.365174in}{0.718694in}}%
\pgfpathclose%
\pgfusepath{stroke,fill}%
\end{pgfscope}%
\begin{pgfscope}%
\pgfpathrectangle{\pgfqpoint{0.100000in}{0.220728in}}{\pgfqpoint{3.696000in}{3.696000in}}%
\pgfusepath{clip}%
\pgfsetbuttcap%
\pgfsetroundjoin%
\definecolor{currentfill}{rgb}{0.121569,0.466667,0.705882}%
\pgfsetfillcolor{currentfill}%
\pgfsetfillopacity{0.998410}%
\pgfsetlinewidth{1.003750pt}%
\definecolor{currentstroke}{rgb}{0.121569,0.466667,0.705882}%
\pgfsetstrokecolor{currentstroke}%
\pgfsetstrokeopacity{0.998410}%
\pgfsetdash{}{0pt}%
\pgfpathmoveto{\pgfqpoint{2.367557in}{0.717587in}}%
\pgfpathcurveto{\pgfqpoint{2.375793in}{0.717587in}}{\pgfqpoint{2.383693in}{0.720859in}}{\pgfqpoint{2.389517in}{0.726683in}}%
\pgfpathcurveto{\pgfqpoint{2.395341in}{0.732507in}}{\pgfqpoint{2.398613in}{0.740407in}}{\pgfqpoint{2.398613in}{0.748643in}}%
\pgfpathcurveto{\pgfqpoint{2.398613in}{0.756880in}}{\pgfqpoint{2.395341in}{0.764780in}}{\pgfqpoint{2.389517in}{0.770604in}}%
\pgfpathcurveto{\pgfqpoint{2.383693in}{0.776428in}}{\pgfqpoint{2.375793in}{0.779700in}}{\pgfqpoint{2.367557in}{0.779700in}}%
\pgfpathcurveto{\pgfqpoint{2.359321in}{0.779700in}}{\pgfqpoint{2.351421in}{0.776428in}}{\pgfqpoint{2.345597in}{0.770604in}}%
\pgfpathcurveto{\pgfqpoint{2.339773in}{0.764780in}}{\pgfqpoint{2.336500in}{0.756880in}}{\pgfqpoint{2.336500in}{0.748643in}}%
\pgfpathcurveto{\pgfqpoint{2.336500in}{0.740407in}}{\pgfqpoint{2.339773in}{0.732507in}}{\pgfqpoint{2.345597in}{0.726683in}}%
\pgfpathcurveto{\pgfqpoint{2.351421in}{0.720859in}}{\pgfqpoint{2.359321in}{0.717587in}}{\pgfqpoint{2.367557in}{0.717587in}}%
\pgfpathclose%
\pgfusepath{stroke,fill}%
\end{pgfscope}%
\begin{pgfscope}%
\pgfpathrectangle{\pgfqpoint{0.100000in}{0.220728in}}{\pgfqpoint{3.696000in}{3.696000in}}%
\pgfusepath{clip}%
\pgfsetbuttcap%
\pgfsetroundjoin%
\definecolor{currentfill}{rgb}{0.121569,0.466667,0.705882}%
\pgfsetfillcolor{currentfill}%
\pgfsetfillopacity{0.998423}%
\pgfsetlinewidth{1.003750pt}%
\definecolor{currentstroke}{rgb}{0.121569,0.466667,0.705882}%
\pgfsetstrokecolor{currentstroke}%
\pgfsetstrokeopacity{0.998423}%
\pgfsetdash{}{0pt}%
\pgfpathmoveto{\pgfqpoint{2.367615in}{0.717553in}}%
\pgfpathcurveto{\pgfqpoint{2.375851in}{0.717553in}}{\pgfqpoint{2.383751in}{0.720825in}}{\pgfqpoint{2.389575in}{0.726649in}}%
\pgfpathcurveto{\pgfqpoint{2.395399in}{0.732473in}}{\pgfqpoint{2.398671in}{0.740373in}}{\pgfqpoint{2.398671in}{0.748609in}}%
\pgfpathcurveto{\pgfqpoint{2.398671in}{0.756846in}}{\pgfqpoint{2.395399in}{0.764746in}}{\pgfqpoint{2.389575in}{0.770570in}}%
\pgfpathcurveto{\pgfqpoint{2.383751in}{0.776394in}}{\pgfqpoint{2.375851in}{0.779666in}}{\pgfqpoint{2.367615in}{0.779666in}}%
\pgfpathcurveto{\pgfqpoint{2.359378in}{0.779666in}}{\pgfqpoint{2.351478in}{0.776394in}}{\pgfqpoint{2.345654in}{0.770570in}}%
\pgfpathcurveto{\pgfqpoint{2.339831in}{0.764746in}}{\pgfqpoint{2.336558in}{0.756846in}}{\pgfqpoint{2.336558in}{0.748609in}}%
\pgfpathcurveto{\pgfqpoint{2.336558in}{0.740373in}}{\pgfqpoint{2.339831in}{0.732473in}}{\pgfqpoint{2.345654in}{0.726649in}}%
\pgfpathcurveto{\pgfqpoint{2.351478in}{0.720825in}}{\pgfqpoint{2.359378in}{0.717553in}}{\pgfqpoint{2.367615in}{0.717553in}}%
\pgfpathclose%
\pgfusepath{stroke,fill}%
\end{pgfscope}%
\begin{pgfscope}%
\pgfpathrectangle{\pgfqpoint{0.100000in}{0.220728in}}{\pgfqpoint{3.696000in}{3.696000in}}%
\pgfusepath{clip}%
\pgfsetbuttcap%
\pgfsetroundjoin%
\definecolor{currentfill}{rgb}{0.121569,0.466667,0.705882}%
\pgfsetfillcolor{currentfill}%
\pgfsetfillopacity{0.998444}%
\pgfsetlinewidth{1.003750pt}%
\definecolor{currentstroke}{rgb}{0.121569,0.466667,0.705882}%
\pgfsetstrokecolor{currentstroke}%
\pgfsetstrokeopacity{0.998444}%
\pgfsetdash{}{0pt}%
\pgfpathmoveto{\pgfqpoint{2.367718in}{0.717484in}}%
\pgfpathcurveto{\pgfqpoint{2.375954in}{0.717484in}}{\pgfqpoint{2.383855in}{0.720756in}}{\pgfqpoint{2.389678in}{0.726580in}}%
\pgfpathcurveto{\pgfqpoint{2.395502in}{0.732404in}}{\pgfqpoint{2.398775in}{0.740304in}}{\pgfqpoint{2.398775in}{0.748540in}}%
\pgfpathcurveto{\pgfqpoint{2.398775in}{0.756777in}}{\pgfqpoint{2.395502in}{0.764677in}}{\pgfqpoint{2.389678in}{0.770501in}}%
\pgfpathcurveto{\pgfqpoint{2.383855in}{0.776324in}}{\pgfqpoint{2.375954in}{0.779597in}}{\pgfqpoint{2.367718in}{0.779597in}}%
\pgfpathcurveto{\pgfqpoint{2.359482in}{0.779597in}}{\pgfqpoint{2.351582in}{0.776324in}}{\pgfqpoint{2.345758in}{0.770501in}}%
\pgfpathcurveto{\pgfqpoint{2.339934in}{0.764677in}}{\pgfqpoint{2.336662in}{0.756777in}}{\pgfqpoint{2.336662in}{0.748540in}}%
\pgfpathcurveto{\pgfqpoint{2.336662in}{0.740304in}}{\pgfqpoint{2.339934in}{0.732404in}}{\pgfqpoint{2.345758in}{0.726580in}}%
\pgfpathcurveto{\pgfqpoint{2.351582in}{0.720756in}}{\pgfqpoint{2.359482in}{0.717484in}}{\pgfqpoint{2.367718in}{0.717484in}}%
\pgfpathclose%
\pgfusepath{stroke,fill}%
\end{pgfscope}%
\begin{pgfscope}%
\pgfpathrectangle{\pgfqpoint{0.100000in}{0.220728in}}{\pgfqpoint{3.696000in}{3.696000in}}%
\pgfusepath{clip}%
\pgfsetbuttcap%
\pgfsetroundjoin%
\definecolor{currentfill}{rgb}{0.121569,0.466667,0.705882}%
\pgfsetfillcolor{currentfill}%
\pgfsetfillopacity{0.998484}%
\pgfsetlinewidth{1.003750pt}%
\definecolor{currentstroke}{rgb}{0.121569,0.466667,0.705882}%
\pgfsetstrokecolor{currentstroke}%
\pgfsetstrokeopacity{0.998484}%
\pgfsetdash{}{0pt}%
\pgfpathmoveto{\pgfqpoint{2.367911in}{0.717374in}}%
\pgfpathcurveto{\pgfqpoint{2.376147in}{0.717374in}}{\pgfqpoint{2.384047in}{0.720646in}}{\pgfqpoint{2.389871in}{0.726470in}}%
\pgfpathcurveto{\pgfqpoint{2.395695in}{0.732294in}}{\pgfqpoint{2.398967in}{0.740194in}}{\pgfqpoint{2.398967in}{0.748431in}}%
\pgfpathcurveto{\pgfqpoint{2.398967in}{0.756667in}}{\pgfqpoint{2.395695in}{0.764567in}}{\pgfqpoint{2.389871in}{0.770391in}}%
\pgfpathcurveto{\pgfqpoint{2.384047in}{0.776215in}}{\pgfqpoint{2.376147in}{0.779487in}}{\pgfqpoint{2.367911in}{0.779487in}}%
\pgfpathcurveto{\pgfqpoint{2.359674in}{0.779487in}}{\pgfqpoint{2.351774in}{0.776215in}}{\pgfqpoint{2.345950in}{0.770391in}}%
\pgfpathcurveto{\pgfqpoint{2.340126in}{0.764567in}}{\pgfqpoint{2.336854in}{0.756667in}}{\pgfqpoint{2.336854in}{0.748431in}}%
\pgfpathcurveto{\pgfqpoint{2.336854in}{0.740194in}}{\pgfqpoint{2.340126in}{0.732294in}}{\pgfqpoint{2.345950in}{0.726470in}}%
\pgfpathcurveto{\pgfqpoint{2.351774in}{0.720646in}}{\pgfqpoint{2.359674in}{0.717374in}}{\pgfqpoint{2.367911in}{0.717374in}}%
\pgfpathclose%
\pgfusepath{stroke,fill}%
\end{pgfscope}%
\begin{pgfscope}%
\pgfpathrectangle{\pgfqpoint{0.100000in}{0.220728in}}{\pgfqpoint{3.696000in}{3.696000in}}%
\pgfusepath{clip}%
\pgfsetbuttcap%
\pgfsetroundjoin%
\definecolor{currentfill}{rgb}{0.121569,0.466667,0.705882}%
\pgfsetfillcolor{currentfill}%
\pgfsetfillopacity{0.998590}%
\pgfsetlinewidth{1.003750pt}%
\definecolor{currentstroke}{rgb}{0.121569,0.466667,0.705882}%
\pgfsetstrokecolor{currentstroke}%
\pgfsetstrokeopacity{0.998590}%
\pgfsetdash{}{0pt}%
\pgfpathmoveto{\pgfqpoint{2.368229in}{0.717212in}}%
\pgfpathcurveto{\pgfqpoint{2.376466in}{0.717212in}}{\pgfqpoint{2.384366in}{0.720484in}}{\pgfqpoint{2.390190in}{0.726308in}}%
\pgfpathcurveto{\pgfqpoint{2.396014in}{0.732132in}}{\pgfqpoint{2.399286in}{0.740032in}}{\pgfqpoint{2.399286in}{0.748269in}}%
\pgfpathcurveto{\pgfqpoint{2.399286in}{0.756505in}}{\pgfqpoint{2.396014in}{0.764405in}}{\pgfqpoint{2.390190in}{0.770229in}}%
\pgfpathcurveto{\pgfqpoint{2.384366in}{0.776053in}}{\pgfqpoint{2.376466in}{0.779325in}}{\pgfqpoint{2.368229in}{0.779325in}}%
\pgfpathcurveto{\pgfqpoint{2.359993in}{0.779325in}}{\pgfqpoint{2.352093in}{0.776053in}}{\pgfqpoint{2.346269in}{0.770229in}}%
\pgfpathcurveto{\pgfqpoint{2.340445in}{0.764405in}}{\pgfqpoint{2.337173in}{0.756505in}}{\pgfqpoint{2.337173in}{0.748269in}}%
\pgfpathcurveto{\pgfqpoint{2.337173in}{0.740032in}}{\pgfqpoint{2.340445in}{0.732132in}}{\pgfqpoint{2.346269in}{0.726308in}}%
\pgfpathcurveto{\pgfqpoint{2.352093in}{0.720484in}}{\pgfqpoint{2.359993in}{0.717212in}}{\pgfqpoint{2.368229in}{0.717212in}}%
\pgfpathclose%
\pgfusepath{stroke,fill}%
\end{pgfscope}%
\begin{pgfscope}%
\pgfpathrectangle{\pgfqpoint{0.100000in}{0.220728in}}{\pgfqpoint{3.696000in}{3.696000in}}%
\pgfusepath{clip}%
\pgfsetbuttcap%
\pgfsetroundjoin%
\definecolor{currentfill}{rgb}{0.121569,0.466667,0.705882}%
\pgfsetfillcolor{currentfill}%
\pgfsetfillopacity{0.998696}%
\pgfsetlinewidth{1.003750pt}%
\definecolor{currentstroke}{rgb}{0.121569,0.466667,0.705882}%
\pgfsetstrokecolor{currentstroke}%
\pgfsetstrokeopacity{0.998696}%
\pgfsetdash{}{0pt}%
\pgfpathmoveto{\pgfqpoint{2.368880in}{0.716801in}}%
\pgfpathcurveto{\pgfqpoint{2.377117in}{0.716801in}}{\pgfqpoint{2.385017in}{0.720073in}}{\pgfqpoint{2.390841in}{0.725897in}}%
\pgfpathcurveto{\pgfqpoint{2.396665in}{0.731721in}}{\pgfqpoint{2.399937in}{0.739621in}}{\pgfqpoint{2.399937in}{0.747857in}}%
\pgfpathcurveto{\pgfqpoint{2.399937in}{0.756094in}}{\pgfqpoint{2.396665in}{0.763994in}}{\pgfqpoint{2.390841in}{0.769818in}}%
\pgfpathcurveto{\pgfqpoint{2.385017in}{0.775641in}}{\pgfqpoint{2.377117in}{0.778914in}}{\pgfqpoint{2.368880in}{0.778914in}}%
\pgfpathcurveto{\pgfqpoint{2.360644in}{0.778914in}}{\pgfqpoint{2.352744in}{0.775641in}}{\pgfqpoint{2.346920in}{0.769818in}}%
\pgfpathcurveto{\pgfqpoint{2.341096in}{0.763994in}}{\pgfqpoint{2.337824in}{0.756094in}}{\pgfqpoint{2.337824in}{0.747857in}}%
\pgfpathcurveto{\pgfqpoint{2.337824in}{0.739621in}}{\pgfqpoint{2.341096in}{0.731721in}}{\pgfqpoint{2.346920in}{0.725897in}}%
\pgfpathcurveto{\pgfqpoint{2.352744in}{0.720073in}}{\pgfqpoint{2.360644in}{0.716801in}}{\pgfqpoint{2.368880in}{0.716801in}}%
\pgfpathclose%
\pgfusepath{stroke,fill}%
\end{pgfscope}%
\begin{pgfscope}%
\pgfpathrectangle{\pgfqpoint{0.100000in}{0.220728in}}{\pgfqpoint{3.696000in}{3.696000in}}%
\pgfusepath{clip}%
\pgfsetbuttcap%
\pgfsetroundjoin%
\definecolor{currentfill}{rgb}{0.121569,0.466667,0.705882}%
\pgfsetfillcolor{currentfill}%
\pgfsetfillopacity{0.998873}%
\pgfsetlinewidth{1.003750pt}%
\definecolor{currentstroke}{rgb}{0.121569,0.466667,0.705882}%
\pgfsetstrokecolor{currentstroke}%
\pgfsetstrokeopacity{0.998873}%
\pgfsetdash{}{0pt}%
\pgfpathmoveto{\pgfqpoint{2.387495in}{0.725646in}}%
\pgfpathcurveto{\pgfqpoint{2.395731in}{0.725646in}}{\pgfqpoint{2.403631in}{0.728918in}}{\pgfqpoint{2.409455in}{0.734742in}}%
\pgfpathcurveto{\pgfqpoint{2.415279in}{0.740566in}}{\pgfqpoint{2.418552in}{0.748466in}}{\pgfqpoint{2.418552in}{0.756702in}}%
\pgfpathcurveto{\pgfqpoint{2.418552in}{0.764939in}}{\pgfqpoint{2.415279in}{0.772839in}}{\pgfqpoint{2.409455in}{0.778663in}}%
\pgfpathcurveto{\pgfqpoint{2.403631in}{0.784487in}}{\pgfqpoint{2.395731in}{0.787759in}}{\pgfqpoint{2.387495in}{0.787759in}}%
\pgfpathcurveto{\pgfqpoint{2.379259in}{0.787759in}}{\pgfqpoint{2.371359in}{0.784487in}}{\pgfqpoint{2.365535in}{0.778663in}}%
\pgfpathcurveto{\pgfqpoint{2.359711in}{0.772839in}}{\pgfqpoint{2.356439in}{0.764939in}}{\pgfqpoint{2.356439in}{0.756702in}}%
\pgfpathcurveto{\pgfqpoint{2.356439in}{0.748466in}}{\pgfqpoint{2.359711in}{0.740566in}}{\pgfqpoint{2.365535in}{0.734742in}}%
\pgfpathcurveto{\pgfqpoint{2.371359in}{0.728918in}}{\pgfqpoint{2.379259in}{0.725646in}}{\pgfqpoint{2.387495in}{0.725646in}}%
\pgfpathclose%
\pgfusepath{stroke,fill}%
\end{pgfscope}%
\begin{pgfscope}%
\pgfpathrectangle{\pgfqpoint{0.100000in}{0.220728in}}{\pgfqpoint{3.696000in}{3.696000in}}%
\pgfusepath{clip}%
\pgfsetbuttcap%
\pgfsetroundjoin%
\definecolor{currentfill}{rgb}{0.121569,0.466667,0.705882}%
\pgfsetfillcolor{currentfill}%
\pgfsetfillopacity{0.998875}%
\pgfsetlinewidth{1.003750pt}%
\definecolor{currentstroke}{rgb}{0.121569,0.466667,0.705882}%
\pgfsetstrokecolor{currentstroke}%
\pgfsetstrokeopacity{0.998875}%
\pgfsetdash{}{0pt}%
\pgfpathmoveto{\pgfqpoint{2.370102in}{0.716128in}}%
\pgfpathcurveto{\pgfqpoint{2.378338in}{0.716128in}}{\pgfqpoint{2.386238in}{0.719400in}}{\pgfqpoint{2.392062in}{0.725224in}}%
\pgfpathcurveto{\pgfqpoint{2.397886in}{0.731048in}}{\pgfqpoint{2.401159in}{0.738948in}}{\pgfqpoint{2.401159in}{0.747184in}}%
\pgfpathcurveto{\pgfqpoint{2.401159in}{0.755420in}}{\pgfqpoint{2.397886in}{0.763320in}}{\pgfqpoint{2.392062in}{0.769144in}}%
\pgfpathcurveto{\pgfqpoint{2.386238in}{0.774968in}}{\pgfqpoint{2.378338in}{0.778241in}}{\pgfqpoint{2.370102in}{0.778241in}}%
\pgfpathcurveto{\pgfqpoint{2.361866in}{0.778241in}}{\pgfqpoint{2.353966in}{0.774968in}}{\pgfqpoint{2.348142in}{0.769144in}}%
\pgfpathcurveto{\pgfqpoint{2.342318in}{0.763320in}}{\pgfqpoint{2.339046in}{0.755420in}}{\pgfqpoint{2.339046in}{0.747184in}}%
\pgfpathcurveto{\pgfqpoint{2.339046in}{0.738948in}}{\pgfqpoint{2.342318in}{0.731048in}}{\pgfqpoint{2.348142in}{0.725224in}}%
\pgfpathcurveto{\pgfqpoint{2.353966in}{0.719400in}}{\pgfqpoint{2.361866in}{0.716128in}}{\pgfqpoint{2.370102in}{0.716128in}}%
\pgfpathclose%
\pgfusepath{stroke,fill}%
\end{pgfscope}%
\begin{pgfscope}%
\pgfpathrectangle{\pgfqpoint{0.100000in}{0.220728in}}{\pgfqpoint{3.696000in}{3.696000in}}%
\pgfusepath{clip}%
\pgfsetbuttcap%
\pgfsetroundjoin%
\definecolor{currentfill}{rgb}{0.121569,0.466667,0.705882}%
\pgfsetfillcolor{currentfill}%
\pgfsetfillopacity{0.999123}%
\pgfsetlinewidth{1.003750pt}%
\definecolor{currentstroke}{rgb}{0.121569,0.466667,0.705882}%
\pgfsetstrokecolor{currentstroke}%
\pgfsetstrokeopacity{0.999123}%
\pgfsetdash{}{0pt}%
\pgfpathmoveto{\pgfqpoint{2.372413in}{0.714970in}}%
\pgfpathcurveto{\pgfqpoint{2.380649in}{0.714970in}}{\pgfqpoint{2.388549in}{0.718243in}}{\pgfqpoint{2.394373in}{0.724067in}}%
\pgfpathcurveto{\pgfqpoint{2.400197in}{0.729891in}}{\pgfqpoint{2.403469in}{0.737791in}}{\pgfqpoint{2.403469in}{0.746027in}}%
\pgfpathcurveto{\pgfqpoint{2.403469in}{0.754263in}}{\pgfqpoint{2.400197in}{0.762163in}}{\pgfqpoint{2.394373in}{0.767987in}}%
\pgfpathcurveto{\pgfqpoint{2.388549in}{0.773811in}}{\pgfqpoint{2.380649in}{0.777083in}}{\pgfqpoint{2.372413in}{0.777083in}}%
\pgfpathcurveto{\pgfqpoint{2.364176in}{0.777083in}}{\pgfqpoint{2.356276in}{0.773811in}}{\pgfqpoint{2.350452in}{0.767987in}}%
\pgfpathcurveto{\pgfqpoint{2.344628in}{0.762163in}}{\pgfqpoint{2.341356in}{0.754263in}}{\pgfqpoint{2.341356in}{0.746027in}}%
\pgfpathcurveto{\pgfqpoint{2.341356in}{0.737791in}}{\pgfqpoint{2.344628in}{0.729891in}}{\pgfqpoint{2.350452in}{0.724067in}}%
\pgfpathcurveto{\pgfqpoint{2.356276in}{0.718243in}}{\pgfqpoint{2.364176in}{0.714970in}}{\pgfqpoint{2.372413in}{0.714970in}}%
\pgfpathclose%
\pgfusepath{stroke,fill}%
\end{pgfscope}%
\begin{pgfscope}%
\pgfpathrectangle{\pgfqpoint{0.100000in}{0.220728in}}{\pgfqpoint{3.696000in}{3.696000in}}%
\pgfusepath{clip}%
\pgfsetbuttcap%
\pgfsetroundjoin%
\definecolor{currentfill}{rgb}{0.121569,0.466667,0.705882}%
\pgfsetfillcolor{currentfill}%
\pgfsetfillopacity{0.999348}%
\pgfsetlinewidth{1.003750pt}%
\definecolor{currentstroke}{rgb}{0.121569,0.466667,0.705882}%
\pgfsetstrokecolor{currentstroke}%
\pgfsetstrokeopacity{0.999348}%
\pgfsetdash{}{0pt}%
\pgfpathmoveto{\pgfqpoint{2.386127in}{0.722195in}}%
\pgfpathcurveto{\pgfqpoint{2.394363in}{0.722195in}}{\pgfqpoint{2.402263in}{0.725467in}}{\pgfqpoint{2.408087in}{0.731291in}}%
\pgfpathcurveto{\pgfqpoint{2.413911in}{0.737115in}}{\pgfqpoint{2.417183in}{0.745015in}}{\pgfqpoint{2.417183in}{0.753251in}}%
\pgfpathcurveto{\pgfqpoint{2.417183in}{0.761487in}}{\pgfqpoint{2.413911in}{0.769387in}}{\pgfqpoint{2.408087in}{0.775211in}}%
\pgfpathcurveto{\pgfqpoint{2.402263in}{0.781035in}}{\pgfqpoint{2.394363in}{0.784308in}}{\pgfqpoint{2.386127in}{0.784308in}}%
\pgfpathcurveto{\pgfqpoint{2.377890in}{0.784308in}}{\pgfqpoint{2.369990in}{0.781035in}}{\pgfqpoint{2.364166in}{0.775211in}}%
\pgfpathcurveto{\pgfqpoint{2.358342in}{0.769387in}}{\pgfqpoint{2.355070in}{0.761487in}}{\pgfqpoint{2.355070in}{0.753251in}}%
\pgfpathcurveto{\pgfqpoint{2.355070in}{0.745015in}}{\pgfqpoint{2.358342in}{0.737115in}}{\pgfqpoint{2.364166in}{0.731291in}}%
\pgfpathcurveto{\pgfqpoint{2.369990in}{0.725467in}}{\pgfqpoint{2.377890in}{0.722195in}}{\pgfqpoint{2.386127in}{0.722195in}}%
\pgfpathclose%
\pgfusepath{stroke,fill}%
\end{pgfscope}%
\begin{pgfscope}%
\pgfpathrectangle{\pgfqpoint{0.100000in}{0.220728in}}{\pgfqpoint{3.696000in}{3.696000in}}%
\pgfusepath{clip}%
\pgfsetbuttcap%
\pgfsetroundjoin%
\definecolor{currentfill}{rgb}{0.121569,0.466667,0.705882}%
\pgfsetfillcolor{currentfill}%
\pgfsetfillopacity{0.999380}%
\pgfsetlinewidth{1.003750pt}%
\definecolor{currentstroke}{rgb}{0.121569,0.466667,0.705882}%
\pgfsetstrokecolor{currentstroke}%
\pgfsetstrokeopacity{0.999380}%
\pgfsetdash{}{0pt}%
\pgfpathmoveto{\pgfqpoint{2.374275in}{0.714569in}}%
\pgfpathcurveto{\pgfqpoint{2.382511in}{0.714569in}}{\pgfqpoint{2.390411in}{0.717841in}}{\pgfqpoint{2.396235in}{0.723665in}}%
\pgfpathcurveto{\pgfqpoint{2.402059in}{0.729489in}}{\pgfqpoint{2.405331in}{0.737389in}}{\pgfqpoint{2.405331in}{0.745625in}}%
\pgfpathcurveto{\pgfqpoint{2.405331in}{0.753861in}}{\pgfqpoint{2.402059in}{0.761761in}}{\pgfqpoint{2.396235in}{0.767585in}}%
\pgfpathcurveto{\pgfqpoint{2.390411in}{0.773409in}}{\pgfqpoint{2.382511in}{0.776682in}}{\pgfqpoint{2.374275in}{0.776682in}}%
\pgfpathcurveto{\pgfqpoint{2.366039in}{0.776682in}}{\pgfqpoint{2.358138in}{0.773409in}}{\pgfqpoint{2.352315in}{0.767585in}}%
\pgfpathcurveto{\pgfqpoint{2.346491in}{0.761761in}}{\pgfqpoint{2.343218in}{0.753861in}}{\pgfqpoint{2.343218in}{0.745625in}}%
\pgfpathcurveto{\pgfqpoint{2.343218in}{0.737389in}}{\pgfqpoint{2.346491in}{0.729489in}}{\pgfqpoint{2.352315in}{0.723665in}}%
\pgfpathcurveto{\pgfqpoint{2.358138in}{0.717841in}}{\pgfqpoint{2.366039in}{0.714569in}}{\pgfqpoint{2.374275in}{0.714569in}}%
\pgfpathclose%
\pgfusepath{stroke,fill}%
\end{pgfscope}%
\begin{pgfscope}%
\pgfpathrectangle{\pgfqpoint{0.100000in}{0.220728in}}{\pgfqpoint{3.696000in}{3.696000in}}%
\pgfusepath{clip}%
\pgfsetbuttcap%
\pgfsetroundjoin%
\definecolor{currentfill}{rgb}{0.121569,0.466667,0.705882}%
\pgfsetfillcolor{currentfill}%
\pgfsetfillopacity{0.999609}%
\pgfsetlinewidth{1.003750pt}%
\definecolor{currentstroke}{rgb}{0.121569,0.466667,0.705882}%
\pgfsetstrokecolor{currentstroke}%
\pgfsetstrokeopacity{0.999609}%
\pgfsetdash{}{0pt}%
\pgfpathmoveto{\pgfqpoint{2.385425in}{0.720243in}}%
\pgfpathcurveto{\pgfqpoint{2.393661in}{0.720243in}}{\pgfqpoint{2.401561in}{0.723515in}}{\pgfqpoint{2.407385in}{0.729339in}}%
\pgfpathcurveto{\pgfqpoint{2.413209in}{0.735163in}}{\pgfqpoint{2.416481in}{0.743063in}}{\pgfqpoint{2.416481in}{0.751299in}}%
\pgfpathcurveto{\pgfqpoint{2.416481in}{0.759535in}}{\pgfqpoint{2.413209in}{0.767435in}}{\pgfqpoint{2.407385in}{0.773259in}}%
\pgfpathcurveto{\pgfqpoint{2.401561in}{0.779083in}}{\pgfqpoint{2.393661in}{0.782356in}}{\pgfqpoint{2.385425in}{0.782356in}}%
\pgfpathcurveto{\pgfqpoint{2.377189in}{0.782356in}}{\pgfqpoint{2.369289in}{0.779083in}}{\pgfqpoint{2.363465in}{0.773259in}}%
\pgfpathcurveto{\pgfqpoint{2.357641in}{0.767435in}}{\pgfqpoint{2.354368in}{0.759535in}}{\pgfqpoint{2.354368in}{0.751299in}}%
\pgfpathcurveto{\pgfqpoint{2.354368in}{0.743063in}}{\pgfqpoint{2.357641in}{0.735163in}}{\pgfqpoint{2.363465in}{0.729339in}}%
\pgfpathcurveto{\pgfqpoint{2.369289in}{0.723515in}}{\pgfqpoint{2.377189in}{0.720243in}}{\pgfqpoint{2.385425in}{0.720243in}}%
\pgfpathclose%
\pgfusepath{stroke,fill}%
\end{pgfscope}%
\begin{pgfscope}%
\pgfpathrectangle{\pgfqpoint{0.100000in}{0.220728in}}{\pgfqpoint{3.696000in}{3.696000in}}%
\pgfusepath{clip}%
\pgfsetbuttcap%
\pgfsetroundjoin%
\definecolor{currentfill}{rgb}{0.121569,0.466667,0.705882}%
\pgfsetfillcolor{currentfill}%
\pgfsetfillopacity{0.999755}%
\pgfsetlinewidth{1.003750pt}%
\definecolor{currentstroke}{rgb}{0.121569,0.466667,0.705882}%
\pgfsetstrokecolor{currentstroke}%
\pgfsetstrokeopacity{0.999755}%
\pgfsetdash{}{0pt}%
\pgfpathmoveto{\pgfqpoint{2.385007in}{0.719213in}}%
\pgfpathcurveto{\pgfqpoint{2.393243in}{0.719213in}}{\pgfqpoint{2.401143in}{0.722486in}}{\pgfqpoint{2.406967in}{0.728310in}}%
\pgfpathcurveto{\pgfqpoint{2.412791in}{0.734134in}}{\pgfqpoint{2.416063in}{0.742034in}}{\pgfqpoint{2.416063in}{0.750270in}}%
\pgfpathcurveto{\pgfqpoint{2.416063in}{0.758506in}}{\pgfqpoint{2.412791in}{0.766406in}}{\pgfqpoint{2.406967in}{0.772230in}}%
\pgfpathcurveto{\pgfqpoint{2.401143in}{0.778054in}}{\pgfqpoint{2.393243in}{0.781326in}}{\pgfqpoint{2.385007in}{0.781326in}}%
\pgfpathcurveto{\pgfqpoint{2.376770in}{0.781326in}}{\pgfqpoint{2.368870in}{0.778054in}}{\pgfqpoint{2.363046in}{0.772230in}}%
\pgfpathcurveto{\pgfqpoint{2.357223in}{0.766406in}}{\pgfqpoint{2.353950in}{0.758506in}}{\pgfqpoint{2.353950in}{0.750270in}}%
\pgfpathcurveto{\pgfqpoint{2.353950in}{0.742034in}}{\pgfqpoint{2.357223in}{0.734134in}}{\pgfqpoint{2.363046in}{0.728310in}}%
\pgfpathcurveto{\pgfqpoint{2.368870in}{0.722486in}}{\pgfqpoint{2.376770in}{0.719213in}}{\pgfqpoint{2.385007in}{0.719213in}}%
\pgfpathclose%
\pgfusepath{stroke,fill}%
\end{pgfscope}%
\begin{pgfscope}%
\pgfpathrectangle{\pgfqpoint{0.100000in}{0.220728in}}{\pgfqpoint{3.696000in}{3.696000in}}%
\pgfusepath{clip}%
\pgfsetbuttcap%
\pgfsetroundjoin%
\definecolor{currentfill}{rgb}{0.121569,0.466667,0.705882}%
\pgfsetfillcolor{currentfill}%
\pgfsetfillopacity{0.999836}%
\pgfsetlinewidth{1.003750pt}%
\definecolor{currentstroke}{rgb}{0.121569,0.466667,0.705882}%
\pgfsetstrokecolor{currentstroke}%
\pgfsetstrokeopacity{0.999836}%
\pgfsetdash{}{0pt}%
\pgfpathmoveto{\pgfqpoint{2.384779in}{0.718645in}}%
\pgfpathcurveto{\pgfqpoint{2.393015in}{0.718645in}}{\pgfqpoint{2.400915in}{0.721917in}}{\pgfqpoint{2.406739in}{0.727741in}}%
\pgfpathcurveto{\pgfqpoint{2.412563in}{0.733565in}}{\pgfqpoint{2.415835in}{0.741465in}}{\pgfqpoint{2.415835in}{0.749701in}}%
\pgfpathcurveto{\pgfqpoint{2.415835in}{0.757937in}}{\pgfqpoint{2.412563in}{0.765837in}}{\pgfqpoint{2.406739in}{0.771661in}}%
\pgfpathcurveto{\pgfqpoint{2.400915in}{0.777485in}}{\pgfqpoint{2.393015in}{0.780758in}}{\pgfqpoint{2.384779in}{0.780758in}}%
\pgfpathcurveto{\pgfqpoint{2.376542in}{0.780758in}}{\pgfqpoint{2.368642in}{0.777485in}}{\pgfqpoint{2.362818in}{0.771661in}}%
\pgfpathcurveto{\pgfqpoint{2.356994in}{0.765837in}}{\pgfqpoint{2.353722in}{0.757937in}}{\pgfqpoint{2.353722in}{0.749701in}}%
\pgfpathcurveto{\pgfqpoint{2.353722in}{0.741465in}}{\pgfqpoint{2.356994in}{0.733565in}}{\pgfqpoint{2.362818in}{0.727741in}}%
\pgfpathcurveto{\pgfqpoint{2.368642in}{0.721917in}}{\pgfqpoint{2.376542in}{0.718645in}}{\pgfqpoint{2.384779in}{0.718645in}}%
\pgfpathclose%
\pgfusepath{stroke,fill}%
\end{pgfscope}%
\begin{pgfscope}%
\pgfpathrectangle{\pgfqpoint{0.100000in}{0.220728in}}{\pgfqpoint{3.696000in}{3.696000in}}%
\pgfusepath{clip}%
\pgfsetbuttcap%
\pgfsetroundjoin%
\definecolor{currentfill}{rgb}{0.121569,0.466667,0.705882}%
\pgfsetfillcolor{currentfill}%
\pgfsetfillopacity{0.999880}%
\pgfsetlinewidth{1.003750pt}%
\definecolor{currentstroke}{rgb}{0.121569,0.466667,0.705882}%
\pgfsetstrokecolor{currentstroke}%
\pgfsetstrokeopacity{0.999880}%
\pgfsetdash{}{0pt}%
\pgfpathmoveto{\pgfqpoint{2.384649in}{0.718337in}}%
\pgfpathcurveto{\pgfqpoint{2.392885in}{0.718337in}}{\pgfqpoint{2.400785in}{0.721610in}}{\pgfqpoint{2.406609in}{0.727434in}}%
\pgfpathcurveto{\pgfqpoint{2.412433in}{0.733258in}}{\pgfqpoint{2.415705in}{0.741158in}}{\pgfqpoint{2.415705in}{0.749394in}}%
\pgfpathcurveto{\pgfqpoint{2.415705in}{0.757630in}}{\pgfqpoint{2.412433in}{0.765530in}}{\pgfqpoint{2.406609in}{0.771354in}}%
\pgfpathcurveto{\pgfqpoint{2.400785in}{0.777178in}}{\pgfqpoint{2.392885in}{0.780450in}}{\pgfqpoint{2.384649in}{0.780450in}}%
\pgfpathcurveto{\pgfqpoint{2.376412in}{0.780450in}}{\pgfqpoint{2.368512in}{0.777178in}}{\pgfqpoint{2.362688in}{0.771354in}}%
\pgfpathcurveto{\pgfqpoint{2.356865in}{0.765530in}}{\pgfqpoint{2.353592in}{0.757630in}}{\pgfqpoint{2.353592in}{0.749394in}}%
\pgfpathcurveto{\pgfqpoint{2.353592in}{0.741158in}}{\pgfqpoint{2.356865in}{0.733258in}}{\pgfqpoint{2.362688in}{0.727434in}}%
\pgfpathcurveto{\pgfqpoint{2.368512in}{0.721610in}}{\pgfqpoint{2.376412in}{0.718337in}}{\pgfqpoint{2.384649in}{0.718337in}}%
\pgfpathclose%
\pgfusepath{stroke,fill}%
\end{pgfscope}%
\begin{pgfscope}%
\pgfpathrectangle{\pgfqpoint{0.100000in}{0.220728in}}{\pgfqpoint{3.696000in}{3.696000in}}%
\pgfusepath{clip}%
\pgfsetbuttcap%
\pgfsetroundjoin%
\definecolor{currentfill}{rgb}{0.121569,0.466667,0.705882}%
\pgfsetfillcolor{currentfill}%
\pgfsetfillopacity{0.999893}%
\pgfsetlinewidth{1.003750pt}%
\definecolor{currentstroke}{rgb}{0.121569,0.466667,0.705882}%
\pgfsetstrokecolor{currentstroke}%
\pgfsetstrokeopacity{0.999893}%
\pgfsetdash{}{0pt}%
\pgfpathmoveto{\pgfqpoint{2.377688in}{0.714138in}}%
\pgfpathcurveto{\pgfqpoint{2.385924in}{0.714138in}}{\pgfqpoint{2.393824in}{0.717411in}}{\pgfqpoint{2.399648in}{0.723235in}}%
\pgfpathcurveto{\pgfqpoint{2.405472in}{0.729059in}}{\pgfqpoint{2.408744in}{0.736959in}}{\pgfqpoint{2.408744in}{0.745195in}}%
\pgfpathcurveto{\pgfqpoint{2.408744in}{0.753431in}}{\pgfqpoint{2.405472in}{0.761331in}}{\pgfqpoint{2.399648in}{0.767155in}}%
\pgfpathcurveto{\pgfqpoint{2.393824in}{0.772979in}}{\pgfqpoint{2.385924in}{0.776251in}}{\pgfqpoint{2.377688in}{0.776251in}}%
\pgfpathcurveto{\pgfqpoint{2.369451in}{0.776251in}}{\pgfqpoint{2.361551in}{0.772979in}}{\pgfqpoint{2.355727in}{0.767155in}}%
\pgfpathcurveto{\pgfqpoint{2.349903in}{0.761331in}}{\pgfqpoint{2.346631in}{0.753431in}}{\pgfqpoint{2.346631in}{0.745195in}}%
\pgfpathcurveto{\pgfqpoint{2.346631in}{0.736959in}}{\pgfqpoint{2.349903in}{0.729059in}}{\pgfqpoint{2.355727in}{0.723235in}}%
\pgfpathcurveto{\pgfqpoint{2.361551in}{0.717411in}}{\pgfqpoint{2.369451in}{0.714138in}}{\pgfqpoint{2.377688in}{0.714138in}}%
\pgfpathclose%
\pgfusepath{stroke,fill}%
\end{pgfscope}%
\begin{pgfscope}%
\pgfpathrectangle{\pgfqpoint{0.100000in}{0.220728in}}{\pgfqpoint{3.696000in}{3.696000in}}%
\pgfusepath{clip}%
\pgfsetbuttcap%
\pgfsetroundjoin%
\definecolor{currentfill}{rgb}{0.121569,0.466667,0.705882}%
\pgfsetfillcolor{currentfill}%
\pgfsetfillopacity{0.999906}%
\pgfsetlinewidth{1.003750pt}%
\definecolor{currentstroke}{rgb}{0.121569,0.466667,0.705882}%
\pgfsetstrokecolor{currentstroke}%
\pgfsetstrokeopacity{0.999906}%
\pgfsetdash{}{0pt}%
\pgfpathmoveto{\pgfqpoint{2.384580in}{0.718171in}}%
\pgfpathcurveto{\pgfqpoint{2.392816in}{0.718171in}}{\pgfqpoint{2.400717in}{0.721443in}}{\pgfqpoint{2.406540in}{0.727267in}}%
\pgfpathcurveto{\pgfqpoint{2.412364in}{0.733091in}}{\pgfqpoint{2.415637in}{0.740991in}}{\pgfqpoint{2.415637in}{0.749227in}}%
\pgfpathcurveto{\pgfqpoint{2.415637in}{0.757463in}}{\pgfqpoint{2.412364in}{0.765363in}}{\pgfqpoint{2.406540in}{0.771187in}}%
\pgfpathcurveto{\pgfqpoint{2.400717in}{0.777011in}}{\pgfqpoint{2.392816in}{0.780284in}}{\pgfqpoint{2.384580in}{0.780284in}}%
\pgfpathcurveto{\pgfqpoint{2.376344in}{0.780284in}}{\pgfqpoint{2.368444in}{0.777011in}}{\pgfqpoint{2.362620in}{0.771187in}}%
\pgfpathcurveto{\pgfqpoint{2.356796in}{0.765363in}}{\pgfqpoint{2.353524in}{0.757463in}}{\pgfqpoint{2.353524in}{0.749227in}}%
\pgfpathcurveto{\pgfqpoint{2.353524in}{0.740991in}}{\pgfqpoint{2.356796in}{0.733091in}}{\pgfqpoint{2.362620in}{0.727267in}}%
\pgfpathcurveto{\pgfqpoint{2.368444in}{0.721443in}}{\pgfqpoint{2.376344in}{0.718171in}}{\pgfqpoint{2.384580in}{0.718171in}}%
\pgfpathclose%
\pgfusepath{stroke,fill}%
\end{pgfscope}%
\begin{pgfscope}%
\pgfpathrectangle{\pgfqpoint{0.100000in}{0.220728in}}{\pgfqpoint{3.696000in}{3.696000in}}%
\pgfusepath{clip}%
\pgfsetbuttcap%
\pgfsetroundjoin%
\definecolor{currentfill}{rgb}{0.121569,0.466667,0.705882}%
\pgfsetfillcolor{currentfill}%
\pgfsetfillopacity{0.999921}%
\pgfsetlinewidth{1.003750pt}%
\definecolor{currentstroke}{rgb}{0.121569,0.466667,0.705882}%
\pgfsetstrokecolor{currentstroke}%
\pgfsetstrokeopacity{0.999921}%
\pgfsetdash{}{0pt}%
\pgfpathmoveto{\pgfqpoint{2.379966in}{0.713896in}}%
\pgfpathcurveto{\pgfqpoint{2.388203in}{0.713896in}}{\pgfqpoint{2.396103in}{0.717168in}}{\pgfqpoint{2.401927in}{0.722992in}}%
\pgfpathcurveto{\pgfqpoint{2.407750in}{0.728816in}}{\pgfqpoint{2.411023in}{0.736716in}}{\pgfqpoint{2.411023in}{0.744952in}}%
\pgfpathcurveto{\pgfqpoint{2.411023in}{0.753188in}}{\pgfqpoint{2.407750in}{0.761089in}}{\pgfqpoint{2.401927in}{0.766912in}}%
\pgfpathcurveto{\pgfqpoint{2.396103in}{0.772736in}}{\pgfqpoint{2.388203in}{0.776009in}}{\pgfqpoint{2.379966in}{0.776009in}}%
\pgfpathcurveto{\pgfqpoint{2.371730in}{0.776009in}}{\pgfqpoint{2.363830in}{0.772736in}}{\pgfqpoint{2.358006in}{0.766912in}}%
\pgfpathcurveto{\pgfqpoint{2.352182in}{0.761089in}}{\pgfqpoint{2.348910in}{0.753188in}}{\pgfqpoint{2.348910in}{0.744952in}}%
\pgfpathcurveto{\pgfqpoint{2.348910in}{0.736716in}}{\pgfqpoint{2.352182in}{0.728816in}}{\pgfqpoint{2.358006in}{0.722992in}}%
\pgfpathcurveto{\pgfqpoint{2.363830in}{0.717168in}}{\pgfqpoint{2.371730in}{0.713896in}}{\pgfqpoint{2.379966in}{0.713896in}}%
\pgfpathclose%
\pgfusepath{stroke,fill}%
\end{pgfscope}%
\begin{pgfscope}%
\pgfpathrectangle{\pgfqpoint{0.100000in}{0.220728in}}{\pgfqpoint{3.696000in}{3.696000in}}%
\pgfusepath{clip}%
\pgfsetbuttcap%
\pgfsetroundjoin%
\definecolor{currentfill}{rgb}{0.121569,0.466667,0.705882}%
\pgfsetfillcolor{currentfill}%
\pgfsetfillopacity{0.999921}%
\pgfsetlinewidth{1.003750pt}%
\definecolor{currentstroke}{rgb}{0.121569,0.466667,0.705882}%
\pgfsetstrokecolor{currentstroke}%
\pgfsetstrokeopacity{0.999921}%
\pgfsetdash{}{0pt}%
\pgfpathmoveto{\pgfqpoint{2.384544in}{0.718080in}}%
\pgfpathcurveto{\pgfqpoint{2.392781in}{0.718080in}}{\pgfqpoint{2.400681in}{0.721352in}}{\pgfqpoint{2.406505in}{0.727176in}}%
\pgfpathcurveto{\pgfqpoint{2.412329in}{0.733000in}}{\pgfqpoint{2.415601in}{0.740900in}}{\pgfqpoint{2.415601in}{0.749136in}}%
\pgfpathcurveto{\pgfqpoint{2.415601in}{0.757372in}}{\pgfqpoint{2.412329in}{0.765272in}}{\pgfqpoint{2.406505in}{0.771096in}}%
\pgfpathcurveto{\pgfqpoint{2.400681in}{0.776920in}}{\pgfqpoint{2.392781in}{0.780193in}}{\pgfqpoint{2.384544in}{0.780193in}}%
\pgfpathcurveto{\pgfqpoint{2.376308in}{0.780193in}}{\pgfqpoint{2.368408in}{0.776920in}}{\pgfqpoint{2.362584in}{0.771096in}}%
\pgfpathcurveto{\pgfqpoint{2.356760in}{0.765272in}}{\pgfqpoint{2.353488in}{0.757372in}}{\pgfqpoint{2.353488in}{0.749136in}}%
\pgfpathcurveto{\pgfqpoint{2.353488in}{0.740900in}}{\pgfqpoint{2.356760in}{0.733000in}}{\pgfqpoint{2.362584in}{0.727176in}}%
\pgfpathcurveto{\pgfqpoint{2.368408in}{0.721352in}}{\pgfqpoint{2.376308in}{0.718080in}}{\pgfqpoint{2.384544in}{0.718080in}}%
\pgfpathclose%
\pgfusepath{stroke,fill}%
\end{pgfscope}%
\begin{pgfscope}%
\pgfpathrectangle{\pgfqpoint{0.100000in}{0.220728in}}{\pgfqpoint{3.696000in}{3.696000in}}%
\pgfusepath{clip}%
\pgfsetbuttcap%
\pgfsetroundjoin%
\definecolor{currentfill}{rgb}{0.121569,0.466667,0.705882}%
\pgfsetfillcolor{currentfill}%
\pgfsetfillopacity{0.999927}%
\pgfsetlinewidth{1.003750pt}%
\definecolor{currentstroke}{rgb}{0.121569,0.466667,0.705882}%
\pgfsetstrokecolor{currentstroke}%
\pgfsetstrokeopacity{0.999927}%
\pgfsetdash{}{0pt}%
\pgfpathmoveto{\pgfqpoint{2.384513in}{0.718040in}}%
\pgfpathcurveto{\pgfqpoint{2.392749in}{0.718040in}}{\pgfqpoint{2.400649in}{0.721313in}}{\pgfqpoint{2.406473in}{0.727136in}}%
\pgfpathcurveto{\pgfqpoint{2.412297in}{0.732960in}}{\pgfqpoint{2.415569in}{0.740860in}}{\pgfqpoint{2.415569in}{0.749097in}}%
\pgfpathcurveto{\pgfqpoint{2.415569in}{0.757333in}}{\pgfqpoint{2.412297in}{0.765233in}}{\pgfqpoint{2.406473in}{0.771057in}}%
\pgfpathcurveto{\pgfqpoint{2.400649in}{0.776881in}}{\pgfqpoint{2.392749in}{0.780153in}}{\pgfqpoint{2.384513in}{0.780153in}}%
\pgfpathcurveto{\pgfqpoint{2.376276in}{0.780153in}}{\pgfqpoint{2.368376in}{0.776881in}}{\pgfqpoint{2.362552in}{0.771057in}}%
\pgfpathcurveto{\pgfqpoint{2.356728in}{0.765233in}}{\pgfqpoint{2.353456in}{0.757333in}}{\pgfqpoint{2.353456in}{0.749097in}}%
\pgfpathcurveto{\pgfqpoint{2.353456in}{0.740860in}}{\pgfqpoint{2.356728in}{0.732960in}}{\pgfqpoint{2.362552in}{0.727136in}}%
\pgfpathcurveto{\pgfqpoint{2.368376in}{0.721313in}}{\pgfqpoint{2.376276in}{0.718040in}}{\pgfqpoint{2.384513in}{0.718040in}}%
\pgfpathclose%
\pgfusepath{stroke,fill}%
\end{pgfscope}%
\begin{pgfscope}%
\pgfpathrectangle{\pgfqpoint{0.100000in}{0.220728in}}{\pgfqpoint{3.696000in}{3.696000in}}%
\pgfusepath{clip}%
\pgfsetbuttcap%
\pgfsetroundjoin%
\definecolor{currentfill}{rgb}{0.121569,0.466667,0.705882}%
\pgfsetfillcolor{currentfill}%
\pgfsetfillopacity{0.999929}%
\pgfsetlinewidth{1.003750pt}%
\definecolor{currentstroke}{rgb}{0.121569,0.466667,0.705882}%
\pgfsetstrokecolor{currentstroke}%
\pgfsetstrokeopacity{0.999929}%
\pgfsetdash{}{0pt}%
\pgfpathmoveto{\pgfqpoint{2.384493in}{0.718016in}}%
\pgfpathcurveto{\pgfqpoint{2.392730in}{0.718016in}}{\pgfqpoint{2.400630in}{0.721288in}}{\pgfqpoint{2.406454in}{0.727112in}}%
\pgfpathcurveto{\pgfqpoint{2.412278in}{0.732936in}}{\pgfqpoint{2.415550in}{0.740836in}}{\pgfqpoint{2.415550in}{0.749072in}}%
\pgfpathcurveto{\pgfqpoint{2.415550in}{0.757308in}}{\pgfqpoint{2.412278in}{0.765208in}}{\pgfqpoint{2.406454in}{0.771032in}}%
\pgfpathcurveto{\pgfqpoint{2.400630in}{0.776856in}}{\pgfqpoint{2.392730in}{0.780129in}}{\pgfqpoint{2.384493in}{0.780129in}}%
\pgfpathcurveto{\pgfqpoint{2.376257in}{0.780129in}}{\pgfqpoint{2.368357in}{0.776856in}}{\pgfqpoint{2.362533in}{0.771032in}}%
\pgfpathcurveto{\pgfqpoint{2.356709in}{0.765208in}}{\pgfqpoint{2.353437in}{0.757308in}}{\pgfqpoint{2.353437in}{0.749072in}}%
\pgfpathcurveto{\pgfqpoint{2.353437in}{0.740836in}}{\pgfqpoint{2.356709in}{0.732936in}}{\pgfqpoint{2.362533in}{0.727112in}}%
\pgfpathcurveto{\pgfqpoint{2.368357in}{0.721288in}}{\pgfqpoint{2.376257in}{0.718016in}}{\pgfqpoint{2.384493in}{0.718016in}}%
\pgfpathclose%
\pgfusepath{stroke,fill}%
\end{pgfscope}%
\begin{pgfscope}%
\pgfpathrectangle{\pgfqpoint{0.100000in}{0.220728in}}{\pgfqpoint{3.696000in}{3.696000in}}%
\pgfusepath{clip}%
\pgfsetbuttcap%
\pgfsetroundjoin%
\definecolor{currentfill}{rgb}{0.121569,0.466667,0.705882}%
\pgfsetfillcolor{currentfill}%
\pgfsetlinewidth{1.003750pt}%
\definecolor{currentstroke}{rgb}{0.121569,0.466667,0.705882}%
\pgfsetstrokecolor{currentstroke}%
\pgfsetdash{}{0pt}%
\pgfpathmoveto{\pgfqpoint{2.381668in}{0.715037in}}%
\pgfpathcurveto{\pgfqpoint{2.389904in}{0.715037in}}{\pgfqpoint{2.397804in}{0.718310in}}{\pgfqpoint{2.403628in}{0.724134in}}%
\pgfpathcurveto{\pgfqpoint{2.409452in}{0.729957in}}{\pgfqpoint{2.412725in}{0.737858in}}{\pgfqpoint{2.412725in}{0.746094in}}%
\pgfpathcurveto{\pgfqpoint{2.412725in}{0.754330in}}{\pgfqpoint{2.409452in}{0.762230in}}{\pgfqpoint{2.403628in}{0.768054in}}%
\pgfpathcurveto{\pgfqpoint{2.397804in}{0.773878in}}{\pgfqpoint{2.389904in}{0.777150in}}{\pgfqpoint{2.381668in}{0.777150in}}%
\pgfpathcurveto{\pgfqpoint{2.373432in}{0.777150in}}{\pgfqpoint{2.365532in}{0.773878in}}{\pgfqpoint{2.359708in}{0.768054in}}%
\pgfpathcurveto{\pgfqpoint{2.353884in}{0.762230in}}{\pgfqpoint{2.350612in}{0.754330in}}{\pgfqpoint{2.350612in}{0.746094in}}%
\pgfpathcurveto{\pgfqpoint{2.350612in}{0.737858in}}{\pgfqpoint{2.353884in}{0.729957in}}{\pgfqpoint{2.359708in}{0.724134in}}%
\pgfpathcurveto{\pgfqpoint{2.365532in}{0.718310in}}{\pgfqpoint{2.373432in}{0.715037in}}{\pgfqpoint{2.381668in}{0.715037in}}%
\pgfpathclose%
\pgfusepath{stroke,fill}%
\end{pgfscope}%
\begin{pgfscope}%
\pgfsetbuttcap%
\pgfsetmiterjoin%
\definecolor{currentfill}{rgb}{1.000000,1.000000,1.000000}%
\pgfsetfillcolor{currentfill}%
\pgfsetfillopacity{0.800000}%
\pgfsetlinewidth{1.003750pt}%
\definecolor{currentstroke}{rgb}{0.800000,0.800000,0.800000}%
\pgfsetstrokecolor{currentstroke}%
\pgfsetstrokeopacity{0.800000}%
\pgfsetdash{}{0pt}%
\pgfpathmoveto{\pgfqpoint{0.197222in}{0.290172in}}%
\pgfpathlineto{\pgfqpoint{1.937579in}{0.290172in}}%
\pgfpathquadraticcurveto{\pgfqpoint{1.965356in}{0.290172in}}{\pgfqpoint{1.965356in}{0.317950in}}%
\pgfpathlineto{\pgfqpoint{1.965356in}{0.915633in}}%
\pgfpathquadraticcurveto{\pgfqpoint{1.965356in}{0.943411in}}{\pgfqpoint{1.937579in}{0.943411in}}%
\pgfpathlineto{\pgfqpoint{0.197222in}{0.943411in}}%
\pgfpathquadraticcurveto{\pgfqpoint{0.169444in}{0.943411in}}{\pgfqpoint{0.169444in}{0.915633in}}%
\pgfpathlineto{\pgfqpoint{0.169444in}{0.317950in}}%
\pgfpathquadraticcurveto{\pgfqpoint{0.169444in}{0.290172in}}{\pgfqpoint{0.197222in}{0.290172in}}%
\pgfpathclose%
\pgfusepath{stroke,fill}%
\end{pgfscope}%
\begin{pgfscope}%
\pgfsetrectcap%
\pgfsetroundjoin%
\pgfsetlinewidth{1.505625pt}%
\definecolor{currentstroke}{rgb}{0.121569,0.466667,0.705882}%
\pgfsetstrokecolor{currentstroke}%
\pgfsetdash{}{0pt}%
\pgfpathmoveto{\pgfqpoint{0.225000in}{0.830943in}}%
\pgfpathlineto{\pgfqpoint{0.502778in}{0.830943in}}%
\pgfusepath{stroke}%
\end{pgfscope}%
\begin{pgfscope}%
\definecolor{textcolor}{rgb}{0.000000,0.000000,0.000000}%
\pgfsetstrokecolor{textcolor}%
\pgfsetfillcolor{textcolor}%
\pgftext[x=0.613889in,y=0.782332in,left,base]{\color{textcolor}\sffamily\fontsize{10.000000}{12.000000}\selectfont Ground truth}%
\end{pgfscope}%
\begin{pgfscope}%
\pgfsetbuttcap%
\pgfsetroundjoin%
\definecolor{currentfill}{rgb}{0.121569,0.466667,0.705882}%
\pgfsetfillcolor{currentfill}%
\pgfsetlinewidth{1.003750pt}%
\definecolor{currentstroke}{rgb}{0.121569,0.466667,0.705882}%
\pgfsetstrokecolor{currentstroke}%
\pgfsetdash{}{0pt}%
\pgfsys@defobject{currentmarker}{\pgfqpoint{-0.031056in}{-0.031056in}}{\pgfqpoint{0.031056in}{0.031056in}}{%
\pgfpathmoveto{\pgfqpoint{0.000000in}{-0.031056in}}%
\pgfpathcurveto{\pgfqpoint{0.008236in}{-0.031056in}}{\pgfqpoint{0.016136in}{-0.027784in}}{\pgfqpoint{0.021960in}{-0.021960in}}%
\pgfpathcurveto{\pgfqpoint{0.027784in}{-0.016136in}}{\pgfqpoint{0.031056in}{-0.008236in}}{\pgfqpoint{0.031056in}{0.000000in}}%
\pgfpathcurveto{\pgfqpoint{0.031056in}{0.008236in}}{\pgfqpoint{0.027784in}{0.016136in}}{\pgfqpoint{0.021960in}{0.021960in}}%
\pgfpathcurveto{\pgfqpoint{0.016136in}{0.027784in}}{\pgfqpoint{0.008236in}{0.031056in}}{\pgfqpoint{0.000000in}{0.031056in}}%
\pgfpathcurveto{\pgfqpoint{-0.008236in}{0.031056in}}{\pgfqpoint{-0.016136in}{0.027784in}}{\pgfqpoint{-0.021960in}{0.021960in}}%
\pgfpathcurveto{\pgfqpoint{-0.027784in}{0.016136in}}{\pgfqpoint{-0.031056in}{0.008236in}}{\pgfqpoint{-0.031056in}{0.000000in}}%
\pgfpathcurveto{\pgfqpoint{-0.031056in}{-0.008236in}}{\pgfqpoint{-0.027784in}{-0.016136in}}{\pgfqpoint{-0.021960in}{-0.021960in}}%
\pgfpathcurveto{\pgfqpoint{-0.016136in}{-0.027784in}}{\pgfqpoint{-0.008236in}{-0.031056in}}{\pgfqpoint{0.000000in}{-0.031056in}}%
\pgfpathclose%
\pgfusepath{stroke,fill}%
}%
\begin{pgfscope}%
\pgfsys@transformshift{0.363889in}{0.614933in}%
\pgfsys@useobject{currentmarker}{}%
\end{pgfscope}%
\end{pgfscope}%
\begin{pgfscope}%
\definecolor{textcolor}{rgb}{0.000000,0.000000,0.000000}%
\pgfsetstrokecolor{textcolor}%
\pgfsetfillcolor{textcolor}%
\pgftext[x=0.613889in,y=0.578475in,left,base]{\color{textcolor}\sffamily\fontsize{10.000000}{12.000000}\selectfont Estimated position}%
\end{pgfscope}%
\begin{pgfscope}%
\pgfsetbuttcap%
\pgfsetroundjoin%
\definecolor{currentfill}{rgb}{1.000000,0.498039,0.054902}%
\pgfsetfillcolor{currentfill}%
\pgfsetlinewidth{1.003750pt}%
\definecolor{currentstroke}{rgb}{1.000000,0.498039,0.054902}%
\pgfsetstrokecolor{currentstroke}%
\pgfsetdash{}{0pt}%
\pgfsys@defobject{currentmarker}{\pgfqpoint{-0.031056in}{-0.031056in}}{\pgfqpoint{0.031056in}{0.031056in}}{%
\pgfpathmoveto{\pgfqpoint{0.000000in}{-0.031056in}}%
\pgfpathcurveto{\pgfqpoint{0.008236in}{-0.031056in}}{\pgfqpoint{0.016136in}{-0.027784in}}{\pgfqpoint{0.021960in}{-0.021960in}}%
\pgfpathcurveto{\pgfqpoint{0.027784in}{-0.016136in}}{\pgfqpoint{0.031056in}{-0.008236in}}{\pgfqpoint{0.031056in}{0.000000in}}%
\pgfpathcurveto{\pgfqpoint{0.031056in}{0.008236in}}{\pgfqpoint{0.027784in}{0.016136in}}{\pgfqpoint{0.021960in}{0.021960in}}%
\pgfpathcurveto{\pgfqpoint{0.016136in}{0.027784in}}{\pgfqpoint{0.008236in}{0.031056in}}{\pgfqpoint{0.000000in}{0.031056in}}%
\pgfpathcurveto{\pgfqpoint{-0.008236in}{0.031056in}}{\pgfqpoint{-0.016136in}{0.027784in}}{\pgfqpoint{-0.021960in}{0.021960in}}%
\pgfpathcurveto{\pgfqpoint{-0.027784in}{0.016136in}}{\pgfqpoint{-0.031056in}{0.008236in}}{\pgfqpoint{-0.031056in}{0.000000in}}%
\pgfpathcurveto{\pgfqpoint{-0.031056in}{-0.008236in}}{\pgfqpoint{-0.027784in}{-0.016136in}}{\pgfqpoint{-0.021960in}{-0.021960in}}%
\pgfpathcurveto{\pgfqpoint{-0.016136in}{-0.027784in}}{\pgfqpoint{-0.008236in}{-0.031056in}}{\pgfqpoint{0.000000in}{-0.031056in}}%
\pgfpathclose%
\pgfusepath{stroke,fill}%
}%
\begin{pgfscope}%
\pgfsys@transformshift{0.363889in}{0.411076in}%
\pgfsys@useobject{currentmarker}{}%
\end{pgfscope}%
\end{pgfscope}%
\begin{pgfscope}%
\definecolor{textcolor}{rgb}{0.000000,0.000000,0.000000}%
\pgfsetstrokecolor{textcolor}%
\pgfsetfillcolor{textcolor}%
\pgftext[x=0.613889in,y=0.374618in,left,base]{\color{textcolor}\sffamily\fontsize{10.000000}{12.000000}\selectfont Estimated turn}%
\end{pgfscope}%
\end{pgfpicture}%
\makeatother%
\endgroup%
}
%         \caption{FLAE's 3D position estimation had the lowest turn error for the 16-meter  side square experiment.}
%         \label{fig:square163D}
%     \end{subfigure}
%     \caption{Position estimation by the best performing algorithms in the 16-meter side square experiment.}
%     \label{fig:square16}
% \end{figure}

% \subsubsection{28 meter}

% For the 28-meter square experiment, the Mahony algorithm which had the lowest displacement error with an average of 2.97 meters (2.65\% of error margin), and ROLEQ with an average of 3.20 meters of turn error (2.86\% of error margin).

% \begin{figure}[!h]
%     \centering
%     \begin{table}[H]
    \begin{center}
        \resizebox{1\linewidth}{!}{

            \begin{tabular}[t]{lcccc}
                \hline
                Algorithm     & Displacement Error[$m$] & Displacement Error[\%] & Turn Error[$m$] & Turn Error[\%] \\
                \hline
                AngularRate   & 34.07                   & 30.42                  & 39.11           & 34.92          \\
                AQUA          & 11.82                   & 10.56                  & 14.64           & 13.07          \\
                Complementary & 12.75                   & 11.38                  & 14.68           & 13.11          \\
                Davenport     & 2.32                    & 2.07                   & 6.44            & 5.75           \\
                EKF           & 3.81                    & 3.41                   & 6.39            & 5.71           \\
                FAMC          & 31.54                   & 28.16                  & 41.11           & 36.71          \\
                FLAE          & 2.28                    & 2.03                   & 6.59            & 5.88           \\
                Fourati       & 54.07                   & 48.28                  & 56.38           & 50.34          \\
                Madgwick      & 3.35                    & 2.99                   & 6.41            & 5.73           \\
                Mahony        & 2.63                    & 2.35                   & 6.42            & 5.73           \\
                OLEQ          & 2.67                    & 2.38                   & 7.16            & 6.40           \\
                QUEST         & 22.92                   & 20.46                  & 33.83           & 30.21          \\
                ROLEQ         & 2.85                    & 2.54                   & 7.54            & 6.73           \\
                SAAM          & 2.65                    & 2.36                   & 6.36            & 5.68           \\
                Tilt          & 2.65                    & 2.36                   & 6.36            & 5.68           \\
                \hline
                Average       & 12.82                   & 11.45                  & 17.30           & 15.44
            \end{tabular}
        }
        \caption{Accelerometer Specifications. }
        \label{tab:accelerometer_specification}
    \end{center}
\end{table}
% \end{figure}

% \begin{figure}[!h]
%     \centering
%     \begin{subfigure}{0.49\textwidth}
%         \centering
%         \resizebox{1\linewidth}{!}{%% Creator: Matplotlib, PGF backend
%%
%% To include the figure in your LaTeX document, write
%%   \input{<filename>.pgf}
%%
%% Make sure the required packages are loaded in your preamble
%%   \usepackage{pgf}
%%
%% and, on pdftex
%%   \usepackage[utf8]{inputenc}\DeclareUnicodeCharacter{2212}{-}
%%
%% or, on luatex and xetex
%%   \usepackage{unicode-math}
%%
%% Figures using additional raster images can only be included by \input if
%% they are in the same directory as the main LaTeX file. For loading figures
%% from other directories you can use the `import` package
%%   \usepackage{import}
%%
%% and then include the figures with
%%   \import{<path to file>}{<filename>.pgf}
%%
%% Matplotlib used the following preamble
%%   \usepackage{fontspec}
%%
\begingroup%
\makeatletter%
\begin{pgfpicture}%
\pgfpathrectangle{\pgfpointorigin}{\pgfqpoint{5.590556in}{4.357861in}}%
\pgfusepath{use as bounding box, clip}%
\begin{pgfscope}%
\pgfsetbuttcap%
\pgfsetmiterjoin%
\definecolor{currentfill}{rgb}{1.000000,1.000000,1.000000}%
\pgfsetfillcolor{currentfill}%
\pgfsetlinewidth{0.000000pt}%
\definecolor{currentstroke}{rgb}{1.000000,1.000000,1.000000}%
\pgfsetstrokecolor{currentstroke}%
\pgfsetdash{}{0pt}%
\pgfpathmoveto{\pgfqpoint{0.000000in}{0.000000in}}%
\pgfpathlineto{\pgfqpoint{5.590556in}{0.000000in}}%
\pgfpathlineto{\pgfqpoint{5.590556in}{4.357861in}}%
\pgfpathlineto{\pgfqpoint{0.000000in}{4.357861in}}%
\pgfpathclose%
\pgfusepath{fill}%
\end{pgfscope}%
\begin{pgfscope}%
\pgfsetbuttcap%
\pgfsetmiterjoin%
\definecolor{currentfill}{rgb}{1.000000,1.000000,1.000000}%
\pgfsetfillcolor{currentfill}%
\pgfsetlinewidth{0.000000pt}%
\definecolor{currentstroke}{rgb}{0.000000,0.000000,0.000000}%
\pgfsetstrokecolor{currentstroke}%
\pgfsetstrokeopacity{0.000000}%
\pgfsetdash{}{0pt}%
\pgfpathmoveto{\pgfqpoint{0.530556in}{0.515000in}}%
\pgfpathlineto{\pgfqpoint{5.490556in}{0.515000in}}%
\pgfpathlineto{\pgfqpoint{5.490556in}{4.211000in}}%
\pgfpathlineto{\pgfqpoint{0.530556in}{4.211000in}}%
\pgfpathclose%
\pgfusepath{fill}%
\end{pgfscope}%
\begin{pgfscope}%
\pgfpathrectangle{\pgfqpoint{0.530556in}{0.515000in}}{\pgfqpoint{4.960000in}{3.696000in}}%
\pgfusepath{clip}%
\pgfsetbuttcap%
\pgfsetroundjoin%
\definecolor{currentfill}{rgb}{0.121569,0.466667,0.705882}%
\pgfsetfillcolor{currentfill}%
\pgfsetlinewidth{1.003750pt}%
\definecolor{currentstroke}{rgb}{0.121569,0.466667,0.705882}%
\pgfsetstrokecolor{currentstroke}%
\pgfsetdash{}{0pt}%
\pgfsys@defobject{currentmarker}{\pgfqpoint{-0.041667in}{-0.041667in}}{\pgfqpoint{0.041667in}{0.041667in}}{%
\pgfpathmoveto{\pgfqpoint{0.000000in}{-0.041667in}}%
\pgfpathcurveto{\pgfqpoint{0.011050in}{-0.041667in}}{\pgfqpoint{0.021649in}{-0.037276in}}{\pgfqpoint{0.029463in}{-0.029463in}}%
\pgfpathcurveto{\pgfqpoint{0.037276in}{-0.021649in}}{\pgfqpoint{0.041667in}{-0.011050in}}{\pgfqpoint{0.041667in}{0.000000in}}%
\pgfpathcurveto{\pgfqpoint{0.041667in}{0.011050in}}{\pgfqpoint{0.037276in}{0.021649in}}{\pgfqpoint{0.029463in}{0.029463in}}%
\pgfpathcurveto{\pgfqpoint{0.021649in}{0.037276in}}{\pgfqpoint{0.011050in}{0.041667in}}{\pgfqpoint{0.000000in}{0.041667in}}%
\pgfpathcurveto{\pgfqpoint{-0.011050in}{0.041667in}}{\pgfqpoint{-0.021649in}{0.037276in}}{\pgfqpoint{-0.029463in}{0.029463in}}%
\pgfpathcurveto{\pgfqpoint{-0.037276in}{0.021649in}}{\pgfqpoint{-0.041667in}{0.011050in}}{\pgfqpoint{-0.041667in}{0.000000in}}%
\pgfpathcurveto{\pgfqpoint{-0.041667in}{-0.011050in}}{\pgfqpoint{-0.037276in}{-0.021649in}}{\pgfqpoint{-0.029463in}{-0.029463in}}%
\pgfpathcurveto{\pgfqpoint{-0.021649in}{-0.037276in}}{\pgfqpoint{-0.011050in}{-0.041667in}}{\pgfqpoint{0.000000in}{-0.041667in}}%
\pgfpathclose%
\pgfusepath{stroke,fill}%
}%
\begin{pgfscope}%
\pgfsys@transformshift{1.378257in}{0.729800in}%
\pgfsys@useobject{currentmarker}{}%
\end{pgfscope}%
\begin{pgfscope}%
\pgfsys@transformshift{1.378257in}{0.729800in}%
\pgfsys@useobject{currentmarker}{}%
\end{pgfscope}%
\begin{pgfscope}%
\pgfsys@transformshift{1.378257in}{0.729800in}%
\pgfsys@useobject{currentmarker}{}%
\end{pgfscope}%
\begin{pgfscope}%
\pgfsys@transformshift{1.378257in}{0.729800in}%
\pgfsys@useobject{currentmarker}{}%
\end{pgfscope}%
\begin{pgfscope}%
\pgfsys@transformshift{1.379932in}{0.729362in}%
\pgfsys@useobject{currentmarker}{}%
\end{pgfscope}%
\begin{pgfscope}%
\pgfsys@transformshift{1.382351in}{0.729736in}%
\pgfsys@useobject{currentmarker}{}%
\end{pgfscope}%
\begin{pgfscope}%
\pgfsys@transformshift{1.385336in}{0.731964in}%
\pgfsys@useobject{currentmarker}{}%
\end{pgfscope}%
\begin{pgfscope}%
\pgfsys@transformshift{1.386216in}{0.733814in}%
\pgfsys@useobject{currentmarker}{}%
\end{pgfscope}%
\begin{pgfscope}%
\pgfsys@transformshift{1.387049in}{0.737018in}%
\pgfsys@useobject{currentmarker}{}%
\end{pgfscope}%
\begin{pgfscope}%
\pgfsys@transformshift{1.387518in}{0.741871in}%
\pgfsys@useobject{currentmarker}{}%
\end{pgfscope}%
\begin{pgfscope}%
\pgfsys@transformshift{1.388148in}{0.748535in}%
\pgfsys@useobject{currentmarker}{}%
\end{pgfscope}%
\begin{pgfscope}%
\pgfsys@transformshift{1.387921in}{0.756558in}%
\pgfsys@useobject{currentmarker}{}%
\end{pgfscope}%
\begin{pgfscope}%
\pgfsys@transformshift{1.389008in}{0.765936in}%
\pgfsys@useobject{currentmarker}{}%
\end{pgfscope}%
\begin{pgfscope}%
\pgfsys@transformshift{1.388739in}{0.776749in}%
\pgfsys@useobject{currentmarker}{}%
\end{pgfscope}%
\begin{pgfscope}%
\pgfsys@transformshift{1.389920in}{0.782580in}%
\pgfsys@useobject{currentmarker}{}%
\end{pgfscope}%
\begin{pgfscope}%
\pgfsys@transformshift{1.389837in}{0.785851in}%
\pgfsys@useobject{currentmarker}{}%
\end{pgfscope}%
\begin{pgfscope}%
\pgfsys@transformshift{1.389974in}{0.787645in}%
\pgfsys@useobject{currentmarker}{}%
\end{pgfscope}%
\begin{pgfscope}%
\pgfsys@transformshift{1.389715in}{0.790227in}%
\pgfsys@useobject{currentmarker}{}%
\end{pgfscope}%
\begin{pgfscope}%
\pgfsys@transformshift{1.389964in}{0.793376in}%
\pgfsys@useobject{currentmarker}{}%
\end{pgfscope}%
\begin{pgfscope}%
\pgfsys@transformshift{1.389805in}{0.795106in}%
\pgfsys@useobject{currentmarker}{}%
\end{pgfscope}%
\begin{pgfscope}%
\pgfsys@transformshift{1.389833in}{0.796061in}%
\pgfsys@useobject{currentmarker}{}%
\end{pgfscope}%
\begin{pgfscope}%
\pgfsys@transformshift{1.389799in}{0.796585in}%
\pgfsys@useobject{currentmarker}{}%
\end{pgfscope}%
\begin{pgfscope}%
\pgfsys@transformshift{1.389938in}{0.797767in}%
\pgfsys@useobject{currentmarker}{}%
\end{pgfscope}%
\begin{pgfscope}%
\pgfsys@transformshift{1.389908in}{0.798421in}%
\pgfsys@useobject{currentmarker}{}%
\end{pgfscope}%
\begin{pgfscope}%
\pgfsys@transformshift{1.389987in}{0.798772in}%
\pgfsys@useobject{currentmarker}{}%
\end{pgfscope}%
\begin{pgfscope}%
\pgfsys@transformshift{1.389984in}{0.798970in}%
\pgfsys@useobject{currentmarker}{}%
\end{pgfscope}%
\begin{pgfscope}%
\pgfsys@transformshift{1.390006in}{0.799077in}%
\pgfsys@useobject{currentmarker}{}%
\end{pgfscope}%
\begin{pgfscope}%
\pgfsys@transformshift{1.390002in}{0.799136in}%
\pgfsys@useobject{currentmarker}{}%
\end{pgfscope}%
\begin{pgfscope}%
\pgfsys@transformshift{1.390010in}{0.799168in}%
\pgfsys@useobject{currentmarker}{}%
\end{pgfscope}%
\begin{pgfscope}%
\pgfsys@transformshift{1.390010in}{0.799187in}%
\pgfsys@useobject{currentmarker}{}%
\end{pgfscope}%
\begin{pgfscope}%
\pgfsys@transformshift{1.390011in}{0.799196in}%
\pgfsys@useobject{currentmarker}{}%
\end{pgfscope}%
\begin{pgfscope}%
\pgfsys@transformshift{1.390011in}{0.799202in}%
\pgfsys@useobject{currentmarker}{}%
\end{pgfscope}%
\begin{pgfscope}%
\pgfsys@transformshift{1.390012in}{0.799205in}%
\pgfsys@useobject{currentmarker}{}%
\end{pgfscope}%
\begin{pgfscope}%
\pgfsys@transformshift{1.390011in}{0.799206in}%
\pgfsys@useobject{currentmarker}{}%
\end{pgfscope}%
\begin{pgfscope}%
\pgfsys@transformshift{1.390012in}{0.799207in}%
\pgfsys@useobject{currentmarker}{}%
\end{pgfscope}%
\begin{pgfscope}%
\pgfsys@transformshift{1.390012in}{0.799208in}%
\pgfsys@useobject{currentmarker}{}%
\end{pgfscope}%
\begin{pgfscope}%
\pgfsys@transformshift{1.389901in}{0.799912in}%
\pgfsys@useobject{currentmarker}{}%
\end{pgfscope}%
\begin{pgfscope}%
\pgfsys@transformshift{1.391790in}{0.804541in}%
\pgfsys@useobject{currentmarker}{}%
\end{pgfscope}%
\begin{pgfscope}%
\pgfsys@transformshift{1.389669in}{0.811598in}%
\pgfsys@useobject{currentmarker}{}%
\end{pgfscope}%
\begin{pgfscope}%
\pgfsys@transformshift{1.392726in}{0.821880in}%
\pgfsys@useobject{currentmarker}{}%
\end{pgfscope}%
\begin{pgfscope}%
\pgfsys@transformshift{1.389860in}{0.835368in}%
\pgfsys@useobject{currentmarker}{}%
\end{pgfscope}%
\begin{pgfscope}%
\pgfsys@transformshift{1.393993in}{0.851297in}%
\pgfsys@useobject{currentmarker}{}%
\end{pgfscope}%
\begin{pgfscope}%
\pgfsys@transformshift{1.390308in}{0.868858in}%
\pgfsys@useobject{currentmarker}{}%
\end{pgfscope}%
\begin{pgfscope}%
\pgfsys@transformshift{1.392222in}{0.887306in}%
\pgfsys@useobject{currentmarker}{}%
\end{pgfscope}%
\begin{pgfscope}%
\pgfsys@transformshift{1.395568in}{0.906090in}%
\pgfsys@useobject{currentmarker}{}%
\end{pgfscope}%
\begin{pgfscope}%
\pgfsys@transformshift{1.390119in}{0.924965in}%
\pgfsys@useobject{currentmarker}{}%
\end{pgfscope}%
\begin{pgfscope}%
\pgfsys@transformshift{1.393119in}{0.935346in}%
\pgfsys@useobject{currentmarker}{}%
\end{pgfscope}%
\begin{pgfscope}%
\pgfsys@transformshift{1.390154in}{0.947926in}%
\pgfsys@useobject{currentmarker}{}%
\end{pgfscope}%
\begin{pgfscope}%
\pgfsys@transformshift{1.394476in}{0.961571in}%
\pgfsys@useobject{currentmarker}{}%
\end{pgfscope}%
\begin{pgfscope}%
\pgfsys@transformshift{1.392625in}{0.969223in}%
\pgfsys@useobject{currentmarker}{}%
\end{pgfscope}%
\begin{pgfscope}%
\pgfsys@transformshift{1.393643in}{0.973432in}%
\pgfsys@useobject{currentmarker}{}%
\end{pgfscope}%
\begin{pgfscope}%
\pgfsys@transformshift{1.392260in}{0.980425in}%
\pgfsys@useobject{currentmarker}{}%
\end{pgfscope}%
\begin{pgfscope}%
\pgfsys@transformshift{1.393828in}{0.988462in}%
\pgfsys@useobject{currentmarker}{}%
\end{pgfscope}%
\begin{pgfscope}%
\pgfsys@transformshift{1.393339in}{0.992939in}%
\pgfsys@useobject{currentmarker}{}%
\end{pgfscope}%
\begin{pgfscope}%
\pgfsys@transformshift{1.390866in}{0.998074in}%
\pgfsys@useobject{currentmarker}{}%
\end{pgfscope}%
\begin{pgfscope}%
\pgfsys@transformshift{1.392343in}{1.006299in}%
\pgfsys@useobject{currentmarker}{}%
\end{pgfscope}%
\begin{pgfscope}%
\pgfsys@transformshift{1.389923in}{1.017168in}%
\pgfsys@useobject{currentmarker}{}%
\end{pgfscope}%
\begin{pgfscope}%
\pgfsys@transformshift{1.392515in}{1.029361in}%
\pgfsys@useobject{currentmarker}{}%
\end{pgfscope}%
\begin{pgfscope}%
\pgfsys@transformshift{1.391198in}{1.036089in}%
\pgfsys@useobject{currentmarker}{}%
\end{pgfscope}%
\begin{pgfscope}%
\pgfsys@transformshift{1.387529in}{1.042582in}%
\pgfsys@useobject{currentmarker}{}%
\end{pgfscope}%
\begin{pgfscope}%
\pgfsys@transformshift{1.390496in}{1.052642in}%
\pgfsys@useobject{currentmarker}{}%
\end{pgfscope}%
\begin{pgfscope}%
\pgfsys@transformshift{1.388266in}{1.064014in}%
\pgfsys@useobject{currentmarker}{}%
\end{pgfscope}%
\begin{pgfscope}%
\pgfsys@transformshift{1.391499in}{1.077107in}%
\pgfsys@useobject{currentmarker}{}%
\end{pgfscope}%
\begin{pgfscope}%
\pgfsys@transformshift{1.391249in}{1.084521in}%
\pgfsys@useobject{currentmarker}{}%
\end{pgfscope}%
\begin{pgfscope}%
\pgfsys@transformshift{1.391857in}{1.092550in}%
\pgfsys@useobject{currentmarker}{}%
\end{pgfscope}%
\begin{pgfscope}%
\pgfsys@transformshift{1.393647in}{1.096601in}%
\pgfsys@useobject{currentmarker}{}%
\end{pgfscope}%
\begin{pgfscope}%
\pgfsys@transformshift{1.392994in}{1.098947in}%
\pgfsys@useobject{currentmarker}{}%
\end{pgfscope}%
\begin{pgfscope}%
\pgfsys@transformshift{1.394021in}{1.102570in}%
\pgfsys@useobject{currentmarker}{}%
\end{pgfscope}%
\begin{pgfscope}%
\pgfsys@transformshift{1.392748in}{1.109787in}%
\pgfsys@useobject{currentmarker}{}%
\end{pgfscope}%
\begin{pgfscope}%
\pgfsys@transformshift{1.394501in}{1.117953in}%
\pgfsys@useobject{currentmarker}{}%
\end{pgfscope}%
\begin{pgfscope}%
\pgfsys@transformshift{1.394231in}{1.122539in}%
\pgfsys@useobject{currentmarker}{}%
\end{pgfscope}%
\begin{pgfscope}%
\pgfsys@transformshift{1.390961in}{1.127838in}%
\pgfsys@useobject{currentmarker}{}%
\end{pgfscope}%
\begin{pgfscope}%
\pgfsys@transformshift{1.392285in}{1.136733in}%
\pgfsys@useobject{currentmarker}{}%
\end{pgfscope}%
\begin{pgfscope}%
\pgfsys@transformshift{1.388962in}{1.147613in}%
\pgfsys@useobject{currentmarker}{}%
\end{pgfscope}%
\begin{pgfscope}%
\pgfsys@transformshift{1.391546in}{1.160127in}%
\pgfsys@useobject{currentmarker}{}%
\end{pgfscope}%
\begin{pgfscope}%
\pgfsys@transformshift{1.390510in}{1.167078in}%
\pgfsys@useobject{currentmarker}{}%
\end{pgfscope}%
\begin{pgfscope}%
\pgfsys@transformshift{1.391982in}{1.175223in}%
\pgfsys@useobject{currentmarker}{}%
\end{pgfscope}%
\begin{pgfscope}%
\pgfsys@transformshift{1.393591in}{1.179482in}%
\pgfsys@useobject{currentmarker}{}%
\end{pgfscope}%
\begin{pgfscope}%
\pgfsys@transformshift{1.391353in}{1.186371in}%
\pgfsys@useobject{currentmarker}{}%
\end{pgfscope}%
\begin{pgfscope}%
\pgfsys@transformshift{1.394311in}{1.194348in}%
\pgfsys@useobject{currentmarker}{}%
\end{pgfscope}%
\begin{pgfscope}%
\pgfsys@transformshift{1.391886in}{1.206072in}%
\pgfsys@useobject{currentmarker}{}%
\end{pgfscope}%
\begin{pgfscope}%
\pgfsys@transformshift{1.393269in}{1.212511in}%
\pgfsys@useobject{currentmarker}{}%
\end{pgfscope}%
\begin{pgfscope}%
\pgfsys@transformshift{1.393130in}{1.219723in}%
\pgfsys@useobject{currentmarker}{}%
\end{pgfscope}%
\begin{pgfscope}%
\pgfsys@transformshift{1.391417in}{1.223302in}%
\pgfsys@useobject{currentmarker}{}%
\end{pgfscope}%
\begin{pgfscope}%
\pgfsys@transformshift{1.392931in}{1.228003in}%
\pgfsys@useobject{currentmarker}{}%
\end{pgfscope}%
\begin{pgfscope}%
\pgfsys@transformshift{1.391314in}{1.235795in}%
\pgfsys@useobject{currentmarker}{}%
\end{pgfscope}%
\begin{pgfscope}%
\pgfsys@transformshift{1.393536in}{1.244213in}%
\pgfsys@useobject{currentmarker}{}%
\end{pgfscope}%
\begin{pgfscope}%
\pgfsys@transformshift{1.390278in}{1.256698in}%
\pgfsys@useobject{currentmarker}{}%
\end{pgfscope}%
\begin{pgfscope}%
\pgfsys@transformshift{1.391632in}{1.263664in}%
\pgfsys@useobject{currentmarker}{}%
\end{pgfscope}%
\begin{pgfscope}%
\pgfsys@transformshift{1.391314in}{1.267554in}%
\pgfsys@useobject{currentmarker}{}%
\end{pgfscope}%
\begin{pgfscope}%
\pgfsys@transformshift{1.391319in}{1.269701in}%
\pgfsys@useobject{currentmarker}{}%
\end{pgfscope}%
\begin{pgfscope}%
\pgfsys@transformshift{1.392634in}{1.272611in}%
\pgfsys@useobject{currentmarker}{}%
\end{pgfscope}%
\begin{pgfscope}%
\pgfsys@transformshift{1.391419in}{1.278167in}%
\pgfsys@useobject{currentmarker}{}%
\end{pgfscope}%
\begin{pgfscope}%
\pgfsys@transformshift{1.393367in}{1.284582in}%
\pgfsys@useobject{currentmarker}{}%
\end{pgfscope}%
\begin{pgfscope}%
\pgfsys@transformshift{1.394277in}{1.288156in}%
\pgfsys@useobject{currentmarker}{}%
\end{pgfscope}%
\begin{pgfscope}%
\pgfsys@transformshift{1.394695in}{1.292326in}%
\pgfsys@useobject{currentmarker}{}%
\end{pgfscope}%
\begin{pgfscope}%
\pgfsys@transformshift{1.395527in}{1.297092in}%
\pgfsys@useobject{currentmarker}{}%
\end{pgfscope}%
\begin{pgfscope}%
\pgfsys@transformshift{1.393248in}{1.305384in}%
\pgfsys@useobject{currentmarker}{}%
\end{pgfscope}%
\begin{pgfscope}%
\pgfsys@transformshift{1.394224in}{1.310011in}%
\pgfsys@useobject{currentmarker}{}%
\end{pgfscope}%
\begin{pgfscope}%
\pgfsys@transformshift{1.393892in}{1.312591in}%
\pgfsys@useobject{currentmarker}{}%
\end{pgfscope}%
\begin{pgfscope}%
\pgfsys@transformshift{1.392242in}{1.315475in}%
\pgfsys@useobject{currentmarker}{}%
\end{pgfscope}%
\begin{pgfscope}%
\pgfsys@transformshift{1.392411in}{1.317294in}%
\pgfsys@useobject{currentmarker}{}%
\end{pgfscope}%
\begin{pgfscope}%
\pgfsys@transformshift{1.390695in}{1.321420in}%
\pgfsys@useobject{currentmarker}{}%
\end{pgfscope}%
\begin{pgfscope}%
\pgfsys@transformshift{1.391759in}{1.326517in}%
\pgfsys@useobject{currentmarker}{}%
\end{pgfscope}%
\begin{pgfscope}%
\pgfsys@transformshift{1.389635in}{1.335189in}%
\pgfsys@useobject{currentmarker}{}%
\end{pgfscope}%
\begin{pgfscope}%
\pgfsys@transformshift{1.390662in}{1.339991in}%
\pgfsys@useobject{currentmarker}{}%
\end{pgfscope}%
\begin{pgfscope}%
\pgfsys@transformshift{1.388360in}{1.344912in}%
\pgfsys@useobject{currentmarker}{}%
\end{pgfscope}%
\begin{pgfscope}%
\pgfsys@transformshift{1.386459in}{1.347217in}%
\pgfsys@useobject{currentmarker}{}%
\end{pgfscope}%
\begin{pgfscope}%
\pgfsys@transformshift{1.387130in}{1.351106in}%
\pgfsys@useobject{currentmarker}{}%
\end{pgfscope}%
\begin{pgfscope}%
\pgfsys@transformshift{1.385653in}{1.357084in}%
\pgfsys@useobject{currentmarker}{}%
\end{pgfscope}%
\begin{pgfscope}%
\pgfsys@transformshift{1.386328in}{1.364350in}%
\pgfsys@useobject{currentmarker}{}%
\end{pgfscope}%
\begin{pgfscope}%
\pgfsys@transformshift{1.385821in}{1.368331in}%
\pgfsys@useobject{currentmarker}{}%
\end{pgfscope}%
\begin{pgfscope}%
\pgfsys@transformshift{1.386456in}{1.373459in}%
\pgfsys@useobject{currentmarker}{}%
\end{pgfscope}%
\begin{pgfscope}%
\pgfsys@transformshift{1.387312in}{1.376170in}%
\pgfsys@useobject{currentmarker}{}%
\end{pgfscope}%
\begin{pgfscope}%
\pgfsys@transformshift{1.386390in}{1.379919in}%
\pgfsys@useobject{currentmarker}{}%
\end{pgfscope}%
\begin{pgfscope}%
\pgfsys@transformshift{1.387527in}{1.384642in}%
\pgfsys@useobject{currentmarker}{}%
\end{pgfscope}%
\begin{pgfscope}%
\pgfsys@transformshift{1.386674in}{1.391861in}%
\pgfsys@useobject{currentmarker}{}%
\end{pgfscope}%
\begin{pgfscope}%
\pgfsys@transformshift{1.388600in}{1.400021in}%
\pgfsys@useobject{currentmarker}{}%
\end{pgfscope}%
\begin{pgfscope}%
\pgfsys@transformshift{1.388308in}{1.404623in}%
\pgfsys@useobject{currentmarker}{}%
\end{pgfscope}%
\begin{pgfscope}%
\pgfsys@transformshift{1.387044in}{1.406822in}%
\pgfsys@useobject{currentmarker}{}%
\end{pgfscope}%
\begin{pgfscope}%
\pgfsys@transformshift{1.388766in}{1.412937in}%
\pgfsys@useobject{currentmarker}{}%
\end{pgfscope}%
\begin{pgfscope}%
\pgfsys@transformshift{1.386241in}{1.420433in}%
\pgfsys@useobject{currentmarker}{}%
\end{pgfscope}%
\begin{pgfscope}%
\pgfsys@transformshift{1.387370in}{1.430019in}%
\pgfsys@useobject{currentmarker}{}%
\end{pgfscope}%
\begin{pgfscope}%
\pgfsys@transformshift{1.391538in}{1.439500in}%
\pgfsys@useobject{currentmarker}{}%
\end{pgfscope}%
\begin{pgfscope}%
\pgfsys@transformshift{1.390580in}{1.451328in}%
\pgfsys@useobject{currentmarker}{}%
\end{pgfscope}%
\begin{pgfscope}%
\pgfsys@transformshift{1.390937in}{1.457846in}%
\pgfsys@useobject{currentmarker}{}%
\end{pgfscope}%
\begin{pgfscope}%
\pgfsys@transformshift{1.388273in}{1.466805in}%
\pgfsys@useobject{currentmarker}{}%
\end{pgfscope}%
\begin{pgfscope}%
\pgfsys@transformshift{1.391288in}{1.476434in}%
\pgfsys@useobject{currentmarker}{}%
\end{pgfscope}%
\begin{pgfscope}%
\pgfsys@transformshift{1.389466in}{1.489984in}%
\pgfsys@useobject{currentmarker}{}%
\end{pgfscope}%
\begin{pgfscope}%
\pgfsys@transformshift{1.390640in}{1.504105in}%
\pgfsys@useobject{currentmarker}{}%
\end{pgfscope}%
\begin{pgfscope}%
\pgfsys@transformshift{1.391547in}{1.511845in}%
\pgfsys@useobject{currentmarker}{}%
\end{pgfscope}%
\begin{pgfscope}%
\pgfsys@transformshift{1.387531in}{1.524187in}%
\pgfsys@useobject{currentmarker}{}%
\end{pgfscope}%
\begin{pgfscope}%
\pgfsys@transformshift{1.391558in}{1.540222in}%
\pgfsys@useobject{currentmarker}{}%
\end{pgfscope}%
\begin{pgfscope}%
\pgfsys@transformshift{1.388014in}{1.557174in}%
\pgfsys@useobject{currentmarker}{}%
\end{pgfscope}%
\begin{pgfscope}%
\pgfsys@transformshift{1.392279in}{1.576704in}%
\pgfsys@useobject{currentmarker}{}%
\end{pgfscope}%
\begin{pgfscope}%
\pgfsys@transformshift{1.397675in}{1.596840in}%
\pgfsys@useobject{currentmarker}{}%
\end{pgfscope}%
\begin{pgfscope}%
\pgfsys@transformshift{1.400189in}{1.618095in}%
\pgfsys@useobject{currentmarker}{}%
\end{pgfscope}%
\begin{pgfscope}%
\pgfsys@transformshift{1.401173in}{1.640267in}%
\pgfsys@useobject{currentmarker}{}%
\end{pgfscope}%
\begin{pgfscope}%
\pgfsys@transformshift{1.395428in}{1.665091in}%
\pgfsys@useobject{currentmarker}{}%
\end{pgfscope}%
\begin{pgfscope}%
\pgfsys@transformshift{1.398300in}{1.678808in}%
\pgfsys@useobject{currentmarker}{}%
\end{pgfscope}%
\begin{pgfscope}%
\pgfsys@transformshift{1.397560in}{1.694203in}%
\pgfsys@useobject{currentmarker}{}%
\end{pgfscope}%
\begin{pgfscope}%
\pgfsys@transformshift{1.399011in}{1.702556in}%
\pgfsys@useobject{currentmarker}{}%
\end{pgfscope}%
\begin{pgfscope}%
\pgfsys@transformshift{1.400226in}{1.711681in}%
\pgfsys@useobject{currentmarker}{}%
\end{pgfscope}%
\begin{pgfscope}%
\pgfsys@transformshift{1.402125in}{1.723563in}%
\pgfsys@useobject{currentmarker}{}%
\end{pgfscope}%
\begin{pgfscope}%
\pgfsys@transformshift{1.405867in}{1.738848in}%
\pgfsys@useobject{currentmarker}{}%
\end{pgfscope}%
\begin{pgfscope}%
\pgfsys@transformshift{1.399919in}{1.755514in}%
\pgfsys@useobject{currentmarker}{}%
\end{pgfscope}%
\begin{pgfscope}%
\pgfsys@transformshift{1.403861in}{1.775651in}%
\pgfsys@useobject{currentmarker}{}%
\end{pgfscope}%
\begin{pgfscope}%
\pgfsys@transformshift{1.407373in}{1.796668in}%
\pgfsys@useobject{currentmarker}{}%
\end{pgfscope}%
\begin{pgfscope}%
\pgfsys@transformshift{1.409639in}{1.818661in}%
\pgfsys@useobject{currentmarker}{}%
\end{pgfscope}%
\begin{pgfscope}%
\pgfsys@transformshift{1.410923in}{1.830753in}%
\pgfsys@useobject{currentmarker}{}%
\end{pgfscope}%
\begin{pgfscope}%
\pgfsys@transformshift{1.408829in}{1.837104in}%
\pgfsys@useobject{currentmarker}{}%
\end{pgfscope}%
\begin{pgfscope}%
\pgfsys@transformshift{1.410043in}{1.840577in}%
\pgfsys@useobject{currentmarker}{}%
\end{pgfscope}%
\begin{pgfscope}%
\pgfsys@transformshift{1.408620in}{1.848594in}%
\pgfsys@useobject{currentmarker}{}%
\end{pgfscope}%
\begin{pgfscope}%
\pgfsys@transformshift{1.409679in}{1.852946in}%
\pgfsys@useobject{currentmarker}{}%
\end{pgfscope}%
\begin{pgfscope}%
\pgfsys@transformshift{1.408768in}{1.855235in}%
\pgfsys@useobject{currentmarker}{}%
\end{pgfscope}%
\begin{pgfscope}%
\pgfsys@transformshift{1.406680in}{1.857730in}%
\pgfsys@useobject{currentmarker}{}%
\end{pgfscope}%
\begin{pgfscope}%
\pgfsys@transformshift{1.406957in}{1.859497in}%
\pgfsys@useobject{currentmarker}{}%
\end{pgfscope}%
\begin{pgfscope}%
\pgfsys@transformshift{1.405292in}{1.864698in}%
\pgfsys@useobject{currentmarker}{}%
\end{pgfscope}%
\begin{pgfscope}%
\pgfsys@transformshift{1.405673in}{1.867677in}%
\pgfsys@useobject{currentmarker}{}%
\end{pgfscope}%
\begin{pgfscope}%
\pgfsys@transformshift{1.405502in}{1.869320in}%
\pgfsys@useobject{currentmarker}{}%
\end{pgfscope}%
\begin{pgfscope}%
\pgfsys@transformshift{1.405703in}{1.871547in}%
\pgfsys@useobject{currentmarker}{}%
\end{pgfscope}%
\begin{pgfscope}%
\pgfsys@transformshift{1.406737in}{1.874262in}%
\pgfsys@useobject{currentmarker}{}%
\end{pgfscope}%
\begin{pgfscope}%
\pgfsys@transformshift{1.405312in}{1.878708in}%
\pgfsys@useobject{currentmarker}{}%
\end{pgfscope}%
\begin{pgfscope}%
\pgfsys@transformshift{1.407042in}{1.884481in}%
\pgfsys@useobject{currentmarker}{}%
\end{pgfscope}%
\begin{pgfscope}%
\pgfsys@transformshift{1.405076in}{1.893862in}%
\pgfsys@useobject{currentmarker}{}%
\end{pgfscope}%
\begin{pgfscope}%
\pgfsys@transformshift{1.406366in}{1.898974in}%
\pgfsys@useobject{currentmarker}{}%
\end{pgfscope}%
\begin{pgfscope}%
\pgfsys@transformshift{1.405966in}{1.901846in}%
\pgfsys@useobject{currentmarker}{}%
\end{pgfscope}%
\begin{pgfscope}%
\pgfsys@transformshift{1.406788in}{1.908997in}%
\pgfsys@useobject{currentmarker}{}%
\end{pgfscope}%
\begin{pgfscope}%
\pgfsys@transformshift{1.408644in}{1.919408in}%
\pgfsys@useobject{currentmarker}{}%
\end{pgfscope}%
\begin{pgfscope}%
\pgfsys@transformshift{1.405816in}{1.930213in}%
\pgfsys@useobject{currentmarker}{}%
\end{pgfscope}%
\begin{pgfscope}%
\pgfsys@transformshift{1.408261in}{1.943538in}%
\pgfsys@useobject{currentmarker}{}%
\end{pgfscope}%
\begin{pgfscope}%
\pgfsys@transformshift{1.405099in}{1.950285in}%
\pgfsys@useobject{currentmarker}{}%
\end{pgfscope}%
\begin{pgfscope}%
\pgfsys@transformshift{1.400416in}{1.956972in}%
\pgfsys@useobject{currentmarker}{}%
\end{pgfscope}%
\begin{pgfscope}%
\pgfsys@transformshift{1.401024in}{1.965803in}%
\pgfsys@useobject{currentmarker}{}%
\end{pgfscope}%
\begin{pgfscope}%
\pgfsys@transformshift{1.398145in}{1.975884in}%
\pgfsys@useobject{currentmarker}{}%
\end{pgfscope}%
\begin{pgfscope}%
\pgfsys@transformshift{1.399129in}{1.981565in}%
\pgfsys@useobject{currentmarker}{}%
\end{pgfscope}%
\begin{pgfscope}%
\pgfsys@transformshift{1.397445in}{1.988295in}%
\pgfsys@useobject{currentmarker}{}%
\end{pgfscope}%
\begin{pgfscope}%
\pgfsys@transformshift{1.397367in}{1.992109in}%
\pgfsys@useobject{currentmarker}{}%
\end{pgfscope}%
\begin{pgfscope}%
\pgfsys@transformshift{1.398594in}{1.996552in}%
\pgfsys@useobject{currentmarker}{}%
\end{pgfscope}%
\begin{pgfscope}%
\pgfsys@transformshift{1.396676in}{2.003600in}%
\pgfsys@useobject{currentmarker}{}%
\end{pgfscope}%
\begin{pgfscope}%
\pgfsys@transformshift{1.399018in}{2.012025in}%
\pgfsys@useobject{currentmarker}{}%
\end{pgfscope}%
\begin{pgfscope}%
\pgfsys@transformshift{1.397004in}{2.024061in}%
\pgfsys@useobject{currentmarker}{}%
\end{pgfscope}%
\begin{pgfscope}%
\pgfsys@transformshift{1.399511in}{2.037281in}%
\pgfsys@useobject{currentmarker}{}%
\end{pgfscope}%
\begin{pgfscope}%
\pgfsys@transformshift{1.398307in}{2.044583in}%
\pgfsys@useobject{currentmarker}{}%
\end{pgfscope}%
\begin{pgfscope}%
\pgfsys@transformshift{1.395540in}{2.055781in}%
\pgfsys@useobject{currentmarker}{}%
\end{pgfscope}%
\begin{pgfscope}%
\pgfsys@transformshift{1.398682in}{2.067737in}%
\pgfsys@useobject{currentmarker}{}%
\end{pgfscope}%
\begin{pgfscope}%
\pgfsys@transformshift{1.392840in}{2.084785in}%
\pgfsys@useobject{currentmarker}{}%
\end{pgfscope}%
\begin{pgfscope}%
\pgfsys@transformshift{1.394947in}{2.103367in}%
\pgfsys@useobject{currentmarker}{}%
\end{pgfscope}%
\begin{pgfscope}%
\pgfsys@transformshift{1.390253in}{2.112519in}%
\pgfsys@useobject{currentmarker}{}%
\end{pgfscope}%
\begin{pgfscope}%
\pgfsys@transformshift{1.386667in}{2.116895in}%
\pgfsys@useobject{currentmarker}{}%
\end{pgfscope}%
\begin{pgfscope}%
\pgfsys@transformshift{1.388158in}{2.124920in}%
\pgfsys@useobject{currentmarker}{}%
\end{pgfscope}%
\begin{pgfscope}%
\pgfsys@transformshift{1.385079in}{2.134389in}%
\pgfsys@useobject{currentmarker}{}%
\end{pgfscope}%
\begin{pgfscope}%
\pgfsys@transformshift{1.386594in}{2.146721in}%
\pgfsys@useobject{currentmarker}{}%
\end{pgfscope}%
\begin{pgfscope}%
\pgfsys@transformshift{1.385413in}{2.153451in}%
\pgfsys@useobject{currentmarker}{}%
\end{pgfscope}%
\begin{pgfscope}%
\pgfsys@transformshift{1.384934in}{2.161682in}%
\pgfsys@useobject{currentmarker}{}%
\end{pgfscope}%
\begin{pgfscope}%
\pgfsys@transformshift{1.386620in}{2.165891in}%
\pgfsys@useobject{currentmarker}{}%
\end{pgfscope}%
\begin{pgfscope}%
\pgfsys@transformshift{1.384568in}{2.173672in}%
\pgfsys@useobject{currentmarker}{}%
\end{pgfscope}%
\begin{pgfscope}%
\pgfsys@transformshift{1.386678in}{2.182877in}%
\pgfsys@useobject{currentmarker}{}%
\end{pgfscope}%
\begin{pgfscope}%
\pgfsys@transformshift{1.383277in}{2.193830in}%
\pgfsys@useobject{currentmarker}{}%
\end{pgfscope}%
\begin{pgfscope}%
\pgfsys@transformshift{1.383886in}{2.200109in}%
\pgfsys@useobject{currentmarker}{}%
\end{pgfscope}%
\begin{pgfscope}%
\pgfsys@transformshift{1.383244in}{2.203519in}%
\pgfsys@useobject{currentmarker}{}%
\end{pgfscope}%
\begin{pgfscope}%
\pgfsys@transformshift{1.382229in}{2.205134in}%
\pgfsys@useobject{currentmarker}{}%
\end{pgfscope}%
\begin{pgfscope}%
\pgfsys@transformshift{1.382915in}{2.210021in}%
\pgfsys@useobject{currentmarker}{}%
\end{pgfscope}%
\begin{pgfscope}%
\pgfsys@transformshift{1.381124in}{2.216344in}%
\pgfsys@useobject{currentmarker}{}%
\end{pgfscope}%
\begin{pgfscope}%
\pgfsys@transformshift{1.383350in}{2.225952in}%
\pgfsys@useobject{currentmarker}{}%
\end{pgfscope}%
\begin{pgfscope}%
\pgfsys@transformshift{1.380622in}{2.236715in}%
\pgfsys@useobject{currentmarker}{}%
\end{pgfscope}%
\begin{pgfscope}%
\pgfsys@transformshift{1.382971in}{2.248401in}%
\pgfsys@useobject{currentmarker}{}%
\end{pgfscope}%
\begin{pgfscope}%
\pgfsys@transformshift{1.382414in}{2.260972in}%
\pgfsys@useobject{currentmarker}{}%
\end{pgfscope}%
\begin{pgfscope}%
\pgfsys@transformshift{1.385279in}{2.276060in}%
\pgfsys@useobject{currentmarker}{}%
\end{pgfscope}%
\begin{pgfscope}%
\pgfsys@transformshift{1.386008in}{2.294976in}%
\pgfsys@useobject{currentmarker}{}%
\end{pgfscope}%
\begin{pgfscope}%
\pgfsys@transformshift{1.380018in}{2.314926in}%
\pgfsys@useobject{currentmarker}{}%
\end{pgfscope}%
\begin{pgfscope}%
\pgfsys@transformshift{1.383135in}{2.338124in}%
\pgfsys@useobject{currentmarker}{}%
\end{pgfscope}%
\begin{pgfscope}%
\pgfsys@transformshift{1.381565in}{2.350902in}%
\pgfsys@useobject{currentmarker}{}%
\end{pgfscope}%
\begin{pgfscope}%
\pgfsys@transformshift{1.381495in}{2.357982in}%
\pgfsys@useobject{currentmarker}{}%
\end{pgfscope}%
\begin{pgfscope}%
\pgfsys@transformshift{1.383014in}{2.361568in}%
\pgfsys@useobject{currentmarker}{}%
\end{pgfscope}%
\begin{pgfscope}%
\pgfsys@transformshift{1.382511in}{2.363650in}%
\pgfsys@useobject{currentmarker}{}%
\end{pgfscope}%
\begin{pgfscope}%
\pgfsys@transformshift{1.383121in}{2.366413in}%
\pgfsys@useobject{currentmarker}{}%
\end{pgfscope}%
\begin{pgfscope}%
\pgfsys@transformshift{1.382128in}{2.372700in}%
\pgfsys@useobject{currentmarker}{}%
\end{pgfscope}%
\begin{pgfscope}%
\pgfsys@transformshift{1.382371in}{2.379595in}%
\pgfsys@useobject{currentmarker}{}%
\end{pgfscope}%
\begin{pgfscope}%
\pgfsys@transformshift{1.382636in}{2.383380in}%
\pgfsys@useobject{currentmarker}{}%
\end{pgfscope}%
\begin{pgfscope}%
\pgfsys@transformshift{1.381151in}{2.391084in}%
\pgfsys@useobject{currentmarker}{}%
\end{pgfscope}%
\begin{pgfscope}%
\pgfsys@transformshift{1.384837in}{2.401791in}%
\pgfsys@useobject{currentmarker}{}%
\end{pgfscope}%
\begin{pgfscope}%
\pgfsys@transformshift{1.381466in}{2.414309in}%
\pgfsys@useobject{currentmarker}{}%
\end{pgfscope}%
\begin{pgfscope}%
\pgfsys@transformshift{1.384888in}{2.429367in}%
\pgfsys@useobject{currentmarker}{}%
\end{pgfscope}%
\begin{pgfscope}%
\pgfsys@transformshift{1.384827in}{2.437860in}%
\pgfsys@useobject{currentmarker}{}%
\end{pgfscope}%
\begin{pgfscope}%
\pgfsys@transformshift{1.387581in}{2.451234in}%
\pgfsys@useobject{currentmarker}{}%
\end{pgfscope}%
\begin{pgfscope}%
\pgfsys@transformshift{1.391247in}{2.467118in}%
\pgfsys@useobject{currentmarker}{}%
\end{pgfscope}%
\begin{pgfscope}%
\pgfsys@transformshift{1.387030in}{2.484245in}%
\pgfsys@useobject{currentmarker}{}%
\end{pgfscope}%
\begin{pgfscope}%
\pgfsys@transformshift{1.392284in}{2.502841in}%
\pgfsys@useobject{currentmarker}{}%
\end{pgfscope}%
\begin{pgfscope}%
\pgfsys@transformshift{1.396902in}{2.522118in}%
\pgfsys@useobject{currentmarker}{}%
\end{pgfscope}%
\begin{pgfscope}%
\pgfsys@transformshift{1.400062in}{2.542495in}%
\pgfsys@useobject{currentmarker}{}%
\end{pgfscope}%
\begin{pgfscope}%
\pgfsys@transformshift{1.405682in}{2.563099in}%
\pgfsys@useobject{currentmarker}{}%
\end{pgfscope}%
\begin{pgfscope}%
\pgfsys@transformshift{1.399835in}{2.585379in}%
\pgfsys@useobject{currentmarker}{}%
\end{pgfscope}%
\begin{pgfscope}%
\pgfsys@transformshift{1.408641in}{2.608375in}%
\pgfsys@useobject{currentmarker}{}%
\end{pgfscope}%
\begin{pgfscope}%
\pgfsys@transformshift{1.406545in}{2.633419in}%
\pgfsys@useobject{currentmarker}{}%
\end{pgfscope}%
\begin{pgfscope}%
\pgfsys@transformshift{1.408570in}{2.647093in}%
\pgfsys@useobject{currentmarker}{}%
\end{pgfscope}%
\begin{pgfscope}%
\pgfsys@transformshift{1.410838in}{2.654349in}%
\pgfsys@useobject{currentmarker}{}%
\end{pgfscope}%
\begin{pgfscope}%
\pgfsys@transformshift{1.409479in}{2.664588in}%
\pgfsys@useobject{currentmarker}{}%
\end{pgfscope}%
\begin{pgfscope}%
\pgfsys@transformshift{1.412344in}{2.675249in}%
\pgfsys@useobject{currentmarker}{}%
\end{pgfscope}%
\begin{pgfscope}%
\pgfsys@transformshift{1.409550in}{2.680639in}%
\pgfsys@useobject{currentmarker}{}%
\end{pgfscope}%
\begin{pgfscope}%
\pgfsys@transformshift{1.405572in}{2.685980in}%
\pgfsys@useobject{currentmarker}{}%
\end{pgfscope}%
\begin{pgfscope}%
\pgfsys@transformshift{1.406864in}{2.694423in}%
\pgfsys@useobject{currentmarker}{}%
\end{pgfscope}%
\begin{pgfscope}%
\pgfsys@transformshift{1.404331in}{2.703859in}%
\pgfsys@useobject{currentmarker}{}%
\end{pgfscope}%
\begin{pgfscope}%
\pgfsys@transformshift{1.406022in}{2.714842in}%
\pgfsys@useobject{currentmarker}{}%
\end{pgfscope}%
\begin{pgfscope}%
\pgfsys@transformshift{1.405720in}{2.720947in}%
\pgfsys@useobject{currentmarker}{}%
\end{pgfscope}%
\begin{pgfscope}%
\pgfsys@transformshift{1.407121in}{2.731311in}%
\pgfsys@useobject{currentmarker}{}%
\end{pgfscope}%
\begin{pgfscope}%
\pgfsys@transformshift{1.410176in}{2.745120in}%
\pgfsys@useobject{currentmarker}{}%
\end{pgfscope}%
\begin{pgfscope}%
\pgfsys@transformshift{1.409683in}{2.760648in}%
\pgfsys@useobject{currentmarker}{}%
\end{pgfscope}%
\begin{pgfscope}%
\pgfsys@transformshift{1.413262in}{2.777597in}%
\pgfsys@useobject{currentmarker}{}%
\end{pgfscope}%
\begin{pgfscope}%
\pgfsys@transformshift{1.415071in}{2.786952in}%
\pgfsys@useobject{currentmarker}{}%
\end{pgfscope}%
\begin{pgfscope}%
\pgfsys@transformshift{1.412411in}{2.796763in}%
\pgfsys@useobject{currentmarker}{}%
\end{pgfscope}%
\begin{pgfscope}%
\pgfsys@transformshift{1.416241in}{2.806761in}%
\pgfsys@useobject{currentmarker}{}%
\end{pgfscope}%
\begin{pgfscope}%
\pgfsys@transformshift{1.415287in}{2.818477in}%
\pgfsys@useobject{currentmarker}{}%
\end{pgfscope}%
\begin{pgfscope}%
\pgfsys@transformshift{1.418422in}{2.831361in}%
\pgfsys@useobject{currentmarker}{}%
\end{pgfscope}%
\begin{pgfscope}%
\pgfsys@transformshift{1.420746in}{2.838274in}%
\pgfsys@useobject{currentmarker}{}%
\end{pgfscope}%
\begin{pgfscope}%
\pgfsys@transformshift{1.419277in}{2.846371in}%
\pgfsys@useobject{currentmarker}{}%
\end{pgfscope}%
\begin{pgfscope}%
\pgfsys@transformshift{1.421986in}{2.855913in}%
\pgfsys@useobject{currentmarker}{}%
\end{pgfscope}%
\begin{pgfscope}%
\pgfsys@transformshift{1.419635in}{2.860836in}%
\pgfsys@useobject{currentmarker}{}%
\end{pgfscope}%
\begin{pgfscope}%
\pgfsys@transformshift{1.416209in}{2.865919in}%
\pgfsys@useobject{currentmarker}{}%
\end{pgfscope}%
\begin{pgfscope}%
\pgfsys@transformshift{1.417628in}{2.873739in}%
\pgfsys@useobject{currentmarker}{}%
\end{pgfscope}%
\begin{pgfscope}%
\pgfsys@transformshift{1.414966in}{2.882706in}%
\pgfsys@useobject{currentmarker}{}%
\end{pgfscope}%
\begin{pgfscope}%
\pgfsys@transformshift{1.417197in}{2.893027in}%
\pgfsys@useobject{currentmarker}{}%
\end{pgfscope}%
\begin{pgfscope}%
\pgfsys@transformshift{1.417573in}{2.904133in}%
\pgfsys@useobject{currentmarker}{}%
\end{pgfscope}%
\begin{pgfscope}%
\pgfsys@transformshift{1.415779in}{2.917274in}%
\pgfsys@useobject{currentmarker}{}%
\end{pgfscope}%
\begin{pgfscope}%
\pgfsys@transformshift{1.421673in}{2.932756in}%
\pgfsys@useobject{currentmarker}{}%
\end{pgfscope}%
\begin{pgfscope}%
\pgfsys@transformshift{1.420940in}{2.950071in}%
\pgfsys@useobject{currentmarker}{}%
\end{pgfscope}%
\begin{pgfscope}%
\pgfsys@transformshift{1.423683in}{2.967968in}%
\pgfsys@useobject{currentmarker}{}%
\end{pgfscope}%
\begin{pgfscope}%
\pgfsys@transformshift{1.425981in}{2.977657in}%
\pgfsys@useobject{currentmarker}{}%
\end{pgfscope}%
\begin{pgfscope}%
\pgfsys@transformshift{1.424985in}{2.983043in}%
\pgfsys@useobject{currentmarker}{}%
\end{pgfscope}%
\begin{pgfscope}%
\pgfsys@transformshift{1.427149in}{2.989253in}%
\pgfsys@useobject{currentmarker}{}%
\end{pgfscope}%
\begin{pgfscope}%
\pgfsys@transformshift{1.428368in}{2.992658in}%
\pgfsys@useobject{currentmarker}{}%
\end{pgfscope}%
\begin{pgfscope}%
\pgfsys@transformshift{1.427877in}{2.994586in}%
\pgfsys@useobject{currentmarker}{}%
\end{pgfscope}%
\begin{pgfscope}%
\pgfsys@transformshift{1.428126in}{2.997159in}%
\pgfsys@useobject{currentmarker}{}%
\end{pgfscope}%
\begin{pgfscope}%
\pgfsys@transformshift{1.427137in}{3.001283in}%
\pgfsys@useobject{currentmarker}{}%
\end{pgfscope}%
\begin{pgfscope}%
\pgfsys@transformshift{1.428192in}{3.006014in}%
\pgfsys@useobject{currentmarker}{}%
\end{pgfscope}%
\begin{pgfscope}%
\pgfsys@transformshift{1.428212in}{3.008680in}%
\pgfsys@useobject{currentmarker}{}%
\end{pgfscope}%
\begin{pgfscope}%
\pgfsys@transformshift{1.429755in}{3.014004in}%
\pgfsys@useobject{currentmarker}{}%
\end{pgfscope}%
\begin{pgfscope}%
\pgfsys@transformshift{1.430641in}{3.021694in}%
\pgfsys@useobject{currentmarker}{}%
\end{pgfscope}%
\begin{pgfscope}%
\pgfsys@transformshift{1.430397in}{3.030828in}%
\pgfsys@useobject{currentmarker}{}%
\end{pgfscope}%
\begin{pgfscope}%
\pgfsys@transformshift{1.432039in}{3.041051in}%
\pgfsys@useobject{currentmarker}{}%
\end{pgfscope}%
\begin{pgfscope}%
\pgfsys@transformshift{1.433251in}{3.046616in}%
\pgfsys@useobject{currentmarker}{}%
\end{pgfscope}%
\begin{pgfscope}%
\pgfsys@transformshift{1.432134in}{3.053796in}%
\pgfsys@useobject{currentmarker}{}%
\end{pgfscope}%
\begin{pgfscope}%
\pgfsys@transformshift{1.434883in}{3.061839in}%
\pgfsys@useobject{currentmarker}{}%
\end{pgfscope}%
\begin{pgfscope}%
\pgfsys@transformshift{1.434488in}{3.066497in}%
\pgfsys@useobject{currentmarker}{}%
\end{pgfscope}%
\begin{pgfscope}%
\pgfsys@transformshift{1.435661in}{3.074975in}%
\pgfsys@useobject{currentmarker}{}%
\end{pgfscope}%
\begin{pgfscope}%
\pgfsys@transformshift{1.439052in}{3.086494in}%
\pgfsys@useobject{currentmarker}{}%
\end{pgfscope}%
\begin{pgfscope}%
\pgfsys@transformshift{1.437142in}{3.099197in}%
\pgfsys@useobject{currentmarker}{}%
\end{pgfscope}%
\begin{pgfscope}%
\pgfsys@transformshift{1.440968in}{3.112472in}%
\pgfsys@useobject{currentmarker}{}%
\end{pgfscope}%
\begin{pgfscope}%
\pgfsys@transformshift{1.441765in}{3.120028in}%
\pgfsys@useobject{currentmarker}{}%
\end{pgfscope}%
\begin{pgfscope}%
\pgfsys@transformshift{1.441015in}{3.124139in}%
\pgfsys@useobject{currentmarker}{}%
\end{pgfscope}%
\begin{pgfscope}%
\pgfsys@transformshift{1.441666in}{3.126344in}%
\pgfsys@useobject{currentmarker}{}%
\end{pgfscope}%
\begin{pgfscope}%
\pgfsys@transformshift{1.441255in}{3.127539in}%
\pgfsys@useobject{currentmarker}{}%
\end{pgfscope}%
\begin{pgfscope}%
\pgfsys@transformshift{1.440226in}{3.129115in}%
\pgfsys@useobject{currentmarker}{}%
\end{pgfscope}%
\begin{pgfscope}%
\pgfsys@transformshift{1.440828in}{3.132792in}%
\pgfsys@useobject{currentmarker}{}%
\end{pgfscope}%
\begin{pgfscope}%
\pgfsys@transformshift{1.440154in}{3.138067in}%
\pgfsys@useobject{currentmarker}{}%
\end{pgfscope}%
\begin{pgfscope}%
\pgfsys@transformshift{1.440452in}{3.140977in}%
\pgfsys@useobject{currentmarker}{}%
\end{pgfscope}%
\begin{pgfscope}%
\pgfsys@transformshift{1.440786in}{3.142550in}%
\pgfsys@useobject{currentmarker}{}%
\end{pgfscope}%
\begin{pgfscope}%
\pgfsys@transformshift{1.440313in}{3.145579in}%
\pgfsys@useobject{currentmarker}{}%
\end{pgfscope}%
\begin{pgfscope}%
\pgfsys@transformshift{1.441633in}{3.150039in}%
\pgfsys@useobject{currentmarker}{}%
\end{pgfscope}%
\begin{pgfscope}%
\pgfsys@transformshift{1.441441in}{3.155192in}%
\pgfsys@useobject{currentmarker}{}%
\end{pgfscope}%
\begin{pgfscope}%
\pgfsys@transformshift{1.441833in}{3.158001in}%
\pgfsys@useobject{currentmarker}{}%
\end{pgfscope}%
\begin{pgfscope}%
\pgfsys@transformshift{1.443265in}{3.163446in}%
\pgfsys@useobject{currentmarker}{}%
\end{pgfscope}%
\begin{pgfscope}%
\pgfsys@transformshift{1.443044in}{3.169815in}%
\pgfsys@useobject{currentmarker}{}%
\end{pgfscope}%
\begin{pgfscope}%
\pgfsys@transformshift{1.443618in}{3.173272in}%
\pgfsys@useobject{currentmarker}{}%
\end{pgfscope}%
\begin{pgfscope}%
\pgfsys@transformshift{1.444387in}{3.175039in}%
\pgfsys@useobject{currentmarker}{}%
\end{pgfscope}%
\begin{pgfscope}%
\pgfsys@transformshift{1.443923in}{3.177599in}%
\pgfsys@useobject{currentmarker}{}%
\end{pgfscope}%
\begin{pgfscope}%
\pgfsys@transformshift{1.444986in}{3.180842in}%
\pgfsys@useobject{currentmarker}{}%
\end{pgfscope}%
\begin{pgfscope}%
\pgfsys@transformshift{1.444715in}{3.184849in}%
\pgfsys@useobject{currentmarker}{}%
\end{pgfscope}%
\begin{pgfscope}%
\pgfsys@transformshift{1.444477in}{3.187045in}%
\pgfsys@useobject{currentmarker}{}%
\end{pgfscope}%
\begin{pgfscope}%
\pgfsys@transformshift{1.444853in}{3.188200in}%
\pgfsys@useobject{currentmarker}{}%
\end{pgfscope}%
\begin{pgfscope}%
\pgfsys@transformshift{1.444978in}{3.190584in}%
\pgfsys@useobject{currentmarker}{}%
\end{pgfscope}%
\begin{pgfscope}%
\pgfsys@transformshift{1.445488in}{3.193524in}%
\pgfsys@useobject{currentmarker}{}%
\end{pgfscope}%
\begin{pgfscope}%
\pgfsys@transformshift{1.447098in}{3.197284in}%
\pgfsys@useobject{currentmarker}{}%
\end{pgfscope}%
\begin{pgfscope}%
\pgfsys@transformshift{1.446021in}{3.203066in}%
\pgfsys@useobject{currentmarker}{}%
\end{pgfscope}%
\begin{pgfscope}%
\pgfsys@transformshift{1.447951in}{3.209170in}%
\pgfsys@useobject{currentmarker}{}%
\end{pgfscope}%
\begin{pgfscope}%
\pgfsys@transformshift{1.448475in}{3.212652in}%
\pgfsys@useobject{currentmarker}{}%
\end{pgfscope}%
\begin{pgfscope}%
\pgfsys@transformshift{1.448231in}{3.214574in}%
\pgfsys@useobject{currentmarker}{}%
\end{pgfscope}%
\begin{pgfscope}%
\pgfsys@transformshift{1.448488in}{3.215607in}%
\pgfsys@useobject{currentmarker}{}%
\end{pgfscope}%
\begin{pgfscope}%
\pgfsys@transformshift{1.448476in}{3.216193in}%
\pgfsys@useobject{currentmarker}{}%
\end{pgfscope}%
\begin{pgfscope}%
\pgfsys@transformshift{1.448661in}{3.220964in}%
\pgfsys@useobject{currentmarker}{}%
\end{pgfscope}%
\begin{pgfscope}%
\pgfsys@transformshift{1.450972in}{3.228716in}%
\pgfsys@useobject{currentmarker}{}%
\end{pgfscope}%
\begin{pgfscope}%
\pgfsys@transformshift{1.452062in}{3.237250in}%
\pgfsys@useobject{currentmarker}{}%
\end{pgfscope}%
\begin{pgfscope}%
\pgfsys@transformshift{1.452813in}{3.241921in}%
\pgfsys@useobject{currentmarker}{}%
\end{pgfscope}%
\begin{pgfscope}%
\pgfsys@transformshift{1.453497in}{3.244432in}%
\pgfsys@useobject{currentmarker}{}%
\end{pgfscope}%
\begin{pgfscope}%
\pgfsys@transformshift{1.452770in}{3.250295in}%
\pgfsys@useobject{currentmarker}{}%
\end{pgfscope}%
\begin{pgfscope}%
\pgfsys@transformshift{1.454106in}{3.256693in}%
\pgfsys@useobject{currentmarker}{}%
\end{pgfscope}%
\begin{pgfscope}%
\pgfsys@transformshift{1.454625in}{3.260251in}%
\pgfsys@useobject{currentmarker}{}%
\end{pgfscope}%
\begin{pgfscope}%
\pgfsys@transformshift{1.454203in}{3.265481in}%
\pgfsys@useobject{currentmarker}{}%
\end{pgfscope}%
\begin{pgfscope}%
\pgfsys@transformshift{1.456132in}{3.272281in}%
\pgfsys@useobject{currentmarker}{}%
\end{pgfscope}%
\begin{pgfscope}%
\pgfsys@transformshift{1.456255in}{3.276166in}%
\pgfsys@useobject{currentmarker}{}%
\end{pgfscope}%
\begin{pgfscope}%
\pgfsys@transformshift{1.455624in}{3.283580in}%
\pgfsys@useobject{currentmarker}{}%
\end{pgfscope}%
\begin{pgfscope}%
\pgfsys@transformshift{1.459400in}{3.293602in}%
\pgfsys@useobject{currentmarker}{}%
\end{pgfscope}%
\begin{pgfscope}%
\pgfsys@transformshift{1.460080in}{3.305022in}%
\pgfsys@useobject{currentmarker}{}%
\end{pgfscope}%
\begin{pgfscope}%
\pgfsys@transformshift{1.455713in}{3.316391in}%
\pgfsys@useobject{currentmarker}{}%
\end{pgfscope}%
\begin{pgfscope}%
\pgfsys@transformshift{1.456784in}{3.329099in}%
\pgfsys@useobject{currentmarker}{}%
\end{pgfscope}%
\begin{pgfscope}%
\pgfsys@transformshift{1.460594in}{3.342935in}%
\pgfsys@useobject{currentmarker}{}%
\end{pgfscope}%
\begin{pgfscope}%
\pgfsys@transformshift{1.461812in}{3.350733in}%
\pgfsys@useobject{currentmarker}{}%
\end{pgfscope}%
\begin{pgfscope}%
\pgfsys@transformshift{1.463938in}{3.359081in}%
\pgfsys@useobject{currentmarker}{}%
\end{pgfscope}%
\begin{pgfscope}%
\pgfsys@transformshift{1.461471in}{3.369715in}%
\pgfsys@useobject{currentmarker}{}%
\end{pgfscope}%
\begin{pgfscope}%
\pgfsys@transformshift{1.463213in}{3.375461in}%
\pgfsys@useobject{currentmarker}{}%
\end{pgfscope}%
\begin{pgfscope}%
\pgfsys@transformshift{1.463558in}{3.382015in}%
\pgfsys@useobject{currentmarker}{}%
\end{pgfscope}%
\begin{pgfscope}%
\pgfsys@transformshift{1.463547in}{3.385626in}%
\pgfsys@useobject{currentmarker}{}%
\end{pgfscope}%
\begin{pgfscope}%
\pgfsys@transformshift{1.463905in}{3.387579in}%
\pgfsys@useobject{currentmarker}{}%
\end{pgfscope}%
\begin{pgfscope}%
\pgfsys@transformshift{1.463973in}{3.390102in}%
\pgfsys@useobject{currentmarker}{}%
\end{pgfscope}%
\begin{pgfscope}%
\pgfsys@transformshift{1.464935in}{3.396322in}%
\pgfsys@useobject{currentmarker}{}%
\end{pgfscope}%
\begin{pgfscope}%
\pgfsys@transformshift{1.466792in}{3.405341in}%
\pgfsys@useobject{currentmarker}{}%
\end{pgfscope}%
\begin{pgfscope}%
\pgfsys@transformshift{1.467505in}{3.415079in}%
\pgfsys@useobject{currentmarker}{}%
\end{pgfscope}%
\begin{pgfscope}%
\pgfsys@transformshift{1.466901in}{3.420416in}%
\pgfsys@useobject{currentmarker}{}%
\end{pgfscope}%
\begin{pgfscope}%
\pgfsys@transformshift{1.469007in}{3.426082in}%
\pgfsys@useobject{currentmarker}{}%
\end{pgfscope}%
\begin{pgfscope}%
\pgfsys@transformshift{1.468762in}{3.434025in}%
\pgfsys@useobject{currentmarker}{}%
\end{pgfscope}%
\begin{pgfscope}%
\pgfsys@transformshift{1.469881in}{3.442556in}%
\pgfsys@useobject{currentmarker}{}%
\end{pgfscope}%
\begin{pgfscope}%
\pgfsys@transformshift{1.471566in}{3.446979in}%
\pgfsys@useobject{currentmarker}{}%
\end{pgfscope}%
\begin{pgfscope}%
\pgfsys@transformshift{1.471065in}{3.449533in}%
\pgfsys@useobject{currentmarker}{}%
\end{pgfscope}%
\begin{pgfscope}%
\pgfsys@transformshift{1.471162in}{3.452751in}%
\pgfsys@useobject{currentmarker}{}%
\end{pgfscope}%
\begin{pgfscope}%
\pgfsys@transformshift{1.471945in}{3.456744in}%
\pgfsys@useobject{currentmarker}{}%
\end{pgfscope}%
\begin{pgfscope}%
\pgfsys@transformshift{1.471992in}{3.458981in}%
\pgfsys@useobject{currentmarker}{}%
\end{pgfscope}%
\begin{pgfscope}%
\pgfsys@transformshift{1.473169in}{3.465489in}%
\pgfsys@useobject{currentmarker}{}%
\end{pgfscope}%
\begin{pgfscope}%
\pgfsys@transformshift{1.475395in}{3.474299in}%
\pgfsys@useobject{currentmarker}{}%
\end{pgfscope}%
\begin{pgfscope}%
\pgfsys@transformshift{1.475668in}{3.483919in}%
\pgfsys@useobject{currentmarker}{}%
\end{pgfscope}%
\begin{pgfscope}%
\pgfsys@transformshift{1.474423in}{3.489063in}%
\pgfsys@useobject{currentmarker}{}%
\end{pgfscope}%
\begin{pgfscope}%
\pgfsys@transformshift{1.476684in}{3.495071in}%
\pgfsys@useobject{currentmarker}{}%
\end{pgfscope}%
\begin{pgfscope}%
\pgfsys@transformshift{1.474287in}{3.502075in}%
\pgfsys@useobject{currentmarker}{}%
\end{pgfscope}%
\begin{pgfscope}%
\pgfsys@transformshift{1.470408in}{3.509231in}%
\pgfsys@useobject{currentmarker}{}%
\end{pgfscope}%
\begin{pgfscope}%
\pgfsys@transformshift{1.472378in}{3.517781in}%
\pgfsys@useobject{currentmarker}{}%
\end{pgfscope}%
\begin{pgfscope}%
\pgfsys@transformshift{1.469694in}{3.527651in}%
\pgfsys@useobject{currentmarker}{}%
\end{pgfscope}%
\begin{pgfscope}%
\pgfsys@transformshift{1.470915in}{3.533143in}%
\pgfsys@useobject{currentmarker}{}%
\end{pgfscope}%
\begin{pgfscope}%
\pgfsys@transformshift{1.471760in}{3.539323in}%
\pgfsys@useobject{currentmarker}{}%
\end{pgfscope}%
\begin{pgfscope}%
\pgfsys@transformshift{1.471323in}{3.542726in}%
\pgfsys@useobject{currentmarker}{}%
\end{pgfscope}%
\begin{pgfscope}%
\pgfsys@transformshift{1.472253in}{3.547349in}%
\pgfsys@useobject{currentmarker}{}%
\end{pgfscope}%
\begin{pgfscope}%
\pgfsys@transformshift{1.472351in}{3.549940in}%
\pgfsys@useobject{currentmarker}{}%
\end{pgfscope}%
\begin{pgfscope}%
\pgfsys@transformshift{1.472293in}{3.551365in}%
\pgfsys@useobject{currentmarker}{}%
\end{pgfscope}%
\begin{pgfscope}%
\pgfsys@transformshift{1.473277in}{3.555099in}%
\pgfsys@useobject{currentmarker}{}%
\end{pgfscope}%
\begin{pgfscope}%
\pgfsys@transformshift{1.473226in}{3.557221in}%
\pgfsys@useobject{currentmarker}{}%
\end{pgfscope}%
\begin{pgfscope}%
\pgfsys@transformshift{1.473313in}{3.558386in}%
\pgfsys@useobject{currentmarker}{}%
\end{pgfscope}%
\begin{pgfscope}%
\pgfsys@transformshift{1.473976in}{3.561626in}%
\pgfsys@useobject{currentmarker}{}%
\end{pgfscope}%
\begin{pgfscope}%
\pgfsys@transformshift{1.473950in}{3.563445in}%
\pgfsys@useobject{currentmarker}{}%
\end{pgfscope}%
\begin{pgfscope}%
\pgfsys@transformshift{1.473427in}{3.566733in}%
\pgfsys@useobject{currentmarker}{}%
\end{pgfscope}%
\begin{pgfscope}%
\pgfsys@transformshift{1.474398in}{3.570667in}%
\pgfsys@useobject{currentmarker}{}%
\end{pgfscope}%
\begin{pgfscope}%
\pgfsys@transformshift{1.473874in}{3.575430in}%
\pgfsys@useobject{currentmarker}{}%
\end{pgfscope}%
\begin{pgfscope}%
\pgfsys@transformshift{1.472702in}{3.580724in}%
\pgfsys@useobject{currentmarker}{}%
\end{pgfscope}%
\begin{pgfscope}%
\pgfsys@transformshift{1.475234in}{3.587536in}%
\pgfsys@useobject{currentmarker}{}%
\end{pgfscope}%
\begin{pgfscope}%
\pgfsys@transformshift{1.474077in}{3.596228in}%
\pgfsys@useobject{currentmarker}{}%
\end{pgfscope}%
\begin{pgfscope}%
\pgfsys@transformshift{1.474820in}{3.600993in}%
\pgfsys@useobject{currentmarker}{}%
\end{pgfscope}%
\begin{pgfscope}%
\pgfsys@transformshift{1.475452in}{3.603569in}%
\pgfsys@useobject{currentmarker}{}%
\end{pgfscope}%
\begin{pgfscope}%
\pgfsys@transformshift{1.474840in}{3.607557in}%
\pgfsys@useobject{currentmarker}{}%
\end{pgfscope}%
\begin{pgfscope}%
\pgfsys@transformshift{1.476142in}{3.612141in}%
\pgfsys@useobject{currentmarker}{}%
\end{pgfscope}%
\begin{pgfscope}%
\pgfsys@transformshift{1.476119in}{3.617514in}%
\pgfsys@useobject{currentmarker}{}%
\end{pgfscope}%
\begin{pgfscope}%
\pgfsys@transformshift{1.476341in}{3.620461in}%
\pgfsys@useobject{currentmarker}{}%
\end{pgfscope}%
\begin{pgfscope}%
\pgfsys@transformshift{1.477843in}{3.625877in}%
\pgfsys@useobject{currentmarker}{}%
\end{pgfscope}%
\begin{pgfscope}%
\pgfsys@transformshift{1.478004in}{3.628964in}%
\pgfsys@useobject{currentmarker}{}%
\end{pgfscope}%
\begin{pgfscope}%
\pgfsys@transformshift{1.479423in}{3.635000in}%
\pgfsys@useobject{currentmarker}{}%
\end{pgfscope}%
\begin{pgfscope}%
\pgfsys@transformshift{1.481694in}{3.644274in}%
\pgfsys@useobject{currentmarker}{}%
\end{pgfscope}%
\begin{pgfscope}%
\pgfsys@transformshift{1.482466in}{3.649468in}%
\pgfsys@useobject{currentmarker}{}%
\end{pgfscope}%
\begin{pgfscope}%
\pgfsys@transformshift{1.483045in}{3.652298in}%
\pgfsys@useobject{currentmarker}{}%
\end{pgfscope}%
\begin{pgfscope}%
\pgfsys@transformshift{1.484081in}{3.655530in}%
\pgfsys@useobject{currentmarker}{}%
\end{pgfscope}%
\begin{pgfscope}%
\pgfsys@transformshift{1.484195in}{3.661393in}%
\pgfsys@useobject{currentmarker}{}%
\end{pgfscope}%
\begin{pgfscope}%
\pgfsys@transformshift{1.484887in}{3.664544in}%
\pgfsys@useobject{currentmarker}{}%
\end{pgfscope}%
\begin{pgfscope}%
\pgfsys@transformshift{1.485526in}{3.666199in}%
\pgfsys@useobject{currentmarker}{}%
\end{pgfscope}%
\begin{pgfscope}%
\pgfsys@transformshift{1.485451in}{3.667172in}%
\pgfsys@useobject{currentmarker}{}%
\end{pgfscope}%
\begin{pgfscope}%
\pgfsys@transformshift{1.486008in}{3.669108in}%
\pgfsys@useobject{currentmarker}{}%
\end{pgfscope}%
\begin{pgfscope}%
\pgfsys@transformshift{1.486173in}{3.670203in}%
\pgfsys@useobject{currentmarker}{}%
\end{pgfscope}%
\begin{pgfscope}%
\pgfsys@transformshift{1.486121in}{3.670811in}%
\pgfsys@useobject{currentmarker}{}%
\end{pgfscope}%
\begin{pgfscope}%
\pgfsys@transformshift{1.486650in}{3.672696in}%
\pgfsys@useobject{currentmarker}{}%
\end{pgfscope}%
\begin{pgfscope}%
\pgfsys@transformshift{1.486748in}{3.673768in}%
\pgfsys@useobject{currentmarker}{}%
\end{pgfscope}%
\begin{pgfscope}%
\pgfsys@transformshift{1.486854in}{3.674351in}%
\pgfsys@useobject{currentmarker}{}%
\end{pgfscope}%
\begin{pgfscope}%
\pgfsys@transformshift{1.487770in}{3.677812in}%
\pgfsys@useobject{currentmarker}{}%
\end{pgfscope}%
\begin{pgfscope}%
\pgfsys@transformshift{1.487928in}{3.679775in}%
\pgfsys@useobject{currentmarker}{}%
\end{pgfscope}%
\begin{pgfscope}%
\pgfsys@transformshift{1.487821in}{3.680853in}%
\pgfsys@useobject{currentmarker}{}%
\end{pgfscope}%
\begin{pgfscope}%
\pgfsys@transformshift{1.488011in}{3.681417in}%
\pgfsys@useobject{currentmarker}{}%
\end{pgfscope}%
\begin{pgfscope}%
\pgfsys@transformshift{1.487972in}{3.681742in}%
\pgfsys@useobject{currentmarker}{}%
\end{pgfscope}%
\begin{pgfscope}%
\pgfsys@transformshift{1.487246in}{3.686518in}%
\pgfsys@useobject{currentmarker}{}%
\end{pgfscope}%
\begin{pgfscope}%
\pgfsys@transformshift{1.489892in}{3.694595in}%
\pgfsys@useobject{currentmarker}{}%
\end{pgfscope}%
\begin{pgfscope}%
\pgfsys@transformshift{1.489886in}{3.699270in}%
\pgfsys@useobject{currentmarker}{}%
\end{pgfscope}%
\begin{pgfscope}%
\pgfsys@transformshift{1.491996in}{3.707755in}%
\pgfsys@useobject{currentmarker}{}%
\end{pgfscope}%
\begin{pgfscope}%
\pgfsys@transformshift{1.494486in}{3.719747in}%
\pgfsys@useobject{currentmarker}{}%
\end{pgfscope}%
\begin{pgfscope}%
\pgfsys@transformshift{1.494877in}{3.732648in}%
\pgfsys@useobject{currentmarker}{}%
\end{pgfscope}%
\begin{pgfscope}%
\pgfsys@transformshift{1.489365in}{3.745078in}%
\pgfsys@useobject{currentmarker}{}%
\end{pgfscope}%
\begin{pgfscope}%
\pgfsys@transformshift{1.492981in}{3.760493in}%
\pgfsys@useobject{currentmarker}{}%
\end{pgfscope}%
\begin{pgfscope}%
\pgfsys@transformshift{1.491466in}{3.776903in}%
\pgfsys@useobject{currentmarker}{}%
\end{pgfscope}%
\begin{pgfscope}%
\pgfsys@transformshift{1.491770in}{3.785962in}%
\pgfsys@useobject{currentmarker}{}%
\end{pgfscope}%
\begin{pgfscope}%
\pgfsys@transformshift{1.493588in}{3.796404in}%
\pgfsys@useobject{currentmarker}{}%
\end{pgfscope}%
\begin{pgfscope}%
\pgfsys@transformshift{1.494072in}{3.808445in}%
\pgfsys@useobject{currentmarker}{}%
\end{pgfscope}%
\begin{pgfscope}%
\pgfsys@transformshift{1.494470in}{3.821229in}%
\pgfsys@useobject{currentmarker}{}%
\end{pgfscope}%
\begin{pgfscope}%
\pgfsys@transformshift{1.497190in}{3.827717in}%
\pgfsys@useobject{currentmarker}{}%
\end{pgfscope}%
\begin{pgfscope}%
\pgfsys@transformshift{1.495565in}{3.838156in}%
\pgfsys@useobject{currentmarker}{}%
\end{pgfscope}%
\begin{pgfscope}%
\pgfsys@transformshift{1.498250in}{3.849046in}%
\pgfsys@useobject{currentmarker}{}%
\end{pgfscope}%
\begin{pgfscope}%
\pgfsys@transformshift{1.500096in}{3.854932in}%
\pgfsys@useobject{currentmarker}{}%
\end{pgfscope}%
\begin{pgfscope}%
\pgfsys@transformshift{1.498880in}{3.863524in}%
\pgfsys@useobject{currentmarker}{}%
\end{pgfscope}%
\begin{pgfscope}%
\pgfsys@transformshift{1.500048in}{3.868152in}%
\pgfsys@useobject{currentmarker}{}%
\end{pgfscope}%
\begin{pgfscope}%
\pgfsys@transformshift{1.500819in}{3.870661in}%
\pgfsys@useobject{currentmarker}{}%
\end{pgfscope}%
\begin{pgfscope}%
\pgfsys@transformshift{1.499820in}{3.875042in}%
\pgfsys@useobject{currentmarker}{}%
\end{pgfscope}%
\begin{pgfscope}%
\pgfsys@transformshift{1.502035in}{3.880991in}%
\pgfsys@useobject{currentmarker}{}%
\end{pgfscope}%
\begin{pgfscope}%
\pgfsys@transformshift{1.501995in}{3.887991in}%
\pgfsys@useobject{currentmarker}{}%
\end{pgfscope}%
\begin{pgfscope}%
\pgfsys@transformshift{1.502126in}{3.891839in}%
\pgfsys@useobject{currentmarker}{}%
\end{pgfscope}%
\begin{pgfscope}%
\pgfsys@transformshift{1.503406in}{3.896078in}%
\pgfsys@useobject{currentmarker}{}%
\end{pgfscope}%
\begin{pgfscope}%
\pgfsys@transformshift{1.503499in}{3.898512in}%
\pgfsys@useobject{currentmarker}{}%
\end{pgfscope}%
\begin{pgfscope}%
\pgfsys@transformshift{1.502061in}{3.906592in}%
\pgfsys@useobject{currentmarker}{}%
\end{pgfscope}%
\begin{pgfscope}%
\pgfsys@transformshift{1.505705in}{3.917309in}%
\pgfsys@useobject{currentmarker}{}%
\end{pgfscope}%
\begin{pgfscope}%
\pgfsys@transformshift{1.505638in}{3.923535in}%
\pgfsys@useobject{currentmarker}{}%
\end{pgfscope}%
\begin{pgfscope}%
\pgfsys@transformshift{1.509209in}{3.933037in}%
\pgfsys@useobject{currentmarker}{}%
\end{pgfscope}%
\begin{pgfscope}%
\pgfsys@transformshift{1.510680in}{3.945253in}%
\pgfsys@useobject{currentmarker}{}%
\end{pgfscope}%
\begin{pgfscope}%
\pgfsys@transformshift{1.515146in}{3.957453in}%
\pgfsys@useobject{currentmarker}{}%
\end{pgfscope}%
\begin{pgfscope}%
\pgfsys@transformshift{1.510848in}{3.970355in}%
\pgfsys@useobject{currentmarker}{}%
\end{pgfscope}%
\begin{pgfscope}%
\pgfsys@transformshift{1.512156in}{3.977718in}%
\pgfsys@useobject{currentmarker}{}%
\end{pgfscope}%
\begin{pgfscope}%
\pgfsys@transformshift{1.510549in}{3.987435in}%
\pgfsys@useobject{currentmarker}{}%
\end{pgfscope}%
\begin{pgfscope}%
\pgfsys@transformshift{1.511909in}{3.992678in}%
\pgfsys@useobject{currentmarker}{}%
\end{pgfscope}%
\begin{pgfscope}%
\pgfsys@transformshift{1.512057in}{3.995653in}%
\pgfsys@useobject{currentmarker}{}%
\end{pgfscope}%
\begin{pgfscope}%
\pgfsys@transformshift{1.512067in}{3.997292in}%
\pgfsys@useobject{currentmarker}{}%
\end{pgfscope}%
\begin{pgfscope}%
\pgfsys@transformshift{1.512827in}{4.000921in}%
\pgfsys@useobject{currentmarker}{}%
\end{pgfscope}%
\begin{pgfscope}%
\pgfsys@transformshift{1.513008in}{4.002952in}%
\pgfsys@useobject{currentmarker}{}%
\end{pgfscope}%
\begin{pgfscope}%
\pgfsys@transformshift{1.512973in}{4.004073in}%
\pgfsys@useobject{currentmarker}{}%
\end{pgfscope}%
\begin{pgfscope}%
\pgfsys@transformshift{1.513959in}{4.006641in}%
\pgfsys@useobject{currentmarker}{}%
\end{pgfscope}%
\begin{pgfscope}%
\pgfsys@transformshift{1.513732in}{4.010015in}%
\pgfsys@useobject{currentmarker}{}%
\end{pgfscope}%
\begin{pgfscope}%
\pgfsys@transformshift{1.513399in}{4.013945in}%
\pgfsys@useobject{currentmarker}{}%
\end{pgfscope}%
\begin{pgfscope}%
\pgfsys@transformshift{1.514082in}{4.016004in}%
\pgfsys@useobject{currentmarker}{}%
\end{pgfscope}%
\begin{pgfscope}%
\pgfsys@transformshift{1.514020in}{4.019122in}%
\pgfsys@useobject{currentmarker}{}%
\end{pgfscope}%
\begin{pgfscope}%
\pgfsys@transformshift{1.515424in}{4.022627in}%
\pgfsys@useobject{currentmarker}{}%
\end{pgfscope}%
\begin{pgfscope}%
\pgfsys@transformshift{1.515388in}{4.024703in}%
\pgfsys@useobject{currentmarker}{}%
\end{pgfscope}%
\begin{pgfscope}%
\pgfsys@transformshift{1.515816in}{4.025762in}%
\pgfsys@useobject{currentmarker}{}%
\end{pgfscope}%
\begin{pgfscope}%
\pgfsys@transformshift{1.515831in}{4.026390in}%
\pgfsys@useobject{currentmarker}{}%
\end{pgfscope}%
\begin{pgfscope}%
\pgfsys@transformshift{1.515926in}{4.026722in}%
\pgfsys@useobject{currentmarker}{}%
\end{pgfscope}%
\begin{pgfscope}%
\pgfsys@transformshift{1.515917in}{4.026912in}%
\pgfsys@useobject{currentmarker}{}%
\end{pgfscope}%
\begin{pgfscope}%
\pgfsys@transformshift{1.515946in}{4.027012in}%
\pgfsys@useobject{currentmarker}{}%
\end{pgfscope}%
\begin{pgfscope}%
\pgfsys@transformshift{1.515944in}{4.027070in}%
\pgfsys@useobject{currentmarker}{}%
\end{pgfscope}%
\begin{pgfscope}%
\pgfsys@transformshift{1.515950in}{4.027101in}%
\pgfsys@useobject{currentmarker}{}%
\end{pgfscope}%
\begin{pgfscope}%
\pgfsys@transformshift{1.515949in}{4.027118in}%
\pgfsys@useobject{currentmarker}{}%
\end{pgfscope}%
\begin{pgfscope}%
\pgfsys@transformshift{1.515947in}{4.027127in}%
\pgfsys@useobject{currentmarker}{}%
\end{pgfscope}%
\begin{pgfscope}%
\pgfsys@transformshift{1.515769in}{4.028494in}%
\pgfsys@useobject{currentmarker}{}%
\end{pgfscope}%
\begin{pgfscope}%
\pgfsys@transformshift{1.515407in}{4.029159in}%
\pgfsys@useobject{currentmarker}{}%
\end{pgfscope}%
\begin{pgfscope}%
\pgfsys@transformshift{1.515206in}{4.029525in}%
\pgfsys@useobject{currentmarker}{}%
\end{pgfscope}%
\begin{pgfscope}%
\pgfsys@transformshift{1.515033in}{4.029675in}%
\pgfsys@useobject{currentmarker}{}%
\end{pgfscope}%
\begin{pgfscope}%
\pgfsys@transformshift{1.514020in}{4.030426in}%
\pgfsys@useobject{currentmarker}{}%
\end{pgfscope}%
\begin{pgfscope}%
\pgfsys@transformshift{1.511561in}{4.031405in}%
\pgfsys@useobject{currentmarker}{}%
\end{pgfscope}%
\begin{pgfscope}%
\pgfsys@transformshift{1.510141in}{4.031729in}%
\pgfsys@useobject{currentmarker}{}%
\end{pgfscope}%
\begin{pgfscope}%
\pgfsys@transformshift{1.509356in}{4.031885in}%
\pgfsys@useobject{currentmarker}{}%
\end{pgfscope}%
\begin{pgfscope}%
\pgfsys@transformshift{1.507823in}{4.031805in}%
\pgfsys@useobject{currentmarker}{}%
\end{pgfscope}%
\begin{pgfscope}%
\pgfsys@transformshift{1.505503in}{4.031787in}%
\pgfsys@useobject{currentmarker}{}%
\end{pgfscope}%
\begin{pgfscope}%
\pgfsys@transformshift{1.504228in}{4.031733in}%
\pgfsys@useobject{currentmarker}{}%
\end{pgfscope}%
\begin{pgfscope}%
\pgfsys@transformshift{1.503530in}{4.031807in}%
\pgfsys@useobject{currentmarker}{}%
\end{pgfscope}%
\begin{pgfscope}%
\pgfsys@transformshift{1.503147in}{4.031761in}%
\pgfsys@useobject{currentmarker}{}%
\end{pgfscope}%
\begin{pgfscope}%
\pgfsys@transformshift{1.502935in}{4.031763in}%
\pgfsys@useobject{currentmarker}{}%
\end{pgfscope}%
\begin{pgfscope}%
\pgfsys@transformshift{1.502820in}{4.031741in}%
\pgfsys@useobject{currentmarker}{}%
\end{pgfscope}%
\begin{pgfscope}%
\pgfsys@transformshift{1.500802in}{4.031708in}%
\pgfsys@useobject{currentmarker}{}%
\end{pgfscope}%
\begin{pgfscope}%
\pgfsys@transformshift{1.499694in}{4.031649in}%
\pgfsys@useobject{currentmarker}{}%
\end{pgfscope}%
\begin{pgfscope}%
\pgfsys@transformshift{1.496275in}{4.031991in}%
\pgfsys@useobject{currentmarker}{}%
\end{pgfscope}%
\begin{pgfscope}%
\pgfsys@transformshift{1.491823in}{4.032031in}%
\pgfsys@useobject{currentmarker}{}%
\end{pgfscope}%
\begin{pgfscope}%
\pgfsys@transformshift{1.486407in}{4.033279in}%
\pgfsys@useobject{currentmarker}{}%
\end{pgfscope}%
\begin{pgfscope}%
\pgfsys@transformshift{1.483362in}{4.033558in}%
\pgfsys@useobject{currentmarker}{}%
\end{pgfscope}%
\begin{pgfscope}%
\pgfsys@transformshift{1.478940in}{4.034139in}%
\pgfsys@useobject{currentmarker}{}%
\end{pgfscope}%
\begin{pgfscope}%
\pgfsys@transformshift{1.476489in}{4.034023in}%
\pgfsys@useobject{currentmarker}{}%
\end{pgfscope}%
\begin{pgfscope}%
\pgfsys@transformshift{1.472956in}{4.034102in}%
\pgfsys@useobject{currentmarker}{}%
\end{pgfscope}%
\begin{pgfscope}%
\pgfsys@transformshift{1.468824in}{4.033764in}%
\pgfsys@useobject{currentmarker}{}%
\end{pgfscope}%
\begin{pgfscope}%
\pgfsys@transformshift{1.463363in}{4.033749in}%
\pgfsys@useobject{currentmarker}{}%
\end{pgfscope}%
\begin{pgfscope}%
\pgfsys@transformshift{1.456653in}{4.033309in}%
\pgfsys@useobject{currentmarker}{}%
\end{pgfscope}%
\begin{pgfscope}%
\pgfsys@transformshift{1.452962in}{4.033556in}%
\pgfsys@useobject{currentmarker}{}%
\end{pgfscope}%
\begin{pgfscope}%
\pgfsys@transformshift{1.447157in}{4.033861in}%
\pgfsys@useobject{currentmarker}{}%
\end{pgfscope}%
\begin{pgfscope}%
\pgfsys@transformshift{1.437405in}{4.034233in}%
\pgfsys@useobject{currentmarker}{}%
\end{pgfscope}%
\begin{pgfscope}%
\pgfsys@transformshift{1.432038in}{4.034194in}%
\pgfsys@useobject{currentmarker}{}%
\end{pgfscope}%
\begin{pgfscope}%
\pgfsys@transformshift{1.424870in}{4.034266in}%
\pgfsys@useobject{currentmarker}{}%
\end{pgfscope}%
\begin{pgfscope}%
\pgfsys@transformshift{1.416775in}{4.035541in}%
\pgfsys@useobject{currentmarker}{}%
\end{pgfscope}%
\begin{pgfscope}%
\pgfsys@transformshift{1.407118in}{4.036398in}%
\pgfsys@useobject{currentmarker}{}%
\end{pgfscope}%
\begin{pgfscope}%
\pgfsys@transformshift{1.401793in}{4.036674in}%
\pgfsys@useobject{currentmarker}{}%
\end{pgfscope}%
\begin{pgfscope}%
\pgfsys@transformshift{1.395857in}{4.036724in}%
\pgfsys@useobject{currentmarker}{}%
\end{pgfscope}%
\begin{pgfscope}%
\pgfsys@transformshift{1.388953in}{4.036881in}%
\pgfsys@useobject{currentmarker}{}%
\end{pgfscope}%
\begin{pgfscope}%
\pgfsys@transformshift{1.379015in}{4.036323in}%
\pgfsys@useobject{currentmarker}{}%
\end{pgfscope}%
\begin{pgfscope}%
\pgfsys@transformshift{1.388284in}{4.036355in}%
\pgfsys@useobject{currentmarker}{}%
\end{pgfscope}%
\begin{pgfscope}%
\pgfsys@transformshift{1.401065in}{4.036411in}%
\pgfsys@useobject{currentmarker}{}%
\end{pgfscope}%
\begin{pgfscope}%
\pgfsys@transformshift{1.414065in}{4.039461in}%
\pgfsys@useobject{currentmarker}{}%
\end{pgfscope}%
\begin{pgfscope}%
\pgfsys@transformshift{1.429185in}{4.041624in}%
\pgfsys@useobject{currentmarker}{}%
\end{pgfscope}%
\begin{pgfscope}%
\pgfsys@transformshift{1.445441in}{4.043000in}%
\pgfsys@useobject{currentmarker}{}%
\end{pgfscope}%
\begin{pgfscope}%
\pgfsys@transformshift{1.463148in}{4.042826in}%
\pgfsys@useobject{currentmarker}{}%
\end{pgfscope}%
\begin{pgfscope}%
\pgfsys@transformshift{1.481748in}{4.042820in}%
\pgfsys@useobject{currentmarker}{}%
\end{pgfscope}%
\begin{pgfscope}%
\pgfsys@transformshift{1.502205in}{4.037143in}%
\pgfsys@useobject{currentmarker}{}%
\end{pgfscope}%
\begin{pgfscope}%
\pgfsys@transformshift{1.513825in}{4.035987in}%
\pgfsys@useobject{currentmarker}{}%
\end{pgfscope}%
\begin{pgfscope}%
\pgfsys@transformshift{1.520228in}{4.035493in}%
\pgfsys@useobject{currentmarker}{}%
\end{pgfscope}%
\begin{pgfscope}%
\pgfsys@transformshift{1.528494in}{4.037340in}%
\pgfsys@useobject{currentmarker}{}%
\end{pgfscope}%
\begin{pgfscope}%
\pgfsys@transformshift{1.540293in}{4.037698in}%
\pgfsys@useobject{currentmarker}{}%
\end{pgfscope}%
\begin{pgfscope}%
\pgfsys@transformshift{1.546775in}{4.038071in}%
\pgfsys@useobject{currentmarker}{}%
\end{pgfscope}%
\begin{pgfscope}%
\pgfsys@transformshift{1.554092in}{4.036839in}%
\pgfsys@useobject{currentmarker}{}%
\end{pgfscope}%
\begin{pgfscope}%
\pgfsys@transformshift{1.563634in}{4.036152in}%
\pgfsys@useobject{currentmarker}{}%
\end{pgfscope}%
\begin{pgfscope}%
\pgfsys@transformshift{1.575643in}{4.033960in}%
\pgfsys@useobject{currentmarker}{}%
\end{pgfscope}%
\begin{pgfscope}%
\pgfsys@transformshift{1.588791in}{4.033662in}%
\pgfsys@useobject{currentmarker}{}%
\end{pgfscope}%
\begin{pgfscope}%
\pgfsys@transformshift{1.595960in}{4.032701in}%
\pgfsys@useobject{currentmarker}{}%
\end{pgfscope}%
\begin{pgfscope}%
\pgfsys@transformshift{1.604060in}{4.033196in}%
\pgfsys@useobject{currentmarker}{}%
\end{pgfscope}%
\begin{pgfscope}%
\pgfsys@transformshift{1.615347in}{4.032878in}%
\pgfsys@useobject{currentmarker}{}%
\end{pgfscope}%
\begin{pgfscope}%
\pgfsys@transformshift{1.627306in}{4.033535in}%
\pgfsys@useobject{currentmarker}{}%
\end{pgfscope}%
\begin{pgfscope}%
\pgfsys@transformshift{1.639738in}{4.032139in}%
\pgfsys@useobject{currentmarker}{}%
\end{pgfscope}%
\begin{pgfscope}%
\pgfsys@transformshift{1.652912in}{4.030919in}%
\pgfsys@useobject{currentmarker}{}%
\end{pgfscope}%
\begin{pgfscope}%
\pgfsys@transformshift{1.668972in}{4.030410in}%
\pgfsys@useobject{currentmarker}{}%
\end{pgfscope}%
\begin{pgfscope}%
\pgfsys@transformshift{1.686081in}{4.027393in}%
\pgfsys@useobject{currentmarker}{}%
\end{pgfscope}%
\begin{pgfscope}%
\pgfsys@transformshift{1.704582in}{4.026491in}%
\pgfsys@useobject{currentmarker}{}%
\end{pgfscope}%
\begin{pgfscope}%
\pgfsys@transformshift{1.726858in}{4.023907in}%
\pgfsys@useobject{currentmarker}{}%
\end{pgfscope}%
\begin{pgfscope}%
\pgfsys@transformshift{1.739099in}{4.025420in}%
\pgfsys@useobject{currentmarker}{}%
\end{pgfscope}%
\begin{pgfscope}%
\pgfsys@transformshift{1.745879in}{4.025179in}%
\pgfsys@useobject{currentmarker}{}%
\end{pgfscope}%
\begin{pgfscope}%
\pgfsys@transformshift{1.754693in}{4.025607in}%
\pgfsys@useobject{currentmarker}{}%
\end{pgfscope}%
\begin{pgfscope}%
\pgfsys@transformshift{1.764714in}{4.024537in}%
\pgfsys@useobject{currentmarker}{}%
\end{pgfscope}%
\begin{pgfscope}%
\pgfsys@transformshift{1.775835in}{4.025325in}%
\pgfsys@useobject{currentmarker}{}%
\end{pgfscope}%
\begin{pgfscope}%
\pgfsys@transformshift{1.781903in}{4.024444in}%
\pgfsys@useobject{currentmarker}{}%
\end{pgfscope}%
\begin{pgfscope}%
\pgfsys@transformshift{1.789345in}{4.024057in}%
\pgfsys@useobject{currentmarker}{}%
\end{pgfscope}%
\begin{pgfscope}%
\pgfsys@transformshift{1.797479in}{4.023232in}%
\pgfsys@useobject{currentmarker}{}%
\end{pgfscope}%
\begin{pgfscope}%
\pgfsys@transformshift{1.801967in}{4.023511in}%
\pgfsys@useobject{currentmarker}{}%
\end{pgfscope}%
\begin{pgfscope}%
\pgfsys@transformshift{1.804438in}{4.023402in}%
\pgfsys@useobject{currentmarker}{}%
\end{pgfscope}%
\begin{pgfscope}%
\pgfsys@transformshift{1.810029in}{4.023723in}%
\pgfsys@useobject{currentmarker}{}%
\end{pgfscope}%
\begin{pgfscope}%
\pgfsys@transformshift{1.816443in}{4.021667in}%
\pgfsys@useobject{currentmarker}{}%
\end{pgfscope}%
\begin{pgfscope}%
\pgfsys@transformshift{1.825300in}{4.019911in}%
\pgfsys@useobject{currentmarker}{}%
\end{pgfscope}%
\begin{pgfscope}%
\pgfsys@transformshift{1.839842in}{4.016999in}%
\pgfsys@useobject{currentmarker}{}%
\end{pgfscope}%
\begin{pgfscope}%
\pgfsys@transformshift{1.847896in}{4.015706in}%
\pgfsys@useobject{currentmarker}{}%
\end{pgfscope}%
\begin{pgfscope}%
\pgfsys@transformshift{1.852324in}{4.014987in}%
\pgfsys@useobject{currentmarker}{}%
\end{pgfscope}%
\begin{pgfscope}%
\pgfsys@transformshift{1.857830in}{4.014941in}%
\pgfsys@useobject{currentmarker}{}%
\end{pgfscope}%
\begin{pgfscope}%
\pgfsys@transformshift{1.864526in}{4.014401in}%
\pgfsys@useobject{currentmarker}{}%
\end{pgfscope}%
\begin{pgfscope}%
\pgfsys@transformshift{1.868206in}{4.014728in}%
\pgfsys@useobject{currentmarker}{}%
\end{pgfscope}%
\begin{pgfscope}%
\pgfsys@transformshift{1.873343in}{4.013813in}%
\pgfsys@useobject{currentmarker}{}%
\end{pgfscope}%
\begin{pgfscope}%
\pgfsys@transformshift{1.879606in}{4.013074in}%
\pgfsys@useobject{currentmarker}{}%
\end{pgfscope}%
\begin{pgfscope}%
\pgfsys@transformshift{1.889603in}{4.011208in}%
\pgfsys@useobject{currentmarker}{}%
\end{pgfscope}%
\begin{pgfscope}%
\pgfsys@transformshift{1.900705in}{4.010294in}%
\pgfsys@useobject{currentmarker}{}%
\end{pgfscope}%
\begin{pgfscope}%
\pgfsys@transformshift{1.912624in}{4.008197in}%
\pgfsys@useobject{currentmarker}{}%
\end{pgfscope}%
\begin{pgfscope}%
\pgfsys@transformshift{1.927012in}{4.006307in}%
\pgfsys@useobject{currentmarker}{}%
\end{pgfscope}%
\begin{pgfscope}%
\pgfsys@transformshift{1.942378in}{4.001087in}%
\pgfsys@useobject{currentmarker}{}%
\end{pgfscope}%
\begin{pgfscope}%
\pgfsys@transformshift{1.959542in}{3.997474in}%
\pgfsys@useobject{currentmarker}{}%
\end{pgfscope}%
\begin{pgfscope}%
\pgfsys@transformshift{1.977665in}{3.993635in}%
\pgfsys@useobject{currentmarker}{}%
\end{pgfscope}%
\begin{pgfscope}%
\pgfsys@transformshift{1.999406in}{3.989507in}%
\pgfsys@useobject{currentmarker}{}%
\end{pgfscope}%
\begin{pgfscope}%
\pgfsys@transformshift{2.022270in}{3.985839in}%
\pgfsys@useobject{currentmarker}{}%
\end{pgfscope}%
\begin{pgfscope}%
\pgfsys@transformshift{2.047820in}{3.984161in}%
\pgfsys@useobject{currentmarker}{}%
\end{pgfscope}%
\begin{pgfscope}%
\pgfsys@transformshift{2.074510in}{3.982651in}%
\pgfsys@useobject{currentmarker}{}%
\end{pgfscope}%
\begin{pgfscope}%
\pgfsys@transformshift{2.102135in}{3.980950in}%
\pgfsys@useobject{currentmarker}{}%
\end{pgfscope}%
\begin{pgfscope}%
\pgfsys@transformshift{2.132017in}{3.982235in}%
\pgfsys@useobject{currentmarker}{}%
\end{pgfscope}%
\begin{pgfscope}%
\pgfsys@transformshift{2.148213in}{3.979351in}%
\pgfsys@useobject{currentmarker}{}%
\end{pgfscope}%
\begin{pgfscope}%
\pgfsys@transformshift{2.166790in}{3.979588in}%
\pgfsys@useobject{currentmarker}{}%
\end{pgfscope}%
\begin{pgfscope}%
\pgfsys@transformshift{2.186673in}{3.976779in}%
\pgfsys@useobject{currentmarker}{}%
\end{pgfscope}%
\begin{pgfscope}%
\pgfsys@transformshift{2.208406in}{3.975035in}%
\pgfsys@useobject{currentmarker}{}%
\end{pgfscope}%
\begin{pgfscope}%
\pgfsys@transformshift{2.230994in}{3.972542in}%
\pgfsys@useobject{currentmarker}{}%
\end{pgfscope}%
\begin{pgfscope}%
\pgfsys@transformshift{2.255877in}{3.969170in}%
\pgfsys@useobject{currentmarker}{}%
\end{pgfscope}%
\begin{pgfscope}%
\pgfsys@transformshift{2.281536in}{3.965644in}%
\pgfsys@useobject{currentmarker}{}%
\end{pgfscope}%
\begin{pgfscope}%
\pgfsys@transformshift{2.310295in}{3.960984in}%
\pgfsys@useobject{currentmarker}{}%
\end{pgfscope}%
\begin{pgfscope}%
\pgfsys@transformshift{2.339943in}{3.953168in}%
\pgfsys@useobject{currentmarker}{}%
\end{pgfscope}%
\begin{pgfscope}%
\pgfsys@transformshift{2.370778in}{3.948069in}%
\pgfsys@useobject{currentmarker}{}%
\end{pgfscope}%
\begin{pgfscope}%
\pgfsys@transformshift{2.387408in}{3.943717in}%
\pgfsys@useobject{currentmarker}{}%
\end{pgfscope}%
\begin{pgfscope}%
\pgfsys@transformshift{2.409227in}{3.940136in}%
\pgfsys@useobject{currentmarker}{}%
\end{pgfscope}%
\begin{pgfscope}%
\pgfsys@transformshift{2.431913in}{3.934681in}%
\pgfsys@useobject{currentmarker}{}%
\end{pgfscope}%
\begin{pgfscope}%
\pgfsys@transformshift{2.457762in}{3.929822in}%
\pgfsys@useobject{currentmarker}{}%
\end{pgfscope}%
\begin{pgfscope}%
\pgfsys@transformshift{2.484905in}{3.923779in}%
\pgfsys@useobject{currentmarker}{}%
\end{pgfscope}%
\begin{pgfscope}%
\pgfsys@transformshift{2.499981in}{3.921206in}%
\pgfsys@useobject{currentmarker}{}%
\end{pgfscope}%
\begin{pgfscope}%
\pgfsys@transformshift{2.516047in}{3.916263in}%
\pgfsys@useobject{currentmarker}{}%
\end{pgfscope}%
\begin{pgfscope}%
\pgfsys@transformshift{2.533999in}{3.914009in}%
\pgfsys@useobject{currentmarker}{}%
\end{pgfscope}%
\begin{pgfscope}%
\pgfsys@transformshift{2.543830in}{3.912463in}%
\pgfsys@useobject{currentmarker}{}%
\end{pgfscope}%
\begin{pgfscope}%
\pgfsys@transformshift{2.556998in}{3.911584in}%
\pgfsys@useobject{currentmarker}{}%
\end{pgfscope}%
\begin{pgfscope}%
\pgfsys@transformshift{2.564110in}{3.910130in}%
\pgfsys@useobject{currentmarker}{}%
\end{pgfscope}%
\begin{pgfscope}%
\pgfsys@transformshift{2.571843in}{3.909305in}%
\pgfsys@useobject{currentmarker}{}%
\end{pgfscope}%
\begin{pgfscope}%
\pgfsys@transformshift{2.580176in}{3.908011in}%
\pgfsys@useobject{currentmarker}{}%
\end{pgfscope}%
\begin{pgfscope}%
\pgfsys@transformshift{2.591277in}{3.906714in}%
\pgfsys@useobject{currentmarker}{}%
\end{pgfscope}%
\begin{pgfscope}%
\pgfsys@transformshift{2.603185in}{3.905583in}%
\pgfsys@useobject{currentmarker}{}%
\end{pgfscope}%
\begin{pgfscope}%
\pgfsys@transformshift{2.616339in}{3.904249in}%
\pgfsys@useobject{currentmarker}{}%
\end{pgfscope}%
\begin{pgfscope}%
\pgfsys@transformshift{2.623458in}{3.902765in}%
\pgfsys@useobject{currentmarker}{}%
\end{pgfscope}%
\begin{pgfscope}%
\pgfsys@transformshift{2.632589in}{3.902382in}%
\pgfsys@useobject{currentmarker}{}%
\end{pgfscope}%
\begin{pgfscope}%
\pgfsys@transformshift{2.643075in}{3.901299in}%
\pgfsys@useobject{currentmarker}{}%
\end{pgfscope}%
\begin{pgfscope}%
\pgfsys@transformshift{2.648852in}{3.900812in}%
\pgfsys@useobject{currentmarker}{}%
\end{pgfscope}%
\begin{pgfscope}%
\pgfsys@transformshift{2.655543in}{3.900552in}%
\pgfsys@useobject{currentmarker}{}%
\end{pgfscope}%
\begin{pgfscope}%
\pgfsys@transformshift{2.659224in}{3.900450in}%
\pgfsys@useobject{currentmarker}{}%
\end{pgfscope}%
\begin{pgfscope}%
\pgfsys@transformshift{2.664024in}{3.899991in}%
\pgfsys@useobject{currentmarker}{}%
\end{pgfscope}%
\begin{pgfscope}%
\pgfsys@transformshift{2.670254in}{3.899987in}%
\pgfsys@useobject{currentmarker}{}%
\end{pgfscope}%
\begin{pgfscope}%
\pgfsys@transformshift{2.678405in}{3.899146in}%
\pgfsys@useobject{currentmarker}{}%
\end{pgfscope}%
\begin{pgfscope}%
\pgfsys@transformshift{2.687802in}{3.898164in}%
\pgfsys@useobject{currentmarker}{}%
\end{pgfscope}%
\begin{pgfscope}%
\pgfsys@transformshift{2.698396in}{3.897004in}%
\pgfsys@useobject{currentmarker}{}%
\end{pgfscope}%
\begin{pgfscope}%
\pgfsys@transformshift{2.714541in}{3.895106in}%
\pgfsys@useobject{currentmarker}{}%
\end{pgfscope}%
\begin{pgfscope}%
\pgfsys@transformshift{2.732090in}{3.895771in}%
\pgfsys@useobject{currentmarker}{}%
\end{pgfscope}%
\begin{pgfscope}%
\pgfsys@transformshift{2.741731in}{3.895177in}%
\pgfsys@useobject{currentmarker}{}%
\end{pgfscope}%
\begin{pgfscope}%
\pgfsys@transformshift{2.747041in}{3.895001in}%
\pgfsys@useobject{currentmarker}{}%
\end{pgfscope}%
\begin{pgfscope}%
\pgfsys@transformshift{2.754291in}{3.894263in}%
\pgfsys@useobject{currentmarker}{}%
\end{pgfscope}%
\begin{pgfscope}%
\pgfsys@transformshift{2.763458in}{3.895636in}%
\pgfsys@useobject{currentmarker}{}%
\end{pgfscope}%
\begin{pgfscope}%
\pgfsys@transformshift{2.774936in}{3.895332in}%
\pgfsys@useobject{currentmarker}{}%
\end{pgfscope}%
\begin{pgfscope}%
\pgfsys@transformshift{2.781216in}{3.895996in}%
\pgfsys@useobject{currentmarker}{}%
\end{pgfscope}%
\begin{pgfscope}%
\pgfsys@transformshift{2.788543in}{3.895358in}%
\pgfsys@useobject{currentmarker}{}%
\end{pgfscope}%
\begin{pgfscope}%
\pgfsys@transformshift{2.792583in}{3.895167in}%
\pgfsys@useobject{currentmarker}{}%
\end{pgfscope}%
\begin{pgfscope}%
\pgfsys@transformshift{2.798843in}{3.895935in}%
\pgfsys@useobject{currentmarker}{}%
\end{pgfscope}%
\begin{pgfscope}%
\pgfsys@transformshift{2.802305in}{3.895739in}%
\pgfsys@useobject{currentmarker}{}%
\end{pgfscope}%
\begin{pgfscope}%
\pgfsys@transformshift{2.804211in}{3.895653in}%
\pgfsys@useobject{currentmarker}{}%
\end{pgfscope}%
\begin{pgfscope}%
\pgfsys@transformshift{2.808437in}{3.895571in}%
\pgfsys@useobject{currentmarker}{}%
\end{pgfscope}%
\begin{pgfscope}%
\pgfsys@transformshift{2.813354in}{3.896154in}%
\pgfsys@useobject{currentmarker}{}%
\end{pgfscope}%
\begin{pgfscope}%
\pgfsys@transformshift{2.816078in}{3.896156in}%
\pgfsys@useobject{currentmarker}{}%
\end{pgfscope}%
\begin{pgfscope}%
\pgfsys@transformshift{2.821226in}{3.896264in}%
\pgfsys@useobject{currentmarker}{}%
\end{pgfscope}%
\begin{pgfscope}%
\pgfsys@transformshift{2.824043in}{3.895974in}%
\pgfsys@useobject{currentmarker}{}%
\end{pgfscope}%
\begin{pgfscope}%
\pgfsys@transformshift{2.825600in}{3.895911in}%
\pgfsys@useobject{currentmarker}{}%
\end{pgfscope}%
\begin{pgfscope}%
\pgfsys@transformshift{2.829287in}{3.895313in}%
\pgfsys@useobject{currentmarker}{}%
\end{pgfscope}%
\begin{pgfscope}%
\pgfsys@transformshift{2.831340in}{3.895364in}%
\pgfsys@useobject{currentmarker}{}%
\end{pgfscope}%
\begin{pgfscope}%
\pgfsys@transformshift{2.834065in}{3.895283in}%
\pgfsys@useobject{currentmarker}{}%
\end{pgfscope}%
\begin{pgfscope}%
\pgfsys@transformshift{2.838478in}{3.895465in}%
\pgfsys@useobject{currentmarker}{}%
\end{pgfscope}%
\begin{pgfscope}%
\pgfsys@transformshift{2.840896in}{3.895236in}%
\pgfsys@useobject{currentmarker}{}%
\end{pgfscope}%
\begin{pgfscope}%
\pgfsys@transformshift{2.845001in}{3.895262in}%
\pgfsys@useobject{currentmarker}{}%
\end{pgfscope}%
\begin{pgfscope}%
\pgfsys@transformshift{2.853505in}{3.893511in}%
\pgfsys@useobject{currentmarker}{}%
\end{pgfscope}%
\begin{pgfscope}%
\pgfsys@transformshift{2.863245in}{3.894476in}%
\pgfsys@useobject{currentmarker}{}%
\end{pgfscope}%
\begin{pgfscope}%
\pgfsys@transformshift{2.868615in}{3.894105in}%
\pgfsys@useobject{currentmarker}{}%
\end{pgfscope}%
\begin{pgfscope}%
\pgfsys@transformshift{2.875847in}{3.893950in}%
\pgfsys@useobject{currentmarker}{}%
\end{pgfscope}%
\begin{pgfscope}%
\pgfsys@transformshift{2.879805in}{3.893543in}%
\pgfsys@useobject{currentmarker}{}%
\end{pgfscope}%
\begin{pgfscope}%
\pgfsys@transformshift{2.885359in}{3.893431in}%
\pgfsys@useobject{currentmarker}{}%
\end{pgfscope}%
\begin{pgfscope}%
\pgfsys@transformshift{2.894918in}{3.893216in}%
\pgfsys@useobject{currentmarker}{}%
\end{pgfscope}%
\begin{pgfscope}%
\pgfsys@transformshift{2.906403in}{3.896335in}%
\pgfsys@useobject{currentmarker}{}%
\end{pgfscope}%
\begin{pgfscope}%
\pgfsys@transformshift{2.919934in}{3.896784in}%
\pgfsys@useobject{currentmarker}{}%
\end{pgfscope}%
\begin{pgfscope}%
\pgfsys@transformshift{2.934720in}{3.898628in}%
\pgfsys@useobject{currentmarker}{}%
\end{pgfscope}%
\begin{pgfscope}%
\pgfsys@transformshift{2.950550in}{3.897972in}%
\pgfsys@useobject{currentmarker}{}%
\end{pgfscope}%
\begin{pgfscope}%
\pgfsys@transformshift{2.967265in}{3.898045in}%
\pgfsys@useobject{currentmarker}{}%
\end{pgfscope}%
\begin{pgfscope}%
\pgfsys@transformshift{2.988131in}{3.897220in}%
\pgfsys@useobject{currentmarker}{}%
\end{pgfscope}%
\begin{pgfscope}%
\pgfsys@transformshift{3.010394in}{3.902609in}%
\pgfsys@useobject{currentmarker}{}%
\end{pgfscope}%
\begin{pgfscope}%
\pgfsys@transformshift{3.035532in}{3.903278in}%
\pgfsys@useobject{currentmarker}{}%
\end{pgfscope}%
\begin{pgfscope}%
\pgfsys@transformshift{3.049305in}{3.904551in}%
\pgfsys@useobject{currentmarker}{}%
\end{pgfscope}%
\begin{pgfscope}%
\pgfsys@transformshift{3.063846in}{3.902803in}%
\pgfsys@useobject{currentmarker}{}%
\end{pgfscope}%
\begin{pgfscope}%
\pgfsys@transformshift{3.071887in}{3.902320in}%
\pgfsys@useobject{currentmarker}{}%
\end{pgfscope}%
\begin{pgfscope}%
\pgfsys@transformshift{3.083475in}{3.901391in}%
\pgfsys@useobject{currentmarker}{}%
\end{pgfscope}%
\begin{pgfscope}%
\pgfsys@transformshift{3.089862in}{3.901693in}%
\pgfsys@useobject{currentmarker}{}%
\end{pgfscope}%
\begin{pgfscope}%
\pgfsys@transformshift{3.098788in}{3.901326in}%
\pgfsys@useobject{currentmarker}{}%
\end{pgfscope}%
\begin{pgfscope}%
\pgfsys@transformshift{3.103697in}{3.901115in}%
\pgfsys@useobject{currentmarker}{}%
\end{pgfscope}%
\begin{pgfscope}%
\pgfsys@transformshift{3.110374in}{3.899756in}%
\pgfsys@useobject{currentmarker}{}%
\end{pgfscope}%
\begin{pgfscope}%
\pgfsys@transformshift{3.121195in}{3.898448in}%
\pgfsys@useobject{currentmarker}{}%
\end{pgfscope}%
\begin{pgfscope}%
\pgfsys@transformshift{3.132936in}{3.896773in}%
\pgfsys@useobject{currentmarker}{}%
\end{pgfscope}%
\begin{pgfscope}%
\pgfsys@transformshift{3.145883in}{3.894771in}%
\pgfsys@useobject{currentmarker}{}%
\end{pgfscope}%
\begin{pgfscope}%
\pgfsys@transformshift{3.160889in}{3.894064in}%
\pgfsys@useobject{currentmarker}{}%
\end{pgfscope}%
\begin{pgfscope}%
\pgfsys@transformshift{3.178708in}{3.892790in}%
\pgfsys@useobject{currentmarker}{}%
\end{pgfscope}%
\begin{pgfscope}%
\pgfsys@transformshift{3.200824in}{3.891848in}%
\pgfsys@useobject{currentmarker}{}%
\end{pgfscope}%
\begin{pgfscope}%
\pgfsys@transformshift{3.224635in}{3.887515in}%
\pgfsys@useobject{currentmarker}{}%
\end{pgfscope}%
\begin{pgfscope}%
\pgfsys@transformshift{3.249913in}{3.885226in}%
\pgfsys@useobject{currentmarker}{}%
\end{pgfscope}%
\begin{pgfscope}%
\pgfsys@transformshift{3.275984in}{3.883396in}%
\pgfsys@useobject{currentmarker}{}%
\end{pgfscope}%
\begin{pgfscope}%
\pgfsys@transformshift{3.302665in}{3.882827in}%
\pgfsys@useobject{currentmarker}{}%
\end{pgfscope}%
\begin{pgfscope}%
\pgfsys@transformshift{3.317343in}{3.882901in}%
\pgfsys@useobject{currentmarker}{}%
\end{pgfscope}%
\begin{pgfscope}%
\pgfsys@transformshift{3.325409in}{3.882546in}%
\pgfsys@useobject{currentmarker}{}%
\end{pgfscope}%
\begin{pgfscope}%
\pgfsys@transformshift{3.334481in}{3.883248in}%
\pgfsys@useobject{currentmarker}{}%
\end{pgfscope}%
\begin{pgfscope}%
\pgfsys@transformshift{3.344297in}{3.882886in}%
\pgfsys@useobject{currentmarker}{}%
\end{pgfscope}%
\begin{pgfscope}%
\pgfsys@transformshift{3.354646in}{3.882536in}%
\pgfsys@useobject{currentmarker}{}%
\end{pgfscope}%
\begin{pgfscope}%
\pgfsys@transformshift{3.360191in}{3.881238in}%
\pgfsys@useobject{currentmarker}{}%
\end{pgfscope}%
\begin{pgfscope}%
\pgfsys@transformshift{3.370575in}{3.882053in}%
\pgfsys@useobject{currentmarker}{}%
\end{pgfscope}%
\begin{pgfscope}%
\pgfsys@transformshift{3.376249in}{3.881255in}%
\pgfsys@useobject{currentmarker}{}%
\end{pgfscope}%
\begin{pgfscope}%
\pgfsys@transformshift{3.383630in}{3.882430in}%
\pgfsys@useobject{currentmarker}{}%
\end{pgfscope}%
\begin{pgfscope}%
\pgfsys@transformshift{3.391932in}{3.881100in}%
\pgfsys@useobject{currentmarker}{}%
\end{pgfscope}%
\begin{pgfscope}%
\pgfsys@transformshift{3.401916in}{3.880800in}%
\pgfsys@useobject{currentmarker}{}%
\end{pgfscope}%
\begin{pgfscope}%
\pgfsys@transformshift{3.412179in}{3.878615in}%
\pgfsys@useobject{currentmarker}{}%
\end{pgfscope}%
\begin{pgfscope}%
\pgfsys@transformshift{3.424445in}{3.876726in}%
\pgfsys@useobject{currentmarker}{}%
\end{pgfscope}%
\begin{pgfscope}%
\pgfsys@transformshift{3.437524in}{3.874765in}%
\pgfsys@useobject{currentmarker}{}%
\end{pgfscope}%
\begin{pgfscope}%
\pgfsys@transformshift{3.453349in}{3.874225in}%
\pgfsys@useobject{currentmarker}{}%
\end{pgfscope}%
\begin{pgfscope}%
\pgfsys@transformshift{3.469806in}{3.875052in}%
\pgfsys@useobject{currentmarker}{}%
\end{pgfscope}%
\begin{pgfscope}%
\pgfsys@transformshift{3.486768in}{3.873181in}%
\pgfsys@useobject{currentmarker}{}%
\end{pgfscope}%
\begin{pgfscope}%
\pgfsys@transformshift{3.504823in}{3.875573in}%
\pgfsys@useobject{currentmarker}{}%
\end{pgfscope}%
\begin{pgfscope}%
\pgfsys@transformshift{3.514556in}{3.873206in}%
\pgfsys@useobject{currentmarker}{}%
\end{pgfscope}%
\begin{pgfscope}%
\pgfsys@transformshift{3.525570in}{3.872674in}%
\pgfsys@useobject{currentmarker}{}%
\end{pgfscope}%
\begin{pgfscope}%
\pgfsys@transformshift{3.539066in}{3.870879in}%
\pgfsys@useobject{currentmarker}{}%
\end{pgfscope}%
\begin{pgfscope}%
\pgfsys@transformshift{3.553561in}{3.870837in}%
\pgfsys@useobject{currentmarker}{}%
\end{pgfscope}%
\begin{pgfscope}%
\pgfsys@transformshift{3.561507in}{3.870195in}%
\pgfsys@useobject{currentmarker}{}%
\end{pgfscope}%
\begin{pgfscope}%
\pgfsys@transformshift{3.571627in}{3.869504in}%
\pgfsys@useobject{currentmarker}{}%
\end{pgfscope}%
\begin{pgfscope}%
\pgfsys@transformshift{3.582356in}{3.867354in}%
\pgfsys@useobject{currentmarker}{}%
\end{pgfscope}%
\begin{pgfscope}%
\pgfsys@transformshift{3.588264in}{3.866210in}%
\pgfsys@useobject{currentmarker}{}%
\end{pgfscope}%
\begin{pgfscope}%
\pgfsys@transformshift{3.596385in}{3.864930in}%
\pgfsys@useobject{currentmarker}{}%
\end{pgfscope}%
\begin{pgfscope}%
\pgfsys@transformshift{3.600775in}{3.863847in}%
\pgfsys@useobject{currentmarker}{}%
\end{pgfscope}%
\begin{pgfscope}%
\pgfsys@transformshift{3.603243in}{3.863539in}%
\pgfsys@useobject{currentmarker}{}%
\end{pgfscope}%
\begin{pgfscope}%
\pgfsys@transformshift{3.606520in}{3.863234in}%
\pgfsys@useobject{currentmarker}{}%
\end{pgfscope}%
\begin{pgfscope}%
\pgfsys@transformshift{3.610560in}{3.863284in}%
\pgfsys@useobject{currentmarker}{}%
\end{pgfscope}%
\begin{pgfscope}%
\pgfsys@transformshift{3.616742in}{3.863254in}%
\pgfsys@useobject{currentmarker}{}%
\end{pgfscope}%
\begin{pgfscope}%
\pgfsys@transformshift{3.625309in}{3.864724in}%
\pgfsys@useobject{currentmarker}{}%
\end{pgfscope}%
\begin{pgfscope}%
\pgfsys@transformshift{3.637357in}{3.864676in}%
\pgfsys@useobject{currentmarker}{}%
\end{pgfscope}%
\begin{pgfscope}%
\pgfsys@transformshift{3.650982in}{3.865671in}%
\pgfsys@useobject{currentmarker}{}%
\end{pgfscope}%
\begin{pgfscope}%
\pgfsys@transformshift{3.658400in}{3.864473in}%
\pgfsys@useobject{currentmarker}{}%
\end{pgfscope}%
\begin{pgfscope}%
\pgfsys@transformshift{3.666683in}{3.865830in}%
\pgfsys@useobject{currentmarker}{}%
\end{pgfscope}%
\begin{pgfscope}%
\pgfsys@transformshift{3.675757in}{3.865924in}%
\pgfsys@useobject{currentmarker}{}%
\end{pgfscope}%
\begin{pgfscope}%
\pgfsys@transformshift{3.685318in}{3.867017in}%
\pgfsys@useobject{currentmarker}{}%
\end{pgfscope}%
\begin{pgfscope}%
\pgfsys@transformshift{3.696229in}{3.867457in}%
\pgfsys@useobject{currentmarker}{}%
\end{pgfscope}%
\begin{pgfscope}%
\pgfsys@transformshift{3.707831in}{3.867654in}%
\pgfsys@useobject{currentmarker}{}%
\end{pgfscope}%
\begin{pgfscope}%
\pgfsys@transformshift{3.720082in}{3.867105in}%
\pgfsys@useobject{currentmarker}{}%
\end{pgfscope}%
\begin{pgfscope}%
\pgfsys@transformshift{3.726825in}{3.867250in}%
\pgfsys@useobject{currentmarker}{}%
\end{pgfscope}%
\begin{pgfscope}%
\pgfsys@transformshift{3.734511in}{3.866680in}%
\pgfsys@useobject{currentmarker}{}%
\end{pgfscope}%
\begin{pgfscope}%
\pgfsys@transformshift{3.743250in}{3.866814in}%
\pgfsys@useobject{currentmarker}{}%
\end{pgfscope}%
\begin{pgfscope}%
\pgfsys@transformshift{3.748010in}{3.866148in}%
\pgfsys@useobject{currentmarker}{}%
\end{pgfscope}%
\begin{pgfscope}%
\pgfsys@transformshift{3.753777in}{3.865732in}%
\pgfsys@useobject{currentmarker}{}%
\end{pgfscope}%
\begin{pgfscope}%
\pgfsys@transformshift{3.756938in}{3.865379in}%
\pgfsys@useobject{currentmarker}{}%
\end{pgfscope}%
\begin{pgfscope}%
\pgfsys@transformshift{3.762491in}{3.865268in}%
\pgfsys@useobject{currentmarker}{}%
\end{pgfscope}%
\begin{pgfscope}%
\pgfsys@transformshift{3.768733in}{3.865068in}%
\pgfsys@useobject{currentmarker}{}%
\end{pgfscope}%
\begin{pgfscope}%
\pgfsys@transformshift{3.772154in}{3.865368in}%
\pgfsys@useobject{currentmarker}{}%
\end{pgfscope}%
\begin{pgfscope}%
\pgfsys@transformshift{3.776300in}{3.865186in}%
\pgfsys@useobject{currentmarker}{}%
\end{pgfscope}%
\begin{pgfscope}%
\pgfsys@transformshift{3.781255in}{3.865098in}%
\pgfsys@useobject{currentmarker}{}%
\end{pgfscope}%
\begin{pgfscope}%
\pgfsys@transformshift{3.787736in}{3.864731in}%
\pgfsys@useobject{currentmarker}{}%
\end{pgfscope}%
\begin{pgfscope}%
\pgfsys@transformshift{3.795570in}{3.864619in}%
\pgfsys@useobject{currentmarker}{}%
\end{pgfscope}%
\begin{pgfscope}%
\pgfsys@transformshift{3.806304in}{3.864607in}%
\pgfsys@useobject{currentmarker}{}%
\end{pgfscope}%
\begin{pgfscope}%
\pgfsys@transformshift{3.817532in}{3.866283in}%
\pgfsys@useobject{currentmarker}{}%
\end{pgfscope}%
\begin{pgfscope}%
\pgfsys@transformshift{3.829621in}{3.866674in}%
\pgfsys@useobject{currentmarker}{}%
\end{pgfscope}%
\begin{pgfscope}%
\pgfsys@transformshift{3.842762in}{3.867578in}%
\pgfsys@useobject{currentmarker}{}%
\end{pgfscope}%
\begin{pgfscope}%
\pgfsys@transformshift{3.849993in}{3.867118in}%
\pgfsys@useobject{currentmarker}{}%
\end{pgfscope}%
\begin{pgfscope}%
\pgfsys@transformshift{3.859148in}{3.867213in}%
\pgfsys@useobject{currentmarker}{}%
\end{pgfscope}%
\begin{pgfscope}%
\pgfsys@transformshift{3.872005in}{3.867246in}%
\pgfsys@useobject{currentmarker}{}%
\end{pgfscope}%
\begin{pgfscope}%
\pgfsys@transformshift{3.886079in}{3.867515in}%
\pgfsys@useobject{currentmarker}{}%
\end{pgfscope}%
\begin{pgfscope}%
\pgfsys@transformshift{3.893812in}{3.867153in}%
\pgfsys@useobject{currentmarker}{}%
\end{pgfscope}%
\begin{pgfscope}%
\pgfsys@transformshift{3.902599in}{3.867210in}%
\pgfsys@useobject{currentmarker}{}%
\end{pgfscope}%
\begin{pgfscope}%
\pgfsys@transformshift{3.907422in}{3.866890in}%
\pgfsys@useobject{currentmarker}{}%
\end{pgfscope}%
\begin{pgfscope}%
\pgfsys@transformshift{3.914133in}{3.866671in}%
\pgfsys@useobject{currentmarker}{}%
\end{pgfscope}%
\begin{pgfscope}%
\pgfsys@transformshift{3.924088in}{3.866706in}%
\pgfsys@useobject{currentmarker}{}%
\end{pgfscope}%
\begin{pgfscope}%
\pgfsys@transformshift{3.935295in}{3.867083in}%
\pgfsys@useobject{currentmarker}{}%
\end{pgfscope}%
\begin{pgfscope}%
\pgfsys@transformshift{3.947164in}{3.866898in}%
\pgfsys@useobject{currentmarker}{}%
\end{pgfscope}%
\begin{pgfscope}%
\pgfsys@transformshift{3.959531in}{3.868144in}%
\pgfsys@useobject{currentmarker}{}%
\end{pgfscope}%
\begin{pgfscope}%
\pgfsys@transformshift{3.966364in}{3.867957in}%
\pgfsys@useobject{currentmarker}{}%
\end{pgfscope}%
\begin{pgfscope}%
\pgfsys@transformshift{3.976041in}{3.868013in}%
\pgfsys@useobject{currentmarker}{}%
\end{pgfscope}%
\begin{pgfscope}%
\pgfsys@transformshift{3.990088in}{3.867295in}%
\pgfsys@useobject{currentmarker}{}%
\end{pgfscope}%
\begin{pgfscope}%
\pgfsys@transformshift{4.006613in}{3.869474in}%
\pgfsys@useobject{currentmarker}{}%
\end{pgfscope}%
\begin{pgfscope}%
\pgfsys@transformshift{4.024064in}{3.868610in}%
\pgfsys@useobject{currentmarker}{}%
\end{pgfscope}%
\begin{pgfscope}%
\pgfsys@transformshift{4.042218in}{3.870768in}%
\pgfsys@useobject{currentmarker}{}%
\end{pgfscope}%
\begin{pgfscope}%
\pgfsys@transformshift{4.052205in}{3.869603in}%
\pgfsys@useobject{currentmarker}{}%
\end{pgfscope}%
\begin{pgfscope}%
\pgfsys@transformshift{4.063717in}{3.869146in}%
\pgfsys@useobject{currentmarker}{}%
\end{pgfscope}%
\begin{pgfscope}%
\pgfsys@transformshift{4.078688in}{3.868158in}%
\pgfsys@useobject{currentmarker}{}%
\end{pgfscope}%
\begin{pgfscope}%
\pgfsys@transformshift{4.094187in}{3.871099in}%
\pgfsys@useobject{currentmarker}{}%
\end{pgfscope}%
\begin{pgfscope}%
\pgfsys@transformshift{4.102864in}{3.871082in}%
\pgfsys@useobject{currentmarker}{}%
\end{pgfscope}%
\begin{pgfscope}%
\pgfsys@transformshift{4.112976in}{3.872267in}%
\pgfsys@useobject{currentmarker}{}%
\end{pgfscope}%
\begin{pgfscope}%
\pgfsys@transformshift{4.124208in}{3.871833in}%
\pgfsys@useobject{currentmarker}{}%
\end{pgfscope}%
\begin{pgfscope}%
\pgfsys@transformshift{4.136430in}{3.871655in}%
\pgfsys@useobject{currentmarker}{}%
\end{pgfscope}%
\begin{pgfscope}%
\pgfsys@transformshift{4.149692in}{3.869918in}%
\pgfsys@useobject{currentmarker}{}%
\end{pgfscope}%
\begin{pgfscope}%
\pgfsys@transformshift{4.164655in}{3.869620in}%
\pgfsys@useobject{currentmarker}{}%
\end{pgfscope}%
\begin{pgfscope}%
\pgfsys@transformshift{4.181151in}{3.867959in}%
\pgfsys@useobject{currentmarker}{}%
\end{pgfscope}%
\begin{pgfscope}%
\pgfsys@transformshift{4.190227in}{3.868845in}%
\pgfsys@useobject{currentmarker}{}%
\end{pgfscope}%
\begin{pgfscope}%
\pgfsys@transformshift{4.201183in}{3.867530in}%
\pgfsys@useobject{currentmarker}{}%
\end{pgfscope}%
\begin{pgfscope}%
\pgfsys@transformshift{4.213533in}{3.867281in}%
\pgfsys@useobject{currentmarker}{}%
\end{pgfscope}%
\begin{pgfscope}%
\pgfsys@transformshift{4.220106in}{3.865562in}%
\pgfsys@useobject{currentmarker}{}%
\end{pgfscope}%
\begin{pgfscope}%
\pgfsys@transformshift{4.229367in}{3.864836in}%
\pgfsys@useobject{currentmarker}{}%
\end{pgfscope}%
\begin{pgfscope}%
\pgfsys@transformshift{4.239396in}{3.863899in}%
\pgfsys@useobject{currentmarker}{}%
\end{pgfscope}%
\begin{pgfscope}%
\pgfsys@transformshift{4.244718in}{3.865438in}%
\pgfsys@useobject{currentmarker}{}%
\end{pgfscope}%
\begin{pgfscope}%
\pgfsys@transformshift{4.251061in}{3.865536in}%
\pgfsys@useobject{currentmarker}{}%
\end{pgfscope}%
\begin{pgfscope}%
\pgfsys@transformshift{4.259639in}{3.866934in}%
\pgfsys@useobject{currentmarker}{}%
\end{pgfscope}%
\begin{pgfscope}%
\pgfsys@transformshift{4.269790in}{3.864556in}%
\pgfsys@useobject{currentmarker}{}%
\end{pgfscope}%
\begin{pgfscope}%
\pgfsys@transformshift{4.275523in}{3.864644in}%
\pgfsys@useobject{currentmarker}{}%
\end{pgfscope}%
\begin{pgfscope}%
\pgfsys@transformshift{4.283326in}{3.863994in}%
\pgfsys@useobject{currentmarker}{}%
\end{pgfscope}%
\begin{pgfscope}%
\pgfsys@transformshift{4.292222in}{3.865654in}%
\pgfsys@useobject{currentmarker}{}%
\end{pgfscope}%
\begin{pgfscope}%
\pgfsys@transformshift{4.297196in}{3.865845in}%
\pgfsys@useobject{currentmarker}{}%
\end{pgfscope}%
\begin{pgfscope}%
\pgfsys@transformshift{4.304656in}{3.867056in}%
\pgfsys@useobject{currentmarker}{}%
\end{pgfscope}%
\begin{pgfscope}%
\pgfsys@transformshift{4.308806in}{3.866813in}%
\pgfsys@useobject{currentmarker}{}%
\end{pgfscope}%
\begin{pgfscope}%
\pgfsys@transformshift{4.313911in}{3.866741in}%
\pgfsys@useobject{currentmarker}{}%
\end{pgfscope}%
\begin{pgfscope}%
\pgfsys@transformshift{4.320305in}{3.866274in}%
\pgfsys@useobject{currentmarker}{}%
\end{pgfscope}%
\begin{pgfscope}%
\pgfsys@transformshift{4.327502in}{3.865902in}%
\pgfsys@useobject{currentmarker}{}%
\end{pgfscope}%
\begin{pgfscope}%
\pgfsys@transformshift{4.331461in}{3.865717in}%
\pgfsys@useobject{currentmarker}{}%
\end{pgfscope}%
\begin{pgfscope}%
\pgfsys@transformshift{4.336481in}{3.866272in}%
\pgfsys@useobject{currentmarker}{}%
\end{pgfscope}%
\begin{pgfscope}%
\pgfsys@transformshift{4.342106in}{3.866100in}%
\pgfsys@useobject{currentmarker}{}%
\end{pgfscope}%
\begin{pgfscope}%
\pgfsys@transformshift{4.349201in}{3.866945in}%
\pgfsys@useobject{currentmarker}{}%
\end{pgfscope}%
\begin{pgfscope}%
\pgfsys@transformshift{4.357787in}{3.865873in}%
\pgfsys@useobject{currentmarker}{}%
\end{pgfscope}%
\begin{pgfscope}%
\pgfsys@transformshift{4.368551in}{3.865325in}%
\pgfsys@useobject{currentmarker}{}%
\end{pgfscope}%
\begin{pgfscope}%
\pgfsys@transformshift{4.383111in}{3.866798in}%
\pgfsys@useobject{currentmarker}{}%
\end{pgfscope}%
\begin{pgfscope}%
\pgfsys@transformshift{4.399305in}{3.868100in}%
\pgfsys@useobject{currentmarker}{}%
\end{pgfscope}%
\begin{pgfscope}%
\pgfsys@transformshift{4.416196in}{3.868558in}%
\pgfsys@useobject{currentmarker}{}%
\end{pgfscope}%
\begin{pgfscope}%
\pgfsys@transformshift{4.435064in}{3.868843in}%
\pgfsys@useobject{currentmarker}{}%
\end{pgfscope}%
\begin{pgfscope}%
\pgfsys@transformshift{4.454941in}{3.868734in}%
\pgfsys@useobject{currentmarker}{}%
\end{pgfscope}%
\begin{pgfscope}%
\pgfsys@transformshift{4.475756in}{3.868502in}%
\pgfsys@useobject{currentmarker}{}%
\end{pgfscope}%
\begin{pgfscope}%
\pgfsys@transformshift{4.487204in}{3.868416in}%
\pgfsys@useobject{currentmarker}{}%
\end{pgfscope}%
\begin{pgfscope}%
\pgfsys@transformshift{4.499869in}{3.868521in}%
\pgfsys@useobject{currentmarker}{}%
\end{pgfscope}%
\begin{pgfscope}%
\pgfsys@transformshift{4.513356in}{3.868132in}%
\pgfsys@useobject{currentmarker}{}%
\end{pgfscope}%
\begin{pgfscope}%
\pgfsys@transformshift{4.527639in}{3.867280in}%
\pgfsys@useobject{currentmarker}{}%
\end{pgfscope}%
\begin{pgfscope}%
\pgfsys@transformshift{4.535483in}{3.866634in}%
\pgfsys@useobject{currentmarker}{}%
\end{pgfscope}%
\begin{pgfscope}%
\pgfsys@transformshift{4.544138in}{3.866709in}%
\pgfsys@useobject{currentmarker}{}%
\end{pgfscope}%
\begin{pgfscope}%
\pgfsys@transformshift{4.548890in}{3.866421in}%
\pgfsys@useobject{currentmarker}{}%
\end{pgfscope}%
\begin{pgfscope}%
\pgfsys@transformshift{4.551508in}{3.866351in}%
\pgfsys@useobject{currentmarker}{}%
\end{pgfscope}%
\begin{pgfscope}%
\pgfsys@transformshift{4.552944in}{3.866240in}%
\pgfsys@useobject{currentmarker}{}%
\end{pgfscope}%
\begin{pgfscope}%
\pgfsys@transformshift{4.553735in}{3.866203in}%
\pgfsys@useobject{currentmarker}{}%
\end{pgfscope}%
\begin{pgfscope}%
\pgfsys@transformshift{4.554169in}{3.866168in}%
\pgfsys@useobject{currentmarker}{}%
\end{pgfscope}%
\begin{pgfscope}%
\pgfsys@transformshift{4.554409in}{3.866167in}%
\pgfsys@useobject{currentmarker}{}%
\end{pgfscope}%
\begin{pgfscope}%
\pgfsys@transformshift{4.554540in}{3.866161in}%
\pgfsys@useobject{currentmarker}{}%
\end{pgfscope}%
\begin{pgfscope}%
\pgfsys@transformshift{4.554613in}{3.866157in}%
\pgfsys@useobject{currentmarker}{}%
\end{pgfscope}%
\begin{pgfscope}%
\pgfsys@transformshift{4.554652in}{3.866148in}%
\pgfsys@useobject{currentmarker}{}%
\end{pgfscope}%
\begin{pgfscope}%
\pgfsys@transformshift{4.554673in}{3.866144in}%
\pgfsys@useobject{currentmarker}{}%
\end{pgfscope}%
\begin{pgfscope}%
\pgfsys@transformshift{4.554683in}{3.866137in}%
\pgfsys@useobject{currentmarker}{}%
\end{pgfscope}%
\begin{pgfscope}%
\pgfsys@transformshift{4.555221in}{3.865798in}%
\pgfsys@useobject{currentmarker}{}%
\end{pgfscope}%
\begin{pgfscope}%
\pgfsys@transformshift{4.555439in}{3.865524in}%
\pgfsys@useobject{currentmarker}{}%
\end{pgfscope}%
\begin{pgfscope}%
\pgfsys@transformshift{4.556383in}{3.864467in}%
\pgfsys@useobject{currentmarker}{}%
\end{pgfscope}%
\begin{pgfscope}%
\pgfsys@transformshift{4.557342in}{3.861405in}%
\pgfsys@useobject{currentmarker}{}%
\end{pgfscope}%
\begin{pgfscope}%
\pgfsys@transformshift{4.558434in}{3.857660in}%
\pgfsys@useobject{currentmarker}{}%
\end{pgfscope}%
\begin{pgfscope}%
\pgfsys@transformshift{4.558606in}{3.853204in}%
\pgfsys@useobject{currentmarker}{}%
\end{pgfscope}%
\begin{pgfscope}%
\pgfsys@transformshift{4.558067in}{3.847787in}%
\pgfsys@useobject{currentmarker}{}%
\end{pgfscope}%
\begin{pgfscope}%
\pgfsys@transformshift{4.557802in}{3.841492in}%
\pgfsys@useobject{currentmarker}{}%
\end{pgfscope}%
\begin{pgfscope}%
\pgfsys@transformshift{4.557193in}{3.838081in}%
\pgfsys@useobject{currentmarker}{}%
\end{pgfscope}%
\begin{pgfscope}%
\pgfsys@transformshift{4.556954in}{3.836190in}%
\pgfsys@useobject{currentmarker}{}%
\end{pgfscope}%
\begin{pgfscope}%
\pgfsys@transformshift{4.556671in}{3.833770in}%
\pgfsys@useobject{currentmarker}{}%
\end{pgfscope}%
\begin{pgfscope}%
\pgfsys@transformshift{4.556416in}{3.832454in}%
\pgfsys@useobject{currentmarker}{}%
\end{pgfscope}%
\begin{pgfscope}%
\pgfsys@transformshift{4.556254in}{3.829735in}%
\pgfsys@useobject{currentmarker}{}%
\end{pgfscope}%
\begin{pgfscope}%
\pgfsys@transformshift{4.555642in}{3.828367in}%
\pgfsys@useobject{currentmarker}{}%
\end{pgfscope}%
\begin{pgfscope}%
\pgfsys@transformshift{4.555552in}{3.827548in}%
\pgfsys@useobject{currentmarker}{}%
\end{pgfscope}%
\begin{pgfscope}%
\pgfsys@transformshift{4.555626in}{3.824921in}%
\pgfsys@useobject{currentmarker}{}%
\end{pgfscope}%
\begin{pgfscope}%
\pgfsys@transformshift{4.554671in}{3.821847in}%
\pgfsys@useobject{currentmarker}{}%
\end{pgfscope}%
\begin{pgfscope}%
\pgfsys@transformshift{4.554214in}{3.820137in}%
\pgfsys@useobject{currentmarker}{}%
\end{pgfscope}%
\begin{pgfscope}%
\pgfsys@transformshift{4.554255in}{3.815194in}%
\pgfsys@useobject{currentmarker}{}%
\end{pgfscope}%
\begin{pgfscope}%
\pgfsys@transformshift{4.553544in}{3.812570in}%
\pgfsys@useobject{currentmarker}{}%
\end{pgfscope}%
\begin{pgfscope}%
\pgfsys@transformshift{4.552933in}{3.811205in}%
\pgfsys@useobject{currentmarker}{}%
\end{pgfscope}%
\begin{pgfscope}%
\pgfsys@transformshift{4.552835in}{3.805797in}%
\pgfsys@useobject{currentmarker}{}%
\end{pgfscope}%
\begin{pgfscope}%
\pgfsys@transformshift{4.552433in}{3.802849in}%
\pgfsys@useobject{currentmarker}{}%
\end{pgfscope}%
\begin{pgfscope}%
\pgfsys@transformshift{4.550364in}{3.797260in}%
\pgfsys@useobject{currentmarker}{}%
\end{pgfscope}%
\begin{pgfscope}%
\pgfsys@transformshift{4.549901in}{3.794015in}%
\pgfsys@useobject{currentmarker}{}%
\end{pgfscope}%
\begin{pgfscope}%
\pgfsys@transformshift{4.550680in}{3.790091in}%
\pgfsys@useobject{currentmarker}{}%
\end{pgfscope}%
\begin{pgfscope}%
\pgfsys@transformshift{4.550298in}{3.787924in}%
\pgfsys@useobject{currentmarker}{}%
\end{pgfscope}%
\begin{pgfscope}%
\pgfsys@transformshift{4.550399in}{3.785221in}%
\pgfsys@useobject{currentmarker}{}%
\end{pgfscope}%
\begin{pgfscope}%
\pgfsys@transformshift{4.550777in}{3.781902in}%
\pgfsys@useobject{currentmarker}{}%
\end{pgfscope}%
\begin{pgfscope}%
\pgfsys@transformshift{4.550965in}{3.776793in}%
\pgfsys@useobject{currentmarker}{}%
\end{pgfscope}%
\begin{pgfscope}%
\pgfsys@transformshift{4.550318in}{3.774056in}%
\pgfsys@useobject{currentmarker}{}%
\end{pgfscope}%
\begin{pgfscope}%
\pgfsys@transformshift{4.550629in}{3.772541in}%
\pgfsys@useobject{currentmarker}{}%
\end{pgfscope}%
\begin{pgfscope}%
\pgfsys@transformshift{4.550526in}{3.771697in}%
\pgfsys@useobject{currentmarker}{}%
\end{pgfscope}%
\begin{pgfscope}%
\pgfsys@transformshift{4.550031in}{3.769804in}%
\pgfsys@useobject{currentmarker}{}%
\end{pgfscope}%
\begin{pgfscope}%
\pgfsys@transformshift{4.551141in}{3.763972in}%
\pgfsys@useobject{currentmarker}{}%
\end{pgfscope}%
\begin{pgfscope}%
\pgfsys@transformshift{4.551355in}{3.757499in}%
\pgfsys@useobject{currentmarker}{}%
\end{pgfscope}%
\begin{pgfscope}%
\pgfsys@transformshift{4.549801in}{3.748789in}%
\pgfsys@useobject{currentmarker}{}%
\end{pgfscope}%
\begin{pgfscope}%
\pgfsys@transformshift{4.548902in}{3.744007in}%
\pgfsys@useobject{currentmarker}{}%
\end{pgfscope}%
\begin{pgfscope}%
\pgfsys@transformshift{4.550178in}{3.738748in}%
\pgfsys@useobject{currentmarker}{}%
\end{pgfscope}%
\begin{pgfscope}%
\pgfsys@transformshift{4.548997in}{3.731671in}%
\pgfsys@useobject{currentmarker}{}%
\end{pgfscope}%
\begin{pgfscope}%
\pgfsys@transformshift{4.548129in}{3.727822in}%
\pgfsys@useobject{currentmarker}{}%
\end{pgfscope}%
\begin{pgfscope}%
\pgfsys@transformshift{4.549462in}{3.722318in}%
\pgfsys@useobject{currentmarker}{}%
\end{pgfscope}%
\begin{pgfscope}%
\pgfsys@transformshift{4.549350in}{3.719206in}%
\pgfsys@useobject{currentmarker}{}%
\end{pgfscope}%
\begin{pgfscope}%
\pgfsys@transformshift{4.548221in}{3.715367in}%
\pgfsys@useobject{currentmarker}{}%
\end{pgfscope}%
\begin{pgfscope}%
\pgfsys@transformshift{4.549397in}{3.709661in}%
\pgfsys@useobject{currentmarker}{}%
\end{pgfscope}%
\begin{pgfscope}%
\pgfsys@transformshift{4.550647in}{3.703424in}%
\pgfsys@useobject{currentmarker}{}%
\end{pgfscope}%
\begin{pgfscope}%
\pgfsys@transformshift{4.549021in}{3.693359in}%
\pgfsys@useobject{currentmarker}{}%
\end{pgfscope}%
\begin{pgfscope}%
\pgfsys@transformshift{4.548603in}{3.682574in}%
\pgfsys@useobject{currentmarker}{}%
\end{pgfscope}%
\begin{pgfscope}%
\pgfsys@transformshift{4.549477in}{3.676703in}%
\pgfsys@useobject{currentmarker}{}%
\end{pgfscope}%
\begin{pgfscope}%
\pgfsys@transformshift{4.549288in}{3.668811in}%
\pgfsys@useobject{currentmarker}{}%
\end{pgfscope}%
\begin{pgfscope}%
\pgfsys@transformshift{4.546176in}{3.659447in}%
\pgfsys@useobject{currentmarker}{}%
\end{pgfscope}%
\begin{pgfscope}%
\pgfsys@transformshift{4.548610in}{3.647413in}%
\pgfsys@useobject{currentmarker}{}%
\end{pgfscope}%
\begin{pgfscope}%
\pgfsys@transformshift{4.549320in}{3.634595in}%
\pgfsys@useobject{currentmarker}{}%
\end{pgfscope}%
\begin{pgfscope}%
\pgfsys@transformshift{4.546523in}{3.619047in}%
\pgfsys@useobject{currentmarker}{}%
\end{pgfscope}%
\begin{pgfscope}%
\pgfsys@transformshift{4.548170in}{3.602382in}%
\pgfsys@useobject{currentmarker}{}%
\end{pgfscope}%
\begin{pgfscope}%
\pgfsys@transformshift{4.550252in}{3.585146in}%
\pgfsys@useobject{currentmarker}{}%
\end{pgfscope}%
\begin{pgfscope}%
\pgfsys@transformshift{4.547912in}{3.563358in}%
\pgfsys@useobject{currentmarker}{}%
\end{pgfscope}%
\begin{pgfscope}%
\pgfsys@transformshift{4.545271in}{3.541045in}%
\pgfsys@useobject{currentmarker}{}%
\end{pgfscope}%
\begin{pgfscope}%
\pgfsys@transformshift{4.549714in}{3.517206in}%
\pgfsys@useobject{currentmarker}{}%
\end{pgfscope}%
\begin{pgfscope}%
\pgfsys@transformshift{4.548625in}{3.503914in}%
\pgfsys@useobject{currentmarker}{}%
\end{pgfscope}%
\begin{pgfscope}%
\pgfsys@transformshift{4.546993in}{3.496762in}%
\pgfsys@useobject{currentmarker}{}%
\end{pgfscope}%
\begin{pgfscope}%
\pgfsys@transformshift{4.549261in}{3.486050in}%
\pgfsys@useobject{currentmarker}{}%
\end{pgfscope}%
\begin{pgfscope}%
\pgfsys@transformshift{4.550177in}{3.480098in}%
\pgfsys@useobject{currentmarker}{}%
\end{pgfscope}%
\begin{pgfscope}%
\pgfsys@transformshift{4.547262in}{3.470864in}%
\pgfsys@useobject{currentmarker}{}%
\end{pgfscope}%
\begin{pgfscope}%
\pgfsys@transformshift{4.549448in}{3.460640in}%
\pgfsys@useobject{currentmarker}{}%
\end{pgfscope}%
\begin{pgfscope}%
\pgfsys@transformshift{4.550607in}{3.449635in}%
\pgfsys@useobject{currentmarker}{}%
\end{pgfscope}%
\begin{pgfscope}%
\pgfsys@transformshift{4.549029in}{3.434757in}%
\pgfsys@useobject{currentmarker}{}%
\end{pgfscope}%
\begin{pgfscope}%
\pgfsys@transformshift{4.548041in}{3.426587in}%
\pgfsys@useobject{currentmarker}{}%
\end{pgfscope}%
\begin{pgfscope}%
\pgfsys@transformshift{4.550493in}{3.418064in}%
\pgfsys@useobject{currentmarker}{}%
\end{pgfscope}%
\begin{pgfscope}%
\pgfsys@transformshift{4.549838in}{3.413231in}%
\pgfsys@useobject{currentmarker}{}%
\end{pgfscope}%
\begin{pgfscope}%
\pgfsys@transformshift{4.549688in}{3.410552in}%
\pgfsys@useobject{currentmarker}{}%
\end{pgfscope}%
\begin{pgfscope}%
\pgfsys@transformshift{4.550243in}{3.405193in}%
\pgfsys@useobject{currentmarker}{}%
\end{pgfscope}%
\begin{pgfscope}%
\pgfsys@transformshift{4.550414in}{3.402234in}%
\pgfsys@useobject{currentmarker}{}%
\end{pgfscope}%
\begin{pgfscope}%
\pgfsys@transformshift{4.549835in}{3.398749in}%
\pgfsys@useobject{currentmarker}{}%
\end{pgfscope}%
\begin{pgfscope}%
\pgfsys@transformshift{4.550497in}{3.394108in}%
\pgfsys@useobject{currentmarker}{}%
\end{pgfscope}%
\begin{pgfscope}%
\pgfsys@transformshift{4.550738in}{3.391541in}%
\pgfsys@useobject{currentmarker}{}%
\end{pgfscope}%
\begin{pgfscope}%
\pgfsys@transformshift{4.550378in}{3.385958in}%
\pgfsys@useobject{currentmarker}{}%
\end{pgfscope}%
\begin{pgfscope}%
\pgfsys@transformshift{4.550199in}{3.379350in}%
\pgfsys@useobject{currentmarker}{}%
\end{pgfscope}%
\begin{pgfscope}%
\pgfsys@transformshift{4.551265in}{3.375874in}%
\pgfsys@useobject{currentmarker}{}%
\end{pgfscope}%
\begin{pgfscope}%
\pgfsys@transformshift{4.550574in}{3.369303in}%
\pgfsys@useobject{currentmarker}{}%
\end{pgfscope}%
\begin{pgfscope}%
\pgfsys@transformshift{4.550880in}{3.361924in}%
\pgfsys@useobject{currentmarker}{}%
\end{pgfscope}%
\begin{pgfscope}%
\pgfsys@transformshift{4.552314in}{3.354027in}%
\pgfsys@useobject{currentmarker}{}%
\end{pgfscope}%
\begin{pgfscope}%
\pgfsys@transformshift{4.552525in}{3.344624in}%
\pgfsys@useobject{currentmarker}{}%
\end{pgfscope}%
\begin{pgfscope}%
\pgfsys@transformshift{4.551672in}{3.334580in}%
\pgfsys@useobject{currentmarker}{}%
\end{pgfscope}%
\begin{pgfscope}%
\pgfsys@transformshift{4.554704in}{3.321642in}%
\pgfsys@useobject{currentmarker}{}%
\end{pgfscope}%
\begin{pgfscope}%
\pgfsys@transformshift{4.555873in}{3.314427in}%
\pgfsys@useobject{currentmarker}{}%
\end{pgfscope}%
\begin{pgfscope}%
\pgfsys@transformshift{4.557298in}{3.305185in}%
\pgfsys@useobject{currentmarker}{}%
\end{pgfscope}%
\begin{pgfscope}%
\pgfsys@transformshift{4.556088in}{3.300186in}%
\pgfsys@useobject{currentmarker}{}%
\end{pgfscope}%
\begin{pgfscope}%
\pgfsys@transformshift{4.558143in}{3.292668in}%
\pgfsys@useobject{currentmarker}{}%
\end{pgfscope}%
\begin{pgfscope}%
\pgfsys@transformshift{4.558795in}{3.288431in}%
\pgfsys@useobject{currentmarker}{}%
\end{pgfscope}%
\begin{pgfscope}%
\pgfsys@transformshift{4.558053in}{3.281981in}%
\pgfsys@useobject{currentmarker}{}%
\end{pgfscope}%
\begin{pgfscope}%
\pgfsys@transformshift{4.558014in}{3.278410in}%
\pgfsys@useobject{currentmarker}{}%
\end{pgfscope}%
\begin{pgfscope}%
\pgfsys@transformshift{4.558907in}{3.273959in}%
\pgfsys@useobject{currentmarker}{}%
\end{pgfscope}%
\begin{pgfscope}%
\pgfsys@transformshift{4.559207in}{3.271480in}%
\pgfsys@useobject{currentmarker}{}%
\end{pgfscope}%
\begin{pgfscope}%
\pgfsys@transformshift{4.558436in}{3.268580in}%
\pgfsys@useobject{currentmarker}{}%
\end{pgfscope}%
\begin{pgfscope}%
\pgfsys@transformshift{4.560180in}{3.262986in}%
\pgfsys@useobject{currentmarker}{}%
\end{pgfscope}%
\begin{pgfscope}%
\pgfsys@transformshift{4.561926in}{3.256366in}%
\pgfsys@useobject{currentmarker}{}%
\end{pgfscope}%
\begin{pgfscope}%
\pgfsys@transformshift{4.561097in}{3.246163in}%
\pgfsys@useobject{currentmarker}{}%
\end{pgfscope}%
\begin{pgfscope}%
\pgfsys@transformshift{4.559617in}{3.240731in}%
\pgfsys@useobject{currentmarker}{}%
\end{pgfscope}%
\begin{pgfscope}%
\pgfsys@transformshift{4.561230in}{3.233135in}%
\pgfsys@useobject{currentmarker}{}%
\end{pgfscope}%
\begin{pgfscope}%
\pgfsys@transformshift{4.561020in}{3.228870in}%
\pgfsys@useobject{currentmarker}{}%
\end{pgfscope}%
\begin{pgfscope}%
\pgfsys@transformshift{4.560128in}{3.223748in}%
\pgfsys@useobject{currentmarker}{}%
\end{pgfscope}%
\begin{pgfscope}%
\pgfsys@transformshift{4.561658in}{3.215005in}%
\pgfsys@useobject{currentmarker}{}%
\end{pgfscope}%
\begin{pgfscope}%
\pgfsys@transformshift{4.562512in}{3.204818in}%
\pgfsys@useobject{currentmarker}{}%
\end{pgfscope}%
\begin{pgfscope}%
\pgfsys@transformshift{4.560386in}{3.191690in}%
\pgfsys@useobject{currentmarker}{}%
\end{pgfscope}%
\begin{pgfscope}%
\pgfsys@transformshift{4.559428in}{3.177598in}%
\pgfsys@useobject{currentmarker}{}%
\end{pgfscope}%
\begin{pgfscope}%
\pgfsys@transformshift{4.560668in}{3.169928in}%
\pgfsys@useobject{currentmarker}{}%
\end{pgfscope}%
\begin{pgfscope}%
\pgfsys@transformshift{4.560638in}{3.161466in}%
\pgfsys@useobject{currentmarker}{}%
\end{pgfscope}%
\begin{pgfscope}%
\pgfsys@transformshift{4.558417in}{3.151525in}%
\pgfsys@useobject{currentmarker}{}%
\end{pgfscope}%
\begin{pgfscope}%
\pgfsys@transformshift{4.561920in}{3.139227in}%
\pgfsys@useobject{currentmarker}{}%
\end{pgfscope}%
\begin{pgfscope}%
\pgfsys@transformshift{4.565197in}{3.125760in}%
\pgfsys@useobject{currentmarker}{}%
\end{pgfscope}%
\begin{pgfscope}%
\pgfsys@transformshift{4.561559in}{3.108332in}%
\pgfsys@useobject{currentmarker}{}%
\end{pgfscope}%
\begin{pgfscope}%
\pgfsys@transformshift{4.561870in}{3.089854in}%
\pgfsys@useobject{currentmarker}{}%
\end{pgfscope}%
\begin{pgfscope}%
\pgfsys@transformshift{4.564591in}{3.070163in}%
\pgfsys@useobject{currentmarker}{}%
\end{pgfscope}%
\begin{pgfscope}%
\pgfsys@transformshift{4.566697in}{3.049345in}%
\pgfsys@useobject{currentmarker}{}%
\end{pgfscope}%
\begin{pgfscope}%
\pgfsys@transformshift{4.558829in}{3.029008in}%
\pgfsys@useobject{currentmarker}{}%
\end{pgfscope}%
\begin{pgfscope}%
\pgfsys@transformshift{4.566300in}{3.006564in}%
\pgfsys@useobject{currentmarker}{}%
\end{pgfscope}%
\begin{pgfscope}%
\pgfsys@transformshift{4.572701in}{2.982004in}%
\pgfsys@useobject{currentmarker}{}%
\end{pgfscope}%
\begin{pgfscope}%
\pgfsys@transformshift{4.571922in}{2.952725in}%
\pgfsys@useobject{currentmarker}{}%
\end{pgfscope}%
\begin{pgfscope}%
\pgfsys@transformshift{4.566980in}{2.923007in}%
\pgfsys@useobject{currentmarker}{}%
\end{pgfscope}%
\begin{pgfscope}%
\pgfsys@transformshift{4.577408in}{2.892483in}%
\pgfsys@useobject{currentmarker}{}%
\end{pgfscope}%
\begin{pgfscope}%
\pgfsys@transformshift{4.584942in}{2.860167in}%
\pgfsys@useobject{currentmarker}{}%
\end{pgfscope}%
\begin{pgfscope}%
\pgfsys@transformshift{4.578380in}{2.824682in}%
\pgfsys@useobject{currentmarker}{}%
\end{pgfscope}%
\begin{pgfscope}%
\pgfsys@transformshift{4.581794in}{2.805130in}%
\pgfsys@useobject{currentmarker}{}%
\end{pgfscope}%
\begin{pgfscope}%
\pgfsys@transformshift{4.583264in}{2.794313in}%
\pgfsys@useobject{currentmarker}{}%
\end{pgfscope}%
\begin{pgfscope}%
\pgfsys@transformshift{4.582639in}{2.779288in}%
\pgfsys@useobject{currentmarker}{}%
\end{pgfscope}%
\begin{pgfscope}%
\pgfsys@transformshift{4.579630in}{2.763565in}%
\pgfsys@useobject{currentmarker}{}%
\end{pgfscope}%
\begin{pgfscope}%
\pgfsys@transformshift{4.583787in}{2.745382in}%
\pgfsys@useobject{currentmarker}{}%
\end{pgfscope}%
\begin{pgfscope}%
\pgfsys@transformshift{4.586337in}{2.725888in}%
\pgfsys@useobject{currentmarker}{}%
\end{pgfscope}%
\begin{pgfscope}%
\pgfsys@transformshift{4.580898in}{2.704172in}%
\pgfsys@useobject{currentmarker}{}%
\end{pgfscope}%
\begin{pgfscope}%
\pgfsys@transformshift{4.583672in}{2.692177in}%
\pgfsys@useobject{currentmarker}{}%
\end{pgfscope}%
\begin{pgfscope}%
\pgfsys@transformshift{4.585136in}{2.685565in}%
\pgfsys@useobject{currentmarker}{}%
\end{pgfscope}%
\begin{pgfscope}%
\pgfsys@transformshift{4.584586in}{2.674506in}%
\pgfsys@useobject{currentmarker}{}%
\end{pgfscope}%
\begin{pgfscope}%
\pgfsys@transformshift{4.582912in}{2.668651in}%
\pgfsys@useobject{currentmarker}{}%
\end{pgfscope}%
\begin{pgfscope}%
\pgfsys@transformshift{4.585687in}{2.661086in}%
\pgfsys@useobject{currentmarker}{}%
\end{pgfscope}%
\begin{pgfscope}%
\pgfsys@transformshift{4.588586in}{2.652549in}%
\pgfsys@useobject{currentmarker}{}%
\end{pgfscope}%
\begin{pgfscope}%
\pgfsys@transformshift{4.585625in}{2.638478in}%
\pgfsys@useobject{currentmarker}{}%
\end{pgfscope}%
\begin{pgfscope}%
\pgfsys@transformshift{4.588500in}{2.631111in}%
\pgfsys@useobject{currentmarker}{}%
\end{pgfscope}%
\begin{pgfscope}%
\pgfsys@transformshift{4.589371in}{2.626849in}%
\pgfsys@useobject{currentmarker}{}%
\end{pgfscope}%
\begin{pgfscope}%
\pgfsys@transformshift{4.587154in}{2.616849in}%
\pgfsys@useobject{currentmarker}{}%
\end{pgfscope}%
\begin{pgfscope}%
\pgfsys@transformshift{4.587955in}{2.605718in}%
\pgfsys@useobject{currentmarker}{}%
\end{pgfscope}%
\begin{pgfscope}%
\pgfsys@transformshift{4.588758in}{2.599634in}%
\pgfsys@useobject{currentmarker}{}%
\end{pgfscope}%
\begin{pgfscope}%
\pgfsys@transformshift{4.588966in}{2.589060in}%
\pgfsys@useobject{currentmarker}{}%
\end{pgfscope}%
\begin{pgfscope}%
\pgfsys@transformshift{4.587245in}{2.583503in}%
\pgfsys@useobject{currentmarker}{}%
\end{pgfscope}%
\begin{pgfscope}%
\pgfsys@transformshift{4.589147in}{2.576173in}%
\pgfsys@useobject{currentmarker}{}%
\end{pgfscope}%
\begin{pgfscope}%
\pgfsys@transformshift{4.589418in}{2.572017in}%
\pgfsys@useobject{currentmarker}{}%
\end{pgfscope}%
\begin{pgfscope}%
\pgfsys@transformshift{4.587880in}{2.565803in}%
\pgfsys@useobject{currentmarker}{}%
\end{pgfscope}%
\begin{pgfscope}%
\pgfsys@transformshift{4.590387in}{2.557634in}%
\pgfsys@useobject{currentmarker}{}%
\end{pgfscope}%
\begin{pgfscope}%
\pgfsys@transformshift{4.592599in}{2.548496in}%
\pgfsys@useobject{currentmarker}{}%
\end{pgfscope}%
\begin{pgfscope}%
\pgfsys@transformshift{4.589453in}{2.534552in}%
\pgfsys@useobject{currentmarker}{}%
\end{pgfscope}%
\begin{pgfscope}%
\pgfsys@transformshift{4.591666in}{2.519519in}%
\pgfsys@useobject{currentmarker}{}%
\end{pgfscope}%
\begin{pgfscope}%
\pgfsys@transformshift{4.592953in}{2.511262in}%
\pgfsys@useobject{currentmarker}{}%
\end{pgfscope}%
\begin{pgfscope}%
\pgfsys@transformshift{4.593488in}{2.498931in}%
\pgfsys@useobject{currentmarker}{}%
\end{pgfscope}%
\begin{pgfscope}%
\pgfsys@transformshift{4.589735in}{2.486210in}%
\pgfsys@useobject{currentmarker}{}%
\end{pgfscope}%
\begin{pgfscope}%
\pgfsys@transformshift{4.593286in}{2.471350in}%
\pgfsys@useobject{currentmarker}{}%
\end{pgfscope}%
\begin{pgfscope}%
\pgfsys@transformshift{4.595894in}{2.455449in}%
\pgfsys@useobject{currentmarker}{}%
\end{pgfscope}%
\begin{pgfscope}%
\pgfsys@transformshift{4.592111in}{2.436775in}%
\pgfsys@useobject{currentmarker}{}%
\end{pgfscope}%
\begin{pgfscope}%
\pgfsys@transformshift{4.598568in}{2.417652in}%
\pgfsys@useobject{currentmarker}{}%
\end{pgfscope}%
\begin{pgfscope}%
\pgfsys@transformshift{4.603964in}{2.397630in}%
\pgfsys@useobject{currentmarker}{}%
\end{pgfscope}%
\begin{pgfscope}%
\pgfsys@transformshift{4.602086in}{2.372674in}%
\pgfsys@useobject{currentmarker}{}%
\end{pgfscope}%
\begin{pgfscope}%
\pgfsys@transformshift{4.600134in}{2.359049in}%
\pgfsys@useobject{currentmarker}{}%
\end{pgfscope}%
\begin{pgfscope}%
\pgfsys@transformshift{4.603630in}{2.343886in}%
\pgfsys@useobject{currentmarker}{}%
\end{pgfscope}%
\begin{pgfscope}%
\pgfsys@transformshift{4.603710in}{2.335328in}%
\pgfsys@useobject{currentmarker}{}%
\end{pgfscope}%
\begin{pgfscope}%
\pgfsys@transformshift{4.601437in}{2.324991in}%
\pgfsys@useobject{currentmarker}{}%
\end{pgfscope}%
\begin{pgfscope}%
\pgfsys@transformshift{4.604793in}{2.312057in}%
\pgfsys@useobject{currentmarker}{}%
\end{pgfscope}%
\begin{pgfscope}%
\pgfsys@transformshift{4.606412in}{2.304888in}%
\pgfsys@useobject{currentmarker}{}%
\end{pgfscope}%
\begin{pgfscope}%
\pgfsys@transformshift{4.604320in}{2.294254in}%
\pgfsys@useobject{currentmarker}{}%
\end{pgfscope}%
\begin{pgfscope}%
\pgfsys@transformshift{4.605911in}{2.288509in}%
\pgfsys@useobject{currentmarker}{}%
\end{pgfscope}%
\begin{pgfscope}%
\pgfsys@transformshift{4.607431in}{2.285605in}%
\pgfsys@useobject{currentmarker}{}%
\end{pgfscope}%
\begin{pgfscope}%
\pgfsys@transformshift{4.606498in}{2.277709in}%
\pgfsys@useobject{currentmarker}{}%
\end{pgfscope}%
\begin{pgfscope}%
\pgfsys@transformshift{4.606099in}{2.273355in}%
\pgfsys@useobject{currentmarker}{}%
\end{pgfscope}%
\begin{pgfscope}%
\pgfsys@transformshift{4.608190in}{2.265688in}%
\pgfsys@useobject{currentmarker}{}%
\end{pgfscope}%
\begin{pgfscope}%
\pgfsys@transformshift{4.608649in}{2.261342in}%
\pgfsys@useobject{currentmarker}{}%
\end{pgfscope}%
\begin{pgfscope}%
\pgfsys@transformshift{4.607331in}{2.256564in}%
\pgfsys@useobject{currentmarker}{}%
\end{pgfscope}%
\begin{pgfscope}%
\pgfsys@transformshift{4.610286in}{2.249193in}%
\pgfsys@useobject{currentmarker}{}%
\end{pgfscope}%
\begin{pgfscope}%
\pgfsys@transformshift{4.611649in}{2.240391in}%
\pgfsys@useobject{currentmarker}{}%
\end{pgfscope}%
\begin{pgfscope}%
\pgfsys@transformshift{4.608983in}{2.227889in}%
\pgfsys@useobject{currentmarker}{}%
\end{pgfscope}%
\begin{pgfscope}%
\pgfsys@transformshift{4.611449in}{2.221306in}%
\pgfsys@useobject{currentmarker}{}%
\end{pgfscope}%
\begin{pgfscope}%
\pgfsys@transformshift{4.613236in}{2.213398in}%
\pgfsys@useobject{currentmarker}{}%
\end{pgfscope}%
\begin{pgfscope}%
\pgfsys@transformshift{4.610370in}{2.199528in}%
\pgfsys@useobject{currentmarker}{}%
\end{pgfscope}%
\begin{pgfscope}%
\pgfsys@transformshift{4.613902in}{2.192585in}%
\pgfsys@useobject{currentmarker}{}%
\end{pgfscope}%
\begin{pgfscope}%
\pgfsys@transformshift{4.615211in}{2.188506in}%
\pgfsys@useobject{currentmarker}{}%
\end{pgfscope}%
\begin{pgfscope}%
\pgfsys@transformshift{4.614030in}{2.176998in}%
\pgfsys@useobject{currentmarker}{}%
\end{pgfscope}%
\begin{pgfscope}%
\pgfsys@transformshift{4.616456in}{2.171116in}%
\pgfsys@useobject{currentmarker}{}%
\end{pgfscope}%
\begin{pgfscope}%
\pgfsys@transformshift{4.618851in}{2.164307in}%
\pgfsys@useobject{currentmarker}{}%
\end{pgfscope}%
\begin{pgfscope}%
\pgfsys@transformshift{4.619064in}{2.152652in}%
\pgfsys@useobject{currentmarker}{}%
\end{pgfscope}%
\begin{pgfscope}%
\pgfsys@transformshift{4.619116in}{2.146240in}%
\pgfsys@useobject{currentmarker}{}%
\end{pgfscope}%
\begin{pgfscope}%
\pgfsys@transformshift{4.621118in}{2.137851in}%
\pgfsys@useobject{currentmarker}{}%
\end{pgfscope}%
\begin{pgfscope}%
\pgfsys@transformshift{4.620572in}{2.127442in}%
\pgfsys@useobject{currentmarker}{}%
\end{pgfscope}%
\begin{pgfscope}%
\pgfsys@transformshift{4.616938in}{2.116755in}%
\pgfsys@useobject{currentmarker}{}%
\end{pgfscope}%
\begin{pgfscope}%
\pgfsys@transformshift{4.621616in}{2.101182in}%
\pgfsys@useobject{currentmarker}{}%
\end{pgfscope}%
\begin{pgfscope}%
\pgfsys@transformshift{4.619954in}{2.084184in}%
\pgfsys@useobject{currentmarker}{}%
\end{pgfscope}%
\begin{pgfscope}%
\pgfsys@transformshift{4.614074in}{2.066632in}%
\pgfsys@useobject{currentmarker}{}%
\end{pgfscope}%
\begin{pgfscope}%
\pgfsys@transformshift{4.620425in}{2.044148in}%
\pgfsys@useobject{currentmarker}{}%
\end{pgfscope}%
\begin{pgfscope}%
\pgfsys@transformshift{4.619324in}{2.031346in}%
\pgfsys@useobject{currentmarker}{}%
\end{pgfscope}%
\begin{pgfscope}%
\pgfsys@transformshift{4.616440in}{2.016876in}%
\pgfsys@useobject{currentmarker}{}%
\end{pgfscope}%
\begin{pgfscope}%
\pgfsys@transformshift{4.621764in}{1.998449in}%
\pgfsys@useobject{currentmarker}{}%
\end{pgfscope}%
\begin{pgfscope}%
\pgfsys@transformshift{4.623423in}{1.978573in}%
\pgfsys@useobject{currentmarker}{}%
\end{pgfscope}%
\begin{pgfscope}%
\pgfsys@transformshift{4.617469in}{1.956142in}%
\pgfsys@useobject{currentmarker}{}%
\end{pgfscope}%
\begin{pgfscope}%
\pgfsys@transformshift{4.625623in}{1.933391in}%
\pgfsys@useobject{currentmarker}{}%
\end{pgfscope}%
\begin{pgfscope}%
\pgfsys@transformshift{4.631444in}{1.908910in}%
\pgfsys@useobject{currentmarker}{}%
\end{pgfscope}%
\begin{pgfscope}%
\pgfsys@transformshift{4.623341in}{1.878724in}%
\pgfsys@useobject{currentmarker}{}%
\end{pgfscope}%
\begin{pgfscope}%
\pgfsys@transformshift{4.629429in}{1.862648in}%
\pgfsys@useobject{currentmarker}{}%
\end{pgfscope}%
\begin{pgfscope}%
\pgfsys@transformshift{4.630179in}{1.844498in}%
\pgfsys@useobject{currentmarker}{}%
\end{pgfscope}%
\begin{pgfscope}%
\pgfsys@transformshift{4.622737in}{1.820517in}%
\pgfsys@useobject{currentmarker}{}%
\end{pgfscope}%
\begin{pgfscope}%
\pgfsys@transformshift{4.631880in}{1.795970in}%
\pgfsys@useobject{currentmarker}{}%
\end{pgfscope}%
\begin{pgfscope}%
\pgfsys@transformshift{4.639736in}{1.769713in}%
\pgfsys@useobject{currentmarker}{}%
\end{pgfscope}%
\begin{pgfscope}%
\pgfsys@transformshift{4.633426in}{1.738501in}%
\pgfsys@useobject{currentmarker}{}%
\end{pgfscope}%
\begin{pgfscope}%
\pgfsys@transformshift{4.639603in}{1.722112in}%
\pgfsys@useobject{currentmarker}{}%
\end{pgfscope}%
\begin{pgfscope}%
\pgfsys@transformshift{4.642855in}{1.703667in}%
\pgfsys@useobject{currentmarker}{}%
\end{pgfscope}%
\begin{pgfscope}%
\pgfsys@transformshift{4.636048in}{1.679476in}%
\pgfsys@useobject{currentmarker}{}%
\end{pgfscope}%
\begin{pgfscope}%
\pgfsys@transformshift{4.639899in}{1.666201in}%
\pgfsys@useobject{currentmarker}{}%
\end{pgfscope}%
\begin{pgfscope}%
\pgfsys@transformshift{4.639183in}{1.651077in}%
\pgfsys@useobject{currentmarker}{}%
\end{pgfscope}%
\begin{pgfscope}%
\pgfsys@transformshift{4.629824in}{1.632693in}%
\pgfsys@useobject{currentmarker}{}%
\end{pgfscope}%
\begin{pgfscope}%
\pgfsys@transformshift{4.631417in}{1.621459in}%
\pgfsys@useobject{currentmarker}{}%
\end{pgfscope}%
\begin{pgfscope}%
\pgfsys@transformshift{4.628437in}{1.609585in}%
\pgfsys@useobject{currentmarker}{}%
\end{pgfscope}%
\begin{pgfscope}%
\pgfsys@transformshift{4.622828in}{1.594438in}%
\pgfsys@useobject{currentmarker}{}%
\end{pgfscope}%
\begin{pgfscope}%
\pgfsys@transformshift{4.623953in}{1.576219in}%
\pgfsys@useobject{currentmarker}{}%
\end{pgfscope}%
\begin{pgfscope}%
\pgfsys@transformshift{4.621569in}{1.556587in}%
\pgfsys@useobject{currentmarker}{}%
\end{pgfscope}%
\begin{pgfscope}%
\pgfsys@transformshift{4.613805in}{1.533975in}%
\pgfsys@useobject{currentmarker}{}%
\end{pgfscope}%
\begin{pgfscope}%
\pgfsys@transformshift{4.614853in}{1.520867in}%
\pgfsys@useobject{currentmarker}{}%
\end{pgfscope}%
\begin{pgfscope}%
\pgfsys@transformshift{4.617311in}{1.506775in}%
\pgfsys@useobject{currentmarker}{}%
\end{pgfscope}%
\begin{pgfscope}%
\pgfsys@transformshift{4.612981in}{1.487872in}%
\pgfsys@useobject{currentmarker}{}%
\end{pgfscope}%
\begin{pgfscope}%
\pgfsys@transformshift{4.609815in}{1.468122in}%
\pgfsys@useobject{currentmarker}{}%
\end{pgfscope}%
\begin{pgfscope}%
\pgfsys@transformshift{4.611962in}{1.445253in}%
\pgfsys@useobject{currentmarker}{}%
\end{pgfscope}%
\begin{pgfscope}%
\pgfsys@transformshift{4.609447in}{1.432872in}%
\pgfsys@useobject{currentmarker}{}%
\end{pgfscope}%
\begin{pgfscope}%
\pgfsys@transformshift{4.603544in}{1.420793in}%
\pgfsys@useobject{currentmarker}{}%
\end{pgfscope}%
\begin{pgfscope}%
\pgfsys@transformshift{4.606766in}{1.404459in}%
\pgfsys@useobject{currentmarker}{}%
\end{pgfscope}%
\begin{pgfscope}%
\pgfsys@transformshift{4.611053in}{1.386446in}%
\pgfsys@useobject{currentmarker}{}%
\end{pgfscope}%
\begin{pgfscope}%
\pgfsys@transformshift{4.603656in}{1.364939in}%
\pgfsys@useobject{currentmarker}{}%
\end{pgfscope}%
\begin{pgfscope}%
\pgfsys@transformshift{4.606812in}{1.352835in}%
\pgfsys@useobject{currentmarker}{}%
\end{pgfscope}%
\begin{pgfscope}%
\pgfsys@transformshift{4.610454in}{1.338569in}%
\pgfsys@useobject{currentmarker}{}%
\end{pgfscope}%
\begin{pgfscope}%
\pgfsys@transformshift{4.608448in}{1.318700in}%
\pgfsys@useobject{currentmarker}{}%
\end{pgfscope}%
\begin{pgfscope}%
\pgfsys@transformshift{4.605527in}{1.308112in}%
\pgfsys@useobject{currentmarker}{}%
\end{pgfscope}%
\begin{pgfscope}%
\pgfsys@transformshift{4.609279in}{1.295922in}%
\pgfsys@useobject{currentmarker}{}%
\end{pgfscope}%
\begin{pgfscope}%
\pgfsys@transformshift{4.609813in}{1.282648in}%
\pgfsys@useobject{currentmarker}{}%
\end{pgfscope}%
\begin{pgfscope}%
\pgfsys@transformshift{4.604849in}{1.268227in}%
\pgfsys@useobject{currentmarker}{}%
\end{pgfscope}%
\begin{pgfscope}%
\pgfsys@transformshift{4.609181in}{1.252263in}%
\pgfsys@useobject{currentmarker}{}%
\end{pgfscope}%
\begin{pgfscope}%
\pgfsys@transformshift{4.610761in}{1.243304in}%
\pgfsys@useobject{currentmarker}{}%
\end{pgfscope}%
\begin{pgfscope}%
\pgfsys@transformshift{4.607935in}{1.229900in}%
\pgfsys@useobject{currentmarker}{}%
\end{pgfscope}%
\begin{pgfscope}%
\pgfsys@transformshift{4.608780in}{1.215270in}%
\pgfsys@useobject{currentmarker}{}%
\end{pgfscope}%
\begin{pgfscope}%
\pgfsys@transformshift{4.611073in}{1.200212in}%
\pgfsys@useobject{currentmarker}{}%
\end{pgfscope}%
\begin{pgfscope}%
\pgfsys@transformshift{4.609390in}{1.180724in}%
\pgfsys@useobject{currentmarker}{}%
\end{pgfscope}%
\begin{pgfscope}%
\pgfsys@transformshift{4.605858in}{1.170563in}%
\pgfsys@useobject{currentmarker}{}%
\end{pgfscope}%
\begin{pgfscope}%
\pgfsys@transformshift{4.608391in}{1.157373in}%
\pgfsys@useobject{currentmarker}{}%
\end{pgfscope}%
\begin{pgfscope}%
\pgfsys@transformshift{4.608954in}{1.142901in}%
\pgfsys@useobject{currentmarker}{}%
\end{pgfscope}%
\begin{pgfscope}%
\pgfsys@transformshift{4.602864in}{1.125883in}%
\pgfsys@useobject{currentmarker}{}%
\end{pgfscope}%
\begin{pgfscope}%
\pgfsys@transformshift{4.605428in}{1.116278in}%
\pgfsys@useobject{currentmarker}{}%
\end{pgfscope}%
\begin{pgfscope}%
\pgfsys@transformshift{4.606646in}{1.110948in}%
\pgfsys@useobject{currentmarker}{}%
\end{pgfscope}%
\begin{pgfscope}%
\pgfsys@transformshift{4.606313in}{1.101757in}%
\pgfsys@useobject{currentmarker}{}%
\end{pgfscope}%
\begin{pgfscope}%
\pgfsys@transformshift{4.605122in}{1.096841in}%
\pgfsys@useobject{currentmarker}{}%
\end{pgfscope}%
\begin{pgfscope}%
\pgfsys@transformshift{4.606697in}{1.089212in}%
\pgfsys@useobject{currentmarker}{}%
\end{pgfscope}%
\begin{pgfscope}%
\pgfsys@transformshift{4.607430in}{1.080519in}%
\pgfsys@useobject{currentmarker}{}%
\end{pgfscope}%
\begin{pgfscope}%
\pgfsys@transformshift{4.605000in}{1.068053in}%
\pgfsys@useobject{currentmarker}{}%
\end{pgfscope}%
\begin{pgfscope}%
\pgfsys@transformshift{4.605633in}{1.061096in}%
\pgfsys@useobject{currentmarker}{}%
\end{pgfscope}%
\begin{pgfscope}%
\pgfsys@transformshift{4.606617in}{1.057383in}%
\pgfsys@useobject{currentmarker}{}%
\end{pgfscope}%
\begin{pgfscope}%
\pgfsys@transformshift{4.605389in}{1.050207in}%
\pgfsys@useobject{currentmarker}{}%
\end{pgfscope}%
\begin{pgfscope}%
\pgfsys@transformshift{4.603218in}{1.041630in}%
\pgfsys@useobject{currentmarker}{}%
\end{pgfscope}%
\begin{pgfscope}%
\pgfsys@transformshift{4.605736in}{1.031067in}%
\pgfsys@useobject{currentmarker}{}%
\end{pgfscope}%
\begin{pgfscope}%
\pgfsys@transformshift{4.609428in}{1.020007in}%
\pgfsys@useobject{currentmarker}{}%
\end{pgfscope}%
\begin{pgfscope}%
\pgfsys@transformshift{4.606028in}{1.004060in}%
\pgfsys@useobject{currentmarker}{}%
\end{pgfscope}%
\begin{pgfscope}%
\pgfsys@transformshift{4.606102in}{0.995092in}%
\pgfsys@useobject{currentmarker}{}%
\end{pgfscope}%
\begin{pgfscope}%
\pgfsys@transformshift{4.605612in}{0.983715in}%
\pgfsys@useobject{currentmarker}{}%
\end{pgfscope}%
\begin{pgfscope}%
\pgfsys@transformshift{4.608684in}{0.971724in}%
\pgfsys@useobject{currentmarker}{}%
\end{pgfscope}%
\begin{pgfscope}%
\pgfsys@transformshift{4.607062in}{0.953916in}%
\pgfsys@useobject{currentmarker}{}%
\end{pgfscope}%
\begin{pgfscope}%
\pgfsys@transformshift{4.602779in}{0.935865in}%
\pgfsys@useobject{currentmarker}{}%
\end{pgfscope}%
\begin{pgfscope}%
\pgfsys@transformshift{4.607717in}{0.915774in}%
\pgfsys@useobject{currentmarker}{}%
\end{pgfscope}%
\begin{pgfscope}%
\pgfsys@transformshift{4.607475in}{0.904397in}%
\pgfsys@useobject{currentmarker}{}%
\end{pgfscope}%
\begin{pgfscope}%
\pgfsys@transformshift{4.607669in}{0.892427in}%
\pgfsys@useobject{currentmarker}{}%
\end{pgfscope}%
\begin{pgfscope}%
\pgfsys@transformshift{4.607536in}{0.885844in}%
\pgfsys@useobject{currentmarker}{}%
\end{pgfscope}%
\begin{pgfscope}%
\pgfsys@transformshift{4.607694in}{0.882226in}%
\pgfsys@useobject{currentmarker}{}%
\end{pgfscope}%
\begin{pgfscope}%
\pgfsys@transformshift{4.607382in}{0.877983in}%
\pgfsys@useobject{currentmarker}{}%
\end{pgfscope}%
\begin{pgfscope}%
\pgfsys@transformshift{4.608043in}{0.873024in}%
\pgfsys@useobject{currentmarker}{}%
\end{pgfscope}%
\begin{pgfscope}%
\pgfsys@transformshift{4.608741in}{0.870362in}%
\pgfsys@useobject{currentmarker}{}%
\end{pgfscope}%
\begin{pgfscope}%
\pgfsys@transformshift{4.606710in}{0.863198in}%
\pgfsys@useobject{currentmarker}{}%
\end{pgfscope}%
\begin{pgfscope}%
\pgfsys@transformshift{4.607797in}{0.855173in}%
\pgfsys@useobject{currentmarker}{}%
\end{pgfscope}%
\begin{pgfscope}%
\pgfsys@transformshift{4.608554in}{0.846607in}%
\pgfsys@useobject{currentmarker}{}%
\end{pgfscope}%
\begin{pgfscope}%
\pgfsys@transformshift{4.607728in}{0.837178in}%
\pgfsys@useobject{currentmarker}{}%
\end{pgfscope}%
\begin{pgfscope}%
\pgfsys@transformshift{4.608419in}{0.826813in}%
\pgfsys@useobject{currentmarker}{}%
\end{pgfscope}%
\begin{pgfscope}%
\pgfsys@transformshift{4.607679in}{0.821148in}%
\pgfsys@useobject{currentmarker}{}%
\end{pgfscope}%
\begin{pgfscope}%
\pgfsys@transformshift{4.607678in}{0.814716in}%
\pgfsys@useobject{currentmarker}{}%
\end{pgfscope}%
\begin{pgfscope}%
\pgfsys@transformshift{4.607399in}{0.811189in}%
\pgfsys@useobject{currentmarker}{}%
\end{pgfscope}%
\begin{pgfscope}%
\pgfsys@transformshift{4.607395in}{0.809244in}%
\pgfsys@useobject{currentmarker}{}%
\end{pgfscope}%
\begin{pgfscope}%
\pgfsys@transformshift{4.607318in}{0.808176in}%
\pgfsys@useobject{currentmarker}{}%
\end{pgfscope}%
\begin{pgfscope}%
\pgfsys@transformshift{4.607211in}{0.807598in}%
\pgfsys@useobject{currentmarker}{}%
\end{pgfscope}%
\begin{pgfscope}%
\pgfsys@transformshift{4.607119in}{0.807287in}%
\pgfsys@useobject{currentmarker}{}%
\end{pgfscope}%
\begin{pgfscope}%
\pgfsys@transformshift{4.606614in}{0.806401in}%
\pgfsys@useobject{currentmarker}{}%
\end{pgfscope}%
\begin{pgfscope}%
\pgfsys@transformshift{4.605298in}{0.805087in}%
\pgfsys@useobject{currentmarker}{}%
\end{pgfscope}%
\begin{pgfscope}%
\pgfsys@transformshift{4.602630in}{0.802970in}%
\pgfsys@useobject{currentmarker}{}%
\end{pgfscope}%
\begin{pgfscope}%
\pgfsys@transformshift{4.597679in}{0.801726in}%
\pgfsys@useobject{currentmarker}{}%
\end{pgfscope}%
\begin{pgfscope}%
\pgfsys@transformshift{4.590400in}{0.800648in}%
\pgfsys@useobject{currentmarker}{}%
\end{pgfscope}%
\begin{pgfscope}%
\pgfsys@transformshift{4.586353in}{0.800702in}%
\pgfsys@useobject{currentmarker}{}%
\end{pgfscope}%
\begin{pgfscope}%
\pgfsys@transformshift{4.581754in}{0.800533in}%
\pgfsys@useobject{currentmarker}{}%
\end{pgfscope}%
\begin{pgfscope}%
\pgfsys@transformshift{4.579254in}{0.800934in}%
\pgfsys@useobject{currentmarker}{}%
\end{pgfscope}%
\begin{pgfscope}%
\pgfsys@transformshift{4.576008in}{0.801058in}%
\pgfsys@useobject{currentmarker}{}%
\end{pgfscope}%
\begin{pgfscope}%
\pgfsys@transformshift{4.572194in}{0.801610in}%
\pgfsys@useobject{currentmarker}{}%
\end{pgfscope}%
\begin{pgfscope}%
\pgfsys@transformshift{4.570076in}{0.801526in}%
\pgfsys@useobject{currentmarker}{}%
\end{pgfscope}%
\begin{pgfscope}%
\pgfsys@transformshift{4.568979in}{0.801921in}%
\pgfsys@useobject{currentmarker}{}%
\end{pgfscope}%
\begin{pgfscope}%
\pgfsys@transformshift{4.568338in}{0.801924in}%
\pgfsys@useobject{currentmarker}{}%
\end{pgfscope}%
\begin{pgfscope}%
\pgfsys@transformshift{4.567998in}{0.802017in}%
\pgfsys@useobject{currentmarker}{}%
\end{pgfscope}%
\begin{pgfscope}%
\pgfsys@transformshift{4.567804in}{0.802029in}%
\pgfsys@useobject{currentmarker}{}%
\end{pgfscope}%
\begin{pgfscope}%
\pgfsys@transformshift{4.567031in}{0.802158in}%
\pgfsys@useobject{currentmarker}{}%
\end{pgfscope}%
\begin{pgfscope}%
\pgfsys@transformshift{4.566603in}{0.802106in}%
\pgfsys@useobject{currentmarker}{}%
\end{pgfscope}%
\begin{pgfscope}%
\pgfsys@transformshift{4.564280in}{0.802360in}%
\pgfsys@useobject{currentmarker}{}%
\end{pgfscope}%
\begin{pgfscope}%
\pgfsys@transformshift{4.559322in}{0.801867in}%
\pgfsys@useobject{currentmarker}{}%
\end{pgfscope}%
\begin{pgfscope}%
\pgfsys@transformshift{4.550430in}{0.802394in}%
\pgfsys@useobject{currentmarker}{}%
\end{pgfscope}%
\begin{pgfscope}%
\pgfsys@transformshift{4.540715in}{0.802389in}%
\pgfsys@useobject{currentmarker}{}%
\end{pgfscope}%
\begin{pgfscope}%
\pgfsys@transformshift{4.530397in}{0.802742in}%
\pgfsys@useobject{currentmarker}{}%
\end{pgfscope}%
\begin{pgfscope}%
\pgfsys@transformshift{4.518836in}{0.803653in}%
\pgfsys@useobject{currentmarker}{}%
\end{pgfscope}%
\begin{pgfscope}%
\pgfsys@transformshift{4.502501in}{0.802023in}%
\pgfsys@useobject{currentmarker}{}%
\end{pgfscope}%
\begin{pgfscope}%
\pgfsys@transformshift{4.482885in}{0.801991in}%
\pgfsys@useobject{currentmarker}{}%
\end{pgfscope}%
\begin{pgfscope}%
\pgfsys@transformshift{4.462723in}{0.802743in}%
\pgfsys@useobject{currentmarker}{}%
\end{pgfscope}%
\begin{pgfscope}%
\pgfsys@transformshift{4.451679in}{0.801669in}%
\pgfsys@useobject{currentmarker}{}%
\end{pgfscope}%
\begin{pgfscope}%
\pgfsys@transformshift{4.439413in}{0.802175in}%
\pgfsys@useobject{currentmarker}{}%
\end{pgfscope}%
\begin{pgfscope}%
\pgfsys@transformshift{4.425114in}{0.800909in}%
\pgfsys@useobject{currentmarker}{}%
\end{pgfscope}%
\begin{pgfscope}%
\pgfsys@transformshift{4.417230in}{0.801332in}%
\pgfsys@useobject{currentmarker}{}%
\end{pgfscope}%
\begin{pgfscope}%
\pgfsys@transformshift{4.412890in}{0.801174in}%
\pgfsys@useobject{currentmarker}{}%
\end{pgfscope}%
\begin{pgfscope}%
\pgfsys@transformshift{4.406182in}{0.800537in}%
\pgfsys@useobject{currentmarker}{}%
\end{pgfscope}%
\begin{pgfscope}%
\pgfsys@transformshift{4.398927in}{0.799791in}%
\pgfsys@useobject{currentmarker}{}%
\end{pgfscope}%
\begin{pgfscope}%
\pgfsys@transformshift{4.390645in}{0.799589in}%
\pgfsys@useobject{currentmarker}{}%
\end{pgfscope}%
\begin{pgfscope}%
\pgfsys@transformshift{4.377326in}{0.798324in}%
\pgfsys@useobject{currentmarker}{}%
\end{pgfscope}%
\begin{pgfscope}%
\pgfsys@transformshift{4.362345in}{0.797567in}%
\pgfsys@useobject{currentmarker}{}%
\end{pgfscope}%
\begin{pgfscope}%
\pgfsys@transformshift{4.346859in}{0.796675in}%
\pgfsys@useobject{currentmarker}{}%
\end{pgfscope}%
\begin{pgfscope}%
\pgfsys@transformshift{4.325998in}{0.795868in}%
\pgfsys@useobject{currentmarker}{}%
\end{pgfscope}%
\begin{pgfscope}%
\pgfsys@transformshift{4.301202in}{0.794116in}%
\pgfsys@useobject{currentmarker}{}%
\end{pgfscope}%
\begin{pgfscope}%
\pgfsys@transformshift{4.287570in}{0.793069in}%
\pgfsys@useobject{currentmarker}{}%
\end{pgfscope}%
\begin{pgfscope}%
\pgfsys@transformshift{4.271258in}{0.791222in}%
\pgfsys@useobject{currentmarker}{}%
\end{pgfscope}%
\begin{pgfscope}%
\pgfsys@transformshift{4.251188in}{0.789528in}%
\pgfsys@useobject{currentmarker}{}%
\end{pgfscope}%
\begin{pgfscope}%
\pgfsys@transformshift{4.229827in}{0.787674in}%
\pgfsys@useobject{currentmarker}{}%
\end{pgfscope}%
\begin{pgfscope}%
\pgfsys@transformshift{4.218072in}{0.788631in}%
\pgfsys@useobject{currentmarker}{}%
\end{pgfscope}%
\begin{pgfscope}%
\pgfsys@transformshift{4.205281in}{0.789035in}%
\pgfsys@useobject{currentmarker}{}%
\end{pgfscope}%
\begin{pgfscope}%
\pgfsys@transformshift{4.189248in}{0.788144in}%
\pgfsys@useobject{currentmarker}{}%
\end{pgfscope}%
\begin{pgfscope}%
\pgfsys@transformshift{4.172570in}{0.786371in}%
\pgfsys@useobject{currentmarker}{}%
\end{pgfscope}%
\begin{pgfscope}%
\pgfsys@transformshift{4.163350in}{0.786673in}%
\pgfsys@useobject{currentmarker}{}%
\end{pgfscope}%
\begin{pgfscope}%
\pgfsys@transformshift{4.149275in}{0.784552in}%
\pgfsys@useobject{currentmarker}{}%
\end{pgfscope}%
\begin{pgfscope}%
\pgfsys@transformshift{4.132004in}{0.785669in}%
\pgfsys@useobject{currentmarker}{}%
\end{pgfscope}%
\begin{pgfscope}%
\pgfsys@transformshift{4.122487in}{0.785854in}%
\pgfsys@useobject{currentmarker}{}%
\end{pgfscope}%
\begin{pgfscope}%
\pgfsys@transformshift{4.107465in}{0.784560in}%
\pgfsys@useobject{currentmarker}{}%
\end{pgfscope}%
\begin{pgfscope}%
\pgfsys@transformshift{4.088880in}{0.783256in}%
\pgfsys@useobject{currentmarker}{}%
\end{pgfscope}%
\begin{pgfscope}%
\pgfsys@transformshift{4.078654in}{0.782610in}%
\pgfsys@useobject{currentmarker}{}%
\end{pgfscope}%
\begin{pgfscope}%
\pgfsys@transformshift{4.064786in}{0.780263in}%
\pgfsys@useobject{currentmarker}{}%
\end{pgfscope}%
\begin{pgfscope}%
\pgfsys@transformshift{4.046857in}{0.779382in}%
\pgfsys@useobject{currentmarker}{}%
\end{pgfscope}%
\begin{pgfscope}%
\pgfsys@transformshift{4.028264in}{0.778448in}%
\pgfsys@useobject{currentmarker}{}%
\end{pgfscope}%
\begin{pgfscope}%
\pgfsys@transformshift{4.006578in}{0.778550in}%
\pgfsys@useobject{currentmarker}{}%
\end{pgfscope}%
\begin{pgfscope}%
\pgfsys@transformshift{3.980629in}{0.777997in}%
\pgfsys@useobject{currentmarker}{}%
\end{pgfscope}%
\begin{pgfscope}%
\pgfsys@transformshift{3.966355in}{0.777860in}%
\pgfsys@useobject{currentmarker}{}%
\end{pgfscope}%
\begin{pgfscope}%
\pgfsys@transformshift{3.951270in}{0.776732in}%
\pgfsys@useobject{currentmarker}{}%
\end{pgfscope}%
\begin{pgfscope}%
\pgfsys@transformshift{3.932987in}{0.777603in}%
\pgfsys@useobject{currentmarker}{}%
\end{pgfscope}%
\begin{pgfscope}%
\pgfsys@transformshift{3.913314in}{0.776591in}%
\pgfsys@useobject{currentmarker}{}%
\end{pgfscope}%
\begin{pgfscope}%
\pgfsys@transformshift{3.902480in}{0.776745in}%
\pgfsys@useobject{currentmarker}{}%
\end{pgfscope}%
\begin{pgfscope}%
\pgfsys@transformshift{3.887950in}{0.776788in}%
\pgfsys@useobject{currentmarker}{}%
\end{pgfscope}%
\begin{pgfscope}%
\pgfsys@transformshift{3.872337in}{0.777442in}%
\pgfsys@useobject{currentmarker}{}%
\end{pgfscope}%
\begin{pgfscope}%
\pgfsys@transformshift{3.863757in}{0.777947in}%
\pgfsys@useobject{currentmarker}{}%
\end{pgfscope}%
\begin{pgfscope}%
\pgfsys@transformshift{3.850725in}{0.778329in}%
\pgfsys@useobject{currentmarker}{}%
\end{pgfscope}%
\begin{pgfscope}%
\pgfsys@transformshift{3.836998in}{0.778815in}%
\pgfsys@useobject{currentmarker}{}%
\end{pgfscope}%
\begin{pgfscope}%
\pgfsys@transformshift{3.822283in}{0.777902in}%
\pgfsys@useobject{currentmarker}{}%
\end{pgfscope}%
\begin{pgfscope}%
\pgfsys@transformshift{3.802439in}{0.776887in}%
\pgfsys@useobject{currentmarker}{}%
\end{pgfscope}%
\begin{pgfscope}%
\pgfsys@transformshift{3.781987in}{0.777037in}%
\pgfsys@useobject{currentmarker}{}%
\end{pgfscope}%
\begin{pgfscope}%
\pgfsys@transformshift{3.759669in}{0.775845in}%
\pgfsys@useobject{currentmarker}{}%
\end{pgfscope}%
\begin{pgfscope}%
\pgfsys@transformshift{3.732339in}{0.775765in}%
\pgfsys@useobject{currentmarker}{}%
\end{pgfscope}%
\begin{pgfscope}%
\pgfsys@transformshift{3.704413in}{0.775547in}%
\pgfsys@useobject{currentmarker}{}%
\end{pgfscope}%
\begin{pgfscope}%
\pgfsys@transformshift{3.675537in}{0.775087in}%
\pgfsys@useobject{currentmarker}{}%
\end{pgfscope}%
\begin{pgfscope}%
\pgfsys@transformshift{3.643324in}{0.777561in}%
\pgfsys@useobject{currentmarker}{}%
\end{pgfscope}%
\begin{pgfscope}%
\pgfsys@transformshift{3.609889in}{0.778037in}%
\pgfsys@useobject{currentmarker}{}%
\end{pgfscope}%
\begin{pgfscope}%
\pgfsys@transformshift{3.591507in}{0.777486in}%
\pgfsys@useobject{currentmarker}{}%
\end{pgfscope}%
\begin{pgfscope}%
\pgfsys@transformshift{3.569699in}{0.780040in}%
\pgfsys@useobject{currentmarker}{}%
\end{pgfscope}%
\begin{pgfscope}%
\pgfsys@transformshift{3.545943in}{0.778986in}%
\pgfsys@useobject{currentmarker}{}%
\end{pgfscope}%
\begin{pgfscope}%
\pgfsys@transformshift{3.532975in}{0.780686in}%
\pgfsys@useobject{currentmarker}{}%
\end{pgfscope}%
\begin{pgfscope}%
\pgfsys@transformshift{3.519190in}{0.782881in}%
\pgfsys@useobject{currentmarker}{}%
\end{pgfscope}%
\begin{pgfscope}%
\pgfsys@transformshift{3.499769in}{0.781421in}%
\pgfsys@useobject{currentmarker}{}%
\end{pgfscope}%
\begin{pgfscope}%
\pgfsys@transformshift{3.478708in}{0.781400in}%
\pgfsys@useobject{currentmarker}{}%
\end{pgfscope}%
\begin{pgfscope}%
\pgfsys@transformshift{3.467125in}{0.781369in}%
\pgfsys@useobject{currentmarker}{}%
\end{pgfscope}%
\begin{pgfscope}%
\pgfsys@transformshift{3.451541in}{0.780531in}%
\pgfsys@useobject{currentmarker}{}%
\end{pgfscope}%
\begin{pgfscope}%
\pgfsys@transformshift{3.433017in}{0.781903in}%
\pgfsys@useobject{currentmarker}{}%
\end{pgfscope}%
\begin{pgfscope}%
\pgfsys@transformshift{3.422802in}{0.782102in}%
\pgfsys@useobject{currentmarker}{}%
\end{pgfscope}%
\begin{pgfscope}%
\pgfsys@transformshift{3.409408in}{0.782503in}%
\pgfsys@useobject{currentmarker}{}%
\end{pgfscope}%
\begin{pgfscope}%
\pgfsys@transformshift{3.390683in}{0.783697in}%
\pgfsys@useobject{currentmarker}{}%
\end{pgfscope}%
\begin{pgfscope}%
\pgfsys@transformshift{3.370254in}{0.783341in}%
\pgfsys@useobject{currentmarker}{}%
\end{pgfscope}%
\begin{pgfscope}%
\pgfsys@transformshift{3.359048in}{0.784193in}%
\pgfsys@useobject{currentmarker}{}%
\end{pgfscope}%
\begin{pgfscope}%
\pgfsys@transformshift{3.343474in}{0.786101in}%
\pgfsys@useobject{currentmarker}{}%
\end{pgfscope}%
\begin{pgfscope}%
\pgfsys@transformshift{3.325814in}{0.784921in}%
\pgfsys@useobject{currentmarker}{}%
\end{pgfscope}%
\begin{pgfscope}%
\pgfsys@transformshift{3.316082in}{0.785162in}%
\pgfsys@useobject{currentmarker}{}%
\end{pgfscope}%
\begin{pgfscope}%
\pgfsys@transformshift{3.303676in}{0.786266in}%
\pgfsys@useobject{currentmarker}{}%
\end{pgfscope}%
\begin{pgfscope}%
\pgfsys@transformshift{3.286113in}{0.784651in}%
\pgfsys@useobject{currentmarker}{}%
\end{pgfscope}%
\begin{pgfscope}%
\pgfsys@transformshift{3.265853in}{0.784655in}%
\pgfsys@useobject{currentmarker}{}%
\end{pgfscope}%
\begin{pgfscope}%
\pgfsys@transformshift{3.254728in}{0.785285in}%
\pgfsys@useobject{currentmarker}{}%
\end{pgfscope}%
\begin{pgfscope}%
\pgfsys@transformshift{3.239547in}{0.783978in}%
\pgfsys@useobject{currentmarker}{}%
\end{pgfscope}%
\begin{pgfscope}%
\pgfsys@transformshift{3.219530in}{0.784435in}%
\pgfsys@useobject{currentmarker}{}%
\end{pgfscope}%
\begin{pgfscope}%
\pgfsys@transformshift{3.198666in}{0.783815in}%
\pgfsys@useobject{currentmarker}{}%
\end{pgfscope}%
\begin{pgfscope}%
\pgfsys@transformshift{3.176318in}{0.783049in}%
\pgfsys@useobject{currentmarker}{}%
\end{pgfscope}%
\begin{pgfscope}%
\pgfsys@transformshift{3.150382in}{0.782109in}%
\pgfsys@useobject{currentmarker}{}%
\end{pgfscope}%
\begin{pgfscope}%
\pgfsys@transformshift{3.121894in}{0.782155in}%
\pgfsys@useobject{currentmarker}{}%
\end{pgfscope}%
\begin{pgfscope}%
\pgfsys@transformshift{3.092740in}{0.776785in}%
\pgfsys@useobject{currentmarker}{}%
\end{pgfscope}%
\begin{pgfscope}%
\pgfsys@transformshift{3.061710in}{0.775343in}%
\pgfsys@useobject{currentmarker}{}%
\end{pgfscope}%
\begin{pgfscope}%
\pgfsys@transformshift{3.024983in}{0.773416in}%
\pgfsys@useobject{currentmarker}{}%
\end{pgfscope}%
\begin{pgfscope}%
\pgfsys@transformshift{2.988254in}{0.764906in}%
\pgfsys@useobject{currentmarker}{}%
\end{pgfscope}%
\begin{pgfscope}%
\pgfsys@transformshift{2.967643in}{0.762635in}%
\pgfsys@useobject{currentmarker}{}%
\end{pgfscope}%
\begin{pgfscope}%
\pgfsys@transformshift{2.942810in}{0.759900in}%
\pgfsys@useobject{currentmarker}{}%
\end{pgfscope}%
\begin{pgfscope}%
\pgfsys@transformshift{2.913140in}{0.754636in}%
\pgfsys@useobject{currentmarker}{}%
\end{pgfscope}%
\begin{pgfscope}%
\pgfsys@transformshift{2.882199in}{0.752169in}%
\pgfsys@useobject{currentmarker}{}%
\end{pgfscope}%
\begin{pgfscope}%
\pgfsys@transformshift{2.847977in}{0.748793in}%
\pgfsys@useobject{currentmarker}{}%
\end{pgfscope}%
\begin{pgfscope}%
\pgfsys@transformshift{2.808151in}{0.740489in}%
\pgfsys@useobject{currentmarker}{}%
\end{pgfscope}%
\begin{pgfscope}%
\pgfsys@transformshift{2.766907in}{0.738676in}%
\pgfsys@useobject{currentmarker}{}%
\end{pgfscope}%
\begin{pgfscope}%
\pgfsys@transformshift{2.744223in}{0.737670in}%
\pgfsys@useobject{currentmarker}{}%
\end{pgfscope}%
\begin{pgfscope}%
\pgfsys@transformshift{2.719506in}{0.737435in}%
\pgfsys@useobject{currentmarker}{}%
\end{pgfscope}%
\begin{pgfscope}%
\pgfsys@transformshift{2.690500in}{0.733187in}%
\pgfsys@useobject{currentmarker}{}%
\end{pgfscope}%
\begin{pgfscope}%
\pgfsys@transformshift{2.660332in}{0.730736in}%
\pgfsys@useobject{currentmarker}{}%
\end{pgfscope}%
\begin{pgfscope}%
\pgfsys@transformshift{2.643685in}{0.730764in}%
\pgfsys@useobject{currentmarker}{}%
\end{pgfscope}%
\begin{pgfscope}%
\pgfsys@transformshift{2.622176in}{0.729139in}%
\pgfsys@useobject{currentmarker}{}%
\end{pgfscope}%
\begin{pgfscope}%
\pgfsys@transformshift{2.597272in}{0.727989in}%
\pgfsys@useobject{currentmarker}{}%
\end{pgfscope}%
\begin{pgfscope}%
\pgfsys@transformshift{2.571634in}{0.727229in}%
\pgfsys@useobject{currentmarker}{}%
\end{pgfscope}%
\begin{pgfscope}%
\pgfsys@transformshift{2.542459in}{0.723786in}%
\pgfsys@useobject{currentmarker}{}%
\end{pgfscope}%
\begin{pgfscope}%
\pgfsys@transformshift{2.508717in}{0.717273in}%
\pgfsys@useobject{currentmarker}{}%
\end{pgfscope}%
\begin{pgfscope}%
\pgfsys@transformshift{2.473915in}{0.714096in}%
\pgfsys@useobject{currentmarker}{}%
\end{pgfscope}%
\begin{pgfscope}%
\pgfsys@transformshift{2.436657in}{0.710371in}%
\pgfsys@useobject{currentmarker}{}%
\end{pgfscope}%
\begin{pgfscope}%
\pgfsys@transformshift{2.395185in}{0.706860in}%
\pgfsys@useobject{currentmarker}{}%
\end{pgfscope}%
\begin{pgfscope}%
\pgfsys@transformshift{2.352850in}{0.705965in}%
\pgfsys@useobject{currentmarker}{}%
\end{pgfscope}%
\begin{pgfscope}%
\pgfsys@transformshift{2.309946in}{0.705921in}%
\pgfsys@useobject{currentmarker}{}%
\end{pgfscope}%
\begin{pgfscope}%
\pgfsys@transformshift{2.263340in}{0.705266in}%
\pgfsys@useobject{currentmarker}{}%
\end{pgfscope}%
\begin{pgfscope}%
\pgfsys@transformshift{2.213870in}{0.706051in}%
\pgfsys@useobject{currentmarker}{}%
\end{pgfscope}%
\begin{pgfscope}%
\pgfsys@transformshift{2.186976in}{0.701900in}%
\pgfsys@useobject{currentmarker}{}%
\end{pgfscope}%
\begin{pgfscope}%
\pgfsys@transformshift{2.156403in}{0.701550in}%
\pgfsys@useobject{currentmarker}{}%
\end{pgfscope}%
\begin{pgfscope}%
\pgfsys@transformshift{2.121774in}{0.699913in}%
\pgfsys@useobject{currentmarker}{}%
\end{pgfscope}%
\begin{pgfscope}%
\pgfsys@transformshift{2.086242in}{0.701110in}%
\pgfsys@useobject{currentmarker}{}%
\end{pgfscope}%
\begin{pgfscope}%
\pgfsys@transformshift{2.050104in}{0.702271in}%
\pgfsys@useobject{currentmarker}{}%
\end{pgfscope}%
\begin{pgfscope}%
\pgfsys@transformshift{2.008653in}{0.698833in}%
\pgfsys@useobject{currentmarker}{}%
\end{pgfscope}%
\begin{pgfscope}%
\pgfsys@transformshift{1.965170in}{0.696756in}%
\pgfsys@useobject{currentmarker}{}%
\end{pgfscope}%
\begin{pgfscope}%
\pgfsys@transformshift{1.921073in}{0.696767in}%
\pgfsys@useobject{currentmarker}{}%
\end{pgfscope}%
\begin{pgfscope}%
\pgfsys@transformshift{1.873837in}{0.692440in}%
\pgfsys@useobject{currentmarker}{}%
\end{pgfscope}%
\begin{pgfscope}%
\pgfsys@transformshift{1.821496in}{0.686702in}%
\pgfsys@useobject{currentmarker}{}%
\end{pgfscope}%
\begin{pgfscope}%
\pgfsys@transformshift{1.768155in}{0.686288in}%
\pgfsys@useobject{currentmarker}{}%
\end{pgfscope}%
\begin{pgfscope}%
\pgfsys@transformshift{1.739002in}{0.683000in}%
\pgfsys@useobject{currentmarker}{}%
\end{pgfscope}%
\begin{pgfscope}%
\pgfsys@transformshift{1.705061in}{0.685225in}%
\pgfsys@useobject{currentmarker}{}%
\end{pgfscope}%
\begin{pgfscope}%
\pgfsys@transformshift{1.668141in}{0.686764in}%
\pgfsys@useobject{currentmarker}{}%
\end{pgfscope}%
\begin{pgfscope}%
\pgfsys@transformshift{1.647865in}{0.688151in}%
\pgfsys@useobject{currentmarker}{}%
\end{pgfscope}%
\begin{pgfscope}%
\pgfsys@transformshift{1.624343in}{0.690592in}%
\pgfsys@useobject{currentmarker}{}%
\end{pgfscope}%
\begin{pgfscope}%
\pgfsys@transformshift{1.597414in}{0.690145in}%
\pgfsys@useobject{currentmarker}{}%
\end{pgfscope}%
\begin{pgfscope}%
\pgfsys@transformshift{1.569879in}{0.690350in}%
\pgfsys@useobject{currentmarker}{}%
\end{pgfscope}%
\begin{pgfscope}%
\pgfsys@transformshift{1.541656in}{0.692297in}%
\pgfsys@useobject{currentmarker}{}%
\end{pgfscope}%
\begin{pgfscope}%
\pgfsys@transformshift{1.526107in}{0.691698in}%
\pgfsys@useobject{currentmarker}{}%
\end{pgfscope}%
\begin{pgfscope}%
\pgfsys@transformshift{1.509940in}{0.689599in}%
\pgfsys@useobject{currentmarker}{}%
\end{pgfscope}%
\begin{pgfscope}%
\pgfsys@transformshift{1.490027in}{0.689807in}%
\pgfsys@useobject{currentmarker}{}%
\end{pgfscope}%
\begin{pgfscope}%
\pgfsys@transformshift{1.466341in}{0.688033in}%
\pgfsys@useobject{currentmarker}{}%
\end{pgfscope}%
\begin{pgfscope}%
\pgfsys@transformshift{1.453291in}{0.688610in}%
\pgfsys@useobject{currentmarker}{}%
\end{pgfscope}%
\begin{pgfscope}%
\pgfsys@transformshift{1.436107in}{0.689517in}%
\pgfsys@useobject{currentmarker}{}%
\end{pgfscope}%
\begin{pgfscope}%
\pgfsys@transformshift{1.415665in}{0.688399in}%
\pgfsys@useobject{currentmarker}{}%
\end{pgfscope}%
\begin{pgfscope}%
\pgfsys@transformshift{1.404441in}{0.687502in}%
\pgfsys@useobject{currentmarker}{}%
\end{pgfscope}%
\begin{pgfscope}%
\pgfsys@transformshift{1.390848in}{0.688769in}%
\pgfsys@useobject{currentmarker}{}%
\end{pgfscope}%
\begin{pgfscope}%
\pgfsys@transformshift{1.383314in}{0.686992in}%
\pgfsys@useobject{currentmarker}{}%
\end{pgfscope}%
\begin{pgfscope}%
\pgfsys@transformshift{1.403703in}{0.686584in}%
\pgfsys@useobject{currentmarker}{}%
\end{pgfscope}%
\begin{pgfscope}%
\pgfsys@transformshift{1.424718in}{0.686575in}%
\pgfsys@useobject{currentmarker}{}%
\end{pgfscope}%
\begin{pgfscope}%
\pgfsys@transformshift{1.446656in}{0.687082in}%
\pgfsys@useobject{currentmarker}{}%
\end{pgfscope}%
\begin{pgfscope}%
\pgfsys@transformshift{1.472300in}{0.687015in}%
\pgfsys@useobject{currentmarker}{}%
\end{pgfscope}%
\begin{pgfscope}%
\pgfsys@transformshift{1.498791in}{0.688386in}%
\pgfsys@useobject{currentmarker}{}%
\end{pgfscope}%
\begin{pgfscope}%
\pgfsys@transformshift{1.513351in}{0.687467in}%
\pgfsys@useobject{currentmarker}{}%
\end{pgfscope}%
\begin{pgfscope}%
\pgfsys@transformshift{1.521375in}{0.687485in}%
\pgfsys@useobject{currentmarker}{}%
\end{pgfscope}%
\begin{pgfscope}%
\pgfsys@transformshift{1.525786in}{0.687639in}%
\pgfsys@useobject{currentmarker}{}%
\end{pgfscope}%
\begin{pgfscope}%
\pgfsys@transformshift{1.528212in}{0.687710in}%
\pgfsys@useobject{currentmarker}{}%
\end{pgfscope}%
\begin{pgfscope}%
\pgfsys@transformshift{1.533033in}{0.687896in}%
\pgfsys@useobject{currentmarker}{}%
\end{pgfscope}%
\begin{pgfscope}%
\pgfsys@transformshift{1.535379in}{0.689136in}%
\pgfsys@useobject{currentmarker}{}%
\end{pgfscope}%
\begin{pgfscope}%
\pgfsys@transformshift{1.539233in}{0.690943in}%
\pgfsys@useobject{currentmarker}{}%
\end{pgfscope}%
\begin{pgfscope}%
\pgfsys@transformshift{1.541349in}{0.691945in}%
\pgfsys@useobject{currentmarker}{}%
\end{pgfscope}%
\begin{pgfscope}%
\pgfsys@transformshift{1.545194in}{0.693570in}%
\pgfsys@useobject{currentmarker}{}%
\end{pgfscope}%
\begin{pgfscope}%
\pgfsys@transformshift{1.550033in}{0.695615in}%
\pgfsys@useobject{currentmarker}{}%
\end{pgfscope}%
\end{pgfscope}%
\begin{pgfscope}%
\pgfsetbuttcap%
\pgfsetroundjoin%
\definecolor{currentfill}{rgb}{0.000000,0.000000,0.000000}%
\pgfsetfillcolor{currentfill}%
\pgfsetlinewidth{0.803000pt}%
\definecolor{currentstroke}{rgb}{0.000000,0.000000,0.000000}%
\pgfsetstrokecolor{currentstroke}%
\pgfsetdash{}{0pt}%
\pgfsys@defobject{currentmarker}{\pgfqpoint{0.000000in}{-0.048611in}}{\pgfqpoint{0.000000in}{0.000000in}}{%
\pgfpathmoveto{\pgfqpoint{0.000000in}{0.000000in}}%
\pgfpathlineto{\pgfqpoint{0.000000in}{-0.048611in}}%
\pgfusepath{stroke,fill}%
}%
\begin{pgfscope}%
\pgfsys@transformshift{1.378257in}{0.515000in}%
\pgfsys@useobject{currentmarker}{}%
\end{pgfscope}%
\end{pgfscope}%
\begin{pgfscope}%
\definecolor{textcolor}{rgb}{0.000000,0.000000,0.000000}%
\pgfsetstrokecolor{textcolor}%
\pgfsetfillcolor{textcolor}%
\pgftext[x=1.378257in,y=0.417777in,,top]{\color{textcolor}\rmfamily\fontsize{10.000000}{12.000000}\selectfont \(\displaystyle {0}\)}%
\end{pgfscope}%
\begin{pgfscope}%
\pgfsetbuttcap%
\pgfsetroundjoin%
\definecolor{currentfill}{rgb}{0.000000,0.000000,0.000000}%
\pgfsetfillcolor{currentfill}%
\pgfsetlinewidth{0.803000pt}%
\definecolor{currentstroke}{rgb}{0.000000,0.000000,0.000000}%
\pgfsetstrokecolor{currentstroke}%
\pgfsetdash{}{0pt}%
\pgfsys@defobject{currentmarker}{\pgfqpoint{0.000000in}{-0.048611in}}{\pgfqpoint{0.000000in}{0.000000in}}{%
\pgfpathmoveto{\pgfqpoint{0.000000in}{0.000000in}}%
\pgfpathlineto{\pgfqpoint{0.000000in}{-0.048611in}}%
\pgfusepath{stroke,fill}%
}%
\begin{pgfscope}%
\pgfsys@transformshift{2.372504in}{0.515000in}%
\pgfsys@useobject{currentmarker}{}%
\end{pgfscope}%
\end{pgfscope}%
\begin{pgfscope}%
\definecolor{textcolor}{rgb}{0.000000,0.000000,0.000000}%
\pgfsetstrokecolor{textcolor}%
\pgfsetfillcolor{textcolor}%
\pgftext[x=2.372504in,y=0.417777in,,top]{\color{textcolor}\rmfamily\fontsize{10.000000}{12.000000}\selectfont \(\displaystyle {10}\)}%
\end{pgfscope}%
\begin{pgfscope}%
\pgfsetbuttcap%
\pgfsetroundjoin%
\definecolor{currentfill}{rgb}{0.000000,0.000000,0.000000}%
\pgfsetfillcolor{currentfill}%
\pgfsetlinewidth{0.803000pt}%
\definecolor{currentstroke}{rgb}{0.000000,0.000000,0.000000}%
\pgfsetstrokecolor{currentstroke}%
\pgfsetdash{}{0pt}%
\pgfsys@defobject{currentmarker}{\pgfqpoint{0.000000in}{-0.048611in}}{\pgfqpoint{0.000000in}{0.000000in}}{%
\pgfpathmoveto{\pgfqpoint{0.000000in}{0.000000in}}%
\pgfpathlineto{\pgfqpoint{0.000000in}{-0.048611in}}%
\pgfusepath{stroke,fill}%
}%
\begin{pgfscope}%
\pgfsys@transformshift{3.366752in}{0.515000in}%
\pgfsys@useobject{currentmarker}{}%
\end{pgfscope}%
\end{pgfscope}%
\begin{pgfscope}%
\definecolor{textcolor}{rgb}{0.000000,0.000000,0.000000}%
\pgfsetstrokecolor{textcolor}%
\pgfsetfillcolor{textcolor}%
\pgftext[x=3.366752in,y=0.417777in,,top]{\color{textcolor}\rmfamily\fontsize{10.000000}{12.000000}\selectfont \(\displaystyle {20}\)}%
\end{pgfscope}%
\begin{pgfscope}%
\pgfsetbuttcap%
\pgfsetroundjoin%
\definecolor{currentfill}{rgb}{0.000000,0.000000,0.000000}%
\pgfsetfillcolor{currentfill}%
\pgfsetlinewidth{0.803000pt}%
\definecolor{currentstroke}{rgb}{0.000000,0.000000,0.000000}%
\pgfsetstrokecolor{currentstroke}%
\pgfsetdash{}{0pt}%
\pgfsys@defobject{currentmarker}{\pgfqpoint{0.000000in}{-0.048611in}}{\pgfqpoint{0.000000in}{0.000000in}}{%
\pgfpathmoveto{\pgfqpoint{0.000000in}{0.000000in}}%
\pgfpathlineto{\pgfqpoint{0.000000in}{-0.048611in}}%
\pgfusepath{stroke,fill}%
}%
\begin{pgfscope}%
\pgfsys@transformshift{4.360999in}{0.515000in}%
\pgfsys@useobject{currentmarker}{}%
\end{pgfscope}%
\end{pgfscope}%
\begin{pgfscope}%
\definecolor{textcolor}{rgb}{0.000000,0.000000,0.000000}%
\pgfsetstrokecolor{textcolor}%
\pgfsetfillcolor{textcolor}%
\pgftext[x=4.360999in,y=0.417777in,,top]{\color{textcolor}\rmfamily\fontsize{10.000000}{12.000000}\selectfont \(\displaystyle {30}\)}%
\end{pgfscope}%
\begin{pgfscope}%
\pgfsetbuttcap%
\pgfsetroundjoin%
\definecolor{currentfill}{rgb}{0.000000,0.000000,0.000000}%
\pgfsetfillcolor{currentfill}%
\pgfsetlinewidth{0.803000pt}%
\definecolor{currentstroke}{rgb}{0.000000,0.000000,0.000000}%
\pgfsetstrokecolor{currentstroke}%
\pgfsetdash{}{0pt}%
\pgfsys@defobject{currentmarker}{\pgfqpoint{0.000000in}{-0.048611in}}{\pgfqpoint{0.000000in}{0.000000in}}{%
\pgfpathmoveto{\pgfqpoint{0.000000in}{0.000000in}}%
\pgfpathlineto{\pgfqpoint{0.000000in}{-0.048611in}}%
\pgfusepath{stroke,fill}%
}%
\begin{pgfscope}%
\pgfsys@transformshift{5.355247in}{0.515000in}%
\pgfsys@useobject{currentmarker}{}%
\end{pgfscope}%
\end{pgfscope}%
\begin{pgfscope}%
\definecolor{textcolor}{rgb}{0.000000,0.000000,0.000000}%
\pgfsetstrokecolor{textcolor}%
\pgfsetfillcolor{textcolor}%
\pgftext[x=5.355247in,y=0.417777in,,top]{\color{textcolor}\rmfamily\fontsize{10.000000}{12.000000}\selectfont \(\displaystyle {40}\)}%
\end{pgfscope}%
\begin{pgfscope}%
\definecolor{textcolor}{rgb}{0.000000,0.000000,0.000000}%
\pgfsetstrokecolor{textcolor}%
\pgfsetfillcolor{textcolor}%
\pgftext[x=3.010556in,y=0.238889in,,top]{\color{textcolor}\rmfamily\fontsize{10.000000}{12.000000}\selectfont Position X [\(\displaystyle m\)]}%
\end{pgfscope}%
\begin{pgfscope}%
\pgfsetbuttcap%
\pgfsetroundjoin%
\definecolor{currentfill}{rgb}{0.000000,0.000000,0.000000}%
\pgfsetfillcolor{currentfill}%
\pgfsetlinewidth{0.803000pt}%
\definecolor{currentstroke}{rgb}{0.000000,0.000000,0.000000}%
\pgfsetstrokecolor{currentstroke}%
\pgfsetdash{}{0pt}%
\pgfsys@defobject{currentmarker}{\pgfqpoint{-0.048611in}{0.000000in}}{\pgfqpoint{-0.000000in}{0.000000in}}{%
\pgfpathmoveto{\pgfqpoint{-0.000000in}{0.000000in}}%
\pgfpathlineto{\pgfqpoint{-0.048611in}{0.000000in}}%
\pgfusepath{stroke,fill}%
}%
\begin{pgfscope}%
\pgfsys@transformshift{0.530556in}{0.729800in}%
\pgfsys@useobject{currentmarker}{}%
\end{pgfscope}%
\end{pgfscope}%
\begin{pgfscope}%
\definecolor{textcolor}{rgb}{0.000000,0.000000,0.000000}%
\pgfsetstrokecolor{textcolor}%
\pgfsetfillcolor{textcolor}%
\pgftext[x=0.363889in, y=0.681606in, left, base]{\color{textcolor}\rmfamily\fontsize{10.000000}{12.000000}\selectfont \(\displaystyle {0}\)}%
\end{pgfscope}%
\begin{pgfscope}%
\pgfsetbuttcap%
\pgfsetroundjoin%
\definecolor{currentfill}{rgb}{0.000000,0.000000,0.000000}%
\pgfsetfillcolor{currentfill}%
\pgfsetlinewidth{0.803000pt}%
\definecolor{currentstroke}{rgb}{0.000000,0.000000,0.000000}%
\pgfsetstrokecolor{currentstroke}%
\pgfsetdash{}{0pt}%
\pgfsys@defobject{currentmarker}{\pgfqpoint{-0.048611in}{0.000000in}}{\pgfqpoint{-0.000000in}{0.000000in}}{%
\pgfpathmoveto{\pgfqpoint{-0.000000in}{0.000000in}}%
\pgfpathlineto{\pgfqpoint{-0.048611in}{0.000000in}}%
\pgfusepath{stroke,fill}%
}%
\begin{pgfscope}%
\pgfsys@transformshift{0.530556in}{1.226924in}%
\pgfsys@useobject{currentmarker}{}%
\end{pgfscope}%
\end{pgfscope}%
\begin{pgfscope}%
\definecolor{textcolor}{rgb}{0.000000,0.000000,0.000000}%
\pgfsetstrokecolor{textcolor}%
\pgfsetfillcolor{textcolor}%
\pgftext[x=0.363889in, y=1.178729in, left, base]{\color{textcolor}\rmfamily\fontsize{10.000000}{12.000000}\selectfont \(\displaystyle {5}\)}%
\end{pgfscope}%
\begin{pgfscope}%
\pgfsetbuttcap%
\pgfsetroundjoin%
\definecolor{currentfill}{rgb}{0.000000,0.000000,0.000000}%
\pgfsetfillcolor{currentfill}%
\pgfsetlinewidth{0.803000pt}%
\definecolor{currentstroke}{rgb}{0.000000,0.000000,0.000000}%
\pgfsetstrokecolor{currentstroke}%
\pgfsetdash{}{0pt}%
\pgfsys@defobject{currentmarker}{\pgfqpoint{-0.048611in}{0.000000in}}{\pgfqpoint{-0.000000in}{0.000000in}}{%
\pgfpathmoveto{\pgfqpoint{-0.000000in}{0.000000in}}%
\pgfpathlineto{\pgfqpoint{-0.048611in}{0.000000in}}%
\pgfusepath{stroke,fill}%
}%
\begin{pgfscope}%
\pgfsys@transformshift{0.530556in}{1.724047in}%
\pgfsys@useobject{currentmarker}{}%
\end{pgfscope}%
\end{pgfscope}%
\begin{pgfscope}%
\definecolor{textcolor}{rgb}{0.000000,0.000000,0.000000}%
\pgfsetstrokecolor{textcolor}%
\pgfsetfillcolor{textcolor}%
\pgftext[x=0.294444in, y=1.675853in, left, base]{\color{textcolor}\rmfamily\fontsize{10.000000}{12.000000}\selectfont \(\displaystyle {10}\)}%
\end{pgfscope}%
\begin{pgfscope}%
\pgfsetbuttcap%
\pgfsetroundjoin%
\definecolor{currentfill}{rgb}{0.000000,0.000000,0.000000}%
\pgfsetfillcolor{currentfill}%
\pgfsetlinewidth{0.803000pt}%
\definecolor{currentstroke}{rgb}{0.000000,0.000000,0.000000}%
\pgfsetstrokecolor{currentstroke}%
\pgfsetdash{}{0pt}%
\pgfsys@defobject{currentmarker}{\pgfqpoint{-0.048611in}{0.000000in}}{\pgfqpoint{-0.000000in}{0.000000in}}{%
\pgfpathmoveto{\pgfqpoint{-0.000000in}{0.000000in}}%
\pgfpathlineto{\pgfqpoint{-0.048611in}{0.000000in}}%
\pgfusepath{stroke,fill}%
}%
\begin{pgfscope}%
\pgfsys@transformshift{0.530556in}{2.221171in}%
\pgfsys@useobject{currentmarker}{}%
\end{pgfscope}%
\end{pgfscope}%
\begin{pgfscope}%
\definecolor{textcolor}{rgb}{0.000000,0.000000,0.000000}%
\pgfsetstrokecolor{textcolor}%
\pgfsetfillcolor{textcolor}%
\pgftext[x=0.294444in, y=2.172977in, left, base]{\color{textcolor}\rmfamily\fontsize{10.000000}{12.000000}\selectfont \(\displaystyle {15}\)}%
\end{pgfscope}%
\begin{pgfscope}%
\pgfsetbuttcap%
\pgfsetroundjoin%
\definecolor{currentfill}{rgb}{0.000000,0.000000,0.000000}%
\pgfsetfillcolor{currentfill}%
\pgfsetlinewidth{0.803000pt}%
\definecolor{currentstroke}{rgb}{0.000000,0.000000,0.000000}%
\pgfsetstrokecolor{currentstroke}%
\pgfsetdash{}{0pt}%
\pgfsys@defobject{currentmarker}{\pgfqpoint{-0.048611in}{0.000000in}}{\pgfqpoint{-0.000000in}{0.000000in}}{%
\pgfpathmoveto{\pgfqpoint{-0.000000in}{0.000000in}}%
\pgfpathlineto{\pgfqpoint{-0.048611in}{0.000000in}}%
\pgfusepath{stroke,fill}%
}%
\begin{pgfscope}%
\pgfsys@transformshift{0.530556in}{2.718295in}%
\pgfsys@useobject{currentmarker}{}%
\end{pgfscope}%
\end{pgfscope}%
\begin{pgfscope}%
\definecolor{textcolor}{rgb}{0.000000,0.000000,0.000000}%
\pgfsetstrokecolor{textcolor}%
\pgfsetfillcolor{textcolor}%
\pgftext[x=0.294444in, y=2.670101in, left, base]{\color{textcolor}\rmfamily\fontsize{10.000000}{12.000000}\selectfont \(\displaystyle {20}\)}%
\end{pgfscope}%
\begin{pgfscope}%
\pgfsetbuttcap%
\pgfsetroundjoin%
\definecolor{currentfill}{rgb}{0.000000,0.000000,0.000000}%
\pgfsetfillcolor{currentfill}%
\pgfsetlinewidth{0.803000pt}%
\definecolor{currentstroke}{rgb}{0.000000,0.000000,0.000000}%
\pgfsetstrokecolor{currentstroke}%
\pgfsetdash{}{0pt}%
\pgfsys@defobject{currentmarker}{\pgfqpoint{-0.048611in}{0.000000in}}{\pgfqpoint{-0.000000in}{0.000000in}}{%
\pgfpathmoveto{\pgfqpoint{-0.000000in}{0.000000in}}%
\pgfpathlineto{\pgfqpoint{-0.048611in}{0.000000in}}%
\pgfusepath{stroke,fill}%
}%
\begin{pgfscope}%
\pgfsys@transformshift{0.530556in}{3.215419in}%
\pgfsys@useobject{currentmarker}{}%
\end{pgfscope}%
\end{pgfscope}%
\begin{pgfscope}%
\definecolor{textcolor}{rgb}{0.000000,0.000000,0.000000}%
\pgfsetstrokecolor{textcolor}%
\pgfsetfillcolor{textcolor}%
\pgftext[x=0.294444in, y=3.167224in, left, base]{\color{textcolor}\rmfamily\fontsize{10.000000}{12.000000}\selectfont \(\displaystyle {25}\)}%
\end{pgfscope}%
\begin{pgfscope}%
\pgfsetbuttcap%
\pgfsetroundjoin%
\definecolor{currentfill}{rgb}{0.000000,0.000000,0.000000}%
\pgfsetfillcolor{currentfill}%
\pgfsetlinewidth{0.803000pt}%
\definecolor{currentstroke}{rgb}{0.000000,0.000000,0.000000}%
\pgfsetstrokecolor{currentstroke}%
\pgfsetdash{}{0pt}%
\pgfsys@defobject{currentmarker}{\pgfqpoint{-0.048611in}{0.000000in}}{\pgfqpoint{-0.000000in}{0.000000in}}{%
\pgfpathmoveto{\pgfqpoint{-0.000000in}{0.000000in}}%
\pgfpathlineto{\pgfqpoint{-0.048611in}{0.000000in}}%
\pgfusepath{stroke,fill}%
}%
\begin{pgfscope}%
\pgfsys@transformshift{0.530556in}{3.712542in}%
\pgfsys@useobject{currentmarker}{}%
\end{pgfscope}%
\end{pgfscope}%
\begin{pgfscope}%
\definecolor{textcolor}{rgb}{0.000000,0.000000,0.000000}%
\pgfsetstrokecolor{textcolor}%
\pgfsetfillcolor{textcolor}%
\pgftext[x=0.294444in, y=3.664348in, left, base]{\color{textcolor}\rmfamily\fontsize{10.000000}{12.000000}\selectfont \(\displaystyle {30}\)}%
\end{pgfscope}%
\begin{pgfscope}%
\pgfsetbuttcap%
\pgfsetroundjoin%
\definecolor{currentfill}{rgb}{0.000000,0.000000,0.000000}%
\pgfsetfillcolor{currentfill}%
\pgfsetlinewidth{0.803000pt}%
\definecolor{currentstroke}{rgb}{0.000000,0.000000,0.000000}%
\pgfsetstrokecolor{currentstroke}%
\pgfsetdash{}{0pt}%
\pgfsys@defobject{currentmarker}{\pgfqpoint{-0.048611in}{0.000000in}}{\pgfqpoint{-0.000000in}{0.000000in}}{%
\pgfpathmoveto{\pgfqpoint{-0.000000in}{0.000000in}}%
\pgfpathlineto{\pgfqpoint{-0.048611in}{0.000000in}}%
\pgfusepath{stroke,fill}%
}%
\begin{pgfscope}%
\pgfsys@transformshift{0.530556in}{4.209666in}%
\pgfsys@useobject{currentmarker}{}%
\end{pgfscope}%
\end{pgfscope}%
\begin{pgfscope}%
\definecolor{textcolor}{rgb}{0.000000,0.000000,0.000000}%
\pgfsetstrokecolor{textcolor}%
\pgfsetfillcolor{textcolor}%
\pgftext[x=0.294444in, y=4.161472in, left, base]{\color{textcolor}\rmfamily\fontsize{10.000000}{12.000000}\selectfont \(\displaystyle {35}\)}%
\end{pgfscope}%
\begin{pgfscope}%
\definecolor{textcolor}{rgb}{0.000000,0.000000,0.000000}%
\pgfsetstrokecolor{textcolor}%
\pgfsetfillcolor{textcolor}%
\pgftext[x=0.238889in,y=2.363000in,,bottom,rotate=90.000000]{\color{textcolor}\rmfamily\fontsize{10.000000}{12.000000}\selectfont Position Y [\(\displaystyle m\)]}%
\end{pgfscope}%
\begin{pgfscope}%
\pgfpathrectangle{\pgfqpoint{0.530556in}{0.515000in}}{\pgfqpoint{4.960000in}{3.696000in}}%
\pgfusepath{clip}%
\pgfsetrectcap%
\pgfsetroundjoin%
\pgfsetlinewidth{1.505625pt}%
\definecolor{currentstroke}{rgb}{0.121569,0.466667,0.705882}%
\pgfsetstrokecolor{currentstroke}%
\pgfsetdash{}{0pt}%
\pgfpathmoveto{\pgfqpoint{1.378257in}{0.729800in}}%
\pgfpathlineto{\pgfqpoint{1.378257in}{3.513693in}}%
\pgfpathlineto{\pgfqpoint{4.162150in}{3.513693in}}%
\pgfpathlineto{\pgfqpoint{4.162150in}{0.729800in}}%
\pgfpathlineto{\pgfqpoint{1.378257in}{0.729800in}}%
\pgfusepath{stroke}%
\end{pgfscope}%
\begin{pgfscope}%
\pgfsetrectcap%
\pgfsetmiterjoin%
\pgfsetlinewidth{0.803000pt}%
\definecolor{currentstroke}{rgb}{0.000000,0.000000,0.000000}%
\pgfsetstrokecolor{currentstroke}%
\pgfsetdash{}{0pt}%
\pgfpathmoveto{\pgfqpoint{0.530556in}{0.515000in}}%
\pgfpathlineto{\pgfqpoint{0.530556in}{4.211000in}}%
\pgfusepath{stroke}%
\end{pgfscope}%
\begin{pgfscope}%
\pgfsetrectcap%
\pgfsetmiterjoin%
\pgfsetlinewidth{0.803000pt}%
\definecolor{currentstroke}{rgb}{0.000000,0.000000,0.000000}%
\pgfsetstrokecolor{currentstroke}%
\pgfsetdash{}{0pt}%
\pgfpathmoveto{\pgfqpoint{5.490556in}{0.515000in}}%
\pgfpathlineto{\pgfqpoint{5.490556in}{4.211000in}}%
\pgfusepath{stroke}%
\end{pgfscope}%
\begin{pgfscope}%
\pgfsetrectcap%
\pgfsetmiterjoin%
\pgfsetlinewidth{0.803000pt}%
\definecolor{currentstroke}{rgb}{0.000000,0.000000,0.000000}%
\pgfsetstrokecolor{currentstroke}%
\pgfsetdash{}{0pt}%
\pgfpathmoveto{\pgfqpoint{0.530556in}{0.515000in}}%
\pgfpathlineto{\pgfqpoint{5.490556in}{0.515000in}}%
\pgfusepath{stroke}%
\end{pgfscope}%
\begin{pgfscope}%
\pgfsetrectcap%
\pgfsetmiterjoin%
\pgfsetlinewidth{0.803000pt}%
\definecolor{currentstroke}{rgb}{0.000000,0.000000,0.000000}%
\pgfsetstrokecolor{currentstroke}%
\pgfsetdash{}{0pt}%
\pgfpathmoveto{\pgfqpoint{0.530556in}{4.211000in}}%
\pgfpathlineto{\pgfqpoint{5.490556in}{4.211000in}}%
\pgfusepath{stroke}%
\end{pgfscope}%
\begin{pgfscope}%
\pgfsetbuttcap%
\pgfsetmiterjoin%
\definecolor{currentfill}{rgb}{1.000000,1.000000,1.000000}%
\pgfsetfillcolor{currentfill}%
\pgfsetfillopacity{0.800000}%
\pgfsetlinewidth{1.003750pt}%
\definecolor{currentstroke}{rgb}{0.800000,0.800000,0.800000}%
\pgfsetstrokecolor{currentstroke}%
\pgfsetstrokeopacity{0.800000}%
\pgfsetdash{}{0pt}%
\pgfpathmoveto{\pgfqpoint{2.213611in}{2.148555in}}%
\pgfpathlineto{\pgfqpoint{3.807500in}{2.148555in}}%
\pgfpathquadraticcurveto{\pgfqpoint{3.835278in}{2.148555in}}{\pgfqpoint{3.835278in}{2.176333in}}%
\pgfpathlineto{\pgfqpoint{3.835278in}{2.549666in}}%
\pgfpathquadraticcurveto{\pgfqpoint{3.835278in}{2.577444in}}{\pgfqpoint{3.807500in}{2.577444in}}%
\pgfpathlineto{\pgfqpoint{2.213611in}{2.577444in}}%
\pgfpathquadraticcurveto{\pgfqpoint{2.185833in}{2.577444in}}{\pgfqpoint{2.185833in}{2.549666in}}%
\pgfpathlineto{\pgfqpoint{2.185833in}{2.176333in}}%
\pgfpathquadraticcurveto{\pgfqpoint{2.185833in}{2.148555in}}{\pgfqpoint{2.213611in}{2.148555in}}%
\pgfpathclose%
\pgfusepath{stroke,fill}%
\end{pgfscope}%
\begin{pgfscope}%
\pgfsetrectcap%
\pgfsetroundjoin%
\pgfsetlinewidth{1.505625pt}%
\definecolor{currentstroke}{rgb}{0.121569,0.466667,0.705882}%
\pgfsetstrokecolor{currentstroke}%
\pgfsetdash{}{0pt}%
\pgfpathmoveto{\pgfqpoint{2.241389in}{2.473277in}}%
\pgfpathlineto{\pgfqpoint{2.519167in}{2.473277in}}%
\pgfusepath{stroke}%
\end{pgfscope}%
\begin{pgfscope}%
\definecolor{textcolor}{rgb}{0.000000,0.000000,0.000000}%
\pgfsetstrokecolor{textcolor}%
\pgfsetfillcolor{textcolor}%
\pgftext[x=2.630278in,y=2.424666in,left,base]{\color{textcolor}\rmfamily\fontsize{10.000000}{12.000000}\selectfont Ground truth}%
\end{pgfscope}%
\begin{pgfscope}%
\pgfsetbuttcap%
\pgfsetroundjoin%
\definecolor{currentfill}{rgb}{0.121569,0.466667,0.705882}%
\pgfsetfillcolor{currentfill}%
\pgfsetlinewidth{1.003750pt}%
\definecolor{currentstroke}{rgb}{0.121569,0.466667,0.705882}%
\pgfsetstrokecolor{currentstroke}%
\pgfsetdash{}{0pt}%
\pgfsys@defobject{currentmarker}{\pgfqpoint{-0.041667in}{-0.041667in}}{\pgfqpoint{0.041667in}{0.041667in}}{%
\pgfpathmoveto{\pgfqpoint{0.000000in}{-0.041667in}}%
\pgfpathcurveto{\pgfqpoint{0.011050in}{-0.041667in}}{\pgfqpoint{0.021649in}{-0.037276in}}{\pgfqpoint{0.029463in}{-0.029463in}}%
\pgfpathcurveto{\pgfqpoint{0.037276in}{-0.021649in}}{\pgfqpoint{0.041667in}{-0.011050in}}{\pgfqpoint{0.041667in}{0.000000in}}%
\pgfpathcurveto{\pgfqpoint{0.041667in}{0.011050in}}{\pgfqpoint{0.037276in}{0.021649in}}{\pgfqpoint{0.029463in}{0.029463in}}%
\pgfpathcurveto{\pgfqpoint{0.021649in}{0.037276in}}{\pgfqpoint{0.011050in}{0.041667in}}{\pgfqpoint{0.000000in}{0.041667in}}%
\pgfpathcurveto{\pgfqpoint{-0.011050in}{0.041667in}}{\pgfqpoint{-0.021649in}{0.037276in}}{\pgfqpoint{-0.029463in}{0.029463in}}%
\pgfpathcurveto{\pgfqpoint{-0.037276in}{0.021649in}}{\pgfqpoint{-0.041667in}{0.011050in}}{\pgfqpoint{-0.041667in}{0.000000in}}%
\pgfpathcurveto{\pgfqpoint{-0.041667in}{-0.011050in}}{\pgfqpoint{-0.037276in}{-0.021649in}}{\pgfqpoint{-0.029463in}{-0.029463in}}%
\pgfpathcurveto{\pgfqpoint{-0.021649in}{-0.037276in}}{\pgfqpoint{-0.011050in}{-0.041667in}}{\pgfqpoint{0.000000in}{-0.041667in}}%
\pgfpathclose%
\pgfusepath{stroke,fill}%
}%
\begin{pgfscope}%
\pgfsys@transformshift{2.380278in}{2.267513in}%
\pgfsys@useobject{currentmarker}{}%
\end{pgfscope}%
\end{pgfscope}%
\begin{pgfscope}%
\definecolor{textcolor}{rgb}{0.000000,0.000000,0.000000}%
\pgfsetstrokecolor{textcolor}%
\pgfsetfillcolor{textcolor}%
\pgftext[x=2.630278in,y=2.231055in,left,base]{\color{textcolor}\rmfamily\fontsize{10.000000}{12.000000}\selectfont Estimated position}%
\end{pgfscope}%
\end{pgfpicture}%
\makeatother%
\endgroup%
}
%         \caption{Davenport's 3D position estimation had the lowest displacement error for the 28-meter side square experiment.}
%         \label{fig:square282D}
%     \end{subfigure}
%     \begin{subfigure}{0.49\textwidth}
%         \centering
%         \resizebox{1\linewidth}{!}{%% Creator: Matplotlib, PGF backend
%%
%% To include the figure in your LaTeX document, write
%%   \input{<filename>.pgf}
%%
%% Make sure the required packages are loaded in your preamble
%%   \usepackage{pgf}
%%
%% and, on pdftex
%%   \usepackage[utf8]{inputenc}\DeclareUnicodeCharacter{2212}{-}
%%
%% or, on luatex and xetex
%%   \usepackage{unicode-math}
%%
%% Figures using additional raster images can only be included by \input if
%% they are in the same directory as the main LaTeX file. For loading figures
%% from other directories you can use the `import` package
%%   \usepackage{import}
%%
%% and then include the figures with
%%   \import{<path to file>}{<filename>.pgf}
%%
%% Matplotlib used the following preamble
%%   \usepackage{fontspec}
%%
\begingroup%
\makeatletter%
\begin{pgfpicture}%
\pgfpathrectangle{\pgfpointorigin}{\pgfqpoint{4.342355in}{4.207622in}}%
\pgfusepath{use as bounding box, clip}%
\begin{pgfscope}%
\pgfsetbuttcap%
\pgfsetmiterjoin%
\definecolor{currentfill}{rgb}{1.000000,1.000000,1.000000}%
\pgfsetfillcolor{currentfill}%
\pgfsetlinewidth{0.000000pt}%
\definecolor{currentstroke}{rgb}{1.000000,1.000000,1.000000}%
\pgfsetstrokecolor{currentstroke}%
\pgfsetdash{}{0pt}%
\pgfpathmoveto{\pgfqpoint{0.000000in}{0.000000in}}%
\pgfpathlineto{\pgfqpoint{4.342355in}{0.000000in}}%
\pgfpathlineto{\pgfqpoint{4.342355in}{4.207622in}}%
\pgfpathlineto{\pgfqpoint{0.000000in}{4.207622in}}%
\pgfpathclose%
\pgfusepath{fill}%
\end{pgfscope}%
\begin{pgfscope}%
\pgfsetbuttcap%
\pgfsetmiterjoin%
\definecolor{currentfill}{rgb}{1.000000,1.000000,1.000000}%
\pgfsetfillcolor{currentfill}%
\pgfsetlinewidth{0.000000pt}%
\definecolor{currentstroke}{rgb}{0.000000,0.000000,0.000000}%
\pgfsetstrokecolor{currentstroke}%
\pgfsetstrokeopacity{0.000000}%
\pgfsetdash{}{0pt}%
\pgfpathmoveto{\pgfqpoint{0.100000in}{0.212622in}}%
\pgfpathlineto{\pgfqpoint{3.796000in}{0.212622in}}%
\pgfpathlineto{\pgfqpoint{3.796000in}{3.908622in}}%
\pgfpathlineto{\pgfqpoint{0.100000in}{3.908622in}}%
\pgfpathclose%
\pgfusepath{fill}%
\end{pgfscope}%
\begin{pgfscope}%
\pgfsetbuttcap%
\pgfsetmiterjoin%
\definecolor{currentfill}{rgb}{0.950000,0.950000,0.950000}%
\pgfsetfillcolor{currentfill}%
\pgfsetfillopacity{0.500000}%
\pgfsetlinewidth{1.003750pt}%
\definecolor{currentstroke}{rgb}{0.950000,0.950000,0.950000}%
\pgfsetstrokecolor{currentstroke}%
\pgfsetstrokeopacity{0.500000}%
\pgfsetdash{}{0pt}%
\pgfpathmoveto{\pgfqpoint{0.379073in}{1.123938in}}%
\pgfpathlineto{\pgfqpoint{1.599613in}{2.147018in}}%
\pgfpathlineto{\pgfqpoint{1.582647in}{3.622484in}}%
\pgfpathlineto{\pgfqpoint{0.303698in}{2.689165in}}%
\pgfusepath{stroke,fill}%
\end{pgfscope}%
\begin{pgfscope}%
\pgfsetbuttcap%
\pgfsetmiterjoin%
\definecolor{currentfill}{rgb}{0.900000,0.900000,0.900000}%
\pgfsetfillcolor{currentfill}%
\pgfsetfillopacity{0.500000}%
\pgfsetlinewidth{1.003750pt}%
\definecolor{currentstroke}{rgb}{0.900000,0.900000,0.900000}%
\pgfsetstrokecolor{currentstroke}%
\pgfsetstrokeopacity{0.500000}%
\pgfsetdash{}{0pt}%
\pgfpathmoveto{\pgfqpoint{1.599613in}{2.147018in}}%
\pgfpathlineto{\pgfqpoint{3.558144in}{1.577751in}}%
\pgfpathlineto{\pgfqpoint{3.628038in}{3.104037in}}%
\pgfpathlineto{\pgfqpoint{1.582647in}{3.622484in}}%
\pgfusepath{stroke,fill}%
\end{pgfscope}%
\begin{pgfscope}%
\pgfsetbuttcap%
\pgfsetmiterjoin%
\definecolor{currentfill}{rgb}{0.925000,0.925000,0.925000}%
\pgfsetfillcolor{currentfill}%
\pgfsetfillopacity{0.500000}%
\pgfsetlinewidth{1.003750pt}%
\definecolor{currentstroke}{rgb}{0.925000,0.925000,0.925000}%
\pgfsetstrokecolor{currentstroke}%
\pgfsetstrokeopacity{0.500000}%
\pgfsetdash{}{0pt}%
\pgfpathmoveto{\pgfqpoint{0.379073in}{1.123938in}}%
\pgfpathlineto{\pgfqpoint{2.455212in}{0.445871in}}%
\pgfpathlineto{\pgfqpoint{3.558144in}{1.577751in}}%
\pgfpathlineto{\pgfqpoint{1.599613in}{2.147018in}}%
\pgfusepath{stroke,fill}%
\end{pgfscope}%
\begin{pgfscope}%
\pgfsetrectcap%
\pgfsetroundjoin%
\pgfsetlinewidth{0.803000pt}%
\definecolor{currentstroke}{rgb}{0.000000,0.000000,0.000000}%
\pgfsetstrokecolor{currentstroke}%
\pgfsetdash{}{0pt}%
\pgfpathmoveto{\pgfqpoint{0.379073in}{1.123938in}}%
\pgfpathlineto{\pgfqpoint{2.455212in}{0.445871in}}%
\pgfusepath{stroke}%
\end{pgfscope}%
\begin{pgfscope}%
\definecolor{textcolor}{rgb}{0.000000,0.000000,0.000000}%
\pgfsetstrokecolor{textcolor}%
\pgfsetfillcolor{textcolor}%
\pgftext[x=0.730374in, y=0.408886in, left, base,rotate=341.912962]{\color{textcolor}\rmfamily\fontsize{10.000000}{12.000000}\selectfont Position X [\(\displaystyle m\)]}%
\end{pgfscope}%
\begin{pgfscope}%
\pgfsetbuttcap%
\pgfsetroundjoin%
\pgfsetlinewidth{0.803000pt}%
\definecolor{currentstroke}{rgb}{0.690196,0.690196,0.690196}%
\pgfsetstrokecolor{currentstroke}%
\pgfsetdash{}{0pt}%
\pgfpathmoveto{\pgfqpoint{0.591495in}{1.054561in}}%
\pgfpathlineto{\pgfqpoint{1.800797in}{2.088542in}}%
\pgfpathlineto{\pgfqpoint{1.792355in}{3.569329in}}%
\pgfusepath{stroke}%
\end{pgfscope}%
\begin{pgfscope}%
\pgfsetbuttcap%
\pgfsetroundjoin%
\pgfsetlinewidth{0.803000pt}%
\definecolor{currentstroke}{rgb}{0.690196,0.690196,0.690196}%
\pgfsetstrokecolor{currentstroke}%
\pgfsetdash{}{0pt}%
\pgfpathmoveto{\pgfqpoint{0.844562in}{0.971909in}}%
\pgfpathlineto{\pgfqpoint{2.040239in}{2.018946in}}%
\pgfpathlineto{\pgfqpoint{2.042062in}{3.506035in}}%
\pgfusepath{stroke}%
\end{pgfscope}%
\begin{pgfscope}%
\pgfsetbuttcap%
\pgfsetroundjoin%
\pgfsetlinewidth{0.803000pt}%
\definecolor{currentstroke}{rgb}{0.690196,0.690196,0.690196}%
\pgfsetstrokecolor{currentstroke}%
\pgfsetdash{}{0pt}%
\pgfpathmoveto{\pgfqpoint{1.100974in}{0.888165in}}%
\pgfpathlineto{\pgfqpoint{2.282583in}{1.948506in}}%
\pgfpathlineto{\pgfqpoint{2.294925in}{3.441942in}}%
\pgfusepath{stroke}%
\end{pgfscope}%
\begin{pgfscope}%
\pgfsetbuttcap%
\pgfsetroundjoin%
\pgfsetlinewidth{0.803000pt}%
\definecolor{currentstroke}{rgb}{0.690196,0.690196,0.690196}%
\pgfsetstrokecolor{currentstroke}%
\pgfsetdash{}{0pt}%
\pgfpathmoveto{\pgfqpoint{1.360797in}{0.803307in}}%
\pgfpathlineto{\pgfqpoint{2.527880in}{1.877208in}}%
\pgfpathlineto{\pgfqpoint{2.551005in}{3.377033in}}%
\pgfusepath{stroke}%
\end{pgfscope}%
\begin{pgfscope}%
\pgfsetbuttcap%
\pgfsetroundjoin%
\pgfsetlinewidth{0.803000pt}%
\definecolor{currentstroke}{rgb}{0.690196,0.690196,0.690196}%
\pgfsetstrokecolor{currentstroke}%
\pgfsetdash{}{0pt}%
\pgfpathmoveto{\pgfqpoint{1.624100in}{0.717312in}}%
\pgfpathlineto{\pgfqpoint{2.776187in}{1.805035in}}%
\pgfpathlineto{\pgfqpoint{2.810365in}{3.311293in}}%
\pgfusepath{stroke}%
\end{pgfscope}%
\begin{pgfscope}%
\pgfsetbuttcap%
\pgfsetroundjoin%
\pgfsetlinewidth{0.803000pt}%
\definecolor{currentstroke}{rgb}{0.690196,0.690196,0.690196}%
\pgfsetstrokecolor{currentstroke}%
\pgfsetdash{}{0pt}%
\pgfpathmoveto{\pgfqpoint{1.890953in}{0.630158in}}%
\pgfpathlineto{\pgfqpoint{3.027558in}{1.731971in}}%
\pgfpathlineto{\pgfqpoint{3.073067in}{3.244706in}}%
\pgfusepath{stroke}%
\end{pgfscope}%
\begin{pgfscope}%
\pgfsetbuttcap%
\pgfsetroundjoin%
\pgfsetlinewidth{0.803000pt}%
\definecolor{currentstroke}{rgb}{0.690196,0.690196,0.690196}%
\pgfsetstrokecolor{currentstroke}%
\pgfsetdash{}{0pt}%
\pgfpathmoveto{\pgfqpoint{2.161428in}{0.541821in}}%
\pgfpathlineto{\pgfqpoint{3.282051in}{1.658000in}}%
\pgfpathlineto{\pgfqpoint{3.339177in}{3.177255in}}%
\pgfusepath{stroke}%
\end{pgfscope}%
\begin{pgfscope}%
\pgfsetrectcap%
\pgfsetroundjoin%
\pgfsetlinewidth{0.803000pt}%
\definecolor{currentstroke}{rgb}{0.000000,0.000000,0.000000}%
\pgfsetstrokecolor{currentstroke}%
\pgfsetdash{}{0pt}%
\pgfpathmoveto{\pgfqpoint{0.602027in}{1.063566in}}%
\pgfpathlineto{\pgfqpoint{0.570385in}{1.036511in}}%
\pgfusepath{stroke}%
\end{pgfscope}%
\begin{pgfscope}%
\definecolor{textcolor}{rgb}{0.000000,0.000000,0.000000}%
\pgfsetstrokecolor{textcolor}%
\pgfsetfillcolor{textcolor}%
\pgftext[x=0.487011in,y=0.835757in,,top]{\color{textcolor}\rmfamily\fontsize{10.000000}{12.000000}\selectfont \(\displaystyle {0}\)}%
\end{pgfscope}%
\begin{pgfscope}%
\pgfsetrectcap%
\pgfsetroundjoin%
\pgfsetlinewidth{0.803000pt}%
\definecolor{currentstroke}{rgb}{0.000000,0.000000,0.000000}%
\pgfsetstrokecolor{currentstroke}%
\pgfsetdash{}{0pt}%
\pgfpathmoveto{\pgfqpoint{0.854981in}{0.981033in}}%
\pgfpathlineto{\pgfqpoint{0.823678in}{0.953621in}}%
\pgfusepath{stroke}%
\end{pgfscope}%
\begin{pgfscope}%
\definecolor{textcolor}{rgb}{0.000000,0.000000,0.000000}%
\pgfsetstrokecolor{textcolor}%
\pgfsetfillcolor{textcolor}%
\pgftext[x=0.740339in,y=0.751354in,,top]{\color{textcolor}\rmfamily\fontsize{10.000000}{12.000000}\selectfont \(\displaystyle {5}\)}%
\end{pgfscope}%
\begin{pgfscope}%
\pgfsetrectcap%
\pgfsetroundjoin%
\pgfsetlinewidth{0.803000pt}%
\definecolor{currentstroke}{rgb}{0.000000,0.000000,0.000000}%
\pgfsetstrokecolor{currentstroke}%
\pgfsetdash{}{0pt}%
\pgfpathmoveto{\pgfqpoint{1.111276in}{0.897410in}}%
\pgfpathlineto{\pgfqpoint{1.080324in}{0.869635in}}%
\pgfusepath{stroke}%
\end{pgfscope}%
\begin{pgfscope}%
\definecolor{textcolor}{rgb}{0.000000,0.000000,0.000000}%
\pgfsetstrokecolor{textcolor}%
\pgfsetfillcolor{textcolor}%
\pgftext[x=0.997027in,y=0.665831in,,top]{\color{textcolor}\rmfamily\fontsize{10.000000}{12.000000}\selectfont \(\displaystyle {10}\)}%
\end{pgfscope}%
\begin{pgfscope}%
\pgfsetrectcap%
\pgfsetroundjoin%
\pgfsetlinewidth{0.803000pt}%
\definecolor{currentstroke}{rgb}{0.000000,0.000000,0.000000}%
\pgfsetstrokecolor{currentstroke}%
\pgfsetdash{}{0pt}%
\pgfpathmoveto{\pgfqpoint{1.370978in}{0.812675in}}%
\pgfpathlineto{\pgfqpoint{1.340390in}{0.784529in}}%
\pgfusepath{stroke}%
\end{pgfscope}%
\begin{pgfscope}%
\definecolor{textcolor}{rgb}{0.000000,0.000000,0.000000}%
\pgfsetstrokecolor{textcolor}%
\pgfsetfillcolor{textcolor}%
\pgftext[x=1.257144in,y=0.579166in,,top]{\color{textcolor}\rmfamily\fontsize{10.000000}{12.000000}\selectfont \(\displaystyle {15}\)}%
\end{pgfscope}%
\begin{pgfscope}%
\pgfsetrectcap%
\pgfsetroundjoin%
\pgfsetlinewidth{0.803000pt}%
\definecolor{currentstroke}{rgb}{0.000000,0.000000,0.000000}%
\pgfsetstrokecolor{currentstroke}%
\pgfsetdash{}{0pt}%
\pgfpathmoveto{\pgfqpoint{1.634156in}{0.726806in}}%
\pgfpathlineto{\pgfqpoint{1.603943in}{0.698282in}}%
\pgfusepath{stroke}%
\end{pgfscope}%
\begin{pgfscope}%
\definecolor{textcolor}{rgb}{0.000000,0.000000,0.000000}%
\pgfsetstrokecolor{textcolor}%
\pgfsetfillcolor{textcolor}%
\pgftext[x=1.520757in,y=0.491335in,,top]{\color{textcolor}\rmfamily\fontsize{10.000000}{12.000000}\selectfont \(\displaystyle {20}\)}%
\end{pgfscope}%
\begin{pgfscope}%
\pgfsetrectcap%
\pgfsetroundjoin%
\pgfsetlinewidth{0.803000pt}%
\definecolor{currentstroke}{rgb}{0.000000,0.000000,0.000000}%
\pgfsetstrokecolor{currentstroke}%
\pgfsetdash{}{0pt}%
\pgfpathmoveto{\pgfqpoint{1.900879in}{0.639780in}}%
\pgfpathlineto{\pgfqpoint{1.871056in}{0.610870in}}%
\pgfusepath{stroke}%
\end{pgfscope}%
\begin{pgfscope}%
\definecolor{textcolor}{rgb}{0.000000,0.000000,0.000000}%
\pgfsetstrokecolor{textcolor}%
\pgfsetfillcolor{textcolor}%
\pgftext[x=1.787939in,y=0.402316in,,top]{\color{textcolor}\rmfamily\fontsize{10.000000}{12.000000}\selectfont \(\displaystyle {25}\)}%
\end{pgfscope}%
\begin{pgfscope}%
\pgfsetrectcap%
\pgfsetroundjoin%
\pgfsetlinewidth{0.803000pt}%
\definecolor{currentstroke}{rgb}{0.000000,0.000000,0.000000}%
\pgfsetstrokecolor{currentstroke}%
\pgfsetdash{}{0pt}%
\pgfpathmoveto{\pgfqpoint{2.171221in}{0.551574in}}%
\pgfpathlineto{\pgfqpoint{2.141799in}{0.522270in}}%
\pgfusepath{stroke}%
\end{pgfscope}%
\begin{pgfscope}%
\definecolor{textcolor}{rgb}{0.000000,0.000000,0.000000}%
\pgfsetstrokecolor{textcolor}%
\pgfsetfillcolor{textcolor}%
\pgftext[x=2.058761in,y=0.312084in,,top]{\color{textcolor}\rmfamily\fontsize{10.000000}{12.000000}\selectfont \(\displaystyle {30}\)}%
\end{pgfscope}%
\begin{pgfscope}%
\pgfsetrectcap%
\pgfsetroundjoin%
\pgfsetlinewidth{0.803000pt}%
\definecolor{currentstroke}{rgb}{0.000000,0.000000,0.000000}%
\pgfsetstrokecolor{currentstroke}%
\pgfsetdash{}{0pt}%
\pgfpathmoveto{\pgfqpoint{3.558144in}{1.577751in}}%
\pgfpathlineto{\pgfqpoint{2.455212in}{0.445871in}}%
\pgfusepath{stroke}%
\end{pgfscope}%
\begin{pgfscope}%
\definecolor{textcolor}{rgb}{0.000000,0.000000,0.000000}%
\pgfsetstrokecolor{textcolor}%
\pgfsetfillcolor{textcolor}%
\pgftext[x=3.120747in, y=0.305657in, left, base,rotate=45.742112]{\color{textcolor}\rmfamily\fontsize{10.000000}{12.000000}\selectfont Position Y [\(\displaystyle m\)]}%
\end{pgfscope}%
\begin{pgfscope}%
\pgfsetbuttcap%
\pgfsetroundjoin%
\pgfsetlinewidth{0.803000pt}%
\definecolor{currentstroke}{rgb}{0.690196,0.690196,0.690196}%
\pgfsetstrokecolor{currentstroke}%
\pgfsetdash{}{0pt}%
\pgfpathmoveto{\pgfqpoint{0.408916in}{2.765949in}}%
\pgfpathlineto{\pgfqpoint{0.479137in}{1.207813in}}%
\pgfpathlineto{\pgfqpoint{2.546001in}{0.539043in}}%
\pgfusepath{stroke}%
\end{pgfscope}%
\begin{pgfscope}%
\pgfsetbuttcap%
\pgfsetroundjoin%
\pgfsetlinewidth{0.803000pt}%
\definecolor{currentstroke}{rgb}{0.690196,0.690196,0.690196}%
\pgfsetstrokecolor{currentstroke}%
\pgfsetdash{}{0pt}%
\pgfpathmoveto{\pgfqpoint{0.584815in}{2.894311in}}%
\pgfpathlineto{\pgfqpoint{0.646558in}{1.348148in}}%
\pgfpathlineto{\pgfqpoint{2.697757in}{0.694781in}}%
\pgfusepath{stroke}%
\end{pgfscope}%
\begin{pgfscope}%
\pgfsetbuttcap%
\pgfsetroundjoin%
\pgfsetlinewidth{0.803000pt}%
\definecolor{currentstroke}{rgb}{0.690196,0.690196,0.690196}%
\pgfsetstrokecolor{currentstroke}%
\pgfsetdash{}{0pt}%
\pgfpathmoveto{\pgfqpoint{0.756611in}{3.019680in}}%
\pgfpathlineto{\pgfqpoint{0.810243in}{1.485352in}}%
\pgfpathlineto{\pgfqpoint{2.845948in}{0.846862in}}%
\pgfusepath{stroke}%
\end{pgfscope}%
\begin{pgfscope}%
\pgfsetbuttcap%
\pgfsetroundjoin%
\pgfsetlinewidth{0.803000pt}%
\definecolor{currentstroke}{rgb}{0.690196,0.690196,0.690196}%
\pgfsetstrokecolor{currentstroke}%
\pgfsetdash{}{0pt}%
\pgfpathmoveto{\pgfqpoint{0.924446in}{3.142159in}}%
\pgfpathlineto{\pgfqpoint{0.970315in}{1.619528in}}%
\pgfpathlineto{\pgfqpoint{2.990699in}{0.995413in}}%
\pgfusepath{stroke}%
\end{pgfscope}%
\begin{pgfscope}%
\pgfsetbuttcap%
\pgfsetroundjoin%
\pgfsetlinewidth{0.803000pt}%
\definecolor{currentstroke}{rgb}{0.690196,0.690196,0.690196}%
\pgfsetstrokecolor{currentstroke}%
\pgfsetdash{}{0pt}%
\pgfpathmoveto{\pgfqpoint{1.088456in}{3.261846in}}%
\pgfpathlineto{\pgfqpoint{1.126893in}{1.750775in}}%
\pgfpathlineto{\pgfqpoint{3.132129in}{1.140554in}}%
\pgfusepath{stroke}%
\end{pgfscope}%
\begin{pgfscope}%
\pgfsetbuttcap%
\pgfsetroundjoin%
\pgfsetlinewidth{0.803000pt}%
\definecolor{currentstroke}{rgb}{0.690196,0.690196,0.690196}%
\pgfsetstrokecolor{currentstroke}%
\pgfsetdash{}{0pt}%
\pgfpathmoveto{\pgfqpoint{1.248770in}{3.378836in}}%
\pgfpathlineto{\pgfqpoint{1.280090in}{1.879188in}}%
\pgfpathlineto{\pgfqpoint{3.270350in}{1.282402in}}%
\pgfusepath{stroke}%
\end{pgfscope}%
\begin{pgfscope}%
\pgfsetbuttcap%
\pgfsetroundjoin%
\pgfsetlinewidth{0.803000pt}%
\definecolor{currentstroke}{rgb}{0.690196,0.690196,0.690196}%
\pgfsetstrokecolor{currentstroke}%
\pgfsetdash{}{0pt}%
\pgfpathmoveto{\pgfqpoint{1.405512in}{3.493219in}}%
\pgfpathlineto{\pgfqpoint{1.430015in}{2.004858in}}%
\pgfpathlineto{\pgfqpoint{3.405470in}{1.421069in}}%
\pgfusepath{stroke}%
\end{pgfscope}%
\begin{pgfscope}%
\pgfsetbuttcap%
\pgfsetroundjoin%
\pgfsetlinewidth{0.803000pt}%
\definecolor{currentstroke}{rgb}{0.690196,0.690196,0.690196}%
\pgfsetstrokecolor{currentstroke}%
\pgfsetdash{}{0pt}%
\pgfpathmoveto{\pgfqpoint{1.558800in}{3.605082in}}%
\pgfpathlineto{\pgfqpoint{1.576771in}{2.127871in}}%
\pgfpathlineto{\pgfqpoint{3.537592in}{1.556659in}}%
\pgfusepath{stroke}%
\end{pgfscope}%
\begin{pgfscope}%
\pgfsetrectcap%
\pgfsetroundjoin%
\pgfsetlinewidth{0.803000pt}%
\definecolor{currentstroke}{rgb}{0.000000,0.000000,0.000000}%
\pgfsetstrokecolor{currentstroke}%
\pgfsetdash{}{0pt}%
\pgfpathmoveto{\pgfqpoint{2.528585in}{0.544678in}}%
\pgfpathlineto{\pgfqpoint{2.580879in}{0.527757in}}%
\pgfusepath{stroke}%
\end{pgfscope}%
\begin{pgfscope}%
\definecolor{textcolor}{rgb}{0.000000,0.000000,0.000000}%
\pgfsetstrokecolor{textcolor}%
\pgfsetfillcolor{textcolor}%
\pgftext[x=2.724705in,y=0.352828in,,top]{\color{textcolor}\rmfamily\fontsize{10.000000}{12.000000}\selectfont \(\displaystyle {0}\)}%
\end{pgfscope}%
\begin{pgfscope}%
\pgfsetrectcap%
\pgfsetroundjoin%
\pgfsetlinewidth{0.803000pt}%
\definecolor{currentstroke}{rgb}{0.000000,0.000000,0.000000}%
\pgfsetstrokecolor{currentstroke}%
\pgfsetdash{}{0pt}%
\pgfpathmoveto{\pgfqpoint{2.680483in}{0.700284in}}%
\pgfpathlineto{\pgfqpoint{2.732349in}{0.683763in}}%
\pgfusepath{stroke}%
\end{pgfscope}%
\begin{pgfscope}%
\definecolor{textcolor}{rgb}{0.000000,0.000000,0.000000}%
\pgfsetstrokecolor{textcolor}%
\pgfsetfillcolor{textcolor}%
\pgftext[x=2.874425in,y=0.510873in,,top]{\color{textcolor}\rmfamily\fontsize{10.000000}{12.000000}\selectfont \(\displaystyle {5}\)}%
\end{pgfscope}%
\begin{pgfscope}%
\pgfsetrectcap%
\pgfsetroundjoin%
\pgfsetlinewidth{0.803000pt}%
\definecolor{currentstroke}{rgb}{0.000000,0.000000,0.000000}%
\pgfsetstrokecolor{currentstroke}%
\pgfsetdash{}{0pt}%
\pgfpathmoveto{\pgfqpoint{2.828815in}{0.852236in}}%
\pgfpathlineto{\pgfqpoint{2.880259in}{0.836101in}}%
\pgfusepath{stroke}%
\end{pgfscope}%
\begin{pgfscope}%
\definecolor{textcolor}{rgb}{0.000000,0.000000,0.000000}%
\pgfsetstrokecolor{textcolor}%
\pgfsetfillcolor{textcolor}%
\pgftext[x=3.020627in,y=0.665204in,,top]{\color{textcolor}\rmfamily\fontsize{10.000000}{12.000000}\selectfont \(\displaystyle {10}\)}%
\end{pgfscope}%
\begin{pgfscope}%
\pgfsetrectcap%
\pgfsetroundjoin%
\pgfsetlinewidth{0.803000pt}%
\definecolor{currentstroke}{rgb}{0.000000,0.000000,0.000000}%
\pgfsetstrokecolor{currentstroke}%
\pgfsetdash{}{0pt}%
\pgfpathmoveto{\pgfqpoint{2.973705in}{1.000662in}}%
\pgfpathlineto{\pgfqpoint{3.024732in}{0.984900in}}%
\pgfusepath{stroke}%
\end{pgfscope}%
\begin{pgfscope}%
\definecolor{textcolor}{rgb}{0.000000,0.000000,0.000000}%
\pgfsetstrokecolor{textcolor}%
\pgfsetfillcolor{textcolor}%
\pgftext[x=3.163434in,y=0.815949in,,top]{\color{textcolor}\rmfamily\fontsize{10.000000}{12.000000}\selectfont \(\displaystyle {15}\)}%
\end{pgfscope}%
\begin{pgfscope}%
\pgfsetrectcap%
\pgfsetroundjoin%
\pgfsetlinewidth{0.803000pt}%
\definecolor{currentstroke}{rgb}{0.000000,0.000000,0.000000}%
\pgfsetstrokecolor{currentstroke}%
\pgfsetdash{}{0pt}%
\pgfpathmoveto{\pgfqpoint{3.115271in}{1.145684in}}%
\pgfpathlineto{\pgfqpoint{3.165886in}{1.130281in}}%
\pgfusepath{stroke}%
\end{pgfscope}%
\begin{pgfscope}%
\definecolor{textcolor}{rgb}{0.000000,0.000000,0.000000}%
\pgfsetstrokecolor{textcolor}%
\pgfsetfillcolor{textcolor}%
\pgftext[x=3.302961in,y=0.963234in,,top]{\color{textcolor}\rmfamily\fontsize{10.000000}{12.000000}\selectfont \(\displaystyle {20}\)}%
\end{pgfscope}%
\begin{pgfscope}%
\pgfsetrectcap%
\pgfsetroundjoin%
\pgfsetlinewidth{0.803000pt}%
\definecolor{currentstroke}{rgb}{0.000000,0.000000,0.000000}%
\pgfsetstrokecolor{currentstroke}%
\pgfsetdash{}{0pt}%
\pgfpathmoveto{\pgfqpoint{3.253627in}{1.287417in}}%
\pgfpathlineto{\pgfqpoint{3.303836in}{1.272361in}}%
\pgfusepath{stroke}%
\end{pgfscope}%
\begin{pgfscope}%
\definecolor{textcolor}{rgb}{0.000000,0.000000,0.000000}%
\pgfsetstrokecolor{textcolor}%
\pgfsetfillcolor{textcolor}%
\pgftext[x=3.439320in,y=1.107175in,,top]{\color{textcolor}\rmfamily\fontsize{10.000000}{12.000000}\selectfont \(\displaystyle {25}\)}%
\end{pgfscope}%
\begin{pgfscope}%
\pgfsetrectcap%
\pgfsetroundjoin%
\pgfsetlinewidth{0.803000pt}%
\definecolor{currentstroke}{rgb}{0.000000,0.000000,0.000000}%
\pgfsetstrokecolor{currentstroke}%
\pgfsetdash{}{0pt}%
\pgfpathmoveto{\pgfqpoint{3.388880in}{1.425971in}}%
\pgfpathlineto{\pgfqpoint{3.438689in}{1.411252in}}%
\pgfusepath{stroke}%
\end{pgfscope}%
\begin{pgfscope}%
\definecolor{textcolor}{rgb}{0.000000,0.000000,0.000000}%
\pgfsetstrokecolor{textcolor}%
\pgfsetfillcolor{textcolor}%
\pgftext[x=3.572620in,y=1.247885in,,top]{\color{textcolor}\rmfamily\fontsize{10.000000}{12.000000}\selectfont \(\displaystyle {30}\)}%
\end{pgfscope}%
\begin{pgfscope}%
\pgfsetrectcap%
\pgfsetroundjoin%
\pgfsetlinewidth{0.803000pt}%
\definecolor{currentstroke}{rgb}{0.000000,0.000000,0.000000}%
\pgfsetstrokecolor{currentstroke}%
\pgfsetdash{}{0pt}%
\pgfpathmoveto{\pgfqpoint{3.521134in}{1.561454in}}%
\pgfpathlineto{\pgfqpoint{3.570548in}{1.547059in}}%
\pgfusepath{stroke}%
\end{pgfscope}%
\begin{pgfscope}%
\definecolor{textcolor}{rgb}{0.000000,0.000000,0.000000}%
\pgfsetstrokecolor{textcolor}%
\pgfsetfillcolor{textcolor}%
\pgftext[x=3.702960in,y=1.385473in,,top]{\color{textcolor}\rmfamily\fontsize{10.000000}{12.000000}\selectfont \(\displaystyle {35}\)}%
\end{pgfscope}%
\begin{pgfscope}%
\pgfsetrectcap%
\pgfsetroundjoin%
\pgfsetlinewidth{0.803000pt}%
\definecolor{currentstroke}{rgb}{0.000000,0.000000,0.000000}%
\pgfsetstrokecolor{currentstroke}%
\pgfsetdash{}{0pt}%
\pgfpathmoveto{\pgfqpoint{3.558144in}{1.577751in}}%
\pgfpathlineto{\pgfqpoint{3.628038in}{3.104037in}}%
\pgfusepath{stroke}%
\end{pgfscope}%
\begin{pgfscope}%
\definecolor{textcolor}{rgb}{0.000000,0.000000,0.000000}%
\pgfsetstrokecolor{textcolor}%
\pgfsetfillcolor{textcolor}%
\pgftext[x=4.167903in, y=1.963517in, left, base,rotate=87.378092]{\color{textcolor}\rmfamily\fontsize{10.000000}{12.000000}\selectfont Position Z [\(\displaystyle m\)]}%
\end{pgfscope}%
\begin{pgfscope}%
\pgfsetbuttcap%
\pgfsetroundjoin%
\pgfsetlinewidth{0.803000pt}%
\definecolor{currentstroke}{rgb}{0.690196,0.690196,0.690196}%
\pgfsetstrokecolor{currentstroke}%
\pgfsetdash{}{0pt}%
\pgfpathmoveto{\pgfqpoint{3.564085in}{1.707472in}}%
\pgfpathlineto{\pgfqpoint{1.598169in}{2.272663in}}%
\pgfpathlineto{\pgfqpoint{0.372677in}{1.256763in}}%
\pgfusepath{stroke}%
\end{pgfscope}%
\begin{pgfscope}%
\pgfsetbuttcap%
\pgfsetroundjoin%
\pgfsetlinewidth{0.803000pt}%
\definecolor{currentstroke}{rgb}{0.690196,0.690196,0.690196}%
\pgfsetstrokecolor{currentstroke}%
\pgfsetdash{}{0pt}%
\pgfpathmoveto{\pgfqpoint{3.578645in}{2.025421in}}%
\pgfpathlineto{\pgfqpoint{1.594630in}{2.580429in}}%
\pgfpathlineto{\pgfqpoint{0.356992in}{1.582483in}}%
\pgfusepath{stroke}%
\end{pgfscope}%
\begin{pgfscope}%
\pgfsetbuttcap%
\pgfsetroundjoin%
\pgfsetlinewidth{0.803000pt}%
\definecolor{currentstroke}{rgb}{0.690196,0.690196,0.690196}%
\pgfsetstrokecolor{currentstroke}%
\pgfsetdash{}{0pt}%
\pgfpathmoveto{\pgfqpoint{3.593478in}{2.349338in}}%
\pgfpathlineto{\pgfqpoint{1.591027in}{2.893693in}}%
\pgfpathlineto{\pgfqpoint{0.341000in}{1.914551in}}%
\pgfusepath{stroke}%
\end{pgfscope}%
\begin{pgfscope}%
\pgfsetbuttcap%
\pgfsetroundjoin%
\pgfsetlinewidth{0.803000pt}%
\definecolor{currentstroke}{rgb}{0.690196,0.690196,0.690196}%
\pgfsetstrokecolor{currentstroke}%
\pgfsetdash{}{0pt}%
\pgfpathmoveto{\pgfqpoint{3.608592in}{2.679391in}}%
\pgfpathlineto{\pgfqpoint{1.587360in}{3.212603in}}%
\pgfpathlineto{\pgfqpoint{0.324694in}{2.253155in}}%
\pgfusepath{stroke}%
\end{pgfscope}%
\begin{pgfscope}%
\pgfsetbuttcap%
\pgfsetroundjoin%
\pgfsetlinewidth{0.803000pt}%
\definecolor{currentstroke}{rgb}{0.690196,0.690196,0.690196}%
\pgfsetstrokecolor{currentstroke}%
\pgfsetdash{}{0pt}%
\pgfpathmoveto{\pgfqpoint{3.623995in}{3.015758in}}%
\pgfpathlineto{\pgfqpoint{1.583626in}{3.537314in}}%
\pgfpathlineto{\pgfqpoint{0.308064in}{2.598490in}}%
\pgfusepath{stroke}%
\end{pgfscope}%
\begin{pgfscope}%
\pgfsetrectcap%
\pgfsetroundjoin%
\pgfsetlinewidth{0.803000pt}%
\definecolor{currentstroke}{rgb}{0.000000,0.000000,0.000000}%
\pgfsetstrokecolor{currentstroke}%
\pgfsetdash{}{0pt}%
\pgfpathmoveto{\pgfqpoint{3.547582in}{1.712216in}}%
\pgfpathlineto{\pgfqpoint{3.597129in}{1.697972in}}%
\pgfusepath{stroke}%
\end{pgfscope}%
\begin{pgfscope}%
\definecolor{textcolor}{rgb}{0.000000,0.000000,0.000000}%
\pgfsetstrokecolor{textcolor}%
\pgfsetfillcolor{textcolor}%
\pgftext[x=3.818482in,y=1.743399in,,top]{\color{textcolor}\rmfamily\fontsize{10.000000}{12.000000}\selectfont \(\displaystyle {-0.2}\)}%
\end{pgfscope}%
\begin{pgfscope}%
\pgfsetrectcap%
\pgfsetroundjoin%
\pgfsetlinewidth{0.803000pt}%
\definecolor{currentstroke}{rgb}{0.000000,0.000000,0.000000}%
\pgfsetstrokecolor{currentstroke}%
\pgfsetdash{}{0pt}%
\pgfpathmoveto{\pgfqpoint{3.561983in}{2.030082in}}%
\pgfpathlineto{\pgfqpoint{3.612008in}{2.016088in}}%
\pgfusepath{stroke}%
\end{pgfscope}%
\begin{pgfscope}%
\definecolor{textcolor}{rgb}{0.000000,0.000000,0.000000}%
\pgfsetstrokecolor{textcolor}%
\pgfsetfillcolor{textcolor}%
\pgftext[x=3.835356in,y=2.060716in,,top]{\color{textcolor}\rmfamily\fontsize{10.000000}{12.000000}\selectfont \(\displaystyle {0.0}\)}%
\end{pgfscope}%
\begin{pgfscope}%
\pgfsetrectcap%
\pgfsetroundjoin%
\pgfsetlinewidth{0.803000pt}%
\definecolor{currentstroke}{rgb}{0.000000,0.000000,0.000000}%
\pgfsetstrokecolor{currentstroke}%
\pgfsetdash{}{0pt}%
\pgfpathmoveto{\pgfqpoint{3.576654in}{2.353911in}}%
\pgfpathlineto{\pgfqpoint{3.627166in}{2.340180in}}%
\pgfusepath{stroke}%
\end{pgfscope}%
\begin{pgfscope}%
\definecolor{textcolor}{rgb}{0.000000,0.000000,0.000000}%
\pgfsetstrokecolor{textcolor}%
\pgfsetfillcolor{textcolor}%
\pgftext[x=3.852545in,y=2.383970in,,top]{\color{textcolor}\rmfamily\fontsize{10.000000}{12.000000}\selectfont \(\displaystyle {0.2}\)}%
\end{pgfscope}%
\begin{pgfscope}%
\pgfsetrectcap%
\pgfsetroundjoin%
\pgfsetlinewidth{0.803000pt}%
\definecolor{currentstroke}{rgb}{0.000000,0.000000,0.000000}%
\pgfsetstrokecolor{currentstroke}%
\pgfsetdash{}{0pt}%
\pgfpathmoveto{\pgfqpoint{3.591602in}{2.683873in}}%
\pgfpathlineto{\pgfqpoint{3.642612in}{2.670417in}}%
\pgfusepath{stroke}%
\end{pgfscope}%
\begin{pgfscope}%
\definecolor{textcolor}{rgb}{0.000000,0.000000,0.000000}%
\pgfsetstrokecolor{textcolor}%
\pgfsetfillcolor{textcolor}%
\pgftext[x=3.870059in,y=2.713329in,,top]{\color{textcolor}\rmfamily\fontsize{10.000000}{12.000000}\selectfont \(\displaystyle {0.4}\)}%
\end{pgfscope}%
\begin{pgfscope}%
\pgfsetrectcap%
\pgfsetroundjoin%
\pgfsetlinewidth{0.803000pt}%
\definecolor{currentstroke}{rgb}{0.000000,0.000000,0.000000}%
\pgfsetstrokecolor{currentstroke}%
\pgfsetdash{}{0pt}%
\pgfpathmoveto{\pgfqpoint{3.606837in}{3.020144in}}%
\pgfpathlineto{\pgfqpoint{3.658354in}{3.006975in}}%
\pgfusepath{stroke}%
\end{pgfscope}%
\begin{pgfscope}%
\definecolor{textcolor}{rgb}{0.000000,0.000000,0.000000}%
\pgfsetstrokecolor{textcolor}%
\pgfsetfillcolor{textcolor}%
\pgftext[x=3.887907in,y=3.048968in,,top]{\color{textcolor}\rmfamily\fontsize{10.000000}{12.000000}\selectfont \(\displaystyle {0.6}\)}%
\end{pgfscope}%
\begin{pgfscope}%
\pgfpathrectangle{\pgfqpoint{0.100000in}{0.212622in}}{\pgfqpoint{3.696000in}{3.696000in}}%
\pgfusepath{clip}%
\pgfsetrectcap%
\pgfsetroundjoin%
\pgfsetlinewidth{1.505625pt}%
\definecolor{currentstroke}{rgb}{0.121569,0.466667,0.705882}%
\pgfsetstrokecolor{currentstroke}%
\pgfsetdash{}{0pt}%
\pgfpathmoveto{\pgfqpoint{0.672869in}{1.597548in}}%
\pgfpathlineto{\pgfqpoint{1.568384in}{2.334283in}}%
\pgfpathlineto{\pgfqpoint{2.981717in}{1.929855in}}%
\pgfpathlineto{\pgfqpoint{2.147590in}{1.137647in}}%
\pgfpathlineto{\pgfqpoint{0.672869in}{1.597548in}}%
\pgfusepath{stroke}%
\end{pgfscope}%
\begin{pgfscope}%
\pgfpathrectangle{\pgfqpoint{0.100000in}{0.212622in}}{\pgfqpoint{3.696000in}{3.696000in}}%
\pgfusepath{clip}%
\pgfsetrectcap%
\pgfsetroundjoin%
\pgfsetlinewidth{1.505625pt}%
\definecolor{currentstroke}{rgb}{1.000000,0.000000,0.000000}%
\pgfsetstrokecolor{currentstroke}%
\pgfsetdash{}{0pt}%
\pgfpathmoveto{\pgfqpoint{0.672869in}{1.597548in}}%
\pgfpathlineto{\pgfqpoint{0.672869in}{1.597548in}}%
\pgfusepath{stroke}%
\end{pgfscope}%
\begin{pgfscope}%
\pgfpathrectangle{\pgfqpoint{0.100000in}{0.212622in}}{\pgfqpoint{3.696000in}{3.696000in}}%
\pgfusepath{clip}%
\pgfsetrectcap%
\pgfsetroundjoin%
\pgfsetlinewidth{1.505625pt}%
\definecolor{currentstroke}{rgb}{1.000000,0.000000,0.000000}%
\pgfsetstrokecolor{currentstroke}%
\pgfsetdash{}{0pt}%
\pgfpathmoveto{\pgfqpoint{1.662771in}{2.135129in}}%
\pgfpathlineto{\pgfqpoint{1.568384in}{2.334283in}}%
\pgfusepath{stroke}%
\end{pgfscope}%
\begin{pgfscope}%
\pgfpathrectangle{\pgfqpoint{0.100000in}{0.212622in}}{\pgfqpoint{3.696000in}{3.696000in}}%
\pgfusepath{clip}%
\pgfsetrectcap%
\pgfsetroundjoin%
\pgfsetlinewidth{1.505625pt}%
\definecolor{currentstroke}{rgb}{1.000000,0.000000,0.000000}%
\pgfsetstrokecolor{currentstroke}%
\pgfsetdash{}{0pt}%
\pgfpathmoveto{\pgfqpoint{3.288684in}{2.027612in}}%
\pgfpathlineto{\pgfqpoint{2.981717in}{1.929855in}}%
\pgfusepath{stroke}%
\end{pgfscope}%
\begin{pgfscope}%
\pgfpathrectangle{\pgfqpoint{0.100000in}{0.212622in}}{\pgfqpoint{3.696000in}{3.696000in}}%
\pgfusepath{clip}%
\pgfsetrectcap%
\pgfsetroundjoin%
\pgfsetlinewidth{1.505625pt}%
\definecolor{currentstroke}{rgb}{1.000000,0.000000,0.000000}%
\pgfsetstrokecolor{currentstroke}%
\pgfsetdash{}{0pt}%
\pgfpathmoveto{\pgfqpoint{2.722088in}{1.738247in}}%
\pgfpathlineto{\pgfqpoint{2.147590in}{1.137647in}}%
\pgfusepath{stroke}%
\end{pgfscope}%
\begin{pgfscope}%
\pgfpathrectangle{\pgfqpoint{0.100000in}{0.212622in}}{\pgfqpoint{3.696000in}{3.696000in}}%
\pgfusepath{clip}%
\pgfsetrectcap%
\pgfsetroundjoin%
\pgfsetlinewidth{1.505625pt}%
\definecolor{currentstroke}{rgb}{1.000000,0.000000,0.000000}%
\pgfsetstrokecolor{currentstroke}%
\pgfsetdash{}{0pt}%
\pgfpathmoveto{\pgfqpoint{0.531950in}{2.613888in}}%
\pgfpathlineto{\pgfqpoint{0.672869in}{1.597548in}}%
\pgfusepath{stroke}%
\end{pgfscope}%
\begin{pgfscope}%
\pgfpathrectangle{\pgfqpoint{0.100000in}{0.212622in}}{\pgfqpoint{3.696000in}{3.696000in}}%
\pgfusepath{clip}%
\pgfsetbuttcap%
\pgfsetroundjoin%
\definecolor{currentfill}{rgb}{1.000000,0.498039,0.054902}%
\pgfsetfillcolor{currentfill}%
\pgfsetfillopacity{0.300000}%
\pgfsetlinewidth{1.003750pt}%
\definecolor{currentstroke}{rgb}{1.000000,0.498039,0.054902}%
\pgfsetstrokecolor{currentstroke}%
\pgfsetstrokeopacity{0.300000}%
\pgfsetdash{}{0pt}%
\pgfpathmoveto{\pgfqpoint{1.662771in}{2.104072in}}%
\pgfpathcurveto{\pgfqpoint{1.671007in}{2.104072in}}{\pgfqpoint{1.678907in}{2.107345in}}{\pgfqpoint{1.684731in}{2.113169in}}%
\pgfpathcurveto{\pgfqpoint{1.690555in}{2.118992in}}{\pgfqpoint{1.693827in}{2.126893in}}{\pgfqpoint{1.693827in}{2.135129in}}%
\pgfpathcurveto{\pgfqpoint{1.693827in}{2.143365in}}{\pgfqpoint{1.690555in}{2.151265in}}{\pgfqpoint{1.684731in}{2.157089in}}%
\pgfpathcurveto{\pgfqpoint{1.678907in}{2.162913in}}{\pgfqpoint{1.671007in}{2.166185in}}{\pgfqpoint{1.662771in}{2.166185in}}%
\pgfpathcurveto{\pgfqpoint{1.654534in}{2.166185in}}{\pgfqpoint{1.646634in}{2.162913in}}{\pgfqpoint{1.640810in}{2.157089in}}%
\pgfpathcurveto{\pgfqpoint{1.634986in}{2.151265in}}{\pgfqpoint{1.631714in}{2.143365in}}{\pgfqpoint{1.631714in}{2.135129in}}%
\pgfpathcurveto{\pgfqpoint{1.631714in}{2.126893in}}{\pgfqpoint{1.634986in}{2.118992in}}{\pgfqpoint{1.640810in}{2.113169in}}%
\pgfpathcurveto{\pgfqpoint{1.646634in}{2.107345in}}{\pgfqpoint{1.654534in}{2.104072in}}{\pgfqpoint{1.662771in}{2.104072in}}%
\pgfpathclose%
\pgfusepath{stroke,fill}%
\end{pgfscope}%
\begin{pgfscope}%
\pgfpathrectangle{\pgfqpoint{0.100000in}{0.212622in}}{\pgfqpoint{3.696000in}{3.696000in}}%
\pgfusepath{clip}%
\pgfsetbuttcap%
\pgfsetroundjoin%
\definecolor{currentfill}{rgb}{1.000000,0.498039,0.054902}%
\pgfsetfillcolor{currentfill}%
\pgfsetfillopacity{0.626957}%
\pgfsetlinewidth{1.003750pt}%
\definecolor{currentstroke}{rgb}{1.000000,0.498039,0.054902}%
\pgfsetstrokecolor{currentstroke}%
\pgfsetstrokeopacity{0.626957}%
\pgfsetdash{}{0pt}%
\pgfpathmoveto{\pgfqpoint{3.288684in}{1.996556in}}%
\pgfpathcurveto{\pgfqpoint{3.296920in}{1.996556in}}{\pgfqpoint{3.304821in}{1.999828in}}{\pgfqpoint{3.310644in}{2.005652in}}%
\pgfpathcurveto{\pgfqpoint{3.316468in}{2.011476in}}{\pgfqpoint{3.319741in}{2.019376in}}{\pgfqpoint{3.319741in}{2.027612in}}%
\pgfpathcurveto{\pgfqpoint{3.319741in}{2.035849in}}{\pgfqpoint{3.316468in}{2.043749in}}{\pgfqpoint{3.310644in}{2.049573in}}%
\pgfpathcurveto{\pgfqpoint{3.304821in}{2.055396in}}{\pgfqpoint{3.296920in}{2.058669in}}{\pgfqpoint{3.288684in}{2.058669in}}%
\pgfpathcurveto{\pgfqpoint{3.280448in}{2.058669in}}{\pgfqpoint{3.272548in}{2.055396in}}{\pgfqpoint{3.266724in}{2.049573in}}%
\pgfpathcurveto{\pgfqpoint{3.260900in}{2.043749in}}{\pgfqpoint{3.257628in}{2.035849in}}{\pgfqpoint{3.257628in}{2.027612in}}%
\pgfpathcurveto{\pgfqpoint{3.257628in}{2.019376in}}{\pgfqpoint{3.260900in}{2.011476in}}{\pgfqpoint{3.266724in}{2.005652in}}%
\pgfpathcurveto{\pgfqpoint{3.272548in}{1.999828in}}{\pgfqpoint{3.280448in}{1.996556in}}{\pgfqpoint{3.288684in}{1.996556in}}%
\pgfpathclose%
\pgfusepath{stroke,fill}%
\end{pgfscope}%
\begin{pgfscope}%
\pgfpathrectangle{\pgfqpoint{0.100000in}{0.212622in}}{\pgfqpoint{3.696000in}{3.696000in}}%
\pgfusepath{clip}%
\pgfsetbuttcap%
\pgfsetroundjoin%
\definecolor{currentfill}{rgb}{1.000000,0.498039,0.054902}%
\pgfsetfillcolor{currentfill}%
\pgfsetfillopacity{0.801706}%
\pgfsetlinewidth{1.003750pt}%
\definecolor{currentstroke}{rgb}{1.000000,0.498039,0.054902}%
\pgfsetstrokecolor{currentstroke}%
\pgfsetstrokeopacity{0.801706}%
\pgfsetdash{}{0pt}%
\pgfpathmoveto{\pgfqpoint{0.672869in}{1.566492in}}%
\pgfpathcurveto{\pgfqpoint{0.681106in}{1.566492in}}{\pgfqpoint{0.689006in}{1.569764in}}{\pgfqpoint{0.694830in}{1.575588in}}%
\pgfpathcurveto{\pgfqpoint{0.700653in}{1.581412in}}{\pgfqpoint{0.703926in}{1.589312in}}{\pgfqpoint{0.703926in}{1.597548in}}%
\pgfpathcurveto{\pgfqpoint{0.703926in}{1.605785in}}{\pgfqpoint{0.700653in}{1.613685in}}{\pgfqpoint{0.694830in}{1.619509in}}%
\pgfpathcurveto{\pgfqpoint{0.689006in}{1.625332in}}{\pgfqpoint{0.681106in}{1.628605in}}{\pgfqpoint{0.672869in}{1.628605in}}%
\pgfpathcurveto{\pgfqpoint{0.664633in}{1.628605in}}{\pgfqpoint{0.656733in}{1.625332in}}{\pgfqpoint{0.650909in}{1.619509in}}%
\pgfpathcurveto{\pgfqpoint{0.645085in}{1.613685in}}{\pgfqpoint{0.641813in}{1.605785in}}{\pgfqpoint{0.641813in}{1.597548in}}%
\pgfpathcurveto{\pgfqpoint{0.641813in}{1.589312in}}{\pgfqpoint{0.645085in}{1.581412in}}{\pgfqpoint{0.650909in}{1.575588in}}%
\pgfpathcurveto{\pgfqpoint{0.656733in}{1.569764in}}{\pgfqpoint{0.664633in}{1.566492in}}{\pgfqpoint{0.672869in}{1.566492in}}%
\pgfpathclose%
\pgfusepath{stroke,fill}%
\end{pgfscope}%
\begin{pgfscope}%
\pgfpathrectangle{\pgfqpoint{0.100000in}{0.212622in}}{\pgfqpoint{3.696000in}{3.696000in}}%
\pgfusepath{clip}%
\pgfsetbuttcap%
\pgfsetroundjoin%
\definecolor{currentfill}{rgb}{1.000000,0.498039,0.054902}%
\pgfsetfillcolor{currentfill}%
\pgfsetfillopacity{0.970918}%
\pgfsetlinewidth{1.003750pt}%
\definecolor{currentstroke}{rgb}{1.000000,0.498039,0.054902}%
\pgfsetstrokecolor{currentstroke}%
\pgfsetstrokeopacity{0.970918}%
\pgfsetdash{}{0pt}%
\pgfpathmoveto{\pgfqpoint{0.531950in}{2.582831in}}%
\pgfpathcurveto{\pgfqpoint{0.540186in}{2.582831in}}{\pgfqpoint{0.548087in}{2.586104in}}{\pgfqpoint{0.553910in}{2.591928in}}%
\pgfpathcurveto{\pgfqpoint{0.559734in}{2.597751in}}{\pgfqpoint{0.563007in}{2.605652in}}{\pgfqpoint{0.563007in}{2.613888in}}%
\pgfpathcurveto{\pgfqpoint{0.563007in}{2.622124in}}{\pgfqpoint{0.559734in}{2.630024in}}{\pgfqpoint{0.553910in}{2.635848in}}%
\pgfpathcurveto{\pgfqpoint{0.548087in}{2.641672in}}{\pgfqpoint{0.540186in}{2.644944in}}{\pgfqpoint{0.531950in}{2.644944in}}%
\pgfpathcurveto{\pgfqpoint{0.523714in}{2.644944in}}{\pgfqpoint{0.515814in}{2.641672in}}{\pgfqpoint{0.509990in}{2.635848in}}%
\pgfpathcurveto{\pgfqpoint{0.504166in}{2.630024in}}{\pgfqpoint{0.500894in}{2.622124in}}{\pgfqpoint{0.500894in}{2.613888in}}%
\pgfpathcurveto{\pgfqpoint{0.500894in}{2.605652in}}{\pgfqpoint{0.504166in}{2.597751in}}{\pgfqpoint{0.509990in}{2.591928in}}%
\pgfpathcurveto{\pgfqpoint{0.515814in}{2.586104in}}{\pgfqpoint{0.523714in}{2.582831in}}{\pgfqpoint{0.531950in}{2.582831in}}%
\pgfpathclose%
\pgfusepath{stroke,fill}%
\end{pgfscope}%
\begin{pgfscope}%
\pgfpathrectangle{\pgfqpoint{0.100000in}{0.212622in}}{\pgfqpoint{3.696000in}{3.696000in}}%
\pgfusepath{clip}%
\pgfsetbuttcap%
\pgfsetroundjoin%
\definecolor{currentfill}{rgb}{1.000000,0.498039,0.054902}%
\pgfsetfillcolor{currentfill}%
\pgfsetlinewidth{1.003750pt}%
\definecolor{currentstroke}{rgb}{1.000000,0.498039,0.054902}%
\pgfsetstrokecolor{currentstroke}%
\pgfsetdash{}{0pt}%
\pgfpathmoveto{\pgfqpoint{2.722088in}{1.707191in}}%
\pgfpathcurveto{\pgfqpoint{2.730324in}{1.707191in}}{\pgfqpoint{2.738224in}{1.710463in}}{\pgfqpoint{2.744048in}{1.716287in}}%
\pgfpathcurveto{\pgfqpoint{2.749872in}{1.722111in}}{\pgfqpoint{2.753144in}{1.730011in}}{\pgfqpoint{2.753144in}{1.738247in}}%
\pgfpathcurveto{\pgfqpoint{2.753144in}{1.746484in}}{\pgfqpoint{2.749872in}{1.754384in}}{\pgfqpoint{2.744048in}{1.760208in}}%
\pgfpathcurveto{\pgfqpoint{2.738224in}{1.766032in}}{\pgfqpoint{2.730324in}{1.769304in}}{\pgfqpoint{2.722088in}{1.769304in}}%
\pgfpathcurveto{\pgfqpoint{2.713852in}{1.769304in}}{\pgfqpoint{2.705951in}{1.766032in}}{\pgfqpoint{2.700128in}{1.760208in}}%
\pgfpathcurveto{\pgfqpoint{2.694304in}{1.754384in}}{\pgfqpoint{2.691031in}{1.746484in}}{\pgfqpoint{2.691031in}{1.738247in}}%
\pgfpathcurveto{\pgfqpoint{2.691031in}{1.730011in}}{\pgfqpoint{2.694304in}{1.722111in}}{\pgfqpoint{2.700128in}{1.716287in}}%
\pgfpathcurveto{\pgfqpoint{2.705951in}{1.710463in}}{\pgfqpoint{2.713852in}{1.707191in}}{\pgfqpoint{2.722088in}{1.707191in}}%
\pgfpathclose%
\pgfusepath{stroke,fill}%
\end{pgfscope}%
\begin{pgfscope}%
\definecolor{textcolor}{rgb}{0.000000,0.000000,0.000000}%
\pgfsetstrokecolor{textcolor}%
\pgfsetfillcolor{textcolor}%
\pgftext[x=1.948000in,y=3.991956in,,base]{\color{textcolor}\rmfamily\fontsize{12.000000}{14.400000}\selectfont Mahony}%
\end{pgfscope}%
\begin{pgfscope}%
\pgfpathrectangle{\pgfqpoint{0.100000in}{0.212622in}}{\pgfqpoint{3.696000in}{3.696000in}}%
\pgfusepath{clip}%
\pgfsetbuttcap%
\pgfsetroundjoin%
\definecolor{currentfill}{rgb}{0.121569,0.466667,0.705882}%
\pgfsetfillcolor{currentfill}%
\pgfsetfillopacity{0.300000}%
\pgfsetlinewidth{1.003750pt}%
\definecolor{currentstroke}{rgb}{0.121569,0.466667,0.705882}%
\pgfsetstrokecolor{currentstroke}%
\pgfsetstrokeopacity{0.300000}%
\pgfsetdash{}{0pt}%
\pgfpathmoveto{\pgfqpoint{1.660109in}{2.103696in}}%
\pgfpathcurveto{\pgfqpoint{1.668346in}{2.103696in}}{\pgfqpoint{1.676246in}{2.106968in}}{\pgfqpoint{1.682070in}{2.112792in}}%
\pgfpathcurveto{\pgfqpoint{1.687893in}{2.118616in}}{\pgfqpoint{1.691166in}{2.126516in}}{\pgfqpoint{1.691166in}{2.134752in}}%
\pgfpathcurveto{\pgfqpoint{1.691166in}{2.142989in}}{\pgfqpoint{1.687893in}{2.150889in}}{\pgfqpoint{1.682070in}{2.156713in}}%
\pgfpathcurveto{\pgfqpoint{1.676246in}{2.162536in}}{\pgfqpoint{1.668346in}{2.165809in}}{\pgfqpoint{1.660109in}{2.165809in}}%
\pgfpathcurveto{\pgfqpoint{1.651873in}{2.165809in}}{\pgfqpoint{1.643973in}{2.162536in}}{\pgfqpoint{1.638149in}{2.156713in}}%
\pgfpathcurveto{\pgfqpoint{1.632325in}{2.150889in}}{\pgfqpoint{1.629053in}{2.142989in}}{\pgfqpoint{1.629053in}{2.134752in}}%
\pgfpathcurveto{\pgfqpoint{1.629053in}{2.126516in}}{\pgfqpoint{1.632325in}{2.118616in}}{\pgfqpoint{1.638149in}{2.112792in}}%
\pgfpathcurveto{\pgfqpoint{1.643973in}{2.106968in}}{\pgfqpoint{1.651873in}{2.103696in}}{\pgfqpoint{1.660109in}{2.103696in}}%
\pgfpathclose%
\pgfusepath{stroke,fill}%
\end{pgfscope}%
\begin{pgfscope}%
\pgfpathrectangle{\pgfqpoint{0.100000in}{0.212622in}}{\pgfqpoint{3.696000in}{3.696000in}}%
\pgfusepath{clip}%
\pgfsetbuttcap%
\pgfsetroundjoin%
\definecolor{currentfill}{rgb}{0.121569,0.466667,0.705882}%
\pgfsetfillcolor{currentfill}%
\pgfsetfillopacity{0.300003}%
\pgfsetlinewidth{1.003750pt}%
\definecolor{currentstroke}{rgb}{0.121569,0.466667,0.705882}%
\pgfsetstrokecolor{currentstroke}%
\pgfsetstrokeopacity{0.300003}%
\pgfsetdash{}{0pt}%
\pgfpathmoveto{\pgfqpoint{1.659404in}{2.103542in}}%
\pgfpathcurveto{\pgfqpoint{1.667640in}{2.103542in}}{\pgfqpoint{1.675540in}{2.106815in}}{\pgfqpoint{1.681364in}{2.112639in}}%
\pgfpathcurveto{\pgfqpoint{1.687188in}{2.118463in}}{\pgfqpoint{1.690460in}{2.126363in}}{\pgfqpoint{1.690460in}{2.134599in}}%
\pgfpathcurveto{\pgfqpoint{1.690460in}{2.142835in}}{\pgfqpoint{1.687188in}{2.150735in}}{\pgfqpoint{1.681364in}{2.156559in}}%
\pgfpathcurveto{\pgfqpoint{1.675540in}{2.162383in}}{\pgfqpoint{1.667640in}{2.165655in}}{\pgfqpoint{1.659404in}{2.165655in}}%
\pgfpathcurveto{\pgfqpoint{1.651168in}{2.165655in}}{\pgfqpoint{1.643267in}{2.162383in}}{\pgfqpoint{1.637444in}{2.156559in}}%
\pgfpathcurveto{\pgfqpoint{1.631620in}{2.150735in}}{\pgfqpoint{1.628347in}{2.142835in}}{\pgfqpoint{1.628347in}{2.134599in}}%
\pgfpathcurveto{\pgfqpoint{1.628347in}{2.126363in}}{\pgfqpoint{1.631620in}{2.118463in}}{\pgfqpoint{1.637444in}{2.112639in}}%
\pgfpathcurveto{\pgfqpoint{1.643267in}{2.106815in}}{\pgfqpoint{1.651168in}{2.103542in}}{\pgfqpoint{1.659404in}{2.103542in}}%
\pgfpathclose%
\pgfusepath{stroke,fill}%
\end{pgfscope}%
\begin{pgfscope}%
\pgfpathrectangle{\pgfqpoint{0.100000in}{0.212622in}}{\pgfqpoint{3.696000in}{3.696000in}}%
\pgfusepath{clip}%
\pgfsetbuttcap%
\pgfsetroundjoin%
\definecolor{currentfill}{rgb}{0.121569,0.466667,0.705882}%
\pgfsetfillcolor{currentfill}%
\pgfsetfillopacity{0.300011}%
\pgfsetlinewidth{1.003750pt}%
\definecolor{currentstroke}{rgb}{0.121569,0.466667,0.705882}%
\pgfsetstrokecolor{currentstroke}%
\pgfsetstrokeopacity{0.300011}%
\pgfsetdash{}{0pt}%
\pgfpathmoveto{\pgfqpoint{1.659276in}{2.103551in}}%
\pgfpathcurveto{\pgfqpoint{1.667513in}{2.103551in}}{\pgfqpoint{1.675413in}{2.106823in}}{\pgfqpoint{1.681237in}{2.112647in}}%
\pgfpathcurveto{\pgfqpoint{1.687061in}{2.118471in}}{\pgfqpoint{1.690333in}{2.126371in}}{\pgfqpoint{1.690333in}{2.134607in}}%
\pgfpathcurveto{\pgfqpoint{1.690333in}{2.142844in}}{\pgfqpoint{1.687061in}{2.150744in}}{\pgfqpoint{1.681237in}{2.156568in}}%
\pgfpathcurveto{\pgfqpoint{1.675413in}{2.162391in}}{\pgfqpoint{1.667513in}{2.165664in}}{\pgfqpoint{1.659276in}{2.165664in}}%
\pgfpathcurveto{\pgfqpoint{1.651040in}{2.165664in}}{\pgfqpoint{1.643140in}{2.162391in}}{\pgfqpoint{1.637316in}{2.156568in}}%
\pgfpathcurveto{\pgfqpoint{1.631492in}{2.150744in}}{\pgfqpoint{1.628220in}{2.142844in}}{\pgfqpoint{1.628220in}{2.134607in}}%
\pgfpathcurveto{\pgfqpoint{1.628220in}{2.126371in}}{\pgfqpoint{1.631492in}{2.118471in}}{\pgfqpoint{1.637316in}{2.112647in}}%
\pgfpathcurveto{\pgfqpoint{1.643140in}{2.106823in}}{\pgfqpoint{1.651040in}{2.103551in}}{\pgfqpoint{1.659276in}{2.103551in}}%
\pgfpathclose%
\pgfusepath{stroke,fill}%
\end{pgfscope}%
\begin{pgfscope}%
\pgfpathrectangle{\pgfqpoint{0.100000in}{0.212622in}}{\pgfqpoint{3.696000in}{3.696000in}}%
\pgfusepath{clip}%
\pgfsetbuttcap%
\pgfsetroundjoin%
\definecolor{currentfill}{rgb}{0.121569,0.466667,0.705882}%
\pgfsetfillcolor{currentfill}%
\pgfsetfillopacity{0.300028}%
\pgfsetlinewidth{1.003750pt}%
\definecolor{currentstroke}{rgb}{0.121569,0.466667,0.705882}%
\pgfsetstrokecolor{currentstroke}%
\pgfsetstrokeopacity{0.300028}%
\pgfsetdash{}{0pt}%
\pgfpathmoveto{\pgfqpoint{1.659074in}{2.103428in}}%
\pgfpathcurveto{\pgfqpoint{1.667310in}{2.103428in}}{\pgfqpoint{1.675210in}{2.106700in}}{\pgfqpoint{1.681034in}{2.112524in}}%
\pgfpathcurveto{\pgfqpoint{1.686858in}{2.118348in}}{\pgfqpoint{1.690130in}{2.126248in}}{\pgfqpoint{1.690130in}{2.134485in}}%
\pgfpathcurveto{\pgfqpoint{1.690130in}{2.142721in}}{\pgfqpoint{1.686858in}{2.150621in}}{\pgfqpoint{1.681034in}{2.156445in}}%
\pgfpathcurveto{\pgfqpoint{1.675210in}{2.162269in}}{\pgfqpoint{1.667310in}{2.165541in}}{\pgfqpoint{1.659074in}{2.165541in}}%
\pgfpathcurveto{\pgfqpoint{1.650837in}{2.165541in}}{\pgfqpoint{1.642937in}{2.162269in}}{\pgfqpoint{1.637113in}{2.156445in}}%
\pgfpathcurveto{\pgfqpoint{1.631289in}{2.150621in}}{\pgfqpoint{1.628017in}{2.142721in}}{\pgfqpoint{1.628017in}{2.134485in}}%
\pgfpathcurveto{\pgfqpoint{1.628017in}{2.126248in}}{\pgfqpoint{1.631289in}{2.118348in}}{\pgfqpoint{1.637113in}{2.112524in}}%
\pgfpathcurveto{\pgfqpoint{1.642937in}{2.106700in}}{\pgfqpoint{1.650837in}{2.103428in}}{\pgfqpoint{1.659074in}{2.103428in}}%
\pgfpathclose%
\pgfusepath{stroke,fill}%
\end{pgfscope}%
\begin{pgfscope}%
\pgfpathrectangle{\pgfqpoint{0.100000in}{0.212622in}}{\pgfqpoint{3.696000in}{3.696000in}}%
\pgfusepath{clip}%
\pgfsetbuttcap%
\pgfsetroundjoin%
\definecolor{currentfill}{rgb}{0.121569,0.466667,0.705882}%
\pgfsetfillcolor{currentfill}%
\pgfsetfillopacity{0.300081}%
\pgfsetlinewidth{1.003750pt}%
\definecolor{currentstroke}{rgb}{0.121569,0.466667,0.705882}%
\pgfsetstrokecolor{currentstroke}%
\pgfsetstrokeopacity{0.300081}%
\pgfsetdash{}{0pt}%
\pgfpathmoveto{\pgfqpoint{1.661574in}{2.103903in}}%
\pgfpathcurveto{\pgfqpoint{1.669811in}{2.103903in}}{\pgfqpoint{1.677711in}{2.107175in}}{\pgfqpoint{1.683535in}{2.112999in}}%
\pgfpathcurveto{\pgfqpoint{1.689359in}{2.118823in}}{\pgfqpoint{1.692631in}{2.126723in}}{\pgfqpoint{1.692631in}{2.134959in}}%
\pgfpathcurveto{\pgfqpoint{1.692631in}{2.143195in}}{\pgfqpoint{1.689359in}{2.151095in}}{\pgfqpoint{1.683535in}{2.156919in}}%
\pgfpathcurveto{\pgfqpoint{1.677711in}{2.162743in}}{\pgfqpoint{1.669811in}{2.166016in}}{\pgfqpoint{1.661574in}{2.166016in}}%
\pgfpathcurveto{\pgfqpoint{1.653338in}{2.166016in}}{\pgfqpoint{1.645438in}{2.162743in}}{\pgfqpoint{1.639614in}{2.156919in}}%
\pgfpathcurveto{\pgfqpoint{1.633790in}{2.151095in}}{\pgfqpoint{1.630518in}{2.143195in}}{\pgfqpoint{1.630518in}{2.134959in}}%
\pgfpathcurveto{\pgfqpoint{1.630518in}{2.126723in}}{\pgfqpoint{1.633790in}{2.118823in}}{\pgfqpoint{1.639614in}{2.112999in}}%
\pgfpathcurveto{\pgfqpoint{1.645438in}{2.107175in}}{\pgfqpoint{1.653338in}{2.103903in}}{\pgfqpoint{1.661574in}{2.103903in}}%
\pgfpathclose%
\pgfusepath{stroke,fill}%
\end{pgfscope}%
\begin{pgfscope}%
\pgfpathrectangle{\pgfqpoint{0.100000in}{0.212622in}}{\pgfqpoint{3.696000in}{3.696000in}}%
\pgfusepath{clip}%
\pgfsetbuttcap%
\pgfsetroundjoin%
\definecolor{currentfill}{rgb}{0.121569,0.466667,0.705882}%
\pgfsetfillcolor{currentfill}%
\pgfsetfillopacity{0.300085}%
\pgfsetlinewidth{1.003750pt}%
\definecolor{currentstroke}{rgb}{0.121569,0.466667,0.705882}%
\pgfsetstrokecolor{currentstroke}%
\pgfsetstrokeopacity{0.300085}%
\pgfsetdash{}{0pt}%
\pgfpathmoveto{\pgfqpoint{1.658704in}{2.103406in}}%
\pgfpathcurveto{\pgfqpoint{1.666940in}{2.103406in}}{\pgfqpoint{1.674840in}{2.106678in}}{\pgfqpoint{1.680664in}{2.112502in}}%
\pgfpathcurveto{\pgfqpoint{1.686488in}{2.118326in}}{\pgfqpoint{1.689760in}{2.126226in}}{\pgfqpoint{1.689760in}{2.134463in}}%
\pgfpathcurveto{\pgfqpoint{1.689760in}{2.142699in}}{\pgfqpoint{1.686488in}{2.150599in}}{\pgfqpoint{1.680664in}{2.156423in}}%
\pgfpathcurveto{\pgfqpoint{1.674840in}{2.162247in}}{\pgfqpoint{1.666940in}{2.165519in}}{\pgfqpoint{1.658704in}{2.165519in}}%
\pgfpathcurveto{\pgfqpoint{1.650467in}{2.165519in}}{\pgfqpoint{1.642567in}{2.162247in}}{\pgfqpoint{1.636743in}{2.156423in}}%
\pgfpathcurveto{\pgfqpoint{1.630919in}{2.150599in}}{\pgfqpoint{1.627647in}{2.142699in}}{\pgfqpoint{1.627647in}{2.134463in}}%
\pgfpathcurveto{\pgfqpoint{1.627647in}{2.126226in}}{\pgfqpoint{1.630919in}{2.118326in}}{\pgfqpoint{1.636743in}{2.112502in}}%
\pgfpathcurveto{\pgfqpoint{1.642567in}{2.106678in}}{\pgfqpoint{1.650467in}{2.103406in}}{\pgfqpoint{1.658704in}{2.103406in}}%
\pgfpathclose%
\pgfusepath{stroke,fill}%
\end{pgfscope}%
\begin{pgfscope}%
\pgfpathrectangle{\pgfqpoint{0.100000in}{0.212622in}}{\pgfqpoint{3.696000in}{3.696000in}}%
\pgfusepath{clip}%
\pgfsetbuttcap%
\pgfsetroundjoin%
\definecolor{currentfill}{rgb}{0.121569,0.466667,0.705882}%
\pgfsetfillcolor{currentfill}%
\pgfsetfillopacity{0.300148}%
\pgfsetlinewidth{1.003750pt}%
\definecolor{currentstroke}{rgb}{0.121569,0.466667,0.705882}%
\pgfsetstrokecolor{currentstroke}%
\pgfsetstrokeopacity{0.300148}%
\pgfsetdash{}{0pt}%
\pgfpathmoveto{\pgfqpoint{1.662350in}{2.103937in}}%
\pgfpathcurveto{\pgfqpoint{1.670586in}{2.103937in}}{\pgfqpoint{1.678486in}{2.107209in}}{\pgfqpoint{1.684310in}{2.113033in}}%
\pgfpathcurveto{\pgfqpoint{1.690134in}{2.118857in}}{\pgfqpoint{1.693406in}{2.126757in}}{\pgfqpoint{1.693406in}{2.134994in}}%
\pgfpathcurveto{\pgfqpoint{1.693406in}{2.143230in}}{\pgfqpoint{1.690134in}{2.151130in}}{\pgfqpoint{1.684310in}{2.156954in}}%
\pgfpathcurveto{\pgfqpoint{1.678486in}{2.162778in}}{\pgfqpoint{1.670586in}{2.166050in}}{\pgfqpoint{1.662350in}{2.166050in}}%
\pgfpathcurveto{\pgfqpoint{1.654113in}{2.166050in}}{\pgfqpoint{1.646213in}{2.162778in}}{\pgfqpoint{1.640389in}{2.156954in}}%
\pgfpathcurveto{\pgfqpoint{1.634566in}{2.151130in}}{\pgfqpoint{1.631293in}{2.143230in}}{\pgfqpoint{1.631293in}{2.134994in}}%
\pgfpathcurveto{\pgfqpoint{1.631293in}{2.126757in}}{\pgfqpoint{1.634566in}{2.118857in}}{\pgfqpoint{1.640389in}{2.113033in}}%
\pgfpathcurveto{\pgfqpoint{1.646213in}{2.107209in}}{\pgfqpoint{1.654113in}{2.103937in}}{\pgfqpoint{1.662350in}{2.103937in}}%
\pgfpathclose%
\pgfusepath{stroke,fill}%
\end{pgfscope}%
\begin{pgfscope}%
\pgfpathrectangle{\pgfqpoint{0.100000in}{0.212622in}}{\pgfqpoint{3.696000in}{3.696000in}}%
\pgfusepath{clip}%
\pgfsetbuttcap%
\pgfsetroundjoin%
\definecolor{currentfill}{rgb}{0.121569,0.466667,0.705882}%
\pgfsetfillcolor{currentfill}%
\pgfsetfillopacity{0.300197}%
\pgfsetlinewidth{1.003750pt}%
\definecolor{currentstroke}{rgb}{0.121569,0.466667,0.705882}%
\pgfsetstrokecolor{currentstroke}%
\pgfsetstrokeopacity{0.300197}%
\pgfsetdash{}{0pt}%
\pgfpathmoveto{\pgfqpoint{1.658218in}{2.103001in}}%
\pgfpathcurveto{\pgfqpoint{1.666455in}{2.103001in}}{\pgfqpoint{1.674355in}{2.106273in}}{\pgfqpoint{1.680179in}{2.112097in}}%
\pgfpathcurveto{\pgfqpoint{1.686003in}{2.117921in}}{\pgfqpoint{1.689275in}{2.125821in}}{\pgfqpoint{1.689275in}{2.134058in}}%
\pgfpathcurveto{\pgfqpoint{1.689275in}{2.142294in}}{\pgfqpoint{1.686003in}{2.150194in}}{\pgfqpoint{1.680179in}{2.156018in}}%
\pgfpathcurveto{\pgfqpoint{1.674355in}{2.161842in}}{\pgfqpoint{1.666455in}{2.165114in}}{\pgfqpoint{1.658218in}{2.165114in}}%
\pgfpathcurveto{\pgfqpoint{1.649982in}{2.165114in}}{\pgfqpoint{1.642082in}{2.161842in}}{\pgfqpoint{1.636258in}{2.156018in}}%
\pgfpathcurveto{\pgfqpoint{1.630434in}{2.150194in}}{\pgfqpoint{1.627162in}{2.142294in}}{\pgfqpoint{1.627162in}{2.134058in}}%
\pgfpathcurveto{\pgfqpoint{1.627162in}{2.125821in}}{\pgfqpoint{1.630434in}{2.117921in}}{\pgfqpoint{1.636258in}{2.112097in}}%
\pgfpathcurveto{\pgfqpoint{1.642082in}{2.106273in}}{\pgfqpoint{1.649982in}{2.103001in}}{\pgfqpoint{1.658218in}{2.103001in}}%
\pgfpathclose%
\pgfusepath{stroke,fill}%
\end{pgfscope}%
\begin{pgfscope}%
\pgfpathrectangle{\pgfqpoint{0.100000in}{0.212622in}}{\pgfqpoint{3.696000in}{3.696000in}}%
\pgfusepath{clip}%
\pgfsetbuttcap%
\pgfsetroundjoin%
\definecolor{currentfill}{rgb}{0.121569,0.466667,0.705882}%
\pgfsetfillcolor{currentfill}%
\pgfsetfillopacity{0.300198}%
\pgfsetlinewidth{1.003750pt}%
\definecolor{currentstroke}{rgb}{0.121569,0.466667,0.705882}%
\pgfsetstrokecolor{currentstroke}%
\pgfsetstrokeopacity{0.300198}%
\pgfsetdash{}{0pt}%
\pgfpathmoveto{\pgfqpoint{1.658215in}{2.102997in}}%
\pgfpathcurveto{\pgfqpoint{1.666451in}{2.102997in}}{\pgfqpoint{1.674351in}{2.106269in}}{\pgfqpoint{1.680175in}{2.112093in}}%
\pgfpathcurveto{\pgfqpoint{1.685999in}{2.117917in}}{\pgfqpoint{1.689271in}{2.125817in}}{\pgfqpoint{1.689271in}{2.134054in}}%
\pgfpathcurveto{\pgfqpoint{1.689271in}{2.142290in}}{\pgfqpoint{1.685999in}{2.150190in}}{\pgfqpoint{1.680175in}{2.156014in}}%
\pgfpathcurveto{\pgfqpoint{1.674351in}{2.161838in}}{\pgfqpoint{1.666451in}{2.165110in}}{\pgfqpoint{1.658215in}{2.165110in}}%
\pgfpathcurveto{\pgfqpoint{1.649978in}{2.165110in}}{\pgfqpoint{1.642078in}{2.161838in}}{\pgfqpoint{1.636255in}{2.156014in}}%
\pgfpathcurveto{\pgfqpoint{1.630431in}{2.150190in}}{\pgfqpoint{1.627158in}{2.142290in}}{\pgfqpoint{1.627158in}{2.134054in}}%
\pgfpathcurveto{\pgfqpoint{1.627158in}{2.125817in}}{\pgfqpoint{1.630431in}{2.117917in}}{\pgfqpoint{1.636255in}{2.112093in}}%
\pgfpathcurveto{\pgfqpoint{1.642078in}{2.106269in}}{\pgfqpoint{1.649978in}{2.102997in}}{\pgfqpoint{1.658215in}{2.102997in}}%
\pgfpathclose%
\pgfusepath{stroke,fill}%
\end{pgfscope}%
\begin{pgfscope}%
\pgfpathrectangle{\pgfqpoint{0.100000in}{0.212622in}}{\pgfqpoint{3.696000in}{3.696000in}}%
\pgfusepath{clip}%
\pgfsetbuttcap%
\pgfsetroundjoin%
\definecolor{currentfill}{rgb}{0.121569,0.466667,0.705882}%
\pgfsetfillcolor{currentfill}%
\pgfsetfillopacity{0.300200}%
\pgfsetlinewidth{1.003750pt}%
\definecolor{currentstroke}{rgb}{0.121569,0.466667,0.705882}%
\pgfsetstrokecolor{currentstroke}%
\pgfsetstrokeopacity{0.300200}%
\pgfsetdash{}{0pt}%
\pgfpathmoveto{\pgfqpoint{1.658209in}{2.102994in}}%
\pgfpathcurveto{\pgfqpoint{1.666445in}{2.102994in}}{\pgfqpoint{1.674345in}{2.106266in}}{\pgfqpoint{1.680169in}{2.112090in}}%
\pgfpathcurveto{\pgfqpoint{1.685993in}{2.117914in}}{\pgfqpoint{1.689265in}{2.125814in}}{\pgfqpoint{1.689265in}{2.134050in}}%
\pgfpathcurveto{\pgfqpoint{1.689265in}{2.142287in}}{\pgfqpoint{1.685993in}{2.150187in}}{\pgfqpoint{1.680169in}{2.156011in}}%
\pgfpathcurveto{\pgfqpoint{1.674345in}{2.161835in}}{\pgfqpoint{1.666445in}{2.165107in}}{\pgfqpoint{1.658209in}{2.165107in}}%
\pgfpathcurveto{\pgfqpoint{1.649973in}{2.165107in}}{\pgfqpoint{1.642073in}{2.161835in}}{\pgfqpoint{1.636249in}{2.156011in}}%
\pgfpathcurveto{\pgfqpoint{1.630425in}{2.150187in}}{\pgfqpoint{1.627152in}{2.142287in}}{\pgfqpoint{1.627152in}{2.134050in}}%
\pgfpathcurveto{\pgfqpoint{1.627152in}{2.125814in}}{\pgfqpoint{1.630425in}{2.117914in}}{\pgfqpoint{1.636249in}{2.112090in}}%
\pgfpathcurveto{\pgfqpoint{1.642073in}{2.106266in}}{\pgfqpoint{1.649973in}{2.102994in}}{\pgfqpoint{1.658209in}{2.102994in}}%
\pgfpathclose%
\pgfusepath{stroke,fill}%
\end{pgfscope}%
\begin{pgfscope}%
\pgfpathrectangle{\pgfqpoint{0.100000in}{0.212622in}}{\pgfqpoint{3.696000in}{3.696000in}}%
\pgfusepath{clip}%
\pgfsetbuttcap%
\pgfsetroundjoin%
\definecolor{currentfill}{rgb}{0.121569,0.466667,0.705882}%
\pgfsetfillcolor{currentfill}%
\pgfsetfillopacity{0.300203}%
\pgfsetlinewidth{1.003750pt}%
\definecolor{currentstroke}{rgb}{0.121569,0.466667,0.705882}%
\pgfsetstrokecolor{currentstroke}%
\pgfsetstrokeopacity{0.300203}%
\pgfsetdash{}{0pt}%
\pgfpathmoveto{\pgfqpoint{1.658203in}{2.102986in}}%
\pgfpathcurveto{\pgfqpoint{1.666439in}{2.102986in}}{\pgfqpoint{1.674339in}{2.106258in}}{\pgfqpoint{1.680163in}{2.112082in}}%
\pgfpathcurveto{\pgfqpoint{1.685987in}{2.117906in}}{\pgfqpoint{1.689260in}{2.125806in}}{\pgfqpoint{1.689260in}{2.134042in}}%
\pgfpathcurveto{\pgfqpoint{1.689260in}{2.142278in}}{\pgfqpoint{1.685987in}{2.150178in}}{\pgfqpoint{1.680163in}{2.156002in}}%
\pgfpathcurveto{\pgfqpoint{1.674339in}{2.161826in}}{\pgfqpoint{1.666439in}{2.165099in}}{\pgfqpoint{1.658203in}{2.165099in}}%
\pgfpathcurveto{\pgfqpoint{1.649967in}{2.165099in}}{\pgfqpoint{1.642067in}{2.161826in}}{\pgfqpoint{1.636243in}{2.156002in}}%
\pgfpathcurveto{\pgfqpoint{1.630419in}{2.150178in}}{\pgfqpoint{1.627147in}{2.142278in}}{\pgfqpoint{1.627147in}{2.134042in}}%
\pgfpathcurveto{\pgfqpoint{1.627147in}{2.125806in}}{\pgfqpoint{1.630419in}{2.117906in}}{\pgfqpoint{1.636243in}{2.112082in}}%
\pgfpathcurveto{\pgfqpoint{1.642067in}{2.106258in}}{\pgfqpoint{1.649967in}{2.102986in}}{\pgfqpoint{1.658203in}{2.102986in}}%
\pgfpathclose%
\pgfusepath{stroke,fill}%
\end{pgfscope}%
\begin{pgfscope}%
\pgfpathrectangle{\pgfqpoint{0.100000in}{0.212622in}}{\pgfqpoint{3.696000in}{3.696000in}}%
\pgfusepath{clip}%
\pgfsetbuttcap%
\pgfsetroundjoin%
\definecolor{currentfill}{rgb}{0.121569,0.466667,0.705882}%
\pgfsetfillcolor{currentfill}%
\pgfsetfillopacity{0.300203}%
\pgfsetlinewidth{1.003750pt}%
\definecolor{currentstroke}{rgb}{0.121569,0.466667,0.705882}%
\pgfsetstrokecolor{currentstroke}%
\pgfsetstrokeopacity{0.300203}%
\pgfsetdash{}{0pt}%
\pgfpathmoveto{\pgfqpoint{1.662771in}{2.104072in}}%
\pgfpathcurveto{\pgfqpoint{1.671007in}{2.104072in}}{\pgfqpoint{1.678907in}{2.107345in}}{\pgfqpoint{1.684731in}{2.113169in}}%
\pgfpathcurveto{\pgfqpoint{1.690555in}{2.118992in}}{\pgfqpoint{1.693827in}{2.126893in}}{\pgfqpoint{1.693827in}{2.135129in}}%
\pgfpathcurveto{\pgfqpoint{1.693827in}{2.143365in}}{\pgfqpoint{1.690555in}{2.151265in}}{\pgfqpoint{1.684731in}{2.157089in}}%
\pgfpathcurveto{\pgfqpoint{1.678907in}{2.162913in}}{\pgfqpoint{1.671007in}{2.166185in}}{\pgfqpoint{1.662771in}{2.166185in}}%
\pgfpathcurveto{\pgfqpoint{1.654534in}{2.166185in}}{\pgfqpoint{1.646634in}{2.162913in}}{\pgfqpoint{1.640810in}{2.157089in}}%
\pgfpathcurveto{\pgfqpoint{1.634986in}{2.151265in}}{\pgfqpoint{1.631714in}{2.143365in}}{\pgfqpoint{1.631714in}{2.135129in}}%
\pgfpathcurveto{\pgfqpoint{1.631714in}{2.126893in}}{\pgfqpoint{1.634986in}{2.118992in}}{\pgfqpoint{1.640810in}{2.113169in}}%
\pgfpathcurveto{\pgfqpoint{1.646634in}{2.107345in}}{\pgfqpoint{1.654534in}{2.104072in}}{\pgfqpoint{1.662771in}{2.104072in}}%
\pgfpathclose%
\pgfusepath{stroke,fill}%
\end{pgfscope}%
\begin{pgfscope}%
\pgfpathrectangle{\pgfqpoint{0.100000in}{0.212622in}}{\pgfqpoint{3.696000in}{3.696000in}}%
\pgfusepath{clip}%
\pgfsetbuttcap%
\pgfsetroundjoin%
\definecolor{currentfill}{rgb}{0.121569,0.466667,0.705882}%
\pgfsetfillcolor{currentfill}%
\pgfsetfillopacity{0.300208}%
\pgfsetlinewidth{1.003750pt}%
\definecolor{currentstroke}{rgb}{0.121569,0.466667,0.705882}%
\pgfsetstrokecolor{currentstroke}%
\pgfsetstrokeopacity{0.300208}%
\pgfsetdash{}{0pt}%
\pgfpathmoveto{\pgfqpoint{1.658185in}{2.102970in}}%
\pgfpathcurveto{\pgfqpoint{1.666421in}{2.102970in}}{\pgfqpoint{1.674321in}{2.106242in}}{\pgfqpoint{1.680145in}{2.112066in}}%
\pgfpathcurveto{\pgfqpoint{1.685969in}{2.117890in}}{\pgfqpoint{1.689241in}{2.125790in}}{\pgfqpoint{1.689241in}{2.134026in}}%
\pgfpathcurveto{\pgfqpoint{1.689241in}{2.142262in}}{\pgfqpoint{1.685969in}{2.150162in}}{\pgfqpoint{1.680145in}{2.155986in}}%
\pgfpathcurveto{\pgfqpoint{1.674321in}{2.161810in}}{\pgfqpoint{1.666421in}{2.165083in}}{\pgfqpoint{1.658185in}{2.165083in}}%
\pgfpathcurveto{\pgfqpoint{1.649949in}{2.165083in}}{\pgfqpoint{1.642049in}{2.161810in}}{\pgfqpoint{1.636225in}{2.155986in}}%
\pgfpathcurveto{\pgfqpoint{1.630401in}{2.150162in}}{\pgfqpoint{1.627128in}{2.142262in}}{\pgfqpoint{1.627128in}{2.134026in}}%
\pgfpathcurveto{\pgfqpoint{1.627128in}{2.125790in}}{\pgfqpoint{1.630401in}{2.117890in}}{\pgfqpoint{1.636225in}{2.112066in}}%
\pgfpathcurveto{\pgfqpoint{1.642049in}{2.106242in}}{\pgfqpoint{1.649949in}{2.102970in}}{\pgfqpoint{1.658185in}{2.102970in}}%
\pgfpathclose%
\pgfusepath{stroke,fill}%
\end{pgfscope}%
\begin{pgfscope}%
\pgfpathrectangle{\pgfqpoint{0.100000in}{0.212622in}}{\pgfqpoint{3.696000in}{3.696000in}}%
\pgfusepath{clip}%
\pgfsetbuttcap%
\pgfsetroundjoin%
\definecolor{currentfill}{rgb}{0.121569,0.466667,0.705882}%
\pgfsetfillcolor{currentfill}%
\pgfsetfillopacity{0.300221}%
\pgfsetlinewidth{1.003750pt}%
\definecolor{currentstroke}{rgb}{0.121569,0.466667,0.705882}%
\pgfsetstrokecolor{currentstroke}%
\pgfsetstrokeopacity{0.300221}%
\pgfsetdash{}{0pt}%
\pgfpathmoveto{\pgfqpoint{1.658170in}{2.102952in}}%
\pgfpathcurveto{\pgfqpoint{1.666406in}{2.102952in}}{\pgfqpoint{1.674306in}{2.106224in}}{\pgfqpoint{1.680130in}{2.112048in}}%
\pgfpathcurveto{\pgfqpoint{1.685954in}{2.117872in}}{\pgfqpoint{1.689226in}{2.125772in}}{\pgfqpoint{1.689226in}{2.134008in}}%
\pgfpathcurveto{\pgfqpoint{1.689226in}{2.142245in}}{\pgfqpoint{1.685954in}{2.150145in}}{\pgfqpoint{1.680130in}{2.155969in}}%
\pgfpathcurveto{\pgfqpoint{1.674306in}{2.161793in}}{\pgfqpoint{1.666406in}{2.165065in}}{\pgfqpoint{1.658170in}{2.165065in}}%
\pgfpathcurveto{\pgfqpoint{1.649933in}{2.165065in}}{\pgfqpoint{1.642033in}{2.161793in}}{\pgfqpoint{1.636209in}{2.155969in}}%
\pgfpathcurveto{\pgfqpoint{1.630385in}{2.150145in}}{\pgfqpoint{1.627113in}{2.142245in}}{\pgfqpoint{1.627113in}{2.134008in}}%
\pgfpathcurveto{\pgfqpoint{1.627113in}{2.125772in}}{\pgfqpoint{1.630385in}{2.117872in}}{\pgfqpoint{1.636209in}{2.112048in}}%
\pgfpathcurveto{\pgfqpoint{1.642033in}{2.106224in}}{\pgfqpoint{1.649933in}{2.102952in}}{\pgfqpoint{1.658170in}{2.102952in}}%
\pgfpathclose%
\pgfusepath{stroke,fill}%
\end{pgfscope}%
\begin{pgfscope}%
\pgfpathrectangle{\pgfqpoint{0.100000in}{0.212622in}}{\pgfqpoint{3.696000in}{3.696000in}}%
\pgfusepath{clip}%
\pgfsetbuttcap%
\pgfsetroundjoin%
\definecolor{currentfill}{rgb}{0.121569,0.466667,0.705882}%
\pgfsetfillcolor{currentfill}%
\pgfsetfillopacity{0.300239}%
\pgfsetlinewidth{1.003750pt}%
\definecolor{currentstroke}{rgb}{0.121569,0.466667,0.705882}%
\pgfsetstrokecolor{currentstroke}%
\pgfsetstrokeopacity{0.300239}%
\pgfsetdash{}{0pt}%
\pgfpathmoveto{\pgfqpoint{1.658110in}{2.102903in}}%
\pgfpathcurveto{\pgfqpoint{1.666346in}{2.102903in}}{\pgfqpoint{1.674246in}{2.106175in}}{\pgfqpoint{1.680070in}{2.111999in}}%
\pgfpathcurveto{\pgfqpoint{1.685894in}{2.117823in}}{\pgfqpoint{1.689166in}{2.125723in}}{\pgfqpoint{1.689166in}{2.133959in}}%
\pgfpathcurveto{\pgfqpoint{1.689166in}{2.142195in}}{\pgfqpoint{1.685894in}{2.150096in}}{\pgfqpoint{1.680070in}{2.155919in}}%
\pgfpathcurveto{\pgfqpoint{1.674246in}{2.161743in}}{\pgfqpoint{1.666346in}{2.165016in}}{\pgfqpoint{1.658110in}{2.165016in}}%
\pgfpathcurveto{\pgfqpoint{1.649873in}{2.165016in}}{\pgfqpoint{1.641973in}{2.161743in}}{\pgfqpoint{1.636149in}{2.155919in}}%
\pgfpathcurveto{\pgfqpoint{1.630325in}{2.150096in}}{\pgfqpoint{1.627053in}{2.142195in}}{\pgfqpoint{1.627053in}{2.133959in}}%
\pgfpathcurveto{\pgfqpoint{1.627053in}{2.125723in}}{\pgfqpoint{1.630325in}{2.117823in}}{\pgfqpoint{1.636149in}{2.111999in}}%
\pgfpathcurveto{\pgfqpoint{1.641973in}{2.106175in}}{\pgfqpoint{1.649873in}{2.102903in}}{\pgfqpoint{1.658110in}{2.102903in}}%
\pgfpathclose%
\pgfusepath{stroke,fill}%
\end{pgfscope}%
\begin{pgfscope}%
\pgfpathrectangle{\pgfqpoint{0.100000in}{0.212622in}}{\pgfqpoint{3.696000in}{3.696000in}}%
\pgfusepath{clip}%
\pgfsetbuttcap%
\pgfsetroundjoin%
\definecolor{currentfill}{rgb}{0.121569,0.466667,0.705882}%
\pgfsetfillcolor{currentfill}%
\pgfsetfillopacity{0.300277}%
\pgfsetlinewidth{1.003750pt}%
\definecolor{currentstroke}{rgb}{0.121569,0.466667,0.705882}%
\pgfsetstrokecolor{currentstroke}%
\pgfsetstrokeopacity{0.300277}%
\pgfsetdash{}{0pt}%
\pgfpathmoveto{\pgfqpoint{1.663485in}{2.103699in}}%
\pgfpathcurveto{\pgfqpoint{1.671721in}{2.103699in}}{\pgfqpoint{1.679621in}{2.106971in}}{\pgfqpoint{1.685445in}{2.112795in}}%
\pgfpathcurveto{\pgfqpoint{1.691269in}{2.118619in}}{\pgfqpoint{1.694541in}{2.126519in}}{\pgfqpoint{1.694541in}{2.134755in}}%
\pgfpathcurveto{\pgfqpoint{1.694541in}{2.142991in}}{\pgfqpoint{1.691269in}{2.150891in}}{\pgfqpoint{1.685445in}{2.156715in}}%
\pgfpathcurveto{\pgfqpoint{1.679621in}{2.162539in}}{\pgfqpoint{1.671721in}{2.165812in}}{\pgfqpoint{1.663485in}{2.165812in}}%
\pgfpathcurveto{\pgfqpoint{1.655249in}{2.165812in}}{\pgfqpoint{1.647349in}{2.162539in}}{\pgfqpoint{1.641525in}{2.156715in}}%
\pgfpathcurveto{\pgfqpoint{1.635701in}{2.150891in}}{\pgfqpoint{1.632428in}{2.142991in}}{\pgfqpoint{1.632428in}{2.134755in}}%
\pgfpathcurveto{\pgfqpoint{1.632428in}{2.126519in}}{\pgfqpoint{1.635701in}{2.118619in}}{\pgfqpoint{1.641525in}{2.112795in}}%
\pgfpathcurveto{\pgfqpoint{1.647349in}{2.106971in}}{\pgfqpoint{1.655249in}{2.103699in}}{\pgfqpoint{1.663485in}{2.103699in}}%
\pgfpathclose%
\pgfusepath{stroke,fill}%
\end{pgfscope}%
\begin{pgfscope}%
\pgfpathrectangle{\pgfqpoint{0.100000in}{0.212622in}}{\pgfqpoint{3.696000in}{3.696000in}}%
\pgfusepath{clip}%
\pgfsetbuttcap%
\pgfsetroundjoin%
\definecolor{currentfill}{rgb}{0.121569,0.466667,0.705882}%
\pgfsetfillcolor{currentfill}%
\pgfsetfillopacity{0.300281}%
\pgfsetlinewidth{1.003750pt}%
\definecolor{currentstroke}{rgb}{0.121569,0.466667,0.705882}%
\pgfsetstrokecolor{currentstroke}%
\pgfsetstrokeopacity{0.300281}%
\pgfsetdash{}{0pt}%
\pgfpathmoveto{\pgfqpoint{1.658058in}{2.102840in}}%
\pgfpathcurveto{\pgfqpoint{1.666294in}{2.102840in}}{\pgfqpoint{1.674194in}{2.106113in}}{\pgfqpoint{1.680018in}{2.111936in}}%
\pgfpathcurveto{\pgfqpoint{1.685842in}{2.117760in}}{\pgfqpoint{1.689114in}{2.125660in}}{\pgfqpoint{1.689114in}{2.133897in}}%
\pgfpathcurveto{\pgfqpoint{1.689114in}{2.142133in}}{\pgfqpoint{1.685842in}{2.150033in}}{\pgfqpoint{1.680018in}{2.155857in}}%
\pgfpathcurveto{\pgfqpoint{1.674194in}{2.161681in}}{\pgfqpoint{1.666294in}{2.164953in}}{\pgfqpoint{1.658058in}{2.164953in}}%
\pgfpathcurveto{\pgfqpoint{1.649821in}{2.164953in}}{\pgfqpoint{1.641921in}{2.161681in}}{\pgfqpoint{1.636097in}{2.155857in}}%
\pgfpathcurveto{\pgfqpoint{1.630273in}{2.150033in}}{\pgfqpoint{1.627001in}{2.142133in}}{\pgfqpoint{1.627001in}{2.133897in}}%
\pgfpathcurveto{\pgfqpoint{1.627001in}{2.125660in}}{\pgfqpoint{1.630273in}{2.117760in}}{\pgfqpoint{1.636097in}{2.111936in}}%
\pgfpathcurveto{\pgfqpoint{1.641921in}{2.106113in}}{\pgfqpoint{1.649821in}{2.102840in}}{\pgfqpoint{1.658058in}{2.102840in}}%
\pgfpathclose%
\pgfusepath{stroke,fill}%
\end{pgfscope}%
\begin{pgfscope}%
\pgfpathrectangle{\pgfqpoint{0.100000in}{0.212622in}}{\pgfqpoint{3.696000in}{3.696000in}}%
\pgfusepath{clip}%
\pgfsetbuttcap%
\pgfsetroundjoin%
\definecolor{currentfill}{rgb}{0.121569,0.466667,0.705882}%
\pgfsetfillcolor{currentfill}%
\pgfsetfillopacity{0.300339}%
\pgfsetlinewidth{1.003750pt}%
\definecolor{currentstroke}{rgb}{0.121569,0.466667,0.705882}%
\pgfsetstrokecolor{currentstroke}%
\pgfsetstrokeopacity{0.300339}%
\pgfsetdash{}{0pt}%
\pgfpathmoveto{\pgfqpoint{1.657881in}{2.102651in}}%
\pgfpathcurveto{\pgfqpoint{1.666117in}{2.102651in}}{\pgfqpoint{1.674017in}{2.105923in}}{\pgfqpoint{1.679841in}{2.111747in}}%
\pgfpathcurveto{\pgfqpoint{1.685665in}{2.117571in}}{\pgfqpoint{1.688937in}{2.125471in}}{\pgfqpoint{1.688937in}{2.133708in}}%
\pgfpathcurveto{\pgfqpoint{1.688937in}{2.141944in}}{\pgfqpoint{1.685665in}{2.149844in}}{\pgfqpoint{1.679841in}{2.155668in}}%
\pgfpathcurveto{\pgfqpoint{1.674017in}{2.161492in}}{\pgfqpoint{1.666117in}{2.164764in}}{\pgfqpoint{1.657881in}{2.164764in}}%
\pgfpathcurveto{\pgfqpoint{1.649644in}{2.164764in}}{\pgfqpoint{1.641744in}{2.161492in}}{\pgfqpoint{1.635920in}{2.155668in}}%
\pgfpathcurveto{\pgfqpoint{1.630097in}{2.149844in}}{\pgfqpoint{1.626824in}{2.141944in}}{\pgfqpoint{1.626824in}{2.133708in}}%
\pgfpathcurveto{\pgfqpoint{1.626824in}{2.125471in}}{\pgfqpoint{1.630097in}{2.117571in}}{\pgfqpoint{1.635920in}{2.111747in}}%
\pgfpathcurveto{\pgfqpoint{1.641744in}{2.105923in}}{\pgfqpoint{1.649644in}{2.102651in}}{\pgfqpoint{1.657881in}{2.102651in}}%
\pgfpathclose%
\pgfusepath{stroke,fill}%
\end{pgfscope}%
\begin{pgfscope}%
\pgfpathrectangle{\pgfqpoint{0.100000in}{0.212622in}}{\pgfqpoint{3.696000in}{3.696000in}}%
\pgfusepath{clip}%
\pgfsetbuttcap%
\pgfsetroundjoin%
\definecolor{currentfill}{rgb}{0.121569,0.466667,0.705882}%
\pgfsetfillcolor{currentfill}%
\pgfsetfillopacity{0.300476}%
\pgfsetlinewidth{1.003750pt}%
\definecolor{currentstroke}{rgb}{0.121569,0.466667,0.705882}%
\pgfsetstrokecolor{currentstroke}%
\pgfsetstrokeopacity{0.300476}%
\pgfsetdash{}{0pt}%
\pgfpathmoveto{\pgfqpoint{1.664589in}{2.103885in}}%
\pgfpathcurveto{\pgfqpoint{1.672825in}{2.103885in}}{\pgfqpoint{1.680725in}{2.107157in}}{\pgfqpoint{1.686549in}{2.112981in}}%
\pgfpathcurveto{\pgfqpoint{1.692373in}{2.118805in}}{\pgfqpoint{1.695645in}{2.126705in}}{\pgfqpoint{1.695645in}{2.134941in}}%
\pgfpathcurveto{\pgfqpoint{1.695645in}{2.143178in}}{\pgfqpoint{1.692373in}{2.151078in}}{\pgfqpoint{1.686549in}{2.156902in}}%
\pgfpathcurveto{\pgfqpoint{1.680725in}{2.162726in}}{\pgfqpoint{1.672825in}{2.165998in}}{\pgfqpoint{1.664589in}{2.165998in}}%
\pgfpathcurveto{\pgfqpoint{1.656352in}{2.165998in}}{\pgfqpoint{1.648452in}{2.162726in}}{\pgfqpoint{1.642628in}{2.156902in}}%
\pgfpathcurveto{\pgfqpoint{1.636805in}{2.151078in}}{\pgfqpoint{1.633532in}{2.143178in}}{\pgfqpoint{1.633532in}{2.134941in}}%
\pgfpathcurveto{\pgfqpoint{1.633532in}{2.126705in}}{\pgfqpoint{1.636805in}{2.118805in}}{\pgfqpoint{1.642628in}{2.112981in}}%
\pgfpathcurveto{\pgfqpoint{1.648452in}{2.107157in}}{\pgfqpoint{1.656352in}{2.103885in}}{\pgfqpoint{1.664589in}{2.103885in}}%
\pgfpathclose%
\pgfusepath{stroke,fill}%
\end{pgfscope}%
\begin{pgfscope}%
\pgfpathrectangle{\pgfqpoint{0.100000in}{0.212622in}}{\pgfqpoint{3.696000in}{3.696000in}}%
\pgfusepath{clip}%
\pgfsetbuttcap%
\pgfsetroundjoin%
\definecolor{currentfill}{rgb}{0.121569,0.466667,0.705882}%
\pgfsetfillcolor{currentfill}%
\pgfsetfillopacity{0.300499}%
\pgfsetlinewidth{1.003750pt}%
\definecolor{currentstroke}{rgb}{0.121569,0.466667,0.705882}%
\pgfsetstrokecolor{currentstroke}%
\pgfsetstrokeopacity{0.300499}%
\pgfsetdash{}{0pt}%
\pgfpathmoveto{\pgfqpoint{1.657773in}{2.102575in}}%
\pgfpathcurveto{\pgfqpoint{1.666010in}{2.102575in}}{\pgfqpoint{1.673910in}{2.105848in}}{\pgfqpoint{1.679734in}{2.111672in}}%
\pgfpathcurveto{\pgfqpoint{1.685558in}{2.117495in}}{\pgfqpoint{1.688830in}{2.125396in}}{\pgfqpoint{1.688830in}{2.133632in}}%
\pgfpathcurveto{\pgfqpoint{1.688830in}{2.141868in}}{\pgfqpoint{1.685558in}{2.149768in}}{\pgfqpoint{1.679734in}{2.155592in}}%
\pgfpathcurveto{\pgfqpoint{1.673910in}{2.161416in}}{\pgfqpoint{1.666010in}{2.164688in}}{\pgfqpoint{1.657773in}{2.164688in}}%
\pgfpathcurveto{\pgfqpoint{1.649537in}{2.164688in}}{\pgfqpoint{1.641637in}{2.161416in}}{\pgfqpoint{1.635813in}{2.155592in}}%
\pgfpathcurveto{\pgfqpoint{1.629989in}{2.149768in}}{\pgfqpoint{1.626717in}{2.141868in}}{\pgfqpoint{1.626717in}{2.133632in}}%
\pgfpathcurveto{\pgfqpoint{1.626717in}{2.125396in}}{\pgfqpoint{1.629989in}{2.117495in}}{\pgfqpoint{1.635813in}{2.111672in}}%
\pgfpathcurveto{\pgfqpoint{1.641637in}{2.105848in}}{\pgfqpoint{1.649537in}{2.102575in}}{\pgfqpoint{1.657773in}{2.102575in}}%
\pgfpathclose%
\pgfusepath{stroke,fill}%
\end{pgfscope}%
\begin{pgfscope}%
\pgfpathrectangle{\pgfqpoint{0.100000in}{0.212622in}}{\pgfqpoint{3.696000in}{3.696000in}}%
\pgfusepath{clip}%
\pgfsetbuttcap%
\pgfsetroundjoin%
\definecolor{currentfill}{rgb}{0.121569,0.466667,0.705882}%
\pgfsetfillcolor{currentfill}%
\pgfsetfillopacity{0.300509}%
\pgfsetlinewidth{1.003750pt}%
\definecolor{currentstroke}{rgb}{0.121569,0.466667,0.705882}%
\pgfsetstrokecolor{currentstroke}%
\pgfsetstrokeopacity{0.300509}%
\pgfsetdash{}{0pt}%
\pgfpathmoveto{\pgfqpoint{1.665188in}{2.103373in}}%
\pgfpathcurveto{\pgfqpoint{1.673425in}{2.103373in}}{\pgfqpoint{1.681325in}{2.106645in}}{\pgfqpoint{1.687148in}{2.112469in}}%
\pgfpathcurveto{\pgfqpoint{1.692972in}{2.118293in}}{\pgfqpoint{1.696245in}{2.126193in}}{\pgfqpoint{1.696245in}{2.134429in}}%
\pgfpathcurveto{\pgfqpoint{1.696245in}{2.142666in}}{\pgfqpoint{1.692972in}{2.150566in}}{\pgfqpoint{1.687148in}{2.156390in}}%
\pgfpathcurveto{\pgfqpoint{1.681325in}{2.162213in}}{\pgfqpoint{1.673425in}{2.165486in}}{\pgfqpoint{1.665188in}{2.165486in}}%
\pgfpathcurveto{\pgfqpoint{1.656952in}{2.165486in}}{\pgfqpoint{1.649052in}{2.162213in}}{\pgfqpoint{1.643228in}{2.156390in}}%
\pgfpathcurveto{\pgfqpoint{1.637404in}{2.150566in}}{\pgfqpoint{1.634132in}{2.142666in}}{\pgfqpoint{1.634132in}{2.134429in}}%
\pgfpathcurveto{\pgfqpoint{1.634132in}{2.126193in}}{\pgfqpoint{1.637404in}{2.118293in}}{\pgfqpoint{1.643228in}{2.112469in}}%
\pgfpathcurveto{\pgfqpoint{1.649052in}{2.106645in}}{\pgfqpoint{1.656952in}{2.103373in}}{\pgfqpoint{1.665188in}{2.103373in}}%
\pgfpathclose%
\pgfusepath{stroke,fill}%
\end{pgfscope}%
\begin{pgfscope}%
\pgfpathrectangle{\pgfqpoint{0.100000in}{0.212622in}}{\pgfqpoint{3.696000in}{3.696000in}}%
\pgfusepath{clip}%
\pgfsetbuttcap%
\pgfsetroundjoin%
\definecolor{currentfill}{rgb}{0.121569,0.466667,0.705882}%
\pgfsetfillcolor{currentfill}%
\pgfsetfillopacity{0.300585}%
\pgfsetlinewidth{1.003750pt}%
\definecolor{currentstroke}{rgb}{0.121569,0.466667,0.705882}%
\pgfsetstrokecolor{currentstroke}%
\pgfsetstrokeopacity{0.300585}%
\pgfsetdash{}{0pt}%
\pgfpathmoveto{\pgfqpoint{1.665542in}{2.103631in}}%
\pgfpathcurveto{\pgfqpoint{1.673778in}{2.103631in}}{\pgfqpoint{1.681678in}{2.106904in}}{\pgfqpoint{1.687502in}{2.112728in}}%
\pgfpathcurveto{\pgfqpoint{1.693326in}{2.118552in}}{\pgfqpoint{1.696598in}{2.126452in}}{\pgfqpoint{1.696598in}{2.134688in}}%
\pgfpathcurveto{\pgfqpoint{1.696598in}{2.142924in}}{\pgfqpoint{1.693326in}{2.150824in}}{\pgfqpoint{1.687502in}{2.156648in}}%
\pgfpathcurveto{\pgfqpoint{1.681678in}{2.162472in}}{\pgfqpoint{1.673778in}{2.165744in}}{\pgfqpoint{1.665542in}{2.165744in}}%
\pgfpathcurveto{\pgfqpoint{1.657306in}{2.165744in}}{\pgfqpoint{1.649406in}{2.162472in}}{\pgfqpoint{1.643582in}{2.156648in}}%
\pgfpathcurveto{\pgfqpoint{1.637758in}{2.150824in}}{\pgfqpoint{1.634485in}{2.142924in}}{\pgfqpoint{1.634485in}{2.134688in}}%
\pgfpathcurveto{\pgfqpoint{1.634485in}{2.126452in}}{\pgfqpoint{1.637758in}{2.118552in}}{\pgfqpoint{1.643582in}{2.112728in}}%
\pgfpathcurveto{\pgfqpoint{1.649406in}{2.106904in}}{\pgfqpoint{1.657306in}{2.103631in}}{\pgfqpoint{1.665542in}{2.103631in}}%
\pgfpathclose%
\pgfusepath{stroke,fill}%
\end{pgfscope}%
\begin{pgfscope}%
\pgfpathrectangle{\pgfqpoint{0.100000in}{0.212622in}}{\pgfqpoint{3.696000in}{3.696000in}}%
\pgfusepath{clip}%
\pgfsetbuttcap%
\pgfsetroundjoin%
\definecolor{currentfill}{rgb}{0.121569,0.466667,0.705882}%
\pgfsetfillcolor{currentfill}%
\pgfsetfillopacity{0.300594}%
\pgfsetlinewidth{1.003750pt}%
\definecolor{currentstroke}{rgb}{0.121569,0.466667,0.705882}%
\pgfsetstrokecolor{currentstroke}%
\pgfsetstrokeopacity{0.300594}%
\pgfsetdash{}{0pt}%
\pgfpathmoveto{\pgfqpoint{1.665713in}{2.103439in}}%
\pgfpathcurveto{\pgfqpoint{1.673950in}{2.103439in}}{\pgfqpoint{1.681850in}{2.106711in}}{\pgfqpoint{1.687674in}{2.112535in}}%
\pgfpathcurveto{\pgfqpoint{1.693498in}{2.118359in}}{\pgfqpoint{1.696770in}{2.126259in}}{\pgfqpoint{1.696770in}{2.134495in}}%
\pgfpathcurveto{\pgfqpoint{1.696770in}{2.142732in}}{\pgfqpoint{1.693498in}{2.150632in}}{\pgfqpoint{1.687674in}{2.156456in}}%
\pgfpathcurveto{\pgfqpoint{1.681850in}{2.162280in}}{\pgfqpoint{1.673950in}{2.165552in}}{\pgfqpoint{1.665713in}{2.165552in}}%
\pgfpathcurveto{\pgfqpoint{1.657477in}{2.165552in}}{\pgfqpoint{1.649577in}{2.162280in}}{\pgfqpoint{1.643753in}{2.156456in}}%
\pgfpathcurveto{\pgfqpoint{1.637929in}{2.150632in}}{\pgfqpoint{1.634657in}{2.142732in}}{\pgfqpoint{1.634657in}{2.134495in}}%
\pgfpathcurveto{\pgfqpoint{1.634657in}{2.126259in}}{\pgfqpoint{1.637929in}{2.118359in}}{\pgfqpoint{1.643753in}{2.112535in}}%
\pgfpathcurveto{\pgfqpoint{1.649577in}{2.106711in}}{\pgfqpoint{1.657477in}{2.103439in}}{\pgfqpoint{1.665713in}{2.103439in}}%
\pgfpathclose%
\pgfusepath{stroke,fill}%
\end{pgfscope}%
\begin{pgfscope}%
\pgfpathrectangle{\pgfqpoint{0.100000in}{0.212622in}}{\pgfqpoint{3.696000in}{3.696000in}}%
\pgfusepath{clip}%
\pgfsetbuttcap%
\pgfsetroundjoin%
\definecolor{currentfill}{rgb}{0.121569,0.466667,0.705882}%
\pgfsetfillcolor{currentfill}%
\pgfsetfillopacity{0.300626}%
\pgfsetlinewidth{1.003750pt}%
\definecolor{currentstroke}{rgb}{0.121569,0.466667,0.705882}%
\pgfsetstrokecolor{currentstroke}%
\pgfsetstrokeopacity{0.300626}%
\pgfsetdash{}{0pt}%
\pgfpathmoveto{\pgfqpoint{1.665815in}{2.103566in}}%
\pgfpathcurveto{\pgfqpoint{1.674051in}{2.103566in}}{\pgfqpoint{1.681951in}{2.106838in}}{\pgfqpoint{1.687775in}{2.112662in}}%
\pgfpathcurveto{\pgfqpoint{1.693599in}{2.118486in}}{\pgfqpoint{1.696871in}{2.126386in}}{\pgfqpoint{1.696871in}{2.134622in}}%
\pgfpathcurveto{\pgfqpoint{1.696871in}{2.142858in}}{\pgfqpoint{1.693599in}{2.150758in}}{\pgfqpoint{1.687775in}{2.156582in}}%
\pgfpathcurveto{\pgfqpoint{1.681951in}{2.162406in}}{\pgfqpoint{1.674051in}{2.165679in}}{\pgfqpoint{1.665815in}{2.165679in}}%
\pgfpathcurveto{\pgfqpoint{1.657578in}{2.165679in}}{\pgfqpoint{1.649678in}{2.162406in}}{\pgfqpoint{1.643854in}{2.156582in}}%
\pgfpathcurveto{\pgfqpoint{1.638031in}{2.150758in}}{\pgfqpoint{1.634758in}{2.142858in}}{\pgfqpoint{1.634758in}{2.134622in}}%
\pgfpathcurveto{\pgfqpoint{1.634758in}{2.126386in}}{\pgfqpoint{1.638031in}{2.118486in}}{\pgfqpoint{1.643854in}{2.112662in}}%
\pgfpathcurveto{\pgfqpoint{1.649678in}{2.106838in}}{\pgfqpoint{1.657578in}{2.103566in}}{\pgfqpoint{1.665815in}{2.103566in}}%
\pgfpathclose%
\pgfusepath{stroke,fill}%
\end{pgfscope}%
\begin{pgfscope}%
\pgfpathrectangle{\pgfqpoint{0.100000in}{0.212622in}}{\pgfqpoint{3.696000in}{3.696000in}}%
\pgfusepath{clip}%
\pgfsetbuttcap%
\pgfsetroundjoin%
\definecolor{currentfill}{rgb}{0.121569,0.466667,0.705882}%
\pgfsetfillcolor{currentfill}%
\pgfsetfillopacity{0.300634}%
\pgfsetlinewidth{1.003750pt}%
\definecolor{currentstroke}{rgb}{0.121569,0.466667,0.705882}%
\pgfsetstrokecolor{currentstroke}%
\pgfsetstrokeopacity{0.300634}%
\pgfsetdash{}{0pt}%
\pgfpathmoveto{\pgfqpoint{1.665863in}{2.103542in}}%
\pgfpathcurveto{\pgfqpoint{1.674100in}{2.103542in}}{\pgfqpoint{1.682000in}{2.106814in}}{\pgfqpoint{1.687824in}{2.112638in}}%
\pgfpathcurveto{\pgfqpoint{1.693648in}{2.118462in}}{\pgfqpoint{1.696920in}{2.126362in}}{\pgfqpoint{1.696920in}{2.134598in}}%
\pgfpathcurveto{\pgfqpoint{1.696920in}{2.142835in}}{\pgfqpoint{1.693648in}{2.150735in}}{\pgfqpoint{1.687824in}{2.156559in}}%
\pgfpathcurveto{\pgfqpoint{1.682000in}{2.162382in}}{\pgfqpoint{1.674100in}{2.165655in}}{\pgfqpoint{1.665863in}{2.165655in}}%
\pgfpathcurveto{\pgfqpoint{1.657627in}{2.165655in}}{\pgfqpoint{1.649727in}{2.162382in}}{\pgfqpoint{1.643903in}{2.156559in}}%
\pgfpathcurveto{\pgfqpoint{1.638079in}{2.150735in}}{\pgfqpoint{1.634807in}{2.142835in}}{\pgfqpoint{1.634807in}{2.134598in}}%
\pgfpathcurveto{\pgfqpoint{1.634807in}{2.126362in}}{\pgfqpoint{1.638079in}{2.118462in}}{\pgfqpoint{1.643903in}{2.112638in}}%
\pgfpathcurveto{\pgfqpoint{1.649727in}{2.106814in}}{\pgfqpoint{1.657627in}{2.103542in}}{\pgfqpoint{1.665863in}{2.103542in}}%
\pgfpathclose%
\pgfusepath{stroke,fill}%
\end{pgfscope}%
\begin{pgfscope}%
\pgfpathrectangle{\pgfqpoint{0.100000in}{0.212622in}}{\pgfqpoint{3.696000in}{3.696000in}}%
\pgfusepath{clip}%
\pgfsetbuttcap%
\pgfsetroundjoin%
\definecolor{currentfill}{rgb}{0.121569,0.466667,0.705882}%
\pgfsetfillcolor{currentfill}%
\pgfsetfillopacity{0.300694}%
\pgfsetlinewidth{1.003750pt}%
\definecolor{currentstroke}{rgb}{0.121569,0.466667,0.705882}%
\pgfsetstrokecolor{currentstroke}%
\pgfsetstrokeopacity{0.300694}%
\pgfsetdash{}{0pt}%
\pgfpathmoveto{\pgfqpoint{1.657147in}{2.102014in}}%
\pgfpathcurveto{\pgfqpoint{1.665384in}{2.102014in}}{\pgfqpoint{1.673284in}{2.105286in}}{\pgfqpoint{1.679108in}{2.111110in}}%
\pgfpathcurveto{\pgfqpoint{1.684932in}{2.116934in}}{\pgfqpoint{1.688204in}{2.124834in}}{\pgfqpoint{1.688204in}{2.133070in}}%
\pgfpathcurveto{\pgfqpoint{1.688204in}{2.141307in}}{\pgfqpoint{1.684932in}{2.149207in}}{\pgfqpoint{1.679108in}{2.155031in}}%
\pgfpathcurveto{\pgfqpoint{1.673284in}{2.160854in}}{\pgfqpoint{1.665384in}{2.164127in}}{\pgfqpoint{1.657147in}{2.164127in}}%
\pgfpathcurveto{\pgfqpoint{1.648911in}{2.164127in}}{\pgfqpoint{1.641011in}{2.160854in}}{\pgfqpoint{1.635187in}{2.155031in}}%
\pgfpathcurveto{\pgfqpoint{1.629363in}{2.149207in}}{\pgfqpoint{1.626091in}{2.141307in}}{\pgfqpoint{1.626091in}{2.133070in}}%
\pgfpathcurveto{\pgfqpoint{1.626091in}{2.124834in}}{\pgfqpoint{1.629363in}{2.116934in}}{\pgfqpoint{1.635187in}{2.111110in}}%
\pgfpathcurveto{\pgfqpoint{1.641011in}{2.105286in}}{\pgfqpoint{1.648911in}{2.102014in}}{\pgfqpoint{1.657147in}{2.102014in}}%
\pgfpathclose%
\pgfusepath{stroke,fill}%
\end{pgfscope}%
\begin{pgfscope}%
\pgfpathrectangle{\pgfqpoint{0.100000in}{0.212622in}}{\pgfqpoint{3.696000in}{3.696000in}}%
\pgfusepath{clip}%
\pgfsetbuttcap%
\pgfsetroundjoin%
\definecolor{currentfill}{rgb}{0.121569,0.466667,0.705882}%
\pgfsetfillcolor{currentfill}%
\pgfsetfillopacity{0.300886}%
\pgfsetlinewidth{1.003750pt}%
\definecolor{currentstroke}{rgb}{0.121569,0.466667,0.705882}%
\pgfsetstrokecolor{currentstroke}%
\pgfsetstrokeopacity{0.300886}%
\pgfsetdash{}{0pt}%
\pgfpathmoveto{\pgfqpoint{1.666813in}{2.104291in}}%
\pgfpathcurveto{\pgfqpoint{1.675049in}{2.104291in}}{\pgfqpoint{1.682949in}{2.107563in}}{\pgfqpoint{1.688773in}{2.113387in}}%
\pgfpathcurveto{\pgfqpoint{1.694597in}{2.119211in}}{\pgfqpoint{1.697869in}{2.127111in}}{\pgfqpoint{1.697869in}{2.135347in}}%
\pgfpathcurveto{\pgfqpoint{1.697869in}{2.143583in}}{\pgfqpoint{1.694597in}{2.151483in}}{\pgfqpoint{1.688773in}{2.157307in}}%
\pgfpathcurveto{\pgfqpoint{1.682949in}{2.163131in}}{\pgfqpoint{1.675049in}{2.166404in}}{\pgfqpoint{1.666813in}{2.166404in}}%
\pgfpathcurveto{\pgfqpoint{1.658577in}{2.166404in}}{\pgfqpoint{1.650677in}{2.163131in}}{\pgfqpoint{1.644853in}{2.157307in}}%
\pgfpathcurveto{\pgfqpoint{1.639029in}{2.151483in}}{\pgfqpoint{1.635756in}{2.143583in}}{\pgfqpoint{1.635756in}{2.135347in}}%
\pgfpathcurveto{\pgfqpoint{1.635756in}{2.127111in}}{\pgfqpoint{1.639029in}{2.119211in}}{\pgfqpoint{1.644853in}{2.113387in}}%
\pgfpathcurveto{\pgfqpoint{1.650677in}{2.107563in}}{\pgfqpoint{1.658577in}{2.104291in}}{\pgfqpoint{1.666813in}{2.104291in}}%
\pgfpathclose%
\pgfusepath{stroke,fill}%
\end{pgfscope}%
\begin{pgfscope}%
\pgfpathrectangle{\pgfqpoint{0.100000in}{0.212622in}}{\pgfqpoint{3.696000in}{3.696000in}}%
\pgfusepath{clip}%
\pgfsetbuttcap%
\pgfsetroundjoin%
\definecolor{currentfill}{rgb}{0.121569,0.466667,0.705882}%
\pgfsetfillcolor{currentfill}%
\pgfsetfillopacity{0.300949}%
\pgfsetlinewidth{1.003750pt}%
\definecolor{currentstroke}{rgb}{0.121569,0.466667,0.705882}%
\pgfsetstrokecolor{currentstroke}%
\pgfsetstrokeopacity{0.300949}%
\pgfsetdash{}{0pt}%
\pgfpathmoveto{\pgfqpoint{1.667328in}{2.104085in}}%
\pgfpathcurveto{\pgfqpoint{1.675565in}{2.104085in}}{\pgfqpoint{1.683465in}{2.107357in}}{\pgfqpoint{1.689289in}{2.113181in}}%
\pgfpathcurveto{\pgfqpoint{1.695113in}{2.119005in}}{\pgfqpoint{1.698385in}{2.126905in}}{\pgfqpoint{1.698385in}{2.135141in}}%
\pgfpathcurveto{\pgfqpoint{1.698385in}{2.143377in}}{\pgfqpoint{1.695113in}{2.151277in}}{\pgfqpoint{1.689289in}{2.157101in}}%
\pgfpathcurveto{\pgfqpoint{1.683465in}{2.162925in}}{\pgfqpoint{1.675565in}{2.166198in}}{\pgfqpoint{1.667328in}{2.166198in}}%
\pgfpathcurveto{\pgfqpoint{1.659092in}{2.166198in}}{\pgfqpoint{1.651192in}{2.162925in}}{\pgfqpoint{1.645368in}{2.157101in}}%
\pgfpathcurveto{\pgfqpoint{1.639544in}{2.151277in}}{\pgfqpoint{1.636272in}{2.143377in}}{\pgfqpoint{1.636272in}{2.135141in}}%
\pgfpathcurveto{\pgfqpoint{1.636272in}{2.126905in}}{\pgfqpoint{1.639544in}{2.119005in}}{\pgfqpoint{1.645368in}{2.113181in}}%
\pgfpathcurveto{\pgfqpoint{1.651192in}{2.107357in}}{\pgfqpoint{1.659092in}{2.104085in}}{\pgfqpoint{1.667328in}{2.104085in}}%
\pgfpathclose%
\pgfusepath{stroke,fill}%
\end{pgfscope}%
\begin{pgfscope}%
\pgfpathrectangle{\pgfqpoint{0.100000in}{0.212622in}}{\pgfqpoint{3.696000in}{3.696000in}}%
\pgfusepath{clip}%
\pgfsetbuttcap%
\pgfsetroundjoin%
\definecolor{currentfill}{rgb}{0.121569,0.466667,0.705882}%
\pgfsetfillcolor{currentfill}%
\pgfsetfillopacity{0.301245}%
\pgfsetlinewidth{1.003750pt}%
\definecolor{currentstroke}{rgb}{0.121569,0.466667,0.705882}%
\pgfsetstrokecolor{currentstroke}%
\pgfsetstrokeopacity{0.301245}%
\pgfsetdash{}{0pt}%
\pgfpathmoveto{\pgfqpoint{1.656784in}{2.101926in}}%
\pgfpathcurveto{\pgfqpoint{1.665020in}{2.101926in}}{\pgfqpoint{1.672920in}{2.105198in}}{\pgfqpoint{1.678744in}{2.111022in}}%
\pgfpathcurveto{\pgfqpoint{1.684568in}{2.116846in}}{\pgfqpoint{1.687841in}{2.124746in}}{\pgfqpoint{1.687841in}{2.132982in}}%
\pgfpathcurveto{\pgfqpoint{1.687841in}{2.141219in}}{\pgfqpoint{1.684568in}{2.149119in}}{\pgfqpoint{1.678744in}{2.154943in}}%
\pgfpathcurveto{\pgfqpoint{1.672920in}{2.160767in}}{\pgfqpoint{1.665020in}{2.164039in}}{\pgfqpoint{1.656784in}{2.164039in}}%
\pgfpathcurveto{\pgfqpoint{1.648548in}{2.164039in}}{\pgfqpoint{1.640648in}{2.160767in}}{\pgfqpoint{1.634824in}{2.154943in}}%
\pgfpathcurveto{\pgfqpoint{1.629000in}{2.149119in}}{\pgfqpoint{1.625728in}{2.141219in}}{\pgfqpoint{1.625728in}{2.132982in}}%
\pgfpathcurveto{\pgfqpoint{1.625728in}{2.124746in}}{\pgfqpoint{1.629000in}{2.116846in}}{\pgfqpoint{1.634824in}{2.111022in}}%
\pgfpathcurveto{\pgfqpoint{1.640648in}{2.105198in}}{\pgfqpoint{1.648548in}{2.101926in}}{\pgfqpoint{1.656784in}{2.101926in}}%
\pgfpathclose%
\pgfusepath{stroke,fill}%
\end{pgfscope}%
\begin{pgfscope}%
\pgfpathrectangle{\pgfqpoint{0.100000in}{0.212622in}}{\pgfqpoint{3.696000in}{3.696000in}}%
\pgfusepath{clip}%
\pgfsetbuttcap%
\pgfsetroundjoin%
\definecolor{currentfill}{rgb}{0.121569,0.466667,0.705882}%
\pgfsetfillcolor{currentfill}%
\pgfsetfillopacity{0.301358}%
\pgfsetlinewidth{1.003750pt}%
\definecolor{currentstroke}{rgb}{0.121569,0.466667,0.705882}%
\pgfsetstrokecolor{currentstroke}%
\pgfsetstrokeopacity{0.301358}%
\pgfsetdash{}{0pt}%
\pgfpathmoveto{\pgfqpoint{1.669053in}{2.105608in}}%
\pgfpathcurveto{\pgfqpoint{1.677289in}{2.105608in}}{\pgfqpoint{1.685190in}{2.108880in}}{\pgfqpoint{1.691013in}{2.114704in}}%
\pgfpathcurveto{\pgfqpoint{1.696837in}{2.120528in}}{\pgfqpoint{1.700110in}{2.128428in}}{\pgfqpoint{1.700110in}{2.136665in}}%
\pgfpathcurveto{\pgfqpoint{1.700110in}{2.144901in}}{\pgfqpoint{1.696837in}{2.152801in}}{\pgfqpoint{1.691013in}{2.158625in}}%
\pgfpathcurveto{\pgfqpoint{1.685190in}{2.164449in}}{\pgfqpoint{1.677289in}{2.167721in}}{\pgfqpoint{1.669053in}{2.167721in}}%
\pgfpathcurveto{\pgfqpoint{1.660817in}{2.167721in}}{\pgfqpoint{1.652917in}{2.164449in}}{\pgfqpoint{1.647093in}{2.158625in}}%
\pgfpathcurveto{\pgfqpoint{1.641269in}{2.152801in}}{\pgfqpoint{1.637997in}{2.144901in}}{\pgfqpoint{1.637997in}{2.136665in}}%
\pgfpathcurveto{\pgfqpoint{1.637997in}{2.128428in}}{\pgfqpoint{1.641269in}{2.120528in}}{\pgfqpoint{1.647093in}{2.114704in}}%
\pgfpathcurveto{\pgfqpoint{1.652917in}{2.108880in}}{\pgfqpoint{1.660817in}{2.105608in}}{\pgfqpoint{1.669053in}{2.105608in}}%
\pgfpathclose%
\pgfusepath{stroke,fill}%
\end{pgfscope}%
\begin{pgfscope}%
\pgfpathrectangle{\pgfqpoint{0.100000in}{0.212622in}}{\pgfqpoint{3.696000in}{3.696000in}}%
\pgfusepath{clip}%
\pgfsetbuttcap%
\pgfsetroundjoin%
\definecolor{currentfill}{rgb}{0.121569,0.466667,0.705882}%
\pgfsetfillcolor{currentfill}%
\pgfsetfillopacity{0.301491}%
\pgfsetlinewidth{1.003750pt}%
\definecolor{currentstroke}{rgb}{0.121569,0.466667,0.705882}%
\pgfsetstrokecolor{currentstroke}%
\pgfsetstrokeopacity{0.301491}%
\pgfsetdash{}{0pt}%
\pgfpathmoveto{\pgfqpoint{1.671211in}{2.104166in}}%
\pgfpathcurveto{\pgfqpoint{1.679447in}{2.104166in}}{\pgfqpoint{1.687348in}{2.107438in}}{\pgfqpoint{1.693171in}{2.113262in}}%
\pgfpathcurveto{\pgfqpoint{1.698995in}{2.119086in}}{\pgfqpoint{1.702268in}{2.126986in}}{\pgfqpoint{1.702268in}{2.135222in}}%
\pgfpathcurveto{\pgfqpoint{1.702268in}{2.143458in}}{\pgfqpoint{1.698995in}{2.151358in}}{\pgfqpoint{1.693171in}{2.157182in}}%
\pgfpathcurveto{\pgfqpoint{1.687348in}{2.163006in}}{\pgfqpoint{1.679447in}{2.166279in}}{\pgfqpoint{1.671211in}{2.166279in}}%
\pgfpathcurveto{\pgfqpoint{1.662975in}{2.166279in}}{\pgfqpoint{1.655075in}{2.163006in}}{\pgfqpoint{1.649251in}{2.157182in}}%
\pgfpathcurveto{\pgfqpoint{1.643427in}{2.151358in}}{\pgfqpoint{1.640155in}{2.143458in}}{\pgfqpoint{1.640155in}{2.135222in}}%
\pgfpathcurveto{\pgfqpoint{1.640155in}{2.126986in}}{\pgfqpoint{1.643427in}{2.119086in}}{\pgfqpoint{1.649251in}{2.113262in}}%
\pgfpathcurveto{\pgfqpoint{1.655075in}{2.107438in}}{\pgfqpoint{1.662975in}{2.104166in}}{\pgfqpoint{1.671211in}{2.104166in}}%
\pgfpathclose%
\pgfusepath{stroke,fill}%
\end{pgfscope}%
\begin{pgfscope}%
\pgfpathrectangle{\pgfqpoint{0.100000in}{0.212622in}}{\pgfqpoint{3.696000in}{3.696000in}}%
\pgfusepath{clip}%
\pgfsetbuttcap%
\pgfsetroundjoin%
\definecolor{currentfill}{rgb}{0.121569,0.466667,0.705882}%
\pgfsetfillcolor{currentfill}%
\pgfsetfillopacity{0.301499}%
\pgfsetlinewidth{1.003750pt}%
\definecolor{currentstroke}{rgb}{0.121569,0.466667,0.705882}%
\pgfsetstrokecolor{currentstroke}%
\pgfsetstrokeopacity{0.301499}%
\pgfsetdash{}{0pt}%
\pgfpathmoveto{\pgfqpoint{1.655843in}{2.100799in}}%
\pgfpathcurveto{\pgfqpoint{1.664079in}{2.100799in}}{\pgfqpoint{1.671979in}{2.104071in}}{\pgfqpoint{1.677803in}{2.109895in}}%
\pgfpathcurveto{\pgfqpoint{1.683627in}{2.115719in}}{\pgfqpoint{1.686899in}{2.123619in}}{\pgfqpoint{1.686899in}{2.131855in}}%
\pgfpathcurveto{\pgfqpoint{1.686899in}{2.140092in}}{\pgfqpoint{1.683627in}{2.147992in}}{\pgfqpoint{1.677803in}{2.153816in}}%
\pgfpathcurveto{\pgfqpoint{1.671979in}{2.159639in}}{\pgfqpoint{1.664079in}{2.162912in}}{\pgfqpoint{1.655843in}{2.162912in}}%
\pgfpathcurveto{\pgfqpoint{1.647607in}{2.162912in}}{\pgfqpoint{1.639707in}{2.159639in}}{\pgfqpoint{1.633883in}{2.153816in}}%
\pgfpathcurveto{\pgfqpoint{1.628059in}{2.147992in}}{\pgfqpoint{1.624786in}{2.140092in}}{\pgfqpoint{1.624786in}{2.131855in}}%
\pgfpathcurveto{\pgfqpoint{1.624786in}{2.123619in}}{\pgfqpoint{1.628059in}{2.115719in}}{\pgfqpoint{1.633883in}{2.109895in}}%
\pgfpathcurveto{\pgfqpoint{1.639707in}{2.104071in}}{\pgfqpoint{1.647607in}{2.100799in}}{\pgfqpoint{1.655843in}{2.100799in}}%
\pgfpathclose%
\pgfusepath{stroke,fill}%
\end{pgfscope}%
\begin{pgfscope}%
\pgfpathrectangle{\pgfqpoint{0.100000in}{0.212622in}}{\pgfqpoint{3.696000in}{3.696000in}}%
\pgfusepath{clip}%
\pgfsetbuttcap%
\pgfsetroundjoin%
\definecolor{currentfill}{rgb}{0.121569,0.466667,0.705882}%
\pgfsetfillcolor{currentfill}%
\pgfsetfillopacity{0.301812}%
\pgfsetlinewidth{1.003750pt}%
\definecolor{currentstroke}{rgb}{0.121569,0.466667,0.705882}%
\pgfsetstrokecolor{currentstroke}%
\pgfsetstrokeopacity{0.301812}%
\pgfsetdash{}{0pt}%
\pgfpathmoveto{\pgfqpoint{1.655562in}{2.100749in}}%
\pgfpathcurveto{\pgfqpoint{1.663798in}{2.100749in}}{\pgfqpoint{1.671698in}{2.104021in}}{\pgfqpoint{1.677522in}{2.109845in}}%
\pgfpathcurveto{\pgfqpoint{1.683346in}{2.115669in}}{\pgfqpoint{1.686618in}{2.123569in}}{\pgfqpoint{1.686618in}{2.131805in}}%
\pgfpathcurveto{\pgfqpoint{1.686618in}{2.140042in}}{\pgfqpoint{1.683346in}{2.147942in}}{\pgfqpoint{1.677522in}{2.153766in}}%
\pgfpathcurveto{\pgfqpoint{1.671698in}{2.159590in}}{\pgfqpoint{1.663798in}{2.162862in}}{\pgfqpoint{1.655562in}{2.162862in}}%
\pgfpathcurveto{\pgfqpoint{1.647325in}{2.162862in}}{\pgfqpoint{1.639425in}{2.159590in}}{\pgfqpoint{1.633601in}{2.153766in}}%
\pgfpathcurveto{\pgfqpoint{1.627777in}{2.147942in}}{\pgfqpoint{1.624505in}{2.140042in}}{\pgfqpoint{1.624505in}{2.131805in}}%
\pgfpathcurveto{\pgfqpoint{1.624505in}{2.123569in}}{\pgfqpoint{1.627777in}{2.115669in}}{\pgfqpoint{1.633601in}{2.109845in}}%
\pgfpathcurveto{\pgfqpoint{1.639425in}{2.104021in}}{\pgfqpoint{1.647325in}{2.100749in}}{\pgfqpoint{1.655562in}{2.100749in}}%
\pgfpathclose%
\pgfusepath{stroke,fill}%
\end{pgfscope}%
\begin{pgfscope}%
\pgfpathrectangle{\pgfqpoint{0.100000in}{0.212622in}}{\pgfqpoint{3.696000in}{3.696000in}}%
\pgfusepath{clip}%
\pgfsetbuttcap%
\pgfsetroundjoin%
\definecolor{currentfill}{rgb}{0.121569,0.466667,0.705882}%
\pgfsetfillcolor{currentfill}%
\pgfsetfillopacity{0.302037}%
\pgfsetlinewidth{1.003750pt}%
\definecolor{currentstroke}{rgb}{0.121569,0.466667,0.705882}%
\pgfsetstrokecolor{currentstroke}%
\pgfsetstrokeopacity{0.302037}%
\pgfsetdash{}{0pt}%
\pgfpathmoveto{\pgfqpoint{1.674154in}{2.106515in}}%
\pgfpathcurveto{\pgfqpoint{1.682390in}{2.106515in}}{\pgfqpoint{1.690290in}{2.109787in}}{\pgfqpoint{1.696114in}{2.115611in}}%
\pgfpathcurveto{\pgfqpoint{1.701938in}{2.121435in}}{\pgfqpoint{1.705210in}{2.129335in}}{\pgfqpoint{1.705210in}{2.137572in}}%
\pgfpathcurveto{\pgfqpoint{1.705210in}{2.145808in}}{\pgfqpoint{1.701938in}{2.153708in}}{\pgfqpoint{1.696114in}{2.159532in}}%
\pgfpathcurveto{\pgfqpoint{1.690290in}{2.165356in}}{\pgfqpoint{1.682390in}{2.168628in}}{\pgfqpoint{1.674154in}{2.168628in}}%
\pgfpathcurveto{\pgfqpoint{1.665917in}{2.168628in}}{\pgfqpoint{1.658017in}{2.165356in}}{\pgfqpoint{1.652193in}{2.159532in}}%
\pgfpathcurveto{\pgfqpoint{1.646370in}{2.153708in}}{\pgfqpoint{1.643097in}{2.145808in}}{\pgfqpoint{1.643097in}{2.137572in}}%
\pgfpathcurveto{\pgfqpoint{1.643097in}{2.129335in}}{\pgfqpoint{1.646370in}{2.121435in}}{\pgfqpoint{1.652193in}{2.115611in}}%
\pgfpathcurveto{\pgfqpoint{1.658017in}{2.109787in}}{\pgfqpoint{1.665917in}{2.106515in}}{\pgfqpoint{1.674154in}{2.106515in}}%
\pgfpathclose%
\pgfusepath{stroke,fill}%
\end{pgfscope}%
\begin{pgfscope}%
\pgfpathrectangle{\pgfqpoint{0.100000in}{0.212622in}}{\pgfqpoint{3.696000in}{3.696000in}}%
\pgfusepath{clip}%
\pgfsetbuttcap%
\pgfsetroundjoin%
\definecolor{currentfill}{rgb}{0.121569,0.466667,0.705882}%
\pgfsetfillcolor{currentfill}%
\pgfsetfillopacity{0.302107}%
\pgfsetlinewidth{1.003750pt}%
\definecolor{currentstroke}{rgb}{0.121569,0.466667,0.705882}%
\pgfsetstrokecolor{currentstroke}%
\pgfsetstrokeopacity{0.302107}%
\pgfsetdash{}{0pt}%
\pgfpathmoveto{\pgfqpoint{1.654254in}{2.099276in}}%
\pgfpathcurveto{\pgfqpoint{1.662490in}{2.099276in}}{\pgfqpoint{1.670390in}{2.102548in}}{\pgfqpoint{1.676214in}{2.108372in}}%
\pgfpathcurveto{\pgfqpoint{1.682038in}{2.114196in}}{\pgfqpoint{1.685310in}{2.122096in}}{\pgfqpoint{1.685310in}{2.130332in}}%
\pgfpathcurveto{\pgfqpoint{1.685310in}{2.138569in}}{\pgfqpoint{1.682038in}{2.146469in}}{\pgfqpoint{1.676214in}{2.152293in}}%
\pgfpathcurveto{\pgfqpoint{1.670390in}{2.158117in}}{\pgfqpoint{1.662490in}{2.161389in}}{\pgfqpoint{1.654254in}{2.161389in}}%
\pgfpathcurveto{\pgfqpoint{1.646017in}{2.161389in}}{\pgfqpoint{1.638117in}{2.158117in}}{\pgfqpoint{1.632293in}{2.152293in}}%
\pgfpathcurveto{\pgfqpoint{1.626469in}{2.146469in}}{\pgfqpoint{1.623197in}{2.138569in}}{\pgfqpoint{1.623197in}{2.130332in}}%
\pgfpathcurveto{\pgfqpoint{1.623197in}{2.122096in}}{\pgfqpoint{1.626469in}{2.114196in}}{\pgfqpoint{1.632293in}{2.108372in}}%
\pgfpathcurveto{\pgfqpoint{1.638117in}{2.102548in}}{\pgfqpoint{1.646017in}{2.099276in}}{\pgfqpoint{1.654254in}{2.099276in}}%
\pgfpathclose%
\pgfusepath{stroke,fill}%
\end{pgfscope}%
\begin{pgfscope}%
\pgfpathrectangle{\pgfqpoint{0.100000in}{0.212622in}}{\pgfqpoint{3.696000in}{3.696000in}}%
\pgfusepath{clip}%
\pgfsetbuttcap%
\pgfsetroundjoin%
\definecolor{currentfill}{rgb}{0.121569,0.466667,0.705882}%
\pgfsetfillcolor{currentfill}%
\pgfsetfillopacity{0.302115}%
\pgfsetlinewidth{1.003750pt}%
\definecolor{currentstroke}{rgb}{0.121569,0.466667,0.705882}%
\pgfsetstrokecolor{currentstroke}%
\pgfsetstrokeopacity{0.302115}%
\pgfsetdash{}{0pt}%
\pgfpathmoveto{\pgfqpoint{1.675703in}{2.105673in}}%
\pgfpathcurveto{\pgfqpoint{1.683939in}{2.105673in}}{\pgfqpoint{1.691840in}{2.108945in}}{\pgfqpoint{1.697663in}{2.114769in}}%
\pgfpathcurveto{\pgfqpoint{1.703487in}{2.120593in}}{\pgfqpoint{1.706760in}{2.128493in}}{\pgfqpoint{1.706760in}{2.136729in}}%
\pgfpathcurveto{\pgfqpoint{1.706760in}{2.144966in}}{\pgfqpoint{1.703487in}{2.152866in}}{\pgfqpoint{1.697663in}{2.158690in}}%
\pgfpathcurveto{\pgfqpoint{1.691840in}{2.164513in}}{\pgfqpoint{1.683939in}{2.167786in}}{\pgfqpoint{1.675703in}{2.167786in}}%
\pgfpathcurveto{\pgfqpoint{1.667467in}{2.167786in}}{\pgfqpoint{1.659567in}{2.164513in}}{\pgfqpoint{1.653743in}{2.158690in}}%
\pgfpathcurveto{\pgfqpoint{1.647919in}{2.152866in}}{\pgfqpoint{1.644647in}{2.144966in}}{\pgfqpoint{1.644647in}{2.136729in}}%
\pgfpathcurveto{\pgfqpoint{1.644647in}{2.128493in}}{\pgfqpoint{1.647919in}{2.120593in}}{\pgfqpoint{1.653743in}{2.114769in}}%
\pgfpathcurveto{\pgfqpoint{1.659567in}{2.108945in}}{\pgfqpoint{1.667467in}{2.105673in}}{\pgfqpoint{1.675703in}{2.105673in}}%
\pgfpathclose%
\pgfusepath{stroke,fill}%
\end{pgfscope}%
\begin{pgfscope}%
\pgfpathrectangle{\pgfqpoint{0.100000in}{0.212622in}}{\pgfqpoint{3.696000in}{3.696000in}}%
\pgfusepath{clip}%
\pgfsetbuttcap%
\pgfsetroundjoin%
\definecolor{currentfill}{rgb}{0.121569,0.466667,0.705882}%
\pgfsetfillcolor{currentfill}%
\pgfsetfillopacity{0.302382}%
\pgfsetlinewidth{1.003750pt}%
\definecolor{currentstroke}{rgb}{0.121569,0.466667,0.705882}%
\pgfsetstrokecolor{currentstroke}%
\pgfsetstrokeopacity{0.302382}%
\pgfsetdash{}{0pt}%
\pgfpathmoveto{\pgfqpoint{1.653158in}{2.098143in}}%
\pgfpathcurveto{\pgfqpoint{1.661394in}{2.098143in}}{\pgfqpoint{1.669294in}{2.101415in}}{\pgfqpoint{1.675118in}{2.107239in}}%
\pgfpathcurveto{\pgfqpoint{1.680942in}{2.113063in}}{\pgfqpoint{1.684215in}{2.120963in}}{\pgfqpoint{1.684215in}{2.129199in}}%
\pgfpathcurveto{\pgfqpoint{1.684215in}{2.137436in}}{\pgfqpoint{1.680942in}{2.145336in}}{\pgfqpoint{1.675118in}{2.151160in}}%
\pgfpathcurveto{\pgfqpoint{1.669294in}{2.156984in}}{\pgfqpoint{1.661394in}{2.160256in}}{\pgfqpoint{1.653158in}{2.160256in}}%
\pgfpathcurveto{\pgfqpoint{1.644922in}{2.160256in}}{\pgfqpoint{1.637022in}{2.156984in}}{\pgfqpoint{1.631198in}{2.151160in}}%
\pgfpathcurveto{\pgfqpoint{1.625374in}{2.145336in}}{\pgfqpoint{1.622102in}{2.137436in}}{\pgfqpoint{1.622102in}{2.129199in}}%
\pgfpathcurveto{\pgfqpoint{1.622102in}{2.120963in}}{\pgfqpoint{1.625374in}{2.113063in}}{\pgfqpoint{1.631198in}{2.107239in}}%
\pgfpathcurveto{\pgfqpoint{1.637022in}{2.101415in}}{\pgfqpoint{1.644922in}{2.098143in}}{\pgfqpoint{1.653158in}{2.098143in}}%
\pgfpathclose%
\pgfusepath{stroke,fill}%
\end{pgfscope}%
\begin{pgfscope}%
\pgfpathrectangle{\pgfqpoint{0.100000in}{0.212622in}}{\pgfqpoint{3.696000in}{3.696000in}}%
\pgfusepath{clip}%
\pgfsetbuttcap%
\pgfsetroundjoin%
\definecolor{currentfill}{rgb}{0.121569,0.466667,0.705882}%
\pgfsetfillcolor{currentfill}%
\pgfsetfillopacity{0.302588}%
\pgfsetlinewidth{1.003750pt}%
\definecolor{currentstroke}{rgb}{0.121569,0.466667,0.705882}%
\pgfsetstrokecolor{currentstroke}%
\pgfsetstrokeopacity{0.302588}%
\pgfsetdash{}{0pt}%
\pgfpathmoveto{\pgfqpoint{1.677979in}{2.107363in}}%
\pgfpathcurveto{\pgfqpoint{1.686215in}{2.107363in}}{\pgfqpoint{1.694115in}{2.110636in}}{\pgfqpoint{1.699939in}{2.116459in}}%
\pgfpathcurveto{\pgfqpoint{1.705763in}{2.122283in}}{\pgfqpoint{1.709035in}{2.130183in}}{\pgfqpoint{1.709035in}{2.138420in}}%
\pgfpathcurveto{\pgfqpoint{1.709035in}{2.146656in}}{\pgfqpoint{1.705763in}{2.154556in}}{\pgfqpoint{1.699939in}{2.160380in}}%
\pgfpathcurveto{\pgfqpoint{1.694115in}{2.166204in}}{\pgfqpoint{1.686215in}{2.169476in}}{\pgfqpoint{1.677979in}{2.169476in}}%
\pgfpathcurveto{\pgfqpoint{1.669743in}{2.169476in}}{\pgfqpoint{1.661843in}{2.166204in}}{\pgfqpoint{1.656019in}{2.160380in}}%
\pgfpathcurveto{\pgfqpoint{1.650195in}{2.154556in}}{\pgfqpoint{1.646922in}{2.146656in}}{\pgfqpoint{1.646922in}{2.138420in}}%
\pgfpathcurveto{\pgfqpoint{1.646922in}{2.130183in}}{\pgfqpoint{1.650195in}{2.122283in}}{\pgfqpoint{1.656019in}{2.116459in}}%
\pgfpathcurveto{\pgfqpoint{1.661843in}{2.110636in}}{\pgfqpoint{1.669743in}{2.107363in}}{\pgfqpoint{1.677979in}{2.107363in}}%
\pgfpathclose%
\pgfusepath{stroke,fill}%
\end{pgfscope}%
\begin{pgfscope}%
\pgfpathrectangle{\pgfqpoint{0.100000in}{0.212622in}}{\pgfqpoint{3.696000in}{3.696000in}}%
\pgfusepath{clip}%
\pgfsetbuttcap%
\pgfsetroundjoin%
\definecolor{currentfill}{rgb}{0.121569,0.466667,0.705882}%
\pgfsetfillcolor{currentfill}%
\pgfsetfillopacity{0.302739}%
\pgfsetlinewidth{1.003750pt}%
\definecolor{currentstroke}{rgb}{0.121569,0.466667,0.705882}%
\pgfsetstrokecolor{currentstroke}%
\pgfsetstrokeopacity{0.302739}%
\pgfsetdash{}{0pt}%
\pgfpathmoveto{\pgfqpoint{1.679123in}{2.107020in}}%
\pgfpathcurveto{\pgfqpoint{1.687359in}{2.107020in}}{\pgfqpoint{1.695259in}{2.110293in}}{\pgfqpoint{1.701083in}{2.116116in}}%
\pgfpathcurveto{\pgfqpoint{1.706907in}{2.121940in}}{\pgfqpoint{1.710179in}{2.129840in}}{\pgfqpoint{1.710179in}{2.138077in}}%
\pgfpathcurveto{\pgfqpoint{1.710179in}{2.146313in}}{\pgfqpoint{1.706907in}{2.154213in}}{\pgfqpoint{1.701083in}{2.160037in}}%
\pgfpathcurveto{\pgfqpoint{1.695259in}{2.165861in}}{\pgfqpoint{1.687359in}{2.169133in}}{\pgfqpoint{1.679123in}{2.169133in}}%
\pgfpathcurveto{\pgfqpoint{1.670886in}{2.169133in}}{\pgfqpoint{1.662986in}{2.165861in}}{\pgfqpoint{1.657162in}{2.160037in}}%
\pgfpathcurveto{\pgfqpoint{1.651338in}{2.154213in}}{\pgfqpoint{1.648066in}{2.146313in}}{\pgfqpoint{1.648066in}{2.138077in}}%
\pgfpathcurveto{\pgfqpoint{1.648066in}{2.129840in}}{\pgfqpoint{1.651338in}{2.121940in}}{\pgfqpoint{1.657162in}{2.116116in}}%
\pgfpathcurveto{\pgfqpoint{1.662986in}{2.110293in}}{\pgfqpoint{1.670886in}{2.107020in}}{\pgfqpoint{1.679123in}{2.107020in}}%
\pgfpathclose%
\pgfusepath{stroke,fill}%
\end{pgfscope}%
\begin{pgfscope}%
\pgfpathrectangle{\pgfqpoint{0.100000in}{0.212622in}}{\pgfqpoint{3.696000in}{3.696000in}}%
\pgfusepath{clip}%
\pgfsetbuttcap%
\pgfsetroundjoin%
\definecolor{currentfill}{rgb}{0.121569,0.466667,0.705882}%
\pgfsetfillcolor{currentfill}%
\pgfsetfillopacity{0.302778}%
\pgfsetlinewidth{1.003750pt}%
\definecolor{currentstroke}{rgb}{0.121569,0.466667,0.705882}%
\pgfsetstrokecolor{currentstroke}%
\pgfsetstrokeopacity{0.302778}%
\pgfsetdash{}{0pt}%
\pgfpathmoveto{\pgfqpoint{1.652871in}{2.098044in}}%
\pgfpathcurveto{\pgfqpoint{1.661107in}{2.098044in}}{\pgfqpoint{1.669007in}{2.101316in}}{\pgfqpoint{1.674831in}{2.107140in}}%
\pgfpathcurveto{\pgfqpoint{1.680655in}{2.112964in}}{\pgfqpoint{1.683927in}{2.120864in}}{\pgfqpoint{1.683927in}{2.129101in}}%
\pgfpathcurveto{\pgfqpoint{1.683927in}{2.137337in}}{\pgfqpoint{1.680655in}{2.145237in}}{\pgfqpoint{1.674831in}{2.151061in}}%
\pgfpathcurveto{\pgfqpoint{1.669007in}{2.156885in}}{\pgfqpoint{1.661107in}{2.160157in}}{\pgfqpoint{1.652871in}{2.160157in}}%
\pgfpathcurveto{\pgfqpoint{1.644635in}{2.160157in}}{\pgfqpoint{1.636735in}{2.156885in}}{\pgfqpoint{1.630911in}{2.151061in}}%
\pgfpathcurveto{\pgfqpoint{1.625087in}{2.145237in}}{\pgfqpoint{1.621814in}{2.137337in}}{\pgfqpoint{1.621814in}{2.129101in}}%
\pgfpathcurveto{\pgfqpoint{1.621814in}{2.120864in}}{\pgfqpoint{1.625087in}{2.112964in}}{\pgfqpoint{1.630911in}{2.107140in}}%
\pgfpathcurveto{\pgfqpoint{1.636735in}{2.101316in}}{\pgfqpoint{1.644635in}{2.098044in}}{\pgfqpoint{1.652871in}{2.098044in}}%
\pgfpathclose%
\pgfusepath{stroke,fill}%
\end{pgfscope}%
\begin{pgfscope}%
\pgfpathrectangle{\pgfqpoint{0.100000in}{0.212622in}}{\pgfqpoint{3.696000in}{3.696000in}}%
\pgfusepath{clip}%
\pgfsetbuttcap%
\pgfsetroundjoin%
\definecolor{currentfill}{rgb}{0.121569,0.466667,0.705882}%
\pgfsetfillcolor{currentfill}%
\pgfsetfillopacity{0.302870}%
\pgfsetlinewidth{1.003750pt}%
\definecolor{currentstroke}{rgb}{0.121569,0.466667,0.705882}%
\pgfsetstrokecolor{currentstroke}%
\pgfsetstrokeopacity{0.302870}%
\pgfsetdash{}{0pt}%
\pgfpathmoveto{\pgfqpoint{1.652527in}{2.097647in}}%
\pgfpathcurveto{\pgfqpoint{1.660763in}{2.097647in}}{\pgfqpoint{1.668663in}{2.100919in}}{\pgfqpoint{1.674487in}{2.106743in}}%
\pgfpathcurveto{\pgfqpoint{1.680311in}{2.112567in}}{\pgfqpoint{1.683583in}{2.120467in}}{\pgfqpoint{1.683583in}{2.128704in}}%
\pgfpathcurveto{\pgfqpoint{1.683583in}{2.136940in}}{\pgfqpoint{1.680311in}{2.144840in}}{\pgfqpoint{1.674487in}{2.150664in}}%
\pgfpathcurveto{\pgfqpoint{1.668663in}{2.156488in}}{\pgfqpoint{1.660763in}{2.159760in}}{\pgfqpoint{1.652527in}{2.159760in}}%
\pgfpathcurveto{\pgfqpoint{1.644290in}{2.159760in}}{\pgfqpoint{1.636390in}{2.156488in}}{\pgfqpoint{1.630566in}{2.150664in}}%
\pgfpathcurveto{\pgfqpoint{1.624742in}{2.144840in}}{\pgfqpoint{1.621470in}{2.136940in}}{\pgfqpoint{1.621470in}{2.128704in}}%
\pgfpathcurveto{\pgfqpoint{1.621470in}{2.120467in}}{\pgfqpoint{1.624742in}{2.112567in}}{\pgfqpoint{1.630566in}{2.106743in}}%
\pgfpathcurveto{\pgfqpoint{1.636390in}{2.100919in}}{\pgfqpoint{1.644290in}{2.097647in}}{\pgfqpoint{1.652527in}{2.097647in}}%
\pgfpathclose%
\pgfusepath{stroke,fill}%
\end{pgfscope}%
\begin{pgfscope}%
\pgfpathrectangle{\pgfqpoint{0.100000in}{0.212622in}}{\pgfqpoint{3.696000in}{3.696000in}}%
\pgfusepath{clip}%
\pgfsetbuttcap%
\pgfsetroundjoin%
\definecolor{currentfill}{rgb}{0.121569,0.466667,0.705882}%
\pgfsetfillcolor{currentfill}%
\pgfsetfillopacity{0.303061}%
\pgfsetlinewidth{1.003750pt}%
\definecolor{currentstroke}{rgb}{0.121569,0.466667,0.705882}%
\pgfsetstrokecolor{currentstroke}%
\pgfsetstrokeopacity{0.303061}%
\pgfsetdash{}{0pt}%
\pgfpathmoveto{\pgfqpoint{1.652018in}{2.096974in}}%
\pgfpathcurveto{\pgfqpoint{1.660255in}{2.096974in}}{\pgfqpoint{1.668155in}{2.100246in}}{\pgfqpoint{1.673979in}{2.106070in}}%
\pgfpathcurveto{\pgfqpoint{1.679802in}{2.111894in}}{\pgfqpoint{1.683075in}{2.119794in}}{\pgfqpoint{1.683075in}{2.128030in}}%
\pgfpathcurveto{\pgfqpoint{1.683075in}{2.136267in}}{\pgfqpoint{1.679802in}{2.144167in}}{\pgfqpoint{1.673979in}{2.149991in}}%
\pgfpathcurveto{\pgfqpoint{1.668155in}{2.155815in}}{\pgfqpoint{1.660255in}{2.159087in}}{\pgfqpoint{1.652018in}{2.159087in}}%
\pgfpathcurveto{\pgfqpoint{1.643782in}{2.159087in}}{\pgfqpoint{1.635882in}{2.155815in}}{\pgfqpoint{1.630058in}{2.149991in}}%
\pgfpathcurveto{\pgfqpoint{1.624234in}{2.144167in}}{\pgfqpoint{1.620962in}{2.136267in}}{\pgfqpoint{1.620962in}{2.128030in}}%
\pgfpathcurveto{\pgfqpoint{1.620962in}{2.119794in}}{\pgfqpoint{1.624234in}{2.111894in}}{\pgfqpoint{1.630058in}{2.106070in}}%
\pgfpathcurveto{\pgfqpoint{1.635882in}{2.100246in}}{\pgfqpoint{1.643782in}{2.096974in}}{\pgfqpoint{1.652018in}{2.096974in}}%
\pgfpathclose%
\pgfusepath{stroke,fill}%
\end{pgfscope}%
\begin{pgfscope}%
\pgfpathrectangle{\pgfqpoint{0.100000in}{0.212622in}}{\pgfqpoint{3.696000in}{3.696000in}}%
\pgfusepath{clip}%
\pgfsetbuttcap%
\pgfsetroundjoin%
\definecolor{currentfill}{rgb}{0.121569,0.466667,0.705882}%
\pgfsetfillcolor{currentfill}%
\pgfsetfillopacity{0.303099}%
\pgfsetlinewidth{1.003750pt}%
\definecolor{currentstroke}{rgb}{0.121569,0.466667,0.705882}%
\pgfsetstrokecolor{currentstroke}%
\pgfsetstrokeopacity{0.303099}%
\pgfsetdash{}{0pt}%
\pgfpathmoveto{\pgfqpoint{1.680832in}{2.107847in}}%
\pgfpathcurveto{\pgfqpoint{1.689069in}{2.107847in}}{\pgfqpoint{1.696969in}{2.111120in}}{\pgfqpoint{1.702793in}{2.116944in}}%
\pgfpathcurveto{\pgfqpoint{1.708617in}{2.122767in}}{\pgfqpoint{1.711889in}{2.130667in}}{\pgfqpoint{1.711889in}{2.138904in}}%
\pgfpathcurveto{\pgfqpoint{1.711889in}{2.147140in}}{\pgfqpoint{1.708617in}{2.155040in}}{\pgfqpoint{1.702793in}{2.160864in}}%
\pgfpathcurveto{\pgfqpoint{1.696969in}{2.166688in}}{\pgfqpoint{1.689069in}{2.169960in}}{\pgfqpoint{1.680832in}{2.169960in}}%
\pgfpathcurveto{\pgfqpoint{1.672596in}{2.169960in}}{\pgfqpoint{1.664696in}{2.166688in}}{\pgfqpoint{1.658872in}{2.160864in}}%
\pgfpathcurveto{\pgfqpoint{1.653048in}{2.155040in}}{\pgfqpoint{1.649776in}{2.147140in}}{\pgfqpoint{1.649776in}{2.138904in}}%
\pgfpathcurveto{\pgfqpoint{1.649776in}{2.130667in}}{\pgfqpoint{1.653048in}{2.122767in}}{\pgfqpoint{1.658872in}{2.116944in}}%
\pgfpathcurveto{\pgfqpoint{1.664696in}{2.111120in}}{\pgfqpoint{1.672596in}{2.107847in}}{\pgfqpoint{1.680832in}{2.107847in}}%
\pgfpathclose%
\pgfusepath{stroke,fill}%
\end{pgfscope}%
\begin{pgfscope}%
\pgfpathrectangle{\pgfqpoint{0.100000in}{0.212622in}}{\pgfqpoint{3.696000in}{3.696000in}}%
\pgfusepath{clip}%
\pgfsetbuttcap%
\pgfsetroundjoin%
\definecolor{currentfill}{rgb}{0.121569,0.466667,0.705882}%
\pgfsetfillcolor{currentfill}%
\pgfsetfillopacity{0.303327}%
\pgfsetlinewidth{1.003750pt}%
\definecolor{currentstroke}{rgb}{0.121569,0.466667,0.705882}%
\pgfsetstrokecolor{currentstroke}%
\pgfsetstrokeopacity{0.303327}%
\pgfsetdash{}{0pt}%
\pgfpathmoveto{\pgfqpoint{1.682722in}{2.106926in}}%
\pgfpathcurveto{\pgfqpoint{1.690959in}{2.106926in}}{\pgfqpoint{1.698859in}{2.110198in}}{\pgfqpoint{1.704683in}{2.116022in}}%
\pgfpathcurveto{\pgfqpoint{1.710506in}{2.121846in}}{\pgfqpoint{1.713779in}{2.129746in}}{\pgfqpoint{1.713779in}{2.137982in}}%
\pgfpathcurveto{\pgfqpoint{1.713779in}{2.146218in}}{\pgfqpoint{1.710506in}{2.154118in}}{\pgfqpoint{1.704683in}{2.159942in}}%
\pgfpathcurveto{\pgfqpoint{1.698859in}{2.165766in}}{\pgfqpoint{1.690959in}{2.169039in}}{\pgfqpoint{1.682722in}{2.169039in}}%
\pgfpathcurveto{\pgfqpoint{1.674486in}{2.169039in}}{\pgfqpoint{1.666586in}{2.165766in}}{\pgfqpoint{1.660762in}{2.159942in}}%
\pgfpathcurveto{\pgfqpoint{1.654938in}{2.154118in}}{\pgfqpoint{1.651666in}{2.146218in}}{\pgfqpoint{1.651666in}{2.137982in}}%
\pgfpathcurveto{\pgfqpoint{1.651666in}{2.129746in}}{\pgfqpoint{1.654938in}{2.121846in}}{\pgfqpoint{1.660762in}{2.116022in}}%
\pgfpathcurveto{\pgfqpoint{1.666586in}{2.110198in}}{\pgfqpoint{1.674486in}{2.106926in}}{\pgfqpoint{1.682722in}{2.106926in}}%
\pgfpathclose%
\pgfusepath{stroke,fill}%
\end{pgfscope}%
\begin{pgfscope}%
\pgfpathrectangle{\pgfqpoint{0.100000in}{0.212622in}}{\pgfqpoint{3.696000in}{3.696000in}}%
\pgfusepath{clip}%
\pgfsetbuttcap%
\pgfsetroundjoin%
\definecolor{currentfill}{rgb}{0.121569,0.466667,0.705882}%
\pgfsetfillcolor{currentfill}%
\pgfsetfillopacity{0.303480}%
\pgfsetlinewidth{1.003750pt}%
\definecolor{currentstroke}{rgb}{0.121569,0.466667,0.705882}%
\pgfsetstrokecolor{currentstroke}%
\pgfsetstrokeopacity{0.303480}%
\pgfsetdash{}{0pt}%
\pgfpathmoveto{\pgfqpoint{1.651317in}{2.096124in}}%
\pgfpathcurveto{\pgfqpoint{1.659553in}{2.096124in}}{\pgfqpoint{1.667453in}{2.099397in}}{\pgfqpoint{1.673277in}{2.105221in}}%
\pgfpathcurveto{\pgfqpoint{1.679101in}{2.111044in}}{\pgfqpoint{1.682373in}{2.118945in}}{\pgfqpoint{1.682373in}{2.127181in}}%
\pgfpathcurveto{\pgfqpoint{1.682373in}{2.135417in}}{\pgfqpoint{1.679101in}{2.143317in}}{\pgfqpoint{1.673277in}{2.149141in}}%
\pgfpathcurveto{\pgfqpoint{1.667453in}{2.154965in}}{\pgfqpoint{1.659553in}{2.158237in}}{\pgfqpoint{1.651317in}{2.158237in}}%
\pgfpathcurveto{\pgfqpoint{1.643080in}{2.158237in}}{\pgfqpoint{1.635180in}{2.154965in}}{\pgfqpoint{1.629356in}{2.149141in}}%
\pgfpathcurveto{\pgfqpoint{1.623532in}{2.143317in}}{\pgfqpoint{1.620260in}{2.135417in}}{\pgfqpoint{1.620260in}{2.127181in}}%
\pgfpathcurveto{\pgfqpoint{1.620260in}{2.118945in}}{\pgfqpoint{1.623532in}{2.111044in}}{\pgfqpoint{1.629356in}{2.105221in}}%
\pgfpathcurveto{\pgfqpoint{1.635180in}{2.099397in}}{\pgfqpoint{1.643080in}{2.096124in}}{\pgfqpoint{1.651317in}{2.096124in}}%
\pgfpathclose%
\pgfusepath{stroke,fill}%
\end{pgfscope}%
\begin{pgfscope}%
\pgfpathrectangle{\pgfqpoint{0.100000in}{0.212622in}}{\pgfqpoint{3.696000in}{3.696000in}}%
\pgfusepath{clip}%
\pgfsetbuttcap%
\pgfsetroundjoin%
\definecolor{currentfill}{rgb}{0.121569,0.466667,0.705882}%
\pgfsetfillcolor{currentfill}%
\pgfsetfillopacity{0.303590}%
\pgfsetlinewidth{1.003750pt}%
\definecolor{currentstroke}{rgb}{0.121569,0.466667,0.705882}%
\pgfsetstrokecolor{currentstroke}%
\pgfsetstrokeopacity{0.303590}%
\pgfsetdash{}{0pt}%
\pgfpathmoveto{\pgfqpoint{1.650844in}{2.095328in}}%
\pgfpathcurveto{\pgfqpoint{1.659080in}{2.095328in}}{\pgfqpoint{1.666980in}{2.098600in}}{\pgfqpoint{1.672804in}{2.104424in}}%
\pgfpathcurveto{\pgfqpoint{1.678628in}{2.110248in}}{\pgfqpoint{1.681900in}{2.118148in}}{\pgfqpoint{1.681900in}{2.126384in}}%
\pgfpathcurveto{\pgfqpoint{1.681900in}{2.134620in}}{\pgfqpoint{1.678628in}{2.142520in}}{\pgfqpoint{1.672804in}{2.148344in}}%
\pgfpathcurveto{\pgfqpoint{1.666980in}{2.154168in}}{\pgfqpoint{1.659080in}{2.157441in}}{\pgfqpoint{1.650844in}{2.157441in}}%
\pgfpathcurveto{\pgfqpoint{1.642608in}{2.157441in}}{\pgfqpoint{1.634708in}{2.154168in}}{\pgfqpoint{1.628884in}{2.148344in}}%
\pgfpathcurveto{\pgfqpoint{1.623060in}{2.142520in}}{\pgfqpoint{1.619787in}{2.134620in}}{\pgfqpoint{1.619787in}{2.126384in}}%
\pgfpathcurveto{\pgfqpoint{1.619787in}{2.118148in}}{\pgfqpoint{1.623060in}{2.110248in}}{\pgfqpoint{1.628884in}{2.104424in}}%
\pgfpathcurveto{\pgfqpoint{1.634708in}{2.098600in}}{\pgfqpoint{1.642608in}{2.095328in}}{\pgfqpoint{1.650844in}{2.095328in}}%
\pgfpathclose%
\pgfusepath{stroke,fill}%
\end{pgfscope}%
\begin{pgfscope}%
\pgfpathrectangle{\pgfqpoint{0.100000in}{0.212622in}}{\pgfqpoint{3.696000in}{3.696000in}}%
\pgfusepath{clip}%
\pgfsetbuttcap%
\pgfsetroundjoin%
\definecolor{currentfill}{rgb}{0.121569,0.466667,0.705882}%
\pgfsetfillcolor{currentfill}%
\pgfsetfillopacity{0.303911}%
\pgfsetlinewidth{1.003750pt}%
\definecolor{currentstroke}{rgb}{0.121569,0.466667,0.705882}%
\pgfsetstrokecolor{currentstroke}%
\pgfsetstrokeopacity{0.303911}%
\pgfsetdash{}{0pt}%
\pgfpathmoveto{\pgfqpoint{1.650040in}{2.094726in}}%
\pgfpathcurveto{\pgfqpoint{1.658276in}{2.094726in}}{\pgfqpoint{1.666176in}{2.097999in}}{\pgfqpoint{1.672000in}{2.103823in}}%
\pgfpathcurveto{\pgfqpoint{1.677824in}{2.109647in}}{\pgfqpoint{1.681097in}{2.117547in}}{\pgfqpoint{1.681097in}{2.125783in}}%
\pgfpathcurveto{\pgfqpoint{1.681097in}{2.134019in}}{\pgfqpoint{1.677824in}{2.141919in}}{\pgfqpoint{1.672000in}{2.147743in}}%
\pgfpathcurveto{\pgfqpoint{1.666176in}{2.153567in}}{\pgfqpoint{1.658276in}{2.156839in}}{\pgfqpoint{1.650040in}{2.156839in}}%
\pgfpathcurveto{\pgfqpoint{1.641804in}{2.156839in}}{\pgfqpoint{1.633904in}{2.153567in}}{\pgfqpoint{1.628080in}{2.147743in}}%
\pgfpathcurveto{\pgfqpoint{1.622256in}{2.141919in}}{\pgfqpoint{1.618984in}{2.134019in}}{\pgfqpoint{1.618984in}{2.125783in}}%
\pgfpathcurveto{\pgfqpoint{1.618984in}{2.117547in}}{\pgfqpoint{1.622256in}{2.109647in}}{\pgfqpoint{1.628080in}{2.103823in}}%
\pgfpathcurveto{\pgfqpoint{1.633904in}{2.097999in}}{\pgfqpoint{1.641804in}{2.094726in}}{\pgfqpoint{1.650040in}{2.094726in}}%
\pgfpathclose%
\pgfusepath{stroke,fill}%
\end{pgfscope}%
\begin{pgfscope}%
\pgfpathrectangle{\pgfqpoint{0.100000in}{0.212622in}}{\pgfqpoint{3.696000in}{3.696000in}}%
\pgfusepath{clip}%
\pgfsetbuttcap%
\pgfsetroundjoin%
\definecolor{currentfill}{rgb}{0.121569,0.466667,0.705882}%
\pgfsetfillcolor{currentfill}%
\pgfsetfillopacity{0.303978}%
\pgfsetlinewidth{1.003750pt}%
\definecolor{currentstroke}{rgb}{0.121569,0.466667,0.705882}%
\pgfsetstrokecolor{currentstroke}%
\pgfsetstrokeopacity{0.303978}%
\pgfsetdash{}{0pt}%
\pgfpathmoveto{\pgfqpoint{1.685320in}{2.108781in}}%
\pgfpathcurveto{\pgfqpoint{1.693556in}{2.108781in}}{\pgfqpoint{1.701456in}{2.112053in}}{\pgfqpoint{1.707280in}{2.117877in}}%
\pgfpathcurveto{\pgfqpoint{1.713104in}{2.123701in}}{\pgfqpoint{1.716377in}{2.131601in}}{\pgfqpoint{1.716377in}{2.139837in}}%
\pgfpathcurveto{\pgfqpoint{1.716377in}{2.148074in}}{\pgfqpoint{1.713104in}{2.155974in}}{\pgfqpoint{1.707280in}{2.161798in}}%
\pgfpathcurveto{\pgfqpoint{1.701456in}{2.167621in}}{\pgfqpoint{1.693556in}{2.170894in}}{\pgfqpoint{1.685320in}{2.170894in}}%
\pgfpathcurveto{\pgfqpoint{1.677084in}{2.170894in}}{\pgfqpoint{1.669184in}{2.167621in}}{\pgfqpoint{1.663360in}{2.161798in}}%
\pgfpathcurveto{\pgfqpoint{1.657536in}{2.155974in}}{\pgfqpoint{1.654264in}{2.148074in}}{\pgfqpoint{1.654264in}{2.139837in}}%
\pgfpathcurveto{\pgfqpoint{1.654264in}{2.131601in}}{\pgfqpoint{1.657536in}{2.123701in}}{\pgfqpoint{1.663360in}{2.117877in}}%
\pgfpathcurveto{\pgfqpoint{1.669184in}{2.112053in}}{\pgfqpoint{1.677084in}{2.108781in}}{\pgfqpoint{1.685320in}{2.108781in}}%
\pgfpathclose%
\pgfusepath{stroke,fill}%
\end{pgfscope}%
\begin{pgfscope}%
\pgfpathrectangle{\pgfqpoint{0.100000in}{0.212622in}}{\pgfqpoint{3.696000in}{3.696000in}}%
\pgfusepath{clip}%
\pgfsetbuttcap%
\pgfsetroundjoin%
\definecolor{currentfill}{rgb}{0.121569,0.466667,0.705882}%
\pgfsetfillcolor{currentfill}%
\pgfsetfillopacity{0.304485}%
\pgfsetlinewidth{1.003750pt}%
\definecolor{currentstroke}{rgb}{0.121569,0.466667,0.705882}%
\pgfsetstrokecolor{currentstroke}%
\pgfsetstrokeopacity{0.304485}%
\pgfsetdash{}{0pt}%
\pgfpathmoveto{\pgfqpoint{1.688412in}{2.108427in}}%
\pgfpathcurveto{\pgfqpoint{1.696649in}{2.108427in}}{\pgfqpoint{1.704549in}{2.111700in}}{\pgfqpoint{1.710373in}{2.117524in}}%
\pgfpathcurveto{\pgfqpoint{1.716197in}{2.123348in}}{\pgfqpoint{1.719469in}{2.131248in}}{\pgfqpoint{1.719469in}{2.139484in}}%
\pgfpathcurveto{\pgfqpoint{1.719469in}{2.147720in}}{\pgfqpoint{1.716197in}{2.155620in}}{\pgfqpoint{1.710373in}{2.161444in}}%
\pgfpathcurveto{\pgfqpoint{1.704549in}{2.167268in}}{\pgfqpoint{1.696649in}{2.170540in}}{\pgfqpoint{1.688412in}{2.170540in}}%
\pgfpathcurveto{\pgfqpoint{1.680176in}{2.170540in}}{\pgfqpoint{1.672276in}{2.167268in}}{\pgfqpoint{1.666452in}{2.161444in}}%
\pgfpathcurveto{\pgfqpoint{1.660628in}{2.155620in}}{\pgfqpoint{1.657356in}{2.147720in}}{\pgfqpoint{1.657356in}{2.139484in}}%
\pgfpathcurveto{\pgfqpoint{1.657356in}{2.131248in}}{\pgfqpoint{1.660628in}{2.123348in}}{\pgfqpoint{1.666452in}{2.117524in}}%
\pgfpathcurveto{\pgfqpoint{1.672276in}{2.111700in}}{\pgfqpoint{1.680176in}{2.108427in}}{\pgfqpoint{1.688412in}{2.108427in}}%
\pgfpathclose%
\pgfusepath{stroke,fill}%
\end{pgfscope}%
\begin{pgfscope}%
\pgfpathrectangle{\pgfqpoint{0.100000in}{0.212622in}}{\pgfqpoint{3.696000in}{3.696000in}}%
\pgfusepath{clip}%
\pgfsetbuttcap%
\pgfsetroundjoin%
\definecolor{currentfill}{rgb}{0.121569,0.466667,0.705882}%
\pgfsetfillcolor{currentfill}%
\pgfsetfillopacity{0.304508}%
\pgfsetlinewidth{1.003750pt}%
\definecolor{currentstroke}{rgb}{0.121569,0.466667,0.705882}%
\pgfsetstrokecolor{currentstroke}%
\pgfsetstrokeopacity{0.304508}%
\pgfsetdash{}{0pt}%
\pgfpathmoveto{\pgfqpoint{1.649153in}{2.093284in}}%
\pgfpathcurveto{\pgfqpoint{1.657390in}{2.093284in}}{\pgfqpoint{1.665290in}{2.096557in}}{\pgfqpoint{1.671114in}{2.102380in}}%
\pgfpathcurveto{\pgfqpoint{1.676937in}{2.108204in}}{\pgfqpoint{1.680210in}{2.116104in}}{\pgfqpoint{1.680210in}{2.124341in}}%
\pgfpathcurveto{\pgfqpoint{1.680210in}{2.132577in}}{\pgfqpoint{1.676937in}{2.140477in}}{\pgfqpoint{1.671114in}{2.146301in}}%
\pgfpathcurveto{\pgfqpoint{1.665290in}{2.152125in}}{\pgfqpoint{1.657390in}{2.155397in}}{\pgfqpoint{1.649153in}{2.155397in}}%
\pgfpathcurveto{\pgfqpoint{1.640917in}{2.155397in}}{\pgfqpoint{1.633017in}{2.152125in}}{\pgfqpoint{1.627193in}{2.146301in}}%
\pgfpathcurveto{\pgfqpoint{1.621369in}{2.140477in}}{\pgfqpoint{1.618097in}{2.132577in}}{\pgfqpoint{1.618097in}{2.124341in}}%
\pgfpathcurveto{\pgfqpoint{1.618097in}{2.116104in}}{\pgfqpoint{1.621369in}{2.108204in}}{\pgfqpoint{1.627193in}{2.102380in}}%
\pgfpathcurveto{\pgfqpoint{1.633017in}{2.096557in}}{\pgfqpoint{1.640917in}{2.093284in}}{\pgfqpoint{1.649153in}{2.093284in}}%
\pgfpathclose%
\pgfusepath{stroke,fill}%
\end{pgfscope}%
\begin{pgfscope}%
\pgfpathrectangle{\pgfqpoint{0.100000in}{0.212622in}}{\pgfqpoint{3.696000in}{3.696000in}}%
\pgfusepath{clip}%
\pgfsetbuttcap%
\pgfsetroundjoin%
\definecolor{currentfill}{rgb}{0.121569,0.466667,0.705882}%
\pgfsetfillcolor{currentfill}%
\pgfsetfillopacity{0.304799}%
\pgfsetlinewidth{1.003750pt}%
\definecolor{currentstroke}{rgb}{0.121569,0.466667,0.705882}%
\pgfsetstrokecolor{currentstroke}%
\pgfsetstrokeopacity{0.304799}%
\pgfsetdash{}{0pt}%
\pgfpathmoveto{\pgfqpoint{1.690251in}{2.108984in}}%
\pgfpathcurveto{\pgfqpoint{1.698487in}{2.108984in}}{\pgfqpoint{1.706387in}{2.112257in}}{\pgfqpoint{1.712211in}{2.118081in}}%
\pgfpathcurveto{\pgfqpoint{1.718035in}{2.123905in}}{\pgfqpoint{1.721308in}{2.131805in}}{\pgfqpoint{1.721308in}{2.140041in}}%
\pgfpathcurveto{\pgfqpoint{1.721308in}{2.148277in}}{\pgfqpoint{1.718035in}{2.156177in}}{\pgfqpoint{1.712211in}{2.162001in}}%
\pgfpathcurveto{\pgfqpoint{1.706387in}{2.167825in}}{\pgfqpoint{1.698487in}{2.171097in}}{\pgfqpoint{1.690251in}{2.171097in}}%
\pgfpathcurveto{\pgfqpoint{1.682015in}{2.171097in}}{\pgfqpoint{1.674115in}{2.167825in}}{\pgfqpoint{1.668291in}{2.162001in}}%
\pgfpathcurveto{\pgfqpoint{1.662467in}{2.156177in}}{\pgfqpoint{1.659195in}{2.148277in}}{\pgfqpoint{1.659195in}{2.140041in}}%
\pgfpathcurveto{\pgfqpoint{1.659195in}{2.131805in}}{\pgfqpoint{1.662467in}{2.123905in}}{\pgfqpoint{1.668291in}{2.118081in}}%
\pgfpathcurveto{\pgfqpoint{1.674115in}{2.112257in}}{\pgfqpoint{1.682015in}{2.108984in}}{\pgfqpoint{1.690251in}{2.108984in}}%
\pgfpathclose%
\pgfusepath{stroke,fill}%
\end{pgfscope}%
\begin{pgfscope}%
\pgfpathrectangle{\pgfqpoint{0.100000in}{0.212622in}}{\pgfqpoint{3.696000in}{3.696000in}}%
\pgfusepath{clip}%
\pgfsetbuttcap%
\pgfsetroundjoin%
\definecolor{currentfill}{rgb}{0.121569,0.466667,0.705882}%
\pgfsetfillcolor{currentfill}%
\pgfsetfillopacity{0.305251}%
\pgfsetlinewidth{1.003750pt}%
\definecolor{currentstroke}{rgb}{0.121569,0.466667,0.705882}%
\pgfsetstrokecolor{currentstroke}%
\pgfsetstrokeopacity{0.305251}%
\pgfsetdash{}{0pt}%
\pgfpathmoveto{\pgfqpoint{1.693122in}{2.109460in}}%
\pgfpathcurveto{\pgfqpoint{1.701359in}{2.109460in}}{\pgfqpoint{1.709259in}{2.112732in}}{\pgfqpoint{1.715083in}{2.118556in}}%
\pgfpathcurveto{\pgfqpoint{1.720907in}{2.124380in}}{\pgfqpoint{1.724179in}{2.132280in}}{\pgfqpoint{1.724179in}{2.140516in}}%
\pgfpathcurveto{\pgfqpoint{1.724179in}{2.148752in}}{\pgfqpoint{1.720907in}{2.156653in}}{\pgfqpoint{1.715083in}{2.162476in}}%
\pgfpathcurveto{\pgfqpoint{1.709259in}{2.168300in}}{\pgfqpoint{1.701359in}{2.171573in}}{\pgfqpoint{1.693122in}{2.171573in}}%
\pgfpathcurveto{\pgfqpoint{1.684886in}{2.171573in}}{\pgfqpoint{1.676986in}{2.168300in}}{\pgfqpoint{1.671162in}{2.162476in}}%
\pgfpathcurveto{\pgfqpoint{1.665338in}{2.156653in}}{\pgfqpoint{1.662066in}{2.148752in}}{\pgfqpoint{1.662066in}{2.140516in}}%
\pgfpathcurveto{\pgfqpoint{1.662066in}{2.132280in}}{\pgfqpoint{1.665338in}{2.124380in}}{\pgfqpoint{1.671162in}{2.118556in}}%
\pgfpathcurveto{\pgfqpoint{1.676986in}{2.112732in}}{\pgfqpoint{1.684886in}{2.109460in}}{\pgfqpoint{1.693122in}{2.109460in}}%
\pgfpathclose%
\pgfusepath{stroke,fill}%
\end{pgfscope}%
\begin{pgfscope}%
\pgfpathrectangle{\pgfqpoint{0.100000in}{0.212622in}}{\pgfqpoint{3.696000in}{3.696000in}}%
\pgfusepath{clip}%
\pgfsetbuttcap%
\pgfsetroundjoin%
\definecolor{currentfill}{rgb}{0.121569,0.466667,0.705882}%
\pgfsetfillcolor{currentfill}%
\pgfsetfillopacity{0.305408}%
\pgfsetlinewidth{1.003750pt}%
\definecolor{currentstroke}{rgb}{0.121569,0.466667,0.705882}%
\pgfsetstrokecolor{currentstroke}%
\pgfsetstrokeopacity{0.305408}%
\pgfsetdash{}{0pt}%
\pgfpathmoveto{\pgfqpoint{1.645523in}{2.091361in}}%
\pgfpathcurveto{\pgfqpoint{1.653760in}{2.091361in}}{\pgfqpoint{1.661660in}{2.094633in}}{\pgfqpoint{1.667484in}{2.100457in}}%
\pgfpathcurveto{\pgfqpoint{1.673307in}{2.106281in}}{\pgfqpoint{1.676580in}{2.114181in}}{\pgfqpoint{1.676580in}{2.122417in}}%
\pgfpathcurveto{\pgfqpoint{1.676580in}{2.130653in}}{\pgfqpoint{1.673307in}{2.138553in}}{\pgfqpoint{1.667484in}{2.144377in}}%
\pgfpathcurveto{\pgfqpoint{1.661660in}{2.150201in}}{\pgfqpoint{1.653760in}{2.153474in}}{\pgfqpoint{1.645523in}{2.153474in}}%
\pgfpathcurveto{\pgfqpoint{1.637287in}{2.153474in}}{\pgfqpoint{1.629387in}{2.150201in}}{\pgfqpoint{1.623563in}{2.144377in}}%
\pgfpathcurveto{\pgfqpoint{1.617739in}{2.138553in}}{\pgfqpoint{1.614467in}{2.130653in}}{\pgfqpoint{1.614467in}{2.122417in}}%
\pgfpathcurveto{\pgfqpoint{1.614467in}{2.114181in}}{\pgfqpoint{1.617739in}{2.106281in}}{\pgfqpoint{1.623563in}{2.100457in}}%
\pgfpathcurveto{\pgfqpoint{1.629387in}{2.094633in}}{\pgfqpoint{1.637287in}{2.091361in}}{\pgfqpoint{1.645523in}{2.091361in}}%
\pgfpathclose%
\pgfusepath{stroke,fill}%
\end{pgfscope}%
\begin{pgfscope}%
\pgfpathrectangle{\pgfqpoint{0.100000in}{0.212622in}}{\pgfqpoint{3.696000in}{3.696000in}}%
\pgfusepath{clip}%
\pgfsetbuttcap%
\pgfsetroundjoin%
\definecolor{currentfill}{rgb}{0.121569,0.466667,0.705882}%
\pgfsetfillcolor{currentfill}%
\pgfsetfillopacity{0.306091}%
\pgfsetlinewidth{1.003750pt}%
\definecolor{currentstroke}{rgb}{0.121569,0.466667,0.705882}%
\pgfsetstrokecolor{currentstroke}%
\pgfsetstrokeopacity{0.306091}%
\pgfsetdash{}{0pt}%
\pgfpathmoveto{\pgfqpoint{1.697902in}{2.110731in}}%
\pgfpathcurveto{\pgfqpoint{1.706138in}{2.110731in}}{\pgfqpoint{1.714038in}{2.114004in}}{\pgfqpoint{1.719862in}{2.119827in}}%
\pgfpathcurveto{\pgfqpoint{1.725686in}{2.125651in}}{\pgfqpoint{1.728958in}{2.133551in}}{\pgfqpoint{1.728958in}{2.141788in}}%
\pgfpathcurveto{\pgfqpoint{1.728958in}{2.150024in}}{\pgfqpoint{1.725686in}{2.157924in}}{\pgfqpoint{1.719862in}{2.163748in}}%
\pgfpathcurveto{\pgfqpoint{1.714038in}{2.169572in}}{\pgfqpoint{1.706138in}{2.172844in}}{\pgfqpoint{1.697902in}{2.172844in}}%
\pgfpathcurveto{\pgfqpoint{1.689666in}{2.172844in}}{\pgfqpoint{1.681765in}{2.169572in}}{\pgfqpoint{1.675942in}{2.163748in}}%
\pgfpathcurveto{\pgfqpoint{1.670118in}{2.157924in}}{\pgfqpoint{1.666845in}{2.150024in}}{\pgfqpoint{1.666845in}{2.141788in}}%
\pgfpathcurveto{\pgfqpoint{1.666845in}{2.133551in}}{\pgfqpoint{1.670118in}{2.125651in}}{\pgfqpoint{1.675942in}{2.119827in}}%
\pgfpathcurveto{\pgfqpoint{1.681765in}{2.114004in}}{\pgfqpoint{1.689666in}{2.110731in}}{\pgfqpoint{1.697902in}{2.110731in}}%
\pgfpathclose%
\pgfusepath{stroke,fill}%
\end{pgfscope}%
\begin{pgfscope}%
\pgfpathrectangle{\pgfqpoint{0.100000in}{0.212622in}}{\pgfqpoint{3.696000in}{3.696000in}}%
\pgfusepath{clip}%
\pgfsetbuttcap%
\pgfsetroundjoin%
\definecolor{currentfill}{rgb}{0.121569,0.466667,0.705882}%
\pgfsetfillcolor{currentfill}%
\pgfsetfillopacity{0.306270}%
\pgfsetlinewidth{1.003750pt}%
\definecolor{currentstroke}{rgb}{0.121569,0.466667,0.705882}%
\pgfsetstrokecolor{currentstroke}%
\pgfsetstrokeopacity{0.306270}%
\pgfsetdash{}{0pt}%
\pgfpathmoveto{\pgfqpoint{1.643983in}{2.089846in}}%
\pgfpathcurveto{\pgfqpoint{1.652219in}{2.089846in}}{\pgfqpoint{1.660119in}{2.093118in}}{\pgfqpoint{1.665943in}{2.098942in}}%
\pgfpathcurveto{\pgfqpoint{1.671767in}{2.104766in}}{\pgfqpoint{1.675039in}{2.112666in}}{\pgfqpoint{1.675039in}{2.120902in}}%
\pgfpathcurveto{\pgfqpoint{1.675039in}{2.129139in}}{\pgfqpoint{1.671767in}{2.137039in}}{\pgfqpoint{1.665943in}{2.142863in}}%
\pgfpathcurveto{\pgfqpoint{1.660119in}{2.148687in}}{\pgfqpoint{1.652219in}{2.151959in}}{\pgfqpoint{1.643983in}{2.151959in}}%
\pgfpathcurveto{\pgfqpoint{1.635747in}{2.151959in}}{\pgfqpoint{1.627847in}{2.148687in}}{\pgfqpoint{1.622023in}{2.142863in}}%
\pgfpathcurveto{\pgfqpoint{1.616199in}{2.137039in}}{\pgfqpoint{1.612926in}{2.129139in}}{\pgfqpoint{1.612926in}{2.120902in}}%
\pgfpathcurveto{\pgfqpoint{1.612926in}{2.112666in}}{\pgfqpoint{1.616199in}{2.104766in}}{\pgfqpoint{1.622023in}{2.098942in}}%
\pgfpathcurveto{\pgfqpoint{1.627847in}{2.093118in}}{\pgfqpoint{1.635747in}{2.089846in}}{\pgfqpoint{1.643983in}{2.089846in}}%
\pgfpathclose%
\pgfusepath{stroke,fill}%
\end{pgfscope}%
\begin{pgfscope}%
\pgfpathrectangle{\pgfqpoint{0.100000in}{0.212622in}}{\pgfqpoint{3.696000in}{3.696000in}}%
\pgfusepath{clip}%
\pgfsetbuttcap%
\pgfsetroundjoin%
\definecolor{currentfill}{rgb}{0.121569,0.466667,0.705882}%
\pgfsetfillcolor{currentfill}%
\pgfsetfillopacity{0.306568}%
\pgfsetlinewidth{1.003750pt}%
\definecolor{currentstroke}{rgb}{0.121569,0.466667,0.705882}%
\pgfsetstrokecolor{currentstroke}%
\pgfsetstrokeopacity{0.306568}%
\pgfsetdash{}{0pt}%
\pgfpathmoveto{\pgfqpoint{1.700462in}{2.111295in}}%
\pgfpathcurveto{\pgfqpoint{1.708698in}{2.111295in}}{\pgfqpoint{1.716598in}{2.114568in}}{\pgfqpoint{1.722422in}{2.120391in}}%
\pgfpathcurveto{\pgfqpoint{1.728246in}{2.126215in}}{\pgfqpoint{1.731519in}{2.134115in}}{\pgfqpoint{1.731519in}{2.142352in}}%
\pgfpathcurveto{\pgfqpoint{1.731519in}{2.150588in}}{\pgfqpoint{1.728246in}{2.158488in}}{\pgfqpoint{1.722422in}{2.164312in}}%
\pgfpathcurveto{\pgfqpoint{1.716598in}{2.170136in}}{\pgfqpoint{1.708698in}{2.173408in}}{\pgfqpoint{1.700462in}{2.173408in}}%
\pgfpathcurveto{\pgfqpoint{1.692226in}{2.173408in}}{\pgfqpoint{1.684326in}{2.170136in}}{\pgfqpoint{1.678502in}{2.164312in}}%
\pgfpathcurveto{\pgfqpoint{1.672678in}{2.158488in}}{\pgfqpoint{1.669406in}{2.150588in}}{\pgfqpoint{1.669406in}{2.142352in}}%
\pgfpathcurveto{\pgfqpoint{1.669406in}{2.134115in}}{\pgfqpoint{1.672678in}{2.126215in}}{\pgfqpoint{1.678502in}{2.120391in}}%
\pgfpathcurveto{\pgfqpoint{1.684326in}{2.114568in}}{\pgfqpoint{1.692226in}{2.111295in}}{\pgfqpoint{1.700462in}{2.111295in}}%
\pgfpathclose%
\pgfusepath{stroke,fill}%
\end{pgfscope}%
\begin{pgfscope}%
\pgfpathrectangle{\pgfqpoint{0.100000in}{0.212622in}}{\pgfqpoint{3.696000in}{3.696000in}}%
\pgfusepath{clip}%
\pgfsetbuttcap%
\pgfsetroundjoin%
\definecolor{currentfill}{rgb}{0.121569,0.466667,0.705882}%
\pgfsetfillcolor{currentfill}%
\pgfsetfillopacity{0.307034}%
\pgfsetlinewidth{1.003750pt}%
\definecolor{currentstroke}{rgb}{0.121569,0.466667,0.705882}%
\pgfsetstrokecolor{currentstroke}%
\pgfsetstrokeopacity{0.307034}%
\pgfsetdash{}{0pt}%
\pgfpathmoveto{\pgfqpoint{1.703929in}{2.110877in}}%
\pgfpathcurveto{\pgfqpoint{1.712166in}{2.110877in}}{\pgfqpoint{1.720066in}{2.114149in}}{\pgfqpoint{1.725890in}{2.119973in}}%
\pgfpathcurveto{\pgfqpoint{1.731714in}{2.125797in}}{\pgfqpoint{1.734986in}{2.133697in}}{\pgfqpoint{1.734986in}{2.141934in}}%
\pgfpathcurveto{\pgfqpoint{1.734986in}{2.150170in}}{\pgfqpoint{1.731714in}{2.158070in}}{\pgfqpoint{1.725890in}{2.163894in}}%
\pgfpathcurveto{\pgfqpoint{1.720066in}{2.169718in}}{\pgfqpoint{1.712166in}{2.172990in}}{\pgfqpoint{1.703929in}{2.172990in}}%
\pgfpathcurveto{\pgfqpoint{1.695693in}{2.172990in}}{\pgfqpoint{1.687793in}{2.169718in}}{\pgfqpoint{1.681969in}{2.163894in}}%
\pgfpathcurveto{\pgfqpoint{1.676145in}{2.158070in}}{\pgfqpoint{1.672873in}{2.150170in}}{\pgfqpoint{1.672873in}{2.141934in}}%
\pgfpathcurveto{\pgfqpoint{1.672873in}{2.133697in}}{\pgfqpoint{1.676145in}{2.125797in}}{\pgfqpoint{1.681969in}{2.119973in}}%
\pgfpathcurveto{\pgfqpoint{1.687793in}{2.114149in}}{\pgfqpoint{1.695693in}{2.110877in}}{\pgfqpoint{1.703929in}{2.110877in}}%
\pgfpathclose%
\pgfusepath{stroke,fill}%
\end{pgfscope}%
\begin{pgfscope}%
\pgfpathrectangle{\pgfqpoint{0.100000in}{0.212622in}}{\pgfqpoint{3.696000in}{3.696000in}}%
\pgfusepath{clip}%
\pgfsetbuttcap%
\pgfsetroundjoin%
\definecolor{currentfill}{rgb}{0.121569,0.466667,0.705882}%
\pgfsetfillcolor{currentfill}%
\pgfsetfillopacity{0.307767}%
\pgfsetlinewidth{1.003750pt}%
\definecolor{currentstroke}{rgb}{0.121569,0.466667,0.705882}%
\pgfsetstrokecolor{currentstroke}%
\pgfsetstrokeopacity{0.307767}%
\pgfsetdash{}{0pt}%
\pgfpathmoveto{\pgfqpoint{1.708174in}{2.113157in}}%
\pgfpathcurveto{\pgfqpoint{1.716411in}{2.113157in}}{\pgfqpoint{1.724311in}{2.116429in}}{\pgfqpoint{1.730135in}{2.122253in}}%
\pgfpathcurveto{\pgfqpoint{1.735959in}{2.128077in}}{\pgfqpoint{1.739231in}{2.135977in}}{\pgfqpoint{1.739231in}{2.144213in}}%
\pgfpathcurveto{\pgfqpoint{1.739231in}{2.152450in}}{\pgfqpoint{1.735959in}{2.160350in}}{\pgfqpoint{1.730135in}{2.166174in}}%
\pgfpathcurveto{\pgfqpoint{1.724311in}{2.171997in}}{\pgfqpoint{1.716411in}{2.175270in}}{\pgfqpoint{1.708174in}{2.175270in}}%
\pgfpathcurveto{\pgfqpoint{1.699938in}{2.175270in}}{\pgfqpoint{1.692038in}{2.171997in}}{\pgfqpoint{1.686214in}{2.166174in}}%
\pgfpathcurveto{\pgfqpoint{1.680390in}{2.160350in}}{\pgfqpoint{1.677118in}{2.152450in}}{\pgfqpoint{1.677118in}{2.144213in}}%
\pgfpathcurveto{\pgfqpoint{1.677118in}{2.135977in}}{\pgfqpoint{1.680390in}{2.128077in}}{\pgfqpoint{1.686214in}{2.122253in}}%
\pgfpathcurveto{\pgfqpoint{1.692038in}{2.116429in}}{\pgfqpoint{1.699938in}{2.113157in}}{\pgfqpoint{1.708174in}{2.113157in}}%
\pgfpathclose%
\pgfusepath{stroke,fill}%
\end{pgfscope}%
\begin{pgfscope}%
\pgfpathrectangle{\pgfqpoint{0.100000in}{0.212622in}}{\pgfqpoint{3.696000in}{3.696000in}}%
\pgfusepath{clip}%
\pgfsetbuttcap%
\pgfsetroundjoin%
\definecolor{currentfill}{rgb}{0.121569,0.466667,0.705882}%
\pgfsetfillcolor{currentfill}%
\pgfsetfillopacity{0.308109}%
\pgfsetlinewidth{1.003750pt}%
\definecolor{currentstroke}{rgb}{0.121569,0.466667,0.705882}%
\pgfsetstrokecolor{currentstroke}%
\pgfsetstrokeopacity{0.308109}%
\pgfsetdash{}{0pt}%
\pgfpathmoveto{\pgfqpoint{1.638060in}{2.093364in}}%
\pgfpathcurveto{\pgfqpoint{1.646296in}{2.093364in}}{\pgfqpoint{1.654196in}{2.096636in}}{\pgfqpoint{1.660020in}{2.102460in}}%
\pgfpathcurveto{\pgfqpoint{1.665844in}{2.108284in}}{\pgfqpoint{1.669116in}{2.116184in}}{\pgfqpoint{1.669116in}{2.124421in}}%
\pgfpathcurveto{\pgfqpoint{1.669116in}{2.132657in}}{\pgfqpoint{1.665844in}{2.140557in}}{\pgfqpoint{1.660020in}{2.146381in}}%
\pgfpathcurveto{\pgfqpoint{1.654196in}{2.152205in}}{\pgfqpoint{1.646296in}{2.155477in}}{\pgfqpoint{1.638060in}{2.155477in}}%
\pgfpathcurveto{\pgfqpoint{1.629824in}{2.155477in}}{\pgfqpoint{1.621923in}{2.152205in}}{\pgfqpoint{1.616100in}{2.146381in}}%
\pgfpathcurveto{\pgfqpoint{1.610276in}{2.140557in}}{\pgfqpoint{1.607003in}{2.132657in}}{\pgfqpoint{1.607003in}{2.124421in}}%
\pgfpathcurveto{\pgfqpoint{1.607003in}{2.116184in}}{\pgfqpoint{1.610276in}{2.108284in}}{\pgfqpoint{1.616100in}{2.102460in}}%
\pgfpathcurveto{\pgfqpoint{1.621923in}{2.096636in}}{\pgfqpoint{1.629824in}{2.093364in}}{\pgfqpoint{1.638060in}{2.093364in}}%
\pgfpathclose%
\pgfusepath{stroke,fill}%
\end{pgfscope}%
\begin{pgfscope}%
\pgfpathrectangle{\pgfqpoint{0.100000in}{0.212622in}}{\pgfqpoint{3.696000in}{3.696000in}}%
\pgfusepath{clip}%
\pgfsetbuttcap%
\pgfsetroundjoin%
\definecolor{currentfill}{rgb}{0.121569,0.466667,0.705882}%
\pgfsetfillcolor{currentfill}%
\pgfsetfillopacity{0.308253}%
\pgfsetlinewidth{1.003750pt}%
\definecolor{currentstroke}{rgb}{0.121569,0.466667,0.705882}%
\pgfsetstrokecolor{currentstroke}%
\pgfsetstrokeopacity{0.308253}%
\pgfsetdash{}{0pt}%
\pgfpathmoveto{\pgfqpoint{1.713074in}{2.112272in}}%
\pgfpathcurveto{\pgfqpoint{1.721311in}{2.112272in}}{\pgfqpoint{1.729211in}{2.115544in}}{\pgfqpoint{1.735035in}{2.121368in}}%
\pgfpathcurveto{\pgfqpoint{1.740859in}{2.127192in}}{\pgfqpoint{1.744131in}{2.135092in}}{\pgfqpoint{1.744131in}{2.143329in}}%
\pgfpathcurveto{\pgfqpoint{1.744131in}{2.151565in}}{\pgfqpoint{1.740859in}{2.159465in}}{\pgfqpoint{1.735035in}{2.165289in}}%
\pgfpathcurveto{\pgfqpoint{1.729211in}{2.171113in}}{\pgfqpoint{1.721311in}{2.174385in}}{\pgfqpoint{1.713074in}{2.174385in}}%
\pgfpathcurveto{\pgfqpoint{1.704838in}{2.174385in}}{\pgfqpoint{1.696938in}{2.171113in}}{\pgfqpoint{1.691114in}{2.165289in}}%
\pgfpathcurveto{\pgfqpoint{1.685290in}{2.159465in}}{\pgfqpoint{1.682018in}{2.151565in}}{\pgfqpoint{1.682018in}{2.143329in}}%
\pgfpathcurveto{\pgfqpoint{1.682018in}{2.135092in}}{\pgfqpoint{1.685290in}{2.127192in}}{\pgfqpoint{1.691114in}{2.121368in}}%
\pgfpathcurveto{\pgfqpoint{1.696938in}{2.115544in}}{\pgfqpoint{1.704838in}{2.112272in}}{\pgfqpoint{1.713074in}{2.112272in}}%
\pgfpathclose%
\pgfusepath{stroke,fill}%
\end{pgfscope}%
\begin{pgfscope}%
\pgfpathrectangle{\pgfqpoint{0.100000in}{0.212622in}}{\pgfqpoint{3.696000in}{3.696000in}}%
\pgfusepath{clip}%
\pgfsetbuttcap%
\pgfsetroundjoin%
\definecolor{currentfill}{rgb}{0.121569,0.466667,0.705882}%
\pgfsetfillcolor{currentfill}%
\pgfsetfillopacity{0.308752}%
\pgfsetlinewidth{1.003750pt}%
\definecolor{currentstroke}{rgb}{0.121569,0.466667,0.705882}%
\pgfsetstrokecolor{currentstroke}%
\pgfsetstrokeopacity{0.308752}%
\pgfsetdash{}{0pt}%
\pgfpathmoveto{\pgfqpoint{1.715706in}{2.113339in}}%
\pgfpathcurveto{\pgfqpoint{1.723942in}{2.113339in}}{\pgfqpoint{1.731842in}{2.116611in}}{\pgfqpoint{1.737666in}{2.122435in}}%
\pgfpathcurveto{\pgfqpoint{1.743490in}{2.128259in}}{\pgfqpoint{1.746763in}{2.136159in}}{\pgfqpoint{1.746763in}{2.144396in}}%
\pgfpathcurveto{\pgfqpoint{1.746763in}{2.152632in}}{\pgfqpoint{1.743490in}{2.160532in}}{\pgfqpoint{1.737666in}{2.166356in}}%
\pgfpathcurveto{\pgfqpoint{1.731842in}{2.172180in}}{\pgfqpoint{1.723942in}{2.175452in}}{\pgfqpoint{1.715706in}{2.175452in}}%
\pgfpathcurveto{\pgfqpoint{1.707470in}{2.175452in}}{\pgfqpoint{1.699570in}{2.172180in}}{\pgfqpoint{1.693746in}{2.166356in}}%
\pgfpathcurveto{\pgfqpoint{1.687922in}{2.160532in}}{\pgfqpoint{1.684650in}{2.152632in}}{\pgfqpoint{1.684650in}{2.144396in}}%
\pgfpathcurveto{\pgfqpoint{1.684650in}{2.136159in}}{\pgfqpoint{1.687922in}{2.128259in}}{\pgfqpoint{1.693746in}{2.122435in}}%
\pgfpathcurveto{\pgfqpoint{1.699570in}{2.116611in}}{\pgfqpoint{1.707470in}{2.113339in}}{\pgfqpoint{1.715706in}{2.113339in}}%
\pgfpathclose%
\pgfusepath{stroke,fill}%
\end{pgfscope}%
\begin{pgfscope}%
\pgfpathrectangle{\pgfqpoint{0.100000in}{0.212622in}}{\pgfqpoint{3.696000in}{3.696000in}}%
\pgfusepath{clip}%
\pgfsetbuttcap%
\pgfsetroundjoin%
\definecolor{currentfill}{rgb}{0.121569,0.466667,0.705882}%
\pgfsetfillcolor{currentfill}%
\pgfsetfillopacity{0.309162}%
\pgfsetlinewidth{1.003750pt}%
\definecolor{currentstroke}{rgb}{0.121569,0.466667,0.705882}%
\pgfsetstrokecolor{currentstroke}%
\pgfsetstrokeopacity{0.309162}%
\pgfsetdash{}{0pt}%
\pgfpathmoveto{\pgfqpoint{1.718576in}{2.113161in}}%
\pgfpathcurveto{\pgfqpoint{1.726812in}{2.113161in}}{\pgfqpoint{1.734712in}{2.116433in}}{\pgfqpoint{1.740536in}{2.122257in}}%
\pgfpathcurveto{\pgfqpoint{1.746360in}{2.128081in}}{\pgfqpoint{1.749632in}{2.135981in}}{\pgfqpoint{1.749632in}{2.144218in}}%
\pgfpathcurveto{\pgfqpoint{1.749632in}{2.152454in}}{\pgfqpoint{1.746360in}{2.160354in}}{\pgfqpoint{1.740536in}{2.166178in}}%
\pgfpathcurveto{\pgfqpoint{1.734712in}{2.172002in}}{\pgfqpoint{1.726812in}{2.175274in}}{\pgfqpoint{1.718576in}{2.175274in}}%
\pgfpathcurveto{\pgfqpoint{1.710339in}{2.175274in}}{\pgfqpoint{1.702439in}{2.172002in}}{\pgfqpoint{1.696615in}{2.166178in}}%
\pgfpathcurveto{\pgfqpoint{1.690792in}{2.160354in}}{\pgfqpoint{1.687519in}{2.152454in}}{\pgfqpoint{1.687519in}{2.144218in}}%
\pgfpathcurveto{\pgfqpoint{1.687519in}{2.135981in}}{\pgfqpoint{1.690792in}{2.128081in}}{\pgfqpoint{1.696615in}{2.122257in}}%
\pgfpathcurveto{\pgfqpoint{1.702439in}{2.116433in}}{\pgfqpoint{1.710339in}{2.113161in}}{\pgfqpoint{1.718576in}{2.113161in}}%
\pgfpathclose%
\pgfusepath{stroke,fill}%
\end{pgfscope}%
\begin{pgfscope}%
\pgfpathrectangle{\pgfqpoint{0.100000in}{0.212622in}}{\pgfqpoint{3.696000in}{3.696000in}}%
\pgfusepath{clip}%
\pgfsetbuttcap%
\pgfsetroundjoin%
\definecolor{currentfill}{rgb}{0.121569,0.466667,0.705882}%
\pgfsetfillcolor{currentfill}%
\pgfsetfillopacity{0.309851}%
\pgfsetlinewidth{1.003750pt}%
\definecolor{currentstroke}{rgb}{0.121569,0.466667,0.705882}%
\pgfsetstrokecolor{currentstroke}%
\pgfsetstrokeopacity{0.309851}%
\pgfsetdash{}{0pt}%
\pgfpathmoveto{\pgfqpoint{1.636594in}{2.091923in}}%
\pgfpathcurveto{\pgfqpoint{1.644830in}{2.091923in}}{\pgfqpoint{1.652730in}{2.095195in}}{\pgfqpoint{1.658554in}{2.101019in}}%
\pgfpathcurveto{\pgfqpoint{1.664378in}{2.106843in}}{\pgfqpoint{1.667650in}{2.114743in}}{\pgfqpoint{1.667650in}{2.122979in}}%
\pgfpathcurveto{\pgfqpoint{1.667650in}{2.131216in}}{\pgfqpoint{1.664378in}{2.139116in}}{\pgfqpoint{1.658554in}{2.144940in}}%
\pgfpathcurveto{\pgfqpoint{1.652730in}{2.150763in}}{\pgfqpoint{1.644830in}{2.154036in}}{\pgfqpoint{1.636594in}{2.154036in}}%
\pgfpathcurveto{\pgfqpoint{1.628358in}{2.154036in}}{\pgfqpoint{1.620458in}{2.150763in}}{\pgfqpoint{1.614634in}{2.144940in}}%
\pgfpathcurveto{\pgfqpoint{1.608810in}{2.139116in}}{\pgfqpoint{1.605537in}{2.131216in}}{\pgfqpoint{1.605537in}{2.122979in}}%
\pgfpathcurveto{\pgfqpoint{1.605537in}{2.114743in}}{\pgfqpoint{1.608810in}{2.106843in}}{\pgfqpoint{1.614634in}{2.101019in}}%
\pgfpathcurveto{\pgfqpoint{1.620458in}{2.095195in}}{\pgfqpoint{1.628358in}{2.091923in}}{\pgfqpoint{1.636594in}{2.091923in}}%
\pgfpathclose%
\pgfusepath{stroke,fill}%
\end{pgfscope}%
\begin{pgfscope}%
\pgfpathrectangle{\pgfqpoint{0.100000in}{0.212622in}}{\pgfqpoint{3.696000in}{3.696000in}}%
\pgfusepath{clip}%
\pgfsetbuttcap%
\pgfsetroundjoin%
\definecolor{currentfill}{rgb}{0.121569,0.466667,0.705882}%
\pgfsetfillcolor{currentfill}%
\pgfsetfillopacity{0.309886}%
\pgfsetlinewidth{1.003750pt}%
\definecolor{currentstroke}{rgb}{0.121569,0.466667,0.705882}%
\pgfsetstrokecolor{currentstroke}%
\pgfsetstrokeopacity{0.309886}%
\pgfsetdash{}{0pt}%
\pgfpathmoveto{\pgfqpoint{1.721927in}{2.114927in}}%
\pgfpathcurveto{\pgfqpoint{1.730164in}{2.114927in}}{\pgfqpoint{1.738064in}{2.118200in}}{\pgfqpoint{1.743888in}{2.124024in}}%
\pgfpathcurveto{\pgfqpoint{1.749711in}{2.129847in}}{\pgfqpoint{1.752984in}{2.137747in}}{\pgfqpoint{1.752984in}{2.145984in}}%
\pgfpathcurveto{\pgfqpoint{1.752984in}{2.154220in}}{\pgfqpoint{1.749711in}{2.162120in}}{\pgfqpoint{1.743888in}{2.167944in}}%
\pgfpathcurveto{\pgfqpoint{1.738064in}{2.173768in}}{\pgfqpoint{1.730164in}{2.177040in}}{\pgfqpoint{1.721927in}{2.177040in}}%
\pgfpathcurveto{\pgfqpoint{1.713691in}{2.177040in}}{\pgfqpoint{1.705791in}{2.173768in}}{\pgfqpoint{1.699967in}{2.167944in}}%
\pgfpathcurveto{\pgfqpoint{1.694143in}{2.162120in}}{\pgfqpoint{1.690871in}{2.154220in}}{\pgfqpoint{1.690871in}{2.145984in}}%
\pgfpathcurveto{\pgfqpoint{1.690871in}{2.137747in}}{\pgfqpoint{1.694143in}{2.129847in}}{\pgfqpoint{1.699967in}{2.124024in}}%
\pgfpathcurveto{\pgfqpoint{1.705791in}{2.118200in}}{\pgfqpoint{1.713691in}{2.114927in}}{\pgfqpoint{1.721927in}{2.114927in}}%
\pgfpathclose%
\pgfusepath{stroke,fill}%
\end{pgfscope}%
\begin{pgfscope}%
\pgfpathrectangle{\pgfqpoint{0.100000in}{0.212622in}}{\pgfqpoint{3.696000in}{3.696000in}}%
\pgfusepath{clip}%
\pgfsetbuttcap%
\pgfsetroundjoin%
\definecolor{currentfill}{rgb}{0.121569,0.466667,0.705882}%
\pgfsetfillcolor{currentfill}%
\pgfsetfillopacity{0.310741}%
\pgfsetlinewidth{1.003750pt}%
\definecolor{currentstroke}{rgb}{0.121569,0.466667,0.705882}%
\pgfsetstrokecolor{currentstroke}%
\pgfsetstrokeopacity{0.310741}%
\pgfsetdash{}{0pt}%
\pgfpathmoveto{\pgfqpoint{1.726540in}{2.115269in}}%
\pgfpathcurveto{\pgfqpoint{1.734776in}{2.115269in}}{\pgfqpoint{1.742676in}{2.118541in}}{\pgfqpoint{1.748500in}{2.124365in}}%
\pgfpathcurveto{\pgfqpoint{1.754324in}{2.130189in}}{\pgfqpoint{1.757596in}{2.138089in}}{\pgfqpoint{1.757596in}{2.146325in}}%
\pgfpathcurveto{\pgfqpoint{1.757596in}{2.154561in}}{\pgfqpoint{1.754324in}{2.162461in}}{\pgfqpoint{1.748500in}{2.168285in}}%
\pgfpathcurveto{\pgfqpoint{1.742676in}{2.174109in}}{\pgfqpoint{1.734776in}{2.177382in}}{\pgfqpoint{1.726540in}{2.177382in}}%
\pgfpathcurveto{\pgfqpoint{1.718303in}{2.177382in}}{\pgfqpoint{1.710403in}{2.174109in}}{\pgfqpoint{1.704579in}{2.168285in}}%
\pgfpathcurveto{\pgfqpoint{1.698756in}{2.162461in}}{\pgfqpoint{1.695483in}{2.154561in}}{\pgfqpoint{1.695483in}{2.146325in}}%
\pgfpathcurveto{\pgfqpoint{1.695483in}{2.138089in}}{\pgfqpoint{1.698756in}{2.130189in}}{\pgfqpoint{1.704579in}{2.124365in}}%
\pgfpathcurveto{\pgfqpoint{1.710403in}{2.118541in}}{\pgfqpoint{1.718303in}{2.115269in}}{\pgfqpoint{1.726540in}{2.115269in}}%
\pgfpathclose%
\pgfusepath{stroke,fill}%
\end{pgfscope}%
\begin{pgfscope}%
\pgfpathrectangle{\pgfqpoint{0.100000in}{0.212622in}}{\pgfqpoint{3.696000in}{3.696000in}}%
\pgfusepath{clip}%
\pgfsetbuttcap%
\pgfsetroundjoin%
\definecolor{currentfill}{rgb}{0.121569,0.466667,0.705882}%
\pgfsetfillcolor{currentfill}%
\pgfsetfillopacity{0.311004}%
\pgfsetlinewidth{1.003750pt}%
\definecolor{currentstroke}{rgb}{0.121569,0.466667,0.705882}%
\pgfsetstrokecolor{currentstroke}%
\pgfsetstrokeopacity{0.311004}%
\pgfsetdash{}{0pt}%
\pgfpathmoveto{\pgfqpoint{1.633716in}{2.087678in}}%
\pgfpathcurveto{\pgfqpoint{1.641952in}{2.087678in}}{\pgfqpoint{1.649852in}{2.090950in}}{\pgfqpoint{1.655676in}{2.096774in}}%
\pgfpathcurveto{\pgfqpoint{1.661500in}{2.102598in}}{\pgfqpoint{1.664772in}{2.110498in}}{\pgfqpoint{1.664772in}{2.118734in}}%
\pgfpathcurveto{\pgfqpoint{1.664772in}{2.126971in}}{\pgfqpoint{1.661500in}{2.134871in}}{\pgfqpoint{1.655676in}{2.140695in}}%
\pgfpathcurveto{\pgfqpoint{1.649852in}{2.146519in}}{\pgfqpoint{1.641952in}{2.149791in}}{\pgfqpoint{1.633716in}{2.149791in}}%
\pgfpathcurveto{\pgfqpoint{1.625479in}{2.149791in}}{\pgfqpoint{1.617579in}{2.146519in}}{\pgfqpoint{1.611755in}{2.140695in}}%
\pgfpathcurveto{\pgfqpoint{1.605931in}{2.134871in}}{\pgfqpoint{1.602659in}{2.126971in}}{\pgfqpoint{1.602659in}{2.118734in}}%
\pgfpathcurveto{\pgfqpoint{1.602659in}{2.110498in}}{\pgfqpoint{1.605931in}{2.102598in}}{\pgfqpoint{1.611755in}{2.096774in}}%
\pgfpathcurveto{\pgfqpoint{1.617579in}{2.090950in}}{\pgfqpoint{1.625479in}{2.087678in}}{\pgfqpoint{1.633716in}{2.087678in}}%
\pgfpathclose%
\pgfusepath{stroke,fill}%
\end{pgfscope}%
\begin{pgfscope}%
\pgfpathrectangle{\pgfqpoint{0.100000in}{0.212622in}}{\pgfqpoint{3.696000in}{3.696000in}}%
\pgfusepath{clip}%
\pgfsetbuttcap%
\pgfsetroundjoin%
\definecolor{currentfill}{rgb}{0.121569,0.466667,0.705882}%
\pgfsetfillcolor{currentfill}%
\pgfsetfillopacity{0.312106}%
\pgfsetlinewidth{1.003750pt}%
\definecolor{currentstroke}{rgb}{0.121569,0.466667,0.705882}%
\pgfsetstrokecolor{currentstroke}%
\pgfsetstrokeopacity{0.312106}%
\pgfsetdash{}{0pt}%
\pgfpathmoveto{\pgfqpoint{1.731705in}{2.119580in}}%
\pgfpathcurveto{\pgfqpoint{1.739942in}{2.119580in}}{\pgfqpoint{1.747842in}{2.122852in}}{\pgfqpoint{1.753666in}{2.128676in}}%
\pgfpathcurveto{\pgfqpoint{1.759489in}{2.134500in}}{\pgfqpoint{1.762762in}{2.142400in}}{\pgfqpoint{1.762762in}{2.150637in}}%
\pgfpathcurveto{\pgfqpoint{1.762762in}{2.158873in}}{\pgfqpoint{1.759489in}{2.166773in}}{\pgfqpoint{1.753666in}{2.172597in}}%
\pgfpathcurveto{\pgfqpoint{1.747842in}{2.178421in}}{\pgfqpoint{1.739942in}{2.181693in}}{\pgfqpoint{1.731705in}{2.181693in}}%
\pgfpathcurveto{\pgfqpoint{1.723469in}{2.181693in}}{\pgfqpoint{1.715569in}{2.178421in}}{\pgfqpoint{1.709745in}{2.172597in}}%
\pgfpathcurveto{\pgfqpoint{1.703921in}{2.166773in}}{\pgfqpoint{1.700649in}{2.158873in}}{\pgfqpoint{1.700649in}{2.150637in}}%
\pgfpathcurveto{\pgfqpoint{1.700649in}{2.142400in}}{\pgfqpoint{1.703921in}{2.134500in}}{\pgfqpoint{1.709745in}{2.128676in}}%
\pgfpathcurveto{\pgfqpoint{1.715569in}{2.122852in}}{\pgfqpoint{1.723469in}{2.119580in}}{\pgfqpoint{1.731705in}{2.119580in}}%
\pgfpathclose%
\pgfusepath{stroke,fill}%
\end{pgfscope}%
\begin{pgfscope}%
\pgfpathrectangle{\pgfqpoint{0.100000in}{0.212622in}}{\pgfqpoint{3.696000in}{3.696000in}}%
\pgfusepath{clip}%
\pgfsetbuttcap%
\pgfsetroundjoin%
\definecolor{currentfill}{rgb}{0.121569,0.466667,0.705882}%
\pgfsetfillcolor{currentfill}%
\pgfsetfillopacity{0.312167}%
\pgfsetlinewidth{1.003750pt}%
\definecolor{currentstroke}{rgb}{0.121569,0.466667,0.705882}%
\pgfsetstrokecolor{currentstroke}%
\pgfsetstrokeopacity{0.312167}%
\pgfsetdash{}{0pt}%
\pgfpathmoveto{\pgfqpoint{1.632633in}{2.085032in}}%
\pgfpathcurveto{\pgfqpoint{1.640869in}{2.085032in}}{\pgfqpoint{1.648769in}{2.088304in}}{\pgfqpoint{1.654593in}{2.094128in}}%
\pgfpathcurveto{\pgfqpoint{1.660417in}{2.099952in}}{\pgfqpoint{1.663689in}{2.107852in}}{\pgfqpoint{1.663689in}{2.116088in}}%
\pgfpathcurveto{\pgfqpoint{1.663689in}{2.124324in}}{\pgfqpoint{1.660417in}{2.132224in}}{\pgfqpoint{1.654593in}{2.138048in}}%
\pgfpathcurveto{\pgfqpoint{1.648769in}{2.143872in}}{\pgfqpoint{1.640869in}{2.147145in}}{\pgfqpoint{1.632633in}{2.147145in}}%
\pgfpathcurveto{\pgfqpoint{1.624396in}{2.147145in}}{\pgfqpoint{1.616496in}{2.143872in}}{\pgfqpoint{1.610672in}{2.138048in}}%
\pgfpathcurveto{\pgfqpoint{1.604848in}{2.132224in}}{\pgfqpoint{1.601576in}{2.124324in}}{\pgfqpoint{1.601576in}{2.116088in}}%
\pgfpathcurveto{\pgfqpoint{1.601576in}{2.107852in}}{\pgfqpoint{1.604848in}{2.099952in}}{\pgfqpoint{1.610672in}{2.094128in}}%
\pgfpathcurveto{\pgfqpoint{1.616496in}{2.088304in}}{\pgfqpoint{1.624396in}{2.085032in}}{\pgfqpoint{1.632633in}{2.085032in}}%
\pgfpathclose%
\pgfusepath{stroke,fill}%
\end{pgfscope}%
\begin{pgfscope}%
\pgfpathrectangle{\pgfqpoint{0.100000in}{0.212622in}}{\pgfqpoint{3.696000in}{3.696000in}}%
\pgfusepath{clip}%
\pgfsetbuttcap%
\pgfsetroundjoin%
\definecolor{currentfill}{rgb}{0.121569,0.466667,0.705882}%
\pgfsetfillcolor{currentfill}%
\pgfsetfillopacity{0.312522}%
\pgfsetlinewidth{1.003750pt}%
\definecolor{currentstroke}{rgb}{0.121569,0.466667,0.705882}%
\pgfsetstrokecolor{currentstroke}%
\pgfsetstrokeopacity{0.312522}%
\pgfsetdash{}{0pt}%
\pgfpathmoveto{\pgfqpoint{1.737904in}{2.115611in}}%
\pgfpathcurveto{\pgfqpoint{1.746140in}{2.115611in}}{\pgfqpoint{1.754040in}{2.118883in}}{\pgfqpoint{1.759864in}{2.124707in}}%
\pgfpathcurveto{\pgfqpoint{1.765688in}{2.130531in}}{\pgfqpoint{1.768960in}{2.138431in}}{\pgfqpoint{1.768960in}{2.146667in}}%
\pgfpathcurveto{\pgfqpoint{1.768960in}{2.154904in}}{\pgfqpoint{1.765688in}{2.162804in}}{\pgfqpoint{1.759864in}{2.168628in}}%
\pgfpathcurveto{\pgfqpoint{1.754040in}{2.174452in}}{\pgfqpoint{1.746140in}{2.177724in}}{\pgfqpoint{1.737904in}{2.177724in}}%
\pgfpathcurveto{\pgfqpoint{1.729667in}{2.177724in}}{\pgfqpoint{1.721767in}{2.174452in}}{\pgfqpoint{1.715943in}{2.168628in}}%
\pgfpathcurveto{\pgfqpoint{1.710120in}{2.162804in}}{\pgfqpoint{1.706847in}{2.154904in}}{\pgfqpoint{1.706847in}{2.146667in}}%
\pgfpathcurveto{\pgfqpoint{1.706847in}{2.138431in}}{\pgfqpoint{1.710120in}{2.130531in}}{\pgfqpoint{1.715943in}{2.124707in}}%
\pgfpathcurveto{\pgfqpoint{1.721767in}{2.118883in}}{\pgfqpoint{1.729667in}{2.115611in}}{\pgfqpoint{1.737904in}{2.115611in}}%
\pgfpathclose%
\pgfusepath{stroke,fill}%
\end{pgfscope}%
\begin{pgfscope}%
\pgfpathrectangle{\pgfqpoint{0.100000in}{0.212622in}}{\pgfqpoint{3.696000in}{3.696000in}}%
\pgfusepath{clip}%
\pgfsetbuttcap%
\pgfsetroundjoin%
\definecolor{currentfill}{rgb}{0.121569,0.466667,0.705882}%
\pgfsetfillcolor{currentfill}%
\pgfsetfillopacity{0.312547}%
\pgfsetlinewidth{1.003750pt}%
\definecolor{currentstroke}{rgb}{0.121569,0.466667,0.705882}%
\pgfsetstrokecolor{currentstroke}%
\pgfsetstrokeopacity{0.312547}%
\pgfsetdash{}{0pt}%
\pgfpathmoveto{\pgfqpoint{1.630785in}{2.081753in}}%
\pgfpathcurveto{\pgfqpoint{1.639021in}{2.081753in}}{\pgfqpoint{1.646921in}{2.085025in}}{\pgfqpoint{1.652745in}{2.090849in}}%
\pgfpathcurveto{\pgfqpoint{1.658569in}{2.096673in}}{\pgfqpoint{1.661841in}{2.104573in}}{\pgfqpoint{1.661841in}{2.112809in}}%
\pgfpathcurveto{\pgfqpoint{1.661841in}{2.121046in}}{\pgfqpoint{1.658569in}{2.128946in}}{\pgfqpoint{1.652745in}{2.134770in}}%
\pgfpathcurveto{\pgfqpoint{1.646921in}{2.140594in}}{\pgfqpoint{1.639021in}{2.143866in}}{\pgfqpoint{1.630785in}{2.143866in}}%
\pgfpathcurveto{\pgfqpoint{1.622548in}{2.143866in}}{\pgfqpoint{1.614648in}{2.140594in}}{\pgfqpoint{1.608824in}{2.134770in}}%
\pgfpathcurveto{\pgfqpoint{1.603000in}{2.128946in}}{\pgfqpoint{1.599728in}{2.121046in}}{\pgfqpoint{1.599728in}{2.112809in}}%
\pgfpathcurveto{\pgfqpoint{1.599728in}{2.104573in}}{\pgfqpoint{1.603000in}{2.096673in}}{\pgfqpoint{1.608824in}{2.090849in}}%
\pgfpathcurveto{\pgfqpoint{1.614648in}{2.085025in}}{\pgfqpoint{1.622548in}{2.081753in}}{\pgfqpoint{1.630785in}{2.081753in}}%
\pgfpathclose%
\pgfusepath{stroke,fill}%
\end{pgfscope}%
\begin{pgfscope}%
\pgfpathrectangle{\pgfqpoint{0.100000in}{0.212622in}}{\pgfqpoint{3.696000in}{3.696000in}}%
\pgfusepath{clip}%
\pgfsetbuttcap%
\pgfsetroundjoin%
\definecolor{currentfill}{rgb}{0.121569,0.466667,0.705882}%
\pgfsetfillcolor{currentfill}%
\pgfsetfillopacity{0.313722}%
\pgfsetlinewidth{1.003750pt}%
\definecolor{currentstroke}{rgb}{0.121569,0.466667,0.705882}%
\pgfsetstrokecolor{currentstroke}%
\pgfsetstrokeopacity{0.313722}%
\pgfsetdash{}{0pt}%
\pgfpathmoveto{\pgfqpoint{1.745025in}{2.120422in}}%
\pgfpathcurveto{\pgfqpoint{1.753262in}{2.120422in}}{\pgfqpoint{1.761162in}{2.123694in}}{\pgfqpoint{1.766986in}{2.129518in}}%
\pgfpathcurveto{\pgfqpoint{1.772809in}{2.135342in}}{\pgfqpoint{1.776082in}{2.143242in}}{\pgfqpoint{1.776082in}{2.151478in}}%
\pgfpathcurveto{\pgfqpoint{1.776082in}{2.159715in}}{\pgfqpoint{1.772809in}{2.167615in}}{\pgfqpoint{1.766986in}{2.173439in}}%
\pgfpathcurveto{\pgfqpoint{1.761162in}{2.179263in}}{\pgfqpoint{1.753262in}{2.182535in}}{\pgfqpoint{1.745025in}{2.182535in}}%
\pgfpathcurveto{\pgfqpoint{1.736789in}{2.182535in}}{\pgfqpoint{1.728889in}{2.179263in}}{\pgfqpoint{1.723065in}{2.173439in}}%
\pgfpathcurveto{\pgfqpoint{1.717241in}{2.167615in}}{\pgfqpoint{1.713969in}{2.159715in}}{\pgfqpoint{1.713969in}{2.151478in}}%
\pgfpathcurveto{\pgfqpoint{1.713969in}{2.143242in}}{\pgfqpoint{1.717241in}{2.135342in}}{\pgfqpoint{1.723065in}{2.129518in}}%
\pgfpathcurveto{\pgfqpoint{1.728889in}{2.123694in}}{\pgfqpoint{1.736789in}{2.120422in}}{\pgfqpoint{1.745025in}{2.120422in}}%
\pgfpathclose%
\pgfusepath{stroke,fill}%
\end{pgfscope}%
\begin{pgfscope}%
\pgfpathrectangle{\pgfqpoint{0.100000in}{0.212622in}}{\pgfqpoint{3.696000in}{3.696000in}}%
\pgfusepath{clip}%
\pgfsetbuttcap%
\pgfsetroundjoin%
\definecolor{currentfill}{rgb}{0.121569,0.466667,0.705882}%
\pgfsetfillcolor{currentfill}%
\pgfsetfillopacity{0.314093}%
\pgfsetlinewidth{1.003750pt}%
\definecolor{currentstroke}{rgb}{0.121569,0.466667,0.705882}%
\pgfsetstrokecolor{currentstroke}%
\pgfsetstrokeopacity{0.314093}%
\pgfsetdash{}{0pt}%
\pgfpathmoveto{\pgfqpoint{1.752970in}{2.116894in}}%
\pgfpathcurveto{\pgfqpoint{1.761206in}{2.116894in}}{\pgfqpoint{1.769106in}{2.120167in}}{\pgfqpoint{1.774930in}{2.125990in}}%
\pgfpathcurveto{\pgfqpoint{1.780754in}{2.131814in}}{\pgfqpoint{1.784027in}{2.139714in}}{\pgfqpoint{1.784027in}{2.147951in}}%
\pgfpathcurveto{\pgfqpoint{1.784027in}{2.156187in}}{\pgfqpoint{1.780754in}{2.164087in}}{\pgfqpoint{1.774930in}{2.169911in}}%
\pgfpathcurveto{\pgfqpoint{1.769106in}{2.175735in}}{\pgfqpoint{1.761206in}{2.179007in}}{\pgfqpoint{1.752970in}{2.179007in}}%
\pgfpathcurveto{\pgfqpoint{1.744734in}{2.179007in}}{\pgfqpoint{1.736834in}{2.175735in}}{\pgfqpoint{1.731010in}{2.169911in}}%
\pgfpathcurveto{\pgfqpoint{1.725186in}{2.164087in}}{\pgfqpoint{1.721914in}{2.156187in}}{\pgfqpoint{1.721914in}{2.147951in}}%
\pgfpathcurveto{\pgfqpoint{1.721914in}{2.139714in}}{\pgfqpoint{1.725186in}{2.131814in}}{\pgfqpoint{1.731010in}{2.125990in}}%
\pgfpathcurveto{\pgfqpoint{1.736834in}{2.120167in}}{\pgfqpoint{1.744734in}{2.116894in}}{\pgfqpoint{1.752970in}{2.116894in}}%
\pgfpathclose%
\pgfusepath{stroke,fill}%
\end{pgfscope}%
\begin{pgfscope}%
\pgfpathrectangle{\pgfqpoint{0.100000in}{0.212622in}}{\pgfqpoint{3.696000in}{3.696000in}}%
\pgfusepath{clip}%
\pgfsetbuttcap%
\pgfsetroundjoin%
\definecolor{currentfill}{rgb}{0.121569,0.466667,0.705882}%
\pgfsetfillcolor{currentfill}%
\pgfsetfillopacity{0.314300}%
\pgfsetlinewidth{1.003750pt}%
\definecolor{currentstroke}{rgb}{0.121569,0.466667,0.705882}%
\pgfsetstrokecolor{currentstroke}%
\pgfsetstrokeopacity{0.314300}%
\pgfsetdash{}{0pt}%
\pgfpathmoveto{\pgfqpoint{1.629340in}{2.082296in}}%
\pgfpathcurveto{\pgfqpoint{1.637576in}{2.082296in}}{\pgfqpoint{1.645476in}{2.085568in}}{\pgfqpoint{1.651300in}{2.091392in}}%
\pgfpathcurveto{\pgfqpoint{1.657124in}{2.097216in}}{\pgfqpoint{1.660397in}{2.105116in}}{\pgfqpoint{1.660397in}{2.113352in}}%
\pgfpathcurveto{\pgfqpoint{1.660397in}{2.121588in}}{\pgfqpoint{1.657124in}{2.129488in}}{\pgfqpoint{1.651300in}{2.135312in}}%
\pgfpathcurveto{\pgfqpoint{1.645476in}{2.141136in}}{\pgfqpoint{1.637576in}{2.144409in}}{\pgfqpoint{1.629340in}{2.144409in}}%
\pgfpathcurveto{\pgfqpoint{1.621104in}{2.144409in}}{\pgfqpoint{1.613204in}{2.141136in}}{\pgfqpoint{1.607380in}{2.135312in}}%
\pgfpathcurveto{\pgfqpoint{1.601556in}{2.129488in}}{\pgfqpoint{1.598284in}{2.121588in}}{\pgfqpoint{1.598284in}{2.113352in}}%
\pgfpathcurveto{\pgfqpoint{1.598284in}{2.105116in}}{\pgfqpoint{1.601556in}{2.097216in}}{\pgfqpoint{1.607380in}{2.091392in}}%
\pgfpathcurveto{\pgfqpoint{1.613204in}{2.085568in}}{\pgfqpoint{1.621104in}{2.082296in}}{\pgfqpoint{1.629340in}{2.082296in}}%
\pgfpathclose%
\pgfusepath{stroke,fill}%
\end{pgfscope}%
\begin{pgfscope}%
\pgfpathrectangle{\pgfqpoint{0.100000in}{0.212622in}}{\pgfqpoint{3.696000in}{3.696000in}}%
\pgfusepath{clip}%
\pgfsetbuttcap%
\pgfsetroundjoin%
\definecolor{currentfill}{rgb}{0.121569,0.466667,0.705882}%
\pgfsetfillcolor{currentfill}%
\pgfsetfillopacity{0.314973}%
\pgfsetlinewidth{1.003750pt}%
\definecolor{currentstroke}{rgb}{0.121569,0.466667,0.705882}%
\pgfsetstrokecolor{currentstroke}%
\pgfsetstrokeopacity{0.314973}%
\pgfsetdash{}{0pt}%
\pgfpathmoveto{\pgfqpoint{1.626269in}{2.080156in}}%
\pgfpathcurveto{\pgfqpoint{1.634506in}{2.080156in}}{\pgfqpoint{1.642406in}{2.083428in}}{\pgfqpoint{1.648230in}{2.089252in}}%
\pgfpathcurveto{\pgfqpoint{1.654054in}{2.095076in}}{\pgfqpoint{1.657326in}{2.102976in}}{\pgfqpoint{1.657326in}{2.111213in}}%
\pgfpathcurveto{\pgfqpoint{1.657326in}{2.119449in}}{\pgfqpoint{1.654054in}{2.127349in}}{\pgfqpoint{1.648230in}{2.133173in}}%
\pgfpathcurveto{\pgfqpoint{1.642406in}{2.138997in}}{\pgfqpoint{1.634506in}{2.142269in}}{\pgfqpoint{1.626269in}{2.142269in}}%
\pgfpathcurveto{\pgfqpoint{1.618033in}{2.142269in}}{\pgfqpoint{1.610133in}{2.138997in}}{\pgfqpoint{1.604309in}{2.133173in}}%
\pgfpathcurveto{\pgfqpoint{1.598485in}{2.127349in}}{\pgfqpoint{1.595213in}{2.119449in}}{\pgfqpoint{1.595213in}{2.111213in}}%
\pgfpathcurveto{\pgfqpoint{1.595213in}{2.102976in}}{\pgfqpoint{1.598485in}{2.095076in}}{\pgfqpoint{1.604309in}{2.089252in}}%
\pgfpathcurveto{\pgfqpoint{1.610133in}{2.083428in}}{\pgfqpoint{1.618033in}{2.080156in}}{\pgfqpoint{1.626269in}{2.080156in}}%
\pgfpathclose%
\pgfusepath{stroke,fill}%
\end{pgfscope}%
\begin{pgfscope}%
\pgfpathrectangle{\pgfqpoint{0.100000in}{0.212622in}}{\pgfqpoint{3.696000in}{3.696000in}}%
\pgfusepath{clip}%
\pgfsetbuttcap%
\pgfsetroundjoin%
\definecolor{currentfill}{rgb}{0.121569,0.466667,0.705882}%
\pgfsetfillcolor{currentfill}%
\pgfsetfillopacity{0.315236}%
\pgfsetlinewidth{1.003750pt}%
\definecolor{currentstroke}{rgb}{0.121569,0.466667,0.705882}%
\pgfsetstrokecolor{currentstroke}%
\pgfsetstrokeopacity{0.315236}%
\pgfsetdash{}{0pt}%
\pgfpathmoveto{\pgfqpoint{1.625595in}{2.079670in}}%
\pgfpathcurveto{\pgfqpoint{1.633831in}{2.079670in}}{\pgfqpoint{1.641731in}{2.082942in}}{\pgfqpoint{1.647555in}{2.088766in}}%
\pgfpathcurveto{\pgfqpoint{1.653379in}{2.094590in}}{\pgfqpoint{1.656651in}{2.102490in}}{\pgfqpoint{1.656651in}{2.110726in}}%
\pgfpathcurveto{\pgfqpoint{1.656651in}{2.118963in}}{\pgfqpoint{1.653379in}{2.126863in}}{\pgfqpoint{1.647555in}{2.132687in}}%
\pgfpathcurveto{\pgfqpoint{1.641731in}{2.138511in}}{\pgfqpoint{1.633831in}{2.141783in}}{\pgfqpoint{1.625595in}{2.141783in}}%
\pgfpathcurveto{\pgfqpoint{1.617358in}{2.141783in}}{\pgfqpoint{1.609458in}{2.138511in}}{\pgfqpoint{1.603634in}{2.132687in}}%
\pgfpathcurveto{\pgfqpoint{1.597811in}{2.126863in}}{\pgfqpoint{1.594538in}{2.118963in}}{\pgfqpoint{1.594538in}{2.110726in}}%
\pgfpathcurveto{\pgfqpoint{1.594538in}{2.102490in}}{\pgfqpoint{1.597811in}{2.094590in}}{\pgfqpoint{1.603634in}{2.088766in}}%
\pgfpathcurveto{\pgfqpoint{1.609458in}{2.082942in}}{\pgfqpoint{1.617358in}{2.079670in}}{\pgfqpoint{1.625595in}{2.079670in}}%
\pgfpathclose%
\pgfusepath{stroke,fill}%
\end{pgfscope}%
\begin{pgfscope}%
\pgfpathrectangle{\pgfqpoint{0.100000in}{0.212622in}}{\pgfqpoint{3.696000in}{3.696000in}}%
\pgfusepath{clip}%
\pgfsetbuttcap%
\pgfsetroundjoin%
\definecolor{currentfill}{rgb}{0.121569,0.466667,0.705882}%
\pgfsetfillcolor{currentfill}%
\pgfsetfillopacity{0.315410}%
\pgfsetlinewidth{1.003750pt}%
\definecolor{currentstroke}{rgb}{0.121569,0.466667,0.705882}%
\pgfsetstrokecolor{currentstroke}%
\pgfsetstrokeopacity{0.315410}%
\pgfsetdash{}{0pt}%
\pgfpathmoveto{\pgfqpoint{1.761193in}{2.119096in}}%
\pgfpathcurveto{\pgfqpoint{1.769430in}{2.119096in}}{\pgfqpoint{1.777330in}{2.122368in}}{\pgfqpoint{1.783154in}{2.128192in}}%
\pgfpathcurveto{\pgfqpoint{1.788978in}{2.134016in}}{\pgfqpoint{1.792250in}{2.141916in}}{\pgfqpoint{1.792250in}{2.150152in}}%
\pgfpathcurveto{\pgfqpoint{1.792250in}{2.158388in}}{\pgfqpoint{1.788978in}{2.166289in}}{\pgfqpoint{1.783154in}{2.172112in}}%
\pgfpathcurveto{\pgfqpoint{1.777330in}{2.177936in}}{\pgfqpoint{1.769430in}{2.181209in}}{\pgfqpoint{1.761193in}{2.181209in}}%
\pgfpathcurveto{\pgfqpoint{1.752957in}{2.181209in}}{\pgfqpoint{1.745057in}{2.177936in}}{\pgfqpoint{1.739233in}{2.172112in}}%
\pgfpathcurveto{\pgfqpoint{1.733409in}{2.166289in}}{\pgfqpoint{1.730137in}{2.158388in}}{\pgfqpoint{1.730137in}{2.150152in}}%
\pgfpathcurveto{\pgfqpoint{1.730137in}{2.141916in}}{\pgfqpoint{1.733409in}{2.134016in}}{\pgfqpoint{1.739233in}{2.128192in}}%
\pgfpathcurveto{\pgfqpoint{1.745057in}{2.122368in}}{\pgfqpoint{1.752957in}{2.119096in}}{\pgfqpoint{1.761193in}{2.119096in}}%
\pgfpathclose%
\pgfusepath{stroke,fill}%
\end{pgfscope}%
\begin{pgfscope}%
\pgfpathrectangle{\pgfqpoint{0.100000in}{0.212622in}}{\pgfqpoint{3.696000in}{3.696000in}}%
\pgfusepath{clip}%
\pgfsetbuttcap%
\pgfsetroundjoin%
\definecolor{currentfill}{rgb}{0.121569,0.466667,0.705882}%
\pgfsetfillcolor{currentfill}%
\pgfsetfillopacity{0.315865}%
\pgfsetlinewidth{1.003750pt}%
\definecolor{currentstroke}{rgb}{0.121569,0.466667,0.705882}%
\pgfsetstrokecolor{currentstroke}%
\pgfsetstrokeopacity{0.315865}%
\pgfsetdash{}{0pt}%
\pgfpathmoveto{\pgfqpoint{1.624949in}{2.079484in}}%
\pgfpathcurveto{\pgfqpoint{1.633186in}{2.079484in}}{\pgfqpoint{1.641086in}{2.082756in}}{\pgfqpoint{1.646910in}{2.088580in}}%
\pgfpathcurveto{\pgfqpoint{1.652734in}{2.094404in}}{\pgfqpoint{1.656006in}{2.102304in}}{\pgfqpoint{1.656006in}{2.110540in}}%
\pgfpathcurveto{\pgfqpoint{1.656006in}{2.118777in}}{\pgfqpoint{1.652734in}{2.126677in}}{\pgfqpoint{1.646910in}{2.132501in}}%
\pgfpathcurveto{\pgfqpoint{1.641086in}{2.138325in}}{\pgfqpoint{1.633186in}{2.141597in}}{\pgfqpoint{1.624949in}{2.141597in}}%
\pgfpathcurveto{\pgfqpoint{1.616713in}{2.141597in}}{\pgfqpoint{1.608813in}{2.138325in}}{\pgfqpoint{1.602989in}{2.132501in}}%
\pgfpathcurveto{\pgfqpoint{1.597165in}{2.126677in}}{\pgfqpoint{1.593893in}{2.118777in}}{\pgfqpoint{1.593893in}{2.110540in}}%
\pgfpathcurveto{\pgfqpoint{1.593893in}{2.102304in}}{\pgfqpoint{1.597165in}{2.094404in}}{\pgfqpoint{1.602989in}{2.088580in}}%
\pgfpathcurveto{\pgfqpoint{1.608813in}{2.082756in}}{\pgfqpoint{1.616713in}{2.079484in}}{\pgfqpoint{1.624949in}{2.079484in}}%
\pgfpathclose%
\pgfusepath{stroke,fill}%
\end{pgfscope}%
\begin{pgfscope}%
\pgfpathrectangle{\pgfqpoint{0.100000in}{0.212622in}}{\pgfqpoint{3.696000in}{3.696000in}}%
\pgfusepath{clip}%
\pgfsetbuttcap%
\pgfsetroundjoin%
\definecolor{currentfill}{rgb}{0.121569,0.466667,0.705882}%
\pgfsetfillcolor{currentfill}%
\pgfsetfillopacity{0.316116}%
\pgfsetlinewidth{1.003750pt}%
\definecolor{currentstroke}{rgb}{0.121569,0.466667,0.705882}%
\pgfsetstrokecolor{currentstroke}%
\pgfsetstrokeopacity{0.316116}%
\pgfsetdash{}{0pt}%
\pgfpathmoveto{\pgfqpoint{1.623888in}{2.077478in}}%
\pgfpathcurveto{\pgfqpoint{1.632125in}{2.077478in}}{\pgfqpoint{1.640025in}{2.080750in}}{\pgfqpoint{1.645848in}{2.086574in}}%
\pgfpathcurveto{\pgfqpoint{1.651672in}{2.092398in}}{\pgfqpoint{1.654945in}{2.100298in}}{\pgfqpoint{1.654945in}{2.108535in}}%
\pgfpathcurveto{\pgfqpoint{1.654945in}{2.116771in}}{\pgfqpoint{1.651672in}{2.124671in}}{\pgfqpoint{1.645848in}{2.130495in}}%
\pgfpathcurveto{\pgfqpoint{1.640025in}{2.136319in}}{\pgfqpoint{1.632125in}{2.139591in}}{\pgfqpoint{1.623888in}{2.139591in}}%
\pgfpathcurveto{\pgfqpoint{1.615652in}{2.139591in}}{\pgfqpoint{1.607752in}{2.136319in}}{\pgfqpoint{1.601928in}{2.130495in}}%
\pgfpathcurveto{\pgfqpoint{1.596104in}{2.124671in}}{\pgfqpoint{1.592832in}{2.116771in}}{\pgfqpoint{1.592832in}{2.108535in}}%
\pgfpathcurveto{\pgfqpoint{1.592832in}{2.100298in}}{\pgfqpoint{1.596104in}{2.092398in}}{\pgfqpoint{1.601928in}{2.086574in}}%
\pgfpathcurveto{\pgfqpoint{1.607752in}{2.080750in}}{\pgfqpoint{1.615652in}{2.077478in}}{\pgfqpoint{1.623888in}{2.077478in}}%
\pgfpathclose%
\pgfusepath{stroke,fill}%
\end{pgfscope}%
\begin{pgfscope}%
\pgfpathrectangle{\pgfqpoint{0.100000in}{0.212622in}}{\pgfqpoint{3.696000in}{3.696000in}}%
\pgfusepath{clip}%
\pgfsetbuttcap%
\pgfsetroundjoin%
\definecolor{currentfill}{rgb}{0.121569,0.466667,0.705882}%
\pgfsetfillcolor{currentfill}%
\pgfsetfillopacity{0.316643}%
\pgfsetlinewidth{1.003750pt}%
\definecolor{currentstroke}{rgb}{0.121569,0.466667,0.705882}%
\pgfsetstrokecolor{currentstroke}%
\pgfsetstrokeopacity{0.316643}%
\pgfsetdash{}{0pt}%
\pgfpathmoveto{\pgfqpoint{1.621817in}{2.074522in}}%
\pgfpathcurveto{\pgfqpoint{1.630053in}{2.074522in}}{\pgfqpoint{1.637953in}{2.077795in}}{\pgfqpoint{1.643777in}{2.083618in}}%
\pgfpathcurveto{\pgfqpoint{1.649601in}{2.089442in}}{\pgfqpoint{1.652873in}{2.097342in}}{\pgfqpoint{1.652873in}{2.105579in}}%
\pgfpathcurveto{\pgfqpoint{1.652873in}{2.113815in}}{\pgfqpoint{1.649601in}{2.121715in}}{\pgfqpoint{1.643777in}{2.127539in}}%
\pgfpathcurveto{\pgfqpoint{1.637953in}{2.133363in}}{\pgfqpoint{1.630053in}{2.136635in}}{\pgfqpoint{1.621817in}{2.136635in}}%
\pgfpathcurveto{\pgfqpoint{1.613581in}{2.136635in}}{\pgfqpoint{1.605680in}{2.133363in}}{\pgfqpoint{1.599857in}{2.127539in}}%
\pgfpathcurveto{\pgfqpoint{1.594033in}{2.121715in}}{\pgfqpoint{1.590760in}{2.113815in}}{\pgfqpoint{1.590760in}{2.105579in}}%
\pgfpathcurveto{\pgfqpoint{1.590760in}{2.097342in}}{\pgfqpoint{1.594033in}{2.089442in}}{\pgfqpoint{1.599857in}{2.083618in}}%
\pgfpathcurveto{\pgfqpoint{1.605680in}{2.077795in}}{\pgfqpoint{1.613581in}{2.074522in}}{\pgfqpoint{1.621817in}{2.074522in}}%
\pgfpathclose%
\pgfusepath{stroke,fill}%
\end{pgfscope}%
\begin{pgfscope}%
\pgfpathrectangle{\pgfqpoint{0.100000in}{0.212622in}}{\pgfqpoint{3.696000in}{3.696000in}}%
\pgfusepath{clip}%
\pgfsetbuttcap%
\pgfsetroundjoin%
\definecolor{currentfill}{rgb}{0.121569,0.466667,0.705882}%
\pgfsetfillcolor{currentfill}%
\pgfsetfillopacity{0.316645}%
\pgfsetlinewidth{1.003750pt}%
\definecolor{currentstroke}{rgb}{0.121569,0.466667,0.705882}%
\pgfsetstrokecolor{currentstroke}%
\pgfsetstrokeopacity{0.316645}%
\pgfsetdash{}{0pt}%
\pgfpathmoveto{\pgfqpoint{1.769686in}{2.118314in}}%
\pgfpathcurveto{\pgfqpoint{1.777923in}{2.118314in}}{\pgfqpoint{1.785823in}{2.121587in}}{\pgfqpoint{1.791647in}{2.127411in}}%
\pgfpathcurveto{\pgfqpoint{1.797471in}{2.133235in}}{\pgfqpoint{1.800743in}{2.141135in}}{\pgfqpoint{1.800743in}{2.149371in}}%
\pgfpathcurveto{\pgfqpoint{1.800743in}{2.157607in}}{\pgfqpoint{1.797471in}{2.165507in}}{\pgfqpoint{1.791647in}{2.171331in}}%
\pgfpathcurveto{\pgfqpoint{1.785823in}{2.177155in}}{\pgfqpoint{1.777923in}{2.180427in}}{\pgfqpoint{1.769686in}{2.180427in}}%
\pgfpathcurveto{\pgfqpoint{1.761450in}{2.180427in}}{\pgfqpoint{1.753550in}{2.177155in}}{\pgfqpoint{1.747726in}{2.171331in}}%
\pgfpathcurveto{\pgfqpoint{1.741902in}{2.165507in}}{\pgfqpoint{1.738630in}{2.157607in}}{\pgfqpoint{1.738630in}{2.149371in}}%
\pgfpathcurveto{\pgfqpoint{1.738630in}{2.141135in}}{\pgfqpoint{1.741902in}{2.133235in}}{\pgfqpoint{1.747726in}{2.127411in}}%
\pgfpathcurveto{\pgfqpoint{1.753550in}{2.121587in}}{\pgfqpoint{1.761450in}{2.118314in}}{\pgfqpoint{1.769686in}{2.118314in}}%
\pgfpathclose%
\pgfusepath{stroke,fill}%
\end{pgfscope}%
\begin{pgfscope}%
\pgfpathrectangle{\pgfqpoint{0.100000in}{0.212622in}}{\pgfqpoint{3.696000in}{3.696000in}}%
\pgfusepath{clip}%
\pgfsetbuttcap%
\pgfsetroundjoin%
\definecolor{currentfill}{rgb}{0.121569,0.466667,0.705882}%
\pgfsetfillcolor{currentfill}%
\pgfsetfillopacity{0.317588}%
\pgfsetlinewidth{1.003750pt}%
\definecolor{currentstroke}{rgb}{0.121569,0.466667,0.705882}%
\pgfsetstrokecolor{currentstroke}%
\pgfsetstrokeopacity{0.317588}%
\pgfsetdash{}{0pt}%
\pgfpathmoveto{\pgfqpoint{1.621108in}{2.074491in}}%
\pgfpathcurveto{\pgfqpoint{1.629345in}{2.074491in}}{\pgfqpoint{1.637245in}{2.077764in}}{\pgfqpoint{1.643069in}{2.083588in}}%
\pgfpathcurveto{\pgfqpoint{1.648893in}{2.089412in}}{\pgfqpoint{1.652165in}{2.097312in}}{\pgfqpoint{1.652165in}{2.105548in}}%
\pgfpathcurveto{\pgfqpoint{1.652165in}{2.113784in}}{\pgfqpoint{1.648893in}{2.121684in}}{\pgfqpoint{1.643069in}{2.127508in}}%
\pgfpathcurveto{\pgfqpoint{1.637245in}{2.133332in}}{\pgfqpoint{1.629345in}{2.136604in}}{\pgfqpoint{1.621108in}{2.136604in}}%
\pgfpathcurveto{\pgfqpoint{1.612872in}{2.136604in}}{\pgfqpoint{1.604972in}{2.133332in}}{\pgfqpoint{1.599148in}{2.127508in}}%
\pgfpathcurveto{\pgfqpoint{1.593324in}{2.121684in}}{\pgfqpoint{1.590052in}{2.113784in}}{\pgfqpoint{1.590052in}{2.105548in}}%
\pgfpathcurveto{\pgfqpoint{1.590052in}{2.097312in}}{\pgfqpoint{1.593324in}{2.089412in}}{\pgfqpoint{1.599148in}{2.083588in}}%
\pgfpathcurveto{\pgfqpoint{1.604972in}{2.077764in}}{\pgfqpoint{1.612872in}{2.074491in}}{\pgfqpoint{1.621108in}{2.074491in}}%
\pgfpathclose%
\pgfusepath{stroke,fill}%
\end{pgfscope}%
\begin{pgfscope}%
\pgfpathrectangle{\pgfqpoint{0.100000in}{0.212622in}}{\pgfqpoint{3.696000in}{3.696000in}}%
\pgfusepath{clip}%
\pgfsetbuttcap%
\pgfsetroundjoin%
\definecolor{currentfill}{rgb}{0.121569,0.466667,0.705882}%
\pgfsetfillcolor{currentfill}%
\pgfsetfillopacity{0.318020}%
\pgfsetlinewidth{1.003750pt}%
\definecolor{currentstroke}{rgb}{0.121569,0.466667,0.705882}%
\pgfsetstrokecolor{currentstroke}%
\pgfsetstrokeopacity{0.318020}%
\pgfsetdash{}{0pt}%
\pgfpathmoveto{\pgfqpoint{1.619330in}{2.073948in}}%
\pgfpathcurveto{\pgfqpoint{1.627566in}{2.073948in}}{\pgfqpoint{1.635466in}{2.077220in}}{\pgfqpoint{1.641290in}{2.083044in}}%
\pgfpathcurveto{\pgfqpoint{1.647114in}{2.088868in}}{\pgfqpoint{1.650386in}{2.096768in}}{\pgfqpoint{1.650386in}{2.105004in}}%
\pgfpathcurveto{\pgfqpoint{1.650386in}{2.113240in}}{\pgfqpoint{1.647114in}{2.121140in}}{\pgfqpoint{1.641290in}{2.126964in}}%
\pgfpathcurveto{\pgfqpoint{1.635466in}{2.132788in}}{\pgfqpoint{1.627566in}{2.136061in}}{\pgfqpoint{1.619330in}{2.136061in}}%
\pgfpathcurveto{\pgfqpoint{1.611093in}{2.136061in}}{\pgfqpoint{1.603193in}{2.132788in}}{\pgfqpoint{1.597369in}{2.126964in}}%
\pgfpathcurveto{\pgfqpoint{1.591546in}{2.121140in}}{\pgfqpoint{1.588273in}{2.113240in}}{\pgfqpoint{1.588273in}{2.105004in}}%
\pgfpathcurveto{\pgfqpoint{1.588273in}{2.096768in}}{\pgfqpoint{1.591546in}{2.088868in}}{\pgfqpoint{1.597369in}{2.083044in}}%
\pgfpathcurveto{\pgfqpoint{1.603193in}{2.077220in}}{\pgfqpoint{1.611093in}{2.073948in}}{\pgfqpoint{1.619330in}{2.073948in}}%
\pgfpathclose%
\pgfusepath{stroke,fill}%
\end{pgfscope}%
\begin{pgfscope}%
\pgfpathrectangle{\pgfqpoint{0.100000in}{0.212622in}}{\pgfqpoint{3.696000in}{3.696000in}}%
\pgfusepath{clip}%
\pgfsetbuttcap%
\pgfsetroundjoin%
\definecolor{currentfill}{rgb}{0.121569,0.466667,0.705882}%
\pgfsetfillcolor{currentfill}%
\pgfsetfillopacity{0.318402}%
\pgfsetlinewidth{1.003750pt}%
\definecolor{currentstroke}{rgb}{0.121569,0.466667,0.705882}%
\pgfsetstrokecolor{currentstroke}%
\pgfsetstrokeopacity{0.318402}%
\pgfsetdash{}{0pt}%
\pgfpathmoveto{\pgfqpoint{1.618953in}{2.073904in}}%
\pgfpathcurveto{\pgfqpoint{1.627189in}{2.073904in}}{\pgfqpoint{1.635089in}{2.077177in}}{\pgfqpoint{1.640913in}{2.083001in}}%
\pgfpathcurveto{\pgfqpoint{1.646737in}{2.088825in}}{\pgfqpoint{1.650009in}{2.096725in}}{\pgfqpoint{1.650009in}{2.104961in}}%
\pgfpathcurveto{\pgfqpoint{1.650009in}{2.113197in}}{\pgfqpoint{1.646737in}{2.121097in}}{\pgfqpoint{1.640913in}{2.126921in}}%
\pgfpathcurveto{\pgfqpoint{1.635089in}{2.132745in}}{\pgfqpoint{1.627189in}{2.136017in}}{\pgfqpoint{1.618953in}{2.136017in}}%
\pgfpathcurveto{\pgfqpoint{1.610716in}{2.136017in}}{\pgfqpoint{1.602816in}{2.132745in}}{\pgfqpoint{1.596992in}{2.126921in}}%
\pgfpathcurveto{\pgfqpoint{1.591169in}{2.121097in}}{\pgfqpoint{1.587896in}{2.113197in}}{\pgfqpoint{1.587896in}{2.104961in}}%
\pgfpathcurveto{\pgfqpoint{1.587896in}{2.096725in}}{\pgfqpoint{1.591169in}{2.088825in}}{\pgfqpoint{1.596992in}{2.083001in}}%
\pgfpathcurveto{\pgfqpoint{1.602816in}{2.077177in}}{\pgfqpoint{1.610716in}{2.073904in}}{\pgfqpoint{1.618953in}{2.073904in}}%
\pgfpathclose%
\pgfusepath{stroke,fill}%
\end{pgfscope}%
\begin{pgfscope}%
\pgfpathrectangle{\pgfqpoint{0.100000in}{0.212622in}}{\pgfqpoint{3.696000in}{3.696000in}}%
\pgfusepath{clip}%
\pgfsetbuttcap%
\pgfsetroundjoin%
\definecolor{currentfill}{rgb}{0.121569,0.466667,0.705882}%
\pgfsetfillcolor{currentfill}%
\pgfsetfillopacity{0.318792}%
\pgfsetlinewidth{1.003750pt}%
\definecolor{currentstroke}{rgb}{0.121569,0.466667,0.705882}%
\pgfsetstrokecolor{currentstroke}%
\pgfsetstrokeopacity{0.318792}%
\pgfsetdash{}{0pt}%
\pgfpathmoveto{\pgfqpoint{1.618159in}{2.071586in}}%
\pgfpathcurveto{\pgfqpoint{1.626396in}{2.071586in}}{\pgfqpoint{1.634296in}{2.074858in}}{\pgfqpoint{1.640120in}{2.080682in}}%
\pgfpathcurveto{\pgfqpoint{1.645944in}{2.086506in}}{\pgfqpoint{1.649216in}{2.094406in}}{\pgfqpoint{1.649216in}{2.102642in}}%
\pgfpathcurveto{\pgfqpoint{1.649216in}{2.110879in}}{\pgfqpoint{1.645944in}{2.118779in}}{\pgfqpoint{1.640120in}{2.124603in}}%
\pgfpathcurveto{\pgfqpoint{1.634296in}{2.130426in}}{\pgfqpoint{1.626396in}{2.133699in}}{\pgfqpoint{1.618159in}{2.133699in}}%
\pgfpathcurveto{\pgfqpoint{1.609923in}{2.133699in}}{\pgfqpoint{1.602023in}{2.130426in}}{\pgfqpoint{1.596199in}{2.124603in}}%
\pgfpathcurveto{\pgfqpoint{1.590375in}{2.118779in}}{\pgfqpoint{1.587103in}{2.110879in}}{\pgfqpoint{1.587103in}{2.102642in}}%
\pgfpathcurveto{\pgfqpoint{1.587103in}{2.094406in}}{\pgfqpoint{1.590375in}{2.086506in}}{\pgfqpoint{1.596199in}{2.080682in}}%
\pgfpathcurveto{\pgfqpoint{1.602023in}{2.074858in}}{\pgfqpoint{1.609923in}{2.071586in}}{\pgfqpoint{1.618159in}{2.071586in}}%
\pgfpathclose%
\pgfusepath{stroke,fill}%
\end{pgfscope}%
\begin{pgfscope}%
\pgfpathrectangle{\pgfqpoint{0.100000in}{0.212622in}}{\pgfqpoint{3.696000in}{3.696000in}}%
\pgfusepath{clip}%
\pgfsetbuttcap%
\pgfsetroundjoin%
\definecolor{currentfill}{rgb}{0.121569,0.466667,0.705882}%
\pgfsetfillcolor{currentfill}%
\pgfsetfillopacity{0.318880}%
\pgfsetlinewidth{1.003750pt}%
\definecolor{currentstroke}{rgb}{0.121569,0.466667,0.705882}%
\pgfsetstrokecolor{currentstroke}%
\pgfsetstrokeopacity{0.318880}%
\pgfsetdash{}{0pt}%
\pgfpathmoveto{\pgfqpoint{1.778618in}{2.124744in}}%
\pgfpathcurveto{\pgfqpoint{1.786854in}{2.124744in}}{\pgfqpoint{1.794754in}{2.128016in}}{\pgfqpoint{1.800578in}{2.133840in}}%
\pgfpathcurveto{\pgfqpoint{1.806402in}{2.139664in}}{\pgfqpoint{1.809674in}{2.147564in}}{\pgfqpoint{1.809674in}{2.155801in}}%
\pgfpathcurveto{\pgfqpoint{1.809674in}{2.164037in}}{\pgfqpoint{1.806402in}{2.171937in}}{\pgfqpoint{1.800578in}{2.177761in}}%
\pgfpathcurveto{\pgfqpoint{1.794754in}{2.183585in}}{\pgfqpoint{1.786854in}{2.186857in}}{\pgfqpoint{1.778618in}{2.186857in}}%
\pgfpathcurveto{\pgfqpoint{1.770382in}{2.186857in}}{\pgfqpoint{1.762482in}{2.183585in}}{\pgfqpoint{1.756658in}{2.177761in}}%
\pgfpathcurveto{\pgfqpoint{1.750834in}{2.171937in}}{\pgfqpoint{1.747561in}{2.164037in}}{\pgfqpoint{1.747561in}{2.155801in}}%
\pgfpathcurveto{\pgfqpoint{1.747561in}{2.147564in}}{\pgfqpoint{1.750834in}{2.139664in}}{\pgfqpoint{1.756658in}{2.133840in}}%
\pgfpathcurveto{\pgfqpoint{1.762482in}{2.128016in}}{\pgfqpoint{1.770382in}{2.124744in}}{\pgfqpoint{1.778618in}{2.124744in}}%
\pgfpathclose%
\pgfusepath{stroke,fill}%
\end{pgfscope}%
\begin{pgfscope}%
\pgfpathrectangle{\pgfqpoint{0.100000in}{0.212622in}}{\pgfqpoint{3.696000in}{3.696000in}}%
\pgfusepath{clip}%
\pgfsetbuttcap%
\pgfsetroundjoin%
\definecolor{currentfill}{rgb}{0.121569,0.466667,0.705882}%
\pgfsetfillcolor{currentfill}%
\pgfsetfillopacity{0.319698}%
\pgfsetlinewidth{1.003750pt}%
\definecolor{currentstroke}{rgb}{0.121569,0.466667,0.705882}%
\pgfsetstrokecolor{currentstroke}%
\pgfsetstrokeopacity{0.319698}%
\pgfsetdash{}{0pt}%
\pgfpathmoveto{\pgfqpoint{1.615026in}{2.070554in}}%
\pgfpathcurveto{\pgfqpoint{1.623262in}{2.070554in}}{\pgfqpoint{1.631162in}{2.073826in}}{\pgfqpoint{1.636986in}{2.079650in}}%
\pgfpathcurveto{\pgfqpoint{1.642810in}{2.085474in}}{\pgfqpoint{1.646082in}{2.093374in}}{\pgfqpoint{1.646082in}{2.101610in}}%
\pgfpathcurveto{\pgfqpoint{1.646082in}{2.109847in}}{\pgfqpoint{1.642810in}{2.117747in}}{\pgfqpoint{1.636986in}{2.123571in}}%
\pgfpathcurveto{\pgfqpoint{1.631162in}{2.129395in}}{\pgfqpoint{1.623262in}{2.132667in}}{\pgfqpoint{1.615026in}{2.132667in}}%
\pgfpathcurveto{\pgfqpoint{1.606790in}{2.132667in}}{\pgfqpoint{1.598890in}{2.129395in}}{\pgfqpoint{1.593066in}{2.123571in}}%
\pgfpathcurveto{\pgfqpoint{1.587242in}{2.117747in}}{\pgfqpoint{1.583969in}{2.109847in}}{\pgfqpoint{1.583969in}{2.101610in}}%
\pgfpathcurveto{\pgfqpoint{1.583969in}{2.093374in}}{\pgfqpoint{1.587242in}{2.085474in}}{\pgfqpoint{1.593066in}{2.079650in}}%
\pgfpathcurveto{\pgfqpoint{1.598890in}{2.073826in}}{\pgfqpoint{1.606790in}{2.070554in}}{\pgfqpoint{1.615026in}{2.070554in}}%
\pgfpathclose%
\pgfusepath{stroke,fill}%
\end{pgfscope}%
\begin{pgfscope}%
\pgfpathrectangle{\pgfqpoint{0.100000in}{0.212622in}}{\pgfqpoint{3.696000in}{3.696000in}}%
\pgfusepath{clip}%
\pgfsetbuttcap%
\pgfsetroundjoin%
\definecolor{currentfill}{rgb}{0.121569,0.466667,0.705882}%
\pgfsetfillcolor{currentfill}%
\pgfsetfillopacity{0.320382}%
\pgfsetlinewidth{1.003750pt}%
\definecolor{currentstroke}{rgb}{0.121569,0.466667,0.705882}%
\pgfsetstrokecolor{currentstroke}%
\pgfsetstrokeopacity{0.320382}%
\pgfsetdash{}{0pt}%
\pgfpathmoveto{\pgfqpoint{1.786867in}{2.118949in}}%
\pgfpathcurveto{\pgfqpoint{1.795103in}{2.118949in}}{\pgfqpoint{1.803003in}{2.122221in}}{\pgfqpoint{1.808827in}{2.128045in}}%
\pgfpathcurveto{\pgfqpoint{1.814651in}{2.133869in}}{\pgfqpoint{1.817923in}{2.141769in}}{\pgfqpoint{1.817923in}{2.150005in}}%
\pgfpathcurveto{\pgfqpoint{1.817923in}{2.158241in}}{\pgfqpoint{1.814651in}{2.166142in}}{\pgfqpoint{1.808827in}{2.171965in}}%
\pgfpathcurveto{\pgfqpoint{1.803003in}{2.177789in}}{\pgfqpoint{1.795103in}{2.181062in}}{\pgfqpoint{1.786867in}{2.181062in}}%
\pgfpathcurveto{\pgfqpoint{1.778631in}{2.181062in}}{\pgfqpoint{1.770731in}{2.177789in}}{\pgfqpoint{1.764907in}{2.171965in}}%
\pgfpathcurveto{\pgfqpoint{1.759083in}{2.166142in}}{\pgfqpoint{1.755810in}{2.158241in}}{\pgfqpoint{1.755810in}{2.150005in}}%
\pgfpathcurveto{\pgfqpoint{1.755810in}{2.141769in}}{\pgfqpoint{1.759083in}{2.133869in}}{\pgfqpoint{1.764907in}{2.128045in}}%
\pgfpathcurveto{\pgfqpoint{1.770731in}{2.122221in}}{\pgfqpoint{1.778631in}{2.118949in}}{\pgfqpoint{1.786867in}{2.118949in}}%
\pgfpathclose%
\pgfusepath{stroke,fill}%
\end{pgfscope}%
\begin{pgfscope}%
\pgfpathrectangle{\pgfqpoint{0.100000in}{0.212622in}}{\pgfqpoint{3.696000in}{3.696000in}}%
\pgfusepath{clip}%
\pgfsetbuttcap%
\pgfsetroundjoin%
\definecolor{currentfill}{rgb}{0.121569,0.466667,0.705882}%
\pgfsetfillcolor{currentfill}%
\pgfsetfillopacity{0.320449}%
\pgfsetlinewidth{1.003750pt}%
\definecolor{currentstroke}{rgb}{0.121569,0.466667,0.705882}%
\pgfsetstrokecolor{currentstroke}%
\pgfsetstrokeopacity{0.320449}%
\pgfsetdash{}{0pt}%
\pgfpathmoveto{\pgfqpoint{1.614170in}{2.069322in}}%
\pgfpathcurveto{\pgfqpoint{1.622406in}{2.069322in}}{\pgfqpoint{1.630306in}{2.072594in}}{\pgfqpoint{1.636130in}{2.078418in}}%
\pgfpathcurveto{\pgfqpoint{1.641954in}{2.084242in}}{\pgfqpoint{1.645227in}{2.092142in}}{\pgfqpoint{1.645227in}{2.100378in}}%
\pgfpathcurveto{\pgfqpoint{1.645227in}{2.108614in}}{\pgfqpoint{1.641954in}{2.116514in}}{\pgfqpoint{1.636130in}{2.122338in}}%
\pgfpathcurveto{\pgfqpoint{1.630306in}{2.128162in}}{\pgfqpoint{1.622406in}{2.131435in}}{\pgfqpoint{1.614170in}{2.131435in}}%
\pgfpathcurveto{\pgfqpoint{1.605934in}{2.131435in}}{\pgfqpoint{1.598034in}{2.128162in}}{\pgfqpoint{1.592210in}{2.122338in}}%
\pgfpathcurveto{\pgfqpoint{1.586386in}{2.116514in}}{\pgfqpoint{1.583114in}{2.108614in}}{\pgfqpoint{1.583114in}{2.100378in}}%
\pgfpathcurveto{\pgfqpoint{1.583114in}{2.092142in}}{\pgfqpoint{1.586386in}{2.084242in}}{\pgfqpoint{1.592210in}{2.078418in}}%
\pgfpathcurveto{\pgfqpoint{1.598034in}{2.072594in}}{\pgfqpoint{1.605934in}{2.069322in}}{\pgfqpoint{1.614170in}{2.069322in}}%
\pgfpathclose%
\pgfusepath{stroke,fill}%
\end{pgfscope}%
\begin{pgfscope}%
\pgfpathrectangle{\pgfqpoint{0.100000in}{0.212622in}}{\pgfqpoint{3.696000in}{3.696000in}}%
\pgfusepath{clip}%
\pgfsetbuttcap%
\pgfsetroundjoin%
\definecolor{currentfill}{rgb}{0.121569,0.466667,0.705882}%
\pgfsetfillcolor{currentfill}%
\pgfsetfillopacity{0.321166}%
\pgfsetlinewidth{1.003750pt}%
\definecolor{currentstroke}{rgb}{0.121569,0.466667,0.705882}%
\pgfsetstrokecolor{currentstroke}%
\pgfsetstrokeopacity{0.321166}%
\pgfsetdash{}{0pt}%
\pgfpathmoveto{\pgfqpoint{1.612288in}{2.062374in}}%
\pgfpathcurveto{\pgfqpoint{1.620524in}{2.062374in}}{\pgfqpoint{1.628424in}{2.065646in}}{\pgfqpoint{1.634248in}{2.071470in}}%
\pgfpathcurveto{\pgfqpoint{1.640072in}{2.077294in}}{\pgfqpoint{1.643345in}{2.085194in}}{\pgfqpoint{1.643345in}{2.093430in}}%
\pgfpathcurveto{\pgfqpoint{1.643345in}{2.101667in}}{\pgfqpoint{1.640072in}{2.109567in}}{\pgfqpoint{1.634248in}{2.115391in}}%
\pgfpathcurveto{\pgfqpoint{1.628424in}{2.121214in}}{\pgfqpoint{1.620524in}{2.124487in}}{\pgfqpoint{1.612288in}{2.124487in}}%
\pgfpathcurveto{\pgfqpoint{1.604052in}{2.124487in}}{\pgfqpoint{1.596152in}{2.121214in}}{\pgfqpoint{1.590328in}{2.115391in}}%
\pgfpathcurveto{\pgfqpoint{1.584504in}{2.109567in}}{\pgfqpoint{1.581232in}{2.101667in}}{\pgfqpoint{1.581232in}{2.093430in}}%
\pgfpathcurveto{\pgfqpoint{1.581232in}{2.085194in}}{\pgfqpoint{1.584504in}{2.077294in}}{\pgfqpoint{1.590328in}{2.071470in}}%
\pgfpathcurveto{\pgfqpoint{1.596152in}{2.065646in}}{\pgfqpoint{1.604052in}{2.062374in}}{\pgfqpoint{1.612288in}{2.062374in}}%
\pgfpathclose%
\pgfusepath{stroke,fill}%
\end{pgfscope}%
\begin{pgfscope}%
\pgfpathrectangle{\pgfqpoint{0.100000in}{0.212622in}}{\pgfqpoint{3.696000in}{3.696000in}}%
\pgfusepath{clip}%
\pgfsetbuttcap%
\pgfsetroundjoin%
\definecolor{currentfill}{rgb}{0.121569,0.466667,0.705882}%
\pgfsetfillcolor{currentfill}%
\pgfsetfillopacity{0.322104}%
\pgfsetlinewidth{1.003750pt}%
\definecolor{currentstroke}{rgb}{0.121569,0.466667,0.705882}%
\pgfsetstrokecolor{currentstroke}%
\pgfsetstrokeopacity{0.322104}%
\pgfsetdash{}{0pt}%
\pgfpathmoveto{\pgfqpoint{1.792110in}{2.124265in}}%
\pgfpathcurveto{\pgfqpoint{1.800346in}{2.124265in}}{\pgfqpoint{1.808246in}{2.127537in}}{\pgfqpoint{1.814070in}{2.133361in}}%
\pgfpathcurveto{\pgfqpoint{1.819894in}{2.139185in}}{\pgfqpoint{1.823166in}{2.147085in}}{\pgfqpoint{1.823166in}{2.155321in}}%
\pgfpathcurveto{\pgfqpoint{1.823166in}{2.163558in}}{\pgfqpoint{1.819894in}{2.171458in}}{\pgfqpoint{1.814070in}{2.177282in}}%
\pgfpathcurveto{\pgfqpoint{1.808246in}{2.183106in}}{\pgfqpoint{1.800346in}{2.186378in}}{\pgfqpoint{1.792110in}{2.186378in}}%
\pgfpathcurveto{\pgfqpoint{1.783874in}{2.186378in}}{\pgfqpoint{1.775974in}{2.183106in}}{\pgfqpoint{1.770150in}{2.177282in}}%
\pgfpathcurveto{\pgfqpoint{1.764326in}{2.171458in}}{\pgfqpoint{1.761053in}{2.163558in}}{\pgfqpoint{1.761053in}{2.155321in}}%
\pgfpathcurveto{\pgfqpoint{1.761053in}{2.147085in}}{\pgfqpoint{1.764326in}{2.139185in}}{\pgfqpoint{1.770150in}{2.133361in}}%
\pgfpathcurveto{\pgfqpoint{1.775974in}{2.127537in}}{\pgfqpoint{1.783874in}{2.124265in}}{\pgfqpoint{1.792110in}{2.124265in}}%
\pgfpathclose%
\pgfusepath{stroke,fill}%
\end{pgfscope}%
\begin{pgfscope}%
\pgfpathrectangle{\pgfqpoint{0.100000in}{0.212622in}}{\pgfqpoint{3.696000in}{3.696000in}}%
\pgfusepath{clip}%
\pgfsetbuttcap%
\pgfsetroundjoin%
\definecolor{currentfill}{rgb}{0.121569,0.466667,0.705882}%
\pgfsetfillcolor{currentfill}%
\pgfsetfillopacity{0.322268}%
\pgfsetlinewidth{1.003750pt}%
\definecolor{currentstroke}{rgb}{0.121569,0.466667,0.705882}%
\pgfsetstrokecolor{currentstroke}%
\pgfsetstrokeopacity{0.322268}%
\pgfsetdash{}{0pt}%
\pgfpathmoveto{\pgfqpoint{1.795077in}{2.121416in}}%
\pgfpathcurveto{\pgfqpoint{1.803313in}{2.121416in}}{\pgfqpoint{1.811213in}{2.124689in}}{\pgfqpoint{1.817037in}{2.130512in}}%
\pgfpathcurveto{\pgfqpoint{1.822861in}{2.136336in}}{\pgfqpoint{1.826133in}{2.144236in}}{\pgfqpoint{1.826133in}{2.152473in}}%
\pgfpathcurveto{\pgfqpoint{1.826133in}{2.160709in}}{\pgfqpoint{1.822861in}{2.168609in}}{\pgfqpoint{1.817037in}{2.174433in}}%
\pgfpathcurveto{\pgfqpoint{1.811213in}{2.180257in}}{\pgfqpoint{1.803313in}{2.183529in}}{\pgfqpoint{1.795077in}{2.183529in}}%
\pgfpathcurveto{\pgfqpoint{1.786840in}{2.183529in}}{\pgfqpoint{1.778940in}{2.180257in}}{\pgfqpoint{1.773117in}{2.174433in}}%
\pgfpathcurveto{\pgfqpoint{1.767293in}{2.168609in}}{\pgfqpoint{1.764020in}{2.160709in}}{\pgfqpoint{1.764020in}{2.152473in}}%
\pgfpathcurveto{\pgfqpoint{1.764020in}{2.144236in}}{\pgfqpoint{1.767293in}{2.136336in}}{\pgfqpoint{1.773117in}{2.130512in}}%
\pgfpathcurveto{\pgfqpoint{1.778940in}{2.124689in}}{\pgfqpoint{1.786840in}{2.121416in}}{\pgfqpoint{1.795077in}{2.121416in}}%
\pgfpathclose%
\pgfusepath{stroke,fill}%
\end{pgfscope}%
\begin{pgfscope}%
\pgfpathrectangle{\pgfqpoint{0.100000in}{0.212622in}}{\pgfqpoint{3.696000in}{3.696000in}}%
\pgfusepath{clip}%
\pgfsetbuttcap%
\pgfsetroundjoin%
\definecolor{currentfill}{rgb}{0.121569,0.466667,0.705882}%
\pgfsetfillcolor{currentfill}%
\pgfsetfillopacity{0.322503}%
\pgfsetlinewidth{1.003750pt}%
\definecolor{currentstroke}{rgb}{0.121569,0.466667,0.705882}%
\pgfsetstrokecolor{currentstroke}%
\pgfsetstrokeopacity{0.322503}%
\pgfsetdash{}{0pt}%
\pgfpathmoveto{\pgfqpoint{1.608385in}{2.062969in}}%
\pgfpathcurveto{\pgfqpoint{1.616622in}{2.062969in}}{\pgfqpoint{1.624522in}{2.066242in}}{\pgfqpoint{1.630346in}{2.072066in}}%
\pgfpathcurveto{\pgfqpoint{1.636170in}{2.077890in}}{\pgfqpoint{1.639442in}{2.085790in}}{\pgfqpoint{1.639442in}{2.094026in}}%
\pgfpathcurveto{\pgfqpoint{1.639442in}{2.102262in}}{\pgfqpoint{1.636170in}{2.110162in}}{\pgfqpoint{1.630346in}{2.115986in}}%
\pgfpathcurveto{\pgfqpoint{1.624522in}{2.121810in}}{\pgfqpoint{1.616622in}{2.125082in}}{\pgfqpoint{1.608385in}{2.125082in}}%
\pgfpathcurveto{\pgfqpoint{1.600149in}{2.125082in}}{\pgfqpoint{1.592249in}{2.121810in}}{\pgfqpoint{1.586425in}{2.115986in}}%
\pgfpathcurveto{\pgfqpoint{1.580601in}{2.110162in}}{\pgfqpoint{1.577329in}{2.102262in}}{\pgfqpoint{1.577329in}{2.094026in}}%
\pgfpathcurveto{\pgfqpoint{1.577329in}{2.085790in}}{\pgfqpoint{1.580601in}{2.077890in}}{\pgfqpoint{1.586425in}{2.072066in}}%
\pgfpathcurveto{\pgfqpoint{1.592249in}{2.066242in}}{\pgfqpoint{1.600149in}{2.062969in}}{\pgfqpoint{1.608385in}{2.062969in}}%
\pgfpathclose%
\pgfusepath{stroke,fill}%
\end{pgfscope}%
\begin{pgfscope}%
\pgfpathrectangle{\pgfqpoint{0.100000in}{0.212622in}}{\pgfqpoint{3.696000in}{3.696000in}}%
\pgfusepath{clip}%
\pgfsetbuttcap%
\pgfsetroundjoin%
\definecolor{currentfill}{rgb}{0.121569,0.466667,0.705882}%
\pgfsetfillcolor{currentfill}%
\pgfsetfillopacity{0.323149}%
\pgfsetlinewidth{1.003750pt}%
\definecolor{currentstroke}{rgb}{0.121569,0.466667,0.705882}%
\pgfsetstrokecolor{currentstroke}%
\pgfsetstrokeopacity{0.323149}%
\pgfsetdash{}{0pt}%
\pgfpathmoveto{\pgfqpoint{1.799592in}{2.125256in}}%
\pgfpathcurveto{\pgfqpoint{1.807829in}{2.125256in}}{\pgfqpoint{1.815729in}{2.128529in}}{\pgfqpoint{1.821553in}{2.134353in}}%
\pgfpathcurveto{\pgfqpoint{1.827376in}{2.140177in}}{\pgfqpoint{1.830649in}{2.148077in}}{\pgfqpoint{1.830649in}{2.156313in}}%
\pgfpathcurveto{\pgfqpoint{1.830649in}{2.164549in}}{\pgfqpoint{1.827376in}{2.172449in}}{\pgfqpoint{1.821553in}{2.178273in}}%
\pgfpathcurveto{\pgfqpoint{1.815729in}{2.184097in}}{\pgfqpoint{1.807829in}{2.187369in}}{\pgfqpoint{1.799592in}{2.187369in}}%
\pgfpathcurveto{\pgfqpoint{1.791356in}{2.187369in}}{\pgfqpoint{1.783456in}{2.184097in}}{\pgfqpoint{1.777632in}{2.178273in}}%
\pgfpathcurveto{\pgfqpoint{1.771808in}{2.172449in}}{\pgfqpoint{1.768536in}{2.164549in}}{\pgfqpoint{1.768536in}{2.156313in}}%
\pgfpathcurveto{\pgfqpoint{1.768536in}{2.148077in}}{\pgfqpoint{1.771808in}{2.140177in}}{\pgfqpoint{1.777632in}{2.134353in}}%
\pgfpathcurveto{\pgfqpoint{1.783456in}{2.128529in}}{\pgfqpoint{1.791356in}{2.125256in}}{\pgfqpoint{1.799592in}{2.125256in}}%
\pgfpathclose%
\pgfusepath{stroke,fill}%
\end{pgfscope}%
\begin{pgfscope}%
\pgfpathrectangle{\pgfqpoint{0.100000in}{0.212622in}}{\pgfqpoint{3.696000in}{3.696000in}}%
\pgfusepath{clip}%
\pgfsetbuttcap%
\pgfsetroundjoin%
\definecolor{currentfill}{rgb}{0.121569,0.466667,0.705882}%
\pgfsetfillcolor{currentfill}%
\pgfsetfillopacity{0.323456}%
\pgfsetlinewidth{1.003750pt}%
\definecolor{currentstroke}{rgb}{0.121569,0.466667,0.705882}%
\pgfsetstrokecolor{currentstroke}%
\pgfsetstrokeopacity{0.323456}%
\pgfsetdash{}{0pt}%
\pgfpathmoveto{\pgfqpoint{1.805425in}{2.121298in}}%
\pgfpathcurveto{\pgfqpoint{1.813661in}{2.121298in}}{\pgfqpoint{1.821562in}{2.124570in}}{\pgfqpoint{1.827385in}{2.130394in}}%
\pgfpathcurveto{\pgfqpoint{1.833209in}{2.136218in}}{\pgfqpoint{1.836482in}{2.144118in}}{\pgfqpoint{1.836482in}{2.152354in}}%
\pgfpathcurveto{\pgfqpoint{1.836482in}{2.160591in}}{\pgfqpoint{1.833209in}{2.168491in}}{\pgfqpoint{1.827385in}{2.174315in}}%
\pgfpathcurveto{\pgfqpoint{1.821562in}{2.180139in}}{\pgfqpoint{1.813661in}{2.183411in}}{\pgfqpoint{1.805425in}{2.183411in}}%
\pgfpathcurveto{\pgfqpoint{1.797189in}{2.183411in}}{\pgfqpoint{1.789289in}{2.180139in}}{\pgfqpoint{1.783465in}{2.174315in}}%
\pgfpathcurveto{\pgfqpoint{1.777641in}{2.168491in}}{\pgfqpoint{1.774369in}{2.160591in}}{\pgfqpoint{1.774369in}{2.152354in}}%
\pgfpathcurveto{\pgfqpoint{1.774369in}{2.144118in}}{\pgfqpoint{1.777641in}{2.136218in}}{\pgfqpoint{1.783465in}{2.130394in}}%
\pgfpathcurveto{\pgfqpoint{1.789289in}{2.124570in}}{\pgfqpoint{1.797189in}{2.121298in}}{\pgfqpoint{1.805425in}{2.121298in}}%
\pgfpathclose%
\pgfusepath{stroke,fill}%
\end{pgfscope}%
\begin{pgfscope}%
\pgfpathrectangle{\pgfqpoint{0.100000in}{0.212622in}}{\pgfqpoint{3.696000in}{3.696000in}}%
\pgfusepath{clip}%
\pgfsetbuttcap%
\pgfsetroundjoin%
\definecolor{currentfill}{rgb}{0.121569,0.466667,0.705882}%
\pgfsetfillcolor{currentfill}%
\pgfsetfillopacity{0.323562}%
\pgfsetlinewidth{1.003750pt}%
\definecolor{currentstroke}{rgb}{0.121569,0.466667,0.705882}%
\pgfsetstrokecolor{currentstroke}%
\pgfsetstrokeopacity{0.323562}%
\pgfsetdash{}{0pt}%
\pgfpathmoveto{\pgfqpoint{1.607755in}{2.062961in}}%
\pgfpathcurveto{\pgfqpoint{1.615991in}{2.062961in}}{\pgfqpoint{1.623891in}{2.066234in}}{\pgfqpoint{1.629715in}{2.072058in}}%
\pgfpathcurveto{\pgfqpoint{1.635539in}{2.077881in}}{\pgfqpoint{1.638811in}{2.085782in}}{\pgfqpoint{1.638811in}{2.094018in}}%
\pgfpathcurveto{\pgfqpoint{1.638811in}{2.102254in}}{\pgfqpoint{1.635539in}{2.110154in}}{\pgfqpoint{1.629715in}{2.115978in}}%
\pgfpathcurveto{\pgfqpoint{1.623891in}{2.121802in}}{\pgfqpoint{1.615991in}{2.125074in}}{\pgfqpoint{1.607755in}{2.125074in}}%
\pgfpathcurveto{\pgfqpoint{1.599518in}{2.125074in}}{\pgfqpoint{1.591618in}{2.121802in}}{\pgfqpoint{1.585794in}{2.115978in}}%
\pgfpathcurveto{\pgfqpoint{1.579970in}{2.110154in}}{\pgfqpoint{1.576698in}{2.102254in}}{\pgfqpoint{1.576698in}{2.094018in}}%
\pgfpathcurveto{\pgfqpoint{1.576698in}{2.085782in}}{\pgfqpoint{1.579970in}{2.077881in}}{\pgfqpoint{1.585794in}{2.072058in}}%
\pgfpathcurveto{\pgfqpoint{1.591618in}{2.066234in}}{\pgfqpoint{1.599518in}{2.062961in}}{\pgfqpoint{1.607755in}{2.062961in}}%
\pgfpathclose%
\pgfusepath{stroke,fill}%
\end{pgfscope}%
\begin{pgfscope}%
\pgfpathrectangle{\pgfqpoint{0.100000in}{0.212622in}}{\pgfqpoint{3.696000in}{3.696000in}}%
\pgfusepath{clip}%
\pgfsetbuttcap%
\pgfsetroundjoin%
\definecolor{currentfill}{rgb}{0.121569,0.466667,0.705882}%
\pgfsetfillcolor{currentfill}%
\pgfsetfillopacity{0.324107}%
\pgfsetlinewidth{1.003750pt}%
\definecolor{currentstroke}{rgb}{0.121569,0.466667,0.705882}%
\pgfsetstrokecolor{currentstroke}%
\pgfsetstrokeopacity{0.324107}%
\pgfsetdash{}{0pt}%
\pgfpathmoveto{\pgfqpoint{1.808661in}{2.122929in}}%
\pgfpathcurveto{\pgfqpoint{1.816897in}{2.122929in}}{\pgfqpoint{1.824797in}{2.126202in}}{\pgfqpoint{1.830621in}{2.132025in}}%
\pgfpathcurveto{\pgfqpoint{1.836445in}{2.137849in}}{\pgfqpoint{1.839717in}{2.145749in}}{\pgfqpoint{1.839717in}{2.153986in}}%
\pgfpathcurveto{\pgfqpoint{1.839717in}{2.162222in}}{\pgfqpoint{1.836445in}{2.170122in}}{\pgfqpoint{1.830621in}{2.175946in}}%
\pgfpathcurveto{\pgfqpoint{1.824797in}{2.181770in}}{\pgfqpoint{1.816897in}{2.185042in}}{\pgfqpoint{1.808661in}{2.185042in}}%
\pgfpathcurveto{\pgfqpoint{1.800424in}{2.185042in}}{\pgfqpoint{1.792524in}{2.181770in}}{\pgfqpoint{1.786700in}{2.175946in}}%
\pgfpathcurveto{\pgfqpoint{1.780876in}{2.170122in}}{\pgfqpoint{1.777604in}{2.162222in}}{\pgfqpoint{1.777604in}{2.153986in}}%
\pgfpathcurveto{\pgfqpoint{1.777604in}{2.145749in}}{\pgfqpoint{1.780876in}{2.137849in}}{\pgfqpoint{1.786700in}{2.132025in}}%
\pgfpathcurveto{\pgfqpoint{1.792524in}{2.126202in}}{\pgfqpoint{1.800424in}{2.122929in}}{\pgfqpoint{1.808661in}{2.122929in}}%
\pgfpathclose%
\pgfusepath{stroke,fill}%
\end{pgfscope}%
\begin{pgfscope}%
\pgfpathrectangle{\pgfqpoint{0.100000in}{0.212622in}}{\pgfqpoint{3.696000in}{3.696000in}}%
\pgfusepath{clip}%
\pgfsetbuttcap%
\pgfsetroundjoin%
\definecolor{currentfill}{rgb}{0.121569,0.466667,0.705882}%
\pgfsetfillcolor{currentfill}%
\pgfsetfillopacity{0.324392}%
\pgfsetlinewidth{1.003750pt}%
\definecolor{currentstroke}{rgb}{0.121569,0.466667,0.705882}%
\pgfsetstrokecolor{currentstroke}%
\pgfsetstrokeopacity{0.324392}%
\pgfsetdash{}{0pt}%
\pgfpathmoveto{\pgfqpoint{1.604203in}{2.056252in}}%
\pgfpathcurveto{\pgfqpoint{1.612439in}{2.056252in}}{\pgfqpoint{1.620339in}{2.059524in}}{\pgfqpoint{1.626163in}{2.065348in}}%
\pgfpathcurveto{\pgfqpoint{1.631987in}{2.071172in}}{\pgfqpoint{1.635259in}{2.079072in}}{\pgfqpoint{1.635259in}{2.087308in}}%
\pgfpathcurveto{\pgfqpoint{1.635259in}{2.095544in}}{\pgfqpoint{1.631987in}{2.103444in}}{\pgfqpoint{1.626163in}{2.109268in}}%
\pgfpathcurveto{\pgfqpoint{1.620339in}{2.115092in}}{\pgfqpoint{1.612439in}{2.118365in}}{\pgfqpoint{1.604203in}{2.118365in}}%
\pgfpathcurveto{\pgfqpoint{1.595967in}{2.118365in}}{\pgfqpoint{1.588067in}{2.115092in}}{\pgfqpoint{1.582243in}{2.109268in}}%
\pgfpathcurveto{\pgfqpoint{1.576419in}{2.103444in}}{\pgfqpoint{1.573146in}{2.095544in}}{\pgfqpoint{1.573146in}{2.087308in}}%
\pgfpathcurveto{\pgfqpoint{1.573146in}{2.079072in}}{\pgfqpoint{1.576419in}{2.071172in}}{\pgfqpoint{1.582243in}{2.065348in}}%
\pgfpathcurveto{\pgfqpoint{1.588067in}{2.059524in}}{\pgfqpoint{1.595967in}{2.056252in}}{\pgfqpoint{1.604203in}{2.056252in}}%
\pgfpathclose%
\pgfusepath{stroke,fill}%
\end{pgfscope}%
\begin{pgfscope}%
\pgfpathrectangle{\pgfqpoint{0.100000in}{0.212622in}}{\pgfqpoint{3.696000in}{3.696000in}}%
\pgfusepath{clip}%
\pgfsetbuttcap%
\pgfsetroundjoin%
\definecolor{currentfill}{rgb}{0.121569,0.466667,0.705882}%
\pgfsetfillcolor{currentfill}%
\pgfsetfillopacity{0.324514}%
\pgfsetlinewidth{1.003750pt}%
\definecolor{currentstroke}{rgb}{0.121569,0.466667,0.705882}%
\pgfsetstrokecolor{currentstroke}%
\pgfsetstrokeopacity{0.324514}%
\pgfsetdash{}{0pt}%
\pgfpathmoveto{\pgfqpoint{1.811852in}{2.120661in}}%
\pgfpathcurveto{\pgfqpoint{1.820088in}{2.120661in}}{\pgfqpoint{1.827988in}{2.123933in}}{\pgfqpoint{1.833812in}{2.129757in}}%
\pgfpathcurveto{\pgfqpoint{1.839636in}{2.135581in}}{\pgfqpoint{1.842909in}{2.143481in}}{\pgfqpoint{1.842909in}{2.151718in}}%
\pgfpathcurveto{\pgfqpoint{1.842909in}{2.159954in}}{\pgfqpoint{1.839636in}{2.167854in}}{\pgfqpoint{1.833812in}{2.173678in}}%
\pgfpathcurveto{\pgfqpoint{1.827988in}{2.179502in}}{\pgfqpoint{1.820088in}{2.182774in}}{\pgfqpoint{1.811852in}{2.182774in}}%
\pgfpathcurveto{\pgfqpoint{1.803616in}{2.182774in}}{\pgfqpoint{1.795716in}{2.179502in}}{\pgfqpoint{1.789892in}{2.173678in}}%
\pgfpathcurveto{\pgfqpoint{1.784068in}{2.167854in}}{\pgfqpoint{1.780796in}{2.159954in}}{\pgfqpoint{1.780796in}{2.151718in}}%
\pgfpathcurveto{\pgfqpoint{1.780796in}{2.143481in}}{\pgfqpoint{1.784068in}{2.135581in}}{\pgfqpoint{1.789892in}{2.129757in}}%
\pgfpathcurveto{\pgfqpoint{1.795716in}{2.123933in}}{\pgfqpoint{1.803616in}{2.120661in}}{\pgfqpoint{1.811852in}{2.120661in}}%
\pgfpathclose%
\pgfusepath{stroke,fill}%
\end{pgfscope}%
\begin{pgfscope}%
\pgfpathrectangle{\pgfqpoint{0.100000in}{0.212622in}}{\pgfqpoint{3.696000in}{3.696000in}}%
\pgfusepath{clip}%
\pgfsetbuttcap%
\pgfsetroundjoin%
\definecolor{currentfill}{rgb}{0.121569,0.466667,0.705882}%
\pgfsetfillcolor{currentfill}%
\pgfsetfillopacity{0.325745}%
\pgfsetlinewidth{1.003750pt}%
\definecolor{currentstroke}{rgb}{0.121569,0.466667,0.705882}%
\pgfsetstrokecolor{currentstroke}%
\pgfsetstrokeopacity{0.325745}%
\pgfsetdash{}{0pt}%
\pgfpathmoveto{\pgfqpoint{1.600859in}{2.054170in}}%
\pgfpathcurveto{\pgfqpoint{1.609096in}{2.054170in}}{\pgfqpoint{1.616996in}{2.057442in}}{\pgfqpoint{1.622820in}{2.063266in}}%
\pgfpathcurveto{\pgfqpoint{1.628644in}{2.069090in}}{\pgfqpoint{1.631916in}{2.076990in}}{\pgfqpoint{1.631916in}{2.085226in}}%
\pgfpathcurveto{\pgfqpoint{1.631916in}{2.093463in}}{\pgfqpoint{1.628644in}{2.101363in}}{\pgfqpoint{1.622820in}{2.107187in}}%
\pgfpathcurveto{\pgfqpoint{1.616996in}{2.113011in}}{\pgfqpoint{1.609096in}{2.116283in}}{\pgfqpoint{1.600859in}{2.116283in}}%
\pgfpathcurveto{\pgfqpoint{1.592623in}{2.116283in}}{\pgfqpoint{1.584723in}{2.113011in}}{\pgfqpoint{1.578899in}{2.107187in}}%
\pgfpathcurveto{\pgfqpoint{1.573075in}{2.101363in}}{\pgfqpoint{1.569803in}{2.093463in}}{\pgfqpoint{1.569803in}{2.085226in}}%
\pgfpathcurveto{\pgfqpoint{1.569803in}{2.076990in}}{\pgfqpoint{1.573075in}{2.069090in}}{\pgfqpoint{1.578899in}{2.063266in}}%
\pgfpathcurveto{\pgfqpoint{1.584723in}{2.057442in}}{\pgfqpoint{1.592623in}{2.054170in}}{\pgfqpoint{1.600859in}{2.054170in}}%
\pgfpathclose%
\pgfusepath{stroke,fill}%
\end{pgfscope}%
\begin{pgfscope}%
\pgfpathrectangle{\pgfqpoint{0.100000in}{0.212622in}}{\pgfqpoint{3.696000in}{3.696000in}}%
\pgfusepath{clip}%
\pgfsetbuttcap%
\pgfsetroundjoin%
\definecolor{currentfill}{rgb}{0.121569,0.466667,0.705882}%
\pgfsetfillcolor{currentfill}%
\pgfsetfillopacity{0.325871}%
\pgfsetlinewidth{1.003750pt}%
\definecolor{currentstroke}{rgb}{0.121569,0.466667,0.705882}%
\pgfsetstrokecolor{currentstroke}%
\pgfsetstrokeopacity{0.325871}%
\pgfsetdash{}{0pt}%
\pgfpathmoveto{\pgfqpoint{1.816246in}{2.124848in}}%
\pgfpathcurveto{\pgfqpoint{1.824483in}{2.124848in}}{\pgfqpoint{1.832383in}{2.128120in}}{\pgfqpoint{1.838207in}{2.133944in}}%
\pgfpathcurveto{\pgfqpoint{1.844031in}{2.139768in}}{\pgfqpoint{1.847303in}{2.147668in}}{\pgfqpoint{1.847303in}{2.155904in}}%
\pgfpathcurveto{\pgfqpoint{1.847303in}{2.164141in}}{\pgfqpoint{1.844031in}{2.172041in}}{\pgfqpoint{1.838207in}{2.177864in}}%
\pgfpathcurveto{\pgfqpoint{1.832383in}{2.183688in}}{\pgfqpoint{1.824483in}{2.186961in}}{\pgfqpoint{1.816246in}{2.186961in}}%
\pgfpathcurveto{\pgfqpoint{1.808010in}{2.186961in}}{\pgfqpoint{1.800110in}{2.183688in}}{\pgfqpoint{1.794286in}{2.177864in}}%
\pgfpathcurveto{\pgfqpoint{1.788462in}{2.172041in}}{\pgfqpoint{1.785190in}{2.164141in}}{\pgfqpoint{1.785190in}{2.155904in}}%
\pgfpathcurveto{\pgfqpoint{1.785190in}{2.147668in}}{\pgfqpoint{1.788462in}{2.139768in}}{\pgfqpoint{1.794286in}{2.133944in}}%
\pgfpathcurveto{\pgfqpoint{1.800110in}{2.128120in}}{\pgfqpoint{1.808010in}{2.124848in}}{\pgfqpoint{1.816246in}{2.124848in}}%
\pgfpathclose%
\pgfusepath{stroke,fill}%
\end{pgfscope}%
\begin{pgfscope}%
\pgfpathrectangle{\pgfqpoint{0.100000in}{0.212622in}}{\pgfqpoint{3.696000in}{3.696000in}}%
\pgfusepath{clip}%
\pgfsetbuttcap%
\pgfsetroundjoin%
\definecolor{currentfill}{rgb}{0.121569,0.466667,0.705882}%
\pgfsetfillcolor{currentfill}%
\pgfsetfillopacity{0.326724}%
\pgfsetlinewidth{1.003750pt}%
\definecolor{currentstroke}{rgb}{0.121569,0.466667,0.705882}%
\pgfsetstrokecolor{currentstroke}%
\pgfsetstrokeopacity{0.326724}%
\pgfsetdash{}{0pt}%
\pgfpathmoveto{\pgfqpoint{1.821432in}{2.122339in}}%
\pgfpathcurveto{\pgfqpoint{1.829669in}{2.122339in}}{\pgfqpoint{1.837569in}{2.125611in}}{\pgfqpoint{1.843393in}{2.131435in}}%
\pgfpathcurveto{\pgfqpoint{1.849217in}{2.137259in}}{\pgfqpoint{1.852489in}{2.145159in}}{\pgfqpoint{1.852489in}{2.153395in}}%
\pgfpathcurveto{\pgfqpoint{1.852489in}{2.161631in}}{\pgfqpoint{1.849217in}{2.169532in}}{\pgfqpoint{1.843393in}{2.175355in}}%
\pgfpathcurveto{\pgfqpoint{1.837569in}{2.181179in}}{\pgfqpoint{1.829669in}{2.184452in}}{\pgfqpoint{1.821432in}{2.184452in}}%
\pgfpathcurveto{\pgfqpoint{1.813196in}{2.184452in}}{\pgfqpoint{1.805296in}{2.181179in}}{\pgfqpoint{1.799472in}{2.175355in}}%
\pgfpathcurveto{\pgfqpoint{1.793648in}{2.169532in}}{\pgfqpoint{1.790376in}{2.161631in}}{\pgfqpoint{1.790376in}{2.153395in}}%
\pgfpathcurveto{\pgfqpoint{1.790376in}{2.145159in}}{\pgfqpoint{1.793648in}{2.137259in}}{\pgfqpoint{1.799472in}{2.131435in}}%
\pgfpathcurveto{\pgfqpoint{1.805296in}{2.125611in}}{\pgfqpoint{1.813196in}{2.122339in}}{\pgfqpoint{1.821432in}{2.122339in}}%
\pgfpathclose%
\pgfusepath{stroke,fill}%
\end{pgfscope}%
\begin{pgfscope}%
\pgfpathrectangle{\pgfqpoint{0.100000in}{0.212622in}}{\pgfqpoint{3.696000in}{3.696000in}}%
\pgfusepath{clip}%
\pgfsetbuttcap%
\pgfsetroundjoin%
\definecolor{currentfill}{rgb}{0.121569,0.466667,0.705882}%
\pgfsetfillcolor{currentfill}%
\pgfsetfillopacity{0.326792}%
\pgfsetlinewidth{1.003750pt}%
\definecolor{currentstroke}{rgb}{0.121569,0.466667,0.705882}%
\pgfsetstrokecolor{currentstroke}%
\pgfsetstrokeopacity{0.326792}%
\pgfsetdash{}{0pt}%
\pgfpathmoveto{\pgfqpoint{1.598651in}{2.050621in}}%
\pgfpathcurveto{\pgfqpoint{1.606888in}{2.050621in}}{\pgfqpoint{1.614788in}{2.053894in}}{\pgfqpoint{1.620612in}{2.059718in}}%
\pgfpathcurveto{\pgfqpoint{1.626435in}{2.065541in}}{\pgfqpoint{1.629708in}{2.073441in}}{\pgfqpoint{1.629708in}{2.081678in}}%
\pgfpathcurveto{\pgfqpoint{1.629708in}{2.089914in}}{\pgfqpoint{1.626435in}{2.097814in}}{\pgfqpoint{1.620612in}{2.103638in}}%
\pgfpathcurveto{\pgfqpoint{1.614788in}{2.109462in}}{\pgfqpoint{1.606888in}{2.112734in}}{\pgfqpoint{1.598651in}{2.112734in}}%
\pgfpathcurveto{\pgfqpoint{1.590415in}{2.112734in}}{\pgfqpoint{1.582515in}{2.109462in}}{\pgfqpoint{1.576691in}{2.103638in}}%
\pgfpathcurveto{\pgfqpoint{1.570867in}{2.097814in}}{\pgfqpoint{1.567595in}{2.089914in}}{\pgfqpoint{1.567595in}{2.081678in}}%
\pgfpathcurveto{\pgfqpoint{1.567595in}{2.073441in}}{\pgfqpoint{1.570867in}{2.065541in}}{\pgfqpoint{1.576691in}{2.059718in}}%
\pgfpathcurveto{\pgfqpoint{1.582515in}{2.053894in}}{\pgfqpoint{1.590415in}{2.050621in}}{\pgfqpoint{1.598651in}{2.050621in}}%
\pgfpathclose%
\pgfusepath{stroke,fill}%
\end{pgfscope}%
\begin{pgfscope}%
\pgfpathrectangle{\pgfqpoint{0.100000in}{0.212622in}}{\pgfqpoint{3.696000in}{3.696000in}}%
\pgfusepath{clip}%
\pgfsetbuttcap%
\pgfsetroundjoin%
\definecolor{currentfill}{rgb}{0.121569,0.466667,0.705882}%
\pgfsetfillcolor{currentfill}%
\pgfsetfillopacity{0.327532}%
\pgfsetlinewidth{1.003750pt}%
\definecolor{currentstroke}{rgb}{0.121569,0.466667,0.705882}%
\pgfsetstrokecolor{currentstroke}%
\pgfsetstrokeopacity{0.327532}%
\pgfsetdash{}{0pt}%
\pgfpathmoveto{\pgfqpoint{1.596129in}{2.046991in}}%
\pgfpathcurveto{\pgfqpoint{1.604366in}{2.046991in}}{\pgfqpoint{1.612266in}{2.050263in}}{\pgfqpoint{1.618090in}{2.056087in}}%
\pgfpathcurveto{\pgfqpoint{1.623914in}{2.061911in}}{\pgfqpoint{1.627186in}{2.069811in}}{\pgfqpoint{1.627186in}{2.078047in}}%
\pgfpathcurveto{\pgfqpoint{1.627186in}{2.086284in}}{\pgfqpoint{1.623914in}{2.094184in}}{\pgfqpoint{1.618090in}{2.100008in}}%
\pgfpathcurveto{\pgfqpoint{1.612266in}{2.105832in}}{\pgfqpoint{1.604366in}{2.109104in}}{\pgfqpoint{1.596129in}{2.109104in}}%
\pgfpathcurveto{\pgfqpoint{1.587893in}{2.109104in}}{\pgfqpoint{1.579993in}{2.105832in}}{\pgfqpoint{1.574169in}{2.100008in}}%
\pgfpathcurveto{\pgfqpoint{1.568345in}{2.094184in}}{\pgfqpoint{1.565073in}{2.086284in}}{\pgfqpoint{1.565073in}{2.078047in}}%
\pgfpathcurveto{\pgfqpoint{1.565073in}{2.069811in}}{\pgfqpoint{1.568345in}{2.061911in}}{\pgfqpoint{1.574169in}{2.056087in}}%
\pgfpathcurveto{\pgfqpoint{1.579993in}{2.050263in}}{\pgfqpoint{1.587893in}{2.046991in}}{\pgfqpoint{1.596129in}{2.046991in}}%
\pgfpathclose%
\pgfusepath{stroke,fill}%
\end{pgfscope}%
\begin{pgfscope}%
\pgfpathrectangle{\pgfqpoint{0.100000in}{0.212622in}}{\pgfqpoint{3.696000in}{3.696000in}}%
\pgfusepath{clip}%
\pgfsetbuttcap%
\pgfsetroundjoin%
\definecolor{currentfill}{rgb}{0.121569,0.466667,0.705882}%
\pgfsetfillcolor{currentfill}%
\pgfsetfillopacity{0.328082}%
\pgfsetlinewidth{1.003750pt}%
\definecolor{currentstroke}{rgb}{0.121569,0.466667,0.705882}%
\pgfsetstrokecolor{currentstroke}%
\pgfsetstrokeopacity{0.328082}%
\pgfsetdash{}{0pt}%
\pgfpathmoveto{\pgfqpoint{1.590621in}{2.035626in}}%
\pgfpathcurveto{\pgfqpoint{1.598858in}{2.035626in}}{\pgfqpoint{1.606758in}{2.038898in}}{\pgfqpoint{1.612582in}{2.044722in}}%
\pgfpathcurveto{\pgfqpoint{1.618406in}{2.050546in}}{\pgfqpoint{1.621678in}{2.058446in}}{\pgfqpoint{1.621678in}{2.066682in}}%
\pgfpathcurveto{\pgfqpoint{1.621678in}{2.074919in}}{\pgfqpoint{1.618406in}{2.082819in}}{\pgfqpoint{1.612582in}{2.088643in}}%
\pgfpathcurveto{\pgfqpoint{1.606758in}{2.094466in}}{\pgfqpoint{1.598858in}{2.097739in}}{\pgfqpoint{1.590621in}{2.097739in}}%
\pgfpathcurveto{\pgfqpoint{1.582385in}{2.097739in}}{\pgfqpoint{1.574485in}{2.094466in}}{\pgfqpoint{1.568661in}{2.088643in}}%
\pgfpathcurveto{\pgfqpoint{1.562837in}{2.082819in}}{\pgfqpoint{1.559565in}{2.074919in}}{\pgfqpoint{1.559565in}{2.066682in}}%
\pgfpathcurveto{\pgfqpoint{1.559565in}{2.058446in}}{\pgfqpoint{1.562837in}{2.050546in}}{\pgfqpoint{1.568661in}{2.044722in}}%
\pgfpathcurveto{\pgfqpoint{1.574485in}{2.038898in}}{\pgfqpoint{1.582385in}{2.035626in}}{\pgfqpoint{1.590621in}{2.035626in}}%
\pgfpathclose%
\pgfusepath{stroke,fill}%
\end{pgfscope}%
\begin{pgfscope}%
\pgfpathrectangle{\pgfqpoint{0.100000in}{0.212622in}}{\pgfqpoint{3.696000in}{3.696000in}}%
\pgfusepath{clip}%
\pgfsetbuttcap%
\pgfsetroundjoin%
\definecolor{currentfill}{rgb}{0.121569,0.466667,0.705882}%
\pgfsetfillcolor{currentfill}%
\pgfsetfillopacity{0.328416}%
\pgfsetlinewidth{1.003750pt}%
\definecolor{currentstroke}{rgb}{0.121569,0.466667,0.705882}%
\pgfsetstrokecolor{currentstroke}%
\pgfsetstrokeopacity{0.328416}%
\pgfsetdash{}{0pt}%
\pgfpathmoveto{\pgfqpoint{1.827688in}{2.127399in}}%
\pgfpathcurveto{\pgfqpoint{1.835925in}{2.127399in}}{\pgfqpoint{1.843825in}{2.130672in}}{\pgfqpoint{1.849649in}{2.136496in}}%
\pgfpathcurveto{\pgfqpoint{1.855473in}{2.142320in}}{\pgfqpoint{1.858745in}{2.150220in}}{\pgfqpoint{1.858745in}{2.158456in}}%
\pgfpathcurveto{\pgfqpoint{1.858745in}{2.166692in}}{\pgfqpoint{1.855473in}{2.174592in}}{\pgfqpoint{1.849649in}{2.180416in}}%
\pgfpathcurveto{\pgfqpoint{1.843825in}{2.186240in}}{\pgfqpoint{1.835925in}{2.189512in}}{\pgfqpoint{1.827688in}{2.189512in}}%
\pgfpathcurveto{\pgfqpoint{1.819452in}{2.189512in}}{\pgfqpoint{1.811552in}{2.186240in}}{\pgfqpoint{1.805728in}{2.180416in}}%
\pgfpathcurveto{\pgfqpoint{1.799904in}{2.174592in}}{\pgfqpoint{1.796632in}{2.166692in}}{\pgfqpoint{1.796632in}{2.158456in}}%
\pgfpathcurveto{\pgfqpoint{1.796632in}{2.150220in}}{\pgfqpoint{1.799904in}{2.142320in}}{\pgfqpoint{1.805728in}{2.136496in}}%
\pgfpathcurveto{\pgfqpoint{1.811552in}{2.130672in}}{\pgfqpoint{1.819452in}{2.127399in}}{\pgfqpoint{1.827688in}{2.127399in}}%
\pgfpathclose%
\pgfusepath{stroke,fill}%
\end{pgfscope}%
\begin{pgfscope}%
\pgfpathrectangle{\pgfqpoint{0.100000in}{0.212622in}}{\pgfqpoint{3.696000in}{3.696000in}}%
\pgfusepath{clip}%
\pgfsetbuttcap%
\pgfsetroundjoin%
\definecolor{currentfill}{rgb}{0.121569,0.466667,0.705882}%
\pgfsetfillcolor{currentfill}%
\pgfsetfillopacity{0.328894}%
\pgfsetlinewidth{1.003750pt}%
\definecolor{currentstroke}{rgb}{0.121569,0.466667,0.705882}%
\pgfsetstrokecolor{currentstroke}%
\pgfsetstrokeopacity{0.328894}%
\pgfsetdash{}{0pt}%
\pgfpathmoveto{\pgfqpoint{1.830888in}{2.126004in}}%
\pgfpathcurveto{\pgfqpoint{1.839124in}{2.126004in}}{\pgfqpoint{1.847024in}{2.129276in}}{\pgfqpoint{1.852848in}{2.135100in}}%
\pgfpathcurveto{\pgfqpoint{1.858672in}{2.140924in}}{\pgfqpoint{1.861944in}{2.148824in}}{\pgfqpoint{1.861944in}{2.157060in}}%
\pgfpathcurveto{\pgfqpoint{1.861944in}{2.165297in}}{\pgfqpoint{1.858672in}{2.173197in}}{\pgfqpoint{1.852848in}{2.179021in}}%
\pgfpathcurveto{\pgfqpoint{1.847024in}{2.184845in}}{\pgfqpoint{1.839124in}{2.188117in}}{\pgfqpoint{1.830888in}{2.188117in}}%
\pgfpathcurveto{\pgfqpoint{1.822652in}{2.188117in}}{\pgfqpoint{1.814752in}{2.184845in}}{\pgfqpoint{1.808928in}{2.179021in}}%
\pgfpathcurveto{\pgfqpoint{1.803104in}{2.173197in}}{\pgfqpoint{1.799831in}{2.165297in}}{\pgfqpoint{1.799831in}{2.157060in}}%
\pgfpathcurveto{\pgfqpoint{1.799831in}{2.148824in}}{\pgfqpoint{1.803104in}{2.140924in}}{\pgfqpoint{1.808928in}{2.135100in}}%
\pgfpathcurveto{\pgfqpoint{1.814752in}{2.129276in}}{\pgfqpoint{1.822652in}{2.126004in}}{\pgfqpoint{1.830888in}{2.126004in}}%
\pgfpathclose%
\pgfusepath{stroke,fill}%
\end{pgfscope}%
\begin{pgfscope}%
\pgfpathrectangle{\pgfqpoint{0.100000in}{0.212622in}}{\pgfqpoint{3.696000in}{3.696000in}}%
\pgfusepath{clip}%
\pgfsetbuttcap%
\pgfsetroundjoin%
\definecolor{currentfill}{rgb}{0.121569,0.466667,0.705882}%
\pgfsetfillcolor{currentfill}%
\pgfsetfillopacity{0.329679}%
\pgfsetlinewidth{1.003750pt}%
\definecolor{currentstroke}{rgb}{0.121569,0.466667,0.705882}%
\pgfsetstrokecolor{currentstroke}%
\pgfsetstrokeopacity{0.329679}%
\pgfsetdash{}{0pt}%
\pgfpathmoveto{\pgfqpoint{1.834950in}{2.127859in}}%
\pgfpathcurveto{\pgfqpoint{1.843186in}{2.127859in}}{\pgfqpoint{1.851086in}{2.131132in}}{\pgfqpoint{1.856910in}{2.136955in}}%
\pgfpathcurveto{\pgfqpoint{1.862734in}{2.142779in}}{\pgfqpoint{1.866007in}{2.150679in}}{\pgfqpoint{1.866007in}{2.158916in}}%
\pgfpathcurveto{\pgfqpoint{1.866007in}{2.167152in}}{\pgfqpoint{1.862734in}{2.175052in}}{\pgfqpoint{1.856910in}{2.180876in}}%
\pgfpathcurveto{\pgfqpoint{1.851086in}{2.186700in}}{\pgfqpoint{1.843186in}{2.189972in}}{\pgfqpoint{1.834950in}{2.189972in}}%
\pgfpathcurveto{\pgfqpoint{1.826714in}{2.189972in}}{\pgfqpoint{1.818814in}{2.186700in}}{\pgfqpoint{1.812990in}{2.180876in}}%
\pgfpathcurveto{\pgfqpoint{1.807166in}{2.175052in}}{\pgfqpoint{1.803894in}{2.167152in}}{\pgfqpoint{1.803894in}{2.158916in}}%
\pgfpathcurveto{\pgfqpoint{1.803894in}{2.150679in}}{\pgfqpoint{1.807166in}{2.142779in}}{\pgfqpoint{1.812990in}{2.136955in}}%
\pgfpathcurveto{\pgfqpoint{1.818814in}{2.131132in}}{\pgfqpoint{1.826714in}{2.127859in}}{\pgfqpoint{1.834950in}{2.127859in}}%
\pgfpathclose%
\pgfusepath{stroke,fill}%
\end{pgfscope}%
\begin{pgfscope}%
\pgfpathrectangle{\pgfqpoint{0.100000in}{0.212622in}}{\pgfqpoint{3.696000in}{3.696000in}}%
\pgfusepath{clip}%
\pgfsetbuttcap%
\pgfsetroundjoin%
\definecolor{currentfill}{rgb}{0.121569,0.466667,0.705882}%
\pgfsetfillcolor{currentfill}%
\pgfsetfillopacity{0.330342}%
\pgfsetlinewidth{1.003750pt}%
\definecolor{currentstroke}{rgb}{0.121569,0.466667,0.705882}%
\pgfsetstrokecolor{currentstroke}%
\pgfsetstrokeopacity{0.330342}%
\pgfsetdash{}{0pt}%
\pgfpathmoveto{\pgfqpoint{1.840338in}{2.126179in}}%
\pgfpathcurveto{\pgfqpoint{1.848574in}{2.126179in}}{\pgfqpoint{1.856474in}{2.129451in}}{\pgfqpoint{1.862298in}{2.135275in}}%
\pgfpathcurveto{\pgfqpoint{1.868122in}{2.141099in}}{\pgfqpoint{1.871394in}{2.148999in}}{\pgfqpoint{1.871394in}{2.157235in}}%
\pgfpathcurveto{\pgfqpoint{1.871394in}{2.165471in}}{\pgfqpoint{1.868122in}{2.173372in}}{\pgfqpoint{1.862298in}{2.179195in}}%
\pgfpathcurveto{\pgfqpoint{1.856474in}{2.185019in}}{\pgfqpoint{1.848574in}{2.188292in}}{\pgfqpoint{1.840338in}{2.188292in}}%
\pgfpathcurveto{\pgfqpoint{1.832101in}{2.188292in}}{\pgfqpoint{1.824201in}{2.185019in}}{\pgfqpoint{1.818377in}{2.179195in}}%
\pgfpathcurveto{\pgfqpoint{1.812554in}{2.173372in}}{\pgfqpoint{1.809281in}{2.165471in}}{\pgfqpoint{1.809281in}{2.157235in}}%
\pgfpathcurveto{\pgfqpoint{1.809281in}{2.148999in}}{\pgfqpoint{1.812554in}{2.141099in}}{\pgfqpoint{1.818377in}{2.135275in}}%
\pgfpathcurveto{\pgfqpoint{1.824201in}{2.129451in}}{\pgfqpoint{1.832101in}{2.126179in}}{\pgfqpoint{1.840338in}{2.126179in}}%
\pgfpathclose%
\pgfusepath{stroke,fill}%
\end{pgfscope}%
\begin{pgfscope}%
\pgfpathrectangle{\pgfqpoint{0.100000in}{0.212622in}}{\pgfqpoint{3.696000in}{3.696000in}}%
\pgfusepath{clip}%
\pgfsetbuttcap%
\pgfsetroundjoin%
\definecolor{currentfill}{rgb}{0.121569,0.466667,0.705882}%
\pgfsetfillcolor{currentfill}%
\pgfsetfillopacity{0.330383}%
\pgfsetlinewidth{1.003750pt}%
\definecolor{currentstroke}{rgb}{0.121569,0.466667,0.705882}%
\pgfsetstrokecolor{currentstroke}%
\pgfsetstrokeopacity{0.330383}%
\pgfsetdash{}{0pt}%
\pgfpathmoveto{\pgfqpoint{1.587743in}{2.035515in}}%
\pgfpathcurveto{\pgfqpoint{1.595979in}{2.035515in}}{\pgfqpoint{1.603879in}{2.038787in}}{\pgfqpoint{1.609703in}{2.044611in}}%
\pgfpathcurveto{\pgfqpoint{1.615527in}{2.050435in}}{\pgfqpoint{1.618800in}{2.058335in}}{\pgfqpoint{1.618800in}{2.066572in}}%
\pgfpathcurveto{\pgfqpoint{1.618800in}{2.074808in}}{\pgfqpoint{1.615527in}{2.082708in}}{\pgfqpoint{1.609703in}{2.088532in}}%
\pgfpathcurveto{\pgfqpoint{1.603879in}{2.094356in}}{\pgfqpoint{1.595979in}{2.097628in}}{\pgfqpoint{1.587743in}{2.097628in}}%
\pgfpathcurveto{\pgfqpoint{1.579507in}{2.097628in}}{\pgfqpoint{1.571607in}{2.094356in}}{\pgfqpoint{1.565783in}{2.088532in}}%
\pgfpathcurveto{\pgfqpoint{1.559959in}{2.082708in}}{\pgfqpoint{1.556687in}{2.074808in}}{\pgfqpoint{1.556687in}{2.066572in}}%
\pgfpathcurveto{\pgfqpoint{1.556687in}{2.058335in}}{\pgfqpoint{1.559959in}{2.050435in}}{\pgfqpoint{1.565783in}{2.044611in}}%
\pgfpathcurveto{\pgfqpoint{1.571607in}{2.038787in}}{\pgfqpoint{1.579507in}{2.035515in}}{\pgfqpoint{1.587743in}{2.035515in}}%
\pgfpathclose%
\pgfusepath{stroke,fill}%
\end{pgfscope}%
\begin{pgfscope}%
\pgfpathrectangle{\pgfqpoint{0.100000in}{0.212622in}}{\pgfqpoint{3.696000in}{3.696000in}}%
\pgfusepath{clip}%
\pgfsetbuttcap%
\pgfsetroundjoin%
\definecolor{currentfill}{rgb}{0.121569,0.466667,0.705882}%
\pgfsetfillcolor{currentfill}%
\pgfsetfillopacity{0.331215}%
\pgfsetlinewidth{1.003750pt}%
\definecolor{currentstroke}{rgb}{0.121569,0.466667,0.705882}%
\pgfsetstrokecolor{currentstroke}%
\pgfsetstrokeopacity{0.331215}%
\pgfsetdash{}{0pt}%
\pgfpathmoveto{\pgfqpoint{1.846328in}{2.126678in}}%
\pgfpathcurveto{\pgfqpoint{1.854564in}{2.126678in}}{\pgfqpoint{1.862464in}{2.129950in}}{\pgfqpoint{1.868288in}{2.135774in}}%
\pgfpathcurveto{\pgfqpoint{1.874112in}{2.141598in}}{\pgfqpoint{1.877385in}{2.149498in}}{\pgfqpoint{1.877385in}{2.157734in}}%
\pgfpathcurveto{\pgfqpoint{1.877385in}{2.165971in}}{\pgfqpoint{1.874112in}{2.173871in}}{\pgfqpoint{1.868288in}{2.179695in}}%
\pgfpathcurveto{\pgfqpoint{1.862464in}{2.185519in}}{\pgfqpoint{1.854564in}{2.188791in}}{\pgfqpoint{1.846328in}{2.188791in}}%
\pgfpathcurveto{\pgfqpoint{1.838092in}{2.188791in}}{\pgfqpoint{1.830192in}{2.185519in}}{\pgfqpoint{1.824368in}{2.179695in}}%
\pgfpathcurveto{\pgfqpoint{1.818544in}{2.173871in}}{\pgfqpoint{1.815272in}{2.165971in}}{\pgfqpoint{1.815272in}{2.157734in}}%
\pgfpathcurveto{\pgfqpoint{1.815272in}{2.149498in}}{\pgfqpoint{1.818544in}{2.141598in}}{\pgfqpoint{1.824368in}{2.135774in}}%
\pgfpathcurveto{\pgfqpoint{1.830192in}{2.129950in}}{\pgfqpoint{1.838092in}{2.126678in}}{\pgfqpoint{1.846328in}{2.126678in}}%
\pgfpathclose%
\pgfusepath{stroke,fill}%
\end{pgfscope}%
\begin{pgfscope}%
\pgfpathrectangle{\pgfqpoint{0.100000in}{0.212622in}}{\pgfqpoint{3.696000in}{3.696000in}}%
\pgfusepath{clip}%
\pgfsetbuttcap%
\pgfsetroundjoin%
\definecolor{currentfill}{rgb}{0.121569,0.466667,0.705882}%
\pgfsetfillcolor{currentfill}%
\pgfsetfillopacity{0.331974}%
\pgfsetlinewidth{1.003750pt}%
\definecolor{currentstroke}{rgb}{0.121569,0.466667,0.705882}%
\pgfsetstrokecolor{currentstroke}%
\pgfsetstrokeopacity{0.331974}%
\pgfsetdash{}{0pt}%
\pgfpathmoveto{\pgfqpoint{1.851964in}{2.124003in}}%
\pgfpathcurveto{\pgfqpoint{1.860200in}{2.124003in}}{\pgfqpoint{1.868100in}{2.127276in}}{\pgfqpoint{1.873924in}{2.133100in}}%
\pgfpathcurveto{\pgfqpoint{1.879748in}{2.138924in}}{\pgfqpoint{1.883020in}{2.146824in}}{\pgfqpoint{1.883020in}{2.155060in}}%
\pgfpathcurveto{\pgfqpoint{1.883020in}{2.163296in}}{\pgfqpoint{1.879748in}{2.171196in}}{\pgfqpoint{1.873924in}{2.177020in}}%
\pgfpathcurveto{\pgfqpoint{1.868100in}{2.182844in}}{\pgfqpoint{1.860200in}{2.186116in}}{\pgfqpoint{1.851964in}{2.186116in}}%
\pgfpathcurveto{\pgfqpoint{1.843727in}{2.186116in}}{\pgfqpoint{1.835827in}{2.182844in}}{\pgfqpoint{1.830003in}{2.177020in}}%
\pgfpathcurveto{\pgfqpoint{1.824180in}{2.171196in}}{\pgfqpoint{1.820907in}{2.163296in}}{\pgfqpoint{1.820907in}{2.155060in}}%
\pgfpathcurveto{\pgfqpoint{1.820907in}{2.146824in}}{\pgfqpoint{1.824180in}{2.138924in}}{\pgfqpoint{1.830003in}{2.133100in}}%
\pgfpathcurveto{\pgfqpoint{1.835827in}{2.127276in}}{\pgfqpoint{1.843727in}{2.124003in}}{\pgfqpoint{1.851964in}{2.124003in}}%
\pgfpathclose%
\pgfusepath{stroke,fill}%
\end{pgfscope}%
\begin{pgfscope}%
\pgfpathrectangle{\pgfqpoint{0.100000in}{0.212622in}}{\pgfqpoint{3.696000in}{3.696000in}}%
\pgfusepath{clip}%
\pgfsetbuttcap%
\pgfsetroundjoin%
\definecolor{currentfill}{rgb}{0.121569,0.466667,0.705882}%
\pgfsetfillcolor{currentfill}%
\pgfsetfillopacity{0.332039}%
\pgfsetlinewidth{1.003750pt}%
\definecolor{currentstroke}{rgb}{0.121569,0.466667,0.705882}%
\pgfsetstrokecolor{currentstroke}%
\pgfsetstrokeopacity{0.332039}%
\pgfsetdash{}{0pt}%
\pgfpathmoveto{\pgfqpoint{1.581328in}{2.038656in}}%
\pgfpathcurveto{\pgfqpoint{1.589564in}{2.038656in}}{\pgfqpoint{1.597465in}{2.041928in}}{\pgfqpoint{1.603288in}{2.047752in}}%
\pgfpathcurveto{\pgfqpoint{1.609112in}{2.053576in}}{\pgfqpoint{1.612385in}{2.061476in}}{\pgfqpoint{1.612385in}{2.069712in}}%
\pgfpathcurveto{\pgfqpoint{1.612385in}{2.077948in}}{\pgfqpoint{1.609112in}{2.085849in}}{\pgfqpoint{1.603288in}{2.091672in}}%
\pgfpathcurveto{\pgfqpoint{1.597465in}{2.097496in}}{\pgfqpoint{1.589564in}{2.100769in}}{\pgfqpoint{1.581328in}{2.100769in}}%
\pgfpathcurveto{\pgfqpoint{1.573092in}{2.100769in}}{\pgfqpoint{1.565192in}{2.097496in}}{\pgfqpoint{1.559368in}{2.091672in}}%
\pgfpathcurveto{\pgfqpoint{1.553544in}{2.085849in}}{\pgfqpoint{1.550272in}{2.077948in}}{\pgfqpoint{1.550272in}{2.069712in}}%
\pgfpathcurveto{\pgfqpoint{1.550272in}{2.061476in}}{\pgfqpoint{1.553544in}{2.053576in}}{\pgfqpoint{1.559368in}{2.047752in}}%
\pgfpathcurveto{\pgfqpoint{1.565192in}{2.041928in}}{\pgfqpoint{1.573092in}{2.038656in}}{\pgfqpoint{1.581328in}{2.038656in}}%
\pgfpathclose%
\pgfusepath{stroke,fill}%
\end{pgfscope}%
\begin{pgfscope}%
\pgfpathrectangle{\pgfqpoint{0.100000in}{0.212622in}}{\pgfqpoint{3.696000in}{3.696000in}}%
\pgfusepath{clip}%
\pgfsetbuttcap%
\pgfsetroundjoin%
\definecolor{currentfill}{rgb}{0.121569,0.466667,0.705882}%
\pgfsetfillcolor{currentfill}%
\pgfsetfillopacity{0.332443}%
\pgfsetlinewidth{1.003750pt}%
\definecolor{currentstroke}{rgb}{0.121569,0.466667,0.705882}%
\pgfsetstrokecolor{currentstroke}%
\pgfsetstrokeopacity{0.332443}%
\pgfsetdash{}{0pt}%
\pgfpathmoveto{\pgfqpoint{1.577767in}{2.028602in}}%
\pgfpathcurveto{\pgfqpoint{1.586003in}{2.028602in}}{\pgfqpoint{1.593903in}{2.031874in}}{\pgfqpoint{1.599727in}{2.037698in}}%
\pgfpathcurveto{\pgfqpoint{1.605551in}{2.043522in}}{\pgfqpoint{1.608823in}{2.051422in}}{\pgfqpoint{1.608823in}{2.059659in}}%
\pgfpathcurveto{\pgfqpoint{1.608823in}{2.067895in}}{\pgfqpoint{1.605551in}{2.075795in}}{\pgfqpoint{1.599727in}{2.081619in}}%
\pgfpathcurveto{\pgfqpoint{1.593903in}{2.087443in}}{\pgfqpoint{1.586003in}{2.090715in}}{\pgfqpoint{1.577767in}{2.090715in}}%
\pgfpathcurveto{\pgfqpoint{1.569531in}{2.090715in}}{\pgfqpoint{1.561631in}{2.087443in}}{\pgfqpoint{1.555807in}{2.081619in}}%
\pgfpathcurveto{\pgfqpoint{1.549983in}{2.075795in}}{\pgfqpoint{1.546710in}{2.067895in}}{\pgfqpoint{1.546710in}{2.059659in}}%
\pgfpathcurveto{\pgfqpoint{1.546710in}{2.051422in}}{\pgfqpoint{1.549983in}{2.043522in}}{\pgfqpoint{1.555807in}{2.037698in}}%
\pgfpathcurveto{\pgfqpoint{1.561631in}{2.031874in}}{\pgfqpoint{1.569531in}{2.028602in}}{\pgfqpoint{1.577767in}{2.028602in}}%
\pgfpathclose%
\pgfusepath{stroke,fill}%
\end{pgfscope}%
\begin{pgfscope}%
\pgfpathrectangle{\pgfqpoint{0.100000in}{0.212622in}}{\pgfqpoint{3.696000in}{3.696000in}}%
\pgfusepath{clip}%
\pgfsetbuttcap%
\pgfsetroundjoin%
\definecolor{currentfill}{rgb}{0.121569,0.466667,0.705882}%
\pgfsetfillcolor{currentfill}%
\pgfsetfillopacity{0.333135}%
\pgfsetlinewidth{1.003750pt}%
\definecolor{currentstroke}{rgb}{0.121569,0.466667,0.705882}%
\pgfsetstrokecolor{currentstroke}%
\pgfsetstrokeopacity{0.333135}%
\pgfsetdash{}{0pt}%
\pgfpathmoveto{\pgfqpoint{1.858002in}{2.124104in}}%
\pgfpathcurveto{\pgfqpoint{1.866238in}{2.124104in}}{\pgfqpoint{1.874138in}{2.127377in}}{\pgfqpoint{1.879962in}{2.133200in}}%
\pgfpathcurveto{\pgfqpoint{1.885786in}{2.139024in}}{\pgfqpoint{1.889058in}{2.146924in}}{\pgfqpoint{1.889058in}{2.155161in}}%
\pgfpathcurveto{\pgfqpoint{1.889058in}{2.163397in}}{\pgfqpoint{1.885786in}{2.171297in}}{\pgfqpoint{1.879962in}{2.177121in}}%
\pgfpathcurveto{\pgfqpoint{1.874138in}{2.182945in}}{\pgfqpoint{1.866238in}{2.186217in}}{\pgfqpoint{1.858002in}{2.186217in}}%
\pgfpathcurveto{\pgfqpoint{1.849765in}{2.186217in}}{\pgfqpoint{1.841865in}{2.182945in}}{\pgfqpoint{1.836041in}{2.177121in}}%
\pgfpathcurveto{\pgfqpoint{1.830217in}{2.171297in}}{\pgfqpoint{1.826945in}{2.163397in}}{\pgfqpoint{1.826945in}{2.155161in}}%
\pgfpathcurveto{\pgfqpoint{1.826945in}{2.146924in}}{\pgfqpoint{1.830217in}{2.139024in}}{\pgfqpoint{1.836041in}{2.133200in}}%
\pgfpathcurveto{\pgfqpoint{1.841865in}{2.127377in}}{\pgfqpoint{1.849765in}{2.124104in}}{\pgfqpoint{1.858002in}{2.124104in}}%
\pgfpathclose%
\pgfusepath{stroke,fill}%
\end{pgfscope}%
\begin{pgfscope}%
\pgfpathrectangle{\pgfqpoint{0.100000in}{0.212622in}}{\pgfqpoint{3.696000in}{3.696000in}}%
\pgfusepath{clip}%
\pgfsetbuttcap%
\pgfsetroundjoin%
\definecolor{currentfill}{rgb}{0.121569,0.466667,0.705882}%
\pgfsetfillcolor{currentfill}%
\pgfsetfillopacity{0.333670}%
\pgfsetlinewidth{1.003750pt}%
\definecolor{currentstroke}{rgb}{0.121569,0.466667,0.705882}%
\pgfsetstrokecolor{currentstroke}%
\pgfsetstrokeopacity{0.333670}%
\pgfsetdash{}{0pt}%
\pgfpathmoveto{\pgfqpoint{1.575415in}{2.024447in}}%
\pgfpathcurveto{\pgfqpoint{1.583651in}{2.024447in}}{\pgfqpoint{1.591551in}{2.027719in}}{\pgfqpoint{1.597375in}{2.033543in}}%
\pgfpathcurveto{\pgfqpoint{1.603199in}{2.039367in}}{\pgfqpoint{1.606471in}{2.047267in}}{\pgfqpoint{1.606471in}{2.055503in}}%
\pgfpathcurveto{\pgfqpoint{1.606471in}{2.063739in}}{\pgfqpoint{1.603199in}{2.071640in}}{\pgfqpoint{1.597375in}{2.077463in}}%
\pgfpathcurveto{\pgfqpoint{1.591551in}{2.083287in}}{\pgfqpoint{1.583651in}{2.086560in}}{\pgfqpoint{1.575415in}{2.086560in}}%
\pgfpathcurveto{\pgfqpoint{1.567179in}{2.086560in}}{\pgfqpoint{1.559279in}{2.083287in}}{\pgfqpoint{1.553455in}{2.077463in}}%
\pgfpathcurveto{\pgfqpoint{1.547631in}{2.071640in}}{\pgfqpoint{1.544358in}{2.063739in}}{\pgfqpoint{1.544358in}{2.055503in}}%
\pgfpathcurveto{\pgfqpoint{1.544358in}{2.047267in}}{\pgfqpoint{1.547631in}{2.039367in}}{\pgfqpoint{1.553455in}{2.033543in}}%
\pgfpathcurveto{\pgfqpoint{1.559279in}{2.027719in}}{\pgfqpoint{1.567179in}{2.024447in}}{\pgfqpoint{1.575415in}{2.024447in}}%
\pgfpathclose%
\pgfusepath{stroke,fill}%
\end{pgfscope}%
\begin{pgfscope}%
\pgfpathrectangle{\pgfqpoint{0.100000in}{0.212622in}}{\pgfqpoint{3.696000in}{3.696000in}}%
\pgfusepath{clip}%
\pgfsetbuttcap%
\pgfsetroundjoin%
\definecolor{currentfill}{rgb}{0.121569,0.466667,0.705882}%
\pgfsetfillcolor{currentfill}%
\pgfsetfillopacity{0.334588}%
\pgfsetlinewidth{1.003750pt}%
\definecolor{currentstroke}{rgb}{0.121569,0.466667,0.705882}%
\pgfsetstrokecolor{currentstroke}%
\pgfsetstrokeopacity{0.334588}%
\pgfsetdash{}{0pt}%
\pgfpathmoveto{\pgfqpoint{1.573916in}{2.021673in}}%
\pgfpathcurveto{\pgfqpoint{1.582152in}{2.021673in}}{\pgfqpoint{1.590052in}{2.024946in}}{\pgfqpoint{1.595876in}{2.030769in}}%
\pgfpathcurveto{\pgfqpoint{1.601700in}{2.036593in}}{\pgfqpoint{1.604972in}{2.044493in}}{\pgfqpoint{1.604972in}{2.052730in}}%
\pgfpathcurveto{\pgfqpoint{1.604972in}{2.060966in}}{\pgfqpoint{1.601700in}{2.068866in}}{\pgfqpoint{1.595876in}{2.074690in}}%
\pgfpathcurveto{\pgfqpoint{1.590052in}{2.080514in}}{\pgfqpoint{1.582152in}{2.083786in}}{\pgfqpoint{1.573916in}{2.083786in}}%
\pgfpathcurveto{\pgfqpoint{1.565679in}{2.083786in}}{\pgfqpoint{1.557779in}{2.080514in}}{\pgfqpoint{1.551955in}{2.074690in}}%
\pgfpathcurveto{\pgfqpoint{1.546131in}{2.068866in}}{\pgfqpoint{1.542859in}{2.060966in}}{\pgfqpoint{1.542859in}{2.052730in}}%
\pgfpathcurveto{\pgfqpoint{1.542859in}{2.044493in}}{\pgfqpoint{1.546131in}{2.036593in}}{\pgfqpoint{1.551955in}{2.030769in}}%
\pgfpathcurveto{\pgfqpoint{1.557779in}{2.024946in}}{\pgfqpoint{1.565679in}{2.021673in}}{\pgfqpoint{1.573916in}{2.021673in}}%
\pgfpathclose%
\pgfusepath{stroke,fill}%
\end{pgfscope}%
\begin{pgfscope}%
\pgfpathrectangle{\pgfqpoint{0.100000in}{0.212622in}}{\pgfqpoint{3.696000in}{3.696000in}}%
\pgfusepath{clip}%
\pgfsetbuttcap%
\pgfsetroundjoin%
\definecolor{currentfill}{rgb}{0.121569,0.466667,0.705882}%
\pgfsetfillcolor{currentfill}%
\pgfsetfillopacity{0.334901}%
\pgfsetlinewidth{1.003750pt}%
\definecolor{currentstroke}{rgb}{0.121569,0.466667,0.705882}%
\pgfsetstrokecolor{currentstroke}%
\pgfsetstrokeopacity{0.334901}%
\pgfsetdash{}{0pt}%
\pgfpathmoveto{\pgfqpoint{1.572537in}{2.019365in}}%
\pgfpathcurveto{\pgfqpoint{1.580773in}{2.019365in}}{\pgfqpoint{1.588673in}{2.022638in}}{\pgfqpoint{1.594497in}{2.028462in}}%
\pgfpathcurveto{\pgfqpoint{1.600321in}{2.034286in}}{\pgfqpoint{1.603593in}{2.042186in}}{\pgfqpoint{1.603593in}{2.050422in}}%
\pgfpathcurveto{\pgfqpoint{1.603593in}{2.058658in}}{\pgfqpoint{1.600321in}{2.066558in}}{\pgfqpoint{1.594497in}{2.072382in}}%
\pgfpathcurveto{\pgfqpoint{1.588673in}{2.078206in}}{\pgfqpoint{1.580773in}{2.081478in}}{\pgfqpoint{1.572537in}{2.081478in}}%
\pgfpathcurveto{\pgfqpoint{1.564300in}{2.081478in}}{\pgfqpoint{1.556400in}{2.078206in}}{\pgfqpoint{1.550576in}{2.072382in}}%
\pgfpathcurveto{\pgfqpoint{1.544753in}{2.066558in}}{\pgfqpoint{1.541480in}{2.058658in}}{\pgfqpoint{1.541480in}{2.050422in}}%
\pgfpathcurveto{\pgfqpoint{1.541480in}{2.042186in}}{\pgfqpoint{1.544753in}{2.034286in}}{\pgfqpoint{1.550576in}{2.028462in}}%
\pgfpathcurveto{\pgfqpoint{1.556400in}{2.022638in}}{\pgfqpoint{1.564300in}{2.019365in}}{\pgfqpoint{1.572537in}{2.019365in}}%
\pgfpathclose%
\pgfusepath{stroke,fill}%
\end{pgfscope}%
\begin{pgfscope}%
\pgfpathrectangle{\pgfqpoint{0.100000in}{0.212622in}}{\pgfqpoint{3.696000in}{3.696000in}}%
\pgfusepath{clip}%
\pgfsetbuttcap%
\pgfsetroundjoin%
\definecolor{currentfill}{rgb}{0.121569,0.466667,0.705882}%
\pgfsetfillcolor{currentfill}%
\pgfsetfillopacity{0.334980}%
\pgfsetlinewidth{1.003750pt}%
\definecolor{currentstroke}{rgb}{0.121569,0.466667,0.705882}%
\pgfsetstrokecolor{currentstroke}%
\pgfsetstrokeopacity{0.334980}%
\pgfsetdash{}{0pt}%
\pgfpathmoveto{\pgfqpoint{1.865636in}{2.128438in}}%
\pgfpathcurveto{\pgfqpoint{1.873872in}{2.128438in}}{\pgfqpoint{1.881772in}{2.131710in}}{\pgfqpoint{1.887596in}{2.137534in}}%
\pgfpathcurveto{\pgfqpoint{1.893420in}{2.143358in}}{\pgfqpoint{1.896692in}{2.151258in}}{\pgfqpoint{1.896692in}{2.159494in}}%
\pgfpathcurveto{\pgfqpoint{1.896692in}{2.167730in}}{\pgfqpoint{1.893420in}{2.175630in}}{\pgfqpoint{1.887596in}{2.181454in}}%
\pgfpathcurveto{\pgfqpoint{1.881772in}{2.187278in}}{\pgfqpoint{1.873872in}{2.190551in}}{\pgfqpoint{1.865636in}{2.190551in}}%
\pgfpathcurveto{\pgfqpoint{1.857400in}{2.190551in}}{\pgfqpoint{1.849500in}{2.187278in}}{\pgfqpoint{1.843676in}{2.181454in}}%
\pgfpathcurveto{\pgfqpoint{1.837852in}{2.175630in}}{\pgfqpoint{1.834579in}{2.167730in}}{\pgfqpoint{1.834579in}{2.159494in}}%
\pgfpathcurveto{\pgfqpoint{1.834579in}{2.151258in}}{\pgfqpoint{1.837852in}{2.143358in}}{\pgfqpoint{1.843676in}{2.137534in}}%
\pgfpathcurveto{\pgfqpoint{1.849500in}{2.131710in}}{\pgfqpoint{1.857400in}{2.128438in}}{\pgfqpoint{1.865636in}{2.128438in}}%
\pgfpathclose%
\pgfusepath{stroke,fill}%
\end{pgfscope}%
\begin{pgfscope}%
\pgfpathrectangle{\pgfqpoint{0.100000in}{0.212622in}}{\pgfqpoint{3.696000in}{3.696000in}}%
\pgfusepath{clip}%
\pgfsetbuttcap%
\pgfsetroundjoin%
\definecolor{currentfill}{rgb}{0.121569,0.466667,0.705882}%
\pgfsetfillcolor{currentfill}%
\pgfsetfillopacity{0.335890}%
\pgfsetlinewidth{1.003750pt}%
\definecolor{currentstroke}{rgb}{0.121569,0.466667,0.705882}%
\pgfsetstrokecolor{currentstroke}%
\pgfsetstrokeopacity{0.335890}%
\pgfsetdash{}{0pt}%
\pgfpathmoveto{\pgfqpoint{1.873091in}{2.122630in}}%
\pgfpathcurveto{\pgfqpoint{1.881327in}{2.122630in}}{\pgfqpoint{1.889227in}{2.125902in}}{\pgfqpoint{1.895051in}{2.131726in}}%
\pgfpathcurveto{\pgfqpoint{1.900875in}{2.137550in}}{\pgfqpoint{1.904147in}{2.145450in}}{\pgfqpoint{1.904147in}{2.153687in}}%
\pgfpathcurveto{\pgfqpoint{1.904147in}{2.161923in}}{\pgfqpoint{1.900875in}{2.169823in}}{\pgfqpoint{1.895051in}{2.175647in}}%
\pgfpathcurveto{\pgfqpoint{1.889227in}{2.181471in}}{\pgfqpoint{1.881327in}{2.184743in}}{\pgfqpoint{1.873091in}{2.184743in}}%
\pgfpathcurveto{\pgfqpoint{1.864855in}{2.184743in}}{\pgfqpoint{1.856955in}{2.181471in}}{\pgfqpoint{1.851131in}{2.175647in}}%
\pgfpathcurveto{\pgfqpoint{1.845307in}{2.169823in}}{\pgfqpoint{1.842034in}{2.161923in}}{\pgfqpoint{1.842034in}{2.153687in}}%
\pgfpathcurveto{\pgfqpoint{1.842034in}{2.145450in}}{\pgfqpoint{1.845307in}{2.137550in}}{\pgfqpoint{1.851131in}{2.131726in}}%
\pgfpathcurveto{\pgfqpoint{1.856955in}{2.125902in}}{\pgfqpoint{1.864855in}{2.122630in}}{\pgfqpoint{1.873091in}{2.122630in}}%
\pgfpathclose%
\pgfusepath{stroke,fill}%
\end{pgfscope}%
\begin{pgfscope}%
\pgfpathrectangle{\pgfqpoint{0.100000in}{0.212622in}}{\pgfqpoint{3.696000in}{3.696000in}}%
\pgfusepath{clip}%
\pgfsetbuttcap%
\pgfsetroundjoin%
\definecolor{currentfill}{rgb}{0.121569,0.466667,0.705882}%
\pgfsetfillcolor{currentfill}%
\pgfsetfillopacity{0.336229}%
\pgfsetlinewidth{1.003750pt}%
\definecolor{currentstroke}{rgb}{0.121569,0.466667,0.705882}%
\pgfsetstrokecolor{currentstroke}%
\pgfsetstrokeopacity{0.336229}%
\pgfsetdash{}{0pt}%
\pgfpathmoveto{\pgfqpoint{1.571378in}{2.019766in}}%
\pgfpathcurveto{\pgfqpoint{1.579614in}{2.019766in}}{\pgfqpoint{1.587515in}{2.023038in}}{\pgfqpoint{1.593338in}{2.028862in}}%
\pgfpathcurveto{\pgfqpoint{1.599162in}{2.034686in}}{\pgfqpoint{1.602435in}{2.042586in}}{\pgfqpoint{1.602435in}{2.050823in}}%
\pgfpathcurveto{\pgfqpoint{1.602435in}{2.059059in}}{\pgfqpoint{1.599162in}{2.066959in}}{\pgfqpoint{1.593338in}{2.072783in}}%
\pgfpathcurveto{\pgfqpoint{1.587515in}{2.078607in}}{\pgfqpoint{1.579614in}{2.081879in}}{\pgfqpoint{1.571378in}{2.081879in}}%
\pgfpathcurveto{\pgfqpoint{1.563142in}{2.081879in}}{\pgfqpoint{1.555242in}{2.078607in}}{\pgfqpoint{1.549418in}{2.072783in}}%
\pgfpathcurveto{\pgfqpoint{1.543594in}{2.066959in}}{\pgfqpoint{1.540322in}{2.059059in}}{\pgfqpoint{1.540322in}{2.050823in}}%
\pgfpathcurveto{\pgfqpoint{1.540322in}{2.042586in}}{\pgfqpoint{1.543594in}{2.034686in}}{\pgfqpoint{1.549418in}{2.028862in}}%
\pgfpathcurveto{\pgfqpoint{1.555242in}{2.023038in}}{\pgfqpoint{1.563142in}{2.019766in}}{\pgfqpoint{1.571378in}{2.019766in}}%
\pgfpathclose%
\pgfusepath{stroke,fill}%
\end{pgfscope}%
\begin{pgfscope}%
\pgfpathrectangle{\pgfqpoint{0.100000in}{0.212622in}}{\pgfqpoint{3.696000in}{3.696000in}}%
\pgfusepath{clip}%
\pgfsetbuttcap%
\pgfsetroundjoin%
\definecolor{currentfill}{rgb}{0.121569,0.466667,0.705882}%
\pgfsetfillcolor{currentfill}%
\pgfsetfillopacity{0.336562}%
\pgfsetlinewidth{1.003750pt}%
\definecolor{currentstroke}{rgb}{0.121569,0.466667,0.705882}%
\pgfsetstrokecolor{currentstroke}%
\pgfsetstrokeopacity{0.336562}%
\pgfsetdash{}{0pt}%
\pgfpathmoveto{\pgfqpoint{1.569616in}{2.017904in}}%
\pgfpathcurveto{\pgfqpoint{1.577852in}{2.017904in}}{\pgfqpoint{1.585753in}{2.021176in}}{\pgfqpoint{1.591576in}{2.027000in}}%
\pgfpathcurveto{\pgfqpoint{1.597400in}{2.032824in}}{\pgfqpoint{1.600673in}{2.040724in}}{\pgfqpoint{1.600673in}{2.048961in}}%
\pgfpathcurveto{\pgfqpoint{1.600673in}{2.057197in}}{\pgfqpoint{1.597400in}{2.065097in}}{\pgfqpoint{1.591576in}{2.070921in}}%
\pgfpathcurveto{\pgfqpoint{1.585753in}{2.076745in}}{\pgfqpoint{1.577852in}{2.080017in}}{\pgfqpoint{1.569616in}{2.080017in}}%
\pgfpathcurveto{\pgfqpoint{1.561380in}{2.080017in}}{\pgfqpoint{1.553480in}{2.076745in}}{\pgfqpoint{1.547656in}{2.070921in}}%
\pgfpathcurveto{\pgfqpoint{1.541832in}{2.065097in}}{\pgfqpoint{1.538560in}{2.057197in}}{\pgfqpoint{1.538560in}{2.048961in}}%
\pgfpathcurveto{\pgfqpoint{1.538560in}{2.040724in}}{\pgfqpoint{1.541832in}{2.032824in}}{\pgfqpoint{1.547656in}{2.027000in}}%
\pgfpathcurveto{\pgfqpoint{1.553480in}{2.021176in}}{\pgfqpoint{1.561380in}{2.017904in}}{\pgfqpoint{1.569616in}{2.017904in}}%
\pgfpathclose%
\pgfusepath{stroke,fill}%
\end{pgfscope}%
\begin{pgfscope}%
\pgfpathrectangle{\pgfqpoint{0.100000in}{0.212622in}}{\pgfqpoint{3.696000in}{3.696000in}}%
\pgfusepath{clip}%
\pgfsetbuttcap%
\pgfsetroundjoin%
\definecolor{currentfill}{rgb}{0.121569,0.466667,0.705882}%
\pgfsetfillcolor{currentfill}%
\pgfsetfillopacity{0.336592}%
\pgfsetlinewidth{1.003750pt}%
\definecolor{currentstroke}{rgb}{0.121569,0.466667,0.705882}%
\pgfsetstrokecolor{currentstroke}%
\pgfsetstrokeopacity{0.336592}%
\pgfsetdash{}{0pt}%
\pgfpathmoveto{\pgfqpoint{1.569501in}{2.017828in}}%
\pgfpathcurveto{\pgfqpoint{1.577737in}{2.017828in}}{\pgfqpoint{1.585637in}{2.021100in}}{\pgfqpoint{1.591461in}{2.026924in}}%
\pgfpathcurveto{\pgfqpoint{1.597285in}{2.032748in}}{\pgfqpoint{1.600557in}{2.040648in}}{\pgfqpoint{1.600557in}{2.048884in}}%
\pgfpathcurveto{\pgfqpoint{1.600557in}{2.057120in}}{\pgfqpoint{1.597285in}{2.065020in}}{\pgfqpoint{1.591461in}{2.070844in}}%
\pgfpathcurveto{\pgfqpoint{1.585637in}{2.076668in}}{\pgfqpoint{1.577737in}{2.079941in}}{\pgfqpoint{1.569501in}{2.079941in}}%
\pgfpathcurveto{\pgfqpoint{1.561264in}{2.079941in}}{\pgfqpoint{1.553364in}{2.076668in}}{\pgfqpoint{1.547540in}{2.070844in}}%
\pgfpathcurveto{\pgfqpoint{1.541716in}{2.065020in}}{\pgfqpoint{1.538444in}{2.057120in}}{\pgfqpoint{1.538444in}{2.048884in}}%
\pgfpathcurveto{\pgfqpoint{1.538444in}{2.040648in}}{\pgfqpoint{1.541716in}{2.032748in}}{\pgfqpoint{1.547540in}{2.026924in}}%
\pgfpathcurveto{\pgfqpoint{1.553364in}{2.021100in}}{\pgfqpoint{1.561264in}{2.017828in}}{\pgfqpoint{1.569501in}{2.017828in}}%
\pgfpathclose%
\pgfusepath{stroke,fill}%
\end{pgfscope}%
\begin{pgfscope}%
\pgfpathrectangle{\pgfqpoint{0.100000in}{0.212622in}}{\pgfqpoint{3.696000in}{3.696000in}}%
\pgfusepath{clip}%
\pgfsetbuttcap%
\pgfsetroundjoin%
\definecolor{currentfill}{rgb}{0.121569,0.466667,0.705882}%
\pgfsetfillcolor{currentfill}%
\pgfsetfillopacity{0.336688}%
\pgfsetlinewidth{1.003750pt}%
\definecolor{currentstroke}{rgb}{0.121569,0.466667,0.705882}%
\pgfsetstrokecolor{currentstroke}%
\pgfsetstrokeopacity{0.336688}%
\pgfsetdash{}{0pt}%
\pgfpathmoveto{\pgfqpoint{1.569422in}{2.017876in}}%
\pgfpathcurveto{\pgfqpoint{1.577658in}{2.017876in}}{\pgfqpoint{1.585558in}{2.021148in}}{\pgfqpoint{1.591382in}{2.026972in}}%
\pgfpathcurveto{\pgfqpoint{1.597206in}{2.032796in}}{\pgfqpoint{1.600478in}{2.040696in}}{\pgfqpoint{1.600478in}{2.048933in}}%
\pgfpathcurveto{\pgfqpoint{1.600478in}{2.057169in}}{\pgfqpoint{1.597206in}{2.065069in}}{\pgfqpoint{1.591382in}{2.070893in}}%
\pgfpathcurveto{\pgfqpoint{1.585558in}{2.076717in}}{\pgfqpoint{1.577658in}{2.079989in}}{\pgfqpoint{1.569422in}{2.079989in}}%
\pgfpathcurveto{\pgfqpoint{1.561186in}{2.079989in}}{\pgfqpoint{1.553286in}{2.076717in}}{\pgfqpoint{1.547462in}{2.070893in}}%
\pgfpathcurveto{\pgfqpoint{1.541638in}{2.065069in}}{\pgfqpoint{1.538365in}{2.057169in}}{\pgfqpoint{1.538365in}{2.048933in}}%
\pgfpathcurveto{\pgfqpoint{1.538365in}{2.040696in}}{\pgfqpoint{1.541638in}{2.032796in}}{\pgfqpoint{1.547462in}{2.026972in}}%
\pgfpathcurveto{\pgfqpoint{1.553286in}{2.021148in}}{\pgfqpoint{1.561186in}{2.017876in}}{\pgfqpoint{1.569422in}{2.017876in}}%
\pgfpathclose%
\pgfusepath{stroke,fill}%
\end{pgfscope}%
\begin{pgfscope}%
\pgfpathrectangle{\pgfqpoint{0.100000in}{0.212622in}}{\pgfqpoint{3.696000in}{3.696000in}}%
\pgfusepath{clip}%
\pgfsetbuttcap%
\pgfsetroundjoin%
\definecolor{currentfill}{rgb}{0.121569,0.466667,0.705882}%
\pgfsetfillcolor{currentfill}%
\pgfsetfillopacity{0.336768}%
\pgfsetlinewidth{1.003750pt}%
\definecolor{currentstroke}{rgb}{0.121569,0.466667,0.705882}%
\pgfsetstrokecolor{currentstroke}%
\pgfsetstrokeopacity{0.336768}%
\pgfsetdash{}{0pt}%
\pgfpathmoveto{\pgfqpoint{1.569051in}{2.017459in}}%
\pgfpathcurveto{\pgfqpoint{1.577288in}{2.017459in}}{\pgfqpoint{1.585188in}{2.020732in}}{\pgfqpoint{1.591012in}{2.026556in}}%
\pgfpathcurveto{\pgfqpoint{1.596836in}{2.032380in}}{\pgfqpoint{1.600108in}{2.040280in}}{\pgfqpoint{1.600108in}{2.048516in}}%
\pgfpathcurveto{\pgfqpoint{1.600108in}{2.056752in}}{\pgfqpoint{1.596836in}{2.064652in}}{\pgfqpoint{1.591012in}{2.070476in}}%
\pgfpathcurveto{\pgfqpoint{1.585188in}{2.076300in}}{\pgfqpoint{1.577288in}{2.079572in}}{\pgfqpoint{1.569051in}{2.079572in}}%
\pgfpathcurveto{\pgfqpoint{1.560815in}{2.079572in}}{\pgfqpoint{1.552915in}{2.076300in}}{\pgfqpoint{1.547091in}{2.070476in}}%
\pgfpathcurveto{\pgfqpoint{1.541267in}{2.064652in}}{\pgfqpoint{1.537995in}{2.056752in}}{\pgfqpoint{1.537995in}{2.048516in}}%
\pgfpathcurveto{\pgfqpoint{1.537995in}{2.040280in}}{\pgfqpoint{1.541267in}{2.032380in}}{\pgfqpoint{1.547091in}{2.026556in}}%
\pgfpathcurveto{\pgfqpoint{1.552915in}{2.020732in}}{\pgfqpoint{1.560815in}{2.017459in}}{\pgfqpoint{1.569051in}{2.017459in}}%
\pgfpathclose%
\pgfusepath{stroke,fill}%
\end{pgfscope}%
\begin{pgfscope}%
\pgfpathrectangle{\pgfqpoint{0.100000in}{0.212622in}}{\pgfqpoint{3.696000in}{3.696000in}}%
\pgfusepath{clip}%
\pgfsetbuttcap%
\pgfsetroundjoin%
\definecolor{currentfill}{rgb}{0.121569,0.466667,0.705882}%
\pgfsetfillcolor{currentfill}%
\pgfsetfillopacity{0.336947}%
\pgfsetlinewidth{1.003750pt}%
\definecolor{currentstroke}{rgb}{0.121569,0.466667,0.705882}%
\pgfsetstrokecolor{currentstroke}%
\pgfsetstrokeopacity{0.336947}%
\pgfsetdash{}{0pt}%
\pgfpathmoveto{\pgfqpoint{1.568546in}{2.016757in}}%
\pgfpathcurveto{\pgfqpoint{1.576782in}{2.016757in}}{\pgfqpoint{1.584682in}{2.020029in}}{\pgfqpoint{1.590506in}{2.025853in}}%
\pgfpathcurveto{\pgfqpoint{1.596330in}{2.031677in}}{\pgfqpoint{1.599603in}{2.039577in}}{\pgfqpoint{1.599603in}{2.047813in}}%
\pgfpathcurveto{\pgfqpoint{1.599603in}{2.056050in}}{\pgfqpoint{1.596330in}{2.063950in}}{\pgfqpoint{1.590506in}{2.069774in}}%
\pgfpathcurveto{\pgfqpoint{1.584682in}{2.075598in}}{\pgfqpoint{1.576782in}{2.078870in}}{\pgfqpoint{1.568546in}{2.078870in}}%
\pgfpathcurveto{\pgfqpoint{1.560310in}{2.078870in}}{\pgfqpoint{1.552410in}{2.075598in}}{\pgfqpoint{1.546586in}{2.069774in}}%
\pgfpathcurveto{\pgfqpoint{1.540762in}{2.063950in}}{\pgfqpoint{1.537490in}{2.056050in}}{\pgfqpoint{1.537490in}{2.047813in}}%
\pgfpathcurveto{\pgfqpoint{1.537490in}{2.039577in}}{\pgfqpoint{1.540762in}{2.031677in}}{\pgfqpoint{1.546586in}{2.025853in}}%
\pgfpathcurveto{\pgfqpoint{1.552410in}{2.020029in}}{\pgfqpoint{1.560310in}{2.016757in}}{\pgfqpoint{1.568546in}{2.016757in}}%
\pgfpathclose%
\pgfusepath{stroke,fill}%
\end{pgfscope}%
\begin{pgfscope}%
\pgfpathrectangle{\pgfqpoint{0.100000in}{0.212622in}}{\pgfqpoint{3.696000in}{3.696000in}}%
\pgfusepath{clip}%
\pgfsetbuttcap%
\pgfsetroundjoin%
\definecolor{currentfill}{rgb}{0.121569,0.466667,0.705882}%
\pgfsetfillcolor{currentfill}%
\pgfsetfillopacity{0.337356}%
\pgfsetlinewidth{1.003750pt}%
\definecolor{currentstroke}{rgb}{0.121569,0.466667,0.705882}%
\pgfsetstrokecolor{currentstroke}%
\pgfsetstrokeopacity{0.337356}%
\pgfsetdash{}{0pt}%
\pgfpathmoveto{\pgfqpoint{1.567957in}{2.015850in}}%
\pgfpathcurveto{\pgfqpoint{1.576194in}{2.015850in}}{\pgfqpoint{1.584094in}{2.019123in}}{\pgfqpoint{1.589918in}{2.024947in}}%
\pgfpathcurveto{\pgfqpoint{1.595742in}{2.030770in}}{\pgfqpoint{1.599014in}{2.038670in}}{\pgfqpoint{1.599014in}{2.046907in}}%
\pgfpathcurveto{\pgfqpoint{1.599014in}{2.055143in}}{\pgfqpoint{1.595742in}{2.063043in}}{\pgfqpoint{1.589918in}{2.068867in}}%
\pgfpathcurveto{\pgfqpoint{1.584094in}{2.074691in}}{\pgfqpoint{1.576194in}{2.077963in}}{\pgfqpoint{1.567957in}{2.077963in}}%
\pgfpathcurveto{\pgfqpoint{1.559721in}{2.077963in}}{\pgfqpoint{1.551821in}{2.074691in}}{\pgfqpoint{1.545997in}{2.068867in}}%
\pgfpathcurveto{\pgfqpoint{1.540173in}{2.063043in}}{\pgfqpoint{1.536901in}{2.055143in}}{\pgfqpoint{1.536901in}{2.046907in}}%
\pgfpathcurveto{\pgfqpoint{1.536901in}{2.038670in}}{\pgfqpoint{1.540173in}{2.030770in}}{\pgfqpoint{1.545997in}{2.024947in}}%
\pgfpathcurveto{\pgfqpoint{1.551821in}{2.019123in}}{\pgfqpoint{1.559721in}{2.015850in}}{\pgfqpoint{1.567957in}{2.015850in}}%
\pgfpathclose%
\pgfusepath{stroke,fill}%
\end{pgfscope}%
\begin{pgfscope}%
\pgfpathrectangle{\pgfqpoint{0.100000in}{0.212622in}}{\pgfqpoint{3.696000in}{3.696000in}}%
\pgfusepath{clip}%
\pgfsetbuttcap%
\pgfsetroundjoin%
\definecolor{currentfill}{rgb}{0.121569,0.466667,0.705882}%
\pgfsetfillcolor{currentfill}%
\pgfsetfillopacity{0.337397}%
\pgfsetlinewidth{1.003750pt}%
\definecolor{currentstroke}{rgb}{0.121569,0.466667,0.705882}%
\pgfsetstrokecolor{currentstroke}%
\pgfsetstrokeopacity{0.337397}%
\pgfsetdash{}{0pt}%
\pgfpathmoveto{\pgfqpoint{1.567837in}{2.015517in}}%
\pgfpathcurveto{\pgfqpoint{1.576073in}{2.015517in}}{\pgfqpoint{1.583973in}{2.018789in}}{\pgfqpoint{1.589797in}{2.024613in}}%
\pgfpathcurveto{\pgfqpoint{1.595621in}{2.030437in}}{\pgfqpoint{1.598893in}{2.038337in}}{\pgfqpoint{1.598893in}{2.046573in}}%
\pgfpathcurveto{\pgfqpoint{1.598893in}{2.054810in}}{\pgfqpoint{1.595621in}{2.062710in}}{\pgfqpoint{1.589797in}{2.068534in}}%
\pgfpathcurveto{\pgfqpoint{1.583973in}{2.074358in}}{\pgfqpoint{1.576073in}{2.077630in}}{\pgfqpoint{1.567837in}{2.077630in}}%
\pgfpathcurveto{\pgfqpoint{1.559601in}{2.077630in}}{\pgfqpoint{1.551701in}{2.074358in}}{\pgfqpoint{1.545877in}{2.068534in}}%
\pgfpathcurveto{\pgfqpoint{1.540053in}{2.062710in}}{\pgfqpoint{1.536780in}{2.054810in}}{\pgfqpoint{1.536780in}{2.046573in}}%
\pgfpathcurveto{\pgfqpoint{1.536780in}{2.038337in}}{\pgfqpoint{1.540053in}{2.030437in}}{\pgfqpoint{1.545877in}{2.024613in}}%
\pgfpathcurveto{\pgfqpoint{1.551701in}{2.018789in}}{\pgfqpoint{1.559601in}{2.015517in}}{\pgfqpoint{1.567837in}{2.015517in}}%
\pgfpathclose%
\pgfusepath{stroke,fill}%
\end{pgfscope}%
\begin{pgfscope}%
\pgfpathrectangle{\pgfqpoint{0.100000in}{0.212622in}}{\pgfqpoint{3.696000in}{3.696000in}}%
\pgfusepath{clip}%
\pgfsetbuttcap%
\pgfsetroundjoin%
\definecolor{currentfill}{rgb}{0.121569,0.466667,0.705882}%
\pgfsetfillcolor{currentfill}%
\pgfsetfillopacity{0.337504}%
\pgfsetlinewidth{1.003750pt}%
\definecolor{currentstroke}{rgb}{0.121569,0.466667,0.705882}%
\pgfsetstrokecolor{currentstroke}%
\pgfsetstrokeopacity{0.337504}%
\pgfsetdash{}{0pt}%
\pgfpathmoveto{\pgfqpoint{1.567566in}{2.015194in}}%
\pgfpathcurveto{\pgfqpoint{1.575802in}{2.015194in}}{\pgfqpoint{1.583702in}{2.018466in}}{\pgfqpoint{1.589526in}{2.024290in}}%
\pgfpathcurveto{\pgfqpoint{1.595350in}{2.030114in}}{\pgfqpoint{1.598622in}{2.038014in}}{\pgfqpoint{1.598622in}{2.046250in}}%
\pgfpathcurveto{\pgfqpoint{1.598622in}{2.054487in}}{\pgfqpoint{1.595350in}{2.062387in}}{\pgfqpoint{1.589526in}{2.068211in}}%
\pgfpathcurveto{\pgfqpoint{1.583702in}{2.074035in}}{\pgfqpoint{1.575802in}{2.077307in}}{\pgfqpoint{1.567566in}{2.077307in}}%
\pgfpathcurveto{\pgfqpoint{1.559329in}{2.077307in}}{\pgfqpoint{1.551429in}{2.074035in}}{\pgfqpoint{1.545605in}{2.068211in}}%
\pgfpathcurveto{\pgfqpoint{1.539781in}{2.062387in}}{\pgfqpoint{1.536509in}{2.054487in}}{\pgfqpoint{1.536509in}{2.046250in}}%
\pgfpathcurveto{\pgfqpoint{1.536509in}{2.038014in}}{\pgfqpoint{1.539781in}{2.030114in}}{\pgfqpoint{1.545605in}{2.024290in}}%
\pgfpathcurveto{\pgfqpoint{1.551429in}{2.018466in}}{\pgfqpoint{1.559329in}{2.015194in}}{\pgfqpoint{1.567566in}{2.015194in}}%
\pgfpathclose%
\pgfusepath{stroke,fill}%
\end{pgfscope}%
\begin{pgfscope}%
\pgfpathrectangle{\pgfqpoint{0.100000in}{0.212622in}}{\pgfqpoint{3.696000in}{3.696000in}}%
\pgfusepath{clip}%
\pgfsetbuttcap%
\pgfsetroundjoin%
\definecolor{currentfill}{rgb}{0.121569,0.466667,0.705882}%
\pgfsetfillcolor{currentfill}%
\pgfsetfillopacity{0.337753}%
\pgfsetlinewidth{1.003750pt}%
\definecolor{currentstroke}{rgb}{0.121569,0.466667,0.705882}%
\pgfsetstrokecolor{currentstroke}%
\pgfsetstrokeopacity{0.337753}%
\pgfsetdash{}{0pt}%
\pgfpathmoveto{\pgfqpoint{1.567257in}{2.014878in}}%
\pgfpathcurveto{\pgfqpoint{1.575493in}{2.014878in}}{\pgfqpoint{1.583393in}{2.018150in}}{\pgfqpoint{1.589217in}{2.023974in}}%
\pgfpathcurveto{\pgfqpoint{1.595041in}{2.029798in}}{\pgfqpoint{1.598314in}{2.037698in}}{\pgfqpoint{1.598314in}{2.045935in}}%
\pgfpathcurveto{\pgfqpoint{1.598314in}{2.054171in}}{\pgfqpoint{1.595041in}{2.062071in}}{\pgfqpoint{1.589217in}{2.067895in}}%
\pgfpathcurveto{\pgfqpoint{1.583393in}{2.073719in}}{\pgfqpoint{1.575493in}{2.076991in}}{\pgfqpoint{1.567257in}{2.076991in}}%
\pgfpathcurveto{\pgfqpoint{1.559021in}{2.076991in}}{\pgfqpoint{1.551121in}{2.073719in}}{\pgfqpoint{1.545297in}{2.067895in}}%
\pgfpathcurveto{\pgfqpoint{1.539473in}{2.062071in}}{\pgfqpoint{1.536201in}{2.054171in}}{\pgfqpoint{1.536201in}{2.045935in}}%
\pgfpathcurveto{\pgfqpoint{1.536201in}{2.037698in}}{\pgfqpoint{1.539473in}{2.029798in}}{\pgfqpoint{1.545297in}{2.023974in}}%
\pgfpathcurveto{\pgfqpoint{1.551121in}{2.018150in}}{\pgfqpoint{1.559021in}{2.014878in}}{\pgfqpoint{1.567257in}{2.014878in}}%
\pgfpathclose%
\pgfusepath{stroke,fill}%
\end{pgfscope}%
\begin{pgfscope}%
\pgfpathrectangle{\pgfqpoint{0.100000in}{0.212622in}}{\pgfqpoint{3.696000in}{3.696000in}}%
\pgfusepath{clip}%
\pgfsetbuttcap%
\pgfsetroundjoin%
\definecolor{currentfill}{rgb}{0.121569,0.466667,0.705882}%
\pgfsetfillcolor{currentfill}%
\pgfsetfillopacity{0.337798}%
\pgfsetlinewidth{1.003750pt}%
\definecolor{currentstroke}{rgb}{0.121569,0.466667,0.705882}%
\pgfsetstrokecolor{currentstroke}%
\pgfsetstrokeopacity{0.337798}%
\pgfsetdash{}{0pt}%
\pgfpathmoveto{\pgfqpoint{1.567052in}{2.014634in}}%
\pgfpathcurveto{\pgfqpoint{1.575289in}{2.014634in}}{\pgfqpoint{1.583189in}{2.017907in}}{\pgfqpoint{1.589013in}{2.023731in}}%
\pgfpathcurveto{\pgfqpoint{1.594837in}{2.029555in}}{\pgfqpoint{1.598109in}{2.037455in}}{\pgfqpoint{1.598109in}{2.045691in}}%
\pgfpathcurveto{\pgfqpoint{1.598109in}{2.053927in}}{\pgfqpoint{1.594837in}{2.061827in}}{\pgfqpoint{1.589013in}{2.067651in}}%
\pgfpathcurveto{\pgfqpoint{1.583189in}{2.073475in}}{\pgfqpoint{1.575289in}{2.076747in}}{\pgfqpoint{1.567052in}{2.076747in}}%
\pgfpathcurveto{\pgfqpoint{1.558816in}{2.076747in}}{\pgfqpoint{1.550916in}{2.073475in}}{\pgfqpoint{1.545092in}{2.067651in}}%
\pgfpathcurveto{\pgfqpoint{1.539268in}{2.061827in}}{\pgfqpoint{1.535996in}{2.053927in}}{\pgfqpoint{1.535996in}{2.045691in}}%
\pgfpathcurveto{\pgfqpoint{1.535996in}{2.037455in}}{\pgfqpoint{1.539268in}{2.029555in}}{\pgfqpoint{1.545092in}{2.023731in}}%
\pgfpathcurveto{\pgfqpoint{1.550916in}{2.017907in}}{\pgfqpoint{1.558816in}{2.014634in}}{\pgfqpoint{1.567052in}{2.014634in}}%
\pgfpathclose%
\pgfusepath{stroke,fill}%
\end{pgfscope}%
\begin{pgfscope}%
\pgfpathrectangle{\pgfqpoint{0.100000in}{0.212622in}}{\pgfqpoint{3.696000in}{3.696000in}}%
\pgfusepath{clip}%
\pgfsetbuttcap%
\pgfsetroundjoin%
\definecolor{currentfill}{rgb}{0.121569,0.466667,0.705882}%
\pgfsetfillcolor{currentfill}%
\pgfsetfillopacity{0.337936}%
\pgfsetlinewidth{1.003750pt}%
\definecolor{currentstroke}{rgb}{0.121569,0.466667,0.705882}%
\pgfsetstrokecolor{currentstroke}%
\pgfsetstrokeopacity{0.337936}%
\pgfsetdash{}{0pt}%
\pgfpathmoveto{\pgfqpoint{1.566804in}{2.014486in}}%
\pgfpathcurveto{\pgfqpoint{1.575041in}{2.014486in}}{\pgfqpoint{1.582941in}{2.017759in}}{\pgfqpoint{1.588765in}{2.023582in}}%
\pgfpathcurveto{\pgfqpoint{1.594588in}{2.029406in}}{\pgfqpoint{1.597861in}{2.037306in}}{\pgfqpoint{1.597861in}{2.045543in}}%
\pgfpathcurveto{\pgfqpoint{1.597861in}{2.053779in}}{\pgfqpoint{1.594588in}{2.061679in}}{\pgfqpoint{1.588765in}{2.067503in}}%
\pgfpathcurveto{\pgfqpoint{1.582941in}{2.073327in}}{\pgfqpoint{1.575041in}{2.076599in}}{\pgfqpoint{1.566804in}{2.076599in}}%
\pgfpathcurveto{\pgfqpoint{1.558568in}{2.076599in}}{\pgfqpoint{1.550668in}{2.073327in}}{\pgfqpoint{1.544844in}{2.067503in}}%
\pgfpathcurveto{\pgfqpoint{1.539020in}{2.061679in}}{\pgfqpoint{1.535748in}{2.053779in}}{\pgfqpoint{1.535748in}{2.045543in}}%
\pgfpathcurveto{\pgfqpoint{1.535748in}{2.037306in}}{\pgfqpoint{1.539020in}{2.029406in}}{\pgfqpoint{1.544844in}{2.023582in}}%
\pgfpathcurveto{\pgfqpoint{1.550668in}{2.017759in}}{\pgfqpoint{1.558568in}{2.014486in}}{\pgfqpoint{1.566804in}{2.014486in}}%
\pgfpathclose%
\pgfusepath{stroke,fill}%
\end{pgfscope}%
\begin{pgfscope}%
\pgfpathrectangle{\pgfqpoint{0.100000in}{0.212622in}}{\pgfqpoint{3.696000in}{3.696000in}}%
\pgfusepath{clip}%
\pgfsetbuttcap%
\pgfsetroundjoin%
\definecolor{currentfill}{rgb}{0.121569,0.466667,0.705882}%
\pgfsetfillcolor{currentfill}%
\pgfsetfillopacity{0.337938}%
\pgfsetlinewidth{1.003750pt}%
\definecolor{currentstroke}{rgb}{0.121569,0.466667,0.705882}%
\pgfsetstrokecolor{currentstroke}%
\pgfsetstrokeopacity{0.337938}%
\pgfsetdash{}{0pt}%
\pgfpathmoveto{\pgfqpoint{1.881809in}{2.126705in}}%
\pgfpathcurveto{\pgfqpoint{1.890046in}{2.126705in}}{\pgfqpoint{1.897946in}{2.129978in}}{\pgfqpoint{1.903770in}{2.135802in}}%
\pgfpathcurveto{\pgfqpoint{1.909594in}{2.141625in}}{\pgfqpoint{1.912866in}{2.149526in}}{\pgfqpoint{1.912866in}{2.157762in}}%
\pgfpathcurveto{\pgfqpoint{1.912866in}{2.165998in}}{\pgfqpoint{1.909594in}{2.173898in}}{\pgfqpoint{1.903770in}{2.179722in}}%
\pgfpathcurveto{\pgfqpoint{1.897946in}{2.185546in}}{\pgfqpoint{1.890046in}{2.188818in}}{\pgfqpoint{1.881809in}{2.188818in}}%
\pgfpathcurveto{\pgfqpoint{1.873573in}{2.188818in}}{\pgfqpoint{1.865673in}{2.185546in}}{\pgfqpoint{1.859849in}{2.179722in}}%
\pgfpathcurveto{\pgfqpoint{1.854025in}{2.173898in}}{\pgfqpoint{1.850753in}{2.165998in}}{\pgfqpoint{1.850753in}{2.157762in}}%
\pgfpathcurveto{\pgfqpoint{1.850753in}{2.149526in}}{\pgfqpoint{1.854025in}{2.141625in}}{\pgfqpoint{1.859849in}{2.135802in}}%
\pgfpathcurveto{\pgfqpoint{1.865673in}{2.129978in}}{\pgfqpoint{1.873573in}{2.126705in}}{\pgfqpoint{1.881809in}{2.126705in}}%
\pgfpathclose%
\pgfusepath{stroke,fill}%
\end{pgfscope}%
\begin{pgfscope}%
\pgfpathrectangle{\pgfqpoint{0.100000in}{0.212622in}}{\pgfqpoint{3.696000in}{3.696000in}}%
\pgfusepath{clip}%
\pgfsetbuttcap%
\pgfsetroundjoin%
\definecolor{currentfill}{rgb}{0.121569,0.466667,0.705882}%
\pgfsetfillcolor{currentfill}%
\pgfsetfillopacity{0.338133}%
\pgfsetlinewidth{1.003750pt}%
\definecolor{currentstroke}{rgb}{0.121569,0.466667,0.705882}%
\pgfsetstrokecolor{currentstroke}%
\pgfsetstrokeopacity{0.338133}%
\pgfsetdash{}{0pt}%
\pgfpathmoveto{\pgfqpoint{1.566500in}{2.013711in}}%
\pgfpathcurveto{\pgfqpoint{1.574736in}{2.013711in}}{\pgfqpoint{1.582636in}{2.016983in}}{\pgfqpoint{1.588460in}{2.022807in}}%
\pgfpathcurveto{\pgfqpoint{1.594284in}{2.028631in}}{\pgfqpoint{1.597556in}{2.036531in}}{\pgfqpoint{1.597556in}{2.044767in}}%
\pgfpathcurveto{\pgfqpoint{1.597556in}{2.053004in}}{\pgfqpoint{1.594284in}{2.060904in}}{\pgfqpoint{1.588460in}{2.066728in}}%
\pgfpathcurveto{\pgfqpoint{1.582636in}{2.072552in}}{\pgfqpoint{1.574736in}{2.075824in}}{\pgfqpoint{1.566500in}{2.075824in}}%
\pgfpathcurveto{\pgfqpoint{1.558264in}{2.075824in}}{\pgfqpoint{1.550364in}{2.072552in}}{\pgfqpoint{1.544540in}{2.066728in}}%
\pgfpathcurveto{\pgfqpoint{1.538716in}{2.060904in}}{\pgfqpoint{1.535443in}{2.053004in}}{\pgfqpoint{1.535443in}{2.044767in}}%
\pgfpathcurveto{\pgfqpoint{1.535443in}{2.036531in}}{\pgfqpoint{1.538716in}{2.028631in}}{\pgfqpoint{1.544540in}{2.022807in}}%
\pgfpathcurveto{\pgfqpoint{1.550364in}{2.016983in}}{\pgfqpoint{1.558264in}{2.013711in}}{\pgfqpoint{1.566500in}{2.013711in}}%
\pgfpathclose%
\pgfusepath{stroke,fill}%
\end{pgfscope}%
\begin{pgfscope}%
\pgfpathrectangle{\pgfqpoint{0.100000in}{0.212622in}}{\pgfqpoint{3.696000in}{3.696000in}}%
\pgfusepath{clip}%
\pgfsetbuttcap%
\pgfsetroundjoin%
\definecolor{currentfill}{rgb}{0.121569,0.466667,0.705882}%
\pgfsetfillcolor{currentfill}%
\pgfsetfillopacity{0.338233}%
\pgfsetlinewidth{1.003750pt}%
\definecolor{currentstroke}{rgb}{0.121569,0.466667,0.705882}%
\pgfsetstrokecolor{currentstroke}%
\pgfsetstrokeopacity{0.338233}%
\pgfsetdash{}{0pt}%
\pgfpathmoveto{\pgfqpoint{1.566173in}{2.013535in}}%
\pgfpathcurveto{\pgfqpoint{1.574409in}{2.013535in}}{\pgfqpoint{1.582309in}{2.016807in}}{\pgfqpoint{1.588133in}{2.022631in}}%
\pgfpathcurveto{\pgfqpoint{1.593957in}{2.028455in}}{\pgfqpoint{1.597230in}{2.036355in}}{\pgfqpoint{1.597230in}{2.044592in}}%
\pgfpathcurveto{\pgfqpoint{1.597230in}{2.052828in}}{\pgfqpoint{1.593957in}{2.060728in}}{\pgfqpoint{1.588133in}{2.066552in}}%
\pgfpathcurveto{\pgfqpoint{1.582309in}{2.072376in}}{\pgfqpoint{1.574409in}{2.075648in}}{\pgfqpoint{1.566173in}{2.075648in}}%
\pgfpathcurveto{\pgfqpoint{1.557937in}{2.075648in}}{\pgfqpoint{1.550037in}{2.072376in}}{\pgfqpoint{1.544213in}{2.066552in}}%
\pgfpathcurveto{\pgfqpoint{1.538389in}{2.060728in}}{\pgfqpoint{1.535117in}{2.052828in}}{\pgfqpoint{1.535117in}{2.044592in}}%
\pgfpathcurveto{\pgfqpoint{1.535117in}{2.036355in}}{\pgfqpoint{1.538389in}{2.028455in}}{\pgfqpoint{1.544213in}{2.022631in}}%
\pgfpathcurveto{\pgfqpoint{1.550037in}{2.016807in}}{\pgfqpoint{1.557937in}{2.013535in}}{\pgfqpoint{1.566173in}{2.013535in}}%
\pgfpathclose%
\pgfusepath{stroke,fill}%
\end{pgfscope}%
\begin{pgfscope}%
\pgfpathrectangle{\pgfqpoint{0.100000in}{0.212622in}}{\pgfqpoint{3.696000in}{3.696000in}}%
\pgfusepath{clip}%
\pgfsetbuttcap%
\pgfsetroundjoin%
\definecolor{currentfill}{rgb}{0.121569,0.466667,0.705882}%
\pgfsetfillcolor{currentfill}%
\pgfsetfillopacity{0.338507}%
\pgfsetlinewidth{1.003750pt}%
\definecolor{currentstroke}{rgb}{0.121569,0.466667,0.705882}%
\pgfsetstrokecolor{currentstroke}%
\pgfsetstrokeopacity{0.338507}%
\pgfsetdash{}{0pt}%
\pgfpathmoveto{\pgfqpoint{1.565982in}{2.013576in}}%
\pgfpathcurveto{\pgfqpoint{1.574218in}{2.013576in}}{\pgfqpoint{1.582119in}{2.016848in}}{\pgfqpoint{1.587942in}{2.022672in}}%
\pgfpathcurveto{\pgfqpoint{1.593766in}{2.028496in}}{\pgfqpoint{1.597039in}{2.036396in}}{\pgfqpoint{1.597039in}{2.044632in}}%
\pgfpathcurveto{\pgfqpoint{1.597039in}{2.052868in}}{\pgfqpoint{1.593766in}{2.060769in}}{\pgfqpoint{1.587942in}{2.066592in}}%
\pgfpathcurveto{\pgfqpoint{1.582119in}{2.072416in}}{\pgfqpoint{1.574218in}{2.075689in}}{\pgfqpoint{1.565982in}{2.075689in}}%
\pgfpathcurveto{\pgfqpoint{1.557746in}{2.075689in}}{\pgfqpoint{1.549846in}{2.072416in}}{\pgfqpoint{1.544022in}{2.066592in}}%
\pgfpathcurveto{\pgfqpoint{1.538198in}{2.060769in}}{\pgfqpoint{1.534926in}{2.052868in}}{\pgfqpoint{1.534926in}{2.044632in}}%
\pgfpathcurveto{\pgfqpoint{1.534926in}{2.036396in}}{\pgfqpoint{1.538198in}{2.028496in}}{\pgfqpoint{1.544022in}{2.022672in}}%
\pgfpathcurveto{\pgfqpoint{1.549846in}{2.016848in}}{\pgfqpoint{1.557746in}{2.013576in}}{\pgfqpoint{1.565982in}{2.013576in}}%
\pgfpathclose%
\pgfusepath{stroke,fill}%
\end{pgfscope}%
\begin{pgfscope}%
\pgfpathrectangle{\pgfqpoint{0.100000in}{0.212622in}}{\pgfqpoint{3.696000in}{3.696000in}}%
\pgfusepath{clip}%
\pgfsetbuttcap%
\pgfsetroundjoin%
\definecolor{currentfill}{rgb}{0.121569,0.466667,0.705882}%
\pgfsetfillcolor{currentfill}%
\pgfsetfillopacity{0.338734}%
\pgfsetlinewidth{1.003750pt}%
\definecolor{currentstroke}{rgb}{0.121569,0.466667,0.705882}%
\pgfsetstrokecolor{currentstroke}%
\pgfsetstrokeopacity{0.338734}%
\pgfsetdash{}{0pt}%
\pgfpathmoveto{\pgfqpoint{1.565391in}{2.011733in}}%
\pgfpathcurveto{\pgfqpoint{1.573627in}{2.011733in}}{\pgfqpoint{1.581527in}{2.015006in}}{\pgfqpoint{1.587351in}{2.020830in}}%
\pgfpathcurveto{\pgfqpoint{1.593175in}{2.026653in}}{\pgfqpoint{1.596447in}{2.034554in}}{\pgfqpoint{1.596447in}{2.042790in}}%
\pgfpathcurveto{\pgfqpoint{1.596447in}{2.051026in}}{\pgfqpoint{1.593175in}{2.058926in}}{\pgfqpoint{1.587351in}{2.064750in}}%
\pgfpathcurveto{\pgfqpoint{1.581527in}{2.070574in}}{\pgfqpoint{1.573627in}{2.073846in}}{\pgfqpoint{1.565391in}{2.073846in}}%
\pgfpathcurveto{\pgfqpoint{1.557154in}{2.073846in}}{\pgfqpoint{1.549254in}{2.070574in}}{\pgfqpoint{1.543430in}{2.064750in}}%
\pgfpathcurveto{\pgfqpoint{1.537606in}{2.058926in}}{\pgfqpoint{1.534334in}{2.051026in}}{\pgfqpoint{1.534334in}{2.042790in}}%
\pgfpathcurveto{\pgfqpoint{1.534334in}{2.034554in}}{\pgfqpoint{1.537606in}{2.026653in}}{\pgfqpoint{1.543430in}{2.020830in}}%
\pgfpathcurveto{\pgfqpoint{1.549254in}{2.015006in}}{\pgfqpoint{1.557154in}{2.011733in}}{\pgfqpoint{1.565391in}{2.011733in}}%
\pgfpathclose%
\pgfusepath{stroke,fill}%
\end{pgfscope}%
\begin{pgfscope}%
\pgfpathrectangle{\pgfqpoint{0.100000in}{0.212622in}}{\pgfqpoint{3.696000in}{3.696000in}}%
\pgfusepath{clip}%
\pgfsetbuttcap%
\pgfsetroundjoin%
\definecolor{currentfill}{rgb}{0.121569,0.466667,0.705882}%
\pgfsetfillcolor{currentfill}%
\pgfsetfillopacity{0.338886}%
\pgfsetlinewidth{1.003750pt}%
\definecolor{currentstroke}{rgb}{0.121569,0.466667,0.705882}%
\pgfsetstrokecolor{currentstroke}%
\pgfsetstrokeopacity{0.338886}%
\pgfsetdash{}{0pt}%
\pgfpathmoveto{\pgfqpoint{1.891920in}{2.118691in}}%
\pgfpathcurveto{\pgfqpoint{1.900156in}{2.118691in}}{\pgfqpoint{1.908056in}{2.121964in}}{\pgfqpoint{1.913880in}{2.127788in}}%
\pgfpathcurveto{\pgfqpoint{1.919704in}{2.133612in}}{\pgfqpoint{1.922976in}{2.141512in}}{\pgfqpoint{1.922976in}{2.149748in}}%
\pgfpathcurveto{\pgfqpoint{1.922976in}{2.157984in}}{\pgfqpoint{1.919704in}{2.165884in}}{\pgfqpoint{1.913880in}{2.171708in}}%
\pgfpathcurveto{\pgfqpoint{1.908056in}{2.177532in}}{\pgfqpoint{1.900156in}{2.180804in}}{\pgfqpoint{1.891920in}{2.180804in}}%
\pgfpathcurveto{\pgfqpoint{1.883684in}{2.180804in}}{\pgfqpoint{1.875784in}{2.177532in}}{\pgfqpoint{1.869960in}{2.171708in}}%
\pgfpathcurveto{\pgfqpoint{1.864136in}{2.165884in}}{\pgfqpoint{1.860863in}{2.157984in}}{\pgfqpoint{1.860863in}{2.149748in}}%
\pgfpathcurveto{\pgfqpoint{1.860863in}{2.141512in}}{\pgfqpoint{1.864136in}{2.133612in}}{\pgfqpoint{1.869960in}{2.127788in}}%
\pgfpathcurveto{\pgfqpoint{1.875784in}{2.121964in}}{\pgfqpoint{1.883684in}{2.118691in}}{\pgfqpoint{1.891920in}{2.118691in}}%
\pgfpathclose%
\pgfusepath{stroke,fill}%
\end{pgfscope}%
\begin{pgfscope}%
\pgfpathrectangle{\pgfqpoint{0.100000in}{0.212622in}}{\pgfqpoint{3.696000in}{3.696000in}}%
\pgfusepath{clip}%
\pgfsetbuttcap%
\pgfsetroundjoin%
\definecolor{currentfill}{rgb}{0.121569,0.466667,0.705882}%
\pgfsetfillcolor{currentfill}%
\pgfsetfillopacity{0.339462}%
\pgfsetlinewidth{1.003750pt}%
\definecolor{currentstroke}{rgb}{0.121569,0.466667,0.705882}%
\pgfsetstrokecolor{currentstroke}%
\pgfsetstrokeopacity{0.339462}%
\pgfsetdash{}{0pt}%
\pgfpathmoveto{\pgfqpoint{1.563687in}{2.011267in}}%
\pgfpathcurveto{\pgfqpoint{1.571923in}{2.011267in}}{\pgfqpoint{1.579823in}{2.014539in}}{\pgfqpoint{1.585647in}{2.020363in}}%
\pgfpathcurveto{\pgfqpoint{1.591471in}{2.026187in}}{\pgfqpoint{1.594743in}{2.034087in}}{\pgfqpoint{1.594743in}{2.042323in}}%
\pgfpathcurveto{\pgfqpoint{1.594743in}{2.050559in}}{\pgfqpoint{1.591471in}{2.058460in}}{\pgfqpoint{1.585647in}{2.064283in}}%
\pgfpathcurveto{\pgfqpoint{1.579823in}{2.070107in}}{\pgfqpoint{1.571923in}{2.073380in}}{\pgfqpoint{1.563687in}{2.073380in}}%
\pgfpathcurveto{\pgfqpoint{1.555450in}{2.073380in}}{\pgfqpoint{1.547550in}{2.070107in}}{\pgfqpoint{1.541726in}{2.064283in}}%
\pgfpathcurveto{\pgfqpoint{1.535902in}{2.058460in}}{\pgfqpoint{1.532630in}{2.050559in}}{\pgfqpoint{1.532630in}{2.042323in}}%
\pgfpathcurveto{\pgfqpoint{1.532630in}{2.034087in}}{\pgfqpoint{1.535902in}{2.026187in}}{\pgfqpoint{1.541726in}{2.020363in}}%
\pgfpathcurveto{\pgfqpoint{1.547550in}{2.014539in}}{\pgfqpoint{1.555450in}{2.011267in}}{\pgfqpoint{1.563687in}{2.011267in}}%
\pgfpathclose%
\pgfusepath{stroke,fill}%
\end{pgfscope}%
\begin{pgfscope}%
\pgfpathrectangle{\pgfqpoint{0.100000in}{0.212622in}}{\pgfqpoint{3.696000in}{3.696000in}}%
\pgfusepath{clip}%
\pgfsetbuttcap%
\pgfsetroundjoin%
\definecolor{currentfill}{rgb}{0.121569,0.466667,0.705882}%
\pgfsetfillcolor{currentfill}%
\pgfsetfillopacity{0.339839}%
\pgfsetlinewidth{1.003750pt}%
\definecolor{currentstroke}{rgb}{0.121569,0.466667,0.705882}%
\pgfsetstrokecolor{currentstroke}%
\pgfsetstrokeopacity{0.339839}%
\pgfsetdash{}{0pt}%
\pgfpathmoveto{\pgfqpoint{1.563225in}{2.010283in}}%
\pgfpathcurveto{\pgfqpoint{1.571461in}{2.010283in}}{\pgfqpoint{1.579361in}{2.013555in}}{\pgfqpoint{1.585185in}{2.019379in}}%
\pgfpathcurveto{\pgfqpoint{1.591009in}{2.025203in}}{\pgfqpoint{1.594282in}{2.033103in}}{\pgfqpoint{1.594282in}{2.041340in}}%
\pgfpathcurveto{\pgfqpoint{1.594282in}{2.049576in}}{\pgfqpoint{1.591009in}{2.057476in}}{\pgfqpoint{1.585185in}{2.063300in}}%
\pgfpathcurveto{\pgfqpoint{1.579361in}{2.069124in}}{\pgfqpoint{1.571461in}{2.072396in}}{\pgfqpoint{1.563225in}{2.072396in}}%
\pgfpathcurveto{\pgfqpoint{1.554989in}{2.072396in}}{\pgfqpoint{1.547089in}{2.069124in}}{\pgfqpoint{1.541265in}{2.063300in}}%
\pgfpathcurveto{\pgfqpoint{1.535441in}{2.057476in}}{\pgfqpoint{1.532169in}{2.049576in}}{\pgfqpoint{1.532169in}{2.041340in}}%
\pgfpathcurveto{\pgfqpoint{1.532169in}{2.033103in}}{\pgfqpoint{1.535441in}{2.025203in}}{\pgfqpoint{1.541265in}{2.019379in}}%
\pgfpathcurveto{\pgfqpoint{1.547089in}{2.013555in}}{\pgfqpoint{1.554989in}{2.010283in}}{\pgfqpoint{1.563225in}{2.010283in}}%
\pgfpathclose%
\pgfusepath{stroke,fill}%
\end{pgfscope}%
\begin{pgfscope}%
\pgfpathrectangle{\pgfqpoint{0.100000in}{0.212622in}}{\pgfqpoint{3.696000in}{3.696000in}}%
\pgfusepath{clip}%
\pgfsetbuttcap%
\pgfsetroundjoin%
\definecolor{currentfill}{rgb}{0.121569,0.466667,0.705882}%
\pgfsetfillcolor{currentfill}%
\pgfsetfillopacity{0.340104}%
\pgfsetlinewidth{1.003750pt}%
\definecolor{currentstroke}{rgb}{0.121569,0.466667,0.705882}%
\pgfsetstrokecolor{currentstroke}%
\pgfsetstrokeopacity{0.340104}%
\pgfsetdash{}{0pt}%
\pgfpathmoveto{\pgfqpoint{1.562667in}{2.009115in}}%
\pgfpathcurveto{\pgfqpoint{1.570903in}{2.009115in}}{\pgfqpoint{1.578803in}{2.012387in}}{\pgfqpoint{1.584627in}{2.018211in}}%
\pgfpathcurveto{\pgfqpoint{1.590451in}{2.024035in}}{\pgfqpoint{1.593723in}{2.031935in}}{\pgfqpoint{1.593723in}{2.040172in}}%
\pgfpathcurveto{\pgfqpoint{1.593723in}{2.048408in}}{\pgfqpoint{1.590451in}{2.056308in}}{\pgfqpoint{1.584627in}{2.062132in}}%
\pgfpathcurveto{\pgfqpoint{1.578803in}{2.067956in}}{\pgfqpoint{1.570903in}{2.071228in}}{\pgfqpoint{1.562667in}{2.071228in}}%
\pgfpathcurveto{\pgfqpoint{1.554430in}{2.071228in}}{\pgfqpoint{1.546530in}{2.067956in}}{\pgfqpoint{1.540706in}{2.062132in}}%
\pgfpathcurveto{\pgfqpoint{1.534882in}{2.056308in}}{\pgfqpoint{1.531610in}{2.048408in}}{\pgfqpoint{1.531610in}{2.040172in}}%
\pgfpathcurveto{\pgfqpoint{1.531610in}{2.031935in}}{\pgfqpoint{1.534882in}{2.024035in}}{\pgfqpoint{1.540706in}{2.018211in}}%
\pgfpathcurveto{\pgfqpoint{1.546530in}{2.012387in}}{\pgfqpoint{1.554430in}{2.009115in}}{\pgfqpoint{1.562667in}{2.009115in}}%
\pgfpathclose%
\pgfusepath{stroke,fill}%
\end{pgfscope}%
\begin{pgfscope}%
\pgfpathrectangle{\pgfqpoint{0.100000in}{0.212622in}}{\pgfqpoint{3.696000in}{3.696000in}}%
\pgfusepath{clip}%
\pgfsetbuttcap%
\pgfsetroundjoin%
\definecolor{currentfill}{rgb}{0.121569,0.466667,0.705882}%
\pgfsetfillcolor{currentfill}%
\pgfsetfillopacity{0.340115}%
\pgfsetlinewidth{1.003750pt}%
\definecolor{currentstroke}{rgb}{0.121569,0.466667,0.705882}%
\pgfsetstrokecolor{currentstroke}%
\pgfsetstrokeopacity{0.340115}%
\pgfsetdash{}{0pt}%
\pgfpathmoveto{\pgfqpoint{1.898308in}{2.122536in}}%
\pgfpathcurveto{\pgfqpoint{1.906545in}{2.122536in}}{\pgfqpoint{1.914445in}{2.125808in}}{\pgfqpoint{1.920269in}{2.131632in}}%
\pgfpathcurveto{\pgfqpoint{1.926092in}{2.137456in}}{\pgfqpoint{1.929365in}{2.145356in}}{\pgfqpoint{1.929365in}{2.153593in}}%
\pgfpathcurveto{\pgfqpoint{1.929365in}{2.161829in}}{\pgfqpoint{1.926092in}{2.169729in}}{\pgfqpoint{1.920269in}{2.175553in}}%
\pgfpathcurveto{\pgfqpoint{1.914445in}{2.181377in}}{\pgfqpoint{1.906545in}{2.184649in}}{\pgfqpoint{1.898308in}{2.184649in}}%
\pgfpathcurveto{\pgfqpoint{1.890072in}{2.184649in}}{\pgfqpoint{1.882172in}{2.181377in}}{\pgfqpoint{1.876348in}{2.175553in}}%
\pgfpathcurveto{\pgfqpoint{1.870524in}{2.169729in}}{\pgfqpoint{1.867252in}{2.161829in}}{\pgfqpoint{1.867252in}{2.153593in}}%
\pgfpathcurveto{\pgfqpoint{1.867252in}{2.145356in}}{\pgfqpoint{1.870524in}{2.137456in}}{\pgfqpoint{1.876348in}{2.131632in}}%
\pgfpathcurveto{\pgfqpoint{1.882172in}{2.125808in}}{\pgfqpoint{1.890072in}{2.122536in}}{\pgfqpoint{1.898308in}{2.122536in}}%
\pgfpathclose%
\pgfusepath{stroke,fill}%
\end{pgfscope}%
\begin{pgfscope}%
\pgfpathrectangle{\pgfqpoint{0.100000in}{0.212622in}}{\pgfqpoint{3.696000in}{3.696000in}}%
\pgfusepath{clip}%
\pgfsetbuttcap%
\pgfsetroundjoin%
\definecolor{currentfill}{rgb}{0.121569,0.466667,0.705882}%
\pgfsetfillcolor{currentfill}%
\pgfsetfillopacity{0.340496}%
\pgfsetlinewidth{1.003750pt}%
\definecolor{currentstroke}{rgb}{0.121569,0.466667,0.705882}%
\pgfsetstrokecolor{currentstroke}%
\pgfsetstrokeopacity{0.340496}%
\pgfsetdash{}{0pt}%
\pgfpathmoveto{\pgfqpoint{1.901543in}{2.121329in}}%
\pgfpathcurveto{\pgfqpoint{1.909779in}{2.121329in}}{\pgfqpoint{1.917679in}{2.124601in}}{\pgfqpoint{1.923503in}{2.130425in}}%
\pgfpathcurveto{\pgfqpoint{1.929327in}{2.136249in}}{\pgfqpoint{1.932599in}{2.144149in}}{\pgfqpoint{1.932599in}{2.152386in}}%
\pgfpathcurveto{\pgfqpoint{1.932599in}{2.160622in}}{\pgfqpoint{1.929327in}{2.168522in}}{\pgfqpoint{1.923503in}{2.174346in}}%
\pgfpathcurveto{\pgfqpoint{1.917679in}{2.180170in}}{\pgfqpoint{1.909779in}{2.183442in}}{\pgfqpoint{1.901543in}{2.183442in}}%
\pgfpathcurveto{\pgfqpoint{1.893307in}{2.183442in}}{\pgfqpoint{1.885406in}{2.180170in}}{\pgfqpoint{1.879583in}{2.174346in}}%
\pgfpathcurveto{\pgfqpoint{1.873759in}{2.168522in}}{\pgfqpoint{1.870486in}{2.160622in}}{\pgfqpoint{1.870486in}{2.152386in}}%
\pgfpathcurveto{\pgfqpoint{1.870486in}{2.144149in}}{\pgfqpoint{1.873759in}{2.136249in}}{\pgfqpoint{1.879583in}{2.130425in}}%
\pgfpathcurveto{\pgfqpoint{1.885406in}{2.124601in}}{\pgfqpoint{1.893307in}{2.121329in}}{\pgfqpoint{1.901543in}{2.121329in}}%
\pgfpathclose%
\pgfusepath{stroke,fill}%
\end{pgfscope}%
\begin{pgfscope}%
\pgfpathrectangle{\pgfqpoint{0.100000in}{0.212622in}}{\pgfqpoint{3.696000in}{3.696000in}}%
\pgfusepath{clip}%
\pgfsetbuttcap%
\pgfsetroundjoin%
\definecolor{currentfill}{rgb}{0.121569,0.466667,0.705882}%
\pgfsetfillcolor{currentfill}%
\pgfsetfillopacity{0.341500}%
\pgfsetlinewidth{1.003750pt}%
\definecolor{currentstroke}{rgb}{0.121569,0.466667,0.705882}%
\pgfsetstrokecolor{currentstroke}%
\pgfsetstrokeopacity{0.341500}%
\pgfsetdash{}{0pt}%
\pgfpathmoveto{\pgfqpoint{1.561413in}{2.013942in}}%
\pgfpathcurveto{\pgfqpoint{1.569650in}{2.013942in}}{\pgfqpoint{1.577550in}{2.017214in}}{\pgfqpoint{1.583373in}{2.023038in}}%
\pgfpathcurveto{\pgfqpoint{1.589197in}{2.028862in}}{\pgfqpoint{1.592470in}{2.036762in}}{\pgfqpoint{1.592470in}{2.044999in}}%
\pgfpathcurveto{\pgfqpoint{1.592470in}{2.053235in}}{\pgfqpoint{1.589197in}{2.061135in}}{\pgfqpoint{1.583373in}{2.066959in}}%
\pgfpathcurveto{\pgfqpoint{1.577550in}{2.072783in}}{\pgfqpoint{1.569650in}{2.076055in}}{\pgfqpoint{1.561413in}{2.076055in}}%
\pgfpathcurveto{\pgfqpoint{1.553177in}{2.076055in}}{\pgfqpoint{1.545277in}{2.072783in}}{\pgfqpoint{1.539453in}{2.066959in}}%
\pgfpathcurveto{\pgfqpoint{1.533629in}{2.061135in}}{\pgfqpoint{1.530357in}{2.053235in}}{\pgfqpoint{1.530357in}{2.044999in}}%
\pgfpathcurveto{\pgfqpoint{1.530357in}{2.036762in}}{\pgfqpoint{1.533629in}{2.028862in}}{\pgfqpoint{1.539453in}{2.023038in}}%
\pgfpathcurveto{\pgfqpoint{1.545277in}{2.017214in}}{\pgfqpoint{1.553177in}{2.013942in}}{\pgfqpoint{1.561413in}{2.013942in}}%
\pgfpathclose%
\pgfusepath{stroke,fill}%
\end{pgfscope}%
\begin{pgfscope}%
\pgfpathrectangle{\pgfqpoint{0.100000in}{0.212622in}}{\pgfqpoint{3.696000in}{3.696000in}}%
\pgfusepath{clip}%
\pgfsetbuttcap%
\pgfsetroundjoin%
\definecolor{currentfill}{rgb}{0.121569,0.466667,0.705882}%
\pgfsetfillcolor{currentfill}%
\pgfsetfillopacity{0.341506}%
\pgfsetlinewidth{1.003750pt}%
\definecolor{currentstroke}{rgb}{0.121569,0.466667,0.705882}%
\pgfsetstrokecolor{currentstroke}%
\pgfsetstrokeopacity{0.341506}%
\pgfsetdash{}{0pt}%
\pgfpathmoveto{\pgfqpoint{1.905953in}{2.124374in}}%
\pgfpathcurveto{\pgfqpoint{1.914189in}{2.124374in}}{\pgfqpoint{1.922089in}{2.127646in}}{\pgfqpoint{1.927913in}{2.133470in}}%
\pgfpathcurveto{\pgfqpoint{1.933737in}{2.139294in}}{\pgfqpoint{1.937009in}{2.147194in}}{\pgfqpoint{1.937009in}{2.155430in}}%
\pgfpathcurveto{\pgfqpoint{1.937009in}{2.163667in}}{\pgfqpoint{1.933737in}{2.171567in}}{\pgfqpoint{1.927913in}{2.177391in}}%
\pgfpathcurveto{\pgfqpoint{1.922089in}{2.183215in}}{\pgfqpoint{1.914189in}{2.186487in}}{\pgfqpoint{1.905953in}{2.186487in}}%
\pgfpathcurveto{\pgfqpoint{1.897717in}{2.186487in}}{\pgfqpoint{1.889816in}{2.183215in}}{\pgfqpoint{1.883993in}{2.177391in}}%
\pgfpathcurveto{\pgfqpoint{1.878169in}{2.171567in}}{\pgfqpoint{1.874896in}{2.163667in}}{\pgfqpoint{1.874896in}{2.155430in}}%
\pgfpathcurveto{\pgfqpoint{1.874896in}{2.147194in}}{\pgfqpoint{1.878169in}{2.139294in}}{\pgfqpoint{1.883993in}{2.133470in}}%
\pgfpathcurveto{\pgfqpoint{1.889816in}{2.127646in}}{\pgfqpoint{1.897717in}{2.124374in}}{\pgfqpoint{1.905953in}{2.124374in}}%
\pgfpathclose%
\pgfusepath{stroke,fill}%
\end{pgfscope}%
\begin{pgfscope}%
\pgfpathrectangle{\pgfqpoint{0.100000in}{0.212622in}}{\pgfqpoint{3.696000in}{3.696000in}}%
\pgfusepath{clip}%
\pgfsetbuttcap%
\pgfsetroundjoin%
\definecolor{currentfill}{rgb}{0.121569,0.466667,0.705882}%
\pgfsetfillcolor{currentfill}%
\pgfsetfillopacity{0.341954}%
\pgfsetlinewidth{1.003750pt}%
\definecolor{currentstroke}{rgb}{0.121569,0.466667,0.705882}%
\pgfsetstrokecolor{currentstroke}%
\pgfsetstrokeopacity{0.341954}%
\pgfsetdash{}{0pt}%
\pgfpathmoveto{\pgfqpoint{1.910532in}{2.121015in}}%
\pgfpathcurveto{\pgfqpoint{1.918768in}{2.121015in}}{\pgfqpoint{1.926668in}{2.124288in}}{\pgfqpoint{1.932492in}{2.130112in}}%
\pgfpathcurveto{\pgfqpoint{1.938316in}{2.135936in}}{\pgfqpoint{1.941589in}{2.143836in}}{\pgfqpoint{1.941589in}{2.152072in}}%
\pgfpathcurveto{\pgfqpoint{1.941589in}{2.160308in}}{\pgfqpoint{1.938316in}{2.168208in}}{\pgfqpoint{1.932492in}{2.174032in}}%
\pgfpathcurveto{\pgfqpoint{1.926668in}{2.179856in}}{\pgfqpoint{1.918768in}{2.183128in}}{\pgfqpoint{1.910532in}{2.183128in}}%
\pgfpathcurveto{\pgfqpoint{1.902296in}{2.183128in}}{\pgfqpoint{1.894396in}{2.179856in}}{\pgfqpoint{1.888572in}{2.174032in}}%
\pgfpathcurveto{\pgfqpoint{1.882748in}{2.168208in}}{\pgfqpoint{1.879476in}{2.160308in}}{\pgfqpoint{1.879476in}{2.152072in}}%
\pgfpathcurveto{\pgfqpoint{1.879476in}{2.143836in}}{\pgfqpoint{1.882748in}{2.135936in}}{\pgfqpoint{1.888572in}{2.130112in}}%
\pgfpathcurveto{\pgfqpoint{1.894396in}{2.124288in}}{\pgfqpoint{1.902296in}{2.121015in}}{\pgfqpoint{1.910532in}{2.121015in}}%
\pgfpathclose%
\pgfusepath{stroke,fill}%
\end{pgfscope}%
\begin{pgfscope}%
\pgfpathrectangle{\pgfqpoint{0.100000in}{0.212622in}}{\pgfqpoint{3.696000in}{3.696000in}}%
\pgfusepath{clip}%
\pgfsetbuttcap%
\pgfsetroundjoin%
\definecolor{currentfill}{rgb}{0.121569,0.466667,0.705882}%
\pgfsetfillcolor{currentfill}%
\pgfsetfillopacity{0.342396}%
\pgfsetlinewidth{1.003750pt}%
\definecolor{currentstroke}{rgb}{0.121569,0.466667,0.705882}%
\pgfsetstrokecolor{currentstroke}%
\pgfsetstrokeopacity{0.342396}%
\pgfsetdash{}{0pt}%
\pgfpathmoveto{\pgfqpoint{1.559762in}{2.010098in}}%
\pgfpathcurveto{\pgfqpoint{1.567998in}{2.010098in}}{\pgfqpoint{1.575898in}{2.013370in}}{\pgfqpoint{1.581722in}{2.019194in}}%
\pgfpathcurveto{\pgfqpoint{1.587546in}{2.025018in}}{\pgfqpoint{1.590818in}{2.032918in}}{\pgfqpoint{1.590818in}{2.041154in}}%
\pgfpathcurveto{\pgfqpoint{1.590818in}{2.049391in}}{\pgfqpoint{1.587546in}{2.057291in}}{\pgfqpoint{1.581722in}{2.063115in}}%
\pgfpathcurveto{\pgfqpoint{1.575898in}{2.068938in}}{\pgfqpoint{1.567998in}{2.072211in}}{\pgfqpoint{1.559762in}{2.072211in}}%
\pgfpathcurveto{\pgfqpoint{1.551526in}{2.072211in}}{\pgfqpoint{1.543626in}{2.068938in}}{\pgfqpoint{1.537802in}{2.063115in}}%
\pgfpathcurveto{\pgfqpoint{1.531978in}{2.057291in}}{\pgfqpoint{1.528705in}{2.049391in}}{\pgfqpoint{1.528705in}{2.041154in}}%
\pgfpathcurveto{\pgfqpoint{1.528705in}{2.032918in}}{\pgfqpoint{1.531978in}{2.025018in}}{\pgfqpoint{1.537802in}{2.019194in}}%
\pgfpathcurveto{\pgfqpoint{1.543626in}{2.013370in}}{\pgfqpoint{1.551526in}{2.010098in}}{\pgfqpoint{1.559762in}{2.010098in}}%
\pgfpathclose%
\pgfusepath{stroke,fill}%
\end{pgfscope}%
\begin{pgfscope}%
\pgfpathrectangle{\pgfqpoint{0.100000in}{0.212622in}}{\pgfqpoint{3.696000in}{3.696000in}}%
\pgfusepath{clip}%
\pgfsetbuttcap%
\pgfsetroundjoin%
\definecolor{currentfill}{rgb}{0.121569,0.466667,0.705882}%
\pgfsetfillcolor{currentfill}%
\pgfsetfillopacity{0.342973}%
\pgfsetlinewidth{1.003750pt}%
\definecolor{currentstroke}{rgb}{0.121569,0.466667,0.705882}%
\pgfsetstrokecolor{currentstroke}%
\pgfsetstrokeopacity{0.342973}%
\pgfsetdash{}{0pt}%
\pgfpathmoveto{\pgfqpoint{1.916177in}{2.123184in}}%
\pgfpathcurveto{\pgfqpoint{1.924413in}{2.123184in}}{\pgfqpoint{1.932313in}{2.126456in}}{\pgfqpoint{1.938137in}{2.132280in}}%
\pgfpathcurveto{\pgfqpoint{1.943961in}{2.138104in}}{\pgfqpoint{1.947234in}{2.146004in}}{\pgfqpoint{1.947234in}{2.154240in}}%
\pgfpathcurveto{\pgfqpoint{1.947234in}{2.162477in}}{\pgfqpoint{1.943961in}{2.170377in}}{\pgfqpoint{1.938137in}{2.176201in}}%
\pgfpathcurveto{\pgfqpoint{1.932313in}{2.182025in}}{\pgfqpoint{1.924413in}{2.185297in}}{\pgfqpoint{1.916177in}{2.185297in}}%
\pgfpathcurveto{\pgfqpoint{1.907941in}{2.185297in}}{\pgfqpoint{1.900041in}{2.182025in}}{\pgfqpoint{1.894217in}{2.176201in}}%
\pgfpathcurveto{\pgfqpoint{1.888393in}{2.170377in}}{\pgfqpoint{1.885121in}{2.162477in}}{\pgfqpoint{1.885121in}{2.154240in}}%
\pgfpathcurveto{\pgfqpoint{1.885121in}{2.146004in}}{\pgfqpoint{1.888393in}{2.138104in}}{\pgfqpoint{1.894217in}{2.132280in}}%
\pgfpathcurveto{\pgfqpoint{1.900041in}{2.126456in}}{\pgfqpoint{1.907941in}{2.123184in}}{\pgfqpoint{1.916177in}{2.123184in}}%
\pgfpathclose%
\pgfusepath{stroke,fill}%
\end{pgfscope}%
\begin{pgfscope}%
\pgfpathrectangle{\pgfqpoint{0.100000in}{0.212622in}}{\pgfqpoint{3.696000in}{3.696000in}}%
\pgfusepath{clip}%
\pgfsetbuttcap%
\pgfsetroundjoin%
\definecolor{currentfill}{rgb}{0.121569,0.466667,0.705882}%
\pgfsetfillcolor{currentfill}%
\pgfsetfillopacity{0.343123}%
\pgfsetlinewidth{1.003750pt}%
\definecolor{currentstroke}{rgb}{0.121569,0.466667,0.705882}%
\pgfsetstrokecolor{currentstroke}%
\pgfsetstrokeopacity{0.343123}%
\pgfsetdash{}{0pt}%
\pgfpathmoveto{\pgfqpoint{1.558644in}{2.008708in}}%
\pgfpathcurveto{\pgfqpoint{1.566880in}{2.008708in}}{\pgfqpoint{1.574780in}{2.011981in}}{\pgfqpoint{1.580604in}{2.017805in}}%
\pgfpathcurveto{\pgfqpoint{1.586428in}{2.023629in}}{\pgfqpoint{1.589701in}{2.031529in}}{\pgfqpoint{1.589701in}{2.039765in}}%
\pgfpathcurveto{\pgfqpoint{1.589701in}{2.048001in}}{\pgfqpoint{1.586428in}{2.055901in}}{\pgfqpoint{1.580604in}{2.061725in}}%
\pgfpathcurveto{\pgfqpoint{1.574780in}{2.067549in}}{\pgfqpoint{1.566880in}{2.070821in}}{\pgfqpoint{1.558644in}{2.070821in}}%
\pgfpathcurveto{\pgfqpoint{1.550408in}{2.070821in}}{\pgfqpoint{1.542508in}{2.067549in}}{\pgfqpoint{1.536684in}{2.061725in}}%
\pgfpathcurveto{\pgfqpoint{1.530860in}{2.055901in}}{\pgfqpoint{1.527588in}{2.048001in}}{\pgfqpoint{1.527588in}{2.039765in}}%
\pgfpathcurveto{\pgfqpoint{1.527588in}{2.031529in}}{\pgfqpoint{1.530860in}{2.023629in}}{\pgfqpoint{1.536684in}{2.017805in}}%
\pgfpathcurveto{\pgfqpoint{1.542508in}{2.011981in}}{\pgfqpoint{1.550408in}{2.008708in}}{\pgfqpoint{1.558644in}{2.008708in}}%
\pgfpathclose%
\pgfusepath{stroke,fill}%
\end{pgfscope}%
\begin{pgfscope}%
\pgfpathrectangle{\pgfqpoint{0.100000in}{0.212622in}}{\pgfqpoint{3.696000in}{3.696000in}}%
\pgfusepath{clip}%
\pgfsetbuttcap%
\pgfsetroundjoin%
\definecolor{currentfill}{rgb}{0.121569,0.466667,0.705882}%
\pgfsetfillcolor{currentfill}%
\pgfsetfillopacity{0.343270}%
\pgfsetlinewidth{1.003750pt}%
\definecolor{currentstroke}{rgb}{0.121569,0.466667,0.705882}%
\pgfsetstrokecolor{currentstroke}%
\pgfsetstrokeopacity{0.343270}%
\pgfsetdash{}{0pt}%
\pgfpathmoveto{\pgfqpoint{1.557818in}{2.006626in}}%
\pgfpathcurveto{\pgfqpoint{1.566054in}{2.006626in}}{\pgfqpoint{1.573954in}{2.009898in}}{\pgfqpoint{1.579778in}{2.015722in}}%
\pgfpathcurveto{\pgfqpoint{1.585602in}{2.021546in}}{\pgfqpoint{1.588874in}{2.029446in}}{\pgfqpoint{1.588874in}{2.037682in}}%
\pgfpathcurveto{\pgfqpoint{1.588874in}{2.045918in}}{\pgfqpoint{1.585602in}{2.053818in}}{\pgfqpoint{1.579778in}{2.059642in}}%
\pgfpathcurveto{\pgfqpoint{1.573954in}{2.065466in}}{\pgfqpoint{1.566054in}{2.068739in}}{\pgfqpoint{1.557818in}{2.068739in}}%
\pgfpathcurveto{\pgfqpoint{1.549582in}{2.068739in}}{\pgfqpoint{1.541681in}{2.065466in}}{\pgfqpoint{1.535858in}{2.059642in}}%
\pgfpathcurveto{\pgfqpoint{1.530034in}{2.053818in}}{\pgfqpoint{1.526761in}{2.045918in}}{\pgfqpoint{1.526761in}{2.037682in}}%
\pgfpathcurveto{\pgfqpoint{1.526761in}{2.029446in}}{\pgfqpoint{1.530034in}{2.021546in}}{\pgfqpoint{1.535858in}{2.015722in}}%
\pgfpathcurveto{\pgfqpoint{1.541681in}{2.009898in}}{\pgfqpoint{1.549582in}{2.006626in}}{\pgfqpoint{1.557818in}{2.006626in}}%
\pgfpathclose%
\pgfusepath{stroke,fill}%
\end{pgfscope}%
\begin{pgfscope}%
\pgfpathrectangle{\pgfqpoint{0.100000in}{0.212622in}}{\pgfqpoint{3.696000in}{3.696000in}}%
\pgfusepath{clip}%
\pgfsetbuttcap%
\pgfsetroundjoin%
\definecolor{currentfill}{rgb}{0.121569,0.466667,0.705882}%
\pgfsetfillcolor{currentfill}%
\pgfsetfillopacity{0.343337}%
\pgfsetlinewidth{1.003750pt}%
\definecolor{currentstroke}{rgb}{0.121569,0.466667,0.705882}%
\pgfsetstrokecolor{currentstroke}%
\pgfsetstrokeopacity{0.343337}%
\pgfsetdash{}{0pt}%
\pgfpathmoveto{\pgfqpoint{1.918883in}{2.121613in}}%
\pgfpathcurveto{\pgfqpoint{1.927119in}{2.121613in}}{\pgfqpoint{1.935019in}{2.124885in}}{\pgfqpoint{1.940843in}{2.130709in}}%
\pgfpathcurveto{\pgfqpoint{1.946667in}{2.136533in}}{\pgfqpoint{1.949939in}{2.144433in}}{\pgfqpoint{1.949939in}{2.152669in}}%
\pgfpathcurveto{\pgfqpoint{1.949939in}{2.160905in}}{\pgfqpoint{1.946667in}{2.168805in}}{\pgfqpoint{1.940843in}{2.174629in}}%
\pgfpathcurveto{\pgfqpoint{1.935019in}{2.180453in}}{\pgfqpoint{1.927119in}{2.183726in}}{\pgfqpoint{1.918883in}{2.183726in}}%
\pgfpathcurveto{\pgfqpoint{1.910647in}{2.183726in}}{\pgfqpoint{1.902746in}{2.180453in}}{\pgfqpoint{1.896923in}{2.174629in}}%
\pgfpathcurveto{\pgfqpoint{1.891099in}{2.168805in}}{\pgfqpoint{1.887826in}{2.160905in}}{\pgfqpoint{1.887826in}{2.152669in}}%
\pgfpathcurveto{\pgfqpoint{1.887826in}{2.144433in}}{\pgfqpoint{1.891099in}{2.136533in}}{\pgfqpoint{1.896923in}{2.130709in}}%
\pgfpathcurveto{\pgfqpoint{1.902746in}{2.124885in}}{\pgfqpoint{1.910647in}{2.121613in}}{\pgfqpoint{1.918883in}{2.121613in}}%
\pgfpathclose%
\pgfusepath{stroke,fill}%
\end{pgfscope}%
\begin{pgfscope}%
\pgfpathrectangle{\pgfqpoint{0.100000in}{0.212622in}}{\pgfqpoint{3.696000in}{3.696000in}}%
\pgfusepath{clip}%
\pgfsetbuttcap%
\pgfsetroundjoin%
\definecolor{currentfill}{rgb}{0.121569,0.466667,0.705882}%
\pgfsetfillcolor{currentfill}%
\pgfsetfillopacity{0.343997}%
\pgfsetlinewidth{1.003750pt}%
\definecolor{currentstroke}{rgb}{0.121569,0.466667,0.705882}%
\pgfsetstrokecolor{currentstroke}%
\pgfsetstrokeopacity{0.343997}%
\pgfsetdash{}{0pt}%
\pgfpathmoveto{\pgfqpoint{1.556919in}{2.005805in}}%
\pgfpathcurveto{\pgfqpoint{1.565155in}{2.005805in}}{\pgfqpoint{1.573055in}{2.009077in}}{\pgfqpoint{1.578879in}{2.014901in}}%
\pgfpathcurveto{\pgfqpoint{1.584703in}{2.020725in}}{\pgfqpoint{1.587975in}{2.028625in}}{\pgfqpoint{1.587975in}{2.036861in}}%
\pgfpathcurveto{\pgfqpoint{1.587975in}{2.045098in}}{\pgfqpoint{1.584703in}{2.052998in}}{\pgfqpoint{1.578879in}{2.058822in}}%
\pgfpathcurveto{\pgfqpoint{1.573055in}{2.064645in}}{\pgfqpoint{1.565155in}{2.067918in}}{\pgfqpoint{1.556919in}{2.067918in}}%
\pgfpathcurveto{\pgfqpoint{1.548683in}{2.067918in}}{\pgfqpoint{1.540783in}{2.064645in}}{\pgfqpoint{1.534959in}{2.058822in}}%
\pgfpathcurveto{\pgfqpoint{1.529135in}{2.052998in}}{\pgfqpoint{1.525862in}{2.045098in}}{\pgfqpoint{1.525862in}{2.036861in}}%
\pgfpathcurveto{\pgfqpoint{1.525862in}{2.028625in}}{\pgfqpoint{1.529135in}{2.020725in}}{\pgfqpoint{1.534959in}{2.014901in}}%
\pgfpathcurveto{\pgfqpoint{1.540783in}{2.009077in}}{\pgfqpoint{1.548683in}{2.005805in}}{\pgfqpoint{1.556919in}{2.005805in}}%
\pgfpathclose%
\pgfusepath{stroke,fill}%
\end{pgfscope}%
\begin{pgfscope}%
\pgfpathrectangle{\pgfqpoint{0.100000in}{0.212622in}}{\pgfqpoint{3.696000in}{3.696000in}}%
\pgfusepath{clip}%
\pgfsetbuttcap%
\pgfsetroundjoin%
\definecolor{currentfill}{rgb}{0.121569,0.466667,0.705882}%
\pgfsetfillcolor{currentfill}%
\pgfsetfillopacity{0.344123}%
\pgfsetlinewidth{1.003750pt}%
\definecolor{currentstroke}{rgb}{0.121569,0.466667,0.705882}%
\pgfsetstrokecolor{currentstroke}%
\pgfsetstrokeopacity{0.344123}%
\pgfsetdash{}{0pt}%
\pgfpathmoveto{\pgfqpoint{1.922396in}{2.122922in}}%
\pgfpathcurveto{\pgfqpoint{1.930633in}{2.122922in}}{\pgfqpoint{1.938533in}{2.126194in}}{\pgfqpoint{1.944357in}{2.132018in}}%
\pgfpathcurveto{\pgfqpoint{1.950181in}{2.137842in}}{\pgfqpoint{1.953453in}{2.145742in}}{\pgfqpoint{1.953453in}{2.153979in}}%
\pgfpathcurveto{\pgfqpoint{1.953453in}{2.162215in}}{\pgfqpoint{1.950181in}{2.170115in}}{\pgfqpoint{1.944357in}{2.175939in}}%
\pgfpathcurveto{\pgfqpoint{1.938533in}{2.181763in}}{\pgfqpoint{1.930633in}{2.185035in}}{\pgfqpoint{1.922396in}{2.185035in}}%
\pgfpathcurveto{\pgfqpoint{1.914160in}{2.185035in}}{\pgfqpoint{1.906260in}{2.181763in}}{\pgfqpoint{1.900436in}{2.175939in}}%
\pgfpathcurveto{\pgfqpoint{1.894612in}{2.170115in}}{\pgfqpoint{1.891340in}{2.162215in}}{\pgfqpoint{1.891340in}{2.153979in}}%
\pgfpathcurveto{\pgfqpoint{1.891340in}{2.145742in}}{\pgfqpoint{1.894612in}{2.137842in}}{\pgfqpoint{1.900436in}{2.132018in}}%
\pgfpathcurveto{\pgfqpoint{1.906260in}{2.126194in}}{\pgfqpoint{1.914160in}{2.122922in}}{\pgfqpoint{1.922396in}{2.122922in}}%
\pgfpathclose%
\pgfusepath{stroke,fill}%
\end{pgfscope}%
\begin{pgfscope}%
\pgfpathrectangle{\pgfqpoint{0.100000in}{0.212622in}}{\pgfqpoint{3.696000in}{3.696000in}}%
\pgfusepath{clip}%
\pgfsetbuttcap%
\pgfsetroundjoin%
\definecolor{currentfill}{rgb}{0.121569,0.466667,0.705882}%
\pgfsetfillcolor{currentfill}%
\pgfsetfillopacity{0.344200}%
\pgfsetlinewidth{1.003750pt}%
\definecolor{currentstroke}{rgb}{0.121569,0.466667,0.705882}%
\pgfsetstrokecolor{currentstroke}%
\pgfsetstrokeopacity{0.344200}%
\pgfsetdash{}{0pt}%
\pgfpathmoveto{\pgfqpoint{1.556157in}{2.004255in}}%
\pgfpathcurveto{\pgfqpoint{1.564394in}{2.004255in}}{\pgfqpoint{1.572294in}{2.007527in}}{\pgfqpoint{1.578118in}{2.013351in}}%
\pgfpathcurveto{\pgfqpoint{1.583942in}{2.019175in}}{\pgfqpoint{1.587214in}{2.027075in}}{\pgfqpoint{1.587214in}{2.035311in}}%
\pgfpathcurveto{\pgfqpoint{1.587214in}{2.043547in}}{\pgfqpoint{1.583942in}{2.051447in}}{\pgfqpoint{1.578118in}{2.057271in}}%
\pgfpathcurveto{\pgfqpoint{1.572294in}{2.063095in}}{\pgfqpoint{1.564394in}{2.066368in}}{\pgfqpoint{1.556157in}{2.066368in}}%
\pgfpathcurveto{\pgfqpoint{1.547921in}{2.066368in}}{\pgfqpoint{1.540021in}{2.063095in}}{\pgfqpoint{1.534197in}{2.057271in}}%
\pgfpathcurveto{\pgfqpoint{1.528373in}{2.051447in}}{\pgfqpoint{1.525101in}{2.043547in}}{\pgfqpoint{1.525101in}{2.035311in}}%
\pgfpathcurveto{\pgfqpoint{1.525101in}{2.027075in}}{\pgfqpoint{1.528373in}{2.019175in}}{\pgfqpoint{1.534197in}{2.013351in}}%
\pgfpathcurveto{\pgfqpoint{1.540021in}{2.007527in}}{\pgfqpoint{1.547921in}{2.004255in}}{\pgfqpoint{1.556157in}{2.004255in}}%
\pgfpathclose%
\pgfusepath{stroke,fill}%
\end{pgfscope}%
\begin{pgfscope}%
\pgfpathrectangle{\pgfqpoint{0.100000in}{0.212622in}}{\pgfqpoint{3.696000in}{3.696000in}}%
\pgfusepath{clip}%
\pgfsetbuttcap%
\pgfsetroundjoin%
\definecolor{currentfill}{rgb}{0.121569,0.466667,0.705882}%
\pgfsetfillcolor{currentfill}%
\pgfsetfillopacity{0.344590}%
\pgfsetlinewidth{1.003750pt}%
\definecolor{currentstroke}{rgb}{0.121569,0.466667,0.705882}%
\pgfsetstrokecolor{currentstroke}%
\pgfsetstrokeopacity{0.344590}%
\pgfsetdash{}{0pt}%
\pgfpathmoveto{\pgfqpoint{1.926128in}{2.121014in}}%
\pgfpathcurveto{\pgfqpoint{1.934364in}{2.121014in}}{\pgfqpoint{1.942264in}{2.124286in}}{\pgfqpoint{1.948088in}{2.130110in}}%
\pgfpathcurveto{\pgfqpoint{1.953912in}{2.135934in}}{\pgfqpoint{1.957185in}{2.143834in}}{\pgfqpoint{1.957185in}{2.152071in}}%
\pgfpathcurveto{\pgfqpoint{1.957185in}{2.160307in}}{\pgfqpoint{1.953912in}{2.168207in}}{\pgfqpoint{1.948088in}{2.174031in}}%
\pgfpathcurveto{\pgfqpoint{1.942264in}{2.179855in}}{\pgfqpoint{1.934364in}{2.183127in}}{\pgfqpoint{1.926128in}{2.183127in}}%
\pgfpathcurveto{\pgfqpoint{1.917892in}{2.183127in}}{\pgfqpoint{1.909992in}{2.179855in}}{\pgfqpoint{1.904168in}{2.174031in}}%
\pgfpathcurveto{\pgfqpoint{1.898344in}{2.168207in}}{\pgfqpoint{1.895072in}{2.160307in}}{\pgfqpoint{1.895072in}{2.152071in}}%
\pgfpathcurveto{\pgfqpoint{1.895072in}{2.143834in}}{\pgfqpoint{1.898344in}{2.135934in}}{\pgfqpoint{1.904168in}{2.130110in}}%
\pgfpathcurveto{\pgfqpoint{1.909992in}{2.124286in}}{\pgfqpoint{1.917892in}{2.121014in}}{\pgfqpoint{1.926128in}{2.121014in}}%
\pgfpathclose%
\pgfusepath{stroke,fill}%
\end{pgfscope}%
\begin{pgfscope}%
\pgfpathrectangle{\pgfqpoint{0.100000in}{0.212622in}}{\pgfqpoint{3.696000in}{3.696000in}}%
\pgfusepath{clip}%
\pgfsetbuttcap%
\pgfsetroundjoin%
\definecolor{currentfill}{rgb}{0.121569,0.466667,0.705882}%
\pgfsetfillcolor{currentfill}%
\pgfsetfillopacity{0.344632}%
\pgfsetlinewidth{1.003750pt}%
\definecolor{currentstroke}{rgb}{0.121569,0.466667,0.705882}%
\pgfsetstrokecolor{currentstroke}%
\pgfsetstrokeopacity{0.344632}%
\pgfsetdash{}{0pt}%
\pgfpathmoveto{\pgfqpoint{1.554554in}{2.002125in}}%
\pgfpathcurveto{\pgfqpoint{1.562790in}{2.002125in}}{\pgfqpoint{1.570690in}{2.005397in}}{\pgfqpoint{1.576514in}{2.011221in}}%
\pgfpathcurveto{\pgfqpoint{1.582338in}{2.017045in}}{\pgfqpoint{1.585610in}{2.024945in}}{\pgfqpoint{1.585610in}{2.033181in}}%
\pgfpathcurveto{\pgfqpoint{1.585610in}{2.041418in}}{\pgfqpoint{1.582338in}{2.049318in}}{\pgfqpoint{1.576514in}{2.055142in}}%
\pgfpathcurveto{\pgfqpoint{1.570690in}{2.060966in}}{\pgfqpoint{1.562790in}{2.064238in}}{\pgfqpoint{1.554554in}{2.064238in}}%
\pgfpathcurveto{\pgfqpoint{1.546317in}{2.064238in}}{\pgfqpoint{1.538417in}{2.060966in}}{\pgfqpoint{1.532593in}{2.055142in}}%
\pgfpathcurveto{\pgfqpoint{1.526769in}{2.049318in}}{\pgfqpoint{1.523497in}{2.041418in}}{\pgfqpoint{1.523497in}{2.033181in}}%
\pgfpathcurveto{\pgfqpoint{1.523497in}{2.024945in}}{\pgfqpoint{1.526769in}{2.017045in}}{\pgfqpoint{1.532593in}{2.011221in}}%
\pgfpathcurveto{\pgfqpoint{1.538417in}{2.005397in}}{\pgfqpoint{1.546317in}{2.002125in}}{\pgfqpoint{1.554554in}{2.002125in}}%
\pgfpathclose%
\pgfusepath{stroke,fill}%
\end{pgfscope}%
\begin{pgfscope}%
\pgfpathrectangle{\pgfqpoint{0.100000in}{0.212622in}}{\pgfqpoint{3.696000in}{3.696000in}}%
\pgfusepath{clip}%
\pgfsetbuttcap%
\pgfsetroundjoin%
\definecolor{currentfill}{rgb}{0.121569,0.466667,0.705882}%
\pgfsetfillcolor{currentfill}%
\pgfsetfillopacity{0.345075}%
\pgfsetlinewidth{1.003750pt}%
\definecolor{currentstroke}{rgb}{0.121569,0.466667,0.705882}%
\pgfsetstrokecolor{currentstroke}%
\pgfsetstrokeopacity{0.345075}%
\pgfsetdash{}{0pt}%
\pgfpathmoveto{\pgfqpoint{1.929573in}{2.120895in}}%
\pgfpathcurveto{\pgfqpoint{1.937809in}{2.120895in}}{\pgfqpoint{1.945709in}{2.124168in}}{\pgfqpoint{1.951533in}{2.129992in}}%
\pgfpathcurveto{\pgfqpoint{1.957357in}{2.135816in}}{\pgfqpoint{1.960629in}{2.143716in}}{\pgfqpoint{1.960629in}{2.151952in}}%
\pgfpathcurveto{\pgfqpoint{1.960629in}{2.160188in}}{\pgfqpoint{1.957357in}{2.168088in}}{\pgfqpoint{1.951533in}{2.173912in}}%
\pgfpathcurveto{\pgfqpoint{1.945709in}{2.179736in}}{\pgfqpoint{1.937809in}{2.183008in}}{\pgfqpoint{1.929573in}{2.183008in}}%
\pgfpathcurveto{\pgfqpoint{1.921336in}{2.183008in}}{\pgfqpoint{1.913436in}{2.179736in}}{\pgfqpoint{1.907612in}{2.173912in}}%
\pgfpathcurveto{\pgfqpoint{1.901788in}{2.168088in}}{\pgfqpoint{1.898516in}{2.160188in}}{\pgfqpoint{1.898516in}{2.151952in}}%
\pgfpathcurveto{\pgfqpoint{1.898516in}{2.143716in}}{\pgfqpoint{1.901788in}{2.135816in}}{\pgfqpoint{1.907612in}{2.129992in}}%
\pgfpathcurveto{\pgfqpoint{1.913436in}{2.124168in}}{\pgfqpoint{1.921336in}{2.120895in}}{\pgfqpoint{1.929573in}{2.120895in}}%
\pgfpathclose%
\pgfusepath{stroke,fill}%
\end{pgfscope}%
\begin{pgfscope}%
\pgfpathrectangle{\pgfqpoint{0.100000in}{0.212622in}}{\pgfqpoint{3.696000in}{3.696000in}}%
\pgfusepath{clip}%
\pgfsetbuttcap%
\pgfsetroundjoin%
\definecolor{currentfill}{rgb}{0.121569,0.466667,0.705882}%
\pgfsetfillcolor{currentfill}%
\pgfsetfillopacity{0.345075}%
\pgfsetlinewidth{1.003750pt}%
\definecolor{currentstroke}{rgb}{0.121569,0.466667,0.705882}%
\pgfsetstrokecolor{currentstroke}%
\pgfsetstrokeopacity{0.345075}%
\pgfsetdash{}{0pt}%
\pgfpathmoveto{\pgfqpoint{1.928396in}{2.122402in}}%
\pgfpathcurveto{\pgfqpoint{1.936632in}{2.122402in}}{\pgfqpoint{1.944532in}{2.125674in}}{\pgfqpoint{1.950356in}{2.131498in}}%
\pgfpathcurveto{\pgfqpoint{1.956180in}{2.137322in}}{\pgfqpoint{1.959452in}{2.145222in}}{\pgfqpoint{1.959452in}{2.153458in}}%
\pgfpathcurveto{\pgfqpoint{1.959452in}{2.161694in}}{\pgfqpoint{1.956180in}{2.169594in}}{\pgfqpoint{1.950356in}{2.175418in}}%
\pgfpathcurveto{\pgfqpoint{1.944532in}{2.181242in}}{\pgfqpoint{1.936632in}{2.184515in}}{\pgfqpoint{1.928396in}{2.184515in}}%
\pgfpathcurveto{\pgfqpoint{1.920160in}{2.184515in}}{\pgfqpoint{1.912260in}{2.181242in}}{\pgfqpoint{1.906436in}{2.175418in}}%
\pgfpathcurveto{\pgfqpoint{1.900612in}{2.169594in}}{\pgfqpoint{1.897339in}{2.161694in}}{\pgfqpoint{1.897339in}{2.153458in}}%
\pgfpathcurveto{\pgfqpoint{1.897339in}{2.145222in}}{\pgfqpoint{1.900612in}{2.137322in}}{\pgfqpoint{1.906436in}{2.131498in}}%
\pgfpathcurveto{\pgfqpoint{1.912260in}{2.125674in}}{\pgfqpoint{1.920160in}{2.122402in}}{\pgfqpoint{1.928396in}{2.122402in}}%
\pgfpathclose%
\pgfusepath{stroke,fill}%
\end{pgfscope}%
\begin{pgfscope}%
\pgfpathrectangle{\pgfqpoint{0.100000in}{0.212622in}}{\pgfqpoint{3.696000in}{3.696000in}}%
\pgfusepath{clip}%
\pgfsetbuttcap%
\pgfsetroundjoin%
\definecolor{currentfill}{rgb}{0.121569,0.466667,0.705882}%
\pgfsetfillcolor{currentfill}%
\pgfsetfillopacity{0.345263}%
\pgfsetlinewidth{1.003750pt}%
\definecolor{currentstroke}{rgb}{0.121569,0.466667,0.705882}%
\pgfsetstrokecolor{currentstroke}%
\pgfsetstrokeopacity{0.345263}%
\pgfsetdash{}{0pt}%
\pgfpathmoveto{\pgfqpoint{1.553806in}{2.001523in}}%
\pgfpathcurveto{\pgfqpoint{1.562042in}{2.001523in}}{\pgfqpoint{1.569942in}{2.004795in}}{\pgfqpoint{1.575766in}{2.010619in}}%
\pgfpathcurveto{\pgfqpoint{1.581590in}{2.016443in}}{\pgfqpoint{1.584862in}{2.024343in}}{\pgfqpoint{1.584862in}{2.032580in}}%
\pgfpathcurveto{\pgfqpoint{1.584862in}{2.040816in}}{\pgfqpoint{1.581590in}{2.048716in}}{\pgfqpoint{1.575766in}{2.054540in}}%
\pgfpathcurveto{\pgfqpoint{1.569942in}{2.060364in}}{\pgfqpoint{1.562042in}{2.063636in}}{\pgfqpoint{1.553806in}{2.063636in}}%
\pgfpathcurveto{\pgfqpoint{1.545569in}{2.063636in}}{\pgfqpoint{1.537669in}{2.060364in}}{\pgfqpoint{1.531845in}{2.054540in}}%
\pgfpathcurveto{\pgfqpoint{1.526022in}{2.048716in}}{\pgfqpoint{1.522749in}{2.040816in}}{\pgfqpoint{1.522749in}{2.032580in}}%
\pgfpathcurveto{\pgfqpoint{1.522749in}{2.024343in}}{\pgfqpoint{1.526022in}{2.016443in}}{\pgfqpoint{1.531845in}{2.010619in}}%
\pgfpathcurveto{\pgfqpoint{1.537669in}{2.004795in}}{\pgfqpoint{1.545569in}{2.001523in}}{\pgfqpoint{1.553806in}{2.001523in}}%
\pgfpathclose%
\pgfusepath{stroke,fill}%
\end{pgfscope}%
\begin{pgfscope}%
\pgfpathrectangle{\pgfqpoint{0.100000in}{0.212622in}}{\pgfqpoint{3.696000in}{3.696000in}}%
\pgfusepath{clip}%
\pgfsetbuttcap%
\pgfsetroundjoin%
\definecolor{currentfill}{rgb}{0.121569,0.466667,0.705882}%
\pgfsetfillcolor{currentfill}%
\pgfsetfillopacity{0.345603}%
\pgfsetlinewidth{1.003750pt}%
\definecolor{currentstroke}{rgb}{0.121569,0.466667,0.705882}%
\pgfsetstrokecolor{currentstroke}%
\pgfsetstrokeopacity{0.345603}%
\pgfsetdash{}{0pt}%
\pgfpathmoveto{\pgfqpoint{1.552322in}{2.000428in}}%
\pgfpathcurveto{\pgfqpoint{1.560558in}{2.000428in}}{\pgfqpoint{1.568458in}{2.003700in}}{\pgfqpoint{1.574282in}{2.009524in}}%
\pgfpathcurveto{\pgfqpoint{1.580106in}{2.015348in}}{\pgfqpoint{1.583378in}{2.023248in}}{\pgfqpoint{1.583378in}{2.031484in}}%
\pgfpathcurveto{\pgfqpoint{1.583378in}{2.039720in}}{\pgfqpoint{1.580106in}{2.047620in}}{\pgfqpoint{1.574282in}{2.053444in}}%
\pgfpathcurveto{\pgfqpoint{1.568458in}{2.059268in}}{\pgfqpoint{1.560558in}{2.062541in}}{\pgfqpoint{1.552322in}{2.062541in}}%
\pgfpathcurveto{\pgfqpoint{1.544086in}{2.062541in}}{\pgfqpoint{1.536186in}{2.059268in}}{\pgfqpoint{1.530362in}{2.053444in}}%
\pgfpathcurveto{\pgfqpoint{1.524538in}{2.047620in}}{\pgfqpoint{1.521265in}{2.039720in}}{\pgfqpoint{1.521265in}{2.031484in}}%
\pgfpathcurveto{\pgfqpoint{1.521265in}{2.023248in}}{\pgfqpoint{1.524538in}{2.015348in}}{\pgfqpoint{1.530362in}{2.009524in}}%
\pgfpathcurveto{\pgfqpoint{1.536186in}{2.003700in}}{\pgfqpoint{1.544086in}{2.000428in}}{\pgfqpoint{1.552322in}{2.000428in}}%
\pgfpathclose%
\pgfusepath{stroke,fill}%
\end{pgfscope}%
\begin{pgfscope}%
\pgfpathrectangle{\pgfqpoint{0.100000in}{0.212622in}}{\pgfqpoint{3.696000in}{3.696000in}}%
\pgfusepath{clip}%
\pgfsetbuttcap%
\pgfsetroundjoin%
\definecolor{currentfill}{rgb}{0.121569,0.466667,0.705882}%
\pgfsetfillcolor{currentfill}%
\pgfsetfillopacity{0.345786}%
\pgfsetlinewidth{1.003750pt}%
\definecolor{currentstroke}{rgb}{0.121569,0.466667,0.705882}%
\pgfsetstrokecolor{currentstroke}%
\pgfsetstrokeopacity{0.345786}%
\pgfsetdash{}{0pt}%
\pgfpathmoveto{\pgfqpoint{1.932390in}{2.123395in}}%
\pgfpathcurveto{\pgfqpoint{1.940626in}{2.123395in}}{\pgfqpoint{1.948526in}{2.126667in}}{\pgfqpoint{1.954350in}{2.132491in}}%
\pgfpathcurveto{\pgfqpoint{1.960174in}{2.138315in}}{\pgfqpoint{1.963446in}{2.146215in}}{\pgfqpoint{1.963446in}{2.154451in}}%
\pgfpathcurveto{\pgfqpoint{1.963446in}{2.162687in}}{\pgfqpoint{1.960174in}{2.170587in}}{\pgfqpoint{1.954350in}{2.176411in}}%
\pgfpathcurveto{\pgfqpoint{1.948526in}{2.182235in}}{\pgfqpoint{1.940626in}{2.185508in}}{\pgfqpoint{1.932390in}{2.185508in}}%
\pgfpathcurveto{\pgfqpoint{1.924153in}{2.185508in}}{\pgfqpoint{1.916253in}{2.182235in}}{\pgfqpoint{1.910429in}{2.176411in}}%
\pgfpathcurveto{\pgfqpoint{1.904605in}{2.170587in}}{\pgfqpoint{1.901333in}{2.162687in}}{\pgfqpoint{1.901333in}{2.154451in}}%
\pgfpathcurveto{\pgfqpoint{1.901333in}{2.146215in}}{\pgfqpoint{1.904605in}{2.138315in}}{\pgfqpoint{1.910429in}{2.132491in}}%
\pgfpathcurveto{\pgfqpoint{1.916253in}{2.126667in}}{\pgfqpoint{1.924153in}{2.123395in}}{\pgfqpoint{1.932390in}{2.123395in}}%
\pgfpathclose%
\pgfusepath{stroke,fill}%
\end{pgfscope}%
\begin{pgfscope}%
\pgfpathrectangle{\pgfqpoint{0.100000in}{0.212622in}}{\pgfqpoint{3.696000in}{3.696000in}}%
\pgfusepath{clip}%
\pgfsetbuttcap%
\pgfsetroundjoin%
\definecolor{currentfill}{rgb}{0.121569,0.466667,0.705882}%
\pgfsetfillcolor{currentfill}%
\pgfsetfillopacity{0.345957}%
\pgfsetlinewidth{1.003750pt}%
\definecolor{currentstroke}{rgb}{0.121569,0.466667,0.705882}%
\pgfsetstrokecolor{currentstroke}%
\pgfsetstrokeopacity{0.345957}%
\pgfsetdash{}{0pt}%
\pgfpathmoveto{\pgfqpoint{1.551853in}{2.000139in}}%
\pgfpathcurveto{\pgfqpoint{1.560089in}{2.000139in}}{\pgfqpoint{1.567989in}{2.003411in}}{\pgfqpoint{1.573813in}{2.009235in}}%
\pgfpathcurveto{\pgfqpoint{1.579637in}{2.015059in}}{\pgfqpoint{1.582909in}{2.022959in}}{\pgfqpoint{1.582909in}{2.031196in}}%
\pgfpathcurveto{\pgfqpoint{1.582909in}{2.039432in}}{\pgfqpoint{1.579637in}{2.047332in}}{\pgfqpoint{1.573813in}{2.053156in}}%
\pgfpathcurveto{\pgfqpoint{1.567989in}{2.058980in}}{\pgfqpoint{1.560089in}{2.062252in}}{\pgfqpoint{1.551853in}{2.062252in}}%
\pgfpathcurveto{\pgfqpoint{1.543616in}{2.062252in}}{\pgfqpoint{1.535716in}{2.058980in}}{\pgfqpoint{1.529892in}{2.053156in}}%
\pgfpathcurveto{\pgfqpoint{1.524069in}{2.047332in}}{\pgfqpoint{1.520796in}{2.039432in}}{\pgfqpoint{1.520796in}{2.031196in}}%
\pgfpathcurveto{\pgfqpoint{1.520796in}{2.022959in}}{\pgfqpoint{1.524069in}{2.015059in}}{\pgfqpoint{1.529892in}{2.009235in}}%
\pgfpathcurveto{\pgfqpoint{1.535716in}{2.003411in}}{\pgfqpoint{1.543616in}{2.000139in}}{\pgfqpoint{1.551853in}{2.000139in}}%
\pgfpathclose%
\pgfusepath{stroke,fill}%
\end{pgfscope}%
\begin{pgfscope}%
\pgfpathrectangle{\pgfqpoint{0.100000in}{0.212622in}}{\pgfqpoint{3.696000in}{3.696000in}}%
\pgfusepath{clip}%
\pgfsetbuttcap%
\pgfsetroundjoin%
\definecolor{currentfill}{rgb}{0.121569,0.466667,0.705882}%
\pgfsetfillcolor{currentfill}%
\pgfsetfillopacity{0.346212}%
\pgfsetlinewidth{1.003750pt}%
\definecolor{currentstroke}{rgb}{0.121569,0.466667,0.705882}%
\pgfsetstrokecolor{currentstroke}%
\pgfsetstrokeopacity{0.346212}%
\pgfsetdash{}{0pt}%
\pgfpathmoveto{\pgfqpoint{1.934927in}{2.120918in}}%
\pgfpathcurveto{\pgfqpoint{1.943164in}{2.120918in}}{\pgfqpoint{1.951064in}{2.124190in}}{\pgfqpoint{1.956888in}{2.130014in}}%
\pgfpathcurveto{\pgfqpoint{1.962712in}{2.135838in}}{\pgfqpoint{1.965984in}{2.143738in}}{\pgfqpoint{1.965984in}{2.151975in}}%
\pgfpathcurveto{\pgfqpoint{1.965984in}{2.160211in}}{\pgfqpoint{1.962712in}{2.168111in}}{\pgfqpoint{1.956888in}{2.173935in}}%
\pgfpathcurveto{\pgfqpoint{1.951064in}{2.179759in}}{\pgfqpoint{1.943164in}{2.183031in}}{\pgfqpoint{1.934927in}{2.183031in}}%
\pgfpathcurveto{\pgfqpoint{1.926691in}{2.183031in}}{\pgfqpoint{1.918791in}{2.179759in}}{\pgfqpoint{1.912967in}{2.173935in}}%
\pgfpathcurveto{\pgfqpoint{1.907143in}{2.168111in}}{\pgfqpoint{1.903871in}{2.160211in}}{\pgfqpoint{1.903871in}{2.151975in}}%
\pgfpathcurveto{\pgfqpoint{1.903871in}{2.143738in}}{\pgfqpoint{1.907143in}{2.135838in}}{\pgfqpoint{1.912967in}{2.130014in}}%
\pgfpathcurveto{\pgfqpoint{1.918791in}{2.124190in}}{\pgfqpoint{1.926691in}{2.120918in}}{\pgfqpoint{1.934927in}{2.120918in}}%
\pgfpathclose%
\pgfusepath{stroke,fill}%
\end{pgfscope}%
\begin{pgfscope}%
\pgfpathrectangle{\pgfqpoint{0.100000in}{0.212622in}}{\pgfqpoint{3.696000in}{3.696000in}}%
\pgfusepath{clip}%
\pgfsetbuttcap%
\pgfsetroundjoin%
\definecolor{currentfill}{rgb}{0.121569,0.466667,0.705882}%
\pgfsetfillcolor{currentfill}%
\pgfsetfillopacity{0.346339}%
\pgfsetlinewidth{1.003750pt}%
\definecolor{currentstroke}{rgb}{0.121569,0.466667,0.705882}%
\pgfsetstrokecolor{currentstroke}%
\pgfsetstrokeopacity{0.346339}%
\pgfsetdash{}{0pt}%
\pgfpathmoveto{\pgfqpoint{1.550803in}{1.997805in}}%
\pgfpathcurveto{\pgfqpoint{1.559039in}{1.997805in}}{\pgfqpoint{1.566939in}{2.001077in}}{\pgfqpoint{1.572763in}{2.006901in}}%
\pgfpathcurveto{\pgfqpoint{1.578587in}{2.012725in}}{\pgfqpoint{1.581859in}{2.020625in}}{\pgfqpoint{1.581859in}{2.028862in}}%
\pgfpathcurveto{\pgfqpoint{1.581859in}{2.037098in}}{\pgfqpoint{1.578587in}{2.044998in}}{\pgfqpoint{1.572763in}{2.050822in}}%
\pgfpathcurveto{\pgfqpoint{1.566939in}{2.056646in}}{\pgfqpoint{1.559039in}{2.059918in}}{\pgfqpoint{1.550803in}{2.059918in}}%
\pgfpathcurveto{\pgfqpoint{1.542567in}{2.059918in}}{\pgfqpoint{1.534667in}{2.056646in}}{\pgfqpoint{1.528843in}{2.050822in}}%
\pgfpathcurveto{\pgfqpoint{1.523019in}{2.044998in}}{\pgfqpoint{1.519746in}{2.037098in}}{\pgfqpoint{1.519746in}{2.028862in}}%
\pgfpathcurveto{\pgfqpoint{1.519746in}{2.020625in}}{\pgfqpoint{1.523019in}{2.012725in}}{\pgfqpoint{1.528843in}{2.006901in}}%
\pgfpathcurveto{\pgfqpoint{1.534667in}{2.001077in}}{\pgfqpoint{1.542567in}{1.997805in}}{\pgfqpoint{1.550803in}{1.997805in}}%
\pgfpathclose%
\pgfusepath{stroke,fill}%
\end{pgfscope}%
\begin{pgfscope}%
\pgfpathrectangle{\pgfqpoint{0.100000in}{0.212622in}}{\pgfqpoint{3.696000in}{3.696000in}}%
\pgfusepath{clip}%
\pgfsetbuttcap%
\pgfsetroundjoin%
\definecolor{currentfill}{rgb}{0.121569,0.466667,0.705882}%
\pgfsetfillcolor{currentfill}%
\pgfsetfillopacity{0.347278}%
\pgfsetlinewidth{1.003750pt}%
\definecolor{currentstroke}{rgb}{0.121569,0.466667,0.705882}%
\pgfsetstrokecolor{currentstroke}%
\pgfsetstrokeopacity{0.347278}%
\pgfsetdash{}{0pt}%
\pgfpathmoveto{\pgfqpoint{1.547634in}{1.996781in}}%
\pgfpathcurveto{\pgfqpoint{1.555871in}{1.996781in}}{\pgfqpoint{1.563771in}{2.000053in}}{\pgfqpoint{1.569595in}{2.005877in}}%
\pgfpathcurveto{\pgfqpoint{1.575419in}{2.011701in}}{\pgfqpoint{1.578691in}{2.019601in}}{\pgfqpoint{1.578691in}{2.027837in}}%
\pgfpathcurveto{\pgfqpoint{1.578691in}{2.036074in}}{\pgfqpoint{1.575419in}{2.043974in}}{\pgfqpoint{1.569595in}{2.049798in}}%
\pgfpathcurveto{\pgfqpoint{1.563771in}{2.055622in}}{\pgfqpoint{1.555871in}{2.058894in}}{\pgfqpoint{1.547634in}{2.058894in}}%
\pgfpathcurveto{\pgfqpoint{1.539398in}{2.058894in}}{\pgfqpoint{1.531498in}{2.055622in}}{\pgfqpoint{1.525674in}{2.049798in}}%
\pgfpathcurveto{\pgfqpoint{1.519850in}{2.043974in}}{\pgfqpoint{1.516578in}{2.036074in}}{\pgfqpoint{1.516578in}{2.027837in}}%
\pgfpathcurveto{\pgfqpoint{1.516578in}{2.019601in}}{\pgfqpoint{1.519850in}{2.011701in}}{\pgfqpoint{1.525674in}{2.005877in}}%
\pgfpathcurveto{\pgfqpoint{1.531498in}{2.000053in}}{\pgfqpoint{1.539398in}{1.996781in}}{\pgfqpoint{1.547634in}{1.996781in}}%
\pgfpathclose%
\pgfusepath{stroke,fill}%
\end{pgfscope}%
\begin{pgfscope}%
\pgfpathrectangle{\pgfqpoint{0.100000in}{0.212622in}}{\pgfqpoint{3.696000in}{3.696000in}}%
\pgfusepath{clip}%
\pgfsetbuttcap%
\pgfsetroundjoin%
\definecolor{currentfill}{rgb}{0.121569,0.466667,0.705882}%
\pgfsetfillcolor{currentfill}%
\pgfsetfillopacity{0.347511}%
\pgfsetlinewidth{1.003750pt}%
\definecolor{currentstroke}{rgb}{0.121569,0.466667,0.705882}%
\pgfsetstrokecolor{currentstroke}%
\pgfsetstrokeopacity{0.347511}%
\pgfsetdash{}{0pt}%
\pgfpathmoveto{\pgfqpoint{1.938736in}{2.123901in}}%
\pgfpathcurveto{\pgfqpoint{1.946973in}{2.123901in}}{\pgfqpoint{1.954873in}{2.127173in}}{\pgfqpoint{1.960697in}{2.132997in}}%
\pgfpathcurveto{\pgfqpoint{1.966521in}{2.138821in}}{\pgfqpoint{1.969793in}{2.146721in}}{\pgfqpoint{1.969793in}{2.154958in}}%
\pgfpathcurveto{\pgfqpoint{1.969793in}{2.163194in}}{\pgfqpoint{1.966521in}{2.171094in}}{\pgfqpoint{1.960697in}{2.176918in}}%
\pgfpathcurveto{\pgfqpoint{1.954873in}{2.182742in}}{\pgfqpoint{1.946973in}{2.186014in}}{\pgfqpoint{1.938736in}{2.186014in}}%
\pgfpathcurveto{\pgfqpoint{1.930500in}{2.186014in}}{\pgfqpoint{1.922600in}{2.182742in}}{\pgfqpoint{1.916776in}{2.176918in}}%
\pgfpathcurveto{\pgfqpoint{1.910952in}{2.171094in}}{\pgfqpoint{1.907680in}{2.163194in}}{\pgfqpoint{1.907680in}{2.154958in}}%
\pgfpathcurveto{\pgfqpoint{1.907680in}{2.146721in}}{\pgfqpoint{1.910952in}{2.138821in}}{\pgfqpoint{1.916776in}{2.132997in}}%
\pgfpathcurveto{\pgfqpoint{1.922600in}{2.127173in}}{\pgfqpoint{1.930500in}{2.123901in}}{\pgfqpoint{1.938736in}{2.123901in}}%
\pgfpathclose%
\pgfusepath{stroke,fill}%
\end{pgfscope}%
\begin{pgfscope}%
\pgfpathrectangle{\pgfqpoint{0.100000in}{0.212622in}}{\pgfqpoint{3.696000in}{3.696000in}}%
\pgfusepath{clip}%
\pgfsetbuttcap%
\pgfsetroundjoin%
\definecolor{currentfill}{rgb}{0.121569,0.466667,0.705882}%
\pgfsetfillcolor{currentfill}%
\pgfsetfillopacity{0.348473}%
\pgfsetlinewidth{1.003750pt}%
\definecolor{currentstroke}{rgb}{0.121569,0.466667,0.705882}%
\pgfsetstrokecolor{currentstroke}%
\pgfsetstrokeopacity{0.348473}%
\pgfsetdash{}{0pt}%
\pgfpathmoveto{\pgfqpoint{1.546789in}{1.997468in}}%
\pgfpathcurveto{\pgfqpoint{1.555025in}{1.997468in}}{\pgfqpoint{1.562925in}{2.000740in}}{\pgfqpoint{1.568749in}{2.006564in}}%
\pgfpathcurveto{\pgfqpoint{1.574573in}{2.012388in}}{\pgfqpoint{1.577845in}{2.020288in}}{\pgfqpoint{1.577845in}{2.028524in}}%
\pgfpathcurveto{\pgfqpoint{1.577845in}{2.036760in}}{\pgfqpoint{1.574573in}{2.044660in}}{\pgfqpoint{1.568749in}{2.050484in}}%
\pgfpathcurveto{\pgfqpoint{1.562925in}{2.056308in}}{\pgfqpoint{1.555025in}{2.059581in}}{\pgfqpoint{1.546789in}{2.059581in}}%
\pgfpathcurveto{\pgfqpoint{1.538552in}{2.059581in}}{\pgfqpoint{1.530652in}{2.056308in}}{\pgfqpoint{1.524828in}{2.050484in}}%
\pgfpathcurveto{\pgfqpoint{1.519004in}{2.044660in}}{\pgfqpoint{1.515732in}{2.036760in}}{\pgfqpoint{1.515732in}{2.028524in}}%
\pgfpathcurveto{\pgfqpoint{1.515732in}{2.020288in}}{\pgfqpoint{1.519004in}{2.012388in}}{\pgfqpoint{1.524828in}{2.006564in}}%
\pgfpathcurveto{\pgfqpoint{1.530652in}{2.000740in}}{\pgfqpoint{1.538552in}{1.997468in}}{\pgfqpoint{1.546789in}{1.997468in}}%
\pgfpathclose%
\pgfusepath{stroke,fill}%
\end{pgfscope}%
\begin{pgfscope}%
\pgfpathrectangle{\pgfqpoint{0.100000in}{0.212622in}}{\pgfqpoint{3.696000in}{3.696000in}}%
\pgfusepath{clip}%
\pgfsetbuttcap%
\pgfsetroundjoin%
\definecolor{currentfill}{rgb}{0.121569,0.466667,0.705882}%
\pgfsetfillcolor{currentfill}%
\pgfsetfillopacity{0.348872}%
\pgfsetlinewidth{1.003750pt}%
\definecolor{currentstroke}{rgb}{0.121569,0.466667,0.705882}%
\pgfsetstrokecolor{currentstroke}%
\pgfsetstrokeopacity{0.348872}%
\pgfsetdash{}{0pt}%
\pgfpathmoveto{\pgfqpoint{1.544646in}{1.995817in}}%
\pgfpathcurveto{\pgfqpoint{1.552882in}{1.995817in}}{\pgfqpoint{1.560782in}{1.999089in}}{\pgfqpoint{1.566606in}{2.004913in}}%
\pgfpathcurveto{\pgfqpoint{1.572430in}{2.010737in}}{\pgfqpoint{1.575702in}{2.018637in}}{\pgfqpoint{1.575702in}{2.026873in}}%
\pgfpathcurveto{\pgfqpoint{1.575702in}{2.035110in}}{\pgfqpoint{1.572430in}{2.043010in}}{\pgfqpoint{1.566606in}{2.048834in}}%
\pgfpathcurveto{\pgfqpoint{1.560782in}{2.054658in}}{\pgfqpoint{1.552882in}{2.057930in}}{\pgfqpoint{1.544646in}{2.057930in}}%
\pgfpathcurveto{\pgfqpoint{1.536410in}{2.057930in}}{\pgfqpoint{1.528510in}{2.054658in}}{\pgfqpoint{1.522686in}{2.048834in}}%
\pgfpathcurveto{\pgfqpoint{1.516862in}{2.043010in}}{\pgfqpoint{1.513589in}{2.035110in}}{\pgfqpoint{1.513589in}{2.026873in}}%
\pgfpathcurveto{\pgfqpoint{1.513589in}{2.018637in}}{\pgfqpoint{1.516862in}{2.010737in}}{\pgfqpoint{1.522686in}{2.004913in}}%
\pgfpathcurveto{\pgfqpoint{1.528510in}{1.999089in}}{\pgfqpoint{1.536410in}{1.995817in}}{\pgfqpoint{1.544646in}{1.995817in}}%
\pgfpathclose%
\pgfusepath{stroke,fill}%
\end{pgfscope}%
\begin{pgfscope}%
\pgfpathrectangle{\pgfqpoint{0.100000in}{0.212622in}}{\pgfqpoint{3.696000in}{3.696000in}}%
\pgfusepath{clip}%
\pgfsetbuttcap%
\pgfsetroundjoin%
\definecolor{currentfill}{rgb}{0.121569,0.466667,0.705882}%
\pgfsetfillcolor{currentfill}%
\pgfsetfillopacity{0.348967}%
\pgfsetlinewidth{1.003750pt}%
\definecolor{currentstroke}{rgb}{0.121569,0.466667,0.705882}%
\pgfsetstrokecolor{currentstroke}%
\pgfsetstrokeopacity{0.348967}%
\pgfsetdash{}{0pt}%
\pgfpathmoveto{\pgfqpoint{1.543004in}{1.992049in}}%
\pgfpathcurveto{\pgfqpoint{1.551240in}{1.992049in}}{\pgfqpoint{1.559140in}{1.995321in}}{\pgfqpoint{1.564964in}{2.001145in}}%
\pgfpathcurveto{\pgfqpoint{1.570788in}{2.006969in}}{\pgfqpoint{1.574060in}{2.014869in}}{\pgfqpoint{1.574060in}{2.023106in}}%
\pgfpathcurveto{\pgfqpoint{1.574060in}{2.031342in}}{\pgfqpoint{1.570788in}{2.039242in}}{\pgfqpoint{1.564964in}{2.045066in}}%
\pgfpathcurveto{\pgfqpoint{1.559140in}{2.050890in}}{\pgfqpoint{1.551240in}{2.054162in}}{\pgfqpoint{1.543004in}{2.054162in}}%
\pgfpathcurveto{\pgfqpoint{1.534767in}{2.054162in}}{\pgfqpoint{1.526867in}{2.050890in}}{\pgfqpoint{1.521043in}{2.045066in}}%
\pgfpathcurveto{\pgfqpoint{1.515220in}{2.039242in}}{\pgfqpoint{1.511947in}{2.031342in}}{\pgfqpoint{1.511947in}{2.023106in}}%
\pgfpathcurveto{\pgfqpoint{1.511947in}{2.014869in}}{\pgfqpoint{1.515220in}{2.006969in}}{\pgfqpoint{1.521043in}{2.001145in}}%
\pgfpathcurveto{\pgfqpoint{1.526867in}{1.995321in}}{\pgfqpoint{1.534767in}{1.992049in}}{\pgfqpoint{1.543004in}{1.992049in}}%
\pgfpathclose%
\pgfusepath{stroke,fill}%
\end{pgfscope}%
\begin{pgfscope}%
\pgfpathrectangle{\pgfqpoint{0.100000in}{0.212622in}}{\pgfqpoint{3.696000in}{3.696000in}}%
\pgfusepath{clip}%
\pgfsetbuttcap%
\pgfsetroundjoin%
\definecolor{currentfill}{rgb}{0.121569,0.466667,0.705882}%
\pgfsetfillcolor{currentfill}%
\pgfsetfillopacity{0.349032}%
\pgfsetlinewidth{1.003750pt}%
\definecolor{currentstroke}{rgb}{0.121569,0.466667,0.705882}%
\pgfsetstrokecolor{currentstroke}%
\pgfsetstrokeopacity{0.349032}%
\pgfsetdash{}{0pt}%
\pgfpathmoveto{\pgfqpoint{1.944993in}{2.124183in}}%
\pgfpathcurveto{\pgfqpoint{1.953230in}{2.124183in}}{\pgfqpoint{1.961130in}{2.127456in}}{\pgfqpoint{1.966954in}{2.133279in}}%
\pgfpathcurveto{\pgfqpoint{1.972777in}{2.139103in}}{\pgfqpoint{1.976050in}{2.147003in}}{\pgfqpoint{1.976050in}{2.155240in}}%
\pgfpathcurveto{\pgfqpoint{1.976050in}{2.163476in}}{\pgfqpoint{1.972777in}{2.171376in}}{\pgfqpoint{1.966954in}{2.177200in}}%
\pgfpathcurveto{\pgfqpoint{1.961130in}{2.183024in}}{\pgfqpoint{1.953230in}{2.186296in}}{\pgfqpoint{1.944993in}{2.186296in}}%
\pgfpathcurveto{\pgfqpoint{1.936757in}{2.186296in}}{\pgfqpoint{1.928857in}{2.183024in}}{\pgfqpoint{1.923033in}{2.177200in}}%
\pgfpathcurveto{\pgfqpoint{1.917209in}{2.171376in}}{\pgfqpoint{1.913937in}{2.163476in}}{\pgfqpoint{1.913937in}{2.155240in}}%
\pgfpathcurveto{\pgfqpoint{1.913937in}{2.147003in}}{\pgfqpoint{1.917209in}{2.139103in}}{\pgfqpoint{1.923033in}{2.133279in}}%
\pgfpathcurveto{\pgfqpoint{1.928857in}{2.127456in}}{\pgfqpoint{1.936757in}{2.124183in}}{\pgfqpoint{1.944993in}{2.124183in}}%
\pgfpathclose%
\pgfusepath{stroke,fill}%
\end{pgfscope}%
\begin{pgfscope}%
\pgfpathrectangle{\pgfqpoint{0.100000in}{0.212622in}}{\pgfqpoint{3.696000in}{3.696000in}}%
\pgfusepath{clip}%
\pgfsetbuttcap%
\pgfsetroundjoin%
\definecolor{currentfill}{rgb}{0.121569,0.466667,0.705882}%
\pgfsetfillcolor{currentfill}%
\pgfsetfillopacity{0.349544}%
\pgfsetlinewidth{1.003750pt}%
\definecolor{currentstroke}{rgb}{0.121569,0.466667,0.705882}%
\pgfsetstrokecolor{currentstroke}%
\pgfsetstrokeopacity{0.349544}%
\pgfsetdash{}{0pt}%
\pgfpathmoveto{\pgfqpoint{1.542285in}{1.991879in}}%
\pgfpathcurveto{\pgfqpoint{1.550521in}{1.991879in}}{\pgfqpoint{1.558421in}{1.995151in}}{\pgfqpoint{1.564245in}{2.000975in}}%
\pgfpathcurveto{\pgfqpoint{1.570069in}{2.006799in}}{\pgfqpoint{1.573341in}{2.014699in}}{\pgfqpoint{1.573341in}{2.022936in}}%
\pgfpathcurveto{\pgfqpoint{1.573341in}{2.031172in}}{\pgfqpoint{1.570069in}{2.039072in}}{\pgfqpoint{1.564245in}{2.044896in}}%
\pgfpathcurveto{\pgfqpoint{1.558421in}{2.050720in}}{\pgfqpoint{1.550521in}{2.053992in}}{\pgfqpoint{1.542285in}{2.053992in}}%
\pgfpathcurveto{\pgfqpoint{1.534049in}{2.053992in}}{\pgfqpoint{1.526148in}{2.050720in}}{\pgfqpoint{1.520325in}{2.044896in}}%
\pgfpathcurveto{\pgfqpoint{1.514501in}{2.039072in}}{\pgfqpoint{1.511228in}{2.031172in}}{\pgfqpoint{1.511228in}{2.022936in}}%
\pgfpathcurveto{\pgfqpoint{1.511228in}{2.014699in}}{\pgfqpoint{1.514501in}{2.006799in}}{\pgfqpoint{1.520325in}{2.000975in}}%
\pgfpathcurveto{\pgfqpoint{1.526148in}{1.995151in}}{\pgfqpoint{1.534049in}{1.991879in}}{\pgfqpoint{1.542285in}{1.991879in}}%
\pgfpathclose%
\pgfusepath{stroke,fill}%
\end{pgfscope}%
\begin{pgfscope}%
\pgfpathrectangle{\pgfqpoint{0.100000in}{0.212622in}}{\pgfqpoint{3.696000in}{3.696000in}}%
\pgfusepath{clip}%
\pgfsetbuttcap%
\pgfsetroundjoin%
\definecolor{currentfill}{rgb}{0.121569,0.466667,0.705882}%
\pgfsetfillcolor{currentfill}%
\pgfsetfillopacity{0.349822}%
\pgfsetlinewidth{1.003750pt}%
\definecolor{currentstroke}{rgb}{0.121569,0.466667,0.705882}%
\pgfsetstrokecolor{currentstroke}%
\pgfsetstrokeopacity{0.349822}%
\pgfsetdash{}{0pt}%
\pgfpathmoveto{\pgfqpoint{1.948553in}{2.124258in}}%
\pgfpathcurveto{\pgfqpoint{1.956789in}{2.124258in}}{\pgfqpoint{1.964689in}{2.127531in}}{\pgfqpoint{1.970513in}{2.133355in}}%
\pgfpathcurveto{\pgfqpoint{1.976337in}{2.139179in}}{\pgfqpoint{1.979609in}{2.147079in}}{\pgfqpoint{1.979609in}{2.155315in}}%
\pgfpathcurveto{\pgfqpoint{1.979609in}{2.163551in}}{\pgfqpoint{1.976337in}{2.171451in}}{\pgfqpoint{1.970513in}{2.177275in}}%
\pgfpathcurveto{\pgfqpoint{1.964689in}{2.183099in}}{\pgfqpoint{1.956789in}{2.186371in}}{\pgfqpoint{1.948553in}{2.186371in}}%
\pgfpathcurveto{\pgfqpoint{1.940317in}{2.186371in}}{\pgfqpoint{1.932417in}{2.183099in}}{\pgfqpoint{1.926593in}{2.177275in}}%
\pgfpathcurveto{\pgfqpoint{1.920769in}{2.171451in}}{\pgfqpoint{1.917496in}{2.163551in}}{\pgfqpoint{1.917496in}{2.155315in}}%
\pgfpathcurveto{\pgfqpoint{1.917496in}{2.147079in}}{\pgfqpoint{1.920769in}{2.139179in}}{\pgfqpoint{1.926593in}{2.133355in}}%
\pgfpathcurveto{\pgfqpoint{1.932417in}{2.127531in}}{\pgfqpoint{1.940317in}{2.124258in}}{\pgfqpoint{1.948553in}{2.124258in}}%
\pgfpathclose%
\pgfusepath{stroke,fill}%
\end{pgfscope}%
\begin{pgfscope}%
\pgfpathrectangle{\pgfqpoint{0.100000in}{0.212622in}}{\pgfqpoint{3.696000in}{3.696000in}}%
\pgfusepath{clip}%
\pgfsetbuttcap%
\pgfsetroundjoin%
\definecolor{currentfill}{rgb}{0.121569,0.466667,0.705882}%
\pgfsetfillcolor{currentfill}%
\pgfsetfillopacity{0.349824}%
\pgfsetlinewidth{1.003750pt}%
\definecolor{currentstroke}{rgb}{0.121569,0.466667,0.705882}%
\pgfsetstrokecolor{currentstroke}%
\pgfsetstrokeopacity{0.349824}%
\pgfsetdash{}{0pt}%
\pgfpathmoveto{\pgfqpoint{1.541051in}{1.990968in}}%
\pgfpathcurveto{\pgfqpoint{1.549287in}{1.990968in}}{\pgfqpoint{1.557187in}{1.994240in}}{\pgfqpoint{1.563011in}{2.000064in}}%
\pgfpathcurveto{\pgfqpoint{1.568835in}{2.005888in}}{\pgfqpoint{1.572107in}{2.013788in}}{\pgfqpoint{1.572107in}{2.022024in}}%
\pgfpathcurveto{\pgfqpoint{1.572107in}{2.030260in}}{\pgfqpoint{1.568835in}{2.038161in}}{\pgfqpoint{1.563011in}{2.043984in}}%
\pgfpathcurveto{\pgfqpoint{1.557187in}{2.049808in}}{\pgfqpoint{1.549287in}{2.053081in}}{\pgfqpoint{1.541051in}{2.053081in}}%
\pgfpathcurveto{\pgfqpoint{1.532815in}{2.053081in}}{\pgfqpoint{1.524915in}{2.049808in}}{\pgfqpoint{1.519091in}{2.043984in}}%
\pgfpathcurveto{\pgfqpoint{1.513267in}{2.038161in}}{\pgfqpoint{1.509994in}{2.030260in}}{\pgfqpoint{1.509994in}{2.022024in}}%
\pgfpathcurveto{\pgfqpoint{1.509994in}{2.013788in}}{\pgfqpoint{1.513267in}{2.005888in}}{\pgfqpoint{1.519091in}{2.000064in}}%
\pgfpathcurveto{\pgfqpoint{1.524915in}{1.994240in}}{\pgfqpoint{1.532815in}{1.990968in}}{\pgfqpoint{1.541051in}{1.990968in}}%
\pgfpathclose%
\pgfusepath{stroke,fill}%
\end{pgfscope}%
\begin{pgfscope}%
\pgfpathrectangle{\pgfqpoint{0.100000in}{0.212622in}}{\pgfqpoint{3.696000in}{3.696000in}}%
\pgfusepath{clip}%
\pgfsetbuttcap%
\pgfsetroundjoin%
\definecolor{currentfill}{rgb}{0.121569,0.466667,0.705882}%
\pgfsetfillcolor{currentfill}%
\pgfsetfillopacity{0.349976}%
\pgfsetlinewidth{1.003750pt}%
\definecolor{currentstroke}{rgb}{0.121569,0.466667,0.705882}%
\pgfsetstrokecolor{currentstroke}%
\pgfsetstrokeopacity{0.349976}%
\pgfsetdash{}{0pt}%
\pgfpathmoveto{\pgfqpoint{1.540497in}{1.990300in}}%
\pgfpathcurveto{\pgfqpoint{1.548734in}{1.990300in}}{\pgfqpoint{1.556634in}{1.993572in}}{\pgfqpoint{1.562458in}{1.999396in}}%
\pgfpathcurveto{\pgfqpoint{1.568282in}{2.005220in}}{\pgfqpoint{1.571554in}{2.013120in}}{\pgfqpoint{1.571554in}{2.021356in}}%
\pgfpathcurveto{\pgfqpoint{1.571554in}{2.029593in}}{\pgfqpoint{1.568282in}{2.037493in}}{\pgfqpoint{1.562458in}{2.043317in}}%
\pgfpathcurveto{\pgfqpoint{1.556634in}{2.049141in}}{\pgfqpoint{1.548734in}{2.052413in}}{\pgfqpoint{1.540497in}{2.052413in}}%
\pgfpathcurveto{\pgfqpoint{1.532261in}{2.052413in}}{\pgfqpoint{1.524361in}{2.049141in}}{\pgfqpoint{1.518537in}{2.043317in}}%
\pgfpathcurveto{\pgfqpoint{1.512713in}{2.037493in}}{\pgfqpoint{1.509441in}{2.029593in}}{\pgfqpoint{1.509441in}{2.021356in}}%
\pgfpathcurveto{\pgfqpoint{1.509441in}{2.013120in}}{\pgfqpoint{1.512713in}{2.005220in}}{\pgfqpoint{1.518537in}{1.999396in}}%
\pgfpathcurveto{\pgfqpoint{1.524361in}{1.993572in}}{\pgfqpoint{1.532261in}{1.990300in}}{\pgfqpoint{1.540497in}{1.990300in}}%
\pgfpathclose%
\pgfusepath{stroke,fill}%
\end{pgfscope}%
\begin{pgfscope}%
\pgfpathrectangle{\pgfqpoint{0.100000in}{0.212622in}}{\pgfqpoint{3.696000in}{3.696000in}}%
\pgfusepath{clip}%
\pgfsetbuttcap%
\pgfsetroundjoin%
\definecolor{currentfill}{rgb}{0.121569,0.466667,0.705882}%
\pgfsetfillcolor{currentfill}%
\pgfsetfillopacity{0.350253}%
\pgfsetlinewidth{1.003750pt}%
\definecolor{currentstroke}{rgb}{0.121569,0.466667,0.705882}%
\pgfsetstrokecolor{currentstroke}%
\pgfsetstrokeopacity{0.350253}%
\pgfsetdash{}{0pt}%
\pgfpathmoveto{\pgfqpoint{1.950509in}{2.124270in}}%
\pgfpathcurveto{\pgfqpoint{1.958745in}{2.124270in}}{\pgfqpoint{1.966645in}{2.127542in}}{\pgfqpoint{1.972469in}{2.133366in}}%
\pgfpathcurveto{\pgfqpoint{1.978293in}{2.139190in}}{\pgfqpoint{1.981565in}{2.147090in}}{\pgfqpoint{1.981565in}{2.155326in}}%
\pgfpathcurveto{\pgfqpoint{1.981565in}{2.163562in}}{\pgfqpoint{1.978293in}{2.171463in}}{\pgfqpoint{1.972469in}{2.177286in}}%
\pgfpathcurveto{\pgfqpoint{1.966645in}{2.183110in}}{\pgfqpoint{1.958745in}{2.186383in}}{\pgfqpoint{1.950509in}{2.186383in}}%
\pgfpathcurveto{\pgfqpoint{1.942272in}{2.186383in}}{\pgfqpoint{1.934372in}{2.183110in}}{\pgfqpoint{1.928548in}{2.177286in}}%
\pgfpathcurveto{\pgfqpoint{1.922724in}{2.171463in}}{\pgfqpoint{1.919452in}{2.163562in}}{\pgfqpoint{1.919452in}{2.155326in}}%
\pgfpathcurveto{\pgfqpoint{1.919452in}{2.147090in}}{\pgfqpoint{1.922724in}{2.139190in}}{\pgfqpoint{1.928548in}{2.133366in}}%
\pgfpathcurveto{\pgfqpoint{1.934372in}{2.127542in}}{\pgfqpoint{1.942272in}{2.124270in}}{\pgfqpoint{1.950509in}{2.124270in}}%
\pgfpathclose%
\pgfusepath{stroke,fill}%
\end{pgfscope}%
\begin{pgfscope}%
\pgfpathrectangle{\pgfqpoint{0.100000in}{0.212622in}}{\pgfqpoint{3.696000in}{3.696000in}}%
\pgfusepath{clip}%
\pgfsetbuttcap%
\pgfsetroundjoin%
\definecolor{currentfill}{rgb}{0.121569,0.466667,0.705882}%
\pgfsetfillcolor{currentfill}%
\pgfsetfillopacity{0.350383}%
\pgfsetlinewidth{1.003750pt}%
\definecolor{currentstroke}{rgb}{0.121569,0.466667,0.705882}%
\pgfsetstrokecolor{currentstroke}%
\pgfsetstrokeopacity{0.350383}%
\pgfsetdash{}{0pt}%
\pgfpathmoveto{\pgfqpoint{1.539845in}{1.989718in}}%
\pgfpathcurveto{\pgfqpoint{1.548081in}{1.989718in}}{\pgfqpoint{1.555981in}{1.992990in}}{\pgfqpoint{1.561805in}{1.998814in}}%
\pgfpathcurveto{\pgfqpoint{1.567629in}{2.004638in}}{\pgfqpoint{1.570902in}{2.012538in}}{\pgfqpoint{1.570902in}{2.020774in}}%
\pgfpathcurveto{\pgfqpoint{1.570902in}{2.029011in}}{\pgfqpoint{1.567629in}{2.036911in}}{\pgfqpoint{1.561805in}{2.042735in}}%
\pgfpathcurveto{\pgfqpoint{1.555981in}{2.048559in}}{\pgfqpoint{1.548081in}{2.051831in}}{\pgfqpoint{1.539845in}{2.051831in}}%
\pgfpathcurveto{\pgfqpoint{1.531609in}{2.051831in}}{\pgfqpoint{1.523709in}{2.048559in}}{\pgfqpoint{1.517885in}{2.042735in}}%
\pgfpathcurveto{\pgfqpoint{1.512061in}{2.036911in}}{\pgfqpoint{1.508789in}{2.029011in}}{\pgfqpoint{1.508789in}{2.020774in}}%
\pgfpathcurveto{\pgfqpoint{1.508789in}{2.012538in}}{\pgfqpoint{1.512061in}{2.004638in}}{\pgfqpoint{1.517885in}{1.998814in}}%
\pgfpathcurveto{\pgfqpoint{1.523709in}{1.992990in}}{\pgfqpoint{1.531609in}{1.989718in}}{\pgfqpoint{1.539845in}{1.989718in}}%
\pgfpathclose%
\pgfusepath{stroke,fill}%
\end{pgfscope}%
\begin{pgfscope}%
\pgfpathrectangle{\pgfqpoint{0.100000in}{0.212622in}}{\pgfqpoint{3.696000in}{3.696000in}}%
\pgfusepath{clip}%
\pgfsetbuttcap%
\pgfsetroundjoin%
\definecolor{currentfill}{rgb}{0.121569,0.466667,0.705882}%
\pgfsetfillcolor{currentfill}%
\pgfsetfillopacity{0.350469}%
\pgfsetlinewidth{1.003750pt}%
\definecolor{currentstroke}{rgb}{0.121569,0.466667,0.705882}%
\pgfsetstrokecolor{currentstroke}%
\pgfsetstrokeopacity{0.350469}%
\pgfsetdash{}{0pt}%
\pgfpathmoveto{\pgfqpoint{1.539542in}{1.989146in}}%
\pgfpathcurveto{\pgfqpoint{1.547778in}{1.989146in}}{\pgfqpoint{1.555678in}{1.992418in}}{\pgfqpoint{1.561502in}{1.998242in}}%
\pgfpathcurveto{\pgfqpoint{1.567326in}{2.004066in}}{\pgfqpoint{1.570599in}{2.011966in}}{\pgfqpoint{1.570599in}{2.020203in}}%
\pgfpathcurveto{\pgfqpoint{1.570599in}{2.028439in}}{\pgfqpoint{1.567326in}{2.036339in}}{\pgfqpoint{1.561502in}{2.042163in}}%
\pgfpathcurveto{\pgfqpoint{1.555678in}{2.047987in}}{\pgfqpoint{1.547778in}{2.051259in}}{\pgfqpoint{1.539542in}{2.051259in}}%
\pgfpathcurveto{\pgfqpoint{1.531306in}{2.051259in}}{\pgfqpoint{1.523406in}{2.047987in}}{\pgfqpoint{1.517582in}{2.042163in}}%
\pgfpathcurveto{\pgfqpoint{1.511758in}{2.036339in}}{\pgfqpoint{1.508486in}{2.028439in}}{\pgfqpoint{1.508486in}{2.020203in}}%
\pgfpathcurveto{\pgfqpoint{1.508486in}{2.011966in}}{\pgfqpoint{1.511758in}{2.004066in}}{\pgfqpoint{1.517582in}{1.998242in}}%
\pgfpathcurveto{\pgfqpoint{1.523406in}{1.992418in}}{\pgfqpoint{1.531306in}{1.989146in}}{\pgfqpoint{1.539542in}{1.989146in}}%
\pgfpathclose%
\pgfusepath{stroke,fill}%
\end{pgfscope}%
\begin{pgfscope}%
\pgfpathrectangle{\pgfqpoint{0.100000in}{0.212622in}}{\pgfqpoint{3.696000in}{3.696000in}}%
\pgfusepath{clip}%
\pgfsetbuttcap%
\pgfsetroundjoin%
\definecolor{currentfill}{rgb}{0.121569,0.466667,0.705882}%
\pgfsetfillcolor{currentfill}%
\pgfsetfillopacity{0.350636}%
\pgfsetlinewidth{1.003750pt}%
\definecolor{currentstroke}{rgb}{0.121569,0.466667,0.705882}%
\pgfsetstrokecolor{currentstroke}%
\pgfsetstrokeopacity{0.350636}%
\pgfsetdash{}{0pt}%
\pgfpathmoveto{\pgfqpoint{1.538887in}{1.988295in}}%
\pgfpathcurveto{\pgfqpoint{1.547123in}{1.988295in}}{\pgfqpoint{1.555023in}{1.991567in}}{\pgfqpoint{1.560847in}{1.997391in}}%
\pgfpathcurveto{\pgfqpoint{1.566671in}{2.003215in}}{\pgfqpoint{1.569944in}{2.011115in}}{\pgfqpoint{1.569944in}{2.019351in}}%
\pgfpathcurveto{\pgfqpoint{1.569944in}{2.027588in}}{\pgfqpoint{1.566671in}{2.035488in}}{\pgfqpoint{1.560847in}{2.041312in}}%
\pgfpathcurveto{\pgfqpoint{1.555023in}{2.047135in}}{\pgfqpoint{1.547123in}{2.050408in}}{\pgfqpoint{1.538887in}{2.050408in}}%
\pgfpathcurveto{\pgfqpoint{1.530651in}{2.050408in}}{\pgfqpoint{1.522751in}{2.047135in}}{\pgfqpoint{1.516927in}{2.041312in}}%
\pgfpathcurveto{\pgfqpoint{1.511103in}{2.035488in}}{\pgfqpoint{1.507831in}{2.027588in}}{\pgfqpoint{1.507831in}{2.019351in}}%
\pgfpathcurveto{\pgfqpoint{1.507831in}{2.011115in}}{\pgfqpoint{1.511103in}{2.003215in}}{\pgfqpoint{1.516927in}{1.997391in}}%
\pgfpathcurveto{\pgfqpoint{1.522751in}{1.991567in}}{\pgfqpoint{1.530651in}{1.988295in}}{\pgfqpoint{1.538887in}{1.988295in}}%
\pgfpathclose%
\pgfusepath{stroke,fill}%
\end{pgfscope}%
\begin{pgfscope}%
\pgfpathrectangle{\pgfqpoint{0.100000in}{0.212622in}}{\pgfqpoint{3.696000in}{3.696000in}}%
\pgfusepath{clip}%
\pgfsetbuttcap%
\pgfsetroundjoin%
\definecolor{currentfill}{rgb}{0.121569,0.466667,0.705882}%
\pgfsetfillcolor{currentfill}%
\pgfsetfillopacity{0.350816}%
\pgfsetlinewidth{1.003750pt}%
\definecolor{currentstroke}{rgb}{0.121569,0.466667,0.705882}%
\pgfsetstrokecolor{currentstroke}%
\pgfsetstrokeopacity{0.350816}%
\pgfsetdash{}{0pt}%
\pgfpathmoveto{\pgfqpoint{1.953186in}{2.125331in}}%
\pgfpathcurveto{\pgfqpoint{1.961422in}{2.125331in}}{\pgfqpoint{1.969322in}{2.128603in}}{\pgfqpoint{1.975146in}{2.134427in}}%
\pgfpathcurveto{\pgfqpoint{1.980970in}{2.140251in}}{\pgfqpoint{1.984242in}{2.148151in}}{\pgfqpoint{1.984242in}{2.156388in}}%
\pgfpathcurveto{\pgfqpoint{1.984242in}{2.164624in}}{\pgfqpoint{1.980970in}{2.172524in}}{\pgfqpoint{1.975146in}{2.178348in}}%
\pgfpathcurveto{\pgfqpoint{1.969322in}{2.184172in}}{\pgfqpoint{1.961422in}{2.187444in}}{\pgfqpoint{1.953186in}{2.187444in}}%
\pgfpathcurveto{\pgfqpoint{1.944950in}{2.187444in}}{\pgfqpoint{1.937049in}{2.184172in}}{\pgfqpoint{1.931226in}{2.178348in}}%
\pgfpathcurveto{\pgfqpoint{1.925402in}{2.172524in}}{\pgfqpoint{1.922129in}{2.164624in}}{\pgfqpoint{1.922129in}{2.156388in}}%
\pgfpathcurveto{\pgfqpoint{1.922129in}{2.148151in}}{\pgfqpoint{1.925402in}{2.140251in}}{\pgfqpoint{1.931226in}{2.134427in}}%
\pgfpathcurveto{\pgfqpoint{1.937049in}{2.128603in}}{\pgfqpoint{1.944950in}{2.125331in}}{\pgfqpoint{1.953186in}{2.125331in}}%
\pgfpathclose%
\pgfusepath{stroke,fill}%
\end{pgfscope}%
\begin{pgfscope}%
\pgfpathrectangle{\pgfqpoint{0.100000in}{0.212622in}}{\pgfqpoint{3.696000in}{3.696000in}}%
\pgfusepath{clip}%
\pgfsetbuttcap%
\pgfsetroundjoin%
\definecolor{currentfill}{rgb}{0.121569,0.466667,0.705882}%
\pgfsetfillcolor{currentfill}%
\pgfsetfillopacity{0.350993}%
\pgfsetlinewidth{1.003750pt}%
\definecolor{currentstroke}{rgb}{0.121569,0.466667,0.705882}%
\pgfsetstrokecolor{currentstroke}%
\pgfsetstrokeopacity{0.350993}%
\pgfsetdash{}{0pt}%
\pgfpathmoveto{\pgfqpoint{1.956307in}{2.122335in}}%
\pgfpathcurveto{\pgfqpoint{1.964543in}{2.122335in}}{\pgfqpoint{1.972443in}{2.125607in}}{\pgfqpoint{1.978267in}{2.131431in}}%
\pgfpathcurveto{\pgfqpoint{1.984091in}{2.137255in}}{\pgfqpoint{1.987363in}{2.145155in}}{\pgfqpoint{1.987363in}{2.153391in}}%
\pgfpathcurveto{\pgfqpoint{1.987363in}{2.161628in}}{\pgfqpoint{1.984091in}{2.169528in}}{\pgfqpoint{1.978267in}{2.175352in}}%
\pgfpathcurveto{\pgfqpoint{1.972443in}{2.181176in}}{\pgfqpoint{1.964543in}{2.184448in}}{\pgfqpoint{1.956307in}{2.184448in}}%
\pgfpathcurveto{\pgfqpoint{1.948070in}{2.184448in}}{\pgfqpoint{1.940170in}{2.181176in}}{\pgfqpoint{1.934346in}{2.175352in}}%
\pgfpathcurveto{\pgfqpoint{1.928522in}{2.169528in}}{\pgfqpoint{1.925250in}{2.161628in}}{\pgfqpoint{1.925250in}{2.153391in}}%
\pgfpathcurveto{\pgfqpoint{1.925250in}{2.145155in}}{\pgfqpoint{1.928522in}{2.137255in}}{\pgfqpoint{1.934346in}{2.131431in}}%
\pgfpathcurveto{\pgfqpoint{1.940170in}{2.125607in}}{\pgfqpoint{1.948070in}{2.122335in}}{\pgfqpoint{1.956307in}{2.122335in}}%
\pgfpathclose%
\pgfusepath{stroke,fill}%
\end{pgfscope}%
\begin{pgfscope}%
\pgfpathrectangle{\pgfqpoint{0.100000in}{0.212622in}}{\pgfqpoint{3.696000in}{3.696000in}}%
\pgfusepath{clip}%
\pgfsetbuttcap%
\pgfsetroundjoin%
\definecolor{currentfill}{rgb}{0.121569,0.466667,0.705882}%
\pgfsetfillcolor{currentfill}%
\pgfsetfillopacity{0.351160}%
\pgfsetlinewidth{1.003750pt}%
\definecolor{currentstroke}{rgb}{0.121569,0.466667,0.705882}%
\pgfsetstrokecolor{currentstroke}%
\pgfsetstrokeopacity{0.351160}%
\pgfsetdash{}{0pt}%
\pgfpathmoveto{\pgfqpoint{1.538237in}{1.987915in}}%
\pgfpathcurveto{\pgfqpoint{1.546473in}{1.987915in}}{\pgfqpoint{1.554374in}{1.991187in}}{\pgfqpoint{1.560197in}{1.997011in}}%
\pgfpathcurveto{\pgfqpoint{1.566021in}{2.002835in}}{\pgfqpoint{1.569294in}{2.010735in}}{\pgfqpoint{1.569294in}{2.018971in}}%
\pgfpathcurveto{\pgfqpoint{1.569294in}{2.027208in}}{\pgfqpoint{1.566021in}{2.035108in}}{\pgfqpoint{1.560197in}{2.040932in}}%
\pgfpathcurveto{\pgfqpoint{1.554374in}{2.046755in}}{\pgfqpoint{1.546473in}{2.050028in}}{\pgfqpoint{1.538237in}{2.050028in}}%
\pgfpathcurveto{\pgfqpoint{1.530001in}{2.050028in}}{\pgfqpoint{1.522101in}{2.046755in}}{\pgfqpoint{1.516277in}{2.040932in}}%
\pgfpathcurveto{\pgfqpoint{1.510453in}{2.035108in}}{\pgfqpoint{1.507181in}{2.027208in}}{\pgfqpoint{1.507181in}{2.018971in}}%
\pgfpathcurveto{\pgfqpoint{1.507181in}{2.010735in}}{\pgfqpoint{1.510453in}{2.002835in}}{\pgfqpoint{1.516277in}{1.997011in}}%
\pgfpathcurveto{\pgfqpoint{1.522101in}{1.991187in}}{\pgfqpoint{1.530001in}{1.987915in}}{\pgfqpoint{1.538237in}{1.987915in}}%
\pgfpathclose%
\pgfusepath{stroke,fill}%
\end{pgfscope}%
\begin{pgfscope}%
\pgfpathrectangle{\pgfqpoint{0.100000in}{0.212622in}}{\pgfqpoint{3.696000in}{3.696000in}}%
\pgfusepath{clip}%
\pgfsetbuttcap%
\pgfsetroundjoin%
\definecolor{currentfill}{rgb}{0.121569,0.466667,0.705882}%
\pgfsetfillcolor{currentfill}%
\pgfsetfillopacity{0.351257}%
\pgfsetlinewidth{1.003750pt}%
\definecolor{currentstroke}{rgb}{0.121569,0.466667,0.705882}%
\pgfsetstrokecolor{currentstroke}%
\pgfsetstrokeopacity{0.351257}%
\pgfsetdash{}{0pt}%
\pgfpathmoveto{\pgfqpoint{1.537787in}{1.987247in}}%
\pgfpathcurveto{\pgfqpoint{1.546023in}{1.987247in}}{\pgfqpoint{1.553923in}{1.990519in}}{\pgfqpoint{1.559747in}{1.996343in}}%
\pgfpathcurveto{\pgfqpoint{1.565571in}{2.002167in}}{\pgfqpoint{1.568843in}{2.010067in}}{\pgfqpoint{1.568843in}{2.018303in}}%
\pgfpathcurveto{\pgfqpoint{1.568843in}{2.026540in}}{\pgfqpoint{1.565571in}{2.034440in}}{\pgfqpoint{1.559747in}{2.040264in}}%
\pgfpathcurveto{\pgfqpoint{1.553923in}{2.046088in}}{\pgfqpoint{1.546023in}{2.049360in}}{\pgfqpoint{1.537787in}{2.049360in}}%
\pgfpathcurveto{\pgfqpoint{1.529551in}{2.049360in}}{\pgfqpoint{1.521650in}{2.046088in}}{\pgfqpoint{1.515827in}{2.040264in}}%
\pgfpathcurveto{\pgfqpoint{1.510003in}{2.034440in}}{\pgfqpoint{1.506730in}{2.026540in}}{\pgfqpoint{1.506730in}{2.018303in}}%
\pgfpathcurveto{\pgfqpoint{1.506730in}{2.010067in}}{\pgfqpoint{1.510003in}{2.002167in}}{\pgfqpoint{1.515827in}{1.996343in}}%
\pgfpathcurveto{\pgfqpoint{1.521650in}{1.990519in}}{\pgfqpoint{1.529551in}{1.987247in}}{\pgfqpoint{1.537787in}{1.987247in}}%
\pgfpathclose%
\pgfusepath{stroke,fill}%
\end{pgfscope}%
\begin{pgfscope}%
\pgfpathrectangle{\pgfqpoint{0.100000in}{0.212622in}}{\pgfqpoint{3.696000in}{3.696000in}}%
\pgfusepath{clip}%
\pgfsetbuttcap%
\pgfsetroundjoin%
\definecolor{currentfill}{rgb}{0.121569,0.466667,0.705882}%
\pgfsetfillcolor{currentfill}%
\pgfsetfillopacity{0.351388}%
\pgfsetlinewidth{1.003750pt}%
\definecolor{currentstroke}{rgb}{0.121569,0.466667,0.705882}%
\pgfsetstrokecolor{currentstroke}%
\pgfsetstrokeopacity{0.351388}%
\pgfsetdash{}{0pt}%
\pgfpathmoveto{\pgfqpoint{1.958200in}{2.123543in}}%
\pgfpathcurveto{\pgfqpoint{1.966436in}{2.123543in}}{\pgfqpoint{1.974336in}{2.126816in}}{\pgfqpoint{1.980160in}{2.132640in}}%
\pgfpathcurveto{\pgfqpoint{1.985984in}{2.138464in}}{\pgfqpoint{1.989256in}{2.146364in}}{\pgfqpoint{1.989256in}{2.154600in}}%
\pgfpathcurveto{\pgfqpoint{1.989256in}{2.162836in}}{\pgfqpoint{1.985984in}{2.170736in}}{\pgfqpoint{1.980160in}{2.176560in}}%
\pgfpathcurveto{\pgfqpoint{1.974336in}{2.182384in}}{\pgfqpoint{1.966436in}{2.185656in}}{\pgfqpoint{1.958200in}{2.185656in}}%
\pgfpathcurveto{\pgfqpoint{1.949964in}{2.185656in}}{\pgfqpoint{1.942064in}{2.182384in}}{\pgfqpoint{1.936240in}{2.176560in}}%
\pgfpathcurveto{\pgfqpoint{1.930416in}{2.170736in}}{\pgfqpoint{1.927143in}{2.162836in}}{\pgfqpoint{1.927143in}{2.154600in}}%
\pgfpathcurveto{\pgfqpoint{1.927143in}{2.146364in}}{\pgfqpoint{1.930416in}{2.138464in}}{\pgfqpoint{1.936240in}{2.132640in}}%
\pgfpathcurveto{\pgfqpoint{1.942064in}{2.126816in}}{\pgfqpoint{1.949964in}{2.123543in}}{\pgfqpoint{1.958200in}{2.123543in}}%
\pgfpathclose%
\pgfusepath{stroke,fill}%
\end{pgfscope}%
\begin{pgfscope}%
\pgfpathrectangle{\pgfqpoint{0.100000in}{0.212622in}}{\pgfqpoint{3.696000in}{3.696000in}}%
\pgfusepath{clip}%
\pgfsetbuttcap%
\pgfsetroundjoin%
\definecolor{currentfill}{rgb}{0.121569,0.466667,0.705882}%
\pgfsetfillcolor{currentfill}%
\pgfsetfillopacity{0.351553}%
\pgfsetlinewidth{1.003750pt}%
\definecolor{currentstroke}{rgb}{0.121569,0.466667,0.705882}%
\pgfsetstrokecolor{currentstroke}%
\pgfsetstrokeopacity{0.351553}%
\pgfsetdash{}{0pt}%
\pgfpathmoveto{\pgfqpoint{1.537056in}{1.986802in}}%
\pgfpathcurveto{\pgfqpoint{1.545292in}{1.986802in}}{\pgfqpoint{1.553192in}{1.990075in}}{\pgfqpoint{1.559016in}{1.995899in}}%
\pgfpathcurveto{\pgfqpoint{1.564840in}{2.001723in}}{\pgfqpoint{1.568112in}{2.009623in}}{\pgfqpoint{1.568112in}{2.017859in}}%
\pgfpathcurveto{\pgfqpoint{1.568112in}{2.026095in}}{\pgfqpoint{1.564840in}{2.033995in}}{\pgfqpoint{1.559016in}{2.039819in}}%
\pgfpathcurveto{\pgfqpoint{1.553192in}{2.045643in}}{\pgfqpoint{1.545292in}{2.048915in}}{\pgfqpoint{1.537056in}{2.048915in}}%
\pgfpathcurveto{\pgfqpoint{1.528820in}{2.048915in}}{\pgfqpoint{1.520920in}{2.045643in}}{\pgfqpoint{1.515096in}{2.039819in}}%
\pgfpathcurveto{\pgfqpoint{1.509272in}{2.033995in}}{\pgfqpoint{1.505999in}{2.026095in}}{\pgfqpoint{1.505999in}{2.017859in}}%
\pgfpathcurveto{\pgfqpoint{1.505999in}{2.009623in}}{\pgfqpoint{1.509272in}{2.001723in}}{\pgfqpoint{1.515096in}{1.995899in}}%
\pgfpathcurveto{\pgfqpoint{1.520920in}{1.990075in}}{\pgfqpoint{1.528820in}{1.986802in}}{\pgfqpoint{1.537056in}{1.986802in}}%
\pgfpathclose%
\pgfusepath{stroke,fill}%
\end{pgfscope}%
\begin{pgfscope}%
\pgfpathrectangle{\pgfqpoint{0.100000in}{0.212622in}}{\pgfqpoint{3.696000in}{3.696000in}}%
\pgfusepath{clip}%
\pgfsetbuttcap%
\pgfsetroundjoin%
\definecolor{currentfill}{rgb}{0.121569,0.466667,0.705882}%
\pgfsetfillcolor{currentfill}%
\pgfsetfillopacity{0.351606}%
\pgfsetlinewidth{1.003750pt}%
\definecolor{currentstroke}{rgb}{0.121569,0.466667,0.705882}%
\pgfsetstrokecolor{currentstroke}%
\pgfsetstrokeopacity{0.351606}%
\pgfsetdash{}{0pt}%
\pgfpathmoveto{\pgfqpoint{1.960450in}{2.121356in}}%
\pgfpathcurveto{\pgfqpoint{1.968686in}{2.121356in}}{\pgfqpoint{1.976586in}{2.124628in}}{\pgfqpoint{1.982410in}{2.130452in}}%
\pgfpathcurveto{\pgfqpoint{1.988234in}{2.136276in}}{\pgfqpoint{1.991506in}{2.144176in}}{\pgfqpoint{1.991506in}{2.152412in}}%
\pgfpathcurveto{\pgfqpoint{1.991506in}{2.160648in}}{\pgfqpoint{1.988234in}{2.168548in}}{\pgfqpoint{1.982410in}{2.174372in}}%
\pgfpathcurveto{\pgfqpoint{1.976586in}{2.180196in}}{\pgfqpoint{1.968686in}{2.183469in}}{\pgfqpoint{1.960450in}{2.183469in}}%
\pgfpathcurveto{\pgfqpoint{1.952213in}{2.183469in}}{\pgfqpoint{1.944313in}{2.180196in}}{\pgfqpoint{1.938489in}{2.174372in}}%
\pgfpathcurveto{\pgfqpoint{1.932665in}{2.168548in}}{\pgfqpoint{1.929393in}{2.160648in}}{\pgfqpoint{1.929393in}{2.152412in}}%
\pgfpathcurveto{\pgfqpoint{1.929393in}{2.144176in}}{\pgfqpoint{1.932665in}{2.136276in}}{\pgfqpoint{1.938489in}{2.130452in}}%
\pgfpathcurveto{\pgfqpoint{1.944313in}{2.124628in}}{\pgfqpoint{1.952213in}{2.121356in}}{\pgfqpoint{1.960450in}{2.121356in}}%
\pgfpathclose%
\pgfusepath{stroke,fill}%
\end{pgfscope}%
\begin{pgfscope}%
\pgfpathrectangle{\pgfqpoint{0.100000in}{0.212622in}}{\pgfqpoint{3.696000in}{3.696000in}}%
\pgfusepath{clip}%
\pgfsetbuttcap%
\pgfsetroundjoin%
\definecolor{currentfill}{rgb}{0.121569,0.466667,0.705882}%
\pgfsetfillcolor{currentfill}%
\pgfsetfillopacity{0.352113}%
\pgfsetlinewidth{1.003750pt}%
\definecolor{currentstroke}{rgb}{0.121569,0.466667,0.705882}%
\pgfsetstrokecolor{currentstroke}%
\pgfsetstrokeopacity{0.352113}%
\pgfsetdash{}{0pt}%
\pgfpathmoveto{\pgfqpoint{1.536119in}{1.985829in}}%
\pgfpathcurveto{\pgfqpoint{1.544355in}{1.985829in}}{\pgfqpoint{1.552255in}{1.989101in}}{\pgfqpoint{1.558079in}{1.994925in}}%
\pgfpathcurveto{\pgfqpoint{1.563903in}{2.000749in}}{\pgfqpoint{1.567175in}{2.008649in}}{\pgfqpoint{1.567175in}{2.016885in}}%
\pgfpathcurveto{\pgfqpoint{1.567175in}{2.025121in}}{\pgfqpoint{1.563903in}{2.033021in}}{\pgfqpoint{1.558079in}{2.038845in}}%
\pgfpathcurveto{\pgfqpoint{1.552255in}{2.044669in}}{\pgfqpoint{1.544355in}{2.047942in}}{\pgfqpoint{1.536119in}{2.047942in}}%
\pgfpathcurveto{\pgfqpoint{1.527883in}{2.047942in}}{\pgfqpoint{1.519983in}{2.044669in}}{\pgfqpoint{1.514159in}{2.038845in}}%
\pgfpathcurveto{\pgfqpoint{1.508335in}{2.033021in}}{\pgfqpoint{1.505062in}{2.025121in}}{\pgfqpoint{1.505062in}{2.016885in}}%
\pgfpathcurveto{\pgfqpoint{1.505062in}{2.008649in}}{\pgfqpoint{1.508335in}{2.000749in}}{\pgfqpoint{1.514159in}{1.994925in}}%
\pgfpathcurveto{\pgfqpoint{1.519983in}{1.989101in}}{\pgfqpoint{1.527883in}{1.985829in}}{\pgfqpoint{1.536119in}{1.985829in}}%
\pgfpathclose%
\pgfusepath{stroke,fill}%
\end{pgfscope}%
\begin{pgfscope}%
\pgfpathrectangle{\pgfqpoint{0.100000in}{0.212622in}}{\pgfqpoint{3.696000in}{3.696000in}}%
\pgfusepath{clip}%
\pgfsetbuttcap%
\pgfsetroundjoin%
\definecolor{currentfill}{rgb}{0.121569,0.466667,0.705882}%
\pgfsetfillcolor{currentfill}%
\pgfsetfillopacity{0.352327}%
\pgfsetlinewidth{1.003750pt}%
\definecolor{currentstroke}{rgb}{0.121569,0.466667,0.705882}%
\pgfsetstrokecolor{currentstroke}%
\pgfsetstrokeopacity{0.352327}%
\pgfsetdash{}{0pt}%
\pgfpathmoveto{\pgfqpoint{1.963298in}{2.122473in}}%
\pgfpathcurveto{\pgfqpoint{1.971534in}{2.122473in}}{\pgfqpoint{1.979434in}{2.125745in}}{\pgfqpoint{1.985258in}{2.131569in}}%
\pgfpathcurveto{\pgfqpoint{1.991082in}{2.137393in}}{\pgfqpoint{1.994354in}{2.145293in}}{\pgfqpoint{1.994354in}{2.153529in}}%
\pgfpathcurveto{\pgfqpoint{1.994354in}{2.161766in}}{\pgfqpoint{1.991082in}{2.169666in}}{\pgfqpoint{1.985258in}{2.175489in}}%
\pgfpathcurveto{\pgfqpoint{1.979434in}{2.181313in}}{\pgfqpoint{1.971534in}{2.184586in}}{\pgfqpoint{1.963298in}{2.184586in}}%
\pgfpathcurveto{\pgfqpoint{1.955061in}{2.184586in}}{\pgfqpoint{1.947161in}{2.181313in}}{\pgfqpoint{1.941337in}{2.175489in}}%
\pgfpathcurveto{\pgfqpoint{1.935513in}{2.169666in}}{\pgfqpoint{1.932241in}{2.161766in}}{\pgfqpoint{1.932241in}{2.153529in}}%
\pgfpathcurveto{\pgfqpoint{1.932241in}{2.145293in}}{\pgfqpoint{1.935513in}{2.137393in}}{\pgfqpoint{1.941337in}{2.131569in}}%
\pgfpathcurveto{\pgfqpoint{1.947161in}{2.125745in}}{\pgfqpoint{1.955061in}{2.122473in}}{\pgfqpoint{1.963298in}{2.122473in}}%
\pgfpathclose%
\pgfusepath{stroke,fill}%
\end{pgfscope}%
\begin{pgfscope}%
\pgfpathrectangle{\pgfqpoint{0.100000in}{0.212622in}}{\pgfqpoint{3.696000in}{3.696000in}}%
\pgfusepath{clip}%
\pgfsetbuttcap%
\pgfsetroundjoin%
\definecolor{currentfill}{rgb}{0.121569,0.466667,0.705882}%
\pgfsetfillcolor{currentfill}%
\pgfsetfillopacity{0.352408}%
\pgfsetlinewidth{1.003750pt}%
\definecolor{currentstroke}{rgb}{0.121569,0.466667,0.705882}%
\pgfsetstrokecolor{currentstroke}%
\pgfsetstrokeopacity{0.352408}%
\pgfsetdash{}{0pt}%
\pgfpathmoveto{\pgfqpoint{1.534891in}{1.984868in}}%
\pgfpathcurveto{\pgfqpoint{1.543128in}{1.984868in}}{\pgfqpoint{1.551028in}{1.988140in}}{\pgfqpoint{1.556851in}{1.993964in}}%
\pgfpathcurveto{\pgfqpoint{1.562675in}{1.999788in}}{\pgfqpoint{1.565948in}{2.007688in}}{\pgfqpoint{1.565948in}{2.015924in}}%
\pgfpathcurveto{\pgfqpoint{1.565948in}{2.024161in}}{\pgfqpoint{1.562675in}{2.032061in}}{\pgfqpoint{1.556851in}{2.037885in}}%
\pgfpathcurveto{\pgfqpoint{1.551028in}{2.043709in}}{\pgfqpoint{1.543128in}{2.046981in}}{\pgfqpoint{1.534891in}{2.046981in}}%
\pgfpathcurveto{\pgfqpoint{1.526655in}{2.046981in}}{\pgfqpoint{1.518755in}{2.043709in}}{\pgfqpoint{1.512931in}{2.037885in}}%
\pgfpathcurveto{\pgfqpoint{1.507107in}{2.032061in}}{\pgfqpoint{1.503835in}{2.024161in}}{\pgfqpoint{1.503835in}{2.015924in}}%
\pgfpathcurveto{\pgfqpoint{1.503835in}{2.007688in}}{\pgfqpoint{1.507107in}{1.999788in}}{\pgfqpoint{1.512931in}{1.993964in}}%
\pgfpathcurveto{\pgfqpoint{1.518755in}{1.988140in}}{\pgfqpoint{1.526655in}{1.984868in}}{\pgfqpoint{1.534891in}{1.984868in}}%
\pgfpathclose%
\pgfusepath{stroke,fill}%
\end{pgfscope}%
\begin{pgfscope}%
\pgfpathrectangle{\pgfqpoint{0.100000in}{0.212622in}}{\pgfqpoint{3.696000in}{3.696000in}}%
\pgfusepath{clip}%
\pgfsetbuttcap%
\pgfsetroundjoin%
\definecolor{currentfill}{rgb}{0.121569,0.466667,0.705882}%
\pgfsetfillcolor{currentfill}%
\pgfsetfillopacity{0.353197}%
\pgfsetlinewidth{1.003750pt}%
\definecolor{currentstroke}{rgb}{0.121569,0.466667,0.705882}%
\pgfsetstrokecolor{currentstroke}%
\pgfsetstrokeopacity{0.353197}%
\pgfsetdash{}{0pt}%
\pgfpathmoveto{\pgfqpoint{1.533438in}{1.984093in}}%
\pgfpathcurveto{\pgfqpoint{1.541674in}{1.984093in}}{\pgfqpoint{1.549574in}{1.987365in}}{\pgfqpoint{1.555398in}{1.993189in}}%
\pgfpathcurveto{\pgfqpoint{1.561222in}{1.999013in}}{\pgfqpoint{1.564495in}{2.006913in}}{\pgfqpoint{1.564495in}{2.015150in}}%
\pgfpathcurveto{\pgfqpoint{1.564495in}{2.023386in}}{\pgfqpoint{1.561222in}{2.031286in}}{\pgfqpoint{1.555398in}{2.037110in}}%
\pgfpathcurveto{\pgfqpoint{1.549574in}{2.042934in}}{\pgfqpoint{1.541674in}{2.046206in}}{\pgfqpoint{1.533438in}{2.046206in}}%
\pgfpathcurveto{\pgfqpoint{1.525202in}{2.046206in}}{\pgfqpoint{1.517302in}{2.042934in}}{\pgfqpoint{1.511478in}{2.037110in}}%
\pgfpathcurveto{\pgfqpoint{1.505654in}{2.031286in}}{\pgfqpoint{1.502382in}{2.023386in}}{\pgfqpoint{1.502382in}{2.015150in}}%
\pgfpathcurveto{\pgfqpoint{1.502382in}{2.006913in}}{\pgfqpoint{1.505654in}{1.999013in}}{\pgfqpoint{1.511478in}{1.993189in}}%
\pgfpathcurveto{\pgfqpoint{1.517302in}{1.987365in}}{\pgfqpoint{1.525202in}{1.984093in}}{\pgfqpoint{1.533438in}{1.984093in}}%
\pgfpathclose%
\pgfusepath{stroke,fill}%
\end{pgfscope}%
\begin{pgfscope}%
\pgfpathrectangle{\pgfqpoint{0.100000in}{0.212622in}}{\pgfqpoint{3.696000in}{3.696000in}}%
\pgfusepath{clip}%
\pgfsetbuttcap%
\pgfsetroundjoin%
\definecolor{currentfill}{rgb}{0.121569,0.466667,0.705882}%
\pgfsetfillcolor{currentfill}%
\pgfsetfillopacity{0.353227}%
\pgfsetlinewidth{1.003750pt}%
\definecolor{currentstroke}{rgb}{0.121569,0.466667,0.705882}%
\pgfsetstrokecolor{currentstroke}%
\pgfsetstrokeopacity{0.353227}%
\pgfsetdash{}{0pt}%
\pgfpathmoveto{\pgfqpoint{1.967649in}{2.121699in}}%
\pgfpathcurveto{\pgfqpoint{1.975885in}{2.121699in}}{\pgfqpoint{1.983785in}{2.124972in}}{\pgfqpoint{1.989609in}{2.130796in}}%
\pgfpathcurveto{\pgfqpoint{1.995433in}{2.136620in}}{\pgfqpoint{1.998705in}{2.144520in}}{\pgfqpoint{1.998705in}{2.152756in}}%
\pgfpathcurveto{\pgfqpoint{1.998705in}{2.160992in}}{\pgfqpoint{1.995433in}{2.168892in}}{\pgfqpoint{1.989609in}{2.174716in}}%
\pgfpathcurveto{\pgfqpoint{1.983785in}{2.180540in}}{\pgfqpoint{1.975885in}{2.183812in}}{\pgfqpoint{1.967649in}{2.183812in}}%
\pgfpathcurveto{\pgfqpoint{1.959412in}{2.183812in}}{\pgfqpoint{1.951512in}{2.180540in}}{\pgfqpoint{1.945688in}{2.174716in}}%
\pgfpathcurveto{\pgfqpoint{1.939864in}{2.168892in}}{\pgfqpoint{1.936592in}{2.160992in}}{\pgfqpoint{1.936592in}{2.152756in}}%
\pgfpathcurveto{\pgfqpoint{1.936592in}{2.144520in}}{\pgfqpoint{1.939864in}{2.136620in}}{\pgfqpoint{1.945688in}{2.130796in}}%
\pgfpathcurveto{\pgfqpoint{1.951512in}{2.124972in}}{\pgfqpoint{1.959412in}{2.121699in}}{\pgfqpoint{1.967649in}{2.121699in}}%
\pgfpathclose%
\pgfusepath{stroke,fill}%
\end{pgfscope}%
\begin{pgfscope}%
\pgfpathrectangle{\pgfqpoint{0.100000in}{0.212622in}}{\pgfqpoint{3.696000in}{3.696000in}}%
\pgfusepath{clip}%
\pgfsetbuttcap%
\pgfsetroundjoin%
\definecolor{currentfill}{rgb}{0.121569,0.466667,0.705882}%
\pgfsetfillcolor{currentfill}%
\pgfsetfillopacity{0.353752}%
\pgfsetlinewidth{1.003750pt}%
\definecolor{currentstroke}{rgb}{0.121569,0.466667,0.705882}%
\pgfsetstrokecolor{currentstroke}%
\pgfsetstrokeopacity{0.353752}%
\pgfsetdash{}{0pt}%
\pgfpathmoveto{\pgfqpoint{1.532392in}{1.982039in}}%
\pgfpathcurveto{\pgfqpoint{1.540628in}{1.982039in}}{\pgfqpoint{1.548528in}{1.985312in}}{\pgfqpoint{1.554352in}{1.991136in}}%
\pgfpathcurveto{\pgfqpoint{1.560176in}{1.996959in}}{\pgfqpoint{1.563449in}{2.004860in}}{\pgfqpoint{1.563449in}{2.013096in}}%
\pgfpathcurveto{\pgfqpoint{1.563449in}{2.021332in}}{\pgfqpoint{1.560176in}{2.029232in}}{\pgfqpoint{1.554352in}{2.035056in}}%
\pgfpathcurveto{\pgfqpoint{1.548528in}{2.040880in}}{\pgfqpoint{1.540628in}{2.044152in}}{\pgfqpoint{1.532392in}{2.044152in}}%
\pgfpathcurveto{\pgfqpoint{1.524156in}{2.044152in}}{\pgfqpoint{1.516256in}{2.040880in}}{\pgfqpoint{1.510432in}{2.035056in}}%
\pgfpathcurveto{\pgfqpoint{1.504608in}{2.029232in}}{\pgfqpoint{1.501336in}{2.021332in}}{\pgfqpoint{1.501336in}{2.013096in}}%
\pgfpathcurveto{\pgfqpoint{1.501336in}{2.004860in}}{\pgfqpoint{1.504608in}{1.996959in}}{\pgfqpoint{1.510432in}{1.991136in}}%
\pgfpathcurveto{\pgfqpoint{1.516256in}{1.985312in}}{\pgfqpoint{1.524156in}{1.982039in}}{\pgfqpoint{1.532392in}{1.982039in}}%
\pgfpathclose%
\pgfusepath{stroke,fill}%
\end{pgfscope}%
\begin{pgfscope}%
\pgfpathrectangle{\pgfqpoint{0.100000in}{0.212622in}}{\pgfqpoint{3.696000in}{3.696000in}}%
\pgfusepath{clip}%
\pgfsetbuttcap%
\pgfsetroundjoin%
\definecolor{currentfill}{rgb}{0.121569,0.466667,0.705882}%
\pgfsetfillcolor{currentfill}%
\pgfsetfillopacity{0.354418}%
\pgfsetlinewidth{1.003750pt}%
\definecolor{currentstroke}{rgb}{0.121569,0.466667,0.705882}%
\pgfsetstrokecolor{currentstroke}%
\pgfsetstrokeopacity{0.354418}%
\pgfsetdash{}{0pt}%
\pgfpathmoveto{\pgfqpoint{1.972817in}{2.123424in}}%
\pgfpathcurveto{\pgfqpoint{1.981053in}{2.123424in}}{\pgfqpoint{1.988953in}{2.126696in}}{\pgfqpoint{1.994777in}{2.132520in}}%
\pgfpathcurveto{\pgfqpoint{2.000601in}{2.138344in}}{\pgfqpoint{2.003874in}{2.146244in}}{\pgfqpoint{2.003874in}{2.154481in}}%
\pgfpathcurveto{\pgfqpoint{2.003874in}{2.162717in}}{\pgfqpoint{2.000601in}{2.170617in}}{\pgfqpoint{1.994777in}{2.176441in}}%
\pgfpathcurveto{\pgfqpoint{1.988953in}{2.182265in}}{\pgfqpoint{1.981053in}{2.185537in}}{\pgfqpoint{1.972817in}{2.185537in}}%
\pgfpathcurveto{\pgfqpoint{1.964581in}{2.185537in}}{\pgfqpoint{1.956681in}{2.182265in}}{\pgfqpoint{1.950857in}{2.176441in}}%
\pgfpathcurveto{\pgfqpoint{1.945033in}{2.170617in}}{\pgfqpoint{1.941761in}{2.162717in}}{\pgfqpoint{1.941761in}{2.154481in}}%
\pgfpathcurveto{\pgfqpoint{1.941761in}{2.146244in}}{\pgfqpoint{1.945033in}{2.138344in}}{\pgfqpoint{1.950857in}{2.132520in}}%
\pgfpathcurveto{\pgfqpoint{1.956681in}{2.126696in}}{\pgfqpoint{1.964581in}{2.123424in}}{\pgfqpoint{1.972817in}{2.123424in}}%
\pgfpathclose%
\pgfusepath{stroke,fill}%
\end{pgfscope}%
\begin{pgfscope}%
\pgfpathrectangle{\pgfqpoint{0.100000in}{0.212622in}}{\pgfqpoint{3.696000in}{3.696000in}}%
\pgfusepath{clip}%
\pgfsetbuttcap%
\pgfsetroundjoin%
\definecolor{currentfill}{rgb}{0.121569,0.466667,0.705882}%
\pgfsetfillcolor{currentfill}%
\pgfsetfillopacity{0.354755}%
\pgfsetlinewidth{1.003750pt}%
\definecolor{currentstroke}{rgb}{0.121569,0.466667,0.705882}%
\pgfsetstrokecolor{currentstroke}%
\pgfsetstrokeopacity{0.354755}%
\pgfsetdash{}{0pt}%
\pgfpathmoveto{\pgfqpoint{1.528119in}{1.981150in}}%
\pgfpathcurveto{\pgfqpoint{1.536355in}{1.981150in}}{\pgfqpoint{1.544255in}{1.984422in}}{\pgfqpoint{1.550079in}{1.990246in}}%
\pgfpathcurveto{\pgfqpoint{1.555903in}{1.996070in}}{\pgfqpoint{1.559176in}{2.003970in}}{\pgfqpoint{1.559176in}{2.012207in}}%
\pgfpathcurveto{\pgfqpoint{1.559176in}{2.020443in}}{\pgfqpoint{1.555903in}{2.028343in}}{\pgfqpoint{1.550079in}{2.034167in}}%
\pgfpathcurveto{\pgfqpoint{1.544255in}{2.039991in}}{\pgfqpoint{1.536355in}{2.043263in}}{\pgfqpoint{1.528119in}{2.043263in}}%
\pgfpathcurveto{\pgfqpoint{1.519883in}{2.043263in}}{\pgfqpoint{1.511983in}{2.039991in}}{\pgfqpoint{1.506159in}{2.034167in}}%
\pgfpathcurveto{\pgfqpoint{1.500335in}{2.028343in}}{\pgfqpoint{1.497063in}{2.020443in}}{\pgfqpoint{1.497063in}{2.012207in}}%
\pgfpathcurveto{\pgfqpoint{1.497063in}{2.003970in}}{\pgfqpoint{1.500335in}{1.996070in}}{\pgfqpoint{1.506159in}{1.990246in}}%
\pgfpathcurveto{\pgfqpoint{1.511983in}{1.984422in}}{\pgfqpoint{1.519883in}{1.981150in}}{\pgfqpoint{1.528119in}{1.981150in}}%
\pgfpathclose%
\pgfusepath{stroke,fill}%
\end{pgfscope}%
\begin{pgfscope}%
\pgfpathrectangle{\pgfqpoint{0.100000in}{0.212622in}}{\pgfqpoint{3.696000in}{3.696000in}}%
\pgfusepath{clip}%
\pgfsetbuttcap%
\pgfsetroundjoin%
\definecolor{currentfill}{rgb}{0.121569,0.466667,0.705882}%
\pgfsetfillcolor{currentfill}%
\pgfsetfillopacity{0.355474}%
\pgfsetlinewidth{1.003750pt}%
\definecolor{currentstroke}{rgb}{0.121569,0.466667,0.705882}%
\pgfsetstrokecolor{currentstroke}%
\pgfsetstrokeopacity{0.355474}%
\pgfsetdash{}{0pt}%
\pgfpathmoveto{\pgfqpoint{1.978046in}{2.122492in}}%
\pgfpathcurveto{\pgfqpoint{1.986283in}{2.122492in}}{\pgfqpoint{1.994183in}{2.125764in}}{\pgfqpoint{2.000007in}{2.131588in}}%
\pgfpathcurveto{\pgfqpoint{2.005831in}{2.137412in}}{\pgfqpoint{2.009103in}{2.145312in}}{\pgfqpoint{2.009103in}{2.153548in}}%
\pgfpathcurveto{\pgfqpoint{2.009103in}{2.161785in}}{\pgfqpoint{2.005831in}{2.169685in}}{\pgfqpoint{2.000007in}{2.175509in}}%
\pgfpathcurveto{\pgfqpoint{1.994183in}{2.181333in}}{\pgfqpoint{1.986283in}{2.184605in}}{\pgfqpoint{1.978046in}{2.184605in}}%
\pgfpathcurveto{\pgfqpoint{1.969810in}{2.184605in}}{\pgfqpoint{1.961910in}{2.181333in}}{\pgfqpoint{1.956086in}{2.175509in}}%
\pgfpathcurveto{\pgfqpoint{1.950262in}{2.169685in}}{\pgfqpoint{1.946990in}{2.161785in}}{\pgfqpoint{1.946990in}{2.153548in}}%
\pgfpathcurveto{\pgfqpoint{1.946990in}{2.145312in}}{\pgfqpoint{1.950262in}{2.137412in}}{\pgfqpoint{1.956086in}{2.131588in}}%
\pgfpathcurveto{\pgfqpoint{1.961910in}{2.125764in}}{\pgfqpoint{1.969810in}{2.122492in}}{\pgfqpoint{1.978046in}{2.122492in}}%
\pgfpathclose%
\pgfusepath{stroke,fill}%
\end{pgfscope}%
\begin{pgfscope}%
\pgfpathrectangle{\pgfqpoint{0.100000in}{0.212622in}}{\pgfqpoint{3.696000in}{3.696000in}}%
\pgfusepath{clip}%
\pgfsetbuttcap%
\pgfsetroundjoin%
\definecolor{currentfill}{rgb}{0.121569,0.466667,0.705882}%
\pgfsetfillcolor{currentfill}%
\pgfsetfillopacity{0.356313}%
\pgfsetlinewidth{1.003750pt}%
\definecolor{currentstroke}{rgb}{0.121569,0.466667,0.705882}%
\pgfsetstrokecolor{currentstroke}%
\pgfsetstrokeopacity{0.356313}%
\pgfsetdash{}{0pt}%
\pgfpathmoveto{\pgfqpoint{1.526452in}{1.983082in}}%
\pgfpathcurveto{\pgfqpoint{1.534688in}{1.983082in}}{\pgfqpoint{1.542588in}{1.986354in}}{\pgfqpoint{1.548412in}{1.992178in}}%
\pgfpathcurveto{\pgfqpoint{1.554236in}{1.998002in}}{\pgfqpoint{1.557508in}{2.005902in}}{\pgfqpoint{1.557508in}{2.014138in}}%
\pgfpathcurveto{\pgfqpoint{1.557508in}{2.022375in}}{\pgfqpoint{1.554236in}{2.030275in}}{\pgfqpoint{1.548412in}{2.036099in}}%
\pgfpathcurveto{\pgfqpoint{1.542588in}{2.041923in}}{\pgfqpoint{1.534688in}{2.045195in}}{\pgfqpoint{1.526452in}{2.045195in}}%
\pgfpathcurveto{\pgfqpoint{1.518216in}{2.045195in}}{\pgfqpoint{1.510316in}{2.041923in}}{\pgfqpoint{1.504492in}{2.036099in}}%
\pgfpathcurveto{\pgfqpoint{1.498668in}{2.030275in}}{\pgfqpoint{1.495395in}{2.022375in}}{\pgfqpoint{1.495395in}{2.014138in}}%
\pgfpathcurveto{\pgfqpoint{1.495395in}{2.005902in}}{\pgfqpoint{1.498668in}{1.998002in}}{\pgfqpoint{1.504492in}{1.992178in}}%
\pgfpathcurveto{\pgfqpoint{1.510316in}{1.986354in}}{\pgfqpoint{1.518216in}{1.983082in}}{\pgfqpoint{1.526452in}{1.983082in}}%
\pgfpathclose%
\pgfusepath{stroke,fill}%
\end{pgfscope}%
\begin{pgfscope}%
\pgfpathrectangle{\pgfqpoint{0.100000in}{0.212622in}}{\pgfqpoint{3.696000in}{3.696000in}}%
\pgfusepath{clip}%
\pgfsetbuttcap%
\pgfsetroundjoin%
\definecolor{currentfill}{rgb}{0.121569,0.466667,0.705882}%
\pgfsetfillcolor{currentfill}%
\pgfsetfillopacity{0.356915}%
\pgfsetlinewidth{1.003750pt}%
\definecolor{currentstroke}{rgb}{0.121569,0.466667,0.705882}%
\pgfsetstrokecolor{currentstroke}%
\pgfsetstrokeopacity{0.356915}%
\pgfsetdash{}{0pt}%
\pgfpathmoveto{\pgfqpoint{1.522417in}{1.982598in}}%
\pgfpathcurveto{\pgfqpoint{1.530653in}{1.982598in}}{\pgfqpoint{1.538553in}{1.985870in}}{\pgfqpoint{1.544377in}{1.991694in}}%
\pgfpathcurveto{\pgfqpoint{1.550201in}{1.997518in}}{\pgfqpoint{1.553473in}{2.005418in}}{\pgfqpoint{1.553473in}{2.013654in}}%
\pgfpathcurveto{\pgfqpoint{1.553473in}{2.021891in}}{\pgfqpoint{1.550201in}{2.029791in}}{\pgfqpoint{1.544377in}{2.035615in}}%
\pgfpathcurveto{\pgfqpoint{1.538553in}{2.041439in}}{\pgfqpoint{1.530653in}{2.044711in}}{\pgfqpoint{1.522417in}{2.044711in}}%
\pgfpathcurveto{\pgfqpoint{1.514180in}{2.044711in}}{\pgfqpoint{1.506280in}{2.041439in}}{\pgfqpoint{1.500456in}{2.035615in}}%
\pgfpathcurveto{\pgfqpoint{1.494633in}{2.029791in}}{\pgfqpoint{1.491360in}{2.021891in}}{\pgfqpoint{1.491360in}{2.013654in}}%
\pgfpathcurveto{\pgfqpoint{1.491360in}{2.005418in}}{\pgfqpoint{1.494633in}{1.997518in}}{\pgfqpoint{1.500456in}{1.991694in}}%
\pgfpathcurveto{\pgfqpoint{1.506280in}{1.985870in}}{\pgfqpoint{1.514180in}{1.982598in}}{\pgfqpoint{1.522417in}{1.982598in}}%
\pgfpathclose%
\pgfusepath{stroke,fill}%
\end{pgfscope}%
\begin{pgfscope}%
\pgfpathrectangle{\pgfqpoint{0.100000in}{0.212622in}}{\pgfqpoint{3.696000in}{3.696000in}}%
\pgfusepath{clip}%
\pgfsetbuttcap%
\pgfsetroundjoin%
\definecolor{currentfill}{rgb}{0.121569,0.466667,0.705882}%
\pgfsetfillcolor{currentfill}%
\pgfsetfillopacity{0.356939}%
\pgfsetlinewidth{1.003750pt}%
\definecolor{currentstroke}{rgb}{0.121569,0.466667,0.705882}%
\pgfsetstrokecolor{currentstroke}%
\pgfsetstrokeopacity{0.356939}%
\pgfsetdash{}{0pt}%
\pgfpathmoveto{\pgfqpoint{1.519211in}{1.977106in}}%
\pgfpathcurveto{\pgfqpoint{1.527447in}{1.977106in}}{\pgfqpoint{1.535347in}{1.980378in}}{\pgfqpoint{1.541171in}{1.986202in}}%
\pgfpathcurveto{\pgfqpoint{1.546995in}{1.992026in}}{\pgfqpoint{1.550267in}{1.999926in}}{\pgfqpoint{1.550267in}{2.008163in}}%
\pgfpathcurveto{\pgfqpoint{1.550267in}{2.016399in}}{\pgfqpoint{1.546995in}{2.024299in}}{\pgfqpoint{1.541171in}{2.030123in}}%
\pgfpathcurveto{\pgfqpoint{1.535347in}{2.035947in}}{\pgfqpoint{1.527447in}{2.039219in}}{\pgfqpoint{1.519211in}{2.039219in}}%
\pgfpathcurveto{\pgfqpoint{1.510975in}{2.039219in}}{\pgfqpoint{1.503075in}{2.035947in}}{\pgfqpoint{1.497251in}{2.030123in}}%
\pgfpathcurveto{\pgfqpoint{1.491427in}{2.024299in}}{\pgfqpoint{1.488154in}{2.016399in}}{\pgfqpoint{1.488154in}{2.008163in}}%
\pgfpathcurveto{\pgfqpoint{1.488154in}{1.999926in}}{\pgfqpoint{1.491427in}{1.992026in}}{\pgfqpoint{1.497251in}{1.986202in}}%
\pgfpathcurveto{\pgfqpoint{1.503075in}{1.980378in}}{\pgfqpoint{1.510975in}{1.977106in}}{\pgfqpoint{1.519211in}{1.977106in}}%
\pgfpathclose%
\pgfusepath{stroke,fill}%
\end{pgfscope}%
\begin{pgfscope}%
\pgfpathrectangle{\pgfqpoint{0.100000in}{0.212622in}}{\pgfqpoint{3.696000in}{3.696000in}}%
\pgfusepath{clip}%
\pgfsetbuttcap%
\pgfsetroundjoin%
\definecolor{currentfill}{rgb}{0.121569,0.466667,0.705882}%
\pgfsetfillcolor{currentfill}%
\pgfsetfillopacity{0.357338}%
\pgfsetlinewidth{1.003750pt}%
\definecolor{currentstroke}{rgb}{0.121569,0.466667,0.705882}%
\pgfsetstrokecolor{currentstroke}%
\pgfsetstrokeopacity{0.357338}%
\pgfsetdash{}{0pt}%
\pgfpathmoveto{\pgfqpoint{1.984546in}{2.126411in}}%
\pgfpathcurveto{\pgfqpoint{1.992783in}{2.126411in}}{\pgfqpoint{2.000683in}{2.129684in}}{\pgfqpoint{2.006507in}{2.135508in}}%
\pgfpathcurveto{\pgfqpoint{2.012331in}{2.141332in}}{\pgfqpoint{2.015603in}{2.149232in}}{\pgfqpoint{2.015603in}{2.157468in}}%
\pgfpathcurveto{\pgfqpoint{2.015603in}{2.165704in}}{\pgfqpoint{2.012331in}{2.173604in}}{\pgfqpoint{2.006507in}{2.179428in}}%
\pgfpathcurveto{\pgfqpoint{2.000683in}{2.185252in}}{\pgfqpoint{1.992783in}{2.188524in}}{\pgfqpoint{1.984546in}{2.188524in}}%
\pgfpathcurveto{\pgfqpoint{1.976310in}{2.188524in}}{\pgfqpoint{1.968410in}{2.185252in}}{\pgfqpoint{1.962586in}{2.179428in}}%
\pgfpathcurveto{\pgfqpoint{1.956762in}{2.173604in}}{\pgfqpoint{1.953490in}{2.165704in}}{\pgfqpoint{1.953490in}{2.157468in}}%
\pgfpathcurveto{\pgfqpoint{1.953490in}{2.149232in}}{\pgfqpoint{1.956762in}{2.141332in}}{\pgfqpoint{1.962586in}{2.135508in}}%
\pgfpathcurveto{\pgfqpoint{1.968410in}{2.129684in}}{\pgfqpoint{1.976310in}{2.126411in}}{\pgfqpoint{1.984546in}{2.126411in}}%
\pgfpathclose%
\pgfusepath{stroke,fill}%
\end{pgfscope}%
\begin{pgfscope}%
\pgfpathrectangle{\pgfqpoint{0.100000in}{0.212622in}}{\pgfqpoint{3.696000in}{3.696000in}}%
\pgfusepath{clip}%
\pgfsetbuttcap%
\pgfsetroundjoin%
\definecolor{currentfill}{rgb}{0.121569,0.466667,0.705882}%
\pgfsetfillcolor{currentfill}%
\pgfsetfillopacity{0.357925}%
\pgfsetlinewidth{1.003750pt}%
\definecolor{currentstroke}{rgb}{0.121569,0.466667,0.705882}%
\pgfsetstrokecolor{currentstroke}%
\pgfsetstrokeopacity{0.357925}%
\pgfsetdash{}{0pt}%
\pgfpathmoveto{\pgfqpoint{1.518478in}{1.977148in}}%
\pgfpathcurveto{\pgfqpoint{1.526714in}{1.977148in}}{\pgfqpoint{1.534615in}{1.980420in}}{\pgfqpoint{1.540438in}{1.986244in}}%
\pgfpathcurveto{\pgfqpoint{1.546262in}{1.992068in}}{\pgfqpoint{1.549535in}{1.999968in}}{\pgfqpoint{1.549535in}{2.008204in}}%
\pgfpathcurveto{\pgfqpoint{1.549535in}{2.016441in}}{\pgfqpoint{1.546262in}{2.024341in}}{\pgfqpoint{1.540438in}{2.030165in}}%
\pgfpathcurveto{\pgfqpoint{1.534615in}{2.035988in}}{\pgfqpoint{1.526714in}{2.039261in}}{\pgfqpoint{1.518478in}{2.039261in}}%
\pgfpathcurveto{\pgfqpoint{1.510242in}{2.039261in}}{\pgfqpoint{1.502342in}{2.035988in}}{\pgfqpoint{1.496518in}{2.030165in}}%
\pgfpathcurveto{\pgfqpoint{1.490694in}{2.024341in}}{\pgfqpoint{1.487422in}{2.016441in}}{\pgfqpoint{1.487422in}{2.008204in}}%
\pgfpathcurveto{\pgfqpoint{1.487422in}{1.999968in}}{\pgfqpoint{1.490694in}{1.992068in}}{\pgfqpoint{1.496518in}{1.986244in}}%
\pgfpathcurveto{\pgfqpoint{1.502342in}{1.980420in}}{\pgfqpoint{1.510242in}{1.977148in}}{\pgfqpoint{1.518478in}{1.977148in}}%
\pgfpathclose%
\pgfusepath{stroke,fill}%
\end{pgfscope}%
\begin{pgfscope}%
\pgfpathrectangle{\pgfqpoint{0.100000in}{0.212622in}}{\pgfqpoint{3.696000in}{3.696000in}}%
\pgfusepath{clip}%
\pgfsetbuttcap%
\pgfsetroundjoin%
\definecolor{currentfill}{rgb}{0.121569,0.466667,0.705882}%
\pgfsetfillcolor{currentfill}%
\pgfsetfillopacity{0.358354}%
\pgfsetlinewidth{1.003750pt}%
\definecolor{currentstroke}{rgb}{0.121569,0.466667,0.705882}%
\pgfsetstrokecolor{currentstroke}%
\pgfsetstrokeopacity{0.358354}%
\pgfsetdash{}{0pt}%
\pgfpathmoveto{\pgfqpoint{1.990568in}{2.120077in}}%
\pgfpathcurveto{\pgfqpoint{1.998804in}{2.120077in}}{\pgfqpoint{2.006705in}{2.123349in}}{\pgfqpoint{2.012528in}{2.129173in}}%
\pgfpathcurveto{\pgfqpoint{2.018352in}{2.134997in}}{\pgfqpoint{2.021625in}{2.142897in}}{\pgfqpoint{2.021625in}{2.151133in}}%
\pgfpathcurveto{\pgfqpoint{2.021625in}{2.159369in}}{\pgfqpoint{2.018352in}{2.167269in}}{\pgfqpoint{2.012528in}{2.173093in}}%
\pgfpathcurveto{\pgfqpoint{2.006705in}{2.178917in}}{\pgfqpoint{1.998804in}{2.182190in}}{\pgfqpoint{1.990568in}{2.182190in}}%
\pgfpathcurveto{\pgfqpoint{1.982332in}{2.182190in}}{\pgfqpoint{1.974432in}{2.178917in}}{\pgfqpoint{1.968608in}{2.173093in}}%
\pgfpathcurveto{\pgfqpoint{1.962784in}{2.167269in}}{\pgfqpoint{1.959512in}{2.159369in}}{\pgfqpoint{1.959512in}{2.151133in}}%
\pgfpathcurveto{\pgfqpoint{1.959512in}{2.142897in}}{\pgfqpoint{1.962784in}{2.134997in}}{\pgfqpoint{1.968608in}{2.129173in}}%
\pgfpathcurveto{\pgfqpoint{1.974432in}{2.123349in}}{\pgfqpoint{1.982332in}{2.120077in}}{\pgfqpoint{1.990568in}{2.120077in}}%
\pgfpathclose%
\pgfusepath{stroke,fill}%
\end{pgfscope}%
\begin{pgfscope}%
\pgfpathrectangle{\pgfqpoint{0.100000in}{0.212622in}}{\pgfqpoint{3.696000in}{3.696000in}}%
\pgfusepath{clip}%
\pgfsetbuttcap%
\pgfsetroundjoin%
\definecolor{currentfill}{rgb}{0.121569,0.466667,0.705882}%
\pgfsetfillcolor{currentfill}%
\pgfsetfillopacity{0.358409}%
\pgfsetlinewidth{1.003750pt}%
\definecolor{currentstroke}{rgb}{0.121569,0.466667,0.705882}%
\pgfsetstrokecolor{currentstroke}%
\pgfsetstrokeopacity{0.358409}%
\pgfsetdash{}{0pt}%
\pgfpathmoveto{\pgfqpoint{1.516327in}{1.976303in}}%
\pgfpathcurveto{\pgfqpoint{1.524564in}{1.976303in}}{\pgfqpoint{1.532464in}{1.979576in}}{\pgfqpoint{1.538288in}{1.985399in}}%
\pgfpathcurveto{\pgfqpoint{1.544111in}{1.991223in}}{\pgfqpoint{1.547384in}{1.999123in}}{\pgfqpoint{1.547384in}{2.007360in}}%
\pgfpathcurveto{\pgfqpoint{1.547384in}{2.015596in}}{\pgfqpoint{1.544111in}{2.023496in}}{\pgfqpoint{1.538288in}{2.029320in}}%
\pgfpathcurveto{\pgfqpoint{1.532464in}{2.035144in}}{\pgfqpoint{1.524564in}{2.038416in}}{\pgfqpoint{1.516327in}{2.038416in}}%
\pgfpathcurveto{\pgfqpoint{1.508091in}{2.038416in}}{\pgfqpoint{1.500191in}{2.035144in}}{\pgfqpoint{1.494367in}{2.029320in}}%
\pgfpathcurveto{\pgfqpoint{1.488543in}{2.023496in}}{\pgfqpoint{1.485271in}{2.015596in}}{\pgfqpoint{1.485271in}{2.007360in}}%
\pgfpathcurveto{\pgfqpoint{1.485271in}{1.999123in}}{\pgfqpoint{1.488543in}{1.991223in}}{\pgfqpoint{1.494367in}{1.985399in}}%
\pgfpathcurveto{\pgfqpoint{1.500191in}{1.979576in}}{\pgfqpoint{1.508091in}{1.976303in}}{\pgfqpoint{1.516327in}{1.976303in}}%
\pgfpathclose%
\pgfusepath{stroke,fill}%
\end{pgfscope}%
\begin{pgfscope}%
\pgfpathrectangle{\pgfqpoint{0.100000in}{0.212622in}}{\pgfqpoint{3.696000in}{3.696000in}}%
\pgfusepath{clip}%
\pgfsetbuttcap%
\pgfsetroundjoin%
\definecolor{currentfill}{rgb}{0.121569,0.466667,0.705882}%
\pgfsetfillcolor{currentfill}%
\pgfsetfillopacity{0.358730}%
\pgfsetlinewidth{1.003750pt}%
\definecolor{currentstroke}{rgb}{0.121569,0.466667,0.705882}%
\pgfsetstrokecolor{currentstroke}%
\pgfsetstrokeopacity{0.358730}%
\pgfsetdash{}{0pt}%
\pgfpathmoveto{\pgfqpoint{1.513642in}{1.968894in}}%
\pgfpathcurveto{\pgfqpoint{1.521878in}{1.968894in}}{\pgfqpoint{1.529778in}{1.972166in}}{\pgfqpoint{1.535602in}{1.977990in}}%
\pgfpathcurveto{\pgfqpoint{1.541426in}{1.983814in}}{\pgfqpoint{1.544698in}{1.991714in}}{\pgfqpoint{1.544698in}{1.999950in}}%
\pgfpathcurveto{\pgfqpoint{1.544698in}{2.008187in}}{\pgfqpoint{1.541426in}{2.016087in}}{\pgfqpoint{1.535602in}{2.021911in}}%
\pgfpathcurveto{\pgfqpoint{1.529778in}{2.027735in}}{\pgfqpoint{1.521878in}{2.031007in}}{\pgfqpoint{1.513642in}{2.031007in}}%
\pgfpathcurveto{\pgfqpoint{1.505405in}{2.031007in}}{\pgfqpoint{1.497505in}{2.027735in}}{\pgfqpoint{1.491681in}{2.021911in}}%
\pgfpathcurveto{\pgfqpoint{1.485857in}{2.016087in}}{\pgfqpoint{1.482585in}{2.008187in}}{\pgfqpoint{1.482585in}{1.999950in}}%
\pgfpathcurveto{\pgfqpoint{1.482585in}{1.991714in}}{\pgfqpoint{1.485857in}{1.983814in}}{\pgfqpoint{1.491681in}{1.977990in}}%
\pgfpathcurveto{\pgfqpoint{1.497505in}{1.972166in}}{\pgfqpoint{1.505405in}{1.968894in}}{\pgfqpoint{1.513642in}{1.968894in}}%
\pgfpathclose%
\pgfusepath{stroke,fill}%
\end{pgfscope}%
\begin{pgfscope}%
\pgfpathrectangle{\pgfqpoint{0.100000in}{0.212622in}}{\pgfqpoint{3.696000in}{3.696000in}}%
\pgfusepath{clip}%
\pgfsetbuttcap%
\pgfsetroundjoin%
\definecolor{currentfill}{rgb}{0.121569,0.466667,0.705882}%
\pgfsetfillcolor{currentfill}%
\pgfsetfillopacity{0.359809}%
\pgfsetlinewidth{1.003750pt}%
\definecolor{currentstroke}{rgb}{0.121569,0.466667,0.705882}%
\pgfsetstrokecolor{currentstroke}%
\pgfsetstrokeopacity{0.359809}%
\pgfsetdash{}{0pt}%
\pgfpathmoveto{\pgfqpoint{1.512069in}{1.966831in}}%
\pgfpathcurveto{\pgfqpoint{1.520305in}{1.966831in}}{\pgfqpoint{1.528205in}{1.970103in}}{\pgfqpoint{1.534029in}{1.975927in}}%
\pgfpathcurveto{\pgfqpoint{1.539853in}{1.981751in}}{\pgfqpoint{1.543125in}{1.989651in}}{\pgfqpoint{1.543125in}{1.997887in}}%
\pgfpathcurveto{\pgfqpoint{1.543125in}{2.006124in}}{\pgfqpoint{1.539853in}{2.014024in}}{\pgfqpoint{1.534029in}{2.019848in}}%
\pgfpathcurveto{\pgfqpoint{1.528205in}{2.025672in}}{\pgfqpoint{1.520305in}{2.028944in}}{\pgfqpoint{1.512069in}{2.028944in}}%
\pgfpathcurveto{\pgfqpoint{1.503832in}{2.028944in}}{\pgfqpoint{1.495932in}{2.025672in}}{\pgfqpoint{1.490108in}{2.019848in}}%
\pgfpathcurveto{\pgfqpoint{1.484284in}{2.014024in}}{\pgfqpoint{1.481012in}{2.006124in}}{\pgfqpoint{1.481012in}{1.997887in}}%
\pgfpathcurveto{\pgfqpoint{1.481012in}{1.989651in}}{\pgfqpoint{1.484284in}{1.981751in}}{\pgfqpoint{1.490108in}{1.975927in}}%
\pgfpathcurveto{\pgfqpoint{1.495932in}{1.970103in}}{\pgfqpoint{1.503832in}{1.966831in}}{\pgfqpoint{1.512069in}{1.966831in}}%
\pgfpathclose%
\pgfusepath{stroke,fill}%
\end{pgfscope}%
\begin{pgfscope}%
\pgfpathrectangle{\pgfqpoint{0.100000in}{0.212622in}}{\pgfqpoint{3.696000in}{3.696000in}}%
\pgfusepath{clip}%
\pgfsetbuttcap%
\pgfsetroundjoin%
\definecolor{currentfill}{rgb}{0.121569,0.466667,0.705882}%
\pgfsetfillcolor{currentfill}%
\pgfsetfillopacity{0.360408}%
\pgfsetlinewidth{1.003750pt}%
\definecolor{currentstroke}{rgb}{0.121569,0.466667,0.705882}%
\pgfsetstrokecolor{currentstroke}%
\pgfsetstrokeopacity{0.360408}%
\pgfsetdash{}{0pt}%
\pgfpathmoveto{\pgfqpoint{1.510698in}{1.964091in}}%
\pgfpathcurveto{\pgfqpoint{1.518934in}{1.964091in}}{\pgfqpoint{1.526834in}{1.967364in}}{\pgfqpoint{1.532658in}{1.973188in}}%
\pgfpathcurveto{\pgfqpoint{1.538482in}{1.979012in}}{\pgfqpoint{1.541755in}{1.986912in}}{\pgfqpoint{1.541755in}{1.995148in}}%
\pgfpathcurveto{\pgfqpoint{1.541755in}{2.003384in}}{\pgfqpoint{1.538482in}{2.011284in}}{\pgfqpoint{1.532658in}{2.017108in}}%
\pgfpathcurveto{\pgfqpoint{1.526834in}{2.022932in}}{\pgfqpoint{1.518934in}{2.026204in}}{\pgfqpoint{1.510698in}{2.026204in}}%
\pgfpathcurveto{\pgfqpoint{1.502462in}{2.026204in}}{\pgfqpoint{1.494562in}{2.022932in}}{\pgfqpoint{1.488738in}{2.017108in}}%
\pgfpathcurveto{\pgfqpoint{1.482914in}{2.011284in}}{\pgfqpoint{1.479642in}{2.003384in}}{\pgfqpoint{1.479642in}{1.995148in}}%
\pgfpathcurveto{\pgfqpoint{1.479642in}{1.986912in}}{\pgfqpoint{1.482914in}{1.979012in}}{\pgfqpoint{1.488738in}{1.973188in}}%
\pgfpathcurveto{\pgfqpoint{1.494562in}{1.967364in}}{\pgfqpoint{1.502462in}{1.964091in}}{\pgfqpoint{1.510698in}{1.964091in}}%
\pgfpathclose%
\pgfusepath{stroke,fill}%
\end{pgfscope}%
\begin{pgfscope}%
\pgfpathrectangle{\pgfqpoint{0.100000in}{0.212622in}}{\pgfqpoint{3.696000in}{3.696000in}}%
\pgfusepath{clip}%
\pgfsetbuttcap%
\pgfsetroundjoin%
\definecolor{currentfill}{rgb}{0.121569,0.466667,0.705882}%
\pgfsetfillcolor{currentfill}%
\pgfsetfillopacity{0.360572}%
\pgfsetlinewidth{1.003750pt}%
\definecolor{currentstroke}{rgb}{0.121569,0.466667,0.705882}%
\pgfsetstrokecolor{currentstroke}%
\pgfsetstrokeopacity{0.360572}%
\pgfsetdash{}{0pt}%
\pgfpathmoveto{\pgfqpoint{1.510055in}{1.963045in}}%
\pgfpathcurveto{\pgfqpoint{1.518292in}{1.963045in}}{\pgfqpoint{1.526192in}{1.966318in}}{\pgfqpoint{1.532016in}{1.972142in}}%
\pgfpathcurveto{\pgfqpoint{1.537840in}{1.977966in}}{\pgfqpoint{1.541112in}{1.985866in}}{\pgfqpoint{1.541112in}{1.994102in}}%
\pgfpathcurveto{\pgfqpoint{1.541112in}{2.002338in}}{\pgfqpoint{1.537840in}{2.010238in}}{\pgfqpoint{1.532016in}{2.016062in}}%
\pgfpathcurveto{\pgfqpoint{1.526192in}{2.021886in}}{\pgfqpoint{1.518292in}{2.025158in}}{\pgfqpoint{1.510055in}{2.025158in}}%
\pgfpathcurveto{\pgfqpoint{1.501819in}{2.025158in}}{\pgfqpoint{1.493919in}{2.021886in}}{\pgfqpoint{1.488095in}{2.016062in}}%
\pgfpathcurveto{\pgfqpoint{1.482271in}{2.010238in}}{\pgfqpoint{1.478999in}{2.002338in}}{\pgfqpoint{1.478999in}{1.994102in}}%
\pgfpathcurveto{\pgfqpoint{1.478999in}{1.985866in}}{\pgfqpoint{1.482271in}{1.977966in}}{\pgfqpoint{1.488095in}{1.972142in}}%
\pgfpathcurveto{\pgfqpoint{1.493919in}{1.966318in}}{\pgfqpoint{1.501819in}{1.963045in}}{\pgfqpoint{1.510055in}{1.963045in}}%
\pgfpathclose%
\pgfusepath{stroke,fill}%
\end{pgfscope}%
\begin{pgfscope}%
\pgfpathrectangle{\pgfqpoint{0.100000in}{0.212622in}}{\pgfqpoint{3.696000in}{3.696000in}}%
\pgfusepath{clip}%
\pgfsetbuttcap%
\pgfsetroundjoin%
\definecolor{currentfill}{rgb}{0.121569,0.466667,0.705882}%
\pgfsetfillcolor{currentfill}%
\pgfsetfillopacity{0.360795}%
\pgfsetlinewidth{1.003750pt}%
\definecolor{currentstroke}{rgb}{0.121569,0.466667,0.705882}%
\pgfsetstrokecolor{currentstroke}%
\pgfsetstrokeopacity{0.360795}%
\pgfsetdash{}{0pt}%
\pgfpathmoveto{\pgfqpoint{1.997941in}{2.125002in}}%
\pgfpathcurveto{\pgfqpoint{2.006177in}{2.125002in}}{\pgfqpoint{2.014077in}{2.128274in}}{\pgfqpoint{2.019901in}{2.134098in}}%
\pgfpathcurveto{\pgfqpoint{2.025725in}{2.139922in}}{\pgfqpoint{2.028997in}{2.147822in}}{\pgfqpoint{2.028997in}{2.156058in}}%
\pgfpathcurveto{\pgfqpoint{2.028997in}{2.164294in}}{\pgfqpoint{2.025725in}{2.172194in}}{\pgfqpoint{2.019901in}{2.178018in}}%
\pgfpathcurveto{\pgfqpoint{2.014077in}{2.183842in}}{\pgfqpoint{2.006177in}{2.187115in}}{\pgfqpoint{1.997941in}{2.187115in}}%
\pgfpathcurveto{\pgfqpoint{1.989704in}{2.187115in}}{\pgfqpoint{1.981804in}{2.183842in}}{\pgfqpoint{1.975980in}{2.178018in}}%
\pgfpathcurveto{\pgfqpoint{1.970157in}{2.172194in}}{\pgfqpoint{1.966884in}{2.164294in}}{\pgfqpoint{1.966884in}{2.156058in}}%
\pgfpathcurveto{\pgfqpoint{1.966884in}{2.147822in}}{\pgfqpoint{1.970157in}{2.139922in}}{\pgfqpoint{1.975980in}{2.134098in}}%
\pgfpathcurveto{\pgfqpoint{1.981804in}{2.128274in}}{\pgfqpoint{1.989704in}{2.125002in}}{\pgfqpoint{1.997941in}{2.125002in}}%
\pgfpathclose%
\pgfusepath{stroke,fill}%
\end{pgfscope}%
\begin{pgfscope}%
\pgfpathrectangle{\pgfqpoint{0.100000in}{0.212622in}}{\pgfqpoint{3.696000in}{3.696000in}}%
\pgfusepath{clip}%
\pgfsetbuttcap%
\pgfsetroundjoin%
\definecolor{currentfill}{rgb}{0.121569,0.466667,0.705882}%
\pgfsetfillcolor{currentfill}%
\pgfsetfillopacity{0.361094}%
\pgfsetlinewidth{1.003750pt}%
\definecolor{currentstroke}{rgb}{0.121569,0.466667,0.705882}%
\pgfsetstrokecolor{currentstroke}%
\pgfsetstrokeopacity{0.361094}%
\pgfsetdash{}{0pt}%
\pgfpathmoveto{\pgfqpoint{1.509228in}{1.962477in}}%
\pgfpathcurveto{\pgfqpoint{1.517464in}{1.962477in}}{\pgfqpoint{1.525364in}{1.965750in}}{\pgfqpoint{1.531188in}{1.971573in}}%
\pgfpathcurveto{\pgfqpoint{1.537012in}{1.977397in}}{\pgfqpoint{1.540284in}{1.985297in}}{\pgfqpoint{1.540284in}{1.993534in}}%
\pgfpathcurveto{\pgfqpoint{1.540284in}{2.001770in}}{\pgfqpoint{1.537012in}{2.009670in}}{\pgfqpoint{1.531188in}{2.015494in}}%
\pgfpathcurveto{\pgfqpoint{1.525364in}{2.021318in}}{\pgfqpoint{1.517464in}{2.024590in}}{\pgfqpoint{1.509228in}{2.024590in}}%
\pgfpathcurveto{\pgfqpoint{1.500991in}{2.024590in}}{\pgfqpoint{1.493091in}{2.021318in}}{\pgfqpoint{1.487267in}{2.015494in}}%
\pgfpathcurveto{\pgfqpoint{1.481443in}{2.009670in}}{\pgfqpoint{1.478171in}{2.001770in}}{\pgfqpoint{1.478171in}{1.993534in}}%
\pgfpathcurveto{\pgfqpoint{1.478171in}{1.985297in}}{\pgfqpoint{1.481443in}{1.977397in}}{\pgfqpoint{1.487267in}{1.971573in}}%
\pgfpathcurveto{\pgfqpoint{1.493091in}{1.965750in}}{\pgfqpoint{1.500991in}{1.962477in}}{\pgfqpoint{1.509228in}{1.962477in}}%
\pgfpathclose%
\pgfusepath{stroke,fill}%
\end{pgfscope}%
\begin{pgfscope}%
\pgfpathrectangle{\pgfqpoint{0.100000in}{0.212622in}}{\pgfqpoint{3.696000in}{3.696000in}}%
\pgfusepath{clip}%
\pgfsetbuttcap%
\pgfsetroundjoin%
\definecolor{currentfill}{rgb}{0.121569,0.466667,0.705882}%
\pgfsetfillcolor{currentfill}%
\pgfsetfillopacity{0.361289}%
\pgfsetlinewidth{1.003750pt}%
\definecolor{currentstroke}{rgb}{0.121569,0.466667,0.705882}%
\pgfsetstrokecolor{currentstroke}%
\pgfsetstrokeopacity{0.361289}%
\pgfsetdash{}{0pt}%
\pgfpathmoveto{\pgfqpoint{1.508322in}{1.960649in}}%
\pgfpathcurveto{\pgfqpoint{1.516558in}{1.960649in}}{\pgfqpoint{1.524458in}{1.963921in}}{\pgfqpoint{1.530282in}{1.969745in}}%
\pgfpathcurveto{\pgfqpoint{1.536106in}{1.975569in}}{\pgfqpoint{1.539378in}{1.983469in}}{\pgfqpoint{1.539378in}{1.991705in}}%
\pgfpathcurveto{\pgfqpoint{1.539378in}{1.999942in}}{\pgfqpoint{1.536106in}{2.007842in}}{\pgfqpoint{1.530282in}{2.013666in}}%
\pgfpathcurveto{\pgfqpoint{1.524458in}{2.019490in}}{\pgfqpoint{1.516558in}{2.022762in}}{\pgfqpoint{1.508322in}{2.022762in}}%
\pgfpathcurveto{\pgfqpoint{1.500086in}{2.022762in}}{\pgfqpoint{1.492186in}{2.019490in}}{\pgfqpoint{1.486362in}{2.013666in}}%
\pgfpathcurveto{\pgfqpoint{1.480538in}{2.007842in}}{\pgfqpoint{1.477265in}{1.999942in}}{\pgfqpoint{1.477265in}{1.991705in}}%
\pgfpathcurveto{\pgfqpoint{1.477265in}{1.983469in}}{\pgfqpoint{1.480538in}{1.975569in}}{\pgfqpoint{1.486362in}{1.969745in}}%
\pgfpathcurveto{\pgfqpoint{1.492186in}{1.963921in}}{\pgfqpoint{1.500086in}{1.960649in}}{\pgfqpoint{1.508322in}{1.960649in}}%
\pgfpathclose%
\pgfusepath{stroke,fill}%
\end{pgfscope}%
\begin{pgfscope}%
\pgfpathrectangle{\pgfqpoint{0.100000in}{0.212622in}}{\pgfqpoint{3.696000in}{3.696000in}}%
\pgfusepath{clip}%
\pgfsetbuttcap%
\pgfsetroundjoin%
\definecolor{currentfill}{rgb}{0.121569,0.466667,0.705882}%
\pgfsetfillcolor{currentfill}%
\pgfsetfillopacity{0.361524}%
\pgfsetlinewidth{1.003750pt}%
\definecolor{currentstroke}{rgb}{0.121569,0.466667,0.705882}%
\pgfsetstrokecolor{currentstroke}%
\pgfsetstrokeopacity{0.361524}%
\pgfsetdash{}{0pt}%
\pgfpathmoveto{\pgfqpoint{1.507311in}{1.960115in}}%
\pgfpathcurveto{\pgfqpoint{1.515547in}{1.960115in}}{\pgfqpoint{1.523448in}{1.963387in}}{\pgfqpoint{1.529271in}{1.969211in}}%
\pgfpathcurveto{\pgfqpoint{1.535095in}{1.975035in}}{\pgfqpoint{1.538368in}{1.982935in}}{\pgfqpoint{1.538368in}{1.991172in}}%
\pgfpathcurveto{\pgfqpoint{1.538368in}{1.999408in}}{\pgfqpoint{1.535095in}{2.007308in}}{\pgfqpoint{1.529271in}{2.013132in}}%
\pgfpathcurveto{\pgfqpoint{1.523448in}{2.018956in}}{\pgfqpoint{1.515547in}{2.022228in}}{\pgfqpoint{1.507311in}{2.022228in}}%
\pgfpathcurveto{\pgfqpoint{1.499075in}{2.022228in}}{\pgfqpoint{1.491175in}{2.018956in}}{\pgfqpoint{1.485351in}{2.013132in}}%
\pgfpathcurveto{\pgfqpoint{1.479527in}{2.007308in}}{\pgfqpoint{1.476255in}{1.999408in}}{\pgfqpoint{1.476255in}{1.991172in}}%
\pgfpathcurveto{\pgfqpoint{1.476255in}{1.982935in}}{\pgfqpoint{1.479527in}{1.975035in}}{\pgfqpoint{1.485351in}{1.969211in}}%
\pgfpathcurveto{\pgfqpoint{1.491175in}{1.963387in}}{\pgfqpoint{1.499075in}{1.960115in}}{\pgfqpoint{1.507311in}{1.960115in}}%
\pgfpathclose%
\pgfusepath{stroke,fill}%
\end{pgfscope}%
\begin{pgfscope}%
\pgfpathrectangle{\pgfqpoint{0.100000in}{0.212622in}}{\pgfqpoint{3.696000in}{3.696000in}}%
\pgfusepath{clip}%
\pgfsetbuttcap%
\pgfsetroundjoin%
\definecolor{currentfill}{rgb}{0.121569,0.466667,0.705882}%
\pgfsetfillcolor{currentfill}%
\pgfsetfillopacity{0.362251}%
\pgfsetlinewidth{1.003750pt}%
\definecolor{currentstroke}{rgb}{0.121569,0.466667,0.705882}%
\pgfsetstrokecolor{currentstroke}%
\pgfsetstrokeopacity{0.362251}%
\pgfsetdash{}{0pt}%
\pgfpathmoveto{\pgfqpoint{1.506779in}{1.960134in}}%
\pgfpathcurveto{\pgfqpoint{1.515015in}{1.960134in}}{\pgfqpoint{1.522916in}{1.963406in}}{\pgfqpoint{1.528739in}{1.969230in}}%
\pgfpathcurveto{\pgfqpoint{1.534563in}{1.975054in}}{\pgfqpoint{1.537836in}{1.982954in}}{\pgfqpoint{1.537836in}{1.991190in}}%
\pgfpathcurveto{\pgfqpoint{1.537836in}{1.999426in}}{\pgfqpoint{1.534563in}{2.007326in}}{\pgfqpoint{1.528739in}{2.013150in}}%
\pgfpathcurveto{\pgfqpoint{1.522916in}{2.018974in}}{\pgfqpoint{1.515015in}{2.022247in}}{\pgfqpoint{1.506779in}{2.022247in}}%
\pgfpathcurveto{\pgfqpoint{1.498543in}{2.022247in}}{\pgfqpoint{1.490643in}{2.018974in}}{\pgfqpoint{1.484819in}{2.013150in}}%
\pgfpathcurveto{\pgfqpoint{1.478995in}{2.007326in}}{\pgfqpoint{1.475723in}{1.999426in}}{\pgfqpoint{1.475723in}{1.991190in}}%
\pgfpathcurveto{\pgfqpoint{1.475723in}{1.982954in}}{\pgfqpoint{1.478995in}{1.975054in}}{\pgfqpoint{1.484819in}{1.969230in}}%
\pgfpathcurveto{\pgfqpoint{1.490643in}{1.963406in}}{\pgfqpoint{1.498543in}{1.960134in}}{\pgfqpoint{1.506779in}{1.960134in}}%
\pgfpathclose%
\pgfusepath{stroke,fill}%
\end{pgfscope}%
\begin{pgfscope}%
\pgfpathrectangle{\pgfqpoint{0.100000in}{0.212622in}}{\pgfqpoint{3.696000in}{3.696000in}}%
\pgfusepath{clip}%
\pgfsetbuttcap%
\pgfsetroundjoin%
\definecolor{currentfill}{rgb}{0.121569,0.466667,0.705882}%
\pgfsetfillcolor{currentfill}%
\pgfsetfillopacity{0.362421}%
\pgfsetlinewidth{1.003750pt}%
\definecolor{currentstroke}{rgb}{0.121569,0.466667,0.705882}%
\pgfsetstrokecolor{currentstroke}%
\pgfsetstrokeopacity{0.362421}%
\pgfsetdash{}{0pt}%
\pgfpathmoveto{\pgfqpoint{2.005724in}{2.123036in}}%
\pgfpathcurveto{\pgfqpoint{2.013960in}{2.123036in}}{\pgfqpoint{2.021861in}{2.126308in}}{\pgfqpoint{2.027684in}{2.132132in}}%
\pgfpathcurveto{\pgfqpoint{2.033508in}{2.137956in}}{\pgfqpoint{2.036781in}{2.145856in}}{\pgfqpoint{2.036781in}{2.154093in}}%
\pgfpathcurveto{\pgfqpoint{2.036781in}{2.162329in}}{\pgfqpoint{2.033508in}{2.170229in}}{\pgfqpoint{2.027684in}{2.176053in}}%
\pgfpathcurveto{\pgfqpoint{2.021861in}{2.181877in}}{\pgfqpoint{2.013960in}{2.185149in}}{\pgfqpoint{2.005724in}{2.185149in}}%
\pgfpathcurveto{\pgfqpoint{1.997488in}{2.185149in}}{\pgfqpoint{1.989588in}{2.181877in}}{\pgfqpoint{1.983764in}{2.176053in}}%
\pgfpathcurveto{\pgfqpoint{1.977940in}{2.170229in}}{\pgfqpoint{1.974668in}{2.162329in}}{\pgfqpoint{1.974668in}{2.154093in}}%
\pgfpathcurveto{\pgfqpoint{1.974668in}{2.145856in}}{\pgfqpoint{1.977940in}{2.137956in}}{\pgfqpoint{1.983764in}{2.132132in}}%
\pgfpathcurveto{\pgfqpoint{1.989588in}{2.126308in}}{\pgfqpoint{1.997488in}{2.123036in}}{\pgfqpoint{2.005724in}{2.123036in}}%
\pgfpathclose%
\pgfusepath{stroke,fill}%
\end{pgfscope}%
\begin{pgfscope}%
\pgfpathrectangle{\pgfqpoint{0.100000in}{0.212622in}}{\pgfqpoint{3.696000in}{3.696000in}}%
\pgfusepath{clip}%
\pgfsetbuttcap%
\pgfsetroundjoin%
\definecolor{currentfill}{rgb}{0.121569,0.466667,0.705882}%
\pgfsetfillcolor{currentfill}%
\pgfsetfillopacity{0.362907}%
\pgfsetlinewidth{1.003750pt}%
\definecolor{currentstroke}{rgb}{0.121569,0.466667,0.705882}%
\pgfsetstrokecolor{currentstroke}%
\pgfsetstrokeopacity{0.362907}%
\pgfsetdash{}{0pt}%
\pgfpathmoveto{\pgfqpoint{1.504786in}{1.955816in}}%
\pgfpathcurveto{\pgfqpoint{1.513022in}{1.955816in}}{\pgfqpoint{1.520922in}{1.959089in}}{\pgfqpoint{1.526746in}{1.964913in}}%
\pgfpathcurveto{\pgfqpoint{1.532570in}{1.970737in}}{\pgfqpoint{1.535842in}{1.978637in}}{\pgfqpoint{1.535842in}{1.986873in}}%
\pgfpathcurveto{\pgfqpoint{1.535842in}{1.995109in}}{\pgfqpoint{1.532570in}{2.003009in}}{\pgfqpoint{1.526746in}{2.008833in}}%
\pgfpathcurveto{\pgfqpoint{1.520922in}{2.014657in}}{\pgfqpoint{1.513022in}{2.017929in}}{\pgfqpoint{1.504786in}{2.017929in}}%
\pgfpathcurveto{\pgfqpoint{1.496549in}{2.017929in}}{\pgfqpoint{1.488649in}{2.014657in}}{\pgfqpoint{1.482825in}{2.008833in}}%
\pgfpathcurveto{\pgfqpoint{1.477002in}{2.003009in}}{\pgfqpoint{1.473729in}{1.995109in}}{\pgfqpoint{1.473729in}{1.986873in}}%
\pgfpathcurveto{\pgfqpoint{1.473729in}{1.978637in}}{\pgfqpoint{1.477002in}{1.970737in}}{\pgfqpoint{1.482825in}{1.964913in}}%
\pgfpathcurveto{\pgfqpoint{1.488649in}{1.959089in}}{\pgfqpoint{1.496549in}{1.955816in}}{\pgfqpoint{1.504786in}{1.955816in}}%
\pgfpathclose%
\pgfusepath{stroke,fill}%
\end{pgfscope}%
\begin{pgfscope}%
\pgfpathrectangle{\pgfqpoint{0.100000in}{0.212622in}}{\pgfqpoint{3.696000in}{3.696000in}}%
\pgfusepath{clip}%
\pgfsetbuttcap%
\pgfsetroundjoin%
\definecolor{currentfill}{rgb}{0.121569,0.466667,0.705882}%
\pgfsetfillcolor{currentfill}%
\pgfsetfillopacity{0.363687}%
\pgfsetlinewidth{1.003750pt}%
\definecolor{currentstroke}{rgb}{0.121569,0.466667,0.705882}%
\pgfsetstrokecolor{currentstroke}%
\pgfsetstrokeopacity{0.363687}%
\pgfsetdash{}{0pt}%
\pgfpathmoveto{\pgfqpoint{1.502279in}{1.953783in}}%
\pgfpathcurveto{\pgfqpoint{1.510516in}{1.953783in}}{\pgfqpoint{1.518416in}{1.957055in}}{\pgfqpoint{1.524240in}{1.962879in}}%
\pgfpathcurveto{\pgfqpoint{1.530064in}{1.968703in}}{\pgfqpoint{1.533336in}{1.976603in}}{\pgfqpoint{1.533336in}{1.984840in}}%
\pgfpathcurveto{\pgfqpoint{1.533336in}{1.993076in}}{\pgfqpoint{1.530064in}{2.000976in}}{\pgfqpoint{1.524240in}{2.006800in}}%
\pgfpathcurveto{\pgfqpoint{1.518416in}{2.012624in}}{\pgfqpoint{1.510516in}{2.015896in}}{\pgfqpoint{1.502279in}{2.015896in}}%
\pgfpathcurveto{\pgfqpoint{1.494043in}{2.015896in}}{\pgfqpoint{1.486143in}{2.012624in}}{\pgfqpoint{1.480319in}{2.006800in}}%
\pgfpathcurveto{\pgfqpoint{1.474495in}{2.000976in}}{\pgfqpoint{1.471223in}{1.993076in}}{\pgfqpoint{1.471223in}{1.984840in}}%
\pgfpathcurveto{\pgfqpoint{1.471223in}{1.976603in}}{\pgfqpoint{1.474495in}{1.968703in}}{\pgfqpoint{1.480319in}{1.962879in}}%
\pgfpathcurveto{\pgfqpoint{1.486143in}{1.957055in}}{\pgfqpoint{1.494043in}{1.953783in}}{\pgfqpoint{1.502279in}{1.953783in}}%
\pgfpathclose%
\pgfusepath{stroke,fill}%
\end{pgfscope}%
\begin{pgfscope}%
\pgfpathrectangle{\pgfqpoint{0.100000in}{0.212622in}}{\pgfqpoint{3.696000in}{3.696000in}}%
\pgfusepath{clip}%
\pgfsetbuttcap%
\pgfsetroundjoin%
\definecolor{currentfill}{rgb}{0.121569,0.466667,0.705882}%
\pgfsetfillcolor{currentfill}%
\pgfsetfillopacity{0.364354}%
\pgfsetlinewidth{1.003750pt}%
\definecolor{currentstroke}{rgb}{0.121569,0.466667,0.705882}%
\pgfsetstrokecolor{currentstroke}%
\pgfsetstrokeopacity{0.364354}%
\pgfsetdash{}{0pt}%
\pgfpathmoveto{\pgfqpoint{2.015207in}{2.121018in}}%
\pgfpathcurveto{\pgfqpoint{2.023443in}{2.121018in}}{\pgfqpoint{2.031343in}{2.124290in}}{\pgfqpoint{2.037167in}{2.130114in}}%
\pgfpathcurveto{\pgfqpoint{2.042991in}{2.135938in}}{\pgfqpoint{2.046264in}{2.143838in}}{\pgfqpoint{2.046264in}{2.152074in}}%
\pgfpathcurveto{\pgfqpoint{2.046264in}{2.160310in}}{\pgfqpoint{2.042991in}{2.168210in}}{\pgfqpoint{2.037167in}{2.174034in}}%
\pgfpathcurveto{\pgfqpoint{2.031343in}{2.179858in}}{\pgfqpoint{2.023443in}{2.183131in}}{\pgfqpoint{2.015207in}{2.183131in}}%
\pgfpathcurveto{\pgfqpoint{2.006971in}{2.183131in}}{\pgfqpoint{1.999071in}{2.179858in}}{\pgfqpoint{1.993247in}{2.174034in}}%
\pgfpathcurveto{\pgfqpoint{1.987423in}{2.168210in}}{\pgfqpoint{1.984151in}{2.160310in}}{\pgfqpoint{1.984151in}{2.152074in}}%
\pgfpathcurveto{\pgfqpoint{1.984151in}{2.143838in}}{\pgfqpoint{1.987423in}{2.135938in}}{\pgfqpoint{1.993247in}{2.130114in}}%
\pgfpathcurveto{\pgfqpoint{1.999071in}{2.124290in}}{\pgfqpoint{2.006971in}{2.121018in}}{\pgfqpoint{2.015207in}{2.121018in}}%
\pgfpathclose%
\pgfusepath{stroke,fill}%
\end{pgfscope}%
\begin{pgfscope}%
\pgfpathrectangle{\pgfqpoint{0.100000in}{0.212622in}}{\pgfqpoint{3.696000in}{3.696000in}}%
\pgfusepath{clip}%
\pgfsetbuttcap%
\pgfsetroundjoin%
\definecolor{currentfill}{rgb}{0.121569,0.466667,0.705882}%
\pgfsetfillcolor{currentfill}%
\pgfsetfillopacity{0.364546}%
\pgfsetlinewidth{1.003750pt}%
\definecolor{currentstroke}{rgb}{0.121569,0.466667,0.705882}%
\pgfsetstrokecolor{currentstroke}%
\pgfsetstrokeopacity{0.364546}%
\pgfsetdash{}{0pt}%
\pgfpathmoveto{\pgfqpoint{1.501579in}{1.953290in}}%
\pgfpathcurveto{\pgfqpoint{1.509815in}{1.953290in}}{\pgfqpoint{1.517715in}{1.956562in}}{\pgfqpoint{1.523539in}{1.962386in}}%
\pgfpathcurveto{\pgfqpoint{1.529363in}{1.968210in}}{\pgfqpoint{1.532635in}{1.976110in}}{\pgfqpoint{1.532635in}{1.984346in}}%
\pgfpathcurveto{\pgfqpoint{1.532635in}{1.992582in}}{\pgfqpoint{1.529363in}{2.000482in}}{\pgfqpoint{1.523539in}{2.006306in}}%
\pgfpathcurveto{\pgfqpoint{1.517715in}{2.012130in}}{\pgfqpoint{1.509815in}{2.015403in}}{\pgfqpoint{1.501579in}{2.015403in}}%
\pgfpathcurveto{\pgfqpoint{1.493343in}{2.015403in}}{\pgfqpoint{1.485443in}{2.012130in}}{\pgfqpoint{1.479619in}{2.006306in}}%
\pgfpathcurveto{\pgfqpoint{1.473795in}{2.000482in}}{\pgfqpoint{1.470522in}{1.992582in}}{\pgfqpoint{1.470522in}{1.984346in}}%
\pgfpathcurveto{\pgfqpoint{1.470522in}{1.976110in}}{\pgfqpoint{1.473795in}{1.968210in}}{\pgfqpoint{1.479619in}{1.962386in}}%
\pgfpathcurveto{\pgfqpoint{1.485443in}{1.956562in}}{\pgfqpoint{1.493343in}{1.953290in}}{\pgfqpoint{1.501579in}{1.953290in}}%
\pgfpathclose%
\pgfusepath{stroke,fill}%
\end{pgfscope}%
\begin{pgfscope}%
\pgfpathrectangle{\pgfqpoint{0.100000in}{0.212622in}}{\pgfqpoint{3.696000in}{3.696000in}}%
\pgfusepath{clip}%
\pgfsetbuttcap%
\pgfsetroundjoin%
\definecolor{currentfill}{rgb}{0.121569,0.466667,0.705882}%
\pgfsetfillcolor{currentfill}%
\pgfsetfillopacity{0.365014}%
\pgfsetlinewidth{1.003750pt}%
\definecolor{currentstroke}{rgb}{0.121569,0.466667,0.705882}%
\pgfsetstrokecolor{currentstroke}%
\pgfsetstrokeopacity{0.365014}%
\pgfsetdash{}{0pt}%
\pgfpathmoveto{\pgfqpoint{1.499684in}{1.951748in}}%
\pgfpathcurveto{\pgfqpoint{1.507921in}{1.951748in}}{\pgfqpoint{1.515821in}{1.955021in}}{\pgfqpoint{1.521645in}{1.960844in}}%
\pgfpathcurveto{\pgfqpoint{1.527469in}{1.966668in}}{\pgfqpoint{1.530741in}{1.974568in}}{\pgfqpoint{1.530741in}{1.982805in}}%
\pgfpathcurveto{\pgfqpoint{1.530741in}{1.991041in}}{\pgfqpoint{1.527469in}{1.998941in}}{\pgfqpoint{1.521645in}{2.004765in}}%
\pgfpathcurveto{\pgfqpoint{1.515821in}{2.010589in}}{\pgfqpoint{1.507921in}{2.013861in}}{\pgfqpoint{1.499684in}{2.013861in}}%
\pgfpathcurveto{\pgfqpoint{1.491448in}{2.013861in}}{\pgfqpoint{1.483548in}{2.010589in}}{\pgfqpoint{1.477724in}{2.004765in}}%
\pgfpathcurveto{\pgfqpoint{1.471900in}{1.998941in}}{\pgfqpoint{1.468628in}{1.991041in}}{\pgfqpoint{1.468628in}{1.982805in}}%
\pgfpathcurveto{\pgfqpoint{1.468628in}{1.974568in}}{\pgfqpoint{1.471900in}{1.966668in}}{\pgfqpoint{1.477724in}{1.960844in}}%
\pgfpathcurveto{\pgfqpoint{1.483548in}{1.955021in}}{\pgfqpoint{1.491448in}{1.951748in}}{\pgfqpoint{1.499684in}{1.951748in}}%
\pgfpathclose%
\pgfusepath{stroke,fill}%
\end{pgfscope}%
\begin{pgfscope}%
\pgfpathrectangle{\pgfqpoint{0.100000in}{0.212622in}}{\pgfqpoint{3.696000in}{3.696000in}}%
\pgfusepath{clip}%
\pgfsetbuttcap%
\pgfsetroundjoin%
\definecolor{currentfill}{rgb}{0.121569,0.466667,0.705882}%
\pgfsetfillcolor{currentfill}%
\pgfsetfillopacity{0.365413}%
\pgfsetlinewidth{1.003750pt}%
\definecolor{currentstroke}{rgb}{0.121569,0.466667,0.705882}%
\pgfsetstrokecolor{currentstroke}%
\pgfsetstrokeopacity{0.365413}%
\pgfsetdash{}{0pt}%
\pgfpathmoveto{\pgfqpoint{1.497167in}{1.944531in}}%
\pgfpathcurveto{\pgfqpoint{1.505404in}{1.944531in}}{\pgfqpoint{1.513304in}{1.947803in}}{\pgfqpoint{1.519128in}{1.953627in}}%
\pgfpathcurveto{\pgfqpoint{1.524952in}{1.959451in}}{\pgfqpoint{1.528224in}{1.967351in}}{\pgfqpoint{1.528224in}{1.975587in}}%
\pgfpathcurveto{\pgfqpoint{1.528224in}{1.983823in}}{\pgfqpoint{1.524952in}{1.991723in}}{\pgfqpoint{1.519128in}{1.997547in}}%
\pgfpathcurveto{\pgfqpoint{1.513304in}{2.003371in}}{\pgfqpoint{1.505404in}{2.006644in}}{\pgfqpoint{1.497167in}{2.006644in}}%
\pgfpathcurveto{\pgfqpoint{1.488931in}{2.006644in}}{\pgfqpoint{1.481031in}{2.003371in}}{\pgfqpoint{1.475207in}{1.997547in}}%
\pgfpathcurveto{\pgfqpoint{1.469383in}{1.991723in}}{\pgfqpoint{1.466111in}{1.983823in}}{\pgfqpoint{1.466111in}{1.975587in}}%
\pgfpathcurveto{\pgfqpoint{1.466111in}{1.967351in}}{\pgfqpoint{1.469383in}{1.959451in}}{\pgfqpoint{1.475207in}{1.953627in}}%
\pgfpathcurveto{\pgfqpoint{1.481031in}{1.947803in}}{\pgfqpoint{1.488931in}{1.944531in}}{\pgfqpoint{1.497167in}{1.944531in}}%
\pgfpathclose%
\pgfusepath{stroke,fill}%
\end{pgfscope}%
\begin{pgfscope}%
\pgfpathrectangle{\pgfqpoint{0.100000in}{0.212622in}}{\pgfqpoint{3.696000in}{3.696000in}}%
\pgfusepath{clip}%
\pgfsetbuttcap%
\pgfsetroundjoin%
\definecolor{currentfill}{rgb}{0.121569,0.466667,0.705882}%
\pgfsetfillcolor{currentfill}%
\pgfsetfillopacity{0.366356}%
\pgfsetlinewidth{1.003750pt}%
\definecolor{currentstroke}{rgb}{0.121569,0.466667,0.705882}%
\pgfsetstrokecolor{currentstroke}%
\pgfsetstrokeopacity{0.366356}%
\pgfsetdash{}{0pt}%
\pgfpathmoveto{\pgfqpoint{1.495361in}{1.941423in}}%
\pgfpathcurveto{\pgfqpoint{1.503597in}{1.941423in}}{\pgfqpoint{1.511498in}{1.944695in}}{\pgfqpoint{1.517321in}{1.950519in}}%
\pgfpathcurveto{\pgfqpoint{1.523145in}{1.956343in}}{\pgfqpoint{1.526418in}{1.964243in}}{\pgfqpoint{1.526418in}{1.972479in}}%
\pgfpathcurveto{\pgfqpoint{1.526418in}{1.980716in}}{\pgfqpoint{1.523145in}{1.988616in}}{\pgfqpoint{1.517321in}{1.994440in}}%
\pgfpathcurveto{\pgfqpoint{1.511498in}{2.000264in}}{\pgfqpoint{1.503597in}{2.003536in}}{\pgfqpoint{1.495361in}{2.003536in}}%
\pgfpathcurveto{\pgfqpoint{1.487125in}{2.003536in}}{\pgfqpoint{1.479225in}{2.000264in}}{\pgfqpoint{1.473401in}{1.994440in}}%
\pgfpathcurveto{\pgfqpoint{1.467577in}{1.988616in}}{\pgfqpoint{1.464305in}{1.980716in}}{\pgfqpoint{1.464305in}{1.972479in}}%
\pgfpathcurveto{\pgfqpoint{1.464305in}{1.964243in}}{\pgfqpoint{1.467577in}{1.956343in}}{\pgfqpoint{1.473401in}{1.950519in}}%
\pgfpathcurveto{\pgfqpoint{1.479225in}{1.944695in}}{\pgfqpoint{1.487125in}{1.941423in}}{\pgfqpoint{1.495361in}{1.941423in}}%
\pgfpathclose%
\pgfusepath{stroke,fill}%
\end{pgfscope}%
\begin{pgfscope}%
\pgfpathrectangle{\pgfqpoint{0.100000in}{0.212622in}}{\pgfqpoint{3.696000in}{3.696000in}}%
\pgfusepath{clip}%
\pgfsetbuttcap%
\pgfsetroundjoin%
\definecolor{currentfill}{rgb}{0.121569,0.466667,0.705882}%
\pgfsetfillcolor{currentfill}%
\pgfsetfillopacity{0.366636}%
\pgfsetlinewidth{1.003750pt}%
\definecolor{currentstroke}{rgb}{0.121569,0.466667,0.705882}%
\pgfsetstrokecolor{currentstroke}%
\pgfsetstrokeopacity{0.366636}%
\pgfsetdash{}{0pt}%
\pgfpathmoveto{\pgfqpoint{2.025386in}{2.121425in}}%
\pgfpathcurveto{\pgfqpoint{2.033622in}{2.121425in}}{\pgfqpoint{2.041522in}{2.124697in}}{\pgfqpoint{2.047346in}{2.130521in}}%
\pgfpathcurveto{\pgfqpoint{2.053170in}{2.136345in}}{\pgfqpoint{2.056442in}{2.144245in}}{\pgfqpoint{2.056442in}{2.152481in}}%
\pgfpathcurveto{\pgfqpoint{2.056442in}{2.160718in}}{\pgfqpoint{2.053170in}{2.168618in}}{\pgfqpoint{2.047346in}{2.174442in}}%
\pgfpathcurveto{\pgfqpoint{2.041522in}{2.180265in}}{\pgfqpoint{2.033622in}{2.183538in}}{\pgfqpoint{2.025386in}{2.183538in}}%
\pgfpathcurveto{\pgfqpoint{2.017150in}{2.183538in}}{\pgfqpoint{2.009250in}{2.180265in}}{\pgfqpoint{2.003426in}{2.174442in}}%
\pgfpathcurveto{\pgfqpoint{1.997602in}{2.168618in}}{\pgfqpoint{1.994329in}{2.160718in}}{\pgfqpoint{1.994329in}{2.152481in}}%
\pgfpathcurveto{\pgfqpoint{1.994329in}{2.144245in}}{\pgfqpoint{1.997602in}{2.136345in}}{\pgfqpoint{2.003426in}{2.130521in}}%
\pgfpathcurveto{\pgfqpoint{2.009250in}{2.124697in}}{\pgfqpoint{2.017150in}{2.121425in}}{\pgfqpoint{2.025386in}{2.121425in}}%
\pgfpathclose%
\pgfusepath{stroke,fill}%
\end{pgfscope}%
\begin{pgfscope}%
\pgfpathrectangle{\pgfqpoint{0.100000in}{0.212622in}}{\pgfqpoint{3.696000in}{3.696000in}}%
\pgfusepath{clip}%
\pgfsetbuttcap%
\pgfsetroundjoin%
\definecolor{currentfill}{rgb}{0.121569,0.466667,0.705882}%
\pgfsetfillcolor{currentfill}%
\pgfsetfillopacity{0.366986}%
\pgfsetlinewidth{1.003750pt}%
\definecolor{currentstroke}{rgb}{0.121569,0.466667,0.705882}%
\pgfsetstrokecolor{currentstroke}%
\pgfsetstrokeopacity{0.366986}%
\pgfsetdash{}{0pt}%
\pgfpathmoveto{\pgfqpoint{1.493961in}{1.939309in}}%
\pgfpathcurveto{\pgfqpoint{1.502197in}{1.939309in}}{\pgfqpoint{1.510097in}{1.942581in}}{\pgfqpoint{1.515921in}{1.948405in}}%
\pgfpathcurveto{\pgfqpoint{1.521745in}{1.954229in}}{\pgfqpoint{1.525017in}{1.962129in}}{\pgfqpoint{1.525017in}{1.970366in}}%
\pgfpathcurveto{\pgfqpoint{1.525017in}{1.978602in}}{\pgfqpoint{1.521745in}{1.986502in}}{\pgfqpoint{1.515921in}{1.992326in}}%
\pgfpathcurveto{\pgfqpoint{1.510097in}{1.998150in}}{\pgfqpoint{1.502197in}{2.001422in}}{\pgfqpoint{1.493961in}{2.001422in}}%
\pgfpathcurveto{\pgfqpoint{1.485724in}{2.001422in}}{\pgfqpoint{1.477824in}{1.998150in}}{\pgfqpoint{1.472000in}{1.992326in}}%
\pgfpathcurveto{\pgfqpoint{1.466177in}{1.986502in}}{\pgfqpoint{1.462904in}{1.978602in}}{\pgfqpoint{1.462904in}{1.970366in}}%
\pgfpathcurveto{\pgfqpoint{1.462904in}{1.962129in}}{\pgfqpoint{1.466177in}{1.954229in}}{\pgfqpoint{1.472000in}{1.948405in}}%
\pgfpathcurveto{\pgfqpoint{1.477824in}{1.942581in}}{\pgfqpoint{1.485724in}{1.939309in}}{\pgfqpoint{1.493961in}{1.939309in}}%
\pgfpathclose%
\pgfusepath{stroke,fill}%
\end{pgfscope}%
\begin{pgfscope}%
\pgfpathrectangle{\pgfqpoint{0.100000in}{0.212622in}}{\pgfqpoint{3.696000in}{3.696000in}}%
\pgfusepath{clip}%
\pgfsetbuttcap%
\pgfsetroundjoin%
\definecolor{currentfill}{rgb}{0.121569,0.466667,0.705882}%
\pgfsetfillcolor{currentfill}%
\pgfsetfillopacity{0.367155}%
\pgfsetlinewidth{1.003750pt}%
\definecolor{currentstroke}{rgb}{0.121569,0.466667,0.705882}%
\pgfsetstrokecolor{currentstroke}%
\pgfsetstrokeopacity{0.367155}%
\pgfsetdash{}{0pt}%
\pgfpathmoveto{\pgfqpoint{1.493244in}{1.937989in}}%
\pgfpathcurveto{\pgfqpoint{1.501480in}{1.937989in}}{\pgfqpoint{1.509380in}{1.941261in}}{\pgfqpoint{1.515204in}{1.947085in}}%
\pgfpathcurveto{\pgfqpoint{1.521028in}{1.952909in}}{\pgfqpoint{1.524300in}{1.960809in}}{\pgfqpoint{1.524300in}{1.969045in}}%
\pgfpathcurveto{\pgfqpoint{1.524300in}{1.977281in}}{\pgfqpoint{1.521028in}{1.985182in}}{\pgfqpoint{1.515204in}{1.991005in}}%
\pgfpathcurveto{\pgfqpoint{1.509380in}{1.996829in}}{\pgfqpoint{1.501480in}{2.000102in}}{\pgfqpoint{1.493244in}{2.000102in}}%
\pgfpathcurveto{\pgfqpoint{1.485008in}{2.000102in}}{\pgfqpoint{1.477108in}{1.996829in}}{\pgfqpoint{1.471284in}{1.991005in}}%
\pgfpathcurveto{\pgfqpoint{1.465460in}{1.985182in}}{\pgfqpoint{1.462187in}{1.977281in}}{\pgfqpoint{1.462187in}{1.969045in}}%
\pgfpathcurveto{\pgfqpoint{1.462187in}{1.960809in}}{\pgfqpoint{1.465460in}{1.952909in}}{\pgfqpoint{1.471284in}{1.947085in}}%
\pgfpathcurveto{\pgfqpoint{1.477108in}{1.941261in}}{\pgfqpoint{1.485008in}{1.937989in}}{\pgfqpoint{1.493244in}{1.937989in}}%
\pgfpathclose%
\pgfusepath{stroke,fill}%
\end{pgfscope}%
\begin{pgfscope}%
\pgfpathrectangle{\pgfqpoint{0.100000in}{0.212622in}}{\pgfqpoint{3.696000in}{3.696000in}}%
\pgfusepath{clip}%
\pgfsetbuttcap%
\pgfsetroundjoin%
\definecolor{currentfill}{rgb}{0.121569,0.466667,0.705882}%
\pgfsetfillcolor{currentfill}%
\pgfsetfillopacity{0.367391}%
\pgfsetlinewidth{1.003750pt}%
\definecolor{currentstroke}{rgb}{0.121569,0.466667,0.705882}%
\pgfsetstrokecolor{currentstroke}%
\pgfsetstrokeopacity{0.367391}%
\pgfsetdash{}{0pt}%
\pgfpathmoveto{\pgfqpoint{1.492827in}{1.937580in}}%
\pgfpathcurveto{\pgfqpoint{1.501064in}{1.937580in}}{\pgfqpoint{1.508964in}{1.940852in}}{\pgfqpoint{1.514788in}{1.946676in}}%
\pgfpathcurveto{\pgfqpoint{1.520612in}{1.952500in}}{\pgfqpoint{1.523884in}{1.960400in}}{\pgfqpoint{1.523884in}{1.968636in}}%
\pgfpathcurveto{\pgfqpoint{1.523884in}{1.976872in}}{\pgfqpoint{1.520612in}{1.984772in}}{\pgfqpoint{1.514788in}{1.990596in}}%
\pgfpathcurveto{\pgfqpoint{1.508964in}{1.996420in}}{\pgfqpoint{1.501064in}{1.999693in}}{\pgfqpoint{1.492827in}{1.999693in}}%
\pgfpathcurveto{\pgfqpoint{1.484591in}{1.999693in}}{\pgfqpoint{1.476691in}{1.996420in}}{\pgfqpoint{1.470867in}{1.990596in}}%
\pgfpathcurveto{\pgfqpoint{1.465043in}{1.984772in}}{\pgfqpoint{1.461771in}{1.976872in}}{\pgfqpoint{1.461771in}{1.968636in}}%
\pgfpathcurveto{\pgfqpoint{1.461771in}{1.960400in}}{\pgfqpoint{1.465043in}{1.952500in}}{\pgfqpoint{1.470867in}{1.946676in}}%
\pgfpathcurveto{\pgfqpoint{1.476691in}{1.940852in}}{\pgfqpoint{1.484591in}{1.937580in}}{\pgfqpoint{1.492827in}{1.937580in}}%
\pgfpathclose%
\pgfusepath{stroke,fill}%
\end{pgfscope}%
\begin{pgfscope}%
\pgfpathrectangle{\pgfqpoint{0.100000in}{0.212622in}}{\pgfqpoint{3.696000in}{3.696000in}}%
\pgfusepath{clip}%
\pgfsetbuttcap%
\pgfsetroundjoin%
\definecolor{currentfill}{rgb}{0.121569,0.466667,0.705882}%
\pgfsetfillcolor{currentfill}%
\pgfsetfillopacity{0.367673}%
\pgfsetlinewidth{1.003750pt}%
\definecolor{currentstroke}{rgb}{0.121569,0.466667,0.705882}%
\pgfsetstrokecolor{currentstroke}%
\pgfsetstrokeopacity{0.367673}%
\pgfsetdash{}{0pt}%
\pgfpathmoveto{\pgfqpoint{1.491743in}{1.936055in}}%
\pgfpathcurveto{\pgfqpoint{1.499980in}{1.936055in}}{\pgfqpoint{1.507880in}{1.939327in}}{\pgfqpoint{1.513704in}{1.945151in}}%
\pgfpathcurveto{\pgfqpoint{1.519527in}{1.950975in}}{\pgfqpoint{1.522800in}{1.958875in}}{\pgfqpoint{1.522800in}{1.967112in}}%
\pgfpathcurveto{\pgfqpoint{1.522800in}{1.975348in}}{\pgfqpoint{1.519527in}{1.983248in}}{\pgfqpoint{1.513704in}{1.989072in}}%
\pgfpathcurveto{\pgfqpoint{1.507880in}{1.994896in}}{\pgfqpoint{1.499980in}{1.998168in}}{\pgfqpoint{1.491743in}{1.998168in}}%
\pgfpathcurveto{\pgfqpoint{1.483507in}{1.998168in}}{\pgfqpoint{1.475607in}{1.994896in}}{\pgfqpoint{1.469783in}{1.989072in}}%
\pgfpathcurveto{\pgfqpoint{1.463959in}{1.983248in}}{\pgfqpoint{1.460687in}{1.975348in}}{\pgfqpoint{1.460687in}{1.967112in}}%
\pgfpathcurveto{\pgfqpoint{1.460687in}{1.958875in}}{\pgfqpoint{1.463959in}{1.950975in}}{\pgfqpoint{1.469783in}{1.945151in}}%
\pgfpathcurveto{\pgfqpoint{1.475607in}{1.939327in}}{\pgfqpoint{1.483507in}{1.936055in}}{\pgfqpoint{1.491743in}{1.936055in}}%
\pgfpathclose%
\pgfusepath{stroke,fill}%
\end{pgfscope}%
\begin{pgfscope}%
\pgfpathrectangle{\pgfqpoint{0.100000in}{0.212622in}}{\pgfqpoint{3.696000in}{3.696000in}}%
\pgfusepath{clip}%
\pgfsetbuttcap%
\pgfsetroundjoin%
\definecolor{currentfill}{rgb}{0.121569,0.466667,0.705882}%
\pgfsetfillcolor{currentfill}%
\pgfsetfillopacity{0.368246}%
\pgfsetlinewidth{1.003750pt}%
\definecolor{currentstroke}{rgb}{0.121569,0.466667,0.705882}%
\pgfsetstrokecolor{currentstroke}%
\pgfsetstrokeopacity{0.368246}%
\pgfsetdash{}{0pt}%
\pgfpathmoveto{\pgfqpoint{1.489948in}{1.933522in}}%
\pgfpathcurveto{\pgfqpoint{1.498184in}{1.933522in}}{\pgfqpoint{1.506084in}{1.936794in}}{\pgfqpoint{1.511908in}{1.942618in}}%
\pgfpathcurveto{\pgfqpoint{1.517732in}{1.948442in}}{\pgfqpoint{1.521005in}{1.956342in}}{\pgfqpoint{1.521005in}{1.964578in}}%
\pgfpathcurveto{\pgfqpoint{1.521005in}{1.972815in}}{\pgfqpoint{1.517732in}{1.980715in}}{\pgfqpoint{1.511908in}{1.986539in}}%
\pgfpathcurveto{\pgfqpoint{1.506084in}{1.992363in}}{\pgfqpoint{1.498184in}{1.995635in}}{\pgfqpoint{1.489948in}{1.995635in}}%
\pgfpathcurveto{\pgfqpoint{1.481712in}{1.995635in}}{\pgfqpoint{1.473812in}{1.992363in}}{\pgfqpoint{1.467988in}{1.986539in}}%
\pgfpathcurveto{\pgfqpoint{1.462164in}{1.980715in}}{\pgfqpoint{1.458892in}{1.972815in}}{\pgfqpoint{1.458892in}{1.964578in}}%
\pgfpathcurveto{\pgfqpoint{1.458892in}{1.956342in}}{\pgfqpoint{1.462164in}{1.948442in}}{\pgfqpoint{1.467988in}{1.942618in}}%
\pgfpathcurveto{\pgfqpoint{1.473812in}{1.936794in}}{\pgfqpoint{1.481712in}{1.933522in}}{\pgfqpoint{1.489948in}{1.933522in}}%
\pgfpathclose%
\pgfusepath{stroke,fill}%
\end{pgfscope}%
\begin{pgfscope}%
\pgfpathrectangle{\pgfqpoint{0.100000in}{0.212622in}}{\pgfqpoint{3.696000in}{3.696000in}}%
\pgfusepath{clip}%
\pgfsetbuttcap%
\pgfsetroundjoin%
\definecolor{currentfill}{rgb}{0.121569,0.466667,0.705882}%
\pgfsetfillcolor{currentfill}%
\pgfsetfillopacity{0.368631}%
\pgfsetlinewidth{1.003750pt}%
\definecolor{currentstroke}{rgb}{0.121569,0.466667,0.705882}%
\pgfsetstrokecolor{currentstroke}%
\pgfsetstrokeopacity{0.368631}%
\pgfsetdash{}{0pt}%
\pgfpathmoveto{\pgfqpoint{2.037470in}{2.120143in}}%
\pgfpathcurveto{\pgfqpoint{2.045706in}{2.120143in}}{\pgfqpoint{2.053606in}{2.123415in}}{\pgfqpoint{2.059430in}{2.129239in}}%
\pgfpathcurveto{\pgfqpoint{2.065254in}{2.135063in}}{\pgfqpoint{2.068526in}{2.142963in}}{\pgfqpoint{2.068526in}{2.151199in}}%
\pgfpathcurveto{\pgfqpoint{2.068526in}{2.159436in}}{\pgfqpoint{2.065254in}{2.167336in}}{\pgfqpoint{2.059430in}{2.173160in}}%
\pgfpathcurveto{\pgfqpoint{2.053606in}{2.178984in}}{\pgfqpoint{2.045706in}{2.182256in}}{\pgfqpoint{2.037470in}{2.182256in}}%
\pgfpathcurveto{\pgfqpoint{2.029234in}{2.182256in}}{\pgfqpoint{2.021334in}{2.178984in}}{\pgfqpoint{2.015510in}{2.173160in}}%
\pgfpathcurveto{\pgfqpoint{2.009686in}{2.167336in}}{\pgfqpoint{2.006413in}{2.159436in}}{\pgfqpoint{2.006413in}{2.151199in}}%
\pgfpathcurveto{\pgfqpoint{2.006413in}{2.142963in}}{\pgfqpoint{2.009686in}{2.135063in}}{\pgfqpoint{2.015510in}{2.129239in}}%
\pgfpathcurveto{\pgfqpoint{2.021334in}{2.123415in}}{\pgfqpoint{2.029234in}{2.120143in}}{\pgfqpoint{2.037470in}{2.120143in}}%
\pgfpathclose%
\pgfusepath{stroke,fill}%
\end{pgfscope}%
\begin{pgfscope}%
\pgfpathrectangle{\pgfqpoint{0.100000in}{0.212622in}}{\pgfqpoint{3.696000in}{3.696000in}}%
\pgfusepath{clip}%
\pgfsetbuttcap%
\pgfsetroundjoin%
\definecolor{currentfill}{rgb}{0.121569,0.466667,0.705882}%
\pgfsetfillcolor{currentfill}%
\pgfsetfillopacity{0.368956}%
\pgfsetlinewidth{1.003750pt}%
\definecolor{currentstroke}{rgb}{0.121569,0.466667,0.705882}%
\pgfsetstrokecolor{currentstroke}%
\pgfsetstrokeopacity{0.368956}%
\pgfsetdash{}{0pt}%
\pgfpathmoveto{\pgfqpoint{1.489059in}{1.932031in}}%
\pgfpathcurveto{\pgfqpoint{1.497295in}{1.932031in}}{\pgfqpoint{1.505195in}{1.935304in}}{\pgfqpoint{1.511019in}{1.941128in}}%
\pgfpathcurveto{\pgfqpoint{1.516843in}{1.946951in}}{\pgfqpoint{1.520115in}{1.954851in}}{\pgfqpoint{1.520115in}{1.963088in}}%
\pgfpathcurveto{\pgfqpoint{1.520115in}{1.971324in}}{\pgfqpoint{1.516843in}{1.979224in}}{\pgfqpoint{1.511019in}{1.985048in}}%
\pgfpathcurveto{\pgfqpoint{1.505195in}{1.990872in}}{\pgfqpoint{1.497295in}{1.994144in}}{\pgfqpoint{1.489059in}{1.994144in}}%
\pgfpathcurveto{\pgfqpoint{1.480822in}{1.994144in}}{\pgfqpoint{1.472922in}{1.990872in}}{\pgfqpoint{1.467098in}{1.985048in}}%
\pgfpathcurveto{\pgfqpoint{1.461275in}{1.979224in}}{\pgfqpoint{1.458002in}{1.971324in}}{\pgfqpoint{1.458002in}{1.963088in}}%
\pgfpathcurveto{\pgfqpoint{1.458002in}{1.954851in}}{\pgfqpoint{1.461275in}{1.946951in}}{\pgfqpoint{1.467098in}{1.941128in}}%
\pgfpathcurveto{\pgfqpoint{1.472922in}{1.935304in}}{\pgfqpoint{1.480822in}{1.932031in}}{\pgfqpoint{1.489059in}{1.932031in}}%
\pgfpathclose%
\pgfusepath{stroke,fill}%
\end{pgfscope}%
\begin{pgfscope}%
\pgfpathrectangle{\pgfqpoint{0.100000in}{0.212622in}}{\pgfqpoint{3.696000in}{3.696000in}}%
\pgfusepath{clip}%
\pgfsetbuttcap%
\pgfsetroundjoin%
\definecolor{currentfill}{rgb}{0.121569,0.466667,0.705882}%
\pgfsetfillcolor{currentfill}%
\pgfsetfillopacity{0.370067}%
\pgfsetlinewidth{1.003750pt}%
\definecolor{currentstroke}{rgb}{0.121569,0.466667,0.705882}%
\pgfsetstrokecolor{currentstroke}%
\pgfsetstrokeopacity{0.370067}%
\pgfsetdash{}{0pt}%
\pgfpathmoveto{\pgfqpoint{1.484636in}{1.930994in}}%
\pgfpathcurveto{\pgfqpoint{1.492873in}{1.930994in}}{\pgfqpoint{1.500773in}{1.934267in}}{\pgfqpoint{1.506597in}{1.940090in}}%
\pgfpathcurveto{\pgfqpoint{1.512421in}{1.945914in}}{\pgfqpoint{1.515693in}{1.953814in}}{\pgfqpoint{1.515693in}{1.962051in}}%
\pgfpathcurveto{\pgfqpoint{1.515693in}{1.970287in}}{\pgfqpoint{1.512421in}{1.978187in}}{\pgfqpoint{1.506597in}{1.984011in}}%
\pgfpathcurveto{\pgfqpoint{1.500773in}{1.989835in}}{\pgfqpoint{1.492873in}{1.993107in}}{\pgfqpoint{1.484636in}{1.993107in}}%
\pgfpathcurveto{\pgfqpoint{1.476400in}{1.993107in}}{\pgfqpoint{1.468500in}{1.989835in}}{\pgfqpoint{1.462676in}{1.984011in}}%
\pgfpathcurveto{\pgfqpoint{1.456852in}{1.978187in}}{\pgfqpoint{1.453580in}{1.970287in}}{\pgfqpoint{1.453580in}{1.962051in}}%
\pgfpathcurveto{\pgfqpoint{1.453580in}{1.953814in}}{\pgfqpoint{1.456852in}{1.945914in}}{\pgfqpoint{1.462676in}{1.940090in}}%
\pgfpathcurveto{\pgfqpoint{1.468500in}{1.934267in}}{\pgfqpoint{1.476400in}{1.930994in}}{\pgfqpoint{1.484636in}{1.930994in}}%
\pgfpathclose%
\pgfusepath{stroke,fill}%
\end{pgfscope}%
\begin{pgfscope}%
\pgfpathrectangle{\pgfqpoint{0.100000in}{0.212622in}}{\pgfqpoint{3.696000in}{3.696000in}}%
\pgfusepath{clip}%
\pgfsetbuttcap%
\pgfsetroundjoin%
\definecolor{currentfill}{rgb}{0.121569,0.466667,0.705882}%
\pgfsetfillcolor{currentfill}%
\pgfsetfillopacity{0.370951}%
\pgfsetlinewidth{1.003750pt}%
\definecolor{currentstroke}{rgb}{0.121569,0.466667,0.705882}%
\pgfsetstrokecolor{currentstroke}%
\pgfsetstrokeopacity{0.370951}%
\pgfsetdash{}{0pt}%
\pgfpathmoveto{\pgfqpoint{2.050177in}{2.120792in}}%
\pgfpathcurveto{\pgfqpoint{2.058414in}{2.120792in}}{\pgfqpoint{2.066314in}{2.124065in}}{\pgfqpoint{2.072137in}{2.129889in}}%
\pgfpathcurveto{\pgfqpoint{2.077961in}{2.135712in}}{\pgfqpoint{2.081234in}{2.143613in}}{\pgfqpoint{2.081234in}{2.151849in}}%
\pgfpathcurveto{\pgfqpoint{2.081234in}{2.160085in}}{\pgfqpoint{2.077961in}{2.167985in}}{\pgfqpoint{2.072137in}{2.173809in}}%
\pgfpathcurveto{\pgfqpoint{2.066314in}{2.179633in}}{\pgfqpoint{2.058414in}{2.182905in}}{\pgfqpoint{2.050177in}{2.182905in}}%
\pgfpathcurveto{\pgfqpoint{2.041941in}{2.182905in}}{\pgfqpoint{2.034041in}{2.179633in}}{\pgfqpoint{2.028217in}{2.173809in}}%
\pgfpathcurveto{\pgfqpoint{2.022393in}{2.167985in}}{\pgfqpoint{2.019121in}{2.160085in}}{\pgfqpoint{2.019121in}{2.151849in}}%
\pgfpathcurveto{\pgfqpoint{2.019121in}{2.143613in}}{\pgfqpoint{2.022393in}{2.135712in}}{\pgfqpoint{2.028217in}{2.129889in}}%
\pgfpathcurveto{\pgfqpoint{2.034041in}{2.124065in}}{\pgfqpoint{2.041941in}{2.120792in}}{\pgfqpoint{2.050177in}{2.120792in}}%
\pgfpathclose%
\pgfusepath{stroke,fill}%
\end{pgfscope}%
\begin{pgfscope}%
\pgfpathrectangle{\pgfqpoint{0.100000in}{0.212622in}}{\pgfqpoint{3.696000in}{3.696000in}}%
\pgfusepath{clip}%
\pgfsetbuttcap%
\pgfsetroundjoin%
\definecolor{currentfill}{rgb}{0.121569,0.466667,0.705882}%
\pgfsetfillcolor{currentfill}%
\pgfsetfillopacity{0.371123}%
\pgfsetlinewidth{1.003750pt}%
\definecolor{currentstroke}{rgb}{0.121569,0.466667,0.705882}%
\pgfsetstrokecolor{currentstroke}%
\pgfsetstrokeopacity{0.371123}%
\pgfsetdash{}{0pt}%
\pgfpathmoveto{\pgfqpoint{1.483141in}{1.929229in}}%
\pgfpathcurveto{\pgfqpoint{1.491377in}{1.929229in}}{\pgfqpoint{1.499277in}{1.932501in}}{\pgfqpoint{1.505101in}{1.938325in}}%
\pgfpathcurveto{\pgfqpoint{1.510925in}{1.944149in}}{\pgfqpoint{1.514197in}{1.952049in}}{\pgfqpoint{1.514197in}{1.960286in}}%
\pgfpathcurveto{\pgfqpoint{1.514197in}{1.968522in}}{\pgfqpoint{1.510925in}{1.976422in}}{\pgfqpoint{1.505101in}{1.982246in}}%
\pgfpathcurveto{\pgfqpoint{1.499277in}{1.988070in}}{\pgfqpoint{1.491377in}{1.991342in}}{\pgfqpoint{1.483141in}{1.991342in}}%
\pgfpathcurveto{\pgfqpoint{1.474905in}{1.991342in}}{\pgfqpoint{1.467005in}{1.988070in}}{\pgfqpoint{1.461181in}{1.982246in}}%
\pgfpathcurveto{\pgfqpoint{1.455357in}{1.976422in}}{\pgfqpoint{1.452084in}{1.968522in}}{\pgfqpoint{1.452084in}{1.960286in}}%
\pgfpathcurveto{\pgfqpoint{1.452084in}{1.952049in}}{\pgfqpoint{1.455357in}{1.944149in}}{\pgfqpoint{1.461181in}{1.938325in}}%
\pgfpathcurveto{\pgfqpoint{1.467005in}{1.932501in}}{\pgfqpoint{1.474905in}{1.929229in}}{\pgfqpoint{1.483141in}{1.929229in}}%
\pgfpathclose%
\pgfusepath{stroke,fill}%
\end{pgfscope}%
\begin{pgfscope}%
\pgfpathrectangle{\pgfqpoint{0.100000in}{0.212622in}}{\pgfqpoint{3.696000in}{3.696000in}}%
\pgfusepath{clip}%
\pgfsetbuttcap%
\pgfsetroundjoin%
\definecolor{currentfill}{rgb}{0.121569,0.466667,0.705882}%
\pgfsetfillcolor{currentfill}%
\pgfsetfillopacity{0.372098}%
\pgfsetlinewidth{1.003750pt}%
\definecolor{currentstroke}{rgb}{0.121569,0.466667,0.705882}%
\pgfsetstrokecolor{currentstroke}%
\pgfsetstrokeopacity{0.372098}%
\pgfsetdash{}{0pt}%
\pgfpathmoveto{\pgfqpoint{1.481367in}{1.927925in}}%
\pgfpathcurveto{\pgfqpoint{1.489603in}{1.927925in}}{\pgfqpoint{1.497503in}{1.931197in}}{\pgfqpoint{1.503327in}{1.937021in}}%
\pgfpathcurveto{\pgfqpoint{1.509151in}{1.942845in}}{\pgfqpoint{1.512423in}{1.950745in}}{\pgfqpoint{1.512423in}{1.958982in}}%
\pgfpathcurveto{\pgfqpoint{1.512423in}{1.967218in}}{\pgfqpoint{1.509151in}{1.975118in}}{\pgfqpoint{1.503327in}{1.980942in}}%
\pgfpathcurveto{\pgfqpoint{1.497503in}{1.986766in}}{\pgfqpoint{1.489603in}{1.990038in}}{\pgfqpoint{1.481367in}{1.990038in}}%
\pgfpathcurveto{\pgfqpoint{1.473131in}{1.990038in}}{\pgfqpoint{1.465231in}{1.986766in}}{\pgfqpoint{1.459407in}{1.980942in}}%
\pgfpathcurveto{\pgfqpoint{1.453583in}{1.975118in}}{\pgfqpoint{1.450310in}{1.967218in}}{\pgfqpoint{1.450310in}{1.958982in}}%
\pgfpathcurveto{\pgfqpoint{1.450310in}{1.950745in}}{\pgfqpoint{1.453583in}{1.942845in}}{\pgfqpoint{1.459407in}{1.937021in}}%
\pgfpathcurveto{\pgfqpoint{1.465231in}{1.931197in}}{\pgfqpoint{1.473131in}{1.927925in}}{\pgfqpoint{1.481367in}{1.927925in}}%
\pgfpathclose%
\pgfusepath{stroke,fill}%
\end{pgfscope}%
\begin{pgfscope}%
\pgfpathrectangle{\pgfqpoint{0.100000in}{0.212622in}}{\pgfqpoint{3.696000in}{3.696000in}}%
\pgfusepath{clip}%
\pgfsetbuttcap%
\pgfsetroundjoin%
\definecolor{currentfill}{rgb}{0.121569,0.466667,0.705882}%
\pgfsetfillcolor{currentfill}%
\pgfsetfillopacity{0.372939}%
\pgfsetlinewidth{1.003750pt}%
\definecolor{currentstroke}{rgb}{0.121569,0.466667,0.705882}%
\pgfsetstrokecolor{currentstroke}%
\pgfsetstrokeopacity{0.372939}%
\pgfsetdash{}{0pt}%
\pgfpathmoveto{\pgfqpoint{2.063297in}{2.118231in}}%
\pgfpathcurveto{\pgfqpoint{2.071533in}{2.118231in}}{\pgfqpoint{2.079434in}{2.121503in}}{\pgfqpoint{2.085257in}{2.127327in}}%
\pgfpathcurveto{\pgfqpoint{2.091081in}{2.133151in}}{\pgfqpoint{2.094354in}{2.141051in}}{\pgfqpoint{2.094354in}{2.149287in}}%
\pgfpathcurveto{\pgfqpoint{2.094354in}{2.157523in}}{\pgfqpoint{2.091081in}{2.165423in}}{\pgfqpoint{2.085257in}{2.171247in}}%
\pgfpathcurveto{\pgfqpoint{2.079434in}{2.177071in}}{\pgfqpoint{2.071533in}{2.180344in}}{\pgfqpoint{2.063297in}{2.180344in}}%
\pgfpathcurveto{\pgfqpoint{2.055061in}{2.180344in}}{\pgfqpoint{2.047161in}{2.177071in}}{\pgfqpoint{2.041337in}{2.171247in}}%
\pgfpathcurveto{\pgfqpoint{2.035513in}{2.165423in}}{\pgfqpoint{2.032241in}{2.157523in}}{\pgfqpoint{2.032241in}{2.149287in}}%
\pgfpathcurveto{\pgfqpoint{2.032241in}{2.141051in}}{\pgfqpoint{2.035513in}{2.133151in}}{\pgfqpoint{2.041337in}{2.127327in}}%
\pgfpathcurveto{\pgfqpoint{2.047161in}{2.121503in}}{\pgfqpoint{2.055061in}{2.118231in}}{\pgfqpoint{2.063297in}{2.118231in}}%
\pgfpathclose%
\pgfusepath{stroke,fill}%
\end{pgfscope}%
\begin{pgfscope}%
\pgfpathrectangle{\pgfqpoint{0.100000in}{0.212622in}}{\pgfqpoint{3.696000in}{3.696000in}}%
\pgfusepath{clip}%
\pgfsetbuttcap%
\pgfsetroundjoin%
\definecolor{currentfill}{rgb}{0.121569,0.466667,0.705882}%
\pgfsetfillcolor{currentfill}%
\pgfsetfillopacity{0.376307}%
\pgfsetlinewidth{1.003750pt}%
\definecolor{currentstroke}{rgb}{0.121569,0.466667,0.705882}%
\pgfsetstrokecolor{currentstroke}%
\pgfsetstrokeopacity{0.376307}%
\pgfsetdash{}{0pt}%
\pgfpathmoveto{\pgfqpoint{1.478752in}{1.943063in}}%
\pgfpathcurveto{\pgfqpoint{1.486988in}{1.943063in}}{\pgfqpoint{1.494888in}{1.946335in}}{\pgfqpoint{1.500712in}{1.952159in}}%
\pgfpathcurveto{\pgfqpoint{1.506536in}{1.957983in}}{\pgfqpoint{1.509808in}{1.965883in}}{\pgfqpoint{1.509808in}{1.974119in}}%
\pgfpathcurveto{\pgfqpoint{1.509808in}{1.982355in}}{\pgfqpoint{1.506536in}{1.990256in}}{\pgfqpoint{1.500712in}{1.996079in}}%
\pgfpathcurveto{\pgfqpoint{1.494888in}{2.001903in}}{\pgfqpoint{1.486988in}{2.005176in}}{\pgfqpoint{1.478752in}{2.005176in}}%
\pgfpathcurveto{\pgfqpoint{1.470515in}{2.005176in}}{\pgfqpoint{1.462615in}{2.001903in}}{\pgfqpoint{1.456791in}{1.996079in}}%
\pgfpathcurveto{\pgfqpoint{1.450967in}{1.990256in}}{\pgfqpoint{1.447695in}{1.982355in}}{\pgfqpoint{1.447695in}{1.974119in}}%
\pgfpathcurveto{\pgfqpoint{1.447695in}{1.965883in}}{\pgfqpoint{1.450967in}{1.957983in}}{\pgfqpoint{1.456791in}{1.952159in}}%
\pgfpathcurveto{\pgfqpoint{1.462615in}{1.946335in}}{\pgfqpoint{1.470515in}{1.943063in}}{\pgfqpoint{1.478752in}{1.943063in}}%
\pgfpathclose%
\pgfusepath{stroke,fill}%
\end{pgfscope}%
\begin{pgfscope}%
\pgfpathrectangle{\pgfqpoint{0.100000in}{0.212622in}}{\pgfqpoint{3.696000in}{3.696000in}}%
\pgfusepath{clip}%
\pgfsetbuttcap%
\pgfsetroundjoin%
\definecolor{currentfill}{rgb}{0.121569,0.466667,0.705882}%
\pgfsetfillcolor{currentfill}%
\pgfsetfillopacity{0.376652}%
\pgfsetlinewidth{1.003750pt}%
\definecolor{currentstroke}{rgb}{0.121569,0.466667,0.705882}%
\pgfsetstrokecolor{currentstroke}%
\pgfsetstrokeopacity{0.376652}%
\pgfsetdash{}{0pt}%
\pgfpathmoveto{\pgfqpoint{2.078437in}{2.130242in}}%
\pgfpathcurveto{\pgfqpoint{2.086674in}{2.130242in}}{\pgfqpoint{2.094574in}{2.133514in}}{\pgfqpoint{2.100398in}{2.139338in}}%
\pgfpathcurveto{\pgfqpoint{2.106221in}{2.145162in}}{\pgfqpoint{2.109494in}{2.153062in}}{\pgfqpoint{2.109494in}{2.161299in}}%
\pgfpathcurveto{\pgfqpoint{2.109494in}{2.169535in}}{\pgfqpoint{2.106221in}{2.177435in}}{\pgfqpoint{2.100398in}{2.183259in}}%
\pgfpathcurveto{\pgfqpoint{2.094574in}{2.189083in}}{\pgfqpoint{2.086674in}{2.192355in}}{\pgfqpoint{2.078437in}{2.192355in}}%
\pgfpathcurveto{\pgfqpoint{2.070201in}{2.192355in}}{\pgfqpoint{2.062301in}{2.189083in}}{\pgfqpoint{2.056477in}{2.183259in}}%
\pgfpathcurveto{\pgfqpoint{2.050653in}{2.177435in}}{\pgfqpoint{2.047381in}{2.169535in}}{\pgfqpoint{2.047381in}{2.161299in}}%
\pgfpathcurveto{\pgfqpoint{2.047381in}{2.153062in}}{\pgfqpoint{2.050653in}{2.145162in}}{\pgfqpoint{2.056477in}{2.139338in}}%
\pgfpathcurveto{\pgfqpoint{2.062301in}{2.133514in}}{\pgfqpoint{2.070201in}{2.130242in}}{\pgfqpoint{2.078437in}{2.130242in}}%
\pgfpathclose%
\pgfusepath{stroke,fill}%
\end{pgfscope}%
\begin{pgfscope}%
\pgfpathrectangle{\pgfqpoint{0.100000in}{0.212622in}}{\pgfqpoint{3.696000in}{3.696000in}}%
\pgfusepath{clip}%
\pgfsetbuttcap%
\pgfsetroundjoin%
\definecolor{currentfill}{rgb}{0.121569,0.466667,0.705882}%
\pgfsetfillcolor{currentfill}%
\pgfsetfillopacity{0.377179}%
\pgfsetlinewidth{1.003750pt}%
\definecolor{currentstroke}{rgb}{0.121569,0.466667,0.705882}%
\pgfsetstrokecolor{currentstroke}%
\pgfsetstrokeopacity{0.377179}%
\pgfsetdash{}{0pt}%
\pgfpathmoveto{\pgfqpoint{2.085584in}{2.122054in}}%
\pgfpathcurveto{\pgfqpoint{2.093820in}{2.122054in}}{\pgfqpoint{2.101720in}{2.125326in}}{\pgfqpoint{2.107544in}{2.131150in}}%
\pgfpathcurveto{\pgfqpoint{2.113368in}{2.136974in}}{\pgfqpoint{2.116640in}{2.144874in}}{\pgfqpoint{2.116640in}{2.153110in}}%
\pgfpathcurveto{\pgfqpoint{2.116640in}{2.161347in}}{\pgfqpoint{2.113368in}{2.169247in}}{\pgfqpoint{2.107544in}{2.175071in}}%
\pgfpathcurveto{\pgfqpoint{2.101720in}{2.180895in}}{\pgfqpoint{2.093820in}{2.184167in}}{\pgfqpoint{2.085584in}{2.184167in}}%
\pgfpathcurveto{\pgfqpoint{2.077347in}{2.184167in}}{\pgfqpoint{2.069447in}{2.180895in}}{\pgfqpoint{2.063623in}{2.175071in}}%
\pgfpathcurveto{\pgfqpoint{2.057799in}{2.169247in}}{\pgfqpoint{2.054527in}{2.161347in}}{\pgfqpoint{2.054527in}{2.153110in}}%
\pgfpathcurveto{\pgfqpoint{2.054527in}{2.144874in}}{\pgfqpoint{2.057799in}{2.136974in}}{\pgfqpoint{2.063623in}{2.131150in}}%
\pgfpathcurveto{\pgfqpoint{2.069447in}{2.125326in}}{\pgfqpoint{2.077347in}{2.122054in}}{\pgfqpoint{2.085584in}{2.122054in}}%
\pgfpathclose%
\pgfusepath{stroke,fill}%
\end{pgfscope}%
\begin{pgfscope}%
\pgfpathrectangle{\pgfqpoint{0.100000in}{0.212622in}}{\pgfqpoint{3.696000in}{3.696000in}}%
\pgfusepath{clip}%
\pgfsetbuttcap%
\pgfsetroundjoin%
\definecolor{currentfill}{rgb}{0.121569,0.466667,0.705882}%
\pgfsetfillcolor{currentfill}%
\pgfsetfillopacity{0.377451}%
\pgfsetlinewidth{1.003750pt}%
\definecolor{currentstroke}{rgb}{0.121569,0.466667,0.705882}%
\pgfsetstrokecolor{currentstroke}%
\pgfsetstrokeopacity{0.377451}%
\pgfsetdash{}{0pt}%
\pgfpathmoveto{\pgfqpoint{1.475456in}{1.938141in}}%
\pgfpathcurveto{\pgfqpoint{1.483693in}{1.938141in}}{\pgfqpoint{1.491593in}{1.941413in}}{\pgfqpoint{1.497417in}{1.947237in}}%
\pgfpathcurveto{\pgfqpoint{1.503240in}{1.953061in}}{\pgfqpoint{1.506513in}{1.960961in}}{\pgfqpoint{1.506513in}{1.969197in}}%
\pgfpathcurveto{\pgfqpoint{1.506513in}{1.977433in}}{\pgfqpoint{1.503240in}{1.985334in}}{\pgfqpoint{1.497417in}{1.991157in}}%
\pgfpathcurveto{\pgfqpoint{1.491593in}{1.996981in}}{\pgfqpoint{1.483693in}{2.000254in}}{\pgfqpoint{1.475456in}{2.000254in}}%
\pgfpathcurveto{\pgfqpoint{1.467220in}{2.000254in}}{\pgfqpoint{1.459320in}{1.996981in}}{\pgfqpoint{1.453496in}{1.991157in}}%
\pgfpathcurveto{\pgfqpoint{1.447672in}{1.985334in}}{\pgfqpoint{1.444400in}{1.977433in}}{\pgfqpoint{1.444400in}{1.969197in}}%
\pgfpathcurveto{\pgfqpoint{1.444400in}{1.960961in}}{\pgfqpoint{1.447672in}{1.953061in}}{\pgfqpoint{1.453496in}{1.947237in}}%
\pgfpathcurveto{\pgfqpoint{1.459320in}{1.941413in}}{\pgfqpoint{1.467220in}{1.938141in}}{\pgfqpoint{1.475456in}{1.938141in}}%
\pgfpathclose%
\pgfusepath{stroke,fill}%
\end{pgfscope}%
\begin{pgfscope}%
\pgfpathrectangle{\pgfqpoint{0.100000in}{0.212622in}}{\pgfqpoint{3.696000in}{3.696000in}}%
\pgfusepath{clip}%
\pgfsetbuttcap%
\pgfsetroundjoin%
\definecolor{currentfill}{rgb}{0.121569,0.466667,0.705882}%
\pgfsetfillcolor{currentfill}%
\pgfsetfillopacity{0.378569}%
\pgfsetlinewidth{1.003750pt}%
\definecolor{currentstroke}{rgb}{0.121569,0.466667,0.705882}%
\pgfsetstrokecolor{currentstroke}%
\pgfsetstrokeopacity{0.378569}%
\pgfsetdash{}{0pt}%
\pgfpathmoveto{\pgfqpoint{1.469872in}{1.937462in}}%
\pgfpathcurveto{\pgfqpoint{1.478109in}{1.937462in}}{\pgfqpoint{1.486009in}{1.940735in}}{\pgfqpoint{1.491833in}{1.946559in}}%
\pgfpathcurveto{\pgfqpoint{1.497657in}{1.952383in}}{\pgfqpoint{1.500929in}{1.960283in}}{\pgfqpoint{1.500929in}{1.968519in}}%
\pgfpathcurveto{\pgfqpoint{1.500929in}{1.976755in}}{\pgfqpoint{1.497657in}{1.984655in}}{\pgfqpoint{1.491833in}{1.990479in}}%
\pgfpathcurveto{\pgfqpoint{1.486009in}{1.996303in}}{\pgfqpoint{1.478109in}{1.999575in}}{\pgfqpoint{1.469872in}{1.999575in}}%
\pgfpathcurveto{\pgfqpoint{1.461636in}{1.999575in}}{\pgfqpoint{1.453736in}{1.996303in}}{\pgfqpoint{1.447912in}{1.990479in}}%
\pgfpathcurveto{\pgfqpoint{1.442088in}{1.984655in}}{\pgfqpoint{1.438816in}{1.976755in}}{\pgfqpoint{1.438816in}{1.968519in}}%
\pgfpathcurveto{\pgfqpoint{1.438816in}{1.960283in}}{\pgfqpoint{1.442088in}{1.952383in}}{\pgfqpoint{1.447912in}{1.946559in}}%
\pgfpathcurveto{\pgfqpoint{1.453736in}{1.940735in}}{\pgfqpoint{1.461636in}{1.937462in}}{\pgfqpoint{1.469872in}{1.937462in}}%
\pgfpathclose%
\pgfusepath{stroke,fill}%
\end{pgfscope}%
\begin{pgfscope}%
\pgfpathrectangle{\pgfqpoint{0.100000in}{0.212622in}}{\pgfqpoint{3.696000in}{3.696000in}}%
\pgfusepath{clip}%
\pgfsetbuttcap%
\pgfsetroundjoin%
\definecolor{currentfill}{rgb}{0.121569,0.466667,0.705882}%
\pgfsetfillcolor{currentfill}%
\pgfsetfillopacity{0.378663}%
\pgfsetlinewidth{1.003750pt}%
\definecolor{currentstroke}{rgb}{0.121569,0.466667,0.705882}%
\pgfsetstrokecolor{currentstroke}%
\pgfsetstrokeopacity{0.378663}%
\pgfsetdash{}{0pt}%
\pgfpathmoveto{\pgfqpoint{2.094832in}{2.122790in}}%
\pgfpathcurveto{\pgfqpoint{2.103068in}{2.122790in}}{\pgfqpoint{2.110968in}{2.126063in}}{\pgfqpoint{2.116792in}{2.131887in}}%
\pgfpathcurveto{\pgfqpoint{2.122616in}{2.137711in}}{\pgfqpoint{2.125889in}{2.145611in}}{\pgfqpoint{2.125889in}{2.153847in}}%
\pgfpathcurveto{\pgfqpoint{2.125889in}{2.162083in}}{\pgfqpoint{2.122616in}{2.169983in}}{\pgfqpoint{2.116792in}{2.175807in}}%
\pgfpathcurveto{\pgfqpoint{2.110968in}{2.181631in}}{\pgfqpoint{2.103068in}{2.184903in}}{\pgfqpoint{2.094832in}{2.184903in}}%
\pgfpathcurveto{\pgfqpoint{2.086596in}{2.184903in}}{\pgfqpoint{2.078696in}{2.181631in}}{\pgfqpoint{2.072872in}{2.175807in}}%
\pgfpathcurveto{\pgfqpoint{2.067048in}{2.169983in}}{\pgfqpoint{2.063776in}{2.162083in}}{\pgfqpoint{2.063776in}{2.153847in}}%
\pgfpathcurveto{\pgfqpoint{2.063776in}{2.145611in}}{\pgfqpoint{2.067048in}{2.137711in}}{\pgfqpoint{2.072872in}{2.131887in}}%
\pgfpathcurveto{\pgfqpoint{2.078696in}{2.126063in}}{\pgfqpoint{2.086596in}{2.122790in}}{\pgfqpoint{2.094832in}{2.122790in}}%
\pgfpathclose%
\pgfusepath{stroke,fill}%
\end{pgfscope}%
\begin{pgfscope}%
\pgfpathrectangle{\pgfqpoint{0.100000in}{0.212622in}}{\pgfqpoint{3.696000in}{3.696000in}}%
\pgfusepath{clip}%
\pgfsetbuttcap%
\pgfsetroundjoin%
\definecolor{currentfill}{rgb}{0.121569,0.466667,0.705882}%
\pgfsetfillcolor{currentfill}%
\pgfsetfillopacity{0.379720}%
\pgfsetlinewidth{1.003750pt}%
\definecolor{currentstroke}{rgb}{0.121569,0.466667,0.705882}%
\pgfsetstrokecolor{currentstroke}%
\pgfsetstrokeopacity{0.379720}%
\pgfsetdash{}{0pt}%
\pgfpathmoveto{\pgfqpoint{2.103834in}{2.116522in}}%
\pgfpathcurveto{\pgfqpoint{2.112070in}{2.116522in}}{\pgfqpoint{2.119970in}{2.119795in}}{\pgfqpoint{2.125794in}{2.125619in}}%
\pgfpathcurveto{\pgfqpoint{2.131618in}{2.131443in}}{\pgfqpoint{2.134890in}{2.139343in}}{\pgfqpoint{2.134890in}{2.147579in}}%
\pgfpathcurveto{\pgfqpoint{2.134890in}{2.155815in}}{\pgfqpoint{2.131618in}{2.163715in}}{\pgfqpoint{2.125794in}{2.169539in}}%
\pgfpathcurveto{\pgfqpoint{2.119970in}{2.175363in}}{\pgfqpoint{2.112070in}{2.178635in}}{\pgfqpoint{2.103834in}{2.178635in}}%
\pgfpathcurveto{\pgfqpoint{2.095597in}{2.178635in}}{\pgfqpoint{2.087697in}{2.175363in}}{\pgfqpoint{2.081873in}{2.169539in}}%
\pgfpathcurveto{\pgfqpoint{2.076050in}{2.163715in}}{\pgfqpoint{2.072777in}{2.155815in}}{\pgfqpoint{2.072777in}{2.147579in}}%
\pgfpathcurveto{\pgfqpoint{2.072777in}{2.139343in}}{\pgfqpoint{2.076050in}{2.131443in}}{\pgfqpoint{2.081873in}{2.125619in}}%
\pgfpathcurveto{\pgfqpoint{2.087697in}{2.119795in}}{\pgfqpoint{2.095597in}{2.116522in}}{\pgfqpoint{2.103834in}{2.116522in}}%
\pgfpathclose%
\pgfusepath{stroke,fill}%
\end{pgfscope}%
\begin{pgfscope}%
\pgfpathrectangle{\pgfqpoint{0.100000in}{0.212622in}}{\pgfqpoint{3.696000in}{3.696000in}}%
\pgfusepath{clip}%
\pgfsetbuttcap%
\pgfsetroundjoin%
\definecolor{currentfill}{rgb}{0.121569,0.466667,0.705882}%
\pgfsetfillcolor{currentfill}%
\pgfsetfillopacity{0.380606}%
\pgfsetlinewidth{1.003750pt}%
\definecolor{currentstroke}{rgb}{0.121569,0.466667,0.705882}%
\pgfsetstrokecolor{currentstroke}%
\pgfsetstrokeopacity{0.380606}%
\pgfsetdash{}{0pt}%
\pgfpathmoveto{\pgfqpoint{1.466653in}{1.940665in}}%
\pgfpathcurveto{\pgfqpoint{1.474889in}{1.940665in}}{\pgfqpoint{1.482789in}{1.943937in}}{\pgfqpoint{1.488613in}{1.949761in}}%
\pgfpathcurveto{\pgfqpoint{1.494437in}{1.955585in}}{\pgfqpoint{1.497709in}{1.963485in}}{\pgfqpoint{1.497709in}{1.971721in}}%
\pgfpathcurveto{\pgfqpoint{1.497709in}{1.979957in}}{\pgfqpoint{1.494437in}{1.987857in}}{\pgfqpoint{1.488613in}{1.993681in}}%
\pgfpathcurveto{\pgfqpoint{1.482789in}{1.999505in}}{\pgfqpoint{1.474889in}{2.002778in}}{\pgfqpoint{1.466653in}{2.002778in}}%
\pgfpathcurveto{\pgfqpoint{1.458416in}{2.002778in}}{\pgfqpoint{1.450516in}{1.999505in}}{\pgfqpoint{1.444692in}{1.993681in}}%
\pgfpathcurveto{\pgfqpoint{1.438868in}{1.987857in}}{\pgfqpoint{1.435596in}{1.979957in}}{\pgfqpoint{1.435596in}{1.971721in}}%
\pgfpathcurveto{\pgfqpoint{1.435596in}{1.963485in}}{\pgfqpoint{1.438868in}{1.955585in}}{\pgfqpoint{1.444692in}{1.949761in}}%
\pgfpathcurveto{\pgfqpoint{1.450516in}{1.943937in}}{\pgfqpoint{1.458416in}{1.940665in}}{\pgfqpoint{1.466653in}{1.940665in}}%
\pgfpathclose%
\pgfusepath{stroke,fill}%
\end{pgfscope}%
\begin{pgfscope}%
\pgfpathrectangle{\pgfqpoint{0.100000in}{0.212622in}}{\pgfqpoint{3.696000in}{3.696000in}}%
\pgfusepath{clip}%
\pgfsetbuttcap%
\pgfsetroundjoin%
\definecolor{currentfill}{rgb}{0.121569,0.466667,0.705882}%
\pgfsetfillcolor{currentfill}%
\pgfsetfillopacity{0.382045}%
\pgfsetlinewidth{1.003750pt}%
\definecolor{currentstroke}{rgb}{0.121569,0.466667,0.705882}%
\pgfsetstrokecolor{currentstroke}%
\pgfsetstrokeopacity{0.382045}%
\pgfsetdash{}{0pt}%
\pgfpathmoveto{\pgfqpoint{2.114093in}{2.119721in}}%
\pgfpathcurveto{\pgfqpoint{2.122329in}{2.119721in}}{\pgfqpoint{2.130229in}{2.122993in}}{\pgfqpoint{2.136053in}{2.128817in}}%
\pgfpathcurveto{\pgfqpoint{2.141877in}{2.134641in}}{\pgfqpoint{2.145149in}{2.142541in}}{\pgfqpoint{2.145149in}{2.150778in}}%
\pgfpathcurveto{\pgfqpoint{2.145149in}{2.159014in}}{\pgfqpoint{2.141877in}{2.166914in}}{\pgfqpoint{2.136053in}{2.172738in}}%
\pgfpathcurveto{\pgfqpoint{2.130229in}{2.178562in}}{\pgfqpoint{2.122329in}{2.181834in}}{\pgfqpoint{2.114093in}{2.181834in}}%
\pgfpathcurveto{\pgfqpoint{2.105856in}{2.181834in}}{\pgfqpoint{2.097956in}{2.178562in}}{\pgfqpoint{2.092132in}{2.172738in}}%
\pgfpathcurveto{\pgfqpoint{2.086309in}{2.166914in}}{\pgfqpoint{2.083036in}{2.159014in}}{\pgfqpoint{2.083036in}{2.150778in}}%
\pgfpathcurveto{\pgfqpoint{2.083036in}{2.142541in}}{\pgfqpoint{2.086309in}{2.134641in}}{\pgfqpoint{2.092132in}{2.128817in}}%
\pgfpathcurveto{\pgfqpoint{2.097956in}{2.122993in}}{\pgfqpoint{2.105856in}{2.119721in}}{\pgfqpoint{2.114093in}{2.119721in}}%
\pgfpathclose%
\pgfusepath{stroke,fill}%
\end{pgfscope}%
\begin{pgfscope}%
\pgfpathrectangle{\pgfqpoint{0.100000in}{0.212622in}}{\pgfqpoint{3.696000in}{3.696000in}}%
\pgfusepath{clip}%
\pgfsetbuttcap%
\pgfsetroundjoin%
\definecolor{currentfill}{rgb}{0.121569,0.466667,0.705882}%
\pgfsetfillcolor{currentfill}%
\pgfsetfillopacity{0.382266}%
\pgfsetlinewidth{1.003750pt}%
\definecolor{currentstroke}{rgb}{0.121569,0.466667,0.705882}%
\pgfsetstrokecolor{currentstroke}%
\pgfsetstrokeopacity{0.382266}%
\pgfsetdash{}{0pt}%
\pgfpathmoveto{\pgfqpoint{1.465407in}{1.940710in}}%
\pgfpathcurveto{\pgfqpoint{1.473644in}{1.940710in}}{\pgfqpoint{1.481544in}{1.943982in}}{\pgfqpoint{1.487368in}{1.949806in}}%
\pgfpathcurveto{\pgfqpoint{1.493192in}{1.955630in}}{\pgfqpoint{1.496464in}{1.963530in}}{\pgfqpoint{1.496464in}{1.971767in}}%
\pgfpathcurveto{\pgfqpoint{1.496464in}{1.980003in}}{\pgfqpoint{1.493192in}{1.987903in}}{\pgfqpoint{1.487368in}{1.993727in}}%
\pgfpathcurveto{\pgfqpoint{1.481544in}{1.999551in}}{\pgfqpoint{1.473644in}{2.002823in}}{\pgfqpoint{1.465407in}{2.002823in}}%
\pgfpathcurveto{\pgfqpoint{1.457171in}{2.002823in}}{\pgfqpoint{1.449271in}{1.999551in}}{\pgfqpoint{1.443447in}{1.993727in}}%
\pgfpathcurveto{\pgfqpoint{1.437623in}{1.987903in}}{\pgfqpoint{1.434351in}{1.980003in}}{\pgfqpoint{1.434351in}{1.971767in}}%
\pgfpathcurveto{\pgfqpoint{1.434351in}{1.963530in}}{\pgfqpoint{1.437623in}{1.955630in}}{\pgfqpoint{1.443447in}{1.949806in}}%
\pgfpathcurveto{\pgfqpoint{1.449271in}{1.943982in}}{\pgfqpoint{1.457171in}{1.940710in}}{\pgfqpoint{1.465407in}{1.940710in}}%
\pgfpathclose%
\pgfusepath{stroke,fill}%
\end{pgfscope}%
\begin{pgfscope}%
\pgfpathrectangle{\pgfqpoint{0.100000in}{0.212622in}}{\pgfqpoint{3.696000in}{3.696000in}}%
\pgfusepath{clip}%
\pgfsetbuttcap%
\pgfsetroundjoin%
\definecolor{currentfill}{rgb}{0.121569,0.466667,0.705882}%
\pgfsetfillcolor{currentfill}%
\pgfsetfillopacity{0.382862}%
\pgfsetlinewidth{1.003750pt}%
\definecolor{currentstroke}{rgb}{0.121569,0.466667,0.705882}%
\pgfsetstrokecolor{currentstroke}%
\pgfsetstrokeopacity{0.382862}%
\pgfsetdash{}{0pt}%
\pgfpathmoveto{\pgfqpoint{1.462869in}{1.938008in}}%
\pgfpathcurveto{\pgfqpoint{1.471106in}{1.938008in}}{\pgfqpoint{1.479006in}{1.941280in}}{\pgfqpoint{1.484830in}{1.947104in}}%
\pgfpathcurveto{\pgfqpoint{1.490654in}{1.952928in}}{\pgfqpoint{1.493926in}{1.960828in}}{\pgfqpoint{1.493926in}{1.969064in}}%
\pgfpathcurveto{\pgfqpoint{1.493926in}{1.977300in}}{\pgfqpoint{1.490654in}{1.985200in}}{\pgfqpoint{1.484830in}{1.991024in}}%
\pgfpathcurveto{\pgfqpoint{1.479006in}{1.996848in}}{\pgfqpoint{1.471106in}{2.000121in}}{\pgfqpoint{1.462869in}{2.000121in}}%
\pgfpathcurveto{\pgfqpoint{1.454633in}{2.000121in}}{\pgfqpoint{1.446733in}{1.996848in}}{\pgfqpoint{1.440909in}{1.991024in}}%
\pgfpathcurveto{\pgfqpoint{1.435085in}{1.985200in}}{\pgfqpoint{1.431813in}{1.977300in}}{\pgfqpoint{1.431813in}{1.969064in}}%
\pgfpathcurveto{\pgfqpoint{1.431813in}{1.960828in}}{\pgfqpoint{1.435085in}{1.952928in}}{\pgfqpoint{1.440909in}{1.947104in}}%
\pgfpathcurveto{\pgfqpoint{1.446733in}{1.941280in}}{\pgfqpoint{1.454633in}{1.938008in}}{\pgfqpoint{1.462869in}{1.938008in}}%
\pgfpathclose%
\pgfusepath{stroke,fill}%
\end{pgfscope}%
\begin{pgfscope}%
\pgfpathrectangle{\pgfqpoint{0.100000in}{0.212622in}}{\pgfqpoint{3.696000in}{3.696000in}}%
\pgfusepath{clip}%
\pgfsetbuttcap%
\pgfsetroundjoin%
\definecolor{currentfill}{rgb}{0.121569,0.466667,0.705882}%
\pgfsetfillcolor{currentfill}%
\pgfsetfillopacity{0.383242}%
\pgfsetlinewidth{1.003750pt}%
\definecolor{currentstroke}{rgb}{0.121569,0.466667,0.705882}%
\pgfsetstrokecolor{currentstroke}%
\pgfsetstrokeopacity{0.383242}%
\pgfsetdash{}{0pt}%
\pgfpathmoveto{\pgfqpoint{1.461755in}{1.936836in}}%
\pgfpathcurveto{\pgfqpoint{1.469991in}{1.936836in}}{\pgfqpoint{1.477891in}{1.940109in}}{\pgfqpoint{1.483715in}{1.945933in}}%
\pgfpathcurveto{\pgfqpoint{1.489539in}{1.951756in}}{\pgfqpoint{1.492812in}{1.959657in}}{\pgfqpoint{1.492812in}{1.967893in}}%
\pgfpathcurveto{\pgfqpoint{1.492812in}{1.976129in}}{\pgfqpoint{1.489539in}{1.984029in}}{\pgfqpoint{1.483715in}{1.989853in}}%
\pgfpathcurveto{\pgfqpoint{1.477891in}{1.995677in}}{\pgfqpoint{1.469991in}{1.998949in}}{\pgfqpoint{1.461755in}{1.998949in}}%
\pgfpathcurveto{\pgfqpoint{1.453519in}{1.998949in}}{\pgfqpoint{1.445619in}{1.995677in}}{\pgfqpoint{1.439795in}{1.989853in}}%
\pgfpathcurveto{\pgfqpoint{1.433971in}{1.984029in}}{\pgfqpoint{1.430699in}{1.976129in}}{\pgfqpoint{1.430699in}{1.967893in}}%
\pgfpathcurveto{\pgfqpoint{1.430699in}{1.959657in}}{\pgfqpoint{1.433971in}{1.951756in}}{\pgfqpoint{1.439795in}{1.945933in}}%
\pgfpathcurveto{\pgfqpoint{1.445619in}{1.940109in}}{\pgfqpoint{1.453519in}{1.936836in}}{\pgfqpoint{1.461755in}{1.936836in}}%
\pgfpathclose%
\pgfusepath{stroke,fill}%
\end{pgfscope}%
\begin{pgfscope}%
\pgfpathrectangle{\pgfqpoint{0.100000in}{0.212622in}}{\pgfqpoint{3.696000in}{3.696000in}}%
\pgfusepath{clip}%
\pgfsetbuttcap%
\pgfsetroundjoin%
\definecolor{currentfill}{rgb}{0.121569,0.466667,0.705882}%
\pgfsetfillcolor{currentfill}%
\pgfsetfillopacity{0.384175}%
\pgfsetlinewidth{1.003750pt}%
\definecolor{currentstroke}{rgb}{0.121569,0.466667,0.705882}%
\pgfsetstrokecolor{currentstroke}%
\pgfsetstrokeopacity{0.384175}%
\pgfsetdash{}{0pt}%
\pgfpathmoveto{\pgfqpoint{1.460619in}{1.935787in}}%
\pgfpathcurveto{\pgfqpoint{1.468855in}{1.935787in}}{\pgfqpoint{1.476755in}{1.939059in}}{\pgfqpoint{1.482579in}{1.944883in}}%
\pgfpathcurveto{\pgfqpoint{1.488403in}{1.950707in}}{\pgfqpoint{1.491675in}{1.958607in}}{\pgfqpoint{1.491675in}{1.966843in}}%
\pgfpathcurveto{\pgfqpoint{1.491675in}{1.975080in}}{\pgfqpoint{1.488403in}{1.982980in}}{\pgfqpoint{1.482579in}{1.988804in}}%
\pgfpathcurveto{\pgfqpoint{1.476755in}{1.994627in}}{\pgfqpoint{1.468855in}{1.997900in}}{\pgfqpoint{1.460619in}{1.997900in}}%
\pgfpathcurveto{\pgfqpoint{1.452382in}{1.997900in}}{\pgfqpoint{1.444482in}{1.994627in}}{\pgfqpoint{1.438658in}{1.988804in}}%
\pgfpathcurveto{\pgfqpoint{1.432834in}{1.982980in}}{\pgfqpoint{1.429562in}{1.975080in}}{\pgfqpoint{1.429562in}{1.966843in}}%
\pgfpathcurveto{\pgfqpoint{1.429562in}{1.958607in}}{\pgfqpoint{1.432834in}{1.950707in}}{\pgfqpoint{1.438658in}{1.944883in}}%
\pgfpathcurveto{\pgfqpoint{1.444482in}{1.939059in}}{\pgfqpoint{1.452382in}{1.935787in}}{\pgfqpoint{1.460619in}{1.935787in}}%
\pgfpathclose%
\pgfusepath{stroke,fill}%
\end{pgfscope}%
\begin{pgfscope}%
\pgfpathrectangle{\pgfqpoint{0.100000in}{0.212622in}}{\pgfqpoint{3.696000in}{3.696000in}}%
\pgfusepath{clip}%
\pgfsetbuttcap%
\pgfsetroundjoin%
\definecolor{currentfill}{rgb}{0.121569,0.466667,0.705882}%
\pgfsetfillcolor{currentfill}%
\pgfsetfillopacity{0.384260}%
\pgfsetlinewidth{1.003750pt}%
\definecolor{currentstroke}{rgb}{0.121569,0.466667,0.705882}%
\pgfsetstrokecolor{currentstroke}%
\pgfsetstrokeopacity{0.384260}%
\pgfsetdash{}{0pt}%
\pgfpathmoveto{\pgfqpoint{2.124563in}{2.120836in}}%
\pgfpathcurveto{\pgfqpoint{2.132799in}{2.120836in}}{\pgfqpoint{2.140699in}{2.124108in}}{\pgfqpoint{2.146523in}{2.129932in}}%
\pgfpathcurveto{\pgfqpoint{2.152347in}{2.135756in}}{\pgfqpoint{2.155620in}{2.143656in}}{\pgfqpoint{2.155620in}{2.151892in}}%
\pgfpathcurveto{\pgfqpoint{2.155620in}{2.160128in}}{\pgfqpoint{2.152347in}{2.168029in}}{\pgfqpoint{2.146523in}{2.173852in}}%
\pgfpathcurveto{\pgfqpoint{2.140699in}{2.179676in}}{\pgfqpoint{2.132799in}{2.182949in}}{\pgfqpoint{2.124563in}{2.182949in}}%
\pgfpathcurveto{\pgfqpoint{2.116327in}{2.182949in}}{\pgfqpoint{2.108427in}{2.179676in}}{\pgfqpoint{2.102603in}{2.173852in}}%
\pgfpathcurveto{\pgfqpoint{2.096779in}{2.168029in}}{\pgfqpoint{2.093507in}{2.160128in}}{\pgfqpoint{2.093507in}{2.151892in}}%
\pgfpathcurveto{\pgfqpoint{2.093507in}{2.143656in}}{\pgfqpoint{2.096779in}{2.135756in}}{\pgfqpoint{2.102603in}{2.129932in}}%
\pgfpathcurveto{\pgfqpoint{2.108427in}{2.124108in}}{\pgfqpoint{2.116327in}{2.120836in}}{\pgfqpoint{2.124563in}{2.120836in}}%
\pgfpathclose%
\pgfusepath{stroke,fill}%
\end{pgfscope}%
\begin{pgfscope}%
\pgfpathrectangle{\pgfqpoint{0.100000in}{0.212622in}}{\pgfqpoint{3.696000in}{3.696000in}}%
\pgfusepath{clip}%
\pgfsetbuttcap%
\pgfsetroundjoin%
\definecolor{currentfill}{rgb}{0.121569,0.466667,0.705882}%
\pgfsetfillcolor{currentfill}%
\pgfsetfillopacity{0.384608}%
\pgfsetlinewidth{1.003750pt}%
\definecolor{currentstroke}{rgb}{0.121569,0.466667,0.705882}%
\pgfsetstrokecolor{currentstroke}%
\pgfsetstrokeopacity{0.384608}%
\pgfsetdash{}{0pt}%
\pgfpathmoveto{\pgfqpoint{1.458837in}{1.933954in}}%
\pgfpathcurveto{\pgfqpoint{1.467073in}{1.933954in}}{\pgfqpoint{1.474973in}{1.937227in}}{\pgfqpoint{1.480797in}{1.943051in}}%
\pgfpathcurveto{\pgfqpoint{1.486621in}{1.948874in}}{\pgfqpoint{1.489894in}{1.956774in}}{\pgfqpoint{1.489894in}{1.965011in}}%
\pgfpathcurveto{\pgfqpoint{1.489894in}{1.973247in}}{\pgfqpoint{1.486621in}{1.981147in}}{\pgfqpoint{1.480797in}{1.986971in}}%
\pgfpathcurveto{\pgfqpoint{1.474973in}{1.992795in}}{\pgfqpoint{1.467073in}{1.996067in}}{\pgfqpoint{1.458837in}{1.996067in}}%
\pgfpathcurveto{\pgfqpoint{1.450601in}{1.996067in}}{\pgfqpoint{1.442701in}{1.992795in}}{\pgfqpoint{1.436877in}{1.986971in}}%
\pgfpathcurveto{\pgfqpoint{1.431053in}{1.981147in}}{\pgfqpoint{1.427781in}{1.973247in}}{\pgfqpoint{1.427781in}{1.965011in}}%
\pgfpathcurveto{\pgfqpoint{1.427781in}{1.956774in}}{\pgfqpoint{1.431053in}{1.948874in}}{\pgfqpoint{1.436877in}{1.943051in}}%
\pgfpathcurveto{\pgfqpoint{1.442701in}{1.937227in}}{\pgfqpoint{1.450601in}{1.933954in}}{\pgfqpoint{1.458837in}{1.933954in}}%
\pgfpathclose%
\pgfusepath{stroke,fill}%
\end{pgfscope}%
\begin{pgfscope}%
\pgfpathrectangle{\pgfqpoint{0.100000in}{0.212622in}}{\pgfqpoint{3.696000in}{3.696000in}}%
\pgfusepath{clip}%
\pgfsetbuttcap%
\pgfsetroundjoin%
\definecolor{currentfill}{rgb}{0.121569,0.466667,0.705882}%
\pgfsetfillcolor{currentfill}%
\pgfsetfillopacity{0.385027}%
\pgfsetlinewidth{1.003750pt}%
\definecolor{currentstroke}{rgb}{0.121569,0.466667,0.705882}%
\pgfsetstrokecolor{currentstroke}%
\pgfsetstrokeopacity{0.385027}%
\pgfsetdash{}{0pt}%
\pgfpathmoveto{\pgfqpoint{1.458008in}{1.933178in}}%
\pgfpathcurveto{\pgfqpoint{1.466244in}{1.933178in}}{\pgfqpoint{1.474144in}{1.936450in}}{\pgfqpoint{1.479968in}{1.942274in}}%
\pgfpathcurveto{\pgfqpoint{1.485792in}{1.948098in}}{\pgfqpoint{1.489065in}{1.955998in}}{\pgfqpoint{1.489065in}{1.964234in}}%
\pgfpathcurveto{\pgfqpoint{1.489065in}{1.972470in}}{\pgfqpoint{1.485792in}{1.980370in}}{\pgfqpoint{1.479968in}{1.986194in}}%
\pgfpathcurveto{\pgfqpoint{1.474144in}{1.992018in}}{\pgfqpoint{1.466244in}{1.995291in}}{\pgfqpoint{1.458008in}{1.995291in}}%
\pgfpathcurveto{\pgfqpoint{1.449772in}{1.995291in}}{\pgfqpoint{1.441872in}{1.992018in}}{\pgfqpoint{1.436048in}{1.986194in}}%
\pgfpathcurveto{\pgfqpoint{1.430224in}{1.980370in}}{\pgfqpoint{1.426952in}{1.972470in}}{\pgfqpoint{1.426952in}{1.964234in}}%
\pgfpathcurveto{\pgfqpoint{1.426952in}{1.955998in}}{\pgfqpoint{1.430224in}{1.948098in}}{\pgfqpoint{1.436048in}{1.942274in}}%
\pgfpathcurveto{\pgfqpoint{1.441872in}{1.936450in}}{\pgfqpoint{1.449772in}{1.933178in}}{\pgfqpoint{1.458008in}{1.933178in}}%
\pgfpathclose%
\pgfusepath{stroke,fill}%
\end{pgfscope}%
\begin{pgfscope}%
\pgfpathrectangle{\pgfqpoint{0.100000in}{0.212622in}}{\pgfqpoint{3.696000in}{3.696000in}}%
\pgfusepath{clip}%
\pgfsetbuttcap%
\pgfsetroundjoin%
\definecolor{currentfill}{rgb}{0.121569,0.466667,0.705882}%
\pgfsetfillcolor{currentfill}%
\pgfsetfillopacity{0.385616}%
\pgfsetlinewidth{1.003750pt}%
\definecolor{currentstroke}{rgb}{0.121569,0.466667,0.705882}%
\pgfsetstrokecolor{currentstroke}%
\pgfsetstrokeopacity{0.385616}%
\pgfsetdash{}{0pt}%
\pgfpathmoveto{\pgfqpoint{1.456735in}{1.930313in}}%
\pgfpathcurveto{\pgfqpoint{1.464972in}{1.930313in}}{\pgfqpoint{1.472872in}{1.933585in}}{\pgfqpoint{1.478696in}{1.939409in}}%
\pgfpathcurveto{\pgfqpoint{1.484520in}{1.945233in}}{\pgfqpoint{1.487792in}{1.953133in}}{\pgfqpoint{1.487792in}{1.961369in}}%
\pgfpathcurveto{\pgfqpoint{1.487792in}{1.969606in}}{\pgfqpoint{1.484520in}{1.977506in}}{\pgfqpoint{1.478696in}{1.983330in}}%
\pgfpathcurveto{\pgfqpoint{1.472872in}{1.989154in}}{\pgfqpoint{1.464972in}{1.992426in}}{\pgfqpoint{1.456735in}{1.992426in}}%
\pgfpathcurveto{\pgfqpoint{1.448499in}{1.992426in}}{\pgfqpoint{1.440599in}{1.989154in}}{\pgfqpoint{1.434775in}{1.983330in}}%
\pgfpathcurveto{\pgfqpoint{1.428951in}{1.977506in}}{\pgfqpoint{1.425679in}{1.969606in}}{\pgfqpoint{1.425679in}{1.961369in}}%
\pgfpathcurveto{\pgfqpoint{1.425679in}{1.953133in}}{\pgfqpoint{1.428951in}{1.945233in}}{\pgfqpoint{1.434775in}{1.939409in}}%
\pgfpathcurveto{\pgfqpoint{1.440599in}{1.933585in}}{\pgfqpoint{1.448499in}{1.930313in}}{\pgfqpoint{1.456735in}{1.930313in}}%
\pgfpathclose%
\pgfusepath{stroke,fill}%
\end{pgfscope}%
\begin{pgfscope}%
\pgfpathrectangle{\pgfqpoint{0.100000in}{0.212622in}}{\pgfqpoint{3.696000in}{3.696000in}}%
\pgfusepath{clip}%
\pgfsetbuttcap%
\pgfsetroundjoin%
\definecolor{currentfill}{rgb}{0.121569,0.466667,0.705882}%
\pgfsetfillcolor{currentfill}%
\pgfsetfillopacity{0.386034}%
\pgfsetlinewidth{1.003750pt}%
\definecolor{currentstroke}{rgb}{0.121569,0.466667,0.705882}%
\pgfsetstrokecolor{currentstroke}%
\pgfsetstrokeopacity{0.386034}%
\pgfsetdash{}{0pt}%
\pgfpathmoveto{\pgfqpoint{2.135906in}{2.116465in}}%
\pgfpathcurveto{\pgfqpoint{2.144143in}{2.116465in}}{\pgfqpoint{2.152043in}{2.119737in}}{\pgfqpoint{2.157866in}{2.125561in}}%
\pgfpathcurveto{\pgfqpoint{2.163690in}{2.131385in}}{\pgfqpoint{2.166963in}{2.139285in}}{\pgfqpoint{2.166963in}{2.147522in}}%
\pgfpathcurveto{\pgfqpoint{2.166963in}{2.155758in}}{\pgfqpoint{2.163690in}{2.163658in}}{\pgfqpoint{2.157866in}{2.169482in}}%
\pgfpathcurveto{\pgfqpoint{2.152043in}{2.175306in}}{\pgfqpoint{2.144143in}{2.178578in}}{\pgfqpoint{2.135906in}{2.178578in}}%
\pgfpathcurveto{\pgfqpoint{2.127670in}{2.178578in}}{\pgfqpoint{2.119770in}{2.175306in}}{\pgfqpoint{2.113946in}{2.169482in}}%
\pgfpathcurveto{\pgfqpoint{2.108122in}{2.163658in}}{\pgfqpoint{2.104850in}{2.155758in}}{\pgfqpoint{2.104850in}{2.147522in}}%
\pgfpathcurveto{\pgfqpoint{2.104850in}{2.139285in}}{\pgfqpoint{2.108122in}{2.131385in}}{\pgfqpoint{2.113946in}{2.125561in}}%
\pgfpathcurveto{\pgfqpoint{2.119770in}{2.119737in}}{\pgfqpoint{2.127670in}{2.116465in}}{\pgfqpoint{2.135906in}{2.116465in}}%
\pgfpathclose%
\pgfusepath{stroke,fill}%
\end{pgfscope}%
\begin{pgfscope}%
\pgfpathrectangle{\pgfqpoint{0.100000in}{0.212622in}}{\pgfqpoint{3.696000in}{3.696000in}}%
\pgfusepath{clip}%
\pgfsetbuttcap%
\pgfsetroundjoin%
\definecolor{currentfill}{rgb}{0.121569,0.466667,0.705882}%
\pgfsetfillcolor{currentfill}%
\pgfsetfillopacity{0.386236}%
\pgfsetlinewidth{1.003750pt}%
\definecolor{currentstroke}{rgb}{0.121569,0.466667,0.705882}%
\pgfsetstrokecolor{currentstroke}%
\pgfsetstrokeopacity{0.386236}%
\pgfsetdash{}{0pt}%
\pgfpathmoveto{\pgfqpoint{1.454596in}{1.929401in}}%
\pgfpathcurveto{\pgfqpoint{1.462833in}{1.929401in}}{\pgfqpoint{1.470733in}{1.932674in}}{\pgfqpoint{1.476557in}{1.938498in}}%
\pgfpathcurveto{\pgfqpoint{1.482381in}{1.944322in}}{\pgfqpoint{1.485653in}{1.952222in}}{\pgfqpoint{1.485653in}{1.960458in}}%
\pgfpathcurveto{\pgfqpoint{1.485653in}{1.968694in}}{\pgfqpoint{1.482381in}{1.976594in}}{\pgfqpoint{1.476557in}{1.982418in}}%
\pgfpathcurveto{\pgfqpoint{1.470733in}{1.988242in}}{\pgfqpoint{1.462833in}{1.991514in}}{\pgfqpoint{1.454596in}{1.991514in}}%
\pgfpathcurveto{\pgfqpoint{1.446360in}{1.991514in}}{\pgfqpoint{1.438460in}{1.988242in}}{\pgfqpoint{1.432636in}{1.982418in}}%
\pgfpathcurveto{\pgfqpoint{1.426812in}{1.976594in}}{\pgfqpoint{1.423540in}{1.968694in}}{\pgfqpoint{1.423540in}{1.960458in}}%
\pgfpathcurveto{\pgfqpoint{1.423540in}{1.952222in}}{\pgfqpoint{1.426812in}{1.944322in}}{\pgfqpoint{1.432636in}{1.938498in}}%
\pgfpathcurveto{\pgfqpoint{1.438460in}{1.932674in}}{\pgfqpoint{1.446360in}{1.929401in}}{\pgfqpoint{1.454596in}{1.929401in}}%
\pgfpathclose%
\pgfusepath{stroke,fill}%
\end{pgfscope}%
\begin{pgfscope}%
\pgfpathrectangle{\pgfqpoint{0.100000in}{0.212622in}}{\pgfqpoint{3.696000in}{3.696000in}}%
\pgfusepath{clip}%
\pgfsetbuttcap%
\pgfsetroundjoin%
\definecolor{currentfill}{rgb}{0.121569,0.466667,0.705882}%
\pgfsetfillcolor{currentfill}%
\pgfsetfillopacity{0.386578}%
\pgfsetlinewidth{1.003750pt}%
\definecolor{currentstroke}{rgb}{0.121569,0.466667,0.705882}%
\pgfsetstrokecolor{currentstroke}%
\pgfsetstrokeopacity{0.386578}%
\pgfsetdash{}{0pt}%
\pgfpathmoveto{\pgfqpoint{1.454162in}{1.929007in}}%
\pgfpathcurveto{\pgfqpoint{1.462399in}{1.929007in}}{\pgfqpoint{1.470299in}{1.932279in}}{\pgfqpoint{1.476122in}{1.938103in}}%
\pgfpathcurveto{\pgfqpoint{1.481946in}{1.943927in}}{\pgfqpoint{1.485219in}{1.951827in}}{\pgfqpoint{1.485219in}{1.960064in}}%
\pgfpathcurveto{\pgfqpoint{1.485219in}{1.968300in}}{\pgfqpoint{1.481946in}{1.976200in}}{\pgfqpoint{1.476122in}{1.982024in}}%
\pgfpathcurveto{\pgfqpoint{1.470299in}{1.987848in}}{\pgfqpoint{1.462399in}{1.991120in}}{\pgfqpoint{1.454162in}{1.991120in}}%
\pgfpathcurveto{\pgfqpoint{1.445926in}{1.991120in}}{\pgfqpoint{1.438026in}{1.987848in}}{\pgfqpoint{1.432202in}{1.982024in}}%
\pgfpathcurveto{\pgfqpoint{1.426378in}{1.976200in}}{\pgfqpoint{1.423106in}{1.968300in}}{\pgfqpoint{1.423106in}{1.960064in}}%
\pgfpathcurveto{\pgfqpoint{1.423106in}{1.951827in}}{\pgfqpoint{1.426378in}{1.943927in}}{\pgfqpoint{1.432202in}{1.938103in}}%
\pgfpathcurveto{\pgfqpoint{1.438026in}{1.932279in}}{\pgfqpoint{1.445926in}{1.929007in}}{\pgfqpoint{1.454162in}{1.929007in}}%
\pgfpathclose%
\pgfusepath{stroke,fill}%
\end{pgfscope}%
\begin{pgfscope}%
\pgfpathrectangle{\pgfqpoint{0.100000in}{0.212622in}}{\pgfqpoint{3.696000in}{3.696000in}}%
\pgfusepath{clip}%
\pgfsetbuttcap%
\pgfsetroundjoin%
\definecolor{currentfill}{rgb}{0.121569,0.466667,0.705882}%
\pgfsetfillcolor{currentfill}%
\pgfsetfillopacity{0.387089}%
\pgfsetlinewidth{1.003750pt}%
\definecolor{currentstroke}{rgb}{0.121569,0.466667,0.705882}%
\pgfsetstrokecolor{currentstroke}%
\pgfsetstrokeopacity{0.387089}%
\pgfsetdash{}{0pt}%
\pgfpathmoveto{\pgfqpoint{1.453112in}{1.927648in}}%
\pgfpathcurveto{\pgfqpoint{1.461348in}{1.927648in}}{\pgfqpoint{1.469248in}{1.930921in}}{\pgfqpoint{1.475072in}{1.936745in}}%
\pgfpathcurveto{\pgfqpoint{1.480896in}{1.942569in}}{\pgfqpoint{1.484168in}{1.950469in}}{\pgfqpoint{1.484168in}{1.958705in}}%
\pgfpathcurveto{\pgfqpoint{1.484168in}{1.966941in}}{\pgfqpoint{1.480896in}{1.974841in}}{\pgfqpoint{1.475072in}{1.980665in}}%
\pgfpathcurveto{\pgfqpoint{1.469248in}{1.986489in}}{\pgfqpoint{1.461348in}{1.989761in}}{\pgfqpoint{1.453112in}{1.989761in}}%
\pgfpathcurveto{\pgfqpoint{1.444875in}{1.989761in}}{\pgfqpoint{1.436975in}{1.986489in}}{\pgfqpoint{1.431151in}{1.980665in}}%
\pgfpathcurveto{\pgfqpoint{1.425328in}{1.974841in}}{\pgfqpoint{1.422055in}{1.966941in}}{\pgfqpoint{1.422055in}{1.958705in}}%
\pgfpathcurveto{\pgfqpoint{1.422055in}{1.950469in}}{\pgfqpoint{1.425328in}{1.942569in}}{\pgfqpoint{1.431151in}{1.936745in}}%
\pgfpathcurveto{\pgfqpoint{1.436975in}{1.930921in}}{\pgfqpoint{1.444875in}{1.927648in}}{\pgfqpoint{1.453112in}{1.927648in}}%
\pgfpathclose%
\pgfusepath{stroke,fill}%
\end{pgfscope}%
\begin{pgfscope}%
\pgfpathrectangle{\pgfqpoint{0.100000in}{0.212622in}}{\pgfqpoint{3.696000in}{3.696000in}}%
\pgfusepath{clip}%
\pgfsetbuttcap%
\pgfsetroundjoin%
\definecolor{currentfill}{rgb}{0.121569,0.466667,0.705882}%
\pgfsetfillcolor{currentfill}%
\pgfsetfillopacity{0.388533}%
\pgfsetlinewidth{1.003750pt}%
\definecolor{currentstroke}{rgb}{0.121569,0.466667,0.705882}%
\pgfsetstrokecolor{currentstroke}%
\pgfsetstrokeopacity{0.388533}%
\pgfsetdash{}{0pt}%
\pgfpathmoveto{\pgfqpoint{2.147619in}{2.116869in}}%
\pgfpathcurveto{\pgfqpoint{2.155855in}{2.116869in}}{\pgfqpoint{2.163755in}{2.120141in}}{\pgfqpoint{2.169579in}{2.125965in}}%
\pgfpathcurveto{\pgfqpoint{2.175403in}{2.131789in}}{\pgfqpoint{2.178676in}{2.139689in}}{\pgfqpoint{2.178676in}{2.147926in}}%
\pgfpathcurveto{\pgfqpoint{2.178676in}{2.156162in}}{\pgfqpoint{2.175403in}{2.164062in}}{\pgfqpoint{2.169579in}{2.169886in}}%
\pgfpathcurveto{\pgfqpoint{2.163755in}{2.175710in}}{\pgfqpoint{2.155855in}{2.178982in}}{\pgfqpoint{2.147619in}{2.178982in}}%
\pgfpathcurveto{\pgfqpoint{2.139383in}{2.178982in}}{\pgfqpoint{2.131483in}{2.175710in}}{\pgfqpoint{2.125659in}{2.169886in}}%
\pgfpathcurveto{\pgfqpoint{2.119835in}{2.164062in}}{\pgfqpoint{2.116563in}{2.156162in}}{\pgfqpoint{2.116563in}{2.147926in}}%
\pgfpathcurveto{\pgfqpoint{2.116563in}{2.139689in}}{\pgfqpoint{2.119835in}{2.131789in}}{\pgfqpoint{2.125659in}{2.125965in}}%
\pgfpathcurveto{\pgfqpoint{2.131483in}{2.120141in}}{\pgfqpoint{2.139383in}{2.116869in}}{\pgfqpoint{2.147619in}{2.116869in}}%
\pgfpathclose%
\pgfusepath{stroke,fill}%
\end{pgfscope}%
\begin{pgfscope}%
\pgfpathrectangle{\pgfqpoint{0.100000in}{0.212622in}}{\pgfqpoint{3.696000in}{3.696000in}}%
\pgfusepath{clip}%
\pgfsetbuttcap%
\pgfsetroundjoin%
\definecolor{currentfill}{rgb}{0.121569,0.466667,0.705882}%
\pgfsetfillcolor{currentfill}%
\pgfsetfillopacity{0.389505}%
\pgfsetlinewidth{1.003750pt}%
\definecolor{currentstroke}{rgb}{0.121569,0.466667,0.705882}%
\pgfsetstrokecolor{currentstroke}%
\pgfsetstrokeopacity{0.389505}%
\pgfsetdash{}{0pt}%
\pgfpathmoveto{\pgfqpoint{1.450879in}{1.936278in}}%
\pgfpathcurveto{\pgfqpoint{1.459116in}{1.936278in}}{\pgfqpoint{1.467016in}{1.939550in}}{\pgfqpoint{1.472840in}{1.945374in}}%
\pgfpathcurveto{\pgfqpoint{1.478664in}{1.951198in}}{\pgfqpoint{1.481936in}{1.959098in}}{\pgfqpoint{1.481936in}{1.967334in}}%
\pgfpathcurveto{\pgfqpoint{1.481936in}{1.975570in}}{\pgfqpoint{1.478664in}{1.983471in}}{\pgfqpoint{1.472840in}{1.989294in}}%
\pgfpathcurveto{\pgfqpoint{1.467016in}{1.995118in}}{\pgfqpoint{1.459116in}{1.998391in}}{\pgfqpoint{1.450879in}{1.998391in}}%
\pgfpathcurveto{\pgfqpoint{1.442643in}{1.998391in}}{\pgfqpoint{1.434743in}{1.995118in}}{\pgfqpoint{1.428919in}{1.989294in}}%
\pgfpathcurveto{\pgfqpoint{1.423095in}{1.983471in}}{\pgfqpoint{1.419823in}{1.975570in}}{\pgfqpoint{1.419823in}{1.967334in}}%
\pgfpathcurveto{\pgfqpoint{1.419823in}{1.959098in}}{\pgfqpoint{1.423095in}{1.951198in}}{\pgfqpoint{1.428919in}{1.945374in}}%
\pgfpathcurveto{\pgfqpoint{1.434743in}{1.939550in}}{\pgfqpoint{1.442643in}{1.936278in}}{\pgfqpoint{1.450879in}{1.936278in}}%
\pgfpathclose%
\pgfusepath{stroke,fill}%
\end{pgfscope}%
\begin{pgfscope}%
\pgfpathrectangle{\pgfqpoint{0.100000in}{0.212622in}}{\pgfqpoint{3.696000in}{3.696000in}}%
\pgfusepath{clip}%
\pgfsetbuttcap%
\pgfsetroundjoin%
\definecolor{currentfill}{rgb}{0.121569,0.466667,0.705882}%
\pgfsetfillcolor{currentfill}%
\pgfsetfillopacity{0.390637}%
\pgfsetlinewidth{1.003750pt}%
\definecolor{currentstroke}{rgb}{0.121569,0.466667,0.705882}%
\pgfsetstrokecolor{currentstroke}%
\pgfsetstrokeopacity{0.390637}%
\pgfsetdash{}{0pt}%
\pgfpathmoveto{\pgfqpoint{1.449627in}{1.935504in}}%
\pgfpathcurveto{\pgfqpoint{1.457863in}{1.935504in}}{\pgfqpoint{1.465764in}{1.938776in}}{\pgfqpoint{1.471587in}{1.944600in}}%
\pgfpathcurveto{\pgfqpoint{1.477411in}{1.950424in}}{\pgfqpoint{1.480684in}{1.958324in}}{\pgfqpoint{1.480684in}{1.966561in}}%
\pgfpathcurveto{\pgfqpoint{1.480684in}{1.974797in}}{\pgfqpoint{1.477411in}{1.982697in}}{\pgfqpoint{1.471587in}{1.988521in}}%
\pgfpathcurveto{\pgfqpoint{1.465764in}{1.994345in}}{\pgfqpoint{1.457863in}{1.997617in}}{\pgfqpoint{1.449627in}{1.997617in}}%
\pgfpathcurveto{\pgfqpoint{1.441391in}{1.997617in}}{\pgfqpoint{1.433491in}{1.994345in}}{\pgfqpoint{1.427667in}{1.988521in}}%
\pgfpathcurveto{\pgfqpoint{1.421843in}{1.982697in}}{\pgfqpoint{1.418571in}{1.974797in}}{\pgfqpoint{1.418571in}{1.966561in}}%
\pgfpathcurveto{\pgfqpoint{1.418571in}{1.958324in}}{\pgfqpoint{1.421843in}{1.950424in}}{\pgfqpoint{1.427667in}{1.944600in}}%
\pgfpathcurveto{\pgfqpoint{1.433491in}{1.938776in}}{\pgfqpoint{1.441391in}{1.935504in}}{\pgfqpoint{1.449627in}{1.935504in}}%
\pgfpathclose%
\pgfusepath{stroke,fill}%
\end{pgfscope}%
\begin{pgfscope}%
\pgfpathrectangle{\pgfqpoint{0.100000in}{0.212622in}}{\pgfqpoint{3.696000in}{3.696000in}}%
\pgfusepath{clip}%
\pgfsetbuttcap%
\pgfsetroundjoin%
\definecolor{currentfill}{rgb}{0.121569,0.466667,0.705882}%
\pgfsetfillcolor{currentfill}%
\pgfsetfillopacity{0.390990}%
\pgfsetlinewidth{1.003750pt}%
\definecolor{currentstroke}{rgb}{0.121569,0.466667,0.705882}%
\pgfsetstrokecolor{currentstroke}%
\pgfsetstrokeopacity{0.390990}%
\pgfsetdash{}{0pt}%
\pgfpathmoveto{\pgfqpoint{1.448286in}{1.933190in}}%
\pgfpathcurveto{\pgfqpoint{1.456522in}{1.933190in}}{\pgfqpoint{1.464422in}{1.936463in}}{\pgfqpoint{1.470246in}{1.942286in}}%
\pgfpathcurveto{\pgfqpoint{1.476070in}{1.948110in}}{\pgfqpoint{1.479342in}{1.956010in}}{\pgfqpoint{1.479342in}{1.964247in}}%
\pgfpathcurveto{\pgfqpoint{1.479342in}{1.972483in}}{\pgfqpoint{1.476070in}{1.980383in}}{\pgfqpoint{1.470246in}{1.986207in}}%
\pgfpathcurveto{\pgfqpoint{1.464422in}{1.992031in}}{\pgfqpoint{1.456522in}{1.995303in}}{\pgfqpoint{1.448286in}{1.995303in}}%
\pgfpathcurveto{\pgfqpoint{1.440049in}{1.995303in}}{\pgfqpoint{1.432149in}{1.992031in}}{\pgfqpoint{1.426325in}{1.986207in}}%
\pgfpathcurveto{\pgfqpoint{1.420501in}{1.980383in}}{\pgfqpoint{1.417229in}{1.972483in}}{\pgfqpoint{1.417229in}{1.964247in}}%
\pgfpathcurveto{\pgfqpoint{1.417229in}{1.956010in}}{\pgfqpoint{1.420501in}{1.948110in}}{\pgfqpoint{1.426325in}{1.942286in}}%
\pgfpathcurveto{\pgfqpoint{1.432149in}{1.936463in}}{\pgfqpoint{1.440049in}{1.933190in}}{\pgfqpoint{1.448286in}{1.933190in}}%
\pgfpathclose%
\pgfusepath{stroke,fill}%
\end{pgfscope}%
\begin{pgfscope}%
\pgfpathrectangle{\pgfqpoint{0.100000in}{0.212622in}}{\pgfqpoint{3.696000in}{3.696000in}}%
\pgfusepath{clip}%
\pgfsetbuttcap%
\pgfsetroundjoin%
\definecolor{currentfill}{rgb}{0.121569,0.466667,0.705882}%
\pgfsetfillcolor{currentfill}%
\pgfsetfillopacity{0.391055}%
\pgfsetlinewidth{1.003750pt}%
\definecolor{currentstroke}{rgb}{0.121569,0.466667,0.705882}%
\pgfsetstrokecolor{currentstroke}%
\pgfsetstrokeopacity{0.391055}%
\pgfsetdash{}{0pt}%
\pgfpathmoveto{\pgfqpoint{1.448101in}{1.933083in}}%
\pgfpathcurveto{\pgfqpoint{1.456338in}{1.933083in}}{\pgfqpoint{1.464238in}{1.936356in}}{\pgfqpoint{1.470062in}{1.942180in}}%
\pgfpathcurveto{\pgfqpoint{1.475886in}{1.948004in}}{\pgfqpoint{1.479158in}{1.955904in}}{\pgfqpoint{1.479158in}{1.964140in}}%
\pgfpathcurveto{\pgfqpoint{1.479158in}{1.972376in}}{\pgfqpoint{1.475886in}{1.980276in}}{\pgfqpoint{1.470062in}{1.986100in}}%
\pgfpathcurveto{\pgfqpoint{1.464238in}{1.991924in}}{\pgfqpoint{1.456338in}{1.995196in}}{\pgfqpoint{1.448101in}{1.995196in}}%
\pgfpathcurveto{\pgfqpoint{1.439865in}{1.995196in}}{\pgfqpoint{1.431965in}{1.991924in}}{\pgfqpoint{1.426141in}{1.986100in}}%
\pgfpathcurveto{\pgfqpoint{1.420317in}{1.980276in}}{\pgfqpoint{1.417045in}{1.972376in}}{\pgfqpoint{1.417045in}{1.964140in}}%
\pgfpathcurveto{\pgfqpoint{1.417045in}{1.955904in}}{\pgfqpoint{1.420317in}{1.948004in}}{\pgfqpoint{1.426141in}{1.942180in}}%
\pgfpathcurveto{\pgfqpoint{1.431965in}{1.936356in}}{\pgfqpoint{1.439865in}{1.933083in}}{\pgfqpoint{1.448101in}{1.933083in}}%
\pgfpathclose%
\pgfusepath{stroke,fill}%
\end{pgfscope}%
\begin{pgfscope}%
\pgfpathrectangle{\pgfqpoint{0.100000in}{0.212622in}}{\pgfqpoint{3.696000in}{3.696000in}}%
\pgfusepath{clip}%
\pgfsetbuttcap%
\pgfsetroundjoin%
\definecolor{currentfill}{rgb}{0.121569,0.466667,0.705882}%
\pgfsetfillcolor{currentfill}%
\pgfsetfillopacity{0.391190}%
\pgfsetlinewidth{1.003750pt}%
\definecolor{currentstroke}{rgb}{0.121569,0.466667,0.705882}%
\pgfsetstrokecolor{currentstroke}%
\pgfsetstrokeopacity{0.391190}%
\pgfsetdash{}{0pt}%
\pgfpathmoveto{\pgfqpoint{1.447911in}{1.932888in}}%
\pgfpathcurveto{\pgfqpoint{1.456147in}{1.932888in}}{\pgfqpoint{1.464047in}{1.936161in}}{\pgfqpoint{1.469871in}{1.941985in}}%
\pgfpathcurveto{\pgfqpoint{1.475695in}{1.947809in}}{\pgfqpoint{1.478967in}{1.955709in}}{\pgfqpoint{1.478967in}{1.963945in}}%
\pgfpathcurveto{\pgfqpoint{1.478967in}{1.972181in}}{\pgfqpoint{1.475695in}{1.980081in}}{\pgfqpoint{1.469871in}{1.985905in}}%
\pgfpathcurveto{\pgfqpoint{1.464047in}{1.991729in}}{\pgfqpoint{1.456147in}{1.995001in}}{\pgfqpoint{1.447911in}{1.995001in}}%
\pgfpathcurveto{\pgfqpoint{1.439675in}{1.995001in}}{\pgfqpoint{1.431775in}{1.991729in}}{\pgfqpoint{1.425951in}{1.985905in}}%
\pgfpathcurveto{\pgfqpoint{1.420127in}{1.980081in}}{\pgfqpoint{1.416854in}{1.972181in}}{\pgfqpoint{1.416854in}{1.963945in}}%
\pgfpathcurveto{\pgfqpoint{1.416854in}{1.955709in}}{\pgfqpoint{1.420127in}{1.947809in}}{\pgfqpoint{1.425951in}{1.941985in}}%
\pgfpathcurveto{\pgfqpoint{1.431775in}{1.936161in}}{\pgfqpoint{1.439675in}{1.932888in}}{\pgfqpoint{1.447911in}{1.932888in}}%
\pgfpathclose%
\pgfusepath{stroke,fill}%
\end{pgfscope}%
\begin{pgfscope}%
\pgfpathrectangle{\pgfqpoint{0.100000in}{0.212622in}}{\pgfqpoint{3.696000in}{3.696000in}}%
\pgfusepath{clip}%
\pgfsetbuttcap%
\pgfsetroundjoin%
\definecolor{currentfill}{rgb}{0.121569,0.466667,0.705882}%
\pgfsetfillcolor{currentfill}%
\pgfsetfillopacity{0.391360}%
\pgfsetlinewidth{1.003750pt}%
\definecolor{currentstroke}{rgb}{0.121569,0.466667,0.705882}%
\pgfsetstrokecolor{currentstroke}%
\pgfsetstrokeopacity{0.391360}%
\pgfsetdash{}{0pt}%
\pgfpathmoveto{\pgfqpoint{1.447210in}{1.932343in}}%
\pgfpathcurveto{\pgfqpoint{1.455446in}{1.932343in}}{\pgfqpoint{1.463346in}{1.935615in}}{\pgfqpoint{1.469170in}{1.941439in}}%
\pgfpathcurveto{\pgfqpoint{1.474994in}{1.947263in}}{\pgfqpoint{1.478266in}{1.955163in}}{\pgfqpoint{1.478266in}{1.963400in}}%
\pgfpathcurveto{\pgfqpoint{1.478266in}{1.971636in}}{\pgfqpoint{1.474994in}{1.979536in}}{\pgfqpoint{1.469170in}{1.985360in}}%
\pgfpathcurveto{\pgfqpoint{1.463346in}{1.991184in}}{\pgfqpoint{1.455446in}{1.994456in}}{\pgfqpoint{1.447210in}{1.994456in}}%
\pgfpathcurveto{\pgfqpoint{1.438973in}{1.994456in}}{\pgfqpoint{1.431073in}{1.991184in}}{\pgfqpoint{1.425249in}{1.985360in}}%
\pgfpathcurveto{\pgfqpoint{1.419425in}{1.979536in}}{\pgfqpoint{1.416153in}{1.971636in}}{\pgfqpoint{1.416153in}{1.963400in}}%
\pgfpathcurveto{\pgfqpoint{1.416153in}{1.955163in}}{\pgfqpoint{1.419425in}{1.947263in}}{\pgfqpoint{1.425249in}{1.941439in}}%
\pgfpathcurveto{\pgfqpoint{1.431073in}{1.935615in}}{\pgfqpoint{1.438973in}{1.932343in}}{\pgfqpoint{1.447210in}{1.932343in}}%
\pgfpathclose%
\pgfusepath{stroke,fill}%
\end{pgfscope}%
\begin{pgfscope}%
\pgfpathrectangle{\pgfqpoint{0.100000in}{0.212622in}}{\pgfqpoint{3.696000in}{3.696000in}}%
\pgfusepath{clip}%
\pgfsetbuttcap%
\pgfsetroundjoin%
\definecolor{currentfill}{rgb}{0.121569,0.466667,0.705882}%
\pgfsetfillcolor{currentfill}%
\pgfsetfillopacity{0.391814}%
\pgfsetlinewidth{1.003750pt}%
\definecolor{currentstroke}{rgb}{0.121569,0.466667,0.705882}%
\pgfsetstrokecolor{currentstroke}%
\pgfsetstrokeopacity{0.391814}%
\pgfsetdash{}{0pt}%
\pgfpathmoveto{\pgfqpoint{1.446399in}{1.931885in}}%
\pgfpathcurveto{\pgfqpoint{1.454636in}{1.931885in}}{\pgfqpoint{1.462536in}{1.935157in}}{\pgfqpoint{1.468360in}{1.940981in}}%
\pgfpathcurveto{\pgfqpoint{1.474184in}{1.946805in}}{\pgfqpoint{1.477456in}{1.954705in}}{\pgfqpoint{1.477456in}{1.962941in}}%
\pgfpathcurveto{\pgfqpoint{1.477456in}{1.971178in}}{\pgfqpoint{1.474184in}{1.979078in}}{\pgfqpoint{1.468360in}{1.984902in}}%
\pgfpathcurveto{\pgfqpoint{1.462536in}{1.990725in}}{\pgfqpoint{1.454636in}{1.993998in}}{\pgfqpoint{1.446399in}{1.993998in}}%
\pgfpathcurveto{\pgfqpoint{1.438163in}{1.993998in}}{\pgfqpoint{1.430263in}{1.990725in}}{\pgfqpoint{1.424439in}{1.984902in}}%
\pgfpathcurveto{\pgfqpoint{1.418615in}{1.979078in}}{\pgfqpoint{1.415343in}{1.971178in}}{\pgfqpoint{1.415343in}{1.962941in}}%
\pgfpathcurveto{\pgfqpoint{1.415343in}{1.954705in}}{\pgfqpoint{1.418615in}{1.946805in}}{\pgfqpoint{1.424439in}{1.940981in}}%
\pgfpathcurveto{\pgfqpoint{1.430263in}{1.935157in}}{\pgfqpoint{1.438163in}{1.931885in}}{\pgfqpoint{1.446399in}{1.931885in}}%
\pgfpathclose%
\pgfusepath{stroke,fill}%
\end{pgfscope}%
\begin{pgfscope}%
\pgfpathrectangle{\pgfqpoint{0.100000in}{0.212622in}}{\pgfqpoint{3.696000in}{3.696000in}}%
\pgfusepath{clip}%
\pgfsetbuttcap%
\pgfsetroundjoin%
\definecolor{currentfill}{rgb}{0.121569,0.466667,0.705882}%
\pgfsetfillcolor{currentfill}%
\pgfsetfillopacity{0.391946}%
\pgfsetlinewidth{1.003750pt}%
\definecolor{currentstroke}{rgb}{0.121569,0.466667,0.705882}%
\pgfsetstrokecolor{currentstroke}%
\pgfsetstrokeopacity{0.391946}%
\pgfsetdash{}{0pt}%
\pgfpathmoveto{\pgfqpoint{2.160577in}{2.121134in}}%
\pgfpathcurveto{\pgfqpoint{2.168813in}{2.121134in}}{\pgfqpoint{2.176713in}{2.124406in}}{\pgfqpoint{2.182537in}{2.130230in}}%
\pgfpathcurveto{\pgfqpoint{2.188361in}{2.136054in}}{\pgfqpoint{2.191633in}{2.143954in}}{\pgfqpoint{2.191633in}{2.152190in}}%
\pgfpathcurveto{\pgfqpoint{2.191633in}{2.160426in}}{\pgfqpoint{2.188361in}{2.168326in}}{\pgfqpoint{2.182537in}{2.174150in}}%
\pgfpathcurveto{\pgfqpoint{2.176713in}{2.179974in}}{\pgfqpoint{2.168813in}{2.183247in}}{\pgfqpoint{2.160577in}{2.183247in}}%
\pgfpathcurveto{\pgfqpoint{2.152341in}{2.183247in}}{\pgfqpoint{2.144440in}{2.179974in}}{\pgfqpoint{2.138617in}{2.174150in}}%
\pgfpathcurveto{\pgfqpoint{2.132793in}{2.168326in}}{\pgfqpoint{2.129520in}{2.160426in}}{\pgfqpoint{2.129520in}{2.152190in}}%
\pgfpathcurveto{\pgfqpoint{2.129520in}{2.143954in}}{\pgfqpoint{2.132793in}{2.136054in}}{\pgfqpoint{2.138617in}{2.130230in}}%
\pgfpathcurveto{\pgfqpoint{2.144440in}{2.124406in}}{\pgfqpoint{2.152341in}{2.121134in}}{\pgfqpoint{2.160577in}{2.121134in}}%
\pgfpathclose%
\pgfusepath{stroke,fill}%
\end{pgfscope}%
\begin{pgfscope}%
\pgfpathrectangle{\pgfqpoint{0.100000in}{0.212622in}}{\pgfqpoint{3.696000in}{3.696000in}}%
\pgfusepath{clip}%
\pgfsetbuttcap%
\pgfsetroundjoin%
\definecolor{currentfill}{rgb}{0.121569,0.466667,0.705882}%
\pgfsetfillcolor{currentfill}%
\pgfsetfillopacity{0.392478}%
\pgfsetlinewidth{1.003750pt}%
\definecolor{currentstroke}{rgb}{0.121569,0.466667,0.705882}%
\pgfsetstrokecolor{currentstroke}%
\pgfsetstrokeopacity{0.392478}%
\pgfsetdash{}{0pt}%
\pgfpathmoveto{\pgfqpoint{1.445493in}{1.929517in}}%
\pgfpathcurveto{\pgfqpoint{1.453730in}{1.929517in}}{\pgfqpoint{1.461630in}{1.932789in}}{\pgfqpoint{1.467454in}{1.938613in}}%
\pgfpathcurveto{\pgfqpoint{1.473278in}{1.944437in}}{\pgfqpoint{1.476550in}{1.952337in}}{\pgfqpoint{1.476550in}{1.960574in}}%
\pgfpathcurveto{\pgfqpoint{1.476550in}{1.968810in}}{\pgfqpoint{1.473278in}{1.976710in}}{\pgfqpoint{1.467454in}{1.982534in}}%
\pgfpathcurveto{\pgfqpoint{1.461630in}{1.988358in}}{\pgfqpoint{1.453730in}{1.991630in}}{\pgfqpoint{1.445493in}{1.991630in}}%
\pgfpathcurveto{\pgfqpoint{1.437257in}{1.991630in}}{\pgfqpoint{1.429357in}{1.988358in}}{\pgfqpoint{1.423533in}{1.982534in}}%
\pgfpathcurveto{\pgfqpoint{1.417709in}{1.976710in}}{\pgfqpoint{1.414437in}{1.968810in}}{\pgfqpoint{1.414437in}{1.960574in}}%
\pgfpathcurveto{\pgfqpoint{1.414437in}{1.952337in}}{\pgfqpoint{1.417709in}{1.944437in}}{\pgfqpoint{1.423533in}{1.938613in}}%
\pgfpathcurveto{\pgfqpoint{1.429357in}{1.932789in}}{\pgfqpoint{1.437257in}{1.929517in}}{\pgfqpoint{1.445493in}{1.929517in}}%
\pgfpathclose%
\pgfusepath{stroke,fill}%
\end{pgfscope}%
\begin{pgfscope}%
\pgfpathrectangle{\pgfqpoint{0.100000in}{0.212622in}}{\pgfqpoint{3.696000in}{3.696000in}}%
\pgfusepath{clip}%
\pgfsetbuttcap%
\pgfsetroundjoin%
\definecolor{currentfill}{rgb}{0.121569,0.466667,0.705882}%
\pgfsetfillcolor{currentfill}%
\pgfsetfillopacity{0.393165}%
\pgfsetlinewidth{1.003750pt}%
\definecolor{currentstroke}{rgb}{0.121569,0.466667,0.705882}%
\pgfsetstrokecolor{currentstroke}%
\pgfsetstrokeopacity{0.393165}%
\pgfsetdash{}{0pt}%
\pgfpathmoveto{\pgfqpoint{1.443193in}{1.929369in}}%
\pgfpathcurveto{\pgfqpoint{1.451429in}{1.929369in}}{\pgfqpoint{1.459329in}{1.932642in}}{\pgfqpoint{1.465153in}{1.938466in}}%
\pgfpathcurveto{\pgfqpoint{1.470977in}{1.944289in}}{\pgfqpoint{1.474249in}{1.952190in}}{\pgfqpoint{1.474249in}{1.960426in}}%
\pgfpathcurveto{\pgfqpoint{1.474249in}{1.968662in}}{\pgfqpoint{1.470977in}{1.976562in}}{\pgfqpoint{1.465153in}{1.982386in}}%
\pgfpathcurveto{\pgfqpoint{1.459329in}{1.988210in}}{\pgfqpoint{1.451429in}{1.991482in}}{\pgfqpoint{1.443193in}{1.991482in}}%
\pgfpathcurveto{\pgfqpoint{1.434956in}{1.991482in}}{\pgfqpoint{1.427056in}{1.988210in}}{\pgfqpoint{1.421232in}{1.982386in}}%
\pgfpathcurveto{\pgfqpoint{1.415408in}{1.976562in}}{\pgfqpoint{1.412136in}{1.968662in}}{\pgfqpoint{1.412136in}{1.960426in}}%
\pgfpathcurveto{\pgfqpoint{1.412136in}{1.952190in}}{\pgfqpoint{1.415408in}{1.944289in}}{\pgfqpoint{1.421232in}{1.938466in}}%
\pgfpathcurveto{\pgfqpoint{1.427056in}{1.932642in}}{\pgfqpoint{1.434956in}{1.929369in}}{\pgfqpoint{1.443193in}{1.929369in}}%
\pgfpathclose%
\pgfusepath{stroke,fill}%
\end{pgfscope}%
\begin{pgfscope}%
\pgfpathrectangle{\pgfqpoint{0.100000in}{0.212622in}}{\pgfqpoint{3.696000in}{3.696000in}}%
\pgfusepath{clip}%
\pgfsetbuttcap%
\pgfsetroundjoin%
\definecolor{currentfill}{rgb}{0.121569,0.466667,0.705882}%
\pgfsetfillcolor{currentfill}%
\pgfsetfillopacity{0.393794}%
\pgfsetlinewidth{1.003750pt}%
\definecolor{currentstroke}{rgb}{0.121569,0.466667,0.705882}%
\pgfsetstrokecolor{currentstroke}%
\pgfsetstrokeopacity{0.393794}%
\pgfsetdash{}{0pt}%
\pgfpathmoveto{\pgfqpoint{1.442816in}{1.929298in}}%
\pgfpathcurveto{\pgfqpoint{1.451052in}{1.929298in}}{\pgfqpoint{1.458952in}{1.932570in}}{\pgfqpoint{1.464776in}{1.938394in}}%
\pgfpathcurveto{\pgfqpoint{1.470600in}{1.944218in}}{\pgfqpoint{1.473872in}{1.952118in}}{\pgfqpoint{1.473872in}{1.960354in}}%
\pgfpathcurveto{\pgfqpoint{1.473872in}{1.968590in}}{\pgfqpoint{1.470600in}{1.976490in}}{\pgfqpoint{1.464776in}{1.982314in}}%
\pgfpathcurveto{\pgfqpoint{1.458952in}{1.988138in}}{\pgfqpoint{1.451052in}{1.991411in}}{\pgfqpoint{1.442816in}{1.991411in}}%
\pgfpathcurveto{\pgfqpoint{1.434580in}{1.991411in}}{\pgfqpoint{1.426680in}{1.988138in}}{\pgfqpoint{1.420856in}{1.982314in}}%
\pgfpathcurveto{\pgfqpoint{1.415032in}{1.976490in}}{\pgfqpoint{1.411759in}{1.968590in}}{\pgfqpoint{1.411759in}{1.960354in}}%
\pgfpathcurveto{\pgfqpoint{1.411759in}{1.952118in}}{\pgfqpoint{1.415032in}{1.944218in}}{\pgfqpoint{1.420856in}{1.938394in}}%
\pgfpathcurveto{\pgfqpoint{1.426680in}{1.932570in}}{\pgfqpoint{1.434580in}{1.929298in}}{\pgfqpoint{1.442816in}{1.929298in}}%
\pgfpathclose%
\pgfusepath{stroke,fill}%
\end{pgfscope}%
\begin{pgfscope}%
\pgfpathrectangle{\pgfqpoint{0.100000in}{0.212622in}}{\pgfqpoint{3.696000in}{3.696000in}}%
\pgfusepath{clip}%
\pgfsetbuttcap%
\pgfsetroundjoin%
\definecolor{currentfill}{rgb}{0.121569,0.466667,0.705882}%
\pgfsetfillcolor{currentfill}%
\pgfsetfillopacity{0.394035}%
\pgfsetlinewidth{1.003750pt}%
\definecolor{currentstroke}{rgb}{0.121569,0.466667,0.705882}%
\pgfsetstrokecolor{currentstroke}%
\pgfsetstrokeopacity{0.394035}%
\pgfsetdash{}{0pt}%
\pgfpathmoveto{\pgfqpoint{1.442182in}{1.927812in}}%
\pgfpathcurveto{\pgfqpoint{1.450418in}{1.927812in}}{\pgfqpoint{1.458318in}{1.931084in}}{\pgfqpoint{1.464142in}{1.936908in}}%
\pgfpathcurveto{\pgfqpoint{1.469966in}{1.942732in}}{\pgfqpoint{1.473238in}{1.950632in}}{\pgfqpoint{1.473238in}{1.958868in}}%
\pgfpathcurveto{\pgfqpoint{1.473238in}{1.967104in}}{\pgfqpoint{1.469966in}{1.975005in}}{\pgfqpoint{1.464142in}{1.980828in}}%
\pgfpathcurveto{\pgfqpoint{1.458318in}{1.986652in}}{\pgfqpoint{1.450418in}{1.989925in}}{\pgfqpoint{1.442182in}{1.989925in}}%
\pgfpathcurveto{\pgfqpoint{1.433945in}{1.989925in}}{\pgfqpoint{1.426045in}{1.986652in}}{\pgfqpoint{1.420222in}{1.980828in}}%
\pgfpathcurveto{\pgfqpoint{1.414398in}{1.975005in}}{\pgfqpoint{1.411125in}{1.967104in}}{\pgfqpoint{1.411125in}{1.958868in}}%
\pgfpathcurveto{\pgfqpoint{1.411125in}{1.950632in}}{\pgfqpoint{1.414398in}{1.942732in}}{\pgfqpoint{1.420222in}{1.936908in}}%
\pgfpathcurveto{\pgfqpoint{1.426045in}{1.931084in}}{\pgfqpoint{1.433945in}{1.927812in}}{\pgfqpoint{1.442182in}{1.927812in}}%
\pgfpathclose%
\pgfusepath{stroke,fill}%
\end{pgfscope}%
\begin{pgfscope}%
\pgfpathrectangle{\pgfqpoint{0.100000in}{0.212622in}}{\pgfqpoint{3.696000in}{3.696000in}}%
\pgfusepath{clip}%
\pgfsetbuttcap%
\pgfsetroundjoin%
\definecolor{currentfill}{rgb}{0.121569,0.466667,0.705882}%
\pgfsetfillcolor{currentfill}%
\pgfsetfillopacity{0.394061}%
\pgfsetlinewidth{1.003750pt}%
\definecolor{currentstroke}{rgb}{0.121569,0.466667,0.705882}%
\pgfsetstrokecolor{currentstroke}%
\pgfsetstrokeopacity{0.394061}%
\pgfsetdash{}{0pt}%
\pgfpathmoveto{\pgfqpoint{2.173023in}{2.112034in}}%
\pgfpathcurveto{\pgfqpoint{2.181260in}{2.112034in}}{\pgfqpoint{2.189160in}{2.115307in}}{\pgfqpoint{2.194984in}{2.121131in}}%
\pgfpathcurveto{\pgfqpoint{2.200807in}{2.126955in}}{\pgfqpoint{2.204080in}{2.134855in}}{\pgfqpoint{2.204080in}{2.143091in}}%
\pgfpathcurveto{\pgfqpoint{2.204080in}{2.151327in}}{\pgfqpoint{2.200807in}{2.159227in}}{\pgfqpoint{2.194984in}{2.165051in}}%
\pgfpathcurveto{\pgfqpoint{2.189160in}{2.170875in}}{\pgfqpoint{2.181260in}{2.174147in}}{\pgfqpoint{2.173023in}{2.174147in}}%
\pgfpathcurveto{\pgfqpoint{2.164787in}{2.174147in}}{\pgfqpoint{2.156887in}{2.170875in}}{\pgfqpoint{2.151063in}{2.165051in}}%
\pgfpathcurveto{\pgfqpoint{2.145239in}{2.159227in}}{\pgfqpoint{2.141967in}{2.151327in}}{\pgfqpoint{2.141967in}{2.143091in}}%
\pgfpathcurveto{\pgfqpoint{2.141967in}{2.134855in}}{\pgfqpoint{2.145239in}{2.126955in}}{\pgfqpoint{2.151063in}{2.121131in}}%
\pgfpathcurveto{\pgfqpoint{2.156887in}{2.115307in}}{\pgfqpoint{2.164787in}{2.112034in}}{\pgfqpoint{2.173023in}{2.112034in}}%
\pgfpathclose%
\pgfusepath{stroke,fill}%
\end{pgfscope}%
\begin{pgfscope}%
\pgfpathrectangle{\pgfqpoint{0.100000in}{0.212622in}}{\pgfqpoint{3.696000in}{3.696000in}}%
\pgfusepath{clip}%
\pgfsetbuttcap%
\pgfsetroundjoin%
\definecolor{currentfill}{rgb}{0.121569,0.466667,0.705882}%
\pgfsetfillcolor{currentfill}%
\pgfsetfillopacity{0.394627}%
\pgfsetlinewidth{1.003750pt}%
\definecolor{currentstroke}{rgb}{0.121569,0.466667,0.705882}%
\pgfsetstrokecolor{currentstroke}%
\pgfsetstrokeopacity{0.394627}%
\pgfsetdash{}{0pt}%
\pgfpathmoveto{\pgfqpoint{1.441477in}{1.929704in}}%
\pgfpathcurveto{\pgfqpoint{1.449713in}{1.929704in}}{\pgfqpoint{1.457614in}{1.932976in}}{\pgfqpoint{1.463437in}{1.938800in}}%
\pgfpathcurveto{\pgfqpoint{1.469261in}{1.944624in}}{\pgfqpoint{1.472534in}{1.952524in}}{\pgfqpoint{1.472534in}{1.960760in}}%
\pgfpathcurveto{\pgfqpoint{1.472534in}{1.968996in}}{\pgfqpoint{1.469261in}{1.976896in}}{\pgfqpoint{1.463437in}{1.982720in}}%
\pgfpathcurveto{\pgfqpoint{1.457614in}{1.988544in}}{\pgfqpoint{1.449713in}{1.991817in}}{\pgfqpoint{1.441477in}{1.991817in}}%
\pgfpathcurveto{\pgfqpoint{1.433241in}{1.991817in}}{\pgfqpoint{1.425341in}{1.988544in}}{\pgfqpoint{1.419517in}{1.982720in}}%
\pgfpathcurveto{\pgfqpoint{1.413693in}{1.976896in}}{\pgfqpoint{1.410421in}{1.968996in}}{\pgfqpoint{1.410421in}{1.960760in}}%
\pgfpathcurveto{\pgfqpoint{1.410421in}{1.952524in}}{\pgfqpoint{1.413693in}{1.944624in}}{\pgfqpoint{1.419517in}{1.938800in}}%
\pgfpathcurveto{\pgfqpoint{1.425341in}{1.932976in}}{\pgfqpoint{1.433241in}{1.929704in}}{\pgfqpoint{1.441477in}{1.929704in}}%
\pgfpathclose%
\pgfusepath{stroke,fill}%
\end{pgfscope}%
\begin{pgfscope}%
\pgfpathrectangle{\pgfqpoint{0.100000in}{0.212622in}}{\pgfqpoint{3.696000in}{3.696000in}}%
\pgfusepath{clip}%
\pgfsetbuttcap%
\pgfsetroundjoin%
\definecolor{currentfill}{rgb}{0.121569,0.466667,0.705882}%
\pgfsetfillcolor{currentfill}%
\pgfsetfillopacity{0.394813}%
\pgfsetlinewidth{1.003750pt}%
\definecolor{currentstroke}{rgb}{0.121569,0.466667,0.705882}%
\pgfsetstrokecolor{currentstroke}%
\pgfsetstrokeopacity{0.394813}%
\pgfsetdash{}{0pt}%
\pgfpathmoveto{\pgfqpoint{1.441304in}{1.929699in}}%
\pgfpathcurveto{\pgfqpoint{1.449540in}{1.929699in}}{\pgfqpoint{1.457440in}{1.932971in}}{\pgfqpoint{1.463264in}{1.938795in}}%
\pgfpathcurveto{\pgfqpoint{1.469088in}{1.944619in}}{\pgfqpoint{1.472360in}{1.952519in}}{\pgfqpoint{1.472360in}{1.960755in}}%
\pgfpathcurveto{\pgfqpoint{1.472360in}{1.968992in}}{\pgfqpoint{1.469088in}{1.976892in}}{\pgfqpoint{1.463264in}{1.982716in}}%
\pgfpathcurveto{\pgfqpoint{1.457440in}{1.988540in}}{\pgfqpoint{1.449540in}{1.991812in}}{\pgfqpoint{1.441304in}{1.991812in}}%
\pgfpathcurveto{\pgfqpoint{1.433067in}{1.991812in}}{\pgfqpoint{1.425167in}{1.988540in}}{\pgfqpoint{1.419343in}{1.982716in}}%
\pgfpathcurveto{\pgfqpoint{1.413519in}{1.976892in}}{\pgfqpoint{1.410247in}{1.968992in}}{\pgfqpoint{1.410247in}{1.960755in}}%
\pgfpathcurveto{\pgfqpoint{1.410247in}{1.952519in}}{\pgfqpoint{1.413519in}{1.944619in}}{\pgfqpoint{1.419343in}{1.938795in}}%
\pgfpathcurveto{\pgfqpoint{1.425167in}{1.932971in}}{\pgfqpoint{1.433067in}{1.929699in}}{\pgfqpoint{1.441304in}{1.929699in}}%
\pgfpathclose%
\pgfusepath{stroke,fill}%
\end{pgfscope}%
\begin{pgfscope}%
\pgfpathrectangle{\pgfqpoint{0.100000in}{0.212622in}}{\pgfqpoint{3.696000in}{3.696000in}}%
\pgfusepath{clip}%
\pgfsetbuttcap%
\pgfsetroundjoin%
\definecolor{currentfill}{rgb}{0.121569,0.466667,0.705882}%
\pgfsetfillcolor{currentfill}%
\pgfsetfillopacity{0.394980}%
\pgfsetlinewidth{1.003750pt}%
\definecolor{currentstroke}{rgb}{0.121569,0.466667,0.705882}%
\pgfsetstrokecolor{currentstroke}%
\pgfsetstrokeopacity{0.394980}%
\pgfsetdash{}{0pt}%
\pgfpathmoveto{\pgfqpoint{1.440525in}{1.928843in}}%
\pgfpathcurveto{\pgfqpoint{1.448761in}{1.928843in}}{\pgfqpoint{1.456661in}{1.932115in}}{\pgfqpoint{1.462485in}{1.937939in}}%
\pgfpathcurveto{\pgfqpoint{1.468309in}{1.943763in}}{\pgfqpoint{1.471582in}{1.951663in}}{\pgfqpoint{1.471582in}{1.959899in}}%
\pgfpathcurveto{\pgfqpoint{1.471582in}{1.968136in}}{\pgfqpoint{1.468309in}{1.976036in}}{\pgfqpoint{1.462485in}{1.981860in}}%
\pgfpathcurveto{\pgfqpoint{1.456661in}{1.987683in}}{\pgfqpoint{1.448761in}{1.990956in}}{\pgfqpoint{1.440525in}{1.990956in}}%
\pgfpathcurveto{\pgfqpoint{1.432289in}{1.990956in}}{\pgfqpoint{1.424389in}{1.987683in}}{\pgfqpoint{1.418565in}{1.981860in}}%
\pgfpathcurveto{\pgfqpoint{1.412741in}{1.976036in}}{\pgfqpoint{1.409469in}{1.968136in}}{\pgfqpoint{1.409469in}{1.959899in}}%
\pgfpathcurveto{\pgfqpoint{1.409469in}{1.951663in}}{\pgfqpoint{1.412741in}{1.943763in}}{\pgfqpoint{1.418565in}{1.937939in}}%
\pgfpathcurveto{\pgfqpoint{1.424389in}{1.932115in}}{\pgfqpoint{1.432289in}{1.928843in}}{\pgfqpoint{1.440525in}{1.928843in}}%
\pgfpathclose%
\pgfusepath{stroke,fill}%
\end{pgfscope}%
\begin{pgfscope}%
\pgfpathrectangle{\pgfqpoint{0.100000in}{0.212622in}}{\pgfqpoint{3.696000in}{3.696000in}}%
\pgfusepath{clip}%
\pgfsetbuttcap%
\pgfsetroundjoin%
\definecolor{currentfill}{rgb}{0.121569,0.466667,0.705882}%
\pgfsetfillcolor{currentfill}%
\pgfsetfillopacity{0.395322}%
\pgfsetlinewidth{1.003750pt}%
\definecolor{currentstroke}{rgb}{0.121569,0.466667,0.705882}%
\pgfsetstrokecolor{currentstroke}%
\pgfsetstrokeopacity{0.395322}%
\pgfsetdash{}{0pt}%
\pgfpathmoveto{\pgfqpoint{1.439180in}{1.927476in}}%
\pgfpathcurveto{\pgfqpoint{1.447416in}{1.927476in}}{\pgfqpoint{1.455316in}{1.930748in}}{\pgfqpoint{1.461140in}{1.936572in}}%
\pgfpathcurveto{\pgfqpoint{1.466964in}{1.942396in}}{\pgfqpoint{1.470237in}{1.950296in}}{\pgfqpoint{1.470237in}{1.958533in}}%
\pgfpathcurveto{\pgfqpoint{1.470237in}{1.966769in}}{\pgfqpoint{1.466964in}{1.974669in}}{\pgfqpoint{1.461140in}{1.980493in}}%
\pgfpathcurveto{\pgfqpoint{1.455316in}{1.986317in}}{\pgfqpoint{1.447416in}{1.989589in}}{\pgfqpoint{1.439180in}{1.989589in}}%
\pgfpathcurveto{\pgfqpoint{1.430944in}{1.989589in}}{\pgfqpoint{1.423044in}{1.986317in}}{\pgfqpoint{1.417220in}{1.980493in}}%
\pgfpathcurveto{\pgfqpoint{1.411396in}{1.974669in}}{\pgfqpoint{1.408124in}{1.966769in}}{\pgfqpoint{1.408124in}{1.958533in}}%
\pgfpathcurveto{\pgfqpoint{1.408124in}{1.950296in}}{\pgfqpoint{1.411396in}{1.942396in}}{\pgfqpoint{1.417220in}{1.936572in}}%
\pgfpathcurveto{\pgfqpoint{1.423044in}{1.930748in}}{\pgfqpoint{1.430944in}{1.927476in}}{\pgfqpoint{1.439180in}{1.927476in}}%
\pgfpathclose%
\pgfusepath{stroke,fill}%
\end{pgfscope}%
\begin{pgfscope}%
\pgfpathrectangle{\pgfqpoint{0.100000in}{0.212622in}}{\pgfqpoint{3.696000in}{3.696000in}}%
\pgfusepath{clip}%
\pgfsetbuttcap%
\pgfsetroundjoin%
\definecolor{currentfill}{rgb}{0.121569,0.466667,0.705882}%
\pgfsetfillcolor{currentfill}%
\pgfsetfillopacity{0.395813}%
\pgfsetlinewidth{1.003750pt}%
\definecolor{currentstroke}{rgb}{0.121569,0.466667,0.705882}%
\pgfsetstrokecolor{currentstroke}%
\pgfsetstrokeopacity{0.395813}%
\pgfsetdash{}{0pt}%
\pgfpathmoveto{\pgfqpoint{1.438700in}{1.927213in}}%
\pgfpathcurveto{\pgfqpoint{1.446937in}{1.927213in}}{\pgfqpoint{1.454837in}{1.930486in}}{\pgfqpoint{1.460661in}{1.936310in}}%
\pgfpathcurveto{\pgfqpoint{1.466485in}{1.942133in}}{\pgfqpoint{1.469757in}{1.950034in}}{\pgfqpoint{1.469757in}{1.958270in}}%
\pgfpathcurveto{\pgfqpoint{1.469757in}{1.966506in}}{\pgfqpoint{1.466485in}{1.974406in}}{\pgfqpoint{1.460661in}{1.980230in}}%
\pgfpathcurveto{\pgfqpoint{1.454837in}{1.986054in}}{\pgfqpoint{1.446937in}{1.989326in}}{\pgfqpoint{1.438700in}{1.989326in}}%
\pgfpathcurveto{\pgfqpoint{1.430464in}{1.989326in}}{\pgfqpoint{1.422564in}{1.986054in}}{\pgfqpoint{1.416740in}{1.980230in}}%
\pgfpathcurveto{\pgfqpoint{1.410916in}{1.974406in}}{\pgfqpoint{1.407644in}{1.966506in}}{\pgfqpoint{1.407644in}{1.958270in}}%
\pgfpathcurveto{\pgfqpoint{1.407644in}{1.950034in}}{\pgfqpoint{1.410916in}{1.942133in}}{\pgfqpoint{1.416740in}{1.936310in}}%
\pgfpathcurveto{\pgfqpoint{1.422564in}{1.930486in}}{\pgfqpoint{1.430464in}{1.927213in}}{\pgfqpoint{1.438700in}{1.927213in}}%
\pgfpathclose%
\pgfusepath{stroke,fill}%
\end{pgfscope}%
\begin{pgfscope}%
\pgfpathrectangle{\pgfqpoint{0.100000in}{0.212622in}}{\pgfqpoint{3.696000in}{3.696000in}}%
\pgfusepath{clip}%
\pgfsetbuttcap%
\pgfsetroundjoin%
\definecolor{currentfill}{rgb}{0.121569,0.466667,0.705882}%
\pgfsetfillcolor{currentfill}%
\pgfsetfillopacity{0.396093}%
\pgfsetlinewidth{1.003750pt}%
\definecolor{currentstroke}{rgb}{0.121569,0.466667,0.705882}%
\pgfsetstrokecolor{currentstroke}%
\pgfsetstrokeopacity{0.396093}%
\pgfsetdash{}{0pt}%
\pgfpathmoveto{\pgfqpoint{1.437690in}{1.926960in}}%
\pgfpathcurveto{\pgfqpoint{1.445926in}{1.926960in}}{\pgfqpoint{1.453826in}{1.930232in}}{\pgfqpoint{1.459650in}{1.936056in}}%
\pgfpathcurveto{\pgfqpoint{1.465474in}{1.941880in}}{\pgfqpoint{1.468746in}{1.949780in}}{\pgfqpoint{1.468746in}{1.958017in}}%
\pgfpathcurveto{\pgfqpoint{1.468746in}{1.966253in}}{\pgfqpoint{1.465474in}{1.974153in}}{\pgfqpoint{1.459650in}{1.979977in}}%
\pgfpathcurveto{\pgfqpoint{1.453826in}{1.985801in}}{\pgfqpoint{1.445926in}{1.989073in}}{\pgfqpoint{1.437690in}{1.989073in}}%
\pgfpathcurveto{\pgfqpoint{1.429454in}{1.989073in}}{\pgfqpoint{1.421553in}{1.985801in}}{\pgfqpoint{1.415730in}{1.979977in}}%
\pgfpathcurveto{\pgfqpoint{1.409906in}{1.974153in}}{\pgfqpoint{1.406633in}{1.966253in}}{\pgfqpoint{1.406633in}{1.958017in}}%
\pgfpathcurveto{\pgfqpoint{1.406633in}{1.949780in}}{\pgfqpoint{1.409906in}{1.941880in}}{\pgfqpoint{1.415730in}{1.936056in}}%
\pgfpathcurveto{\pgfqpoint{1.421553in}{1.930232in}}{\pgfqpoint{1.429454in}{1.926960in}}{\pgfqpoint{1.437690in}{1.926960in}}%
\pgfpathclose%
\pgfusepath{stroke,fill}%
\end{pgfscope}%
\begin{pgfscope}%
\pgfpathrectangle{\pgfqpoint{0.100000in}{0.212622in}}{\pgfqpoint{3.696000in}{3.696000in}}%
\pgfusepath{clip}%
\pgfsetbuttcap%
\pgfsetroundjoin%
\definecolor{currentfill}{rgb}{0.121569,0.466667,0.705882}%
\pgfsetfillcolor{currentfill}%
\pgfsetfillopacity{0.396429}%
\pgfsetlinewidth{1.003750pt}%
\definecolor{currentstroke}{rgb}{0.121569,0.466667,0.705882}%
\pgfsetstrokecolor{currentstroke}%
\pgfsetstrokeopacity{0.396429}%
\pgfsetdash{}{0pt}%
\pgfpathmoveto{\pgfqpoint{1.437514in}{1.927211in}}%
\pgfpathcurveto{\pgfqpoint{1.445750in}{1.927211in}}{\pgfqpoint{1.453650in}{1.930483in}}{\pgfqpoint{1.459474in}{1.936307in}}%
\pgfpathcurveto{\pgfqpoint{1.465298in}{1.942131in}}{\pgfqpoint{1.468570in}{1.950031in}}{\pgfqpoint{1.468570in}{1.958267in}}%
\pgfpathcurveto{\pgfqpoint{1.468570in}{1.966503in}}{\pgfqpoint{1.465298in}{1.974404in}}{\pgfqpoint{1.459474in}{1.980227in}}%
\pgfpathcurveto{\pgfqpoint{1.453650in}{1.986051in}}{\pgfqpoint{1.445750in}{1.989324in}}{\pgfqpoint{1.437514in}{1.989324in}}%
\pgfpathcurveto{\pgfqpoint{1.429278in}{1.989324in}}{\pgfqpoint{1.421378in}{1.986051in}}{\pgfqpoint{1.415554in}{1.980227in}}%
\pgfpathcurveto{\pgfqpoint{1.409730in}{1.974404in}}{\pgfqpoint{1.406457in}{1.966503in}}{\pgfqpoint{1.406457in}{1.958267in}}%
\pgfpathcurveto{\pgfqpoint{1.406457in}{1.950031in}}{\pgfqpoint{1.409730in}{1.942131in}}{\pgfqpoint{1.415554in}{1.936307in}}%
\pgfpathcurveto{\pgfqpoint{1.421378in}{1.930483in}}{\pgfqpoint{1.429278in}{1.927211in}}{\pgfqpoint{1.437514in}{1.927211in}}%
\pgfpathclose%
\pgfusepath{stroke,fill}%
\end{pgfscope}%
\begin{pgfscope}%
\pgfpathrectangle{\pgfqpoint{0.100000in}{0.212622in}}{\pgfqpoint{3.696000in}{3.696000in}}%
\pgfusepath{clip}%
\pgfsetbuttcap%
\pgfsetroundjoin%
\definecolor{currentfill}{rgb}{0.121569,0.466667,0.705882}%
\pgfsetfillcolor{currentfill}%
\pgfsetfillopacity{0.396692}%
\pgfsetlinewidth{1.003750pt}%
\definecolor{currentstroke}{rgb}{0.121569,0.466667,0.705882}%
\pgfsetstrokecolor{currentstroke}%
\pgfsetstrokeopacity{0.396692}%
\pgfsetdash{}{0pt}%
\pgfpathmoveto{\pgfqpoint{1.436757in}{1.925325in}}%
\pgfpathcurveto{\pgfqpoint{1.444993in}{1.925325in}}{\pgfqpoint{1.452893in}{1.928597in}}{\pgfqpoint{1.458717in}{1.934421in}}%
\pgfpathcurveto{\pgfqpoint{1.464541in}{1.940245in}}{\pgfqpoint{1.467813in}{1.948145in}}{\pgfqpoint{1.467813in}{1.956382in}}%
\pgfpathcurveto{\pgfqpoint{1.467813in}{1.964618in}}{\pgfqpoint{1.464541in}{1.972518in}}{\pgfqpoint{1.458717in}{1.978342in}}%
\pgfpathcurveto{\pgfqpoint{1.452893in}{1.984166in}}{\pgfqpoint{1.444993in}{1.987438in}}{\pgfqpoint{1.436757in}{1.987438in}}%
\pgfpathcurveto{\pgfqpoint{1.428521in}{1.987438in}}{\pgfqpoint{1.420621in}{1.984166in}}{\pgfqpoint{1.414797in}{1.978342in}}%
\pgfpathcurveto{\pgfqpoint{1.408973in}{1.972518in}}{\pgfqpoint{1.405700in}{1.964618in}}{\pgfqpoint{1.405700in}{1.956382in}}%
\pgfpathcurveto{\pgfqpoint{1.405700in}{1.948145in}}{\pgfqpoint{1.408973in}{1.940245in}}{\pgfqpoint{1.414797in}{1.934421in}}%
\pgfpathcurveto{\pgfqpoint{1.420621in}{1.928597in}}{\pgfqpoint{1.428521in}{1.925325in}}{\pgfqpoint{1.436757in}{1.925325in}}%
\pgfpathclose%
\pgfusepath{stroke,fill}%
\end{pgfscope}%
\begin{pgfscope}%
\pgfpathrectangle{\pgfqpoint{0.100000in}{0.212622in}}{\pgfqpoint{3.696000in}{3.696000in}}%
\pgfusepath{clip}%
\pgfsetbuttcap%
\pgfsetroundjoin%
\definecolor{currentfill}{rgb}{0.121569,0.466667,0.705882}%
\pgfsetfillcolor{currentfill}%
\pgfsetfillopacity{0.397404}%
\pgfsetlinewidth{1.003750pt}%
\definecolor{currentstroke}{rgb}{0.121569,0.466667,0.705882}%
\pgfsetstrokecolor{currentstroke}%
\pgfsetstrokeopacity{0.397404}%
\pgfsetdash{}{0pt}%
\pgfpathmoveto{\pgfqpoint{1.434702in}{1.924270in}}%
\pgfpathcurveto{\pgfqpoint{1.442938in}{1.924270in}}{\pgfqpoint{1.450838in}{1.927543in}}{\pgfqpoint{1.456662in}{1.933366in}}%
\pgfpathcurveto{\pgfqpoint{1.462486in}{1.939190in}}{\pgfqpoint{1.465758in}{1.947090in}}{\pgfqpoint{1.465758in}{1.955327in}}%
\pgfpathcurveto{\pgfqpoint{1.465758in}{1.963563in}}{\pgfqpoint{1.462486in}{1.971463in}}{\pgfqpoint{1.456662in}{1.977287in}}%
\pgfpathcurveto{\pgfqpoint{1.450838in}{1.983111in}}{\pgfqpoint{1.442938in}{1.986383in}}{\pgfqpoint{1.434702in}{1.986383in}}%
\pgfpathcurveto{\pgfqpoint{1.426466in}{1.986383in}}{\pgfqpoint{1.418566in}{1.983111in}}{\pgfqpoint{1.412742in}{1.977287in}}%
\pgfpathcurveto{\pgfqpoint{1.406918in}{1.971463in}}{\pgfqpoint{1.403645in}{1.963563in}}{\pgfqpoint{1.403645in}{1.955327in}}%
\pgfpathcurveto{\pgfqpoint{1.403645in}{1.947090in}}{\pgfqpoint{1.406918in}{1.939190in}}{\pgfqpoint{1.412742in}{1.933366in}}%
\pgfpathcurveto{\pgfqpoint{1.418566in}{1.927543in}}{\pgfqpoint{1.426466in}{1.924270in}}{\pgfqpoint{1.434702in}{1.924270in}}%
\pgfpathclose%
\pgfusepath{stroke,fill}%
\end{pgfscope}%
\begin{pgfscope}%
\pgfpathrectangle{\pgfqpoint{0.100000in}{0.212622in}}{\pgfqpoint{3.696000in}{3.696000in}}%
\pgfusepath{clip}%
\pgfsetbuttcap%
\pgfsetroundjoin%
\definecolor{currentfill}{rgb}{0.121569,0.466667,0.705882}%
\pgfsetfillcolor{currentfill}%
\pgfsetfillopacity{0.397977}%
\pgfsetlinewidth{1.003750pt}%
\definecolor{currentstroke}{rgb}{0.121569,0.466667,0.705882}%
\pgfsetstrokecolor{currentstroke}%
\pgfsetstrokeopacity{0.397977}%
\pgfsetdash{}{0pt}%
\pgfpathmoveto{\pgfqpoint{1.433750in}{1.922166in}}%
\pgfpathcurveto{\pgfqpoint{1.441987in}{1.922166in}}{\pgfqpoint{1.449887in}{1.925439in}}{\pgfqpoint{1.455711in}{1.931262in}}%
\pgfpathcurveto{\pgfqpoint{1.461535in}{1.937086in}}{\pgfqpoint{1.464807in}{1.944986in}}{\pgfqpoint{1.464807in}{1.953223in}}%
\pgfpathcurveto{\pgfqpoint{1.464807in}{1.961459in}}{\pgfqpoint{1.461535in}{1.969359in}}{\pgfqpoint{1.455711in}{1.975183in}}%
\pgfpathcurveto{\pgfqpoint{1.449887in}{1.981007in}}{\pgfqpoint{1.441987in}{1.984279in}}{\pgfqpoint{1.433750in}{1.984279in}}%
\pgfpathcurveto{\pgfqpoint{1.425514in}{1.984279in}}{\pgfqpoint{1.417614in}{1.981007in}}{\pgfqpoint{1.411790in}{1.975183in}}%
\pgfpathcurveto{\pgfqpoint{1.405966in}{1.969359in}}{\pgfqpoint{1.402694in}{1.961459in}}{\pgfqpoint{1.402694in}{1.953223in}}%
\pgfpathcurveto{\pgfqpoint{1.402694in}{1.944986in}}{\pgfqpoint{1.405966in}{1.937086in}}{\pgfqpoint{1.411790in}{1.931262in}}%
\pgfpathcurveto{\pgfqpoint{1.417614in}{1.925439in}}{\pgfqpoint{1.425514in}{1.922166in}}{\pgfqpoint{1.433750in}{1.922166in}}%
\pgfpathclose%
\pgfusepath{stroke,fill}%
\end{pgfscope}%
\begin{pgfscope}%
\pgfpathrectangle{\pgfqpoint{0.100000in}{0.212622in}}{\pgfqpoint{3.696000in}{3.696000in}}%
\pgfusepath{clip}%
\pgfsetbuttcap%
\pgfsetroundjoin%
\definecolor{currentfill}{rgb}{0.121569,0.466667,0.705882}%
\pgfsetfillcolor{currentfill}%
\pgfsetfillopacity{0.398257}%
\pgfsetlinewidth{1.003750pt}%
\definecolor{currentstroke}{rgb}{0.121569,0.466667,0.705882}%
\pgfsetstrokecolor{currentstroke}%
\pgfsetstrokeopacity{0.398257}%
\pgfsetdash{}{0pt}%
\pgfpathmoveto{\pgfqpoint{1.433090in}{1.921170in}}%
\pgfpathcurveto{\pgfqpoint{1.441326in}{1.921170in}}{\pgfqpoint{1.449226in}{1.924442in}}{\pgfqpoint{1.455050in}{1.930266in}}%
\pgfpathcurveto{\pgfqpoint{1.460874in}{1.936090in}}{\pgfqpoint{1.464147in}{1.943990in}}{\pgfqpoint{1.464147in}{1.952226in}}%
\pgfpathcurveto{\pgfqpoint{1.464147in}{1.960463in}}{\pgfqpoint{1.460874in}{1.968363in}}{\pgfqpoint{1.455050in}{1.974187in}}%
\pgfpathcurveto{\pgfqpoint{1.449226in}{1.980011in}}{\pgfqpoint{1.441326in}{1.983283in}}{\pgfqpoint{1.433090in}{1.983283in}}%
\pgfpathcurveto{\pgfqpoint{1.424854in}{1.983283in}}{\pgfqpoint{1.416954in}{1.980011in}}{\pgfqpoint{1.411130in}{1.974187in}}%
\pgfpathcurveto{\pgfqpoint{1.405306in}{1.968363in}}{\pgfqpoint{1.402034in}{1.960463in}}{\pgfqpoint{1.402034in}{1.952226in}}%
\pgfpathcurveto{\pgfqpoint{1.402034in}{1.943990in}}{\pgfqpoint{1.405306in}{1.936090in}}{\pgfqpoint{1.411130in}{1.930266in}}%
\pgfpathcurveto{\pgfqpoint{1.416954in}{1.924442in}}{\pgfqpoint{1.424854in}{1.921170in}}{\pgfqpoint{1.433090in}{1.921170in}}%
\pgfpathclose%
\pgfusepath{stroke,fill}%
\end{pgfscope}%
\begin{pgfscope}%
\pgfpathrectangle{\pgfqpoint{0.100000in}{0.212622in}}{\pgfqpoint{3.696000in}{3.696000in}}%
\pgfusepath{clip}%
\pgfsetbuttcap%
\pgfsetroundjoin%
\definecolor{currentfill}{rgb}{0.121569,0.466667,0.705882}%
\pgfsetfillcolor{currentfill}%
\pgfsetfillopacity{0.398477}%
\pgfsetlinewidth{1.003750pt}%
\definecolor{currentstroke}{rgb}{0.121569,0.466667,0.705882}%
\pgfsetstrokecolor{currentstroke}%
\pgfsetstrokeopacity{0.398477}%
\pgfsetdash{}{0pt}%
\pgfpathmoveto{\pgfqpoint{2.186954in}{2.122051in}}%
\pgfpathcurveto{\pgfqpoint{2.195191in}{2.122051in}}{\pgfqpoint{2.203091in}{2.125323in}}{\pgfqpoint{2.208915in}{2.131147in}}%
\pgfpathcurveto{\pgfqpoint{2.214739in}{2.136971in}}{\pgfqpoint{2.218011in}{2.144871in}}{\pgfqpoint{2.218011in}{2.153107in}}%
\pgfpathcurveto{\pgfqpoint{2.218011in}{2.161343in}}{\pgfqpoint{2.214739in}{2.169243in}}{\pgfqpoint{2.208915in}{2.175067in}}%
\pgfpathcurveto{\pgfqpoint{2.203091in}{2.180891in}}{\pgfqpoint{2.195191in}{2.184164in}}{\pgfqpoint{2.186954in}{2.184164in}}%
\pgfpathcurveto{\pgfqpoint{2.178718in}{2.184164in}}{\pgfqpoint{2.170818in}{2.180891in}}{\pgfqpoint{2.164994in}{2.175067in}}%
\pgfpathcurveto{\pgfqpoint{2.159170in}{2.169243in}}{\pgfqpoint{2.155898in}{2.161343in}}{\pgfqpoint{2.155898in}{2.153107in}}%
\pgfpathcurveto{\pgfqpoint{2.155898in}{2.144871in}}{\pgfqpoint{2.159170in}{2.136971in}}{\pgfqpoint{2.164994in}{2.131147in}}%
\pgfpathcurveto{\pgfqpoint{2.170818in}{2.125323in}}{\pgfqpoint{2.178718in}{2.122051in}}{\pgfqpoint{2.186954in}{2.122051in}}%
\pgfpathclose%
\pgfusepath{stroke,fill}%
\end{pgfscope}%
\begin{pgfscope}%
\pgfpathrectangle{\pgfqpoint{0.100000in}{0.212622in}}{\pgfqpoint{3.696000in}{3.696000in}}%
\pgfusepath{clip}%
\pgfsetbuttcap%
\pgfsetroundjoin%
\definecolor{currentfill}{rgb}{0.121569,0.466667,0.705882}%
\pgfsetfillcolor{currentfill}%
\pgfsetfillopacity{0.398488}%
\pgfsetlinewidth{1.003750pt}%
\definecolor{currentstroke}{rgb}{0.121569,0.466667,0.705882}%
\pgfsetstrokecolor{currentstroke}%
\pgfsetstrokeopacity{0.398488}%
\pgfsetdash{}{0pt}%
\pgfpathmoveto{\pgfqpoint{1.431462in}{1.917768in}}%
\pgfpathcurveto{\pgfqpoint{1.439698in}{1.917768in}}{\pgfqpoint{1.447598in}{1.921040in}}{\pgfqpoint{1.453422in}{1.926864in}}%
\pgfpathcurveto{\pgfqpoint{1.459246in}{1.932688in}}{\pgfqpoint{1.462518in}{1.940588in}}{\pgfqpoint{1.462518in}{1.948825in}}%
\pgfpathcurveto{\pgfqpoint{1.462518in}{1.957061in}}{\pgfqpoint{1.459246in}{1.964961in}}{\pgfqpoint{1.453422in}{1.970785in}}%
\pgfpathcurveto{\pgfqpoint{1.447598in}{1.976609in}}{\pgfqpoint{1.439698in}{1.979881in}}{\pgfqpoint{1.431462in}{1.979881in}}%
\pgfpathcurveto{\pgfqpoint{1.423226in}{1.979881in}}{\pgfqpoint{1.415326in}{1.976609in}}{\pgfqpoint{1.409502in}{1.970785in}}%
\pgfpathcurveto{\pgfqpoint{1.403678in}{1.964961in}}{\pgfqpoint{1.400405in}{1.957061in}}{\pgfqpoint{1.400405in}{1.948825in}}%
\pgfpathcurveto{\pgfqpoint{1.400405in}{1.940588in}}{\pgfqpoint{1.403678in}{1.932688in}}{\pgfqpoint{1.409502in}{1.926864in}}%
\pgfpathcurveto{\pgfqpoint{1.415326in}{1.921040in}}{\pgfqpoint{1.423226in}{1.917768in}}{\pgfqpoint{1.431462in}{1.917768in}}%
\pgfpathclose%
\pgfusepath{stroke,fill}%
\end{pgfscope}%
\begin{pgfscope}%
\pgfpathrectangle{\pgfqpoint{0.100000in}{0.212622in}}{\pgfqpoint{3.696000in}{3.696000in}}%
\pgfusepath{clip}%
\pgfsetbuttcap%
\pgfsetroundjoin%
\definecolor{currentfill}{rgb}{0.121569,0.466667,0.705882}%
\pgfsetfillcolor{currentfill}%
\pgfsetfillopacity{0.399165}%
\pgfsetlinewidth{1.003750pt}%
\definecolor{currentstroke}{rgb}{0.121569,0.466667,0.705882}%
\pgfsetstrokecolor{currentstroke}%
\pgfsetstrokeopacity{0.399165}%
\pgfsetdash{}{0pt}%
\pgfpathmoveto{\pgfqpoint{1.430730in}{1.917497in}}%
\pgfpathcurveto{\pgfqpoint{1.438966in}{1.917497in}}{\pgfqpoint{1.446866in}{1.920769in}}{\pgfqpoint{1.452690in}{1.926593in}}%
\pgfpathcurveto{\pgfqpoint{1.458514in}{1.932417in}}{\pgfqpoint{1.461787in}{1.940317in}}{\pgfqpoint{1.461787in}{1.948554in}}%
\pgfpathcurveto{\pgfqpoint{1.461787in}{1.956790in}}{\pgfqpoint{1.458514in}{1.964690in}}{\pgfqpoint{1.452690in}{1.970514in}}%
\pgfpathcurveto{\pgfqpoint{1.446866in}{1.976338in}}{\pgfqpoint{1.438966in}{1.979610in}}{\pgfqpoint{1.430730in}{1.979610in}}%
\pgfpathcurveto{\pgfqpoint{1.422494in}{1.979610in}}{\pgfqpoint{1.414594in}{1.976338in}}{\pgfqpoint{1.408770in}{1.970514in}}%
\pgfpathcurveto{\pgfqpoint{1.402946in}{1.964690in}}{\pgfqpoint{1.399674in}{1.956790in}}{\pgfqpoint{1.399674in}{1.948554in}}%
\pgfpathcurveto{\pgfqpoint{1.399674in}{1.940317in}}{\pgfqpoint{1.402946in}{1.932417in}}{\pgfqpoint{1.408770in}{1.926593in}}%
\pgfpathcurveto{\pgfqpoint{1.414594in}{1.920769in}}{\pgfqpoint{1.422494in}{1.917497in}}{\pgfqpoint{1.430730in}{1.917497in}}%
\pgfpathclose%
\pgfusepath{stroke,fill}%
\end{pgfscope}%
\begin{pgfscope}%
\pgfpathrectangle{\pgfqpoint{0.100000in}{0.212622in}}{\pgfqpoint{3.696000in}{3.696000in}}%
\pgfusepath{clip}%
\pgfsetbuttcap%
\pgfsetroundjoin%
\definecolor{currentfill}{rgb}{0.121569,0.466667,0.705882}%
\pgfsetfillcolor{currentfill}%
\pgfsetfillopacity{0.399462}%
\pgfsetlinewidth{1.003750pt}%
\definecolor{currentstroke}{rgb}{0.121569,0.466667,0.705882}%
\pgfsetstrokecolor{currentstroke}%
\pgfsetstrokeopacity{0.399462}%
\pgfsetdash{}{0pt}%
\pgfpathmoveto{\pgfqpoint{1.429574in}{1.916894in}}%
\pgfpathcurveto{\pgfqpoint{1.437811in}{1.916894in}}{\pgfqpoint{1.445711in}{1.920166in}}{\pgfqpoint{1.451535in}{1.925990in}}%
\pgfpathcurveto{\pgfqpoint{1.457359in}{1.931814in}}{\pgfqpoint{1.460631in}{1.939714in}}{\pgfqpoint{1.460631in}{1.947951in}}%
\pgfpathcurveto{\pgfqpoint{1.460631in}{1.956187in}}{\pgfqpoint{1.457359in}{1.964087in}}{\pgfqpoint{1.451535in}{1.969911in}}%
\pgfpathcurveto{\pgfqpoint{1.445711in}{1.975735in}}{\pgfqpoint{1.437811in}{1.979007in}}{\pgfqpoint{1.429574in}{1.979007in}}%
\pgfpathcurveto{\pgfqpoint{1.421338in}{1.979007in}}{\pgfqpoint{1.413438in}{1.975735in}}{\pgfqpoint{1.407614in}{1.969911in}}%
\pgfpathcurveto{\pgfqpoint{1.401790in}{1.964087in}}{\pgfqpoint{1.398518in}{1.956187in}}{\pgfqpoint{1.398518in}{1.947951in}}%
\pgfpathcurveto{\pgfqpoint{1.398518in}{1.939714in}}{\pgfqpoint{1.401790in}{1.931814in}}{\pgfqpoint{1.407614in}{1.925990in}}%
\pgfpathcurveto{\pgfqpoint{1.413438in}{1.920166in}}{\pgfqpoint{1.421338in}{1.916894in}}{\pgfqpoint{1.429574in}{1.916894in}}%
\pgfpathclose%
\pgfusepath{stroke,fill}%
\end{pgfscope}%
\begin{pgfscope}%
\pgfpathrectangle{\pgfqpoint{0.100000in}{0.212622in}}{\pgfqpoint{3.696000in}{3.696000in}}%
\pgfusepath{clip}%
\pgfsetbuttcap%
\pgfsetroundjoin%
\definecolor{currentfill}{rgb}{0.121569,0.466667,0.705882}%
\pgfsetfillcolor{currentfill}%
\pgfsetfillopacity{0.399673}%
\pgfsetlinewidth{1.003750pt}%
\definecolor{currentstroke}{rgb}{0.121569,0.466667,0.705882}%
\pgfsetstrokecolor{currentstroke}%
\pgfsetstrokeopacity{0.399673}%
\pgfsetdash{}{0pt}%
\pgfpathmoveto{\pgfqpoint{1.429255in}{1.916665in}}%
\pgfpathcurveto{\pgfqpoint{1.437491in}{1.916665in}}{\pgfqpoint{1.445391in}{1.919938in}}{\pgfqpoint{1.451215in}{1.925762in}}%
\pgfpathcurveto{\pgfqpoint{1.457039in}{1.931585in}}{\pgfqpoint{1.460311in}{1.939486in}}{\pgfqpoint{1.460311in}{1.947722in}}%
\pgfpathcurveto{\pgfqpoint{1.460311in}{1.955958in}}{\pgfqpoint{1.457039in}{1.963858in}}{\pgfqpoint{1.451215in}{1.969682in}}%
\pgfpathcurveto{\pgfqpoint{1.445391in}{1.975506in}}{\pgfqpoint{1.437491in}{1.978778in}}{\pgfqpoint{1.429255in}{1.978778in}}%
\pgfpathcurveto{\pgfqpoint{1.421018in}{1.978778in}}{\pgfqpoint{1.413118in}{1.975506in}}{\pgfqpoint{1.407294in}{1.969682in}}%
\pgfpathcurveto{\pgfqpoint{1.401470in}{1.963858in}}{\pgfqpoint{1.398198in}{1.955958in}}{\pgfqpoint{1.398198in}{1.947722in}}%
\pgfpathcurveto{\pgfqpoint{1.398198in}{1.939486in}}{\pgfqpoint{1.401470in}{1.931585in}}{\pgfqpoint{1.407294in}{1.925762in}}%
\pgfpathcurveto{\pgfqpoint{1.413118in}{1.919938in}}{\pgfqpoint{1.421018in}{1.916665in}}{\pgfqpoint{1.429255in}{1.916665in}}%
\pgfpathclose%
\pgfusepath{stroke,fill}%
\end{pgfscope}%
\begin{pgfscope}%
\pgfpathrectangle{\pgfqpoint{0.100000in}{0.212622in}}{\pgfqpoint{3.696000in}{3.696000in}}%
\pgfusepath{clip}%
\pgfsetbuttcap%
\pgfsetroundjoin%
\definecolor{currentfill}{rgb}{0.121569,0.466667,0.705882}%
\pgfsetfillcolor{currentfill}%
\pgfsetfillopacity{0.399896}%
\pgfsetlinewidth{1.003750pt}%
\definecolor{currentstroke}{rgb}{0.121569,0.466667,0.705882}%
\pgfsetstrokecolor{currentstroke}%
\pgfsetstrokeopacity{0.399896}%
\pgfsetdash{}{0pt}%
\pgfpathmoveto{\pgfqpoint{1.428525in}{1.915193in}}%
\pgfpathcurveto{\pgfqpoint{1.436761in}{1.915193in}}{\pgfqpoint{1.444661in}{1.918465in}}{\pgfqpoint{1.450485in}{1.924289in}}%
\pgfpathcurveto{\pgfqpoint{1.456309in}{1.930113in}}{\pgfqpoint{1.459581in}{1.938013in}}{\pgfqpoint{1.459581in}{1.946249in}}%
\pgfpathcurveto{\pgfqpoint{1.459581in}{1.954486in}}{\pgfqpoint{1.456309in}{1.962386in}}{\pgfqpoint{1.450485in}{1.968210in}}%
\pgfpathcurveto{\pgfqpoint{1.444661in}{1.974034in}}{\pgfqpoint{1.436761in}{1.977306in}}{\pgfqpoint{1.428525in}{1.977306in}}%
\pgfpathcurveto{\pgfqpoint{1.420288in}{1.977306in}}{\pgfqpoint{1.412388in}{1.974034in}}{\pgfqpoint{1.406564in}{1.968210in}}%
\pgfpathcurveto{\pgfqpoint{1.400740in}{1.962386in}}{\pgfqpoint{1.397468in}{1.954486in}}{\pgfqpoint{1.397468in}{1.946249in}}%
\pgfpathcurveto{\pgfqpoint{1.397468in}{1.938013in}}{\pgfqpoint{1.400740in}{1.930113in}}{\pgfqpoint{1.406564in}{1.924289in}}%
\pgfpathcurveto{\pgfqpoint{1.412388in}{1.918465in}}{\pgfqpoint{1.420288in}{1.915193in}}{\pgfqpoint{1.428525in}{1.915193in}}%
\pgfpathclose%
\pgfusepath{stroke,fill}%
\end{pgfscope}%
\begin{pgfscope}%
\pgfpathrectangle{\pgfqpoint{0.100000in}{0.212622in}}{\pgfqpoint{3.696000in}{3.696000in}}%
\pgfusepath{clip}%
\pgfsetbuttcap%
\pgfsetroundjoin%
\definecolor{currentfill}{rgb}{0.121569,0.466667,0.705882}%
\pgfsetfillcolor{currentfill}%
\pgfsetfillopacity{0.400047}%
\pgfsetlinewidth{1.003750pt}%
\definecolor{currentstroke}{rgb}{0.121569,0.466667,0.705882}%
\pgfsetstrokecolor{currentstroke}%
\pgfsetstrokeopacity{0.400047}%
\pgfsetdash{}{0pt}%
\pgfpathmoveto{\pgfqpoint{2.193976in}{2.119796in}}%
\pgfpathcurveto{\pgfqpoint{2.202212in}{2.119796in}}{\pgfqpoint{2.210112in}{2.123068in}}{\pgfqpoint{2.215936in}{2.128892in}}%
\pgfpathcurveto{\pgfqpoint{2.221760in}{2.134716in}}{\pgfqpoint{2.225032in}{2.142616in}}{\pgfqpoint{2.225032in}{2.150853in}}%
\pgfpathcurveto{\pgfqpoint{2.225032in}{2.159089in}}{\pgfqpoint{2.221760in}{2.166989in}}{\pgfqpoint{2.215936in}{2.172813in}}%
\pgfpathcurveto{\pgfqpoint{2.210112in}{2.178637in}}{\pgfqpoint{2.202212in}{2.181909in}}{\pgfqpoint{2.193976in}{2.181909in}}%
\pgfpathcurveto{\pgfqpoint{2.185740in}{2.181909in}}{\pgfqpoint{2.177840in}{2.178637in}}{\pgfqpoint{2.172016in}{2.172813in}}%
\pgfpathcurveto{\pgfqpoint{2.166192in}{2.166989in}}{\pgfqpoint{2.162919in}{2.159089in}}{\pgfqpoint{2.162919in}{2.150853in}}%
\pgfpathcurveto{\pgfqpoint{2.162919in}{2.142616in}}{\pgfqpoint{2.166192in}{2.134716in}}{\pgfqpoint{2.172016in}{2.128892in}}%
\pgfpathcurveto{\pgfqpoint{2.177840in}{2.123068in}}{\pgfqpoint{2.185740in}{2.119796in}}{\pgfqpoint{2.193976in}{2.119796in}}%
\pgfpathclose%
\pgfusepath{stroke,fill}%
\end{pgfscope}%
\begin{pgfscope}%
\pgfpathrectangle{\pgfqpoint{0.100000in}{0.212622in}}{\pgfqpoint{3.696000in}{3.696000in}}%
\pgfusepath{clip}%
\pgfsetbuttcap%
\pgfsetroundjoin%
\definecolor{currentfill}{rgb}{0.121569,0.466667,0.705882}%
\pgfsetfillcolor{currentfill}%
\pgfsetfillopacity{0.400417}%
\pgfsetlinewidth{1.003750pt}%
\definecolor{currentstroke}{rgb}{0.121569,0.466667,0.705882}%
\pgfsetstrokecolor{currentstroke}%
\pgfsetstrokeopacity{0.400417}%
\pgfsetdash{}{0pt}%
\pgfpathmoveto{\pgfqpoint{1.426586in}{1.914082in}}%
\pgfpathcurveto{\pgfqpoint{1.434822in}{1.914082in}}{\pgfqpoint{1.442722in}{1.917355in}}{\pgfqpoint{1.448546in}{1.923179in}}%
\pgfpathcurveto{\pgfqpoint{1.454370in}{1.929003in}}{\pgfqpoint{1.457643in}{1.936903in}}{\pgfqpoint{1.457643in}{1.945139in}}%
\pgfpathcurveto{\pgfqpoint{1.457643in}{1.953375in}}{\pgfqpoint{1.454370in}{1.961275in}}{\pgfqpoint{1.448546in}{1.967099in}}%
\pgfpathcurveto{\pgfqpoint{1.442722in}{1.972923in}}{\pgfqpoint{1.434822in}{1.976195in}}{\pgfqpoint{1.426586in}{1.976195in}}%
\pgfpathcurveto{\pgfqpoint{1.418350in}{1.976195in}}{\pgfqpoint{1.410450in}{1.972923in}}{\pgfqpoint{1.404626in}{1.967099in}}%
\pgfpathcurveto{\pgfqpoint{1.398802in}{1.961275in}}{\pgfqpoint{1.395530in}{1.953375in}}{\pgfqpoint{1.395530in}{1.945139in}}%
\pgfpathcurveto{\pgfqpoint{1.395530in}{1.936903in}}{\pgfqpoint{1.398802in}{1.929003in}}{\pgfqpoint{1.404626in}{1.923179in}}%
\pgfpathcurveto{\pgfqpoint{1.410450in}{1.917355in}}{\pgfqpoint{1.418350in}{1.914082in}}{\pgfqpoint{1.426586in}{1.914082in}}%
\pgfpathclose%
\pgfusepath{stroke,fill}%
\end{pgfscope}%
\begin{pgfscope}%
\pgfpathrectangle{\pgfqpoint{0.100000in}{0.212622in}}{\pgfqpoint{3.696000in}{3.696000in}}%
\pgfusepath{clip}%
\pgfsetbuttcap%
\pgfsetroundjoin%
\definecolor{currentfill}{rgb}{0.121569,0.466667,0.705882}%
\pgfsetfillcolor{currentfill}%
\pgfsetfillopacity{0.400981}%
\pgfsetlinewidth{1.003750pt}%
\definecolor{currentstroke}{rgb}{0.121569,0.466667,0.705882}%
\pgfsetstrokecolor{currentstroke}%
\pgfsetstrokeopacity{0.400981}%
\pgfsetdash{}{0pt}%
\pgfpathmoveto{\pgfqpoint{1.425740in}{1.914177in}}%
\pgfpathcurveto{\pgfqpoint{1.433976in}{1.914177in}}{\pgfqpoint{1.441876in}{1.917449in}}{\pgfqpoint{1.447700in}{1.923273in}}%
\pgfpathcurveto{\pgfqpoint{1.453524in}{1.929097in}}{\pgfqpoint{1.456796in}{1.936997in}}{\pgfqpoint{1.456796in}{1.945234in}}%
\pgfpathcurveto{\pgfqpoint{1.456796in}{1.953470in}}{\pgfqpoint{1.453524in}{1.961370in}}{\pgfqpoint{1.447700in}{1.967194in}}%
\pgfpathcurveto{\pgfqpoint{1.441876in}{1.973018in}}{\pgfqpoint{1.433976in}{1.976290in}}{\pgfqpoint{1.425740in}{1.976290in}}%
\pgfpathcurveto{\pgfqpoint{1.417504in}{1.976290in}}{\pgfqpoint{1.409604in}{1.973018in}}{\pgfqpoint{1.403780in}{1.967194in}}%
\pgfpathcurveto{\pgfqpoint{1.397956in}{1.961370in}}{\pgfqpoint{1.394683in}{1.953470in}}{\pgfqpoint{1.394683in}{1.945234in}}%
\pgfpathcurveto{\pgfqpoint{1.394683in}{1.936997in}}{\pgfqpoint{1.397956in}{1.929097in}}{\pgfqpoint{1.403780in}{1.923273in}}%
\pgfpathcurveto{\pgfqpoint{1.409604in}{1.917449in}}{\pgfqpoint{1.417504in}{1.914177in}}{\pgfqpoint{1.425740in}{1.914177in}}%
\pgfpathclose%
\pgfusepath{stroke,fill}%
\end{pgfscope}%
\begin{pgfscope}%
\pgfpathrectangle{\pgfqpoint{0.100000in}{0.212622in}}{\pgfqpoint{3.696000in}{3.696000in}}%
\pgfusepath{clip}%
\pgfsetbuttcap%
\pgfsetroundjoin%
\definecolor{currentfill}{rgb}{0.121569,0.466667,0.705882}%
\pgfsetfillcolor{currentfill}%
\pgfsetfillopacity{0.401118}%
\pgfsetlinewidth{1.003750pt}%
\definecolor{currentstroke}{rgb}{0.121569,0.466667,0.705882}%
\pgfsetstrokecolor{currentstroke}%
\pgfsetstrokeopacity{0.401118}%
\pgfsetdash{}{0pt}%
\pgfpathmoveto{\pgfqpoint{1.424201in}{1.913199in}}%
\pgfpathcurveto{\pgfqpoint{1.432437in}{1.913199in}}{\pgfqpoint{1.440337in}{1.916471in}}{\pgfqpoint{1.446161in}{1.922295in}}%
\pgfpathcurveto{\pgfqpoint{1.451985in}{1.928119in}}{\pgfqpoint{1.455257in}{1.936019in}}{\pgfqpoint{1.455257in}{1.944255in}}%
\pgfpathcurveto{\pgfqpoint{1.455257in}{1.952491in}}{\pgfqpoint{1.451985in}{1.960391in}}{\pgfqpoint{1.446161in}{1.966215in}}%
\pgfpathcurveto{\pgfqpoint{1.440337in}{1.972039in}}{\pgfqpoint{1.432437in}{1.975312in}}{\pgfqpoint{1.424201in}{1.975312in}}%
\pgfpathcurveto{\pgfqpoint{1.415964in}{1.975312in}}{\pgfqpoint{1.408064in}{1.972039in}}{\pgfqpoint{1.402240in}{1.966215in}}%
\pgfpathcurveto{\pgfqpoint{1.396416in}{1.960391in}}{\pgfqpoint{1.393144in}{1.952491in}}{\pgfqpoint{1.393144in}{1.944255in}}%
\pgfpathcurveto{\pgfqpoint{1.393144in}{1.936019in}}{\pgfqpoint{1.396416in}{1.928119in}}{\pgfqpoint{1.402240in}{1.922295in}}%
\pgfpathcurveto{\pgfqpoint{1.408064in}{1.916471in}}{\pgfqpoint{1.415964in}{1.913199in}}{\pgfqpoint{1.424201in}{1.913199in}}%
\pgfpathclose%
\pgfusepath{stroke,fill}%
\end{pgfscope}%
\begin{pgfscope}%
\pgfpathrectangle{\pgfqpoint{0.100000in}{0.212622in}}{\pgfqpoint{3.696000in}{3.696000in}}%
\pgfusepath{clip}%
\pgfsetbuttcap%
\pgfsetroundjoin%
\definecolor{currentfill}{rgb}{0.121569,0.466667,0.705882}%
\pgfsetfillcolor{currentfill}%
\pgfsetfillopacity{0.401144}%
\pgfsetlinewidth{1.003750pt}%
\definecolor{currentstroke}{rgb}{0.121569,0.466667,0.705882}%
\pgfsetstrokecolor{currentstroke}%
\pgfsetstrokeopacity{0.401144}%
\pgfsetdash{}{0pt}%
\pgfpathmoveto{\pgfqpoint{1.424750in}{1.914365in}}%
\pgfpathcurveto{\pgfqpoint{1.432986in}{1.914365in}}{\pgfqpoint{1.440887in}{1.917638in}}{\pgfqpoint{1.446710in}{1.923462in}}%
\pgfpathcurveto{\pgfqpoint{1.452534in}{1.929286in}}{\pgfqpoint{1.455807in}{1.937186in}}{\pgfqpoint{1.455807in}{1.945422in}}%
\pgfpathcurveto{\pgfqpoint{1.455807in}{1.953658in}}{\pgfqpoint{1.452534in}{1.961558in}}{\pgfqpoint{1.446710in}{1.967382in}}%
\pgfpathcurveto{\pgfqpoint{1.440887in}{1.973206in}}{\pgfqpoint{1.432986in}{1.976478in}}{\pgfqpoint{1.424750in}{1.976478in}}%
\pgfpathcurveto{\pgfqpoint{1.416514in}{1.976478in}}{\pgfqpoint{1.408614in}{1.973206in}}{\pgfqpoint{1.402790in}{1.967382in}}%
\pgfpathcurveto{\pgfqpoint{1.396966in}{1.961558in}}{\pgfqpoint{1.393694in}{1.953658in}}{\pgfqpoint{1.393694in}{1.945422in}}%
\pgfpathcurveto{\pgfqpoint{1.393694in}{1.937186in}}{\pgfqpoint{1.396966in}{1.929286in}}{\pgfqpoint{1.402790in}{1.923462in}}%
\pgfpathcurveto{\pgfqpoint{1.408614in}{1.917638in}}{\pgfqpoint{1.416514in}{1.914365in}}{\pgfqpoint{1.424750in}{1.914365in}}%
\pgfpathclose%
\pgfusepath{stroke,fill}%
\end{pgfscope}%
\begin{pgfscope}%
\pgfpathrectangle{\pgfqpoint{0.100000in}{0.212622in}}{\pgfqpoint{3.696000in}{3.696000in}}%
\pgfusepath{clip}%
\pgfsetbuttcap%
\pgfsetroundjoin%
\definecolor{currentfill}{rgb}{0.121569,0.466667,0.705882}%
\pgfsetfillcolor{currentfill}%
\pgfsetfillopacity{0.401454}%
\pgfsetlinewidth{1.003750pt}%
\definecolor{currentstroke}{rgb}{0.121569,0.466667,0.705882}%
\pgfsetstrokecolor{currentstroke}%
\pgfsetstrokeopacity{0.401454}%
\pgfsetdash{}{0pt}%
\pgfpathmoveto{\pgfqpoint{1.423837in}{1.913075in}}%
\pgfpathcurveto{\pgfqpoint{1.432074in}{1.913075in}}{\pgfqpoint{1.439974in}{1.916348in}}{\pgfqpoint{1.445798in}{1.922172in}}%
\pgfpathcurveto{\pgfqpoint{1.451621in}{1.927996in}}{\pgfqpoint{1.454894in}{1.935896in}}{\pgfqpoint{1.454894in}{1.944132in}}%
\pgfpathcurveto{\pgfqpoint{1.454894in}{1.952368in}}{\pgfqpoint{1.451621in}{1.960268in}}{\pgfqpoint{1.445798in}{1.966092in}}%
\pgfpathcurveto{\pgfqpoint{1.439974in}{1.971916in}}{\pgfqpoint{1.432074in}{1.975188in}}{\pgfqpoint{1.423837in}{1.975188in}}%
\pgfpathcurveto{\pgfqpoint{1.415601in}{1.975188in}}{\pgfqpoint{1.407701in}{1.971916in}}{\pgfqpoint{1.401877in}{1.966092in}}%
\pgfpathcurveto{\pgfqpoint{1.396053in}{1.960268in}}{\pgfqpoint{1.392781in}{1.952368in}}{\pgfqpoint{1.392781in}{1.944132in}}%
\pgfpathcurveto{\pgfqpoint{1.392781in}{1.935896in}}{\pgfqpoint{1.396053in}{1.927996in}}{\pgfqpoint{1.401877in}{1.922172in}}%
\pgfpathcurveto{\pgfqpoint{1.407701in}{1.916348in}}{\pgfqpoint{1.415601in}{1.913075in}}{\pgfqpoint{1.423837in}{1.913075in}}%
\pgfpathclose%
\pgfusepath{stroke,fill}%
\end{pgfscope}%
\begin{pgfscope}%
\pgfpathrectangle{\pgfqpoint{0.100000in}{0.212622in}}{\pgfqpoint{3.696000in}{3.696000in}}%
\pgfusepath{clip}%
\pgfsetbuttcap%
\pgfsetroundjoin%
\definecolor{currentfill}{rgb}{0.121569,0.466667,0.705882}%
\pgfsetfillcolor{currentfill}%
\pgfsetfillopacity{0.401793}%
\pgfsetlinewidth{1.003750pt}%
\definecolor{currentstroke}{rgb}{0.121569,0.466667,0.705882}%
\pgfsetstrokecolor{currentstroke}%
\pgfsetstrokeopacity{0.401793}%
\pgfsetdash{}{0pt}%
\pgfpathmoveto{\pgfqpoint{1.422223in}{1.911835in}}%
\pgfpathcurveto{\pgfqpoint{1.430459in}{1.911835in}}{\pgfqpoint{1.438359in}{1.915108in}}{\pgfqpoint{1.444183in}{1.920932in}}%
\pgfpathcurveto{\pgfqpoint{1.450007in}{1.926756in}}{\pgfqpoint{1.453279in}{1.934656in}}{\pgfqpoint{1.453279in}{1.942892in}}%
\pgfpathcurveto{\pgfqpoint{1.453279in}{1.951128in}}{\pgfqpoint{1.450007in}{1.959028in}}{\pgfqpoint{1.444183in}{1.964852in}}%
\pgfpathcurveto{\pgfqpoint{1.438359in}{1.970676in}}{\pgfqpoint{1.430459in}{1.973948in}}{\pgfqpoint{1.422223in}{1.973948in}}%
\pgfpathcurveto{\pgfqpoint{1.413986in}{1.973948in}}{\pgfqpoint{1.406086in}{1.970676in}}{\pgfqpoint{1.400262in}{1.964852in}}%
\pgfpathcurveto{\pgfqpoint{1.394438in}{1.959028in}}{\pgfqpoint{1.391166in}{1.951128in}}{\pgfqpoint{1.391166in}{1.942892in}}%
\pgfpathcurveto{\pgfqpoint{1.391166in}{1.934656in}}{\pgfqpoint{1.394438in}{1.926756in}}{\pgfqpoint{1.400262in}{1.920932in}}%
\pgfpathcurveto{\pgfqpoint{1.406086in}{1.915108in}}{\pgfqpoint{1.413986in}{1.911835in}}{\pgfqpoint{1.422223in}{1.911835in}}%
\pgfpathclose%
\pgfusepath{stroke,fill}%
\end{pgfscope}%
\begin{pgfscope}%
\pgfpathrectangle{\pgfqpoint{0.100000in}{0.212622in}}{\pgfqpoint{3.696000in}{3.696000in}}%
\pgfusepath{clip}%
\pgfsetbuttcap%
\pgfsetroundjoin%
\definecolor{currentfill}{rgb}{0.121569,0.466667,0.705882}%
\pgfsetfillcolor{currentfill}%
\pgfsetfillopacity{0.402658}%
\pgfsetlinewidth{1.003750pt}%
\definecolor{currentstroke}{rgb}{0.121569,0.466667,0.705882}%
\pgfsetstrokecolor{currentstroke}%
\pgfsetstrokeopacity{0.402658}%
\pgfsetdash{}{0pt}%
\pgfpathmoveto{\pgfqpoint{1.420301in}{1.910108in}}%
\pgfpathcurveto{\pgfqpoint{1.428538in}{1.910108in}}{\pgfqpoint{1.436438in}{1.913381in}}{\pgfqpoint{1.442262in}{1.919205in}}%
\pgfpathcurveto{\pgfqpoint{1.448086in}{1.925028in}}{\pgfqpoint{1.451358in}{1.932928in}}{\pgfqpoint{1.451358in}{1.941165in}}%
\pgfpathcurveto{\pgfqpoint{1.451358in}{1.949401in}}{\pgfqpoint{1.448086in}{1.957301in}}{\pgfqpoint{1.442262in}{1.963125in}}%
\pgfpathcurveto{\pgfqpoint{1.436438in}{1.968949in}}{\pgfqpoint{1.428538in}{1.972221in}}{\pgfqpoint{1.420301in}{1.972221in}}%
\pgfpathcurveto{\pgfqpoint{1.412065in}{1.972221in}}{\pgfqpoint{1.404165in}{1.968949in}}{\pgfqpoint{1.398341in}{1.963125in}}%
\pgfpathcurveto{\pgfqpoint{1.392517in}{1.957301in}}{\pgfqpoint{1.389245in}{1.949401in}}{\pgfqpoint{1.389245in}{1.941165in}}%
\pgfpathcurveto{\pgfqpoint{1.389245in}{1.932928in}}{\pgfqpoint{1.392517in}{1.925028in}}{\pgfqpoint{1.398341in}{1.919205in}}%
\pgfpathcurveto{\pgfqpoint{1.404165in}{1.913381in}}{\pgfqpoint{1.412065in}{1.910108in}}{\pgfqpoint{1.420301in}{1.910108in}}%
\pgfpathclose%
\pgfusepath{stroke,fill}%
\end{pgfscope}%
\begin{pgfscope}%
\pgfpathrectangle{\pgfqpoint{0.100000in}{0.212622in}}{\pgfqpoint{3.696000in}{3.696000in}}%
\pgfusepath{clip}%
\pgfsetbuttcap%
\pgfsetroundjoin%
\definecolor{currentfill}{rgb}{0.121569,0.466667,0.705882}%
\pgfsetfillcolor{currentfill}%
\pgfsetfillopacity{0.402838}%
\pgfsetlinewidth{1.003750pt}%
\definecolor{currentstroke}{rgb}{0.121569,0.466667,0.705882}%
\pgfsetstrokecolor{currentstroke}%
\pgfsetstrokeopacity{0.402838}%
\pgfsetdash{}{0pt}%
\pgfpathmoveto{\pgfqpoint{2.203850in}{2.124399in}}%
\pgfpathcurveto{\pgfqpoint{2.212087in}{2.124399in}}{\pgfqpoint{2.219987in}{2.127672in}}{\pgfqpoint{2.225811in}{2.133496in}}%
\pgfpathcurveto{\pgfqpoint{2.231635in}{2.139320in}}{\pgfqpoint{2.234907in}{2.147220in}}{\pgfqpoint{2.234907in}{2.155456in}}%
\pgfpathcurveto{\pgfqpoint{2.234907in}{2.163692in}}{\pgfqpoint{2.231635in}{2.171592in}}{\pgfqpoint{2.225811in}{2.177416in}}%
\pgfpathcurveto{\pgfqpoint{2.219987in}{2.183240in}}{\pgfqpoint{2.212087in}{2.186512in}}{\pgfqpoint{2.203850in}{2.186512in}}%
\pgfpathcurveto{\pgfqpoint{2.195614in}{2.186512in}}{\pgfqpoint{2.187714in}{2.183240in}}{\pgfqpoint{2.181890in}{2.177416in}}%
\pgfpathcurveto{\pgfqpoint{2.176066in}{2.171592in}}{\pgfqpoint{2.172794in}{2.163692in}}{\pgfqpoint{2.172794in}{2.155456in}}%
\pgfpathcurveto{\pgfqpoint{2.172794in}{2.147220in}}{\pgfqpoint{2.176066in}{2.139320in}}{\pgfqpoint{2.181890in}{2.133496in}}%
\pgfpathcurveto{\pgfqpoint{2.187714in}{2.127672in}}{\pgfqpoint{2.195614in}{2.124399in}}{\pgfqpoint{2.203850in}{2.124399in}}%
\pgfpathclose%
\pgfusepath{stroke,fill}%
\end{pgfscope}%
\begin{pgfscope}%
\pgfpathrectangle{\pgfqpoint{0.100000in}{0.212622in}}{\pgfqpoint{3.696000in}{3.696000in}}%
\pgfusepath{clip}%
\pgfsetbuttcap%
\pgfsetroundjoin%
\definecolor{currentfill}{rgb}{0.121569,0.466667,0.705882}%
\pgfsetfillcolor{currentfill}%
\pgfsetfillopacity{0.404237}%
\pgfsetlinewidth{1.003750pt}%
\definecolor{currentstroke}{rgb}{0.121569,0.466667,0.705882}%
\pgfsetstrokecolor{currentstroke}%
\pgfsetstrokeopacity{0.404237}%
\pgfsetdash{}{0pt}%
\pgfpathmoveto{\pgfqpoint{1.418119in}{1.906102in}}%
\pgfpathcurveto{\pgfqpoint{1.426355in}{1.906102in}}{\pgfqpoint{1.434255in}{1.909374in}}{\pgfqpoint{1.440079in}{1.915198in}}%
\pgfpathcurveto{\pgfqpoint{1.445903in}{1.921022in}}{\pgfqpoint{1.449175in}{1.928922in}}{\pgfqpoint{1.449175in}{1.937158in}}%
\pgfpathcurveto{\pgfqpoint{1.449175in}{1.945394in}}{\pgfqpoint{1.445903in}{1.953294in}}{\pgfqpoint{1.440079in}{1.959118in}}%
\pgfpathcurveto{\pgfqpoint{1.434255in}{1.964942in}}{\pgfqpoint{1.426355in}{1.968215in}}{\pgfqpoint{1.418119in}{1.968215in}}%
\pgfpathcurveto{\pgfqpoint{1.409882in}{1.968215in}}{\pgfqpoint{1.401982in}{1.964942in}}{\pgfqpoint{1.396158in}{1.959118in}}%
\pgfpathcurveto{\pgfqpoint{1.390334in}{1.953294in}}{\pgfqpoint{1.387062in}{1.945394in}}{\pgfqpoint{1.387062in}{1.937158in}}%
\pgfpathcurveto{\pgfqpoint{1.387062in}{1.928922in}}{\pgfqpoint{1.390334in}{1.921022in}}{\pgfqpoint{1.396158in}{1.915198in}}%
\pgfpathcurveto{\pgfqpoint{1.401982in}{1.909374in}}{\pgfqpoint{1.409882in}{1.906102in}}{\pgfqpoint{1.418119in}{1.906102in}}%
\pgfpathclose%
\pgfusepath{stroke,fill}%
\end{pgfscope}%
\begin{pgfscope}%
\pgfpathrectangle{\pgfqpoint{0.100000in}{0.212622in}}{\pgfqpoint{3.696000in}{3.696000in}}%
\pgfusepath{clip}%
\pgfsetbuttcap%
\pgfsetroundjoin%
\definecolor{currentfill}{rgb}{0.121569,0.466667,0.705882}%
\pgfsetfillcolor{currentfill}%
\pgfsetfillopacity{0.405566}%
\pgfsetlinewidth{1.003750pt}%
\definecolor{currentstroke}{rgb}{0.121569,0.466667,0.705882}%
\pgfsetstrokecolor{currentstroke}%
\pgfsetstrokeopacity{0.405566}%
\pgfsetdash{}{0pt}%
\pgfpathmoveto{\pgfqpoint{2.213615in}{2.126122in}}%
\pgfpathcurveto{\pgfqpoint{2.221851in}{2.126122in}}{\pgfqpoint{2.229751in}{2.129395in}}{\pgfqpoint{2.235575in}{2.135218in}}%
\pgfpathcurveto{\pgfqpoint{2.241399in}{2.141042in}}{\pgfqpoint{2.244671in}{2.148942in}}{\pgfqpoint{2.244671in}{2.157179in}}%
\pgfpathcurveto{\pgfqpoint{2.244671in}{2.165415in}}{\pgfqpoint{2.241399in}{2.173315in}}{\pgfqpoint{2.235575in}{2.179139in}}%
\pgfpathcurveto{\pgfqpoint{2.229751in}{2.184963in}}{\pgfqpoint{2.221851in}{2.188235in}}{\pgfqpoint{2.213615in}{2.188235in}}%
\pgfpathcurveto{\pgfqpoint{2.205379in}{2.188235in}}{\pgfqpoint{2.197479in}{2.184963in}}{\pgfqpoint{2.191655in}{2.179139in}}%
\pgfpathcurveto{\pgfqpoint{2.185831in}{2.173315in}}{\pgfqpoint{2.182558in}{2.165415in}}{\pgfqpoint{2.182558in}{2.157179in}}%
\pgfpathcurveto{\pgfqpoint{2.182558in}{2.148942in}}{\pgfqpoint{2.185831in}{2.141042in}}{\pgfqpoint{2.191655in}{2.135218in}}%
\pgfpathcurveto{\pgfqpoint{2.197479in}{2.129395in}}{\pgfqpoint{2.205379in}{2.126122in}}{\pgfqpoint{2.213615in}{2.126122in}}%
\pgfpathclose%
\pgfusepath{stroke,fill}%
\end{pgfscope}%
\begin{pgfscope}%
\pgfpathrectangle{\pgfqpoint{0.100000in}{0.212622in}}{\pgfqpoint{3.696000in}{3.696000in}}%
\pgfusepath{clip}%
\pgfsetbuttcap%
\pgfsetroundjoin%
\definecolor{currentfill}{rgb}{0.121569,0.466667,0.705882}%
\pgfsetfillcolor{currentfill}%
\pgfsetfillopacity{0.405582}%
\pgfsetlinewidth{1.003750pt}%
\definecolor{currentstroke}{rgb}{0.121569,0.466667,0.705882}%
\pgfsetstrokecolor{currentstroke}%
\pgfsetstrokeopacity{0.405582}%
\pgfsetdash{}{0pt}%
\pgfpathmoveto{\pgfqpoint{1.413289in}{1.904233in}}%
\pgfpathcurveto{\pgfqpoint{1.421525in}{1.904233in}}{\pgfqpoint{1.429425in}{1.907505in}}{\pgfqpoint{1.435249in}{1.913329in}}%
\pgfpathcurveto{\pgfqpoint{1.441073in}{1.919153in}}{\pgfqpoint{1.444345in}{1.927053in}}{\pgfqpoint{1.444345in}{1.935290in}}%
\pgfpathcurveto{\pgfqpoint{1.444345in}{1.943526in}}{\pgfqpoint{1.441073in}{1.951426in}}{\pgfqpoint{1.435249in}{1.957250in}}%
\pgfpathcurveto{\pgfqpoint{1.429425in}{1.963074in}}{\pgfqpoint{1.421525in}{1.966346in}}{\pgfqpoint{1.413289in}{1.966346in}}%
\pgfpathcurveto{\pgfqpoint{1.405053in}{1.966346in}}{\pgfqpoint{1.397153in}{1.963074in}}{\pgfqpoint{1.391329in}{1.957250in}}%
\pgfpathcurveto{\pgfqpoint{1.385505in}{1.951426in}}{\pgfqpoint{1.382232in}{1.943526in}}{\pgfqpoint{1.382232in}{1.935290in}}%
\pgfpathcurveto{\pgfqpoint{1.382232in}{1.927053in}}{\pgfqpoint{1.385505in}{1.919153in}}{\pgfqpoint{1.391329in}{1.913329in}}%
\pgfpathcurveto{\pgfqpoint{1.397153in}{1.907505in}}{\pgfqpoint{1.405053in}{1.904233in}}{\pgfqpoint{1.413289in}{1.904233in}}%
\pgfpathclose%
\pgfusepath{stroke,fill}%
\end{pgfscope}%
\begin{pgfscope}%
\pgfpathrectangle{\pgfqpoint{0.100000in}{0.212622in}}{\pgfqpoint{3.696000in}{3.696000in}}%
\pgfusepath{clip}%
\pgfsetbuttcap%
\pgfsetroundjoin%
\definecolor{currentfill}{rgb}{0.121569,0.466667,0.705882}%
\pgfsetfillcolor{currentfill}%
\pgfsetfillopacity{0.407010}%
\pgfsetlinewidth{1.003750pt}%
\definecolor{currentstroke}{rgb}{0.121569,0.466667,0.705882}%
\pgfsetstrokecolor{currentstroke}%
\pgfsetstrokeopacity{0.407010}%
\pgfsetdash{}{0pt}%
\pgfpathmoveto{\pgfqpoint{1.411420in}{1.901129in}}%
\pgfpathcurveto{\pgfqpoint{1.419656in}{1.901129in}}{\pgfqpoint{1.427556in}{1.904401in}}{\pgfqpoint{1.433380in}{1.910225in}}%
\pgfpathcurveto{\pgfqpoint{1.439204in}{1.916049in}}{\pgfqpoint{1.442476in}{1.923949in}}{\pgfqpoint{1.442476in}{1.932186in}}%
\pgfpathcurveto{\pgfqpoint{1.442476in}{1.940422in}}{\pgfqpoint{1.439204in}{1.948322in}}{\pgfqpoint{1.433380in}{1.954146in}}%
\pgfpathcurveto{\pgfqpoint{1.427556in}{1.959970in}}{\pgfqpoint{1.419656in}{1.963242in}}{\pgfqpoint{1.411420in}{1.963242in}}%
\pgfpathcurveto{\pgfqpoint{1.403183in}{1.963242in}}{\pgfqpoint{1.395283in}{1.959970in}}{\pgfqpoint{1.389459in}{1.954146in}}%
\pgfpathcurveto{\pgfqpoint{1.383635in}{1.948322in}}{\pgfqpoint{1.380363in}{1.940422in}}{\pgfqpoint{1.380363in}{1.932186in}}%
\pgfpathcurveto{\pgfqpoint{1.380363in}{1.923949in}}{\pgfqpoint{1.383635in}{1.916049in}}{\pgfqpoint{1.389459in}{1.910225in}}%
\pgfpathcurveto{\pgfqpoint{1.395283in}{1.904401in}}{\pgfqpoint{1.403183in}{1.901129in}}{\pgfqpoint{1.411420in}{1.901129in}}%
\pgfpathclose%
\pgfusepath{stroke,fill}%
\end{pgfscope}%
\begin{pgfscope}%
\pgfpathrectangle{\pgfqpoint{0.100000in}{0.212622in}}{\pgfqpoint{3.696000in}{3.696000in}}%
\pgfusepath{clip}%
\pgfsetbuttcap%
\pgfsetroundjoin%
\definecolor{currentfill}{rgb}{0.121569,0.466667,0.705882}%
\pgfsetfillcolor{currentfill}%
\pgfsetfillopacity{0.407726}%
\pgfsetlinewidth{1.003750pt}%
\definecolor{currentstroke}{rgb}{0.121569,0.466667,0.705882}%
\pgfsetstrokecolor{currentstroke}%
\pgfsetstrokeopacity{0.407726}%
\pgfsetdash{}{0pt}%
\pgfpathmoveto{\pgfqpoint{1.409430in}{1.897133in}}%
\pgfpathcurveto{\pgfqpoint{1.417666in}{1.897133in}}{\pgfqpoint{1.425566in}{1.900405in}}{\pgfqpoint{1.431390in}{1.906229in}}%
\pgfpathcurveto{\pgfqpoint{1.437214in}{1.912053in}}{\pgfqpoint{1.440487in}{1.919953in}}{\pgfqpoint{1.440487in}{1.928189in}}%
\pgfpathcurveto{\pgfqpoint{1.440487in}{1.936425in}}{\pgfqpoint{1.437214in}{1.944325in}}{\pgfqpoint{1.431390in}{1.950149in}}%
\pgfpathcurveto{\pgfqpoint{1.425566in}{1.955973in}}{\pgfqpoint{1.417666in}{1.959246in}}{\pgfqpoint{1.409430in}{1.959246in}}%
\pgfpathcurveto{\pgfqpoint{1.401194in}{1.959246in}}{\pgfqpoint{1.393294in}{1.955973in}}{\pgfqpoint{1.387470in}{1.950149in}}%
\pgfpathcurveto{\pgfqpoint{1.381646in}{1.944325in}}{\pgfqpoint{1.378374in}{1.936425in}}{\pgfqpoint{1.378374in}{1.928189in}}%
\pgfpathcurveto{\pgfqpoint{1.378374in}{1.919953in}}{\pgfqpoint{1.381646in}{1.912053in}}{\pgfqpoint{1.387470in}{1.906229in}}%
\pgfpathcurveto{\pgfqpoint{1.393294in}{1.900405in}}{\pgfqpoint{1.401194in}{1.897133in}}{\pgfqpoint{1.409430in}{1.897133in}}%
\pgfpathclose%
\pgfusepath{stroke,fill}%
\end{pgfscope}%
\begin{pgfscope}%
\pgfpathrectangle{\pgfqpoint{0.100000in}{0.212622in}}{\pgfqpoint{3.696000in}{3.696000in}}%
\pgfusepath{clip}%
\pgfsetbuttcap%
\pgfsetroundjoin%
\definecolor{currentfill}{rgb}{0.121569,0.466667,0.705882}%
\pgfsetfillcolor{currentfill}%
\pgfsetfillopacity{0.408119}%
\pgfsetlinewidth{1.003750pt}%
\definecolor{currentstroke}{rgb}{0.121569,0.466667,0.705882}%
\pgfsetstrokecolor{currentstroke}%
\pgfsetstrokeopacity{0.408119}%
\pgfsetdash{}{0pt}%
\pgfpathmoveto{\pgfqpoint{1.407821in}{1.895523in}}%
\pgfpathcurveto{\pgfqpoint{1.416058in}{1.895523in}}{\pgfqpoint{1.423958in}{1.898795in}}{\pgfqpoint{1.429782in}{1.904619in}}%
\pgfpathcurveto{\pgfqpoint{1.435606in}{1.910443in}}{\pgfqpoint{1.438878in}{1.918343in}}{\pgfqpoint{1.438878in}{1.926580in}}%
\pgfpathcurveto{\pgfqpoint{1.438878in}{1.934816in}}{\pgfqpoint{1.435606in}{1.942716in}}{\pgfqpoint{1.429782in}{1.948540in}}%
\pgfpathcurveto{\pgfqpoint{1.423958in}{1.954364in}}{\pgfqpoint{1.416058in}{1.957636in}}{\pgfqpoint{1.407821in}{1.957636in}}%
\pgfpathcurveto{\pgfqpoint{1.399585in}{1.957636in}}{\pgfqpoint{1.391685in}{1.954364in}}{\pgfqpoint{1.385861in}{1.948540in}}%
\pgfpathcurveto{\pgfqpoint{1.380037in}{1.942716in}}{\pgfqpoint{1.376765in}{1.934816in}}{\pgfqpoint{1.376765in}{1.926580in}}%
\pgfpathcurveto{\pgfqpoint{1.376765in}{1.918343in}}{\pgfqpoint{1.380037in}{1.910443in}}{\pgfqpoint{1.385861in}{1.904619in}}%
\pgfpathcurveto{\pgfqpoint{1.391685in}{1.898795in}}{\pgfqpoint{1.399585in}{1.895523in}}{\pgfqpoint{1.407821in}{1.895523in}}%
\pgfpathclose%
\pgfusepath{stroke,fill}%
\end{pgfscope}%
\begin{pgfscope}%
\pgfpathrectangle{\pgfqpoint{0.100000in}{0.212622in}}{\pgfqpoint{3.696000in}{3.696000in}}%
\pgfusepath{clip}%
\pgfsetbuttcap%
\pgfsetroundjoin%
\definecolor{currentfill}{rgb}{0.121569,0.466667,0.705882}%
\pgfsetfillcolor{currentfill}%
\pgfsetfillopacity{0.408199}%
\pgfsetlinewidth{1.003750pt}%
\definecolor{currentstroke}{rgb}{0.121569,0.466667,0.705882}%
\pgfsetstrokecolor{currentstroke}%
\pgfsetstrokeopacity{0.408199}%
\pgfsetdash{}{0pt}%
\pgfpathmoveto{\pgfqpoint{2.225129in}{2.125908in}}%
\pgfpathcurveto{\pgfqpoint{2.233365in}{2.125908in}}{\pgfqpoint{2.241265in}{2.129180in}}{\pgfqpoint{2.247089in}{2.135004in}}%
\pgfpathcurveto{\pgfqpoint{2.252913in}{2.140828in}}{\pgfqpoint{2.256185in}{2.148728in}}{\pgfqpoint{2.256185in}{2.156964in}}%
\pgfpathcurveto{\pgfqpoint{2.256185in}{2.165201in}}{\pgfqpoint{2.252913in}{2.173101in}}{\pgfqpoint{2.247089in}{2.178925in}}%
\pgfpathcurveto{\pgfqpoint{2.241265in}{2.184748in}}{\pgfqpoint{2.233365in}{2.188021in}}{\pgfqpoint{2.225129in}{2.188021in}}%
\pgfpathcurveto{\pgfqpoint{2.216892in}{2.188021in}}{\pgfqpoint{2.208992in}{2.184748in}}{\pgfqpoint{2.203168in}{2.178925in}}%
\pgfpathcurveto{\pgfqpoint{2.197344in}{2.173101in}}{\pgfqpoint{2.194072in}{2.165201in}}{\pgfqpoint{2.194072in}{2.156964in}}%
\pgfpathcurveto{\pgfqpoint{2.194072in}{2.148728in}}{\pgfqpoint{2.197344in}{2.140828in}}{\pgfqpoint{2.203168in}{2.135004in}}%
\pgfpathcurveto{\pgfqpoint{2.208992in}{2.129180in}}{\pgfqpoint{2.216892in}{2.125908in}}{\pgfqpoint{2.225129in}{2.125908in}}%
\pgfpathclose%
\pgfusepath{stroke,fill}%
\end{pgfscope}%
\begin{pgfscope}%
\pgfpathrectangle{\pgfqpoint{0.100000in}{0.212622in}}{\pgfqpoint{3.696000in}{3.696000in}}%
\pgfusepath{clip}%
\pgfsetbuttcap%
\pgfsetroundjoin%
\definecolor{currentfill}{rgb}{0.121569,0.466667,0.705882}%
\pgfsetfillcolor{currentfill}%
\pgfsetfillopacity{0.409484}%
\pgfsetlinewidth{1.003750pt}%
\definecolor{currentstroke}{rgb}{0.121569,0.466667,0.705882}%
\pgfsetstrokecolor{currentstroke}%
\pgfsetstrokeopacity{0.409484}%
\pgfsetdash{}{0pt}%
\pgfpathmoveto{\pgfqpoint{1.406657in}{1.895838in}}%
\pgfpathcurveto{\pgfqpoint{1.414894in}{1.895838in}}{\pgfqpoint{1.422794in}{1.899111in}}{\pgfqpoint{1.428618in}{1.904935in}}%
\pgfpathcurveto{\pgfqpoint{1.434442in}{1.910759in}}{\pgfqpoint{1.437714in}{1.918659in}}{\pgfqpoint{1.437714in}{1.926895in}}%
\pgfpathcurveto{\pgfqpoint{1.437714in}{1.935131in}}{\pgfqpoint{1.434442in}{1.943031in}}{\pgfqpoint{1.428618in}{1.948855in}}%
\pgfpathcurveto{\pgfqpoint{1.422794in}{1.954679in}}{\pgfqpoint{1.414894in}{1.957951in}}{\pgfqpoint{1.406657in}{1.957951in}}%
\pgfpathcurveto{\pgfqpoint{1.398421in}{1.957951in}}{\pgfqpoint{1.390521in}{1.954679in}}{\pgfqpoint{1.384697in}{1.948855in}}%
\pgfpathcurveto{\pgfqpoint{1.378873in}{1.943031in}}{\pgfqpoint{1.375601in}{1.935131in}}{\pgfqpoint{1.375601in}{1.926895in}}%
\pgfpathcurveto{\pgfqpoint{1.375601in}{1.918659in}}{\pgfqpoint{1.378873in}{1.910759in}}{\pgfqpoint{1.384697in}{1.904935in}}%
\pgfpathcurveto{\pgfqpoint{1.390521in}{1.899111in}}{\pgfqpoint{1.398421in}{1.895838in}}{\pgfqpoint{1.406657in}{1.895838in}}%
\pgfpathclose%
\pgfusepath{stroke,fill}%
\end{pgfscope}%
\begin{pgfscope}%
\pgfpathrectangle{\pgfqpoint{0.100000in}{0.212622in}}{\pgfqpoint{3.696000in}{3.696000in}}%
\pgfusepath{clip}%
\pgfsetbuttcap%
\pgfsetroundjoin%
\definecolor{currentfill}{rgb}{0.121569,0.466667,0.705882}%
\pgfsetfillcolor{currentfill}%
\pgfsetfillopacity{0.410179}%
\pgfsetlinewidth{1.003750pt}%
\definecolor{currentstroke}{rgb}{0.121569,0.466667,0.705882}%
\pgfsetstrokecolor{currentstroke}%
\pgfsetstrokeopacity{0.410179}%
\pgfsetdash{}{0pt}%
\pgfpathmoveto{\pgfqpoint{1.403913in}{1.894309in}}%
\pgfpathcurveto{\pgfqpoint{1.412149in}{1.894309in}}{\pgfqpoint{1.420049in}{1.897581in}}{\pgfqpoint{1.425873in}{1.903405in}}%
\pgfpathcurveto{\pgfqpoint{1.431697in}{1.909229in}}{\pgfqpoint{1.434969in}{1.917129in}}{\pgfqpoint{1.434969in}{1.925366in}}%
\pgfpathcurveto{\pgfqpoint{1.434969in}{1.933602in}}{\pgfqpoint{1.431697in}{1.941502in}}{\pgfqpoint{1.425873in}{1.947326in}}%
\pgfpathcurveto{\pgfqpoint{1.420049in}{1.953150in}}{\pgfqpoint{1.412149in}{1.956422in}}{\pgfqpoint{1.403913in}{1.956422in}}%
\pgfpathcurveto{\pgfqpoint{1.395676in}{1.956422in}}{\pgfqpoint{1.387776in}{1.953150in}}{\pgfqpoint{1.381952in}{1.947326in}}%
\pgfpathcurveto{\pgfqpoint{1.376128in}{1.941502in}}{\pgfqpoint{1.372856in}{1.933602in}}{\pgfqpoint{1.372856in}{1.925366in}}%
\pgfpathcurveto{\pgfqpoint{1.372856in}{1.917129in}}{\pgfqpoint{1.376128in}{1.909229in}}{\pgfqpoint{1.381952in}{1.903405in}}%
\pgfpathcurveto{\pgfqpoint{1.387776in}{1.897581in}}{\pgfqpoint{1.395676in}{1.894309in}}{\pgfqpoint{1.403913in}{1.894309in}}%
\pgfpathclose%
\pgfusepath{stroke,fill}%
\end{pgfscope}%
\begin{pgfscope}%
\pgfpathrectangle{\pgfqpoint{0.100000in}{0.212622in}}{\pgfqpoint{3.696000in}{3.696000in}}%
\pgfusepath{clip}%
\pgfsetbuttcap%
\pgfsetroundjoin%
\definecolor{currentfill}{rgb}{0.121569,0.466667,0.705882}%
\pgfsetfillcolor{currentfill}%
\pgfsetfillopacity{0.410494}%
\pgfsetlinewidth{1.003750pt}%
\definecolor{currentstroke}{rgb}{0.121569,0.466667,0.705882}%
\pgfsetstrokecolor{currentstroke}%
\pgfsetstrokeopacity{0.410494}%
\pgfsetdash{}{0pt}%
\pgfpathmoveto{\pgfqpoint{2.236933in}{2.121238in}}%
\pgfpathcurveto{\pgfqpoint{2.245169in}{2.121238in}}{\pgfqpoint{2.253069in}{2.124510in}}{\pgfqpoint{2.258893in}{2.130334in}}%
\pgfpathcurveto{\pgfqpoint{2.264717in}{2.136158in}}{\pgfqpoint{2.267990in}{2.144058in}}{\pgfqpoint{2.267990in}{2.152295in}}%
\pgfpathcurveto{\pgfqpoint{2.267990in}{2.160531in}}{\pgfqpoint{2.264717in}{2.168431in}}{\pgfqpoint{2.258893in}{2.174255in}}%
\pgfpathcurveto{\pgfqpoint{2.253069in}{2.180079in}}{\pgfqpoint{2.245169in}{2.183351in}}{\pgfqpoint{2.236933in}{2.183351in}}%
\pgfpathcurveto{\pgfqpoint{2.228697in}{2.183351in}}{\pgfqpoint{2.220797in}{2.180079in}}{\pgfqpoint{2.214973in}{2.174255in}}%
\pgfpathcurveto{\pgfqpoint{2.209149in}{2.168431in}}{\pgfqpoint{2.205877in}{2.160531in}}{\pgfqpoint{2.205877in}{2.152295in}}%
\pgfpathcurveto{\pgfqpoint{2.205877in}{2.144058in}}{\pgfqpoint{2.209149in}{2.136158in}}{\pgfqpoint{2.214973in}{2.130334in}}%
\pgfpathcurveto{\pgfqpoint{2.220797in}{2.124510in}}{\pgfqpoint{2.228697in}{2.121238in}}{\pgfqpoint{2.236933in}{2.121238in}}%
\pgfpathclose%
\pgfusepath{stroke,fill}%
\end{pgfscope}%
\begin{pgfscope}%
\pgfpathrectangle{\pgfqpoint{0.100000in}{0.212622in}}{\pgfqpoint{3.696000in}{3.696000in}}%
\pgfusepath{clip}%
\pgfsetbuttcap%
\pgfsetroundjoin%
\definecolor{currentfill}{rgb}{0.121569,0.466667,0.705882}%
\pgfsetfillcolor{currentfill}%
\pgfsetfillopacity{0.410833}%
\pgfsetlinewidth{1.003750pt}%
\definecolor{currentstroke}{rgb}{0.121569,0.466667,0.705882}%
\pgfsetstrokecolor{currentstroke}%
\pgfsetstrokeopacity{0.410833}%
\pgfsetdash{}{0pt}%
\pgfpathmoveto{\pgfqpoint{1.402801in}{1.892777in}}%
\pgfpathcurveto{\pgfqpoint{1.411038in}{1.892777in}}{\pgfqpoint{1.418938in}{1.896050in}}{\pgfqpoint{1.424762in}{1.901874in}}%
\pgfpathcurveto{\pgfqpoint{1.430585in}{1.907698in}}{\pgfqpoint{1.433858in}{1.915598in}}{\pgfqpoint{1.433858in}{1.923834in}}%
\pgfpathcurveto{\pgfqpoint{1.433858in}{1.932070in}}{\pgfqpoint{1.430585in}{1.939970in}}{\pgfqpoint{1.424762in}{1.945794in}}%
\pgfpathcurveto{\pgfqpoint{1.418938in}{1.951618in}}{\pgfqpoint{1.411038in}{1.954890in}}{\pgfqpoint{1.402801in}{1.954890in}}%
\pgfpathcurveto{\pgfqpoint{1.394565in}{1.954890in}}{\pgfqpoint{1.386665in}{1.951618in}}{\pgfqpoint{1.380841in}{1.945794in}}%
\pgfpathcurveto{\pgfqpoint{1.375017in}{1.939970in}}{\pgfqpoint{1.371745in}{1.932070in}}{\pgfqpoint{1.371745in}{1.923834in}}%
\pgfpathcurveto{\pgfqpoint{1.371745in}{1.915598in}}{\pgfqpoint{1.375017in}{1.907698in}}{\pgfqpoint{1.380841in}{1.901874in}}%
\pgfpathcurveto{\pgfqpoint{1.386665in}{1.896050in}}{\pgfqpoint{1.394565in}{1.892777in}}{\pgfqpoint{1.402801in}{1.892777in}}%
\pgfpathclose%
\pgfusepath{stroke,fill}%
\end{pgfscope}%
\begin{pgfscope}%
\pgfpathrectangle{\pgfqpoint{0.100000in}{0.212622in}}{\pgfqpoint{3.696000in}{3.696000in}}%
\pgfusepath{clip}%
\pgfsetbuttcap%
\pgfsetroundjoin%
\definecolor{currentfill}{rgb}{0.121569,0.466667,0.705882}%
\pgfsetfillcolor{currentfill}%
\pgfsetfillopacity{0.411770}%
\pgfsetlinewidth{1.003750pt}%
\definecolor{currentstroke}{rgb}{0.121569,0.466667,0.705882}%
\pgfsetstrokecolor{currentstroke}%
\pgfsetstrokeopacity{0.411770}%
\pgfsetdash{}{0pt}%
\pgfpathmoveto{\pgfqpoint{1.400502in}{1.888351in}}%
\pgfpathcurveto{\pgfqpoint{1.408738in}{1.888351in}}{\pgfqpoint{1.416638in}{1.891624in}}{\pgfqpoint{1.422462in}{1.897448in}}%
\pgfpathcurveto{\pgfqpoint{1.428286in}{1.903272in}}{\pgfqpoint{1.431558in}{1.911172in}}{\pgfqpoint{1.431558in}{1.919408in}}%
\pgfpathcurveto{\pgfqpoint{1.431558in}{1.927644in}}{\pgfqpoint{1.428286in}{1.935544in}}{\pgfqpoint{1.422462in}{1.941368in}}%
\pgfpathcurveto{\pgfqpoint{1.416638in}{1.947192in}}{\pgfqpoint{1.408738in}{1.950464in}}{\pgfqpoint{1.400502in}{1.950464in}}%
\pgfpathcurveto{\pgfqpoint{1.392265in}{1.950464in}}{\pgfqpoint{1.384365in}{1.947192in}}{\pgfqpoint{1.378541in}{1.941368in}}%
\pgfpathcurveto{\pgfqpoint{1.372717in}{1.935544in}}{\pgfqpoint{1.369445in}{1.927644in}}{\pgfqpoint{1.369445in}{1.919408in}}%
\pgfpathcurveto{\pgfqpoint{1.369445in}{1.911172in}}{\pgfqpoint{1.372717in}{1.903272in}}{\pgfqpoint{1.378541in}{1.897448in}}%
\pgfpathcurveto{\pgfqpoint{1.384365in}{1.891624in}}{\pgfqpoint{1.392265in}{1.888351in}}{\pgfqpoint{1.400502in}{1.888351in}}%
\pgfpathclose%
\pgfusepath{stroke,fill}%
\end{pgfscope}%
\begin{pgfscope}%
\pgfpathrectangle{\pgfqpoint{0.100000in}{0.212622in}}{\pgfqpoint{3.696000in}{3.696000in}}%
\pgfusepath{clip}%
\pgfsetbuttcap%
\pgfsetroundjoin%
\definecolor{currentfill}{rgb}{0.121569,0.466667,0.705882}%
\pgfsetfillcolor{currentfill}%
\pgfsetfillopacity{0.412618}%
\pgfsetlinewidth{1.003750pt}%
\definecolor{currentstroke}{rgb}{0.121569,0.466667,0.705882}%
\pgfsetstrokecolor{currentstroke}%
\pgfsetstrokeopacity{0.412618}%
\pgfsetdash{}{0pt}%
\pgfpathmoveto{\pgfqpoint{2.243769in}{2.125706in}}%
\pgfpathcurveto{\pgfqpoint{2.252005in}{2.125706in}}{\pgfqpoint{2.259905in}{2.128978in}}{\pgfqpoint{2.265729in}{2.134802in}}%
\pgfpathcurveto{\pgfqpoint{2.271553in}{2.140626in}}{\pgfqpoint{2.274825in}{2.148526in}}{\pgfqpoint{2.274825in}{2.156762in}}%
\pgfpathcurveto{\pgfqpoint{2.274825in}{2.164998in}}{\pgfqpoint{2.271553in}{2.172898in}}{\pgfqpoint{2.265729in}{2.178722in}}%
\pgfpathcurveto{\pgfqpoint{2.259905in}{2.184546in}}{\pgfqpoint{2.252005in}{2.187819in}}{\pgfqpoint{2.243769in}{2.187819in}}%
\pgfpathcurveto{\pgfqpoint{2.235532in}{2.187819in}}{\pgfqpoint{2.227632in}{2.184546in}}{\pgfqpoint{2.221808in}{2.178722in}}%
\pgfpathcurveto{\pgfqpoint{2.215984in}{2.172898in}}{\pgfqpoint{2.212712in}{2.164998in}}{\pgfqpoint{2.212712in}{2.156762in}}%
\pgfpathcurveto{\pgfqpoint{2.212712in}{2.148526in}}{\pgfqpoint{2.215984in}{2.140626in}}{\pgfqpoint{2.221808in}{2.134802in}}%
\pgfpathcurveto{\pgfqpoint{2.227632in}{2.128978in}}{\pgfqpoint{2.235532in}{2.125706in}}{\pgfqpoint{2.243769in}{2.125706in}}%
\pgfpathclose%
\pgfusepath{stroke,fill}%
\end{pgfscope}%
\begin{pgfscope}%
\pgfpathrectangle{\pgfqpoint{0.100000in}{0.212622in}}{\pgfqpoint{3.696000in}{3.696000in}}%
\pgfusepath{clip}%
\pgfsetbuttcap%
\pgfsetroundjoin%
\definecolor{currentfill}{rgb}{0.121569,0.466667,0.705882}%
\pgfsetfillcolor{currentfill}%
\pgfsetfillopacity{0.412684}%
\pgfsetlinewidth{1.003750pt}%
\definecolor{currentstroke}{rgb}{0.121569,0.466667,0.705882}%
\pgfsetstrokecolor{currentstroke}%
\pgfsetstrokeopacity{0.412684}%
\pgfsetdash{}{0pt}%
\pgfpathmoveto{\pgfqpoint{1.397588in}{1.885937in}}%
\pgfpathcurveto{\pgfqpoint{1.405824in}{1.885937in}}{\pgfqpoint{1.413724in}{1.889209in}}{\pgfqpoint{1.419548in}{1.895033in}}%
\pgfpathcurveto{\pgfqpoint{1.425372in}{1.900857in}}{\pgfqpoint{1.428644in}{1.908757in}}{\pgfqpoint{1.428644in}{1.916993in}}%
\pgfpathcurveto{\pgfqpoint{1.428644in}{1.925229in}}{\pgfqpoint{1.425372in}{1.933129in}}{\pgfqpoint{1.419548in}{1.938953in}}%
\pgfpathcurveto{\pgfqpoint{1.413724in}{1.944777in}}{\pgfqpoint{1.405824in}{1.948050in}}{\pgfqpoint{1.397588in}{1.948050in}}%
\pgfpathcurveto{\pgfqpoint{1.389351in}{1.948050in}}{\pgfqpoint{1.381451in}{1.944777in}}{\pgfqpoint{1.375627in}{1.938953in}}%
\pgfpathcurveto{\pgfqpoint{1.369803in}{1.933129in}}{\pgfqpoint{1.366531in}{1.925229in}}{\pgfqpoint{1.366531in}{1.916993in}}%
\pgfpathcurveto{\pgfqpoint{1.366531in}{1.908757in}}{\pgfqpoint{1.369803in}{1.900857in}}{\pgfqpoint{1.375627in}{1.895033in}}%
\pgfpathcurveto{\pgfqpoint{1.381451in}{1.889209in}}{\pgfqpoint{1.389351in}{1.885937in}}{\pgfqpoint{1.397588in}{1.885937in}}%
\pgfpathclose%
\pgfusepath{stroke,fill}%
\end{pgfscope}%
\begin{pgfscope}%
\pgfpathrectangle{\pgfqpoint{0.100000in}{0.212622in}}{\pgfqpoint{3.696000in}{3.696000in}}%
\pgfusepath{clip}%
\pgfsetbuttcap%
\pgfsetroundjoin%
\definecolor{currentfill}{rgb}{0.121569,0.466667,0.705882}%
\pgfsetfillcolor{currentfill}%
\pgfsetfillopacity{0.413377}%
\pgfsetlinewidth{1.003750pt}%
\definecolor{currentstroke}{rgb}{0.121569,0.466667,0.705882}%
\pgfsetstrokecolor{currentstroke}%
\pgfsetstrokeopacity{0.413377}%
\pgfsetdash{}{0pt}%
\pgfpathmoveto{\pgfqpoint{1.395685in}{1.882719in}}%
\pgfpathcurveto{\pgfqpoint{1.403921in}{1.882719in}}{\pgfqpoint{1.411822in}{1.885992in}}{\pgfqpoint{1.417645in}{1.891816in}}%
\pgfpathcurveto{\pgfqpoint{1.423469in}{1.897640in}}{\pgfqpoint{1.426742in}{1.905540in}}{\pgfqpoint{1.426742in}{1.913776in}}%
\pgfpathcurveto{\pgfqpoint{1.426742in}{1.922012in}}{\pgfqpoint{1.423469in}{1.929912in}}{\pgfqpoint{1.417645in}{1.935736in}}%
\pgfpathcurveto{\pgfqpoint{1.411822in}{1.941560in}}{\pgfqpoint{1.403921in}{1.944832in}}{\pgfqpoint{1.395685in}{1.944832in}}%
\pgfpathcurveto{\pgfqpoint{1.387449in}{1.944832in}}{\pgfqpoint{1.379549in}{1.941560in}}{\pgfqpoint{1.373725in}{1.935736in}}%
\pgfpathcurveto{\pgfqpoint{1.367901in}{1.929912in}}{\pgfqpoint{1.364629in}{1.922012in}}{\pgfqpoint{1.364629in}{1.913776in}}%
\pgfpathcurveto{\pgfqpoint{1.364629in}{1.905540in}}{\pgfqpoint{1.367901in}{1.897640in}}{\pgfqpoint{1.373725in}{1.891816in}}%
\pgfpathcurveto{\pgfqpoint{1.379549in}{1.885992in}}{\pgfqpoint{1.387449in}{1.882719in}}{\pgfqpoint{1.395685in}{1.882719in}}%
\pgfpathclose%
\pgfusepath{stroke,fill}%
\end{pgfscope}%
\begin{pgfscope}%
\pgfpathrectangle{\pgfqpoint{0.100000in}{0.212622in}}{\pgfqpoint{3.696000in}{3.696000in}}%
\pgfusepath{clip}%
\pgfsetbuttcap%
\pgfsetroundjoin%
\definecolor{currentfill}{rgb}{0.121569,0.466667,0.705882}%
\pgfsetfillcolor{currentfill}%
\pgfsetfillopacity{0.413964}%
\pgfsetlinewidth{1.003750pt}%
\definecolor{currentstroke}{rgb}{0.121569,0.466667,0.705882}%
\pgfsetstrokecolor{currentstroke}%
\pgfsetstrokeopacity{0.413964}%
\pgfsetdash{}{0pt}%
\pgfpathmoveto{\pgfqpoint{2.250365in}{2.121439in}}%
\pgfpathcurveto{\pgfqpoint{2.258601in}{2.121439in}}{\pgfqpoint{2.266501in}{2.124712in}}{\pgfqpoint{2.272325in}{2.130536in}}%
\pgfpathcurveto{\pgfqpoint{2.278149in}{2.136360in}}{\pgfqpoint{2.281421in}{2.144260in}}{\pgfqpoint{2.281421in}{2.152496in}}%
\pgfpathcurveto{\pgfqpoint{2.281421in}{2.160732in}}{\pgfqpoint{2.278149in}{2.168632in}}{\pgfqpoint{2.272325in}{2.174456in}}%
\pgfpathcurveto{\pgfqpoint{2.266501in}{2.180280in}}{\pgfqpoint{2.258601in}{2.183552in}}{\pgfqpoint{2.250365in}{2.183552in}}%
\pgfpathcurveto{\pgfqpoint{2.242129in}{2.183552in}}{\pgfqpoint{2.234228in}{2.180280in}}{\pgfqpoint{2.228405in}{2.174456in}}%
\pgfpathcurveto{\pgfqpoint{2.222581in}{2.168632in}}{\pgfqpoint{2.219308in}{2.160732in}}{\pgfqpoint{2.219308in}{2.152496in}}%
\pgfpathcurveto{\pgfqpoint{2.219308in}{2.144260in}}{\pgfqpoint{2.222581in}{2.136360in}}{\pgfqpoint{2.228405in}{2.130536in}}%
\pgfpathcurveto{\pgfqpoint{2.234228in}{2.124712in}}{\pgfqpoint{2.242129in}{2.121439in}}{\pgfqpoint{2.250365in}{2.121439in}}%
\pgfpathclose%
\pgfusepath{stroke,fill}%
\end{pgfscope}%
\begin{pgfscope}%
\pgfpathrectangle{\pgfqpoint{0.100000in}{0.212622in}}{\pgfqpoint{3.696000in}{3.696000in}}%
\pgfusepath{clip}%
\pgfsetbuttcap%
\pgfsetroundjoin%
\definecolor{currentfill}{rgb}{0.121569,0.466667,0.705882}%
\pgfsetfillcolor{currentfill}%
\pgfsetfillopacity{0.414009}%
\pgfsetlinewidth{1.003750pt}%
\definecolor{currentstroke}{rgb}{0.121569,0.466667,0.705882}%
\pgfsetstrokecolor{currentstroke}%
\pgfsetstrokeopacity{0.414009}%
\pgfsetdash{}{0pt}%
\pgfpathmoveto{\pgfqpoint{1.394809in}{1.881072in}}%
\pgfpathcurveto{\pgfqpoint{1.403046in}{1.881072in}}{\pgfqpoint{1.410946in}{1.884344in}}{\pgfqpoint{1.416770in}{1.890168in}}%
\pgfpathcurveto{\pgfqpoint{1.422594in}{1.895992in}}{\pgfqpoint{1.425866in}{1.903892in}}{\pgfqpoint{1.425866in}{1.912128in}}%
\pgfpathcurveto{\pgfqpoint{1.425866in}{1.920365in}}{\pgfqpoint{1.422594in}{1.928265in}}{\pgfqpoint{1.416770in}{1.934089in}}%
\pgfpathcurveto{\pgfqpoint{1.410946in}{1.939913in}}{\pgfqpoint{1.403046in}{1.943185in}}{\pgfqpoint{1.394809in}{1.943185in}}%
\pgfpathcurveto{\pgfqpoint{1.386573in}{1.943185in}}{\pgfqpoint{1.378673in}{1.939913in}}{\pgfqpoint{1.372849in}{1.934089in}}%
\pgfpathcurveto{\pgfqpoint{1.367025in}{1.928265in}}{\pgfqpoint{1.363753in}{1.920365in}}{\pgfqpoint{1.363753in}{1.912128in}}%
\pgfpathcurveto{\pgfqpoint{1.363753in}{1.903892in}}{\pgfqpoint{1.367025in}{1.895992in}}{\pgfqpoint{1.372849in}{1.890168in}}%
\pgfpathcurveto{\pgfqpoint{1.378673in}{1.884344in}}{\pgfqpoint{1.386573in}{1.881072in}}{\pgfqpoint{1.394809in}{1.881072in}}%
\pgfpathclose%
\pgfusepath{stroke,fill}%
\end{pgfscope}%
\begin{pgfscope}%
\pgfpathrectangle{\pgfqpoint{0.100000in}{0.212622in}}{\pgfqpoint{3.696000in}{3.696000in}}%
\pgfusepath{clip}%
\pgfsetbuttcap%
\pgfsetroundjoin%
\definecolor{currentfill}{rgb}{0.121569,0.466667,0.705882}%
\pgfsetfillcolor{currentfill}%
\pgfsetfillopacity{0.414169}%
\pgfsetlinewidth{1.003750pt}%
\definecolor{currentstroke}{rgb}{0.121569,0.466667,0.705882}%
\pgfsetstrokecolor{currentstroke}%
\pgfsetstrokeopacity{0.414169}%
\pgfsetdash{}{0pt}%
\pgfpathmoveto{\pgfqpoint{1.394017in}{1.879529in}}%
\pgfpathcurveto{\pgfqpoint{1.402253in}{1.879529in}}{\pgfqpoint{1.410153in}{1.882801in}}{\pgfqpoint{1.415977in}{1.888625in}}%
\pgfpathcurveto{\pgfqpoint{1.421801in}{1.894449in}}{\pgfqpoint{1.425074in}{1.902349in}}{\pgfqpoint{1.425074in}{1.910585in}}%
\pgfpathcurveto{\pgfqpoint{1.425074in}{1.918822in}}{\pgfqpoint{1.421801in}{1.926722in}}{\pgfqpoint{1.415977in}{1.932546in}}%
\pgfpathcurveto{\pgfqpoint{1.410153in}{1.938369in}}{\pgfqpoint{1.402253in}{1.941642in}}{\pgfqpoint{1.394017in}{1.941642in}}%
\pgfpathcurveto{\pgfqpoint{1.385781in}{1.941642in}}{\pgfqpoint{1.377881in}{1.938369in}}{\pgfqpoint{1.372057in}{1.932546in}}%
\pgfpathcurveto{\pgfqpoint{1.366233in}{1.926722in}}{\pgfqpoint{1.362961in}{1.918822in}}{\pgfqpoint{1.362961in}{1.910585in}}%
\pgfpathcurveto{\pgfqpoint{1.362961in}{1.902349in}}{\pgfqpoint{1.366233in}{1.894449in}}{\pgfqpoint{1.372057in}{1.888625in}}%
\pgfpathcurveto{\pgfqpoint{1.377881in}{1.882801in}}{\pgfqpoint{1.385781in}{1.879529in}}{\pgfqpoint{1.394017in}{1.879529in}}%
\pgfpathclose%
\pgfusepath{stroke,fill}%
\end{pgfscope}%
\begin{pgfscope}%
\pgfpathrectangle{\pgfqpoint{0.100000in}{0.212622in}}{\pgfqpoint{3.696000in}{3.696000in}}%
\pgfusepath{clip}%
\pgfsetbuttcap%
\pgfsetroundjoin%
\definecolor{currentfill}{rgb}{0.121569,0.466667,0.705882}%
\pgfsetfillcolor{currentfill}%
\pgfsetfillopacity{0.414924}%
\pgfsetlinewidth{1.003750pt}%
\definecolor{currentstroke}{rgb}{0.121569,0.466667,0.705882}%
\pgfsetstrokecolor{currentstroke}%
\pgfsetstrokeopacity{0.414924}%
\pgfsetdash{}{0pt}%
\pgfpathmoveto{\pgfqpoint{1.393057in}{1.879659in}}%
\pgfpathcurveto{\pgfqpoint{1.401293in}{1.879659in}}{\pgfqpoint{1.409193in}{1.882932in}}{\pgfqpoint{1.415017in}{1.888756in}}%
\pgfpathcurveto{\pgfqpoint{1.420841in}{1.894580in}}{\pgfqpoint{1.424113in}{1.902480in}}{\pgfqpoint{1.424113in}{1.910716in}}%
\pgfpathcurveto{\pgfqpoint{1.424113in}{1.918952in}}{\pgfqpoint{1.420841in}{1.926852in}}{\pgfqpoint{1.415017in}{1.932676in}}%
\pgfpathcurveto{\pgfqpoint{1.409193in}{1.938500in}}{\pgfqpoint{1.401293in}{1.941772in}}{\pgfqpoint{1.393057in}{1.941772in}}%
\pgfpathcurveto{\pgfqpoint{1.384821in}{1.941772in}}{\pgfqpoint{1.376920in}{1.938500in}}{\pgfqpoint{1.371097in}{1.932676in}}%
\pgfpathcurveto{\pgfqpoint{1.365273in}{1.926852in}}{\pgfqpoint{1.362000in}{1.918952in}}{\pgfqpoint{1.362000in}{1.910716in}}%
\pgfpathcurveto{\pgfqpoint{1.362000in}{1.902480in}}{\pgfqpoint{1.365273in}{1.894580in}}{\pgfqpoint{1.371097in}{1.888756in}}%
\pgfpathcurveto{\pgfqpoint{1.376920in}{1.882932in}}{\pgfqpoint{1.384821in}{1.879659in}}{\pgfqpoint{1.393057in}{1.879659in}}%
\pgfpathclose%
\pgfusepath{stroke,fill}%
\end{pgfscope}%
\begin{pgfscope}%
\pgfpathrectangle{\pgfqpoint{0.100000in}{0.212622in}}{\pgfqpoint{3.696000in}{3.696000in}}%
\pgfusepath{clip}%
\pgfsetbuttcap%
\pgfsetroundjoin%
\definecolor{currentfill}{rgb}{0.121569,0.466667,0.705882}%
\pgfsetfillcolor{currentfill}%
\pgfsetfillopacity{0.415246}%
\pgfsetlinewidth{1.003750pt}%
\definecolor{currentstroke}{rgb}{0.121569,0.466667,0.705882}%
\pgfsetstrokecolor{currentstroke}%
\pgfsetstrokeopacity{0.415246}%
\pgfsetdash{}{0pt}%
\pgfpathmoveto{\pgfqpoint{1.391315in}{1.878405in}}%
\pgfpathcurveto{\pgfqpoint{1.399551in}{1.878405in}}{\pgfqpoint{1.407452in}{1.881677in}}{\pgfqpoint{1.413275in}{1.887501in}}%
\pgfpathcurveto{\pgfqpoint{1.419099in}{1.893325in}}{\pgfqpoint{1.422372in}{1.901225in}}{\pgfqpoint{1.422372in}{1.909461in}}%
\pgfpathcurveto{\pgfqpoint{1.422372in}{1.917698in}}{\pgfqpoint{1.419099in}{1.925598in}}{\pgfqpoint{1.413275in}{1.931422in}}%
\pgfpathcurveto{\pgfqpoint{1.407452in}{1.937245in}}{\pgfqpoint{1.399551in}{1.940518in}}{\pgfqpoint{1.391315in}{1.940518in}}%
\pgfpathcurveto{\pgfqpoint{1.383079in}{1.940518in}}{\pgfqpoint{1.375179in}{1.937245in}}{\pgfqpoint{1.369355in}{1.931422in}}%
\pgfpathcurveto{\pgfqpoint{1.363531in}{1.925598in}}{\pgfqpoint{1.360259in}{1.917698in}}{\pgfqpoint{1.360259in}{1.909461in}}%
\pgfpathcurveto{\pgfqpoint{1.360259in}{1.901225in}}{\pgfqpoint{1.363531in}{1.893325in}}{\pgfqpoint{1.369355in}{1.887501in}}%
\pgfpathcurveto{\pgfqpoint{1.375179in}{1.881677in}}{\pgfqpoint{1.383079in}{1.878405in}}{\pgfqpoint{1.391315in}{1.878405in}}%
\pgfpathclose%
\pgfusepath{stroke,fill}%
\end{pgfscope}%
\begin{pgfscope}%
\pgfpathrectangle{\pgfqpoint{0.100000in}{0.212622in}}{\pgfqpoint{3.696000in}{3.696000in}}%
\pgfusepath{clip}%
\pgfsetbuttcap%
\pgfsetroundjoin%
\definecolor{currentfill}{rgb}{0.121569,0.466667,0.705882}%
\pgfsetfillcolor{currentfill}%
\pgfsetfillopacity{0.415637}%
\pgfsetlinewidth{1.003750pt}%
\definecolor{currentstroke}{rgb}{0.121569,0.466667,0.705882}%
\pgfsetstrokecolor{currentstroke}%
\pgfsetstrokeopacity{0.415637}%
\pgfsetdash{}{0pt}%
\pgfpathmoveto{\pgfqpoint{1.390634in}{1.878526in}}%
\pgfpathcurveto{\pgfqpoint{1.398870in}{1.878526in}}{\pgfqpoint{1.406770in}{1.881798in}}{\pgfqpoint{1.412594in}{1.887622in}}%
\pgfpathcurveto{\pgfqpoint{1.418418in}{1.893446in}}{\pgfqpoint{1.421690in}{1.901346in}}{\pgfqpoint{1.421690in}{1.909582in}}%
\pgfpathcurveto{\pgfqpoint{1.421690in}{1.917818in}}{\pgfqpoint{1.418418in}{1.925718in}}{\pgfqpoint{1.412594in}{1.931542in}}%
\pgfpathcurveto{\pgfqpoint{1.406770in}{1.937366in}}{\pgfqpoint{1.398870in}{1.940639in}}{\pgfqpoint{1.390634in}{1.940639in}}%
\pgfpathcurveto{\pgfqpoint{1.382397in}{1.940639in}}{\pgfqpoint{1.374497in}{1.937366in}}{\pgfqpoint{1.368673in}{1.931542in}}%
\pgfpathcurveto{\pgfqpoint{1.362849in}{1.925718in}}{\pgfqpoint{1.359577in}{1.917818in}}{\pgfqpoint{1.359577in}{1.909582in}}%
\pgfpathcurveto{\pgfqpoint{1.359577in}{1.901346in}}{\pgfqpoint{1.362849in}{1.893446in}}{\pgfqpoint{1.368673in}{1.887622in}}%
\pgfpathcurveto{\pgfqpoint{1.374497in}{1.881798in}}{\pgfqpoint{1.382397in}{1.878526in}}{\pgfqpoint{1.390634in}{1.878526in}}%
\pgfpathclose%
\pgfusepath{stroke,fill}%
\end{pgfscope}%
\begin{pgfscope}%
\pgfpathrectangle{\pgfqpoint{0.100000in}{0.212622in}}{\pgfqpoint{3.696000in}{3.696000in}}%
\pgfusepath{clip}%
\pgfsetbuttcap%
\pgfsetroundjoin%
\definecolor{currentfill}{rgb}{0.121569,0.466667,0.705882}%
\pgfsetfillcolor{currentfill}%
\pgfsetfillopacity{0.415678}%
\pgfsetlinewidth{1.003750pt}%
\definecolor{currentstroke}{rgb}{0.121569,0.466667,0.705882}%
\pgfsetstrokecolor{currentstroke}%
\pgfsetstrokeopacity{0.415678}%
\pgfsetdash{}{0pt}%
\pgfpathmoveto{\pgfqpoint{2.258711in}{2.121538in}}%
\pgfpathcurveto{\pgfqpoint{2.266947in}{2.121538in}}{\pgfqpoint{2.274847in}{2.124810in}}{\pgfqpoint{2.280671in}{2.130634in}}%
\pgfpathcurveto{\pgfqpoint{2.286495in}{2.136458in}}{\pgfqpoint{2.289767in}{2.144358in}}{\pgfqpoint{2.289767in}{2.152594in}}%
\pgfpathcurveto{\pgfqpoint{2.289767in}{2.160831in}}{\pgfqpoint{2.286495in}{2.168731in}}{\pgfqpoint{2.280671in}{2.174555in}}%
\pgfpathcurveto{\pgfqpoint{2.274847in}{2.180379in}}{\pgfqpoint{2.266947in}{2.183651in}}{\pgfqpoint{2.258711in}{2.183651in}}%
\pgfpathcurveto{\pgfqpoint{2.250474in}{2.183651in}}{\pgfqpoint{2.242574in}{2.180379in}}{\pgfqpoint{2.236750in}{2.174555in}}%
\pgfpathcurveto{\pgfqpoint{2.230926in}{2.168731in}}{\pgfqpoint{2.227654in}{2.160831in}}{\pgfqpoint{2.227654in}{2.152594in}}%
\pgfpathcurveto{\pgfqpoint{2.227654in}{2.144358in}}{\pgfqpoint{2.230926in}{2.136458in}}{\pgfqpoint{2.236750in}{2.130634in}}%
\pgfpathcurveto{\pgfqpoint{2.242574in}{2.124810in}}{\pgfqpoint{2.250474in}{2.121538in}}{\pgfqpoint{2.258711in}{2.121538in}}%
\pgfpathclose%
\pgfusepath{stroke,fill}%
\end{pgfscope}%
\begin{pgfscope}%
\pgfpathrectangle{\pgfqpoint{0.100000in}{0.212622in}}{\pgfqpoint{3.696000in}{3.696000in}}%
\pgfusepath{clip}%
\pgfsetbuttcap%
\pgfsetroundjoin%
\definecolor{currentfill}{rgb}{0.121569,0.466667,0.705882}%
\pgfsetfillcolor{currentfill}%
\pgfsetfillopacity{0.415849}%
\pgfsetlinewidth{1.003750pt}%
\definecolor{currentstroke}{rgb}{0.121569,0.466667,0.705882}%
\pgfsetstrokecolor{currentstroke}%
\pgfsetstrokeopacity{0.415849}%
\pgfsetdash{}{0pt}%
\pgfpathmoveto{\pgfqpoint{1.389797in}{1.878405in}}%
\pgfpathcurveto{\pgfqpoint{1.398033in}{1.878405in}}{\pgfqpoint{1.405934in}{1.881677in}}{\pgfqpoint{1.411757in}{1.887501in}}%
\pgfpathcurveto{\pgfqpoint{1.417581in}{1.893325in}}{\pgfqpoint{1.420854in}{1.901225in}}{\pgfqpoint{1.420854in}{1.909461in}}%
\pgfpathcurveto{\pgfqpoint{1.420854in}{1.917697in}}{\pgfqpoint{1.417581in}{1.925597in}}{\pgfqpoint{1.411757in}{1.931421in}}%
\pgfpathcurveto{\pgfqpoint{1.405934in}{1.937245in}}{\pgfqpoint{1.398033in}{1.940518in}}{\pgfqpoint{1.389797in}{1.940518in}}%
\pgfpathcurveto{\pgfqpoint{1.381561in}{1.940518in}}{\pgfqpoint{1.373661in}{1.937245in}}{\pgfqpoint{1.367837in}{1.931421in}}%
\pgfpathcurveto{\pgfqpoint{1.362013in}{1.925597in}}{\pgfqpoint{1.358741in}{1.917697in}}{\pgfqpoint{1.358741in}{1.909461in}}%
\pgfpathcurveto{\pgfqpoint{1.358741in}{1.901225in}}{\pgfqpoint{1.362013in}{1.893325in}}{\pgfqpoint{1.367837in}{1.887501in}}%
\pgfpathcurveto{\pgfqpoint{1.373661in}{1.881677in}}{\pgfqpoint{1.381561in}{1.878405in}}{\pgfqpoint{1.389797in}{1.878405in}}%
\pgfpathclose%
\pgfusepath{stroke,fill}%
\end{pgfscope}%
\begin{pgfscope}%
\pgfpathrectangle{\pgfqpoint{0.100000in}{0.212622in}}{\pgfqpoint{3.696000in}{3.696000in}}%
\pgfusepath{clip}%
\pgfsetbuttcap%
\pgfsetroundjoin%
\definecolor{currentfill}{rgb}{0.121569,0.466667,0.705882}%
\pgfsetfillcolor{currentfill}%
\pgfsetfillopacity{0.416500}%
\pgfsetlinewidth{1.003750pt}%
\definecolor{currentstroke}{rgb}{0.121569,0.466667,0.705882}%
\pgfsetstrokecolor{currentstroke}%
\pgfsetstrokeopacity{0.416500}%
\pgfsetdash{}{0pt}%
\pgfpathmoveto{\pgfqpoint{2.263185in}{2.120405in}}%
\pgfpathcurveto{\pgfqpoint{2.271421in}{2.120405in}}{\pgfqpoint{2.279321in}{2.123677in}}{\pgfqpoint{2.285145in}{2.129501in}}%
\pgfpathcurveto{\pgfqpoint{2.290969in}{2.135325in}}{\pgfqpoint{2.294242in}{2.143225in}}{\pgfqpoint{2.294242in}{2.151461in}}%
\pgfpathcurveto{\pgfqpoint{2.294242in}{2.159698in}}{\pgfqpoint{2.290969in}{2.167598in}}{\pgfqpoint{2.285145in}{2.173422in}}%
\pgfpathcurveto{\pgfqpoint{2.279321in}{2.179246in}}{\pgfqpoint{2.271421in}{2.182518in}}{\pgfqpoint{2.263185in}{2.182518in}}%
\pgfpathcurveto{\pgfqpoint{2.254949in}{2.182518in}}{\pgfqpoint{2.247049in}{2.179246in}}{\pgfqpoint{2.241225in}{2.173422in}}%
\pgfpathcurveto{\pgfqpoint{2.235401in}{2.167598in}}{\pgfqpoint{2.232129in}{2.159698in}}{\pgfqpoint{2.232129in}{2.151461in}}%
\pgfpathcurveto{\pgfqpoint{2.232129in}{2.143225in}}{\pgfqpoint{2.235401in}{2.135325in}}{\pgfqpoint{2.241225in}{2.129501in}}%
\pgfpathcurveto{\pgfqpoint{2.247049in}{2.123677in}}{\pgfqpoint{2.254949in}{2.120405in}}{\pgfqpoint{2.263185in}{2.120405in}}%
\pgfpathclose%
\pgfusepath{stroke,fill}%
\end{pgfscope}%
\begin{pgfscope}%
\pgfpathrectangle{\pgfqpoint{0.100000in}{0.212622in}}{\pgfqpoint{3.696000in}{3.696000in}}%
\pgfusepath{clip}%
\pgfsetbuttcap%
\pgfsetroundjoin%
\definecolor{currentfill}{rgb}{0.121569,0.466667,0.705882}%
\pgfsetfillcolor{currentfill}%
\pgfsetfillopacity{0.416751}%
\pgfsetlinewidth{1.003750pt}%
\definecolor{currentstroke}{rgb}{0.121569,0.466667,0.705882}%
\pgfsetstrokecolor{currentstroke}%
\pgfsetstrokeopacity{0.416751}%
\pgfsetdash{}{0pt}%
\pgfpathmoveto{\pgfqpoint{1.389287in}{1.880866in}}%
\pgfpathcurveto{\pgfqpoint{1.397523in}{1.880866in}}{\pgfqpoint{1.405423in}{1.884138in}}{\pgfqpoint{1.411247in}{1.889962in}}%
\pgfpathcurveto{\pgfqpoint{1.417071in}{1.895786in}}{\pgfqpoint{1.420343in}{1.903686in}}{\pgfqpoint{1.420343in}{1.911922in}}%
\pgfpathcurveto{\pgfqpoint{1.420343in}{1.920159in}}{\pgfqpoint{1.417071in}{1.928059in}}{\pgfqpoint{1.411247in}{1.933882in}}%
\pgfpathcurveto{\pgfqpoint{1.405423in}{1.939706in}}{\pgfqpoint{1.397523in}{1.942979in}}{\pgfqpoint{1.389287in}{1.942979in}}%
\pgfpathcurveto{\pgfqpoint{1.381050in}{1.942979in}}{\pgfqpoint{1.373150in}{1.939706in}}{\pgfqpoint{1.367326in}{1.933882in}}%
\pgfpathcurveto{\pgfqpoint{1.361502in}{1.928059in}}{\pgfqpoint{1.358230in}{1.920159in}}{\pgfqpoint{1.358230in}{1.911922in}}%
\pgfpathcurveto{\pgfqpoint{1.358230in}{1.903686in}}{\pgfqpoint{1.361502in}{1.895786in}}{\pgfqpoint{1.367326in}{1.889962in}}%
\pgfpathcurveto{\pgfqpoint{1.373150in}{1.884138in}}{\pgfqpoint{1.381050in}{1.880866in}}{\pgfqpoint{1.389287in}{1.880866in}}%
\pgfpathclose%
\pgfusepath{stroke,fill}%
\end{pgfscope}%
\begin{pgfscope}%
\pgfpathrectangle{\pgfqpoint{0.100000in}{0.212622in}}{\pgfqpoint{3.696000in}{3.696000in}}%
\pgfusepath{clip}%
\pgfsetbuttcap%
\pgfsetroundjoin%
\definecolor{currentfill}{rgb}{0.121569,0.466667,0.705882}%
\pgfsetfillcolor{currentfill}%
\pgfsetfillopacity{0.417700}%
\pgfsetlinewidth{1.003750pt}%
\definecolor{currentstroke}{rgb}{0.121569,0.466667,0.705882}%
\pgfsetstrokecolor{currentstroke}%
\pgfsetstrokeopacity{0.417700}%
\pgfsetdash{}{0pt}%
\pgfpathmoveto{\pgfqpoint{1.388417in}{1.880279in}}%
\pgfpathcurveto{\pgfqpoint{1.396653in}{1.880279in}}{\pgfqpoint{1.404553in}{1.883551in}}{\pgfqpoint{1.410377in}{1.889375in}}%
\pgfpathcurveto{\pgfqpoint{1.416201in}{1.895199in}}{\pgfqpoint{1.419473in}{1.903099in}}{\pgfqpoint{1.419473in}{1.911335in}}%
\pgfpathcurveto{\pgfqpoint{1.419473in}{1.919572in}}{\pgfqpoint{1.416201in}{1.927472in}}{\pgfqpoint{1.410377in}{1.933296in}}%
\pgfpathcurveto{\pgfqpoint{1.404553in}{1.939120in}}{\pgfqpoint{1.396653in}{1.942392in}}{\pgfqpoint{1.388417in}{1.942392in}}%
\pgfpathcurveto{\pgfqpoint{1.380181in}{1.942392in}}{\pgfqpoint{1.372281in}{1.939120in}}{\pgfqpoint{1.366457in}{1.933296in}}%
\pgfpathcurveto{\pgfqpoint{1.360633in}{1.927472in}}{\pgfqpoint{1.357360in}{1.919572in}}{\pgfqpoint{1.357360in}{1.911335in}}%
\pgfpathcurveto{\pgfqpoint{1.357360in}{1.903099in}}{\pgfqpoint{1.360633in}{1.895199in}}{\pgfqpoint{1.366457in}{1.889375in}}%
\pgfpathcurveto{\pgfqpoint{1.372281in}{1.883551in}}{\pgfqpoint{1.380181in}{1.880279in}}{\pgfqpoint{1.388417in}{1.880279in}}%
\pgfpathclose%
\pgfusepath{stroke,fill}%
\end{pgfscope}%
\begin{pgfscope}%
\pgfpathrectangle{\pgfqpoint{0.100000in}{0.212622in}}{\pgfqpoint{3.696000in}{3.696000in}}%
\pgfusepath{clip}%
\pgfsetbuttcap%
\pgfsetroundjoin%
\definecolor{currentfill}{rgb}{0.121569,0.466667,0.705882}%
\pgfsetfillcolor{currentfill}%
\pgfsetfillopacity{0.417796}%
\pgfsetlinewidth{1.003750pt}%
\definecolor{currentstroke}{rgb}{0.121569,0.466667,0.705882}%
\pgfsetstrokecolor{currentstroke}%
\pgfsetstrokeopacity{0.417796}%
\pgfsetdash{}{0pt}%
\pgfpathmoveto{\pgfqpoint{2.269542in}{2.121540in}}%
\pgfpathcurveto{\pgfqpoint{2.277779in}{2.121540in}}{\pgfqpoint{2.285679in}{2.124813in}}{\pgfqpoint{2.291503in}{2.130636in}}%
\pgfpathcurveto{\pgfqpoint{2.297327in}{2.136460in}}{\pgfqpoint{2.300599in}{2.144360in}}{\pgfqpoint{2.300599in}{2.152597in}}%
\pgfpathcurveto{\pgfqpoint{2.300599in}{2.160833in}}{\pgfqpoint{2.297327in}{2.168733in}}{\pgfqpoint{2.291503in}{2.174557in}}%
\pgfpathcurveto{\pgfqpoint{2.285679in}{2.180381in}}{\pgfqpoint{2.277779in}{2.183653in}}{\pgfqpoint{2.269542in}{2.183653in}}%
\pgfpathcurveto{\pgfqpoint{2.261306in}{2.183653in}}{\pgfqpoint{2.253406in}{2.180381in}}{\pgfqpoint{2.247582in}{2.174557in}}%
\pgfpathcurveto{\pgfqpoint{2.241758in}{2.168733in}}{\pgfqpoint{2.238486in}{2.160833in}}{\pgfqpoint{2.238486in}{2.152597in}}%
\pgfpathcurveto{\pgfqpoint{2.238486in}{2.144360in}}{\pgfqpoint{2.241758in}{2.136460in}}{\pgfqpoint{2.247582in}{2.130636in}}%
\pgfpathcurveto{\pgfqpoint{2.253406in}{2.124813in}}{\pgfqpoint{2.261306in}{2.121540in}}{\pgfqpoint{2.269542in}{2.121540in}}%
\pgfpathclose%
\pgfusepath{stroke,fill}%
\end{pgfscope}%
\begin{pgfscope}%
\pgfpathrectangle{\pgfqpoint{0.100000in}{0.212622in}}{\pgfqpoint{3.696000in}{3.696000in}}%
\pgfusepath{clip}%
\pgfsetbuttcap%
\pgfsetroundjoin%
\definecolor{currentfill}{rgb}{0.121569,0.466667,0.705882}%
\pgfsetfillcolor{currentfill}%
\pgfsetfillopacity{0.418198}%
\pgfsetlinewidth{1.003750pt}%
\definecolor{currentstroke}{rgb}{0.121569,0.466667,0.705882}%
\pgfsetstrokecolor{currentstroke}%
\pgfsetstrokeopacity{0.418198}%
\pgfsetdash{}{0pt}%
\pgfpathmoveto{\pgfqpoint{2.272672in}{2.118956in}}%
\pgfpathcurveto{\pgfqpoint{2.280909in}{2.118956in}}{\pgfqpoint{2.288809in}{2.122228in}}{\pgfqpoint{2.294633in}{2.128052in}}%
\pgfpathcurveto{\pgfqpoint{2.300456in}{2.133876in}}{\pgfqpoint{2.303729in}{2.141776in}}{\pgfqpoint{2.303729in}{2.150013in}}%
\pgfpathcurveto{\pgfqpoint{2.303729in}{2.158249in}}{\pgfqpoint{2.300456in}{2.166149in}}{\pgfqpoint{2.294633in}{2.171973in}}%
\pgfpathcurveto{\pgfqpoint{2.288809in}{2.177797in}}{\pgfqpoint{2.280909in}{2.181069in}}{\pgfqpoint{2.272672in}{2.181069in}}%
\pgfpathcurveto{\pgfqpoint{2.264436in}{2.181069in}}{\pgfqpoint{2.256536in}{2.177797in}}{\pgfqpoint{2.250712in}{2.171973in}}%
\pgfpathcurveto{\pgfqpoint{2.244888in}{2.166149in}}{\pgfqpoint{2.241616in}{2.158249in}}{\pgfqpoint{2.241616in}{2.150013in}}%
\pgfpathcurveto{\pgfqpoint{2.241616in}{2.141776in}}{\pgfqpoint{2.244888in}{2.133876in}}{\pgfqpoint{2.250712in}{2.128052in}}%
\pgfpathcurveto{\pgfqpoint{2.256536in}{2.122228in}}{\pgfqpoint{2.264436in}{2.118956in}}{\pgfqpoint{2.272672in}{2.118956in}}%
\pgfpathclose%
\pgfusepath{stroke,fill}%
\end{pgfscope}%
\begin{pgfscope}%
\pgfpathrectangle{\pgfqpoint{0.100000in}{0.212622in}}{\pgfqpoint{3.696000in}{3.696000in}}%
\pgfusepath{clip}%
\pgfsetbuttcap%
\pgfsetroundjoin%
\definecolor{currentfill}{rgb}{0.121569,0.466667,0.705882}%
\pgfsetfillcolor{currentfill}%
\pgfsetfillopacity{0.418296}%
\pgfsetlinewidth{1.003750pt}%
\definecolor{currentstroke}{rgb}{0.121569,0.466667,0.705882}%
\pgfsetstrokecolor{currentstroke}%
\pgfsetstrokeopacity{0.418296}%
\pgfsetdash{}{0pt}%
\pgfpathmoveto{\pgfqpoint{1.386265in}{1.879754in}}%
\pgfpathcurveto{\pgfqpoint{1.394501in}{1.879754in}}{\pgfqpoint{1.402401in}{1.883027in}}{\pgfqpoint{1.408225in}{1.888851in}}%
\pgfpathcurveto{\pgfqpoint{1.414049in}{1.894675in}}{\pgfqpoint{1.417321in}{1.902575in}}{\pgfqpoint{1.417321in}{1.910811in}}%
\pgfpathcurveto{\pgfqpoint{1.417321in}{1.919047in}}{\pgfqpoint{1.414049in}{1.926947in}}{\pgfqpoint{1.408225in}{1.932771in}}%
\pgfpathcurveto{\pgfqpoint{1.402401in}{1.938595in}}{\pgfqpoint{1.394501in}{1.941867in}}{\pgfqpoint{1.386265in}{1.941867in}}%
\pgfpathcurveto{\pgfqpoint{1.378029in}{1.941867in}}{\pgfqpoint{1.370129in}{1.938595in}}{\pgfqpoint{1.364305in}{1.932771in}}%
\pgfpathcurveto{\pgfqpoint{1.358481in}{1.926947in}}{\pgfqpoint{1.355208in}{1.919047in}}{\pgfqpoint{1.355208in}{1.910811in}}%
\pgfpathcurveto{\pgfqpoint{1.355208in}{1.902575in}}{\pgfqpoint{1.358481in}{1.894675in}}{\pgfqpoint{1.364305in}{1.888851in}}%
\pgfpathcurveto{\pgfqpoint{1.370129in}{1.883027in}}{\pgfqpoint{1.378029in}{1.879754in}}{\pgfqpoint{1.386265in}{1.879754in}}%
\pgfpathclose%
\pgfusepath{stroke,fill}%
\end{pgfscope}%
\begin{pgfscope}%
\pgfpathrectangle{\pgfqpoint{0.100000in}{0.212622in}}{\pgfqpoint{3.696000in}{3.696000in}}%
\pgfusepath{clip}%
\pgfsetbuttcap%
\pgfsetroundjoin%
\definecolor{currentfill}{rgb}{0.121569,0.466667,0.705882}%
\pgfsetfillcolor{currentfill}%
\pgfsetfillopacity{0.419085}%
\pgfsetlinewidth{1.003750pt}%
\definecolor{currentstroke}{rgb}{0.121569,0.466667,0.705882}%
\pgfsetstrokecolor{currentstroke}%
\pgfsetstrokeopacity{0.419085}%
\pgfsetdash{}{0pt}%
\pgfpathmoveto{\pgfqpoint{2.276325in}{2.120236in}}%
\pgfpathcurveto{\pgfqpoint{2.284561in}{2.120236in}}{\pgfqpoint{2.292461in}{2.123508in}}{\pgfqpoint{2.298285in}{2.129332in}}%
\pgfpathcurveto{\pgfqpoint{2.304109in}{2.135156in}}{\pgfqpoint{2.307381in}{2.143056in}}{\pgfqpoint{2.307381in}{2.151292in}}%
\pgfpathcurveto{\pgfqpoint{2.307381in}{2.159528in}}{\pgfqpoint{2.304109in}{2.167428in}}{\pgfqpoint{2.298285in}{2.173252in}}%
\pgfpathcurveto{\pgfqpoint{2.292461in}{2.179076in}}{\pgfqpoint{2.284561in}{2.182349in}}{\pgfqpoint{2.276325in}{2.182349in}}%
\pgfpathcurveto{\pgfqpoint{2.268088in}{2.182349in}}{\pgfqpoint{2.260188in}{2.179076in}}{\pgfqpoint{2.254364in}{2.173252in}}%
\pgfpathcurveto{\pgfqpoint{2.248540in}{2.167428in}}{\pgfqpoint{2.245268in}{2.159528in}}{\pgfqpoint{2.245268in}{2.151292in}}%
\pgfpathcurveto{\pgfqpoint{2.245268in}{2.143056in}}{\pgfqpoint{2.248540in}{2.135156in}}{\pgfqpoint{2.254364in}{2.129332in}}%
\pgfpathcurveto{\pgfqpoint{2.260188in}{2.123508in}}{\pgfqpoint{2.268088in}{2.120236in}}{\pgfqpoint{2.276325in}{2.120236in}}%
\pgfpathclose%
\pgfusepath{stroke,fill}%
\end{pgfscope}%
\begin{pgfscope}%
\pgfpathrectangle{\pgfqpoint{0.100000in}{0.212622in}}{\pgfqpoint{3.696000in}{3.696000in}}%
\pgfusepath{clip}%
\pgfsetbuttcap%
\pgfsetroundjoin%
\definecolor{currentfill}{rgb}{0.121569,0.466667,0.705882}%
\pgfsetfillcolor{currentfill}%
\pgfsetfillopacity{0.419524}%
\pgfsetlinewidth{1.003750pt}%
\definecolor{currentstroke}{rgb}{0.121569,0.466667,0.705882}%
\pgfsetstrokecolor{currentstroke}%
\pgfsetstrokeopacity{0.419524}%
\pgfsetdash{}{0pt}%
\pgfpathmoveto{\pgfqpoint{1.384403in}{1.877597in}}%
\pgfpathcurveto{\pgfqpoint{1.392639in}{1.877597in}}{\pgfqpoint{1.400539in}{1.880870in}}{\pgfqpoint{1.406363in}{1.886694in}}%
\pgfpathcurveto{\pgfqpoint{1.412187in}{1.892518in}}{\pgfqpoint{1.415460in}{1.900418in}}{\pgfqpoint{1.415460in}{1.908654in}}%
\pgfpathcurveto{\pgfqpoint{1.415460in}{1.916890in}}{\pgfqpoint{1.412187in}{1.924790in}}{\pgfqpoint{1.406363in}{1.930614in}}%
\pgfpathcurveto{\pgfqpoint{1.400539in}{1.936438in}}{\pgfqpoint{1.392639in}{1.939710in}}{\pgfqpoint{1.384403in}{1.939710in}}%
\pgfpathcurveto{\pgfqpoint{1.376167in}{1.939710in}}{\pgfqpoint{1.368267in}{1.936438in}}{\pgfqpoint{1.362443in}{1.930614in}}%
\pgfpathcurveto{\pgfqpoint{1.356619in}{1.924790in}}{\pgfqpoint{1.353347in}{1.916890in}}{\pgfqpoint{1.353347in}{1.908654in}}%
\pgfpathcurveto{\pgfqpoint{1.353347in}{1.900418in}}{\pgfqpoint{1.356619in}{1.892518in}}{\pgfqpoint{1.362443in}{1.886694in}}%
\pgfpathcurveto{\pgfqpoint{1.368267in}{1.880870in}}{\pgfqpoint{1.376167in}{1.877597in}}{\pgfqpoint{1.384403in}{1.877597in}}%
\pgfpathclose%
\pgfusepath{stroke,fill}%
\end{pgfscope}%
\begin{pgfscope}%
\pgfpathrectangle{\pgfqpoint{0.100000in}{0.212622in}}{\pgfqpoint{3.696000in}{3.696000in}}%
\pgfusepath{clip}%
\pgfsetbuttcap%
\pgfsetroundjoin%
\definecolor{currentfill}{rgb}{0.121569,0.466667,0.705882}%
\pgfsetfillcolor{currentfill}%
\pgfsetfillopacity{0.419700}%
\pgfsetlinewidth{1.003750pt}%
\definecolor{currentstroke}{rgb}{0.121569,0.466667,0.705882}%
\pgfsetstrokecolor{currentstroke}%
\pgfsetstrokeopacity{0.419700}%
\pgfsetdash{}{0pt}%
\pgfpathmoveto{\pgfqpoint{2.280123in}{2.118679in}}%
\pgfpathcurveto{\pgfqpoint{2.288359in}{2.118679in}}{\pgfqpoint{2.296259in}{2.121952in}}{\pgfqpoint{2.302083in}{2.127775in}}%
\pgfpathcurveto{\pgfqpoint{2.307907in}{2.133599in}}{\pgfqpoint{2.311179in}{2.141499in}}{\pgfqpoint{2.311179in}{2.149736in}}%
\pgfpathcurveto{\pgfqpoint{2.311179in}{2.157972in}}{\pgfqpoint{2.307907in}{2.165872in}}{\pgfqpoint{2.302083in}{2.171696in}}%
\pgfpathcurveto{\pgfqpoint{2.296259in}{2.177520in}}{\pgfqpoint{2.288359in}{2.180792in}}{\pgfqpoint{2.280123in}{2.180792in}}%
\pgfpathcurveto{\pgfqpoint{2.271886in}{2.180792in}}{\pgfqpoint{2.263986in}{2.177520in}}{\pgfqpoint{2.258162in}{2.171696in}}%
\pgfpathcurveto{\pgfqpoint{2.252338in}{2.165872in}}{\pgfqpoint{2.249066in}{2.157972in}}{\pgfqpoint{2.249066in}{2.149736in}}%
\pgfpathcurveto{\pgfqpoint{2.249066in}{2.141499in}}{\pgfqpoint{2.252338in}{2.133599in}}{\pgfqpoint{2.258162in}{2.127775in}}%
\pgfpathcurveto{\pgfqpoint{2.263986in}{2.121952in}}{\pgfqpoint{2.271886in}{2.118679in}}{\pgfqpoint{2.280123in}{2.118679in}}%
\pgfpathclose%
\pgfusepath{stroke,fill}%
\end{pgfscope}%
\begin{pgfscope}%
\pgfpathrectangle{\pgfqpoint{0.100000in}{0.212622in}}{\pgfqpoint{3.696000in}{3.696000in}}%
\pgfusepath{clip}%
\pgfsetbuttcap%
\pgfsetroundjoin%
\definecolor{currentfill}{rgb}{0.121569,0.466667,0.705882}%
\pgfsetfillcolor{currentfill}%
\pgfsetfillopacity{0.420472}%
\pgfsetlinewidth{1.003750pt}%
\definecolor{currentstroke}{rgb}{0.121569,0.466667,0.705882}%
\pgfsetstrokecolor{currentstroke}%
\pgfsetstrokeopacity{0.420472}%
\pgfsetdash{}{0pt}%
\pgfpathmoveto{\pgfqpoint{2.285306in}{2.116695in}}%
\pgfpathcurveto{\pgfqpoint{2.293543in}{2.116695in}}{\pgfqpoint{2.301443in}{2.119967in}}{\pgfqpoint{2.307267in}{2.125791in}}%
\pgfpathcurveto{\pgfqpoint{2.313090in}{2.131615in}}{\pgfqpoint{2.316363in}{2.139515in}}{\pgfqpoint{2.316363in}{2.147751in}}%
\pgfpathcurveto{\pgfqpoint{2.316363in}{2.155987in}}{\pgfqpoint{2.313090in}{2.163887in}}{\pgfqpoint{2.307267in}{2.169711in}}%
\pgfpathcurveto{\pgfqpoint{2.301443in}{2.175535in}}{\pgfqpoint{2.293543in}{2.178808in}}{\pgfqpoint{2.285306in}{2.178808in}}%
\pgfpathcurveto{\pgfqpoint{2.277070in}{2.178808in}}{\pgfqpoint{2.269170in}{2.175535in}}{\pgfqpoint{2.263346in}{2.169711in}}%
\pgfpathcurveto{\pgfqpoint{2.257522in}{2.163887in}}{\pgfqpoint{2.254250in}{2.155987in}}{\pgfqpoint{2.254250in}{2.147751in}}%
\pgfpathcurveto{\pgfqpoint{2.254250in}{2.139515in}}{\pgfqpoint{2.257522in}{2.131615in}}{\pgfqpoint{2.263346in}{2.125791in}}%
\pgfpathcurveto{\pgfqpoint{2.269170in}{2.119967in}}{\pgfqpoint{2.277070in}{2.116695in}}{\pgfqpoint{2.285306in}{2.116695in}}%
\pgfpathclose%
\pgfusepath{stroke,fill}%
\end{pgfscope}%
\begin{pgfscope}%
\pgfpathrectangle{\pgfqpoint{0.100000in}{0.212622in}}{\pgfqpoint{3.696000in}{3.696000in}}%
\pgfusepath{clip}%
\pgfsetbuttcap%
\pgfsetroundjoin%
\definecolor{currentfill}{rgb}{0.121569,0.466667,0.705882}%
\pgfsetfillcolor{currentfill}%
\pgfsetfillopacity{0.421027}%
\pgfsetlinewidth{1.003750pt}%
\definecolor{currentstroke}{rgb}{0.121569,0.466667,0.705882}%
\pgfsetstrokecolor{currentstroke}%
\pgfsetstrokeopacity{0.421027}%
\pgfsetdash{}{0pt}%
\pgfpathmoveto{\pgfqpoint{1.380317in}{1.868855in}}%
\pgfpathcurveto{\pgfqpoint{1.388553in}{1.868855in}}{\pgfqpoint{1.396453in}{1.872127in}}{\pgfqpoint{1.402277in}{1.877951in}}%
\pgfpathcurveto{\pgfqpoint{1.408101in}{1.883775in}}{\pgfqpoint{1.411373in}{1.891675in}}{\pgfqpoint{1.411373in}{1.899911in}}%
\pgfpathcurveto{\pgfqpoint{1.411373in}{1.908147in}}{\pgfqpoint{1.408101in}{1.916047in}}{\pgfqpoint{1.402277in}{1.921871in}}%
\pgfpathcurveto{\pgfqpoint{1.396453in}{1.927695in}}{\pgfqpoint{1.388553in}{1.930968in}}{\pgfqpoint{1.380317in}{1.930968in}}%
\pgfpathcurveto{\pgfqpoint{1.372081in}{1.930968in}}{\pgfqpoint{1.364181in}{1.927695in}}{\pgfqpoint{1.358357in}{1.921871in}}%
\pgfpathcurveto{\pgfqpoint{1.352533in}{1.916047in}}{\pgfqpoint{1.349260in}{1.908147in}}{\pgfqpoint{1.349260in}{1.899911in}}%
\pgfpathcurveto{\pgfqpoint{1.349260in}{1.891675in}}{\pgfqpoint{1.352533in}{1.883775in}}{\pgfqpoint{1.358357in}{1.877951in}}%
\pgfpathcurveto{\pgfqpoint{1.364181in}{1.872127in}}{\pgfqpoint{1.372081in}{1.868855in}}{\pgfqpoint{1.380317in}{1.868855in}}%
\pgfpathclose%
\pgfusepath{stroke,fill}%
\end{pgfscope}%
\begin{pgfscope}%
\pgfpathrectangle{\pgfqpoint{0.100000in}{0.212622in}}{\pgfqpoint{3.696000in}{3.696000in}}%
\pgfusepath{clip}%
\pgfsetbuttcap%
\pgfsetroundjoin%
\definecolor{currentfill}{rgb}{0.121569,0.466667,0.705882}%
\pgfsetfillcolor{currentfill}%
\pgfsetfillopacity{0.421505}%
\pgfsetlinewidth{1.003750pt}%
\definecolor{currentstroke}{rgb}{0.121569,0.466667,0.705882}%
\pgfsetstrokecolor{currentstroke}%
\pgfsetstrokeopacity{0.421505}%
\pgfsetdash{}{0pt}%
\pgfpathmoveto{\pgfqpoint{2.290958in}{2.116338in}}%
\pgfpathcurveto{\pgfqpoint{2.299194in}{2.116338in}}{\pgfqpoint{2.307094in}{2.119611in}}{\pgfqpoint{2.312918in}{2.125434in}}%
\pgfpathcurveto{\pgfqpoint{2.318742in}{2.131258in}}{\pgfqpoint{2.322014in}{2.139158in}}{\pgfqpoint{2.322014in}{2.147395in}}%
\pgfpathcurveto{\pgfqpoint{2.322014in}{2.155631in}}{\pgfqpoint{2.318742in}{2.163531in}}{\pgfqpoint{2.312918in}{2.169355in}}%
\pgfpathcurveto{\pgfqpoint{2.307094in}{2.175179in}}{\pgfqpoint{2.299194in}{2.178451in}}{\pgfqpoint{2.290958in}{2.178451in}}%
\pgfpathcurveto{\pgfqpoint{2.282722in}{2.178451in}}{\pgfqpoint{2.274822in}{2.175179in}}{\pgfqpoint{2.268998in}{2.169355in}}%
\pgfpathcurveto{\pgfqpoint{2.263174in}{2.163531in}}{\pgfqpoint{2.259901in}{2.155631in}}{\pgfqpoint{2.259901in}{2.147395in}}%
\pgfpathcurveto{\pgfqpoint{2.259901in}{2.139158in}}{\pgfqpoint{2.263174in}{2.131258in}}{\pgfqpoint{2.268998in}{2.125434in}}%
\pgfpathcurveto{\pgfqpoint{2.274822in}{2.119611in}}{\pgfqpoint{2.282722in}{2.116338in}}{\pgfqpoint{2.290958in}{2.116338in}}%
\pgfpathclose%
\pgfusepath{stroke,fill}%
\end{pgfscope}%
\begin{pgfscope}%
\pgfpathrectangle{\pgfqpoint{0.100000in}{0.212622in}}{\pgfqpoint{3.696000in}{3.696000in}}%
\pgfusepath{clip}%
\pgfsetbuttcap%
\pgfsetroundjoin%
\definecolor{currentfill}{rgb}{0.121569,0.466667,0.705882}%
\pgfsetfillcolor{currentfill}%
\pgfsetfillopacity{0.423018}%
\pgfsetlinewidth{1.003750pt}%
\definecolor{currentstroke}{rgb}{0.121569,0.466667,0.705882}%
\pgfsetstrokecolor{currentstroke}%
\pgfsetstrokeopacity{0.423018}%
\pgfsetdash{}{0pt}%
\pgfpathmoveto{\pgfqpoint{1.374605in}{1.866262in}}%
\pgfpathcurveto{\pgfqpoint{1.382842in}{1.866262in}}{\pgfqpoint{1.390742in}{1.869534in}}{\pgfqpoint{1.396566in}{1.875358in}}%
\pgfpathcurveto{\pgfqpoint{1.402390in}{1.881182in}}{\pgfqpoint{1.405662in}{1.889082in}}{\pgfqpoint{1.405662in}{1.897318in}}%
\pgfpathcurveto{\pgfqpoint{1.405662in}{1.905554in}}{\pgfqpoint{1.402390in}{1.913454in}}{\pgfqpoint{1.396566in}{1.919278in}}%
\pgfpathcurveto{\pgfqpoint{1.390742in}{1.925102in}}{\pgfqpoint{1.382842in}{1.928375in}}{\pgfqpoint{1.374605in}{1.928375in}}%
\pgfpathcurveto{\pgfqpoint{1.366369in}{1.928375in}}{\pgfqpoint{1.358469in}{1.925102in}}{\pgfqpoint{1.352645in}{1.919278in}}%
\pgfpathcurveto{\pgfqpoint{1.346821in}{1.913454in}}{\pgfqpoint{1.343549in}{1.905554in}}{\pgfqpoint{1.343549in}{1.897318in}}%
\pgfpathcurveto{\pgfqpoint{1.343549in}{1.889082in}}{\pgfqpoint{1.346821in}{1.881182in}}{\pgfqpoint{1.352645in}{1.875358in}}%
\pgfpathcurveto{\pgfqpoint{1.358469in}{1.869534in}}{\pgfqpoint{1.366369in}{1.866262in}}{\pgfqpoint{1.374605in}{1.866262in}}%
\pgfpathclose%
\pgfusepath{stroke,fill}%
\end{pgfscope}%
\begin{pgfscope}%
\pgfpathrectangle{\pgfqpoint{0.100000in}{0.212622in}}{\pgfqpoint{3.696000in}{3.696000in}}%
\pgfusepath{clip}%
\pgfsetbuttcap%
\pgfsetroundjoin%
\definecolor{currentfill}{rgb}{0.121569,0.466667,0.705882}%
\pgfsetfillcolor{currentfill}%
\pgfsetfillopacity{0.423028}%
\pgfsetlinewidth{1.003750pt}%
\definecolor{currentstroke}{rgb}{0.121569,0.466667,0.705882}%
\pgfsetstrokecolor{currentstroke}%
\pgfsetstrokeopacity{0.423028}%
\pgfsetdash{}{0pt}%
\pgfpathmoveto{\pgfqpoint{2.297203in}{2.118675in}}%
\pgfpathcurveto{\pgfqpoint{2.305439in}{2.118675in}}{\pgfqpoint{2.313340in}{2.121947in}}{\pgfqpoint{2.319163in}{2.127771in}}%
\pgfpathcurveto{\pgfqpoint{2.324987in}{2.133595in}}{\pgfqpoint{2.328260in}{2.141495in}}{\pgfqpoint{2.328260in}{2.149731in}}%
\pgfpathcurveto{\pgfqpoint{2.328260in}{2.157968in}}{\pgfqpoint{2.324987in}{2.165868in}}{\pgfqpoint{2.319163in}{2.171692in}}%
\pgfpathcurveto{\pgfqpoint{2.313340in}{2.177515in}}{\pgfqpoint{2.305439in}{2.180788in}}{\pgfqpoint{2.297203in}{2.180788in}}%
\pgfpathcurveto{\pgfqpoint{2.288967in}{2.180788in}}{\pgfqpoint{2.281067in}{2.177515in}}{\pgfqpoint{2.275243in}{2.171692in}}%
\pgfpathcurveto{\pgfqpoint{2.269419in}{2.165868in}}{\pgfqpoint{2.266147in}{2.157968in}}{\pgfqpoint{2.266147in}{2.149731in}}%
\pgfpathcurveto{\pgfqpoint{2.266147in}{2.141495in}}{\pgfqpoint{2.269419in}{2.133595in}}{\pgfqpoint{2.275243in}{2.127771in}}%
\pgfpathcurveto{\pgfqpoint{2.281067in}{2.121947in}}{\pgfqpoint{2.288967in}{2.118675in}}{\pgfqpoint{2.297203in}{2.118675in}}%
\pgfpathclose%
\pgfusepath{stroke,fill}%
\end{pgfscope}%
\begin{pgfscope}%
\pgfpathrectangle{\pgfqpoint{0.100000in}{0.212622in}}{\pgfqpoint{3.696000in}{3.696000in}}%
\pgfusepath{clip}%
\pgfsetbuttcap%
\pgfsetroundjoin%
\definecolor{currentfill}{rgb}{0.121569,0.466667,0.705882}%
\pgfsetfillcolor{currentfill}%
\pgfsetfillopacity{0.423427}%
\pgfsetlinewidth{1.003750pt}%
\definecolor{currentstroke}{rgb}{0.121569,0.466667,0.705882}%
\pgfsetstrokecolor{currentstroke}%
\pgfsetstrokeopacity{0.423427}%
\pgfsetdash{}{0pt}%
\pgfpathmoveto{\pgfqpoint{2.300333in}{2.116023in}}%
\pgfpathcurveto{\pgfqpoint{2.308570in}{2.116023in}}{\pgfqpoint{2.316470in}{2.119295in}}{\pgfqpoint{2.322294in}{2.125119in}}%
\pgfpathcurveto{\pgfqpoint{2.328118in}{2.130943in}}{\pgfqpoint{2.331390in}{2.138843in}}{\pgfqpoint{2.331390in}{2.147079in}}%
\pgfpathcurveto{\pgfqpoint{2.331390in}{2.155315in}}{\pgfqpoint{2.328118in}{2.163215in}}{\pgfqpoint{2.322294in}{2.169039in}}%
\pgfpathcurveto{\pgfqpoint{2.316470in}{2.174863in}}{\pgfqpoint{2.308570in}{2.178136in}}{\pgfqpoint{2.300333in}{2.178136in}}%
\pgfpathcurveto{\pgfqpoint{2.292097in}{2.178136in}}{\pgfqpoint{2.284197in}{2.174863in}}{\pgfqpoint{2.278373in}{2.169039in}}%
\pgfpathcurveto{\pgfqpoint{2.272549in}{2.163215in}}{\pgfqpoint{2.269277in}{2.155315in}}{\pgfqpoint{2.269277in}{2.147079in}}%
\pgfpathcurveto{\pgfqpoint{2.269277in}{2.138843in}}{\pgfqpoint{2.272549in}{2.130943in}}{\pgfqpoint{2.278373in}{2.125119in}}%
\pgfpathcurveto{\pgfqpoint{2.284197in}{2.119295in}}{\pgfqpoint{2.292097in}{2.116023in}}{\pgfqpoint{2.300333in}{2.116023in}}%
\pgfpathclose%
\pgfusepath{stroke,fill}%
\end{pgfscope}%
\begin{pgfscope}%
\pgfpathrectangle{\pgfqpoint{0.100000in}{0.212622in}}{\pgfqpoint{3.696000in}{3.696000in}}%
\pgfusepath{clip}%
\pgfsetbuttcap%
\pgfsetroundjoin%
\definecolor{currentfill}{rgb}{0.121569,0.466667,0.705882}%
\pgfsetfillcolor{currentfill}%
\pgfsetfillopacity{0.424269}%
\pgfsetlinewidth{1.003750pt}%
\definecolor{currentstroke}{rgb}{0.121569,0.466667,0.705882}%
\pgfsetstrokecolor{currentstroke}%
\pgfsetstrokeopacity{0.424269}%
\pgfsetdash{}{0pt}%
\pgfpathmoveto{\pgfqpoint{2.304817in}{2.116603in}}%
\pgfpathcurveto{\pgfqpoint{2.313053in}{2.116603in}}{\pgfqpoint{2.320954in}{2.119875in}}{\pgfqpoint{2.326777in}{2.125699in}}%
\pgfpathcurveto{\pgfqpoint{2.332601in}{2.131523in}}{\pgfqpoint{2.335874in}{2.139423in}}{\pgfqpoint{2.335874in}{2.147660in}}%
\pgfpathcurveto{\pgfqpoint{2.335874in}{2.155896in}}{\pgfqpoint{2.332601in}{2.163796in}}{\pgfqpoint{2.326777in}{2.169620in}}%
\pgfpathcurveto{\pgfqpoint{2.320954in}{2.175444in}}{\pgfqpoint{2.313053in}{2.178716in}}{\pgfqpoint{2.304817in}{2.178716in}}%
\pgfpathcurveto{\pgfqpoint{2.296581in}{2.178716in}}{\pgfqpoint{2.288681in}{2.175444in}}{\pgfqpoint{2.282857in}{2.169620in}}%
\pgfpathcurveto{\pgfqpoint{2.277033in}{2.163796in}}{\pgfqpoint{2.273761in}{2.155896in}}{\pgfqpoint{2.273761in}{2.147660in}}%
\pgfpathcurveto{\pgfqpoint{2.273761in}{2.139423in}}{\pgfqpoint{2.277033in}{2.131523in}}{\pgfqpoint{2.282857in}{2.125699in}}%
\pgfpathcurveto{\pgfqpoint{2.288681in}{2.119875in}}{\pgfqpoint{2.296581in}{2.116603in}}{\pgfqpoint{2.304817in}{2.116603in}}%
\pgfpathclose%
\pgfusepath{stroke,fill}%
\end{pgfscope}%
\begin{pgfscope}%
\pgfpathrectangle{\pgfqpoint{0.100000in}{0.212622in}}{\pgfqpoint{3.696000in}{3.696000in}}%
\pgfusepath{clip}%
\pgfsetbuttcap%
\pgfsetroundjoin%
\definecolor{currentfill}{rgb}{0.121569,0.466667,0.705882}%
\pgfsetfillcolor{currentfill}%
\pgfsetfillopacity{0.425007}%
\pgfsetlinewidth{1.003750pt}%
\definecolor{currentstroke}{rgb}{0.121569,0.466667,0.705882}%
\pgfsetstrokecolor{currentstroke}%
\pgfsetstrokeopacity{0.425007}%
\pgfsetdash{}{0pt}%
\pgfpathmoveto{\pgfqpoint{2.309764in}{2.114919in}}%
\pgfpathcurveto{\pgfqpoint{2.318001in}{2.114919in}}{\pgfqpoint{2.325901in}{2.118192in}}{\pgfqpoint{2.331725in}{2.124016in}}%
\pgfpathcurveto{\pgfqpoint{2.337549in}{2.129840in}}{\pgfqpoint{2.340821in}{2.137740in}}{\pgfqpoint{2.340821in}{2.145976in}}%
\pgfpathcurveto{\pgfqpoint{2.340821in}{2.154212in}}{\pgfqpoint{2.337549in}{2.162112in}}{\pgfqpoint{2.331725in}{2.167936in}}%
\pgfpathcurveto{\pgfqpoint{2.325901in}{2.173760in}}{\pgfqpoint{2.318001in}{2.177032in}}{\pgfqpoint{2.309764in}{2.177032in}}%
\pgfpathcurveto{\pgfqpoint{2.301528in}{2.177032in}}{\pgfqpoint{2.293628in}{2.173760in}}{\pgfqpoint{2.287804in}{2.167936in}}%
\pgfpathcurveto{\pgfqpoint{2.281980in}{2.162112in}}{\pgfqpoint{2.278708in}{2.154212in}}{\pgfqpoint{2.278708in}{2.145976in}}%
\pgfpathcurveto{\pgfqpoint{2.278708in}{2.137740in}}{\pgfqpoint{2.281980in}{2.129840in}}{\pgfqpoint{2.287804in}{2.124016in}}%
\pgfpathcurveto{\pgfqpoint{2.293628in}{2.118192in}}{\pgfqpoint{2.301528in}{2.114919in}}{\pgfqpoint{2.309764in}{2.114919in}}%
\pgfpathclose%
\pgfusepath{stroke,fill}%
\end{pgfscope}%
\begin{pgfscope}%
\pgfpathrectangle{\pgfqpoint{0.100000in}{0.212622in}}{\pgfqpoint{3.696000in}{3.696000in}}%
\pgfusepath{clip}%
\pgfsetbuttcap%
\pgfsetroundjoin%
\definecolor{currentfill}{rgb}{0.121569,0.466667,0.705882}%
\pgfsetfillcolor{currentfill}%
\pgfsetfillopacity{0.425443}%
\pgfsetlinewidth{1.003750pt}%
\definecolor{currentstroke}{rgb}{0.121569,0.466667,0.705882}%
\pgfsetstrokecolor{currentstroke}%
\pgfsetstrokeopacity{0.425443}%
\pgfsetdash{}{0pt}%
\pgfpathmoveto{\pgfqpoint{1.372667in}{1.864889in}}%
\pgfpathcurveto{\pgfqpoint{1.380904in}{1.864889in}}{\pgfqpoint{1.388804in}{1.868161in}}{\pgfqpoint{1.394628in}{1.873985in}}%
\pgfpathcurveto{\pgfqpoint{1.400452in}{1.879809in}}{\pgfqpoint{1.403724in}{1.887709in}}{\pgfqpoint{1.403724in}{1.895945in}}%
\pgfpathcurveto{\pgfqpoint{1.403724in}{1.904181in}}{\pgfqpoint{1.400452in}{1.912081in}}{\pgfqpoint{1.394628in}{1.917905in}}%
\pgfpathcurveto{\pgfqpoint{1.388804in}{1.923729in}}{\pgfqpoint{1.380904in}{1.927002in}}{\pgfqpoint{1.372667in}{1.927002in}}%
\pgfpathcurveto{\pgfqpoint{1.364431in}{1.927002in}}{\pgfqpoint{1.356531in}{1.923729in}}{\pgfqpoint{1.350707in}{1.917905in}}%
\pgfpathcurveto{\pgfqpoint{1.344883in}{1.912081in}}{\pgfqpoint{1.341611in}{1.904181in}}{\pgfqpoint{1.341611in}{1.895945in}}%
\pgfpathcurveto{\pgfqpoint{1.341611in}{1.887709in}}{\pgfqpoint{1.344883in}{1.879809in}}{\pgfqpoint{1.350707in}{1.873985in}}%
\pgfpathcurveto{\pgfqpoint{1.356531in}{1.868161in}}{\pgfqpoint{1.364431in}{1.864889in}}{\pgfqpoint{1.372667in}{1.864889in}}%
\pgfpathclose%
\pgfusepath{stroke,fill}%
\end{pgfscope}%
\begin{pgfscope}%
\pgfpathrectangle{\pgfqpoint{0.100000in}{0.212622in}}{\pgfqpoint{3.696000in}{3.696000in}}%
\pgfusepath{clip}%
\pgfsetbuttcap%
\pgfsetroundjoin%
\definecolor{currentfill}{rgb}{0.121569,0.466667,0.705882}%
\pgfsetfillcolor{currentfill}%
\pgfsetfillopacity{0.425535}%
\pgfsetlinewidth{1.003750pt}%
\definecolor{currentstroke}{rgb}{0.121569,0.466667,0.705882}%
\pgfsetstrokecolor{currentstroke}%
\pgfsetstrokeopacity{0.425535}%
\pgfsetdash{}{0pt}%
\pgfpathmoveto{\pgfqpoint{2.312531in}{2.115002in}}%
\pgfpathcurveto{\pgfqpoint{2.320767in}{2.115002in}}{\pgfqpoint{2.328667in}{2.118274in}}{\pgfqpoint{2.334491in}{2.124098in}}%
\pgfpathcurveto{\pgfqpoint{2.340315in}{2.129922in}}{\pgfqpoint{2.343587in}{2.137822in}}{\pgfqpoint{2.343587in}{2.146059in}}%
\pgfpathcurveto{\pgfqpoint{2.343587in}{2.154295in}}{\pgfqpoint{2.340315in}{2.162195in}}{\pgfqpoint{2.334491in}{2.168019in}}%
\pgfpathcurveto{\pgfqpoint{2.328667in}{2.173843in}}{\pgfqpoint{2.320767in}{2.177115in}}{\pgfqpoint{2.312531in}{2.177115in}}%
\pgfpathcurveto{\pgfqpoint{2.304295in}{2.177115in}}{\pgfqpoint{2.296395in}{2.173843in}}{\pgfqpoint{2.290571in}{2.168019in}}%
\pgfpathcurveto{\pgfqpoint{2.284747in}{2.162195in}}{\pgfqpoint{2.281474in}{2.154295in}}{\pgfqpoint{2.281474in}{2.146059in}}%
\pgfpathcurveto{\pgfqpoint{2.281474in}{2.137822in}}{\pgfqpoint{2.284747in}{2.129922in}}{\pgfqpoint{2.290571in}{2.124098in}}%
\pgfpathcurveto{\pgfqpoint{2.296395in}{2.118274in}}{\pgfqpoint{2.304295in}{2.115002in}}{\pgfqpoint{2.312531in}{2.115002in}}%
\pgfpathclose%
\pgfusepath{stroke,fill}%
\end{pgfscope}%
\begin{pgfscope}%
\pgfpathrectangle{\pgfqpoint{0.100000in}{0.212622in}}{\pgfqpoint{3.696000in}{3.696000in}}%
\pgfusepath{clip}%
\pgfsetbuttcap%
\pgfsetroundjoin%
\definecolor{currentfill}{rgb}{0.121569,0.466667,0.705882}%
\pgfsetfillcolor{currentfill}%
\pgfsetfillopacity{0.426029}%
\pgfsetlinewidth{1.003750pt}%
\definecolor{currentstroke}{rgb}{0.121569,0.466667,0.705882}%
\pgfsetstrokecolor{currentstroke}%
\pgfsetstrokeopacity{0.426029}%
\pgfsetdash{}{0pt}%
\pgfpathmoveto{\pgfqpoint{2.315817in}{2.114535in}}%
\pgfpathcurveto{\pgfqpoint{2.324054in}{2.114535in}}{\pgfqpoint{2.331954in}{2.117808in}}{\pgfqpoint{2.337777in}{2.123632in}}%
\pgfpathcurveto{\pgfqpoint{2.343601in}{2.129456in}}{\pgfqpoint{2.346874in}{2.137356in}}{\pgfqpoint{2.346874in}{2.145592in}}%
\pgfpathcurveto{\pgfqpoint{2.346874in}{2.153828in}}{\pgfqpoint{2.343601in}{2.161728in}}{\pgfqpoint{2.337777in}{2.167552in}}%
\pgfpathcurveto{\pgfqpoint{2.331954in}{2.173376in}}{\pgfqpoint{2.324054in}{2.176648in}}{\pgfqpoint{2.315817in}{2.176648in}}%
\pgfpathcurveto{\pgfqpoint{2.307581in}{2.176648in}}{\pgfqpoint{2.299681in}{2.173376in}}{\pgfqpoint{2.293857in}{2.167552in}}%
\pgfpathcurveto{\pgfqpoint{2.288033in}{2.161728in}}{\pgfqpoint{2.284761in}{2.153828in}}{\pgfqpoint{2.284761in}{2.145592in}}%
\pgfpathcurveto{\pgfqpoint{2.284761in}{2.137356in}}{\pgfqpoint{2.288033in}{2.129456in}}{\pgfqpoint{2.293857in}{2.123632in}}%
\pgfpathcurveto{\pgfqpoint{2.299681in}{2.117808in}}{\pgfqpoint{2.307581in}{2.114535in}}{\pgfqpoint{2.315817in}{2.114535in}}%
\pgfpathclose%
\pgfusepath{stroke,fill}%
\end{pgfscope}%
\begin{pgfscope}%
\pgfpathrectangle{\pgfqpoint{0.100000in}{0.212622in}}{\pgfqpoint{3.696000in}{3.696000in}}%
\pgfusepath{clip}%
\pgfsetbuttcap%
\pgfsetroundjoin%
\definecolor{currentfill}{rgb}{0.121569,0.466667,0.705882}%
\pgfsetfillcolor{currentfill}%
\pgfsetfillopacity{0.426359}%
\pgfsetlinewidth{1.003750pt}%
\definecolor{currentstroke}{rgb}{0.121569,0.466667,0.705882}%
\pgfsetstrokecolor{currentstroke}%
\pgfsetstrokeopacity{0.426359}%
\pgfsetdash{}{0pt}%
\pgfpathmoveto{\pgfqpoint{2.317642in}{2.114753in}}%
\pgfpathcurveto{\pgfqpoint{2.325878in}{2.114753in}}{\pgfqpoint{2.333778in}{2.118025in}}{\pgfqpoint{2.339602in}{2.123849in}}%
\pgfpathcurveto{\pgfqpoint{2.345426in}{2.129673in}}{\pgfqpoint{2.348698in}{2.137573in}}{\pgfqpoint{2.348698in}{2.145809in}}%
\pgfpathcurveto{\pgfqpoint{2.348698in}{2.154045in}}{\pgfqpoint{2.345426in}{2.161946in}}{\pgfqpoint{2.339602in}{2.167769in}}%
\pgfpathcurveto{\pgfqpoint{2.333778in}{2.173593in}}{\pgfqpoint{2.325878in}{2.176866in}}{\pgfqpoint{2.317642in}{2.176866in}}%
\pgfpathcurveto{\pgfqpoint{2.309405in}{2.176866in}}{\pgfqpoint{2.301505in}{2.173593in}}{\pgfqpoint{2.295681in}{2.167769in}}%
\pgfpathcurveto{\pgfqpoint{2.289857in}{2.161946in}}{\pgfqpoint{2.286585in}{2.154045in}}{\pgfqpoint{2.286585in}{2.145809in}}%
\pgfpathcurveto{\pgfqpoint{2.286585in}{2.137573in}}{\pgfqpoint{2.289857in}{2.129673in}}{\pgfqpoint{2.295681in}{2.123849in}}%
\pgfpathcurveto{\pgfqpoint{2.301505in}{2.118025in}}{\pgfqpoint{2.309405in}{2.114753in}}{\pgfqpoint{2.317642in}{2.114753in}}%
\pgfpathclose%
\pgfusepath{stroke,fill}%
\end{pgfscope}%
\begin{pgfscope}%
\pgfpathrectangle{\pgfqpoint{0.100000in}{0.212622in}}{\pgfqpoint{3.696000in}{3.696000in}}%
\pgfusepath{clip}%
\pgfsetbuttcap%
\pgfsetroundjoin%
\definecolor{currentfill}{rgb}{0.121569,0.466667,0.705882}%
\pgfsetfillcolor{currentfill}%
\pgfsetfillopacity{0.426613}%
\pgfsetlinewidth{1.003750pt}%
\definecolor{currentstroke}{rgb}{0.121569,0.466667,0.705882}%
\pgfsetstrokecolor{currentstroke}%
\pgfsetstrokeopacity{0.426613}%
\pgfsetdash{}{0pt}%
\pgfpathmoveto{\pgfqpoint{1.367757in}{1.861107in}}%
\pgfpathcurveto{\pgfqpoint{1.375994in}{1.861107in}}{\pgfqpoint{1.383894in}{1.864379in}}{\pgfqpoint{1.389718in}{1.870203in}}%
\pgfpathcurveto{\pgfqpoint{1.395541in}{1.876027in}}{\pgfqpoint{1.398814in}{1.883927in}}{\pgfqpoint{1.398814in}{1.892164in}}%
\pgfpathcurveto{\pgfqpoint{1.398814in}{1.900400in}}{\pgfqpoint{1.395541in}{1.908300in}}{\pgfqpoint{1.389718in}{1.914124in}}%
\pgfpathcurveto{\pgfqpoint{1.383894in}{1.919948in}}{\pgfqpoint{1.375994in}{1.923220in}}{\pgfqpoint{1.367757in}{1.923220in}}%
\pgfpathcurveto{\pgfqpoint{1.359521in}{1.923220in}}{\pgfqpoint{1.351621in}{1.919948in}}{\pgfqpoint{1.345797in}{1.914124in}}%
\pgfpathcurveto{\pgfqpoint{1.339973in}{1.908300in}}{\pgfqpoint{1.336701in}{1.900400in}}{\pgfqpoint{1.336701in}{1.892164in}}%
\pgfpathcurveto{\pgfqpoint{1.336701in}{1.883927in}}{\pgfqpoint{1.339973in}{1.876027in}}{\pgfqpoint{1.345797in}{1.870203in}}%
\pgfpathcurveto{\pgfqpoint{1.351621in}{1.864379in}}{\pgfqpoint{1.359521in}{1.861107in}}{\pgfqpoint{1.367757in}{1.861107in}}%
\pgfpathclose%
\pgfusepath{stroke,fill}%
\end{pgfscope}%
\begin{pgfscope}%
\pgfpathrectangle{\pgfqpoint{0.100000in}{0.212622in}}{\pgfqpoint{3.696000in}{3.696000in}}%
\pgfusepath{clip}%
\pgfsetbuttcap%
\pgfsetroundjoin%
\definecolor{currentfill}{rgb}{0.121569,0.466667,0.705882}%
\pgfsetfillcolor{currentfill}%
\pgfsetfillopacity{0.426702}%
\pgfsetlinewidth{1.003750pt}%
\definecolor{currentstroke}{rgb}{0.121569,0.466667,0.705882}%
\pgfsetstrokecolor{currentstroke}%
\pgfsetstrokeopacity{0.426702}%
\pgfsetdash{}{0pt}%
\pgfpathmoveto{\pgfqpoint{2.319919in}{2.114057in}}%
\pgfpathcurveto{\pgfqpoint{2.328155in}{2.114057in}}{\pgfqpoint{2.336055in}{2.117329in}}{\pgfqpoint{2.341879in}{2.123153in}}%
\pgfpathcurveto{\pgfqpoint{2.347703in}{2.128977in}}{\pgfqpoint{2.350975in}{2.136877in}}{\pgfqpoint{2.350975in}{2.145114in}}%
\pgfpathcurveto{\pgfqpoint{2.350975in}{2.153350in}}{\pgfqpoint{2.347703in}{2.161250in}}{\pgfqpoint{2.341879in}{2.167074in}}%
\pgfpathcurveto{\pgfqpoint{2.336055in}{2.172898in}}{\pgfqpoint{2.328155in}{2.176170in}}{\pgfqpoint{2.319919in}{2.176170in}}%
\pgfpathcurveto{\pgfqpoint{2.311683in}{2.176170in}}{\pgfqpoint{2.303783in}{2.172898in}}{\pgfqpoint{2.297959in}{2.167074in}}%
\pgfpathcurveto{\pgfqpoint{2.292135in}{2.161250in}}{\pgfqpoint{2.288862in}{2.153350in}}{\pgfqpoint{2.288862in}{2.145114in}}%
\pgfpathcurveto{\pgfqpoint{2.288862in}{2.136877in}}{\pgfqpoint{2.292135in}{2.128977in}}{\pgfqpoint{2.297959in}{2.123153in}}%
\pgfpathcurveto{\pgfqpoint{2.303783in}{2.117329in}}{\pgfqpoint{2.311683in}{2.114057in}}{\pgfqpoint{2.319919in}{2.114057in}}%
\pgfpathclose%
\pgfusepath{stroke,fill}%
\end{pgfscope}%
\begin{pgfscope}%
\pgfpathrectangle{\pgfqpoint{0.100000in}{0.212622in}}{\pgfqpoint{3.696000in}{3.696000in}}%
\pgfusepath{clip}%
\pgfsetbuttcap%
\pgfsetroundjoin%
\definecolor{currentfill}{rgb}{0.121569,0.466667,0.705882}%
\pgfsetfillcolor{currentfill}%
\pgfsetfillopacity{0.427261}%
\pgfsetlinewidth{1.003750pt}%
\definecolor{currentstroke}{rgb}{0.121569,0.466667,0.705882}%
\pgfsetstrokecolor{currentstroke}%
\pgfsetstrokeopacity{0.427261}%
\pgfsetdash{}{0pt}%
\pgfpathmoveto{\pgfqpoint{2.323057in}{2.114592in}}%
\pgfpathcurveto{\pgfqpoint{2.331293in}{2.114592in}}{\pgfqpoint{2.339193in}{2.117864in}}{\pgfqpoint{2.345017in}{2.123688in}}%
\pgfpathcurveto{\pgfqpoint{2.350841in}{2.129512in}}{\pgfqpoint{2.354113in}{2.137412in}}{\pgfqpoint{2.354113in}{2.145649in}}%
\pgfpathcurveto{\pgfqpoint{2.354113in}{2.153885in}}{\pgfqpoint{2.350841in}{2.161785in}}{\pgfqpoint{2.345017in}{2.167609in}}%
\pgfpathcurveto{\pgfqpoint{2.339193in}{2.173433in}}{\pgfqpoint{2.331293in}{2.176705in}}{\pgfqpoint{2.323057in}{2.176705in}}%
\pgfpathcurveto{\pgfqpoint{2.314821in}{2.176705in}}{\pgfqpoint{2.306921in}{2.173433in}}{\pgfqpoint{2.301097in}{2.167609in}}%
\pgfpathcurveto{\pgfqpoint{2.295273in}{2.161785in}}{\pgfqpoint{2.292000in}{2.153885in}}{\pgfqpoint{2.292000in}{2.145649in}}%
\pgfpathcurveto{\pgfqpoint{2.292000in}{2.137412in}}{\pgfqpoint{2.295273in}{2.129512in}}{\pgfqpoint{2.301097in}{2.123688in}}%
\pgfpathcurveto{\pgfqpoint{2.306921in}{2.117864in}}{\pgfqpoint{2.314821in}{2.114592in}}{\pgfqpoint{2.323057in}{2.114592in}}%
\pgfpathclose%
\pgfusepath{stroke,fill}%
\end{pgfscope}%
\begin{pgfscope}%
\pgfpathrectangle{\pgfqpoint{0.100000in}{0.212622in}}{\pgfqpoint{3.696000in}{3.696000in}}%
\pgfusepath{clip}%
\pgfsetbuttcap%
\pgfsetroundjoin%
\definecolor{currentfill}{rgb}{0.121569,0.466667,0.705882}%
\pgfsetfillcolor{currentfill}%
\pgfsetfillopacity{0.427785}%
\pgfsetlinewidth{1.003750pt}%
\definecolor{currentstroke}{rgb}{0.121569,0.466667,0.705882}%
\pgfsetstrokecolor{currentstroke}%
\pgfsetstrokeopacity{0.427785}%
\pgfsetdash{}{0pt}%
\pgfpathmoveto{\pgfqpoint{1.364522in}{1.858164in}}%
\pgfpathcurveto{\pgfqpoint{1.372759in}{1.858164in}}{\pgfqpoint{1.380659in}{1.861437in}}{\pgfqpoint{1.386483in}{1.867261in}}%
\pgfpathcurveto{\pgfqpoint{1.392307in}{1.873085in}}{\pgfqpoint{1.395579in}{1.880985in}}{\pgfqpoint{1.395579in}{1.889221in}}%
\pgfpathcurveto{\pgfqpoint{1.395579in}{1.897457in}}{\pgfqpoint{1.392307in}{1.905357in}}{\pgfqpoint{1.386483in}{1.911181in}}%
\pgfpathcurveto{\pgfqpoint{1.380659in}{1.917005in}}{\pgfqpoint{1.372759in}{1.920277in}}{\pgfqpoint{1.364522in}{1.920277in}}%
\pgfpathcurveto{\pgfqpoint{1.356286in}{1.920277in}}{\pgfqpoint{1.348386in}{1.917005in}}{\pgfqpoint{1.342562in}{1.911181in}}%
\pgfpathcurveto{\pgfqpoint{1.336738in}{1.905357in}}{\pgfqpoint{1.333466in}{1.897457in}}{\pgfqpoint{1.333466in}{1.889221in}}%
\pgfpathcurveto{\pgfqpoint{1.333466in}{1.880985in}}{\pgfqpoint{1.336738in}{1.873085in}}{\pgfqpoint{1.342562in}{1.867261in}}%
\pgfpathcurveto{\pgfqpoint{1.348386in}{1.861437in}}{\pgfqpoint{1.356286in}{1.858164in}}{\pgfqpoint{1.364522in}{1.858164in}}%
\pgfpathclose%
\pgfusepath{stroke,fill}%
\end{pgfscope}%
\begin{pgfscope}%
\pgfpathrectangle{\pgfqpoint{0.100000in}{0.212622in}}{\pgfqpoint{3.696000in}{3.696000in}}%
\pgfusepath{clip}%
\pgfsetbuttcap%
\pgfsetroundjoin%
\definecolor{currentfill}{rgb}{0.121569,0.466667,0.705882}%
\pgfsetfillcolor{currentfill}%
\pgfsetfillopacity{0.427821}%
\pgfsetlinewidth{1.003750pt}%
\definecolor{currentstroke}{rgb}{0.121569,0.466667,0.705882}%
\pgfsetstrokecolor{currentstroke}%
\pgfsetstrokeopacity{0.427821}%
\pgfsetdash{}{0pt}%
\pgfpathmoveto{\pgfqpoint{2.326907in}{2.113185in}}%
\pgfpathcurveto{\pgfqpoint{2.335143in}{2.113185in}}{\pgfqpoint{2.343043in}{2.116457in}}{\pgfqpoint{2.348867in}{2.122281in}}%
\pgfpathcurveto{\pgfqpoint{2.354691in}{2.128105in}}{\pgfqpoint{2.357963in}{2.136005in}}{\pgfqpoint{2.357963in}{2.144241in}}%
\pgfpathcurveto{\pgfqpoint{2.357963in}{2.152477in}}{\pgfqpoint{2.354691in}{2.160377in}}{\pgfqpoint{2.348867in}{2.166201in}}%
\pgfpathcurveto{\pgfqpoint{2.343043in}{2.172025in}}{\pgfqpoint{2.335143in}{2.175298in}}{\pgfqpoint{2.326907in}{2.175298in}}%
\pgfpathcurveto{\pgfqpoint{2.318670in}{2.175298in}}{\pgfqpoint{2.310770in}{2.172025in}}{\pgfqpoint{2.304946in}{2.166201in}}%
\pgfpathcurveto{\pgfqpoint{2.299122in}{2.160377in}}{\pgfqpoint{2.295850in}{2.152477in}}{\pgfqpoint{2.295850in}{2.144241in}}%
\pgfpathcurveto{\pgfqpoint{2.295850in}{2.136005in}}{\pgfqpoint{2.299122in}{2.128105in}}{\pgfqpoint{2.304946in}{2.122281in}}%
\pgfpathcurveto{\pgfqpoint{2.310770in}{2.116457in}}{\pgfqpoint{2.318670in}{2.113185in}}{\pgfqpoint{2.326907in}{2.113185in}}%
\pgfpathclose%
\pgfusepath{stroke,fill}%
\end{pgfscope}%
\begin{pgfscope}%
\pgfpathrectangle{\pgfqpoint{0.100000in}{0.212622in}}{\pgfqpoint{3.696000in}{3.696000in}}%
\pgfusepath{clip}%
\pgfsetbuttcap%
\pgfsetroundjoin%
\definecolor{currentfill}{rgb}{0.121569,0.466667,0.705882}%
\pgfsetfillcolor{currentfill}%
\pgfsetfillopacity{0.428616}%
\pgfsetlinewidth{1.003750pt}%
\definecolor{currentstroke}{rgb}{0.121569,0.466667,0.705882}%
\pgfsetstrokecolor{currentstroke}%
\pgfsetstrokeopacity{0.428616}%
\pgfsetdash{}{0pt}%
\pgfpathmoveto{\pgfqpoint{2.331353in}{2.112645in}}%
\pgfpathcurveto{\pgfqpoint{2.339589in}{2.112645in}}{\pgfqpoint{2.347489in}{2.115917in}}{\pgfqpoint{2.353313in}{2.121741in}}%
\pgfpathcurveto{\pgfqpoint{2.359137in}{2.127565in}}{\pgfqpoint{2.362409in}{2.135465in}}{\pgfqpoint{2.362409in}{2.143702in}}%
\pgfpathcurveto{\pgfqpoint{2.362409in}{2.151938in}}{\pgfqpoint{2.359137in}{2.159838in}}{\pgfqpoint{2.353313in}{2.165662in}}%
\pgfpathcurveto{\pgfqpoint{2.347489in}{2.171486in}}{\pgfqpoint{2.339589in}{2.174758in}}{\pgfqpoint{2.331353in}{2.174758in}}%
\pgfpathcurveto{\pgfqpoint{2.323116in}{2.174758in}}{\pgfqpoint{2.315216in}{2.171486in}}{\pgfqpoint{2.309392in}{2.165662in}}%
\pgfpathcurveto{\pgfqpoint{2.303568in}{2.159838in}}{\pgfqpoint{2.300296in}{2.151938in}}{\pgfqpoint{2.300296in}{2.143702in}}%
\pgfpathcurveto{\pgfqpoint{2.300296in}{2.135465in}}{\pgfqpoint{2.303568in}{2.127565in}}{\pgfqpoint{2.309392in}{2.121741in}}%
\pgfpathcurveto{\pgfqpoint{2.315216in}{2.115917in}}{\pgfqpoint{2.323116in}{2.112645in}}{\pgfqpoint{2.331353in}{2.112645in}}%
\pgfpathclose%
\pgfusepath{stroke,fill}%
\end{pgfscope}%
\begin{pgfscope}%
\pgfpathrectangle{\pgfqpoint{0.100000in}{0.212622in}}{\pgfqpoint{3.696000in}{3.696000in}}%
\pgfusepath{clip}%
\pgfsetbuttcap%
\pgfsetroundjoin%
\definecolor{currentfill}{rgb}{0.121569,0.466667,0.705882}%
\pgfsetfillcolor{currentfill}%
\pgfsetfillopacity{0.428937}%
\pgfsetlinewidth{1.003750pt}%
\definecolor{currentstroke}{rgb}{0.121569,0.466667,0.705882}%
\pgfsetstrokecolor{currentstroke}%
\pgfsetstrokeopacity{0.428937}%
\pgfsetdash{}{0pt}%
\pgfpathmoveto{\pgfqpoint{1.362443in}{1.854801in}}%
\pgfpathcurveto{\pgfqpoint{1.370679in}{1.854801in}}{\pgfqpoint{1.378579in}{1.858074in}}{\pgfqpoint{1.384403in}{1.863898in}}%
\pgfpathcurveto{\pgfqpoint{1.390227in}{1.869722in}}{\pgfqpoint{1.393500in}{1.877622in}}{\pgfqpoint{1.393500in}{1.885858in}}%
\pgfpathcurveto{\pgfqpoint{1.393500in}{1.894094in}}{\pgfqpoint{1.390227in}{1.901994in}}{\pgfqpoint{1.384403in}{1.907818in}}%
\pgfpathcurveto{\pgfqpoint{1.378579in}{1.913642in}}{\pgfqpoint{1.370679in}{1.916914in}}{\pgfqpoint{1.362443in}{1.916914in}}%
\pgfpathcurveto{\pgfqpoint{1.354207in}{1.916914in}}{\pgfqpoint{1.346307in}{1.913642in}}{\pgfqpoint{1.340483in}{1.907818in}}%
\pgfpathcurveto{\pgfqpoint{1.334659in}{1.901994in}}{\pgfqpoint{1.331387in}{1.894094in}}{\pgfqpoint{1.331387in}{1.885858in}}%
\pgfpathcurveto{\pgfqpoint{1.331387in}{1.877622in}}{\pgfqpoint{1.334659in}{1.869722in}}{\pgfqpoint{1.340483in}{1.863898in}}%
\pgfpathcurveto{\pgfqpoint{1.346307in}{1.858074in}}{\pgfqpoint{1.354207in}{1.854801in}}{\pgfqpoint{1.362443in}{1.854801in}}%
\pgfpathclose%
\pgfusepath{stroke,fill}%
\end{pgfscope}%
\begin{pgfscope}%
\pgfpathrectangle{\pgfqpoint{0.100000in}{0.212622in}}{\pgfqpoint{3.696000in}{3.696000in}}%
\pgfusepath{clip}%
\pgfsetbuttcap%
\pgfsetroundjoin%
\definecolor{currentfill}{rgb}{0.121569,0.466667,0.705882}%
\pgfsetfillcolor{currentfill}%
\pgfsetfillopacity{0.429484}%
\pgfsetlinewidth{1.003750pt}%
\definecolor{currentstroke}{rgb}{0.121569,0.466667,0.705882}%
\pgfsetstrokecolor{currentstroke}%
\pgfsetstrokeopacity{0.429484}%
\pgfsetdash{}{0pt}%
\pgfpathmoveto{\pgfqpoint{2.336349in}{2.111770in}}%
\pgfpathcurveto{\pgfqpoint{2.344585in}{2.111770in}}{\pgfqpoint{2.352485in}{2.115043in}}{\pgfqpoint{2.358309in}{2.120867in}}%
\pgfpathcurveto{\pgfqpoint{2.364133in}{2.126691in}}{\pgfqpoint{2.367405in}{2.134591in}}{\pgfqpoint{2.367405in}{2.142827in}}%
\pgfpathcurveto{\pgfqpoint{2.367405in}{2.151063in}}{\pgfqpoint{2.364133in}{2.158963in}}{\pgfqpoint{2.358309in}{2.164787in}}%
\pgfpathcurveto{\pgfqpoint{2.352485in}{2.170611in}}{\pgfqpoint{2.344585in}{2.173883in}}{\pgfqpoint{2.336349in}{2.173883in}}%
\pgfpathcurveto{\pgfqpoint{2.328112in}{2.173883in}}{\pgfqpoint{2.320212in}{2.170611in}}{\pgfqpoint{2.314388in}{2.164787in}}%
\pgfpathcurveto{\pgfqpoint{2.308564in}{2.158963in}}{\pgfqpoint{2.305292in}{2.151063in}}{\pgfqpoint{2.305292in}{2.142827in}}%
\pgfpathcurveto{\pgfqpoint{2.305292in}{2.134591in}}{\pgfqpoint{2.308564in}{2.126691in}}{\pgfqpoint{2.314388in}{2.120867in}}%
\pgfpathcurveto{\pgfqpoint{2.320212in}{2.115043in}}{\pgfqpoint{2.328112in}{2.111770in}}{\pgfqpoint{2.336349in}{2.111770in}}%
\pgfpathclose%
\pgfusepath{stroke,fill}%
\end{pgfscope}%
\begin{pgfscope}%
\pgfpathrectangle{\pgfqpoint{0.100000in}{0.212622in}}{\pgfqpoint{3.696000in}{3.696000in}}%
\pgfusepath{clip}%
\pgfsetbuttcap%
\pgfsetroundjoin%
\definecolor{currentfill}{rgb}{0.121569,0.466667,0.705882}%
\pgfsetfillcolor{currentfill}%
\pgfsetfillopacity{0.429853}%
\pgfsetlinewidth{1.003750pt}%
\definecolor{currentstroke}{rgb}{0.121569,0.466667,0.705882}%
\pgfsetstrokecolor{currentstroke}%
\pgfsetstrokeopacity{0.429853}%
\pgfsetdash{}{0pt}%
\pgfpathmoveto{\pgfqpoint{1.358356in}{1.853900in}}%
\pgfpathcurveto{\pgfqpoint{1.366593in}{1.853900in}}{\pgfqpoint{1.374493in}{1.857172in}}{\pgfqpoint{1.380317in}{1.862996in}}%
\pgfpathcurveto{\pgfqpoint{1.386141in}{1.868820in}}{\pgfqpoint{1.389413in}{1.876720in}}{\pgfqpoint{1.389413in}{1.884956in}}%
\pgfpathcurveto{\pgfqpoint{1.389413in}{1.893192in}}{\pgfqpoint{1.386141in}{1.901092in}}{\pgfqpoint{1.380317in}{1.906916in}}%
\pgfpathcurveto{\pgfqpoint{1.374493in}{1.912740in}}{\pgfqpoint{1.366593in}{1.916013in}}{\pgfqpoint{1.358356in}{1.916013in}}%
\pgfpathcurveto{\pgfqpoint{1.350120in}{1.916013in}}{\pgfqpoint{1.342220in}{1.912740in}}{\pgfqpoint{1.336396in}{1.906916in}}%
\pgfpathcurveto{\pgfqpoint{1.330572in}{1.901092in}}{\pgfqpoint{1.327300in}{1.893192in}}{\pgfqpoint{1.327300in}{1.884956in}}%
\pgfpathcurveto{\pgfqpoint{1.327300in}{1.876720in}}{\pgfqpoint{1.330572in}{1.868820in}}{\pgfqpoint{1.336396in}{1.862996in}}%
\pgfpathcurveto{\pgfqpoint{1.342220in}{1.857172in}}{\pgfqpoint{1.350120in}{1.853900in}}{\pgfqpoint{1.358356in}{1.853900in}}%
\pgfpathclose%
\pgfusepath{stroke,fill}%
\end{pgfscope}%
\begin{pgfscope}%
\pgfpathrectangle{\pgfqpoint{0.100000in}{0.212622in}}{\pgfqpoint{3.696000in}{3.696000in}}%
\pgfusepath{clip}%
\pgfsetbuttcap%
\pgfsetroundjoin%
\definecolor{currentfill}{rgb}{0.121569,0.466667,0.705882}%
\pgfsetfillcolor{currentfill}%
\pgfsetfillopacity{0.430303}%
\pgfsetlinewidth{1.003750pt}%
\definecolor{currentstroke}{rgb}{0.121569,0.466667,0.705882}%
\pgfsetstrokecolor{currentstroke}%
\pgfsetstrokeopacity{0.430303}%
\pgfsetdash{}{0pt}%
\pgfpathmoveto{\pgfqpoint{2.343891in}{2.106588in}}%
\pgfpathcurveto{\pgfqpoint{2.352128in}{2.106588in}}{\pgfqpoint{2.360028in}{2.109861in}}{\pgfqpoint{2.365852in}{2.115685in}}%
\pgfpathcurveto{\pgfqpoint{2.371676in}{2.121509in}}{\pgfqpoint{2.374948in}{2.129409in}}{\pgfqpoint{2.374948in}{2.137645in}}%
\pgfpathcurveto{\pgfqpoint{2.374948in}{2.145881in}}{\pgfqpoint{2.371676in}{2.153781in}}{\pgfqpoint{2.365852in}{2.159605in}}%
\pgfpathcurveto{\pgfqpoint{2.360028in}{2.165429in}}{\pgfqpoint{2.352128in}{2.168701in}}{\pgfqpoint{2.343891in}{2.168701in}}%
\pgfpathcurveto{\pgfqpoint{2.335655in}{2.168701in}}{\pgfqpoint{2.327755in}{2.165429in}}{\pgfqpoint{2.321931in}{2.159605in}}%
\pgfpathcurveto{\pgfqpoint{2.316107in}{2.153781in}}{\pgfqpoint{2.312835in}{2.145881in}}{\pgfqpoint{2.312835in}{2.137645in}}%
\pgfpathcurveto{\pgfqpoint{2.312835in}{2.129409in}}{\pgfqpoint{2.316107in}{2.121509in}}{\pgfqpoint{2.321931in}{2.115685in}}%
\pgfpathcurveto{\pgfqpoint{2.327755in}{2.109861in}}{\pgfqpoint{2.335655in}{2.106588in}}{\pgfqpoint{2.343891in}{2.106588in}}%
\pgfpathclose%
\pgfusepath{stroke,fill}%
\end{pgfscope}%
\begin{pgfscope}%
\pgfpathrectangle{\pgfqpoint{0.100000in}{0.212622in}}{\pgfqpoint{3.696000in}{3.696000in}}%
\pgfusepath{clip}%
\pgfsetbuttcap%
\pgfsetroundjoin%
\definecolor{currentfill}{rgb}{0.121569,0.466667,0.705882}%
\pgfsetfillcolor{currentfill}%
\pgfsetfillopacity{0.431290}%
\pgfsetlinewidth{1.003750pt}%
\definecolor{currentstroke}{rgb}{0.121569,0.466667,0.705882}%
\pgfsetstrokecolor{currentstroke}%
\pgfsetstrokeopacity{0.431290}%
\pgfsetdash{}{0pt}%
\pgfpathmoveto{\pgfqpoint{1.356567in}{1.855629in}}%
\pgfpathcurveto{\pgfqpoint{1.364803in}{1.855629in}}{\pgfqpoint{1.372704in}{1.858901in}}{\pgfqpoint{1.378527in}{1.864725in}}%
\pgfpathcurveto{\pgfqpoint{1.384351in}{1.870549in}}{\pgfqpoint{1.387624in}{1.878449in}}{\pgfqpoint{1.387624in}{1.886685in}}%
\pgfpathcurveto{\pgfqpoint{1.387624in}{1.894922in}}{\pgfqpoint{1.384351in}{1.902822in}}{\pgfqpoint{1.378527in}{1.908646in}}%
\pgfpathcurveto{\pgfqpoint{1.372704in}{1.914470in}}{\pgfqpoint{1.364803in}{1.917742in}}{\pgfqpoint{1.356567in}{1.917742in}}%
\pgfpathcurveto{\pgfqpoint{1.348331in}{1.917742in}}{\pgfqpoint{1.340431in}{1.914470in}}{\pgfqpoint{1.334607in}{1.908646in}}%
\pgfpathcurveto{\pgfqpoint{1.328783in}{1.902822in}}{\pgfqpoint{1.325511in}{1.894922in}}{\pgfqpoint{1.325511in}{1.886685in}}%
\pgfpathcurveto{\pgfqpoint{1.325511in}{1.878449in}}{\pgfqpoint{1.328783in}{1.870549in}}{\pgfqpoint{1.334607in}{1.864725in}}%
\pgfpathcurveto{\pgfqpoint{1.340431in}{1.858901in}}{\pgfqpoint{1.348331in}{1.855629in}}{\pgfqpoint{1.356567in}{1.855629in}}%
\pgfpathclose%
\pgfusepath{stroke,fill}%
\end{pgfscope}%
\begin{pgfscope}%
\pgfpathrectangle{\pgfqpoint{0.100000in}{0.212622in}}{\pgfqpoint{3.696000in}{3.696000in}}%
\pgfusepath{clip}%
\pgfsetbuttcap%
\pgfsetroundjoin%
\definecolor{currentfill}{rgb}{0.121569,0.466667,0.705882}%
\pgfsetfillcolor{currentfill}%
\pgfsetfillopacity{0.431539}%
\pgfsetlinewidth{1.003750pt}%
\definecolor{currentstroke}{rgb}{0.121569,0.466667,0.705882}%
\pgfsetstrokecolor{currentstroke}%
\pgfsetstrokeopacity{0.431539}%
\pgfsetdash{}{0pt}%
\pgfpathmoveto{\pgfqpoint{2.352921in}{2.106271in}}%
\pgfpathcurveto{\pgfqpoint{2.361158in}{2.106271in}}{\pgfqpoint{2.369058in}{2.109544in}}{\pgfqpoint{2.374881in}{2.115368in}}%
\pgfpathcurveto{\pgfqpoint{2.380705in}{2.121192in}}{\pgfqpoint{2.383978in}{2.129092in}}{\pgfqpoint{2.383978in}{2.137328in}}%
\pgfpathcurveto{\pgfqpoint{2.383978in}{2.145564in}}{\pgfqpoint{2.380705in}{2.153464in}}{\pgfqpoint{2.374881in}{2.159288in}}%
\pgfpathcurveto{\pgfqpoint{2.369058in}{2.165112in}}{\pgfqpoint{2.361158in}{2.168384in}}{\pgfqpoint{2.352921in}{2.168384in}}%
\pgfpathcurveto{\pgfqpoint{2.344685in}{2.168384in}}{\pgfqpoint{2.336785in}{2.165112in}}{\pgfqpoint{2.330961in}{2.159288in}}%
\pgfpathcurveto{\pgfqpoint{2.325137in}{2.153464in}}{\pgfqpoint{2.321865in}{2.145564in}}{\pgfqpoint{2.321865in}{2.137328in}}%
\pgfpathcurveto{\pgfqpoint{2.321865in}{2.129092in}}{\pgfqpoint{2.325137in}{2.121192in}}{\pgfqpoint{2.330961in}{2.115368in}}%
\pgfpathcurveto{\pgfqpoint{2.336785in}{2.109544in}}{\pgfqpoint{2.344685in}{2.106271in}}{\pgfqpoint{2.352921in}{2.106271in}}%
\pgfpathclose%
\pgfusepath{stroke,fill}%
\end{pgfscope}%
\begin{pgfscope}%
\pgfpathrectangle{\pgfqpoint{0.100000in}{0.212622in}}{\pgfqpoint{3.696000in}{3.696000in}}%
\pgfusepath{clip}%
\pgfsetbuttcap%
\pgfsetroundjoin%
\definecolor{currentfill}{rgb}{0.121569,0.466667,0.705882}%
\pgfsetfillcolor{currentfill}%
\pgfsetfillopacity{0.431610}%
\pgfsetlinewidth{1.003750pt}%
\definecolor{currentstroke}{rgb}{0.121569,0.466667,0.705882}%
\pgfsetstrokecolor{currentstroke}%
\pgfsetstrokeopacity{0.431610}%
\pgfsetdash{}{0pt}%
\pgfpathmoveto{\pgfqpoint{1.350702in}{1.851097in}}%
\pgfpathcurveto{\pgfqpoint{1.358938in}{1.851097in}}{\pgfqpoint{1.366838in}{1.854369in}}{\pgfqpoint{1.372662in}{1.860193in}}%
\pgfpathcurveto{\pgfqpoint{1.378486in}{1.866017in}}{\pgfqpoint{1.381758in}{1.873917in}}{\pgfqpoint{1.381758in}{1.882153in}}%
\pgfpathcurveto{\pgfqpoint{1.381758in}{1.890390in}}{\pgfqpoint{1.378486in}{1.898290in}}{\pgfqpoint{1.372662in}{1.904113in}}%
\pgfpathcurveto{\pgfqpoint{1.366838in}{1.909937in}}{\pgfqpoint{1.358938in}{1.913210in}}{\pgfqpoint{1.350702in}{1.913210in}}%
\pgfpathcurveto{\pgfqpoint{1.342466in}{1.913210in}}{\pgfqpoint{1.334566in}{1.909937in}}{\pgfqpoint{1.328742in}{1.904113in}}%
\pgfpathcurveto{\pgfqpoint{1.322918in}{1.898290in}}{\pgfqpoint{1.319645in}{1.890390in}}{\pgfqpoint{1.319645in}{1.882153in}}%
\pgfpathcurveto{\pgfqpoint{1.319645in}{1.873917in}}{\pgfqpoint{1.322918in}{1.866017in}}{\pgfqpoint{1.328742in}{1.860193in}}%
\pgfpathcurveto{\pgfqpoint{1.334566in}{1.854369in}}{\pgfqpoint{1.342466in}{1.851097in}}{\pgfqpoint{1.350702in}{1.851097in}}%
\pgfpathclose%
\pgfusepath{stroke,fill}%
\end{pgfscope}%
\begin{pgfscope}%
\pgfpathrectangle{\pgfqpoint{0.100000in}{0.212622in}}{\pgfqpoint{3.696000in}{3.696000in}}%
\pgfusepath{clip}%
\pgfsetbuttcap%
\pgfsetroundjoin%
\definecolor{currentfill}{rgb}{0.121569,0.466667,0.705882}%
\pgfsetfillcolor{currentfill}%
\pgfsetfillopacity{0.431743}%
\pgfsetlinewidth{1.003750pt}%
\definecolor{currentstroke}{rgb}{0.121569,0.466667,0.705882}%
\pgfsetstrokecolor{currentstroke}%
\pgfsetstrokeopacity{0.431743}%
\pgfsetdash{}{0pt}%
\pgfpathmoveto{\pgfqpoint{1.353303in}{1.855731in}}%
\pgfpathcurveto{\pgfqpoint{1.361540in}{1.855731in}}{\pgfqpoint{1.369440in}{1.859003in}}{\pgfqpoint{1.375264in}{1.864827in}}%
\pgfpathcurveto{\pgfqpoint{1.381088in}{1.870651in}}{\pgfqpoint{1.384360in}{1.878551in}}{\pgfqpoint{1.384360in}{1.886787in}}%
\pgfpathcurveto{\pgfqpoint{1.384360in}{1.895024in}}{\pgfqpoint{1.381088in}{1.902924in}}{\pgfqpoint{1.375264in}{1.908748in}}%
\pgfpathcurveto{\pgfqpoint{1.369440in}{1.914572in}}{\pgfqpoint{1.361540in}{1.917844in}}{\pgfqpoint{1.353303in}{1.917844in}}%
\pgfpathcurveto{\pgfqpoint{1.345067in}{1.917844in}}{\pgfqpoint{1.337167in}{1.914572in}}{\pgfqpoint{1.331343in}{1.908748in}}%
\pgfpathcurveto{\pgfqpoint{1.325519in}{1.902924in}}{\pgfqpoint{1.322247in}{1.895024in}}{\pgfqpoint{1.322247in}{1.886787in}}%
\pgfpathcurveto{\pgfqpoint{1.322247in}{1.878551in}}{\pgfqpoint{1.325519in}{1.870651in}}{\pgfqpoint{1.331343in}{1.864827in}}%
\pgfpathcurveto{\pgfqpoint{1.337167in}{1.859003in}}{\pgfqpoint{1.345067in}{1.855731in}}{\pgfqpoint{1.353303in}{1.855731in}}%
\pgfpathclose%
\pgfusepath{stroke,fill}%
\end{pgfscope}%
\begin{pgfscope}%
\pgfpathrectangle{\pgfqpoint{0.100000in}{0.212622in}}{\pgfqpoint{3.696000in}{3.696000in}}%
\pgfusepath{clip}%
\pgfsetbuttcap%
\pgfsetroundjoin%
\definecolor{currentfill}{rgb}{0.121569,0.466667,0.705882}%
\pgfsetfillcolor{currentfill}%
\pgfsetfillopacity{0.432137}%
\pgfsetlinewidth{1.003750pt}%
\definecolor{currentstroke}{rgb}{0.121569,0.466667,0.705882}%
\pgfsetstrokecolor{currentstroke}%
\pgfsetstrokeopacity{0.432137}%
\pgfsetdash{}{0pt}%
\pgfpathmoveto{\pgfqpoint{2.357595in}{2.104523in}}%
\pgfpathcurveto{\pgfqpoint{2.365832in}{2.104523in}}{\pgfqpoint{2.373732in}{2.107795in}}{\pgfqpoint{2.379556in}{2.113619in}}%
\pgfpathcurveto{\pgfqpoint{2.385380in}{2.119443in}}{\pgfqpoint{2.388652in}{2.127343in}}{\pgfqpoint{2.388652in}{2.135579in}}%
\pgfpathcurveto{\pgfqpoint{2.388652in}{2.143816in}}{\pgfqpoint{2.385380in}{2.151716in}}{\pgfqpoint{2.379556in}{2.157540in}}%
\pgfpathcurveto{\pgfqpoint{2.373732in}{2.163364in}}{\pgfqpoint{2.365832in}{2.166636in}}{\pgfqpoint{2.357595in}{2.166636in}}%
\pgfpathcurveto{\pgfqpoint{2.349359in}{2.166636in}}{\pgfqpoint{2.341459in}{2.163364in}}{\pgfqpoint{2.335635in}{2.157540in}}%
\pgfpathcurveto{\pgfqpoint{2.329811in}{2.151716in}}{\pgfqpoint{2.326539in}{2.143816in}}{\pgfqpoint{2.326539in}{2.135579in}}%
\pgfpathcurveto{\pgfqpoint{2.326539in}{2.127343in}}{\pgfqpoint{2.329811in}{2.119443in}}{\pgfqpoint{2.335635in}{2.113619in}}%
\pgfpathcurveto{\pgfqpoint{2.341459in}{2.107795in}}{\pgfqpoint{2.349359in}{2.104523in}}{\pgfqpoint{2.357595in}{2.104523in}}%
\pgfpathclose%
\pgfusepath{stroke,fill}%
\end{pgfscope}%
\begin{pgfscope}%
\pgfpathrectangle{\pgfqpoint{0.100000in}{0.212622in}}{\pgfqpoint{3.696000in}{3.696000in}}%
\pgfusepath{clip}%
\pgfsetbuttcap%
\pgfsetroundjoin%
\definecolor{currentfill}{rgb}{0.121569,0.466667,0.705882}%
\pgfsetfillcolor{currentfill}%
\pgfsetfillopacity{0.432524}%
\pgfsetlinewidth{1.003750pt}%
\definecolor{currentstroke}{rgb}{0.121569,0.466667,0.705882}%
\pgfsetstrokecolor{currentstroke}%
\pgfsetstrokeopacity{0.432524}%
\pgfsetdash{}{0pt}%
\pgfpathmoveto{\pgfqpoint{2.360219in}{2.104145in}}%
\pgfpathcurveto{\pgfqpoint{2.368455in}{2.104145in}}{\pgfqpoint{2.376356in}{2.107417in}}{\pgfqpoint{2.382179in}{2.113241in}}%
\pgfpathcurveto{\pgfqpoint{2.388003in}{2.119065in}}{\pgfqpoint{2.391276in}{2.126965in}}{\pgfqpoint{2.391276in}{2.135201in}}%
\pgfpathcurveto{\pgfqpoint{2.391276in}{2.143437in}}{\pgfqpoint{2.388003in}{2.151338in}}{\pgfqpoint{2.382179in}{2.157161in}}%
\pgfpathcurveto{\pgfqpoint{2.376356in}{2.162985in}}{\pgfqpoint{2.368455in}{2.166258in}}{\pgfqpoint{2.360219in}{2.166258in}}%
\pgfpathcurveto{\pgfqpoint{2.351983in}{2.166258in}}{\pgfqpoint{2.344083in}{2.162985in}}{\pgfqpoint{2.338259in}{2.157161in}}%
\pgfpathcurveto{\pgfqpoint{2.332435in}{2.151338in}}{\pgfqpoint{2.329163in}{2.143437in}}{\pgfqpoint{2.329163in}{2.135201in}}%
\pgfpathcurveto{\pgfqpoint{2.329163in}{2.126965in}}{\pgfqpoint{2.332435in}{2.119065in}}{\pgfqpoint{2.338259in}{2.113241in}}%
\pgfpathcurveto{\pgfqpoint{2.344083in}{2.107417in}}{\pgfqpoint{2.351983in}{2.104145in}}{\pgfqpoint{2.360219in}{2.104145in}}%
\pgfpathclose%
\pgfusepath{stroke,fill}%
\end{pgfscope}%
\begin{pgfscope}%
\pgfpathrectangle{\pgfqpoint{0.100000in}{0.212622in}}{\pgfqpoint{3.696000in}{3.696000in}}%
\pgfusepath{clip}%
\pgfsetbuttcap%
\pgfsetroundjoin%
\definecolor{currentfill}{rgb}{0.121569,0.466667,0.705882}%
\pgfsetfillcolor{currentfill}%
\pgfsetfillopacity{0.432676}%
\pgfsetlinewidth{1.003750pt}%
\definecolor{currentstroke}{rgb}{0.121569,0.466667,0.705882}%
\pgfsetstrokecolor{currentstroke}%
\pgfsetstrokeopacity{0.432676}%
\pgfsetdash{}{0pt}%
\pgfpathmoveto{\pgfqpoint{2.363627in}{2.100336in}}%
\pgfpathcurveto{\pgfqpoint{2.371864in}{2.100336in}}{\pgfqpoint{2.379764in}{2.103608in}}{\pgfqpoint{2.385587in}{2.109432in}}%
\pgfpathcurveto{\pgfqpoint{2.391411in}{2.115256in}}{\pgfqpoint{2.394684in}{2.123156in}}{\pgfqpoint{2.394684in}{2.131393in}}%
\pgfpathcurveto{\pgfqpoint{2.394684in}{2.139629in}}{\pgfqpoint{2.391411in}{2.147529in}}{\pgfqpoint{2.385587in}{2.153353in}}%
\pgfpathcurveto{\pgfqpoint{2.379764in}{2.159177in}}{\pgfqpoint{2.371864in}{2.162449in}}{\pgfqpoint{2.363627in}{2.162449in}}%
\pgfpathcurveto{\pgfqpoint{2.355391in}{2.162449in}}{\pgfqpoint{2.347491in}{2.159177in}}{\pgfqpoint{2.341667in}{2.153353in}}%
\pgfpathcurveto{\pgfqpoint{2.335843in}{2.147529in}}{\pgfqpoint{2.332571in}{2.139629in}}{\pgfqpoint{2.332571in}{2.131393in}}%
\pgfpathcurveto{\pgfqpoint{2.332571in}{2.123156in}}{\pgfqpoint{2.335843in}{2.115256in}}{\pgfqpoint{2.341667in}{2.109432in}}%
\pgfpathcurveto{\pgfqpoint{2.347491in}{2.103608in}}{\pgfqpoint{2.355391in}{2.100336in}}{\pgfqpoint{2.363627in}{2.100336in}}%
\pgfpathclose%
\pgfusepath{stroke,fill}%
\end{pgfscope}%
\begin{pgfscope}%
\pgfpathrectangle{\pgfqpoint{0.100000in}{0.212622in}}{\pgfqpoint{3.696000in}{3.696000in}}%
\pgfusepath{clip}%
\pgfsetbuttcap%
\pgfsetroundjoin%
\definecolor{currentfill}{rgb}{0.121569,0.466667,0.705882}%
\pgfsetfillcolor{currentfill}%
\pgfsetfillopacity{0.433000}%
\pgfsetlinewidth{1.003750pt}%
\definecolor{currentstroke}{rgb}{0.121569,0.466667,0.705882}%
\pgfsetstrokecolor{currentstroke}%
\pgfsetstrokeopacity{0.433000}%
\pgfsetdash{}{0pt}%
\pgfpathmoveto{\pgfqpoint{1.349052in}{1.849969in}}%
\pgfpathcurveto{\pgfqpoint{1.357289in}{1.849969in}}{\pgfqpoint{1.365189in}{1.853242in}}{\pgfqpoint{1.371013in}{1.859066in}}%
\pgfpathcurveto{\pgfqpoint{1.376837in}{1.864890in}}{\pgfqpoint{1.380109in}{1.872790in}}{\pgfqpoint{1.380109in}{1.881026in}}%
\pgfpathcurveto{\pgfqpoint{1.380109in}{1.889262in}}{\pgfqpoint{1.376837in}{1.897162in}}{\pgfqpoint{1.371013in}{1.902986in}}%
\pgfpathcurveto{\pgfqpoint{1.365189in}{1.908810in}}{\pgfqpoint{1.357289in}{1.912082in}}{\pgfqpoint{1.349052in}{1.912082in}}%
\pgfpathcurveto{\pgfqpoint{1.340816in}{1.912082in}}{\pgfqpoint{1.332916in}{1.908810in}}{\pgfqpoint{1.327092in}{1.902986in}}%
\pgfpathcurveto{\pgfqpoint{1.321268in}{1.897162in}}{\pgfqpoint{1.317996in}{1.889262in}}{\pgfqpoint{1.317996in}{1.881026in}}%
\pgfpathcurveto{\pgfqpoint{1.317996in}{1.872790in}}{\pgfqpoint{1.321268in}{1.864890in}}{\pgfqpoint{1.327092in}{1.859066in}}%
\pgfpathcurveto{\pgfqpoint{1.332916in}{1.853242in}}{\pgfqpoint{1.340816in}{1.849969in}}{\pgfqpoint{1.349052in}{1.849969in}}%
\pgfpathclose%
\pgfusepath{stroke,fill}%
\end{pgfscope}%
\begin{pgfscope}%
\pgfpathrectangle{\pgfqpoint{0.100000in}{0.212622in}}{\pgfqpoint{3.696000in}{3.696000in}}%
\pgfusepath{clip}%
\pgfsetbuttcap%
\pgfsetroundjoin%
\definecolor{currentfill}{rgb}{0.121569,0.466667,0.705882}%
\pgfsetfillcolor{currentfill}%
\pgfsetfillopacity{0.433457}%
\pgfsetlinewidth{1.003750pt}%
\definecolor{currentstroke}{rgb}{0.121569,0.466667,0.705882}%
\pgfsetstrokecolor{currentstroke}%
\pgfsetstrokeopacity{0.433457}%
\pgfsetdash{}{0pt}%
\pgfpathmoveto{\pgfqpoint{2.368661in}{2.102213in}}%
\pgfpathcurveto{\pgfqpoint{2.376897in}{2.102213in}}{\pgfqpoint{2.384797in}{2.105485in}}{\pgfqpoint{2.390621in}{2.111309in}}%
\pgfpathcurveto{\pgfqpoint{2.396445in}{2.117133in}}{\pgfqpoint{2.399717in}{2.125033in}}{\pgfqpoint{2.399717in}{2.133269in}}%
\pgfpathcurveto{\pgfqpoint{2.399717in}{2.141506in}}{\pgfqpoint{2.396445in}{2.149406in}}{\pgfqpoint{2.390621in}{2.155230in}}%
\pgfpathcurveto{\pgfqpoint{2.384797in}{2.161054in}}{\pgfqpoint{2.376897in}{2.164326in}}{\pgfqpoint{2.368661in}{2.164326in}}%
\pgfpathcurveto{\pgfqpoint{2.360425in}{2.164326in}}{\pgfqpoint{2.352524in}{2.161054in}}{\pgfqpoint{2.346701in}{2.155230in}}%
\pgfpathcurveto{\pgfqpoint{2.340877in}{2.149406in}}{\pgfqpoint{2.337604in}{2.141506in}}{\pgfqpoint{2.337604in}{2.133269in}}%
\pgfpathcurveto{\pgfqpoint{2.337604in}{2.125033in}}{\pgfqpoint{2.340877in}{2.117133in}}{\pgfqpoint{2.346701in}{2.111309in}}%
\pgfpathcurveto{\pgfqpoint{2.352524in}{2.105485in}}{\pgfqpoint{2.360425in}{2.102213in}}{\pgfqpoint{2.368661in}{2.102213in}}%
\pgfpathclose%
\pgfusepath{stroke,fill}%
\end{pgfscope}%
\begin{pgfscope}%
\pgfpathrectangle{\pgfqpoint{0.100000in}{0.212622in}}{\pgfqpoint{3.696000in}{3.696000in}}%
\pgfusepath{clip}%
\pgfsetbuttcap%
\pgfsetroundjoin%
\definecolor{currentfill}{rgb}{0.121569,0.466667,0.705882}%
\pgfsetfillcolor{currentfill}%
\pgfsetfillopacity{0.433591}%
\pgfsetlinewidth{1.003750pt}%
\definecolor{currentstroke}{rgb}{0.121569,0.466667,0.705882}%
\pgfsetstrokecolor{currentstroke}%
\pgfsetstrokeopacity{0.433591}%
\pgfsetdash{}{0pt}%
\pgfpathmoveto{\pgfqpoint{2.374303in}{2.096284in}}%
\pgfpathcurveto{\pgfqpoint{2.382539in}{2.096284in}}{\pgfqpoint{2.390439in}{2.099556in}}{\pgfqpoint{2.396263in}{2.105380in}}%
\pgfpathcurveto{\pgfqpoint{2.402087in}{2.111204in}}{\pgfqpoint{2.405360in}{2.119104in}}{\pgfqpoint{2.405360in}{2.127341in}}%
\pgfpathcurveto{\pgfqpoint{2.405360in}{2.135577in}}{\pgfqpoint{2.402087in}{2.143477in}}{\pgfqpoint{2.396263in}{2.149301in}}%
\pgfpathcurveto{\pgfqpoint{2.390439in}{2.155125in}}{\pgfqpoint{2.382539in}{2.158397in}}{\pgfqpoint{2.374303in}{2.158397in}}%
\pgfpathcurveto{\pgfqpoint{2.366067in}{2.158397in}}{\pgfqpoint{2.358167in}{2.155125in}}{\pgfqpoint{2.352343in}{2.149301in}}%
\pgfpathcurveto{\pgfqpoint{2.346519in}{2.143477in}}{\pgfqpoint{2.343247in}{2.135577in}}{\pgfqpoint{2.343247in}{2.127341in}}%
\pgfpathcurveto{\pgfqpoint{2.343247in}{2.119104in}}{\pgfqpoint{2.346519in}{2.111204in}}{\pgfqpoint{2.352343in}{2.105380in}}%
\pgfpathcurveto{\pgfqpoint{2.358167in}{2.099556in}}{\pgfqpoint{2.366067in}{2.096284in}}{\pgfqpoint{2.374303in}{2.096284in}}%
\pgfpathclose%
\pgfusepath{stroke,fill}%
\end{pgfscope}%
\begin{pgfscope}%
\pgfpathrectangle{\pgfqpoint{0.100000in}{0.212622in}}{\pgfqpoint{3.696000in}{3.696000in}}%
\pgfusepath{clip}%
\pgfsetbuttcap%
\pgfsetroundjoin%
\definecolor{currentfill}{rgb}{0.121569,0.466667,0.705882}%
\pgfsetfillcolor{currentfill}%
\pgfsetfillopacity{0.433795}%
\pgfsetlinewidth{1.003750pt}%
\definecolor{currentstroke}{rgb}{0.121569,0.466667,0.705882}%
\pgfsetstrokecolor{currentstroke}%
\pgfsetstrokeopacity{0.433795}%
\pgfsetdash{}{0pt}%
\pgfpathmoveto{\pgfqpoint{1.345827in}{1.848371in}}%
\pgfpathcurveto{\pgfqpoint{1.354064in}{1.848371in}}{\pgfqpoint{1.361964in}{1.851643in}}{\pgfqpoint{1.367788in}{1.857467in}}%
\pgfpathcurveto{\pgfqpoint{1.373612in}{1.863291in}}{\pgfqpoint{1.376884in}{1.871191in}}{\pgfqpoint{1.376884in}{1.879427in}}%
\pgfpathcurveto{\pgfqpoint{1.376884in}{1.887664in}}{\pgfqpoint{1.373612in}{1.895564in}}{\pgfqpoint{1.367788in}{1.901388in}}%
\pgfpathcurveto{\pgfqpoint{1.361964in}{1.907212in}}{\pgfqpoint{1.354064in}{1.910484in}}{\pgfqpoint{1.345827in}{1.910484in}}%
\pgfpathcurveto{\pgfqpoint{1.337591in}{1.910484in}}{\pgfqpoint{1.329691in}{1.907212in}}{\pgfqpoint{1.323867in}{1.901388in}}%
\pgfpathcurveto{\pgfqpoint{1.318043in}{1.895564in}}{\pgfqpoint{1.314771in}{1.887664in}}{\pgfqpoint{1.314771in}{1.879427in}}%
\pgfpathcurveto{\pgfqpoint{1.314771in}{1.871191in}}{\pgfqpoint{1.318043in}{1.863291in}}{\pgfqpoint{1.323867in}{1.857467in}}%
\pgfpathcurveto{\pgfqpoint{1.329691in}{1.851643in}}{\pgfqpoint{1.337591in}{1.848371in}}{\pgfqpoint{1.345827in}{1.848371in}}%
\pgfpathclose%
\pgfusepath{stroke,fill}%
\end{pgfscope}%
\begin{pgfscope}%
\pgfpathrectangle{\pgfqpoint{0.100000in}{0.212622in}}{\pgfqpoint{3.696000in}{3.696000in}}%
\pgfusepath{clip}%
\pgfsetbuttcap%
\pgfsetroundjoin%
\definecolor{currentfill}{rgb}{0.121569,0.466667,0.705882}%
\pgfsetfillcolor{currentfill}%
\pgfsetfillopacity{0.434045}%
\pgfsetlinewidth{1.003750pt}%
\definecolor{currentstroke}{rgb}{0.121569,0.466667,0.705882}%
\pgfsetstrokecolor{currentstroke}%
\pgfsetstrokeopacity{0.434045}%
\pgfsetdash{}{0pt}%
\pgfpathmoveto{\pgfqpoint{2.377665in}{2.096684in}}%
\pgfpathcurveto{\pgfqpoint{2.385902in}{2.096684in}}{\pgfqpoint{2.393802in}{2.099956in}}{\pgfqpoint{2.399626in}{2.105780in}}%
\pgfpathcurveto{\pgfqpoint{2.405450in}{2.111604in}}{\pgfqpoint{2.408722in}{2.119504in}}{\pgfqpoint{2.408722in}{2.127740in}}%
\pgfpathcurveto{\pgfqpoint{2.408722in}{2.135977in}}{\pgfqpoint{2.405450in}{2.143877in}}{\pgfqpoint{2.399626in}{2.149701in}}%
\pgfpathcurveto{\pgfqpoint{2.393802in}{2.155525in}}{\pgfqpoint{2.385902in}{2.158797in}}{\pgfqpoint{2.377665in}{2.158797in}}%
\pgfpathcurveto{\pgfqpoint{2.369429in}{2.158797in}}{\pgfqpoint{2.361529in}{2.155525in}}{\pgfqpoint{2.355705in}{2.149701in}}%
\pgfpathcurveto{\pgfqpoint{2.349881in}{2.143877in}}{\pgfqpoint{2.346609in}{2.135977in}}{\pgfqpoint{2.346609in}{2.127740in}}%
\pgfpathcurveto{\pgfqpoint{2.346609in}{2.119504in}}{\pgfqpoint{2.349881in}{2.111604in}}{\pgfqpoint{2.355705in}{2.105780in}}%
\pgfpathcurveto{\pgfqpoint{2.361529in}{2.099956in}}{\pgfqpoint{2.369429in}{2.096684in}}{\pgfqpoint{2.377665in}{2.096684in}}%
\pgfpathclose%
\pgfusepath{stroke,fill}%
\end{pgfscope}%
\begin{pgfscope}%
\pgfpathrectangle{\pgfqpoint{0.100000in}{0.212622in}}{\pgfqpoint{3.696000in}{3.696000in}}%
\pgfusepath{clip}%
\pgfsetbuttcap%
\pgfsetroundjoin%
\definecolor{currentfill}{rgb}{0.121569,0.466667,0.705882}%
\pgfsetfillcolor{currentfill}%
\pgfsetfillopacity{0.434449}%
\pgfsetlinewidth{1.003750pt}%
\definecolor{currentstroke}{rgb}{0.121569,0.466667,0.705882}%
\pgfsetstrokecolor{currentstroke}%
\pgfsetstrokeopacity{0.434449}%
\pgfsetdash{}{0pt}%
\pgfpathmoveto{\pgfqpoint{2.381163in}{2.094793in}}%
\pgfpathcurveto{\pgfqpoint{2.389399in}{2.094793in}}{\pgfqpoint{2.397299in}{2.098065in}}{\pgfqpoint{2.403123in}{2.103889in}}%
\pgfpathcurveto{\pgfqpoint{2.408947in}{2.109713in}}{\pgfqpoint{2.412219in}{2.117613in}}{\pgfqpoint{2.412219in}{2.125850in}}%
\pgfpathcurveto{\pgfqpoint{2.412219in}{2.134086in}}{\pgfqpoint{2.408947in}{2.141986in}}{\pgfqpoint{2.403123in}{2.147810in}}%
\pgfpathcurveto{\pgfqpoint{2.397299in}{2.153634in}}{\pgfqpoint{2.389399in}{2.156906in}}{\pgfqpoint{2.381163in}{2.156906in}}%
\pgfpathcurveto{\pgfqpoint{2.372927in}{2.156906in}}{\pgfqpoint{2.365027in}{2.153634in}}{\pgfqpoint{2.359203in}{2.147810in}}%
\pgfpathcurveto{\pgfqpoint{2.353379in}{2.141986in}}{\pgfqpoint{2.350106in}{2.134086in}}{\pgfqpoint{2.350106in}{2.125850in}}%
\pgfpathcurveto{\pgfqpoint{2.350106in}{2.117613in}}{\pgfqpoint{2.353379in}{2.109713in}}{\pgfqpoint{2.359203in}{2.103889in}}%
\pgfpathcurveto{\pgfqpoint{2.365027in}{2.098065in}}{\pgfqpoint{2.372927in}{2.094793in}}{\pgfqpoint{2.381163in}{2.094793in}}%
\pgfpathclose%
\pgfusepath{stroke,fill}%
\end{pgfscope}%
\begin{pgfscope}%
\pgfpathrectangle{\pgfqpoint{0.100000in}{0.212622in}}{\pgfqpoint{3.696000in}{3.696000in}}%
\pgfusepath{clip}%
\pgfsetbuttcap%
\pgfsetroundjoin%
\definecolor{currentfill}{rgb}{0.121569,0.466667,0.705882}%
\pgfsetfillcolor{currentfill}%
\pgfsetfillopacity{0.434589}%
\pgfsetlinewidth{1.003750pt}%
\definecolor{currentstroke}{rgb}{0.121569,0.466667,0.705882}%
\pgfsetstrokecolor{currentstroke}%
\pgfsetstrokeopacity{0.434589}%
\pgfsetdash{}{0pt}%
\pgfpathmoveto{\pgfqpoint{2.383132in}{2.093301in}}%
\pgfpathcurveto{\pgfqpoint{2.391369in}{2.093301in}}{\pgfqpoint{2.399269in}{2.096573in}}{\pgfqpoint{2.405093in}{2.102397in}}%
\pgfpathcurveto{\pgfqpoint{2.410916in}{2.108221in}}{\pgfqpoint{2.414189in}{2.116121in}}{\pgfqpoint{2.414189in}{2.124357in}}%
\pgfpathcurveto{\pgfqpoint{2.414189in}{2.132593in}}{\pgfqpoint{2.410916in}{2.140493in}}{\pgfqpoint{2.405093in}{2.146317in}}%
\pgfpathcurveto{\pgfqpoint{2.399269in}{2.152141in}}{\pgfqpoint{2.391369in}{2.155414in}}{\pgfqpoint{2.383132in}{2.155414in}}%
\pgfpathcurveto{\pgfqpoint{2.374896in}{2.155414in}}{\pgfqpoint{2.366996in}{2.152141in}}{\pgfqpoint{2.361172in}{2.146317in}}%
\pgfpathcurveto{\pgfqpoint{2.355348in}{2.140493in}}{\pgfqpoint{2.352076in}{2.132593in}}{\pgfqpoint{2.352076in}{2.124357in}}%
\pgfpathcurveto{\pgfqpoint{2.352076in}{2.116121in}}{\pgfqpoint{2.355348in}{2.108221in}}{\pgfqpoint{2.361172in}{2.102397in}}%
\pgfpathcurveto{\pgfqpoint{2.366996in}{2.096573in}}{\pgfqpoint{2.374896in}{2.093301in}}{\pgfqpoint{2.383132in}{2.093301in}}%
\pgfpathclose%
\pgfusepath{stroke,fill}%
\end{pgfscope}%
\begin{pgfscope}%
\pgfpathrectangle{\pgfqpoint{0.100000in}{0.212622in}}{\pgfqpoint{3.696000in}{3.696000in}}%
\pgfusepath{clip}%
\pgfsetbuttcap%
\pgfsetroundjoin%
\definecolor{currentfill}{rgb}{0.121569,0.466667,0.705882}%
\pgfsetfillcolor{currentfill}%
\pgfsetfillopacity{0.434912}%
\pgfsetlinewidth{1.003750pt}%
\definecolor{currentstroke}{rgb}{0.121569,0.466667,0.705882}%
\pgfsetstrokecolor{currentstroke}%
\pgfsetstrokeopacity{0.434912}%
\pgfsetdash{}{0pt}%
\pgfpathmoveto{\pgfqpoint{1.344796in}{1.848149in}}%
\pgfpathcurveto{\pgfqpoint{1.353032in}{1.848149in}}{\pgfqpoint{1.360932in}{1.851421in}}{\pgfqpoint{1.366756in}{1.857245in}}%
\pgfpathcurveto{\pgfqpoint{1.372580in}{1.863069in}}{\pgfqpoint{1.375853in}{1.870969in}}{\pgfqpoint{1.375853in}{1.879205in}}%
\pgfpathcurveto{\pgfqpoint{1.375853in}{1.887441in}}{\pgfqpoint{1.372580in}{1.895341in}}{\pgfqpoint{1.366756in}{1.901165in}}%
\pgfpathcurveto{\pgfqpoint{1.360932in}{1.906989in}}{\pgfqpoint{1.353032in}{1.910262in}}{\pgfqpoint{1.344796in}{1.910262in}}%
\pgfpathcurveto{\pgfqpoint{1.336560in}{1.910262in}}{\pgfqpoint{1.328660in}{1.906989in}}{\pgfqpoint{1.322836in}{1.901165in}}%
\pgfpathcurveto{\pgfqpoint{1.317012in}{1.895341in}}{\pgfqpoint{1.313740in}{1.887441in}}{\pgfqpoint{1.313740in}{1.879205in}}%
\pgfpathcurveto{\pgfqpoint{1.313740in}{1.870969in}}{\pgfqpoint{1.317012in}{1.863069in}}{\pgfqpoint{1.322836in}{1.857245in}}%
\pgfpathcurveto{\pgfqpoint{1.328660in}{1.851421in}}{\pgfqpoint{1.336560in}{1.848149in}}{\pgfqpoint{1.344796in}{1.848149in}}%
\pgfpathclose%
\pgfusepath{stroke,fill}%
\end{pgfscope}%
\begin{pgfscope}%
\pgfpathrectangle{\pgfqpoint{0.100000in}{0.212622in}}{\pgfqpoint{3.696000in}{3.696000in}}%
\pgfusepath{clip}%
\pgfsetbuttcap%
\pgfsetroundjoin%
\definecolor{currentfill}{rgb}{0.121569,0.466667,0.705882}%
\pgfsetfillcolor{currentfill}%
\pgfsetfillopacity{0.435015}%
\pgfsetlinewidth{1.003750pt}%
\definecolor{currentstroke}{rgb}{0.121569,0.466667,0.705882}%
\pgfsetstrokecolor{currentstroke}%
\pgfsetstrokeopacity{0.435015}%
\pgfsetdash{}{0pt}%
\pgfpathmoveto{\pgfqpoint{2.386514in}{2.093613in}}%
\pgfpathcurveto{\pgfqpoint{2.394750in}{2.093613in}}{\pgfqpoint{2.402650in}{2.096885in}}{\pgfqpoint{2.408474in}{2.102709in}}%
\pgfpathcurveto{\pgfqpoint{2.414298in}{2.108533in}}{\pgfqpoint{2.417570in}{2.116433in}}{\pgfqpoint{2.417570in}{2.124669in}}%
\pgfpathcurveto{\pgfqpoint{2.417570in}{2.132906in}}{\pgfqpoint{2.414298in}{2.140806in}}{\pgfqpoint{2.408474in}{2.146630in}}%
\pgfpathcurveto{\pgfqpoint{2.402650in}{2.152453in}}{\pgfqpoint{2.394750in}{2.155726in}}{\pgfqpoint{2.386514in}{2.155726in}}%
\pgfpathcurveto{\pgfqpoint{2.378278in}{2.155726in}}{\pgfqpoint{2.370378in}{2.152453in}}{\pgfqpoint{2.364554in}{2.146630in}}%
\pgfpathcurveto{\pgfqpoint{2.358730in}{2.140806in}}{\pgfqpoint{2.355457in}{2.132906in}}{\pgfqpoint{2.355457in}{2.124669in}}%
\pgfpathcurveto{\pgfqpoint{2.355457in}{2.116433in}}{\pgfqpoint{2.358730in}{2.108533in}}{\pgfqpoint{2.364554in}{2.102709in}}%
\pgfpathcurveto{\pgfqpoint{2.370378in}{2.096885in}}{\pgfqpoint{2.378278in}{2.093613in}}{\pgfqpoint{2.386514in}{2.093613in}}%
\pgfpathclose%
\pgfusepath{stroke,fill}%
\end{pgfscope}%
\begin{pgfscope}%
\pgfpathrectangle{\pgfqpoint{0.100000in}{0.212622in}}{\pgfqpoint{3.696000in}{3.696000in}}%
\pgfusepath{clip}%
\pgfsetbuttcap%
\pgfsetroundjoin%
\definecolor{currentfill}{rgb}{0.121569,0.466667,0.705882}%
\pgfsetfillcolor{currentfill}%
\pgfsetfillopacity{0.435231}%
\pgfsetlinewidth{1.003750pt}%
\definecolor{currentstroke}{rgb}{0.121569,0.466667,0.705882}%
\pgfsetstrokecolor{currentstroke}%
\pgfsetstrokeopacity{0.435231}%
\pgfsetdash{}{0pt}%
\pgfpathmoveto{\pgfqpoint{2.388201in}{2.093010in}}%
\pgfpathcurveto{\pgfqpoint{2.396437in}{2.093010in}}{\pgfqpoint{2.404337in}{2.096282in}}{\pgfqpoint{2.410161in}{2.102106in}}%
\pgfpathcurveto{\pgfqpoint{2.415985in}{2.107930in}}{\pgfqpoint{2.419257in}{2.115830in}}{\pgfqpoint{2.419257in}{2.124066in}}%
\pgfpathcurveto{\pgfqpoint{2.419257in}{2.132302in}}{\pgfqpoint{2.415985in}{2.140202in}}{\pgfqpoint{2.410161in}{2.146026in}}%
\pgfpathcurveto{\pgfqpoint{2.404337in}{2.151850in}}{\pgfqpoint{2.396437in}{2.155123in}}{\pgfqpoint{2.388201in}{2.155123in}}%
\pgfpathcurveto{\pgfqpoint{2.379965in}{2.155123in}}{\pgfqpoint{2.372065in}{2.151850in}}{\pgfqpoint{2.366241in}{2.146026in}}%
\pgfpathcurveto{\pgfqpoint{2.360417in}{2.140202in}}{\pgfqpoint{2.357144in}{2.132302in}}{\pgfqpoint{2.357144in}{2.124066in}}%
\pgfpathcurveto{\pgfqpoint{2.357144in}{2.115830in}}{\pgfqpoint{2.360417in}{2.107930in}}{\pgfqpoint{2.366241in}{2.102106in}}%
\pgfpathcurveto{\pgfqpoint{2.372065in}{2.096282in}}{\pgfqpoint{2.379965in}{2.093010in}}{\pgfqpoint{2.388201in}{2.093010in}}%
\pgfpathclose%
\pgfusepath{stroke,fill}%
\end{pgfscope}%
\begin{pgfscope}%
\pgfpathrectangle{\pgfqpoint{0.100000in}{0.212622in}}{\pgfqpoint{3.696000in}{3.696000in}}%
\pgfusepath{clip}%
\pgfsetbuttcap%
\pgfsetroundjoin%
\definecolor{currentfill}{rgb}{0.121569,0.466667,0.705882}%
\pgfsetfillcolor{currentfill}%
\pgfsetfillopacity{0.435374}%
\pgfsetlinewidth{1.003750pt}%
\definecolor{currentstroke}{rgb}{0.121569,0.466667,0.705882}%
\pgfsetstrokecolor{currentstroke}%
\pgfsetstrokeopacity{0.435374}%
\pgfsetdash{}{0pt}%
\pgfpathmoveto{\pgfqpoint{2.389138in}{2.092878in}}%
\pgfpathcurveto{\pgfqpoint{2.397374in}{2.092878in}}{\pgfqpoint{2.405274in}{2.096151in}}{\pgfqpoint{2.411098in}{2.101975in}}%
\pgfpathcurveto{\pgfqpoint{2.416922in}{2.107799in}}{\pgfqpoint{2.420194in}{2.115699in}}{\pgfqpoint{2.420194in}{2.123935in}}%
\pgfpathcurveto{\pgfqpoint{2.420194in}{2.132171in}}{\pgfqpoint{2.416922in}{2.140071in}}{\pgfqpoint{2.411098in}{2.145895in}}%
\pgfpathcurveto{\pgfqpoint{2.405274in}{2.151719in}}{\pgfqpoint{2.397374in}{2.154991in}}{\pgfqpoint{2.389138in}{2.154991in}}%
\pgfpathcurveto{\pgfqpoint{2.380901in}{2.154991in}}{\pgfqpoint{2.373001in}{2.151719in}}{\pgfqpoint{2.367177in}{2.145895in}}%
\pgfpathcurveto{\pgfqpoint{2.361353in}{2.140071in}}{\pgfqpoint{2.358081in}{2.132171in}}{\pgfqpoint{2.358081in}{2.123935in}}%
\pgfpathcurveto{\pgfqpoint{2.358081in}{2.115699in}}{\pgfqpoint{2.361353in}{2.107799in}}{\pgfqpoint{2.367177in}{2.101975in}}%
\pgfpathcurveto{\pgfqpoint{2.373001in}{2.096151in}}{\pgfqpoint{2.380901in}{2.092878in}}{\pgfqpoint{2.389138in}{2.092878in}}%
\pgfpathclose%
\pgfusepath{stroke,fill}%
\end{pgfscope}%
\begin{pgfscope}%
\pgfpathrectangle{\pgfqpoint{0.100000in}{0.212622in}}{\pgfqpoint{3.696000in}{3.696000in}}%
\pgfusepath{clip}%
\pgfsetbuttcap%
\pgfsetroundjoin%
\definecolor{currentfill}{rgb}{0.121569,0.466667,0.705882}%
\pgfsetfillcolor{currentfill}%
\pgfsetfillopacity{0.435553}%
\pgfsetlinewidth{1.003750pt}%
\definecolor{currentstroke}{rgb}{0.121569,0.466667,0.705882}%
\pgfsetstrokecolor{currentstroke}%
\pgfsetstrokeopacity{0.435553}%
\pgfsetdash{}{0pt}%
\pgfpathmoveto{\pgfqpoint{2.391236in}{2.091678in}}%
\pgfpathcurveto{\pgfqpoint{2.399472in}{2.091678in}}{\pgfqpoint{2.407372in}{2.094950in}}{\pgfqpoint{2.413196in}{2.100774in}}%
\pgfpathcurveto{\pgfqpoint{2.419020in}{2.106598in}}{\pgfqpoint{2.422292in}{2.114498in}}{\pgfqpoint{2.422292in}{2.122734in}}%
\pgfpathcurveto{\pgfqpoint{2.422292in}{2.130971in}}{\pgfqpoint{2.419020in}{2.138871in}}{\pgfqpoint{2.413196in}{2.144695in}}%
\pgfpathcurveto{\pgfqpoint{2.407372in}{2.150519in}}{\pgfqpoint{2.399472in}{2.153791in}}{\pgfqpoint{2.391236in}{2.153791in}}%
\pgfpathcurveto{\pgfqpoint{2.382999in}{2.153791in}}{\pgfqpoint{2.375099in}{2.150519in}}{\pgfqpoint{2.369275in}{2.144695in}}%
\pgfpathcurveto{\pgfqpoint{2.363452in}{2.138871in}}{\pgfqpoint{2.360179in}{2.130971in}}{\pgfqpoint{2.360179in}{2.122734in}}%
\pgfpathcurveto{\pgfqpoint{2.360179in}{2.114498in}}{\pgfqpoint{2.363452in}{2.106598in}}{\pgfqpoint{2.369275in}{2.100774in}}%
\pgfpathcurveto{\pgfqpoint{2.375099in}{2.094950in}}{\pgfqpoint{2.382999in}{2.091678in}}{\pgfqpoint{2.391236in}{2.091678in}}%
\pgfpathclose%
\pgfusepath{stroke,fill}%
\end{pgfscope}%
\begin{pgfscope}%
\pgfpathrectangle{\pgfqpoint{0.100000in}{0.212622in}}{\pgfqpoint{3.696000in}{3.696000in}}%
\pgfusepath{clip}%
\pgfsetbuttcap%
\pgfsetroundjoin%
\definecolor{currentfill}{rgb}{0.121569,0.466667,0.705882}%
\pgfsetfillcolor{currentfill}%
\pgfsetfillopacity{0.435968}%
\pgfsetlinewidth{1.003750pt}%
\definecolor{currentstroke}{rgb}{0.121569,0.466667,0.705882}%
\pgfsetstrokecolor{currentstroke}%
\pgfsetstrokeopacity{0.435968}%
\pgfsetdash{}{0pt}%
\pgfpathmoveto{\pgfqpoint{2.393894in}{2.092493in}}%
\pgfpathcurveto{\pgfqpoint{2.402130in}{2.092493in}}{\pgfqpoint{2.410030in}{2.095765in}}{\pgfqpoint{2.415854in}{2.101589in}}%
\pgfpathcurveto{\pgfqpoint{2.421678in}{2.107413in}}{\pgfqpoint{2.424951in}{2.115313in}}{\pgfqpoint{2.424951in}{2.123549in}}%
\pgfpathcurveto{\pgfqpoint{2.424951in}{2.131785in}}{\pgfqpoint{2.421678in}{2.139685in}}{\pgfqpoint{2.415854in}{2.145509in}}%
\pgfpathcurveto{\pgfqpoint{2.410030in}{2.151333in}}{\pgfqpoint{2.402130in}{2.154606in}}{\pgfqpoint{2.393894in}{2.154606in}}%
\pgfpathcurveto{\pgfqpoint{2.385658in}{2.154606in}}{\pgfqpoint{2.377758in}{2.151333in}}{\pgfqpoint{2.371934in}{2.145509in}}%
\pgfpathcurveto{\pgfqpoint{2.366110in}{2.139685in}}{\pgfqpoint{2.362838in}{2.131785in}}{\pgfqpoint{2.362838in}{2.123549in}}%
\pgfpathcurveto{\pgfqpoint{2.362838in}{2.115313in}}{\pgfqpoint{2.366110in}{2.107413in}}{\pgfqpoint{2.371934in}{2.101589in}}%
\pgfpathcurveto{\pgfqpoint{2.377758in}{2.095765in}}{\pgfqpoint{2.385658in}{2.092493in}}{\pgfqpoint{2.393894in}{2.092493in}}%
\pgfpathclose%
\pgfusepath{stroke,fill}%
\end{pgfscope}%
\begin{pgfscope}%
\pgfpathrectangle{\pgfqpoint{0.100000in}{0.212622in}}{\pgfqpoint{3.696000in}{3.696000in}}%
\pgfusepath{clip}%
\pgfsetbuttcap%
\pgfsetroundjoin%
\definecolor{currentfill}{rgb}{0.121569,0.466667,0.705882}%
\pgfsetfillcolor{currentfill}%
\pgfsetfillopacity{0.435988}%
\pgfsetlinewidth{1.003750pt}%
\definecolor{currentstroke}{rgb}{0.121569,0.466667,0.705882}%
\pgfsetstrokecolor{currentstroke}%
\pgfsetstrokeopacity{0.435988}%
\pgfsetdash{}{0pt}%
\pgfpathmoveto{\pgfqpoint{1.342416in}{1.841058in}}%
\pgfpathcurveto{\pgfqpoint{1.350653in}{1.841058in}}{\pgfqpoint{1.358553in}{1.844330in}}{\pgfqpoint{1.364377in}{1.850154in}}%
\pgfpathcurveto{\pgfqpoint{1.370201in}{1.855978in}}{\pgfqpoint{1.373473in}{1.863878in}}{\pgfqpoint{1.373473in}{1.872115in}}%
\pgfpathcurveto{\pgfqpoint{1.373473in}{1.880351in}}{\pgfqpoint{1.370201in}{1.888251in}}{\pgfqpoint{1.364377in}{1.894075in}}%
\pgfpathcurveto{\pgfqpoint{1.358553in}{1.899899in}}{\pgfqpoint{1.350653in}{1.903171in}}{\pgfqpoint{1.342416in}{1.903171in}}%
\pgfpathcurveto{\pgfqpoint{1.334180in}{1.903171in}}{\pgfqpoint{1.326280in}{1.899899in}}{\pgfqpoint{1.320456in}{1.894075in}}%
\pgfpathcurveto{\pgfqpoint{1.314632in}{1.888251in}}{\pgfqpoint{1.311360in}{1.880351in}}{\pgfqpoint{1.311360in}{1.872115in}}%
\pgfpathcurveto{\pgfqpoint{1.311360in}{1.863878in}}{\pgfqpoint{1.314632in}{1.855978in}}{\pgfqpoint{1.320456in}{1.850154in}}%
\pgfpathcurveto{\pgfqpoint{1.326280in}{1.844330in}}{\pgfqpoint{1.334180in}{1.841058in}}{\pgfqpoint{1.342416in}{1.841058in}}%
\pgfpathclose%
\pgfusepath{stroke,fill}%
\end{pgfscope}%
\begin{pgfscope}%
\pgfpathrectangle{\pgfqpoint{0.100000in}{0.212622in}}{\pgfqpoint{3.696000in}{3.696000in}}%
\pgfusepath{clip}%
\pgfsetbuttcap%
\pgfsetroundjoin%
\definecolor{currentfill}{rgb}{0.121569,0.466667,0.705882}%
\pgfsetfillcolor{currentfill}%
\pgfsetfillopacity{0.436071}%
\pgfsetlinewidth{1.003750pt}%
\definecolor{currentstroke}{rgb}{0.121569,0.466667,0.705882}%
\pgfsetstrokecolor{currentstroke}%
\pgfsetstrokeopacity{0.436071}%
\pgfsetdash{}{0pt}%
\pgfpathmoveto{\pgfqpoint{2.395262in}{2.091686in}}%
\pgfpathcurveto{\pgfqpoint{2.403498in}{2.091686in}}{\pgfqpoint{2.411398in}{2.094959in}}{\pgfqpoint{2.417222in}{2.100782in}}%
\pgfpathcurveto{\pgfqpoint{2.423046in}{2.106606in}}{\pgfqpoint{2.426318in}{2.114506in}}{\pgfqpoint{2.426318in}{2.122743in}}%
\pgfpathcurveto{\pgfqpoint{2.426318in}{2.130979in}}{\pgfqpoint{2.423046in}{2.138879in}}{\pgfqpoint{2.417222in}{2.144703in}}%
\pgfpathcurveto{\pgfqpoint{2.411398in}{2.150527in}}{\pgfqpoint{2.403498in}{2.153799in}}{\pgfqpoint{2.395262in}{2.153799in}}%
\pgfpathcurveto{\pgfqpoint{2.387025in}{2.153799in}}{\pgfqpoint{2.379125in}{2.150527in}}{\pgfqpoint{2.373301in}{2.144703in}}%
\pgfpathcurveto{\pgfqpoint{2.367477in}{2.138879in}}{\pgfqpoint{2.364205in}{2.130979in}}{\pgfqpoint{2.364205in}{2.122743in}}%
\pgfpathcurveto{\pgfqpoint{2.364205in}{2.114506in}}{\pgfqpoint{2.367477in}{2.106606in}}{\pgfqpoint{2.373301in}{2.100782in}}%
\pgfpathcurveto{\pgfqpoint{2.379125in}{2.094959in}}{\pgfqpoint{2.387025in}{2.091686in}}{\pgfqpoint{2.395262in}{2.091686in}}%
\pgfpathclose%
\pgfusepath{stroke,fill}%
\end{pgfscope}%
\begin{pgfscope}%
\pgfpathrectangle{\pgfqpoint{0.100000in}{0.212622in}}{\pgfqpoint{3.696000in}{3.696000in}}%
\pgfusepath{clip}%
\pgfsetbuttcap%
\pgfsetroundjoin%
\definecolor{currentfill}{rgb}{0.121569,0.466667,0.705882}%
\pgfsetfillcolor{currentfill}%
\pgfsetfillopacity{0.436481}%
\pgfsetlinewidth{1.003750pt}%
\definecolor{currentstroke}{rgb}{0.121569,0.466667,0.705882}%
\pgfsetstrokecolor{currentstroke}%
\pgfsetstrokeopacity{0.436481}%
\pgfsetdash{}{0pt}%
\pgfpathmoveto{\pgfqpoint{2.397896in}{2.091845in}}%
\pgfpathcurveto{\pgfqpoint{2.406132in}{2.091845in}}{\pgfqpoint{2.414032in}{2.095118in}}{\pgfqpoint{2.419856in}{2.100941in}}%
\pgfpathcurveto{\pgfqpoint{2.425680in}{2.106765in}}{\pgfqpoint{2.428952in}{2.114665in}}{\pgfqpoint{2.428952in}{2.122902in}}%
\pgfpathcurveto{\pgfqpoint{2.428952in}{2.131138in}}{\pgfqpoint{2.425680in}{2.139038in}}{\pgfqpoint{2.419856in}{2.144862in}}%
\pgfpathcurveto{\pgfqpoint{2.414032in}{2.150686in}}{\pgfqpoint{2.406132in}{2.153958in}}{\pgfqpoint{2.397896in}{2.153958in}}%
\pgfpathcurveto{\pgfqpoint{2.389660in}{2.153958in}}{\pgfqpoint{2.381759in}{2.150686in}}{\pgfqpoint{2.375936in}{2.144862in}}%
\pgfpathcurveto{\pgfqpoint{2.370112in}{2.139038in}}{\pgfqpoint{2.366839in}{2.131138in}}{\pgfqpoint{2.366839in}{2.122902in}}%
\pgfpathcurveto{\pgfqpoint{2.366839in}{2.114665in}}{\pgfqpoint{2.370112in}{2.106765in}}{\pgfqpoint{2.375936in}{2.100941in}}%
\pgfpathcurveto{\pgfqpoint{2.381759in}{2.095118in}}{\pgfqpoint{2.389660in}{2.091845in}}{\pgfqpoint{2.397896in}{2.091845in}}%
\pgfpathclose%
\pgfusepath{stroke,fill}%
\end{pgfscope}%
\begin{pgfscope}%
\pgfpathrectangle{\pgfqpoint{0.100000in}{0.212622in}}{\pgfqpoint{3.696000in}{3.696000in}}%
\pgfusepath{clip}%
\pgfsetbuttcap%
\pgfsetroundjoin%
\definecolor{currentfill}{rgb}{0.121569,0.466667,0.705882}%
\pgfsetfillcolor{currentfill}%
\pgfsetfillopacity{0.436598}%
\pgfsetlinewidth{1.003750pt}%
\definecolor{currentstroke}{rgb}{0.121569,0.466667,0.705882}%
\pgfsetstrokecolor{currentstroke}%
\pgfsetstrokeopacity{0.436598}%
\pgfsetdash{}{0pt}%
\pgfpathmoveto{\pgfqpoint{2.399225in}{2.090792in}}%
\pgfpathcurveto{\pgfqpoint{2.407462in}{2.090792in}}{\pgfqpoint{2.415362in}{2.094064in}}{\pgfqpoint{2.421186in}{2.099888in}}%
\pgfpathcurveto{\pgfqpoint{2.427010in}{2.105712in}}{\pgfqpoint{2.430282in}{2.113612in}}{\pgfqpoint{2.430282in}{2.121848in}}%
\pgfpathcurveto{\pgfqpoint{2.430282in}{2.130085in}}{\pgfqpoint{2.427010in}{2.137985in}}{\pgfqpoint{2.421186in}{2.143809in}}%
\pgfpathcurveto{\pgfqpoint{2.415362in}{2.149633in}}{\pgfqpoint{2.407462in}{2.152905in}}{\pgfqpoint{2.399225in}{2.152905in}}%
\pgfpathcurveto{\pgfqpoint{2.390989in}{2.152905in}}{\pgfqpoint{2.383089in}{2.149633in}}{\pgfqpoint{2.377265in}{2.143809in}}%
\pgfpathcurveto{\pgfqpoint{2.371441in}{2.137985in}}{\pgfqpoint{2.368169in}{2.130085in}}{\pgfqpoint{2.368169in}{2.121848in}}%
\pgfpathcurveto{\pgfqpoint{2.368169in}{2.113612in}}{\pgfqpoint{2.371441in}{2.105712in}}{\pgfqpoint{2.377265in}{2.099888in}}%
\pgfpathcurveto{\pgfqpoint{2.383089in}{2.094064in}}{\pgfqpoint{2.390989in}{2.090792in}}{\pgfqpoint{2.399225in}{2.090792in}}%
\pgfpathclose%
\pgfusepath{stroke,fill}%
\end{pgfscope}%
\begin{pgfscope}%
\pgfpathrectangle{\pgfqpoint{0.100000in}{0.212622in}}{\pgfqpoint{3.696000in}{3.696000in}}%
\pgfusepath{clip}%
\pgfsetbuttcap%
\pgfsetroundjoin%
\definecolor{currentfill}{rgb}{0.121569,0.466667,0.705882}%
\pgfsetfillcolor{currentfill}%
\pgfsetfillopacity{0.436755}%
\pgfsetlinewidth{1.003750pt}%
\definecolor{currentstroke}{rgb}{0.121569,0.466667,0.705882}%
\pgfsetstrokecolor{currentstroke}%
\pgfsetstrokeopacity{0.436755}%
\pgfsetdash{}{0pt}%
\pgfpathmoveto{\pgfqpoint{2.399997in}{2.090980in}}%
\pgfpathcurveto{\pgfqpoint{2.408233in}{2.090980in}}{\pgfqpoint{2.416133in}{2.094252in}}{\pgfqpoint{2.421957in}{2.100076in}}%
\pgfpathcurveto{\pgfqpoint{2.427781in}{2.105900in}}{\pgfqpoint{2.431053in}{2.113800in}}{\pgfqpoint{2.431053in}{2.122036in}}%
\pgfpathcurveto{\pgfqpoint{2.431053in}{2.130273in}}{\pgfqpoint{2.427781in}{2.138173in}}{\pgfqpoint{2.421957in}{2.143997in}}%
\pgfpathcurveto{\pgfqpoint{2.416133in}{2.149820in}}{\pgfqpoint{2.408233in}{2.153093in}}{\pgfqpoint{2.399997in}{2.153093in}}%
\pgfpathcurveto{\pgfqpoint{2.391760in}{2.153093in}}{\pgfqpoint{2.383860in}{2.149820in}}{\pgfqpoint{2.378036in}{2.143997in}}%
\pgfpathcurveto{\pgfqpoint{2.372212in}{2.138173in}}{\pgfqpoint{2.368940in}{2.130273in}}{\pgfqpoint{2.368940in}{2.122036in}}%
\pgfpathcurveto{\pgfqpoint{2.368940in}{2.113800in}}{\pgfqpoint{2.372212in}{2.105900in}}{\pgfqpoint{2.378036in}{2.100076in}}%
\pgfpathcurveto{\pgfqpoint{2.383860in}{2.094252in}}{\pgfqpoint{2.391760in}{2.090980in}}{\pgfqpoint{2.399997in}{2.090980in}}%
\pgfpathclose%
\pgfusepath{stroke,fill}%
\end{pgfscope}%
\begin{pgfscope}%
\pgfpathrectangle{\pgfqpoint{0.100000in}{0.212622in}}{\pgfqpoint{3.696000in}{3.696000in}}%
\pgfusepath{clip}%
\pgfsetbuttcap%
\pgfsetroundjoin%
\definecolor{currentfill}{rgb}{0.121569,0.466667,0.705882}%
\pgfsetfillcolor{currentfill}%
\pgfsetfillopacity{0.437015}%
\pgfsetlinewidth{1.003750pt}%
\definecolor{currentstroke}{rgb}{0.121569,0.466667,0.705882}%
\pgfsetstrokecolor{currentstroke}%
\pgfsetstrokeopacity{0.437015}%
\pgfsetdash{}{0pt}%
\pgfpathmoveto{\pgfqpoint{2.401682in}{2.090160in}}%
\pgfpathcurveto{\pgfqpoint{2.409918in}{2.090160in}}{\pgfqpoint{2.417818in}{2.093433in}}{\pgfqpoint{2.423642in}{2.099257in}}%
\pgfpathcurveto{\pgfqpoint{2.429466in}{2.105081in}}{\pgfqpoint{2.432738in}{2.112981in}}{\pgfqpoint{2.432738in}{2.121217in}}%
\pgfpathcurveto{\pgfqpoint{2.432738in}{2.129453in}}{\pgfqpoint{2.429466in}{2.137353in}}{\pgfqpoint{2.423642in}{2.143177in}}%
\pgfpathcurveto{\pgfqpoint{2.417818in}{2.149001in}}{\pgfqpoint{2.409918in}{2.152273in}}{\pgfqpoint{2.401682in}{2.152273in}}%
\pgfpathcurveto{\pgfqpoint{2.393446in}{2.152273in}}{\pgfqpoint{2.385546in}{2.149001in}}{\pgfqpoint{2.379722in}{2.143177in}}%
\pgfpathcurveto{\pgfqpoint{2.373898in}{2.137353in}}{\pgfqpoint{2.370625in}{2.129453in}}{\pgfqpoint{2.370625in}{2.121217in}}%
\pgfpathcurveto{\pgfqpoint{2.370625in}{2.112981in}}{\pgfqpoint{2.373898in}{2.105081in}}{\pgfqpoint{2.379722in}{2.099257in}}%
\pgfpathcurveto{\pgfqpoint{2.385546in}{2.093433in}}{\pgfqpoint{2.393446in}{2.090160in}}{\pgfqpoint{2.401682in}{2.090160in}}%
\pgfpathclose%
\pgfusepath{stroke,fill}%
\end{pgfscope}%
\begin{pgfscope}%
\pgfpathrectangle{\pgfqpoint{0.100000in}{0.212622in}}{\pgfqpoint{3.696000in}{3.696000in}}%
\pgfusepath{clip}%
\pgfsetbuttcap%
\pgfsetroundjoin%
\definecolor{currentfill}{rgb}{0.121569,0.466667,0.705882}%
\pgfsetfillcolor{currentfill}%
\pgfsetfillopacity{0.437160}%
\pgfsetlinewidth{1.003750pt}%
\definecolor{currentstroke}{rgb}{0.121569,0.466667,0.705882}%
\pgfsetstrokecolor{currentstroke}%
\pgfsetstrokeopacity{0.437160}%
\pgfsetdash{}{0pt}%
\pgfpathmoveto{\pgfqpoint{2.402734in}{2.090097in}}%
\pgfpathcurveto{\pgfqpoint{2.410970in}{2.090097in}}{\pgfqpoint{2.418870in}{2.093369in}}{\pgfqpoint{2.424694in}{2.099193in}}%
\pgfpathcurveto{\pgfqpoint{2.430518in}{2.105017in}}{\pgfqpoint{2.433790in}{2.112917in}}{\pgfqpoint{2.433790in}{2.121153in}}%
\pgfpathcurveto{\pgfqpoint{2.433790in}{2.129390in}}{\pgfqpoint{2.430518in}{2.137290in}}{\pgfqpoint{2.424694in}{2.143114in}}%
\pgfpathcurveto{\pgfqpoint{2.418870in}{2.148938in}}{\pgfqpoint{2.410970in}{2.152210in}}{\pgfqpoint{2.402734in}{2.152210in}}%
\pgfpathcurveto{\pgfqpoint{2.394497in}{2.152210in}}{\pgfqpoint{2.386597in}{2.148938in}}{\pgfqpoint{2.380773in}{2.143114in}}%
\pgfpathcurveto{\pgfqpoint{2.374949in}{2.137290in}}{\pgfqpoint{2.371677in}{2.129390in}}{\pgfqpoint{2.371677in}{2.121153in}}%
\pgfpathcurveto{\pgfqpoint{2.371677in}{2.112917in}}{\pgfqpoint{2.374949in}{2.105017in}}{\pgfqpoint{2.380773in}{2.099193in}}%
\pgfpathcurveto{\pgfqpoint{2.386597in}{2.093369in}}{\pgfqpoint{2.394497in}{2.090097in}}{\pgfqpoint{2.402734in}{2.090097in}}%
\pgfpathclose%
\pgfusepath{stroke,fill}%
\end{pgfscope}%
\begin{pgfscope}%
\pgfpathrectangle{\pgfqpoint{0.100000in}{0.212622in}}{\pgfqpoint{3.696000in}{3.696000in}}%
\pgfusepath{clip}%
\pgfsetbuttcap%
\pgfsetroundjoin%
\definecolor{currentfill}{rgb}{0.121569,0.466667,0.705882}%
\pgfsetfillcolor{currentfill}%
\pgfsetfillopacity{0.437175}%
\pgfsetlinewidth{1.003750pt}%
\definecolor{currentstroke}{rgb}{0.121569,0.466667,0.705882}%
\pgfsetstrokecolor{currentstroke}%
\pgfsetstrokeopacity{0.437175}%
\pgfsetdash{}{0pt}%
\pgfpathmoveto{\pgfqpoint{1.338324in}{1.838327in}}%
\pgfpathcurveto{\pgfqpoint{1.346560in}{1.838327in}}{\pgfqpoint{1.354460in}{1.841599in}}{\pgfqpoint{1.360284in}{1.847423in}}%
\pgfpathcurveto{\pgfqpoint{1.366108in}{1.853247in}}{\pgfqpoint{1.369380in}{1.861147in}}{\pgfqpoint{1.369380in}{1.869383in}}%
\pgfpathcurveto{\pgfqpoint{1.369380in}{1.877620in}}{\pgfqpoint{1.366108in}{1.885520in}}{\pgfqpoint{1.360284in}{1.891343in}}%
\pgfpathcurveto{\pgfqpoint{1.354460in}{1.897167in}}{\pgfqpoint{1.346560in}{1.900440in}}{\pgfqpoint{1.338324in}{1.900440in}}%
\pgfpathcurveto{\pgfqpoint{1.330088in}{1.900440in}}{\pgfqpoint{1.322188in}{1.897167in}}{\pgfqpoint{1.316364in}{1.891343in}}%
\pgfpathcurveto{\pgfqpoint{1.310540in}{1.885520in}}{\pgfqpoint{1.307267in}{1.877620in}}{\pgfqpoint{1.307267in}{1.869383in}}%
\pgfpathcurveto{\pgfqpoint{1.307267in}{1.861147in}}{\pgfqpoint{1.310540in}{1.853247in}}{\pgfqpoint{1.316364in}{1.847423in}}%
\pgfpathcurveto{\pgfqpoint{1.322188in}{1.841599in}}{\pgfqpoint{1.330088in}{1.838327in}}{\pgfqpoint{1.338324in}{1.838327in}}%
\pgfpathclose%
\pgfusepath{stroke,fill}%
\end{pgfscope}%
\begin{pgfscope}%
\pgfpathrectangle{\pgfqpoint{0.100000in}{0.212622in}}{\pgfqpoint{3.696000in}{3.696000in}}%
\pgfusepath{clip}%
\pgfsetbuttcap%
\pgfsetroundjoin%
\definecolor{currentfill}{rgb}{0.121569,0.466667,0.705882}%
\pgfsetfillcolor{currentfill}%
\pgfsetfillopacity{0.437324}%
\pgfsetlinewidth{1.003750pt}%
\definecolor{currentstroke}{rgb}{0.121569,0.466667,0.705882}%
\pgfsetstrokecolor{currentstroke}%
\pgfsetstrokeopacity{0.437324}%
\pgfsetdash{}{0pt}%
\pgfpathmoveto{\pgfqpoint{2.404083in}{2.089649in}}%
\pgfpathcurveto{\pgfqpoint{2.412319in}{2.089649in}}{\pgfqpoint{2.420219in}{2.092921in}}{\pgfqpoint{2.426043in}{2.098745in}}%
\pgfpathcurveto{\pgfqpoint{2.431867in}{2.104569in}}{\pgfqpoint{2.435140in}{2.112469in}}{\pgfqpoint{2.435140in}{2.120706in}}%
\pgfpathcurveto{\pgfqpoint{2.435140in}{2.128942in}}{\pgfqpoint{2.431867in}{2.136842in}}{\pgfqpoint{2.426043in}{2.142666in}}%
\pgfpathcurveto{\pgfqpoint{2.420219in}{2.148490in}}{\pgfqpoint{2.412319in}{2.151762in}}{\pgfqpoint{2.404083in}{2.151762in}}%
\pgfpathcurveto{\pgfqpoint{2.395847in}{2.151762in}}{\pgfqpoint{2.387947in}{2.148490in}}{\pgfqpoint{2.382123in}{2.142666in}}%
\pgfpathcurveto{\pgfqpoint{2.376299in}{2.136842in}}{\pgfqpoint{2.373027in}{2.128942in}}{\pgfqpoint{2.373027in}{2.120706in}}%
\pgfpathcurveto{\pgfqpoint{2.373027in}{2.112469in}}{\pgfqpoint{2.376299in}{2.104569in}}{\pgfqpoint{2.382123in}{2.098745in}}%
\pgfpathcurveto{\pgfqpoint{2.387947in}{2.092921in}}{\pgfqpoint{2.395847in}{2.089649in}}{\pgfqpoint{2.404083in}{2.089649in}}%
\pgfpathclose%
\pgfusepath{stroke,fill}%
\end{pgfscope}%
\begin{pgfscope}%
\pgfpathrectangle{\pgfqpoint{0.100000in}{0.212622in}}{\pgfqpoint{3.696000in}{3.696000in}}%
\pgfusepath{clip}%
\pgfsetbuttcap%
\pgfsetroundjoin%
\definecolor{currentfill}{rgb}{0.121569,0.466667,0.705882}%
\pgfsetfillcolor{currentfill}%
\pgfsetfillopacity{0.437652}%
\pgfsetlinewidth{1.003750pt}%
\definecolor{currentstroke}{rgb}{0.121569,0.466667,0.705882}%
\pgfsetstrokecolor{currentstroke}%
\pgfsetstrokeopacity{0.437652}%
\pgfsetdash{}{0pt}%
\pgfpathmoveto{\pgfqpoint{2.406366in}{2.089707in}}%
\pgfpathcurveto{\pgfqpoint{2.414603in}{2.089707in}}{\pgfqpoint{2.422503in}{2.092980in}}{\pgfqpoint{2.428327in}{2.098804in}}%
\pgfpathcurveto{\pgfqpoint{2.434150in}{2.104628in}}{\pgfqpoint{2.437423in}{2.112528in}}{\pgfqpoint{2.437423in}{2.120764in}}%
\pgfpathcurveto{\pgfqpoint{2.437423in}{2.129000in}}{\pgfqpoint{2.434150in}{2.136900in}}{\pgfqpoint{2.428327in}{2.142724in}}%
\pgfpathcurveto{\pgfqpoint{2.422503in}{2.148548in}}{\pgfqpoint{2.414603in}{2.151820in}}{\pgfqpoint{2.406366in}{2.151820in}}%
\pgfpathcurveto{\pgfqpoint{2.398130in}{2.151820in}}{\pgfqpoint{2.390230in}{2.148548in}}{\pgfqpoint{2.384406in}{2.142724in}}%
\pgfpathcurveto{\pgfqpoint{2.378582in}{2.136900in}}{\pgfqpoint{2.375310in}{2.129000in}}{\pgfqpoint{2.375310in}{2.120764in}}%
\pgfpathcurveto{\pgfqpoint{2.375310in}{2.112528in}}{\pgfqpoint{2.378582in}{2.104628in}}{\pgfqpoint{2.384406in}{2.098804in}}%
\pgfpathcurveto{\pgfqpoint{2.390230in}{2.092980in}}{\pgfqpoint{2.398130in}{2.089707in}}{\pgfqpoint{2.406366in}{2.089707in}}%
\pgfpathclose%
\pgfusepath{stroke,fill}%
\end{pgfscope}%
\begin{pgfscope}%
\pgfpathrectangle{\pgfqpoint{0.100000in}{0.212622in}}{\pgfqpoint{3.696000in}{3.696000in}}%
\pgfusepath{clip}%
\pgfsetbuttcap%
\pgfsetroundjoin%
\definecolor{currentfill}{rgb}{0.121569,0.466667,0.705882}%
\pgfsetfillcolor{currentfill}%
\pgfsetfillopacity{0.437723}%
\pgfsetlinewidth{1.003750pt}%
\definecolor{currentstroke}{rgb}{0.121569,0.466667,0.705882}%
\pgfsetstrokecolor{currentstroke}%
\pgfsetstrokeopacity{0.437723}%
\pgfsetdash{}{0pt}%
\pgfpathmoveto{\pgfqpoint{2.407511in}{2.088599in}}%
\pgfpathcurveto{\pgfqpoint{2.415748in}{2.088599in}}{\pgfqpoint{2.423648in}{2.091871in}}{\pgfqpoint{2.429472in}{2.097695in}}%
\pgfpathcurveto{\pgfqpoint{2.435296in}{2.103519in}}{\pgfqpoint{2.438568in}{2.111419in}}{\pgfqpoint{2.438568in}{2.119655in}}%
\pgfpathcurveto{\pgfqpoint{2.438568in}{2.127892in}}{\pgfqpoint{2.435296in}{2.135792in}}{\pgfqpoint{2.429472in}{2.141616in}}%
\pgfpathcurveto{\pgfqpoint{2.423648in}{2.147439in}}{\pgfqpoint{2.415748in}{2.150712in}}{\pgfqpoint{2.407511in}{2.150712in}}%
\pgfpathcurveto{\pgfqpoint{2.399275in}{2.150712in}}{\pgfqpoint{2.391375in}{2.147439in}}{\pgfqpoint{2.385551in}{2.141616in}}%
\pgfpathcurveto{\pgfqpoint{2.379727in}{2.135792in}}{\pgfqpoint{2.376455in}{2.127892in}}{\pgfqpoint{2.376455in}{2.119655in}}%
\pgfpathcurveto{\pgfqpoint{2.376455in}{2.111419in}}{\pgfqpoint{2.379727in}{2.103519in}}{\pgfqpoint{2.385551in}{2.097695in}}%
\pgfpathcurveto{\pgfqpoint{2.391375in}{2.091871in}}{\pgfqpoint{2.399275in}{2.088599in}}{\pgfqpoint{2.407511in}{2.088599in}}%
\pgfpathclose%
\pgfusepath{stroke,fill}%
\end{pgfscope}%
\begin{pgfscope}%
\pgfpathrectangle{\pgfqpoint{0.100000in}{0.212622in}}{\pgfqpoint{3.696000in}{3.696000in}}%
\pgfusepath{clip}%
\pgfsetbuttcap%
\pgfsetroundjoin%
\definecolor{currentfill}{rgb}{0.121569,0.466667,0.705882}%
\pgfsetfillcolor{currentfill}%
\pgfsetfillopacity{0.438176}%
\pgfsetlinewidth{1.003750pt}%
\definecolor{currentstroke}{rgb}{0.121569,0.466667,0.705882}%
\pgfsetstrokecolor{currentstroke}%
\pgfsetstrokeopacity{0.438176}%
\pgfsetdash{}{0pt}%
\pgfpathmoveto{\pgfqpoint{2.409607in}{2.089596in}}%
\pgfpathcurveto{\pgfqpoint{2.417843in}{2.089596in}}{\pgfqpoint{2.425743in}{2.092868in}}{\pgfqpoint{2.431567in}{2.098692in}}%
\pgfpathcurveto{\pgfqpoint{2.437391in}{2.104516in}}{\pgfqpoint{2.440663in}{2.112416in}}{\pgfqpoint{2.440663in}{2.120653in}}%
\pgfpathcurveto{\pgfqpoint{2.440663in}{2.128889in}}{\pgfqpoint{2.437391in}{2.136789in}}{\pgfqpoint{2.431567in}{2.142613in}}%
\pgfpathcurveto{\pgfqpoint{2.425743in}{2.148437in}}{\pgfqpoint{2.417843in}{2.151709in}}{\pgfqpoint{2.409607in}{2.151709in}}%
\pgfpathcurveto{\pgfqpoint{2.401370in}{2.151709in}}{\pgfqpoint{2.393470in}{2.148437in}}{\pgfqpoint{2.387646in}{2.142613in}}%
\pgfpathcurveto{\pgfqpoint{2.381822in}{2.136789in}}{\pgfqpoint{2.378550in}{2.128889in}}{\pgfqpoint{2.378550in}{2.120653in}}%
\pgfpathcurveto{\pgfqpoint{2.378550in}{2.112416in}}{\pgfqpoint{2.381822in}{2.104516in}}{\pgfqpoint{2.387646in}{2.098692in}}%
\pgfpathcurveto{\pgfqpoint{2.393470in}{2.092868in}}{\pgfqpoint{2.401370in}{2.089596in}}{\pgfqpoint{2.409607in}{2.089596in}}%
\pgfpathclose%
\pgfusepath{stroke,fill}%
\end{pgfscope}%
\begin{pgfscope}%
\pgfpathrectangle{\pgfqpoint{0.100000in}{0.212622in}}{\pgfqpoint{3.696000in}{3.696000in}}%
\pgfusepath{clip}%
\pgfsetbuttcap%
\pgfsetroundjoin%
\definecolor{currentfill}{rgb}{0.121569,0.466667,0.705882}%
\pgfsetfillcolor{currentfill}%
\pgfsetfillopacity{0.438557}%
\pgfsetlinewidth{1.003750pt}%
\definecolor{currentstroke}{rgb}{0.121569,0.466667,0.705882}%
\pgfsetstrokecolor{currentstroke}%
\pgfsetstrokeopacity{0.438557}%
\pgfsetdash{}{0pt}%
\pgfpathmoveto{\pgfqpoint{2.413367in}{2.085709in}}%
\pgfpathcurveto{\pgfqpoint{2.421604in}{2.085709in}}{\pgfqpoint{2.429504in}{2.088982in}}{\pgfqpoint{2.435328in}{2.094806in}}%
\pgfpathcurveto{\pgfqpoint{2.441152in}{2.100630in}}{\pgfqpoint{2.444424in}{2.108530in}}{\pgfqpoint{2.444424in}{2.116766in}}%
\pgfpathcurveto{\pgfqpoint{2.444424in}{2.125002in}}{\pgfqpoint{2.441152in}{2.132902in}}{\pgfqpoint{2.435328in}{2.138726in}}%
\pgfpathcurveto{\pgfqpoint{2.429504in}{2.144550in}}{\pgfqpoint{2.421604in}{2.147822in}}{\pgfqpoint{2.413367in}{2.147822in}}%
\pgfpathcurveto{\pgfqpoint{2.405131in}{2.147822in}}{\pgfqpoint{2.397231in}{2.144550in}}{\pgfqpoint{2.391407in}{2.138726in}}%
\pgfpathcurveto{\pgfqpoint{2.385583in}{2.132902in}}{\pgfqpoint{2.382311in}{2.125002in}}{\pgfqpoint{2.382311in}{2.116766in}}%
\pgfpathcurveto{\pgfqpoint{2.382311in}{2.108530in}}{\pgfqpoint{2.385583in}{2.100630in}}{\pgfqpoint{2.391407in}{2.094806in}}%
\pgfpathcurveto{\pgfqpoint{2.397231in}{2.088982in}}{\pgfqpoint{2.405131in}{2.085709in}}{\pgfqpoint{2.413367in}{2.085709in}}%
\pgfpathclose%
\pgfusepath{stroke,fill}%
\end{pgfscope}%
\begin{pgfscope}%
\pgfpathrectangle{\pgfqpoint{0.100000in}{0.212622in}}{\pgfqpoint{3.696000in}{3.696000in}}%
\pgfusepath{clip}%
\pgfsetbuttcap%
\pgfsetroundjoin%
\definecolor{currentfill}{rgb}{0.121569,0.466667,0.705882}%
\pgfsetfillcolor{currentfill}%
\pgfsetfillopacity{0.438852}%
\pgfsetlinewidth{1.003750pt}%
\definecolor{currentstroke}{rgb}{0.121569,0.466667,0.705882}%
\pgfsetstrokecolor{currentstroke}%
\pgfsetstrokeopacity{0.438852}%
\pgfsetdash{}{0pt}%
\pgfpathmoveto{\pgfqpoint{1.337054in}{1.838157in}}%
\pgfpathcurveto{\pgfqpoint{1.345291in}{1.838157in}}{\pgfqpoint{1.353191in}{1.841430in}}{\pgfqpoint{1.359015in}{1.847254in}}%
\pgfpathcurveto{\pgfqpoint{1.364839in}{1.853077in}}{\pgfqpoint{1.368111in}{1.860977in}}{\pgfqpoint{1.368111in}{1.869214in}}%
\pgfpathcurveto{\pgfqpoint{1.368111in}{1.877450in}}{\pgfqpoint{1.364839in}{1.885350in}}{\pgfqpoint{1.359015in}{1.891174in}}%
\pgfpathcurveto{\pgfqpoint{1.353191in}{1.896998in}}{\pgfqpoint{1.345291in}{1.900270in}}{\pgfqpoint{1.337054in}{1.900270in}}%
\pgfpathcurveto{\pgfqpoint{1.328818in}{1.900270in}}{\pgfqpoint{1.320918in}{1.896998in}}{\pgfqpoint{1.315094in}{1.891174in}}%
\pgfpathcurveto{\pgfqpoint{1.309270in}{1.885350in}}{\pgfqpoint{1.305998in}{1.877450in}}{\pgfqpoint{1.305998in}{1.869214in}}%
\pgfpathcurveto{\pgfqpoint{1.305998in}{1.860977in}}{\pgfqpoint{1.309270in}{1.853077in}}{\pgfqpoint{1.315094in}{1.847254in}}%
\pgfpathcurveto{\pgfqpoint{1.320918in}{1.841430in}}{\pgfqpoint{1.328818in}{1.838157in}}{\pgfqpoint{1.337054in}{1.838157in}}%
\pgfpathclose%
\pgfusepath{stroke,fill}%
\end{pgfscope}%
\begin{pgfscope}%
\pgfpathrectangle{\pgfqpoint{0.100000in}{0.212622in}}{\pgfqpoint{3.696000in}{3.696000in}}%
\pgfusepath{clip}%
\pgfsetbuttcap%
\pgfsetroundjoin%
\definecolor{currentfill}{rgb}{0.121569,0.466667,0.705882}%
\pgfsetfillcolor{currentfill}%
\pgfsetfillopacity{0.439731}%
\pgfsetlinewidth{1.003750pt}%
\definecolor{currentstroke}{rgb}{0.121569,0.466667,0.705882}%
\pgfsetstrokecolor{currentstroke}%
\pgfsetstrokeopacity{0.439731}%
\pgfsetdash{}{0pt}%
\pgfpathmoveto{\pgfqpoint{2.418616in}{2.089715in}}%
\pgfpathcurveto{\pgfqpoint{2.426853in}{2.089715in}}{\pgfqpoint{2.434753in}{2.092987in}}{\pgfqpoint{2.440576in}{2.098811in}}%
\pgfpathcurveto{\pgfqpoint{2.446400in}{2.104635in}}{\pgfqpoint{2.449673in}{2.112535in}}{\pgfqpoint{2.449673in}{2.120771in}}%
\pgfpathcurveto{\pgfqpoint{2.449673in}{2.129008in}}{\pgfqpoint{2.446400in}{2.136908in}}{\pgfqpoint{2.440576in}{2.142732in}}%
\pgfpathcurveto{\pgfqpoint{2.434753in}{2.148556in}}{\pgfqpoint{2.426853in}{2.151828in}}{\pgfqpoint{2.418616in}{2.151828in}}%
\pgfpathcurveto{\pgfqpoint{2.410380in}{2.151828in}}{\pgfqpoint{2.402480in}{2.148556in}}{\pgfqpoint{2.396656in}{2.142732in}}%
\pgfpathcurveto{\pgfqpoint{2.390832in}{2.136908in}}{\pgfqpoint{2.387560in}{2.129008in}}{\pgfqpoint{2.387560in}{2.120771in}}%
\pgfpathcurveto{\pgfqpoint{2.387560in}{2.112535in}}{\pgfqpoint{2.390832in}{2.104635in}}{\pgfqpoint{2.396656in}{2.098811in}}%
\pgfpathcurveto{\pgfqpoint{2.402480in}{2.092987in}}{\pgfqpoint{2.410380in}{2.089715in}}{\pgfqpoint{2.418616in}{2.089715in}}%
\pgfpathclose%
\pgfusepath{stroke,fill}%
\end{pgfscope}%
\begin{pgfscope}%
\pgfpathrectangle{\pgfqpoint{0.100000in}{0.212622in}}{\pgfqpoint{3.696000in}{3.696000in}}%
\pgfusepath{clip}%
\pgfsetbuttcap%
\pgfsetroundjoin%
\definecolor{currentfill}{rgb}{0.121569,0.466667,0.705882}%
\pgfsetfillcolor{currentfill}%
\pgfsetfillopacity{0.439853}%
\pgfsetlinewidth{1.003750pt}%
\definecolor{currentstroke}{rgb}{0.121569,0.466667,0.705882}%
\pgfsetstrokecolor{currentstroke}%
\pgfsetstrokeopacity{0.439853}%
\pgfsetdash{}{0pt}%
\pgfpathmoveto{\pgfqpoint{2.421198in}{2.087134in}}%
\pgfpathcurveto{\pgfqpoint{2.429434in}{2.087134in}}{\pgfqpoint{2.437334in}{2.090407in}}{\pgfqpoint{2.443158in}{2.096231in}}%
\pgfpathcurveto{\pgfqpoint{2.448982in}{2.102054in}}{\pgfqpoint{2.452254in}{2.109955in}}{\pgfqpoint{2.452254in}{2.118191in}}%
\pgfpathcurveto{\pgfqpoint{2.452254in}{2.126427in}}{\pgfqpoint{2.448982in}{2.134327in}}{\pgfqpoint{2.443158in}{2.140151in}}%
\pgfpathcurveto{\pgfqpoint{2.437334in}{2.145975in}}{\pgfqpoint{2.429434in}{2.149247in}}{\pgfqpoint{2.421198in}{2.149247in}}%
\pgfpathcurveto{\pgfqpoint{2.412962in}{2.149247in}}{\pgfqpoint{2.405061in}{2.145975in}}{\pgfqpoint{2.399238in}{2.140151in}}%
\pgfpathcurveto{\pgfqpoint{2.393414in}{2.134327in}}{\pgfqpoint{2.390141in}{2.126427in}}{\pgfqpoint{2.390141in}{2.118191in}}%
\pgfpathcurveto{\pgfqpoint{2.390141in}{2.109955in}}{\pgfqpoint{2.393414in}{2.102054in}}{\pgfqpoint{2.399238in}{2.096231in}}%
\pgfpathcurveto{\pgfqpoint{2.405061in}{2.090407in}}{\pgfqpoint{2.412962in}{2.087134in}}{\pgfqpoint{2.421198in}{2.087134in}}%
\pgfpathclose%
\pgfusepath{stroke,fill}%
\end{pgfscope}%
\begin{pgfscope}%
\pgfpathrectangle{\pgfqpoint{0.100000in}{0.212622in}}{\pgfqpoint{3.696000in}{3.696000in}}%
\pgfusepath{clip}%
\pgfsetbuttcap%
\pgfsetroundjoin%
\definecolor{currentfill}{rgb}{0.121569,0.466667,0.705882}%
\pgfsetfillcolor{currentfill}%
\pgfsetfillopacity{0.439930}%
\pgfsetlinewidth{1.003750pt}%
\definecolor{currentstroke}{rgb}{0.121569,0.466667,0.705882}%
\pgfsetstrokecolor{currentstroke}%
\pgfsetstrokeopacity{0.439930}%
\pgfsetdash{}{0pt}%
\pgfpathmoveto{\pgfqpoint{1.332684in}{1.837539in}}%
\pgfpathcurveto{\pgfqpoint{1.340921in}{1.837539in}}{\pgfqpoint{1.348821in}{1.840812in}}{\pgfqpoint{1.354645in}{1.846636in}}%
\pgfpathcurveto{\pgfqpoint{1.360468in}{1.852460in}}{\pgfqpoint{1.363741in}{1.860360in}}{\pgfqpoint{1.363741in}{1.868596in}}%
\pgfpathcurveto{\pgfqpoint{1.363741in}{1.876832in}}{\pgfqpoint{1.360468in}{1.884732in}}{\pgfqpoint{1.354645in}{1.890556in}}%
\pgfpathcurveto{\pgfqpoint{1.348821in}{1.896380in}}{\pgfqpoint{1.340921in}{1.899652in}}{\pgfqpoint{1.332684in}{1.899652in}}%
\pgfpathcurveto{\pgfqpoint{1.324448in}{1.899652in}}{\pgfqpoint{1.316548in}{1.896380in}}{\pgfqpoint{1.310724in}{1.890556in}}%
\pgfpathcurveto{\pgfqpoint{1.304900in}{1.884732in}}{\pgfqpoint{1.301628in}{1.876832in}}{\pgfqpoint{1.301628in}{1.868596in}}%
\pgfpathcurveto{\pgfqpoint{1.301628in}{1.860360in}}{\pgfqpoint{1.304900in}{1.852460in}}{\pgfqpoint{1.310724in}{1.846636in}}%
\pgfpathcurveto{\pgfqpoint{1.316548in}{1.840812in}}{\pgfqpoint{1.324448in}{1.837539in}}{\pgfqpoint{1.332684in}{1.837539in}}%
\pgfpathclose%
\pgfusepath{stroke,fill}%
\end{pgfscope}%
\begin{pgfscope}%
\pgfpathrectangle{\pgfqpoint{0.100000in}{0.212622in}}{\pgfqpoint{3.696000in}{3.696000in}}%
\pgfusepath{clip}%
\pgfsetbuttcap%
\pgfsetroundjoin%
\definecolor{currentfill}{rgb}{0.121569,0.466667,0.705882}%
\pgfsetfillcolor{currentfill}%
\pgfsetfillopacity{0.440390}%
\pgfsetlinewidth{1.003750pt}%
\definecolor{currentstroke}{rgb}{0.121569,0.466667,0.705882}%
\pgfsetstrokecolor{currentstroke}%
\pgfsetstrokeopacity{0.440390}%
\pgfsetdash{}{0pt}%
\pgfpathmoveto{\pgfqpoint{2.424811in}{2.086752in}}%
\pgfpathcurveto{\pgfqpoint{2.433047in}{2.086752in}}{\pgfqpoint{2.440947in}{2.090025in}}{\pgfqpoint{2.446771in}{2.095849in}}%
\pgfpathcurveto{\pgfqpoint{2.452595in}{2.101673in}}{\pgfqpoint{2.455867in}{2.109573in}}{\pgfqpoint{2.455867in}{2.117809in}}%
\pgfpathcurveto{\pgfqpoint{2.455867in}{2.126045in}}{\pgfqpoint{2.452595in}{2.133945in}}{\pgfqpoint{2.446771in}{2.139769in}}%
\pgfpathcurveto{\pgfqpoint{2.440947in}{2.145593in}}{\pgfqpoint{2.433047in}{2.148865in}}{\pgfqpoint{2.424811in}{2.148865in}}%
\pgfpathcurveto{\pgfqpoint{2.416574in}{2.148865in}}{\pgfqpoint{2.408674in}{2.145593in}}{\pgfqpoint{2.402850in}{2.139769in}}%
\pgfpathcurveto{\pgfqpoint{2.397026in}{2.133945in}}{\pgfqpoint{2.393754in}{2.126045in}}{\pgfqpoint{2.393754in}{2.117809in}}%
\pgfpathcurveto{\pgfqpoint{2.393754in}{2.109573in}}{\pgfqpoint{2.397026in}{2.101673in}}{\pgfqpoint{2.402850in}{2.095849in}}%
\pgfpathcurveto{\pgfqpoint{2.408674in}{2.090025in}}{\pgfqpoint{2.416574in}{2.086752in}}{\pgfqpoint{2.424811in}{2.086752in}}%
\pgfpathclose%
\pgfusepath{stroke,fill}%
\end{pgfscope}%
\begin{pgfscope}%
\pgfpathrectangle{\pgfqpoint{0.100000in}{0.212622in}}{\pgfqpoint{3.696000in}{3.696000in}}%
\pgfusepath{clip}%
\pgfsetbuttcap%
\pgfsetroundjoin%
\definecolor{currentfill}{rgb}{0.121569,0.466667,0.705882}%
\pgfsetfillcolor{currentfill}%
\pgfsetfillopacity{0.440529}%
\pgfsetlinewidth{1.003750pt}%
\definecolor{currentstroke}{rgb}{0.121569,0.466667,0.705882}%
\pgfsetstrokecolor{currentstroke}%
\pgfsetstrokeopacity{0.440529}%
\pgfsetdash{}{0pt}%
\pgfpathmoveto{\pgfqpoint{2.426679in}{2.085080in}}%
\pgfpathcurveto{\pgfqpoint{2.434915in}{2.085080in}}{\pgfqpoint{2.442815in}{2.088352in}}{\pgfqpoint{2.448639in}{2.094176in}}%
\pgfpathcurveto{\pgfqpoint{2.454463in}{2.100000in}}{\pgfqpoint{2.457736in}{2.107900in}}{\pgfqpoint{2.457736in}{2.116136in}}%
\pgfpathcurveto{\pgfqpoint{2.457736in}{2.124372in}}{\pgfqpoint{2.454463in}{2.132272in}}{\pgfqpoint{2.448639in}{2.138096in}}%
\pgfpathcurveto{\pgfqpoint{2.442815in}{2.143920in}}{\pgfqpoint{2.434915in}{2.147193in}}{\pgfqpoint{2.426679in}{2.147193in}}%
\pgfpathcurveto{\pgfqpoint{2.418443in}{2.147193in}}{\pgfqpoint{2.410543in}{2.143920in}}{\pgfqpoint{2.404719in}{2.138096in}}%
\pgfpathcurveto{\pgfqpoint{2.398895in}{2.132272in}}{\pgfqpoint{2.395623in}{2.124372in}}{\pgfqpoint{2.395623in}{2.116136in}}%
\pgfpathcurveto{\pgfqpoint{2.395623in}{2.107900in}}{\pgfqpoint{2.398895in}{2.100000in}}{\pgfqpoint{2.404719in}{2.094176in}}%
\pgfpathcurveto{\pgfqpoint{2.410543in}{2.088352in}}{\pgfqpoint{2.418443in}{2.085080in}}{\pgfqpoint{2.426679in}{2.085080in}}%
\pgfpathclose%
\pgfusepath{stroke,fill}%
\end{pgfscope}%
\begin{pgfscope}%
\pgfpathrectangle{\pgfqpoint{0.100000in}{0.212622in}}{\pgfqpoint{3.696000in}{3.696000in}}%
\pgfusepath{clip}%
\pgfsetbuttcap%
\pgfsetroundjoin%
\definecolor{currentfill}{rgb}{0.121569,0.466667,0.705882}%
\pgfsetfillcolor{currentfill}%
\pgfsetfillopacity{0.440924}%
\pgfsetlinewidth{1.003750pt}%
\definecolor{currentstroke}{rgb}{0.121569,0.466667,0.705882}%
\pgfsetstrokecolor{currentstroke}%
\pgfsetstrokeopacity{0.440924}%
\pgfsetdash{}{0pt}%
\pgfpathmoveto{\pgfqpoint{2.434163in}{2.078938in}}%
\pgfpathcurveto{\pgfqpoint{2.442400in}{2.078938in}}{\pgfqpoint{2.450300in}{2.082210in}}{\pgfqpoint{2.456124in}{2.088034in}}%
\pgfpathcurveto{\pgfqpoint{2.461947in}{2.093858in}}{\pgfqpoint{2.465220in}{2.101758in}}{\pgfqpoint{2.465220in}{2.109995in}}%
\pgfpathcurveto{\pgfqpoint{2.465220in}{2.118231in}}{\pgfqpoint{2.461947in}{2.126131in}}{\pgfqpoint{2.456124in}{2.131955in}}%
\pgfpathcurveto{\pgfqpoint{2.450300in}{2.137779in}}{\pgfqpoint{2.442400in}{2.141051in}}{\pgfqpoint{2.434163in}{2.141051in}}%
\pgfpathcurveto{\pgfqpoint{2.425927in}{2.141051in}}{\pgfqpoint{2.418027in}{2.137779in}}{\pgfqpoint{2.412203in}{2.131955in}}%
\pgfpathcurveto{\pgfqpoint{2.406379in}{2.126131in}}{\pgfqpoint{2.403107in}{2.118231in}}{\pgfqpoint{2.403107in}{2.109995in}}%
\pgfpathcurveto{\pgfqpoint{2.403107in}{2.101758in}}{\pgfqpoint{2.406379in}{2.093858in}}{\pgfqpoint{2.412203in}{2.088034in}}%
\pgfpathcurveto{\pgfqpoint{2.418027in}{2.082210in}}{\pgfqpoint{2.425927in}{2.078938in}}{\pgfqpoint{2.434163in}{2.078938in}}%
\pgfpathclose%
\pgfusepath{stroke,fill}%
\end{pgfscope}%
\begin{pgfscope}%
\pgfpathrectangle{\pgfqpoint{0.100000in}{0.212622in}}{\pgfqpoint{3.696000in}{3.696000in}}%
\pgfusepath{clip}%
\pgfsetbuttcap%
\pgfsetroundjoin%
\definecolor{currentfill}{rgb}{0.121569,0.466667,0.705882}%
\pgfsetfillcolor{currentfill}%
\pgfsetfillopacity{0.440946}%
\pgfsetlinewidth{1.003750pt}%
\definecolor{currentstroke}{rgb}{0.121569,0.466667,0.705882}%
\pgfsetstrokecolor{currentstroke}%
\pgfsetstrokeopacity{0.440946}%
\pgfsetdash{}{0pt}%
\pgfpathmoveto{\pgfqpoint{2.429457in}{2.084830in}}%
\pgfpathcurveto{\pgfqpoint{2.437693in}{2.084830in}}{\pgfqpoint{2.445593in}{2.088102in}}{\pgfqpoint{2.451417in}{2.093926in}}%
\pgfpathcurveto{\pgfqpoint{2.457241in}{2.099750in}}{\pgfqpoint{2.460513in}{2.107650in}}{\pgfqpoint{2.460513in}{2.115886in}}%
\pgfpathcurveto{\pgfqpoint{2.460513in}{2.124122in}}{\pgfqpoint{2.457241in}{2.132023in}}{\pgfqpoint{2.451417in}{2.137846in}}%
\pgfpathcurveto{\pgfqpoint{2.445593in}{2.143670in}}{\pgfqpoint{2.437693in}{2.146943in}}{\pgfqpoint{2.429457in}{2.146943in}}%
\pgfpathcurveto{\pgfqpoint{2.421220in}{2.146943in}}{\pgfqpoint{2.413320in}{2.143670in}}{\pgfqpoint{2.407496in}{2.137846in}}%
\pgfpathcurveto{\pgfqpoint{2.401673in}{2.132023in}}{\pgfqpoint{2.398400in}{2.124122in}}{\pgfqpoint{2.398400in}{2.115886in}}%
\pgfpathcurveto{\pgfqpoint{2.398400in}{2.107650in}}{\pgfqpoint{2.401673in}{2.099750in}}{\pgfqpoint{2.407496in}{2.093926in}}%
\pgfpathcurveto{\pgfqpoint{2.413320in}{2.088102in}}{\pgfqpoint{2.421220in}{2.084830in}}{\pgfqpoint{2.429457in}{2.084830in}}%
\pgfpathclose%
\pgfusepath{stroke,fill}%
\end{pgfscope}%
\begin{pgfscope}%
\pgfpathrectangle{\pgfqpoint{0.100000in}{0.212622in}}{\pgfqpoint{3.696000in}{3.696000in}}%
\pgfusepath{clip}%
\pgfsetbuttcap%
\pgfsetroundjoin%
\definecolor{currentfill}{rgb}{0.121569,0.466667,0.705882}%
\pgfsetfillcolor{currentfill}%
\pgfsetfillopacity{0.441020}%
\pgfsetlinewidth{1.003750pt}%
\definecolor{currentstroke}{rgb}{0.121569,0.466667,0.705882}%
\pgfsetstrokecolor{currentstroke}%
\pgfsetstrokeopacity{0.441020}%
\pgfsetdash{}{0pt}%
\pgfpathmoveto{\pgfqpoint{1.330676in}{1.834899in}}%
\pgfpathcurveto{\pgfqpoint{1.338912in}{1.834899in}}{\pgfqpoint{1.346812in}{1.838172in}}{\pgfqpoint{1.352636in}{1.843996in}}%
\pgfpathcurveto{\pgfqpoint{1.358460in}{1.849820in}}{\pgfqpoint{1.361732in}{1.857720in}}{\pgfqpoint{1.361732in}{1.865956in}}%
\pgfpathcurveto{\pgfqpoint{1.361732in}{1.874192in}}{\pgfqpoint{1.358460in}{1.882092in}}{\pgfqpoint{1.352636in}{1.887916in}}%
\pgfpathcurveto{\pgfqpoint{1.346812in}{1.893740in}}{\pgfqpoint{1.338912in}{1.897012in}}{\pgfqpoint{1.330676in}{1.897012in}}%
\pgfpathcurveto{\pgfqpoint{1.322439in}{1.897012in}}{\pgfqpoint{1.314539in}{1.893740in}}{\pgfqpoint{1.308715in}{1.887916in}}%
\pgfpathcurveto{\pgfqpoint{1.302891in}{1.882092in}}{\pgfqpoint{1.299619in}{1.874192in}}{\pgfqpoint{1.299619in}{1.865956in}}%
\pgfpathcurveto{\pgfqpoint{1.299619in}{1.857720in}}{\pgfqpoint{1.302891in}{1.849820in}}{\pgfqpoint{1.308715in}{1.843996in}}%
\pgfpathcurveto{\pgfqpoint{1.314539in}{1.838172in}}{\pgfqpoint{1.322439in}{1.834899in}}{\pgfqpoint{1.330676in}{1.834899in}}%
\pgfpathclose%
\pgfusepath{stroke,fill}%
\end{pgfscope}%
\begin{pgfscope}%
\pgfpathrectangle{\pgfqpoint{0.100000in}{0.212622in}}{\pgfqpoint{3.696000in}{3.696000in}}%
\pgfusepath{clip}%
\pgfsetbuttcap%
\pgfsetroundjoin%
\definecolor{currentfill}{rgb}{0.121569,0.466667,0.705882}%
\pgfsetfillcolor{currentfill}%
\pgfsetfillopacity{0.441628}%
\pgfsetlinewidth{1.003750pt}%
\definecolor{currentstroke}{rgb}{0.121569,0.466667,0.705882}%
\pgfsetstrokecolor{currentstroke}%
\pgfsetstrokeopacity{0.441628}%
\pgfsetdash{}{0pt}%
\pgfpathmoveto{\pgfqpoint{2.447771in}{2.073332in}}%
\pgfpathcurveto{\pgfqpoint{2.456007in}{2.073332in}}{\pgfqpoint{2.463907in}{2.076604in}}{\pgfqpoint{2.469731in}{2.082428in}}%
\pgfpathcurveto{\pgfqpoint{2.475555in}{2.088252in}}{\pgfqpoint{2.478827in}{2.096152in}}{\pgfqpoint{2.478827in}{2.104388in}}%
\pgfpathcurveto{\pgfqpoint{2.478827in}{2.112624in}}{\pgfqpoint{2.475555in}{2.120524in}}{\pgfqpoint{2.469731in}{2.126348in}}%
\pgfpathcurveto{\pgfqpoint{2.463907in}{2.132172in}}{\pgfqpoint{2.456007in}{2.135445in}}{\pgfqpoint{2.447771in}{2.135445in}}%
\pgfpathcurveto{\pgfqpoint{2.439534in}{2.135445in}}{\pgfqpoint{2.431634in}{2.132172in}}{\pgfqpoint{2.425810in}{2.126348in}}%
\pgfpathcurveto{\pgfqpoint{2.419986in}{2.120524in}}{\pgfqpoint{2.416714in}{2.112624in}}{\pgfqpoint{2.416714in}{2.104388in}}%
\pgfpathcurveto{\pgfqpoint{2.416714in}{2.096152in}}{\pgfqpoint{2.419986in}{2.088252in}}{\pgfqpoint{2.425810in}{2.082428in}}%
\pgfpathcurveto{\pgfqpoint{2.431634in}{2.076604in}}{\pgfqpoint{2.439534in}{2.073332in}}{\pgfqpoint{2.447771in}{2.073332in}}%
\pgfpathclose%
\pgfusepath{stroke,fill}%
\end{pgfscope}%
\begin{pgfscope}%
\pgfpathrectangle{\pgfqpoint{0.100000in}{0.212622in}}{\pgfqpoint{3.696000in}{3.696000in}}%
\pgfusepath{clip}%
\pgfsetbuttcap%
\pgfsetroundjoin%
\definecolor{currentfill}{rgb}{0.121569,0.466667,0.705882}%
\pgfsetfillcolor{currentfill}%
\pgfsetfillopacity{0.441760}%
\pgfsetlinewidth{1.003750pt}%
\definecolor{currentstroke}{rgb}{0.121569,0.466667,0.705882}%
\pgfsetstrokecolor{currentstroke}%
\pgfsetstrokeopacity{0.441760}%
\pgfsetdash{}{0pt}%
\pgfpathmoveto{\pgfqpoint{2.440894in}{2.081642in}}%
\pgfpathcurveto{\pgfqpoint{2.449131in}{2.081642in}}{\pgfqpoint{2.457031in}{2.084915in}}{\pgfqpoint{2.462855in}{2.090738in}}%
\pgfpathcurveto{\pgfqpoint{2.468679in}{2.096562in}}{\pgfqpoint{2.471951in}{2.104462in}}{\pgfqpoint{2.471951in}{2.112699in}}%
\pgfpathcurveto{\pgfqpoint{2.471951in}{2.120935in}}{\pgfqpoint{2.468679in}{2.128835in}}{\pgfqpoint{2.462855in}{2.134659in}}%
\pgfpathcurveto{\pgfqpoint{2.457031in}{2.140483in}}{\pgfqpoint{2.449131in}{2.143755in}}{\pgfqpoint{2.440894in}{2.143755in}}%
\pgfpathcurveto{\pgfqpoint{2.432658in}{2.143755in}}{\pgfqpoint{2.424758in}{2.140483in}}{\pgfqpoint{2.418934in}{2.134659in}}%
\pgfpathcurveto{\pgfqpoint{2.413110in}{2.128835in}}{\pgfqpoint{2.409838in}{2.120935in}}{\pgfqpoint{2.409838in}{2.112699in}}%
\pgfpathcurveto{\pgfqpoint{2.409838in}{2.104462in}}{\pgfqpoint{2.413110in}{2.096562in}}{\pgfqpoint{2.418934in}{2.090738in}}%
\pgfpathcurveto{\pgfqpoint{2.424758in}{2.084915in}}{\pgfqpoint{2.432658in}{2.081642in}}{\pgfqpoint{2.440894in}{2.081642in}}%
\pgfpathclose%
\pgfusepath{stroke,fill}%
\end{pgfscope}%
\begin{pgfscope}%
\pgfpathrectangle{\pgfqpoint{0.100000in}{0.212622in}}{\pgfqpoint{3.696000in}{3.696000in}}%
\pgfusepath{clip}%
\pgfsetbuttcap%
\pgfsetroundjoin%
\definecolor{currentfill}{rgb}{0.121569,0.466667,0.705882}%
\pgfsetfillcolor{currentfill}%
\pgfsetfillopacity{0.442309}%
\pgfsetlinewidth{1.003750pt}%
\definecolor{currentstroke}{rgb}{0.121569,0.466667,0.705882}%
\pgfsetstrokecolor{currentstroke}%
\pgfsetstrokeopacity{0.442309}%
\pgfsetdash{}{0pt}%
\pgfpathmoveto{\pgfqpoint{2.455763in}{2.071684in}}%
\pgfpathcurveto{\pgfqpoint{2.463999in}{2.071684in}}{\pgfqpoint{2.471899in}{2.074957in}}{\pgfqpoint{2.477723in}{2.080780in}}%
\pgfpathcurveto{\pgfqpoint{2.483547in}{2.086604in}}{\pgfqpoint{2.486819in}{2.094504in}}{\pgfqpoint{2.486819in}{2.102741in}}%
\pgfpathcurveto{\pgfqpoint{2.486819in}{2.110977in}}{\pgfqpoint{2.483547in}{2.118877in}}{\pgfqpoint{2.477723in}{2.124701in}}%
\pgfpathcurveto{\pgfqpoint{2.471899in}{2.130525in}}{\pgfqpoint{2.463999in}{2.133797in}}{\pgfqpoint{2.455763in}{2.133797in}}%
\pgfpathcurveto{\pgfqpoint{2.447527in}{2.133797in}}{\pgfqpoint{2.439627in}{2.130525in}}{\pgfqpoint{2.433803in}{2.124701in}}%
\pgfpathcurveto{\pgfqpoint{2.427979in}{2.118877in}}{\pgfqpoint{2.424706in}{2.110977in}}{\pgfqpoint{2.424706in}{2.102741in}}%
\pgfpathcurveto{\pgfqpoint{2.424706in}{2.094504in}}{\pgfqpoint{2.427979in}{2.086604in}}{\pgfqpoint{2.433803in}{2.080780in}}%
\pgfpathcurveto{\pgfqpoint{2.439627in}{2.074957in}}{\pgfqpoint{2.447527in}{2.071684in}}{\pgfqpoint{2.455763in}{2.071684in}}%
\pgfpathclose%
\pgfusepath{stroke,fill}%
\end{pgfscope}%
\begin{pgfscope}%
\pgfpathrectangle{\pgfqpoint{0.100000in}{0.212622in}}{\pgfqpoint{3.696000in}{3.696000in}}%
\pgfusepath{clip}%
\pgfsetbuttcap%
\pgfsetroundjoin%
\definecolor{currentfill}{rgb}{0.121569,0.466667,0.705882}%
\pgfsetfillcolor{currentfill}%
\pgfsetfillopacity{0.442434}%
\pgfsetlinewidth{1.003750pt}%
\definecolor{currentstroke}{rgb}{0.121569,0.466667,0.705882}%
\pgfsetstrokecolor{currentstroke}%
\pgfsetstrokeopacity{0.442434}%
\pgfsetdash{}{0pt}%
\pgfpathmoveto{\pgfqpoint{1.327265in}{1.825835in}}%
\pgfpathcurveto{\pgfqpoint{1.335501in}{1.825835in}}{\pgfqpoint{1.343401in}{1.829107in}}{\pgfqpoint{1.349225in}{1.834931in}}%
\pgfpathcurveto{\pgfqpoint{1.355049in}{1.840755in}}{\pgfqpoint{1.358321in}{1.848655in}}{\pgfqpoint{1.358321in}{1.856891in}}%
\pgfpathcurveto{\pgfqpoint{1.358321in}{1.865128in}}{\pgfqpoint{1.355049in}{1.873028in}}{\pgfqpoint{1.349225in}{1.878852in}}%
\pgfpathcurveto{\pgfqpoint{1.343401in}{1.884676in}}{\pgfqpoint{1.335501in}{1.887948in}}{\pgfqpoint{1.327265in}{1.887948in}}%
\pgfpathcurveto{\pgfqpoint{1.319028in}{1.887948in}}{\pgfqpoint{1.311128in}{1.884676in}}{\pgfqpoint{1.305304in}{1.878852in}}%
\pgfpathcurveto{\pgfqpoint{1.299480in}{1.873028in}}{\pgfqpoint{1.296208in}{1.865128in}}{\pgfqpoint{1.296208in}{1.856891in}}%
\pgfpathcurveto{\pgfqpoint{1.296208in}{1.848655in}}{\pgfqpoint{1.299480in}{1.840755in}}{\pgfqpoint{1.305304in}{1.834931in}}%
\pgfpathcurveto{\pgfqpoint{1.311128in}{1.829107in}}{\pgfqpoint{1.319028in}{1.825835in}}{\pgfqpoint{1.327265in}{1.825835in}}%
\pgfpathclose%
\pgfusepath{stroke,fill}%
\end{pgfscope}%
\begin{pgfscope}%
\pgfpathrectangle{\pgfqpoint{0.100000in}{0.212622in}}{\pgfqpoint{3.696000in}{3.696000in}}%
\pgfusepath{clip}%
\pgfsetbuttcap%
\pgfsetroundjoin%
\definecolor{currentfill}{rgb}{0.121569,0.466667,0.705882}%
\pgfsetfillcolor{currentfill}%
\pgfsetfillopacity{0.442842}%
\pgfsetlinewidth{1.003750pt}%
\definecolor{currentstroke}{rgb}{0.121569,0.466667,0.705882}%
\pgfsetstrokecolor{currentstroke}%
\pgfsetstrokeopacity{0.442842}%
\pgfsetdash{}{0pt}%
\pgfpathmoveto{\pgfqpoint{2.463527in}{2.065801in}}%
\pgfpathcurveto{\pgfqpoint{2.471764in}{2.065801in}}{\pgfqpoint{2.479664in}{2.069073in}}{\pgfqpoint{2.485488in}{2.074897in}}%
\pgfpathcurveto{\pgfqpoint{2.491311in}{2.080721in}}{\pgfqpoint{2.494584in}{2.088621in}}{\pgfqpoint{2.494584in}{2.096857in}}%
\pgfpathcurveto{\pgfqpoint{2.494584in}{2.105094in}}{\pgfqpoint{2.491311in}{2.112994in}}{\pgfqpoint{2.485488in}{2.118818in}}%
\pgfpathcurveto{\pgfqpoint{2.479664in}{2.124642in}}{\pgfqpoint{2.471764in}{2.127914in}}{\pgfqpoint{2.463527in}{2.127914in}}%
\pgfpathcurveto{\pgfqpoint{2.455291in}{2.127914in}}{\pgfqpoint{2.447391in}{2.124642in}}{\pgfqpoint{2.441567in}{2.118818in}}%
\pgfpathcurveto{\pgfqpoint{2.435743in}{2.112994in}}{\pgfqpoint{2.432471in}{2.105094in}}{\pgfqpoint{2.432471in}{2.096857in}}%
\pgfpathcurveto{\pgfqpoint{2.432471in}{2.088621in}}{\pgfqpoint{2.435743in}{2.080721in}}{\pgfqpoint{2.441567in}{2.074897in}}%
\pgfpathcurveto{\pgfqpoint{2.447391in}{2.069073in}}{\pgfqpoint{2.455291in}{2.065801in}}{\pgfqpoint{2.463527in}{2.065801in}}%
\pgfpathclose%
\pgfusepath{stroke,fill}%
\end{pgfscope}%
\begin{pgfscope}%
\pgfpathrectangle{\pgfqpoint{0.100000in}{0.212622in}}{\pgfqpoint{3.696000in}{3.696000in}}%
\pgfusepath{clip}%
\pgfsetbuttcap%
\pgfsetroundjoin%
\definecolor{currentfill}{rgb}{0.121569,0.466667,0.705882}%
\pgfsetfillcolor{currentfill}%
\pgfsetfillopacity{0.443718}%
\pgfsetlinewidth{1.003750pt}%
\definecolor{currentstroke}{rgb}{0.121569,0.466667,0.705882}%
\pgfsetstrokecolor{currentstroke}%
\pgfsetstrokeopacity{0.443718}%
\pgfsetdash{}{0pt}%
\pgfpathmoveto{\pgfqpoint{2.471984in}{2.062665in}}%
\pgfpathcurveto{\pgfqpoint{2.480220in}{2.062665in}}{\pgfqpoint{2.488120in}{2.065937in}}{\pgfqpoint{2.493944in}{2.071761in}}%
\pgfpathcurveto{\pgfqpoint{2.499768in}{2.077585in}}{\pgfqpoint{2.503041in}{2.085485in}}{\pgfqpoint{2.503041in}{2.093721in}}%
\pgfpathcurveto{\pgfqpoint{2.503041in}{2.101958in}}{\pgfqpoint{2.499768in}{2.109858in}}{\pgfqpoint{2.493944in}{2.115682in}}%
\pgfpathcurveto{\pgfqpoint{2.488120in}{2.121506in}}{\pgfqpoint{2.480220in}{2.124778in}}{\pgfqpoint{2.471984in}{2.124778in}}%
\pgfpathcurveto{\pgfqpoint{2.463748in}{2.124778in}}{\pgfqpoint{2.455848in}{2.121506in}}{\pgfqpoint{2.450024in}{2.115682in}}%
\pgfpathcurveto{\pgfqpoint{2.444200in}{2.109858in}}{\pgfqpoint{2.440928in}{2.101958in}}{\pgfqpoint{2.440928in}{2.093721in}}%
\pgfpathcurveto{\pgfqpoint{2.440928in}{2.085485in}}{\pgfqpoint{2.444200in}{2.077585in}}{\pgfqpoint{2.450024in}{2.071761in}}%
\pgfpathcurveto{\pgfqpoint{2.455848in}{2.065937in}}{\pgfqpoint{2.463748in}{2.062665in}}{\pgfqpoint{2.471984in}{2.062665in}}%
\pgfpathclose%
\pgfusepath{stroke,fill}%
\end{pgfscope}%
\begin{pgfscope}%
\pgfpathrectangle{\pgfqpoint{0.100000in}{0.212622in}}{\pgfqpoint{3.696000in}{3.696000in}}%
\pgfusepath{clip}%
\pgfsetbuttcap%
\pgfsetroundjoin%
\definecolor{currentfill}{rgb}{0.121569,0.466667,0.705882}%
\pgfsetfillcolor{currentfill}%
\pgfsetfillopacity{0.444021}%
\pgfsetlinewidth{1.003750pt}%
\definecolor{currentstroke}{rgb}{0.121569,0.466667,0.705882}%
\pgfsetstrokecolor{currentstroke}%
\pgfsetstrokeopacity{0.444021}%
\pgfsetdash{}{0pt}%
\pgfpathmoveto{\pgfqpoint{1.322212in}{1.821820in}}%
\pgfpathcurveto{\pgfqpoint{1.330448in}{1.821820in}}{\pgfqpoint{1.338348in}{1.825093in}}{\pgfqpoint{1.344172in}{1.830917in}}%
\pgfpathcurveto{\pgfqpoint{1.349996in}{1.836741in}}{\pgfqpoint{1.353269in}{1.844641in}}{\pgfqpoint{1.353269in}{1.852877in}}%
\pgfpathcurveto{\pgfqpoint{1.353269in}{1.861113in}}{\pgfqpoint{1.349996in}{1.869013in}}{\pgfqpoint{1.344172in}{1.874837in}}%
\pgfpathcurveto{\pgfqpoint{1.338348in}{1.880661in}}{\pgfqpoint{1.330448in}{1.883933in}}{\pgfqpoint{1.322212in}{1.883933in}}%
\pgfpathcurveto{\pgfqpoint{1.313976in}{1.883933in}}{\pgfqpoint{1.306076in}{1.880661in}}{\pgfqpoint{1.300252in}{1.874837in}}%
\pgfpathcurveto{\pgfqpoint{1.294428in}{1.869013in}}{\pgfqpoint{1.291156in}{1.861113in}}{\pgfqpoint{1.291156in}{1.852877in}}%
\pgfpathcurveto{\pgfqpoint{1.291156in}{1.844641in}}{\pgfqpoint{1.294428in}{1.836741in}}{\pgfqpoint{1.300252in}{1.830917in}}%
\pgfpathcurveto{\pgfqpoint{1.306076in}{1.825093in}}{\pgfqpoint{1.313976in}{1.821820in}}{\pgfqpoint{1.322212in}{1.821820in}}%
\pgfpathclose%
\pgfusepath{stroke,fill}%
\end{pgfscope}%
\begin{pgfscope}%
\pgfpathrectangle{\pgfqpoint{0.100000in}{0.212622in}}{\pgfqpoint{3.696000in}{3.696000in}}%
\pgfusepath{clip}%
\pgfsetbuttcap%
\pgfsetroundjoin%
\definecolor{currentfill}{rgb}{0.121569,0.466667,0.705882}%
\pgfsetfillcolor{currentfill}%
\pgfsetfillopacity{0.444373}%
\pgfsetlinewidth{1.003750pt}%
\definecolor{currentstroke}{rgb}{0.121569,0.466667,0.705882}%
\pgfsetstrokecolor{currentstroke}%
\pgfsetstrokeopacity{0.444373}%
\pgfsetdash{}{0pt}%
\pgfpathmoveto{\pgfqpoint{2.482231in}{2.054593in}}%
\pgfpathcurveto{\pgfqpoint{2.490468in}{2.054593in}}{\pgfqpoint{2.498368in}{2.057865in}}{\pgfqpoint{2.504192in}{2.063689in}}%
\pgfpathcurveto{\pgfqpoint{2.510016in}{2.069513in}}{\pgfqpoint{2.513288in}{2.077413in}}{\pgfqpoint{2.513288in}{2.085649in}}%
\pgfpathcurveto{\pgfqpoint{2.513288in}{2.093886in}}{\pgfqpoint{2.510016in}{2.101786in}}{\pgfqpoint{2.504192in}{2.107610in}}%
\pgfpathcurveto{\pgfqpoint{2.498368in}{2.113434in}}{\pgfqpoint{2.490468in}{2.116706in}}{\pgfqpoint{2.482231in}{2.116706in}}%
\pgfpathcurveto{\pgfqpoint{2.473995in}{2.116706in}}{\pgfqpoint{2.466095in}{2.113434in}}{\pgfqpoint{2.460271in}{2.107610in}}%
\pgfpathcurveto{\pgfqpoint{2.454447in}{2.101786in}}{\pgfqpoint{2.451175in}{2.093886in}}{\pgfqpoint{2.451175in}{2.085649in}}%
\pgfpathcurveto{\pgfqpoint{2.451175in}{2.077413in}}{\pgfqpoint{2.454447in}{2.069513in}}{\pgfqpoint{2.460271in}{2.063689in}}%
\pgfpathcurveto{\pgfqpoint{2.466095in}{2.057865in}}{\pgfqpoint{2.473995in}{2.054593in}}{\pgfqpoint{2.482231in}{2.054593in}}%
\pgfpathclose%
\pgfusepath{stroke,fill}%
\end{pgfscope}%
\begin{pgfscope}%
\pgfpathrectangle{\pgfqpoint{0.100000in}{0.212622in}}{\pgfqpoint{3.696000in}{3.696000in}}%
\pgfusepath{clip}%
\pgfsetbuttcap%
\pgfsetroundjoin%
\definecolor{currentfill}{rgb}{0.121569,0.466667,0.705882}%
\pgfsetfillcolor{currentfill}%
\pgfsetfillopacity{0.445439}%
\pgfsetlinewidth{1.003750pt}%
\definecolor{currentstroke}{rgb}{0.121569,0.466667,0.705882}%
\pgfsetstrokecolor{currentstroke}%
\pgfsetstrokeopacity{0.445439}%
\pgfsetdash{}{0pt}%
\pgfpathmoveto{\pgfqpoint{1.319461in}{1.816273in}}%
\pgfpathcurveto{\pgfqpoint{1.327698in}{1.816273in}}{\pgfqpoint{1.335598in}{1.819545in}}{\pgfqpoint{1.341422in}{1.825369in}}%
\pgfpathcurveto{\pgfqpoint{1.347246in}{1.831193in}}{\pgfqpoint{1.350518in}{1.839093in}}{\pgfqpoint{1.350518in}{1.847329in}}%
\pgfpathcurveto{\pgfqpoint{1.350518in}{1.855566in}}{\pgfqpoint{1.347246in}{1.863466in}}{\pgfqpoint{1.341422in}{1.869290in}}%
\pgfpathcurveto{\pgfqpoint{1.335598in}{1.875114in}}{\pgfqpoint{1.327698in}{1.878386in}}{\pgfqpoint{1.319461in}{1.878386in}}%
\pgfpathcurveto{\pgfqpoint{1.311225in}{1.878386in}}{\pgfqpoint{1.303325in}{1.875114in}}{\pgfqpoint{1.297501in}{1.869290in}}%
\pgfpathcurveto{\pgfqpoint{1.291677in}{1.863466in}}{\pgfqpoint{1.288405in}{1.855566in}}{\pgfqpoint{1.288405in}{1.847329in}}%
\pgfpathcurveto{\pgfqpoint{1.288405in}{1.839093in}}{\pgfqpoint{1.291677in}{1.831193in}}{\pgfqpoint{1.297501in}{1.825369in}}%
\pgfpathcurveto{\pgfqpoint{1.303325in}{1.819545in}}{\pgfqpoint{1.311225in}{1.816273in}}{\pgfqpoint{1.319461in}{1.816273in}}%
\pgfpathclose%
\pgfusepath{stroke,fill}%
\end{pgfscope}%
\begin{pgfscope}%
\pgfpathrectangle{\pgfqpoint{0.100000in}{0.212622in}}{\pgfqpoint{3.696000in}{3.696000in}}%
\pgfusepath{clip}%
\pgfsetbuttcap%
\pgfsetroundjoin%
\definecolor{currentfill}{rgb}{0.121569,0.466667,0.705882}%
\pgfsetfillcolor{currentfill}%
\pgfsetfillopacity{0.446094}%
\pgfsetlinewidth{1.003750pt}%
\definecolor{currentstroke}{rgb}{0.121569,0.466667,0.705882}%
\pgfsetstrokecolor{currentstroke}%
\pgfsetstrokeopacity{0.446094}%
\pgfsetdash{}{0pt}%
\pgfpathmoveto{\pgfqpoint{2.495125in}{2.059873in}}%
\pgfpathcurveto{\pgfqpoint{2.503361in}{2.059873in}}{\pgfqpoint{2.511261in}{2.063145in}}{\pgfqpoint{2.517085in}{2.068969in}}%
\pgfpathcurveto{\pgfqpoint{2.522909in}{2.074793in}}{\pgfqpoint{2.526181in}{2.082693in}}{\pgfqpoint{2.526181in}{2.090929in}}%
\pgfpathcurveto{\pgfqpoint{2.526181in}{2.099165in}}{\pgfqpoint{2.522909in}{2.107066in}}{\pgfqpoint{2.517085in}{2.112889in}}%
\pgfpathcurveto{\pgfqpoint{2.511261in}{2.118713in}}{\pgfqpoint{2.503361in}{2.121986in}}{\pgfqpoint{2.495125in}{2.121986in}}%
\pgfpathcurveto{\pgfqpoint{2.486889in}{2.121986in}}{\pgfqpoint{2.478989in}{2.118713in}}{\pgfqpoint{2.473165in}{2.112889in}}%
\pgfpathcurveto{\pgfqpoint{2.467341in}{2.107066in}}{\pgfqpoint{2.464068in}{2.099165in}}{\pgfqpoint{2.464068in}{2.090929in}}%
\pgfpathcurveto{\pgfqpoint{2.464068in}{2.082693in}}{\pgfqpoint{2.467341in}{2.074793in}}{\pgfqpoint{2.473165in}{2.068969in}}%
\pgfpathcurveto{\pgfqpoint{2.478989in}{2.063145in}}{\pgfqpoint{2.486889in}{2.059873in}}{\pgfqpoint{2.495125in}{2.059873in}}%
\pgfpathclose%
\pgfusepath{stroke,fill}%
\end{pgfscope}%
\begin{pgfscope}%
\pgfpathrectangle{\pgfqpoint{0.100000in}{0.212622in}}{\pgfqpoint{3.696000in}{3.696000in}}%
\pgfusepath{clip}%
\pgfsetbuttcap%
\pgfsetroundjoin%
\definecolor{currentfill}{rgb}{0.121569,0.466667,0.705882}%
\pgfsetfillcolor{currentfill}%
\pgfsetfillopacity{0.446253}%
\pgfsetlinewidth{1.003750pt}%
\definecolor{currentstroke}{rgb}{0.121569,0.466667,0.705882}%
\pgfsetstrokecolor{currentstroke}%
\pgfsetstrokeopacity{0.446253}%
\pgfsetdash{}{0pt}%
\pgfpathmoveto{\pgfqpoint{2.507902in}{2.047220in}}%
\pgfpathcurveto{\pgfqpoint{2.516139in}{2.047220in}}{\pgfqpoint{2.524039in}{2.050493in}}{\pgfqpoint{2.529863in}{2.056316in}}%
\pgfpathcurveto{\pgfqpoint{2.535687in}{2.062140in}}{\pgfqpoint{2.538959in}{2.070040in}}{\pgfqpoint{2.538959in}{2.078277in}}%
\pgfpathcurveto{\pgfqpoint{2.538959in}{2.086513in}}{\pgfqpoint{2.535687in}{2.094413in}}{\pgfqpoint{2.529863in}{2.100237in}}%
\pgfpathcurveto{\pgfqpoint{2.524039in}{2.106061in}}{\pgfqpoint{2.516139in}{2.109333in}}{\pgfqpoint{2.507902in}{2.109333in}}%
\pgfpathcurveto{\pgfqpoint{2.499666in}{2.109333in}}{\pgfqpoint{2.491766in}{2.106061in}}{\pgfqpoint{2.485942in}{2.100237in}}%
\pgfpathcurveto{\pgfqpoint{2.480118in}{2.094413in}}{\pgfqpoint{2.476846in}{2.086513in}}{\pgfqpoint{2.476846in}{2.078277in}}%
\pgfpathcurveto{\pgfqpoint{2.476846in}{2.070040in}}{\pgfqpoint{2.480118in}{2.062140in}}{\pgfqpoint{2.485942in}{2.056316in}}%
\pgfpathcurveto{\pgfqpoint{2.491766in}{2.050493in}}{\pgfqpoint{2.499666in}{2.047220in}}{\pgfqpoint{2.507902in}{2.047220in}}%
\pgfpathclose%
\pgfusepath{stroke,fill}%
\end{pgfscope}%
\begin{pgfscope}%
\pgfpathrectangle{\pgfqpoint{0.100000in}{0.212622in}}{\pgfqpoint{3.696000in}{3.696000in}}%
\pgfusepath{clip}%
\pgfsetbuttcap%
\pgfsetroundjoin%
\definecolor{currentfill}{rgb}{0.121569,0.466667,0.705882}%
\pgfsetfillcolor{currentfill}%
\pgfsetfillopacity{0.446443}%
\pgfsetlinewidth{1.003750pt}%
\definecolor{currentstroke}{rgb}{0.121569,0.466667,0.705882}%
\pgfsetstrokecolor{currentstroke}%
\pgfsetstrokeopacity{0.446443}%
\pgfsetdash{}{0pt}%
\pgfpathmoveto{\pgfqpoint{1.316958in}{1.812184in}}%
\pgfpathcurveto{\pgfqpoint{1.325195in}{1.812184in}}{\pgfqpoint{1.333095in}{1.815457in}}{\pgfqpoint{1.338919in}{1.821281in}}%
\pgfpathcurveto{\pgfqpoint{1.344742in}{1.827105in}}{\pgfqpoint{1.348015in}{1.835005in}}{\pgfqpoint{1.348015in}{1.843241in}}%
\pgfpathcurveto{\pgfqpoint{1.348015in}{1.851477in}}{\pgfqpoint{1.344742in}{1.859377in}}{\pgfqpoint{1.338919in}{1.865201in}}%
\pgfpathcurveto{\pgfqpoint{1.333095in}{1.871025in}}{\pgfqpoint{1.325195in}{1.874297in}}{\pgfqpoint{1.316958in}{1.874297in}}%
\pgfpathcurveto{\pgfqpoint{1.308722in}{1.874297in}}{\pgfqpoint{1.300822in}{1.871025in}}{\pgfqpoint{1.294998in}{1.865201in}}%
\pgfpathcurveto{\pgfqpoint{1.289174in}{1.859377in}}{\pgfqpoint{1.285902in}{1.851477in}}{\pgfqpoint{1.285902in}{1.843241in}}%
\pgfpathcurveto{\pgfqpoint{1.285902in}{1.835005in}}{\pgfqpoint{1.289174in}{1.827105in}}{\pgfqpoint{1.294998in}{1.821281in}}%
\pgfpathcurveto{\pgfqpoint{1.300822in}{1.815457in}}{\pgfqpoint{1.308722in}{1.812184in}}{\pgfqpoint{1.316958in}{1.812184in}}%
\pgfpathclose%
\pgfusepath{stroke,fill}%
\end{pgfscope}%
\begin{pgfscope}%
\pgfpathrectangle{\pgfqpoint{0.100000in}{0.212622in}}{\pgfqpoint{3.696000in}{3.696000in}}%
\pgfusepath{clip}%
\pgfsetbuttcap%
\pgfsetroundjoin%
\definecolor{currentfill}{rgb}{0.121569,0.466667,0.705882}%
\pgfsetfillcolor{currentfill}%
\pgfsetfillopacity{0.446860}%
\pgfsetlinewidth{1.003750pt}%
\definecolor{currentstroke}{rgb}{0.121569,0.466667,0.705882}%
\pgfsetstrokecolor{currentstroke}%
\pgfsetstrokeopacity{0.446860}%
\pgfsetdash{}{0pt}%
\pgfpathmoveto{\pgfqpoint{1.314945in}{1.809150in}}%
\pgfpathcurveto{\pgfqpoint{1.323182in}{1.809150in}}{\pgfqpoint{1.331082in}{1.812422in}}{\pgfqpoint{1.336906in}{1.818246in}}%
\pgfpathcurveto{\pgfqpoint{1.342730in}{1.824070in}}{\pgfqpoint{1.346002in}{1.831970in}}{\pgfqpoint{1.346002in}{1.840206in}}%
\pgfpathcurveto{\pgfqpoint{1.346002in}{1.848443in}}{\pgfqpoint{1.342730in}{1.856343in}}{\pgfqpoint{1.336906in}{1.862167in}}%
\pgfpathcurveto{\pgfqpoint{1.331082in}{1.867990in}}{\pgfqpoint{1.323182in}{1.871263in}}{\pgfqpoint{1.314945in}{1.871263in}}%
\pgfpathcurveto{\pgfqpoint{1.306709in}{1.871263in}}{\pgfqpoint{1.298809in}{1.867990in}}{\pgfqpoint{1.292985in}{1.862167in}}%
\pgfpathcurveto{\pgfqpoint{1.287161in}{1.856343in}}{\pgfqpoint{1.283889in}{1.848443in}}{\pgfqpoint{1.283889in}{1.840206in}}%
\pgfpathcurveto{\pgfqpoint{1.283889in}{1.831970in}}{\pgfqpoint{1.287161in}{1.824070in}}{\pgfqpoint{1.292985in}{1.818246in}}%
\pgfpathcurveto{\pgfqpoint{1.298809in}{1.812422in}}{\pgfqpoint{1.306709in}{1.809150in}}{\pgfqpoint{1.314945in}{1.809150in}}%
\pgfpathclose%
\pgfusepath{stroke,fill}%
\end{pgfscope}%
\begin{pgfscope}%
\pgfpathrectangle{\pgfqpoint{0.100000in}{0.212622in}}{\pgfqpoint{3.696000in}{3.696000in}}%
\pgfusepath{clip}%
\pgfsetbuttcap%
\pgfsetroundjoin%
\definecolor{currentfill}{rgb}{0.121569,0.466667,0.705882}%
\pgfsetfillcolor{currentfill}%
\pgfsetfillopacity{0.447220}%
\pgfsetlinewidth{1.003750pt}%
\definecolor{currentstroke}{rgb}{0.121569,0.466667,0.705882}%
\pgfsetstrokecolor{currentstroke}%
\pgfsetstrokeopacity{0.447220}%
\pgfsetdash{}{0pt}%
\pgfpathmoveto{\pgfqpoint{2.515267in}{2.047657in}}%
\pgfpathcurveto{\pgfqpoint{2.523503in}{2.047657in}}{\pgfqpoint{2.531403in}{2.050929in}}{\pgfqpoint{2.537227in}{2.056753in}}%
\pgfpathcurveto{\pgfqpoint{2.543051in}{2.062577in}}{\pgfqpoint{2.546324in}{2.070477in}}{\pgfqpoint{2.546324in}{2.078713in}}%
\pgfpathcurveto{\pgfqpoint{2.546324in}{2.086949in}}{\pgfqpoint{2.543051in}{2.094849in}}{\pgfqpoint{2.537227in}{2.100673in}}%
\pgfpathcurveto{\pgfqpoint{2.531403in}{2.106497in}}{\pgfqpoint{2.523503in}{2.109770in}}{\pgfqpoint{2.515267in}{2.109770in}}%
\pgfpathcurveto{\pgfqpoint{2.507031in}{2.109770in}}{\pgfqpoint{2.499131in}{2.106497in}}{\pgfqpoint{2.493307in}{2.100673in}}%
\pgfpathcurveto{\pgfqpoint{2.487483in}{2.094849in}}{\pgfqpoint{2.484211in}{2.086949in}}{\pgfqpoint{2.484211in}{2.078713in}}%
\pgfpathcurveto{\pgfqpoint{2.484211in}{2.070477in}}{\pgfqpoint{2.487483in}{2.062577in}}{\pgfqpoint{2.493307in}{2.056753in}}%
\pgfpathcurveto{\pgfqpoint{2.499131in}{2.050929in}}{\pgfqpoint{2.507031in}{2.047657in}}{\pgfqpoint{2.515267in}{2.047657in}}%
\pgfpathclose%
\pgfusepath{stroke,fill}%
\end{pgfscope}%
\begin{pgfscope}%
\pgfpathrectangle{\pgfqpoint{0.100000in}{0.212622in}}{\pgfqpoint{3.696000in}{3.696000in}}%
\pgfusepath{clip}%
\pgfsetbuttcap%
\pgfsetroundjoin%
\definecolor{currentfill}{rgb}{0.121569,0.466667,0.705882}%
\pgfsetfillcolor{currentfill}%
\pgfsetfillopacity{0.447987}%
\pgfsetlinewidth{1.003750pt}%
\definecolor{currentstroke}{rgb}{0.121569,0.466667,0.705882}%
\pgfsetstrokecolor{currentstroke}%
\pgfsetstrokeopacity{0.447987}%
\pgfsetdash{}{0pt}%
\pgfpathmoveto{\pgfqpoint{2.522106in}{2.043097in}}%
\pgfpathcurveto{\pgfqpoint{2.530342in}{2.043097in}}{\pgfqpoint{2.538242in}{2.046370in}}{\pgfqpoint{2.544066in}{2.052194in}}%
\pgfpathcurveto{\pgfqpoint{2.549890in}{2.058018in}}{\pgfqpoint{2.553162in}{2.065918in}}{\pgfqpoint{2.553162in}{2.074154in}}%
\pgfpathcurveto{\pgfqpoint{2.553162in}{2.082390in}}{\pgfqpoint{2.549890in}{2.090290in}}{\pgfqpoint{2.544066in}{2.096114in}}%
\pgfpathcurveto{\pgfqpoint{2.538242in}{2.101938in}}{\pgfqpoint{2.530342in}{2.105210in}}{\pgfqpoint{2.522106in}{2.105210in}}%
\pgfpathcurveto{\pgfqpoint{2.513869in}{2.105210in}}{\pgfqpoint{2.505969in}{2.101938in}}{\pgfqpoint{2.500145in}{2.096114in}}%
\pgfpathcurveto{\pgfqpoint{2.494321in}{2.090290in}}{\pgfqpoint{2.491049in}{2.082390in}}{\pgfqpoint{2.491049in}{2.074154in}}%
\pgfpathcurveto{\pgfqpoint{2.491049in}{2.065918in}}{\pgfqpoint{2.494321in}{2.058018in}}{\pgfqpoint{2.500145in}{2.052194in}}%
\pgfpathcurveto{\pgfqpoint{2.505969in}{2.046370in}}{\pgfqpoint{2.513869in}{2.043097in}}{\pgfqpoint{2.522106in}{2.043097in}}%
\pgfpathclose%
\pgfusepath{stroke,fill}%
\end{pgfscope}%
\begin{pgfscope}%
\pgfpathrectangle{\pgfqpoint{0.100000in}{0.212622in}}{\pgfqpoint{3.696000in}{3.696000in}}%
\pgfusepath{clip}%
\pgfsetbuttcap%
\pgfsetroundjoin%
\definecolor{currentfill}{rgb}{0.121569,0.466667,0.705882}%
\pgfsetfillcolor{currentfill}%
\pgfsetfillopacity{0.448227}%
\pgfsetlinewidth{1.003750pt}%
\definecolor{currentstroke}{rgb}{0.121569,0.466667,0.705882}%
\pgfsetstrokecolor{currentstroke}%
\pgfsetstrokeopacity{0.448227}%
\pgfsetdash{}{0pt}%
\pgfpathmoveto{\pgfqpoint{1.312350in}{1.806892in}}%
\pgfpathcurveto{\pgfqpoint{1.320587in}{1.806892in}}{\pgfqpoint{1.328487in}{1.810164in}}{\pgfqpoint{1.334311in}{1.815988in}}%
\pgfpathcurveto{\pgfqpoint{1.340135in}{1.821812in}}{\pgfqpoint{1.343407in}{1.829712in}}{\pgfqpoint{1.343407in}{1.837948in}}%
\pgfpathcurveto{\pgfqpoint{1.343407in}{1.846185in}}{\pgfqpoint{1.340135in}{1.854085in}}{\pgfqpoint{1.334311in}{1.859909in}}%
\pgfpathcurveto{\pgfqpoint{1.328487in}{1.865732in}}{\pgfqpoint{1.320587in}{1.869005in}}{\pgfqpoint{1.312350in}{1.869005in}}%
\pgfpathcurveto{\pgfqpoint{1.304114in}{1.869005in}}{\pgfqpoint{1.296214in}{1.865732in}}{\pgfqpoint{1.290390in}{1.859909in}}%
\pgfpathcurveto{\pgfqpoint{1.284566in}{1.854085in}}{\pgfqpoint{1.281294in}{1.846185in}}{\pgfqpoint{1.281294in}{1.837948in}}%
\pgfpathcurveto{\pgfqpoint{1.281294in}{1.829712in}}{\pgfqpoint{1.284566in}{1.821812in}}{\pgfqpoint{1.290390in}{1.815988in}}%
\pgfpathcurveto{\pgfqpoint{1.296214in}{1.810164in}}{\pgfqpoint{1.304114in}{1.806892in}}{\pgfqpoint{1.312350in}{1.806892in}}%
\pgfpathclose%
\pgfusepath{stroke,fill}%
\end{pgfscope}%
\begin{pgfscope}%
\pgfpathrectangle{\pgfqpoint{0.100000in}{0.212622in}}{\pgfqpoint{3.696000in}{3.696000in}}%
\pgfusepath{clip}%
\pgfsetbuttcap%
\pgfsetroundjoin%
\definecolor{currentfill}{rgb}{0.121569,0.466667,0.705882}%
\pgfsetfillcolor{currentfill}%
\pgfsetfillopacity{0.448635}%
\pgfsetlinewidth{1.003750pt}%
\definecolor{currentstroke}{rgb}{0.121569,0.466667,0.705882}%
\pgfsetstrokecolor{currentstroke}%
\pgfsetstrokeopacity{0.448635}%
\pgfsetdash{}{0pt}%
\pgfpathmoveto{\pgfqpoint{2.526056in}{2.042708in}}%
\pgfpathcurveto{\pgfqpoint{2.534292in}{2.042708in}}{\pgfqpoint{2.542192in}{2.045981in}}{\pgfqpoint{2.548016in}{2.051804in}}%
\pgfpathcurveto{\pgfqpoint{2.553840in}{2.057628in}}{\pgfqpoint{2.557112in}{2.065528in}}{\pgfqpoint{2.557112in}{2.073765in}}%
\pgfpathcurveto{\pgfqpoint{2.557112in}{2.082001in}}{\pgfqpoint{2.553840in}{2.089901in}}{\pgfqpoint{2.548016in}{2.095725in}}%
\pgfpathcurveto{\pgfqpoint{2.542192in}{2.101549in}}{\pgfqpoint{2.534292in}{2.104821in}}{\pgfqpoint{2.526056in}{2.104821in}}%
\pgfpathcurveto{\pgfqpoint{2.517819in}{2.104821in}}{\pgfqpoint{2.509919in}{2.101549in}}{\pgfqpoint{2.504095in}{2.095725in}}%
\pgfpathcurveto{\pgfqpoint{2.498271in}{2.089901in}}{\pgfqpoint{2.494999in}{2.082001in}}{\pgfqpoint{2.494999in}{2.073765in}}%
\pgfpathcurveto{\pgfqpoint{2.494999in}{2.065528in}}{\pgfqpoint{2.498271in}{2.057628in}}{\pgfqpoint{2.504095in}{2.051804in}}%
\pgfpathcurveto{\pgfqpoint{2.509919in}{2.045981in}}{\pgfqpoint{2.517819in}{2.042708in}}{\pgfqpoint{2.526056in}{2.042708in}}%
\pgfpathclose%
\pgfusepath{stroke,fill}%
\end{pgfscope}%
\begin{pgfscope}%
\pgfpathrectangle{\pgfqpoint{0.100000in}{0.212622in}}{\pgfqpoint{3.696000in}{3.696000in}}%
\pgfusepath{clip}%
\pgfsetbuttcap%
\pgfsetroundjoin%
\definecolor{currentfill}{rgb}{0.121569,0.466667,0.705882}%
\pgfsetfillcolor{currentfill}%
\pgfsetfillopacity{0.449149}%
\pgfsetlinewidth{1.003750pt}%
\definecolor{currentstroke}{rgb}{0.121569,0.466667,0.705882}%
\pgfsetstrokecolor{currentstroke}%
\pgfsetstrokeopacity{0.449149}%
\pgfsetdash{}{0pt}%
\pgfpathmoveto{\pgfqpoint{1.308163in}{1.805412in}}%
\pgfpathcurveto{\pgfqpoint{1.316400in}{1.805412in}}{\pgfqpoint{1.324300in}{1.808684in}}{\pgfqpoint{1.330124in}{1.814508in}}%
\pgfpathcurveto{\pgfqpoint{1.335948in}{1.820332in}}{\pgfqpoint{1.339220in}{1.828232in}}{\pgfqpoint{1.339220in}{1.836468in}}%
\pgfpathcurveto{\pgfqpoint{1.339220in}{1.844705in}}{\pgfqpoint{1.335948in}{1.852605in}}{\pgfqpoint{1.330124in}{1.858429in}}%
\pgfpathcurveto{\pgfqpoint{1.324300in}{1.864253in}}{\pgfqpoint{1.316400in}{1.867525in}}{\pgfqpoint{1.308163in}{1.867525in}}%
\pgfpathcurveto{\pgfqpoint{1.299927in}{1.867525in}}{\pgfqpoint{1.292027in}{1.864253in}}{\pgfqpoint{1.286203in}{1.858429in}}%
\pgfpathcurveto{\pgfqpoint{1.280379in}{1.852605in}}{\pgfqpoint{1.277107in}{1.844705in}}{\pgfqpoint{1.277107in}{1.836468in}}%
\pgfpathcurveto{\pgfqpoint{1.277107in}{1.828232in}}{\pgfqpoint{1.280379in}{1.820332in}}{\pgfqpoint{1.286203in}{1.814508in}}%
\pgfpathcurveto{\pgfqpoint{1.292027in}{1.808684in}}{\pgfqpoint{1.299927in}{1.805412in}}{\pgfqpoint{1.308163in}{1.805412in}}%
\pgfpathclose%
\pgfusepath{stroke,fill}%
\end{pgfscope}%
\begin{pgfscope}%
\pgfpathrectangle{\pgfqpoint{0.100000in}{0.212622in}}{\pgfqpoint{3.696000in}{3.696000in}}%
\pgfusepath{clip}%
\pgfsetbuttcap%
\pgfsetroundjoin%
\definecolor{currentfill}{rgb}{0.121569,0.466667,0.705882}%
\pgfsetfillcolor{currentfill}%
\pgfsetfillopacity{0.449179}%
\pgfsetlinewidth{1.003750pt}%
\definecolor{currentstroke}{rgb}{0.121569,0.466667,0.705882}%
\pgfsetstrokecolor{currentstroke}%
\pgfsetstrokeopacity{0.449179}%
\pgfsetdash{}{0pt}%
\pgfpathmoveto{\pgfqpoint{2.531637in}{2.039052in}}%
\pgfpathcurveto{\pgfqpoint{2.539874in}{2.039052in}}{\pgfqpoint{2.547774in}{2.042324in}}{\pgfqpoint{2.553598in}{2.048148in}}%
\pgfpathcurveto{\pgfqpoint{2.559422in}{2.053972in}}{\pgfqpoint{2.562694in}{2.061872in}}{\pgfqpoint{2.562694in}{2.070108in}}%
\pgfpathcurveto{\pgfqpoint{2.562694in}{2.078345in}}{\pgfqpoint{2.559422in}{2.086245in}}{\pgfqpoint{2.553598in}{2.092069in}}%
\pgfpathcurveto{\pgfqpoint{2.547774in}{2.097893in}}{\pgfqpoint{2.539874in}{2.101165in}}{\pgfqpoint{2.531637in}{2.101165in}}%
\pgfpathcurveto{\pgfqpoint{2.523401in}{2.101165in}}{\pgfqpoint{2.515501in}{2.097893in}}{\pgfqpoint{2.509677in}{2.092069in}}%
\pgfpathcurveto{\pgfqpoint{2.503853in}{2.086245in}}{\pgfqpoint{2.500581in}{2.078345in}}{\pgfqpoint{2.500581in}{2.070108in}}%
\pgfpathcurveto{\pgfqpoint{2.500581in}{2.061872in}}{\pgfqpoint{2.503853in}{2.053972in}}{\pgfqpoint{2.509677in}{2.048148in}}%
\pgfpathcurveto{\pgfqpoint{2.515501in}{2.042324in}}{\pgfqpoint{2.523401in}{2.039052in}}{\pgfqpoint{2.531637in}{2.039052in}}%
\pgfpathclose%
\pgfusepath{stroke,fill}%
\end{pgfscope}%
\begin{pgfscope}%
\pgfpathrectangle{\pgfqpoint{0.100000in}{0.212622in}}{\pgfqpoint{3.696000in}{3.696000in}}%
\pgfusepath{clip}%
\pgfsetbuttcap%
\pgfsetroundjoin%
\definecolor{currentfill}{rgb}{0.121569,0.466667,0.705882}%
\pgfsetfillcolor{currentfill}%
\pgfsetfillopacity{0.449652}%
\pgfsetlinewidth{1.003750pt}%
\definecolor{currentstroke}{rgb}{0.121569,0.466667,0.705882}%
\pgfsetstrokecolor{currentstroke}%
\pgfsetstrokeopacity{0.449652}%
\pgfsetdash{}{0pt}%
\pgfpathmoveto{\pgfqpoint{2.534974in}{2.039138in}}%
\pgfpathcurveto{\pgfqpoint{2.543210in}{2.039138in}}{\pgfqpoint{2.551111in}{2.042411in}}{\pgfqpoint{2.556934in}{2.048235in}}%
\pgfpathcurveto{\pgfqpoint{2.562758in}{2.054059in}}{\pgfqpoint{2.566031in}{2.061959in}}{\pgfqpoint{2.566031in}{2.070195in}}%
\pgfpathcurveto{\pgfqpoint{2.566031in}{2.078431in}}{\pgfqpoint{2.562758in}{2.086331in}}{\pgfqpoint{2.556934in}{2.092155in}}%
\pgfpathcurveto{\pgfqpoint{2.551111in}{2.097979in}}{\pgfqpoint{2.543210in}{2.101251in}}{\pgfqpoint{2.534974in}{2.101251in}}%
\pgfpathcurveto{\pgfqpoint{2.526738in}{2.101251in}}{\pgfqpoint{2.518838in}{2.097979in}}{\pgfqpoint{2.513014in}{2.092155in}}%
\pgfpathcurveto{\pgfqpoint{2.507190in}{2.086331in}}{\pgfqpoint{2.503918in}{2.078431in}}{\pgfqpoint{2.503918in}{2.070195in}}%
\pgfpathcurveto{\pgfqpoint{2.503918in}{2.061959in}}{\pgfqpoint{2.507190in}{2.054059in}}{\pgfqpoint{2.513014in}{2.048235in}}%
\pgfpathcurveto{\pgfqpoint{2.518838in}{2.042411in}}{\pgfqpoint{2.526738in}{2.039138in}}{\pgfqpoint{2.534974in}{2.039138in}}%
\pgfpathclose%
\pgfusepath{stroke,fill}%
\end{pgfscope}%
\begin{pgfscope}%
\pgfpathrectangle{\pgfqpoint{0.100000in}{0.212622in}}{\pgfqpoint{3.696000in}{3.696000in}}%
\pgfusepath{clip}%
\pgfsetbuttcap%
\pgfsetroundjoin%
\definecolor{currentfill}{rgb}{0.121569,0.466667,0.705882}%
\pgfsetfillcolor{currentfill}%
\pgfsetfillopacity{0.450037}%
\pgfsetlinewidth{1.003750pt}%
\definecolor{currentstroke}{rgb}{0.121569,0.466667,0.705882}%
\pgfsetstrokecolor{currentstroke}%
\pgfsetstrokeopacity{0.450037}%
\pgfsetdash{}{0pt}%
\pgfpathmoveto{\pgfqpoint{2.539374in}{2.036426in}}%
\pgfpathcurveto{\pgfqpoint{2.547611in}{2.036426in}}{\pgfqpoint{2.555511in}{2.039699in}}{\pgfqpoint{2.561335in}{2.045523in}}%
\pgfpathcurveto{\pgfqpoint{2.567159in}{2.051346in}}{\pgfqpoint{2.570431in}{2.059247in}}{\pgfqpoint{2.570431in}{2.067483in}}%
\pgfpathcurveto{\pgfqpoint{2.570431in}{2.075719in}}{\pgfqpoint{2.567159in}{2.083619in}}{\pgfqpoint{2.561335in}{2.089443in}}%
\pgfpathcurveto{\pgfqpoint{2.555511in}{2.095267in}}{\pgfqpoint{2.547611in}{2.098539in}}{\pgfqpoint{2.539374in}{2.098539in}}%
\pgfpathcurveto{\pgfqpoint{2.531138in}{2.098539in}}{\pgfqpoint{2.523238in}{2.095267in}}{\pgfqpoint{2.517414in}{2.089443in}}%
\pgfpathcurveto{\pgfqpoint{2.511590in}{2.083619in}}{\pgfqpoint{2.508318in}{2.075719in}}{\pgfqpoint{2.508318in}{2.067483in}}%
\pgfpathcurveto{\pgfqpoint{2.508318in}{2.059247in}}{\pgfqpoint{2.511590in}{2.051346in}}{\pgfqpoint{2.517414in}{2.045523in}}%
\pgfpathcurveto{\pgfqpoint{2.523238in}{2.039699in}}{\pgfqpoint{2.531138in}{2.036426in}}{\pgfqpoint{2.539374in}{2.036426in}}%
\pgfpathclose%
\pgfusepath{stroke,fill}%
\end{pgfscope}%
\begin{pgfscope}%
\pgfpathrectangle{\pgfqpoint{0.100000in}{0.212622in}}{\pgfqpoint{3.696000in}{3.696000in}}%
\pgfusepath{clip}%
\pgfsetbuttcap%
\pgfsetroundjoin%
\definecolor{currentfill}{rgb}{0.121569,0.466667,0.705882}%
\pgfsetfillcolor{currentfill}%
\pgfsetfillopacity{0.450522}%
\pgfsetlinewidth{1.003750pt}%
\definecolor{currentstroke}{rgb}{0.121569,0.466667,0.705882}%
\pgfsetstrokecolor{currentstroke}%
\pgfsetstrokeopacity{0.450522}%
\pgfsetdash{}{0pt}%
\pgfpathmoveto{\pgfqpoint{2.541823in}{2.036926in}}%
\pgfpathcurveto{\pgfqpoint{2.550060in}{2.036926in}}{\pgfqpoint{2.557960in}{2.040198in}}{\pgfqpoint{2.563784in}{2.046022in}}%
\pgfpathcurveto{\pgfqpoint{2.569607in}{2.051846in}}{\pgfqpoint{2.572880in}{2.059746in}}{\pgfqpoint{2.572880in}{2.067982in}}%
\pgfpathcurveto{\pgfqpoint{2.572880in}{2.076218in}}{\pgfqpoint{2.569607in}{2.084118in}}{\pgfqpoint{2.563784in}{2.089942in}}%
\pgfpathcurveto{\pgfqpoint{2.557960in}{2.095766in}}{\pgfqpoint{2.550060in}{2.099039in}}{\pgfqpoint{2.541823in}{2.099039in}}%
\pgfpathcurveto{\pgfqpoint{2.533587in}{2.099039in}}{\pgfqpoint{2.525687in}{2.095766in}}{\pgfqpoint{2.519863in}{2.089942in}}%
\pgfpathcurveto{\pgfqpoint{2.514039in}{2.084118in}}{\pgfqpoint{2.510767in}{2.076218in}}{\pgfqpoint{2.510767in}{2.067982in}}%
\pgfpathcurveto{\pgfqpoint{2.510767in}{2.059746in}}{\pgfqpoint{2.514039in}{2.051846in}}{\pgfqpoint{2.519863in}{2.046022in}}%
\pgfpathcurveto{\pgfqpoint{2.525687in}{2.040198in}}{\pgfqpoint{2.533587in}{2.036926in}}{\pgfqpoint{2.541823in}{2.036926in}}%
\pgfpathclose%
\pgfusepath{stroke,fill}%
\end{pgfscope}%
\begin{pgfscope}%
\pgfpathrectangle{\pgfqpoint{0.100000in}{0.212622in}}{\pgfqpoint{3.696000in}{3.696000in}}%
\pgfusepath{clip}%
\pgfsetbuttcap%
\pgfsetroundjoin%
\definecolor{currentfill}{rgb}{0.121569,0.466667,0.705882}%
\pgfsetfillcolor{currentfill}%
\pgfsetfillopacity{0.450671}%
\pgfsetlinewidth{1.003750pt}%
\definecolor{currentstroke}{rgb}{0.121569,0.466667,0.705882}%
\pgfsetstrokecolor{currentstroke}%
\pgfsetstrokeopacity{0.450671}%
\pgfsetdash{}{0pt}%
\pgfpathmoveto{\pgfqpoint{1.306093in}{1.807104in}}%
\pgfpathcurveto{\pgfqpoint{1.314329in}{1.807104in}}{\pgfqpoint{1.322229in}{1.810377in}}{\pgfqpoint{1.328053in}{1.816200in}}%
\pgfpathcurveto{\pgfqpoint{1.333877in}{1.822024in}}{\pgfqpoint{1.337149in}{1.829924in}}{\pgfqpoint{1.337149in}{1.838161in}}%
\pgfpathcurveto{\pgfqpoint{1.337149in}{1.846397in}}{\pgfqpoint{1.333877in}{1.854297in}}{\pgfqpoint{1.328053in}{1.860121in}}%
\pgfpathcurveto{\pgfqpoint{1.322229in}{1.865945in}}{\pgfqpoint{1.314329in}{1.869217in}}{\pgfqpoint{1.306093in}{1.869217in}}%
\pgfpathcurveto{\pgfqpoint{1.297857in}{1.869217in}}{\pgfqpoint{1.289956in}{1.865945in}}{\pgfqpoint{1.284133in}{1.860121in}}%
\pgfpathcurveto{\pgfqpoint{1.278309in}{1.854297in}}{\pgfqpoint{1.275036in}{1.846397in}}{\pgfqpoint{1.275036in}{1.838161in}}%
\pgfpathcurveto{\pgfqpoint{1.275036in}{1.829924in}}{\pgfqpoint{1.278309in}{1.822024in}}{\pgfqpoint{1.284133in}{1.816200in}}%
\pgfpathcurveto{\pgfqpoint{1.289956in}{1.810377in}}{\pgfqpoint{1.297857in}{1.807104in}}{\pgfqpoint{1.306093in}{1.807104in}}%
\pgfpathclose%
\pgfusepath{stroke,fill}%
\end{pgfscope}%
\begin{pgfscope}%
\pgfpathrectangle{\pgfqpoint{0.100000in}{0.212622in}}{\pgfqpoint{3.696000in}{3.696000in}}%
\pgfusepath{clip}%
\pgfsetbuttcap%
\pgfsetroundjoin%
\definecolor{currentfill}{rgb}{0.121569,0.466667,0.705882}%
\pgfsetfillcolor{currentfill}%
\pgfsetfillopacity{0.450850}%
\pgfsetlinewidth{1.003750pt}%
\definecolor{currentstroke}{rgb}{0.121569,0.466667,0.705882}%
\pgfsetstrokecolor{currentstroke}%
\pgfsetstrokeopacity{0.450850}%
\pgfsetdash{}{0pt}%
\pgfpathmoveto{\pgfqpoint{1.299509in}{1.801335in}}%
\pgfpathcurveto{\pgfqpoint{1.307746in}{1.801335in}}{\pgfqpoint{1.315646in}{1.804607in}}{\pgfqpoint{1.321470in}{1.810431in}}%
\pgfpathcurveto{\pgfqpoint{1.327293in}{1.816255in}}{\pgfqpoint{1.330566in}{1.824155in}}{\pgfqpoint{1.330566in}{1.832391in}}%
\pgfpathcurveto{\pgfqpoint{1.330566in}{1.840627in}}{\pgfqpoint{1.327293in}{1.848527in}}{\pgfqpoint{1.321470in}{1.854351in}}%
\pgfpathcurveto{\pgfqpoint{1.315646in}{1.860175in}}{\pgfqpoint{1.307746in}{1.863448in}}{\pgfqpoint{1.299509in}{1.863448in}}%
\pgfpathcurveto{\pgfqpoint{1.291273in}{1.863448in}}{\pgfqpoint{1.283373in}{1.860175in}}{\pgfqpoint{1.277549in}{1.854351in}}%
\pgfpathcurveto{\pgfqpoint{1.271725in}{1.848527in}}{\pgfqpoint{1.268453in}{1.840627in}}{\pgfqpoint{1.268453in}{1.832391in}}%
\pgfpathcurveto{\pgfqpoint{1.268453in}{1.824155in}}{\pgfqpoint{1.271725in}{1.816255in}}{\pgfqpoint{1.277549in}{1.810431in}}%
\pgfpathcurveto{\pgfqpoint{1.283373in}{1.804607in}}{\pgfqpoint{1.291273in}{1.801335in}}{\pgfqpoint{1.299509in}{1.801335in}}%
\pgfpathclose%
\pgfusepath{stroke,fill}%
\end{pgfscope}%
\begin{pgfscope}%
\pgfpathrectangle{\pgfqpoint{0.100000in}{0.212622in}}{\pgfqpoint{3.696000in}{3.696000in}}%
\pgfusepath{clip}%
\pgfsetbuttcap%
\pgfsetroundjoin%
\definecolor{currentfill}{rgb}{0.121569,0.466667,0.705882}%
\pgfsetfillcolor{currentfill}%
\pgfsetfillopacity{0.451029}%
\pgfsetlinewidth{1.003750pt}%
\definecolor{currentstroke}{rgb}{0.121569,0.466667,0.705882}%
\pgfsetstrokecolor{currentstroke}%
\pgfsetstrokeopacity{0.451029}%
\pgfsetdash{}{0pt}%
\pgfpathmoveto{\pgfqpoint{2.544821in}{2.035390in}}%
\pgfpathcurveto{\pgfqpoint{2.553057in}{2.035390in}}{\pgfqpoint{2.560957in}{2.038663in}}{\pgfqpoint{2.566781in}{2.044486in}}%
\pgfpathcurveto{\pgfqpoint{2.572605in}{2.050310in}}{\pgfqpoint{2.575877in}{2.058210in}}{\pgfqpoint{2.575877in}{2.066447in}}%
\pgfpathcurveto{\pgfqpoint{2.575877in}{2.074683in}}{\pgfqpoint{2.572605in}{2.082583in}}{\pgfqpoint{2.566781in}{2.088407in}}%
\pgfpathcurveto{\pgfqpoint{2.560957in}{2.094231in}}{\pgfqpoint{2.553057in}{2.097503in}}{\pgfqpoint{2.544821in}{2.097503in}}%
\pgfpathcurveto{\pgfqpoint{2.536584in}{2.097503in}}{\pgfqpoint{2.528684in}{2.094231in}}{\pgfqpoint{2.522860in}{2.088407in}}%
\pgfpathcurveto{\pgfqpoint{2.517036in}{2.082583in}}{\pgfqpoint{2.513764in}{2.074683in}}{\pgfqpoint{2.513764in}{2.066447in}}%
\pgfpathcurveto{\pgfqpoint{2.513764in}{2.058210in}}{\pgfqpoint{2.517036in}{2.050310in}}{\pgfqpoint{2.522860in}{2.044486in}}%
\pgfpathcurveto{\pgfqpoint{2.528684in}{2.038663in}}{\pgfqpoint{2.536584in}{2.035390in}}{\pgfqpoint{2.544821in}{2.035390in}}%
\pgfpathclose%
\pgfusepath{stroke,fill}%
\end{pgfscope}%
\begin{pgfscope}%
\pgfpathrectangle{\pgfqpoint{0.100000in}{0.212622in}}{\pgfqpoint{3.696000in}{3.696000in}}%
\pgfusepath{clip}%
\pgfsetbuttcap%
\pgfsetroundjoin%
\definecolor{currentfill}{rgb}{0.121569,0.466667,0.705882}%
\pgfsetfillcolor{currentfill}%
\pgfsetfillopacity{0.451110}%
\pgfsetlinewidth{1.003750pt}%
\definecolor{currentstroke}{rgb}{0.121569,0.466667,0.705882}%
\pgfsetstrokecolor{currentstroke}%
\pgfsetstrokeopacity{0.451110}%
\pgfsetdash{}{0pt}%
\pgfpathmoveto{\pgfqpoint{1.302461in}{1.807123in}}%
\pgfpathcurveto{\pgfqpoint{1.310697in}{1.807123in}}{\pgfqpoint{1.318597in}{1.810395in}}{\pgfqpoint{1.324421in}{1.816219in}}%
\pgfpathcurveto{\pgfqpoint{1.330245in}{1.822043in}}{\pgfqpoint{1.333517in}{1.829943in}}{\pgfqpoint{1.333517in}{1.838179in}}%
\pgfpathcurveto{\pgfqpoint{1.333517in}{1.846416in}}{\pgfqpoint{1.330245in}{1.854316in}}{\pgfqpoint{1.324421in}{1.860140in}}%
\pgfpathcurveto{\pgfqpoint{1.318597in}{1.865964in}}{\pgfqpoint{1.310697in}{1.869236in}}{\pgfqpoint{1.302461in}{1.869236in}}%
\pgfpathcurveto{\pgfqpoint{1.294225in}{1.869236in}}{\pgfqpoint{1.286325in}{1.865964in}}{\pgfqpoint{1.280501in}{1.860140in}}%
\pgfpathcurveto{\pgfqpoint{1.274677in}{1.854316in}}{\pgfqpoint{1.271404in}{1.846416in}}{\pgfqpoint{1.271404in}{1.838179in}}%
\pgfpathcurveto{\pgfqpoint{1.271404in}{1.829943in}}{\pgfqpoint{1.274677in}{1.822043in}}{\pgfqpoint{1.280501in}{1.816219in}}%
\pgfpathcurveto{\pgfqpoint{1.286325in}{1.810395in}}{\pgfqpoint{1.294225in}{1.807123in}}{\pgfqpoint{1.302461in}{1.807123in}}%
\pgfpathclose%
\pgfusepath{stroke,fill}%
\end{pgfscope}%
\begin{pgfscope}%
\pgfpathrectangle{\pgfqpoint{0.100000in}{0.212622in}}{\pgfqpoint{3.696000in}{3.696000in}}%
\pgfusepath{clip}%
\pgfsetbuttcap%
\pgfsetroundjoin%
\definecolor{currentfill}{rgb}{0.121569,0.466667,0.705882}%
\pgfsetfillcolor{currentfill}%
\pgfsetfillopacity{0.452044}%
\pgfsetlinewidth{1.003750pt}%
\definecolor{currentstroke}{rgb}{0.121569,0.466667,0.705882}%
\pgfsetstrokecolor{currentstroke}%
\pgfsetstrokeopacity{0.452044}%
\pgfsetdash{}{0pt}%
\pgfpathmoveto{\pgfqpoint{2.549967in}{2.035227in}}%
\pgfpathcurveto{\pgfqpoint{2.558203in}{2.035227in}}{\pgfqpoint{2.566103in}{2.038500in}}{\pgfqpoint{2.571927in}{2.044324in}}%
\pgfpathcurveto{\pgfqpoint{2.577751in}{2.050147in}}{\pgfqpoint{2.581023in}{2.058047in}}{\pgfqpoint{2.581023in}{2.066284in}}%
\pgfpathcurveto{\pgfqpoint{2.581023in}{2.074520in}}{\pgfqpoint{2.577751in}{2.082420in}}{\pgfqpoint{2.571927in}{2.088244in}}%
\pgfpathcurveto{\pgfqpoint{2.566103in}{2.094068in}}{\pgfqpoint{2.558203in}{2.097340in}}{\pgfqpoint{2.549967in}{2.097340in}}%
\pgfpathcurveto{\pgfqpoint{2.541731in}{2.097340in}}{\pgfqpoint{2.533830in}{2.094068in}}{\pgfqpoint{2.528007in}{2.088244in}}%
\pgfpathcurveto{\pgfqpoint{2.522183in}{2.082420in}}{\pgfqpoint{2.518910in}{2.074520in}}{\pgfqpoint{2.518910in}{2.066284in}}%
\pgfpathcurveto{\pgfqpoint{2.518910in}{2.058047in}}{\pgfqpoint{2.522183in}{2.050147in}}{\pgfqpoint{2.528007in}{2.044324in}}%
\pgfpathcurveto{\pgfqpoint{2.533830in}{2.038500in}}{\pgfqpoint{2.541731in}{2.035227in}}{\pgfqpoint{2.549967in}{2.035227in}}%
\pgfpathclose%
\pgfusepath{stroke,fill}%
\end{pgfscope}%
\begin{pgfscope}%
\pgfpathrectangle{\pgfqpoint{0.100000in}{0.212622in}}{\pgfqpoint{3.696000in}{3.696000in}}%
\pgfusepath{clip}%
\pgfsetbuttcap%
\pgfsetroundjoin%
\definecolor{currentfill}{rgb}{0.121569,0.466667,0.705882}%
\pgfsetfillcolor{currentfill}%
\pgfsetfillopacity{0.452428}%
\pgfsetlinewidth{1.003750pt}%
\definecolor{currentstroke}{rgb}{0.121569,0.466667,0.705882}%
\pgfsetstrokecolor{currentstroke}%
\pgfsetstrokeopacity{0.452428}%
\pgfsetdash{}{0pt}%
\pgfpathmoveto{\pgfqpoint{1.297559in}{1.800239in}}%
\pgfpathcurveto{\pgfqpoint{1.305796in}{1.800239in}}{\pgfqpoint{1.313696in}{1.803511in}}{\pgfqpoint{1.319520in}{1.809335in}}%
\pgfpathcurveto{\pgfqpoint{1.325344in}{1.815159in}}{\pgfqpoint{1.328616in}{1.823059in}}{\pgfqpoint{1.328616in}{1.831295in}}%
\pgfpathcurveto{\pgfqpoint{1.328616in}{1.839532in}}{\pgfqpoint{1.325344in}{1.847432in}}{\pgfqpoint{1.319520in}{1.853256in}}%
\pgfpathcurveto{\pgfqpoint{1.313696in}{1.859080in}}{\pgfqpoint{1.305796in}{1.862352in}}{\pgfqpoint{1.297559in}{1.862352in}}%
\pgfpathcurveto{\pgfqpoint{1.289323in}{1.862352in}}{\pgfqpoint{1.281423in}{1.859080in}}{\pgfqpoint{1.275599in}{1.853256in}}%
\pgfpathcurveto{\pgfqpoint{1.269775in}{1.847432in}}{\pgfqpoint{1.266503in}{1.839532in}}{\pgfqpoint{1.266503in}{1.831295in}}%
\pgfpathcurveto{\pgfqpoint{1.266503in}{1.823059in}}{\pgfqpoint{1.269775in}{1.815159in}}{\pgfqpoint{1.275599in}{1.809335in}}%
\pgfpathcurveto{\pgfqpoint{1.281423in}{1.803511in}}{\pgfqpoint{1.289323in}{1.800239in}}{\pgfqpoint{1.297559in}{1.800239in}}%
\pgfpathclose%
\pgfusepath{stroke,fill}%
\end{pgfscope}%
\begin{pgfscope}%
\pgfpathrectangle{\pgfqpoint{0.100000in}{0.212622in}}{\pgfqpoint{3.696000in}{3.696000in}}%
\pgfusepath{clip}%
\pgfsetbuttcap%
\pgfsetroundjoin%
\definecolor{currentfill}{rgb}{0.121569,0.466667,0.705882}%
\pgfsetfillcolor{currentfill}%
\pgfsetfillopacity{0.453444}%
\pgfsetlinewidth{1.003750pt}%
\definecolor{currentstroke}{rgb}{0.121569,0.466667,0.705882}%
\pgfsetstrokecolor{currentstroke}%
\pgfsetstrokeopacity{0.453444}%
\pgfsetdash{}{0pt}%
\pgfpathmoveto{\pgfqpoint{1.293702in}{1.797930in}}%
\pgfpathcurveto{\pgfqpoint{1.301938in}{1.797930in}}{\pgfqpoint{1.309838in}{1.801203in}}{\pgfqpoint{1.315662in}{1.807027in}}%
\pgfpathcurveto{\pgfqpoint{1.321486in}{1.812851in}}{\pgfqpoint{1.324758in}{1.820751in}}{\pgfqpoint{1.324758in}{1.828987in}}%
\pgfpathcurveto{\pgfqpoint{1.324758in}{1.837223in}}{\pgfqpoint{1.321486in}{1.845123in}}{\pgfqpoint{1.315662in}{1.850947in}}%
\pgfpathcurveto{\pgfqpoint{1.309838in}{1.856771in}}{\pgfqpoint{1.301938in}{1.860043in}}{\pgfqpoint{1.293702in}{1.860043in}}%
\pgfpathcurveto{\pgfqpoint{1.285466in}{1.860043in}}{\pgfqpoint{1.277566in}{1.856771in}}{\pgfqpoint{1.271742in}{1.850947in}}%
\pgfpathcurveto{\pgfqpoint{1.265918in}{1.845123in}}{\pgfqpoint{1.262645in}{1.837223in}}{\pgfqpoint{1.262645in}{1.828987in}}%
\pgfpathcurveto{\pgfqpoint{1.262645in}{1.820751in}}{\pgfqpoint{1.265918in}{1.812851in}}{\pgfqpoint{1.271742in}{1.807027in}}%
\pgfpathcurveto{\pgfqpoint{1.277566in}{1.801203in}}{\pgfqpoint{1.285466in}{1.797930in}}{\pgfqpoint{1.293702in}{1.797930in}}%
\pgfpathclose%
\pgfusepath{stroke,fill}%
\end{pgfscope}%
\begin{pgfscope}%
\pgfpathrectangle{\pgfqpoint{0.100000in}{0.212622in}}{\pgfqpoint{3.696000in}{3.696000in}}%
\pgfusepath{clip}%
\pgfsetbuttcap%
\pgfsetroundjoin%
\definecolor{currentfill}{rgb}{0.121569,0.466667,0.705882}%
\pgfsetfillcolor{currentfill}%
\pgfsetfillopacity{0.453479}%
\pgfsetlinewidth{1.003750pt}%
\definecolor{currentstroke}{rgb}{0.121569,0.466667,0.705882}%
\pgfsetstrokecolor{currentstroke}%
\pgfsetstrokeopacity{0.453479}%
\pgfsetdash{}{0pt}%
\pgfpathmoveto{\pgfqpoint{2.555517in}{2.037235in}}%
\pgfpathcurveto{\pgfqpoint{2.563753in}{2.037235in}}{\pgfqpoint{2.571653in}{2.040508in}}{\pgfqpoint{2.577477in}{2.046332in}}%
\pgfpathcurveto{\pgfqpoint{2.583301in}{2.052156in}}{\pgfqpoint{2.586573in}{2.060056in}}{\pgfqpoint{2.586573in}{2.068292in}}%
\pgfpathcurveto{\pgfqpoint{2.586573in}{2.076528in}}{\pgfqpoint{2.583301in}{2.084428in}}{\pgfqpoint{2.577477in}{2.090252in}}%
\pgfpathcurveto{\pgfqpoint{2.571653in}{2.096076in}}{\pgfqpoint{2.563753in}{2.099348in}}{\pgfqpoint{2.555517in}{2.099348in}}%
\pgfpathcurveto{\pgfqpoint{2.547280in}{2.099348in}}{\pgfqpoint{2.539380in}{2.096076in}}{\pgfqpoint{2.533556in}{2.090252in}}%
\pgfpathcurveto{\pgfqpoint{2.527732in}{2.084428in}}{\pgfqpoint{2.524460in}{2.076528in}}{\pgfqpoint{2.524460in}{2.068292in}}%
\pgfpathcurveto{\pgfqpoint{2.524460in}{2.060056in}}{\pgfqpoint{2.527732in}{2.052156in}}{\pgfqpoint{2.533556in}{2.046332in}}%
\pgfpathcurveto{\pgfqpoint{2.539380in}{2.040508in}}{\pgfqpoint{2.547280in}{2.037235in}}{\pgfqpoint{2.555517in}{2.037235in}}%
\pgfpathclose%
\pgfusepath{stroke,fill}%
\end{pgfscope}%
\begin{pgfscope}%
\pgfpathrectangle{\pgfqpoint{0.100000in}{0.212622in}}{\pgfqpoint{3.696000in}{3.696000in}}%
\pgfusepath{clip}%
\pgfsetbuttcap%
\pgfsetroundjoin%
\definecolor{currentfill}{rgb}{0.121569,0.466667,0.705882}%
\pgfsetfillcolor{currentfill}%
\pgfsetfillopacity{0.454506}%
\pgfsetlinewidth{1.003750pt}%
\definecolor{currentstroke}{rgb}{0.121569,0.466667,0.705882}%
\pgfsetstrokecolor{currentstroke}%
\pgfsetstrokeopacity{0.454506}%
\pgfsetdash{}{0pt}%
\pgfpathmoveto{\pgfqpoint{2.561528in}{2.035213in}}%
\pgfpathcurveto{\pgfqpoint{2.569764in}{2.035213in}}{\pgfqpoint{2.577664in}{2.038485in}}{\pgfqpoint{2.583488in}{2.044309in}}%
\pgfpathcurveto{\pgfqpoint{2.589312in}{2.050133in}}{\pgfqpoint{2.592584in}{2.058033in}}{\pgfqpoint{2.592584in}{2.066269in}}%
\pgfpathcurveto{\pgfqpoint{2.592584in}{2.074506in}}{\pgfqpoint{2.589312in}{2.082406in}}{\pgfqpoint{2.583488in}{2.088230in}}%
\pgfpathcurveto{\pgfqpoint{2.577664in}{2.094054in}}{\pgfqpoint{2.569764in}{2.097326in}}{\pgfqpoint{2.561528in}{2.097326in}}%
\pgfpathcurveto{\pgfqpoint{2.553291in}{2.097326in}}{\pgfqpoint{2.545391in}{2.094054in}}{\pgfqpoint{2.539567in}{2.088230in}}%
\pgfpathcurveto{\pgfqpoint{2.533743in}{2.082406in}}{\pgfqpoint{2.530471in}{2.074506in}}{\pgfqpoint{2.530471in}{2.066269in}}%
\pgfpathcurveto{\pgfqpoint{2.530471in}{2.058033in}}{\pgfqpoint{2.533743in}{2.050133in}}{\pgfqpoint{2.539567in}{2.044309in}}%
\pgfpathcurveto{\pgfqpoint{2.545391in}{2.038485in}}{\pgfqpoint{2.553291in}{2.035213in}}{\pgfqpoint{2.561528in}{2.035213in}}%
\pgfpathclose%
\pgfusepath{stroke,fill}%
\end{pgfscope}%
\begin{pgfscope}%
\pgfpathrectangle{\pgfqpoint{0.100000in}{0.212622in}}{\pgfqpoint{3.696000in}{3.696000in}}%
\pgfusepath{clip}%
\pgfsetbuttcap%
\pgfsetroundjoin%
\definecolor{currentfill}{rgb}{0.121569,0.466667,0.705882}%
\pgfsetfillcolor{currentfill}%
\pgfsetfillopacity{0.454630}%
\pgfsetlinewidth{1.003750pt}%
\definecolor{currentstroke}{rgb}{0.121569,0.466667,0.705882}%
\pgfsetstrokecolor{currentstroke}%
\pgfsetstrokeopacity{0.454630}%
\pgfsetdash{}{0pt}%
\pgfpathmoveto{\pgfqpoint{1.292516in}{1.797803in}}%
\pgfpathcurveto{\pgfqpoint{1.300752in}{1.797803in}}{\pgfqpoint{1.308652in}{1.801075in}}{\pgfqpoint{1.314476in}{1.806899in}}%
\pgfpathcurveto{\pgfqpoint{1.320300in}{1.812723in}}{\pgfqpoint{1.323572in}{1.820623in}}{\pgfqpoint{1.323572in}{1.828859in}}%
\pgfpathcurveto{\pgfqpoint{1.323572in}{1.837096in}}{\pgfqpoint{1.320300in}{1.844996in}}{\pgfqpoint{1.314476in}{1.850820in}}%
\pgfpathcurveto{\pgfqpoint{1.308652in}{1.856644in}}{\pgfqpoint{1.300752in}{1.859916in}}{\pgfqpoint{1.292516in}{1.859916in}}%
\pgfpathcurveto{\pgfqpoint{1.284280in}{1.859916in}}{\pgfqpoint{1.276380in}{1.856644in}}{\pgfqpoint{1.270556in}{1.850820in}}%
\pgfpathcurveto{\pgfqpoint{1.264732in}{1.844996in}}{\pgfqpoint{1.261459in}{1.837096in}}{\pgfqpoint{1.261459in}{1.828859in}}%
\pgfpathcurveto{\pgfqpoint{1.261459in}{1.820623in}}{\pgfqpoint{1.264732in}{1.812723in}}{\pgfqpoint{1.270556in}{1.806899in}}%
\pgfpathcurveto{\pgfqpoint{1.276380in}{1.801075in}}{\pgfqpoint{1.284280in}{1.797803in}}{\pgfqpoint{1.292516in}{1.797803in}}%
\pgfpathclose%
\pgfusepath{stroke,fill}%
\end{pgfscope}%
\begin{pgfscope}%
\pgfpathrectangle{\pgfqpoint{0.100000in}{0.212622in}}{\pgfqpoint{3.696000in}{3.696000in}}%
\pgfusepath{clip}%
\pgfsetbuttcap%
\pgfsetroundjoin%
\definecolor{currentfill}{rgb}{0.121569,0.466667,0.705882}%
\pgfsetfillcolor{currentfill}%
\pgfsetfillopacity{0.455501}%
\pgfsetlinewidth{1.003750pt}%
\definecolor{currentstroke}{rgb}{0.121569,0.466667,0.705882}%
\pgfsetstrokecolor{currentstroke}%
\pgfsetstrokeopacity{0.455501}%
\pgfsetdash{}{0pt}%
\pgfpathmoveto{\pgfqpoint{1.289357in}{1.788955in}}%
\pgfpathcurveto{\pgfqpoint{1.297593in}{1.788955in}}{\pgfqpoint{1.305493in}{1.792228in}}{\pgfqpoint{1.311317in}{1.798052in}}%
\pgfpathcurveto{\pgfqpoint{1.317141in}{1.803875in}}{\pgfqpoint{1.320413in}{1.811776in}}{\pgfqpoint{1.320413in}{1.820012in}}%
\pgfpathcurveto{\pgfqpoint{1.320413in}{1.828248in}}{\pgfqpoint{1.317141in}{1.836148in}}{\pgfqpoint{1.311317in}{1.841972in}}%
\pgfpathcurveto{\pgfqpoint{1.305493in}{1.847796in}}{\pgfqpoint{1.297593in}{1.851068in}}{\pgfqpoint{1.289357in}{1.851068in}}%
\pgfpathcurveto{\pgfqpoint{1.281121in}{1.851068in}}{\pgfqpoint{1.273220in}{1.847796in}}{\pgfqpoint{1.267397in}{1.841972in}}%
\pgfpathcurveto{\pgfqpoint{1.261573in}{1.836148in}}{\pgfqpoint{1.258300in}{1.828248in}}{\pgfqpoint{1.258300in}{1.820012in}}%
\pgfpathcurveto{\pgfqpoint{1.258300in}{1.811776in}}{\pgfqpoint{1.261573in}{1.803875in}}{\pgfqpoint{1.267397in}{1.798052in}}%
\pgfpathcurveto{\pgfqpoint{1.273220in}{1.792228in}}{\pgfqpoint{1.281121in}{1.788955in}}{\pgfqpoint{1.289357in}{1.788955in}}%
\pgfpathclose%
\pgfusepath{stroke,fill}%
\end{pgfscope}%
\begin{pgfscope}%
\pgfpathrectangle{\pgfqpoint{0.100000in}{0.212622in}}{\pgfqpoint{3.696000in}{3.696000in}}%
\pgfusepath{clip}%
\pgfsetbuttcap%
\pgfsetroundjoin%
\definecolor{currentfill}{rgb}{0.121569,0.466667,0.705882}%
\pgfsetfillcolor{currentfill}%
\pgfsetfillopacity{0.455823}%
\pgfsetlinewidth{1.003750pt}%
\definecolor{currentstroke}{rgb}{0.121569,0.466667,0.705882}%
\pgfsetstrokecolor{currentstroke}%
\pgfsetstrokeopacity{0.455823}%
\pgfsetdash{}{0pt}%
\pgfpathmoveto{\pgfqpoint{2.568991in}{2.035447in}}%
\pgfpathcurveto{\pgfqpoint{2.577227in}{2.035447in}}{\pgfqpoint{2.585127in}{2.038720in}}{\pgfqpoint{2.590951in}{2.044543in}}%
\pgfpathcurveto{\pgfqpoint{2.596775in}{2.050367in}}{\pgfqpoint{2.600047in}{2.058267in}}{\pgfqpoint{2.600047in}{2.066504in}}%
\pgfpathcurveto{\pgfqpoint{2.600047in}{2.074740in}}{\pgfqpoint{2.596775in}{2.082640in}}{\pgfqpoint{2.590951in}{2.088464in}}%
\pgfpathcurveto{\pgfqpoint{2.585127in}{2.094288in}}{\pgfqpoint{2.577227in}{2.097560in}}{\pgfqpoint{2.568991in}{2.097560in}}%
\pgfpathcurveto{\pgfqpoint{2.560754in}{2.097560in}}{\pgfqpoint{2.552854in}{2.094288in}}{\pgfqpoint{2.547030in}{2.088464in}}%
\pgfpathcurveto{\pgfqpoint{2.541206in}{2.082640in}}{\pgfqpoint{2.537934in}{2.074740in}}{\pgfqpoint{2.537934in}{2.066504in}}%
\pgfpathcurveto{\pgfqpoint{2.537934in}{2.058267in}}{\pgfqpoint{2.541206in}{2.050367in}}{\pgfqpoint{2.547030in}{2.044543in}}%
\pgfpathcurveto{\pgfqpoint{2.552854in}{2.038720in}}{\pgfqpoint{2.560754in}{2.035447in}}{\pgfqpoint{2.568991in}{2.035447in}}%
\pgfpathclose%
\pgfusepath{stroke,fill}%
\end{pgfscope}%
\begin{pgfscope}%
\pgfpathrectangle{\pgfqpoint{0.100000in}{0.212622in}}{\pgfqpoint{3.696000in}{3.696000in}}%
\pgfusepath{clip}%
\pgfsetbuttcap%
\pgfsetroundjoin%
\definecolor{currentfill}{rgb}{0.121569,0.466667,0.705882}%
\pgfsetfillcolor{currentfill}%
\pgfsetfillopacity{0.457448}%
\pgfsetlinewidth{1.003750pt}%
\definecolor{currentstroke}{rgb}{0.121569,0.466667,0.705882}%
\pgfsetstrokecolor{currentstroke}%
\pgfsetstrokeopacity{0.457448}%
\pgfsetdash{}{0pt}%
\pgfpathmoveto{\pgfqpoint{2.577737in}{2.035716in}}%
\pgfpathcurveto{\pgfqpoint{2.585973in}{2.035716in}}{\pgfqpoint{2.593873in}{2.038989in}}{\pgfqpoint{2.599697in}{2.044813in}}%
\pgfpathcurveto{\pgfqpoint{2.605521in}{2.050637in}}{\pgfqpoint{2.608794in}{2.058537in}}{\pgfqpoint{2.608794in}{2.066773in}}%
\pgfpathcurveto{\pgfqpoint{2.608794in}{2.075009in}}{\pgfqpoint{2.605521in}{2.082909in}}{\pgfqpoint{2.599697in}{2.088733in}}%
\pgfpathcurveto{\pgfqpoint{2.593873in}{2.094557in}}{\pgfqpoint{2.585973in}{2.097829in}}{\pgfqpoint{2.577737in}{2.097829in}}%
\pgfpathcurveto{\pgfqpoint{2.569501in}{2.097829in}}{\pgfqpoint{2.561601in}{2.094557in}}{\pgfqpoint{2.555777in}{2.088733in}}%
\pgfpathcurveto{\pgfqpoint{2.549953in}{2.082909in}}{\pgfqpoint{2.546681in}{2.075009in}}{\pgfqpoint{2.546681in}{2.066773in}}%
\pgfpathcurveto{\pgfqpoint{2.546681in}{2.058537in}}{\pgfqpoint{2.549953in}{2.050637in}}{\pgfqpoint{2.555777in}{2.044813in}}%
\pgfpathcurveto{\pgfqpoint{2.561601in}{2.038989in}}{\pgfqpoint{2.569501in}{2.035716in}}{\pgfqpoint{2.577737in}{2.035716in}}%
\pgfpathclose%
\pgfusepath{stroke,fill}%
\end{pgfscope}%
\begin{pgfscope}%
\pgfpathrectangle{\pgfqpoint{0.100000in}{0.212622in}}{\pgfqpoint{3.696000in}{3.696000in}}%
\pgfusepath{clip}%
\pgfsetbuttcap%
\pgfsetroundjoin%
\definecolor{currentfill}{rgb}{0.121569,0.466667,0.705882}%
\pgfsetfillcolor{currentfill}%
\pgfsetfillopacity{0.458397}%
\pgfsetlinewidth{1.003750pt}%
\definecolor{currentstroke}{rgb}{0.121569,0.466667,0.705882}%
\pgfsetstrokecolor{currentstroke}%
\pgfsetstrokeopacity{0.458397}%
\pgfsetdash{}{0pt}%
\pgfpathmoveto{\pgfqpoint{1.280462in}{1.785529in}}%
\pgfpathcurveto{\pgfqpoint{1.288698in}{1.785529in}}{\pgfqpoint{1.296599in}{1.788801in}}{\pgfqpoint{1.302422in}{1.794625in}}%
\pgfpathcurveto{\pgfqpoint{1.308246in}{1.800449in}}{\pgfqpoint{1.311519in}{1.808349in}}{\pgfqpoint{1.311519in}{1.816585in}}%
\pgfpathcurveto{\pgfqpoint{1.311519in}{1.824821in}}{\pgfqpoint{1.308246in}{1.832721in}}{\pgfqpoint{1.302422in}{1.838545in}}%
\pgfpathcurveto{\pgfqpoint{1.296599in}{1.844369in}}{\pgfqpoint{1.288698in}{1.847642in}}{\pgfqpoint{1.280462in}{1.847642in}}%
\pgfpathcurveto{\pgfqpoint{1.272226in}{1.847642in}}{\pgfqpoint{1.264326in}{1.844369in}}{\pgfqpoint{1.258502in}{1.838545in}}%
\pgfpathcurveto{\pgfqpoint{1.252678in}{1.832721in}}{\pgfqpoint{1.249406in}{1.824821in}}{\pgfqpoint{1.249406in}{1.816585in}}%
\pgfpathcurveto{\pgfqpoint{1.249406in}{1.808349in}}{\pgfqpoint{1.252678in}{1.800449in}}{\pgfqpoint{1.258502in}{1.794625in}}%
\pgfpathcurveto{\pgfqpoint{1.264326in}{1.788801in}}{\pgfqpoint{1.272226in}{1.785529in}}{\pgfqpoint{1.280462in}{1.785529in}}%
\pgfpathclose%
\pgfusepath{stroke,fill}%
\end{pgfscope}%
\begin{pgfscope}%
\pgfpathrectangle{\pgfqpoint{0.100000in}{0.212622in}}{\pgfqpoint{3.696000in}{3.696000in}}%
\pgfusepath{clip}%
\pgfsetbuttcap%
\pgfsetroundjoin%
\definecolor{currentfill}{rgb}{0.121569,0.466667,0.705882}%
\pgfsetfillcolor{currentfill}%
\pgfsetfillopacity{0.459533}%
\pgfsetlinewidth{1.003750pt}%
\definecolor{currentstroke}{rgb}{0.121569,0.466667,0.705882}%
\pgfsetstrokecolor{currentstroke}%
\pgfsetstrokeopacity{0.459533}%
\pgfsetdash{}{0pt}%
\pgfpathmoveto{\pgfqpoint{2.588798in}{2.037195in}}%
\pgfpathcurveto{\pgfqpoint{2.597034in}{2.037195in}}{\pgfqpoint{2.604935in}{2.040468in}}{\pgfqpoint{2.610758in}{2.046291in}}%
\pgfpathcurveto{\pgfqpoint{2.616582in}{2.052115in}}{\pgfqpoint{2.619855in}{2.060015in}}{\pgfqpoint{2.619855in}{2.068252in}}%
\pgfpathcurveto{\pgfqpoint{2.619855in}{2.076488in}}{\pgfqpoint{2.616582in}{2.084388in}}{\pgfqpoint{2.610758in}{2.090212in}}%
\pgfpathcurveto{\pgfqpoint{2.604935in}{2.096036in}}{\pgfqpoint{2.597034in}{2.099308in}}{\pgfqpoint{2.588798in}{2.099308in}}%
\pgfpathcurveto{\pgfqpoint{2.580562in}{2.099308in}}{\pgfqpoint{2.572662in}{2.096036in}}{\pgfqpoint{2.566838in}{2.090212in}}%
\pgfpathcurveto{\pgfqpoint{2.561014in}{2.084388in}}{\pgfqpoint{2.557742in}{2.076488in}}{\pgfqpoint{2.557742in}{2.068252in}}%
\pgfpathcurveto{\pgfqpoint{2.557742in}{2.060015in}}{\pgfqpoint{2.561014in}{2.052115in}}{\pgfqpoint{2.566838in}{2.046291in}}%
\pgfpathcurveto{\pgfqpoint{2.572662in}{2.040468in}}{\pgfqpoint{2.580562in}{2.037195in}}{\pgfqpoint{2.588798in}{2.037195in}}%
\pgfpathclose%
\pgfusepath{stroke,fill}%
\end{pgfscope}%
\begin{pgfscope}%
\pgfpathrectangle{\pgfqpoint{0.100000in}{0.212622in}}{\pgfqpoint{3.696000in}{3.696000in}}%
\pgfusepath{clip}%
\pgfsetbuttcap%
\pgfsetroundjoin%
\definecolor{currentfill}{rgb}{0.121569,0.466667,0.705882}%
\pgfsetfillcolor{currentfill}%
\pgfsetfillopacity{0.460910}%
\pgfsetlinewidth{1.003750pt}%
\definecolor{currentstroke}{rgb}{0.121569,0.466667,0.705882}%
\pgfsetstrokecolor{currentstroke}%
\pgfsetstrokeopacity{0.460910}%
\pgfsetdash{}{0pt}%
\pgfpathmoveto{\pgfqpoint{2.599625in}{2.029031in}}%
\pgfpathcurveto{\pgfqpoint{2.607861in}{2.029031in}}{\pgfqpoint{2.615761in}{2.032304in}}{\pgfqpoint{2.621585in}{2.038127in}}%
\pgfpathcurveto{\pgfqpoint{2.627409in}{2.043951in}}{\pgfqpoint{2.630681in}{2.051851in}}{\pgfqpoint{2.630681in}{2.060088in}}%
\pgfpathcurveto{\pgfqpoint{2.630681in}{2.068324in}}{\pgfqpoint{2.627409in}{2.076224in}}{\pgfqpoint{2.621585in}{2.082048in}}%
\pgfpathcurveto{\pgfqpoint{2.615761in}{2.087872in}}{\pgfqpoint{2.607861in}{2.091144in}}{\pgfqpoint{2.599625in}{2.091144in}}%
\pgfpathcurveto{\pgfqpoint{2.591389in}{2.091144in}}{\pgfqpoint{2.583489in}{2.087872in}}{\pgfqpoint{2.577665in}{2.082048in}}%
\pgfpathcurveto{\pgfqpoint{2.571841in}{2.076224in}}{\pgfqpoint{2.568568in}{2.068324in}}{\pgfqpoint{2.568568in}{2.060088in}}%
\pgfpathcurveto{\pgfqpoint{2.568568in}{2.051851in}}{\pgfqpoint{2.571841in}{2.043951in}}{\pgfqpoint{2.577665in}{2.038127in}}%
\pgfpathcurveto{\pgfqpoint{2.583489in}{2.032304in}}{\pgfqpoint{2.591389in}{2.029031in}}{\pgfqpoint{2.599625in}{2.029031in}}%
\pgfpathclose%
\pgfusepath{stroke,fill}%
\end{pgfscope}%
\begin{pgfscope}%
\pgfpathrectangle{\pgfqpoint{0.100000in}{0.212622in}}{\pgfqpoint{3.696000in}{3.696000in}}%
\pgfusepath{clip}%
\pgfsetbuttcap%
\pgfsetroundjoin%
\definecolor{currentfill}{rgb}{0.121569,0.466667,0.705882}%
\pgfsetfillcolor{currentfill}%
\pgfsetfillopacity{0.462302}%
\pgfsetlinewidth{1.003750pt}%
\definecolor{currentstroke}{rgb}{0.121569,0.466667,0.705882}%
\pgfsetstrokecolor{currentstroke}%
\pgfsetstrokeopacity{0.462302}%
\pgfsetdash{}{0pt}%
\pgfpathmoveto{\pgfqpoint{1.277476in}{1.785180in}}%
\pgfpathcurveto{\pgfqpoint{1.285713in}{1.785180in}}{\pgfqpoint{1.293613in}{1.788452in}}{\pgfqpoint{1.299436in}{1.794276in}}%
\pgfpathcurveto{\pgfqpoint{1.305260in}{1.800100in}}{\pgfqpoint{1.308533in}{1.808000in}}{\pgfqpoint{1.308533in}{1.816237in}}%
\pgfpathcurveto{\pgfqpoint{1.308533in}{1.824473in}}{\pgfqpoint{1.305260in}{1.832373in}}{\pgfqpoint{1.299436in}{1.838197in}}%
\pgfpathcurveto{\pgfqpoint{1.293613in}{1.844021in}}{\pgfqpoint{1.285713in}{1.847293in}}{\pgfqpoint{1.277476in}{1.847293in}}%
\pgfpathcurveto{\pgfqpoint{1.269240in}{1.847293in}}{\pgfqpoint{1.261340in}{1.844021in}}{\pgfqpoint{1.255516in}{1.838197in}}%
\pgfpathcurveto{\pgfqpoint{1.249692in}{1.832373in}}{\pgfqpoint{1.246420in}{1.824473in}}{\pgfqpoint{1.246420in}{1.816237in}}%
\pgfpathcurveto{\pgfqpoint{1.246420in}{1.808000in}}{\pgfqpoint{1.249692in}{1.800100in}}{\pgfqpoint{1.255516in}{1.794276in}}%
\pgfpathcurveto{\pgfqpoint{1.261340in}{1.788452in}}{\pgfqpoint{1.269240in}{1.785180in}}{\pgfqpoint{1.277476in}{1.785180in}}%
\pgfpathclose%
\pgfusepath{stroke,fill}%
\end{pgfscope}%
\begin{pgfscope}%
\pgfpathrectangle{\pgfqpoint{0.100000in}{0.212622in}}{\pgfqpoint{3.696000in}{3.696000in}}%
\pgfusepath{clip}%
\pgfsetbuttcap%
\pgfsetroundjoin%
\definecolor{currentfill}{rgb}{0.121569,0.466667,0.705882}%
\pgfsetfillcolor{currentfill}%
\pgfsetfillopacity{0.462953}%
\pgfsetlinewidth{1.003750pt}%
\definecolor{currentstroke}{rgb}{0.121569,0.466667,0.705882}%
\pgfsetstrokecolor{currentstroke}%
\pgfsetstrokeopacity{0.462953}%
\pgfsetdash{}{0pt}%
\pgfpathmoveto{\pgfqpoint{2.611885in}{2.026965in}}%
\pgfpathcurveto{\pgfqpoint{2.620121in}{2.026965in}}{\pgfqpoint{2.628021in}{2.030237in}}{\pgfqpoint{2.633845in}{2.036061in}}%
\pgfpathcurveto{\pgfqpoint{2.639669in}{2.041885in}}{\pgfqpoint{2.642941in}{2.049785in}}{\pgfqpoint{2.642941in}{2.058021in}}%
\pgfpathcurveto{\pgfqpoint{2.642941in}{2.066257in}}{\pgfqpoint{2.639669in}{2.074158in}}{\pgfqpoint{2.633845in}{2.079981in}}%
\pgfpathcurveto{\pgfqpoint{2.628021in}{2.085805in}}{\pgfqpoint{2.620121in}{2.089078in}}{\pgfqpoint{2.611885in}{2.089078in}}%
\pgfpathcurveto{\pgfqpoint{2.603649in}{2.089078in}}{\pgfqpoint{2.595748in}{2.085805in}}{\pgfqpoint{2.589925in}{2.079981in}}%
\pgfpathcurveto{\pgfqpoint{2.584101in}{2.074158in}}{\pgfqpoint{2.580828in}{2.066257in}}{\pgfqpoint{2.580828in}{2.058021in}}%
\pgfpathcurveto{\pgfqpoint{2.580828in}{2.049785in}}{\pgfqpoint{2.584101in}{2.041885in}}{\pgfqpoint{2.589925in}{2.036061in}}%
\pgfpathcurveto{\pgfqpoint{2.595748in}{2.030237in}}{\pgfqpoint{2.603649in}{2.026965in}}{\pgfqpoint{2.611885in}{2.026965in}}%
\pgfpathclose%
\pgfusepath{stroke,fill}%
\end{pgfscope}%
\begin{pgfscope}%
\pgfpathrectangle{\pgfqpoint{0.100000in}{0.212622in}}{\pgfqpoint{3.696000in}{3.696000in}}%
\pgfusepath{clip}%
\pgfsetbuttcap%
\pgfsetroundjoin%
\definecolor{currentfill}{rgb}{0.121569,0.466667,0.705882}%
\pgfsetfillcolor{currentfill}%
\pgfsetfillopacity{0.464555}%
\pgfsetlinewidth{1.003750pt}%
\definecolor{currentstroke}{rgb}{0.121569,0.466667,0.705882}%
\pgfsetstrokecolor{currentstroke}%
\pgfsetstrokeopacity{0.464555}%
\pgfsetdash{}{0pt}%
\pgfpathmoveto{\pgfqpoint{1.267599in}{1.782040in}}%
\pgfpathcurveto{\pgfqpoint{1.275836in}{1.782040in}}{\pgfqpoint{1.283736in}{1.785312in}}{\pgfqpoint{1.289560in}{1.791136in}}%
\pgfpathcurveto{\pgfqpoint{1.295384in}{1.796960in}}{\pgfqpoint{1.298656in}{1.804860in}}{\pgfqpoint{1.298656in}{1.813097in}}%
\pgfpathcurveto{\pgfqpoint{1.298656in}{1.821333in}}{\pgfqpoint{1.295384in}{1.829233in}}{\pgfqpoint{1.289560in}{1.835057in}}%
\pgfpathcurveto{\pgfqpoint{1.283736in}{1.840881in}}{\pgfqpoint{1.275836in}{1.844153in}}{\pgfqpoint{1.267599in}{1.844153in}}%
\pgfpathcurveto{\pgfqpoint{1.259363in}{1.844153in}}{\pgfqpoint{1.251463in}{1.840881in}}{\pgfqpoint{1.245639in}{1.835057in}}%
\pgfpathcurveto{\pgfqpoint{1.239815in}{1.829233in}}{\pgfqpoint{1.236543in}{1.821333in}}{\pgfqpoint{1.236543in}{1.813097in}}%
\pgfpathcurveto{\pgfqpoint{1.236543in}{1.804860in}}{\pgfqpoint{1.239815in}{1.796960in}}{\pgfqpoint{1.245639in}{1.791136in}}%
\pgfpathcurveto{\pgfqpoint{1.251463in}{1.785312in}}{\pgfqpoint{1.259363in}{1.782040in}}{\pgfqpoint{1.267599in}{1.782040in}}%
\pgfpathclose%
\pgfusepath{stroke,fill}%
\end{pgfscope}%
\begin{pgfscope}%
\pgfpathrectangle{\pgfqpoint{0.100000in}{0.212622in}}{\pgfqpoint{3.696000in}{3.696000in}}%
\pgfusepath{clip}%
\pgfsetbuttcap%
\pgfsetroundjoin%
\definecolor{currentfill}{rgb}{0.121569,0.466667,0.705882}%
\pgfsetfillcolor{currentfill}%
\pgfsetfillopacity{0.464918}%
\pgfsetlinewidth{1.003750pt}%
\definecolor{currentstroke}{rgb}{0.121569,0.466667,0.705882}%
\pgfsetstrokecolor{currentstroke}%
\pgfsetstrokeopacity{0.464918}%
\pgfsetdash{}{0pt}%
\pgfpathmoveto{\pgfqpoint{2.624679in}{2.024326in}}%
\pgfpathcurveto{\pgfqpoint{2.632915in}{2.024326in}}{\pgfqpoint{2.640816in}{2.027598in}}{\pgfqpoint{2.646639in}{2.033422in}}%
\pgfpathcurveto{\pgfqpoint{2.652463in}{2.039246in}}{\pgfqpoint{2.655736in}{2.047146in}}{\pgfqpoint{2.655736in}{2.055382in}}%
\pgfpathcurveto{\pgfqpoint{2.655736in}{2.063619in}}{\pgfqpoint{2.652463in}{2.071519in}}{\pgfqpoint{2.646639in}{2.077343in}}%
\pgfpathcurveto{\pgfqpoint{2.640816in}{2.083167in}}{\pgfqpoint{2.632915in}{2.086439in}}{\pgfqpoint{2.624679in}{2.086439in}}%
\pgfpathcurveto{\pgfqpoint{2.616443in}{2.086439in}}{\pgfqpoint{2.608543in}{2.083167in}}{\pgfqpoint{2.602719in}{2.077343in}}%
\pgfpathcurveto{\pgfqpoint{2.596895in}{2.071519in}}{\pgfqpoint{2.593623in}{2.063619in}}{\pgfqpoint{2.593623in}{2.055382in}}%
\pgfpathcurveto{\pgfqpoint{2.593623in}{2.047146in}}{\pgfqpoint{2.596895in}{2.039246in}}{\pgfqpoint{2.602719in}{2.033422in}}%
\pgfpathcurveto{\pgfqpoint{2.608543in}{2.027598in}}{\pgfqpoint{2.616443in}{2.024326in}}{\pgfqpoint{2.624679in}{2.024326in}}%
\pgfpathclose%
\pgfusepath{stroke,fill}%
\end{pgfscope}%
\begin{pgfscope}%
\pgfpathrectangle{\pgfqpoint{0.100000in}{0.212622in}}{\pgfqpoint{3.696000in}{3.696000in}}%
\pgfusepath{clip}%
\pgfsetbuttcap%
\pgfsetroundjoin%
\definecolor{currentfill}{rgb}{0.121569,0.466667,0.705882}%
\pgfsetfillcolor{currentfill}%
\pgfsetfillopacity{0.466641}%
\pgfsetlinewidth{1.003750pt}%
\definecolor{currentstroke}{rgb}{0.121569,0.466667,0.705882}%
\pgfsetstrokecolor{currentstroke}%
\pgfsetstrokeopacity{0.466641}%
\pgfsetdash{}{0pt}%
\pgfpathmoveto{\pgfqpoint{2.638131in}{2.020854in}}%
\pgfpathcurveto{\pgfqpoint{2.646367in}{2.020854in}}{\pgfqpoint{2.654267in}{2.024127in}}{\pgfqpoint{2.660091in}{2.029950in}}%
\pgfpathcurveto{\pgfqpoint{2.665915in}{2.035774in}}{\pgfqpoint{2.669188in}{2.043674in}}{\pgfqpoint{2.669188in}{2.051911in}}%
\pgfpathcurveto{\pgfqpoint{2.669188in}{2.060147in}}{\pgfqpoint{2.665915in}{2.068047in}}{\pgfqpoint{2.660091in}{2.073871in}}%
\pgfpathcurveto{\pgfqpoint{2.654267in}{2.079695in}}{\pgfqpoint{2.646367in}{2.082967in}}{\pgfqpoint{2.638131in}{2.082967in}}%
\pgfpathcurveto{\pgfqpoint{2.629895in}{2.082967in}}{\pgfqpoint{2.621995in}{2.079695in}}{\pgfqpoint{2.616171in}{2.073871in}}%
\pgfpathcurveto{\pgfqpoint{2.610347in}{2.068047in}}{\pgfqpoint{2.607075in}{2.060147in}}{\pgfqpoint{2.607075in}{2.051911in}}%
\pgfpathcurveto{\pgfqpoint{2.607075in}{2.043674in}}{\pgfqpoint{2.610347in}{2.035774in}}{\pgfqpoint{2.616171in}{2.029950in}}%
\pgfpathcurveto{\pgfqpoint{2.621995in}{2.024127in}}{\pgfqpoint{2.629895in}{2.020854in}}{\pgfqpoint{2.638131in}{2.020854in}}%
\pgfpathclose%
\pgfusepath{stroke,fill}%
\end{pgfscope}%
\begin{pgfscope}%
\pgfpathrectangle{\pgfqpoint{0.100000in}{0.212622in}}{\pgfqpoint{3.696000in}{3.696000in}}%
\pgfusepath{clip}%
\pgfsetbuttcap%
\pgfsetroundjoin%
\definecolor{currentfill}{rgb}{0.121569,0.466667,0.705882}%
\pgfsetfillcolor{currentfill}%
\pgfsetfillopacity{0.467490}%
\pgfsetlinewidth{1.003750pt}%
\definecolor{currentstroke}{rgb}{0.121569,0.466667,0.705882}%
\pgfsetstrokecolor{currentstroke}%
\pgfsetstrokeopacity{0.467490}%
\pgfsetdash{}{0pt}%
\pgfpathmoveto{\pgfqpoint{2.645637in}{2.018612in}}%
\pgfpathcurveto{\pgfqpoint{2.653873in}{2.018612in}}{\pgfqpoint{2.661773in}{2.021884in}}{\pgfqpoint{2.667597in}{2.027708in}}%
\pgfpathcurveto{\pgfqpoint{2.673421in}{2.033532in}}{\pgfqpoint{2.676693in}{2.041432in}}{\pgfqpoint{2.676693in}{2.049668in}}%
\pgfpathcurveto{\pgfqpoint{2.676693in}{2.057905in}}{\pgfqpoint{2.673421in}{2.065805in}}{\pgfqpoint{2.667597in}{2.071629in}}%
\pgfpathcurveto{\pgfqpoint{2.661773in}{2.077453in}}{\pgfqpoint{2.653873in}{2.080725in}}{\pgfqpoint{2.645637in}{2.080725in}}%
\pgfpathcurveto{\pgfqpoint{2.637400in}{2.080725in}}{\pgfqpoint{2.629500in}{2.077453in}}{\pgfqpoint{2.623676in}{2.071629in}}%
\pgfpathcurveto{\pgfqpoint{2.617853in}{2.065805in}}{\pgfqpoint{2.614580in}{2.057905in}}{\pgfqpoint{2.614580in}{2.049668in}}%
\pgfpathcurveto{\pgfqpoint{2.614580in}{2.041432in}}{\pgfqpoint{2.617853in}{2.033532in}}{\pgfqpoint{2.623676in}{2.027708in}}%
\pgfpathcurveto{\pgfqpoint{2.629500in}{2.021884in}}{\pgfqpoint{2.637400in}{2.018612in}}{\pgfqpoint{2.645637in}{2.018612in}}%
\pgfpathclose%
\pgfusepath{stroke,fill}%
\end{pgfscope}%
\begin{pgfscope}%
\pgfpathrectangle{\pgfqpoint{0.100000in}{0.212622in}}{\pgfqpoint{3.696000in}{3.696000in}}%
\pgfusepath{clip}%
\pgfsetbuttcap%
\pgfsetroundjoin%
\definecolor{currentfill}{rgb}{0.121569,0.466667,0.705882}%
\pgfsetfillcolor{currentfill}%
\pgfsetfillopacity{0.467680}%
\pgfsetlinewidth{1.003750pt}%
\definecolor{currentstroke}{rgb}{0.121569,0.466667,0.705882}%
\pgfsetstrokecolor{currentstroke}%
\pgfsetstrokeopacity{0.467680}%
\pgfsetdash{}{0pt}%
\pgfpathmoveto{\pgfqpoint{1.263831in}{1.780267in}}%
\pgfpathcurveto{\pgfqpoint{1.272067in}{1.780267in}}{\pgfqpoint{1.279967in}{1.783540in}}{\pgfqpoint{1.285791in}{1.789364in}}%
\pgfpathcurveto{\pgfqpoint{1.291615in}{1.795188in}}{\pgfqpoint{1.294887in}{1.803088in}}{\pgfqpoint{1.294887in}{1.811324in}}%
\pgfpathcurveto{\pgfqpoint{1.294887in}{1.819560in}}{\pgfqpoint{1.291615in}{1.827460in}}{\pgfqpoint{1.285791in}{1.833284in}}%
\pgfpathcurveto{\pgfqpoint{1.279967in}{1.839108in}}{\pgfqpoint{1.272067in}{1.842380in}}{\pgfqpoint{1.263831in}{1.842380in}}%
\pgfpathcurveto{\pgfqpoint{1.255594in}{1.842380in}}{\pgfqpoint{1.247694in}{1.839108in}}{\pgfqpoint{1.241870in}{1.833284in}}%
\pgfpathcurveto{\pgfqpoint{1.236046in}{1.827460in}}{\pgfqpoint{1.232774in}{1.819560in}}{\pgfqpoint{1.232774in}{1.811324in}}%
\pgfpathcurveto{\pgfqpoint{1.232774in}{1.803088in}}{\pgfqpoint{1.236046in}{1.795188in}}{\pgfqpoint{1.241870in}{1.789364in}}%
\pgfpathcurveto{\pgfqpoint{1.247694in}{1.783540in}}{\pgfqpoint{1.255594in}{1.780267in}}{\pgfqpoint{1.263831in}{1.780267in}}%
\pgfpathclose%
\pgfusepath{stroke,fill}%
\end{pgfscope}%
\begin{pgfscope}%
\pgfpathrectangle{\pgfqpoint{0.100000in}{0.212622in}}{\pgfqpoint{3.696000in}{3.696000in}}%
\pgfusepath{clip}%
\pgfsetbuttcap%
\pgfsetroundjoin%
\definecolor{currentfill}{rgb}{0.121569,0.466667,0.705882}%
\pgfsetfillcolor{currentfill}%
\pgfsetfillopacity{0.467928}%
\pgfsetlinewidth{1.003750pt}%
\definecolor{currentstroke}{rgb}{0.121569,0.466667,0.705882}%
\pgfsetstrokecolor{currentstroke}%
\pgfsetstrokeopacity{0.467928}%
\pgfsetdash{}{0pt}%
\pgfpathmoveto{\pgfqpoint{2.649641in}{2.016765in}}%
\pgfpathcurveto{\pgfqpoint{2.657877in}{2.016765in}}{\pgfqpoint{2.665777in}{2.020037in}}{\pgfqpoint{2.671601in}{2.025861in}}%
\pgfpathcurveto{\pgfqpoint{2.677425in}{2.031685in}}{\pgfqpoint{2.680697in}{2.039585in}}{\pgfqpoint{2.680697in}{2.047822in}}%
\pgfpathcurveto{\pgfqpoint{2.680697in}{2.056058in}}{\pgfqpoint{2.677425in}{2.063958in}}{\pgfqpoint{2.671601in}{2.069782in}}%
\pgfpathcurveto{\pgfqpoint{2.665777in}{2.075606in}}{\pgfqpoint{2.657877in}{2.078878in}}{\pgfqpoint{2.649641in}{2.078878in}}%
\pgfpathcurveto{\pgfqpoint{2.641404in}{2.078878in}}{\pgfqpoint{2.633504in}{2.075606in}}{\pgfqpoint{2.627680in}{2.069782in}}%
\pgfpathcurveto{\pgfqpoint{2.621857in}{2.063958in}}{\pgfqpoint{2.618584in}{2.056058in}}{\pgfqpoint{2.618584in}{2.047822in}}%
\pgfpathcurveto{\pgfqpoint{2.618584in}{2.039585in}}{\pgfqpoint{2.621857in}{2.031685in}}{\pgfqpoint{2.627680in}{2.025861in}}%
\pgfpathcurveto{\pgfqpoint{2.633504in}{2.020037in}}{\pgfqpoint{2.641404in}{2.016765in}}{\pgfqpoint{2.649641in}{2.016765in}}%
\pgfpathclose%
\pgfusepath{stroke,fill}%
\end{pgfscope}%
\begin{pgfscope}%
\pgfpathrectangle{\pgfqpoint{0.100000in}{0.212622in}}{\pgfqpoint{3.696000in}{3.696000in}}%
\pgfusepath{clip}%
\pgfsetbuttcap%
\pgfsetroundjoin%
\definecolor{currentfill}{rgb}{0.121569,0.466667,0.705882}%
\pgfsetfillcolor{currentfill}%
\pgfsetfillopacity{0.468706}%
\pgfsetlinewidth{1.003750pt}%
\definecolor{currentstroke}{rgb}{0.121569,0.466667,0.705882}%
\pgfsetstrokecolor{currentstroke}%
\pgfsetstrokeopacity{0.468706}%
\pgfsetdash{}{0pt}%
\pgfpathmoveto{\pgfqpoint{2.654511in}{2.017895in}}%
\pgfpathcurveto{\pgfqpoint{2.662748in}{2.017895in}}{\pgfqpoint{2.670648in}{2.021168in}}{\pgfqpoint{2.676472in}{2.026992in}}%
\pgfpathcurveto{\pgfqpoint{2.682296in}{2.032815in}}{\pgfqpoint{2.685568in}{2.040715in}}{\pgfqpoint{2.685568in}{2.048952in}}%
\pgfpathcurveto{\pgfqpoint{2.685568in}{2.057188in}}{\pgfqpoint{2.682296in}{2.065088in}}{\pgfqpoint{2.676472in}{2.070912in}}%
\pgfpathcurveto{\pgfqpoint{2.670648in}{2.076736in}}{\pgfqpoint{2.662748in}{2.080008in}}{\pgfqpoint{2.654511in}{2.080008in}}%
\pgfpathcurveto{\pgfqpoint{2.646275in}{2.080008in}}{\pgfqpoint{2.638375in}{2.076736in}}{\pgfqpoint{2.632551in}{2.070912in}}%
\pgfpathcurveto{\pgfqpoint{2.626727in}{2.065088in}}{\pgfqpoint{2.623455in}{2.057188in}}{\pgfqpoint{2.623455in}{2.048952in}}%
\pgfpathcurveto{\pgfqpoint{2.623455in}{2.040715in}}{\pgfqpoint{2.626727in}{2.032815in}}{\pgfqpoint{2.632551in}{2.026992in}}%
\pgfpathcurveto{\pgfqpoint{2.638375in}{2.021168in}}{\pgfqpoint{2.646275in}{2.017895in}}{\pgfqpoint{2.654511in}{2.017895in}}%
\pgfpathclose%
\pgfusepath{stroke,fill}%
\end{pgfscope}%
\begin{pgfscope}%
\pgfpathrectangle{\pgfqpoint{0.100000in}{0.212622in}}{\pgfqpoint{3.696000in}{3.696000in}}%
\pgfusepath{clip}%
\pgfsetbuttcap%
\pgfsetroundjoin%
\definecolor{currentfill}{rgb}{0.121569,0.466667,0.705882}%
\pgfsetfillcolor{currentfill}%
\pgfsetfillopacity{0.469176}%
\pgfsetlinewidth{1.003750pt}%
\definecolor{currentstroke}{rgb}{0.121569,0.466667,0.705882}%
\pgfsetstrokecolor{currentstroke}%
\pgfsetstrokeopacity{0.469176}%
\pgfsetdash{}{0pt}%
\pgfpathmoveto{\pgfqpoint{2.659399in}{2.015260in}}%
\pgfpathcurveto{\pgfqpoint{2.667635in}{2.015260in}}{\pgfqpoint{2.675535in}{2.018532in}}{\pgfqpoint{2.681359in}{2.024356in}}%
\pgfpathcurveto{\pgfqpoint{2.687183in}{2.030180in}}{\pgfqpoint{2.690455in}{2.038080in}}{\pgfqpoint{2.690455in}{2.046317in}}%
\pgfpathcurveto{\pgfqpoint{2.690455in}{2.054553in}}{\pgfqpoint{2.687183in}{2.062453in}}{\pgfqpoint{2.681359in}{2.068277in}}%
\pgfpathcurveto{\pgfqpoint{2.675535in}{2.074101in}}{\pgfqpoint{2.667635in}{2.077373in}}{\pgfqpoint{2.659399in}{2.077373in}}%
\pgfpathcurveto{\pgfqpoint{2.651163in}{2.077373in}}{\pgfqpoint{2.643263in}{2.074101in}}{\pgfqpoint{2.637439in}{2.068277in}}%
\pgfpathcurveto{\pgfqpoint{2.631615in}{2.062453in}}{\pgfqpoint{2.628342in}{2.054553in}}{\pgfqpoint{2.628342in}{2.046317in}}%
\pgfpathcurveto{\pgfqpoint{2.628342in}{2.038080in}}{\pgfqpoint{2.631615in}{2.030180in}}{\pgfqpoint{2.637439in}{2.024356in}}%
\pgfpathcurveto{\pgfqpoint{2.643263in}{2.018532in}}{\pgfqpoint{2.651163in}{2.015260in}}{\pgfqpoint{2.659399in}{2.015260in}}%
\pgfpathclose%
\pgfusepath{stroke,fill}%
\end{pgfscope}%
\begin{pgfscope}%
\pgfpathrectangle{\pgfqpoint{0.100000in}{0.212622in}}{\pgfqpoint{3.696000in}{3.696000in}}%
\pgfusepath{clip}%
\pgfsetbuttcap%
\pgfsetroundjoin%
\definecolor{currentfill}{rgb}{0.121569,0.466667,0.705882}%
\pgfsetfillcolor{currentfill}%
\pgfsetfillopacity{0.470428}%
\pgfsetlinewidth{1.003750pt}%
\definecolor{currentstroke}{rgb}{0.121569,0.466667,0.705882}%
\pgfsetstrokecolor{currentstroke}%
\pgfsetstrokeopacity{0.470428}%
\pgfsetdash{}{0pt}%
\pgfpathmoveto{\pgfqpoint{2.664668in}{2.018020in}}%
\pgfpathcurveto{\pgfqpoint{2.672905in}{2.018020in}}{\pgfqpoint{2.680805in}{2.021292in}}{\pgfqpoint{2.686629in}{2.027116in}}%
\pgfpathcurveto{\pgfqpoint{2.692453in}{2.032940in}}{\pgfqpoint{2.695725in}{2.040840in}}{\pgfqpoint{2.695725in}{2.049076in}}%
\pgfpathcurveto{\pgfqpoint{2.695725in}{2.057313in}}{\pgfqpoint{2.692453in}{2.065213in}}{\pgfqpoint{2.686629in}{2.071037in}}%
\pgfpathcurveto{\pgfqpoint{2.680805in}{2.076861in}}{\pgfqpoint{2.672905in}{2.080133in}}{\pgfqpoint{2.664668in}{2.080133in}}%
\pgfpathcurveto{\pgfqpoint{2.656432in}{2.080133in}}{\pgfqpoint{2.648532in}{2.076861in}}{\pgfqpoint{2.642708in}{2.071037in}}%
\pgfpathcurveto{\pgfqpoint{2.636884in}{2.065213in}}{\pgfqpoint{2.633612in}{2.057313in}}{\pgfqpoint{2.633612in}{2.049076in}}%
\pgfpathcurveto{\pgfqpoint{2.633612in}{2.040840in}}{\pgfqpoint{2.636884in}{2.032940in}}{\pgfqpoint{2.642708in}{2.027116in}}%
\pgfpathcurveto{\pgfqpoint{2.648532in}{2.021292in}}{\pgfqpoint{2.656432in}{2.018020in}}{\pgfqpoint{2.664668in}{2.018020in}}%
\pgfpathclose%
\pgfusepath{stroke,fill}%
\end{pgfscope}%
\begin{pgfscope}%
\pgfpathrectangle{\pgfqpoint{0.100000in}{0.212622in}}{\pgfqpoint{3.696000in}{3.696000in}}%
\pgfusepath{clip}%
\pgfsetbuttcap%
\pgfsetroundjoin%
\definecolor{currentfill}{rgb}{0.121569,0.466667,0.705882}%
\pgfsetfillcolor{currentfill}%
\pgfsetfillopacity{0.470615}%
\pgfsetlinewidth{1.003750pt}%
\definecolor{currentstroke}{rgb}{0.121569,0.466667,0.705882}%
\pgfsetstrokecolor{currentstroke}%
\pgfsetstrokeopacity{0.470615}%
\pgfsetdash{}{0pt}%
\pgfpathmoveto{\pgfqpoint{1.258919in}{1.778745in}}%
\pgfpathcurveto{\pgfqpoint{1.267155in}{1.778745in}}{\pgfqpoint{1.275055in}{1.782017in}}{\pgfqpoint{1.280879in}{1.787841in}}%
\pgfpathcurveto{\pgfqpoint{1.286703in}{1.793665in}}{\pgfqpoint{1.289975in}{1.801565in}}{\pgfqpoint{1.289975in}{1.809801in}}%
\pgfpathcurveto{\pgfqpoint{1.289975in}{1.818037in}}{\pgfqpoint{1.286703in}{1.825938in}}{\pgfqpoint{1.280879in}{1.831761in}}%
\pgfpathcurveto{\pgfqpoint{1.275055in}{1.837585in}}{\pgfqpoint{1.267155in}{1.840858in}}{\pgfqpoint{1.258919in}{1.840858in}}%
\pgfpathcurveto{\pgfqpoint{1.250683in}{1.840858in}}{\pgfqpoint{1.242783in}{1.837585in}}{\pgfqpoint{1.236959in}{1.831761in}}%
\pgfpathcurveto{\pgfqpoint{1.231135in}{1.825938in}}{\pgfqpoint{1.227862in}{1.818037in}}{\pgfqpoint{1.227862in}{1.809801in}}%
\pgfpathcurveto{\pgfqpoint{1.227862in}{1.801565in}}{\pgfqpoint{1.231135in}{1.793665in}}{\pgfqpoint{1.236959in}{1.787841in}}%
\pgfpathcurveto{\pgfqpoint{1.242783in}{1.782017in}}{\pgfqpoint{1.250683in}{1.778745in}}{\pgfqpoint{1.258919in}{1.778745in}}%
\pgfpathclose%
\pgfusepath{stroke,fill}%
\end{pgfscope}%
\begin{pgfscope}%
\pgfpathrectangle{\pgfqpoint{0.100000in}{0.212622in}}{\pgfqpoint{3.696000in}{3.696000in}}%
\pgfusepath{clip}%
\pgfsetbuttcap%
\pgfsetroundjoin%
\definecolor{currentfill}{rgb}{0.121569,0.466667,0.705882}%
\pgfsetfillcolor{currentfill}%
\pgfsetfillopacity{0.470693}%
\pgfsetlinewidth{1.003750pt}%
\definecolor{currentstroke}{rgb}{0.121569,0.466667,0.705882}%
\pgfsetstrokecolor{currentstroke}%
\pgfsetstrokeopacity{0.470693}%
\pgfsetdash{}{0pt}%
\pgfpathmoveto{\pgfqpoint{2.667107in}{2.015404in}}%
\pgfpathcurveto{\pgfqpoint{2.675343in}{2.015404in}}{\pgfqpoint{2.683243in}{2.018677in}}{\pgfqpoint{2.689067in}{2.024501in}}%
\pgfpathcurveto{\pgfqpoint{2.694891in}{2.030325in}}{\pgfqpoint{2.698163in}{2.038225in}}{\pgfqpoint{2.698163in}{2.046461in}}%
\pgfpathcurveto{\pgfqpoint{2.698163in}{2.054697in}}{\pgfqpoint{2.694891in}{2.062597in}}{\pgfqpoint{2.689067in}{2.068421in}}%
\pgfpathcurveto{\pgfqpoint{2.683243in}{2.074245in}}{\pgfqpoint{2.675343in}{2.077517in}}{\pgfqpoint{2.667107in}{2.077517in}}%
\pgfpathcurveto{\pgfqpoint{2.658871in}{2.077517in}}{\pgfqpoint{2.650970in}{2.074245in}}{\pgfqpoint{2.645147in}{2.068421in}}%
\pgfpathcurveto{\pgfqpoint{2.639323in}{2.062597in}}{\pgfqpoint{2.636050in}{2.054697in}}{\pgfqpoint{2.636050in}{2.046461in}}%
\pgfpathcurveto{\pgfqpoint{2.636050in}{2.038225in}}{\pgfqpoint{2.639323in}{2.030325in}}{\pgfqpoint{2.645147in}{2.024501in}}%
\pgfpathcurveto{\pgfqpoint{2.650970in}{2.018677in}}{\pgfqpoint{2.658871in}{2.015404in}}{\pgfqpoint{2.667107in}{2.015404in}}%
\pgfpathclose%
\pgfusepath{stroke,fill}%
\end{pgfscope}%
\begin{pgfscope}%
\pgfpathrectangle{\pgfqpoint{0.100000in}{0.212622in}}{\pgfqpoint{3.696000in}{3.696000in}}%
\pgfusepath{clip}%
\pgfsetbuttcap%
\pgfsetroundjoin%
\definecolor{currentfill}{rgb}{0.121569,0.466667,0.705882}%
\pgfsetfillcolor{currentfill}%
\pgfsetfillopacity{0.472004}%
\pgfsetlinewidth{1.003750pt}%
\definecolor{currentstroke}{rgb}{0.121569,0.466667,0.705882}%
\pgfsetstrokecolor{currentstroke}%
\pgfsetstrokeopacity{0.472004}%
\pgfsetdash{}{0pt}%
\pgfpathmoveto{\pgfqpoint{2.672752in}{2.019769in}}%
\pgfpathcurveto{\pgfqpoint{2.680988in}{2.019769in}}{\pgfqpoint{2.688888in}{2.023041in}}{\pgfqpoint{2.694712in}{2.028865in}}%
\pgfpathcurveto{\pgfqpoint{2.700536in}{2.034689in}}{\pgfqpoint{2.703808in}{2.042589in}}{\pgfqpoint{2.703808in}{2.050825in}}%
\pgfpathcurveto{\pgfqpoint{2.703808in}{2.059062in}}{\pgfqpoint{2.700536in}{2.066962in}}{\pgfqpoint{2.694712in}{2.072786in}}%
\pgfpathcurveto{\pgfqpoint{2.688888in}{2.078610in}}{\pgfqpoint{2.680988in}{2.081882in}}{\pgfqpoint{2.672752in}{2.081882in}}%
\pgfpathcurveto{\pgfqpoint{2.664516in}{2.081882in}}{\pgfqpoint{2.656616in}{2.078610in}}{\pgfqpoint{2.650792in}{2.072786in}}%
\pgfpathcurveto{\pgfqpoint{2.644968in}{2.066962in}}{\pgfqpoint{2.641695in}{2.059062in}}{\pgfqpoint{2.641695in}{2.050825in}}%
\pgfpathcurveto{\pgfqpoint{2.641695in}{2.042589in}}{\pgfqpoint{2.644968in}{2.034689in}}{\pgfqpoint{2.650792in}{2.028865in}}%
\pgfpathcurveto{\pgfqpoint{2.656616in}{2.023041in}}{\pgfqpoint{2.664516in}{2.019769in}}{\pgfqpoint{2.672752in}{2.019769in}}%
\pgfpathclose%
\pgfusepath{stroke,fill}%
\end{pgfscope}%
\begin{pgfscope}%
\pgfpathrectangle{\pgfqpoint{0.100000in}{0.212622in}}{\pgfqpoint{3.696000in}{3.696000in}}%
\pgfusepath{clip}%
\pgfsetbuttcap%
\pgfsetroundjoin%
\definecolor{currentfill}{rgb}{0.121569,0.466667,0.705882}%
\pgfsetfillcolor{currentfill}%
\pgfsetfillopacity{0.472202}%
\pgfsetlinewidth{1.003750pt}%
\definecolor{currentstroke}{rgb}{0.121569,0.466667,0.705882}%
\pgfsetstrokecolor{currentstroke}%
\pgfsetstrokeopacity{0.472202}%
\pgfsetdash{}{0pt}%
\pgfpathmoveto{\pgfqpoint{2.675394in}{2.017097in}}%
\pgfpathcurveto{\pgfqpoint{2.683630in}{2.017097in}}{\pgfqpoint{2.691531in}{2.020369in}}{\pgfqpoint{2.697354in}{2.026193in}}%
\pgfpathcurveto{\pgfqpoint{2.703178in}{2.032017in}}{\pgfqpoint{2.706451in}{2.039917in}}{\pgfqpoint{2.706451in}{2.048153in}}%
\pgfpathcurveto{\pgfqpoint{2.706451in}{2.056389in}}{\pgfqpoint{2.703178in}{2.064289in}}{\pgfqpoint{2.697354in}{2.070113in}}%
\pgfpathcurveto{\pgfqpoint{2.691531in}{2.075937in}}{\pgfqpoint{2.683630in}{2.079210in}}{\pgfqpoint{2.675394in}{2.079210in}}%
\pgfpathcurveto{\pgfqpoint{2.667158in}{2.079210in}}{\pgfqpoint{2.659258in}{2.075937in}}{\pgfqpoint{2.653434in}{2.070113in}}%
\pgfpathcurveto{\pgfqpoint{2.647610in}{2.064289in}}{\pgfqpoint{2.644338in}{2.056389in}}{\pgfqpoint{2.644338in}{2.048153in}}%
\pgfpathcurveto{\pgfqpoint{2.644338in}{2.039917in}}{\pgfqpoint{2.647610in}{2.032017in}}{\pgfqpoint{2.653434in}{2.026193in}}%
\pgfpathcurveto{\pgfqpoint{2.659258in}{2.020369in}}{\pgfqpoint{2.667158in}{2.017097in}}{\pgfqpoint{2.675394in}{2.017097in}}%
\pgfpathclose%
\pgfusepath{stroke,fill}%
\end{pgfscope}%
\begin{pgfscope}%
\pgfpathrectangle{\pgfqpoint{0.100000in}{0.212622in}}{\pgfqpoint{3.696000in}{3.696000in}}%
\pgfusepath{clip}%
\pgfsetbuttcap%
\pgfsetroundjoin%
\definecolor{currentfill}{rgb}{0.121569,0.466667,0.705882}%
\pgfsetfillcolor{currentfill}%
\pgfsetfillopacity{0.472937}%
\pgfsetlinewidth{1.003750pt}%
\definecolor{currentstroke}{rgb}{0.121569,0.466667,0.705882}%
\pgfsetstrokecolor{currentstroke}%
\pgfsetstrokeopacity{0.472937}%
\pgfsetdash{}{0pt}%
\pgfpathmoveto{\pgfqpoint{2.679558in}{2.019375in}}%
\pgfpathcurveto{\pgfqpoint{2.687794in}{2.019375in}}{\pgfqpoint{2.695694in}{2.022647in}}{\pgfqpoint{2.701518in}{2.028471in}}%
\pgfpathcurveto{\pgfqpoint{2.707342in}{2.034295in}}{\pgfqpoint{2.710614in}{2.042195in}}{\pgfqpoint{2.710614in}{2.050431in}}%
\pgfpathcurveto{\pgfqpoint{2.710614in}{2.058668in}}{\pgfqpoint{2.707342in}{2.066568in}}{\pgfqpoint{2.701518in}{2.072392in}}%
\pgfpathcurveto{\pgfqpoint{2.695694in}{2.078216in}}{\pgfqpoint{2.687794in}{2.081488in}}{\pgfqpoint{2.679558in}{2.081488in}}%
\pgfpathcurveto{\pgfqpoint{2.671322in}{2.081488in}}{\pgfqpoint{2.663421in}{2.078216in}}{\pgfqpoint{2.657598in}{2.072392in}}%
\pgfpathcurveto{\pgfqpoint{2.651774in}{2.066568in}}{\pgfqpoint{2.648501in}{2.058668in}}{\pgfqpoint{2.648501in}{2.050431in}}%
\pgfpathcurveto{\pgfqpoint{2.648501in}{2.042195in}}{\pgfqpoint{2.651774in}{2.034295in}}{\pgfqpoint{2.657598in}{2.028471in}}%
\pgfpathcurveto{\pgfqpoint{2.663421in}{2.022647in}}{\pgfqpoint{2.671322in}{2.019375in}}{\pgfqpoint{2.679558in}{2.019375in}}%
\pgfpathclose%
\pgfusepath{stroke,fill}%
\end{pgfscope}%
\begin{pgfscope}%
\pgfpathrectangle{\pgfqpoint{0.100000in}{0.212622in}}{\pgfqpoint{3.696000in}{3.696000in}}%
\pgfusepath{clip}%
\pgfsetbuttcap%
\pgfsetroundjoin%
\definecolor{currentfill}{rgb}{0.121569,0.466667,0.705882}%
\pgfsetfillcolor{currentfill}%
\pgfsetfillopacity{0.473349}%
\pgfsetlinewidth{1.003750pt}%
\definecolor{currentstroke}{rgb}{0.121569,0.466667,0.705882}%
\pgfsetstrokecolor{currentstroke}%
\pgfsetstrokeopacity{0.473349}%
\pgfsetdash{}{0pt}%
\pgfpathmoveto{\pgfqpoint{2.683395in}{2.016192in}}%
\pgfpathcurveto{\pgfqpoint{2.691632in}{2.016192in}}{\pgfqpoint{2.699532in}{2.019464in}}{\pgfqpoint{2.705356in}{2.025288in}}%
\pgfpathcurveto{\pgfqpoint{2.711180in}{2.031112in}}{\pgfqpoint{2.714452in}{2.039012in}}{\pgfqpoint{2.714452in}{2.047248in}}%
\pgfpathcurveto{\pgfqpoint{2.714452in}{2.055485in}}{\pgfqpoint{2.711180in}{2.063385in}}{\pgfqpoint{2.705356in}{2.069209in}}%
\pgfpathcurveto{\pgfqpoint{2.699532in}{2.075033in}}{\pgfqpoint{2.691632in}{2.078305in}}{\pgfqpoint{2.683395in}{2.078305in}}%
\pgfpathcurveto{\pgfqpoint{2.675159in}{2.078305in}}{\pgfqpoint{2.667259in}{2.075033in}}{\pgfqpoint{2.661435in}{2.069209in}}%
\pgfpathcurveto{\pgfqpoint{2.655611in}{2.063385in}}{\pgfqpoint{2.652339in}{2.055485in}}{\pgfqpoint{2.652339in}{2.047248in}}%
\pgfpathcurveto{\pgfqpoint{2.652339in}{2.039012in}}{\pgfqpoint{2.655611in}{2.031112in}}{\pgfqpoint{2.661435in}{2.025288in}}%
\pgfpathcurveto{\pgfqpoint{2.667259in}{2.019464in}}{\pgfqpoint{2.675159in}{2.016192in}}{\pgfqpoint{2.683395in}{2.016192in}}%
\pgfpathclose%
\pgfusepath{stroke,fill}%
\end{pgfscope}%
\begin{pgfscope}%
\pgfpathrectangle{\pgfqpoint{0.100000in}{0.212622in}}{\pgfqpoint{3.696000in}{3.696000in}}%
\pgfusepath{clip}%
\pgfsetbuttcap%
\pgfsetroundjoin%
\definecolor{currentfill}{rgb}{0.121569,0.466667,0.705882}%
\pgfsetfillcolor{currentfill}%
\pgfsetfillopacity{0.474555}%
\pgfsetlinewidth{1.003750pt}%
\definecolor{currentstroke}{rgb}{0.121569,0.466667,0.705882}%
\pgfsetstrokecolor{currentstroke}%
\pgfsetstrokeopacity{0.474555}%
\pgfsetdash{}{0pt}%
\pgfpathmoveto{\pgfqpoint{2.688500in}{2.018868in}}%
\pgfpathcurveto{\pgfqpoint{2.696737in}{2.018868in}}{\pgfqpoint{2.704637in}{2.022140in}}{\pgfqpoint{2.710461in}{2.027964in}}%
\pgfpathcurveto{\pgfqpoint{2.716284in}{2.033788in}}{\pgfqpoint{2.719557in}{2.041688in}}{\pgfqpoint{2.719557in}{2.049924in}}%
\pgfpathcurveto{\pgfqpoint{2.719557in}{2.058161in}}{\pgfqpoint{2.716284in}{2.066061in}}{\pgfqpoint{2.710461in}{2.071885in}}%
\pgfpathcurveto{\pgfqpoint{2.704637in}{2.077709in}}{\pgfqpoint{2.696737in}{2.080981in}}{\pgfqpoint{2.688500in}{2.080981in}}%
\pgfpathcurveto{\pgfqpoint{2.680264in}{2.080981in}}{\pgfqpoint{2.672364in}{2.077709in}}{\pgfqpoint{2.666540in}{2.071885in}}%
\pgfpathcurveto{\pgfqpoint{2.660716in}{2.066061in}}{\pgfqpoint{2.657444in}{2.058161in}}{\pgfqpoint{2.657444in}{2.049924in}}%
\pgfpathcurveto{\pgfqpoint{2.657444in}{2.041688in}}{\pgfqpoint{2.660716in}{2.033788in}}{\pgfqpoint{2.666540in}{2.027964in}}%
\pgfpathcurveto{\pgfqpoint{2.672364in}{2.022140in}}{\pgfqpoint{2.680264in}{2.018868in}}{\pgfqpoint{2.688500in}{2.018868in}}%
\pgfpathclose%
\pgfusepath{stroke,fill}%
\end{pgfscope}%
\begin{pgfscope}%
\pgfpathrectangle{\pgfqpoint{0.100000in}{0.212622in}}{\pgfqpoint{3.696000in}{3.696000in}}%
\pgfusepath{clip}%
\pgfsetbuttcap%
\pgfsetroundjoin%
\definecolor{currentfill}{rgb}{0.121569,0.466667,0.705882}%
\pgfsetfillcolor{currentfill}%
\pgfsetfillopacity{0.475635}%
\pgfsetlinewidth{1.003750pt}%
\definecolor{currentstroke}{rgb}{0.121569,0.466667,0.705882}%
\pgfsetstrokecolor{currentstroke}%
\pgfsetstrokeopacity{0.475635}%
\pgfsetdash{}{0pt}%
\pgfpathmoveto{\pgfqpoint{2.693169in}{2.018524in}}%
\pgfpathcurveto{\pgfqpoint{2.701406in}{2.018524in}}{\pgfqpoint{2.709306in}{2.021796in}}{\pgfqpoint{2.715130in}{2.027620in}}%
\pgfpathcurveto{\pgfqpoint{2.720954in}{2.033444in}}{\pgfqpoint{2.724226in}{2.041344in}}{\pgfqpoint{2.724226in}{2.049580in}}%
\pgfpathcurveto{\pgfqpoint{2.724226in}{2.057816in}}{\pgfqpoint{2.720954in}{2.065716in}}{\pgfqpoint{2.715130in}{2.071540in}}%
\pgfpathcurveto{\pgfqpoint{2.709306in}{2.077364in}}{\pgfqpoint{2.701406in}{2.080637in}}{\pgfqpoint{2.693169in}{2.080637in}}%
\pgfpathcurveto{\pgfqpoint{2.684933in}{2.080637in}}{\pgfqpoint{2.677033in}{2.077364in}}{\pgfqpoint{2.671209in}{2.071540in}}%
\pgfpathcurveto{\pgfqpoint{2.665385in}{2.065716in}}{\pgfqpoint{2.662113in}{2.057816in}}{\pgfqpoint{2.662113in}{2.049580in}}%
\pgfpathcurveto{\pgfqpoint{2.662113in}{2.041344in}}{\pgfqpoint{2.665385in}{2.033444in}}{\pgfqpoint{2.671209in}{2.027620in}}%
\pgfpathcurveto{\pgfqpoint{2.677033in}{2.021796in}}{\pgfqpoint{2.684933in}{2.018524in}}{\pgfqpoint{2.693169in}{2.018524in}}%
\pgfpathclose%
\pgfusepath{stroke,fill}%
\end{pgfscope}%
\begin{pgfscope}%
\pgfpathrectangle{\pgfqpoint{0.100000in}{0.212622in}}{\pgfqpoint{3.696000in}{3.696000in}}%
\pgfusepath{clip}%
\pgfsetbuttcap%
\pgfsetroundjoin%
\definecolor{currentfill}{rgb}{0.121569,0.466667,0.705882}%
\pgfsetfillcolor{currentfill}%
\pgfsetfillopacity{0.476837}%
\pgfsetlinewidth{1.003750pt}%
\definecolor{currentstroke}{rgb}{0.121569,0.466667,0.705882}%
\pgfsetstrokecolor{currentstroke}%
\pgfsetstrokeopacity{0.476837}%
\pgfsetdash{}{0pt}%
\pgfpathmoveto{\pgfqpoint{1.254546in}{1.801257in}}%
\pgfpathcurveto{\pgfqpoint{1.262782in}{1.801257in}}{\pgfqpoint{1.270683in}{1.804530in}}{\pgfqpoint{1.276506in}{1.810354in}}%
\pgfpathcurveto{\pgfqpoint{1.282330in}{1.816177in}}{\pgfqpoint{1.285603in}{1.824078in}}{\pgfqpoint{1.285603in}{1.832314in}}%
\pgfpathcurveto{\pgfqpoint{1.285603in}{1.840550in}}{\pgfqpoint{1.282330in}{1.848450in}}{\pgfqpoint{1.276506in}{1.854274in}}%
\pgfpathcurveto{\pgfqpoint{1.270683in}{1.860098in}}{\pgfqpoint{1.262782in}{1.863370in}}{\pgfqpoint{1.254546in}{1.863370in}}%
\pgfpathcurveto{\pgfqpoint{1.246310in}{1.863370in}}{\pgfqpoint{1.238410in}{1.860098in}}{\pgfqpoint{1.232586in}{1.854274in}}%
\pgfpathcurveto{\pgfqpoint{1.226762in}{1.848450in}}{\pgfqpoint{1.223490in}{1.840550in}}{\pgfqpoint{1.223490in}{1.832314in}}%
\pgfpathcurveto{\pgfqpoint{1.223490in}{1.824078in}}{\pgfqpoint{1.226762in}{1.816177in}}{\pgfqpoint{1.232586in}{1.810354in}}%
\pgfpathcurveto{\pgfqpoint{1.238410in}{1.804530in}}{\pgfqpoint{1.246310in}{1.801257in}}{\pgfqpoint{1.254546in}{1.801257in}}%
\pgfpathclose%
\pgfusepath{stroke,fill}%
\end{pgfscope}%
\begin{pgfscope}%
\pgfpathrectangle{\pgfqpoint{0.100000in}{0.212622in}}{\pgfqpoint{3.696000in}{3.696000in}}%
\pgfusepath{clip}%
\pgfsetbuttcap%
\pgfsetroundjoin%
\definecolor{currentfill}{rgb}{0.121569,0.466667,0.705882}%
\pgfsetfillcolor{currentfill}%
\pgfsetfillopacity{0.476886}%
\pgfsetlinewidth{1.003750pt}%
\definecolor{currentstroke}{rgb}{0.121569,0.466667,0.705882}%
\pgfsetstrokecolor{currentstroke}%
\pgfsetstrokeopacity{0.476886}%
\pgfsetdash{}{0pt}%
\pgfpathmoveto{\pgfqpoint{2.698962in}{2.018560in}}%
\pgfpathcurveto{\pgfqpoint{2.707198in}{2.018560in}}{\pgfqpoint{2.715098in}{2.021832in}}{\pgfqpoint{2.720922in}{2.027656in}}%
\pgfpathcurveto{\pgfqpoint{2.726746in}{2.033480in}}{\pgfqpoint{2.730019in}{2.041380in}}{\pgfqpoint{2.730019in}{2.049616in}}%
\pgfpathcurveto{\pgfqpoint{2.730019in}{2.057853in}}{\pgfqpoint{2.726746in}{2.065753in}}{\pgfqpoint{2.720922in}{2.071577in}}%
\pgfpathcurveto{\pgfqpoint{2.715098in}{2.077401in}}{\pgfqpoint{2.707198in}{2.080673in}}{\pgfqpoint{2.698962in}{2.080673in}}%
\pgfpathcurveto{\pgfqpoint{2.690726in}{2.080673in}}{\pgfqpoint{2.682826in}{2.077401in}}{\pgfqpoint{2.677002in}{2.071577in}}%
\pgfpathcurveto{\pgfqpoint{2.671178in}{2.065753in}}{\pgfqpoint{2.667906in}{2.057853in}}{\pgfqpoint{2.667906in}{2.049616in}}%
\pgfpathcurveto{\pgfqpoint{2.667906in}{2.041380in}}{\pgfqpoint{2.671178in}{2.033480in}}{\pgfqpoint{2.677002in}{2.027656in}}%
\pgfpathcurveto{\pgfqpoint{2.682826in}{2.021832in}}{\pgfqpoint{2.690726in}{2.018560in}}{\pgfqpoint{2.698962in}{2.018560in}}%
\pgfpathclose%
\pgfusepath{stroke,fill}%
\end{pgfscope}%
\begin{pgfscope}%
\pgfpathrectangle{\pgfqpoint{0.100000in}{0.212622in}}{\pgfqpoint{3.696000in}{3.696000in}}%
\pgfusepath{clip}%
\pgfsetbuttcap%
\pgfsetroundjoin%
\definecolor{currentfill}{rgb}{0.121569,0.466667,0.705882}%
\pgfsetfillcolor{currentfill}%
\pgfsetfillopacity{0.478140}%
\pgfsetlinewidth{1.003750pt}%
\definecolor{currentstroke}{rgb}{0.121569,0.466667,0.705882}%
\pgfsetstrokecolor{currentstroke}%
\pgfsetstrokeopacity{0.478140}%
\pgfsetdash{}{0pt}%
\pgfpathmoveto{\pgfqpoint{2.705146in}{2.018074in}}%
\pgfpathcurveto{\pgfqpoint{2.713382in}{2.018074in}}{\pgfqpoint{2.721282in}{2.021346in}}{\pgfqpoint{2.727106in}{2.027170in}}%
\pgfpathcurveto{\pgfqpoint{2.732930in}{2.032994in}}{\pgfqpoint{2.736202in}{2.040894in}}{\pgfqpoint{2.736202in}{2.049130in}}%
\pgfpathcurveto{\pgfqpoint{2.736202in}{2.057367in}}{\pgfqpoint{2.732930in}{2.065267in}}{\pgfqpoint{2.727106in}{2.071091in}}%
\pgfpathcurveto{\pgfqpoint{2.721282in}{2.076914in}}{\pgfqpoint{2.713382in}{2.080187in}}{\pgfqpoint{2.705146in}{2.080187in}}%
\pgfpathcurveto{\pgfqpoint{2.696909in}{2.080187in}}{\pgfqpoint{2.689009in}{2.076914in}}{\pgfqpoint{2.683185in}{2.071091in}}%
\pgfpathcurveto{\pgfqpoint{2.677361in}{2.065267in}}{\pgfqpoint{2.674089in}{2.057367in}}{\pgfqpoint{2.674089in}{2.049130in}}%
\pgfpathcurveto{\pgfqpoint{2.674089in}{2.040894in}}{\pgfqpoint{2.677361in}{2.032994in}}{\pgfqpoint{2.683185in}{2.027170in}}%
\pgfpathcurveto{\pgfqpoint{2.689009in}{2.021346in}}{\pgfqpoint{2.696909in}{2.018074in}}{\pgfqpoint{2.705146in}{2.018074in}}%
\pgfpathclose%
\pgfusepath{stroke,fill}%
\end{pgfscope}%
\begin{pgfscope}%
\pgfpathrectangle{\pgfqpoint{0.100000in}{0.212622in}}{\pgfqpoint{3.696000in}{3.696000in}}%
\pgfusepath{clip}%
\pgfsetbuttcap%
\pgfsetroundjoin%
\definecolor{currentfill}{rgb}{0.121569,0.466667,0.705882}%
\pgfsetfillcolor{currentfill}%
\pgfsetfillopacity{0.479219}%
\pgfsetlinewidth{1.003750pt}%
\definecolor{currentstroke}{rgb}{0.121569,0.466667,0.705882}%
\pgfsetstrokecolor{currentstroke}%
\pgfsetstrokeopacity{0.479219}%
\pgfsetdash{}{0pt}%
\pgfpathmoveto{\pgfqpoint{2.713110in}{2.016166in}}%
\pgfpathcurveto{\pgfqpoint{2.721347in}{2.016166in}}{\pgfqpoint{2.729247in}{2.019438in}}{\pgfqpoint{2.735071in}{2.025262in}}%
\pgfpathcurveto{\pgfqpoint{2.740894in}{2.031086in}}{\pgfqpoint{2.744167in}{2.038986in}}{\pgfqpoint{2.744167in}{2.047222in}}%
\pgfpathcurveto{\pgfqpoint{2.744167in}{2.055459in}}{\pgfqpoint{2.740894in}{2.063359in}}{\pgfqpoint{2.735071in}{2.069183in}}%
\pgfpathcurveto{\pgfqpoint{2.729247in}{2.075007in}}{\pgfqpoint{2.721347in}{2.078279in}}{\pgfqpoint{2.713110in}{2.078279in}}%
\pgfpathcurveto{\pgfqpoint{2.704874in}{2.078279in}}{\pgfqpoint{2.696974in}{2.075007in}}{\pgfqpoint{2.691150in}{2.069183in}}%
\pgfpathcurveto{\pgfqpoint{2.685326in}{2.063359in}}{\pgfqpoint{2.682054in}{2.055459in}}{\pgfqpoint{2.682054in}{2.047222in}}%
\pgfpathcurveto{\pgfqpoint{2.682054in}{2.038986in}}{\pgfqpoint{2.685326in}{2.031086in}}{\pgfqpoint{2.691150in}{2.025262in}}%
\pgfpathcurveto{\pgfqpoint{2.696974in}{2.019438in}}{\pgfqpoint{2.704874in}{2.016166in}}{\pgfqpoint{2.713110in}{2.016166in}}%
\pgfpathclose%
\pgfusepath{stroke,fill}%
\end{pgfscope}%
\begin{pgfscope}%
\pgfpathrectangle{\pgfqpoint{0.100000in}{0.212622in}}{\pgfqpoint{3.696000in}{3.696000in}}%
\pgfusepath{clip}%
\pgfsetbuttcap%
\pgfsetroundjoin%
\definecolor{currentfill}{rgb}{0.121569,0.466667,0.705882}%
\pgfsetfillcolor{currentfill}%
\pgfsetfillopacity{0.479695}%
\pgfsetlinewidth{1.003750pt}%
\definecolor{currentstroke}{rgb}{0.121569,0.466667,0.705882}%
\pgfsetstrokecolor{currentstroke}%
\pgfsetstrokeopacity{0.479695}%
\pgfsetdash{}{0pt}%
\pgfpathmoveto{\pgfqpoint{1.251216in}{1.799723in}}%
\pgfpathcurveto{\pgfqpoint{1.259452in}{1.799723in}}{\pgfqpoint{1.267352in}{1.802996in}}{\pgfqpoint{1.273176in}{1.808820in}}%
\pgfpathcurveto{\pgfqpoint{1.279000in}{1.814644in}}{\pgfqpoint{1.282273in}{1.822544in}}{\pgfqpoint{1.282273in}{1.830780in}}%
\pgfpathcurveto{\pgfqpoint{1.282273in}{1.839016in}}{\pgfqpoint{1.279000in}{1.846916in}}{\pgfqpoint{1.273176in}{1.852740in}}%
\pgfpathcurveto{\pgfqpoint{1.267352in}{1.858564in}}{\pgfqpoint{1.259452in}{1.861836in}}{\pgfqpoint{1.251216in}{1.861836in}}%
\pgfpathcurveto{\pgfqpoint{1.242980in}{1.861836in}}{\pgfqpoint{1.235080in}{1.858564in}}{\pgfqpoint{1.229256in}{1.852740in}}%
\pgfpathcurveto{\pgfqpoint{1.223432in}{1.846916in}}{\pgfqpoint{1.220160in}{1.839016in}}{\pgfqpoint{1.220160in}{1.830780in}}%
\pgfpathcurveto{\pgfqpoint{1.220160in}{1.822544in}}{\pgfqpoint{1.223432in}{1.814644in}}{\pgfqpoint{1.229256in}{1.808820in}}%
\pgfpathcurveto{\pgfqpoint{1.235080in}{1.802996in}}{\pgfqpoint{1.242980in}{1.799723in}}{\pgfqpoint{1.251216in}{1.799723in}}%
\pgfpathclose%
\pgfusepath{stroke,fill}%
\end{pgfscope}%
\begin{pgfscope}%
\pgfpathrectangle{\pgfqpoint{0.100000in}{0.212622in}}{\pgfqpoint{3.696000in}{3.696000in}}%
\pgfusepath{clip}%
\pgfsetbuttcap%
\pgfsetroundjoin%
\definecolor{currentfill}{rgb}{0.121569,0.466667,0.705882}%
\pgfsetfillcolor{currentfill}%
\pgfsetfillopacity{0.480864}%
\pgfsetlinewidth{1.003750pt}%
\definecolor{currentstroke}{rgb}{0.121569,0.466667,0.705882}%
\pgfsetstrokecolor{currentstroke}%
\pgfsetstrokeopacity{0.480864}%
\pgfsetdash{}{0pt}%
\pgfpathmoveto{\pgfqpoint{2.721889in}{2.019419in}}%
\pgfpathcurveto{\pgfqpoint{2.730126in}{2.019419in}}{\pgfqpoint{2.738026in}{2.022692in}}{\pgfqpoint{2.743850in}{2.028516in}}%
\pgfpathcurveto{\pgfqpoint{2.749674in}{2.034339in}}{\pgfqpoint{2.752946in}{2.042240in}}{\pgfqpoint{2.752946in}{2.050476in}}%
\pgfpathcurveto{\pgfqpoint{2.752946in}{2.058712in}}{\pgfqpoint{2.749674in}{2.066612in}}{\pgfqpoint{2.743850in}{2.072436in}}%
\pgfpathcurveto{\pgfqpoint{2.738026in}{2.078260in}}{\pgfqpoint{2.730126in}{2.081532in}}{\pgfqpoint{2.721889in}{2.081532in}}%
\pgfpathcurveto{\pgfqpoint{2.713653in}{2.081532in}}{\pgfqpoint{2.705753in}{2.078260in}}{\pgfqpoint{2.699929in}{2.072436in}}%
\pgfpathcurveto{\pgfqpoint{2.694105in}{2.066612in}}{\pgfqpoint{2.690833in}{2.058712in}}{\pgfqpoint{2.690833in}{2.050476in}}%
\pgfpathcurveto{\pgfqpoint{2.690833in}{2.042240in}}{\pgfqpoint{2.694105in}{2.034339in}}{\pgfqpoint{2.699929in}{2.028516in}}%
\pgfpathcurveto{\pgfqpoint{2.705753in}{2.022692in}}{\pgfqpoint{2.713653in}{2.019419in}}{\pgfqpoint{2.721889in}{2.019419in}}%
\pgfpathclose%
\pgfusepath{stroke,fill}%
\end{pgfscope}%
\begin{pgfscope}%
\pgfpathrectangle{\pgfqpoint{0.100000in}{0.212622in}}{\pgfqpoint{3.696000in}{3.696000in}}%
\pgfusepath{clip}%
\pgfsetbuttcap%
\pgfsetroundjoin%
\definecolor{currentfill}{rgb}{0.121569,0.466667,0.705882}%
\pgfsetfillcolor{currentfill}%
\pgfsetfillopacity{0.481245}%
\pgfsetlinewidth{1.003750pt}%
\definecolor{currentstroke}{rgb}{0.121569,0.466667,0.705882}%
\pgfsetstrokecolor{currentstroke}%
\pgfsetstrokeopacity{0.481245}%
\pgfsetdash{}{0pt}%
\pgfpathmoveto{\pgfqpoint{2.729931in}{2.010403in}}%
\pgfpathcurveto{\pgfqpoint{2.738168in}{2.010403in}}{\pgfqpoint{2.746068in}{2.013675in}}{\pgfqpoint{2.751892in}{2.019499in}}%
\pgfpathcurveto{\pgfqpoint{2.757716in}{2.025323in}}{\pgfqpoint{2.760988in}{2.033223in}}{\pgfqpoint{2.760988in}{2.041459in}}%
\pgfpathcurveto{\pgfqpoint{2.760988in}{2.049695in}}{\pgfqpoint{2.757716in}{2.057595in}}{\pgfqpoint{2.751892in}{2.063419in}}%
\pgfpathcurveto{\pgfqpoint{2.746068in}{2.069243in}}{\pgfqpoint{2.738168in}{2.072516in}}{\pgfqpoint{2.729931in}{2.072516in}}%
\pgfpathcurveto{\pgfqpoint{2.721695in}{2.072516in}}{\pgfqpoint{2.713795in}{2.069243in}}{\pgfqpoint{2.707971in}{2.063419in}}%
\pgfpathcurveto{\pgfqpoint{2.702147in}{2.057595in}}{\pgfqpoint{2.698875in}{2.049695in}}{\pgfqpoint{2.698875in}{2.041459in}}%
\pgfpathcurveto{\pgfqpoint{2.698875in}{2.033223in}}{\pgfqpoint{2.702147in}{2.025323in}}{\pgfqpoint{2.707971in}{2.019499in}}%
\pgfpathcurveto{\pgfqpoint{2.713795in}{2.013675in}}{\pgfqpoint{2.721695in}{2.010403in}}{\pgfqpoint{2.729931in}{2.010403in}}%
\pgfpathclose%
\pgfusepath{stroke,fill}%
\end{pgfscope}%
\begin{pgfscope}%
\pgfpathrectangle{\pgfqpoint{0.100000in}{0.212622in}}{\pgfqpoint{3.696000in}{3.696000in}}%
\pgfusepath{clip}%
\pgfsetbuttcap%
\pgfsetroundjoin%
\definecolor{currentfill}{rgb}{0.121569,0.466667,0.705882}%
\pgfsetfillcolor{currentfill}%
\pgfsetfillopacity{0.481324}%
\pgfsetlinewidth{1.003750pt}%
\definecolor{currentstroke}{rgb}{0.121569,0.466667,0.705882}%
\pgfsetstrokecolor{currentstroke}%
\pgfsetstrokeopacity{0.481324}%
\pgfsetdash{}{0pt}%
\pgfpathmoveto{\pgfqpoint{1.243745in}{1.796310in}}%
\pgfpathcurveto{\pgfqpoint{1.251981in}{1.796310in}}{\pgfqpoint{1.259881in}{1.799582in}}{\pgfqpoint{1.265705in}{1.805406in}}%
\pgfpathcurveto{\pgfqpoint{1.271529in}{1.811230in}}{\pgfqpoint{1.274801in}{1.819130in}}{\pgfqpoint{1.274801in}{1.827366in}}%
\pgfpathcurveto{\pgfqpoint{1.274801in}{1.835602in}}{\pgfqpoint{1.271529in}{1.843503in}}{\pgfqpoint{1.265705in}{1.849326in}}%
\pgfpathcurveto{\pgfqpoint{1.259881in}{1.855150in}}{\pgfqpoint{1.251981in}{1.858423in}}{\pgfqpoint{1.243745in}{1.858423in}}%
\pgfpathcurveto{\pgfqpoint{1.235509in}{1.858423in}}{\pgfqpoint{1.227609in}{1.855150in}}{\pgfqpoint{1.221785in}{1.849326in}}%
\pgfpathcurveto{\pgfqpoint{1.215961in}{1.843503in}}{\pgfqpoint{1.212688in}{1.835602in}}{\pgfqpoint{1.212688in}{1.827366in}}%
\pgfpathcurveto{\pgfqpoint{1.212688in}{1.819130in}}{\pgfqpoint{1.215961in}{1.811230in}}{\pgfqpoint{1.221785in}{1.805406in}}%
\pgfpathcurveto{\pgfqpoint{1.227609in}{1.799582in}}{\pgfqpoint{1.235509in}{1.796310in}}{\pgfqpoint{1.243745in}{1.796310in}}%
\pgfpathclose%
\pgfusepath{stroke,fill}%
\end{pgfscope}%
\begin{pgfscope}%
\pgfpathrectangle{\pgfqpoint{0.100000in}{0.212622in}}{\pgfqpoint{3.696000in}{3.696000in}}%
\pgfusepath{clip}%
\pgfsetbuttcap%
\pgfsetroundjoin%
\definecolor{currentfill}{rgb}{0.121569,0.466667,0.705882}%
\pgfsetfillcolor{currentfill}%
\pgfsetfillopacity{0.483003}%
\pgfsetlinewidth{1.003750pt}%
\definecolor{currentstroke}{rgb}{0.121569,0.466667,0.705882}%
\pgfsetstrokecolor{currentstroke}%
\pgfsetstrokeopacity{0.483003}%
\pgfsetdash{}{0pt}%
\pgfpathmoveto{\pgfqpoint{2.740007in}{2.015143in}}%
\pgfpathcurveto{\pgfqpoint{2.748244in}{2.015143in}}{\pgfqpoint{2.756144in}{2.018415in}}{\pgfqpoint{2.761968in}{2.024239in}}%
\pgfpathcurveto{\pgfqpoint{2.767792in}{2.030063in}}{\pgfqpoint{2.771064in}{2.037963in}}{\pgfqpoint{2.771064in}{2.046199in}}%
\pgfpathcurveto{\pgfqpoint{2.771064in}{2.054435in}}{\pgfqpoint{2.767792in}{2.062335in}}{\pgfqpoint{2.761968in}{2.068159in}}%
\pgfpathcurveto{\pgfqpoint{2.756144in}{2.073983in}}{\pgfqpoint{2.748244in}{2.077256in}}{\pgfqpoint{2.740007in}{2.077256in}}%
\pgfpathcurveto{\pgfqpoint{2.731771in}{2.077256in}}{\pgfqpoint{2.723871in}{2.073983in}}{\pgfqpoint{2.718047in}{2.068159in}}%
\pgfpathcurveto{\pgfqpoint{2.712223in}{2.062335in}}{\pgfqpoint{2.708951in}{2.054435in}}{\pgfqpoint{2.708951in}{2.046199in}}%
\pgfpathcurveto{\pgfqpoint{2.708951in}{2.037963in}}{\pgfqpoint{2.712223in}{2.030063in}}{\pgfqpoint{2.718047in}{2.024239in}}%
\pgfpathcurveto{\pgfqpoint{2.723871in}{2.018415in}}{\pgfqpoint{2.731771in}{2.015143in}}{\pgfqpoint{2.740007in}{2.015143in}}%
\pgfpathclose%
\pgfusepath{stroke,fill}%
\end{pgfscope}%
\begin{pgfscope}%
\pgfpathrectangle{\pgfqpoint{0.100000in}{0.212622in}}{\pgfqpoint{3.696000in}{3.696000in}}%
\pgfusepath{clip}%
\pgfsetbuttcap%
\pgfsetroundjoin%
\definecolor{currentfill}{rgb}{0.121569,0.466667,0.705882}%
\pgfsetfillcolor{currentfill}%
\pgfsetfillopacity{0.483241}%
\pgfsetlinewidth{1.003750pt}%
\definecolor{currentstroke}{rgb}{0.121569,0.466667,0.705882}%
\pgfsetstrokecolor{currentstroke}%
\pgfsetstrokeopacity{0.483241}%
\pgfsetdash{}{0pt}%
\pgfpathmoveto{\pgfqpoint{2.744243in}{2.008774in}}%
\pgfpathcurveto{\pgfqpoint{2.752479in}{2.008774in}}{\pgfqpoint{2.760379in}{2.012046in}}{\pgfqpoint{2.766203in}{2.017870in}}%
\pgfpathcurveto{\pgfqpoint{2.772027in}{2.023694in}}{\pgfqpoint{2.775300in}{2.031594in}}{\pgfqpoint{2.775300in}{2.039831in}}%
\pgfpathcurveto{\pgfqpoint{2.775300in}{2.048067in}}{\pgfqpoint{2.772027in}{2.055967in}}{\pgfqpoint{2.766203in}{2.061791in}}%
\pgfpathcurveto{\pgfqpoint{2.760379in}{2.067615in}}{\pgfqpoint{2.752479in}{2.070887in}}{\pgfqpoint{2.744243in}{2.070887in}}%
\pgfpathcurveto{\pgfqpoint{2.736007in}{2.070887in}}{\pgfqpoint{2.728107in}{2.067615in}}{\pgfqpoint{2.722283in}{2.061791in}}%
\pgfpathcurveto{\pgfqpoint{2.716459in}{2.055967in}}{\pgfqpoint{2.713187in}{2.048067in}}{\pgfqpoint{2.713187in}{2.039831in}}%
\pgfpathcurveto{\pgfqpoint{2.713187in}{2.031594in}}{\pgfqpoint{2.716459in}{2.023694in}}{\pgfqpoint{2.722283in}{2.017870in}}%
\pgfpathcurveto{\pgfqpoint{2.728107in}{2.012046in}}{\pgfqpoint{2.736007in}{2.008774in}}{\pgfqpoint{2.744243in}{2.008774in}}%
\pgfpathclose%
\pgfusepath{stroke,fill}%
\end{pgfscope}%
\begin{pgfscope}%
\pgfpathrectangle{\pgfqpoint{0.100000in}{0.212622in}}{\pgfqpoint{3.696000in}{3.696000in}}%
\pgfusepath{clip}%
\pgfsetbuttcap%
\pgfsetroundjoin%
\definecolor{currentfill}{rgb}{0.121569,0.466667,0.705882}%
\pgfsetfillcolor{currentfill}%
\pgfsetfillopacity{0.483617}%
\pgfsetlinewidth{1.003750pt}%
\definecolor{currentstroke}{rgb}{0.121569,0.466667,0.705882}%
\pgfsetstrokecolor{currentstroke}%
\pgfsetstrokeopacity{0.483617}%
\pgfsetdash{}{0pt}%
\pgfpathmoveto{\pgfqpoint{1.240503in}{1.794378in}}%
\pgfpathcurveto{\pgfqpoint{1.248739in}{1.794378in}}{\pgfqpoint{1.256639in}{1.797650in}}{\pgfqpoint{1.262463in}{1.803474in}}%
\pgfpathcurveto{\pgfqpoint{1.268287in}{1.809298in}}{\pgfqpoint{1.271559in}{1.817198in}}{\pgfqpoint{1.271559in}{1.825435in}}%
\pgfpathcurveto{\pgfqpoint{1.271559in}{1.833671in}}{\pgfqpoint{1.268287in}{1.841571in}}{\pgfqpoint{1.262463in}{1.847395in}}%
\pgfpathcurveto{\pgfqpoint{1.256639in}{1.853219in}}{\pgfqpoint{1.248739in}{1.856491in}}{\pgfqpoint{1.240503in}{1.856491in}}%
\pgfpathcurveto{\pgfqpoint{1.232266in}{1.856491in}}{\pgfqpoint{1.224366in}{1.853219in}}{\pgfqpoint{1.218543in}{1.847395in}}%
\pgfpathcurveto{\pgfqpoint{1.212719in}{1.841571in}}{\pgfqpoint{1.209446in}{1.833671in}}{\pgfqpoint{1.209446in}{1.825435in}}%
\pgfpathcurveto{\pgfqpoint{1.209446in}{1.817198in}}{\pgfqpoint{1.212719in}{1.809298in}}{\pgfqpoint{1.218543in}{1.803474in}}%
\pgfpathcurveto{\pgfqpoint{1.224366in}{1.797650in}}{\pgfqpoint{1.232266in}{1.794378in}}{\pgfqpoint{1.240503in}{1.794378in}}%
\pgfpathclose%
\pgfusepath{stroke,fill}%
\end{pgfscope}%
\begin{pgfscope}%
\pgfpathrectangle{\pgfqpoint{0.100000in}{0.212622in}}{\pgfqpoint{3.696000in}{3.696000in}}%
\pgfusepath{clip}%
\pgfsetbuttcap%
\pgfsetroundjoin%
\definecolor{currentfill}{rgb}{0.121569,0.466667,0.705882}%
\pgfsetfillcolor{currentfill}%
\pgfsetfillopacity{0.484605}%
\pgfsetlinewidth{1.003750pt}%
\definecolor{currentstroke}{rgb}{0.121569,0.466667,0.705882}%
\pgfsetstrokecolor{currentstroke}%
\pgfsetstrokeopacity{0.484605}%
\pgfsetdash{}{0pt}%
\pgfpathmoveto{\pgfqpoint{2.749848in}{2.011726in}}%
\pgfpathcurveto{\pgfqpoint{2.758084in}{2.011726in}}{\pgfqpoint{2.765985in}{2.014998in}}{\pgfqpoint{2.771808in}{2.020822in}}%
\pgfpathcurveto{\pgfqpoint{2.777632in}{2.026646in}}{\pgfqpoint{2.780905in}{2.034546in}}{\pgfqpoint{2.780905in}{2.042782in}}%
\pgfpathcurveto{\pgfqpoint{2.780905in}{2.051018in}}{\pgfqpoint{2.777632in}{2.058918in}}{\pgfqpoint{2.771808in}{2.064742in}}%
\pgfpathcurveto{\pgfqpoint{2.765985in}{2.070566in}}{\pgfqpoint{2.758084in}{2.073839in}}{\pgfqpoint{2.749848in}{2.073839in}}%
\pgfpathcurveto{\pgfqpoint{2.741612in}{2.073839in}}{\pgfqpoint{2.733712in}{2.070566in}}{\pgfqpoint{2.727888in}{2.064742in}}%
\pgfpathcurveto{\pgfqpoint{2.722064in}{2.058918in}}{\pgfqpoint{2.718792in}{2.051018in}}{\pgfqpoint{2.718792in}{2.042782in}}%
\pgfpathcurveto{\pgfqpoint{2.718792in}{2.034546in}}{\pgfqpoint{2.722064in}{2.026646in}}{\pgfqpoint{2.727888in}{2.020822in}}%
\pgfpathcurveto{\pgfqpoint{2.733712in}{2.014998in}}{\pgfqpoint{2.741612in}{2.011726in}}{\pgfqpoint{2.749848in}{2.011726in}}%
\pgfpathclose%
\pgfusepath{stroke,fill}%
\end{pgfscope}%
\begin{pgfscope}%
\pgfpathrectangle{\pgfqpoint{0.100000in}{0.212622in}}{\pgfqpoint{3.696000in}{3.696000in}}%
\pgfusepath{clip}%
\pgfsetbuttcap%
\pgfsetroundjoin%
\definecolor{currentfill}{rgb}{0.121569,0.466667,0.705882}%
\pgfsetfillcolor{currentfill}%
\pgfsetfillopacity{0.484971}%
\pgfsetlinewidth{1.003750pt}%
\definecolor{currentstroke}{rgb}{0.121569,0.466667,0.705882}%
\pgfsetstrokecolor{currentstroke}%
\pgfsetstrokeopacity{0.484971}%
\pgfsetdash{}{0pt}%
\pgfpathmoveto{\pgfqpoint{1.237696in}{1.788762in}}%
\pgfpathcurveto{\pgfqpoint{1.245932in}{1.788762in}}{\pgfqpoint{1.253832in}{1.792034in}}{\pgfqpoint{1.259656in}{1.797858in}}%
\pgfpathcurveto{\pgfqpoint{1.265480in}{1.803682in}}{\pgfqpoint{1.268752in}{1.811582in}}{\pgfqpoint{1.268752in}{1.819818in}}%
\pgfpathcurveto{\pgfqpoint{1.268752in}{1.828054in}}{\pgfqpoint{1.265480in}{1.835954in}}{\pgfqpoint{1.259656in}{1.841778in}}%
\pgfpathcurveto{\pgfqpoint{1.253832in}{1.847602in}}{\pgfqpoint{1.245932in}{1.850875in}}{\pgfqpoint{1.237696in}{1.850875in}}%
\pgfpathcurveto{\pgfqpoint{1.229459in}{1.850875in}}{\pgfqpoint{1.221559in}{1.847602in}}{\pgfqpoint{1.215735in}{1.841778in}}%
\pgfpathcurveto{\pgfqpoint{1.209912in}{1.835954in}}{\pgfqpoint{1.206639in}{1.828054in}}{\pgfqpoint{1.206639in}{1.819818in}}%
\pgfpathcurveto{\pgfqpoint{1.206639in}{1.811582in}}{\pgfqpoint{1.209912in}{1.803682in}}{\pgfqpoint{1.215735in}{1.797858in}}%
\pgfpathcurveto{\pgfqpoint{1.221559in}{1.792034in}}{\pgfqpoint{1.229459in}{1.788762in}}{\pgfqpoint{1.237696in}{1.788762in}}%
\pgfpathclose%
\pgfusepath{stroke,fill}%
\end{pgfscope}%
\begin{pgfscope}%
\pgfpathrectangle{\pgfqpoint{0.100000in}{0.212622in}}{\pgfqpoint{3.696000in}{3.696000in}}%
\pgfusepath{clip}%
\pgfsetbuttcap%
\pgfsetroundjoin%
\definecolor{currentfill}{rgb}{0.121569,0.466667,0.705882}%
\pgfsetfillcolor{currentfill}%
\pgfsetfillopacity{0.485459}%
\pgfsetlinewidth{1.003750pt}%
\definecolor{currentstroke}{rgb}{0.121569,0.466667,0.705882}%
\pgfsetstrokecolor{currentstroke}%
\pgfsetstrokeopacity{0.485459}%
\pgfsetdash{}{0pt}%
\pgfpathmoveto{\pgfqpoint{2.756250in}{2.008225in}}%
\pgfpathcurveto{\pgfqpoint{2.764486in}{2.008225in}}{\pgfqpoint{2.772386in}{2.011497in}}{\pgfqpoint{2.778210in}{2.017321in}}%
\pgfpathcurveto{\pgfqpoint{2.784034in}{2.023145in}}{\pgfqpoint{2.787306in}{2.031045in}}{\pgfqpoint{2.787306in}{2.039282in}}%
\pgfpathcurveto{\pgfqpoint{2.787306in}{2.047518in}}{\pgfqpoint{2.784034in}{2.055418in}}{\pgfqpoint{2.778210in}{2.061242in}}%
\pgfpathcurveto{\pgfqpoint{2.772386in}{2.067066in}}{\pgfqpoint{2.764486in}{2.070338in}}{\pgfqpoint{2.756250in}{2.070338in}}%
\pgfpathcurveto{\pgfqpoint{2.748014in}{2.070338in}}{\pgfqpoint{2.740114in}{2.067066in}}{\pgfqpoint{2.734290in}{2.061242in}}%
\pgfpathcurveto{\pgfqpoint{2.728466in}{2.055418in}}{\pgfqpoint{2.725193in}{2.047518in}}{\pgfqpoint{2.725193in}{2.039282in}}%
\pgfpathcurveto{\pgfqpoint{2.725193in}{2.031045in}}{\pgfqpoint{2.728466in}{2.023145in}}{\pgfqpoint{2.734290in}{2.017321in}}%
\pgfpathcurveto{\pgfqpoint{2.740114in}{2.011497in}}{\pgfqpoint{2.748014in}{2.008225in}}{\pgfqpoint{2.756250in}{2.008225in}}%
\pgfpathclose%
\pgfusepath{stroke,fill}%
\end{pgfscope}%
\begin{pgfscope}%
\pgfpathrectangle{\pgfqpoint{0.100000in}{0.212622in}}{\pgfqpoint{3.696000in}{3.696000in}}%
\pgfusepath{clip}%
\pgfsetbuttcap%
\pgfsetroundjoin%
\definecolor{currentfill}{rgb}{0.121569,0.466667,0.705882}%
\pgfsetfillcolor{currentfill}%
\pgfsetfillopacity{0.485563}%
\pgfsetlinewidth{1.003750pt}%
\definecolor{currentstroke}{rgb}{0.121569,0.466667,0.705882}%
\pgfsetstrokecolor{currentstroke}%
\pgfsetstrokeopacity{0.485563}%
\pgfsetdash{}{0pt}%
\pgfpathmoveto{\pgfqpoint{1.235043in}{1.784314in}}%
\pgfpathcurveto{\pgfqpoint{1.243279in}{1.784314in}}{\pgfqpoint{1.251179in}{1.787586in}}{\pgfqpoint{1.257003in}{1.793410in}}%
\pgfpathcurveto{\pgfqpoint{1.262827in}{1.799234in}}{\pgfqpoint{1.266100in}{1.807134in}}{\pgfqpoint{1.266100in}{1.815371in}}%
\pgfpathcurveto{\pgfqpoint{1.266100in}{1.823607in}}{\pgfqpoint{1.262827in}{1.831507in}}{\pgfqpoint{1.257003in}{1.837331in}}%
\pgfpathcurveto{\pgfqpoint{1.251179in}{1.843155in}}{\pgfqpoint{1.243279in}{1.846427in}}{\pgfqpoint{1.235043in}{1.846427in}}%
\pgfpathcurveto{\pgfqpoint{1.226807in}{1.846427in}}{\pgfqpoint{1.218907in}{1.843155in}}{\pgfqpoint{1.213083in}{1.837331in}}%
\pgfpathcurveto{\pgfqpoint{1.207259in}{1.831507in}}{\pgfqpoint{1.203987in}{1.823607in}}{\pgfqpoint{1.203987in}{1.815371in}}%
\pgfpathcurveto{\pgfqpoint{1.203987in}{1.807134in}}{\pgfqpoint{1.207259in}{1.799234in}}{\pgfqpoint{1.213083in}{1.793410in}}%
\pgfpathcurveto{\pgfqpoint{1.218907in}{1.787586in}}{\pgfqpoint{1.226807in}{1.784314in}}{\pgfqpoint{1.235043in}{1.784314in}}%
\pgfpathclose%
\pgfusepath{stroke,fill}%
\end{pgfscope}%
\begin{pgfscope}%
\pgfpathrectangle{\pgfqpoint{0.100000in}{0.212622in}}{\pgfqpoint{3.696000in}{3.696000in}}%
\pgfusepath{clip}%
\pgfsetbuttcap%
\pgfsetroundjoin%
\definecolor{currentfill}{rgb}{0.121569,0.466667,0.705882}%
\pgfsetfillcolor{currentfill}%
\pgfsetfillopacity{0.486840}%
\pgfsetlinewidth{1.003750pt}%
\definecolor{currentstroke}{rgb}{0.121569,0.466667,0.705882}%
\pgfsetstrokecolor{currentstroke}%
\pgfsetstrokeopacity{0.486840}%
\pgfsetdash{}{0pt}%
\pgfpathmoveto{\pgfqpoint{2.763765in}{2.009778in}}%
\pgfpathcurveto{\pgfqpoint{2.772001in}{2.009778in}}{\pgfqpoint{2.779901in}{2.013050in}}{\pgfqpoint{2.785725in}{2.018874in}}%
\pgfpathcurveto{\pgfqpoint{2.791549in}{2.024698in}}{\pgfqpoint{2.794821in}{2.032598in}}{\pgfqpoint{2.794821in}{2.040834in}}%
\pgfpathcurveto{\pgfqpoint{2.794821in}{2.049070in}}{\pgfqpoint{2.791549in}{2.056971in}}{\pgfqpoint{2.785725in}{2.062794in}}%
\pgfpathcurveto{\pgfqpoint{2.779901in}{2.068618in}}{\pgfqpoint{2.772001in}{2.071891in}}{\pgfqpoint{2.763765in}{2.071891in}}%
\pgfpathcurveto{\pgfqpoint{2.755528in}{2.071891in}}{\pgfqpoint{2.747628in}{2.068618in}}{\pgfqpoint{2.741804in}{2.062794in}}%
\pgfpathcurveto{\pgfqpoint{2.735980in}{2.056971in}}{\pgfqpoint{2.732708in}{2.049070in}}{\pgfqpoint{2.732708in}{2.040834in}}%
\pgfpathcurveto{\pgfqpoint{2.732708in}{2.032598in}}{\pgfqpoint{2.735980in}{2.024698in}}{\pgfqpoint{2.741804in}{2.018874in}}%
\pgfpathcurveto{\pgfqpoint{2.747628in}{2.013050in}}{\pgfqpoint{2.755528in}{2.009778in}}{\pgfqpoint{2.763765in}{2.009778in}}%
\pgfpathclose%
\pgfusepath{stroke,fill}%
\end{pgfscope}%
\begin{pgfscope}%
\pgfpathrectangle{\pgfqpoint{0.100000in}{0.212622in}}{\pgfqpoint{3.696000in}{3.696000in}}%
\pgfusepath{clip}%
\pgfsetbuttcap%
\pgfsetroundjoin%
\definecolor{currentfill}{rgb}{0.121569,0.466667,0.705882}%
\pgfsetfillcolor{currentfill}%
\pgfsetfillopacity{0.487446}%
\pgfsetlinewidth{1.003750pt}%
\definecolor{currentstroke}{rgb}{0.121569,0.466667,0.705882}%
\pgfsetstrokecolor{currentstroke}%
\pgfsetstrokeopacity{0.487446}%
\pgfsetdash{}{0pt}%
\pgfpathmoveto{\pgfqpoint{2.767677in}{2.008890in}}%
\pgfpathcurveto{\pgfqpoint{2.775913in}{2.008890in}}{\pgfqpoint{2.783813in}{2.012162in}}{\pgfqpoint{2.789637in}{2.017986in}}%
\pgfpathcurveto{\pgfqpoint{2.795461in}{2.023810in}}{\pgfqpoint{2.798734in}{2.031710in}}{\pgfqpoint{2.798734in}{2.039946in}}%
\pgfpathcurveto{\pgfqpoint{2.798734in}{2.048182in}}{\pgfqpoint{2.795461in}{2.056082in}}{\pgfqpoint{2.789637in}{2.061906in}}%
\pgfpathcurveto{\pgfqpoint{2.783813in}{2.067730in}}{\pgfqpoint{2.775913in}{2.071003in}}{\pgfqpoint{2.767677in}{2.071003in}}%
\pgfpathcurveto{\pgfqpoint{2.759441in}{2.071003in}}{\pgfqpoint{2.751541in}{2.067730in}}{\pgfqpoint{2.745717in}{2.061906in}}%
\pgfpathcurveto{\pgfqpoint{2.739893in}{2.056082in}}{\pgfqpoint{2.736621in}{2.048182in}}{\pgfqpoint{2.736621in}{2.039946in}}%
\pgfpathcurveto{\pgfqpoint{2.736621in}{2.031710in}}{\pgfqpoint{2.739893in}{2.023810in}}{\pgfqpoint{2.745717in}{2.017986in}}%
\pgfpathcurveto{\pgfqpoint{2.751541in}{2.012162in}}{\pgfqpoint{2.759441in}{2.008890in}}{\pgfqpoint{2.767677in}{2.008890in}}%
\pgfpathclose%
\pgfusepath{stroke,fill}%
\end{pgfscope}%
\begin{pgfscope}%
\pgfpathrectangle{\pgfqpoint{0.100000in}{0.212622in}}{\pgfqpoint{3.696000in}{3.696000in}}%
\pgfusepath{clip}%
\pgfsetbuttcap%
\pgfsetroundjoin%
\definecolor{currentfill}{rgb}{0.121569,0.466667,0.705882}%
\pgfsetfillcolor{currentfill}%
\pgfsetfillopacity{0.487636}%
\pgfsetlinewidth{1.003750pt}%
\definecolor{currentstroke}{rgb}{0.121569,0.466667,0.705882}%
\pgfsetstrokecolor{currentstroke}%
\pgfsetstrokeopacity{0.487636}%
\pgfsetdash{}{0pt}%
\pgfpathmoveto{\pgfqpoint{1.231953in}{1.781749in}}%
\pgfpathcurveto{\pgfqpoint{1.240189in}{1.781749in}}{\pgfqpoint{1.248089in}{1.785021in}}{\pgfqpoint{1.253913in}{1.790845in}}%
\pgfpathcurveto{\pgfqpoint{1.259737in}{1.796669in}}{\pgfqpoint{1.263010in}{1.804569in}}{\pgfqpoint{1.263010in}{1.812805in}}%
\pgfpathcurveto{\pgfqpoint{1.263010in}{1.821041in}}{\pgfqpoint{1.259737in}{1.828941in}}{\pgfqpoint{1.253913in}{1.834765in}}%
\pgfpathcurveto{\pgfqpoint{1.248089in}{1.840589in}}{\pgfqpoint{1.240189in}{1.843862in}}{\pgfqpoint{1.231953in}{1.843862in}}%
\pgfpathcurveto{\pgfqpoint{1.223717in}{1.843862in}}{\pgfqpoint{1.215817in}{1.840589in}}{\pgfqpoint{1.209993in}{1.834765in}}%
\pgfpathcurveto{\pgfqpoint{1.204169in}{1.828941in}}{\pgfqpoint{1.200897in}{1.821041in}}{\pgfqpoint{1.200897in}{1.812805in}}%
\pgfpathcurveto{\pgfqpoint{1.200897in}{1.804569in}}{\pgfqpoint{1.204169in}{1.796669in}}{\pgfqpoint{1.209993in}{1.790845in}}%
\pgfpathcurveto{\pgfqpoint{1.215817in}{1.785021in}}{\pgfqpoint{1.223717in}{1.781749in}}{\pgfqpoint{1.231953in}{1.781749in}}%
\pgfpathclose%
\pgfusepath{stroke,fill}%
\end{pgfscope}%
\begin{pgfscope}%
\pgfpathrectangle{\pgfqpoint{0.100000in}{0.212622in}}{\pgfqpoint{3.696000in}{3.696000in}}%
\pgfusepath{clip}%
\pgfsetbuttcap%
\pgfsetroundjoin%
\definecolor{currentfill}{rgb}{0.121569,0.466667,0.705882}%
\pgfsetfillcolor{currentfill}%
\pgfsetfillopacity{0.488511}%
\pgfsetlinewidth{1.003750pt}%
\definecolor{currentstroke}{rgb}{0.121569,0.466667,0.705882}%
\pgfsetstrokecolor{currentstroke}%
\pgfsetstrokeopacity{0.488511}%
\pgfsetdash{}{0pt}%
\pgfpathmoveto{\pgfqpoint{2.772747in}{2.010010in}}%
\pgfpathcurveto{\pgfqpoint{2.780983in}{2.010010in}}{\pgfqpoint{2.788883in}{2.013283in}}{\pgfqpoint{2.794707in}{2.019107in}}%
\pgfpathcurveto{\pgfqpoint{2.800531in}{2.024931in}}{\pgfqpoint{2.803803in}{2.032831in}}{\pgfqpoint{2.803803in}{2.041067in}}%
\pgfpathcurveto{\pgfqpoint{2.803803in}{2.049303in}}{\pgfqpoint{2.800531in}{2.057203in}}{\pgfqpoint{2.794707in}{2.063027in}}%
\pgfpathcurveto{\pgfqpoint{2.788883in}{2.068851in}}{\pgfqpoint{2.780983in}{2.072123in}}{\pgfqpoint{2.772747in}{2.072123in}}%
\pgfpathcurveto{\pgfqpoint{2.764510in}{2.072123in}}{\pgfqpoint{2.756610in}{2.068851in}}{\pgfqpoint{2.750786in}{2.063027in}}%
\pgfpathcurveto{\pgfqpoint{2.744962in}{2.057203in}}{\pgfqpoint{2.741690in}{2.049303in}}{\pgfqpoint{2.741690in}{2.041067in}}%
\pgfpathcurveto{\pgfqpoint{2.741690in}{2.032831in}}{\pgfqpoint{2.744962in}{2.024931in}}{\pgfqpoint{2.750786in}{2.019107in}}%
\pgfpathcurveto{\pgfqpoint{2.756610in}{2.013283in}}{\pgfqpoint{2.764510in}{2.010010in}}{\pgfqpoint{2.772747in}{2.010010in}}%
\pgfpathclose%
\pgfusepath{stroke,fill}%
\end{pgfscope}%
\begin{pgfscope}%
\pgfpathrectangle{\pgfqpoint{0.100000in}{0.212622in}}{\pgfqpoint{3.696000in}{3.696000in}}%
\pgfusepath{clip}%
\pgfsetbuttcap%
\pgfsetroundjoin%
\definecolor{currentfill}{rgb}{0.121569,0.466667,0.705882}%
\pgfsetfillcolor{currentfill}%
\pgfsetfillopacity{0.489073}%
\pgfsetlinewidth{1.003750pt}%
\definecolor{currentstroke}{rgb}{0.121569,0.466667,0.705882}%
\pgfsetstrokecolor{currentstroke}%
\pgfsetstrokeopacity{0.489073}%
\pgfsetdash{}{0pt}%
\pgfpathmoveto{\pgfqpoint{1.226294in}{1.781167in}}%
\pgfpathcurveto{\pgfqpoint{1.234530in}{1.781167in}}{\pgfqpoint{1.242430in}{1.784439in}}{\pgfqpoint{1.248254in}{1.790263in}}%
\pgfpathcurveto{\pgfqpoint{1.254078in}{1.796087in}}{\pgfqpoint{1.257350in}{1.803987in}}{\pgfqpoint{1.257350in}{1.812224in}}%
\pgfpathcurveto{\pgfqpoint{1.257350in}{1.820460in}}{\pgfqpoint{1.254078in}{1.828360in}}{\pgfqpoint{1.248254in}{1.834184in}}%
\pgfpathcurveto{\pgfqpoint{1.242430in}{1.840008in}}{\pgfqpoint{1.234530in}{1.843280in}}{\pgfqpoint{1.226294in}{1.843280in}}%
\pgfpathcurveto{\pgfqpoint{1.218057in}{1.843280in}}{\pgfqpoint{1.210157in}{1.840008in}}{\pgfqpoint{1.204333in}{1.834184in}}%
\pgfpathcurveto{\pgfqpoint{1.198509in}{1.828360in}}{\pgfqpoint{1.195237in}{1.820460in}}{\pgfqpoint{1.195237in}{1.812224in}}%
\pgfpathcurveto{\pgfqpoint{1.195237in}{1.803987in}}{\pgfqpoint{1.198509in}{1.796087in}}{\pgfqpoint{1.204333in}{1.790263in}}%
\pgfpathcurveto{\pgfqpoint{1.210157in}{1.784439in}}{\pgfqpoint{1.218057in}{1.781167in}}{\pgfqpoint{1.226294in}{1.781167in}}%
\pgfpathclose%
\pgfusepath{stroke,fill}%
\end{pgfscope}%
\begin{pgfscope}%
\pgfpathrectangle{\pgfqpoint{0.100000in}{0.212622in}}{\pgfqpoint{3.696000in}{3.696000in}}%
\pgfusepath{clip}%
\pgfsetbuttcap%
\pgfsetroundjoin%
\definecolor{currentfill}{rgb}{0.121569,0.466667,0.705882}%
\pgfsetfillcolor{currentfill}%
\pgfsetfillopacity{0.489408}%
\pgfsetlinewidth{1.003750pt}%
\definecolor{currentstroke}{rgb}{0.121569,0.466667,0.705882}%
\pgfsetstrokecolor{currentstroke}%
\pgfsetstrokeopacity{0.489408}%
\pgfsetdash{}{0pt}%
\pgfpathmoveto{\pgfqpoint{2.777664in}{2.008061in}}%
\pgfpathcurveto{\pgfqpoint{2.785900in}{2.008061in}}{\pgfqpoint{2.793800in}{2.011333in}}{\pgfqpoint{2.799624in}{2.017157in}}%
\pgfpathcurveto{\pgfqpoint{2.805448in}{2.022981in}}{\pgfqpoint{2.808720in}{2.030881in}}{\pgfqpoint{2.808720in}{2.039118in}}%
\pgfpathcurveto{\pgfqpoint{2.808720in}{2.047354in}}{\pgfqpoint{2.805448in}{2.055254in}}{\pgfqpoint{2.799624in}{2.061078in}}%
\pgfpathcurveto{\pgfqpoint{2.793800in}{2.066902in}}{\pgfqpoint{2.785900in}{2.070174in}}{\pgfqpoint{2.777664in}{2.070174in}}%
\pgfpathcurveto{\pgfqpoint{2.769427in}{2.070174in}}{\pgfqpoint{2.761527in}{2.066902in}}{\pgfqpoint{2.755703in}{2.061078in}}%
\pgfpathcurveto{\pgfqpoint{2.749880in}{2.055254in}}{\pgfqpoint{2.746607in}{2.047354in}}{\pgfqpoint{2.746607in}{2.039118in}}%
\pgfpathcurveto{\pgfqpoint{2.746607in}{2.030881in}}{\pgfqpoint{2.749880in}{2.022981in}}{\pgfqpoint{2.755703in}{2.017157in}}%
\pgfpathcurveto{\pgfqpoint{2.761527in}{2.011333in}}{\pgfqpoint{2.769427in}{2.008061in}}{\pgfqpoint{2.777664in}{2.008061in}}%
\pgfpathclose%
\pgfusepath{stroke,fill}%
\end{pgfscope}%
\begin{pgfscope}%
\pgfpathrectangle{\pgfqpoint{0.100000in}{0.212622in}}{\pgfqpoint{3.696000in}{3.696000in}}%
\pgfusepath{clip}%
\pgfsetbuttcap%
\pgfsetroundjoin%
\definecolor{currentfill}{rgb}{0.121569,0.466667,0.705882}%
\pgfsetfillcolor{currentfill}%
\pgfsetfillopacity{0.490089}%
\pgfsetlinewidth{1.003750pt}%
\definecolor{currentstroke}{rgb}{0.121569,0.466667,0.705882}%
\pgfsetstrokecolor{currentstroke}%
\pgfsetstrokeopacity{0.490089}%
\pgfsetdash{}{0pt}%
\pgfpathmoveto{\pgfqpoint{2.780415in}{2.008378in}}%
\pgfpathcurveto{\pgfqpoint{2.788651in}{2.008378in}}{\pgfqpoint{2.796551in}{2.011650in}}{\pgfqpoint{2.802375in}{2.017474in}}%
\pgfpathcurveto{\pgfqpoint{2.808199in}{2.023298in}}{\pgfqpoint{2.811471in}{2.031198in}}{\pgfqpoint{2.811471in}{2.039434in}}%
\pgfpathcurveto{\pgfqpoint{2.811471in}{2.047671in}}{\pgfqpoint{2.808199in}{2.055571in}}{\pgfqpoint{2.802375in}{2.061395in}}%
\pgfpathcurveto{\pgfqpoint{2.796551in}{2.067219in}}{\pgfqpoint{2.788651in}{2.070491in}}{\pgfqpoint{2.780415in}{2.070491in}}%
\pgfpathcurveto{\pgfqpoint{2.772178in}{2.070491in}}{\pgfqpoint{2.764278in}{2.067219in}}{\pgfqpoint{2.758454in}{2.061395in}}%
\pgfpathcurveto{\pgfqpoint{2.752630in}{2.055571in}}{\pgfqpoint{2.749358in}{2.047671in}}{\pgfqpoint{2.749358in}{2.039434in}}%
\pgfpathcurveto{\pgfqpoint{2.749358in}{2.031198in}}{\pgfqpoint{2.752630in}{2.023298in}}{\pgfqpoint{2.758454in}{2.017474in}}%
\pgfpathcurveto{\pgfqpoint{2.764278in}{2.011650in}}{\pgfqpoint{2.772178in}{2.008378in}}{\pgfqpoint{2.780415in}{2.008378in}}%
\pgfpathclose%
\pgfusepath{stroke,fill}%
\end{pgfscope}%
\begin{pgfscope}%
\pgfpathrectangle{\pgfqpoint{0.100000in}{0.212622in}}{\pgfqpoint{3.696000in}{3.696000in}}%
\pgfusepath{clip}%
\pgfsetbuttcap%
\pgfsetroundjoin%
\definecolor{currentfill}{rgb}{0.121569,0.466667,0.705882}%
\pgfsetfillcolor{currentfill}%
\pgfsetfillopacity{0.490929}%
\pgfsetlinewidth{1.003750pt}%
\definecolor{currentstroke}{rgb}{0.121569,0.466667,0.705882}%
\pgfsetstrokecolor{currentstroke}%
\pgfsetstrokeopacity{0.490929}%
\pgfsetdash{}{0pt}%
\pgfpathmoveto{\pgfqpoint{1.224663in}{1.781348in}}%
\pgfpathcurveto{\pgfqpoint{1.232899in}{1.781348in}}{\pgfqpoint{1.240799in}{1.784621in}}{\pgfqpoint{1.246623in}{1.790445in}}%
\pgfpathcurveto{\pgfqpoint{1.252447in}{1.796268in}}{\pgfqpoint{1.255719in}{1.804169in}}{\pgfqpoint{1.255719in}{1.812405in}}%
\pgfpathcurveto{\pgfqpoint{1.255719in}{1.820641in}}{\pgfqpoint{1.252447in}{1.828541in}}{\pgfqpoint{1.246623in}{1.834365in}}%
\pgfpathcurveto{\pgfqpoint{1.240799in}{1.840189in}}{\pgfqpoint{1.232899in}{1.843461in}}{\pgfqpoint{1.224663in}{1.843461in}}%
\pgfpathcurveto{\pgfqpoint{1.216426in}{1.843461in}}{\pgfqpoint{1.208526in}{1.840189in}}{\pgfqpoint{1.202702in}{1.834365in}}%
\pgfpathcurveto{\pgfqpoint{1.196878in}{1.828541in}}{\pgfqpoint{1.193606in}{1.820641in}}{\pgfqpoint{1.193606in}{1.812405in}}%
\pgfpathcurveto{\pgfqpoint{1.193606in}{1.804169in}}{\pgfqpoint{1.196878in}{1.796268in}}{\pgfqpoint{1.202702in}{1.790445in}}%
\pgfpathcurveto{\pgfqpoint{1.208526in}{1.784621in}}{\pgfqpoint{1.216426in}{1.781348in}}{\pgfqpoint{1.224663in}{1.781348in}}%
\pgfpathclose%
\pgfusepath{stroke,fill}%
\end{pgfscope}%
\begin{pgfscope}%
\pgfpathrectangle{\pgfqpoint{0.100000in}{0.212622in}}{\pgfqpoint{3.696000in}{3.696000in}}%
\pgfusepath{clip}%
\pgfsetbuttcap%
\pgfsetroundjoin%
\definecolor{currentfill}{rgb}{0.121569,0.466667,0.705882}%
\pgfsetfillcolor{currentfill}%
\pgfsetfillopacity{0.491192}%
\pgfsetlinewidth{1.003750pt}%
\definecolor{currentstroke}{rgb}{0.121569,0.466667,0.705882}%
\pgfsetstrokecolor{currentstroke}%
\pgfsetstrokeopacity{0.491192}%
\pgfsetdash{}{0pt}%
\pgfpathmoveto{\pgfqpoint{2.784312in}{2.010326in}}%
\pgfpathcurveto{\pgfqpoint{2.792548in}{2.010326in}}{\pgfqpoint{2.800448in}{2.013598in}}{\pgfqpoint{2.806272in}{2.019422in}}%
\pgfpathcurveto{\pgfqpoint{2.812096in}{2.025246in}}{\pgfqpoint{2.815368in}{2.033146in}}{\pgfqpoint{2.815368in}{2.041382in}}%
\pgfpathcurveto{\pgfqpoint{2.815368in}{2.049619in}}{\pgfqpoint{2.812096in}{2.057519in}}{\pgfqpoint{2.806272in}{2.063343in}}%
\pgfpathcurveto{\pgfqpoint{2.800448in}{2.069167in}}{\pgfqpoint{2.792548in}{2.072439in}}{\pgfqpoint{2.784312in}{2.072439in}}%
\pgfpathcurveto{\pgfqpoint{2.776076in}{2.072439in}}{\pgfqpoint{2.768176in}{2.069167in}}{\pgfqpoint{2.762352in}{2.063343in}}%
\pgfpathcurveto{\pgfqpoint{2.756528in}{2.057519in}}{\pgfqpoint{2.753255in}{2.049619in}}{\pgfqpoint{2.753255in}{2.041382in}}%
\pgfpathcurveto{\pgfqpoint{2.753255in}{2.033146in}}{\pgfqpoint{2.756528in}{2.025246in}}{\pgfqpoint{2.762352in}{2.019422in}}%
\pgfpathcurveto{\pgfqpoint{2.768176in}{2.013598in}}{\pgfqpoint{2.776076in}{2.010326in}}{\pgfqpoint{2.784312in}{2.010326in}}%
\pgfpathclose%
\pgfusepath{stroke,fill}%
\end{pgfscope}%
\begin{pgfscope}%
\pgfpathrectangle{\pgfqpoint{0.100000in}{0.212622in}}{\pgfqpoint{3.696000in}{3.696000in}}%
\pgfusepath{clip}%
\pgfsetbuttcap%
\pgfsetroundjoin%
\definecolor{currentfill}{rgb}{0.121569,0.466667,0.705882}%
\pgfsetfillcolor{currentfill}%
\pgfsetfillopacity{0.491640}%
\pgfsetlinewidth{1.003750pt}%
\definecolor{currentstroke}{rgb}{0.121569,0.466667,0.705882}%
\pgfsetstrokecolor{currentstroke}%
\pgfsetstrokeopacity{0.491640}%
\pgfsetdash{}{0pt}%
\pgfpathmoveto{\pgfqpoint{1.221503in}{1.779383in}}%
\pgfpathcurveto{\pgfqpoint{1.229739in}{1.779383in}}{\pgfqpoint{1.237639in}{1.782656in}}{\pgfqpoint{1.243463in}{1.788480in}}%
\pgfpathcurveto{\pgfqpoint{1.249287in}{1.794304in}}{\pgfqpoint{1.252559in}{1.802204in}}{\pgfqpoint{1.252559in}{1.810440in}}%
\pgfpathcurveto{\pgfqpoint{1.252559in}{1.818676in}}{\pgfqpoint{1.249287in}{1.826576in}}{\pgfqpoint{1.243463in}{1.832400in}}%
\pgfpathcurveto{\pgfqpoint{1.237639in}{1.838224in}}{\pgfqpoint{1.229739in}{1.841496in}}{\pgfqpoint{1.221503in}{1.841496in}}%
\pgfpathcurveto{\pgfqpoint{1.213267in}{1.841496in}}{\pgfqpoint{1.205366in}{1.838224in}}{\pgfqpoint{1.199543in}{1.832400in}}%
\pgfpathcurveto{\pgfqpoint{1.193719in}{1.826576in}}{\pgfqpoint{1.190446in}{1.818676in}}{\pgfqpoint{1.190446in}{1.810440in}}%
\pgfpathcurveto{\pgfqpoint{1.190446in}{1.802204in}}{\pgfqpoint{1.193719in}{1.794304in}}{\pgfqpoint{1.199543in}{1.788480in}}%
\pgfpathcurveto{\pgfqpoint{1.205366in}{1.782656in}}{\pgfqpoint{1.213267in}{1.779383in}}{\pgfqpoint{1.221503in}{1.779383in}}%
\pgfpathclose%
\pgfusepath{stroke,fill}%
\end{pgfscope}%
\begin{pgfscope}%
\pgfpathrectangle{\pgfqpoint{0.100000in}{0.212622in}}{\pgfqpoint{3.696000in}{3.696000in}}%
\pgfusepath{clip}%
\pgfsetbuttcap%
\pgfsetroundjoin%
\definecolor{currentfill}{rgb}{0.121569,0.466667,0.705882}%
\pgfsetfillcolor{currentfill}%
\pgfsetfillopacity{0.491647}%
\pgfsetlinewidth{1.003750pt}%
\definecolor{currentstroke}{rgb}{0.121569,0.466667,0.705882}%
\pgfsetstrokecolor{currentstroke}%
\pgfsetstrokeopacity{0.491647}%
\pgfsetdash{}{0pt}%
\pgfpathmoveto{\pgfqpoint{2.786280in}{2.009952in}}%
\pgfpathcurveto{\pgfqpoint{2.794516in}{2.009952in}}{\pgfqpoint{2.802416in}{2.013224in}}{\pgfqpoint{2.808240in}{2.019048in}}%
\pgfpathcurveto{\pgfqpoint{2.814064in}{2.024872in}}{\pgfqpoint{2.817336in}{2.032772in}}{\pgfqpoint{2.817336in}{2.041009in}}%
\pgfpathcurveto{\pgfqpoint{2.817336in}{2.049245in}}{\pgfqpoint{2.814064in}{2.057145in}}{\pgfqpoint{2.808240in}{2.062969in}}%
\pgfpathcurveto{\pgfqpoint{2.802416in}{2.068793in}}{\pgfqpoint{2.794516in}{2.072065in}}{\pgfqpoint{2.786280in}{2.072065in}}%
\pgfpathcurveto{\pgfqpoint{2.778044in}{2.072065in}}{\pgfqpoint{2.770144in}{2.068793in}}{\pgfqpoint{2.764320in}{2.062969in}}%
\pgfpathcurveto{\pgfqpoint{2.758496in}{2.057145in}}{\pgfqpoint{2.755223in}{2.049245in}}{\pgfqpoint{2.755223in}{2.041009in}}%
\pgfpathcurveto{\pgfqpoint{2.755223in}{2.032772in}}{\pgfqpoint{2.758496in}{2.024872in}}{\pgfqpoint{2.764320in}{2.019048in}}%
\pgfpathcurveto{\pgfqpoint{2.770144in}{2.013224in}}{\pgfqpoint{2.778044in}{2.009952in}}{\pgfqpoint{2.786280in}{2.009952in}}%
\pgfpathclose%
\pgfusepath{stroke,fill}%
\end{pgfscope}%
\begin{pgfscope}%
\pgfpathrectangle{\pgfqpoint{0.100000in}{0.212622in}}{\pgfqpoint{3.696000in}{3.696000in}}%
\pgfusepath{clip}%
\pgfsetbuttcap%
\pgfsetroundjoin%
\definecolor{currentfill}{rgb}{0.121569,0.466667,0.705882}%
\pgfsetfillcolor{currentfill}%
\pgfsetfillopacity{0.491962}%
\pgfsetlinewidth{1.003750pt}%
\definecolor{currentstroke}{rgb}{0.121569,0.466667,0.705882}%
\pgfsetstrokecolor{currentstroke}%
\pgfsetstrokeopacity{0.491962}%
\pgfsetdash{}{0pt}%
\pgfpathmoveto{\pgfqpoint{2.787485in}{2.010480in}}%
\pgfpathcurveto{\pgfqpoint{2.795722in}{2.010480in}}{\pgfqpoint{2.803622in}{2.013753in}}{\pgfqpoint{2.809446in}{2.019577in}}%
\pgfpathcurveto{\pgfqpoint{2.815270in}{2.025401in}}{\pgfqpoint{2.818542in}{2.033301in}}{\pgfqpoint{2.818542in}{2.041537in}}%
\pgfpathcurveto{\pgfqpoint{2.818542in}{2.049773in}}{\pgfqpoint{2.815270in}{2.057673in}}{\pgfqpoint{2.809446in}{2.063497in}}%
\pgfpathcurveto{\pgfqpoint{2.803622in}{2.069321in}}{\pgfqpoint{2.795722in}{2.072593in}}{\pgfqpoint{2.787485in}{2.072593in}}%
\pgfpathcurveto{\pgfqpoint{2.779249in}{2.072593in}}{\pgfqpoint{2.771349in}{2.069321in}}{\pgfqpoint{2.765525in}{2.063497in}}%
\pgfpathcurveto{\pgfqpoint{2.759701in}{2.057673in}}{\pgfqpoint{2.756429in}{2.049773in}}{\pgfqpoint{2.756429in}{2.041537in}}%
\pgfpathcurveto{\pgfqpoint{2.756429in}{2.033301in}}{\pgfqpoint{2.759701in}{2.025401in}}{\pgfqpoint{2.765525in}{2.019577in}}%
\pgfpathcurveto{\pgfqpoint{2.771349in}{2.013753in}}{\pgfqpoint{2.779249in}{2.010480in}}{\pgfqpoint{2.787485in}{2.010480in}}%
\pgfpathclose%
\pgfusepath{stroke,fill}%
\end{pgfscope}%
\begin{pgfscope}%
\pgfpathrectangle{\pgfqpoint{0.100000in}{0.212622in}}{\pgfqpoint{3.696000in}{3.696000in}}%
\pgfusepath{clip}%
\pgfsetbuttcap%
\pgfsetroundjoin%
\definecolor{currentfill}{rgb}{0.121569,0.466667,0.705882}%
\pgfsetfillcolor{currentfill}%
\pgfsetfillopacity{0.492018}%
\pgfsetlinewidth{1.003750pt}%
\definecolor{currentstroke}{rgb}{0.121569,0.466667,0.705882}%
\pgfsetstrokecolor{currentstroke}%
\pgfsetstrokeopacity{0.492018}%
\pgfsetdash{}{0pt}%
\pgfpathmoveto{\pgfqpoint{1.220447in}{1.778045in}}%
\pgfpathcurveto{\pgfqpoint{1.228683in}{1.778045in}}{\pgfqpoint{1.236583in}{1.781317in}}{\pgfqpoint{1.242407in}{1.787141in}}%
\pgfpathcurveto{\pgfqpoint{1.248231in}{1.792965in}}{\pgfqpoint{1.251504in}{1.800865in}}{\pgfqpoint{1.251504in}{1.809101in}}%
\pgfpathcurveto{\pgfqpoint{1.251504in}{1.817338in}}{\pgfqpoint{1.248231in}{1.825238in}}{\pgfqpoint{1.242407in}{1.831062in}}%
\pgfpathcurveto{\pgfqpoint{1.236583in}{1.836886in}}{\pgfqpoint{1.228683in}{1.840158in}}{\pgfqpoint{1.220447in}{1.840158in}}%
\pgfpathcurveto{\pgfqpoint{1.212211in}{1.840158in}}{\pgfqpoint{1.204311in}{1.836886in}}{\pgfqpoint{1.198487in}{1.831062in}}%
\pgfpathcurveto{\pgfqpoint{1.192663in}{1.825238in}}{\pgfqpoint{1.189391in}{1.817338in}}{\pgfqpoint{1.189391in}{1.809101in}}%
\pgfpathcurveto{\pgfqpoint{1.189391in}{1.800865in}}{\pgfqpoint{1.192663in}{1.792965in}}{\pgfqpoint{1.198487in}{1.787141in}}%
\pgfpathcurveto{\pgfqpoint{1.204311in}{1.781317in}}{\pgfqpoint{1.212211in}{1.778045in}}{\pgfqpoint{1.220447in}{1.778045in}}%
\pgfpathclose%
\pgfusepath{stroke,fill}%
\end{pgfscope}%
\begin{pgfscope}%
\pgfpathrectangle{\pgfqpoint{0.100000in}{0.212622in}}{\pgfqpoint{3.696000in}{3.696000in}}%
\pgfusepath{clip}%
\pgfsetbuttcap%
\pgfsetroundjoin%
\definecolor{currentfill}{rgb}{0.121569,0.466667,0.705882}%
\pgfsetfillcolor{currentfill}%
\pgfsetfillopacity{0.492261}%
\pgfsetlinewidth{1.003750pt}%
\definecolor{currentstroke}{rgb}{0.121569,0.466667,0.705882}%
\pgfsetstrokecolor{currentstroke}%
\pgfsetstrokeopacity{0.492261}%
\pgfsetdash{}{0pt}%
\pgfpathmoveto{\pgfqpoint{2.789098in}{2.010423in}}%
\pgfpathcurveto{\pgfqpoint{2.797334in}{2.010423in}}{\pgfqpoint{2.805234in}{2.013695in}}{\pgfqpoint{2.811058in}{2.019519in}}%
\pgfpathcurveto{\pgfqpoint{2.816882in}{2.025343in}}{\pgfqpoint{2.820154in}{2.033243in}}{\pgfqpoint{2.820154in}{2.041479in}}%
\pgfpathcurveto{\pgfqpoint{2.820154in}{2.049716in}}{\pgfqpoint{2.816882in}{2.057616in}}{\pgfqpoint{2.811058in}{2.063440in}}%
\pgfpathcurveto{\pgfqpoint{2.805234in}{2.069263in}}{\pgfqpoint{2.797334in}{2.072536in}}{\pgfqpoint{2.789098in}{2.072536in}}%
\pgfpathcurveto{\pgfqpoint{2.780862in}{2.072536in}}{\pgfqpoint{2.772962in}{2.069263in}}{\pgfqpoint{2.767138in}{2.063440in}}%
\pgfpathcurveto{\pgfqpoint{2.761314in}{2.057616in}}{\pgfqpoint{2.758041in}{2.049716in}}{\pgfqpoint{2.758041in}{2.041479in}}%
\pgfpathcurveto{\pgfqpoint{2.758041in}{2.033243in}}{\pgfqpoint{2.761314in}{2.025343in}}{\pgfqpoint{2.767138in}{2.019519in}}%
\pgfpathcurveto{\pgfqpoint{2.772962in}{2.013695in}}{\pgfqpoint{2.780862in}{2.010423in}}{\pgfqpoint{2.789098in}{2.010423in}}%
\pgfpathclose%
\pgfusepath{stroke,fill}%
\end{pgfscope}%
\begin{pgfscope}%
\pgfpathrectangle{\pgfqpoint{0.100000in}{0.212622in}}{\pgfqpoint{3.696000in}{3.696000in}}%
\pgfusepath{clip}%
\pgfsetbuttcap%
\pgfsetroundjoin%
\definecolor{currentfill}{rgb}{0.121569,0.466667,0.705882}%
\pgfsetfillcolor{currentfill}%
\pgfsetfillopacity{0.492624}%
\pgfsetlinewidth{1.003750pt}%
\definecolor{currentstroke}{rgb}{0.121569,0.466667,0.705882}%
\pgfsetstrokecolor{currentstroke}%
\pgfsetstrokeopacity{0.492624}%
\pgfsetdash{}{0pt}%
\pgfpathmoveto{\pgfqpoint{1.218416in}{1.775152in}}%
\pgfpathcurveto{\pgfqpoint{1.226652in}{1.775152in}}{\pgfqpoint{1.234552in}{1.778425in}}{\pgfqpoint{1.240376in}{1.784249in}}%
\pgfpathcurveto{\pgfqpoint{1.246200in}{1.790073in}}{\pgfqpoint{1.249472in}{1.797973in}}{\pgfqpoint{1.249472in}{1.806209in}}%
\pgfpathcurveto{\pgfqpoint{1.249472in}{1.814445in}}{\pgfqpoint{1.246200in}{1.822345in}}{\pgfqpoint{1.240376in}{1.828169in}}%
\pgfpathcurveto{\pgfqpoint{1.234552in}{1.833993in}}{\pgfqpoint{1.226652in}{1.837265in}}{\pgfqpoint{1.218416in}{1.837265in}}%
\pgfpathcurveto{\pgfqpoint{1.210180in}{1.837265in}}{\pgfqpoint{1.202280in}{1.833993in}}{\pgfqpoint{1.196456in}{1.828169in}}%
\pgfpathcurveto{\pgfqpoint{1.190632in}{1.822345in}}{\pgfqpoint{1.187359in}{1.814445in}}{\pgfqpoint{1.187359in}{1.806209in}}%
\pgfpathcurveto{\pgfqpoint{1.187359in}{1.797973in}}{\pgfqpoint{1.190632in}{1.790073in}}{\pgfqpoint{1.196456in}{1.784249in}}%
\pgfpathcurveto{\pgfqpoint{1.202280in}{1.778425in}}{\pgfqpoint{1.210180in}{1.775152in}}{\pgfqpoint{1.218416in}{1.775152in}}%
\pgfpathclose%
\pgfusepath{stroke,fill}%
\end{pgfscope}%
\begin{pgfscope}%
\pgfpathrectangle{\pgfqpoint{0.100000in}{0.212622in}}{\pgfqpoint{3.696000in}{3.696000in}}%
\pgfusepath{clip}%
\pgfsetbuttcap%
\pgfsetroundjoin%
\definecolor{currentfill}{rgb}{0.121569,0.466667,0.705882}%
\pgfsetfillcolor{currentfill}%
\pgfsetfillopacity{0.492741}%
\pgfsetlinewidth{1.003750pt}%
\definecolor{currentstroke}{rgb}{0.121569,0.466667,0.705882}%
\pgfsetstrokecolor{currentstroke}%
\pgfsetstrokeopacity{0.492741}%
\pgfsetdash{}{0pt}%
\pgfpathmoveto{\pgfqpoint{2.791232in}{2.011603in}}%
\pgfpathcurveto{\pgfqpoint{2.799468in}{2.011603in}}{\pgfqpoint{2.807368in}{2.014875in}}{\pgfqpoint{2.813192in}{2.020699in}}%
\pgfpathcurveto{\pgfqpoint{2.819016in}{2.026523in}}{\pgfqpoint{2.822288in}{2.034423in}}{\pgfqpoint{2.822288in}{2.042659in}}%
\pgfpathcurveto{\pgfqpoint{2.822288in}{2.050895in}}{\pgfqpoint{2.819016in}{2.058795in}}{\pgfqpoint{2.813192in}{2.064619in}}%
\pgfpathcurveto{\pgfqpoint{2.807368in}{2.070443in}}{\pgfqpoint{2.799468in}{2.073716in}}{\pgfqpoint{2.791232in}{2.073716in}}%
\pgfpathcurveto{\pgfqpoint{2.782995in}{2.073716in}}{\pgfqpoint{2.775095in}{2.070443in}}{\pgfqpoint{2.769271in}{2.064619in}}%
\pgfpathcurveto{\pgfqpoint{2.763447in}{2.058795in}}{\pgfqpoint{2.760175in}{2.050895in}}{\pgfqpoint{2.760175in}{2.042659in}}%
\pgfpathcurveto{\pgfqpoint{2.760175in}{2.034423in}}{\pgfqpoint{2.763447in}{2.026523in}}{\pgfqpoint{2.769271in}{2.020699in}}%
\pgfpathcurveto{\pgfqpoint{2.775095in}{2.014875in}}{\pgfqpoint{2.782995in}{2.011603in}}{\pgfqpoint{2.791232in}{2.011603in}}%
\pgfpathclose%
\pgfusepath{stroke,fill}%
\end{pgfscope}%
\begin{pgfscope}%
\pgfpathrectangle{\pgfqpoint{0.100000in}{0.212622in}}{\pgfqpoint{3.696000in}{3.696000in}}%
\pgfusepath{clip}%
\pgfsetbuttcap%
\pgfsetroundjoin%
\definecolor{currentfill}{rgb}{0.121569,0.466667,0.705882}%
\pgfsetfillcolor{currentfill}%
\pgfsetfillopacity{0.493011}%
\pgfsetlinewidth{1.003750pt}%
\definecolor{currentstroke}{rgb}{0.121569,0.466667,0.705882}%
\pgfsetstrokecolor{currentstroke}%
\pgfsetstrokeopacity{0.493011}%
\pgfsetdash{}{0pt}%
\pgfpathmoveto{\pgfqpoint{2.794388in}{2.009923in}}%
\pgfpathcurveto{\pgfqpoint{2.802624in}{2.009923in}}{\pgfqpoint{2.810524in}{2.013196in}}{\pgfqpoint{2.816348in}{2.019020in}}%
\pgfpathcurveto{\pgfqpoint{2.822172in}{2.024844in}}{\pgfqpoint{2.825444in}{2.032744in}}{\pgfqpoint{2.825444in}{2.040980in}}%
\pgfpathcurveto{\pgfqpoint{2.825444in}{2.049216in}}{\pgfqpoint{2.822172in}{2.057116in}}{\pgfqpoint{2.816348in}{2.062940in}}%
\pgfpathcurveto{\pgfqpoint{2.810524in}{2.068764in}}{\pgfqpoint{2.802624in}{2.072036in}}{\pgfqpoint{2.794388in}{2.072036in}}%
\pgfpathcurveto{\pgfqpoint{2.786152in}{2.072036in}}{\pgfqpoint{2.778252in}{2.068764in}}{\pgfqpoint{2.772428in}{2.062940in}}%
\pgfpathcurveto{\pgfqpoint{2.766604in}{2.057116in}}{\pgfqpoint{2.763331in}{2.049216in}}{\pgfqpoint{2.763331in}{2.040980in}}%
\pgfpathcurveto{\pgfqpoint{2.763331in}{2.032744in}}{\pgfqpoint{2.766604in}{2.024844in}}{\pgfqpoint{2.772428in}{2.019020in}}%
\pgfpathcurveto{\pgfqpoint{2.778252in}{2.013196in}}{\pgfqpoint{2.786152in}{2.009923in}}{\pgfqpoint{2.794388in}{2.009923in}}%
\pgfpathclose%
\pgfusepath{stroke,fill}%
\end{pgfscope}%
\begin{pgfscope}%
\pgfpathrectangle{\pgfqpoint{0.100000in}{0.212622in}}{\pgfqpoint{3.696000in}{3.696000in}}%
\pgfusepath{clip}%
\pgfsetbuttcap%
\pgfsetroundjoin%
\definecolor{currentfill}{rgb}{0.121569,0.466667,0.705882}%
\pgfsetfillcolor{currentfill}%
\pgfsetfillopacity{0.493356}%
\pgfsetlinewidth{1.003750pt}%
\definecolor{currentstroke}{rgb}{0.121569,0.466667,0.705882}%
\pgfsetstrokecolor{currentstroke}%
\pgfsetstrokeopacity{0.493356}%
\pgfsetdash{}{0pt}%
\pgfpathmoveto{\pgfqpoint{1.215921in}{1.774511in}}%
\pgfpathcurveto{\pgfqpoint{1.224157in}{1.774511in}}{\pgfqpoint{1.232057in}{1.777783in}}{\pgfqpoint{1.237881in}{1.783607in}}%
\pgfpathcurveto{\pgfqpoint{1.243705in}{1.789431in}}{\pgfqpoint{1.246978in}{1.797331in}}{\pgfqpoint{1.246978in}{1.805567in}}%
\pgfpathcurveto{\pgfqpoint{1.246978in}{1.813803in}}{\pgfqpoint{1.243705in}{1.821704in}}{\pgfqpoint{1.237881in}{1.827527in}}%
\pgfpathcurveto{\pgfqpoint{1.232057in}{1.833351in}}{\pgfqpoint{1.224157in}{1.836624in}}{\pgfqpoint{1.215921in}{1.836624in}}%
\pgfpathcurveto{\pgfqpoint{1.207685in}{1.836624in}}{\pgfqpoint{1.199785in}{1.833351in}}{\pgfqpoint{1.193961in}{1.827527in}}%
\pgfpathcurveto{\pgfqpoint{1.188137in}{1.821704in}}{\pgfqpoint{1.184865in}{1.813803in}}{\pgfqpoint{1.184865in}{1.805567in}}%
\pgfpathcurveto{\pgfqpoint{1.184865in}{1.797331in}}{\pgfqpoint{1.188137in}{1.789431in}}{\pgfqpoint{1.193961in}{1.783607in}}%
\pgfpathcurveto{\pgfqpoint{1.199785in}{1.777783in}}{\pgfqpoint{1.207685in}{1.774511in}}{\pgfqpoint{1.215921in}{1.774511in}}%
\pgfpathclose%
\pgfusepath{stroke,fill}%
\end{pgfscope}%
\begin{pgfscope}%
\pgfpathrectangle{\pgfqpoint{0.100000in}{0.212622in}}{\pgfqpoint{3.696000in}{3.696000in}}%
\pgfusepath{clip}%
\pgfsetbuttcap%
\pgfsetroundjoin%
\definecolor{currentfill}{rgb}{0.121569,0.466667,0.705882}%
\pgfsetfillcolor{currentfill}%
\pgfsetfillopacity{0.493725}%
\pgfsetlinewidth{1.003750pt}%
\definecolor{currentstroke}{rgb}{0.121569,0.466667,0.705882}%
\pgfsetstrokecolor{currentstroke}%
\pgfsetstrokeopacity{0.493725}%
\pgfsetdash{}{0pt}%
\pgfpathmoveto{\pgfqpoint{1.215345in}{1.773964in}}%
\pgfpathcurveto{\pgfqpoint{1.223581in}{1.773964in}}{\pgfqpoint{1.231481in}{1.777236in}}{\pgfqpoint{1.237305in}{1.783060in}}%
\pgfpathcurveto{\pgfqpoint{1.243129in}{1.788884in}}{\pgfqpoint{1.246401in}{1.796784in}}{\pgfqpoint{1.246401in}{1.805020in}}%
\pgfpathcurveto{\pgfqpoint{1.246401in}{1.813256in}}{\pgfqpoint{1.243129in}{1.821156in}}{\pgfqpoint{1.237305in}{1.826980in}}%
\pgfpathcurveto{\pgfqpoint{1.231481in}{1.832804in}}{\pgfqpoint{1.223581in}{1.836077in}}{\pgfqpoint{1.215345in}{1.836077in}}%
\pgfpathcurveto{\pgfqpoint{1.207108in}{1.836077in}}{\pgfqpoint{1.199208in}{1.832804in}}{\pgfqpoint{1.193384in}{1.826980in}}%
\pgfpathcurveto{\pgfqpoint{1.187560in}{1.821156in}}{\pgfqpoint{1.184288in}{1.813256in}}{\pgfqpoint{1.184288in}{1.805020in}}%
\pgfpathcurveto{\pgfqpoint{1.184288in}{1.796784in}}{\pgfqpoint{1.187560in}{1.788884in}}{\pgfqpoint{1.193384in}{1.783060in}}%
\pgfpathcurveto{\pgfqpoint{1.199208in}{1.777236in}}{\pgfqpoint{1.207108in}{1.773964in}}{\pgfqpoint{1.215345in}{1.773964in}}%
\pgfpathclose%
\pgfusepath{stroke,fill}%
\end{pgfscope}%
\begin{pgfscope}%
\pgfpathrectangle{\pgfqpoint{0.100000in}{0.212622in}}{\pgfqpoint{3.696000in}{3.696000in}}%
\pgfusepath{clip}%
\pgfsetbuttcap%
\pgfsetroundjoin%
\definecolor{currentfill}{rgb}{0.121569,0.466667,0.705882}%
\pgfsetfillcolor{currentfill}%
\pgfsetfillopacity{0.493829}%
\pgfsetlinewidth{1.003750pt}%
\definecolor{currentstroke}{rgb}{0.121569,0.466667,0.705882}%
\pgfsetstrokecolor{currentstroke}%
\pgfsetstrokeopacity{0.493829}%
\pgfsetdash{}{0pt}%
\pgfpathmoveto{\pgfqpoint{2.799288in}{2.012382in}}%
\pgfpathcurveto{\pgfqpoint{2.807524in}{2.012382in}}{\pgfqpoint{2.815424in}{2.015654in}}{\pgfqpoint{2.821248in}{2.021478in}}%
\pgfpathcurveto{\pgfqpoint{2.827072in}{2.027302in}}{\pgfqpoint{2.830344in}{2.035202in}}{\pgfqpoint{2.830344in}{2.043438in}}%
\pgfpathcurveto{\pgfqpoint{2.830344in}{2.051675in}}{\pgfqpoint{2.827072in}{2.059575in}}{\pgfqpoint{2.821248in}{2.065398in}}%
\pgfpathcurveto{\pgfqpoint{2.815424in}{2.071222in}}{\pgfqpoint{2.807524in}{2.074495in}}{\pgfqpoint{2.799288in}{2.074495in}}%
\pgfpathcurveto{\pgfqpoint{2.791051in}{2.074495in}}{\pgfqpoint{2.783151in}{2.071222in}}{\pgfqpoint{2.777327in}{2.065398in}}%
\pgfpathcurveto{\pgfqpoint{2.771503in}{2.059575in}}{\pgfqpoint{2.768231in}{2.051675in}}{\pgfqpoint{2.768231in}{2.043438in}}%
\pgfpathcurveto{\pgfqpoint{2.768231in}{2.035202in}}{\pgfqpoint{2.771503in}{2.027302in}}{\pgfqpoint{2.777327in}{2.021478in}}%
\pgfpathcurveto{\pgfqpoint{2.783151in}{2.015654in}}{\pgfqpoint{2.791051in}{2.012382in}}{\pgfqpoint{2.799288in}{2.012382in}}%
\pgfpathclose%
\pgfusepath{stroke,fill}%
\end{pgfscope}%
\begin{pgfscope}%
\pgfpathrectangle{\pgfqpoint{0.100000in}{0.212622in}}{\pgfqpoint{3.696000in}{3.696000in}}%
\pgfusepath{clip}%
\pgfsetbuttcap%
\pgfsetroundjoin%
\definecolor{currentfill}{rgb}{0.121569,0.466667,0.705882}%
\pgfsetfillcolor{currentfill}%
\pgfsetfillopacity{0.493991}%
\pgfsetlinewidth{1.003750pt}%
\definecolor{currentstroke}{rgb}{0.121569,0.466667,0.705882}%
\pgfsetstrokecolor{currentstroke}%
\pgfsetstrokeopacity{0.493991}%
\pgfsetdash{}{0pt}%
\pgfpathmoveto{\pgfqpoint{1.214426in}{1.774023in}}%
\pgfpathcurveto{\pgfqpoint{1.222662in}{1.774023in}}{\pgfqpoint{1.230562in}{1.777296in}}{\pgfqpoint{1.236386in}{1.783119in}}%
\pgfpathcurveto{\pgfqpoint{1.242210in}{1.788943in}}{\pgfqpoint{1.245483in}{1.796843in}}{\pgfqpoint{1.245483in}{1.805080in}}%
\pgfpathcurveto{\pgfqpoint{1.245483in}{1.813316in}}{\pgfqpoint{1.242210in}{1.821216in}}{\pgfqpoint{1.236386in}{1.827040in}}%
\pgfpathcurveto{\pgfqpoint{1.230562in}{1.832864in}}{\pgfqpoint{1.222662in}{1.836136in}}{\pgfqpoint{1.214426in}{1.836136in}}%
\pgfpathcurveto{\pgfqpoint{1.206190in}{1.836136in}}{\pgfqpoint{1.198290in}{1.832864in}}{\pgfqpoint{1.192466in}{1.827040in}}%
\pgfpathcurveto{\pgfqpoint{1.186642in}{1.821216in}}{\pgfqpoint{1.183370in}{1.813316in}}{\pgfqpoint{1.183370in}{1.805080in}}%
\pgfpathcurveto{\pgfqpoint{1.183370in}{1.796843in}}{\pgfqpoint{1.186642in}{1.788943in}}{\pgfqpoint{1.192466in}{1.783119in}}%
\pgfpathcurveto{\pgfqpoint{1.198290in}{1.777296in}}{\pgfqpoint{1.206190in}{1.774023in}}{\pgfqpoint{1.214426in}{1.774023in}}%
\pgfpathclose%
\pgfusepath{stroke,fill}%
\end{pgfscope}%
\begin{pgfscope}%
\pgfpathrectangle{\pgfqpoint{0.100000in}{0.212622in}}{\pgfqpoint{3.696000in}{3.696000in}}%
\pgfusepath{clip}%
\pgfsetbuttcap%
\pgfsetroundjoin%
\definecolor{currentfill}{rgb}{0.121569,0.466667,0.705882}%
\pgfsetfillcolor{currentfill}%
\pgfsetfillopacity{0.494255}%
\pgfsetlinewidth{1.003750pt}%
\definecolor{currentstroke}{rgb}{0.121569,0.466667,0.705882}%
\pgfsetstrokecolor{currentstroke}%
\pgfsetstrokeopacity{0.494255}%
\pgfsetdash{}{0pt}%
\pgfpathmoveto{\pgfqpoint{2.805428in}{2.008394in}}%
\pgfpathcurveto{\pgfqpoint{2.813664in}{2.008394in}}{\pgfqpoint{2.821564in}{2.011667in}}{\pgfqpoint{2.827388in}{2.017491in}}%
\pgfpathcurveto{\pgfqpoint{2.833212in}{2.023315in}}{\pgfqpoint{2.836484in}{2.031215in}}{\pgfqpoint{2.836484in}{2.039451in}}%
\pgfpathcurveto{\pgfqpoint{2.836484in}{2.047687in}}{\pgfqpoint{2.833212in}{2.055587in}}{\pgfqpoint{2.827388in}{2.061411in}}%
\pgfpathcurveto{\pgfqpoint{2.821564in}{2.067235in}}{\pgfqpoint{2.813664in}{2.070507in}}{\pgfqpoint{2.805428in}{2.070507in}}%
\pgfpathcurveto{\pgfqpoint{2.797191in}{2.070507in}}{\pgfqpoint{2.789291in}{2.067235in}}{\pgfqpoint{2.783467in}{2.061411in}}%
\pgfpathcurveto{\pgfqpoint{2.777644in}{2.055587in}}{\pgfqpoint{2.774371in}{2.047687in}}{\pgfqpoint{2.774371in}{2.039451in}}%
\pgfpathcurveto{\pgfqpoint{2.774371in}{2.031215in}}{\pgfqpoint{2.777644in}{2.023315in}}{\pgfqpoint{2.783467in}{2.017491in}}%
\pgfpathcurveto{\pgfqpoint{2.789291in}{2.011667in}}{\pgfqpoint{2.797191in}{2.008394in}}{\pgfqpoint{2.805428in}{2.008394in}}%
\pgfpathclose%
\pgfusepath{stroke,fill}%
\end{pgfscope}%
\begin{pgfscope}%
\pgfpathrectangle{\pgfqpoint{0.100000in}{0.212622in}}{\pgfqpoint{3.696000in}{3.696000in}}%
\pgfusepath{clip}%
\pgfsetbuttcap%
\pgfsetroundjoin%
\definecolor{currentfill}{rgb}{0.121569,0.466667,0.705882}%
\pgfsetfillcolor{currentfill}%
\pgfsetfillopacity{0.494671}%
\pgfsetlinewidth{1.003750pt}%
\definecolor{currentstroke}{rgb}{0.121569,0.466667,0.705882}%
\pgfsetstrokecolor{currentstroke}%
\pgfsetstrokeopacity{0.494671}%
\pgfsetdash{}{0pt}%
\pgfpathmoveto{\pgfqpoint{1.214012in}{1.774328in}}%
\pgfpathcurveto{\pgfqpoint{1.222249in}{1.774328in}}{\pgfqpoint{1.230149in}{1.777600in}}{\pgfqpoint{1.235973in}{1.783424in}}%
\pgfpathcurveto{\pgfqpoint{1.241797in}{1.789248in}}{\pgfqpoint{1.245069in}{1.797148in}}{\pgfqpoint{1.245069in}{1.805384in}}%
\pgfpathcurveto{\pgfqpoint{1.245069in}{1.813621in}}{\pgfqpoint{1.241797in}{1.821521in}}{\pgfqpoint{1.235973in}{1.827345in}}%
\pgfpathcurveto{\pgfqpoint{1.230149in}{1.833169in}}{\pgfqpoint{1.222249in}{1.836441in}}{\pgfqpoint{1.214012in}{1.836441in}}%
\pgfpathcurveto{\pgfqpoint{1.205776in}{1.836441in}}{\pgfqpoint{1.197876in}{1.833169in}}{\pgfqpoint{1.192052in}{1.827345in}}%
\pgfpathcurveto{\pgfqpoint{1.186228in}{1.821521in}}{\pgfqpoint{1.182956in}{1.813621in}}{\pgfqpoint{1.182956in}{1.805384in}}%
\pgfpathcurveto{\pgfqpoint{1.182956in}{1.797148in}}{\pgfqpoint{1.186228in}{1.789248in}}{\pgfqpoint{1.192052in}{1.783424in}}%
\pgfpathcurveto{\pgfqpoint{1.197876in}{1.777600in}}{\pgfqpoint{1.205776in}{1.774328in}}{\pgfqpoint{1.214012in}{1.774328in}}%
\pgfpathclose%
\pgfusepath{stroke,fill}%
\end{pgfscope}%
\begin{pgfscope}%
\pgfpathrectangle{\pgfqpoint{0.100000in}{0.212622in}}{\pgfqpoint{3.696000in}{3.696000in}}%
\pgfusepath{clip}%
\pgfsetbuttcap%
\pgfsetroundjoin%
\definecolor{currentfill}{rgb}{0.121569,0.466667,0.705882}%
\pgfsetfillcolor{currentfill}%
\pgfsetfillopacity{0.495097}%
\pgfsetlinewidth{1.003750pt}%
\definecolor{currentstroke}{rgb}{0.121569,0.466667,0.705882}%
\pgfsetstrokecolor{currentstroke}%
\pgfsetstrokeopacity{0.495097}%
\pgfsetdash{}{0pt}%
\pgfpathmoveto{\pgfqpoint{1.211796in}{1.770119in}}%
\pgfpathcurveto{\pgfqpoint{1.220032in}{1.770119in}}{\pgfqpoint{1.227932in}{1.773392in}}{\pgfqpoint{1.233756in}{1.779215in}}%
\pgfpathcurveto{\pgfqpoint{1.239580in}{1.785039in}}{\pgfqpoint{1.242852in}{1.792939in}}{\pgfqpoint{1.242852in}{1.801176in}}%
\pgfpathcurveto{\pgfqpoint{1.242852in}{1.809412in}}{\pgfqpoint{1.239580in}{1.817312in}}{\pgfqpoint{1.233756in}{1.823136in}}%
\pgfpathcurveto{\pgfqpoint{1.227932in}{1.828960in}}{\pgfqpoint{1.220032in}{1.832232in}}{\pgfqpoint{1.211796in}{1.832232in}}%
\pgfpathcurveto{\pgfqpoint{1.203560in}{1.832232in}}{\pgfqpoint{1.195660in}{1.828960in}}{\pgfqpoint{1.189836in}{1.823136in}}%
\pgfpathcurveto{\pgfqpoint{1.184012in}{1.817312in}}{\pgfqpoint{1.180739in}{1.809412in}}{\pgfqpoint{1.180739in}{1.801176in}}%
\pgfpathcurveto{\pgfqpoint{1.180739in}{1.792939in}}{\pgfqpoint{1.184012in}{1.785039in}}{\pgfqpoint{1.189836in}{1.779215in}}%
\pgfpathcurveto{\pgfqpoint{1.195660in}{1.773392in}}{\pgfqpoint{1.203560in}{1.770119in}}{\pgfqpoint{1.211796in}{1.770119in}}%
\pgfpathclose%
\pgfusepath{stroke,fill}%
\end{pgfscope}%
\begin{pgfscope}%
\pgfpathrectangle{\pgfqpoint{0.100000in}{0.212622in}}{\pgfqpoint{3.696000in}{3.696000in}}%
\pgfusepath{clip}%
\pgfsetbuttcap%
\pgfsetroundjoin%
\definecolor{currentfill}{rgb}{0.121569,0.466667,0.705882}%
\pgfsetfillcolor{currentfill}%
\pgfsetfillopacity{0.495893}%
\pgfsetlinewidth{1.003750pt}%
\definecolor{currentstroke}{rgb}{0.121569,0.466667,0.705882}%
\pgfsetstrokecolor{currentstroke}%
\pgfsetstrokeopacity{0.495893}%
\pgfsetdash{}{0pt}%
\pgfpathmoveto{\pgfqpoint{2.812883in}{2.013349in}}%
\pgfpathcurveto{\pgfqpoint{2.821119in}{2.013349in}}{\pgfqpoint{2.829019in}{2.016622in}}{\pgfqpoint{2.834843in}{2.022445in}}%
\pgfpathcurveto{\pgfqpoint{2.840667in}{2.028269in}}{\pgfqpoint{2.843940in}{2.036169in}}{\pgfqpoint{2.843940in}{2.044406in}}%
\pgfpathcurveto{\pgfqpoint{2.843940in}{2.052642in}}{\pgfqpoint{2.840667in}{2.060542in}}{\pgfqpoint{2.834843in}{2.066366in}}%
\pgfpathcurveto{\pgfqpoint{2.829019in}{2.072190in}}{\pgfqpoint{2.821119in}{2.075462in}}{\pgfqpoint{2.812883in}{2.075462in}}%
\pgfpathcurveto{\pgfqpoint{2.804647in}{2.075462in}}{\pgfqpoint{2.796747in}{2.072190in}}{\pgfqpoint{2.790923in}{2.066366in}}%
\pgfpathcurveto{\pgfqpoint{2.785099in}{2.060542in}}{\pgfqpoint{2.781827in}{2.052642in}}{\pgfqpoint{2.781827in}{2.044406in}}%
\pgfpathcurveto{\pgfqpoint{2.781827in}{2.036169in}}{\pgfqpoint{2.785099in}{2.028269in}}{\pgfqpoint{2.790923in}{2.022445in}}%
\pgfpathcurveto{\pgfqpoint{2.796747in}{2.016622in}}{\pgfqpoint{2.804647in}{2.013349in}}{\pgfqpoint{2.812883in}{2.013349in}}%
\pgfpathclose%
\pgfusepath{stroke,fill}%
\end{pgfscope}%
\begin{pgfscope}%
\pgfpathrectangle{\pgfqpoint{0.100000in}{0.212622in}}{\pgfqpoint{3.696000in}{3.696000in}}%
\pgfusepath{clip}%
\pgfsetbuttcap%
\pgfsetroundjoin%
\definecolor{currentfill}{rgb}{0.121569,0.466667,0.705882}%
\pgfsetfillcolor{currentfill}%
\pgfsetfillopacity{0.496040}%
\pgfsetlinewidth{1.003750pt}%
\definecolor{currentstroke}{rgb}{0.121569,0.466667,0.705882}%
\pgfsetstrokecolor{currentstroke}%
\pgfsetstrokeopacity{0.496040}%
\pgfsetdash{}{0pt}%
\pgfpathmoveto{\pgfqpoint{2.816298in}{2.008859in}}%
\pgfpathcurveto{\pgfqpoint{2.824534in}{2.008859in}}{\pgfqpoint{2.832434in}{2.012131in}}{\pgfqpoint{2.838258in}{2.017955in}}%
\pgfpathcurveto{\pgfqpoint{2.844082in}{2.023779in}}{\pgfqpoint{2.847354in}{2.031679in}}{\pgfqpoint{2.847354in}{2.039915in}}%
\pgfpathcurveto{\pgfqpoint{2.847354in}{2.048152in}}{\pgfqpoint{2.844082in}{2.056052in}}{\pgfqpoint{2.838258in}{2.061876in}}%
\pgfpathcurveto{\pgfqpoint{2.832434in}{2.067700in}}{\pgfqpoint{2.824534in}{2.070972in}}{\pgfqpoint{2.816298in}{2.070972in}}%
\pgfpathcurveto{\pgfqpoint{2.808061in}{2.070972in}}{\pgfqpoint{2.800161in}{2.067700in}}{\pgfqpoint{2.794338in}{2.061876in}}%
\pgfpathcurveto{\pgfqpoint{2.788514in}{2.056052in}}{\pgfqpoint{2.785241in}{2.048152in}}{\pgfqpoint{2.785241in}{2.039915in}}%
\pgfpathcurveto{\pgfqpoint{2.785241in}{2.031679in}}{\pgfqpoint{2.788514in}{2.023779in}}{\pgfqpoint{2.794338in}{2.017955in}}%
\pgfpathcurveto{\pgfqpoint{2.800161in}{2.012131in}}{\pgfqpoint{2.808061in}{2.008859in}}{\pgfqpoint{2.816298in}{2.008859in}}%
\pgfpathclose%
\pgfusepath{stroke,fill}%
\end{pgfscope}%
\begin{pgfscope}%
\pgfpathrectangle{\pgfqpoint{0.100000in}{0.212622in}}{\pgfqpoint{3.696000in}{3.696000in}}%
\pgfusepath{clip}%
\pgfsetbuttcap%
\pgfsetroundjoin%
\definecolor{currentfill}{rgb}{0.121569,0.466667,0.705882}%
\pgfsetfillcolor{currentfill}%
\pgfsetfillopacity{0.496485}%
\pgfsetlinewidth{1.003750pt}%
\definecolor{currentstroke}{rgb}{0.121569,0.466667,0.705882}%
\pgfsetstrokecolor{currentstroke}%
\pgfsetstrokeopacity{0.496485}%
\pgfsetdash{}{0pt}%
\pgfpathmoveto{\pgfqpoint{1.206963in}{1.767824in}}%
\pgfpathcurveto{\pgfqpoint{1.215199in}{1.767824in}}{\pgfqpoint{1.223099in}{1.771096in}}{\pgfqpoint{1.228923in}{1.776920in}}%
\pgfpathcurveto{\pgfqpoint{1.234747in}{1.782744in}}{\pgfqpoint{1.238019in}{1.790644in}}{\pgfqpoint{1.238019in}{1.798881in}}%
\pgfpathcurveto{\pgfqpoint{1.238019in}{1.807117in}}{\pgfqpoint{1.234747in}{1.815017in}}{\pgfqpoint{1.228923in}{1.820841in}}%
\pgfpathcurveto{\pgfqpoint{1.223099in}{1.826665in}}{\pgfqpoint{1.215199in}{1.829937in}}{\pgfqpoint{1.206963in}{1.829937in}}%
\pgfpathcurveto{\pgfqpoint{1.198726in}{1.829937in}}{\pgfqpoint{1.190826in}{1.826665in}}{\pgfqpoint{1.185003in}{1.820841in}}%
\pgfpathcurveto{\pgfqpoint{1.179179in}{1.815017in}}{\pgfqpoint{1.175906in}{1.807117in}}{\pgfqpoint{1.175906in}{1.798881in}}%
\pgfpathcurveto{\pgfqpoint{1.175906in}{1.790644in}}{\pgfqpoint{1.179179in}{1.782744in}}{\pgfqpoint{1.185003in}{1.776920in}}%
\pgfpathcurveto{\pgfqpoint{1.190826in}{1.771096in}}{\pgfqpoint{1.198726in}{1.767824in}}{\pgfqpoint{1.206963in}{1.767824in}}%
\pgfpathclose%
\pgfusepath{stroke,fill}%
\end{pgfscope}%
\begin{pgfscope}%
\pgfpathrectangle{\pgfqpoint{0.100000in}{0.212622in}}{\pgfqpoint{3.696000in}{3.696000in}}%
\pgfusepath{clip}%
\pgfsetbuttcap%
\pgfsetroundjoin%
\definecolor{currentfill}{rgb}{0.121569,0.466667,0.705882}%
\pgfsetfillcolor{currentfill}%
\pgfsetfillopacity{0.496788}%
\pgfsetlinewidth{1.003750pt}%
\definecolor{currentstroke}{rgb}{0.121569,0.466667,0.705882}%
\pgfsetstrokecolor{currentstroke}%
\pgfsetstrokeopacity{0.496788}%
\pgfsetdash{}{0pt}%
\pgfpathmoveto{\pgfqpoint{2.821015in}{2.010849in}}%
\pgfpathcurveto{\pgfqpoint{2.829251in}{2.010849in}}{\pgfqpoint{2.837151in}{2.014121in}}{\pgfqpoint{2.842975in}{2.019945in}}%
\pgfpathcurveto{\pgfqpoint{2.848799in}{2.025769in}}{\pgfqpoint{2.852071in}{2.033669in}}{\pgfqpoint{2.852071in}{2.041905in}}%
\pgfpathcurveto{\pgfqpoint{2.852071in}{2.050141in}}{\pgfqpoint{2.848799in}{2.058041in}}{\pgfqpoint{2.842975in}{2.063865in}}%
\pgfpathcurveto{\pgfqpoint{2.837151in}{2.069689in}}{\pgfqpoint{2.829251in}{2.072962in}}{\pgfqpoint{2.821015in}{2.072962in}}%
\pgfpathcurveto{\pgfqpoint{2.812779in}{2.072962in}}{\pgfqpoint{2.804879in}{2.069689in}}{\pgfqpoint{2.799055in}{2.063865in}}%
\pgfpathcurveto{\pgfqpoint{2.793231in}{2.058041in}}{\pgfqpoint{2.789958in}{2.050141in}}{\pgfqpoint{2.789958in}{2.041905in}}%
\pgfpathcurveto{\pgfqpoint{2.789958in}{2.033669in}}{\pgfqpoint{2.793231in}{2.025769in}}{\pgfqpoint{2.799055in}{2.019945in}}%
\pgfpathcurveto{\pgfqpoint{2.804879in}{2.014121in}}{\pgfqpoint{2.812779in}{2.010849in}}{\pgfqpoint{2.821015in}{2.010849in}}%
\pgfpathclose%
\pgfusepath{stroke,fill}%
\end{pgfscope}%
\begin{pgfscope}%
\pgfpathrectangle{\pgfqpoint{0.100000in}{0.212622in}}{\pgfqpoint{3.696000in}{3.696000in}}%
\pgfusepath{clip}%
\pgfsetbuttcap%
\pgfsetroundjoin%
\definecolor{currentfill}{rgb}{0.121569,0.466667,0.705882}%
\pgfsetfillcolor{currentfill}%
\pgfsetfillopacity{0.497188}%
\pgfsetlinewidth{1.003750pt}%
\definecolor{currentstroke}{rgb}{0.121569,0.466667,0.705882}%
\pgfsetstrokecolor{currentstroke}%
\pgfsetstrokeopacity{0.497188}%
\pgfsetdash{}{0pt}%
\pgfpathmoveto{\pgfqpoint{2.825696in}{2.008549in}}%
\pgfpathcurveto{\pgfqpoint{2.833933in}{2.008549in}}{\pgfqpoint{2.841833in}{2.011821in}}{\pgfqpoint{2.847657in}{2.017645in}}%
\pgfpathcurveto{\pgfqpoint{2.853481in}{2.023469in}}{\pgfqpoint{2.856753in}{2.031369in}}{\pgfqpoint{2.856753in}{2.039606in}}%
\pgfpathcurveto{\pgfqpoint{2.856753in}{2.047842in}}{\pgfqpoint{2.853481in}{2.055742in}}{\pgfqpoint{2.847657in}{2.061566in}}%
\pgfpathcurveto{\pgfqpoint{2.841833in}{2.067390in}}{\pgfqpoint{2.833933in}{2.070662in}}{\pgfqpoint{2.825696in}{2.070662in}}%
\pgfpathcurveto{\pgfqpoint{2.817460in}{2.070662in}}{\pgfqpoint{2.809560in}{2.067390in}}{\pgfqpoint{2.803736in}{2.061566in}}%
\pgfpathcurveto{\pgfqpoint{2.797912in}{2.055742in}}{\pgfqpoint{2.794640in}{2.047842in}}{\pgfqpoint{2.794640in}{2.039606in}}%
\pgfpathcurveto{\pgfqpoint{2.794640in}{2.031369in}}{\pgfqpoint{2.797912in}{2.023469in}}{\pgfqpoint{2.803736in}{2.017645in}}%
\pgfpathcurveto{\pgfqpoint{2.809560in}{2.011821in}}{\pgfqpoint{2.817460in}{2.008549in}}{\pgfqpoint{2.825696in}{2.008549in}}%
\pgfpathclose%
\pgfusepath{stroke,fill}%
\end{pgfscope}%
\begin{pgfscope}%
\pgfpathrectangle{\pgfqpoint{0.100000in}{0.212622in}}{\pgfqpoint{3.696000in}{3.696000in}}%
\pgfusepath{clip}%
\pgfsetbuttcap%
\pgfsetroundjoin%
\definecolor{currentfill}{rgb}{0.121569,0.466667,0.705882}%
\pgfsetfillcolor{currentfill}%
\pgfsetfillopacity{0.497944}%
\pgfsetlinewidth{1.003750pt}%
\definecolor{currentstroke}{rgb}{0.121569,0.466667,0.705882}%
\pgfsetstrokecolor{currentstroke}%
\pgfsetstrokeopacity{0.497944}%
\pgfsetdash{}{0pt}%
\pgfpathmoveto{\pgfqpoint{2.830984in}{2.009575in}}%
\pgfpathcurveto{\pgfqpoint{2.839221in}{2.009575in}}{\pgfqpoint{2.847121in}{2.012847in}}{\pgfqpoint{2.852945in}{2.018671in}}%
\pgfpathcurveto{\pgfqpoint{2.858768in}{2.024495in}}{\pgfqpoint{2.862041in}{2.032395in}}{\pgfqpoint{2.862041in}{2.040631in}}%
\pgfpathcurveto{\pgfqpoint{2.862041in}{2.048867in}}{\pgfqpoint{2.858768in}{2.056767in}}{\pgfqpoint{2.852945in}{2.062591in}}%
\pgfpathcurveto{\pgfqpoint{2.847121in}{2.068415in}}{\pgfqpoint{2.839221in}{2.071688in}}{\pgfqpoint{2.830984in}{2.071688in}}%
\pgfpathcurveto{\pgfqpoint{2.822748in}{2.071688in}}{\pgfqpoint{2.814848in}{2.068415in}}{\pgfqpoint{2.809024in}{2.062591in}}%
\pgfpathcurveto{\pgfqpoint{2.803200in}{2.056767in}}{\pgfqpoint{2.799928in}{2.048867in}}{\pgfqpoint{2.799928in}{2.040631in}}%
\pgfpathcurveto{\pgfqpoint{2.799928in}{2.032395in}}{\pgfqpoint{2.803200in}{2.024495in}}{\pgfqpoint{2.809024in}{2.018671in}}%
\pgfpathcurveto{\pgfqpoint{2.814848in}{2.012847in}}{\pgfqpoint{2.822748in}{2.009575in}}{\pgfqpoint{2.830984in}{2.009575in}}%
\pgfpathclose%
\pgfusepath{stroke,fill}%
\end{pgfscope}%
\begin{pgfscope}%
\pgfpathrectangle{\pgfqpoint{0.100000in}{0.212622in}}{\pgfqpoint{3.696000in}{3.696000in}}%
\pgfusepath{clip}%
\pgfsetbuttcap%
\pgfsetroundjoin%
\definecolor{currentfill}{rgb}{0.121569,0.466667,0.705882}%
\pgfsetfillcolor{currentfill}%
\pgfsetfillopacity{0.498576}%
\pgfsetlinewidth{1.003750pt}%
\definecolor{currentstroke}{rgb}{0.121569,0.466667,0.705882}%
\pgfsetstrokecolor{currentstroke}%
\pgfsetstrokeopacity{0.498576}%
\pgfsetdash{}{0pt}%
\pgfpathmoveto{\pgfqpoint{2.836740in}{2.008240in}}%
\pgfpathcurveto{\pgfqpoint{2.844976in}{2.008240in}}{\pgfqpoint{2.852876in}{2.011513in}}{\pgfqpoint{2.858700in}{2.017337in}}%
\pgfpathcurveto{\pgfqpoint{2.864524in}{2.023160in}}{\pgfqpoint{2.867796in}{2.031061in}}{\pgfqpoint{2.867796in}{2.039297in}}%
\pgfpathcurveto{\pgfqpoint{2.867796in}{2.047533in}}{\pgfqpoint{2.864524in}{2.055433in}}{\pgfqpoint{2.858700in}{2.061257in}}%
\pgfpathcurveto{\pgfqpoint{2.852876in}{2.067081in}}{\pgfqpoint{2.844976in}{2.070353in}}{\pgfqpoint{2.836740in}{2.070353in}}%
\pgfpathcurveto{\pgfqpoint{2.828503in}{2.070353in}}{\pgfqpoint{2.820603in}{2.067081in}}{\pgfqpoint{2.814779in}{2.061257in}}%
\pgfpathcurveto{\pgfqpoint{2.808955in}{2.055433in}}{\pgfqpoint{2.805683in}{2.047533in}}{\pgfqpoint{2.805683in}{2.039297in}}%
\pgfpathcurveto{\pgfqpoint{2.805683in}{2.031061in}}{\pgfqpoint{2.808955in}{2.023160in}}{\pgfqpoint{2.814779in}{2.017337in}}%
\pgfpathcurveto{\pgfqpoint{2.820603in}{2.011513in}}{\pgfqpoint{2.828503in}{2.008240in}}{\pgfqpoint{2.836740in}{2.008240in}}%
\pgfpathclose%
\pgfusepath{stroke,fill}%
\end{pgfscope}%
\begin{pgfscope}%
\pgfpathrectangle{\pgfqpoint{0.100000in}{0.212622in}}{\pgfqpoint{3.696000in}{3.696000in}}%
\pgfusepath{clip}%
\pgfsetbuttcap%
\pgfsetroundjoin%
\definecolor{currentfill}{rgb}{0.121569,0.466667,0.705882}%
\pgfsetfillcolor{currentfill}%
\pgfsetfillopacity{0.499570}%
\pgfsetlinewidth{1.003750pt}%
\definecolor{currentstroke}{rgb}{0.121569,0.466667,0.705882}%
\pgfsetstrokecolor{currentstroke}%
\pgfsetstrokeopacity{0.499570}%
\pgfsetdash{}{0pt}%
\pgfpathmoveto{\pgfqpoint{2.842837in}{2.008872in}}%
\pgfpathcurveto{\pgfqpoint{2.851074in}{2.008872in}}{\pgfqpoint{2.858974in}{2.012145in}}{\pgfqpoint{2.864798in}{2.017968in}}%
\pgfpathcurveto{\pgfqpoint{2.870622in}{2.023792in}}{\pgfqpoint{2.873894in}{2.031692in}}{\pgfqpoint{2.873894in}{2.039929in}}%
\pgfpathcurveto{\pgfqpoint{2.873894in}{2.048165in}}{\pgfqpoint{2.870622in}{2.056065in}}{\pgfqpoint{2.864798in}{2.061889in}}%
\pgfpathcurveto{\pgfqpoint{2.858974in}{2.067713in}}{\pgfqpoint{2.851074in}{2.070985in}}{\pgfqpoint{2.842837in}{2.070985in}}%
\pgfpathcurveto{\pgfqpoint{2.834601in}{2.070985in}}{\pgfqpoint{2.826701in}{2.067713in}}{\pgfqpoint{2.820877in}{2.061889in}}%
\pgfpathcurveto{\pgfqpoint{2.815053in}{2.056065in}}{\pgfqpoint{2.811781in}{2.048165in}}{\pgfqpoint{2.811781in}{2.039929in}}%
\pgfpathcurveto{\pgfqpoint{2.811781in}{2.031692in}}{\pgfqpoint{2.815053in}{2.023792in}}{\pgfqpoint{2.820877in}{2.017968in}}%
\pgfpathcurveto{\pgfqpoint{2.826701in}{2.012145in}}{\pgfqpoint{2.834601in}{2.008872in}}{\pgfqpoint{2.842837in}{2.008872in}}%
\pgfpathclose%
\pgfusepath{stroke,fill}%
\end{pgfscope}%
\begin{pgfscope}%
\pgfpathrectangle{\pgfqpoint{0.100000in}{0.212622in}}{\pgfqpoint{3.696000in}{3.696000in}}%
\pgfusepath{clip}%
\pgfsetbuttcap%
\pgfsetroundjoin%
\definecolor{currentfill}{rgb}{0.121569,0.466667,0.705882}%
\pgfsetfillcolor{currentfill}%
\pgfsetfillopacity{0.499587}%
\pgfsetlinewidth{1.003750pt}%
\definecolor{currentstroke}{rgb}{0.121569,0.466667,0.705882}%
\pgfsetstrokecolor{currentstroke}%
\pgfsetstrokeopacity{0.499587}%
\pgfsetdash{}{0pt}%
\pgfpathmoveto{\pgfqpoint{1.201100in}{1.764338in}}%
\pgfpathcurveto{\pgfqpoint{1.209336in}{1.764338in}}{\pgfqpoint{1.217236in}{1.767610in}}{\pgfqpoint{1.223060in}{1.773434in}}%
\pgfpathcurveto{\pgfqpoint{1.228884in}{1.779258in}}{\pgfqpoint{1.232156in}{1.787158in}}{\pgfqpoint{1.232156in}{1.795394in}}%
\pgfpathcurveto{\pgfqpoint{1.232156in}{1.803630in}}{\pgfqpoint{1.228884in}{1.811530in}}{\pgfqpoint{1.223060in}{1.817354in}}%
\pgfpathcurveto{\pgfqpoint{1.217236in}{1.823178in}}{\pgfqpoint{1.209336in}{1.826451in}}{\pgfqpoint{1.201100in}{1.826451in}}%
\pgfpathcurveto{\pgfqpoint{1.192864in}{1.826451in}}{\pgfqpoint{1.184964in}{1.823178in}}{\pgfqpoint{1.179140in}{1.817354in}}%
\pgfpathcurveto{\pgfqpoint{1.173316in}{1.811530in}}{\pgfqpoint{1.170043in}{1.803630in}}{\pgfqpoint{1.170043in}{1.795394in}}%
\pgfpathcurveto{\pgfqpoint{1.170043in}{1.787158in}}{\pgfqpoint{1.173316in}{1.779258in}}{\pgfqpoint{1.179140in}{1.773434in}}%
\pgfpathcurveto{\pgfqpoint{1.184964in}{1.767610in}}{\pgfqpoint{1.192864in}{1.764338in}}{\pgfqpoint{1.201100in}{1.764338in}}%
\pgfpathclose%
\pgfusepath{stroke,fill}%
\end{pgfscope}%
\begin{pgfscope}%
\pgfpathrectangle{\pgfqpoint{0.100000in}{0.212622in}}{\pgfqpoint{3.696000in}{3.696000in}}%
\pgfusepath{clip}%
\pgfsetbuttcap%
\pgfsetroundjoin%
\definecolor{currentfill}{rgb}{0.121569,0.466667,0.705882}%
\pgfsetfillcolor{currentfill}%
\pgfsetfillopacity{0.500151}%
\pgfsetlinewidth{1.003750pt}%
\definecolor{currentstroke}{rgb}{0.121569,0.466667,0.705882}%
\pgfsetstrokecolor{currentstroke}%
\pgfsetstrokeopacity{0.500151}%
\pgfsetdash{}{0pt}%
\pgfpathmoveto{\pgfqpoint{2.848969in}{2.005364in}}%
\pgfpathcurveto{\pgfqpoint{2.857205in}{2.005364in}}{\pgfqpoint{2.865105in}{2.008636in}}{\pgfqpoint{2.870929in}{2.014460in}}%
\pgfpathcurveto{\pgfqpoint{2.876753in}{2.020284in}}{\pgfqpoint{2.880026in}{2.028184in}}{\pgfqpoint{2.880026in}{2.036420in}}%
\pgfpathcurveto{\pgfqpoint{2.880026in}{2.044657in}}{\pgfqpoint{2.876753in}{2.052557in}}{\pgfqpoint{2.870929in}{2.058381in}}%
\pgfpathcurveto{\pgfqpoint{2.865105in}{2.064205in}}{\pgfqpoint{2.857205in}{2.067477in}}{\pgfqpoint{2.848969in}{2.067477in}}%
\pgfpathcurveto{\pgfqpoint{2.840733in}{2.067477in}}{\pgfqpoint{2.832833in}{2.064205in}}{\pgfqpoint{2.827009in}{2.058381in}}%
\pgfpathcurveto{\pgfqpoint{2.821185in}{2.052557in}}{\pgfqpoint{2.817913in}{2.044657in}}{\pgfqpoint{2.817913in}{2.036420in}}%
\pgfpathcurveto{\pgfqpoint{2.817913in}{2.028184in}}{\pgfqpoint{2.821185in}{2.020284in}}{\pgfqpoint{2.827009in}{2.014460in}}%
\pgfpathcurveto{\pgfqpoint{2.832833in}{2.008636in}}{\pgfqpoint{2.840733in}{2.005364in}}{\pgfqpoint{2.848969in}{2.005364in}}%
\pgfpathclose%
\pgfusepath{stroke,fill}%
\end{pgfscope}%
\begin{pgfscope}%
\pgfpathrectangle{\pgfqpoint{0.100000in}{0.212622in}}{\pgfqpoint{3.696000in}{3.696000in}}%
\pgfusepath{clip}%
\pgfsetbuttcap%
\pgfsetroundjoin%
\definecolor{currentfill}{rgb}{0.121569,0.466667,0.705882}%
\pgfsetfillcolor{currentfill}%
\pgfsetfillopacity{0.500718}%
\pgfsetlinewidth{1.003750pt}%
\definecolor{currentstroke}{rgb}{0.121569,0.466667,0.705882}%
\pgfsetstrokecolor{currentstroke}%
\pgfsetstrokeopacity{0.500718}%
\pgfsetdash{}{0pt}%
\pgfpathmoveto{\pgfqpoint{2.852522in}{2.005679in}}%
\pgfpathcurveto{\pgfqpoint{2.860759in}{2.005679in}}{\pgfqpoint{2.868659in}{2.008952in}}{\pgfqpoint{2.874483in}{2.014776in}}%
\pgfpathcurveto{\pgfqpoint{2.880307in}{2.020600in}}{\pgfqpoint{2.883579in}{2.028500in}}{\pgfqpoint{2.883579in}{2.036736in}}%
\pgfpathcurveto{\pgfqpoint{2.883579in}{2.044972in}}{\pgfqpoint{2.880307in}{2.052872in}}{\pgfqpoint{2.874483in}{2.058696in}}%
\pgfpathcurveto{\pgfqpoint{2.868659in}{2.064520in}}{\pgfqpoint{2.860759in}{2.067792in}}{\pgfqpoint{2.852522in}{2.067792in}}%
\pgfpathcurveto{\pgfqpoint{2.844286in}{2.067792in}}{\pgfqpoint{2.836386in}{2.064520in}}{\pgfqpoint{2.830562in}{2.058696in}}%
\pgfpathcurveto{\pgfqpoint{2.824738in}{2.052872in}}{\pgfqpoint{2.821466in}{2.044972in}}{\pgfqpoint{2.821466in}{2.036736in}}%
\pgfpathcurveto{\pgfqpoint{2.821466in}{2.028500in}}{\pgfqpoint{2.824738in}{2.020600in}}{\pgfqpoint{2.830562in}{2.014776in}}%
\pgfpathcurveto{\pgfqpoint{2.836386in}{2.008952in}}{\pgfqpoint{2.844286in}{2.005679in}}{\pgfqpoint{2.852522in}{2.005679in}}%
\pgfpathclose%
\pgfusepath{stroke,fill}%
\end{pgfscope}%
\begin{pgfscope}%
\pgfpathrectangle{\pgfqpoint{0.100000in}{0.212622in}}{\pgfqpoint{3.696000in}{3.696000in}}%
\pgfusepath{clip}%
\pgfsetbuttcap%
\pgfsetroundjoin%
\definecolor{currentfill}{rgb}{0.121569,0.466667,0.705882}%
\pgfsetfillcolor{currentfill}%
\pgfsetfillopacity{0.501159}%
\pgfsetlinewidth{1.003750pt}%
\definecolor{currentstroke}{rgb}{0.121569,0.466667,0.705882}%
\pgfsetstrokecolor{currentstroke}%
\pgfsetstrokeopacity{0.501159}%
\pgfsetdash{}{0pt}%
\pgfpathmoveto{\pgfqpoint{2.856317in}{2.003796in}}%
\pgfpathcurveto{\pgfqpoint{2.864554in}{2.003796in}}{\pgfqpoint{2.872454in}{2.007068in}}{\pgfqpoint{2.878278in}{2.012892in}}%
\pgfpathcurveto{\pgfqpoint{2.884102in}{2.018716in}}{\pgfqpoint{2.887374in}{2.026616in}}{\pgfqpoint{2.887374in}{2.034852in}}%
\pgfpathcurveto{\pgfqpoint{2.887374in}{2.043088in}}{\pgfqpoint{2.884102in}{2.050988in}}{\pgfqpoint{2.878278in}{2.056812in}}%
\pgfpathcurveto{\pgfqpoint{2.872454in}{2.062636in}}{\pgfqpoint{2.864554in}{2.065909in}}{\pgfqpoint{2.856317in}{2.065909in}}%
\pgfpathcurveto{\pgfqpoint{2.848081in}{2.065909in}}{\pgfqpoint{2.840181in}{2.062636in}}{\pgfqpoint{2.834357in}{2.056812in}}%
\pgfpathcurveto{\pgfqpoint{2.828533in}{2.050988in}}{\pgfqpoint{2.825261in}{2.043088in}}{\pgfqpoint{2.825261in}{2.034852in}}%
\pgfpathcurveto{\pgfqpoint{2.825261in}{2.026616in}}{\pgfqpoint{2.828533in}{2.018716in}}{\pgfqpoint{2.834357in}{2.012892in}}%
\pgfpathcurveto{\pgfqpoint{2.840181in}{2.007068in}}{\pgfqpoint{2.848081in}{2.003796in}}{\pgfqpoint{2.856317in}{2.003796in}}%
\pgfpathclose%
\pgfusepath{stroke,fill}%
\end{pgfscope}%
\begin{pgfscope}%
\pgfpathrectangle{\pgfqpoint{0.100000in}{0.212622in}}{\pgfqpoint{3.696000in}{3.696000in}}%
\pgfusepath{clip}%
\pgfsetbuttcap%
\pgfsetroundjoin%
\definecolor{currentfill}{rgb}{0.121569,0.466667,0.705882}%
\pgfsetfillcolor{currentfill}%
\pgfsetfillopacity{0.501301}%
\pgfsetlinewidth{1.003750pt}%
\definecolor{currentstroke}{rgb}{0.121569,0.466667,0.705882}%
\pgfsetstrokecolor{currentstroke}%
\pgfsetstrokeopacity{0.501301}%
\pgfsetdash{}{0pt}%
\pgfpathmoveto{\pgfqpoint{1.191829in}{1.759527in}}%
\pgfpathcurveto{\pgfqpoint{1.200066in}{1.759527in}}{\pgfqpoint{1.207966in}{1.762800in}}{\pgfqpoint{1.213790in}{1.768623in}}%
\pgfpathcurveto{\pgfqpoint{1.219614in}{1.774447in}}{\pgfqpoint{1.222886in}{1.782347in}}{\pgfqpoint{1.222886in}{1.790584in}}%
\pgfpathcurveto{\pgfqpoint{1.222886in}{1.798820in}}{\pgfqpoint{1.219614in}{1.806720in}}{\pgfqpoint{1.213790in}{1.812544in}}%
\pgfpathcurveto{\pgfqpoint{1.207966in}{1.818368in}}{\pgfqpoint{1.200066in}{1.821640in}}{\pgfqpoint{1.191829in}{1.821640in}}%
\pgfpathcurveto{\pgfqpoint{1.183593in}{1.821640in}}{\pgfqpoint{1.175693in}{1.818368in}}{\pgfqpoint{1.169869in}{1.812544in}}%
\pgfpathcurveto{\pgfqpoint{1.164045in}{1.806720in}}{\pgfqpoint{1.160773in}{1.798820in}}{\pgfqpoint{1.160773in}{1.790584in}}%
\pgfpathcurveto{\pgfqpoint{1.160773in}{1.782347in}}{\pgfqpoint{1.164045in}{1.774447in}}{\pgfqpoint{1.169869in}{1.768623in}}%
\pgfpathcurveto{\pgfqpoint{1.175693in}{1.762800in}}{\pgfqpoint{1.183593in}{1.759527in}}{\pgfqpoint{1.191829in}{1.759527in}}%
\pgfpathclose%
\pgfusepath{stroke,fill}%
\end{pgfscope}%
\begin{pgfscope}%
\pgfpathrectangle{\pgfqpoint{0.100000in}{0.212622in}}{\pgfqpoint{3.696000in}{3.696000in}}%
\pgfusepath{clip}%
\pgfsetbuttcap%
\pgfsetroundjoin%
\definecolor{currentfill}{rgb}{0.121569,0.466667,0.705882}%
\pgfsetfillcolor{currentfill}%
\pgfsetfillopacity{0.502088}%
\pgfsetlinewidth{1.003750pt}%
\definecolor{currentstroke}{rgb}{0.121569,0.466667,0.705882}%
\pgfsetstrokecolor{currentstroke}%
\pgfsetstrokeopacity{0.502088}%
\pgfsetdash{}{0pt}%
\pgfpathmoveto{\pgfqpoint{2.860944in}{2.005557in}}%
\pgfpathcurveto{\pgfqpoint{2.869180in}{2.005557in}}{\pgfqpoint{2.877080in}{2.008830in}}{\pgfqpoint{2.882904in}{2.014653in}}%
\pgfpathcurveto{\pgfqpoint{2.888728in}{2.020477in}}{\pgfqpoint{2.892000in}{2.028377in}}{\pgfqpoint{2.892000in}{2.036614in}}%
\pgfpathcurveto{\pgfqpoint{2.892000in}{2.044850in}}{\pgfqpoint{2.888728in}{2.052750in}}{\pgfqpoint{2.882904in}{2.058574in}}%
\pgfpathcurveto{\pgfqpoint{2.877080in}{2.064398in}}{\pgfqpoint{2.869180in}{2.067670in}}{\pgfqpoint{2.860944in}{2.067670in}}%
\pgfpathcurveto{\pgfqpoint{2.852707in}{2.067670in}}{\pgfqpoint{2.844807in}{2.064398in}}{\pgfqpoint{2.838984in}{2.058574in}}%
\pgfpathcurveto{\pgfqpoint{2.833160in}{2.052750in}}{\pgfqpoint{2.829887in}{2.044850in}}{\pgfqpoint{2.829887in}{2.036614in}}%
\pgfpathcurveto{\pgfqpoint{2.829887in}{2.028377in}}{\pgfqpoint{2.833160in}{2.020477in}}{\pgfqpoint{2.838984in}{2.014653in}}%
\pgfpathcurveto{\pgfqpoint{2.844807in}{2.008830in}}{\pgfqpoint{2.852707in}{2.005557in}}{\pgfqpoint{2.860944in}{2.005557in}}%
\pgfpathclose%
\pgfusepath{stroke,fill}%
\end{pgfscope}%
\begin{pgfscope}%
\pgfpathrectangle{\pgfqpoint{0.100000in}{0.212622in}}{\pgfqpoint{3.696000in}{3.696000in}}%
\pgfusepath{clip}%
\pgfsetbuttcap%
\pgfsetroundjoin%
\definecolor{currentfill}{rgb}{0.121569,0.466667,0.705882}%
\pgfsetfillcolor{currentfill}%
\pgfsetfillopacity{0.502435}%
\pgfsetlinewidth{1.003750pt}%
\definecolor{currentstroke}{rgb}{0.121569,0.466667,0.705882}%
\pgfsetstrokecolor{currentstroke}%
\pgfsetstrokeopacity{0.502435}%
\pgfsetdash{}{0pt}%
\pgfpathmoveto{\pgfqpoint{2.863215in}{2.004597in}}%
\pgfpathcurveto{\pgfqpoint{2.871451in}{2.004597in}}{\pgfqpoint{2.879351in}{2.007869in}}{\pgfqpoint{2.885175in}{2.013693in}}%
\pgfpathcurveto{\pgfqpoint{2.890999in}{2.019517in}}{\pgfqpoint{2.894272in}{2.027417in}}{\pgfqpoint{2.894272in}{2.035654in}}%
\pgfpathcurveto{\pgfqpoint{2.894272in}{2.043890in}}{\pgfqpoint{2.890999in}{2.051790in}}{\pgfqpoint{2.885175in}{2.057614in}}%
\pgfpathcurveto{\pgfqpoint{2.879351in}{2.063438in}}{\pgfqpoint{2.871451in}{2.066710in}}{\pgfqpoint{2.863215in}{2.066710in}}%
\pgfpathcurveto{\pgfqpoint{2.854979in}{2.066710in}}{\pgfqpoint{2.847079in}{2.063438in}}{\pgfqpoint{2.841255in}{2.057614in}}%
\pgfpathcurveto{\pgfqpoint{2.835431in}{2.051790in}}{\pgfqpoint{2.832159in}{2.043890in}}{\pgfqpoint{2.832159in}{2.035654in}}%
\pgfpathcurveto{\pgfqpoint{2.832159in}{2.027417in}}{\pgfqpoint{2.835431in}{2.019517in}}{\pgfqpoint{2.841255in}{2.013693in}}%
\pgfpathcurveto{\pgfqpoint{2.847079in}{2.007869in}}{\pgfqpoint{2.854979in}{2.004597in}}{\pgfqpoint{2.863215in}{2.004597in}}%
\pgfpathclose%
\pgfusepath{stroke,fill}%
\end{pgfscope}%
\begin{pgfscope}%
\pgfpathrectangle{\pgfqpoint{0.100000in}{0.212622in}}{\pgfqpoint{3.696000in}{3.696000in}}%
\pgfusepath{clip}%
\pgfsetbuttcap%
\pgfsetroundjoin%
\definecolor{currentfill}{rgb}{0.121569,0.466667,0.705882}%
\pgfsetfillcolor{currentfill}%
\pgfsetfillopacity{0.503092}%
\pgfsetlinewidth{1.003750pt}%
\definecolor{currentstroke}{rgb}{0.121569,0.466667,0.705882}%
\pgfsetstrokecolor{currentstroke}%
\pgfsetstrokeopacity{0.503092}%
\pgfsetdash{}{0pt}%
\pgfpathmoveto{\pgfqpoint{2.866127in}{2.005555in}}%
\pgfpathcurveto{\pgfqpoint{2.874363in}{2.005555in}}{\pgfqpoint{2.882264in}{2.008828in}}{\pgfqpoint{2.888087in}{2.014652in}}%
\pgfpathcurveto{\pgfqpoint{2.893911in}{2.020476in}}{\pgfqpoint{2.897184in}{2.028376in}}{\pgfqpoint{2.897184in}{2.036612in}}%
\pgfpathcurveto{\pgfqpoint{2.897184in}{2.044848in}}{\pgfqpoint{2.893911in}{2.052748in}}{\pgfqpoint{2.888087in}{2.058572in}}%
\pgfpathcurveto{\pgfqpoint{2.882264in}{2.064396in}}{\pgfqpoint{2.874363in}{2.067668in}}{\pgfqpoint{2.866127in}{2.067668in}}%
\pgfpathcurveto{\pgfqpoint{2.857891in}{2.067668in}}{\pgfqpoint{2.849991in}{2.064396in}}{\pgfqpoint{2.844167in}{2.058572in}}%
\pgfpathcurveto{\pgfqpoint{2.838343in}{2.052748in}}{\pgfqpoint{2.835071in}{2.044848in}}{\pgfqpoint{2.835071in}{2.036612in}}%
\pgfpathcurveto{\pgfqpoint{2.835071in}{2.028376in}}{\pgfqpoint{2.838343in}{2.020476in}}{\pgfqpoint{2.844167in}{2.014652in}}%
\pgfpathcurveto{\pgfqpoint{2.849991in}{2.008828in}}{\pgfqpoint{2.857891in}{2.005555in}}{\pgfqpoint{2.866127in}{2.005555in}}%
\pgfpathclose%
\pgfusepath{stroke,fill}%
\end{pgfscope}%
\begin{pgfscope}%
\pgfpathrectangle{\pgfqpoint{0.100000in}{0.212622in}}{\pgfqpoint{3.696000in}{3.696000in}}%
\pgfusepath{clip}%
\pgfsetbuttcap%
\pgfsetroundjoin%
\definecolor{currentfill}{rgb}{0.121569,0.466667,0.705882}%
\pgfsetfillcolor{currentfill}%
\pgfsetfillopacity{0.503378}%
\pgfsetlinewidth{1.003750pt}%
\definecolor{currentstroke}{rgb}{0.121569,0.466667,0.705882}%
\pgfsetstrokecolor{currentstroke}%
\pgfsetstrokeopacity{0.503378}%
\pgfsetdash{}{0pt}%
\pgfpathmoveto{\pgfqpoint{2.867672in}{2.005412in}}%
\pgfpathcurveto{\pgfqpoint{2.875908in}{2.005412in}}{\pgfqpoint{2.883809in}{2.008684in}}{\pgfqpoint{2.889632in}{2.014508in}}%
\pgfpathcurveto{\pgfqpoint{2.895456in}{2.020332in}}{\pgfqpoint{2.898729in}{2.028232in}}{\pgfqpoint{2.898729in}{2.036468in}}%
\pgfpathcurveto{\pgfqpoint{2.898729in}{2.044705in}}{\pgfqpoint{2.895456in}{2.052605in}}{\pgfqpoint{2.889632in}{2.058429in}}%
\pgfpathcurveto{\pgfqpoint{2.883809in}{2.064252in}}{\pgfqpoint{2.875908in}{2.067525in}}{\pgfqpoint{2.867672in}{2.067525in}}%
\pgfpathcurveto{\pgfqpoint{2.859436in}{2.067525in}}{\pgfqpoint{2.851536in}{2.064252in}}{\pgfqpoint{2.845712in}{2.058429in}}%
\pgfpathcurveto{\pgfqpoint{2.839888in}{2.052605in}}{\pgfqpoint{2.836616in}{2.044705in}}{\pgfqpoint{2.836616in}{2.036468in}}%
\pgfpathcurveto{\pgfqpoint{2.836616in}{2.028232in}}{\pgfqpoint{2.839888in}{2.020332in}}{\pgfqpoint{2.845712in}{2.014508in}}%
\pgfpathcurveto{\pgfqpoint{2.851536in}{2.008684in}}{\pgfqpoint{2.859436in}{2.005412in}}{\pgfqpoint{2.867672in}{2.005412in}}%
\pgfpathclose%
\pgfusepath{stroke,fill}%
\end{pgfscope}%
\begin{pgfscope}%
\pgfpathrectangle{\pgfqpoint{0.100000in}{0.212622in}}{\pgfqpoint{3.696000in}{3.696000in}}%
\pgfusepath{clip}%
\pgfsetbuttcap%
\pgfsetroundjoin%
\definecolor{currentfill}{rgb}{0.121569,0.466667,0.705882}%
\pgfsetfillcolor{currentfill}%
\pgfsetfillopacity{0.503453}%
\pgfsetlinewidth{1.003750pt}%
\definecolor{currentstroke}{rgb}{0.121569,0.466667,0.705882}%
\pgfsetstrokecolor{currentstroke}%
\pgfsetstrokeopacity{0.503453}%
\pgfsetdash{}{0pt}%
\pgfpathmoveto{\pgfqpoint{1.186184in}{1.754995in}}%
\pgfpathcurveto{\pgfqpoint{1.194421in}{1.754995in}}{\pgfqpoint{1.202321in}{1.758267in}}{\pgfqpoint{1.208145in}{1.764091in}}%
\pgfpathcurveto{\pgfqpoint{1.213969in}{1.769915in}}{\pgfqpoint{1.217241in}{1.777815in}}{\pgfqpoint{1.217241in}{1.786051in}}%
\pgfpathcurveto{\pgfqpoint{1.217241in}{1.794288in}}{\pgfqpoint{1.213969in}{1.802188in}}{\pgfqpoint{1.208145in}{1.808012in}}%
\pgfpathcurveto{\pgfqpoint{1.202321in}{1.813836in}}{\pgfqpoint{1.194421in}{1.817108in}}{\pgfqpoint{1.186184in}{1.817108in}}%
\pgfpathcurveto{\pgfqpoint{1.177948in}{1.817108in}}{\pgfqpoint{1.170048in}{1.813836in}}{\pgfqpoint{1.164224in}{1.808012in}}%
\pgfpathcurveto{\pgfqpoint{1.158400in}{1.802188in}}{\pgfqpoint{1.155128in}{1.794288in}}{\pgfqpoint{1.155128in}{1.786051in}}%
\pgfpathcurveto{\pgfqpoint{1.155128in}{1.777815in}}{\pgfqpoint{1.158400in}{1.769915in}}{\pgfqpoint{1.164224in}{1.764091in}}%
\pgfpathcurveto{\pgfqpoint{1.170048in}{1.758267in}}{\pgfqpoint{1.177948in}{1.754995in}}{\pgfqpoint{1.186184in}{1.754995in}}%
\pgfpathclose%
\pgfusepath{stroke,fill}%
\end{pgfscope}%
\begin{pgfscope}%
\pgfpathrectangle{\pgfqpoint{0.100000in}{0.212622in}}{\pgfqpoint{3.696000in}{3.696000in}}%
\pgfusepath{clip}%
\pgfsetbuttcap%
\pgfsetroundjoin%
\definecolor{currentfill}{rgb}{0.121569,0.466667,0.705882}%
\pgfsetfillcolor{currentfill}%
\pgfsetfillopacity{0.504020}%
\pgfsetlinewidth{1.003750pt}%
\definecolor{currentstroke}{rgb}{0.121569,0.466667,0.705882}%
\pgfsetstrokecolor{currentstroke}%
\pgfsetstrokeopacity{0.504020}%
\pgfsetdash{}{0pt}%
\pgfpathmoveto{\pgfqpoint{2.870565in}{2.006691in}}%
\pgfpathcurveto{\pgfqpoint{2.878802in}{2.006691in}}{\pgfqpoint{2.886702in}{2.009964in}}{\pgfqpoint{2.892526in}{2.015788in}}%
\pgfpathcurveto{\pgfqpoint{2.898350in}{2.021612in}}{\pgfqpoint{2.901622in}{2.029512in}}{\pgfqpoint{2.901622in}{2.037748in}}%
\pgfpathcurveto{\pgfqpoint{2.901622in}{2.045984in}}{\pgfqpoint{2.898350in}{2.053884in}}{\pgfqpoint{2.892526in}{2.059708in}}%
\pgfpathcurveto{\pgfqpoint{2.886702in}{2.065532in}}{\pgfqpoint{2.878802in}{2.068804in}}{\pgfqpoint{2.870565in}{2.068804in}}%
\pgfpathcurveto{\pgfqpoint{2.862329in}{2.068804in}}{\pgfqpoint{2.854429in}{2.065532in}}{\pgfqpoint{2.848605in}{2.059708in}}%
\pgfpathcurveto{\pgfqpoint{2.842781in}{2.053884in}}{\pgfqpoint{2.839509in}{2.045984in}}{\pgfqpoint{2.839509in}{2.037748in}}%
\pgfpathcurveto{\pgfqpoint{2.839509in}{2.029512in}}{\pgfqpoint{2.842781in}{2.021612in}}{\pgfqpoint{2.848605in}{2.015788in}}%
\pgfpathcurveto{\pgfqpoint{2.854429in}{2.009964in}}{\pgfqpoint{2.862329in}{2.006691in}}{\pgfqpoint{2.870565in}{2.006691in}}%
\pgfpathclose%
\pgfusepath{stroke,fill}%
\end{pgfscope}%
\begin{pgfscope}%
\pgfpathrectangle{\pgfqpoint{0.100000in}{0.212622in}}{\pgfqpoint{3.696000in}{3.696000in}}%
\pgfusepath{clip}%
\pgfsetbuttcap%
\pgfsetroundjoin%
\definecolor{currentfill}{rgb}{0.121569,0.466667,0.705882}%
\pgfsetfillcolor{currentfill}%
\pgfsetfillopacity{0.504364}%
\pgfsetlinewidth{1.003750pt}%
\definecolor{currentstroke}{rgb}{0.121569,0.466667,0.705882}%
\pgfsetstrokecolor{currentstroke}%
\pgfsetstrokeopacity{0.504364}%
\pgfsetdash{}{0pt}%
\pgfpathmoveto{\pgfqpoint{2.873727in}{2.005328in}}%
\pgfpathcurveto{\pgfqpoint{2.881963in}{2.005328in}}{\pgfqpoint{2.889863in}{2.008600in}}{\pgfqpoint{2.895687in}{2.014424in}}%
\pgfpathcurveto{\pgfqpoint{2.901511in}{2.020248in}}{\pgfqpoint{2.904783in}{2.028148in}}{\pgfqpoint{2.904783in}{2.036384in}}%
\pgfpathcurveto{\pgfqpoint{2.904783in}{2.044621in}}{\pgfqpoint{2.901511in}{2.052521in}}{\pgfqpoint{2.895687in}{2.058345in}}%
\pgfpathcurveto{\pgfqpoint{2.889863in}{2.064169in}}{\pgfqpoint{2.881963in}{2.067441in}}{\pgfqpoint{2.873727in}{2.067441in}}%
\pgfpathcurveto{\pgfqpoint{2.865490in}{2.067441in}}{\pgfqpoint{2.857590in}{2.064169in}}{\pgfqpoint{2.851766in}{2.058345in}}%
\pgfpathcurveto{\pgfqpoint{2.845942in}{2.052521in}}{\pgfqpoint{2.842670in}{2.044621in}}{\pgfqpoint{2.842670in}{2.036384in}}%
\pgfpathcurveto{\pgfqpoint{2.842670in}{2.028148in}}{\pgfqpoint{2.845942in}{2.020248in}}{\pgfqpoint{2.851766in}{2.014424in}}%
\pgfpathcurveto{\pgfqpoint{2.857590in}{2.008600in}}{\pgfqpoint{2.865490in}{2.005328in}}{\pgfqpoint{2.873727in}{2.005328in}}%
\pgfpathclose%
\pgfusepath{stroke,fill}%
\end{pgfscope}%
\begin{pgfscope}%
\pgfpathrectangle{\pgfqpoint{0.100000in}{0.212622in}}{\pgfqpoint{3.696000in}{3.696000in}}%
\pgfusepath{clip}%
\pgfsetbuttcap%
\pgfsetroundjoin%
\definecolor{currentfill}{rgb}{0.121569,0.466667,0.705882}%
\pgfsetfillcolor{currentfill}%
\pgfsetfillopacity{0.504728}%
\pgfsetlinewidth{1.003750pt}%
\definecolor{currentstroke}{rgb}{0.121569,0.466667,0.705882}%
\pgfsetstrokecolor{currentstroke}%
\pgfsetstrokeopacity{0.504728}%
\pgfsetdash{}{0pt}%
\pgfpathmoveto{\pgfqpoint{2.875614in}{2.006271in}}%
\pgfpathcurveto{\pgfqpoint{2.883851in}{2.006271in}}{\pgfqpoint{2.891751in}{2.009544in}}{\pgfqpoint{2.897575in}{2.015368in}}%
\pgfpathcurveto{\pgfqpoint{2.903399in}{2.021191in}}{\pgfqpoint{2.906671in}{2.029092in}}{\pgfqpoint{2.906671in}{2.037328in}}%
\pgfpathcurveto{\pgfqpoint{2.906671in}{2.045564in}}{\pgfqpoint{2.903399in}{2.053464in}}{\pgfqpoint{2.897575in}{2.059288in}}%
\pgfpathcurveto{\pgfqpoint{2.891751in}{2.065112in}}{\pgfqpoint{2.883851in}{2.068384in}}{\pgfqpoint{2.875614in}{2.068384in}}%
\pgfpathcurveto{\pgfqpoint{2.867378in}{2.068384in}}{\pgfqpoint{2.859478in}{2.065112in}}{\pgfqpoint{2.853654in}{2.059288in}}%
\pgfpathcurveto{\pgfqpoint{2.847830in}{2.053464in}}{\pgfqpoint{2.844558in}{2.045564in}}{\pgfqpoint{2.844558in}{2.037328in}}%
\pgfpathcurveto{\pgfqpoint{2.844558in}{2.029092in}}{\pgfqpoint{2.847830in}{2.021191in}}{\pgfqpoint{2.853654in}{2.015368in}}%
\pgfpathcurveto{\pgfqpoint{2.859478in}{2.009544in}}{\pgfqpoint{2.867378in}{2.006271in}}{\pgfqpoint{2.875614in}{2.006271in}}%
\pgfpathclose%
\pgfusepath{stroke,fill}%
\end{pgfscope}%
\begin{pgfscope}%
\pgfpathrectangle{\pgfqpoint{0.100000in}{0.212622in}}{\pgfqpoint{3.696000in}{3.696000in}}%
\pgfusepath{clip}%
\pgfsetbuttcap%
\pgfsetroundjoin%
\definecolor{currentfill}{rgb}{0.121569,0.466667,0.705882}%
\pgfsetfillcolor{currentfill}%
\pgfsetfillopacity{0.504945}%
\pgfsetlinewidth{1.003750pt}%
\definecolor{currentstroke}{rgb}{0.121569,0.466667,0.705882}%
\pgfsetstrokecolor{currentstroke}%
\pgfsetstrokeopacity{0.504945}%
\pgfsetdash{}{0pt}%
\pgfpathmoveto{\pgfqpoint{2.877697in}{2.005231in}}%
\pgfpathcurveto{\pgfqpoint{2.885934in}{2.005231in}}{\pgfqpoint{2.893834in}{2.008503in}}{\pgfqpoint{2.899658in}{2.014327in}}%
\pgfpathcurveto{\pgfqpoint{2.905482in}{2.020151in}}{\pgfqpoint{2.908754in}{2.028051in}}{\pgfqpoint{2.908754in}{2.036287in}}%
\pgfpathcurveto{\pgfqpoint{2.908754in}{2.044523in}}{\pgfqpoint{2.905482in}{2.052423in}}{\pgfqpoint{2.899658in}{2.058247in}}%
\pgfpathcurveto{\pgfqpoint{2.893834in}{2.064071in}}{\pgfqpoint{2.885934in}{2.067344in}}{\pgfqpoint{2.877697in}{2.067344in}}%
\pgfpathcurveto{\pgfqpoint{2.869461in}{2.067344in}}{\pgfqpoint{2.861561in}{2.064071in}}{\pgfqpoint{2.855737in}{2.058247in}}%
\pgfpathcurveto{\pgfqpoint{2.849913in}{2.052423in}}{\pgfqpoint{2.846641in}{2.044523in}}{\pgfqpoint{2.846641in}{2.036287in}}%
\pgfpathcurveto{\pgfqpoint{2.846641in}{2.028051in}}{\pgfqpoint{2.849913in}{2.020151in}}{\pgfqpoint{2.855737in}{2.014327in}}%
\pgfpathcurveto{\pgfqpoint{2.861561in}{2.008503in}}{\pgfqpoint{2.869461in}{2.005231in}}{\pgfqpoint{2.877697in}{2.005231in}}%
\pgfpathclose%
\pgfusepath{stroke,fill}%
\end{pgfscope}%
\begin{pgfscope}%
\pgfpathrectangle{\pgfqpoint{0.100000in}{0.212622in}}{\pgfqpoint{3.696000in}{3.696000in}}%
\pgfusepath{clip}%
\pgfsetbuttcap%
\pgfsetroundjoin%
\definecolor{currentfill}{rgb}{0.121569,0.466667,0.705882}%
\pgfsetfillcolor{currentfill}%
\pgfsetfillopacity{0.505305}%
\pgfsetlinewidth{1.003750pt}%
\definecolor{currentstroke}{rgb}{0.121569,0.466667,0.705882}%
\pgfsetstrokecolor{currentstroke}%
\pgfsetstrokeopacity{0.505305}%
\pgfsetdash{}{0pt}%
\pgfpathmoveto{\pgfqpoint{1.182811in}{1.751005in}}%
\pgfpathcurveto{\pgfqpoint{1.191047in}{1.751005in}}{\pgfqpoint{1.198947in}{1.754277in}}{\pgfqpoint{1.204771in}{1.760101in}}%
\pgfpathcurveto{\pgfqpoint{1.210595in}{1.765925in}}{\pgfqpoint{1.213867in}{1.773825in}}{\pgfqpoint{1.213867in}{1.782062in}}%
\pgfpathcurveto{\pgfqpoint{1.213867in}{1.790298in}}{\pgfqpoint{1.210595in}{1.798198in}}{\pgfqpoint{1.204771in}{1.804022in}}%
\pgfpathcurveto{\pgfqpoint{1.198947in}{1.809846in}}{\pgfqpoint{1.191047in}{1.813118in}}{\pgfqpoint{1.182811in}{1.813118in}}%
\pgfpathcurveto{\pgfqpoint{1.174574in}{1.813118in}}{\pgfqpoint{1.166674in}{1.809846in}}{\pgfqpoint{1.160850in}{1.804022in}}%
\pgfpathcurveto{\pgfqpoint{1.155026in}{1.798198in}}{\pgfqpoint{1.151754in}{1.790298in}}{\pgfqpoint{1.151754in}{1.782062in}}%
\pgfpathcurveto{\pgfqpoint{1.151754in}{1.773825in}}{\pgfqpoint{1.155026in}{1.765925in}}{\pgfqpoint{1.160850in}{1.760101in}}%
\pgfpathcurveto{\pgfqpoint{1.166674in}{1.754277in}}{\pgfqpoint{1.174574in}{1.751005in}}{\pgfqpoint{1.182811in}{1.751005in}}%
\pgfpathclose%
\pgfusepath{stroke,fill}%
\end{pgfscope}%
\begin{pgfscope}%
\pgfpathrectangle{\pgfqpoint{0.100000in}{0.212622in}}{\pgfqpoint{3.696000in}{3.696000in}}%
\pgfusepath{clip}%
\pgfsetbuttcap%
\pgfsetroundjoin%
\definecolor{currentfill}{rgb}{0.121569,0.466667,0.705882}%
\pgfsetfillcolor{currentfill}%
\pgfsetfillopacity{0.505447}%
\pgfsetlinewidth{1.003750pt}%
\definecolor{currentstroke}{rgb}{0.121569,0.466667,0.705882}%
\pgfsetstrokecolor{currentstroke}%
\pgfsetstrokeopacity{0.505447}%
\pgfsetdash{}{0pt}%
\pgfpathmoveto{\pgfqpoint{2.880271in}{2.005877in}}%
\pgfpathcurveto{\pgfqpoint{2.888507in}{2.005877in}}{\pgfqpoint{2.896407in}{2.009149in}}{\pgfqpoint{2.902231in}{2.014973in}}%
\pgfpathcurveto{\pgfqpoint{2.908055in}{2.020797in}}{\pgfqpoint{2.911328in}{2.028697in}}{\pgfqpoint{2.911328in}{2.036933in}}%
\pgfpathcurveto{\pgfqpoint{2.911328in}{2.045169in}}{\pgfqpoint{2.908055in}{2.053069in}}{\pgfqpoint{2.902231in}{2.058893in}}%
\pgfpathcurveto{\pgfqpoint{2.896407in}{2.064717in}}{\pgfqpoint{2.888507in}{2.067990in}}{\pgfqpoint{2.880271in}{2.067990in}}%
\pgfpathcurveto{\pgfqpoint{2.872035in}{2.067990in}}{\pgfqpoint{2.864135in}{2.064717in}}{\pgfqpoint{2.858311in}{2.058893in}}%
\pgfpathcurveto{\pgfqpoint{2.852487in}{2.053069in}}{\pgfqpoint{2.849215in}{2.045169in}}{\pgfqpoint{2.849215in}{2.036933in}}%
\pgfpathcurveto{\pgfqpoint{2.849215in}{2.028697in}}{\pgfqpoint{2.852487in}{2.020797in}}{\pgfqpoint{2.858311in}{2.014973in}}%
\pgfpathcurveto{\pgfqpoint{2.864135in}{2.009149in}}{\pgfqpoint{2.872035in}{2.005877in}}{\pgfqpoint{2.880271in}{2.005877in}}%
\pgfpathclose%
\pgfusepath{stroke,fill}%
\end{pgfscope}%
\begin{pgfscope}%
\pgfpathrectangle{\pgfqpoint{0.100000in}{0.212622in}}{\pgfqpoint{3.696000in}{3.696000in}}%
\pgfusepath{clip}%
\pgfsetbuttcap%
\pgfsetroundjoin%
\definecolor{currentfill}{rgb}{0.121569,0.466667,0.705882}%
\pgfsetfillcolor{currentfill}%
\pgfsetfillopacity{0.505900}%
\pgfsetlinewidth{1.003750pt}%
\definecolor{currentstroke}{rgb}{0.121569,0.466667,0.705882}%
\pgfsetstrokecolor{currentstroke}%
\pgfsetstrokeopacity{0.505900}%
\pgfsetdash{}{0pt}%
\pgfpathmoveto{\pgfqpoint{2.883525in}{2.004988in}}%
\pgfpathcurveto{\pgfqpoint{2.891762in}{2.004988in}}{\pgfqpoint{2.899662in}{2.008260in}}{\pgfqpoint{2.905486in}{2.014084in}}%
\pgfpathcurveto{\pgfqpoint{2.911310in}{2.019908in}}{\pgfqpoint{2.914582in}{2.027808in}}{\pgfqpoint{2.914582in}{2.036045in}}%
\pgfpathcurveto{\pgfqpoint{2.914582in}{2.044281in}}{\pgfqpoint{2.911310in}{2.052181in}}{\pgfqpoint{2.905486in}{2.058005in}}%
\pgfpathcurveto{\pgfqpoint{2.899662in}{2.063829in}}{\pgfqpoint{2.891762in}{2.067101in}}{\pgfqpoint{2.883525in}{2.067101in}}%
\pgfpathcurveto{\pgfqpoint{2.875289in}{2.067101in}}{\pgfqpoint{2.867389in}{2.063829in}}{\pgfqpoint{2.861565in}{2.058005in}}%
\pgfpathcurveto{\pgfqpoint{2.855741in}{2.052181in}}{\pgfqpoint{2.852469in}{2.044281in}}{\pgfqpoint{2.852469in}{2.036045in}}%
\pgfpathcurveto{\pgfqpoint{2.852469in}{2.027808in}}{\pgfqpoint{2.855741in}{2.019908in}}{\pgfqpoint{2.861565in}{2.014084in}}%
\pgfpathcurveto{\pgfqpoint{2.867389in}{2.008260in}}{\pgfqpoint{2.875289in}{2.004988in}}{\pgfqpoint{2.883525in}{2.004988in}}%
\pgfpathclose%
\pgfusepath{stroke,fill}%
\end{pgfscope}%
\begin{pgfscope}%
\pgfpathrectangle{\pgfqpoint{0.100000in}{0.212622in}}{\pgfqpoint{3.696000in}{3.696000in}}%
\pgfusepath{clip}%
\pgfsetbuttcap%
\pgfsetroundjoin%
\definecolor{currentfill}{rgb}{0.121569,0.466667,0.705882}%
\pgfsetfillcolor{currentfill}%
\pgfsetfillopacity{0.506382}%
\pgfsetlinewidth{1.003750pt}%
\definecolor{currentstroke}{rgb}{0.121569,0.466667,0.705882}%
\pgfsetstrokecolor{currentstroke}%
\pgfsetstrokeopacity{0.506382}%
\pgfsetdash{}{0pt}%
\pgfpathmoveto{\pgfqpoint{1.178588in}{1.746048in}}%
\pgfpathcurveto{\pgfqpoint{1.186824in}{1.746048in}}{\pgfqpoint{1.194724in}{1.749321in}}{\pgfqpoint{1.200548in}{1.755145in}}%
\pgfpathcurveto{\pgfqpoint{1.206372in}{1.760968in}}{\pgfqpoint{1.209645in}{1.768869in}}{\pgfqpoint{1.209645in}{1.777105in}}%
\pgfpathcurveto{\pgfqpoint{1.209645in}{1.785341in}}{\pgfqpoint{1.206372in}{1.793241in}}{\pgfqpoint{1.200548in}{1.799065in}}%
\pgfpathcurveto{\pgfqpoint{1.194724in}{1.804889in}}{\pgfqpoint{1.186824in}{1.808161in}}{\pgfqpoint{1.178588in}{1.808161in}}%
\pgfpathcurveto{\pgfqpoint{1.170352in}{1.808161in}}{\pgfqpoint{1.162452in}{1.804889in}}{\pgfqpoint{1.156628in}{1.799065in}}%
\pgfpathcurveto{\pgfqpoint{1.150804in}{1.793241in}}{\pgfqpoint{1.147532in}{1.785341in}}{\pgfqpoint{1.147532in}{1.777105in}}%
\pgfpathcurveto{\pgfqpoint{1.147532in}{1.768869in}}{\pgfqpoint{1.150804in}{1.760968in}}{\pgfqpoint{1.156628in}{1.755145in}}%
\pgfpathcurveto{\pgfqpoint{1.162452in}{1.749321in}}{\pgfqpoint{1.170352in}{1.746048in}}{\pgfqpoint{1.178588in}{1.746048in}}%
\pgfpathclose%
\pgfusepath{stroke,fill}%
\end{pgfscope}%
\begin{pgfscope}%
\pgfpathrectangle{\pgfqpoint{0.100000in}{0.212622in}}{\pgfqpoint{3.696000in}{3.696000in}}%
\pgfusepath{clip}%
\pgfsetbuttcap%
\pgfsetroundjoin%
\definecolor{currentfill}{rgb}{0.121569,0.466667,0.705882}%
\pgfsetfillcolor{currentfill}%
\pgfsetfillopacity{0.506583}%
\pgfsetlinewidth{1.003750pt}%
\definecolor{currentstroke}{rgb}{0.121569,0.466667,0.705882}%
\pgfsetstrokecolor{currentstroke}%
\pgfsetstrokeopacity{0.506583}%
\pgfsetdash{}{0pt}%
\pgfpathmoveto{\pgfqpoint{2.887584in}{2.005227in}}%
\pgfpathcurveto{\pgfqpoint{2.895820in}{2.005227in}}{\pgfqpoint{2.903720in}{2.008499in}}{\pgfqpoint{2.909544in}{2.014323in}}%
\pgfpathcurveto{\pgfqpoint{2.915368in}{2.020147in}}{\pgfqpoint{2.918640in}{2.028047in}}{\pgfqpoint{2.918640in}{2.036284in}}%
\pgfpathcurveto{\pgfqpoint{2.918640in}{2.044520in}}{\pgfqpoint{2.915368in}{2.052420in}}{\pgfqpoint{2.909544in}{2.058244in}}%
\pgfpathcurveto{\pgfqpoint{2.903720in}{2.064068in}}{\pgfqpoint{2.895820in}{2.067340in}}{\pgfqpoint{2.887584in}{2.067340in}}%
\pgfpathcurveto{\pgfqpoint{2.879348in}{2.067340in}}{\pgfqpoint{2.871447in}{2.064068in}}{\pgfqpoint{2.865624in}{2.058244in}}%
\pgfpathcurveto{\pgfqpoint{2.859800in}{2.052420in}}{\pgfqpoint{2.856527in}{2.044520in}}{\pgfqpoint{2.856527in}{2.036284in}}%
\pgfpathcurveto{\pgfqpoint{2.856527in}{2.028047in}}{\pgfqpoint{2.859800in}{2.020147in}}{\pgfqpoint{2.865624in}{2.014323in}}%
\pgfpathcurveto{\pgfqpoint{2.871447in}{2.008499in}}{\pgfqpoint{2.879348in}{2.005227in}}{\pgfqpoint{2.887584in}{2.005227in}}%
\pgfpathclose%
\pgfusepath{stroke,fill}%
\end{pgfscope}%
\begin{pgfscope}%
\pgfpathrectangle{\pgfqpoint{0.100000in}{0.212622in}}{\pgfqpoint{3.696000in}{3.696000in}}%
\pgfusepath{clip}%
\pgfsetbuttcap%
\pgfsetroundjoin%
\definecolor{currentfill}{rgb}{0.121569,0.466667,0.705882}%
\pgfsetfillcolor{currentfill}%
\pgfsetfillopacity{0.507146}%
\pgfsetlinewidth{1.003750pt}%
\definecolor{currentstroke}{rgb}{0.121569,0.466667,0.705882}%
\pgfsetstrokecolor{currentstroke}%
\pgfsetstrokeopacity{0.507146}%
\pgfsetdash{}{0pt}%
\pgfpathmoveto{\pgfqpoint{2.893119in}{2.003014in}}%
\pgfpathcurveto{\pgfqpoint{2.901356in}{2.003014in}}{\pgfqpoint{2.909256in}{2.006286in}}{\pgfqpoint{2.915080in}{2.012110in}}%
\pgfpathcurveto{\pgfqpoint{2.920904in}{2.017934in}}{\pgfqpoint{2.924176in}{2.025834in}}{\pgfqpoint{2.924176in}{2.034070in}}%
\pgfpathcurveto{\pgfqpoint{2.924176in}{2.042307in}}{\pgfqpoint{2.920904in}{2.050207in}}{\pgfqpoint{2.915080in}{2.056031in}}%
\pgfpathcurveto{\pgfqpoint{2.909256in}{2.061854in}}{\pgfqpoint{2.901356in}{2.065127in}}{\pgfqpoint{2.893119in}{2.065127in}}%
\pgfpathcurveto{\pgfqpoint{2.884883in}{2.065127in}}{\pgfqpoint{2.876983in}{2.061854in}}{\pgfqpoint{2.871159in}{2.056031in}}%
\pgfpathcurveto{\pgfqpoint{2.865335in}{2.050207in}}{\pgfqpoint{2.862063in}{2.042307in}}{\pgfqpoint{2.862063in}{2.034070in}}%
\pgfpathcurveto{\pgfqpoint{2.862063in}{2.025834in}}{\pgfqpoint{2.865335in}{2.017934in}}{\pgfqpoint{2.871159in}{2.012110in}}%
\pgfpathcurveto{\pgfqpoint{2.876983in}{2.006286in}}{\pgfqpoint{2.884883in}{2.003014in}}{\pgfqpoint{2.893119in}{2.003014in}}%
\pgfpathclose%
\pgfusepath{stroke,fill}%
\end{pgfscope}%
\begin{pgfscope}%
\pgfpathrectangle{\pgfqpoint{0.100000in}{0.212622in}}{\pgfqpoint{3.696000in}{3.696000in}}%
\pgfusepath{clip}%
\pgfsetbuttcap%
\pgfsetroundjoin%
\definecolor{currentfill}{rgb}{0.121569,0.466667,0.705882}%
\pgfsetfillcolor{currentfill}%
\pgfsetfillopacity{0.507757}%
\pgfsetlinewidth{1.003750pt}%
\definecolor{currentstroke}{rgb}{0.121569,0.466667,0.705882}%
\pgfsetstrokecolor{currentstroke}%
\pgfsetstrokeopacity{0.507757}%
\pgfsetdash{}{0pt}%
\pgfpathmoveto{\pgfqpoint{1.176048in}{1.742434in}}%
\pgfpathcurveto{\pgfqpoint{1.184284in}{1.742434in}}{\pgfqpoint{1.192184in}{1.745706in}}{\pgfqpoint{1.198008in}{1.751530in}}%
\pgfpathcurveto{\pgfqpoint{1.203832in}{1.757354in}}{\pgfqpoint{1.207104in}{1.765254in}}{\pgfqpoint{1.207104in}{1.773490in}}%
\pgfpathcurveto{\pgfqpoint{1.207104in}{1.781727in}}{\pgfqpoint{1.203832in}{1.789627in}}{\pgfqpoint{1.198008in}{1.795450in}}%
\pgfpathcurveto{\pgfqpoint{1.192184in}{1.801274in}}{\pgfqpoint{1.184284in}{1.804547in}}{\pgfqpoint{1.176048in}{1.804547in}}%
\pgfpathcurveto{\pgfqpoint{1.167811in}{1.804547in}}{\pgfqpoint{1.159911in}{1.801274in}}{\pgfqpoint{1.154087in}{1.795450in}}%
\pgfpathcurveto{\pgfqpoint{1.148264in}{1.789627in}}{\pgfqpoint{1.144991in}{1.781727in}}{\pgfqpoint{1.144991in}{1.773490in}}%
\pgfpathcurveto{\pgfqpoint{1.144991in}{1.765254in}}{\pgfqpoint{1.148264in}{1.757354in}}{\pgfqpoint{1.154087in}{1.751530in}}%
\pgfpathcurveto{\pgfqpoint{1.159911in}{1.745706in}}{\pgfqpoint{1.167811in}{1.742434in}}{\pgfqpoint{1.176048in}{1.742434in}}%
\pgfpathclose%
\pgfusepath{stroke,fill}%
\end{pgfscope}%
\begin{pgfscope}%
\pgfpathrectangle{\pgfqpoint{0.100000in}{0.212622in}}{\pgfqpoint{3.696000in}{3.696000in}}%
\pgfusepath{clip}%
\pgfsetbuttcap%
\pgfsetroundjoin%
\definecolor{currentfill}{rgb}{0.121569,0.466667,0.705882}%
\pgfsetfillcolor{currentfill}%
\pgfsetfillopacity{0.508224}%
\pgfsetlinewidth{1.003750pt}%
\definecolor{currentstroke}{rgb}{0.121569,0.466667,0.705882}%
\pgfsetstrokecolor{currentstroke}%
\pgfsetstrokeopacity{0.508224}%
\pgfsetdash{}{0pt}%
\pgfpathmoveto{\pgfqpoint{2.899507in}{2.005975in}}%
\pgfpathcurveto{\pgfqpoint{2.907744in}{2.005975in}}{\pgfqpoint{2.915644in}{2.009248in}}{\pgfqpoint{2.921468in}{2.015072in}}%
\pgfpathcurveto{\pgfqpoint{2.927292in}{2.020895in}}{\pgfqpoint{2.930564in}{2.028796in}}{\pgfqpoint{2.930564in}{2.037032in}}%
\pgfpathcurveto{\pgfqpoint{2.930564in}{2.045268in}}{\pgfqpoint{2.927292in}{2.053168in}}{\pgfqpoint{2.921468in}{2.058992in}}%
\pgfpathcurveto{\pgfqpoint{2.915644in}{2.064816in}}{\pgfqpoint{2.907744in}{2.068088in}}{\pgfqpoint{2.899507in}{2.068088in}}%
\pgfpathcurveto{\pgfqpoint{2.891271in}{2.068088in}}{\pgfqpoint{2.883371in}{2.064816in}}{\pgfqpoint{2.877547in}{2.058992in}}%
\pgfpathcurveto{\pgfqpoint{2.871723in}{2.053168in}}{\pgfqpoint{2.868451in}{2.045268in}}{\pgfqpoint{2.868451in}{2.037032in}}%
\pgfpathcurveto{\pgfqpoint{2.868451in}{2.028796in}}{\pgfqpoint{2.871723in}{2.020895in}}{\pgfqpoint{2.877547in}{2.015072in}}%
\pgfpathcurveto{\pgfqpoint{2.883371in}{2.009248in}}{\pgfqpoint{2.891271in}{2.005975in}}{\pgfqpoint{2.899507in}{2.005975in}}%
\pgfpathclose%
\pgfusepath{stroke,fill}%
\end{pgfscope}%
\begin{pgfscope}%
\pgfpathrectangle{\pgfqpoint{0.100000in}{0.212622in}}{\pgfqpoint{3.696000in}{3.696000in}}%
\pgfusepath{clip}%
\pgfsetbuttcap%
\pgfsetroundjoin%
\definecolor{currentfill}{rgb}{0.121569,0.466667,0.705882}%
\pgfsetfillcolor{currentfill}%
\pgfsetfillopacity{0.508515}%
\pgfsetlinewidth{1.003750pt}%
\definecolor{currentstroke}{rgb}{0.121569,0.466667,0.705882}%
\pgfsetstrokecolor{currentstroke}%
\pgfsetstrokeopacity{0.508515}%
\pgfsetdash{}{0pt}%
\pgfpathmoveto{\pgfqpoint{2.905801in}{2.001415in}}%
\pgfpathcurveto{\pgfqpoint{2.914037in}{2.001415in}}{\pgfqpoint{2.921937in}{2.004687in}}{\pgfqpoint{2.927761in}{2.010511in}}%
\pgfpathcurveto{\pgfqpoint{2.933585in}{2.016335in}}{\pgfqpoint{2.936857in}{2.024235in}}{\pgfqpoint{2.936857in}{2.032471in}}%
\pgfpathcurveto{\pgfqpoint{2.936857in}{2.040708in}}{\pgfqpoint{2.933585in}{2.048608in}}{\pgfqpoint{2.927761in}{2.054432in}}%
\pgfpathcurveto{\pgfqpoint{2.921937in}{2.060256in}}{\pgfqpoint{2.914037in}{2.063528in}}{\pgfqpoint{2.905801in}{2.063528in}}%
\pgfpathcurveto{\pgfqpoint{2.897565in}{2.063528in}}{\pgfqpoint{2.889665in}{2.060256in}}{\pgfqpoint{2.883841in}{2.054432in}}%
\pgfpathcurveto{\pgfqpoint{2.878017in}{2.048608in}}{\pgfqpoint{2.874744in}{2.040708in}}{\pgfqpoint{2.874744in}{2.032471in}}%
\pgfpathcurveto{\pgfqpoint{2.874744in}{2.024235in}}{\pgfqpoint{2.878017in}{2.016335in}}{\pgfqpoint{2.883841in}{2.010511in}}%
\pgfpathcurveto{\pgfqpoint{2.889665in}{2.004687in}}{\pgfqpoint{2.897565in}{2.001415in}}{\pgfqpoint{2.905801in}{2.001415in}}%
\pgfpathclose%
\pgfusepath{stroke,fill}%
\end{pgfscope}%
\begin{pgfscope}%
\pgfpathrectangle{\pgfqpoint{0.100000in}{0.212622in}}{\pgfqpoint{3.696000in}{3.696000in}}%
\pgfusepath{clip}%
\pgfsetbuttcap%
\pgfsetroundjoin%
\definecolor{currentfill}{rgb}{0.121569,0.466667,0.705882}%
\pgfsetfillcolor{currentfill}%
\pgfsetfillopacity{0.508908}%
\pgfsetlinewidth{1.003750pt}%
\definecolor{currentstroke}{rgb}{0.121569,0.466667,0.705882}%
\pgfsetstrokecolor{currentstroke}%
\pgfsetstrokeopacity{0.508908}%
\pgfsetdash{}{0pt}%
\pgfpathmoveto{\pgfqpoint{1.171250in}{1.741141in}}%
\pgfpathcurveto{\pgfqpoint{1.179486in}{1.741141in}}{\pgfqpoint{1.187386in}{1.744413in}}{\pgfqpoint{1.193210in}{1.750237in}}%
\pgfpathcurveto{\pgfqpoint{1.199034in}{1.756061in}}{\pgfqpoint{1.202306in}{1.763961in}}{\pgfqpoint{1.202306in}{1.772197in}}%
\pgfpathcurveto{\pgfqpoint{1.202306in}{1.780433in}}{\pgfqpoint{1.199034in}{1.788333in}}{\pgfqpoint{1.193210in}{1.794157in}}%
\pgfpathcurveto{\pgfqpoint{1.187386in}{1.799981in}}{\pgfqpoint{1.179486in}{1.803254in}}{\pgfqpoint{1.171250in}{1.803254in}}%
\pgfpathcurveto{\pgfqpoint{1.163013in}{1.803254in}}{\pgfqpoint{1.155113in}{1.799981in}}{\pgfqpoint{1.149289in}{1.794157in}}%
\pgfpathcurveto{\pgfqpoint{1.143465in}{1.788333in}}{\pgfqpoint{1.140193in}{1.780433in}}{\pgfqpoint{1.140193in}{1.772197in}}%
\pgfpathcurveto{\pgfqpoint{1.140193in}{1.763961in}}{\pgfqpoint{1.143465in}{1.756061in}}{\pgfqpoint{1.149289in}{1.750237in}}%
\pgfpathcurveto{\pgfqpoint{1.155113in}{1.744413in}}{\pgfqpoint{1.163013in}{1.741141in}}{\pgfqpoint{1.171250in}{1.741141in}}%
\pgfpathclose%
\pgfusepath{stroke,fill}%
\end{pgfscope}%
\begin{pgfscope}%
\pgfpathrectangle{\pgfqpoint{0.100000in}{0.212622in}}{\pgfqpoint{3.696000in}{3.696000in}}%
\pgfusepath{clip}%
\pgfsetbuttcap%
\pgfsetroundjoin%
\definecolor{currentfill}{rgb}{0.121569,0.466667,0.705882}%
\pgfsetfillcolor{currentfill}%
\pgfsetfillopacity{0.509902}%
\pgfsetlinewidth{1.003750pt}%
\definecolor{currentstroke}{rgb}{0.121569,0.466667,0.705882}%
\pgfsetstrokecolor{currentstroke}%
\pgfsetstrokeopacity{0.509902}%
\pgfsetdash{}{0pt}%
\pgfpathmoveto{\pgfqpoint{2.912991in}{2.004686in}}%
\pgfpathcurveto{\pgfqpoint{2.921228in}{2.004686in}}{\pgfqpoint{2.929128in}{2.007959in}}{\pgfqpoint{2.934952in}{2.013783in}}%
\pgfpathcurveto{\pgfqpoint{2.940775in}{2.019607in}}{\pgfqpoint{2.944048in}{2.027507in}}{\pgfqpoint{2.944048in}{2.035743in}}%
\pgfpathcurveto{\pgfqpoint{2.944048in}{2.043979in}}{\pgfqpoint{2.940775in}{2.051879in}}{\pgfqpoint{2.934952in}{2.057703in}}%
\pgfpathcurveto{\pgfqpoint{2.929128in}{2.063527in}}{\pgfqpoint{2.921228in}{2.066799in}}{\pgfqpoint{2.912991in}{2.066799in}}%
\pgfpathcurveto{\pgfqpoint{2.904755in}{2.066799in}}{\pgfqpoint{2.896855in}{2.063527in}}{\pgfqpoint{2.891031in}{2.057703in}}%
\pgfpathcurveto{\pgfqpoint{2.885207in}{2.051879in}}{\pgfqpoint{2.881935in}{2.043979in}}{\pgfqpoint{2.881935in}{2.035743in}}%
\pgfpathcurveto{\pgfqpoint{2.881935in}{2.027507in}}{\pgfqpoint{2.885207in}{2.019607in}}{\pgfqpoint{2.891031in}{2.013783in}}%
\pgfpathcurveto{\pgfqpoint{2.896855in}{2.007959in}}{\pgfqpoint{2.904755in}{2.004686in}}{\pgfqpoint{2.912991in}{2.004686in}}%
\pgfpathclose%
\pgfusepath{stroke,fill}%
\end{pgfscope}%
\begin{pgfscope}%
\pgfpathrectangle{\pgfqpoint{0.100000in}{0.212622in}}{\pgfqpoint{3.696000in}{3.696000in}}%
\pgfusepath{clip}%
\pgfsetbuttcap%
\pgfsetroundjoin%
\definecolor{currentfill}{rgb}{0.121569,0.466667,0.705882}%
\pgfsetfillcolor{currentfill}%
\pgfsetfillopacity{0.510165}%
\pgfsetlinewidth{1.003750pt}%
\definecolor{currentstroke}{rgb}{0.121569,0.466667,0.705882}%
\pgfsetstrokecolor{currentstroke}%
\pgfsetstrokeopacity{0.510165}%
\pgfsetdash{}{0pt}%
\pgfpathmoveto{\pgfqpoint{1.169286in}{1.738893in}}%
\pgfpathcurveto{\pgfqpoint{1.177522in}{1.738893in}}{\pgfqpoint{1.185422in}{1.742165in}}{\pgfqpoint{1.191246in}{1.747989in}}%
\pgfpathcurveto{\pgfqpoint{1.197070in}{1.753813in}}{\pgfqpoint{1.200343in}{1.761713in}}{\pgfqpoint{1.200343in}{1.769950in}}%
\pgfpathcurveto{\pgfqpoint{1.200343in}{1.778186in}}{\pgfqpoint{1.197070in}{1.786086in}}{\pgfqpoint{1.191246in}{1.791910in}}%
\pgfpathcurveto{\pgfqpoint{1.185422in}{1.797734in}}{\pgfqpoint{1.177522in}{1.801006in}}{\pgfqpoint{1.169286in}{1.801006in}}%
\pgfpathcurveto{\pgfqpoint{1.161050in}{1.801006in}}{\pgfqpoint{1.153150in}{1.797734in}}{\pgfqpoint{1.147326in}{1.791910in}}%
\pgfpathcurveto{\pgfqpoint{1.141502in}{1.786086in}}{\pgfqpoint{1.138230in}{1.778186in}}{\pgfqpoint{1.138230in}{1.769950in}}%
\pgfpathcurveto{\pgfqpoint{1.138230in}{1.761713in}}{\pgfqpoint{1.141502in}{1.753813in}}{\pgfqpoint{1.147326in}{1.747989in}}%
\pgfpathcurveto{\pgfqpoint{1.153150in}{1.742165in}}{\pgfqpoint{1.161050in}{1.738893in}}{\pgfqpoint{1.169286in}{1.738893in}}%
\pgfpathclose%
\pgfusepath{stroke,fill}%
\end{pgfscope}%
\begin{pgfscope}%
\pgfpathrectangle{\pgfqpoint{0.100000in}{0.212622in}}{\pgfqpoint{3.696000in}{3.696000in}}%
\pgfusepath{clip}%
\pgfsetbuttcap%
\pgfsetroundjoin%
\definecolor{currentfill}{rgb}{0.121569,0.466667,0.705882}%
\pgfsetfillcolor{currentfill}%
\pgfsetfillopacity{0.510284}%
\pgfsetlinewidth{1.003750pt}%
\definecolor{currentstroke}{rgb}{0.121569,0.466667,0.705882}%
\pgfsetstrokecolor{currentstroke}%
\pgfsetstrokeopacity{0.510284}%
\pgfsetdash{}{0pt}%
\pgfpathmoveto{\pgfqpoint{2.916589in}{2.002739in}}%
\pgfpathcurveto{\pgfqpoint{2.924826in}{2.002739in}}{\pgfqpoint{2.932726in}{2.006011in}}{\pgfqpoint{2.938550in}{2.011835in}}%
\pgfpathcurveto{\pgfqpoint{2.944374in}{2.017659in}}{\pgfqpoint{2.947646in}{2.025559in}}{\pgfqpoint{2.947646in}{2.033795in}}%
\pgfpathcurveto{\pgfqpoint{2.947646in}{2.042031in}}{\pgfqpoint{2.944374in}{2.049931in}}{\pgfqpoint{2.938550in}{2.055755in}}%
\pgfpathcurveto{\pgfqpoint{2.932726in}{2.061579in}}{\pgfqpoint{2.924826in}{2.064852in}}{\pgfqpoint{2.916589in}{2.064852in}}%
\pgfpathcurveto{\pgfqpoint{2.908353in}{2.064852in}}{\pgfqpoint{2.900453in}{2.061579in}}{\pgfqpoint{2.894629in}{2.055755in}}%
\pgfpathcurveto{\pgfqpoint{2.888805in}{2.049931in}}{\pgfqpoint{2.885533in}{2.042031in}}{\pgfqpoint{2.885533in}{2.033795in}}%
\pgfpathcurveto{\pgfqpoint{2.885533in}{2.025559in}}{\pgfqpoint{2.888805in}{2.017659in}}{\pgfqpoint{2.894629in}{2.011835in}}%
\pgfpathcurveto{\pgfqpoint{2.900453in}{2.006011in}}{\pgfqpoint{2.908353in}{2.002739in}}{\pgfqpoint{2.916589in}{2.002739in}}%
\pgfpathclose%
\pgfusepath{stroke,fill}%
\end{pgfscope}%
\begin{pgfscope}%
\pgfpathrectangle{\pgfqpoint{0.100000in}{0.212622in}}{\pgfqpoint{3.696000in}{3.696000in}}%
\pgfusepath{clip}%
\pgfsetbuttcap%
\pgfsetroundjoin%
\definecolor{currentfill}{rgb}{0.121569,0.466667,0.705882}%
\pgfsetfillcolor{currentfill}%
\pgfsetfillopacity{0.510771}%
\pgfsetlinewidth{1.003750pt}%
\definecolor{currentstroke}{rgb}{0.121569,0.466667,0.705882}%
\pgfsetstrokecolor{currentstroke}%
\pgfsetstrokeopacity{0.510771}%
\pgfsetdash{}{0pt}%
\pgfpathmoveto{\pgfqpoint{1.166382in}{1.737733in}}%
\pgfpathcurveto{\pgfqpoint{1.174618in}{1.737733in}}{\pgfqpoint{1.182518in}{1.741006in}}{\pgfqpoint{1.188342in}{1.746830in}}%
\pgfpathcurveto{\pgfqpoint{1.194166in}{1.752653in}}{\pgfqpoint{1.197438in}{1.760554in}}{\pgfqpoint{1.197438in}{1.768790in}}%
\pgfpathcurveto{\pgfqpoint{1.197438in}{1.777026in}}{\pgfqpoint{1.194166in}{1.784926in}}{\pgfqpoint{1.188342in}{1.790750in}}%
\pgfpathcurveto{\pgfqpoint{1.182518in}{1.796574in}}{\pgfqpoint{1.174618in}{1.799846in}}{\pgfqpoint{1.166382in}{1.799846in}}%
\pgfpathcurveto{\pgfqpoint{1.158146in}{1.799846in}}{\pgfqpoint{1.150246in}{1.796574in}}{\pgfqpoint{1.144422in}{1.790750in}}%
\pgfpathcurveto{\pgfqpoint{1.138598in}{1.784926in}}{\pgfqpoint{1.135325in}{1.777026in}}{\pgfqpoint{1.135325in}{1.768790in}}%
\pgfpathcurveto{\pgfqpoint{1.135325in}{1.760554in}}{\pgfqpoint{1.138598in}{1.752653in}}{\pgfqpoint{1.144422in}{1.746830in}}%
\pgfpathcurveto{\pgfqpoint{1.150246in}{1.741006in}}{\pgfqpoint{1.158146in}{1.737733in}}{\pgfqpoint{1.166382in}{1.737733in}}%
\pgfpathclose%
\pgfusepath{stroke,fill}%
\end{pgfscope}%
\begin{pgfscope}%
\pgfpathrectangle{\pgfqpoint{0.100000in}{0.212622in}}{\pgfqpoint{3.696000in}{3.696000in}}%
\pgfusepath{clip}%
\pgfsetbuttcap%
\pgfsetroundjoin%
\definecolor{currentfill}{rgb}{0.121569,0.466667,0.705882}%
\pgfsetfillcolor{currentfill}%
\pgfsetfillopacity{0.511058}%
\pgfsetlinewidth{1.003750pt}%
\definecolor{currentstroke}{rgb}{0.121569,0.466667,0.705882}%
\pgfsetstrokecolor{currentstroke}%
\pgfsetstrokeopacity{0.511058}%
\pgfsetdash{}{0pt}%
\pgfpathmoveto{\pgfqpoint{2.921405in}{2.003074in}}%
\pgfpathcurveto{\pgfqpoint{2.929642in}{2.003074in}}{\pgfqpoint{2.937542in}{2.006346in}}{\pgfqpoint{2.943366in}{2.012170in}}%
\pgfpathcurveto{\pgfqpoint{2.949190in}{2.017994in}}{\pgfqpoint{2.952462in}{2.025894in}}{\pgfqpoint{2.952462in}{2.034130in}}%
\pgfpathcurveto{\pgfqpoint{2.952462in}{2.042366in}}{\pgfqpoint{2.949190in}{2.050267in}}{\pgfqpoint{2.943366in}{2.056090in}}%
\pgfpathcurveto{\pgfqpoint{2.937542in}{2.061914in}}{\pgfqpoint{2.929642in}{2.065187in}}{\pgfqpoint{2.921405in}{2.065187in}}%
\pgfpathcurveto{\pgfqpoint{2.913169in}{2.065187in}}{\pgfqpoint{2.905269in}{2.061914in}}{\pgfqpoint{2.899445in}{2.056090in}}%
\pgfpathcurveto{\pgfqpoint{2.893621in}{2.050267in}}{\pgfqpoint{2.890349in}{2.042366in}}{\pgfqpoint{2.890349in}{2.034130in}}%
\pgfpathcurveto{\pgfqpoint{2.890349in}{2.025894in}}{\pgfqpoint{2.893621in}{2.017994in}}{\pgfqpoint{2.899445in}{2.012170in}}%
\pgfpathcurveto{\pgfqpoint{2.905269in}{2.006346in}}{\pgfqpoint{2.913169in}{2.003074in}}{\pgfqpoint{2.921405in}{2.003074in}}%
\pgfpathclose%
\pgfusepath{stroke,fill}%
\end{pgfscope}%
\begin{pgfscope}%
\pgfpathrectangle{\pgfqpoint{0.100000in}{0.212622in}}{\pgfqpoint{3.696000in}{3.696000in}}%
\pgfusepath{clip}%
\pgfsetbuttcap%
\pgfsetroundjoin%
\definecolor{currentfill}{rgb}{0.121569,0.466667,0.705882}%
\pgfsetfillcolor{currentfill}%
\pgfsetfillopacity{0.511726}%
\pgfsetlinewidth{1.003750pt}%
\definecolor{currentstroke}{rgb}{0.121569,0.466667,0.705882}%
\pgfsetstrokecolor{currentstroke}%
\pgfsetstrokeopacity{0.511726}%
\pgfsetdash{}{0pt}%
\pgfpathmoveto{\pgfqpoint{1.165101in}{1.739110in}}%
\pgfpathcurveto{\pgfqpoint{1.173338in}{1.739110in}}{\pgfqpoint{1.181238in}{1.742383in}}{\pgfqpoint{1.187062in}{1.748206in}}%
\pgfpathcurveto{\pgfqpoint{1.192886in}{1.754030in}}{\pgfqpoint{1.196158in}{1.761930in}}{\pgfqpoint{1.196158in}{1.770167in}}%
\pgfpathcurveto{\pgfqpoint{1.196158in}{1.778403in}}{\pgfqpoint{1.192886in}{1.786303in}}{\pgfqpoint{1.187062in}{1.792127in}}%
\pgfpathcurveto{\pgfqpoint{1.181238in}{1.797951in}}{\pgfqpoint{1.173338in}{1.801223in}}{\pgfqpoint{1.165101in}{1.801223in}}%
\pgfpathcurveto{\pgfqpoint{1.156865in}{1.801223in}}{\pgfqpoint{1.148965in}{1.797951in}}{\pgfqpoint{1.143141in}{1.792127in}}%
\pgfpathcurveto{\pgfqpoint{1.137317in}{1.786303in}}{\pgfqpoint{1.134045in}{1.778403in}}{\pgfqpoint{1.134045in}{1.770167in}}%
\pgfpathcurveto{\pgfqpoint{1.134045in}{1.761930in}}{\pgfqpoint{1.137317in}{1.754030in}}{\pgfqpoint{1.143141in}{1.748206in}}%
\pgfpathcurveto{\pgfqpoint{1.148965in}{1.742383in}}{\pgfqpoint{1.156865in}{1.739110in}}{\pgfqpoint{1.165101in}{1.739110in}}%
\pgfpathclose%
\pgfusepath{stroke,fill}%
\end{pgfscope}%
\begin{pgfscope}%
\pgfpathrectangle{\pgfqpoint{0.100000in}{0.212622in}}{\pgfqpoint{3.696000in}{3.696000in}}%
\pgfusepath{clip}%
\pgfsetbuttcap%
\pgfsetroundjoin%
\definecolor{currentfill}{rgb}{0.121569,0.466667,0.705882}%
\pgfsetfillcolor{currentfill}%
\pgfsetfillopacity{0.511902}%
\pgfsetlinewidth{1.003750pt}%
\definecolor{currentstroke}{rgb}{0.121569,0.466667,0.705882}%
\pgfsetstrokecolor{currentstroke}%
\pgfsetstrokeopacity{0.511902}%
\pgfsetdash{}{0pt}%
\pgfpathmoveto{\pgfqpoint{1.164073in}{1.739292in}}%
\pgfpathcurveto{\pgfqpoint{1.172309in}{1.739292in}}{\pgfqpoint{1.180209in}{1.742565in}}{\pgfqpoint{1.186033in}{1.748389in}}%
\pgfpathcurveto{\pgfqpoint{1.191857in}{1.754212in}}{\pgfqpoint{1.195129in}{1.762113in}}{\pgfqpoint{1.195129in}{1.770349in}}%
\pgfpathcurveto{\pgfqpoint{1.195129in}{1.778585in}}{\pgfqpoint{1.191857in}{1.786485in}}{\pgfqpoint{1.186033in}{1.792309in}}%
\pgfpathcurveto{\pgfqpoint{1.180209in}{1.798133in}}{\pgfqpoint{1.172309in}{1.801405in}}{\pgfqpoint{1.164073in}{1.801405in}}%
\pgfpathcurveto{\pgfqpoint{1.155837in}{1.801405in}}{\pgfqpoint{1.147936in}{1.798133in}}{\pgfqpoint{1.142113in}{1.792309in}}%
\pgfpathcurveto{\pgfqpoint{1.136289in}{1.786485in}}{\pgfqpoint{1.133016in}{1.778585in}}{\pgfqpoint{1.133016in}{1.770349in}}%
\pgfpathcurveto{\pgfqpoint{1.133016in}{1.762113in}}{\pgfqpoint{1.136289in}{1.754212in}}{\pgfqpoint{1.142113in}{1.748389in}}%
\pgfpathcurveto{\pgfqpoint{1.147936in}{1.742565in}}{\pgfqpoint{1.155837in}{1.739292in}}{\pgfqpoint{1.164073in}{1.739292in}}%
\pgfpathclose%
\pgfusepath{stroke,fill}%
\end{pgfscope}%
\begin{pgfscope}%
\pgfpathrectangle{\pgfqpoint{0.100000in}{0.212622in}}{\pgfqpoint{3.696000in}{3.696000in}}%
\pgfusepath{clip}%
\pgfsetbuttcap%
\pgfsetroundjoin%
\definecolor{currentfill}{rgb}{0.121569,0.466667,0.705882}%
\pgfsetfillcolor{currentfill}%
\pgfsetfillopacity{0.512036}%
\pgfsetlinewidth{1.003750pt}%
\definecolor{currentstroke}{rgb}{0.121569,0.466667,0.705882}%
\pgfsetstrokecolor{currentstroke}%
\pgfsetstrokeopacity{0.512036}%
\pgfsetdash{}{0pt}%
\pgfpathmoveto{\pgfqpoint{1.162704in}{1.737095in}}%
\pgfpathcurveto{\pgfqpoint{1.170940in}{1.737095in}}{\pgfqpoint{1.178840in}{1.740367in}}{\pgfqpoint{1.184664in}{1.746191in}}%
\pgfpathcurveto{\pgfqpoint{1.190488in}{1.752015in}}{\pgfqpoint{1.193761in}{1.759915in}}{\pgfqpoint{1.193761in}{1.768152in}}%
\pgfpathcurveto{\pgfqpoint{1.193761in}{1.776388in}}{\pgfqpoint{1.190488in}{1.784288in}}{\pgfqpoint{1.184664in}{1.790112in}}%
\pgfpathcurveto{\pgfqpoint{1.178840in}{1.795936in}}{\pgfqpoint{1.170940in}{1.799208in}}{\pgfqpoint{1.162704in}{1.799208in}}%
\pgfpathcurveto{\pgfqpoint{1.154468in}{1.799208in}}{\pgfqpoint{1.146568in}{1.795936in}}{\pgfqpoint{1.140744in}{1.790112in}}%
\pgfpathcurveto{\pgfqpoint{1.134920in}{1.784288in}}{\pgfqpoint{1.131648in}{1.776388in}}{\pgfqpoint{1.131648in}{1.768152in}}%
\pgfpathcurveto{\pgfqpoint{1.131648in}{1.759915in}}{\pgfqpoint{1.134920in}{1.752015in}}{\pgfqpoint{1.140744in}{1.746191in}}%
\pgfpathcurveto{\pgfqpoint{1.146568in}{1.740367in}}{\pgfqpoint{1.154468in}{1.737095in}}{\pgfqpoint{1.162704in}{1.737095in}}%
\pgfpathclose%
\pgfusepath{stroke,fill}%
\end{pgfscope}%
\begin{pgfscope}%
\pgfpathrectangle{\pgfqpoint{0.100000in}{0.212622in}}{\pgfqpoint{3.696000in}{3.696000in}}%
\pgfusepath{clip}%
\pgfsetbuttcap%
\pgfsetroundjoin%
\definecolor{currentfill}{rgb}{0.121569,0.466667,0.705882}%
\pgfsetfillcolor{currentfill}%
\pgfsetfillopacity{0.512118}%
\pgfsetlinewidth{1.003750pt}%
\definecolor{currentstroke}{rgb}{0.121569,0.466667,0.705882}%
\pgfsetstrokecolor{currentstroke}%
\pgfsetstrokeopacity{0.512118}%
\pgfsetdash{}{0pt}%
\pgfpathmoveto{\pgfqpoint{2.928135in}{2.003240in}}%
\pgfpathcurveto{\pgfqpoint{2.936371in}{2.003240in}}{\pgfqpoint{2.944271in}{2.006512in}}{\pgfqpoint{2.950095in}{2.012336in}}%
\pgfpathcurveto{\pgfqpoint{2.955919in}{2.018160in}}{\pgfqpoint{2.959192in}{2.026060in}}{\pgfqpoint{2.959192in}{2.034296in}}%
\pgfpathcurveto{\pgfqpoint{2.959192in}{2.042533in}}{\pgfqpoint{2.955919in}{2.050433in}}{\pgfqpoint{2.950095in}{2.056257in}}%
\pgfpathcurveto{\pgfqpoint{2.944271in}{2.062081in}}{\pgfqpoint{2.936371in}{2.065353in}}{\pgfqpoint{2.928135in}{2.065353in}}%
\pgfpathcurveto{\pgfqpoint{2.919899in}{2.065353in}}{\pgfqpoint{2.911999in}{2.062081in}}{\pgfqpoint{2.906175in}{2.056257in}}%
\pgfpathcurveto{\pgfqpoint{2.900351in}{2.050433in}}{\pgfqpoint{2.897079in}{2.042533in}}{\pgfqpoint{2.897079in}{2.034296in}}%
\pgfpathcurveto{\pgfqpoint{2.897079in}{2.026060in}}{\pgfqpoint{2.900351in}{2.018160in}}{\pgfqpoint{2.906175in}{2.012336in}}%
\pgfpathcurveto{\pgfqpoint{2.911999in}{2.006512in}}{\pgfqpoint{2.919899in}{2.003240in}}{\pgfqpoint{2.928135in}{2.003240in}}%
\pgfpathclose%
\pgfusepath{stroke,fill}%
\end{pgfscope}%
\begin{pgfscope}%
\pgfpathrectangle{\pgfqpoint{0.100000in}{0.212622in}}{\pgfqpoint{3.696000in}{3.696000in}}%
\pgfusepath{clip}%
\pgfsetbuttcap%
\pgfsetroundjoin%
\definecolor{currentfill}{rgb}{0.121569,0.466667,0.705882}%
\pgfsetfillcolor{currentfill}%
\pgfsetfillopacity{0.512632}%
\pgfsetlinewidth{1.003750pt}%
\definecolor{currentstroke}{rgb}{0.121569,0.466667,0.705882}%
\pgfsetstrokecolor{currentstroke}%
\pgfsetstrokeopacity{0.512632}%
\pgfsetdash{}{0pt}%
\pgfpathmoveto{\pgfqpoint{1.161029in}{1.734499in}}%
\pgfpathcurveto{\pgfqpoint{1.169266in}{1.734499in}}{\pgfqpoint{1.177166in}{1.737771in}}{\pgfqpoint{1.182990in}{1.743595in}}%
\pgfpathcurveto{\pgfqpoint{1.188814in}{1.749419in}}{\pgfqpoint{1.192086in}{1.757319in}}{\pgfqpoint{1.192086in}{1.765556in}}%
\pgfpathcurveto{\pgfqpoint{1.192086in}{1.773792in}}{\pgfqpoint{1.188814in}{1.781692in}}{\pgfqpoint{1.182990in}{1.787516in}}%
\pgfpathcurveto{\pgfqpoint{1.177166in}{1.793340in}}{\pgfqpoint{1.169266in}{1.796612in}}{\pgfqpoint{1.161029in}{1.796612in}}%
\pgfpathcurveto{\pgfqpoint{1.152793in}{1.796612in}}{\pgfqpoint{1.144893in}{1.793340in}}{\pgfqpoint{1.139069in}{1.787516in}}%
\pgfpathcurveto{\pgfqpoint{1.133245in}{1.781692in}}{\pgfqpoint{1.129973in}{1.773792in}}{\pgfqpoint{1.129973in}{1.765556in}}%
\pgfpathcurveto{\pgfqpoint{1.129973in}{1.757319in}}{\pgfqpoint{1.133245in}{1.749419in}}{\pgfqpoint{1.139069in}{1.743595in}}%
\pgfpathcurveto{\pgfqpoint{1.144893in}{1.737771in}}{\pgfqpoint{1.152793in}{1.734499in}}{\pgfqpoint{1.161029in}{1.734499in}}%
\pgfpathclose%
\pgfusepath{stroke,fill}%
\end{pgfscope}%
\begin{pgfscope}%
\pgfpathrectangle{\pgfqpoint{0.100000in}{0.212622in}}{\pgfqpoint{3.696000in}{3.696000in}}%
\pgfusepath{clip}%
\pgfsetbuttcap%
\pgfsetroundjoin%
\definecolor{currentfill}{rgb}{0.121569,0.466667,0.705882}%
\pgfsetfillcolor{currentfill}%
\pgfsetfillopacity{0.513365}%
\pgfsetlinewidth{1.003750pt}%
\definecolor{currentstroke}{rgb}{0.121569,0.466667,0.705882}%
\pgfsetstrokecolor{currentstroke}%
\pgfsetstrokeopacity{0.513365}%
\pgfsetdash{}{0pt}%
\pgfpathmoveto{\pgfqpoint{2.935592in}{2.004285in}}%
\pgfpathcurveto{\pgfqpoint{2.943829in}{2.004285in}}{\pgfqpoint{2.951729in}{2.007558in}}{\pgfqpoint{2.957553in}{2.013382in}}%
\pgfpathcurveto{\pgfqpoint{2.963376in}{2.019205in}}{\pgfqpoint{2.966649in}{2.027106in}}{\pgfqpoint{2.966649in}{2.035342in}}%
\pgfpathcurveto{\pgfqpoint{2.966649in}{2.043578in}}{\pgfqpoint{2.963376in}{2.051478in}}{\pgfqpoint{2.957553in}{2.057302in}}%
\pgfpathcurveto{\pgfqpoint{2.951729in}{2.063126in}}{\pgfqpoint{2.943829in}{2.066398in}}{\pgfqpoint{2.935592in}{2.066398in}}%
\pgfpathcurveto{\pgfqpoint{2.927356in}{2.066398in}}{\pgfqpoint{2.919456in}{2.063126in}}{\pgfqpoint{2.913632in}{2.057302in}}%
\pgfpathcurveto{\pgfqpoint{2.907808in}{2.051478in}}{\pgfqpoint{2.904536in}{2.043578in}}{\pgfqpoint{2.904536in}{2.035342in}}%
\pgfpathcurveto{\pgfqpoint{2.904536in}{2.027106in}}{\pgfqpoint{2.907808in}{2.019205in}}{\pgfqpoint{2.913632in}{2.013382in}}%
\pgfpathcurveto{\pgfqpoint{2.919456in}{2.007558in}}{\pgfqpoint{2.927356in}{2.004285in}}{\pgfqpoint{2.935592in}{2.004285in}}%
\pgfpathclose%
\pgfusepath{stroke,fill}%
\end{pgfscope}%
\begin{pgfscope}%
\pgfpathrectangle{\pgfqpoint{0.100000in}{0.212622in}}{\pgfqpoint{3.696000in}{3.696000in}}%
\pgfusepath{clip}%
\pgfsetbuttcap%
\pgfsetroundjoin%
\definecolor{currentfill}{rgb}{0.121569,0.466667,0.705882}%
\pgfsetfillcolor{currentfill}%
\pgfsetfillopacity{0.513639}%
\pgfsetlinewidth{1.003750pt}%
\definecolor{currentstroke}{rgb}{0.121569,0.466667,0.705882}%
\pgfsetstrokecolor{currentstroke}%
\pgfsetstrokeopacity{0.513639}%
\pgfsetdash{}{0pt}%
\pgfpathmoveto{\pgfqpoint{1.155851in}{1.732347in}}%
\pgfpathcurveto{\pgfqpoint{1.164087in}{1.732347in}}{\pgfqpoint{1.171987in}{1.735619in}}{\pgfqpoint{1.177811in}{1.741443in}}%
\pgfpathcurveto{\pgfqpoint{1.183635in}{1.747267in}}{\pgfqpoint{1.186907in}{1.755167in}}{\pgfqpoint{1.186907in}{1.763403in}}%
\pgfpathcurveto{\pgfqpoint{1.186907in}{1.771640in}}{\pgfqpoint{1.183635in}{1.779540in}}{\pgfqpoint{1.177811in}{1.785364in}}%
\pgfpathcurveto{\pgfqpoint{1.171987in}{1.791188in}}{\pgfqpoint{1.164087in}{1.794460in}}{\pgfqpoint{1.155851in}{1.794460in}}%
\pgfpathcurveto{\pgfqpoint{1.147614in}{1.794460in}}{\pgfqpoint{1.139714in}{1.791188in}}{\pgfqpoint{1.133890in}{1.785364in}}%
\pgfpathcurveto{\pgfqpoint{1.128066in}{1.779540in}}{\pgfqpoint{1.124794in}{1.771640in}}{\pgfqpoint{1.124794in}{1.763403in}}%
\pgfpathcurveto{\pgfqpoint{1.124794in}{1.755167in}}{\pgfqpoint{1.128066in}{1.747267in}}{\pgfqpoint{1.133890in}{1.741443in}}%
\pgfpathcurveto{\pgfqpoint{1.139714in}{1.735619in}}{\pgfqpoint{1.147614in}{1.732347in}}{\pgfqpoint{1.155851in}{1.732347in}}%
\pgfpathclose%
\pgfusepath{stroke,fill}%
\end{pgfscope}%
\begin{pgfscope}%
\pgfpathrectangle{\pgfqpoint{0.100000in}{0.212622in}}{\pgfqpoint{3.696000in}{3.696000in}}%
\pgfusepath{clip}%
\pgfsetbuttcap%
\pgfsetroundjoin%
\definecolor{currentfill}{rgb}{0.121569,0.466667,0.705882}%
\pgfsetfillcolor{currentfill}%
\pgfsetfillopacity{0.513808}%
\pgfsetlinewidth{1.003750pt}%
\definecolor{currentstroke}{rgb}{0.121569,0.466667,0.705882}%
\pgfsetstrokecolor{currentstroke}%
\pgfsetstrokeopacity{0.513808}%
\pgfsetdash{}{0pt}%
\pgfpathmoveto{\pgfqpoint{2.939491in}{2.002579in}}%
\pgfpathcurveto{\pgfqpoint{2.947727in}{2.002579in}}{\pgfqpoint{2.955627in}{2.005852in}}{\pgfqpoint{2.961451in}{2.011676in}}%
\pgfpathcurveto{\pgfqpoint{2.967275in}{2.017500in}}{\pgfqpoint{2.970548in}{2.025400in}}{\pgfqpoint{2.970548in}{2.033636in}}%
\pgfpathcurveto{\pgfqpoint{2.970548in}{2.041872in}}{\pgfqpoint{2.967275in}{2.049772in}}{\pgfqpoint{2.961451in}{2.055596in}}%
\pgfpathcurveto{\pgfqpoint{2.955627in}{2.061420in}}{\pgfqpoint{2.947727in}{2.064692in}}{\pgfqpoint{2.939491in}{2.064692in}}%
\pgfpathcurveto{\pgfqpoint{2.931255in}{2.064692in}}{\pgfqpoint{2.923355in}{2.061420in}}{\pgfqpoint{2.917531in}{2.055596in}}%
\pgfpathcurveto{\pgfqpoint{2.911707in}{2.049772in}}{\pgfqpoint{2.908435in}{2.041872in}}{\pgfqpoint{2.908435in}{2.033636in}}%
\pgfpathcurveto{\pgfqpoint{2.908435in}{2.025400in}}{\pgfqpoint{2.911707in}{2.017500in}}{\pgfqpoint{2.917531in}{2.011676in}}%
\pgfpathcurveto{\pgfqpoint{2.923355in}{2.005852in}}{\pgfqpoint{2.931255in}{2.002579in}}{\pgfqpoint{2.939491in}{2.002579in}}%
\pgfpathclose%
\pgfusepath{stroke,fill}%
\end{pgfscope}%
\begin{pgfscope}%
\pgfpathrectangle{\pgfqpoint{0.100000in}{0.212622in}}{\pgfqpoint{3.696000in}{3.696000in}}%
\pgfusepath{clip}%
\pgfsetbuttcap%
\pgfsetroundjoin%
\definecolor{currentfill}{rgb}{0.121569,0.466667,0.705882}%
\pgfsetfillcolor{currentfill}%
\pgfsetfillopacity{0.514708}%
\pgfsetlinewidth{1.003750pt}%
\definecolor{currentstroke}{rgb}{0.121569,0.466667,0.705882}%
\pgfsetstrokecolor{currentstroke}%
\pgfsetstrokeopacity{0.514708}%
\pgfsetdash{}{0pt}%
\pgfpathmoveto{\pgfqpoint{2.944141in}{2.003988in}}%
\pgfpathcurveto{\pgfqpoint{2.952377in}{2.003988in}}{\pgfqpoint{2.960277in}{2.007261in}}{\pgfqpoint{2.966101in}{2.013085in}}%
\pgfpathcurveto{\pgfqpoint{2.971925in}{2.018909in}}{\pgfqpoint{2.975197in}{2.026809in}}{\pgfqpoint{2.975197in}{2.035045in}}%
\pgfpathcurveto{\pgfqpoint{2.975197in}{2.043281in}}{\pgfqpoint{2.971925in}{2.051181in}}{\pgfqpoint{2.966101in}{2.057005in}}%
\pgfpathcurveto{\pgfqpoint{2.960277in}{2.062829in}}{\pgfqpoint{2.952377in}{2.066101in}}{\pgfqpoint{2.944141in}{2.066101in}}%
\pgfpathcurveto{\pgfqpoint{2.935904in}{2.066101in}}{\pgfqpoint{2.928004in}{2.062829in}}{\pgfqpoint{2.922180in}{2.057005in}}%
\pgfpathcurveto{\pgfqpoint{2.916356in}{2.051181in}}{\pgfqpoint{2.913084in}{2.043281in}}{\pgfqpoint{2.913084in}{2.035045in}}%
\pgfpathcurveto{\pgfqpoint{2.913084in}{2.026809in}}{\pgfqpoint{2.916356in}{2.018909in}}{\pgfqpoint{2.922180in}{2.013085in}}%
\pgfpathcurveto{\pgfqpoint{2.928004in}{2.007261in}}{\pgfqpoint{2.935904in}{2.003988in}}{\pgfqpoint{2.944141in}{2.003988in}}%
\pgfpathclose%
\pgfusepath{stroke,fill}%
\end{pgfscope}%
\begin{pgfscope}%
\pgfpathrectangle{\pgfqpoint{0.100000in}{0.212622in}}{\pgfqpoint{3.696000in}{3.696000in}}%
\pgfusepath{clip}%
\pgfsetbuttcap%
\pgfsetroundjoin%
\definecolor{currentfill}{rgb}{0.121569,0.466667,0.705882}%
\pgfsetfillcolor{currentfill}%
\pgfsetfillopacity{0.515077}%
\pgfsetlinewidth{1.003750pt}%
\definecolor{currentstroke}{rgb}{0.121569,0.466667,0.705882}%
\pgfsetstrokecolor{currentstroke}%
\pgfsetstrokeopacity{0.515077}%
\pgfsetdash{}{0pt}%
\pgfpathmoveto{\pgfqpoint{2.946563in}{2.003501in}}%
\pgfpathcurveto{\pgfqpoint{2.954799in}{2.003501in}}{\pgfqpoint{2.962699in}{2.006773in}}{\pgfqpoint{2.968523in}{2.012597in}}%
\pgfpathcurveto{\pgfqpoint{2.974347in}{2.018421in}}{\pgfqpoint{2.977620in}{2.026321in}}{\pgfqpoint{2.977620in}{2.034558in}}%
\pgfpathcurveto{\pgfqpoint{2.977620in}{2.042794in}}{\pgfqpoint{2.974347in}{2.050694in}}{\pgfqpoint{2.968523in}{2.056518in}}%
\pgfpathcurveto{\pgfqpoint{2.962699in}{2.062342in}}{\pgfqpoint{2.954799in}{2.065614in}}{\pgfqpoint{2.946563in}{2.065614in}}%
\pgfpathcurveto{\pgfqpoint{2.938327in}{2.065614in}}{\pgfqpoint{2.930427in}{2.062342in}}{\pgfqpoint{2.924603in}{2.056518in}}%
\pgfpathcurveto{\pgfqpoint{2.918779in}{2.050694in}}{\pgfqpoint{2.915507in}{2.042794in}}{\pgfqpoint{2.915507in}{2.034558in}}%
\pgfpathcurveto{\pgfqpoint{2.915507in}{2.026321in}}{\pgfqpoint{2.918779in}{2.018421in}}{\pgfqpoint{2.924603in}{2.012597in}}%
\pgfpathcurveto{\pgfqpoint{2.930427in}{2.006773in}}{\pgfqpoint{2.938327in}{2.003501in}}{\pgfqpoint{2.946563in}{2.003501in}}%
\pgfpathclose%
\pgfusepath{stroke,fill}%
\end{pgfscope}%
\begin{pgfscope}%
\pgfpathrectangle{\pgfqpoint{0.100000in}{0.212622in}}{\pgfqpoint{3.696000in}{3.696000in}}%
\pgfusepath{clip}%
\pgfsetbuttcap%
\pgfsetroundjoin%
\definecolor{currentfill}{rgb}{0.121569,0.466667,0.705882}%
\pgfsetfillcolor{currentfill}%
\pgfsetfillopacity{0.515107}%
\pgfsetlinewidth{1.003750pt}%
\definecolor{currentstroke}{rgb}{0.121569,0.466667,0.705882}%
\pgfsetstrokecolor{currentstroke}%
\pgfsetstrokeopacity{0.515107}%
\pgfsetdash{}{0pt}%
\pgfpathmoveto{\pgfqpoint{1.153906in}{1.732078in}}%
\pgfpathcurveto{\pgfqpoint{1.162142in}{1.732078in}}{\pgfqpoint{1.170042in}{1.735350in}}{\pgfqpoint{1.175866in}{1.741174in}}%
\pgfpathcurveto{\pgfqpoint{1.181690in}{1.746998in}}{\pgfqpoint{1.184962in}{1.754898in}}{\pgfqpoint{1.184962in}{1.763134in}}%
\pgfpathcurveto{\pgfqpoint{1.184962in}{1.771371in}}{\pgfqpoint{1.181690in}{1.779271in}}{\pgfqpoint{1.175866in}{1.785095in}}%
\pgfpathcurveto{\pgfqpoint{1.170042in}{1.790919in}}{\pgfqpoint{1.162142in}{1.794191in}}{\pgfqpoint{1.153906in}{1.794191in}}%
\pgfpathcurveto{\pgfqpoint{1.145670in}{1.794191in}}{\pgfqpoint{1.137770in}{1.790919in}}{\pgfqpoint{1.131946in}{1.785095in}}%
\pgfpathcurveto{\pgfqpoint{1.126122in}{1.779271in}}{\pgfqpoint{1.122849in}{1.771371in}}{\pgfqpoint{1.122849in}{1.763134in}}%
\pgfpathcurveto{\pgfqpoint{1.122849in}{1.754898in}}{\pgfqpoint{1.126122in}{1.746998in}}{\pgfqpoint{1.131946in}{1.741174in}}%
\pgfpathcurveto{\pgfqpoint{1.137770in}{1.735350in}}{\pgfqpoint{1.145670in}{1.732078in}}{\pgfqpoint{1.153906in}{1.732078in}}%
\pgfpathclose%
\pgfusepath{stroke,fill}%
\end{pgfscope}%
\begin{pgfscope}%
\pgfpathrectangle{\pgfqpoint{0.100000in}{0.212622in}}{\pgfqpoint{3.696000in}{3.696000in}}%
\pgfusepath{clip}%
\pgfsetbuttcap%
\pgfsetroundjoin%
\definecolor{currentfill}{rgb}{0.121569,0.466667,0.705882}%
\pgfsetfillcolor{currentfill}%
\pgfsetfillopacity{0.515616}%
\pgfsetlinewidth{1.003750pt}%
\definecolor{currentstroke}{rgb}{0.121569,0.466667,0.705882}%
\pgfsetstrokecolor{currentstroke}%
\pgfsetstrokeopacity{0.515616}%
\pgfsetdash{}{0pt}%
\pgfpathmoveto{\pgfqpoint{2.950008in}{2.003233in}}%
\pgfpathcurveto{\pgfqpoint{2.958244in}{2.003233in}}{\pgfqpoint{2.966144in}{2.006505in}}{\pgfqpoint{2.971968in}{2.012329in}}%
\pgfpathcurveto{\pgfqpoint{2.977792in}{2.018153in}}{\pgfqpoint{2.981065in}{2.026053in}}{\pgfqpoint{2.981065in}{2.034289in}}%
\pgfpathcurveto{\pgfqpoint{2.981065in}{2.042525in}}{\pgfqpoint{2.977792in}{2.050425in}}{\pgfqpoint{2.971968in}{2.056249in}}%
\pgfpathcurveto{\pgfqpoint{2.966144in}{2.062073in}}{\pgfqpoint{2.958244in}{2.065346in}}{\pgfqpoint{2.950008in}{2.065346in}}%
\pgfpathcurveto{\pgfqpoint{2.941772in}{2.065346in}}{\pgfqpoint{2.933872in}{2.062073in}}{\pgfqpoint{2.928048in}{2.056249in}}%
\pgfpathcurveto{\pgfqpoint{2.922224in}{2.050425in}}{\pgfqpoint{2.918952in}{2.042525in}}{\pgfqpoint{2.918952in}{2.034289in}}%
\pgfpathcurveto{\pgfqpoint{2.918952in}{2.026053in}}{\pgfqpoint{2.922224in}{2.018153in}}{\pgfqpoint{2.928048in}{2.012329in}}%
\pgfpathcurveto{\pgfqpoint{2.933872in}{2.006505in}}{\pgfqpoint{2.941772in}{2.003233in}}{\pgfqpoint{2.950008in}{2.003233in}}%
\pgfpathclose%
\pgfusepath{stroke,fill}%
\end{pgfscope}%
\begin{pgfscope}%
\pgfpathrectangle{\pgfqpoint{0.100000in}{0.212622in}}{\pgfqpoint{3.696000in}{3.696000in}}%
\pgfusepath{clip}%
\pgfsetbuttcap%
\pgfsetroundjoin%
\definecolor{currentfill}{rgb}{0.121569,0.466667,0.705882}%
\pgfsetfillcolor{currentfill}%
\pgfsetfillopacity{0.516049}%
\pgfsetlinewidth{1.003750pt}%
\definecolor{currentstroke}{rgb}{0.121569,0.466667,0.705882}%
\pgfsetstrokecolor{currentstroke}%
\pgfsetstrokeopacity{0.516049}%
\pgfsetdash{}{0pt}%
\pgfpathmoveto{\pgfqpoint{1.150369in}{1.731946in}}%
\pgfpathcurveto{\pgfqpoint{1.158605in}{1.731946in}}{\pgfqpoint{1.166505in}{1.735219in}}{\pgfqpoint{1.172329in}{1.741043in}}%
\pgfpathcurveto{\pgfqpoint{1.178153in}{1.746867in}}{\pgfqpoint{1.181425in}{1.754767in}}{\pgfqpoint{1.181425in}{1.763003in}}%
\pgfpathcurveto{\pgfqpoint{1.181425in}{1.771239in}}{\pgfqpoint{1.178153in}{1.779139in}}{\pgfqpoint{1.172329in}{1.784963in}}%
\pgfpathcurveto{\pgfqpoint{1.166505in}{1.790787in}}{\pgfqpoint{1.158605in}{1.794059in}}{\pgfqpoint{1.150369in}{1.794059in}}%
\pgfpathcurveto{\pgfqpoint{1.142133in}{1.794059in}}{\pgfqpoint{1.134232in}{1.790787in}}{\pgfqpoint{1.128409in}{1.784963in}}%
\pgfpathcurveto{\pgfqpoint{1.122585in}{1.779139in}}{\pgfqpoint{1.119312in}{1.771239in}}{\pgfqpoint{1.119312in}{1.763003in}}%
\pgfpathcurveto{\pgfqpoint{1.119312in}{1.754767in}}{\pgfqpoint{1.122585in}{1.746867in}}{\pgfqpoint{1.128409in}{1.741043in}}%
\pgfpathcurveto{\pgfqpoint{1.134232in}{1.735219in}}{\pgfqpoint{1.142133in}{1.731946in}}{\pgfqpoint{1.150369in}{1.731946in}}%
\pgfpathclose%
\pgfusepath{stroke,fill}%
\end{pgfscope}%
\begin{pgfscope}%
\pgfpathrectangle{\pgfqpoint{0.100000in}{0.212622in}}{\pgfqpoint{3.696000in}{3.696000in}}%
\pgfusepath{clip}%
\pgfsetbuttcap%
\pgfsetroundjoin%
\definecolor{currentfill}{rgb}{0.121569,0.466667,0.705882}%
\pgfsetfillcolor{currentfill}%
\pgfsetfillopacity{0.516292}%
\pgfsetlinewidth{1.003750pt}%
\definecolor{currentstroke}{rgb}{0.121569,0.466667,0.705882}%
\pgfsetstrokecolor{currentstroke}%
\pgfsetstrokeopacity{0.516292}%
\pgfsetdash{}{0pt}%
\pgfpathmoveto{\pgfqpoint{2.955202in}{2.002317in}}%
\pgfpathcurveto{\pgfqpoint{2.963438in}{2.002317in}}{\pgfqpoint{2.971338in}{2.005589in}}{\pgfqpoint{2.977162in}{2.011413in}}%
\pgfpathcurveto{\pgfqpoint{2.982986in}{2.017237in}}{\pgfqpoint{2.986258in}{2.025137in}}{\pgfqpoint{2.986258in}{2.033373in}}%
\pgfpathcurveto{\pgfqpoint{2.986258in}{2.041610in}}{\pgfqpoint{2.982986in}{2.049510in}}{\pgfqpoint{2.977162in}{2.055334in}}%
\pgfpathcurveto{\pgfqpoint{2.971338in}{2.061157in}}{\pgfqpoint{2.963438in}{2.064430in}}{\pgfqpoint{2.955202in}{2.064430in}}%
\pgfpathcurveto{\pgfqpoint{2.946966in}{2.064430in}}{\pgfqpoint{2.939066in}{2.061157in}}{\pgfqpoint{2.933242in}{2.055334in}}%
\pgfpathcurveto{\pgfqpoint{2.927418in}{2.049510in}}{\pgfqpoint{2.924145in}{2.041610in}}{\pgfqpoint{2.924145in}{2.033373in}}%
\pgfpathcurveto{\pgfqpoint{2.924145in}{2.025137in}}{\pgfqpoint{2.927418in}{2.017237in}}{\pgfqpoint{2.933242in}{2.011413in}}%
\pgfpathcurveto{\pgfqpoint{2.939066in}{2.005589in}}{\pgfqpoint{2.946966in}{2.002317in}}{\pgfqpoint{2.955202in}{2.002317in}}%
\pgfpathclose%
\pgfusepath{stroke,fill}%
\end{pgfscope}%
\begin{pgfscope}%
\pgfpathrectangle{\pgfqpoint{0.100000in}{0.212622in}}{\pgfqpoint{3.696000in}{3.696000in}}%
\pgfusepath{clip}%
\pgfsetbuttcap%
\pgfsetroundjoin%
\definecolor{currentfill}{rgb}{0.121569,0.466667,0.705882}%
\pgfsetfillcolor{currentfill}%
\pgfsetfillopacity{0.517036}%
\pgfsetlinewidth{1.003750pt}%
\definecolor{currentstroke}{rgb}{0.121569,0.466667,0.705882}%
\pgfsetstrokecolor{currentstroke}%
\pgfsetstrokeopacity{0.517036}%
\pgfsetdash{}{0pt}%
\pgfpathmoveto{\pgfqpoint{1.149789in}{1.733655in}}%
\pgfpathcurveto{\pgfqpoint{1.158025in}{1.733655in}}{\pgfqpoint{1.165925in}{1.736928in}}{\pgfqpoint{1.171749in}{1.742752in}}%
\pgfpathcurveto{\pgfqpoint{1.177573in}{1.748575in}}{\pgfqpoint{1.180845in}{1.756476in}}{\pgfqpoint{1.180845in}{1.764712in}}%
\pgfpathcurveto{\pgfqpoint{1.180845in}{1.772948in}}{\pgfqpoint{1.177573in}{1.780848in}}{\pgfqpoint{1.171749in}{1.786672in}}%
\pgfpathcurveto{\pgfqpoint{1.165925in}{1.792496in}}{\pgfqpoint{1.158025in}{1.795768in}}{\pgfqpoint{1.149789in}{1.795768in}}%
\pgfpathcurveto{\pgfqpoint{1.141552in}{1.795768in}}{\pgfqpoint{1.133652in}{1.792496in}}{\pgfqpoint{1.127828in}{1.786672in}}%
\pgfpathcurveto{\pgfqpoint{1.122004in}{1.780848in}}{\pgfqpoint{1.118732in}{1.772948in}}{\pgfqpoint{1.118732in}{1.764712in}}%
\pgfpathcurveto{\pgfqpoint{1.118732in}{1.756476in}}{\pgfqpoint{1.122004in}{1.748575in}}{\pgfqpoint{1.127828in}{1.742752in}}%
\pgfpathcurveto{\pgfqpoint{1.133652in}{1.736928in}}{\pgfqpoint{1.141552in}{1.733655in}}{\pgfqpoint{1.149789in}{1.733655in}}%
\pgfpathclose%
\pgfusepath{stroke,fill}%
\end{pgfscope}%
\begin{pgfscope}%
\pgfpathrectangle{\pgfqpoint{0.100000in}{0.212622in}}{\pgfqpoint{3.696000in}{3.696000in}}%
\pgfusepath{clip}%
\pgfsetbuttcap%
\pgfsetroundjoin%
\definecolor{currentfill}{rgb}{0.121569,0.466667,0.705882}%
\pgfsetfillcolor{currentfill}%
\pgfsetfillopacity{0.517207}%
\pgfsetlinewidth{1.003750pt}%
\definecolor{currentstroke}{rgb}{0.121569,0.466667,0.705882}%
\pgfsetstrokecolor{currentstroke}%
\pgfsetstrokeopacity{0.517207}%
\pgfsetdash{}{0pt}%
\pgfpathmoveto{\pgfqpoint{2.961183in}{2.002742in}}%
\pgfpathcurveto{\pgfqpoint{2.969419in}{2.002742in}}{\pgfqpoint{2.977320in}{2.006015in}}{\pgfqpoint{2.983143in}{2.011839in}}%
\pgfpathcurveto{\pgfqpoint{2.988967in}{2.017663in}}{\pgfqpoint{2.992240in}{2.025563in}}{\pgfqpoint{2.992240in}{2.033799in}}%
\pgfpathcurveto{\pgfqpoint{2.992240in}{2.042035in}}{\pgfqpoint{2.988967in}{2.049935in}}{\pgfqpoint{2.983143in}{2.055759in}}%
\pgfpathcurveto{\pgfqpoint{2.977320in}{2.061583in}}{\pgfqpoint{2.969419in}{2.064855in}}{\pgfqpoint{2.961183in}{2.064855in}}%
\pgfpathcurveto{\pgfqpoint{2.952947in}{2.064855in}}{\pgfqpoint{2.945047in}{2.061583in}}{\pgfqpoint{2.939223in}{2.055759in}}%
\pgfpathcurveto{\pgfqpoint{2.933399in}{2.049935in}}{\pgfqpoint{2.930127in}{2.042035in}}{\pgfqpoint{2.930127in}{2.033799in}}%
\pgfpathcurveto{\pgfqpoint{2.930127in}{2.025563in}}{\pgfqpoint{2.933399in}{2.017663in}}{\pgfqpoint{2.939223in}{2.011839in}}%
\pgfpathcurveto{\pgfqpoint{2.945047in}{2.006015in}}{\pgfqpoint{2.952947in}{2.002742in}}{\pgfqpoint{2.961183in}{2.002742in}}%
\pgfpathclose%
\pgfusepath{stroke,fill}%
\end{pgfscope}%
\begin{pgfscope}%
\pgfpathrectangle{\pgfqpoint{0.100000in}{0.212622in}}{\pgfqpoint{3.696000in}{3.696000in}}%
\pgfusepath{clip}%
\pgfsetbuttcap%
\pgfsetroundjoin%
\definecolor{currentfill}{rgb}{0.121569,0.466667,0.705882}%
\pgfsetfillcolor{currentfill}%
\pgfsetfillopacity{0.517498}%
\pgfsetlinewidth{1.003750pt}%
\definecolor{currentstroke}{rgb}{0.121569,0.466667,0.705882}%
\pgfsetstrokecolor{currentstroke}%
\pgfsetstrokeopacity{0.517498}%
\pgfsetdash{}{0pt}%
\pgfpathmoveto{\pgfqpoint{1.146998in}{1.728706in}}%
\pgfpathcurveto{\pgfqpoint{1.155234in}{1.728706in}}{\pgfqpoint{1.163134in}{1.731979in}}{\pgfqpoint{1.168958in}{1.737802in}}%
\pgfpathcurveto{\pgfqpoint{1.174782in}{1.743626in}}{\pgfqpoint{1.178055in}{1.751526in}}{\pgfqpoint{1.178055in}{1.759763in}}%
\pgfpathcurveto{\pgfqpoint{1.178055in}{1.767999in}}{\pgfqpoint{1.174782in}{1.775899in}}{\pgfqpoint{1.168958in}{1.781723in}}%
\pgfpathcurveto{\pgfqpoint{1.163134in}{1.787547in}}{\pgfqpoint{1.155234in}{1.790819in}}{\pgfqpoint{1.146998in}{1.790819in}}%
\pgfpathcurveto{\pgfqpoint{1.138762in}{1.790819in}}{\pgfqpoint{1.130862in}{1.787547in}}{\pgfqpoint{1.125038in}{1.781723in}}%
\pgfpathcurveto{\pgfqpoint{1.119214in}{1.775899in}}{\pgfqpoint{1.115942in}{1.767999in}}{\pgfqpoint{1.115942in}{1.759763in}}%
\pgfpathcurveto{\pgfqpoint{1.115942in}{1.751526in}}{\pgfqpoint{1.119214in}{1.743626in}}{\pgfqpoint{1.125038in}{1.737802in}}%
\pgfpathcurveto{\pgfqpoint{1.130862in}{1.731979in}}{\pgfqpoint{1.138762in}{1.728706in}}{\pgfqpoint{1.146998in}{1.728706in}}%
\pgfpathclose%
\pgfusepath{stroke,fill}%
\end{pgfscope}%
\begin{pgfscope}%
\pgfpathrectangle{\pgfqpoint{0.100000in}{0.212622in}}{\pgfqpoint{3.696000in}{3.696000in}}%
\pgfusepath{clip}%
\pgfsetbuttcap%
\pgfsetroundjoin%
\definecolor{currentfill}{rgb}{0.121569,0.466667,0.705882}%
\pgfsetfillcolor{currentfill}%
\pgfsetfillopacity{0.517914}%
\pgfsetlinewidth{1.003750pt}%
\definecolor{currentstroke}{rgb}{0.121569,0.466667,0.705882}%
\pgfsetstrokecolor{currentstroke}%
\pgfsetstrokeopacity{0.517914}%
\pgfsetdash{}{0pt}%
\pgfpathmoveto{\pgfqpoint{2.967292in}{2.000698in}}%
\pgfpathcurveto{\pgfqpoint{2.975529in}{2.000698in}}{\pgfqpoint{2.983429in}{2.003971in}}{\pgfqpoint{2.989253in}{2.009795in}}%
\pgfpathcurveto{\pgfqpoint{2.995077in}{2.015619in}}{\pgfqpoint{2.998349in}{2.023519in}}{\pgfqpoint{2.998349in}{2.031755in}}%
\pgfpathcurveto{\pgfqpoint{2.998349in}{2.039991in}}{\pgfqpoint{2.995077in}{2.047891in}}{\pgfqpoint{2.989253in}{2.053715in}}%
\pgfpathcurveto{\pgfqpoint{2.983429in}{2.059539in}}{\pgfqpoint{2.975529in}{2.062811in}}{\pgfqpoint{2.967292in}{2.062811in}}%
\pgfpathcurveto{\pgfqpoint{2.959056in}{2.062811in}}{\pgfqpoint{2.951156in}{2.059539in}}{\pgfqpoint{2.945332in}{2.053715in}}%
\pgfpathcurveto{\pgfqpoint{2.939508in}{2.047891in}}{\pgfqpoint{2.936236in}{2.039991in}}{\pgfqpoint{2.936236in}{2.031755in}}%
\pgfpathcurveto{\pgfqpoint{2.936236in}{2.023519in}}{\pgfqpoint{2.939508in}{2.015619in}}{\pgfqpoint{2.945332in}{2.009795in}}%
\pgfpathcurveto{\pgfqpoint{2.951156in}{2.003971in}}{\pgfqpoint{2.959056in}{2.000698in}}{\pgfqpoint{2.967292in}{2.000698in}}%
\pgfpathclose%
\pgfusepath{stroke,fill}%
\end{pgfscope}%
\begin{pgfscope}%
\pgfpathrectangle{\pgfqpoint{0.100000in}{0.212622in}}{\pgfqpoint{3.696000in}{3.696000in}}%
\pgfusepath{clip}%
\pgfsetbuttcap%
\pgfsetroundjoin%
\definecolor{currentfill}{rgb}{0.121569,0.466667,0.705882}%
\pgfsetfillcolor{currentfill}%
\pgfsetfillopacity{0.518186}%
\pgfsetlinewidth{1.003750pt}%
\definecolor{currentstroke}{rgb}{0.121569,0.466667,0.705882}%
\pgfsetstrokecolor{currentstroke}%
\pgfsetstrokeopacity{0.518186}%
\pgfsetdash{}{0pt}%
\pgfpathmoveto{\pgfqpoint{1.144260in}{1.727337in}}%
\pgfpathcurveto{\pgfqpoint{1.152496in}{1.727337in}}{\pgfqpoint{1.160396in}{1.730609in}}{\pgfqpoint{1.166220in}{1.736433in}}%
\pgfpathcurveto{\pgfqpoint{1.172044in}{1.742257in}}{\pgfqpoint{1.175317in}{1.750157in}}{\pgfqpoint{1.175317in}{1.758393in}}%
\pgfpathcurveto{\pgfqpoint{1.175317in}{1.766630in}}{\pgfqpoint{1.172044in}{1.774530in}}{\pgfqpoint{1.166220in}{1.780354in}}%
\pgfpathcurveto{\pgfqpoint{1.160396in}{1.786178in}}{\pgfqpoint{1.152496in}{1.789450in}}{\pgfqpoint{1.144260in}{1.789450in}}%
\pgfpathcurveto{\pgfqpoint{1.136024in}{1.789450in}}{\pgfqpoint{1.128124in}{1.786178in}}{\pgfqpoint{1.122300in}{1.780354in}}%
\pgfpathcurveto{\pgfqpoint{1.116476in}{1.774530in}}{\pgfqpoint{1.113204in}{1.766630in}}{\pgfqpoint{1.113204in}{1.758393in}}%
\pgfpathcurveto{\pgfqpoint{1.113204in}{1.750157in}}{\pgfqpoint{1.116476in}{1.742257in}}{\pgfqpoint{1.122300in}{1.736433in}}%
\pgfpathcurveto{\pgfqpoint{1.128124in}{1.730609in}}{\pgfqpoint{1.136024in}{1.727337in}}{\pgfqpoint{1.144260in}{1.727337in}}%
\pgfpathclose%
\pgfusepath{stroke,fill}%
\end{pgfscope}%
\begin{pgfscope}%
\pgfpathrectangle{\pgfqpoint{0.100000in}{0.212622in}}{\pgfqpoint{3.696000in}{3.696000in}}%
\pgfusepath{clip}%
\pgfsetbuttcap%
\pgfsetroundjoin%
\definecolor{currentfill}{rgb}{0.121569,0.466667,0.705882}%
\pgfsetfillcolor{currentfill}%
\pgfsetfillopacity{0.519000}%
\pgfsetlinewidth{1.003750pt}%
\definecolor{currentstroke}{rgb}{0.121569,0.466667,0.705882}%
\pgfsetstrokecolor{currentstroke}%
\pgfsetstrokeopacity{0.519000}%
\pgfsetdash{}{0pt}%
\pgfpathmoveto{\pgfqpoint{2.974161in}{2.002574in}}%
\pgfpathcurveto{\pgfqpoint{2.982398in}{2.002574in}}{\pgfqpoint{2.990298in}{2.005847in}}{\pgfqpoint{2.996122in}{2.011671in}}%
\pgfpathcurveto{\pgfqpoint{3.001946in}{2.017494in}}{\pgfqpoint{3.005218in}{2.025394in}}{\pgfqpoint{3.005218in}{2.033631in}}%
\pgfpathcurveto{\pgfqpoint{3.005218in}{2.041867in}}{\pgfqpoint{3.001946in}{2.049767in}}{\pgfqpoint{2.996122in}{2.055591in}}%
\pgfpathcurveto{\pgfqpoint{2.990298in}{2.061415in}}{\pgfqpoint{2.982398in}{2.064687in}}{\pgfqpoint{2.974161in}{2.064687in}}%
\pgfpathcurveto{\pgfqpoint{2.965925in}{2.064687in}}{\pgfqpoint{2.958025in}{2.061415in}}{\pgfqpoint{2.952201in}{2.055591in}}%
\pgfpathcurveto{\pgfqpoint{2.946377in}{2.049767in}}{\pgfqpoint{2.943105in}{2.041867in}}{\pgfqpoint{2.943105in}{2.033631in}}%
\pgfpathcurveto{\pgfqpoint{2.943105in}{2.025394in}}{\pgfqpoint{2.946377in}{2.017494in}}{\pgfqpoint{2.952201in}{2.011671in}}%
\pgfpathcurveto{\pgfqpoint{2.958025in}{2.005847in}}{\pgfqpoint{2.965925in}{2.002574in}}{\pgfqpoint{2.974161in}{2.002574in}}%
\pgfpathclose%
\pgfusepath{stroke,fill}%
\end{pgfscope}%
\begin{pgfscope}%
\pgfpathrectangle{\pgfqpoint{0.100000in}{0.212622in}}{\pgfqpoint{3.696000in}{3.696000in}}%
\pgfusepath{clip}%
\pgfsetbuttcap%
\pgfsetroundjoin%
\definecolor{currentfill}{rgb}{0.121569,0.466667,0.705882}%
\pgfsetfillcolor{currentfill}%
\pgfsetfillopacity{0.519376}%
\pgfsetlinewidth{1.003750pt}%
\definecolor{currentstroke}{rgb}{0.121569,0.466667,0.705882}%
\pgfsetstrokecolor{currentstroke}%
\pgfsetstrokeopacity{0.519376}%
\pgfsetdash{}{0pt}%
\pgfpathmoveto{\pgfqpoint{2.977650in}{2.001088in}}%
\pgfpathcurveto{\pgfqpoint{2.985887in}{2.001088in}}{\pgfqpoint{2.993787in}{2.004361in}}{\pgfqpoint{2.999611in}{2.010185in}}%
\pgfpathcurveto{\pgfqpoint{3.005435in}{2.016009in}}{\pgfqpoint{3.008707in}{2.023909in}}{\pgfqpoint{3.008707in}{2.032145in}}%
\pgfpathcurveto{\pgfqpoint{3.008707in}{2.040381in}}{\pgfqpoint{3.005435in}{2.048281in}}{\pgfqpoint{2.999611in}{2.054105in}}%
\pgfpathcurveto{\pgfqpoint{2.993787in}{2.059929in}}{\pgfqpoint{2.985887in}{2.063201in}}{\pgfqpoint{2.977650in}{2.063201in}}%
\pgfpathcurveto{\pgfqpoint{2.969414in}{2.063201in}}{\pgfqpoint{2.961514in}{2.059929in}}{\pgfqpoint{2.955690in}{2.054105in}}%
\pgfpathcurveto{\pgfqpoint{2.949866in}{2.048281in}}{\pgfqpoint{2.946594in}{2.040381in}}{\pgfqpoint{2.946594in}{2.032145in}}%
\pgfpathcurveto{\pgfqpoint{2.946594in}{2.023909in}}{\pgfqpoint{2.949866in}{2.016009in}}{\pgfqpoint{2.955690in}{2.010185in}}%
\pgfpathcurveto{\pgfqpoint{2.961514in}{2.004361in}}{\pgfqpoint{2.969414in}{2.001088in}}{\pgfqpoint{2.977650in}{2.001088in}}%
\pgfpathclose%
\pgfusepath{stroke,fill}%
\end{pgfscope}%
\begin{pgfscope}%
\pgfpathrectangle{\pgfqpoint{0.100000in}{0.212622in}}{\pgfqpoint{3.696000in}{3.696000in}}%
\pgfusepath{clip}%
\pgfsetbuttcap%
\pgfsetroundjoin%
\definecolor{currentfill}{rgb}{0.121569,0.466667,0.705882}%
\pgfsetfillcolor{currentfill}%
\pgfsetfillopacity{0.519957}%
\pgfsetlinewidth{1.003750pt}%
\definecolor{currentstroke}{rgb}{0.121569,0.466667,0.705882}%
\pgfsetstrokecolor{currentstroke}%
\pgfsetstrokeopacity{0.519957}%
\pgfsetdash{}{0pt}%
\pgfpathmoveto{\pgfqpoint{1.141017in}{1.726310in}}%
\pgfpathcurveto{\pgfqpoint{1.149254in}{1.726310in}}{\pgfqpoint{1.157154in}{1.729583in}}{\pgfqpoint{1.162978in}{1.735407in}}%
\pgfpathcurveto{\pgfqpoint{1.168801in}{1.741231in}}{\pgfqpoint{1.172074in}{1.749131in}}{\pgfqpoint{1.172074in}{1.757367in}}%
\pgfpathcurveto{\pgfqpoint{1.172074in}{1.765603in}}{\pgfqpoint{1.168801in}{1.773503in}}{\pgfqpoint{1.162978in}{1.779327in}}%
\pgfpathcurveto{\pgfqpoint{1.157154in}{1.785151in}}{\pgfqpoint{1.149254in}{1.788423in}}{\pgfqpoint{1.141017in}{1.788423in}}%
\pgfpathcurveto{\pgfqpoint{1.132781in}{1.788423in}}{\pgfqpoint{1.124881in}{1.785151in}}{\pgfqpoint{1.119057in}{1.779327in}}%
\pgfpathcurveto{\pgfqpoint{1.113233in}{1.773503in}}{\pgfqpoint{1.109961in}{1.765603in}}{\pgfqpoint{1.109961in}{1.757367in}}%
\pgfpathcurveto{\pgfqpoint{1.109961in}{1.749131in}}{\pgfqpoint{1.113233in}{1.741231in}}{\pgfqpoint{1.119057in}{1.735407in}}%
\pgfpathcurveto{\pgfqpoint{1.124881in}{1.729583in}}{\pgfqpoint{1.132781in}{1.726310in}}{\pgfqpoint{1.141017in}{1.726310in}}%
\pgfpathclose%
\pgfusepath{stroke,fill}%
\end{pgfscope}%
\begin{pgfscope}%
\pgfpathrectangle{\pgfqpoint{0.100000in}{0.212622in}}{\pgfqpoint{3.696000in}{3.696000in}}%
\pgfusepath{clip}%
\pgfsetbuttcap%
\pgfsetroundjoin%
\definecolor{currentfill}{rgb}{0.121569,0.466667,0.705882}%
\pgfsetfillcolor{currentfill}%
\pgfsetfillopacity{0.520392}%
\pgfsetlinewidth{1.003750pt}%
\definecolor{currentstroke}{rgb}{0.121569,0.466667,0.705882}%
\pgfsetstrokecolor{currentstroke}%
\pgfsetstrokeopacity{0.520392}%
\pgfsetdash{}{0pt}%
\pgfpathmoveto{\pgfqpoint{2.982787in}{2.002795in}}%
\pgfpathcurveto{\pgfqpoint{2.991023in}{2.002795in}}{\pgfqpoint{2.998923in}{2.006067in}}{\pgfqpoint{3.004747in}{2.011891in}}%
\pgfpathcurveto{\pgfqpoint{3.010571in}{2.017715in}}{\pgfqpoint{3.013844in}{2.025615in}}{\pgfqpoint{3.013844in}{2.033852in}}%
\pgfpathcurveto{\pgfqpoint{3.013844in}{2.042088in}}{\pgfqpoint{3.010571in}{2.049988in}}{\pgfqpoint{3.004747in}{2.055812in}}%
\pgfpathcurveto{\pgfqpoint{2.998923in}{2.061636in}}{\pgfqpoint{2.991023in}{2.064908in}}{\pgfqpoint{2.982787in}{2.064908in}}%
\pgfpathcurveto{\pgfqpoint{2.974551in}{2.064908in}}{\pgfqpoint{2.966651in}{2.061636in}}{\pgfqpoint{2.960827in}{2.055812in}}%
\pgfpathcurveto{\pgfqpoint{2.955003in}{2.049988in}}{\pgfqpoint{2.951731in}{2.042088in}}{\pgfqpoint{2.951731in}{2.033852in}}%
\pgfpathcurveto{\pgfqpoint{2.951731in}{2.025615in}}{\pgfqpoint{2.955003in}{2.017715in}}{\pgfqpoint{2.960827in}{2.011891in}}%
\pgfpathcurveto{\pgfqpoint{2.966651in}{2.006067in}}{\pgfqpoint{2.974551in}{2.002795in}}{\pgfqpoint{2.982787in}{2.002795in}}%
\pgfpathclose%
\pgfusepath{stroke,fill}%
\end{pgfscope}%
\begin{pgfscope}%
\pgfpathrectangle{\pgfqpoint{0.100000in}{0.212622in}}{\pgfqpoint{3.696000in}{3.696000in}}%
\pgfusepath{clip}%
\pgfsetbuttcap%
\pgfsetroundjoin%
\definecolor{currentfill}{rgb}{0.121569,0.466667,0.705882}%
\pgfsetfillcolor{currentfill}%
\pgfsetfillopacity{0.520909}%
\pgfsetlinewidth{1.003750pt}%
\definecolor{currentstroke}{rgb}{0.121569,0.466667,0.705882}%
\pgfsetstrokecolor{currentstroke}%
\pgfsetstrokeopacity{0.520909}%
\pgfsetdash{}{0pt}%
\pgfpathmoveto{\pgfqpoint{1.136445in}{1.725092in}}%
\pgfpathcurveto{\pgfqpoint{1.144681in}{1.725092in}}{\pgfqpoint{1.152582in}{1.728364in}}{\pgfqpoint{1.158405in}{1.734188in}}%
\pgfpathcurveto{\pgfqpoint{1.164229in}{1.740012in}}{\pgfqpoint{1.167502in}{1.747912in}}{\pgfqpoint{1.167502in}{1.756149in}}%
\pgfpathcurveto{\pgfqpoint{1.167502in}{1.764385in}}{\pgfqpoint{1.164229in}{1.772285in}}{\pgfqpoint{1.158405in}{1.778109in}}%
\pgfpathcurveto{\pgfqpoint{1.152582in}{1.783933in}}{\pgfqpoint{1.144681in}{1.787205in}}{\pgfqpoint{1.136445in}{1.787205in}}%
\pgfpathcurveto{\pgfqpoint{1.128209in}{1.787205in}}{\pgfqpoint{1.120309in}{1.783933in}}{\pgfqpoint{1.114485in}{1.778109in}}%
\pgfpathcurveto{\pgfqpoint{1.108661in}{1.772285in}}{\pgfqpoint{1.105389in}{1.764385in}}{\pgfqpoint{1.105389in}{1.756149in}}%
\pgfpathcurveto{\pgfqpoint{1.105389in}{1.747912in}}{\pgfqpoint{1.108661in}{1.740012in}}{\pgfqpoint{1.114485in}{1.734188in}}%
\pgfpathcurveto{\pgfqpoint{1.120309in}{1.728364in}}{\pgfqpoint{1.128209in}{1.725092in}}{\pgfqpoint{1.136445in}{1.725092in}}%
\pgfpathclose%
\pgfusepath{stroke,fill}%
\end{pgfscope}%
\begin{pgfscope}%
\pgfpathrectangle{\pgfqpoint{0.100000in}{0.212622in}}{\pgfqpoint{3.696000in}{3.696000in}}%
\pgfusepath{clip}%
\pgfsetbuttcap%
\pgfsetroundjoin%
\definecolor{currentfill}{rgb}{0.121569,0.466667,0.705882}%
\pgfsetfillcolor{currentfill}%
\pgfsetfillopacity{0.521182}%
\pgfsetlinewidth{1.003750pt}%
\definecolor{currentstroke}{rgb}{0.121569,0.466667,0.705882}%
\pgfsetstrokecolor{currentstroke}%
\pgfsetstrokeopacity{0.521182}%
\pgfsetdash{}{0pt}%
\pgfpathmoveto{\pgfqpoint{2.989867in}{1.999513in}}%
\pgfpathcurveto{\pgfqpoint{2.998103in}{1.999513in}}{\pgfqpoint{3.006003in}{2.002786in}}{\pgfqpoint{3.011827in}{2.008610in}}%
\pgfpathcurveto{\pgfqpoint{3.017651in}{2.014433in}}{\pgfqpoint{3.020923in}{2.022334in}}{\pgfqpoint{3.020923in}{2.030570in}}%
\pgfpathcurveto{\pgfqpoint{3.020923in}{2.038806in}}{\pgfqpoint{3.017651in}{2.046706in}}{\pgfqpoint{3.011827in}{2.052530in}}%
\pgfpathcurveto{\pgfqpoint{3.006003in}{2.058354in}}{\pgfqpoint{2.998103in}{2.061626in}}{\pgfqpoint{2.989867in}{2.061626in}}%
\pgfpathcurveto{\pgfqpoint{2.981630in}{2.061626in}}{\pgfqpoint{2.973730in}{2.058354in}}{\pgfqpoint{2.967906in}{2.052530in}}%
\pgfpathcurveto{\pgfqpoint{2.962082in}{2.046706in}}{\pgfqpoint{2.958810in}{2.038806in}}{\pgfqpoint{2.958810in}{2.030570in}}%
\pgfpathcurveto{\pgfqpoint{2.958810in}{2.022334in}}{\pgfqpoint{2.962082in}{2.014433in}}{\pgfqpoint{2.967906in}{2.008610in}}%
\pgfpathcurveto{\pgfqpoint{2.973730in}{2.002786in}}{\pgfqpoint{2.981630in}{1.999513in}}{\pgfqpoint{2.989867in}{1.999513in}}%
\pgfpathclose%
\pgfusepath{stroke,fill}%
\end{pgfscope}%
\begin{pgfscope}%
\pgfpathrectangle{\pgfqpoint{0.100000in}{0.212622in}}{\pgfqpoint{3.696000in}{3.696000in}}%
\pgfusepath{clip}%
\pgfsetbuttcap%
\pgfsetroundjoin%
\definecolor{currentfill}{rgb}{0.121569,0.466667,0.705882}%
\pgfsetfillcolor{currentfill}%
\pgfsetfillopacity{0.522691}%
\pgfsetlinewidth{1.003750pt}%
\definecolor{currentstroke}{rgb}{0.121569,0.466667,0.705882}%
\pgfsetstrokecolor{currentstroke}%
\pgfsetstrokeopacity{0.522691}%
\pgfsetdash{}{0pt}%
\pgfpathmoveto{\pgfqpoint{1.134475in}{1.728619in}}%
\pgfpathcurveto{\pgfqpoint{1.142711in}{1.728619in}}{\pgfqpoint{1.150612in}{1.731892in}}{\pgfqpoint{1.156435in}{1.737716in}}%
\pgfpathcurveto{\pgfqpoint{1.162259in}{1.743540in}}{\pgfqpoint{1.165532in}{1.751440in}}{\pgfqpoint{1.165532in}{1.759676in}}%
\pgfpathcurveto{\pgfqpoint{1.165532in}{1.767912in}}{\pgfqpoint{1.162259in}{1.775812in}}{\pgfqpoint{1.156435in}{1.781636in}}%
\pgfpathcurveto{\pgfqpoint{1.150612in}{1.787460in}}{\pgfqpoint{1.142711in}{1.790732in}}{\pgfqpoint{1.134475in}{1.790732in}}%
\pgfpathcurveto{\pgfqpoint{1.126239in}{1.790732in}}{\pgfqpoint{1.118339in}{1.787460in}}{\pgfqpoint{1.112515in}{1.781636in}}%
\pgfpathcurveto{\pgfqpoint{1.106691in}{1.775812in}}{\pgfqpoint{1.103419in}{1.767912in}}{\pgfqpoint{1.103419in}{1.759676in}}%
\pgfpathcurveto{\pgfqpoint{1.103419in}{1.751440in}}{\pgfqpoint{1.106691in}{1.743540in}}{\pgfqpoint{1.112515in}{1.737716in}}%
\pgfpathcurveto{\pgfqpoint{1.118339in}{1.731892in}}{\pgfqpoint{1.126239in}{1.728619in}}{\pgfqpoint{1.134475in}{1.728619in}}%
\pgfpathclose%
\pgfusepath{stroke,fill}%
\end{pgfscope}%
\begin{pgfscope}%
\pgfpathrectangle{\pgfqpoint{0.100000in}{0.212622in}}{\pgfqpoint{3.696000in}{3.696000in}}%
\pgfusepath{clip}%
\pgfsetbuttcap%
\pgfsetroundjoin%
\definecolor{currentfill}{rgb}{0.121569,0.466667,0.705882}%
\pgfsetfillcolor{currentfill}%
\pgfsetfillopacity{0.522719}%
\pgfsetlinewidth{1.003750pt}%
\definecolor{currentstroke}{rgb}{0.121569,0.466667,0.705882}%
\pgfsetstrokecolor{currentstroke}%
\pgfsetstrokeopacity{0.522719}%
\pgfsetdash{}{0pt}%
\pgfpathmoveto{\pgfqpoint{1.126208in}{1.719623in}}%
\pgfpathcurveto{\pgfqpoint{1.134444in}{1.719623in}}{\pgfqpoint{1.142344in}{1.722895in}}{\pgfqpoint{1.148168in}{1.728719in}}%
\pgfpathcurveto{\pgfqpoint{1.153992in}{1.734543in}}{\pgfqpoint{1.157265in}{1.742443in}}{\pgfqpoint{1.157265in}{1.750679in}}%
\pgfpathcurveto{\pgfqpoint{1.157265in}{1.758916in}}{\pgfqpoint{1.153992in}{1.766816in}}{\pgfqpoint{1.148168in}{1.772640in}}%
\pgfpathcurveto{\pgfqpoint{1.142344in}{1.778464in}}{\pgfqpoint{1.134444in}{1.781736in}}{\pgfqpoint{1.126208in}{1.781736in}}%
\pgfpathcurveto{\pgfqpoint{1.117972in}{1.781736in}}{\pgfqpoint{1.110072in}{1.778464in}}{\pgfqpoint{1.104248in}{1.772640in}}%
\pgfpathcurveto{\pgfqpoint{1.098424in}{1.766816in}}{\pgfqpoint{1.095152in}{1.758916in}}{\pgfqpoint{1.095152in}{1.750679in}}%
\pgfpathcurveto{\pgfqpoint{1.095152in}{1.742443in}}{\pgfqpoint{1.098424in}{1.734543in}}{\pgfqpoint{1.104248in}{1.728719in}}%
\pgfpathcurveto{\pgfqpoint{1.110072in}{1.722895in}}{\pgfqpoint{1.117972in}{1.719623in}}{\pgfqpoint{1.126208in}{1.719623in}}%
\pgfpathclose%
\pgfusepath{stroke,fill}%
\end{pgfscope}%
\begin{pgfscope}%
\pgfpathrectangle{\pgfqpoint{0.100000in}{0.212622in}}{\pgfqpoint{3.696000in}{3.696000in}}%
\pgfusepath{clip}%
\pgfsetbuttcap%
\pgfsetroundjoin%
\definecolor{currentfill}{rgb}{0.121569,0.466667,0.705882}%
\pgfsetfillcolor{currentfill}%
\pgfsetfillopacity{0.523107}%
\pgfsetlinewidth{1.003750pt}%
\definecolor{currentstroke}{rgb}{0.121569,0.466667,0.705882}%
\pgfsetstrokecolor{currentstroke}%
\pgfsetstrokeopacity{0.523107}%
\pgfsetdash{}{0pt}%
\pgfpathmoveto{\pgfqpoint{1.131267in}{1.729149in}}%
\pgfpathcurveto{\pgfqpoint{1.139503in}{1.729149in}}{\pgfqpoint{1.147403in}{1.732421in}}{\pgfqpoint{1.153227in}{1.738245in}}%
\pgfpathcurveto{\pgfqpoint{1.159051in}{1.744069in}}{\pgfqpoint{1.162323in}{1.751969in}}{\pgfqpoint{1.162323in}{1.760206in}}%
\pgfpathcurveto{\pgfqpoint{1.162323in}{1.768442in}}{\pgfqpoint{1.159051in}{1.776342in}}{\pgfqpoint{1.153227in}{1.782166in}}%
\pgfpathcurveto{\pgfqpoint{1.147403in}{1.787990in}}{\pgfqpoint{1.139503in}{1.791262in}}{\pgfqpoint{1.131267in}{1.791262in}}%
\pgfpathcurveto{\pgfqpoint{1.123030in}{1.791262in}}{\pgfqpoint{1.115130in}{1.787990in}}{\pgfqpoint{1.109306in}{1.782166in}}%
\pgfpathcurveto{\pgfqpoint{1.103482in}{1.776342in}}{\pgfqpoint{1.100210in}{1.768442in}}{\pgfqpoint{1.100210in}{1.760206in}}%
\pgfpathcurveto{\pgfqpoint{1.100210in}{1.751969in}}{\pgfqpoint{1.103482in}{1.744069in}}{\pgfqpoint{1.109306in}{1.738245in}}%
\pgfpathcurveto{\pgfqpoint{1.115130in}{1.732421in}}{\pgfqpoint{1.123030in}{1.729149in}}{\pgfqpoint{1.131267in}{1.729149in}}%
\pgfpathclose%
\pgfusepath{stroke,fill}%
\end{pgfscope}%
\begin{pgfscope}%
\pgfpathrectangle{\pgfqpoint{0.100000in}{0.212622in}}{\pgfqpoint{3.696000in}{3.696000in}}%
\pgfusepath{clip}%
\pgfsetbuttcap%
\pgfsetroundjoin%
\definecolor{currentfill}{rgb}{0.121569,0.466667,0.705882}%
\pgfsetfillcolor{currentfill}%
\pgfsetfillopacity{0.523273}%
\pgfsetlinewidth{1.003750pt}%
\definecolor{currentstroke}{rgb}{0.121569,0.466667,0.705882}%
\pgfsetstrokecolor{currentstroke}%
\pgfsetstrokeopacity{0.523273}%
\pgfsetdash{}{0pt}%
\pgfpathmoveto{\pgfqpoint{2.999347in}{2.007145in}}%
\pgfpathcurveto{\pgfqpoint{3.007583in}{2.007145in}}{\pgfqpoint{3.015483in}{2.010418in}}{\pgfqpoint{3.021307in}{2.016242in}}%
\pgfpathcurveto{\pgfqpoint{3.027131in}{2.022066in}}{\pgfqpoint{3.030403in}{2.029966in}}{\pgfqpoint{3.030403in}{2.038202in}}%
\pgfpathcurveto{\pgfqpoint{3.030403in}{2.046438in}}{\pgfqpoint{3.027131in}{2.054338in}}{\pgfqpoint{3.021307in}{2.060162in}}%
\pgfpathcurveto{\pgfqpoint{3.015483in}{2.065986in}}{\pgfqpoint{3.007583in}{2.069258in}}{\pgfqpoint{2.999347in}{2.069258in}}%
\pgfpathcurveto{\pgfqpoint{2.991110in}{2.069258in}}{\pgfqpoint{2.983210in}{2.065986in}}{\pgfqpoint{2.977386in}{2.060162in}}%
\pgfpathcurveto{\pgfqpoint{2.971563in}{2.054338in}}{\pgfqpoint{2.968290in}{2.046438in}}{\pgfqpoint{2.968290in}{2.038202in}}%
\pgfpathcurveto{\pgfqpoint{2.968290in}{2.029966in}}{\pgfqpoint{2.971563in}{2.022066in}}{\pgfqpoint{2.977386in}{2.016242in}}%
\pgfpathcurveto{\pgfqpoint{2.983210in}{2.010418in}}{\pgfqpoint{2.991110in}{2.007145in}}{\pgfqpoint{2.999347in}{2.007145in}}%
\pgfpathclose%
\pgfusepath{stroke,fill}%
\end{pgfscope}%
\begin{pgfscope}%
\pgfpathrectangle{\pgfqpoint{0.100000in}{0.212622in}}{\pgfqpoint{3.696000in}{3.696000in}}%
\pgfusepath{clip}%
\pgfsetbuttcap%
\pgfsetroundjoin%
\definecolor{currentfill}{rgb}{0.121569,0.466667,0.705882}%
\pgfsetfillcolor{currentfill}%
\pgfsetfillopacity{0.523710}%
\pgfsetlinewidth{1.003750pt}%
\definecolor{currentstroke}{rgb}{0.121569,0.466667,0.705882}%
\pgfsetstrokecolor{currentstroke}%
\pgfsetstrokeopacity{0.523710}%
\pgfsetdash{}{0pt}%
\pgfpathmoveto{\pgfqpoint{3.008043in}{1.999160in}}%
\pgfpathcurveto{\pgfqpoint{3.016280in}{1.999160in}}{\pgfqpoint{3.024180in}{2.002432in}}{\pgfqpoint{3.030004in}{2.008256in}}%
\pgfpathcurveto{\pgfqpoint{3.035828in}{2.014080in}}{\pgfqpoint{3.039100in}{2.021980in}}{\pgfqpoint{3.039100in}{2.030216in}}%
\pgfpathcurveto{\pgfqpoint{3.039100in}{2.038452in}}{\pgfqpoint{3.035828in}{2.046352in}}{\pgfqpoint{3.030004in}{2.052176in}}%
\pgfpathcurveto{\pgfqpoint{3.024180in}{2.058000in}}{\pgfqpoint{3.016280in}{2.061273in}}{\pgfqpoint{3.008043in}{2.061273in}}%
\pgfpathcurveto{\pgfqpoint{2.999807in}{2.061273in}}{\pgfqpoint{2.991907in}{2.058000in}}{\pgfqpoint{2.986083in}{2.052176in}}%
\pgfpathcurveto{\pgfqpoint{2.980259in}{2.046352in}}{\pgfqpoint{2.976987in}{2.038452in}}{\pgfqpoint{2.976987in}{2.030216in}}%
\pgfpathcurveto{\pgfqpoint{2.976987in}{2.021980in}}{\pgfqpoint{2.980259in}{2.014080in}}{\pgfqpoint{2.986083in}{2.008256in}}%
\pgfpathcurveto{\pgfqpoint{2.991907in}{2.002432in}}{\pgfqpoint{2.999807in}{1.999160in}}{\pgfqpoint{3.008043in}{1.999160in}}%
\pgfpathclose%
\pgfusepath{stroke,fill}%
\end{pgfscope}%
\begin{pgfscope}%
\pgfpathrectangle{\pgfqpoint{0.100000in}{0.212622in}}{\pgfqpoint{3.696000in}{3.696000in}}%
\pgfusepath{clip}%
\pgfsetbuttcap%
\pgfsetroundjoin%
\definecolor{currentfill}{rgb}{0.121569,0.466667,0.705882}%
\pgfsetfillcolor{currentfill}%
\pgfsetfillopacity{0.525087}%
\pgfsetlinewidth{1.003750pt}%
\definecolor{currentstroke}{rgb}{0.121569,0.466667,0.705882}%
\pgfsetstrokecolor{currentstroke}%
\pgfsetstrokeopacity{0.525087}%
\pgfsetdash{}{0pt}%
\pgfpathmoveto{\pgfqpoint{1.121273in}{1.715993in}}%
\pgfpathcurveto{\pgfqpoint{1.129509in}{1.715993in}}{\pgfqpoint{1.137409in}{1.719265in}}{\pgfqpoint{1.143233in}{1.725089in}}%
\pgfpathcurveto{\pgfqpoint{1.149057in}{1.730913in}}{\pgfqpoint{1.152329in}{1.738813in}}{\pgfqpoint{1.152329in}{1.747050in}}%
\pgfpathcurveto{\pgfqpoint{1.152329in}{1.755286in}}{\pgfqpoint{1.149057in}{1.763186in}}{\pgfqpoint{1.143233in}{1.769010in}}%
\pgfpathcurveto{\pgfqpoint{1.137409in}{1.774834in}}{\pgfqpoint{1.129509in}{1.778106in}}{\pgfqpoint{1.121273in}{1.778106in}}%
\pgfpathcurveto{\pgfqpoint{1.113036in}{1.778106in}}{\pgfqpoint{1.105136in}{1.774834in}}{\pgfqpoint{1.099312in}{1.769010in}}%
\pgfpathcurveto{\pgfqpoint{1.093488in}{1.763186in}}{\pgfqpoint{1.090216in}{1.755286in}}{\pgfqpoint{1.090216in}{1.747050in}}%
\pgfpathcurveto{\pgfqpoint{1.090216in}{1.738813in}}{\pgfqpoint{1.093488in}{1.730913in}}{\pgfqpoint{1.099312in}{1.725089in}}%
\pgfpathcurveto{\pgfqpoint{1.105136in}{1.719265in}}{\pgfqpoint{1.113036in}{1.715993in}}{\pgfqpoint{1.121273in}{1.715993in}}%
\pgfpathclose%
\pgfusepath{stroke,fill}%
\end{pgfscope}%
\begin{pgfscope}%
\pgfpathrectangle{\pgfqpoint{0.100000in}{0.212622in}}{\pgfqpoint{3.696000in}{3.696000in}}%
\pgfusepath{clip}%
\pgfsetbuttcap%
\pgfsetroundjoin%
\definecolor{currentfill}{rgb}{0.121569,0.466667,0.705882}%
\pgfsetfillcolor{currentfill}%
\pgfsetfillopacity{0.525662}%
\pgfsetlinewidth{1.003750pt}%
\definecolor{currentstroke}{rgb}{0.121569,0.466667,0.705882}%
\pgfsetstrokecolor{currentstroke}%
\pgfsetstrokeopacity{0.525662}%
\pgfsetdash{}{0pt}%
\pgfpathmoveto{\pgfqpoint{3.018328in}{2.004812in}}%
\pgfpathcurveto{\pgfqpoint{3.026565in}{2.004812in}}{\pgfqpoint{3.034465in}{2.008084in}}{\pgfqpoint{3.040288in}{2.013908in}}%
\pgfpathcurveto{\pgfqpoint{3.046112in}{2.019732in}}{\pgfqpoint{3.049385in}{2.027632in}}{\pgfqpoint{3.049385in}{2.035868in}}%
\pgfpathcurveto{\pgfqpoint{3.049385in}{2.044105in}}{\pgfqpoint{3.046112in}{2.052005in}}{\pgfqpoint{3.040288in}{2.057829in}}%
\pgfpathcurveto{\pgfqpoint{3.034465in}{2.063653in}}{\pgfqpoint{3.026565in}{2.066925in}}{\pgfqpoint{3.018328in}{2.066925in}}%
\pgfpathcurveto{\pgfqpoint{3.010092in}{2.066925in}}{\pgfqpoint{3.002192in}{2.063653in}}{\pgfqpoint{2.996368in}{2.057829in}}%
\pgfpathcurveto{\pgfqpoint{2.990544in}{2.052005in}}{\pgfqpoint{2.987272in}{2.044105in}}{\pgfqpoint{2.987272in}{2.035868in}}%
\pgfpathcurveto{\pgfqpoint{2.987272in}{2.027632in}}{\pgfqpoint{2.990544in}{2.019732in}}{\pgfqpoint{2.996368in}{2.013908in}}%
\pgfpathcurveto{\pgfqpoint{3.002192in}{2.008084in}}{\pgfqpoint{3.010092in}{2.004812in}}{\pgfqpoint{3.018328in}{2.004812in}}%
\pgfpathclose%
\pgfusepath{stroke,fill}%
\end{pgfscope}%
\begin{pgfscope}%
\pgfpathrectangle{\pgfqpoint{0.100000in}{0.212622in}}{\pgfqpoint{3.696000in}{3.696000in}}%
\pgfusepath{clip}%
\pgfsetbuttcap%
\pgfsetroundjoin%
\definecolor{currentfill}{rgb}{0.121569,0.466667,0.705882}%
\pgfsetfillcolor{currentfill}%
\pgfsetfillopacity{0.526095}%
\pgfsetlinewidth{1.003750pt}%
\definecolor{currentstroke}{rgb}{0.121569,0.466667,0.705882}%
\pgfsetstrokecolor{currentstroke}%
\pgfsetstrokeopacity{0.526095}%
\pgfsetdash{}{0pt}%
\pgfpathmoveto{\pgfqpoint{3.023147in}{2.000858in}}%
\pgfpathcurveto{\pgfqpoint{3.031383in}{2.000858in}}{\pgfqpoint{3.039283in}{2.004130in}}{\pgfqpoint{3.045107in}{2.009954in}}%
\pgfpathcurveto{\pgfqpoint{3.050931in}{2.015778in}}{\pgfqpoint{3.054203in}{2.023678in}}{\pgfqpoint{3.054203in}{2.031915in}}%
\pgfpathcurveto{\pgfqpoint{3.054203in}{2.040151in}}{\pgfqpoint{3.050931in}{2.048051in}}{\pgfqpoint{3.045107in}{2.053875in}}%
\pgfpathcurveto{\pgfqpoint{3.039283in}{2.059699in}}{\pgfqpoint{3.031383in}{2.062971in}}{\pgfqpoint{3.023147in}{2.062971in}}%
\pgfpathcurveto{\pgfqpoint{3.014910in}{2.062971in}}{\pgfqpoint{3.007010in}{2.059699in}}{\pgfqpoint{3.001186in}{2.053875in}}%
\pgfpathcurveto{\pgfqpoint{2.995363in}{2.048051in}}{\pgfqpoint{2.992090in}{2.040151in}}{\pgfqpoint{2.992090in}{2.031915in}}%
\pgfpathcurveto{\pgfqpoint{2.992090in}{2.023678in}}{\pgfqpoint{2.995363in}{2.015778in}}{\pgfqpoint{3.001186in}{2.009954in}}%
\pgfpathcurveto{\pgfqpoint{3.007010in}{2.004130in}}{\pgfqpoint{3.014910in}{2.000858in}}{\pgfqpoint{3.023147in}{2.000858in}}%
\pgfpathclose%
\pgfusepath{stroke,fill}%
\end{pgfscope}%
\begin{pgfscope}%
\pgfpathrectangle{\pgfqpoint{0.100000in}{0.212622in}}{\pgfqpoint{3.696000in}{3.696000in}}%
\pgfusepath{clip}%
\pgfsetbuttcap%
\pgfsetroundjoin%
\definecolor{currentfill}{rgb}{0.121569,0.466667,0.705882}%
\pgfsetfillcolor{currentfill}%
\pgfsetfillopacity{0.526975}%
\pgfsetlinewidth{1.003750pt}%
\definecolor{currentstroke}{rgb}{0.121569,0.466667,0.705882}%
\pgfsetstrokecolor{currentstroke}%
\pgfsetstrokeopacity{0.526975}%
\pgfsetdash{}{0pt}%
\pgfpathmoveto{\pgfqpoint{1.112841in}{1.715179in}}%
\pgfpathcurveto{\pgfqpoint{1.121077in}{1.715179in}}{\pgfqpoint{1.128977in}{1.718451in}}{\pgfqpoint{1.134801in}{1.724275in}}%
\pgfpathcurveto{\pgfqpoint{1.140625in}{1.730099in}}{\pgfqpoint{1.143897in}{1.737999in}}{\pgfqpoint{1.143897in}{1.746235in}}%
\pgfpathcurveto{\pgfqpoint{1.143897in}{1.754472in}}{\pgfqpoint{1.140625in}{1.762372in}}{\pgfqpoint{1.134801in}{1.768196in}}%
\pgfpathcurveto{\pgfqpoint{1.128977in}{1.774020in}}{\pgfqpoint{1.121077in}{1.777292in}}{\pgfqpoint{1.112841in}{1.777292in}}%
\pgfpathcurveto{\pgfqpoint{1.104604in}{1.777292in}}{\pgfqpoint{1.096704in}{1.774020in}}{\pgfqpoint{1.090880in}{1.768196in}}%
\pgfpathcurveto{\pgfqpoint{1.085056in}{1.762372in}}{\pgfqpoint{1.081784in}{1.754472in}}{\pgfqpoint{1.081784in}{1.746235in}}%
\pgfpathcurveto{\pgfqpoint{1.081784in}{1.737999in}}{\pgfqpoint{1.085056in}{1.730099in}}{\pgfqpoint{1.090880in}{1.724275in}}%
\pgfpathcurveto{\pgfqpoint{1.096704in}{1.718451in}}{\pgfqpoint{1.104604in}{1.715179in}}{\pgfqpoint{1.112841in}{1.715179in}}%
\pgfpathclose%
\pgfusepath{stroke,fill}%
\end{pgfscope}%
\begin{pgfscope}%
\pgfpathrectangle{\pgfqpoint{0.100000in}{0.212622in}}{\pgfqpoint{3.696000in}{3.696000in}}%
\pgfusepath{clip}%
\pgfsetbuttcap%
\pgfsetroundjoin%
\definecolor{currentfill}{rgb}{0.121569,0.466667,0.705882}%
\pgfsetfillcolor{currentfill}%
\pgfsetfillopacity{0.527339}%
\pgfsetlinewidth{1.003750pt}%
\definecolor{currentstroke}{rgb}{0.121569,0.466667,0.705882}%
\pgfsetstrokecolor{currentstroke}%
\pgfsetstrokeopacity{0.527339}%
\pgfsetdash{}{0pt}%
\pgfpathmoveto{\pgfqpoint{3.029128in}{2.002582in}}%
\pgfpathcurveto{\pgfqpoint{3.037364in}{2.002582in}}{\pgfqpoint{3.045264in}{2.005854in}}{\pgfqpoint{3.051088in}{2.011678in}}%
\pgfpathcurveto{\pgfqpoint{3.056912in}{2.017502in}}{\pgfqpoint{3.060185in}{2.025402in}}{\pgfqpoint{3.060185in}{2.033638in}}%
\pgfpathcurveto{\pgfqpoint{3.060185in}{2.041874in}}{\pgfqpoint{3.056912in}{2.049774in}}{\pgfqpoint{3.051088in}{2.055598in}}%
\pgfpathcurveto{\pgfqpoint{3.045264in}{2.061422in}}{\pgfqpoint{3.037364in}{2.064695in}}{\pgfqpoint{3.029128in}{2.064695in}}%
\pgfpathcurveto{\pgfqpoint{3.020892in}{2.064695in}}{\pgfqpoint{3.012992in}{2.061422in}}{\pgfqpoint{3.007168in}{2.055598in}}%
\pgfpathcurveto{\pgfqpoint{3.001344in}{2.049774in}}{\pgfqpoint{2.998072in}{2.041874in}}{\pgfqpoint{2.998072in}{2.033638in}}%
\pgfpathcurveto{\pgfqpoint{2.998072in}{2.025402in}}{\pgfqpoint{3.001344in}{2.017502in}}{\pgfqpoint{3.007168in}{2.011678in}}%
\pgfpathcurveto{\pgfqpoint{3.012992in}{2.005854in}}{\pgfqpoint{3.020892in}{2.002582in}}{\pgfqpoint{3.029128in}{2.002582in}}%
\pgfpathclose%
\pgfusepath{stroke,fill}%
\end{pgfscope}%
\begin{pgfscope}%
\pgfpathrectangle{\pgfqpoint{0.100000in}{0.212622in}}{\pgfqpoint{3.696000in}{3.696000in}}%
\pgfusepath{clip}%
\pgfsetbuttcap%
\pgfsetroundjoin%
\definecolor{currentfill}{rgb}{0.121569,0.466667,0.705882}%
\pgfsetfillcolor{currentfill}%
\pgfsetfillopacity{0.528046}%
\pgfsetlinewidth{1.003750pt}%
\definecolor{currentstroke}{rgb}{0.121569,0.466667,0.705882}%
\pgfsetstrokecolor{currentstroke}%
\pgfsetstrokeopacity{0.528046}%
\pgfsetdash{}{0pt}%
\pgfpathmoveto{\pgfqpoint{3.036596in}{1.997856in}}%
\pgfpathcurveto{\pgfqpoint{3.044832in}{1.997856in}}{\pgfqpoint{3.052732in}{2.001128in}}{\pgfqpoint{3.058556in}{2.006952in}}%
\pgfpathcurveto{\pgfqpoint{3.064380in}{2.012776in}}{\pgfqpoint{3.067652in}{2.020676in}}{\pgfqpoint{3.067652in}{2.028912in}}%
\pgfpathcurveto{\pgfqpoint{3.067652in}{2.037148in}}{\pgfqpoint{3.064380in}{2.045049in}}{\pgfqpoint{3.058556in}{2.050872in}}%
\pgfpathcurveto{\pgfqpoint{3.052732in}{2.056696in}}{\pgfqpoint{3.044832in}{2.059969in}}{\pgfqpoint{3.036596in}{2.059969in}}%
\pgfpathcurveto{\pgfqpoint{3.028359in}{2.059969in}}{\pgfqpoint{3.020459in}{2.056696in}}{\pgfqpoint{3.014635in}{2.050872in}}%
\pgfpathcurveto{\pgfqpoint{3.008812in}{2.045049in}}{\pgfqpoint{3.005539in}{2.037148in}}{\pgfqpoint{3.005539in}{2.028912in}}%
\pgfpathcurveto{\pgfqpoint{3.005539in}{2.020676in}}{\pgfqpoint{3.008812in}{2.012776in}}{\pgfqpoint{3.014635in}{2.006952in}}%
\pgfpathcurveto{\pgfqpoint{3.020459in}{2.001128in}}{\pgfqpoint{3.028359in}{1.997856in}}{\pgfqpoint{3.036596in}{1.997856in}}%
\pgfpathclose%
\pgfusepath{stroke,fill}%
\end{pgfscope}%
\begin{pgfscope}%
\pgfpathrectangle{\pgfqpoint{0.100000in}{0.212622in}}{\pgfqpoint{3.696000in}{3.696000in}}%
\pgfusepath{clip}%
\pgfsetbuttcap%
\pgfsetroundjoin%
\definecolor{currentfill}{rgb}{0.121569,0.466667,0.705882}%
\pgfsetfillcolor{currentfill}%
\pgfsetfillopacity{0.529165}%
\pgfsetlinewidth{1.003750pt}%
\definecolor{currentstroke}{rgb}{0.121569,0.466667,0.705882}%
\pgfsetstrokecolor{currentstroke}%
\pgfsetstrokeopacity{0.529165}%
\pgfsetdash{}{0pt}%
\pgfpathmoveto{\pgfqpoint{1.110444in}{1.716532in}}%
\pgfpathcurveto{\pgfqpoint{1.118680in}{1.716532in}}{\pgfqpoint{1.126580in}{1.719804in}}{\pgfqpoint{1.132404in}{1.725628in}}%
\pgfpathcurveto{\pgfqpoint{1.138228in}{1.731452in}}{\pgfqpoint{1.141501in}{1.739352in}}{\pgfqpoint{1.141501in}{1.747588in}}%
\pgfpathcurveto{\pgfqpoint{1.141501in}{1.755824in}}{\pgfqpoint{1.138228in}{1.763725in}}{\pgfqpoint{1.132404in}{1.769548in}}%
\pgfpathcurveto{\pgfqpoint{1.126580in}{1.775372in}}{\pgfqpoint{1.118680in}{1.778645in}}{\pgfqpoint{1.110444in}{1.778645in}}%
\pgfpathcurveto{\pgfqpoint{1.102208in}{1.778645in}}{\pgfqpoint{1.094308in}{1.775372in}}{\pgfqpoint{1.088484in}{1.769548in}}%
\pgfpathcurveto{\pgfqpoint{1.082660in}{1.763725in}}{\pgfqpoint{1.079388in}{1.755824in}}{\pgfqpoint{1.079388in}{1.747588in}}%
\pgfpathcurveto{\pgfqpoint{1.079388in}{1.739352in}}{\pgfqpoint{1.082660in}{1.731452in}}{\pgfqpoint{1.088484in}{1.725628in}}%
\pgfpathcurveto{\pgfqpoint{1.094308in}{1.719804in}}{\pgfqpoint{1.102208in}{1.716532in}}{\pgfqpoint{1.110444in}{1.716532in}}%
\pgfpathclose%
\pgfusepath{stroke,fill}%
\end{pgfscope}%
\begin{pgfscope}%
\pgfpathrectangle{\pgfqpoint{0.100000in}{0.212622in}}{\pgfqpoint{3.696000in}{3.696000in}}%
\pgfusepath{clip}%
\pgfsetbuttcap%
\pgfsetroundjoin%
\definecolor{currentfill}{rgb}{0.121569,0.466667,0.705882}%
\pgfsetfillcolor{currentfill}%
\pgfsetfillopacity{0.529918}%
\pgfsetlinewidth{1.003750pt}%
\definecolor{currentstroke}{rgb}{0.121569,0.466667,0.705882}%
\pgfsetstrokecolor{currentstroke}%
\pgfsetstrokeopacity{0.529918}%
\pgfsetdash{}{0pt}%
\pgfpathmoveto{\pgfqpoint{1.105534in}{1.711873in}}%
\pgfpathcurveto{\pgfqpoint{1.113771in}{1.711873in}}{\pgfqpoint{1.121671in}{1.715145in}}{\pgfqpoint{1.127495in}{1.720969in}}%
\pgfpathcurveto{\pgfqpoint{1.133319in}{1.726793in}}{\pgfqpoint{1.136591in}{1.734693in}}{\pgfqpoint{1.136591in}{1.742930in}}%
\pgfpathcurveto{\pgfqpoint{1.136591in}{1.751166in}}{\pgfqpoint{1.133319in}{1.759066in}}{\pgfqpoint{1.127495in}{1.764890in}}%
\pgfpathcurveto{\pgfqpoint{1.121671in}{1.770714in}}{\pgfqpoint{1.113771in}{1.773986in}}{\pgfqpoint{1.105534in}{1.773986in}}%
\pgfpathcurveto{\pgfqpoint{1.097298in}{1.773986in}}{\pgfqpoint{1.089398in}{1.770714in}}{\pgfqpoint{1.083574in}{1.764890in}}%
\pgfpathcurveto{\pgfqpoint{1.077750in}{1.759066in}}{\pgfqpoint{1.074478in}{1.751166in}}{\pgfqpoint{1.074478in}{1.742930in}}%
\pgfpathcurveto{\pgfqpoint{1.074478in}{1.734693in}}{\pgfqpoint{1.077750in}{1.726793in}}{\pgfqpoint{1.083574in}{1.720969in}}%
\pgfpathcurveto{\pgfqpoint{1.089398in}{1.715145in}}{\pgfqpoint{1.097298in}{1.711873in}}{\pgfqpoint{1.105534in}{1.711873in}}%
\pgfpathclose%
\pgfusepath{stroke,fill}%
\end{pgfscope}%
\begin{pgfscope}%
\pgfpathrectangle{\pgfqpoint{0.100000in}{0.212622in}}{\pgfqpoint{3.696000in}{3.696000in}}%
\pgfusepath{clip}%
\pgfsetbuttcap%
\pgfsetroundjoin%
\definecolor{currentfill}{rgb}{0.121569,0.466667,0.705882}%
\pgfsetfillcolor{currentfill}%
\pgfsetfillopacity{0.530052}%
\pgfsetlinewidth{1.003750pt}%
\definecolor{currentstroke}{rgb}{0.121569,0.466667,0.705882}%
\pgfsetstrokecolor{currentstroke}%
\pgfsetstrokeopacity{0.530052}%
\pgfsetdash{}{0pt}%
\pgfpathmoveto{\pgfqpoint{3.045797in}{2.006244in}}%
\pgfpathcurveto{\pgfqpoint{3.054033in}{2.006244in}}{\pgfqpoint{3.061933in}{2.009516in}}{\pgfqpoint{3.067757in}{2.015340in}}%
\pgfpathcurveto{\pgfqpoint{3.073581in}{2.021164in}}{\pgfqpoint{3.076853in}{2.029064in}}{\pgfqpoint{3.076853in}{2.037300in}}%
\pgfpathcurveto{\pgfqpoint{3.076853in}{2.045537in}}{\pgfqpoint{3.073581in}{2.053437in}}{\pgfqpoint{3.067757in}{2.059261in}}%
\pgfpathcurveto{\pgfqpoint{3.061933in}{2.065085in}}{\pgfqpoint{3.054033in}{2.068357in}}{\pgfqpoint{3.045797in}{2.068357in}}%
\pgfpathcurveto{\pgfqpoint{3.037561in}{2.068357in}}{\pgfqpoint{3.029661in}{2.065085in}}{\pgfqpoint{3.023837in}{2.059261in}}%
\pgfpathcurveto{\pgfqpoint{3.018013in}{2.053437in}}{\pgfqpoint{3.014740in}{2.045537in}}{\pgfqpoint{3.014740in}{2.037300in}}%
\pgfpathcurveto{\pgfqpoint{3.014740in}{2.029064in}}{\pgfqpoint{3.018013in}{2.021164in}}{\pgfqpoint{3.023837in}{2.015340in}}%
\pgfpathcurveto{\pgfqpoint{3.029661in}{2.009516in}}{\pgfqpoint{3.037561in}{2.006244in}}{\pgfqpoint{3.045797in}{2.006244in}}%
\pgfpathclose%
\pgfusepath{stroke,fill}%
\end{pgfscope}%
\begin{pgfscope}%
\pgfpathrectangle{\pgfqpoint{0.100000in}{0.212622in}}{\pgfqpoint{3.696000in}{3.696000in}}%
\pgfusepath{clip}%
\pgfsetbuttcap%
\pgfsetroundjoin%
\definecolor{currentfill}{rgb}{0.121569,0.466667,0.705882}%
\pgfsetfillcolor{currentfill}%
\pgfsetfillopacity{0.530370}%
\pgfsetlinewidth{1.003750pt}%
\definecolor{currentstroke}{rgb}{0.121569,0.466667,0.705882}%
\pgfsetstrokecolor{currentstroke}%
\pgfsetstrokeopacity{0.530370}%
\pgfsetdash{}{0pt}%
\pgfpathmoveto{\pgfqpoint{3.050275in}{2.003419in}}%
\pgfpathcurveto{\pgfqpoint{3.058511in}{2.003419in}}{\pgfqpoint{3.066411in}{2.006691in}}{\pgfqpoint{3.072235in}{2.012515in}}%
\pgfpathcurveto{\pgfqpoint{3.078059in}{2.018339in}}{\pgfqpoint{3.081331in}{2.026239in}}{\pgfqpoint{3.081331in}{2.034475in}}%
\pgfpathcurveto{\pgfqpoint{3.081331in}{2.042712in}}{\pgfqpoint{3.078059in}{2.050612in}}{\pgfqpoint{3.072235in}{2.056435in}}%
\pgfpathcurveto{\pgfqpoint{3.066411in}{2.062259in}}{\pgfqpoint{3.058511in}{2.065532in}}{\pgfqpoint{3.050275in}{2.065532in}}%
\pgfpathcurveto{\pgfqpoint{3.042039in}{2.065532in}}{\pgfqpoint{3.034138in}{2.062259in}}{\pgfqpoint{3.028315in}{2.056435in}}%
\pgfpathcurveto{\pgfqpoint{3.022491in}{2.050612in}}{\pgfqpoint{3.019218in}{2.042712in}}{\pgfqpoint{3.019218in}{2.034475in}}%
\pgfpathcurveto{\pgfqpoint{3.019218in}{2.026239in}}{\pgfqpoint{3.022491in}{2.018339in}}{\pgfqpoint{3.028315in}{2.012515in}}%
\pgfpathcurveto{\pgfqpoint{3.034138in}{2.006691in}}{\pgfqpoint{3.042039in}{2.003419in}}{\pgfqpoint{3.050275in}{2.003419in}}%
\pgfpathclose%
\pgfusepath{stroke,fill}%
\end{pgfscope}%
\begin{pgfscope}%
\pgfpathrectangle{\pgfqpoint{0.100000in}{0.212622in}}{\pgfqpoint{3.696000in}{3.696000in}}%
\pgfusepath{clip}%
\pgfsetbuttcap%
\pgfsetroundjoin%
\definecolor{currentfill}{rgb}{0.121569,0.466667,0.705882}%
\pgfsetfillcolor{currentfill}%
\pgfsetfillopacity{0.530447}%
\pgfsetlinewidth{1.003750pt}%
\definecolor{currentstroke}{rgb}{0.121569,0.466667,0.705882}%
\pgfsetstrokecolor{currentstroke}%
\pgfsetstrokeopacity{0.530447}%
\pgfsetdash{}{0pt}%
\pgfpathmoveto{\pgfqpoint{1.102630in}{1.708759in}}%
\pgfpathcurveto{\pgfqpoint{1.110866in}{1.708759in}}{\pgfqpoint{1.118766in}{1.712032in}}{\pgfqpoint{1.124590in}{1.717856in}}%
\pgfpathcurveto{\pgfqpoint{1.130414in}{1.723680in}}{\pgfqpoint{1.133686in}{1.731580in}}{\pgfqpoint{1.133686in}{1.739816in}}%
\pgfpathcurveto{\pgfqpoint{1.133686in}{1.748052in}}{\pgfqpoint{1.130414in}{1.755952in}}{\pgfqpoint{1.124590in}{1.761776in}}%
\pgfpathcurveto{\pgfqpoint{1.118766in}{1.767600in}}{\pgfqpoint{1.110866in}{1.770872in}}{\pgfqpoint{1.102630in}{1.770872in}}%
\pgfpathcurveto{\pgfqpoint{1.094393in}{1.770872in}}{\pgfqpoint{1.086493in}{1.767600in}}{\pgfqpoint{1.080669in}{1.761776in}}%
\pgfpathcurveto{\pgfqpoint{1.074846in}{1.755952in}}{\pgfqpoint{1.071573in}{1.748052in}}{\pgfqpoint{1.071573in}{1.739816in}}%
\pgfpathcurveto{\pgfqpoint{1.071573in}{1.731580in}}{\pgfqpoint{1.074846in}{1.723680in}}{\pgfqpoint{1.080669in}{1.717856in}}%
\pgfpathcurveto{\pgfqpoint{1.086493in}{1.712032in}}{\pgfqpoint{1.094393in}{1.708759in}}{\pgfqpoint{1.102630in}{1.708759in}}%
\pgfpathclose%
\pgfusepath{stroke,fill}%
\end{pgfscope}%
\begin{pgfscope}%
\pgfpathrectangle{\pgfqpoint{0.100000in}{0.212622in}}{\pgfqpoint{3.696000in}{3.696000in}}%
\pgfusepath{clip}%
\pgfsetbuttcap%
\pgfsetroundjoin%
\definecolor{currentfill}{rgb}{0.121569,0.466667,0.705882}%
\pgfsetfillcolor{currentfill}%
\pgfsetfillopacity{0.531159}%
\pgfsetlinewidth{1.003750pt}%
\definecolor{currentstroke}{rgb}{0.121569,0.466667,0.705882}%
\pgfsetstrokecolor{currentstroke}%
\pgfsetstrokeopacity{0.531159}%
\pgfsetdash{}{0pt}%
\pgfpathmoveto{\pgfqpoint{3.055948in}{2.004381in}}%
\pgfpathcurveto{\pgfqpoint{3.064185in}{2.004381in}}{\pgfqpoint{3.072085in}{2.007653in}}{\pgfqpoint{3.077909in}{2.013477in}}%
\pgfpathcurveto{\pgfqpoint{3.083733in}{2.019301in}}{\pgfqpoint{3.087005in}{2.027201in}}{\pgfqpoint{3.087005in}{2.035437in}}%
\pgfpathcurveto{\pgfqpoint{3.087005in}{2.043674in}}{\pgfqpoint{3.083733in}{2.051574in}}{\pgfqpoint{3.077909in}{2.057398in}}%
\pgfpathcurveto{\pgfqpoint{3.072085in}{2.063221in}}{\pgfqpoint{3.064185in}{2.066494in}}{\pgfqpoint{3.055948in}{2.066494in}}%
\pgfpathcurveto{\pgfqpoint{3.047712in}{2.066494in}}{\pgfqpoint{3.039812in}{2.063221in}}{\pgfqpoint{3.033988in}{2.057398in}}%
\pgfpathcurveto{\pgfqpoint{3.028164in}{2.051574in}}{\pgfqpoint{3.024892in}{2.043674in}}{\pgfqpoint{3.024892in}{2.035437in}}%
\pgfpathcurveto{\pgfqpoint{3.024892in}{2.027201in}}{\pgfqpoint{3.028164in}{2.019301in}}{\pgfqpoint{3.033988in}{2.013477in}}%
\pgfpathcurveto{\pgfqpoint{3.039812in}{2.007653in}}{\pgfqpoint{3.047712in}{2.004381in}}{\pgfqpoint{3.055948in}{2.004381in}}%
\pgfpathclose%
\pgfusepath{stroke,fill}%
\end{pgfscope}%
\begin{pgfscope}%
\pgfpathrectangle{\pgfqpoint{0.100000in}{0.212622in}}{\pgfqpoint{3.696000in}{3.696000in}}%
\pgfusepath{clip}%
\pgfsetbuttcap%
\pgfsetroundjoin%
\definecolor{currentfill}{rgb}{0.121569,0.466667,0.705882}%
\pgfsetfillcolor{currentfill}%
\pgfsetfillopacity{0.531779}%
\pgfsetlinewidth{1.003750pt}%
\definecolor{currentstroke}{rgb}{0.121569,0.466667,0.705882}%
\pgfsetstrokecolor{currentstroke}%
\pgfsetstrokeopacity{0.531779}%
\pgfsetdash{}{0pt}%
\pgfpathmoveto{\pgfqpoint{3.061670in}{2.001789in}}%
\pgfpathcurveto{\pgfqpoint{3.069907in}{2.001789in}}{\pgfqpoint{3.077807in}{2.005062in}}{\pgfqpoint{3.083631in}{2.010886in}}%
\pgfpathcurveto{\pgfqpoint{3.089455in}{2.016709in}}{\pgfqpoint{3.092727in}{2.024609in}}{\pgfqpoint{3.092727in}{2.032846in}}%
\pgfpathcurveto{\pgfqpoint{3.092727in}{2.041082in}}{\pgfqpoint{3.089455in}{2.048982in}}{\pgfqpoint{3.083631in}{2.054806in}}%
\pgfpathcurveto{\pgfqpoint{3.077807in}{2.060630in}}{\pgfqpoint{3.069907in}{2.063902in}}{\pgfqpoint{3.061670in}{2.063902in}}%
\pgfpathcurveto{\pgfqpoint{3.053434in}{2.063902in}}{\pgfqpoint{3.045534in}{2.060630in}}{\pgfqpoint{3.039710in}{2.054806in}}%
\pgfpathcurveto{\pgfqpoint{3.033886in}{2.048982in}}{\pgfqpoint{3.030614in}{2.041082in}}{\pgfqpoint{3.030614in}{2.032846in}}%
\pgfpathcurveto{\pgfqpoint{3.030614in}{2.024609in}}{\pgfqpoint{3.033886in}{2.016709in}}{\pgfqpoint{3.039710in}{2.010886in}}%
\pgfpathcurveto{\pgfqpoint{3.045534in}{2.005062in}}{\pgfqpoint{3.053434in}{2.001789in}}{\pgfqpoint{3.061670in}{2.001789in}}%
\pgfpathclose%
\pgfusepath{stroke,fill}%
\end{pgfscope}%
\begin{pgfscope}%
\pgfpathrectangle{\pgfqpoint{0.100000in}{0.212622in}}{\pgfqpoint{3.696000in}{3.696000in}}%
\pgfusepath{clip}%
\pgfsetbuttcap%
\pgfsetroundjoin%
\definecolor{currentfill}{rgb}{0.121569,0.466667,0.705882}%
\pgfsetfillcolor{currentfill}%
\pgfsetfillopacity{0.532322}%
\pgfsetlinewidth{1.003750pt}%
\definecolor{currentstroke}{rgb}{0.121569,0.466667,0.705882}%
\pgfsetstrokecolor{currentstroke}%
\pgfsetstrokeopacity{0.532322}%
\pgfsetdash{}{0pt}%
\pgfpathmoveto{\pgfqpoint{1.099584in}{1.706897in}}%
\pgfpathcurveto{\pgfqpoint{1.107820in}{1.706897in}}{\pgfqpoint{1.115720in}{1.710169in}}{\pgfqpoint{1.121544in}{1.715993in}}%
\pgfpathcurveto{\pgfqpoint{1.127368in}{1.721817in}}{\pgfqpoint{1.130640in}{1.729717in}}{\pgfqpoint{1.130640in}{1.737953in}}%
\pgfpathcurveto{\pgfqpoint{1.130640in}{1.746189in}}{\pgfqpoint{1.127368in}{1.754089in}}{\pgfqpoint{1.121544in}{1.759913in}}%
\pgfpathcurveto{\pgfqpoint{1.115720in}{1.765737in}}{\pgfqpoint{1.107820in}{1.769010in}}{\pgfqpoint{1.099584in}{1.769010in}}%
\pgfpathcurveto{\pgfqpoint{1.091347in}{1.769010in}}{\pgfqpoint{1.083447in}{1.765737in}}{\pgfqpoint{1.077623in}{1.759913in}}%
\pgfpathcurveto{\pgfqpoint{1.071799in}{1.754089in}}{\pgfqpoint{1.068527in}{1.746189in}}{\pgfqpoint{1.068527in}{1.737953in}}%
\pgfpathcurveto{\pgfqpoint{1.068527in}{1.729717in}}{\pgfqpoint{1.071799in}{1.721817in}}{\pgfqpoint{1.077623in}{1.715993in}}%
\pgfpathcurveto{\pgfqpoint{1.083447in}{1.710169in}}{\pgfqpoint{1.091347in}{1.706897in}}{\pgfqpoint{1.099584in}{1.706897in}}%
\pgfpathclose%
\pgfusepath{stroke,fill}%
\end{pgfscope}%
\begin{pgfscope}%
\pgfpathrectangle{\pgfqpoint{0.100000in}{0.212622in}}{\pgfqpoint{3.696000in}{3.696000in}}%
\pgfusepath{clip}%
\pgfsetbuttcap%
\pgfsetroundjoin%
\definecolor{currentfill}{rgb}{0.121569,0.466667,0.705882}%
\pgfsetfillcolor{currentfill}%
\pgfsetfillopacity{0.533238}%
\pgfsetlinewidth{1.003750pt}%
\definecolor{currentstroke}{rgb}{0.121569,0.466667,0.705882}%
\pgfsetstrokecolor{currentstroke}%
\pgfsetstrokeopacity{0.533238}%
\pgfsetdash{}{0pt}%
\pgfpathmoveto{\pgfqpoint{3.068161in}{2.004916in}}%
\pgfpathcurveto{\pgfqpoint{3.076397in}{2.004916in}}{\pgfqpoint{3.084297in}{2.008188in}}{\pgfqpoint{3.090121in}{2.014012in}}%
\pgfpathcurveto{\pgfqpoint{3.095945in}{2.019836in}}{\pgfqpoint{3.099217in}{2.027736in}}{\pgfqpoint{3.099217in}{2.035972in}}%
\pgfpathcurveto{\pgfqpoint{3.099217in}{2.044209in}}{\pgfqpoint{3.095945in}{2.052109in}}{\pgfqpoint{3.090121in}{2.057933in}}%
\pgfpathcurveto{\pgfqpoint{3.084297in}{2.063757in}}{\pgfqpoint{3.076397in}{2.067029in}}{\pgfqpoint{3.068161in}{2.067029in}}%
\pgfpathcurveto{\pgfqpoint{3.059925in}{2.067029in}}{\pgfqpoint{3.052024in}{2.063757in}}{\pgfqpoint{3.046201in}{2.057933in}}%
\pgfpathcurveto{\pgfqpoint{3.040377in}{2.052109in}}{\pgfqpoint{3.037104in}{2.044209in}}{\pgfqpoint{3.037104in}{2.035972in}}%
\pgfpathcurveto{\pgfqpoint{3.037104in}{2.027736in}}{\pgfqpoint{3.040377in}{2.019836in}}{\pgfqpoint{3.046201in}{2.014012in}}%
\pgfpathcurveto{\pgfqpoint{3.052024in}{2.008188in}}{\pgfqpoint{3.059925in}{2.004916in}}{\pgfqpoint{3.068161in}{2.004916in}}%
\pgfpathclose%
\pgfusepath{stroke,fill}%
\end{pgfscope}%
\begin{pgfscope}%
\pgfpathrectangle{\pgfqpoint{0.100000in}{0.212622in}}{\pgfqpoint{3.696000in}{3.696000in}}%
\pgfusepath{clip}%
\pgfsetbuttcap%
\pgfsetroundjoin%
\definecolor{currentfill}{rgb}{0.121569,0.466667,0.705882}%
\pgfsetfillcolor{currentfill}%
\pgfsetfillopacity{0.533432}%
\pgfsetlinewidth{1.003750pt}%
\definecolor{currentstroke}{rgb}{0.121569,0.466667,0.705882}%
\pgfsetstrokecolor{currentstroke}%
\pgfsetstrokeopacity{0.533432}%
\pgfsetdash{}{0pt}%
\pgfpathmoveto{\pgfqpoint{1.094733in}{1.703460in}}%
\pgfpathcurveto{\pgfqpoint{1.102969in}{1.703460in}}{\pgfqpoint{1.110869in}{1.706732in}}{\pgfqpoint{1.116693in}{1.712556in}}%
\pgfpathcurveto{\pgfqpoint{1.122517in}{1.718380in}}{\pgfqpoint{1.125790in}{1.726280in}}{\pgfqpoint{1.125790in}{1.734517in}}%
\pgfpathcurveto{\pgfqpoint{1.125790in}{1.742753in}}{\pgfqpoint{1.122517in}{1.750653in}}{\pgfqpoint{1.116693in}{1.756477in}}%
\pgfpathcurveto{\pgfqpoint{1.110869in}{1.762301in}}{\pgfqpoint{1.102969in}{1.765573in}}{\pgfqpoint{1.094733in}{1.765573in}}%
\pgfpathcurveto{\pgfqpoint{1.086497in}{1.765573in}}{\pgfqpoint{1.078597in}{1.762301in}}{\pgfqpoint{1.072773in}{1.756477in}}%
\pgfpathcurveto{\pgfqpoint{1.066949in}{1.750653in}}{\pgfqpoint{1.063677in}{1.742753in}}{\pgfqpoint{1.063677in}{1.734517in}}%
\pgfpathcurveto{\pgfqpoint{1.063677in}{1.726280in}}{\pgfqpoint{1.066949in}{1.718380in}}{\pgfqpoint{1.072773in}{1.712556in}}%
\pgfpathcurveto{\pgfqpoint{1.078597in}{1.706732in}}{\pgfqpoint{1.086497in}{1.703460in}}{\pgfqpoint{1.094733in}{1.703460in}}%
\pgfpathclose%
\pgfusepath{stroke,fill}%
\end{pgfscope}%
\begin{pgfscope}%
\pgfpathrectangle{\pgfqpoint{0.100000in}{0.212622in}}{\pgfqpoint{3.696000in}{3.696000in}}%
\pgfusepath{clip}%
\pgfsetbuttcap%
\pgfsetroundjoin%
\definecolor{currentfill}{rgb}{0.121569,0.466667,0.705882}%
\pgfsetfillcolor{currentfill}%
\pgfsetfillopacity{0.534294}%
\pgfsetlinewidth{1.003750pt}%
\definecolor{currentstroke}{rgb}{0.121569,0.466667,0.705882}%
\pgfsetstrokecolor{currentstroke}%
\pgfsetstrokeopacity{0.534294}%
\pgfsetdash{}{0pt}%
\pgfpathmoveto{\pgfqpoint{3.074626in}{2.002903in}}%
\pgfpathcurveto{\pgfqpoint{3.082862in}{2.002903in}}{\pgfqpoint{3.090762in}{2.006175in}}{\pgfqpoint{3.096586in}{2.011999in}}%
\pgfpathcurveto{\pgfqpoint{3.102410in}{2.017823in}}{\pgfqpoint{3.105682in}{2.025723in}}{\pgfqpoint{3.105682in}{2.033959in}}%
\pgfpathcurveto{\pgfqpoint{3.105682in}{2.042196in}}{\pgfqpoint{3.102410in}{2.050096in}}{\pgfqpoint{3.096586in}{2.055919in}}%
\pgfpathcurveto{\pgfqpoint{3.090762in}{2.061743in}}{\pgfqpoint{3.082862in}{2.065016in}}{\pgfqpoint{3.074626in}{2.065016in}}%
\pgfpathcurveto{\pgfqpoint{3.066390in}{2.065016in}}{\pgfqpoint{3.058490in}{2.061743in}}{\pgfqpoint{3.052666in}{2.055919in}}%
\pgfpathcurveto{\pgfqpoint{3.046842in}{2.050096in}}{\pgfqpoint{3.043569in}{2.042196in}}{\pgfqpoint{3.043569in}{2.033959in}}%
\pgfpathcurveto{\pgfqpoint{3.043569in}{2.025723in}}{\pgfqpoint{3.046842in}{2.017823in}}{\pgfqpoint{3.052666in}{2.011999in}}%
\pgfpathcurveto{\pgfqpoint{3.058490in}{2.006175in}}{\pgfqpoint{3.066390in}{2.002903in}}{\pgfqpoint{3.074626in}{2.002903in}}%
\pgfpathclose%
\pgfusepath{stroke,fill}%
\end{pgfscope}%
\begin{pgfscope}%
\pgfpathrectangle{\pgfqpoint{0.100000in}{0.212622in}}{\pgfqpoint{3.696000in}{3.696000in}}%
\pgfusepath{clip}%
\pgfsetbuttcap%
\pgfsetroundjoin%
\definecolor{currentfill}{rgb}{0.121569,0.466667,0.705882}%
\pgfsetfillcolor{currentfill}%
\pgfsetfillopacity{0.534801}%
\pgfsetlinewidth{1.003750pt}%
\definecolor{currentstroke}{rgb}{0.121569,0.466667,0.705882}%
\pgfsetstrokecolor{currentstroke}%
\pgfsetstrokeopacity{0.534801}%
\pgfsetdash{}{0pt}%
\pgfpathmoveto{\pgfqpoint{1.093135in}{1.703064in}}%
\pgfpathcurveto{\pgfqpoint{1.101371in}{1.703064in}}{\pgfqpoint{1.109271in}{1.706336in}}{\pgfqpoint{1.115095in}{1.712160in}}%
\pgfpathcurveto{\pgfqpoint{1.120919in}{1.717984in}}{\pgfqpoint{1.124192in}{1.725884in}}{\pgfqpoint{1.124192in}{1.734120in}}%
\pgfpathcurveto{\pgfqpoint{1.124192in}{1.742356in}}{\pgfqpoint{1.120919in}{1.750256in}}{\pgfqpoint{1.115095in}{1.756080in}}%
\pgfpathcurveto{\pgfqpoint{1.109271in}{1.761904in}}{\pgfqpoint{1.101371in}{1.765177in}}{\pgfqpoint{1.093135in}{1.765177in}}%
\pgfpathcurveto{\pgfqpoint{1.084899in}{1.765177in}}{\pgfqpoint{1.076999in}{1.761904in}}{\pgfqpoint{1.071175in}{1.756080in}}%
\pgfpathcurveto{\pgfqpoint{1.065351in}{1.750256in}}{\pgfqpoint{1.062079in}{1.742356in}}{\pgfqpoint{1.062079in}{1.734120in}}%
\pgfpathcurveto{\pgfqpoint{1.062079in}{1.725884in}}{\pgfqpoint{1.065351in}{1.717984in}}{\pgfqpoint{1.071175in}{1.712160in}}%
\pgfpathcurveto{\pgfqpoint{1.076999in}{1.706336in}}{\pgfqpoint{1.084899in}{1.703064in}}{\pgfqpoint{1.093135in}{1.703064in}}%
\pgfpathclose%
\pgfusepath{stroke,fill}%
\end{pgfscope}%
\begin{pgfscope}%
\pgfpathrectangle{\pgfqpoint{0.100000in}{0.212622in}}{\pgfqpoint{3.696000in}{3.696000in}}%
\pgfusepath{clip}%
\pgfsetbuttcap%
\pgfsetroundjoin%
\definecolor{currentfill}{rgb}{0.121569,0.466667,0.705882}%
\pgfsetfillcolor{currentfill}%
\pgfsetfillopacity{0.535550}%
\pgfsetlinewidth{1.003750pt}%
\definecolor{currentstroke}{rgb}{0.121569,0.466667,0.705882}%
\pgfsetstrokecolor{currentstroke}%
\pgfsetstrokeopacity{0.535550}%
\pgfsetdash{}{0pt}%
\pgfpathmoveto{\pgfqpoint{1.089898in}{1.702184in}}%
\pgfpathcurveto{\pgfqpoint{1.098134in}{1.702184in}}{\pgfqpoint{1.106034in}{1.705456in}}{\pgfqpoint{1.111858in}{1.711280in}}%
\pgfpathcurveto{\pgfqpoint{1.117682in}{1.717104in}}{\pgfqpoint{1.120955in}{1.725004in}}{\pgfqpoint{1.120955in}{1.733241in}}%
\pgfpathcurveto{\pgfqpoint{1.120955in}{1.741477in}}{\pgfqpoint{1.117682in}{1.749377in}}{\pgfqpoint{1.111858in}{1.755201in}}%
\pgfpathcurveto{\pgfqpoint{1.106034in}{1.761025in}}{\pgfqpoint{1.098134in}{1.764297in}}{\pgfqpoint{1.089898in}{1.764297in}}%
\pgfpathcurveto{\pgfqpoint{1.081662in}{1.764297in}}{\pgfqpoint{1.073762in}{1.761025in}}{\pgfqpoint{1.067938in}{1.755201in}}%
\pgfpathcurveto{\pgfqpoint{1.062114in}{1.749377in}}{\pgfqpoint{1.058842in}{1.741477in}}{\pgfqpoint{1.058842in}{1.733241in}}%
\pgfpathcurveto{\pgfqpoint{1.058842in}{1.725004in}}{\pgfqpoint{1.062114in}{1.717104in}}{\pgfqpoint{1.067938in}{1.711280in}}%
\pgfpathcurveto{\pgfqpoint{1.073762in}{1.705456in}}{\pgfqpoint{1.081662in}{1.702184in}}{\pgfqpoint{1.089898in}{1.702184in}}%
\pgfpathclose%
\pgfusepath{stroke,fill}%
\end{pgfscope}%
\begin{pgfscope}%
\pgfpathrectangle{\pgfqpoint{0.100000in}{0.212622in}}{\pgfqpoint{3.696000in}{3.696000in}}%
\pgfusepath{clip}%
\pgfsetbuttcap%
\pgfsetroundjoin%
\definecolor{currentfill}{rgb}{0.121569,0.466667,0.705882}%
\pgfsetfillcolor{currentfill}%
\pgfsetfillopacity{0.536144}%
\pgfsetlinewidth{1.003750pt}%
\definecolor{currentstroke}{rgb}{0.121569,0.466667,0.705882}%
\pgfsetstrokecolor{currentstroke}%
\pgfsetstrokeopacity{0.536144}%
\pgfsetdash{}{0pt}%
\pgfpathmoveto{\pgfqpoint{3.082572in}{2.007087in}}%
\pgfpathcurveto{\pgfqpoint{3.090808in}{2.007087in}}{\pgfqpoint{3.098708in}{2.010359in}}{\pgfqpoint{3.104532in}{2.016183in}}%
\pgfpathcurveto{\pgfqpoint{3.110356in}{2.022007in}}{\pgfqpoint{3.113628in}{2.029907in}}{\pgfqpoint{3.113628in}{2.038144in}}%
\pgfpathcurveto{\pgfqpoint{3.113628in}{2.046380in}}{\pgfqpoint{3.110356in}{2.054280in}}{\pgfqpoint{3.104532in}{2.060104in}}%
\pgfpathcurveto{\pgfqpoint{3.098708in}{2.065928in}}{\pgfqpoint{3.090808in}{2.069200in}}{\pgfqpoint{3.082572in}{2.069200in}}%
\pgfpathcurveto{\pgfqpoint{3.074335in}{2.069200in}}{\pgfqpoint{3.066435in}{2.065928in}}{\pgfqpoint{3.060612in}{2.060104in}}%
\pgfpathcurveto{\pgfqpoint{3.054788in}{2.054280in}}{\pgfqpoint{3.051515in}{2.046380in}}{\pgfqpoint{3.051515in}{2.038144in}}%
\pgfpathcurveto{\pgfqpoint{3.051515in}{2.029907in}}{\pgfqpoint{3.054788in}{2.022007in}}{\pgfqpoint{3.060612in}{2.016183in}}%
\pgfpathcurveto{\pgfqpoint{3.066435in}{2.010359in}}{\pgfqpoint{3.074335in}{2.007087in}}{\pgfqpoint{3.082572in}{2.007087in}}%
\pgfpathclose%
\pgfusepath{stroke,fill}%
\end{pgfscope}%
\begin{pgfscope}%
\pgfpathrectangle{\pgfqpoint{0.100000in}{0.212622in}}{\pgfqpoint{3.696000in}{3.696000in}}%
\pgfusepath{clip}%
\pgfsetbuttcap%
\pgfsetroundjoin%
\definecolor{currentfill}{rgb}{0.121569,0.466667,0.705882}%
\pgfsetfillcolor{currentfill}%
\pgfsetfillopacity{0.536349}%
\pgfsetlinewidth{1.003750pt}%
\definecolor{currentstroke}{rgb}{0.121569,0.466667,0.705882}%
\pgfsetstrokecolor{currentstroke}%
\pgfsetstrokeopacity{0.536349}%
\pgfsetdash{}{0pt}%
\pgfpathmoveto{\pgfqpoint{1.089036in}{1.702524in}}%
\pgfpathcurveto{\pgfqpoint{1.097272in}{1.702524in}}{\pgfqpoint{1.105172in}{1.705797in}}{\pgfqpoint{1.110996in}{1.711621in}}%
\pgfpathcurveto{\pgfqpoint{1.116820in}{1.717444in}}{\pgfqpoint{1.120093in}{1.725345in}}{\pgfqpoint{1.120093in}{1.733581in}}%
\pgfpathcurveto{\pgfqpoint{1.120093in}{1.741817in}}{\pgfqpoint{1.116820in}{1.749717in}}{\pgfqpoint{1.110996in}{1.755541in}}%
\pgfpathcurveto{\pgfqpoint{1.105172in}{1.761365in}}{\pgfqpoint{1.097272in}{1.764637in}}{\pgfqpoint{1.089036in}{1.764637in}}%
\pgfpathcurveto{\pgfqpoint{1.080800in}{1.764637in}}{\pgfqpoint{1.072900in}{1.761365in}}{\pgfqpoint{1.067076in}{1.755541in}}%
\pgfpathcurveto{\pgfqpoint{1.061252in}{1.749717in}}{\pgfqpoint{1.057980in}{1.741817in}}{\pgfqpoint{1.057980in}{1.733581in}}%
\pgfpathcurveto{\pgfqpoint{1.057980in}{1.725345in}}{\pgfqpoint{1.061252in}{1.717444in}}{\pgfqpoint{1.067076in}{1.711621in}}%
\pgfpathcurveto{\pgfqpoint{1.072900in}{1.705797in}}{\pgfqpoint{1.080800in}{1.702524in}}{\pgfqpoint{1.089036in}{1.702524in}}%
\pgfpathclose%
\pgfusepath{stroke,fill}%
\end{pgfscope}%
\begin{pgfscope}%
\pgfpathrectangle{\pgfqpoint{0.100000in}{0.212622in}}{\pgfqpoint{3.696000in}{3.696000in}}%
\pgfusepath{clip}%
\pgfsetbuttcap%
\pgfsetroundjoin%
\definecolor{currentfill}{rgb}{0.121569,0.466667,0.705882}%
\pgfsetfillcolor{currentfill}%
\pgfsetfillopacity{0.536597}%
\pgfsetlinewidth{1.003750pt}%
\definecolor{currentstroke}{rgb}{0.121569,0.466667,0.705882}%
\pgfsetstrokecolor{currentstroke}%
\pgfsetstrokeopacity{0.536597}%
\pgfsetdash{}{0pt}%
\pgfpathmoveto{\pgfqpoint{1.087802in}{1.700367in}}%
\pgfpathcurveto{\pgfqpoint{1.096038in}{1.700367in}}{\pgfqpoint{1.103938in}{1.703639in}}{\pgfqpoint{1.109762in}{1.709463in}}%
\pgfpathcurveto{\pgfqpoint{1.115586in}{1.715287in}}{\pgfqpoint{1.118859in}{1.723187in}}{\pgfqpoint{1.118859in}{1.731423in}}%
\pgfpathcurveto{\pgfqpoint{1.118859in}{1.739660in}}{\pgfqpoint{1.115586in}{1.747560in}}{\pgfqpoint{1.109762in}{1.753384in}}%
\pgfpathcurveto{\pgfqpoint{1.103938in}{1.759207in}}{\pgfqpoint{1.096038in}{1.762480in}}{\pgfqpoint{1.087802in}{1.762480in}}%
\pgfpathcurveto{\pgfqpoint{1.079566in}{1.762480in}}{\pgfqpoint{1.071666in}{1.759207in}}{\pgfqpoint{1.065842in}{1.753384in}}%
\pgfpathcurveto{\pgfqpoint{1.060018in}{1.747560in}}{\pgfqpoint{1.056746in}{1.739660in}}{\pgfqpoint{1.056746in}{1.731423in}}%
\pgfpathcurveto{\pgfqpoint{1.056746in}{1.723187in}}{\pgfqpoint{1.060018in}{1.715287in}}{\pgfqpoint{1.065842in}{1.709463in}}%
\pgfpathcurveto{\pgfqpoint{1.071666in}{1.703639in}}{\pgfqpoint{1.079566in}{1.700367in}}{\pgfqpoint{1.087802in}{1.700367in}}%
\pgfpathclose%
\pgfusepath{stroke,fill}%
\end{pgfscope}%
\begin{pgfscope}%
\pgfpathrectangle{\pgfqpoint{0.100000in}{0.212622in}}{\pgfqpoint{3.696000in}{3.696000in}}%
\pgfusepath{clip}%
\pgfsetbuttcap%
\pgfsetroundjoin%
\definecolor{currentfill}{rgb}{0.121569,0.466667,0.705882}%
\pgfsetfillcolor{currentfill}%
\pgfsetfillopacity{0.537012}%
\pgfsetlinewidth{1.003750pt}%
\definecolor{currentstroke}{rgb}{0.121569,0.466667,0.705882}%
\pgfsetstrokecolor{currentstroke}%
\pgfsetstrokeopacity{0.537012}%
\pgfsetdash{}{0pt}%
\pgfpathmoveto{\pgfqpoint{1.084841in}{1.697301in}}%
\pgfpathcurveto{\pgfqpoint{1.093078in}{1.697301in}}{\pgfqpoint{1.100978in}{1.700573in}}{\pgfqpoint{1.106802in}{1.706397in}}%
\pgfpathcurveto{\pgfqpoint{1.112626in}{1.712221in}}{\pgfqpoint{1.115898in}{1.720121in}}{\pgfqpoint{1.115898in}{1.728357in}}%
\pgfpathcurveto{\pgfqpoint{1.115898in}{1.736593in}}{\pgfqpoint{1.112626in}{1.744494in}}{\pgfqpoint{1.106802in}{1.750317in}}%
\pgfpathcurveto{\pgfqpoint{1.100978in}{1.756141in}}{\pgfqpoint{1.093078in}{1.759414in}}{\pgfqpoint{1.084841in}{1.759414in}}%
\pgfpathcurveto{\pgfqpoint{1.076605in}{1.759414in}}{\pgfqpoint{1.068705in}{1.756141in}}{\pgfqpoint{1.062881in}{1.750317in}}%
\pgfpathcurveto{\pgfqpoint{1.057057in}{1.744494in}}{\pgfqpoint{1.053785in}{1.736593in}}{\pgfqpoint{1.053785in}{1.728357in}}%
\pgfpathcurveto{\pgfqpoint{1.053785in}{1.720121in}}{\pgfqpoint{1.057057in}{1.712221in}}{\pgfqpoint{1.062881in}{1.706397in}}%
\pgfpathcurveto{\pgfqpoint{1.068705in}{1.700573in}}{\pgfqpoint{1.076605in}{1.697301in}}{\pgfqpoint{1.084841in}{1.697301in}}%
\pgfpathclose%
\pgfusepath{stroke,fill}%
\end{pgfscope}%
\begin{pgfscope}%
\pgfpathrectangle{\pgfqpoint{0.100000in}{0.212622in}}{\pgfqpoint{3.696000in}{3.696000in}}%
\pgfusepath{clip}%
\pgfsetbuttcap%
\pgfsetroundjoin%
\definecolor{currentfill}{rgb}{0.121569,0.466667,0.705882}%
\pgfsetfillcolor{currentfill}%
\pgfsetfillopacity{0.537206}%
\pgfsetlinewidth{1.003750pt}%
\definecolor{currentstroke}{rgb}{0.121569,0.466667,0.705882}%
\pgfsetstrokecolor{currentstroke}%
\pgfsetstrokeopacity{0.537206}%
\pgfsetdash{}{0pt}%
\pgfpathmoveto{\pgfqpoint{3.090716in}{2.003287in}}%
\pgfpathcurveto{\pgfqpoint{3.098952in}{2.003287in}}{\pgfqpoint{3.106852in}{2.006559in}}{\pgfqpoint{3.112676in}{2.012383in}}%
\pgfpathcurveto{\pgfqpoint{3.118500in}{2.018207in}}{\pgfqpoint{3.121772in}{2.026107in}}{\pgfqpoint{3.121772in}{2.034343in}}%
\pgfpathcurveto{\pgfqpoint{3.121772in}{2.042580in}}{\pgfqpoint{3.118500in}{2.050480in}}{\pgfqpoint{3.112676in}{2.056304in}}%
\pgfpathcurveto{\pgfqpoint{3.106852in}{2.062127in}}{\pgfqpoint{3.098952in}{2.065400in}}{\pgfqpoint{3.090716in}{2.065400in}}%
\pgfpathcurveto{\pgfqpoint{3.082479in}{2.065400in}}{\pgfqpoint{3.074579in}{2.062127in}}{\pgfqpoint{3.068755in}{2.056304in}}%
\pgfpathcurveto{\pgfqpoint{3.062931in}{2.050480in}}{\pgfqpoint{3.059659in}{2.042580in}}{\pgfqpoint{3.059659in}{2.034343in}}%
\pgfpathcurveto{\pgfqpoint{3.059659in}{2.026107in}}{\pgfqpoint{3.062931in}{2.018207in}}{\pgfqpoint{3.068755in}{2.012383in}}%
\pgfpathcurveto{\pgfqpoint{3.074579in}{2.006559in}}{\pgfqpoint{3.082479in}{2.003287in}}{\pgfqpoint{3.090716in}{2.003287in}}%
\pgfpathclose%
\pgfusepath{stroke,fill}%
\end{pgfscope}%
\begin{pgfscope}%
\pgfpathrectangle{\pgfqpoint{0.100000in}{0.212622in}}{\pgfqpoint{3.696000in}{3.696000in}}%
\pgfusepath{clip}%
\pgfsetbuttcap%
\pgfsetroundjoin%
\definecolor{currentfill}{rgb}{0.121569,0.466667,0.705882}%
\pgfsetfillcolor{currentfill}%
\pgfsetfillopacity{0.537880}%
\pgfsetlinewidth{1.003750pt}%
\definecolor{currentstroke}{rgb}{0.121569,0.466667,0.705882}%
\pgfsetstrokecolor{currentstroke}%
\pgfsetstrokeopacity{0.537880}%
\pgfsetdash{}{0pt}%
\pgfpathmoveto{\pgfqpoint{1.083469in}{1.696981in}}%
\pgfpathcurveto{\pgfqpoint{1.091705in}{1.696981in}}{\pgfqpoint{1.099605in}{1.700253in}}{\pgfqpoint{1.105429in}{1.706077in}}%
\pgfpathcurveto{\pgfqpoint{1.111253in}{1.711901in}}{\pgfqpoint{1.114525in}{1.719801in}}{\pgfqpoint{1.114525in}{1.728037in}}%
\pgfpathcurveto{\pgfqpoint{1.114525in}{1.736273in}}{\pgfqpoint{1.111253in}{1.744173in}}{\pgfqpoint{1.105429in}{1.749997in}}%
\pgfpathcurveto{\pgfqpoint{1.099605in}{1.755821in}}{\pgfqpoint{1.091705in}{1.759094in}}{\pgfqpoint{1.083469in}{1.759094in}}%
\pgfpathcurveto{\pgfqpoint{1.075232in}{1.759094in}}{\pgfqpoint{1.067332in}{1.755821in}}{\pgfqpoint{1.061508in}{1.749997in}}%
\pgfpathcurveto{\pgfqpoint{1.055684in}{1.744173in}}{\pgfqpoint{1.052412in}{1.736273in}}{\pgfqpoint{1.052412in}{1.728037in}}%
\pgfpathcurveto{\pgfqpoint{1.052412in}{1.719801in}}{\pgfqpoint{1.055684in}{1.711901in}}{\pgfqpoint{1.061508in}{1.706077in}}%
\pgfpathcurveto{\pgfqpoint{1.067332in}{1.700253in}}{\pgfqpoint{1.075232in}{1.696981in}}{\pgfqpoint{1.083469in}{1.696981in}}%
\pgfpathclose%
\pgfusepath{stroke,fill}%
\end{pgfscope}%
\begin{pgfscope}%
\pgfpathrectangle{\pgfqpoint{0.100000in}{0.212622in}}{\pgfqpoint{3.696000in}{3.696000in}}%
\pgfusepath{clip}%
\pgfsetbuttcap%
\pgfsetroundjoin%
\definecolor{currentfill}{rgb}{0.121569,0.466667,0.705882}%
\pgfsetfillcolor{currentfill}%
\pgfsetfillopacity{0.538247}%
\pgfsetlinewidth{1.003750pt}%
\definecolor{currentstroke}{rgb}{0.121569,0.466667,0.705882}%
\pgfsetstrokecolor{currentstroke}%
\pgfsetstrokeopacity{0.538247}%
\pgfsetdash{}{0pt}%
\pgfpathmoveto{\pgfqpoint{3.095843in}{2.006347in}}%
\pgfpathcurveto{\pgfqpoint{3.104079in}{2.006347in}}{\pgfqpoint{3.111979in}{2.009619in}}{\pgfqpoint{3.117803in}{2.015443in}}%
\pgfpathcurveto{\pgfqpoint{3.123627in}{2.021267in}}{\pgfqpoint{3.126900in}{2.029167in}}{\pgfqpoint{3.126900in}{2.037404in}}%
\pgfpathcurveto{\pgfqpoint{3.126900in}{2.045640in}}{\pgfqpoint{3.123627in}{2.053540in}}{\pgfqpoint{3.117803in}{2.059364in}}%
\pgfpathcurveto{\pgfqpoint{3.111979in}{2.065188in}}{\pgfqpoint{3.104079in}{2.068460in}}{\pgfqpoint{3.095843in}{2.068460in}}%
\pgfpathcurveto{\pgfqpoint{3.087607in}{2.068460in}}{\pgfqpoint{3.079707in}{2.065188in}}{\pgfqpoint{3.073883in}{2.059364in}}%
\pgfpathcurveto{\pgfqpoint{3.068059in}{2.053540in}}{\pgfqpoint{3.064787in}{2.045640in}}{\pgfqpoint{3.064787in}{2.037404in}}%
\pgfpathcurveto{\pgfqpoint{3.064787in}{2.029167in}}{\pgfqpoint{3.068059in}{2.021267in}}{\pgfqpoint{3.073883in}{2.015443in}}%
\pgfpathcurveto{\pgfqpoint{3.079707in}{2.009619in}}{\pgfqpoint{3.087607in}{2.006347in}}{\pgfqpoint{3.095843in}{2.006347in}}%
\pgfpathclose%
\pgfusepath{stroke,fill}%
\end{pgfscope}%
\begin{pgfscope}%
\pgfpathrectangle{\pgfqpoint{0.100000in}{0.212622in}}{\pgfqpoint{3.696000in}{3.696000in}}%
\pgfusepath{clip}%
\pgfsetbuttcap%
\pgfsetroundjoin%
\definecolor{currentfill}{rgb}{0.121569,0.466667,0.705882}%
\pgfsetfillcolor{currentfill}%
\pgfsetfillopacity{0.538537}%
\pgfsetlinewidth{1.003750pt}%
\definecolor{currentstroke}{rgb}{0.121569,0.466667,0.705882}%
\pgfsetstrokecolor{currentstroke}%
\pgfsetstrokeopacity{0.538537}%
\pgfsetdash{}{0pt}%
\pgfpathmoveto{\pgfqpoint{3.101096in}{2.000611in}}%
\pgfpathcurveto{\pgfqpoint{3.109333in}{2.000611in}}{\pgfqpoint{3.117233in}{2.003884in}}{\pgfqpoint{3.123057in}{2.009708in}}%
\pgfpathcurveto{\pgfqpoint{3.128880in}{2.015531in}}{\pgfqpoint{3.132153in}{2.023431in}}{\pgfqpoint{3.132153in}{2.031668in}}%
\pgfpathcurveto{\pgfqpoint{3.132153in}{2.039904in}}{\pgfqpoint{3.128880in}{2.047804in}}{\pgfqpoint{3.123057in}{2.053628in}}%
\pgfpathcurveto{\pgfqpoint{3.117233in}{2.059452in}}{\pgfqpoint{3.109333in}{2.062724in}}{\pgfqpoint{3.101096in}{2.062724in}}%
\pgfpathcurveto{\pgfqpoint{3.092860in}{2.062724in}}{\pgfqpoint{3.084960in}{2.059452in}}{\pgfqpoint{3.079136in}{2.053628in}}%
\pgfpathcurveto{\pgfqpoint{3.073312in}{2.047804in}}{\pgfqpoint{3.070040in}{2.039904in}}{\pgfqpoint{3.070040in}{2.031668in}}%
\pgfpathcurveto{\pgfqpoint{3.070040in}{2.023431in}}{\pgfqpoint{3.073312in}{2.015531in}}{\pgfqpoint{3.079136in}{2.009708in}}%
\pgfpathcurveto{\pgfqpoint{3.084960in}{2.003884in}}{\pgfqpoint{3.092860in}{2.000611in}}{\pgfqpoint{3.101096in}{2.000611in}}%
\pgfpathclose%
\pgfusepath{stroke,fill}%
\end{pgfscope}%
\begin{pgfscope}%
\pgfpathrectangle{\pgfqpoint{0.100000in}{0.212622in}}{\pgfqpoint{3.696000in}{3.696000in}}%
\pgfusepath{clip}%
\pgfsetbuttcap%
\pgfsetroundjoin%
\definecolor{currentfill}{rgb}{0.121569,0.466667,0.705882}%
\pgfsetfillcolor{currentfill}%
\pgfsetfillopacity{0.538725}%
\pgfsetlinewidth{1.003750pt}%
\definecolor{currentstroke}{rgb}{0.121569,0.466667,0.705882}%
\pgfsetstrokecolor{currentstroke}%
\pgfsetstrokeopacity{0.538725}%
\pgfsetdash{}{0pt}%
\pgfpathmoveto{\pgfqpoint{1.078807in}{1.694184in}}%
\pgfpathcurveto{\pgfqpoint{1.087044in}{1.694184in}}{\pgfqpoint{1.094944in}{1.697456in}}{\pgfqpoint{1.100768in}{1.703280in}}%
\pgfpathcurveto{\pgfqpoint{1.106591in}{1.709104in}}{\pgfqpoint{1.109864in}{1.717004in}}{\pgfqpoint{1.109864in}{1.725240in}}%
\pgfpathcurveto{\pgfqpoint{1.109864in}{1.733477in}}{\pgfqpoint{1.106591in}{1.741377in}}{\pgfqpoint{1.100768in}{1.747201in}}%
\pgfpathcurveto{\pgfqpoint{1.094944in}{1.753025in}}{\pgfqpoint{1.087044in}{1.756297in}}{\pgfqpoint{1.078807in}{1.756297in}}%
\pgfpathcurveto{\pgfqpoint{1.070571in}{1.756297in}}{\pgfqpoint{1.062671in}{1.753025in}}{\pgfqpoint{1.056847in}{1.747201in}}%
\pgfpathcurveto{\pgfqpoint{1.051023in}{1.741377in}}{\pgfqpoint{1.047751in}{1.733477in}}{\pgfqpoint{1.047751in}{1.725240in}}%
\pgfpathcurveto{\pgfqpoint{1.047751in}{1.717004in}}{\pgfqpoint{1.051023in}{1.709104in}}{\pgfqpoint{1.056847in}{1.703280in}}%
\pgfpathcurveto{\pgfqpoint{1.062671in}{1.697456in}}{\pgfqpoint{1.070571in}{1.694184in}}{\pgfqpoint{1.078807in}{1.694184in}}%
\pgfpathclose%
\pgfusepath{stroke,fill}%
\end{pgfscope}%
\begin{pgfscope}%
\pgfpathrectangle{\pgfqpoint{0.100000in}{0.212622in}}{\pgfqpoint{3.696000in}{3.696000in}}%
\pgfusepath{clip}%
\pgfsetbuttcap%
\pgfsetroundjoin%
\definecolor{currentfill}{rgb}{0.121569,0.466667,0.705882}%
\pgfsetfillcolor{currentfill}%
\pgfsetfillopacity{0.540305}%
\pgfsetlinewidth{1.003750pt}%
\definecolor{currentstroke}{rgb}{0.121569,0.466667,0.705882}%
\pgfsetstrokecolor{currentstroke}%
\pgfsetstrokeopacity{0.540305}%
\pgfsetdash{}{0pt}%
\pgfpathmoveto{\pgfqpoint{3.107723in}{2.005777in}}%
\pgfpathcurveto{\pgfqpoint{3.115959in}{2.005777in}}{\pgfqpoint{3.123859in}{2.009049in}}{\pgfqpoint{3.129683in}{2.014873in}}%
\pgfpathcurveto{\pgfqpoint{3.135507in}{2.020697in}}{\pgfqpoint{3.138779in}{2.028597in}}{\pgfqpoint{3.138779in}{2.036833in}}%
\pgfpathcurveto{\pgfqpoint{3.138779in}{2.045070in}}{\pgfqpoint{3.135507in}{2.052970in}}{\pgfqpoint{3.129683in}{2.058794in}}%
\pgfpathcurveto{\pgfqpoint{3.123859in}{2.064618in}}{\pgfqpoint{3.115959in}{2.067890in}}{\pgfqpoint{3.107723in}{2.067890in}}%
\pgfpathcurveto{\pgfqpoint{3.099486in}{2.067890in}}{\pgfqpoint{3.091586in}{2.064618in}}{\pgfqpoint{3.085762in}{2.058794in}}%
\pgfpathcurveto{\pgfqpoint{3.079938in}{2.052970in}}{\pgfqpoint{3.076666in}{2.045070in}}{\pgfqpoint{3.076666in}{2.036833in}}%
\pgfpathcurveto{\pgfqpoint{3.076666in}{2.028597in}}{\pgfqpoint{3.079938in}{2.020697in}}{\pgfqpoint{3.085762in}{2.014873in}}%
\pgfpathcurveto{\pgfqpoint{3.091586in}{2.009049in}}{\pgfqpoint{3.099486in}{2.005777in}}{\pgfqpoint{3.107723in}{2.005777in}}%
\pgfpathclose%
\pgfusepath{stroke,fill}%
\end{pgfscope}%
\begin{pgfscope}%
\pgfpathrectangle{\pgfqpoint{0.100000in}{0.212622in}}{\pgfqpoint{3.696000in}{3.696000in}}%
\pgfusepath{clip}%
\pgfsetbuttcap%
\pgfsetroundjoin%
\definecolor{currentfill}{rgb}{0.121569,0.466667,0.705882}%
\pgfsetfillcolor{currentfill}%
\pgfsetfillopacity{0.540339}%
\pgfsetlinewidth{1.003750pt}%
\definecolor{currentstroke}{rgb}{0.121569,0.466667,0.705882}%
\pgfsetstrokecolor{currentstroke}%
\pgfsetstrokeopacity{0.540339}%
\pgfsetdash{}{0pt}%
\pgfpathmoveto{\pgfqpoint{1.076151in}{1.696098in}}%
\pgfpathcurveto{\pgfqpoint{1.084387in}{1.696098in}}{\pgfqpoint{1.092287in}{1.699370in}}{\pgfqpoint{1.098111in}{1.705194in}}%
\pgfpathcurveto{\pgfqpoint{1.103935in}{1.711018in}}{\pgfqpoint{1.107207in}{1.718918in}}{\pgfqpoint{1.107207in}{1.727154in}}%
\pgfpathcurveto{\pgfqpoint{1.107207in}{1.735391in}}{\pgfqpoint{1.103935in}{1.743291in}}{\pgfqpoint{1.098111in}{1.749115in}}%
\pgfpathcurveto{\pgfqpoint{1.092287in}{1.754938in}}{\pgfqpoint{1.084387in}{1.758211in}}{\pgfqpoint{1.076151in}{1.758211in}}%
\pgfpathcurveto{\pgfqpoint{1.067915in}{1.758211in}}{\pgfqpoint{1.060015in}{1.754938in}}{\pgfqpoint{1.054191in}{1.749115in}}%
\pgfpathcurveto{\pgfqpoint{1.048367in}{1.743291in}}{\pgfqpoint{1.045094in}{1.735391in}}{\pgfqpoint{1.045094in}{1.727154in}}%
\pgfpathcurveto{\pgfqpoint{1.045094in}{1.718918in}}{\pgfqpoint{1.048367in}{1.711018in}}{\pgfqpoint{1.054191in}{1.705194in}}%
\pgfpathcurveto{\pgfqpoint{1.060015in}{1.699370in}}{\pgfqpoint{1.067915in}{1.696098in}}{\pgfqpoint{1.076151in}{1.696098in}}%
\pgfpathclose%
\pgfusepath{stroke,fill}%
\end{pgfscope}%
\begin{pgfscope}%
\pgfpathrectangle{\pgfqpoint{0.100000in}{0.212622in}}{\pgfqpoint{3.696000in}{3.696000in}}%
\pgfusepath{clip}%
\pgfsetbuttcap%
\pgfsetroundjoin%
\definecolor{currentfill}{rgb}{0.121569,0.466667,0.705882}%
\pgfsetfillcolor{currentfill}%
\pgfsetfillopacity{0.540842}%
\pgfsetlinewidth{1.003750pt}%
\definecolor{currentstroke}{rgb}{0.121569,0.466667,0.705882}%
\pgfsetstrokecolor{currentstroke}%
\pgfsetstrokeopacity{0.540842}%
\pgfsetdash{}{0pt}%
\pgfpathmoveto{\pgfqpoint{3.110680in}{2.003967in}}%
\pgfpathcurveto{\pgfqpoint{3.118916in}{2.003967in}}{\pgfqpoint{3.126816in}{2.007239in}}{\pgfqpoint{3.132640in}{2.013063in}}%
\pgfpathcurveto{\pgfqpoint{3.138464in}{2.018887in}}{\pgfqpoint{3.141736in}{2.026787in}}{\pgfqpoint{3.141736in}{2.035023in}}%
\pgfpathcurveto{\pgfqpoint{3.141736in}{2.043260in}}{\pgfqpoint{3.138464in}{2.051160in}}{\pgfqpoint{3.132640in}{2.056984in}}%
\pgfpathcurveto{\pgfqpoint{3.126816in}{2.062808in}}{\pgfqpoint{3.118916in}{2.066080in}}{\pgfqpoint{3.110680in}{2.066080in}}%
\pgfpathcurveto{\pgfqpoint{3.102443in}{2.066080in}}{\pgfqpoint{3.094543in}{2.062808in}}{\pgfqpoint{3.088719in}{2.056984in}}%
\pgfpathcurveto{\pgfqpoint{3.082895in}{2.051160in}}{\pgfqpoint{3.079623in}{2.043260in}}{\pgfqpoint{3.079623in}{2.035023in}}%
\pgfpathcurveto{\pgfqpoint{3.079623in}{2.026787in}}{\pgfqpoint{3.082895in}{2.018887in}}{\pgfqpoint{3.088719in}{2.013063in}}%
\pgfpathcurveto{\pgfqpoint{3.094543in}{2.007239in}}{\pgfqpoint{3.102443in}{2.003967in}}{\pgfqpoint{3.110680in}{2.003967in}}%
\pgfpathclose%
\pgfusepath{stroke,fill}%
\end{pgfscope}%
\begin{pgfscope}%
\pgfpathrectangle{\pgfqpoint{0.100000in}{0.212622in}}{\pgfqpoint{3.696000in}{3.696000in}}%
\pgfusepath{clip}%
\pgfsetbuttcap%
\pgfsetroundjoin%
\definecolor{currentfill}{rgb}{0.121569,0.466667,0.705882}%
\pgfsetfillcolor{currentfill}%
\pgfsetfillopacity{0.540914}%
\pgfsetlinewidth{1.003750pt}%
\definecolor{currentstroke}{rgb}{0.121569,0.466667,0.705882}%
\pgfsetstrokecolor{currentstroke}%
\pgfsetstrokeopacity{0.540914}%
\pgfsetdash{}{0pt}%
\pgfpathmoveto{\pgfqpoint{1.071641in}{1.695973in}}%
\pgfpathcurveto{\pgfqpoint{1.079877in}{1.695973in}}{\pgfqpoint{1.087777in}{1.699245in}}{\pgfqpoint{1.093601in}{1.705069in}}%
\pgfpathcurveto{\pgfqpoint{1.099425in}{1.710893in}}{\pgfqpoint{1.102697in}{1.718793in}}{\pgfqpoint{1.102697in}{1.727030in}}%
\pgfpathcurveto{\pgfqpoint{1.102697in}{1.735266in}}{\pgfqpoint{1.099425in}{1.743166in}}{\pgfqpoint{1.093601in}{1.748990in}}%
\pgfpathcurveto{\pgfqpoint{1.087777in}{1.754814in}}{\pgfqpoint{1.079877in}{1.758086in}}{\pgfqpoint{1.071641in}{1.758086in}}%
\pgfpathcurveto{\pgfqpoint{1.063404in}{1.758086in}}{\pgfqpoint{1.055504in}{1.754814in}}{\pgfqpoint{1.049680in}{1.748990in}}%
\pgfpathcurveto{\pgfqpoint{1.043856in}{1.743166in}}{\pgfqpoint{1.040584in}{1.735266in}}{\pgfqpoint{1.040584in}{1.727030in}}%
\pgfpathcurveto{\pgfqpoint{1.040584in}{1.718793in}}{\pgfqpoint{1.043856in}{1.710893in}}{\pgfqpoint{1.049680in}{1.705069in}}%
\pgfpathcurveto{\pgfqpoint{1.055504in}{1.699245in}}{\pgfqpoint{1.063404in}{1.695973in}}{\pgfqpoint{1.071641in}{1.695973in}}%
\pgfpathclose%
\pgfusepath{stroke,fill}%
\end{pgfscope}%
\begin{pgfscope}%
\pgfpathrectangle{\pgfqpoint{0.100000in}{0.212622in}}{\pgfqpoint{3.696000in}{3.696000in}}%
\pgfusepath{clip}%
\pgfsetbuttcap%
\pgfsetroundjoin%
\definecolor{currentfill}{rgb}{0.121569,0.466667,0.705882}%
\pgfsetfillcolor{currentfill}%
\pgfsetfillopacity{0.540996}%
\pgfsetlinewidth{1.003750pt}%
\definecolor{currentstroke}{rgb}{0.121569,0.466667,0.705882}%
\pgfsetstrokecolor{currentstroke}%
\pgfsetstrokeopacity{0.540996}%
\pgfsetdash{}{0pt}%
\pgfpathmoveto{\pgfqpoint{1.067982in}{1.691385in}}%
\pgfpathcurveto{\pgfqpoint{1.076218in}{1.691385in}}{\pgfqpoint{1.084118in}{1.694657in}}{\pgfqpoint{1.089942in}{1.700481in}}%
\pgfpathcurveto{\pgfqpoint{1.095766in}{1.706305in}}{\pgfqpoint{1.099038in}{1.714205in}}{\pgfqpoint{1.099038in}{1.722441in}}%
\pgfpathcurveto{\pgfqpoint{1.099038in}{1.730677in}}{\pgfqpoint{1.095766in}{1.738577in}}{\pgfqpoint{1.089942in}{1.744401in}}%
\pgfpathcurveto{\pgfqpoint{1.084118in}{1.750225in}}{\pgfqpoint{1.076218in}{1.753498in}}{\pgfqpoint{1.067982in}{1.753498in}}%
\pgfpathcurveto{\pgfqpoint{1.059745in}{1.753498in}}{\pgfqpoint{1.051845in}{1.750225in}}{\pgfqpoint{1.046021in}{1.744401in}}%
\pgfpathcurveto{\pgfqpoint{1.040197in}{1.738577in}}{\pgfqpoint{1.036925in}{1.730677in}}{\pgfqpoint{1.036925in}{1.722441in}}%
\pgfpathcurveto{\pgfqpoint{1.036925in}{1.714205in}}{\pgfqpoint{1.040197in}{1.706305in}}{\pgfqpoint{1.046021in}{1.700481in}}%
\pgfpathcurveto{\pgfqpoint{1.051845in}{1.694657in}}{\pgfqpoint{1.059745in}{1.691385in}}{\pgfqpoint{1.067982in}{1.691385in}}%
\pgfpathclose%
\pgfusepath{stroke,fill}%
\end{pgfscope}%
\begin{pgfscope}%
\pgfpathrectangle{\pgfqpoint{0.100000in}{0.212622in}}{\pgfqpoint{3.696000in}{3.696000in}}%
\pgfusepath{clip}%
\pgfsetbuttcap%
\pgfsetroundjoin%
\definecolor{currentfill}{rgb}{0.121569,0.466667,0.705882}%
\pgfsetfillcolor{currentfill}%
\pgfsetfillopacity{0.541984}%
\pgfsetlinewidth{1.003750pt}%
\definecolor{currentstroke}{rgb}{0.121569,0.466667,0.705882}%
\pgfsetstrokecolor{currentstroke}%
\pgfsetstrokeopacity{0.541984}%
\pgfsetdash{}{0pt}%
\pgfpathmoveto{\pgfqpoint{3.115448in}{2.005957in}}%
\pgfpathcurveto{\pgfqpoint{3.123684in}{2.005957in}}{\pgfqpoint{3.131584in}{2.009229in}}{\pgfqpoint{3.137408in}{2.015053in}}%
\pgfpathcurveto{\pgfqpoint{3.143232in}{2.020877in}}{\pgfqpoint{3.146504in}{2.028777in}}{\pgfqpoint{3.146504in}{2.037013in}}%
\pgfpathcurveto{\pgfqpoint{3.146504in}{2.045249in}}{\pgfqpoint{3.143232in}{2.053149in}}{\pgfqpoint{3.137408in}{2.058973in}}%
\pgfpathcurveto{\pgfqpoint{3.131584in}{2.064797in}}{\pgfqpoint{3.123684in}{2.068070in}}{\pgfqpoint{3.115448in}{2.068070in}}%
\pgfpathcurveto{\pgfqpoint{3.107211in}{2.068070in}}{\pgfqpoint{3.099311in}{2.064797in}}{\pgfqpoint{3.093487in}{2.058973in}}%
\pgfpathcurveto{\pgfqpoint{3.087663in}{2.053149in}}{\pgfqpoint{3.084391in}{2.045249in}}{\pgfqpoint{3.084391in}{2.037013in}}%
\pgfpathcurveto{\pgfqpoint{3.084391in}{2.028777in}}{\pgfqpoint{3.087663in}{2.020877in}}{\pgfqpoint{3.093487in}{2.015053in}}%
\pgfpathcurveto{\pgfqpoint{3.099311in}{2.009229in}}{\pgfqpoint{3.107211in}{2.005957in}}{\pgfqpoint{3.115448in}{2.005957in}}%
\pgfpathclose%
\pgfusepath{stroke,fill}%
\end{pgfscope}%
\begin{pgfscope}%
\pgfpathrectangle{\pgfqpoint{0.100000in}{0.212622in}}{\pgfqpoint{3.696000in}{3.696000in}}%
\pgfusepath{clip}%
\pgfsetbuttcap%
\pgfsetroundjoin%
\definecolor{currentfill}{rgb}{0.121569,0.466667,0.705882}%
\pgfsetfillcolor{currentfill}%
\pgfsetfillopacity{0.542100}%
\pgfsetlinewidth{1.003750pt}%
\definecolor{currentstroke}{rgb}{0.121569,0.466667,0.705882}%
\pgfsetstrokecolor{currentstroke}%
\pgfsetstrokeopacity{0.542100}%
\pgfsetdash{}{0pt}%
\pgfpathmoveto{\pgfqpoint{3.120307in}{1.999904in}}%
\pgfpathcurveto{\pgfqpoint{3.128543in}{1.999904in}}{\pgfqpoint{3.136443in}{2.003176in}}{\pgfqpoint{3.142267in}{2.009000in}}%
\pgfpathcurveto{\pgfqpoint{3.148091in}{2.014824in}}{\pgfqpoint{3.151363in}{2.022724in}}{\pgfqpoint{3.151363in}{2.030960in}}%
\pgfpathcurveto{\pgfqpoint{3.151363in}{2.039197in}}{\pgfqpoint{3.148091in}{2.047097in}}{\pgfqpoint{3.142267in}{2.052921in}}%
\pgfpathcurveto{\pgfqpoint{3.136443in}{2.058744in}}{\pgfqpoint{3.128543in}{2.062017in}}{\pgfqpoint{3.120307in}{2.062017in}}%
\pgfpathcurveto{\pgfqpoint{3.112070in}{2.062017in}}{\pgfqpoint{3.104170in}{2.058744in}}{\pgfqpoint{3.098346in}{2.052921in}}%
\pgfpathcurveto{\pgfqpoint{3.092522in}{2.047097in}}{\pgfqpoint{3.089250in}{2.039197in}}{\pgfqpoint{3.089250in}{2.030960in}}%
\pgfpathcurveto{\pgfqpoint{3.089250in}{2.022724in}}{\pgfqpoint{3.092522in}{2.014824in}}{\pgfqpoint{3.098346in}{2.009000in}}%
\pgfpathcurveto{\pgfqpoint{3.104170in}{2.003176in}}{\pgfqpoint{3.112070in}{1.999904in}}{\pgfqpoint{3.120307in}{1.999904in}}%
\pgfpathclose%
\pgfusepath{stroke,fill}%
\end{pgfscope}%
\begin{pgfscope}%
\pgfpathrectangle{\pgfqpoint{0.100000in}{0.212622in}}{\pgfqpoint{3.696000in}{3.696000in}}%
\pgfusepath{clip}%
\pgfsetbuttcap%
\pgfsetroundjoin%
\definecolor{currentfill}{rgb}{0.121569,0.466667,0.705882}%
\pgfsetfillcolor{currentfill}%
\pgfsetfillopacity{0.542703}%
\pgfsetlinewidth{1.003750pt}%
\definecolor{currentstroke}{rgb}{0.121569,0.466667,0.705882}%
\pgfsetstrokecolor{currentstroke}%
\pgfsetstrokeopacity{0.542703}%
\pgfsetdash{}{0pt}%
\pgfpathmoveto{\pgfqpoint{3.126889in}{1.999071in}}%
\pgfpathcurveto{\pgfqpoint{3.135126in}{1.999071in}}{\pgfqpoint{3.143026in}{2.002344in}}{\pgfqpoint{3.148850in}{2.008167in}}%
\pgfpathcurveto{\pgfqpoint{3.154674in}{2.013991in}}{\pgfqpoint{3.157946in}{2.021891in}}{\pgfqpoint{3.157946in}{2.030128in}}%
\pgfpathcurveto{\pgfqpoint{3.157946in}{2.038364in}}{\pgfqpoint{3.154674in}{2.046264in}}{\pgfqpoint{3.148850in}{2.052088in}}%
\pgfpathcurveto{\pgfqpoint{3.143026in}{2.057912in}}{\pgfqpoint{3.135126in}{2.061184in}}{\pgfqpoint{3.126889in}{2.061184in}}%
\pgfpathcurveto{\pgfqpoint{3.118653in}{2.061184in}}{\pgfqpoint{3.110753in}{2.057912in}}{\pgfqpoint{3.104929in}{2.052088in}}%
\pgfpathcurveto{\pgfqpoint{3.099105in}{2.046264in}}{\pgfqpoint{3.095833in}{2.038364in}}{\pgfqpoint{3.095833in}{2.030128in}}%
\pgfpathcurveto{\pgfqpoint{3.095833in}{2.021891in}}{\pgfqpoint{3.099105in}{2.013991in}}{\pgfqpoint{3.104929in}{2.008167in}}%
\pgfpathcurveto{\pgfqpoint{3.110753in}{2.002344in}}{\pgfqpoint{3.118653in}{1.999071in}}{\pgfqpoint{3.126889in}{1.999071in}}%
\pgfpathclose%
\pgfusepath{stroke,fill}%
\end{pgfscope}%
\begin{pgfscope}%
\pgfpathrectangle{\pgfqpoint{0.100000in}{0.212622in}}{\pgfqpoint{3.696000in}{3.696000in}}%
\pgfusepath{clip}%
\pgfsetbuttcap%
\pgfsetroundjoin%
\definecolor{currentfill}{rgb}{0.121569,0.466667,0.705882}%
\pgfsetfillcolor{currentfill}%
\pgfsetfillopacity{0.542879}%
\pgfsetlinewidth{1.003750pt}%
\definecolor{currentstroke}{rgb}{0.121569,0.466667,0.705882}%
\pgfsetstrokecolor{currentstroke}%
\pgfsetstrokeopacity{0.542879}%
\pgfsetdash{}{0pt}%
\pgfpathmoveto{\pgfqpoint{3.123666in}{2.003967in}}%
\pgfpathcurveto{\pgfqpoint{3.131903in}{2.003967in}}{\pgfqpoint{3.139803in}{2.007240in}}{\pgfqpoint{3.145627in}{2.013064in}}%
\pgfpathcurveto{\pgfqpoint{3.151450in}{2.018888in}}{\pgfqpoint{3.154723in}{2.026788in}}{\pgfqpoint{3.154723in}{2.035024in}}%
\pgfpathcurveto{\pgfqpoint{3.154723in}{2.043260in}}{\pgfqpoint{3.151450in}{2.051160in}}{\pgfqpoint{3.145627in}{2.056984in}}%
\pgfpathcurveto{\pgfqpoint{3.139803in}{2.062808in}}{\pgfqpoint{3.131903in}{2.066080in}}{\pgfqpoint{3.123666in}{2.066080in}}%
\pgfpathcurveto{\pgfqpoint{3.115430in}{2.066080in}}{\pgfqpoint{3.107530in}{2.062808in}}{\pgfqpoint{3.101706in}{2.056984in}}%
\pgfpathcurveto{\pgfqpoint{3.095882in}{2.051160in}}{\pgfqpoint{3.092610in}{2.043260in}}{\pgfqpoint{3.092610in}{2.035024in}}%
\pgfpathcurveto{\pgfqpoint{3.092610in}{2.026788in}}{\pgfqpoint{3.095882in}{2.018888in}}{\pgfqpoint{3.101706in}{2.013064in}}%
\pgfpathcurveto{\pgfqpoint{3.107530in}{2.007240in}}{\pgfqpoint{3.115430in}{2.003967in}}{\pgfqpoint{3.123666in}{2.003967in}}%
\pgfpathclose%
\pgfusepath{stroke,fill}%
\end{pgfscope}%
\begin{pgfscope}%
\pgfpathrectangle{\pgfqpoint{0.100000in}{0.212622in}}{\pgfqpoint{3.696000in}{3.696000in}}%
\pgfusepath{clip}%
\pgfsetbuttcap%
\pgfsetroundjoin%
\definecolor{currentfill}{rgb}{0.121569,0.466667,0.705882}%
\pgfsetfillcolor{currentfill}%
\pgfsetfillopacity{0.542995}%
\pgfsetlinewidth{1.003750pt}%
\definecolor{currentstroke}{rgb}{0.121569,0.466667,0.705882}%
\pgfsetstrokecolor{currentstroke}%
\pgfsetstrokeopacity{0.542995}%
\pgfsetdash{}{0pt}%
\pgfpathmoveto{\pgfqpoint{1.064829in}{1.690258in}}%
\pgfpathcurveto{\pgfqpoint{1.073065in}{1.690258in}}{\pgfqpoint{1.080965in}{1.693530in}}{\pgfqpoint{1.086789in}{1.699354in}}%
\pgfpathcurveto{\pgfqpoint{1.092613in}{1.705178in}}{\pgfqpoint{1.095886in}{1.713078in}}{\pgfqpoint{1.095886in}{1.721314in}}%
\pgfpathcurveto{\pgfqpoint{1.095886in}{1.729551in}}{\pgfqpoint{1.092613in}{1.737451in}}{\pgfqpoint{1.086789in}{1.743275in}}%
\pgfpathcurveto{\pgfqpoint{1.080965in}{1.749098in}}{\pgfqpoint{1.073065in}{1.752371in}}{\pgfqpoint{1.064829in}{1.752371in}}%
\pgfpathcurveto{\pgfqpoint{1.056593in}{1.752371in}}{\pgfqpoint{1.048693in}{1.749098in}}{\pgfqpoint{1.042869in}{1.743275in}}%
\pgfpathcurveto{\pgfqpoint{1.037045in}{1.737451in}}{\pgfqpoint{1.033773in}{1.729551in}}{\pgfqpoint{1.033773in}{1.721314in}}%
\pgfpathcurveto{\pgfqpoint{1.033773in}{1.713078in}}{\pgfqpoint{1.037045in}{1.705178in}}{\pgfqpoint{1.042869in}{1.699354in}}%
\pgfpathcurveto{\pgfqpoint{1.048693in}{1.693530in}}{\pgfqpoint{1.056593in}{1.690258in}}{\pgfqpoint{1.064829in}{1.690258in}}%
\pgfpathclose%
\pgfusepath{stroke,fill}%
\end{pgfscope}%
\begin{pgfscope}%
\pgfpathrectangle{\pgfqpoint{0.100000in}{0.212622in}}{\pgfqpoint{3.696000in}{3.696000in}}%
\pgfusepath{clip}%
\pgfsetbuttcap%
\pgfsetroundjoin%
\definecolor{currentfill}{rgb}{0.121569,0.466667,0.705882}%
\pgfsetfillcolor{currentfill}%
\pgfsetfillopacity{0.543564}%
\pgfsetlinewidth{1.003750pt}%
\definecolor{currentstroke}{rgb}{0.121569,0.466667,0.705882}%
\pgfsetstrokecolor{currentstroke}%
\pgfsetstrokeopacity{0.543564}%
\pgfsetdash{}{0pt}%
\pgfpathmoveto{\pgfqpoint{3.131888in}{2.001650in}}%
\pgfpathcurveto{\pgfqpoint{3.140124in}{2.001650in}}{\pgfqpoint{3.148024in}{2.004923in}}{\pgfqpoint{3.153848in}{2.010747in}}%
\pgfpathcurveto{\pgfqpoint{3.159672in}{2.016571in}}{\pgfqpoint{3.162944in}{2.024471in}}{\pgfqpoint{3.162944in}{2.032707in}}%
\pgfpathcurveto{\pgfqpoint{3.162944in}{2.040943in}}{\pgfqpoint{3.159672in}{2.048843in}}{\pgfqpoint{3.153848in}{2.054667in}}%
\pgfpathcurveto{\pgfqpoint{3.148024in}{2.060491in}}{\pgfqpoint{3.140124in}{2.063763in}}{\pgfqpoint{3.131888in}{2.063763in}}%
\pgfpathcurveto{\pgfqpoint{3.123652in}{2.063763in}}{\pgfqpoint{3.115752in}{2.060491in}}{\pgfqpoint{3.109928in}{2.054667in}}%
\pgfpathcurveto{\pgfqpoint{3.104104in}{2.048843in}}{\pgfqpoint{3.100831in}{2.040943in}}{\pgfqpoint{3.100831in}{2.032707in}}%
\pgfpathcurveto{\pgfqpoint{3.100831in}{2.024471in}}{\pgfqpoint{3.104104in}{2.016571in}}{\pgfqpoint{3.109928in}{2.010747in}}%
\pgfpathcurveto{\pgfqpoint{3.115752in}{2.004923in}}{\pgfqpoint{3.123652in}{2.001650in}}{\pgfqpoint{3.131888in}{2.001650in}}%
\pgfpathclose%
\pgfusepath{stroke,fill}%
\end{pgfscope}%
\begin{pgfscope}%
\pgfpathrectangle{\pgfqpoint{0.100000in}{0.212622in}}{\pgfqpoint{3.696000in}{3.696000in}}%
\pgfusepath{clip}%
\pgfsetbuttcap%
\pgfsetroundjoin%
\definecolor{currentfill}{rgb}{0.121569,0.466667,0.705882}%
\pgfsetfillcolor{currentfill}%
\pgfsetfillopacity{0.543780}%
\pgfsetlinewidth{1.003750pt}%
\definecolor{currentstroke}{rgb}{0.121569,0.466667,0.705882}%
\pgfsetstrokecolor{currentstroke}%
\pgfsetstrokeopacity{0.543780}%
\pgfsetdash{}{0pt}%
\pgfpathmoveto{\pgfqpoint{3.136401in}{1.994741in}}%
\pgfpathcurveto{\pgfqpoint{3.144637in}{1.994741in}}{\pgfqpoint{3.152537in}{1.998014in}}{\pgfqpoint{3.158361in}{2.003838in}}%
\pgfpathcurveto{\pgfqpoint{3.164185in}{2.009662in}}{\pgfqpoint{3.167457in}{2.017562in}}{\pgfqpoint{3.167457in}{2.025798in}}%
\pgfpathcurveto{\pgfqpoint{3.167457in}{2.034034in}}{\pgfqpoint{3.164185in}{2.041934in}}{\pgfqpoint{3.158361in}{2.047758in}}%
\pgfpathcurveto{\pgfqpoint{3.152537in}{2.053582in}}{\pgfqpoint{3.144637in}{2.056854in}}{\pgfqpoint{3.136401in}{2.056854in}}%
\pgfpathcurveto{\pgfqpoint{3.128164in}{2.056854in}}{\pgfqpoint{3.120264in}{2.053582in}}{\pgfqpoint{3.114440in}{2.047758in}}%
\pgfpathcurveto{\pgfqpoint{3.108616in}{2.041934in}}{\pgfqpoint{3.105344in}{2.034034in}}{\pgfqpoint{3.105344in}{2.025798in}}%
\pgfpathcurveto{\pgfqpoint{3.105344in}{2.017562in}}{\pgfqpoint{3.108616in}{2.009662in}}{\pgfqpoint{3.114440in}{2.003838in}}%
\pgfpathcurveto{\pgfqpoint{3.120264in}{1.998014in}}{\pgfqpoint{3.128164in}{1.994741in}}{\pgfqpoint{3.136401in}{1.994741in}}%
\pgfpathclose%
\pgfusepath{stroke,fill}%
\end{pgfscope}%
\begin{pgfscope}%
\pgfpathrectangle{\pgfqpoint{0.100000in}{0.212622in}}{\pgfqpoint{3.696000in}{3.696000in}}%
\pgfusepath{clip}%
\pgfsetbuttcap%
\pgfsetroundjoin%
\definecolor{currentfill}{rgb}{0.121569,0.466667,0.705882}%
\pgfsetfillcolor{currentfill}%
\pgfsetfillopacity{0.544128}%
\pgfsetlinewidth{1.003750pt}%
\definecolor{currentstroke}{rgb}{0.121569,0.466667,0.705882}%
\pgfsetstrokecolor{currentstroke}%
\pgfsetstrokeopacity{0.544128}%
\pgfsetdash{}{0pt}%
\pgfpathmoveto{\pgfqpoint{1.059912in}{1.688705in}}%
\pgfpathcurveto{\pgfqpoint{1.068148in}{1.688705in}}{\pgfqpoint{1.076048in}{1.691977in}}{\pgfqpoint{1.081872in}{1.697801in}}%
\pgfpathcurveto{\pgfqpoint{1.087696in}{1.703625in}}{\pgfqpoint{1.090968in}{1.711525in}}{\pgfqpoint{1.090968in}{1.719761in}}%
\pgfpathcurveto{\pgfqpoint{1.090968in}{1.727997in}}{\pgfqpoint{1.087696in}{1.735897in}}{\pgfqpoint{1.081872in}{1.741721in}}%
\pgfpathcurveto{\pgfqpoint{1.076048in}{1.747545in}}{\pgfqpoint{1.068148in}{1.750818in}}{\pgfqpoint{1.059912in}{1.750818in}}%
\pgfpathcurveto{\pgfqpoint{1.051675in}{1.750818in}}{\pgfqpoint{1.043775in}{1.747545in}}{\pgfqpoint{1.037951in}{1.741721in}}%
\pgfpathcurveto{\pgfqpoint{1.032128in}{1.735897in}}{\pgfqpoint{1.028855in}{1.727997in}}{\pgfqpoint{1.028855in}{1.719761in}}%
\pgfpathcurveto{\pgfqpoint{1.028855in}{1.711525in}}{\pgfqpoint{1.032128in}{1.703625in}}{\pgfqpoint{1.037951in}{1.697801in}}%
\pgfpathcurveto{\pgfqpoint{1.043775in}{1.691977in}}{\pgfqpoint{1.051675in}{1.688705in}}{\pgfqpoint{1.059912in}{1.688705in}}%
\pgfpathclose%
\pgfusepath{stroke,fill}%
\end{pgfscope}%
\begin{pgfscope}%
\pgfpathrectangle{\pgfqpoint{0.100000in}{0.212622in}}{\pgfqpoint{3.696000in}{3.696000in}}%
\pgfusepath{clip}%
\pgfsetbuttcap%
\pgfsetroundjoin%
\definecolor{currentfill}{rgb}{0.121569,0.466667,0.705882}%
\pgfsetfillcolor{currentfill}%
\pgfsetfillopacity{0.544495}%
\pgfsetlinewidth{1.003750pt}%
\definecolor{currentstroke}{rgb}{0.121569,0.466667,0.705882}%
\pgfsetstrokecolor{currentstroke}%
\pgfsetstrokeopacity{0.544495}%
\pgfsetdash{}{0pt}%
\pgfpathmoveto{\pgfqpoint{3.139521in}{1.996583in}}%
\pgfpathcurveto{\pgfqpoint{3.147758in}{1.996583in}}{\pgfqpoint{3.155658in}{1.999856in}}{\pgfqpoint{3.161482in}{2.005680in}}%
\pgfpathcurveto{\pgfqpoint{3.167306in}{2.011504in}}{\pgfqpoint{3.170578in}{2.019404in}}{\pgfqpoint{3.170578in}{2.027640in}}%
\pgfpathcurveto{\pgfqpoint{3.170578in}{2.035876in}}{\pgfqpoint{3.167306in}{2.043776in}}{\pgfqpoint{3.161482in}{2.049600in}}%
\pgfpathcurveto{\pgfqpoint{3.155658in}{2.055424in}}{\pgfqpoint{3.147758in}{2.058696in}}{\pgfqpoint{3.139521in}{2.058696in}}%
\pgfpathcurveto{\pgfqpoint{3.131285in}{2.058696in}}{\pgfqpoint{3.123385in}{2.055424in}}{\pgfqpoint{3.117561in}{2.049600in}}%
\pgfpathcurveto{\pgfqpoint{3.111737in}{2.043776in}}{\pgfqpoint{3.108465in}{2.035876in}}{\pgfqpoint{3.108465in}{2.027640in}}%
\pgfpathcurveto{\pgfqpoint{3.108465in}{2.019404in}}{\pgfqpoint{3.111737in}{2.011504in}}{\pgfqpoint{3.117561in}{2.005680in}}%
\pgfpathcurveto{\pgfqpoint{3.123385in}{1.999856in}}{\pgfqpoint{3.131285in}{1.996583in}}{\pgfqpoint{3.139521in}{1.996583in}}%
\pgfpathclose%
\pgfusepath{stroke,fill}%
\end{pgfscope}%
\begin{pgfscope}%
\pgfpathrectangle{\pgfqpoint{0.100000in}{0.212622in}}{\pgfqpoint{3.696000in}{3.696000in}}%
\pgfusepath{clip}%
\pgfsetbuttcap%
\pgfsetroundjoin%
\definecolor{currentfill}{rgb}{0.121569,0.466667,0.705882}%
\pgfsetfillcolor{currentfill}%
\pgfsetfillopacity{0.544752}%
\pgfsetlinewidth{1.003750pt}%
\definecolor{currentstroke}{rgb}{0.121569,0.466667,0.705882}%
\pgfsetstrokecolor{currentstroke}%
\pgfsetstrokeopacity{0.544752}%
\pgfsetdash{}{0pt}%
\pgfpathmoveto{\pgfqpoint{3.143368in}{1.993150in}}%
\pgfpathcurveto{\pgfqpoint{3.151604in}{1.993150in}}{\pgfqpoint{3.159504in}{1.996423in}}{\pgfqpoint{3.165328in}{2.002247in}}%
\pgfpathcurveto{\pgfqpoint{3.171152in}{2.008071in}}{\pgfqpoint{3.174424in}{2.015971in}}{\pgfqpoint{3.174424in}{2.024207in}}%
\pgfpathcurveto{\pgfqpoint{3.174424in}{2.032443in}}{\pgfqpoint{3.171152in}{2.040343in}}{\pgfqpoint{3.165328in}{2.046167in}}%
\pgfpathcurveto{\pgfqpoint{3.159504in}{2.051991in}}{\pgfqpoint{3.151604in}{2.055263in}}{\pgfqpoint{3.143368in}{2.055263in}}%
\pgfpathcurveto{\pgfqpoint{3.135132in}{2.055263in}}{\pgfqpoint{3.127232in}{2.051991in}}{\pgfqpoint{3.121408in}{2.046167in}}%
\pgfpathcurveto{\pgfqpoint{3.115584in}{2.040343in}}{\pgfqpoint{3.112311in}{2.032443in}}{\pgfqpoint{3.112311in}{2.024207in}}%
\pgfpathcurveto{\pgfqpoint{3.112311in}{2.015971in}}{\pgfqpoint{3.115584in}{2.008071in}}{\pgfqpoint{3.121408in}{2.002247in}}%
\pgfpathcurveto{\pgfqpoint{3.127232in}{1.996423in}}{\pgfqpoint{3.135132in}{1.993150in}}{\pgfqpoint{3.143368in}{1.993150in}}%
\pgfpathclose%
\pgfusepath{stroke,fill}%
\end{pgfscope}%
\begin{pgfscope}%
\pgfpathrectangle{\pgfqpoint{0.100000in}{0.212622in}}{\pgfqpoint{3.696000in}{3.696000in}}%
\pgfusepath{clip}%
\pgfsetbuttcap%
\pgfsetroundjoin%
\definecolor{currentfill}{rgb}{0.121569,0.466667,0.705882}%
\pgfsetfillcolor{currentfill}%
\pgfsetfillopacity{0.545653}%
\pgfsetlinewidth{1.003750pt}%
\definecolor{currentstroke}{rgb}{0.121569,0.466667,0.705882}%
\pgfsetstrokecolor{currentstroke}%
\pgfsetstrokeopacity{0.545653}%
\pgfsetdash{}{0pt}%
\pgfpathmoveto{\pgfqpoint{1.057425in}{1.687583in}}%
\pgfpathcurveto{\pgfqpoint{1.065661in}{1.687583in}}{\pgfqpoint{1.073561in}{1.690855in}}{\pgfqpoint{1.079385in}{1.696679in}}%
\pgfpathcurveto{\pgfqpoint{1.085209in}{1.702503in}}{\pgfqpoint{1.088481in}{1.710403in}}{\pgfqpoint{1.088481in}{1.718639in}}%
\pgfpathcurveto{\pgfqpoint{1.088481in}{1.726875in}}{\pgfqpoint{1.085209in}{1.734775in}}{\pgfqpoint{1.079385in}{1.740599in}}%
\pgfpathcurveto{\pgfqpoint{1.073561in}{1.746423in}}{\pgfqpoint{1.065661in}{1.749696in}}{\pgfqpoint{1.057425in}{1.749696in}}%
\pgfpathcurveto{\pgfqpoint{1.049188in}{1.749696in}}{\pgfqpoint{1.041288in}{1.746423in}}{\pgfqpoint{1.035464in}{1.740599in}}%
\pgfpathcurveto{\pgfqpoint{1.029640in}{1.734775in}}{\pgfqpoint{1.026368in}{1.726875in}}{\pgfqpoint{1.026368in}{1.718639in}}%
\pgfpathcurveto{\pgfqpoint{1.026368in}{1.710403in}}{\pgfqpoint{1.029640in}{1.702503in}}{\pgfqpoint{1.035464in}{1.696679in}}%
\pgfpathcurveto{\pgfqpoint{1.041288in}{1.690855in}}{\pgfqpoint{1.049188in}{1.687583in}}{\pgfqpoint{1.057425in}{1.687583in}}%
\pgfpathclose%
\pgfusepath{stroke,fill}%
\end{pgfscope}%
\begin{pgfscope}%
\pgfpathrectangle{\pgfqpoint{0.100000in}{0.212622in}}{\pgfqpoint{3.696000in}{3.696000in}}%
\pgfusepath{clip}%
\pgfsetbuttcap%
\pgfsetroundjoin%
\definecolor{currentfill}{rgb}{0.121569,0.466667,0.705882}%
\pgfsetfillcolor{currentfill}%
\pgfsetfillopacity{0.545775}%
\pgfsetlinewidth{1.003750pt}%
\definecolor{currentstroke}{rgb}{0.121569,0.466667,0.705882}%
\pgfsetstrokecolor{currentstroke}%
\pgfsetstrokeopacity{0.545775}%
\pgfsetdash{}{0pt}%
\pgfpathmoveto{\pgfqpoint{3.148652in}{1.996979in}}%
\pgfpathcurveto{\pgfqpoint{3.156888in}{1.996979in}}{\pgfqpoint{3.164788in}{2.000252in}}{\pgfqpoint{3.170612in}{2.006075in}}%
\pgfpathcurveto{\pgfqpoint{3.176436in}{2.011899in}}{\pgfqpoint{3.179708in}{2.019799in}}{\pgfqpoint{3.179708in}{2.028036in}}%
\pgfpathcurveto{\pgfqpoint{3.179708in}{2.036272in}}{\pgfqpoint{3.176436in}{2.044172in}}{\pgfqpoint{3.170612in}{2.049996in}}%
\pgfpathcurveto{\pgfqpoint{3.164788in}{2.055820in}}{\pgfqpoint{3.156888in}{2.059092in}}{\pgfqpoint{3.148652in}{2.059092in}}%
\pgfpathcurveto{\pgfqpoint{3.140415in}{2.059092in}}{\pgfqpoint{3.132515in}{2.055820in}}{\pgfqpoint{3.126691in}{2.049996in}}%
\pgfpathcurveto{\pgfqpoint{3.120867in}{2.044172in}}{\pgfqpoint{3.117595in}{2.036272in}}{\pgfqpoint{3.117595in}{2.028036in}}%
\pgfpathcurveto{\pgfqpoint{3.117595in}{2.019799in}}{\pgfqpoint{3.120867in}{2.011899in}}{\pgfqpoint{3.126691in}{2.006075in}}%
\pgfpathcurveto{\pgfqpoint{3.132515in}{2.000252in}}{\pgfqpoint{3.140415in}{1.996979in}}{\pgfqpoint{3.148652in}{1.996979in}}%
\pgfpathclose%
\pgfusepath{stroke,fill}%
\end{pgfscope}%
\begin{pgfscope}%
\pgfpathrectangle{\pgfqpoint{0.100000in}{0.212622in}}{\pgfqpoint{3.696000in}{3.696000in}}%
\pgfusepath{clip}%
\pgfsetbuttcap%
\pgfsetroundjoin%
\definecolor{currentfill}{rgb}{0.121569,0.466667,0.705882}%
\pgfsetfillcolor{currentfill}%
\pgfsetfillopacity{0.545828}%
\pgfsetlinewidth{1.003750pt}%
\definecolor{currentstroke}{rgb}{0.121569,0.466667,0.705882}%
\pgfsetstrokecolor{currentstroke}%
\pgfsetstrokeopacity{0.545828}%
\pgfsetdash{}{0pt}%
\pgfpathmoveto{\pgfqpoint{3.151260in}{1.994624in}}%
\pgfpathcurveto{\pgfqpoint{3.159496in}{1.994624in}}{\pgfqpoint{3.167396in}{1.997896in}}{\pgfqpoint{3.173220in}{2.003720in}}%
\pgfpathcurveto{\pgfqpoint{3.179044in}{2.009544in}}{\pgfqpoint{3.182316in}{2.017444in}}{\pgfqpoint{3.182316in}{2.025680in}}%
\pgfpathcurveto{\pgfqpoint{3.182316in}{2.033917in}}{\pgfqpoint{3.179044in}{2.041817in}}{\pgfqpoint{3.173220in}{2.047641in}}%
\pgfpathcurveto{\pgfqpoint{3.167396in}{2.053464in}}{\pgfqpoint{3.159496in}{2.056737in}}{\pgfqpoint{3.151260in}{2.056737in}}%
\pgfpathcurveto{\pgfqpoint{3.143024in}{2.056737in}}{\pgfqpoint{3.135123in}{2.053464in}}{\pgfqpoint{3.129300in}{2.047641in}}%
\pgfpathcurveto{\pgfqpoint{3.123476in}{2.041817in}}{\pgfqpoint{3.120203in}{2.033917in}}{\pgfqpoint{3.120203in}{2.025680in}}%
\pgfpathcurveto{\pgfqpoint{3.120203in}{2.017444in}}{\pgfqpoint{3.123476in}{2.009544in}}{\pgfqpoint{3.129300in}{2.003720in}}%
\pgfpathcurveto{\pgfqpoint{3.135123in}{1.997896in}}{\pgfqpoint{3.143024in}{1.994624in}}{\pgfqpoint{3.151260in}{1.994624in}}%
\pgfpathclose%
\pgfusepath{stroke,fill}%
\end{pgfscope}%
\begin{pgfscope}%
\pgfpathrectangle{\pgfqpoint{0.100000in}{0.212622in}}{\pgfqpoint{3.696000in}{3.696000in}}%
\pgfusepath{clip}%
\pgfsetbuttcap%
\pgfsetroundjoin%
\definecolor{currentfill}{rgb}{0.121569,0.466667,0.705882}%
\pgfsetfillcolor{currentfill}%
\pgfsetfillopacity{0.546289}%
\pgfsetlinewidth{1.003750pt}%
\definecolor{currentstroke}{rgb}{0.121569,0.466667,0.705882}%
\pgfsetstrokecolor{currentstroke}%
\pgfsetstrokeopacity{0.546289}%
\pgfsetdash{}{0pt}%
\pgfpathmoveto{\pgfqpoint{1.055565in}{1.684152in}}%
\pgfpathcurveto{\pgfqpoint{1.063801in}{1.684152in}}{\pgfqpoint{1.071701in}{1.687424in}}{\pgfqpoint{1.077525in}{1.693248in}}%
\pgfpathcurveto{\pgfqpoint{1.083349in}{1.699072in}}{\pgfqpoint{1.086621in}{1.706972in}}{\pgfqpoint{1.086621in}{1.715208in}}%
\pgfpathcurveto{\pgfqpoint{1.086621in}{1.723445in}}{\pgfqpoint{1.083349in}{1.731345in}}{\pgfqpoint{1.077525in}{1.737169in}}%
\pgfpathcurveto{\pgfqpoint{1.071701in}{1.742993in}}{\pgfqpoint{1.063801in}{1.746265in}}{\pgfqpoint{1.055565in}{1.746265in}}%
\pgfpathcurveto{\pgfqpoint{1.047329in}{1.746265in}}{\pgfqpoint{1.039429in}{1.742993in}}{\pgfqpoint{1.033605in}{1.737169in}}%
\pgfpathcurveto{\pgfqpoint{1.027781in}{1.731345in}}{\pgfqpoint{1.024508in}{1.723445in}}{\pgfqpoint{1.024508in}{1.715208in}}%
\pgfpathcurveto{\pgfqpoint{1.024508in}{1.706972in}}{\pgfqpoint{1.027781in}{1.699072in}}{\pgfqpoint{1.033605in}{1.693248in}}%
\pgfpathcurveto{\pgfqpoint{1.039429in}{1.687424in}}{\pgfqpoint{1.047329in}{1.684152in}}{\pgfqpoint{1.055565in}{1.684152in}}%
\pgfpathclose%
\pgfusepath{stroke,fill}%
\end{pgfscope}%
\begin{pgfscope}%
\pgfpathrectangle{\pgfqpoint{0.100000in}{0.212622in}}{\pgfqpoint{3.696000in}{3.696000in}}%
\pgfusepath{clip}%
\pgfsetbuttcap%
\pgfsetroundjoin%
\definecolor{currentfill}{rgb}{0.121569,0.466667,0.705882}%
\pgfsetfillcolor{currentfill}%
\pgfsetfillopacity{0.546555}%
\pgfsetlinewidth{1.003750pt}%
\definecolor{currentstroke}{rgb}{0.121569,0.466667,0.705882}%
\pgfsetstrokecolor{currentstroke}%
\pgfsetstrokeopacity{0.546555}%
\pgfsetdash{}{0pt}%
\pgfpathmoveto{\pgfqpoint{1.054443in}{1.683278in}}%
\pgfpathcurveto{\pgfqpoint{1.062680in}{1.683278in}}{\pgfqpoint{1.070580in}{1.686550in}}{\pgfqpoint{1.076404in}{1.692374in}}%
\pgfpathcurveto{\pgfqpoint{1.082227in}{1.698198in}}{\pgfqpoint{1.085500in}{1.706098in}}{\pgfqpoint{1.085500in}{1.714334in}}%
\pgfpathcurveto{\pgfqpoint{1.085500in}{1.722571in}}{\pgfqpoint{1.082227in}{1.730471in}}{\pgfqpoint{1.076404in}{1.736295in}}%
\pgfpathcurveto{\pgfqpoint{1.070580in}{1.742119in}}{\pgfqpoint{1.062680in}{1.745391in}}{\pgfqpoint{1.054443in}{1.745391in}}%
\pgfpathcurveto{\pgfqpoint{1.046207in}{1.745391in}}{\pgfqpoint{1.038307in}{1.742119in}}{\pgfqpoint{1.032483in}{1.736295in}}%
\pgfpathcurveto{\pgfqpoint{1.026659in}{1.730471in}}{\pgfqpoint{1.023387in}{1.722571in}}{\pgfqpoint{1.023387in}{1.714334in}}%
\pgfpathcurveto{\pgfqpoint{1.023387in}{1.706098in}}{\pgfqpoint{1.026659in}{1.698198in}}{\pgfqpoint{1.032483in}{1.692374in}}%
\pgfpathcurveto{\pgfqpoint{1.038307in}{1.686550in}}{\pgfqpoint{1.046207in}{1.683278in}}{\pgfqpoint{1.054443in}{1.683278in}}%
\pgfpathclose%
\pgfusepath{stroke,fill}%
\end{pgfscope}%
\begin{pgfscope}%
\pgfpathrectangle{\pgfqpoint{0.100000in}{0.212622in}}{\pgfqpoint{3.696000in}{3.696000in}}%
\pgfusepath{clip}%
\pgfsetbuttcap%
\pgfsetroundjoin%
\definecolor{currentfill}{rgb}{0.121569,0.466667,0.705882}%
\pgfsetfillcolor{currentfill}%
\pgfsetfillopacity{0.546589}%
\pgfsetlinewidth{1.003750pt}%
\definecolor{currentstroke}{rgb}{0.121569,0.466667,0.705882}%
\pgfsetstrokecolor{currentstroke}%
\pgfsetstrokeopacity{0.546589}%
\pgfsetdash{}{0pt}%
\pgfpathmoveto{\pgfqpoint{3.155615in}{1.996949in}}%
\pgfpathcurveto{\pgfqpoint{3.163851in}{1.996949in}}{\pgfqpoint{3.171752in}{2.000221in}}{\pgfqpoint{3.177575in}{2.006045in}}%
\pgfpathcurveto{\pgfqpoint{3.183399in}{2.011869in}}{\pgfqpoint{3.186672in}{2.019769in}}{\pgfqpoint{3.186672in}{2.028006in}}%
\pgfpathcurveto{\pgfqpoint{3.186672in}{2.036242in}}{\pgfqpoint{3.183399in}{2.044142in}}{\pgfqpoint{3.177575in}{2.049966in}}%
\pgfpathcurveto{\pgfqpoint{3.171752in}{2.055790in}}{\pgfqpoint{3.163851in}{2.059062in}}{\pgfqpoint{3.155615in}{2.059062in}}%
\pgfpathcurveto{\pgfqpoint{3.147379in}{2.059062in}}{\pgfqpoint{3.139479in}{2.055790in}}{\pgfqpoint{3.133655in}{2.049966in}}%
\pgfpathcurveto{\pgfqpoint{3.127831in}{2.044142in}}{\pgfqpoint{3.124559in}{2.036242in}}{\pgfqpoint{3.124559in}{2.028006in}}%
\pgfpathcurveto{\pgfqpoint{3.124559in}{2.019769in}}{\pgfqpoint{3.127831in}{2.011869in}}{\pgfqpoint{3.133655in}{2.006045in}}%
\pgfpathcurveto{\pgfqpoint{3.139479in}{2.000221in}}{\pgfqpoint{3.147379in}{1.996949in}}{\pgfqpoint{3.155615in}{1.996949in}}%
\pgfpathclose%
\pgfusepath{stroke,fill}%
\end{pgfscope}%
\begin{pgfscope}%
\pgfpathrectangle{\pgfqpoint{0.100000in}{0.212622in}}{\pgfqpoint{3.696000in}{3.696000in}}%
\pgfusepath{clip}%
\pgfsetbuttcap%
\pgfsetroundjoin%
\definecolor{currentfill}{rgb}{0.121569,0.466667,0.705882}%
\pgfsetfillcolor{currentfill}%
\pgfsetfillopacity{0.546794}%
\pgfsetlinewidth{1.003750pt}%
\definecolor{currentstroke}{rgb}{0.121569,0.466667,0.705882}%
\pgfsetstrokecolor{currentstroke}%
\pgfsetstrokeopacity{0.546794}%
\pgfsetdash{}{0pt}%
\pgfpathmoveto{\pgfqpoint{3.157709in}{1.995713in}}%
\pgfpathcurveto{\pgfqpoint{3.165946in}{1.995713in}}{\pgfqpoint{3.173846in}{1.998986in}}{\pgfqpoint{3.179670in}{2.004809in}}%
\pgfpathcurveto{\pgfqpoint{3.185493in}{2.010633in}}{\pgfqpoint{3.188766in}{2.018533in}}{\pgfqpoint{3.188766in}{2.026770in}}%
\pgfpathcurveto{\pgfqpoint{3.188766in}{2.035006in}}{\pgfqpoint{3.185493in}{2.042906in}}{\pgfqpoint{3.179670in}{2.048730in}}%
\pgfpathcurveto{\pgfqpoint{3.173846in}{2.054554in}}{\pgfqpoint{3.165946in}{2.057826in}}{\pgfqpoint{3.157709in}{2.057826in}}%
\pgfpathcurveto{\pgfqpoint{3.149473in}{2.057826in}}{\pgfqpoint{3.141573in}{2.054554in}}{\pgfqpoint{3.135749in}{2.048730in}}%
\pgfpathcurveto{\pgfqpoint{3.129925in}{2.042906in}}{\pgfqpoint{3.126653in}{2.035006in}}{\pgfqpoint{3.126653in}{2.026770in}}%
\pgfpathcurveto{\pgfqpoint{3.126653in}{2.018533in}}{\pgfqpoint{3.129925in}{2.010633in}}{\pgfqpoint{3.135749in}{2.004809in}}%
\pgfpathcurveto{\pgfqpoint{3.141573in}{1.998986in}}{\pgfqpoint{3.149473in}{1.995713in}}{\pgfqpoint{3.157709in}{1.995713in}}%
\pgfpathclose%
\pgfusepath{stroke,fill}%
\end{pgfscope}%
\begin{pgfscope}%
\pgfpathrectangle{\pgfqpoint{0.100000in}{0.212622in}}{\pgfqpoint{3.696000in}{3.696000in}}%
\pgfusepath{clip}%
\pgfsetbuttcap%
\pgfsetroundjoin%
\definecolor{currentfill}{rgb}{0.121569,0.466667,0.705882}%
\pgfsetfillcolor{currentfill}%
\pgfsetfillopacity{0.547316}%
\pgfsetlinewidth{1.003750pt}%
\definecolor{currentstroke}{rgb}{0.121569,0.466667,0.705882}%
\pgfsetstrokecolor{currentstroke}%
\pgfsetstrokeopacity{0.547316}%
\pgfsetdash{}{0pt}%
\pgfpathmoveto{\pgfqpoint{3.160419in}{1.996371in}}%
\pgfpathcurveto{\pgfqpoint{3.168656in}{1.996371in}}{\pgfqpoint{3.176556in}{1.999643in}}{\pgfqpoint{3.182380in}{2.005467in}}%
\pgfpathcurveto{\pgfqpoint{3.188203in}{2.011291in}}{\pgfqpoint{3.191476in}{2.019191in}}{\pgfqpoint{3.191476in}{2.027427in}}%
\pgfpathcurveto{\pgfqpoint{3.191476in}{2.035663in}}{\pgfqpoint{3.188203in}{2.043563in}}{\pgfqpoint{3.182380in}{2.049387in}}%
\pgfpathcurveto{\pgfqpoint{3.176556in}{2.055211in}}{\pgfqpoint{3.168656in}{2.058484in}}{\pgfqpoint{3.160419in}{2.058484in}}%
\pgfpathcurveto{\pgfqpoint{3.152183in}{2.058484in}}{\pgfqpoint{3.144283in}{2.055211in}}{\pgfqpoint{3.138459in}{2.049387in}}%
\pgfpathcurveto{\pgfqpoint{3.132635in}{2.043563in}}{\pgfqpoint{3.129363in}{2.035663in}}{\pgfqpoint{3.129363in}{2.027427in}}%
\pgfpathcurveto{\pgfqpoint{3.129363in}{2.019191in}}{\pgfqpoint{3.132635in}{2.011291in}}{\pgfqpoint{3.138459in}{2.005467in}}%
\pgfpathcurveto{\pgfqpoint{3.144283in}{1.999643in}}{\pgfqpoint{3.152183in}{1.996371in}}{\pgfqpoint{3.160419in}{1.996371in}}%
\pgfpathclose%
\pgfusepath{stroke,fill}%
\end{pgfscope}%
\begin{pgfscope}%
\pgfpathrectangle{\pgfqpoint{0.100000in}{0.212622in}}{\pgfqpoint{3.696000in}{3.696000in}}%
\pgfusepath{clip}%
\pgfsetbuttcap%
\pgfsetroundjoin%
\definecolor{currentfill}{rgb}{0.121569,0.466667,0.705882}%
\pgfsetfillcolor{currentfill}%
\pgfsetfillopacity{0.547365}%
\pgfsetlinewidth{1.003750pt}%
\definecolor{currentstroke}{rgb}{0.121569,0.466667,0.705882}%
\pgfsetstrokecolor{currentstroke}%
\pgfsetstrokeopacity{0.547365}%
\pgfsetdash{}{0pt}%
\pgfpathmoveto{\pgfqpoint{1.053402in}{1.682935in}}%
\pgfpathcurveto{\pgfqpoint{1.061638in}{1.682935in}}{\pgfqpoint{1.069538in}{1.686208in}}{\pgfqpoint{1.075362in}{1.692032in}}%
\pgfpathcurveto{\pgfqpoint{1.081186in}{1.697855in}}{\pgfqpoint{1.084459in}{1.705756in}}{\pgfqpoint{1.084459in}{1.713992in}}%
\pgfpathcurveto{\pgfqpoint{1.084459in}{1.722228in}}{\pgfqpoint{1.081186in}{1.730128in}}{\pgfqpoint{1.075362in}{1.735952in}}%
\pgfpathcurveto{\pgfqpoint{1.069538in}{1.741776in}}{\pgfqpoint{1.061638in}{1.745048in}}{\pgfqpoint{1.053402in}{1.745048in}}%
\pgfpathcurveto{\pgfqpoint{1.045166in}{1.745048in}}{\pgfqpoint{1.037266in}{1.741776in}}{\pgfqpoint{1.031442in}{1.735952in}}%
\pgfpathcurveto{\pgfqpoint{1.025618in}{1.730128in}}{\pgfqpoint{1.022346in}{1.722228in}}{\pgfqpoint{1.022346in}{1.713992in}}%
\pgfpathcurveto{\pgfqpoint{1.022346in}{1.705756in}}{\pgfqpoint{1.025618in}{1.697855in}}{\pgfqpoint{1.031442in}{1.692032in}}%
\pgfpathcurveto{\pgfqpoint{1.037266in}{1.686208in}}{\pgfqpoint{1.045166in}{1.682935in}}{\pgfqpoint{1.053402in}{1.682935in}}%
\pgfpathclose%
\pgfusepath{stroke,fill}%
\end{pgfscope}%
\begin{pgfscope}%
\pgfpathrectangle{\pgfqpoint{0.100000in}{0.212622in}}{\pgfqpoint{3.696000in}{3.696000in}}%
\pgfusepath{clip}%
\pgfsetbuttcap%
\pgfsetroundjoin%
\definecolor{currentfill}{rgb}{0.121569,0.466667,0.705882}%
\pgfsetfillcolor{currentfill}%
\pgfsetfillopacity{0.547906}%
\pgfsetlinewidth{1.003750pt}%
\definecolor{currentstroke}{rgb}{0.121569,0.466667,0.705882}%
\pgfsetstrokecolor{currentstroke}%
\pgfsetstrokeopacity{0.547906}%
\pgfsetdash{}{0pt}%
\pgfpathmoveto{\pgfqpoint{3.163685in}{1.996340in}}%
\pgfpathcurveto{\pgfqpoint{3.171922in}{1.996340in}}{\pgfqpoint{3.179822in}{1.999613in}}{\pgfqpoint{3.185646in}{2.005437in}}%
\pgfpathcurveto{\pgfqpoint{3.191470in}{2.011261in}}{\pgfqpoint{3.194742in}{2.019161in}}{\pgfqpoint{3.194742in}{2.027397in}}%
\pgfpathcurveto{\pgfqpoint{3.194742in}{2.035633in}}{\pgfqpoint{3.191470in}{2.043533in}}{\pgfqpoint{3.185646in}{2.049357in}}%
\pgfpathcurveto{\pgfqpoint{3.179822in}{2.055181in}}{\pgfqpoint{3.171922in}{2.058453in}}{\pgfqpoint{3.163685in}{2.058453in}}%
\pgfpathcurveto{\pgfqpoint{3.155449in}{2.058453in}}{\pgfqpoint{3.147549in}{2.055181in}}{\pgfqpoint{3.141725in}{2.049357in}}%
\pgfpathcurveto{\pgfqpoint{3.135901in}{2.043533in}}{\pgfqpoint{3.132629in}{2.035633in}}{\pgfqpoint{3.132629in}{2.027397in}}%
\pgfpathcurveto{\pgfqpoint{3.132629in}{2.019161in}}{\pgfqpoint{3.135901in}{2.011261in}}{\pgfqpoint{3.141725in}{2.005437in}}%
\pgfpathcurveto{\pgfqpoint{3.147549in}{1.999613in}}{\pgfqpoint{3.155449in}{1.996340in}}{\pgfqpoint{3.163685in}{1.996340in}}%
\pgfpathclose%
\pgfusepath{stroke,fill}%
\end{pgfscope}%
\begin{pgfscope}%
\pgfpathrectangle{\pgfqpoint{0.100000in}{0.212622in}}{\pgfqpoint{3.696000in}{3.696000in}}%
\pgfusepath{clip}%
\pgfsetbuttcap%
\pgfsetroundjoin%
\definecolor{currentfill}{rgb}{0.121569,0.466667,0.705882}%
\pgfsetfillcolor{currentfill}%
\pgfsetfillopacity{0.548250}%
\pgfsetlinewidth{1.003750pt}%
\definecolor{currentstroke}{rgb}{0.121569,0.466667,0.705882}%
\pgfsetstrokecolor{currentstroke}%
\pgfsetstrokeopacity{0.548250}%
\pgfsetdash{}{0pt}%
\pgfpathmoveto{\pgfqpoint{1.049396in}{1.680577in}}%
\pgfpathcurveto{\pgfqpoint{1.057633in}{1.680577in}}{\pgfqpoint{1.065533in}{1.683850in}}{\pgfqpoint{1.071357in}{1.689674in}}%
\pgfpathcurveto{\pgfqpoint{1.077180in}{1.695498in}}{\pgfqpoint{1.080453in}{1.703398in}}{\pgfqpoint{1.080453in}{1.711634in}}%
\pgfpathcurveto{\pgfqpoint{1.080453in}{1.719870in}}{\pgfqpoint{1.077180in}{1.727770in}}{\pgfqpoint{1.071357in}{1.733594in}}%
\pgfpathcurveto{\pgfqpoint{1.065533in}{1.739418in}}{\pgfqpoint{1.057633in}{1.742690in}}{\pgfqpoint{1.049396in}{1.742690in}}%
\pgfpathcurveto{\pgfqpoint{1.041160in}{1.742690in}}{\pgfqpoint{1.033260in}{1.739418in}}{\pgfqpoint{1.027436in}{1.733594in}}%
\pgfpathcurveto{\pgfqpoint{1.021612in}{1.727770in}}{\pgfqpoint{1.018340in}{1.719870in}}{\pgfqpoint{1.018340in}{1.711634in}}%
\pgfpathcurveto{\pgfqpoint{1.018340in}{1.703398in}}{\pgfqpoint{1.021612in}{1.695498in}}{\pgfqpoint{1.027436in}{1.689674in}}%
\pgfpathcurveto{\pgfqpoint{1.033260in}{1.683850in}}{\pgfqpoint{1.041160in}{1.680577in}}{\pgfqpoint{1.049396in}{1.680577in}}%
\pgfpathclose%
\pgfusepath{stroke,fill}%
\end{pgfscope}%
\begin{pgfscope}%
\pgfpathrectangle{\pgfqpoint{0.100000in}{0.212622in}}{\pgfqpoint{3.696000in}{3.696000in}}%
\pgfusepath{clip}%
\pgfsetbuttcap%
\pgfsetroundjoin%
\definecolor{currentfill}{rgb}{0.121569,0.466667,0.705882}%
\pgfsetfillcolor{currentfill}%
\pgfsetfillopacity{0.548647}%
\pgfsetlinewidth{1.003750pt}%
\definecolor{currentstroke}{rgb}{0.121569,0.466667,0.705882}%
\pgfsetstrokecolor{currentstroke}%
\pgfsetstrokeopacity{0.548647}%
\pgfsetdash{}{0pt}%
\pgfpathmoveto{\pgfqpoint{3.167427in}{1.997023in}}%
\pgfpathcurveto{\pgfqpoint{3.175664in}{1.997023in}}{\pgfqpoint{3.183564in}{2.000295in}}{\pgfqpoint{3.189388in}{2.006119in}}%
\pgfpathcurveto{\pgfqpoint{3.195212in}{2.011943in}}{\pgfqpoint{3.198484in}{2.019843in}}{\pgfqpoint{3.198484in}{2.028079in}}%
\pgfpathcurveto{\pgfqpoint{3.198484in}{2.036315in}}{\pgfqpoint{3.195212in}{2.044215in}}{\pgfqpoint{3.189388in}{2.050039in}}%
\pgfpathcurveto{\pgfqpoint{3.183564in}{2.055863in}}{\pgfqpoint{3.175664in}{2.059136in}}{\pgfqpoint{3.167427in}{2.059136in}}%
\pgfpathcurveto{\pgfqpoint{3.159191in}{2.059136in}}{\pgfqpoint{3.151291in}{2.055863in}}{\pgfqpoint{3.145467in}{2.050039in}}%
\pgfpathcurveto{\pgfqpoint{3.139643in}{2.044215in}}{\pgfqpoint{3.136371in}{2.036315in}}{\pgfqpoint{3.136371in}{2.028079in}}%
\pgfpathcurveto{\pgfqpoint{3.136371in}{2.019843in}}{\pgfqpoint{3.139643in}{2.011943in}}{\pgfqpoint{3.145467in}{2.006119in}}%
\pgfpathcurveto{\pgfqpoint{3.151291in}{2.000295in}}{\pgfqpoint{3.159191in}{1.997023in}}{\pgfqpoint{3.167427in}{1.997023in}}%
\pgfpathclose%
\pgfusepath{stroke,fill}%
\end{pgfscope}%
\begin{pgfscope}%
\pgfpathrectangle{\pgfqpoint{0.100000in}{0.212622in}}{\pgfqpoint{3.696000in}{3.696000in}}%
\pgfusepath{clip}%
\pgfsetbuttcap%
\pgfsetroundjoin%
\definecolor{currentfill}{rgb}{0.121569,0.466667,0.705882}%
\pgfsetfillcolor{currentfill}%
\pgfsetfillopacity{0.548867}%
\pgfsetlinewidth{1.003750pt}%
\definecolor{currentstroke}{rgb}{0.121569,0.466667,0.705882}%
\pgfsetstrokecolor{currentstroke}%
\pgfsetstrokeopacity{0.548867}%
\pgfsetdash{}{0pt}%
\pgfpathmoveto{\pgfqpoint{3.169447in}{1.996071in}}%
\pgfpathcurveto{\pgfqpoint{3.177683in}{1.996071in}}{\pgfqpoint{3.185583in}{1.999343in}}{\pgfqpoint{3.191407in}{2.005167in}}%
\pgfpathcurveto{\pgfqpoint{3.197231in}{2.010991in}}{\pgfqpoint{3.200503in}{2.018891in}}{\pgfqpoint{3.200503in}{2.027127in}}%
\pgfpathcurveto{\pgfqpoint{3.200503in}{2.035364in}}{\pgfqpoint{3.197231in}{2.043264in}}{\pgfqpoint{3.191407in}{2.049088in}}%
\pgfpathcurveto{\pgfqpoint{3.185583in}{2.054912in}}{\pgfqpoint{3.177683in}{2.058184in}}{\pgfqpoint{3.169447in}{2.058184in}}%
\pgfpathcurveto{\pgfqpoint{3.161211in}{2.058184in}}{\pgfqpoint{3.153311in}{2.054912in}}{\pgfqpoint{3.147487in}{2.049088in}}%
\pgfpathcurveto{\pgfqpoint{3.141663in}{2.043264in}}{\pgfqpoint{3.138390in}{2.035364in}}{\pgfqpoint{3.138390in}{2.027127in}}%
\pgfpathcurveto{\pgfqpoint{3.138390in}{2.018891in}}{\pgfqpoint{3.141663in}{2.010991in}}{\pgfqpoint{3.147487in}{2.005167in}}%
\pgfpathcurveto{\pgfqpoint{3.153311in}{1.999343in}}{\pgfqpoint{3.161211in}{1.996071in}}{\pgfqpoint{3.169447in}{1.996071in}}%
\pgfpathclose%
\pgfusepath{stroke,fill}%
\end{pgfscope}%
\begin{pgfscope}%
\pgfpathrectangle{\pgfqpoint{0.100000in}{0.212622in}}{\pgfqpoint{3.696000in}{3.696000in}}%
\pgfusepath{clip}%
\pgfsetbuttcap%
\pgfsetroundjoin%
\definecolor{currentfill}{rgb}{0.121569,0.466667,0.705882}%
\pgfsetfillcolor{currentfill}%
\pgfsetfillopacity{0.549242}%
\pgfsetlinewidth{1.003750pt}%
\definecolor{currentstroke}{rgb}{0.121569,0.466667,0.705882}%
\pgfsetstrokecolor{currentstroke}%
\pgfsetstrokeopacity{0.549242}%
\pgfsetdash{}{0pt}%
\pgfpathmoveto{\pgfqpoint{1.048344in}{1.680575in}}%
\pgfpathcurveto{\pgfqpoint{1.056580in}{1.680575in}}{\pgfqpoint{1.064480in}{1.683847in}}{\pgfqpoint{1.070304in}{1.689671in}}%
\pgfpathcurveto{\pgfqpoint{1.076128in}{1.695495in}}{\pgfqpoint{1.079400in}{1.703395in}}{\pgfqpoint{1.079400in}{1.711631in}}%
\pgfpathcurveto{\pgfqpoint{1.079400in}{1.719868in}}{\pgfqpoint{1.076128in}{1.727768in}}{\pgfqpoint{1.070304in}{1.733592in}}%
\pgfpathcurveto{\pgfqpoint{1.064480in}{1.739416in}}{\pgfqpoint{1.056580in}{1.742688in}}{\pgfqpoint{1.048344in}{1.742688in}}%
\pgfpathcurveto{\pgfqpoint{1.040107in}{1.742688in}}{\pgfqpoint{1.032207in}{1.739416in}}{\pgfqpoint{1.026383in}{1.733592in}}%
\pgfpathcurveto{\pgfqpoint{1.020559in}{1.727768in}}{\pgfqpoint{1.017287in}{1.719868in}}{\pgfqpoint{1.017287in}{1.711631in}}%
\pgfpathcurveto{\pgfqpoint{1.017287in}{1.703395in}}{\pgfqpoint{1.020559in}{1.695495in}}{\pgfqpoint{1.026383in}{1.689671in}}%
\pgfpathcurveto{\pgfqpoint{1.032207in}{1.683847in}}{\pgfqpoint{1.040107in}{1.680575in}}{\pgfqpoint{1.048344in}{1.680575in}}%
\pgfpathclose%
\pgfusepath{stroke,fill}%
\end{pgfscope}%
\begin{pgfscope}%
\pgfpathrectangle{\pgfqpoint{0.100000in}{0.212622in}}{\pgfqpoint{3.696000in}{3.696000in}}%
\pgfusepath{clip}%
\pgfsetbuttcap%
\pgfsetroundjoin%
\definecolor{currentfill}{rgb}{0.121569,0.466667,0.705882}%
\pgfsetfillcolor{currentfill}%
\pgfsetfillopacity{0.549475}%
\pgfsetlinewidth{1.003750pt}%
\definecolor{currentstroke}{rgb}{0.121569,0.466667,0.705882}%
\pgfsetstrokecolor{currentstroke}%
\pgfsetstrokeopacity{0.549475}%
\pgfsetdash{}{0pt}%
\pgfpathmoveto{\pgfqpoint{3.172326in}{1.998049in}}%
\pgfpathcurveto{\pgfqpoint{3.180562in}{1.998049in}}{\pgfqpoint{3.188462in}{2.001321in}}{\pgfqpoint{3.194286in}{2.007145in}}%
\pgfpathcurveto{\pgfqpoint{3.200110in}{2.012969in}}{\pgfqpoint{3.203383in}{2.020869in}}{\pgfqpoint{3.203383in}{2.029105in}}%
\pgfpathcurveto{\pgfqpoint{3.203383in}{2.037341in}}{\pgfqpoint{3.200110in}{2.045241in}}{\pgfqpoint{3.194286in}{2.051065in}}%
\pgfpathcurveto{\pgfqpoint{3.188462in}{2.056889in}}{\pgfqpoint{3.180562in}{2.060162in}}{\pgfqpoint{3.172326in}{2.060162in}}%
\pgfpathcurveto{\pgfqpoint{3.164090in}{2.060162in}}{\pgfqpoint{3.156190in}{2.056889in}}{\pgfqpoint{3.150366in}{2.051065in}}%
\pgfpathcurveto{\pgfqpoint{3.144542in}{2.045241in}}{\pgfqpoint{3.141270in}{2.037341in}}{\pgfqpoint{3.141270in}{2.029105in}}%
\pgfpathcurveto{\pgfqpoint{3.141270in}{2.020869in}}{\pgfqpoint{3.144542in}{2.012969in}}{\pgfqpoint{3.150366in}{2.007145in}}%
\pgfpathcurveto{\pgfqpoint{3.156190in}{2.001321in}}{\pgfqpoint{3.164090in}{1.998049in}}{\pgfqpoint{3.172326in}{1.998049in}}%
\pgfpathclose%
\pgfusepath{stroke,fill}%
\end{pgfscope}%
\begin{pgfscope}%
\pgfpathrectangle{\pgfqpoint{0.100000in}{0.212622in}}{\pgfqpoint{3.696000in}{3.696000in}}%
\pgfusepath{clip}%
\pgfsetbuttcap%
\pgfsetroundjoin%
\definecolor{currentfill}{rgb}{0.121569,0.466667,0.705882}%
\pgfsetfillcolor{currentfill}%
\pgfsetfillopacity{0.549564}%
\pgfsetlinewidth{1.003750pt}%
\definecolor{currentstroke}{rgb}{0.121569,0.466667,0.705882}%
\pgfsetstrokecolor{currentstroke}%
\pgfsetstrokeopacity{0.549564}%
\pgfsetdash{}{0pt}%
\pgfpathmoveto{\pgfqpoint{3.175168in}{1.995185in}}%
\pgfpathcurveto{\pgfqpoint{3.183405in}{1.995185in}}{\pgfqpoint{3.191305in}{1.998457in}}{\pgfqpoint{3.197129in}{2.004281in}}%
\pgfpathcurveto{\pgfqpoint{3.202952in}{2.010105in}}{\pgfqpoint{3.206225in}{2.018005in}}{\pgfqpoint{3.206225in}{2.026241in}}%
\pgfpathcurveto{\pgfqpoint{3.206225in}{2.034477in}}{\pgfqpoint{3.202952in}{2.042377in}}{\pgfqpoint{3.197129in}{2.048201in}}%
\pgfpathcurveto{\pgfqpoint{3.191305in}{2.054025in}}{\pgfqpoint{3.183405in}{2.057298in}}{\pgfqpoint{3.175168in}{2.057298in}}%
\pgfpathcurveto{\pgfqpoint{3.166932in}{2.057298in}}{\pgfqpoint{3.159032in}{2.054025in}}{\pgfqpoint{3.153208in}{2.048201in}}%
\pgfpathcurveto{\pgfqpoint{3.147384in}{2.042377in}}{\pgfqpoint{3.144112in}{2.034477in}}{\pgfqpoint{3.144112in}{2.026241in}}%
\pgfpathcurveto{\pgfqpoint{3.144112in}{2.018005in}}{\pgfqpoint{3.147384in}{2.010105in}}{\pgfqpoint{3.153208in}{2.004281in}}%
\pgfpathcurveto{\pgfqpoint{3.159032in}{1.998457in}}{\pgfqpoint{3.166932in}{1.995185in}}{\pgfqpoint{3.175168in}{1.995185in}}%
\pgfpathclose%
\pgfusepath{stroke,fill}%
\end{pgfscope}%
\begin{pgfscope}%
\pgfpathrectangle{\pgfqpoint{0.100000in}{0.212622in}}{\pgfqpoint{3.696000in}{3.696000in}}%
\pgfusepath{clip}%
\pgfsetbuttcap%
\pgfsetroundjoin%
\definecolor{currentfill}{rgb}{0.121569,0.466667,0.705882}%
\pgfsetfillcolor{currentfill}%
\pgfsetfillopacity{0.549868}%
\pgfsetlinewidth{1.003750pt}%
\definecolor{currentstroke}{rgb}{0.121569,0.466667,0.705882}%
\pgfsetstrokecolor{currentstroke}%
\pgfsetstrokeopacity{0.549868}%
\pgfsetdash{}{0pt}%
\pgfpathmoveto{\pgfqpoint{1.046152in}{1.681210in}}%
\pgfpathcurveto{\pgfqpoint{1.054388in}{1.681210in}}{\pgfqpoint{1.062288in}{1.684483in}}{\pgfqpoint{1.068112in}{1.690307in}}%
\pgfpathcurveto{\pgfqpoint{1.073936in}{1.696131in}}{\pgfqpoint{1.077208in}{1.704031in}}{\pgfqpoint{1.077208in}{1.712267in}}%
\pgfpathcurveto{\pgfqpoint{1.077208in}{1.720503in}}{\pgfqpoint{1.073936in}{1.728403in}}{\pgfqpoint{1.068112in}{1.734227in}}%
\pgfpathcurveto{\pgfqpoint{1.062288in}{1.740051in}}{\pgfqpoint{1.054388in}{1.743323in}}{\pgfqpoint{1.046152in}{1.743323in}}%
\pgfpathcurveto{\pgfqpoint{1.037915in}{1.743323in}}{\pgfqpoint{1.030015in}{1.740051in}}{\pgfqpoint{1.024191in}{1.734227in}}%
\pgfpathcurveto{\pgfqpoint{1.018368in}{1.728403in}}{\pgfqpoint{1.015095in}{1.720503in}}{\pgfqpoint{1.015095in}{1.712267in}}%
\pgfpathcurveto{\pgfqpoint{1.015095in}{1.704031in}}{\pgfqpoint{1.018368in}{1.696131in}}{\pgfqpoint{1.024191in}{1.690307in}}%
\pgfpathcurveto{\pgfqpoint{1.030015in}{1.684483in}}{\pgfqpoint{1.037915in}{1.681210in}}{\pgfqpoint{1.046152in}{1.681210in}}%
\pgfpathclose%
\pgfusepath{stroke,fill}%
\end{pgfscope}%
\begin{pgfscope}%
\pgfpathrectangle{\pgfqpoint{0.100000in}{0.212622in}}{\pgfqpoint{3.696000in}{3.696000in}}%
\pgfusepath{clip}%
\pgfsetbuttcap%
\pgfsetroundjoin%
\definecolor{currentfill}{rgb}{0.121569,0.466667,0.705882}%
\pgfsetfillcolor{currentfill}%
\pgfsetfillopacity{0.550234}%
\pgfsetlinewidth{1.003750pt}%
\definecolor{currentstroke}{rgb}{0.121569,0.466667,0.705882}%
\pgfsetstrokecolor{currentstroke}%
\pgfsetstrokeopacity{0.550234}%
\pgfsetdash{}{0pt}%
\pgfpathmoveto{\pgfqpoint{3.179211in}{1.996680in}}%
\pgfpathcurveto{\pgfqpoint{3.187447in}{1.996680in}}{\pgfqpoint{3.195347in}{1.999953in}}{\pgfqpoint{3.201171in}{2.005777in}}%
\pgfpathcurveto{\pgfqpoint{3.206995in}{2.011600in}}{\pgfqpoint{3.210268in}{2.019500in}}{\pgfqpoint{3.210268in}{2.027737in}}%
\pgfpathcurveto{\pgfqpoint{3.210268in}{2.035973in}}{\pgfqpoint{3.206995in}{2.043873in}}{\pgfqpoint{3.201171in}{2.049697in}}%
\pgfpathcurveto{\pgfqpoint{3.195347in}{2.055521in}}{\pgfqpoint{3.187447in}{2.058793in}}{\pgfqpoint{3.179211in}{2.058793in}}%
\pgfpathcurveto{\pgfqpoint{3.170975in}{2.058793in}}{\pgfqpoint{3.163075in}{2.055521in}}{\pgfqpoint{3.157251in}{2.049697in}}%
\pgfpathcurveto{\pgfqpoint{3.151427in}{2.043873in}}{\pgfqpoint{3.148155in}{2.035973in}}{\pgfqpoint{3.148155in}{2.027737in}}%
\pgfpathcurveto{\pgfqpoint{3.148155in}{2.019500in}}{\pgfqpoint{3.151427in}{2.011600in}}{\pgfqpoint{3.157251in}{2.005777in}}%
\pgfpathcurveto{\pgfqpoint{3.163075in}{1.999953in}}{\pgfqpoint{3.170975in}{1.996680in}}{\pgfqpoint{3.179211in}{1.996680in}}%
\pgfpathclose%
\pgfusepath{stroke,fill}%
\end{pgfscope}%
\begin{pgfscope}%
\pgfpathrectangle{\pgfqpoint{0.100000in}{0.212622in}}{\pgfqpoint{3.696000in}{3.696000in}}%
\pgfusepath{clip}%
\pgfsetbuttcap%
\pgfsetroundjoin%
\definecolor{currentfill}{rgb}{0.121569,0.466667,0.705882}%
\pgfsetfillcolor{currentfill}%
\pgfsetfillopacity{0.550267}%
\pgfsetlinewidth{1.003750pt}%
\definecolor{currentstroke}{rgb}{0.121569,0.466667,0.705882}%
\pgfsetstrokecolor{currentstroke}%
\pgfsetstrokeopacity{0.550267}%
\pgfsetdash{}{0pt}%
\pgfpathmoveto{\pgfqpoint{1.045782in}{1.680593in}}%
\pgfpathcurveto{\pgfqpoint{1.054019in}{1.680593in}}{\pgfqpoint{1.061919in}{1.683865in}}{\pgfqpoint{1.067742in}{1.689689in}}%
\pgfpathcurveto{\pgfqpoint{1.073566in}{1.695513in}}{\pgfqpoint{1.076839in}{1.703413in}}{\pgfqpoint{1.076839in}{1.711649in}}%
\pgfpathcurveto{\pgfqpoint{1.076839in}{1.719886in}}{\pgfqpoint{1.073566in}{1.727786in}}{\pgfqpoint{1.067742in}{1.733610in}}%
\pgfpathcurveto{\pgfqpoint{1.061919in}{1.739434in}}{\pgfqpoint{1.054019in}{1.742706in}}{\pgfqpoint{1.045782in}{1.742706in}}%
\pgfpathcurveto{\pgfqpoint{1.037546in}{1.742706in}}{\pgfqpoint{1.029646in}{1.739434in}}{\pgfqpoint{1.023822in}{1.733610in}}%
\pgfpathcurveto{\pgfqpoint{1.017998in}{1.727786in}}{\pgfqpoint{1.014726in}{1.719886in}}{\pgfqpoint{1.014726in}{1.711649in}}%
\pgfpathcurveto{\pgfqpoint{1.014726in}{1.703413in}}{\pgfqpoint{1.017998in}{1.695513in}}{\pgfqpoint{1.023822in}{1.689689in}}%
\pgfpathcurveto{\pgfqpoint{1.029646in}{1.683865in}}{\pgfqpoint{1.037546in}{1.680593in}}{\pgfqpoint{1.045782in}{1.680593in}}%
\pgfpathclose%
\pgfusepath{stroke,fill}%
\end{pgfscope}%
\begin{pgfscope}%
\pgfpathrectangle{\pgfqpoint{0.100000in}{0.212622in}}{\pgfqpoint{3.696000in}{3.696000in}}%
\pgfusepath{clip}%
\pgfsetbuttcap%
\pgfsetroundjoin%
\definecolor{currentfill}{rgb}{0.121569,0.466667,0.705882}%
\pgfsetfillcolor{currentfill}%
\pgfsetfillopacity{0.550393}%
\pgfsetlinewidth{1.003750pt}%
\definecolor{currentstroke}{rgb}{0.121569,0.466667,0.705882}%
\pgfsetstrokecolor{currentstroke}%
\pgfsetstrokeopacity{0.550393}%
\pgfsetdash{}{0pt}%
\pgfpathmoveto{\pgfqpoint{1.045188in}{1.679046in}}%
\pgfpathcurveto{\pgfqpoint{1.053424in}{1.679046in}}{\pgfqpoint{1.061324in}{1.682318in}}{\pgfqpoint{1.067148in}{1.688142in}}%
\pgfpathcurveto{\pgfqpoint{1.072972in}{1.693966in}}{\pgfqpoint{1.076245in}{1.701866in}}{\pgfqpoint{1.076245in}{1.710102in}}%
\pgfpathcurveto{\pgfqpoint{1.076245in}{1.718339in}}{\pgfqpoint{1.072972in}{1.726239in}}{\pgfqpoint{1.067148in}{1.732063in}}%
\pgfpathcurveto{\pgfqpoint{1.061324in}{1.737886in}}{\pgfqpoint{1.053424in}{1.741159in}}{\pgfqpoint{1.045188in}{1.741159in}}%
\pgfpathcurveto{\pgfqpoint{1.036952in}{1.741159in}}{\pgfqpoint{1.029052in}{1.737886in}}{\pgfqpoint{1.023228in}{1.732063in}}%
\pgfpathcurveto{\pgfqpoint{1.017404in}{1.726239in}}{\pgfqpoint{1.014132in}{1.718339in}}{\pgfqpoint{1.014132in}{1.710102in}}%
\pgfpathcurveto{\pgfqpoint{1.014132in}{1.701866in}}{\pgfqpoint{1.017404in}{1.693966in}}{\pgfqpoint{1.023228in}{1.688142in}}%
\pgfpathcurveto{\pgfqpoint{1.029052in}{1.682318in}}{\pgfqpoint{1.036952in}{1.679046in}}{\pgfqpoint{1.045188in}{1.679046in}}%
\pgfpathclose%
\pgfusepath{stroke,fill}%
\end{pgfscope}%
\begin{pgfscope}%
\pgfpathrectangle{\pgfqpoint{0.100000in}{0.212622in}}{\pgfqpoint{3.696000in}{3.696000in}}%
\pgfusepath{clip}%
\pgfsetbuttcap%
\pgfsetroundjoin%
\definecolor{currentfill}{rgb}{0.121569,0.466667,0.705882}%
\pgfsetfillcolor{currentfill}%
\pgfsetfillopacity{0.550575}%
\pgfsetlinewidth{1.003750pt}%
\definecolor{currentstroke}{rgb}{0.121569,0.466667,0.705882}%
\pgfsetstrokecolor{currentstroke}%
\pgfsetstrokeopacity{0.550575}%
\pgfsetdash{}{0pt}%
\pgfpathmoveto{\pgfqpoint{1.044569in}{1.678722in}}%
\pgfpathcurveto{\pgfqpoint{1.052805in}{1.678722in}}{\pgfqpoint{1.060705in}{1.681994in}}{\pgfqpoint{1.066529in}{1.687818in}}%
\pgfpathcurveto{\pgfqpoint{1.072353in}{1.693642in}}{\pgfqpoint{1.075626in}{1.701542in}}{\pgfqpoint{1.075626in}{1.709778in}}%
\pgfpathcurveto{\pgfqpoint{1.075626in}{1.718014in}}{\pgfqpoint{1.072353in}{1.725915in}}{\pgfqpoint{1.066529in}{1.731738in}}%
\pgfpathcurveto{\pgfqpoint{1.060705in}{1.737562in}}{\pgfqpoint{1.052805in}{1.740835in}}{\pgfqpoint{1.044569in}{1.740835in}}%
\pgfpathcurveto{\pgfqpoint{1.036333in}{1.740835in}}{\pgfqpoint{1.028433in}{1.737562in}}{\pgfqpoint{1.022609in}{1.731738in}}%
\pgfpathcurveto{\pgfqpoint{1.016785in}{1.725915in}}{\pgfqpoint{1.013513in}{1.718014in}}{\pgfqpoint{1.013513in}{1.709778in}}%
\pgfpathcurveto{\pgfqpoint{1.013513in}{1.701542in}}{\pgfqpoint{1.016785in}{1.693642in}}{\pgfqpoint{1.022609in}{1.687818in}}%
\pgfpathcurveto{\pgfqpoint{1.028433in}{1.681994in}}{\pgfqpoint{1.036333in}{1.678722in}}{\pgfqpoint{1.044569in}{1.678722in}}%
\pgfpathclose%
\pgfusepath{stroke,fill}%
\end{pgfscope}%
\begin{pgfscope}%
\pgfpathrectangle{\pgfqpoint{0.100000in}{0.212622in}}{\pgfqpoint{3.696000in}{3.696000in}}%
\pgfusepath{clip}%
\pgfsetbuttcap%
\pgfsetroundjoin%
\definecolor{currentfill}{rgb}{0.121569,0.466667,0.705882}%
\pgfsetfillcolor{currentfill}%
\pgfsetfillopacity{0.550740}%
\pgfsetlinewidth{1.003750pt}%
\definecolor{currentstroke}{rgb}{0.121569,0.466667,0.705882}%
\pgfsetstrokecolor{currentstroke}%
\pgfsetstrokeopacity{0.550740}%
\pgfsetdash{}{0pt}%
\pgfpathmoveto{\pgfqpoint{3.183398in}{1.994122in}}%
\pgfpathcurveto{\pgfqpoint{3.191634in}{1.994122in}}{\pgfqpoint{3.199534in}{1.997395in}}{\pgfqpoint{3.205358in}{2.003218in}}%
\pgfpathcurveto{\pgfqpoint{3.211182in}{2.009042in}}{\pgfqpoint{3.214455in}{2.016942in}}{\pgfqpoint{3.214455in}{2.025179in}}%
\pgfpathcurveto{\pgfqpoint{3.214455in}{2.033415in}}{\pgfqpoint{3.211182in}{2.041315in}}{\pgfqpoint{3.205358in}{2.047139in}}%
\pgfpathcurveto{\pgfqpoint{3.199534in}{2.052963in}}{\pgfqpoint{3.191634in}{2.056235in}}{\pgfqpoint{3.183398in}{2.056235in}}%
\pgfpathcurveto{\pgfqpoint{3.175162in}{2.056235in}}{\pgfqpoint{3.167262in}{2.052963in}}{\pgfqpoint{3.161438in}{2.047139in}}%
\pgfpathcurveto{\pgfqpoint{3.155614in}{2.041315in}}{\pgfqpoint{3.152342in}{2.033415in}}{\pgfqpoint{3.152342in}{2.025179in}}%
\pgfpathcurveto{\pgfqpoint{3.152342in}{2.016942in}}{\pgfqpoint{3.155614in}{2.009042in}}{\pgfqpoint{3.161438in}{2.003218in}}%
\pgfpathcurveto{\pgfqpoint{3.167262in}{1.997395in}}{\pgfqpoint{3.175162in}{1.994122in}}{\pgfqpoint{3.183398in}{1.994122in}}%
\pgfpathclose%
\pgfusepath{stroke,fill}%
\end{pgfscope}%
\begin{pgfscope}%
\pgfpathrectangle{\pgfqpoint{0.100000in}{0.212622in}}{\pgfqpoint{3.696000in}{3.696000in}}%
\pgfusepath{clip}%
\pgfsetbuttcap%
\pgfsetroundjoin%
\definecolor{currentfill}{rgb}{0.121569,0.466667,0.705882}%
\pgfsetfillcolor{currentfill}%
\pgfsetfillopacity{0.550948}%
\pgfsetlinewidth{1.003750pt}%
\definecolor{currentstroke}{rgb}{0.121569,0.466667,0.705882}%
\pgfsetstrokecolor{currentstroke}%
\pgfsetstrokeopacity{0.550948}%
\pgfsetdash{}{0pt}%
\pgfpathmoveto{\pgfqpoint{1.043792in}{1.678026in}}%
\pgfpathcurveto{\pgfqpoint{1.052028in}{1.678026in}}{\pgfqpoint{1.059928in}{1.681298in}}{\pgfqpoint{1.065752in}{1.687122in}}%
\pgfpathcurveto{\pgfqpoint{1.071576in}{1.692946in}}{\pgfqpoint{1.074848in}{1.700846in}}{\pgfqpoint{1.074848in}{1.709082in}}%
\pgfpathcurveto{\pgfqpoint{1.074848in}{1.717318in}}{\pgfqpoint{1.071576in}{1.725218in}}{\pgfqpoint{1.065752in}{1.731042in}}%
\pgfpathcurveto{\pgfqpoint{1.059928in}{1.736866in}}{\pgfqpoint{1.052028in}{1.740139in}}{\pgfqpoint{1.043792in}{1.740139in}}%
\pgfpathcurveto{\pgfqpoint{1.035555in}{1.740139in}}{\pgfqpoint{1.027655in}{1.736866in}}{\pgfqpoint{1.021831in}{1.731042in}}%
\pgfpathcurveto{\pgfqpoint{1.016007in}{1.725218in}}{\pgfqpoint{1.012735in}{1.717318in}}{\pgfqpoint{1.012735in}{1.709082in}}%
\pgfpathcurveto{\pgfqpoint{1.012735in}{1.700846in}}{\pgfqpoint{1.016007in}{1.692946in}}{\pgfqpoint{1.021831in}{1.687122in}}%
\pgfpathcurveto{\pgfqpoint{1.027655in}{1.681298in}}{\pgfqpoint{1.035555in}{1.678026in}}{\pgfqpoint{1.043792in}{1.678026in}}%
\pgfpathclose%
\pgfusepath{stroke,fill}%
\end{pgfscope}%
\begin{pgfscope}%
\pgfpathrectangle{\pgfqpoint{0.100000in}{0.212622in}}{\pgfqpoint{3.696000in}{3.696000in}}%
\pgfusepath{clip}%
\pgfsetbuttcap%
\pgfsetroundjoin%
\definecolor{currentfill}{rgb}{0.121569,0.466667,0.705882}%
\pgfsetfillcolor{currentfill}%
\pgfsetfillopacity{0.551479}%
\pgfsetlinewidth{1.003750pt}%
\definecolor{currentstroke}{rgb}{0.121569,0.466667,0.705882}%
\pgfsetstrokecolor{currentstroke}%
\pgfsetstrokeopacity{0.551479}%
\pgfsetdash{}{0pt}%
\pgfpathmoveto{\pgfqpoint{1.041266in}{1.677341in}}%
\pgfpathcurveto{\pgfqpoint{1.049503in}{1.677341in}}{\pgfqpoint{1.057403in}{1.680613in}}{\pgfqpoint{1.063227in}{1.686437in}}%
\pgfpathcurveto{\pgfqpoint{1.069051in}{1.692261in}}{\pgfqpoint{1.072323in}{1.700161in}}{\pgfqpoint{1.072323in}{1.708397in}}%
\pgfpathcurveto{\pgfqpoint{1.072323in}{1.716634in}}{\pgfqpoint{1.069051in}{1.724534in}}{\pgfqpoint{1.063227in}{1.730357in}}%
\pgfpathcurveto{\pgfqpoint{1.057403in}{1.736181in}}{\pgfqpoint{1.049503in}{1.739454in}}{\pgfqpoint{1.041266in}{1.739454in}}%
\pgfpathcurveto{\pgfqpoint{1.033030in}{1.739454in}}{\pgfqpoint{1.025130in}{1.736181in}}{\pgfqpoint{1.019306in}{1.730357in}}%
\pgfpathcurveto{\pgfqpoint{1.013482in}{1.724534in}}{\pgfqpoint{1.010210in}{1.716634in}}{\pgfqpoint{1.010210in}{1.708397in}}%
\pgfpathcurveto{\pgfqpoint{1.010210in}{1.700161in}}{\pgfqpoint{1.013482in}{1.692261in}}{\pgfqpoint{1.019306in}{1.686437in}}%
\pgfpathcurveto{\pgfqpoint{1.025130in}{1.680613in}}{\pgfqpoint{1.033030in}{1.677341in}}{\pgfqpoint{1.041266in}{1.677341in}}%
\pgfpathclose%
\pgfusepath{stroke,fill}%
\end{pgfscope}%
\begin{pgfscope}%
\pgfpathrectangle{\pgfqpoint{0.100000in}{0.212622in}}{\pgfqpoint{3.696000in}{3.696000in}}%
\pgfusepath{clip}%
\pgfsetbuttcap%
\pgfsetroundjoin%
\definecolor{currentfill}{rgb}{0.121569,0.466667,0.705882}%
\pgfsetfillcolor{currentfill}%
\pgfsetfillopacity{0.551585}%
\pgfsetlinewidth{1.003750pt}%
\definecolor{currentstroke}{rgb}{0.121569,0.466667,0.705882}%
\pgfsetstrokecolor{currentstroke}%
\pgfsetstrokeopacity{0.551585}%
\pgfsetdash{}{0pt}%
\pgfpathmoveto{\pgfqpoint{3.188940in}{1.993250in}}%
\pgfpathcurveto{\pgfqpoint{3.197176in}{1.993250in}}{\pgfqpoint{3.205077in}{1.996522in}}{\pgfqpoint{3.210900in}{2.002346in}}%
\pgfpathcurveto{\pgfqpoint{3.216724in}{2.008170in}}{\pgfqpoint{3.219997in}{2.016070in}}{\pgfqpoint{3.219997in}{2.024306in}}%
\pgfpathcurveto{\pgfqpoint{3.219997in}{2.032542in}}{\pgfqpoint{3.216724in}{2.040442in}}{\pgfqpoint{3.210900in}{2.046266in}}%
\pgfpathcurveto{\pgfqpoint{3.205077in}{2.052090in}}{\pgfqpoint{3.197176in}{2.055363in}}{\pgfqpoint{3.188940in}{2.055363in}}%
\pgfpathcurveto{\pgfqpoint{3.180704in}{2.055363in}}{\pgfqpoint{3.172804in}{2.052090in}}{\pgfqpoint{3.166980in}{2.046266in}}%
\pgfpathcurveto{\pgfqpoint{3.161156in}{2.040442in}}{\pgfqpoint{3.157884in}{2.032542in}}{\pgfqpoint{3.157884in}{2.024306in}}%
\pgfpathcurveto{\pgfqpoint{3.157884in}{2.016070in}}{\pgfqpoint{3.161156in}{2.008170in}}{\pgfqpoint{3.166980in}{2.002346in}}%
\pgfpathcurveto{\pgfqpoint{3.172804in}{1.996522in}}{\pgfqpoint{3.180704in}{1.993250in}}{\pgfqpoint{3.188940in}{1.993250in}}%
\pgfpathclose%
\pgfusepath{stroke,fill}%
\end{pgfscope}%
\begin{pgfscope}%
\pgfpathrectangle{\pgfqpoint{0.100000in}{0.212622in}}{\pgfqpoint{3.696000in}{3.696000in}}%
\pgfusepath{clip}%
\pgfsetbuttcap%
\pgfsetroundjoin%
\definecolor{currentfill}{rgb}{0.121569,0.466667,0.705882}%
\pgfsetfillcolor{currentfill}%
\pgfsetfillopacity{0.551860}%
\pgfsetlinewidth{1.003750pt}%
\definecolor{currentstroke}{rgb}{0.121569,0.466667,0.705882}%
\pgfsetstrokecolor{currentstroke}%
\pgfsetstrokeopacity{0.551860}%
\pgfsetdash{}{0pt}%
\pgfpathmoveto{\pgfqpoint{1.040799in}{1.678032in}}%
\pgfpathcurveto{\pgfqpoint{1.049036in}{1.678032in}}{\pgfqpoint{1.056936in}{1.681305in}}{\pgfqpoint{1.062760in}{1.687129in}}%
\pgfpathcurveto{\pgfqpoint{1.068583in}{1.692952in}}{\pgfqpoint{1.071856in}{1.700853in}}{\pgfqpoint{1.071856in}{1.709089in}}%
\pgfpathcurveto{\pgfqpoint{1.071856in}{1.717325in}}{\pgfqpoint{1.068583in}{1.725225in}}{\pgfqpoint{1.062760in}{1.731049in}}%
\pgfpathcurveto{\pgfqpoint{1.056936in}{1.736873in}}{\pgfqpoint{1.049036in}{1.740145in}}{\pgfqpoint{1.040799in}{1.740145in}}%
\pgfpathcurveto{\pgfqpoint{1.032563in}{1.740145in}}{\pgfqpoint{1.024663in}{1.736873in}}{\pgfqpoint{1.018839in}{1.731049in}}%
\pgfpathcurveto{\pgfqpoint{1.013015in}{1.725225in}}{\pgfqpoint{1.009743in}{1.717325in}}{\pgfqpoint{1.009743in}{1.709089in}}%
\pgfpathcurveto{\pgfqpoint{1.009743in}{1.700853in}}{\pgfqpoint{1.013015in}{1.692952in}}{\pgfqpoint{1.018839in}{1.687129in}}%
\pgfpathcurveto{\pgfqpoint{1.024663in}{1.681305in}}{\pgfqpoint{1.032563in}{1.678032in}}{\pgfqpoint{1.040799in}{1.678032in}}%
\pgfpathclose%
\pgfusepath{stroke,fill}%
\end{pgfscope}%
\begin{pgfscope}%
\pgfpathrectangle{\pgfqpoint{0.100000in}{0.212622in}}{\pgfqpoint{3.696000in}{3.696000in}}%
\pgfusepath{clip}%
\pgfsetbuttcap%
\pgfsetroundjoin%
\definecolor{currentfill}{rgb}{0.121569,0.466667,0.705882}%
\pgfsetfillcolor{currentfill}%
\pgfsetfillopacity{0.552070}%
\pgfsetlinewidth{1.003750pt}%
\definecolor{currentstroke}{rgb}{0.121569,0.466667,0.705882}%
\pgfsetstrokecolor{currentstroke}%
\pgfsetstrokeopacity{0.552070}%
\pgfsetdash{}{0pt}%
\pgfpathmoveto{\pgfqpoint{1.037782in}{1.676296in}}%
\pgfpathcurveto{\pgfqpoint{1.046018in}{1.676296in}}{\pgfqpoint{1.053918in}{1.679568in}}{\pgfqpoint{1.059742in}{1.685392in}}%
\pgfpathcurveto{\pgfqpoint{1.065566in}{1.691216in}}{\pgfqpoint{1.068838in}{1.699116in}}{\pgfqpoint{1.068838in}{1.707352in}}%
\pgfpathcurveto{\pgfqpoint{1.068838in}{1.715588in}}{\pgfqpoint{1.065566in}{1.723488in}}{\pgfqpoint{1.059742in}{1.729312in}}%
\pgfpathcurveto{\pgfqpoint{1.053918in}{1.735136in}}{\pgfqpoint{1.046018in}{1.738409in}}{\pgfqpoint{1.037782in}{1.738409in}}%
\pgfpathcurveto{\pgfqpoint{1.029545in}{1.738409in}}{\pgfqpoint{1.021645in}{1.735136in}}{\pgfqpoint{1.015821in}{1.729312in}}%
\pgfpathcurveto{\pgfqpoint{1.009997in}{1.723488in}}{\pgfqpoint{1.006725in}{1.715588in}}{\pgfqpoint{1.006725in}{1.707352in}}%
\pgfpathcurveto{\pgfqpoint{1.006725in}{1.699116in}}{\pgfqpoint{1.009997in}{1.691216in}}{\pgfqpoint{1.015821in}{1.685392in}}%
\pgfpathcurveto{\pgfqpoint{1.021645in}{1.679568in}}{\pgfqpoint{1.029545in}{1.676296in}}{\pgfqpoint{1.037782in}{1.676296in}}%
\pgfpathclose%
\pgfusepath{stroke,fill}%
\end{pgfscope}%
\begin{pgfscope}%
\pgfpathrectangle{\pgfqpoint{0.100000in}{0.212622in}}{\pgfqpoint{3.696000in}{3.696000in}}%
\pgfusepath{clip}%
\pgfsetbuttcap%
\pgfsetroundjoin%
\definecolor{currentfill}{rgb}{0.121569,0.466667,0.705882}%
\pgfsetfillcolor{currentfill}%
\pgfsetfillopacity{0.552118}%
\pgfsetlinewidth{1.003750pt}%
\definecolor{currentstroke}{rgb}{0.121569,0.466667,0.705882}%
\pgfsetstrokecolor{currentstroke}%
\pgfsetstrokeopacity{0.552118}%
\pgfsetdash{}{0pt}%
\pgfpathmoveto{\pgfqpoint{1.038928in}{1.678492in}}%
\pgfpathcurveto{\pgfqpoint{1.047165in}{1.678492in}}{\pgfqpoint{1.055065in}{1.681764in}}{\pgfqpoint{1.060889in}{1.687588in}}%
\pgfpathcurveto{\pgfqpoint{1.066712in}{1.693412in}}{\pgfqpoint{1.069985in}{1.701312in}}{\pgfqpoint{1.069985in}{1.709548in}}%
\pgfpathcurveto{\pgfqpoint{1.069985in}{1.717784in}}{\pgfqpoint{1.066712in}{1.725684in}}{\pgfqpoint{1.060889in}{1.731508in}}%
\pgfpathcurveto{\pgfqpoint{1.055065in}{1.737332in}}{\pgfqpoint{1.047165in}{1.740605in}}{\pgfqpoint{1.038928in}{1.740605in}}%
\pgfpathcurveto{\pgfqpoint{1.030692in}{1.740605in}}{\pgfqpoint{1.022792in}{1.737332in}}{\pgfqpoint{1.016968in}{1.731508in}}%
\pgfpathcurveto{\pgfqpoint{1.011144in}{1.725684in}}{\pgfqpoint{1.007872in}{1.717784in}}{\pgfqpoint{1.007872in}{1.709548in}}%
\pgfpathcurveto{\pgfqpoint{1.007872in}{1.701312in}}{\pgfqpoint{1.011144in}{1.693412in}}{\pgfqpoint{1.016968in}{1.687588in}}%
\pgfpathcurveto{\pgfqpoint{1.022792in}{1.681764in}}{\pgfqpoint{1.030692in}{1.678492in}}{\pgfqpoint{1.038928in}{1.678492in}}%
\pgfpathclose%
\pgfusepath{stroke,fill}%
\end{pgfscope}%
\begin{pgfscope}%
\pgfpathrectangle{\pgfqpoint{0.100000in}{0.212622in}}{\pgfqpoint{3.696000in}{3.696000in}}%
\pgfusepath{clip}%
\pgfsetbuttcap%
\pgfsetroundjoin%
\definecolor{currentfill}{rgb}{0.121569,0.466667,0.705882}%
\pgfsetfillcolor{currentfill}%
\pgfsetfillopacity{0.552618}%
\pgfsetlinewidth{1.003750pt}%
\definecolor{currentstroke}{rgb}{0.121569,0.466667,0.705882}%
\pgfsetstrokecolor{currentstroke}%
\pgfsetstrokeopacity{0.552618}%
\pgfsetdash{}{0pt}%
\pgfpathmoveto{\pgfqpoint{3.197081in}{1.993592in}}%
\pgfpathcurveto{\pgfqpoint{3.205317in}{1.993592in}}{\pgfqpoint{3.213217in}{1.996864in}}{\pgfqpoint{3.219041in}{2.002688in}}%
\pgfpathcurveto{\pgfqpoint{3.224865in}{2.008512in}}{\pgfqpoint{3.228137in}{2.016412in}}{\pgfqpoint{3.228137in}{2.024648in}}%
\pgfpathcurveto{\pgfqpoint{3.228137in}{2.032884in}}{\pgfqpoint{3.224865in}{2.040784in}}{\pgfqpoint{3.219041in}{2.046608in}}%
\pgfpathcurveto{\pgfqpoint{3.213217in}{2.052432in}}{\pgfqpoint{3.205317in}{2.055705in}}{\pgfqpoint{3.197081in}{2.055705in}}%
\pgfpathcurveto{\pgfqpoint{3.188845in}{2.055705in}}{\pgfqpoint{3.180945in}{2.052432in}}{\pgfqpoint{3.175121in}{2.046608in}}%
\pgfpathcurveto{\pgfqpoint{3.169297in}{2.040784in}}{\pgfqpoint{3.166024in}{2.032884in}}{\pgfqpoint{3.166024in}{2.024648in}}%
\pgfpathcurveto{\pgfqpoint{3.166024in}{2.016412in}}{\pgfqpoint{3.169297in}{2.008512in}}{\pgfqpoint{3.175121in}{2.002688in}}%
\pgfpathcurveto{\pgfqpoint{3.180945in}{1.996864in}}{\pgfqpoint{3.188845in}{1.993592in}}{\pgfqpoint{3.197081in}{1.993592in}}%
\pgfpathclose%
\pgfusepath{stroke,fill}%
\end{pgfscope}%
\begin{pgfscope}%
\pgfpathrectangle{\pgfqpoint{0.100000in}{0.212622in}}{\pgfqpoint{3.696000in}{3.696000in}}%
\pgfusepath{clip}%
\pgfsetbuttcap%
\pgfsetroundjoin%
\definecolor{currentfill}{rgb}{0.121569,0.466667,0.705882}%
\pgfsetfillcolor{currentfill}%
\pgfsetfillopacity{0.552672}%
\pgfsetlinewidth{1.003750pt}%
\definecolor{currentstroke}{rgb}{0.121569,0.466667,0.705882}%
\pgfsetstrokecolor{currentstroke}%
\pgfsetstrokeopacity{0.552672}%
\pgfsetdash{}{0pt}%
\pgfpathmoveto{\pgfqpoint{1.036890in}{1.675388in}}%
\pgfpathcurveto{\pgfqpoint{1.045126in}{1.675388in}}{\pgfqpoint{1.053026in}{1.678660in}}{\pgfqpoint{1.058850in}{1.684484in}}%
\pgfpathcurveto{\pgfqpoint{1.064674in}{1.690308in}}{\pgfqpoint{1.067946in}{1.698208in}}{\pgfqpoint{1.067946in}{1.706445in}}%
\pgfpathcurveto{\pgfqpoint{1.067946in}{1.714681in}}{\pgfqpoint{1.064674in}{1.722581in}}{\pgfqpoint{1.058850in}{1.728405in}}%
\pgfpathcurveto{\pgfqpoint{1.053026in}{1.734229in}}{\pgfqpoint{1.045126in}{1.737501in}}{\pgfqpoint{1.036890in}{1.737501in}}%
\pgfpathcurveto{\pgfqpoint{1.028654in}{1.737501in}}{\pgfqpoint{1.020753in}{1.734229in}}{\pgfqpoint{1.014930in}{1.728405in}}%
\pgfpathcurveto{\pgfqpoint{1.009106in}{1.722581in}}{\pgfqpoint{1.005833in}{1.714681in}}{\pgfqpoint{1.005833in}{1.706445in}}%
\pgfpathcurveto{\pgfqpoint{1.005833in}{1.698208in}}{\pgfqpoint{1.009106in}{1.690308in}}{\pgfqpoint{1.014930in}{1.684484in}}%
\pgfpathcurveto{\pgfqpoint{1.020753in}{1.678660in}}{\pgfqpoint{1.028654in}{1.675388in}}{\pgfqpoint{1.036890in}{1.675388in}}%
\pgfpathclose%
\pgfusepath{stroke,fill}%
\end{pgfscope}%
\begin{pgfscope}%
\pgfpathrectangle{\pgfqpoint{0.100000in}{0.212622in}}{\pgfqpoint{3.696000in}{3.696000in}}%
\pgfusepath{clip}%
\pgfsetbuttcap%
\pgfsetroundjoin%
\definecolor{currentfill}{rgb}{0.121569,0.466667,0.705882}%
\pgfsetfillcolor{currentfill}%
\pgfsetfillopacity{0.553510}%
\pgfsetlinewidth{1.003750pt}%
\definecolor{currentstroke}{rgb}{0.121569,0.466667,0.705882}%
\pgfsetstrokecolor{currentstroke}%
\pgfsetstrokeopacity{0.553510}%
\pgfsetdash{}{0pt}%
\pgfpathmoveto{\pgfqpoint{1.033575in}{1.673792in}}%
\pgfpathcurveto{\pgfqpoint{1.041812in}{1.673792in}}{\pgfqpoint{1.049712in}{1.677064in}}{\pgfqpoint{1.055536in}{1.682888in}}%
\pgfpathcurveto{\pgfqpoint{1.061360in}{1.688712in}}{\pgfqpoint{1.064632in}{1.696612in}}{\pgfqpoint{1.064632in}{1.704848in}}%
\pgfpathcurveto{\pgfqpoint{1.064632in}{1.713084in}}{\pgfqpoint{1.061360in}{1.720985in}}{\pgfqpoint{1.055536in}{1.726808in}}%
\pgfpathcurveto{\pgfqpoint{1.049712in}{1.732632in}}{\pgfqpoint{1.041812in}{1.735905in}}{\pgfqpoint{1.033575in}{1.735905in}}%
\pgfpathcurveto{\pgfqpoint{1.025339in}{1.735905in}}{\pgfqpoint{1.017439in}{1.732632in}}{\pgfqpoint{1.011615in}{1.726808in}}%
\pgfpathcurveto{\pgfqpoint{1.005791in}{1.720985in}}{\pgfqpoint{1.002519in}{1.713084in}}{\pgfqpoint{1.002519in}{1.704848in}}%
\pgfpathcurveto{\pgfqpoint{1.002519in}{1.696612in}}{\pgfqpoint{1.005791in}{1.688712in}}{\pgfqpoint{1.011615in}{1.682888in}}%
\pgfpathcurveto{\pgfqpoint{1.017439in}{1.677064in}}{\pgfqpoint{1.025339in}{1.673792in}}{\pgfqpoint{1.033575in}{1.673792in}}%
\pgfpathclose%
\pgfusepath{stroke,fill}%
\end{pgfscope}%
\begin{pgfscope}%
\pgfpathrectangle{\pgfqpoint{0.100000in}{0.212622in}}{\pgfqpoint{3.696000in}{3.696000in}}%
\pgfusepath{clip}%
\pgfsetbuttcap%
\pgfsetroundjoin%
\definecolor{currentfill}{rgb}{0.121569,0.466667,0.705882}%
\pgfsetfillcolor{currentfill}%
\pgfsetfillopacity{0.553873}%
\pgfsetlinewidth{1.003750pt}%
\definecolor{currentstroke}{rgb}{0.121569,0.466667,0.705882}%
\pgfsetstrokecolor{currentstroke}%
\pgfsetstrokeopacity{0.553873}%
\pgfsetdash{}{0pt}%
\pgfpathmoveto{\pgfqpoint{3.206064in}{1.994376in}}%
\pgfpathcurveto{\pgfqpoint{3.214300in}{1.994376in}}{\pgfqpoint{3.222200in}{1.997648in}}{\pgfqpoint{3.228024in}{2.003472in}}%
\pgfpathcurveto{\pgfqpoint{3.233848in}{2.009296in}}{\pgfqpoint{3.237120in}{2.017196in}}{\pgfqpoint{3.237120in}{2.025433in}}%
\pgfpathcurveto{\pgfqpoint{3.237120in}{2.033669in}}{\pgfqpoint{3.233848in}{2.041569in}}{\pgfqpoint{3.228024in}{2.047393in}}%
\pgfpathcurveto{\pgfqpoint{3.222200in}{2.053217in}}{\pgfqpoint{3.214300in}{2.056489in}}{\pgfqpoint{3.206064in}{2.056489in}}%
\pgfpathcurveto{\pgfqpoint{3.197828in}{2.056489in}}{\pgfqpoint{3.189928in}{2.053217in}}{\pgfqpoint{3.184104in}{2.047393in}}%
\pgfpathcurveto{\pgfqpoint{3.178280in}{2.041569in}}{\pgfqpoint{3.175007in}{2.033669in}}{\pgfqpoint{3.175007in}{2.025433in}}%
\pgfpathcurveto{\pgfqpoint{3.175007in}{2.017196in}}{\pgfqpoint{3.178280in}{2.009296in}}{\pgfqpoint{3.184104in}{2.003472in}}%
\pgfpathcurveto{\pgfqpoint{3.189928in}{1.997648in}}{\pgfqpoint{3.197828in}{1.994376in}}{\pgfqpoint{3.206064in}{1.994376in}}%
\pgfpathclose%
\pgfusepath{stroke,fill}%
\end{pgfscope}%
\begin{pgfscope}%
\pgfpathrectangle{\pgfqpoint{0.100000in}{0.212622in}}{\pgfqpoint{3.696000in}{3.696000in}}%
\pgfusepath{clip}%
\pgfsetbuttcap%
\pgfsetroundjoin%
\definecolor{currentfill}{rgb}{0.121569,0.466667,0.705882}%
\pgfsetfillcolor{currentfill}%
\pgfsetfillopacity{0.554031}%
\pgfsetlinewidth{1.003750pt}%
\definecolor{currentstroke}{rgb}{0.121569,0.466667,0.705882}%
\pgfsetstrokecolor{currentstroke}%
\pgfsetstrokeopacity{0.554031}%
\pgfsetdash{}{0pt}%
\pgfpathmoveto{\pgfqpoint{1.033044in}{1.673136in}}%
\pgfpathcurveto{\pgfqpoint{1.041280in}{1.673136in}}{\pgfqpoint{1.049180in}{1.676408in}}{\pgfqpoint{1.055004in}{1.682232in}}%
\pgfpathcurveto{\pgfqpoint{1.060828in}{1.688056in}}{\pgfqpoint{1.064100in}{1.695956in}}{\pgfqpoint{1.064100in}{1.704192in}}%
\pgfpathcurveto{\pgfqpoint{1.064100in}{1.712429in}}{\pgfqpoint{1.060828in}{1.720329in}}{\pgfqpoint{1.055004in}{1.726153in}}%
\pgfpathcurveto{\pgfqpoint{1.049180in}{1.731977in}}{\pgfqpoint{1.041280in}{1.735249in}}{\pgfqpoint{1.033044in}{1.735249in}}%
\pgfpathcurveto{\pgfqpoint{1.024807in}{1.735249in}}{\pgfqpoint{1.016907in}{1.731977in}}{\pgfqpoint{1.011083in}{1.726153in}}%
\pgfpathcurveto{\pgfqpoint{1.005260in}{1.720329in}}{\pgfqpoint{1.001987in}{1.712429in}}{\pgfqpoint{1.001987in}{1.704192in}}%
\pgfpathcurveto{\pgfqpoint{1.001987in}{1.695956in}}{\pgfqpoint{1.005260in}{1.688056in}}{\pgfqpoint{1.011083in}{1.682232in}}%
\pgfpathcurveto{\pgfqpoint{1.016907in}{1.676408in}}{\pgfqpoint{1.024807in}{1.673136in}}{\pgfqpoint{1.033044in}{1.673136in}}%
\pgfpathclose%
\pgfusepath{stroke,fill}%
\end{pgfscope}%
\begin{pgfscope}%
\pgfpathrectangle{\pgfqpoint{0.100000in}{0.212622in}}{\pgfqpoint{3.696000in}{3.696000in}}%
\pgfusepath{clip}%
\pgfsetbuttcap%
\pgfsetroundjoin%
\definecolor{currentfill}{rgb}{0.121569,0.466667,0.705882}%
\pgfsetfillcolor{currentfill}%
\pgfsetfillopacity{0.554715}%
\pgfsetlinewidth{1.003750pt}%
\definecolor{currentstroke}{rgb}{0.121569,0.466667,0.705882}%
\pgfsetstrokecolor{currentstroke}%
\pgfsetstrokeopacity{0.554715}%
\pgfsetdash{}{0pt}%
\pgfpathmoveto{\pgfqpoint{1.029919in}{1.672578in}}%
\pgfpathcurveto{\pgfqpoint{1.038155in}{1.672578in}}{\pgfqpoint{1.046055in}{1.675850in}}{\pgfqpoint{1.051879in}{1.681674in}}%
\pgfpathcurveto{\pgfqpoint{1.057703in}{1.687498in}}{\pgfqpoint{1.060975in}{1.695398in}}{\pgfqpoint{1.060975in}{1.703634in}}%
\pgfpathcurveto{\pgfqpoint{1.060975in}{1.711871in}}{\pgfqpoint{1.057703in}{1.719771in}}{\pgfqpoint{1.051879in}{1.725595in}}%
\pgfpathcurveto{\pgfqpoint{1.046055in}{1.731419in}}{\pgfqpoint{1.038155in}{1.734691in}}{\pgfqpoint{1.029919in}{1.734691in}}%
\pgfpathcurveto{\pgfqpoint{1.021683in}{1.734691in}}{\pgfqpoint{1.013783in}{1.731419in}}{\pgfqpoint{1.007959in}{1.725595in}}%
\pgfpathcurveto{\pgfqpoint{1.002135in}{1.719771in}}{\pgfqpoint{0.998862in}{1.711871in}}{\pgfqpoint{0.998862in}{1.703634in}}%
\pgfpathcurveto{\pgfqpoint{0.998862in}{1.695398in}}{\pgfqpoint{1.002135in}{1.687498in}}{\pgfqpoint{1.007959in}{1.681674in}}%
\pgfpathcurveto{\pgfqpoint{1.013783in}{1.675850in}}{\pgfqpoint{1.021683in}{1.672578in}}{\pgfqpoint{1.029919in}{1.672578in}}%
\pgfpathclose%
\pgfusepath{stroke,fill}%
\end{pgfscope}%
\begin{pgfscope}%
\pgfpathrectangle{\pgfqpoint{0.100000in}{0.212622in}}{\pgfqpoint{3.696000in}{3.696000in}}%
\pgfusepath{clip}%
\pgfsetbuttcap%
\pgfsetroundjoin%
\definecolor{currentfill}{rgb}{0.121569,0.466667,0.705882}%
\pgfsetfillcolor{currentfill}%
\pgfsetfillopacity{0.554900}%
\pgfsetlinewidth{1.003750pt}%
\definecolor{currentstroke}{rgb}{0.121569,0.466667,0.705882}%
\pgfsetstrokecolor{currentstroke}%
\pgfsetstrokeopacity{0.554900}%
\pgfsetdash{}{0pt}%
\pgfpathmoveto{\pgfqpoint{3.215096in}{1.992236in}}%
\pgfpathcurveto{\pgfqpoint{3.223332in}{1.992236in}}{\pgfqpoint{3.231232in}{1.995508in}}{\pgfqpoint{3.237056in}{2.001332in}}%
\pgfpathcurveto{\pgfqpoint{3.242880in}{2.007156in}}{\pgfqpoint{3.246152in}{2.015056in}}{\pgfqpoint{3.246152in}{2.023292in}}%
\pgfpathcurveto{\pgfqpoint{3.246152in}{2.031529in}}{\pgfqpoint{3.242880in}{2.039429in}}{\pgfqpoint{3.237056in}{2.045253in}}%
\pgfpathcurveto{\pgfqpoint{3.231232in}{2.051076in}}{\pgfqpoint{3.223332in}{2.054349in}}{\pgfqpoint{3.215096in}{2.054349in}}%
\pgfpathcurveto{\pgfqpoint{3.206860in}{2.054349in}}{\pgfqpoint{3.198960in}{2.051076in}}{\pgfqpoint{3.193136in}{2.045253in}}%
\pgfpathcurveto{\pgfqpoint{3.187312in}{2.039429in}}{\pgfqpoint{3.184039in}{2.031529in}}{\pgfqpoint{3.184039in}{2.023292in}}%
\pgfpathcurveto{\pgfqpoint{3.184039in}{2.015056in}}{\pgfqpoint{3.187312in}{2.007156in}}{\pgfqpoint{3.193136in}{2.001332in}}%
\pgfpathcurveto{\pgfqpoint{3.198960in}{1.995508in}}{\pgfqpoint{3.206860in}{1.992236in}}{\pgfqpoint{3.215096in}{1.992236in}}%
\pgfpathclose%
\pgfusepath{stroke,fill}%
\end{pgfscope}%
\begin{pgfscope}%
\pgfpathrectangle{\pgfqpoint{0.100000in}{0.212622in}}{\pgfqpoint{3.696000in}{3.696000in}}%
\pgfusepath{clip}%
\pgfsetbuttcap%
\pgfsetroundjoin%
\definecolor{currentfill}{rgb}{0.121569,0.466667,0.705882}%
\pgfsetfillcolor{currentfill}%
\pgfsetfillopacity{0.556204}%
\pgfsetlinewidth{1.003750pt}%
\definecolor{currentstroke}{rgb}{0.121569,0.466667,0.705882}%
\pgfsetstrokecolor{currentstroke}%
\pgfsetstrokeopacity{0.556204}%
\pgfsetdash{}{0pt}%
\pgfpathmoveto{\pgfqpoint{1.026625in}{1.669725in}}%
\pgfpathcurveto{\pgfqpoint{1.034861in}{1.669725in}}{\pgfqpoint{1.042761in}{1.672998in}}{\pgfqpoint{1.048585in}{1.678822in}}%
\pgfpathcurveto{\pgfqpoint{1.054409in}{1.684645in}}{\pgfqpoint{1.057682in}{1.692546in}}{\pgfqpoint{1.057682in}{1.700782in}}%
\pgfpathcurveto{\pgfqpoint{1.057682in}{1.709018in}}{\pgfqpoint{1.054409in}{1.716918in}}{\pgfqpoint{1.048585in}{1.722742in}}%
\pgfpathcurveto{\pgfqpoint{1.042761in}{1.728566in}}{\pgfqpoint{1.034861in}{1.731838in}}{\pgfqpoint{1.026625in}{1.731838in}}%
\pgfpathcurveto{\pgfqpoint{1.018389in}{1.731838in}}{\pgfqpoint{1.010489in}{1.728566in}}{\pgfqpoint{1.004665in}{1.722742in}}%
\pgfpathcurveto{\pgfqpoint{0.998841in}{1.716918in}}{\pgfqpoint{0.995569in}{1.709018in}}{\pgfqpoint{0.995569in}{1.700782in}}%
\pgfpathcurveto{\pgfqpoint{0.995569in}{1.692546in}}{\pgfqpoint{0.998841in}{1.684645in}}{\pgfqpoint{1.004665in}{1.678822in}}%
\pgfpathcurveto{\pgfqpoint{1.010489in}{1.672998in}}{\pgfqpoint{1.018389in}{1.669725in}}{\pgfqpoint{1.026625in}{1.669725in}}%
\pgfpathclose%
\pgfusepath{stroke,fill}%
\end{pgfscope}%
\begin{pgfscope}%
\pgfpathrectangle{\pgfqpoint{0.100000in}{0.212622in}}{\pgfqpoint{3.696000in}{3.696000in}}%
\pgfusepath{clip}%
\pgfsetbuttcap%
\pgfsetroundjoin%
\definecolor{currentfill}{rgb}{0.121569,0.466667,0.705882}%
\pgfsetfillcolor{currentfill}%
\pgfsetfillopacity{0.556669}%
\pgfsetlinewidth{1.003750pt}%
\definecolor{currentstroke}{rgb}{0.121569,0.466667,0.705882}%
\pgfsetstrokecolor{currentstroke}%
\pgfsetstrokeopacity{0.556669}%
\pgfsetdash{}{0pt}%
\pgfpathmoveto{\pgfqpoint{3.225279in}{1.994049in}}%
\pgfpathcurveto{\pgfqpoint{3.233515in}{1.994049in}}{\pgfqpoint{3.241416in}{1.997321in}}{\pgfqpoint{3.247239in}{2.003145in}}%
\pgfpathcurveto{\pgfqpoint{3.253063in}{2.008969in}}{\pgfqpoint{3.256336in}{2.016869in}}{\pgfqpoint{3.256336in}{2.025105in}}%
\pgfpathcurveto{\pgfqpoint{3.256336in}{2.033341in}}{\pgfqpoint{3.253063in}{2.041242in}}{\pgfqpoint{3.247239in}{2.047065in}}%
\pgfpathcurveto{\pgfqpoint{3.241416in}{2.052889in}}{\pgfqpoint{3.233515in}{2.056162in}}{\pgfqpoint{3.225279in}{2.056162in}}%
\pgfpathcurveto{\pgfqpoint{3.217043in}{2.056162in}}{\pgfqpoint{3.209143in}{2.052889in}}{\pgfqpoint{3.203319in}{2.047065in}}%
\pgfpathcurveto{\pgfqpoint{3.197495in}{2.041242in}}{\pgfqpoint{3.194223in}{2.033341in}}{\pgfqpoint{3.194223in}{2.025105in}}%
\pgfpathcurveto{\pgfqpoint{3.194223in}{2.016869in}}{\pgfqpoint{3.197495in}{2.008969in}}{\pgfqpoint{3.203319in}{2.003145in}}%
\pgfpathcurveto{\pgfqpoint{3.209143in}{1.997321in}}{\pgfqpoint{3.217043in}{1.994049in}}{\pgfqpoint{3.225279in}{1.994049in}}%
\pgfpathclose%
\pgfusepath{stroke,fill}%
\end{pgfscope}%
\begin{pgfscope}%
\pgfpathrectangle{\pgfqpoint{0.100000in}{0.212622in}}{\pgfqpoint{3.696000in}{3.696000in}}%
\pgfusepath{clip}%
\pgfsetbuttcap%
\pgfsetroundjoin%
\definecolor{currentfill}{rgb}{0.121569,0.466667,0.705882}%
\pgfsetfillcolor{currentfill}%
\pgfsetfillopacity{0.558394}%
\pgfsetlinewidth{1.003750pt}%
\definecolor{currentstroke}{rgb}{0.121569,0.466667,0.705882}%
\pgfsetstrokecolor{currentstroke}%
\pgfsetstrokeopacity{0.558394}%
\pgfsetdash{}{0pt}%
\pgfpathmoveto{\pgfqpoint{3.235855in}{1.994521in}}%
\pgfpathcurveto{\pgfqpoint{3.244091in}{1.994521in}}{\pgfqpoint{3.251991in}{1.997794in}}{\pgfqpoint{3.257815in}{2.003618in}}%
\pgfpathcurveto{\pgfqpoint{3.263639in}{2.009442in}}{\pgfqpoint{3.266912in}{2.017342in}}{\pgfqpoint{3.266912in}{2.025578in}}%
\pgfpathcurveto{\pgfqpoint{3.266912in}{2.033814in}}{\pgfqpoint{3.263639in}{2.041714in}}{\pgfqpoint{3.257815in}{2.047538in}}%
\pgfpathcurveto{\pgfqpoint{3.251991in}{2.053362in}}{\pgfqpoint{3.244091in}{2.056634in}}{\pgfqpoint{3.235855in}{2.056634in}}%
\pgfpathcurveto{\pgfqpoint{3.227619in}{2.056634in}}{\pgfqpoint{3.219719in}{2.053362in}}{\pgfqpoint{3.213895in}{2.047538in}}%
\pgfpathcurveto{\pgfqpoint{3.208071in}{2.041714in}}{\pgfqpoint{3.204799in}{2.033814in}}{\pgfqpoint{3.204799in}{2.025578in}}%
\pgfpathcurveto{\pgfqpoint{3.204799in}{2.017342in}}{\pgfqpoint{3.208071in}{2.009442in}}{\pgfqpoint{3.213895in}{2.003618in}}%
\pgfpathcurveto{\pgfqpoint{3.219719in}{1.997794in}}{\pgfqpoint{3.227619in}{1.994521in}}{\pgfqpoint{3.235855in}{1.994521in}}%
\pgfpathclose%
\pgfusepath{stroke,fill}%
\end{pgfscope}%
\begin{pgfscope}%
\pgfpathrectangle{\pgfqpoint{0.100000in}{0.212622in}}{\pgfqpoint{3.696000in}{3.696000in}}%
\pgfusepath{clip}%
\pgfsetbuttcap%
\pgfsetroundjoin%
\definecolor{currentfill}{rgb}{0.121569,0.466667,0.705882}%
\pgfsetfillcolor{currentfill}%
\pgfsetfillopacity{0.559945}%
\pgfsetlinewidth{1.003750pt}%
\definecolor{currentstroke}{rgb}{0.121569,0.466667,0.705882}%
\pgfsetstrokecolor{currentstroke}%
\pgfsetstrokeopacity{0.559945}%
\pgfsetdash{}{0pt}%
\pgfpathmoveto{\pgfqpoint{3.246836in}{1.993069in}}%
\pgfpathcurveto{\pgfqpoint{3.255073in}{1.993069in}}{\pgfqpoint{3.262973in}{1.996342in}}{\pgfqpoint{3.268797in}{2.002166in}}%
\pgfpathcurveto{\pgfqpoint{3.274621in}{2.007990in}}{\pgfqpoint{3.277893in}{2.015890in}}{\pgfqpoint{3.277893in}{2.024126in}}%
\pgfpathcurveto{\pgfqpoint{3.277893in}{2.032362in}}{\pgfqpoint{3.274621in}{2.040262in}}{\pgfqpoint{3.268797in}{2.046086in}}%
\pgfpathcurveto{\pgfqpoint{3.262973in}{2.051910in}}{\pgfqpoint{3.255073in}{2.055182in}}{\pgfqpoint{3.246836in}{2.055182in}}%
\pgfpathcurveto{\pgfqpoint{3.238600in}{2.055182in}}{\pgfqpoint{3.230700in}{2.051910in}}{\pgfqpoint{3.224876in}{2.046086in}}%
\pgfpathcurveto{\pgfqpoint{3.219052in}{2.040262in}}{\pgfqpoint{3.215780in}{2.032362in}}{\pgfqpoint{3.215780in}{2.024126in}}%
\pgfpathcurveto{\pgfqpoint{3.215780in}{2.015890in}}{\pgfqpoint{3.219052in}{2.007990in}}{\pgfqpoint{3.224876in}{2.002166in}}%
\pgfpathcurveto{\pgfqpoint{3.230700in}{1.996342in}}{\pgfqpoint{3.238600in}{1.993069in}}{\pgfqpoint{3.246836in}{1.993069in}}%
\pgfpathclose%
\pgfusepath{stroke,fill}%
\end{pgfscope}%
\begin{pgfscope}%
\pgfpathrectangle{\pgfqpoint{0.100000in}{0.212622in}}{\pgfqpoint{3.696000in}{3.696000in}}%
\pgfusepath{clip}%
\pgfsetbuttcap%
\pgfsetroundjoin%
\definecolor{currentfill}{rgb}{0.121569,0.466667,0.705882}%
\pgfsetfillcolor{currentfill}%
\pgfsetfillopacity{0.559953}%
\pgfsetlinewidth{1.003750pt}%
\definecolor{currentstroke}{rgb}{0.121569,0.466667,0.705882}%
\pgfsetstrokecolor{currentstroke}%
\pgfsetstrokeopacity{0.559953}%
\pgfsetdash{}{0pt}%
\pgfpathmoveto{\pgfqpoint{1.020384in}{1.671923in}}%
\pgfpathcurveto{\pgfqpoint{1.028620in}{1.671923in}}{\pgfqpoint{1.036520in}{1.675196in}}{\pgfqpoint{1.042344in}{1.681020in}}%
\pgfpathcurveto{\pgfqpoint{1.048168in}{1.686844in}}{\pgfqpoint{1.051441in}{1.694744in}}{\pgfqpoint{1.051441in}{1.702980in}}%
\pgfpathcurveto{\pgfqpoint{1.051441in}{1.711216in}}{\pgfqpoint{1.048168in}{1.719116in}}{\pgfqpoint{1.042344in}{1.724940in}}%
\pgfpathcurveto{\pgfqpoint{1.036520in}{1.730764in}}{\pgfqpoint{1.028620in}{1.734036in}}{\pgfqpoint{1.020384in}{1.734036in}}%
\pgfpathcurveto{\pgfqpoint{1.012148in}{1.734036in}}{\pgfqpoint{1.004248in}{1.730764in}}{\pgfqpoint{0.998424in}{1.724940in}}%
\pgfpathcurveto{\pgfqpoint{0.992600in}{1.719116in}}{\pgfqpoint{0.989328in}{1.711216in}}{\pgfqpoint{0.989328in}{1.702980in}}%
\pgfpathcurveto{\pgfqpoint{0.989328in}{1.694744in}}{\pgfqpoint{0.992600in}{1.686844in}}{\pgfqpoint{0.998424in}{1.681020in}}%
\pgfpathcurveto{\pgfqpoint{1.004248in}{1.675196in}}{\pgfqpoint{1.012148in}{1.671923in}}{\pgfqpoint{1.020384in}{1.671923in}}%
\pgfpathclose%
\pgfusepath{stroke,fill}%
\end{pgfscope}%
\begin{pgfscope}%
\pgfpathrectangle{\pgfqpoint{0.100000in}{0.212622in}}{\pgfqpoint{3.696000in}{3.696000in}}%
\pgfusepath{clip}%
\pgfsetbuttcap%
\pgfsetroundjoin%
\definecolor{currentfill}{rgb}{0.121569,0.466667,0.705882}%
\pgfsetfillcolor{currentfill}%
\pgfsetfillopacity{0.560900}%
\pgfsetlinewidth{1.003750pt}%
\definecolor{currentstroke}{rgb}{0.121569,0.466667,0.705882}%
\pgfsetstrokecolor{currentstroke}%
\pgfsetstrokeopacity{0.560900}%
\pgfsetdash{}{0pt}%
\pgfpathmoveto{\pgfqpoint{3.252918in}{1.993039in}}%
\pgfpathcurveto{\pgfqpoint{3.261154in}{1.993039in}}{\pgfqpoint{3.269054in}{1.996311in}}{\pgfqpoint{3.274878in}{2.002135in}}%
\pgfpathcurveto{\pgfqpoint{3.280702in}{2.007959in}}{\pgfqpoint{3.283975in}{2.015859in}}{\pgfqpoint{3.283975in}{2.024095in}}%
\pgfpathcurveto{\pgfqpoint{3.283975in}{2.032332in}}{\pgfqpoint{3.280702in}{2.040232in}}{\pgfqpoint{3.274878in}{2.046056in}}%
\pgfpathcurveto{\pgfqpoint{3.269054in}{2.051880in}}{\pgfqpoint{3.261154in}{2.055152in}}{\pgfqpoint{3.252918in}{2.055152in}}%
\pgfpathcurveto{\pgfqpoint{3.244682in}{2.055152in}}{\pgfqpoint{3.236782in}{2.051880in}}{\pgfqpoint{3.230958in}{2.046056in}}%
\pgfpathcurveto{\pgfqpoint{3.225134in}{2.040232in}}{\pgfqpoint{3.221862in}{2.032332in}}{\pgfqpoint{3.221862in}{2.024095in}}%
\pgfpathcurveto{\pgfqpoint{3.221862in}{2.015859in}}{\pgfqpoint{3.225134in}{2.007959in}}{\pgfqpoint{3.230958in}{2.002135in}}%
\pgfpathcurveto{\pgfqpoint{3.236782in}{1.996311in}}{\pgfqpoint{3.244682in}{1.993039in}}{\pgfqpoint{3.252918in}{1.993039in}}%
\pgfpathclose%
\pgfusepath{stroke,fill}%
\end{pgfscope}%
\begin{pgfscope}%
\pgfpathrectangle{\pgfqpoint{0.100000in}{0.212622in}}{\pgfqpoint{3.696000in}{3.696000in}}%
\pgfusepath{clip}%
\pgfsetbuttcap%
\pgfsetroundjoin%
\definecolor{currentfill}{rgb}{0.121569,0.466667,0.705882}%
\pgfsetfillcolor{currentfill}%
\pgfsetfillopacity{0.562063}%
\pgfsetlinewidth{1.003750pt}%
\definecolor{currentstroke}{rgb}{0.121569,0.466667,0.705882}%
\pgfsetstrokecolor{currentstroke}%
\pgfsetstrokeopacity{0.562063}%
\pgfsetdash{}{0pt}%
\pgfpathmoveto{\pgfqpoint{3.259737in}{1.993972in}}%
\pgfpathcurveto{\pgfqpoint{3.267973in}{1.993972in}}{\pgfqpoint{3.275873in}{1.997245in}}{\pgfqpoint{3.281697in}{2.003069in}}%
\pgfpathcurveto{\pgfqpoint{3.287521in}{2.008893in}}{\pgfqpoint{3.290793in}{2.016793in}}{\pgfqpoint{3.290793in}{2.025029in}}%
\pgfpathcurveto{\pgfqpoint{3.290793in}{2.033265in}}{\pgfqpoint{3.287521in}{2.041165in}}{\pgfqpoint{3.281697in}{2.046989in}}%
\pgfpathcurveto{\pgfqpoint{3.275873in}{2.052813in}}{\pgfqpoint{3.267973in}{2.056085in}}{\pgfqpoint{3.259737in}{2.056085in}}%
\pgfpathcurveto{\pgfqpoint{3.251501in}{2.056085in}}{\pgfqpoint{3.243601in}{2.052813in}}{\pgfqpoint{3.237777in}{2.046989in}}%
\pgfpathcurveto{\pgfqpoint{3.231953in}{2.041165in}}{\pgfqpoint{3.228680in}{2.033265in}}{\pgfqpoint{3.228680in}{2.025029in}}%
\pgfpathcurveto{\pgfqpoint{3.228680in}{2.016793in}}{\pgfqpoint{3.231953in}{2.008893in}}{\pgfqpoint{3.237777in}{2.003069in}}%
\pgfpathcurveto{\pgfqpoint{3.243601in}{1.997245in}}{\pgfqpoint{3.251501in}{1.993972in}}{\pgfqpoint{3.259737in}{1.993972in}}%
\pgfpathclose%
\pgfusepath{stroke,fill}%
\end{pgfscope}%
\begin{pgfscope}%
\pgfpathrectangle{\pgfqpoint{0.100000in}{0.212622in}}{\pgfqpoint{3.696000in}{3.696000in}}%
\pgfusepath{clip}%
\pgfsetbuttcap%
\pgfsetroundjoin%
\definecolor{currentfill}{rgb}{0.121569,0.466667,0.705882}%
\pgfsetfillcolor{currentfill}%
\pgfsetfillopacity{0.563467}%
\pgfsetlinewidth{1.003750pt}%
\definecolor{currentstroke}{rgb}{0.121569,0.466667,0.705882}%
\pgfsetstrokecolor{currentstroke}%
\pgfsetstrokeopacity{0.563467}%
\pgfsetdash{}{0pt}%
\pgfpathmoveto{\pgfqpoint{3.266894in}{1.995615in}}%
\pgfpathcurveto{\pgfqpoint{3.275131in}{1.995615in}}{\pgfqpoint{3.283031in}{1.998887in}}{\pgfqpoint{3.288855in}{2.004711in}}%
\pgfpathcurveto{\pgfqpoint{3.294678in}{2.010535in}}{\pgfqpoint{3.297951in}{2.018435in}}{\pgfqpoint{3.297951in}{2.026671in}}%
\pgfpathcurveto{\pgfqpoint{3.297951in}{2.034908in}}{\pgfqpoint{3.294678in}{2.042808in}}{\pgfqpoint{3.288855in}{2.048632in}}%
\pgfpathcurveto{\pgfqpoint{3.283031in}{2.054456in}}{\pgfqpoint{3.275131in}{2.057728in}}{\pgfqpoint{3.266894in}{2.057728in}}%
\pgfpathcurveto{\pgfqpoint{3.258658in}{2.057728in}}{\pgfqpoint{3.250758in}{2.054456in}}{\pgfqpoint{3.244934in}{2.048632in}}%
\pgfpathcurveto{\pgfqpoint{3.239110in}{2.042808in}}{\pgfqpoint{3.235838in}{2.034908in}}{\pgfqpoint{3.235838in}{2.026671in}}%
\pgfpathcurveto{\pgfqpoint{3.235838in}{2.018435in}}{\pgfqpoint{3.239110in}{2.010535in}}{\pgfqpoint{3.244934in}{2.004711in}}%
\pgfpathcurveto{\pgfqpoint{3.250758in}{1.998887in}}{\pgfqpoint{3.258658in}{1.995615in}}{\pgfqpoint{3.266894in}{1.995615in}}%
\pgfpathclose%
\pgfusepath{stroke,fill}%
\end{pgfscope}%
\begin{pgfscope}%
\pgfpathrectangle{\pgfqpoint{0.100000in}{0.212622in}}{\pgfqpoint{3.696000in}{3.696000in}}%
\pgfusepath{clip}%
\pgfsetbuttcap%
\pgfsetroundjoin%
\definecolor{currentfill}{rgb}{0.121569,0.466667,0.705882}%
\pgfsetfillcolor{currentfill}%
\pgfsetfillopacity{0.564914}%
\pgfsetlinewidth{1.003750pt}%
\definecolor{currentstroke}{rgb}{0.121569,0.466667,0.705882}%
\pgfsetstrokecolor{currentstroke}%
\pgfsetstrokeopacity{0.564914}%
\pgfsetdash{}{0pt}%
\pgfpathmoveto{\pgfqpoint{3.274334in}{1.996606in}}%
\pgfpathcurveto{\pgfqpoint{3.282571in}{1.996606in}}{\pgfqpoint{3.290471in}{1.999878in}}{\pgfqpoint{3.296295in}{2.005702in}}%
\pgfpathcurveto{\pgfqpoint{3.302119in}{2.011526in}}{\pgfqpoint{3.305391in}{2.019426in}}{\pgfqpoint{3.305391in}{2.027662in}}%
\pgfpathcurveto{\pgfqpoint{3.305391in}{2.035899in}}{\pgfqpoint{3.302119in}{2.043799in}}{\pgfqpoint{3.296295in}{2.049623in}}%
\pgfpathcurveto{\pgfqpoint{3.290471in}{2.055446in}}{\pgfqpoint{3.282571in}{2.058719in}}{\pgfqpoint{3.274334in}{2.058719in}}%
\pgfpathcurveto{\pgfqpoint{3.266098in}{2.058719in}}{\pgfqpoint{3.258198in}{2.055446in}}{\pgfqpoint{3.252374in}{2.049623in}}%
\pgfpathcurveto{\pgfqpoint{3.246550in}{2.043799in}}{\pgfqpoint{3.243278in}{2.035899in}}{\pgfqpoint{3.243278in}{2.027662in}}%
\pgfpathcurveto{\pgfqpoint{3.243278in}{2.019426in}}{\pgfqpoint{3.246550in}{2.011526in}}{\pgfqpoint{3.252374in}{2.005702in}}%
\pgfpathcurveto{\pgfqpoint{3.258198in}{1.999878in}}{\pgfqpoint{3.266098in}{1.996606in}}{\pgfqpoint{3.274334in}{1.996606in}}%
\pgfpathclose%
\pgfusepath{stroke,fill}%
\end{pgfscope}%
\begin{pgfscope}%
\pgfpathrectangle{\pgfqpoint{0.100000in}{0.212622in}}{\pgfqpoint{3.696000in}{3.696000in}}%
\pgfusepath{clip}%
\pgfsetbuttcap%
\pgfsetroundjoin%
\definecolor{currentfill}{rgb}{0.121569,0.466667,0.705882}%
\pgfsetfillcolor{currentfill}%
\pgfsetfillopacity{0.565649}%
\pgfsetlinewidth{1.003750pt}%
\definecolor{currentstroke}{rgb}{0.121569,0.466667,0.705882}%
\pgfsetstrokecolor{currentstroke}%
\pgfsetstrokeopacity{0.565649}%
\pgfsetdash{}{0pt}%
\pgfpathmoveto{\pgfqpoint{3.278352in}{1.996539in}}%
\pgfpathcurveto{\pgfqpoint{3.286589in}{1.996539in}}{\pgfqpoint{3.294489in}{1.999812in}}{\pgfqpoint{3.300313in}{2.005636in}}%
\pgfpathcurveto{\pgfqpoint{3.306137in}{2.011460in}}{\pgfqpoint{3.309409in}{2.019360in}}{\pgfqpoint{3.309409in}{2.027596in}}%
\pgfpathcurveto{\pgfqpoint{3.309409in}{2.035832in}}{\pgfqpoint{3.306137in}{2.043732in}}{\pgfqpoint{3.300313in}{2.049556in}}%
\pgfpathcurveto{\pgfqpoint{3.294489in}{2.055380in}}{\pgfqpoint{3.286589in}{2.058652in}}{\pgfqpoint{3.278352in}{2.058652in}}%
\pgfpathcurveto{\pgfqpoint{3.270116in}{2.058652in}}{\pgfqpoint{3.262216in}{2.055380in}}{\pgfqpoint{3.256392in}{2.049556in}}%
\pgfpathcurveto{\pgfqpoint{3.250568in}{2.043732in}}{\pgfqpoint{3.247296in}{2.035832in}}{\pgfqpoint{3.247296in}{2.027596in}}%
\pgfpathcurveto{\pgfqpoint{3.247296in}{2.019360in}}{\pgfqpoint{3.250568in}{2.011460in}}{\pgfqpoint{3.256392in}{2.005636in}}%
\pgfpathcurveto{\pgfqpoint{3.262216in}{1.999812in}}{\pgfqpoint{3.270116in}{1.996539in}}{\pgfqpoint{3.278352in}{1.996539in}}%
\pgfpathclose%
\pgfusepath{stroke,fill}%
\end{pgfscope}%
\begin{pgfscope}%
\pgfpathrectangle{\pgfqpoint{0.100000in}{0.212622in}}{\pgfqpoint{3.696000in}{3.696000in}}%
\pgfusepath{clip}%
\pgfsetbuttcap%
\pgfsetroundjoin%
\definecolor{currentfill}{rgb}{0.121569,0.466667,0.705882}%
\pgfsetfillcolor{currentfill}%
\pgfsetfillopacity{0.566432}%
\pgfsetlinewidth{1.003750pt}%
\definecolor{currentstroke}{rgb}{0.121569,0.466667,0.705882}%
\pgfsetstrokecolor{currentstroke}%
\pgfsetstrokeopacity{0.566432}%
\pgfsetdash{}{0pt}%
\pgfpathmoveto{\pgfqpoint{3.283018in}{1.997081in}}%
\pgfpathcurveto{\pgfqpoint{3.291254in}{1.997081in}}{\pgfqpoint{3.299155in}{2.000354in}}{\pgfqpoint{3.304978in}{2.006178in}}%
\pgfpathcurveto{\pgfqpoint{3.310802in}{2.012002in}}{\pgfqpoint{3.314075in}{2.019902in}}{\pgfqpoint{3.314075in}{2.028138in}}%
\pgfpathcurveto{\pgfqpoint{3.314075in}{2.036374in}}{\pgfqpoint{3.310802in}{2.044274in}}{\pgfqpoint{3.304978in}{2.050098in}}%
\pgfpathcurveto{\pgfqpoint{3.299155in}{2.055922in}}{\pgfqpoint{3.291254in}{2.059194in}}{\pgfqpoint{3.283018in}{2.059194in}}%
\pgfpathcurveto{\pgfqpoint{3.274782in}{2.059194in}}{\pgfqpoint{3.266882in}{2.055922in}}{\pgfqpoint{3.261058in}{2.050098in}}%
\pgfpathcurveto{\pgfqpoint{3.255234in}{2.044274in}}{\pgfqpoint{3.251962in}{2.036374in}}{\pgfqpoint{3.251962in}{2.028138in}}%
\pgfpathcurveto{\pgfqpoint{3.251962in}{2.019902in}}{\pgfqpoint{3.255234in}{2.012002in}}{\pgfqpoint{3.261058in}{2.006178in}}%
\pgfpathcurveto{\pgfqpoint{3.266882in}{2.000354in}}{\pgfqpoint{3.274782in}{1.997081in}}{\pgfqpoint{3.283018in}{1.997081in}}%
\pgfpathclose%
\pgfusepath{stroke,fill}%
\end{pgfscope}%
\begin{pgfscope}%
\pgfpathrectangle{\pgfqpoint{0.100000in}{0.212622in}}{\pgfqpoint{3.696000in}{3.696000in}}%
\pgfusepath{clip}%
\pgfsetbuttcap%
\pgfsetroundjoin%
\definecolor{currentfill}{rgb}{0.121569,0.466667,0.705882}%
\pgfsetfillcolor{currentfill}%
\pgfsetfillopacity{0.566758}%
\pgfsetlinewidth{1.003750pt}%
\definecolor{currentstroke}{rgb}{0.121569,0.466667,0.705882}%
\pgfsetstrokecolor{currentstroke}%
\pgfsetstrokeopacity{0.566758}%
\pgfsetdash{}{0pt}%
\pgfpathmoveto{\pgfqpoint{3.285453in}{1.996293in}}%
\pgfpathcurveto{\pgfqpoint{3.293690in}{1.996293in}}{\pgfqpoint{3.301590in}{1.999565in}}{\pgfqpoint{3.307414in}{2.005389in}}%
\pgfpathcurveto{\pgfqpoint{3.313238in}{2.011213in}}{\pgfqpoint{3.316510in}{2.019113in}}{\pgfqpoint{3.316510in}{2.027349in}}%
\pgfpathcurveto{\pgfqpoint{3.316510in}{2.035585in}}{\pgfqpoint{3.313238in}{2.043485in}}{\pgfqpoint{3.307414in}{2.049309in}}%
\pgfpathcurveto{\pgfqpoint{3.301590in}{2.055133in}}{\pgfqpoint{3.293690in}{2.058406in}}{\pgfqpoint{3.285453in}{2.058406in}}%
\pgfpathcurveto{\pgfqpoint{3.277217in}{2.058406in}}{\pgfqpoint{3.269317in}{2.055133in}}{\pgfqpoint{3.263493in}{2.049309in}}%
\pgfpathcurveto{\pgfqpoint{3.257669in}{2.043485in}}{\pgfqpoint{3.254397in}{2.035585in}}{\pgfqpoint{3.254397in}{2.027349in}}%
\pgfpathcurveto{\pgfqpoint{3.254397in}{2.019113in}}{\pgfqpoint{3.257669in}{2.011213in}}{\pgfqpoint{3.263493in}{2.005389in}}%
\pgfpathcurveto{\pgfqpoint{3.269317in}{1.999565in}}{\pgfqpoint{3.277217in}{1.996293in}}{\pgfqpoint{3.285453in}{1.996293in}}%
\pgfpathclose%
\pgfusepath{stroke,fill}%
\end{pgfscope}%
\begin{pgfscope}%
\pgfpathrectangle{\pgfqpoint{0.100000in}{0.212622in}}{\pgfqpoint{3.696000in}{3.696000in}}%
\pgfusepath{clip}%
\pgfsetbuttcap%
\pgfsetroundjoin%
\definecolor{currentfill}{rgb}{0.121569,0.466667,0.705882}%
\pgfsetfillcolor{currentfill}%
\pgfsetfillopacity{0.567014}%
\pgfsetlinewidth{1.003750pt}%
\definecolor{currentstroke}{rgb}{0.121569,0.466667,0.705882}%
\pgfsetstrokecolor{currentstroke}%
\pgfsetstrokeopacity{0.567014}%
\pgfsetdash{}{0pt}%
\pgfpathmoveto{\pgfqpoint{3.286842in}{1.996500in}}%
\pgfpathcurveto{\pgfqpoint{3.295079in}{1.996500in}}{\pgfqpoint{3.302979in}{1.999772in}}{\pgfqpoint{3.308803in}{2.005596in}}%
\pgfpathcurveto{\pgfqpoint{3.314626in}{2.011420in}}{\pgfqpoint{3.317899in}{2.019320in}}{\pgfqpoint{3.317899in}{2.027557in}}%
\pgfpathcurveto{\pgfqpoint{3.317899in}{2.035793in}}{\pgfqpoint{3.314626in}{2.043693in}}{\pgfqpoint{3.308803in}{2.049517in}}%
\pgfpathcurveto{\pgfqpoint{3.302979in}{2.055341in}}{\pgfqpoint{3.295079in}{2.058613in}}{\pgfqpoint{3.286842in}{2.058613in}}%
\pgfpathcurveto{\pgfqpoint{3.278606in}{2.058613in}}{\pgfqpoint{3.270706in}{2.055341in}}{\pgfqpoint{3.264882in}{2.049517in}}%
\pgfpathcurveto{\pgfqpoint{3.259058in}{2.043693in}}{\pgfqpoint{3.255786in}{2.035793in}}{\pgfqpoint{3.255786in}{2.027557in}}%
\pgfpathcurveto{\pgfqpoint{3.255786in}{2.019320in}}{\pgfqpoint{3.259058in}{2.011420in}}{\pgfqpoint{3.264882in}{2.005596in}}%
\pgfpathcurveto{\pgfqpoint{3.270706in}{1.999772in}}{\pgfqpoint{3.278606in}{1.996500in}}{\pgfqpoint{3.286842in}{1.996500in}}%
\pgfpathclose%
\pgfusepath{stroke,fill}%
\end{pgfscope}%
\begin{pgfscope}%
\pgfpathrectangle{\pgfqpoint{0.100000in}{0.212622in}}{\pgfqpoint{3.696000in}{3.696000in}}%
\pgfusepath{clip}%
\pgfsetbuttcap%
\pgfsetroundjoin%
\definecolor{currentfill}{rgb}{0.121569,0.466667,0.705882}%
\pgfsetfillcolor{currentfill}%
\pgfsetfillopacity{0.567119}%
\pgfsetlinewidth{1.003750pt}%
\definecolor{currentstroke}{rgb}{0.121569,0.466667,0.705882}%
\pgfsetstrokecolor{currentstroke}%
\pgfsetstrokeopacity{0.567119}%
\pgfsetdash{}{0pt}%
\pgfpathmoveto{\pgfqpoint{1.014372in}{1.698548in}}%
\pgfpathcurveto{\pgfqpoint{1.022609in}{1.698548in}}{\pgfqpoint{1.030509in}{1.701820in}}{\pgfqpoint{1.036333in}{1.707644in}}%
\pgfpathcurveto{\pgfqpoint{1.042156in}{1.713468in}}{\pgfqpoint{1.045429in}{1.721368in}}{\pgfqpoint{1.045429in}{1.729605in}}%
\pgfpathcurveto{\pgfqpoint{1.045429in}{1.737841in}}{\pgfqpoint{1.042156in}{1.745741in}}{\pgfqpoint{1.036333in}{1.751565in}}%
\pgfpathcurveto{\pgfqpoint{1.030509in}{1.757389in}}{\pgfqpoint{1.022609in}{1.760661in}}{\pgfqpoint{1.014372in}{1.760661in}}%
\pgfpathcurveto{\pgfqpoint{1.006136in}{1.760661in}}{\pgfqpoint{0.998236in}{1.757389in}}{\pgfqpoint{0.992412in}{1.751565in}}%
\pgfpathcurveto{\pgfqpoint{0.986588in}{1.745741in}}{\pgfqpoint{0.983316in}{1.737841in}}{\pgfqpoint{0.983316in}{1.729605in}}%
\pgfpathcurveto{\pgfqpoint{0.983316in}{1.721368in}}{\pgfqpoint{0.986588in}{1.713468in}}{\pgfqpoint{0.992412in}{1.707644in}}%
\pgfpathcurveto{\pgfqpoint{0.998236in}{1.701820in}}{\pgfqpoint{1.006136in}{1.698548in}}{\pgfqpoint{1.014372in}{1.698548in}}%
\pgfpathclose%
\pgfusepath{stroke,fill}%
\end{pgfscope}%
\begin{pgfscope}%
\pgfpathrectangle{\pgfqpoint{0.100000in}{0.212622in}}{\pgfqpoint{3.696000in}{3.696000in}}%
\pgfusepath{clip}%
\pgfsetbuttcap%
\pgfsetroundjoin%
\definecolor{currentfill}{rgb}{0.121569,0.466667,0.705882}%
\pgfsetfillcolor{currentfill}%
\pgfsetfillopacity{0.567133}%
\pgfsetlinewidth{1.003750pt}%
\definecolor{currentstroke}{rgb}{0.121569,0.466667,0.705882}%
\pgfsetstrokecolor{currentstroke}%
\pgfsetstrokeopacity{0.567133}%
\pgfsetdash{}{0pt}%
\pgfpathmoveto{\pgfqpoint{3.287576in}{1.996379in}}%
\pgfpathcurveto{\pgfqpoint{3.295812in}{1.996379in}}{\pgfqpoint{3.303712in}{1.999652in}}{\pgfqpoint{3.309536in}{2.005476in}}%
\pgfpathcurveto{\pgfqpoint{3.315360in}{2.011299in}}{\pgfqpoint{3.318633in}{2.019200in}}{\pgfqpoint{3.318633in}{2.027436in}}%
\pgfpathcurveto{\pgfqpoint{3.318633in}{2.035672in}}{\pgfqpoint{3.315360in}{2.043572in}}{\pgfqpoint{3.309536in}{2.049396in}}%
\pgfpathcurveto{\pgfqpoint{3.303712in}{2.055220in}}{\pgfqpoint{3.295812in}{2.058492in}}{\pgfqpoint{3.287576in}{2.058492in}}%
\pgfpathcurveto{\pgfqpoint{3.279340in}{2.058492in}}{\pgfqpoint{3.271440in}{2.055220in}}{\pgfqpoint{3.265616in}{2.049396in}}%
\pgfpathcurveto{\pgfqpoint{3.259792in}{2.043572in}}{\pgfqpoint{3.256520in}{2.035672in}}{\pgfqpoint{3.256520in}{2.027436in}}%
\pgfpathcurveto{\pgfqpoint{3.256520in}{2.019200in}}{\pgfqpoint{3.259792in}{2.011299in}}{\pgfqpoint{3.265616in}{2.005476in}}%
\pgfpathcurveto{\pgfqpoint{3.271440in}{1.999652in}}{\pgfqpoint{3.279340in}{1.996379in}}{\pgfqpoint{3.287576in}{1.996379in}}%
\pgfpathclose%
\pgfusepath{stroke,fill}%
\end{pgfscope}%
\begin{pgfscope}%
\pgfpathrectangle{\pgfqpoint{0.100000in}{0.212622in}}{\pgfqpoint{3.696000in}{3.696000in}}%
\pgfusepath{clip}%
\pgfsetbuttcap%
\pgfsetroundjoin%
\definecolor{currentfill}{rgb}{0.121569,0.466667,0.705882}%
\pgfsetfillcolor{currentfill}%
\pgfsetfillopacity{0.567214}%
\pgfsetlinewidth{1.003750pt}%
\definecolor{currentstroke}{rgb}{0.121569,0.466667,0.705882}%
\pgfsetstrokecolor{currentstroke}%
\pgfsetstrokeopacity{0.567214}%
\pgfsetdash{}{0pt}%
\pgfpathmoveto{\pgfqpoint{3.287992in}{1.996455in}}%
\pgfpathcurveto{\pgfqpoint{3.296229in}{1.996455in}}{\pgfqpoint{3.304129in}{1.999727in}}{\pgfqpoint{3.309953in}{2.005551in}}%
\pgfpathcurveto{\pgfqpoint{3.315776in}{2.011375in}}{\pgfqpoint{3.319049in}{2.019275in}}{\pgfqpoint{3.319049in}{2.027511in}}%
\pgfpathcurveto{\pgfqpoint{3.319049in}{2.035748in}}{\pgfqpoint{3.315776in}{2.043648in}}{\pgfqpoint{3.309953in}{2.049472in}}%
\pgfpathcurveto{\pgfqpoint{3.304129in}{2.055295in}}{\pgfqpoint{3.296229in}{2.058568in}}{\pgfqpoint{3.287992in}{2.058568in}}%
\pgfpathcurveto{\pgfqpoint{3.279756in}{2.058568in}}{\pgfqpoint{3.271856in}{2.055295in}}{\pgfqpoint{3.266032in}{2.049472in}}%
\pgfpathcurveto{\pgfqpoint{3.260208in}{2.043648in}}{\pgfqpoint{3.256936in}{2.035748in}}{\pgfqpoint{3.256936in}{2.027511in}}%
\pgfpathcurveto{\pgfqpoint{3.256936in}{2.019275in}}{\pgfqpoint{3.260208in}{2.011375in}}{\pgfqpoint{3.266032in}{2.005551in}}%
\pgfpathcurveto{\pgfqpoint{3.271856in}{1.999727in}}{\pgfqpoint{3.279756in}{1.996455in}}{\pgfqpoint{3.287992in}{1.996455in}}%
\pgfpathclose%
\pgfusepath{stroke,fill}%
\end{pgfscope}%
\begin{pgfscope}%
\pgfpathrectangle{\pgfqpoint{0.100000in}{0.212622in}}{\pgfqpoint{3.696000in}{3.696000in}}%
\pgfusepath{clip}%
\pgfsetbuttcap%
\pgfsetroundjoin%
\definecolor{currentfill}{rgb}{0.121569,0.466667,0.705882}%
\pgfsetfillcolor{currentfill}%
\pgfsetfillopacity{0.567254}%
\pgfsetlinewidth{1.003750pt}%
\definecolor{currentstroke}{rgb}{0.121569,0.466667,0.705882}%
\pgfsetstrokecolor{currentstroke}%
\pgfsetstrokeopacity{0.567254}%
\pgfsetdash{}{0pt}%
\pgfpathmoveto{\pgfqpoint{3.288215in}{1.996443in}}%
\pgfpathcurveto{\pgfqpoint{3.296451in}{1.996443in}}{\pgfqpoint{3.304351in}{1.999715in}}{\pgfqpoint{3.310175in}{2.005539in}}%
\pgfpathcurveto{\pgfqpoint{3.315999in}{2.011363in}}{\pgfqpoint{3.319271in}{2.019263in}}{\pgfqpoint{3.319271in}{2.027499in}}%
\pgfpathcurveto{\pgfqpoint{3.319271in}{2.035736in}}{\pgfqpoint{3.315999in}{2.043636in}}{\pgfqpoint{3.310175in}{2.049460in}}%
\pgfpathcurveto{\pgfqpoint{3.304351in}{2.055284in}}{\pgfqpoint{3.296451in}{2.058556in}}{\pgfqpoint{3.288215in}{2.058556in}}%
\pgfpathcurveto{\pgfqpoint{3.279978in}{2.058556in}}{\pgfqpoint{3.272078in}{2.055284in}}{\pgfqpoint{3.266255in}{2.049460in}}%
\pgfpathcurveto{\pgfqpoint{3.260431in}{2.043636in}}{\pgfqpoint{3.257158in}{2.035736in}}{\pgfqpoint{3.257158in}{2.027499in}}%
\pgfpathcurveto{\pgfqpoint{3.257158in}{2.019263in}}{\pgfqpoint{3.260431in}{2.011363in}}{\pgfqpoint{3.266255in}{2.005539in}}%
\pgfpathcurveto{\pgfqpoint{3.272078in}{1.999715in}}{\pgfqpoint{3.279978in}{1.996443in}}{\pgfqpoint{3.288215in}{1.996443in}}%
\pgfpathclose%
\pgfusepath{stroke,fill}%
\end{pgfscope}%
\begin{pgfscope}%
\pgfpathrectangle{\pgfqpoint{0.100000in}{0.212622in}}{\pgfqpoint{3.696000in}{3.696000in}}%
\pgfusepath{clip}%
\pgfsetbuttcap%
\pgfsetroundjoin%
\definecolor{currentfill}{rgb}{0.121569,0.466667,0.705882}%
\pgfsetfillcolor{currentfill}%
\pgfsetfillopacity{0.567277}%
\pgfsetlinewidth{1.003750pt}%
\definecolor{currentstroke}{rgb}{0.121569,0.466667,0.705882}%
\pgfsetstrokecolor{currentstroke}%
\pgfsetstrokeopacity{0.567277}%
\pgfsetdash{}{0pt}%
\pgfpathmoveto{\pgfqpoint{3.288343in}{1.996464in}}%
\pgfpathcurveto{\pgfqpoint{3.296580in}{1.996464in}}{\pgfqpoint{3.304480in}{1.999737in}}{\pgfqpoint{3.310304in}{2.005561in}}%
\pgfpathcurveto{\pgfqpoint{3.316127in}{2.011384in}}{\pgfqpoint{3.319400in}{2.019285in}}{\pgfqpoint{3.319400in}{2.027521in}}%
\pgfpathcurveto{\pgfqpoint{3.319400in}{2.035757in}}{\pgfqpoint{3.316127in}{2.043657in}}{\pgfqpoint{3.310304in}{2.049481in}}%
\pgfpathcurveto{\pgfqpoint{3.304480in}{2.055305in}}{\pgfqpoint{3.296580in}{2.058577in}}{\pgfqpoint{3.288343in}{2.058577in}}%
\pgfpathcurveto{\pgfqpoint{3.280107in}{2.058577in}}{\pgfqpoint{3.272207in}{2.055305in}}{\pgfqpoint{3.266383in}{2.049481in}}%
\pgfpathcurveto{\pgfqpoint{3.260559in}{2.043657in}}{\pgfqpoint{3.257287in}{2.035757in}}{\pgfqpoint{3.257287in}{2.027521in}}%
\pgfpathcurveto{\pgfqpoint{3.257287in}{2.019285in}}{\pgfqpoint{3.260559in}{2.011384in}}{\pgfqpoint{3.266383in}{2.005561in}}%
\pgfpathcurveto{\pgfqpoint{3.272207in}{1.999737in}}{\pgfqpoint{3.280107in}{1.996464in}}{\pgfqpoint{3.288343in}{1.996464in}}%
\pgfpathclose%
\pgfusepath{stroke,fill}%
\end{pgfscope}%
\begin{pgfscope}%
\pgfpathrectangle{\pgfqpoint{0.100000in}{0.212622in}}{\pgfqpoint{3.696000in}{3.696000in}}%
\pgfusepath{clip}%
\pgfsetbuttcap%
\pgfsetroundjoin%
\definecolor{currentfill}{rgb}{0.121569,0.466667,0.705882}%
\pgfsetfillcolor{currentfill}%
\pgfsetfillopacity{0.567287}%
\pgfsetlinewidth{1.003750pt}%
\definecolor{currentstroke}{rgb}{0.121569,0.466667,0.705882}%
\pgfsetstrokecolor{currentstroke}%
\pgfsetstrokeopacity{0.567287}%
\pgfsetdash{}{0pt}%
\pgfpathmoveto{\pgfqpoint{3.288412in}{1.996456in}}%
\pgfpathcurveto{\pgfqpoint{3.296648in}{1.996456in}}{\pgfqpoint{3.304548in}{1.999728in}}{\pgfqpoint{3.310372in}{2.005552in}}%
\pgfpathcurveto{\pgfqpoint{3.316196in}{2.011376in}}{\pgfqpoint{3.319468in}{2.019276in}}{\pgfqpoint{3.319468in}{2.027512in}}%
\pgfpathcurveto{\pgfqpoint{3.319468in}{2.035748in}}{\pgfqpoint{3.316196in}{2.043648in}}{\pgfqpoint{3.310372in}{2.049472in}}%
\pgfpathcurveto{\pgfqpoint{3.304548in}{2.055296in}}{\pgfqpoint{3.296648in}{2.058569in}}{\pgfqpoint{3.288412in}{2.058569in}}%
\pgfpathcurveto{\pgfqpoint{3.280175in}{2.058569in}}{\pgfqpoint{3.272275in}{2.055296in}}{\pgfqpoint{3.266452in}{2.049472in}}%
\pgfpathcurveto{\pgfqpoint{3.260628in}{2.043648in}}{\pgfqpoint{3.257355in}{2.035748in}}{\pgfqpoint{3.257355in}{2.027512in}}%
\pgfpathcurveto{\pgfqpoint{3.257355in}{2.019276in}}{\pgfqpoint{3.260628in}{2.011376in}}{\pgfqpoint{3.266452in}{2.005552in}}%
\pgfpathcurveto{\pgfqpoint{3.272275in}{1.999728in}}{\pgfqpoint{3.280175in}{1.996456in}}{\pgfqpoint{3.288412in}{1.996456in}}%
\pgfpathclose%
\pgfusepath{stroke,fill}%
\end{pgfscope}%
\begin{pgfscope}%
\pgfpathrectangle{\pgfqpoint{0.100000in}{0.212622in}}{\pgfqpoint{3.696000in}{3.696000in}}%
\pgfusepath{clip}%
\pgfsetbuttcap%
\pgfsetroundjoin%
\definecolor{currentfill}{rgb}{0.121569,0.466667,0.705882}%
\pgfsetfillcolor{currentfill}%
\pgfsetfillopacity{0.567294}%
\pgfsetlinewidth{1.003750pt}%
\definecolor{currentstroke}{rgb}{0.121569,0.466667,0.705882}%
\pgfsetstrokecolor{currentstroke}%
\pgfsetstrokeopacity{0.567294}%
\pgfsetdash{}{0pt}%
\pgfpathmoveto{\pgfqpoint{3.288450in}{1.996460in}}%
\pgfpathcurveto{\pgfqpoint{3.296686in}{1.996460in}}{\pgfqpoint{3.304586in}{1.999732in}}{\pgfqpoint{3.310410in}{2.005556in}}%
\pgfpathcurveto{\pgfqpoint{3.316234in}{2.011380in}}{\pgfqpoint{3.319506in}{2.019280in}}{\pgfqpoint{3.319506in}{2.027517in}}%
\pgfpathcurveto{\pgfqpoint{3.319506in}{2.035753in}}{\pgfqpoint{3.316234in}{2.043653in}}{\pgfqpoint{3.310410in}{2.049477in}}%
\pgfpathcurveto{\pgfqpoint{3.304586in}{2.055301in}}{\pgfqpoint{3.296686in}{2.058573in}}{\pgfqpoint{3.288450in}{2.058573in}}%
\pgfpathcurveto{\pgfqpoint{3.280213in}{2.058573in}}{\pgfqpoint{3.272313in}{2.055301in}}{\pgfqpoint{3.266489in}{2.049477in}}%
\pgfpathcurveto{\pgfqpoint{3.260666in}{2.043653in}}{\pgfqpoint{3.257393in}{2.035753in}}{\pgfqpoint{3.257393in}{2.027517in}}%
\pgfpathcurveto{\pgfqpoint{3.257393in}{2.019280in}}{\pgfqpoint{3.260666in}{2.011380in}}{\pgfqpoint{3.266489in}{2.005556in}}%
\pgfpathcurveto{\pgfqpoint{3.272313in}{1.999732in}}{\pgfqpoint{3.280213in}{1.996460in}}{\pgfqpoint{3.288450in}{1.996460in}}%
\pgfpathclose%
\pgfusepath{stroke,fill}%
\end{pgfscope}%
\begin{pgfscope}%
\pgfpathrectangle{\pgfqpoint{0.100000in}{0.212622in}}{\pgfqpoint{3.696000in}{3.696000in}}%
\pgfusepath{clip}%
\pgfsetbuttcap%
\pgfsetroundjoin%
\definecolor{currentfill}{rgb}{0.121569,0.466667,0.705882}%
\pgfsetfillcolor{currentfill}%
\pgfsetfillopacity{0.567297}%
\pgfsetlinewidth{1.003750pt}%
\definecolor{currentstroke}{rgb}{0.121569,0.466667,0.705882}%
\pgfsetstrokecolor{currentstroke}%
\pgfsetstrokeopacity{0.567297}%
\pgfsetdash{}{0pt}%
\pgfpathmoveto{\pgfqpoint{3.288468in}{1.996448in}}%
\pgfpathcurveto{\pgfqpoint{3.296704in}{1.996448in}}{\pgfqpoint{3.304604in}{1.999721in}}{\pgfqpoint{3.310428in}{2.005544in}}%
\pgfpathcurveto{\pgfqpoint{3.316252in}{2.011368in}}{\pgfqpoint{3.319524in}{2.019268in}}{\pgfqpoint{3.319524in}{2.027505in}}%
\pgfpathcurveto{\pgfqpoint{3.319524in}{2.035741in}}{\pgfqpoint{3.316252in}{2.043641in}}{\pgfqpoint{3.310428in}{2.049465in}}%
\pgfpathcurveto{\pgfqpoint{3.304604in}{2.055289in}}{\pgfqpoint{3.296704in}{2.058561in}}{\pgfqpoint{3.288468in}{2.058561in}}%
\pgfpathcurveto{\pgfqpoint{3.280232in}{2.058561in}}{\pgfqpoint{3.272331in}{2.055289in}}{\pgfqpoint{3.266508in}{2.049465in}}%
\pgfpathcurveto{\pgfqpoint{3.260684in}{2.043641in}}{\pgfqpoint{3.257411in}{2.035741in}}{\pgfqpoint{3.257411in}{2.027505in}}%
\pgfpathcurveto{\pgfqpoint{3.257411in}{2.019268in}}{\pgfqpoint{3.260684in}{2.011368in}}{\pgfqpoint{3.266508in}{2.005544in}}%
\pgfpathcurveto{\pgfqpoint{3.272331in}{1.999721in}}{\pgfqpoint{3.280232in}{1.996448in}}{\pgfqpoint{3.288468in}{1.996448in}}%
\pgfpathclose%
\pgfusepath{stroke,fill}%
\end{pgfscope}%
\begin{pgfscope}%
\pgfpathrectangle{\pgfqpoint{0.100000in}{0.212622in}}{\pgfqpoint{3.696000in}{3.696000in}}%
\pgfusepath{clip}%
\pgfsetbuttcap%
\pgfsetroundjoin%
\definecolor{currentfill}{rgb}{0.121569,0.466667,0.705882}%
\pgfsetfillcolor{currentfill}%
\pgfsetfillopacity{0.567300}%
\pgfsetlinewidth{1.003750pt}%
\definecolor{currentstroke}{rgb}{0.121569,0.466667,0.705882}%
\pgfsetstrokecolor{currentstroke}%
\pgfsetstrokeopacity{0.567300}%
\pgfsetdash{}{0pt}%
\pgfpathmoveto{\pgfqpoint{3.288478in}{1.996454in}}%
\pgfpathcurveto{\pgfqpoint{3.296715in}{1.996454in}}{\pgfqpoint{3.304615in}{1.999726in}}{\pgfqpoint{3.310439in}{2.005550in}}%
\pgfpathcurveto{\pgfqpoint{3.316263in}{2.011374in}}{\pgfqpoint{3.319535in}{2.019274in}}{\pgfqpoint{3.319535in}{2.027510in}}%
\pgfpathcurveto{\pgfqpoint{3.319535in}{2.035746in}}{\pgfqpoint{3.316263in}{2.043646in}}{\pgfqpoint{3.310439in}{2.049470in}}%
\pgfpathcurveto{\pgfqpoint{3.304615in}{2.055294in}}{\pgfqpoint{3.296715in}{2.058567in}}{\pgfqpoint{3.288478in}{2.058567in}}%
\pgfpathcurveto{\pgfqpoint{3.280242in}{2.058567in}}{\pgfqpoint{3.272342in}{2.055294in}}{\pgfqpoint{3.266518in}{2.049470in}}%
\pgfpathcurveto{\pgfqpoint{3.260694in}{2.043646in}}{\pgfqpoint{3.257422in}{2.035746in}}{\pgfqpoint{3.257422in}{2.027510in}}%
\pgfpathcurveto{\pgfqpoint{3.257422in}{2.019274in}}{\pgfqpoint{3.260694in}{2.011374in}}{\pgfqpoint{3.266518in}{2.005550in}}%
\pgfpathcurveto{\pgfqpoint{3.272342in}{1.999726in}}{\pgfqpoint{3.280242in}{1.996454in}}{\pgfqpoint{3.288478in}{1.996454in}}%
\pgfpathclose%
\pgfusepath{stroke,fill}%
\end{pgfscope}%
\begin{pgfscope}%
\pgfpathrectangle{\pgfqpoint{0.100000in}{0.212622in}}{\pgfqpoint{3.696000in}{3.696000in}}%
\pgfusepath{clip}%
\pgfsetbuttcap%
\pgfsetroundjoin%
\definecolor{currentfill}{rgb}{0.121569,0.466667,0.705882}%
\pgfsetfillcolor{currentfill}%
\pgfsetfillopacity{0.567302}%
\pgfsetlinewidth{1.003750pt}%
\definecolor{currentstroke}{rgb}{0.121569,0.466667,0.705882}%
\pgfsetstrokecolor{currentstroke}%
\pgfsetstrokeopacity{0.567302}%
\pgfsetdash{}{0pt}%
\pgfpathmoveto{\pgfqpoint{3.288482in}{1.996448in}}%
\pgfpathcurveto{\pgfqpoint{3.296718in}{1.996448in}}{\pgfqpoint{3.304618in}{1.999721in}}{\pgfqpoint{3.310442in}{2.005545in}}%
\pgfpathcurveto{\pgfqpoint{3.316266in}{2.011369in}}{\pgfqpoint{3.319538in}{2.019269in}}{\pgfqpoint{3.319538in}{2.027505in}}%
\pgfpathcurveto{\pgfqpoint{3.319538in}{2.035741in}}{\pgfqpoint{3.316266in}{2.043641in}}{\pgfqpoint{3.310442in}{2.049465in}}%
\pgfpathcurveto{\pgfqpoint{3.304618in}{2.055289in}}{\pgfqpoint{3.296718in}{2.058561in}}{\pgfqpoint{3.288482in}{2.058561in}}%
\pgfpathcurveto{\pgfqpoint{3.280245in}{2.058561in}}{\pgfqpoint{3.272345in}{2.055289in}}{\pgfqpoint{3.266521in}{2.049465in}}%
\pgfpathcurveto{\pgfqpoint{3.260698in}{2.043641in}}{\pgfqpoint{3.257425in}{2.035741in}}{\pgfqpoint{3.257425in}{2.027505in}}%
\pgfpathcurveto{\pgfqpoint{3.257425in}{2.019269in}}{\pgfqpoint{3.260698in}{2.011369in}}{\pgfqpoint{3.266521in}{2.005545in}}%
\pgfpathcurveto{\pgfqpoint{3.272345in}{1.999721in}}{\pgfqpoint{3.280245in}{1.996448in}}{\pgfqpoint{3.288482in}{1.996448in}}%
\pgfpathclose%
\pgfusepath{stroke,fill}%
\end{pgfscope}%
\begin{pgfscope}%
\pgfpathrectangle{\pgfqpoint{0.100000in}{0.212622in}}{\pgfqpoint{3.696000in}{3.696000in}}%
\pgfusepath{clip}%
\pgfsetbuttcap%
\pgfsetroundjoin%
\definecolor{currentfill}{rgb}{0.121569,0.466667,0.705882}%
\pgfsetfillcolor{currentfill}%
\pgfsetfillopacity{0.567411}%
\pgfsetlinewidth{1.003750pt}%
\definecolor{currentstroke}{rgb}{0.121569,0.466667,0.705882}%
\pgfsetstrokecolor{currentstroke}%
\pgfsetstrokeopacity{0.567411}%
\pgfsetdash{}{0pt}%
\pgfpathmoveto{\pgfqpoint{3.288684in}{1.996556in}}%
\pgfpathcurveto{\pgfqpoint{3.296920in}{1.996556in}}{\pgfqpoint{3.304821in}{1.999828in}}{\pgfqpoint{3.310644in}{2.005652in}}%
\pgfpathcurveto{\pgfqpoint{3.316468in}{2.011476in}}{\pgfqpoint{3.319741in}{2.019376in}}{\pgfqpoint{3.319741in}{2.027612in}}%
\pgfpathcurveto{\pgfqpoint{3.319741in}{2.035849in}}{\pgfqpoint{3.316468in}{2.043749in}}{\pgfqpoint{3.310644in}{2.049573in}}%
\pgfpathcurveto{\pgfqpoint{3.304821in}{2.055396in}}{\pgfqpoint{3.296920in}{2.058669in}}{\pgfqpoint{3.288684in}{2.058669in}}%
\pgfpathcurveto{\pgfqpoint{3.280448in}{2.058669in}}{\pgfqpoint{3.272548in}{2.055396in}}{\pgfqpoint{3.266724in}{2.049573in}}%
\pgfpathcurveto{\pgfqpoint{3.260900in}{2.043749in}}{\pgfqpoint{3.257628in}{2.035849in}}{\pgfqpoint{3.257628in}{2.027612in}}%
\pgfpathcurveto{\pgfqpoint{3.257628in}{2.019376in}}{\pgfqpoint{3.260900in}{2.011476in}}{\pgfqpoint{3.266724in}{2.005652in}}%
\pgfpathcurveto{\pgfqpoint{3.272548in}{1.999828in}}{\pgfqpoint{3.280448in}{1.996556in}}{\pgfqpoint{3.288684in}{1.996556in}}%
\pgfpathclose%
\pgfusepath{stroke,fill}%
\end{pgfscope}%
\begin{pgfscope}%
\pgfpathrectangle{\pgfqpoint{0.100000in}{0.212622in}}{\pgfqpoint{3.696000in}{3.696000in}}%
\pgfusepath{clip}%
\pgfsetbuttcap%
\pgfsetroundjoin%
\definecolor{currentfill}{rgb}{0.121569,0.466667,0.705882}%
\pgfsetfillcolor{currentfill}%
\pgfsetfillopacity{0.567455}%
\pgfsetlinewidth{1.003750pt}%
\definecolor{currentstroke}{rgb}{0.121569,0.466667,0.705882}%
\pgfsetstrokecolor{currentstroke}%
\pgfsetstrokeopacity{0.567455}%
\pgfsetdash{}{0pt}%
\pgfpathmoveto{\pgfqpoint{3.288724in}{1.996453in}}%
\pgfpathcurveto{\pgfqpoint{3.296961in}{1.996453in}}{\pgfqpoint{3.304861in}{1.999725in}}{\pgfqpoint{3.310685in}{2.005549in}}%
\pgfpathcurveto{\pgfqpoint{3.316509in}{2.011373in}}{\pgfqpoint{3.319781in}{2.019273in}}{\pgfqpoint{3.319781in}{2.027510in}}%
\pgfpathcurveto{\pgfqpoint{3.319781in}{2.035746in}}{\pgfqpoint{3.316509in}{2.043646in}}{\pgfqpoint{3.310685in}{2.049470in}}%
\pgfpathcurveto{\pgfqpoint{3.304861in}{2.055294in}}{\pgfqpoint{3.296961in}{2.058566in}}{\pgfqpoint{3.288724in}{2.058566in}}%
\pgfpathcurveto{\pgfqpoint{3.280488in}{2.058566in}}{\pgfqpoint{3.272588in}{2.055294in}}{\pgfqpoint{3.266764in}{2.049470in}}%
\pgfpathcurveto{\pgfqpoint{3.260940in}{2.043646in}}{\pgfqpoint{3.257668in}{2.035746in}}{\pgfqpoint{3.257668in}{2.027510in}}%
\pgfpathcurveto{\pgfqpoint{3.257668in}{2.019273in}}{\pgfqpoint{3.260940in}{2.011373in}}{\pgfqpoint{3.266764in}{2.005549in}}%
\pgfpathcurveto{\pgfqpoint{3.272588in}{1.999725in}}{\pgfqpoint{3.280488in}{1.996453in}}{\pgfqpoint{3.288724in}{1.996453in}}%
\pgfpathclose%
\pgfusepath{stroke,fill}%
\end{pgfscope}%
\begin{pgfscope}%
\pgfpathrectangle{\pgfqpoint{0.100000in}{0.212622in}}{\pgfqpoint{3.696000in}{3.696000in}}%
\pgfusepath{clip}%
\pgfsetbuttcap%
\pgfsetroundjoin%
\definecolor{currentfill}{rgb}{0.121569,0.466667,0.705882}%
\pgfsetfillcolor{currentfill}%
\pgfsetfillopacity{0.567637}%
\pgfsetlinewidth{1.003750pt}%
\definecolor{currentstroke}{rgb}{0.121569,0.466667,0.705882}%
\pgfsetstrokecolor{currentstroke}%
\pgfsetstrokeopacity{0.567637}%
\pgfsetdash{}{0pt}%
\pgfpathmoveto{\pgfqpoint{3.288937in}{1.996086in}}%
\pgfpathcurveto{\pgfqpoint{3.297173in}{1.996086in}}{\pgfqpoint{3.305073in}{1.999358in}}{\pgfqpoint{3.310897in}{2.005182in}}%
\pgfpathcurveto{\pgfqpoint{3.316721in}{2.011006in}}{\pgfqpoint{3.319994in}{2.018906in}}{\pgfqpoint{3.319994in}{2.027143in}}%
\pgfpathcurveto{\pgfqpoint{3.319994in}{2.035379in}}{\pgfqpoint{3.316721in}{2.043279in}}{\pgfqpoint{3.310897in}{2.049103in}}%
\pgfpathcurveto{\pgfqpoint{3.305073in}{2.054927in}}{\pgfqpoint{3.297173in}{2.058199in}}{\pgfqpoint{3.288937in}{2.058199in}}%
\pgfpathcurveto{\pgfqpoint{3.280701in}{2.058199in}}{\pgfqpoint{3.272801in}{2.054927in}}{\pgfqpoint{3.266977in}{2.049103in}}%
\pgfpathcurveto{\pgfqpoint{3.261153in}{2.043279in}}{\pgfqpoint{3.257881in}{2.035379in}}{\pgfqpoint{3.257881in}{2.027143in}}%
\pgfpathcurveto{\pgfqpoint{3.257881in}{2.018906in}}{\pgfqpoint{3.261153in}{2.011006in}}{\pgfqpoint{3.266977in}{2.005182in}}%
\pgfpathcurveto{\pgfqpoint{3.272801in}{1.999358in}}{\pgfqpoint{3.280701in}{1.996086in}}{\pgfqpoint{3.288937in}{1.996086in}}%
\pgfpathclose%
\pgfusepath{stroke,fill}%
\end{pgfscope}%
\begin{pgfscope}%
\pgfpathrectangle{\pgfqpoint{0.100000in}{0.212622in}}{\pgfqpoint{3.696000in}{3.696000in}}%
\pgfusepath{clip}%
\pgfsetbuttcap%
\pgfsetroundjoin%
\definecolor{currentfill}{rgb}{0.121569,0.466667,0.705882}%
\pgfsetfillcolor{currentfill}%
\pgfsetfillopacity{0.568126}%
\pgfsetlinewidth{1.003750pt}%
\definecolor{currentstroke}{rgb}{0.121569,0.466667,0.705882}%
\pgfsetstrokecolor{currentstroke}%
\pgfsetstrokeopacity{0.568126}%
\pgfsetdash{}{0pt}%
\pgfpathmoveto{\pgfqpoint{3.288632in}{1.995795in}}%
\pgfpathcurveto{\pgfqpoint{3.296869in}{1.995795in}}{\pgfqpoint{3.304769in}{1.999068in}}{\pgfqpoint{3.310593in}{2.004892in}}%
\pgfpathcurveto{\pgfqpoint{3.316417in}{2.010716in}}{\pgfqpoint{3.319689in}{2.018616in}}{\pgfqpoint{3.319689in}{2.026852in}}%
\pgfpathcurveto{\pgfqpoint{3.319689in}{2.035088in}}{\pgfqpoint{3.316417in}{2.042988in}}{\pgfqpoint{3.310593in}{2.048812in}}%
\pgfpathcurveto{\pgfqpoint{3.304769in}{2.054636in}}{\pgfqpoint{3.296869in}{2.057908in}}{\pgfqpoint{3.288632in}{2.057908in}}%
\pgfpathcurveto{\pgfqpoint{3.280396in}{2.057908in}}{\pgfqpoint{3.272496in}{2.054636in}}{\pgfqpoint{3.266672in}{2.048812in}}%
\pgfpathcurveto{\pgfqpoint{3.260848in}{2.042988in}}{\pgfqpoint{3.257576in}{2.035088in}}{\pgfqpoint{3.257576in}{2.026852in}}%
\pgfpathcurveto{\pgfqpoint{3.257576in}{2.018616in}}{\pgfqpoint{3.260848in}{2.010716in}}{\pgfqpoint{3.266672in}{2.004892in}}%
\pgfpathcurveto{\pgfqpoint{3.272496in}{1.999068in}}{\pgfqpoint{3.280396in}{1.995795in}}{\pgfqpoint{3.288632in}{1.995795in}}%
\pgfpathclose%
\pgfusepath{stroke,fill}%
\end{pgfscope}%
\begin{pgfscope}%
\pgfpathrectangle{\pgfqpoint{0.100000in}{0.212622in}}{\pgfqpoint{3.696000in}{3.696000in}}%
\pgfusepath{clip}%
\pgfsetbuttcap%
\pgfsetroundjoin%
\definecolor{currentfill}{rgb}{0.121569,0.466667,0.705882}%
\pgfsetfillcolor{currentfill}%
\pgfsetfillopacity{0.568681}%
\pgfsetlinewidth{1.003750pt}%
\definecolor{currentstroke}{rgb}{0.121569,0.466667,0.705882}%
\pgfsetstrokecolor{currentstroke}%
\pgfsetstrokeopacity{0.568681}%
\pgfsetdash{}{0pt}%
\pgfpathmoveto{\pgfqpoint{3.288206in}{1.995180in}}%
\pgfpathcurveto{\pgfqpoint{3.296442in}{1.995180in}}{\pgfqpoint{3.304342in}{1.998452in}}{\pgfqpoint{3.310166in}{2.004276in}}%
\pgfpathcurveto{\pgfqpoint{3.315990in}{2.010100in}}{\pgfqpoint{3.319263in}{2.018000in}}{\pgfqpoint{3.319263in}{2.026236in}}%
\pgfpathcurveto{\pgfqpoint{3.319263in}{2.034473in}}{\pgfqpoint{3.315990in}{2.042373in}}{\pgfqpoint{3.310166in}{2.048197in}}%
\pgfpathcurveto{\pgfqpoint{3.304342in}{2.054021in}}{\pgfqpoint{3.296442in}{2.057293in}}{\pgfqpoint{3.288206in}{2.057293in}}%
\pgfpathcurveto{\pgfqpoint{3.279970in}{2.057293in}}{\pgfqpoint{3.272070in}{2.054021in}}{\pgfqpoint{3.266246in}{2.048197in}}%
\pgfpathcurveto{\pgfqpoint{3.260422in}{2.042373in}}{\pgfqpoint{3.257150in}{2.034473in}}{\pgfqpoint{3.257150in}{2.026236in}}%
\pgfpathcurveto{\pgfqpoint{3.257150in}{2.018000in}}{\pgfqpoint{3.260422in}{2.010100in}}{\pgfqpoint{3.266246in}{2.004276in}}%
\pgfpathcurveto{\pgfqpoint{3.272070in}{1.998452in}}{\pgfqpoint{3.279970in}{1.995180in}}{\pgfqpoint{3.288206in}{1.995180in}}%
\pgfpathclose%
\pgfusepath{stroke,fill}%
\end{pgfscope}%
\begin{pgfscope}%
\pgfpathrectangle{\pgfqpoint{0.100000in}{0.212622in}}{\pgfqpoint{3.696000in}{3.696000in}}%
\pgfusepath{clip}%
\pgfsetbuttcap%
\pgfsetroundjoin%
\definecolor{currentfill}{rgb}{0.121569,0.466667,0.705882}%
\pgfsetfillcolor{currentfill}%
\pgfsetfillopacity{0.569276}%
\pgfsetlinewidth{1.003750pt}%
\definecolor{currentstroke}{rgb}{0.121569,0.466667,0.705882}%
\pgfsetstrokecolor{currentstroke}%
\pgfsetstrokeopacity{0.569276}%
\pgfsetdash{}{0pt}%
\pgfpathmoveto{\pgfqpoint{3.287104in}{1.994647in}}%
\pgfpathcurveto{\pgfqpoint{3.295340in}{1.994647in}}{\pgfqpoint{3.303240in}{1.997920in}}{\pgfqpoint{3.309064in}{2.003743in}}%
\pgfpathcurveto{\pgfqpoint{3.314888in}{2.009567in}}{\pgfqpoint{3.318160in}{2.017467in}}{\pgfqpoint{3.318160in}{2.025704in}}%
\pgfpathcurveto{\pgfqpoint{3.318160in}{2.033940in}}{\pgfqpoint{3.314888in}{2.041840in}}{\pgfqpoint{3.309064in}{2.047664in}}%
\pgfpathcurveto{\pgfqpoint{3.303240in}{2.053488in}}{\pgfqpoint{3.295340in}{2.056760in}}{\pgfqpoint{3.287104in}{2.056760in}}%
\pgfpathcurveto{\pgfqpoint{3.278868in}{2.056760in}}{\pgfqpoint{3.270967in}{2.053488in}}{\pgfqpoint{3.265144in}{2.047664in}}%
\pgfpathcurveto{\pgfqpoint{3.259320in}{2.041840in}}{\pgfqpoint{3.256047in}{2.033940in}}{\pgfqpoint{3.256047in}{2.025704in}}%
\pgfpathcurveto{\pgfqpoint{3.256047in}{2.017467in}}{\pgfqpoint{3.259320in}{2.009567in}}{\pgfqpoint{3.265144in}{2.003743in}}%
\pgfpathcurveto{\pgfqpoint{3.270967in}{1.997920in}}{\pgfqpoint{3.278868in}{1.994647in}}{\pgfqpoint{3.287104in}{1.994647in}}%
\pgfpathclose%
\pgfusepath{stroke,fill}%
\end{pgfscope}%
\begin{pgfscope}%
\pgfpathrectangle{\pgfqpoint{0.100000in}{0.212622in}}{\pgfqpoint{3.696000in}{3.696000in}}%
\pgfusepath{clip}%
\pgfsetbuttcap%
\pgfsetroundjoin%
\definecolor{currentfill}{rgb}{0.121569,0.466667,0.705882}%
\pgfsetfillcolor{currentfill}%
\pgfsetfillopacity{0.570195}%
\pgfsetlinewidth{1.003750pt}%
\definecolor{currentstroke}{rgb}{0.121569,0.466667,0.705882}%
\pgfsetstrokecolor{currentstroke}%
\pgfsetstrokeopacity{0.570195}%
\pgfsetdash{}{0pt}%
\pgfpathmoveto{\pgfqpoint{3.285432in}{1.995842in}}%
\pgfpathcurveto{\pgfqpoint{3.293668in}{1.995842in}}{\pgfqpoint{3.301568in}{1.999114in}}{\pgfqpoint{3.307392in}{2.004938in}}%
\pgfpathcurveto{\pgfqpoint{3.313216in}{2.010762in}}{\pgfqpoint{3.316488in}{2.018662in}}{\pgfqpoint{3.316488in}{2.026898in}}%
\pgfpathcurveto{\pgfqpoint{3.316488in}{2.035134in}}{\pgfqpoint{3.313216in}{2.043034in}}{\pgfqpoint{3.307392in}{2.048858in}}%
\pgfpathcurveto{\pgfqpoint{3.301568in}{2.054682in}}{\pgfqpoint{3.293668in}{2.057955in}}{\pgfqpoint{3.285432in}{2.057955in}}%
\pgfpathcurveto{\pgfqpoint{3.277195in}{2.057955in}}{\pgfqpoint{3.269295in}{2.054682in}}{\pgfqpoint{3.263471in}{2.048858in}}%
\pgfpathcurveto{\pgfqpoint{3.257647in}{2.043034in}}{\pgfqpoint{3.254375in}{2.035134in}}{\pgfqpoint{3.254375in}{2.026898in}}%
\pgfpathcurveto{\pgfqpoint{3.254375in}{2.018662in}}{\pgfqpoint{3.257647in}{2.010762in}}{\pgfqpoint{3.263471in}{2.004938in}}%
\pgfpathcurveto{\pgfqpoint{3.269295in}{1.999114in}}{\pgfqpoint{3.277195in}{1.995842in}}{\pgfqpoint{3.285432in}{1.995842in}}%
\pgfpathclose%
\pgfusepath{stroke,fill}%
\end{pgfscope}%
\begin{pgfscope}%
\pgfpathrectangle{\pgfqpoint{0.100000in}{0.212622in}}{\pgfqpoint{3.696000in}{3.696000in}}%
\pgfusepath{clip}%
\pgfsetbuttcap%
\pgfsetroundjoin%
\definecolor{currentfill}{rgb}{0.121569,0.466667,0.705882}%
\pgfsetfillcolor{currentfill}%
\pgfsetfillopacity{0.570262}%
\pgfsetlinewidth{1.003750pt}%
\definecolor{currentstroke}{rgb}{0.121569,0.466667,0.705882}%
\pgfsetstrokecolor{currentstroke}%
\pgfsetstrokeopacity{0.570262}%
\pgfsetdash{}{0pt}%
\pgfpathmoveto{\pgfqpoint{1.009657in}{1.697324in}}%
\pgfpathcurveto{\pgfqpoint{1.017893in}{1.697324in}}{\pgfqpoint{1.025793in}{1.700596in}}{\pgfqpoint{1.031617in}{1.706420in}}%
\pgfpathcurveto{\pgfqpoint{1.037441in}{1.712244in}}{\pgfqpoint{1.040713in}{1.720144in}}{\pgfqpoint{1.040713in}{1.728380in}}%
\pgfpathcurveto{\pgfqpoint{1.040713in}{1.736616in}}{\pgfqpoint{1.037441in}{1.744516in}}{\pgfqpoint{1.031617in}{1.750340in}}%
\pgfpathcurveto{\pgfqpoint{1.025793in}{1.756164in}}{\pgfqpoint{1.017893in}{1.759437in}}{\pgfqpoint{1.009657in}{1.759437in}}%
\pgfpathcurveto{\pgfqpoint{1.001420in}{1.759437in}}{\pgfqpoint{0.993520in}{1.756164in}}{\pgfqpoint{0.987696in}{1.750340in}}%
\pgfpathcurveto{\pgfqpoint{0.981872in}{1.744516in}}{\pgfqpoint{0.978600in}{1.736616in}}{\pgfqpoint{0.978600in}{1.728380in}}%
\pgfpathcurveto{\pgfqpoint{0.978600in}{1.720144in}}{\pgfqpoint{0.981872in}{1.712244in}}{\pgfqpoint{0.987696in}{1.706420in}}%
\pgfpathcurveto{\pgfqpoint{0.993520in}{1.700596in}}{\pgfqpoint{1.001420in}{1.697324in}}{\pgfqpoint{1.009657in}{1.697324in}}%
\pgfpathclose%
\pgfusepath{stroke,fill}%
\end{pgfscope}%
\begin{pgfscope}%
\pgfpathrectangle{\pgfqpoint{0.100000in}{0.212622in}}{\pgfqpoint{3.696000in}{3.696000in}}%
\pgfusepath{clip}%
\pgfsetbuttcap%
\pgfsetroundjoin%
\definecolor{currentfill}{rgb}{0.121569,0.466667,0.705882}%
\pgfsetfillcolor{currentfill}%
\pgfsetfillopacity{0.571678}%
\pgfsetlinewidth{1.003750pt}%
\definecolor{currentstroke}{rgb}{0.121569,0.466667,0.705882}%
\pgfsetstrokecolor{currentstroke}%
\pgfsetstrokeopacity{0.571678}%
\pgfsetdash{}{0pt}%
\pgfpathmoveto{\pgfqpoint{3.283783in}{1.999968in}}%
\pgfpathcurveto{\pgfqpoint{3.292019in}{1.999968in}}{\pgfqpoint{3.299919in}{2.003240in}}{\pgfqpoint{3.305743in}{2.009064in}}%
\pgfpathcurveto{\pgfqpoint{3.311567in}{2.014888in}}{\pgfqpoint{3.314840in}{2.022788in}}{\pgfqpoint{3.314840in}{2.031024in}}%
\pgfpathcurveto{\pgfqpoint{3.314840in}{2.039260in}}{\pgfqpoint{3.311567in}{2.047160in}}{\pgfqpoint{3.305743in}{2.052984in}}%
\pgfpathcurveto{\pgfqpoint{3.299919in}{2.058808in}}{\pgfqpoint{3.292019in}{2.062081in}}{\pgfqpoint{3.283783in}{2.062081in}}%
\pgfpathcurveto{\pgfqpoint{3.275547in}{2.062081in}}{\pgfqpoint{3.267647in}{2.058808in}}{\pgfqpoint{3.261823in}{2.052984in}}%
\pgfpathcurveto{\pgfqpoint{3.255999in}{2.047160in}}{\pgfqpoint{3.252727in}{2.039260in}}{\pgfqpoint{3.252727in}{2.031024in}}%
\pgfpathcurveto{\pgfqpoint{3.252727in}{2.022788in}}{\pgfqpoint{3.255999in}{2.014888in}}{\pgfqpoint{3.261823in}{2.009064in}}%
\pgfpathcurveto{\pgfqpoint{3.267647in}{2.003240in}}{\pgfqpoint{3.275547in}{1.999968in}}{\pgfqpoint{3.283783in}{1.999968in}}%
\pgfpathclose%
\pgfusepath{stroke,fill}%
\end{pgfscope}%
\begin{pgfscope}%
\pgfpathrectangle{\pgfqpoint{0.100000in}{0.212622in}}{\pgfqpoint{3.696000in}{3.696000in}}%
\pgfusepath{clip}%
\pgfsetbuttcap%
\pgfsetroundjoin%
\definecolor{currentfill}{rgb}{0.121569,0.466667,0.705882}%
\pgfsetfillcolor{currentfill}%
\pgfsetfillopacity{0.572090}%
\pgfsetlinewidth{1.003750pt}%
\definecolor{currentstroke}{rgb}{0.121569,0.466667,0.705882}%
\pgfsetstrokecolor{currentstroke}%
\pgfsetstrokeopacity{0.572090}%
\pgfsetdash{}{0pt}%
\pgfpathmoveto{\pgfqpoint{3.282547in}{1.999690in}}%
\pgfpathcurveto{\pgfqpoint{3.290783in}{1.999690in}}{\pgfqpoint{3.298683in}{2.002962in}}{\pgfqpoint{3.304507in}{2.008786in}}%
\pgfpathcurveto{\pgfqpoint{3.310331in}{2.014610in}}{\pgfqpoint{3.313603in}{2.022510in}}{\pgfqpoint{3.313603in}{2.030747in}}%
\pgfpathcurveto{\pgfqpoint{3.313603in}{2.038983in}}{\pgfqpoint{3.310331in}{2.046883in}}{\pgfqpoint{3.304507in}{2.052707in}}%
\pgfpathcurveto{\pgfqpoint{3.298683in}{2.058531in}}{\pgfqpoint{3.290783in}{2.061803in}}{\pgfqpoint{3.282547in}{2.061803in}}%
\pgfpathcurveto{\pgfqpoint{3.274311in}{2.061803in}}{\pgfqpoint{3.266410in}{2.058531in}}{\pgfqpoint{3.260587in}{2.052707in}}%
\pgfpathcurveto{\pgfqpoint{3.254763in}{2.046883in}}{\pgfqpoint{3.251490in}{2.038983in}}{\pgfqpoint{3.251490in}{2.030747in}}%
\pgfpathcurveto{\pgfqpoint{3.251490in}{2.022510in}}{\pgfqpoint{3.254763in}{2.014610in}}{\pgfqpoint{3.260587in}{2.008786in}}%
\pgfpathcurveto{\pgfqpoint{3.266410in}{2.002962in}}{\pgfqpoint{3.274311in}{1.999690in}}{\pgfqpoint{3.282547in}{1.999690in}}%
\pgfpathclose%
\pgfusepath{stroke,fill}%
\end{pgfscope}%
\begin{pgfscope}%
\pgfpathrectangle{\pgfqpoint{0.100000in}{0.212622in}}{\pgfqpoint{3.696000in}{3.696000in}}%
\pgfusepath{clip}%
\pgfsetbuttcap%
\pgfsetroundjoin%
\definecolor{currentfill}{rgb}{0.121569,0.466667,0.705882}%
\pgfsetfillcolor{currentfill}%
\pgfsetfillopacity{0.572273}%
\pgfsetlinewidth{1.003750pt}%
\definecolor{currentstroke}{rgb}{0.121569,0.466667,0.705882}%
\pgfsetstrokecolor{currentstroke}%
\pgfsetstrokeopacity{0.572273}%
\pgfsetdash{}{0pt}%
\pgfpathmoveto{\pgfqpoint{1.001144in}{1.697579in}}%
\pgfpathcurveto{\pgfqpoint{1.009380in}{1.697579in}}{\pgfqpoint{1.017280in}{1.700851in}}{\pgfqpoint{1.023104in}{1.706675in}}%
\pgfpathcurveto{\pgfqpoint{1.028928in}{1.712499in}}{\pgfqpoint{1.032200in}{1.720399in}}{\pgfqpoint{1.032200in}{1.728635in}}%
\pgfpathcurveto{\pgfqpoint{1.032200in}{1.736872in}}{\pgfqpoint{1.028928in}{1.744772in}}{\pgfqpoint{1.023104in}{1.750596in}}%
\pgfpathcurveto{\pgfqpoint{1.017280in}{1.756420in}}{\pgfqpoint{1.009380in}{1.759692in}}{\pgfqpoint{1.001144in}{1.759692in}}%
\pgfpathcurveto{\pgfqpoint{0.992907in}{1.759692in}}{\pgfqpoint{0.985007in}{1.756420in}}{\pgfqpoint{0.979183in}{1.750596in}}%
\pgfpathcurveto{\pgfqpoint{0.973359in}{1.744772in}}{\pgfqpoint{0.970087in}{1.736872in}}{\pgfqpoint{0.970087in}{1.728635in}}%
\pgfpathcurveto{\pgfqpoint{0.970087in}{1.720399in}}{\pgfqpoint{0.973359in}{1.712499in}}{\pgfqpoint{0.979183in}{1.706675in}}%
\pgfpathcurveto{\pgfqpoint{0.985007in}{1.700851in}}{\pgfqpoint{0.992907in}{1.697579in}}{\pgfqpoint{1.001144in}{1.697579in}}%
\pgfpathclose%
\pgfusepath{stroke,fill}%
\end{pgfscope}%
\begin{pgfscope}%
\pgfpathrectangle{\pgfqpoint{0.100000in}{0.212622in}}{\pgfqpoint{3.696000in}{3.696000in}}%
\pgfusepath{clip}%
\pgfsetbuttcap%
\pgfsetroundjoin%
\definecolor{currentfill}{rgb}{0.121569,0.466667,0.705882}%
\pgfsetfillcolor{currentfill}%
\pgfsetfillopacity{0.572361}%
\pgfsetlinewidth{1.003750pt}%
\definecolor{currentstroke}{rgb}{0.121569,0.466667,0.705882}%
\pgfsetstrokecolor{currentstroke}%
\pgfsetstrokeopacity{0.572361}%
\pgfsetdash{}{0pt}%
\pgfpathmoveto{\pgfqpoint{3.281923in}{1.999781in}}%
\pgfpathcurveto{\pgfqpoint{3.290160in}{1.999781in}}{\pgfqpoint{3.298060in}{2.003054in}}{\pgfqpoint{3.303884in}{2.008878in}}%
\pgfpathcurveto{\pgfqpoint{3.309707in}{2.014701in}}{\pgfqpoint{3.312980in}{2.022602in}}{\pgfqpoint{3.312980in}{2.030838in}}%
\pgfpathcurveto{\pgfqpoint{3.312980in}{2.039074in}}{\pgfqpoint{3.309707in}{2.046974in}}{\pgfqpoint{3.303884in}{2.052798in}}%
\pgfpathcurveto{\pgfqpoint{3.298060in}{2.058622in}}{\pgfqpoint{3.290160in}{2.061894in}}{\pgfqpoint{3.281923in}{2.061894in}}%
\pgfpathcurveto{\pgfqpoint{3.273687in}{2.061894in}}{\pgfqpoint{3.265787in}{2.058622in}}{\pgfqpoint{3.259963in}{2.052798in}}%
\pgfpathcurveto{\pgfqpoint{3.254139in}{2.046974in}}{\pgfqpoint{3.250867in}{2.039074in}}{\pgfqpoint{3.250867in}{2.030838in}}%
\pgfpathcurveto{\pgfqpoint{3.250867in}{2.022602in}}{\pgfqpoint{3.254139in}{2.014701in}}{\pgfqpoint{3.259963in}{2.008878in}}%
\pgfpathcurveto{\pgfqpoint{3.265787in}{2.003054in}}{\pgfqpoint{3.273687in}{1.999781in}}{\pgfqpoint{3.281923in}{1.999781in}}%
\pgfpathclose%
\pgfusepath{stroke,fill}%
\end{pgfscope}%
\begin{pgfscope}%
\pgfpathrectangle{\pgfqpoint{0.100000in}{0.212622in}}{\pgfqpoint{3.696000in}{3.696000in}}%
\pgfusepath{clip}%
\pgfsetbuttcap%
\pgfsetroundjoin%
\definecolor{currentfill}{rgb}{0.121569,0.466667,0.705882}%
\pgfsetfillcolor{currentfill}%
\pgfsetfillopacity{0.572743}%
\pgfsetlinewidth{1.003750pt}%
\definecolor{currentstroke}{rgb}{0.121569,0.466667,0.705882}%
\pgfsetstrokecolor{currentstroke}%
\pgfsetstrokeopacity{0.572743}%
\pgfsetdash{}{0pt}%
\pgfpathmoveto{\pgfqpoint{3.281145in}{2.000126in}}%
\pgfpathcurveto{\pgfqpoint{3.289382in}{2.000126in}}{\pgfqpoint{3.297282in}{2.003399in}}{\pgfqpoint{3.303106in}{2.009223in}}%
\pgfpathcurveto{\pgfqpoint{3.308929in}{2.015046in}}{\pgfqpoint{3.312202in}{2.022946in}}{\pgfqpoint{3.312202in}{2.031183in}}%
\pgfpathcurveto{\pgfqpoint{3.312202in}{2.039419in}}{\pgfqpoint{3.308929in}{2.047319in}}{\pgfqpoint{3.303106in}{2.053143in}}%
\pgfpathcurveto{\pgfqpoint{3.297282in}{2.058967in}}{\pgfqpoint{3.289382in}{2.062239in}}{\pgfqpoint{3.281145in}{2.062239in}}%
\pgfpathcurveto{\pgfqpoint{3.272909in}{2.062239in}}{\pgfqpoint{3.265009in}{2.058967in}}{\pgfqpoint{3.259185in}{2.053143in}}%
\pgfpathcurveto{\pgfqpoint{3.253361in}{2.047319in}}{\pgfqpoint{3.250089in}{2.039419in}}{\pgfqpoint{3.250089in}{2.031183in}}%
\pgfpathcurveto{\pgfqpoint{3.250089in}{2.022946in}}{\pgfqpoint{3.253361in}{2.015046in}}{\pgfqpoint{3.259185in}{2.009223in}}%
\pgfpathcurveto{\pgfqpoint{3.265009in}{2.003399in}}{\pgfqpoint{3.272909in}{2.000126in}}{\pgfqpoint{3.281145in}{2.000126in}}%
\pgfpathclose%
\pgfusepath{stroke,fill}%
\end{pgfscope}%
\begin{pgfscope}%
\pgfpathrectangle{\pgfqpoint{0.100000in}{0.212622in}}{\pgfqpoint{3.696000in}{3.696000in}}%
\pgfusepath{clip}%
\pgfsetbuttcap%
\pgfsetroundjoin%
\definecolor{currentfill}{rgb}{0.121569,0.466667,0.705882}%
\pgfsetfillcolor{currentfill}%
\pgfsetfillopacity{0.572930}%
\pgfsetlinewidth{1.003750pt}%
\definecolor{currentstroke}{rgb}{0.121569,0.466667,0.705882}%
\pgfsetstrokecolor{currentstroke}%
\pgfsetstrokeopacity{0.572930}%
\pgfsetdash{}{0pt}%
\pgfpathmoveto{\pgfqpoint{3.280666in}{2.000234in}}%
\pgfpathcurveto{\pgfqpoint{3.288902in}{2.000234in}}{\pgfqpoint{3.296802in}{2.003507in}}{\pgfqpoint{3.302626in}{2.009331in}}%
\pgfpathcurveto{\pgfqpoint{3.308450in}{2.015154in}}{\pgfqpoint{3.311722in}{2.023054in}}{\pgfqpoint{3.311722in}{2.031291in}}%
\pgfpathcurveto{\pgfqpoint{3.311722in}{2.039527in}}{\pgfqpoint{3.308450in}{2.047427in}}{\pgfqpoint{3.302626in}{2.053251in}}%
\pgfpathcurveto{\pgfqpoint{3.296802in}{2.059075in}}{\pgfqpoint{3.288902in}{2.062347in}}{\pgfqpoint{3.280666in}{2.062347in}}%
\pgfpathcurveto{\pgfqpoint{3.272429in}{2.062347in}}{\pgfqpoint{3.264529in}{2.059075in}}{\pgfqpoint{3.258705in}{2.053251in}}%
\pgfpathcurveto{\pgfqpoint{3.252881in}{2.047427in}}{\pgfqpoint{3.249609in}{2.039527in}}{\pgfqpoint{3.249609in}{2.031291in}}%
\pgfpathcurveto{\pgfqpoint{3.249609in}{2.023054in}}{\pgfqpoint{3.252881in}{2.015154in}}{\pgfqpoint{3.258705in}{2.009331in}}%
\pgfpathcurveto{\pgfqpoint{3.264529in}{2.003507in}}{\pgfqpoint{3.272429in}{2.000234in}}{\pgfqpoint{3.280666in}{2.000234in}}%
\pgfpathclose%
\pgfusepath{stroke,fill}%
\end{pgfscope}%
\begin{pgfscope}%
\pgfpathrectangle{\pgfqpoint{0.100000in}{0.212622in}}{\pgfqpoint{3.696000in}{3.696000in}}%
\pgfusepath{clip}%
\pgfsetbuttcap%
\pgfsetroundjoin%
\definecolor{currentfill}{rgb}{0.121569,0.466667,0.705882}%
\pgfsetfillcolor{currentfill}%
\pgfsetfillopacity{0.573291}%
\pgfsetlinewidth{1.003750pt}%
\definecolor{currentstroke}{rgb}{0.121569,0.466667,0.705882}%
\pgfsetstrokecolor{currentstroke}%
\pgfsetstrokeopacity{0.573291}%
\pgfsetdash{}{0pt}%
\pgfpathmoveto{\pgfqpoint{3.279854in}{2.000048in}}%
\pgfpathcurveto{\pgfqpoint{3.288090in}{2.000048in}}{\pgfqpoint{3.295990in}{2.003320in}}{\pgfqpoint{3.301814in}{2.009144in}}%
\pgfpathcurveto{\pgfqpoint{3.307638in}{2.014968in}}{\pgfqpoint{3.310910in}{2.022868in}}{\pgfqpoint{3.310910in}{2.031104in}}%
\pgfpathcurveto{\pgfqpoint{3.310910in}{2.039340in}}{\pgfqpoint{3.307638in}{2.047240in}}{\pgfqpoint{3.301814in}{2.053064in}}%
\pgfpathcurveto{\pgfqpoint{3.295990in}{2.058888in}}{\pgfqpoint{3.288090in}{2.062161in}}{\pgfqpoint{3.279854in}{2.062161in}}%
\pgfpathcurveto{\pgfqpoint{3.271617in}{2.062161in}}{\pgfqpoint{3.263717in}{2.058888in}}{\pgfqpoint{3.257894in}{2.053064in}}%
\pgfpathcurveto{\pgfqpoint{3.252070in}{2.047240in}}{\pgfqpoint{3.248797in}{2.039340in}}{\pgfqpoint{3.248797in}{2.031104in}}%
\pgfpathcurveto{\pgfqpoint{3.248797in}{2.022868in}}{\pgfqpoint{3.252070in}{2.014968in}}{\pgfqpoint{3.257894in}{2.009144in}}%
\pgfpathcurveto{\pgfqpoint{3.263717in}{2.003320in}}{\pgfqpoint{3.271617in}{2.000048in}}{\pgfqpoint{3.279854in}{2.000048in}}%
\pgfpathclose%
\pgfusepath{stroke,fill}%
\end{pgfscope}%
\begin{pgfscope}%
\pgfpathrectangle{\pgfqpoint{0.100000in}{0.212622in}}{\pgfqpoint{3.696000in}{3.696000in}}%
\pgfusepath{clip}%
\pgfsetbuttcap%
\pgfsetroundjoin%
\definecolor{currentfill}{rgb}{0.121569,0.466667,0.705882}%
\pgfsetfillcolor{currentfill}%
\pgfsetfillopacity{0.573489}%
\pgfsetlinewidth{1.003750pt}%
\definecolor{currentstroke}{rgb}{0.121569,0.466667,0.705882}%
\pgfsetstrokecolor{currentstroke}%
\pgfsetstrokeopacity{0.573489}%
\pgfsetdash{}{0pt}%
\pgfpathmoveto{\pgfqpoint{3.279178in}{2.000384in}}%
\pgfpathcurveto{\pgfqpoint{3.287414in}{2.000384in}}{\pgfqpoint{3.295314in}{2.003656in}}{\pgfqpoint{3.301138in}{2.009480in}}%
\pgfpathcurveto{\pgfqpoint{3.306962in}{2.015304in}}{\pgfqpoint{3.310234in}{2.023204in}}{\pgfqpoint{3.310234in}{2.031441in}}%
\pgfpathcurveto{\pgfqpoint{3.310234in}{2.039677in}}{\pgfqpoint{3.306962in}{2.047577in}}{\pgfqpoint{3.301138in}{2.053401in}}%
\pgfpathcurveto{\pgfqpoint{3.295314in}{2.059225in}}{\pgfqpoint{3.287414in}{2.062497in}}{\pgfqpoint{3.279178in}{2.062497in}}%
\pgfpathcurveto{\pgfqpoint{3.270941in}{2.062497in}}{\pgfqpoint{3.263041in}{2.059225in}}{\pgfqpoint{3.257217in}{2.053401in}}%
\pgfpathcurveto{\pgfqpoint{3.251393in}{2.047577in}}{\pgfqpoint{3.248121in}{2.039677in}}{\pgfqpoint{3.248121in}{2.031441in}}%
\pgfpathcurveto{\pgfqpoint{3.248121in}{2.023204in}}{\pgfqpoint{3.251393in}{2.015304in}}{\pgfqpoint{3.257217in}{2.009480in}}%
\pgfpathcurveto{\pgfqpoint{3.263041in}{2.003656in}}{\pgfqpoint{3.270941in}{2.000384in}}{\pgfqpoint{3.279178in}{2.000384in}}%
\pgfpathclose%
\pgfusepath{stroke,fill}%
\end{pgfscope}%
\begin{pgfscope}%
\pgfpathrectangle{\pgfqpoint{0.100000in}{0.212622in}}{\pgfqpoint{3.696000in}{3.696000in}}%
\pgfusepath{clip}%
\pgfsetbuttcap%
\pgfsetroundjoin%
\definecolor{currentfill}{rgb}{0.121569,0.466667,0.705882}%
\pgfsetfillcolor{currentfill}%
\pgfsetfillopacity{0.573547}%
\pgfsetlinewidth{1.003750pt}%
\definecolor{currentstroke}{rgb}{0.121569,0.466667,0.705882}%
\pgfsetstrokecolor{currentstroke}%
\pgfsetstrokeopacity{0.573547}%
\pgfsetdash{}{0pt}%
\pgfpathmoveto{\pgfqpoint{3.278899in}{1.999990in}}%
\pgfpathcurveto{\pgfqpoint{3.287135in}{1.999990in}}{\pgfqpoint{3.295035in}{2.003263in}}{\pgfqpoint{3.300859in}{2.009087in}}%
\pgfpathcurveto{\pgfqpoint{3.306683in}{2.014911in}}{\pgfqpoint{3.309955in}{2.022811in}}{\pgfqpoint{3.309955in}{2.031047in}}%
\pgfpathcurveto{\pgfqpoint{3.309955in}{2.039283in}}{\pgfqpoint{3.306683in}{2.047183in}}{\pgfqpoint{3.300859in}{2.053007in}}%
\pgfpathcurveto{\pgfqpoint{3.295035in}{2.058831in}}{\pgfqpoint{3.287135in}{2.062103in}}{\pgfqpoint{3.278899in}{2.062103in}}%
\pgfpathcurveto{\pgfqpoint{3.270662in}{2.062103in}}{\pgfqpoint{3.262762in}{2.058831in}}{\pgfqpoint{3.256938in}{2.053007in}}%
\pgfpathcurveto{\pgfqpoint{3.251115in}{2.047183in}}{\pgfqpoint{3.247842in}{2.039283in}}{\pgfqpoint{3.247842in}{2.031047in}}%
\pgfpathcurveto{\pgfqpoint{3.247842in}{2.022811in}}{\pgfqpoint{3.251115in}{2.014911in}}{\pgfqpoint{3.256938in}{2.009087in}}%
\pgfpathcurveto{\pgfqpoint{3.262762in}{2.003263in}}{\pgfqpoint{3.270662in}{1.999990in}}{\pgfqpoint{3.278899in}{1.999990in}}%
\pgfpathclose%
\pgfusepath{stroke,fill}%
\end{pgfscope}%
\begin{pgfscope}%
\pgfpathrectangle{\pgfqpoint{0.100000in}{0.212622in}}{\pgfqpoint{3.696000in}{3.696000in}}%
\pgfusepath{clip}%
\pgfsetbuttcap%
\pgfsetroundjoin%
\definecolor{currentfill}{rgb}{0.121569,0.466667,0.705882}%
\pgfsetfillcolor{currentfill}%
\pgfsetfillopacity{0.573978}%
\pgfsetlinewidth{1.003750pt}%
\definecolor{currentstroke}{rgb}{0.121569,0.466667,0.705882}%
\pgfsetstrokecolor{currentstroke}%
\pgfsetstrokeopacity{0.573978}%
\pgfsetdash{}{0pt}%
\pgfpathmoveto{\pgfqpoint{3.278254in}{2.000261in}}%
\pgfpathcurveto{\pgfqpoint{3.286491in}{2.000261in}}{\pgfqpoint{3.294391in}{2.003533in}}{\pgfqpoint{3.300215in}{2.009357in}}%
\pgfpathcurveto{\pgfqpoint{3.306039in}{2.015181in}}{\pgfqpoint{3.309311in}{2.023081in}}{\pgfqpoint{3.309311in}{2.031318in}}%
\pgfpathcurveto{\pgfqpoint{3.309311in}{2.039554in}}{\pgfqpoint{3.306039in}{2.047454in}}{\pgfqpoint{3.300215in}{2.053278in}}%
\pgfpathcurveto{\pgfqpoint{3.294391in}{2.059102in}}{\pgfqpoint{3.286491in}{2.062374in}}{\pgfqpoint{3.278254in}{2.062374in}}%
\pgfpathcurveto{\pgfqpoint{3.270018in}{2.062374in}}{\pgfqpoint{3.262118in}{2.059102in}}{\pgfqpoint{3.256294in}{2.053278in}}%
\pgfpathcurveto{\pgfqpoint{3.250470in}{2.047454in}}{\pgfqpoint{3.247198in}{2.039554in}}{\pgfqpoint{3.247198in}{2.031318in}}%
\pgfpathcurveto{\pgfqpoint{3.247198in}{2.023081in}}{\pgfqpoint{3.250470in}{2.015181in}}{\pgfqpoint{3.256294in}{2.009357in}}%
\pgfpathcurveto{\pgfqpoint{3.262118in}{2.003533in}}{\pgfqpoint{3.270018in}{2.000261in}}{\pgfqpoint{3.278254in}{2.000261in}}%
\pgfpathclose%
\pgfusepath{stroke,fill}%
\end{pgfscope}%
\begin{pgfscope}%
\pgfpathrectangle{\pgfqpoint{0.100000in}{0.212622in}}{\pgfqpoint{3.696000in}{3.696000in}}%
\pgfusepath{clip}%
\pgfsetbuttcap%
\pgfsetroundjoin%
\definecolor{currentfill}{rgb}{0.121569,0.466667,0.705882}%
\pgfsetfillcolor{currentfill}%
\pgfsetfillopacity{0.574406}%
\pgfsetlinewidth{1.003750pt}%
\definecolor{currentstroke}{rgb}{0.121569,0.466667,0.705882}%
\pgfsetstrokecolor{currentstroke}%
\pgfsetstrokeopacity{0.574406}%
\pgfsetdash{}{0pt}%
\pgfpathmoveto{\pgfqpoint{0.997992in}{1.694384in}}%
\pgfpathcurveto{\pgfqpoint{1.006228in}{1.694384in}}{\pgfqpoint{1.014129in}{1.697656in}}{\pgfqpoint{1.019952in}{1.703480in}}%
\pgfpathcurveto{\pgfqpoint{1.025776in}{1.709304in}}{\pgfqpoint{1.029049in}{1.717204in}}{\pgfqpoint{1.029049in}{1.725440in}}%
\pgfpathcurveto{\pgfqpoint{1.029049in}{1.733676in}}{\pgfqpoint{1.025776in}{1.741576in}}{\pgfqpoint{1.019952in}{1.747400in}}%
\pgfpathcurveto{\pgfqpoint{1.014129in}{1.753224in}}{\pgfqpoint{1.006228in}{1.756497in}}{\pgfqpoint{0.997992in}{1.756497in}}%
\pgfpathcurveto{\pgfqpoint{0.989756in}{1.756497in}}{\pgfqpoint{0.981856in}{1.753224in}}{\pgfqpoint{0.976032in}{1.747400in}}%
\pgfpathcurveto{\pgfqpoint{0.970208in}{1.741576in}}{\pgfqpoint{0.966936in}{1.733676in}}{\pgfqpoint{0.966936in}{1.725440in}}%
\pgfpathcurveto{\pgfqpoint{0.966936in}{1.717204in}}{\pgfqpoint{0.970208in}{1.709304in}}{\pgfqpoint{0.976032in}{1.703480in}}%
\pgfpathcurveto{\pgfqpoint{0.981856in}{1.697656in}}{\pgfqpoint{0.989756in}{1.694384in}}{\pgfqpoint{0.997992in}{1.694384in}}%
\pgfpathclose%
\pgfusepath{stroke,fill}%
\end{pgfscope}%
\begin{pgfscope}%
\pgfpathrectangle{\pgfqpoint{0.100000in}{0.212622in}}{\pgfqpoint{3.696000in}{3.696000in}}%
\pgfusepath{clip}%
\pgfsetbuttcap%
\pgfsetroundjoin%
\definecolor{currentfill}{rgb}{0.121569,0.466667,0.705882}%
\pgfsetfillcolor{currentfill}%
\pgfsetfillopacity{0.574407}%
\pgfsetlinewidth{1.003750pt}%
\definecolor{currentstroke}{rgb}{0.121569,0.466667,0.705882}%
\pgfsetstrokecolor{currentstroke}%
\pgfsetstrokeopacity{0.574407}%
\pgfsetdash{}{0pt}%
\pgfpathmoveto{\pgfqpoint{3.276946in}{2.000655in}}%
\pgfpathcurveto{\pgfqpoint{3.285182in}{2.000655in}}{\pgfqpoint{3.293082in}{2.003927in}}{\pgfqpoint{3.298906in}{2.009751in}}%
\pgfpathcurveto{\pgfqpoint{3.304730in}{2.015575in}}{\pgfqpoint{3.308002in}{2.023475in}}{\pgfqpoint{3.308002in}{2.031711in}}%
\pgfpathcurveto{\pgfqpoint{3.308002in}{2.039948in}}{\pgfqpoint{3.304730in}{2.047848in}}{\pgfqpoint{3.298906in}{2.053672in}}%
\pgfpathcurveto{\pgfqpoint{3.293082in}{2.059496in}}{\pgfqpoint{3.285182in}{2.062768in}}{\pgfqpoint{3.276946in}{2.062768in}}%
\pgfpathcurveto{\pgfqpoint{3.268709in}{2.062768in}}{\pgfqpoint{3.260809in}{2.059496in}}{\pgfqpoint{3.254985in}{2.053672in}}%
\pgfpathcurveto{\pgfqpoint{3.249161in}{2.047848in}}{\pgfqpoint{3.245889in}{2.039948in}}{\pgfqpoint{3.245889in}{2.031711in}}%
\pgfpathcurveto{\pgfqpoint{3.245889in}{2.023475in}}{\pgfqpoint{3.249161in}{2.015575in}}{\pgfqpoint{3.254985in}{2.009751in}}%
\pgfpathcurveto{\pgfqpoint{3.260809in}{2.003927in}}{\pgfqpoint{3.268709in}{2.000655in}}{\pgfqpoint{3.276946in}{2.000655in}}%
\pgfpathclose%
\pgfusepath{stroke,fill}%
\end{pgfscope}%
\begin{pgfscope}%
\pgfpathrectangle{\pgfqpoint{0.100000in}{0.212622in}}{\pgfqpoint{3.696000in}{3.696000in}}%
\pgfusepath{clip}%
\pgfsetbuttcap%
\pgfsetroundjoin%
\definecolor{currentfill}{rgb}{0.121569,0.466667,0.705882}%
\pgfsetfillcolor{currentfill}%
\pgfsetfillopacity{0.574650}%
\pgfsetlinewidth{1.003750pt}%
\definecolor{currentstroke}{rgb}{0.121569,0.466667,0.705882}%
\pgfsetstrokecolor{currentstroke}%
\pgfsetstrokeopacity{0.574650}%
\pgfsetdash{}{0pt}%
\pgfpathmoveto{\pgfqpoint{3.276257in}{2.000866in}}%
\pgfpathcurveto{\pgfqpoint{3.284493in}{2.000866in}}{\pgfqpoint{3.292393in}{2.004138in}}{\pgfqpoint{3.298217in}{2.009962in}}%
\pgfpathcurveto{\pgfqpoint{3.304041in}{2.015786in}}{\pgfqpoint{3.307313in}{2.023686in}}{\pgfqpoint{3.307313in}{2.031922in}}%
\pgfpathcurveto{\pgfqpoint{3.307313in}{2.040158in}}{\pgfqpoint{3.304041in}{2.048059in}}{\pgfqpoint{3.298217in}{2.053882in}}%
\pgfpathcurveto{\pgfqpoint{3.292393in}{2.059706in}}{\pgfqpoint{3.284493in}{2.062979in}}{\pgfqpoint{3.276257in}{2.062979in}}%
\pgfpathcurveto{\pgfqpoint{3.268021in}{2.062979in}}{\pgfqpoint{3.260121in}{2.059706in}}{\pgfqpoint{3.254297in}{2.053882in}}%
\pgfpathcurveto{\pgfqpoint{3.248473in}{2.048059in}}{\pgfqpoint{3.245200in}{2.040158in}}{\pgfqpoint{3.245200in}{2.031922in}}%
\pgfpathcurveto{\pgfqpoint{3.245200in}{2.023686in}}{\pgfqpoint{3.248473in}{2.015786in}}{\pgfqpoint{3.254297in}{2.009962in}}%
\pgfpathcurveto{\pgfqpoint{3.260121in}{2.004138in}}{\pgfqpoint{3.268021in}{2.000866in}}{\pgfqpoint{3.276257in}{2.000866in}}%
\pgfpathclose%
\pgfusepath{stroke,fill}%
\end{pgfscope}%
\begin{pgfscope}%
\pgfpathrectangle{\pgfqpoint{0.100000in}{0.212622in}}{\pgfqpoint{3.696000in}{3.696000in}}%
\pgfusepath{clip}%
\pgfsetbuttcap%
\pgfsetroundjoin%
\definecolor{currentfill}{rgb}{0.121569,0.466667,0.705882}%
\pgfsetfillcolor{currentfill}%
\pgfsetfillopacity{0.575413}%
\pgfsetlinewidth{1.003750pt}%
\definecolor{currentstroke}{rgb}{0.121569,0.466667,0.705882}%
\pgfsetstrokecolor{currentstroke}%
\pgfsetstrokeopacity{0.575413}%
\pgfsetdash{}{0pt}%
\pgfpathmoveto{\pgfqpoint{0.995157in}{1.687892in}}%
\pgfpathcurveto{\pgfqpoint{1.003393in}{1.687892in}}{\pgfqpoint{1.011293in}{1.691165in}}{\pgfqpoint{1.017117in}{1.696989in}}%
\pgfpathcurveto{\pgfqpoint{1.022941in}{1.702813in}}{\pgfqpoint{1.026213in}{1.710713in}}{\pgfqpoint{1.026213in}{1.718949in}}%
\pgfpathcurveto{\pgfqpoint{1.026213in}{1.727185in}}{\pgfqpoint{1.022941in}{1.735085in}}{\pgfqpoint{1.017117in}{1.740909in}}%
\pgfpathcurveto{\pgfqpoint{1.011293in}{1.746733in}}{\pgfqpoint{1.003393in}{1.750005in}}{\pgfqpoint{0.995157in}{1.750005in}}%
\pgfpathcurveto{\pgfqpoint{0.986921in}{1.750005in}}{\pgfqpoint{0.979021in}{1.746733in}}{\pgfqpoint{0.973197in}{1.740909in}}%
\pgfpathcurveto{\pgfqpoint{0.967373in}{1.735085in}}{\pgfqpoint{0.964100in}{1.727185in}}{\pgfqpoint{0.964100in}{1.718949in}}%
\pgfpathcurveto{\pgfqpoint{0.964100in}{1.710713in}}{\pgfqpoint{0.967373in}{1.702813in}}{\pgfqpoint{0.973197in}{1.696989in}}%
\pgfpathcurveto{\pgfqpoint{0.979021in}{1.691165in}}{\pgfqpoint{0.986921in}{1.687892in}}{\pgfqpoint{0.995157in}{1.687892in}}%
\pgfpathclose%
\pgfusepath{stroke,fill}%
\end{pgfscope}%
\begin{pgfscope}%
\pgfpathrectangle{\pgfqpoint{0.100000in}{0.212622in}}{\pgfqpoint{3.696000in}{3.696000in}}%
\pgfusepath{clip}%
\pgfsetbuttcap%
\pgfsetroundjoin%
\definecolor{currentfill}{rgb}{0.121569,0.466667,0.705882}%
\pgfsetfillcolor{currentfill}%
\pgfsetfillopacity{0.575459}%
\pgfsetlinewidth{1.003750pt}%
\definecolor{currentstroke}{rgb}{0.121569,0.466667,0.705882}%
\pgfsetstrokecolor{currentstroke}%
\pgfsetstrokeopacity{0.575459}%
\pgfsetdash{}{0pt}%
\pgfpathmoveto{\pgfqpoint{3.274992in}{2.001413in}}%
\pgfpathcurveto{\pgfqpoint{3.283229in}{2.001413in}}{\pgfqpoint{3.291129in}{2.004686in}}{\pgfqpoint{3.296953in}{2.010510in}}%
\pgfpathcurveto{\pgfqpoint{3.302777in}{2.016334in}}{\pgfqpoint{3.306049in}{2.024234in}}{\pgfqpoint{3.306049in}{2.032470in}}%
\pgfpathcurveto{\pgfqpoint{3.306049in}{2.040706in}}{\pgfqpoint{3.302777in}{2.048606in}}{\pgfqpoint{3.296953in}{2.054430in}}%
\pgfpathcurveto{\pgfqpoint{3.291129in}{2.060254in}}{\pgfqpoint{3.283229in}{2.063526in}}{\pgfqpoint{3.274992in}{2.063526in}}%
\pgfpathcurveto{\pgfqpoint{3.266756in}{2.063526in}}{\pgfqpoint{3.258856in}{2.060254in}}{\pgfqpoint{3.253032in}{2.054430in}}%
\pgfpathcurveto{\pgfqpoint{3.247208in}{2.048606in}}{\pgfqpoint{3.243936in}{2.040706in}}{\pgfqpoint{3.243936in}{2.032470in}}%
\pgfpathcurveto{\pgfqpoint{3.243936in}{2.024234in}}{\pgfqpoint{3.247208in}{2.016334in}}{\pgfqpoint{3.253032in}{2.010510in}}%
\pgfpathcurveto{\pgfqpoint{3.258856in}{2.004686in}}{\pgfqpoint{3.266756in}{2.001413in}}{\pgfqpoint{3.274992in}{2.001413in}}%
\pgfpathclose%
\pgfusepath{stroke,fill}%
\end{pgfscope}%
\begin{pgfscope}%
\pgfpathrectangle{\pgfqpoint{0.100000in}{0.212622in}}{\pgfqpoint{3.696000in}{3.696000in}}%
\pgfusepath{clip}%
\pgfsetbuttcap%
\pgfsetroundjoin%
\definecolor{currentfill}{rgb}{0.121569,0.466667,0.705882}%
\pgfsetfillcolor{currentfill}%
\pgfsetfillopacity{0.575696}%
\pgfsetlinewidth{1.003750pt}%
\definecolor{currentstroke}{rgb}{0.121569,0.466667,0.705882}%
\pgfsetstrokecolor{currentstroke}%
\pgfsetstrokeopacity{0.575696}%
\pgfsetdash{}{0pt}%
\pgfpathmoveto{\pgfqpoint{3.273894in}{2.000768in}}%
\pgfpathcurveto{\pgfqpoint{3.282131in}{2.000768in}}{\pgfqpoint{3.290031in}{2.004041in}}{\pgfqpoint{3.295855in}{2.009865in}}%
\pgfpathcurveto{\pgfqpoint{3.301678in}{2.015688in}}{\pgfqpoint{3.304951in}{2.023589in}}{\pgfqpoint{3.304951in}{2.031825in}}%
\pgfpathcurveto{\pgfqpoint{3.304951in}{2.040061in}}{\pgfqpoint{3.301678in}{2.047961in}}{\pgfqpoint{3.295855in}{2.053785in}}%
\pgfpathcurveto{\pgfqpoint{3.290031in}{2.059609in}}{\pgfqpoint{3.282131in}{2.062881in}}{\pgfqpoint{3.273894in}{2.062881in}}%
\pgfpathcurveto{\pgfqpoint{3.265658in}{2.062881in}}{\pgfqpoint{3.257758in}{2.059609in}}{\pgfqpoint{3.251934in}{2.053785in}}%
\pgfpathcurveto{\pgfqpoint{3.246110in}{2.047961in}}{\pgfqpoint{3.242838in}{2.040061in}}{\pgfqpoint{3.242838in}{2.031825in}}%
\pgfpathcurveto{\pgfqpoint{3.242838in}{2.023589in}}{\pgfqpoint{3.246110in}{2.015688in}}{\pgfqpoint{3.251934in}{2.009865in}}%
\pgfpathcurveto{\pgfqpoint{3.257758in}{2.004041in}}{\pgfqpoint{3.265658in}{2.000768in}}{\pgfqpoint{3.273894in}{2.000768in}}%
\pgfpathclose%
\pgfusepath{stroke,fill}%
\end{pgfscope}%
\begin{pgfscope}%
\pgfpathrectangle{\pgfqpoint{0.100000in}{0.212622in}}{\pgfqpoint{3.696000in}{3.696000in}}%
\pgfusepath{clip}%
\pgfsetbuttcap%
\pgfsetroundjoin%
\definecolor{currentfill}{rgb}{0.121569,0.466667,0.705882}%
\pgfsetfillcolor{currentfill}%
\pgfsetfillopacity{0.575891}%
\pgfsetlinewidth{1.003750pt}%
\definecolor{currentstroke}{rgb}{0.121569,0.466667,0.705882}%
\pgfsetstrokecolor{currentstroke}%
\pgfsetstrokeopacity{0.575891}%
\pgfsetdash{}{0pt}%
\pgfpathmoveto{\pgfqpoint{3.273218in}{2.001084in}}%
\pgfpathcurveto{\pgfqpoint{3.281454in}{2.001084in}}{\pgfqpoint{3.289354in}{2.004357in}}{\pgfqpoint{3.295178in}{2.010181in}}%
\pgfpathcurveto{\pgfqpoint{3.301002in}{2.016004in}}{\pgfqpoint{3.304274in}{2.023904in}}{\pgfqpoint{3.304274in}{2.032141in}}%
\pgfpathcurveto{\pgfqpoint{3.304274in}{2.040377in}}{\pgfqpoint{3.301002in}{2.048277in}}{\pgfqpoint{3.295178in}{2.054101in}}%
\pgfpathcurveto{\pgfqpoint{3.289354in}{2.059925in}}{\pgfqpoint{3.281454in}{2.063197in}}{\pgfqpoint{3.273218in}{2.063197in}}%
\pgfpathcurveto{\pgfqpoint{3.264982in}{2.063197in}}{\pgfqpoint{3.257082in}{2.059925in}}{\pgfqpoint{3.251258in}{2.054101in}}%
\pgfpathcurveto{\pgfqpoint{3.245434in}{2.048277in}}{\pgfqpoint{3.242161in}{2.040377in}}{\pgfqpoint{3.242161in}{2.032141in}}%
\pgfpathcurveto{\pgfqpoint{3.242161in}{2.023904in}}{\pgfqpoint{3.245434in}{2.016004in}}{\pgfqpoint{3.251258in}{2.010181in}}%
\pgfpathcurveto{\pgfqpoint{3.257082in}{2.004357in}}{\pgfqpoint{3.264982in}{2.001084in}}{\pgfqpoint{3.273218in}{2.001084in}}%
\pgfpathclose%
\pgfusepath{stroke,fill}%
\end{pgfscope}%
\begin{pgfscope}%
\pgfpathrectangle{\pgfqpoint{0.100000in}{0.212622in}}{\pgfqpoint{3.696000in}{3.696000in}}%
\pgfusepath{clip}%
\pgfsetbuttcap%
\pgfsetroundjoin%
\definecolor{currentfill}{rgb}{0.121569,0.466667,0.705882}%
\pgfsetfillcolor{currentfill}%
\pgfsetfillopacity{0.576262}%
\pgfsetlinewidth{1.003750pt}%
\definecolor{currentstroke}{rgb}{0.121569,0.466667,0.705882}%
\pgfsetstrokecolor{currentstroke}%
\pgfsetstrokeopacity{0.576262}%
\pgfsetdash{}{0pt}%
\pgfpathmoveto{\pgfqpoint{0.992838in}{1.683580in}}%
\pgfpathcurveto{\pgfqpoint{1.001075in}{1.683580in}}{\pgfqpoint{1.008975in}{1.686852in}}{\pgfqpoint{1.014799in}{1.692676in}}%
\pgfpathcurveto{\pgfqpoint{1.020623in}{1.698500in}}{\pgfqpoint{1.023895in}{1.706400in}}{\pgfqpoint{1.023895in}{1.714636in}}%
\pgfpathcurveto{\pgfqpoint{1.023895in}{1.722872in}}{\pgfqpoint{1.020623in}{1.730772in}}{\pgfqpoint{1.014799in}{1.736596in}}%
\pgfpathcurveto{\pgfqpoint{1.008975in}{1.742420in}}{\pgfqpoint{1.001075in}{1.745693in}}{\pgfqpoint{0.992838in}{1.745693in}}%
\pgfpathcurveto{\pgfqpoint{0.984602in}{1.745693in}}{\pgfqpoint{0.976702in}{1.742420in}}{\pgfqpoint{0.970878in}{1.736596in}}%
\pgfpathcurveto{\pgfqpoint{0.965054in}{1.730772in}}{\pgfqpoint{0.961782in}{1.722872in}}{\pgfqpoint{0.961782in}{1.714636in}}%
\pgfpathcurveto{\pgfqpoint{0.961782in}{1.706400in}}{\pgfqpoint{0.965054in}{1.698500in}}{\pgfqpoint{0.970878in}{1.692676in}}%
\pgfpathcurveto{\pgfqpoint{0.976702in}{1.686852in}}{\pgfqpoint{0.984602in}{1.683580in}}{\pgfqpoint{0.992838in}{1.683580in}}%
\pgfpathclose%
\pgfusepath{stroke,fill}%
\end{pgfscope}%
\begin{pgfscope}%
\pgfpathrectangle{\pgfqpoint{0.100000in}{0.212622in}}{\pgfqpoint{3.696000in}{3.696000in}}%
\pgfusepath{clip}%
\pgfsetbuttcap%
\pgfsetroundjoin%
\definecolor{currentfill}{rgb}{0.121569,0.466667,0.705882}%
\pgfsetfillcolor{currentfill}%
\pgfsetfillopacity{0.576705}%
\pgfsetlinewidth{1.003750pt}%
\definecolor{currentstroke}{rgb}{0.121569,0.466667,0.705882}%
\pgfsetstrokecolor{currentstroke}%
\pgfsetstrokeopacity{0.576705}%
\pgfsetdash{}{0pt}%
\pgfpathmoveto{\pgfqpoint{3.271742in}{2.001264in}}%
\pgfpathcurveto{\pgfqpoint{3.279978in}{2.001264in}}{\pgfqpoint{3.287878in}{2.004536in}}{\pgfqpoint{3.293702in}{2.010360in}}%
\pgfpathcurveto{\pgfqpoint{3.299526in}{2.016184in}}{\pgfqpoint{3.302798in}{2.024084in}}{\pgfqpoint{3.302798in}{2.032321in}}%
\pgfpathcurveto{\pgfqpoint{3.302798in}{2.040557in}}{\pgfqpoint{3.299526in}{2.048457in}}{\pgfqpoint{3.293702in}{2.054281in}}%
\pgfpathcurveto{\pgfqpoint{3.287878in}{2.060105in}}{\pgfqpoint{3.279978in}{2.063377in}}{\pgfqpoint{3.271742in}{2.063377in}}%
\pgfpathcurveto{\pgfqpoint{3.263505in}{2.063377in}}{\pgfqpoint{3.255605in}{2.060105in}}{\pgfqpoint{3.249781in}{2.054281in}}%
\pgfpathcurveto{\pgfqpoint{3.243958in}{2.048457in}}{\pgfqpoint{3.240685in}{2.040557in}}{\pgfqpoint{3.240685in}{2.032321in}}%
\pgfpathcurveto{\pgfqpoint{3.240685in}{2.024084in}}{\pgfqpoint{3.243958in}{2.016184in}}{\pgfqpoint{3.249781in}{2.010360in}}%
\pgfpathcurveto{\pgfqpoint{3.255605in}{2.004536in}}{\pgfqpoint{3.263505in}{2.001264in}}{\pgfqpoint{3.271742in}{2.001264in}}%
\pgfpathclose%
\pgfusepath{stroke,fill}%
\end{pgfscope}%
\begin{pgfscope}%
\pgfpathrectangle{\pgfqpoint{0.100000in}{0.212622in}}{\pgfqpoint{3.696000in}{3.696000in}}%
\pgfusepath{clip}%
\pgfsetbuttcap%
\pgfsetroundjoin%
\definecolor{currentfill}{rgb}{0.121569,0.466667,0.705882}%
\pgfsetfillcolor{currentfill}%
\pgfsetfillopacity{0.577039}%
\pgfsetlinewidth{1.003750pt}%
\definecolor{currentstroke}{rgb}{0.121569,0.466667,0.705882}%
\pgfsetstrokecolor{currentstroke}%
\pgfsetstrokeopacity{0.577039}%
\pgfsetdash{}{0pt}%
\pgfpathmoveto{\pgfqpoint{0.990889in}{1.679429in}}%
\pgfpathcurveto{\pgfqpoint{0.999125in}{1.679429in}}{\pgfqpoint{1.007025in}{1.682702in}}{\pgfqpoint{1.012849in}{1.688526in}}%
\pgfpathcurveto{\pgfqpoint{1.018673in}{1.694349in}}{\pgfqpoint{1.021945in}{1.702250in}}{\pgfqpoint{1.021945in}{1.710486in}}%
\pgfpathcurveto{\pgfqpoint{1.021945in}{1.718722in}}{\pgfqpoint{1.018673in}{1.726622in}}{\pgfqpoint{1.012849in}{1.732446in}}%
\pgfpathcurveto{\pgfqpoint{1.007025in}{1.738270in}}{\pgfqpoint{0.999125in}{1.741542in}}{\pgfqpoint{0.990889in}{1.741542in}}%
\pgfpathcurveto{\pgfqpoint{0.982653in}{1.741542in}}{\pgfqpoint{0.974753in}{1.738270in}}{\pgfqpoint{0.968929in}{1.732446in}}%
\pgfpathcurveto{\pgfqpoint{0.963105in}{1.726622in}}{\pgfqpoint{0.959832in}{1.718722in}}{\pgfqpoint{0.959832in}{1.710486in}}%
\pgfpathcurveto{\pgfqpoint{0.959832in}{1.702250in}}{\pgfqpoint{0.963105in}{1.694349in}}{\pgfqpoint{0.968929in}{1.688526in}}%
\pgfpathcurveto{\pgfqpoint{0.974753in}{1.682702in}}{\pgfqpoint{0.982653in}{1.679429in}}{\pgfqpoint{0.990889in}{1.679429in}}%
\pgfpathclose%
\pgfusepath{stroke,fill}%
\end{pgfscope}%
\begin{pgfscope}%
\pgfpathrectangle{\pgfqpoint{0.100000in}{0.212622in}}{\pgfqpoint{3.696000in}{3.696000in}}%
\pgfusepath{clip}%
\pgfsetbuttcap%
\pgfsetroundjoin%
\definecolor{currentfill}{rgb}{0.121569,0.466667,0.705882}%
\pgfsetfillcolor{currentfill}%
\pgfsetfillopacity{0.577057}%
\pgfsetlinewidth{1.003750pt}%
\definecolor{currentstroke}{rgb}{0.121569,0.466667,0.705882}%
\pgfsetstrokecolor{currentstroke}%
\pgfsetstrokeopacity{0.577057}%
\pgfsetdash{}{0pt}%
\pgfpathmoveto{\pgfqpoint{3.270734in}{2.000910in}}%
\pgfpathcurveto{\pgfqpoint{3.278970in}{2.000910in}}{\pgfqpoint{3.286870in}{2.004182in}}{\pgfqpoint{3.292694in}{2.010006in}}%
\pgfpathcurveto{\pgfqpoint{3.298518in}{2.015830in}}{\pgfqpoint{3.301790in}{2.023730in}}{\pgfqpoint{3.301790in}{2.031966in}}%
\pgfpathcurveto{\pgfqpoint{3.301790in}{2.040202in}}{\pgfqpoint{3.298518in}{2.048102in}}{\pgfqpoint{3.292694in}{2.053926in}}%
\pgfpathcurveto{\pgfqpoint{3.286870in}{2.059750in}}{\pgfqpoint{3.278970in}{2.063023in}}{\pgfqpoint{3.270734in}{2.063023in}}%
\pgfpathcurveto{\pgfqpoint{3.262498in}{2.063023in}}{\pgfqpoint{3.254598in}{2.059750in}}{\pgfqpoint{3.248774in}{2.053926in}}%
\pgfpathcurveto{\pgfqpoint{3.242950in}{2.048102in}}{\pgfqpoint{3.239677in}{2.040202in}}{\pgfqpoint{3.239677in}{2.031966in}}%
\pgfpathcurveto{\pgfqpoint{3.239677in}{2.023730in}}{\pgfqpoint{3.242950in}{2.015830in}}{\pgfqpoint{3.248774in}{2.010006in}}%
\pgfpathcurveto{\pgfqpoint{3.254598in}{2.004182in}}{\pgfqpoint{3.262498in}{2.000910in}}{\pgfqpoint{3.270734in}{2.000910in}}%
\pgfpathclose%
\pgfusepath{stroke,fill}%
\end{pgfscope}%
\begin{pgfscope}%
\pgfpathrectangle{\pgfqpoint{0.100000in}{0.212622in}}{\pgfqpoint{3.696000in}{3.696000in}}%
\pgfusepath{clip}%
\pgfsetbuttcap%
\pgfsetroundjoin%
\definecolor{currentfill}{rgb}{0.121569,0.466667,0.705882}%
\pgfsetfillcolor{currentfill}%
\pgfsetfillopacity{0.577770}%
\pgfsetlinewidth{1.003750pt}%
\definecolor{currentstroke}{rgb}{0.121569,0.466667,0.705882}%
\pgfsetstrokecolor{currentstroke}%
\pgfsetstrokeopacity{0.577770}%
\pgfsetdash{}{0pt}%
\pgfpathmoveto{\pgfqpoint{3.268163in}{2.001346in}}%
\pgfpathcurveto{\pgfqpoint{3.276399in}{2.001346in}}{\pgfqpoint{3.284299in}{2.004618in}}{\pgfqpoint{3.290123in}{2.010442in}}%
\pgfpathcurveto{\pgfqpoint{3.295947in}{2.016266in}}{\pgfqpoint{3.299219in}{2.024166in}}{\pgfqpoint{3.299219in}{2.032402in}}%
\pgfpathcurveto{\pgfqpoint{3.299219in}{2.040639in}}{\pgfqpoint{3.295947in}{2.048539in}}{\pgfqpoint{3.290123in}{2.054363in}}%
\pgfpathcurveto{\pgfqpoint{3.284299in}{2.060187in}}{\pgfqpoint{3.276399in}{2.063459in}}{\pgfqpoint{3.268163in}{2.063459in}}%
\pgfpathcurveto{\pgfqpoint{3.259927in}{2.063459in}}{\pgfqpoint{3.252027in}{2.060187in}}{\pgfqpoint{3.246203in}{2.054363in}}%
\pgfpathcurveto{\pgfqpoint{3.240379in}{2.048539in}}{\pgfqpoint{3.237106in}{2.040639in}}{\pgfqpoint{3.237106in}{2.032402in}}%
\pgfpathcurveto{\pgfqpoint{3.237106in}{2.024166in}}{\pgfqpoint{3.240379in}{2.016266in}}{\pgfqpoint{3.246203in}{2.010442in}}%
\pgfpathcurveto{\pgfqpoint{3.252027in}{2.004618in}}{\pgfqpoint{3.259927in}{2.001346in}}{\pgfqpoint{3.268163in}{2.001346in}}%
\pgfpathclose%
\pgfusepath{stroke,fill}%
\end{pgfscope}%
\begin{pgfscope}%
\pgfpathrectangle{\pgfqpoint{0.100000in}{0.212622in}}{\pgfqpoint{3.696000in}{3.696000in}}%
\pgfusepath{clip}%
\pgfsetbuttcap%
\pgfsetroundjoin%
\definecolor{currentfill}{rgb}{0.121569,0.466667,0.705882}%
\pgfsetfillcolor{currentfill}%
\pgfsetfillopacity{0.578222}%
\pgfsetlinewidth{1.003750pt}%
\definecolor{currentstroke}{rgb}{0.121569,0.466667,0.705882}%
\pgfsetstrokecolor{currentstroke}%
\pgfsetstrokeopacity{0.578222}%
\pgfsetdash{}{0pt}%
\pgfpathmoveto{\pgfqpoint{0.985593in}{1.672040in}}%
\pgfpathcurveto{\pgfqpoint{0.993829in}{1.672040in}}{\pgfqpoint{1.001729in}{1.675313in}}{\pgfqpoint{1.007553in}{1.681136in}}%
\pgfpathcurveto{\pgfqpoint{1.013377in}{1.686960in}}{\pgfqpoint{1.016649in}{1.694860in}}{\pgfqpoint{1.016649in}{1.703097in}}%
\pgfpathcurveto{\pgfqpoint{1.016649in}{1.711333in}}{\pgfqpoint{1.013377in}{1.719233in}}{\pgfqpoint{1.007553in}{1.725057in}}%
\pgfpathcurveto{\pgfqpoint{1.001729in}{1.730881in}}{\pgfqpoint{0.993829in}{1.734153in}}{\pgfqpoint{0.985593in}{1.734153in}}%
\pgfpathcurveto{\pgfqpoint{0.977356in}{1.734153in}}{\pgfqpoint{0.969456in}{1.730881in}}{\pgfqpoint{0.963633in}{1.725057in}}%
\pgfpathcurveto{\pgfqpoint{0.957809in}{1.719233in}}{\pgfqpoint{0.954536in}{1.711333in}}{\pgfqpoint{0.954536in}{1.703097in}}%
\pgfpathcurveto{\pgfqpoint{0.954536in}{1.694860in}}{\pgfqpoint{0.957809in}{1.686960in}}{\pgfqpoint{0.963633in}{1.681136in}}%
\pgfpathcurveto{\pgfqpoint{0.969456in}{1.675313in}}{\pgfqpoint{0.977356in}{1.672040in}}{\pgfqpoint{0.985593in}{1.672040in}}%
\pgfpathclose%
\pgfusepath{stroke,fill}%
\end{pgfscope}%
\begin{pgfscope}%
\pgfpathrectangle{\pgfqpoint{0.100000in}{0.212622in}}{\pgfqpoint{3.696000in}{3.696000in}}%
\pgfusepath{clip}%
\pgfsetbuttcap%
\pgfsetroundjoin%
\definecolor{currentfill}{rgb}{0.121569,0.466667,0.705882}%
\pgfsetfillcolor{currentfill}%
\pgfsetfillopacity{0.578293}%
\pgfsetlinewidth{1.003750pt}%
\definecolor{currentstroke}{rgb}{0.121569,0.466667,0.705882}%
\pgfsetstrokecolor{currentstroke}%
\pgfsetstrokeopacity{0.578293}%
\pgfsetdash{}{0pt}%
\pgfpathmoveto{\pgfqpoint{3.267078in}{2.001936in}}%
\pgfpathcurveto{\pgfqpoint{3.275314in}{2.001936in}}{\pgfqpoint{3.283214in}{2.005208in}}{\pgfqpoint{3.289038in}{2.011032in}}%
\pgfpathcurveto{\pgfqpoint{3.294862in}{2.016856in}}{\pgfqpoint{3.298135in}{2.024756in}}{\pgfqpoint{3.298135in}{2.032992in}}%
\pgfpathcurveto{\pgfqpoint{3.298135in}{2.041229in}}{\pgfqpoint{3.294862in}{2.049129in}}{\pgfqpoint{3.289038in}{2.054953in}}%
\pgfpathcurveto{\pgfqpoint{3.283214in}{2.060777in}}{\pgfqpoint{3.275314in}{2.064049in}}{\pgfqpoint{3.267078in}{2.064049in}}%
\pgfpathcurveto{\pgfqpoint{3.258842in}{2.064049in}}{\pgfqpoint{3.250942in}{2.060777in}}{\pgfqpoint{3.245118in}{2.054953in}}%
\pgfpathcurveto{\pgfqpoint{3.239294in}{2.049129in}}{\pgfqpoint{3.236022in}{2.041229in}}{\pgfqpoint{3.236022in}{2.032992in}}%
\pgfpathcurveto{\pgfqpoint{3.236022in}{2.024756in}}{\pgfqpoint{3.239294in}{2.016856in}}{\pgfqpoint{3.245118in}{2.011032in}}%
\pgfpathcurveto{\pgfqpoint{3.250942in}{2.005208in}}{\pgfqpoint{3.258842in}{2.001936in}}{\pgfqpoint{3.267078in}{2.001936in}}%
\pgfpathclose%
\pgfusepath{stroke,fill}%
\end{pgfscope}%
\begin{pgfscope}%
\pgfpathrectangle{\pgfqpoint{0.100000in}{0.212622in}}{\pgfqpoint{3.696000in}{3.696000in}}%
\pgfusepath{clip}%
\pgfsetbuttcap%
\pgfsetroundjoin%
\definecolor{currentfill}{rgb}{0.121569,0.466667,0.705882}%
\pgfsetfillcolor{currentfill}%
\pgfsetfillopacity{0.578865}%
\pgfsetlinewidth{1.003750pt}%
\definecolor{currentstroke}{rgb}{0.121569,0.466667,0.705882}%
\pgfsetstrokecolor{currentstroke}%
\pgfsetstrokeopacity{0.578865}%
\pgfsetdash{}{0pt}%
\pgfpathmoveto{\pgfqpoint{3.266437in}{2.001421in}}%
\pgfpathcurveto{\pgfqpoint{3.274673in}{2.001421in}}{\pgfqpoint{3.282573in}{2.004693in}}{\pgfqpoint{3.288397in}{2.010517in}}%
\pgfpathcurveto{\pgfqpoint{3.294221in}{2.016341in}}{\pgfqpoint{3.297494in}{2.024241in}}{\pgfqpoint{3.297494in}{2.032477in}}%
\pgfpathcurveto{\pgfqpoint{3.297494in}{2.040713in}}{\pgfqpoint{3.294221in}{2.048613in}}{\pgfqpoint{3.288397in}{2.054437in}}%
\pgfpathcurveto{\pgfqpoint{3.282573in}{2.060261in}}{\pgfqpoint{3.274673in}{2.063534in}}{\pgfqpoint{3.266437in}{2.063534in}}%
\pgfpathcurveto{\pgfqpoint{3.258201in}{2.063534in}}{\pgfqpoint{3.250301in}{2.060261in}}{\pgfqpoint{3.244477in}{2.054437in}}%
\pgfpathcurveto{\pgfqpoint{3.238653in}{2.048613in}}{\pgfqpoint{3.235381in}{2.040713in}}{\pgfqpoint{3.235381in}{2.032477in}}%
\pgfpathcurveto{\pgfqpoint{3.235381in}{2.024241in}}{\pgfqpoint{3.238653in}{2.016341in}}{\pgfqpoint{3.244477in}{2.010517in}}%
\pgfpathcurveto{\pgfqpoint{3.250301in}{2.004693in}}{\pgfqpoint{3.258201in}{2.001421in}}{\pgfqpoint{3.266437in}{2.001421in}}%
\pgfpathclose%
\pgfusepath{stroke,fill}%
\end{pgfscope}%
\begin{pgfscope}%
\pgfpathrectangle{\pgfqpoint{0.100000in}{0.212622in}}{\pgfqpoint{3.696000in}{3.696000in}}%
\pgfusepath{clip}%
\pgfsetbuttcap%
\pgfsetroundjoin%
\definecolor{currentfill}{rgb}{0.121569,0.466667,0.705882}%
\pgfsetfillcolor{currentfill}%
\pgfsetfillopacity{0.579180}%
\pgfsetlinewidth{1.003750pt}%
\definecolor{currentstroke}{rgb}{0.121569,0.466667,0.705882}%
\pgfsetstrokecolor{currentstroke}%
\pgfsetstrokeopacity{0.579180}%
\pgfsetdash{}{0pt}%
\pgfpathmoveto{\pgfqpoint{3.265666in}{2.001616in}}%
\pgfpathcurveto{\pgfqpoint{3.273902in}{2.001616in}}{\pgfqpoint{3.281802in}{2.004888in}}{\pgfqpoint{3.287626in}{2.010712in}}%
\pgfpathcurveto{\pgfqpoint{3.293450in}{2.016536in}}{\pgfqpoint{3.296722in}{2.024436in}}{\pgfqpoint{3.296722in}{2.032672in}}%
\pgfpathcurveto{\pgfqpoint{3.296722in}{2.040908in}}{\pgfqpoint{3.293450in}{2.048808in}}{\pgfqpoint{3.287626in}{2.054632in}}%
\pgfpathcurveto{\pgfqpoint{3.281802in}{2.060456in}}{\pgfqpoint{3.273902in}{2.063729in}}{\pgfqpoint{3.265666in}{2.063729in}}%
\pgfpathcurveto{\pgfqpoint{3.257430in}{2.063729in}}{\pgfqpoint{3.249530in}{2.060456in}}{\pgfqpoint{3.243706in}{2.054632in}}%
\pgfpathcurveto{\pgfqpoint{3.237882in}{2.048808in}}{\pgfqpoint{3.234609in}{2.040908in}}{\pgfqpoint{3.234609in}{2.032672in}}%
\pgfpathcurveto{\pgfqpoint{3.234609in}{2.024436in}}{\pgfqpoint{3.237882in}{2.016536in}}{\pgfqpoint{3.243706in}{2.010712in}}%
\pgfpathcurveto{\pgfqpoint{3.249530in}{2.004888in}}{\pgfqpoint{3.257430in}{2.001616in}}{\pgfqpoint{3.265666in}{2.001616in}}%
\pgfpathclose%
\pgfusepath{stroke,fill}%
\end{pgfscope}%
\begin{pgfscope}%
\pgfpathrectangle{\pgfqpoint{0.100000in}{0.212622in}}{\pgfqpoint{3.696000in}{3.696000in}}%
\pgfusepath{clip}%
\pgfsetbuttcap%
\pgfsetroundjoin%
\definecolor{currentfill}{rgb}{0.121569,0.466667,0.705882}%
\pgfsetfillcolor{currentfill}%
\pgfsetfillopacity{0.579614}%
\pgfsetlinewidth{1.003750pt}%
\definecolor{currentstroke}{rgb}{0.121569,0.466667,0.705882}%
\pgfsetstrokecolor{currentstroke}%
\pgfsetstrokeopacity{0.579614}%
\pgfsetdash{}{0pt}%
\pgfpathmoveto{\pgfqpoint{3.265011in}{2.001805in}}%
\pgfpathcurveto{\pgfqpoint{3.273247in}{2.001805in}}{\pgfqpoint{3.281147in}{2.005078in}}{\pgfqpoint{3.286971in}{2.010901in}}%
\pgfpathcurveto{\pgfqpoint{3.292795in}{2.016725in}}{\pgfqpoint{3.296067in}{2.024625in}}{\pgfqpoint{3.296067in}{2.032862in}}%
\pgfpathcurveto{\pgfqpoint{3.296067in}{2.041098in}}{\pgfqpoint{3.292795in}{2.048998in}}{\pgfqpoint{3.286971in}{2.054822in}}%
\pgfpathcurveto{\pgfqpoint{3.281147in}{2.060646in}}{\pgfqpoint{3.273247in}{2.063918in}}{\pgfqpoint{3.265011in}{2.063918in}}%
\pgfpathcurveto{\pgfqpoint{3.256775in}{2.063918in}}{\pgfqpoint{3.248875in}{2.060646in}}{\pgfqpoint{3.243051in}{2.054822in}}%
\pgfpathcurveto{\pgfqpoint{3.237227in}{2.048998in}}{\pgfqpoint{3.233954in}{2.041098in}}{\pgfqpoint{3.233954in}{2.032862in}}%
\pgfpathcurveto{\pgfqpoint{3.233954in}{2.024625in}}{\pgfqpoint{3.237227in}{2.016725in}}{\pgfqpoint{3.243051in}{2.010901in}}%
\pgfpathcurveto{\pgfqpoint{3.248875in}{2.005078in}}{\pgfqpoint{3.256775in}{2.001805in}}{\pgfqpoint{3.265011in}{2.001805in}}%
\pgfpathclose%
\pgfusepath{stroke,fill}%
\end{pgfscope}%
\begin{pgfscope}%
\pgfpathrectangle{\pgfqpoint{0.100000in}{0.212622in}}{\pgfqpoint{3.696000in}{3.696000in}}%
\pgfusepath{clip}%
\pgfsetbuttcap%
\pgfsetroundjoin%
\definecolor{currentfill}{rgb}{0.121569,0.466667,0.705882}%
\pgfsetfillcolor{currentfill}%
\pgfsetfillopacity{0.580092}%
\pgfsetlinewidth{1.003750pt}%
\definecolor{currentstroke}{rgb}{0.121569,0.466667,0.705882}%
\pgfsetstrokecolor{currentstroke}%
\pgfsetstrokeopacity{0.580092}%
\pgfsetdash{}{0pt}%
\pgfpathmoveto{\pgfqpoint{3.264323in}{2.001490in}}%
\pgfpathcurveto{\pgfqpoint{3.272559in}{2.001490in}}{\pgfqpoint{3.280459in}{2.004763in}}{\pgfqpoint{3.286283in}{2.010587in}}%
\pgfpathcurveto{\pgfqpoint{3.292107in}{2.016411in}}{\pgfqpoint{3.295379in}{2.024311in}}{\pgfqpoint{3.295379in}{2.032547in}}%
\pgfpathcurveto{\pgfqpoint{3.295379in}{2.040783in}}{\pgfqpoint{3.292107in}{2.048683in}}{\pgfqpoint{3.286283in}{2.054507in}}%
\pgfpathcurveto{\pgfqpoint{3.280459in}{2.060331in}}{\pgfqpoint{3.272559in}{2.063603in}}{\pgfqpoint{3.264323in}{2.063603in}}%
\pgfpathcurveto{\pgfqpoint{3.256086in}{2.063603in}}{\pgfqpoint{3.248186in}{2.060331in}}{\pgfqpoint{3.242362in}{2.054507in}}%
\pgfpathcurveto{\pgfqpoint{3.236538in}{2.048683in}}{\pgfqpoint{3.233266in}{2.040783in}}{\pgfqpoint{3.233266in}{2.032547in}}%
\pgfpathcurveto{\pgfqpoint{3.233266in}{2.024311in}}{\pgfqpoint{3.236538in}{2.016411in}}{\pgfqpoint{3.242362in}{2.010587in}}%
\pgfpathcurveto{\pgfqpoint{3.248186in}{2.004763in}}{\pgfqpoint{3.256086in}{2.001490in}}{\pgfqpoint{3.264323in}{2.001490in}}%
\pgfpathclose%
\pgfusepath{stroke,fill}%
\end{pgfscope}%
\begin{pgfscope}%
\pgfpathrectangle{\pgfqpoint{0.100000in}{0.212622in}}{\pgfqpoint{3.696000in}{3.696000in}}%
\pgfusepath{clip}%
\pgfsetbuttcap%
\pgfsetroundjoin%
\definecolor{currentfill}{rgb}{0.121569,0.466667,0.705882}%
\pgfsetfillcolor{currentfill}%
\pgfsetfillopacity{0.580253}%
\pgfsetlinewidth{1.003750pt}%
\definecolor{currentstroke}{rgb}{0.121569,0.466667,0.705882}%
\pgfsetstrokecolor{currentstroke}%
\pgfsetstrokeopacity{0.580253}%
\pgfsetdash{}{0pt}%
\pgfpathmoveto{\pgfqpoint{0.982482in}{1.670242in}}%
\pgfpathcurveto{\pgfqpoint{0.990718in}{1.670242in}}{\pgfqpoint{0.998618in}{1.673514in}}{\pgfqpoint{1.004442in}{1.679338in}}%
\pgfpathcurveto{\pgfqpoint{1.010266in}{1.685162in}}{\pgfqpoint{1.013538in}{1.693062in}}{\pgfqpoint{1.013538in}{1.701298in}}%
\pgfpathcurveto{\pgfqpoint{1.013538in}{1.709534in}}{\pgfqpoint{1.010266in}{1.717435in}}{\pgfqpoint{1.004442in}{1.723258in}}%
\pgfpathcurveto{\pgfqpoint{0.998618in}{1.729082in}}{\pgfqpoint{0.990718in}{1.732355in}}{\pgfqpoint{0.982482in}{1.732355in}}%
\pgfpathcurveto{\pgfqpoint{0.974245in}{1.732355in}}{\pgfqpoint{0.966345in}{1.729082in}}{\pgfqpoint{0.960521in}{1.723258in}}%
\pgfpathcurveto{\pgfqpoint{0.954698in}{1.717435in}}{\pgfqpoint{0.951425in}{1.709534in}}{\pgfqpoint{0.951425in}{1.701298in}}%
\pgfpathcurveto{\pgfqpoint{0.951425in}{1.693062in}}{\pgfqpoint{0.954698in}{1.685162in}}{\pgfqpoint{0.960521in}{1.679338in}}%
\pgfpathcurveto{\pgfqpoint{0.966345in}{1.673514in}}{\pgfqpoint{0.974245in}{1.670242in}}{\pgfqpoint{0.982482in}{1.670242in}}%
\pgfpathclose%
\pgfusepath{stroke,fill}%
\end{pgfscope}%
\begin{pgfscope}%
\pgfpathrectangle{\pgfqpoint{0.100000in}{0.212622in}}{\pgfqpoint{3.696000in}{3.696000in}}%
\pgfusepath{clip}%
\pgfsetbuttcap%
\pgfsetroundjoin%
\definecolor{currentfill}{rgb}{0.121569,0.466667,0.705882}%
\pgfsetfillcolor{currentfill}%
\pgfsetfillopacity{0.580824}%
\pgfsetlinewidth{1.003750pt}%
\definecolor{currentstroke}{rgb}{0.121569,0.466667,0.705882}%
\pgfsetstrokecolor{currentstroke}%
\pgfsetstrokeopacity{0.580824}%
\pgfsetdash{}{0pt}%
\pgfpathmoveto{\pgfqpoint{3.263060in}{2.001223in}}%
\pgfpathcurveto{\pgfqpoint{3.271296in}{2.001223in}}{\pgfqpoint{3.279196in}{2.004495in}}{\pgfqpoint{3.285020in}{2.010319in}}%
\pgfpathcurveto{\pgfqpoint{3.290844in}{2.016143in}}{\pgfqpoint{3.294116in}{2.024043in}}{\pgfqpoint{3.294116in}{2.032279in}}%
\pgfpathcurveto{\pgfqpoint{3.294116in}{2.040516in}}{\pgfqpoint{3.290844in}{2.048416in}}{\pgfqpoint{3.285020in}{2.054240in}}%
\pgfpathcurveto{\pgfqpoint{3.279196in}{2.060064in}}{\pgfqpoint{3.271296in}{2.063336in}}{\pgfqpoint{3.263060in}{2.063336in}}%
\pgfpathcurveto{\pgfqpoint{3.254824in}{2.063336in}}{\pgfqpoint{3.246924in}{2.060064in}}{\pgfqpoint{3.241100in}{2.054240in}}%
\pgfpathcurveto{\pgfqpoint{3.235276in}{2.048416in}}{\pgfqpoint{3.232003in}{2.040516in}}{\pgfqpoint{3.232003in}{2.032279in}}%
\pgfpathcurveto{\pgfqpoint{3.232003in}{2.024043in}}{\pgfqpoint{3.235276in}{2.016143in}}{\pgfqpoint{3.241100in}{2.010319in}}%
\pgfpathcurveto{\pgfqpoint{3.246924in}{2.004495in}}{\pgfqpoint{3.254824in}{2.001223in}}{\pgfqpoint{3.263060in}{2.001223in}}%
\pgfpathclose%
\pgfusepath{stroke,fill}%
\end{pgfscope}%
\begin{pgfscope}%
\pgfpathrectangle{\pgfqpoint{0.100000in}{0.212622in}}{\pgfqpoint{3.696000in}{3.696000in}}%
\pgfusepath{clip}%
\pgfsetbuttcap%
\pgfsetroundjoin%
\definecolor{currentfill}{rgb}{0.121569,0.466667,0.705882}%
\pgfsetfillcolor{currentfill}%
\pgfsetfillopacity{0.581238}%
\pgfsetlinewidth{1.003750pt}%
\definecolor{currentstroke}{rgb}{0.121569,0.466667,0.705882}%
\pgfsetstrokecolor{currentstroke}%
\pgfsetstrokeopacity{0.581238}%
\pgfsetdash{}{0pt}%
\pgfpathmoveto{\pgfqpoint{3.262004in}{2.001677in}}%
\pgfpathcurveto{\pgfqpoint{3.270241in}{2.001677in}}{\pgfqpoint{3.278141in}{2.004949in}}{\pgfqpoint{3.283965in}{2.010773in}}%
\pgfpathcurveto{\pgfqpoint{3.289789in}{2.016597in}}{\pgfqpoint{3.293061in}{2.024497in}}{\pgfqpoint{3.293061in}{2.032733in}}%
\pgfpathcurveto{\pgfqpoint{3.293061in}{2.040969in}}{\pgfqpoint{3.289789in}{2.048869in}}{\pgfqpoint{3.283965in}{2.054693in}}%
\pgfpathcurveto{\pgfqpoint{3.278141in}{2.060517in}}{\pgfqpoint{3.270241in}{2.063790in}}{\pgfqpoint{3.262004in}{2.063790in}}%
\pgfpathcurveto{\pgfqpoint{3.253768in}{2.063790in}}{\pgfqpoint{3.245868in}{2.060517in}}{\pgfqpoint{3.240044in}{2.054693in}}%
\pgfpathcurveto{\pgfqpoint{3.234220in}{2.048869in}}{\pgfqpoint{3.230948in}{2.040969in}}{\pgfqpoint{3.230948in}{2.032733in}}%
\pgfpathcurveto{\pgfqpoint{3.230948in}{2.024497in}}{\pgfqpoint{3.234220in}{2.016597in}}{\pgfqpoint{3.240044in}{2.010773in}}%
\pgfpathcurveto{\pgfqpoint{3.245868in}{2.004949in}}{\pgfqpoint{3.253768in}{2.001677in}}{\pgfqpoint{3.262004in}{2.001677in}}%
\pgfpathclose%
\pgfusepath{stroke,fill}%
\end{pgfscope}%
\begin{pgfscope}%
\pgfpathrectangle{\pgfqpoint{0.100000in}{0.212622in}}{\pgfqpoint{3.696000in}{3.696000in}}%
\pgfusepath{clip}%
\pgfsetbuttcap%
\pgfsetroundjoin%
\definecolor{currentfill}{rgb}{0.121569,0.466667,0.705882}%
\pgfsetfillcolor{currentfill}%
\pgfsetfillopacity{0.581509}%
\pgfsetlinewidth{1.003750pt}%
\definecolor{currentstroke}{rgb}{0.121569,0.466667,0.705882}%
\pgfsetstrokecolor{currentstroke}%
\pgfsetstrokeopacity{0.581509}%
\pgfsetdash{}{0pt}%
\pgfpathmoveto{\pgfqpoint{3.261775in}{2.001823in}}%
\pgfpathcurveto{\pgfqpoint{3.270011in}{2.001823in}}{\pgfqpoint{3.277911in}{2.005096in}}{\pgfqpoint{3.283735in}{2.010920in}}%
\pgfpathcurveto{\pgfqpoint{3.289559in}{2.016744in}}{\pgfqpoint{3.292831in}{2.024644in}}{\pgfqpoint{3.292831in}{2.032880in}}%
\pgfpathcurveto{\pgfqpoint{3.292831in}{2.041116in}}{\pgfqpoint{3.289559in}{2.049016in}}{\pgfqpoint{3.283735in}{2.054840in}}%
\pgfpathcurveto{\pgfqpoint{3.277911in}{2.060664in}}{\pgfqpoint{3.270011in}{2.063936in}}{\pgfqpoint{3.261775in}{2.063936in}}%
\pgfpathcurveto{\pgfqpoint{3.253539in}{2.063936in}}{\pgfqpoint{3.245638in}{2.060664in}}{\pgfqpoint{3.239815in}{2.054840in}}%
\pgfpathcurveto{\pgfqpoint{3.233991in}{2.049016in}}{\pgfqpoint{3.230718in}{2.041116in}}{\pgfqpoint{3.230718in}{2.032880in}}%
\pgfpathcurveto{\pgfqpoint{3.230718in}{2.024644in}}{\pgfqpoint{3.233991in}{2.016744in}}{\pgfqpoint{3.239815in}{2.010920in}}%
\pgfpathcurveto{\pgfqpoint{3.245638in}{2.005096in}}{\pgfqpoint{3.253539in}{2.001823in}}{\pgfqpoint{3.261775in}{2.001823in}}%
\pgfpathclose%
\pgfusepath{stroke,fill}%
\end{pgfscope}%
\begin{pgfscope}%
\pgfpathrectangle{\pgfqpoint{0.100000in}{0.212622in}}{\pgfqpoint{3.696000in}{3.696000in}}%
\pgfusepath{clip}%
\pgfsetbuttcap%
\pgfsetroundjoin%
\definecolor{currentfill}{rgb}{0.121569,0.466667,0.705882}%
\pgfsetfillcolor{currentfill}%
\pgfsetfillopacity{0.581617}%
\pgfsetlinewidth{1.003750pt}%
\definecolor{currentstroke}{rgb}{0.121569,0.466667,0.705882}%
\pgfsetstrokecolor{currentstroke}%
\pgfsetstrokeopacity{0.581617}%
\pgfsetdash{}{0pt}%
\pgfpathmoveto{\pgfqpoint{3.261494in}{2.001768in}}%
\pgfpathcurveto{\pgfqpoint{3.269730in}{2.001768in}}{\pgfqpoint{3.277630in}{2.005041in}}{\pgfqpoint{3.283454in}{2.010865in}}%
\pgfpathcurveto{\pgfqpoint{3.289278in}{2.016689in}}{\pgfqpoint{3.292551in}{2.024589in}}{\pgfqpoint{3.292551in}{2.032825in}}%
\pgfpathcurveto{\pgfqpoint{3.292551in}{2.041061in}}{\pgfqpoint{3.289278in}{2.048961in}}{\pgfqpoint{3.283454in}{2.054785in}}%
\pgfpathcurveto{\pgfqpoint{3.277630in}{2.060609in}}{\pgfqpoint{3.269730in}{2.063881in}}{\pgfqpoint{3.261494in}{2.063881in}}%
\pgfpathcurveto{\pgfqpoint{3.253258in}{2.063881in}}{\pgfqpoint{3.245358in}{2.060609in}}{\pgfqpoint{3.239534in}{2.054785in}}%
\pgfpathcurveto{\pgfqpoint{3.233710in}{2.048961in}}{\pgfqpoint{3.230438in}{2.041061in}}{\pgfqpoint{3.230438in}{2.032825in}}%
\pgfpathcurveto{\pgfqpoint{3.230438in}{2.024589in}}{\pgfqpoint{3.233710in}{2.016689in}}{\pgfqpoint{3.239534in}{2.010865in}}%
\pgfpathcurveto{\pgfqpoint{3.245358in}{2.005041in}}{\pgfqpoint{3.253258in}{2.001768in}}{\pgfqpoint{3.261494in}{2.001768in}}%
\pgfpathclose%
\pgfusepath{stroke,fill}%
\end{pgfscope}%
\begin{pgfscope}%
\pgfpathrectangle{\pgfqpoint{0.100000in}{0.212622in}}{\pgfqpoint{3.696000in}{3.696000in}}%
\pgfusepath{clip}%
\pgfsetbuttcap%
\pgfsetroundjoin%
\definecolor{currentfill}{rgb}{0.121569,0.466667,0.705882}%
\pgfsetfillcolor{currentfill}%
\pgfsetfillopacity{0.581907}%
\pgfsetlinewidth{1.003750pt}%
\definecolor{currentstroke}{rgb}{0.121569,0.466667,0.705882}%
\pgfsetstrokecolor{currentstroke}%
\pgfsetstrokeopacity{0.581907}%
\pgfsetdash{}{0pt}%
\pgfpathmoveto{\pgfqpoint{3.260740in}{2.002135in}}%
\pgfpathcurveto{\pgfqpoint{3.268977in}{2.002135in}}{\pgfqpoint{3.276877in}{2.005407in}}{\pgfqpoint{3.282701in}{2.011231in}}%
\pgfpathcurveto{\pgfqpoint{3.288525in}{2.017055in}}{\pgfqpoint{3.291797in}{2.024955in}}{\pgfqpoint{3.291797in}{2.033191in}}%
\pgfpathcurveto{\pgfqpoint{3.291797in}{2.041427in}}{\pgfqpoint{3.288525in}{2.049327in}}{\pgfqpoint{3.282701in}{2.055151in}}%
\pgfpathcurveto{\pgfqpoint{3.276877in}{2.060975in}}{\pgfqpoint{3.268977in}{2.064248in}}{\pgfqpoint{3.260740in}{2.064248in}}%
\pgfpathcurveto{\pgfqpoint{3.252504in}{2.064248in}}{\pgfqpoint{3.244604in}{2.060975in}}{\pgfqpoint{3.238780in}{2.055151in}}%
\pgfpathcurveto{\pgfqpoint{3.232956in}{2.049327in}}{\pgfqpoint{3.229684in}{2.041427in}}{\pgfqpoint{3.229684in}{2.033191in}}%
\pgfpathcurveto{\pgfqpoint{3.229684in}{2.024955in}}{\pgfqpoint{3.232956in}{2.017055in}}{\pgfqpoint{3.238780in}{2.011231in}}%
\pgfpathcurveto{\pgfqpoint{3.244604in}{2.005407in}}{\pgfqpoint{3.252504in}{2.002135in}}{\pgfqpoint{3.260740in}{2.002135in}}%
\pgfpathclose%
\pgfusepath{stroke,fill}%
\end{pgfscope}%
\begin{pgfscope}%
\pgfpathrectangle{\pgfqpoint{0.100000in}{0.212622in}}{\pgfqpoint{3.696000in}{3.696000in}}%
\pgfusepath{clip}%
\pgfsetbuttcap%
\pgfsetroundjoin%
\definecolor{currentfill}{rgb}{0.121569,0.466667,0.705882}%
\pgfsetfillcolor{currentfill}%
\pgfsetfillopacity{0.582246}%
\pgfsetlinewidth{1.003750pt}%
\definecolor{currentstroke}{rgb}{0.121569,0.466667,0.705882}%
\pgfsetstrokecolor{currentstroke}%
\pgfsetstrokeopacity{0.582246}%
\pgfsetdash{}{0pt}%
\pgfpathmoveto{\pgfqpoint{0.971573in}{1.661591in}}%
\pgfpathcurveto{\pgfqpoint{0.979810in}{1.661591in}}{\pgfqpoint{0.987710in}{1.664863in}}{\pgfqpoint{0.993534in}{1.670687in}}%
\pgfpathcurveto{\pgfqpoint{0.999357in}{1.676511in}}{\pgfqpoint{1.002630in}{1.684411in}}{\pgfqpoint{1.002630in}{1.692647in}}%
\pgfpathcurveto{\pgfqpoint{1.002630in}{1.700884in}}{\pgfqpoint{0.999357in}{1.708784in}}{\pgfqpoint{0.993534in}{1.714608in}}%
\pgfpathcurveto{\pgfqpoint{0.987710in}{1.720431in}}{\pgfqpoint{0.979810in}{1.723704in}}{\pgfqpoint{0.971573in}{1.723704in}}%
\pgfpathcurveto{\pgfqpoint{0.963337in}{1.723704in}}{\pgfqpoint{0.955437in}{1.720431in}}{\pgfqpoint{0.949613in}{1.714608in}}%
\pgfpathcurveto{\pgfqpoint{0.943789in}{1.708784in}}{\pgfqpoint{0.940517in}{1.700884in}}{\pgfqpoint{0.940517in}{1.692647in}}%
\pgfpathcurveto{\pgfqpoint{0.940517in}{1.684411in}}{\pgfqpoint{0.943789in}{1.676511in}}{\pgfqpoint{0.949613in}{1.670687in}}%
\pgfpathcurveto{\pgfqpoint{0.955437in}{1.664863in}}{\pgfqpoint{0.963337in}{1.661591in}}{\pgfqpoint{0.971573in}{1.661591in}}%
\pgfpathclose%
\pgfusepath{stroke,fill}%
\end{pgfscope}%
\begin{pgfscope}%
\pgfpathrectangle{\pgfqpoint{0.100000in}{0.212622in}}{\pgfqpoint{3.696000in}{3.696000in}}%
\pgfusepath{clip}%
\pgfsetbuttcap%
\pgfsetroundjoin%
\definecolor{currentfill}{rgb}{0.121569,0.466667,0.705882}%
\pgfsetfillcolor{currentfill}%
\pgfsetfillopacity{0.582740}%
\pgfsetlinewidth{1.003750pt}%
\definecolor{currentstroke}{rgb}{0.121569,0.466667,0.705882}%
\pgfsetstrokecolor{currentstroke}%
\pgfsetstrokeopacity{0.582740}%
\pgfsetdash{}{0pt}%
\pgfpathmoveto{\pgfqpoint{3.259757in}{2.001266in}}%
\pgfpathcurveto{\pgfqpoint{3.267993in}{2.001266in}}{\pgfqpoint{3.275893in}{2.004539in}}{\pgfqpoint{3.281717in}{2.010363in}}%
\pgfpathcurveto{\pgfqpoint{3.287541in}{2.016187in}}{\pgfqpoint{3.290813in}{2.024087in}}{\pgfqpoint{3.290813in}{2.032323in}}%
\pgfpathcurveto{\pgfqpoint{3.290813in}{2.040559in}}{\pgfqpoint{3.287541in}{2.048459in}}{\pgfqpoint{3.281717in}{2.054283in}}%
\pgfpathcurveto{\pgfqpoint{3.275893in}{2.060107in}}{\pgfqpoint{3.267993in}{2.063379in}}{\pgfqpoint{3.259757in}{2.063379in}}%
\pgfpathcurveto{\pgfqpoint{3.251521in}{2.063379in}}{\pgfqpoint{3.243621in}{2.060107in}}{\pgfqpoint{3.237797in}{2.054283in}}%
\pgfpathcurveto{\pgfqpoint{3.231973in}{2.048459in}}{\pgfqpoint{3.228700in}{2.040559in}}{\pgfqpoint{3.228700in}{2.032323in}}%
\pgfpathcurveto{\pgfqpoint{3.228700in}{2.024087in}}{\pgfqpoint{3.231973in}{2.016187in}}{\pgfqpoint{3.237797in}{2.010363in}}%
\pgfpathcurveto{\pgfqpoint{3.243621in}{2.004539in}}{\pgfqpoint{3.251521in}{2.001266in}}{\pgfqpoint{3.259757in}{2.001266in}}%
\pgfpathclose%
\pgfusepath{stroke,fill}%
\end{pgfscope}%
\begin{pgfscope}%
\pgfpathrectangle{\pgfqpoint{0.100000in}{0.212622in}}{\pgfqpoint{3.696000in}{3.696000in}}%
\pgfusepath{clip}%
\pgfsetbuttcap%
\pgfsetroundjoin%
\definecolor{currentfill}{rgb}{0.121569,0.466667,0.705882}%
\pgfsetfillcolor{currentfill}%
\pgfsetfillopacity{0.583569}%
\pgfsetlinewidth{1.003750pt}%
\definecolor{currentstroke}{rgb}{0.121569,0.466667,0.705882}%
\pgfsetstrokecolor{currentstroke}%
\pgfsetstrokeopacity{0.583569}%
\pgfsetdash{}{0pt}%
\pgfpathmoveto{\pgfqpoint{3.258117in}{2.000228in}}%
\pgfpathcurveto{\pgfqpoint{3.266353in}{2.000228in}}{\pgfqpoint{3.274253in}{2.003501in}}{\pgfqpoint{3.280077in}{2.009325in}}%
\pgfpathcurveto{\pgfqpoint{3.285901in}{2.015149in}}{\pgfqpoint{3.289173in}{2.023049in}}{\pgfqpoint{3.289173in}{2.031285in}}%
\pgfpathcurveto{\pgfqpoint{3.289173in}{2.039521in}}{\pgfqpoint{3.285901in}{2.047421in}}{\pgfqpoint{3.280077in}{2.053245in}}%
\pgfpathcurveto{\pgfqpoint{3.274253in}{2.059069in}}{\pgfqpoint{3.266353in}{2.062341in}}{\pgfqpoint{3.258117in}{2.062341in}}%
\pgfpathcurveto{\pgfqpoint{3.249880in}{2.062341in}}{\pgfqpoint{3.241980in}{2.059069in}}{\pgfqpoint{3.236156in}{2.053245in}}%
\pgfpathcurveto{\pgfqpoint{3.230333in}{2.047421in}}{\pgfqpoint{3.227060in}{2.039521in}}{\pgfqpoint{3.227060in}{2.031285in}}%
\pgfpathcurveto{\pgfqpoint{3.227060in}{2.023049in}}{\pgfqpoint{3.230333in}{2.015149in}}{\pgfqpoint{3.236156in}{2.009325in}}%
\pgfpathcurveto{\pgfqpoint{3.241980in}{2.003501in}}{\pgfqpoint{3.249880in}{2.000228in}}{\pgfqpoint{3.258117in}{2.000228in}}%
\pgfpathclose%
\pgfusepath{stroke,fill}%
\end{pgfscope}%
\begin{pgfscope}%
\pgfpathrectangle{\pgfqpoint{0.100000in}{0.212622in}}{\pgfqpoint{3.696000in}{3.696000in}}%
\pgfusepath{clip}%
\pgfsetbuttcap%
\pgfsetroundjoin%
\definecolor{currentfill}{rgb}{0.121569,0.466667,0.705882}%
\pgfsetfillcolor{currentfill}%
\pgfsetfillopacity{0.584779}%
\pgfsetlinewidth{1.003750pt}%
\definecolor{currentstroke}{rgb}{0.121569,0.466667,0.705882}%
\pgfsetstrokecolor{currentstroke}%
\pgfsetstrokeopacity{0.584779}%
\pgfsetdash{}{0pt}%
\pgfpathmoveto{\pgfqpoint{3.254986in}{2.000595in}}%
\pgfpathcurveto{\pgfqpoint{3.263222in}{2.000595in}}{\pgfqpoint{3.271122in}{2.003867in}}{\pgfqpoint{3.276946in}{2.009691in}}%
\pgfpathcurveto{\pgfqpoint{3.282770in}{2.015515in}}{\pgfqpoint{3.286042in}{2.023415in}}{\pgfqpoint{3.286042in}{2.031651in}}%
\pgfpathcurveto{\pgfqpoint{3.286042in}{2.039888in}}{\pgfqpoint{3.282770in}{2.047788in}}{\pgfqpoint{3.276946in}{2.053612in}}%
\pgfpathcurveto{\pgfqpoint{3.271122in}{2.059436in}}{\pgfqpoint{3.263222in}{2.062708in}}{\pgfqpoint{3.254986in}{2.062708in}}%
\pgfpathcurveto{\pgfqpoint{3.246749in}{2.062708in}}{\pgfqpoint{3.238849in}{2.059436in}}{\pgfqpoint{3.233025in}{2.053612in}}%
\pgfpathcurveto{\pgfqpoint{3.227201in}{2.047788in}}{\pgfqpoint{3.223929in}{2.039888in}}{\pgfqpoint{3.223929in}{2.031651in}}%
\pgfpathcurveto{\pgfqpoint{3.223929in}{2.023415in}}{\pgfqpoint{3.227201in}{2.015515in}}{\pgfqpoint{3.233025in}{2.009691in}}%
\pgfpathcurveto{\pgfqpoint{3.238849in}{2.003867in}}{\pgfqpoint{3.246749in}{2.000595in}}{\pgfqpoint{3.254986in}{2.000595in}}%
\pgfpathclose%
\pgfusepath{stroke,fill}%
\end{pgfscope}%
\begin{pgfscope}%
\pgfpathrectangle{\pgfqpoint{0.100000in}{0.212622in}}{\pgfqpoint{3.696000in}{3.696000in}}%
\pgfusepath{clip}%
\pgfsetbuttcap%
\pgfsetroundjoin%
\definecolor{currentfill}{rgb}{0.121569,0.466667,0.705882}%
\pgfsetfillcolor{currentfill}%
\pgfsetfillopacity{0.585314}%
\pgfsetlinewidth{1.003750pt}%
\definecolor{currentstroke}{rgb}{0.121569,0.466667,0.705882}%
\pgfsetstrokecolor{currentstroke}%
\pgfsetstrokeopacity{0.585314}%
\pgfsetdash{}{0pt}%
\pgfpathmoveto{\pgfqpoint{0.964695in}{1.659293in}}%
\pgfpathcurveto{\pgfqpoint{0.972931in}{1.659293in}}{\pgfqpoint{0.980831in}{1.662565in}}{\pgfqpoint{0.986655in}{1.668389in}}%
\pgfpathcurveto{\pgfqpoint{0.992479in}{1.674213in}}{\pgfqpoint{0.995751in}{1.682113in}}{\pgfqpoint{0.995751in}{1.690350in}}%
\pgfpathcurveto{\pgfqpoint{0.995751in}{1.698586in}}{\pgfqpoint{0.992479in}{1.706486in}}{\pgfqpoint{0.986655in}{1.712310in}}%
\pgfpathcurveto{\pgfqpoint{0.980831in}{1.718134in}}{\pgfqpoint{0.972931in}{1.721406in}}{\pgfqpoint{0.964695in}{1.721406in}}%
\pgfpathcurveto{\pgfqpoint{0.956458in}{1.721406in}}{\pgfqpoint{0.948558in}{1.718134in}}{\pgfqpoint{0.942734in}{1.712310in}}%
\pgfpathcurveto{\pgfqpoint{0.936911in}{1.706486in}}{\pgfqpoint{0.933638in}{1.698586in}}{\pgfqpoint{0.933638in}{1.690350in}}%
\pgfpathcurveto{\pgfqpoint{0.933638in}{1.682113in}}{\pgfqpoint{0.936911in}{1.674213in}}{\pgfqpoint{0.942734in}{1.668389in}}%
\pgfpathcurveto{\pgfqpoint{0.948558in}{1.662565in}}{\pgfqpoint{0.956458in}{1.659293in}}{\pgfqpoint{0.964695in}{1.659293in}}%
\pgfpathclose%
\pgfusepath{stroke,fill}%
\end{pgfscope}%
\begin{pgfscope}%
\pgfpathrectangle{\pgfqpoint{0.100000in}{0.212622in}}{\pgfqpoint{3.696000in}{3.696000in}}%
\pgfusepath{clip}%
\pgfsetbuttcap%
\pgfsetroundjoin%
\definecolor{currentfill}{rgb}{0.121569,0.466667,0.705882}%
\pgfsetfillcolor{currentfill}%
\pgfsetfillopacity{0.585332}%
\pgfsetlinewidth{1.003750pt}%
\definecolor{currentstroke}{rgb}{0.121569,0.466667,0.705882}%
\pgfsetstrokecolor{currentstroke}%
\pgfsetstrokeopacity{0.585332}%
\pgfsetdash{}{0pt}%
\pgfpathmoveto{\pgfqpoint{3.253214in}{2.000029in}}%
\pgfpathcurveto{\pgfqpoint{3.261450in}{2.000029in}}{\pgfqpoint{3.269350in}{2.003301in}}{\pgfqpoint{3.275174in}{2.009125in}}%
\pgfpathcurveto{\pgfqpoint{3.280998in}{2.014949in}}{\pgfqpoint{3.284270in}{2.022849in}}{\pgfqpoint{3.284270in}{2.031085in}}%
\pgfpathcurveto{\pgfqpoint{3.284270in}{2.039321in}}{\pgfqpoint{3.280998in}{2.047221in}}{\pgfqpoint{3.275174in}{2.053045in}}%
\pgfpathcurveto{\pgfqpoint{3.269350in}{2.058869in}}{\pgfqpoint{3.261450in}{2.062142in}}{\pgfqpoint{3.253214in}{2.062142in}}%
\pgfpathcurveto{\pgfqpoint{3.244977in}{2.062142in}}{\pgfqpoint{3.237077in}{2.058869in}}{\pgfqpoint{3.231253in}{2.053045in}}%
\pgfpathcurveto{\pgfqpoint{3.225429in}{2.047221in}}{\pgfqpoint{3.222157in}{2.039321in}}{\pgfqpoint{3.222157in}{2.031085in}}%
\pgfpathcurveto{\pgfqpoint{3.222157in}{2.022849in}}{\pgfqpoint{3.225429in}{2.014949in}}{\pgfqpoint{3.231253in}{2.009125in}}%
\pgfpathcurveto{\pgfqpoint{3.237077in}{2.003301in}}{\pgfqpoint{3.244977in}{2.000029in}}{\pgfqpoint{3.253214in}{2.000029in}}%
\pgfpathclose%
\pgfusepath{stroke,fill}%
\end{pgfscope}%
\begin{pgfscope}%
\pgfpathrectangle{\pgfqpoint{0.100000in}{0.212622in}}{\pgfqpoint{3.696000in}{3.696000in}}%
\pgfusepath{clip}%
\pgfsetbuttcap%
\pgfsetroundjoin%
\definecolor{currentfill}{rgb}{0.121569,0.466667,0.705882}%
\pgfsetfillcolor{currentfill}%
\pgfsetfillopacity{0.586256}%
\pgfsetlinewidth{1.003750pt}%
\definecolor{currentstroke}{rgb}{0.121569,0.466667,0.705882}%
\pgfsetstrokecolor{currentstroke}%
\pgfsetstrokeopacity{0.586256}%
\pgfsetdash{}{0pt}%
\pgfpathmoveto{\pgfqpoint{3.252511in}{2.000307in}}%
\pgfpathcurveto{\pgfqpoint{3.260747in}{2.000307in}}{\pgfqpoint{3.268647in}{2.003579in}}{\pgfqpoint{3.274471in}{2.009403in}}%
\pgfpathcurveto{\pgfqpoint{3.280295in}{2.015227in}}{\pgfqpoint{3.283567in}{2.023127in}}{\pgfqpoint{3.283567in}{2.031364in}}%
\pgfpathcurveto{\pgfqpoint{3.283567in}{2.039600in}}{\pgfqpoint{3.280295in}{2.047500in}}{\pgfqpoint{3.274471in}{2.053324in}}%
\pgfpathcurveto{\pgfqpoint{3.268647in}{2.059148in}}{\pgfqpoint{3.260747in}{2.062420in}}{\pgfqpoint{3.252511in}{2.062420in}}%
\pgfpathcurveto{\pgfqpoint{3.244275in}{2.062420in}}{\pgfqpoint{3.236375in}{2.059148in}}{\pgfqpoint{3.230551in}{2.053324in}}%
\pgfpathcurveto{\pgfqpoint{3.224727in}{2.047500in}}{\pgfqpoint{3.221454in}{2.039600in}}{\pgfqpoint{3.221454in}{2.031364in}}%
\pgfpathcurveto{\pgfqpoint{3.221454in}{2.023127in}}{\pgfqpoint{3.224727in}{2.015227in}}{\pgfqpoint{3.230551in}{2.009403in}}%
\pgfpathcurveto{\pgfqpoint{3.236375in}{2.003579in}}{\pgfqpoint{3.244275in}{2.000307in}}{\pgfqpoint{3.252511in}{2.000307in}}%
\pgfpathclose%
\pgfusepath{stroke,fill}%
\end{pgfscope}%
\begin{pgfscope}%
\pgfpathrectangle{\pgfqpoint{0.100000in}{0.212622in}}{\pgfqpoint{3.696000in}{3.696000in}}%
\pgfusepath{clip}%
\pgfsetbuttcap%
\pgfsetroundjoin%
\definecolor{currentfill}{rgb}{0.121569,0.466667,0.705882}%
\pgfsetfillcolor{currentfill}%
\pgfsetfillopacity{0.587156}%
\pgfsetlinewidth{1.003750pt}%
\definecolor{currentstroke}{rgb}{0.121569,0.466667,0.705882}%
\pgfsetstrokecolor{currentstroke}%
\pgfsetstrokeopacity{0.587156}%
\pgfsetdash{}{0pt}%
\pgfpathmoveto{\pgfqpoint{3.249985in}{1.999953in}}%
\pgfpathcurveto{\pgfqpoint{3.258221in}{1.999953in}}{\pgfqpoint{3.266121in}{2.003226in}}{\pgfqpoint{3.271945in}{2.009050in}}%
\pgfpathcurveto{\pgfqpoint{3.277769in}{2.014874in}}{\pgfqpoint{3.281041in}{2.022774in}}{\pgfqpoint{3.281041in}{2.031010in}}%
\pgfpathcurveto{\pgfqpoint{3.281041in}{2.039246in}}{\pgfqpoint{3.277769in}{2.047146in}}{\pgfqpoint{3.271945in}{2.052970in}}%
\pgfpathcurveto{\pgfqpoint{3.266121in}{2.058794in}}{\pgfqpoint{3.258221in}{2.062066in}}{\pgfqpoint{3.249985in}{2.062066in}}%
\pgfpathcurveto{\pgfqpoint{3.241748in}{2.062066in}}{\pgfqpoint{3.233848in}{2.058794in}}{\pgfqpoint{3.228025in}{2.052970in}}%
\pgfpathcurveto{\pgfqpoint{3.222201in}{2.047146in}}{\pgfqpoint{3.218928in}{2.039246in}}{\pgfqpoint{3.218928in}{2.031010in}}%
\pgfpathcurveto{\pgfqpoint{3.218928in}{2.022774in}}{\pgfqpoint{3.222201in}{2.014874in}}{\pgfqpoint{3.228025in}{2.009050in}}%
\pgfpathcurveto{\pgfqpoint{3.233848in}{2.003226in}}{\pgfqpoint{3.241748in}{1.999953in}}{\pgfqpoint{3.249985in}{1.999953in}}%
\pgfpathclose%
\pgfusepath{stroke,fill}%
\end{pgfscope}%
\begin{pgfscope}%
\pgfpathrectangle{\pgfqpoint{0.100000in}{0.212622in}}{\pgfqpoint{3.696000in}{3.696000in}}%
\pgfusepath{clip}%
\pgfsetbuttcap%
\pgfsetroundjoin%
\definecolor{currentfill}{rgb}{0.121569,0.466667,0.705882}%
\pgfsetfillcolor{currentfill}%
\pgfsetfillopacity{0.587632}%
\pgfsetlinewidth{1.003750pt}%
\definecolor{currentstroke}{rgb}{0.121569,0.466667,0.705882}%
\pgfsetstrokecolor{currentstroke}%
\pgfsetstrokeopacity{0.587632}%
\pgfsetdash{}{0pt}%
\pgfpathmoveto{\pgfqpoint{3.248491in}{1.999804in}}%
\pgfpathcurveto{\pgfqpoint{3.256727in}{1.999804in}}{\pgfqpoint{3.264627in}{2.003076in}}{\pgfqpoint{3.270451in}{2.008900in}}%
\pgfpathcurveto{\pgfqpoint{3.276275in}{2.014724in}}{\pgfqpoint{3.279547in}{2.022624in}}{\pgfqpoint{3.279547in}{2.030860in}}%
\pgfpathcurveto{\pgfqpoint{3.279547in}{2.039096in}}{\pgfqpoint{3.276275in}{2.046996in}}{\pgfqpoint{3.270451in}{2.052820in}}%
\pgfpathcurveto{\pgfqpoint{3.264627in}{2.058644in}}{\pgfqpoint{3.256727in}{2.061917in}}{\pgfqpoint{3.248491in}{2.061917in}}%
\pgfpathcurveto{\pgfqpoint{3.240254in}{2.061917in}}{\pgfqpoint{3.232354in}{2.058644in}}{\pgfqpoint{3.226530in}{2.052820in}}%
\pgfpathcurveto{\pgfqpoint{3.220706in}{2.046996in}}{\pgfqpoint{3.217434in}{2.039096in}}{\pgfqpoint{3.217434in}{2.030860in}}%
\pgfpathcurveto{\pgfqpoint{3.217434in}{2.022624in}}{\pgfqpoint{3.220706in}{2.014724in}}{\pgfqpoint{3.226530in}{2.008900in}}%
\pgfpathcurveto{\pgfqpoint{3.232354in}{2.003076in}}{\pgfqpoint{3.240254in}{1.999804in}}{\pgfqpoint{3.248491in}{1.999804in}}%
\pgfpathclose%
\pgfusepath{stroke,fill}%
\end{pgfscope}%
\begin{pgfscope}%
\pgfpathrectangle{\pgfqpoint{0.100000in}{0.212622in}}{\pgfqpoint{3.696000in}{3.696000in}}%
\pgfusepath{clip}%
\pgfsetbuttcap%
\pgfsetroundjoin%
\definecolor{currentfill}{rgb}{0.121569,0.466667,0.705882}%
\pgfsetfillcolor{currentfill}%
\pgfsetfillopacity{0.588504}%
\pgfsetlinewidth{1.003750pt}%
\definecolor{currentstroke}{rgb}{0.121569,0.466667,0.705882}%
\pgfsetstrokecolor{currentstroke}%
\pgfsetstrokeopacity{0.588504}%
\pgfsetdash{}{0pt}%
\pgfpathmoveto{\pgfqpoint{3.247728in}{1.999411in}}%
\pgfpathcurveto{\pgfqpoint{3.255965in}{1.999411in}}{\pgfqpoint{3.263865in}{2.002683in}}{\pgfqpoint{3.269689in}{2.008507in}}%
\pgfpathcurveto{\pgfqpoint{3.275513in}{2.014331in}}{\pgfqpoint{3.278785in}{2.022231in}}{\pgfqpoint{3.278785in}{2.030467in}}%
\pgfpathcurveto{\pgfqpoint{3.278785in}{2.038704in}}{\pgfqpoint{3.275513in}{2.046604in}}{\pgfqpoint{3.269689in}{2.052428in}}%
\pgfpathcurveto{\pgfqpoint{3.263865in}{2.058252in}}{\pgfqpoint{3.255965in}{2.061524in}}{\pgfqpoint{3.247728in}{2.061524in}}%
\pgfpathcurveto{\pgfqpoint{3.239492in}{2.061524in}}{\pgfqpoint{3.231592in}{2.058252in}}{\pgfqpoint{3.225768in}{2.052428in}}%
\pgfpathcurveto{\pgfqpoint{3.219944in}{2.046604in}}{\pgfqpoint{3.216672in}{2.038704in}}{\pgfqpoint{3.216672in}{2.030467in}}%
\pgfpathcurveto{\pgfqpoint{3.216672in}{2.022231in}}{\pgfqpoint{3.219944in}{2.014331in}}{\pgfqpoint{3.225768in}{2.008507in}}%
\pgfpathcurveto{\pgfqpoint{3.231592in}{2.002683in}}{\pgfqpoint{3.239492in}{1.999411in}}{\pgfqpoint{3.247728in}{1.999411in}}%
\pgfpathclose%
\pgfusepath{stroke,fill}%
\end{pgfscope}%
\begin{pgfscope}%
\pgfpathrectangle{\pgfqpoint{0.100000in}{0.212622in}}{\pgfqpoint{3.696000in}{3.696000in}}%
\pgfusepath{clip}%
\pgfsetbuttcap%
\pgfsetroundjoin%
\definecolor{currentfill}{rgb}{0.121569,0.466667,0.705882}%
\pgfsetfillcolor{currentfill}%
\pgfsetfillopacity{0.588886}%
\pgfsetlinewidth{1.003750pt}%
\definecolor{currentstroke}{rgb}{0.121569,0.466667,0.705882}%
\pgfsetstrokecolor{currentstroke}%
\pgfsetstrokeopacity{0.588886}%
\pgfsetdash{}{0pt}%
\pgfpathmoveto{\pgfqpoint{3.246823in}{1.998915in}}%
\pgfpathcurveto{\pgfqpoint{3.255059in}{1.998915in}}{\pgfqpoint{3.262959in}{2.002187in}}{\pgfqpoint{3.268783in}{2.008011in}}%
\pgfpathcurveto{\pgfqpoint{3.274607in}{2.013835in}}{\pgfqpoint{3.277879in}{2.021735in}}{\pgfqpoint{3.277879in}{2.029971in}}%
\pgfpathcurveto{\pgfqpoint{3.277879in}{2.038208in}}{\pgfqpoint{3.274607in}{2.046108in}}{\pgfqpoint{3.268783in}{2.051931in}}%
\pgfpathcurveto{\pgfqpoint{3.262959in}{2.057755in}}{\pgfqpoint{3.255059in}{2.061028in}}{\pgfqpoint{3.246823in}{2.061028in}}%
\pgfpathcurveto{\pgfqpoint{3.238586in}{2.061028in}}{\pgfqpoint{3.230686in}{2.057755in}}{\pgfqpoint{3.224862in}{2.051931in}}%
\pgfpathcurveto{\pgfqpoint{3.219038in}{2.046108in}}{\pgfqpoint{3.215766in}{2.038208in}}{\pgfqpoint{3.215766in}{2.029971in}}%
\pgfpathcurveto{\pgfqpoint{3.215766in}{2.021735in}}{\pgfqpoint{3.219038in}{2.013835in}}{\pgfqpoint{3.224862in}{2.008011in}}%
\pgfpathcurveto{\pgfqpoint{3.230686in}{2.002187in}}{\pgfqpoint{3.238586in}{1.998915in}}{\pgfqpoint{3.246823in}{1.998915in}}%
\pgfpathclose%
\pgfusepath{stroke,fill}%
\end{pgfscope}%
\begin{pgfscope}%
\pgfpathrectangle{\pgfqpoint{0.100000in}{0.212622in}}{\pgfqpoint{3.696000in}{3.696000in}}%
\pgfusepath{clip}%
\pgfsetbuttcap%
\pgfsetroundjoin%
\definecolor{currentfill}{rgb}{0.121569,0.466667,0.705882}%
\pgfsetfillcolor{currentfill}%
\pgfsetfillopacity{0.588928}%
\pgfsetlinewidth{1.003750pt}%
\definecolor{currentstroke}{rgb}{0.121569,0.466667,0.705882}%
\pgfsetstrokecolor{currentstroke}%
\pgfsetstrokeopacity{0.588928}%
\pgfsetdash{}{0pt}%
\pgfpathmoveto{\pgfqpoint{0.958757in}{1.660855in}}%
\pgfpathcurveto{\pgfqpoint{0.966994in}{1.660855in}}{\pgfqpoint{0.974894in}{1.664127in}}{\pgfqpoint{0.980718in}{1.669951in}}%
\pgfpathcurveto{\pgfqpoint{0.986542in}{1.675775in}}{\pgfqpoint{0.989814in}{1.683675in}}{\pgfqpoint{0.989814in}{1.691911in}}%
\pgfpathcurveto{\pgfqpoint{0.989814in}{1.700147in}}{\pgfqpoint{0.986542in}{1.708047in}}{\pgfqpoint{0.980718in}{1.713871in}}%
\pgfpathcurveto{\pgfqpoint{0.974894in}{1.719695in}}{\pgfqpoint{0.966994in}{1.722968in}}{\pgfqpoint{0.958757in}{1.722968in}}%
\pgfpathcurveto{\pgfqpoint{0.950521in}{1.722968in}}{\pgfqpoint{0.942621in}{1.719695in}}{\pgfqpoint{0.936797in}{1.713871in}}%
\pgfpathcurveto{\pgfqpoint{0.930973in}{1.708047in}}{\pgfqpoint{0.927701in}{1.700147in}}{\pgfqpoint{0.927701in}{1.691911in}}%
\pgfpathcurveto{\pgfqpoint{0.927701in}{1.683675in}}{\pgfqpoint{0.930973in}{1.675775in}}{\pgfqpoint{0.936797in}{1.669951in}}%
\pgfpathcurveto{\pgfqpoint{0.942621in}{1.664127in}}{\pgfqpoint{0.950521in}{1.660855in}}{\pgfqpoint{0.958757in}{1.660855in}}%
\pgfpathclose%
\pgfusepath{stroke,fill}%
\end{pgfscope}%
\begin{pgfscope}%
\pgfpathrectangle{\pgfqpoint{0.100000in}{0.212622in}}{\pgfqpoint{3.696000in}{3.696000in}}%
\pgfusepath{clip}%
\pgfsetbuttcap%
\pgfsetroundjoin%
\definecolor{currentfill}{rgb}{0.121569,0.466667,0.705882}%
\pgfsetfillcolor{currentfill}%
\pgfsetfillopacity{0.589426}%
\pgfsetlinewidth{1.003750pt}%
\definecolor{currentstroke}{rgb}{0.121569,0.466667,0.705882}%
\pgfsetstrokecolor{currentstroke}%
\pgfsetstrokeopacity{0.589426}%
\pgfsetdash{}{0pt}%
\pgfpathmoveto{\pgfqpoint{3.245213in}{1.999382in}}%
\pgfpathcurveto{\pgfqpoint{3.253449in}{1.999382in}}{\pgfqpoint{3.261349in}{2.002654in}}{\pgfqpoint{3.267173in}{2.008478in}}%
\pgfpathcurveto{\pgfqpoint{3.272997in}{2.014302in}}{\pgfqpoint{3.276269in}{2.022202in}}{\pgfqpoint{3.276269in}{2.030438in}}%
\pgfpathcurveto{\pgfqpoint{3.276269in}{2.038675in}}{\pgfqpoint{3.272997in}{2.046575in}}{\pgfqpoint{3.267173in}{2.052399in}}%
\pgfpathcurveto{\pgfqpoint{3.261349in}{2.058223in}}{\pgfqpoint{3.253449in}{2.061495in}}{\pgfqpoint{3.245213in}{2.061495in}}%
\pgfpathcurveto{\pgfqpoint{3.236977in}{2.061495in}}{\pgfqpoint{3.229077in}{2.058223in}}{\pgfqpoint{3.223253in}{2.052399in}}%
\pgfpathcurveto{\pgfqpoint{3.217429in}{2.046575in}}{\pgfqpoint{3.214156in}{2.038675in}}{\pgfqpoint{3.214156in}{2.030438in}}%
\pgfpathcurveto{\pgfqpoint{3.214156in}{2.022202in}}{\pgfqpoint{3.217429in}{2.014302in}}{\pgfqpoint{3.223253in}{2.008478in}}%
\pgfpathcurveto{\pgfqpoint{3.229077in}{2.002654in}}{\pgfqpoint{3.236977in}{1.999382in}}{\pgfqpoint{3.245213in}{1.999382in}}%
\pgfpathclose%
\pgfusepath{stroke,fill}%
\end{pgfscope}%
\begin{pgfscope}%
\pgfpathrectangle{\pgfqpoint{0.100000in}{0.212622in}}{\pgfqpoint{3.696000in}{3.696000in}}%
\pgfusepath{clip}%
\pgfsetbuttcap%
\pgfsetroundjoin%
\definecolor{currentfill}{rgb}{0.121569,0.466667,0.705882}%
\pgfsetfillcolor{currentfill}%
\pgfsetfillopacity{0.590046}%
\pgfsetlinewidth{1.003750pt}%
\definecolor{currentstroke}{rgb}{0.121569,0.466667,0.705882}%
\pgfsetstrokecolor{currentstroke}%
\pgfsetstrokeopacity{0.590046}%
\pgfsetdash{}{0pt}%
\pgfpathmoveto{\pgfqpoint{3.244243in}{1.997075in}}%
\pgfpathcurveto{\pgfqpoint{3.252479in}{1.997075in}}{\pgfqpoint{3.260379in}{2.000348in}}{\pgfqpoint{3.266203in}{2.006172in}}%
\pgfpathcurveto{\pgfqpoint{3.272027in}{2.011996in}}{\pgfqpoint{3.275299in}{2.019896in}}{\pgfqpoint{3.275299in}{2.028132in}}%
\pgfpathcurveto{\pgfqpoint{3.275299in}{2.036368in}}{\pgfqpoint{3.272027in}{2.044268in}}{\pgfqpoint{3.266203in}{2.050092in}}%
\pgfpathcurveto{\pgfqpoint{3.260379in}{2.055916in}}{\pgfqpoint{3.252479in}{2.059188in}}{\pgfqpoint{3.244243in}{2.059188in}}%
\pgfpathcurveto{\pgfqpoint{3.236006in}{2.059188in}}{\pgfqpoint{3.228106in}{2.055916in}}{\pgfqpoint{3.222282in}{2.050092in}}%
\pgfpathcurveto{\pgfqpoint{3.216458in}{2.044268in}}{\pgfqpoint{3.213186in}{2.036368in}}{\pgfqpoint{3.213186in}{2.028132in}}%
\pgfpathcurveto{\pgfqpoint{3.213186in}{2.019896in}}{\pgfqpoint{3.216458in}{2.011996in}}{\pgfqpoint{3.222282in}{2.006172in}}%
\pgfpathcurveto{\pgfqpoint{3.228106in}{2.000348in}}{\pgfqpoint{3.236006in}{1.997075in}}{\pgfqpoint{3.244243in}{1.997075in}}%
\pgfpathclose%
\pgfusepath{stroke,fill}%
\end{pgfscope}%
\begin{pgfscope}%
\pgfpathrectangle{\pgfqpoint{0.100000in}{0.212622in}}{\pgfqpoint{3.696000in}{3.696000in}}%
\pgfusepath{clip}%
\pgfsetbuttcap%
\pgfsetroundjoin%
\definecolor{currentfill}{rgb}{0.121569,0.466667,0.705882}%
\pgfsetfillcolor{currentfill}%
\pgfsetfillopacity{0.590891}%
\pgfsetlinewidth{1.003750pt}%
\definecolor{currentstroke}{rgb}{0.121569,0.466667,0.705882}%
\pgfsetstrokecolor{currentstroke}%
\pgfsetstrokeopacity{0.590891}%
\pgfsetdash{}{0pt}%
\pgfpathmoveto{\pgfqpoint{3.243206in}{1.995762in}}%
\pgfpathcurveto{\pgfqpoint{3.251442in}{1.995762in}}{\pgfqpoint{3.259342in}{1.999035in}}{\pgfqpoint{3.265166in}{2.004858in}}%
\pgfpathcurveto{\pgfqpoint{3.270990in}{2.010682in}}{\pgfqpoint{3.274263in}{2.018582in}}{\pgfqpoint{3.274263in}{2.026819in}}%
\pgfpathcurveto{\pgfqpoint{3.274263in}{2.035055in}}{\pgfqpoint{3.270990in}{2.042955in}}{\pgfqpoint{3.265166in}{2.048779in}}%
\pgfpathcurveto{\pgfqpoint{3.259342in}{2.054603in}}{\pgfqpoint{3.251442in}{2.057875in}}{\pgfqpoint{3.243206in}{2.057875in}}%
\pgfpathcurveto{\pgfqpoint{3.234970in}{2.057875in}}{\pgfqpoint{3.227070in}{2.054603in}}{\pgfqpoint{3.221246in}{2.048779in}}%
\pgfpathcurveto{\pgfqpoint{3.215422in}{2.042955in}}{\pgfqpoint{3.212150in}{2.035055in}}{\pgfqpoint{3.212150in}{2.026819in}}%
\pgfpathcurveto{\pgfqpoint{3.212150in}{2.018582in}}{\pgfqpoint{3.215422in}{2.010682in}}{\pgfqpoint{3.221246in}{2.004858in}}%
\pgfpathcurveto{\pgfqpoint{3.227070in}{1.999035in}}{\pgfqpoint{3.234970in}{1.995762in}}{\pgfqpoint{3.243206in}{1.995762in}}%
\pgfpathclose%
\pgfusepath{stroke,fill}%
\end{pgfscope}%
\begin{pgfscope}%
\pgfpathrectangle{\pgfqpoint{0.100000in}{0.212622in}}{\pgfqpoint{3.696000in}{3.696000in}}%
\pgfusepath{clip}%
\pgfsetbuttcap%
\pgfsetroundjoin%
\definecolor{currentfill}{rgb}{0.121569,0.466667,0.705882}%
\pgfsetfillcolor{currentfill}%
\pgfsetfillopacity{0.592347}%
\pgfsetlinewidth{1.003750pt}%
\definecolor{currentstroke}{rgb}{0.121569,0.466667,0.705882}%
\pgfsetstrokecolor{currentstroke}%
\pgfsetstrokeopacity{0.592347}%
\pgfsetdash{}{0pt}%
\pgfpathmoveto{\pgfqpoint{3.239679in}{1.996461in}}%
\pgfpathcurveto{\pgfqpoint{3.247915in}{1.996461in}}{\pgfqpoint{3.255815in}{1.999733in}}{\pgfqpoint{3.261639in}{2.005557in}}%
\pgfpathcurveto{\pgfqpoint{3.267463in}{2.011381in}}{\pgfqpoint{3.270735in}{2.019281in}}{\pgfqpoint{3.270735in}{2.027517in}}%
\pgfpathcurveto{\pgfqpoint{3.270735in}{2.035753in}}{\pgfqpoint{3.267463in}{2.043653in}}{\pgfqpoint{3.261639in}{2.049477in}}%
\pgfpathcurveto{\pgfqpoint{3.255815in}{2.055301in}}{\pgfqpoint{3.247915in}{2.058574in}}{\pgfqpoint{3.239679in}{2.058574in}}%
\pgfpathcurveto{\pgfqpoint{3.231443in}{2.058574in}}{\pgfqpoint{3.223542in}{2.055301in}}{\pgfqpoint{3.217719in}{2.049477in}}%
\pgfpathcurveto{\pgfqpoint{3.211895in}{2.043653in}}{\pgfqpoint{3.208622in}{2.035753in}}{\pgfqpoint{3.208622in}{2.027517in}}%
\pgfpathcurveto{\pgfqpoint{3.208622in}{2.019281in}}{\pgfqpoint{3.211895in}{2.011381in}}{\pgfqpoint{3.217719in}{2.005557in}}%
\pgfpathcurveto{\pgfqpoint{3.223542in}{1.999733in}}{\pgfqpoint{3.231443in}{1.996461in}}{\pgfqpoint{3.239679in}{1.996461in}}%
\pgfpathclose%
\pgfusepath{stroke,fill}%
\end{pgfscope}%
\begin{pgfscope}%
\pgfpathrectangle{\pgfqpoint{0.100000in}{0.212622in}}{\pgfqpoint{3.696000in}{3.696000in}}%
\pgfusepath{clip}%
\pgfsetbuttcap%
\pgfsetroundjoin%
\definecolor{currentfill}{rgb}{0.121569,0.466667,0.705882}%
\pgfsetfillcolor{currentfill}%
\pgfsetfillopacity{0.593532}%
\pgfsetlinewidth{1.003750pt}%
\definecolor{currentstroke}{rgb}{0.121569,0.466667,0.705882}%
\pgfsetstrokecolor{currentstroke}%
\pgfsetstrokeopacity{0.593532}%
\pgfsetdash{}{0pt}%
\pgfpathmoveto{\pgfqpoint{3.236484in}{1.993752in}}%
\pgfpathcurveto{\pgfqpoint{3.244720in}{1.993752in}}{\pgfqpoint{3.252620in}{1.997024in}}{\pgfqpoint{3.258444in}{2.002848in}}%
\pgfpathcurveto{\pgfqpoint{3.264268in}{2.008672in}}{\pgfqpoint{3.267540in}{2.016572in}}{\pgfqpoint{3.267540in}{2.024808in}}%
\pgfpathcurveto{\pgfqpoint{3.267540in}{2.033045in}}{\pgfqpoint{3.264268in}{2.040945in}}{\pgfqpoint{3.258444in}{2.046769in}}%
\pgfpathcurveto{\pgfqpoint{3.252620in}{2.052592in}}{\pgfqpoint{3.244720in}{2.055865in}}{\pgfqpoint{3.236484in}{2.055865in}}%
\pgfpathcurveto{\pgfqpoint{3.228247in}{2.055865in}}{\pgfqpoint{3.220347in}{2.052592in}}{\pgfqpoint{3.214524in}{2.046769in}}%
\pgfpathcurveto{\pgfqpoint{3.208700in}{2.040945in}}{\pgfqpoint{3.205427in}{2.033045in}}{\pgfqpoint{3.205427in}{2.024808in}}%
\pgfpathcurveto{\pgfqpoint{3.205427in}{2.016572in}}{\pgfqpoint{3.208700in}{2.008672in}}{\pgfqpoint{3.214524in}{2.002848in}}%
\pgfpathcurveto{\pgfqpoint{3.220347in}{1.997024in}}{\pgfqpoint{3.228247in}{1.993752in}}{\pgfqpoint{3.236484in}{1.993752in}}%
\pgfpathclose%
\pgfusepath{stroke,fill}%
\end{pgfscope}%
\begin{pgfscope}%
\pgfpathrectangle{\pgfqpoint{0.100000in}{0.212622in}}{\pgfqpoint{3.696000in}{3.696000in}}%
\pgfusepath{clip}%
\pgfsetbuttcap%
\pgfsetroundjoin%
\definecolor{currentfill}{rgb}{0.121569,0.466667,0.705882}%
\pgfsetfillcolor{currentfill}%
\pgfsetfillopacity{0.594610}%
\pgfsetlinewidth{1.003750pt}%
\definecolor{currentstroke}{rgb}{0.121569,0.466667,0.705882}%
\pgfsetstrokecolor{currentstroke}%
\pgfsetstrokeopacity{0.594610}%
\pgfsetdash{}{0pt}%
\pgfpathmoveto{\pgfqpoint{3.235422in}{1.994695in}}%
\pgfpathcurveto{\pgfqpoint{3.243658in}{1.994695in}}{\pgfqpoint{3.251558in}{1.997967in}}{\pgfqpoint{3.257382in}{2.003791in}}%
\pgfpathcurveto{\pgfqpoint{3.263206in}{2.009615in}}{\pgfqpoint{3.266478in}{2.017515in}}{\pgfqpoint{3.266478in}{2.025752in}}%
\pgfpathcurveto{\pgfqpoint{3.266478in}{2.033988in}}{\pgfqpoint{3.263206in}{2.041888in}}{\pgfqpoint{3.257382in}{2.047712in}}%
\pgfpathcurveto{\pgfqpoint{3.251558in}{2.053536in}}{\pgfqpoint{3.243658in}{2.056808in}}{\pgfqpoint{3.235422in}{2.056808in}}%
\pgfpathcurveto{\pgfqpoint{3.227185in}{2.056808in}}{\pgfqpoint{3.219285in}{2.053536in}}{\pgfqpoint{3.213461in}{2.047712in}}%
\pgfpathcurveto{\pgfqpoint{3.207638in}{2.041888in}}{\pgfqpoint{3.204365in}{2.033988in}}{\pgfqpoint{3.204365in}{2.025752in}}%
\pgfpathcurveto{\pgfqpoint{3.204365in}{2.017515in}}{\pgfqpoint{3.207638in}{2.009615in}}{\pgfqpoint{3.213461in}{2.003791in}}%
\pgfpathcurveto{\pgfqpoint{3.219285in}{1.997967in}}{\pgfqpoint{3.227185in}{1.994695in}}{\pgfqpoint{3.235422in}{1.994695in}}%
\pgfpathclose%
\pgfusepath{stroke,fill}%
\end{pgfscope}%
\begin{pgfscope}%
\pgfpathrectangle{\pgfqpoint{0.100000in}{0.212622in}}{\pgfqpoint{3.696000in}{3.696000in}}%
\pgfusepath{clip}%
\pgfsetbuttcap%
\pgfsetroundjoin%
\definecolor{currentfill}{rgb}{0.121569,0.466667,0.705882}%
\pgfsetfillcolor{currentfill}%
\pgfsetfillopacity{0.595579}%
\pgfsetlinewidth{1.003750pt}%
\definecolor{currentstroke}{rgb}{0.121569,0.466667,0.705882}%
\pgfsetstrokecolor{currentstroke}%
\pgfsetstrokeopacity{0.595579}%
\pgfsetdash{}{0pt}%
\pgfpathmoveto{\pgfqpoint{3.233168in}{1.993361in}}%
\pgfpathcurveto{\pgfqpoint{3.241404in}{1.993361in}}{\pgfqpoint{3.249304in}{1.996633in}}{\pgfqpoint{3.255128in}{2.002457in}}%
\pgfpathcurveto{\pgfqpoint{3.260952in}{2.008281in}}{\pgfqpoint{3.264224in}{2.016181in}}{\pgfqpoint{3.264224in}{2.024417in}}%
\pgfpathcurveto{\pgfqpoint{3.264224in}{2.032653in}}{\pgfqpoint{3.260952in}{2.040554in}}{\pgfqpoint{3.255128in}{2.046377in}}%
\pgfpathcurveto{\pgfqpoint{3.249304in}{2.052201in}}{\pgfqpoint{3.241404in}{2.055474in}}{\pgfqpoint{3.233168in}{2.055474in}}%
\pgfpathcurveto{\pgfqpoint{3.224932in}{2.055474in}}{\pgfqpoint{3.217032in}{2.052201in}}{\pgfqpoint{3.211208in}{2.046377in}}%
\pgfpathcurveto{\pgfqpoint{3.205384in}{2.040554in}}{\pgfqpoint{3.202111in}{2.032653in}}{\pgfqpoint{3.202111in}{2.024417in}}%
\pgfpathcurveto{\pgfqpoint{3.202111in}{2.016181in}}{\pgfqpoint{3.205384in}{2.008281in}}{\pgfqpoint{3.211208in}{2.002457in}}%
\pgfpathcurveto{\pgfqpoint{3.217032in}{1.996633in}}{\pgfqpoint{3.224932in}{1.993361in}}{\pgfqpoint{3.233168in}{1.993361in}}%
\pgfpathclose%
\pgfusepath{stroke,fill}%
\end{pgfscope}%
\begin{pgfscope}%
\pgfpathrectangle{\pgfqpoint{0.100000in}{0.212622in}}{\pgfqpoint{3.696000in}{3.696000in}}%
\pgfusepath{clip}%
\pgfsetbuttcap%
\pgfsetroundjoin%
\definecolor{currentfill}{rgb}{0.121569,0.466667,0.705882}%
\pgfsetfillcolor{currentfill}%
\pgfsetfillopacity{0.596025}%
\pgfsetlinewidth{1.003750pt}%
\definecolor{currentstroke}{rgb}{0.121569,0.466667,0.705882}%
\pgfsetstrokecolor{currentstroke}%
\pgfsetstrokeopacity{0.596025}%
\pgfsetdash{}{0pt}%
\pgfpathmoveto{\pgfqpoint{0.953891in}{1.686337in}}%
\pgfpathcurveto{\pgfqpoint{0.962127in}{1.686337in}}{\pgfqpoint{0.970027in}{1.689610in}}{\pgfqpoint{0.975851in}{1.695434in}}%
\pgfpathcurveto{\pgfqpoint{0.981675in}{1.701258in}}{\pgfqpoint{0.984948in}{1.709158in}}{\pgfqpoint{0.984948in}{1.717394in}}%
\pgfpathcurveto{\pgfqpoint{0.984948in}{1.725630in}}{\pgfqpoint{0.981675in}{1.733530in}}{\pgfqpoint{0.975851in}{1.739354in}}%
\pgfpathcurveto{\pgfqpoint{0.970027in}{1.745178in}}{\pgfqpoint{0.962127in}{1.748450in}}{\pgfqpoint{0.953891in}{1.748450in}}%
\pgfpathcurveto{\pgfqpoint{0.945655in}{1.748450in}}{\pgfqpoint{0.937755in}{1.745178in}}{\pgfqpoint{0.931931in}{1.739354in}}%
\pgfpathcurveto{\pgfqpoint{0.926107in}{1.733530in}}{\pgfqpoint{0.922835in}{1.725630in}}{\pgfqpoint{0.922835in}{1.717394in}}%
\pgfpathcurveto{\pgfqpoint{0.922835in}{1.709158in}}{\pgfqpoint{0.926107in}{1.701258in}}{\pgfqpoint{0.931931in}{1.695434in}}%
\pgfpathcurveto{\pgfqpoint{0.937755in}{1.689610in}}{\pgfqpoint{0.945655in}{1.686337in}}{\pgfqpoint{0.953891in}{1.686337in}}%
\pgfpathclose%
\pgfusepath{stroke,fill}%
\end{pgfscope}%
\begin{pgfscope}%
\pgfpathrectangle{\pgfqpoint{0.100000in}{0.212622in}}{\pgfqpoint{3.696000in}{3.696000in}}%
\pgfusepath{clip}%
\pgfsetbuttcap%
\pgfsetroundjoin%
\definecolor{currentfill}{rgb}{0.121569,0.466667,0.705882}%
\pgfsetfillcolor{currentfill}%
\pgfsetfillopacity{0.597035}%
\pgfsetlinewidth{1.003750pt}%
\definecolor{currentstroke}{rgb}{0.121569,0.466667,0.705882}%
\pgfsetstrokecolor{currentstroke}%
\pgfsetstrokeopacity{0.597035}%
\pgfsetdash{}{0pt}%
\pgfpathmoveto{\pgfqpoint{3.229081in}{1.995664in}}%
\pgfpathcurveto{\pgfqpoint{3.237317in}{1.995664in}}{\pgfqpoint{3.245217in}{1.998936in}}{\pgfqpoint{3.251041in}{2.004760in}}%
\pgfpathcurveto{\pgfqpoint{3.256865in}{2.010584in}}{\pgfqpoint{3.260137in}{2.018484in}}{\pgfqpoint{3.260137in}{2.026720in}}%
\pgfpathcurveto{\pgfqpoint{3.260137in}{2.034957in}}{\pgfqpoint{3.256865in}{2.042857in}}{\pgfqpoint{3.251041in}{2.048681in}}%
\pgfpathcurveto{\pgfqpoint{3.245217in}{2.054505in}}{\pgfqpoint{3.237317in}{2.057777in}}{\pgfqpoint{3.229081in}{2.057777in}}%
\pgfpathcurveto{\pgfqpoint{3.220845in}{2.057777in}}{\pgfqpoint{3.212945in}{2.054505in}}{\pgfqpoint{3.207121in}{2.048681in}}%
\pgfpathcurveto{\pgfqpoint{3.201297in}{2.042857in}}{\pgfqpoint{3.198024in}{2.034957in}}{\pgfqpoint{3.198024in}{2.026720in}}%
\pgfpathcurveto{\pgfqpoint{3.198024in}{2.018484in}}{\pgfqpoint{3.201297in}{2.010584in}}{\pgfqpoint{3.207121in}{2.004760in}}%
\pgfpathcurveto{\pgfqpoint{3.212945in}{1.998936in}}{\pgfqpoint{3.220845in}{1.995664in}}{\pgfqpoint{3.229081in}{1.995664in}}%
\pgfpathclose%
\pgfusepath{stroke,fill}%
\end{pgfscope}%
\begin{pgfscope}%
\pgfpathrectangle{\pgfqpoint{0.100000in}{0.212622in}}{\pgfqpoint{3.696000in}{3.696000in}}%
\pgfusepath{clip}%
\pgfsetbuttcap%
\pgfsetroundjoin%
\definecolor{currentfill}{rgb}{0.121569,0.466667,0.705882}%
\pgfsetfillcolor{currentfill}%
\pgfsetfillopacity{0.598780}%
\pgfsetlinewidth{1.003750pt}%
\definecolor{currentstroke}{rgb}{0.121569,0.466667,0.705882}%
\pgfsetstrokecolor{currentstroke}%
\pgfsetstrokeopacity{0.598780}%
\pgfsetdash{}{0pt}%
\pgfpathmoveto{\pgfqpoint{3.227112in}{1.993894in}}%
\pgfpathcurveto{\pgfqpoint{3.235349in}{1.993894in}}{\pgfqpoint{3.243249in}{1.997166in}}{\pgfqpoint{3.249073in}{2.002990in}}%
\pgfpathcurveto{\pgfqpoint{3.254897in}{2.008814in}}{\pgfqpoint{3.258169in}{2.016714in}}{\pgfqpoint{3.258169in}{2.024950in}}%
\pgfpathcurveto{\pgfqpoint{3.258169in}{2.033186in}}{\pgfqpoint{3.254897in}{2.041086in}}{\pgfqpoint{3.249073in}{2.046910in}}%
\pgfpathcurveto{\pgfqpoint{3.243249in}{2.052734in}}{\pgfqpoint{3.235349in}{2.056007in}}{\pgfqpoint{3.227112in}{2.056007in}}%
\pgfpathcurveto{\pgfqpoint{3.218876in}{2.056007in}}{\pgfqpoint{3.210976in}{2.052734in}}{\pgfqpoint{3.205152in}{2.046910in}}%
\pgfpathcurveto{\pgfqpoint{3.199328in}{2.041086in}}{\pgfqpoint{3.196056in}{2.033186in}}{\pgfqpoint{3.196056in}{2.024950in}}%
\pgfpathcurveto{\pgfqpoint{3.196056in}{2.016714in}}{\pgfqpoint{3.199328in}{2.008814in}}{\pgfqpoint{3.205152in}{2.002990in}}%
\pgfpathcurveto{\pgfqpoint{3.210976in}{1.997166in}}{\pgfqpoint{3.218876in}{1.993894in}}{\pgfqpoint{3.227112in}{1.993894in}}%
\pgfpathclose%
\pgfusepath{stroke,fill}%
\end{pgfscope}%
\begin{pgfscope}%
\pgfpathrectangle{\pgfqpoint{0.100000in}{0.212622in}}{\pgfqpoint{3.696000in}{3.696000in}}%
\pgfusepath{clip}%
\pgfsetbuttcap%
\pgfsetroundjoin%
\definecolor{currentfill}{rgb}{0.121569,0.466667,0.705882}%
\pgfsetfillcolor{currentfill}%
\pgfsetfillopacity{0.599068}%
\pgfsetlinewidth{1.003750pt}%
\definecolor{currentstroke}{rgb}{0.121569,0.466667,0.705882}%
\pgfsetstrokecolor{currentstroke}%
\pgfsetstrokeopacity{0.599068}%
\pgfsetdash{}{0pt}%
\pgfpathmoveto{\pgfqpoint{0.949488in}{1.684578in}}%
\pgfpathcurveto{\pgfqpoint{0.957725in}{1.684578in}}{\pgfqpoint{0.965625in}{1.687851in}}{\pgfqpoint{0.971449in}{1.693675in}}%
\pgfpathcurveto{\pgfqpoint{0.977273in}{1.699499in}}{\pgfqpoint{0.980545in}{1.707399in}}{\pgfqpoint{0.980545in}{1.715635in}}%
\pgfpathcurveto{\pgfqpoint{0.980545in}{1.723871in}}{\pgfqpoint{0.977273in}{1.731771in}}{\pgfqpoint{0.971449in}{1.737595in}}%
\pgfpathcurveto{\pgfqpoint{0.965625in}{1.743419in}}{\pgfqpoint{0.957725in}{1.746691in}}{\pgfqpoint{0.949488in}{1.746691in}}%
\pgfpathcurveto{\pgfqpoint{0.941252in}{1.746691in}}{\pgfqpoint{0.933352in}{1.743419in}}{\pgfqpoint{0.927528in}{1.737595in}}%
\pgfpathcurveto{\pgfqpoint{0.921704in}{1.731771in}}{\pgfqpoint{0.918432in}{1.723871in}}{\pgfqpoint{0.918432in}{1.715635in}}%
\pgfpathcurveto{\pgfqpoint{0.918432in}{1.707399in}}{\pgfqpoint{0.921704in}{1.699499in}}{\pgfqpoint{0.927528in}{1.693675in}}%
\pgfpathcurveto{\pgfqpoint{0.933352in}{1.687851in}}{\pgfqpoint{0.941252in}{1.684578in}}{\pgfqpoint{0.949488in}{1.684578in}}%
\pgfpathclose%
\pgfusepath{stroke,fill}%
\end{pgfscope}%
\begin{pgfscope}%
\pgfpathrectangle{\pgfqpoint{0.100000in}{0.212622in}}{\pgfqpoint{3.696000in}{3.696000in}}%
\pgfusepath{clip}%
\pgfsetbuttcap%
\pgfsetroundjoin%
\definecolor{currentfill}{rgb}{0.121569,0.466667,0.705882}%
\pgfsetfillcolor{currentfill}%
\pgfsetfillopacity{0.600540}%
\pgfsetlinewidth{1.003750pt}%
\definecolor{currentstroke}{rgb}{0.121569,0.466667,0.705882}%
\pgfsetstrokecolor{currentstroke}%
\pgfsetstrokeopacity{0.600540}%
\pgfsetdash{}{0pt}%
\pgfpathmoveto{\pgfqpoint{3.224016in}{1.992413in}}%
\pgfpathcurveto{\pgfqpoint{3.232253in}{1.992413in}}{\pgfqpoint{3.240153in}{1.995685in}}{\pgfqpoint{3.245977in}{2.001509in}}%
\pgfpathcurveto{\pgfqpoint{3.251801in}{2.007333in}}{\pgfqpoint{3.255073in}{2.015233in}}{\pgfqpoint{3.255073in}{2.023469in}}%
\pgfpathcurveto{\pgfqpoint{3.255073in}{2.031705in}}{\pgfqpoint{3.251801in}{2.039605in}}{\pgfqpoint{3.245977in}{2.045429in}}%
\pgfpathcurveto{\pgfqpoint{3.240153in}{2.051253in}}{\pgfqpoint{3.232253in}{2.054526in}}{\pgfqpoint{3.224016in}{2.054526in}}%
\pgfpathcurveto{\pgfqpoint{3.215780in}{2.054526in}}{\pgfqpoint{3.207880in}{2.051253in}}{\pgfqpoint{3.202056in}{2.045429in}}%
\pgfpathcurveto{\pgfqpoint{3.196232in}{2.039605in}}{\pgfqpoint{3.192960in}{2.031705in}}{\pgfqpoint{3.192960in}{2.023469in}}%
\pgfpathcurveto{\pgfqpoint{3.192960in}{2.015233in}}{\pgfqpoint{3.196232in}{2.007333in}}{\pgfqpoint{3.202056in}{2.001509in}}%
\pgfpathcurveto{\pgfqpoint{3.207880in}{1.995685in}}{\pgfqpoint{3.215780in}{1.992413in}}{\pgfqpoint{3.224016in}{1.992413in}}%
\pgfpathclose%
\pgfusepath{stroke,fill}%
\end{pgfscope}%
\begin{pgfscope}%
\pgfpathrectangle{\pgfqpoint{0.100000in}{0.212622in}}{\pgfqpoint{3.696000in}{3.696000in}}%
\pgfusepath{clip}%
\pgfsetbuttcap%
\pgfsetroundjoin%
\definecolor{currentfill}{rgb}{0.121569,0.466667,0.705882}%
\pgfsetfillcolor{currentfill}%
\pgfsetfillopacity{0.600829}%
\pgfsetlinewidth{1.003750pt}%
\definecolor{currentstroke}{rgb}{0.121569,0.466667,0.705882}%
\pgfsetstrokecolor{currentstroke}%
\pgfsetstrokeopacity{0.600829}%
\pgfsetdash{}{0pt}%
\pgfpathmoveto{\pgfqpoint{0.942113in}{1.681165in}}%
\pgfpathcurveto{\pgfqpoint{0.950350in}{1.681165in}}{\pgfqpoint{0.958250in}{1.684437in}}{\pgfqpoint{0.964074in}{1.690261in}}%
\pgfpathcurveto{\pgfqpoint{0.969898in}{1.696085in}}{\pgfqpoint{0.973170in}{1.703985in}}{\pgfqpoint{0.973170in}{1.712221in}}%
\pgfpathcurveto{\pgfqpoint{0.973170in}{1.720458in}}{\pgfqpoint{0.969898in}{1.728358in}}{\pgfqpoint{0.964074in}{1.734182in}}%
\pgfpathcurveto{\pgfqpoint{0.958250in}{1.740006in}}{\pgfqpoint{0.950350in}{1.743278in}}{\pgfqpoint{0.942113in}{1.743278in}}%
\pgfpathcurveto{\pgfqpoint{0.933877in}{1.743278in}}{\pgfqpoint{0.925977in}{1.740006in}}{\pgfqpoint{0.920153in}{1.734182in}}%
\pgfpathcurveto{\pgfqpoint{0.914329in}{1.728358in}}{\pgfqpoint{0.911057in}{1.720458in}}{\pgfqpoint{0.911057in}{1.712221in}}%
\pgfpathcurveto{\pgfqpoint{0.911057in}{1.703985in}}{\pgfqpoint{0.914329in}{1.696085in}}{\pgfqpoint{0.920153in}{1.690261in}}%
\pgfpathcurveto{\pgfqpoint{0.925977in}{1.684437in}}{\pgfqpoint{0.933877in}{1.681165in}}{\pgfqpoint{0.942113in}{1.681165in}}%
\pgfpathclose%
\pgfusepath{stroke,fill}%
\end{pgfscope}%
\begin{pgfscope}%
\pgfpathrectangle{\pgfqpoint{0.100000in}{0.212622in}}{\pgfqpoint{3.696000in}{3.696000in}}%
\pgfusepath{clip}%
\pgfsetbuttcap%
\pgfsetroundjoin%
\definecolor{currentfill}{rgb}{0.121569,0.466667,0.705882}%
\pgfsetfillcolor{currentfill}%
\pgfsetfillopacity{0.602725}%
\pgfsetlinewidth{1.003750pt}%
\definecolor{currentstroke}{rgb}{0.121569,0.466667,0.705882}%
\pgfsetstrokecolor{currentstroke}%
\pgfsetstrokeopacity{0.602725}%
\pgfsetdash{}{0pt}%
\pgfpathmoveto{\pgfqpoint{3.218381in}{1.993146in}}%
\pgfpathcurveto{\pgfqpoint{3.226618in}{1.993146in}}{\pgfqpoint{3.234518in}{1.996418in}}{\pgfqpoint{3.240342in}{2.002242in}}%
\pgfpathcurveto{\pgfqpoint{3.246166in}{2.008066in}}{\pgfqpoint{3.249438in}{2.015966in}}{\pgfqpoint{3.249438in}{2.024202in}}%
\pgfpathcurveto{\pgfqpoint{3.249438in}{2.032438in}}{\pgfqpoint{3.246166in}{2.040338in}}{\pgfqpoint{3.240342in}{2.046162in}}%
\pgfpathcurveto{\pgfqpoint{3.234518in}{2.051986in}}{\pgfqpoint{3.226618in}{2.055259in}}{\pgfqpoint{3.218381in}{2.055259in}}%
\pgfpathcurveto{\pgfqpoint{3.210145in}{2.055259in}}{\pgfqpoint{3.202245in}{2.051986in}}{\pgfqpoint{3.196421in}{2.046162in}}%
\pgfpathcurveto{\pgfqpoint{3.190597in}{2.040338in}}{\pgfqpoint{3.187325in}{2.032438in}}{\pgfqpoint{3.187325in}{2.024202in}}%
\pgfpathcurveto{\pgfqpoint{3.187325in}{2.015966in}}{\pgfqpoint{3.190597in}{2.008066in}}{\pgfqpoint{3.196421in}{2.002242in}}%
\pgfpathcurveto{\pgfqpoint{3.202245in}{1.996418in}}{\pgfqpoint{3.210145in}{1.993146in}}{\pgfqpoint{3.218381in}{1.993146in}}%
\pgfpathclose%
\pgfusepath{stroke,fill}%
\end{pgfscope}%
\begin{pgfscope}%
\pgfpathrectangle{\pgfqpoint{0.100000in}{0.212622in}}{\pgfqpoint{3.696000in}{3.696000in}}%
\pgfusepath{clip}%
\pgfsetbuttcap%
\pgfsetroundjoin%
\definecolor{currentfill}{rgb}{0.121569,0.466667,0.705882}%
\pgfsetfillcolor{currentfill}%
\pgfsetfillopacity{0.604480}%
\pgfsetlinewidth{1.003750pt}%
\definecolor{currentstroke}{rgb}{0.121569,0.466667,0.705882}%
\pgfsetstrokecolor{currentstroke}%
\pgfsetstrokeopacity{0.604480}%
\pgfsetdash{}{0pt}%
\pgfpathmoveto{\pgfqpoint{3.214589in}{1.986990in}}%
\pgfpathcurveto{\pgfqpoint{3.222826in}{1.986990in}}{\pgfqpoint{3.230726in}{1.990262in}}{\pgfqpoint{3.236550in}{1.996086in}}%
\pgfpathcurveto{\pgfqpoint{3.242374in}{2.001910in}}{\pgfqpoint{3.245646in}{2.009810in}}{\pgfqpoint{3.245646in}{2.018046in}}%
\pgfpathcurveto{\pgfqpoint{3.245646in}{2.026283in}}{\pgfqpoint{3.242374in}{2.034183in}}{\pgfqpoint{3.236550in}{2.040007in}}%
\pgfpathcurveto{\pgfqpoint{3.230726in}{2.045830in}}{\pgfqpoint{3.222826in}{2.049103in}}{\pgfqpoint{3.214589in}{2.049103in}}%
\pgfpathcurveto{\pgfqpoint{3.206353in}{2.049103in}}{\pgfqpoint{3.198453in}{2.045830in}}{\pgfqpoint{3.192629in}{2.040007in}}%
\pgfpathcurveto{\pgfqpoint{3.186805in}{2.034183in}}{\pgfqpoint{3.183533in}{2.026283in}}{\pgfqpoint{3.183533in}{2.018046in}}%
\pgfpathcurveto{\pgfqpoint{3.183533in}{2.009810in}}{\pgfqpoint{3.186805in}{2.001910in}}{\pgfqpoint{3.192629in}{1.996086in}}%
\pgfpathcurveto{\pgfqpoint{3.198453in}{1.990262in}}{\pgfqpoint{3.206353in}{1.986990in}}{\pgfqpoint{3.214589in}{1.986990in}}%
\pgfpathclose%
\pgfusepath{stroke,fill}%
\end{pgfscope}%
\begin{pgfscope}%
\pgfpathrectangle{\pgfqpoint{0.100000in}{0.212622in}}{\pgfqpoint{3.696000in}{3.696000in}}%
\pgfusepath{clip}%
\pgfsetbuttcap%
\pgfsetroundjoin%
\definecolor{currentfill}{rgb}{0.121569,0.466667,0.705882}%
\pgfsetfillcolor{currentfill}%
\pgfsetfillopacity{0.604759}%
\pgfsetlinewidth{1.003750pt}%
\definecolor{currentstroke}{rgb}{0.121569,0.466667,0.705882}%
\pgfsetstrokecolor{currentstroke}%
\pgfsetstrokeopacity{0.604759}%
\pgfsetdash{}{0pt}%
\pgfpathmoveto{\pgfqpoint{0.938451in}{1.689422in}}%
\pgfpathcurveto{\pgfqpoint{0.946687in}{1.689422in}}{\pgfqpoint{0.954587in}{1.692695in}}{\pgfqpoint{0.960411in}{1.698519in}}%
\pgfpathcurveto{\pgfqpoint{0.966235in}{1.704343in}}{\pgfqpoint{0.969507in}{1.712243in}}{\pgfqpoint{0.969507in}{1.720479in}}%
\pgfpathcurveto{\pgfqpoint{0.969507in}{1.728715in}}{\pgfqpoint{0.966235in}{1.736615in}}{\pgfqpoint{0.960411in}{1.742439in}}%
\pgfpathcurveto{\pgfqpoint{0.954587in}{1.748263in}}{\pgfqpoint{0.946687in}{1.751535in}}{\pgfqpoint{0.938451in}{1.751535in}}%
\pgfpathcurveto{\pgfqpoint{0.930214in}{1.751535in}}{\pgfqpoint{0.922314in}{1.748263in}}{\pgfqpoint{0.916490in}{1.742439in}}%
\pgfpathcurveto{\pgfqpoint{0.910667in}{1.736615in}}{\pgfqpoint{0.907394in}{1.728715in}}{\pgfqpoint{0.907394in}{1.720479in}}%
\pgfpathcurveto{\pgfqpoint{0.907394in}{1.712243in}}{\pgfqpoint{0.910667in}{1.704343in}}{\pgfqpoint{0.916490in}{1.698519in}}%
\pgfpathcurveto{\pgfqpoint{0.922314in}{1.692695in}}{\pgfqpoint{0.930214in}{1.689422in}}{\pgfqpoint{0.938451in}{1.689422in}}%
\pgfpathclose%
\pgfusepath{stroke,fill}%
\end{pgfscope}%
\begin{pgfscope}%
\pgfpathrectangle{\pgfqpoint{0.100000in}{0.212622in}}{\pgfqpoint{3.696000in}{3.696000in}}%
\pgfusepath{clip}%
\pgfsetbuttcap%
\pgfsetroundjoin%
\definecolor{currentfill}{rgb}{0.121569,0.466667,0.705882}%
\pgfsetfillcolor{currentfill}%
\pgfsetfillopacity{0.606090}%
\pgfsetlinewidth{1.003750pt}%
\definecolor{currentstroke}{rgb}{0.121569,0.466667,0.705882}%
\pgfsetstrokecolor{currentstroke}%
\pgfsetstrokeopacity{0.606090}%
\pgfsetdash{}{0pt}%
\pgfpathmoveto{\pgfqpoint{0.932321in}{1.688113in}}%
\pgfpathcurveto{\pgfqpoint{0.940557in}{1.688113in}}{\pgfqpoint{0.948457in}{1.691385in}}{\pgfqpoint{0.954281in}{1.697209in}}%
\pgfpathcurveto{\pgfqpoint{0.960105in}{1.703033in}}{\pgfqpoint{0.963377in}{1.710933in}}{\pgfqpoint{0.963377in}{1.719170in}}%
\pgfpathcurveto{\pgfqpoint{0.963377in}{1.727406in}}{\pgfqpoint{0.960105in}{1.735306in}}{\pgfqpoint{0.954281in}{1.741130in}}%
\pgfpathcurveto{\pgfqpoint{0.948457in}{1.746954in}}{\pgfqpoint{0.940557in}{1.750226in}}{\pgfqpoint{0.932321in}{1.750226in}}%
\pgfpathcurveto{\pgfqpoint{0.924084in}{1.750226in}}{\pgfqpoint{0.916184in}{1.746954in}}{\pgfqpoint{0.910360in}{1.741130in}}%
\pgfpathcurveto{\pgfqpoint{0.904536in}{1.735306in}}{\pgfqpoint{0.901264in}{1.727406in}}{\pgfqpoint{0.901264in}{1.719170in}}%
\pgfpathcurveto{\pgfqpoint{0.901264in}{1.710933in}}{\pgfqpoint{0.904536in}{1.703033in}}{\pgfqpoint{0.910360in}{1.697209in}}%
\pgfpathcurveto{\pgfqpoint{0.916184in}{1.691385in}}{\pgfqpoint{0.924084in}{1.688113in}}{\pgfqpoint{0.932321in}{1.688113in}}%
\pgfpathclose%
\pgfusepath{stroke,fill}%
\end{pgfscope}%
\begin{pgfscope}%
\pgfpathrectangle{\pgfqpoint{0.100000in}{0.212622in}}{\pgfqpoint{3.696000in}{3.696000in}}%
\pgfusepath{clip}%
\pgfsetbuttcap%
\pgfsetroundjoin%
\definecolor{currentfill}{rgb}{0.121569,0.466667,0.705882}%
\pgfsetfillcolor{currentfill}%
\pgfsetfillopacity{0.606789}%
\pgfsetlinewidth{1.003750pt}%
\definecolor{currentstroke}{rgb}{0.121569,0.466667,0.705882}%
\pgfsetstrokecolor{currentstroke}%
\pgfsetstrokeopacity{0.606789}%
\pgfsetdash{}{0pt}%
\pgfpathmoveto{\pgfqpoint{3.210982in}{1.983874in}}%
\pgfpathcurveto{\pgfqpoint{3.219218in}{1.983874in}}{\pgfqpoint{3.227118in}{1.987147in}}{\pgfqpoint{3.232942in}{1.992970in}}%
\pgfpathcurveto{\pgfqpoint{3.238766in}{1.998794in}}{\pgfqpoint{3.242038in}{2.006694in}}{\pgfqpoint{3.242038in}{2.014931in}}%
\pgfpathcurveto{\pgfqpoint{3.242038in}{2.023167in}}{\pgfqpoint{3.238766in}{2.031067in}}{\pgfqpoint{3.232942in}{2.036891in}}%
\pgfpathcurveto{\pgfqpoint{3.227118in}{2.042715in}}{\pgfqpoint{3.219218in}{2.045987in}}{\pgfqpoint{3.210982in}{2.045987in}}%
\pgfpathcurveto{\pgfqpoint{3.202746in}{2.045987in}}{\pgfqpoint{3.194846in}{2.042715in}}{\pgfqpoint{3.189022in}{2.036891in}}%
\pgfpathcurveto{\pgfqpoint{3.183198in}{2.031067in}}{\pgfqpoint{3.179925in}{2.023167in}}{\pgfqpoint{3.179925in}{2.014931in}}%
\pgfpathcurveto{\pgfqpoint{3.179925in}{2.006694in}}{\pgfqpoint{3.183198in}{1.998794in}}{\pgfqpoint{3.189022in}{1.992970in}}%
\pgfpathcurveto{\pgfqpoint{3.194846in}{1.987147in}}{\pgfqpoint{3.202746in}{1.983874in}}{\pgfqpoint{3.210982in}{1.983874in}}%
\pgfpathclose%
\pgfusepath{stroke,fill}%
\end{pgfscope}%
\begin{pgfscope}%
\pgfpathrectangle{\pgfqpoint{0.100000in}{0.212622in}}{\pgfqpoint{3.696000in}{3.696000in}}%
\pgfusepath{clip}%
\pgfsetbuttcap%
\pgfsetroundjoin%
\definecolor{currentfill}{rgb}{0.121569,0.466667,0.705882}%
\pgfsetfillcolor{currentfill}%
\pgfsetfillopacity{0.606900}%
\pgfsetlinewidth{1.003750pt}%
\definecolor{currentstroke}{rgb}{0.121569,0.466667,0.705882}%
\pgfsetstrokecolor{currentstroke}%
\pgfsetstrokeopacity{0.606900}%
\pgfsetdash{}{0pt}%
\pgfpathmoveto{\pgfqpoint{0.930250in}{1.685076in}}%
\pgfpathcurveto{\pgfqpoint{0.938487in}{1.685076in}}{\pgfqpoint{0.946387in}{1.688348in}}{\pgfqpoint{0.952211in}{1.694172in}}%
\pgfpathcurveto{\pgfqpoint{0.958035in}{1.699996in}}{\pgfqpoint{0.961307in}{1.707896in}}{\pgfqpoint{0.961307in}{1.716132in}}%
\pgfpathcurveto{\pgfqpoint{0.961307in}{1.724369in}}{\pgfqpoint{0.958035in}{1.732269in}}{\pgfqpoint{0.952211in}{1.738093in}}%
\pgfpathcurveto{\pgfqpoint{0.946387in}{1.743917in}}{\pgfqpoint{0.938487in}{1.747189in}}{\pgfqpoint{0.930250in}{1.747189in}}%
\pgfpathcurveto{\pgfqpoint{0.922014in}{1.747189in}}{\pgfqpoint{0.914114in}{1.743917in}}{\pgfqpoint{0.908290in}{1.738093in}}%
\pgfpathcurveto{\pgfqpoint{0.902466in}{1.732269in}}{\pgfqpoint{0.899194in}{1.724369in}}{\pgfqpoint{0.899194in}{1.716132in}}%
\pgfpathcurveto{\pgfqpoint{0.899194in}{1.707896in}}{\pgfqpoint{0.902466in}{1.699996in}}{\pgfqpoint{0.908290in}{1.694172in}}%
\pgfpathcurveto{\pgfqpoint{0.914114in}{1.688348in}}{\pgfqpoint{0.922014in}{1.685076in}}{\pgfqpoint{0.930250in}{1.685076in}}%
\pgfpathclose%
\pgfusepath{stroke,fill}%
\end{pgfscope}%
\begin{pgfscope}%
\pgfpathrectangle{\pgfqpoint{0.100000in}{0.212622in}}{\pgfqpoint{3.696000in}{3.696000in}}%
\pgfusepath{clip}%
\pgfsetbuttcap%
\pgfsetroundjoin%
\definecolor{currentfill}{rgb}{0.121569,0.466667,0.705882}%
\pgfsetfillcolor{currentfill}%
\pgfsetfillopacity{0.608166}%
\pgfsetlinewidth{1.003750pt}%
\definecolor{currentstroke}{rgb}{0.121569,0.466667,0.705882}%
\pgfsetstrokecolor{currentstroke}%
\pgfsetstrokeopacity{0.608166}%
\pgfsetdash{}{0pt}%
\pgfpathmoveto{\pgfqpoint{0.926270in}{1.678325in}}%
\pgfpathcurveto{\pgfqpoint{0.934506in}{1.678325in}}{\pgfqpoint{0.942406in}{1.681597in}}{\pgfqpoint{0.948230in}{1.687421in}}%
\pgfpathcurveto{\pgfqpoint{0.954054in}{1.693245in}}{\pgfqpoint{0.957326in}{1.701145in}}{\pgfqpoint{0.957326in}{1.709381in}}%
\pgfpathcurveto{\pgfqpoint{0.957326in}{1.717617in}}{\pgfqpoint{0.954054in}{1.725517in}}{\pgfqpoint{0.948230in}{1.731341in}}%
\pgfpathcurveto{\pgfqpoint{0.942406in}{1.737165in}}{\pgfqpoint{0.934506in}{1.740438in}}{\pgfqpoint{0.926270in}{1.740438in}}%
\pgfpathcurveto{\pgfqpoint{0.918034in}{1.740438in}}{\pgfqpoint{0.910133in}{1.737165in}}{\pgfqpoint{0.904310in}{1.731341in}}%
\pgfpathcurveto{\pgfqpoint{0.898486in}{1.725517in}}{\pgfqpoint{0.895213in}{1.717617in}}{\pgfqpoint{0.895213in}{1.709381in}}%
\pgfpathcurveto{\pgfqpoint{0.895213in}{1.701145in}}{\pgfqpoint{0.898486in}{1.693245in}}{\pgfqpoint{0.904310in}{1.687421in}}%
\pgfpathcurveto{\pgfqpoint{0.910133in}{1.681597in}}{\pgfqpoint{0.918034in}{1.678325in}}{\pgfqpoint{0.926270in}{1.678325in}}%
\pgfpathclose%
\pgfusepath{stroke,fill}%
\end{pgfscope}%
\begin{pgfscope}%
\pgfpathrectangle{\pgfqpoint{0.100000in}{0.212622in}}{\pgfqpoint{3.696000in}{3.696000in}}%
\pgfusepath{clip}%
\pgfsetbuttcap%
\pgfsetroundjoin%
\definecolor{currentfill}{rgb}{0.121569,0.466667,0.705882}%
\pgfsetfillcolor{currentfill}%
\pgfsetfillopacity{0.609593}%
\pgfsetlinewidth{1.003750pt}%
\definecolor{currentstroke}{rgb}{0.121569,0.466667,0.705882}%
\pgfsetstrokecolor{currentstroke}%
\pgfsetstrokeopacity{0.609593}%
\pgfsetdash{}{0pt}%
\pgfpathmoveto{\pgfqpoint{0.920877in}{1.675089in}}%
\pgfpathcurveto{\pgfqpoint{0.929113in}{1.675089in}}{\pgfqpoint{0.937013in}{1.678362in}}{\pgfqpoint{0.942837in}{1.684186in}}%
\pgfpathcurveto{\pgfqpoint{0.948661in}{1.690010in}}{\pgfqpoint{0.951933in}{1.697910in}}{\pgfqpoint{0.951933in}{1.706146in}}%
\pgfpathcurveto{\pgfqpoint{0.951933in}{1.714382in}}{\pgfqpoint{0.948661in}{1.722282in}}{\pgfqpoint{0.942837in}{1.728106in}}%
\pgfpathcurveto{\pgfqpoint{0.937013in}{1.733930in}}{\pgfqpoint{0.929113in}{1.737202in}}{\pgfqpoint{0.920877in}{1.737202in}}%
\pgfpathcurveto{\pgfqpoint{0.912641in}{1.737202in}}{\pgfqpoint{0.904740in}{1.733930in}}{\pgfqpoint{0.898917in}{1.728106in}}%
\pgfpathcurveto{\pgfqpoint{0.893093in}{1.722282in}}{\pgfqpoint{0.889820in}{1.714382in}}{\pgfqpoint{0.889820in}{1.706146in}}%
\pgfpathcurveto{\pgfqpoint{0.889820in}{1.697910in}}{\pgfqpoint{0.893093in}{1.690010in}}{\pgfqpoint{0.898917in}{1.684186in}}%
\pgfpathcurveto{\pgfqpoint{0.904740in}{1.678362in}}{\pgfqpoint{0.912641in}{1.675089in}}{\pgfqpoint{0.920877in}{1.675089in}}%
\pgfpathclose%
\pgfusepath{stroke,fill}%
\end{pgfscope}%
\begin{pgfscope}%
\pgfpathrectangle{\pgfqpoint{0.100000in}{0.212622in}}{\pgfqpoint{3.696000in}{3.696000in}}%
\pgfusepath{clip}%
\pgfsetbuttcap%
\pgfsetroundjoin%
\definecolor{currentfill}{rgb}{0.121569,0.466667,0.705882}%
\pgfsetfillcolor{currentfill}%
\pgfsetfillopacity{0.609795}%
\pgfsetlinewidth{1.003750pt}%
\definecolor{currentstroke}{rgb}{0.121569,0.466667,0.705882}%
\pgfsetstrokecolor{currentstroke}%
\pgfsetstrokeopacity{0.609795}%
\pgfsetdash{}{0pt}%
\pgfpathmoveto{\pgfqpoint{3.203867in}{1.983496in}}%
\pgfpathcurveto{\pgfqpoint{3.212103in}{1.983496in}}{\pgfqpoint{3.220003in}{1.986768in}}{\pgfqpoint{3.225827in}{1.992592in}}%
\pgfpathcurveto{\pgfqpoint{3.231651in}{1.998416in}}{\pgfqpoint{3.234923in}{2.006316in}}{\pgfqpoint{3.234923in}{2.014552in}}%
\pgfpathcurveto{\pgfqpoint{3.234923in}{2.022788in}}{\pgfqpoint{3.231651in}{2.030688in}}{\pgfqpoint{3.225827in}{2.036512in}}%
\pgfpathcurveto{\pgfqpoint{3.220003in}{2.042336in}}{\pgfqpoint{3.212103in}{2.045609in}}{\pgfqpoint{3.203867in}{2.045609in}}%
\pgfpathcurveto{\pgfqpoint{3.195630in}{2.045609in}}{\pgfqpoint{3.187730in}{2.042336in}}{\pgfqpoint{3.181906in}{2.036512in}}%
\pgfpathcurveto{\pgfqpoint{3.176082in}{2.030688in}}{\pgfqpoint{3.172810in}{2.022788in}}{\pgfqpoint{3.172810in}{2.014552in}}%
\pgfpathcurveto{\pgfqpoint{3.172810in}{2.006316in}}{\pgfqpoint{3.176082in}{1.998416in}}{\pgfqpoint{3.181906in}{1.992592in}}%
\pgfpathcurveto{\pgfqpoint{3.187730in}{1.986768in}}{\pgfqpoint{3.195630in}{1.983496in}}{\pgfqpoint{3.203867in}{1.983496in}}%
\pgfpathclose%
\pgfusepath{stroke,fill}%
\end{pgfscope}%
\begin{pgfscope}%
\pgfpathrectangle{\pgfqpoint{0.100000in}{0.212622in}}{\pgfqpoint{3.696000in}{3.696000in}}%
\pgfusepath{clip}%
\pgfsetbuttcap%
\pgfsetroundjoin%
\definecolor{currentfill}{rgb}{0.121569,0.466667,0.705882}%
\pgfsetfillcolor{currentfill}%
\pgfsetfillopacity{0.611096}%
\pgfsetlinewidth{1.003750pt}%
\definecolor{currentstroke}{rgb}{0.121569,0.466667,0.705882}%
\pgfsetstrokecolor{currentstroke}%
\pgfsetstrokeopacity{0.611096}%
\pgfsetdash{}{0pt}%
\pgfpathmoveto{\pgfqpoint{0.919164in}{1.673633in}}%
\pgfpathcurveto{\pgfqpoint{0.927400in}{1.673633in}}{\pgfqpoint{0.935300in}{1.676906in}}{\pgfqpoint{0.941124in}{1.682730in}}%
\pgfpathcurveto{\pgfqpoint{0.946948in}{1.688554in}}{\pgfqpoint{0.950220in}{1.696454in}}{\pgfqpoint{0.950220in}{1.704690in}}%
\pgfpathcurveto{\pgfqpoint{0.950220in}{1.712926in}}{\pgfqpoint{0.946948in}{1.720826in}}{\pgfqpoint{0.941124in}{1.726650in}}%
\pgfpathcurveto{\pgfqpoint{0.935300in}{1.732474in}}{\pgfqpoint{0.927400in}{1.735746in}}{\pgfqpoint{0.919164in}{1.735746in}}%
\pgfpathcurveto{\pgfqpoint{0.910927in}{1.735746in}}{\pgfqpoint{0.903027in}{1.732474in}}{\pgfqpoint{0.897203in}{1.726650in}}%
\pgfpathcurveto{\pgfqpoint{0.891379in}{1.720826in}}{\pgfqpoint{0.888107in}{1.712926in}}{\pgfqpoint{0.888107in}{1.704690in}}%
\pgfpathcurveto{\pgfqpoint{0.888107in}{1.696454in}}{\pgfqpoint{0.891379in}{1.688554in}}{\pgfqpoint{0.897203in}{1.682730in}}%
\pgfpathcurveto{\pgfqpoint{0.903027in}{1.676906in}}{\pgfqpoint{0.910927in}{1.673633in}}{\pgfqpoint{0.919164in}{1.673633in}}%
\pgfpathclose%
\pgfusepath{stroke,fill}%
\end{pgfscope}%
\begin{pgfscope}%
\pgfpathrectangle{\pgfqpoint{0.100000in}{0.212622in}}{\pgfqpoint{3.696000in}{3.696000in}}%
\pgfusepath{clip}%
\pgfsetbuttcap%
\pgfsetroundjoin%
\definecolor{currentfill}{rgb}{0.121569,0.466667,0.705882}%
\pgfsetfillcolor{currentfill}%
\pgfsetfillopacity{0.611866}%
\pgfsetlinewidth{1.003750pt}%
\definecolor{currentstroke}{rgb}{0.121569,0.466667,0.705882}%
\pgfsetstrokecolor{currentstroke}%
\pgfsetstrokeopacity{0.611866}%
\pgfsetdash{}{0pt}%
\pgfpathmoveto{\pgfqpoint{0.914878in}{1.671137in}}%
\pgfpathcurveto{\pgfqpoint{0.923115in}{1.671137in}}{\pgfqpoint{0.931015in}{1.674410in}}{\pgfqpoint{0.936839in}{1.680233in}}%
\pgfpathcurveto{\pgfqpoint{0.942662in}{1.686057in}}{\pgfqpoint{0.945935in}{1.693957in}}{\pgfqpoint{0.945935in}{1.702194in}}%
\pgfpathcurveto{\pgfqpoint{0.945935in}{1.710430in}}{\pgfqpoint{0.942662in}{1.718330in}}{\pgfqpoint{0.936839in}{1.724154in}}%
\pgfpathcurveto{\pgfqpoint{0.931015in}{1.729978in}}{\pgfqpoint{0.923115in}{1.733250in}}{\pgfqpoint{0.914878in}{1.733250in}}%
\pgfpathcurveto{\pgfqpoint{0.906642in}{1.733250in}}{\pgfqpoint{0.898742in}{1.729978in}}{\pgfqpoint{0.892918in}{1.724154in}}%
\pgfpathcurveto{\pgfqpoint{0.887094in}{1.718330in}}{\pgfqpoint{0.883822in}{1.710430in}}{\pgfqpoint{0.883822in}{1.702194in}}%
\pgfpathcurveto{\pgfqpoint{0.883822in}{1.693957in}}{\pgfqpoint{0.887094in}{1.686057in}}{\pgfqpoint{0.892918in}{1.680233in}}%
\pgfpathcurveto{\pgfqpoint{0.898742in}{1.674410in}}{\pgfqpoint{0.906642in}{1.671137in}}{\pgfqpoint{0.914878in}{1.671137in}}%
\pgfpathclose%
\pgfusepath{stroke,fill}%
\end{pgfscope}%
\begin{pgfscope}%
\pgfpathrectangle{\pgfqpoint{0.100000in}{0.212622in}}{\pgfqpoint{3.696000in}{3.696000in}}%
\pgfusepath{clip}%
\pgfsetbuttcap%
\pgfsetroundjoin%
\definecolor{currentfill}{rgb}{0.121569,0.466667,0.705882}%
\pgfsetfillcolor{currentfill}%
\pgfsetfillopacity{0.612254}%
\pgfsetlinewidth{1.003750pt}%
\definecolor{currentstroke}{rgb}{0.121569,0.466667,0.705882}%
\pgfsetstrokecolor{currentstroke}%
\pgfsetstrokeopacity{0.612254}%
\pgfsetdash{}{0pt}%
\pgfpathmoveto{\pgfqpoint{3.196292in}{1.978852in}}%
\pgfpathcurveto{\pgfqpoint{3.204528in}{1.978852in}}{\pgfqpoint{3.212428in}{1.982124in}}{\pgfqpoint{3.218252in}{1.987948in}}%
\pgfpathcurveto{\pgfqpoint{3.224076in}{1.993772in}}{\pgfqpoint{3.227348in}{2.001672in}}{\pgfqpoint{3.227348in}{2.009908in}}%
\pgfpathcurveto{\pgfqpoint{3.227348in}{2.018145in}}{\pgfqpoint{3.224076in}{2.026045in}}{\pgfqpoint{3.218252in}{2.031869in}}%
\pgfpathcurveto{\pgfqpoint{3.212428in}{2.037693in}}{\pgfqpoint{3.204528in}{2.040965in}}{\pgfqpoint{3.196292in}{2.040965in}}%
\pgfpathcurveto{\pgfqpoint{3.188055in}{2.040965in}}{\pgfqpoint{3.180155in}{2.037693in}}{\pgfqpoint{3.174331in}{2.031869in}}%
\pgfpathcurveto{\pgfqpoint{3.168507in}{2.026045in}}{\pgfqpoint{3.165235in}{2.018145in}}{\pgfqpoint{3.165235in}{2.009908in}}%
\pgfpathcurveto{\pgfqpoint{3.165235in}{2.001672in}}{\pgfqpoint{3.168507in}{1.993772in}}{\pgfqpoint{3.174331in}{1.987948in}}%
\pgfpathcurveto{\pgfqpoint{3.180155in}{1.982124in}}{\pgfqpoint{3.188055in}{1.978852in}}{\pgfqpoint{3.196292in}{1.978852in}}%
\pgfpathclose%
\pgfusepath{stroke,fill}%
\end{pgfscope}%
\begin{pgfscope}%
\pgfpathrectangle{\pgfqpoint{0.100000in}{0.212622in}}{\pgfqpoint{3.696000in}{3.696000in}}%
\pgfusepath{clip}%
\pgfsetbuttcap%
\pgfsetroundjoin%
\definecolor{currentfill}{rgb}{0.121569,0.466667,0.705882}%
\pgfsetfillcolor{currentfill}%
\pgfsetfillopacity{0.612838}%
\pgfsetlinewidth{1.003750pt}%
\definecolor{currentstroke}{rgb}{0.121569,0.466667,0.705882}%
\pgfsetstrokecolor{currentstroke}%
\pgfsetstrokeopacity{0.612838}%
\pgfsetdash{}{0pt}%
\pgfpathmoveto{\pgfqpoint{0.912856in}{1.670837in}}%
\pgfpathcurveto{\pgfqpoint{0.921093in}{1.670837in}}{\pgfqpoint{0.928993in}{1.674110in}}{\pgfqpoint{0.934817in}{1.679934in}}%
\pgfpathcurveto{\pgfqpoint{0.940640in}{1.685757in}}{\pgfqpoint{0.943913in}{1.693658in}}{\pgfqpoint{0.943913in}{1.701894in}}%
\pgfpathcurveto{\pgfqpoint{0.943913in}{1.710130in}}{\pgfqpoint{0.940640in}{1.718030in}}{\pgfqpoint{0.934817in}{1.723854in}}%
\pgfpathcurveto{\pgfqpoint{0.928993in}{1.729678in}}{\pgfqpoint{0.921093in}{1.732950in}}{\pgfqpoint{0.912856in}{1.732950in}}%
\pgfpathcurveto{\pgfqpoint{0.904620in}{1.732950in}}{\pgfqpoint{0.896720in}{1.729678in}}{\pgfqpoint{0.890896in}{1.723854in}}%
\pgfpathcurveto{\pgfqpoint{0.885072in}{1.718030in}}{\pgfqpoint{0.881800in}{1.710130in}}{\pgfqpoint{0.881800in}{1.701894in}}%
\pgfpathcurveto{\pgfqpoint{0.881800in}{1.693658in}}{\pgfqpoint{0.885072in}{1.685757in}}{\pgfqpoint{0.890896in}{1.679934in}}%
\pgfpathcurveto{\pgfqpoint{0.896720in}{1.674110in}}{\pgfqpoint{0.904620in}{1.670837in}}{\pgfqpoint{0.912856in}{1.670837in}}%
\pgfpathclose%
\pgfusepath{stroke,fill}%
\end{pgfscope}%
\begin{pgfscope}%
\pgfpathrectangle{\pgfqpoint{0.100000in}{0.212622in}}{\pgfqpoint{3.696000in}{3.696000in}}%
\pgfusepath{clip}%
\pgfsetbuttcap%
\pgfsetroundjoin%
\definecolor{currentfill}{rgb}{0.121569,0.466667,0.705882}%
\pgfsetfillcolor{currentfill}%
\pgfsetfillopacity{0.614604}%
\pgfsetlinewidth{1.003750pt}%
\definecolor{currentstroke}{rgb}{0.121569,0.466667,0.705882}%
\pgfsetstrokecolor{currentstroke}%
\pgfsetstrokeopacity{0.614604}%
\pgfsetdash{}{0pt}%
\pgfpathmoveto{\pgfqpoint{0.908352in}{1.671256in}}%
\pgfpathcurveto{\pgfqpoint{0.916589in}{1.671256in}}{\pgfqpoint{0.924489in}{1.674528in}}{\pgfqpoint{0.930313in}{1.680352in}}%
\pgfpathcurveto{\pgfqpoint{0.936137in}{1.686176in}}{\pgfqpoint{0.939409in}{1.694076in}}{\pgfqpoint{0.939409in}{1.702312in}}%
\pgfpathcurveto{\pgfqpoint{0.939409in}{1.710549in}}{\pgfqpoint{0.936137in}{1.718449in}}{\pgfqpoint{0.930313in}{1.724273in}}%
\pgfpathcurveto{\pgfqpoint{0.924489in}{1.730097in}}{\pgfqpoint{0.916589in}{1.733369in}}{\pgfqpoint{0.908352in}{1.733369in}}%
\pgfpathcurveto{\pgfqpoint{0.900116in}{1.733369in}}{\pgfqpoint{0.892216in}{1.730097in}}{\pgfqpoint{0.886392in}{1.724273in}}%
\pgfpathcurveto{\pgfqpoint{0.880568in}{1.718449in}}{\pgfqpoint{0.877296in}{1.710549in}}{\pgfqpoint{0.877296in}{1.702312in}}%
\pgfpathcurveto{\pgfqpoint{0.877296in}{1.694076in}}{\pgfqpoint{0.880568in}{1.686176in}}{\pgfqpoint{0.886392in}{1.680352in}}%
\pgfpathcurveto{\pgfqpoint{0.892216in}{1.674528in}}{\pgfqpoint{0.900116in}{1.671256in}}{\pgfqpoint{0.908352in}{1.671256in}}%
\pgfpathclose%
\pgfusepath{stroke,fill}%
\end{pgfscope}%
\begin{pgfscope}%
\pgfpathrectangle{\pgfqpoint{0.100000in}{0.212622in}}{\pgfqpoint{3.696000in}{3.696000in}}%
\pgfusepath{clip}%
\pgfsetbuttcap%
\pgfsetroundjoin%
\definecolor{currentfill}{rgb}{0.121569,0.466667,0.705882}%
\pgfsetfillcolor{currentfill}%
\pgfsetfillopacity{0.616432}%
\pgfsetlinewidth{1.003750pt}%
\definecolor{currentstroke}{rgb}{0.121569,0.466667,0.705882}%
\pgfsetstrokecolor{currentstroke}%
\pgfsetstrokeopacity{0.616432}%
\pgfsetdash{}{0pt}%
\pgfpathmoveto{\pgfqpoint{3.192329in}{1.980477in}}%
\pgfpathcurveto{\pgfqpoint{3.200565in}{1.980477in}}{\pgfqpoint{3.208465in}{1.983750in}}{\pgfqpoint{3.214289in}{1.989574in}}%
\pgfpathcurveto{\pgfqpoint{3.220113in}{1.995397in}}{\pgfqpoint{3.223385in}{2.003298in}}{\pgfqpoint{3.223385in}{2.011534in}}%
\pgfpathcurveto{\pgfqpoint{3.223385in}{2.019770in}}{\pgfqpoint{3.220113in}{2.027670in}}{\pgfqpoint{3.214289in}{2.033494in}}%
\pgfpathcurveto{\pgfqpoint{3.208465in}{2.039318in}}{\pgfqpoint{3.200565in}{2.042590in}}{\pgfqpoint{3.192329in}{2.042590in}}%
\pgfpathcurveto{\pgfqpoint{3.184093in}{2.042590in}}{\pgfqpoint{3.176192in}{2.039318in}}{\pgfqpoint{3.170369in}{2.033494in}}%
\pgfpathcurveto{\pgfqpoint{3.164545in}{2.027670in}}{\pgfqpoint{3.161272in}{2.019770in}}{\pgfqpoint{3.161272in}{2.011534in}}%
\pgfpathcurveto{\pgfqpoint{3.161272in}{2.003298in}}{\pgfqpoint{3.164545in}{1.995397in}}{\pgfqpoint{3.170369in}{1.989574in}}%
\pgfpathcurveto{\pgfqpoint{3.176192in}{1.983750in}}{\pgfqpoint{3.184093in}{1.980477in}}{\pgfqpoint{3.192329in}{1.980477in}}%
\pgfpathclose%
\pgfusepath{stroke,fill}%
\end{pgfscope}%
\begin{pgfscope}%
\pgfpathrectangle{\pgfqpoint{0.100000in}{0.212622in}}{\pgfqpoint{3.696000in}{3.696000in}}%
\pgfusepath{clip}%
\pgfsetbuttcap%
\pgfsetroundjoin%
\definecolor{currentfill}{rgb}{0.121569,0.466667,0.705882}%
\pgfsetfillcolor{currentfill}%
\pgfsetfillopacity{0.617984}%
\pgfsetlinewidth{1.003750pt}%
\definecolor{currentstroke}{rgb}{0.121569,0.466667,0.705882}%
\pgfsetstrokecolor{currentstroke}%
\pgfsetstrokeopacity{0.617984}%
\pgfsetdash{}{0pt}%
\pgfpathmoveto{\pgfqpoint{0.906858in}{1.682376in}}%
\pgfpathcurveto{\pgfqpoint{0.915094in}{1.682376in}}{\pgfqpoint{0.922994in}{1.685648in}}{\pgfqpoint{0.928818in}{1.691472in}}%
\pgfpathcurveto{\pgfqpoint{0.934642in}{1.697296in}}{\pgfqpoint{0.937914in}{1.705196in}}{\pgfqpoint{0.937914in}{1.713432in}}%
\pgfpathcurveto{\pgfqpoint{0.937914in}{1.721668in}}{\pgfqpoint{0.934642in}{1.729569in}}{\pgfqpoint{0.928818in}{1.735392in}}%
\pgfpathcurveto{\pgfqpoint{0.922994in}{1.741216in}}{\pgfqpoint{0.915094in}{1.744489in}}{\pgfqpoint{0.906858in}{1.744489in}}%
\pgfpathcurveto{\pgfqpoint{0.898622in}{1.744489in}}{\pgfqpoint{0.890722in}{1.741216in}}{\pgfqpoint{0.884898in}{1.735392in}}%
\pgfpathcurveto{\pgfqpoint{0.879074in}{1.729569in}}{\pgfqpoint{0.875801in}{1.721668in}}{\pgfqpoint{0.875801in}{1.713432in}}%
\pgfpathcurveto{\pgfqpoint{0.875801in}{1.705196in}}{\pgfqpoint{0.879074in}{1.697296in}}{\pgfqpoint{0.884898in}{1.691472in}}%
\pgfpathcurveto{\pgfqpoint{0.890722in}{1.685648in}}{\pgfqpoint{0.898622in}{1.682376in}}{\pgfqpoint{0.906858in}{1.682376in}}%
\pgfpathclose%
\pgfusepath{stroke,fill}%
\end{pgfscope}%
\begin{pgfscope}%
\pgfpathrectangle{\pgfqpoint{0.100000in}{0.212622in}}{\pgfqpoint{3.696000in}{3.696000in}}%
\pgfusepath{clip}%
\pgfsetbuttcap%
\pgfsetroundjoin%
\definecolor{currentfill}{rgb}{0.121569,0.466667,0.705882}%
\pgfsetfillcolor{currentfill}%
\pgfsetfillopacity{0.618158}%
\pgfsetlinewidth{1.003750pt}%
\definecolor{currentstroke}{rgb}{0.121569,0.466667,0.705882}%
\pgfsetstrokecolor{currentstroke}%
\pgfsetstrokeopacity{0.618158}%
\pgfsetdash{}{0pt}%
\pgfpathmoveto{\pgfqpoint{3.188121in}{1.979230in}}%
\pgfpathcurveto{\pgfqpoint{3.196357in}{1.979230in}}{\pgfqpoint{3.204257in}{1.982502in}}{\pgfqpoint{3.210081in}{1.988326in}}%
\pgfpathcurveto{\pgfqpoint{3.215905in}{1.994150in}}{\pgfqpoint{3.219177in}{2.002050in}}{\pgfqpoint{3.219177in}{2.010286in}}%
\pgfpathcurveto{\pgfqpoint{3.219177in}{2.018522in}}{\pgfqpoint{3.215905in}{2.026422in}}{\pgfqpoint{3.210081in}{2.032246in}}%
\pgfpathcurveto{\pgfqpoint{3.204257in}{2.038070in}}{\pgfqpoint{3.196357in}{2.041343in}}{\pgfqpoint{3.188121in}{2.041343in}}%
\pgfpathcurveto{\pgfqpoint{3.179885in}{2.041343in}}{\pgfqpoint{3.171985in}{2.038070in}}{\pgfqpoint{3.166161in}{2.032246in}}%
\pgfpathcurveto{\pgfqpoint{3.160337in}{2.026422in}}{\pgfqpoint{3.157064in}{2.018522in}}{\pgfqpoint{3.157064in}{2.010286in}}%
\pgfpathcurveto{\pgfqpoint{3.157064in}{2.002050in}}{\pgfqpoint{3.160337in}{1.994150in}}{\pgfqpoint{3.166161in}{1.988326in}}%
\pgfpathcurveto{\pgfqpoint{3.171985in}{1.982502in}}{\pgfqpoint{3.179885in}{1.979230in}}{\pgfqpoint{3.188121in}{1.979230in}}%
\pgfpathclose%
\pgfusepath{stroke,fill}%
\end{pgfscope}%
\begin{pgfscope}%
\pgfpathrectangle{\pgfqpoint{0.100000in}{0.212622in}}{\pgfqpoint{3.696000in}{3.696000in}}%
\pgfusepath{clip}%
\pgfsetbuttcap%
\pgfsetroundjoin%
\definecolor{currentfill}{rgb}{0.121569,0.466667,0.705882}%
\pgfsetfillcolor{currentfill}%
\pgfsetfillopacity{0.619080}%
\pgfsetlinewidth{1.003750pt}%
\definecolor{currentstroke}{rgb}{0.121569,0.466667,0.705882}%
\pgfsetstrokecolor{currentstroke}%
\pgfsetstrokeopacity{0.619080}%
\pgfsetdash{}{0pt}%
\pgfpathmoveto{\pgfqpoint{3.185312in}{1.979134in}}%
\pgfpathcurveto{\pgfqpoint{3.193548in}{1.979134in}}{\pgfqpoint{3.201448in}{1.982406in}}{\pgfqpoint{3.207272in}{1.988230in}}%
\pgfpathcurveto{\pgfqpoint{3.213096in}{1.994054in}}{\pgfqpoint{3.216369in}{2.001954in}}{\pgfqpoint{3.216369in}{2.010191in}}%
\pgfpathcurveto{\pgfqpoint{3.216369in}{2.018427in}}{\pgfqpoint{3.213096in}{2.026327in}}{\pgfqpoint{3.207272in}{2.032151in}}%
\pgfpathcurveto{\pgfqpoint{3.201448in}{2.037975in}}{\pgfqpoint{3.193548in}{2.041247in}}{\pgfqpoint{3.185312in}{2.041247in}}%
\pgfpathcurveto{\pgfqpoint{3.177076in}{2.041247in}}{\pgfqpoint{3.169176in}{2.037975in}}{\pgfqpoint{3.163352in}{2.032151in}}%
\pgfpathcurveto{\pgfqpoint{3.157528in}{2.026327in}}{\pgfqpoint{3.154256in}{2.018427in}}{\pgfqpoint{3.154256in}{2.010191in}}%
\pgfpathcurveto{\pgfqpoint{3.154256in}{2.001954in}}{\pgfqpoint{3.157528in}{1.994054in}}{\pgfqpoint{3.163352in}{1.988230in}}%
\pgfpathcurveto{\pgfqpoint{3.169176in}{1.982406in}}{\pgfqpoint{3.177076in}{1.979134in}}{\pgfqpoint{3.185312in}{1.979134in}}%
\pgfpathclose%
\pgfusepath{stroke,fill}%
\end{pgfscope}%
\begin{pgfscope}%
\pgfpathrectangle{\pgfqpoint{0.100000in}{0.212622in}}{\pgfqpoint{3.696000in}{3.696000in}}%
\pgfusepath{clip}%
\pgfsetbuttcap%
\pgfsetroundjoin%
\definecolor{currentfill}{rgb}{0.121569,0.466667,0.705882}%
\pgfsetfillcolor{currentfill}%
\pgfsetfillopacity{0.619142}%
\pgfsetlinewidth{1.003750pt}%
\definecolor{currentstroke}{rgb}{0.121569,0.466667,0.705882}%
\pgfsetstrokecolor{currentstroke}%
\pgfsetstrokeopacity{0.619142}%
\pgfsetdash{}{0pt}%
\pgfpathmoveto{\pgfqpoint{0.904252in}{1.679651in}}%
\pgfpathcurveto{\pgfqpoint{0.912488in}{1.679651in}}{\pgfqpoint{0.920388in}{1.682923in}}{\pgfqpoint{0.926212in}{1.688747in}}%
\pgfpathcurveto{\pgfqpoint{0.932036in}{1.694571in}}{\pgfqpoint{0.935308in}{1.702471in}}{\pgfqpoint{0.935308in}{1.710707in}}%
\pgfpathcurveto{\pgfqpoint{0.935308in}{1.718944in}}{\pgfqpoint{0.932036in}{1.726844in}}{\pgfqpoint{0.926212in}{1.732668in}}%
\pgfpathcurveto{\pgfqpoint{0.920388in}{1.738492in}}{\pgfqpoint{0.912488in}{1.741764in}}{\pgfqpoint{0.904252in}{1.741764in}}%
\pgfpathcurveto{\pgfqpoint{0.896015in}{1.741764in}}{\pgfqpoint{0.888115in}{1.738492in}}{\pgfqpoint{0.882291in}{1.732668in}}%
\pgfpathcurveto{\pgfqpoint{0.876468in}{1.726844in}}{\pgfqpoint{0.873195in}{1.718944in}}{\pgfqpoint{0.873195in}{1.710707in}}%
\pgfpathcurveto{\pgfqpoint{0.873195in}{1.702471in}}{\pgfqpoint{0.876468in}{1.694571in}}{\pgfqpoint{0.882291in}{1.688747in}}%
\pgfpathcurveto{\pgfqpoint{0.888115in}{1.682923in}}{\pgfqpoint{0.896015in}{1.679651in}}{\pgfqpoint{0.904252in}{1.679651in}}%
\pgfpathclose%
\pgfusepath{stroke,fill}%
\end{pgfscope}%
\begin{pgfscope}%
\pgfpathrectangle{\pgfqpoint{0.100000in}{0.212622in}}{\pgfqpoint{3.696000in}{3.696000in}}%
\pgfusepath{clip}%
\pgfsetbuttcap%
\pgfsetroundjoin%
\definecolor{currentfill}{rgb}{0.121569,0.466667,0.705882}%
\pgfsetfillcolor{currentfill}%
\pgfsetfillopacity{0.620032}%
\pgfsetlinewidth{1.003750pt}%
\definecolor{currentstroke}{rgb}{0.121569,0.466667,0.705882}%
\pgfsetstrokecolor{currentstroke}%
\pgfsetstrokeopacity{0.620032}%
\pgfsetdash{}{0pt}%
\pgfpathmoveto{\pgfqpoint{0.900451in}{1.679442in}}%
\pgfpathcurveto{\pgfqpoint{0.908687in}{1.679442in}}{\pgfqpoint{0.916587in}{1.682715in}}{\pgfqpoint{0.922411in}{1.688539in}}%
\pgfpathcurveto{\pgfqpoint{0.928235in}{1.694363in}}{\pgfqpoint{0.931507in}{1.702263in}}{\pgfqpoint{0.931507in}{1.710499in}}%
\pgfpathcurveto{\pgfqpoint{0.931507in}{1.718735in}}{\pgfqpoint{0.928235in}{1.726635in}}{\pgfqpoint{0.922411in}{1.732459in}}%
\pgfpathcurveto{\pgfqpoint{0.916587in}{1.738283in}}{\pgfqpoint{0.908687in}{1.741555in}}{\pgfqpoint{0.900451in}{1.741555in}}%
\pgfpathcurveto{\pgfqpoint{0.892214in}{1.741555in}}{\pgfqpoint{0.884314in}{1.738283in}}{\pgfqpoint{0.878490in}{1.732459in}}%
\pgfpathcurveto{\pgfqpoint{0.872666in}{1.726635in}}{\pgfqpoint{0.869394in}{1.718735in}}{\pgfqpoint{0.869394in}{1.710499in}}%
\pgfpathcurveto{\pgfqpoint{0.869394in}{1.702263in}}{\pgfqpoint{0.872666in}{1.694363in}}{\pgfqpoint{0.878490in}{1.688539in}}%
\pgfpathcurveto{\pgfqpoint{0.884314in}{1.682715in}}{\pgfqpoint{0.892214in}{1.679442in}}{\pgfqpoint{0.900451in}{1.679442in}}%
\pgfpathclose%
\pgfusepath{stroke,fill}%
\end{pgfscope}%
\begin{pgfscope}%
\pgfpathrectangle{\pgfqpoint{0.100000in}{0.212622in}}{\pgfqpoint{3.696000in}{3.696000in}}%
\pgfusepath{clip}%
\pgfsetbuttcap%
\pgfsetroundjoin%
\definecolor{currentfill}{rgb}{0.121569,0.466667,0.705882}%
\pgfsetfillcolor{currentfill}%
\pgfsetfillopacity{0.620643}%
\pgfsetlinewidth{1.003750pt}%
\definecolor{currentstroke}{rgb}{0.121569,0.466667,0.705882}%
\pgfsetstrokecolor{currentstroke}%
\pgfsetstrokeopacity{0.620643}%
\pgfsetdash{}{0pt}%
\pgfpathmoveto{\pgfqpoint{3.183589in}{1.977454in}}%
\pgfpathcurveto{\pgfqpoint{3.191825in}{1.977454in}}{\pgfqpoint{3.199726in}{1.980726in}}{\pgfqpoint{3.205549in}{1.986550in}}%
\pgfpathcurveto{\pgfqpoint{3.211373in}{1.992374in}}{\pgfqpoint{3.214646in}{2.000274in}}{\pgfqpoint{3.214646in}{2.008510in}}%
\pgfpathcurveto{\pgfqpoint{3.214646in}{2.016747in}}{\pgfqpoint{3.211373in}{2.024647in}}{\pgfqpoint{3.205549in}{2.030471in}}%
\pgfpathcurveto{\pgfqpoint{3.199726in}{2.036295in}}{\pgfqpoint{3.191825in}{2.039567in}}{\pgfqpoint{3.183589in}{2.039567in}}%
\pgfpathcurveto{\pgfqpoint{3.175353in}{2.039567in}}{\pgfqpoint{3.167453in}{2.036295in}}{\pgfqpoint{3.161629in}{2.030471in}}%
\pgfpathcurveto{\pgfqpoint{3.155805in}{2.024647in}}{\pgfqpoint{3.152533in}{2.016747in}}{\pgfqpoint{3.152533in}{2.008510in}}%
\pgfpathcurveto{\pgfqpoint{3.152533in}{2.000274in}}{\pgfqpoint{3.155805in}{1.992374in}}{\pgfqpoint{3.161629in}{1.986550in}}%
\pgfpathcurveto{\pgfqpoint{3.167453in}{1.980726in}}{\pgfqpoint{3.175353in}{1.977454in}}{\pgfqpoint{3.183589in}{1.977454in}}%
\pgfpathclose%
\pgfusepath{stroke,fill}%
\end{pgfscope}%
\begin{pgfscope}%
\pgfpathrectangle{\pgfqpoint{0.100000in}{0.212622in}}{\pgfqpoint{3.696000in}{3.696000in}}%
\pgfusepath{clip}%
\pgfsetbuttcap%
\pgfsetroundjoin%
\definecolor{currentfill}{rgb}{0.121569,0.466667,0.705882}%
\pgfsetfillcolor{currentfill}%
\pgfsetfillopacity{0.621506}%
\pgfsetlinewidth{1.003750pt}%
\definecolor{currentstroke}{rgb}{0.121569,0.466667,0.705882}%
\pgfsetstrokecolor{currentstroke}%
\pgfsetstrokeopacity{0.621506}%
\pgfsetdash{}{0pt}%
\pgfpathmoveto{\pgfqpoint{3.182451in}{1.976682in}}%
\pgfpathcurveto{\pgfqpoint{3.190688in}{1.976682in}}{\pgfqpoint{3.198588in}{1.979954in}}{\pgfqpoint{3.204412in}{1.985778in}}%
\pgfpathcurveto{\pgfqpoint{3.210236in}{1.991602in}}{\pgfqpoint{3.213508in}{1.999502in}}{\pgfqpoint{3.213508in}{2.007738in}}%
\pgfpathcurveto{\pgfqpoint{3.213508in}{2.015975in}}{\pgfqpoint{3.210236in}{2.023875in}}{\pgfqpoint{3.204412in}{2.029699in}}%
\pgfpathcurveto{\pgfqpoint{3.198588in}{2.035522in}}{\pgfqpoint{3.190688in}{2.038795in}}{\pgfqpoint{3.182451in}{2.038795in}}%
\pgfpathcurveto{\pgfqpoint{3.174215in}{2.038795in}}{\pgfqpoint{3.166315in}{2.035522in}}{\pgfqpoint{3.160491in}{2.029699in}}%
\pgfpathcurveto{\pgfqpoint{3.154667in}{2.023875in}}{\pgfqpoint{3.151395in}{2.015975in}}{\pgfqpoint{3.151395in}{2.007738in}}%
\pgfpathcurveto{\pgfqpoint{3.151395in}{1.999502in}}{\pgfqpoint{3.154667in}{1.991602in}}{\pgfqpoint{3.160491in}{1.985778in}}%
\pgfpathcurveto{\pgfqpoint{3.166315in}{1.979954in}}{\pgfqpoint{3.174215in}{1.976682in}}{\pgfqpoint{3.182451in}{1.976682in}}%
\pgfpathclose%
\pgfusepath{stroke,fill}%
\end{pgfscope}%
\begin{pgfscope}%
\pgfpathrectangle{\pgfqpoint{0.100000in}{0.212622in}}{\pgfqpoint{3.696000in}{3.696000in}}%
\pgfusepath{clip}%
\pgfsetbuttcap%
\pgfsetroundjoin%
\definecolor{currentfill}{rgb}{0.121569,0.466667,0.705882}%
\pgfsetfillcolor{currentfill}%
\pgfsetfillopacity{0.621547}%
\pgfsetlinewidth{1.003750pt}%
\definecolor{currentstroke}{rgb}{0.121569,0.466667,0.705882}%
\pgfsetstrokecolor{currentstroke}%
\pgfsetstrokeopacity{0.621547}%
\pgfsetdash{}{0pt}%
\pgfpathmoveto{\pgfqpoint{0.899132in}{1.682521in}}%
\pgfpathcurveto{\pgfqpoint{0.907368in}{1.682521in}}{\pgfqpoint{0.915268in}{1.685793in}}{\pgfqpoint{0.921092in}{1.691617in}}%
\pgfpathcurveto{\pgfqpoint{0.926916in}{1.697441in}}{\pgfqpoint{0.930188in}{1.705341in}}{\pgfqpoint{0.930188in}{1.713578in}}%
\pgfpathcurveto{\pgfqpoint{0.930188in}{1.721814in}}{\pgfqpoint{0.926916in}{1.729714in}}{\pgfqpoint{0.921092in}{1.735538in}}%
\pgfpathcurveto{\pgfqpoint{0.915268in}{1.741362in}}{\pgfqpoint{0.907368in}{1.744634in}}{\pgfqpoint{0.899132in}{1.744634in}}%
\pgfpathcurveto{\pgfqpoint{0.890895in}{1.744634in}}{\pgfqpoint{0.882995in}{1.741362in}}{\pgfqpoint{0.877171in}{1.735538in}}%
\pgfpathcurveto{\pgfqpoint{0.871347in}{1.729714in}}{\pgfqpoint{0.868075in}{1.721814in}}{\pgfqpoint{0.868075in}{1.713578in}}%
\pgfpathcurveto{\pgfqpoint{0.868075in}{1.705341in}}{\pgfqpoint{0.871347in}{1.697441in}}{\pgfqpoint{0.877171in}{1.691617in}}%
\pgfpathcurveto{\pgfqpoint{0.882995in}{1.685793in}}{\pgfqpoint{0.890895in}{1.682521in}}{\pgfqpoint{0.899132in}{1.682521in}}%
\pgfpathclose%
\pgfusepath{stroke,fill}%
\end{pgfscope}%
\begin{pgfscope}%
\pgfpathrectangle{\pgfqpoint{0.100000in}{0.212622in}}{\pgfqpoint{3.696000in}{3.696000in}}%
\pgfusepath{clip}%
\pgfsetbuttcap%
\pgfsetroundjoin%
\definecolor{currentfill}{rgb}{0.121569,0.466667,0.705882}%
\pgfsetfillcolor{currentfill}%
\pgfsetfillopacity{0.621772}%
\pgfsetlinewidth{1.003750pt}%
\definecolor{currentstroke}{rgb}{0.121569,0.466667,0.705882}%
\pgfsetstrokecolor{currentstroke}%
\pgfsetstrokeopacity{0.621772}%
\pgfsetdash{}{0pt}%
\pgfpathmoveto{\pgfqpoint{0.897752in}{1.682526in}}%
\pgfpathcurveto{\pgfqpoint{0.905989in}{1.682526in}}{\pgfqpoint{0.913889in}{1.685798in}}{\pgfqpoint{0.919713in}{1.691622in}}%
\pgfpathcurveto{\pgfqpoint{0.925537in}{1.697446in}}{\pgfqpoint{0.928809in}{1.705346in}}{\pgfqpoint{0.928809in}{1.713582in}}%
\pgfpathcurveto{\pgfqpoint{0.928809in}{1.721818in}}{\pgfqpoint{0.925537in}{1.729718in}}{\pgfqpoint{0.919713in}{1.735542in}}%
\pgfpathcurveto{\pgfqpoint{0.913889in}{1.741366in}}{\pgfqpoint{0.905989in}{1.744639in}}{\pgfqpoint{0.897752in}{1.744639in}}%
\pgfpathcurveto{\pgfqpoint{0.889516in}{1.744639in}}{\pgfqpoint{0.881616in}{1.741366in}}{\pgfqpoint{0.875792in}{1.735542in}}%
\pgfpathcurveto{\pgfqpoint{0.869968in}{1.729718in}}{\pgfqpoint{0.866696in}{1.721818in}}{\pgfqpoint{0.866696in}{1.713582in}}%
\pgfpathcurveto{\pgfqpoint{0.866696in}{1.705346in}}{\pgfqpoint{0.869968in}{1.697446in}}{\pgfqpoint{0.875792in}{1.691622in}}%
\pgfpathcurveto{\pgfqpoint{0.881616in}{1.685798in}}{\pgfqpoint{0.889516in}{1.682526in}}{\pgfqpoint{0.897752in}{1.682526in}}%
\pgfpathclose%
\pgfusepath{stroke,fill}%
\end{pgfscope}%
\begin{pgfscope}%
\pgfpathrectangle{\pgfqpoint{0.100000in}{0.212622in}}{\pgfqpoint{3.696000in}{3.696000in}}%
\pgfusepath{clip}%
\pgfsetbuttcap%
\pgfsetroundjoin%
\definecolor{currentfill}{rgb}{0.121569,0.466667,0.705882}%
\pgfsetfillcolor{currentfill}%
\pgfsetfillopacity{0.622393}%
\pgfsetlinewidth{1.003750pt}%
\definecolor{currentstroke}{rgb}{0.121569,0.466667,0.705882}%
\pgfsetstrokecolor{currentstroke}%
\pgfsetstrokeopacity{0.622393}%
\pgfsetdash{}{0pt}%
\pgfpathmoveto{\pgfqpoint{0.896053in}{1.682157in}}%
\pgfpathcurveto{\pgfqpoint{0.904289in}{1.682157in}}{\pgfqpoint{0.912189in}{1.685429in}}{\pgfqpoint{0.918013in}{1.691253in}}%
\pgfpathcurveto{\pgfqpoint{0.923837in}{1.697077in}}{\pgfqpoint{0.927109in}{1.704977in}}{\pgfqpoint{0.927109in}{1.713213in}}%
\pgfpathcurveto{\pgfqpoint{0.927109in}{1.721449in}}{\pgfqpoint{0.923837in}{1.729350in}}{\pgfqpoint{0.918013in}{1.735173in}}%
\pgfpathcurveto{\pgfqpoint{0.912189in}{1.740997in}}{\pgfqpoint{0.904289in}{1.744270in}}{\pgfqpoint{0.896053in}{1.744270in}}%
\pgfpathcurveto{\pgfqpoint{0.887817in}{1.744270in}}{\pgfqpoint{0.879917in}{1.740997in}}{\pgfqpoint{0.874093in}{1.735173in}}%
\pgfpathcurveto{\pgfqpoint{0.868269in}{1.729350in}}{\pgfqpoint{0.864996in}{1.721449in}}{\pgfqpoint{0.864996in}{1.713213in}}%
\pgfpathcurveto{\pgfqpoint{0.864996in}{1.704977in}}{\pgfqpoint{0.868269in}{1.697077in}}{\pgfqpoint{0.874093in}{1.691253in}}%
\pgfpathcurveto{\pgfqpoint{0.879917in}{1.685429in}}{\pgfqpoint{0.887817in}{1.682157in}}{\pgfqpoint{0.896053in}{1.682157in}}%
\pgfpathclose%
\pgfusepath{stroke,fill}%
\end{pgfscope}%
\begin{pgfscope}%
\pgfpathrectangle{\pgfqpoint{0.100000in}{0.212622in}}{\pgfqpoint{3.696000in}{3.696000in}}%
\pgfusepath{clip}%
\pgfsetbuttcap%
\pgfsetroundjoin%
\definecolor{currentfill}{rgb}{0.121569,0.466667,0.705882}%
\pgfsetfillcolor{currentfill}%
\pgfsetfillopacity{0.623234}%
\pgfsetlinewidth{1.003750pt}%
\definecolor{currentstroke}{rgb}{0.121569,0.466667,0.705882}%
\pgfsetstrokecolor{currentstroke}%
\pgfsetstrokeopacity{0.623234}%
\pgfsetdash{}{0pt}%
\pgfpathmoveto{\pgfqpoint{3.178531in}{1.980825in}}%
\pgfpathcurveto{\pgfqpoint{3.186767in}{1.980825in}}{\pgfqpoint{3.194667in}{1.984097in}}{\pgfqpoint{3.200491in}{1.989921in}}%
\pgfpathcurveto{\pgfqpoint{3.206315in}{1.995745in}}{\pgfqpoint{3.209587in}{2.003645in}}{\pgfqpoint{3.209587in}{2.011881in}}%
\pgfpathcurveto{\pgfqpoint{3.209587in}{2.020118in}}{\pgfqpoint{3.206315in}{2.028018in}}{\pgfqpoint{3.200491in}{2.033842in}}%
\pgfpathcurveto{\pgfqpoint{3.194667in}{2.039665in}}{\pgfqpoint{3.186767in}{2.042938in}}{\pgfqpoint{3.178531in}{2.042938in}}%
\pgfpathcurveto{\pgfqpoint{3.170295in}{2.042938in}}{\pgfqpoint{3.162395in}{2.039665in}}{\pgfqpoint{3.156571in}{2.033842in}}%
\pgfpathcurveto{\pgfqpoint{3.150747in}{2.028018in}}{\pgfqpoint{3.147474in}{2.020118in}}{\pgfqpoint{3.147474in}{2.011881in}}%
\pgfpathcurveto{\pgfqpoint{3.147474in}{2.003645in}}{\pgfqpoint{3.150747in}{1.995745in}}{\pgfqpoint{3.156571in}{1.989921in}}%
\pgfpathcurveto{\pgfqpoint{3.162395in}{1.984097in}}{\pgfqpoint{3.170295in}{1.980825in}}{\pgfqpoint{3.178531in}{1.980825in}}%
\pgfpathclose%
\pgfusepath{stroke,fill}%
\end{pgfscope}%
\begin{pgfscope}%
\pgfpathrectangle{\pgfqpoint{0.100000in}{0.212622in}}{\pgfqpoint{3.696000in}{3.696000in}}%
\pgfusepath{clip}%
\pgfsetbuttcap%
\pgfsetroundjoin%
\definecolor{currentfill}{rgb}{0.121569,0.466667,0.705882}%
\pgfsetfillcolor{currentfill}%
\pgfsetfillopacity{0.623560}%
\pgfsetlinewidth{1.003750pt}%
\definecolor{currentstroke}{rgb}{0.121569,0.466667,0.705882}%
\pgfsetstrokecolor{currentstroke}%
\pgfsetstrokeopacity{0.623560}%
\pgfsetdash{}{0pt}%
\pgfpathmoveto{\pgfqpoint{0.894290in}{1.680508in}}%
\pgfpathcurveto{\pgfqpoint{0.902526in}{1.680508in}}{\pgfqpoint{0.910426in}{1.683780in}}{\pgfqpoint{0.916250in}{1.689604in}}%
\pgfpathcurveto{\pgfqpoint{0.922074in}{1.695428in}}{\pgfqpoint{0.925347in}{1.703328in}}{\pgfqpoint{0.925347in}{1.711564in}}%
\pgfpathcurveto{\pgfqpoint{0.925347in}{1.719801in}}{\pgfqpoint{0.922074in}{1.727701in}}{\pgfqpoint{0.916250in}{1.733525in}}%
\pgfpathcurveto{\pgfqpoint{0.910426in}{1.739349in}}{\pgfqpoint{0.902526in}{1.742621in}}{\pgfqpoint{0.894290in}{1.742621in}}%
\pgfpathcurveto{\pgfqpoint{0.886054in}{1.742621in}}{\pgfqpoint{0.878154in}{1.739349in}}{\pgfqpoint{0.872330in}{1.733525in}}%
\pgfpathcurveto{\pgfqpoint{0.866506in}{1.727701in}}{\pgfqpoint{0.863234in}{1.719801in}}{\pgfqpoint{0.863234in}{1.711564in}}%
\pgfpathcurveto{\pgfqpoint{0.863234in}{1.703328in}}{\pgfqpoint{0.866506in}{1.695428in}}{\pgfqpoint{0.872330in}{1.689604in}}%
\pgfpathcurveto{\pgfqpoint{0.878154in}{1.683780in}}{\pgfqpoint{0.886054in}{1.680508in}}{\pgfqpoint{0.894290in}{1.680508in}}%
\pgfpathclose%
\pgfusepath{stroke,fill}%
\end{pgfscope}%
\begin{pgfscope}%
\pgfpathrectangle{\pgfqpoint{0.100000in}{0.212622in}}{\pgfqpoint{3.696000in}{3.696000in}}%
\pgfusepath{clip}%
\pgfsetbuttcap%
\pgfsetroundjoin%
\definecolor{currentfill}{rgb}{0.121569,0.466667,0.705882}%
\pgfsetfillcolor{currentfill}%
\pgfsetfillopacity{0.624182}%
\pgfsetlinewidth{1.003750pt}%
\definecolor{currentstroke}{rgb}{0.121569,0.466667,0.705882}%
\pgfsetstrokecolor{currentstroke}%
\pgfsetstrokeopacity{0.624182}%
\pgfsetdash{}{0pt}%
\pgfpathmoveto{\pgfqpoint{0.891496in}{1.677714in}}%
\pgfpathcurveto{\pgfqpoint{0.899733in}{1.677714in}}{\pgfqpoint{0.907633in}{1.680986in}}{\pgfqpoint{0.913457in}{1.686810in}}%
\pgfpathcurveto{\pgfqpoint{0.919281in}{1.692634in}}{\pgfqpoint{0.922553in}{1.700534in}}{\pgfqpoint{0.922553in}{1.708770in}}%
\pgfpathcurveto{\pgfqpoint{0.922553in}{1.717006in}}{\pgfqpoint{0.919281in}{1.724907in}}{\pgfqpoint{0.913457in}{1.730730in}}%
\pgfpathcurveto{\pgfqpoint{0.907633in}{1.736554in}}{\pgfqpoint{0.899733in}{1.739827in}}{\pgfqpoint{0.891496in}{1.739827in}}%
\pgfpathcurveto{\pgfqpoint{0.883260in}{1.739827in}}{\pgfqpoint{0.875360in}{1.736554in}}{\pgfqpoint{0.869536in}{1.730730in}}%
\pgfpathcurveto{\pgfqpoint{0.863712in}{1.724907in}}{\pgfqpoint{0.860440in}{1.717006in}}{\pgfqpoint{0.860440in}{1.708770in}}%
\pgfpathcurveto{\pgfqpoint{0.860440in}{1.700534in}}{\pgfqpoint{0.863712in}{1.692634in}}{\pgfqpoint{0.869536in}{1.686810in}}%
\pgfpathcurveto{\pgfqpoint{0.875360in}{1.680986in}}{\pgfqpoint{0.883260in}{1.677714in}}{\pgfqpoint{0.891496in}{1.677714in}}%
\pgfpathclose%
\pgfusepath{stroke,fill}%
\end{pgfscope}%
\begin{pgfscope}%
\pgfpathrectangle{\pgfqpoint{0.100000in}{0.212622in}}{\pgfqpoint{3.696000in}{3.696000in}}%
\pgfusepath{clip}%
\pgfsetbuttcap%
\pgfsetroundjoin%
\definecolor{currentfill}{rgb}{0.121569,0.466667,0.705882}%
\pgfsetfillcolor{currentfill}%
\pgfsetfillopacity{0.624460}%
\pgfsetlinewidth{1.003750pt}%
\definecolor{currentstroke}{rgb}{0.121569,0.466667,0.705882}%
\pgfsetstrokecolor{currentstroke}%
\pgfsetstrokeopacity{0.624460}%
\pgfsetdash{}{0pt}%
\pgfpathmoveto{\pgfqpoint{3.176830in}{1.977316in}}%
\pgfpathcurveto{\pgfqpoint{3.185066in}{1.977316in}}{\pgfqpoint{3.192966in}{1.980588in}}{\pgfqpoint{3.198790in}{1.986412in}}%
\pgfpathcurveto{\pgfqpoint{3.204614in}{1.992236in}}{\pgfqpoint{3.207886in}{2.000136in}}{\pgfqpoint{3.207886in}{2.008372in}}%
\pgfpathcurveto{\pgfqpoint{3.207886in}{2.016608in}}{\pgfqpoint{3.204614in}{2.024508in}}{\pgfqpoint{3.198790in}{2.030332in}}%
\pgfpathcurveto{\pgfqpoint{3.192966in}{2.036156in}}{\pgfqpoint{3.185066in}{2.039429in}}{\pgfqpoint{3.176830in}{2.039429in}}%
\pgfpathcurveto{\pgfqpoint{3.168593in}{2.039429in}}{\pgfqpoint{3.160693in}{2.036156in}}{\pgfqpoint{3.154869in}{2.030332in}}%
\pgfpathcurveto{\pgfqpoint{3.149045in}{2.024508in}}{\pgfqpoint{3.145773in}{2.016608in}}{\pgfqpoint{3.145773in}{2.008372in}}%
\pgfpathcurveto{\pgfqpoint{3.145773in}{2.000136in}}{\pgfqpoint{3.149045in}{1.992236in}}{\pgfqpoint{3.154869in}{1.986412in}}%
\pgfpathcurveto{\pgfqpoint{3.160693in}{1.980588in}}{\pgfqpoint{3.168593in}{1.977316in}}{\pgfqpoint{3.176830in}{1.977316in}}%
\pgfpathclose%
\pgfusepath{stroke,fill}%
\end{pgfscope}%
\begin{pgfscope}%
\pgfpathrectangle{\pgfqpoint{0.100000in}{0.212622in}}{\pgfqpoint{3.696000in}{3.696000in}}%
\pgfusepath{clip}%
\pgfsetbuttcap%
\pgfsetroundjoin%
\definecolor{currentfill}{rgb}{0.121569,0.466667,0.705882}%
\pgfsetfillcolor{currentfill}%
\pgfsetfillopacity{0.624841}%
\pgfsetlinewidth{1.003750pt}%
\definecolor{currentstroke}{rgb}{0.121569,0.466667,0.705882}%
\pgfsetstrokecolor{currentstroke}%
\pgfsetstrokeopacity{0.624841}%
\pgfsetdash{}{0pt}%
\pgfpathmoveto{\pgfqpoint{0.890491in}{1.676623in}}%
\pgfpathcurveto{\pgfqpoint{0.898727in}{1.676623in}}{\pgfqpoint{0.906627in}{1.679895in}}{\pgfqpoint{0.912451in}{1.685719in}}%
\pgfpathcurveto{\pgfqpoint{0.918275in}{1.691543in}}{\pgfqpoint{0.921547in}{1.699443in}}{\pgfqpoint{0.921547in}{1.707680in}}%
\pgfpathcurveto{\pgfqpoint{0.921547in}{1.715916in}}{\pgfqpoint{0.918275in}{1.723816in}}{\pgfqpoint{0.912451in}{1.729640in}}%
\pgfpathcurveto{\pgfqpoint{0.906627in}{1.735464in}}{\pgfqpoint{0.898727in}{1.738736in}}{\pgfqpoint{0.890491in}{1.738736in}}%
\pgfpathcurveto{\pgfqpoint{0.882254in}{1.738736in}}{\pgfqpoint{0.874354in}{1.735464in}}{\pgfqpoint{0.868530in}{1.729640in}}%
\pgfpathcurveto{\pgfqpoint{0.862706in}{1.723816in}}{\pgfqpoint{0.859434in}{1.715916in}}{\pgfqpoint{0.859434in}{1.707680in}}%
\pgfpathcurveto{\pgfqpoint{0.859434in}{1.699443in}}{\pgfqpoint{0.862706in}{1.691543in}}{\pgfqpoint{0.868530in}{1.685719in}}%
\pgfpathcurveto{\pgfqpoint{0.874354in}{1.679895in}}{\pgfqpoint{0.882254in}{1.676623in}}{\pgfqpoint{0.890491in}{1.676623in}}%
\pgfpathclose%
\pgfusepath{stroke,fill}%
\end{pgfscope}%
\begin{pgfscope}%
\pgfpathrectangle{\pgfqpoint{0.100000in}{0.212622in}}{\pgfqpoint{3.696000in}{3.696000in}}%
\pgfusepath{clip}%
\pgfsetbuttcap%
\pgfsetroundjoin%
\definecolor{currentfill}{rgb}{0.121569,0.466667,0.705882}%
\pgfsetfillcolor{currentfill}%
\pgfsetfillopacity{0.625299}%
\pgfsetlinewidth{1.003750pt}%
\definecolor{currentstroke}{rgb}{0.121569,0.466667,0.705882}%
\pgfsetstrokecolor{currentstroke}%
\pgfsetstrokeopacity{0.625299}%
\pgfsetdash{}{0pt}%
\pgfpathmoveto{\pgfqpoint{0.888760in}{1.676400in}}%
\pgfpathcurveto{\pgfqpoint{0.896997in}{1.676400in}}{\pgfqpoint{0.904897in}{1.679672in}}{\pgfqpoint{0.910721in}{1.685496in}}%
\pgfpathcurveto{\pgfqpoint{0.916545in}{1.691320in}}{\pgfqpoint{0.919817in}{1.699220in}}{\pgfqpoint{0.919817in}{1.707456in}}%
\pgfpathcurveto{\pgfqpoint{0.919817in}{1.715692in}}{\pgfqpoint{0.916545in}{1.723592in}}{\pgfqpoint{0.910721in}{1.729416in}}%
\pgfpathcurveto{\pgfqpoint{0.904897in}{1.735240in}}{\pgfqpoint{0.896997in}{1.738513in}}{\pgfqpoint{0.888760in}{1.738513in}}%
\pgfpathcurveto{\pgfqpoint{0.880524in}{1.738513in}}{\pgfqpoint{0.872624in}{1.735240in}}{\pgfqpoint{0.866800in}{1.729416in}}%
\pgfpathcurveto{\pgfqpoint{0.860976in}{1.723592in}}{\pgfqpoint{0.857704in}{1.715692in}}{\pgfqpoint{0.857704in}{1.707456in}}%
\pgfpathcurveto{\pgfqpoint{0.857704in}{1.699220in}}{\pgfqpoint{0.860976in}{1.691320in}}{\pgfqpoint{0.866800in}{1.685496in}}%
\pgfpathcurveto{\pgfqpoint{0.872624in}{1.679672in}}{\pgfqpoint{0.880524in}{1.676400in}}{\pgfqpoint{0.888760in}{1.676400in}}%
\pgfpathclose%
\pgfusepath{stroke,fill}%
\end{pgfscope}%
\begin{pgfscope}%
\pgfpathrectangle{\pgfqpoint{0.100000in}{0.212622in}}{\pgfqpoint{3.696000in}{3.696000in}}%
\pgfusepath{clip}%
\pgfsetbuttcap%
\pgfsetroundjoin%
\definecolor{currentfill}{rgb}{0.121569,0.466667,0.705882}%
\pgfsetfillcolor{currentfill}%
\pgfsetfillopacity{0.625786}%
\pgfsetlinewidth{1.003750pt}%
\definecolor{currentstroke}{rgb}{0.121569,0.466667,0.705882}%
\pgfsetstrokecolor{currentstroke}%
\pgfsetstrokeopacity{0.625786}%
\pgfsetdash{}{0pt}%
\pgfpathmoveto{\pgfqpoint{0.888264in}{1.676415in}}%
\pgfpathcurveto{\pgfqpoint{0.896500in}{1.676415in}}{\pgfqpoint{0.904400in}{1.679687in}}{\pgfqpoint{0.910224in}{1.685511in}}%
\pgfpathcurveto{\pgfqpoint{0.916048in}{1.691335in}}{\pgfqpoint{0.919320in}{1.699235in}}{\pgfqpoint{0.919320in}{1.707471in}}%
\pgfpathcurveto{\pgfqpoint{0.919320in}{1.715707in}}{\pgfqpoint{0.916048in}{1.723608in}}{\pgfqpoint{0.910224in}{1.729431in}}%
\pgfpathcurveto{\pgfqpoint{0.904400in}{1.735255in}}{\pgfqpoint{0.896500in}{1.738528in}}{\pgfqpoint{0.888264in}{1.738528in}}%
\pgfpathcurveto{\pgfqpoint{0.880028in}{1.738528in}}{\pgfqpoint{0.872127in}{1.735255in}}{\pgfqpoint{0.866304in}{1.729431in}}%
\pgfpathcurveto{\pgfqpoint{0.860480in}{1.723608in}}{\pgfqpoint{0.857207in}{1.715707in}}{\pgfqpoint{0.857207in}{1.707471in}}%
\pgfpathcurveto{\pgfqpoint{0.857207in}{1.699235in}}{\pgfqpoint{0.860480in}{1.691335in}}{\pgfqpoint{0.866304in}{1.685511in}}%
\pgfpathcurveto{\pgfqpoint{0.872127in}{1.679687in}}{\pgfqpoint{0.880028in}{1.676415in}}{\pgfqpoint{0.888264in}{1.676415in}}%
\pgfpathclose%
\pgfusepath{stroke,fill}%
\end{pgfscope}%
\begin{pgfscope}%
\pgfpathrectangle{\pgfqpoint{0.100000in}{0.212622in}}{\pgfqpoint{3.696000in}{3.696000in}}%
\pgfusepath{clip}%
\pgfsetbuttcap%
\pgfsetroundjoin%
\definecolor{currentfill}{rgb}{0.121569,0.466667,0.705882}%
\pgfsetfillcolor{currentfill}%
\pgfsetfillopacity{0.626038}%
\pgfsetlinewidth{1.003750pt}%
\definecolor{currentstroke}{rgb}{0.121569,0.466667,0.705882}%
\pgfsetstrokecolor{currentstroke}%
\pgfsetstrokeopacity{0.626038}%
\pgfsetdash{}{0pt}%
\pgfpathmoveto{\pgfqpoint{3.174439in}{1.976067in}}%
\pgfpathcurveto{\pgfqpoint{3.182676in}{1.976067in}}{\pgfqpoint{3.190576in}{1.979339in}}{\pgfqpoint{3.196400in}{1.985163in}}%
\pgfpathcurveto{\pgfqpoint{3.202224in}{1.990987in}}{\pgfqpoint{3.205496in}{1.998887in}}{\pgfqpoint{3.205496in}{2.007123in}}%
\pgfpathcurveto{\pgfqpoint{3.205496in}{2.015360in}}{\pgfqpoint{3.202224in}{2.023260in}}{\pgfqpoint{3.196400in}{2.029084in}}%
\pgfpathcurveto{\pgfqpoint{3.190576in}{2.034908in}}{\pgfqpoint{3.182676in}{2.038180in}}{\pgfqpoint{3.174439in}{2.038180in}}%
\pgfpathcurveto{\pgfqpoint{3.166203in}{2.038180in}}{\pgfqpoint{3.158303in}{2.034908in}}{\pgfqpoint{3.152479in}{2.029084in}}%
\pgfpathcurveto{\pgfqpoint{3.146655in}{2.023260in}}{\pgfqpoint{3.143383in}{2.015360in}}{\pgfqpoint{3.143383in}{2.007123in}}%
\pgfpathcurveto{\pgfqpoint{3.143383in}{1.998887in}}{\pgfqpoint{3.146655in}{1.990987in}}{\pgfqpoint{3.152479in}{1.985163in}}%
\pgfpathcurveto{\pgfqpoint{3.158303in}{1.979339in}}{\pgfqpoint{3.166203in}{1.976067in}}{\pgfqpoint{3.174439in}{1.976067in}}%
\pgfpathclose%
\pgfusepath{stroke,fill}%
\end{pgfscope}%
\begin{pgfscope}%
\pgfpathrectangle{\pgfqpoint{0.100000in}{0.212622in}}{\pgfqpoint{3.696000in}{3.696000in}}%
\pgfusepath{clip}%
\pgfsetbuttcap%
\pgfsetroundjoin%
\definecolor{currentfill}{rgb}{0.121569,0.466667,0.705882}%
\pgfsetfillcolor{currentfill}%
\pgfsetfillopacity{0.626187}%
\pgfsetlinewidth{1.003750pt}%
\definecolor{currentstroke}{rgb}{0.121569,0.466667,0.705882}%
\pgfsetstrokecolor{currentstroke}%
\pgfsetstrokeopacity{0.626187}%
\pgfsetdash{}{0pt}%
\pgfpathmoveto{\pgfqpoint{0.886931in}{1.673393in}}%
\pgfpathcurveto{\pgfqpoint{0.895167in}{1.673393in}}{\pgfqpoint{0.903067in}{1.676665in}}{\pgfqpoint{0.908891in}{1.682489in}}%
\pgfpathcurveto{\pgfqpoint{0.914715in}{1.688313in}}{\pgfqpoint{0.917987in}{1.696213in}}{\pgfqpoint{0.917987in}{1.704450in}}%
\pgfpathcurveto{\pgfqpoint{0.917987in}{1.712686in}}{\pgfqpoint{0.914715in}{1.720586in}}{\pgfqpoint{0.908891in}{1.726410in}}%
\pgfpathcurveto{\pgfqpoint{0.903067in}{1.732234in}}{\pgfqpoint{0.895167in}{1.735506in}}{\pgfqpoint{0.886931in}{1.735506in}}%
\pgfpathcurveto{\pgfqpoint{0.878694in}{1.735506in}}{\pgfqpoint{0.870794in}{1.732234in}}{\pgfqpoint{0.864970in}{1.726410in}}%
\pgfpathcurveto{\pgfqpoint{0.859147in}{1.720586in}}{\pgfqpoint{0.855874in}{1.712686in}}{\pgfqpoint{0.855874in}{1.704450in}}%
\pgfpathcurveto{\pgfqpoint{0.855874in}{1.696213in}}{\pgfqpoint{0.859147in}{1.688313in}}{\pgfqpoint{0.864970in}{1.682489in}}%
\pgfpathcurveto{\pgfqpoint{0.870794in}{1.676665in}}{\pgfqpoint{0.878694in}{1.673393in}}{\pgfqpoint{0.886931in}{1.673393in}}%
\pgfpathclose%
\pgfusepath{stroke,fill}%
\end{pgfscope}%
\begin{pgfscope}%
\pgfpathrectangle{\pgfqpoint{0.100000in}{0.212622in}}{\pgfqpoint{3.696000in}{3.696000in}}%
\pgfusepath{clip}%
\pgfsetbuttcap%
\pgfsetroundjoin%
\definecolor{currentfill}{rgb}{0.121569,0.466667,0.705882}%
\pgfsetfillcolor{currentfill}%
\pgfsetfillopacity{0.626584}%
\pgfsetlinewidth{1.003750pt}%
\definecolor{currentstroke}{rgb}{0.121569,0.466667,0.705882}%
\pgfsetstrokecolor{currentstroke}%
\pgfsetstrokeopacity{0.626584}%
\pgfsetdash{}{0pt}%
\pgfpathmoveto{\pgfqpoint{0.885358in}{1.672240in}}%
\pgfpathcurveto{\pgfqpoint{0.893594in}{1.672240in}}{\pgfqpoint{0.901494in}{1.675512in}}{\pgfqpoint{0.907318in}{1.681336in}}%
\pgfpathcurveto{\pgfqpoint{0.913142in}{1.687160in}}{\pgfqpoint{0.916415in}{1.695060in}}{\pgfqpoint{0.916415in}{1.703296in}}%
\pgfpathcurveto{\pgfqpoint{0.916415in}{1.711533in}}{\pgfqpoint{0.913142in}{1.719433in}}{\pgfqpoint{0.907318in}{1.725257in}}%
\pgfpathcurveto{\pgfqpoint{0.901494in}{1.731081in}}{\pgfqpoint{0.893594in}{1.734353in}}{\pgfqpoint{0.885358in}{1.734353in}}%
\pgfpathcurveto{\pgfqpoint{0.877122in}{1.734353in}}{\pgfqpoint{0.869222in}{1.731081in}}{\pgfqpoint{0.863398in}{1.725257in}}%
\pgfpathcurveto{\pgfqpoint{0.857574in}{1.719433in}}{\pgfqpoint{0.854302in}{1.711533in}}{\pgfqpoint{0.854302in}{1.703296in}}%
\pgfpathcurveto{\pgfqpoint{0.854302in}{1.695060in}}{\pgfqpoint{0.857574in}{1.687160in}}{\pgfqpoint{0.863398in}{1.681336in}}%
\pgfpathcurveto{\pgfqpoint{0.869222in}{1.675512in}}{\pgfqpoint{0.877122in}{1.672240in}}{\pgfqpoint{0.885358in}{1.672240in}}%
\pgfpathclose%
\pgfusepath{stroke,fill}%
\end{pgfscope}%
\begin{pgfscope}%
\pgfpathrectangle{\pgfqpoint{0.100000in}{0.212622in}}{\pgfqpoint{3.696000in}{3.696000in}}%
\pgfusepath{clip}%
\pgfsetbuttcap%
\pgfsetroundjoin%
\definecolor{currentfill}{rgb}{0.121569,0.466667,0.705882}%
\pgfsetfillcolor{currentfill}%
\pgfsetfillopacity{0.627516}%
\pgfsetlinewidth{1.003750pt}%
\definecolor{currentstroke}{rgb}{0.121569,0.466667,0.705882}%
\pgfsetstrokecolor{currentstroke}%
\pgfsetstrokeopacity{0.627516}%
\pgfsetdash{}{0pt}%
\pgfpathmoveto{\pgfqpoint{0.883270in}{1.670636in}}%
\pgfpathcurveto{\pgfqpoint{0.891506in}{1.670636in}}{\pgfqpoint{0.899406in}{1.673908in}}{\pgfqpoint{0.905230in}{1.679732in}}%
\pgfpathcurveto{\pgfqpoint{0.911054in}{1.685556in}}{\pgfqpoint{0.914327in}{1.693456in}}{\pgfqpoint{0.914327in}{1.701693in}}%
\pgfpathcurveto{\pgfqpoint{0.914327in}{1.709929in}}{\pgfqpoint{0.911054in}{1.717829in}}{\pgfqpoint{0.905230in}{1.723653in}}%
\pgfpathcurveto{\pgfqpoint{0.899406in}{1.729477in}}{\pgfqpoint{0.891506in}{1.732749in}}{\pgfqpoint{0.883270in}{1.732749in}}%
\pgfpathcurveto{\pgfqpoint{0.875034in}{1.732749in}}{\pgfqpoint{0.867134in}{1.729477in}}{\pgfqpoint{0.861310in}{1.723653in}}%
\pgfpathcurveto{\pgfqpoint{0.855486in}{1.717829in}}{\pgfqpoint{0.852214in}{1.709929in}}{\pgfqpoint{0.852214in}{1.701693in}}%
\pgfpathcurveto{\pgfqpoint{0.852214in}{1.693456in}}{\pgfqpoint{0.855486in}{1.685556in}}{\pgfqpoint{0.861310in}{1.679732in}}%
\pgfpathcurveto{\pgfqpoint{0.867134in}{1.673908in}}{\pgfqpoint{0.875034in}{1.670636in}}{\pgfqpoint{0.883270in}{1.670636in}}%
\pgfpathclose%
\pgfusepath{stroke,fill}%
\end{pgfscope}%
\begin{pgfscope}%
\pgfpathrectangle{\pgfqpoint{0.100000in}{0.212622in}}{\pgfqpoint{3.696000in}{3.696000in}}%
\pgfusepath{clip}%
\pgfsetbuttcap%
\pgfsetroundjoin%
\definecolor{currentfill}{rgb}{0.121569,0.466667,0.705882}%
\pgfsetfillcolor{currentfill}%
\pgfsetfillopacity{0.627795}%
\pgfsetlinewidth{1.003750pt}%
\definecolor{currentstroke}{rgb}{0.121569,0.466667,0.705882}%
\pgfsetstrokecolor{currentstroke}%
\pgfsetstrokeopacity{0.627795}%
\pgfsetdash{}{0pt}%
\pgfpathmoveto{\pgfqpoint{3.169483in}{1.973594in}}%
\pgfpathcurveto{\pgfqpoint{3.177719in}{1.973594in}}{\pgfqpoint{3.185619in}{1.976866in}}{\pgfqpoint{3.191443in}{1.982690in}}%
\pgfpathcurveto{\pgfqpoint{3.197267in}{1.988514in}}{\pgfqpoint{3.200539in}{1.996414in}}{\pgfqpoint{3.200539in}{2.004650in}}%
\pgfpathcurveto{\pgfqpoint{3.200539in}{2.012886in}}{\pgfqpoint{3.197267in}{2.020786in}}{\pgfqpoint{3.191443in}{2.026610in}}%
\pgfpathcurveto{\pgfqpoint{3.185619in}{2.032434in}}{\pgfqpoint{3.177719in}{2.035707in}}{\pgfqpoint{3.169483in}{2.035707in}}%
\pgfpathcurveto{\pgfqpoint{3.161246in}{2.035707in}}{\pgfqpoint{3.153346in}{2.032434in}}{\pgfqpoint{3.147522in}{2.026610in}}%
\pgfpathcurveto{\pgfqpoint{3.141698in}{2.020786in}}{\pgfqpoint{3.138426in}{2.012886in}}{\pgfqpoint{3.138426in}{2.004650in}}%
\pgfpathcurveto{\pgfqpoint{3.138426in}{1.996414in}}{\pgfqpoint{3.141698in}{1.988514in}}{\pgfqpoint{3.147522in}{1.982690in}}%
\pgfpathcurveto{\pgfqpoint{3.153346in}{1.976866in}}{\pgfqpoint{3.161246in}{1.973594in}}{\pgfqpoint{3.169483in}{1.973594in}}%
\pgfpathclose%
\pgfusepath{stroke,fill}%
\end{pgfscope}%
\begin{pgfscope}%
\pgfpathrectangle{\pgfqpoint{0.100000in}{0.212622in}}{\pgfqpoint{3.696000in}{3.696000in}}%
\pgfusepath{clip}%
\pgfsetbuttcap%
\pgfsetroundjoin%
\definecolor{currentfill}{rgb}{0.121569,0.466667,0.705882}%
\pgfsetfillcolor{currentfill}%
\pgfsetfillopacity{0.628106}%
\pgfsetlinewidth{1.003750pt}%
\definecolor{currentstroke}{rgb}{0.121569,0.466667,0.705882}%
\pgfsetstrokecolor{currentstroke}%
\pgfsetstrokeopacity{0.628106}%
\pgfsetdash{}{0pt}%
\pgfpathmoveto{\pgfqpoint{0.880537in}{1.669293in}}%
\pgfpathcurveto{\pgfqpoint{0.888773in}{1.669293in}}{\pgfqpoint{0.896673in}{1.672565in}}{\pgfqpoint{0.902497in}{1.678389in}}%
\pgfpathcurveto{\pgfqpoint{0.908321in}{1.684213in}}{\pgfqpoint{0.911593in}{1.692113in}}{\pgfqpoint{0.911593in}{1.700349in}}%
\pgfpathcurveto{\pgfqpoint{0.911593in}{1.708586in}}{\pgfqpoint{0.908321in}{1.716486in}}{\pgfqpoint{0.902497in}{1.722310in}}%
\pgfpathcurveto{\pgfqpoint{0.896673in}{1.728134in}}{\pgfqpoint{0.888773in}{1.731406in}}{\pgfqpoint{0.880537in}{1.731406in}}%
\pgfpathcurveto{\pgfqpoint{0.872300in}{1.731406in}}{\pgfqpoint{0.864400in}{1.728134in}}{\pgfqpoint{0.858576in}{1.722310in}}%
\pgfpathcurveto{\pgfqpoint{0.852752in}{1.716486in}}{\pgfqpoint{0.849480in}{1.708586in}}{\pgfqpoint{0.849480in}{1.700349in}}%
\pgfpathcurveto{\pgfqpoint{0.849480in}{1.692113in}}{\pgfqpoint{0.852752in}{1.684213in}}{\pgfqpoint{0.858576in}{1.678389in}}%
\pgfpathcurveto{\pgfqpoint{0.864400in}{1.672565in}}{\pgfqpoint{0.872300in}{1.669293in}}{\pgfqpoint{0.880537in}{1.669293in}}%
\pgfpathclose%
\pgfusepath{stroke,fill}%
\end{pgfscope}%
\begin{pgfscope}%
\pgfpathrectangle{\pgfqpoint{0.100000in}{0.212622in}}{\pgfqpoint{3.696000in}{3.696000in}}%
\pgfusepath{clip}%
\pgfsetbuttcap%
\pgfsetroundjoin%
\definecolor{currentfill}{rgb}{0.121569,0.466667,0.705882}%
\pgfsetfillcolor{currentfill}%
\pgfsetfillopacity{0.628710}%
\pgfsetlinewidth{1.003750pt}%
\definecolor{currentstroke}{rgb}{0.121569,0.466667,0.705882}%
\pgfsetstrokecolor{currentstroke}%
\pgfsetstrokeopacity{0.628710}%
\pgfsetdash{}{0pt}%
\pgfpathmoveto{\pgfqpoint{3.166684in}{1.971959in}}%
\pgfpathcurveto{\pgfqpoint{3.174921in}{1.971959in}}{\pgfqpoint{3.182821in}{1.975231in}}{\pgfqpoint{3.188645in}{1.981055in}}%
\pgfpathcurveto{\pgfqpoint{3.194469in}{1.986879in}}{\pgfqpoint{3.197741in}{1.994779in}}{\pgfqpoint{3.197741in}{2.003016in}}%
\pgfpathcurveto{\pgfqpoint{3.197741in}{2.011252in}}{\pgfqpoint{3.194469in}{2.019152in}}{\pgfqpoint{3.188645in}{2.024976in}}%
\pgfpathcurveto{\pgfqpoint{3.182821in}{2.030800in}}{\pgfqpoint{3.174921in}{2.034072in}}{\pgfqpoint{3.166684in}{2.034072in}}%
\pgfpathcurveto{\pgfqpoint{3.158448in}{2.034072in}}{\pgfqpoint{3.150548in}{2.030800in}}{\pgfqpoint{3.144724in}{2.024976in}}%
\pgfpathcurveto{\pgfqpoint{3.138900in}{2.019152in}}{\pgfqpoint{3.135628in}{2.011252in}}{\pgfqpoint{3.135628in}{2.003016in}}%
\pgfpathcurveto{\pgfqpoint{3.135628in}{1.994779in}}{\pgfqpoint{3.138900in}{1.986879in}}{\pgfqpoint{3.144724in}{1.981055in}}%
\pgfpathcurveto{\pgfqpoint{3.150548in}{1.975231in}}{\pgfqpoint{3.158448in}{1.971959in}}{\pgfqpoint{3.166684in}{1.971959in}}%
\pgfpathclose%
\pgfusepath{stroke,fill}%
\end{pgfscope}%
\begin{pgfscope}%
\pgfpathrectangle{\pgfqpoint{0.100000in}{0.212622in}}{\pgfqpoint{3.696000in}{3.696000in}}%
\pgfusepath{clip}%
\pgfsetbuttcap%
\pgfsetroundjoin%
\definecolor{currentfill}{rgb}{0.121569,0.466667,0.705882}%
\pgfsetfillcolor{currentfill}%
\pgfsetfillopacity{0.628979}%
\pgfsetlinewidth{1.003750pt}%
\definecolor{currentstroke}{rgb}{0.121569,0.466667,0.705882}%
\pgfsetstrokecolor{currentstroke}%
\pgfsetstrokeopacity{0.628979}%
\pgfsetdash{}{0pt}%
\pgfpathmoveto{\pgfqpoint{0.875097in}{1.665698in}}%
\pgfpathcurveto{\pgfqpoint{0.883333in}{1.665698in}}{\pgfqpoint{0.891233in}{1.668970in}}{\pgfqpoint{0.897057in}{1.674794in}}%
\pgfpathcurveto{\pgfqpoint{0.902881in}{1.680618in}}{\pgfqpoint{0.906153in}{1.688518in}}{\pgfqpoint{0.906153in}{1.696754in}}%
\pgfpathcurveto{\pgfqpoint{0.906153in}{1.704990in}}{\pgfqpoint{0.902881in}{1.712890in}}{\pgfqpoint{0.897057in}{1.718714in}}%
\pgfpathcurveto{\pgfqpoint{0.891233in}{1.724538in}}{\pgfqpoint{0.883333in}{1.727811in}}{\pgfqpoint{0.875097in}{1.727811in}}%
\pgfpathcurveto{\pgfqpoint{0.866861in}{1.727811in}}{\pgfqpoint{0.858961in}{1.724538in}}{\pgfqpoint{0.853137in}{1.718714in}}%
\pgfpathcurveto{\pgfqpoint{0.847313in}{1.712890in}}{\pgfqpoint{0.844040in}{1.704990in}}{\pgfqpoint{0.844040in}{1.696754in}}%
\pgfpathcurveto{\pgfqpoint{0.844040in}{1.688518in}}{\pgfqpoint{0.847313in}{1.680618in}}{\pgfqpoint{0.853137in}{1.674794in}}%
\pgfpathcurveto{\pgfqpoint{0.858961in}{1.668970in}}{\pgfqpoint{0.866861in}{1.665698in}}{\pgfqpoint{0.875097in}{1.665698in}}%
\pgfpathclose%
\pgfusepath{stroke,fill}%
\end{pgfscope}%
\begin{pgfscope}%
\pgfpathrectangle{\pgfqpoint{0.100000in}{0.212622in}}{\pgfqpoint{3.696000in}{3.696000in}}%
\pgfusepath{clip}%
\pgfsetbuttcap%
\pgfsetroundjoin%
\definecolor{currentfill}{rgb}{0.121569,0.466667,0.705882}%
\pgfsetfillcolor{currentfill}%
\pgfsetfillopacity{0.629005}%
\pgfsetlinewidth{1.003750pt}%
\definecolor{currentstroke}{rgb}{0.121569,0.466667,0.705882}%
\pgfsetstrokecolor{currentstroke}%
\pgfsetstrokeopacity{0.629005}%
\pgfsetdash{}{0pt}%
\pgfpathmoveto{\pgfqpoint{0.879488in}{1.671038in}}%
\pgfpathcurveto{\pgfqpoint{0.887724in}{1.671038in}}{\pgfqpoint{0.895624in}{1.674310in}}{\pgfqpoint{0.901448in}{1.680134in}}%
\pgfpathcurveto{\pgfqpoint{0.907272in}{1.685958in}}{\pgfqpoint{0.910545in}{1.693858in}}{\pgfqpoint{0.910545in}{1.702094in}}%
\pgfpathcurveto{\pgfqpoint{0.910545in}{1.710330in}}{\pgfqpoint{0.907272in}{1.718231in}}{\pgfqpoint{0.901448in}{1.724054in}}%
\pgfpathcurveto{\pgfqpoint{0.895624in}{1.729878in}}{\pgfqpoint{0.887724in}{1.733151in}}{\pgfqpoint{0.879488in}{1.733151in}}%
\pgfpathcurveto{\pgfqpoint{0.871252in}{1.733151in}}{\pgfqpoint{0.863352in}{1.729878in}}{\pgfqpoint{0.857528in}{1.724054in}}%
\pgfpathcurveto{\pgfqpoint{0.851704in}{1.718231in}}{\pgfqpoint{0.848432in}{1.710330in}}{\pgfqpoint{0.848432in}{1.702094in}}%
\pgfpathcurveto{\pgfqpoint{0.848432in}{1.693858in}}{\pgfqpoint{0.851704in}{1.685958in}}{\pgfqpoint{0.857528in}{1.680134in}}%
\pgfpathcurveto{\pgfqpoint{0.863352in}{1.674310in}}{\pgfqpoint{0.871252in}{1.671038in}}{\pgfqpoint{0.879488in}{1.671038in}}%
\pgfpathclose%
\pgfusepath{stroke,fill}%
\end{pgfscope}%
\begin{pgfscope}%
\pgfpathrectangle{\pgfqpoint{0.100000in}{0.212622in}}{\pgfqpoint{3.696000in}{3.696000in}}%
\pgfusepath{clip}%
\pgfsetbuttcap%
\pgfsetroundjoin%
\definecolor{currentfill}{rgb}{0.121569,0.466667,0.705882}%
\pgfsetfillcolor{currentfill}%
\pgfsetfillopacity{0.629229}%
\pgfsetlinewidth{1.003750pt}%
\definecolor{currentstroke}{rgb}{0.121569,0.466667,0.705882}%
\pgfsetstrokecolor{currentstroke}%
\pgfsetstrokeopacity{0.629229}%
\pgfsetdash{}{0pt}%
\pgfpathmoveto{\pgfqpoint{0.877743in}{1.671294in}}%
\pgfpathcurveto{\pgfqpoint{0.885979in}{1.671294in}}{\pgfqpoint{0.893879in}{1.674566in}}{\pgfqpoint{0.899703in}{1.680390in}}%
\pgfpathcurveto{\pgfqpoint{0.905527in}{1.686214in}}{\pgfqpoint{0.908800in}{1.694114in}}{\pgfqpoint{0.908800in}{1.702350in}}%
\pgfpathcurveto{\pgfqpoint{0.908800in}{1.710586in}}{\pgfqpoint{0.905527in}{1.718486in}}{\pgfqpoint{0.899703in}{1.724310in}}%
\pgfpathcurveto{\pgfqpoint{0.893879in}{1.730134in}}{\pgfqpoint{0.885979in}{1.733407in}}{\pgfqpoint{0.877743in}{1.733407in}}%
\pgfpathcurveto{\pgfqpoint{0.869507in}{1.733407in}}{\pgfqpoint{0.861607in}{1.730134in}}{\pgfqpoint{0.855783in}{1.724310in}}%
\pgfpathcurveto{\pgfqpoint{0.849959in}{1.718486in}}{\pgfqpoint{0.846687in}{1.710586in}}{\pgfqpoint{0.846687in}{1.702350in}}%
\pgfpathcurveto{\pgfqpoint{0.846687in}{1.694114in}}{\pgfqpoint{0.849959in}{1.686214in}}{\pgfqpoint{0.855783in}{1.680390in}}%
\pgfpathcurveto{\pgfqpoint{0.861607in}{1.674566in}}{\pgfqpoint{0.869507in}{1.671294in}}{\pgfqpoint{0.877743in}{1.671294in}}%
\pgfpathclose%
\pgfusepath{stroke,fill}%
\end{pgfscope}%
\begin{pgfscope}%
\pgfpathrectangle{\pgfqpoint{0.100000in}{0.212622in}}{\pgfqpoint{3.696000in}{3.696000in}}%
\pgfusepath{clip}%
\pgfsetbuttcap%
\pgfsetroundjoin%
\definecolor{currentfill}{rgb}{0.121569,0.466667,0.705882}%
\pgfsetfillcolor{currentfill}%
\pgfsetfillopacity{0.629646}%
\pgfsetlinewidth{1.003750pt}%
\definecolor{currentstroke}{rgb}{0.121569,0.466667,0.705882}%
\pgfsetstrokecolor{currentstroke}%
\pgfsetstrokeopacity{0.629646}%
\pgfsetdash{}{0pt}%
\pgfpathmoveto{\pgfqpoint{0.874005in}{1.664643in}}%
\pgfpathcurveto{\pgfqpoint{0.882241in}{1.664643in}}{\pgfqpoint{0.890141in}{1.667915in}}{\pgfqpoint{0.895965in}{1.673739in}}%
\pgfpathcurveto{\pgfqpoint{0.901789in}{1.679563in}}{\pgfqpoint{0.905061in}{1.687463in}}{\pgfqpoint{0.905061in}{1.695699in}}%
\pgfpathcurveto{\pgfqpoint{0.905061in}{1.703935in}}{\pgfqpoint{0.901789in}{1.711835in}}{\pgfqpoint{0.895965in}{1.717659in}}%
\pgfpathcurveto{\pgfqpoint{0.890141in}{1.723483in}}{\pgfqpoint{0.882241in}{1.726756in}}{\pgfqpoint{0.874005in}{1.726756in}}%
\pgfpathcurveto{\pgfqpoint{0.865769in}{1.726756in}}{\pgfqpoint{0.857868in}{1.723483in}}{\pgfqpoint{0.852045in}{1.717659in}}%
\pgfpathcurveto{\pgfqpoint{0.846221in}{1.711835in}}{\pgfqpoint{0.842948in}{1.703935in}}{\pgfqpoint{0.842948in}{1.695699in}}%
\pgfpathcurveto{\pgfqpoint{0.842948in}{1.687463in}}{\pgfqpoint{0.846221in}{1.679563in}}{\pgfqpoint{0.852045in}{1.673739in}}%
\pgfpathcurveto{\pgfqpoint{0.857868in}{1.667915in}}{\pgfqpoint{0.865769in}{1.664643in}}{\pgfqpoint{0.874005in}{1.664643in}}%
\pgfpathclose%
\pgfusepath{stroke,fill}%
\end{pgfscope}%
\begin{pgfscope}%
\pgfpathrectangle{\pgfqpoint{0.100000in}{0.212622in}}{\pgfqpoint{3.696000in}{3.696000in}}%
\pgfusepath{clip}%
\pgfsetbuttcap%
\pgfsetroundjoin%
\definecolor{currentfill}{rgb}{0.121569,0.466667,0.705882}%
\pgfsetfillcolor{currentfill}%
\pgfsetfillopacity{0.630238}%
\pgfsetlinewidth{1.003750pt}%
\definecolor{currentstroke}{rgb}{0.121569,0.466667,0.705882}%
\pgfsetstrokecolor{currentstroke}%
\pgfsetstrokeopacity{0.630238}%
\pgfsetdash{}{0pt}%
\pgfpathmoveto{\pgfqpoint{3.165709in}{1.972196in}}%
\pgfpathcurveto{\pgfqpoint{3.173945in}{1.972196in}}{\pgfqpoint{3.181845in}{1.975468in}}{\pgfqpoint{3.187669in}{1.981292in}}%
\pgfpathcurveto{\pgfqpoint{3.193493in}{1.987116in}}{\pgfqpoint{3.196765in}{1.995016in}}{\pgfqpoint{3.196765in}{2.003252in}}%
\pgfpathcurveto{\pgfqpoint{3.196765in}{2.011489in}}{\pgfqpoint{3.193493in}{2.019389in}}{\pgfqpoint{3.187669in}{2.025212in}}%
\pgfpathcurveto{\pgfqpoint{3.181845in}{2.031036in}}{\pgfqpoint{3.173945in}{2.034309in}}{\pgfqpoint{3.165709in}{2.034309in}}%
\pgfpathcurveto{\pgfqpoint{3.157473in}{2.034309in}}{\pgfqpoint{3.149573in}{2.031036in}}{\pgfqpoint{3.143749in}{2.025212in}}%
\pgfpathcurveto{\pgfqpoint{3.137925in}{2.019389in}}{\pgfqpoint{3.134652in}{2.011489in}}{\pgfqpoint{3.134652in}{2.003252in}}%
\pgfpathcurveto{\pgfqpoint{3.134652in}{1.995016in}}{\pgfqpoint{3.137925in}{1.987116in}}{\pgfqpoint{3.143749in}{1.981292in}}%
\pgfpathcurveto{\pgfqpoint{3.149573in}{1.975468in}}{\pgfqpoint{3.157473in}{1.972196in}}{\pgfqpoint{3.165709in}{1.972196in}}%
\pgfpathclose%
\pgfusepath{stroke,fill}%
\end{pgfscope}%
\begin{pgfscope}%
\pgfpathrectangle{\pgfqpoint{0.100000in}{0.212622in}}{\pgfqpoint{3.696000in}{3.696000in}}%
\pgfusepath{clip}%
\pgfsetbuttcap%
\pgfsetroundjoin%
\definecolor{currentfill}{rgb}{0.121569,0.466667,0.705882}%
\pgfsetfillcolor{currentfill}%
\pgfsetfillopacity{0.630637}%
\pgfsetlinewidth{1.003750pt}%
\definecolor{currentstroke}{rgb}{0.121569,0.466667,0.705882}%
\pgfsetstrokecolor{currentstroke}%
\pgfsetstrokeopacity{0.630637}%
\pgfsetdash{}{0pt}%
\pgfpathmoveto{\pgfqpoint{0.870015in}{1.663620in}}%
\pgfpathcurveto{\pgfqpoint{0.878251in}{1.663620in}}{\pgfqpoint{0.886151in}{1.666892in}}{\pgfqpoint{0.891975in}{1.672716in}}%
\pgfpathcurveto{\pgfqpoint{0.897799in}{1.678540in}}{\pgfqpoint{0.901072in}{1.686440in}}{\pgfqpoint{0.901072in}{1.694676in}}%
\pgfpathcurveto{\pgfqpoint{0.901072in}{1.702913in}}{\pgfqpoint{0.897799in}{1.710813in}}{\pgfqpoint{0.891975in}{1.716637in}}%
\pgfpathcurveto{\pgfqpoint{0.886151in}{1.722461in}}{\pgfqpoint{0.878251in}{1.725733in}}{\pgfqpoint{0.870015in}{1.725733in}}%
\pgfpathcurveto{\pgfqpoint{0.861779in}{1.725733in}}{\pgfqpoint{0.853879in}{1.722461in}}{\pgfqpoint{0.848055in}{1.716637in}}%
\pgfpathcurveto{\pgfqpoint{0.842231in}{1.710813in}}{\pgfqpoint{0.838959in}{1.702913in}}{\pgfqpoint{0.838959in}{1.694676in}}%
\pgfpathcurveto{\pgfqpoint{0.838959in}{1.686440in}}{\pgfqpoint{0.842231in}{1.678540in}}{\pgfqpoint{0.848055in}{1.672716in}}%
\pgfpathcurveto{\pgfqpoint{0.853879in}{1.666892in}}{\pgfqpoint{0.861779in}{1.663620in}}{\pgfqpoint{0.870015in}{1.663620in}}%
\pgfpathclose%
\pgfusepath{stroke,fill}%
\end{pgfscope}%
\begin{pgfscope}%
\pgfpathrectangle{\pgfqpoint{0.100000in}{0.212622in}}{\pgfqpoint{3.696000in}{3.696000in}}%
\pgfusepath{clip}%
\pgfsetbuttcap%
\pgfsetroundjoin%
\definecolor{currentfill}{rgb}{0.121569,0.466667,0.705882}%
\pgfsetfillcolor{currentfill}%
\pgfsetfillopacity{0.630957}%
\pgfsetlinewidth{1.003750pt}%
\definecolor{currentstroke}{rgb}{0.121569,0.466667,0.705882}%
\pgfsetstrokecolor{currentstroke}%
\pgfsetstrokeopacity{0.630957}%
\pgfsetdash{}{0pt}%
\pgfpathmoveto{\pgfqpoint{3.164058in}{1.972514in}}%
\pgfpathcurveto{\pgfqpoint{3.172295in}{1.972514in}}{\pgfqpoint{3.180195in}{1.975787in}}{\pgfqpoint{3.186019in}{1.981611in}}%
\pgfpathcurveto{\pgfqpoint{3.191842in}{1.987435in}}{\pgfqpoint{3.195115in}{1.995335in}}{\pgfqpoint{3.195115in}{2.003571in}}%
\pgfpathcurveto{\pgfqpoint{3.195115in}{2.011807in}}{\pgfqpoint{3.191842in}{2.019707in}}{\pgfqpoint{3.186019in}{2.025531in}}%
\pgfpathcurveto{\pgfqpoint{3.180195in}{2.031355in}}{\pgfqpoint{3.172295in}{2.034627in}}{\pgfqpoint{3.164058in}{2.034627in}}%
\pgfpathcurveto{\pgfqpoint{3.155822in}{2.034627in}}{\pgfqpoint{3.147922in}{2.031355in}}{\pgfqpoint{3.142098in}{2.025531in}}%
\pgfpathcurveto{\pgfqpoint{3.136274in}{2.019707in}}{\pgfqpoint{3.133002in}{2.011807in}}{\pgfqpoint{3.133002in}{2.003571in}}%
\pgfpathcurveto{\pgfqpoint{3.133002in}{1.995335in}}{\pgfqpoint{3.136274in}{1.987435in}}{\pgfqpoint{3.142098in}{1.981611in}}%
\pgfpathcurveto{\pgfqpoint{3.147922in}{1.975787in}}{\pgfqpoint{3.155822in}{1.972514in}}{\pgfqpoint{3.164058in}{1.972514in}}%
\pgfpathclose%
\pgfusepath{stroke,fill}%
\end{pgfscope}%
\begin{pgfscope}%
\pgfpathrectangle{\pgfqpoint{0.100000in}{0.212622in}}{\pgfqpoint{3.696000in}{3.696000in}}%
\pgfusepath{clip}%
\pgfsetbuttcap%
\pgfsetroundjoin%
\definecolor{currentfill}{rgb}{0.121569,0.466667,0.705882}%
\pgfsetfillcolor{currentfill}%
\pgfsetfillopacity{0.631249}%
\pgfsetlinewidth{1.003750pt}%
\definecolor{currentstroke}{rgb}{0.121569,0.466667,0.705882}%
\pgfsetstrokecolor{currentstroke}%
\pgfsetstrokeopacity{0.631249}%
\pgfsetdash{}{0pt}%
\pgfpathmoveto{\pgfqpoint{3.163228in}{1.971813in}}%
\pgfpathcurveto{\pgfqpoint{3.171464in}{1.971813in}}{\pgfqpoint{3.179365in}{1.975085in}}{\pgfqpoint{3.185188in}{1.980909in}}%
\pgfpathcurveto{\pgfqpoint{3.191012in}{1.986733in}}{\pgfqpoint{3.194285in}{1.994633in}}{\pgfqpoint{3.194285in}{2.002869in}}%
\pgfpathcurveto{\pgfqpoint{3.194285in}{2.011106in}}{\pgfqpoint{3.191012in}{2.019006in}}{\pgfqpoint{3.185188in}{2.024830in}}%
\pgfpathcurveto{\pgfqpoint{3.179365in}{2.030654in}}{\pgfqpoint{3.171464in}{2.033926in}}{\pgfqpoint{3.163228in}{2.033926in}}%
\pgfpathcurveto{\pgfqpoint{3.154992in}{2.033926in}}{\pgfqpoint{3.147092in}{2.030654in}}{\pgfqpoint{3.141268in}{2.024830in}}%
\pgfpathcurveto{\pgfqpoint{3.135444in}{2.019006in}}{\pgfqpoint{3.132172in}{2.011106in}}{\pgfqpoint{3.132172in}{2.002869in}}%
\pgfpathcurveto{\pgfqpoint{3.132172in}{1.994633in}}{\pgfqpoint{3.135444in}{1.986733in}}{\pgfqpoint{3.141268in}{1.980909in}}%
\pgfpathcurveto{\pgfqpoint{3.147092in}{1.975085in}}{\pgfqpoint{3.154992in}{1.971813in}}{\pgfqpoint{3.163228in}{1.971813in}}%
\pgfpathclose%
\pgfusepath{stroke,fill}%
\end{pgfscope}%
\begin{pgfscope}%
\pgfpathrectangle{\pgfqpoint{0.100000in}{0.212622in}}{\pgfqpoint{3.696000in}{3.696000in}}%
\pgfusepath{clip}%
\pgfsetbuttcap%
\pgfsetroundjoin%
\definecolor{currentfill}{rgb}{0.121569,0.466667,0.705882}%
\pgfsetfillcolor{currentfill}%
\pgfsetfillopacity{0.631402}%
\pgfsetlinewidth{1.003750pt}%
\definecolor{currentstroke}{rgb}{0.121569,0.466667,0.705882}%
\pgfsetstrokecolor{currentstroke}%
\pgfsetstrokeopacity{0.631402}%
\pgfsetdash{}{0pt}%
\pgfpathmoveto{\pgfqpoint{0.868829in}{1.662902in}}%
\pgfpathcurveto{\pgfqpoint{0.877065in}{1.662902in}}{\pgfqpoint{0.884965in}{1.666174in}}{\pgfqpoint{0.890789in}{1.671998in}}%
\pgfpathcurveto{\pgfqpoint{0.896613in}{1.677822in}}{\pgfqpoint{0.899885in}{1.685722in}}{\pgfqpoint{0.899885in}{1.693958in}}%
\pgfpathcurveto{\pgfqpoint{0.899885in}{1.702195in}}{\pgfqpoint{0.896613in}{1.710095in}}{\pgfqpoint{0.890789in}{1.715919in}}%
\pgfpathcurveto{\pgfqpoint{0.884965in}{1.721743in}}{\pgfqpoint{0.877065in}{1.725015in}}{\pgfqpoint{0.868829in}{1.725015in}}%
\pgfpathcurveto{\pgfqpoint{0.860593in}{1.725015in}}{\pgfqpoint{0.852692in}{1.721743in}}{\pgfqpoint{0.846869in}{1.715919in}}%
\pgfpathcurveto{\pgfqpoint{0.841045in}{1.710095in}}{\pgfqpoint{0.837772in}{1.702195in}}{\pgfqpoint{0.837772in}{1.693958in}}%
\pgfpathcurveto{\pgfqpoint{0.837772in}{1.685722in}}{\pgfqpoint{0.841045in}{1.677822in}}{\pgfqpoint{0.846869in}{1.671998in}}%
\pgfpathcurveto{\pgfqpoint{0.852692in}{1.666174in}}{\pgfqpoint{0.860593in}{1.662902in}}{\pgfqpoint{0.868829in}{1.662902in}}%
\pgfpathclose%
\pgfusepath{stroke,fill}%
\end{pgfscope}%
\begin{pgfscope}%
\pgfpathrectangle{\pgfqpoint{0.100000in}{0.212622in}}{\pgfqpoint{3.696000in}{3.696000in}}%
\pgfusepath{clip}%
\pgfsetbuttcap%
\pgfsetroundjoin%
\definecolor{currentfill}{rgb}{0.121569,0.466667,0.705882}%
\pgfsetfillcolor{currentfill}%
\pgfsetfillopacity{0.631817}%
\pgfsetlinewidth{1.003750pt}%
\definecolor{currentstroke}{rgb}{0.121569,0.466667,0.705882}%
\pgfsetstrokecolor{currentstroke}%
\pgfsetstrokeopacity{0.631817}%
\pgfsetdash{}{0pt}%
\pgfpathmoveto{\pgfqpoint{0.866576in}{1.662444in}}%
\pgfpathcurveto{\pgfqpoint{0.874813in}{1.662444in}}{\pgfqpoint{0.882713in}{1.665716in}}{\pgfqpoint{0.888537in}{1.671540in}}%
\pgfpathcurveto{\pgfqpoint{0.894361in}{1.677364in}}{\pgfqpoint{0.897633in}{1.685264in}}{\pgfqpoint{0.897633in}{1.693501in}}%
\pgfpathcurveto{\pgfqpoint{0.897633in}{1.701737in}}{\pgfqpoint{0.894361in}{1.709637in}}{\pgfqpoint{0.888537in}{1.715461in}}%
\pgfpathcurveto{\pgfqpoint{0.882713in}{1.721285in}}{\pgfqpoint{0.874813in}{1.724557in}}{\pgfqpoint{0.866576in}{1.724557in}}%
\pgfpathcurveto{\pgfqpoint{0.858340in}{1.724557in}}{\pgfqpoint{0.850440in}{1.721285in}}{\pgfqpoint{0.844616in}{1.715461in}}%
\pgfpathcurveto{\pgfqpoint{0.838792in}{1.709637in}}{\pgfqpoint{0.835520in}{1.701737in}}{\pgfqpoint{0.835520in}{1.693501in}}%
\pgfpathcurveto{\pgfqpoint{0.835520in}{1.685264in}}{\pgfqpoint{0.838792in}{1.677364in}}{\pgfqpoint{0.844616in}{1.671540in}}%
\pgfpathcurveto{\pgfqpoint{0.850440in}{1.665716in}}{\pgfqpoint{0.858340in}{1.662444in}}{\pgfqpoint{0.866576in}{1.662444in}}%
\pgfpathclose%
\pgfusepath{stroke,fill}%
\end{pgfscope}%
\begin{pgfscope}%
\pgfpathrectangle{\pgfqpoint{0.100000in}{0.212622in}}{\pgfqpoint{3.696000in}{3.696000in}}%
\pgfusepath{clip}%
\pgfsetbuttcap%
\pgfsetroundjoin%
\definecolor{currentfill}{rgb}{0.121569,0.466667,0.705882}%
\pgfsetfillcolor{currentfill}%
\pgfsetfillopacity{0.631970}%
\pgfsetlinewidth{1.003750pt}%
\definecolor{currentstroke}{rgb}{0.121569,0.466667,0.705882}%
\pgfsetstrokecolor{currentstroke}%
\pgfsetstrokeopacity{0.631970}%
\pgfsetdash{}{0pt}%
\pgfpathmoveto{\pgfqpoint{3.162044in}{1.970863in}}%
\pgfpathcurveto{\pgfqpoint{3.170280in}{1.970863in}}{\pgfqpoint{3.178180in}{1.974135in}}{\pgfqpoint{3.184004in}{1.979959in}}%
\pgfpathcurveto{\pgfqpoint{3.189828in}{1.985783in}}{\pgfqpoint{3.193101in}{1.993683in}}{\pgfqpoint{3.193101in}{2.001919in}}%
\pgfpathcurveto{\pgfqpoint{3.193101in}{2.010155in}}{\pgfqpoint{3.189828in}{2.018055in}}{\pgfqpoint{3.184004in}{2.023879in}}%
\pgfpathcurveto{\pgfqpoint{3.178180in}{2.029703in}}{\pgfqpoint{3.170280in}{2.032976in}}{\pgfqpoint{3.162044in}{2.032976in}}%
\pgfpathcurveto{\pgfqpoint{3.153808in}{2.032976in}}{\pgfqpoint{3.145908in}{2.029703in}}{\pgfqpoint{3.140084in}{2.023879in}}%
\pgfpathcurveto{\pgfqpoint{3.134260in}{2.018055in}}{\pgfqpoint{3.130988in}{2.010155in}}{\pgfqpoint{3.130988in}{2.001919in}}%
\pgfpathcurveto{\pgfqpoint{3.130988in}{1.993683in}}{\pgfqpoint{3.134260in}{1.985783in}}{\pgfqpoint{3.140084in}{1.979959in}}%
\pgfpathcurveto{\pgfqpoint{3.145908in}{1.974135in}}{\pgfqpoint{3.153808in}{1.970863in}}{\pgfqpoint{3.162044in}{1.970863in}}%
\pgfpathclose%
\pgfusepath{stroke,fill}%
\end{pgfscope}%
\begin{pgfscope}%
\pgfpathrectangle{\pgfqpoint{0.100000in}{0.212622in}}{\pgfqpoint{3.696000in}{3.696000in}}%
\pgfusepath{clip}%
\pgfsetbuttcap%
\pgfsetroundjoin%
\definecolor{currentfill}{rgb}{0.121569,0.466667,0.705882}%
\pgfsetfillcolor{currentfill}%
\pgfsetfillopacity{0.632109}%
\pgfsetlinewidth{1.003750pt}%
\definecolor{currentstroke}{rgb}{0.121569,0.466667,0.705882}%
\pgfsetstrokecolor{currentstroke}%
\pgfsetstrokeopacity{0.632109}%
\pgfsetdash{}{0pt}%
\pgfpathmoveto{\pgfqpoint{0.866039in}{1.662440in}}%
\pgfpathcurveto{\pgfqpoint{0.874275in}{1.662440in}}{\pgfqpoint{0.882175in}{1.665713in}}{\pgfqpoint{0.887999in}{1.671537in}}%
\pgfpathcurveto{\pgfqpoint{0.893823in}{1.677361in}}{\pgfqpoint{0.897095in}{1.685261in}}{\pgfqpoint{0.897095in}{1.693497in}}%
\pgfpathcurveto{\pgfqpoint{0.897095in}{1.701733in}}{\pgfqpoint{0.893823in}{1.709633in}}{\pgfqpoint{0.887999in}{1.715457in}}%
\pgfpathcurveto{\pgfqpoint{0.882175in}{1.721281in}}{\pgfqpoint{0.874275in}{1.724553in}}{\pgfqpoint{0.866039in}{1.724553in}}%
\pgfpathcurveto{\pgfqpoint{0.857803in}{1.724553in}}{\pgfqpoint{0.849903in}{1.721281in}}{\pgfqpoint{0.844079in}{1.715457in}}%
\pgfpathcurveto{\pgfqpoint{0.838255in}{1.709633in}}{\pgfqpoint{0.834982in}{1.701733in}}{\pgfqpoint{0.834982in}{1.693497in}}%
\pgfpathcurveto{\pgfqpoint{0.834982in}{1.685261in}}{\pgfqpoint{0.838255in}{1.677361in}}{\pgfqpoint{0.844079in}{1.671537in}}%
\pgfpathcurveto{\pgfqpoint{0.849903in}{1.665713in}}{\pgfqpoint{0.857803in}{1.662440in}}{\pgfqpoint{0.866039in}{1.662440in}}%
\pgfpathclose%
\pgfusepath{stroke,fill}%
\end{pgfscope}%
\begin{pgfscope}%
\pgfpathrectangle{\pgfqpoint{0.100000in}{0.212622in}}{\pgfqpoint{3.696000in}{3.696000in}}%
\pgfusepath{clip}%
\pgfsetbuttcap%
\pgfsetroundjoin%
\definecolor{currentfill}{rgb}{0.121569,0.466667,0.705882}%
\pgfsetfillcolor{currentfill}%
\pgfsetfillopacity{0.632414}%
\pgfsetlinewidth{1.003750pt}%
\definecolor{currentstroke}{rgb}{0.121569,0.466667,0.705882}%
\pgfsetstrokecolor{currentstroke}%
\pgfsetstrokeopacity{0.632414}%
\pgfsetdash{}{0pt}%
\pgfpathmoveto{\pgfqpoint{3.161331in}{1.970741in}}%
\pgfpathcurveto{\pgfqpoint{3.169567in}{1.970741in}}{\pgfqpoint{3.177467in}{1.974014in}}{\pgfqpoint{3.183291in}{1.979837in}}%
\pgfpathcurveto{\pgfqpoint{3.189115in}{1.985661in}}{\pgfqpoint{3.192387in}{1.993561in}}{\pgfqpoint{3.192387in}{2.001798in}}%
\pgfpathcurveto{\pgfqpoint{3.192387in}{2.010034in}}{\pgfqpoint{3.189115in}{2.017934in}}{\pgfqpoint{3.183291in}{2.023758in}}%
\pgfpathcurveto{\pgfqpoint{3.177467in}{2.029582in}}{\pgfqpoint{3.169567in}{2.032854in}}{\pgfqpoint{3.161331in}{2.032854in}}%
\pgfpathcurveto{\pgfqpoint{3.153094in}{2.032854in}}{\pgfqpoint{3.145194in}{2.029582in}}{\pgfqpoint{3.139370in}{2.023758in}}%
\pgfpathcurveto{\pgfqpoint{3.133546in}{2.017934in}}{\pgfqpoint{3.130274in}{2.010034in}}{\pgfqpoint{3.130274in}{2.001798in}}%
\pgfpathcurveto{\pgfqpoint{3.130274in}{1.993561in}}{\pgfqpoint{3.133546in}{1.985661in}}{\pgfqpoint{3.139370in}{1.979837in}}%
\pgfpathcurveto{\pgfqpoint{3.145194in}{1.974014in}}{\pgfqpoint{3.153094in}{1.970741in}}{\pgfqpoint{3.161331in}{1.970741in}}%
\pgfpathclose%
\pgfusepath{stroke,fill}%
\end{pgfscope}%
\begin{pgfscope}%
\pgfpathrectangle{\pgfqpoint{0.100000in}{0.212622in}}{\pgfqpoint{3.696000in}{3.696000in}}%
\pgfusepath{clip}%
\pgfsetbuttcap%
\pgfsetroundjoin%
\definecolor{currentfill}{rgb}{0.121569,0.466667,0.705882}%
\pgfsetfillcolor{currentfill}%
\pgfsetfillopacity{0.632438}%
\pgfsetlinewidth{1.003750pt}%
\definecolor{currentstroke}{rgb}{0.121569,0.466667,0.705882}%
\pgfsetstrokecolor{currentstroke}%
\pgfsetstrokeopacity{0.632438}%
\pgfsetdash{}{0pt}%
\pgfpathmoveto{\pgfqpoint{0.864221in}{1.662661in}}%
\pgfpathcurveto{\pgfqpoint{0.872457in}{1.662661in}}{\pgfqpoint{0.880357in}{1.665933in}}{\pgfqpoint{0.886181in}{1.671757in}}%
\pgfpathcurveto{\pgfqpoint{0.892005in}{1.677581in}}{\pgfqpoint{0.895277in}{1.685481in}}{\pgfqpoint{0.895277in}{1.693718in}}%
\pgfpathcurveto{\pgfqpoint{0.895277in}{1.701954in}}{\pgfqpoint{0.892005in}{1.709854in}}{\pgfqpoint{0.886181in}{1.715678in}}%
\pgfpathcurveto{\pgfqpoint{0.880357in}{1.721502in}}{\pgfqpoint{0.872457in}{1.724774in}}{\pgfqpoint{0.864221in}{1.724774in}}%
\pgfpathcurveto{\pgfqpoint{0.855985in}{1.724774in}}{\pgfqpoint{0.848084in}{1.721502in}}{\pgfqpoint{0.842261in}{1.715678in}}%
\pgfpathcurveto{\pgfqpoint{0.836437in}{1.709854in}}{\pgfqpoint{0.833164in}{1.701954in}}{\pgfqpoint{0.833164in}{1.693718in}}%
\pgfpathcurveto{\pgfqpoint{0.833164in}{1.685481in}}{\pgfqpoint{0.836437in}{1.677581in}}{\pgfqpoint{0.842261in}{1.671757in}}%
\pgfpathcurveto{\pgfqpoint{0.848084in}{1.665933in}}{\pgfqpoint{0.855985in}{1.662661in}}{\pgfqpoint{0.864221in}{1.662661in}}%
\pgfpathclose%
\pgfusepath{stroke,fill}%
\end{pgfscope}%
\begin{pgfscope}%
\pgfpathrectangle{\pgfqpoint{0.100000in}{0.212622in}}{\pgfqpoint{3.696000in}{3.696000in}}%
\pgfusepath{clip}%
\pgfsetbuttcap%
\pgfsetroundjoin%
\definecolor{currentfill}{rgb}{0.121569,0.466667,0.705882}%
\pgfsetfillcolor{currentfill}%
\pgfsetfillopacity{0.632643}%
\pgfsetlinewidth{1.003750pt}%
\definecolor{currentstroke}{rgb}{0.121569,0.466667,0.705882}%
\pgfsetstrokecolor{currentstroke}%
\pgfsetstrokeopacity{0.632643}%
\pgfsetdash{}{0pt}%
\pgfpathmoveto{\pgfqpoint{0.863210in}{1.661553in}}%
\pgfpathcurveto{\pgfqpoint{0.871446in}{1.661553in}}{\pgfqpoint{0.879346in}{1.664825in}}{\pgfqpoint{0.885170in}{1.670649in}}%
\pgfpathcurveto{\pgfqpoint{0.890994in}{1.676473in}}{\pgfqpoint{0.894266in}{1.684373in}}{\pgfqpoint{0.894266in}{1.692609in}}%
\pgfpathcurveto{\pgfqpoint{0.894266in}{1.700846in}}{\pgfqpoint{0.890994in}{1.708746in}}{\pgfqpoint{0.885170in}{1.714570in}}%
\pgfpathcurveto{\pgfqpoint{0.879346in}{1.720394in}}{\pgfqpoint{0.871446in}{1.723666in}}{\pgfqpoint{0.863210in}{1.723666in}}%
\pgfpathcurveto{\pgfqpoint{0.854973in}{1.723666in}}{\pgfqpoint{0.847073in}{1.720394in}}{\pgfqpoint{0.841249in}{1.714570in}}%
\pgfpathcurveto{\pgfqpoint{0.835425in}{1.708746in}}{\pgfqpoint{0.832153in}{1.700846in}}{\pgfqpoint{0.832153in}{1.692609in}}%
\pgfpathcurveto{\pgfqpoint{0.832153in}{1.684373in}}{\pgfqpoint{0.835425in}{1.676473in}}{\pgfqpoint{0.841249in}{1.670649in}}%
\pgfpathcurveto{\pgfqpoint{0.847073in}{1.664825in}}{\pgfqpoint{0.854973in}{1.661553in}}{\pgfqpoint{0.863210in}{1.661553in}}%
\pgfpathclose%
\pgfusepath{stroke,fill}%
\end{pgfscope}%
\begin{pgfscope}%
\pgfpathrectangle{\pgfqpoint{0.100000in}{0.212622in}}{\pgfqpoint{3.696000in}{3.696000in}}%
\pgfusepath{clip}%
\pgfsetbuttcap%
\pgfsetroundjoin%
\definecolor{currentfill}{rgb}{0.121569,0.466667,0.705882}%
\pgfsetfillcolor{currentfill}%
\pgfsetfillopacity{0.632890}%
\pgfsetlinewidth{1.003750pt}%
\definecolor{currentstroke}{rgb}{0.121569,0.466667,0.705882}%
\pgfsetstrokecolor{currentstroke}%
\pgfsetstrokeopacity{0.632890}%
\pgfsetdash{}{0pt}%
\pgfpathmoveto{\pgfqpoint{3.160074in}{1.970735in}}%
\pgfpathcurveto{\pgfqpoint{3.168310in}{1.970735in}}{\pgfqpoint{3.176210in}{1.974007in}}{\pgfqpoint{3.182034in}{1.979831in}}%
\pgfpathcurveto{\pgfqpoint{3.187858in}{1.985655in}}{\pgfqpoint{3.191130in}{1.993555in}}{\pgfqpoint{3.191130in}{2.001791in}}%
\pgfpathcurveto{\pgfqpoint{3.191130in}{2.010028in}}{\pgfqpoint{3.187858in}{2.017928in}}{\pgfqpoint{3.182034in}{2.023752in}}%
\pgfpathcurveto{\pgfqpoint{3.176210in}{2.029576in}}{\pgfqpoint{3.168310in}{2.032848in}}{\pgfqpoint{3.160074in}{2.032848in}}%
\pgfpathcurveto{\pgfqpoint{3.151838in}{2.032848in}}{\pgfqpoint{3.143937in}{2.029576in}}{\pgfqpoint{3.138114in}{2.023752in}}%
\pgfpathcurveto{\pgfqpoint{3.132290in}{2.017928in}}{\pgfqpoint{3.129017in}{2.010028in}}{\pgfqpoint{3.129017in}{2.001791in}}%
\pgfpathcurveto{\pgfqpoint{3.129017in}{1.993555in}}{\pgfqpoint{3.132290in}{1.985655in}}{\pgfqpoint{3.138114in}{1.979831in}}%
\pgfpathcurveto{\pgfqpoint{3.143937in}{1.974007in}}{\pgfqpoint{3.151838in}{1.970735in}}{\pgfqpoint{3.160074in}{1.970735in}}%
\pgfpathclose%
\pgfusepath{stroke,fill}%
\end{pgfscope}%
\begin{pgfscope}%
\pgfpathrectangle{\pgfqpoint{0.100000in}{0.212622in}}{\pgfqpoint{3.696000in}{3.696000in}}%
\pgfusepath{clip}%
\pgfsetbuttcap%
\pgfsetroundjoin%
\definecolor{currentfill}{rgb}{0.121569,0.466667,0.705882}%
\pgfsetfillcolor{currentfill}%
\pgfsetfillopacity{0.633265}%
\pgfsetlinewidth{1.003750pt}%
\definecolor{currentstroke}{rgb}{0.121569,0.466667,0.705882}%
\pgfsetstrokecolor{currentstroke}%
\pgfsetstrokeopacity{0.633265}%
\pgfsetdash{}{0pt}%
\pgfpathmoveto{\pgfqpoint{0.862153in}{1.660388in}}%
\pgfpathcurveto{\pgfqpoint{0.870389in}{1.660388in}}{\pgfqpoint{0.878289in}{1.663660in}}{\pgfqpoint{0.884113in}{1.669484in}}%
\pgfpathcurveto{\pgfqpoint{0.889937in}{1.675308in}}{\pgfqpoint{0.893209in}{1.683208in}}{\pgfqpoint{0.893209in}{1.691444in}}%
\pgfpathcurveto{\pgfqpoint{0.893209in}{1.699681in}}{\pgfqpoint{0.889937in}{1.707581in}}{\pgfqpoint{0.884113in}{1.713405in}}%
\pgfpathcurveto{\pgfqpoint{0.878289in}{1.719228in}}{\pgfqpoint{0.870389in}{1.722501in}}{\pgfqpoint{0.862153in}{1.722501in}}%
\pgfpathcurveto{\pgfqpoint{0.853916in}{1.722501in}}{\pgfqpoint{0.846016in}{1.719228in}}{\pgfqpoint{0.840192in}{1.713405in}}%
\pgfpathcurveto{\pgfqpoint{0.834368in}{1.707581in}}{\pgfqpoint{0.831096in}{1.699681in}}{\pgfqpoint{0.831096in}{1.691444in}}%
\pgfpathcurveto{\pgfqpoint{0.831096in}{1.683208in}}{\pgfqpoint{0.834368in}{1.675308in}}{\pgfqpoint{0.840192in}{1.669484in}}%
\pgfpathcurveto{\pgfqpoint{0.846016in}{1.663660in}}{\pgfqpoint{0.853916in}{1.660388in}}{\pgfqpoint{0.862153in}{1.660388in}}%
\pgfpathclose%
\pgfusepath{stroke,fill}%
\end{pgfscope}%
\begin{pgfscope}%
\pgfpathrectangle{\pgfqpoint{0.100000in}{0.212622in}}{\pgfqpoint{3.696000in}{3.696000in}}%
\pgfusepath{clip}%
\pgfsetbuttcap%
\pgfsetroundjoin%
\definecolor{currentfill}{rgb}{0.121569,0.466667,0.705882}%
\pgfsetfillcolor{currentfill}%
\pgfsetfillopacity{0.633439}%
\pgfsetlinewidth{1.003750pt}%
\definecolor{currentstroke}{rgb}{0.121569,0.466667,0.705882}%
\pgfsetstrokecolor{currentstroke}%
\pgfsetstrokeopacity{0.633439}%
\pgfsetdash{}{0pt}%
\pgfpathmoveto{\pgfqpoint{3.159123in}{1.969265in}}%
\pgfpathcurveto{\pgfqpoint{3.167360in}{1.969265in}}{\pgfqpoint{3.175260in}{1.972537in}}{\pgfqpoint{3.181084in}{1.978361in}}%
\pgfpathcurveto{\pgfqpoint{3.186908in}{1.984185in}}{\pgfqpoint{3.190180in}{1.992085in}}{\pgfqpoint{3.190180in}{2.000321in}}%
\pgfpathcurveto{\pgfqpoint{3.190180in}{2.008558in}}{\pgfqpoint{3.186908in}{2.016458in}}{\pgfqpoint{3.181084in}{2.022282in}}%
\pgfpathcurveto{\pgfqpoint{3.175260in}{2.028106in}}{\pgfqpoint{3.167360in}{2.031378in}}{\pgfqpoint{3.159123in}{2.031378in}}%
\pgfpathcurveto{\pgfqpoint{3.150887in}{2.031378in}}{\pgfqpoint{3.142987in}{2.028106in}}{\pgfqpoint{3.137163in}{2.022282in}}%
\pgfpathcurveto{\pgfqpoint{3.131339in}{2.016458in}}{\pgfqpoint{3.128067in}{2.008558in}}{\pgfqpoint{3.128067in}{2.000321in}}%
\pgfpathcurveto{\pgfqpoint{3.128067in}{1.992085in}}{\pgfqpoint{3.131339in}{1.984185in}}{\pgfqpoint{3.137163in}{1.978361in}}%
\pgfpathcurveto{\pgfqpoint{3.142987in}{1.972537in}}{\pgfqpoint{3.150887in}{1.969265in}}{\pgfqpoint{3.159123in}{1.969265in}}%
\pgfpathclose%
\pgfusepath{stroke,fill}%
\end{pgfscope}%
\begin{pgfscope}%
\pgfpathrectangle{\pgfqpoint{0.100000in}{0.212622in}}{\pgfqpoint{3.696000in}{3.696000in}}%
\pgfusepath{clip}%
\pgfsetbuttcap%
\pgfsetroundjoin%
\definecolor{currentfill}{rgb}{0.121569,0.466667,0.705882}%
\pgfsetfillcolor{currentfill}%
\pgfsetfillopacity{0.633784}%
\pgfsetlinewidth{1.003750pt}%
\definecolor{currentstroke}{rgb}{0.121569,0.466667,0.705882}%
\pgfsetstrokecolor{currentstroke}%
\pgfsetstrokeopacity{0.633784}%
\pgfsetdash{}{0pt}%
\pgfpathmoveto{\pgfqpoint{3.158543in}{1.968819in}}%
\pgfpathcurveto{\pgfqpoint{3.166780in}{1.968819in}}{\pgfqpoint{3.174680in}{1.972091in}}{\pgfqpoint{3.180503in}{1.977915in}}%
\pgfpathcurveto{\pgfqpoint{3.186327in}{1.983739in}}{\pgfqpoint{3.189600in}{1.991639in}}{\pgfqpoint{3.189600in}{1.999875in}}%
\pgfpathcurveto{\pgfqpoint{3.189600in}{2.008111in}}{\pgfqpoint{3.186327in}{2.016011in}}{\pgfqpoint{3.180503in}{2.021835in}}%
\pgfpathcurveto{\pgfqpoint{3.174680in}{2.027659in}}{\pgfqpoint{3.166780in}{2.030932in}}{\pgfqpoint{3.158543in}{2.030932in}}%
\pgfpathcurveto{\pgfqpoint{3.150307in}{2.030932in}}{\pgfqpoint{3.142407in}{2.027659in}}{\pgfqpoint{3.136583in}{2.021835in}}%
\pgfpathcurveto{\pgfqpoint{3.130759in}{2.016011in}}{\pgfqpoint{3.127487in}{2.008111in}}{\pgfqpoint{3.127487in}{1.999875in}}%
\pgfpathcurveto{\pgfqpoint{3.127487in}{1.991639in}}{\pgfqpoint{3.130759in}{1.983739in}}{\pgfqpoint{3.136583in}{1.977915in}}%
\pgfpathcurveto{\pgfqpoint{3.142407in}{1.972091in}}{\pgfqpoint{3.150307in}{1.968819in}}{\pgfqpoint{3.158543in}{1.968819in}}%
\pgfpathclose%
\pgfusepath{stroke,fill}%
\end{pgfscope}%
\begin{pgfscope}%
\pgfpathrectangle{\pgfqpoint{0.100000in}{0.212622in}}{\pgfqpoint{3.696000in}{3.696000in}}%
\pgfusepath{clip}%
\pgfsetbuttcap%
\pgfsetroundjoin%
\definecolor{currentfill}{rgb}{0.121569,0.466667,0.705882}%
\pgfsetfillcolor{currentfill}%
\pgfsetfillopacity{0.634242}%
\pgfsetlinewidth{1.003750pt}%
\definecolor{currentstroke}{rgb}{0.121569,0.466667,0.705882}%
\pgfsetstrokecolor{currentstroke}%
\pgfsetstrokeopacity{0.634242}%
\pgfsetdash{}{0pt}%
\pgfpathmoveto{\pgfqpoint{0.858198in}{1.659753in}}%
\pgfpathcurveto{\pgfqpoint{0.866434in}{1.659753in}}{\pgfqpoint{0.874334in}{1.663025in}}{\pgfqpoint{0.880158in}{1.668849in}}%
\pgfpathcurveto{\pgfqpoint{0.885982in}{1.674673in}}{\pgfqpoint{0.889254in}{1.682573in}}{\pgfqpoint{0.889254in}{1.690810in}}%
\pgfpathcurveto{\pgfqpoint{0.889254in}{1.699046in}}{\pgfqpoint{0.885982in}{1.706946in}}{\pgfqpoint{0.880158in}{1.712770in}}%
\pgfpathcurveto{\pgfqpoint{0.874334in}{1.718594in}}{\pgfqpoint{0.866434in}{1.721866in}}{\pgfqpoint{0.858198in}{1.721866in}}%
\pgfpathcurveto{\pgfqpoint{0.849962in}{1.721866in}}{\pgfqpoint{0.842062in}{1.718594in}}{\pgfqpoint{0.836238in}{1.712770in}}%
\pgfpathcurveto{\pgfqpoint{0.830414in}{1.706946in}}{\pgfqpoint{0.827141in}{1.699046in}}{\pgfqpoint{0.827141in}{1.690810in}}%
\pgfpathcurveto{\pgfqpoint{0.827141in}{1.682573in}}{\pgfqpoint{0.830414in}{1.674673in}}{\pgfqpoint{0.836238in}{1.668849in}}%
\pgfpathcurveto{\pgfqpoint{0.842062in}{1.663025in}}{\pgfqpoint{0.849962in}{1.659753in}}{\pgfqpoint{0.858198in}{1.659753in}}%
\pgfpathclose%
\pgfusepath{stroke,fill}%
\end{pgfscope}%
\begin{pgfscope}%
\pgfpathrectangle{\pgfqpoint{0.100000in}{0.212622in}}{\pgfqpoint{3.696000in}{3.696000in}}%
\pgfusepath{clip}%
\pgfsetbuttcap%
\pgfsetroundjoin%
\definecolor{currentfill}{rgb}{0.121569,0.466667,0.705882}%
\pgfsetfillcolor{currentfill}%
\pgfsetfillopacity{0.634592}%
\pgfsetlinewidth{1.003750pt}%
\definecolor{currentstroke}{rgb}{0.121569,0.466667,0.705882}%
\pgfsetstrokecolor{currentstroke}%
\pgfsetstrokeopacity{0.634592}%
\pgfsetdash{}{0pt}%
\pgfpathmoveto{\pgfqpoint{3.156834in}{1.968784in}}%
\pgfpathcurveto{\pgfqpoint{3.165070in}{1.968784in}}{\pgfqpoint{3.172970in}{1.972056in}}{\pgfqpoint{3.178794in}{1.977880in}}%
\pgfpathcurveto{\pgfqpoint{3.184618in}{1.983704in}}{\pgfqpoint{3.187890in}{1.991604in}}{\pgfqpoint{3.187890in}{1.999841in}}%
\pgfpathcurveto{\pgfqpoint{3.187890in}{2.008077in}}{\pgfqpoint{3.184618in}{2.015977in}}{\pgfqpoint{3.178794in}{2.021801in}}%
\pgfpathcurveto{\pgfqpoint{3.172970in}{2.027625in}}{\pgfqpoint{3.165070in}{2.030897in}}{\pgfqpoint{3.156834in}{2.030897in}}%
\pgfpathcurveto{\pgfqpoint{3.148597in}{2.030897in}}{\pgfqpoint{3.140697in}{2.027625in}}{\pgfqpoint{3.134873in}{2.021801in}}%
\pgfpathcurveto{\pgfqpoint{3.129050in}{2.015977in}}{\pgfqpoint{3.125777in}{2.008077in}}{\pgfqpoint{3.125777in}{1.999841in}}%
\pgfpathcurveto{\pgfqpoint{3.125777in}{1.991604in}}{\pgfqpoint{3.129050in}{1.983704in}}{\pgfqpoint{3.134873in}{1.977880in}}%
\pgfpathcurveto{\pgfqpoint{3.140697in}{1.972056in}}{\pgfqpoint{3.148597in}{1.968784in}}{\pgfqpoint{3.156834in}{1.968784in}}%
\pgfpathclose%
\pgfusepath{stroke,fill}%
\end{pgfscope}%
\begin{pgfscope}%
\pgfpathrectangle{\pgfqpoint{0.100000in}{0.212622in}}{\pgfqpoint{3.696000in}{3.696000in}}%
\pgfusepath{clip}%
\pgfsetbuttcap%
\pgfsetroundjoin%
\definecolor{currentfill}{rgb}{0.121569,0.466667,0.705882}%
\pgfsetfillcolor{currentfill}%
\pgfsetfillopacity{0.634828}%
\pgfsetlinewidth{1.003750pt}%
\definecolor{currentstroke}{rgb}{0.121569,0.466667,0.705882}%
\pgfsetstrokecolor{currentstroke}%
\pgfsetstrokeopacity{0.634828}%
\pgfsetdash{}{0pt}%
\pgfpathmoveto{\pgfqpoint{0.857031in}{1.658270in}}%
\pgfpathcurveto{\pgfqpoint{0.865268in}{1.658270in}}{\pgfqpoint{0.873168in}{1.661542in}}{\pgfqpoint{0.878992in}{1.667366in}}%
\pgfpathcurveto{\pgfqpoint{0.884815in}{1.673190in}}{\pgfqpoint{0.888088in}{1.681090in}}{\pgfqpoint{0.888088in}{1.689327in}}%
\pgfpathcurveto{\pgfqpoint{0.888088in}{1.697563in}}{\pgfqpoint{0.884815in}{1.705463in}}{\pgfqpoint{0.878992in}{1.711287in}}%
\pgfpathcurveto{\pgfqpoint{0.873168in}{1.717111in}}{\pgfqpoint{0.865268in}{1.720383in}}{\pgfqpoint{0.857031in}{1.720383in}}%
\pgfpathcurveto{\pgfqpoint{0.848795in}{1.720383in}}{\pgfqpoint{0.840895in}{1.717111in}}{\pgfqpoint{0.835071in}{1.711287in}}%
\pgfpathcurveto{\pgfqpoint{0.829247in}{1.705463in}}{\pgfqpoint{0.825975in}{1.697563in}}{\pgfqpoint{0.825975in}{1.689327in}}%
\pgfpathcurveto{\pgfqpoint{0.825975in}{1.681090in}}{\pgfqpoint{0.829247in}{1.673190in}}{\pgfqpoint{0.835071in}{1.667366in}}%
\pgfpathcurveto{\pgfqpoint{0.840895in}{1.661542in}}{\pgfqpoint{0.848795in}{1.658270in}}{\pgfqpoint{0.857031in}{1.658270in}}%
\pgfpathclose%
\pgfusepath{stroke,fill}%
\end{pgfscope}%
\begin{pgfscope}%
\pgfpathrectangle{\pgfqpoint{0.100000in}{0.212622in}}{\pgfqpoint{3.696000in}{3.696000in}}%
\pgfusepath{clip}%
\pgfsetbuttcap%
\pgfsetroundjoin%
\definecolor{currentfill}{rgb}{0.121569,0.466667,0.705882}%
\pgfsetfillcolor{currentfill}%
\pgfsetfillopacity{0.635248}%
\pgfsetlinewidth{1.003750pt}%
\definecolor{currentstroke}{rgb}{0.121569,0.466667,0.705882}%
\pgfsetstrokecolor{currentstroke}%
\pgfsetstrokeopacity{0.635248}%
\pgfsetdash{}{0pt}%
\pgfpathmoveto{\pgfqpoint{3.154868in}{1.966484in}}%
\pgfpathcurveto{\pgfqpoint{3.163104in}{1.966484in}}{\pgfqpoint{3.171004in}{1.969756in}}{\pgfqpoint{3.176828in}{1.975580in}}%
\pgfpathcurveto{\pgfqpoint{3.182652in}{1.981404in}}{\pgfqpoint{3.185924in}{1.989304in}}{\pgfqpoint{3.185924in}{1.997541in}}%
\pgfpathcurveto{\pgfqpoint{3.185924in}{2.005777in}}{\pgfqpoint{3.182652in}{2.013677in}}{\pgfqpoint{3.176828in}{2.019501in}}%
\pgfpathcurveto{\pgfqpoint{3.171004in}{2.025325in}}{\pgfqpoint{3.163104in}{2.028597in}}{\pgfqpoint{3.154868in}{2.028597in}}%
\pgfpathcurveto{\pgfqpoint{3.146631in}{2.028597in}}{\pgfqpoint{3.138731in}{2.025325in}}{\pgfqpoint{3.132907in}{2.019501in}}%
\pgfpathcurveto{\pgfqpoint{3.127083in}{2.013677in}}{\pgfqpoint{3.123811in}{2.005777in}}{\pgfqpoint{3.123811in}{1.997541in}}%
\pgfpathcurveto{\pgfqpoint{3.123811in}{1.989304in}}{\pgfqpoint{3.127083in}{1.981404in}}{\pgfqpoint{3.132907in}{1.975580in}}%
\pgfpathcurveto{\pgfqpoint{3.138731in}{1.969756in}}{\pgfqpoint{3.146631in}{1.966484in}}{\pgfqpoint{3.154868in}{1.966484in}}%
\pgfpathclose%
\pgfusepath{stroke,fill}%
\end{pgfscope}%
\begin{pgfscope}%
\pgfpathrectangle{\pgfqpoint{0.100000in}{0.212622in}}{\pgfqpoint{3.696000in}{3.696000in}}%
\pgfusepath{clip}%
\pgfsetbuttcap%
\pgfsetroundjoin%
\definecolor{currentfill}{rgb}{0.121569,0.466667,0.705882}%
\pgfsetfillcolor{currentfill}%
\pgfsetfillopacity{0.635495}%
\pgfsetlinewidth{1.003750pt}%
\definecolor{currentstroke}{rgb}{0.121569,0.466667,0.705882}%
\pgfsetstrokecolor{currentstroke}%
\pgfsetstrokeopacity{0.635495}%
\pgfsetdash{}{0pt}%
\pgfpathmoveto{\pgfqpoint{0.855813in}{1.658233in}}%
\pgfpathcurveto{\pgfqpoint{0.864050in}{1.658233in}}{\pgfqpoint{0.871950in}{1.661505in}}{\pgfqpoint{0.877774in}{1.667329in}}%
\pgfpathcurveto{\pgfqpoint{0.883598in}{1.673153in}}{\pgfqpoint{0.886870in}{1.681053in}}{\pgfqpoint{0.886870in}{1.689289in}}%
\pgfpathcurveto{\pgfqpoint{0.886870in}{1.697525in}}{\pgfqpoint{0.883598in}{1.705425in}}{\pgfqpoint{0.877774in}{1.711249in}}%
\pgfpathcurveto{\pgfqpoint{0.871950in}{1.717073in}}{\pgfqpoint{0.864050in}{1.720346in}}{\pgfqpoint{0.855813in}{1.720346in}}%
\pgfpathcurveto{\pgfqpoint{0.847577in}{1.720346in}}{\pgfqpoint{0.839677in}{1.717073in}}{\pgfqpoint{0.833853in}{1.711249in}}%
\pgfpathcurveto{\pgfqpoint{0.828029in}{1.705425in}}{\pgfqpoint{0.824757in}{1.697525in}}{\pgfqpoint{0.824757in}{1.689289in}}%
\pgfpathcurveto{\pgfqpoint{0.824757in}{1.681053in}}{\pgfqpoint{0.828029in}{1.673153in}}{\pgfqpoint{0.833853in}{1.667329in}}%
\pgfpathcurveto{\pgfqpoint{0.839677in}{1.661505in}}{\pgfqpoint{0.847577in}{1.658233in}}{\pgfqpoint{0.855813in}{1.658233in}}%
\pgfpathclose%
\pgfusepath{stroke,fill}%
\end{pgfscope}%
\begin{pgfscope}%
\pgfpathrectangle{\pgfqpoint{0.100000in}{0.212622in}}{\pgfqpoint{3.696000in}{3.696000in}}%
\pgfusepath{clip}%
\pgfsetbuttcap%
\pgfsetroundjoin%
\definecolor{currentfill}{rgb}{0.121569,0.466667,0.705882}%
\pgfsetfillcolor{currentfill}%
\pgfsetfillopacity{0.635862}%
\pgfsetlinewidth{1.003750pt}%
\definecolor{currentstroke}{rgb}{0.121569,0.466667,0.705882}%
\pgfsetstrokecolor{currentstroke}%
\pgfsetstrokeopacity{0.635862}%
\pgfsetdash{}{0pt}%
\pgfpathmoveto{\pgfqpoint{3.154499in}{1.966464in}}%
\pgfpathcurveto{\pgfqpoint{3.162736in}{1.966464in}}{\pgfqpoint{3.170636in}{1.969736in}}{\pgfqpoint{3.176460in}{1.975560in}}%
\pgfpathcurveto{\pgfqpoint{3.182284in}{1.981384in}}{\pgfqpoint{3.185556in}{1.989284in}}{\pgfqpoint{3.185556in}{1.997520in}}%
\pgfpathcurveto{\pgfqpoint{3.185556in}{2.005757in}}{\pgfqpoint{3.182284in}{2.013657in}}{\pgfqpoint{3.176460in}{2.019481in}}%
\pgfpathcurveto{\pgfqpoint{3.170636in}{2.025305in}}{\pgfqpoint{3.162736in}{2.028577in}}{\pgfqpoint{3.154499in}{2.028577in}}%
\pgfpathcurveto{\pgfqpoint{3.146263in}{2.028577in}}{\pgfqpoint{3.138363in}{2.025305in}}{\pgfqpoint{3.132539in}{2.019481in}}%
\pgfpathcurveto{\pgfqpoint{3.126715in}{2.013657in}}{\pgfqpoint{3.123443in}{2.005757in}}{\pgfqpoint{3.123443in}{1.997520in}}%
\pgfpathcurveto{\pgfqpoint{3.123443in}{1.989284in}}{\pgfqpoint{3.126715in}{1.981384in}}{\pgfqpoint{3.132539in}{1.975560in}}%
\pgfpathcurveto{\pgfqpoint{3.138363in}{1.969736in}}{\pgfqpoint{3.146263in}{1.966464in}}{\pgfqpoint{3.154499in}{1.966464in}}%
\pgfpathclose%
\pgfusepath{stroke,fill}%
\end{pgfscope}%
\begin{pgfscope}%
\pgfpathrectangle{\pgfqpoint{0.100000in}{0.212622in}}{\pgfqpoint{3.696000in}{3.696000in}}%
\pgfusepath{clip}%
\pgfsetbuttcap%
\pgfsetroundjoin%
\definecolor{currentfill}{rgb}{0.121569,0.466667,0.705882}%
\pgfsetfillcolor{currentfill}%
\pgfsetfillopacity{0.636715}%
\pgfsetlinewidth{1.003750pt}%
\definecolor{currentstroke}{rgb}{0.121569,0.466667,0.705882}%
\pgfsetstrokecolor{currentstroke}%
\pgfsetstrokeopacity{0.636715}%
\pgfsetdash{}{0pt}%
\pgfpathmoveto{\pgfqpoint{0.854905in}{1.662390in}}%
\pgfpathcurveto{\pgfqpoint{0.863142in}{1.662390in}}{\pgfqpoint{0.871042in}{1.665662in}}{\pgfqpoint{0.876866in}{1.671486in}}%
\pgfpathcurveto{\pgfqpoint{0.882689in}{1.677310in}}{\pgfqpoint{0.885962in}{1.685210in}}{\pgfqpoint{0.885962in}{1.693447in}}%
\pgfpathcurveto{\pgfqpoint{0.885962in}{1.701683in}}{\pgfqpoint{0.882689in}{1.709583in}}{\pgfqpoint{0.876866in}{1.715407in}}%
\pgfpathcurveto{\pgfqpoint{0.871042in}{1.721231in}}{\pgfqpoint{0.863142in}{1.724503in}}{\pgfqpoint{0.854905in}{1.724503in}}%
\pgfpathcurveto{\pgfqpoint{0.846669in}{1.724503in}}{\pgfqpoint{0.838769in}{1.721231in}}{\pgfqpoint{0.832945in}{1.715407in}}%
\pgfpathcurveto{\pgfqpoint{0.827121in}{1.709583in}}{\pgfqpoint{0.823849in}{1.701683in}}{\pgfqpoint{0.823849in}{1.693447in}}%
\pgfpathcurveto{\pgfqpoint{0.823849in}{1.685210in}}{\pgfqpoint{0.827121in}{1.677310in}}{\pgfqpoint{0.832945in}{1.671486in}}%
\pgfpathcurveto{\pgfqpoint{0.838769in}{1.665662in}}{\pgfqpoint{0.846669in}{1.662390in}}{\pgfqpoint{0.854905in}{1.662390in}}%
\pgfpathclose%
\pgfusepath{stroke,fill}%
\end{pgfscope}%
\begin{pgfscope}%
\pgfpathrectangle{\pgfqpoint{0.100000in}{0.212622in}}{\pgfqpoint{3.696000in}{3.696000in}}%
\pgfusepath{clip}%
\pgfsetbuttcap%
\pgfsetroundjoin%
\definecolor{currentfill}{rgb}{0.121569,0.466667,0.705882}%
\pgfsetfillcolor{currentfill}%
\pgfsetfillopacity{0.636762}%
\pgfsetlinewidth{1.003750pt}%
\definecolor{currentstroke}{rgb}{0.121569,0.466667,0.705882}%
\pgfsetstrokecolor{currentstroke}%
\pgfsetstrokeopacity{0.636762}%
\pgfsetdash{}{0pt}%
\pgfpathmoveto{\pgfqpoint{3.152335in}{1.966216in}}%
\pgfpathcurveto{\pgfqpoint{3.160571in}{1.966216in}}{\pgfqpoint{3.168471in}{1.969488in}}{\pgfqpoint{3.174295in}{1.975312in}}%
\pgfpathcurveto{\pgfqpoint{3.180119in}{1.981136in}}{\pgfqpoint{3.183391in}{1.989036in}}{\pgfqpoint{3.183391in}{1.997272in}}%
\pgfpathcurveto{\pgfqpoint{3.183391in}{2.005509in}}{\pgfqpoint{3.180119in}{2.013409in}}{\pgfqpoint{3.174295in}{2.019233in}}%
\pgfpathcurveto{\pgfqpoint{3.168471in}{2.025057in}}{\pgfqpoint{3.160571in}{2.028329in}}{\pgfqpoint{3.152335in}{2.028329in}}%
\pgfpathcurveto{\pgfqpoint{3.144099in}{2.028329in}}{\pgfqpoint{3.136199in}{2.025057in}}{\pgfqpoint{3.130375in}{2.019233in}}%
\pgfpathcurveto{\pgfqpoint{3.124551in}{2.013409in}}{\pgfqpoint{3.121278in}{2.005509in}}{\pgfqpoint{3.121278in}{1.997272in}}%
\pgfpathcurveto{\pgfqpoint{3.121278in}{1.989036in}}{\pgfqpoint{3.124551in}{1.981136in}}{\pgfqpoint{3.130375in}{1.975312in}}%
\pgfpathcurveto{\pgfqpoint{3.136199in}{1.969488in}}{\pgfqpoint{3.144099in}{1.966216in}}{\pgfqpoint{3.152335in}{1.966216in}}%
\pgfpathclose%
\pgfusepath{stroke,fill}%
\end{pgfscope}%
\begin{pgfscope}%
\pgfpathrectangle{\pgfqpoint{0.100000in}{0.212622in}}{\pgfqpoint{3.696000in}{3.696000in}}%
\pgfusepath{clip}%
\pgfsetbuttcap%
\pgfsetroundjoin%
\definecolor{currentfill}{rgb}{0.121569,0.466667,0.705882}%
\pgfsetfillcolor{currentfill}%
\pgfsetfillopacity{0.637549}%
\pgfsetlinewidth{1.003750pt}%
\definecolor{currentstroke}{rgb}{0.121569,0.466667,0.705882}%
\pgfsetstrokecolor{currentstroke}%
\pgfsetstrokeopacity{0.637549}%
\pgfsetdash{}{0pt}%
\pgfpathmoveto{\pgfqpoint{3.150413in}{1.963718in}}%
\pgfpathcurveto{\pgfqpoint{3.158650in}{1.963718in}}{\pgfqpoint{3.166550in}{1.966990in}}{\pgfqpoint{3.172374in}{1.972814in}}%
\pgfpathcurveto{\pgfqpoint{3.178198in}{1.978638in}}{\pgfqpoint{3.181470in}{1.986538in}}{\pgfqpoint{3.181470in}{1.994774in}}%
\pgfpathcurveto{\pgfqpoint{3.181470in}{2.003010in}}{\pgfqpoint{3.178198in}{2.010910in}}{\pgfqpoint{3.172374in}{2.016734in}}%
\pgfpathcurveto{\pgfqpoint{3.166550in}{2.022558in}}{\pgfqpoint{3.158650in}{2.025831in}}{\pgfqpoint{3.150413in}{2.025831in}}%
\pgfpathcurveto{\pgfqpoint{3.142177in}{2.025831in}}{\pgfqpoint{3.134277in}{2.022558in}}{\pgfqpoint{3.128453in}{2.016734in}}%
\pgfpathcurveto{\pgfqpoint{3.122629in}{2.010910in}}{\pgfqpoint{3.119357in}{2.003010in}}{\pgfqpoint{3.119357in}{1.994774in}}%
\pgfpathcurveto{\pgfqpoint{3.119357in}{1.986538in}}{\pgfqpoint{3.122629in}{1.978638in}}{\pgfqpoint{3.128453in}{1.972814in}}%
\pgfpathcurveto{\pgfqpoint{3.134277in}{1.966990in}}{\pgfqpoint{3.142177in}{1.963718in}}{\pgfqpoint{3.150413in}{1.963718in}}%
\pgfpathclose%
\pgfusepath{stroke,fill}%
\end{pgfscope}%
\begin{pgfscope}%
\pgfpathrectangle{\pgfqpoint{0.100000in}{0.212622in}}{\pgfqpoint{3.696000in}{3.696000in}}%
\pgfusepath{clip}%
\pgfsetbuttcap%
\pgfsetroundjoin%
\definecolor{currentfill}{rgb}{0.121569,0.466667,0.705882}%
\pgfsetfillcolor{currentfill}%
\pgfsetfillopacity{0.637592}%
\pgfsetlinewidth{1.003750pt}%
\definecolor{currentstroke}{rgb}{0.121569,0.466667,0.705882}%
\pgfsetstrokecolor{currentstroke}%
\pgfsetstrokeopacity{0.637592}%
\pgfsetdash{}{0pt}%
\pgfpathmoveto{\pgfqpoint{0.853744in}{1.660504in}}%
\pgfpathcurveto{\pgfqpoint{0.861980in}{1.660504in}}{\pgfqpoint{0.869881in}{1.663776in}}{\pgfqpoint{0.875704in}{1.669600in}}%
\pgfpathcurveto{\pgfqpoint{0.881528in}{1.675424in}}{\pgfqpoint{0.884801in}{1.683324in}}{\pgfqpoint{0.884801in}{1.691560in}}%
\pgfpathcurveto{\pgfqpoint{0.884801in}{1.699796in}}{\pgfqpoint{0.881528in}{1.707696in}}{\pgfqpoint{0.875704in}{1.713520in}}%
\pgfpathcurveto{\pgfqpoint{0.869881in}{1.719344in}}{\pgfqpoint{0.861980in}{1.722617in}}{\pgfqpoint{0.853744in}{1.722617in}}%
\pgfpathcurveto{\pgfqpoint{0.845508in}{1.722617in}}{\pgfqpoint{0.837608in}{1.719344in}}{\pgfqpoint{0.831784in}{1.713520in}}%
\pgfpathcurveto{\pgfqpoint{0.825960in}{1.707696in}}{\pgfqpoint{0.822688in}{1.699796in}}{\pgfqpoint{0.822688in}{1.691560in}}%
\pgfpathcurveto{\pgfqpoint{0.822688in}{1.683324in}}{\pgfqpoint{0.825960in}{1.675424in}}{\pgfqpoint{0.831784in}{1.669600in}}%
\pgfpathcurveto{\pgfqpoint{0.837608in}{1.663776in}}{\pgfqpoint{0.845508in}{1.660504in}}{\pgfqpoint{0.853744in}{1.660504in}}%
\pgfpathclose%
\pgfusepath{stroke,fill}%
\end{pgfscope}%
\begin{pgfscope}%
\pgfpathrectangle{\pgfqpoint{0.100000in}{0.212622in}}{\pgfqpoint{3.696000in}{3.696000in}}%
\pgfusepath{clip}%
\pgfsetbuttcap%
\pgfsetroundjoin%
\definecolor{currentfill}{rgb}{0.121569,0.466667,0.705882}%
\pgfsetfillcolor{currentfill}%
\pgfsetfillopacity{0.638352}%
\pgfsetlinewidth{1.003750pt}%
\definecolor{currentstroke}{rgb}{0.121569,0.466667,0.705882}%
\pgfsetstrokecolor{currentstroke}%
\pgfsetstrokeopacity{0.638352}%
\pgfsetdash{}{0pt}%
\pgfpathmoveto{\pgfqpoint{0.851228in}{1.660618in}}%
\pgfpathcurveto{\pgfqpoint{0.859465in}{1.660618in}}{\pgfqpoint{0.867365in}{1.663890in}}{\pgfqpoint{0.873189in}{1.669714in}}%
\pgfpathcurveto{\pgfqpoint{0.879012in}{1.675538in}}{\pgfqpoint{0.882285in}{1.683438in}}{\pgfqpoint{0.882285in}{1.691674in}}%
\pgfpathcurveto{\pgfqpoint{0.882285in}{1.699910in}}{\pgfqpoint{0.879012in}{1.707811in}}{\pgfqpoint{0.873189in}{1.713634in}}%
\pgfpathcurveto{\pgfqpoint{0.867365in}{1.719458in}}{\pgfqpoint{0.859465in}{1.722731in}}{\pgfqpoint{0.851228in}{1.722731in}}%
\pgfpathcurveto{\pgfqpoint{0.842992in}{1.722731in}}{\pgfqpoint{0.835092in}{1.719458in}}{\pgfqpoint{0.829268in}{1.713634in}}%
\pgfpathcurveto{\pgfqpoint{0.823444in}{1.707811in}}{\pgfqpoint{0.820172in}{1.699910in}}{\pgfqpoint{0.820172in}{1.691674in}}%
\pgfpathcurveto{\pgfqpoint{0.820172in}{1.683438in}}{\pgfqpoint{0.823444in}{1.675538in}}{\pgfqpoint{0.829268in}{1.669714in}}%
\pgfpathcurveto{\pgfqpoint{0.835092in}{1.663890in}}{\pgfqpoint{0.842992in}{1.660618in}}{\pgfqpoint{0.851228in}{1.660618in}}%
\pgfpathclose%
\pgfusepath{stroke,fill}%
\end{pgfscope}%
\begin{pgfscope}%
\pgfpathrectangle{\pgfqpoint{0.100000in}{0.212622in}}{\pgfqpoint{3.696000in}{3.696000in}}%
\pgfusepath{clip}%
\pgfsetbuttcap%
\pgfsetroundjoin%
\definecolor{currentfill}{rgb}{0.121569,0.466667,0.705882}%
\pgfsetfillcolor{currentfill}%
\pgfsetfillopacity{0.638769}%
\pgfsetlinewidth{1.003750pt}%
\definecolor{currentstroke}{rgb}{0.121569,0.466667,0.705882}%
\pgfsetstrokecolor{currentstroke}%
\pgfsetstrokeopacity{0.638769}%
\pgfsetdash{}{0pt}%
\pgfpathmoveto{\pgfqpoint{3.149020in}{1.963022in}}%
\pgfpathcurveto{\pgfqpoint{3.157256in}{1.963022in}}{\pgfqpoint{3.165156in}{1.966295in}}{\pgfqpoint{3.170980in}{1.972119in}}%
\pgfpathcurveto{\pgfqpoint{3.176804in}{1.977942in}}{\pgfqpoint{3.180076in}{1.985843in}}{\pgfqpoint{3.180076in}{1.994079in}}%
\pgfpathcurveto{\pgfqpoint{3.180076in}{2.002315in}}{\pgfqpoint{3.176804in}{2.010215in}}{\pgfqpoint{3.170980in}{2.016039in}}%
\pgfpathcurveto{\pgfqpoint{3.165156in}{2.021863in}}{\pgfqpoint{3.157256in}{2.025135in}}{\pgfqpoint{3.149020in}{2.025135in}}%
\pgfpathcurveto{\pgfqpoint{3.140784in}{2.025135in}}{\pgfqpoint{3.132884in}{2.021863in}}{\pgfqpoint{3.127060in}{2.016039in}}%
\pgfpathcurveto{\pgfqpoint{3.121236in}{2.010215in}}{\pgfqpoint{3.117963in}{2.002315in}}{\pgfqpoint{3.117963in}{1.994079in}}%
\pgfpathcurveto{\pgfqpoint{3.117963in}{1.985843in}}{\pgfqpoint{3.121236in}{1.977942in}}{\pgfqpoint{3.127060in}{1.972119in}}%
\pgfpathcurveto{\pgfqpoint{3.132884in}{1.966295in}}{\pgfqpoint{3.140784in}{1.963022in}}{\pgfqpoint{3.149020in}{1.963022in}}%
\pgfpathclose%
\pgfusepath{stroke,fill}%
\end{pgfscope}%
\begin{pgfscope}%
\pgfpathrectangle{\pgfqpoint{0.100000in}{0.212622in}}{\pgfqpoint{3.696000in}{3.696000in}}%
\pgfusepath{clip}%
\pgfsetbuttcap%
\pgfsetroundjoin%
\definecolor{currentfill}{rgb}{0.121569,0.466667,0.705882}%
\pgfsetfillcolor{currentfill}%
\pgfsetfillopacity{0.639057}%
\pgfsetlinewidth{1.003750pt}%
\definecolor{currentstroke}{rgb}{0.121569,0.466667,0.705882}%
\pgfsetstrokecolor{currentstroke}%
\pgfsetstrokeopacity{0.639057}%
\pgfsetdash{}{0pt}%
\pgfpathmoveto{\pgfqpoint{0.850854in}{1.661637in}}%
\pgfpathcurveto{\pgfqpoint{0.859090in}{1.661637in}}{\pgfqpoint{0.866990in}{1.664909in}}{\pgfqpoint{0.872814in}{1.670733in}}%
\pgfpathcurveto{\pgfqpoint{0.878638in}{1.676557in}}{\pgfqpoint{0.881911in}{1.684457in}}{\pgfqpoint{0.881911in}{1.692694in}}%
\pgfpathcurveto{\pgfqpoint{0.881911in}{1.700930in}}{\pgfqpoint{0.878638in}{1.708830in}}{\pgfqpoint{0.872814in}{1.714654in}}%
\pgfpathcurveto{\pgfqpoint{0.866990in}{1.720478in}}{\pgfqpoint{0.859090in}{1.723750in}}{\pgfqpoint{0.850854in}{1.723750in}}%
\pgfpathcurveto{\pgfqpoint{0.842618in}{1.723750in}}{\pgfqpoint{0.834718in}{1.720478in}}{\pgfqpoint{0.828894in}{1.714654in}}%
\pgfpathcurveto{\pgfqpoint{0.823070in}{1.708830in}}{\pgfqpoint{0.819798in}{1.700930in}}{\pgfqpoint{0.819798in}{1.692694in}}%
\pgfpathcurveto{\pgfqpoint{0.819798in}{1.684457in}}{\pgfqpoint{0.823070in}{1.676557in}}{\pgfqpoint{0.828894in}{1.670733in}}%
\pgfpathcurveto{\pgfqpoint{0.834718in}{1.664909in}}{\pgfqpoint{0.842618in}{1.661637in}}{\pgfqpoint{0.850854in}{1.661637in}}%
\pgfpathclose%
\pgfusepath{stroke,fill}%
\end{pgfscope}%
\begin{pgfscope}%
\pgfpathrectangle{\pgfqpoint{0.100000in}{0.212622in}}{\pgfqpoint{3.696000in}{3.696000in}}%
\pgfusepath{clip}%
\pgfsetbuttcap%
\pgfsetroundjoin%
\definecolor{currentfill}{rgb}{0.121569,0.466667,0.705882}%
\pgfsetfillcolor{currentfill}%
\pgfsetfillopacity{0.639198}%
\pgfsetlinewidth{1.003750pt}%
\definecolor{currentstroke}{rgb}{0.121569,0.466667,0.705882}%
\pgfsetstrokecolor{currentstroke}%
\pgfsetstrokeopacity{0.639198}%
\pgfsetdash{}{0pt}%
\pgfpathmoveto{\pgfqpoint{0.850164in}{1.660338in}}%
\pgfpathcurveto{\pgfqpoint{0.858400in}{1.660338in}}{\pgfqpoint{0.866300in}{1.663610in}}{\pgfqpoint{0.872124in}{1.669434in}}%
\pgfpathcurveto{\pgfqpoint{0.877948in}{1.675258in}}{\pgfqpoint{0.881220in}{1.683158in}}{\pgfqpoint{0.881220in}{1.691394in}}%
\pgfpathcurveto{\pgfqpoint{0.881220in}{1.699630in}}{\pgfqpoint{0.877948in}{1.707530in}}{\pgfqpoint{0.872124in}{1.713354in}}%
\pgfpathcurveto{\pgfqpoint{0.866300in}{1.719178in}}{\pgfqpoint{0.858400in}{1.722451in}}{\pgfqpoint{0.850164in}{1.722451in}}%
\pgfpathcurveto{\pgfqpoint{0.841927in}{1.722451in}}{\pgfqpoint{0.834027in}{1.719178in}}{\pgfqpoint{0.828203in}{1.713354in}}%
\pgfpathcurveto{\pgfqpoint{0.822380in}{1.707530in}}{\pgfqpoint{0.819107in}{1.699630in}}{\pgfqpoint{0.819107in}{1.691394in}}%
\pgfpathcurveto{\pgfqpoint{0.819107in}{1.683158in}}{\pgfqpoint{0.822380in}{1.675258in}}{\pgfqpoint{0.828203in}{1.669434in}}%
\pgfpathcurveto{\pgfqpoint{0.834027in}{1.663610in}}{\pgfqpoint{0.841927in}{1.660338in}}{\pgfqpoint{0.850164in}{1.660338in}}%
\pgfpathclose%
\pgfusepath{stroke,fill}%
\end{pgfscope}%
\begin{pgfscope}%
\pgfpathrectangle{\pgfqpoint{0.100000in}{0.212622in}}{\pgfqpoint{3.696000in}{3.696000in}}%
\pgfusepath{clip}%
\pgfsetbuttcap%
\pgfsetroundjoin%
\definecolor{currentfill}{rgb}{0.121569,0.466667,0.705882}%
\pgfsetfillcolor{currentfill}%
\pgfsetfillopacity{0.639668}%
\pgfsetlinewidth{1.003750pt}%
\definecolor{currentstroke}{rgb}{0.121569,0.466667,0.705882}%
\pgfsetstrokecolor{currentstroke}%
\pgfsetstrokeopacity{0.639668}%
\pgfsetdash{}{0pt}%
\pgfpathmoveto{\pgfqpoint{0.848693in}{1.659662in}}%
\pgfpathcurveto{\pgfqpoint{0.856930in}{1.659662in}}{\pgfqpoint{0.864830in}{1.662934in}}{\pgfqpoint{0.870654in}{1.668758in}}%
\pgfpathcurveto{\pgfqpoint{0.876478in}{1.674582in}}{\pgfqpoint{0.879750in}{1.682482in}}{\pgfqpoint{0.879750in}{1.690718in}}%
\pgfpathcurveto{\pgfqpoint{0.879750in}{1.698954in}}{\pgfqpoint{0.876478in}{1.706855in}}{\pgfqpoint{0.870654in}{1.712678in}}%
\pgfpathcurveto{\pgfqpoint{0.864830in}{1.718502in}}{\pgfqpoint{0.856930in}{1.721775in}}{\pgfqpoint{0.848693in}{1.721775in}}%
\pgfpathcurveto{\pgfqpoint{0.840457in}{1.721775in}}{\pgfqpoint{0.832557in}{1.718502in}}{\pgfqpoint{0.826733in}{1.712678in}}%
\pgfpathcurveto{\pgfqpoint{0.820909in}{1.706855in}}{\pgfqpoint{0.817637in}{1.698954in}}{\pgfqpoint{0.817637in}{1.690718in}}%
\pgfpathcurveto{\pgfqpoint{0.817637in}{1.682482in}}{\pgfqpoint{0.820909in}{1.674582in}}{\pgfqpoint{0.826733in}{1.668758in}}%
\pgfpathcurveto{\pgfqpoint{0.832557in}{1.662934in}}{\pgfqpoint{0.840457in}{1.659662in}}{\pgfqpoint{0.848693in}{1.659662in}}%
\pgfpathclose%
\pgfusepath{stroke,fill}%
\end{pgfscope}%
\begin{pgfscope}%
\pgfpathrectangle{\pgfqpoint{0.100000in}{0.212622in}}{\pgfqpoint{3.696000in}{3.696000in}}%
\pgfusepath{clip}%
\pgfsetbuttcap%
\pgfsetroundjoin%
\definecolor{currentfill}{rgb}{0.121569,0.466667,0.705882}%
\pgfsetfillcolor{currentfill}%
\pgfsetfillopacity{0.640086}%
\pgfsetlinewidth{1.003750pt}%
\definecolor{currentstroke}{rgb}{0.121569,0.466667,0.705882}%
\pgfsetstrokecolor{currentstroke}%
\pgfsetstrokeopacity{0.640086}%
\pgfsetdash{}{0pt}%
\pgfpathmoveto{\pgfqpoint{3.146550in}{1.962143in}}%
\pgfpathcurveto{\pgfqpoint{3.154786in}{1.962143in}}{\pgfqpoint{3.162686in}{1.965415in}}{\pgfqpoint{3.168510in}{1.971239in}}%
\pgfpathcurveto{\pgfqpoint{3.174334in}{1.977063in}}{\pgfqpoint{3.177606in}{1.984963in}}{\pgfqpoint{3.177606in}{1.993199in}}%
\pgfpathcurveto{\pgfqpoint{3.177606in}{2.001436in}}{\pgfqpoint{3.174334in}{2.009336in}}{\pgfqpoint{3.168510in}{2.015160in}}%
\pgfpathcurveto{\pgfqpoint{3.162686in}{2.020984in}}{\pgfqpoint{3.154786in}{2.024256in}}{\pgfqpoint{3.146550in}{2.024256in}}%
\pgfpathcurveto{\pgfqpoint{3.138313in}{2.024256in}}{\pgfqpoint{3.130413in}{2.020984in}}{\pgfqpoint{3.124589in}{2.015160in}}%
\pgfpathcurveto{\pgfqpoint{3.118765in}{2.009336in}}{\pgfqpoint{3.115493in}{2.001436in}}{\pgfqpoint{3.115493in}{1.993199in}}%
\pgfpathcurveto{\pgfqpoint{3.115493in}{1.984963in}}{\pgfqpoint{3.118765in}{1.977063in}}{\pgfqpoint{3.124589in}{1.971239in}}%
\pgfpathcurveto{\pgfqpoint{3.130413in}{1.965415in}}{\pgfqpoint{3.138313in}{1.962143in}}{\pgfqpoint{3.146550in}{1.962143in}}%
\pgfpathclose%
\pgfusepath{stroke,fill}%
\end{pgfscope}%
\begin{pgfscope}%
\pgfpathrectangle{\pgfqpoint{0.100000in}{0.212622in}}{\pgfqpoint{3.696000in}{3.696000in}}%
\pgfusepath{clip}%
\pgfsetbuttcap%
\pgfsetroundjoin%
\definecolor{currentfill}{rgb}{0.121569,0.466667,0.705882}%
\pgfsetfillcolor{currentfill}%
\pgfsetfillopacity{0.640615}%
\pgfsetlinewidth{1.003750pt}%
\definecolor{currentstroke}{rgb}{0.121569,0.466667,0.705882}%
\pgfsetstrokecolor{currentstroke}%
\pgfsetstrokeopacity{0.640615}%
\pgfsetdash{}{0pt}%
\pgfpathmoveto{\pgfqpoint{0.847036in}{1.658023in}}%
\pgfpathcurveto{\pgfqpoint{0.855272in}{1.658023in}}{\pgfqpoint{0.863172in}{1.661295in}}{\pgfqpoint{0.868996in}{1.667119in}}%
\pgfpathcurveto{\pgfqpoint{0.874820in}{1.672943in}}{\pgfqpoint{0.878093in}{1.680843in}}{\pgfqpoint{0.878093in}{1.689080in}}%
\pgfpathcurveto{\pgfqpoint{0.878093in}{1.697316in}}{\pgfqpoint{0.874820in}{1.705216in}}{\pgfqpoint{0.868996in}{1.711040in}}%
\pgfpathcurveto{\pgfqpoint{0.863172in}{1.716864in}}{\pgfqpoint{0.855272in}{1.720136in}}{\pgfqpoint{0.847036in}{1.720136in}}%
\pgfpathcurveto{\pgfqpoint{0.838800in}{1.720136in}}{\pgfqpoint{0.830900in}{1.716864in}}{\pgfqpoint{0.825076in}{1.711040in}}%
\pgfpathcurveto{\pgfqpoint{0.819252in}{1.705216in}}{\pgfqpoint{0.815980in}{1.697316in}}{\pgfqpoint{0.815980in}{1.689080in}}%
\pgfpathcurveto{\pgfqpoint{0.815980in}{1.680843in}}{\pgfqpoint{0.819252in}{1.672943in}}{\pgfqpoint{0.825076in}{1.667119in}}%
\pgfpathcurveto{\pgfqpoint{0.830900in}{1.661295in}}{\pgfqpoint{0.838800in}{1.658023in}}{\pgfqpoint{0.847036in}{1.658023in}}%
\pgfpathclose%
\pgfusepath{stroke,fill}%
\end{pgfscope}%
\begin{pgfscope}%
\pgfpathrectangle{\pgfqpoint{0.100000in}{0.212622in}}{\pgfqpoint{3.696000in}{3.696000in}}%
\pgfusepath{clip}%
\pgfsetbuttcap%
\pgfsetroundjoin%
\definecolor{currentfill}{rgb}{0.121569,0.466667,0.705882}%
\pgfsetfillcolor{currentfill}%
\pgfsetfillopacity{0.641648}%
\pgfsetlinewidth{1.003750pt}%
\definecolor{currentstroke}{rgb}{0.121569,0.466667,0.705882}%
\pgfsetstrokecolor{currentstroke}%
\pgfsetstrokeopacity{0.641648}%
\pgfsetdash{}{0pt}%
\pgfpathmoveto{\pgfqpoint{3.143385in}{1.962938in}}%
\pgfpathcurveto{\pgfqpoint{3.151621in}{1.962938in}}{\pgfqpoint{3.159521in}{1.966210in}}{\pgfqpoint{3.165345in}{1.972034in}}%
\pgfpathcurveto{\pgfqpoint{3.171169in}{1.977858in}}{\pgfqpoint{3.174441in}{1.985758in}}{\pgfqpoint{3.174441in}{1.993994in}}%
\pgfpathcurveto{\pgfqpoint{3.174441in}{2.002231in}}{\pgfqpoint{3.171169in}{2.010131in}}{\pgfqpoint{3.165345in}{2.015955in}}%
\pgfpathcurveto{\pgfqpoint{3.159521in}{2.021779in}}{\pgfqpoint{3.151621in}{2.025051in}}{\pgfqpoint{3.143385in}{2.025051in}}%
\pgfpathcurveto{\pgfqpoint{3.135148in}{2.025051in}}{\pgfqpoint{3.127248in}{2.021779in}}{\pgfqpoint{3.121424in}{2.015955in}}%
\pgfpathcurveto{\pgfqpoint{3.115600in}{2.010131in}}{\pgfqpoint{3.112328in}{2.002231in}}{\pgfqpoint{3.112328in}{1.993994in}}%
\pgfpathcurveto{\pgfqpoint{3.112328in}{1.985758in}}{\pgfqpoint{3.115600in}{1.977858in}}{\pgfqpoint{3.121424in}{1.972034in}}%
\pgfpathcurveto{\pgfqpoint{3.127248in}{1.966210in}}{\pgfqpoint{3.135148in}{1.962938in}}{\pgfqpoint{3.143385in}{1.962938in}}%
\pgfpathclose%
\pgfusepath{stroke,fill}%
\end{pgfscope}%
\begin{pgfscope}%
\pgfpathrectangle{\pgfqpoint{0.100000in}{0.212622in}}{\pgfqpoint{3.696000in}{3.696000in}}%
\pgfusepath{clip}%
\pgfsetbuttcap%
\pgfsetroundjoin%
\definecolor{currentfill}{rgb}{0.121569,0.466667,0.705882}%
\pgfsetfillcolor{currentfill}%
\pgfsetfillopacity{0.642074}%
\pgfsetlinewidth{1.003750pt}%
\definecolor{currentstroke}{rgb}{0.121569,0.466667,0.705882}%
\pgfsetstrokecolor{currentstroke}%
\pgfsetstrokeopacity{0.642074}%
\pgfsetdash{}{0pt}%
\pgfpathmoveto{\pgfqpoint{0.841161in}{1.656846in}}%
\pgfpathcurveto{\pgfqpoint{0.849397in}{1.656846in}}{\pgfqpoint{0.857297in}{1.660118in}}{\pgfqpoint{0.863121in}{1.665942in}}%
\pgfpathcurveto{\pgfqpoint{0.868945in}{1.671766in}}{\pgfqpoint{0.872217in}{1.679666in}}{\pgfqpoint{0.872217in}{1.687902in}}%
\pgfpathcurveto{\pgfqpoint{0.872217in}{1.696138in}}{\pgfqpoint{0.868945in}{1.704039in}}{\pgfqpoint{0.863121in}{1.709862in}}%
\pgfpathcurveto{\pgfqpoint{0.857297in}{1.715686in}}{\pgfqpoint{0.849397in}{1.718959in}}{\pgfqpoint{0.841161in}{1.718959in}}%
\pgfpathcurveto{\pgfqpoint{0.832924in}{1.718959in}}{\pgfqpoint{0.825024in}{1.715686in}}{\pgfqpoint{0.819200in}{1.709862in}}%
\pgfpathcurveto{\pgfqpoint{0.813377in}{1.704039in}}{\pgfqpoint{0.810104in}{1.696138in}}{\pgfqpoint{0.810104in}{1.687902in}}%
\pgfpathcurveto{\pgfqpoint{0.810104in}{1.679666in}}{\pgfqpoint{0.813377in}{1.671766in}}{\pgfqpoint{0.819200in}{1.665942in}}%
\pgfpathcurveto{\pgfqpoint{0.825024in}{1.660118in}}{\pgfqpoint{0.832924in}{1.656846in}}{\pgfqpoint{0.841161in}{1.656846in}}%
\pgfpathclose%
\pgfusepath{stroke,fill}%
\end{pgfscope}%
\begin{pgfscope}%
\pgfpathrectangle{\pgfqpoint{0.100000in}{0.212622in}}{\pgfqpoint{3.696000in}{3.696000in}}%
\pgfusepath{clip}%
\pgfsetbuttcap%
\pgfsetroundjoin%
\definecolor{currentfill}{rgb}{0.121569,0.466667,0.705882}%
\pgfsetfillcolor{currentfill}%
\pgfsetfillopacity{0.643490}%
\pgfsetlinewidth{1.003750pt}%
\definecolor{currentstroke}{rgb}{0.121569,0.466667,0.705882}%
\pgfsetstrokecolor{currentstroke}%
\pgfsetstrokeopacity{0.643490}%
\pgfsetdash{}{0pt}%
\pgfpathmoveto{\pgfqpoint{3.141415in}{1.960266in}}%
\pgfpathcurveto{\pgfqpoint{3.149651in}{1.960266in}}{\pgfqpoint{3.157551in}{1.963538in}}{\pgfqpoint{3.163375in}{1.969362in}}%
\pgfpathcurveto{\pgfqpoint{3.169199in}{1.975186in}}{\pgfqpoint{3.172471in}{1.983086in}}{\pgfqpoint{3.172471in}{1.991323in}}%
\pgfpathcurveto{\pgfqpoint{3.172471in}{1.999559in}}{\pgfqpoint{3.169199in}{2.007459in}}{\pgfqpoint{3.163375in}{2.013283in}}%
\pgfpathcurveto{\pgfqpoint{3.157551in}{2.019107in}}{\pgfqpoint{3.149651in}{2.022379in}}{\pgfqpoint{3.141415in}{2.022379in}}%
\pgfpathcurveto{\pgfqpoint{3.133179in}{2.022379in}}{\pgfqpoint{3.125279in}{2.019107in}}{\pgfqpoint{3.119455in}{2.013283in}}%
\pgfpathcurveto{\pgfqpoint{3.113631in}{2.007459in}}{\pgfqpoint{3.110358in}{1.999559in}}{\pgfqpoint{3.110358in}{1.991323in}}%
\pgfpathcurveto{\pgfqpoint{3.110358in}{1.983086in}}{\pgfqpoint{3.113631in}{1.975186in}}{\pgfqpoint{3.119455in}{1.969362in}}%
\pgfpathcurveto{\pgfqpoint{3.125279in}{1.963538in}}{\pgfqpoint{3.133179in}{1.960266in}}{\pgfqpoint{3.141415in}{1.960266in}}%
\pgfpathclose%
\pgfusepath{stroke,fill}%
\end{pgfscope}%
\begin{pgfscope}%
\pgfpathrectangle{\pgfqpoint{0.100000in}{0.212622in}}{\pgfqpoint{3.696000in}{3.696000in}}%
\pgfusepath{clip}%
\pgfsetbuttcap%
\pgfsetroundjoin%
\definecolor{currentfill}{rgb}{0.121569,0.466667,0.705882}%
\pgfsetfillcolor{currentfill}%
\pgfsetfillopacity{0.643552}%
\pgfsetlinewidth{1.003750pt}%
\definecolor{currentstroke}{rgb}{0.121569,0.466667,0.705882}%
\pgfsetstrokecolor{currentstroke}%
\pgfsetstrokeopacity{0.643552}%
\pgfsetdash{}{0pt}%
\pgfpathmoveto{\pgfqpoint{0.839382in}{1.656827in}}%
\pgfpathcurveto{\pgfqpoint{0.847618in}{1.656827in}}{\pgfqpoint{0.855518in}{1.660099in}}{\pgfqpoint{0.861342in}{1.665923in}}%
\pgfpathcurveto{\pgfqpoint{0.867166in}{1.671747in}}{\pgfqpoint{0.870438in}{1.679647in}}{\pgfqpoint{0.870438in}{1.687883in}}%
\pgfpathcurveto{\pgfqpoint{0.870438in}{1.696119in}}{\pgfqpoint{0.867166in}{1.704019in}}{\pgfqpoint{0.861342in}{1.709843in}}%
\pgfpathcurveto{\pgfqpoint{0.855518in}{1.715667in}}{\pgfqpoint{0.847618in}{1.718940in}}{\pgfqpoint{0.839382in}{1.718940in}}%
\pgfpathcurveto{\pgfqpoint{0.831146in}{1.718940in}}{\pgfqpoint{0.823246in}{1.715667in}}{\pgfqpoint{0.817422in}{1.709843in}}%
\pgfpathcurveto{\pgfqpoint{0.811598in}{1.704019in}}{\pgfqpoint{0.808325in}{1.696119in}}{\pgfqpoint{0.808325in}{1.687883in}}%
\pgfpathcurveto{\pgfqpoint{0.808325in}{1.679647in}}{\pgfqpoint{0.811598in}{1.671747in}}{\pgfqpoint{0.817422in}{1.665923in}}%
\pgfpathcurveto{\pgfqpoint{0.823246in}{1.660099in}}{\pgfqpoint{0.831146in}{1.656827in}}{\pgfqpoint{0.839382in}{1.656827in}}%
\pgfpathclose%
\pgfusepath{stroke,fill}%
\end{pgfscope}%
\begin{pgfscope}%
\pgfpathrectangle{\pgfqpoint{0.100000in}{0.212622in}}{\pgfqpoint{3.696000in}{3.696000in}}%
\pgfusepath{clip}%
\pgfsetbuttcap%
\pgfsetroundjoin%
\definecolor{currentfill}{rgb}{0.121569,0.466667,0.705882}%
\pgfsetfillcolor{currentfill}%
\pgfsetfillopacity{0.644377}%
\pgfsetlinewidth{1.003750pt}%
\definecolor{currentstroke}{rgb}{0.121569,0.466667,0.705882}%
\pgfsetstrokecolor{currentstroke}%
\pgfsetstrokeopacity{0.644377}%
\pgfsetdash{}{0pt}%
\pgfpathmoveto{\pgfqpoint{0.835949in}{1.655238in}}%
\pgfpathcurveto{\pgfqpoint{0.844185in}{1.655238in}}{\pgfqpoint{0.852085in}{1.658511in}}{\pgfqpoint{0.857909in}{1.664334in}}%
\pgfpathcurveto{\pgfqpoint{0.863733in}{1.670158in}}{\pgfqpoint{0.867006in}{1.678058in}}{\pgfqpoint{0.867006in}{1.686295in}}%
\pgfpathcurveto{\pgfqpoint{0.867006in}{1.694531in}}{\pgfqpoint{0.863733in}{1.702431in}}{\pgfqpoint{0.857909in}{1.708255in}}%
\pgfpathcurveto{\pgfqpoint{0.852085in}{1.714079in}}{\pgfqpoint{0.844185in}{1.717351in}}{\pgfqpoint{0.835949in}{1.717351in}}%
\pgfpathcurveto{\pgfqpoint{0.827713in}{1.717351in}}{\pgfqpoint{0.819813in}{1.714079in}}{\pgfqpoint{0.813989in}{1.708255in}}%
\pgfpathcurveto{\pgfqpoint{0.808165in}{1.702431in}}{\pgfqpoint{0.804893in}{1.694531in}}{\pgfqpoint{0.804893in}{1.686295in}}%
\pgfpathcurveto{\pgfqpoint{0.804893in}{1.678058in}}{\pgfqpoint{0.808165in}{1.670158in}}{\pgfqpoint{0.813989in}{1.664334in}}%
\pgfpathcurveto{\pgfqpoint{0.819813in}{1.658511in}}{\pgfqpoint{0.827713in}{1.655238in}}{\pgfqpoint{0.835949in}{1.655238in}}%
\pgfpathclose%
\pgfusepath{stroke,fill}%
\end{pgfscope}%
\begin{pgfscope}%
\pgfpathrectangle{\pgfqpoint{0.100000in}{0.212622in}}{\pgfqpoint{3.696000in}{3.696000in}}%
\pgfusepath{clip}%
\pgfsetbuttcap%
\pgfsetroundjoin%
\definecolor{currentfill}{rgb}{0.121569,0.466667,0.705882}%
\pgfsetfillcolor{currentfill}%
\pgfsetfillopacity{0.644518}%
\pgfsetlinewidth{1.003750pt}%
\definecolor{currentstroke}{rgb}{0.121569,0.466667,0.705882}%
\pgfsetstrokecolor{currentstroke}%
\pgfsetstrokeopacity{0.644518}%
\pgfsetdash{}{0pt}%
\pgfpathmoveto{\pgfqpoint{3.140044in}{1.959094in}}%
\pgfpathcurveto{\pgfqpoint{3.148281in}{1.959094in}}{\pgfqpoint{3.156181in}{1.962366in}}{\pgfqpoint{3.162005in}{1.968190in}}%
\pgfpathcurveto{\pgfqpoint{3.167829in}{1.974014in}}{\pgfqpoint{3.171101in}{1.981914in}}{\pgfqpoint{3.171101in}{1.990150in}}%
\pgfpathcurveto{\pgfqpoint{3.171101in}{1.998387in}}{\pgfqpoint{3.167829in}{2.006287in}}{\pgfqpoint{3.162005in}{2.012111in}}%
\pgfpathcurveto{\pgfqpoint{3.156181in}{2.017935in}}{\pgfqpoint{3.148281in}{2.021207in}}{\pgfqpoint{3.140044in}{2.021207in}}%
\pgfpathcurveto{\pgfqpoint{3.131808in}{2.021207in}}{\pgfqpoint{3.123908in}{2.017935in}}{\pgfqpoint{3.118084in}{2.012111in}}%
\pgfpathcurveto{\pgfqpoint{3.112260in}{2.006287in}}{\pgfqpoint{3.108988in}{1.998387in}}{\pgfqpoint{3.108988in}{1.990150in}}%
\pgfpathcurveto{\pgfqpoint{3.108988in}{1.981914in}}{\pgfqpoint{3.112260in}{1.974014in}}{\pgfqpoint{3.118084in}{1.968190in}}%
\pgfpathcurveto{\pgfqpoint{3.123908in}{1.962366in}}{\pgfqpoint{3.131808in}{1.959094in}}{\pgfqpoint{3.140044in}{1.959094in}}%
\pgfpathclose%
\pgfusepath{stroke,fill}%
\end{pgfscope}%
\begin{pgfscope}%
\pgfpathrectangle{\pgfqpoint{0.100000in}{0.212622in}}{\pgfqpoint{3.696000in}{3.696000in}}%
\pgfusepath{clip}%
\pgfsetbuttcap%
\pgfsetroundjoin%
\definecolor{currentfill}{rgb}{0.121569,0.466667,0.705882}%
\pgfsetfillcolor{currentfill}%
\pgfsetfillopacity{0.645495}%
\pgfsetlinewidth{1.003750pt}%
\definecolor{currentstroke}{rgb}{0.121569,0.466667,0.705882}%
\pgfsetstrokecolor{currentstroke}%
\pgfsetstrokeopacity{0.645495}%
\pgfsetdash{}{0pt}%
\pgfpathmoveto{\pgfqpoint{0.835027in}{1.657100in}}%
\pgfpathcurveto{\pgfqpoint{0.843263in}{1.657100in}}{\pgfqpoint{0.851163in}{1.660373in}}{\pgfqpoint{0.856987in}{1.666197in}}%
\pgfpathcurveto{\pgfqpoint{0.862811in}{1.672021in}}{\pgfqpoint{0.866084in}{1.679921in}}{\pgfqpoint{0.866084in}{1.688157in}}%
\pgfpathcurveto{\pgfqpoint{0.866084in}{1.696393in}}{\pgfqpoint{0.862811in}{1.704293in}}{\pgfqpoint{0.856987in}{1.710117in}}%
\pgfpathcurveto{\pgfqpoint{0.851163in}{1.715941in}}{\pgfqpoint{0.843263in}{1.719213in}}{\pgfqpoint{0.835027in}{1.719213in}}%
\pgfpathcurveto{\pgfqpoint{0.826791in}{1.719213in}}{\pgfqpoint{0.818891in}{1.715941in}}{\pgfqpoint{0.813067in}{1.710117in}}%
\pgfpathcurveto{\pgfqpoint{0.807243in}{1.704293in}}{\pgfqpoint{0.803971in}{1.696393in}}{\pgfqpoint{0.803971in}{1.688157in}}%
\pgfpathcurveto{\pgfqpoint{0.803971in}{1.679921in}}{\pgfqpoint{0.807243in}{1.672021in}}{\pgfqpoint{0.813067in}{1.666197in}}%
\pgfpathcurveto{\pgfqpoint{0.818891in}{1.660373in}}{\pgfqpoint{0.826791in}{1.657100in}}{\pgfqpoint{0.835027in}{1.657100in}}%
\pgfpathclose%
\pgfusepath{stroke,fill}%
\end{pgfscope}%
\begin{pgfscope}%
\pgfpathrectangle{\pgfqpoint{0.100000in}{0.212622in}}{\pgfqpoint{3.696000in}{3.696000in}}%
\pgfusepath{clip}%
\pgfsetbuttcap%
\pgfsetroundjoin%
\definecolor{currentfill}{rgb}{0.121569,0.466667,0.705882}%
\pgfsetfillcolor{currentfill}%
\pgfsetfillopacity{0.645909}%
\pgfsetlinewidth{1.003750pt}%
\definecolor{currentstroke}{rgb}{0.121569,0.466667,0.705882}%
\pgfsetstrokecolor{currentstroke}%
\pgfsetstrokeopacity{0.645909}%
\pgfsetdash{}{0pt}%
\pgfpathmoveto{\pgfqpoint{0.832936in}{1.657190in}}%
\pgfpathcurveto{\pgfqpoint{0.841173in}{1.657190in}}{\pgfqpoint{0.849073in}{1.660463in}}{\pgfqpoint{0.854897in}{1.666287in}}%
\pgfpathcurveto{\pgfqpoint{0.860721in}{1.672111in}}{\pgfqpoint{0.863993in}{1.680011in}}{\pgfqpoint{0.863993in}{1.688247in}}%
\pgfpathcurveto{\pgfqpoint{0.863993in}{1.696483in}}{\pgfqpoint{0.860721in}{1.704383in}}{\pgfqpoint{0.854897in}{1.710207in}}%
\pgfpathcurveto{\pgfqpoint{0.849073in}{1.716031in}}{\pgfqpoint{0.841173in}{1.719303in}}{\pgfqpoint{0.832936in}{1.719303in}}%
\pgfpathcurveto{\pgfqpoint{0.824700in}{1.719303in}}{\pgfqpoint{0.816800in}{1.716031in}}{\pgfqpoint{0.810976in}{1.710207in}}%
\pgfpathcurveto{\pgfqpoint{0.805152in}{1.704383in}}{\pgfqpoint{0.801880in}{1.696483in}}{\pgfqpoint{0.801880in}{1.688247in}}%
\pgfpathcurveto{\pgfqpoint{0.801880in}{1.680011in}}{\pgfqpoint{0.805152in}{1.672111in}}{\pgfqpoint{0.810976in}{1.666287in}}%
\pgfpathcurveto{\pgfqpoint{0.816800in}{1.660463in}}{\pgfqpoint{0.824700in}{1.657190in}}{\pgfqpoint{0.832936in}{1.657190in}}%
\pgfpathclose%
\pgfusepath{stroke,fill}%
\end{pgfscope}%
\begin{pgfscope}%
\pgfpathrectangle{\pgfqpoint{0.100000in}{0.212622in}}{\pgfqpoint{3.696000in}{3.696000in}}%
\pgfusepath{clip}%
\pgfsetbuttcap%
\pgfsetroundjoin%
\definecolor{currentfill}{rgb}{0.121569,0.466667,0.705882}%
\pgfsetfillcolor{currentfill}%
\pgfsetfillopacity{0.645946}%
\pgfsetlinewidth{1.003750pt}%
\definecolor{currentstroke}{rgb}{0.121569,0.466667,0.705882}%
\pgfsetstrokecolor{currentstroke}%
\pgfsetstrokeopacity{0.645946}%
\pgfsetdash{}{0pt}%
\pgfpathmoveto{\pgfqpoint{3.138275in}{1.958405in}}%
\pgfpathcurveto{\pgfqpoint{3.146512in}{1.958405in}}{\pgfqpoint{3.154412in}{1.961678in}}{\pgfqpoint{3.160236in}{1.967502in}}%
\pgfpathcurveto{\pgfqpoint{3.166059in}{1.973325in}}{\pgfqpoint{3.169332in}{1.981226in}}{\pgfqpoint{3.169332in}{1.989462in}}%
\pgfpathcurveto{\pgfqpoint{3.169332in}{1.997698in}}{\pgfqpoint{3.166059in}{2.005598in}}{\pgfqpoint{3.160236in}{2.011422in}}%
\pgfpathcurveto{\pgfqpoint{3.154412in}{2.017246in}}{\pgfqpoint{3.146512in}{2.020518in}}{\pgfqpoint{3.138275in}{2.020518in}}%
\pgfpathcurveto{\pgfqpoint{3.130039in}{2.020518in}}{\pgfqpoint{3.122139in}{2.017246in}}{\pgfqpoint{3.116315in}{2.011422in}}%
\pgfpathcurveto{\pgfqpoint{3.110491in}{2.005598in}}{\pgfqpoint{3.107219in}{1.997698in}}{\pgfqpoint{3.107219in}{1.989462in}}%
\pgfpathcurveto{\pgfqpoint{3.107219in}{1.981226in}}{\pgfqpoint{3.110491in}{1.973325in}}{\pgfqpoint{3.116315in}{1.967502in}}%
\pgfpathcurveto{\pgfqpoint{3.122139in}{1.961678in}}{\pgfqpoint{3.130039in}{1.958405in}}{\pgfqpoint{3.138275in}{1.958405in}}%
\pgfpathclose%
\pgfusepath{stroke,fill}%
\end{pgfscope}%
\begin{pgfscope}%
\pgfpathrectangle{\pgfqpoint{0.100000in}{0.212622in}}{\pgfqpoint{3.696000in}{3.696000in}}%
\pgfusepath{clip}%
\pgfsetbuttcap%
\pgfsetroundjoin%
\definecolor{currentfill}{rgb}{0.121569,0.466667,0.705882}%
\pgfsetfillcolor{currentfill}%
\pgfsetfillopacity{0.646164}%
\pgfsetlinewidth{1.003750pt}%
\definecolor{currentstroke}{rgb}{0.121569,0.466667,0.705882}%
\pgfsetstrokecolor{currentstroke}%
\pgfsetstrokeopacity{0.646164}%
\pgfsetdash{}{0pt}%
\pgfpathmoveto{\pgfqpoint{0.830585in}{1.651364in}}%
\pgfpathcurveto{\pgfqpoint{0.838821in}{1.651364in}}{\pgfqpoint{0.846721in}{1.654636in}}{\pgfqpoint{0.852545in}{1.660460in}}%
\pgfpathcurveto{\pgfqpoint{0.858369in}{1.666284in}}{\pgfqpoint{0.861641in}{1.674184in}}{\pgfqpoint{0.861641in}{1.682420in}}%
\pgfpathcurveto{\pgfqpoint{0.861641in}{1.690657in}}{\pgfqpoint{0.858369in}{1.698557in}}{\pgfqpoint{0.852545in}{1.704381in}}%
\pgfpathcurveto{\pgfqpoint{0.846721in}{1.710204in}}{\pgfqpoint{0.838821in}{1.713477in}}{\pgfqpoint{0.830585in}{1.713477in}}%
\pgfpathcurveto{\pgfqpoint{0.822348in}{1.713477in}}{\pgfqpoint{0.814448in}{1.710204in}}{\pgfqpoint{0.808624in}{1.704381in}}%
\pgfpathcurveto{\pgfqpoint{0.802800in}{1.698557in}}{\pgfqpoint{0.799528in}{1.690657in}}{\pgfqpoint{0.799528in}{1.682420in}}%
\pgfpathcurveto{\pgfqpoint{0.799528in}{1.674184in}}{\pgfqpoint{0.802800in}{1.666284in}}{\pgfqpoint{0.808624in}{1.660460in}}%
\pgfpathcurveto{\pgfqpoint{0.814448in}{1.654636in}}{\pgfqpoint{0.822348in}{1.651364in}}{\pgfqpoint{0.830585in}{1.651364in}}%
\pgfpathclose%
\pgfusepath{stroke,fill}%
\end{pgfscope}%
\begin{pgfscope}%
\pgfpathrectangle{\pgfqpoint{0.100000in}{0.212622in}}{\pgfqpoint{3.696000in}{3.696000in}}%
\pgfusepath{clip}%
\pgfsetbuttcap%
\pgfsetroundjoin%
\definecolor{currentfill}{rgb}{0.121569,0.466667,0.705882}%
\pgfsetfillcolor{currentfill}%
\pgfsetfillopacity{0.646791}%
\pgfsetlinewidth{1.003750pt}%
\definecolor{currentstroke}{rgb}{0.121569,0.466667,0.705882}%
\pgfsetstrokecolor{currentstroke}%
\pgfsetstrokeopacity{0.646791}%
\pgfsetdash{}{0pt}%
\pgfpathmoveto{\pgfqpoint{3.136303in}{1.959735in}}%
\pgfpathcurveto{\pgfqpoint{3.144539in}{1.959735in}}{\pgfqpoint{3.152439in}{1.963007in}}{\pgfqpoint{3.158263in}{1.968831in}}%
\pgfpathcurveto{\pgfqpoint{3.164087in}{1.974655in}}{\pgfqpoint{3.167359in}{1.982555in}}{\pgfqpoint{3.167359in}{1.990792in}}%
\pgfpathcurveto{\pgfqpoint{3.167359in}{1.999028in}}{\pgfqpoint{3.164087in}{2.006928in}}{\pgfqpoint{3.158263in}{2.012752in}}%
\pgfpathcurveto{\pgfqpoint{3.152439in}{2.018576in}}{\pgfqpoint{3.144539in}{2.021848in}}{\pgfqpoint{3.136303in}{2.021848in}}%
\pgfpathcurveto{\pgfqpoint{3.128067in}{2.021848in}}{\pgfqpoint{3.120167in}{2.018576in}}{\pgfqpoint{3.114343in}{2.012752in}}%
\pgfpathcurveto{\pgfqpoint{3.108519in}{2.006928in}}{\pgfqpoint{3.105246in}{1.999028in}}{\pgfqpoint{3.105246in}{1.990792in}}%
\pgfpathcurveto{\pgfqpoint{3.105246in}{1.982555in}}{\pgfqpoint{3.108519in}{1.974655in}}{\pgfqpoint{3.114343in}{1.968831in}}%
\pgfpathcurveto{\pgfqpoint{3.120167in}{1.963007in}}{\pgfqpoint{3.128067in}{1.959735in}}{\pgfqpoint{3.136303in}{1.959735in}}%
\pgfpathclose%
\pgfusepath{stroke,fill}%
\end{pgfscope}%
\begin{pgfscope}%
\pgfpathrectangle{\pgfqpoint{0.100000in}{0.212622in}}{\pgfqpoint{3.696000in}{3.696000in}}%
\pgfusepath{clip}%
\pgfsetbuttcap%
\pgfsetroundjoin%
\definecolor{currentfill}{rgb}{0.121569,0.466667,0.705882}%
\pgfsetfillcolor{currentfill}%
\pgfsetfillopacity{0.646964}%
\pgfsetlinewidth{1.003750pt}%
\definecolor{currentstroke}{rgb}{0.121569,0.466667,0.705882}%
\pgfsetstrokecolor{currentstroke}%
\pgfsetstrokeopacity{0.646964}%
\pgfsetdash{}{0pt}%
\pgfpathmoveto{\pgfqpoint{0.829134in}{1.649230in}}%
\pgfpathcurveto{\pgfqpoint{0.837371in}{1.649230in}}{\pgfqpoint{0.845271in}{1.652503in}}{\pgfqpoint{0.851095in}{1.658326in}}%
\pgfpathcurveto{\pgfqpoint{0.856918in}{1.664150in}}{\pgfqpoint{0.860191in}{1.672050in}}{\pgfqpoint{0.860191in}{1.680287in}}%
\pgfpathcurveto{\pgfqpoint{0.860191in}{1.688523in}}{\pgfqpoint{0.856918in}{1.696423in}}{\pgfqpoint{0.851095in}{1.702247in}}%
\pgfpathcurveto{\pgfqpoint{0.845271in}{1.708071in}}{\pgfqpoint{0.837371in}{1.711343in}}{\pgfqpoint{0.829134in}{1.711343in}}%
\pgfpathcurveto{\pgfqpoint{0.820898in}{1.711343in}}{\pgfqpoint{0.812998in}{1.708071in}}{\pgfqpoint{0.807174in}{1.702247in}}%
\pgfpathcurveto{\pgfqpoint{0.801350in}{1.696423in}}{\pgfqpoint{0.798078in}{1.688523in}}{\pgfqpoint{0.798078in}{1.680287in}}%
\pgfpathcurveto{\pgfqpoint{0.798078in}{1.672050in}}{\pgfqpoint{0.801350in}{1.664150in}}{\pgfqpoint{0.807174in}{1.658326in}}%
\pgfpathcurveto{\pgfqpoint{0.812998in}{1.652503in}}{\pgfqpoint{0.820898in}{1.649230in}}{\pgfqpoint{0.829134in}{1.649230in}}%
\pgfpathclose%
\pgfusepath{stroke,fill}%
\end{pgfscope}%
\begin{pgfscope}%
\pgfpathrectangle{\pgfqpoint{0.100000in}{0.212622in}}{\pgfqpoint{3.696000in}{3.696000in}}%
\pgfusepath{clip}%
\pgfsetbuttcap%
\pgfsetroundjoin%
\definecolor{currentfill}{rgb}{0.121569,0.466667,0.705882}%
\pgfsetfillcolor{currentfill}%
\pgfsetfillopacity{0.647733}%
\pgfsetlinewidth{1.003750pt}%
\definecolor{currentstroke}{rgb}{0.121569,0.466667,0.705882}%
\pgfsetstrokecolor{currentstroke}%
\pgfsetstrokeopacity{0.647733}%
\pgfsetdash{}{0pt}%
\pgfpathmoveto{\pgfqpoint{3.135279in}{1.957095in}}%
\pgfpathcurveto{\pgfqpoint{3.143515in}{1.957095in}}{\pgfqpoint{3.151415in}{1.960367in}}{\pgfqpoint{3.157239in}{1.966191in}}%
\pgfpathcurveto{\pgfqpoint{3.163063in}{1.972015in}}{\pgfqpoint{3.166335in}{1.979915in}}{\pgfqpoint{3.166335in}{1.988151in}}%
\pgfpathcurveto{\pgfqpoint{3.166335in}{1.996387in}}{\pgfqpoint{3.163063in}{2.004287in}}{\pgfqpoint{3.157239in}{2.010111in}}%
\pgfpathcurveto{\pgfqpoint{3.151415in}{2.015935in}}{\pgfqpoint{3.143515in}{2.019208in}}{\pgfqpoint{3.135279in}{2.019208in}}%
\pgfpathcurveto{\pgfqpoint{3.127042in}{2.019208in}}{\pgfqpoint{3.119142in}{2.015935in}}{\pgfqpoint{3.113318in}{2.010111in}}%
\pgfpathcurveto{\pgfqpoint{3.107495in}{2.004287in}}{\pgfqpoint{3.104222in}{1.996387in}}{\pgfqpoint{3.104222in}{1.988151in}}%
\pgfpathcurveto{\pgfqpoint{3.104222in}{1.979915in}}{\pgfqpoint{3.107495in}{1.972015in}}{\pgfqpoint{3.113318in}{1.966191in}}%
\pgfpathcurveto{\pgfqpoint{3.119142in}{1.960367in}}{\pgfqpoint{3.127042in}{1.957095in}}{\pgfqpoint{3.135279in}{1.957095in}}%
\pgfpathclose%
\pgfusepath{stroke,fill}%
\end{pgfscope}%
\begin{pgfscope}%
\pgfpathrectangle{\pgfqpoint{0.100000in}{0.212622in}}{\pgfqpoint{3.696000in}{3.696000in}}%
\pgfusepath{clip}%
\pgfsetbuttcap%
\pgfsetroundjoin%
\definecolor{currentfill}{rgb}{0.121569,0.466667,0.705882}%
\pgfsetfillcolor{currentfill}%
\pgfsetfillopacity{0.648274}%
\pgfsetlinewidth{1.003750pt}%
\definecolor{currentstroke}{rgb}{0.121569,0.466667,0.705882}%
\pgfsetstrokecolor{currentstroke}%
\pgfsetstrokeopacity{0.648274}%
\pgfsetdash{}{0pt}%
\pgfpathmoveto{\pgfqpoint{3.134440in}{1.955980in}}%
\pgfpathcurveto{\pgfqpoint{3.142676in}{1.955980in}}{\pgfqpoint{3.150576in}{1.959253in}}{\pgfqpoint{3.156400in}{1.965077in}}%
\pgfpathcurveto{\pgfqpoint{3.162224in}{1.970901in}}{\pgfqpoint{3.165496in}{1.978801in}}{\pgfqpoint{3.165496in}{1.987037in}}%
\pgfpathcurveto{\pgfqpoint{3.165496in}{1.995273in}}{\pgfqpoint{3.162224in}{2.003173in}}{\pgfqpoint{3.156400in}{2.008997in}}%
\pgfpathcurveto{\pgfqpoint{3.150576in}{2.014821in}}{\pgfqpoint{3.142676in}{2.018093in}}{\pgfqpoint{3.134440in}{2.018093in}}%
\pgfpathcurveto{\pgfqpoint{3.126204in}{2.018093in}}{\pgfqpoint{3.118304in}{2.014821in}}{\pgfqpoint{3.112480in}{2.008997in}}%
\pgfpathcurveto{\pgfqpoint{3.106656in}{2.003173in}}{\pgfqpoint{3.103383in}{1.995273in}}{\pgfqpoint{3.103383in}{1.987037in}}%
\pgfpathcurveto{\pgfqpoint{3.103383in}{1.978801in}}{\pgfqpoint{3.106656in}{1.970901in}}{\pgfqpoint{3.112480in}{1.965077in}}%
\pgfpathcurveto{\pgfqpoint{3.118304in}{1.959253in}}{\pgfqpoint{3.126204in}{1.955980in}}{\pgfqpoint{3.134440in}{1.955980in}}%
\pgfpathclose%
\pgfusepath{stroke,fill}%
\end{pgfscope}%
\begin{pgfscope}%
\pgfpathrectangle{\pgfqpoint{0.100000in}{0.212622in}}{\pgfqpoint{3.696000in}{3.696000in}}%
\pgfusepath{clip}%
\pgfsetbuttcap%
\pgfsetroundjoin%
\definecolor{currentfill}{rgb}{0.121569,0.466667,0.705882}%
\pgfsetfillcolor{currentfill}%
\pgfsetfillopacity{0.648293}%
\pgfsetlinewidth{1.003750pt}%
\definecolor{currentstroke}{rgb}{0.121569,0.466667,0.705882}%
\pgfsetstrokecolor{currentstroke}%
\pgfsetstrokeopacity{0.648293}%
\pgfsetdash{}{0pt}%
\pgfpathmoveto{\pgfqpoint{0.823946in}{1.647428in}}%
\pgfpathcurveto{\pgfqpoint{0.832183in}{1.647428in}}{\pgfqpoint{0.840083in}{1.650700in}}{\pgfqpoint{0.845907in}{1.656524in}}%
\pgfpathcurveto{\pgfqpoint{0.851730in}{1.662348in}}{\pgfqpoint{0.855003in}{1.670248in}}{\pgfqpoint{0.855003in}{1.678485in}}%
\pgfpathcurveto{\pgfqpoint{0.855003in}{1.686721in}}{\pgfqpoint{0.851730in}{1.694621in}}{\pgfqpoint{0.845907in}{1.700445in}}%
\pgfpathcurveto{\pgfqpoint{0.840083in}{1.706269in}}{\pgfqpoint{0.832183in}{1.709541in}}{\pgfqpoint{0.823946in}{1.709541in}}%
\pgfpathcurveto{\pgfqpoint{0.815710in}{1.709541in}}{\pgfqpoint{0.807810in}{1.706269in}}{\pgfqpoint{0.801986in}{1.700445in}}%
\pgfpathcurveto{\pgfqpoint{0.796162in}{1.694621in}}{\pgfqpoint{0.792890in}{1.686721in}}{\pgfqpoint{0.792890in}{1.678485in}}%
\pgfpathcurveto{\pgfqpoint{0.792890in}{1.670248in}}{\pgfqpoint{0.796162in}{1.662348in}}{\pgfqpoint{0.801986in}{1.656524in}}%
\pgfpathcurveto{\pgfqpoint{0.807810in}{1.650700in}}{\pgfqpoint{0.815710in}{1.647428in}}{\pgfqpoint{0.823946in}{1.647428in}}%
\pgfpathclose%
\pgfusepath{stroke,fill}%
\end{pgfscope}%
\begin{pgfscope}%
\pgfpathrectangle{\pgfqpoint{0.100000in}{0.212622in}}{\pgfqpoint{3.696000in}{3.696000in}}%
\pgfusepath{clip}%
\pgfsetbuttcap%
\pgfsetroundjoin%
\definecolor{currentfill}{rgb}{0.121569,0.466667,0.705882}%
\pgfsetfillcolor{currentfill}%
\pgfsetfillopacity{0.649164}%
\pgfsetlinewidth{1.003750pt}%
\definecolor{currentstroke}{rgb}{0.121569,0.466667,0.705882}%
\pgfsetstrokecolor{currentstroke}%
\pgfsetstrokeopacity{0.649164}%
\pgfsetdash{}{0pt}%
\pgfpathmoveto{\pgfqpoint{3.132274in}{1.955789in}}%
\pgfpathcurveto{\pgfqpoint{3.140510in}{1.955789in}}{\pgfqpoint{3.148410in}{1.959061in}}{\pgfqpoint{3.154234in}{1.964885in}}%
\pgfpathcurveto{\pgfqpoint{3.160058in}{1.970709in}}{\pgfqpoint{3.163330in}{1.978609in}}{\pgfqpoint{3.163330in}{1.986845in}}%
\pgfpathcurveto{\pgfqpoint{3.163330in}{1.995082in}}{\pgfqpoint{3.160058in}{2.002982in}}{\pgfqpoint{3.154234in}{2.008806in}}%
\pgfpathcurveto{\pgfqpoint{3.148410in}{2.014630in}}{\pgfqpoint{3.140510in}{2.017902in}}{\pgfqpoint{3.132274in}{2.017902in}}%
\pgfpathcurveto{\pgfqpoint{3.124037in}{2.017902in}}{\pgfqpoint{3.116137in}{2.014630in}}{\pgfqpoint{3.110313in}{2.008806in}}%
\pgfpathcurveto{\pgfqpoint{3.104489in}{2.002982in}}{\pgfqpoint{3.101217in}{1.995082in}}{\pgfqpoint{3.101217in}{1.986845in}}%
\pgfpathcurveto{\pgfqpoint{3.101217in}{1.978609in}}{\pgfqpoint{3.104489in}{1.970709in}}{\pgfqpoint{3.110313in}{1.964885in}}%
\pgfpathcurveto{\pgfqpoint{3.116137in}{1.959061in}}{\pgfqpoint{3.124037in}{1.955789in}}{\pgfqpoint{3.132274in}{1.955789in}}%
\pgfpathclose%
\pgfusepath{stroke,fill}%
\end{pgfscope}%
\begin{pgfscope}%
\pgfpathrectangle{\pgfqpoint{0.100000in}{0.212622in}}{\pgfqpoint{3.696000in}{3.696000in}}%
\pgfusepath{clip}%
\pgfsetbuttcap%
\pgfsetroundjoin%
\definecolor{currentfill}{rgb}{0.121569,0.466667,0.705882}%
\pgfsetfillcolor{currentfill}%
\pgfsetfillopacity{0.649528}%
\pgfsetlinewidth{1.003750pt}%
\definecolor{currentstroke}{rgb}{0.121569,0.466667,0.705882}%
\pgfsetstrokecolor{currentstroke}%
\pgfsetstrokeopacity{0.649528}%
\pgfsetdash{}{0pt}%
\pgfpathmoveto{\pgfqpoint{3.131238in}{1.954551in}}%
\pgfpathcurveto{\pgfqpoint{3.139474in}{1.954551in}}{\pgfqpoint{3.147374in}{1.957823in}}{\pgfqpoint{3.153198in}{1.963647in}}%
\pgfpathcurveto{\pgfqpoint{3.159022in}{1.969471in}}{\pgfqpoint{3.162294in}{1.977371in}}{\pgfqpoint{3.162294in}{1.985607in}}%
\pgfpathcurveto{\pgfqpoint{3.162294in}{1.993843in}}{\pgfqpoint{3.159022in}{2.001743in}}{\pgfqpoint{3.153198in}{2.007567in}}%
\pgfpathcurveto{\pgfqpoint{3.147374in}{2.013391in}}{\pgfqpoint{3.139474in}{2.016664in}}{\pgfqpoint{3.131238in}{2.016664in}}%
\pgfpathcurveto{\pgfqpoint{3.123001in}{2.016664in}}{\pgfqpoint{3.115101in}{2.013391in}}{\pgfqpoint{3.109277in}{2.007567in}}%
\pgfpathcurveto{\pgfqpoint{3.103453in}{2.001743in}}{\pgfqpoint{3.100181in}{1.993843in}}{\pgfqpoint{3.100181in}{1.985607in}}%
\pgfpathcurveto{\pgfqpoint{3.100181in}{1.977371in}}{\pgfqpoint{3.103453in}{1.969471in}}{\pgfqpoint{3.109277in}{1.963647in}}%
\pgfpathcurveto{\pgfqpoint{3.115101in}{1.957823in}}{\pgfqpoint{3.123001in}{1.954551in}}{\pgfqpoint{3.131238in}{1.954551in}}%
\pgfpathclose%
\pgfusepath{stroke,fill}%
\end{pgfscope}%
\begin{pgfscope}%
\pgfpathrectangle{\pgfqpoint{0.100000in}{0.212622in}}{\pgfqpoint{3.696000in}{3.696000in}}%
\pgfusepath{clip}%
\pgfsetbuttcap%
\pgfsetroundjoin%
\definecolor{currentfill}{rgb}{0.121569,0.466667,0.705882}%
\pgfsetfillcolor{currentfill}%
\pgfsetfillopacity{0.650041}%
\pgfsetlinewidth{1.003750pt}%
\definecolor{currentstroke}{rgb}{0.121569,0.466667,0.705882}%
\pgfsetstrokecolor{currentstroke}%
\pgfsetstrokeopacity{0.650041}%
\pgfsetdash{}{0pt}%
\pgfpathmoveto{\pgfqpoint{0.822616in}{1.649288in}}%
\pgfpathcurveto{\pgfqpoint{0.830852in}{1.649288in}}{\pgfqpoint{0.838752in}{1.652560in}}{\pgfqpoint{0.844576in}{1.658384in}}%
\pgfpathcurveto{\pgfqpoint{0.850400in}{1.664208in}}{\pgfqpoint{0.853672in}{1.672108in}}{\pgfqpoint{0.853672in}{1.680344in}}%
\pgfpathcurveto{\pgfqpoint{0.853672in}{1.688581in}}{\pgfqpoint{0.850400in}{1.696481in}}{\pgfqpoint{0.844576in}{1.702305in}}%
\pgfpathcurveto{\pgfqpoint{0.838752in}{1.708129in}}{\pgfqpoint{0.830852in}{1.711401in}}{\pgfqpoint{0.822616in}{1.711401in}}%
\pgfpathcurveto{\pgfqpoint{0.814379in}{1.711401in}}{\pgfqpoint{0.806479in}{1.708129in}}{\pgfqpoint{0.800655in}{1.702305in}}%
\pgfpathcurveto{\pgfqpoint{0.794831in}{1.696481in}}{\pgfqpoint{0.791559in}{1.688581in}}{\pgfqpoint{0.791559in}{1.680344in}}%
\pgfpathcurveto{\pgfqpoint{0.791559in}{1.672108in}}{\pgfqpoint{0.794831in}{1.664208in}}{\pgfqpoint{0.800655in}{1.658384in}}%
\pgfpathcurveto{\pgfqpoint{0.806479in}{1.652560in}}{\pgfqpoint{0.814379in}{1.649288in}}{\pgfqpoint{0.822616in}{1.649288in}}%
\pgfpathclose%
\pgfusepath{stroke,fill}%
\end{pgfscope}%
\begin{pgfscope}%
\pgfpathrectangle{\pgfqpoint{0.100000in}{0.212622in}}{\pgfqpoint{3.696000in}{3.696000in}}%
\pgfusepath{clip}%
\pgfsetbuttcap%
\pgfsetroundjoin%
\definecolor{currentfill}{rgb}{0.121569,0.466667,0.705882}%
\pgfsetfillcolor{currentfill}%
\pgfsetfillopacity{0.650238}%
\pgfsetlinewidth{1.003750pt}%
\definecolor{currentstroke}{rgb}{0.121569,0.466667,0.705882}%
\pgfsetstrokecolor{currentstroke}%
\pgfsetstrokeopacity{0.650238}%
\pgfsetdash{}{0pt}%
\pgfpathmoveto{\pgfqpoint{3.130496in}{1.954244in}}%
\pgfpathcurveto{\pgfqpoint{3.138733in}{1.954244in}}{\pgfqpoint{3.146633in}{1.957517in}}{\pgfqpoint{3.152457in}{1.963341in}}%
\pgfpathcurveto{\pgfqpoint{3.158281in}{1.969165in}}{\pgfqpoint{3.161553in}{1.977065in}}{\pgfqpoint{3.161553in}{1.985301in}}%
\pgfpathcurveto{\pgfqpoint{3.161553in}{1.993537in}}{\pgfqpoint{3.158281in}{2.001437in}}{\pgfqpoint{3.152457in}{2.007261in}}%
\pgfpathcurveto{\pgfqpoint{3.146633in}{2.013085in}}{\pgfqpoint{3.138733in}{2.016357in}}{\pgfqpoint{3.130496in}{2.016357in}}%
\pgfpathcurveto{\pgfqpoint{3.122260in}{2.016357in}}{\pgfqpoint{3.114360in}{2.013085in}}{\pgfqpoint{3.108536in}{2.007261in}}%
\pgfpathcurveto{\pgfqpoint{3.102712in}{2.001437in}}{\pgfqpoint{3.099440in}{1.993537in}}{\pgfqpoint{3.099440in}{1.985301in}}%
\pgfpathcurveto{\pgfqpoint{3.099440in}{1.977065in}}{\pgfqpoint{3.102712in}{1.969165in}}{\pgfqpoint{3.108536in}{1.963341in}}%
\pgfpathcurveto{\pgfqpoint{3.114360in}{1.957517in}}{\pgfqpoint{3.122260in}{1.954244in}}{\pgfqpoint{3.130496in}{1.954244in}}%
\pgfpathclose%
\pgfusepath{stroke,fill}%
\end{pgfscope}%
\begin{pgfscope}%
\pgfpathrectangle{\pgfqpoint{0.100000in}{0.212622in}}{\pgfqpoint{3.696000in}{3.696000in}}%
\pgfusepath{clip}%
\pgfsetbuttcap%
\pgfsetroundjoin%
\definecolor{currentfill}{rgb}{0.121569,0.466667,0.705882}%
\pgfsetfillcolor{currentfill}%
\pgfsetfillopacity{0.650589}%
\pgfsetlinewidth{1.003750pt}%
\definecolor{currentstroke}{rgb}{0.121569,0.466667,0.705882}%
\pgfsetstrokecolor{currentstroke}%
\pgfsetstrokeopacity{0.650589}%
\pgfsetdash{}{0pt}%
\pgfpathmoveto{\pgfqpoint{3.129970in}{1.953879in}}%
\pgfpathcurveto{\pgfqpoint{3.138207in}{1.953879in}}{\pgfqpoint{3.146107in}{1.957152in}}{\pgfqpoint{3.151931in}{1.962976in}}%
\pgfpathcurveto{\pgfqpoint{3.157755in}{1.968800in}}{\pgfqpoint{3.161027in}{1.976700in}}{\pgfqpoint{3.161027in}{1.984936in}}%
\pgfpathcurveto{\pgfqpoint{3.161027in}{1.993172in}}{\pgfqpoint{3.157755in}{2.001072in}}{\pgfqpoint{3.151931in}{2.006896in}}%
\pgfpathcurveto{\pgfqpoint{3.146107in}{2.012720in}}{\pgfqpoint{3.138207in}{2.015992in}}{\pgfqpoint{3.129970in}{2.015992in}}%
\pgfpathcurveto{\pgfqpoint{3.121734in}{2.015992in}}{\pgfqpoint{3.113834in}{2.012720in}}{\pgfqpoint{3.108010in}{2.006896in}}%
\pgfpathcurveto{\pgfqpoint{3.102186in}{2.001072in}}{\pgfqpoint{3.098914in}{1.993172in}}{\pgfqpoint{3.098914in}{1.984936in}}%
\pgfpathcurveto{\pgfqpoint{3.098914in}{1.976700in}}{\pgfqpoint{3.102186in}{1.968800in}}{\pgfqpoint{3.108010in}{1.962976in}}%
\pgfpathcurveto{\pgfqpoint{3.113834in}{1.957152in}}{\pgfqpoint{3.121734in}{1.953879in}}{\pgfqpoint{3.129970in}{1.953879in}}%
\pgfpathclose%
\pgfusepath{stroke,fill}%
\end{pgfscope}%
\begin{pgfscope}%
\pgfpathrectangle{\pgfqpoint{0.100000in}{0.212622in}}{\pgfqpoint{3.696000in}{3.696000in}}%
\pgfusepath{clip}%
\pgfsetbuttcap%
\pgfsetroundjoin%
\definecolor{currentfill}{rgb}{0.121569,0.466667,0.705882}%
\pgfsetfillcolor{currentfill}%
\pgfsetfillopacity{0.651052}%
\pgfsetlinewidth{1.003750pt}%
\definecolor{currentstroke}{rgb}{0.121569,0.466667,0.705882}%
\pgfsetstrokecolor{currentstroke}%
\pgfsetstrokeopacity{0.651052}%
\pgfsetdash{}{0pt}%
\pgfpathmoveto{\pgfqpoint{0.819086in}{1.650309in}}%
\pgfpathcurveto{\pgfqpoint{0.827322in}{1.650309in}}{\pgfqpoint{0.835222in}{1.653581in}}{\pgfqpoint{0.841046in}{1.659405in}}%
\pgfpathcurveto{\pgfqpoint{0.846870in}{1.665229in}}{\pgfqpoint{0.850142in}{1.673129in}}{\pgfqpoint{0.850142in}{1.681365in}}%
\pgfpathcurveto{\pgfqpoint{0.850142in}{1.689601in}}{\pgfqpoint{0.846870in}{1.697501in}}{\pgfqpoint{0.841046in}{1.703325in}}%
\pgfpathcurveto{\pgfqpoint{0.835222in}{1.709149in}}{\pgfqpoint{0.827322in}{1.712422in}}{\pgfqpoint{0.819086in}{1.712422in}}%
\pgfpathcurveto{\pgfqpoint{0.810850in}{1.712422in}}{\pgfqpoint{0.802949in}{1.709149in}}{\pgfqpoint{0.797126in}{1.703325in}}%
\pgfpathcurveto{\pgfqpoint{0.791302in}{1.697501in}}{\pgfqpoint{0.788029in}{1.689601in}}{\pgfqpoint{0.788029in}{1.681365in}}%
\pgfpathcurveto{\pgfqpoint{0.788029in}{1.673129in}}{\pgfqpoint{0.791302in}{1.665229in}}{\pgfqpoint{0.797126in}{1.659405in}}%
\pgfpathcurveto{\pgfqpoint{0.802949in}{1.653581in}}{\pgfqpoint{0.810850in}{1.650309in}}{\pgfqpoint{0.819086in}{1.650309in}}%
\pgfpathclose%
\pgfusepath{stroke,fill}%
\end{pgfscope}%
\begin{pgfscope}%
\pgfpathrectangle{\pgfqpoint{0.100000in}{0.212622in}}{\pgfqpoint{3.696000in}{3.696000in}}%
\pgfusepath{clip}%
\pgfsetbuttcap%
\pgfsetroundjoin%
\definecolor{currentfill}{rgb}{0.121569,0.466667,0.705882}%
\pgfsetfillcolor{currentfill}%
\pgfsetfillopacity{0.651061}%
\pgfsetlinewidth{1.003750pt}%
\definecolor{currentstroke}{rgb}{0.121569,0.466667,0.705882}%
\pgfsetstrokecolor{currentstroke}%
\pgfsetstrokeopacity{0.651061}%
\pgfsetdash{}{0pt}%
\pgfpathmoveto{\pgfqpoint{3.128784in}{1.954555in}}%
\pgfpathcurveto{\pgfqpoint{3.137020in}{1.954555in}}{\pgfqpoint{3.144920in}{1.957828in}}{\pgfqpoint{3.150744in}{1.963652in}}%
\pgfpathcurveto{\pgfqpoint{3.156568in}{1.969476in}}{\pgfqpoint{3.159840in}{1.977376in}}{\pgfqpoint{3.159840in}{1.985612in}}%
\pgfpathcurveto{\pgfqpoint{3.159840in}{1.993848in}}{\pgfqpoint{3.156568in}{2.001748in}}{\pgfqpoint{3.150744in}{2.007572in}}%
\pgfpathcurveto{\pgfqpoint{3.144920in}{2.013396in}}{\pgfqpoint{3.137020in}{2.016668in}}{\pgfqpoint{3.128784in}{2.016668in}}%
\pgfpathcurveto{\pgfqpoint{3.120548in}{2.016668in}}{\pgfqpoint{3.112648in}{2.013396in}}{\pgfqpoint{3.106824in}{2.007572in}}%
\pgfpathcurveto{\pgfqpoint{3.101000in}{2.001748in}}{\pgfqpoint{3.097727in}{1.993848in}}{\pgfqpoint{3.097727in}{1.985612in}}%
\pgfpathcurveto{\pgfqpoint{3.097727in}{1.977376in}}{\pgfqpoint{3.101000in}{1.969476in}}{\pgfqpoint{3.106824in}{1.963652in}}%
\pgfpathcurveto{\pgfqpoint{3.112648in}{1.957828in}}{\pgfqpoint{3.120548in}{1.954555in}}{\pgfqpoint{3.128784in}{1.954555in}}%
\pgfpathclose%
\pgfusepath{stroke,fill}%
\end{pgfscope}%
\begin{pgfscope}%
\pgfpathrectangle{\pgfqpoint{0.100000in}{0.212622in}}{\pgfqpoint{3.696000in}{3.696000in}}%
\pgfusepath{clip}%
\pgfsetbuttcap%
\pgfsetroundjoin%
\definecolor{currentfill}{rgb}{0.121569,0.466667,0.705882}%
\pgfsetfillcolor{currentfill}%
\pgfsetfillopacity{0.651765}%
\pgfsetlinewidth{1.003750pt}%
\definecolor{currentstroke}{rgb}{0.121569,0.466667,0.705882}%
\pgfsetstrokecolor{currentstroke}%
\pgfsetstrokeopacity{0.651765}%
\pgfsetdash{}{0pt}%
\pgfpathmoveto{\pgfqpoint{0.818442in}{1.649748in}}%
\pgfpathcurveto{\pgfqpoint{0.826678in}{1.649748in}}{\pgfqpoint{0.834578in}{1.653021in}}{\pgfqpoint{0.840402in}{1.658844in}}%
\pgfpathcurveto{\pgfqpoint{0.846226in}{1.664668in}}{\pgfqpoint{0.849499in}{1.672568in}}{\pgfqpoint{0.849499in}{1.680805in}}%
\pgfpathcurveto{\pgfqpoint{0.849499in}{1.689041in}}{\pgfqpoint{0.846226in}{1.696941in}}{\pgfqpoint{0.840402in}{1.702765in}}%
\pgfpathcurveto{\pgfqpoint{0.834578in}{1.708589in}}{\pgfqpoint{0.826678in}{1.711861in}}{\pgfqpoint{0.818442in}{1.711861in}}%
\pgfpathcurveto{\pgfqpoint{0.810206in}{1.711861in}}{\pgfqpoint{0.802306in}{1.708589in}}{\pgfqpoint{0.796482in}{1.702765in}}%
\pgfpathcurveto{\pgfqpoint{0.790658in}{1.696941in}}{\pgfqpoint{0.787386in}{1.689041in}}{\pgfqpoint{0.787386in}{1.680805in}}%
\pgfpathcurveto{\pgfqpoint{0.787386in}{1.672568in}}{\pgfqpoint{0.790658in}{1.664668in}}{\pgfqpoint{0.796482in}{1.658844in}}%
\pgfpathcurveto{\pgfqpoint{0.802306in}{1.653021in}}{\pgfqpoint{0.810206in}{1.649748in}}{\pgfqpoint{0.818442in}{1.649748in}}%
\pgfpathclose%
\pgfusepath{stroke,fill}%
\end{pgfscope}%
\begin{pgfscope}%
\pgfpathrectangle{\pgfqpoint{0.100000in}{0.212622in}}{\pgfqpoint{3.696000in}{3.696000in}}%
\pgfusepath{clip}%
\pgfsetbuttcap%
\pgfsetroundjoin%
\definecolor{currentfill}{rgb}{0.121569,0.466667,0.705882}%
\pgfsetfillcolor{currentfill}%
\pgfsetfillopacity{0.651804}%
\pgfsetlinewidth{1.003750pt}%
\definecolor{currentstroke}{rgb}{0.121569,0.466667,0.705882}%
\pgfsetstrokecolor{currentstroke}%
\pgfsetstrokeopacity{0.651804}%
\pgfsetdash{}{0pt}%
\pgfpathmoveto{\pgfqpoint{3.128142in}{1.952750in}}%
\pgfpathcurveto{\pgfqpoint{3.136378in}{1.952750in}}{\pgfqpoint{3.144278in}{1.956022in}}{\pgfqpoint{3.150102in}{1.961846in}}%
\pgfpathcurveto{\pgfqpoint{3.155926in}{1.967670in}}{\pgfqpoint{3.159198in}{1.975570in}}{\pgfqpoint{3.159198in}{1.983806in}}%
\pgfpathcurveto{\pgfqpoint{3.159198in}{1.992042in}}{\pgfqpoint{3.155926in}{1.999942in}}{\pgfqpoint{3.150102in}{2.005766in}}%
\pgfpathcurveto{\pgfqpoint{3.144278in}{2.011590in}}{\pgfqpoint{3.136378in}{2.014863in}}{\pgfqpoint{3.128142in}{2.014863in}}%
\pgfpathcurveto{\pgfqpoint{3.119906in}{2.014863in}}{\pgfqpoint{3.112006in}{2.011590in}}{\pgfqpoint{3.106182in}{2.005766in}}%
\pgfpathcurveto{\pgfqpoint{3.100358in}{1.999942in}}{\pgfqpoint{3.097085in}{1.992042in}}{\pgfqpoint{3.097085in}{1.983806in}}%
\pgfpathcurveto{\pgfqpoint{3.097085in}{1.975570in}}{\pgfqpoint{3.100358in}{1.967670in}}{\pgfqpoint{3.106182in}{1.961846in}}%
\pgfpathcurveto{\pgfqpoint{3.112006in}{1.956022in}}{\pgfqpoint{3.119906in}{1.952750in}}{\pgfqpoint{3.128142in}{1.952750in}}%
\pgfpathclose%
\pgfusepath{stroke,fill}%
\end{pgfscope}%
\begin{pgfscope}%
\pgfpathrectangle{\pgfqpoint{0.100000in}{0.212622in}}{\pgfqpoint{3.696000in}{3.696000in}}%
\pgfusepath{clip}%
\pgfsetbuttcap%
\pgfsetroundjoin%
\definecolor{currentfill}{rgb}{0.121569,0.466667,0.705882}%
\pgfsetfillcolor{currentfill}%
\pgfsetfillopacity{0.652256}%
\pgfsetlinewidth{1.003750pt}%
\definecolor{currentstroke}{rgb}{0.121569,0.466667,0.705882}%
\pgfsetstrokecolor{currentstroke}%
\pgfsetstrokeopacity{0.652256}%
\pgfsetdash{}{0pt}%
\pgfpathmoveto{\pgfqpoint{0.816583in}{1.643583in}}%
\pgfpathcurveto{\pgfqpoint{0.824820in}{1.643583in}}{\pgfqpoint{0.832720in}{1.646856in}}{\pgfqpoint{0.838544in}{1.652680in}}%
\pgfpathcurveto{\pgfqpoint{0.844368in}{1.658504in}}{\pgfqpoint{0.847640in}{1.666404in}}{\pgfqpoint{0.847640in}{1.674640in}}%
\pgfpathcurveto{\pgfqpoint{0.847640in}{1.682876in}}{\pgfqpoint{0.844368in}{1.690776in}}{\pgfqpoint{0.838544in}{1.696600in}}%
\pgfpathcurveto{\pgfqpoint{0.832720in}{1.702424in}}{\pgfqpoint{0.824820in}{1.705696in}}{\pgfqpoint{0.816583in}{1.705696in}}%
\pgfpathcurveto{\pgfqpoint{0.808347in}{1.705696in}}{\pgfqpoint{0.800447in}{1.702424in}}{\pgfqpoint{0.794623in}{1.696600in}}%
\pgfpathcurveto{\pgfqpoint{0.788799in}{1.690776in}}{\pgfqpoint{0.785527in}{1.682876in}}{\pgfqpoint{0.785527in}{1.674640in}}%
\pgfpathcurveto{\pgfqpoint{0.785527in}{1.666404in}}{\pgfqpoint{0.788799in}{1.658504in}}{\pgfqpoint{0.794623in}{1.652680in}}%
\pgfpathcurveto{\pgfqpoint{0.800447in}{1.646856in}}{\pgfqpoint{0.808347in}{1.643583in}}{\pgfqpoint{0.816583in}{1.643583in}}%
\pgfpathclose%
\pgfusepath{stroke,fill}%
\end{pgfscope}%
\begin{pgfscope}%
\pgfpathrectangle{\pgfqpoint{0.100000in}{0.212622in}}{\pgfqpoint{3.696000in}{3.696000in}}%
\pgfusepath{clip}%
\pgfsetbuttcap%
\pgfsetroundjoin%
\definecolor{currentfill}{rgb}{0.121569,0.466667,0.705882}%
\pgfsetfillcolor{currentfill}%
\pgfsetfillopacity{0.652704}%
\pgfsetlinewidth{1.003750pt}%
\definecolor{currentstroke}{rgb}{0.121569,0.466667,0.705882}%
\pgfsetstrokecolor{currentstroke}%
\pgfsetstrokeopacity{0.652704}%
\pgfsetdash{}{0pt}%
\pgfpathmoveto{\pgfqpoint{3.127220in}{1.950939in}}%
\pgfpathcurveto{\pgfqpoint{3.135456in}{1.950939in}}{\pgfqpoint{3.143356in}{1.954211in}}{\pgfqpoint{3.149180in}{1.960035in}}%
\pgfpathcurveto{\pgfqpoint{3.155004in}{1.965859in}}{\pgfqpoint{3.158277in}{1.973759in}}{\pgfqpoint{3.158277in}{1.981996in}}%
\pgfpathcurveto{\pgfqpoint{3.158277in}{1.990232in}}{\pgfqpoint{3.155004in}{1.998132in}}{\pgfqpoint{3.149180in}{2.003956in}}%
\pgfpathcurveto{\pgfqpoint{3.143356in}{2.009780in}}{\pgfqpoint{3.135456in}{2.013052in}}{\pgfqpoint{3.127220in}{2.013052in}}%
\pgfpathcurveto{\pgfqpoint{3.118984in}{2.013052in}}{\pgfqpoint{3.111084in}{2.009780in}}{\pgfqpoint{3.105260in}{2.003956in}}%
\pgfpathcurveto{\pgfqpoint{3.099436in}{1.998132in}}{\pgfqpoint{3.096164in}{1.990232in}}{\pgfqpoint{3.096164in}{1.981996in}}%
\pgfpathcurveto{\pgfqpoint{3.096164in}{1.973759in}}{\pgfqpoint{3.099436in}{1.965859in}}{\pgfqpoint{3.105260in}{1.960035in}}%
\pgfpathcurveto{\pgfqpoint{3.111084in}{1.954211in}}{\pgfqpoint{3.118984in}{1.950939in}}{\pgfqpoint{3.127220in}{1.950939in}}%
\pgfpathclose%
\pgfusepath{stroke,fill}%
\end{pgfscope}%
\begin{pgfscope}%
\pgfpathrectangle{\pgfqpoint{0.100000in}{0.212622in}}{\pgfqpoint{3.696000in}{3.696000in}}%
\pgfusepath{clip}%
\pgfsetbuttcap%
\pgfsetroundjoin%
\definecolor{currentfill}{rgb}{0.121569,0.466667,0.705882}%
\pgfsetfillcolor{currentfill}%
\pgfsetfillopacity{0.653057}%
\pgfsetlinewidth{1.003750pt}%
\definecolor{currentstroke}{rgb}{0.121569,0.466667,0.705882}%
\pgfsetstrokecolor{currentstroke}%
\pgfsetstrokeopacity{0.653057}%
\pgfsetdash{}{0pt}%
\pgfpathmoveto{\pgfqpoint{0.813710in}{1.642348in}}%
\pgfpathcurveto{\pgfqpoint{0.821946in}{1.642348in}}{\pgfqpoint{0.829846in}{1.645620in}}{\pgfqpoint{0.835670in}{1.651444in}}%
\pgfpathcurveto{\pgfqpoint{0.841494in}{1.657268in}}{\pgfqpoint{0.844766in}{1.665168in}}{\pgfqpoint{0.844766in}{1.673404in}}%
\pgfpathcurveto{\pgfqpoint{0.844766in}{1.681640in}}{\pgfqpoint{0.841494in}{1.689541in}}{\pgfqpoint{0.835670in}{1.695364in}}%
\pgfpathcurveto{\pgfqpoint{0.829846in}{1.701188in}}{\pgfqpoint{0.821946in}{1.704461in}}{\pgfqpoint{0.813710in}{1.704461in}}%
\pgfpathcurveto{\pgfqpoint{0.805474in}{1.704461in}}{\pgfqpoint{0.797573in}{1.701188in}}{\pgfqpoint{0.791750in}{1.695364in}}%
\pgfpathcurveto{\pgfqpoint{0.785926in}{1.689541in}}{\pgfqpoint{0.782653in}{1.681640in}}{\pgfqpoint{0.782653in}{1.673404in}}%
\pgfpathcurveto{\pgfqpoint{0.782653in}{1.665168in}}{\pgfqpoint{0.785926in}{1.657268in}}{\pgfqpoint{0.791750in}{1.651444in}}%
\pgfpathcurveto{\pgfqpoint{0.797573in}{1.645620in}}{\pgfqpoint{0.805474in}{1.642348in}}{\pgfqpoint{0.813710in}{1.642348in}}%
\pgfpathclose%
\pgfusepath{stroke,fill}%
\end{pgfscope}%
\begin{pgfscope}%
\pgfpathrectangle{\pgfqpoint{0.100000in}{0.212622in}}{\pgfqpoint{3.696000in}{3.696000in}}%
\pgfusepath{clip}%
\pgfsetbuttcap%
\pgfsetroundjoin%
\definecolor{currentfill}{rgb}{0.121569,0.466667,0.705882}%
\pgfsetfillcolor{currentfill}%
\pgfsetfillopacity{0.653823}%
\pgfsetlinewidth{1.003750pt}%
\definecolor{currentstroke}{rgb}{0.121569,0.466667,0.705882}%
\pgfsetstrokecolor{currentstroke}%
\pgfsetstrokeopacity{0.653823}%
\pgfsetdash{}{0pt}%
\pgfpathmoveto{\pgfqpoint{3.123901in}{1.948381in}}%
\pgfpathcurveto{\pgfqpoint{3.132137in}{1.948381in}}{\pgfqpoint{3.140037in}{1.951653in}}{\pgfqpoint{3.145861in}{1.957477in}}%
\pgfpathcurveto{\pgfqpoint{3.151685in}{1.963301in}}{\pgfqpoint{3.154957in}{1.971201in}}{\pgfqpoint{3.154957in}{1.979437in}}%
\pgfpathcurveto{\pgfqpoint{3.154957in}{1.987674in}}{\pgfqpoint{3.151685in}{1.995574in}}{\pgfqpoint{3.145861in}{2.001398in}}%
\pgfpathcurveto{\pgfqpoint{3.140037in}{2.007222in}}{\pgfqpoint{3.132137in}{2.010494in}}{\pgfqpoint{3.123901in}{2.010494in}}%
\pgfpathcurveto{\pgfqpoint{3.115664in}{2.010494in}}{\pgfqpoint{3.107764in}{2.007222in}}{\pgfqpoint{3.101940in}{2.001398in}}%
\pgfpathcurveto{\pgfqpoint{3.096116in}{1.995574in}}{\pgfqpoint{3.092844in}{1.987674in}}{\pgfqpoint{3.092844in}{1.979437in}}%
\pgfpathcurveto{\pgfqpoint{3.092844in}{1.971201in}}{\pgfqpoint{3.096116in}{1.963301in}}{\pgfqpoint{3.101940in}{1.957477in}}%
\pgfpathcurveto{\pgfqpoint{3.107764in}{1.951653in}}{\pgfqpoint{3.115664in}{1.948381in}}{\pgfqpoint{3.123901in}{1.948381in}}%
\pgfpathclose%
\pgfusepath{stroke,fill}%
\end{pgfscope}%
\begin{pgfscope}%
\pgfpathrectangle{\pgfqpoint{0.100000in}{0.212622in}}{\pgfqpoint{3.696000in}{3.696000in}}%
\pgfusepath{clip}%
\pgfsetbuttcap%
\pgfsetroundjoin%
\definecolor{currentfill}{rgb}{0.121569,0.466667,0.705882}%
\pgfsetfillcolor{currentfill}%
\pgfsetfillopacity{0.654547}%
\pgfsetlinewidth{1.003750pt}%
\definecolor{currentstroke}{rgb}{0.121569,0.466667,0.705882}%
\pgfsetstrokecolor{currentstroke}%
\pgfsetstrokeopacity{0.654547}%
\pgfsetdash{}{0pt}%
\pgfpathmoveto{\pgfqpoint{3.121620in}{1.948547in}}%
\pgfpathcurveto{\pgfqpoint{3.129857in}{1.948547in}}{\pgfqpoint{3.137757in}{1.951820in}}{\pgfqpoint{3.143581in}{1.957644in}}%
\pgfpathcurveto{\pgfqpoint{3.149404in}{1.963468in}}{\pgfqpoint{3.152677in}{1.971368in}}{\pgfqpoint{3.152677in}{1.979604in}}%
\pgfpathcurveto{\pgfqpoint{3.152677in}{1.987840in}}{\pgfqpoint{3.149404in}{1.995740in}}{\pgfqpoint{3.143581in}{2.001564in}}%
\pgfpathcurveto{\pgfqpoint{3.137757in}{2.007388in}}{\pgfqpoint{3.129857in}{2.010660in}}{\pgfqpoint{3.121620in}{2.010660in}}%
\pgfpathcurveto{\pgfqpoint{3.113384in}{2.010660in}}{\pgfqpoint{3.105484in}{2.007388in}}{\pgfqpoint{3.099660in}{2.001564in}}%
\pgfpathcurveto{\pgfqpoint{3.093836in}{1.995740in}}{\pgfqpoint{3.090564in}{1.987840in}}{\pgfqpoint{3.090564in}{1.979604in}}%
\pgfpathcurveto{\pgfqpoint{3.090564in}{1.971368in}}{\pgfqpoint{3.093836in}{1.963468in}}{\pgfqpoint{3.099660in}{1.957644in}}%
\pgfpathcurveto{\pgfqpoint{3.105484in}{1.951820in}}{\pgfqpoint{3.113384in}{1.948547in}}{\pgfqpoint{3.121620in}{1.948547in}}%
\pgfpathclose%
\pgfusepath{stroke,fill}%
\end{pgfscope}%
\begin{pgfscope}%
\pgfpathrectangle{\pgfqpoint{0.100000in}{0.212622in}}{\pgfqpoint{3.696000in}{3.696000in}}%
\pgfusepath{clip}%
\pgfsetbuttcap%
\pgfsetroundjoin%
\definecolor{currentfill}{rgb}{0.121569,0.466667,0.705882}%
\pgfsetfillcolor{currentfill}%
\pgfsetfillopacity{0.654742}%
\pgfsetlinewidth{1.003750pt}%
\definecolor{currentstroke}{rgb}{0.121569,0.466667,0.705882}%
\pgfsetstrokecolor{currentstroke}%
\pgfsetstrokeopacity{0.654742}%
\pgfsetdash{}{0pt}%
\pgfpathmoveto{\pgfqpoint{0.810796in}{1.639129in}}%
\pgfpathcurveto{\pgfqpoint{0.819033in}{1.639129in}}{\pgfqpoint{0.826933in}{1.642401in}}{\pgfqpoint{0.832757in}{1.648225in}}%
\pgfpathcurveto{\pgfqpoint{0.838581in}{1.654049in}}{\pgfqpoint{0.841853in}{1.661949in}}{\pgfqpoint{0.841853in}{1.670186in}}%
\pgfpathcurveto{\pgfqpoint{0.841853in}{1.678422in}}{\pgfqpoint{0.838581in}{1.686322in}}{\pgfqpoint{0.832757in}{1.692146in}}%
\pgfpathcurveto{\pgfqpoint{0.826933in}{1.697970in}}{\pgfqpoint{0.819033in}{1.701242in}}{\pgfqpoint{0.810796in}{1.701242in}}%
\pgfpathcurveto{\pgfqpoint{0.802560in}{1.701242in}}{\pgfqpoint{0.794660in}{1.697970in}}{\pgfqpoint{0.788836in}{1.692146in}}%
\pgfpathcurveto{\pgfqpoint{0.783012in}{1.686322in}}{\pgfqpoint{0.779740in}{1.678422in}}{\pgfqpoint{0.779740in}{1.670186in}}%
\pgfpathcurveto{\pgfqpoint{0.779740in}{1.661949in}}{\pgfqpoint{0.783012in}{1.654049in}}{\pgfqpoint{0.788836in}{1.648225in}}%
\pgfpathcurveto{\pgfqpoint{0.794660in}{1.642401in}}{\pgfqpoint{0.802560in}{1.639129in}}{\pgfqpoint{0.810796in}{1.639129in}}%
\pgfpathclose%
\pgfusepath{stroke,fill}%
\end{pgfscope}%
\begin{pgfscope}%
\pgfpathrectangle{\pgfqpoint{0.100000in}{0.212622in}}{\pgfqpoint{3.696000in}{3.696000in}}%
\pgfusepath{clip}%
\pgfsetbuttcap%
\pgfsetroundjoin%
\definecolor{currentfill}{rgb}{0.121569,0.466667,0.705882}%
\pgfsetfillcolor{currentfill}%
\pgfsetfillopacity{0.655798}%
\pgfsetlinewidth{1.003750pt}%
\definecolor{currentstroke}{rgb}{0.121569,0.466667,0.705882}%
\pgfsetstrokecolor{currentstroke}%
\pgfsetstrokeopacity{0.655798}%
\pgfsetdash{}{0pt}%
\pgfpathmoveto{\pgfqpoint{0.805454in}{1.636840in}}%
\pgfpathcurveto{\pgfqpoint{0.813690in}{1.636840in}}{\pgfqpoint{0.821590in}{1.640112in}}{\pgfqpoint{0.827414in}{1.645936in}}%
\pgfpathcurveto{\pgfqpoint{0.833238in}{1.651760in}}{\pgfqpoint{0.836510in}{1.659660in}}{\pgfqpoint{0.836510in}{1.667896in}}%
\pgfpathcurveto{\pgfqpoint{0.836510in}{1.676133in}}{\pgfqpoint{0.833238in}{1.684033in}}{\pgfqpoint{0.827414in}{1.689857in}}%
\pgfpathcurveto{\pgfqpoint{0.821590in}{1.695681in}}{\pgfqpoint{0.813690in}{1.698953in}}{\pgfqpoint{0.805454in}{1.698953in}}%
\pgfpathcurveto{\pgfqpoint{0.797218in}{1.698953in}}{\pgfqpoint{0.789318in}{1.695681in}}{\pgfqpoint{0.783494in}{1.689857in}}%
\pgfpathcurveto{\pgfqpoint{0.777670in}{1.684033in}}{\pgfqpoint{0.774397in}{1.676133in}}{\pgfqpoint{0.774397in}{1.667896in}}%
\pgfpathcurveto{\pgfqpoint{0.774397in}{1.659660in}}{\pgfqpoint{0.777670in}{1.651760in}}{\pgfqpoint{0.783494in}{1.645936in}}%
\pgfpathcurveto{\pgfqpoint{0.789318in}{1.640112in}}{\pgfqpoint{0.797218in}{1.636840in}}{\pgfqpoint{0.805454in}{1.636840in}}%
\pgfpathclose%
\pgfusepath{stroke,fill}%
\end{pgfscope}%
\begin{pgfscope}%
\pgfpathrectangle{\pgfqpoint{0.100000in}{0.212622in}}{\pgfqpoint{3.696000in}{3.696000in}}%
\pgfusepath{clip}%
\pgfsetbuttcap%
\pgfsetroundjoin%
\definecolor{currentfill}{rgb}{0.121569,0.466667,0.705882}%
\pgfsetfillcolor{currentfill}%
\pgfsetfillopacity{0.655957}%
\pgfsetlinewidth{1.003750pt}%
\definecolor{currentstroke}{rgb}{0.121569,0.466667,0.705882}%
\pgfsetstrokecolor{currentstroke}%
\pgfsetstrokeopacity{0.655957}%
\pgfsetdash{}{0pt}%
\pgfpathmoveto{\pgfqpoint{3.120443in}{1.949334in}}%
\pgfpathcurveto{\pgfqpoint{3.128679in}{1.949334in}}{\pgfqpoint{3.136579in}{1.952606in}}{\pgfqpoint{3.142403in}{1.958430in}}%
\pgfpathcurveto{\pgfqpoint{3.148227in}{1.964254in}}{\pgfqpoint{3.151499in}{1.972154in}}{\pgfqpoint{3.151499in}{1.980391in}}%
\pgfpathcurveto{\pgfqpoint{3.151499in}{1.988627in}}{\pgfqpoint{3.148227in}{1.996527in}}{\pgfqpoint{3.142403in}{2.002351in}}%
\pgfpathcurveto{\pgfqpoint{3.136579in}{2.008175in}}{\pgfqpoint{3.128679in}{2.011447in}}{\pgfqpoint{3.120443in}{2.011447in}}%
\pgfpathcurveto{\pgfqpoint{3.112206in}{2.011447in}}{\pgfqpoint{3.104306in}{2.008175in}}{\pgfqpoint{3.098482in}{2.002351in}}%
\pgfpathcurveto{\pgfqpoint{3.092658in}{1.996527in}}{\pgfqpoint{3.089386in}{1.988627in}}{\pgfqpoint{3.089386in}{1.980391in}}%
\pgfpathcurveto{\pgfqpoint{3.089386in}{1.972154in}}{\pgfqpoint{3.092658in}{1.964254in}}{\pgfqpoint{3.098482in}{1.958430in}}%
\pgfpathcurveto{\pgfqpoint{3.104306in}{1.952606in}}{\pgfqpoint{3.112206in}{1.949334in}}{\pgfqpoint{3.120443in}{1.949334in}}%
\pgfpathclose%
\pgfusepath{stroke,fill}%
\end{pgfscope}%
\begin{pgfscope}%
\pgfpathrectangle{\pgfqpoint{0.100000in}{0.212622in}}{\pgfqpoint{3.696000in}{3.696000in}}%
\pgfusepath{clip}%
\pgfsetbuttcap%
\pgfsetroundjoin%
\definecolor{currentfill}{rgb}{0.121569,0.466667,0.705882}%
\pgfsetfillcolor{currentfill}%
\pgfsetfillopacity{0.656504}%
\pgfsetlinewidth{1.003750pt}%
\definecolor{currentstroke}{rgb}{0.121569,0.466667,0.705882}%
\pgfsetstrokecolor{currentstroke}%
\pgfsetstrokeopacity{0.656504}%
\pgfsetdash{}{0pt}%
\pgfpathmoveto{\pgfqpoint{3.119143in}{1.948737in}}%
\pgfpathcurveto{\pgfqpoint{3.127379in}{1.948737in}}{\pgfqpoint{3.135279in}{1.952010in}}{\pgfqpoint{3.141103in}{1.957834in}}%
\pgfpathcurveto{\pgfqpoint{3.146927in}{1.963658in}}{\pgfqpoint{3.150199in}{1.971558in}}{\pgfqpoint{3.150199in}{1.979794in}}%
\pgfpathcurveto{\pgfqpoint{3.150199in}{1.988030in}}{\pgfqpoint{3.146927in}{1.995930in}}{\pgfqpoint{3.141103in}{2.001754in}}%
\pgfpathcurveto{\pgfqpoint{3.135279in}{2.007578in}}{\pgfqpoint{3.127379in}{2.010850in}}{\pgfqpoint{3.119143in}{2.010850in}}%
\pgfpathcurveto{\pgfqpoint{3.110906in}{2.010850in}}{\pgfqpoint{3.103006in}{2.007578in}}{\pgfqpoint{3.097182in}{2.001754in}}%
\pgfpathcurveto{\pgfqpoint{3.091359in}{1.995930in}}{\pgfqpoint{3.088086in}{1.988030in}}{\pgfqpoint{3.088086in}{1.979794in}}%
\pgfpathcurveto{\pgfqpoint{3.088086in}{1.971558in}}{\pgfqpoint{3.091359in}{1.963658in}}{\pgfqpoint{3.097182in}{1.957834in}}%
\pgfpathcurveto{\pgfqpoint{3.103006in}{1.952010in}}{\pgfqpoint{3.110906in}{1.948737in}}{\pgfqpoint{3.119143in}{1.948737in}}%
\pgfpathclose%
\pgfusepath{stroke,fill}%
\end{pgfscope}%
\begin{pgfscope}%
\pgfpathrectangle{\pgfqpoint{0.100000in}{0.212622in}}{\pgfqpoint{3.696000in}{3.696000in}}%
\pgfusepath{clip}%
\pgfsetbuttcap%
\pgfsetroundjoin%
\definecolor{currentfill}{rgb}{0.121569,0.466667,0.705882}%
\pgfsetfillcolor{currentfill}%
\pgfsetfillopacity{0.657131}%
\pgfsetlinewidth{1.003750pt}%
\definecolor{currentstroke}{rgb}{0.121569,0.466667,0.705882}%
\pgfsetstrokecolor{currentstroke}%
\pgfsetstrokeopacity{0.657131}%
\pgfsetdash{}{0pt}%
\pgfpathmoveto{\pgfqpoint{3.117240in}{1.948188in}}%
\pgfpathcurveto{\pgfqpoint{3.125477in}{1.948188in}}{\pgfqpoint{3.133377in}{1.951460in}}{\pgfqpoint{3.139201in}{1.957284in}}%
\pgfpathcurveto{\pgfqpoint{3.145025in}{1.963108in}}{\pgfqpoint{3.148297in}{1.971008in}}{\pgfqpoint{3.148297in}{1.979244in}}%
\pgfpathcurveto{\pgfqpoint{3.148297in}{1.987480in}}{\pgfqpoint{3.145025in}{1.995381in}}{\pgfqpoint{3.139201in}{2.001204in}}%
\pgfpathcurveto{\pgfqpoint{3.133377in}{2.007028in}}{\pgfqpoint{3.125477in}{2.010301in}}{\pgfqpoint{3.117240in}{2.010301in}}%
\pgfpathcurveto{\pgfqpoint{3.109004in}{2.010301in}}{\pgfqpoint{3.101104in}{2.007028in}}{\pgfqpoint{3.095280in}{2.001204in}}%
\pgfpathcurveto{\pgfqpoint{3.089456in}{1.995381in}}{\pgfqpoint{3.086184in}{1.987480in}}{\pgfqpoint{3.086184in}{1.979244in}}%
\pgfpathcurveto{\pgfqpoint{3.086184in}{1.971008in}}{\pgfqpoint{3.089456in}{1.963108in}}{\pgfqpoint{3.095280in}{1.957284in}}%
\pgfpathcurveto{\pgfqpoint{3.101104in}{1.951460in}}{\pgfqpoint{3.109004in}{1.948188in}}{\pgfqpoint{3.117240in}{1.948188in}}%
\pgfpathclose%
\pgfusepath{stroke,fill}%
\end{pgfscope}%
\begin{pgfscope}%
\pgfpathrectangle{\pgfqpoint{0.100000in}{0.212622in}}{\pgfqpoint{3.696000in}{3.696000in}}%
\pgfusepath{clip}%
\pgfsetbuttcap%
\pgfsetroundjoin%
\definecolor{currentfill}{rgb}{0.121569,0.466667,0.705882}%
\pgfsetfillcolor{currentfill}%
\pgfsetfillopacity{0.657566}%
\pgfsetlinewidth{1.003750pt}%
\definecolor{currentstroke}{rgb}{0.121569,0.466667,0.705882}%
\pgfsetstrokecolor{currentstroke}%
\pgfsetstrokeopacity{0.657566}%
\pgfsetdash{}{0pt}%
\pgfpathmoveto{\pgfqpoint{0.802975in}{1.638803in}}%
\pgfpathcurveto{\pgfqpoint{0.811211in}{1.638803in}}{\pgfqpoint{0.819111in}{1.642076in}}{\pgfqpoint{0.824935in}{1.647900in}}%
\pgfpathcurveto{\pgfqpoint{0.830759in}{1.653724in}}{\pgfqpoint{0.834031in}{1.661624in}}{\pgfqpoint{0.834031in}{1.669860in}}%
\pgfpathcurveto{\pgfqpoint{0.834031in}{1.678096in}}{\pgfqpoint{0.830759in}{1.685996in}}{\pgfqpoint{0.824935in}{1.691820in}}%
\pgfpathcurveto{\pgfqpoint{0.819111in}{1.697644in}}{\pgfqpoint{0.811211in}{1.700916in}}{\pgfqpoint{0.802975in}{1.700916in}}%
\pgfpathcurveto{\pgfqpoint{0.794739in}{1.700916in}}{\pgfqpoint{0.786839in}{1.697644in}}{\pgfqpoint{0.781015in}{1.691820in}}%
\pgfpathcurveto{\pgfqpoint{0.775191in}{1.685996in}}{\pgfqpoint{0.771918in}{1.678096in}}{\pgfqpoint{0.771918in}{1.669860in}}%
\pgfpathcurveto{\pgfqpoint{0.771918in}{1.661624in}}{\pgfqpoint{0.775191in}{1.653724in}}{\pgfqpoint{0.781015in}{1.647900in}}%
\pgfpathcurveto{\pgfqpoint{0.786839in}{1.642076in}}{\pgfqpoint{0.794739in}{1.638803in}}{\pgfqpoint{0.802975in}{1.638803in}}%
\pgfpathclose%
\pgfusepath{stroke,fill}%
\end{pgfscope}%
\begin{pgfscope}%
\pgfpathrectangle{\pgfqpoint{0.100000in}{0.212622in}}{\pgfqpoint{3.696000in}{3.696000in}}%
\pgfusepath{clip}%
\pgfsetbuttcap%
\pgfsetroundjoin%
\definecolor{currentfill}{rgb}{0.121569,0.466667,0.705882}%
\pgfsetfillcolor{currentfill}%
\pgfsetfillopacity{0.658305}%
\pgfsetlinewidth{1.003750pt}%
\definecolor{currentstroke}{rgb}{0.121569,0.466667,0.705882}%
\pgfsetstrokecolor{currentstroke}%
\pgfsetstrokeopacity{0.658305}%
\pgfsetdash{}{0pt}%
\pgfpathmoveto{\pgfqpoint{3.115610in}{1.946141in}}%
\pgfpathcurveto{\pgfqpoint{3.123846in}{1.946141in}}{\pgfqpoint{3.131747in}{1.949414in}}{\pgfqpoint{3.137570in}{1.955238in}}%
\pgfpathcurveto{\pgfqpoint{3.143394in}{1.961062in}}{\pgfqpoint{3.146667in}{1.968962in}}{\pgfqpoint{3.146667in}{1.977198in}}%
\pgfpathcurveto{\pgfqpoint{3.146667in}{1.985434in}}{\pgfqpoint{3.143394in}{1.993334in}}{\pgfqpoint{3.137570in}{1.999158in}}%
\pgfpathcurveto{\pgfqpoint{3.131747in}{2.004982in}}{\pgfqpoint{3.123846in}{2.008254in}}{\pgfqpoint{3.115610in}{2.008254in}}%
\pgfpathcurveto{\pgfqpoint{3.107374in}{2.008254in}}{\pgfqpoint{3.099474in}{2.004982in}}{\pgfqpoint{3.093650in}{1.999158in}}%
\pgfpathcurveto{\pgfqpoint{3.087826in}{1.993334in}}{\pgfqpoint{3.084554in}{1.985434in}}{\pgfqpoint{3.084554in}{1.977198in}}%
\pgfpathcurveto{\pgfqpoint{3.084554in}{1.968962in}}{\pgfqpoint{3.087826in}{1.961062in}}{\pgfqpoint{3.093650in}{1.955238in}}%
\pgfpathcurveto{\pgfqpoint{3.099474in}{1.949414in}}{\pgfqpoint{3.107374in}{1.946141in}}{\pgfqpoint{3.115610in}{1.946141in}}%
\pgfpathclose%
\pgfusepath{stroke,fill}%
\end{pgfscope}%
\begin{pgfscope}%
\pgfpathrectangle{\pgfqpoint{0.100000in}{0.212622in}}{\pgfqpoint{3.696000in}{3.696000in}}%
\pgfusepath{clip}%
\pgfsetbuttcap%
\pgfsetroundjoin%
\definecolor{currentfill}{rgb}{0.121569,0.466667,0.705882}%
\pgfsetfillcolor{currentfill}%
\pgfsetfillopacity{0.658356}%
\pgfsetlinewidth{1.003750pt}%
\definecolor{currentstroke}{rgb}{0.121569,0.466667,0.705882}%
\pgfsetstrokecolor{currentstroke}%
\pgfsetstrokeopacity{0.658356}%
\pgfsetdash{}{0pt}%
\pgfpathmoveto{\pgfqpoint{0.799441in}{1.640593in}}%
\pgfpathcurveto{\pgfqpoint{0.807677in}{1.640593in}}{\pgfqpoint{0.815577in}{1.643865in}}{\pgfqpoint{0.821401in}{1.649689in}}%
\pgfpathcurveto{\pgfqpoint{0.827225in}{1.655513in}}{\pgfqpoint{0.830498in}{1.663413in}}{\pgfqpoint{0.830498in}{1.671649in}}%
\pgfpathcurveto{\pgfqpoint{0.830498in}{1.679886in}}{\pgfqpoint{0.827225in}{1.687786in}}{\pgfqpoint{0.821401in}{1.693610in}}%
\pgfpathcurveto{\pgfqpoint{0.815577in}{1.699434in}}{\pgfqpoint{0.807677in}{1.702706in}}{\pgfqpoint{0.799441in}{1.702706in}}%
\pgfpathcurveto{\pgfqpoint{0.791205in}{1.702706in}}{\pgfqpoint{0.783305in}{1.699434in}}{\pgfqpoint{0.777481in}{1.693610in}}%
\pgfpathcurveto{\pgfqpoint{0.771657in}{1.687786in}}{\pgfqpoint{0.768385in}{1.679886in}}{\pgfqpoint{0.768385in}{1.671649in}}%
\pgfpathcurveto{\pgfqpoint{0.768385in}{1.663413in}}{\pgfqpoint{0.771657in}{1.655513in}}{\pgfqpoint{0.777481in}{1.649689in}}%
\pgfpathcurveto{\pgfqpoint{0.783305in}{1.643865in}}{\pgfqpoint{0.791205in}{1.640593in}}{\pgfqpoint{0.799441in}{1.640593in}}%
\pgfpathclose%
\pgfusepath{stroke,fill}%
\end{pgfscope}%
\begin{pgfscope}%
\pgfpathrectangle{\pgfqpoint{0.100000in}{0.212622in}}{\pgfqpoint{3.696000in}{3.696000in}}%
\pgfusepath{clip}%
\pgfsetbuttcap%
\pgfsetroundjoin%
\definecolor{currentfill}{rgb}{0.121569,0.466667,0.705882}%
\pgfsetfillcolor{currentfill}%
\pgfsetfillopacity{0.658531}%
\pgfsetlinewidth{1.003750pt}%
\definecolor{currentstroke}{rgb}{0.121569,0.466667,0.705882}%
\pgfsetstrokecolor{currentstroke}%
\pgfsetstrokeopacity{0.658531}%
\pgfsetdash{}{0pt}%
\pgfpathmoveto{\pgfqpoint{0.797843in}{1.637070in}}%
\pgfpathcurveto{\pgfqpoint{0.806080in}{1.637070in}}{\pgfqpoint{0.813980in}{1.640342in}}{\pgfqpoint{0.819804in}{1.646166in}}%
\pgfpathcurveto{\pgfqpoint{0.825628in}{1.651990in}}{\pgfqpoint{0.828900in}{1.659890in}}{\pgfqpoint{0.828900in}{1.668126in}}%
\pgfpathcurveto{\pgfqpoint{0.828900in}{1.676363in}}{\pgfqpoint{0.825628in}{1.684263in}}{\pgfqpoint{0.819804in}{1.690087in}}%
\pgfpathcurveto{\pgfqpoint{0.813980in}{1.695911in}}{\pgfqpoint{0.806080in}{1.699183in}}{\pgfqpoint{0.797843in}{1.699183in}}%
\pgfpathcurveto{\pgfqpoint{0.789607in}{1.699183in}}{\pgfqpoint{0.781707in}{1.695911in}}{\pgfqpoint{0.775883in}{1.690087in}}%
\pgfpathcurveto{\pgfqpoint{0.770059in}{1.684263in}}{\pgfqpoint{0.766787in}{1.676363in}}{\pgfqpoint{0.766787in}{1.668126in}}%
\pgfpathcurveto{\pgfqpoint{0.766787in}{1.659890in}}{\pgfqpoint{0.770059in}{1.651990in}}{\pgfqpoint{0.775883in}{1.646166in}}%
\pgfpathcurveto{\pgfqpoint{0.781707in}{1.640342in}}{\pgfqpoint{0.789607in}{1.637070in}}{\pgfqpoint{0.797843in}{1.637070in}}%
\pgfpathclose%
\pgfusepath{stroke,fill}%
\end{pgfscope}%
\begin{pgfscope}%
\pgfpathrectangle{\pgfqpoint{0.100000in}{0.212622in}}{\pgfqpoint{3.696000in}{3.696000in}}%
\pgfusepath{clip}%
\pgfsetbuttcap%
\pgfsetroundjoin%
\definecolor{currentfill}{rgb}{0.121569,0.466667,0.705882}%
\pgfsetfillcolor{currentfill}%
\pgfsetfillopacity{0.659475}%
\pgfsetlinewidth{1.003750pt}%
\definecolor{currentstroke}{rgb}{0.121569,0.466667,0.705882}%
\pgfsetstrokecolor{currentstroke}%
\pgfsetstrokeopacity{0.659475}%
\pgfsetdash{}{0pt}%
\pgfpathmoveto{\pgfqpoint{0.796010in}{1.633831in}}%
\pgfpathcurveto{\pgfqpoint{0.804247in}{1.633831in}}{\pgfqpoint{0.812147in}{1.637103in}}{\pgfqpoint{0.817971in}{1.642927in}}%
\pgfpathcurveto{\pgfqpoint{0.823794in}{1.648751in}}{\pgfqpoint{0.827067in}{1.656651in}}{\pgfqpoint{0.827067in}{1.664888in}}%
\pgfpathcurveto{\pgfqpoint{0.827067in}{1.673124in}}{\pgfqpoint{0.823794in}{1.681024in}}{\pgfqpoint{0.817971in}{1.686848in}}%
\pgfpathcurveto{\pgfqpoint{0.812147in}{1.692672in}}{\pgfqpoint{0.804247in}{1.695944in}}{\pgfqpoint{0.796010in}{1.695944in}}%
\pgfpathcurveto{\pgfqpoint{0.787774in}{1.695944in}}{\pgfqpoint{0.779874in}{1.692672in}}{\pgfqpoint{0.774050in}{1.686848in}}%
\pgfpathcurveto{\pgfqpoint{0.768226in}{1.681024in}}{\pgfqpoint{0.764954in}{1.673124in}}{\pgfqpoint{0.764954in}{1.664888in}}%
\pgfpathcurveto{\pgfqpoint{0.764954in}{1.656651in}}{\pgfqpoint{0.768226in}{1.648751in}}{\pgfqpoint{0.774050in}{1.642927in}}%
\pgfpathcurveto{\pgfqpoint{0.779874in}{1.637103in}}{\pgfqpoint{0.787774in}{1.633831in}}{\pgfqpoint{0.796010in}{1.633831in}}%
\pgfpathclose%
\pgfusepath{stroke,fill}%
\end{pgfscope}%
\begin{pgfscope}%
\pgfpathrectangle{\pgfqpoint{0.100000in}{0.212622in}}{\pgfqpoint{3.696000in}{3.696000in}}%
\pgfusepath{clip}%
\pgfsetbuttcap%
\pgfsetroundjoin%
\definecolor{currentfill}{rgb}{0.121569,0.466667,0.705882}%
\pgfsetfillcolor{currentfill}%
\pgfsetfillopacity{0.659586}%
\pgfsetlinewidth{1.003750pt}%
\definecolor{currentstroke}{rgb}{0.121569,0.466667,0.705882}%
\pgfsetstrokecolor{currentstroke}%
\pgfsetstrokeopacity{0.659586}%
\pgfsetdash{}{0pt}%
\pgfpathmoveto{\pgfqpoint{3.113204in}{1.943699in}}%
\pgfpathcurveto{\pgfqpoint{3.121441in}{1.943699in}}{\pgfqpoint{3.129341in}{1.946971in}}{\pgfqpoint{3.135165in}{1.952795in}}%
\pgfpathcurveto{\pgfqpoint{3.140989in}{1.958619in}}{\pgfqpoint{3.144261in}{1.966519in}}{\pgfqpoint{3.144261in}{1.974755in}}%
\pgfpathcurveto{\pgfqpoint{3.144261in}{1.982992in}}{\pgfqpoint{3.140989in}{1.990892in}}{\pgfqpoint{3.135165in}{1.996716in}}%
\pgfpathcurveto{\pgfqpoint{3.129341in}{2.002540in}}{\pgfqpoint{3.121441in}{2.005812in}}{\pgfqpoint{3.113204in}{2.005812in}}%
\pgfpathcurveto{\pgfqpoint{3.104968in}{2.005812in}}{\pgfqpoint{3.097068in}{2.002540in}}{\pgfqpoint{3.091244in}{1.996716in}}%
\pgfpathcurveto{\pgfqpoint{3.085420in}{1.990892in}}{\pgfqpoint{3.082148in}{1.982992in}}{\pgfqpoint{3.082148in}{1.974755in}}%
\pgfpathcurveto{\pgfqpoint{3.082148in}{1.966519in}}{\pgfqpoint{3.085420in}{1.958619in}}{\pgfqpoint{3.091244in}{1.952795in}}%
\pgfpathcurveto{\pgfqpoint{3.097068in}{1.946971in}}{\pgfqpoint{3.104968in}{1.943699in}}{\pgfqpoint{3.113204in}{1.943699in}}%
\pgfpathclose%
\pgfusepath{stroke,fill}%
\end{pgfscope}%
\begin{pgfscope}%
\pgfpathrectangle{\pgfqpoint{0.100000in}{0.212622in}}{\pgfqpoint{3.696000in}{3.696000in}}%
\pgfusepath{clip}%
\pgfsetbuttcap%
\pgfsetroundjoin%
\definecolor{currentfill}{rgb}{0.121569,0.466667,0.705882}%
\pgfsetfillcolor{currentfill}%
\pgfsetfillopacity{0.660172}%
\pgfsetlinewidth{1.003750pt}%
\definecolor{currentstroke}{rgb}{0.121569,0.466667,0.705882}%
\pgfsetstrokecolor{currentstroke}%
\pgfsetstrokeopacity{0.660172}%
\pgfsetdash{}{0pt}%
\pgfpathmoveto{\pgfqpoint{0.792949in}{1.631714in}}%
\pgfpathcurveto{\pgfqpoint{0.801185in}{1.631714in}}{\pgfqpoint{0.809086in}{1.634987in}}{\pgfqpoint{0.814909in}{1.640811in}}%
\pgfpathcurveto{\pgfqpoint{0.820733in}{1.646634in}}{\pgfqpoint{0.824006in}{1.654535in}}{\pgfqpoint{0.824006in}{1.662771in}}%
\pgfpathcurveto{\pgfqpoint{0.824006in}{1.671007in}}{\pgfqpoint{0.820733in}{1.678907in}}{\pgfqpoint{0.814909in}{1.684731in}}%
\pgfpathcurveto{\pgfqpoint{0.809086in}{1.690555in}}{\pgfqpoint{0.801185in}{1.693827in}}{\pgfqpoint{0.792949in}{1.693827in}}%
\pgfpathcurveto{\pgfqpoint{0.784713in}{1.693827in}}{\pgfqpoint{0.776813in}{1.690555in}}{\pgfqpoint{0.770989in}{1.684731in}}%
\pgfpathcurveto{\pgfqpoint{0.765165in}{1.678907in}}{\pgfqpoint{0.761893in}{1.671007in}}{\pgfqpoint{0.761893in}{1.662771in}}%
\pgfpathcurveto{\pgfqpoint{0.761893in}{1.654535in}}{\pgfqpoint{0.765165in}{1.646634in}}{\pgfqpoint{0.770989in}{1.640811in}}%
\pgfpathcurveto{\pgfqpoint{0.776813in}{1.634987in}}{\pgfqpoint{0.784713in}{1.631714in}}{\pgfqpoint{0.792949in}{1.631714in}}%
\pgfpathclose%
\pgfusepath{stroke,fill}%
\end{pgfscope}%
\begin{pgfscope}%
\pgfpathrectangle{\pgfqpoint{0.100000in}{0.212622in}}{\pgfqpoint{3.696000in}{3.696000in}}%
\pgfusepath{clip}%
\pgfsetbuttcap%
\pgfsetroundjoin%
\definecolor{currentfill}{rgb}{0.121569,0.466667,0.705882}%
\pgfsetfillcolor{currentfill}%
\pgfsetfillopacity{0.660754}%
\pgfsetlinewidth{1.003750pt}%
\definecolor{currentstroke}{rgb}{0.121569,0.466667,0.705882}%
\pgfsetstrokecolor{currentstroke}%
\pgfsetstrokeopacity{0.660754}%
\pgfsetdash{}{0pt}%
\pgfpathmoveto{\pgfqpoint{0.792226in}{1.631257in}}%
\pgfpathcurveto{\pgfqpoint{0.800462in}{1.631257in}}{\pgfqpoint{0.808362in}{1.634530in}}{\pgfqpoint{0.814186in}{1.640354in}}%
\pgfpathcurveto{\pgfqpoint{0.820010in}{1.646178in}}{\pgfqpoint{0.823282in}{1.654078in}}{\pgfqpoint{0.823282in}{1.662314in}}%
\pgfpathcurveto{\pgfqpoint{0.823282in}{1.670550in}}{\pgfqpoint{0.820010in}{1.678450in}}{\pgfqpoint{0.814186in}{1.684274in}}%
\pgfpathcurveto{\pgfqpoint{0.808362in}{1.690098in}}{\pgfqpoint{0.800462in}{1.693370in}}{\pgfqpoint{0.792226in}{1.693370in}}%
\pgfpathcurveto{\pgfqpoint{0.783989in}{1.693370in}}{\pgfqpoint{0.776089in}{1.690098in}}{\pgfqpoint{0.770265in}{1.684274in}}%
\pgfpathcurveto{\pgfqpoint{0.764441in}{1.678450in}}{\pgfqpoint{0.761169in}{1.670550in}}{\pgfqpoint{0.761169in}{1.662314in}}%
\pgfpathcurveto{\pgfqpoint{0.761169in}{1.654078in}}{\pgfqpoint{0.764441in}{1.646178in}}{\pgfqpoint{0.770265in}{1.640354in}}%
\pgfpathcurveto{\pgfqpoint{0.776089in}{1.634530in}}{\pgfqpoint{0.783989in}{1.631257in}}{\pgfqpoint{0.792226in}{1.631257in}}%
\pgfpathclose%
\pgfusepath{stroke,fill}%
\end{pgfscope}%
\begin{pgfscope}%
\pgfpathrectangle{\pgfqpoint{0.100000in}{0.212622in}}{\pgfqpoint{3.696000in}{3.696000in}}%
\pgfusepath{clip}%
\pgfsetbuttcap%
\pgfsetroundjoin%
\definecolor{currentfill}{rgb}{0.121569,0.466667,0.705882}%
\pgfsetfillcolor{currentfill}%
\pgfsetfillopacity{0.661110}%
\pgfsetlinewidth{1.003750pt}%
\definecolor{currentstroke}{rgb}{0.121569,0.466667,0.705882}%
\pgfsetstrokecolor{currentstroke}%
\pgfsetstrokeopacity{0.661110}%
\pgfsetdash{}{0pt}%
\pgfpathmoveto{\pgfqpoint{0.791073in}{1.631608in}}%
\pgfpathcurveto{\pgfqpoint{0.799309in}{1.631608in}}{\pgfqpoint{0.807209in}{1.634881in}}{\pgfqpoint{0.813033in}{1.640705in}}%
\pgfpathcurveto{\pgfqpoint{0.818857in}{1.646529in}}{\pgfqpoint{0.822130in}{1.654429in}}{\pgfqpoint{0.822130in}{1.662665in}}%
\pgfpathcurveto{\pgfqpoint{0.822130in}{1.670901in}}{\pgfqpoint{0.818857in}{1.678801in}}{\pgfqpoint{0.813033in}{1.684625in}}%
\pgfpathcurveto{\pgfqpoint{0.807209in}{1.690449in}}{\pgfqpoint{0.799309in}{1.693721in}}{\pgfqpoint{0.791073in}{1.693721in}}%
\pgfpathcurveto{\pgfqpoint{0.782837in}{1.693721in}}{\pgfqpoint{0.774937in}{1.690449in}}{\pgfqpoint{0.769113in}{1.684625in}}%
\pgfpathcurveto{\pgfqpoint{0.763289in}{1.678801in}}{\pgfqpoint{0.760017in}{1.670901in}}{\pgfqpoint{0.760017in}{1.662665in}}%
\pgfpathcurveto{\pgfqpoint{0.760017in}{1.654429in}}{\pgfqpoint{0.763289in}{1.646529in}}{\pgfqpoint{0.769113in}{1.640705in}}%
\pgfpathcurveto{\pgfqpoint{0.774937in}{1.634881in}}{\pgfqpoint{0.782837in}{1.631608in}}{\pgfqpoint{0.791073in}{1.631608in}}%
\pgfpathclose%
\pgfusepath{stroke,fill}%
\end{pgfscope}%
\begin{pgfscope}%
\pgfpathrectangle{\pgfqpoint{0.100000in}{0.212622in}}{\pgfqpoint{3.696000in}{3.696000in}}%
\pgfusepath{clip}%
\pgfsetbuttcap%
\pgfsetroundjoin%
\definecolor{currentfill}{rgb}{0.121569,0.466667,0.705882}%
\pgfsetfillcolor{currentfill}%
\pgfsetfillopacity{0.661241}%
\pgfsetlinewidth{1.003750pt}%
\definecolor{currentstroke}{rgb}{0.121569,0.466667,0.705882}%
\pgfsetstrokecolor{currentstroke}%
\pgfsetstrokeopacity{0.661241}%
\pgfsetdash{}{0pt}%
\pgfpathmoveto{\pgfqpoint{3.108418in}{1.942526in}}%
\pgfpathcurveto{\pgfqpoint{3.116654in}{1.942526in}}{\pgfqpoint{3.124554in}{1.945799in}}{\pgfqpoint{3.130378in}{1.951623in}}%
\pgfpathcurveto{\pgfqpoint{3.136202in}{1.957446in}}{\pgfqpoint{3.139474in}{1.965346in}}{\pgfqpoint{3.139474in}{1.973583in}}%
\pgfpathcurveto{\pgfqpoint{3.139474in}{1.981819in}}{\pgfqpoint{3.136202in}{1.989719in}}{\pgfqpoint{3.130378in}{1.995543in}}%
\pgfpathcurveto{\pgfqpoint{3.124554in}{2.001367in}}{\pgfqpoint{3.116654in}{2.004639in}}{\pgfqpoint{3.108418in}{2.004639in}}%
\pgfpathcurveto{\pgfqpoint{3.100181in}{2.004639in}}{\pgfqpoint{3.092281in}{2.001367in}}{\pgfqpoint{3.086457in}{1.995543in}}%
\pgfpathcurveto{\pgfqpoint{3.080633in}{1.989719in}}{\pgfqpoint{3.077361in}{1.981819in}}{\pgfqpoint{3.077361in}{1.973583in}}%
\pgfpathcurveto{\pgfqpoint{3.077361in}{1.965346in}}{\pgfqpoint{3.080633in}{1.957446in}}{\pgfqpoint{3.086457in}{1.951623in}}%
\pgfpathcurveto{\pgfqpoint{3.092281in}{1.945799in}}{\pgfqpoint{3.100181in}{1.942526in}}{\pgfqpoint{3.108418in}{1.942526in}}%
\pgfpathclose%
\pgfusepath{stroke,fill}%
\end{pgfscope}%
\begin{pgfscope}%
\pgfpathrectangle{\pgfqpoint{0.100000in}{0.212622in}}{\pgfqpoint{3.696000in}{3.696000in}}%
\pgfusepath{clip}%
\pgfsetbuttcap%
\pgfsetroundjoin%
\definecolor{currentfill}{rgb}{0.121569,0.466667,0.705882}%
\pgfsetfillcolor{currentfill}%
\pgfsetfillopacity{0.661873}%
\pgfsetlinewidth{1.003750pt}%
\definecolor{currentstroke}{rgb}{0.121569,0.466667,0.705882}%
\pgfsetstrokecolor{currentstroke}%
\pgfsetstrokeopacity{0.661873}%
\pgfsetdash{}{0pt}%
\pgfpathmoveto{\pgfqpoint{0.790572in}{1.631482in}}%
\pgfpathcurveto{\pgfqpoint{0.798808in}{1.631482in}}{\pgfqpoint{0.806709in}{1.634754in}}{\pgfqpoint{0.812532in}{1.640578in}}%
\pgfpathcurveto{\pgfqpoint{0.818356in}{1.646402in}}{\pgfqpoint{0.821629in}{1.654302in}}{\pgfqpoint{0.821629in}{1.662539in}}%
\pgfpathcurveto{\pgfqpoint{0.821629in}{1.670775in}}{\pgfqpoint{0.818356in}{1.678675in}}{\pgfqpoint{0.812532in}{1.684499in}}%
\pgfpathcurveto{\pgfqpoint{0.806709in}{1.690323in}}{\pgfqpoint{0.798808in}{1.693595in}}{\pgfqpoint{0.790572in}{1.693595in}}%
\pgfpathcurveto{\pgfqpoint{0.782336in}{1.693595in}}{\pgfqpoint{0.774436in}{1.690323in}}{\pgfqpoint{0.768612in}{1.684499in}}%
\pgfpathcurveto{\pgfqpoint{0.762788in}{1.678675in}}{\pgfqpoint{0.759516in}{1.670775in}}{\pgfqpoint{0.759516in}{1.662539in}}%
\pgfpathcurveto{\pgfqpoint{0.759516in}{1.654302in}}{\pgfqpoint{0.762788in}{1.646402in}}{\pgfqpoint{0.768612in}{1.640578in}}%
\pgfpathcurveto{\pgfqpoint{0.774436in}{1.634754in}}{\pgfqpoint{0.782336in}{1.631482in}}{\pgfqpoint{0.790572in}{1.631482in}}%
\pgfpathclose%
\pgfusepath{stroke,fill}%
\end{pgfscope}%
\begin{pgfscope}%
\pgfpathrectangle{\pgfqpoint{0.100000in}{0.212622in}}{\pgfqpoint{3.696000in}{3.696000in}}%
\pgfusepath{clip}%
\pgfsetbuttcap%
\pgfsetroundjoin%
\definecolor{currentfill}{rgb}{0.121569,0.466667,0.705882}%
\pgfsetfillcolor{currentfill}%
\pgfsetfillopacity{0.662447}%
\pgfsetlinewidth{1.003750pt}%
\definecolor{currentstroke}{rgb}{0.121569,0.466667,0.705882}%
\pgfsetstrokecolor{currentstroke}%
\pgfsetstrokeopacity{0.662447}%
\pgfsetdash{}{0pt}%
\pgfpathmoveto{\pgfqpoint{0.788271in}{1.626508in}}%
\pgfpathcurveto{\pgfqpoint{0.796507in}{1.626508in}}{\pgfqpoint{0.804407in}{1.629780in}}{\pgfqpoint{0.810231in}{1.635604in}}%
\pgfpathcurveto{\pgfqpoint{0.816055in}{1.641428in}}{\pgfqpoint{0.819328in}{1.649328in}}{\pgfqpoint{0.819328in}{1.657565in}}%
\pgfpathcurveto{\pgfqpoint{0.819328in}{1.665801in}}{\pgfqpoint{0.816055in}{1.673701in}}{\pgfqpoint{0.810231in}{1.679525in}}%
\pgfpathcurveto{\pgfqpoint{0.804407in}{1.685349in}}{\pgfqpoint{0.796507in}{1.688621in}}{\pgfqpoint{0.788271in}{1.688621in}}%
\pgfpathcurveto{\pgfqpoint{0.780035in}{1.688621in}}{\pgfqpoint{0.772135in}{1.685349in}}{\pgfqpoint{0.766311in}{1.679525in}}%
\pgfpathcurveto{\pgfqpoint{0.760487in}{1.673701in}}{\pgfqpoint{0.757215in}{1.665801in}}{\pgfqpoint{0.757215in}{1.657565in}}%
\pgfpathcurveto{\pgfqpoint{0.757215in}{1.649328in}}{\pgfqpoint{0.760487in}{1.641428in}}{\pgfqpoint{0.766311in}{1.635604in}}%
\pgfpathcurveto{\pgfqpoint{0.772135in}{1.629780in}}{\pgfqpoint{0.780035in}{1.626508in}}{\pgfqpoint{0.788271in}{1.626508in}}%
\pgfpathclose%
\pgfusepath{stroke,fill}%
\end{pgfscope}%
\begin{pgfscope}%
\pgfpathrectangle{\pgfqpoint{0.100000in}{0.212622in}}{\pgfqpoint{3.696000in}{3.696000in}}%
\pgfusepath{clip}%
\pgfsetbuttcap%
\pgfsetroundjoin%
\definecolor{currentfill}{rgb}{0.121569,0.466667,0.705882}%
\pgfsetfillcolor{currentfill}%
\pgfsetfillopacity{0.662824}%
\pgfsetlinewidth{1.003750pt}%
\definecolor{currentstroke}{rgb}{0.121569,0.466667,0.705882}%
\pgfsetstrokecolor{currentstroke}%
\pgfsetstrokeopacity{0.662824}%
\pgfsetdash{}{0pt}%
\pgfpathmoveto{\pgfqpoint{3.103923in}{1.939100in}}%
\pgfpathcurveto{\pgfqpoint{3.112159in}{1.939100in}}{\pgfqpoint{3.120060in}{1.942373in}}{\pgfqpoint{3.125883in}{1.948196in}}%
\pgfpathcurveto{\pgfqpoint{3.131707in}{1.954020in}}{\pgfqpoint{3.134980in}{1.961920in}}{\pgfqpoint{3.134980in}{1.970157in}}%
\pgfpathcurveto{\pgfqpoint{3.134980in}{1.978393in}}{\pgfqpoint{3.131707in}{1.986293in}}{\pgfqpoint{3.125883in}{1.992117in}}%
\pgfpathcurveto{\pgfqpoint{3.120060in}{1.997941in}}{\pgfqpoint{3.112159in}{2.001213in}}{\pgfqpoint{3.103923in}{2.001213in}}%
\pgfpathcurveto{\pgfqpoint{3.095687in}{2.001213in}}{\pgfqpoint{3.087787in}{1.997941in}}{\pgfqpoint{3.081963in}{1.992117in}}%
\pgfpathcurveto{\pgfqpoint{3.076139in}{1.986293in}}{\pgfqpoint{3.072867in}{1.978393in}}{\pgfqpoint{3.072867in}{1.970157in}}%
\pgfpathcurveto{\pgfqpoint{3.072867in}{1.961920in}}{\pgfqpoint{3.076139in}{1.954020in}}{\pgfqpoint{3.081963in}{1.948196in}}%
\pgfpathcurveto{\pgfqpoint{3.087787in}{1.942373in}}{\pgfqpoint{3.095687in}{1.939100in}}{\pgfqpoint{3.103923in}{1.939100in}}%
\pgfpathclose%
\pgfusepath{stroke,fill}%
\end{pgfscope}%
\begin{pgfscope}%
\pgfpathrectangle{\pgfqpoint{0.100000in}{0.212622in}}{\pgfqpoint{3.696000in}{3.696000in}}%
\pgfusepath{clip}%
\pgfsetbuttcap%
\pgfsetroundjoin%
\definecolor{currentfill}{rgb}{0.121569,0.466667,0.705882}%
\pgfsetfillcolor{currentfill}%
\pgfsetfillopacity{0.663307}%
\pgfsetlinewidth{1.003750pt}%
\definecolor{currentstroke}{rgb}{0.121569,0.466667,0.705882}%
\pgfsetstrokecolor{currentstroke}%
\pgfsetstrokeopacity{0.663307}%
\pgfsetdash{}{0pt}%
\pgfpathmoveto{\pgfqpoint{0.785641in}{1.624687in}}%
\pgfpathcurveto{\pgfqpoint{0.793877in}{1.624687in}}{\pgfqpoint{0.801777in}{1.627959in}}{\pgfqpoint{0.807601in}{1.633783in}}%
\pgfpathcurveto{\pgfqpoint{0.813425in}{1.639607in}}{\pgfqpoint{0.816697in}{1.647507in}}{\pgfqpoint{0.816697in}{1.655743in}}%
\pgfpathcurveto{\pgfqpoint{0.816697in}{1.663979in}}{\pgfqpoint{0.813425in}{1.671879in}}{\pgfqpoint{0.807601in}{1.677703in}}%
\pgfpathcurveto{\pgfqpoint{0.801777in}{1.683527in}}{\pgfqpoint{0.793877in}{1.686800in}}{\pgfqpoint{0.785641in}{1.686800in}}%
\pgfpathcurveto{\pgfqpoint{0.777405in}{1.686800in}}{\pgfqpoint{0.769505in}{1.683527in}}{\pgfqpoint{0.763681in}{1.677703in}}%
\pgfpathcurveto{\pgfqpoint{0.757857in}{1.671879in}}{\pgfqpoint{0.754584in}{1.663979in}}{\pgfqpoint{0.754584in}{1.655743in}}%
\pgfpathcurveto{\pgfqpoint{0.754584in}{1.647507in}}{\pgfqpoint{0.757857in}{1.639607in}}{\pgfqpoint{0.763681in}{1.633783in}}%
\pgfpathcurveto{\pgfqpoint{0.769505in}{1.627959in}}{\pgfqpoint{0.777405in}{1.624687in}}{\pgfqpoint{0.785641in}{1.624687in}}%
\pgfpathclose%
\pgfusepath{stroke,fill}%
\end{pgfscope}%
\begin{pgfscope}%
\pgfpathrectangle{\pgfqpoint{0.100000in}{0.212622in}}{\pgfqpoint{3.696000in}{3.696000in}}%
\pgfusepath{clip}%
\pgfsetbuttcap%
\pgfsetroundjoin%
\definecolor{currentfill}{rgb}{0.121569,0.466667,0.705882}%
\pgfsetfillcolor{currentfill}%
\pgfsetfillopacity{0.664070}%
\pgfsetlinewidth{1.003750pt}%
\definecolor{currentstroke}{rgb}{0.121569,0.466667,0.705882}%
\pgfsetstrokecolor{currentstroke}%
\pgfsetstrokeopacity{0.664070}%
\pgfsetdash{}{0pt}%
\pgfpathmoveto{\pgfqpoint{3.102483in}{1.938855in}}%
\pgfpathcurveto{\pgfqpoint{3.110719in}{1.938855in}}{\pgfqpoint{3.118619in}{1.942127in}}{\pgfqpoint{3.124443in}{1.947951in}}%
\pgfpathcurveto{\pgfqpoint{3.130267in}{1.953775in}}{\pgfqpoint{3.133540in}{1.961675in}}{\pgfqpoint{3.133540in}{1.969911in}}%
\pgfpathcurveto{\pgfqpoint{3.133540in}{1.978147in}}{\pgfqpoint{3.130267in}{1.986047in}}{\pgfqpoint{3.124443in}{1.991871in}}%
\pgfpathcurveto{\pgfqpoint{3.118619in}{1.997695in}}{\pgfqpoint{3.110719in}{2.000968in}}{\pgfqpoint{3.102483in}{2.000968in}}%
\pgfpathcurveto{\pgfqpoint{3.094247in}{2.000968in}}{\pgfqpoint{3.086347in}{1.997695in}}{\pgfqpoint{3.080523in}{1.991871in}}%
\pgfpathcurveto{\pgfqpoint{3.074699in}{1.986047in}}{\pgfqpoint{3.071427in}{1.978147in}}{\pgfqpoint{3.071427in}{1.969911in}}%
\pgfpathcurveto{\pgfqpoint{3.071427in}{1.961675in}}{\pgfqpoint{3.074699in}{1.953775in}}{\pgfqpoint{3.080523in}{1.947951in}}%
\pgfpathcurveto{\pgfqpoint{3.086347in}{1.942127in}}{\pgfqpoint{3.094247in}{1.938855in}}{\pgfqpoint{3.102483in}{1.938855in}}%
\pgfpathclose%
\pgfusepath{stroke,fill}%
\end{pgfscope}%
\begin{pgfscope}%
\pgfpathrectangle{\pgfqpoint{0.100000in}{0.212622in}}{\pgfqpoint{3.696000in}{3.696000in}}%
\pgfusepath{clip}%
\pgfsetbuttcap%
\pgfsetroundjoin%
\definecolor{currentfill}{rgb}{0.121569,0.466667,0.705882}%
\pgfsetfillcolor{currentfill}%
\pgfsetfillopacity{0.665106}%
\pgfsetlinewidth{1.003750pt}%
\definecolor{currentstroke}{rgb}{0.121569,0.466667,0.705882}%
\pgfsetstrokecolor{currentstroke}%
\pgfsetstrokeopacity{0.665106}%
\pgfsetdash{}{0pt}%
\pgfpathmoveto{\pgfqpoint{0.782847in}{1.621181in}}%
\pgfpathcurveto{\pgfqpoint{0.791083in}{1.621181in}}{\pgfqpoint{0.798983in}{1.624454in}}{\pgfqpoint{0.804807in}{1.630277in}}%
\pgfpathcurveto{\pgfqpoint{0.810631in}{1.636101in}}{\pgfqpoint{0.813903in}{1.644001in}}{\pgfqpoint{0.813903in}{1.652238in}}%
\pgfpathcurveto{\pgfqpoint{0.813903in}{1.660474in}}{\pgfqpoint{0.810631in}{1.668374in}}{\pgfqpoint{0.804807in}{1.674198in}}%
\pgfpathcurveto{\pgfqpoint{0.798983in}{1.680022in}}{\pgfqpoint{0.791083in}{1.683294in}}{\pgfqpoint{0.782847in}{1.683294in}}%
\pgfpathcurveto{\pgfqpoint{0.774611in}{1.683294in}}{\pgfqpoint{0.766711in}{1.680022in}}{\pgfqpoint{0.760887in}{1.674198in}}%
\pgfpathcurveto{\pgfqpoint{0.755063in}{1.668374in}}{\pgfqpoint{0.751790in}{1.660474in}}{\pgfqpoint{0.751790in}{1.652238in}}%
\pgfpathcurveto{\pgfqpoint{0.751790in}{1.644001in}}{\pgfqpoint{0.755063in}{1.636101in}}{\pgfqpoint{0.760887in}{1.630277in}}%
\pgfpathcurveto{\pgfqpoint{0.766711in}{1.624454in}}{\pgfqpoint{0.774611in}{1.621181in}}{\pgfqpoint{0.782847in}{1.621181in}}%
\pgfpathclose%
\pgfusepath{stroke,fill}%
\end{pgfscope}%
\begin{pgfscope}%
\pgfpathrectangle{\pgfqpoint{0.100000in}{0.212622in}}{\pgfqpoint{3.696000in}{3.696000in}}%
\pgfusepath{clip}%
\pgfsetbuttcap%
\pgfsetroundjoin%
\definecolor{currentfill}{rgb}{0.121569,0.466667,0.705882}%
\pgfsetfillcolor{currentfill}%
\pgfsetfillopacity{0.665129}%
\pgfsetlinewidth{1.003750pt}%
\definecolor{currentstroke}{rgb}{0.121569,0.466667,0.705882}%
\pgfsetstrokecolor{currentstroke}%
\pgfsetstrokeopacity{0.665129}%
\pgfsetdash{}{0pt}%
\pgfpathmoveto{\pgfqpoint{3.100091in}{1.937217in}}%
\pgfpathcurveto{\pgfqpoint{3.108328in}{1.937217in}}{\pgfqpoint{3.116228in}{1.940489in}}{\pgfqpoint{3.122052in}{1.946313in}}%
\pgfpathcurveto{\pgfqpoint{3.127876in}{1.952137in}}{\pgfqpoint{3.131148in}{1.960037in}}{\pgfqpoint{3.131148in}{1.968274in}}%
\pgfpathcurveto{\pgfqpoint{3.131148in}{1.976510in}}{\pgfqpoint{3.127876in}{1.984410in}}{\pgfqpoint{3.122052in}{1.990234in}}%
\pgfpathcurveto{\pgfqpoint{3.116228in}{1.996058in}}{\pgfqpoint{3.108328in}{1.999330in}}{\pgfqpoint{3.100091in}{1.999330in}}%
\pgfpathcurveto{\pgfqpoint{3.091855in}{1.999330in}}{\pgfqpoint{3.083955in}{1.996058in}}{\pgfqpoint{3.078131in}{1.990234in}}%
\pgfpathcurveto{\pgfqpoint{3.072307in}{1.984410in}}{\pgfqpoint{3.069035in}{1.976510in}}{\pgfqpoint{3.069035in}{1.968274in}}%
\pgfpathcurveto{\pgfqpoint{3.069035in}{1.960037in}}{\pgfqpoint{3.072307in}{1.952137in}}{\pgfqpoint{3.078131in}{1.946313in}}%
\pgfpathcurveto{\pgfqpoint{3.083955in}{1.940489in}}{\pgfqpoint{3.091855in}{1.937217in}}{\pgfqpoint{3.100091in}{1.937217in}}%
\pgfpathclose%
\pgfusepath{stroke,fill}%
\end{pgfscope}%
\begin{pgfscope}%
\pgfpathrectangle{\pgfqpoint{0.100000in}{0.212622in}}{\pgfqpoint{3.696000in}{3.696000in}}%
\pgfusepath{clip}%
\pgfsetbuttcap%
\pgfsetroundjoin%
\definecolor{currentfill}{rgb}{0.121569,0.466667,0.705882}%
\pgfsetfillcolor{currentfill}%
\pgfsetfillopacity{0.666413}%
\pgfsetlinewidth{1.003750pt}%
\definecolor{currentstroke}{rgb}{0.121569,0.466667,0.705882}%
\pgfsetstrokecolor{currentstroke}%
\pgfsetstrokeopacity{0.666413}%
\pgfsetdash{}{0pt}%
\pgfpathmoveto{\pgfqpoint{0.777846in}{1.619403in}}%
\pgfpathcurveto{\pgfqpoint{0.786082in}{1.619403in}}{\pgfqpoint{0.793982in}{1.622676in}}{\pgfqpoint{0.799806in}{1.628500in}}%
\pgfpathcurveto{\pgfqpoint{0.805630in}{1.634324in}}{\pgfqpoint{0.808902in}{1.642224in}}{\pgfqpoint{0.808902in}{1.650460in}}%
\pgfpathcurveto{\pgfqpoint{0.808902in}{1.658696in}}{\pgfqpoint{0.805630in}{1.666596in}}{\pgfqpoint{0.799806in}{1.672420in}}%
\pgfpathcurveto{\pgfqpoint{0.793982in}{1.678244in}}{\pgfqpoint{0.786082in}{1.681516in}}{\pgfqpoint{0.777846in}{1.681516in}}%
\pgfpathcurveto{\pgfqpoint{0.769609in}{1.681516in}}{\pgfqpoint{0.761709in}{1.678244in}}{\pgfqpoint{0.755885in}{1.672420in}}%
\pgfpathcurveto{\pgfqpoint{0.750061in}{1.666596in}}{\pgfqpoint{0.746789in}{1.658696in}}{\pgfqpoint{0.746789in}{1.650460in}}%
\pgfpathcurveto{\pgfqpoint{0.746789in}{1.642224in}}{\pgfqpoint{0.750061in}{1.634324in}}{\pgfqpoint{0.755885in}{1.628500in}}%
\pgfpathcurveto{\pgfqpoint{0.761709in}{1.622676in}}{\pgfqpoint{0.769609in}{1.619403in}}{\pgfqpoint{0.777846in}{1.619403in}}%
\pgfpathclose%
\pgfusepath{stroke,fill}%
\end{pgfscope}%
\begin{pgfscope}%
\pgfpathrectangle{\pgfqpoint{0.100000in}{0.212622in}}{\pgfqpoint{3.696000in}{3.696000in}}%
\pgfusepath{clip}%
\pgfsetbuttcap%
\pgfsetroundjoin%
\definecolor{currentfill}{rgb}{0.121569,0.466667,0.705882}%
\pgfsetfillcolor{currentfill}%
\pgfsetfillopacity{0.666734}%
\pgfsetlinewidth{1.003750pt}%
\definecolor{currentstroke}{rgb}{0.121569,0.466667,0.705882}%
\pgfsetstrokecolor{currentstroke}%
\pgfsetstrokeopacity{0.666734}%
\pgfsetdash{}{0pt}%
\pgfpathmoveto{\pgfqpoint{3.096217in}{1.939128in}}%
\pgfpathcurveto{\pgfqpoint{3.104453in}{1.939128in}}{\pgfqpoint{3.112353in}{1.942400in}}{\pgfqpoint{3.118177in}{1.948224in}}%
\pgfpathcurveto{\pgfqpoint{3.124001in}{1.954048in}}{\pgfqpoint{3.127273in}{1.961948in}}{\pgfqpoint{3.127273in}{1.970185in}}%
\pgfpathcurveto{\pgfqpoint{3.127273in}{1.978421in}}{\pgfqpoint{3.124001in}{1.986321in}}{\pgfqpoint{3.118177in}{1.992145in}}%
\pgfpathcurveto{\pgfqpoint{3.112353in}{1.997969in}}{\pgfqpoint{3.104453in}{2.001241in}}{\pgfqpoint{3.096217in}{2.001241in}}%
\pgfpathcurveto{\pgfqpoint{3.087981in}{2.001241in}}{\pgfqpoint{3.080081in}{1.997969in}}{\pgfqpoint{3.074257in}{1.992145in}}%
\pgfpathcurveto{\pgfqpoint{3.068433in}{1.986321in}}{\pgfqpoint{3.065160in}{1.978421in}}{\pgfqpoint{3.065160in}{1.970185in}}%
\pgfpathcurveto{\pgfqpoint{3.065160in}{1.961948in}}{\pgfqpoint{3.068433in}{1.954048in}}{\pgfqpoint{3.074257in}{1.948224in}}%
\pgfpathcurveto{\pgfqpoint{3.080081in}{1.942400in}}{\pgfqpoint{3.087981in}{1.939128in}}{\pgfqpoint{3.096217in}{1.939128in}}%
\pgfpathclose%
\pgfusepath{stroke,fill}%
\end{pgfscope}%
\begin{pgfscope}%
\pgfpathrectangle{\pgfqpoint{0.100000in}{0.212622in}}{\pgfqpoint{3.696000in}{3.696000in}}%
\pgfusepath{clip}%
\pgfsetbuttcap%
\pgfsetroundjoin%
\definecolor{currentfill}{rgb}{0.121569,0.466667,0.705882}%
\pgfsetfillcolor{currentfill}%
\pgfsetfillopacity{0.668186}%
\pgfsetlinewidth{1.003750pt}%
\definecolor{currentstroke}{rgb}{0.121569,0.466667,0.705882}%
\pgfsetstrokecolor{currentstroke}%
\pgfsetstrokeopacity{0.668186}%
\pgfsetdash{}{0pt}%
\pgfpathmoveto{\pgfqpoint{3.094573in}{1.933987in}}%
\pgfpathcurveto{\pgfqpoint{3.102809in}{1.933987in}}{\pgfqpoint{3.110709in}{1.937259in}}{\pgfqpoint{3.116533in}{1.943083in}}%
\pgfpathcurveto{\pgfqpoint{3.122357in}{1.948907in}}{\pgfqpoint{3.125630in}{1.956807in}}{\pgfqpoint{3.125630in}{1.965043in}}%
\pgfpathcurveto{\pgfqpoint{3.125630in}{1.973280in}}{\pgfqpoint{3.122357in}{1.981180in}}{\pgfqpoint{3.116533in}{1.987004in}}%
\pgfpathcurveto{\pgfqpoint{3.110709in}{1.992828in}}{\pgfqpoint{3.102809in}{1.996100in}}{\pgfqpoint{3.094573in}{1.996100in}}%
\pgfpathcurveto{\pgfqpoint{3.086337in}{1.996100in}}{\pgfqpoint{3.078437in}{1.992828in}}{\pgfqpoint{3.072613in}{1.987004in}}%
\pgfpathcurveto{\pgfqpoint{3.066789in}{1.981180in}}{\pgfqpoint{3.063517in}{1.973280in}}{\pgfqpoint{3.063517in}{1.965043in}}%
\pgfpathcurveto{\pgfqpoint{3.063517in}{1.956807in}}{\pgfqpoint{3.066789in}{1.948907in}}{\pgfqpoint{3.072613in}{1.943083in}}%
\pgfpathcurveto{\pgfqpoint{3.078437in}{1.937259in}}{\pgfqpoint{3.086337in}{1.933987in}}{\pgfqpoint{3.094573in}{1.933987in}}%
\pgfpathclose%
\pgfusepath{stroke,fill}%
\end{pgfscope}%
\begin{pgfscope}%
\pgfpathrectangle{\pgfqpoint{0.100000in}{0.212622in}}{\pgfqpoint{3.696000in}{3.696000in}}%
\pgfusepath{clip}%
\pgfsetbuttcap%
\pgfsetroundjoin%
\definecolor{currentfill}{rgb}{0.121569,0.466667,0.705882}%
\pgfsetfillcolor{currentfill}%
\pgfsetfillopacity{0.668725}%
\pgfsetlinewidth{1.003750pt}%
\definecolor{currentstroke}{rgb}{0.121569,0.466667,0.705882}%
\pgfsetstrokecolor{currentstroke}%
\pgfsetstrokeopacity{0.668725}%
\pgfsetdash{}{0pt}%
\pgfpathmoveto{\pgfqpoint{0.775718in}{1.622833in}}%
\pgfpathcurveto{\pgfqpoint{0.783954in}{1.622833in}}{\pgfqpoint{0.791854in}{1.626105in}}{\pgfqpoint{0.797678in}{1.631929in}}%
\pgfpathcurveto{\pgfqpoint{0.803502in}{1.637753in}}{\pgfqpoint{0.806775in}{1.645653in}}{\pgfqpoint{0.806775in}{1.653889in}}%
\pgfpathcurveto{\pgfqpoint{0.806775in}{1.662125in}}{\pgfqpoint{0.803502in}{1.670025in}}{\pgfqpoint{0.797678in}{1.675849in}}%
\pgfpathcurveto{\pgfqpoint{0.791854in}{1.681673in}}{\pgfqpoint{0.783954in}{1.684946in}}{\pgfqpoint{0.775718in}{1.684946in}}%
\pgfpathcurveto{\pgfqpoint{0.767482in}{1.684946in}}{\pgfqpoint{0.759582in}{1.681673in}}{\pgfqpoint{0.753758in}{1.675849in}}%
\pgfpathcurveto{\pgfqpoint{0.747934in}{1.670025in}}{\pgfqpoint{0.744662in}{1.662125in}}{\pgfqpoint{0.744662in}{1.653889in}}%
\pgfpathcurveto{\pgfqpoint{0.744662in}{1.645653in}}{\pgfqpoint{0.747934in}{1.637753in}}{\pgfqpoint{0.753758in}{1.631929in}}%
\pgfpathcurveto{\pgfqpoint{0.759582in}{1.626105in}}{\pgfqpoint{0.767482in}{1.622833in}}{\pgfqpoint{0.775718in}{1.622833in}}%
\pgfpathclose%
\pgfusepath{stroke,fill}%
\end{pgfscope}%
\begin{pgfscope}%
\pgfpathrectangle{\pgfqpoint{0.100000in}{0.212622in}}{\pgfqpoint{3.696000in}{3.696000in}}%
\pgfusepath{clip}%
\pgfsetbuttcap%
\pgfsetroundjoin%
\definecolor{currentfill}{rgb}{0.121569,0.466667,0.705882}%
\pgfsetfillcolor{currentfill}%
\pgfsetfillopacity{0.669449}%
\pgfsetlinewidth{1.003750pt}%
\definecolor{currentstroke}{rgb}{0.121569,0.466667,0.705882}%
\pgfsetstrokecolor{currentstroke}%
\pgfsetstrokeopacity{0.669449}%
\pgfsetdash{}{0pt}%
\pgfpathmoveto{\pgfqpoint{0.768843in}{1.616694in}}%
\pgfpathcurveto{\pgfqpoint{0.777079in}{1.616694in}}{\pgfqpoint{0.784979in}{1.619966in}}{\pgfqpoint{0.790803in}{1.625790in}}%
\pgfpathcurveto{\pgfqpoint{0.796627in}{1.631614in}}{\pgfqpoint{0.799899in}{1.639514in}}{\pgfqpoint{0.799899in}{1.647750in}}%
\pgfpathcurveto{\pgfqpoint{0.799899in}{1.655987in}}{\pgfqpoint{0.796627in}{1.663887in}}{\pgfqpoint{0.790803in}{1.669711in}}%
\pgfpathcurveto{\pgfqpoint{0.784979in}{1.675535in}}{\pgfqpoint{0.777079in}{1.678807in}}{\pgfqpoint{0.768843in}{1.678807in}}%
\pgfpathcurveto{\pgfqpoint{0.760606in}{1.678807in}}{\pgfqpoint{0.752706in}{1.675535in}}{\pgfqpoint{0.746882in}{1.669711in}}%
\pgfpathcurveto{\pgfqpoint{0.741059in}{1.663887in}}{\pgfqpoint{0.737786in}{1.655987in}}{\pgfqpoint{0.737786in}{1.647750in}}%
\pgfpathcurveto{\pgfqpoint{0.737786in}{1.639514in}}{\pgfqpoint{0.741059in}{1.631614in}}{\pgfqpoint{0.746882in}{1.625790in}}%
\pgfpathcurveto{\pgfqpoint{0.752706in}{1.619966in}}{\pgfqpoint{0.760606in}{1.616694in}}{\pgfqpoint{0.768843in}{1.616694in}}%
\pgfpathclose%
\pgfusepath{stroke,fill}%
\end{pgfscope}%
\begin{pgfscope}%
\pgfpathrectangle{\pgfqpoint{0.100000in}{0.212622in}}{\pgfqpoint{3.696000in}{3.696000in}}%
\pgfusepath{clip}%
\pgfsetbuttcap%
\pgfsetroundjoin%
\definecolor{currentfill}{rgb}{0.121569,0.466667,0.705882}%
\pgfsetfillcolor{currentfill}%
\pgfsetfillopacity{0.669525}%
\pgfsetlinewidth{1.003750pt}%
\definecolor{currentstroke}{rgb}{0.121569,0.466667,0.705882}%
\pgfsetstrokecolor{currentstroke}%
\pgfsetstrokeopacity{0.669525}%
\pgfsetdash{}{0pt}%
\pgfpathmoveto{\pgfqpoint{0.771593in}{1.623627in}}%
\pgfpathcurveto{\pgfqpoint{0.779829in}{1.623627in}}{\pgfqpoint{0.787729in}{1.626900in}}{\pgfqpoint{0.793553in}{1.632724in}}%
\pgfpathcurveto{\pgfqpoint{0.799377in}{1.638548in}}{\pgfqpoint{0.802649in}{1.646448in}}{\pgfqpoint{0.802649in}{1.654684in}}%
\pgfpathcurveto{\pgfqpoint{0.802649in}{1.662920in}}{\pgfqpoint{0.799377in}{1.670820in}}{\pgfqpoint{0.793553in}{1.676644in}}%
\pgfpathcurveto{\pgfqpoint{0.787729in}{1.682468in}}{\pgfqpoint{0.779829in}{1.685740in}}{\pgfqpoint{0.771593in}{1.685740in}}%
\pgfpathcurveto{\pgfqpoint{0.763357in}{1.685740in}}{\pgfqpoint{0.755457in}{1.682468in}}{\pgfqpoint{0.749633in}{1.676644in}}%
\pgfpathcurveto{\pgfqpoint{0.743809in}{1.670820in}}{\pgfqpoint{0.740536in}{1.662920in}}{\pgfqpoint{0.740536in}{1.654684in}}%
\pgfpathcurveto{\pgfqpoint{0.740536in}{1.646448in}}{\pgfqpoint{0.743809in}{1.638548in}}{\pgfqpoint{0.749633in}{1.632724in}}%
\pgfpathcurveto{\pgfqpoint{0.755457in}{1.626900in}}{\pgfqpoint{0.763357in}{1.623627in}}{\pgfqpoint{0.771593in}{1.623627in}}%
\pgfpathclose%
\pgfusepath{stroke,fill}%
\end{pgfscope}%
\begin{pgfscope}%
\pgfpathrectangle{\pgfqpoint{0.100000in}{0.212622in}}{\pgfqpoint{3.696000in}{3.696000in}}%
\pgfusepath{clip}%
\pgfsetbuttcap%
\pgfsetroundjoin%
\definecolor{currentfill}{rgb}{0.121569,0.466667,0.705882}%
\pgfsetfillcolor{currentfill}%
\pgfsetfillopacity{0.669941}%
\pgfsetlinewidth{1.003750pt}%
\definecolor{currentstroke}{rgb}{0.121569,0.466667,0.705882}%
\pgfsetstrokecolor{currentstroke}%
\pgfsetstrokeopacity{0.669941}%
\pgfsetdash{}{0pt}%
\pgfpathmoveto{\pgfqpoint{3.092516in}{1.929830in}}%
\pgfpathcurveto{\pgfqpoint{3.100752in}{1.929830in}}{\pgfqpoint{3.108652in}{1.933102in}}{\pgfqpoint{3.114476in}{1.938926in}}%
\pgfpathcurveto{\pgfqpoint{3.120300in}{1.944750in}}{\pgfqpoint{3.123572in}{1.952650in}}{\pgfqpoint{3.123572in}{1.960886in}}%
\pgfpathcurveto{\pgfqpoint{3.123572in}{1.969123in}}{\pgfqpoint{3.120300in}{1.977023in}}{\pgfqpoint{3.114476in}{1.982846in}}%
\pgfpathcurveto{\pgfqpoint{3.108652in}{1.988670in}}{\pgfqpoint{3.100752in}{1.991943in}}{\pgfqpoint{3.092516in}{1.991943in}}%
\pgfpathcurveto{\pgfqpoint{3.084279in}{1.991943in}}{\pgfqpoint{3.076379in}{1.988670in}}{\pgfqpoint{3.070555in}{1.982846in}}%
\pgfpathcurveto{\pgfqpoint{3.064731in}{1.977023in}}{\pgfqpoint{3.061459in}{1.969123in}}{\pgfqpoint{3.061459in}{1.960886in}}%
\pgfpathcurveto{\pgfqpoint{3.061459in}{1.952650in}}{\pgfqpoint{3.064731in}{1.944750in}}{\pgfqpoint{3.070555in}{1.938926in}}%
\pgfpathcurveto{\pgfqpoint{3.076379in}{1.933102in}}{\pgfqpoint{3.084279in}{1.929830in}}{\pgfqpoint{3.092516in}{1.929830in}}%
\pgfpathclose%
\pgfusepath{stroke,fill}%
\end{pgfscope}%
\begin{pgfscope}%
\pgfpathrectangle{\pgfqpoint{0.100000in}{0.212622in}}{\pgfqpoint{3.696000in}{3.696000in}}%
\pgfusepath{clip}%
\pgfsetbuttcap%
\pgfsetroundjoin%
\definecolor{currentfill}{rgb}{0.121569,0.466667,0.705882}%
\pgfsetfillcolor{currentfill}%
\pgfsetfillopacity{0.671059}%
\pgfsetlinewidth{1.003750pt}%
\definecolor{currentstroke}{rgb}{0.121569,0.466667,0.705882}%
\pgfsetstrokecolor{currentstroke}%
\pgfsetstrokeopacity{0.671059}%
\pgfsetdash{}{0pt}%
\pgfpathmoveto{\pgfqpoint{0.766030in}{1.613198in}}%
\pgfpathcurveto{\pgfqpoint{0.774266in}{1.613198in}}{\pgfqpoint{0.782166in}{1.616470in}}{\pgfqpoint{0.787990in}{1.622294in}}%
\pgfpathcurveto{\pgfqpoint{0.793814in}{1.628118in}}{\pgfqpoint{0.797086in}{1.636018in}}{\pgfqpoint{0.797086in}{1.644254in}}%
\pgfpathcurveto{\pgfqpoint{0.797086in}{1.652491in}}{\pgfqpoint{0.793814in}{1.660391in}}{\pgfqpoint{0.787990in}{1.666215in}}%
\pgfpathcurveto{\pgfqpoint{0.782166in}{1.672038in}}{\pgfqpoint{0.774266in}{1.675311in}}{\pgfqpoint{0.766030in}{1.675311in}}%
\pgfpathcurveto{\pgfqpoint{0.757794in}{1.675311in}}{\pgfqpoint{0.749893in}{1.672038in}}{\pgfqpoint{0.744070in}{1.666215in}}%
\pgfpathcurveto{\pgfqpoint{0.738246in}{1.660391in}}{\pgfqpoint{0.734973in}{1.652491in}}{\pgfqpoint{0.734973in}{1.644254in}}%
\pgfpathcurveto{\pgfqpoint{0.734973in}{1.636018in}}{\pgfqpoint{0.738246in}{1.628118in}}{\pgfqpoint{0.744070in}{1.622294in}}%
\pgfpathcurveto{\pgfqpoint{0.749893in}{1.616470in}}{\pgfqpoint{0.757794in}{1.613198in}}{\pgfqpoint{0.766030in}{1.613198in}}%
\pgfpathclose%
\pgfusepath{stroke,fill}%
\end{pgfscope}%
\begin{pgfscope}%
\pgfpathrectangle{\pgfqpoint{0.100000in}{0.212622in}}{\pgfqpoint{3.696000in}{3.696000in}}%
\pgfusepath{clip}%
\pgfsetbuttcap%
\pgfsetroundjoin%
\definecolor{currentfill}{rgb}{0.121569,0.466667,0.705882}%
\pgfsetfillcolor{currentfill}%
\pgfsetfillopacity{0.672122}%
\pgfsetlinewidth{1.003750pt}%
\definecolor{currentstroke}{rgb}{0.121569,0.466667,0.705882}%
\pgfsetstrokecolor{currentstroke}%
\pgfsetstrokeopacity{0.672122}%
\pgfsetdash{}{0pt}%
\pgfpathmoveto{\pgfqpoint{0.761136in}{1.610366in}}%
\pgfpathcurveto{\pgfqpoint{0.769372in}{1.610366in}}{\pgfqpoint{0.777272in}{1.613638in}}{\pgfqpoint{0.783096in}{1.619462in}}%
\pgfpathcurveto{\pgfqpoint{0.788920in}{1.625286in}}{\pgfqpoint{0.792192in}{1.633186in}}{\pgfqpoint{0.792192in}{1.641423in}}%
\pgfpathcurveto{\pgfqpoint{0.792192in}{1.649659in}}{\pgfqpoint{0.788920in}{1.657559in}}{\pgfqpoint{0.783096in}{1.663383in}}%
\pgfpathcurveto{\pgfqpoint{0.777272in}{1.669207in}}{\pgfqpoint{0.769372in}{1.672479in}}{\pgfqpoint{0.761136in}{1.672479in}}%
\pgfpathcurveto{\pgfqpoint{0.752899in}{1.672479in}}{\pgfqpoint{0.744999in}{1.669207in}}{\pgfqpoint{0.739175in}{1.663383in}}%
\pgfpathcurveto{\pgfqpoint{0.733351in}{1.657559in}}{\pgfqpoint{0.730079in}{1.649659in}}{\pgfqpoint{0.730079in}{1.641423in}}%
\pgfpathcurveto{\pgfqpoint{0.730079in}{1.633186in}}{\pgfqpoint{0.733351in}{1.625286in}}{\pgfqpoint{0.739175in}{1.619462in}}%
\pgfpathcurveto{\pgfqpoint{0.744999in}{1.613638in}}{\pgfqpoint{0.752899in}{1.610366in}}{\pgfqpoint{0.761136in}{1.610366in}}%
\pgfpathclose%
\pgfusepath{stroke,fill}%
\end{pgfscope}%
\begin{pgfscope}%
\pgfpathrectangle{\pgfqpoint{0.100000in}{0.212622in}}{\pgfqpoint{3.696000in}{3.696000in}}%
\pgfusepath{clip}%
\pgfsetbuttcap%
\pgfsetroundjoin%
\definecolor{currentfill}{rgb}{0.121569,0.466667,0.705882}%
\pgfsetfillcolor{currentfill}%
\pgfsetfillopacity{0.672468}%
\pgfsetlinewidth{1.003750pt}%
\definecolor{currentstroke}{rgb}{0.121569,0.466667,0.705882}%
\pgfsetstrokecolor{currentstroke}%
\pgfsetstrokeopacity{0.672468}%
\pgfsetdash{}{0pt}%
\pgfpathmoveto{\pgfqpoint{3.085780in}{1.930989in}}%
\pgfpathcurveto{\pgfqpoint{3.094017in}{1.930989in}}{\pgfqpoint{3.101917in}{1.934261in}}{\pgfqpoint{3.107741in}{1.940085in}}%
\pgfpathcurveto{\pgfqpoint{3.113565in}{1.945909in}}{\pgfqpoint{3.116837in}{1.953809in}}{\pgfqpoint{3.116837in}{1.962045in}}%
\pgfpathcurveto{\pgfqpoint{3.116837in}{1.970282in}}{\pgfqpoint{3.113565in}{1.978182in}}{\pgfqpoint{3.107741in}{1.984006in}}%
\pgfpathcurveto{\pgfqpoint{3.101917in}{1.989829in}}{\pgfqpoint{3.094017in}{1.993102in}}{\pgfqpoint{3.085780in}{1.993102in}}%
\pgfpathcurveto{\pgfqpoint{3.077544in}{1.993102in}}{\pgfqpoint{3.069644in}{1.989829in}}{\pgfqpoint{3.063820in}{1.984006in}}%
\pgfpathcurveto{\pgfqpoint{3.057996in}{1.978182in}}{\pgfqpoint{3.054724in}{1.970282in}}{\pgfqpoint{3.054724in}{1.962045in}}%
\pgfpathcurveto{\pgfqpoint{3.054724in}{1.953809in}}{\pgfqpoint{3.057996in}{1.945909in}}{\pgfqpoint{3.063820in}{1.940085in}}%
\pgfpathcurveto{\pgfqpoint{3.069644in}{1.934261in}}{\pgfqpoint{3.077544in}{1.930989in}}{\pgfqpoint{3.085780in}{1.930989in}}%
\pgfpathclose%
\pgfusepath{stroke,fill}%
\end{pgfscope}%
\begin{pgfscope}%
\pgfpathrectangle{\pgfqpoint{0.100000in}{0.212622in}}{\pgfqpoint{3.696000in}{3.696000in}}%
\pgfusepath{clip}%
\pgfsetbuttcap%
\pgfsetroundjoin%
\definecolor{currentfill}{rgb}{0.121569,0.466667,0.705882}%
\pgfsetfillcolor{currentfill}%
\pgfsetfillopacity{0.673577}%
\pgfsetlinewidth{1.003750pt}%
\definecolor{currentstroke}{rgb}{0.121569,0.466667,0.705882}%
\pgfsetstrokecolor{currentstroke}%
\pgfsetstrokeopacity{0.673577}%
\pgfsetdash{}{0pt}%
\pgfpathmoveto{\pgfqpoint{0.759001in}{1.610705in}}%
\pgfpathcurveto{\pgfqpoint{0.767237in}{1.610705in}}{\pgfqpoint{0.775138in}{1.613978in}}{\pgfqpoint{0.780961in}{1.619802in}}%
\pgfpathcurveto{\pgfqpoint{0.786785in}{1.625625in}}{\pgfqpoint{0.790058in}{1.633526in}}{\pgfqpoint{0.790058in}{1.641762in}}%
\pgfpathcurveto{\pgfqpoint{0.790058in}{1.649998in}}{\pgfqpoint{0.786785in}{1.657898in}}{\pgfqpoint{0.780961in}{1.663722in}}%
\pgfpathcurveto{\pgfqpoint{0.775138in}{1.669546in}}{\pgfqpoint{0.767237in}{1.672818in}}{\pgfqpoint{0.759001in}{1.672818in}}%
\pgfpathcurveto{\pgfqpoint{0.750765in}{1.672818in}}{\pgfqpoint{0.742865in}{1.669546in}}{\pgfqpoint{0.737041in}{1.663722in}}%
\pgfpathcurveto{\pgfqpoint{0.731217in}{1.657898in}}{\pgfqpoint{0.727945in}{1.649998in}}{\pgfqpoint{0.727945in}{1.641762in}}%
\pgfpathcurveto{\pgfqpoint{0.727945in}{1.633526in}}{\pgfqpoint{0.731217in}{1.625625in}}{\pgfqpoint{0.737041in}{1.619802in}}%
\pgfpathcurveto{\pgfqpoint{0.742865in}{1.613978in}}{\pgfqpoint{0.750765in}{1.610705in}}{\pgfqpoint{0.759001in}{1.610705in}}%
\pgfpathclose%
\pgfusepath{stroke,fill}%
\end{pgfscope}%
\begin{pgfscope}%
\pgfpathrectangle{\pgfqpoint{0.100000in}{0.212622in}}{\pgfqpoint{3.696000in}{3.696000in}}%
\pgfusepath{clip}%
\pgfsetbuttcap%
\pgfsetroundjoin%
\definecolor{currentfill}{rgb}{0.121569,0.466667,0.705882}%
\pgfsetfillcolor{currentfill}%
\pgfsetfillopacity{0.674207}%
\pgfsetlinewidth{1.003750pt}%
\definecolor{currentstroke}{rgb}{0.121569,0.466667,0.705882}%
\pgfsetstrokecolor{currentstroke}%
\pgfsetstrokeopacity{0.674207}%
\pgfsetdash{}{0pt}%
\pgfpathmoveto{\pgfqpoint{0.755959in}{1.611050in}}%
\pgfpathcurveto{\pgfqpoint{0.764196in}{1.611050in}}{\pgfqpoint{0.772096in}{1.614322in}}{\pgfqpoint{0.777920in}{1.620146in}}%
\pgfpathcurveto{\pgfqpoint{0.783744in}{1.625970in}}{\pgfqpoint{0.787016in}{1.633870in}}{\pgfqpoint{0.787016in}{1.642106in}}%
\pgfpathcurveto{\pgfqpoint{0.787016in}{1.650342in}}{\pgfqpoint{0.783744in}{1.658242in}}{\pgfqpoint{0.777920in}{1.664066in}}%
\pgfpathcurveto{\pgfqpoint{0.772096in}{1.669890in}}{\pgfqpoint{0.764196in}{1.673163in}}{\pgfqpoint{0.755959in}{1.673163in}}%
\pgfpathcurveto{\pgfqpoint{0.747723in}{1.673163in}}{\pgfqpoint{0.739823in}{1.669890in}}{\pgfqpoint{0.733999in}{1.664066in}}%
\pgfpathcurveto{\pgfqpoint{0.728175in}{1.658242in}}{\pgfqpoint{0.724903in}{1.650342in}}{\pgfqpoint{0.724903in}{1.642106in}}%
\pgfpathcurveto{\pgfqpoint{0.724903in}{1.633870in}}{\pgfqpoint{0.728175in}{1.625970in}}{\pgfqpoint{0.733999in}{1.620146in}}%
\pgfpathcurveto{\pgfqpoint{0.739823in}{1.614322in}}{\pgfqpoint{0.747723in}{1.611050in}}{\pgfqpoint{0.755959in}{1.611050in}}%
\pgfpathclose%
\pgfusepath{stroke,fill}%
\end{pgfscope}%
\begin{pgfscope}%
\pgfpathrectangle{\pgfqpoint{0.100000in}{0.212622in}}{\pgfqpoint{3.696000in}{3.696000in}}%
\pgfusepath{clip}%
\pgfsetbuttcap%
\pgfsetroundjoin%
\definecolor{currentfill}{rgb}{0.121569,0.466667,0.705882}%
\pgfsetfillcolor{currentfill}%
\pgfsetfillopacity{0.674407}%
\pgfsetlinewidth{1.003750pt}%
\definecolor{currentstroke}{rgb}{0.121569,0.466667,0.705882}%
\pgfsetstrokecolor{currentstroke}%
\pgfsetstrokeopacity{0.674407}%
\pgfsetdash{}{0pt}%
\pgfpathmoveto{\pgfqpoint{3.080654in}{1.924516in}}%
\pgfpathcurveto{\pgfqpoint{3.088890in}{1.924516in}}{\pgfqpoint{3.096790in}{1.927788in}}{\pgfqpoint{3.102614in}{1.933612in}}%
\pgfpathcurveto{\pgfqpoint{3.108438in}{1.939436in}}{\pgfqpoint{3.111710in}{1.947336in}}{\pgfqpoint{3.111710in}{1.955572in}}%
\pgfpathcurveto{\pgfqpoint{3.111710in}{1.963808in}}{\pgfqpoint{3.108438in}{1.971709in}}{\pgfqpoint{3.102614in}{1.977532in}}%
\pgfpathcurveto{\pgfqpoint{3.096790in}{1.983356in}}{\pgfqpoint{3.088890in}{1.986629in}}{\pgfqpoint{3.080654in}{1.986629in}}%
\pgfpathcurveto{\pgfqpoint{3.072417in}{1.986629in}}{\pgfqpoint{3.064517in}{1.983356in}}{\pgfqpoint{3.058693in}{1.977532in}}%
\pgfpathcurveto{\pgfqpoint{3.052869in}{1.971709in}}{\pgfqpoint{3.049597in}{1.963808in}}{\pgfqpoint{3.049597in}{1.955572in}}%
\pgfpathcurveto{\pgfqpoint{3.049597in}{1.947336in}}{\pgfqpoint{3.052869in}{1.939436in}}{\pgfqpoint{3.058693in}{1.933612in}}%
\pgfpathcurveto{\pgfqpoint{3.064517in}{1.927788in}}{\pgfqpoint{3.072417in}{1.924516in}}{\pgfqpoint{3.080654in}{1.924516in}}%
\pgfpathclose%
\pgfusepath{stroke,fill}%
\end{pgfscope}%
\begin{pgfscope}%
\pgfpathrectangle{\pgfqpoint{0.100000in}{0.212622in}}{\pgfqpoint{3.696000in}{3.696000in}}%
\pgfusepath{clip}%
\pgfsetbuttcap%
\pgfsetroundjoin%
\definecolor{currentfill}{rgb}{0.121569,0.466667,0.705882}%
\pgfsetfillcolor{currentfill}%
\pgfsetfillopacity{0.674469}%
\pgfsetlinewidth{1.003750pt}%
\definecolor{currentstroke}{rgb}{0.121569,0.466667,0.705882}%
\pgfsetstrokecolor{currentstroke}%
\pgfsetstrokeopacity{0.674469}%
\pgfsetdash{}{0pt}%
\pgfpathmoveto{\pgfqpoint{0.754252in}{1.608357in}}%
\pgfpathcurveto{\pgfqpoint{0.762488in}{1.608357in}}{\pgfqpoint{0.770388in}{1.611629in}}{\pgfqpoint{0.776212in}{1.617453in}}%
\pgfpathcurveto{\pgfqpoint{0.782036in}{1.623277in}}{\pgfqpoint{0.785308in}{1.631177in}}{\pgfqpoint{0.785308in}{1.639414in}}%
\pgfpathcurveto{\pgfqpoint{0.785308in}{1.647650in}}{\pgfqpoint{0.782036in}{1.655550in}}{\pgfqpoint{0.776212in}{1.661374in}}%
\pgfpathcurveto{\pgfqpoint{0.770388in}{1.667198in}}{\pgfqpoint{0.762488in}{1.670470in}}{\pgfqpoint{0.754252in}{1.670470in}}%
\pgfpathcurveto{\pgfqpoint{0.746016in}{1.670470in}}{\pgfqpoint{0.738116in}{1.667198in}}{\pgfqpoint{0.732292in}{1.661374in}}%
\pgfpathcurveto{\pgfqpoint{0.726468in}{1.655550in}}{\pgfqpoint{0.723195in}{1.647650in}}{\pgfqpoint{0.723195in}{1.639414in}}%
\pgfpathcurveto{\pgfqpoint{0.723195in}{1.631177in}}{\pgfqpoint{0.726468in}{1.623277in}}{\pgfqpoint{0.732292in}{1.617453in}}%
\pgfpathcurveto{\pgfqpoint{0.738116in}{1.611629in}}{\pgfqpoint{0.746016in}{1.608357in}}{\pgfqpoint{0.754252in}{1.608357in}}%
\pgfpathclose%
\pgfusepath{stroke,fill}%
\end{pgfscope}%
\begin{pgfscope}%
\pgfpathrectangle{\pgfqpoint{0.100000in}{0.212622in}}{\pgfqpoint{3.696000in}{3.696000in}}%
\pgfusepath{clip}%
\pgfsetbuttcap%
\pgfsetroundjoin%
\definecolor{currentfill}{rgb}{0.121569,0.466667,0.705882}%
\pgfsetfillcolor{currentfill}%
\pgfsetfillopacity{0.675550}%
\pgfsetlinewidth{1.003750pt}%
\definecolor{currentstroke}{rgb}{0.121569,0.466667,0.705882}%
\pgfsetstrokecolor{currentstroke}%
\pgfsetstrokeopacity{0.675550}%
\pgfsetdash{}{0pt}%
\pgfpathmoveto{\pgfqpoint{0.752315in}{1.606252in}}%
\pgfpathcurveto{\pgfqpoint{0.760551in}{1.606252in}}{\pgfqpoint{0.768451in}{1.609525in}}{\pgfqpoint{0.774275in}{1.615348in}}%
\pgfpathcurveto{\pgfqpoint{0.780099in}{1.621172in}}{\pgfqpoint{0.783371in}{1.629072in}}{\pgfqpoint{0.783371in}{1.637309in}}%
\pgfpathcurveto{\pgfqpoint{0.783371in}{1.645545in}}{\pgfqpoint{0.780099in}{1.653445in}}{\pgfqpoint{0.774275in}{1.659269in}}%
\pgfpathcurveto{\pgfqpoint{0.768451in}{1.665093in}}{\pgfqpoint{0.760551in}{1.668365in}}{\pgfqpoint{0.752315in}{1.668365in}}%
\pgfpathcurveto{\pgfqpoint{0.744078in}{1.668365in}}{\pgfqpoint{0.736178in}{1.665093in}}{\pgfqpoint{0.730354in}{1.659269in}}%
\pgfpathcurveto{\pgfqpoint{0.724530in}{1.653445in}}{\pgfqpoint{0.721258in}{1.645545in}}{\pgfqpoint{0.721258in}{1.637309in}}%
\pgfpathcurveto{\pgfqpoint{0.721258in}{1.629072in}}{\pgfqpoint{0.724530in}{1.621172in}}{\pgfqpoint{0.730354in}{1.615348in}}%
\pgfpathcurveto{\pgfqpoint{0.736178in}{1.609525in}}{\pgfqpoint{0.744078in}{1.606252in}}{\pgfqpoint{0.752315in}{1.606252in}}%
\pgfpathclose%
\pgfusepath{stroke,fill}%
\end{pgfscope}%
\begin{pgfscope}%
\pgfpathrectangle{\pgfqpoint{0.100000in}{0.212622in}}{\pgfqpoint{3.696000in}{3.696000in}}%
\pgfusepath{clip}%
\pgfsetbuttcap%
\pgfsetroundjoin%
\definecolor{currentfill}{rgb}{0.121569,0.466667,0.705882}%
\pgfsetfillcolor{currentfill}%
\pgfsetfillopacity{0.676209}%
\pgfsetlinewidth{1.003750pt}%
\definecolor{currentstroke}{rgb}{0.121569,0.466667,0.705882}%
\pgfsetstrokecolor{currentstroke}%
\pgfsetstrokeopacity{0.676209}%
\pgfsetdash{}{0pt}%
\pgfpathmoveto{\pgfqpoint{0.749260in}{1.604141in}}%
\pgfpathcurveto{\pgfqpoint{0.757496in}{1.604141in}}{\pgfqpoint{0.765396in}{1.607413in}}{\pgfqpoint{0.771220in}{1.613237in}}%
\pgfpathcurveto{\pgfqpoint{0.777044in}{1.619061in}}{\pgfqpoint{0.780316in}{1.626961in}}{\pgfqpoint{0.780316in}{1.635197in}}%
\pgfpathcurveto{\pgfqpoint{0.780316in}{1.643434in}}{\pgfqpoint{0.777044in}{1.651334in}}{\pgfqpoint{0.771220in}{1.657158in}}%
\pgfpathcurveto{\pgfqpoint{0.765396in}{1.662981in}}{\pgfqpoint{0.757496in}{1.666254in}}{\pgfqpoint{0.749260in}{1.666254in}}%
\pgfpathcurveto{\pgfqpoint{0.741023in}{1.666254in}}{\pgfqpoint{0.733123in}{1.662981in}}{\pgfqpoint{0.727299in}{1.657158in}}%
\pgfpathcurveto{\pgfqpoint{0.721475in}{1.651334in}}{\pgfqpoint{0.718203in}{1.643434in}}{\pgfqpoint{0.718203in}{1.635197in}}%
\pgfpathcurveto{\pgfqpoint{0.718203in}{1.626961in}}{\pgfqpoint{0.721475in}{1.619061in}}{\pgfqpoint{0.727299in}{1.613237in}}%
\pgfpathcurveto{\pgfqpoint{0.733123in}{1.607413in}}{\pgfqpoint{0.741023in}{1.604141in}}{\pgfqpoint{0.749260in}{1.604141in}}%
\pgfpathclose%
\pgfusepath{stroke,fill}%
\end{pgfscope}%
\begin{pgfscope}%
\pgfpathrectangle{\pgfqpoint{0.100000in}{0.212622in}}{\pgfqpoint{3.696000in}{3.696000in}}%
\pgfusepath{clip}%
\pgfsetbuttcap%
\pgfsetroundjoin%
\definecolor{currentfill}{rgb}{0.121569,0.466667,0.705882}%
\pgfsetfillcolor{currentfill}%
\pgfsetfillopacity{0.676824}%
\pgfsetlinewidth{1.003750pt}%
\definecolor{currentstroke}{rgb}{0.121569,0.466667,0.705882}%
\pgfsetstrokecolor{currentstroke}%
\pgfsetstrokeopacity{0.676824}%
\pgfsetdash{}{0pt}%
\pgfpathmoveto{\pgfqpoint{0.748335in}{1.603260in}}%
\pgfpathcurveto{\pgfqpoint{0.756571in}{1.603260in}}{\pgfqpoint{0.764472in}{1.606533in}}{\pgfqpoint{0.770295in}{1.612357in}}%
\pgfpathcurveto{\pgfqpoint{0.776119in}{1.618181in}}{\pgfqpoint{0.779392in}{1.626081in}}{\pgfqpoint{0.779392in}{1.634317in}}%
\pgfpathcurveto{\pgfqpoint{0.779392in}{1.642553in}}{\pgfqpoint{0.776119in}{1.650453in}}{\pgfqpoint{0.770295in}{1.656277in}}%
\pgfpathcurveto{\pgfqpoint{0.764472in}{1.662101in}}{\pgfqpoint{0.756571in}{1.665373in}}{\pgfqpoint{0.748335in}{1.665373in}}%
\pgfpathcurveto{\pgfqpoint{0.740099in}{1.665373in}}{\pgfqpoint{0.732199in}{1.662101in}}{\pgfqpoint{0.726375in}{1.656277in}}%
\pgfpathcurveto{\pgfqpoint{0.720551in}{1.650453in}}{\pgfqpoint{0.717279in}{1.642553in}}{\pgfqpoint{0.717279in}{1.634317in}}%
\pgfpathcurveto{\pgfqpoint{0.717279in}{1.626081in}}{\pgfqpoint{0.720551in}{1.618181in}}{\pgfqpoint{0.726375in}{1.612357in}}%
\pgfpathcurveto{\pgfqpoint{0.732199in}{1.606533in}}{\pgfqpoint{0.740099in}{1.603260in}}{\pgfqpoint{0.748335in}{1.603260in}}%
\pgfpathclose%
\pgfusepath{stroke,fill}%
\end{pgfscope}%
\begin{pgfscope}%
\pgfpathrectangle{\pgfqpoint{0.100000in}{0.212622in}}{\pgfqpoint{3.696000in}{3.696000in}}%
\pgfusepath{clip}%
\pgfsetbuttcap%
\pgfsetroundjoin%
\definecolor{currentfill}{rgb}{0.121569,0.466667,0.705882}%
\pgfsetfillcolor{currentfill}%
\pgfsetfillopacity{0.677073}%
\pgfsetlinewidth{1.003750pt}%
\definecolor{currentstroke}{rgb}{0.121569,0.466667,0.705882}%
\pgfsetstrokecolor{currentstroke}%
\pgfsetstrokeopacity{0.677073}%
\pgfsetdash{}{0pt}%
\pgfpathmoveto{\pgfqpoint{3.076568in}{1.920327in}}%
\pgfpathcurveto{\pgfqpoint{3.084804in}{1.920327in}}{\pgfqpoint{3.092704in}{1.923599in}}{\pgfqpoint{3.098528in}{1.929423in}}%
\pgfpathcurveto{\pgfqpoint{3.104352in}{1.935247in}}{\pgfqpoint{3.107625in}{1.943147in}}{\pgfqpoint{3.107625in}{1.951383in}}%
\pgfpathcurveto{\pgfqpoint{3.107625in}{1.959619in}}{\pgfqpoint{3.104352in}{1.967520in}}{\pgfqpoint{3.098528in}{1.973343in}}%
\pgfpathcurveto{\pgfqpoint{3.092704in}{1.979167in}}{\pgfqpoint{3.084804in}{1.982440in}}{\pgfqpoint{3.076568in}{1.982440in}}%
\pgfpathcurveto{\pgfqpoint{3.068332in}{1.982440in}}{\pgfqpoint{3.060432in}{1.979167in}}{\pgfqpoint{3.054608in}{1.973343in}}%
\pgfpathcurveto{\pgfqpoint{3.048784in}{1.967520in}}{\pgfqpoint{3.045512in}{1.959619in}}{\pgfqpoint{3.045512in}{1.951383in}}%
\pgfpathcurveto{\pgfqpoint{3.045512in}{1.943147in}}{\pgfqpoint{3.048784in}{1.935247in}}{\pgfqpoint{3.054608in}{1.929423in}}%
\pgfpathcurveto{\pgfqpoint{3.060432in}{1.923599in}}{\pgfqpoint{3.068332in}{1.920327in}}{\pgfqpoint{3.076568in}{1.920327in}}%
\pgfpathclose%
\pgfusepath{stroke,fill}%
\end{pgfscope}%
\begin{pgfscope}%
\pgfpathrectangle{\pgfqpoint{0.100000in}{0.212622in}}{\pgfqpoint{3.696000in}{3.696000in}}%
\pgfusepath{clip}%
\pgfsetbuttcap%
\pgfsetroundjoin%
\definecolor{currentfill}{rgb}{0.121569,0.466667,0.705882}%
\pgfsetfillcolor{currentfill}%
\pgfsetfillopacity{0.677860}%
\pgfsetlinewidth{1.003750pt}%
\definecolor{currentstroke}{rgb}{0.121569,0.466667,0.705882}%
\pgfsetstrokecolor{currentstroke}%
\pgfsetstrokeopacity{0.677860}%
\pgfsetdash{}{0pt}%
\pgfpathmoveto{\pgfqpoint{0.744734in}{1.603316in}}%
\pgfpathcurveto{\pgfqpoint{0.752970in}{1.603316in}}{\pgfqpoint{0.760870in}{1.606589in}}{\pgfqpoint{0.766694in}{1.612413in}}%
\pgfpathcurveto{\pgfqpoint{0.772518in}{1.618237in}}{\pgfqpoint{0.775791in}{1.626137in}}{\pgfqpoint{0.775791in}{1.634373in}}%
\pgfpathcurveto{\pgfqpoint{0.775791in}{1.642609in}}{\pgfqpoint{0.772518in}{1.650509in}}{\pgfqpoint{0.766694in}{1.656333in}}%
\pgfpathcurveto{\pgfqpoint{0.760870in}{1.662157in}}{\pgfqpoint{0.752970in}{1.665429in}}{\pgfqpoint{0.744734in}{1.665429in}}%
\pgfpathcurveto{\pgfqpoint{0.736498in}{1.665429in}}{\pgfqpoint{0.728598in}{1.662157in}}{\pgfqpoint{0.722774in}{1.656333in}}%
\pgfpathcurveto{\pgfqpoint{0.716950in}{1.650509in}}{\pgfqpoint{0.713678in}{1.642609in}}{\pgfqpoint{0.713678in}{1.634373in}}%
\pgfpathcurveto{\pgfqpoint{0.713678in}{1.626137in}}{\pgfqpoint{0.716950in}{1.618237in}}{\pgfqpoint{0.722774in}{1.612413in}}%
\pgfpathcurveto{\pgfqpoint{0.728598in}{1.606589in}}{\pgfqpoint{0.736498in}{1.603316in}}{\pgfqpoint{0.744734in}{1.603316in}}%
\pgfpathclose%
\pgfusepath{stroke,fill}%
\end{pgfscope}%
\begin{pgfscope}%
\pgfpathrectangle{\pgfqpoint{0.100000in}{0.212622in}}{\pgfqpoint{3.696000in}{3.696000in}}%
\pgfusepath{clip}%
\pgfsetbuttcap%
\pgfsetroundjoin%
\definecolor{currentfill}{rgb}{0.121569,0.466667,0.705882}%
\pgfsetfillcolor{currentfill}%
\pgfsetfillopacity{0.679875}%
\pgfsetlinewidth{1.003750pt}%
\definecolor{currentstroke}{rgb}{0.121569,0.466667,0.705882}%
\pgfsetstrokecolor{currentstroke}%
\pgfsetstrokeopacity{0.679875}%
\pgfsetdash{}{0pt}%
\pgfpathmoveto{\pgfqpoint{3.071836in}{1.916223in}}%
\pgfpathcurveto{\pgfqpoint{3.080072in}{1.916223in}}{\pgfqpoint{3.087972in}{1.919495in}}{\pgfqpoint{3.093796in}{1.925319in}}%
\pgfpathcurveto{\pgfqpoint{3.099620in}{1.931143in}}{\pgfqpoint{3.102892in}{1.939043in}}{\pgfqpoint{3.102892in}{1.947280in}}%
\pgfpathcurveto{\pgfqpoint{3.102892in}{1.955516in}}{\pgfqpoint{3.099620in}{1.963416in}}{\pgfqpoint{3.093796in}{1.969240in}}%
\pgfpathcurveto{\pgfqpoint{3.087972in}{1.975064in}}{\pgfqpoint{3.080072in}{1.978336in}}{\pgfqpoint{3.071836in}{1.978336in}}%
\pgfpathcurveto{\pgfqpoint{3.063600in}{1.978336in}}{\pgfqpoint{3.055700in}{1.975064in}}{\pgfqpoint{3.049876in}{1.969240in}}%
\pgfpathcurveto{\pgfqpoint{3.044052in}{1.963416in}}{\pgfqpoint{3.040779in}{1.955516in}}{\pgfqpoint{3.040779in}{1.947280in}}%
\pgfpathcurveto{\pgfqpoint{3.040779in}{1.939043in}}{\pgfqpoint{3.044052in}{1.931143in}}{\pgfqpoint{3.049876in}{1.925319in}}%
\pgfpathcurveto{\pgfqpoint{3.055700in}{1.919495in}}{\pgfqpoint{3.063600in}{1.916223in}}{\pgfqpoint{3.071836in}{1.916223in}}%
\pgfpathclose%
\pgfusepath{stroke,fill}%
\end{pgfscope}%
\begin{pgfscope}%
\pgfpathrectangle{\pgfqpoint{0.100000in}{0.212622in}}{\pgfqpoint{3.696000in}{3.696000in}}%
\pgfusepath{clip}%
\pgfsetbuttcap%
\pgfsetroundjoin%
\definecolor{currentfill}{rgb}{0.121569,0.466667,0.705882}%
\pgfsetfillcolor{currentfill}%
\pgfsetfillopacity{0.680699}%
\pgfsetlinewidth{1.003750pt}%
\definecolor{currentstroke}{rgb}{0.121569,0.466667,0.705882}%
\pgfsetstrokecolor{currentstroke}%
\pgfsetstrokeopacity{0.680699}%
\pgfsetdash{}{0pt}%
\pgfpathmoveto{\pgfqpoint{0.742055in}{1.605642in}}%
\pgfpathcurveto{\pgfqpoint{0.750291in}{1.605642in}}{\pgfqpoint{0.758191in}{1.608914in}}{\pgfqpoint{0.764015in}{1.614738in}}%
\pgfpathcurveto{\pgfqpoint{0.769839in}{1.620562in}}{\pgfqpoint{0.773111in}{1.628462in}}{\pgfqpoint{0.773111in}{1.636699in}}%
\pgfpathcurveto{\pgfqpoint{0.773111in}{1.644935in}}{\pgfqpoint{0.769839in}{1.652835in}}{\pgfqpoint{0.764015in}{1.658659in}}%
\pgfpathcurveto{\pgfqpoint{0.758191in}{1.664483in}}{\pgfqpoint{0.750291in}{1.667755in}}{\pgfqpoint{0.742055in}{1.667755in}}%
\pgfpathcurveto{\pgfqpoint{0.733819in}{1.667755in}}{\pgfqpoint{0.725919in}{1.664483in}}{\pgfqpoint{0.720095in}{1.658659in}}%
\pgfpathcurveto{\pgfqpoint{0.714271in}{1.652835in}}{\pgfqpoint{0.710998in}{1.644935in}}{\pgfqpoint{0.710998in}{1.636699in}}%
\pgfpathcurveto{\pgfqpoint{0.710998in}{1.628462in}}{\pgfqpoint{0.714271in}{1.620562in}}{\pgfqpoint{0.720095in}{1.614738in}}%
\pgfpathcurveto{\pgfqpoint{0.725919in}{1.608914in}}{\pgfqpoint{0.733819in}{1.605642in}}{\pgfqpoint{0.742055in}{1.605642in}}%
\pgfpathclose%
\pgfusepath{stroke,fill}%
\end{pgfscope}%
\begin{pgfscope}%
\pgfpathrectangle{\pgfqpoint{0.100000in}{0.212622in}}{\pgfqpoint{3.696000in}{3.696000in}}%
\pgfusepath{clip}%
\pgfsetbuttcap%
\pgfsetroundjoin%
\definecolor{currentfill}{rgb}{0.121569,0.466667,0.705882}%
\pgfsetfillcolor{currentfill}%
\pgfsetfillopacity{0.682407}%
\pgfsetlinewidth{1.003750pt}%
\definecolor{currentstroke}{rgb}{0.121569,0.466667,0.705882}%
\pgfsetstrokecolor{currentstroke}%
\pgfsetstrokeopacity{0.682407}%
\pgfsetdash{}{0pt}%
\pgfpathmoveto{\pgfqpoint{0.736167in}{1.605711in}}%
\pgfpathcurveto{\pgfqpoint{0.744404in}{1.605711in}}{\pgfqpoint{0.752304in}{1.608983in}}{\pgfqpoint{0.758128in}{1.614807in}}%
\pgfpathcurveto{\pgfqpoint{0.763951in}{1.620631in}}{\pgfqpoint{0.767224in}{1.628531in}}{\pgfqpoint{0.767224in}{1.636768in}}%
\pgfpathcurveto{\pgfqpoint{0.767224in}{1.645004in}}{\pgfqpoint{0.763951in}{1.652904in}}{\pgfqpoint{0.758128in}{1.658728in}}%
\pgfpathcurveto{\pgfqpoint{0.752304in}{1.664552in}}{\pgfqpoint{0.744404in}{1.667824in}}{\pgfqpoint{0.736167in}{1.667824in}}%
\pgfpathcurveto{\pgfqpoint{0.727931in}{1.667824in}}{\pgfqpoint{0.720031in}{1.664552in}}{\pgfqpoint{0.714207in}{1.658728in}}%
\pgfpathcurveto{\pgfqpoint{0.708383in}{1.652904in}}{\pgfqpoint{0.705111in}{1.645004in}}{\pgfqpoint{0.705111in}{1.636768in}}%
\pgfpathcurveto{\pgfqpoint{0.705111in}{1.628531in}}{\pgfqpoint{0.708383in}{1.620631in}}{\pgfqpoint{0.714207in}{1.614807in}}%
\pgfpathcurveto{\pgfqpoint{0.720031in}{1.608983in}}{\pgfqpoint{0.727931in}{1.605711in}}{\pgfqpoint{0.736167in}{1.605711in}}%
\pgfpathclose%
\pgfusepath{stroke,fill}%
\end{pgfscope}%
\begin{pgfscope}%
\pgfpathrectangle{\pgfqpoint{0.100000in}{0.212622in}}{\pgfqpoint{3.696000in}{3.696000in}}%
\pgfusepath{clip}%
\pgfsetbuttcap%
\pgfsetroundjoin%
\definecolor{currentfill}{rgb}{0.121569,0.466667,0.705882}%
\pgfsetfillcolor{currentfill}%
\pgfsetfillopacity{0.683587}%
\pgfsetlinewidth{1.003750pt}%
\definecolor{currentstroke}{rgb}{0.121569,0.466667,0.705882}%
\pgfsetstrokecolor{currentstroke}%
\pgfsetstrokeopacity{0.683587}%
\pgfsetdash{}{0pt}%
\pgfpathmoveto{\pgfqpoint{3.062208in}{1.924991in}}%
\pgfpathcurveto{\pgfqpoint{3.070444in}{1.924991in}}{\pgfqpoint{3.078344in}{1.928263in}}{\pgfqpoint{3.084168in}{1.934087in}}%
\pgfpathcurveto{\pgfqpoint{3.089992in}{1.939911in}}{\pgfqpoint{3.093264in}{1.947811in}}{\pgfqpoint{3.093264in}{1.956048in}}%
\pgfpathcurveto{\pgfqpoint{3.093264in}{1.964284in}}{\pgfqpoint{3.089992in}{1.972184in}}{\pgfqpoint{3.084168in}{1.978008in}}%
\pgfpathcurveto{\pgfqpoint{3.078344in}{1.983832in}}{\pgfqpoint{3.070444in}{1.987104in}}{\pgfqpoint{3.062208in}{1.987104in}}%
\pgfpathcurveto{\pgfqpoint{3.053972in}{1.987104in}}{\pgfqpoint{3.046072in}{1.983832in}}{\pgfqpoint{3.040248in}{1.978008in}}%
\pgfpathcurveto{\pgfqpoint{3.034424in}{1.972184in}}{\pgfqpoint{3.031151in}{1.964284in}}{\pgfqpoint{3.031151in}{1.956048in}}%
\pgfpathcurveto{\pgfqpoint{3.031151in}{1.947811in}}{\pgfqpoint{3.034424in}{1.939911in}}{\pgfqpoint{3.040248in}{1.934087in}}%
\pgfpathcurveto{\pgfqpoint{3.046072in}{1.928263in}}{\pgfqpoint{3.053972in}{1.924991in}}{\pgfqpoint{3.062208in}{1.924991in}}%
\pgfpathclose%
\pgfusepath{stroke,fill}%
\end{pgfscope}%
\begin{pgfscope}%
\pgfpathrectangle{\pgfqpoint{0.100000in}{0.212622in}}{\pgfqpoint{3.696000in}{3.696000in}}%
\pgfusepath{clip}%
\pgfsetbuttcap%
\pgfsetroundjoin%
\definecolor{currentfill}{rgb}{0.121569,0.466667,0.705882}%
\pgfsetfillcolor{currentfill}%
\pgfsetfillopacity{0.684425}%
\pgfsetlinewidth{1.003750pt}%
\definecolor{currentstroke}{rgb}{0.121569,0.466667,0.705882}%
\pgfsetstrokecolor{currentstroke}%
\pgfsetstrokeopacity{0.684425}%
\pgfsetdash{}{0pt}%
\pgfpathmoveto{\pgfqpoint{0.734010in}{1.606651in}}%
\pgfpathcurveto{\pgfqpoint{0.742247in}{1.606651in}}{\pgfqpoint{0.750147in}{1.609924in}}{\pgfqpoint{0.755970in}{1.615748in}}%
\pgfpathcurveto{\pgfqpoint{0.761794in}{1.621572in}}{\pgfqpoint{0.765067in}{1.629472in}}{\pgfqpoint{0.765067in}{1.637708in}}%
\pgfpathcurveto{\pgfqpoint{0.765067in}{1.645944in}}{\pgfqpoint{0.761794in}{1.653844in}}{\pgfqpoint{0.755970in}{1.659668in}}%
\pgfpathcurveto{\pgfqpoint{0.750147in}{1.665492in}}{\pgfqpoint{0.742247in}{1.668764in}}{\pgfqpoint{0.734010in}{1.668764in}}%
\pgfpathcurveto{\pgfqpoint{0.725774in}{1.668764in}}{\pgfqpoint{0.717874in}{1.665492in}}{\pgfqpoint{0.712050in}{1.659668in}}%
\pgfpathcurveto{\pgfqpoint{0.706226in}{1.653844in}}{\pgfqpoint{0.702954in}{1.645944in}}{\pgfqpoint{0.702954in}{1.637708in}}%
\pgfpathcurveto{\pgfqpoint{0.702954in}{1.629472in}}{\pgfqpoint{0.706226in}{1.621572in}}{\pgfqpoint{0.712050in}{1.615748in}}%
\pgfpathcurveto{\pgfqpoint{0.717874in}{1.609924in}}{\pgfqpoint{0.725774in}{1.606651in}}{\pgfqpoint{0.734010in}{1.606651in}}%
\pgfpathclose%
\pgfusepath{stroke,fill}%
\end{pgfscope}%
\begin{pgfscope}%
\pgfpathrectangle{\pgfqpoint{0.100000in}{0.212622in}}{\pgfqpoint{3.696000in}{3.696000in}}%
\pgfusepath{clip}%
\pgfsetbuttcap%
\pgfsetroundjoin%
\definecolor{currentfill}{rgb}{0.121569,0.466667,0.705882}%
\pgfsetfillcolor{currentfill}%
\pgfsetfillopacity{0.686294}%
\pgfsetlinewidth{1.003750pt}%
\definecolor{currentstroke}{rgb}{0.121569,0.466667,0.705882}%
\pgfsetstrokecolor{currentstroke}%
\pgfsetstrokeopacity{0.686294}%
\pgfsetdash{}{0pt}%
\pgfpathmoveto{\pgfqpoint{3.059753in}{1.915169in}}%
\pgfpathcurveto{\pgfqpoint{3.067989in}{1.915169in}}{\pgfqpoint{3.075889in}{1.918441in}}{\pgfqpoint{3.081713in}{1.924265in}}%
\pgfpathcurveto{\pgfqpoint{3.087537in}{1.930089in}}{\pgfqpoint{3.090809in}{1.937989in}}{\pgfqpoint{3.090809in}{1.946225in}}%
\pgfpathcurveto{\pgfqpoint{3.090809in}{1.954461in}}{\pgfqpoint{3.087537in}{1.962362in}}{\pgfqpoint{3.081713in}{1.968185in}}%
\pgfpathcurveto{\pgfqpoint{3.075889in}{1.974009in}}{\pgfqpoint{3.067989in}{1.977282in}}{\pgfqpoint{3.059753in}{1.977282in}}%
\pgfpathcurveto{\pgfqpoint{3.051516in}{1.977282in}}{\pgfqpoint{3.043616in}{1.974009in}}{\pgfqpoint{3.037793in}{1.968185in}}%
\pgfpathcurveto{\pgfqpoint{3.031969in}{1.962362in}}{\pgfqpoint{3.028696in}{1.954461in}}{\pgfqpoint{3.028696in}{1.946225in}}%
\pgfpathcurveto{\pgfqpoint{3.028696in}{1.937989in}}{\pgfqpoint{3.031969in}{1.930089in}}{\pgfqpoint{3.037793in}{1.924265in}}%
\pgfpathcurveto{\pgfqpoint{3.043616in}{1.918441in}}{\pgfqpoint{3.051516in}{1.915169in}}{\pgfqpoint{3.059753in}{1.915169in}}%
\pgfpathclose%
\pgfusepath{stroke,fill}%
\end{pgfscope}%
\begin{pgfscope}%
\pgfpathrectangle{\pgfqpoint{0.100000in}{0.212622in}}{\pgfqpoint{3.696000in}{3.696000in}}%
\pgfusepath{clip}%
\pgfsetbuttcap%
\pgfsetroundjoin%
\definecolor{currentfill}{rgb}{0.121569,0.466667,0.705882}%
\pgfsetfillcolor{currentfill}%
\pgfsetfillopacity{0.686765}%
\pgfsetlinewidth{1.003750pt}%
\definecolor{currentstroke}{rgb}{0.121569,0.466667,0.705882}%
\pgfsetstrokecolor{currentstroke}%
\pgfsetstrokeopacity{0.686765}%
\pgfsetdash{}{0pt}%
\pgfpathmoveto{\pgfqpoint{0.724697in}{1.605798in}}%
\pgfpathcurveto{\pgfqpoint{0.732933in}{1.605798in}}{\pgfqpoint{0.740833in}{1.609070in}}{\pgfqpoint{0.746657in}{1.614894in}}%
\pgfpathcurveto{\pgfqpoint{0.752481in}{1.620718in}}{\pgfqpoint{0.755753in}{1.628618in}}{\pgfqpoint{0.755753in}{1.636854in}}%
\pgfpathcurveto{\pgfqpoint{0.755753in}{1.645091in}}{\pgfqpoint{0.752481in}{1.652991in}}{\pgfqpoint{0.746657in}{1.658815in}}%
\pgfpathcurveto{\pgfqpoint{0.740833in}{1.664638in}}{\pgfqpoint{0.732933in}{1.667911in}}{\pgfqpoint{0.724697in}{1.667911in}}%
\pgfpathcurveto{\pgfqpoint{0.716460in}{1.667911in}}{\pgfqpoint{0.708560in}{1.664638in}}{\pgfqpoint{0.702736in}{1.658815in}}%
\pgfpathcurveto{\pgfqpoint{0.696912in}{1.652991in}}{\pgfqpoint{0.693640in}{1.645091in}}{\pgfqpoint{0.693640in}{1.636854in}}%
\pgfpathcurveto{\pgfqpoint{0.693640in}{1.628618in}}{\pgfqpoint{0.696912in}{1.620718in}}{\pgfqpoint{0.702736in}{1.614894in}}%
\pgfpathcurveto{\pgfqpoint{0.708560in}{1.609070in}}{\pgfqpoint{0.716460in}{1.605798in}}{\pgfqpoint{0.724697in}{1.605798in}}%
\pgfpathclose%
\pgfusepath{stroke,fill}%
\end{pgfscope}%
\begin{pgfscope}%
\pgfpathrectangle{\pgfqpoint{0.100000in}{0.212622in}}{\pgfqpoint{3.696000in}{3.696000in}}%
\pgfusepath{clip}%
\pgfsetbuttcap%
\pgfsetroundjoin%
\definecolor{currentfill}{rgb}{0.121569,0.466667,0.705882}%
\pgfsetfillcolor{currentfill}%
\pgfsetfillopacity{0.689354}%
\pgfsetlinewidth{1.003750pt}%
\definecolor{currentstroke}{rgb}{0.121569,0.466667,0.705882}%
\pgfsetstrokecolor{currentstroke}%
\pgfsetstrokeopacity{0.689354}%
\pgfsetdash{}{0pt}%
\pgfpathmoveto{\pgfqpoint{3.056151in}{1.906147in}}%
\pgfpathcurveto{\pgfqpoint{3.064387in}{1.906147in}}{\pgfqpoint{3.072287in}{1.909419in}}{\pgfqpoint{3.078111in}{1.915243in}}%
\pgfpathcurveto{\pgfqpoint{3.083935in}{1.921067in}}{\pgfqpoint{3.087207in}{1.928967in}}{\pgfqpoint{3.087207in}{1.937203in}}%
\pgfpathcurveto{\pgfqpoint{3.087207in}{1.945440in}}{\pgfqpoint{3.083935in}{1.953340in}}{\pgfqpoint{3.078111in}{1.959164in}}%
\pgfpathcurveto{\pgfqpoint{3.072287in}{1.964988in}}{\pgfqpoint{3.064387in}{1.968260in}}{\pgfqpoint{3.056151in}{1.968260in}}%
\pgfpathcurveto{\pgfqpoint{3.047915in}{1.968260in}}{\pgfqpoint{3.040015in}{1.964988in}}{\pgfqpoint{3.034191in}{1.959164in}}%
\pgfpathcurveto{\pgfqpoint{3.028367in}{1.953340in}}{\pgfqpoint{3.025094in}{1.945440in}}{\pgfqpoint{3.025094in}{1.937203in}}%
\pgfpathcurveto{\pgfqpoint{3.025094in}{1.928967in}}{\pgfqpoint{3.028367in}{1.921067in}}{\pgfqpoint{3.034191in}{1.915243in}}%
\pgfpathcurveto{\pgfqpoint{3.040015in}{1.909419in}}{\pgfqpoint{3.047915in}{1.906147in}}{\pgfqpoint{3.056151in}{1.906147in}}%
\pgfpathclose%
\pgfusepath{stroke,fill}%
\end{pgfscope}%
\begin{pgfscope}%
\pgfpathrectangle{\pgfqpoint{0.100000in}{0.212622in}}{\pgfqpoint{3.696000in}{3.696000in}}%
\pgfusepath{clip}%
\pgfsetbuttcap%
\pgfsetroundjoin%
\definecolor{currentfill}{rgb}{0.121569,0.466667,0.705882}%
\pgfsetfillcolor{currentfill}%
\pgfsetfillopacity{0.689564}%
\pgfsetlinewidth{1.003750pt}%
\definecolor{currentstroke}{rgb}{0.121569,0.466667,0.705882}%
\pgfsetstrokecolor{currentstroke}%
\pgfsetstrokeopacity{0.689564}%
\pgfsetdash{}{0pt}%
\pgfpathmoveto{\pgfqpoint{0.719907in}{1.602847in}}%
\pgfpathcurveto{\pgfqpoint{0.728143in}{1.602847in}}{\pgfqpoint{0.736043in}{1.606119in}}{\pgfqpoint{0.741867in}{1.611943in}}%
\pgfpathcurveto{\pgfqpoint{0.747691in}{1.617767in}}{\pgfqpoint{0.750963in}{1.625667in}}{\pgfqpoint{0.750963in}{1.633903in}}%
\pgfpathcurveto{\pgfqpoint{0.750963in}{1.642140in}}{\pgfqpoint{0.747691in}{1.650040in}}{\pgfqpoint{0.741867in}{1.655864in}}%
\pgfpathcurveto{\pgfqpoint{0.736043in}{1.661687in}}{\pgfqpoint{0.728143in}{1.664960in}}{\pgfqpoint{0.719907in}{1.664960in}}%
\pgfpathcurveto{\pgfqpoint{0.711670in}{1.664960in}}{\pgfqpoint{0.703770in}{1.661687in}}{\pgfqpoint{0.697946in}{1.655864in}}%
\pgfpathcurveto{\pgfqpoint{0.692122in}{1.650040in}}{\pgfqpoint{0.688850in}{1.642140in}}{\pgfqpoint{0.688850in}{1.633903in}}%
\pgfpathcurveto{\pgfqpoint{0.688850in}{1.625667in}}{\pgfqpoint{0.692122in}{1.617767in}}{\pgfqpoint{0.697946in}{1.611943in}}%
\pgfpathcurveto{\pgfqpoint{0.703770in}{1.606119in}}{\pgfqpoint{0.711670in}{1.602847in}}{\pgfqpoint{0.719907in}{1.602847in}}%
\pgfpathclose%
\pgfusepath{stroke,fill}%
\end{pgfscope}%
\begin{pgfscope}%
\pgfpathrectangle{\pgfqpoint{0.100000in}{0.212622in}}{\pgfqpoint{3.696000in}{3.696000in}}%
\pgfusepath{clip}%
\pgfsetbuttcap%
\pgfsetroundjoin%
\definecolor{currentfill}{rgb}{0.121569,0.466667,0.705882}%
\pgfsetfillcolor{currentfill}%
\pgfsetfillopacity{0.691298}%
\pgfsetlinewidth{1.003750pt}%
\definecolor{currentstroke}{rgb}{0.121569,0.466667,0.705882}%
\pgfsetstrokecolor{currentstroke}%
\pgfsetstrokeopacity{0.691298}%
\pgfsetdash{}{0pt}%
\pgfpathmoveto{\pgfqpoint{0.714739in}{1.593781in}}%
\pgfpathcurveto{\pgfqpoint{0.722975in}{1.593781in}}{\pgfqpoint{0.730876in}{1.597053in}}{\pgfqpoint{0.736699in}{1.602877in}}%
\pgfpathcurveto{\pgfqpoint{0.742523in}{1.608701in}}{\pgfqpoint{0.745796in}{1.616601in}}{\pgfqpoint{0.745796in}{1.624837in}}%
\pgfpathcurveto{\pgfqpoint{0.745796in}{1.633074in}}{\pgfqpoint{0.742523in}{1.640974in}}{\pgfqpoint{0.736699in}{1.646798in}}%
\pgfpathcurveto{\pgfqpoint{0.730876in}{1.652621in}}{\pgfqpoint{0.722975in}{1.655894in}}{\pgfqpoint{0.714739in}{1.655894in}}%
\pgfpathcurveto{\pgfqpoint{0.706503in}{1.655894in}}{\pgfqpoint{0.698603in}{1.652621in}}{\pgfqpoint{0.692779in}{1.646798in}}%
\pgfpathcurveto{\pgfqpoint{0.686955in}{1.640974in}}{\pgfqpoint{0.683683in}{1.633074in}}{\pgfqpoint{0.683683in}{1.624837in}}%
\pgfpathcurveto{\pgfqpoint{0.683683in}{1.616601in}}{\pgfqpoint{0.686955in}{1.608701in}}{\pgfqpoint{0.692779in}{1.602877in}}%
\pgfpathcurveto{\pgfqpoint{0.698603in}{1.597053in}}{\pgfqpoint{0.706503in}{1.593781in}}{\pgfqpoint{0.714739in}{1.593781in}}%
\pgfpathclose%
\pgfusepath{stroke,fill}%
\end{pgfscope}%
\begin{pgfscope}%
\pgfpathrectangle{\pgfqpoint{0.100000in}{0.212622in}}{\pgfqpoint{3.696000in}{3.696000in}}%
\pgfusepath{clip}%
\pgfsetbuttcap%
\pgfsetroundjoin%
\definecolor{currentfill}{rgb}{0.121569,0.466667,0.705882}%
\pgfsetfillcolor{currentfill}%
\pgfsetfillopacity{0.692948}%
\pgfsetlinewidth{1.003750pt}%
\definecolor{currentstroke}{rgb}{0.121569,0.466667,0.705882}%
\pgfsetstrokecolor{currentstroke}%
\pgfsetstrokeopacity{0.692948}%
\pgfsetdash{}{0pt}%
\pgfpathmoveto{\pgfqpoint{3.047416in}{1.900075in}}%
\pgfpathcurveto{\pgfqpoint{3.055652in}{1.900075in}}{\pgfqpoint{3.063552in}{1.903348in}}{\pgfqpoint{3.069376in}{1.909172in}}%
\pgfpathcurveto{\pgfqpoint{3.075200in}{1.914996in}}{\pgfqpoint{3.078472in}{1.922896in}}{\pgfqpoint{3.078472in}{1.931132in}}%
\pgfpathcurveto{\pgfqpoint{3.078472in}{1.939368in}}{\pgfqpoint{3.075200in}{1.947268in}}{\pgfqpoint{3.069376in}{1.953092in}}%
\pgfpathcurveto{\pgfqpoint{3.063552in}{1.958916in}}{\pgfqpoint{3.055652in}{1.962188in}}{\pgfqpoint{3.047416in}{1.962188in}}%
\pgfpathcurveto{\pgfqpoint{3.039179in}{1.962188in}}{\pgfqpoint{3.031279in}{1.958916in}}{\pgfqpoint{3.025455in}{1.953092in}}%
\pgfpathcurveto{\pgfqpoint{3.019631in}{1.947268in}}{\pgfqpoint{3.016359in}{1.939368in}}{\pgfqpoint{3.016359in}{1.931132in}}%
\pgfpathcurveto{\pgfqpoint{3.016359in}{1.922896in}}{\pgfqpoint{3.019631in}{1.914996in}}{\pgfqpoint{3.025455in}{1.909172in}}%
\pgfpathcurveto{\pgfqpoint{3.031279in}{1.903348in}}{\pgfqpoint{3.039179in}{1.900075in}}{\pgfqpoint{3.047416in}{1.900075in}}%
\pgfpathclose%
\pgfusepath{stroke,fill}%
\end{pgfscope}%
\begin{pgfscope}%
\pgfpathrectangle{\pgfqpoint{0.100000in}{0.212622in}}{\pgfqpoint{3.696000in}{3.696000in}}%
\pgfusepath{clip}%
\pgfsetbuttcap%
\pgfsetroundjoin%
\definecolor{currentfill}{rgb}{0.121569,0.466667,0.705882}%
\pgfsetfillcolor{currentfill}%
\pgfsetfillopacity{0.693326}%
\pgfsetlinewidth{1.003750pt}%
\definecolor{currentstroke}{rgb}{0.121569,0.466667,0.705882}%
\pgfsetstrokecolor{currentstroke}%
\pgfsetstrokeopacity{0.693326}%
\pgfsetdash{}{0pt}%
\pgfpathmoveto{\pgfqpoint{0.706823in}{1.591073in}}%
\pgfpathcurveto{\pgfqpoint{0.715059in}{1.591073in}}{\pgfqpoint{0.722959in}{1.594346in}}{\pgfqpoint{0.728783in}{1.600170in}}%
\pgfpathcurveto{\pgfqpoint{0.734607in}{1.605994in}}{\pgfqpoint{0.737879in}{1.613894in}}{\pgfqpoint{0.737879in}{1.622130in}}%
\pgfpathcurveto{\pgfqpoint{0.737879in}{1.630366in}}{\pgfqpoint{0.734607in}{1.638266in}}{\pgfqpoint{0.728783in}{1.644090in}}%
\pgfpathcurveto{\pgfqpoint{0.722959in}{1.649914in}}{\pgfqpoint{0.715059in}{1.653186in}}{\pgfqpoint{0.706823in}{1.653186in}}%
\pgfpathcurveto{\pgfqpoint{0.698586in}{1.653186in}}{\pgfqpoint{0.690686in}{1.649914in}}{\pgfqpoint{0.684862in}{1.644090in}}%
\pgfpathcurveto{\pgfqpoint{0.679038in}{1.638266in}}{\pgfqpoint{0.675766in}{1.630366in}}{\pgfqpoint{0.675766in}{1.622130in}}%
\pgfpathcurveto{\pgfqpoint{0.675766in}{1.613894in}}{\pgfqpoint{0.679038in}{1.605994in}}{\pgfqpoint{0.684862in}{1.600170in}}%
\pgfpathcurveto{\pgfqpoint{0.690686in}{1.594346in}}{\pgfqpoint{0.698586in}{1.591073in}}{\pgfqpoint{0.706823in}{1.591073in}}%
\pgfpathclose%
\pgfusepath{stroke,fill}%
\end{pgfscope}%
\begin{pgfscope}%
\pgfpathrectangle{\pgfqpoint{0.100000in}{0.212622in}}{\pgfqpoint{3.696000in}{3.696000in}}%
\pgfusepath{clip}%
\pgfsetbuttcap%
\pgfsetroundjoin%
\definecolor{currentfill}{rgb}{0.121569,0.466667,0.705882}%
\pgfsetfillcolor{currentfill}%
\pgfsetfillopacity{0.695953}%
\pgfsetlinewidth{1.003750pt}%
\definecolor{currentstroke}{rgb}{0.121569,0.466667,0.705882}%
\pgfsetstrokecolor{currentstroke}%
\pgfsetstrokeopacity{0.695953}%
\pgfsetdash{}{0pt}%
\pgfpathmoveto{\pgfqpoint{0.703364in}{1.589473in}}%
\pgfpathcurveto{\pgfqpoint{0.711601in}{1.589473in}}{\pgfqpoint{0.719501in}{1.592745in}}{\pgfqpoint{0.725324in}{1.598569in}}%
\pgfpathcurveto{\pgfqpoint{0.731148in}{1.604393in}}{\pgfqpoint{0.734421in}{1.612293in}}{\pgfqpoint{0.734421in}{1.620529in}}%
\pgfpathcurveto{\pgfqpoint{0.734421in}{1.628765in}}{\pgfqpoint{0.731148in}{1.636665in}}{\pgfqpoint{0.725324in}{1.642489in}}%
\pgfpathcurveto{\pgfqpoint{0.719501in}{1.648313in}}{\pgfqpoint{0.711601in}{1.651586in}}{\pgfqpoint{0.703364in}{1.651586in}}%
\pgfpathcurveto{\pgfqpoint{0.695128in}{1.651586in}}{\pgfqpoint{0.687228in}{1.648313in}}{\pgfqpoint{0.681404in}{1.642489in}}%
\pgfpathcurveto{\pgfqpoint{0.675580in}{1.636665in}}{\pgfqpoint{0.672308in}{1.628765in}}{\pgfqpoint{0.672308in}{1.620529in}}%
\pgfpathcurveto{\pgfqpoint{0.672308in}{1.612293in}}{\pgfqpoint{0.675580in}{1.604393in}}{\pgfqpoint{0.681404in}{1.598569in}}%
\pgfpathcurveto{\pgfqpoint{0.687228in}{1.592745in}}{\pgfqpoint{0.695128in}{1.589473in}}{\pgfqpoint{0.703364in}{1.589473in}}%
\pgfpathclose%
\pgfusepath{stroke,fill}%
\end{pgfscope}%
\begin{pgfscope}%
\pgfpathrectangle{\pgfqpoint{0.100000in}{0.212622in}}{\pgfqpoint{3.696000in}{3.696000in}}%
\pgfusepath{clip}%
\pgfsetbuttcap%
\pgfsetroundjoin%
\definecolor{currentfill}{rgb}{0.121569,0.466667,0.705882}%
\pgfsetfillcolor{currentfill}%
\pgfsetfillopacity{0.697187}%
\pgfsetlinewidth{1.003750pt}%
\definecolor{currentstroke}{rgb}{0.121569,0.466667,0.705882}%
\pgfsetstrokecolor{currentstroke}%
\pgfsetstrokeopacity{0.697187}%
\pgfsetdash{}{0pt}%
\pgfpathmoveto{\pgfqpoint{3.036474in}{1.900490in}}%
\pgfpathcurveto{\pgfqpoint{3.044710in}{1.900490in}}{\pgfqpoint{3.052610in}{1.903762in}}{\pgfqpoint{3.058434in}{1.909586in}}%
\pgfpathcurveto{\pgfqpoint{3.064258in}{1.915410in}}{\pgfqpoint{3.067531in}{1.923310in}}{\pgfqpoint{3.067531in}{1.931546in}}%
\pgfpathcurveto{\pgfqpoint{3.067531in}{1.939782in}}{\pgfqpoint{3.064258in}{1.947683in}}{\pgfqpoint{3.058434in}{1.953506in}}%
\pgfpathcurveto{\pgfqpoint{3.052610in}{1.959330in}}{\pgfqpoint{3.044710in}{1.962603in}}{\pgfqpoint{3.036474in}{1.962603in}}%
\pgfpathcurveto{\pgfqpoint{3.028238in}{1.962603in}}{\pgfqpoint{3.020338in}{1.959330in}}{\pgfqpoint{3.014514in}{1.953506in}}%
\pgfpathcurveto{\pgfqpoint{3.008690in}{1.947683in}}{\pgfqpoint{3.005418in}{1.939782in}}{\pgfqpoint{3.005418in}{1.931546in}}%
\pgfpathcurveto{\pgfqpoint{3.005418in}{1.923310in}}{\pgfqpoint{3.008690in}{1.915410in}}{\pgfqpoint{3.014514in}{1.909586in}}%
\pgfpathcurveto{\pgfqpoint{3.020338in}{1.903762in}}{\pgfqpoint{3.028238in}{1.900490in}}{\pgfqpoint{3.036474in}{1.900490in}}%
\pgfpathclose%
\pgfusepath{stroke,fill}%
\end{pgfscope}%
\begin{pgfscope}%
\pgfpathrectangle{\pgfqpoint{0.100000in}{0.212622in}}{\pgfqpoint{3.696000in}{3.696000in}}%
\pgfusepath{clip}%
\pgfsetbuttcap%
\pgfsetroundjoin%
\definecolor{currentfill}{rgb}{0.121569,0.466667,0.705882}%
\pgfsetfillcolor{currentfill}%
\pgfsetfillopacity{0.697815}%
\pgfsetlinewidth{1.003750pt}%
\definecolor{currentstroke}{rgb}{0.121569,0.466667,0.705882}%
\pgfsetstrokecolor{currentstroke}%
\pgfsetstrokeopacity{0.697815}%
\pgfsetdash{}{0pt}%
\pgfpathmoveto{\pgfqpoint{0.697178in}{1.589481in}}%
\pgfpathcurveto{\pgfqpoint{0.705414in}{1.589481in}}{\pgfqpoint{0.713314in}{1.592753in}}{\pgfqpoint{0.719138in}{1.598577in}}%
\pgfpathcurveto{\pgfqpoint{0.724962in}{1.604401in}}{\pgfqpoint{0.728234in}{1.612301in}}{\pgfqpoint{0.728234in}{1.620538in}}%
\pgfpathcurveto{\pgfqpoint{0.728234in}{1.628774in}}{\pgfqpoint{0.724962in}{1.636674in}}{\pgfqpoint{0.719138in}{1.642498in}}%
\pgfpathcurveto{\pgfqpoint{0.713314in}{1.648322in}}{\pgfqpoint{0.705414in}{1.651594in}}{\pgfqpoint{0.697178in}{1.651594in}}%
\pgfpathcurveto{\pgfqpoint{0.688941in}{1.651594in}}{\pgfqpoint{0.681041in}{1.648322in}}{\pgfqpoint{0.675217in}{1.642498in}}%
\pgfpathcurveto{\pgfqpoint{0.669393in}{1.636674in}}{\pgfqpoint{0.666121in}{1.628774in}}{\pgfqpoint{0.666121in}{1.620538in}}%
\pgfpathcurveto{\pgfqpoint{0.666121in}{1.612301in}}{\pgfqpoint{0.669393in}{1.604401in}}{\pgfqpoint{0.675217in}{1.598577in}}%
\pgfpathcurveto{\pgfqpoint{0.681041in}{1.592753in}}{\pgfqpoint{0.688941in}{1.589481in}}{\pgfqpoint{0.697178in}{1.589481in}}%
\pgfpathclose%
\pgfusepath{stroke,fill}%
\end{pgfscope}%
\begin{pgfscope}%
\pgfpathrectangle{\pgfqpoint{0.100000in}{0.212622in}}{\pgfqpoint{3.696000in}{3.696000in}}%
\pgfusepath{clip}%
\pgfsetbuttcap%
\pgfsetroundjoin%
\definecolor{currentfill}{rgb}{0.121569,0.466667,0.705882}%
\pgfsetfillcolor{currentfill}%
\pgfsetfillopacity{0.700035}%
\pgfsetlinewidth{1.003750pt}%
\definecolor{currentstroke}{rgb}{0.121569,0.466667,0.705882}%
\pgfsetstrokecolor{currentstroke}%
\pgfsetstrokeopacity{0.700035}%
\pgfsetdash{}{0pt}%
\pgfpathmoveto{\pgfqpoint{0.695009in}{1.591793in}}%
\pgfpathcurveto{\pgfqpoint{0.703245in}{1.591793in}}{\pgfqpoint{0.711145in}{1.595065in}}{\pgfqpoint{0.716969in}{1.600889in}}%
\pgfpathcurveto{\pgfqpoint{0.722793in}{1.606713in}}{\pgfqpoint{0.726065in}{1.614613in}}{\pgfqpoint{0.726065in}{1.622849in}}%
\pgfpathcurveto{\pgfqpoint{0.726065in}{1.631086in}}{\pgfqpoint{0.722793in}{1.638986in}}{\pgfqpoint{0.716969in}{1.644810in}}%
\pgfpathcurveto{\pgfqpoint{0.711145in}{1.650633in}}{\pgfqpoint{0.703245in}{1.653906in}}{\pgfqpoint{0.695009in}{1.653906in}}%
\pgfpathcurveto{\pgfqpoint{0.686772in}{1.653906in}}{\pgfqpoint{0.678872in}{1.650633in}}{\pgfqpoint{0.673048in}{1.644810in}}%
\pgfpathcurveto{\pgfqpoint{0.667224in}{1.638986in}}{\pgfqpoint{0.663952in}{1.631086in}}{\pgfqpoint{0.663952in}{1.622849in}}%
\pgfpathcurveto{\pgfqpoint{0.663952in}{1.614613in}}{\pgfqpoint{0.667224in}{1.606713in}}{\pgfqpoint{0.673048in}{1.600889in}}%
\pgfpathcurveto{\pgfqpoint{0.678872in}{1.595065in}}{\pgfqpoint{0.686772in}{1.591793in}}{\pgfqpoint{0.695009in}{1.591793in}}%
\pgfpathclose%
\pgfusepath{stroke,fill}%
\end{pgfscope}%
\begin{pgfscope}%
\pgfpathrectangle{\pgfqpoint{0.100000in}{0.212622in}}{\pgfqpoint{3.696000in}{3.696000in}}%
\pgfusepath{clip}%
\pgfsetbuttcap%
\pgfsetroundjoin%
\definecolor{currentfill}{rgb}{0.121569,0.466667,0.705882}%
\pgfsetfillcolor{currentfill}%
\pgfsetfillopacity{0.701208}%
\pgfsetlinewidth{1.003750pt}%
\definecolor{currentstroke}{rgb}{0.121569,0.466667,0.705882}%
\pgfsetstrokecolor{currentstroke}%
\pgfsetstrokeopacity{0.701208}%
\pgfsetdash{}{0pt}%
\pgfpathmoveto{\pgfqpoint{0.691389in}{1.593526in}}%
\pgfpathcurveto{\pgfqpoint{0.699625in}{1.593526in}}{\pgfqpoint{0.707525in}{1.596799in}}{\pgfqpoint{0.713349in}{1.602623in}}%
\pgfpathcurveto{\pgfqpoint{0.719173in}{1.608447in}}{\pgfqpoint{0.722445in}{1.616347in}}{\pgfqpoint{0.722445in}{1.624583in}}%
\pgfpathcurveto{\pgfqpoint{0.722445in}{1.632819in}}{\pgfqpoint{0.719173in}{1.640719in}}{\pgfqpoint{0.713349in}{1.646543in}}%
\pgfpathcurveto{\pgfqpoint{0.707525in}{1.652367in}}{\pgfqpoint{0.699625in}{1.655639in}}{\pgfqpoint{0.691389in}{1.655639in}}%
\pgfpathcurveto{\pgfqpoint{0.683153in}{1.655639in}}{\pgfqpoint{0.675253in}{1.652367in}}{\pgfqpoint{0.669429in}{1.646543in}}%
\pgfpathcurveto{\pgfqpoint{0.663605in}{1.640719in}}{\pgfqpoint{0.660332in}{1.632819in}}{\pgfqpoint{0.660332in}{1.624583in}}%
\pgfpathcurveto{\pgfqpoint{0.660332in}{1.616347in}}{\pgfqpoint{0.663605in}{1.608447in}}{\pgfqpoint{0.669429in}{1.602623in}}%
\pgfpathcurveto{\pgfqpoint{0.675253in}{1.596799in}}{\pgfqpoint{0.683153in}{1.593526in}}{\pgfqpoint{0.691389in}{1.593526in}}%
\pgfpathclose%
\pgfusepath{stroke,fill}%
\end{pgfscope}%
\begin{pgfscope}%
\pgfpathrectangle{\pgfqpoint{0.100000in}{0.212622in}}{\pgfqpoint{3.696000in}{3.696000in}}%
\pgfusepath{clip}%
\pgfsetbuttcap%
\pgfsetroundjoin%
\definecolor{currentfill}{rgb}{0.121569,0.466667,0.705882}%
\pgfsetfillcolor{currentfill}%
\pgfsetfillopacity{0.701961}%
\pgfsetlinewidth{1.003750pt}%
\definecolor{currentstroke}{rgb}{0.121569,0.466667,0.705882}%
\pgfsetstrokecolor{currentstroke}%
\pgfsetstrokeopacity{0.701961}%
\pgfsetdash{}{0pt}%
\pgfpathmoveto{\pgfqpoint{3.033470in}{1.894338in}}%
\pgfpathcurveto{\pgfqpoint{3.041706in}{1.894338in}}{\pgfqpoint{3.049607in}{1.897610in}}{\pgfqpoint{3.055430in}{1.903434in}}%
\pgfpathcurveto{\pgfqpoint{3.061254in}{1.909258in}}{\pgfqpoint{3.064527in}{1.917158in}}{\pgfqpoint{3.064527in}{1.925394in}}%
\pgfpathcurveto{\pgfqpoint{3.064527in}{1.933631in}}{\pgfqpoint{3.061254in}{1.941531in}}{\pgfqpoint{3.055430in}{1.947355in}}%
\pgfpathcurveto{\pgfqpoint{3.049607in}{1.953179in}}{\pgfqpoint{3.041706in}{1.956451in}}{\pgfqpoint{3.033470in}{1.956451in}}%
\pgfpathcurveto{\pgfqpoint{3.025234in}{1.956451in}}{\pgfqpoint{3.017334in}{1.953179in}}{\pgfqpoint{3.011510in}{1.947355in}}%
\pgfpathcurveto{\pgfqpoint{3.005686in}{1.941531in}}{\pgfqpoint{3.002414in}{1.933631in}}{\pgfqpoint{3.002414in}{1.925394in}}%
\pgfpathcurveto{\pgfqpoint{3.002414in}{1.917158in}}{\pgfqpoint{3.005686in}{1.909258in}}{\pgfqpoint{3.011510in}{1.903434in}}%
\pgfpathcurveto{\pgfqpoint{3.017334in}{1.897610in}}{\pgfqpoint{3.025234in}{1.894338in}}{\pgfqpoint{3.033470in}{1.894338in}}%
\pgfpathclose%
\pgfusepath{stroke,fill}%
\end{pgfscope}%
\begin{pgfscope}%
\pgfpathrectangle{\pgfqpoint{0.100000in}{0.212622in}}{\pgfqpoint{3.696000in}{3.696000in}}%
\pgfusepath{clip}%
\pgfsetbuttcap%
\pgfsetroundjoin%
\definecolor{currentfill}{rgb}{0.121569,0.466667,0.705882}%
\pgfsetfillcolor{currentfill}%
\pgfsetfillopacity{0.702211}%
\pgfsetlinewidth{1.003750pt}%
\definecolor{currentstroke}{rgb}{0.121569,0.466667,0.705882}%
\pgfsetstrokecolor{currentstroke}%
\pgfsetstrokeopacity{0.702211}%
\pgfsetdash{}{0pt}%
\pgfpathmoveto{\pgfqpoint{0.690680in}{1.594274in}}%
\pgfpathcurveto{\pgfqpoint{0.698916in}{1.594274in}}{\pgfqpoint{0.706816in}{1.597547in}}{\pgfqpoint{0.712640in}{1.603370in}}%
\pgfpathcurveto{\pgfqpoint{0.718464in}{1.609194in}}{\pgfqpoint{0.721737in}{1.617094in}}{\pgfqpoint{0.721737in}{1.625331in}}%
\pgfpathcurveto{\pgfqpoint{0.721737in}{1.633567in}}{\pgfqpoint{0.718464in}{1.641467in}}{\pgfqpoint{0.712640in}{1.647291in}}%
\pgfpathcurveto{\pgfqpoint{0.706816in}{1.653115in}}{\pgfqpoint{0.698916in}{1.656387in}}{\pgfqpoint{0.690680in}{1.656387in}}%
\pgfpathcurveto{\pgfqpoint{0.682444in}{1.656387in}}{\pgfqpoint{0.674544in}{1.653115in}}{\pgfqpoint{0.668720in}{1.647291in}}%
\pgfpathcurveto{\pgfqpoint{0.662896in}{1.641467in}}{\pgfqpoint{0.659624in}{1.633567in}}{\pgfqpoint{0.659624in}{1.625331in}}%
\pgfpathcurveto{\pgfqpoint{0.659624in}{1.617094in}}{\pgfqpoint{0.662896in}{1.609194in}}{\pgfqpoint{0.668720in}{1.603370in}}%
\pgfpathcurveto{\pgfqpoint{0.674544in}{1.597547in}}{\pgfqpoint{0.682444in}{1.594274in}}{\pgfqpoint{0.690680in}{1.594274in}}%
\pgfpathclose%
\pgfusepath{stroke,fill}%
\end{pgfscope}%
\begin{pgfscope}%
\pgfpathrectangle{\pgfqpoint{0.100000in}{0.212622in}}{\pgfqpoint{3.696000in}{3.696000in}}%
\pgfusepath{clip}%
\pgfsetbuttcap%
\pgfsetroundjoin%
\definecolor{currentfill}{rgb}{0.121569,0.466667,0.705882}%
\pgfsetfillcolor{currentfill}%
\pgfsetfillopacity{0.702284}%
\pgfsetlinewidth{1.003750pt}%
\definecolor{currentstroke}{rgb}{0.121569,0.466667,0.705882}%
\pgfsetstrokecolor{currentstroke}%
\pgfsetstrokeopacity{0.702284}%
\pgfsetdash{}{0pt}%
\pgfpathmoveto{\pgfqpoint{0.690383in}{1.594080in}}%
\pgfpathcurveto{\pgfqpoint{0.698620in}{1.594080in}}{\pgfqpoint{0.706520in}{1.597353in}}{\pgfqpoint{0.712344in}{1.603177in}}%
\pgfpathcurveto{\pgfqpoint{0.718168in}{1.609001in}}{\pgfqpoint{0.721440in}{1.616901in}}{\pgfqpoint{0.721440in}{1.625137in}}%
\pgfpathcurveto{\pgfqpoint{0.721440in}{1.633373in}}{\pgfqpoint{0.718168in}{1.641273in}}{\pgfqpoint{0.712344in}{1.647097in}}%
\pgfpathcurveto{\pgfqpoint{0.706520in}{1.652921in}}{\pgfqpoint{0.698620in}{1.656193in}}{\pgfqpoint{0.690383in}{1.656193in}}%
\pgfpathcurveto{\pgfqpoint{0.682147in}{1.656193in}}{\pgfqpoint{0.674247in}{1.652921in}}{\pgfqpoint{0.668423in}{1.647097in}}%
\pgfpathcurveto{\pgfqpoint{0.662599in}{1.641273in}}{\pgfqpoint{0.659327in}{1.633373in}}{\pgfqpoint{0.659327in}{1.625137in}}%
\pgfpathcurveto{\pgfqpoint{0.659327in}{1.616901in}}{\pgfqpoint{0.662599in}{1.609001in}}{\pgfqpoint{0.668423in}{1.603177in}}%
\pgfpathcurveto{\pgfqpoint{0.674247in}{1.597353in}}{\pgfqpoint{0.682147in}{1.594080in}}{\pgfqpoint{0.690383in}{1.594080in}}%
\pgfpathclose%
\pgfusepath{stroke,fill}%
\end{pgfscope}%
\begin{pgfscope}%
\pgfpathrectangle{\pgfqpoint{0.100000in}{0.212622in}}{\pgfqpoint{3.696000in}{3.696000in}}%
\pgfusepath{clip}%
\pgfsetbuttcap%
\pgfsetroundjoin%
\definecolor{currentfill}{rgb}{0.121569,0.466667,0.705882}%
\pgfsetfillcolor{currentfill}%
\pgfsetfillopacity{0.702284}%
\pgfsetlinewidth{1.003750pt}%
\definecolor{currentstroke}{rgb}{0.121569,0.466667,0.705882}%
\pgfsetstrokecolor{currentstroke}%
\pgfsetstrokeopacity{0.702284}%
\pgfsetdash{}{0pt}%
\pgfpathmoveto{\pgfqpoint{0.690383in}{1.594080in}}%
\pgfpathcurveto{\pgfqpoint{0.698620in}{1.594080in}}{\pgfqpoint{0.706520in}{1.597353in}}{\pgfqpoint{0.712344in}{1.603177in}}%
\pgfpathcurveto{\pgfqpoint{0.718167in}{1.609001in}}{\pgfqpoint{0.721440in}{1.616901in}}{\pgfqpoint{0.721440in}{1.625137in}}%
\pgfpathcurveto{\pgfqpoint{0.721440in}{1.633373in}}{\pgfqpoint{0.718167in}{1.641273in}}{\pgfqpoint{0.712344in}{1.647097in}}%
\pgfpathcurveto{\pgfqpoint{0.706520in}{1.652921in}}{\pgfqpoint{0.698620in}{1.656193in}}{\pgfqpoint{0.690383in}{1.656193in}}%
\pgfpathcurveto{\pgfqpoint{0.682147in}{1.656193in}}{\pgfqpoint{0.674247in}{1.652921in}}{\pgfqpoint{0.668423in}{1.647097in}}%
\pgfpathcurveto{\pgfqpoint{0.662599in}{1.641273in}}{\pgfqpoint{0.659327in}{1.633373in}}{\pgfqpoint{0.659327in}{1.625137in}}%
\pgfpathcurveto{\pgfqpoint{0.659327in}{1.616901in}}{\pgfqpoint{0.662599in}{1.609001in}}{\pgfqpoint{0.668423in}{1.603177in}}%
\pgfpathcurveto{\pgfqpoint{0.674247in}{1.597353in}}{\pgfqpoint{0.682147in}{1.594080in}}{\pgfqpoint{0.690383in}{1.594080in}}%
\pgfpathclose%
\pgfusepath{stroke,fill}%
\end{pgfscope}%
\begin{pgfscope}%
\pgfpathrectangle{\pgfqpoint{0.100000in}{0.212622in}}{\pgfqpoint{3.696000in}{3.696000in}}%
\pgfusepath{clip}%
\pgfsetbuttcap%
\pgfsetroundjoin%
\definecolor{currentfill}{rgb}{0.121569,0.466667,0.705882}%
\pgfsetfillcolor{currentfill}%
\pgfsetfillopacity{0.702284}%
\pgfsetlinewidth{1.003750pt}%
\definecolor{currentstroke}{rgb}{0.121569,0.466667,0.705882}%
\pgfsetstrokecolor{currentstroke}%
\pgfsetstrokeopacity{0.702284}%
\pgfsetdash{}{0pt}%
\pgfpathmoveto{\pgfqpoint{0.690383in}{1.594080in}}%
\pgfpathcurveto{\pgfqpoint{0.698619in}{1.594080in}}{\pgfqpoint{0.706519in}{1.597353in}}{\pgfqpoint{0.712343in}{1.603176in}}%
\pgfpathcurveto{\pgfqpoint{0.718167in}{1.609000in}}{\pgfqpoint{0.721440in}{1.616900in}}{\pgfqpoint{0.721440in}{1.625137in}}%
\pgfpathcurveto{\pgfqpoint{0.721440in}{1.633373in}}{\pgfqpoint{0.718167in}{1.641273in}}{\pgfqpoint{0.712343in}{1.647097in}}%
\pgfpathcurveto{\pgfqpoint{0.706519in}{1.652921in}}{\pgfqpoint{0.698619in}{1.656193in}}{\pgfqpoint{0.690383in}{1.656193in}}%
\pgfpathcurveto{\pgfqpoint{0.682147in}{1.656193in}}{\pgfqpoint{0.674247in}{1.652921in}}{\pgfqpoint{0.668423in}{1.647097in}}%
\pgfpathcurveto{\pgfqpoint{0.662599in}{1.641273in}}{\pgfqpoint{0.659327in}{1.633373in}}{\pgfqpoint{0.659327in}{1.625137in}}%
\pgfpathcurveto{\pgfqpoint{0.659327in}{1.616900in}}{\pgfqpoint{0.662599in}{1.609000in}}{\pgfqpoint{0.668423in}{1.603176in}}%
\pgfpathcurveto{\pgfqpoint{0.674247in}{1.597353in}}{\pgfqpoint{0.682147in}{1.594080in}}{\pgfqpoint{0.690383in}{1.594080in}}%
\pgfpathclose%
\pgfusepath{stroke,fill}%
\end{pgfscope}%
\begin{pgfscope}%
\pgfpathrectangle{\pgfqpoint{0.100000in}{0.212622in}}{\pgfqpoint{3.696000in}{3.696000in}}%
\pgfusepath{clip}%
\pgfsetbuttcap%
\pgfsetroundjoin%
\definecolor{currentfill}{rgb}{0.121569,0.466667,0.705882}%
\pgfsetfillcolor{currentfill}%
\pgfsetfillopacity{0.702284}%
\pgfsetlinewidth{1.003750pt}%
\definecolor{currentstroke}{rgb}{0.121569,0.466667,0.705882}%
\pgfsetstrokecolor{currentstroke}%
\pgfsetstrokeopacity{0.702284}%
\pgfsetdash{}{0pt}%
\pgfpathmoveto{\pgfqpoint{0.690382in}{1.594080in}}%
\pgfpathcurveto{\pgfqpoint{0.698619in}{1.594080in}}{\pgfqpoint{0.706519in}{1.597352in}}{\pgfqpoint{0.712343in}{1.603176in}}%
\pgfpathcurveto{\pgfqpoint{0.718167in}{1.609000in}}{\pgfqpoint{0.721439in}{1.616900in}}{\pgfqpoint{0.721439in}{1.625136in}}%
\pgfpathcurveto{\pgfqpoint{0.721439in}{1.633373in}}{\pgfqpoint{0.718167in}{1.641273in}}{\pgfqpoint{0.712343in}{1.647097in}}%
\pgfpathcurveto{\pgfqpoint{0.706519in}{1.652920in}}{\pgfqpoint{0.698619in}{1.656193in}}{\pgfqpoint{0.690382in}{1.656193in}}%
\pgfpathcurveto{\pgfqpoint{0.682146in}{1.656193in}}{\pgfqpoint{0.674246in}{1.652920in}}{\pgfqpoint{0.668422in}{1.647097in}}%
\pgfpathcurveto{\pgfqpoint{0.662598in}{1.641273in}}{\pgfqpoint{0.659326in}{1.633373in}}{\pgfqpoint{0.659326in}{1.625136in}}%
\pgfpathcurveto{\pgfqpoint{0.659326in}{1.616900in}}{\pgfqpoint{0.662598in}{1.609000in}}{\pgfqpoint{0.668422in}{1.603176in}}%
\pgfpathcurveto{\pgfqpoint{0.674246in}{1.597352in}}{\pgfqpoint{0.682146in}{1.594080in}}{\pgfqpoint{0.690382in}{1.594080in}}%
\pgfpathclose%
\pgfusepath{stroke,fill}%
\end{pgfscope}%
\begin{pgfscope}%
\pgfpathrectangle{\pgfqpoint{0.100000in}{0.212622in}}{\pgfqpoint{3.696000in}{3.696000in}}%
\pgfusepath{clip}%
\pgfsetbuttcap%
\pgfsetroundjoin%
\definecolor{currentfill}{rgb}{0.121569,0.466667,0.705882}%
\pgfsetfillcolor{currentfill}%
\pgfsetfillopacity{0.702284}%
\pgfsetlinewidth{1.003750pt}%
\definecolor{currentstroke}{rgb}{0.121569,0.466667,0.705882}%
\pgfsetstrokecolor{currentstroke}%
\pgfsetstrokeopacity{0.702284}%
\pgfsetdash{}{0pt}%
\pgfpathmoveto{\pgfqpoint{0.690382in}{1.594079in}}%
\pgfpathcurveto{\pgfqpoint{0.698618in}{1.594079in}}{\pgfqpoint{0.706518in}{1.597351in}}{\pgfqpoint{0.712342in}{1.603175in}}%
\pgfpathcurveto{\pgfqpoint{0.718166in}{1.608999in}}{\pgfqpoint{0.721438in}{1.616899in}}{\pgfqpoint{0.721438in}{1.625135in}}%
\pgfpathcurveto{\pgfqpoint{0.721438in}{1.633372in}}{\pgfqpoint{0.718166in}{1.641272in}}{\pgfqpoint{0.712342in}{1.647096in}}%
\pgfpathcurveto{\pgfqpoint{0.706518in}{1.652920in}}{\pgfqpoint{0.698618in}{1.656192in}}{\pgfqpoint{0.690382in}{1.656192in}}%
\pgfpathcurveto{\pgfqpoint{0.682146in}{1.656192in}}{\pgfqpoint{0.674246in}{1.652920in}}{\pgfqpoint{0.668422in}{1.647096in}}%
\pgfpathcurveto{\pgfqpoint{0.662598in}{1.641272in}}{\pgfqpoint{0.659325in}{1.633372in}}{\pgfqpoint{0.659325in}{1.625135in}}%
\pgfpathcurveto{\pgfqpoint{0.659325in}{1.616899in}}{\pgfqpoint{0.662598in}{1.608999in}}{\pgfqpoint{0.668422in}{1.603175in}}%
\pgfpathcurveto{\pgfqpoint{0.674246in}{1.597351in}}{\pgfqpoint{0.682146in}{1.594079in}}{\pgfqpoint{0.690382in}{1.594079in}}%
\pgfpathclose%
\pgfusepath{stroke,fill}%
\end{pgfscope}%
\begin{pgfscope}%
\pgfpathrectangle{\pgfqpoint{0.100000in}{0.212622in}}{\pgfqpoint{3.696000in}{3.696000in}}%
\pgfusepath{clip}%
\pgfsetbuttcap%
\pgfsetroundjoin%
\definecolor{currentfill}{rgb}{0.121569,0.466667,0.705882}%
\pgfsetfillcolor{currentfill}%
\pgfsetfillopacity{0.702285}%
\pgfsetlinewidth{1.003750pt}%
\definecolor{currentstroke}{rgb}{0.121569,0.466667,0.705882}%
\pgfsetstrokecolor{currentstroke}%
\pgfsetstrokeopacity{0.702285}%
\pgfsetdash{}{0pt}%
\pgfpathmoveto{\pgfqpoint{0.690380in}{1.594077in}}%
\pgfpathcurveto{\pgfqpoint{0.698616in}{1.594077in}}{\pgfqpoint{0.706516in}{1.597349in}}{\pgfqpoint{0.712340in}{1.603173in}}%
\pgfpathcurveto{\pgfqpoint{0.718164in}{1.608997in}}{\pgfqpoint{0.721436in}{1.616897in}}{\pgfqpoint{0.721436in}{1.625134in}}%
\pgfpathcurveto{\pgfqpoint{0.721436in}{1.633370in}}{\pgfqpoint{0.718164in}{1.641270in}}{\pgfqpoint{0.712340in}{1.647094in}}%
\pgfpathcurveto{\pgfqpoint{0.706516in}{1.652918in}}{\pgfqpoint{0.698616in}{1.656190in}}{\pgfqpoint{0.690380in}{1.656190in}}%
\pgfpathcurveto{\pgfqpoint{0.682144in}{1.656190in}}{\pgfqpoint{0.674244in}{1.652918in}}{\pgfqpoint{0.668420in}{1.647094in}}%
\pgfpathcurveto{\pgfqpoint{0.662596in}{1.641270in}}{\pgfqpoint{0.659323in}{1.633370in}}{\pgfqpoint{0.659323in}{1.625134in}}%
\pgfpathcurveto{\pgfqpoint{0.659323in}{1.616897in}}{\pgfqpoint{0.662596in}{1.608997in}}{\pgfqpoint{0.668420in}{1.603173in}}%
\pgfpathcurveto{\pgfqpoint{0.674244in}{1.597349in}}{\pgfqpoint{0.682144in}{1.594077in}}{\pgfqpoint{0.690380in}{1.594077in}}%
\pgfpathclose%
\pgfusepath{stroke,fill}%
\end{pgfscope}%
\begin{pgfscope}%
\pgfpathrectangle{\pgfqpoint{0.100000in}{0.212622in}}{\pgfqpoint{3.696000in}{3.696000in}}%
\pgfusepath{clip}%
\pgfsetbuttcap%
\pgfsetroundjoin%
\definecolor{currentfill}{rgb}{0.121569,0.466667,0.705882}%
\pgfsetfillcolor{currentfill}%
\pgfsetfillopacity{0.702286}%
\pgfsetlinewidth{1.003750pt}%
\definecolor{currentstroke}{rgb}{0.121569,0.466667,0.705882}%
\pgfsetstrokecolor{currentstroke}%
\pgfsetstrokeopacity{0.702286}%
\pgfsetdash{}{0pt}%
\pgfpathmoveto{\pgfqpoint{0.690377in}{1.594074in}}%
\pgfpathcurveto{\pgfqpoint{0.698614in}{1.594074in}}{\pgfqpoint{0.706514in}{1.597346in}}{\pgfqpoint{0.712338in}{1.603170in}}%
\pgfpathcurveto{\pgfqpoint{0.718162in}{1.608994in}}{\pgfqpoint{0.721434in}{1.616894in}}{\pgfqpoint{0.721434in}{1.625130in}}%
\pgfpathcurveto{\pgfqpoint{0.721434in}{1.633367in}}{\pgfqpoint{0.718162in}{1.641267in}}{\pgfqpoint{0.712338in}{1.647091in}}%
\pgfpathcurveto{\pgfqpoint{0.706514in}{1.652914in}}{\pgfqpoint{0.698614in}{1.656187in}}{\pgfqpoint{0.690377in}{1.656187in}}%
\pgfpathcurveto{\pgfqpoint{0.682141in}{1.656187in}}{\pgfqpoint{0.674241in}{1.652914in}}{\pgfqpoint{0.668417in}{1.647091in}}%
\pgfpathcurveto{\pgfqpoint{0.662593in}{1.641267in}}{\pgfqpoint{0.659321in}{1.633367in}}{\pgfqpoint{0.659321in}{1.625130in}}%
\pgfpathcurveto{\pgfqpoint{0.659321in}{1.616894in}}{\pgfqpoint{0.662593in}{1.608994in}}{\pgfqpoint{0.668417in}{1.603170in}}%
\pgfpathcurveto{\pgfqpoint{0.674241in}{1.597346in}}{\pgfqpoint{0.682141in}{1.594074in}}{\pgfqpoint{0.690377in}{1.594074in}}%
\pgfpathclose%
\pgfusepath{stroke,fill}%
\end{pgfscope}%
\begin{pgfscope}%
\pgfpathrectangle{\pgfqpoint{0.100000in}{0.212622in}}{\pgfqpoint{3.696000in}{3.696000in}}%
\pgfusepath{clip}%
\pgfsetbuttcap%
\pgfsetroundjoin%
\definecolor{currentfill}{rgb}{0.121569,0.466667,0.705882}%
\pgfsetfillcolor{currentfill}%
\pgfsetfillopacity{0.702288}%
\pgfsetlinewidth{1.003750pt}%
\definecolor{currentstroke}{rgb}{0.121569,0.466667,0.705882}%
\pgfsetstrokecolor{currentstroke}%
\pgfsetstrokeopacity{0.702288}%
\pgfsetdash{}{0pt}%
\pgfpathmoveto{\pgfqpoint{0.690371in}{1.594067in}}%
\pgfpathcurveto{\pgfqpoint{0.698607in}{1.594067in}}{\pgfqpoint{0.706507in}{1.597340in}}{\pgfqpoint{0.712331in}{1.603164in}}%
\pgfpathcurveto{\pgfqpoint{0.718155in}{1.608988in}}{\pgfqpoint{0.721428in}{1.616888in}}{\pgfqpoint{0.721428in}{1.625124in}}%
\pgfpathcurveto{\pgfqpoint{0.721428in}{1.633360in}}{\pgfqpoint{0.718155in}{1.641260in}}{\pgfqpoint{0.712331in}{1.647084in}}%
\pgfpathcurveto{\pgfqpoint{0.706507in}{1.652908in}}{\pgfqpoint{0.698607in}{1.656180in}}{\pgfqpoint{0.690371in}{1.656180in}}%
\pgfpathcurveto{\pgfqpoint{0.682135in}{1.656180in}}{\pgfqpoint{0.674235in}{1.652908in}}{\pgfqpoint{0.668411in}{1.647084in}}%
\pgfpathcurveto{\pgfqpoint{0.662587in}{1.641260in}}{\pgfqpoint{0.659315in}{1.633360in}}{\pgfqpoint{0.659315in}{1.625124in}}%
\pgfpathcurveto{\pgfqpoint{0.659315in}{1.616888in}}{\pgfqpoint{0.662587in}{1.608988in}}{\pgfqpoint{0.668411in}{1.603164in}}%
\pgfpathcurveto{\pgfqpoint{0.674235in}{1.597340in}}{\pgfqpoint{0.682135in}{1.594067in}}{\pgfqpoint{0.690371in}{1.594067in}}%
\pgfpathclose%
\pgfusepath{stroke,fill}%
\end{pgfscope}%
\begin{pgfscope}%
\pgfpathrectangle{\pgfqpoint{0.100000in}{0.212622in}}{\pgfqpoint{3.696000in}{3.696000in}}%
\pgfusepath{clip}%
\pgfsetbuttcap%
\pgfsetroundjoin%
\definecolor{currentfill}{rgb}{0.121569,0.466667,0.705882}%
\pgfsetfillcolor{currentfill}%
\pgfsetfillopacity{0.702292}%
\pgfsetlinewidth{1.003750pt}%
\definecolor{currentstroke}{rgb}{0.121569,0.466667,0.705882}%
\pgfsetstrokecolor{currentstroke}%
\pgfsetstrokeopacity{0.702292}%
\pgfsetdash{}{0pt}%
\pgfpathmoveto{\pgfqpoint{0.690364in}{1.594059in}}%
\pgfpathcurveto{\pgfqpoint{0.698600in}{1.594059in}}{\pgfqpoint{0.706500in}{1.597331in}}{\pgfqpoint{0.712324in}{1.603155in}}%
\pgfpathcurveto{\pgfqpoint{0.718148in}{1.608979in}}{\pgfqpoint{0.721420in}{1.616879in}}{\pgfqpoint{0.721420in}{1.625115in}}%
\pgfpathcurveto{\pgfqpoint{0.721420in}{1.633351in}}{\pgfqpoint{0.718148in}{1.641251in}}{\pgfqpoint{0.712324in}{1.647075in}}%
\pgfpathcurveto{\pgfqpoint{0.706500in}{1.652899in}}{\pgfqpoint{0.698600in}{1.656172in}}{\pgfqpoint{0.690364in}{1.656172in}}%
\pgfpathcurveto{\pgfqpoint{0.682128in}{1.656172in}}{\pgfqpoint{0.674227in}{1.652899in}}{\pgfqpoint{0.668404in}{1.647075in}}%
\pgfpathcurveto{\pgfqpoint{0.662580in}{1.641251in}}{\pgfqpoint{0.659307in}{1.633351in}}{\pgfqpoint{0.659307in}{1.625115in}}%
\pgfpathcurveto{\pgfqpoint{0.659307in}{1.616879in}}{\pgfqpoint{0.662580in}{1.608979in}}{\pgfqpoint{0.668404in}{1.603155in}}%
\pgfpathcurveto{\pgfqpoint{0.674227in}{1.597331in}}{\pgfqpoint{0.682128in}{1.594059in}}{\pgfqpoint{0.690364in}{1.594059in}}%
\pgfpathclose%
\pgfusepath{stroke,fill}%
\end{pgfscope}%
\begin{pgfscope}%
\pgfpathrectangle{\pgfqpoint{0.100000in}{0.212622in}}{\pgfqpoint{3.696000in}{3.696000in}}%
\pgfusepath{clip}%
\pgfsetbuttcap%
\pgfsetroundjoin%
\definecolor{currentfill}{rgb}{0.121569,0.466667,0.705882}%
\pgfsetfillcolor{currentfill}%
\pgfsetfillopacity{0.702299}%
\pgfsetlinewidth{1.003750pt}%
\definecolor{currentstroke}{rgb}{0.121569,0.466667,0.705882}%
\pgfsetstrokecolor{currentstroke}%
\pgfsetstrokeopacity{0.702299}%
\pgfsetdash{}{0pt}%
\pgfpathmoveto{\pgfqpoint{0.690342in}{1.594040in}}%
\pgfpathcurveto{\pgfqpoint{0.698578in}{1.594040in}}{\pgfqpoint{0.706478in}{1.597312in}}{\pgfqpoint{0.712302in}{1.603136in}}%
\pgfpathcurveto{\pgfqpoint{0.718126in}{1.608960in}}{\pgfqpoint{0.721398in}{1.616860in}}{\pgfqpoint{0.721398in}{1.625096in}}%
\pgfpathcurveto{\pgfqpoint{0.721398in}{1.633333in}}{\pgfqpoint{0.718126in}{1.641233in}}{\pgfqpoint{0.712302in}{1.647057in}}%
\pgfpathcurveto{\pgfqpoint{0.706478in}{1.652881in}}{\pgfqpoint{0.698578in}{1.656153in}}{\pgfqpoint{0.690342in}{1.656153in}}%
\pgfpathcurveto{\pgfqpoint{0.682105in}{1.656153in}}{\pgfqpoint{0.674205in}{1.652881in}}{\pgfqpoint{0.668381in}{1.647057in}}%
\pgfpathcurveto{\pgfqpoint{0.662558in}{1.641233in}}{\pgfqpoint{0.659285in}{1.633333in}}{\pgfqpoint{0.659285in}{1.625096in}}%
\pgfpathcurveto{\pgfqpoint{0.659285in}{1.616860in}}{\pgfqpoint{0.662558in}{1.608960in}}{\pgfqpoint{0.668381in}{1.603136in}}%
\pgfpathcurveto{\pgfqpoint{0.674205in}{1.597312in}}{\pgfqpoint{0.682105in}{1.594040in}}{\pgfqpoint{0.690342in}{1.594040in}}%
\pgfpathclose%
\pgfusepath{stroke,fill}%
\end{pgfscope}%
\begin{pgfscope}%
\pgfpathrectangle{\pgfqpoint{0.100000in}{0.212622in}}{\pgfqpoint{3.696000in}{3.696000in}}%
\pgfusepath{clip}%
\pgfsetbuttcap%
\pgfsetroundjoin%
\definecolor{currentfill}{rgb}{0.121569,0.466667,0.705882}%
\pgfsetfillcolor{currentfill}%
\pgfsetfillopacity{0.702312}%
\pgfsetlinewidth{1.003750pt}%
\definecolor{currentstroke}{rgb}{0.121569,0.466667,0.705882}%
\pgfsetstrokecolor{currentstroke}%
\pgfsetstrokeopacity{0.702312}%
\pgfsetdash{}{0pt}%
\pgfpathmoveto{\pgfqpoint{0.690317in}{1.594004in}}%
\pgfpathcurveto{\pgfqpoint{0.698553in}{1.594004in}}{\pgfqpoint{0.706453in}{1.597277in}}{\pgfqpoint{0.712277in}{1.603101in}}%
\pgfpathcurveto{\pgfqpoint{0.718101in}{1.608925in}}{\pgfqpoint{0.721373in}{1.616825in}}{\pgfqpoint{0.721373in}{1.625061in}}%
\pgfpathcurveto{\pgfqpoint{0.721373in}{1.633297in}}{\pgfqpoint{0.718101in}{1.641197in}}{\pgfqpoint{0.712277in}{1.647021in}}%
\pgfpathcurveto{\pgfqpoint{0.706453in}{1.652845in}}{\pgfqpoint{0.698553in}{1.656117in}}{\pgfqpoint{0.690317in}{1.656117in}}%
\pgfpathcurveto{\pgfqpoint{0.682080in}{1.656117in}}{\pgfqpoint{0.674180in}{1.652845in}}{\pgfqpoint{0.668356in}{1.647021in}}%
\pgfpathcurveto{\pgfqpoint{0.662532in}{1.641197in}}{\pgfqpoint{0.659260in}{1.633297in}}{\pgfqpoint{0.659260in}{1.625061in}}%
\pgfpathcurveto{\pgfqpoint{0.659260in}{1.616825in}}{\pgfqpoint{0.662532in}{1.608925in}}{\pgfqpoint{0.668356in}{1.603101in}}%
\pgfpathcurveto{\pgfqpoint{0.674180in}{1.597277in}}{\pgfqpoint{0.682080in}{1.594004in}}{\pgfqpoint{0.690317in}{1.594004in}}%
\pgfpathclose%
\pgfusepath{stroke,fill}%
\end{pgfscope}%
\begin{pgfscope}%
\pgfpathrectangle{\pgfqpoint{0.100000in}{0.212622in}}{\pgfqpoint{3.696000in}{3.696000in}}%
\pgfusepath{clip}%
\pgfsetbuttcap%
\pgfsetroundjoin%
\definecolor{currentfill}{rgb}{0.121569,0.466667,0.705882}%
\pgfsetfillcolor{currentfill}%
\pgfsetfillopacity{0.702334}%
\pgfsetlinewidth{1.003750pt}%
\definecolor{currentstroke}{rgb}{0.121569,0.466667,0.705882}%
\pgfsetstrokecolor{currentstroke}%
\pgfsetstrokeopacity{0.702334}%
\pgfsetdash{}{0pt}%
\pgfpathmoveto{\pgfqpoint{0.690248in}{1.593942in}}%
\pgfpathcurveto{\pgfqpoint{0.698484in}{1.593942in}}{\pgfqpoint{0.706384in}{1.597215in}}{\pgfqpoint{0.712208in}{1.603039in}}%
\pgfpathcurveto{\pgfqpoint{0.718032in}{1.608862in}}{\pgfqpoint{0.721304in}{1.616763in}}{\pgfqpoint{0.721304in}{1.624999in}}%
\pgfpathcurveto{\pgfqpoint{0.721304in}{1.633235in}}{\pgfqpoint{0.718032in}{1.641135in}}{\pgfqpoint{0.712208in}{1.646959in}}%
\pgfpathcurveto{\pgfqpoint{0.706384in}{1.652783in}}{\pgfqpoint{0.698484in}{1.656055in}}{\pgfqpoint{0.690248in}{1.656055in}}%
\pgfpathcurveto{\pgfqpoint{0.682011in}{1.656055in}}{\pgfqpoint{0.674111in}{1.652783in}}{\pgfqpoint{0.668287in}{1.646959in}}%
\pgfpathcurveto{\pgfqpoint{0.662464in}{1.641135in}}{\pgfqpoint{0.659191in}{1.633235in}}{\pgfqpoint{0.659191in}{1.624999in}}%
\pgfpathcurveto{\pgfqpoint{0.659191in}{1.616763in}}{\pgfqpoint{0.662464in}{1.608862in}}{\pgfqpoint{0.668287in}{1.603039in}}%
\pgfpathcurveto{\pgfqpoint{0.674111in}{1.597215in}}{\pgfqpoint{0.682011in}{1.593942in}}{\pgfqpoint{0.690248in}{1.593942in}}%
\pgfpathclose%
\pgfusepath{stroke,fill}%
\end{pgfscope}%
\begin{pgfscope}%
\pgfpathrectangle{\pgfqpoint{0.100000in}{0.212622in}}{\pgfqpoint{3.696000in}{3.696000in}}%
\pgfusepath{clip}%
\pgfsetbuttcap%
\pgfsetroundjoin%
\definecolor{currentfill}{rgb}{0.121569,0.466667,0.705882}%
\pgfsetfillcolor{currentfill}%
\pgfsetfillopacity{0.702380}%
\pgfsetlinewidth{1.003750pt}%
\definecolor{currentstroke}{rgb}{0.121569,0.466667,0.705882}%
\pgfsetstrokecolor{currentstroke}%
\pgfsetstrokeopacity{0.702380}%
\pgfsetdash{}{0pt}%
\pgfpathmoveto{\pgfqpoint{0.690168in}{1.593831in}}%
\pgfpathcurveto{\pgfqpoint{0.698404in}{1.593831in}}{\pgfqpoint{0.706304in}{1.597103in}}{\pgfqpoint{0.712128in}{1.602927in}}%
\pgfpathcurveto{\pgfqpoint{0.717952in}{1.608751in}}{\pgfqpoint{0.721225in}{1.616651in}}{\pgfqpoint{0.721225in}{1.624887in}}%
\pgfpathcurveto{\pgfqpoint{0.721225in}{1.633124in}}{\pgfqpoint{0.717952in}{1.641024in}}{\pgfqpoint{0.712128in}{1.646848in}}%
\pgfpathcurveto{\pgfqpoint{0.706304in}{1.652671in}}{\pgfqpoint{0.698404in}{1.655944in}}{\pgfqpoint{0.690168in}{1.655944in}}%
\pgfpathcurveto{\pgfqpoint{0.681932in}{1.655944in}}{\pgfqpoint{0.674032in}{1.652671in}}{\pgfqpoint{0.668208in}{1.646848in}}%
\pgfpathcurveto{\pgfqpoint{0.662384in}{1.641024in}}{\pgfqpoint{0.659112in}{1.633124in}}{\pgfqpoint{0.659112in}{1.624887in}}%
\pgfpathcurveto{\pgfqpoint{0.659112in}{1.616651in}}{\pgfqpoint{0.662384in}{1.608751in}}{\pgfqpoint{0.668208in}{1.602927in}}%
\pgfpathcurveto{\pgfqpoint{0.674032in}{1.597103in}}{\pgfqpoint{0.681932in}{1.593831in}}{\pgfqpoint{0.690168in}{1.593831in}}%
\pgfpathclose%
\pgfusepath{stroke,fill}%
\end{pgfscope}%
\begin{pgfscope}%
\pgfpathrectangle{\pgfqpoint{0.100000in}{0.212622in}}{\pgfqpoint{3.696000in}{3.696000in}}%
\pgfusepath{clip}%
\pgfsetbuttcap%
\pgfsetroundjoin%
\definecolor{currentfill}{rgb}{0.121569,0.466667,0.705882}%
\pgfsetfillcolor{currentfill}%
\pgfsetfillopacity{0.702454}%
\pgfsetlinewidth{1.003750pt}%
\definecolor{currentstroke}{rgb}{0.121569,0.466667,0.705882}%
\pgfsetstrokecolor{currentstroke}%
\pgfsetstrokeopacity{0.702454}%
\pgfsetdash{}{0pt}%
\pgfpathmoveto{\pgfqpoint{0.689929in}{1.593651in}}%
\pgfpathcurveto{\pgfqpoint{0.698166in}{1.593651in}}{\pgfqpoint{0.706066in}{1.596923in}}{\pgfqpoint{0.711890in}{1.602747in}}%
\pgfpathcurveto{\pgfqpoint{0.717714in}{1.608571in}}{\pgfqpoint{0.720986in}{1.616471in}}{\pgfqpoint{0.720986in}{1.624707in}}%
\pgfpathcurveto{\pgfqpoint{0.720986in}{1.632943in}}{\pgfqpoint{0.717714in}{1.640844in}}{\pgfqpoint{0.711890in}{1.646667in}}%
\pgfpathcurveto{\pgfqpoint{0.706066in}{1.652491in}}{\pgfqpoint{0.698166in}{1.655764in}}{\pgfqpoint{0.689929in}{1.655764in}}%
\pgfpathcurveto{\pgfqpoint{0.681693in}{1.655764in}}{\pgfqpoint{0.673793in}{1.652491in}}{\pgfqpoint{0.667969in}{1.646667in}}%
\pgfpathcurveto{\pgfqpoint{0.662145in}{1.640844in}}{\pgfqpoint{0.658873in}{1.632943in}}{\pgfqpoint{0.658873in}{1.624707in}}%
\pgfpathcurveto{\pgfqpoint{0.658873in}{1.616471in}}{\pgfqpoint{0.662145in}{1.608571in}}{\pgfqpoint{0.667969in}{1.602747in}}%
\pgfpathcurveto{\pgfqpoint{0.673793in}{1.596923in}}{\pgfqpoint{0.681693in}{1.593651in}}{\pgfqpoint{0.689929in}{1.593651in}}%
\pgfpathclose%
\pgfusepath{stroke,fill}%
\end{pgfscope}%
\begin{pgfscope}%
\pgfpathrectangle{\pgfqpoint{0.100000in}{0.212622in}}{\pgfqpoint{3.696000in}{3.696000in}}%
\pgfusepath{clip}%
\pgfsetbuttcap%
\pgfsetroundjoin%
\definecolor{currentfill}{rgb}{0.121569,0.466667,0.705882}%
\pgfsetfillcolor{currentfill}%
\pgfsetfillopacity{0.702602}%
\pgfsetlinewidth{1.003750pt}%
\definecolor{currentstroke}{rgb}{0.121569,0.466667,0.705882}%
\pgfsetstrokecolor{currentstroke}%
\pgfsetstrokeopacity{0.702602}%
\pgfsetdash{}{0pt}%
\pgfpathmoveto{\pgfqpoint{0.689596in}{1.593306in}}%
\pgfpathcurveto{\pgfqpoint{0.697832in}{1.593306in}}{\pgfqpoint{0.705732in}{1.596578in}}{\pgfqpoint{0.711556in}{1.602402in}}%
\pgfpathcurveto{\pgfqpoint{0.717380in}{1.608226in}}{\pgfqpoint{0.720653in}{1.616126in}}{\pgfqpoint{0.720653in}{1.624363in}}%
\pgfpathcurveto{\pgfqpoint{0.720653in}{1.632599in}}{\pgfqpoint{0.717380in}{1.640499in}}{\pgfqpoint{0.711556in}{1.646323in}}%
\pgfpathcurveto{\pgfqpoint{0.705732in}{1.652147in}}{\pgfqpoint{0.697832in}{1.655419in}}{\pgfqpoint{0.689596in}{1.655419in}}%
\pgfpathcurveto{\pgfqpoint{0.681360in}{1.655419in}}{\pgfqpoint{0.673460in}{1.652147in}}{\pgfqpoint{0.667636in}{1.646323in}}%
\pgfpathcurveto{\pgfqpoint{0.661812in}{1.640499in}}{\pgfqpoint{0.658540in}{1.632599in}}{\pgfqpoint{0.658540in}{1.624363in}}%
\pgfpathcurveto{\pgfqpoint{0.658540in}{1.616126in}}{\pgfqpoint{0.661812in}{1.608226in}}{\pgfqpoint{0.667636in}{1.602402in}}%
\pgfpathcurveto{\pgfqpoint{0.673460in}{1.596578in}}{\pgfqpoint{0.681360in}{1.593306in}}{\pgfqpoint{0.689596in}{1.593306in}}%
\pgfpathclose%
\pgfusepath{stroke,fill}%
\end{pgfscope}%
\begin{pgfscope}%
\pgfpathrectangle{\pgfqpoint{0.100000in}{0.212622in}}{\pgfqpoint{3.696000in}{3.696000in}}%
\pgfusepath{clip}%
\pgfsetbuttcap%
\pgfsetroundjoin%
\definecolor{currentfill}{rgb}{0.121569,0.466667,0.705882}%
\pgfsetfillcolor{currentfill}%
\pgfsetfillopacity{0.702660}%
\pgfsetlinewidth{1.003750pt}%
\definecolor{currentstroke}{rgb}{0.121569,0.466667,0.705882}%
\pgfsetstrokecolor{currentstroke}%
\pgfsetstrokeopacity{0.702660}%
\pgfsetdash{}{0pt}%
\pgfpathmoveto{\pgfqpoint{0.689400in}{1.593163in}}%
\pgfpathcurveto{\pgfqpoint{0.697636in}{1.593163in}}{\pgfqpoint{0.705536in}{1.596436in}}{\pgfqpoint{0.711360in}{1.602260in}}%
\pgfpathcurveto{\pgfqpoint{0.717184in}{1.608084in}}{\pgfqpoint{0.720456in}{1.615984in}}{\pgfqpoint{0.720456in}{1.624220in}}%
\pgfpathcurveto{\pgfqpoint{0.720456in}{1.632456in}}{\pgfqpoint{0.717184in}{1.640356in}}{\pgfqpoint{0.711360in}{1.646180in}}%
\pgfpathcurveto{\pgfqpoint{0.705536in}{1.652004in}}{\pgfqpoint{0.697636in}{1.655276in}}{\pgfqpoint{0.689400in}{1.655276in}}%
\pgfpathcurveto{\pgfqpoint{0.681163in}{1.655276in}}{\pgfqpoint{0.673263in}{1.652004in}}{\pgfqpoint{0.667439in}{1.646180in}}%
\pgfpathcurveto{\pgfqpoint{0.661615in}{1.640356in}}{\pgfqpoint{0.658343in}{1.632456in}}{\pgfqpoint{0.658343in}{1.624220in}}%
\pgfpathcurveto{\pgfqpoint{0.658343in}{1.615984in}}{\pgfqpoint{0.661615in}{1.608084in}}{\pgfqpoint{0.667439in}{1.602260in}}%
\pgfpathcurveto{\pgfqpoint{0.673263in}{1.596436in}}{\pgfqpoint{0.681163in}{1.593163in}}{\pgfqpoint{0.689400in}{1.593163in}}%
\pgfpathclose%
\pgfusepath{stroke,fill}%
\end{pgfscope}%
\begin{pgfscope}%
\pgfpathrectangle{\pgfqpoint{0.100000in}{0.212622in}}{\pgfqpoint{3.696000in}{3.696000in}}%
\pgfusepath{clip}%
\pgfsetbuttcap%
\pgfsetroundjoin%
\definecolor{currentfill}{rgb}{0.121569,0.466667,0.705882}%
\pgfsetfillcolor{currentfill}%
\pgfsetfillopacity{0.702771}%
\pgfsetlinewidth{1.003750pt}%
\definecolor{currentstroke}{rgb}{0.121569,0.466667,0.705882}%
\pgfsetstrokecolor{currentstroke}%
\pgfsetstrokeopacity{0.702771}%
\pgfsetdash{}{0pt}%
\pgfpathmoveto{\pgfqpoint{0.689088in}{1.592880in}}%
\pgfpathcurveto{\pgfqpoint{0.697324in}{1.592880in}}{\pgfqpoint{0.705224in}{1.596153in}}{\pgfqpoint{0.711048in}{1.601977in}}%
\pgfpathcurveto{\pgfqpoint{0.716872in}{1.607801in}}{\pgfqpoint{0.720144in}{1.615701in}}{\pgfqpoint{0.720144in}{1.623937in}}%
\pgfpathcurveto{\pgfqpoint{0.720144in}{1.632173in}}{\pgfqpoint{0.716872in}{1.640073in}}{\pgfqpoint{0.711048in}{1.645897in}}%
\pgfpathcurveto{\pgfqpoint{0.705224in}{1.651721in}}{\pgfqpoint{0.697324in}{1.654993in}}{\pgfqpoint{0.689088in}{1.654993in}}%
\pgfpathcurveto{\pgfqpoint{0.680851in}{1.654993in}}{\pgfqpoint{0.672951in}{1.651721in}}{\pgfqpoint{0.667127in}{1.645897in}}%
\pgfpathcurveto{\pgfqpoint{0.661304in}{1.640073in}}{\pgfqpoint{0.658031in}{1.632173in}}{\pgfqpoint{0.658031in}{1.623937in}}%
\pgfpathcurveto{\pgfqpoint{0.658031in}{1.615701in}}{\pgfqpoint{0.661304in}{1.607801in}}{\pgfqpoint{0.667127in}{1.601977in}}%
\pgfpathcurveto{\pgfqpoint{0.672951in}{1.596153in}}{\pgfqpoint{0.680851in}{1.592880in}}{\pgfqpoint{0.689088in}{1.592880in}}%
\pgfpathclose%
\pgfusepath{stroke,fill}%
\end{pgfscope}%
\begin{pgfscope}%
\pgfpathrectangle{\pgfqpoint{0.100000in}{0.212622in}}{\pgfqpoint{3.696000in}{3.696000in}}%
\pgfusepath{clip}%
\pgfsetbuttcap%
\pgfsetroundjoin%
\definecolor{currentfill}{rgb}{0.121569,0.466667,0.705882}%
\pgfsetfillcolor{currentfill}%
\pgfsetfillopacity{0.702963}%
\pgfsetlinewidth{1.003750pt}%
\definecolor{currentstroke}{rgb}{0.121569,0.466667,0.705882}%
\pgfsetstrokecolor{currentstroke}%
\pgfsetstrokeopacity{0.702963}%
\pgfsetdash{}{0pt}%
\pgfpathmoveto{\pgfqpoint{0.688415in}{1.592425in}}%
\pgfpathcurveto{\pgfqpoint{0.696652in}{1.592425in}}{\pgfqpoint{0.704552in}{1.595698in}}{\pgfqpoint{0.710376in}{1.601522in}}%
\pgfpathcurveto{\pgfqpoint{0.716199in}{1.607346in}}{\pgfqpoint{0.719472in}{1.615246in}}{\pgfqpoint{0.719472in}{1.623482in}}%
\pgfpathcurveto{\pgfqpoint{0.719472in}{1.631718in}}{\pgfqpoint{0.716199in}{1.639618in}}{\pgfqpoint{0.710376in}{1.645442in}}%
\pgfpathcurveto{\pgfqpoint{0.704552in}{1.651266in}}{\pgfqpoint{0.696652in}{1.654538in}}{\pgfqpoint{0.688415in}{1.654538in}}%
\pgfpathcurveto{\pgfqpoint{0.680179in}{1.654538in}}{\pgfqpoint{0.672279in}{1.651266in}}{\pgfqpoint{0.666455in}{1.645442in}}%
\pgfpathcurveto{\pgfqpoint{0.660631in}{1.639618in}}{\pgfqpoint{0.657359in}{1.631718in}}{\pgfqpoint{0.657359in}{1.623482in}}%
\pgfpathcurveto{\pgfqpoint{0.657359in}{1.615246in}}{\pgfqpoint{0.660631in}{1.607346in}}{\pgfqpoint{0.666455in}{1.601522in}}%
\pgfpathcurveto{\pgfqpoint{0.672279in}{1.595698in}}{\pgfqpoint{0.680179in}{1.592425in}}{\pgfqpoint{0.688415in}{1.592425in}}%
\pgfpathclose%
\pgfusepath{stroke,fill}%
\end{pgfscope}%
\begin{pgfscope}%
\pgfpathrectangle{\pgfqpoint{0.100000in}{0.212622in}}{\pgfqpoint{3.696000in}{3.696000in}}%
\pgfusepath{clip}%
\pgfsetbuttcap%
\pgfsetroundjoin%
\definecolor{currentfill}{rgb}{0.121569,0.466667,0.705882}%
\pgfsetfillcolor{currentfill}%
\pgfsetfillopacity{0.703387}%
\pgfsetlinewidth{1.003750pt}%
\definecolor{currentstroke}{rgb}{0.121569,0.466667,0.705882}%
\pgfsetstrokecolor{currentstroke}%
\pgfsetstrokeopacity{0.703387}%
\pgfsetdash{}{0pt}%
\pgfpathmoveto{\pgfqpoint{0.687455in}{1.591791in}}%
\pgfpathcurveto{\pgfqpoint{0.695691in}{1.591791in}}{\pgfqpoint{0.703591in}{1.595063in}}{\pgfqpoint{0.709415in}{1.600887in}}%
\pgfpathcurveto{\pgfqpoint{0.715239in}{1.606711in}}{\pgfqpoint{0.718511in}{1.614611in}}{\pgfqpoint{0.718511in}{1.622847in}}%
\pgfpathcurveto{\pgfqpoint{0.718511in}{1.631083in}}{\pgfqpoint{0.715239in}{1.638983in}}{\pgfqpoint{0.709415in}{1.644807in}}%
\pgfpathcurveto{\pgfqpoint{0.703591in}{1.650631in}}{\pgfqpoint{0.695691in}{1.653904in}}{\pgfqpoint{0.687455in}{1.653904in}}%
\pgfpathcurveto{\pgfqpoint{0.679219in}{1.653904in}}{\pgfqpoint{0.671318in}{1.650631in}}{\pgfqpoint{0.665495in}{1.644807in}}%
\pgfpathcurveto{\pgfqpoint{0.659671in}{1.638983in}}{\pgfqpoint{0.656398in}{1.631083in}}{\pgfqpoint{0.656398in}{1.622847in}}%
\pgfpathcurveto{\pgfqpoint{0.656398in}{1.614611in}}{\pgfqpoint{0.659671in}{1.606711in}}{\pgfqpoint{0.665495in}{1.600887in}}%
\pgfpathcurveto{\pgfqpoint{0.671318in}{1.595063in}}{\pgfqpoint{0.679219in}{1.591791in}}{\pgfqpoint{0.687455in}{1.591791in}}%
\pgfpathclose%
\pgfusepath{stroke,fill}%
\end{pgfscope}%
\begin{pgfscope}%
\pgfpathrectangle{\pgfqpoint{0.100000in}{0.212622in}}{\pgfqpoint{3.696000in}{3.696000in}}%
\pgfusepath{clip}%
\pgfsetbuttcap%
\pgfsetroundjoin%
\definecolor{currentfill}{rgb}{0.121569,0.466667,0.705882}%
\pgfsetfillcolor{currentfill}%
\pgfsetfillopacity{0.703669}%
\pgfsetlinewidth{1.003750pt}%
\definecolor{currentstroke}{rgb}{0.121569,0.466667,0.705882}%
\pgfsetstrokecolor{currentstroke}%
\pgfsetstrokeopacity{0.703669}%
\pgfsetdash{}{0pt}%
\pgfpathmoveto{\pgfqpoint{0.686441in}{1.591100in}}%
\pgfpathcurveto{\pgfqpoint{0.694677in}{1.591100in}}{\pgfqpoint{0.702577in}{1.594373in}}{\pgfqpoint{0.708401in}{1.600197in}}%
\pgfpathcurveto{\pgfqpoint{0.714225in}{1.606021in}}{\pgfqpoint{0.717497in}{1.613921in}}{\pgfqpoint{0.717497in}{1.622157in}}%
\pgfpathcurveto{\pgfqpoint{0.717497in}{1.630393in}}{\pgfqpoint{0.714225in}{1.638293in}}{\pgfqpoint{0.708401in}{1.644117in}}%
\pgfpathcurveto{\pgfqpoint{0.702577in}{1.649941in}}{\pgfqpoint{0.694677in}{1.653213in}}{\pgfqpoint{0.686441in}{1.653213in}}%
\pgfpathcurveto{\pgfqpoint{0.678204in}{1.653213in}}{\pgfqpoint{0.670304in}{1.649941in}}{\pgfqpoint{0.664480in}{1.644117in}}%
\pgfpathcurveto{\pgfqpoint{0.658656in}{1.638293in}}{\pgfqpoint{0.655384in}{1.630393in}}{\pgfqpoint{0.655384in}{1.622157in}}%
\pgfpathcurveto{\pgfqpoint{0.655384in}{1.613921in}}{\pgfqpoint{0.658656in}{1.606021in}}{\pgfqpoint{0.664480in}{1.600197in}}%
\pgfpathcurveto{\pgfqpoint{0.670304in}{1.594373in}}{\pgfqpoint{0.678204in}{1.591100in}}{\pgfqpoint{0.686441in}{1.591100in}}%
\pgfpathclose%
\pgfusepath{stroke,fill}%
\end{pgfscope}%
\begin{pgfscope}%
\pgfpathrectangle{\pgfqpoint{0.100000in}{0.212622in}}{\pgfqpoint{3.696000in}{3.696000in}}%
\pgfusepath{clip}%
\pgfsetbuttcap%
\pgfsetroundjoin%
\definecolor{currentfill}{rgb}{0.121569,0.466667,0.705882}%
\pgfsetfillcolor{currentfill}%
\pgfsetfillopacity{0.703908}%
\pgfsetlinewidth{1.003750pt}%
\definecolor{currentstroke}{rgb}{0.121569,0.466667,0.705882}%
\pgfsetstrokecolor{currentstroke}%
\pgfsetstrokeopacity{0.703908}%
\pgfsetdash{}{0pt}%
\pgfpathmoveto{\pgfqpoint{0.685891in}{1.590719in}}%
\pgfpathcurveto{\pgfqpoint{0.694128in}{1.590719in}}{\pgfqpoint{0.702028in}{1.593991in}}{\pgfqpoint{0.707852in}{1.599815in}}%
\pgfpathcurveto{\pgfqpoint{0.713676in}{1.605639in}}{\pgfqpoint{0.716948in}{1.613539in}}{\pgfqpoint{0.716948in}{1.621775in}}%
\pgfpathcurveto{\pgfqpoint{0.716948in}{1.630011in}}{\pgfqpoint{0.713676in}{1.637911in}}{\pgfqpoint{0.707852in}{1.643735in}}%
\pgfpathcurveto{\pgfqpoint{0.702028in}{1.649559in}}{\pgfqpoint{0.694128in}{1.652832in}}{\pgfqpoint{0.685891in}{1.652832in}}%
\pgfpathcurveto{\pgfqpoint{0.677655in}{1.652832in}}{\pgfqpoint{0.669755in}{1.649559in}}{\pgfqpoint{0.663931in}{1.643735in}}%
\pgfpathcurveto{\pgfqpoint{0.658107in}{1.637911in}}{\pgfqpoint{0.654835in}{1.630011in}}{\pgfqpoint{0.654835in}{1.621775in}}%
\pgfpathcurveto{\pgfqpoint{0.654835in}{1.613539in}}{\pgfqpoint{0.658107in}{1.605639in}}{\pgfqpoint{0.663931in}{1.599815in}}%
\pgfpathcurveto{\pgfqpoint{0.669755in}{1.593991in}}{\pgfqpoint{0.677655in}{1.590719in}}{\pgfqpoint{0.685891in}{1.590719in}}%
\pgfpathclose%
\pgfusepath{stroke,fill}%
\end{pgfscope}%
\begin{pgfscope}%
\pgfpathrectangle{\pgfqpoint{0.100000in}{0.212622in}}{\pgfqpoint{3.696000in}{3.696000in}}%
\pgfusepath{clip}%
\pgfsetbuttcap%
\pgfsetroundjoin%
\definecolor{currentfill}{rgb}{0.121569,0.466667,0.705882}%
\pgfsetfillcolor{currentfill}%
\pgfsetfillopacity{0.704267}%
\pgfsetlinewidth{1.003750pt}%
\definecolor{currentstroke}{rgb}{0.121569,0.466667,0.705882}%
\pgfsetstrokecolor{currentstroke}%
\pgfsetstrokeopacity{0.704267}%
\pgfsetdash{}{0pt}%
\pgfpathmoveto{\pgfqpoint{0.684735in}{1.589707in}}%
\pgfpathcurveto{\pgfqpoint{0.692971in}{1.589707in}}{\pgfqpoint{0.700871in}{1.592980in}}{\pgfqpoint{0.706695in}{1.598804in}}%
\pgfpathcurveto{\pgfqpoint{0.712519in}{1.604627in}}{\pgfqpoint{0.715791in}{1.612528in}}{\pgfqpoint{0.715791in}{1.620764in}}%
\pgfpathcurveto{\pgfqpoint{0.715791in}{1.629000in}}{\pgfqpoint{0.712519in}{1.636900in}}{\pgfqpoint{0.706695in}{1.642724in}}%
\pgfpathcurveto{\pgfqpoint{0.700871in}{1.648548in}}{\pgfqpoint{0.692971in}{1.651820in}}{\pgfqpoint{0.684735in}{1.651820in}}%
\pgfpathcurveto{\pgfqpoint{0.676498in}{1.651820in}}{\pgfqpoint{0.668598in}{1.648548in}}{\pgfqpoint{0.662774in}{1.642724in}}%
\pgfpathcurveto{\pgfqpoint{0.656950in}{1.636900in}}{\pgfqpoint{0.653678in}{1.629000in}}{\pgfqpoint{0.653678in}{1.620764in}}%
\pgfpathcurveto{\pgfqpoint{0.653678in}{1.612528in}}{\pgfqpoint{0.656950in}{1.604627in}}{\pgfqpoint{0.662774in}{1.598804in}}%
\pgfpathcurveto{\pgfqpoint{0.668598in}{1.592980in}}{\pgfqpoint{0.676498in}{1.589707in}}{\pgfqpoint{0.684735in}{1.589707in}}%
\pgfpathclose%
\pgfusepath{stroke,fill}%
\end{pgfscope}%
\begin{pgfscope}%
\pgfpathrectangle{\pgfqpoint{0.100000in}{0.212622in}}{\pgfqpoint{3.696000in}{3.696000in}}%
\pgfusepath{clip}%
\pgfsetbuttcap%
\pgfsetroundjoin%
\definecolor{currentfill}{rgb}{0.121569,0.466667,0.705882}%
\pgfsetfillcolor{currentfill}%
\pgfsetfillopacity{0.705083}%
\pgfsetlinewidth{1.003750pt}%
\definecolor{currentstroke}{rgb}{0.121569,0.466667,0.705882}%
\pgfsetstrokecolor{currentstroke}%
\pgfsetstrokeopacity{0.705083}%
\pgfsetdash{}{0pt}%
\pgfpathmoveto{\pgfqpoint{0.683327in}{1.588295in}}%
\pgfpathcurveto{\pgfqpoint{0.691563in}{1.588295in}}{\pgfqpoint{0.699463in}{1.591567in}}{\pgfqpoint{0.705287in}{1.597391in}}%
\pgfpathcurveto{\pgfqpoint{0.711111in}{1.603215in}}{\pgfqpoint{0.714383in}{1.611115in}}{\pgfqpoint{0.714383in}{1.619351in}}%
\pgfpathcurveto{\pgfqpoint{0.714383in}{1.627587in}}{\pgfqpoint{0.711111in}{1.635487in}}{\pgfqpoint{0.705287in}{1.641311in}}%
\pgfpathcurveto{\pgfqpoint{0.699463in}{1.647135in}}{\pgfqpoint{0.691563in}{1.650408in}}{\pgfqpoint{0.683327in}{1.650408in}}%
\pgfpathcurveto{\pgfqpoint{0.675091in}{1.650408in}}{\pgfqpoint{0.667191in}{1.647135in}}{\pgfqpoint{0.661367in}{1.641311in}}%
\pgfpathcurveto{\pgfqpoint{0.655543in}{1.635487in}}{\pgfqpoint{0.652270in}{1.627587in}}{\pgfqpoint{0.652270in}{1.619351in}}%
\pgfpathcurveto{\pgfqpoint{0.652270in}{1.611115in}}{\pgfqpoint{0.655543in}{1.603215in}}{\pgfqpoint{0.661367in}{1.597391in}}%
\pgfpathcurveto{\pgfqpoint{0.667191in}{1.591567in}}{\pgfqpoint{0.675091in}{1.588295in}}{\pgfqpoint{0.683327in}{1.588295in}}%
\pgfpathclose%
\pgfusepath{stroke,fill}%
\end{pgfscope}%
\begin{pgfscope}%
\pgfpathrectangle{\pgfqpoint{0.100000in}{0.212622in}}{\pgfqpoint{3.696000in}{3.696000in}}%
\pgfusepath{clip}%
\pgfsetbuttcap%
\pgfsetroundjoin%
\definecolor{currentfill}{rgb}{0.121569,0.466667,0.705882}%
\pgfsetfillcolor{currentfill}%
\pgfsetfillopacity{0.706147}%
\pgfsetlinewidth{1.003750pt}%
\definecolor{currentstroke}{rgb}{0.121569,0.466667,0.705882}%
\pgfsetstrokecolor{currentstroke}%
\pgfsetstrokeopacity{0.706147}%
\pgfsetdash{}{0pt}%
\pgfpathmoveto{\pgfqpoint{0.679537in}{1.584090in}}%
\pgfpathcurveto{\pgfqpoint{0.687773in}{1.584090in}}{\pgfqpoint{0.695673in}{1.587362in}}{\pgfqpoint{0.701497in}{1.593186in}}%
\pgfpathcurveto{\pgfqpoint{0.707321in}{1.599010in}}{\pgfqpoint{0.710593in}{1.606910in}}{\pgfqpoint{0.710593in}{1.615146in}}%
\pgfpathcurveto{\pgfqpoint{0.710593in}{1.623383in}}{\pgfqpoint{0.707321in}{1.631283in}}{\pgfqpoint{0.701497in}{1.637106in}}%
\pgfpathcurveto{\pgfqpoint{0.695673in}{1.642930in}}{\pgfqpoint{0.687773in}{1.646203in}}{\pgfqpoint{0.679537in}{1.646203in}}%
\pgfpathcurveto{\pgfqpoint{0.671300in}{1.646203in}}{\pgfqpoint{0.663400in}{1.642930in}}{\pgfqpoint{0.657576in}{1.637106in}}%
\pgfpathcurveto{\pgfqpoint{0.651752in}{1.631283in}}{\pgfqpoint{0.648480in}{1.623383in}}{\pgfqpoint{0.648480in}{1.615146in}}%
\pgfpathcurveto{\pgfqpoint{0.648480in}{1.606910in}}{\pgfqpoint{0.651752in}{1.599010in}}{\pgfqpoint{0.657576in}{1.593186in}}%
\pgfpathcurveto{\pgfqpoint{0.663400in}{1.587362in}}{\pgfqpoint{0.671300in}{1.584090in}}{\pgfqpoint{0.679537in}{1.584090in}}%
\pgfpathclose%
\pgfusepath{stroke,fill}%
\end{pgfscope}%
\begin{pgfscope}%
\pgfpathrectangle{\pgfqpoint{0.100000in}{0.212622in}}{\pgfqpoint{3.696000in}{3.696000in}}%
\pgfusepath{clip}%
\pgfsetbuttcap%
\pgfsetroundjoin%
\definecolor{currentfill}{rgb}{0.121569,0.466667,0.705882}%
\pgfsetfillcolor{currentfill}%
\pgfsetfillopacity{0.706258}%
\pgfsetlinewidth{1.003750pt}%
\definecolor{currentstroke}{rgb}{0.121569,0.466667,0.705882}%
\pgfsetstrokecolor{currentstroke}%
\pgfsetstrokeopacity{0.706258}%
\pgfsetdash{}{0pt}%
\pgfpathmoveto{\pgfqpoint{3.028276in}{1.884641in}}%
\pgfpathcurveto{\pgfqpoint{3.036512in}{1.884641in}}{\pgfqpoint{3.044412in}{1.887913in}}{\pgfqpoint{3.050236in}{1.893737in}}%
\pgfpathcurveto{\pgfqpoint{3.056060in}{1.899561in}}{\pgfqpoint{3.059332in}{1.907461in}}{\pgfqpoint{3.059332in}{1.915698in}}%
\pgfpathcurveto{\pgfqpoint{3.059332in}{1.923934in}}{\pgfqpoint{3.056060in}{1.931834in}}{\pgfqpoint{3.050236in}{1.937658in}}%
\pgfpathcurveto{\pgfqpoint{3.044412in}{1.943482in}}{\pgfqpoint{3.036512in}{1.946754in}}{\pgfqpoint{3.028276in}{1.946754in}}%
\pgfpathcurveto{\pgfqpoint{3.020039in}{1.946754in}}{\pgfqpoint{3.012139in}{1.943482in}}{\pgfqpoint{3.006315in}{1.937658in}}%
\pgfpathcurveto{\pgfqpoint{3.000492in}{1.931834in}}{\pgfqpoint{2.997219in}{1.923934in}}{\pgfqpoint{2.997219in}{1.915698in}}%
\pgfpathcurveto{\pgfqpoint{2.997219in}{1.907461in}}{\pgfqpoint{3.000492in}{1.899561in}}{\pgfqpoint{3.006315in}{1.893737in}}%
\pgfpathcurveto{\pgfqpoint{3.012139in}{1.887913in}}{\pgfqpoint{3.020039in}{1.884641in}}{\pgfqpoint{3.028276in}{1.884641in}}%
\pgfpathclose%
\pgfusepath{stroke,fill}%
\end{pgfscope}%
\begin{pgfscope}%
\pgfpathrectangle{\pgfqpoint{0.100000in}{0.212622in}}{\pgfqpoint{3.696000in}{3.696000in}}%
\pgfusepath{clip}%
\pgfsetbuttcap%
\pgfsetroundjoin%
\definecolor{currentfill}{rgb}{0.121569,0.466667,0.705882}%
\pgfsetfillcolor{currentfill}%
\pgfsetfillopacity{0.707716}%
\pgfsetlinewidth{1.003750pt}%
\definecolor{currentstroke}{rgb}{0.121569,0.466667,0.705882}%
\pgfsetstrokecolor{currentstroke}%
\pgfsetstrokeopacity{0.707716}%
\pgfsetdash{}{0pt}%
\pgfpathmoveto{\pgfqpoint{0.676773in}{1.584067in}}%
\pgfpathcurveto{\pgfqpoint{0.685010in}{1.584067in}}{\pgfqpoint{0.692910in}{1.587340in}}{\pgfqpoint{0.698734in}{1.593164in}}%
\pgfpathcurveto{\pgfqpoint{0.704558in}{1.598988in}}{\pgfqpoint{0.707830in}{1.606888in}}{\pgfqpoint{0.707830in}{1.615124in}}%
\pgfpathcurveto{\pgfqpoint{0.707830in}{1.623360in}}{\pgfqpoint{0.704558in}{1.631260in}}{\pgfqpoint{0.698734in}{1.637084in}}%
\pgfpathcurveto{\pgfqpoint{0.692910in}{1.642908in}}{\pgfqpoint{0.685010in}{1.646180in}}{\pgfqpoint{0.676773in}{1.646180in}}%
\pgfpathcurveto{\pgfqpoint{0.668537in}{1.646180in}}{\pgfqpoint{0.660637in}{1.642908in}}{\pgfqpoint{0.654813in}{1.637084in}}%
\pgfpathcurveto{\pgfqpoint{0.648989in}{1.631260in}}{\pgfqpoint{0.645717in}{1.623360in}}{\pgfqpoint{0.645717in}{1.615124in}}%
\pgfpathcurveto{\pgfqpoint{0.645717in}{1.606888in}}{\pgfqpoint{0.648989in}{1.598988in}}{\pgfqpoint{0.654813in}{1.593164in}}%
\pgfpathcurveto{\pgfqpoint{0.660637in}{1.587340in}}{\pgfqpoint{0.668537in}{1.584067in}}{\pgfqpoint{0.676773in}{1.584067in}}%
\pgfpathclose%
\pgfusepath{stroke,fill}%
\end{pgfscope}%
\begin{pgfscope}%
\pgfpathrectangle{\pgfqpoint{0.100000in}{0.212622in}}{\pgfqpoint{3.696000in}{3.696000in}}%
\pgfusepath{clip}%
\pgfsetbuttcap%
\pgfsetroundjoin%
\definecolor{currentfill}{rgb}{0.121569,0.466667,0.705882}%
\pgfsetfillcolor{currentfill}%
\pgfsetfillopacity{0.708108}%
\pgfsetlinewidth{1.003750pt}%
\definecolor{currentstroke}{rgb}{0.121569,0.466667,0.705882}%
\pgfsetstrokecolor{currentstroke}%
\pgfsetstrokeopacity{0.708108}%
\pgfsetdash{}{0pt}%
\pgfpathmoveto{\pgfqpoint{0.674050in}{1.578237in}}%
\pgfpathcurveto{\pgfqpoint{0.682286in}{1.578237in}}{\pgfqpoint{0.690187in}{1.581509in}}{\pgfqpoint{0.696010in}{1.587333in}}%
\pgfpathcurveto{\pgfqpoint{0.701834in}{1.593157in}}{\pgfqpoint{0.705107in}{1.601057in}}{\pgfqpoint{0.705107in}{1.609293in}}%
\pgfpathcurveto{\pgfqpoint{0.705107in}{1.617530in}}{\pgfqpoint{0.701834in}{1.625430in}}{\pgfqpoint{0.696010in}{1.631254in}}%
\pgfpathcurveto{\pgfqpoint{0.690187in}{1.637078in}}{\pgfqpoint{0.682286in}{1.640350in}}{\pgfqpoint{0.674050in}{1.640350in}}%
\pgfpathcurveto{\pgfqpoint{0.665814in}{1.640350in}}{\pgfqpoint{0.657914in}{1.637078in}}{\pgfqpoint{0.652090in}{1.631254in}}%
\pgfpathcurveto{\pgfqpoint{0.646266in}{1.625430in}}{\pgfqpoint{0.642994in}{1.617530in}}{\pgfqpoint{0.642994in}{1.609293in}}%
\pgfpathcurveto{\pgfqpoint{0.642994in}{1.601057in}}{\pgfqpoint{0.646266in}{1.593157in}}{\pgfqpoint{0.652090in}{1.587333in}}%
\pgfpathcurveto{\pgfqpoint{0.657914in}{1.581509in}}{\pgfqpoint{0.665814in}{1.578237in}}{\pgfqpoint{0.674050in}{1.578237in}}%
\pgfpathclose%
\pgfusepath{stroke,fill}%
\end{pgfscope}%
\begin{pgfscope}%
\pgfpathrectangle{\pgfqpoint{0.100000in}{0.212622in}}{\pgfqpoint{3.696000in}{3.696000in}}%
\pgfusepath{clip}%
\pgfsetbuttcap%
\pgfsetroundjoin%
\definecolor{currentfill}{rgb}{0.121569,0.466667,0.705882}%
\pgfsetfillcolor{currentfill}%
\pgfsetfillopacity{0.708930}%
\pgfsetlinewidth{1.003750pt}%
\definecolor{currentstroke}{rgb}{0.121569,0.466667,0.705882}%
\pgfsetstrokecolor{currentstroke}%
\pgfsetstrokeopacity{0.708930}%
\pgfsetdash{}{0pt}%
\pgfpathmoveto{\pgfqpoint{0.672089in}{1.576291in}}%
\pgfpathcurveto{\pgfqpoint{0.680325in}{1.576291in}}{\pgfqpoint{0.688225in}{1.579563in}}{\pgfqpoint{0.694049in}{1.585387in}}%
\pgfpathcurveto{\pgfqpoint{0.699873in}{1.591211in}}{\pgfqpoint{0.703145in}{1.599111in}}{\pgfqpoint{0.703145in}{1.607347in}}%
\pgfpathcurveto{\pgfqpoint{0.703145in}{1.615584in}}{\pgfqpoint{0.699873in}{1.623484in}}{\pgfqpoint{0.694049in}{1.629308in}}%
\pgfpathcurveto{\pgfqpoint{0.688225in}{1.635132in}}{\pgfqpoint{0.680325in}{1.638404in}}{\pgfqpoint{0.672089in}{1.638404in}}%
\pgfpathcurveto{\pgfqpoint{0.663852in}{1.638404in}}{\pgfqpoint{0.655952in}{1.635132in}}{\pgfqpoint{0.650128in}{1.629308in}}%
\pgfpathcurveto{\pgfqpoint{0.644304in}{1.623484in}}{\pgfqpoint{0.641032in}{1.615584in}}{\pgfqpoint{0.641032in}{1.607347in}}%
\pgfpathcurveto{\pgfqpoint{0.641032in}{1.599111in}}{\pgfqpoint{0.644304in}{1.591211in}}{\pgfqpoint{0.650128in}{1.585387in}}%
\pgfpathcurveto{\pgfqpoint{0.655952in}{1.579563in}}{\pgfqpoint{0.663852in}{1.576291in}}{\pgfqpoint{0.672089in}{1.576291in}}%
\pgfpathclose%
\pgfusepath{stroke,fill}%
\end{pgfscope}%
\begin{pgfscope}%
\pgfpathrectangle{\pgfqpoint{0.100000in}{0.212622in}}{\pgfqpoint{3.696000in}{3.696000in}}%
\pgfusepath{clip}%
\pgfsetbuttcap%
\pgfsetroundjoin%
\definecolor{currentfill}{rgb}{0.121569,0.466667,0.705882}%
\pgfsetfillcolor{currentfill}%
\pgfsetfillopacity{0.709479}%
\pgfsetlinewidth{1.003750pt}%
\definecolor{currentstroke}{rgb}{0.121569,0.466667,0.705882}%
\pgfsetstrokecolor{currentstroke}%
\pgfsetstrokeopacity{0.709479}%
\pgfsetdash{}{0pt}%
\pgfpathmoveto{\pgfqpoint{0.670678in}{1.574527in}}%
\pgfpathcurveto{\pgfqpoint{0.678914in}{1.574527in}}{\pgfqpoint{0.686814in}{1.577800in}}{\pgfqpoint{0.692638in}{1.583623in}}%
\pgfpathcurveto{\pgfqpoint{0.698462in}{1.589447in}}{\pgfqpoint{0.701734in}{1.597347in}}{\pgfqpoint{0.701734in}{1.605584in}}%
\pgfpathcurveto{\pgfqpoint{0.701734in}{1.613820in}}{\pgfqpoint{0.698462in}{1.621720in}}{\pgfqpoint{0.692638in}{1.627544in}}%
\pgfpathcurveto{\pgfqpoint{0.686814in}{1.633368in}}{\pgfqpoint{0.678914in}{1.636640in}}{\pgfqpoint{0.670678in}{1.636640in}}%
\pgfpathcurveto{\pgfqpoint{0.662441in}{1.636640in}}{\pgfqpoint{0.654541in}{1.633368in}}{\pgfqpoint{0.648718in}{1.627544in}}%
\pgfpathcurveto{\pgfqpoint{0.642894in}{1.621720in}}{\pgfqpoint{0.639621in}{1.613820in}}{\pgfqpoint{0.639621in}{1.605584in}}%
\pgfpathcurveto{\pgfqpoint{0.639621in}{1.597347in}}{\pgfqpoint{0.642894in}{1.589447in}}{\pgfqpoint{0.648718in}{1.583623in}}%
\pgfpathcurveto{\pgfqpoint{0.654541in}{1.577800in}}{\pgfqpoint{0.662441in}{1.574527in}}{\pgfqpoint{0.670678in}{1.574527in}}%
\pgfpathclose%
\pgfusepath{stroke,fill}%
\end{pgfscope}%
\begin{pgfscope}%
\pgfpathrectangle{\pgfqpoint{0.100000in}{0.212622in}}{\pgfqpoint{3.696000in}{3.696000in}}%
\pgfusepath{clip}%
\pgfsetbuttcap%
\pgfsetroundjoin%
\definecolor{currentfill}{rgb}{0.121569,0.466667,0.705882}%
\pgfsetfillcolor{currentfill}%
\pgfsetfillopacity{0.709747}%
\pgfsetlinewidth{1.003750pt}%
\definecolor{currentstroke}{rgb}{0.121569,0.466667,0.705882}%
\pgfsetstrokecolor{currentstroke}%
\pgfsetstrokeopacity{0.709747}%
\pgfsetdash{}{0pt}%
\pgfpathmoveto{\pgfqpoint{0.670048in}{1.572393in}}%
\pgfpathcurveto{\pgfqpoint{0.678284in}{1.572393in}}{\pgfqpoint{0.686184in}{1.575665in}}{\pgfqpoint{0.692008in}{1.581489in}}%
\pgfpathcurveto{\pgfqpoint{0.697832in}{1.587313in}}{\pgfqpoint{0.701104in}{1.595213in}}{\pgfqpoint{0.701104in}{1.603449in}}%
\pgfpathcurveto{\pgfqpoint{0.701104in}{1.611686in}}{\pgfqpoint{0.697832in}{1.619586in}}{\pgfqpoint{0.692008in}{1.625410in}}%
\pgfpathcurveto{\pgfqpoint{0.686184in}{1.631234in}}{\pgfqpoint{0.678284in}{1.634506in}}{\pgfqpoint{0.670048in}{1.634506in}}%
\pgfpathcurveto{\pgfqpoint{0.661811in}{1.634506in}}{\pgfqpoint{0.653911in}{1.631234in}}{\pgfqpoint{0.648087in}{1.625410in}}%
\pgfpathcurveto{\pgfqpoint{0.642263in}{1.619586in}}{\pgfqpoint{0.638991in}{1.611686in}}{\pgfqpoint{0.638991in}{1.603449in}}%
\pgfpathcurveto{\pgfqpoint{0.638991in}{1.595213in}}{\pgfqpoint{0.642263in}{1.587313in}}{\pgfqpoint{0.648087in}{1.581489in}}%
\pgfpathcurveto{\pgfqpoint{0.653911in}{1.575665in}}{\pgfqpoint{0.661811in}{1.572393in}}{\pgfqpoint{0.670048in}{1.572393in}}%
\pgfpathclose%
\pgfusepath{stroke,fill}%
\end{pgfscope}%
\begin{pgfscope}%
\pgfpathrectangle{\pgfqpoint{0.100000in}{0.212622in}}{\pgfqpoint{3.696000in}{3.696000in}}%
\pgfusepath{clip}%
\pgfsetbuttcap%
\pgfsetroundjoin%
\definecolor{currentfill}{rgb}{0.121569,0.466667,0.705882}%
\pgfsetfillcolor{currentfill}%
\pgfsetfillopacity{0.709991}%
\pgfsetlinewidth{1.003750pt}%
\definecolor{currentstroke}{rgb}{0.121569,0.466667,0.705882}%
\pgfsetstrokecolor{currentstroke}%
\pgfsetstrokeopacity{0.709991}%
\pgfsetdash{}{0pt}%
\pgfpathmoveto{\pgfqpoint{0.669870in}{1.571530in}}%
\pgfpathcurveto{\pgfqpoint{0.678106in}{1.571530in}}{\pgfqpoint{0.686006in}{1.574803in}}{\pgfqpoint{0.691830in}{1.580627in}}%
\pgfpathcurveto{\pgfqpoint{0.697654in}{1.586451in}}{\pgfqpoint{0.700926in}{1.594351in}}{\pgfqpoint{0.700926in}{1.602587in}}%
\pgfpathcurveto{\pgfqpoint{0.700926in}{1.610823in}}{\pgfqpoint{0.697654in}{1.618723in}}{\pgfqpoint{0.691830in}{1.624547in}}%
\pgfpathcurveto{\pgfqpoint{0.686006in}{1.630371in}}{\pgfqpoint{0.678106in}{1.633643in}}{\pgfqpoint{0.669870in}{1.633643in}}%
\pgfpathcurveto{\pgfqpoint{0.661634in}{1.633643in}}{\pgfqpoint{0.653734in}{1.630371in}}{\pgfqpoint{0.647910in}{1.624547in}}%
\pgfpathcurveto{\pgfqpoint{0.642086in}{1.618723in}}{\pgfqpoint{0.638813in}{1.610823in}}{\pgfqpoint{0.638813in}{1.602587in}}%
\pgfpathcurveto{\pgfqpoint{0.638813in}{1.594351in}}{\pgfqpoint{0.642086in}{1.586451in}}{\pgfqpoint{0.647910in}{1.580627in}}%
\pgfpathcurveto{\pgfqpoint{0.653734in}{1.574803in}}{\pgfqpoint{0.661634in}{1.571530in}}{\pgfqpoint{0.669870in}{1.571530in}}%
\pgfpathclose%
\pgfusepath{stroke,fill}%
\end{pgfscope}%
\begin{pgfscope}%
\pgfpathrectangle{\pgfqpoint{0.100000in}{0.212622in}}{\pgfqpoint{3.696000in}{3.696000in}}%
\pgfusepath{clip}%
\pgfsetbuttcap%
\pgfsetroundjoin%
\definecolor{currentfill}{rgb}{0.121569,0.466667,0.705882}%
\pgfsetfillcolor{currentfill}%
\pgfsetfillopacity{0.710226}%
\pgfsetlinewidth{1.003750pt}%
\definecolor{currentstroke}{rgb}{0.121569,0.466667,0.705882}%
\pgfsetstrokecolor{currentstroke}%
\pgfsetstrokeopacity{0.710226}%
\pgfsetdash{}{0pt}%
\pgfpathmoveto{\pgfqpoint{0.672869in}{1.566492in}}%
\pgfpathcurveto{\pgfqpoint{0.681106in}{1.566492in}}{\pgfqpoint{0.689006in}{1.569764in}}{\pgfqpoint{0.694830in}{1.575588in}}%
\pgfpathcurveto{\pgfqpoint{0.700653in}{1.581412in}}{\pgfqpoint{0.703926in}{1.589312in}}{\pgfqpoint{0.703926in}{1.597548in}}%
\pgfpathcurveto{\pgfqpoint{0.703926in}{1.605785in}}{\pgfqpoint{0.700653in}{1.613685in}}{\pgfqpoint{0.694830in}{1.619509in}}%
\pgfpathcurveto{\pgfqpoint{0.689006in}{1.625332in}}{\pgfqpoint{0.681106in}{1.628605in}}{\pgfqpoint{0.672869in}{1.628605in}}%
\pgfpathcurveto{\pgfqpoint{0.664633in}{1.628605in}}{\pgfqpoint{0.656733in}{1.625332in}}{\pgfqpoint{0.650909in}{1.619509in}}%
\pgfpathcurveto{\pgfqpoint{0.645085in}{1.613685in}}{\pgfqpoint{0.641813in}{1.605785in}}{\pgfqpoint{0.641813in}{1.597548in}}%
\pgfpathcurveto{\pgfqpoint{0.641813in}{1.589312in}}{\pgfqpoint{0.645085in}{1.581412in}}{\pgfqpoint{0.650909in}{1.575588in}}%
\pgfpathcurveto{\pgfqpoint{0.656733in}{1.569764in}}{\pgfqpoint{0.664633in}{1.566492in}}{\pgfqpoint{0.672869in}{1.566492in}}%
\pgfpathclose%
\pgfusepath{stroke,fill}%
\end{pgfscope}%
\begin{pgfscope}%
\pgfpathrectangle{\pgfqpoint{0.100000in}{0.212622in}}{\pgfqpoint{3.696000in}{3.696000in}}%
\pgfusepath{clip}%
\pgfsetbuttcap%
\pgfsetroundjoin%
\definecolor{currentfill}{rgb}{0.121569,0.466667,0.705882}%
\pgfsetfillcolor{currentfill}%
\pgfsetfillopacity{0.710226}%
\pgfsetlinewidth{1.003750pt}%
\definecolor{currentstroke}{rgb}{0.121569,0.466667,0.705882}%
\pgfsetstrokecolor{currentstroke}%
\pgfsetstrokeopacity{0.710226}%
\pgfsetdash{}{0pt}%
\pgfpathmoveto{\pgfqpoint{0.672869in}{1.566492in}}%
\pgfpathcurveto{\pgfqpoint{0.681106in}{1.566492in}}{\pgfqpoint{0.689006in}{1.569764in}}{\pgfqpoint{0.694830in}{1.575588in}}%
\pgfpathcurveto{\pgfqpoint{0.700653in}{1.581412in}}{\pgfqpoint{0.703926in}{1.589312in}}{\pgfqpoint{0.703926in}{1.597548in}}%
\pgfpathcurveto{\pgfqpoint{0.703926in}{1.605785in}}{\pgfqpoint{0.700653in}{1.613685in}}{\pgfqpoint{0.694830in}{1.619509in}}%
\pgfpathcurveto{\pgfqpoint{0.689006in}{1.625332in}}{\pgfqpoint{0.681106in}{1.628605in}}{\pgfqpoint{0.672869in}{1.628605in}}%
\pgfpathcurveto{\pgfqpoint{0.664633in}{1.628605in}}{\pgfqpoint{0.656733in}{1.625332in}}{\pgfqpoint{0.650909in}{1.619509in}}%
\pgfpathcurveto{\pgfqpoint{0.645085in}{1.613685in}}{\pgfqpoint{0.641813in}{1.605785in}}{\pgfqpoint{0.641813in}{1.597548in}}%
\pgfpathcurveto{\pgfqpoint{0.641813in}{1.589312in}}{\pgfqpoint{0.645085in}{1.581412in}}{\pgfqpoint{0.650909in}{1.575588in}}%
\pgfpathcurveto{\pgfqpoint{0.656733in}{1.569764in}}{\pgfqpoint{0.664633in}{1.566492in}}{\pgfqpoint{0.672869in}{1.566492in}}%
\pgfpathclose%
\pgfusepath{stroke,fill}%
\end{pgfscope}%
\begin{pgfscope}%
\pgfpathrectangle{\pgfqpoint{0.100000in}{0.212622in}}{\pgfqpoint{3.696000in}{3.696000in}}%
\pgfusepath{clip}%
\pgfsetbuttcap%
\pgfsetroundjoin%
\definecolor{currentfill}{rgb}{0.121569,0.466667,0.705882}%
\pgfsetfillcolor{currentfill}%
\pgfsetfillopacity{0.710226}%
\pgfsetlinewidth{1.003750pt}%
\definecolor{currentstroke}{rgb}{0.121569,0.466667,0.705882}%
\pgfsetstrokecolor{currentstroke}%
\pgfsetstrokeopacity{0.710226}%
\pgfsetdash{}{0pt}%
\pgfpathmoveto{\pgfqpoint{0.672869in}{1.566492in}}%
\pgfpathcurveto{\pgfqpoint{0.681106in}{1.566492in}}{\pgfqpoint{0.689006in}{1.569764in}}{\pgfqpoint{0.694830in}{1.575588in}}%
\pgfpathcurveto{\pgfqpoint{0.700653in}{1.581412in}}{\pgfqpoint{0.703926in}{1.589312in}}{\pgfqpoint{0.703926in}{1.597548in}}%
\pgfpathcurveto{\pgfqpoint{0.703926in}{1.605785in}}{\pgfqpoint{0.700653in}{1.613685in}}{\pgfqpoint{0.694830in}{1.619509in}}%
\pgfpathcurveto{\pgfqpoint{0.689006in}{1.625332in}}{\pgfqpoint{0.681106in}{1.628605in}}{\pgfqpoint{0.672869in}{1.628605in}}%
\pgfpathcurveto{\pgfqpoint{0.664633in}{1.628605in}}{\pgfqpoint{0.656733in}{1.625332in}}{\pgfqpoint{0.650909in}{1.619509in}}%
\pgfpathcurveto{\pgfqpoint{0.645085in}{1.613685in}}{\pgfqpoint{0.641813in}{1.605785in}}{\pgfqpoint{0.641813in}{1.597548in}}%
\pgfpathcurveto{\pgfqpoint{0.641813in}{1.589312in}}{\pgfqpoint{0.645085in}{1.581412in}}{\pgfqpoint{0.650909in}{1.575588in}}%
\pgfpathcurveto{\pgfqpoint{0.656733in}{1.569764in}}{\pgfqpoint{0.664633in}{1.566492in}}{\pgfqpoint{0.672869in}{1.566492in}}%
\pgfpathclose%
\pgfusepath{stroke,fill}%
\end{pgfscope}%
\begin{pgfscope}%
\pgfpathrectangle{\pgfqpoint{0.100000in}{0.212622in}}{\pgfqpoint{3.696000in}{3.696000in}}%
\pgfusepath{clip}%
\pgfsetbuttcap%
\pgfsetroundjoin%
\definecolor{currentfill}{rgb}{0.121569,0.466667,0.705882}%
\pgfsetfillcolor{currentfill}%
\pgfsetfillopacity{0.710226}%
\pgfsetlinewidth{1.003750pt}%
\definecolor{currentstroke}{rgb}{0.121569,0.466667,0.705882}%
\pgfsetstrokecolor{currentstroke}%
\pgfsetstrokeopacity{0.710226}%
\pgfsetdash{}{0pt}%
\pgfpathmoveto{\pgfqpoint{0.672869in}{1.566492in}}%
\pgfpathcurveto{\pgfqpoint{0.681106in}{1.566492in}}{\pgfqpoint{0.689006in}{1.569764in}}{\pgfqpoint{0.694830in}{1.575588in}}%
\pgfpathcurveto{\pgfqpoint{0.700653in}{1.581412in}}{\pgfqpoint{0.703926in}{1.589312in}}{\pgfqpoint{0.703926in}{1.597548in}}%
\pgfpathcurveto{\pgfqpoint{0.703926in}{1.605785in}}{\pgfqpoint{0.700653in}{1.613685in}}{\pgfqpoint{0.694830in}{1.619509in}}%
\pgfpathcurveto{\pgfqpoint{0.689006in}{1.625332in}}{\pgfqpoint{0.681106in}{1.628605in}}{\pgfqpoint{0.672869in}{1.628605in}}%
\pgfpathcurveto{\pgfqpoint{0.664633in}{1.628605in}}{\pgfqpoint{0.656733in}{1.625332in}}{\pgfqpoint{0.650909in}{1.619509in}}%
\pgfpathcurveto{\pgfqpoint{0.645085in}{1.613685in}}{\pgfqpoint{0.641813in}{1.605785in}}{\pgfqpoint{0.641813in}{1.597548in}}%
\pgfpathcurveto{\pgfqpoint{0.641813in}{1.589312in}}{\pgfqpoint{0.645085in}{1.581412in}}{\pgfqpoint{0.650909in}{1.575588in}}%
\pgfpathcurveto{\pgfqpoint{0.656733in}{1.569764in}}{\pgfqpoint{0.664633in}{1.566492in}}{\pgfqpoint{0.672869in}{1.566492in}}%
\pgfpathclose%
\pgfusepath{stroke,fill}%
\end{pgfscope}%
\begin{pgfscope}%
\pgfpathrectangle{\pgfqpoint{0.100000in}{0.212622in}}{\pgfqpoint{3.696000in}{3.696000in}}%
\pgfusepath{clip}%
\pgfsetbuttcap%
\pgfsetroundjoin%
\definecolor{currentfill}{rgb}{0.121569,0.466667,0.705882}%
\pgfsetfillcolor{currentfill}%
\pgfsetfillopacity{0.710248}%
\pgfsetlinewidth{1.003750pt}%
\definecolor{currentstroke}{rgb}{0.121569,0.466667,0.705882}%
\pgfsetstrokecolor{currentstroke}%
\pgfsetstrokeopacity{0.710248}%
\pgfsetdash{}{0pt}%
\pgfpathmoveto{\pgfqpoint{0.670682in}{1.569071in}}%
\pgfpathcurveto{\pgfqpoint{0.678919in}{1.569071in}}{\pgfqpoint{0.686819in}{1.572344in}}{\pgfqpoint{0.692642in}{1.578168in}}%
\pgfpathcurveto{\pgfqpoint{0.698466in}{1.583992in}}{\pgfqpoint{0.701739in}{1.591892in}}{\pgfqpoint{0.701739in}{1.600128in}}%
\pgfpathcurveto{\pgfqpoint{0.701739in}{1.608364in}}{\pgfqpoint{0.698466in}{1.616264in}}{\pgfqpoint{0.692642in}{1.622088in}}%
\pgfpathcurveto{\pgfqpoint{0.686819in}{1.627912in}}{\pgfqpoint{0.678919in}{1.631184in}}{\pgfqpoint{0.670682in}{1.631184in}}%
\pgfpathcurveto{\pgfqpoint{0.662446in}{1.631184in}}{\pgfqpoint{0.654546in}{1.627912in}}{\pgfqpoint{0.648722in}{1.622088in}}%
\pgfpathcurveto{\pgfqpoint{0.642898in}{1.616264in}}{\pgfqpoint{0.639626in}{1.608364in}}{\pgfqpoint{0.639626in}{1.600128in}}%
\pgfpathcurveto{\pgfqpoint{0.639626in}{1.591892in}}{\pgfqpoint{0.642898in}{1.583992in}}{\pgfqpoint{0.648722in}{1.578168in}}%
\pgfpathcurveto{\pgfqpoint{0.654546in}{1.572344in}}{\pgfqpoint{0.662446in}{1.569071in}}{\pgfqpoint{0.670682in}{1.569071in}}%
\pgfpathclose%
\pgfusepath{stroke,fill}%
\end{pgfscope}%
\begin{pgfscope}%
\pgfpathrectangle{\pgfqpoint{0.100000in}{0.212622in}}{\pgfqpoint{3.696000in}{3.696000in}}%
\pgfusepath{clip}%
\pgfsetbuttcap%
\pgfsetroundjoin%
\definecolor{currentfill}{rgb}{0.121569,0.466667,0.705882}%
\pgfsetfillcolor{currentfill}%
\pgfsetfillopacity{0.710250}%
\pgfsetlinewidth{1.003750pt}%
\definecolor{currentstroke}{rgb}{0.121569,0.466667,0.705882}%
\pgfsetstrokecolor{currentstroke}%
\pgfsetstrokeopacity{0.710250}%
\pgfsetdash{}{0pt}%
\pgfpathmoveto{\pgfqpoint{0.671842in}{1.567206in}}%
\pgfpathcurveto{\pgfqpoint{0.680078in}{1.567206in}}{\pgfqpoint{0.687978in}{1.570479in}}{\pgfqpoint{0.693802in}{1.576303in}}%
\pgfpathcurveto{\pgfqpoint{0.699626in}{1.582127in}}{\pgfqpoint{0.702898in}{1.590027in}}{\pgfqpoint{0.702898in}{1.598263in}}%
\pgfpathcurveto{\pgfqpoint{0.702898in}{1.606499in}}{\pgfqpoint{0.699626in}{1.614399in}}{\pgfqpoint{0.693802in}{1.620223in}}%
\pgfpathcurveto{\pgfqpoint{0.687978in}{1.626047in}}{\pgfqpoint{0.680078in}{1.629319in}}{\pgfqpoint{0.671842in}{1.629319in}}%
\pgfpathcurveto{\pgfqpoint{0.663606in}{1.629319in}}{\pgfqpoint{0.655706in}{1.626047in}}{\pgfqpoint{0.649882in}{1.620223in}}%
\pgfpathcurveto{\pgfqpoint{0.644058in}{1.614399in}}{\pgfqpoint{0.640785in}{1.606499in}}{\pgfqpoint{0.640785in}{1.598263in}}%
\pgfpathcurveto{\pgfqpoint{0.640785in}{1.590027in}}{\pgfqpoint{0.644058in}{1.582127in}}{\pgfqpoint{0.649882in}{1.576303in}}%
\pgfpathcurveto{\pgfqpoint{0.655706in}{1.570479in}}{\pgfqpoint{0.663606in}{1.567206in}}{\pgfqpoint{0.671842in}{1.567206in}}%
\pgfpathclose%
\pgfusepath{stroke,fill}%
\end{pgfscope}%
\begin{pgfscope}%
\pgfpathrectangle{\pgfqpoint{0.100000in}{0.212622in}}{\pgfqpoint{3.696000in}{3.696000in}}%
\pgfusepath{clip}%
\pgfsetbuttcap%
\pgfsetroundjoin%
\definecolor{currentfill}{rgb}{0.121569,0.466667,0.705882}%
\pgfsetfillcolor{currentfill}%
\pgfsetfillopacity{0.711932}%
\pgfsetlinewidth{1.003750pt}%
\definecolor{currentstroke}{rgb}{0.121569,0.466667,0.705882}%
\pgfsetstrokecolor{currentstroke}%
\pgfsetstrokeopacity{0.711932}%
\pgfsetdash{}{0pt}%
\pgfpathmoveto{\pgfqpoint{3.014924in}{1.889554in}}%
\pgfpathcurveto{\pgfqpoint{3.023160in}{1.889554in}}{\pgfqpoint{3.031060in}{1.892827in}}{\pgfqpoint{3.036884in}{1.898651in}}%
\pgfpathcurveto{\pgfqpoint{3.042708in}{1.904475in}}{\pgfqpoint{3.045981in}{1.912375in}}{\pgfqpoint{3.045981in}{1.920611in}}%
\pgfpathcurveto{\pgfqpoint{3.045981in}{1.928847in}}{\pgfqpoint{3.042708in}{1.936747in}}{\pgfqpoint{3.036884in}{1.942571in}}%
\pgfpathcurveto{\pgfqpoint{3.031060in}{1.948395in}}{\pgfqpoint{3.023160in}{1.951667in}}{\pgfqpoint{3.014924in}{1.951667in}}%
\pgfpathcurveto{\pgfqpoint{3.006688in}{1.951667in}}{\pgfqpoint{2.998788in}{1.948395in}}{\pgfqpoint{2.992964in}{1.942571in}}%
\pgfpathcurveto{\pgfqpoint{2.987140in}{1.936747in}}{\pgfqpoint{2.983868in}{1.928847in}}{\pgfqpoint{2.983868in}{1.920611in}}%
\pgfpathcurveto{\pgfqpoint{2.983868in}{1.912375in}}{\pgfqpoint{2.987140in}{1.904475in}}{\pgfqpoint{2.992964in}{1.898651in}}%
\pgfpathcurveto{\pgfqpoint{2.998788in}{1.892827in}}{\pgfqpoint{3.006688in}{1.889554in}}{\pgfqpoint{3.014924in}{1.889554in}}%
\pgfpathclose%
\pgfusepath{stroke,fill}%
\end{pgfscope}%
\begin{pgfscope}%
\pgfpathrectangle{\pgfqpoint{0.100000in}{0.212622in}}{\pgfqpoint{3.696000in}{3.696000in}}%
\pgfusepath{clip}%
\pgfsetbuttcap%
\pgfsetroundjoin%
\definecolor{currentfill}{rgb}{0.121569,0.466667,0.705882}%
\pgfsetfillcolor{currentfill}%
\pgfsetfillopacity{0.713689}%
\pgfsetlinewidth{1.003750pt}%
\definecolor{currentstroke}{rgb}{0.121569,0.466667,0.705882}%
\pgfsetstrokecolor{currentstroke}%
\pgfsetstrokeopacity{0.713689}%
\pgfsetdash{}{0pt}%
\pgfpathmoveto{\pgfqpoint{3.010989in}{1.878491in}}%
\pgfpathcurveto{\pgfqpoint{3.019225in}{1.878491in}}{\pgfqpoint{3.027125in}{1.881763in}}{\pgfqpoint{3.032949in}{1.887587in}}%
\pgfpathcurveto{\pgfqpoint{3.038773in}{1.893411in}}{\pgfqpoint{3.042046in}{1.901311in}}{\pgfqpoint{3.042046in}{1.909547in}}%
\pgfpathcurveto{\pgfqpoint{3.042046in}{1.917783in}}{\pgfqpoint{3.038773in}{1.925683in}}{\pgfqpoint{3.032949in}{1.931507in}}%
\pgfpathcurveto{\pgfqpoint{3.027125in}{1.937331in}}{\pgfqpoint{3.019225in}{1.940604in}}{\pgfqpoint{3.010989in}{1.940604in}}%
\pgfpathcurveto{\pgfqpoint{3.002753in}{1.940604in}}{\pgfqpoint{2.994853in}{1.937331in}}{\pgfqpoint{2.989029in}{1.931507in}}%
\pgfpathcurveto{\pgfqpoint{2.983205in}{1.925683in}}{\pgfqpoint{2.979933in}{1.917783in}}{\pgfqpoint{2.979933in}{1.909547in}}%
\pgfpathcurveto{\pgfqpoint{2.979933in}{1.901311in}}{\pgfqpoint{2.983205in}{1.893411in}}{\pgfqpoint{2.989029in}{1.887587in}}%
\pgfpathcurveto{\pgfqpoint{2.994853in}{1.881763in}}{\pgfqpoint{3.002753in}{1.878491in}}{\pgfqpoint{3.010989in}{1.878491in}}%
\pgfpathclose%
\pgfusepath{stroke,fill}%
\end{pgfscope}%
\begin{pgfscope}%
\pgfpathrectangle{\pgfqpoint{0.100000in}{0.212622in}}{\pgfqpoint{3.696000in}{3.696000in}}%
\pgfusepath{clip}%
\pgfsetbuttcap%
\pgfsetroundjoin%
\definecolor{currentfill}{rgb}{0.121569,0.466667,0.705882}%
\pgfsetfillcolor{currentfill}%
\pgfsetfillopacity{0.715374}%
\pgfsetlinewidth{1.003750pt}%
\definecolor{currentstroke}{rgb}{0.121569,0.466667,0.705882}%
\pgfsetstrokecolor{currentstroke}%
\pgfsetstrokeopacity{0.715374}%
\pgfsetdash{}{0pt}%
\pgfpathmoveto{\pgfqpoint{3.008734in}{1.877537in}}%
\pgfpathcurveto{\pgfqpoint{3.016970in}{1.877537in}}{\pgfqpoint{3.024870in}{1.880809in}}{\pgfqpoint{3.030694in}{1.886633in}}%
\pgfpathcurveto{\pgfqpoint{3.036518in}{1.892457in}}{\pgfqpoint{3.039790in}{1.900357in}}{\pgfqpoint{3.039790in}{1.908593in}}%
\pgfpathcurveto{\pgfqpoint{3.039790in}{1.916830in}}{\pgfqpoint{3.036518in}{1.924730in}}{\pgfqpoint{3.030694in}{1.930554in}}%
\pgfpathcurveto{\pgfqpoint{3.024870in}{1.936378in}}{\pgfqpoint{3.016970in}{1.939650in}}{\pgfqpoint{3.008734in}{1.939650in}}%
\pgfpathcurveto{\pgfqpoint{3.000498in}{1.939650in}}{\pgfqpoint{2.992598in}{1.936378in}}{\pgfqpoint{2.986774in}{1.930554in}}%
\pgfpathcurveto{\pgfqpoint{2.980950in}{1.924730in}}{\pgfqpoint{2.977677in}{1.916830in}}{\pgfqpoint{2.977677in}{1.908593in}}%
\pgfpathcurveto{\pgfqpoint{2.977677in}{1.900357in}}{\pgfqpoint{2.980950in}{1.892457in}}{\pgfqpoint{2.986774in}{1.886633in}}%
\pgfpathcurveto{\pgfqpoint{2.992598in}{1.880809in}}{\pgfqpoint{3.000498in}{1.877537in}}{\pgfqpoint{3.008734in}{1.877537in}}%
\pgfpathclose%
\pgfusepath{stroke,fill}%
\end{pgfscope}%
\begin{pgfscope}%
\pgfpathrectangle{\pgfqpoint{0.100000in}{0.212622in}}{\pgfqpoint{3.696000in}{3.696000in}}%
\pgfusepath{clip}%
\pgfsetbuttcap%
\pgfsetroundjoin%
\definecolor{currentfill}{rgb}{0.121569,0.466667,0.705882}%
\pgfsetfillcolor{currentfill}%
\pgfsetfillopacity{0.717241}%
\pgfsetlinewidth{1.003750pt}%
\definecolor{currentstroke}{rgb}{0.121569,0.466667,0.705882}%
\pgfsetstrokecolor{currentstroke}%
\pgfsetstrokeopacity{0.717241}%
\pgfsetdash{}{0pt}%
\pgfpathmoveto{\pgfqpoint{3.004093in}{1.874558in}}%
\pgfpathcurveto{\pgfqpoint{3.012329in}{1.874558in}}{\pgfqpoint{3.020229in}{1.877831in}}{\pgfqpoint{3.026053in}{1.883655in}}%
\pgfpathcurveto{\pgfqpoint{3.031877in}{1.889479in}}{\pgfqpoint{3.035149in}{1.897379in}}{\pgfqpoint{3.035149in}{1.905615in}}%
\pgfpathcurveto{\pgfqpoint{3.035149in}{1.913851in}}{\pgfqpoint{3.031877in}{1.921751in}}{\pgfqpoint{3.026053in}{1.927575in}}%
\pgfpathcurveto{\pgfqpoint{3.020229in}{1.933399in}}{\pgfqpoint{3.012329in}{1.936671in}}{\pgfqpoint{3.004093in}{1.936671in}}%
\pgfpathcurveto{\pgfqpoint{2.995857in}{1.936671in}}{\pgfqpoint{2.987957in}{1.933399in}}{\pgfqpoint{2.982133in}{1.927575in}}%
\pgfpathcurveto{\pgfqpoint{2.976309in}{1.921751in}}{\pgfqpoint{2.973036in}{1.913851in}}{\pgfqpoint{2.973036in}{1.905615in}}%
\pgfpathcurveto{\pgfqpoint{2.973036in}{1.897379in}}{\pgfqpoint{2.976309in}{1.889479in}}{\pgfqpoint{2.982133in}{1.883655in}}%
\pgfpathcurveto{\pgfqpoint{2.987957in}{1.877831in}}{\pgfqpoint{2.995857in}{1.874558in}}{\pgfqpoint{3.004093in}{1.874558in}}%
\pgfpathclose%
\pgfusepath{stroke,fill}%
\end{pgfscope}%
\begin{pgfscope}%
\pgfpathrectangle{\pgfqpoint{0.100000in}{0.212622in}}{\pgfqpoint{3.696000in}{3.696000in}}%
\pgfusepath{clip}%
\pgfsetbuttcap%
\pgfsetroundjoin%
\definecolor{currentfill}{rgb}{0.121569,0.466667,0.705882}%
\pgfsetfillcolor{currentfill}%
\pgfsetfillopacity{0.719340}%
\pgfsetlinewidth{1.003750pt}%
\definecolor{currentstroke}{rgb}{0.121569,0.466667,0.705882}%
\pgfsetstrokecolor{currentstroke}%
\pgfsetstrokeopacity{0.719340}%
\pgfsetdash{}{0pt}%
\pgfpathmoveto{\pgfqpoint{2.998017in}{1.873873in}}%
\pgfpathcurveto{\pgfqpoint{3.006253in}{1.873873in}}{\pgfqpoint{3.014153in}{1.877145in}}{\pgfqpoint{3.019977in}{1.882969in}}%
\pgfpathcurveto{\pgfqpoint{3.025801in}{1.888793in}}{\pgfqpoint{3.029073in}{1.896693in}}{\pgfqpoint{3.029073in}{1.904929in}}%
\pgfpathcurveto{\pgfqpoint{3.029073in}{1.913166in}}{\pgfqpoint{3.025801in}{1.921066in}}{\pgfqpoint{3.019977in}{1.926890in}}%
\pgfpathcurveto{\pgfqpoint{3.014153in}{1.932713in}}{\pgfqpoint{3.006253in}{1.935986in}}{\pgfqpoint{2.998017in}{1.935986in}}%
\pgfpathcurveto{\pgfqpoint{2.989781in}{1.935986in}}{\pgfqpoint{2.981880in}{1.932713in}}{\pgfqpoint{2.976057in}{1.926890in}}%
\pgfpathcurveto{\pgfqpoint{2.970233in}{1.921066in}}{\pgfqpoint{2.966960in}{1.913166in}}{\pgfqpoint{2.966960in}{1.904929in}}%
\pgfpathcurveto{\pgfqpoint{2.966960in}{1.896693in}}{\pgfqpoint{2.970233in}{1.888793in}}{\pgfqpoint{2.976057in}{1.882969in}}%
\pgfpathcurveto{\pgfqpoint{2.981880in}{1.877145in}}{\pgfqpoint{2.989781in}{1.873873in}}{\pgfqpoint{2.998017in}{1.873873in}}%
\pgfpathclose%
\pgfusepath{stroke,fill}%
\end{pgfscope}%
\begin{pgfscope}%
\pgfpathrectangle{\pgfqpoint{0.100000in}{0.212622in}}{\pgfqpoint{3.696000in}{3.696000in}}%
\pgfusepath{clip}%
\pgfsetbuttcap%
\pgfsetroundjoin%
\definecolor{currentfill}{rgb}{0.121569,0.466667,0.705882}%
\pgfsetfillcolor{currentfill}%
\pgfsetfillopacity{0.722269}%
\pgfsetlinewidth{1.003750pt}%
\definecolor{currentstroke}{rgb}{0.121569,0.466667,0.705882}%
\pgfsetstrokecolor{currentstroke}%
\pgfsetstrokeopacity{0.722269}%
\pgfsetdash{}{0pt}%
\pgfpathmoveto{\pgfqpoint{2.995123in}{1.871850in}}%
\pgfpathcurveto{\pgfqpoint{3.003359in}{1.871850in}}{\pgfqpoint{3.011260in}{1.875123in}}{\pgfqpoint{3.017083in}{1.880947in}}%
\pgfpathcurveto{\pgfqpoint{3.022907in}{1.886770in}}{\pgfqpoint{3.026180in}{1.894671in}}{\pgfqpoint{3.026180in}{1.902907in}}%
\pgfpathcurveto{\pgfqpoint{3.026180in}{1.911143in}}{\pgfqpoint{3.022907in}{1.919043in}}{\pgfqpoint{3.017083in}{1.924867in}}%
\pgfpathcurveto{\pgfqpoint{3.011260in}{1.930691in}}{\pgfqpoint{3.003359in}{1.933963in}}{\pgfqpoint{2.995123in}{1.933963in}}%
\pgfpathcurveto{\pgfqpoint{2.986887in}{1.933963in}}{\pgfqpoint{2.978987in}{1.930691in}}{\pgfqpoint{2.973163in}{1.924867in}}%
\pgfpathcurveto{\pgfqpoint{2.967339in}{1.919043in}}{\pgfqpoint{2.964067in}{1.911143in}}{\pgfqpoint{2.964067in}{1.902907in}}%
\pgfpathcurveto{\pgfqpoint{2.964067in}{1.894671in}}{\pgfqpoint{2.967339in}{1.886770in}}{\pgfqpoint{2.973163in}{1.880947in}}%
\pgfpathcurveto{\pgfqpoint{2.978987in}{1.875123in}}{\pgfqpoint{2.986887in}{1.871850in}}{\pgfqpoint{2.995123in}{1.871850in}}%
\pgfpathclose%
\pgfusepath{stroke,fill}%
\end{pgfscope}%
\begin{pgfscope}%
\pgfpathrectangle{\pgfqpoint{0.100000in}{0.212622in}}{\pgfqpoint{3.696000in}{3.696000in}}%
\pgfusepath{clip}%
\pgfsetbuttcap%
\pgfsetroundjoin%
\definecolor{currentfill}{rgb}{0.121569,0.466667,0.705882}%
\pgfsetfillcolor{currentfill}%
\pgfsetfillopacity{0.724888}%
\pgfsetlinewidth{1.003750pt}%
\definecolor{currentstroke}{rgb}{0.121569,0.466667,0.705882}%
\pgfsetstrokecolor{currentstroke}%
\pgfsetstrokeopacity{0.724888}%
\pgfsetdash{}{0pt}%
\pgfpathmoveto{\pgfqpoint{2.990904in}{1.867234in}}%
\pgfpathcurveto{\pgfqpoint{2.999140in}{1.867234in}}{\pgfqpoint{3.007040in}{1.870506in}}{\pgfqpoint{3.012864in}{1.876330in}}%
\pgfpathcurveto{\pgfqpoint{3.018688in}{1.882154in}}{\pgfqpoint{3.021960in}{1.890054in}}{\pgfqpoint{3.021960in}{1.898290in}}%
\pgfpathcurveto{\pgfqpoint{3.021960in}{1.906526in}}{\pgfqpoint{3.018688in}{1.914426in}}{\pgfqpoint{3.012864in}{1.920250in}}%
\pgfpathcurveto{\pgfqpoint{3.007040in}{1.926074in}}{\pgfqpoint{2.999140in}{1.929347in}}{\pgfqpoint{2.990904in}{1.929347in}}%
\pgfpathcurveto{\pgfqpoint{2.982667in}{1.929347in}}{\pgfqpoint{2.974767in}{1.926074in}}{\pgfqpoint{2.968943in}{1.920250in}}%
\pgfpathcurveto{\pgfqpoint{2.963119in}{1.914426in}}{\pgfqpoint{2.959847in}{1.906526in}}{\pgfqpoint{2.959847in}{1.898290in}}%
\pgfpathcurveto{\pgfqpoint{2.959847in}{1.890054in}}{\pgfqpoint{2.963119in}{1.882154in}}{\pgfqpoint{2.968943in}{1.876330in}}%
\pgfpathcurveto{\pgfqpoint{2.974767in}{1.870506in}}{\pgfqpoint{2.982667in}{1.867234in}}{\pgfqpoint{2.990904in}{1.867234in}}%
\pgfpathclose%
\pgfusepath{stroke,fill}%
\end{pgfscope}%
\begin{pgfscope}%
\pgfpathrectangle{\pgfqpoint{0.100000in}{0.212622in}}{\pgfqpoint{3.696000in}{3.696000in}}%
\pgfusepath{clip}%
\pgfsetbuttcap%
\pgfsetroundjoin%
\definecolor{currentfill}{rgb}{0.121569,0.466667,0.705882}%
\pgfsetfillcolor{currentfill}%
\pgfsetfillopacity{0.728112}%
\pgfsetlinewidth{1.003750pt}%
\definecolor{currentstroke}{rgb}{0.121569,0.466667,0.705882}%
\pgfsetstrokecolor{currentstroke}%
\pgfsetstrokeopacity{0.728112}%
\pgfsetdash{}{0pt}%
\pgfpathmoveto{\pgfqpoint{2.981872in}{1.869229in}}%
\pgfpathcurveto{\pgfqpoint{2.990108in}{1.869229in}}{\pgfqpoint{2.998008in}{1.872501in}}{\pgfqpoint{3.003832in}{1.878325in}}%
\pgfpathcurveto{\pgfqpoint{3.009656in}{1.884149in}}{\pgfqpoint{3.012929in}{1.892049in}}{\pgfqpoint{3.012929in}{1.900285in}}%
\pgfpathcurveto{\pgfqpoint{3.012929in}{1.908522in}}{\pgfqpoint{3.009656in}{1.916422in}}{\pgfqpoint{3.003832in}{1.922246in}}%
\pgfpathcurveto{\pgfqpoint{2.998008in}{1.928070in}}{\pgfqpoint{2.990108in}{1.931342in}}{\pgfqpoint{2.981872in}{1.931342in}}%
\pgfpathcurveto{\pgfqpoint{2.973636in}{1.931342in}}{\pgfqpoint{2.965736in}{1.928070in}}{\pgfqpoint{2.959912in}{1.922246in}}%
\pgfpathcurveto{\pgfqpoint{2.954088in}{1.916422in}}{\pgfqpoint{2.950816in}{1.908522in}}{\pgfqpoint{2.950816in}{1.900285in}}%
\pgfpathcurveto{\pgfqpoint{2.950816in}{1.892049in}}{\pgfqpoint{2.954088in}{1.884149in}}{\pgfqpoint{2.959912in}{1.878325in}}%
\pgfpathcurveto{\pgfqpoint{2.965736in}{1.872501in}}{\pgfqpoint{2.973636in}{1.869229in}}{\pgfqpoint{2.981872in}{1.869229in}}%
\pgfpathclose%
\pgfusepath{stroke,fill}%
\end{pgfscope}%
\begin{pgfscope}%
\pgfpathrectangle{\pgfqpoint{0.100000in}{0.212622in}}{\pgfqpoint{3.696000in}{3.696000in}}%
\pgfusepath{clip}%
\pgfsetbuttcap%
\pgfsetroundjoin%
\definecolor{currentfill}{rgb}{0.121569,0.466667,0.705882}%
\pgfsetfillcolor{currentfill}%
\pgfsetfillopacity{0.729568}%
\pgfsetlinewidth{1.003750pt}%
\definecolor{currentstroke}{rgb}{0.121569,0.466667,0.705882}%
\pgfsetstrokecolor{currentstroke}%
\pgfsetstrokeopacity{0.729568}%
\pgfsetdash{}{0pt}%
\pgfpathmoveto{\pgfqpoint{2.979876in}{1.864553in}}%
\pgfpathcurveto{\pgfqpoint{2.988112in}{1.864553in}}{\pgfqpoint{2.996012in}{1.867826in}}{\pgfqpoint{3.001836in}{1.873650in}}%
\pgfpathcurveto{\pgfqpoint{3.007660in}{1.879473in}}{\pgfqpoint{3.010932in}{1.887374in}}{\pgfqpoint{3.010932in}{1.895610in}}%
\pgfpathcurveto{\pgfqpoint{3.010932in}{1.903846in}}{\pgfqpoint{3.007660in}{1.911746in}}{\pgfqpoint{3.001836in}{1.917570in}}%
\pgfpathcurveto{\pgfqpoint{2.996012in}{1.923394in}}{\pgfqpoint{2.988112in}{1.926666in}}{\pgfqpoint{2.979876in}{1.926666in}}%
\pgfpathcurveto{\pgfqpoint{2.971639in}{1.926666in}}{\pgfqpoint{2.963739in}{1.923394in}}{\pgfqpoint{2.957915in}{1.917570in}}%
\pgfpathcurveto{\pgfqpoint{2.952091in}{1.911746in}}{\pgfqpoint{2.948819in}{1.903846in}}{\pgfqpoint{2.948819in}{1.895610in}}%
\pgfpathcurveto{\pgfqpoint{2.948819in}{1.887374in}}{\pgfqpoint{2.952091in}{1.879473in}}{\pgfqpoint{2.957915in}{1.873650in}}%
\pgfpathcurveto{\pgfqpoint{2.963739in}{1.867826in}}{\pgfqpoint{2.971639in}{1.864553in}}{\pgfqpoint{2.979876in}{1.864553in}}%
\pgfpathclose%
\pgfusepath{stroke,fill}%
\end{pgfscope}%
\begin{pgfscope}%
\pgfpathrectangle{\pgfqpoint{0.100000in}{0.212622in}}{\pgfqpoint{3.696000in}{3.696000in}}%
\pgfusepath{clip}%
\pgfsetbuttcap%
\pgfsetroundjoin%
\definecolor{currentfill}{rgb}{0.121569,0.466667,0.705882}%
\pgfsetfillcolor{currentfill}%
\pgfsetfillopacity{0.730510}%
\pgfsetlinewidth{1.003750pt}%
\definecolor{currentstroke}{rgb}{0.121569,0.466667,0.705882}%
\pgfsetstrokecolor{currentstroke}%
\pgfsetstrokeopacity{0.730510}%
\pgfsetdash{}{0pt}%
\pgfpathmoveto{\pgfqpoint{2.978768in}{1.862974in}}%
\pgfpathcurveto{\pgfqpoint{2.987004in}{1.862974in}}{\pgfqpoint{2.994904in}{1.866246in}}{\pgfqpoint{3.000728in}{1.872070in}}%
\pgfpathcurveto{\pgfqpoint{3.006552in}{1.877894in}}{\pgfqpoint{3.009824in}{1.885794in}}{\pgfqpoint{3.009824in}{1.894030in}}%
\pgfpathcurveto{\pgfqpoint{3.009824in}{1.902267in}}{\pgfqpoint{3.006552in}{1.910167in}}{\pgfqpoint{3.000728in}{1.915991in}}%
\pgfpathcurveto{\pgfqpoint{2.994904in}{1.921815in}}{\pgfqpoint{2.987004in}{1.925087in}}{\pgfqpoint{2.978768in}{1.925087in}}%
\pgfpathcurveto{\pgfqpoint{2.970531in}{1.925087in}}{\pgfqpoint{2.962631in}{1.921815in}}{\pgfqpoint{2.956808in}{1.915991in}}%
\pgfpathcurveto{\pgfqpoint{2.950984in}{1.910167in}}{\pgfqpoint{2.947711in}{1.902267in}}{\pgfqpoint{2.947711in}{1.894030in}}%
\pgfpathcurveto{\pgfqpoint{2.947711in}{1.885794in}}{\pgfqpoint{2.950984in}{1.877894in}}{\pgfqpoint{2.956808in}{1.872070in}}%
\pgfpathcurveto{\pgfqpoint{2.962631in}{1.866246in}}{\pgfqpoint{2.970531in}{1.862974in}}{\pgfqpoint{2.978768in}{1.862974in}}%
\pgfpathclose%
\pgfusepath{stroke,fill}%
\end{pgfscope}%
\begin{pgfscope}%
\pgfpathrectangle{\pgfqpoint{0.100000in}{0.212622in}}{\pgfqpoint{3.696000in}{3.696000in}}%
\pgfusepath{clip}%
\pgfsetbuttcap%
\pgfsetroundjoin%
\definecolor{currentfill}{rgb}{0.121569,0.466667,0.705882}%
\pgfsetfillcolor{currentfill}%
\pgfsetfillopacity{0.731731}%
\pgfsetlinewidth{1.003750pt}%
\definecolor{currentstroke}{rgb}{0.121569,0.466667,0.705882}%
\pgfsetstrokecolor{currentstroke}%
\pgfsetstrokeopacity{0.731731}%
\pgfsetdash{}{0pt}%
\pgfpathmoveto{\pgfqpoint{2.975255in}{1.859718in}}%
\pgfpathcurveto{\pgfqpoint{2.983491in}{1.859718in}}{\pgfqpoint{2.991391in}{1.862991in}}{\pgfqpoint{2.997215in}{1.868815in}}%
\pgfpathcurveto{\pgfqpoint{3.003039in}{1.874639in}}{\pgfqpoint{3.006311in}{1.882539in}}{\pgfqpoint{3.006311in}{1.890775in}}%
\pgfpathcurveto{\pgfqpoint{3.006311in}{1.899011in}}{\pgfqpoint{3.003039in}{1.906911in}}{\pgfqpoint{2.997215in}{1.912735in}}%
\pgfpathcurveto{\pgfqpoint{2.991391in}{1.918559in}}{\pgfqpoint{2.983491in}{1.921831in}}{\pgfqpoint{2.975255in}{1.921831in}}%
\pgfpathcurveto{\pgfqpoint{2.967019in}{1.921831in}}{\pgfqpoint{2.959119in}{1.918559in}}{\pgfqpoint{2.953295in}{1.912735in}}%
\pgfpathcurveto{\pgfqpoint{2.947471in}{1.906911in}}{\pgfqpoint{2.944198in}{1.899011in}}{\pgfqpoint{2.944198in}{1.890775in}}%
\pgfpathcurveto{\pgfqpoint{2.944198in}{1.882539in}}{\pgfqpoint{2.947471in}{1.874639in}}{\pgfqpoint{2.953295in}{1.868815in}}%
\pgfpathcurveto{\pgfqpoint{2.959119in}{1.862991in}}{\pgfqpoint{2.967019in}{1.859718in}}{\pgfqpoint{2.975255in}{1.859718in}}%
\pgfpathclose%
\pgfusepath{stroke,fill}%
\end{pgfscope}%
\begin{pgfscope}%
\pgfpathrectangle{\pgfqpoint{0.100000in}{0.212622in}}{\pgfqpoint{3.696000in}{3.696000in}}%
\pgfusepath{clip}%
\pgfsetbuttcap%
\pgfsetroundjoin%
\definecolor{currentfill}{rgb}{0.121569,0.466667,0.705882}%
\pgfsetfillcolor{currentfill}%
\pgfsetfillopacity{0.732635}%
\pgfsetlinewidth{1.003750pt}%
\definecolor{currentstroke}{rgb}{0.121569,0.466667,0.705882}%
\pgfsetstrokecolor{currentstroke}%
\pgfsetstrokeopacity{0.732635}%
\pgfsetdash{}{0pt}%
\pgfpathmoveto{\pgfqpoint{2.972711in}{1.860607in}}%
\pgfpathcurveto{\pgfqpoint{2.980947in}{1.860607in}}{\pgfqpoint{2.988847in}{1.863879in}}{\pgfqpoint{2.994671in}{1.869703in}}%
\pgfpathcurveto{\pgfqpoint{3.000495in}{1.875527in}}{\pgfqpoint{3.003768in}{1.883427in}}{\pgfqpoint{3.003768in}{1.891663in}}%
\pgfpathcurveto{\pgfqpoint{3.003768in}{1.899900in}}{\pgfqpoint{3.000495in}{1.907800in}}{\pgfqpoint{2.994671in}{1.913624in}}%
\pgfpathcurveto{\pgfqpoint{2.988847in}{1.919448in}}{\pgfqpoint{2.980947in}{1.922720in}}{\pgfqpoint{2.972711in}{1.922720in}}%
\pgfpathcurveto{\pgfqpoint{2.964475in}{1.922720in}}{\pgfqpoint{2.956575in}{1.919448in}}{\pgfqpoint{2.950751in}{1.913624in}}%
\pgfpathcurveto{\pgfqpoint{2.944927in}{1.907800in}}{\pgfqpoint{2.941655in}{1.899900in}}{\pgfqpoint{2.941655in}{1.891663in}}%
\pgfpathcurveto{\pgfqpoint{2.941655in}{1.883427in}}{\pgfqpoint{2.944927in}{1.875527in}}{\pgfqpoint{2.950751in}{1.869703in}}%
\pgfpathcurveto{\pgfqpoint{2.956575in}{1.863879in}}{\pgfqpoint{2.964475in}{1.860607in}}{\pgfqpoint{2.972711in}{1.860607in}}%
\pgfpathclose%
\pgfusepath{stroke,fill}%
\end{pgfscope}%
\begin{pgfscope}%
\pgfpathrectangle{\pgfqpoint{0.100000in}{0.212622in}}{\pgfqpoint{3.696000in}{3.696000in}}%
\pgfusepath{clip}%
\pgfsetbuttcap%
\pgfsetroundjoin%
\definecolor{currentfill}{rgb}{0.121569,0.466667,0.705882}%
\pgfsetfillcolor{currentfill}%
\pgfsetfillopacity{0.733469}%
\pgfsetlinewidth{1.003750pt}%
\definecolor{currentstroke}{rgb}{0.121569,0.466667,0.705882}%
\pgfsetstrokecolor{currentstroke}%
\pgfsetstrokeopacity{0.733469}%
\pgfsetdash{}{0pt}%
\pgfpathmoveto{\pgfqpoint{2.971977in}{1.856436in}}%
\pgfpathcurveto{\pgfqpoint{2.980213in}{1.856436in}}{\pgfqpoint{2.988113in}{1.859709in}}{\pgfqpoint{2.993937in}{1.865533in}}%
\pgfpathcurveto{\pgfqpoint{2.999761in}{1.871357in}}{\pgfqpoint{3.003033in}{1.879257in}}{\pgfqpoint{3.003033in}{1.887493in}}%
\pgfpathcurveto{\pgfqpoint{3.003033in}{1.895729in}}{\pgfqpoint{2.999761in}{1.903629in}}{\pgfqpoint{2.993937in}{1.909453in}}%
\pgfpathcurveto{\pgfqpoint{2.988113in}{1.915277in}}{\pgfqpoint{2.980213in}{1.918549in}}{\pgfqpoint{2.971977in}{1.918549in}}%
\pgfpathcurveto{\pgfqpoint{2.963740in}{1.918549in}}{\pgfqpoint{2.955840in}{1.915277in}}{\pgfqpoint{2.950016in}{1.909453in}}%
\pgfpathcurveto{\pgfqpoint{2.944192in}{1.903629in}}{\pgfqpoint{2.940920in}{1.895729in}}{\pgfqpoint{2.940920in}{1.887493in}}%
\pgfpathcurveto{\pgfqpoint{2.940920in}{1.879257in}}{\pgfqpoint{2.944192in}{1.871357in}}{\pgfqpoint{2.950016in}{1.865533in}}%
\pgfpathcurveto{\pgfqpoint{2.955840in}{1.859709in}}{\pgfqpoint{2.963740in}{1.856436in}}{\pgfqpoint{2.971977in}{1.856436in}}%
\pgfpathclose%
\pgfusepath{stroke,fill}%
\end{pgfscope}%
\begin{pgfscope}%
\pgfpathrectangle{\pgfqpoint{0.100000in}{0.212622in}}{\pgfqpoint{3.696000in}{3.696000in}}%
\pgfusepath{clip}%
\pgfsetbuttcap%
\pgfsetroundjoin%
\definecolor{currentfill}{rgb}{0.121569,0.466667,0.705882}%
\pgfsetfillcolor{currentfill}%
\pgfsetfillopacity{0.734827}%
\pgfsetlinewidth{1.003750pt}%
\definecolor{currentstroke}{rgb}{0.121569,0.466667,0.705882}%
\pgfsetstrokecolor{currentstroke}%
\pgfsetstrokeopacity{0.734827}%
\pgfsetdash{}{0pt}%
\pgfpathmoveto{\pgfqpoint{2.971105in}{1.854735in}}%
\pgfpathcurveto{\pgfqpoint{2.979341in}{1.854735in}}{\pgfqpoint{2.987241in}{1.858007in}}{\pgfqpoint{2.993065in}{1.863831in}}%
\pgfpathcurveto{\pgfqpoint{2.998889in}{1.869655in}}{\pgfqpoint{3.002161in}{1.877555in}}{\pgfqpoint{3.002161in}{1.885791in}}%
\pgfpathcurveto{\pgfqpoint{3.002161in}{1.894027in}}{\pgfqpoint{2.998889in}{1.901927in}}{\pgfqpoint{2.993065in}{1.907751in}}%
\pgfpathcurveto{\pgfqpoint{2.987241in}{1.913575in}}{\pgfqpoint{2.979341in}{1.916848in}}{\pgfqpoint{2.971105in}{1.916848in}}%
\pgfpathcurveto{\pgfqpoint{2.962869in}{1.916848in}}{\pgfqpoint{2.954968in}{1.913575in}}{\pgfqpoint{2.949145in}{1.907751in}}%
\pgfpathcurveto{\pgfqpoint{2.943321in}{1.901927in}}{\pgfqpoint{2.940048in}{1.894027in}}{\pgfqpoint{2.940048in}{1.885791in}}%
\pgfpathcurveto{\pgfqpoint{2.940048in}{1.877555in}}{\pgfqpoint{2.943321in}{1.869655in}}{\pgfqpoint{2.949145in}{1.863831in}}%
\pgfpathcurveto{\pgfqpoint{2.954968in}{1.858007in}}{\pgfqpoint{2.962869in}{1.854735in}}{\pgfqpoint{2.971105in}{1.854735in}}%
\pgfpathclose%
\pgfusepath{stroke,fill}%
\end{pgfscope}%
\begin{pgfscope}%
\pgfpathrectangle{\pgfqpoint{0.100000in}{0.212622in}}{\pgfqpoint{3.696000in}{3.696000in}}%
\pgfusepath{clip}%
\pgfsetbuttcap%
\pgfsetroundjoin%
\definecolor{currentfill}{rgb}{0.121569,0.466667,0.705882}%
\pgfsetfillcolor{currentfill}%
\pgfsetfillopacity{0.737485}%
\pgfsetlinewidth{1.003750pt}%
\definecolor{currentstroke}{rgb}{0.121569,0.466667,0.705882}%
\pgfsetstrokecolor{currentstroke}%
\pgfsetstrokeopacity{0.737485}%
\pgfsetdash{}{0pt}%
\pgfpathmoveto{\pgfqpoint{2.965641in}{1.859530in}}%
\pgfpathcurveto{\pgfqpoint{2.973878in}{1.859530in}}{\pgfqpoint{2.981778in}{1.862802in}}{\pgfqpoint{2.987602in}{1.868626in}}%
\pgfpathcurveto{\pgfqpoint{2.993426in}{1.874450in}}{\pgfqpoint{2.996698in}{1.882350in}}{\pgfqpoint{2.996698in}{1.890586in}}%
\pgfpathcurveto{\pgfqpoint{2.996698in}{1.898822in}}{\pgfqpoint{2.993426in}{1.906722in}}{\pgfqpoint{2.987602in}{1.912546in}}%
\pgfpathcurveto{\pgfqpoint{2.981778in}{1.918370in}}{\pgfqpoint{2.973878in}{1.921643in}}{\pgfqpoint{2.965641in}{1.921643in}}%
\pgfpathcurveto{\pgfqpoint{2.957405in}{1.921643in}}{\pgfqpoint{2.949505in}{1.918370in}}{\pgfqpoint{2.943681in}{1.912546in}}%
\pgfpathcurveto{\pgfqpoint{2.937857in}{1.906722in}}{\pgfqpoint{2.934585in}{1.898822in}}{\pgfqpoint{2.934585in}{1.890586in}}%
\pgfpathcurveto{\pgfqpoint{2.934585in}{1.882350in}}{\pgfqpoint{2.937857in}{1.874450in}}{\pgfqpoint{2.943681in}{1.868626in}}%
\pgfpathcurveto{\pgfqpoint{2.949505in}{1.862802in}}{\pgfqpoint{2.957405in}{1.859530in}}{\pgfqpoint{2.965641in}{1.859530in}}%
\pgfpathclose%
\pgfusepath{stroke,fill}%
\end{pgfscope}%
\begin{pgfscope}%
\pgfpathrectangle{\pgfqpoint{0.100000in}{0.212622in}}{\pgfqpoint{3.696000in}{3.696000in}}%
\pgfusepath{clip}%
\pgfsetbuttcap%
\pgfsetroundjoin%
\definecolor{currentfill}{rgb}{0.121569,0.466667,0.705882}%
\pgfsetfillcolor{currentfill}%
\pgfsetfillopacity{0.738491}%
\pgfsetlinewidth{1.003750pt}%
\definecolor{currentstroke}{rgb}{0.121569,0.466667,0.705882}%
\pgfsetstrokecolor{currentstroke}%
\pgfsetstrokeopacity{0.738491}%
\pgfsetdash{}{0pt}%
\pgfpathmoveto{\pgfqpoint{2.965053in}{1.856682in}}%
\pgfpathcurveto{\pgfqpoint{2.973289in}{1.856682in}}{\pgfqpoint{2.981189in}{1.859954in}}{\pgfqpoint{2.987013in}{1.865778in}}%
\pgfpathcurveto{\pgfqpoint{2.992837in}{1.871602in}}{\pgfqpoint{2.996109in}{1.879502in}}{\pgfqpoint{2.996109in}{1.887738in}}%
\pgfpathcurveto{\pgfqpoint{2.996109in}{1.895974in}}{\pgfqpoint{2.992837in}{1.903874in}}{\pgfqpoint{2.987013in}{1.909698in}}%
\pgfpathcurveto{\pgfqpoint{2.981189in}{1.915522in}}{\pgfqpoint{2.973289in}{1.918795in}}{\pgfqpoint{2.965053in}{1.918795in}}%
\pgfpathcurveto{\pgfqpoint{2.956817in}{1.918795in}}{\pgfqpoint{2.948917in}{1.915522in}}{\pgfqpoint{2.943093in}{1.909698in}}%
\pgfpathcurveto{\pgfqpoint{2.937269in}{1.903874in}}{\pgfqpoint{2.933996in}{1.895974in}}{\pgfqpoint{2.933996in}{1.887738in}}%
\pgfpathcurveto{\pgfqpoint{2.933996in}{1.879502in}}{\pgfqpoint{2.937269in}{1.871602in}}{\pgfqpoint{2.943093in}{1.865778in}}%
\pgfpathcurveto{\pgfqpoint{2.948917in}{1.859954in}}{\pgfqpoint{2.956817in}{1.856682in}}{\pgfqpoint{2.965053in}{1.856682in}}%
\pgfpathclose%
\pgfusepath{stroke,fill}%
\end{pgfscope}%
\begin{pgfscope}%
\pgfpathrectangle{\pgfqpoint{0.100000in}{0.212622in}}{\pgfqpoint{3.696000in}{3.696000in}}%
\pgfusepath{clip}%
\pgfsetbuttcap%
\pgfsetroundjoin%
\definecolor{currentfill}{rgb}{0.121569,0.466667,0.705882}%
\pgfsetfillcolor{currentfill}%
\pgfsetfillopacity{0.739014}%
\pgfsetlinewidth{1.003750pt}%
\definecolor{currentstroke}{rgb}{0.121569,0.466667,0.705882}%
\pgfsetstrokecolor{currentstroke}%
\pgfsetstrokeopacity{0.739014}%
\pgfsetdash{}{0pt}%
\pgfpathmoveto{\pgfqpoint{2.964281in}{1.855112in}}%
\pgfpathcurveto{\pgfqpoint{2.972517in}{1.855112in}}{\pgfqpoint{2.980417in}{1.858385in}}{\pgfqpoint{2.986241in}{1.864209in}}%
\pgfpathcurveto{\pgfqpoint{2.992065in}{1.870033in}}{\pgfqpoint{2.995338in}{1.877933in}}{\pgfqpoint{2.995338in}{1.886169in}}%
\pgfpathcurveto{\pgfqpoint{2.995338in}{1.894405in}}{\pgfqpoint{2.992065in}{1.902305in}}{\pgfqpoint{2.986241in}{1.908129in}}%
\pgfpathcurveto{\pgfqpoint{2.980417in}{1.913953in}}{\pgfqpoint{2.972517in}{1.917225in}}{\pgfqpoint{2.964281in}{1.917225in}}%
\pgfpathcurveto{\pgfqpoint{2.956045in}{1.917225in}}{\pgfqpoint{2.948145in}{1.913953in}}{\pgfqpoint{2.942321in}{1.908129in}}%
\pgfpathcurveto{\pgfqpoint{2.936497in}{1.902305in}}{\pgfqpoint{2.933225in}{1.894405in}}{\pgfqpoint{2.933225in}{1.886169in}}%
\pgfpathcurveto{\pgfqpoint{2.933225in}{1.877933in}}{\pgfqpoint{2.936497in}{1.870033in}}{\pgfqpoint{2.942321in}{1.864209in}}%
\pgfpathcurveto{\pgfqpoint{2.948145in}{1.858385in}}{\pgfqpoint{2.956045in}{1.855112in}}{\pgfqpoint{2.964281in}{1.855112in}}%
\pgfpathclose%
\pgfusepath{stroke,fill}%
\end{pgfscope}%
\begin{pgfscope}%
\pgfpathrectangle{\pgfqpoint{0.100000in}{0.212622in}}{\pgfqpoint{3.696000in}{3.696000in}}%
\pgfusepath{clip}%
\pgfsetbuttcap%
\pgfsetroundjoin%
\definecolor{currentfill}{rgb}{0.121569,0.466667,0.705882}%
\pgfsetfillcolor{currentfill}%
\pgfsetfillopacity{0.740642}%
\pgfsetlinewidth{1.003750pt}%
\definecolor{currentstroke}{rgb}{0.121569,0.466667,0.705882}%
\pgfsetstrokecolor{currentstroke}%
\pgfsetstrokeopacity{0.740642}%
\pgfsetdash{}{0pt}%
\pgfpathmoveto{\pgfqpoint{2.960282in}{1.856784in}}%
\pgfpathcurveto{\pgfqpoint{2.968518in}{1.856784in}}{\pgfqpoint{2.976418in}{1.860056in}}{\pgfqpoint{2.982242in}{1.865880in}}%
\pgfpathcurveto{\pgfqpoint{2.988066in}{1.871704in}}{\pgfqpoint{2.991339in}{1.879604in}}{\pgfqpoint{2.991339in}{1.887840in}}%
\pgfpathcurveto{\pgfqpoint{2.991339in}{1.896077in}}{\pgfqpoint{2.988066in}{1.903977in}}{\pgfqpoint{2.982242in}{1.909801in}}%
\pgfpathcurveto{\pgfqpoint{2.976418in}{1.915625in}}{\pgfqpoint{2.968518in}{1.918897in}}{\pgfqpoint{2.960282in}{1.918897in}}%
\pgfpathcurveto{\pgfqpoint{2.952046in}{1.918897in}}{\pgfqpoint{2.944146in}{1.915625in}}{\pgfqpoint{2.938322in}{1.909801in}}%
\pgfpathcurveto{\pgfqpoint{2.932498in}{1.903977in}}{\pgfqpoint{2.929226in}{1.896077in}}{\pgfqpoint{2.929226in}{1.887840in}}%
\pgfpathcurveto{\pgfqpoint{2.929226in}{1.879604in}}{\pgfqpoint{2.932498in}{1.871704in}}{\pgfqpoint{2.938322in}{1.865880in}}%
\pgfpathcurveto{\pgfqpoint{2.944146in}{1.860056in}}{\pgfqpoint{2.952046in}{1.856784in}}{\pgfqpoint{2.960282in}{1.856784in}}%
\pgfpathclose%
\pgfusepath{stroke,fill}%
\end{pgfscope}%
\begin{pgfscope}%
\pgfpathrectangle{\pgfqpoint{0.100000in}{0.212622in}}{\pgfqpoint{3.696000in}{3.696000in}}%
\pgfusepath{clip}%
\pgfsetbuttcap%
\pgfsetroundjoin%
\definecolor{currentfill}{rgb}{0.121569,0.466667,0.705882}%
\pgfsetfillcolor{currentfill}%
\pgfsetfillopacity{0.741810}%
\pgfsetlinewidth{1.003750pt}%
\definecolor{currentstroke}{rgb}{0.121569,0.466667,0.705882}%
\pgfsetstrokecolor{currentstroke}%
\pgfsetstrokeopacity{0.741810}%
\pgfsetdash{}{0pt}%
\pgfpathmoveto{\pgfqpoint{2.957441in}{1.852212in}}%
\pgfpathcurveto{\pgfqpoint{2.965678in}{1.852212in}}{\pgfqpoint{2.973578in}{1.855484in}}{\pgfqpoint{2.979402in}{1.861308in}}%
\pgfpathcurveto{\pgfqpoint{2.985225in}{1.867132in}}{\pgfqpoint{2.988498in}{1.875032in}}{\pgfqpoint{2.988498in}{1.883268in}}%
\pgfpathcurveto{\pgfqpoint{2.988498in}{1.891504in}}{\pgfqpoint{2.985225in}{1.899404in}}{\pgfqpoint{2.979402in}{1.905228in}}%
\pgfpathcurveto{\pgfqpoint{2.973578in}{1.911052in}}{\pgfqpoint{2.965678in}{1.914325in}}{\pgfqpoint{2.957441in}{1.914325in}}%
\pgfpathcurveto{\pgfqpoint{2.949205in}{1.914325in}}{\pgfqpoint{2.941305in}{1.911052in}}{\pgfqpoint{2.935481in}{1.905228in}}%
\pgfpathcurveto{\pgfqpoint{2.929657in}{1.899404in}}{\pgfqpoint{2.926385in}{1.891504in}}{\pgfqpoint{2.926385in}{1.883268in}}%
\pgfpathcurveto{\pgfqpoint{2.926385in}{1.875032in}}{\pgfqpoint{2.929657in}{1.867132in}}{\pgfqpoint{2.935481in}{1.861308in}}%
\pgfpathcurveto{\pgfqpoint{2.941305in}{1.855484in}}{\pgfqpoint{2.949205in}{1.852212in}}{\pgfqpoint{2.957441in}{1.852212in}}%
\pgfpathclose%
\pgfusepath{stroke,fill}%
\end{pgfscope}%
\begin{pgfscope}%
\pgfpathrectangle{\pgfqpoint{0.100000in}{0.212622in}}{\pgfqpoint{3.696000in}{3.696000in}}%
\pgfusepath{clip}%
\pgfsetbuttcap%
\pgfsetroundjoin%
\definecolor{currentfill}{rgb}{0.121569,0.466667,0.705882}%
\pgfsetfillcolor{currentfill}%
\pgfsetfillopacity{0.742695}%
\pgfsetlinewidth{1.003750pt}%
\definecolor{currentstroke}{rgb}{0.121569,0.466667,0.705882}%
\pgfsetstrokecolor{currentstroke}%
\pgfsetstrokeopacity{0.742695}%
\pgfsetdash{}{0pt}%
\pgfpathmoveto{\pgfqpoint{2.956129in}{1.851182in}}%
\pgfpathcurveto{\pgfqpoint{2.964365in}{1.851182in}}{\pgfqpoint{2.972265in}{1.854454in}}{\pgfqpoint{2.978089in}{1.860278in}}%
\pgfpathcurveto{\pgfqpoint{2.983913in}{1.866102in}}{\pgfqpoint{2.987185in}{1.874002in}}{\pgfqpoint{2.987185in}{1.882238in}}%
\pgfpathcurveto{\pgfqpoint{2.987185in}{1.890475in}}{\pgfqpoint{2.983913in}{1.898375in}}{\pgfqpoint{2.978089in}{1.904199in}}%
\pgfpathcurveto{\pgfqpoint{2.972265in}{1.910022in}}{\pgfqpoint{2.964365in}{1.913295in}}{\pgfqpoint{2.956129in}{1.913295in}}%
\pgfpathcurveto{\pgfqpoint{2.947893in}{1.913295in}}{\pgfqpoint{2.939993in}{1.910022in}}{\pgfqpoint{2.934169in}{1.904199in}}%
\pgfpathcurveto{\pgfqpoint{2.928345in}{1.898375in}}{\pgfqpoint{2.925072in}{1.890475in}}{\pgfqpoint{2.925072in}{1.882238in}}%
\pgfpathcurveto{\pgfqpoint{2.925072in}{1.874002in}}{\pgfqpoint{2.928345in}{1.866102in}}{\pgfqpoint{2.934169in}{1.860278in}}%
\pgfpathcurveto{\pgfqpoint{2.939993in}{1.854454in}}{\pgfqpoint{2.947893in}{1.851182in}}{\pgfqpoint{2.956129in}{1.851182in}}%
\pgfpathclose%
\pgfusepath{stroke,fill}%
\end{pgfscope}%
\begin{pgfscope}%
\pgfpathrectangle{\pgfqpoint{0.100000in}{0.212622in}}{\pgfqpoint{3.696000in}{3.696000in}}%
\pgfusepath{clip}%
\pgfsetbuttcap%
\pgfsetroundjoin%
\definecolor{currentfill}{rgb}{0.121569,0.466667,0.705882}%
\pgfsetfillcolor{currentfill}%
\pgfsetfillopacity{0.743961}%
\pgfsetlinewidth{1.003750pt}%
\definecolor{currentstroke}{rgb}{0.121569,0.466667,0.705882}%
\pgfsetstrokecolor{currentstroke}%
\pgfsetstrokeopacity{0.743961}%
\pgfsetdash{}{0pt}%
\pgfpathmoveto{\pgfqpoint{2.953164in}{1.848243in}}%
\pgfpathcurveto{\pgfqpoint{2.961400in}{1.848243in}}{\pgfqpoint{2.969300in}{1.851516in}}{\pgfqpoint{2.975124in}{1.857340in}}%
\pgfpathcurveto{\pgfqpoint{2.980948in}{1.863163in}}{\pgfqpoint{2.984220in}{1.871063in}}{\pgfqpoint{2.984220in}{1.879300in}}%
\pgfpathcurveto{\pgfqpoint{2.984220in}{1.887536in}}{\pgfqpoint{2.980948in}{1.895436in}}{\pgfqpoint{2.975124in}{1.901260in}}%
\pgfpathcurveto{\pgfqpoint{2.969300in}{1.907084in}}{\pgfqpoint{2.961400in}{1.910356in}}{\pgfqpoint{2.953164in}{1.910356in}}%
\pgfpathcurveto{\pgfqpoint{2.944928in}{1.910356in}}{\pgfqpoint{2.937028in}{1.907084in}}{\pgfqpoint{2.931204in}{1.901260in}}%
\pgfpathcurveto{\pgfqpoint{2.925380in}{1.895436in}}{\pgfqpoint{2.922107in}{1.887536in}}{\pgfqpoint{2.922107in}{1.879300in}}%
\pgfpathcurveto{\pgfqpoint{2.922107in}{1.871063in}}{\pgfqpoint{2.925380in}{1.863163in}}{\pgfqpoint{2.931204in}{1.857340in}}%
\pgfpathcurveto{\pgfqpoint{2.937028in}{1.851516in}}{\pgfqpoint{2.944928in}{1.848243in}}{\pgfqpoint{2.953164in}{1.848243in}}%
\pgfpathclose%
\pgfusepath{stroke,fill}%
\end{pgfscope}%
\begin{pgfscope}%
\pgfpathrectangle{\pgfqpoint{0.100000in}{0.212622in}}{\pgfqpoint{3.696000in}{3.696000in}}%
\pgfusepath{clip}%
\pgfsetbuttcap%
\pgfsetroundjoin%
\definecolor{currentfill}{rgb}{0.121569,0.466667,0.705882}%
\pgfsetfillcolor{currentfill}%
\pgfsetfillopacity{0.744796}%
\pgfsetlinewidth{1.003750pt}%
\definecolor{currentstroke}{rgb}{0.121569,0.466667,0.705882}%
\pgfsetstrokecolor{currentstroke}%
\pgfsetstrokeopacity{0.744796}%
\pgfsetdash{}{0pt}%
\pgfpathmoveto{\pgfqpoint{2.950665in}{1.848986in}}%
\pgfpathcurveto{\pgfqpoint{2.958901in}{1.848986in}}{\pgfqpoint{2.966801in}{1.852258in}}{\pgfqpoint{2.972625in}{1.858082in}}%
\pgfpathcurveto{\pgfqpoint{2.978449in}{1.863906in}}{\pgfqpoint{2.981722in}{1.871806in}}{\pgfqpoint{2.981722in}{1.880042in}}%
\pgfpathcurveto{\pgfqpoint{2.981722in}{1.888278in}}{\pgfqpoint{2.978449in}{1.896178in}}{\pgfqpoint{2.972625in}{1.902002in}}%
\pgfpathcurveto{\pgfqpoint{2.966801in}{1.907826in}}{\pgfqpoint{2.958901in}{1.911099in}}{\pgfqpoint{2.950665in}{1.911099in}}%
\pgfpathcurveto{\pgfqpoint{2.942429in}{1.911099in}}{\pgfqpoint{2.934529in}{1.907826in}}{\pgfqpoint{2.928705in}{1.902002in}}%
\pgfpathcurveto{\pgfqpoint{2.922881in}{1.896178in}}{\pgfqpoint{2.919609in}{1.888278in}}{\pgfqpoint{2.919609in}{1.880042in}}%
\pgfpathcurveto{\pgfqpoint{2.919609in}{1.871806in}}{\pgfqpoint{2.922881in}{1.863906in}}{\pgfqpoint{2.928705in}{1.858082in}}%
\pgfpathcurveto{\pgfqpoint{2.934529in}{1.852258in}}{\pgfqpoint{2.942429in}{1.848986in}}{\pgfqpoint{2.950665in}{1.848986in}}%
\pgfpathclose%
\pgfusepath{stroke,fill}%
\end{pgfscope}%
\begin{pgfscope}%
\pgfpathrectangle{\pgfqpoint{0.100000in}{0.212622in}}{\pgfqpoint{3.696000in}{3.696000in}}%
\pgfusepath{clip}%
\pgfsetbuttcap%
\pgfsetroundjoin%
\definecolor{currentfill}{rgb}{0.121569,0.466667,0.705882}%
\pgfsetfillcolor{currentfill}%
\pgfsetfillopacity{0.745989}%
\pgfsetlinewidth{1.003750pt}%
\definecolor{currentstroke}{rgb}{0.121569,0.466667,0.705882}%
\pgfsetstrokecolor{currentstroke}%
\pgfsetstrokeopacity{0.745989}%
\pgfsetdash{}{0pt}%
\pgfpathmoveto{\pgfqpoint{2.949602in}{1.848025in}}%
\pgfpathcurveto{\pgfqpoint{2.957838in}{1.848025in}}{\pgfqpoint{2.965739in}{1.851297in}}{\pgfqpoint{2.971562in}{1.857121in}}%
\pgfpathcurveto{\pgfqpoint{2.977386in}{1.862945in}}{\pgfqpoint{2.980659in}{1.870845in}}{\pgfqpoint{2.980659in}{1.879081in}}%
\pgfpathcurveto{\pgfqpoint{2.980659in}{1.887318in}}{\pgfqpoint{2.977386in}{1.895218in}}{\pgfqpoint{2.971562in}{1.901042in}}%
\pgfpathcurveto{\pgfqpoint{2.965739in}{1.906865in}}{\pgfqpoint{2.957838in}{1.910138in}}{\pgfqpoint{2.949602in}{1.910138in}}%
\pgfpathcurveto{\pgfqpoint{2.941366in}{1.910138in}}{\pgfqpoint{2.933466in}{1.906865in}}{\pgfqpoint{2.927642in}{1.901042in}}%
\pgfpathcurveto{\pgfqpoint{2.921818in}{1.895218in}}{\pgfqpoint{2.918546in}{1.887318in}}{\pgfqpoint{2.918546in}{1.879081in}}%
\pgfpathcurveto{\pgfqpoint{2.918546in}{1.870845in}}{\pgfqpoint{2.921818in}{1.862945in}}{\pgfqpoint{2.927642in}{1.857121in}}%
\pgfpathcurveto{\pgfqpoint{2.933466in}{1.851297in}}{\pgfqpoint{2.941366in}{1.848025in}}{\pgfqpoint{2.949602in}{1.848025in}}%
\pgfpathclose%
\pgfusepath{stroke,fill}%
\end{pgfscope}%
\begin{pgfscope}%
\pgfpathrectangle{\pgfqpoint{0.100000in}{0.212622in}}{\pgfqpoint{3.696000in}{3.696000in}}%
\pgfusepath{clip}%
\pgfsetbuttcap%
\pgfsetroundjoin%
\definecolor{currentfill}{rgb}{0.121569,0.466667,0.705882}%
\pgfsetfillcolor{currentfill}%
\pgfsetfillopacity{0.746460}%
\pgfsetlinewidth{1.003750pt}%
\definecolor{currentstroke}{rgb}{0.121569,0.466667,0.705882}%
\pgfsetstrokecolor{currentstroke}%
\pgfsetstrokeopacity{0.746460}%
\pgfsetdash{}{0pt}%
\pgfpathmoveto{\pgfqpoint{2.948531in}{1.846568in}}%
\pgfpathcurveto{\pgfqpoint{2.956767in}{1.846568in}}{\pgfqpoint{2.964667in}{1.849840in}}{\pgfqpoint{2.970491in}{1.855664in}}%
\pgfpathcurveto{\pgfqpoint{2.976315in}{1.861488in}}{\pgfqpoint{2.979587in}{1.869388in}}{\pgfqpoint{2.979587in}{1.877624in}}%
\pgfpathcurveto{\pgfqpoint{2.979587in}{1.885861in}}{\pgfqpoint{2.976315in}{1.893761in}}{\pgfqpoint{2.970491in}{1.899585in}}%
\pgfpathcurveto{\pgfqpoint{2.964667in}{1.905409in}}{\pgfqpoint{2.956767in}{1.908681in}}{\pgfqpoint{2.948531in}{1.908681in}}%
\pgfpathcurveto{\pgfqpoint{2.940294in}{1.908681in}}{\pgfqpoint{2.932394in}{1.905409in}}{\pgfqpoint{2.926570in}{1.899585in}}%
\pgfpathcurveto{\pgfqpoint{2.920747in}{1.893761in}}{\pgfqpoint{2.917474in}{1.885861in}}{\pgfqpoint{2.917474in}{1.877624in}}%
\pgfpathcurveto{\pgfqpoint{2.917474in}{1.869388in}}{\pgfqpoint{2.920747in}{1.861488in}}{\pgfqpoint{2.926570in}{1.855664in}}%
\pgfpathcurveto{\pgfqpoint{2.932394in}{1.849840in}}{\pgfqpoint{2.940294in}{1.846568in}}{\pgfqpoint{2.948531in}{1.846568in}}%
\pgfpathclose%
\pgfusepath{stroke,fill}%
\end{pgfscope}%
\begin{pgfscope}%
\pgfpathrectangle{\pgfqpoint{0.100000in}{0.212622in}}{\pgfqpoint{3.696000in}{3.696000in}}%
\pgfusepath{clip}%
\pgfsetbuttcap%
\pgfsetroundjoin%
\definecolor{currentfill}{rgb}{0.121569,0.466667,0.705882}%
\pgfsetfillcolor{currentfill}%
\pgfsetfillopacity{0.747427}%
\pgfsetlinewidth{1.003750pt}%
\definecolor{currentstroke}{rgb}{0.121569,0.466667,0.705882}%
\pgfsetstrokecolor{currentstroke}%
\pgfsetstrokeopacity{0.747427}%
\pgfsetdash{}{0pt}%
\pgfpathmoveto{\pgfqpoint{2.945946in}{1.847377in}}%
\pgfpathcurveto{\pgfqpoint{2.954182in}{1.847377in}}{\pgfqpoint{2.962082in}{1.850649in}}{\pgfqpoint{2.967906in}{1.856473in}}%
\pgfpathcurveto{\pgfqpoint{2.973730in}{1.862297in}}{\pgfqpoint{2.977002in}{1.870197in}}{\pgfqpoint{2.977002in}{1.878433in}}%
\pgfpathcurveto{\pgfqpoint{2.977002in}{1.886669in}}{\pgfqpoint{2.973730in}{1.894569in}}{\pgfqpoint{2.967906in}{1.900393in}}%
\pgfpathcurveto{\pgfqpoint{2.962082in}{1.906217in}}{\pgfqpoint{2.954182in}{1.909490in}}{\pgfqpoint{2.945946in}{1.909490in}}%
\pgfpathcurveto{\pgfqpoint{2.937709in}{1.909490in}}{\pgfqpoint{2.929809in}{1.906217in}}{\pgfqpoint{2.923985in}{1.900393in}}%
\pgfpathcurveto{\pgfqpoint{2.918161in}{1.894569in}}{\pgfqpoint{2.914889in}{1.886669in}}{\pgfqpoint{2.914889in}{1.878433in}}%
\pgfpathcurveto{\pgfqpoint{2.914889in}{1.870197in}}{\pgfqpoint{2.918161in}{1.862297in}}{\pgfqpoint{2.923985in}{1.856473in}}%
\pgfpathcurveto{\pgfqpoint{2.929809in}{1.850649in}}{\pgfqpoint{2.937709in}{1.847377in}}{\pgfqpoint{2.945946in}{1.847377in}}%
\pgfpathclose%
\pgfusepath{stroke,fill}%
\end{pgfscope}%
\begin{pgfscope}%
\pgfpathrectangle{\pgfqpoint{0.100000in}{0.212622in}}{\pgfqpoint{3.696000in}{3.696000in}}%
\pgfusepath{clip}%
\pgfsetbuttcap%
\pgfsetroundjoin%
\definecolor{currentfill}{rgb}{0.121569,0.466667,0.705882}%
\pgfsetfillcolor{currentfill}%
\pgfsetfillopacity{0.748386}%
\pgfsetlinewidth{1.003750pt}%
\definecolor{currentstroke}{rgb}{0.121569,0.466667,0.705882}%
\pgfsetstrokecolor{currentstroke}%
\pgfsetstrokeopacity{0.748386}%
\pgfsetdash{}{0pt}%
\pgfpathmoveto{\pgfqpoint{2.944895in}{1.843524in}}%
\pgfpathcurveto{\pgfqpoint{2.953131in}{1.843524in}}{\pgfqpoint{2.961031in}{1.846796in}}{\pgfqpoint{2.966855in}{1.852620in}}%
\pgfpathcurveto{\pgfqpoint{2.972679in}{1.858444in}}{\pgfqpoint{2.975951in}{1.866344in}}{\pgfqpoint{2.975951in}{1.874580in}}%
\pgfpathcurveto{\pgfqpoint{2.975951in}{1.882816in}}{\pgfqpoint{2.972679in}{1.890716in}}{\pgfqpoint{2.966855in}{1.896540in}}%
\pgfpathcurveto{\pgfqpoint{2.961031in}{1.902364in}}{\pgfqpoint{2.953131in}{1.905637in}}{\pgfqpoint{2.944895in}{1.905637in}}%
\pgfpathcurveto{\pgfqpoint{2.936659in}{1.905637in}}{\pgfqpoint{2.928759in}{1.902364in}}{\pgfqpoint{2.922935in}{1.896540in}}%
\pgfpathcurveto{\pgfqpoint{2.917111in}{1.890716in}}{\pgfqpoint{2.913838in}{1.882816in}}{\pgfqpoint{2.913838in}{1.874580in}}%
\pgfpathcurveto{\pgfqpoint{2.913838in}{1.866344in}}{\pgfqpoint{2.917111in}{1.858444in}}{\pgfqpoint{2.922935in}{1.852620in}}%
\pgfpathcurveto{\pgfqpoint{2.928759in}{1.846796in}}{\pgfqpoint{2.936659in}{1.843524in}}{\pgfqpoint{2.944895in}{1.843524in}}%
\pgfpathclose%
\pgfusepath{stroke,fill}%
\end{pgfscope}%
\begin{pgfscope}%
\pgfpathrectangle{\pgfqpoint{0.100000in}{0.212622in}}{\pgfqpoint{3.696000in}{3.696000in}}%
\pgfusepath{clip}%
\pgfsetbuttcap%
\pgfsetroundjoin%
\definecolor{currentfill}{rgb}{0.121569,0.466667,0.705882}%
\pgfsetfillcolor{currentfill}%
\pgfsetfillopacity{0.749416}%
\pgfsetlinewidth{1.003750pt}%
\definecolor{currentstroke}{rgb}{0.121569,0.466667,0.705882}%
\pgfsetstrokecolor{currentstroke}%
\pgfsetstrokeopacity{0.749416}%
\pgfsetdash{}{0pt}%
\pgfpathmoveto{\pgfqpoint{2.943395in}{1.839276in}}%
\pgfpathcurveto{\pgfqpoint{2.951632in}{1.839276in}}{\pgfqpoint{2.959532in}{1.842548in}}{\pgfqpoint{2.965356in}{1.848372in}}%
\pgfpathcurveto{\pgfqpoint{2.971180in}{1.854196in}}{\pgfqpoint{2.974452in}{1.862096in}}{\pgfqpoint{2.974452in}{1.870332in}}%
\pgfpathcurveto{\pgfqpoint{2.974452in}{1.878568in}}{\pgfqpoint{2.971180in}{1.886468in}}{\pgfqpoint{2.965356in}{1.892292in}}%
\pgfpathcurveto{\pgfqpoint{2.959532in}{1.898116in}}{\pgfqpoint{2.951632in}{1.901389in}}{\pgfqpoint{2.943395in}{1.901389in}}%
\pgfpathcurveto{\pgfqpoint{2.935159in}{1.901389in}}{\pgfqpoint{2.927259in}{1.898116in}}{\pgfqpoint{2.921435in}{1.892292in}}%
\pgfpathcurveto{\pgfqpoint{2.915611in}{1.886468in}}{\pgfqpoint{2.912339in}{1.878568in}}{\pgfqpoint{2.912339in}{1.870332in}}%
\pgfpathcurveto{\pgfqpoint{2.912339in}{1.862096in}}{\pgfqpoint{2.915611in}{1.854196in}}{\pgfqpoint{2.921435in}{1.848372in}}%
\pgfpathcurveto{\pgfqpoint{2.927259in}{1.842548in}}{\pgfqpoint{2.935159in}{1.839276in}}{\pgfqpoint{2.943395in}{1.839276in}}%
\pgfpathclose%
\pgfusepath{stroke,fill}%
\end{pgfscope}%
\begin{pgfscope}%
\pgfpathrectangle{\pgfqpoint{0.100000in}{0.212622in}}{\pgfqpoint{3.696000in}{3.696000in}}%
\pgfusepath{clip}%
\pgfsetbuttcap%
\pgfsetroundjoin%
\definecolor{currentfill}{rgb}{0.121569,0.466667,0.705882}%
\pgfsetfillcolor{currentfill}%
\pgfsetfillopacity{0.751828}%
\pgfsetlinewidth{1.003750pt}%
\definecolor{currentstroke}{rgb}{0.121569,0.466667,0.705882}%
\pgfsetstrokecolor{currentstroke}%
\pgfsetstrokeopacity{0.751828}%
\pgfsetdash{}{0pt}%
\pgfpathmoveto{\pgfqpoint{2.937794in}{1.842546in}}%
\pgfpathcurveto{\pgfqpoint{2.946031in}{1.842546in}}{\pgfqpoint{2.953931in}{1.845818in}}{\pgfqpoint{2.959755in}{1.851642in}}%
\pgfpathcurveto{\pgfqpoint{2.965579in}{1.857466in}}{\pgfqpoint{2.968851in}{1.865366in}}{\pgfqpoint{2.968851in}{1.873602in}}%
\pgfpathcurveto{\pgfqpoint{2.968851in}{1.881838in}}{\pgfqpoint{2.965579in}{1.889738in}}{\pgfqpoint{2.959755in}{1.895562in}}%
\pgfpathcurveto{\pgfqpoint{2.953931in}{1.901386in}}{\pgfqpoint{2.946031in}{1.904659in}}{\pgfqpoint{2.937794in}{1.904659in}}%
\pgfpathcurveto{\pgfqpoint{2.929558in}{1.904659in}}{\pgfqpoint{2.921658in}{1.901386in}}{\pgfqpoint{2.915834in}{1.895562in}}%
\pgfpathcurveto{\pgfqpoint{2.910010in}{1.889738in}}{\pgfqpoint{2.906738in}{1.881838in}}{\pgfqpoint{2.906738in}{1.873602in}}%
\pgfpathcurveto{\pgfqpoint{2.906738in}{1.865366in}}{\pgfqpoint{2.910010in}{1.857466in}}{\pgfqpoint{2.915834in}{1.851642in}}%
\pgfpathcurveto{\pgfqpoint{2.921658in}{1.845818in}}{\pgfqpoint{2.929558in}{1.842546in}}{\pgfqpoint{2.937794in}{1.842546in}}%
\pgfpathclose%
\pgfusepath{stroke,fill}%
\end{pgfscope}%
\begin{pgfscope}%
\pgfpathrectangle{\pgfqpoint{0.100000in}{0.212622in}}{\pgfqpoint{3.696000in}{3.696000in}}%
\pgfusepath{clip}%
\pgfsetbuttcap%
\pgfsetroundjoin%
\definecolor{currentfill}{rgb}{0.121569,0.466667,0.705882}%
\pgfsetfillcolor{currentfill}%
\pgfsetfillopacity{0.753303}%
\pgfsetlinewidth{1.003750pt}%
\definecolor{currentstroke}{rgb}{0.121569,0.466667,0.705882}%
\pgfsetstrokecolor{currentstroke}%
\pgfsetstrokeopacity{0.753303}%
\pgfsetdash{}{0pt}%
\pgfpathmoveto{\pgfqpoint{2.934523in}{1.834913in}}%
\pgfpathcurveto{\pgfqpoint{2.942759in}{1.834913in}}{\pgfqpoint{2.950659in}{1.838185in}}{\pgfqpoint{2.956483in}{1.844009in}}%
\pgfpathcurveto{\pgfqpoint{2.962307in}{1.849833in}}{\pgfqpoint{2.965579in}{1.857733in}}{\pgfqpoint{2.965579in}{1.865969in}}%
\pgfpathcurveto{\pgfqpoint{2.965579in}{1.874206in}}{\pgfqpoint{2.962307in}{1.882106in}}{\pgfqpoint{2.956483in}{1.887930in}}%
\pgfpathcurveto{\pgfqpoint{2.950659in}{1.893753in}}{\pgfqpoint{2.942759in}{1.897026in}}{\pgfqpoint{2.934523in}{1.897026in}}%
\pgfpathcurveto{\pgfqpoint{2.926287in}{1.897026in}}{\pgfqpoint{2.918387in}{1.893753in}}{\pgfqpoint{2.912563in}{1.887930in}}%
\pgfpathcurveto{\pgfqpoint{2.906739in}{1.882106in}}{\pgfqpoint{2.903466in}{1.874206in}}{\pgfqpoint{2.903466in}{1.865969in}}%
\pgfpathcurveto{\pgfqpoint{2.903466in}{1.857733in}}{\pgfqpoint{2.906739in}{1.849833in}}{\pgfqpoint{2.912563in}{1.844009in}}%
\pgfpathcurveto{\pgfqpoint{2.918387in}{1.838185in}}{\pgfqpoint{2.926287in}{1.834913in}}{\pgfqpoint{2.934523in}{1.834913in}}%
\pgfpathclose%
\pgfusepath{stroke,fill}%
\end{pgfscope}%
\begin{pgfscope}%
\pgfpathrectangle{\pgfqpoint{0.100000in}{0.212622in}}{\pgfqpoint{3.696000in}{3.696000in}}%
\pgfusepath{clip}%
\pgfsetbuttcap%
\pgfsetroundjoin%
\definecolor{currentfill}{rgb}{0.121569,0.466667,0.705882}%
\pgfsetfillcolor{currentfill}%
\pgfsetfillopacity{0.754335}%
\pgfsetlinewidth{1.003750pt}%
\definecolor{currentstroke}{rgb}{0.121569,0.466667,0.705882}%
\pgfsetstrokecolor{currentstroke}%
\pgfsetstrokeopacity{0.754335}%
\pgfsetdash{}{0pt}%
\pgfpathmoveto{\pgfqpoint{2.932805in}{1.832189in}}%
\pgfpathcurveto{\pgfqpoint{2.941041in}{1.832189in}}{\pgfqpoint{2.948942in}{1.835462in}}{\pgfqpoint{2.954765in}{1.841286in}}%
\pgfpathcurveto{\pgfqpoint{2.960589in}{1.847110in}}{\pgfqpoint{2.963862in}{1.855010in}}{\pgfqpoint{2.963862in}{1.863246in}}%
\pgfpathcurveto{\pgfqpoint{2.963862in}{1.871482in}}{\pgfqpoint{2.960589in}{1.879382in}}{\pgfqpoint{2.954765in}{1.885206in}}%
\pgfpathcurveto{\pgfqpoint{2.948942in}{1.891030in}}{\pgfqpoint{2.941041in}{1.894302in}}{\pgfqpoint{2.932805in}{1.894302in}}%
\pgfpathcurveto{\pgfqpoint{2.924569in}{1.894302in}}{\pgfqpoint{2.916669in}{1.891030in}}{\pgfqpoint{2.910845in}{1.885206in}}%
\pgfpathcurveto{\pgfqpoint{2.905021in}{1.879382in}}{\pgfqpoint{2.901749in}{1.871482in}}{\pgfqpoint{2.901749in}{1.863246in}}%
\pgfpathcurveto{\pgfqpoint{2.901749in}{1.855010in}}{\pgfqpoint{2.905021in}{1.847110in}}{\pgfqpoint{2.910845in}{1.841286in}}%
\pgfpathcurveto{\pgfqpoint{2.916669in}{1.835462in}}{\pgfqpoint{2.924569in}{1.832189in}}{\pgfqpoint{2.932805in}{1.832189in}}%
\pgfpathclose%
\pgfusepath{stroke,fill}%
\end{pgfscope}%
\begin{pgfscope}%
\pgfpathrectangle{\pgfqpoint{0.100000in}{0.212622in}}{\pgfqpoint{3.696000in}{3.696000in}}%
\pgfusepath{clip}%
\pgfsetbuttcap%
\pgfsetroundjoin%
\definecolor{currentfill}{rgb}{0.121569,0.466667,0.705882}%
\pgfsetfillcolor{currentfill}%
\pgfsetfillopacity{0.755893}%
\pgfsetlinewidth{1.003750pt}%
\definecolor{currentstroke}{rgb}{0.121569,0.466667,0.705882}%
\pgfsetstrokecolor{currentstroke}%
\pgfsetstrokeopacity{0.755893}%
\pgfsetdash{}{0pt}%
\pgfpathmoveto{\pgfqpoint{2.929501in}{1.829075in}}%
\pgfpathcurveto{\pgfqpoint{2.937738in}{1.829075in}}{\pgfqpoint{2.945638in}{1.832347in}}{\pgfqpoint{2.951462in}{1.838171in}}%
\pgfpathcurveto{\pgfqpoint{2.957286in}{1.843995in}}{\pgfqpoint{2.960558in}{1.851895in}}{\pgfqpoint{2.960558in}{1.860132in}}%
\pgfpathcurveto{\pgfqpoint{2.960558in}{1.868368in}}{\pgfqpoint{2.957286in}{1.876268in}}{\pgfqpoint{2.951462in}{1.882092in}}%
\pgfpathcurveto{\pgfqpoint{2.945638in}{1.887916in}}{\pgfqpoint{2.937738in}{1.891188in}}{\pgfqpoint{2.929501in}{1.891188in}}%
\pgfpathcurveto{\pgfqpoint{2.921265in}{1.891188in}}{\pgfqpoint{2.913365in}{1.887916in}}{\pgfqpoint{2.907541in}{1.882092in}}%
\pgfpathcurveto{\pgfqpoint{2.901717in}{1.876268in}}{\pgfqpoint{2.898445in}{1.868368in}}{\pgfqpoint{2.898445in}{1.860132in}}%
\pgfpathcurveto{\pgfqpoint{2.898445in}{1.851895in}}{\pgfqpoint{2.901717in}{1.843995in}}{\pgfqpoint{2.907541in}{1.838171in}}%
\pgfpathcurveto{\pgfqpoint{2.913365in}{1.832347in}}{\pgfqpoint{2.921265in}{1.829075in}}{\pgfqpoint{2.929501in}{1.829075in}}%
\pgfpathclose%
\pgfusepath{stroke,fill}%
\end{pgfscope}%
\begin{pgfscope}%
\pgfpathrectangle{\pgfqpoint{0.100000in}{0.212622in}}{\pgfqpoint{3.696000in}{3.696000in}}%
\pgfusepath{clip}%
\pgfsetbuttcap%
\pgfsetroundjoin%
\definecolor{currentfill}{rgb}{0.121569,0.466667,0.705882}%
\pgfsetfillcolor{currentfill}%
\pgfsetfillopacity{0.757841}%
\pgfsetlinewidth{1.003750pt}%
\definecolor{currentstroke}{rgb}{0.121569,0.466667,0.705882}%
\pgfsetstrokecolor{currentstroke}%
\pgfsetstrokeopacity{0.757841}%
\pgfsetdash{}{0pt}%
\pgfpathmoveto{\pgfqpoint{2.923861in}{1.830839in}}%
\pgfpathcurveto{\pgfqpoint{2.932097in}{1.830839in}}{\pgfqpoint{2.939997in}{1.834111in}}{\pgfqpoint{2.945821in}{1.839935in}}%
\pgfpathcurveto{\pgfqpoint{2.951645in}{1.845759in}}{\pgfqpoint{2.954917in}{1.853659in}}{\pgfqpoint{2.954917in}{1.861895in}}%
\pgfpathcurveto{\pgfqpoint{2.954917in}{1.870132in}}{\pgfqpoint{2.951645in}{1.878032in}}{\pgfqpoint{2.945821in}{1.883856in}}%
\pgfpathcurveto{\pgfqpoint{2.939997in}{1.889679in}}{\pgfqpoint{2.932097in}{1.892952in}}{\pgfqpoint{2.923861in}{1.892952in}}%
\pgfpathcurveto{\pgfqpoint{2.915625in}{1.892952in}}{\pgfqpoint{2.907725in}{1.889679in}}{\pgfqpoint{2.901901in}{1.883856in}}%
\pgfpathcurveto{\pgfqpoint{2.896077in}{1.878032in}}{\pgfqpoint{2.892804in}{1.870132in}}{\pgfqpoint{2.892804in}{1.861895in}}%
\pgfpathcurveto{\pgfqpoint{2.892804in}{1.853659in}}{\pgfqpoint{2.896077in}{1.845759in}}{\pgfqpoint{2.901901in}{1.839935in}}%
\pgfpathcurveto{\pgfqpoint{2.907725in}{1.834111in}}{\pgfqpoint{2.915625in}{1.830839in}}{\pgfqpoint{2.923861in}{1.830839in}}%
\pgfpathclose%
\pgfusepath{stroke,fill}%
\end{pgfscope}%
\begin{pgfscope}%
\pgfpathrectangle{\pgfqpoint{0.100000in}{0.212622in}}{\pgfqpoint{3.696000in}{3.696000in}}%
\pgfusepath{clip}%
\pgfsetbuttcap%
\pgfsetroundjoin%
\definecolor{currentfill}{rgb}{0.121569,0.466667,0.705882}%
\pgfsetfillcolor{currentfill}%
\pgfsetfillopacity{0.759929}%
\pgfsetlinewidth{1.003750pt}%
\definecolor{currentstroke}{rgb}{0.121569,0.466667,0.705882}%
\pgfsetstrokecolor{currentstroke}%
\pgfsetstrokeopacity{0.759929}%
\pgfsetdash{}{0pt}%
\pgfpathmoveto{\pgfqpoint{2.921462in}{1.826719in}}%
\pgfpathcurveto{\pgfqpoint{2.929698in}{1.826719in}}{\pgfqpoint{2.937598in}{1.829992in}}{\pgfqpoint{2.943422in}{1.835816in}}%
\pgfpathcurveto{\pgfqpoint{2.949246in}{1.841640in}}{\pgfqpoint{2.952518in}{1.849540in}}{\pgfqpoint{2.952518in}{1.857776in}}%
\pgfpathcurveto{\pgfqpoint{2.952518in}{1.866012in}}{\pgfqpoint{2.949246in}{1.873912in}}{\pgfqpoint{2.943422in}{1.879736in}}%
\pgfpathcurveto{\pgfqpoint{2.937598in}{1.885560in}}{\pgfqpoint{2.929698in}{1.888832in}}{\pgfqpoint{2.921462in}{1.888832in}}%
\pgfpathcurveto{\pgfqpoint{2.913225in}{1.888832in}}{\pgfqpoint{2.905325in}{1.885560in}}{\pgfqpoint{2.899501in}{1.879736in}}%
\pgfpathcurveto{\pgfqpoint{2.893677in}{1.873912in}}{\pgfqpoint{2.890405in}{1.866012in}}{\pgfqpoint{2.890405in}{1.857776in}}%
\pgfpathcurveto{\pgfqpoint{2.890405in}{1.849540in}}{\pgfqpoint{2.893677in}{1.841640in}}{\pgfqpoint{2.899501in}{1.835816in}}%
\pgfpathcurveto{\pgfqpoint{2.905325in}{1.829992in}}{\pgfqpoint{2.913225in}{1.826719in}}{\pgfqpoint{2.921462in}{1.826719in}}%
\pgfpathclose%
\pgfusepath{stroke,fill}%
\end{pgfscope}%
\begin{pgfscope}%
\pgfpathrectangle{\pgfqpoint{0.100000in}{0.212622in}}{\pgfqpoint{3.696000in}{3.696000in}}%
\pgfusepath{clip}%
\pgfsetbuttcap%
\pgfsetroundjoin%
\definecolor{currentfill}{rgb}{0.121569,0.466667,0.705882}%
\pgfsetfillcolor{currentfill}%
\pgfsetfillopacity{0.762082}%
\pgfsetlinewidth{1.003750pt}%
\definecolor{currentstroke}{rgb}{0.121569,0.466667,0.705882}%
\pgfsetstrokecolor{currentstroke}%
\pgfsetstrokeopacity{0.762082}%
\pgfsetdash{}{0pt}%
\pgfpathmoveto{\pgfqpoint{2.918241in}{1.822478in}}%
\pgfpathcurveto{\pgfqpoint{2.926477in}{1.822478in}}{\pgfqpoint{2.934377in}{1.825750in}}{\pgfqpoint{2.940201in}{1.831574in}}%
\pgfpathcurveto{\pgfqpoint{2.946025in}{1.837398in}}{\pgfqpoint{2.949297in}{1.845298in}}{\pgfqpoint{2.949297in}{1.853534in}}%
\pgfpathcurveto{\pgfqpoint{2.949297in}{1.861770in}}{\pgfqpoint{2.946025in}{1.869670in}}{\pgfqpoint{2.940201in}{1.875494in}}%
\pgfpathcurveto{\pgfqpoint{2.934377in}{1.881318in}}{\pgfqpoint{2.926477in}{1.884591in}}{\pgfqpoint{2.918241in}{1.884591in}}%
\pgfpathcurveto{\pgfqpoint{2.910005in}{1.884591in}}{\pgfqpoint{2.902104in}{1.881318in}}{\pgfqpoint{2.896281in}{1.875494in}}%
\pgfpathcurveto{\pgfqpoint{2.890457in}{1.869670in}}{\pgfqpoint{2.887184in}{1.861770in}}{\pgfqpoint{2.887184in}{1.853534in}}%
\pgfpathcurveto{\pgfqpoint{2.887184in}{1.845298in}}{\pgfqpoint{2.890457in}{1.837398in}}{\pgfqpoint{2.896281in}{1.831574in}}%
\pgfpathcurveto{\pgfqpoint{2.902104in}{1.825750in}}{\pgfqpoint{2.910005in}{1.822478in}}{\pgfqpoint{2.918241in}{1.822478in}}%
\pgfpathclose%
\pgfusepath{stroke,fill}%
\end{pgfscope}%
\begin{pgfscope}%
\pgfpathrectangle{\pgfqpoint{0.100000in}{0.212622in}}{\pgfqpoint{3.696000in}{3.696000in}}%
\pgfusepath{clip}%
\pgfsetbuttcap%
\pgfsetroundjoin%
\definecolor{currentfill}{rgb}{0.121569,0.466667,0.705882}%
\pgfsetfillcolor{currentfill}%
\pgfsetfillopacity{0.765470}%
\pgfsetlinewidth{1.003750pt}%
\definecolor{currentstroke}{rgb}{0.121569,0.466667,0.705882}%
\pgfsetstrokecolor{currentstroke}%
\pgfsetstrokeopacity{0.765470}%
\pgfsetdash{}{0pt}%
\pgfpathmoveto{\pgfqpoint{2.910958in}{1.827548in}}%
\pgfpathcurveto{\pgfqpoint{2.919194in}{1.827548in}}{\pgfqpoint{2.927094in}{1.830820in}}{\pgfqpoint{2.932918in}{1.836644in}}%
\pgfpathcurveto{\pgfqpoint{2.938742in}{1.842468in}}{\pgfqpoint{2.942015in}{1.850368in}}{\pgfqpoint{2.942015in}{1.858604in}}%
\pgfpathcurveto{\pgfqpoint{2.942015in}{1.866840in}}{\pgfqpoint{2.938742in}{1.874740in}}{\pgfqpoint{2.932918in}{1.880564in}}%
\pgfpathcurveto{\pgfqpoint{2.927094in}{1.886388in}}{\pgfqpoint{2.919194in}{1.889661in}}{\pgfqpoint{2.910958in}{1.889661in}}%
\pgfpathcurveto{\pgfqpoint{2.902722in}{1.889661in}}{\pgfqpoint{2.894822in}{1.886388in}}{\pgfqpoint{2.888998in}{1.880564in}}%
\pgfpathcurveto{\pgfqpoint{2.883174in}{1.874740in}}{\pgfqpoint{2.879902in}{1.866840in}}{\pgfqpoint{2.879902in}{1.858604in}}%
\pgfpathcurveto{\pgfqpoint{2.879902in}{1.850368in}}{\pgfqpoint{2.883174in}{1.842468in}}{\pgfqpoint{2.888998in}{1.836644in}}%
\pgfpathcurveto{\pgfqpoint{2.894822in}{1.830820in}}{\pgfqpoint{2.902722in}{1.827548in}}{\pgfqpoint{2.910958in}{1.827548in}}%
\pgfpathclose%
\pgfusepath{stroke,fill}%
\end{pgfscope}%
\begin{pgfscope}%
\pgfpathrectangle{\pgfqpoint{0.100000in}{0.212622in}}{\pgfqpoint{3.696000in}{3.696000in}}%
\pgfusepath{clip}%
\pgfsetbuttcap%
\pgfsetroundjoin%
\definecolor{currentfill}{rgb}{0.121569,0.466667,0.705882}%
\pgfsetfillcolor{currentfill}%
\pgfsetfillopacity{0.767836}%
\pgfsetlinewidth{1.003750pt}%
\definecolor{currentstroke}{rgb}{0.121569,0.466667,0.705882}%
\pgfsetstrokecolor{currentstroke}%
\pgfsetstrokeopacity{0.767836}%
\pgfsetdash{}{0pt}%
\pgfpathmoveto{\pgfqpoint{2.908806in}{1.818826in}}%
\pgfpathcurveto{\pgfqpoint{2.917042in}{1.818826in}}{\pgfqpoint{2.924942in}{1.822098in}}{\pgfqpoint{2.930766in}{1.827922in}}%
\pgfpathcurveto{\pgfqpoint{2.936590in}{1.833746in}}{\pgfqpoint{2.939862in}{1.841646in}}{\pgfqpoint{2.939862in}{1.849883in}}%
\pgfpathcurveto{\pgfqpoint{2.939862in}{1.858119in}}{\pgfqpoint{2.936590in}{1.866019in}}{\pgfqpoint{2.930766in}{1.871843in}}%
\pgfpathcurveto{\pgfqpoint{2.924942in}{1.877667in}}{\pgfqpoint{2.917042in}{1.880939in}}{\pgfqpoint{2.908806in}{1.880939in}}%
\pgfpathcurveto{\pgfqpoint{2.900569in}{1.880939in}}{\pgfqpoint{2.892669in}{1.877667in}}{\pgfqpoint{2.886845in}{1.871843in}}%
\pgfpathcurveto{\pgfqpoint{2.881021in}{1.866019in}}{\pgfqpoint{2.877749in}{1.858119in}}{\pgfqpoint{2.877749in}{1.849883in}}%
\pgfpathcurveto{\pgfqpoint{2.877749in}{1.841646in}}{\pgfqpoint{2.881021in}{1.833746in}}{\pgfqpoint{2.886845in}{1.827922in}}%
\pgfpathcurveto{\pgfqpoint{2.892669in}{1.822098in}}{\pgfqpoint{2.900569in}{1.818826in}}{\pgfqpoint{2.908806in}{1.818826in}}%
\pgfpathclose%
\pgfusepath{stroke,fill}%
\end{pgfscope}%
\begin{pgfscope}%
\pgfpathrectangle{\pgfqpoint{0.100000in}{0.212622in}}{\pgfqpoint{3.696000in}{3.696000in}}%
\pgfusepath{clip}%
\pgfsetbuttcap%
\pgfsetroundjoin%
\definecolor{currentfill}{rgb}{0.121569,0.466667,0.705882}%
\pgfsetfillcolor{currentfill}%
\pgfsetfillopacity{0.770651}%
\pgfsetlinewidth{1.003750pt}%
\definecolor{currentstroke}{rgb}{0.121569,0.466667,0.705882}%
\pgfsetstrokecolor{currentstroke}%
\pgfsetstrokeopacity{0.770651}%
\pgfsetdash{}{0pt}%
\pgfpathmoveto{\pgfqpoint{2.905882in}{1.812818in}}%
\pgfpathcurveto{\pgfqpoint{2.914119in}{1.812818in}}{\pgfqpoint{2.922019in}{1.816090in}}{\pgfqpoint{2.927843in}{1.821914in}}%
\pgfpathcurveto{\pgfqpoint{2.933666in}{1.827738in}}{\pgfqpoint{2.936939in}{1.835638in}}{\pgfqpoint{2.936939in}{1.843874in}}%
\pgfpathcurveto{\pgfqpoint{2.936939in}{1.852110in}}{\pgfqpoint{2.933666in}{1.860011in}}{\pgfqpoint{2.927843in}{1.865834in}}%
\pgfpathcurveto{\pgfqpoint{2.922019in}{1.871658in}}{\pgfqpoint{2.914119in}{1.874931in}}{\pgfqpoint{2.905882in}{1.874931in}}%
\pgfpathcurveto{\pgfqpoint{2.897646in}{1.874931in}}{\pgfqpoint{2.889746in}{1.871658in}}{\pgfqpoint{2.883922in}{1.865834in}}%
\pgfpathcurveto{\pgfqpoint{2.878098in}{1.860011in}}{\pgfqpoint{2.874826in}{1.852110in}}{\pgfqpoint{2.874826in}{1.843874in}}%
\pgfpathcurveto{\pgfqpoint{2.874826in}{1.835638in}}{\pgfqpoint{2.878098in}{1.827738in}}{\pgfqpoint{2.883922in}{1.821914in}}%
\pgfpathcurveto{\pgfqpoint{2.889746in}{1.816090in}}{\pgfqpoint{2.897646in}{1.812818in}}{\pgfqpoint{2.905882in}{1.812818in}}%
\pgfpathclose%
\pgfusepath{stroke,fill}%
\end{pgfscope}%
\begin{pgfscope}%
\pgfpathrectangle{\pgfqpoint{0.100000in}{0.212622in}}{\pgfqpoint{3.696000in}{3.696000in}}%
\pgfusepath{clip}%
\pgfsetbuttcap%
\pgfsetroundjoin%
\definecolor{currentfill}{rgb}{0.121569,0.466667,0.705882}%
\pgfsetfillcolor{currentfill}%
\pgfsetfillopacity{0.773292}%
\pgfsetlinewidth{1.003750pt}%
\definecolor{currentstroke}{rgb}{0.121569,0.466667,0.705882}%
\pgfsetstrokecolor{currentstroke}%
\pgfsetstrokeopacity{0.773292}%
\pgfsetdash{}{0pt}%
\pgfpathmoveto{\pgfqpoint{2.897479in}{1.804688in}}%
\pgfpathcurveto{\pgfqpoint{2.905715in}{1.804688in}}{\pgfqpoint{2.913615in}{1.807960in}}{\pgfqpoint{2.919439in}{1.813784in}}%
\pgfpathcurveto{\pgfqpoint{2.925263in}{1.819608in}}{\pgfqpoint{2.928536in}{1.827508in}}{\pgfqpoint{2.928536in}{1.835745in}}%
\pgfpathcurveto{\pgfqpoint{2.928536in}{1.843981in}}{\pgfqpoint{2.925263in}{1.851881in}}{\pgfqpoint{2.919439in}{1.857705in}}%
\pgfpathcurveto{\pgfqpoint{2.913615in}{1.863529in}}{\pgfqpoint{2.905715in}{1.866801in}}{\pgfqpoint{2.897479in}{1.866801in}}%
\pgfpathcurveto{\pgfqpoint{2.889243in}{1.866801in}}{\pgfqpoint{2.881343in}{1.863529in}}{\pgfqpoint{2.875519in}{1.857705in}}%
\pgfpathcurveto{\pgfqpoint{2.869695in}{1.851881in}}{\pgfqpoint{2.866423in}{1.843981in}}{\pgfqpoint{2.866423in}{1.835745in}}%
\pgfpathcurveto{\pgfqpoint{2.866423in}{1.827508in}}{\pgfqpoint{2.869695in}{1.819608in}}{\pgfqpoint{2.875519in}{1.813784in}}%
\pgfpathcurveto{\pgfqpoint{2.881343in}{1.807960in}}{\pgfqpoint{2.889243in}{1.804688in}}{\pgfqpoint{2.897479in}{1.804688in}}%
\pgfpathclose%
\pgfusepath{stroke,fill}%
\end{pgfscope}%
\begin{pgfscope}%
\pgfpathrectangle{\pgfqpoint{0.100000in}{0.212622in}}{\pgfqpoint{3.696000in}{3.696000in}}%
\pgfusepath{clip}%
\pgfsetbuttcap%
\pgfsetroundjoin%
\definecolor{currentfill}{rgb}{0.121569,0.466667,0.705882}%
\pgfsetfillcolor{currentfill}%
\pgfsetfillopacity{0.775162}%
\pgfsetlinewidth{1.003750pt}%
\definecolor{currentstroke}{rgb}{0.121569,0.466667,0.705882}%
\pgfsetstrokecolor{currentstroke}%
\pgfsetstrokeopacity{0.775162}%
\pgfsetdash{}{0pt}%
\pgfpathmoveto{\pgfqpoint{2.892467in}{1.803718in}}%
\pgfpathcurveto{\pgfqpoint{2.900704in}{1.803718in}}{\pgfqpoint{2.908604in}{1.806991in}}{\pgfqpoint{2.914427in}{1.812815in}}%
\pgfpathcurveto{\pgfqpoint{2.920251in}{1.818639in}}{\pgfqpoint{2.923524in}{1.826539in}}{\pgfqpoint{2.923524in}{1.834775in}}%
\pgfpathcurveto{\pgfqpoint{2.923524in}{1.843011in}}{\pgfqpoint{2.920251in}{1.850911in}}{\pgfqpoint{2.914427in}{1.856735in}}%
\pgfpathcurveto{\pgfqpoint{2.908604in}{1.862559in}}{\pgfqpoint{2.900704in}{1.865831in}}{\pgfqpoint{2.892467in}{1.865831in}}%
\pgfpathcurveto{\pgfqpoint{2.884231in}{1.865831in}}{\pgfqpoint{2.876331in}{1.862559in}}{\pgfqpoint{2.870507in}{1.856735in}}%
\pgfpathcurveto{\pgfqpoint{2.864683in}{1.850911in}}{\pgfqpoint{2.861411in}{1.843011in}}{\pgfqpoint{2.861411in}{1.834775in}}%
\pgfpathcurveto{\pgfqpoint{2.861411in}{1.826539in}}{\pgfqpoint{2.864683in}{1.818639in}}{\pgfqpoint{2.870507in}{1.812815in}}%
\pgfpathcurveto{\pgfqpoint{2.876331in}{1.806991in}}{\pgfqpoint{2.884231in}{1.803718in}}{\pgfqpoint{2.892467in}{1.803718in}}%
\pgfpathclose%
\pgfusepath{stroke,fill}%
\end{pgfscope}%
\begin{pgfscope}%
\pgfpathrectangle{\pgfqpoint{0.100000in}{0.212622in}}{\pgfqpoint{3.696000in}{3.696000in}}%
\pgfusepath{clip}%
\pgfsetbuttcap%
\pgfsetroundjoin%
\definecolor{currentfill}{rgb}{0.121569,0.466667,0.705882}%
\pgfsetfillcolor{currentfill}%
\pgfsetfillopacity{0.777925}%
\pgfsetlinewidth{1.003750pt}%
\definecolor{currentstroke}{rgb}{0.121569,0.466667,0.705882}%
\pgfsetstrokecolor{currentstroke}%
\pgfsetstrokeopacity{0.777925}%
\pgfsetdash{}{0pt}%
\pgfpathmoveto{\pgfqpoint{2.890038in}{1.803768in}}%
\pgfpathcurveto{\pgfqpoint{2.898274in}{1.803768in}}{\pgfqpoint{2.906175in}{1.807040in}}{\pgfqpoint{2.911998in}{1.812864in}}%
\pgfpathcurveto{\pgfqpoint{2.917822in}{1.818688in}}{\pgfqpoint{2.921095in}{1.826588in}}{\pgfqpoint{2.921095in}{1.834824in}}%
\pgfpathcurveto{\pgfqpoint{2.921095in}{1.843060in}}{\pgfqpoint{2.917822in}{1.850960in}}{\pgfqpoint{2.911998in}{1.856784in}}%
\pgfpathcurveto{\pgfqpoint{2.906175in}{1.862608in}}{\pgfqpoint{2.898274in}{1.865881in}}{\pgfqpoint{2.890038in}{1.865881in}}%
\pgfpathcurveto{\pgfqpoint{2.881802in}{1.865881in}}{\pgfqpoint{2.873902in}{1.862608in}}{\pgfqpoint{2.868078in}{1.856784in}}%
\pgfpathcurveto{\pgfqpoint{2.862254in}{1.850960in}}{\pgfqpoint{2.858982in}{1.843060in}}{\pgfqpoint{2.858982in}{1.834824in}}%
\pgfpathcurveto{\pgfqpoint{2.858982in}{1.826588in}}{\pgfqpoint{2.862254in}{1.818688in}}{\pgfqpoint{2.868078in}{1.812864in}}%
\pgfpathcurveto{\pgfqpoint{2.873902in}{1.807040in}}{\pgfqpoint{2.881802in}{1.803768in}}{\pgfqpoint{2.890038in}{1.803768in}}%
\pgfpathclose%
\pgfusepath{stroke,fill}%
\end{pgfscope}%
\begin{pgfscope}%
\pgfpathrectangle{\pgfqpoint{0.100000in}{0.212622in}}{\pgfqpoint{3.696000in}{3.696000in}}%
\pgfusepath{clip}%
\pgfsetbuttcap%
\pgfsetroundjoin%
\definecolor{currentfill}{rgb}{0.121569,0.466667,0.705882}%
\pgfsetfillcolor{currentfill}%
\pgfsetfillopacity{0.778963}%
\pgfsetlinewidth{1.003750pt}%
\definecolor{currentstroke}{rgb}{0.121569,0.466667,0.705882}%
\pgfsetstrokecolor{currentstroke}%
\pgfsetstrokeopacity{0.778963}%
\pgfsetdash{}{0pt}%
\pgfpathmoveto{\pgfqpoint{2.887560in}{1.801411in}}%
\pgfpathcurveto{\pgfqpoint{2.895797in}{1.801411in}}{\pgfqpoint{2.903697in}{1.804684in}}{\pgfqpoint{2.909521in}{1.810508in}}%
\pgfpathcurveto{\pgfqpoint{2.915345in}{1.816332in}}{\pgfqpoint{2.918617in}{1.824232in}}{\pgfqpoint{2.918617in}{1.832468in}}%
\pgfpathcurveto{\pgfqpoint{2.918617in}{1.840704in}}{\pgfqpoint{2.915345in}{1.848604in}}{\pgfqpoint{2.909521in}{1.854428in}}%
\pgfpathcurveto{\pgfqpoint{2.903697in}{1.860252in}}{\pgfqpoint{2.895797in}{1.863524in}}{\pgfqpoint{2.887560in}{1.863524in}}%
\pgfpathcurveto{\pgfqpoint{2.879324in}{1.863524in}}{\pgfqpoint{2.871424in}{1.860252in}}{\pgfqpoint{2.865600in}{1.854428in}}%
\pgfpathcurveto{\pgfqpoint{2.859776in}{1.848604in}}{\pgfqpoint{2.856504in}{1.840704in}}{\pgfqpoint{2.856504in}{1.832468in}}%
\pgfpathcurveto{\pgfqpoint{2.856504in}{1.824232in}}{\pgfqpoint{2.859776in}{1.816332in}}{\pgfqpoint{2.865600in}{1.810508in}}%
\pgfpathcurveto{\pgfqpoint{2.871424in}{1.804684in}}{\pgfqpoint{2.879324in}{1.801411in}}{\pgfqpoint{2.887560in}{1.801411in}}%
\pgfpathclose%
\pgfusepath{stroke,fill}%
\end{pgfscope}%
\begin{pgfscope}%
\pgfpathrectangle{\pgfqpoint{0.100000in}{0.212622in}}{\pgfqpoint{3.696000in}{3.696000in}}%
\pgfusepath{clip}%
\pgfsetbuttcap%
\pgfsetroundjoin%
\definecolor{currentfill}{rgb}{0.121569,0.466667,0.705882}%
\pgfsetfillcolor{currentfill}%
\pgfsetfillopacity{0.780590}%
\pgfsetlinewidth{1.003750pt}%
\definecolor{currentstroke}{rgb}{0.121569,0.466667,0.705882}%
\pgfsetstrokecolor{currentstroke}%
\pgfsetstrokeopacity{0.780590}%
\pgfsetdash{}{0pt}%
\pgfpathmoveto{\pgfqpoint{2.883364in}{1.802567in}}%
\pgfpathcurveto{\pgfqpoint{2.891601in}{1.802567in}}{\pgfqpoint{2.899501in}{1.805839in}}{\pgfqpoint{2.905325in}{1.811663in}}%
\pgfpathcurveto{\pgfqpoint{2.911148in}{1.817487in}}{\pgfqpoint{2.914421in}{1.825387in}}{\pgfqpoint{2.914421in}{1.833623in}}%
\pgfpathcurveto{\pgfqpoint{2.914421in}{1.841859in}}{\pgfqpoint{2.911148in}{1.849759in}}{\pgfqpoint{2.905325in}{1.855583in}}%
\pgfpathcurveto{\pgfqpoint{2.899501in}{1.861407in}}{\pgfqpoint{2.891601in}{1.864680in}}{\pgfqpoint{2.883364in}{1.864680in}}%
\pgfpathcurveto{\pgfqpoint{2.875128in}{1.864680in}}{\pgfqpoint{2.867228in}{1.861407in}}{\pgfqpoint{2.861404in}{1.855583in}}%
\pgfpathcurveto{\pgfqpoint{2.855580in}{1.849759in}}{\pgfqpoint{2.852308in}{1.841859in}}{\pgfqpoint{2.852308in}{1.833623in}}%
\pgfpathcurveto{\pgfqpoint{2.852308in}{1.825387in}}{\pgfqpoint{2.855580in}{1.817487in}}{\pgfqpoint{2.861404in}{1.811663in}}%
\pgfpathcurveto{\pgfqpoint{2.867228in}{1.805839in}}{\pgfqpoint{2.875128in}{1.802567in}}{\pgfqpoint{2.883364in}{1.802567in}}%
\pgfpathclose%
\pgfusepath{stroke,fill}%
\end{pgfscope}%
\begin{pgfscope}%
\pgfpathrectangle{\pgfqpoint{0.100000in}{0.212622in}}{\pgfqpoint{3.696000in}{3.696000in}}%
\pgfusepath{clip}%
\pgfsetbuttcap%
\pgfsetroundjoin%
\definecolor{currentfill}{rgb}{0.121569,0.466667,0.705882}%
\pgfsetfillcolor{currentfill}%
\pgfsetfillopacity{0.782518}%
\pgfsetlinewidth{1.003750pt}%
\definecolor{currentstroke}{rgb}{0.121569,0.466667,0.705882}%
\pgfsetstrokecolor{currentstroke}%
\pgfsetstrokeopacity{0.782518}%
\pgfsetdash{}{0pt}%
\pgfpathmoveto{\pgfqpoint{2.881409in}{1.799426in}}%
\pgfpathcurveto{\pgfqpoint{2.889646in}{1.799426in}}{\pgfqpoint{2.897546in}{1.802698in}}{\pgfqpoint{2.903370in}{1.808522in}}%
\pgfpathcurveto{\pgfqpoint{2.909194in}{1.814346in}}{\pgfqpoint{2.912466in}{1.822246in}}{\pgfqpoint{2.912466in}{1.830482in}}%
\pgfpathcurveto{\pgfqpoint{2.912466in}{1.838718in}}{\pgfqpoint{2.909194in}{1.846619in}}{\pgfqpoint{2.903370in}{1.852442in}}%
\pgfpathcurveto{\pgfqpoint{2.897546in}{1.858266in}}{\pgfqpoint{2.889646in}{1.861539in}}{\pgfqpoint{2.881409in}{1.861539in}}%
\pgfpathcurveto{\pgfqpoint{2.873173in}{1.861539in}}{\pgfqpoint{2.865273in}{1.858266in}}{\pgfqpoint{2.859449in}{1.852442in}}%
\pgfpathcurveto{\pgfqpoint{2.853625in}{1.846619in}}{\pgfqpoint{2.850353in}{1.838718in}}{\pgfqpoint{2.850353in}{1.830482in}}%
\pgfpathcurveto{\pgfqpoint{2.850353in}{1.822246in}}{\pgfqpoint{2.853625in}{1.814346in}}{\pgfqpoint{2.859449in}{1.808522in}}%
\pgfpathcurveto{\pgfqpoint{2.865273in}{1.802698in}}{\pgfqpoint{2.873173in}{1.799426in}}{\pgfqpoint{2.881409in}{1.799426in}}%
\pgfpathclose%
\pgfusepath{stroke,fill}%
\end{pgfscope}%
\begin{pgfscope}%
\pgfpathrectangle{\pgfqpoint{0.100000in}{0.212622in}}{\pgfqpoint{3.696000in}{3.696000in}}%
\pgfusepath{clip}%
\pgfsetbuttcap%
\pgfsetroundjoin%
\definecolor{currentfill}{rgb}{0.121569,0.466667,0.705882}%
\pgfsetfillcolor{currentfill}%
\pgfsetfillopacity{0.783408}%
\pgfsetlinewidth{1.003750pt}%
\definecolor{currentstroke}{rgb}{0.121569,0.466667,0.705882}%
\pgfsetstrokecolor{currentstroke}%
\pgfsetstrokeopacity{0.783408}%
\pgfsetdash{}{0pt}%
\pgfpathmoveto{\pgfqpoint{2.880166in}{1.796618in}}%
\pgfpathcurveto{\pgfqpoint{2.888402in}{1.796618in}}{\pgfqpoint{2.896302in}{1.799890in}}{\pgfqpoint{2.902126in}{1.805714in}}%
\pgfpathcurveto{\pgfqpoint{2.907950in}{1.811538in}}{\pgfqpoint{2.911222in}{1.819438in}}{\pgfqpoint{2.911222in}{1.827674in}}%
\pgfpathcurveto{\pgfqpoint{2.911222in}{1.835911in}}{\pgfqpoint{2.907950in}{1.843811in}}{\pgfqpoint{2.902126in}{1.849635in}}%
\pgfpathcurveto{\pgfqpoint{2.896302in}{1.855459in}}{\pgfqpoint{2.888402in}{1.858731in}}{\pgfqpoint{2.880166in}{1.858731in}}%
\pgfpathcurveto{\pgfqpoint{2.871929in}{1.858731in}}{\pgfqpoint{2.864029in}{1.855459in}}{\pgfqpoint{2.858205in}{1.849635in}}%
\pgfpathcurveto{\pgfqpoint{2.852381in}{1.843811in}}{\pgfqpoint{2.849109in}{1.835911in}}{\pgfqpoint{2.849109in}{1.827674in}}%
\pgfpathcurveto{\pgfqpoint{2.849109in}{1.819438in}}{\pgfqpoint{2.852381in}{1.811538in}}{\pgfqpoint{2.858205in}{1.805714in}}%
\pgfpathcurveto{\pgfqpoint{2.864029in}{1.799890in}}{\pgfqpoint{2.871929in}{1.796618in}}{\pgfqpoint{2.880166in}{1.796618in}}%
\pgfpathclose%
\pgfusepath{stroke,fill}%
\end{pgfscope}%
\begin{pgfscope}%
\pgfpathrectangle{\pgfqpoint{0.100000in}{0.212622in}}{\pgfqpoint{3.696000in}{3.696000in}}%
\pgfusepath{clip}%
\pgfsetbuttcap%
\pgfsetroundjoin%
\definecolor{currentfill}{rgb}{0.121569,0.466667,0.705882}%
\pgfsetfillcolor{currentfill}%
\pgfsetfillopacity{0.785058}%
\pgfsetlinewidth{1.003750pt}%
\definecolor{currentstroke}{rgb}{0.121569,0.466667,0.705882}%
\pgfsetstrokecolor{currentstroke}%
\pgfsetstrokeopacity{0.785058}%
\pgfsetdash{}{0pt}%
\pgfpathmoveto{\pgfqpoint{2.875969in}{1.797470in}}%
\pgfpathcurveto{\pgfqpoint{2.884205in}{1.797470in}}{\pgfqpoint{2.892105in}{1.800743in}}{\pgfqpoint{2.897929in}{1.806567in}}%
\pgfpathcurveto{\pgfqpoint{2.903753in}{1.812391in}}{\pgfqpoint{2.907026in}{1.820291in}}{\pgfqpoint{2.907026in}{1.828527in}}%
\pgfpathcurveto{\pgfqpoint{2.907026in}{1.836763in}}{\pgfqpoint{2.903753in}{1.844663in}}{\pgfqpoint{2.897929in}{1.850487in}}%
\pgfpathcurveto{\pgfqpoint{2.892105in}{1.856311in}}{\pgfqpoint{2.884205in}{1.859583in}}{\pgfqpoint{2.875969in}{1.859583in}}%
\pgfpathcurveto{\pgfqpoint{2.867733in}{1.859583in}}{\pgfqpoint{2.859833in}{1.856311in}}{\pgfqpoint{2.854009in}{1.850487in}}%
\pgfpathcurveto{\pgfqpoint{2.848185in}{1.844663in}}{\pgfqpoint{2.844913in}{1.836763in}}{\pgfqpoint{2.844913in}{1.828527in}}%
\pgfpathcurveto{\pgfqpoint{2.844913in}{1.820291in}}{\pgfqpoint{2.848185in}{1.812391in}}{\pgfqpoint{2.854009in}{1.806567in}}%
\pgfpathcurveto{\pgfqpoint{2.859833in}{1.800743in}}{\pgfqpoint{2.867733in}{1.797470in}}{\pgfqpoint{2.875969in}{1.797470in}}%
\pgfpathclose%
\pgfusepath{stroke,fill}%
\end{pgfscope}%
\begin{pgfscope}%
\pgfpathrectangle{\pgfqpoint{0.100000in}{0.212622in}}{\pgfqpoint{3.696000in}{3.696000in}}%
\pgfusepath{clip}%
\pgfsetbuttcap%
\pgfsetroundjoin%
\definecolor{currentfill}{rgb}{0.121569,0.466667,0.705882}%
\pgfsetfillcolor{currentfill}%
\pgfsetfillopacity{0.785669}%
\pgfsetlinewidth{1.003750pt}%
\definecolor{currentstroke}{rgb}{0.121569,0.466667,0.705882}%
\pgfsetstrokecolor{currentstroke}%
\pgfsetstrokeopacity{0.785669}%
\pgfsetdash{}{0pt}%
\pgfpathmoveto{\pgfqpoint{2.875110in}{1.794342in}}%
\pgfpathcurveto{\pgfqpoint{2.883346in}{1.794342in}}{\pgfqpoint{2.891247in}{1.797614in}}{\pgfqpoint{2.897070in}{1.803438in}}%
\pgfpathcurveto{\pgfqpoint{2.902894in}{1.809262in}}{\pgfqpoint{2.906167in}{1.817162in}}{\pgfqpoint{2.906167in}{1.825398in}}%
\pgfpathcurveto{\pgfqpoint{2.906167in}{1.833634in}}{\pgfqpoint{2.902894in}{1.841535in}}{\pgfqpoint{2.897070in}{1.847358in}}%
\pgfpathcurveto{\pgfqpoint{2.891247in}{1.853182in}}{\pgfqpoint{2.883346in}{1.856455in}}{\pgfqpoint{2.875110in}{1.856455in}}%
\pgfpathcurveto{\pgfqpoint{2.866874in}{1.856455in}}{\pgfqpoint{2.858974in}{1.853182in}}{\pgfqpoint{2.853150in}{1.847358in}}%
\pgfpathcurveto{\pgfqpoint{2.847326in}{1.841535in}}{\pgfqpoint{2.844054in}{1.833634in}}{\pgfqpoint{2.844054in}{1.825398in}}%
\pgfpathcurveto{\pgfqpoint{2.844054in}{1.817162in}}{\pgfqpoint{2.847326in}{1.809262in}}{\pgfqpoint{2.853150in}{1.803438in}}%
\pgfpathcurveto{\pgfqpoint{2.858974in}{1.797614in}}{\pgfqpoint{2.866874in}{1.794342in}}{\pgfqpoint{2.875110in}{1.794342in}}%
\pgfpathclose%
\pgfusepath{stroke,fill}%
\end{pgfscope}%
\begin{pgfscope}%
\pgfpathrectangle{\pgfqpoint{0.100000in}{0.212622in}}{\pgfqpoint{3.696000in}{3.696000in}}%
\pgfusepath{clip}%
\pgfsetbuttcap%
\pgfsetroundjoin%
\definecolor{currentfill}{rgb}{0.121569,0.466667,0.705882}%
\pgfsetfillcolor{currentfill}%
\pgfsetfillopacity{0.786205}%
\pgfsetlinewidth{1.003750pt}%
\definecolor{currentstroke}{rgb}{0.121569,0.466667,0.705882}%
\pgfsetstrokecolor{currentstroke}%
\pgfsetstrokeopacity{0.786205}%
\pgfsetdash{}{0pt}%
\pgfpathmoveto{\pgfqpoint{2.875099in}{1.793854in}}%
\pgfpathcurveto{\pgfqpoint{2.883335in}{1.793854in}}{\pgfqpoint{2.891235in}{1.797126in}}{\pgfqpoint{2.897059in}{1.802950in}}%
\pgfpathcurveto{\pgfqpoint{2.902883in}{1.808774in}}{\pgfqpoint{2.906155in}{1.816674in}}{\pgfqpoint{2.906155in}{1.824910in}}%
\pgfpathcurveto{\pgfqpoint{2.906155in}{1.833147in}}{\pgfqpoint{2.902883in}{1.841047in}}{\pgfqpoint{2.897059in}{1.846871in}}%
\pgfpathcurveto{\pgfqpoint{2.891235in}{1.852695in}}{\pgfqpoint{2.883335in}{1.855967in}}{\pgfqpoint{2.875099in}{1.855967in}}%
\pgfpathcurveto{\pgfqpoint{2.866863in}{1.855967in}}{\pgfqpoint{2.858963in}{1.852695in}}{\pgfqpoint{2.853139in}{1.846871in}}%
\pgfpathcurveto{\pgfqpoint{2.847315in}{1.841047in}}{\pgfqpoint{2.844042in}{1.833147in}}{\pgfqpoint{2.844042in}{1.824910in}}%
\pgfpathcurveto{\pgfqpoint{2.844042in}{1.816674in}}{\pgfqpoint{2.847315in}{1.808774in}}{\pgfqpoint{2.853139in}{1.802950in}}%
\pgfpathcurveto{\pgfqpoint{2.858963in}{1.797126in}}{\pgfqpoint{2.866863in}{1.793854in}}{\pgfqpoint{2.875099in}{1.793854in}}%
\pgfpathclose%
\pgfusepath{stroke,fill}%
\end{pgfscope}%
\begin{pgfscope}%
\pgfpathrectangle{\pgfqpoint{0.100000in}{0.212622in}}{\pgfqpoint{3.696000in}{3.696000in}}%
\pgfusepath{clip}%
\pgfsetbuttcap%
\pgfsetroundjoin%
\definecolor{currentfill}{rgb}{0.121569,0.466667,0.705882}%
\pgfsetfillcolor{currentfill}%
\pgfsetfillopacity{0.786933}%
\pgfsetlinewidth{1.003750pt}%
\definecolor{currentstroke}{rgb}{0.121569,0.466667,0.705882}%
\pgfsetstrokecolor{currentstroke}%
\pgfsetstrokeopacity{0.786933}%
\pgfsetdash{}{0pt}%
\pgfpathmoveto{\pgfqpoint{2.872227in}{1.790719in}}%
\pgfpathcurveto{\pgfqpoint{2.880463in}{1.790719in}}{\pgfqpoint{2.888363in}{1.793991in}}{\pgfqpoint{2.894187in}{1.799815in}}%
\pgfpathcurveto{\pgfqpoint{2.900011in}{1.805639in}}{\pgfqpoint{2.903284in}{1.813539in}}{\pgfqpoint{2.903284in}{1.821775in}}%
\pgfpathcurveto{\pgfqpoint{2.903284in}{1.830012in}}{\pgfqpoint{2.900011in}{1.837912in}}{\pgfqpoint{2.894187in}{1.843736in}}%
\pgfpathcurveto{\pgfqpoint{2.888363in}{1.849560in}}{\pgfqpoint{2.880463in}{1.852832in}}{\pgfqpoint{2.872227in}{1.852832in}}%
\pgfpathcurveto{\pgfqpoint{2.863991in}{1.852832in}}{\pgfqpoint{2.856091in}{1.849560in}}{\pgfqpoint{2.850267in}{1.843736in}}%
\pgfpathcurveto{\pgfqpoint{2.844443in}{1.837912in}}{\pgfqpoint{2.841171in}{1.830012in}}{\pgfqpoint{2.841171in}{1.821775in}}%
\pgfpathcurveto{\pgfqpoint{2.841171in}{1.813539in}}{\pgfqpoint{2.844443in}{1.805639in}}{\pgfqpoint{2.850267in}{1.799815in}}%
\pgfpathcurveto{\pgfqpoint{2.856091in}{1.793991in}}{\pgfqpoint{2.863991in}{1.790719in}}{\pgfqpoint{2.872227in}{1.790719in}}%
\pgfpathclose%
\pgfusepath{stroke,fill}%
\end{pgfscope}%
\begin{pgfscope}%
\pgfpathrectangle{\pgfqpoint{0.100000in}{0.212622in}}{\pgfqpoint{3.696000in}{3.696000in}}%
\pgfusepath{clip}%
\pgfsetbuttcap%
\pgfsetroundjoin%
\definecolor{currentfill}{rgb}{0.121569,0.466667,0.705882}%
\pgfsetfillcolor{currentfill}%
\pgfsetfillopacity{0.787447}%
\pgfsetlinewidth{1.003750pt}%
\definecolor{currentstroke}{rgb}{0.121569,0.466667,0.705882}%
\pgfsetstrokecolor{currentstroke}%
\pgfsetstrokeopacity{0.787447}%
\pgfsetdash{}{0pt}%
\pgfpathmoveto{\pgfqpoint{2.870724in}{1.789678in}}%
\pgfpathcurveto{\pgfqpoint{2.878960in}{1.789678in}}{\pgfqpoint{2.886860in}{1.792951in}}{\pgfqpoint{2.892684in}{1.798775in}}%
\pgfpathcurveto{\pgfqpoint{2.898508in}{1.804599in}}{\pgfqpoint{2.901781in}{1.812499in}}{\pgfqpoint{2.901781in}{1.820735in}}%
\pgfpathcurveto{\pgfqpoint{2.901781in}{1.828971in}}{\pgfqpoint{2.898508in}{1.836871in}}{\pgfqpoint{2.892684in}{1.842695in}}%
\pgfpathcurveto{\pgfqpoint{2.886860in}{1.848519in}}{\pgfqpoint{2.878960in}{1.851791in}}{\pgfqpoint{2.870724in}{1.851791in}}%
\pgfpathcurveto{\pgfqpoint{2.862488in}{1.851791in}}{\pgfqpoint{2.854588in}{1.848519in}}{\pgfqpoint{2.848764in}{1.842695in}}%
\pgfpathcurveto{\pgfqpoint{2.842940in}{1.836871in}}{\pgfqpoint{2.839668in}{1.828971in}}{\pgfqpoint{2.839668in}{1.820735in}}%
\pgfpathcurveto{\pgfqpoint{2.839668in}{1.812499in}}{\pgfqpoint{2.842940in}{1.804599in}}{\pgfqpoint{2.848764in}{1.798775in}}%
\pgfpathcurveto{\pgfqpoint{2.854588in}{1.792951in}}{\pgfqpoint{2.862488in}{1.789678in}}{\pgfqpoint{2.870724in}{1.789678in}}%
\pgfpathclose%
\pgfusepath{stroke,fill}%
\end{pgfscope}%
\begin{pgfscope}%
\pgfpathrectangle{\pgfqpoint{0.100000in}{0.212622in}}{\pgfqpoint{3.696000in}{3.696000in}}%
\pgfusepath{clip}%
\pgfsetbuttcap%
\pgfsetroundjoin%
\definecolor{currentfill}{rgb}{0.121569,0.466667,0.705882}%
\pgfsetfillcolor{currentfill}%
\pgfsetfillopacity{0.788639}%
\pgfsetlinewidth{1.003750pt}%
\definecolor{currentstroke}{rgb}{0.121569,0.466667,0.705882}%
\pgfsetstrokecolor{currentstroke}%
\pgfsetstrokeopacity{0.788639}%
\pgfsetdash{}{0pt}%
\pgfpathmoveto{\pgfqpoint{2.869624in}{1.788069in}}%
\pgfpathcurveto{\pgfqpoint{2.877860in}{1.788069in}}{\pgfqpoint{2.885760in}{1.791341in}}{\pgfqpoint{2.891584in}{1.797165in}}%
\pgfpathcurveto{\pgfqpoint{2.897408in}{1.802989in}}{\pgfqpoint{2.900681in}{1.810889in}}{\pgfqpoint{2.900681in}{1.819125in}}%
\pgfpathcurveto{\pgfqpoint{2.900681in}{1.827361in}}{\pgfqpoint{2.897408in}{1.835261in}}{\pgfqpoint{2.891584in}{1.841085in}}%
\pgfpathcurveto{\pgfqpoint{2.885760in}{1.846909in}}{\pgfqpoint{2.877860in}{1.850182in}}{\pgfqpoint{2.869624in}{1.850182in}}%
\pgfpathcurveto{\pgfqpoint{2.861388in}{1.850182in}}{\pgfqpoint{2.853488in}{1.846909in}}{\pgfqpoint{2.847664in}{1.841085in}}%
\pgfpathcurveto{\pgfqpoint{2.841840in}{1.835261in}}{\pgfqpoint{2.838568in}{1.827361in}}{\pgfqpoint{2.838568in}{1.819125in}}%
\pgfpathcurveto{\pgfqpoint{2.838568in}{1.810889in}}{\pgfqpoint{2.841840in}{1.802989in}}{\pgfqpoint{2.847664in}{1.797165in}}%
\pgfpathcurveto{\pgfqpoint{2.853488in}{1.791341in}}{\pgfqpoint{2.861388in}{1.788069in}}{\pgfqpoint{2.869624in}{1.788069in}}%
\pgfpathclose%
\pgfusepath{stroke,fill}%
\end{pgfscope}%
\begin{pgfscope}%
\pgfpathrectangle{\pgfqpoint{0.100000in}{0.212622in}}{\pgfqpoint{3.696000in}{3.696000in}}%
\pgfusepath{clip}%
\pgfsetbuttcap%
\pgfsetroundjoin%
\definecolor{currentfill}{rgb}{0.121569,0.466667,0.705882}%
\pgfsetfillcolor{currentfill}%
\pgfsetfillopacity{0.789087}%
\pgfsetlinewidth{1.003750pt}%
\definecolor{currentstroke}{rgb}{0.121569,0.466667,0.705882}%
\pgfsetstrokecolor{currentstroke}%
\pgfsetstrokeopacity{0.789087}%
\pgfsetdash{}{0pt}%
\pgfpathmoveto{\pgfqpoint{2.868571in}{1.786058in}}%
\pgfpathcurveto{\pgfqpoint{2.876808in}{1.786058in}}{\pgfqpoint{2.884708in}{1.789330in}}{\pgfqpoint{2.890532in}{1.795154in}}%
\pgfpathcurveto{\pgfqpoint{2.896355in}{1.800978in}}{\pgfqpoint{2.899628in}{1.808878in}}{\pgfqpoint{2.899628in}{1.817114in}}%
\pgfpathcurveto{\pgfqpoint{2.899628in}{1.825350in}}{\pgfqpoint{2.896355in}{1.833251in}}{\pgfqpoint{2.890532in}{1.839074in}}%
\pgfpathcurveto{\pgfqpoint{2.884708in}{1.844898in}}{\pgfqpoint{2.876808in}{1.848171in}}{\pgfqpoint{2.868571in}{1.848171in}}%
\pgfpathcurveto{\pgfqpoint{2.860335in}{1.848171in}}{\pgfqpoint{2.852435in}{1.844898in}}{\pgfqpoint{2.846611in}{1.839074in}}%
\pgfpathcurveto{\pgfqpoint{2.840787in}{1.833251in}}{\pgfqpoint{2.837515in}{1.825350in}}{\pgfqpoint{2.837515in}{1.817114in}}%
\pgfpathcurveto{\pgfqpoint{2.837515in}{1.808878in}}{\pgfqpoint{2.840787in}{1.800978in}}{\pgfqpoint{2.846611in}{1.795154in}}%
\pgfpathcurveto{\pgfqpoint{2.852435in}{1.789330in}}{\pgfqpoint{2.860335in}{1.786058in}}{\pgfqpoint{2.868571in}{1.786058in}}%
\pgfpathclose%
\pgfusepath{stroke,fill}%
\end{pgfscope}%
\begin{pgfscope}%
\pgfpathrectangle{\pgfqpoint{0.100000in}{0.212622in}}{\pgfqpoint{3.696000in}{3.696000in}}%
\pgfusepath{clip}%
\pgfsetbuttcap%
\pgfsetroundjoin%
\definecolor{currentfill}{rgb}{0.121569,0.466667,0.705882}%
\pgfsetfillcolor{currentfill}%
\pgfsetfillopacity{0.789884}%
\pgfsetlinewidth{1.003750pt}%
\definecolor{currentstroke}{rgb}{0.121569,0.466667,0.705882}%
\pgfsetstrokecolor{currentstroke}%
\pgfsetstrokeopacity{0.789884}%
\pgfsetdash{}{0pt}%
\pgfpathmoveto{\pgfqpoint{2.866489in}{1.787035in}}%
\pgfpathcurveto{\pgfqpoint{2.874725in}{1.787035in}}{\pgfqpoint{2.882625in}{1.790307in}}{\pgfqpoint{2.888449in}{1.796131in}}%
\pgfpathcurveto{\pgfqpoint{2.894273in}{1.801955in}}{\pgfqpoint{2.897545in}{1.809855in}}{\pgfqpoint{2.897545in}{1.818091in}}%
\pgfpathcurveto{\pgfqpoint{2.897545in}{1.826328in}}{\pgfqpoint{2.894273in}{1.834228in}}{\pgfqpoint{2.888449in}{1.840051in}}%
\pgfpathcurveto{\pgfqpoint{2.882625in}{1.845875in}}{\pgfqpoint{2.874725in}{1.849148in}}{\pgfqpoint{2.866489in}{1.849148in}}%
\pgfpathcurveto{\pgfqpoint{2.858252in}{1.849148in}}{\pgfqpoint{2.850352in}{1.845875in}}{\pgfqpoint{2.844528in}{1.840051in}}%
\pgfpathcurveto{\pgfqpoint{2.838705in}{1.834228in}}{\pgfqpoint{2.835432in}{1.826328in}}{\pgfqpoint{2.835432in}{1.818091in}}%
\pgfpathcurveto{\pgfqpoint{2.835432in}{1.809855in}}{\pgfqpoint{2.838705in}{1.801955in}}{\pgfqpoint{2.844528in}{1.796131in}}%
\pgfpathcurveto{\pgfqpoint{2.850352in}{1.790307in}}{\pgfqpoint{2.858252in}{1.787035in}}{\pgfqpoint{2.866489in}{1.787035in}}%
\pgfpathclose%
\pgfusepath{stroke,fill}%
\end{pgfscope}%
\begin{pgfscope}%
\pgfpathrectangle{\pgfqpoint{0.100000in}{0.212622in}}{\pgfqpoint{3.696000in}{3.696000in}}%
\pgfusepath{clip}%
\pgfsetbuttcap%
\pgfsetroundjoin%
\definecolor{currentfill}{rgb}{0.121569,0.466667,0.705882}%
\pgfsetfillcolor{currentfill}%
\pgfsetfillopacity{0.791090}%
\pgfsetlinewidth{1.003750pt}%
\definecolor{currentstroke}{rgb}{0.121569,0.466667,0.705882}%
\pgfsetstrokecolor{currentstroke}%
\pgfsetstrokeopacity{0.791090}%
\pgfsetdash{}{0pt}%
\pgfpathmoveto{\pgfqpoint{2.865946in}{1.785298in}}%
\pgfpathcurveto{\pgfqpoint{2.874183in}{1.785298in}}{\pgfqpoint{2.882083in}{1.788570in}}{\pgfqpoint{2.887907in}{1.794394in}}%
\pgfpathcurveto{\pgfqpoint{2.893731in}{1.800218in}}{\pgfqpoint{2.897003in}{1.808118in}}{\pgfqpoint{2.897003in}{1.816354in}}%
\pgfpathcurveto{\pgfqpoint{2.897003in}{1.824591in}}{\pgfqpoint{2.893731in}{1.832491in}}{\pgfqpoint{2.887907in}{1.838315in}}%
\pgfpathcurveto{\pgfqpoint{2.882083in}{1.844139in}}{\pgfqpoint{2.874183in}{1.847411in}}{\pgfqpoint{2.865946in}{1.847411in}}%
\pgfpathcurveto{\pgfqpoint{2.857710in}{1.847411in}}{\pgfqpoint{2.849810in}{1.844139in}}{\pgfqpoint{2.843986in}{1.838315in}}%
\pgfpathcurveto{\pgfqpoint{2.838162in}{1.832491in}}{\pgfqpoint{2.834890in}{1.824591in}}{\pgfqpoint{2.834890in}{1.816354in}}%
\pgfpathcurveto{\pgfqpoint{2.834890in}{1.808118in}}{\pgfqpoint{2.838162in}{1.800218in}}{\pgfqpoint{2.843986in}{1.794394in}}%
\pgfpathcurveto{\pgfqpoint{2.849810in}{1.788570in}}{\pgfqpoint{2.857710in}{1.785298in}}{\pgfqpoint{2.865946in}{1.785298in}}%
\pgfpathclose%
\pgfusepath{stroke,fill}%
\end{pgfscope}%
\begin{pgfscope}%
\pgfpathrectangle{\pgfqpoint{0.100000in}{0.212622in}}{\pgfqpoint{3.696000in}{3.696000in}}%
\pgfusepath{clip}%
\pgfsetbuttcap%
\pgfsetroundjoin%
\definecolor{currentfill}{rgb}{0.121569,0.466667,0.705882}%
\pgfsetfillcolor{currentfill}%
\pgfsetfillopacity{0.792016}%
\pgfsetlinewidth{1.003750pt}%
\definecolor{currentstroke}{rgb}{0.121569,0.466667,0.705882}%
\pgfsetstrokecolor{currentstroke}%
\pgfsetstrokeopacity{0.792016}%
\pgfsetdash{}{0pt}%
\pgfpathmoveto{\pgfqpoint{2.864048in}{1.781071in}}%
\pgfpathcurveto{\pgfqpoint{2.872285in}{1.781071in}}{\pgfqpoint{2.880185in}{1.784344in}}{\pgfqpoint{2.886008in}{1.790167in}}%
\pgfpathcurveto{\pgfqpoint{2.891832in}{1.795991in}}{\pgfqpoint{2.895105in}{1.803891in}}{\pgfqpoint{2.895105in}{1.812128in}}%
\pgfpathcurveto{\pgfqpoint{2.895105in}{1.820364in}}{\pgfqpoint{2.891832in}{1.828264in}}{\pgfqpoint{2.886008in}{1.834088in}}%
\pgfpathcurveto{\pgfqpoint{2.880185in}{1.839912in}}{\pgfqpoint{2.872285in}{1.843184in}}{\pgfqpoint{2.864048in}{1.843184in}}%
\pgfpathcurveto{\pgfqpoint{2.855812in}{1.843184in}}{\pgfqpoint{2.847912in}{1.839912in}}{\pgfqpoint{2.842088in}{1.834088in}}%
\pgfpathcurveto{\pgfqpoint{2.836264in}{1.828264in}}{\pgfqpoint{2.832992in}{1.820364in}}{\pgfqpoint{2.832992in}{1.812128in}}%
\pgfpathcurveto{\pgfqpoint{2.832992in}{1.803891in}}{\pgfqpoint{2.836264in}{1.795991in}}{\pgfqpoint{2.842088in}{1.790167in}}%
\pgfpathcurveto{\pgfqpoint{2.847912in}{1.784344in}}{\pgfqpoint{2.855812in}{1.781071in}}{\pgfqpoint{2.864048in}{1.781071in}}%
\pgfpathclose%
\pgfusepath{stroke,fill}%
\end{pgfscope}%
\begin{pgfscope}%
\pgfpathrectangle{\pgfqpoint{0.100000in}{0.212622in}}{\pgfqpoint{3.696000in}{3.696000in}}%
\pgfusepath{clip}%
\pgfsetbuttcap%
\pgfsetroundjoin%
\definecolor{currentfill}{rgb}{0.121569,0.466667,0.705882}%
\pgfsetfillcolor{currentfill}%
\pgfsetfillopacity{0.794039}%
\pgfsetlinewidth{1.003750pt}%
\definecolor{currentstroke}{rgb}{0.121569,0.466667,0.705882}%
\pgfsetstrokecolor{currentstroke}%
\pgfsetstrokeopacity{0.794039}%
\pgfsetdash{}{0pt}%
\pgfpathmoveto{\pgfqpoint{2.859003in}{1.782716in}}%
\pgfpathcurveto{\pgfqpoint{2.867239in}{1.782716in}}{\pgfqpoint{2.875139in}{1.785988in}}{\pgfqpoint{2.880963in}{1.791812in}}%
\pgfpathcurveto{\pgfqpoint{2.886787in}{1.797636in}}{\pgfqpoint{2.890059in}{1.805536in}}{\pgfqpoint{2.890059in}{1.813772in}}%
\pgfpathcurveto{\pgfqpoint{2.890059in}{1.822009in}}{\pgfqpoint{2.886787in}{1.829909in}}{\pgfqpoint{2.880963in}{1.835733in}}%
\pgfpathcurveto{\pgfqpoint{2.875139in}{1.841557in}}{\pgfqpoint{2.867239in}{1.844829in}}{\pgfqpoint{2.859003in}{1.844829in}}%
\pgfpathcurveto{\pgfqpoint{2.850767in}{1.844829in}}{\pgfqpoint{2.842867in}{1.841557in}}{\pgfqpoint{2.837043in}{1.835733in}}%
\pgfpathcurveto{\pgfqpoint{2.831219in}{1.829909in}}{\pgfqpoint{2.827946in}{1.822009in}}{\pgfqpoint{2.827946in}{1.813772in}}%
\pgfpathcurveto{\pgfqpoint{2.827946in}{1.805536in}}{\pgfqpoint{2.831219in}{1.797636in}}{\pgfqpoint{2.837043in}{1.791812in}}%
\pgfpathcurveto{\pgfqpoint{2.842867in}{1.785988in}}{\pgfqpoint{2.850767in}{1.782716in}}{\pgfqpoint{2.859003in}{1.782716in}}%
\pgfpathclose%
\pgfusepath{stroke,fill}%
\end{pgfscope}%
\begin{pgfscope}%
\pgfpathrectangle{\pgfqpoint{0.100000in}{0.212622in}}{\pgfqpoint{3.696000in}{3.696000in}}%
\pgfusepath{clip}%
\pgfsetbuttcap%
\pgfsetroundjoin%
\definecolor{currentfill}{rgb}{0.121569,0.466667,0.705882}%
\pgfsetfillcolor{currentfill}%
\pgfsetfillopacity{0.795066}%
\pgfsetlinewidth{1.003750pt}%
\definecolor{currentstroke}{rgb}{0.121569,0.466667,0.705882}%
\pgfsetstrokecolor{currentstroke}%
\pgfsetstrokeopacity{0.795066}%
\pgfsetdash{}{0pt}%
\pgfpathmoveto{\pgfqpoint{2.858414in}{1.780914in}}%
\pgfpathcurveto{\pgfqpoint{2.866651in}{1.780914in}}{\pgfqpoint{2.874551in}{1.784186in}}{\pgfqpoint{2.880375in}{1.790010in}}%
\pgfpathcurveto{\pgfqpoint{2.886199in}{1.795834in}}{\pgfqpoint{2.889471in}{1.803734in}}{\pgfqpoint{2.889471in}{1.811970in}}%
\pgfpathcurveto{\pgfqpoint{2.889471in}{1.820206in}}{\pgfqpoint{2.886199in}{1.828106in}}{\pgfqpoint{2.880375in}{1.833930in}}%
\pgfpathcurveto{\pgfqpoint{2.874551in}{1.839754in}}{\pgfqpoint{2.866651in}{1.843027in}}{\pgfqpoint{2.858414in}{1.843027in}}%
\pgfpathcurveto{\pgfqpoint{2.850178in}{1.843027in}}{\pgfqpoint{2.842278in}{1.839754in}}{\pgfqpoint{2.836454in}{1.833930in}}%
\pgfpathcurveto{\pgfqpoint{2.830630in}{1.828106in}}{\pgfqpoint{2.827358in}{1.820206in}}{\pgfqpoint{2.827358in}{1.811970in}}%
\pgfpathcurveto{\pgfqpoint{2.827358in}{1.803734in}}{\pgfqpoint{2.830630in}{1.795834in}}{\pgfqpoint{2.836454in}{1.790010in}}%
\pgfpathcurveto{\pgfqpoint{2.842278in}{1.784186in}}{\pgfqpoint{2.850178in}{1.780914in}}{\pgfqpoint{2.858414in}{1.780914in}}%
\pgfpathclose%
\pgfusepath{stroke,fill}%
\end{pgfscope}%
\begin{pgfscope}%
\pgfpathrectangle{\pgfqpoint{0.100000in}{0.212622in}}{\pgfqpoint{3.696000in}{3.696000in}}%
\pgfusepath{clip}%
\pgfsetbuttcap%
\pgfsetroundjoin%
\definecolor{currentfill}{rgb}{0.121569,0.466667,0.705882}%
\pgfsetfillcolor{currentfill}%
\pgfsetfillopacity{0.796010}%
\pgfsetlinewidth{1.003750pt}%
\definecolor{currentstroke}{rgb}{0.121569,0.466667,0.705882}%
\pgfsetstrokecolor{currentstroke}%
\pgfsetstrokeopacity{0.796010}%
\pgfsetdash{}{0pt}%
\pgfpathmoveto{\pgfqpoint{2.857027in}{1.777516in}}%
\pgfpathcurveto{\pgfqpoint{2.865263in}{1.777516in}}{\pgfqpoint{2.873163in}{1.780789in}}{\pgfqpoint{2.878987in}{1.786613in}}%
\pgfpathcurveto{\pgfqpoint{2.884811in}{1.792436in}}{\pgfqpoint{2.888083in}{1.800337in}}{\pgfqpoint{2.888083in}{1.808573in}}%
\pgfpathcurveto{\pgfqpoint{2.888083in}{1.816809in}}{\pgfqpoint{2.884811in}{1.824709in}}{\pgfqpoint{2.878987in}{1.830533in}}%
\pgfpathcurveto{\pgfqpoint{2.873163in}{1.836357in}}{\pgfqpoint{2.865263in}{1.839629in}}{\pgfqpoint{2.857027in}{1.839629in}}%
\pgfpathcurveto{\pgfqpoint{2.848791in}{1.839629in}}{\pgfqpoint{2.840891in}{1.836357in}}{\pgfqpoint{2.835067in}{1.830533in}}%
\pgfpathcurveto{\pgfqpoint{2.829243in}{1.824709in}}{\pgfqpoint{2.825970in}{1.816809in}}{\pgfqpoint{2.825970in}{1.808573in}}%
\pgfpathcurveto{\pgfqpoint{2.825970in}{1.800337in}}{\pgfqpoint{2.829243in}{1.792436in}}{\pgfqpoint{2.835067in}{1.786613in}}%
\pgfpathcurveto{\pgfqpoint{2.840891in}{1.780789in}}{\pgfqpoint{2.848791in}{1.777516in}}{\pgfqpoint{2.857027in}{1.777516in}}%
\pgfpathclose%
\pgfusepath{stroke,fill}%
\end{pgfscope}%
\begin{pgfscope}%
\pgfpathrectangle{\pgfqpoint{0.100000in}{0.212622in}}{\pgfqpoint{3.696000in}{3.696000in}}%
\pgfusepath{clip}%
\pgfsetbuttcap%
\pgfsetroundjoin%
\definecolor{currentfill}{rgb}{0.121569,0.466667,0.705882}%
\pgfsetfillcolor{currentfill}%
\pgfsetfillopacity{0.798341}%
\pgfsetlinewidth{1.003750pt}%
\definecolor{currentstroke}{rgb}{0.121569,0.466667,0.705882}%
\pgfsetstrokecolor{currentstroke}%
\pgfsetstrokeopacity{0.798341}%
\pgfsetdash{}{0pt}%
\pgfpathmoveto{\pgfqpoint{2.851485in}{1.779841in}}%
\pgfpathcurveto{\pgfqpoint{2.859721in}{1.779841in}}{\pgfqpoint{2.867621in}{1.783113in}}{\pgfqpoint{2.873445in}{1.788937in}}%
\pgfpathcurveto{\pgfqpoint{2.879269in}{1.794761in}}{\pgfqpoint{2.882541in}{1.802661in}}{\pgfqpoint{2.882541in}{1.810898in}}%
\pgfpathcurveto{\pgfqpoint{2.882541in}{1.819134in}}{\pgfqpoint{2.879269in}{1.827034in}}{\pgfqpoint{2.873445in}{1.832858in}}%
\pgfpathcurveto{\pgfqpoint{2.867621in}{1.838682in}}{\pgfqpoint{2.859721in}{1.841954in}}{\pgfqpoint{2.851485in}{1.841954in}}%
\pgfpathcurveto{\pgfqpoint{2.843248in}{1.841954in}}{\pgfqpoint{2.835348in}{1.838682in}}{\pgfqpoint{2.829524in}{1.832858in}}%
\pgfpathcurveto{\pgfqpoint{2.823700in}{1.827034in}}{\pgfqpoint{2.820428in}{1.819134in}}{\pgfqpoint{2.820428in}{1.810898in}}%
\pgfpathcurveto{\pgfqpoint{2.820428in}{1.802661in}}{\pgfqpoint{2.823700in}{1.794761in}}{\pgfqpoint{2.829524in}{1.788937in}}%
\pgfpathcurveto{\pgfqpoint{2.835348in}{1.783113in}}{\pgfqpoint{2.843248in}{1.779841in}}{\pgfqpoint{2.851485in}{1.779841in}}%
\pgfpathclose%
\pgfusepath{stroke,fill}%
\end{pgfscope}%
\begin{pgfscope}%
\pgfpathrectangle{\pgfqpoint{0.100000in}{0.212622in}}{\pgfqpoint{3.696000in}{3.696000in}}%
\pgfusepath{clip}%
\pgfsetbuttcap%
\pgfsetroundjoin%
\definecolor{currentfill}{rgb}{0.121569,0.466667,0.705882}%
\pgfsetfillcolor{currentfill}%
\pgfsetfillopacity{0.799347}%
\pgfsetlinewidth{1.003750pt}%
\definecolor{currentstroke}{rgb}{0.121569,0.466667,0.705882}%
\pgfsetstrokecolor{currentstroke}%
\pgfsetstrokeopacity{0.799347}%
\pgfsetdash{}{0pt}%
\pgfpathmoveto{\pgfqpoint{2.851349in}{1.776831in}}%
\pgfpathcurveto{\pgfqpoint{2.859585in}{1.776831in}}{\pgfqpoint{2.867485in}{1.780103in}}{\pgfqpoint{2.873309in}{1.785927in}}%
\pgfpathcurveto{\pgfqpoint{2.879133in}{1.791751in}}{\pgfqpoint{2.882405in}{1.799651in}}{\pgfqpoint{2.882405in}{1.807888in}}%
\pgfpathcurveto{\pgfqpoint{2.882405in}{1.816124in}}{\pgfqpoint{2.879133in}{1.824024in}}{\pgfqpoint{2.873309in}{1.829848in}}%
\pgfpathcurveto{\pgfqpoint{2.867485in}{1.835672in}}{\pgfqpoint{2.859585in}{1.838944in}}{\pgfqpoint{2.851349in}{1.838944in}}%
\pgfpathcurveto{\pgfqpoint{2.843113in}{1.838944in}}{\pgfqpoint{2.835213in}{1.835672in}}{\pgfqpoint{2.829389in}{1.829848in}}%
\pgfpathcurveto{\pgfqpoint{2.823565in}{1.824024in}}{\pgfqpoint{2.820292in}{1.816124in}}{\pgfqpoint{2.820292in}{1.807888in}}%
\pgfpathcurveto{\pgfqpoint{2.820292in}{1.799651in}}{\pgfqpoint{2.823565in}{1.791751in}}{\pgfqpoint{2.829389in}{1.785927in}}%
\pgfpathcurveto{\pgfqpoint{2.835213in}{1.780103in}}{\pgfqpoint{2.843113in}{1.776831in}}{\pgfqpoint{2.851349in}{1.776831in}}%
\pgfpathclose%
\pgfusepath{stroke,fill}%
\end{pgfscope}%
\begin{pgfscope}%
\pgfpathrectangle{\pgfqpoint{0.100000in}{0.212622in}}{\pgfqpoint{3.696000in}{3.696000in}}%
\pgfusepath{clip}%
\pgfsetbuttcap%
\pgfsetroundjoin%
\definecolor{currentfill}{rgb}{0.121569,0.466667,0.705882}%
\pgfsetfillcolor{currentfill}%
\pgfsetfillopacity{0.799791}%
\pgfsetlinewidth{1.003750pt}%
\definecolor{currentstroke}{rgb}{0.121569,0.466667,0.705882}%
\pgfsetstrokecolor{currentstroke}%
\pgfsetstrokeopacity{0.799791}%
\pgfsetdash{}{0pt}%
\pgfpathmoveto{\pgfqpoint{2.850833in}{1.774545in}}%
\pgfpathcurveto{\pgfqpoint{2.859069in}{1.774545in}}{\pgfqpoint{2.866970in}{1.777817in}}{\pgfqpoint{2.872793in}{1.783641in}}%
\pgfpathcurveto{\pgfqpoint{2.878617in}{1.789465in}}{\pgfqpoint{2.881890in}{1.797365in}}{\pgfqpoint{2.881890in}{1.805601in}}%
\pgfpathcurveto{\pgfqpoint{2.881890in}{1.813838in}}{\pgfqpoint{2.878617in}{1.821738in}}{\pgfqpoint{2.872793in}{1.827562in}}%
\pgfpathcurveto{\pgfqpoint{2.866970in}{1.833386in}}{\pgfqpoint{2.859069in}{1.836658in}}{\pgfqpoint{2.850833in}{1.836658in}}%
\pgfpathcurveto{\pgfqpoint{2.842597in}{1.836658in}}{\pgfqpoint{2.834697in}{1.833386in}}{\pgfqpoint{2.828873in}{1.827562in}}%
\pgfpathcurveto{\pgfqpoint{2.823049in}{1.821738in}}{\pgfqpoint{2.819777in}{1.813838in}}{\pgfqpoint{2.819777in}{1.805601in}}%
\pgfpathcurveto{\pgfqpoint{2.819777in}{1.797365in}}{\pgfqpoint{2.823049in}{1.789465in}}{\pgfqpoint{2.828873in}{1.783641in}}%
\pgfpathcurveto{\pgfqpoint{2.834697in}{1.777817in}}{\pgfqpoint{2.842597in}{1.774545in}}{\pgfqpoint{2.850833in}{1.774545in}}%
\pgfpathclose%
\pgfusepath{stroke,fill}%
\end{pgfscope}%
\begin{pgfscope}%
\pgfpathrectangle{\pgfqpoint{0.100000in}{0.212622in}}{\pgfqpoint{3.696000in}{3.696000in}}%
\pgfusepath{clip}%
\pgfsetbuttcap%
\pgfsetroundjoin%
\definecolor{currentfill}{rgb}{0.121569,0.466667,0.705882}%
\pgfsetfillcolor{currentfill}%
\pgfsetfillopacity{0.801700}%
\pgfsetlinewidth{1.003750pt}%
\definecolor{currentstroke}{rgb}{0.121569,0.466667,0.705882}%
\pgfsetstrokecolor{currentstroke}%
\pgfsetstrokeopacity{0.801700}%
\pgfsetdash{}{0pt}%
\pgfpathmoveto{\pgfqpoint{2.846868in}{1.775538in}}%
\pgfpathcurveto{\pgfqpoint{2.855104in}{1.775538in}}{\pgfqpoint{2.863004in}{1.778811in}}{\pgfqpoint{2.868828in}{1.784635in}}%
\pgfpathcurveto{\pgfqpoint{2.874652in}{1.790459in}}{\pgfqpoint{2.877924in}{1.798359in}}{\pgfqpoint{2.877924in}{1.806595in}}%
\pgfpathcurveto{\pgfqpoint{2.877924in}{1.814831in}}{\pgfqpoint{2.874652in}{1.822731in}}{\pgfqpoint{2.868828in}{1.828555in}}%
\pgfpathcurveto{\pgfqpoint{2.863004in}{1.834379in}}{\pgfqpoint{2.855104in}{1.837651in}}{\pgfqpoint{2.846868in}{1.837651in}}%
\pgfpathcurveto{\pgfqpoint{2.838631in}{1.837651in}}{\pgfqpoint{2.830731in}{1.834379in}}{\pgfqpoint{2.824907in}{1.828555in}}%
\pgfpathcurveto{\pgfqpoint{2.819083in}{1.822731in}}{\pgfqpoint{2.815811in}{1.814831in}}{\pgfqpoint{2.815811in}{1.806595in}}%
\pgfpathcurveto{\pgfqpoint{2.815811in}{1.798359in}}{\pgfqpoint{2.819083in}{1.790459in}}{\pgfqpoint{2.824907in}{1.784635in}}%
\pgfpathcurveto{\pgfqpoint{2.830731in}{1.778811in}}{\pgfqpoint{2.838631in}{1.775538in}}{\pgfqpoint{2.846868in}{1.775538in}}%
\pgfpathclose%
\pgfusepath{stroke,fill}%
\end{pgfscope}%
\begin{pgfscope}%
\pgfpathrectangle{\pgfqpoint{0.100000in}{0.212622in}}{\pgfqpoint{3.696000in}{3.696000in}}%
\pgfusepath{clip}%
\pgfsetbuttcap%
\pgfsetroundjoin%
\definecolor{currentfill}{rgb}{0.121569,0.466667,0.705882}%
\pgfsetfillcolor{currentfill}%
\pgfsetfillopacity{0.802588}%
\pgfsetlinewidth{1.003750pt}%
\definecolor{currentstroke}{rgb}{0.121569,0.466667,0.705882}%
\pgfsetstrokecolor{currentstroke}%
\pgfsetstrokeopacity{0.802588}%
\pgfsetdash{}{0pt}%
\pgfpathmoveto{\pgfqpoint{2.846453in}{1.773573in}}%
\pgfpathcurveto{\pgfqpoint{2.854689in}{1.773573in}}{\pgfqpoint{2.862589in}{1.776845in}}{\pgfqpoint{2.868413in}{1.782669in}}%
\pgfpathcurveto{\pgfqpoint{2.874237in}{1.788493in}}{\pgfqpoint{2.877510in}{1.796393in}}{\pgfqpoint{2.877510in}{1.804629in}}%
\pgfpathcurveto{\pgfqpoint{2.877510in}{1.812866in}}{\pgfqpoint{2.874237in}{1.820766in}}{\pgfqpoint{2.868413in}{1.826590in}}%
\pgfpathcurveto{\pgfqpoint{2.862589in}{1.832414in}}{\pgfqpoint{2.854689in}{1.835686in}}{\pgfqpoint{2.846453in}{1.835686in}}%
\pgfpathcurveto{\pgfqpoint{2.838217in}{1.835686in}}{\pgfqpoint{2.830317in}{1.832414in}}{\pgfqpoint{2.824493in}{1.826590in}}%
\pgfpathcurveto{\pgfqpoint{2.818669in}{1.820766in}}{\pgfqpoint{2.815397in}{1.812866in}}{\pgfqpoint{2.815397in}{1.804629in}}%
\pgfpathcurveto{\pgfqpoint{2.815397in}{1.796393in}}{\pgfqpoint{2.818669in}{1.788493in}}{\pgfqpoint{2.824493in}{1.782669in}}%
\pgfpathcurveto{\pgfqpoint{2.830317in}{1.776845in}}{\pgfqpoint{2.838217in}{1.773573in}}{\pgfqpoint{2.846453in}{1.773573in}}%
\pgfpathclose%
\pgfusepath{stroke,fill}%
\end{pgfscope}%
\begin{pgfscope}%
\pgfpathrectangle{\pgfqpoint{0.100000in}{0.212622in}}{\pgfqpoint{3.696000in}{3.696000in}}%
\pgfusepath{clip}%
\pgfsetbuttcap%
\pgfsetroundjoin%
\definecolor{currentfill}{rgb}{0.121569,0.466667,0.705882}%
\pgfsetfillcolor{currentfill}%
\pgfsetfillopacity{0.803358}%
\pgfsetlinewidth{1.003750pt}%
\definecolor{currentstroke}{rgb}{0.121569,0.466667,0.705882}%
\pgfsetstrokecolor{currentstroke}%
\pgfsetstrokeopacity{0.803358}%
\pgfsetdash{}{0pt}%
\pgfpathmoveto{\pgfqpoint{2.845708in}{1.769805in}}%
\pgfpathcurveto{\pgfqpoint{2.853944in}{1.769805in}}{\pgfqpoint{2.861844in}{1.773078in}}{\pgfqpoint{2.867668in}{1.778902in}}%
\pgfpathcurveto{\pgfqpoint{2.873492in}{1.784726in}}{\pgfqpoint{2.876764in}{1.792626in}}{\pgfqpoint{2.876764in}{1.800862in}}%
\pgfpathcurveto{\pgfqpoint{2.876764in}{1.809098in}}{\pgfqpoint{2.873492in}{1.816998in}}{\pgfqpoint{2.867668in}{1.822822in}}%
\pgfpathcurveto{\pgfqpoint{2.861844in}{1.828646in}}{\pgfqpoint{2.853944in}{1.831918in}}{\pgfqpoint{2.845708in}{1.831918in}}%
\pgfpathcurveto{\pgfqpoint{2.837472in}{1.831918in}}{\pgfqpoint{2.829572in}{1.828646in}}{\pgfqpoint{2.823748in}{1.822822in}}%
\pgfpathcurveto{\pgfqpoint{2.817924in}{1.816998in}}{\pgfqpoint{2.814651in}{1.809098in}}{\pgfqpoint{2.814651in}{1.800862in}}%
\pgfpathcurveto{\pgfqpoint{2.814651in}{1.792626in}}{\pgfqpoint{2.817924in}{1.784726in}}{\pgfqpoint{2.823748in}{1.778902in}}%
\pgfpathcurveto{\pgfqpoint{2.829572in}{1.773078in}}{\pgfqpoint{2.837472in}{1.769805in}}{\pgfqpoint{2.845708in}{1.769805in}}%
\pgfpathclose%
\pgfusepath{stroke,fill}%
\end{pgfscope}%
\begin{pgfscope}%
\pgfpathrectangle{\pgfqpoint{0.100000in}{0.212622in}}{\pgfqpoint{3.696000in}{3.696000in}}%
\pgfusepath{clip}%
\pgfsetbuttcap%
\pgfsetroundjoin%
\definecolor{currentfill}{rgb}{0.121569,0.466667,0.705882}%
\pgfsetfillcolor{currentfill}%
\pgfsetfillopacity{0.804856}%
\pgfsetlinewidth{1.003750pt}%
\definecolor{currentstroke}{rgb}{0.121569,0.466667,0.705882}%
\pgfsetstrokecolor{currentstroke}%
\pgfsetstrokeopacity{0.804856}%
\pgfsetdash{}{0pt}%
\pgfpathmoveto{\pgfqpoint{2.842372in}{1.766978in}}%
\pgfpathcurveto{\pgfqpoint{2.850608in}{1.766978in}}{\pgfqpoint{2.858508in}{1.770251in}}{\pgfqpoint{2.864332in}{1.776075in}}%
\pgfpathcurveto{\pgfqpoint{2.870156in}{1.781899in}}{\pgfqpoint{2.873428in}{1.789799in}}{\pgfqpoint{2.873428in}{1.798035in}}%
\pgfpathcurveto{\pgfqpoint{2.873428in}{1.806271in}}{\pgfqpoint{2.870156in}{1.814171in}}{\pgfqpoint{2.864332in}{1.819995in}}%
\pgfpathcurveto{\pgfqpoint{2.858508in}{1.825819in}}{\pgfqpoint{2.850608in}{1.829091in}}{\pgfqpoint{2.842372in}{1.829091in}}%
\pgfpathcurveto{\pgfqpoint{2.834135in}{1.829091in}}{\pgfqpoint{2.826235in}{1.825819in}}{\pgfqpoint{2.820411in}{1.819995in}}%
\pgfpathcurveto{\pgfqpoint{2.814587in}{1.814171in}}{\pgfqpoint{2.811315in}{1.806271in}}{\pgfqpoint{2.811315in}{1.798035in}}%
\pgfpathcurveto{\pgfqpoint{2.811315in}{1.789799in}}{\pgfqpoint{2.814587in}{1.781899in}}{\pgfqpoint{2.820411in}{1.776075in}}%
\pgfpathcurveto{\pgfqpoint{2.826235in}{1.770251in}}{\pgfqpoint{2.834135in}{1.766978in}}{\pgfqpoint{2.842372in}{1.766978in}}%
\pgfpathclose%
\pgfusepath{stroke,fill}%
\end{pgfscope}%
\begin{pgfscope}%
\pgfpathrectangle{\pgfqpoint{0.100000in}{0.212622in}}{\pgfqpoint{3.696000in}{3.696000in}}%
\pgfusepath{clip}%
\pgfsetbuttcap%
\pgfsetroundjoin%
\definecolor{currentfill}{rgb}{0.121569,0.466667,0.705882}%
\pgfsetfillcolor{currentfill}%
\pgfsetfillopacity{0.805487}%
\pgfsetlinewidth{1.003750pt}%
\definecolor{currentstroke}{rgb}{0.121569,0.466667,0.705882}%
\pgfsetstrokecolor{currentstroke}%
\pgfsetstrokeopacity{0.805487}%
\pgfsetdash{}{0pt}%
\pgfpathmoveto{\pgfqpoint{2.840469in}{1.764156in}}%
\pgfpathcurveto{\pgfqpoint{2.848705in}{1.764156in}}{\pgfqpoint{2.856605in}{1.767428in}}{\pgfqpoint{2.862429in}{1.773252in}}%
\pgfpathcurveto{\pgfqpoint{2.868253in}{1.779076in}}{\pgfqpoint{2.871525in}{1.786976in}}{\pgfqpoint{2.871525in}{1.795213in}}%
\pgfpathcurveto{\pgfqpoint{2.871525in}{1.803449in}}{\pgfqpoint{2.868253in}{1.811349in}}{\pgfqpoint{2.862429in}{1.817173in}}%
\pgfpathcurveto{\pgfqpoint{2.856605in}{1.822997in}}{\pgfqpoint{2.848705in}{1.826269in}}{\pgfqpoint{2.840469in}{1.826269in}}%
\pgfpathcurveto{\pgfqpoint{2.832233in}{1.826269in}}{\pgfqpoint{2.824332in}{1.822997in}}{\pgfqpoint{2.818509in}{1.817173in}}%
\pgfpathcurveto{\pgfqpoint{2.812685in}{1.811349in}}{\pgfqpoint{2.809412in}{1.803449in}}{\pgfqpoint{2.809412in}{1.795213in}}%
\pgfpathcurveto{\pgfqpoint{2.809412in}{1.786976in}}{\pgfqpoint{2.812685in}{1.779076in}}{\pgfqpoint{2.818509in}{1.773252in}}%
\pgfpathcurveto{\pgfqpoint{2.824332in}{1.767428in}}{\pgfqpoint{2.832233in}{1.764156in}}{\pgfqpoint{2.840469in}{1.764156in}}%
\pgfpathclose%
\pgfusepath{stroke,fill}%
\end{pgfscope}%
\begin{pgfscope}%
\pgfpathrectangle{\pgfqpoint{0.100000in}{0.212622in}}{\pgfqpoint{3.696000in}{3.696000in}}%
\pgfusepath{clip}%
\pgfsetbuttcap%
\pgfsetroundjoin%
\definecolor{currentfill}{rgb}{0.121569,0.466667,0.705882}%
\pgfsetfillcolor{currentfill}%
\pgfsetfillopacity{0.806785}%
\pgfsetlinewidth{1.003750pt}%
\definecolor{currentstroke}{rgb}{0.121569,0.466667,0.705882}%
\pgfsetstrokecolor{currentstroke}%
\pgfsetstrokeopacity{0.806785}%
\pgfsetdash{}{0pt}%
\pgfpathmoveto{\pgfqpoint{2.839096in}{1.762448in}}%
\pgfpathcurveto{\pgfqpoint{2.847333in}{1.762448in}}{\pgfqpoint{2.855233in}{1.765720in}}{\pgfqpoint{2.861057in}{1.771544in}}%
\pgfpathcurveto{\pgfqpoint{2.866881in}{1.777368in}}{\pgfqpoint{2.870153in}{1.785268in}}{\pgfqpoint{2.870153in}{1.793504in}}%
\pgfpathcurveto{\pgfqpoint{2.870153in}{1.801740in}}{\pgfqpoint{2.866881in}{1.809641in}}{\pgfqpoint{2.861057in}{1.815464in}}%
\pgfpathcurveto{\pgfqpoint{2.855233in}{1.821288in}}{\pgfqpoint{2.847333in}{1.824561in}}{\pgfqpoint{2.839096in}{1.824561in}}%
\pgfpathcurveto{\pgfqpoint{2.830860in}{1.824561in}}{\pgfqpoint{2.822960in}{1.821288in}}{\pgfqpoint{2.817136in}{1.815464in}}%
\pgfpathcurveto{\pgfqpoint{2.811312in}{1.809641in}}{\pgfqpoint{2.808040in}{1.801740in}}{\pgfqpoint{2.808040in}{1.793504in}}%
\pgfpathcurveto{\pgfqpoint{2.808040in}{1.785268in}}{\pgfqpoint{2.811312in}{1.777368in}}{\pgfqpoint{2.817136in}{1.771544in}}%
\pgfpathcurveto{\pgfqpoint{2.822960in}{1.765720in}}{\pgfqpoint{2.830860in}{1.762448in}}{\pgfqpoint{2.839096in}{1.762448in}}%
\pgfpathclose%
\pgfusepath{stroke,fill}%
\end{pgfscope}%
\begin{pgfscope}%
\pgfpathrectangle{\pgfqpoint{0.100000in}{0.212622in}}{\pgfqpoint{3.696000in}{3.696000in}}%
\pgfusepath{clip}%
\pgfsetbuttcap%
\pgfsetroundjoin%
\definecolor{currentfill}{rgb}{0.121569,0.466667,0.705882}%
\pgfsetfillcolor{currentfill}%
\pgfsetfillopacity{0.807826}%
\pgfsetlinewidth{1.003750pt}%
\definecolor{currentstroke}{rgb}{0.121569,0.466667,0.705882}%
\pgfsetstrokecolor{currentstroke}%
\pgfsetstrokeopacity{0.807826}%
\pgfsetdash{}{0pt}%
\pgfpathmoveto{\pgfqpoint{2.835667in}{1.758360in}}%
\pgfpathcurveto{\pgfqpoint{2.843903in}{1.758360in}}{\pgfqpoint{2.851803in}{1.761632in}}{\pgfqpoint{2.857627in}{1.767456in}}%
\pgfpathcurveto{\pgfqpoint{2.863451in}{1.773280in}}{\pgfqpoint{2.866724in}{1.781180in}}{\pgfqpoint{2.866724in}{1.789416in}}%
\pgfpathcurveto{\pgfqpoint{2.866724in}{1.797652in}}{\pgfqpoint{2.863451in}{1.805552in}}{\pgfqpoint{2.857627in}{1.811376in}}%
\pgfpathcurveto{\pgfqpoint{2.851803in}{1.817200in}}{\pgfqpoint{2.843903in}{1.820473in}}{\pgfqpoint{2.835667in}{1.820473in}}%
\pgfpathcurveto{\pgfqpoint{2.827431in}{1.820473in}}{\pgfqpoint{2.819531in}{1.817200in}}{\pgfqpoint{2.813707in}{1.811376in}}%
\pgfpathcurveto{\pgfqpoint{2.807883in}{1.805552in}}{\pgfqpoint{2.804611in}{1.797652in}}{\pgfqpoint{2.804611in}{1.789416in}}%
\pgfpathcurveto{\pgfqpoint{2.804611in}{1.781180in}}{\pgfqpoint{2.807883in}{1.773280in}}{\pgfqpoint{2.813707in}{1.767456in}}%
\pgfpathcurveto{\pgfqpoint{2.819531in}{1.761632in}}{\pgfqpoint{2.827431in}{1.758360in}}{\pgfqpoint{2.835667in}{1.758360in}}%
\pgfpathclose%
\pgfusepath{stroke,fill}%
\end{pgfscope}%
\begin{pgfscope}%
\pgfpathrectangle{\pgfqpoint{0.100000in}{0.212622in}}{\pgfqpoint{3.696000in}{3.696000in}}%
\pgfusepath{clip}%
\pgfsetbuttcap%
\pgfsetroundjoin%
\definecolor{currentfill}{rgb}{0.121569,0.466667,0.705882}%
\pgfsetfillcolor{currentfill}%
\pgfsetfillopacity{0.809499}%
\pgfsetlinewidth{1.003750pt}%
\definecolor{currentstroke}{rgb}{0.121569,0.466667,0.705882}%
\pgfsetstrokecolor{currentstroke}%
\pgfsetstrokeopacity{0.809499}%
\pgfsetdash{}{0pt}%
\pgfpathmoveto{\pgfqpoint{2.830591in}{1.760148in}}%
\pgfpathcurveto{\pgfqpoint{2.838827in}{1.760148in}}{\pgfqpoint{2.846727in}{1.763420in}}{\pgfqpoint{2.852551in}{1.769244in}}%
\pgfpathcurveto{\pgfqpoint{2.858375in}{1.775068in}}{\pgfqpoint{2.861647in}{1.782968in}}{\pgfqpoint{2.861647in}{1.791205in}}%
\pgfpathcurveto{\pgfqpoint{2.861647in}{1.799441in}}{\pgfqpoint{2.858375in}{1.807341in}}{\pgfqpoint{2.852551in}{1.813165in}}%
\pgfpathcurveto{\pgfqpoint{2.846727in}{1.818989in}}{\pgfqpoint{2.838827in}{1.822261in}}{\pgfqpoint{2.830591in}{1.822261in}}%
\pgfpathcurveto{\pgfqpoint{2.822355in}{1.822261in}}{\pgfqpoint{2.814455in}{1.818989in}}{\pgfqpoint{2.808631in}{1.813165in}}%
\pgfpathcurveto{\pgfqpoint{2.802807in}{1.807341in}}{\pgfqpoint{2.799534in}{1.799441in}}{\pgfqpoint{2.799534in}{1.791205in}}%
\pgfpathcurveto{\pgfqpoint{2.799534in}{1.782968in}}{\pgfqpoint{2.802807in}{1.775068in}}{\pgfqpoint{2.808631in}{1.769244in}}%
\pgfpathcurveto{\pgfqpoint{2.814455in}{1.763420in}}{\pgfqpoint{2.822355in}{1.760148in}}{\pgfqpoint{2.830591in}{1.760148in}}%
\pgfpathclose%
\pgfusepath{stroke,fill}%
\end{pgfscope}%
\begin{pgfscope}%
\pgfpathrectangle{\pgfqpoint{0.100000in}{0.212622in}}{\pgfqpoint{3.696000in}{3.696000in}}%
\pgfusepath{clip}%
\pgfsetbuttcap%
\pgfsetroundjoin%
\definecolor{currentfill}{rgb}{0.121569,0.466667,0.705882}%
\pgfsetfillcolor{currentfill}%
\pgfsetfillopacity{0.812511}%
\pgfsetlinewidth{1.003750pt}%
\definecolor{currentstroke}{rgb}{0.121569,0.466667,0.705882}%
\pgfsetstrokecolor{currentstroke}%
\pgfsetstrokeopacity{0.812511}%
\pgfsetdash{}{0pt}%
\pgfpathmoveto{\pgfqpoint{2.828651in}{1.760415in}}%
\pgfpathcurveto{\pgfqpoint{2.836888in}{1.760415in}}{\pgfqpoint{2.844788in}{1.763687in}}{\pgfqpoint{2.850612in}{1.769511in}}%
\pgfpathcurveto{\pgfqpoint{2.856436in}{1.775335in}}{\pgfqpoint{2.859708in}{1.783235in}}{\pgfqpoint{2.859708in}{1.791471in}}%
\pgfpathcurveto{\pgfqpoint{2.859708in}{1.799708in}}{\pgfqpoint{2.856436in}{1.807608in}}{\pgfqpoint{2.850612in}{1.813432in}}%
\pgfpathcurveto{\pgfqpoint{2.844788in}{1.819255in}}{\pgfqpoint{2.836888in}{1.822528in}}{\pgfqpoint{2.828651in}{1.822528in}}%
\pgfpathcurveto{\pgfqpoint{2.820415in}{1.822528in}}{\pgfqpoint{2.812515in}{1.819255in}}{\pgfqpoint{2.806691in}{1.813432in}}%
\pgfpathcurveto{\pgfqpoint{2.800867in}{1.807608in}}{\pgfqpoint{2.797595in}{1.799708in}}{\pgfqpoint{2.797595in}{1.791471in}}%
\pgfpathcurveto{\pgfqpoint{2.797595in}{1.783235in}}{\pgfqpoint{2.800867in}{1.775335in}}{\pgfqpoint{2.806691in}{1.769511in}}%
\pgfpathcurveto{\pgfqpoint{2.812515in}{1.763687in}}{\pgfqpoint{2.820415in}{1.760415in}}{\pgfqpoint{2.828651in}{1.760415in}}%
\pgfpathclose%
\pgfusepath{stroke,fill}%
\end{pgfscope}%
\begin{pgfscope}%
\pgfpathrectangle{\pgfqpoint{0.100000in}{0.212622in}}{\pgfqpoint{3.696000in}{3.696000in}}%
\pgfusepath{clip}%
\pgfsetbuttcap%
\pgfsetroundjoin%
\definecolor{currentfill}{rgb}{0.121569,0.466667,0.705882}%
\pgfsetfillcolor{currentfill}%
\pgfsetfillopacity{0.813757}%
\pgfsetlinewidth{1.003750pt}%
\definecolor{currentstroke}{rgb}{0.121569,0.466667,0.705882}%
\pgfsetstrokecolor{currentstroke}%
\pgfsetstrokeopacity{0.813757}%
\pgfsetdash{}{0pt}%
\pgfpathmoveto{\pgfqpoint{2.822552in}{1.751121in}}%
\pgfpathcurveto{\pgfqpoint{2.830788in}{1.751121in}}{\pgfqpoint{2.838689in}{1.754394in}}{\pgfqpoint{2.844512in}{1.760218in}}%
\pgfpathcurveto{\pgfqpoint{2.850336in}{1.766042in}}{\pgfqpoint{2.853609in}{1.773942in}}{\pgfqpoint{2.853609in}{1.782178in}}%
\pgfpathcurveto{\pgfqpoint{2.853609in}{1.790414in}}{\pgfqpoint{2.850336in}{1.798314in}}{\pgfqpoint{2.844512in}{1.804138in}}%
\pgfpathcurveto{\pgfqpoint{2.838689in}{1.809962in}}{\pgfqpoint{2.830788in}{1.813234in}}{\pgfqpoint{2.822552in}{1.813234in}}%
\pgfpathcurveto{\pgfqpoint{2.814316in}{1.813234in}}{\pgfqpoint{2.806416in}{1.809962in}}{\pgfqpoint{2.800592in}{1.804138in}}%
\pgfpathcurveto{\pgfqpoint{2.794768in}{1.798314in}}{\pgfqpoint{2.791496in}{1.790414in}}{\pgfqpoint{2.791496in}{1.782178in}}%
\pgfpathcurveto{\pgfqpoint{2.791496in}{1.773942in}}{\pgfqpoint{2.794768in}{1.766042in}}{\pgfqpoint{2.800592in}{1.760218in}}%
\pgfpathcurveto{\pgfqpoint{2.806416in}{1.754394in}}{\pgfqpoint{2.814316in}{1.751121in}}{\pgfqpoint{2.822552in}{1.751121in}}%
\pgfpathclose%
\pgfusepath{stroke,fill}%
\end{pgfscope}%
\begin{pgfscope}%
\pgfpathrectangle{\pgfqpoint{0.100000in}{0.212622in}}{\pgfqpoint{3.696000in}{3.696000in}}%
\pgfusepath{clip}%
\pgfsetbuttcap%
\pgfsetroundjoin%
\definecolor{currentfill}{rgb}{0.121569,0.466667,0.705882}%
\pgfsetfillcolor{currentfill}%
\pgfsetfillopacity{0.816183}%
\pgfsetlinewidth{1.003750pt}%
\definecolor{currentstroke}{rgb}{0.121569,0.466667,0.705882}%
\pgfsetstrokecolor{currentstroke}%
\pgfsetstrokeopacity{0.816183}%
\pgfsetdash{}{0pt}%
\pgfpathmoveto{\pgfqpoint{2.814191in}{1.751790in}}%
\pgfpathcurveto{\pgfqpoint{2.822427in}{1.751790in}}{\pgfqpoint{2.830327in}{1.755063in}}{\pgfqpoint{2.836151in}{1.760887in}}%
\pgfpathcurveto{\pgfqpoint{2.841975in}{1.766710in}}{\pgfqpoint{2.845247in}{1.774611in}}{\pgfqpoint{2.845247in}{1.782847in}}%
\pgfpathcurveto{\pgfqpoint{2.845247in}{1.791083in}}{\pgfqpoint{2.841975in}{1.798983in}}{\pgfqpoint{2.836151in}{1.804807in}}%
\pgfpathcurveto{\pgfqpoint{2.830327in}{1.810631in}}{\pgfqpoint{2.822427in}{1.813903in}}{\pgfqpoint{2.814191in}{1.813903in}}%
\pgfpathcurveto{\pgfqpoint{2.805955in}{1.813903in}}{\pgfqpoint{2.798055in}{1.810631in}}{\pgfqpoint{2.792231in}{1.804807in}}%
\pgfpathcurveto{\pgfqpoint{2.786407in}{1.798983in}}{\pgfqpoint{2.783134in}{1.791083in}}{\pgfqpoint{2.783134in}{1.782847in}}%
\pgfpathcurveto{\pgfqpoint{2.783134in}{1.774611in}}{\pgfqpoint{2.786407in}{1.766710in}}{\pgfqpoint{2.792231in}{1.760887in}}%
\pgfpathcurveto{\pgfqpoint{2.798055in}{1.755063in}}{\pgfqpoint{2.805955in}{1.751790in}}{\pgfqpoint{2.814191in}{1.751790in}}%
\pgfpathclose%
\pgfusepath{stroke,fill}%
\end{pgfscope}%
\begin{pgfscope}%
\pgfpathrectangle{\pgfqpoint{0.100000in}{0.212622in}}{\pgfqpoint{3.696000in}{3.696000in}}%
\pgfusepath{clip}%
\pgfsetbuttcap%
\pgfsetroundjoin%
\definecolor{currentfill}{rgb}{0.121569,0.466667,0.705882}%
\pgfsetfillcolor{currentfill}%
\pgfsetfillopacity{0.821027}%
\pgfsetlinewidth{1.003750pt}%
\definecolor{currentstroke}{rgb}{0.121569,0.466667,0.705882}%
\pgfsetstrokecolor{currentstroke}%
\pgfsetstrokeopacity{0.821027}%
\pgfsetdash{}{0pt}%
\pgfpathmoveto{\pgfqpoint{2.811232in}{1.755664in}}%
\pgfpathcurveto{\pgfqpoint{2.819468in}{1.755664in}}{\pgfqpoint{2.827368in}{1.758937in}}{\pgfqpoint{2.833192in}{1.764761in}}%
\pgfpathcurveto{\pgfqpoint{2.839016in}{1.770585in}}{\pgfqpoint{2.842289in}{1.778485in}}{\pgfqpoint{2.842289in}{1.786721in}}%
\pgfpathcurveto{\pgfqpoint{2.842289in}{1.794957in}}{\pgfqpoint{2.839016in}{1.802857in}}{\pgfqpoint{2.833192in}{1.808681in}}%
\pgfpathcurveto{\pgfqpoint{2.827368in}{1.814505in}}{\pgfqpoint{2.819468in}{1.817777in}}{\pgfqpoint{2.811232in}{1.817777in}}%
\pgfpathcurveto{\pgfqpoint{2.802996in}{1.817777in}}{\pgfqpoint{2.795096in}{1.814505in}}{\pgfqpoint{2.789272in}{1.808681in}}%
\pgfpathcurveto{\pgfqpoint{2.783448in}{1.802857in}}{\pgfqpoint{2.780176in}{1.794957in}}{\pgfqpoint{2.780176in}{1.786721in}}%
\pgfpathcurveto{\pgfqpoint{2.780176in}{1.778485in}}{\pgfqpoint{2.783448in}{1.770585in}}{\pgfqpoint{2.789272in}{1.764761in}}%
\pgfpathcurveto{\pgfqpoint{2.795096in}{1.758937in}}{\pgfqpoint{2.802996in}{1.755664in}}{\pgfqpoint{2.811232in}{1.755664in}}%
\pgfpathclose%
\pgfusepath{stroke,fill}%
\end{pgfscope}%
\begin{pgfscope}%
\pgfpathrectangle{\pgfqpoint{0.100000in}{0.212622in}}{\pgfqpoint{3.696000in}{3.696000in}}%
\pgfusepath{clip}%
\pgfsetbuttcap%
\pgfsetroundjoin%
\definecolor{currentfill}{rgb}{0.121569,0.466667,0.705882}%
\pgfsetfillcolor{currentfill}%
\pgfsetfillopacity{0.822142}%
\pgfsetlinewidth{1.003750pt}%
\definecolor{currentstroke}{rgb}{0.121569,0.466667,0.705882}%
\pgfsetstrokecolor{currentstroke}%
\pgfsetstrokeopacity{0.822142}%
\pgfsetdash{}{0pt}%
\pgfpathmoveto{\pgfqpoint{2.806737in}{1.749730in}}%
\pgfpathcurveto{\pgfqpoint{2.814973in}{1.749730in}}{\pgfqpoint{2.822873in}{1.753003in}}{\pgfqpoint{2.828697in}{1.758826in}}%
\pgfpathcurveto{\pgfqpoint{2.834521in}{1.764650in}}{\pgfqpoint{2.837793in}{1.772550in}}{\pgfqpoint{2.837793in}{1.780787in}}%
\pgfpathcurveto{\pgfqpoint{2.837793in}{1.789023in}}{\pgfqpoint{2.834521in}{1.796923in}}{\pgfqpoint{2.828697in}{1.802747in}}%
\pgfpathcurveto{\pgfqpoint{2.822873in}{1.808571in}}{\pgfqpoint{2.814973in}{1.811843in}}{\pgfqpoint{2.806737in}{1.811843in}}%
\pgfpathcurveto{\pgfqpoint{2.798501in}{1.811843in}}{\pgfqpoint{2.790600in}{1.808571in}}{\pgfqpoint{2.784777in}{1.802747in}}%
\pgfpathcurveto{\pgfqpoint{2.778953in}{1.796923in}}{\pgfqpoint{2.775680in}{1.789023in}}{\pgfqpoint{2.775680in}{1.780787in}}%
\pgfpathcurveto{\pgfqpoint{2.775680in}{1.772550in}}{\pgfqpoint{2.778953in}{1.764650in}}{\pgfqpoint{2.784777in}{1.758826in}}%
\pgfpathcurveto{\pgfqpoint{2.790600in}{1.753003in}}{\pgfqpoint{2.798501in}{1.749730in}}{\pgfqpoint{2.806737in}{1.749730in}}%
\pgfpathclose%
\pgfusepath{stroke,fill}%
\end{pgfscope}%
\begin{pgfscope}%
\pgfpathrectangle{\pgfqpoint{0.100000in}{0.212622in}}{\pgfqpoint{3.696000in}{3.696000in}}%
\pgfusepath{clip}%
\pgfsetbuttcap%
\pgfsetroundjoin%
\definecolor{currentfill}{rgb}{0.121569,0.466667,0.705882}%
\pgfsetfillcolor{currentfill}%
\pgfsetfillopacity{0.824275}%
\pgfsetlinewidth{1.003750pt}%
\definecolor{currentstroke}{rgb}{0.121569,0.466667,0.705882}%
\pgfsetstrokecolor{currentstroke}%
\pgfsetstrokeopacity{0.824275}%
\pgfsetdash{}{0pt}%
\pgfpathmoveto{\pgfqpoint{2.800906in}{1.749907in}}%
\pgfpathcurveto{\pgfqpoint{2.809143in}{1.749907in}}{\pgfqpoint{2.817043in}{1.753180in}}{\pgfqpoint{2.822867in}{1.759004in}}%
\pgfpathcurveto{\pgfqpoint{2.828691in}{1.764828in}}{\pgfqpoint{2.831963in}{1.772728in}}{\pgfqpoint{2.831963in}{1.780964in}}%
\pgfpathcurveto{\pgfqpoint{2.831963in}{1.789200in}}{\pgfqpoint{2.828691in}{1.797100in}}{\pgfqpoint{2.822867in}{1.802924in}}%
\pgfpathcurveto{\pgfqpoint{2.817043in}{1.808748in}}{\pgfqpoint{2.809143in}{1.812020in}}{\pgfqpoint{2.800906in}{1.812020in}}%
\pgfpathcurveto{\pgfqpoint{2.792670in}{1.812020in}}{\pgfqpoint{2.784770in}{1.808748in}}{\pgfqpoint{2.778946in}{1.802924in}}%
\pgfpathcurveto{\pgfqpoint{2.773122in}{1.797100in}}{\pgfqpoint{2.769850in}{1.789200in}}{\pgfqpoint{2.769850in}{1.780964in}}%
\pgfpathcurveto{\pgfqpoint{2.769850in}{1.772728in}}{\pgfqpoint{2.773122in}{1.764828in}}{\pgfqpoint{2.778946in}{1.759004in}}%
\pgfpathcurveto{\pgfqpoint{2.784770in}{1.753180in}}{\pgfqpoint{2.792670in}{1.749907in}}{\pgfqpoint{2.800906in}{1.749907in}}%
\pgfpathclose%
\pgfusepath{stroke,fill}%
\end{pgfscope}%
\begin{pgfscope}%
\pgfpathrectangle{\pgfqpoint{0.100000in}{0.212622in}}{\pgfqpoint{3.696000in}{3.696000in}}%
\pgfusepath{clip}%
\pgfsetbuttcap%
\pgfsetroundjoin%
\definecolor{currentfill}{rgb}{0.121569,0.466667,0.705882}%
\pgfsetfillcolor{currentfill}%
\pgfsetfillopacity{0.827622}%
\pgfsetlinewidth{1.003750pt}%
\definecolor{currentstroke}{rgb}{0.121569,0.466667,0.705882}%
\pgfsetstrokecolor{currentstroke}%
\pgfsetstrokeopacity{0.827622}%
\pgfsetdash{}{0pt}%
\pgfpathmoveto{\pgfqpoint{2.798432in}{1.748747in}}%
\pgfpathcurveto{\pgfqpoint{2.806668in}{1.748747in}}{\pgfqpoint{2.814568in}{1.752019in}}{\pgfqpoint{2.820392in}{1.757843in}}%
\pgfpathcurveto{\pgfqpoint{2.826216in}{1.763667in}}{\pgfqpoint{2.829488in}{1.771567in}}{\pgfqpoint{2.829488in}{1.779804in}}%
\pgfpathcurveto{\pgfqpoint{2.829488in}{1.788040in}}{\pgfqpoint{2.826216in}{1.795940in}}{\pgfqpoint{2.820392in}{1.801764in}}%
\pgfpathcurveto{\pgfqpoint{2.814568in}{1.807588in}}{\pgfqpoint{2.806668in}{1.810860in}}{\pgfqpoint{2.798432in}{1.810860in}}%
\pgfpathcurveto{\pgfqpoint{2.790196in}{1.810860in}}{\pgfqpoint{2.782296in}{1.807588in}}{\pgfqpoint{2.776472in}{1.801764in}}%
\pgfpathcurveto{\pgfqpoint{2.770648in}{1.795940in}}{\pgfqpoint{2.767375in}{1.788040in}}{\pgfqpoint{2.767375in}{1.779804in}}%
\pgfpathcurveto{\pgfqpoint{2.767375in}{1.771567in}}{\pgfqpoint{2.770648in}{1.763667in}}{\pgfqpoint{2.776472in}{1.757843in}}%
\pgfpathcurveto{\pgfqpoint{2.782296in}{1.752019in}}{\pgfqpoint{2.790196in}{1.748747in}}{\pgfqpoint{2.798432in}{1.748747in}}%
\pgfpathclose%
\pgfusepath{stroke,fill}%
\end{pgfscope}%
\begin{pgfscope}%
\pgfpathrectangle{\pgfqpoint{0.100000in}{0.212622in}}{\pgfqpoint{3.696000in}{3.696000in}}%
\pgfusepath{clip}%
\pgfsetbuttcap%
\pgfsetroundjoin%
\definecolor{currentfill}{rgb}{0.121569,0.466667,0.705882}%
\pgfsetfillcolor{currentfill}%
\pgfsetfillopacity{0.829622}%
\pgfsetlinewidth{1.003750pt}%
\definecolor{currentstroke}{rgb}{0.121569,0.466667,0.705882}%
\pgfsetstrokecolor{currentstroke}%
\pgfsetstrokeopacity{0.829622}%
\pgfsetdash{}{0pt}%
\pgfpathmoveto{\pgfqpoint{2.793300in}{1.739190in}}%
\pgfpathcurveto{\pgfqpoint{2.801537in}{1.739190in}}{\pgfqpoint{2.809437in}{1.742463in}}{\pgfqpoint{2.815261in}{1.748287in}}%
\pgfpathcurveto{\pgfqpoint{2.821085in}{1.754111in}}{\pgfqpoint{2.824357in}{1.762011in}}{\pgfqpoint{2.824357in}{1.770247in}}%
\pgfpathcurveto{\pgfqpoint{2.824357in}{1.778483in}}{\pgfqpoint{2.821085in}{1.786383in}}{\pgfqpoint{2.815261in}{1.792207in}}%
\pgfpathcurveto{\pgfqpoint{2.809437in}{1.798031in}}{\pgfqpoint{2.801537in}{1.801303in}}{\pgfqpoint{2.793300in}{1.801303in}}%
\pgfpathcurveto{\pgfqpoint{2.785064in}{1.801303in}}{\pgfqpoint{2.777164in}{1.798031in}}{\pgfqpoint{2.771340in}{1.792207in}}%
\pgfpathcurveto{\pgfqpoint{2.765516in}{1.786383in}}{\pgfqpoint{2.762244in}{1.778483in}}{\pgfqpoint{2.762244in}{1.770247in}}%
\pgfpathcurveto{\pgfqpoint{2.762244in}{1.762011in}}{\pgfqpoint{2.765516in}{1.754111in}}{\pgfqpoint{2.771340in}{1.748287in}}%
\pgfpathcurveto{\pgfqpoint{2.777164in}{1.742463in}}{\pgfqpoint{2.785064in}{1.739190in}}{\pgfqpoint{2.793300in}{1.739190in}}%
\pgfpathclose%
\pgfusepath{stroke,fill}%
\end{pgfscope}%
\begin{pgfscope}%
\pgfpathrectangle{\pgfqpoint{0.100000in}{0.212622in}}{\pgfqpoint{3.696000in}{3.696000in}}%
\pgfusepath{clip}%
\pgfsetbuttcap%
\pgfsetroundjoin%
\definecolor{currentfill}{rgb}{0.121569,0.466667,0.705882}%
\pgfsetfillcolor{currentfill}%
\pgfsetfillopacity{0.833798}%
\pgfsetlinewidth{1.003750pt}%
\definecolor{currentstroke}{rgb}{0.121569,0.466667,0.705882}%
\pgfsetstrokecolor{currentstroke}%
\pgfsetstrokeopacity{0.833798}%
\pgfsetdash{}{0pt}%
\pgfpathmoveto{\pgfqpoint{2.783568in}{1.746148in}}%
\pgfpathcurveto{\pgfqpoint{2.791804in}{1.746148in}}{\pgfqpoint{2.799704in}{1.749420in}}{\pgfqpoint{2.805528in}{1.755244in}}%
\pgfpathcurveto{\pgfqpoint{2.811352in}{1.761068in}}{\pgfqpoint{2.814624in}{1.768968in}}{\pgfqpoint{2.814624in}{1.777204in}}%
\pgfpathcurveto{\pgfqpoint{2.814624in}{1.785441in}}{\pgfqpoint{2.811352in}{1.793341in}}{\pgfqpoint{2.805528in}{1.799165in}}%
\pgfpathcurveto{\pgfqpoint{2.799704in}{1.804989in}}{\pgfqpoint{2.791804in}{1.808261in}}{\pgfqpoint{2.783568in}{1.808261in}}%
\pgfpathcurveto{\pgfqpoint{2.775332in}{1.808261in}}{\pgfqpoint{2.767432in}{1.804989in}}{\pgfqpoint{2.761608in}{1.799165in}}%
\pgfpathcurveto{\pgfqpoint{2.755784in}{1.793341in}}{\pgfqpoint{2.752511in}{1.785441in}}{\pgfqpoint{2.752511in}{1.777204in}}%
\pgfpathcurveto{\pgfqpoint{2.752511in}{1.768968in}}{\pgfqpoint{2.755784in}{1.761068in}}{\pgfqpoint{2.761608in}{1.755244in}}%
\pgfpathcurveto{\pgfqpoint{2.767432in}{1.749420in}}{\pgfqpoint{2.775332in}{1.746148in}}{\pgfqpoint{2.783568in}{1.746148in}}%
\pgfpathclose%
\pgfusepath{stroke,fill}%
\end{pgfscope}%
\begin{pgfscope}%
\pgfpathrectangle{\pgfqpoint{0.100000in}{0.212622in}}{\pgfqpoint{3.696000in}{3.696000in}}%
\pgfusepath{clip}%
\pgfsetbuttcap%
\pgfsetroundjoin%
\definecolor{currentfill}{rgb}{0.121569,0.466667,0.705882}%
\pgfsetfillcolor{currentfill}%
\pgfsetfillopacity{0.837279}%
\pgfsetlinewidth{1.003750pt}%
\definecolor{currentstroke}{rgb}{0.121569,0.466667,0.705882}%
\pgfsetstrokecolor{currentstroke}%
\pgfsetstrokeopacity{0.837279}%
\pgfsetdash{}{0pt}%
\pgfpathmoveto{\pgfqpoint{2.781246in}{1.739243in}}%
\pgfpathcurveto{\pgfqpoint{2.789483in}{1.739243in}}{\pgfqpoint{2.797383in}{1.742515in}}{\pgfqpoint{2.803206in}{1.748339in}}%
\pgfpathcurveto{\pgfqpoint{2.809030in}{1.754163in}}{\pgfqpoint{2.812303in}{1.762063in}}{\pgfqpoint{2.812303in}{1.770299in}}%
\pgfpathcurveto{\pgfqpoint{2.812303in}{1.778536in}}{\pgfqpoint{2.809030in}{1.786436in}}{\pgfqpoint{2.803206in}{1.792260in}}%
\pgfpathcurveto{\pgfqpoint{2.797383in}{1.798084in}}{\pgfqpoint{2.789483in}{1.801356in}}{\pgfqpoint{2.781246in}{1.801356in}}%
\pgfpathcurveto{\pgfqpoint{2.773010in}{1.801356in}}{\pgfqpoint{2.765110in}{1.798084in}}{\pgfqpoint{2.759286in}{1.792260in}}%
\pgfpathcurveto{\pgfqpoint{2.753462in}{1.786436in}}{\pgfqpoint{2.750190in}{1.778536in}}{\pgfqpoint{2.750190in}{1.770299in}}%
\pgfpathcurveto{\pgfqpoint{2.750190in}{1.762063in}}{\pgfqpoint{2.753462in}{1.754163in}}{\pgfqpoint{2.759286in}{1.748339in}}%
\pgfpathcurveto{\pgfqpoint{2.765110in}{1.742515in}}{\pgfqpoint{2.773010in}{1.739243in}}{\pgfqpoint{2.781246in}{1.739243in}}%
\pgfpathclose%
\pgfusepath{stroke,fill}%
\end{pgfscope}%
\begin{pgfscope}%
\pgfpathrectangle{\pgfqpoint{0.100000in}{0.212622in}}{\pgfqpoint{3.696000in}{3.696000in}}%
\pgfusepath{clip}%
\pgfsetbuttcap%
\pgfsetroundjoin%
\definecolor{currentfill}{rgb}{0.121569,0.466667,0.705882}%
\pgfsetfillcolor{currentfill}%
\pgfsetfillopacity{0.840121}%
\pgfsetlinewidth{1.003750pt}%
\definecolor{currentstroke}{rgb}{0.121569,0.466667,0.705882}%
\pgfsetstrokecolor{currentstroke}%
\pgfsetstrokeopacity{0.840121}%
\pgfsetdash{}{0pt}%
\pgfpathmoveto{\pgfqpoint{2.777006in}{1.727558in}}%
\pgfpathcurveto{\pgfqpoint{2.785243in}{1.727558in}}{\pgfqpoint{2.793143in}{1.730830in}}{\pgfqpoint{2.798967in}{1.736654in}}%
\pgfpathcurveto{\pgfqpoint{2.804791in}{1.742478in}}{\pgfqpoint{2.808063in}{1.750378in}}{\pgfqpoint{2.808063in}{1.758614in}}%
\pgfpathcurveto{\pgfqpoint{2.808063in}{1.766850in}}{\pgfqpoint{2.804791in}{1.774750in}}{\pgfqpoint{2.798967in}{1.780574in}}%
\pgfpathcurveto{\pgfqpoint{2.793143in}{1.786398in}}{\pgfqpoint{2.785243in}{1.789671in}}{\pgfqpoint{2.777006in}{1.789671in}}%
\pgfpathcurveto{\pgfqpoint{2.768770in}{1.789671in}}{\pgfqpoint{2.760870in}{1.786398in}}{\pgfqpoint{2.755046in}{1.780574in}}%
\pgfpathcurveto{\pgfqpoint{2.749222in}{1.774750in}}{\pgfqpoint{2.745950in}{1.766850in}}{\pgfqpoint{2.745950in}{1.758614in}}%
\pgfpathcurveto{\pgfqpoint{2.745950in}{1.750378in}}{\pgfqpoint{2.749222in}{1.742478in}}{\pgfqpoint{2.755046in}{1.736654in}}%
\pgfpathcurveto{\pgfqpoint{2.760870in}{1.730830in}}{\pgfqpoint{2.768770in}{1.727558in}}{\pgfqpoint{2.777006in}{1.727558in}}%
\pgfpathclose%
\pgfusepath{stroke,fill}%
\end{pgfscope}%
\begin{pgfscope}%
\pgfpathrectangle{\pgfqpoint{0.100000in}{0.212622in}}{\pgfqpoint{3.696000in}{3.696000in}}%
\pgfusepath{clip}%
\pgfsetbuttcap%
\pgfsetroundjoin%
\definecolor{currentfill}{rgb}{0.121569,0.466667,0.705882}%
\pgfsetfillcolor{currentfill}%
\pgfsetfillopacity{0.845803}%
\pgfsetlinewidth{1.003750pt}%
\definecolor{currentstroke}{rgb}{0.121569,0.466667,0.705882}%
\pgfsetstrokecolor{currentstroke}%
\pgfsetstrokeopacity{0.845803}%
\pgfsetdash{}{0pt}%
\pgfpathmoveto{\pgfqpoint{2.763817in}{1.737240in}}%
\pgfpathcurveto{\pgfqpoint{2.772053in}{1.737240in}}{\pgfqpoint{2.779953in}{1.740512in}}{\pgfqpoint{2.785777in}{1.746336in}}%
\pgfpathcurveto{\pgfqpoint{2.791601in}{1.752160in}}{\pgfqpoint{2.794874in}{1.760060in}}{\pgfqpoint{2.794874in}{1.768296in}}%
\pgfpathcurveto{\pgfqpoint{2.794874in}{1.776533in}}{\pgfqpoint{2.791601in}{1.784433in}}{\pgfqpoint{2.785777in}{1.790257in}}%
\pgfpathcurveto{\pgfqpoint{2.779953in}{1.796081in}}{\pgfqpoint{2.772053in}{1.799353in}}{\pgfqpoint{2.763817in}{1.799353in}}%
\pgfpathcurveto{\pgfqpoint{2.755581in}{1.799353in}}{\pgfqpoint{2.747681in}{1.796081in}}{\pgfqpoint{2.741857in}{1.790257in}}%
\pgfpathcurveto{\pgfqpoint{2.736033in}{1.784433in}}{\pgfqpoint{2.732761in}{1.776533in}}{\pgfqpoint{2.732761in}{1.768296in}}%
\pgfpathcurveto{\pgfqpoint{2.732761in}{1.760060in}}{\pgfqpoint{2.736033in}{1.752160in}}{\pgfqpoint{2.741857in}{1.746336in}}%
\pgfpathcurveto{\pgfqpoint{2.747681in}{1.740512in}}{\pgfqpoint{2.755581in}{1.737240in}}{\pgfqpoint{2.763817in}{1.737240in}}%
\pgfpathclose%
\pgfusepath{stroke,fill}%
\end{pgfscope}%
\begin{pgfscope}%
\pgfpathrectangle{\pgfqpoint{0.100000in}{0.212622in}}{\pgfqpoint{3.696000in}{3.696000in}}%
\pgfusepath{clip}%
\pgfsetbuttcap%
\pgfsetroundjoin%
\definecolor{currentfill}{rgb}{0.121569,0.466667,0.705882}%
\pgfsetfillcolor{currentfill}%
\pgfsetfillopacity{0.848516}%
\pgfsetlinewidth{1.003750pt}%
\definecolor{currentstroke}{rgb}{0.121569,0.466667,0.705882}%
\pgfsetstrokecolor{currentstroke}%
\pgfsetstrokeopacity{0.848516}%
\pgfsetdash{}{0pt}%
\pgfpathmoveto{\pgfqpoint{0.531950in}{2.582831in}}%
\pgfpathcurveto{\pgfqpoint{0.540186in}{2.582831in}}{\pgfqpoint{0.548087in}{2.586104in}}{\pgfqpoint{0.553910in}{2.591928in}}%
\pgfpathcurveto{\pgfqpoint{0.559734in}{2.597751in}}{\pgfqpoint{0.563007in}{2.605652in}}{\pgfqpoint{0.563007in}{2.613888in}}%
\pgfpathcurveto{\pgfqpoint{0.563007in}{2.622124in}}{\pgfqpoint{0.559734in}{2.630024in}}{\pgfqpoint{0.553910in}{2.635848in}}%
\pgfpathcurveto{\pgfqpoint{0.548087in}{2.641672in}}{\pgfqpoint{0.540186in}{2.644944in}}{\pgfqpoint{0.531950in}{2.644944in}}%
\pgfpathcurveto{\pgfqpoint{0.523714in}{2.644944in}}{\pgfqpoint{0.515814in}{2.641672in}}{\pgfqpoint{0.509990in}{2.635848in}}%
\pgfpathcurveto{\pgfqpoint{0.504166in}{2.630024in}}{\pgfqpoint{0.500894in}{2.622124in}}{\pgfqpoint{0.500894in}{2.613888in}}%
\pgfpathcurveto{\pgfqpoint{0.500894in}{2.605652in}}{\pgfqpoint{0.504166in}{2.597751in}}{\pgfqpoint{0.509990in}{2.591928in}}%
\pgfpathcurveto{\pgfqpoint{0.515814in}{2.586104in}}{\pgfqpoint{0.523714in}{2.582831in}}{\pgfqpoint{0.531950in}{2.582831in}}%
\pgfpathclose%
\pgfusepath{stroke,fill}%
\end{pgfscope}%
\begin{pgfscope}%
\pgfpathrectangle{\pgfqpoint{0.100000in}{0.212622in}}{\pgfqpoint{3.696000in}{3.696000in}}%
\pgfusepath{clip}%
\pgfsetbuttcap%
\pgfsetroundjoin%
\definecolor{currentfill}{rgb}{0.121569,0.466667,0.705882}%
\pgfsetfillcolor{currentfill}%
\pgfsetfillopacity{0.848532}%
\pgfsetlinewidth{1.003750pt}%
\definecolor{currentstroke}{rgb}{0.121569,0.466667,0.705882}%
\pgfsetstrokecolor{currentstroke}%
\pgfsetstrokeopacity{0.848532}%
\pgfsetdash{}{0pt}%
\pgfpathmoveto{\pgfqpoint{2.762379in}{1.733869in}}%
\pgfpathcurveto{\pgfqpoint{2.770615in}{1.733869in}}{\pgfqpoint{2.778515in}{1.737141in}}{\pgfqpoint{2.784339in}{1.742965in}}%
\pgfpathcurveto{\pgfqpoint{2.790163in}{1.748789in}}{\pgfqpoint{2.793436in}{1.756689in}}{\pgfqpoint{2.793436in}{1.764925in}}%
\pgfpathcurveto{\pgfqpoint{2.793436in}{1.773162in}}{\pgfqpoint{2.790163in}{1.781062in}}{\pgfqpoint{2.784339in}{1.786886in}}%
\pgfpathcurveto{\pgfqpoint{2.778515in}{1.792710in}}{\pgfqpoint{2.770615in}{1.795982in}}{\pgfqpoint{2.762379in}{1.795982in}}%
\pgfpathcurveto{\pgfqpoint{2.754143in}{1.795982in}}{\pgfqpoint{2.746243in}{1.792710in}}{\pgfqpoint{2.740419in}{1.786886in}}%
\pgfpathcurveto{\pgfqpoint{2.734595in}{1.781062in}}{\pgfqpoint{2.731323in}{1.773162in}}{\pgfqpoint{2.731323in}{1.764925in}}%
\pgfpathcurveto{\pgfqpoint{2.731323in}{1.756689in}}{\pgfqpoint{2.734595in}{1.748789in}}{\pgfqpoint{2.740419in}{1.742965in}}%
\pgfpathcurveto{\pgfqpoint{2.746243in}{1.737141in}}{\pgfqpoint{2.754143in}{1.733869in}}{\pgfqpoint{2.762379in}{1.733869in}}%
\pgfpathclose%
\pgfusepath{stroke,fill}%
\end{pgfscope}%
\begin{pgfscope}%
\pgfpathrectangle{\pgfqpoint{0.100000in}{0.212622in}}{\pgfqpoint{3.696000in}{3.696000in}}%
\pgfusepath{clip}%
\pgfsetbuttcap%
\pgfsetroundjoin%
\definecolor{currentfill}{rgb}{0.121569,0.466667,0.705882}%
\pgfsetfillcolor{currentfill}%
\pgfsetfillopacity{0.849285}%
\pgfsetlinewidth{1.003750pt}%
\definecolor{currentstroke}{rgb}{0.121569,0.466667,0.705882}%
\pgfsetstrokecolor{currentstroke}%
\pgfsetstrokeopacity{0.849285}%
\pgfsetdash{}{0pt}%
\pgfpathmoveto{\pgfqpoint{0.533700in}{2.582809in}}%
\pgfpathcurveto{\pgfqpoint{0.541936in}{2.582809in}}{\pgfqpoint{0.549836in}{2.586081in}}{\pgfqpoint{0.555660in}{2.591905in}}%
\pgfpathcurveto{\pgfqpoint{0.561484in}{2.597729in}}{\pgfqpoint{0.564756in}{2.605629in}}{\pgfqpoint{0.564756in}{2.613865in}}%
\pgfpathcurveto{\pgfqpoint{0.564756in}{2.622102in}}{\pgfqpoint{0.561484in}{2.630002in}}{\pgfqpoint{0.555660in}{2.635826in}}%
\pgfpathcurveto{\pgfqpoint{0.549836in}{2.641650in}}{\pgfqpoint{0.541936in}{2.644922in}}{\pgfqpoint{0.533700in}{2.644922in}}%
\pgfpathcurveto{\pgfqpoint{0.525464in}{2.644922in}}{\pgfqpoint{0.517564in}{2.641650in}}{\pgfqpoint{0.511740in}{2.635826in}}%
\pgfpathcurveto{\pgfqpoint{0.505916in}{2.630002in}}{\pgfqpoint{0.502643in}{2.622102in}}{\pgfqpoint{0.502643in}{2.613865in}}%
\pgfpathcurveto{\pgfqpoint{0.502643in}{2.605629in}}{\pgfqpoint{0.505916in}{2.597729in}}{\pgfqpoint{0.511740in}{2.591905in}}%
\pgfpathcurveto{\pgfqpoint{0.517564in}{2.586081in}}{\pgfqpoint{0.525464in}{2.582809in}}{\pgfqpoint{0.533700in}{2.582809in}}%
\pgfpathclose%
\pgfusepath{stroke,fill}%
\end{pgfscope}%
\begin{pgfscope}%
\pgfpathrectangle{\pgfqpoint{0.100000in}{0.212622in}}{\pgfqpoint{3.696000in}{3.696000in}}%
\pgfusepath{clip}%
\pgfsetbuttcap%
\pgfsetroundjoin%
\definecolor{currentfill}{rgb}{0.121569,0.466667,0.705882}%
\pgfsetfillcolor{currentfill}%
\pgfsetfillopacity{0.849897}%
\pgfsetlinewidth{1.003750pt}%
\definecolor{currentstroke}{rgb}{0.121569,0.466667,0.705882}%
\pgfsetstrokecolor{currentstroke}%
\pgfsetstrokeopacity{0.849897}%
\pgfsetdash{}{0pt}%
\pgfpathmoveto{\pgfqpoint{0.535091in}{2.582791in}}%
\pgfpathcurveto{\pgfqpoint{0.543327in}{2.582791in}}{\pgfqpoint{0.551227in}{2.586063in}}{\pgfqpoint{0.557051in}{2.591887in}}%
\pgfpathcurveto{\pgfqpoint{0.562875in}{2.597711in}}{\pgfqpoint{0.566147in}{2.605611in}}{\pgfqpoint{0.566147in}{2.613847in}}%
\pgfpathcurveto{\pgfqpoint{0.566147in}{2.622084in}}{\pgfqpoint{0.562875in}{2.629984in}}{\pgfqpoint{0.557051in}{2.635808in}}%
\pgfpathcurveto{\pgfqpoint{0.551227in}{2.641632in}}{\pgfqpoint{0.543327in}{2.644904in}}{\pgfqpoint{0.535091in}{2.644904in}}%
\pgfpathcurveto{\pgfqpoint{0.526854in}{2.644904in}}{\pgfqpoint{0.518954in}{2.641632in}}{\pgfqpoint{0.513130in}{2.635808in}}%
\pgfpathcurveto{\pgfqpoint{0.507306in}{2.629984in}}{\pgfqpoint{0.504034in}{2.622084in}}{\pgfqpoint{0.504034in}{2.613847in}}%
\pgfpathcurveto{\pgfqpoint{0.504034in}{2.605611in}}{\pgfqpoint{0.507306in}{2.597711in}}{\pgfqpoint{0.513130in}{2.591887in}}%
\pgfpathcurveto{\pgfqpoint{0.518954in}{2.586063in}}{\pgfqpoint{0.526854in}{2.582791in}}{\pgfqpoint{0.535091in}{2.582791in}}%
\pgfpathclose%
\pgfusepath{stroke,fill}%
\end{pgfscope}%
\begin{pgfscope}%
\pgfpathrectangle{\pgfqpoint{0.100000in}{0.212622in}}{\pgfqpoint{3.696000in}{3.696000in}}%
\pgfusepath{clip}%
\pgfsetbuttcap%
\pgfsetroundjoin%
\definecolor{currentfill}{rgb}{0.121569,0.466667,0.705882}%
\pgfsetfillcolor{currentfill}%
\pgfsetfillopacity{0.850168}%
\pgfsetlinewidth{1.003750pt}%
\definecolor{currentstroke}{rgb}{0.121569,0.466667,0.705882}%
\pgfsetstrokecolor{currentstroke}%
\pgfsetstrokeopacity{0.850168}%
\pgfsetdash{}{0pt}%
\pgfpathmoveto{\pgfqpoint{0.535840in}{2.582213in}}%
\pgfpathcurveto{\pgfqpoint{0.544076in}{2.582213in}}{\pgfqpoint{0.551976in}{2.585485in}}{\pgfqpoint{0.557800in}{2.591309in}}%
\pgfpathcurveto{\pgfqpoint{0.563624in}{2.597133in}}{\pgfqpoint{0.566896in}{2.605033in}}{\pgfqpoint{0.566896in}{2.613270in}}%
\pgfpathcurveto{\pgfqpoint{0.566896in}{2.621506in}}{\pgfqpoint{0.563624in}{2.629406in}}{\pgfqpoint{0.557800in}{2.635230in}}%
\pgfpathcurveto{\pgfqpoint{0.551976in}{2.641054in}}{\pgfqpoint{0.544076in}{2.644326in}}{\pgfqpoint{0.535840in}{2.644326in}}%
\pgfpathcurveto{\pgfqpoint{0.527604in}{2.644326in}}{\pgfqpoint{0.519704in}{2.641054in}}{\pgfqpoint{0.513880in}{2.635230in}}%
\pgfpathcurveto{\pgfqpoint{0.508056in}{2.629406in}}{\pgfqpoint{0.504783in}{2.621506in}}{\pgfqpoint{0.504783in}{2.613270in}}%
\pgfpathcurveto{\pgfqpoint{0.504783in}{2.605033in}}{\pgfqpoint{0.508056in}{2.597133in}}{\pgfqpoint{0.513880in}{2.591309in}}%
\pgfpathcurveto{\pgfqpoint{0.519704in}{2.585485in}}{\pgfqpoint{0.527604in}{2.582213in}}{\pgfqpoint{0.535840in}{2.582213in}}%
\pgfpathclose%
\pgfusepath{stroke,fill}%
\end{pgfscope}%
\begin{pgfscope}%
\pgfpathrectangle{\pgfqpoint{0.100000in}{0.212622in}}{\pgfqpoint{3.696000in}{3.696000in}}%
\pgfusepath{clip}%
\pgfsetbuttcap%
\pgfsetroundjoin%
\definecolor{currentfill}{rgb}{0.121569,0.466667,0.705882}%
\pgfsetfillcolor{currentfill}%
\pgfsetfillopacity{0.850201}%
\pgfsetlinewidth{1.003750pt}%
\definecolor{currentstroke}{rgb}{0.121569,0.466667,0.705882}%
\pgfsetstrokecolor{currentstroke}%
\pgfsetstrokeopacity{0.850201}%
\pgfsetdash{}{0pt}%
\pgfpathmoveto{\pgfqpoint{2.757220in}{1.724426in}}%
\pgfpathcurveto{\pgfqpoint{2.765456in}{1.724426in}}{\pgfqpoint{2.773356in}{1.727698in}}{\pgfqpoint{2.779180in}{1.733522in}}%
\pgfpathcurveto{\pgfqpoint{2.785004in}{1.739346in}}{\pgfqpoint{2.788277in}{1.747246in}}{\pgfqpoint{2.788277in}{1.755483in}}%
\pgfpathcurveto{\pgfqpoint{2.788277in}{1.763719in}}{\pgfqpoint{2.785004in}{1.771619in}}{\pgfqpoint{2.779180in}{1.777443in}}%
\pgfpathcurveto{\pgfqpoint{2.773356in}{1.783267in}}{\pgfqpoint{2.765456in}{1.786539in}}{\pgfqpoint{2.757220in}{1.786539in}}%
\pgfpathcurveto{\pgfqpoint{2.748984in}{1.786539in}}{\pgfqpoint{2.741084in}{1.783267in}}{\pgfqpoint{2.735260in}{1.777443in}}%
\pgfpathcurveto{\pgfqpoint{2.729436in}{1.771619in}}{\pgfqpoint{2.726164in}{1.763719in}}{\pgfqpoint{2.726164in}{1.755483in}}%
\pgfpathcurveto{\pgfqpoint{2.726164in}{1.747246in}}{\pgfqpoint{2.729436in}{1.739346in}}{\pgfqpoint{2.735260in}{1.733522in}}%
\pgfpathcurveto{\pgfqpoint{2.741084in}{1.727698in}}{\pgfqpoint{2.748984in}{1.724426in}}{\pgfqpoint{2.757220in}{1.724426in}}%
\pgfpathclose%
\pgfusepath{stroke,fill}%
\end{pgfscope}%
\begin{pgfscope}%
\pgfpathrectangle{\pgfqpoint{0.100000in}{0.212622in}}{\pgfqpoint{3.696000in}{3.696000in}}%
\pgfusepath{clip}%
\pgfsetbuttcap%
\pgfsetroundjoin%
\definecolor{currentfill}{rgb}{0.121569,0.466667,0.705882}%
\pgfsetfillcolor{currentfill}%
\pgfsetfillopacity{0.850602}%
\pgfsetlinewidth{1.003750pt}%
\definecolor{currentstroke}{rgb}{0.121569,0.466667,0.705882}%
\pgfsetstrokecolor{currentstroke}%
\pgfsetstrokeopacity{0.850602}%
\pgfsetdash{}{0pt}%
\pgfpathmoveto{\pgfqpoint{0.537228in}{2.580767in}}%
\pgfpathcurveto{\pgfqpoint{0.545465in}{2.580767in}}{\pgfqpoint{0.553365in}{2.584039in}}{\pgfqpoint{0.559189in}{2.589863in}}%
\pgfpathcurveto{\pgfqpoint{0.565012in}{2.595687in}}{\pgfqpoint{0.568285in}{2.603587in}}{\pgfqpoint{0.568285in}{2.611823in}}%
\pgfpathcurveto{\pgfqpoint{0.568285in}{2.620060in}}{\pgfqpoint{0.565012in}{2.627960in}}{\pgfqpoint{0.559189in}{2.633784in}}%
\pgfpathcurveto{\pgfqpoint{0.553365in}{2.639608in}}{\pgfqpoint{0.545465in}{2.642880in}}{\pgfqpoint{0.537228in}{2.642880in}}%
\pgfpathcurveto{\pgfqpoint{0.528992in}{2.642880in}}{\pgfqpoint{0.521092in}{2.639608in}}{\pgfqpoint{0.515268in}{2.633784in}}%
\pgfpathcurveto{\pgfqpoint{0.509444in}{2.627960in}}{\pgfqpoint{0.506172in}{2.620060in}}{\pgfqpoint{0.506172in}{2.611823in}}%
\pgfpathcurveto{\pgfqpoint{0.506172in}{2.603587in}}{\pgfqpoint{0.509444in}{2.595687in}}{\pgfqpoint{0.515268in}{2.589863in}}%
\pgfpathcurveto{\pgfqpoint{0.521092in}{2.584039in}}{\pgfqpoint{0.528992in}{2.580767in}}{\pgfqpoint{0.537228in}{2.580767in}}%
\pgfpathclose%
\pgfusepath{stroke,fill}%
\end{pgfscope}%
\begin{pgfscope}%
\pgfpathrectangle{\pgfqpoint{0.100000in}{0.212622in}}{\pgfqpoint{3.696000in}{3.696000in}}%
\pgfusepath{clip}%
\pgfsetbuttcap%
\pgfsetroundjoin%
\definecolor{currentfill}{rgb}{0.121569,0.466667,0.705882}%
\pgfsetfillcolor{currentfill}%
\pgfsetfillopacity{0.850866}%
\pgfsetlinewidth{1.003750pt}%
\definecolor{currentstroke}{rgb}{0.121569,0.466667,0.705882}%
\pgfsetstrokecolor{currentstroke}%
\pgfsetstrokeopacity{0.850866}%
\pgfsetdash{}{0pt}%
\pgfpathmoveto{\pgfqpoint{0.538029in}{2.579728in}}%
\pgfpathcurveto{\pgfqpoint{0.546265in}{2.579728in}}{\pgfqpoint{0.554165in}{2.583001in}}{\pgfqpoint{0.559989in}{2.588824in}}%
\pgfpathcurveto{\pgfqpoint{0.565813in}{2.594648in}}{\pgfqpoint{0.569085in}{2.602548in}}{\pgfqpoint{0.569085in}{2.610785in}}%
\pgfpathcurveto{\pgfqpoint{0.569085in}{2.619021in}}{\pgfqpoint{0.565813in}{2.626921in}}{\pgfqpoint{0.559989in}{2.632745in}}%
\pgfpathcurveto{\pgfqpoint{0.554165in}{2.638569in}}{\pgfqpoint{0.546265in}{2.641841in}}{\pgfqpoint{0.538029in}{2.641841in}}%
\pgfpathcurveto{\pgfqpoint{0.529793in}{2.641841in}}{\pgfqpoint{0.521892in}{2.638569in}}{\pgfqpoint{0.516069in}{2.632745in}}%
\pgfpathcurveto{\pgfqpoint{0.510245in}{2.626921in}}{\pgfqpoint{0.506972in}{2.619021in}}{\pgfqpoint{0.506972in}{2.610785in}}%
\pgfpathcurveto{\pgfqpoint{0.506972in}{2.602548in}}{\pgfqpoint{0.510245in}{2.594648in}}{\pgfqpoint{0.516069in}{2.588824in}}%
\pgfpathcurveto{\pgfqpoint{0.521892in}{2.583001in}}{\pgfqpoint{0.529793in}{2.579728in}}{\pgfqpoint{0.538029in}{2.579728in}}%
\pgfpathclose%
\pgfusepath{stroke,fill}%
\end{pgfscope}%
\begin{pgfscope}%
\pgfpathrectangle{\pgfqpoint{0.100000in}{0.212622in}}{\pgfqpoint{3.696000in}{3.696000in}}%
\pgfusepath{clip}%
\pgfsetbuttcap%
\pgfsetroundjoin%
\definecolor{currentfill}{rgb}{0.121569,0.466667,0.705882}%
\pgfsetfillcolor{currentfill}%
\pgfsetfillopacity{0.851186}%
\pgfsetlinewidth{1.003750pt}%
\definecolor{currentstroke}{rgb}{0.121569,0.466667,0.705882}%
\pgfsetstrokecolor{currentstroke}%
\pgfsetstrokeopacity{0.851186}%
\pgfsetdash{}{0pt}%
\pgfpathmoveto{\pgfqpoint{0.540494in}{2.578699in}}%
\pgfpathcurveto{\pgfqpoint{0.548730in}{2.578699in}}{\pgfqpoint{0.556630in}{2.581971in}}{\pgfqpoint{0.562454in}{2.587795in}}%
\pgfpathcurveto{\pgfqpoint{0.568278in}{2.593619in}}{\pgfqpoint{0.571551in}{2.601519in}}{\pgfqpoint{0.571551in}{2.609755in}}%
\pgfpathcurveto{\pgfqpoint{0.571551in}{2.617992in}}{\pgfqpoint{0.568278in}{2.625892in}}{\pgfqpoint{0.562454in}{2.631716in}}%
\pgfpathcurveto{\pgfqpoint{0.556630in}{2.637539in}}{\pgfqpoint{0.548730in}{2.640812in}}{\pgfqpoint{0.540494in}{2.640812in}}%
\pgfpathcurveto{\pgfqpoint{0.532258in}{2.640812in}}{\pgfqpoint{0.524358in}{2.637539in}}{\pgfqpoint{0.518534in}{2.631716in}}%
\pgfpathcurveto{\pgfqpoint{0.512710in}{2.625892in}}{\pgfqpoint{0.509438in}{2.617992in}}{\pgfqpoint{0.509438in}{2.609755in}}%
\pgfpathcurveto{\pgfqpoint{0.509438in}{2.601519in}}{\pgfqpoint{0.512710in}{2.593619in}}{\pgfqpoint{0.518534in}{2.587795in}}%
\pgfpathcurveto{\pgfqpoint{0.524358in}{2.581971in}}{\pgfqpoint{0.532258in}{2.578699in}}{\pgfqpoint{0.540494in}{2.578699in}}%
\pgfpathclose%
\pgfusepath{stroke,fill}%
\end{pgfscope}%
\begin{pgfscope}%
\pgfpathrectangle{\pgfqpoint{0.100000in}{0.212622in}}{\pgfqpoint{3.696000in}{3.696000in}}%
\pgfusepath{clip}%
\pgfsetbuttcap%
\pgfsetroundjoin%
\definecolor{currentfill}{rgb}{0.121569,0.466667,0.705882}%
\pgfsetfillcolor{currentfill}%
\pgfsetfillopacity{0.851224}%
\pgfsetlinewidth{1.003750pt}%
\definecolor{currentstroke}{rgb}{0.121569,0.466667,0.705882}%
\pgfsetstrokecolor{currentstroke}%
\pgfsetstrokeopacity{0.851224}%
\pgfsetdash{}{0pt}%
\pgfpathmoveto{\pgfqpoint{0.541777in}{2.577359in}}%
\pgfpathcurveto{\pgfqpoint{0.550013in}{2.577359in}}{\pgfqpoint{0.557913in}{2.580632in}}{\pgfqpoint{0.563737in}{2.586455in}}%
\pgfpathcurveto{\pgfqpoint{0.569561in}{2.592279in}}{\pgfqpoint{0.572834in}{2.600179in}}{\pgfqpoint{0.572834in}{2.608416in}}%
\pgfpathcurveto{\pgfqpoint{0.572834in}{2.616652in}}{\pgfqpoint{0.569561in}{2.624552in}}{\pgfqpoint{0.563737in}{2.630376in}}%
\pgfpathcurveto{\pgfqpoint{0.557913in}{2.636200in}}{\pgfqpoint{0.550013in}{2.639472in}}{\pgfqpoint{0.541777in}{2.639472in}}%
\pgfpathcurveto{\pgfqpoint{0.533541in}{2.639472in}}{\pgfqpoint{0.525641in}{2.636200in}}{\pgfqpoint{0.519817in}{2.630376in}}%
\pgfpathcurveto{\pgfqpoint{0.513993in}{2.624552in}}{\pgfqpoint{0.510721in}{2.616652in}}{\pgfqpoint{0.510721in}{2.608416in}}%
\pgfpathcurveto{\pgfqpoint{0.510721in}{2.600179in}}{\pgfqpoint{0.513993in}{2.592279in}}{\pgfqpoint{0.519817in}{2.586455in}}%
\pgfpathcurveto{\pgfqpoint{0.525641in}{2.580632in}}{\pgfqpoint{0.533541in}{2.577359in}}{\pgfqpoint{0.541777in}{2.577359in}}%
\pgfpathclose%
\pgfusepath{stroke,fill}%
\end{pgfscope}%
\begin{pgfscope}%
\pgfpathrectangle{\pgfqpoint{0.100000in}{0.212622in}}{\pgfqpoint{3.696000in}{3.696000in}}%
\pgfusepath{clip}%
\pgfsetbuttcap%
\pgfsetroundjoin%
\definecolor{currentfill}{rgb}{0.121569,0.466667,0.705882}%
\pgfsetfillcolor{currentfill}%
\pgfsetfillopacity{0.851291}%
\pgfsetlinewidth{1.003750pt}%
\definecolor{currentstroke}{rgb}{0.121569,0.466667,0.705882}%
\pgfsetstrokecolor{currentstroke}%
\pgfsetstrokeopacity{0.851291}%
\pgfsetdash{}{0pt}%
\pgfpathmoveto{\pgfqpoint{0.544102in}{2.574883in}}%
\pgfpathcurveto{\pgfqpoint{0.552339in}{2.574883in}}{\pgfqpoint{0.560239in}{2.578156in}}{\pgfqpoint{0.566063in}{2.583980in}}%
\pgfpathcurveto{\pgfqpoint{0.571887in}{2.589803in}}{\pgfqpoint{0.575159in}{2.597703in}}{\pgfqpoint{0.575159in}{2.605940in}}%
\pgfpathcurveto{\pgfqpoint{0.575159in}{2.614176in}}{\pgfqpoint{0.571887in}{2.622076in}}{\pgfqpoint{0.566063in}{2.627900in}}%
\pgfpathcurveto{\pgfqpoint{0.560239in}{2.633724in}}{\pgfqpoint{0.552339in}{2.636996in}}{\pgfqpoint{0.544102in}{2.636996in}}%
\pgfpathcurveto{\pgfqpoint{0.535866in}{2.636996in}}{\pgfqpoint{0.527966in}{2.633724in}}{\pgfqpoint{0.522142in}{2.627900in}}%
\pgfpathcurveto{\pgfqpoint{0.516318in}{2.622076in}}{\pgfqpoint{0.513046in}{2.614176in}}{\pgfqpoint{0.513046in}{2.605940in}}%
\pgfpathcurveto{\pgfqpoint{0.513046in}{2.597703in}}{\pgfqpoint{0.516318in}{2.589803in}}{\pgfqpoint{0.522142in}{2.583980in}}%
\pgfpathcurveto{\pgfqpoint{0.527966in}{2.578156in}}{\pgfqpoint{0.535866in}{2.574883in}}{\pgfqpoint{0.544102in}{2.574883in}}%
\pgfpathclose%
\pgfusepath{stroke,fill}%
\end{pgfscope}%
\begin{pgfscope}%
\pgfpathrectangle{\pgfqpoint{0.100000in}{0.212622in}}{\pgfqpoint{3.696000in}{3.696000in}}%
\pgfusepath{clip}%
\pgfsetbuttcap%
\pgfsetroundjoin%
\definecolor{currentfill}{rgb}{0.121569,0.466667,0.705882}%
\pgfsetfillcolor{currentfill}%
\pgfsetfillopacity{0.851602}%
\pgfsetlinewidth{1.003750pt}%
\definecolor{currentstroke}{rgb}{0.121569,0.466667,0.705882}%
\pgfsetstrokecolor{currentstroke}%
\pgfsetstrokeopacity{0.851602}%
\pgfsetdash{}{0pt}%
\pgfpathmoveto{\pgfqpoint{0.548364in}{2.571964in}}%
\pgfpathcurveto{\pgfqpoint{0.556600in}{2.571964in}}{\pgfqpoint{0.564500in}{2.575237in}}{\pgfqpoint{0.570324in}{2.581061in}}%
\pgfpathcurveto{\pgfqpoint{0.576148in}{2.586885in}}{\pgfqpoint{0.579420in}{2.594785in}}{\pgfqpoint{0.579420in}{2.603021in}}%
\pgfpathcurveto{\pgfqpoint{0.579420in}{2.611257in}}{\pgfqpoint{0.576148in}{2.619157in}}{\pgfqpoint{0.570324in}{2.624981in}}%
\pgfpathcurveto{\pgfqpoint{0.564500in}{2.630805in}}{\pgfqpoint{0.556600in}{2.634077in}}{\pgfqpoint{0.548364in}{2.634077in}}%
\pgfpathcurveto{\pgfqpoint{0.540127in}{2.634077in}}{\pgfqpoint{0.532227in}{2.630805in}}{\pgfqpoint{0.526403in}{2.624981in}}%
\pgfpathcurveto{\pgfqpoint{0.520579in}{2.619157in}}{\pgfqpoint{0.517307in}{2.611257in}}{\pgfqpoint{0.517307in}{2.603021in}}%
\pgfpathcurveto{\pgfqpoint{0.517307in}{2.594785in}}{\pgfqpoint{0.520579in}{2.586885in}}{\pgfqpoint{0.526403in}{2.581061in}}%
\pgfpathcurveto{\pgfqpoint{0.532227in}{2.575237in}}{\pgfqpoint{0.540127in}{2.571964in}}{\pgfqpoint{0.548364in}{2.571964in}}%
\pgfpathclose%
\pgfusepath{stroke,fill}%
\end{pgfscope}%
\begin{pgfscope}%
\pgfpathrectangle{\pgfqpoint{0.100000in}{0.212622in}}{\pgfqpoint{3.696000in}{3.696000in}}%
\pgfusepath{clip}%
\pgfsetbuttcap%
\pgfsetroundjoin%
\definecolor{currentfill}{rgb}{0.121569,0.466667,0.705882}%
\pgfsetfillcolor{currentfill}%
\pgfsetfillopacity{0.852035}%
\pgfsetlinewidth{1.003750pt}%
\definecolor{currentstroke}{rgb}{0.121569,0.466667,0.705882}%
\pgfsetstrokecolor{currentstroke}%
\pgfsetstrokeopacity{0.852035}%
\pgfsetdash{}{0pt}%
\pgfpathmoveto{\pgfqpoint{0.556439in}{2.566825in}}%
\pgfpathcurveto{\pgfqpoint{0.564675in}{2.566825in}}{\pgfqpoint{0.572575in}{2.570098in}}{\pgfqpoint{0.578399in}{2.575922in}}%
\pgfpathcurveto{\pgfqpoint{0.584223in}{2.581745in}}{\pgfqpoint{0.587495in}{2.589646in}}{\pgfqpoint{0.587495in}{2.597882in}}%
\pgfpathcurveto{\pgfqpoint{0.587495in}{2.606118in}}{\pgfqpoint{0.584223in}{2.614018in}}{\pgfqpoint{0.578399in}{2.619842in}}%
\pgfpathcurveto{\pgfqpoint{0.572575in}{2.625666in}}{\pgfqpoint{0.564675in}{2.628938in}}{\pgfqpoint{0.556439in}{2.628938in}}%
\pgfpathcurveto{\pgfqpoint{0.548203in}{2.628938in}}{\pgfqpoint{0.540303in}{2.625666in}}{\pgfqpoint{0.534479in}{2.619842in}}%
\pgfpathcurveto{\pgfqpoint{0.528655in}{2.614018in}}{\pgfqpoint{0.525382in}{2.606118in}}{\pgfqpoint{0.525382in}{2.597882in}}%
\pgfpathcurveto{\pgfqpoint{0.525382in}{2.589646in}}{\pgfqpoint{0.528655in}{2.581745in}}{\pgfqpoint{0.534479in}{2.575922in}}%
\pgfpathcurveto{\pgfqpoint{0.540303in}{2.570098in}}{\pgfqpoint{0.548203in}{2.566825in}}{\pgfqpoint{0.556439in}{2.566825in}}%
\pgfpathclose%
\pgfusepath{stroke,fill}%
\end{pgfscope}%
\begin{pgfscope}%
\pgfpathrectangle{\pgfqpoint{0.100000in}{0.212622in}}{\pgfqpoint{3.696000in}{3.696000in}}%
\pgfusepath{clip}%
\pgfsetbuttcap%
\pgfsetroundjoin%
\definecolor{currentfill}{rgb}{0.121569,0.466667,0.705882}%
\pgfsetfillcolor{currentfill}%
\pgfsetfillopacity{0.852932}%
\pgfsetlinewidth{1.003750pt}%
\definecolor{currentstroke}{rgb}{0.121569,0.466667,0.705882}%
\pgfsetstrokecolor{currentstroke}%
\pgfsetstrokeopacity{0.852932}%
\pgfsetdash{}{0pt}%
\pgfpathmoveto{\pgfqpoint{0.570128in}{2.554844in}}%
\pgfpathcurveto{\pgfqpoint{0.578365in}{2.554844in}}{\pgfqpoint{0.586265in}{2.558116in}}{\pgfqpoint{0.592089in}{2.563940in}}%
\pgfpathcurveto{\pgfqpoint{0.597913in}{2.569764in}}{\pgfqpoint{0.601185in}{2.577664in}}{\pgfqpoint{0.601185in}{2.585901in}}%
\pgfpathcurveto{\pgfqpoint{0.601185in}{2.594137in}}{\pgfqpoint{0.597913in}{2.602037in}}{\pgfqpoint{0.592089in}{2.607861in}}%
\pgfpathcurveto{\pgfqpoint{0.586265in}{2.613685in}}{\pgfqpoint{0.578365in}{2.616957in}}{\pgfqpoint{0.570128in}{2.616957in}}%
\pgfpathcurveto{\pgfqpoint{0.561892in}{2.616957in}}{\pgfqpoint{0.553992in}{2.613685in}}{\pgfqpoint{0.548168in}{2.607861in}}%
\pgfpathcurveto{\pgfqpoint{0.542344in}{2.602037in}}{\pgfqpoint{0.539072in}{2.594137in}}{\pgfqpoint{0.539072in}{2.585901in}}%
\pgfpathcurveto{\pgfqpoint{0.539072in}{2.577664in}}{\pgfqpoint{0.542344in}{2.569764in}}{\pgfqpoint{0.548168in}{2.563940in}}%
\pgfpathcurveto{\pgfqpoint{0.553992in}{2.558116in}}{\pgfqpoint{0.561892in}{2.554844in}}{\pgfqpoint{0.570128in}{2.554844in}}%
\pgfpathclose%
\pgfusepath{stroke,fill}%
\end{pgfscope}%
\begin{pgfscope}%
\pgfpathrectangle{\pgfqpoint{0.100000in}{0.212622in}}{\pgfqpoint{3.696000in}{3.696000in}}%
\pgfusepath{clip}%
\pgfsetbuttcap%
\pgfsetroundjoin%
\definecolor{currentfill}{rgb}{0.121569,0.466667,0.705882}%
\pgfsetfillcolor{currentfill}%
\pgfsetfillopacity{0.853962}%
\pgfsetlinewidth{1.003750pt}%
\definecolor{currentstroke}{rgb}{0.121569,0.466667,0.705882}%
\pgfsetstrokecolor{currentstroke}%
\pgfsetstrokeopacity{0.853962}%
\pgfsetdash{}{0pt}%
\pgfpathmoveto{\pgfqpoint{2.746048in}{1.727613in}}%
\pgfpathcurveto{\pgfqpoint{2.754285in}{1.727613in}}{\pgfqpoint{2.762185in}{1.730886in}}{\pgfqpoint{2.768009in}{1.736710in}}%
\pgfpathcurveto{\pgfqpoint{2.773832in}{1.742534in}}{\pgfqpoint{2.777105in}{1.750434in}}{\pgfqpoint{2.777105in}{1.758670in}}%
\pgfpathcurveto{\pgfqpoint{2.777105in}{1.766906in}}{\pgfqpoint{2.773832in}{1.774806in}}{\pgfqpoint{2.768009in}{1.780630in}}%
\pgfpathcurveto{\pgfqpoint{2.762185in}{1.786454in}}{\pgfqpoint{2.754285in}{1.789726in}}{\pgfqpoint{2.746048in}{1.789726in}}%
\pgfpathcurveto{\pgfqpoint{2.737812in}{1.789726in}}{\pgfqpoint{2.729912in}{1.786454in}}{\pgfqpoint{2.724088in}{1.780630in}}%
\pgfpathcurveto{\pgfqpoint{2.718264in}{1.774806in}}{\pgfqpoint{2.714992in}{1.766906in}}{\pgfqpoint{2.714992in}{1.758670in}}%
\pgfpathcurveto{\pgfqpoint{2.714992in}{1.750434in}}{\pgfqpoint{2.718264in}{1.742534in}}{\pgfqpoint{2.724088in}{1.736710in}}%
\pgfpathcurveto{\pgfqpoint{2.729912in}{1.730886in}}{\pgfqpoint{2.737812in}{1.727613in}}{\pgfqpoint{2.746048in}{1.727613in}}%
\pgfpathclose%
\pgfusepath{stroke,fill}%
\end{pgfscope}%
\begin{pgfscope}%
\pgfpathrectangle{\pgfqpoint{0.100000in}{0.212622in}}{\pgfqpoint{3.696000in}{3.696000in}}%
\pgfusepath{clip}%
\pgfsetbuttcap%
\pgfsetroundjoin%
\definecolor{currentfill}{rgb}{0.121569,0.466667,0.705882}%
\pgfsetfillcolor{currentfill}%
\pgfsetfillopacity{0.854322}%
\pgfsetlinewidth{1.003750pt}%
\definecolor{currentstroke}{rgb}{0.121569,0.466667,0.705882}%
\pgfsetstrokecolor{currentstroke}%
\pgfsetstrokeopacity{0.854322}%
\pgfsetdash{}{0pt}%
\pgfpathmoveto{\pgfqpoint{0.583688in}{2.548358in}}%
\pgfpathcurveto{\pgfqpoint{0.591924in}{2.548358in}}{\pgfqpoint{0.599824in}{2.551630in}}{\pgfqpoint{0.605648in}{2.557454in}}%
\pgfpathcurveto{\pgfqpoint{0.611472in}{2.563278in}}{\pgfqpoint{0.614744in}{2.571178in}}{\pgfqpoint{0.614744in}{2.579415in}}%
\pgfpathcurveto{\pgfqpoint{0.614744in}{2.587651in}}{\pgfqpoint{0.611472in}{2.595551in}}{\pgfqpoint{0.605648in}{2.601375in}}%
\pgfpathcurveto{\pgfqpoint{0.599824in}{2.607199in}}{\pgfqpoint{0.591924in}{2.610471in}}{\pgfqpoint{0.583688in}{2.610471in}}%
\pgfpathcurveto{\pgfqpoint{0.575451in}{2.610471in}}{\pgfqpoint{0.567551in}{2.607199in}}{\pgfqpoint{0.561727in}{2.601375in}}%
\pgfpathcurveto{\pgfqpoint{0.555903in}{2.595551in}}{\pgfqpoint{0.552631in}{2.587651in}}{\pgfqpoint{0.552631in}{2.579415in}}%
\pgfpathcurveto{\pgfqpoint{0.552631in}{2.571178in}}{\pgfqpoint{0.555903in}{2.563278in}}{\pgfqpoint{0.561727in}{2.557454in}}%
\pgfpathcurveto{\pgfqpoint{0.567551in}{2.551630in}}{\pgfqpoint{0.575451in}{2.548358in}}{\pgfqpoint{0.583688in}{2.548358in}}%
\pgfpathclose%
\pgfusepath{stroke,fill}%
\end{pgfscope}%
\begin{pgfscope}%
\pgfpathrectangle{\pgfqpoint{0.100000in}{0.212622in}}{\pgfqpoint{3.696000in}{3.696000in}}%
\pgfusepath{clip}%
\pgfsetbuttcap%
\pgfsetroundjoin%
\definecolor{currentfill}{rgb}{0.121569,0.466667,0.705882}%
\pgfsetfillcolor{currentfill}%
\pgfsetfillopacity{0.855454}%
\pgfsetlinewidth{1.003750pt}%
\definecolor{currentstroke}{rgb}{0.121569,0.466667,0.705882}%
\pgfsetstrokecolor{currentstroke}%
\pgfsetstrokeopacity{0.855454}%
\pgfsetdash{}{0pt}%
\pgfpathmoveto{\pgfqpoint{0.595128in}{2.541782in}}%
\pgfpathcurveto{\pgfqpoint{0.603364in}{2.541782in}}{\pgfqpoint{0.611264in}{2.545055in}}{\pgfqpoint{0.617088in}{2.550879in}}%
\pgfpathcurveto{\pgfqpoint{0.622912in}{2.556702in}}{\pgfqpoint{0.626185in}{2.564603in}}{\pgfqpoint{0.626185in}{2.572839in}}%
\pgfpathcurveto{\pgfqpoint{0.626185in}{2.581075in}}{\pgfqpoint{0.622912in}{2.588975in}}{\pgfqpoint{0.617088in}{2.594799in}}%
\pgfpathcurveto{\pgfqpoint{0.611264in}{2.600623in}}{\pgfqpoint{0.603364in}{2.603895in}}{\pgfqpoint{0.595128in}{2.603895in}}%
\pgfpathcurveto{\pgfqpoint{0.586892in}{2.603895in}}{\pgfqpoint{0.578992in}{2.600623in}}{\pgfqpoint{0.573168in}{2.594799in}}%
\pgfpathcurveto{\pgfqpoint{0.567344in}{2.588975in}}{\pgfqpoint{0.564072in}{2.581075in}}{\pgfqpoint{0.564072in}{2.572839in}}%
\pgfpathcurveto{\pgfqpoint{0.564072in}{2.564603in}}{\pgfqpoint{0.567344in}{2.556702in}}{\pgfqpoint{0.573168in}{2.550879in}}%
\pgfpathcurveto{\pgfqpoint{0.578992in}{2.545055in}}{\pgfqpoint{0.586892in}{2.541782in}}{\pgfqpoint{0.595128in}{2.541782in}}%
\pgfpathclose%
\pgfusepath{stroke,fill}%
\end{pgfscope}%
\begin{pgfscope}%
\pgfpathrectangle{\pgfqpoint{0.100000in}{0.212622in}}{\pgfqpoint{3.696000in}{3.696000in}}%
\pgfusepath{clip}%
\pgfsetbuttcap%
\pgfsetroundjoin%
\definecolor{currentfill}{rgb}{0.121569,0.466667,0.705882}%
\pgfsetfillcolor{currentfill}%
\pgfsetfillopacity{0.856387}%
\pgfsetlinewidth{1.003750pt}%
\definecolor{currentstroke}{rgb}{0.121569,0.466667,0.705882}%
\pgfsetstrokecolor{currentstroke}%
\pgfsetstrokeopacity{0.856387}%
\pgfsetdash{}{0pt}%
\pgfpathmoveto{\pgfqpoint{0.606292in}{2.534995in}}%
\pgfpathcurveto{\pgfqpoint{0.614528in}{2.534995in}}{\pgfqpoint{0.622428in}{2.538268in}}{\pgfqpoint{0.628252in}{2.544092in}}%
\pgfpathcurveto{\pgfqpoint{0.634076in}{2.549915in}}{\pgfqpoint{0.637348in}{2.557815in}}{\pgfqpoint{0.637348in}{2.566052in}}%
\pgfpathcurveto{\pgfqpoint{0.637348in}{2.574288in}}{\pgfqpoint{0.634076in}{2.582188in}}{\pgfqpoint{0.628252in}{2.588012in}}%
\pgfpathcurveto{\pgfqpoint{0.622428in}{2.593836in}}{\pgfqpoint{0.614528in}{2.597108in}}{\pgfqpoint{0.606292in}{2.597108in}}%
\pgfpathcurveto{\pgfqpoint{0.598055in}{2.597108in}}{\pgfqpoint{0.590155in}{2.593836in}}{\pgfqpoint{0.584331in}{2.588012in}}%
\pgfpathcurveto{\pgfqpoint{0.578507in}{2.582188in}}{\pgfqpoint{0.575235in}{2.574288in}}{\pgfqpoint{0.575235in}{2.566052in}}%
\pgfpathcurveto{\pgfqpoint{0.575235in}{2.557815in}}{\pgfqpoint{0.578507in}{2.549915in}}{\pgfqpoint{0.584331in}{2.544092in}}%
\pgfpathcurveto{\pgfqpoint{0.590155in}{2.538268in}}{\pgfqpoint{0.598055in}{2.534995in}}{\pgfqpoint{0.606292in}{2.534995in}}%
\pgfpathclose%
\pgfusepath{stroke,fill}%
\end{pgfscope}%
\begin{pgfscope}%
\pgfpathrectangle{\pgfqpoint{0.100000in}{0.212622in}}{\pgfqpoint{3.696000in}{3.696000in}}%
\pgfusepath{clip}%
\pgfsetbuttcap%
\pgfsetroundjoin%
\definecolor{currentfill}{rgb}{0.121569,0.466667,0.705882}%
\pgfsetfillcolor{currentfill}%
\pgfsetfillopacity{0.857394}%
\pgfsetlinewidth{1.003750pt}%
\definecolor{currentstroke}{rgb}{0.121569,0.466667,0.705882}%
\pgfsetstrokecolor{currentstroke}%
\pgfsetstrokeopacity{0.857394}%
\pgfsetdash{}{0pt}%
\pgfpathmoveto{\pgfqpoint{0.617228in}{2.529541in}}%
\pgfpathcurveto{\pgfqpoint{0.625464in}{2.529541in}}{\pgfqpoint{0.633364in}{2.532814in}}{\pgfqpoint{0.639188in}{2.538638in}}%
\pgfpathcurveto{\pgfqpoint{0.645012in}{2.544462in}}{\pgfqpoint{0.648284in}{2.552362in}}{\pgfqpoint{0.648284in}{2.560598in}}%
\pgfpathcurveto{\pgfqpoint{0.648284in}{2.568834in}}{\pgfqpoint{0.645012in}{2.576734in}}{\pgfqpoint{0.639188in}{2.582558in}}%
\pgfpathcurveto{\pgfqpoint{0.633364in}{2.588382in}}{\pgfqpoint{0.625464in}{2.591654in}}{\pgfqpoint{0.617228in}{2.591654in}}%
\pgfpathcurveto{\pgfqpoint{0.608992in}{2.591654in}}{\pgfqpoint{0.601092in}{2.588382in}}{\pgfqpoint{0.595268in}{2.582558in}}%
\pgfpathcurveto{\pgfqpoint{0.589444in}{2.576734in}}{\pgfqpoint{0.586171in}{2.568834in}}{\pgfqpoint{0.586171in}{2.560598in}}%
\pgfpathcurveto{\pgfqpoint{0.586171in}{2.552362in}}{\pgfqpoint{0.589444in}{2.544462in}}{\pgfqpoint{0.595268in}{2.538638in}}%
\pgfpathcurveto{\pgfqpoint{0.601092in}{2.532814in}}{\pgfqpoint{0.608992in}{2.529541in}}{\pgfqpoint{0.617228in}{2.529541in}}%
\pgfpathclose%
\pgfusepath{stroke,fill}%
\end{pgfscope}%
\begin{pgfscope}%
\pgfpathrectangle{\pgfqpoint{0.100000in}{0.212622in}}{\pgfqpoint{3.696000in}{3.696000in}}%
\pgfusepath{clip}%
\pgfsetbuttcap%
\pgfsetroundjoin%
\definecolor{currentfill}{rgb}{0.121569,0.466667,0.705882}%
\pgfsetfillcolor{currentfill}%
\pgfsetfillopacity{0.858030}%
\pgfsetlinewidth{1.003750pt}%
\definecolor{currentstroke}{rgb}{0.121569,0.466667,0.705882}%
\pgfsetstrokecolor{currentstroke}%
\pgfsetstrokeopacity{0.858030}%
\pgfsetdash{}{0pt}%
\pgfpathmoveto{\pgfqpoint{2.743734in}{1.721788in}}%
\pgfpathcurveto{\pgfqpoint{2.751971in}{1.721788in}}{\pgfqpoint{2.759871in}{1.725061in}}{\pgfqpoint{2.765695in}{1.730885in}}%
\pgfpathcurveto{\pgfqpoint{2.771518in}{1.736709in}}{\pgfqpoint{2.774791in}{1.744609in}}{\pgfqpoint{2.774791in}{1.752845in}}%
\pgfpathcurveto{\pgfqpoint{2.774791in}{1.761081in}}{\pgfqpoint{2.771518in}{1.768981in}}{\pgfqpoint{2.765695in}{1.774805in}}%
\pgfpathcurveto{\pgfqpoint{2.759871in}{1.780629in}}{\pgfqpoint{2.751971in}{1.783901in}}{\pgfqpoint{2.743734in}{1.783901in}}%
\pgfpathcurveto{\pgfqpoint{2.735498in}{1.783901in}}{\pgfqpoint{2.727598in}{1.780629in}}{\pgfqpoint{2.721774in}{1.774805in}}%
\pgfpathcurveto{\pgfqpoint{2.715950in}{1.768981in}}{\pgfqpoint{2.712678in}{1.761081in}}{\pgfqpoint{2.712678in}{1.752845in}}%
\pgfpathcurveto{\pgfqpoint{2.712678in}{1.744609in}}{\pgfqpoint{2.715950in}{1.736709in}}{\pgfqpoint{2.721774in}{1.730885in}}%
\pgfpathcurveto{\pgfqpoint{2.727598in}{1.725061in}}{\pgfqpoint{2.735498in}{1.721788in}}{\pgfqpoint{2.743734in}{1.721788in}}%
\pgfpathclose%
\pgfusepath{stroke,fill}%
\end{pgfscope}%
\begin{pgfscope}%
\pgfpathrectangle{\pgfqpoint{0.100000in}{0.212622in}}{\pgfqpoint{3.696000in}{3.696000in}}%
\pgfusepath{clip}%
\pgfsetbuttcap%
\pgfsetroundjoin%
\definecolor{currentfill}{rgb}{0.121569,0.466667,0.705882}%
\pgfsetfillcolor{currentfill}%
\pgfsetfillopacity{0.858251}%
\pgfsetlinewidth{1.003750pt}%
\definecolor{currentstroke}{rgb}{0.121569,0.466667,0.705882}%
\pgfsetstrokecolor{currentstroke}%
\pgfsetstrokeopacity{0.858251}%
\pgfsetdash{}{0pt}%
\pgfpathmoveto{\pgfqpoint{0.634139in}{2.516362in}}%
\pgfpathcurveto{\pgfqpoint{0.642375in}{2.516362in}}{\pgfqpoint{0.650275in}{2.519634in}}{\pgfqpoint{0.656099in}{2.525458in}}%
\pgfpathcurveto{\pgfqpoint{0.661923in}{2.531282in}}{\pgfqpoint{0.665196in}{2.539182in}}{\pgfqpoint{0.665196in}{2.547419in}}%
\pgfpathcurveto{\pgfqpoint{0.665196in}{2.555655in}}{\pgfqpoint{0.661923in}{2.563555in}}{\pgfqpoint{0.656099in}{2.569379in}}%
\pgfpathcurveto{\pgfqpoint{0.650275in}{2.575203in}}{\pgfqpoint{0.642375in}{2.578475in}}{\pgfqpoint{0.634139in}{2.578475in}}%
\pgfpathcurveto{\pgfqpoint{0.625903in}{2.578475in}}{\pgfqpoint{0.618003in}{2.575203in}}{\pgfqpoint{0.612179in}{2.569379in}}%
\pgfpathcurveto{\pgfqpoint{0.606355in}{2.563555in}}{\pgfqpoint{0.603083in}{2.555655in}}{\pgfqpoint{0.603083in}{2.547419in}}%
\pgfpathcurveto{\pgfqpoint{0.603083in}{2.539182in}}{\pgfqpoint{0.606355in}{2.531282in}}{\pgfqpoint{0.612179in}{2.525458in}}%
\pgfpathcurveto{\pgfqpoint{0.618003in}{2.519634in}}{\pgfqpoint{0.625903in}{2.516362in}}{\pgfqpoint{0.634139in}{2.516362in}}%
\pgfpathclose%
\pgfusepath{stroke,fill}%
\end{pgfscope}%
\begin{pgfscope}%
\pgfpathrectangle{\pgfqpoint{0.100000in}{0.212622in}}{\pgfqpoint{3.696000in}{3.696000in}}%
\pgfusepath{clip}%
\pgfsetbuttcap%
\pgfsetroundjoin%
\definecolor{currentfill}{rgb}{0.121569,0.466667,0.705882}%
\pgfsetfillcolor{currentfill}%
\pgfsetfillopacity{0.858253}%
\pgfsetlinewidth{1.003750pt}%
\definecolor{currentstroke}{rgb}{0.121569,0.466667,0.705882}%
\pgfsetstrokecolor{currentstroke}%
\pgfsetstrokeopacity{0.858253}%
\pgfsetdash{}{0pt}%
\pgfpathmoveto{\pgfqpoint{0.627158in}{2.526314in}}%
\pgfpathcurveto{\pgfqpoint{0.635394in}{2.526314in}}{\pgfqpoint{0.643294in}{2.529586in}}{\pgfqpoint{0.649118in}{2.535410in}}%
\pgfpathcurveto{\pgfqpoint{0.654942in}{2.541234in}}{\pgfqpoint{0.658214in}{2.549134in}}{\pgfqpoint{0.658214in}{2.557370in}}%
\pgfpathcurveto{\pgfqpoint{0.658214in}{2.565607in}}{\pgfqpoint{0.654942in}{2.573507in}}{\pgfqpoint{0.649118in}{2.579331in}}%
\pgfpathcurveto{\pgfqpoint{0.643294in}{2.585155in}}{\pgfqpoint{0.635394in}{2.588427in}}{\pgfqpoint{0.627158in}{2.588427in}}%
\pgfpathcurveto{\pgfqpoint{0.618922in}{2.588427in}}{\pgfqpoint{0.611022in}{2.585155in}}{\pgfqpoint{0.605198in}{2.579331in}}%
\pgfpathcurveto{\pgfqpoint{0.599374in}{2.573507in}}{\pgfqpoint{0.596101in}{2.565607in}}{\pgfqpoint{0.596101in}{2.557370in}}%
\pgfpathcurveto{\pgfqpoint{0.596101in}{2.549134in}}{\pgfqpoint{0.599374in}{2.541234in}}{\pgfqpoint{0.605198in}{2.535410in}}%
\pgfpathcurveto{\pgfqpoint{0.611022in}{2.529586in}}{\pgfqpoint{0.618922in}{2.526314in}}{\pgfqpoint{0.627158in}{2.526314in}}%
\pgfpathclose%
\pgfusepath{stroke,fill}%
\end{pgfscope}%
\begin{pgfscope}%
\pgfpathrectangle{\pgfqpoint{0.100000in}{0.212622in}}{\pgfqpoint{3.696000in}{3.696000in}}%
\pgfusepath{clip}%
\pgfsetbuttcap%
\pgfsetroundjoin%
\definecolor{currentfill}{rgb}{0.121569,0.466667,0.705882}%
\pgfsetfillcolor{currentfill}%
\pgfsetfillopacity{0.859014}%
\pgfsetlinewidth{1.003750pt}%
\definecolor{currentstroke}{rgb}{0.121569,0.466667,0.705882}%
\pgfsetstrokecolor{currentstroke}%
\pgfsetstrokeopacity{0.859014}%
\pgfsetdash{}{0pt}%
\pgfpathmoveto{\pgfqpoint{0.640325in}{2.515529in}}%
\pgfpathcurveto{\pgfqpoint{0.648561in}{2.515529in}}{\pgfqpoint{0.656461in}{2.518801in}}{\pgfqpoint{0.662285in}{2.524625in}}%
\pgfpathcurveto{\pgfqpoint{0.668109in}{2.530449in}}{\pgfqpoint{0.671381in}{2.538349in}}{\pgfqpoint{0.671381in}{2.546585in}}%
\pgfpathcurveto{\pgfqpoint{0.671381in}{2.554822in}}{\pgfqpoint{0.668109in}{2.562722in}}{\pgfqpoint{0.662285in}{2.568546in}}%
\pgfpathcurveto{\pgfqpoint{0.656461in}{2.574370in}}{\pgfqpoint{0.648561in}{2.577642in}}{\pgfqpoint{0.640325in}{2.577642in}}%
\pgfpathcurveto{\pgfqpoint{0.632088in}{2.577642in}}{\pgfqpoint{0.624188in}{2.574370in}}{\pgfqpoint{0.618364in}{2.568546in}}%
\pgfpathcurveto{\pgfqpoint{0.612541in}{2.562722in}}{\pgfqpoint{0.609268in}{2.554822in}}{\pgfqpoint{0.609268in}{2.546585in}}%
\pgfpathcurveto{\pgfqpoint{0.609268in}{2.538349in}}{\pgfqpoint{0.612541in}{2.530449in}}{\pgfqpoint{0.618364in}{2.524625in}}%
\pgfpathcurveto{\pgfqpoint{0.624188in}{2.518801in}}{\pgfqpoint{0.632088in}{2.515529in}}{\pgfqpoint{0.640325in}{2.515529in}}%
\pgfpathclose%
\pgfusepath{stroke,fill}%
\end{pgfscope}%
\begin{pgfscope}%
\pgfpathrectangle{\pgfqpoint{0.100000in}{0.212622in}}{\pgfqpoint{3.696000in}{3.696000in}}%
\pgfusepath{clip}%
\pgfsetbuttcap%
\pgfsetroundjoin%
\definecolor{currentfill}{rgb}{0.121569,0.466667,0.705882}%
\pgfsetfillcolor{currentfill}%
\pgfsetfillopacity{0.859951}%
\pgfsetlinewidth{1.003750pt}%
\definecolor{currentstroke}{rgb}{0.121569,0.466667,0.705882}%
\pgfsetstrokecolor{currentstroke}%
\pgfsetstrokeopacity{0.859951}%
\pgfsetdash{}{0pt}%
\pgfpathmoveto{\pgfqpoint{0.651548in}{2.510348in}}%
\pgfpathcurveto{\pgfqpoint{0.659785in}{2.510348in}}{\pgfqpoint{0.667685in}{2.513620in}}{\pgfqpoint{0.673509in}{2.519444in}}%
\pgfpathcurveto{\pgfqpoint{0.679333in}{2.525268in}}{\pgfqpoint{0.682605in}{2.533168in}}{\pgfqpoint{0.682605in}{2.541404in}}%
\pgfpathcurveto{\pgfqpoint{0.682605in}{2.549640in}}{\pgfqpoint{0.679333in}{2.557540in}}{\pgfqpoint{0.673509in}{2.563364in}}%
\pgfpathcurveto{\pgfqpoint{0.667685in}{2.569188in}}{\pgfqpoint{0.659785in}{2.572461in}}{\pgfqpoint{0.651548in}{2.572461in}}%
\pgfpathcurveto{\pgfqpoint{0.643312in}{2.572461in}}{\pgfqpoint{0.635412in}{2.569188in}}{\pgfqpoint{0.629588in}{2.563364in}}%
\pgfpathcurveto{\pgfqpoint{0.623764in}{2.557540in}}{\pgfqpoint{0.620492in}{2.549640in}}{\pgfqpoint{0.620492in}{2.541404in}}%
\pgfpathcurveto{\pgfqpoint{0.620492in}{2.533168in}}{\pgfqpoint{0.623764in}{2.525268in}}{\pgfqpoint{0.629588in}{2.519444in}}%
\pgfpathcurveto{\pgfqpoint{0.635412in}{2.513620in}}{\pgfqpoint{0.643312in}{2.510348in}}{\pgfqpoint{0.651548in}{2.510348in}}%
\pgfpathclose%
\pgfusepath{stroke,fill}%
\end{pgfscope}%
\begin{pgfscope}%
\pgfpathrectangle{\pgfqpoint{0.100000in}{0.212622in}}{\pgfqpoint{3.696000in}{3.696000in}}%
\pgfusepath{clip}%
\pgfsetbuttcap%
\pgfsetroundjoin%
\definecolor{currentfill}{rgb}{0.121569,0.466667,0.705882}%
\pgfsetfillcolor{currentfill}%
\pgfsetfillopacity{0.860425}%
\pgfsetlinewidth{1.003750pt}%
\definecolor{currentstroke}{rgb}{0.121569,0.466667,0.705882}%
\pgfsetstrokecolor{currentstroke}%
\pgfsetstrokeopacity{0.860425}%
\pgfsetdash{}{0pt}%
\pgfpathmoveto{\pgfqpoint{0.660475in}{2.501785in}}%
\pgfpathcurveto{\pgfqpoint{0.668711in}{2.501785in}}{\pgfqpoint{0.676611in}{2.505057in}}{\pgfqpoint{0.682435in}{2.510881in}}%
\pgfpathcurveto{\pgfqpoint{0.688259in}{2.516705in}}{\pgfqpoint{0.691531in}{2.524605in}}{\pgfqpoint{0.691531in}{2.532842in}}%
\pgfpathcurveto{\pgfqpoint{0.691531in}{2.541078in}}{\pgfqpoint{0.688259in}{2.548978in}}{\pgfqpoint{0.682435in}{2.554802in}}%
\pgfpathcurveto{\pgfqpoint{0.676611in}{2.560626in}}{\pgfqpoint{0.668711in}{2.563898in}}{\pgfqpoint{0.660475in}{2.563898in}}%
\pgfpathcurveto{\pgfqpoint{0.652238in}{2.563898in}}{\pgfqpoint{0.644338in}{2.560626in}}{\pgfqpoint{0.638514in}{2.554802in}}%
\pgfpathcurveto{\pgfqpoint{0.632691in}{2.548978in}}{\pgfqpoint{0.629418in}{2.541078in}}{\pgfqpoint{0.629418in}{2.532842in}}%
\pgfpathcurveto{\pgfqpoint{0.629418in}{2.524605in}}{\pgfqpoint{0.632691in}{2.516705in}}{\pgfqpoint{0.638514in}{2.510881in}}%
\pgfpathcurveto{\pgfqpoint{0.644338in}{2.505057in}}{\pgfqpoint{0.652238in}{2.501785in}}{\pgfqpoint{0.660475in}{2.501785in}}%
\pgfpathclose%
\pgfusepath{stroke,fill}%
\end{pgfscope}%
\begin{pgfscope}%
\pgfpathrectangle{\pgfqpoint{0.100000in}{0.212622in}}{\pgfqpoint{3.696000in}{3.696000in}}%
\pgfusepath{clip}%
\pgfsetbuttcap%
\pgfsetroundjoin%
\definecolor{currentfill}{rgb}{0.121569,0.466667,0.705882}%
\pgfsetfillcolor{currentfill}%
\pgfsetfillopacity{0.860522}%
\pgfsetlinewidth{1.003750pt}%
\definecolor{currentstroke}{rgb}{0.121569,0.466667,0.705882}%
\pgfsetstrokecolor{currentstroke}%
\pgfsetstrokeopacity{0.860522}%
\pgfsetdash{}{0pt}%
\pgfpathmoveto{\pgfqpoint{2.739939in}{1.704299in}}%
\pgfpathcurveto{\pgfqpoint{2.748175in}{1.704299in}}{\pgfqpoint{2.756075in}{1.707572in}}{\pgfqpoint{2.761899in}{1.713396in}}%
\pgfpathcurveto{\pgfqpoint{2.767723in}{1.719220in}}{\pgfqpoint{2.770995in}{1.727120in}}{\pgfqpoint{2.770995in}{1.735356in}}%
\pgfpathcurveto{\pgfqpoint{2.770995in}{1.743592in}}{\pgfqpoint{2.767723in}{1.751492in}}{\pgfqpoint{2.761899in}{1.757316in}}%
\pgfpathcurveto{\pgfqpoint{2.756075in}{1.763140in}}{\pgfqpoint{2.748175in}{1.766412in}}{\pgfqpoint{2.739939in}{1.766412in}}%
\pgfpathcurveto{\pgfqpoint{2.731702in}{1.766412in}}{\pgfqpoint{2.723802in}{1.763140in}}{\pgfqpoint{2.717978in}{1.757316in}}%
\pgfpathcurveto{\pgfqpoint{2.712154in}{1.751492in}}{\pgfqpoint{2.708882in}{1.743592in}}{\pgfqpoint{2.708882in}{1.735356in}}%
\pgfpathcurveto{\pgfqpoint{2.708882in}{1.727120in}}{\pgfqpoint{2.712154in}{1.719220in}}{\pgfqpoint{2.717978in}{1.713396in}}%
\pgfpathcurveto{\pgfqpoint{2.723802in}{1.707572in}}{\pgfqpoint{2.731702in}{1.704299in}}{\pgfqpoint{2.739939in}{1.704299in}}%
\pgfpathclose%
\pgfusepath{stroke,fill}%
\end{pgfscope}%
\begin{pgfscope}%
\pgfpathrectangle{\pgfqpoint{0.100000in}{0.212622in}}{\pgfqpoint{3.696000in}{3.696000in}}%
\pgfusepath{clip}%
\pgfsetbuttcap%
\pgfsetroundjoin%
\definecolor{currentfill}{rgb}{0.121569,0.466667,0.705882}%
\pgfsetfillcolor{currentfill}%
\pgfsetfillopacity{0.861107}%
\pgfsetlinewidth{1.003750pt}%
\definecolor{currentstroke}{rgb}{0.121569,0.466667,0.705882}%
\pgfsetstrokecolor{currentstroke}%
\pgfsetstrokeopacity{0.861107}%
\pgfsetdash{}{0pt}%
\pgfpathmoveto{\pgfqpoint{0.667208in}{2.497607in}}%
\pgfpathcurveto{\pgfqpoint{0.675444in}{2.497607in}}{\pgfqpoint{0.683344in}{2.500879in}}{\pgfqpoint{0.689168in}{2.506703in}}%
\pgfpathcurveto{\pgfqpoint{0.694992in}{2.512527in}}{\pgfqpoint{0.698265in}{2.520427in}}{\pgfqpoint{0.698265in}{2.528663in}}%
\pgfpathcurveto{\pgfqpoint{0.698265in}{2.536900in}}{\pgfqpoint{0.694992in}{2.544800in}}{\pgfqpoint{0.689168in}{2.550624in}}%
\pgfpathcurveto{\pgfqpoint{0.683344in}{2.556448in}}{\pgfqpoint{0.675444in}{2.559720in}}{\pgfqpoint{0.667208in}{2.559720in}}%
\pgfpathcurveto{\pgfqpoint{0.658972in}{2.559720in}}{\pgfqpoint{0.651072in}{2.556448in}}{\pgfqpoint{0.645248in}{2.550624in}}%
\pgfpathcurveto{\pgfqpoint{0.639424in}{2.544800in}}{\pgfqpoint{0.636152in}{2.536900in}}{\pgfqpoint{0.636152in}{2.528663in}}%
\pgfpathcurveto{\pgfqpoint{0.636152in}{2.520427in}}{\pgfqpoint{0.639424in}{2.512527in}}{\pgfqpoint{0.645248in}{2.506703in}}%
\pgfpathcurveto{\pgfqpoint{0.651072in}{2.500879in}}{\pgfqpoint{0.658972in}{2.497607in}}{\pgfqpoint{0.667208in}{2.497607in}}%
\pgfpathclose%
\pgfusepath{stroke,fill}%
\end{pgfscope}%
\begin{pgfscope}%
\pgfpathrectangle{\pgfqpoint{0.100000in}{0.212622in}}{\pgfqpoint{3.696000in}{3.696000in}}%
\pgfusepath{clip}%
\pgfsetbuttcap%
\pgfsetroundjoin%
\definecolor{currentfill}{rgb}{0.121569,0.466667,0.705882}%
\pgfsetfillcolor{currentfill}%
\pgfsetfillopacity{0.862623}%
\pgfsetlinewidth{1.003750pt}%
\definecolor{currentstroke}{rgb}{0.121569,0.466667,0.705882}%
\pgfsetstrokecolor{currentstroke}%
\pgfsetstrokeopacity{0.862623}%
\pgfsetdash{}{0pt}%
\pgfpathmoveto{\pgfqpoint{0.680267in}{2.495052in}}%
\pgfpathcurveto{\pgfqpoint{0.688504in}{2.495052in}}{\pgfqpoint{0.696404in}{2.498324in}}{\pgfqpoint{0.702228in}{2.504148in}}%
\pgfpathcurveto{\pgfqpoint{0.708051in}{2.509972in}}{\pgfqpoint{0.711324in}{2.517872in}}{\pgfqpoint{0.711324in}{2.526109in}}%
\pgfpathcurveto{\pgfqpoint{0.711324in}{2.534345in}}{\pgfqpoint{0.708051in}{2.542245in}}{\pgfqpoint{0.702228in}{2.548069in}}%
\pgfpathcurveto{\pgfqpoint{0.696404in}{2.553893in}}{\pgfqpoint{0.688504in}{2.557165in}}{\pgfqpoint{0.680267in}{2.557165in}}%
\pgfpathcurveto{\pgfqpoint{0.672031in}{2.557165in}}{\pgfqpoint{0.664131in}{2.553893in}}{\pgfqpoint{0.658307in}{2.548069in}}%
\pgfpathcurveto{\pgfqpoint{0.652483in}{2.542245in}}{\pgfqpoint{0.649211in}{2.534345in}}{\pgfqpoint{0.649211in}{2.526109in}}%
\pgfpathcurveto{\pgfqpoint{0.649211in}{2.517872in}}{\pgfqpoint{0.652483in}{2.509972in}}{\pgfqpoint{0.658307in}{2.504148in}}%
\pgfpathcurveto{\pgfqpoint{0.664131in}{2.498324in}}{\pgfqpoint{0.672031in}{2.495052in}}{\pgfqpoint{0.680267in}{2.495052in}}%
\pgfpathclose%
\pgfusepath{stroke,fill}%
\end{pgfscope}%
\begin{pgfscope}%
\pgfpathrectangle{\pgfqpoint{0.100000in}{0.212622in}}{\pgfqpoint{3.696000in}{3.696000in}}%
\pgfusepath{clip}%
\pgfsetbuttcap%
\pgfsetroundjoin%
\definecolor{currentfill}{rgb}{0.121569,0.466667,0.705882}%
\pgfsetfillcolor{currentfill}%
\pgfsetfillopacity{0.862756}%
\pgfsetlinewidth{1.003750pt}%
\definecolor{currentstroke}{rgb}{0.121569,0.466667,0.705882}%
\pgfsetstrokecolor{currentstroke}%
\pgfsetstrokeopacity{0.862756}%
\pgfsetdash{}{0pt}%
\pgfpathmoveto{\pgfqpoint{0.690997in}{2.483221in}}%
\pgfpathcurveto{\pgfqpoint{0.699234in}{2.483221in}}{\pgfqpoint{0.707134in}{2.486493in}}{\pgfqpoint{0.712958in}{2.492317in}}%
\pgfpathcurveto{\pgfqpoint{0.718781in}{2.498141in}}{\pgfqpoint{0.722054in}{2.506041in}}{\pgfqpoint{0.722054in}{2.514277in}}%
\pgfpathcurveto{\pgfqpoint{0.722054in}{2.522513in}}{\pgfqpoint{0.718781in}{2.530413in}}{\pgfqpoint{0.712958in}{2.536237in}}%
\pgfpathcurveto{\pgfqpoint{0.707134in}{2.542061in}}{\pgfqpoint{0.699234in}{2.545334in}}{\pgfqpoint{0.690997in}{2.545334in}}%
\pgfpathcurveto{\pgfqpoint{0.682761in}{2.545334in}}{\pgfqpoint{0.674861in}{2.542061in}}{\pgfqpoint{0.669037in}{2.536237in}}%
\pgfpathcurveto{\pgfqpoint{0.663213in}{2.530413in}}{\pgfqpoint{0.659941in}{2.522513in}}{\pgfqpoint{0.659941in}{2.514277in}}%
\pgfpathcurveto{\pgfqpoint{0.659941in}{2.506041in}}{\pgfqpoint{0.663213in}{2.498141in}}{\pgfqpoint{0.669037in}{2.492317in}}%
\pgfpathcurveto{\pgfqpoint{0.674861in}{2.486493in}}{\pgfqpoint{0.682761in}{2.483221in}}{\pgfqpoint{0.690997in}{2.483221in}}%
\pgfpathclose%
\pgfusepath{stroke,fill}%
\end{pgfscope}%
\begin{pgfscope}%
\pgfpathrectangle{\pgfqpoint{0.100000in}{0.212622in}}{\pgfqpoint{3.696000in}{3.696000in}}%
\pgfusepath{clip}%
\pgfsetbuttcap%
\pgfsetroundjoin%
\definecolor{currentfill}{rgb}{0.121569,0.466667,0.705882}%
\pgfsetfillcolor{currentfill}%
\pgfsetfillopacity{0.863569}%
\pgfsetlinewidth{1.003750pt}%
\definecolor{currentstroke}{rgb}{0.121569,0.466667,0.705882}%
\pgfsetstrokecolor{currentstroke}%
\pgfsetstrokeopacity{0.863569}%
\pgfsetdash{}{0pt}%
\pgfpathmoveto{\pgfqpoint{0.700260in}{2.480946in}}%
\pgfpathcurveto{\pgfqpoint{0.708496in}{2.480946in}}{\pgfqpoint{0.716396in}{2.484218in}}{\pgfqpoint{0.722220in}{2.490042in}}%
\pgfpathcurveto{\pgfqpoint{0.728044in}{2.495866in}}{\pgfqpoint{0.731316in}{2.503766in}}{\pgfqpoint{0.731316in}{2.512002in}}%
\pgfpathcurveto{\pgfqpoint{0.731316in}{2.520239in}}{\pgfqpoint{0.728044in}{2.528139in}}{\pgfqpoint{0.722220in}{2.533963in}}%
\pgfpathcurveto{\pgfqpoint{0.716396in}{2.539786in}}{\pgfqpoint{0.708496in}{2.543059in}}{\pgfqpoint{0.700260in}{2.543059in}}%
\pgfpathcurveto{\pgfqpoint{0.692024in}{2.543059in}}{\pgfqpoint{0.684123in}{2.539786in}}{\pgfqpoint{0.678300in}{2.533963in}}%
\pgfpathcurveto{\pgfqpoint{0.672476in}{2.528139in}}{\pgfqpoint{0.669203in}{2.520239in}}{\pgfqpoint{0.669203in}{2.512002in}}%
\pgfpathcurveto{\pgfqpoint{0.669203in}{2.503766in}}{\pgfqpoint{0.672476in}{2.495866in}}{\pgfqpoint{0.678300in}{2.490042in}}%
\pgfpathcurveto{\pgfqpoint{0.684123in}{2.484218in}}{\pgfqpoint{0.692024in}{2.480946in}}{\pgfqpoint{0.700260in}{2.480946in}}%
\pgfpathclose%
\pgfusepath{stroke,fill}%
\end{pgfscope}%
\begin{pgfscope}%
\pgfpathrectangle{\pgfqpoint{0.100000in}{0.212622in}}{\pgfqpoint{3.696000in}{3.696000in}}%
\pgfusepath{clip}%
\pgfsetbuttcap%
\pgfsetroundjoin%
\definecolor{currentfill}{rgb}{0.121569,0.466667,0.705882}%
\pgfsetfillcolor{currentfill}%
\pgfsetfillopacity{0.864134}%
\pgfsetlinewidth{1.003750pt}%
\definecolor{currentstroke}{rgb}{0.121569,0.466667,0.705882}%
\pgfsetstrokecolor{currentstroke}%
\pgfsetstrokeopacity{0.864134}%
\pgfsetdash{}{0pt}%
\pgfpathmoveto{\pgfqpoint{0.708769in}{2.475702in}}%
\pgfpathcurveto{\pgfqpoint{0.717005in}{2.475702in}}{\pgfqpoint{0.724905in}{2.478974in}}{\pgfqpoint{0.730729in}{2.484798in}}%
\pgfpathcurveto{\pgfqpoint{0.736553in}{2.490622in}}{\pgfqpoint{0.739825in}{2.498522in}}{\pgfqpoint{0.739825in}{2.506759in}}%
\pgfpathcurveto{\pgfqpoint{0.739825in}{2.514995in}}{\pgfqpoint{0.736553in}{2.522895in}}{\pgfqpoint{0.730729in}{2.528719in}}%
\pgfpathcurveto{\pgfqpoint{0.724905in}{2.534543in}}{\pgfqpoint{0.717005in}{2.537815in}}{\pgfqpoint{0.708769in}{2.537815in}}%
\pgfpathcurveto{\pgfqpoint{0.700532in}{2.537815in}}{\pgfqpoint{0.692632in}{2.534543in}}{\pgfqpoint{0.686808in}{2.528719in}}%
\pgfpathcurveto{\pgfqpoint{0.680984in}{2.522895in}}{\pgfqpoint{0.677712in}{2.514995in}}{\pgfqpoint{0.677712in}{2.506759in}}%
\pgfpathcurveto{\pgfqpoint{0.677712in}{2.498522in}}{\pgfqpoint{0.680984in}{2.490622in}}{\pgfqpoint{0.686808in}{2.484798in}}%
\pgfpathcurveto{\pgfqpoint{0.692632in}{2.478974in}}{\pgfqpoint{0.700532in}{2.475702in}}{\pgfqpoint{0.708769in}{2.475702in}}%
\pgfpathclose%
\pgfusepath{stroke,fill}%
\end{pgfscope}%
\begin{pgfscope}%
\pgfpathrectangle{\pgfqpoint{0.100000in}{0.212622in}}{\pgfqpoint{3.696000in}{3.696000in}}%
\pgfusepath{clip}%
\pgfsetbuttcap%
\pgfsetroundjoin%
\definecolor{currentfill}{rgb}{0.121569,0.466667,0.705882}%
\pgfsetfillcolor{currentfill}%
\pgfsetfillopacity{0.865674}%
\pgfsetlinewidth{1.003750pt}%
\definecolor{currentstroke}{rgb}{0.121569,0.466667,0.705882}%
\pgfsetstrokecolor{currentstroke}%
\pgfsetstrokeopacity{0.865674}%
\pgfsetdash{}{0pt}%
\pgfpathmoveto{\pgfqpoint{0.723117in}{2.466294in}}%
\pgfpathcurveto{\pgfqpoint{0.731353in}{2.466294in}}{\pgfqpoint{0.739253in}{2.469566in}}{\pgfqpoint{0.745077in}{2.475390in}}%
\pgfpathcurveto{\pgfqpoint{0.750901in}{2.481214in}}{\pgfqpoint{0.754173in}{2.489114in}}{\pgfqpoint{0.754173in}{2.497350in}}%
\pgfpathcurveto{\pgfqpoint{0.754173in}{2.505586in}}{\pgfqpoint{0.750901in}{2.513486in}}{\pgfqpoint{0.745077in}{2.519310in}}%
\pgfpathcurveto{\pgfqpoint{0.739253in}{2.525134in}}{\pgfqpoint{0.731353in}{2.528407in}}{\pgfqpoint{0.723117in}{2.528407in}}%
\pgfpathcurveto{\pgfqpoint{0.714880in}{2.528407in}}{\pgfqpoint{0.706980in}{2.525134in}}{\pgfqpoint{0.701156in}{2.519310in}}%
\pgfpathcurveto{\pgfqpoint{0.695332in}{2.513486in}}{\pgfqpoint{0.692060in}{2.505586in}}{\pgfqpoint{0.692060in}{2.497350in}}%
\pgfpathcurveto{\pgfqpoint{0.692060in}{2.489114in}}{\pgfqpoint{0.695332in}{2.481214in}}{\pgfqpoint{0.701156in}{2.475390in}}%
\pgfpathcurveto{\pgfqpoint{0.706980in}{2.469566in}}{\pgfqpoint{0.714880in}{2.466294in}}{\pgfqpoint{0.723117in}{2.466294in}}%
\pgfpathclose%
\pgfusepath{stroke,fill}%
\end{pgfscope}%
\begin{pgfscope}%
\pgfpathrectangle{\pgfqpoint{0.100000in}{0.212622in}}{\pgfqpoint{3.696000in}{3.696000in}}%
\pgfusepath{clip}%
\pgfsetbuttcap%
\pgfsetroundjoin%
\definecolor{currentfill}{rgb}{0.121569,0.466667,0.705882}%
\pgfsetfillcolor{currentfill}%
\pgfsetfillopacity{0.866902}%
\pgfsetlinewidth{1.003750pt}%
\definecolor{currentstroke}{rgb}{0.121569,0.466667,0.705882}%
\pgfsetstrokecolor{currentstroke}%
\pgfsetstrokeopacity{0.866902}%
\pgfsetdash{}{0pt}%
\pgfpathmoveto{\pgfqpoint{2.727394in}{1.716076in}}%
\pgfpathcurveto{\pgfqpoint{2.735630in}{1.716076in}}{\pgfqpoint{2.743530in}{1.719348in}}{\pgfqpoint{2.749354in}{1.725172in}}%
\pgfpathcurveto{\pgfqpoint{2.755178in}{1.730996in}}{\pgfqpoint{2.758450in}{1.738896in}}{\pgfqpoint{2.758450in}{1.747132in}}%
\pgfpathcurveto{\pgfqpoint{2.758450in}{1.755369in}}{\pgfqpoint{2.755178in}{1.763269in}}{\pgfqpoint{2.749354in}{1.769093in}}%
\pgfpathcurveto{\pgfqpoint{2.743530in}{1.774916in}}{\pgfqpoint{2.735630in}{1.778189in}}{\pgfqpoint{2.727394in}{1.778189in}}%
\pgfpathcurveto{\pgfqpoint{2.719158in}{1.778189in}}{\pgfqpoint{2.711258in}{1.774916in}}{\pgfqpoint{2.705434in}{1.769093in}}%
\pgfpathcurveto{\pgfqpoint{2.699610in}{1.763269in}}{\pgfqpoint{2.696337in}{1.755369in}}{\pgfqpoint{2.696337in}{1.747132in}}%
\pgfpathcurveto{\pgfqpoint{2.696337in}{1.738896in}}{\pgfqpoint{2.699610in}{1.730996in}}{\pgfqpoint{2.705434in}{1.725172in}}%
\pgfpathcurveto{\pgfqpoint{2.711258in}{1.719348in}}{\pgfqpoint{2.719158in}{1.716076in}}{\pgfqpoint{2.727394in}{1.716076in}}%
\pgfpathclose%
\pgfusepath{stroke,fill}%
\end{pgfscope}%
\begin{pgfscope}%
\pgfpathrectangle{\pgfqpoint{0.100000in}{0.212622in}}{\pgfqpoint{3.696000in}{3.696000in}}%
\pgfusepath{clip}%
\pgfsetbuttcap%
\pgfsetroundjoin%
\definecolor{currentfill}{rgb}{0.121569,0.466667,0.705882}%
\pgfsetfillcolor{currentfill}%
\pgfsetfillopacity{0.867722}%
\pgfsetlinewidth{1.003750pt}%
\definecolor{currentstroke}{rgb}{0.121569,0.466667,0.705882}%
\pgfsetstrokecolor{currentstroke}%
\pgfsetstrokeopacity{0.867722}%
\pgfsetdash{}{0pt}%
\pgfpathmoveto{\pgfqpoint{0.737534in}{2.462708in}}%
\pgfpathcurveto{\pgfqpoint{0.745770in}{2.462708in}}{\pgfqpoint{0.753670in}{2.465980in}}{\pgfqpoint{0.759494in}{2.471804in}}%
\pgfpathcurveto{\pgfqpoint{0.765318in}{2.477628in}}{\pgfqpoint{0.768590in}{2.485528in}}{\pgfqpoint{0.768590in}{2.493764in}}%
\pgfpathcurveto{\pgfqpoint{0.768590in}{2.502000in}}{\pgfqpoint{0.765318in}{2.509900in}}{\pgfqpoint{0.759494in}{2.515724in}}%
\pgfpathcurveto{\pgfqpoint{0.753670in}{2.521548in}}{\pgfqpoint{0.745770in}{2.524821in}}{\pgfqpoint{0.737534in}{2.524821in}}%
\pgfpathcurveto{\pgfqpoint{0.729297in}{2.524821in}}{\pgfqpoint{0.721397in}{2.521548in}}{\pgfqpoint{0.715573in}{2.515724in}}%
\pgfpathcurveto{\pgfqpoint{0.709749in}{2.509900in}}{\pgfqpoint{0.706477in}{2.502000in}}{\pgfqpoint{0.706477in}{2.493764in}}%
\pgfpathcurveto{\pgfqpoint{0.706477in}{2.485528in}}{\pgfqpoint{0.709749in}{2.477628in}}{\pgfqpoint{0.715573in}{2.471804in}}%
\pgfpathcurveto{\pgfqpoint{0.721397in}{2.465980in}}{\pgfqpoint{0.729297in}{2.462708in}}{\pgfqpoint{0.737534in}{2.462708in}}%
\pgfpathclose%
\pgfusepath{stroke,fill}%
\end{pgfscope}%
\begin{pgfscope}%
\pgfpathrectangle{\pgfqpoint{0.100000in}{0.212622in}}{\pgfqpoint{3.696000in}{3.696000in}}%
\pgfusepath{clip}%
\pgfsetbuttcap%
\pgfsetroundjoin%
\definecolor{currentfill}{rgb}{0.121569,0.466667,0.705882}%
\pgfsetfillcolor{currentfill}%
\pgfsetfillopacity{0.869373}%
\pgfsetlinewidth{1.003750pt}%
\definecolor{currentstroke}{rgb}{0.121569,0.466667,0.705882}%
\pgfsetstrokecolor{currentstroke}%
\pgfsetstrokeopacity{0.869373}%
\pgfsetdash{}{0pt}%
\pgfpathmoveto{\pgfqpoint{0.751938in}{2.457519in}}%
\pgfpathcurveto{\pgfqpoint{0.760174in}{2.457519in}}{\pgfqpoint{0.768074in}{2.460792in}}{\pgfqpoint{0.773898in}{2.466616in}}%
\pgfpathcurveto{\pgfqpoint{0.779722in}{2.472439in}}{\pgfqpoint{0.782995in}{2.480340in}}{\pgfqpoint{0.782995in}{2.488576in}}%
\pgfpathcurveto{\pgfqpoint{0.782995in}{2.496812in}}{\pgfqpoint{0.779722in}{2.504712in}}{\pgfqpoint{0.773898in}{2.510536in}}%
\pgfpathcurveto{\pgfqpoint{0.768074in}{2.516360in}}{\pgfqpoint{0.760174in}{2.519632in}}{\pgfqpoint{0.751938in}{2.519632in}}%
\pgfpathcurveto{\pgfqpoint{0.743702in}{2.519632in}}{\pgfqpoint{0.735802in}{2.516360in}}{\pgfqpoint{0.729978in}{2.510536in}}%
\pgfpathcurveto{\pgfqpoint{0.724154in}{2.504712in}}{\pgfqpoint{0.720882in}{2.496812in}}{\pgfqpoint{0.720882in}{2.488576in}}%
\pgfpathcurveto{\pgfqpoint{0.720882in}{2.480340in}}{\pgfqpoint{0.724154in}{2.472439in}}{\pgfqpoint{0.729978in}{2.466616in}}%
\pgfpathcurveto{\pgfqpoint{0.735802in}{2.460792in}}{\pgfqpoint{0.743702in}{2.457519in}}{\pgfqpoint{0.751938in}{2.457519in}}%
\pgfpathclose%
\pgfusepath{stroke,fill}%
\end{pgfscope}%
\begin{pgfscope}%
\pgfpathrectangle{\pgfqpoint{0.100000in}{0.212622in}}{\pgfqpoint{3.696000in}{3.696000in}}%
\pgfusepath{clip}%
\pgfsetbuttcap%
\pgfsetroundjoin%
\definecolor{currentfill}{rgb}{0.121569,0.466667,0.705882}%
\pgfsetfillcolor{currentfill}%
\pgfsetfillopacity{0.870636}%
\pgfsetlinewidth{1.003750pt}%
\definecolor{currentstroke}{rgb}{0.121569,0.466667,0.705882}%
\pgfsetstrokecolor{currentstroke}%
\pgfsetstrokeopacity{0.870636}%
\pgfsetdash{}{0pt}%
\pgfpathmoveto{\pgfqpoint{0.763661in}{2.448596in}}%
\pgfpathcurveto{\pgfqpoint{0.771897in}{2.448596in}}{\pgfqpoint{0.779797in}{2.451869in}}{\pgfqpoint{0.785621in}{2.457692in}}%
\pgfpathcurveto{\pgfqpoint{0.791445in}{2.463516in}}{\pgfqpoint{0.794717in}{2.471416in}}{\pgfqpoint{0.794717in}{2.479653in}}%
\pgfpathcurveto{\pgfqpoint{0.794717in}{2.487889in}}{\pgfqpoint{0.791445in}{2.495789in}}{\pgfqpoint{0.785621in}{2.501613in}}%
\pgfpathcurveto{\pgfqpoint{0.779797in}{2.507437in}}{\pgfqpoint{0.771897in}{2.510709in}}{\pgfqpoint{0.763661in}{2.510709in}}%
\pgfpathcurveto{\pgfqpoint{0.755424in}{2.510709in}}{\pgfqpoint{0.747524in}{2.507437in}}{\pgfqpoint{0.741700in}{2.501613in}}%
\pgfpathcurveto{\pgfqpoint{0.735876in}{2.495789in}}{\pgfqpoint{0.732604in}{2.487889in}}{\pgfqpoint{0.732604in}{2.479653in}}%
\pgfpathcurveto{\pgfqpoint{0.732604in}{2.471416in}}{\pgfqpoint{0.735876in}{2.463516in}}{\pgfqpoint{0.741700in}{2.457692in}}%
\pgfpathcurveto{\pgfqpoint{0.747524in}{2.451869in}}{\pgfqpoint{0.755424in}{2.448596in}}{\pgfqpoint{0.763661in}{2.448596in}}%
\pgfpathclose%
\pgfusepath{stroke,fill}%
\end{pgfscope}%
\begin{pgfscope}%
\pgfpathrectangle{\pgfqpoint{0.100000in}{0.212622in}}{\pgfqpoint{3.696000in}{3.696000in}}%
\pgfusepath{clip}%
\pgfsetbuttcap%
\pgfsetroundjoin%
\definecolor{currentfill}{rgb}{0.121569,0.466667,0.705882}%
\pgfsetfillcolor{currentfill}%
\pgfsetfillopacity{0.870736}%
\pgfsetlinewidth{1.003750pt}%
\definecolor{currentstroke}{rgb}{0.121569,0.466667,0.705882}%
\pgfsetstrokecolor{currentstroke}%
\pgfsetstrokeopacity{0.870736}%
\pgfsetdash{}{0pt}%
\pgfpathmoveto{\pgfqpoint{2.726032in}{1.719490in}}%
\pgfpathcurveto{\pgfqpoint{2.734268in}{1.719490in}}{\pgfqpoint{2.742168in}{1.722762in}}{\pgfqpoint{2.747992in}{1.728586in}}%
\pgfpathcurveto{\pgfqpoint{2.753816in}{1.734410in}}{\pgfqpoint{2.757088in}{1.742310in}}{\pgfqpoint{2.757088in}{1.750546in}}%
\pgfpathcurveto{\pgfqpoint{2.757088in}{1.758782in}}{\pgfqpoint{2.753816in}{1.766682in}}{\pgfqpoint{2.747992in}{1.772506in}}%
\pgfpathcurveto{\pgfqpoint{2.742168in}{1.778330in}}{\pgfqpoint{2.734268in}{1.781603in}}{\pgfqpoint{2.726032in}{1.781603in}}%
\pgfpathcurveto{\pgfqpoint{2.717795in}{1.781603in}}{\pgfqpoint{2.709895in}{1.778330in}}{\pgfqpoint{2.704071in}{1.772506in}}%
\pgfpathcurveto{\pgfqpoint{2.698247in}{1.766682in}}{\pgfqpoint{2.694975in}{1.758782in}}{\pgfqpoint{2.694975in}{1.750546in}}%
\pgfpathcurveto{\pgfqpoint{2.694975in}{1.742310in}}{\pgfqpoint{2.698247in}{1.734410in}}{\pgfqpoint{2.704071in}{1.728586in}}%
\pgfpathcurveto{\pgfqpoint{2.709895in}{1.722762in}}{\pgfqpoint{2.717795in}{1.719490in}}{\pgfqpoint{2.726032in}{1.719490in}}%
\pgfpathclose%
\pgfusepath{stroke,fill}%
\end{pgfscope}%
\begin{pgfscope}%
\pgfpathrectangle{\pgfqpoint{0.100000in}{0.212622in}}{\pgfqpoint{3.696000in}{3.696000in}}%
\pgfusepath{clip}%
\pgfsetbuttcap%
\pgfsetroundjoin%
\definecolor{currentfill}{rgb}{0.121569,0.466667,0.705882}%
\pgfsetfillcolor{currentfill}%
\pgfsetfillopacity{0.872008}%
\pgfsetlinewidth{1.003750pt}%
\definecolor{currentstroke}{rgb}{0.121569,0.466667,0.705882}%
\pgfsetstrokecolor{currentstroke}%
\pgfsetstrokeopacity{0.872008}%
\pgfsetdash{}{0pt}%
\pgfpathmoveto{\pgfqpoint{0.773933in}{2.443627in}}%
\pgfpathcurveto{\pgfqpoint{0.782170in}{2.443627in}}{\pgfqpoint{0.790070in}{2.446900in}}{\pgfqpoint{0.795894in}{2.452724in}}%
\pgfpathcurveto{\pgfqpoint{0.801718in}{2.458548in}}{\pgfqpoint{0.804990in}{2.466448in}}{\pgfqpoint{0.804990in}{2.474684in}}%
\pgfpathcurveto{\pgfqpoint{0.804990in}{2.482920in}}{\pgfqpoint{0.801718in}{2.490820in}}{\pgfqpoint{0.795894in}{2.496644in}}%
\pgfpathcurveto{\pgfqpoint{0.790070in}{2.502468in}}{\pgfqpoint{0.782170in}{2.505740in}}{\pgfqpoint{0.773933in}{2.505740in}}%
\pgfpathcurveto{\pgfqpoint{0.765697in}{2.505740in}}{\pgfqpoint{0.757797in}{2.502468in}}{\pgfqpoint{0.751973in}{2.496644in}}%
\pgfpathcurveto{\pgfqpoint{0.746149in}{2.490820in}}{\pgfqpoint{0.742877in}{2.482920in}}{\pgfqpoint{0.742877in}{2.474684in}}%
\pgfpathcurveto{\pgfqpoint{0.742877in}{2.466448in}}{\pgfqpoint{0.746149in}{2.458548in}}{\pgfqpoint{0.751973in}{2.452724in}}%
\pgfpathcurveto{\pgfqpoint{0.757797in}{2.446900in}}{\pgfqpoint{0.765697in}{2.443627in}}{\pgfqpoint{0.773933in}{2.443627in}}%
\pgfpathclose%
\pgfusepath{stroke,fill}%
\end{pgfscope}%
\begin{pgfscope}%
\pgfpathrectangle{\pgfqpoint{0.100000in}{0.212622in}}{\pgfqpoint{3.696000in}{3.696000in}}%
\pgfusepath{clip}%
\pgfsetbuttcap%
\pgfsetroundjoin%
\definecolor{currentfill}{rgb}{0.121569,0.466667,0.705882}%
\pgfsetfillcolor{currentfill}%
\pgfsetfillopacity{0.872283}%
\pgfsetlinewidth{1.003750pt}%
\definecolor{currentstroke}{rgb}{0.121569,0.466667,0.705882}%
\pgfsetstrokecolor{currentstroke}%
\pgfsetstrokeopacity{0.872283}%
\pgfsetdash{}{0pt}%
\pgfpathmoveto{\pgfqpoint{2.722088in}{1.707191in}}%
\pgfpathcurveto{\pgfqpoint{2.730324in}{1.707191in}}{\pgfqpoint{2.738224in}{1.710463in}}{\pgfqpoint{2.744048in}{1.716287in}}%
\pgfpathcurveto{\pgfqpoint{2.749872in}{1.722111in}}{\pgfqpoint{2.753144in}{1.730011in}}{\pgfqpoint{2.753144in}{1.738247in}}%
\pgfpathcurveto{\pgfqpoint{2.753144in}{1.746484in}}{\pgfqpoint{2.749872in}{1.754384in}}{\pgfqpoint{2.744048in}{1.760208in}}%
\pgfpathcurveto{\pgfqpoint{2.738224in}{1.766032in}}{\pgfqpoint{2.730324in}{1.769304in}}{\pgfqpoint{2.722088in}{1.769304in}}%
\pgfpathcurveto{\pgfqpoint{2.713852in}{1.769304in}}{\pgfqpoint{2.705951in}{1.766032in}}{\pgfqpoint{2.700128in}{1.760208in}}%
\pgfpathcurveto{\pgfqpoint{2.694304in}{1.754384in}}{\pgfqpoint{2.691031in}{1.746484in}}{\pgfqpoint{2.691031in}{1.738247in}}%
\pgfpathcurveto{\pgfqpoint{2.691031in}{1.730011in}}{\pgfqpoint{2.694304in}{1.722111in}}{\pgfqpoint{2.700128in}{1.716287in}}%
\pgfpathcurveto{\pgfqpoint{2.705951in}{1.710463in}}{\pgfqpoint{2.713852in}{1.707191in}}{\pgfqpoint{2.722088in}{1.707191in}}%
\pgfpathclose%
\pgfusepath{stroke,fill}%
\end{pgfscope}%
\begin{pgfscope}%
\pgfpathrectangle{\pgfqpoint{0.100000in}{0.212622in}}{\pgfqpoint{3.696000in}{3.696000in}}%
\pgfusepath{clip}%
\pgfsetbuttcap%
\pgfsetroundjoin%
\definecolor{currentfill}{rgb}{0.121569,0.466667,0.705882}%
\pgfsetfillcolor{currentfill}%
\pgfsetfillopacity{0.875273}%
\pgfsetlinewidth{1.003750pt}%
\definecolor{currentstroke}{rgb}{0.121569,0.466667,0.705882}%
\pgfsetstrokecolor{currentstroke}%
\pgfsetstrokeopacity{0.875273}%
\pgfsetdash{}{0pt}%
\pgfpathmoveto{\pgfqpoint{0.792791in}{2.440872in}}%
\pgfpathcurveto{\pgfqpoint{0.801027in}{2.440872in}}{\pgfqpoint{0.808927in}{2.444144in}}{\pgfqpoint{0.814751in}{2.449968in}}%
\pgfpathcurveto{\pgfqpoint{0.820575in}{2.455792in}}{\pgfqpoint{0.823847in}{2.463692in}}{\pgfqpoint{0.823847in}{2.471928in}}%
\pgfpathcurveto{\pgfqpoint{0.823847in}{2.480164in}}{\pgfqpoint{0.820575in}{2.488064in}}{\pgfqpoint{0.814751in}{2.493888in}}%
\pgfpathcurveto{\pgfqpoint{0.808927in}{2.499712in}}{\pgfqpoint{0.801027in}{2.502985in}}{\pgfqpoint{0.792791in}{2.502985in}}%
\pgfpathcurveto{\pgfqpoint{0.784554in}{2.502985in}}{\pgfqpoint{0.776654in}{2.499712in}}{\pgfqpoint{0.770830in}{2.493888in}}%
\pgfpathcurveto{\pgfqpoint{0.765007in}{2.488064in}}{\pgfqpoint{0.761734in}{2.480164in}}{\pgfqpoint{0.761734in}{2.471928in}}%
\pgfpathcurveto{\pgfqpoint{0.761734in}{2.463692in}}{\pgfqpoint{0.765007in}{2.455792in}}{\pgfqpoint{0.770830in}{2.449968in}}%
\pgfpathcurveto{\pgfqpoint{0.776654in}{2.444144in}}{\pgfqpoint{0.784554in}{2.440872in}}{\pgfqpoint{0.792791in}{2.440872in}}%
\pgfpathclose%
\pgfusepath{stroke,fill}%
\end{pgfscope}%
\begin{pgfscope}%
\pgfpathrectangle{\pgfqpoint{0.100000in}{0.212622in}}{\pgfqpoint{3.696000in}{3.696000in}}%
\pgfusepath{clip}%
\pgfsetbuttcap%
\pgfsetroundjoin%
\definecolor{currentfill}{rgb}{0.121569,0.466667,0.705882}%
\pgfsetfillcolor{currentfill}%
\pgfsetfillopacity{0.876164}%
\pgfsetlinewidth{1.003750pt}%
\definecolor{currentstroke}{rgb}{0.121569,0.466667,0.705882}%
\pgfsetstrokecolor{currentstroke}%
\pgfsetstrokeopacity{0.876164}%
\pgfsetdash{}{0pt}%
\pgfpathmoveto{\pgfqpoint{0.810375in}{2.423069in}}%
\pgfpathcurveto{\pgfqpoint{0.818612in}{2.423069in}}{\pgfqpoint{0.826512in}{2.426342in}}{\pgfqpoint{0.832336in}{2.432165in}}%
\pgfpathcurveto{\pgfqpoint{0.838160in}{2.437989in}}{\pgfqpoint{0.841432in}{2.445889in}}{\pgfqpoint{0.841432in}{2.454126in}}%
\pgfpathcurveto{\pgfqpoint{0.841432in}{2.462362in}}{\pgfqpoint{0.838160in}{2.470262in}}{\pgfqpoint{0.832336in}{2.476086in}}%
\pgfpathcurveto{\pgfqpoint{0.826512in}{2.481910in}}{\pgfqpoint{0.818612in}{2.485182in}}{\pgfqpoint{0.810375in}{2.485182in}}%
\pgfpathcurveto{\pgfqpoint{0.802139in}{2.485182in}}{\pgfqpoint{0.794239in}{2.481910in}}{\pgfqpoint{0.788415in}{2.476086in}}%
\pgfpathcurveto{\pgfqpoint{0.782591in}{2.470262in}}{\pgfqpoint{0.779319in}{2.462362in}}{\pgfqpoint{0.779319in}{2.454126in}}%
\pgfpathcurveto{\pgfqpoint{0.779319in}{2.445889in}}{\pgfqpoint{0.782591in}{2.437989in}}{\pgfqpoint{0.788415in}{2.432165in}}%
\pgfpathcurveto{\pgfqpoint{0.794239in}{2.426342in}}{\pgfqpoint{0.802139in}{2.423069in}}{\pgfqpoint{0.810375in}{2.423069in}}%
\pgfpathclose%
\pgfusepath{stroke,fill}%
\end{pgfscope}%
\begin{pgfscope}%
\pgfpathrectangle{\pgfqpoint{0.100000in}{0.212622in}}{\pgfqpoint{3.696000in}{3.696000in}}%
\pgfusepath{clip}%
\pgfsetbuttcap%
\pgfsetroundjoin%
\definecolor{currentfill}{rgb}{0.121569,0.466667,0.705882}%
\pgfsetfillcolor{currentfill}%
\pgfsetfillopacity{0.876574}%
\pgfsetlinewidth{1.003750pt}%
\definecolor{currentstroke}{rgb}{0.121569,0.466667,0.705882}%
\pgfsetstrokecolor{currentstroke}%
\pgfsetstrokeopacity{0.876574}%
\pgfsetdash{}{0pt}%
\pgfpathmoveto{\pgfqpoint{2.711189in}{1.713056in}}%
\pgfpathcurveto{\pgfqpoint{2.719426in}{1.713056in}}{\pgfqpoint{2.727326in}{1.716329in}}{\pgfqpoint{2.733149in}{1.722153in}}%
\pgfpathcurveto{\pgfqpoint{2.738973in}{1.727976in}}{\pgfqpoint{2.742246in}{1.735877in}}{\pgfqpoint{2.742246in}{1.744113in}}%
\pgfpathcurveto{\pgfqpoint{2.742246in}{1.752349in}}{\pgfqpoint{2.738973in}{1.760249in}}{\pgfqpoint{2.733149in}{1.766073in}}%
\pgfpathcurveto{\pgfqpoint{2.727326in}{1.771897in}}{\pgfqpoint{2.719426in}{1.775169in}}{\pgfqpoint{2.711189in}{1.775169in}}%
\pgfpathcurveto{\pgfqpoint{2.702953in}{1.775169in}}{\pgfqpoint{2.695053in}{1.771897in}}{\pgfqpoint{2.689229in}{1.766073in}}%
\pgfpathcurveto{\pgfqpoint{2.683405in}{1.760249in}}{\pgfqpoint{2.680133in}{1.752349in}}{\pgfqpoint{2.680133in}{1.744113in}}%
\pgfpathcurveto{\pgfqpoint{2.680133in}{1.735877in}}{\pgfqpoint{2.683405in}{1.727976in}}{\pgfqpoint{2.689229in}{1.722153in}}%
\pgfpathcurveto{\pgfqpoint{2.695053in}{1.716329in}}{\pgfqpoint{2.702953in}{1.713056in}}{\pgfqpoint{2.711189in}{1.713056in}}%
\pgfpathclose%
\pgfusepath{stroke,fill}%
\end{pgfscope}%
\begin{pgfscope}%
\pgfpathrectangle{\pgfqpoint{0.100000in}{0.212622in}}{\pgfqpoint{3.696000in}{3.696000in}}%
\pgfusepath{clip}%
\pgfsetbuttcap%
\pgfsetroundjoin%
\definecolor{currentfill}{rgb}{0.121569,0.466667,0.705882}%
\pgfsetfillcolor{currentfill}%
\pgfsetfillopacity{0.877566}%
\pgfsetlinewidth{1.003750pt}%
\definecolor{currentstroke}{rgb}{0.121569,0.466667,0.705882}%
\pgfsetstrokecolor{currentstroke}%
\pgfsetstrokeopacity{0.877566}%
\pgfsetdash{}{0pt}%
\pgfpathmoveto{\pgfqpoint{0.827014in}{2.417900in}}%
\pgfpathcurveto{\pgfqpoint{0.835250in}{2.417900in}}{\pgfqpoint{0.843150in}{2.421172in}}{\pgfqpoint{0.848974in}{2.426996in}}%
\pgfpathcurveto{\pgfqpoint{0.854798in}{2.432820in}}{\pgfqpoint{0.858070in}{2.440720in}}{\pgfqpoint{0.858070in}{2.448956in}}%
\pgfpathcurveto{\pgfqpoint{0.858070in}{2.457193in}}{\pgfqpoint{0.854798in}{2.465093in}}{\pgfqpoint{0.848974in}{2.470917in}}%
\pgfpathcurveto{\pgfqpoint{0.843150in}{2.476741in}}{\pgfqpoint{0.835250in}{2.480013in}}{\pgfqpoint{0.827014in}{2.480013in}}%
\pgfpathcurveto{\pgfqpoint{0.818777in}{2.480013in}}{\pgfqpoint{0.810877in}{2.476741in}}{\pgfqpoint{0.805054in}{2.470917in}}%
\pgfpathcurveto{\pgfqpoint{0.799230in}{2.465093in}}{\pgfqpoint{0.795957in}{2.457193in}}{\pgfqpoint{0.795957in}{2.448956in}}%
\pgfpathcurveto{\pgfqpoint{0.795957in}{2.440720in}}{\pgfqpoint{0.799230in}{2.432820in}}{\pgfqpoint{0.805054in}{2.426996in}}%
\pgfpathcurveto{\pgfqpoint{0.810877in}{2.421172in}}{\pgfqpoint{0.818777in}{2.417900in}}{\pgfqpoint{0.827014in}{2.417900in}}%
\pgfpathclose%
\pgfusepath{stroke,fill}%
\end{pgfscope}%
\begin{pgfscope}%
\pgfpathrectangle{\pgfqpoint{0.100000in}{0.212622in}}{\pgfqpoint{3.696000in}{3.696000in}}%
\pgfusepath{clip}%
\pgfsetbuttcap%
\pgfsetroundjoin%
\definecolor{currentfill}{rgb}{0.121569,0.466667,0.705882}%
\pgfsetfillcolor{currentfill}%
\pgfsetfillopacity{0.878968}%
\pgfsetlinewidth{1.003750pt}%
\definecolor{currentstroke}{rgb}{0.121569,0.466667,0.705882}%
\pgfsetstrokecolor{currentstroke}%
\pgfsetstrokeopacity{0.878968}%
\pgfsetdash{}{0pt}%
\pgfpathmoveto{\pgfqpoint{2.709345in}{1.711737in}}%
\pgfpathcurveto{\pgfqpoint{2.717581in}{1.711737in}}{\pgfqpoint{2.725481in}{1.715010in}}{\pgfqpoint{2.731305in}{1.720834in}}%
\pgfpathcurveto{\pgfqpoint{2.737129in}{1.726658in}}{\pgfqpoint{2.740401in}{1.734558in}}{\pgfqpoint{2.740401in}{1.742794in}}%
\pgfpathcurveto{\pgfqpoint{2.740401in}{1.751030in}}{\pgfqpoint{2.737129in}{1.758930in}}{\pgfqpoint{2.731305in}{1.764754in}}%
\pgfpathcurveto{\pgfqpoint{2.725481in}{1.770578in}}{\pgfqpoint{2.717581in}{1.773850in}}{\pgfqpoint{2.709345in}{1.773850in}}%
\pgfpathcurveto{\pgfqpoint{2.701109in}{1.773850in}}{\pgfqpoint{2.693209in}{1.770578in}}{\pgfqpoint{2.687385in}{1.764754in}}%
\pgfpathcurveto{\pgfqpoint{2.681561in}{1.758930in}}{\pgfqpoint{2.678288in}{1.751030in}}{\pgfqpoint{2.678288in}{1.742794in}}%
\pgfpathcurveto{\pgfqpoint{2.678288in}{1.734558in}}{\pgfqpoint{2.681561in}{1.726658in}}{\pgfqpoint{2.687385in}{1.720834in}}%
\pgfpathcurveto{\pgfqpoint{2.693209in}{1.715010in}}{\pgfqpoint{2.701109in}{1.711737in}}{\pgfqpoint{2.709345in}{1.711737in}}%
\pgfpathclose%
\pgfusepath{stroke,fill}%
\end{pgfscope}%
\begin{pgfscope}%
\pgfpathrectangle{\pgfqpoint{0.100000in}{0.212622in}}{\pgfqpoint{3.696000in}{3.696000in}}%
\pgfusepath{clip}%
\pgfsetbuttcap%
\pgfsetroundjoin%
\definecolor{currentfill}{rgb}{0.121569,0.466667,0.705882}%
\pgfsetfillcolor{currentfill}%
\pgfsetfillopacity{0.879312}%
\pgfsetlinewidth{1.003750pt}%
\definecolor{currentstroke}{rgb}{0.121569,0.466667,0.705882}%
\pgfsetstrokecolor{currentstroke}%
\pgfsetstrokeopacity{0.879312}%
\pgfsetdash{}{0pt}%
\pgfpathmoveto{\pgfqpoint{0.855857in}{2.396604in}}%
\pgfpathcurveto{\pgfqpoint{0.864093in}{2.396604in}}{\pgfqpoint{0.871993in}{2.399876in}}{\pgfqpoint{0.877817in}{2.405700in}}%
\pgfpathcurveto{\pgfqpoint{0.883641in}{2.411524in}}{\pgfqpoint{0.886913in}{2.419424in}}{\pgfqpoint{0.886913in}{2.427660in}}%
\pgfpathcurveto{\pgfqpoint{0.886913in}{2.435897in}}{\pgfqpoint{0.883641in}{2.443797in}}{\pgfqpoint{0.877817in}{2.449621in}}%
\pgfpathcurveto{\pgfqpoint{0.871993in}{2.455445in}}{\pgfqpoint{0.864093in}{2.458717in}}{\pgfqpoint{0.855857in}{2.458717in}}%
\pgfpathcurveto{\pgfqpoint{0.847620in}{2.458717in}}{\pgfqpoint{0.839720in}{2.455445in}}{\pgfqpoint{0.833896in}{2.449621in}}%
\pgfpathcurveto{\pgfqpoint{0.828072in}{2.443797in}}{\pgfqpoint{0.824800in}{2.435897in}}{\pgfqpoint{0.824800in}{2.427660in}}%
\pgfpathcurveto{\pgfqpoint{0.824800in}{2.419424in}}{\pgfqpoint{0.828072in}{2.411524in}}{\pgfqpoint{0.833896in}{2.405700in}}%
\pgfpathcurveto{\pgfqpoint{0.839720in}{2.399876in}}{\pgfqpoint{0.847620in}{2.396604in}}{\pgfqpoint{0.855857in}{2.396604in}}%
\pgfpathclose%
\pgfusepath{stroke,fill}%
\end{pgfscope}%
\begin{pgfscope}%
\pgfpathrectangle{\pgfqpoint{0.100000in}{0.212622in}}{\pgfqpoint{3.696000in}{3.696000in}}%
\pgfusepath{clip}%
\pgfsetbuttcap%
\pgfsetroundjoin%
\definecolor{currentfill}{rgb}{0.121569,0.466667,0.705882}%
\pgfsetfillcolor{currentfill}%
\pgfsetfillopacity{0.880235}%
\pgfsetlinewidth{1.003750pt}%
\definecolor{currentstroke}{rgb}{0.121569,0.466667,0.705882}%
\pgfsetstrokecolor{currentstroke}%
\pgfsetstrokeopacity{0.880235}%
\pgfsetdash{}{0pt}%
\pgfpathmoveto{\pgfqpoint{2.704256in}{1.703734in}}%
\pgfpathcurveto{\pgfqpoint{2.712492in}{1.703734in}}{\pgfqpoint{2.720392in}{1.707006in}}{\pgfqpoint{2.726216in}{1.712830in}}%
\pgfpathcurveto{\pgfqpoint{2.732040in}{1.718654in}}{\pgfqpoint{2.735312in}{1.726554in}}{\pgfqpoint{2.735312in}{1.734790in}}%
\pgfpathcurveto{\pgfqpoint{2.735312in}{1.743027in}}{\pgfqpoint{2.732040in}{1.750927in}}{\pgfqpoint{2.726216in}{1.756751in}}%
\pgfpathcurveto{\pgfqpoint{2.720392in}{1.762575in}}{\pgfqpoint{2.712492in}{1.765847in}}{\pgfqpoint{2.704256in}{1.765847in}}%
\pgfpathcurveto{\pgfqpoint{2.696019in}{1.765847in}}{\pgfqpoint{2.688119in}{1.762575in}}{\pgfqpoint{2.682295in}{1.756751in}}%
\pgfpathcurveto{\pgfqpoint{2.676471in}{1.750927in}}{\pgfqpoint{2.673199in}{1.743027in}}{\pgfqpoint{2.673199in}{1.734790in}}%
\pgfpathcurveto{\pgfqpoint{2.673199in}{1.726554in}}{\pgfqpoint{2.676471in}{1.718654in}}{\pgfqpoint{2.682295in}{1.712830in}}%
\pgfpathcurveto{\pgfqpoint{2.688119in}{1.707006in}}{\pgfqpoint{2.696019in}{1.703734in}}{\pgfqpoint{2.704256in}{1.703734in}}%
\pgfpathclose%
\pgfusepath{stroke,fill}%
\end{pgfscope}%
\begin{pgfscope}%
\pgfpathrectangle{\pgfqpoint{0.100000in}{0.212622in}}{\pgfqpoint{3.696000in}{3.696000in}}%
\pgfusepath{clip}%
\pgfsetbuttcap%
\pgfsetroundjoin%
\definecolor{currentfill}{rgb}{0.121569,0.466667,0.705882}%
\pgfsetfillcolor{currentfill}%
\pgfsetfillopacity{0.881157}%
\pgfsetlinewidth{1.003750pt}%
\definecolor{currentstroke}{rgb}{0.121569,0.466667,0.705882}%
\pgfsetstrokecolor{currentstroke}%
\pgfsetstrokeopacity{0.881157}%
\pgfsetdash{}{0pt}%
\pgfpathmoveto{\pgfqpoint{0.885869in}{2.382530in}}%
\pgfpathcurveto{\pgfqpoint{0.894105in}{2.382530in}}{\pgfqpoint{0.902005in}{2.385802in}}{\pgfqpoint{0.907829in}{2.391626in}}%
\pgfpathcurveto{\pgfqpoint{0.913653in}{2.397450in}}{\pgfqpoint{0.916925in}{2.405350in}}{\pgfqpoint{0.916925in}{2.413586in}}%
\pgfpathcurveto{\pgfqpoint{0.916925in}{2.421823in}}{\pgfqpoint{0.913653in}{2.429723in}}{\pgfqpoint{0.907829in}{2.435547in}}%
\pgfpathcurveto{\pgfqpoint{0.902005in}{2.441371in}}{\pgfqpoint{0.894105in}{2.444643in}}{\pgfqpoint{0.885869in}{2.444643in}}%
\pgfpathcurveto{\pgfqpoint{0.877633in}{2.444643in}}{\pgfqpoint{0.869732in}{2.441371in}}{\pgfqpoint{0.863909in}{2.435547in}}%
\pgfpathcurveto{\pgfqpoint{0.858085in}{2.429723in}}{\pgfqpoint{0.854812in}{2.421823in}}{\pgfqpoint{0.854812in}{2.413586in}}%
\pgfpathcurveto{\pgfqpoint{0.854812in}{2.405350in}}{\pgfqpoint{0.858085in}{2.397450in}}{\pgfqpoint{0.863909in}{2.391626in}}%
\pgfpathcurveto{\pgfqpoint{0.869732in}{2.385802in}}{\pgfqpoint{0.877633in}{2.382530in}}{\pgfqpoint{0.885869in}{2.382530in}}%
\pgfpathclose%
\pgfusepath{stroke,fill}%
\end{pgfscope}%
\begin{pgfscope}%
\pgfpathrectangle{\pgfqpoint{0.100000in}{0.212622in}}{\pgfqpoint{3.696000in}{3.696000in}}%
\pgfusepath{clip}%
\pgfsetbuttcap%
\pgfsetroundjoin%
\definecolor{currentfill}{rgb}{0.121569,0.466667,0.705882}%
\pgfsetfillcolor{currentfill}%
\pgfsetfillopacity{0.881904}%
\pgfsetlinewidth{1.003750pt}%
\definecolor{currentstroke}{rgb}{0.121569,0.466667,0.705882}%
\pgfsetstrokecolor{currentstroke}%
\pgfsetstrokeopacity{0.881904}%
\pgfsetdash{}{0pt}%
\pgfpathmoveto{\pgfqpoint{0.912891in}{2.362647in}}%
\pgfpathcurveto{\pgfqpoint{0.921127in}{2.362647in}}{\pgfqpoint{0.929027in}{2.365920in}}{\pgfqpoint{0.934851in}{2.371744in}}%
\pgfpathcurveto{\pgfqpoint{0.940675in}{2.377567in}}{\pgfqpoint{0.943947in}{2.385468in}}{\pgfqpoint{0.943947in}{2.393704in}}%
\pgfpathcurveto{\pgfqpoint{0.943947in}{2.401940in}}{\pgfqpoint{0.940675in}{2.409840in}}{\pgfqpoint{0.934851in}{2.415664in}}%
\pgfpathcurveto{\pgfqpoint{0.929027in}{2.421488in}}{\pgfqpoint{0.921127in}{2.424760in}}{\pgfqpoint{0.912891in}{2.424760in}}%
\pgfpathcurveto{\pgfqpoint{0.904654in}{2.424760in}}{\pgfqpoint{0.896754in}{2.421488in}}{\pgfqpoint{0.890930in}{2.415664in}}%
\pgfpathcurveto{\pgfqpoint{0.885107in}{2.409840in}}{\pgfqpoint{0.881834in}{2.401940in}}{\pgfqpoint{0.881834in}{2.393704in}}%
\pgfpathcurveto{\pgfqpoint{0.881834in}{2.385468in}}{\pgfqpoint{0.885107in}{2.377567in}}{\pgfqpoint{0.890930in}{2.371744in}}%
\pgfpathcurveto{\pgfqpoint{0.896754in}{2.365920in}}{\pgfqpoint{0.904654in}{2.362647in}}{\pgfqpoint{0.912891in}{2.362647in}}%
\pgfpathclose%
\pgfusepath{stroke,fill}%
\end{pgfscope}%
\begin{pgfscope}%
\pgfpathrectangle{\pgfqpoint{0.100000in}{0.212622in}}{\pgfqpoint{3.696000in}{3.696000in}}%
\pgfusepath{clip}%
\pgfsetbuttcap%
\pgfsetroundjoin%
\definecolor{currentfill}{rgb}{0.121569,0.466667,0.705882}%
\pgfsetfillcolor{currentfill}%
\pgfsetfillopacity{0.883416}%
\pgfsetlinewidth{1.003750pt}%
\definecolor{currentstroke}{rgb}{0.121569,0.466667,0.705882}%
\pgfsetstrokecolor{currentstroke}%
\pgfsetstrokeopacity{0.883416}%
\pgfsetdash{}{0pt}%
\pgfpathmoveto{\pgfqpoint{0.936651in}{2.345068in}}%
\pgfpathcurveto{\pgfqpoint{0.944887in}{2.345068in}}{\pgfqpoint{0.952787in}{2.348340in}}{\pgfqpoint{0.958611in}{2.354164in}}%
\pgfpathcurveto{\pgfqpoint{0.964435in}{2.359988in}}{\pgfqpoint{0.967707in}{2.367888in}}{\pgfqpoint{0.967707in}{2.376124in}}%
\pgfpathcurveto{\pgfqpoint{0.967707in}{2.384360in}}{\pgfqpoint{0.964435in}{2.392261in}}{\pgfqpoint{0.958611in}{2.398084in}}%
\pgfpathcurveto{\pgfqpoint{0.952787in}{2.403908in}}{\pgfqpoint{0.944887in}{2.407181in}}{\pgfqpoint{0.936651in}{2.407181in}}%
\pgfpathcurveto{\pgfqpoint{0.928414in}{2.407181in}}{\pgfqpoint{0.920514in}{2.403908in}}{\pgfqpoint{0.914690in}{2.398084in}}%
\pgfpathcurveto{\pgfqpoint{0.908866in}{2.392261in}}{\pgfqpoint{0.905594in}{2.384360in}}{\pgfqpoint{0.905594in}{2.376124in}}%
\pgfpathcurveto{\pgfqpoint{0.905594in}{2.367888in}}{\pgfqpoint{0.908866in}{2.359988in}}{\pgfqpoint{0.914690in}{2.354164in}}%
\pgfpathcurveto{\pgfqpoint{0.920514in}{2.348340in}}{\pgfqpoint{0.928414in}{2.345068in}}{\pgfqpoint{0.936651in}{2.345068in}}%
\pgfpathclose%
\pgfusepath{stroke,fill}%
\end{pgfscope}%
\begin{pgfscope}%
\pgfpathrectangle{\pgfqpoint{0.100000in}{0.212622in}}{\pgfqpoint{3.696000in}{3.696000in}}%
\pgfusepath{clip}%
\pgfsetbuttcap%
\pgfsetroundjoin%
\definecolor{currentfill}{rgb}{0.121569,0.466667,0.705882}%
\pgfsetfillcolor{currentfill}%
\pgfsetfillopacity{0.883709}%
\pgfsetlinewidth{1.003750pt}%
\definecolor{currentstroke}{rgb}{0.121569,0.466667,0.705882}%
\pgfsetstrokecolor{currentstroke}%
\pgfsetstrokeopacity{0.883709}%
\pgfsetdash{}{0pt}%
\pgfpathmoveto{\pgfqpoint{2.693692in}{1.712187in}}%
\pgfpathcurveto{\pgfqpoint{2.701929in}{1.712187in}}{\pgfqpoint{2.709829in}{1.715459in}}{\pgfqpoint{2.715653in}{1.721283in}}%
\pgfpathcurveto{\pgfqpoint{2.721477in}{1.727107in}}{\pgfqpoint{2.724749in}{1.735007in}}{\pgfqpoint{2.724749in}{1.743243in}}%
\pgfpathcurveto{\pgfqpoint{2.724749in}{1.751479in}}{\pgfqpoint{2.721477in}{1.759379in}}{\pgfqpoint{2.715653in}{1.765203in}}%
\pgfpathcurveto{\pgfqpoint{2.709829in}{1.771027in}}{\pgfqpoint{2.701929in}{1.774300in}}{\pgfqpoint{2.693692in}{1.774300in}}%
\pgfpathcurveto{\pgfqpoint{2.685456in}{1.774300in}}{\pgfqpoint{2.677556in}{1.771027in}}{\pgfqpoint{2.671732in}{1.765203in}}%
\pgfpathcurveto{\pgfqpoint{2.665908in}{1.759379in}}{\pgfqpoint{2.662636in}{1.751479in}}{\pgfqpoint{2.662636in}{1.743243in}}%
\pgfpathcurveto{\pgfqpoint{2.662636in}{1.735007in}}{\pgfqpoint{2.665908in}{1.727107in}}{\pgfqpoint{2.671732in}{1.721283in}}%
\pgfpathcurveto{\pgfqpoint{2.677556in}{1.715459in}}{\pgfqpoint{2.685456in}{1.712187in}}{\pgfqpoint{2.693692in}{1.712187in}}%
\pgfpathclose%
\pgfusepath{stroke,fill}%
\end{pgfscope}%
\begin{pgfscope}%
\pgfpathrectangle{\pgfqpoint{0.100000in}{0.212622in}}{\pgfqpoint{3.696000in}{3.696000in}}%
\pgfusepath{clip}%
\pgfsetbuttcap%
\pgfsetroundjoin%
\definecolor{currentfill}{rgb}{0.121569,0.466667,0.705882}%
\pgfsetfillcolor{currentfill}%
\pgfsetfillopacity{0.885036}%
\pgfsetlinewidth{1.003750pt}%
\definecolor{currentstroke}{rgb}{0.121569,0.466667,0.705882}%
\pgfsetstrokecolor{currentstroke}%
\pgfsetstrokeopacity{0.885036}%
\pgfsetdash{}{0pt}%
\pgfpathmoveto{\pgfqpoint{0.960741in}{2.330924in}}%
\pgfpathcurveto{\pgfqpoint{0.968977in}{2.330924in}}{\pgfqpoint{0.976877in}{2.334196in}}{\pgfqpoint{0.982701in}{2.340020in}}%
\pgfpathcurveto{\pgfqpoint{0.988525in}{2.345844in}}{\pgfqpoint{0.991797in}{2.353744in}}{\pgfqpoint{0.991797in}{2.361981in}}%
\pgfpathcurveto{\pgfqpoint{0.991797in}{2.370217in}}{\pgfqpoint{0.988525in}{2.378117in}}{\pgfqpoint{0.982701in}{2.383941in}}%
\pgfpathcurveto{\pgfqpoint{0.976877in}{2.389765in}}{\pgfqpoint{0.968977in}{2.393037in}}{\pgfqpoint{0.960741in}{2.393037in}}%
\pgfpathcurveto{\pgfqpoint{0.952504in}{2.393037in}}{\pgfqpoint{0.944604in}{2.389765in}}{\pgfqpoint{0.938780in}{2.383941in}}%
\pgfpathcurveto{\pgfqpoint{0.932956in}{2.378117in}}{\pgfqpoint{0.929684in}{2.370217in}}{\pgfqpoint{0.929684in}{2.361981in}}%
\pgfpathcurveto{\pgfqpoint{0.929684in}{2.353744in}}{\pgfqpoint{0.932956in}{2.345844in}}{\pgfqpoint{0.938780in}{2.340020in}}%
\pgfpathcurveto{\pgfqpoint{0.944604in}{2.334196in}}{\pgfqpoint{0.952504in}{2.330924in}}{\pgfqpoint{0.960741in}{2.330924in}}%
\pgfpathclose%
\pgfusepath{stroke,fill}%
\end{pgfscope}%
\begin{pgfscope}%
\pgfpathrectangle{\pgfqpoint{0.100000in}{0.212622in}}{\pgfqpoint{3.696000in}{3.696000in}}%
\pgfusepath{clip}%
\pgfsetbuttcap%
\pgfsetroundjoin%
\definecolor{currentfill}{rgb}{0.121569,0.466667,0.705882}%
\pgfsetfillcolor{currentfill}%
\pgfsetfillopacity{0.885440}%
\pgfsetlinewidth{1.003750pt}%
\definecolor{currentstroke}{rgb}{0.121569,0.466667,0.705882}%
\pgfsetstrokecolor{currentstroke}%
\pgfsetstrokeopacity{0.885440}%
\pgfsetdash{}{0pt}%
\pgfpathmoveto{\pgfqpoint{2.691172in}{1.710127in}}%
\pgfpathcurveto{\pgfqpoint{2.699409in}{1.710127in}}{\pgfqpoint{2.707309in}{1.713399in}}{\pgfqpoint{2.713133in}{1.719223in}}%
\pgfpathcurveto{\pgfqpoint{2.718957in}{1.725047in}}{\pgfqpoint{2.722229in}{1.732947in}}{\pgfqpoint{2.722229in}{1.741184in}}%
\pgfpathcurveto{\pgfqpoint{2.722229in}{1.749420in}}{\pgfqpoint{2.718957in}{1.757320in}}{\pgfqpoint{2.713133in}{1.763144in}}%
\pgfpathcurveto{\pgfqpoint{2.707309in}{1.768968in}}{\pgfqpoint{2.699409in}{1.772240in}}{\pgfqpoint{2.691172in}{1.772240in}}%
\pgfpathcurveto{\pgfqpoint{2.682936in}{1.772240in}}{\pgfqpoint{2.675036in}{1.768968in}}{\pgfqpoint{2.669212in}{1.763144in}}%
\pgfpathcurveto{\pgfqpoint{2.663388in}{1.757320in}}{\pgfqpoint{2.660116in}{1.749420in}}{\pgfqpoint{2.660116in}{1.741184in}}%
\pgfpathcurveto{\pgfqpoint{2.660116in}{1.732947in}}{\pgfqpoint{2.663388in}{1.725047in}}{\pgfqpoint{2.669212in}{1.719223in}}%
\pgfpathcurveto{\pgfqpoint{2.675036in}{1.713399in}}{\pgfqpoint{2.682936in}{1.710127in}}{\pgfqpoint{2.691172in}{1.710127in}}%
\pgfpathclose%
\pgfusepath{stroke,fill}%
\end{pgfscope}%
\begin{pgfscope}%
\pgfpathrectangle{\pgfqpoint{0.100000in}{0.212622in}}{\pgfqpoint{3.696000in}{3.696000in}}%
\pgfusepath{clip}%
\pgfsetbuttcap%
\pgfsetroundjoin%
\definecolor{currentfill}{rgb}{0.121569,0.466667,0.705882}%
\pgfsetfillcolor{currentfill}%
\pgfsetfillopacity{0.886565}%
\pgfsetlinewidth{1.003750pt}%
\definecolor{currentstroke}{rgb}{0.121569,0.466667,0.705882}%
\pgfsetstrokecolor{currentstroke}%
\pgfsetstrokeopacity{0.886565}%
\pgfsetdash{}{0pt}%
\pgfpathmoveto{\pgfqpoint{2.685864in}{1.706205in}}%
\pgfpathcurveto{\pgfqpoint{2.694100in}{1.706205in}}{\pgfqpoint{2.702000in}{1.709478in}}{\pgfqpoint{2.707824in}{1.715302in}}%
\pgfpathcurveto{\pgfqpoint{2.713648in}{1.721126in}}{\pgfqpoint{2.716920in}{1.729026in}}{\pgfqpoint{2.716920in}{1.737262in}}%
\pgfpathcurveto{\pgfqpoint{2.716920in}{1.745498in}}{\pgfqpoint{2.713648in}{1.753398in}}{\pgfqpoint{2.707824in}{1.759222in}}%
\pgfpathcurveto{\pgfqpoint{2.702000in}{1.765046in}}{\pgfqpoint{2.694100in}{1.768318in}}{\pgfqpoint{2.685864in}{1.768318in}}%
\pgfpathcurveto{\pgfqpoint{2.677628in}{1.768318in}}{\pgfqpoint{2.669728in}{1.765046in}}{\pgfqpoint{2.663904in}{1.759222in}}%
\pgfpathcurveto{\pgfqpoint{2.658080in}{1.753398in}}{\pgfqpoint{2.654807in}{1.745498in}}{\pgfqpoint{2.654807in}{1.737262in}}%
\pgfpathcurveto{\pgfqpoint{2.654807in}{1.729026in}}{\pgfqpoint{2.658080in}{1.721126in}}{\pgfqpoint{2.663904in}{1.715302in}}%
\pgfpathcurveto{\pgfqpoint{2.669728in}{1.709478in}}{\pgfqpoint{2.677628in}{1.706205in}}{\pgfqpoint{2.685864in}{1.706205in}}%
\pgfpathclose%
\pgfusepath{stroke,fill}%
\end{pgfscope}%
\begin{pgfscope}%
\pgfpathrectangle{\pgfqpoint{0.100000in}{0.212622in}}{\pgfqpoint{3.696000in}{3.696000in}}%
\pgfusepath{clip}%
\pgfsetbuttcap%
\pgfsetroundjoin%
\definecolor{currentfill}{rgb}{0.121569,0.466667,0.705882}%
\pgfsetfillcolor{currentfill}%
\pgfsetfillopacity{0.886845}%
\pgfsetlinewidth{1.003750pt}%
\definecolor{currentstroke}{rgb}{0.121569,0.466667,0.705882}%
\pgfsetstrokecolor{currentstroke}%
\pgfsetstrokeopacity{0.886845}%
\pgfsetdash{}{0pt}%
\pgfpathmoveto{\pgfqpoint{0.984138in}{2.320841in}}%
\pgfpathcurveto{\pgfqpoint{0.992374in}{2.320841in}}{\pgfqpoint{1.000274in}{2.324113in}}{\pgfqpoint{1.006098in}{2.329937in}}%
\pgfpathcurveto{\pgfqpoint{1.011922in}{2.335761in}}{\pgfqpoint{1.015194in}{2.343661in}}{\pgfqpoint{1.015194in}{2.351897in}}%
\pgfpathcurveto{\pgfqpoint{1.015194in}{2.360133in}}{\pgfqpoint{1.011922in}{2.368033in}}{\pgfqpoint{1.006098in}{2.373857in}}%
\pgfpathcurveto{\pgfqpoint{1.000274in}{2.379681in}}{\pgfqpoint{0.992374in}{2.382954in}}{\pgfqpoint{0.984138in}{2.382954in}}%
\pgfpathcurveto{\pgfqpoint{0.975902in}{2.382954in}}{\pgfqpoint{0.968001in}{2.379681in}}{\pgfqpoint{0.962178in}{2.373857in}}%
\pgfpathcurveto{\pgfqpoint{0.956354in}{2.368033in}}{\pgfqpoint{0.953081in}{2.360133in}}{\pgfqpoint{0.953081in}{2.351897in}}%
\pgfpathcurveto{\pgfqpoint{0.953081in}{2.343661in}}{\pgfqpoint{0.956354in}{2.335761in}}{\pgfqpoint{0.962178in}{2.329937in}}%
\pgfpathcurveto{\pgfqpoint{0.968001in}{2.324113in}}{\pgfqpoint{0.975902in}{2.320841in}}{\pgfqpoint{0.984138in}{2.320841in}}%
\pgfpathclose%
\pgfusepath{stroke,fill}%
\end{pgfscope}%
\begin{pgfscope}%
\pgfpathrectangle{\pgfqpoint{0.100000in}{0.212622in}}{\pgfqpoint{3.696000in}{3.696000in}}%
\pgfusepath{clip}%
\pgfsetbuttcap%
\pgfsetroundjoin%
\definecolor{currentfill}{rgb}{0.121569,0.466667,0.705882}%
\pgfsetfillcolor{currentfill}%
\pgfsetfillopacity{0.887676}%
\pgfsetlinewidth{1.003750pt}%
\definecolor{currentstroke}{rgb}{0.121569,0.466667,0.705882}%
\pgfsetstrokecolor{currentstroke}%
\pgfsetstrokeopacity{0.887676}%
\pgfsetdash{}{0pt}%
\pgfpathmoveto{\pgfqpoint{1.003351in}{2.302354in}}%
\pgfpathcurveto{\pgfqpoint{1.011587in}{2.302354in}}{\pgfqpoint{1.019488in}{2.305626in}}{\pgfqpoint{1.025311in}{2.311450in}}%
\pgfpathcurveto{\pgfqpoint{1.031135in}{2.317274in}}{\pgfqpoint{1.034408in}{2.325174in}}{\pgfqpoint{1.034408in}{2.333411in}}%
\pgfpathcurveto{\pgfqpoint{1.034408in}{2.341647in}}{\pgfqpoint{1.031135in}{2.349547in}}{\pgfqpoint{1.025311in}{2.355371in}}%
\pgfpathcurveto{\pgfqpoint{1.019488in}{2.361195in}}{\pgfqpoint{1.011587in}{2.364467in}}{\pgfqpoint{1.003351in}{2.364467in}}%
\pgfpathcurveto{\pgfqpoint{0.995115in}{2.364467in}}{\pgfqpoint{0.987215in}{2.361195in}}{\pgfqpoint{0.981391in}{2.355371in}}%
\pgfpathcurveto{\pgfqpoint{0.975567in}{2.349547in}}{\pgfqpoint{0.972295in}{2.341647in}}{\pgfqpoint{0.972295in}{2.333411in}}%
\pgfpathcurveto{\pgfqpoint{0.972295in}{2.325174in}}{\pgfqpoint{0.975567in}{2.317274in}}{\pgfqpoint{0.981391in}{2.311450in}}%
\pgfpathcurveto{\pgfqpoint{0.987215in}{2.305626in}}{\pgfqpoint{0.995115in}{2.302354in}}{\pgfqpoint{1.003351in}{2.302354in}}%
\pgfpathclose%
\pgfusepath{stroke,fill}%
\end{pgfscope}%
\begin{pgfscope}%
\pgfpathrectangle{\pgfqpoint{0.100000in}{0.212622in}}{\pgfqpoint{3.696000in}{3.696000in}}%
\pgfusepath{clip}%
\pgfsetbuttcap%
\pgfsetroundjoin%
\definecolor{currentfill}{rgb}{0.121569,0.466667,0.705882}%
\pgfsetfillcolor{currentfill}%
\pgfsetfillopacity{0.888475}%
\pgfsetlinewidth{1.003750pt}%
\definecolor{currentstroke}{rgb}{0.121569,0.466667,0.705882}%
\pgfsetstrokecolor{currentstroke}%
\pgfsetstrokeopacity{0.888475}%
\pgfsetdash{}{0pt}%
\pgfpathmoveto{\pgfqpoint{2.678163in}{1.705489in}}%
\pgfpathcurveto{\pgfqpoint{2.686399in}{1.705489in}}{\pgfqpoint{2.694299in}{1.708762in}}{\pgfqpoint{2.700123in}{1.714586in}}%
\pgfpathcurveto{\pgfqpoint{2.705947in}{1.720409in}}{\pgfqpoint{2.709220in}{1.728310in}}{\pgfqpoint{2.709220in}{1.736546in}}%
\pgfpathcurveto{\pgfqpoint{2.709220in}{1.744782in}}{\pgfqpoint{2.705947in}{1.752682in}}{\pgfqpoint{2.700123in}{1.758506in}}%
\pgfpathcurveto{\pgfqpoint{2.694299in}{1.764330in}}{\pgfqpoint{2.686399in}{1.767602in}}{\pgfqpoint{2.678163in}{1.767602in}}%
\pgfpathcurveto{\pgfqpoint{2.669927in}{1.767602in}}{\pgfqpoint{2.662027in}{1.764330in}}{\pgfqpoint{2.656203in}{1.758506in}}%
\pgfpathcurveto{\pgfqpoint{2.650379in}{1.752682in}}{\pgfqpoint{2.647107in}{1.744782in}}{\pgfqpoint{2.647107in}{1.736546in}}%
\pgfpathcurveto{\pgfqpoint{2.647107in}{1.728310in}}{\pgfqpoint{2.650379in}{1.720409in}}{\pgfqpoint{2.656203in}{1.714586in}}%
\pgfpathcurveto{\pgfqpoint{2.662027in}{1.708762in}}{\pgfqpoint{2.669927in}{1.705489in}}{\pgfqpoint{2.678163in}{1.705489in}}%
\pgfpathclose%
\pgfusepath{stroke,fill}%
\end{pgfscope}%
\begin{pgfscope}%
\pgfpathrectangle{\pgfqpoint{0.100000in}{0.212622in}}{\pgfqpoint{3.696000in}{3.696000in}}%
\pgfusepath{clip}%
\pgfsetbuttcap%
\pgfsetroundjoin%
\definecolor{currentfill}{rgb}{0.121569,0.466667,0.705882}%
\pgfsetfillcolor{currentfill}%
\pgfsetfillopacity{0.889091}%
\pgfsetlinewidth{1.003750pt}%
\definecolor{currentstroke}{rgb}{0.121569,0.466667,0.705882}%
\pgfsetstrokecolor{currentstroke}%
\pgfsetstrokeopacity{0.889091}%
\pgfsetdash{}{0pt}%
\pgfpathmoveto{\pgfqpoint{1.022118in}{2.288140in}}%
\pgfpathcurveto{\pgfqpoint{1.030354in}{2.288140in}}{\pgfqpoint{1.038254in}{2.291412in}}{\pgfqpoint{1.044078in}{2.297236in}}%
\pgfpathcurveto{\pgfqpoint{1.049902in}{2.303060in}}{\pgfqpoint{1.053175in}{2.310960in}}{\pgfqpoint{1.053175in}{2.319196in}}%
\pgfpathcurveto{\pgfqpoint{1.053175in}{2.327433in}}{\pgfqpoint{1.049902in}{2.335333in}}{\pgfqpoint{1.044078in}{2.341157in}}%
\pgfpathcurveto{\pgfqpoint{1.038254in}{2.346981in}}{\pgfqpoint{1.030354in}{2.350253in}}{\pgfqpoint{1.022118in}{2.350253in}}%
\pgfpathcurveto{\pgfqpoint{1.013882in}{2.350253in}}{\pgfqpoint{1.005982in}{2.346981in}}{\pgfqpoint{1.000158in}{2.341157in}}%
\pgfpathcurveto{\pgfqpoint{0.994334in}{2.335333in}}{\pgfqpoint{0.991062in}{2.327433in}}{\pgfqpoint{0.991062in}{2.319196in}}%
\pgfpathcurveto{\pgfqpoint{0.991062in}{2.310960in}}{\pgfqpoint{0.994334in}{2.303060in}}{\pgfqpoint{1.000158in}{2.297236in}}%
\pgfpathcurveto{\pgfqpoint{1.005982in}{2.291412in}}{\pgfqpoint{1.013882in}{2.288140in}}{\pgfqpoint{1.022118in}{2.288140in}}%
\pgfpathclose%
\pgfusepath{stroke,fill}%
\end{pgfscope}%
\begin{pgfscope}%
\pgfpathrectangle{\pgfqpoint{0.100000in}{0.212622in}}{\pgfqpoint{3.696000in}{3.696000in}}%
\pgfusepath{clip}%
\pgfsetbuttcap%
\pgfsetroundjoin%
\definecolor{currentfill}{rgb}{0.121569,0.466667,0.705882}%
\pgfsetfillcolor{currentfill}%
\pgfsetfillopacity{0.891037}%
\pgfsetlinewidth{1.003750pt}%
\definecolor{currentstroke}{rgb}{0.121569,0.466667,0.705882}%
\pgfsetstrokecolor{currentstroke}%
\pgfsetstrokeopacity{0.891037}%
\pgfsetdash{}{0pt}%
\pgfpathmoveto{\pgfqpoint{2.673233in}{1.701400in}}%
\pgfpathcurveto{\pgfqpoint{2.681469in}{1.701400in}}{\pgfqpoint{2.689369in}{1.704672in}}{\pgfqpoint{2.695193in}{1.710496in}}%
\pgfpathcurveto{\pgfqpoint{2.701017in}{1.716320in}}{\pgfqpoint{2.704289in}{1.724220in}}{\pgfqpoint{2.704289in}{1.732456in}}%
\pgfpathcurveto{\pgfqpoint{2.704289in}{1.740692in}}{\pgfqpoint{2.701017in}{1.748592in}}{\pgfqpoint{2.695193in}{1.754416in}}%
\pgfpathcurveto{\pgfqpoint{2.689369in}{1.760240in}}{\pgfqpoint{2.681469in}{1.763513in}}{\pgfqpoint{2.673233in}{1.763513in}}%
\pgfpathcurveto{\pgfqpoint{2.664997in}{1.763513in}}{\pgfqpoint{2.657097in}{1.760240in}}{\pgfqpoint{2.651273in}{1.754416in}}%
\pgfpathcurveto{\pgfqpoint{2.645449in}{1.748592in}}{\pgfqpoint{2.642176in}{1.740692in}}{\pgfqpoint{2.642176in}{1.732456in}}%
\pgfpathcurveto{\pgfqpoint{2.642176in}{1.724220in}}{\pgfqpoint{2.645449in}{1.716320in}}{\pgfqpoint{2.651273in}{1.710496in}}%
\pgfpathcurveto{\pgfqpoint{2.657097in}{1.704672in}}{\pgfqpoint{2.664997in}{1.701400in}}{\pgfqpoint{2.673233in}{1.701400in}}%
\pgfpathclose%
\pgfusepath{stroke,fill}%
\end{pgfscope}%
\begin{pgfscope}%
\pgfpathrectangle{\pgfqpoint{0.100000in}{0.212622in}}{\pgfqpoint{3.696000in}{3.696000in}}%
\pgfusepath{clip}%
\pgfsetbuttcap%
\pgfsetroundjoin%
\definecolor{currentfill}{rgb}{0.121569,0.466667,0.705882}%
\pgfsetfillcolor{currentfill}%
\pgfsetfillopacity{0.891315}%
\pgfsetlinewidth{1.003750pt}%
\definecolor{currentstroke}{rgb}{0.121569,0.466667,0.705882}%
\pgfsetstrokecolor{currentstroke}%
\pgfsetstrokeopacity{0.891315}%
\pgfsetdash{}{0pt}%
\pgfpathmoveto{\pgfqpoint{1.041166in}{2.283106in}}%
\pgfpathcurveto{\pgfqpoint{1.049403in}{2.283106in}}{\pgfqpoint{1.057303in}{2.286378in}}{\pgfqpoint{1.063127in}{2.292202in}}%
\pgfpathcurveto{\pgfqpoint{1.068950in}{2.298026in}}{\pgfqpoint{1.072223in}{2.305926in}}{\pgfqpoint{1.072223in}{2.314162in}}%
\pgfpathcurveto{\pgfqpoint{1.072223in}{2.322398in}}{\pgfqpoint{1.068950in}{2.330299in}}{\pgfqpoint{1.063127in}{2.336122in}}%
\pgfpathcurveto{\pgfqpoint{1.057303in}{2.341946in}}{\pgfqpoint{1.049403in}{2.345219in}}{\pgfqpoint{1.041166in}{2.345219in}}%
\pgfpathcurveto{\pgfqpoint{1.032930in}{2.345219in}}{\pgfqpoint{1.025030in}{2.341946in}}{\pgfqpoint{1.019206in}{2.336122in}}%
\pgfpathcurveto{\pgfqpoint{1.013382in}{2.330299in}}{\pgfqpoint{1.010110in}{2.322398in}}{\pgfqpoint{1.010110in}{2.314162in}}%
\pgfpathcurveto{\pgfqpoint{1.010110in}{2.305926in}}{\pgfqpoint{1.013382in}{2.298026in}}{\pgfqpoint{1.019206in}{2.292202in}}%
\pgfpathcurveto{\pgfqpoint{1.025030in}{2.286378in}}{\pgfqpoint{1.032930in}{2.283106in}}{\pgfqpoint{1.041166in}{2.283106in}}%
\pgfpathclose%
\pgfusepath{stroke,fill}%
\end{pgfscope}%
\begin{pgfscope}%
\pgfpathrectangle{\pgfqpoint{0.100000in}{0.212622in}}{\pgfqpoint{3.696000in}{3.696000in}}%
\pgfusepath{clip}%
\pgfsetbuttcap%
\pgfsetroundjoin%
\definecolor{currentfill}{rgb}{0.121569,0.466667,0.705882}%
\pgfsetfillcolor{currentfill}%
\pgfsetfillopacity{0.891605}%
\pgfsetlinewidth{1.003750pt}%
\definecolor{currentstroke}{rgb}{0.121569,0.466667,0.705882}%
\pgfsetstrokecolor{currentstroke}%
\pgfsetstrokeopacity{0.891605}%
\pgfsetdash{}{0pt}%
\pgfpathmoveto{\pgfqpoint{1.057949in}{2.266147in}}%
\pgfpathcurveto{\pgfqpoint{1.066185in}{2.266147in}}{\pgfqpoint{1.074085in}{2.269419in}}{\pgfqpoint{1.079909in}{2.275243in}}%
\pgfpathcurveto{\pgfqpoint{1.085733in}{2.281067in}}{\pgfqpoint{1.089005in}{2.288967in}}{\pgfqpoint{1.089005in}{2.297203in}}%
\pgfpathcurveto{\pgfqpoint{1.089005in}{2.305439in}}{\pgfqpoint{1.085733in}{2.313339in}}{\pgfqpoint{1.079909in}{2.319163in}}%
\pgfpathcurveto{\pgfqpoint{1.074085in}{2.324987in}}{\pgfqpoint{1.066185in}{2.328260in}}{\pgfqpoint{1.057949in}{2.328260in}}%
\pgfpathcurveto{\pgfqpoint{1.049712in}{2.328260in}}{\pgfqpoint{1.041812in}{2.324987in}}{\pgfqpoint{1.035988in}{2.319163in}}%
\pgfpathcurveto{\pgfqpoint{1.030164in}{2.313339in}}{\pgfqpoint{1.026892in}{2.305439in}}{\pgfqpoint{1.026892in}{2.297203in}}%
\pgfpathcurveto{\pgfqpoint{1.026892in}{2.288967in}}{\pgfqpoint{1.030164in}{2.281067in}}{\pgfqpoint{1.035988in}{2.275243in}}%
\pgfpathcurveto{\pgfqpoint{1.041812in}{2.269419in}}{\pgfqpoint{1.049712in}{2.266147in}}{\pgfqpoint{1.057949in}{2.266147in}}%
\pgfpathclose%
\pgfusepath{stroke,fill}%
\end{pgfscope}%
\begin{pgfscope}%
\pgfpathrectangle{\pgfqpoint{0.100000in}{0.212622in}}{\pgfqpoint{3.696000in}{3.696000in}}%
\pgfusepath{clip}%
\pgfsetbuttcap%
\pgfsetroundjoin%
\definecolor{currentfill}{rgb}{0.121569,0.466667,0.705882}%
\pgfsetfillcolor{currentfill}%
\pgfsetfillopacity{0.893064}%
\pgfsetlinewidth{1.003750pt}%
\definecolor{currentstroke}{rgb}{0.121569,0.466667,0.705882}%
\pgfsetstrokecolor{currentstroke}%
\pgfsetstrokeopacity{0.893064}%
\pgfsetdash{}{0pt}%
\pgfpathmoveto{\pgfqpoint{1.073738in}{2.263631in}}%
\pgfpathcurveto{\pgfqpoint{1.081975in}{2.263631in}}{\pgfqpoint{1.089875in}{2.266904in}}{\pgfqpoint{1.095699in}{2.272728in}}%
\pgfpathcurveto{\pgfqpoint{1.101523in}{2.278551in}}{\pgfqpoint{1.104795in}{2.286452in}}{\pgfqpoint{1.104795in}{2.294688in}}%
\pgfpathcurveto{\pgfqpoint{1.104795in}{2.302924in}}{\pgfqpoint{1.101523in}{2.310824in}}{\pgfqpoint{1.095699in}{2.316648in}}%
\pgfpathcurveto{\pgfqpoint{1.089875in}{2.322472in}}{\pgfqpoint{1.081975in}{2.325744in}}{\pgfqpoint{1.073738in}{2.325744in}}%
\pgfpathcurveto{\pgfqpoint{1.065502in}{2.325744in}}{\pgfqpoint{1.057602in}{2.322472in}}{\pgfqpoint{1.051778in}{2.316648in}}%
\pgfpathcurveto{\pgfqpoint{1.045954in}{2.310824in}}{\pgfqpoint{1.042682in}{2.302924in}}{\pgfqpoint{1.042682in}{2.294688in}}%
\pgfpathcurveto{\pgfqpoint{1.042682in}{2.286452in}}{\pgfqpoint{1.045954in}{2.278551in}}{\pgfqpoint{1.051778in}{2.272728in}}%
\pgfpathcurveto{\pgfqpoint{1.057602in}{2.266904in}}{\pgfqpoint{1.065502in}{2.263631in}}{\pgfqpoint{1.073738in}{2.263631in}}%
\pgfpathclose%
\pgfusepath{stroke,fill}%
\end{pgfscope}%
\begin{pgfscope}%
\pgfpathrectangle{\pgfqpoint{0.100000in}{0.212622in}}{\pgfqpoint{3.696000in}{3.696000in}}%
\pgfusepath{clip}%
\pgfsetbuttcap%
\pgfsetroundjoin%
\definecolor{currentfill}{rgb}{0.121569,0.466667,0.705882}%
\pgfsetfillcolor{currentfill}%
\pgfsetfillopacity{0.893197}%
\pgfsetlinewidth{1.003750pt}%
\definecolor{currentstroke}{rgb}{0.121569,0.466667,0.705882}%
\pgfsetstrokecolor{currentstroke}%
\pgfsetstrokeopacity{0.893197}%
\pgfsetdash{}{0pt}%
\pgfpathmoveto{\pgfqpoint{2.665868in}{1.695260in}}%
\pgfpathcurveto{\pgfqpoint{2.674105in}{1.695260in}}{\pgfqpoint{2.682005in}{1.698532in}}{\pgfqpoint{2.687829in}{1.704356in}}%
\pgfpathcurveto{\pgfqpoint{2.693652in}{1.710180in}}{\pgfqpoint{2.696925in}{1.718080in}}{\pgfqpoint{2.696925in}{1.726316in}}%
\pgfpathcurveto{\pgfqpoint{2.696925in}{1.734553in}}{\pgfqpoint{2.693652in}{1.742453in}}{\pgfqpoint{2.687829in}{1.748277in}}%
\pgfpathcurveto{\pgfqpoint{2.682005in}{1.754101in}}{\pgfqpoint{2.674105in}{1.757373in}}{\pgfqpoint{2.665868in}{1.757373in}}%
\pgfpathcurveto{\pgfqpoint{2.657632in}{1.757373in}}{\pgfqpoint{2.649732in}{1.754101in}}{\pgfqpoint{2.643908in}{1.748277in}}%
\pgfpathcurveto{\pgfqpoint{2.638084in}{1.742453in}}{\pgfqpoint{2.634812in}{1.734553in}}{\pgfqpoint{2.634812in}{1.726316in}}%
\pgfpathcurveto{\pgfqpoint{2.634812in}{1.718080in}}{\pgfqpoint{2.638084in}{1.710180in}}{\pgfqpoint{2.643908in}{1.704356in}}%
\pgfpathcurveto{\pgfqpoint{2.649732in}{1.698532in}}{\pgfqpoint{2.657632in}{1.695260in}}{\pgfqpoint{2.665868in}{1.695260in}}%
\pgfpathclose%
\pgfusepath{stroke,fill}%
\end{pgfscope}%
\begin{pgfscope}%
\pgfpathrectangle{\pgfqpoint{0.100000in}{0.212622in}}{\pgfqpoint{3.696000in}{3.696000in}}%
\pgfusepath{clip}%
\pgfsetbuttcap%
\pgfsetroundjoin%
\definecolor{currentfill}{rgb}{0.121569,0.466667,0.705882}%
\pgfsetfillcolor{currentfill}%
\pgfsetfillopacity{0.894405}%
\pgfsetlinewidth{1.003750pt}%
\definecolor{currentstroke}{rgb}{0.121569,0.466667,0.705882}%
\pgfsetstrokecolor{currentstroke}%
\pgfsetstrokeopacity{0.894405}%
\pgfsetdash{}{0pt}%
\pgfpathmoveto{\pgfqpoint{1.100318in}{2.240531in}}%
\pgfpathcurveto{\pgfqpoint{1.108554in}{2.240531in}}{\pgfqpoint{1.116454in}{2.243804in}}{\pgfqpoint{1.122278in}{2.249627in}}%
\pgfpathcurveto{\pgfqpoint{1.128102in}{2.255451in}}{\pgfqpoint{1.131375in}{2.263351in}}{\pgfqpoint{1.131375in}{2.271588in}}%
\pgfpathcurveto{\pgfqpoint{1.131375in}{2.279824in}}{\pgfqpoint{1.128102in}{2.287724in}}{\pgfqpoint{1.122278in}{2.293548in}}%
\pgfpathcurveto{\pgfqpoint{1.116454in}{2.299372in}}{\pgfqpoint{1.108554in}{2.302644in}}{\pgfqpoint{1.100318in}{2.302644in}}%
\pgfpathcurveto{\pgfqpoint{1.092082in}{2.302644in}}{\pgfqpoint{1.084182in}{2.299372in}}{\pgfqpoint{1.078358in}{2.293548in}}%
\pgfpathcurveto{\pgfqpoint{1.072534in}{2.287724in}}{\pgfqpoint{1.069262in}{2.279824in}}{\pgfqpoint{1.069262in}{2.271588in}}%
\pgfpathcurveto{\pgfqpoint{1.069262in}{2.263351in}}{\pgfqpoint{1.072534in}{2.255451in}}{\pgfqpoint{1.078358in}{2.249627in}}%
\pgfpathcurveto{\pgfqpoint{1.084182in}{2.243804in}}{\pgfqpoint{1.092082in}{2.240531in}}{\pgfqpoint{1.100318in}{2.240531in}}%
\pgfpathclose%
\pgfusepath{stroke,fill}%
\end{pgfscope}%
\begin{pgfscope}%
\pgfpathrectangle{\pgfqpoint{0.100000in}{0.212622in}}{\pgfqpoint{3.696000in}{3.696000in}}%
\pgfusepath{clip}%
\pgfsetbuttcap%
\pgfsetroundjoin%
\definecolor{currentfill}{rgb}{0.121569,0.466667,0.705882}%
\pgfsetfillcolor{currentfill}%
\pgfsetfillopacity{0.895914}%
\pgfsetlinewidth{1.003750pt}%
\definecolor{currentstroke}{rgb}{0.121569,0.466667,0.705882}%
\pgfsetstrokecolor{currentstroke}%
\pgfsetstrokeopacity{0.895914}%
\pgfsetdash{}{0pt}%
\pgfpathmoveto{\pgfqpoint{1.125781in}{2.221957in}}%
\pgfpathcurveto{\pgfqpoint{1.134018in}{2.221957in}}{\pgfqpoint{1.141918in}{2.225229in}}{\pgfqpoint{1.147742in}{2.231053in}}%
\pgfpathcurveto{\pgfqpoint{1.153566in}{2.236877in}}{\pgfqpoint{1.156838in}{2.244777in}}{\pgfqpoint{1.156838in}{2.253013in}}%
\pgfpathcurveto{\pgfqpoint{1.156838in}{2.261250in}}{\pgfqpoint{1.153566in}{2.269150in}}{\pgfqpoint{1.147742in}{2.274974in}}%
\pgfpathcurveto{\pgfqpoint{1.141918in}{2.280798in}}{\pgfqpoint{1.134018in}{2.284070in}}{\pgfqpoint{1.125781in}{2.284070in}}%
\pgfpathcurveto{\pgfqpoint{1.117545in}{2.284070in}}{\pgfqpoint{1.109645in}{2.280798in}}{\pgfqpoint{1.103821in}{2.274974in}}%
\pgfpathcurveto{\pgfqpoint{1.097997in}{2.269150in}}{\pgfqpoint{1.094725in}{2.261250in}}{\pgfqpoint{1.094725in}{2.253013in}}%
\pgfpathcurveto{\pgfqpoint{1.094725in}{2.244777in}}{\pgfqpoint{1.097997in}{2.236877in}}{\pgfqpoint{1.103821in}{2.231053in}}%
\pgfpathcurveto{\pgfqpoint{1.109645in}{2.225229in}}{\pgfqpoint{1.117545in}{2.221957in}}{\pgfqpoint{1.125781in}{2.221957in}}%
\pgfpathclose%
\pgfusepath{stroke,fill}%
\end{pgfscope}%
\begin{pgfscope}%
\pgfpathrectangle{\pgfqpoint{0.100000in}{0.212622in}}{\pgfqpoint{3.696000in}{3.696000in}}%
\pgfusepath{clip}%
\pgfsetbuttcap%
\pgfsetroundjoin%
\definecolor{currentfill}{rgb}{0.121569,0.466667,0.705882}%
\pgfsetfillcolor{currentfill}%
\pgfsetfillopacity{0.896280}%
\pgfsetlinewidth{1.003750pt}%
\definecolor{currentstroke}{rgb}{0.121569,0.466667,0.705882}%
\pgfsetstrokecolor{currentstroke}%
\pgfsetstrokeopacity{0.896280}%
\pgfsetdash{}{0pt}%
\pgfpathmoveto{\pgfqpoint{2.654707in}{1.695277in}}%
\pgfpathcurveto{\pgfqpoint{2.662943in}{1.695277in}}{\pgfqpoint{2.670843in}{1.698550in}}{\pgfqpoint{2.676667in}{1.704373in}}%
\pgfpathcurveto{\pgfqpoint{2.682491in}{1.710197in}}{\pgfqpoint{2.685763in}{1.718097in}}{\pgfqpoint{2.685763in}{1.726334in}}%
\pgfpathcurveto{\pgfqpoint{2.685763in}{1.734570in}}{\pgfqpoint{2.682491in}{1.742470in}}{\pgfqpoint{2.676667in}{1.748294in}}%
\pgfpathcurveto{\pgfqpoint{2.670843in}{1.754118in}}{\pgfqpoint{2.662943in}{1.757390in}}{\pgfqpoint{2.654707in}{1.757390in}}%
\pgfpathcurveto{\pgfqpoint{2.646470in}{1.757390in}}{\pgfqpoint{2.638570in}{1.754118in}}{\pgfqpoint{2.632746in}{1.748294in}}%
\pgfpathcurveto{\pgfqpoint{2.626922in}{1.742470in}}{\pgfqpoint{2.623650in}{1.734570in}}{\pgfqpoint{2.623650in}{1.726334in}}%
\pgfpathcurveto{\pgfqpoint{2.623650in}{1.718097in}}{\pgfqpoint{2.626922in}{1.710197in}}{\pgfqpoint{2.632746in}{1.704373in}}%
\pgfpathcurveto{\pgfqpoint{2.638570in}{1.698550in}}{\pgfqpoint{2.646470in}{1.695277in}}{\pgfqpoint{2.654707in}{1.695277in}}%
\pgfpathclose%
\pgfusepath{stroke,fill}%
\end{pgfscope}%
\begin{pgfscope}%
\pgfpathrectangle{\pgfqpoint{0.100000in}{0.212622in}}{\pgfqpoint{3.696000in}{3.696000in}}%
\pgfusepath{clip}%
\pgfsetbuttcap%
\pgfsetroundjoin%
\definecolor{currentfill}{rgb}{0.121569,0.466667,0.705882}%
\pgfsetfillcolor{currentfill}%
\pgfsetfillopacity{0.897367}%
\pgfsetlinewidth{1.003750pt}%
\definecolor{currentstroke}{rgb}{0.121569,0.466667,0.705882}%
\pgfsetstrokecolor{currentstroke}%
\pgfsetstrokeopacity{0.897367}%
\pgfsetdash{}{0pt}%
\pgfpathmoveto{\pgfqpoint{2.651181in}{1.687134in}}%
\pgfpathcurveto{\pgfqpoint{2.659417in}{1.687134in}}{\pgfqpoint{2.667317in}{1.690407in}}{\pgfqpoint{2.673141in}{1.696231in}}%
\pgfpathcurveto{\pgfqpoint{2.678965in}{1.702055in}}{\pgfqpoint{2.682237in}{1.709955in}}{\pgfqpoint{2.682237in}{1.718191in}}%
\pgfpathcurveto{\pgfqpoint{2.682237in}{1.726427in}}{\pgfqpoint{2.678965in}{1.734327in}}{\pgfqpoint{2.673141in}{1.740151in}}%
\pgfpathcurveto{\pgfqpoint{2.667317in}{1.745975in}}{\pgfqpoint{2.659417in}{1.749247in}}{\pgfqpoint{2.651181in}{1.749247in}}%
\pgfpathcurveto{\pgfqpoint{2.642944in}{1.749247in}}{\pgfqpoint{2.635044in}{1.745975in}}{\pgfqpoint{2.629220in}{1.740151in}}%
\pgfpathcurveto{\pgfqpoint{2.623396in}{1.734327in}}{\pgfqpoint{2.620124in}{1.726427in}}{\pgfqpoint{2.620124in}{1.718191in}}%
\pgfpathcurveto{\pgfqpoint{2.620124in}{1.709955in}}{\pgfqpoint{2.623396in}{1.702055in}}{\pgfqpoint{2.629220in}{1.696231in}}%
\pgfpathcurveto{\pgfqpoint{2.635044in}{1.690407in}}{\pgfqpoint{2.642944in}{1.687134in}}{\pgfqpoint{2.651181in}{1.687134in}}%
\pgfpathclose%
\pgfusepath{stroke,fill}%
\end{pgfscope}%
\begin{pgfscope}%
\pgfpathrectangle{\pgfqpoint{0.100000in}{0.212622in}}{\pgfqpoint{3.696000in}{3.696000in}}%
\pgfusepath{clip}%
\pgfsetbuttcap%
\pgfsetroundjoin%
\definecolor{currentfill}{rgb}{0.121569,0.466667,0.705882}%
\pgfsetfillcolor{currentfill}%
\pgfsetfillopacity{0.897426}%
\pgfsetlinewidth{1.003750pt}%
\definecolor{currentstroke}{rgb}{0.121569,0.466667,0.705882}%
\pgfsetstrokecolor{currentstroke}%
\pgfsetstrokeopacity{0.897426}%
\pgfsetdash{}{0pt}%
\pgfpathmoveto{\pgfqpoint{1.149034in}{2.205052in}}%
\pgfpathcurveto{\pgfqpoint{1.157270in}{2.205052in}}{\pgfqpoint{1.165170in}{2.208324in}}{\pgfqpoint{1.170994in}{2.214148in}}%
\pgfpathcurveto{\pgfqpoint{1.176818in}{2.219972in}}{\pgfqpoint{1.180091in}{2.227872in}}{\pgfqpoint{1.180091in}{2.236108in}}%
\pgfpathcurveto{\pgfqpoint{1.180091in}{2.244344in}}{\pgfqpoint{1.176818in}{2.252244in}}{\pgfqpoint{1.170994in}{2.258068in}}%
\pgfpathcurveto{\pgfqpoint{1.165170in}{2.263892in}}{\pgfqpoint{1.157270in}{2.267165in}}{\pgfqpoint{1.149034in}{2.267165in}}%
\pgfpathcurveto{\pgfqpoint{1.140798in}{2.267165in}}{\pgfqpoint{1.132898in}{2.263892in}}{\pgfqpoint{1.127074in}{2.258068in}}%
\pgfpathcurveto{\pgfqpoint{1.121250in}{2.252244in}}{\pgfqpoint{1.117978in}{2.244344in}}{\pgfqpoint{1.117978in}{2.236108in}}%
\pgfpathcurveto{\pgfqpoint{1.117978in}{2.227872in}}{\pgfqpoint{1.121250in}{2.219972in}}{\pgfqpoint{1.127074in}{2.214148in}}%
\pgfpathcurveto{\pgfqpoint{1.132898in}{2.208324in}}{\pgfqpoint{1.140798in}{2.205052in}}{\pgfqpoint{1.149034in}{2.205052in}}%
\pgfpathclose%
\pgfusepath{stroke,fill}%
\end{pgfscope}%
\begin{pgfscope}%
\pgfpathrectangle{\pgfqpoint{0.100000in}{0.212622in}}{\pgfqpoint{3.696000in}{3.696000in}}%
\pgfusepath{clip}%
\pgfsetbuttcap%
\pgfsetroundjoin%
\definecolor{currentfill}{rgb}{0.121569,0.466667,0.705882}%
\pgfsetfillcolor{currentfill}%
\pgfsetfillopacity{0.898710}%
\pgfsetlinewidth{1.003750pt}%
\definecolor{currentstroke}{rgb}{0.121569,0.466667,0.705882}%
\pgfsetstrokecolor{currentstroke}%
\pgfsetstrokeopacity{0.898710}%
\pgfsetdash{}{0pt}%
\pgfpathmoveto{\pgfqpoint{1.172301in}{2.187911in}}%
\pgfpathcurveto{\pgfqpoint{1.180537in}{2.187911in}}{\pgfqpoint{1.188437in}{2.191183in}}{\pgfqpoint{1.194261in}{2.197007in}}%
\pgfpathcurveto{\pgfqpoint{1.200085in}{2.202831in}}{\pgfqpoint{1.203358in}{2.210731in}}{\pgfqpoint{1.203358in}{2.218967in}}%
\pgfpathcurveto{\pgfqpoint{1.203358in}{2.227204in}}{\pgfqpoint{1.200085in}{2.235104in}}{\pgfqpoint{1.194261in}{2.240928in}}%
\pgfpathcurveto{\pgfqpoint{1.188437in}{2.246752in}}{\pgfqpoint{1.180537in}{2.250024in}}{\pgfqpoint{1.172301in}{2.250024in}}%
\pgfpathcurveto{\pgfqpoint{1.164065in}{2.250024in}}{\pgfqpoint{1.156165in}{2.246752in}}{\pgfqpoint{1.150341in}{2.240928in}}%
\pgfpathcurveto{\pgfqpoint{1.144517in}{2.235104in}}{\pgfqpoint{1.141245in}{2.227204in}}{\pgfqpoint{1.141245in}{2.218967in}}%
\pgfpathcurveto{\pgfqpoint{1.141245in}{2.210731in}}{\pgfqpoint{1.144517in}{2.202831in}}{\pgfqpoint{1.150341in}{2.197007in}}%
\pgfpathcurveto{\pgfqpoint{1.156165in}{2.191183in}}{\pgfqpoint{1.164065in}{2.187911in}}{\pgfqpoint{1.172301in}{2.187911in}}%
\pgfpathclose%
\pgfusepath{stroke,fill}%
\end{pgfscope}%
\begin{pgfscope}%
\pgfpathrectangle{\pgfqpoint{0.100000in}{0.212622in}}{\pgfqpoint{3.696000in}{3.696000in}}%
\pgfusepath{clip}%
\pgfsetbuttcap%
\pgfsetroundjoin%
\definecolor{currentfill}{rgb}{0.121569,0.466667,0.705882}%
\pgfsetfillcolor{currentfill}%
\pgfsetfillopacity{0.899489}%
\pgfsetlinewidth{1.003750pt}%
\definecolor{currentstroke}{rgb}{0.121569,0.466667,0.705882}%
\pgfsetstrokecolor{currentstroke}%
\pgfsetstrokeopacity{0.899489}%
\pgfsetdash{}{0pt}%
\pgfpathmoveto{\pgfqpoint{2.648237in}{1.683840in}}%
\pgfpathcurveto{\pgfqpoint{2.656473in}{1.683840in}}{\pgfqpoint{2.664374in}{1.687112in}}{\pgfqpoint{2.670197in}{1.692936in}}%
\pgfpathcurveto{\pgfqpoint{2.676021in}{1.698760in}}{\pgfqpoint{2.679294in}{1.706660in}}{\pgfqpoint{2.679294in}{1.714896in}}%
\pgfpathcurveto{\pgfqpoint{2.679294in}{1.723132in}}{\pgfqpoint{2.676021in}{1.731032in}}{\pgfqpoint{2.670197in}{1.736856in}}%
\pgfpathcurveto{\pgfqpoint{2.664374in}{1.742680in}}{\pgfqpoint{2.656473in}{1.745953in}}{\pgfqpoint{2.648237in}{1.745953in}}%
\pgfpathcurveto{\pgfqpoint{2.640001in}{1.745953in}}{\pgfqpoint{2.632101in}{1.742680in}}{\pgfqpoint{2.626277in}{1.736856in}}%
\pgfpathcurveto{\pgfqpoint{2.620453in}{1.731032in}}{\pgfqpoint{2.617181in}{1.723132in}}{\pgfqpoint{2.617181in}{1.714896in}}%
\pgfpathcurveto{\pgfqpoint{2.617181in}{1.706660in}}{\pgfqpoint{2.620453in}{1.698760in}}{\pgfqpoint{2.626277in}{1.692936in}}%
\pgfpathcurveto{\pgfqpoint{2.632101in}{1.687112in}}{\pgfqpoint{2.640001in}{1.683840in}}{\pgfqpoint{2.648237in}{1.683840in}}%
\pgfpathclose%
\pgfusepath{stroke,fill}%
\end{pgfscope}%
\begin{pgfscope}%
\pgfpathrectangle{\pgfqpoint{0.100000in}{0.212622in}}{\pgfqpoint{3.696000in}{3.696000in}}%
\pgfusepath{clip}%
\pgfsetbuttcap%
\pgfsetroundjoin%
\definecolor{currentfill}{rgb}{0.121569,0.466667,0.705882}%
\pgfsetfillcolor{currentfill}%
\pgfsetfillopacity{0.900586}%
\pgfsetlinewidth{1.003750pt}%
\definecolor{currentstroke}{rgb}{0.121569,0.466667,0.705882}%
\pgfsetstrokecolor{currentstroke}%
\pgfsetstrokeopacity{0.900586}%
\pgfsetdash{}{0pt}%
\pgfpathmoveto{\pgfqpoint{1.195861in}{2.178205in}}%
\pgfpathcurveto{\pgfqpoint{1.204097in}{2.178205in}}{\pgfqpoint{1.211998in}{2.181477in}}{\pgfqpoint{1.217821in}{2.187301in}}%
\pgfpathcurveto{\pgfqpoint{1.223645in}{2.193125in}}{\pgfqpoint{1.226918in}{2.201025in}}{\pgfqpoint{1.226918in}{2.209261in}}%
\pgfpathcurveto{\pgfqpoint{1.226918in}{2.217498in}}{\pgfqpoint{1.223645in}{2.225398in}}{\pgfqpoint{1.217821in}{2.231222in}}%
\pgfpathcurveto{\pgfqpoint{1.211998in}{2.237045in}}{\pgfqpoint{1.204097in}{2.240318in}}{\pgfqpoint{1.195861in}{2.240318in}}%
\pgfpathcurveto{\pgfqpoint{1.187625in}{2.240318in}}{\pgfqpoint{1.179725in}{2.237045in}}{\pgfqpoint{1.173901in}{2.231222in}}%
\pgfpathcurveto{\pgfqpoint{1.168077in}{2.225398in}}{\pgfqpoint{1.164805in}{2.217498in}}{\pgfqpoint{1.164805in}{2.209261in}}%
\pgfpathcurveto{\pgfqpoint{1.164805in}{2.201025in}}{\pgfqpoint{1.168077in}{2.193125in}}{\pgfqpoint{1.173901in}{2.187301in}}%
\pgfpathcurveto{\pgfqpoint{1.179725in}{2.181477in}}{\pgfqpoint{1.187625in}{2.178205in}}{\pgfqpoint{1.195861in}{2.178205in}}%
\pgfpathclose%
\pgfusepath{stroke,fill}%
\end{pgfscope}%
\begin{pgfscope}%
\pgfpathrectangle{\pgfqpoint{0.100000in}{0.212622in}}{\pgfqpoint{3.696000in}{3.696000in}}%
\pgfusepath{clip}%
\pgfsetbuttcap%
\pgfsetroundjoin%
\definecolor{currentfill}{rgb}{0.121569,0.466667,0.705882}%
\pgfsetfillcolor{currentfill}%
\pgfsetfillopacity{0.901821}%
\pgfsetlinewidth{1.003750pt}%
\definecolor{currentstroke}{rgb}{0.121569,0.466667,0.705882}%
\pgfsetstrokecolor{currentstroke}%
\pgfsetstrokeopacity{0.901821}%
\pgfsetdash{}{0pt}%
\pgfpathmoveto{\pgfqpoint{1.217295in}{2.167088in}}%
\pgfpathcurveto{\pgfqpoint{1.225531in}{2.167088in}}{\pgfqpoint{1.233431in}{2.170360in}}{\pgfqpoint{1.239255in}{2.176184in}}%
\pgfpathcurveto{\pgfqpoint{1.245079in}{2.182008in}}{\pgfqpoint{1.248351in}{2.189908in}}{\pgfqpoint{1.248351in}{2.198145in}}%
\pgfpathcurveto{\pgfqpoint{1.248351in}{2.206381in}}{\pgfqpoint{1.245079in}{2.214281in}}{\pgfqpoint{1.239255in}{2.220105in}}%
\pgfpathcurveto{\pgfqpoint{1.233431in}{2.225929in}}{\pgfqpoint{1.225531in}{2.229201in}}{\pgfqpoint{1.217295in}{2.229201in}}%
\pgfpathcurveto{\pgfqpoint{1.209059in}{2.229201in}}{\pgfqpoint{1.201159in}{2.225929in}}{\pgfqpoint{1.195335in}{2.220105in}}%
\pgfpathcurveto{\pgfqpoint{1.189511in}{2.214281in}}{\pgfqpoint{1.186238in}{2.206381in}}{\pgfqpoint{1.186238in}{2.198145in}}%
\pgfpathcurveto{\pgfqpoint{1.186238in}{2.189908in}}{\pgfqpoint{1.189511in}{2.182008in}}{\pgfqpoint{1.195335in}{2.176184in}}%
\pgfpathcurveto{\pgfqpoint{1.201159in}{2.170360in}}{\pgfqpoint{1.209059in}{2.167088in}}{\pgfqpoint{1.217295in}{2.167088in}}%
\pgfpathclose%
\pgfusepath{stroke,fill}%
\end{pgfscope}%
\begin{pgfscope}%
\pgfpathrectangle{\pgfqpoint{0.100000in}{0.212622in}}{\pgfqpoint{3.696000in}{3.696000in}}%
\pgfusepath{clip}%
\pgfsetbuttcap%
\pgfsetroundjoin%
\definecolor{currentfill}{rgb}{0.121569,0.466667,0.705882}%
\pgfsetfillcolor{currentfill}%
\pgfsetfillopacity{0.902301}%
\pgfsetlinewidth{1.003750pt}%
\definecolor{currentstroke}{rgb}{0.121569,0.466667,0.705882}%
\pgfsetstrokecolor{currentstroke}%
\pgfsetstrokeopacity{0.902301}%
\pgfsetdash{}{0pt}%
\pgfpathmoveto{\pgfqpoint{2.640100in}{1.683995in}}%
\pgfpathcurveto{\pgfqpoint{2.648336in}{1.683995in}}{\pgfqpoint{2.656236in}{1.687268in}}{\pgfqpoint{2.662060in}{1.693092in}}%
\pgfpathcurveto{\pgfqpoint{2.667884in}{1.698916in}}{\pgfqpoint{2.671156in}{1.706816in}}{\pgfqpoint{2.671156in}{1.715052in}}%
\pgfpathcurveto{\pgfqpoint{2.671156in}{1.723288in}}{\pgfqpoint{2.667884in}{1.731188in}}{\pgfqpoint{2.662060in}{1.737012in}}%
\pgfpathcurveto{\pgfqpoint{2.656236in}{1.742836in}}{\pgfqpoint{2.648336in}{1.746108in}}{\pgfqpoint{2.640100in}{1.746108in}}%
\pgfpathcurveto{\pgfqpoint{2.631863in}{1.746108in}}{\pgfqpoint{2.623963in}{1.742836in}}{\pgfqpoint{2.618139in}{1.737012in}}%
\pgfpathcurveto{\pgfqpoint{2.612315in}{1.731188in}}{\pgfqpoint{2.609043in}{1.723288in}}{\pgfqpoint{2.609043in}{1.715052in}}%
\pgfpathcurveto{\pgfqpoint{2.609043in}{1.706816in}}{\pgfqpoint{2.612315in}{1.698916in}}{\pgfqpoint{2.618139in}{1.693092in}}%
\pgfpathcurveto{\pgfqpoint{2.623963in}{1.687268in}}{\pgfqpoint{2.631863in}{1.683995in}}{\pgfqpoint{2.640100in}{1.683995in}}%
\pgfpathclose%
\pgfusepath{stroke,fill}%
\end{pgfscope}%
\begin{pgfscope}%
\pgfpathrectangle{\pgfqpoint{0.100000in}{0.212622in}}{\pgfqpoint{3.696000in}{3.696000in}}%
\pgfusepath{clip}%
\pgfsetbuttcap%
\pgfsetroundjoin%
\definecolor{currentfill}{rgb}{0.121569,0.466667,0.705882}%
\pgfsetfillcolor{currentfill}%
\pgfsetfillopacity{0.902330}%
\pgfsetlinewidth{1.003750pt}%
\definecolor{currentstroke}{rgb}{0.121569,0.466667,0.705882}%
\pgfsetstrokecolor{currentstroke}%
\pgfsetstrokeopacity{0.902330}%
\pgfsetdash{}{0pt}%
\pgfpathmoveto{\pgfqpoint{1.237321in}{2.152040in}}%
\pgfpathcurveto{\pgfqpoint{1.245558in}{2.152040in}}{\pgfqpoint{1.253458in}{2.155312in}}{\pgfqpoint{1.259282in}{2.161136in}}%
\pgfpathcurveto{\pgfqpoint{1.265106in}{2.166960in}}{\pgfqpoint{1.268378in}{2.174860in}}{\pgfqpoint{1.268378in}{2.183097in}}%
\pgfpathcurveto{\pgfqpoint{1.268378in}{2.191333in}}{\pgfqpoint{1.265106in}{2.199233in}}{\pgfqpoint{1.259282in}{2.205057in}}%
\pgfpathcurveto{\pgfqpoint{1.253458in}{2.210881in}}{\pgfqpoint{1.245558in}{2.214153in}}{\pgfqpoint{1.237321in}{2.214153in}}%
\pgfpathcurveto{\pgfqpoint{1.229085in}{2.214153in}}{\pgfqpoint{1.221185in}{2.210881in}}{\pgfqpoint{1.215361in}{2.205057in}}%
\pgfpathcurveto{\pgfqpoint{1.209537in}{2.199233in}}{\pgfqpoint{1.206265in}{2.191333in}}{\pgfqpoint{1.206265in}{2.183097in}}%
\pgfpathcurveto{\pgfqpoint{1.206265in}{2.174860in}}{\pgfqpoint{1.209537in}{2.166960in}}{\pgfqpoint{1.215361in}{2.161136in}}%
\pgfpathcurveto{\pgfqpoint{1.221185in}{2.155312in}}{\pgfqpoint{1.229085in}{2.152040in}}{\pgfqpoint{1.237321in}{2.152040in}}%
\pgfpathclose%
\pgfusepath{stroke,fill}%
\end{pgfscope}%
\begin{pgfscope}%
\pgfpathrectangle{\pgfqpoint{0.100000in}{0.212622in}}{\pgfqpoint{3.696000in}{3.696000in}}%
\pgfusepath{clip}%
\pgfsetbuttcap%
\pgfsetroundjoin%
\definecolor{currentfill}{rgb}{0.121569,0.466667,0.705882}%
\pgfsetfillcolor{currentfill}%
\pgfsetfillopacity{0.904159}%
\pgfsetlinewidth{1.003750pt}%
\definecolor{currentstroke}{rgb}{0.121569,0.466667,0.705882}%
\pgfsetstrokecolor{currentstroke}%
\pgfsetstrokeopacity{0.904159}%
\pgfsetdash{}{0pt}%
\pgfpathmoveto{\pgfqpoint{2.632200in}{1.676137in}}%
\pgfpathcurveto{\pgfqpoint{2.640437in}{1.676137in}}{\pgfqpoint{2.648337in}{1.679409in}}{\pgfqpoint{2.654161in}{1.685233in}}%
\pgfpathcurveto{\pgfqpoint{2.659985in}{1.691057in}}{\pgfqpoint{2.663257in}{1.698957in}}{\pgfqpoint{2.663257in}{1.707193in}}%
\pgfpathcurveto{\pgfqpoint{2.663257in}{1.715429in}}{\pgfqpoint{2.659985in}{1.723329in}}{\pgfqpoint{2.654161in}{1.729153in}}%
\pgfpathcurveto{\pgfqpoint{2.648337in}{1.734977in}}{\pgfqpoint{2.640437in}{1.738250in}}{\pgfqpoint{2.632200in}{1.738250in}}%
\pgfpathcurveto{\pgfqpoint{2.623964in}{1.738250in}}{\pgfqpoint{2.616064in}{1.734977in}}{\pgfqpoint{2.610240in}{1.729153in}}%
\pgfpathcurveto{\pgfqpoint{2.604416in}{1.723329in}}{\pgfqpoint{2.601144in}{1.715429in}}{\pgfqpoint{2.601144in}{1.707193in}}%
\pgfpathcurveto{\pgfqpoint{2.601144in}{1.698957in}}{\pgfqpoint{2.604416in}{1.691057in}}{\pgfqpoint{2.610240in}{1.685233in}}%
\pgfpathcurveto{\pgfqpoint{2.616064in}{1.679409in}}{\pgfqpoint{2.623964in}{1.676137in}}{\pgfqpoint{2.632200in}{1.676137in}}%
\pgfpathclose%
\pgfusepath{stroke,fill}%
\end{pgfscope}%
\begin{pgfscope}%
\pgfpathrectangle{\pgfqpoint{0.100000in}{0.212622in}}{\pgfqpoint{3.696000in}{3.696000in}}%
\pgfusepath{clip}%
\pgfsetbuttcap%
\pgfsetroundjoin%
\definecolor{currentfill}{rgb}{0.121569,0.466667,0.705882}%
\pgfsetfillcolor{currentfill}%
\pgfsetfillopacity{0.904653}%
\pgfsetlinewidth{1.003750pt}%
\definecolor{currentstroke}{rgb}{0.121569,0.466667,0.705882}%
\pgfsetstrokecolor{currentstroke}%
\pgfsetstrokeopacity{0.904653}%
\pgfsetdash{}{0pt}%
\pgfpathmoveto{\pgfqpoint{1.257592in}{2.153554in}}%
\pgfpathcurveto{\pgfqpoint{1.265828in}{2.153554in}}{\pgfqpoint{1.273728in}{2.156826in}}{\pgfqpoint{1.279552in}{2.162650in}}%
\pgfpathcurveto{\pgfqpoint{1.285376in}{2.168474in}}{\pgfqpoint{1.288648in}{2.176374in}}{\pgfqpoint{1.288648in}{2.184611in}}%
\pgfpathcurveto{\pgfqpoint{1.288648in}{2.192847in}}{\pgfqpoint{1.285376in}{2.200747in}}{\pgfqpoint{1.279552in}{2.206571in}}%
\pgfpathcurveto{\pgfqpoint{1.273728in}{2.212395in}}{\pgfqpoint{1.265828in}{2.215667in}}{\pgfqpoint{1.257592in}{2.215667in}}%
\pgfpathcurveto{\pgfqpoint{1.249355in}{2.215667in}}{\pgfqpoint{1.241455in}{2.212395in}}{\pgfqpoint{1.235631in}{2.206571in}}%
\pgfpathcurveto{\pgfqpoint{1.229807in}{2.200747in}}{\pgfqpoint{1.226535in}{2.192847in}}{\pgfqpoint{1.226535in}{2.184611in}}%
\pgfpathcurveto{\pgfqpoint{1.226535in}{2.176374in}}{\pgfqpoint{1.229807in}{2.168474in}}{\pgfqpoint{1.235631in}{2.162650in}}%
\pgfpathcurveto{\pgfqpoint{1.241455in}{2.156826in}}{\pgfqpoint{1.249355in}{2.153554in}}{\pgfqpoint{1.257592in}{2.153554in}}%
\pgfpathclose%
\pgfusepath{stroke,fill}%
\end{pgfscope}%
\begin{pgfscope}%
\pgfpathrectangle{\pgfqpoint{0.100000in}{0.212622in}}{\pgfqpoint{3.696000in}{3.696000in}}%
\pgfusepath{clip}%
\pgfsetbuttcap%
\pgfsetroundjoin%
\definecolor{currentfill}{rgb}{0.121569,0.466667,0.705882}%
\pgfsetfillcolor{currentfill}%
\pgfsetfillopacity{0.905261}%
\pgfsetlinewidth{1.003750pt}%
\definecolor{currentstroke}{rgb}{0.121569,0.466667,0.705882}%
\pgfsetstrokecolor{currentstroke}%
\pgfsetstrokeopacity{0.905261}%
\pgfsetdash{}{0pt}%
\pgfpathmoveto{\pgfqpoint{1.274620in}{2.143017in}}%
\pgfpathcurveto{\pgfqpoint{1.282856in}{2.143017in}}{\pgfqpoint{1.290756in}{2.146289in}}{\pgfqpoint{1.296580in}{2.152113in}}%
\pgfpathcurveto{\pgfqpoint{1.302404in}{2.157937in}}{\pgfqpoint{1.305676in}{2.165837in}}{\pgfqpoint{1.305676in}{2.174073in}}%
\pgfpathcurveto{\pgfqpoint{1.305676in}{2.182310in}}{\pgfqpoint{1.302404in}{2.190210in}}{\pgfqpoint{1.296580in}{2.196034in}}%
\pgfpathcurveto{\pgfqpoint{1.290756in}{2.201858in}}{\pgfqpoint{1.282856in}{2.205130in}}{\pgfqpoint{1.274620in}{2.205130in}}%
\pgfpathcurveto{\pgfqpoint{1.266384in}{2.205130in}}{\pgfqpoint{1.258484in}{2.201858in}}{\pgfqpoint{1.252660in}{2.196034in}}%
\pgfpathcurveto{\pgfqpoint{1.246836in}{2.190210in}}{\pgfqpoint{1.243563in}{2.182310in}}{\pgfqpoint{1.243563in}{2.174073in}}%
\pgfpathcurveto{\pgfqpoint{1.243563in}{2.165837in}}{\pgfqpoint{1.246836in}{2.157937in}}{\pgfqpoint{1.252660in}{2.152113in}}%
\pgfpathcurveto{\pgfqpoint{1.258484in}{2.146289in}}{\pgfqpoint{1.266384in}{2.143017in}}{\pgfqpoint{1.274620in}{2.143017in}}%
\pgfpathclose%
\pgfusepath{stroke,fill}%
\end{pgfscope}%
\begin{pgfscope}%
\pgfpathrectangle{\pgfqpoint{0.100000in}{0.212622in}}{\pgfqpoint{3.696000in}{3.696000in}}%
\pgfusepath{clip}%
\pgfsetbuttcap%
\pgfsetroundjoin%
\definecolor{currentfill}{rgb}{0.121569,0.466667,0.705882}%
\pgfsetfillcolor{currentfill}%
\pgfsetfillopacity{0.905514}%
\pgfsetlinewidth{1.003750pt}%
\definecolor{currentstroke}{rgb}{0.121569,0.466667,0.705882}%
\pgfsetstrokecolor{currentstroke}%
\pgfsetstrokeopacity{0.905514}%
\pgfsetdash{}{0pt}%
\pgfpathmoveto{\pgfqpoint{1.288892in}{2.129327in}}%
\pgfpathcurveto{\pgfqpoint{1.297128in}{2.129327in}}{\pgfqpoint{1.305028in}{2.132599in}}{\pgfqpoint{1.310852in}{2.138423in}}%
\pgfpathcurveto{\pgfqpoint{1.316676in}{2.144247in}}{\pgfqpoint{1.319949in}{2.152147in}}{\pgfqpoint{1.319949in}{2.160383in}}%
\pgfpathcurveto{\pgfqpoint{1.319949in}{2.168620in}}{\pgfqpoint{1.316676in}{2.176520in}}{\pgfqpoint{1.310852in}{2.182344in}}%
\pgfpathcurveto{\pgfqpoint{1.305028in}{2.188167in}}{\pgfqpoint{1.297128in}{2.191440in}}{\pgfqpoint{1.288892in}{2.191440in}}%
\pgfpathcurveto{\pgfqpoint{1.280656in}{2.191440in}}{\pgfqpoint{1.272756in}{2.188167in}}{\pgfqpoint{1.266932in}{2.182344in}}%
\pgfpathcurveto{\pgfqpoint{1.261108in}{2.176520in}}{\pgfqpoint{1.257836in}{2.168620in}}{\pgfqpoint{1.257836in}{2.160383in}}%
\pgfpathcurveto{\pgfqpoint{1.257836in}{2.152147in}}{\pgfqpoint{1.261108in}{2.144247in}}{\pgfqpoint{1.266932in}{2.138423in}}%
\pgfpathcurveto{\pgfqpoint{1.272756in}{2.132599in}}{\pgfqpoint{1.280656in}{2.129327in}}{\pgfqpoint{1.288892in}{2.129327in}}%
\pgfpathclose%
\pgfusepath{stroke,fill}%
\end{pgfscope}%
\begin{pgfscope}%
\pgfpathrectangle{\pgfqpoint{0.100000in}{0.212622in}}{\pgfqpoint{3.696000in}{3.696000in}}%
\pgfusepath{clip}%
\pgfsetbuttcap%
\pgfsetroundjoin%
\definecolor{currentfill}{rgb}{0.121569,0.466667,0.705882}%
\pgfsetfillcolor{currentfill}%
\pgfsetfillopacity{0.906885}%
\pgfsetlinewidth{1.003750pt}%
\definecolor{currentstroke}{rgb}{0.121569,0.466667,0.705882}%
\pgfsetstrokecolor{currentstroke}%
\pgfsetstrokeopacity{0.906885}%
\pgfsetdash{}{0pt}%
\pgfpathmoveto{\pgfqpoint{1.302739in}{2.124022in}}%
\pgfpathcurveto{\pgfqpoint{1.310975in}{2.124022in}}{\pgfqpoint{1.318875in}{2.127295in}}{\pgfqpoint{1.324699in}{2.133119in}}%
\pgfpathcurveto{\pgfqpoint{1.330523in}{2.138943in}}{\pgfqpoint{1.333795in}{2.146843in}}{\pgfqpoint{1.333795in}{2.155079in}}%
\pgfpathcurveto{\pgfqpoint{1.333795in}{2.163315in}}{\pgfqpoint{1.330523in}{2.171215in}}{\pgfqpoint{1.324699in}{2.177039in}}%
\pgfpathcurveto{\pgfqpoint{1.318875in}{2.182863in}}{\pgfqpoint{1.310975in}{2.186135in}}{\pgfqpoint{1.302739in}{2.186135in}}%
\pgfpathcurveto{\pgfqpoint{1.294503in}{2.186135in}}{\pgfqpoint{1.286603in}{2.182863in}}{\pgfqpoint{1.280779in}{2.177039in}}%
\pgfpathcurveto{\pgfqpoint{1.274955in}{2.171215in}}{\pgfqpoint{1.271682in}{2.163315in}}{\pgfqpoint{1.271682in}{2.155079in}}%
\pgfpathcurveto{\pgfqpoint{1.271682in}{2.146843in}}{\pgfqpoint{1.274955in}{2.138943in}}{\pgfqpoint{1.280779in}{2.133119in}}%
\pgfpathcurveto{\pgfqpoint{1.286603in}{2.127295in}}{\pgfqpoint{1.294503in}{2.124022in}}{\pgfqpoint{1.302739in}{2.124022in}}%
\pgfpathclose%
\pgfusepath{stroke,fill}%
\end{pgfscope}%
\begin{pgfscope}%
\pgfpathrectangle{\pgfqpoint{0.100000in}{0.212622in}}{\pgfqpoint{3.696000in}{3.696000in}}%
\pgfusepath{clip}%
\pgfsetbuttcap%
\pgfsetroundjoin%
\definecolor{currentfill}{rgb}{0.121569,0.466667,0.705882}%
\pgfsetfillcolor{currentfill}%
\pgfsetfillopacity{0.907680}%
\pgfsetlinewidth{1.003750pt}%
\definecolor{currentstroke}{rgb}{0.121569,0.466667,0.705882}%
\pgfsetstrokecolor{currentstroke}%
\pgfsetstrokeopacity{0.907680}%
\pgfsetdash{}{0pt}%
\pgfpathmoveto{\pgfqpoint{1.314960in}{2.117539in}}%
\pgfpathcurveto{\pgfqpoint{1.323196in}{2.117539in}}{\pgfqpoint{1.331096in}{2.120811in}}{\pgfqpoint{1.336920in}{2.126635in}}%
\pgfpathcurveto{\pgfqpoint{1.342744in}{2.132459in}}{\pgfqpoint{1.346017in}{2.140359in}}{\pgfqpoint{1.346017in}{2.148595in}}%
\pgfpathcurveto{\pgfqpoint{1.346017in}{2.156832in}}{\pgfqpoint{1.342744in}{2.164732in}}{\pgfqpoint{1.336920in}{2.170556in}}%
\pgfpathcurveto{\pgfqpoint{1.331096in}{2.176380in}}{\pgfqpoint{1.323196in}{2.179652in}}{\pgfqpoint{1.314960in}{2.179652in}}%
\pgfpathcurveto{\pgfqpoint{1.306724in}{2.179652in}}{\pgfqpoint{1.298824in}{2.176380in}}{\pgfqpoint{1.293000in}{2.170556in}}%
\pgfpathcurveto{\pgfqpoint{1.287176in}{2.164732in}}{\pgfqpoint{1.283904in}{2.156832in}}{\pgfqpoint{1.283904in}{2.148595in}}%
\pgfpathcurveto{\pgfqpoint{1.283904in}{2.140359in}}{\pgfqpoint{1.287176in}{2.132459in}}{\pgfqpoint{1.293000in}{2.126635in}}%
\pgfpathcurveto{\pgfqpoint{1.298824in}{2.120811in}}{\pgfqpoint{1.306724in}{2.117539in}}{\pgfqpoint{1.314960in}{2.117539in}}%
\pgfpathclose%
\pgfusepath{stroke,fill}%
\end{pgfscope}%
\begin{pgfscope}%
\pgfpathrectangle{\pgfqpoint{0.100000in}{0.212622in}}{\pgfqpoint{3.696000in}{3.696000in}}%
\pgfusepath{clip}%
\pgfsetbuttcap%
\pgfsetroundjoin%
\definecolor{currentfill}{rgb}{0.121569,0.466667,0.705882}%
\pgfsetfillcolor{currentfill}%
\pgfsetfillopacity{0.907697}%
\pgfsetlinewidth{1.003750pt}%
\definecolor{currentstroke}{rgb}{0.121569,0.466667,0.705882}%
\pgfsetstrokecolor{currentstroke}%
\pgfsetstrokeopacity{0.907697}%
\pgfsetdash{}{0pt}%
\pgfpathmoveto{\pgfqpoint{1.324083in}{2.107076in}}%
\pgfpathcurveto{\pgfqpoint{1.332319in}{2.107076in}}{\pgfqpoint{1.340219in}{2.110348in}}{\pgfqpoint{1.346043in}{2.116172in}}%
\pgfpathcurveto{\pgfqpoint{1.351867in}{2.121996in}}{\pgfqpoint{1.355139in}{2.129896in}}{\pgfqpoint{1.355139in}{2.138132in}}%
\pgfpathcurveto{\pgfqpoint{1.355139in}{2.146368in}}{\pgfqpoint{1.351867in}{2.154268in}}{\pgfqpoint{1.346043in}{2.160092in}}%
\pgfpathcurveto{\pgfqpoint{1.340219in}{2.165916in}}{\pgfqpoint{1.332319in}{2.169189in}}{\pgfqpoint{1.324083in}{2.169189in}}%
\pgfpathcurveto{\pgfqpoint{1.315847in}{2.169189in}}{\pgfqpoint{1.307946in}{2.165916in}}{\pgfqpoint{1.302123in}{2.160092in}}%
\pgfpathcurveto{\pgfqpoint{1.296299in}{2.154268in}}{\pgfqpoint{1.293026in}{2.146368in}}{\pgfqpoint{1.293026in}{2.138132in}}%
\pgfpathcurveto{\pgfqpoint{1.293026in}{2.129896in}}{\pgfqpoint{1.296299in}{2.121996in}}{\pgfqpoint{1.302123in}{2.116172in}}%
\pgfpathcurveto{\pgfqpoint{1.307946in}{2.110348in}}{\pgfqpoint{1.315847in}{2.107076in}}{\pgfqpoint{1.324083in}{2.107076in}}%
\pgfpathclose%
\pgfusepath{stroke,fill}%
\end{pgfscope}%
\begin{pgfscope}%
\pgfpathrectangle{\pgfqpoint{0.100000in}{0.212622in}}{\pgfqpoint{3.696000in}{3.696000in}}%
\pgfusepath{clip}%
\pgfsetbuttcap%
\pgfsetroundjoin%
\definecolor{currentfill}{rgb}{0.121569,0.466667,0.705882}%
\pgfsetfillcolor{currentfill}%
\pgfsetfillopacity{0.908492}%
\pgfsetlinewidth{1.003750pt}%
\definecolor{currentstroke}{rgb}{0.121569,0.466667,0.705882}%
\pgfsetstrokecolor{currentstroke}%
\pgfsetstrokeopacity{0.908492}%
\pgfsetdash{}{0pt}%
\pgfpathmoveto{\pgfqpoint{2.626499in}{1.677749in}}%
\pgfpathcurveto{\pgfqpoint{2.634735in}{1.677749in}}{\pgfqpoint{2.642636in}{1.681022in}}{\pgfqpoint{2.648459in}{1.686846in}}%
\pgfpathcurveto{\pgfqpoint{2.654283in}{1.692670in}}{\pgfqpoint{2.657556in}{1.700570in}}{\pgfqpoint{2.657556in}{1.708806in}}%
\pgfpathcurveto{\pgfqpoint{2.657556in}{1.717042in}}{\pgfqpoint{2.654283in}{1.724942in}}{\pgfqpoint{2.648459in}{1.730766in}}%
\pgfpathcurveto{\pgfqpoint{2.642636in}{1.736590in}}{\pgfqpoint{2.634735in}{1.739862in}}{\pgfqpoint{2.626499in}{1.739862in}}%
\pgfpathcurveto{\pgfqpoint{2.618263in}{1.739862in}}{\pgfqpoint{2.610363in}{1.736590in}}{\pgfqpoint{2.604539in}{1.730766in}}%
\pgfpathcurveto{\pgfqpoint{2.598715in}{1.724942in}}{\pgfqpoint{2.595443in}{1.717042in}}{\pgfqpoint{2.595443in}{1.708806in}}%
\pgfpathcurveto{\pgfqpoint{2.595443in}{1.700570in}}{\pgfqpoint{2.598715in}{1.692670in}}{\pgfqpoint{2.604539in}{1.686846in}}%
\pgfpathcurveto{\pgfqpoint{2.610363in}{1.681022in}}{\pgfqpoint{2.618263in}{1.677749in}}{\pgfqpoint{2.626499in}{1.677749in}}%
\pgfpathclose%
\pgfusepath{stroke,fill}%
\end{pgfscope}%
\begin{pgfscope}%
\pgfpathrectangle{\pgfqpoint{0.100000in}{0.212622in}}{\pgfqpoint{3.696000in}{3.696000in}}%
\pgfusepath{clip}%
\pgfsetbuttcap%
\pgfsetroundjoin%
\definecolor{currentfill}{rgb}{0.121569,0.466667,0.705882}%
\pgfsetfillcolor{currentfill}%
\pgfsetfillopacity{0.909107}%
\pgfsetlinewidth{1.003750pt}%
\definecolor{currentstroke}{rgb}{0.121569,0.466667,0.705882}%
\pgfsetstrokecolor{currentstroke}%
\pgfsetstrokeopacity{0.909107}%
\pgfsetdash{}{0pt}%
\pgfpathmoveto{\pgfqpoint{1.341251in}{2.100136in}}%
\pgfpathcurveto{\pgfqpoint{1.349487in}{2.100136in}}{\pgfqpoint{1.357387in}{2.103408in}}{\pgfqpoint{1.363211in}{2.109232in}}%
\pgfpathcurveto{\pgfqpoint{1.369035in}{2.115056in}}{\pgfqpoint{1.372307in}{2.122956in}}{\pgfqpoint{1.372307in}{2.131193in}}%
\pgfpathcurveto{\pgfqpoint{1.372307in}{2.139429in}}{\pgfqpoint{1.369035in}{2.147329in}}{\pgfqpoint{1.363211in}{2.153153in}}%
\pgfpathcurveto{\pgfqpoint{1.357387in}{2.158977in}}{\pgfqpoint{1.349487in}{2.162249in}}{\pgfqpoint{1.341251in}{2.162249in}}%
\pgfpathcurveto{\pgfqpoint{1.333014in}{2.162249in}}{\pgfqpoint{1.325114in}{2.158977in}}{\pgfqpoint{1.319291in}{2.153153in}}%
\pgfpathcurveto{\pgfqpoint{1.313467in}{2.147329in}}{\pgfqpoint{1.310194in}{2.139429in}}{\pgfqpoint{1.310194in}{2.131193in}}%
\pgfpathcurveto{\pgfqpoint{1.310194in}{2.122956in}}{\pgfqpoint{1.313467in}{2.115056in}}{\pgfqpoint{1.319291in}{2.109232in}}%
\pgfpathcurveto{\pgfqpoint{1.325114in}{2.103408in}}{\pgfqpoint{1.333014in}{2.100136in}}{\pgfqpoint{1.341251in}{2.100136in}}%
\pgfpathclose%
\pgfusepath{stroke,fill}%
\end{pgfscope}%
\begin{pgfscope}%
\pgfpathrectangle{\pgfqpoint{0.100000in}{0.212622in}}{\pgfqpoint{3.696000in}{3.696000in}}%
\pgfusepath{clip}%
\pgfsetbuttcap%
\pgfsetroundjoin%
\definecolor{currentfill}{rgb}{0.121569,0.466667,0.705882}%
\pgfsetfillcolor{currentfill}%
\pgfsetfillopacity{0.909572}%
\pgfsetlinewidth{1.003750pt}%
\definecolor{currentstroke}{rgb}{0.121569,0.466667,0.705882}%
\pgfsetstrokecolor{currentstroke}%
\pgfsetstrokeopacity{0.909572}%
\pgfsetdash{}{0pt}%
\pgfpathmoveto{\pgfqpoint{2.621237in}{1.672579in}}%
\pgfpathcurveto{\pgfqpoint{2.629473in}{1.672579in}}{\pgfqpoint{2.637373in}{1.675851in}}{\pgfqpoint{2.643197in}{1.681675in}}%
\pgfpathcurveto{\pgfqpoint{2.649021in}{1.687499in}}{\pgfqpoint{2.652294in}{1.695399in}}{\pgfqpoint{2.652294in}{1.703635in}}%
\pgfpathcurveto{\pgfqpoint{2.652294in}{1.711872in}}{\pgfqpoint{2.649021in}{1.719772in}}{\pgfqpoint{2.643197in}{1.725596in}}%
\pgfpathcurveto{\pgfqpoint{2.637373in}{1.731419in}}{\pgfqpoint{2.629473in}{1.734692in}}{\pgfqpoint{2.621237in}{1.734692in}}%
\pgfpathcurveto{\pgfqpoint{2.613001in}{1.734692in}}{\pgfqpoint{2.605101in}{1.731419in}}{\pgfqpoint{2.599277in}{1.725596in}}%
\pgfpathcurveto{\pgfqpoint{2.593453in}{1.719772in}}{\pgfqpoint{2.590181in}{1.711872in}}{\pgfqpoint{2.590181in}{1.703635in}}%
\pgfpathcurveto{\pgfqpoint{2.590181in}{1.695399in}}{\pgfqpoint{2.593453in}{1.687499in}}{\pgfqpoint{2.599277in}{1.681675in}}%
\pgfpathcurveto{\pgfqpoint{2.605101in}{1.675851in}}{\pgfqpoint{2.613001in}{1.672579in}}{\pgfqpoint{2.621237in}{1.672579in}}%
\pgfpathclose%
\pgfusepath{stroke,fill}%
\end{pgfscope}%
\begin{pgfscope}%
\pgfpathrectangle{\pgfqpoint{0.100000in}{0.212622in}}{\pgfqpoint{3.696000in}{3.696000in}}%
\pgfusepath{clip}%
\pgfsetbuttcap%
\pgfsetroundjoin%
\definecolor{currentfill}{rgb}{0.121569,0.466667,0.705882}%
\pgfsetfillcolor{currentfill}%
\pgfsetfillopacity{0.909656}%
\pgfsetlinewidth{1.003750pt}%
\definecolor{currentstroke}{rgb}{0.121569,0.466667,0.705882}%
\pgfsetstrokecolor{currentstroke}%
\pgfsetstrokeopacity{0.909656}%
\pgfsetdash{}{0pt}%
\pgfpathmoveto{\pgfqpoint{1.358479in}{2.090195in}}%
\pgfpathcurveto{\pgfqpoint{1.366716in}{2.090195in}}{\pgfqpoint{1.374616in}{2.093468in}}{\pgfqpoint{1.380439in}{2.099292in}}%
\pgfpathcurveto{\pgfqpoint{1.386263in}{2.105116in}}{\pgfqpoint{1.389536in}{2.113016in}}{\pgfqpoint{1.389536in}{2.121252in}}%
\pgfpathcurveto{\pgfqpoint{1.389536in}{2.129488in}}{\pgfqpoint{1.386263in}{2.137388in}}{\pgfqpoint{1.380439in}{2.143212in}}%
\pgfpathcurveto{\pgfqpoint{1.374616in}{2.149036in}}{\pgfqpoint{1.366716in}{2.152308in}}{\pgfqpoint{1.358479in}{2.152308in}}%
\pgfpathcurveto{\pgfqpoint{1.350243in}{2.152308in}}{\pgfqpoint{1.342343in}{2.149036in}}{\pgfqpoint{1.336519in}{2.143212in}}%
\pgfpathcurveto{\pgfqpoint{1.330695in}{2.137388in}}{\pgfqpoint{1.327423in}{2.129488in}}{\pgfqpoint{1.327423in}{2.121252in}}%
\pgfpathcurveto{\pgfqpoint{1.327423in}{2.113016in}}{\pgfqpoint{1.330695in}{2.105116in}}{\pgfqpoint{1.336519in}{2.099292in}}%
\pgfpathcurveto{\pgfqpoint{1.342343in}{2.093468in}}{\pgfqpoint{1.350243in}{2.090195in}}{\pgfqpoint{1.358479in}{2.090195in}}%
\pgfpathclose%
\pgfusepath{stroke,fill}%
\end{pgfscope}%
\begin{pgfscope}%
\pgfpathrectangle{\pgfqpoint{0.100000in}{0.212622in}}{\pgfqpoint{3.696000in}{3.696000in}}%
\pgfusepath{clip}%
\pgfsetbuttcap%
\pgfsetroundjoin%
\definecolor{currentfill}{rgb}{0.121569,0.466667,0.705882}%
\pgfsetfillcolor{currentfill}%
\pgfsetfillopacity{0.910256}%
\pgfsetlinewidth{1.003750pt}%
\definecolor{currentstroke}{rgb}{0.121569,0.466667,0.705882}%
\pgfsetstrokecolor{currentstroke}%
\pgfsetstrokeopacity{0.910256}%
\pgfsetdash{}{0pt}%
\pgfpathmoveto{\pgfqpoint{1.372050in}{2.078801in}}%
\pgfpathcurveto{\pgfqpoint{1.380286in}{2.078801in}}{\pgfqpoint{1.388186in}{2.082074in}}{\pgfqpoint{1.394010in}{2.087898in}}%
\pgfpathcurveto{\pgfqpoint{1.399834in}{2.093721in}}{\pgfqpoint{1.403106in}{2.101621in}}{\pgfqpoint{1.403106in}{2.109858in}}%
\pgfpathcurveto{\pgfqpoint{1.403106in}{2.118094in}}{\pgfqpoint{1.399834in}{2.125994in}}{\pgfqpoint{1.394010in}{2.131818in}}%
\pgfpathcurveto{\pgfqpoint{1.388186in}{2.137642in}}{\pgfqpoint{1.380286in}{2.140914in}}{\pgfqpoint{1.372050in}{2.140914in}}%
\pgfpathcurveto{\pgfqpoint{1.363813in}{2.140914in}}{\pgfqpoint{1.355913in}{2.137642in}}{\pgfqpoint{1.350089in}{2.131818in}}%
\pgfpathcurveto{\pgfqpoint{1.344265in}{2.125994in}}{\pgfqpoint{1.340993in}{2.118094in}}{\pgfqpoint{1.340993in}{2.109858in}}%
\pgfpathcurveto{\pgfqpoint{1.340993in}{2.101621in}}{\pgfqpoint{1.344265in}{2.093721in}}{\pgfqpoint{1.350089in}{2.087898in}}%
\pgfpathcurveto{\pgfqpoint{1.355913in}{2.082074in}}{\pgfqpoint{1.363813in}{2.078801in}}{\pgfqpoint{1.372050in}{2.078801in}}%
\pgfpathclose%
\pgfusepath{stroke,fill}%
\end{pgfscope}%
\begin{pgfscope}%
\pgfpathrectangle{\pgfqpoint{0.100000in}{0.212622in}}{\pgfqpoint{3.696000in}{3.696000in}}%
\pgfusepath{clip}%
\pgfsetbuttcap%
\pgfsetroundjoin%
\definecolor{currentfill}{rgb}{0.121569,0.466667,0.705882}%
\pgfsetfillcolor{currentfill}%
\pgfsetfillopacity{0.911430}%
\pgfsetlinewidth{1.003750pt}%
\definecolor{currentstroke}{rgb}{0.121569,0.466667,0.705882}%
\pgfsetstrokecolor{currentstroke}%
\pgfsetstrokeopacity{0.911430}%
\pgfsetdash{}{0pt}%
\pgfpathmoveto{\pgfqpoint{2.614304in}{1.675021in}}%
\pgfpathcurveto{\pgfqpoint{2.622540in}{1.675021in}}{\pgfqpoint{2.630440in}{1.678293in}}{\pgfqpoint{2.636264in}{1.684117in}}%
\pgfpathcurveto{\pgfqpoint{2.642088in}{1.689941in}}{\pgfqpoint{2.645360in}{1.697841in}}{\pgfqpoint{2.645360in}{1.706077in}}%
\pgfpathcurveto{\pgfqpoint{2.645360in}{1.714313in}}{\pgfqpoint{2.642088in}{1.722213in}}{\pgfqpoint{2.636264in}{1.728037in}}%
\pgfpathcurveto{\pgfqpoint{2.630440in}{1.733861in}}{\pgfqpoint{2.622540in}{1.737134in}}{\pgfqpoint{2.614304in}{1.737134in}}%
\pgfpathcurveto{\pgfqpoint{2.606067in}{1.737134in}}{\pgfqpoint{2.598167in}{1.733861in}}{\pgfqpoint{2.592343in}{1.728037in}}%
\pgfpathcurveto{\pgfqpoint{2.586520in}{1.722213in}}{\pgfqpoint{2.583247in}{1.714313in}}{\pgfqpoint{2.583247in}{1.706077in}}%
\pgfpathcurveto{\pgfqpoint{2.583247in}{1.697841in}}{\pgfqpoint{2.586520in}{1.689941in}}{\pgfqpoint{2.592343in}{1.684117in}}%
\pgfpathcurveto{\pgfqpoint{2.598167in}{1.678293in}}{\pgfqpoint{2.606067in}{1.675021in}}{\pgfqpoint{2.614304in}{1.675021in}}%
\pgfpathclose%
\pgfusepath{stroke,fill}%
\end{pgfscope}%
\begin{pgfscope}%
\pgfpathrectangle{\pgfqpoint{0.100000in}{0.212622in}}{\pgfqpoint{3.696000in}{3.696000in}}%
\pgfusepath{clip}%
\pgfsetbuttcap%
\pgfsetroundjoin%
\definecolor{currentfill}{rgb}{0.121569,0.466667,0.705882}%
\pgfsetfillcolor{currentfill}%
\pgfsetfillopacity{0.911585}%
\pgfsetlinewidth{1.003750pt}%
\definecolor{currentstroke}{rgb}{0.121569,0.466667,0.705882}%
\pgfsetstrokecolor{currentstroke}%
\pgfsetstrokeopacity{0.911585}%
\pgfsetdash{}{0pt}%
\pgfpathmoveto{\pgfqpoint{1.384671in}{2.074427in}}%
\pgfpathcurveto{\pgfqpoint{1.392907in}{2.074427in}}{\pgfqpoint{1.400807in}{2.077699in}}{\pgfqpoint{1.406631in}{2.083523in}}%
\pgfpathcurveto{\pgfqpoint{1.412455in}{2.089347in}}{\pgfqpoint{1.415727in}{2.097247in}}{\pgfqpoint{1.415727in}{2.105483in}}%
\pgfpathcurveto{\pgfqpoint{1.415727in}{2.113719in}}{\pgfqpoint{1.412455in}{2.121619in}}{\pgfqpoint{1.406631in}{2.127443in}}%
\pgfpathcurveto{\pgfqpoint{1.400807in}{2.133267in}}{\pgfqpoint{1.392907in}{2.136540in}}{\pgfqpoint{1.384671in}{2.136540in}}%
\pgfpathcurveto{\pgfqpoint{1.376435in}{2.136540in}}{\pgfqpoint{1.368535in}{2.133267in}}{\pgfqpoint{1.362711in}{2.127443in}}%
\pgfpathcurveto{\pgfqpoint{1.356887in}{2.121619in}}{\pgfqpoint{1.353614in}{2.113719in}}{\pgfqpoint{1.353614in}{2.105483in}}%
\pgfpathcurveto{\pgfqpoint{1.353614in}{2.097247in}}{\pgfqpoint{1.356887in}{2.089347in}}{\pgfqpoint{1.362711in}{2.083523in}}%
\pgfpathcurveto{\pgfqpoint{1.368535in}{2.077699in}}{\pgfqpoint{1.376435in}{2.074427in}}{\pgfqpoint{1.384671in}{2.074427in}}%
\pgfpathclose%
\pgfusepath{stroke,fill}%
\end{pgfscope}%
\begin{pgfscope}%
\pgfpathrectangle{\pgfqpoint{0.100000in}{0.212622in}}{\pgfqpoint{3.696000in}{3.696000in}}%
\pgfusepath{clip}%
\pgfsetbuttcap%
\pgfsetroundjoin%
\definecolor{currentfill}{rgb}{0.121569,0.466667,0.705882}%
\pgfsetfillcolor{currentfill}%
\pgfsetfillopacity{0.913166}%
\pgfsetlinewidth{1.003750pt}%
\definecolor{currentstroke}{rgb}{0.121569,0.466667,0.705882}%
\pgfsetstrokecolor{currentstroke}%
\pgfsetstrokeopacity{0.913166}%
\pgfsetdash{}{0pt}%
\pgfpathmoveto{\pgfqpoint{2.610982in}{1.666085in}}%
\pgfpathcurveto{\pgfqpoint{2.619218in}{1.666085in}}{\pgfqpoint{2.627118in}{1.669358in}}{\pgfqpoint{2.632942in}{1.675182in}}%
\pgfpathcurveto{\pgfqpoint{2.638766in}{1.681006in}}{\pgfqpoint{2.642039in}{1.688906in}}{\pgfqpoint{2.642039in}{1.697142in}}%
\pgfpathcurveto{\pgfqpoint{2.642039in}{1.705378in}}{\pgfqpoint{2.638766in}{1.713278in}}{\pgfqpoint{2.632942in}{1.719102in}}%
\pgfpathcurveto{\pgfqpoint{2.627118in}{1.724926in}}{\pgfqpoint{2.619218in}{1.728198in}}{\pgfqpoint{2.610982in}{1.728198in}}%
\pgfpathcurveto{\pgfqpoint{2.602746in}{1.728198in}}{\pgfqpoint{2.594846in}{1.724926in}}{\pgfqpoint{2.589022in}{1.719102in}}%
\pgfpathcurveto{\pgfqpoint{2.583198in}{1.713278in}}{\pgfqpoint{2.579926in}{1.705378in}}{\pgfqpoint{2.579926in}{1.697142in}}%
\pgfpathcurveto{\pgfqpoint{2.579926in}{1.688906in}}{\pgfqpoint{2.583198in}{1.681006in}}{\pgfqpoint{2.589022in}{1.675182in}}%
\pgfpathcurveto{\pgfqpoint{2.594846in}{1.669358in}}{\pgfqpoint{2.602746in}{1.666085in}}{\pgfqpoint{2.610982in}{1.666085in}}%
\pgfpathclose%
\pgfusepath{stroke,fill}%
\end{pgfscope}%
\begin{pgfscope}%
\pgfpathrectangle{\pgfqpoint{0.100000in}{0.212622in}}{\pgfqpoint{3.696000in}{3.696000in}}%
\pgfusepath{clip}%
\pgfsetbuttcap%
\pgfsetroundjoin%
\definecolor{currentfill}{rgb}{0.121569,0.466667,0.705882}%
\pgfsetfillcolor{currentfill}%
\pgfsetfillopacity{0.914560}%
\pgfsetlinewidth{1.003750pt}%
\definecolor{currentstroke}{rgb}{0.121569,0.466667,0.705882}%
\pgfsetstrokecolor{currentstroke}%
\pgfsetstrokeopacity{0.914560}%
\pgfsetdash{}{0pt}%
\pgfpathmoveto{\pgfqpoint{1.407567in}{2.070166in}}%
\pgfpathcurveto{\pgfqpoint{1.415803in}{2.070166in}}{\pgfqpoint{1.423703in}{2.073438in}}{\pgfqpoint{1.429527in}{2.079262in}}%
\pgfpathcurveto{\pgfqpoint{1.435351in}{2.085086in}}{\pgfqpoint{1.438623in}{2.092986in}}{\pgfqpoint{1.438623in}{2.101222in}}%
\pgfpathcurveto{\pgfqpoint{1.438623in}{2.109459in}}{\pgfqpoint{1.435351in}{2.117359in}}{\pgfqpoint{1.429527in}{2.123183in}}%
\pgfpathcurveto{\pgfqpoint{1.423703in}{2.129007in}}{\pgfqpoint{1.415803in}{2.132279in}}{\pgfqpoint{1.407567in}{2.132279in}}%
\pgfpathcurveto{\pgfqpoint{1.399331in}{2.132279in}}{\pgfqpoint{1.391431in}{2.129007in}}{\pgfqpoint{1.385607in}{2.123183in}}%
\pgfpathcurveto{\pgfqpoint{1.379783in}{2.117359in}}{\pgfqpoint{1.376510in}{2.109459in}}{\pgfqpoint{1.376510in}{2.101222in}}%
\pgfpathcurveto{\pgfqpoint{1.376510in}{2.092986in}}{\pgfqpoint{1.379783in}{2.085086in}}{\pgfqpoint{1.385607in}{2.079262in}}%
\pgfpathcurveto{\pgfqpoint{1.391431in}{2.073438in}}{\pgfqpoint{1.399331in}{2.070166in}}{\pgfqpoint{1.407567in}{2.070166in}}%
\pgfpathclose%
\pgfusepath{stroke,fill}%
\end{pgfscope}%
\begin{pgfscope}%
\pgfpathrectangle{\pgfqpoint{0.100000in}{0.212622in}}{\pgfqpoint{3.696000in}{3.696000in}}%
\pgfusepath{clip}%
\pgfsetbuttcap%
\pgfsetroundjoin%
\definecolor{currentfill}{rgb}{0.121569,0.466667,0.705882}%
\pgfsetfillcolor{currentfill}%
\pgfsetfillopacity{0.915772}%
\pgfsetlinewidth{1.003750pt}%
\definecolor{currentstroke}{rgb}{0.121569,0.466667,0.705882}%
\pgfsetstrokecolor{currentstroke}%
\pgfsetstrokeopacity{0.915772}%
\pgfsetdash{}{0pt}%
\pgfpathmoveto{\pgfqpoint{2.607801in}{1.660308in}}%
\pgfpathcurveto{\pgfqpoint{2.616038in}{1.660308in}}{\pgfqpoint{2.623938in}{1.663581in}}{\pgfqpoint{2.629762in}{1.669405in}}%
\pgfpathcurveto{\pgfqpoint{2.635586in}{1.675229in}}{\pgfqpoint{2.638858in}{1.683129in}}{\pgfqpoint{2.638858in}{1.691365in}}%
\pgfpathcurveto{\pgfqpoint{2.638858in}{1.699601in}}{\pgfqpoint{2.635586in}{1.707501in}}{\pgfqpoint{2.629762in}{1.713325in}}%
\pgfpathcurveto{\pgfqpoint{2.623938in}{1.719149in}}{\pgfqpoint{2.616038in}{1.722421in}}{\pgfqpoint{2.607801in}{1.722421in}}%
\pgfpathcurveto{\pgfqpoint{2.599565in}{1.722421in}}{\pgfqpoint{2.591665in}{1.719149in}}{\pgfqpoint{2.585841in}{1.713325in}}%
\pgfpathcurveto{\pgfqpoint{2.580017in}{1.707501in}}{\pgfqpoint{2.576745in}{1.699601in}}{\pgfqpoint{2.576745in}{1.691365in}}%
\pgfpathcurveto{\pgfqpoint{2.576745in}{1.683129in}}{\pgfqpoint{2.580017in}{1.675229in}}{\pgfqpoint{2.585841in}{1.669405in}}%
\pgfpathcurveto{\pgfqpoint{2.591665in}{1.663581in}}{\pgfqpoint{2.599565in}{1.660308in}}{\pgfqpoint{2.607801in}{1.660308in}}%
\pgfpathclose%
\pgfusepath{stroke,fill}%
\end{pgfscope}%
\begin{pgfscope}%
\pgfpathrectangle{\pgfqpoint{0.100000in}{0.212622in}}{\pgfqpoint{3.696000in}{3.696000in}}%
\pgfusepath{clip}%
\pgfsetbuttcap%
\pgfsetroundjoin%
\definecolor{currentfill}{rgb}{0.121569,0.466667,0.705882}%
\pgfsetfillcolor{currentfill}%
\pgfsetfillopacity{0.916753}%
\pgfsetlinewidth{1.003750pt}%
\definecolor{currentstroke}{rgb}{0.121569,0.466667,0.705882}%
\pgfsetstrokecolor{currentstroke}%
\pgfsetstrokeopacity{0.916753}%
\pgfsetdash{}{0pt}%
\pgfpathmoveto{\pgfqpoint{1.431902in}{2.068856in}}%
\pgfpathcurveto{\pgfqpoint{1.440139in}{2.068856in}}{\pgfqpoint{1.448039in}{2.072128in}}{\pgfqpoint{1.453863in}{2.077952in}}%
\pgfpathcurveto{\pgfqpoint{1.459687in}{2.083776in}}{\pgfqpoint{1.462959in}{2.091676in}}{\pgfqpoint{1.462959in}{2.099912in}}%
\pgfpathcurveto{\pgfqpoint{1.462959in}{2.108149in}}{\pgfqpoint{1.459687in}{2.116049in}}{\pgfqpoint{1.453863in}{2.121873in}}%
\pgfpathcurveto{\pgfqpoint{1.448039in}{2.127697in}}{\pgfqpoint{1.440139in}{2.130969in}}{\pgfqpoint{1.431902in}{2.130969in}}%
\pgfpathcurveto{\pgfqpoint{1.423666in}{2.130969in}}{\pgfqpoint{1.415766in}{2.127697in}}{\pgfqpoint{1.409942in}{2.121873in}}%
\pgfpathcurveto{\pgfqpoint{1.404118in}{2.116049in}}{\pgfqpoint{1.400846in}{2.108149in}}{\pgfqpoint{1.400846in}{2.099912in}}%
\pgfpathcurveto{\pgfqpoint{1.400846in}{2.091676in}}{\pgfqpoint{1.404118in}{2.083776in}}{\pgfqpoint{1.409942in}{2.077952in}}%
\pgfpathcurveto{\pgfqpoint{1.415766in}{2.072128in}}{\pgfqpoint{1.423666in}{2.068856in}}{\pgfqpoint{1.431902in}{2.068856in}}%
\pgfpathclose%
\pgfusepath{stroke,fill}%
\end{pgfscope}%
\begin{pgfscope}%
\pgfpathrectangle{\pgfqpoint{0.100000in}{0.212622in}}{\pgfqpoint{3.696000in}{3.696000in}}%
\pgfusepath{clip}%
\pgfsetbuttcap%
\pgfsetroundjoin%
\definecolor{currentfill}{rgb}{0.121569,0.466667,0.705882}%
\pgfsetfillcolor{currentfill}%
\pgfsetfillopacity{0.917547}%
\pgfsetlinewidth{1.003750pt}%
\definecolor{currentstroke}{rgb}{0.121569,0.466667,0.705882}%
\pgfsetstrokecolor{currentstroke}%
\pgfsetstrokeopacity{0.917547}%
\pgfsetdash{}{0pt}%
\pgfpathmoveto{\pgfqpoint{1.451724in}{2.056226in}}%
\pgfpathcurveto{\pgfqpoint{1.459961in}{2.056226in}}{\pgfqpoint{1.467861in}{2.059499in}}{\pgfqpoint{1.473685in}{2.065323in}}%
\pgfpathcurveto{\pgfqpoint{1.479509in}{2.071147in}}{\pgfqpoint{1.482781in}{2.079047in}}{\pgfqpoint{1.482781in}{2.087283in}}%
\pgfpathcurveto{\pgfqpoint{1.482781in}{2.095519in}}{\pgfqpoint{1.479509in}{2.103419in}}{\pgfqpoint{1.473685in}{2.109243in}}%
\pgfpathcurveto{\pgfqpoint{1.467861in}{2.115067in}}{\pgfqpoint{1.459961in}{2.118339in}}{\pgfqpoint{1.451724in}{2.118339in}}%
\pgfpathcurveto{\pgfqpoint{1.443488in}{2.118339in}}{\pgfqpoint{1.435588in}{2.115067in}}{\pgfqpoint{1.429764in}{2.109243in}}%
\pgfpathcurveto{\pgfqpoint{1.423940in}{2.103419in}}{\pgfqpoint{1.420668in}{2.095519in}}{\pgfqpoint{1.420668in}{2.087283in}}%
\pgfpathcurveto{\pgfqpoint{1.420668in}{2.079047in}}{\pgfqpoint{1.423940in}{2.071147in}}{\pgfqpoint{1.429764in}{2.065323in}}%
\pgfpathcurveto{\pgfqpoint{1.435588in}{2.059499in}}{\pgfqpoint{1.443488in}{2.056226in}}{\pgfqpoint{1.451724in}{2.056226in}}%
\pgfpathclose%
\pgfusepath{stroke,fill}%
\end{pgfscope}%
\begin{pgfscope}%
\pgfpathrectangle{\pgfqpoint{0.100000in}{0.212622in}}{\pgfqpoint{3.696000in}{3.696000in}}%
\pgfusepath{clip}%
\pgfsetbuttcap%
\pgfsetroundjoin%
\definecolor{currentfill}{rgb}{0.121569,0.466667,0.705882}%
\pgfsetfillcolor{currentfill}%
\pgfsetfillopacity{0.918200}%
\pgfsetlinewidth{1.003750pt}%
\definecolor{currentstroke}{rgb}{0.121569,0.466667,0.705882}%
\pgfsetstrokecolor{currentstroke}%
\pgfsetstrokeopacity{0.918200}%
\pgfsetdash{}{0pt}%
\pgfpathmoveto{\pgfqpoint{1.469468in}{2.043710in}}%
\pgfpathcurveto{\pgfqpoint{1.477704in}{2.043710in}}{\pgfqpoint{1.485604in}{2.046983in}}{\pgfqpoint{1.491428in}{2.052806in}}%
\pgfpathcurveto{\pgfqpoint{1.497252in}{2.058630in}}{\pgfqpoint{1.500524in}{2.066530in}}{\pgfqpoint{1.500524in}{2.074767in}}%
\pgfpathcurveto{\pgfqpoint{1.500524in}{2.083003in}}{\pgfqpoint{1.497252in}{2.090903in}}{\pgfqpoint{1.491428in}{2.096727in}}%
\pgfpathcurveto{\pgfqpoint{1.485604in}{2.102551in}}{\pgfqpoint{1.477704in}{2.105823in}}{\pgfqpoint{1.469468in}{2.105823in}}%
\pgfpathcurveto{\pgfqpoint{1.461231in}{2.105823in}}{\pgfqpoint{1.453331in}{2.102551in}}{\pgfqpoint{1.447507in}{2.096727in}}%
\pgfpathcurveto{\pgfqpoint{1.441683in}{2.090903in}}{\pgfqpoint{1.438411in}{2.083003in}}{\pgfqpoint{1.438411in}{2.074767in}}%
\pgfpathcurveto{\pgfqpoint{1.438411in}{2.066530in}}{\pgfqpoint{1.441683in}{2.058630in}}{\pgfqpoint{1.447507in}{2.052806in}}%
\pgfpathcurveto{\pgfqpoint{1.453331in}{2.046983in}}{\pgfqpoint{1.461231in}{2.043710in}}{\pgfqpoint{1.469468in}{2.043710in}}%
\pgfpathclose%
\pgfusepath{stroke,fill}%
\end{pgfscope}%
\begin{pgfscope}%
\pgfpathrectangle{\pgfqpoint{0.100000in}{0.212622in}}{\pgfqpoint{3.696000in}{3.696000in}}%
\pgfusepath{clip}%
\pgfsetbuttcap%
\pgfsetroundjoin%
\definecolor{currentfill}{rgb}{0.121569,0.466667,0.705882}%
\pgfsetfillcolor{currentfill}%
\pgfsetfillopacity{0.919042}%
\pgfsetlinewidth{1.003750pt}%
\definecolor{currentstroke}{rgb}{0.121569,0.466667,0.705882}%
\pgfsetstrokecolor{currentstroke}%
\pgfsetstrokeopacity{0.919042}%
\pgfsetdash{}{0pt}%
\pgfpathmoveto{\pgfqpoint{1.487406in}{2.036403in}}%
\pgfpathcurveto{\pgfqpoint{1.495642in}{2.036403in}}{\pgfqpoint{1.503542in}{2.039676in}}{\pgfqpoint{1.509366in}{2.045500in}}%
\pgfpathcurveto{\pgfqpoint{1.515190in}{2.051324in}}{\pgfqpoint{1.518463in}{2.059224in}}{\pgfqpoint{1.518463in}{2.067460in}}%
\pgfpathcurveto{\pgfqpoint{1.518463in}{2.075696in}}{\pgfqpoint{1.515190in}{2.083596in}}{\pgfqpoint{1.509366in}{2.089420in}}%
\pgfpathcurveto{\pgfqpoint{1.503542in}{2.095244in}}{\pgfqpoint{1.495642in}{2.098516in}}{\pgfqpoint{1.487406in}{2.098516in}}%
\pgfpathcurveto{\pgfqpoint{1.479170in}{2.098516in}}{\pgfqpoint{1.471270in}{2.095244in}}{\pgfqpoint{1.465446in}{2.089420in}}%
\pgfpathcurveto{\pgfqpoint{1.459622in}{2.083596in}}{\pgfqpoint{1.456350in}{2.075696in}}{\pgfqpoint{1.456350in}{2.067460in}}%
\pgfpathcurveto{\pgfqpoint{1.456350in}{2.059224in}}{\pgfqpoint{1.459622in}{2.051324in}}{\pgfqpoint{1.465446in}{2.045500in}}%
\pgfpathcurveto{\pgfqpoint{1.471270in}{2.039676in}}{\pgfqpoint{1.479170in}{2.036403in}}{\pgfqpoint{1.487406in}{2.036403in}}%
\pgfpathclose%
\pgfusepath{stroke,fill}%
\end{pgfscope}%
\begin{pgfscope}%
\pgfpathrectangle{\pgfqpoint{0.100000in}{0.212622in}}{\pgfqpoint{3.696000in}{3.696000in}}%
\pgfusepath{clip}%
\pgfsetbuttcap%
\pgfsetroundjoin%
\definecolor{currentfill}{rgb}{0.121569,0.466667,0.705882}%
\pgfsetfillcolor{currentfill}%
\pgfsetfillopacity{0.919059}%
\pgfsetlinewidth{1.003750pt}%
\definecolor{currentstroke}{rgb}{0.121569,0.466667,0.705882}%
\pgfsetstrokecolor{currentstroke}%
\pgfsetstrokeopacity{0.919059}%
\pgfsetdash{}{0pt}%
\pgfpathmoveto{\pgfqpoint{1.514019in}{2.013289in}}%
\pgfpathcurveto{\pgfqpoint{1.522256in}{2.013289in}}{\pgfqpoint{1.530156in}{2.016561in}}{\pgfqpoint{1.535980in}{2.022385in}}%
\pgfpathcurveto{\pgfqpoint{1.541804in}{2.028209in}}{\pgfqpoint{1.545076in}{2.036109in}}{\pgfqpoint{1.545076in}{2.044345in}}%
\pgfpathcurveto{\pgfqpoint{1.545076in}{2.052581in}}{\pgfqpoint{1.541804in}{2.060481in}}{\pgfqpoint{1.535980in}{2.066305in}}%
\pgfpathcurveto{\pgfqpoint{1.530156in}{2.072129in}}{\pgfqpoint{1.522256in}{2.075402in}}{\pgfqpoint{1.514019in}{2.075402in}}%
\pgfpathcurveto{\pgfqpoint{1.505783in}{2.075402in}}{\pgfqpoint{1.497883in}{2.072129in}}{\pgfqpoint{1.492059in}{2.066305in}}%
\pgfpathcurveto{\pgfqpoint{1.486235in}{2.060481in}}{\pgfqpoint{1.482963in}{2.052581in}}{\pgfqpoint{1.482963in}{2.044345in}}%
\pgfpathcurveto{\pgfqpoint{1.482963in}{2.036109in}}{\pgfqpoint{1.486235in}{2.028209in}}{\pgfqpoint{1.492059in}{2.022385in}}%
\pgfpathcurveto{\pgfqpoint{1.497883in}{2.016561in}}{\pgfqpoint{1.505783in}{2.013289in}}{\pgfqpoint{1.514019in}{2.013289in}}%
\pgfpathclose%
\pgfusepath{stroke,fill}%
\end{pgfscope}%
\begin{pgfscope}%
\pgfpathrectangle{\pgfqpoint{0.100000in}{0.212622in}}{\pgfqpoint{3.696000in}{3.696000in}}%
\pgfusepath{clip}%
\pgfsetbuttcap%
\pgfsetroundjoin%
\definecolor{currentfill}{rgb}{0.121569,0.466667,0.705882}%
\pgfsetfillcolor{currentfill}%
\pgfsetfillopacity{0.919152}%
\pgfsetlinewidth{1.003750pt}%
\definecolor{currentstroke}{rgb}{0.121569,0.466667,0.705882}%
\pgfsetstrokecolor{currentstroke}%
\pgfsetstrokeopacity{0.919152}%
\pgfsetdash{}{0pt}%
\pgfpathmoveto{\pgfqpoint{1.501939in}{2.024442in}}%
\pgfpathcurveto{\pgfqpoint{1.510176in}{2.024442in}}{\pgfqpoint{1.518076in}{2.027714in}}{\pgfqpoint{1.523899in}{2.033538in}}%
\pgfpathcurveto{\pgfqpoint{1.529723in}{2.039362in}}{\pgfqpoint{1.532996in}{2.047262in}}{\pgfqpoint{1.532996in}{2.055499in}}%
\pgfpathcurveto{\pgfqpoint{1.532996in}{2.063735in}}{\pgfqpoint{1.529723in}{2.071635in}}{\pgfqpoint{1.523899in}{2.077459in}}%
\pgfpathcurveto{\pgfqpoint{1.518076in}{2.083283in}}{\pgfqpoint{1.510176in}{2.086555in}}{\pgfqpoint{1.501939in}{2.086555in}}%
\pgfpathcurveto{\pgfqpoint{1.493703in}{2.086555in}}{\pgfqpoint{1.485803in}{2.083283in}}{\pgfqpoint{1.479979in}{2.077459in}}%
\pgfpathcurveto{\pgfqpoint{1.474155in}{2.071635in}}{\pgfqpoint{1.470883in}{2.063735in}}{\pgfqpoint{1.470883in}{2.055499in}}%
\pgfpathcurveto{\pgfqpoint{1.470883in}{2.047262in}}{\pgfqpoint{1.474155in}{2.039362in}}{\pgfqpoint{1.479979in}{2.033538in}}%
\pgfpathcurveto{\pgfqpoint{1.485803in}{2.027714in}}{\pgfqpoint{1.493703in}{2.024442in}}{\pgfqpoint{1.501939in}{2.024442in}}%
\pgfpathclose%
\pgfusepath{stroke,fill}%
\end{pgfscope}%
\begin{pgfscope}%
\pgfpathrectangle{\pgfqpoint{0.100000in}{0.212622in}}{\pgfqpoint{3.696000in}{3.696000in}}%
\pgfusepath{clip}%
\pgfsetbuttcap%
\pgfsetroundjoin%
\definecolor{currentfill}{rgb}{0.121569,0.466667,0.705882}%
\pgfsetfillcolor{currentfill}%
\pgfsetfillopacity{0.919804}%
\pgfsetlinewidth{1.003750pt}%
\definecolor{currentstroke}{rgb}{0.121569,0.466667,0.705882}%
\pgfsetstrokecolor{currentstroke}%
\pgfsetstrokeopacity{0.919804}%
\pgfsetdash{}{0pt}%
\pgfpathmoveto{\pgfqpoint{2.597223in}{1.667393in}}%
\pgfpathcurveto{\pgfqpoint{2.605460in}{1.667393in}}{\pgfqpoint{2.613360in}{1.670666in}}{\pgfqpoint{2.619184in}{1.676489in}}%
\pgfpathcurveto{\pgfqpoint{2.625008in}{1.682313in}}{\pgfqpoint{2.628280in}{1.690213in}}{\pgfqpoint{2.628280in}{1.698450in}}%
\pgfpathcurveto{\pgfqpoint{2.628280in}{1.706686in}}{\pgfqpoint{2.625008in}{1.714586in}}{\pgfqpoint{2.619184in}{1.720410in}}%
\pgfpathcurveto{\pgfqpoint{2.613360in}{1.726234in}}{\pgfqpoint{2.605460in}{1.729506in}}{\pgfqpoint{2.597223in}{1.729506in}}%
\pgfpathcurveto{\pgfqpoint{2.588987in}{1.729506in}}{\pgfqpoint{2.581087in}{1.726234in}}{\pgfqpoint{2.575263in}{1.720410in}}%
\pgfpathcurveto{\pgfqpoint{2.569439in}{1.714586in}}{\pgfqpoint{2.566167in}{1.706686in}}{\pgfqpoint{2.566167in}{1.698450in}}%
\pgfpathcurveto{\pgfqpoint{2.566167in}{1.690213in}}{\pgfqpoint{2.569439in}{1.682313in}}{\pgfqpoint{2.575263in}{1.676489in}}%
\pgfpathcurveto{\pgfqpoint{2.581087in}{1.670666in}}{\pgfqpoint{2.588987in}{1.667393in}}{\pgfqpoint{2.597223in}{1.667393in}}%
\pgfpathclose%
\pgfusepath{stroke,fill}%
\end{pgfscope}%
\begin{pgfscope}%
\pgfpathrectangle{\pgfqpoint{0.100000in}{0.212622in}}{\pgfqpoint{3.696000in}{3.696000in}}%
\pgfusepath{clip}%
\pgfsetbuttcap%
\pgfsetroundjoin%
\definecolor{currentfill}{rgb}{0.121569,0.466667,0.705882}%
\pgfsetfillcolor{currentfill}%
\pgfsetfillopacity{0.921149}%
\pgfsetlinewidth{1.003750pt}%
\definecolor{currentstroke}{rgb}{0.121569,0.466667,0.705882}%
\pgfsetstrokecolor{currentstroke}%
\pgfsetstrokeopacity{0.921149}%
\pgfsetdash{}{0pt}%
\pgfpathmoveto{\pgfqpoint{2.595183in}{1.660629in}}%
\pgfpathcurveto{\pgfqpoint{2.603419in}{1.660629in}}{\pgfqpoint{2.611319in}{1.663901in}}{\pgfqpoint{2.617143in}{1.669725in}}%
\pgfpathcurveto{\pgfqpoint{2.622967in}{1.675549in}}{\pgfqpoint{2.626239in}{1.683449in}}{\pgfqpoint{2.626239in}{1.691685in}}%
\pgfpathcurveto{\pgfqpoint{2.626239in}{1.699922in}}{\pgfqpoint{2.622967in}{1.707822in}}{\pgfqpoint{2.617143in}{1.713646in}}%
\pgfpathcurveto{\pgfqpoint{2.611319in}{1.719470in}}{\pgfqpoint{2.603419in}{1.722742in}}{\pgfqpoint{2.595183in}{1.722742in}}%
\pgfpathcurveto{\pgfqpoint{2.586946in}{1.722742in}}{\pgfqpoint{2.579046in}{1.719470in}}{\pgfqpoint{2.573222in}{1.713646in}}%
\pgfpathcurveto{\pgfqpoint{2.567398in}{1.707822in}}{\pgfqpoint{2.564126in}{1.699922in}}{\pgfqpoint{2.564126in}{1.691685in}}%
\pgfpathcurveto{\pgfqpoint{2.564126in}{1.683449in}}{\pgfqpoint{2.567398in}{1.675549in}}{\pgfqpoint{2.573222in}{1.669725in}}%
\pgfpathcurveto{\pgfqpoint{2.579046in}{1.663901in}}{\pgfqpoint{2.586946in}{1.660629in}}{\pgfqpoint{2.595183in}{1.660629in}}%
\pgfpathclose%
\pgfusepath{stroke,fill}%
\end{pgfscope}%
\begin{pgfscope}%
\pgfpathrectangle{\pgfqpoint{0.100000in}{0.212622in}}{\pgfqpoint{3.696000in}{3.696000in}}%
\pgfusepath{clip}%
\pgfsetbuttcap%
\pgfsetroundjoin%
\definecolor{currentfill}{rgb}{0.121569,0.466667,0.705882}%
\pgfsetfillcolor{currentfill}%
\pgfsetfillopacity{0.921251}%
\pgfsetlinewidth{1.003750pt}%
\definecolor{currentstroke}{rgb}{0.121569,0.466667,0.705882}%
\pgfsetstrokecolor{currentstroke}%
\pgfsetstrokeopacity{0.921251}%
\pgfsetdash{}{0pt}%
\pgfpathmoveto{\pgfqpoint{1.536784in}{2.014129in}}%
\pgfpathcurveto{\pgfqpoint{1.545021in}{2.014129in}}{\pgfqpoint{1.552921in}{2.017401in}}{\pgfqpoint{1.558745in}{2.023225in}}%
\pgfpathcurveto{\pgfqpoint{1.564569in}{2.029049in}}{\pgfqpoint{1.567841in}{2.036949in}}{\pgfqpoint{1.567841in}{2.045185in}}%
\pgfpathcurveto{\pgfqpoint{1.567841in}{2.053421in}}{\pgfqpoint{1.564569in}{2.061321in}}{\pgfqpoint{1.558745in}{2.067145in}}%
\pgfpathcurveto{\pgfqpoint{1.552921in}{2.072969in}}{\pgfqpoint{1.545021in}{2.076242in}}{\pgfqpoint{1.536784in}{2.076242in}}%
\pgfpathcurveto{\pgfqpoint{1.528548in}{2.076242in}}{\pgfqpoint{1.520648in}{2.072969in}}{\pgfqpoint{1.514824in}{2.067145in}}%
\pgfpathcurveto{\pgfqpoint{1.509000in}{2.061321in}}{\pgfqpoint{1.505728in}{2.053421in}}{\pgfqpoint{1.505728in}{2.045185in}}%
\pgfpathcurveto{\pgfqpoint{1.505728in}{2.036949in}}{\pgfqpoint{1.509000in}{2.029049in}}{\pgfqpoint{1.514824in}{2.023225in}}%
\pgfpathcurveto{\pgfqpoint{1.520648in}{2.017401in}}{\pgfqpoint{1.528548in}{2.014129in}}{\pgfqpoint{1.536784in}{2.014129in}}%
\pgfpathclose%
\pgfusepath{stroke,fill}%
\end{pgfscope}%
\begin{pgfscope}%
\pgfpathrectangle{\pgfqpoint{0.100000in}{0.212622in}}{\pgfqpoint{3.696000in}{3.696000in}}%
\pgfusepath{clip}%
\pgfsetbuttcap%
\pgfsetroundjoin%
\definecolor{currentfill}{rgb}{0.121569,0.466667,0.705882}%
\pgfsetfillcolor{currentfill}%
\pgfsetfillopacity{0.921586}%
\pgfsetlinewidth{1.003750pt}%
\definecolor{currentstroke}{rgb}{0.121569,0.466667,0.705882}%
\pgfsetstrokecolor{currentstroke}%
\pgfsetstrokeopacity{0.921586}%
\pgfsetdash{}{0pt}%
\pgfpathmoveto{\pgfqpoint{1.557586in}{1.995195in}}%
\pgfpathcurveto{\pgfqpoint{1.565822in}{1.995195in}}{\pgfqpoint{1.573723in}{1.998467in}}{\pgfqpoint{1.579546in}{2.004291in}}%
\pgfpathcurveto{\pgfqpoint{1.585370in}{2.010115in}}{\pgfqpoint{1.588643in}{2.018015in}}{\pgfqpoint{1.588643in}{2.026252in}}%
\pgfpathcurveto{\pgfqpoint{1.588643in}{2.034488in}}{\pgfqpoint{1.585370in}{2.042388in}}{\pgfqpoint{1.579546in}{2.048212in}}%
\pgfpathcurveto{\pgfqpoint{1.573723in}{2.054036in}}{\pgfqpoint{1.565822in}{2.057308in}}{\pgfqpoint{1.557586in}{2.057308in}}%
\pgfpathcurveto{\pgfqpoint{1.549350in}{2.057308in}}{\pgfqpoint{1.541450in}{2.054036in}}{\pgfqpoint{1.535626in}{2.048212in}}%
\pgfpathcurveto{\pgfqpoint{1.529802in}{2.042388in}}{\pgfqpoint{1.526530in}{2.034488in}}{\pgfqpoint{1.526530in}{2.026252in}}%
\pgfpathcurveto{\pgfqpoint{1.526530in}{2.018015in}}{\pgfqpoint{1.529802in}{2.010115in}}{\pgfqpoint{1.535626in}{2.004291in}}%
\pgfpathcurveto{\pgfqpoint{1.541450in}{1.998467in}}{\pgfqpoint{1.549350in}{1.995195in}}{\pgfqpoint{1.557586in}{1.995195in}}%
\pgfpathclose%
\pgfusepath{stroke,fill}%
\end{pgfscope}%
\begin{pgfscope}%
\pgfpathrectangle{\pgfqpoint{0.100000in}{0.212622in}}{\pgfqpoint{3.696000in}{3.696000in}}%
\pgfusepath{clip}%
\pgfsetbuttcap%
\pgfsetroundjoin%
\definecolor{currentfill}{rgb}{0.121569,0.466667,0.705882}%
\pgfsetfillcolor{currentfill}%
\pgfsetfillopacity{0.922146}%
\pgfsetlinewidth{1.003750pt}%
\definecolor{currentstroke}{rgb}{0.121569,0.466667,0.705882}%
\pgfsetstrokecolor{currentstroke}%
\pgfsetstrokeopacity{0.922146}%
\pgfsetdash{}{0pt}%
\pgfpathmoveto{\pgfqpoint{1.575081in}{1.980822in}}%
\pgfpathcurveto{\pgfqpoint{1.583317in}{1.980822in}}{\pgfqpoint{1.591217in}{1.984094in}}{\pgfqpoint{1.597041in}{1.989918in}}%
\pgfpathcurveto{\pgfqpoint{1.602865in}{1.995742in}}{\pgfqpoint{1.606137in}{2.003642in}}{\pgfqpoint{1.606137in}{2.011878in}}%
\pgfpathcurveto{\pgfqpoint{1.606137in}{2.020115in}}{\pgfqpoint{1.602865in}{2.028015in}}{\pgfqpoint{1.597041in}{2.033839in}}%
\pgfpathcurveto{\pgfqpoint{1.591217in}{2.039663in}}{\pgfqpoint{1.583317in}{2.042935in}}{\pgfqpoint{1.575081in}{2.042935in}}%
\pgfpathcurveto{\pgfqpoint{1.566844in}{2.042935in}}{\pgfqpoint{1.558944in}{2.039663in}}{\pgfqpoint{1.553120in}{2.033839in}}%
\pgfpathcurveto{\pgfqpoint{1.547296in}{2.028015in}}{\pgfqpoint{1.544024in}{2.020115in}}{\pgfqpoint{1.544024in}{2.011878in}}%
\pgfpathcurveto{\pgfqpoint{1.544024in}{2.003642in}}{\pgfqpoint{1.547296in}{1.995742in}}{\pgfqpoint{1.553120in}{1.989918in}}%
\pgfpathcurveto{\pgfqpoint{1.558944in}{1.984094in}}{\pgfqpoint{1.566844in}{1.980822in}}{\pgfqpoint{1.575081in}{1.980822in}}%
\pgfpathclose%
\pgfusepath{stroke,fill}%
\end{pgfscope}%
\begin{pgfscope}%
\pgfpathrectangle{\pgfqpoint{0.100000in}{0.212622in}}{\pgfqpoint{3.696000in}{3.696000in}}%
\pgfusepath{clip}%
\pgfsetbuttcap%
\pgfsetroundjoin%
\definecolor{currentfill}{rgb}{0.121569,0.466667,0.705882}%
\pgfsetfillcolor{currentfill}%
\pgfsetfillopacity{0.923304}%
\pgfsetlinewidth{1.003750pt}%
\definecolor{currentstroke}{rgb}{0.121569,0.466667,0.705882}%
\pgfsetstrokecolor{currentstroke}%
\pgfsetstrokeopacity{0.923304}%
\pgfsetdash{}{0pt}%
\pgfpathmoveto{\pgfqpoint{2.592800in}{1.656446in}}%
\pgfpathcurveto{\pgfqpoint{2.601036in}{1.656446in}}{\pgfqpoint{2.608936in}{1.659719in}}{\pgfqpoint{2.614760in}{1.665543in}}%
\pgfpathcurveto{\pgfqpoint{2.620584in}{1.671367in}}{\pgfqpoint{2.623856in}{1.679267in}}{\pgfqpoint{2.623856in}{1.687503in}}%
\pgfpathcurveto{\pgfqpoint{2.623856in}{1.695739in}}{\pgfqpoint{2.620584in}{1.703639in}}{\pgfqpoint{2.614760in}{1.709463in}}%
\pgfpathcurveto{\pgfqpoint{2.608936in}{1.715287in}}{\pgfqpoint{2.601036in}{1.718559in}}{\pgfqpoint{2.592800in}{1.718559in}}%
\pgfpathcurveto{\pgfqpoint{2.584563in}{1.718559in}}{\pgfqpoint{2.576663in}{1.715287in}}{\pgfqpoint{2.570839in}{1.709463in}}%
\pgfpathcurveto{\pgfqpoint{2.565015in}{1.703639in}}{\pgfqpoint{2.561743in}{1.695739in}}{\pgfqpoint{2.561743in}{1.687503in}}%
\pgfpathcurveto{\pgfqpoint{2.561743in}{1.679267in}}{\pgfqpoint{2.565015in}{1.671367in}}{\pgfqpoint{2.570839in}{1.665543in}}%
\pgfpathcurveto{\pgfqpoint{2.576663in}{1.659719in}}{\pgfqpoint{2.584563in}{1.656446in}}{\pgfqpoint{2.592800in}{1.656446in}}%
\pgfpathclose%
\pgfusepath{stroke,fill}%
\end{pgfscope}%
\begin{pgfscope}%
\pgfpathrectangle{\pgfqpoint{0.100000in}{0.212622in}}{\pgfqpoint{3.696000in}{3.696000in}}%
\pgfusepath{clip}%
\pgfsetbuttcap%
\pgfsetroundjoin%
\definecolor{currentfill}{rgb}{0.121569,0.466667,0.705882}%
\pgfsetfillcolor{currentfill}%
\pgfsetfillopacity{0.924415}%
\pgfsetlinewidth{1.003750pt}%
\definecolor{currentstroke}{rgb}{0.121569,0.466667,0.705882}%
\pgfsetstrokecolor{currentstroke}%
\pgfsetstrokeopacity{0.924415}%
\pgfsetdash{}{0pt}%
\pgfpathmoveto{\pgfqpoint{1.592689in}{1.983416in}}%
\pgfpathcurveto{\pgfqpoint{1.600925in}{1.983416in}}{\pgfqpoint{1.608825in}{1.986689in}}{\pgfqpoint{1.614649in}{1.992513in}}%
\pgfpathcurveto{\pgfqpoint{1.620473in}{1.998337in}}{\pgfqpoint{1.623746in}{2.006237in}}{\pgfqpoint{1.623746in}{2.014473in}}%
\pgfpathcurveto{\pgfqpoint{1.623746in}{2.022709in}}{\pgfqpoint{1.620473in}{2.030609in}}{\pgfqpoint{1.614649in}{2.036433in}}%
\pgfpathcurveto{\pgfqpoint{1.608825in}{2.042257in}}{\pgfqpoint{1.600925in}{2.045529in}}{\pgfqpoint{1.592689in}{2.045529in}}%
\pgfpathcurveto{\pgfqpoint{1.584453in}{2.045529in}}{\pgfqpoint{1.576553in}{2.042257in}}{\pgfqpoint{1.570729in}{2.036433in}}%
\pgfpathcurveto{\pgfqpoint{1.564905in}{2.030609in}}{\pgfqpoint{1.561633in}{2.022709in}}{\pgfqpoint{1.561633in}{2.014473in}}%
\pgfpathcurveto{\pgfqpoint{1.561633in}{2.006237in}}{\pgfqpoint{1.564905in}{1.998337in}}{\pgfqpoint{1.570729in}{1.992513in}}%
\pgfpathcurveto{\pgfqpoint{1.576553in}{1.986689in}}{\pgfqpoint{1.584453in}{1.983416in}}{\pgfqpoint{1.592689in}{1.983416in}}%
\pgfpathclose%
\pgfusepath{stroke,fill}%
\end{pgfscope}%
\begin{pgfscope}%
\pgfpathrectangle{\pgfqpoint{0.100000in}{0.212622in}}{\pgfqpoint{3.696000in}{3.696000in}}%
\pgfusepath{clip}%
\pgfsetbuttcap%
\pgfsetroundjoin%
\definecolor{currentfill}{rgb}{0.121569,0.466667,0.705882}%
\pgfsetfillcolor{currentfill}%
\pgfsetfillopacity{0.925857}%
\pgfsetlinewidth{1.003750pt}%
\definecolor{currentstroke}{rgb}{0.121569,0.466667,0.705882}%
\pgfsetstrokecolor{currentstroke}%
\pgfsetstrokeopacity{0.925857}%
\pgfsetdash{}{0pt}%
\pgfpathmoveto{\pgfqpoint{1.608248in}{1.974838in}}%
\pgfpathcurveto{\pgfqpoint{1.616484in}{1.974838in}}{\pgfqpoint{1.624384in}{1.978110in}}{\pgfqpoint{1.630208in}{1.983934in}}%
\pgfpathcurveto{\pgfqpoint{1.636032in}{1.989758in}}{\pgfqpoint{1.639304in}{1.997658in}}{\pgfqpoint{1.639304in}{2.005894in}}%
\pgfpathcurveto{\pgfqpoint{1.639304in}{2.014130in}}{\pgfqpoint{1.636032in}{2.022031in}}{\pgfqpoint{1.630208in}{2.027854in}}%
\pgfpathcurveto{\pgfqpoint{1.624384in}{2.033678in}}{\pgfqpoint{1.616484in}{2.036951in}}{\pgfqpoint{1.608248in}{2.036951in}}%
\pgfpathcurveto{\pgfqpoint{1.600011in}{2.036951in}}{\pgfqpoint{1.592111in}{2.033678in}}{\pgfqpoint{1.586287in}{2.027854in}}%
\pgfpathcurveto{\pgfqpoint{1.580463in}{2.022031in}}{\pgfqpoint{1.577191in}{2.014130in}}{\pgfqpoint{1.577191in}{2.005894in}}%
\pgfpathcurveto{\pgfqpoint{1.577191in}{1.997658in}}{\pgfqpoint{1.580463in}{1.989758in}}{\pgfqpoint{1.586287in}{1.983934in}}%
\pgfpathcurveto{\pgfqpoint{1.592111in}{1.978110in}}{\pgfqpoint{1.600011in}{1.974838in}}{\pgfqpoint{1.608248in}{1.974838in}}%
\pgfpathclose%
\pgfusepath{stroke,fill}%
\end{pgfscope}%
\begin{pgfscope}%
\pgfpathrectangle{\pgfqpoint{0.100000in}{0.212622in}}{\pgfqpoint{3.696000in}{3.696000in}}%
\pgfusepath{clip}%
\pgfsetbuttcap%
\pgfsetroundjoin%
\definecolor{currentfill}{rgb}{0.121569,0.466667,0.705882}%
\pgfsetfillcolor{currentfill}%
\pgfsetfillopacity{0.925897}%
\pgfsetlinewidth{1.003750pt}%
\definecolor{currentstroke}{rgb}{0.121569,0.466667,0.705882}%
\pgfsetstrokecolor{currentstroke}%
\pgfsetstrokeopacity{0.925897}%
\pgfsetdash{}{0pt}%
\pgfpathmoveto{\pgfqpoint{2.585532in}{1.652411in}}%
\pgfpathcurveto{\pgfqpoint{2.593768in}{1.652411in}}{\pgfqpoint{2.601668in}{1.655683in}}{\pgfqpoint{2.607492in}{1.661507in}}%
\pgfpathcurveto{\pgfqpoint{2.613316in}{1.667331in}}{\pgfqpoint{2.616588in}{1.675231in}}{\pgfqpoint{2.616588in}{1.683468in}}%
\pgfpathcurveto{\pgfqpoint{2.616588in}{1.691704in}}{\pgfqpoint{2.613316in}{1.699604in}}{\pgfqpoint{2.607492in}{1.705428in}}%
\pgfpathcurveto{\pgfqpoint{2.601668in}{1.711252in}}{\pgfqpoint{2.593768in}{1.714524in}}{\pgfqpoint{2.585532in}{1.714524in}}%
\pgfpathcurveto{\pgfqpoint{2.577295in}{1.714524in}}{\pgfqpoint{2.569395in}{1.711252in}}{\pgfqpoint{2.563571in}{1.705428in}}%
\pgfpathcurveto{\pgfqpoint{2.557748in}{1.699604in}}{\pgfqpoint{2.554475in}{1.691704in}}{\pgfqpoint{2.554475in}{1.683468in}}%
\pgfpathcurveto{\pgfqpoint{2.554475in}{1.675231in}}{\pgfqpoint{2.557748in}{1.667331in}}{\pgfqpoint{2.563571in}{1.661507in}}%
\pgfpathcurveto{\pgfqpoint{2.569395in}{1.655683in}}{\pgfqpoint{2.577295in}{1.652411in}}{\pgfqpoint{2.585532in}{1.652411in}}%
\pgfpathclose%
\pgfusepath{stroke,fill}%
\end{pgfscope}%
\begin{pgfscope}%
\pgfpathrectangle{\pgfqpoint{0.100000in}{0.212622in}}{\pgfqpoint{3.696000in}{3.696000in}}%
\pgfusepath{clip}%
\pgfsetbuttcap%
\pgfsetroundjoin%
\definecolor{currentfill}{rgb}{0.121569,0.466667,0.705882}%
\pgfsetfillcolor{currentfill}%
\pgfsetfillopacity{0.927056}%
\pgfsetlinewidth{1.003750pt}%
\definecolor{currentstroke}{rgb}{0.121569,0.466667,0.705882}%
\pgfsetstrokecolor{currentstroke}%
\pgfsetstrokeopacity{0.927056}%
\pgfsetdash{}{0pt}%
\pgfpathmoveto{\pgfqpoint{1.622747in}{1.967363in}}%
\pgfpathcurveto{\pgfqpoint{1.630983in}{1.967363in}}{\pgfqpoint{1.638883in}{1.970636in}}{\pgfqpoint{1.644707in}{1.976460in}}%
\pgfpathcurveto{\pgfqpoint{1.650531in}{1.982284in}}{\pgfqpoint{1.653803in}{1.990184in}}{\pgfqpoint{1.653803in}{1.998420in}}%
\pgfpathcurveto{\pgfqpoint{1.653803in}{2.006656in}}{\pgfqpoint{1.650531in}{2.014556in}}{\pgfqpoint{1.644707in}{2.020380in}}%
\pgfpathcurveto{\pgfqpoint{1.638883in}{2.026204in}}{\pgfqpoint{1.630983in}{2.029476in}}{\pgfqpoint{1.622747in}{2.029476in}}%
\pgfpathcurveto{\pgfqpoint{1.614510in}{2.029476in}}{\pgfqpoint{1.606610in}{2.026204in}}{\pgfqpoint{1.600786in}{2.020380in}}%
\pgfpathcurveto{\pgfqpoint{1.594962in}{2.014556in}}{\pgfqpoint{1.591690in}{2.006656in}}{\pgfqpoint{1.591690in}{1.998420in}}%
\pgfpathcurveto{\pgfqpoint{1.591690in}{1.990184in}}{\pgfqpoint{1.594962in}{1.982284in}}{\pgfqpoint{1.600786in}{1.976460in}}%
\pgfpathcurveto{\pgfqpoint{1.606610in}{1.970636in}}{\pgfqpoint{1.614510in}{1.967363in}}{\pgfqpoint{1.622747in}{1.967363in}}%
\pgfpathclose%
\pgfusepath{stroke,fill}%
\end{pgfscope}%
\begin{pgfscope}%
\pgfpathrectangle{\pgfqpoint{0.100000in}{0.212622in}}{\pgfqpoint{3.696000in}{3.696000in}}%
\pgfusepath{clip}%
\pgfsetbuttcap%
\pgfsetroundjoin%
\definecolor{currentfill}{rgb}{0.121569,0.466667,0.705882}%
\pgfsetfillcolor{currentfill}%
\pgfsetfillopacity{0.927525}%
\pgfsetlinewidth{1.003750pt}%
\definecolor{currentstroke}{rgb}{0.121569,0.466667,0.705882}%
\pgfsetstrokecolor{currentstroke}%
\pgfsetstrokeopacity{0.927525}%
\pgfsetdash{}{0pt}%
\pgfpathmoveto{\pgfqpoint{2.580664in}{1.653052in}}%
\pgfpathcurveto{\pgfqpoint{2.588900in}{1.653052in}}{\pgfqpoint{2.596800in}{1.656324in}}{\pgfqpoint{2.602624in}{1.662148in}}%
\pgfpathcurveto{\pgfqpoint{2.608448in}{1.667972in}}{\pgfqpoint{2.611720in}{1.675872in}}{\pgfqpoint{2.611720in}{1.684108in}}%
\pgfpathcurveto{\pgfqpoint{2.611720in}{1.692344in}}{\pgfqpoint{2.608448in}{1.700244in}}{\pgfqpoint{2.602624in}{1.706068in}}%
\pgfpathcurveto{\pgfqpoint{2.596800in}{1.711892in}}{\pgfqpoint{2.588900in}{1.715165in}}{\pgfqpoint{2.580664in}{1.715165in}}%
\pgfpathcurveto{\pgfqpoint{2.572428in}{1.715165in}}{\pgfqpoint{2.564528in}{1.711892in}}{\pgfqpoint{2.558704in}{1.706068in}}%
\pgfpathcurveto{\pgfqpoint{2.552880in}{1.700244in}}{\pgfqpoint{2.549607in}{1.692344in}}{\pgfqpoint{2.549607in}{1.684108in}}%
\pgfpathcurveto{\pgfqpoint{2.549607in}{1.675872in}}{\pgfqpoint{2.552880in}{1.667972in}}{\pgfqpoint{2.558704in}{1.662148in}}%
\pgfpathcurveto{\pgfqpoint{2.564528in}{1.656324in}}{\pgfqpoint{2.572428in}{1.653052in}}{\pgfqpoint{2.580664in}{1.653052in}}%
\pgfpathclose%
\pgfusepath{stroke,fill}%
\end{pgfscope}%
\begin{pgfscope}%
\pgfpathrectangle{\pgfqpoint{0.100000in}{0.212622in}}{\pgfqpoint{3.696000in}{3.696000in}}%
\pgfusepath{clip}%
\pgfsetbuttcap%
\pgfsetroundjoin%
\definecolor{currentfill}{rgb}{0.121569,0.466667,0.705882}%
\pgfsetfillcolor{currentfill}%
\pgfsetfillopacity{0.928307}%
\pgfsetlinewidth{1.003750pt}%
\definecolor{currentstroke}{rgb}{0.121569,0.466667,0.705882}%
\pgfsetstrokecolor{currentstroke}%
\pgfsetstrokeopacity{0.928307}%
\pgfsetdash{}{0pt}%
\pgfpathmoveto{\pgfqpoint{1.635217in}{1.962312in}}%
\pgfpathcurveto{\pgfqpoint{1.643453in}{1.962312in}}{\pgfqpoint{1.651353in}{1.965584in}}{\pgfqpoint{1.657177in}{1.971408in}}%
\pgfpathcurveto{\pgfqpoint{1.663001in}{1.977232in}}{\pgfqpoint{1.666273in}{1.985132in}}{\pgfqpoint{1.666273in}{1.993368in}}%
\pgfpathcurveto{\pgfqpoint{1.666273in}{2.001604in}}{\pgfqpoint{1.663001in}{2.009504in}}{\pgfqpoint{1.657177in}{2.015328in}}%
\pgfpathcurveto{\pgfqpoint{1.651353in}{2.021152in}}{\pgfqpoint{1.643453in}{2.024425in}}{\pgfqpoint{1.635217in}{2.024425in}}%
\pgfpathcurveto{\pgfqpoint{1.626981in}{2.024425in}}{\pgfqpoint{1.619081in}{2.021152in}}{\pgfqpoint{1.613257in}{2.015328in}}%
\pgfpathcurveto{\pgfqpoint{1.607433in}{2.009504in}}{\pgfqpoint{1.604160in}{2.001604in}}{\pgfqpoint{1.604160in}{1.993368in}}%
\pgfpathcurveto{\pgfqpoint{1.604160in}{1.985132in}}{\pgfqpoint{1.607433in}{1.977232in}}{\pgfqpoint{1.613257in}{1.971408in}}%
\pgfpathcurveto{\pgfqpoint{1.619081in}{1.965584in}}{\pgfqpoint{1.626981in}{1.962312in}}{\pgfqpoint{1.635217in}{1.962312in}}%
\pgfpathclose%
\pgfusepath{stroke,fill}%
\end{pgfscope}%
\begin{pgfscope}%
\pgfpathrectangle{\pgfqpoint{0.100000in}{0.212622in}}{\pgfqpoint{3.696000in}{3.696000in}}%
\pgfusepath{clip}%
\pgfsetbuttcap%
\pgfsetroundjoin%
\definecolor{currentfill}{rgb}{0.121569,0.466667,0.705882}%
\pgfsetfillcolor{currentfill}%
\pgfsetfillopacity{0.929101}%
\pgfsetlinewidth{1.003750pt}%
\definecolor{currentstroke}{rgb}{0.121569,0.466667,0.705882}%
\pgfsetstrokecolor{currentstroke}%
\pgfsetstrokeopacity{0.929101}%
\pgfsetdash{}{0pt}%
\pgfpathmoveto{\pgfqpoint{1.646858in}{1.955007in}}%
\pgfpathcurveto{\pgfqpoint{1.655094in}{1.955007in}}{\pgfqpoint{1.662994in}{1.958279in}}{\pgfqpoint{1.668818in}{1.964103in}}%
\pgfpathcurveto{\pgfqpoint{1.674642in}{1.969927in}}{\pgfqpoint{1.677914in}{1.977827in}}{\pgfqpoint{1.677914in}{1.986063in}}%
\pgfpathcurveto{\pgfqpoint{1.677914in}{1.994300in}}{\pgfqpoint{1.674642in}{2.002200in}}{\pgfqpoint{1.668818in}{2.008024in}}%
\pgfpathcurveto{\pgfqpoint{1.662994in}{2.013848in}}{\pgfqpoint{1.655094in}{2.017120in}}{\pgfqpoint{1.646858in}{2.017120in}}%
\pgfpathcurveto{\pgfqpoint{1.638622in}{2.017120in}}{\pgfqpoint{1.630722in}{2.013848in}}{\pgfqpoint{1.624898in}{2.008024in}}%
\pgfpathcurveto{\pgfqpoint{1.619074in}{2.002200in}}{\pgfqpoint{1.615801in}{1.994300in}}{\pgfqpoint{1.615801in}{1.986063in}}%
\pgfpathcurveto{\pgfqpoint{1.615801in}{1.977827in}}{\pgfqpoint{1.619074in}{1.969927in}}{\pgfqpoint{1.624898in}{1.964103in}}%
\pgfpathcurveto{\pgfqpoint{1.630722in}{1.958279in}}{\pgfqpoint{1.638622in}{1.955007in}}{\pgfqpoint{1.646858in}{1.955007in}}%
\pgfpathclose%
\pgfusepath{stroke,fill}%
\end{pgfscope}%
\begin{pgfscope}%
\pgfpathrectangle{\pgfqpoint{0.100000in}{0.212622in}}{\pgfqpoint{3.696000in}{3.696000in}}%
\pgfusepath{clip}%
\pgfsetbuttcap%
\pgfsetroundjoin%
\definecolor{currentfill}{rgb}{0.121569,0.466667,0.705882}%
\pgfsetfillcolor{currentfill}%
\pgfsetfillopacity{0.929582}%
\pgfsetlinewidth{1.003750pt}%
\definecolor{currentstroke}{rgb}{0.121569,0.466667,0.705882}%
\pgfsetstrokecolor{currentstroke}%
\pgfsetstrokeopacity{0.929582}%
\pgfsetdash{}{0pt}%
\pgfpathmoveto{\pgfqpoint{1.657706in}{1.944946in}}%
\pgfpathcurveto{\pgfqpoint{1.665943in}{1.944946in}}{\pgfqpoint{1.673843in}{1.948219in}}{\pgfqpoint{1.679667in}{1.954043in}}%
\pgfpathcurveto{\pgfqpoint{1.685491in}{1.959867in}}{\pgfqpoint{1.688763in}{1.967767in}}{\pgfqpoint{1.688763in}{1.976003in}}%
\pgfpathcurveto{\pgfqpoint{1.688763in}{1.984239in}}{\pgfqpoint{1.685491in}{1.992139in}}{\pgfqpoint{1.679667in}{1.997963in}}%
\pgfpathcurveto{\pgfqpoint{1.673843in}{2.003787in}}{\pgfqpoint{1.665943in}{2.007059in}}{\pgfqpoint{1.657706in}{2.007059in}}%
\pgfpathcurveto{\pgfqpoint{1.649470in}{2.007059in}}{\pgfqpoint{1.641570in}{2.003787in}}{\pgfqpoint{1.635746in}{1.997963in}}%
\pgfpathcurveto{\pgfqpoint{1.629922in}{1.992139in}}{\pgfqpoint{1.626650in}{1.984239in}}{\pgfqpoint{1.626650in}{1.976003in}}%
\pgfpathcurveto{\pgfqpoint{1.626650in}{1.967767in}}{\pgfqpoint{1.629922in}{1.959867in}}{\pgfqpoint{1.635746in}{1.954043in}}%
\pgfpathcurveto{\pgfqpoint{1.641570in}{1.948219in}}{\pgfqpoint{1.649470in}{1.944946in}}{\pgfqpoint{1.657706in}{1.944946in}}%
\pgfpathclose%
\pgfusepath{stroke,fill}%
\end{pgfscope}%
\begin{pgfscope}%
\pgfpathrectangle{\pgfqpoint{0.100000in}{0.212622in}}{\pgfqpoint{3.696000in}{3.696000in}}%
\pgfusepath{clip}%
\pgfsetbuttcap%
\pgfsetroundjoin%
\definecolor{currentfill}{rgb}{0.121569,0.466667,0.705882}%
\pgfsetfillcolor{currentfill}%
\pgfsetfillopacity{0.929706}%
\pgfsetlinewidth{1.003750pt}%
\definecolor{currentstroke}{rgb}{0.121569,0.466667,0.705882}%
\pgfsetstrokecolor{currentstroke}%
\pgfsetstrokeopacity{0.929706}%
\pgfsetdash{}{0pt}%
\pgfpathmoveto{\pgfqpoint{2.579010in}{1.651269in}}%
\pgfpathcurveto{\pgfqpoint{2.587246in}{1.651269in}}{\pgfqpoint{2.595146in}{1.654541in}}{\pgfqpoint{2.600970in}{1.660365in}}%
\pgfpathcurveto{\pgfqpoint{2.606794in}{1.666189in}}{\pgfqpoint{2.610066in}{1.674089in}}{\pgfqpoint{2.610066in}{1.682325in}}%
\pgfpathcurveto{\pgfqpoint{2.610066in}{1.690561in}}{\pgfqpoint{2.606794in}{1.698461in}}{\pgfqpoint{2.600970in}{1.704285in}}%
\pgfpathcurveto{\pgfqpoint{2.595146in}{1.710109in}}{\pgfqpoint{2.587246in}{1.713382in}}{\pgfqpoint{2.579010in}{1.713382in}}%
\pgfpathcurveto{\pgfqpoint{2.570773in}{1.713382in}}{\pgfqpoint{2.562873in}{1.710109in}}{\pgfqpoint{2.557049in}{1.704285in}}%
\pgfpathcurveto{\pgfqpoint{2.551225in}{1.698461in}}{\pgfqpoint{2.547953in}{1.690561in}}{\pgfqpoint{2.547953in}{1.682325in}}%
\pgfpathcurveto{\pgfqpoint{2.547953in}{1.674089in}}{\pgfqpoint{2.551225in}{1.666189in}}{\pgfqpoint{2.557049in}{1.660365in}}%
\pgfpathcurveto{\pgfqpoint{2.562873in}{1.654541in}}{\pgfqpoint{2.570773in}{1.651269in}}{\pgfqpoint{2.579010in}{1.651269in}}%
\pgfpathclose%
\pgfusepath{stroke,fill}%
\end{pgfscope}%
\begin{pgfscope}%
\pgfpathrectangle{\pgfqpoint{0.100000in}{0.212622in}}{\pgfqpoint{3.696000in}{3.696000in}}%
\pgfusepath{clip}%
\pgfsetbuttcap%
\pgfsetroundjoin%
\definecolor{currentfill}{rgb}{0.121569,0.466667,0.705882}%
\pgfsetfillcolor{currentfill}%
\pgfsetfillopacity{0.930062}%
\pgfsetlinewidth{1.003750pt}%
\definecolor{currentstroke}{rgb}{0.121569,0.466667,0.705882}%
\pgfsetstrokecolor{currentstroke}%
\pgfsetstrokeopacity{0.930062}%
\pgfsetdash{}{0pt}%
\pgfpathmoveto{\pgfqpoint{1.666466in}{1.939944in}}%
\pgfpathcurveto{\pgfqpoint{1.674702in}{1.939944in}}{\pgfqpoint{1.682602in}{1.943216in}}{\pgfqpoint{1.688426in}{1.949040in}}%
\pgfpathcurveto{\pgfqpoint{1.694250in}{1.954864in}}{\pgfqpoint{1.697522in}{1.962764in}}{\pgfqpoint{1.697522in}{1.971001in}}%
\pgfpathcurveto{\pgfqpoint{1.697522in}{1.979237in}}{\pgfqpoint{1.694250in}{1.987137in}}{\pgfqpoint{1.688426in}{1.992961in}}%
\pgfpathcurveto{\pgfqpoint{1.682602in}{1.998785in}}{\pgfqpoint{1.674702in}{2.002057in}}{\pgfqpoint{1.666466in}{2.002057in}}%
\pgfpathcurveto{\pgfqpoint{1.658229in}{2.002057in}}{\pgfqpoint{1.650329in}{1.998785in}}{\pgfqpoint{1.644505in}{1.992961in}}%
\pgfpathcurveto{\pgfqpoint{1.638681in}{1.987137in}}{\pgfqpoint{1.635409in}{1.979237in}}{\pgfqpoint{1.635409in}{1.971001in}}%
\pgfpathcurveto{\pgfqpoint{1.635409in}{1.962764in}}{\pgfqpoint{1.638681in}{1.954864in}}{\pgfqpoint{1.644505in}{1.949040in}}%
\pgfpathcurveto{\pgfqpoint{1.650329in}{1.943216in}}{\pgfqpoint{1.658229in}{1.939944in}}{\pgfqpoint{1.666466in}{1.939944in}}%
\pgfpathclose%
\pgfusepath{stroke,fill}%
\end{pgfscope}%
\begin{pgfscope}%
\pgfpathrectangle{\pgfqpoint{0.100000in}{0.212622in}}{\pgfqpoint{3.696000in}{3.696000in}}%
\pgfusepath{clip}%
\pgfsetbuttcap%
\pgfsetroundjoin%
\definecolor{currentfill}{rgb}{0.121569,0.466667,0.705882}%
\pgfsetfillcolor{currentfill}%
\pgfsetfillopacity{0.930469}%
\pgfsetlinewidth{1.003750pt}%
\definecolor{currentstroke}{rgb}{0.121569,0.466667,0.705882}%
\pgfsetstrokecolor{currentstroke}%
\pgfsetstrokeopacity{0.930469}%
\pgfsetdash{}{0pt}%
\pgfpathmoveto{\pgfqpoint{1.672366in}{1.934852in}}%
\pgfpathcurveto{\pgfqpoint{1.680602in}{1.934852in}}{\pgfqpoint{1.688502in}{1.938125in}}{\pgfqpoint{1.694326in}{1.943949in}}%
\pgfpathcurveto{\pgfqpoint{1.700150in}{1.949773in}}{\pgfqpoint{1.703422in}{1.957673in}}{\pgfqpoint{1.703422in}{1.965909in}}%
\pgfpathcurveto{\pgfqpoint{1.703422in}{1.974145in}}{\pgfqpoint{1.700150in}{1.982045in}}{\pgfqpoint{1.694326in}{1.987869in}}%
\pgfpathcurveto{\pgfqpoint{1.688502in}{1.993693in}}{\pgfqpoint{1.680602in}{1.996965in}}{\pgfqpoint{1.672366in}{1.996965in}}%
\pgfpathcurveto{\pgfqpoint{1.664129in}{1.996965in}}{\pgfqpoint{1.656229in}{1.993693in}}{\pgfqpoint{1.650405in}{1.987869in}}%
\pgfpathcurveto{\pgfqpoint{1.644582in}{1.982045in}}{\pgfqpoint{1.641309in}{1.974145in}}{\pgfqpoint{1.641309in}{1.965909in}}%
\pgfpathcurveto{\pgfqpoint{1.641309in}{1.957673in}}{\pgfqpoint{1.644582in}{1.949773in}}{\pgfqpoint{1.650405in}{1.943949in}}%
\pgfpathcurveto{\pgfqpoint{1.656229in}{1.938125in}}{\pgfqpoint{1.664129in}{1.934852in}}{\pgfqpoint{1.672366in}{1.934852in}}%
\pgfpathclose%
\pgfusepath{stroke,fill}%
\end{pgfscope}%
\begin{pgfscope}%
\pgfpathrectangle{\pgfqpoint{0.100000in}{0.212622in}}{\pgfqpoint{3.696000in}{3.696000in}}%
\pgfusepath{clip}%
\pgfsetbuttcap%
\pgfsetroundjoin%
\definecolor{currentfill}{rgb}{0.121569,0.466667,0.705882}%
\pgfsetfillcolor{currentfill}%
\pgfsetfillopacity{0.930947}%
\pgfsetlinewidth{1.003750pt}%
\definecolor{currentstroke}{rgb}{0.121569,0.466667,0.705882}%
\pgfsetstrokecolor{currentstroke}%
\pgfsetstrokeopacity{0.930947}%
\pgfsetdash{}{0pt}%
\pgfpathmoveto{\pgfqpoint{2.575123in}{1.644121in}}%
\pgfpathcurveto{\pgfqpoint{2.583359in}{1.644121in}}{\pgfqpoint{2.591259in}{1.647393in}}{\pgfqpoint{2.597083in}{1.653217in}}%
\pgfpathcurveto{\pgfqpoint{2.602907in}{1.659041in}}{\pgfqpoint{2.606179in}{1.666941in}}{\pgfqpoint{2.606179in}{1.675177in}}%
\pgfpathcurveto{\pgfqpoint{2.606179in}{1.683414in}}{\pgfqpoint{2.602907in}{1.691314in}}{\pgfqpoint{2.597083in}{1.697138in}}%
\pgfpathcurveto{\pgfqpoint{2.591259in}{1.702962in}}{\pgfqpoint{2.583359in}{1.706234in}}{\pgfqpoint{2.575123in}{1.706234in}}%
\pgfpathcurveto{\pgfqpoint{2.566887in}{1.706234in}}{\pgfqpoint{2.558986in}{1.702962in}}{\pgfqpoint{2.553163in}{1.697138in}}%
\pgfpathcurveto{\pgfqpoint{2.547339in}{1.691314in}}{\pgfqpoint{2.544066in}{1.683414in}}{\pgfqpoint{2.544066in}{1.675177in}}%
\pgfpathcurveto{\pgfqpoint{2.544066in}{1.666941in}}{\pgfqpoint{2.547339in}{1.659041in}}{\pgfqpoint{2.553163in}{1.653217in}}%
\pgfpathcurveto{\pgfqpoint{2.558986in}{1.647393in}}{\pgfqpoint{2.566887in}{1.644121in}}{\pgfqpoint{2.575123in}{1.644121in}}%
\pgfpathclose%
\pgfusepath{stroke,fill}%
\end{pgfscope}%
\begin{pgfscope}%
\pgfpathrectangle{\pgfqpoint{0.100000in}{0.212622in}}{\pgfqpoint{3.696000in}{3.696000in}}%
\pgfusepath{clip}%
\pgfsetbuttcap%
\pgfsetroundjoin%
\definecolor{currentfill}{rgb}{0.121569,0.466667,0.705882}%
\pgfsetfillcolor{currentfill}%
\pgfsetfillopacity{0.931343}%
\pgfsetlinewidth{1.003750pt}%
\definecolor{currentstroke}{rgb}{0.121569,0.466667,0.705882}%
\pgfsetstrokecolor{currentstroke}%
\pgfsetstrokeopacity{0.931343}%
\pgfsetdash{}{0pt}%
\pgfpathmoveto{\pgfqpoint{1.683477in}{1.927739in}}%
\pgfpathcurveto{\pgfqpoint{1.691713in}{1.927739in}}{\pgfqpoint{1.699613in}{1.931011in}}{\pgfqpoint{1.705437in}{1.936835in}}%
\pgfpathcurveto{\pgfqpoint{1.711261in}{1.942659in}}{\pgfqpoint{1.714533in}{1.950559in}}{\pgfqpoint{1.714533in}{1.958795in}}%
\pgfpathcurveto{\pgfqpoint{1.714533in}{1.967031in}}{\pgfqpoint{1.711261in}{1.974931in}}{\pgfqpoint{1.705437in}{1.980755in}}%
\pgfpathcurveto{\pgfqpoint{1.699613in}{1.986579in}}{\pgfqpoint{1.691713in}{1.989852in}}{\pgfqpoint{1.683477in}{1.989852in}}%
\pgfpathcurveto{\pgfqpoint{1.675240in}{1.989852in}}{\pgfqpoint{1.667340in}{1.986579in}}{\pgfqpoint{1.661516in}{1.980755in}}%
\pgfpathcurveto{\pgfqpoint{1.655692in}{1.974931in}}{\pgfqpoint{1.652420in}{1.967031in}}{\pgfqpoint{1.652420in}{1.958795in}}%
\pgfpathcurveto{\pgfqpoint{1.652420in}{1.950559in}}{\pgfqpoint{1.655692in}{1.942659in}}{\pgfqpoint{1.661516in}{1.936835in}}%
\pgfpathcurveto{\pgfqpoint{1.667340in}{1.931011in}}{\pgfqpoint{1.675240in}{1.927739in}}{\pgfqpoint{1.683477in}{1.927739in}}%
\pgfpathclose%
\pgfusepath{stroke,fill}%
\end{pgfscope}%
\begin{pgfscope}%
\pgfpathrectangle{\pgfqpoint{0.100000in}{0.212622in}}{\pgfqpoint{3.696000in}{3.696000in}}%
\pgfusepath{clip}%
\pgfsetbuttcap%
\pgfsetroundjoin%
\definecolor{currentfill}{rgb}{0.121569,0.466667,0.705882}%
\pgfsetfillcolor{currentfill}%
\pgfsetfillopacity{0.932207}%
\pgfsetlinewidth{1.003750pt}%
\definecolor{currentstroke}{rgb}{0.121569,0.466667,0.705882}%
\pgfsetstrokecolor{currentstroke}%
\pgfsetstrokeopacity{0.932207}%
\pgfsetdash{}{0pt}%
\pgfpathmoveto{\pgfqpoint{1.693633in}{1.924107in}}%
\pgfpathcurveto{\pgfqpoint{1.701869in}{1.924107in}}{\pgfqpoint{1.709769in}{1.927380in}}{\pgfqpoint{1.715593in}{1.933204in}}%
\pgfpathcurveto{\pgfqpoint{1.721417in}{1.939027in}}{\pgfqpoint{1.724689in}{1.946928in}}{\pgfqpoint{1.724689in}{1.955164in}}%
\pgfpathcurveto{\pgfqpoint{1.724689in}{1.963400in}}{\pgfqpoint{1.721417in}{1.971300in}}{\pgfqpoint{1.715593in}{1.977124in}}%
\pgfpathcurveto{\pgfqpoint{1.709769in}{1.982948in}}{\pgfqpoint{1.701869in}{1.986220in}}{\pgfqpoint{1.693633in}{1.986220in}}%
\pgfpathcurveto{\pgfqpoint{1.685396in}{1.986220in}}{\pgfqpoint{1.677496in}{1.982948in}}{\pgfqpoint{1.671672in}{1.977124in}}%
\pgfpathcurveto{\pgfqpoint{1.665848in}{1.971300in}}{\pgfqpoint{1.662576in}{1.963400in}}{\pgfqpoint{1.662576in}{1.955164in}}%
\pgfpathcurveto{\pgfqpoint{1.662576in}{1.946928in}}{\pgfqpoint{1.665848in}{1.939027in}}{\pgfqpoint{1.671672in}{1.933204in}}%
\pgfpathcurveto{\pgfqpoint{1.677496in}{1.927380in}}{\pgfqpoint{1.685396in}{1.924107in}}{\pgfqpoint{1.693633in}{1.924107in}}%
\pgfpathclose%
\pgfusepath{stroke,fill}%
\end{pgfscope}%
\begin{pgfscope}%
\pgfpathrectangle{\pgfqpoint{0.100000in}{0.212622in}}{\pgfqpoint{3.696000in}{3.696000in}}%
\pgfusepath{clip}%
\pgfsetbuttcap%
\pgfsetroundjoin%
\definecolor{currentfill}{rgb}{0.121569,0.466667,0.705882}%
\pgfsetfillcolor{currentfill}%
\pgfsetfillopacity{0.932475}%
\pgfsetlinewidth{1.003750pt}%
\definecolor{currentstroke}{rgb}{0.121569,0.466667,0.705882}%
\pgfsetstrokecolor{currentstroke}%
\pgfsetstrokeopacity{0.932475}%
\pgfsetdash{}{0pt}%
\pgfpathmoveto{\pgfqpoint{1.700097in}{1.916736in}}%
\pgfpathcurveto{\pgfqpoint{1.708333in}{1.916736in}}{\pgfqpoint{1.716233in}{1.920008in}}{\pgfqpoint{1.722057in}{1.925832in}}%
\pgfpathcurveto{\pgfqpoint{1.727881in}{1.931656in}}{\pgfqpoint{1.731153in}{1.939556in}}{\pgfqpoint{1.731153in}{1.947793in}}%
\pgfpathcurveto{\pgfqpoint{1.731153in}{1.956029in}}{\pgfqpoint{1.727881in}{1.963929in}}{\pgfqpoint{1.722057in}{1.969753in}}%
\pgfpathcurveto{\pgfqpoint{1.716233in}{1.975577in}}{\pgfqpoint{1.708333in}{1.978849in}}{\pgfqpoint{1.700097in}{1.978849in}}%
\pgfpathcurveto{\pgfqpoint{1.691861in}{1.978849in}}{\pgfqpoint{1.683961in}{1.975577in}}{\pgfqpoint{1.678137in}{1.969753in}}%
\pgfpathcurveto{\pgfqpoint{1.672313in}{1.963929in}}{\pgfqpoint{1.669040in}{1.956029in}}{\pgfqpoint{1.669040in}{1.947793in}}%
\pgfpathcurveto{\pgfqpoint{1.669040in}{1.939556in}}{\pgfqpoint{1.672313in}{1.931656in}}{\pgfqpoint{1.678137in}{1.925832in}}%
\pgfpathcurveto{\pgfqpoint{1.683961in}{1.920008in}}{\pgfqpoint{1.691861in}{1.916736in}}{\pgfqpoint{1.700097in}{1.916736in}}%
\pgfpathclose%
\pgfusepath{stroke,fill}%
\end{pgfscope}%
\begin{pgfscope}%
\pgfpathrectangle{\pgfqpoint{0.100000in}{0.212622in}}{\pgfqpoint{3.696000in}{3.696000in}}%
\pgfusepath{clip}%
\pgfsetbuttcap%
\pgfsetroundjoin%
\definecolor{currentfill}{rgb}{0.121569,0.466667,0.705882}%
\pgfsetfillcolor{currentfill}%
\pgfsetfillopacity{0.932944}%
\pgfsetlinewidth{1.003750pt}%
\definecolor{currentstroke}{rgb}{0.121569,0.466667,0.705882}%
\pgfsetstrokecolor{currentstroke}%
\pgfsetstrokeopacity{0.932944}%
\pgfsetdash{}{0pt}%
\pgfpathmoveto{\pgfqpoint{1.705357in}{1.913374in}}%
\pgfpathcurveto{\pgfqpoint{1.713593in}{1.913374in}}{\pgfqpoint{1.721493in}{1.916646in}}{\pgfqpoint{1.727317in}{1.922470in}}%
\pgfpathcurveto{\pgfqpoint{1.733141in}{1.928294in}}{\pgfqpoint{1.736414in}{1.936194in}}{\pgfqpoint{1.736414in}{1.944430in}}%
\pgfpathcurveto{\pgfqpoint{1.736414in}{1.952666in}}{\pgfqpoint{1.733141in}{1.960567in}}{\pgfqpoint{1.727317in}{1.966390in}}%
\pgfpathcurveto{\pgfqpoint{1.721493in}{1.972214in}}{\pgfqpoint{1.713593in}{1.975487in}}{\pgfqpoint{1.705357in}{1.975487in}}%
\pgfpathcurveto{\pgfqpoint{1.697121in}{1.975487in}}{\pgfqpoint{1.689221in}{1.972214in}}{\pgfqpoint{1.683397in}{1.966390in}}%
\pgfpathcurveto{\pgfqpoint{1.677573in}{1.960567in}}{\pgfqpoint{1.674301in}{1.952666in}}{\pgfqpoint{1.674301in}{1.944430in}}%
\pgfpathcurveto{\pgfqpoint{1.674301in}{1.936194in}}{\pgfqpoint{1.677573in}{1.928294in}}{\pgfqpoint{1.683397in}{1.922470in}}%
\pgfpathcurveto{\pgfqpoint{1.689221in}{1.916646in}}{\pgfqpoint{1.697121in}{1.913374in}}{\pgfqpoint{1.705357in}{1.913374in}}%
\pgfpathclose%
\pgfusepath{stroke,fill}%
\end{pgfscope}%
\begin{pgfscope}%
\pgfpathrectangle{\pgfqpoint{0.100000in}{0.212622in}}{\pgfqpoint{3.696000in}{3.696000in}}%
\pgfusepath{clip}%
\pgfsetbuttcap%
\pgfsetroundjoin%
\definecolor{currentfill}{rgb}{0.121569,0.466667,0.705882}%
\pgfsetfillcolor{currentfill}%
\pgfsetfillopacity{0.933573}%
\pgfsetlinewidth{1.003750pt}%
\definecolor{currentstroke}{rgb}{0.121569,0.466667,0.705882}%
\pgfsetstrokecolor{currentstroke}%
\pgfsetstrokeopacity{0.933573}%
\pgfsetdash{}{0pt}%
\pgfpathmoveto{\pgfqpoint{2.567981in}{1.648264in}}%
\pgfpathcurveto{\pgfqpoint{2.576218in}{1.648264in}}{\pgfqpoint{2.584118in}{1.651536in}}{\pgfqpoint{2.589942in}{1.657360in}}%
\pgfpathcurveto{\pgfqpoint{2.595765in}{1.663184in}}{\pgfqpoint{2.599038in}{1.671084in}}{\pgfqpoint{2.599038in}{1.679320in}}%
\pgfpathcurveto{\pgfqpoint{2.599038in}{1.687557in}}{\pgfqpoint{2.595765in}{1.695457in}}{\pgfqpoint{2.589942in}{1.701281in}}%
\pgfpathcurveto{\pgfqpoint{2.584118in}{1.707105in}}{\pgfqpoint{2.576218in}{1.710377in}}{\pgfqpoint{2.567981in}{1.710377in}}%
\pgfpathcurveto{\pgfqpoint{2.559745in}{1.710377in}}{\pgfqpoint{2.551845in}{1.707105in}}{\pgfqpoint{2.546021in}{1.701281in}}%
\pgfpathcurveto{\pgfqpoint{2.540197in}{1.695457in}}{\pgfqpoint{2.536925in}{1.687557in}}{\pgfqpoint{2.536925in}{1.679320in}}%
\pgfpathcurveto{\pgfqpoint{2.536925in}{1.671084in}}{\pgfqpoint{2.540197in}{1.663184in}}{\pgfqpoint{2.546021in}{1.657360in}}%
\pgfpathcurveto{\pgfqpoint{2.551845in}{1.651536in}}{\pgfqpoint{2.559745in}{1.648264in}}{\pgfqpoint{2.567981in}{1.648264in}}%
\pgfpathclose%
\pgfusepath{stroke,fill}%
\end{pgfscope}%
\begin{pgfscope}%
\pgfpathrectangle{\pgfqpoint{0.100000in}{0.212622in}}{\pgfqpoint{3.696000in}{3.696000in}}%
\pgfusepath{clip}%
\pgfsetbuttcap%
\pgfsetroundjoin%
\definecolor{currentfill}{rgb}{0.121569,0.466667,0.705882}%
\pgfsetfillcolor{currentfill}%
\pgfsetfillopacity{0.933890}%
\pgfsetlinewidth{1.003750pt}%
\definecolor{currentstroke}{rgb}{0.121569,0.466667,0.705882}%
\pgfsetstrokecolor{currentstroke}%
\pgfsetstrokeopacity{0.933890}%
\pgfsetdash{}{0pt}%
\pgfpathmoveto{\pgfqpoint{1.715427in}{1.909727in}}%
\pgfpathcurveto{\pgfqpoint{1.723663in}{1.909727in}}{\pgfqpoint{1.731563in}{1.912999in}}{\pgfqpoint{1.737387in}{1.918823in}}%
\pgfpathcurveto{\pgfqpoint{1.743211in}{1.924647in}}{\pgfqpoint{1.746484in}{1.932547in}}{\pgfqpoint{1.746484in}{1.940783in}}%
\pgfpathcurveto{\pgfqpoint{1.746484in}{1.949019in}}{\pgfqpoint{1.743211in}{1.956919in}}{\pgfqpoint{1.737387in}{1.962743in}}%
\pgfpathcurveto{\pgfqpoint{1.731563in}{1.968567in}}{\pgfqpoint{1.723663in}{1.971840in}}{\pgfqpoint{1.715427in}{1.971840in}}%
\pgfpathcurveto{\pgfqpoint{1.707191in}{1.971840in}}{\pgfqpoint{1.699291in}{1.968567in}}{\pgfqpoint{1.693467in}{1.962743in}}%
\pgfpathcurveto{\pgfqpoint{1.687643in}{1.956919in}}{\pgfqpoint{1.684371in}{1.949019in}}{\pgfqpoint{1.684371in}{1.940783in}}%
\pgfpathcurveto{\pgfqpoint{1.684371in}{1.932547in}}{\pgfqpoint{1.687643in}{1.924647in}}{\pgfqpoint{1.693467in}{1.918823in}}%
\pgfpathcurveto{\pgfqpoint{1.699291in}{1.912999in}}{\pgfqpoint{1.707191in}{1.909727in}}{\pgfqpoint{1.715427in}{1.909727in}}%
\pgfpathclose%
\pgfusepath{stroke,fill}%
\end{pgfscope}%
\begin{pgfscope}%
\pgfpathrectangle{\pgfqpoint{0.100000in}{0.212622in}}{\pgfqpoint{3.696000in}{3.696000in}}%
\pgfusepath{clip}%
\pgfsetbuttcap%
\pgfsetroundjoin%
\definecolor{currentfill}{rgb}{0.121569,0.466667,0.705882}%
\pgfsetfillcolor{currentfill}%
\pgfsetfillopacity{0.934417}%
\pgfsetlinewidth{1.003750pt}%
\definecolor{currentstroke}{rgb}{0.121569,0.466667,0.705882}%
\pgfsetstrokecolor{currentstroke}%
\pgfsetstrokeopacity{0.934417}%
\pgfsetdash{}{0pt}%
\pgfpathmoveto{\pgfqpoint{1.723367in}{1.901138in}}%
\pgfpathcurveto{\pgfqpoint{1.731603in}{1.901138in}}{\pgfqpoint{1.739503in}{1.904410in}}{\pgfqpoint{1.745327in}{1.910234in}}%
\pgfpathcurveto{\pgfqpoint{1.751151in}{1.916058in}}{\pgfqpoint{1.754423in}{1.923958in}}{\pgfqpoint{1.754423in}{1.932194in}}%
\pgfpathcurveto{\pgfqpoint{1.754423in}{1.940430in}}{\pgfqpoint{1.751151in}{1.948330in}}{\pgfqpoint{1.745327in}{1.954154in}}%
\pgfpathcurveto{\pgfqpoint{1.739503in}{1.959978in}}{\pgfqpoint{1.731603in}{1.963251in}}{\pgfqpoint{1.723367in}{1.963251in}}%
\pgfpathcurveto{\pgfqpoint{1.715130in}{1.963251in}}{\pgfqpoint{1.707230in}{1.959978in}}{\pgfqpoint{1.701406in}{1.954154in}}%
\pgfpathcurveto{\pgfqpoint{1.695582in}{1.948330in}}{\pgfqpoint{1.692310in}{1.940430in}}{\pgfqpoint{1.692310in}{1.932194in}}%
\pgfpathcurveto{\pgfqpoint{1.692310in}{1.923958in}}{\pgfqpoint{1.695582in}{1.916058in}}{\pgfqpoint{1.701406in}{1.910234in}}%
\pgfpathcurveto{\pgfqpoint{1.707230in}{1.904410in}}{\pgfqpoint{1.715130in}{1.901138in}}{\pgfqpoint{1.723367in}{1.901138in}}%
\pgfpathclose%
\pgfusepath{stroke,fill}%
\end{pgfscope}%
\begin{pgfscope}%
\pgfpathrectangle{\pgfqpoint{0.100000in}{0.212622in}}{\pgfqpoint{3.696000in}{3.696000in}}%
\pgfusepath{clip}%
\pgfsetbuttcap%
\pgfsetroundjoin%
\definecolor{currentfill}{rgb}{0.121569,0.466667,0.705882}%
\pgfsetfillcolor{currentfill}%
\pgfsetfillopacity{0.935175}%
\pgfsetlinewidth{1.003750pt}%
\definecolor{currentstroke}{rgb}{0.121569,0.466667,0.705882}%
\pgfsetstrokecolor{currentstroke}%
\pgfsetstrokeopacity{0.935175}%
\pgfsetdash{}{0pt}%
\pgfpathmoveto{\pgfqpoint{1.729230in}{1.898054in}}%
\pgfpathcurveto{\pgfqpoint{1.737466in}{1.898054in}}{\pgfqpoint{1.745366in}{1.901326in}}{\pgfqpoint{1.751190in}{1.907150in}}%
\pgfpathcurveto{\pgfqpoint{1.757014in}{1.912974in}}{\pgfqpoint{1.760286in}{1.920874in}}{\pgfqpoint{1.760286in}{1.929111in}}%
\pgfpathcurveto{\pgfqpoint{1.760286in}{1.937347in}}{\pgfqpoint{1.757014in}{1.945247in}}{\pgfqpoint{1.751190in}{1.951071in}}%
\pgfpathcurveto{\pgfqpoint{1.745366in}{1.956895in}}{\pgfqpoint{1.737466in}{1.960167in}}{\pgfqpoint{1.729230in}{1.960167in}}%
\pgfpathcurveto{\pgfqpoint{1.720994in}{1.960167in}}{\pgfqpoint{1.713094in}{1.956895in}}{\pgfqpoint{1.707270in}{1.951071in}}%
\pgfpathcurveto{\pgfqpoint{1.701446in}{1.945247in}}{\pgfqpoint{1.698173in}{1.937347in}}{\pgfqpoint{1.698173in}{1.929111in}}%
\pgfpathcurveto{\pgfqpoint{1.698173in}{1.920874in}}{\pgfqpoint{1.701446in}{1.912974in}}{\pgfqpoint{1.707270in}{1.907150in}}%
\pgfpathcurveto{\pgfqpoint{1.713094in}{1.901326in}}{\pgfqpoint{1.720994in}{1.898054in}}{\pgfqpoint{1.729230in}{1.898054in}}%
\pgfpathclose%
\pgfusepath{stroke,fill}%
\end{pgfscope}%
\begin{pgfscope}%
\pgfpathrectangle{\pgfqpoint{0.100000in}{0.212622in}}{\pgfqpoint{3.696000in}{3.696000in}}%
\pgfusepath{clip}%
\pgfsetbuttcap%
\pgfsetroundjoin%
\definecolor{currentfill}{rgb}{0.121569,0.466667,0.705882}%
\pgfsetfillcolor{currentfill}%
\pgfsetfillopacity{0.935852}%
\pgfsetlinewidth{1.003750pt}%
\definecolor{currentstroke}{rgb}{0.121569,0.466667,0.705882}%
\pgfsetstrokecolor{currentstroke}%
\pgfsetstrokeopacity{0.935852}%
\pgfsetdash{}{0pt}%
\pgfpathmoveto{\pgfqpoint{2.565421in}{1.642459in}}%
\pgfpathcurveto{\pgfqpoint{2.573657in}{1.642459in}}{\pgfqpoint{2.581557in}{1.645731in}}{\pgfqpoint{2.587381in}{1.651555in}}%
\pgfpathcurveto{\pgfqpoint{2.593205in}{1.657379in}}{\pgfqpoint{2.596477in}{1.665279in}}{\pgfqpoint{2.596477in}{1.673515in}}%
\pgfpathcurveto{\pgfqpoint{2.596477in}{1.681751in}}{\pgfqpoint{2.593205in}{1.689651in}}{\pgfqpoint{2.587381in}{1.695475in}}%
\pgfpathcurveto{\pgfqpoint{2.581557in}{1.701299in}}{\pgfqpoint{2.573657in}{1.704572in}}{\pgfqpoint{2.565421in}{1.704572in}}%
\pgfpathcurveto{\pgfqpoint{2.557184in}{1.704572in}}{\pgfqpoint{2.549284in}{1.701299in}}{\pgfqpoint{2.543460in}{1.695475in}}%
\pgfpathcurveto{\pgfqpoint{2.537637in}{1.689651in}}{\pgfqpoint{2.534364in}{1.681751in}}{\pgfqpoint{2.534364in}{1.673515in}}%
\pgfpathcurveto{\pgfqpoint{2.534364in}{1.665279in}}{\pgfqpoint{2.537637in}{1.657379in}}{\pgfqpoint{2.543460in}{1.651555in}}%
\pgfpathcurveto{\pgfqpoint{2.549284in}{1.645731in}}{\pgfqpoint{2.557184in}{1.642459in}}{\pgfqpoint{2.565421in}{1.642459in}}%
\pgfpathclose%
\pgfusepath{stroke,fill}%
\end{pgfscope}%
\begin{pgfscope}%
\pgfpathrectangle{\pgfqpoint{0.100000in}{0.212622in}}{\pgfqpoint{3.696000in}{3.696000in}}%
\pgfusepath{clip}%
\pgfsetbuttcap%
\pgfsetroundjoin%
\definecolor{currentfill}{rgb}{0.121569,0.466667,0.705882}%
\pgfsetfillcolor{currentfill}%
\pgfsetfillopacity{0.936971}%
\pgfsetlinewidth{1.003750pt}%
\definecolor{currentstroke}{rgb}{0.121569,0.466667,0.705882}%
\pgfsetstrokecolor{currentstroke}%
\pgfsetstrokeopacity{0.936971}%
\pgfsetdash{}{0pt}%
\pgfpathmoveto{\pgfqpoint{2.563495in}{1.638686in}}%
\pgfpathcurveto{\pgfqpoint{2.571732in}{1.638686in}}{\pgfqpoint{2.579632in}{1.641958in}}{\pgfqpoint{2.585456in}{1.647782in}}%
\pgfpathcurveto{\pgfqpoint{2.591279in}{1.653606in}}{\pgfqpoint{2.594552in}{1.661506in}}{\pgfqpoint{2.594552in}{1.669742in}}%
\pgfpathcurveto{\pgfqpoint{2.594552in}{1.677978in}}{\pgfqpoint{2.591279in}{1.685879in}}{\pgfqpoint{2.585456in}{1.691702in}}%
\pgfpathcurveto{\pgfqpoint{2.579632in}{1.697526in}}{\pgfqpoint{2.571732in}{1.700799in}}{\pgfqpoint{2.563495in}{1.700799in}}%
\pgfpathcurveto{\pgfqpoint{2.555259in}{1.700799in}}{\pgfqpoint{2.547359in}{1.697526in}}{\pgfqpoint{2.541535in}{1.691702in}}%
\pgfpathcurveto{\pgfqpoint{2.535711in}{1.685879in}}{\pgfqpoint{2.532439in}{1.677978in}}{\pgfqpoint{2.532439in}{1.669742in}}%
\pgfpathcurveto{\pgfqpoint{2.532439in}{1.661506in}}{\pgfqpoint{2.535711in}{1.653606in}}{\pgfqpoint{2.541535in}{1.647782in}}%
\pgfpathcurveto{\pgfqpoint{2.547359in}{1.641958in}}{\pgfqpoint{2.555259in}{1.638686in}}{\pgfqpoint{2.563495in}{1.638686in}}%
\pgfpathclose%
\pgfusepath{stroke,fill}%
\end{pgfscope}%
\begin{pgfscope}%
\pgfpathrectangle{\pgfqpoint{0.100000in}{0.212622in}}{\pgfqpoint{3.696000in}{3.696000in}}%
\pgfusepath{clip}%
\pgfsetbuttcap%
\pgfsetroundjoin%
\definecolor{currentfill}{rgb}{0.121569,0.466667,0.705882}%
\pgfsetfillcolor{currentfill}%
\pgfsetfillopacity{0.936984}%
\pgfsetlinewidth{1.003750pt}%
\definecolor{currentstroke}{rgb}{0.121569,0.466667,0.705882}%
\pgfsetstrokecolor{currentstroke}%
\pgfsetstrokeopacity{0.936984}%
\pgfsetdash{}{0pt}%
\pgfpathmoveto{\pgfqpoint{1.740525in}{1.897404in}}%
\pgfpathcurveto{\pgfqpoint{1.748762in}{1.897404in}}{\pgfqpoint{1.756662in}{1.900676in}}{\pgfqpoint{1.762486in}{1.906500in}}%
\pgfpathcurveto{\pgfqpoint{1.768310in}{1.912324in}}{\pgfqpoint{1.771582in}{1.920224in}}{\pgfqpoint{1.771582in}{1.928460in}}%
\pgfpathcurveto{\pgfqpoint{1.771582in}{1.936697in}}{\pgfqpoint{1.768310in}{1.944597in}}{\pgfqpoint{1.762486in}{1.950421in}}%
\pgfpathcurveto{\pgfqpoint{1.756662in}{1.956245in}}{\pgfqpoint{1.748762in}{1.959517in}}{\pgfqpoint{1.740525in}{1.959517in}}%
\pgfpathcurveto{\pgfqpoint{1.732289in}{1.959517in}}{\pgfqpoint{1.724389in}{1.956245in}}{\pgfqpoint{1.718565in}{1.950421in}}%
\pgfpathcurveto{\pgfqpoint{1.712741in}{1.944597in}}{\pgfqpoint{1.709469in}{1.936697in}}{\pgfqpoint{1.709469in}{1.928460in}}%
\pgfpathcurveto{\pgfqpoint{1.709469in}{1.920224in}}{\pgfqpoint{1.712741in}{1.912324in}}{\pgfqpoint{1.718565in}{1.906500in}}%
\pgfpathcurveto{\pgfqpoint{1.724389in}{1.900676in}}{\pgfqpoint{1.732289in}{1.897404in}}{\pgfqpoint{1.740525in}{1.897404in}}%
\pgfpathclose%
\pgfusepath{stroke,fill}%
\end{pgfscope}%
\begin{pgfscope}%
\pgfpathrectangle{\pgfqpoint{0.100000in}{0.212622in}}{\pgfqpoint{3.696000in}{3.696000in}}%
\pgfusepath{clip}%
\pgfsetbuttcap%
\pgfsetroundjoin%
\definecolor{currentfill}{rgb}{0.121569,0.466667,0.705882}%
\pgfsetfillcolor{currentfill}%
\pgfsetfillopacity{0.937407}%
\pgfsetlinewidth{1.003750pt}%
\definecolor{currentstroke}{rgb}{0.121569,0.466667,0.705882}%
\pgfsetstrokecolor{currentstroke}%
\pgfsetstrokeopacity{0.937407}%
\pgfsetdash{}{0pt}%
\pgfpathmoveto{\pgfqpoint{1.750448in}{1.886921in}}%
\pgfpathcurveto{\pgfqpoint{1.758684in}{1.886921in}}{\pgfqpoint{1.766585in}{1.890193in}}{\pgfqpoint{1.772408in}{1.896017in}}%
\pgfpathcurveto{\pgfqpoint{1.778232in}{1.901841in}}{\pgfqpoint{1.781505in}{1.909741in}}{\pgfqpoint{1.781505in}{1.917977in}}%
\pgfpathcurveto{\pgfqpoint{1.781505in}{1.926213in}}{\pgfqpoint{1.778232in}{1.934113in}}{\pgfqpoint{1.772408in}{1.939937in}}%
\pgfpathcurveto{\pgfqpoint{1.766585in}{1.945761in}}{\pgfqpoint{1.758684in}{1.949034in}}{\pgfqpoint{1.750448in}{1.949034in}}%
\pgfpathcurveto{\pgfqpoint{1.742212in}{1.949034in}}{\pgfqpoint{1.734312in}{1.945761in}}{\pgfqpoint{1.728488in}{1.939937in}}%
\pgfpathcurveto{\pgfqpoint{1.722664in}{1.934113in}}{\pgfqpoint{1.719392in}{1.926213in}}{\pgfqpoint{1.719392in}{1.917977in}}%
\pgfpathcurveto{\pgfqpoint{1.719392in}{1.909741in}}{\pgfqpoint{1.722664in}{1.901841in}}{\pgfqpoint{1.728488in}{1.896017in}}%
\pgfpathcurveto{\pgfqpoint{1.734312in}{1.890193in}}{\pgfqpoint{1.742212in}{1.886921in}}{\pgfqpoint{1.750448in}{1.886921in}}%
\pgfpathclose%
\pgfusepath{stroke,fill}%
\end{pgfscope}%
\begin{pgfscope}%
\pgfpathrectangle{\pgfqpoint{0.100000in}{0.212622in}}{\pgfqpoint{3.696000in}{3.696000in}}%
\pgfusepath{clip}%
\pgfsetbuttcap%
\pgfsetroundjoin%
\definecolor{currentfill}{rgb}{0.121569,0.466667,0.705882}%
\pgfsetfillcolor{currentfill}%
\pgfsetfillopacity{0.938005}%
\pgfsetlinewidth{1.003750pt}%
\definecolor{currentstroke}{rgb}{0.121569,0.466667,0.705882}%
\pgfsetstrokecolor{currentstroke}%
\pgfsetstrokeopacity{0.938005}%
\pgfsetdash{}{0pt}%
\pgfpathmoveto{\pgfqpoint{1.757682in}{1.881885in}}%
\pgfpathcurveto{\pgfqpoint{1.765918in}{1.881885in}}{\pgfqpoint{1.773818in}{1.885157in}}{\pgfqpoint{1.779642in}{1.890981in}}%
\pgfpathcurveto{\pgfqpoint{1.785466in}{1.896805in}}{\pgfqpoint{1.788738in}{1.904705in}}{\pgfqpoint{1.788738in}{1.912941in}}%
\pgfpathcurveto{\pgfqpoint{1.788738in}{1.921178in}}{\pgfqpoint{1.785466in}{1.929078in}}{\pgfqpoint{1.779642in}{1.934901in}}%
\pgfpathcurveto{\pgfqpoint{1.773818in}{1.940725in}}{\pgfqpoint{1.765918in}{1.943998in}}{\pgfqpoint{1.757682in}{1.943998in}}%
\pgfpathcurveto{\pgfqpoint{1.749445in}{1.943998in}}{\pgfqpoint{1.741545in}{1.940725in}}{\pgfqpoint{1.735721in}{1.934901in}}%
\pgfpathcurveto{\pgfqpoint{1.729897in}{1.929078in}}{\pgfqpoint{1.726625in}{1.921178in}}{\pgfqpoint{1.726625in}{1.912941in}}%
\pgfpathcurveto{\pgfqpoint{1.726625in}{1.904705in}}{\pgfqpoint{1.729897in}{1.896805in}}{\pgfqpoint{1.735721in}{1.890981in}}%
\pgfpathcurveto{\pgfqpoint{1.741545in}{1.885157in}}{\pgfqpoint{1.749445in}{1.881885in}}{\pgfqpoint{1.757682in}{1.881885in}}%
\pgfpathclose%
\pgfusepath{stroke,fill}%
\end{pgfscope}%
\begin{pgfscope}%
\pgfpathrectangle{\pgfqpoint{0.100000in}{0.212622in}}{\pgfqpoint{3.696000in}{3.696000in}}%
\pgfusepath{clip}%
\pgfsetbuttcap%
\pgfsetroundjoin%
\definecolor{currentfill}{rgb}{0.121569,0.466667,0.705882}%
\pgfsetfillcolor{currentfill}%
\pgfsetfillopacity{0.938415}%
\pgfsetlinewidth{1.003750pt}%
\definecolor{currentstroke}{rgb}{0.121569,0.466667,0.705882}%
\pgfsetstrokecolor{currentstroke}%
\pgfsetstrokeopacity{0.938415}%
\pgfsetdash{}{0pt}%
\pgfpathmoveto{\pgfqpoint{1.763236in}{1.877863in}}%
\pgfpathcurveto{\pgfqpoint{1.771472in}{1.877863in}}{\pgfqpoint{1.779372in}{1.881135in}}{\pgfqpoint{1.785196in}{1.886959in}}%
\pgfpathcurveto{\pgfqpoint{1.791020in}{1.892783in}}{\pgfqpoint{1.794292in}{1.900683in}}{\pgfqpoint{1.794292in}{1.908919in}}%
\pgfpathcurveto{\pgfqpoint{1.794292in}{1.917155in}}{\pgfqpoint{1.791020in}{1.925055in}}{\pgfqpoint{1.785196in}{1.930879in}}%
\pgfpathcurveto{\pgfqpoint{1.779372in}{1.936703in}}{\pgfqpoint{1.771472in}{1.939976in}}{\pgfqpoint{1.763236in}{1.939976in}}%
\pgfpathcurveto{\pgfqpoint{1.754999in}{1.939976in}}{\pgfqpoint{1.747099in}{1.936703in}}{\pgfqpoint{1.741275in}{1.930879in}}%
\pgfpathcurveto{\pgfqpoint{1.735451in}{1.925055in}}{\pgfqpoint{1.732179in}{1.917155in}}{\pgfqpoint{1.732179in}{1.908919in}}%
\pgfpathcurveto{\pgfqpoint{1.732179in}{1.900683in}}{\pgfqpoint{1.735451in}{1.892783in}}{\pgfqpoint{1.741275in}{1.886959in}}%
\pgfpathcurveto{\pgfqpoint{1.747099in}{1.881135in}}{\pgfqpoint{1.754999in}{1.877863in}}{\pgfqpoint{1.763236in}{1.877863in}}%
\pgfpathclose%
\pgfusepath{stroke,fill}%
\end{pgfscope}%
\begin{pgfscope}%
\pgfpathrectangle{\pgfqpoint{0.100000in}{0.212622in}}{\pgfqpoint{3.696000in}{3.696000in}}%
\pgfusepath{clip}%
\pgfsetbuttcap%
\pgfsetroundjoin%
\definecolor{currentfill}{rgb}{0.121569,0.466667,0.705882}%
\pgfsetfillcolor{currentfill}%
\pgfsetfillopacity{0.939024}%
\pgfsetlinewidth{1.003750pt}%
\definecolor{currentstroke}{rgb}{0.121569,0.466667,0.705882}%
\pgfsetstrokecolor{currentstroke}%
\pgfsetstrokeopacity{0.939024}%
\pgfsetdash{}{0pt}%
\pgfpathmoveto{\pgfqpoint{1.772986in}{1.868529in}}%
\pgfpathcurveto{\pgfqpoint{1.781223in}{1.868529in}}{\pgfqpoint{1.789123in}{1.871801in}}{\pgfqpoint{1.794947in}{1.877625in}}%
\pgfpathcurveto{\pgfqpoint{1.800771in}{1.883449in}}{\pgfqpoint{1.804043in}{1.891349in}}{\pgfqpoint{1.804043in}{1.899585in}}%
\pgfpathcurveto{\pgfqpoint{1.804043in}{1.907821in}}{\pgfqpoint{1.800771in}{1.915721in}}{\pgfqpoint{1.794947in}{1.921545in}}%
\pgfpathcurveto{\pgfqpoint{1.789123in}{1.927369in}}{\pgfqpoint{1.781223in}{1.930642in}}{\pgfqpoint{1.772986in}{1.930642in}}%
\pgfpathcurveto{\pgfqpoint{1.764750in}{1.930642in}}{\pgfqpoint{1.756850in}{1.927369in}}{\pgfqpoint{1.751026in}{1.921545in}}%
\pgfpathcurveto{\pgfqpoint{1.745202in}{1.915721in}}{\pgfqpoint{1.741930in}{1.907821in}}{\pgfqpoint{1.741930in}{1.899585in}}%
\pgfpathcurveto{\pgfqpoint{1.741930in}{1.891349in}}{\pgfqpoint{1.745202in}{1.883449in}}{\pgfqpoint{1.751026in}{1.877625in}}%
\pgfpathcurveto{\pgfqpoint{1.756850in}{1.871801in}}{\pgfqpoint{1.764750in}{1.868529in}}{\pgfqpoint{1.772986in}{1.868529in}}%
\pgfpathclose%
\pgfusepath{stroke,fill}%
\end{pgfscope}%
\begin{pgfscope}%
\pgfpathrectangle{\pgfqpoint{0.100000in}{0.212622in}}{\pgfqpoint{3.696000in}{3.696000in}}%
\pgfusepath{clip}%
\pgfsetbuttcap%
\pgfsetroundjoin%
\definecolor{currentfill}{rgb}{0.121569,0.466667,0.705882}%
\pgfsetfillcolor{currentfill}%
\pgfsetfillopacity{0.939153}%
\pgfsetlinewidth{1.003750pt}%
\definecolor{currentstroke}{rgb}{0.121569,0.466667,0.705882}%
\pgfsetstrokecolor{currentstroke}%
\pgfsetstrokeopacity{0.939153}%
\pgfsetdash{}{0pt}%
\pgfpathmoveto{\pgfqpoint{2.557803in}{1.639669in}}%
\pgfpathcurveto{\pgfqpoint{2.566040in}{1.639669in}}{\pgfqpoint{2.573940in}{1.642941in}}{\pgfqpoint{2.579764in}{1.648765in}}%
\pgfpathcurveto{\pgfqpoint{2.585588in}{1.654589in}}{\pgfqpoint{2.588860in}{1.662489in}}{\pgfqpoint{2.588860in}{1.670726in}}%
\pgfpathcurveto{\pgfqpoint{2.588860in}{1.678962in}}{\pgfqpoint{2.585588in}{1.686862in}}{\pgfqpoint{2.579764in}{1.692686in}}%
\pgfpathcurveto{\pgfqpoint{2.573940in}{1.698510in}}{\pgfqpoint{2.566040in}{1.701782in}}{\pgfqpoint{2.557803in}{1.701782in}}%
\pgfpathcurveto{\pgfqpoint{2.549567in}{1.701782in}}{\pgfqpoint{2.541667in}{1.698510in}}{\pgfqpoint{2.535843in}{1.692686in}}%
\pgfpathcurveto{\pgfqpoint{2.530019in}{1.686862in}}{\pgfqpoint{2.526747in}{1.678962in}}{\pgfqpoint{2.526747in}{1.670726in}}%
\pgfpathcurveto{\pgfqpoint{2.526747in}{1.662489in}}{\pgfqpoint{2.530019in}{1.654589in}}{\pgfqpoint{2.535843in}{1.648765in}}%
\pgfpathcurveto{\pgfqpoint{2.541667in}{1.642941in}}{\pgfqpoint{2.549567in}{1.639669in}}{\pgfqpoint{2.557803in}{1.639669in}}%
\pgfpathclose%
\pgfusepath{stroke,fill}%
\end{pgfscope}%
\begin{pgfscope}%
\pgfpathrectangle{\pgfqpoint{0.100000in}{0.212622in}}{\pgfqpoint{3.696000in}{3.696000in}}%
\pgfusepath{clip}%
\pgfsetbuttcap%
\pgfsetroundjoin%
\definecolor{currentfill}{rgb}{0.121569,0.466667,0.705882}%
\pgfsetfillcolor{currentfill}%
\pgfsetfillopacity{0.939722}%
\pgfsetlinewidth{1.003750pt}%
\definecolor{currentstroke}{rgb}{0.121569,0.466667,0.705882}%
\pgfsetstrokecolor{currentstroke}%
\pgfsetstrokeopacity{0.939722}%
\pgfsetdash{}{0pt}%
\pgfpathmoveto{\pgfqpoint{1.781833in}{1.864148in}}%
\pgfpathcurveto{\pgfqpoint{1.790069in}{1.864148in}}{\pgfqpoint{1.797969in}{1.867420in}}{\pgfqpoint{1.803793in}{1.873244in}}%
\pgfpathcurveto{\pgfqpoint{1.809617in}{1.879068in}}{\pgfqpoint{1.812889in}{1.886968in}}{\pgfqpoint{1.812889in}{1.895204in}}%
\pgfpathcurveto{\pgfqpoint{1.812889in}{1.903441in}}{\pgfqpoint{1.809617in}{1.911341in}}{\pgfqpoint{1.803793in}{1.917165in}}%
\pgfpathcurveto{\pgfqpoint{1.797969in}{1.922989in}}{\pgfqpoint{1.790069in}{1.926261in}}{\pgfqpoint{1.781833in}{1.926261in}}%
\pgfpathcurveto{\pgfqpoint{1.773596in}{1.926261in}}{\pgfqpoint{1.765696in}{1.922989in}}{\pgfqpoint{1.759872in}{1.917165in}}%
\pgfpathcurveto{\pgfqpoint{1.754048in}{1.911341in}}{\pgfqpoint{1.750776in}{1.903441in}}{\pgfqpoint{1.750776in}{1.895204in}}%
\pgfpathcurveto{\pgfqpoint{1.750776in}{1.886968in}}{\pgfqpoint{1.754048in}{1.879068in}}{\pgfqpoint{1.759872in}{1.873244in}}%
\pgfpathcurveto{\pgfqpoint{1.765696in}{1.867420in}}{\pgfqpoint{1.773596in}{1.864148in}}{\pgfqpoint{1.781833in}{1.864148in}}%
\pgfpathclose%
\pgfusepath{stroke,fill}%
\end{pgfscope}%
\begin{pgfscope}%
\pgfpathrectangle{\pgfqpoint{0.100000in}{0.212622in}}{\pgfqpoint{3.696000in}{3.696000in}}%
\pgfusepath{clip}%
\pgfsetbuttcap%
\pgfsetroundjoin%
\definecolor{currentfill}{rgb}{0.121569,0.466667,0.705882}%
\pgfsetfillcolor{currentfill}%
\pgfsetfillopacity{0.939954}%
\pgfsetlinewidth{1.003750pt}%
\definecolor{currentstroke}{rgb}{0.121569,0.466667,0.705882}%
\pgfsetstrokecolor{currentstroke}%
\pgfsetstrokeopacity{0.939954}%
\pgfsetdash{}{0pt}%
\pgfpathmoveto{\pgfqpoint{1.788228in}{1.858328in}}%
\pgfpathcurveto{\pgfqpoint{1.796465in}{1.858328in}}{\pgfqpoint{1.804365in}{1.861601in}}{\pgfqpoint{1.810189in}{1.867424in}}%
\pgfpathcurveto{\pgfqpoint{1.816013in}{1.873248in}}{\pgfqpoint{1.819285in}{1.881148in}}{\pgfqpoint{1.819285in}{1.889385in}}%
\pgfpathcurveto{\pgfqpoint{1.819285in}{1.897621in}}{\pgfqpoint{1.816013in}{1.905521in}}{\pgfqpoint{1.810189in}{1.911345in}}%
\pgfpathcurveto{\pgfqpoint{1.804365in}{1.917169in}}{\pgfqpoint{1.796465in}{1.920441in}}{\pgfqpoint{1.788228in}{1.920441in}}%
\pgfpathcurveto{\pgfqpoint{1.779992in}{1.920441in}}{\pgfqpoint{1.772092in}{1.917169in}}{\pgfqpoint{1.766268in}{1.911345in}}%
\pgfpathcurveto{\pgfqpoint{1.760444in}{1.905521in}}{\pgfqpoint{1.757172in}{1.897621in}}{\pgfqpoint{1.757172in}{1.889385in}}%
\pgfpathcurveto{\pgfqpoint{1.757172in}{1.881148in}}{\pgfqpoint{1.760444in}{1.873248in}}{\pgfqpoint{1.766268in}{1.867424in}}%
\pgfpathcurveto{\pgfqpoint{1.772092in}{1.861601in}}{\pgfqpoint{1.779992in}{1.858328in}}{\pgfqpoint{1.788228in}{1.858328in}}%
\pgfpathclose%
\pgfusepath{stroke,fill}%
\end{pgfscope}%
\begin{pgfscope}%
\pgfpathrectangle{\pgfqpoint{0.100000in}{0.212622in}}{\pgfqpoint{3.696000in}{3.696000in}}%
\pgfusepath{clip}%
\pgfsetbuttcap%
\pgfsetroundjoin%
\definecolor{currentfill}{rgb}{0.121569,0.466667,0.705882}%
\pgfsetfillcolor{currentfill}%
\pgfsetfillopacity{0.940833}%
\pgfsetlinewidth{1.003750pt}%
\definecolor{currentstroke}{rgb}{0.121569,0.466667,0.705882}%
\pgfsetstrokecolor{currentstroke}%
\pgfsetstrokeopacity{0.940833}%
\pgfsetdash{}{0pt}%
\pgfpathmoveto{\pgfqpoint{2.553680in}{1.633625in}}%
\pgfpathcurveto{\pgfqpoint{2.561916in}{1.633625in}}{\pgfqpoint{2.569816in}{1.636897in}}{\pgfqpoint{2.575640in}{1.642721in}}%
\pgfpathcurveto{\pgfqpoint{2.581464in}{1.648545in}}{\pgfqpoint{2.584737in}{1.656445in}}{\pgfqpoint{2.584737in}{1.664681in}}%
\pgfpathcurveto{\pgfqpoint{2.584737in}{1.672917in}}{\pgfqpoint{2.581464in}{1.680817in}}{\pgfqpoint{2.575640in}{1.686641in}}%
\pgfpathcurveto{\pgfqpoint{2.569816in}{1.692465in}}{\pgfqpoint{2.561916in}{1.695738in}}{\pgfqpoint{2.553680in}{1.695738in}}%
\pgfpathcurveto{\pgfqpoint{2.545444in}{1.695738in}}{\pgfqpoint{2.537544in}{1.692465in}}{\pgfqpoint{2.531720in}{1.686641in}}%
\pgfpathcurveto{\pgfqpoint{2.525896in}{1.680817in}}{\pgfqpoint{2.522624in}{1.672917in}}{\pgfqpoint{2.522624in}{1.664681in}}%
\pgfpathcurveto{\pgfqpoint{2.522624in}{1.656445in}}{\pgfqpoint{2.525896in}{1.648545in}}{\pgfqpoint{2.531720in}{1.642721in}}%
\pgfpathcurveto{\pgfqpoint{2.537544in}{1.636897in}}{\pgfqpoint{2.545444in}{1.633625in}}{\pgfqpoint{2.553680in}{1.633625in}}%
\pgfpathclose%
\pgfusepath{stroke,fill}%
\end{pgfscope}%
\begin{pgfscope}%
\pgfpathrectangle{\pgfqpoint{0.100000in}{0.212622in}}{\pgfqpoint{3.696000in}{3.696000in}}%
\pgfusepath{clip}%
\pgfsetbuttcap%
\pgfsetroundjoin%
\definecolor{currentfill}{rgb}{0.121569,0.466667,0.705882}%
\pgfsetfillcolor{currentfill}%
\pgfsetfillopacity{0.941062}%
\pgfsetlinewidth{1.003750pt}%
\definecolor{currentstroke}{rgb}{0.121569,0.466667,0.705882}%
\pgfsetstrokecolor{currentstroke}%
\pgfsetstrokeopacity{0.941062}%
\pgfsetdash{}{0pt}%
\pgfpathmoveto{\pgfqpoint{1.799821in}{1.852233in}}%
\pgfpathcurveto{\pgfqpoint{1.808057in}{1.852233in}}{\pgfqpoint{1.815957in}{1.855506in}}{\pgfqpoint{1.821781in}{1.861330in}}%
\pgfpathcurveto{\pgfqpoint{1.827605in}{1.867154in}}{\pgfqpoint{1.830877in}{1.875054in}}{\pgfqpoint{1.830877in}{1.883290in}}%
\pgfpathcurveto{\pgfqpoint{1.830877in}{1.891526in}}{\pgfqpoint{1.827605in}{1.899426in}}{\pgfqpoint{1.821781in}{1.905250in}}%
\pgfpathcurveto{\pgfqpoint{1.815957in}{1.911074in}}{\pgfqpoint{1.808057in}{1.914346in}}{\pgfqpoint{1.799821in}{1.914346in}}%
\pgfpathcurveto{\pgfqpoint{1.791584in}{1.914346in}}{\pgfqpoint{1.783684in}{1.911074in}}{\pgfqpoint{1.777860in}{1.905250in}}%
\pgfpathcurveto{\pgfqpoint{1.772037in}{1.899426in}}{\pgfqpoint{1.768764in}{1.891526in}}{\pgfqpoint{1.768764in}{1.883290in}}%
\pgfpathcurveto{\pgfqpoint{1.768764in}{1.875054in}}{\pgfqpoint{1.772037in}{1.867154in}}{\pgfqpoint{1.777860in}{1.861330in}}%
\pgfpathcurveto{\pgfqpoint{1.783684in}{1.855506in}}{\pgfqpoint{1.791584in}{1.852233in}}{\pgfqpoint{1.799821in}{1.852233in}}%
\pgfpathclose%
\pgfusepath{stroke,fill}%
\end{pgfscope}%
\begin{pgfscope}%
\pgfpathrectangle{\pgfqpoint{0.100000in}{0.212622in}}{\pgfqpoint{3.696000in}{3.696000in}}%
\pgfusepath{clip}%
\pgfsetbuttcap%
\pgfsetroundjoin%
\definecolor{currentfill}{rgb}{0.121569,0.466667,0.705882}%
\pgfsetfillcolor{currentfill}%
\pgfsetfillopacity{0.942162}%
\pgfsetlinewidth{1.003750pt}%
\definecolor{currentstroke}{rgb}{0.121569,0.466667,0.705882}%
\pgfsetstrokecolor{currentstroke}%
\pgfsetstrokeopacity{0.942162}%
\pgfsetdash{}{0pt}%
\pgfpathmoveto{\pgfqpoint{1.810982in}{1.848757in}}%
\pgfpathcurveto{\pgfqpoint{1.819218in}{1.848757in}}{\pgfqpoint{1.827119in}{1.852030in}}{\pgfqpoint{1.832942in}{1.857854in}}%
\pgfpathcurveto{\pgfqpoint{1.838766in}{1.863677in}}{\pgfqpoint{1.842039in}{1.871577in}}{\pgfqpoint{1.842039in}{1.879814in}}%
\pgfpathcurveto{\pgfqpoint{1.842039in}{1.888050in}}{\pgfqpoint{1.838766in}{1.895950in}}{\pgfqpoint{1.832942in}{1.901774in}}%
\pgfpathcurveto{\pgfqpoint{1.827119in}{1.907598in}}{\pgfqpoint{1.819218in}{1.910870in}}{\pgfqpoint{1.810982in}{1.910870in}}%
\pgfpathcurveto{\pgfqpoint{1.802746in}{1.910870in}}{\pgfqpoint{1.794846in}{1.907598in}}{\pgfqpoint{1.789022in}{1.901774in}}%
\pgfpathcurveto{\pgfqpoint{1.783198in}{1.895950in}}{\pgfqpoint{1.779926in}{1.888050in}}{\pgfqpoint{1.779926in}{1.879814in}}%
\pgfpathcurveto{\pgfqpoint{1.779926in}{1.871577in}}{\pgfqpoint{1.783198in}{1.863677in}}{\pgfqpoint{1.789022in}{1.857854in}}%
\pgfpathcurveto{\pgfqpoint{1.794846in}{1.852030in}}{\pgfqpoint{1.802746in}{1.848757in}}{\pgfqpoint{1.810982in}{1.848757in}}%
\pgfpathclose%
\pgfusepath{stroke,fill}%
\end{pgfscope}%
\begin{pgfscope}%
\pgfpathrectangle{\pgfqpoint{0.100000in}{0.212622in}}{\pgfqpoint{3.696000in}{3.696000in}}%
\pgfusepath{clip}%
\pgfsetbuttcap%
\pgfsetroundjoin%
\definecolor{currentfill}{rgb}{0.121569,0.466667,0.705882}%
\pgfsetfillcolor{currentfill}%
\pgfsetfillopacity{0.942758}%
\pgfsetlinewidth{1.003750pt}%
\definecolor{currentstroke}{rgb}{0.121569,0.466667,0.705882}%
\pgfsetstrokecolor{currentstroke}%
\pgfsetstrokeopacity{0.942758}%
\pgfsetdash{}{0pt}%
\pgfpathmoveto{\pgfqpoint{1.817860in}{1.841415in}}%
\pgfpathcurveto{\pgfqpoint{1.826096in}{1.841415in}}{\pgfqpoint{1.833996in}{1.844687in}}{\pgfqpoint{1.839820in}{1.850511in}}%
\pgfpathcurveto{\pgfqpoint{1.845644in}{1.856335in}}{\pgfqpoint{1.848916in}{1.864235in}}{\pgfqpoint{1.848916in}{1.872471in}}%
\pgfpathcurveto{\pgfqpoint{1.848916in}{1.880708in}}{\pgfqpoint{1.845644in}{1.888608in}}{\pgfqpoint{1.839820in}{1.894432in}}%
\pgfpathcurveto{\pgfqpoint{1.833996in}{1.900255in}}{\pgfqpoint{1.826096in}{1.903528in}}{\pgfqpoint{1.817860in}{1.903528in}}%
\pgfpathcurveto{\pgfqpoint{1.809624in}{1.903528in}}{\pgfqpoint{1.801724in}{1.900255in}}{\pgfqpoint{1.795900in}{1.894432in}}%
\pgfpathcurveto{\pgfqpoint{1.790076in}{1.888608in}}{\pgfqpoint{1.786803in}{1.880708in}}{\pgfqpoint{1.786803in}{1.872471in}}%
\pgfpathcurveto{\pgfqpoint{1.786803in}{1.864235in}}{\pgfqpoint{1.790076in}{1.856335in}}{\pgfqpoint{1.795900in}{1.850511in}}%
\pgfpathcurveto{\pgfqpoint{1.801724in}{1.844687in}}{\pgfqpoint{1.809624in}{1.841415in}}{\pgfqpoint{1.817860in}{1.841415in}}%
\pgfpathclose%
\pgfusepath{stroke,fill}%
\end{pgfscope}%
\begin{pgfscope}%
\pgfpathrectangle{\pgfqpoint{0.100000in}{0.212622in}}{\pgfqpoint{3.696000in}{3.696000in}}%
\pgfusepath{clip}%
\pgfsetbuttcap%
\pgfsetroundjoin%
\definecolor{currentfill}{rgb}{0.121569,0.466667,0.705882}%
\pgfsetfillcolor{currentfill}%
\pgfsetfillopacity{0.943149}%
\pgfsetlinewidth{1.003750pt}%
\definecolor{currentstroke}{rgb}{0.121569,0.466667,0.705882}%
\pgfsetstrokecolor{currentstroke}%
\pgfsetstrokeopacity{0.943149}%
\pgfsetdash{}{0pt}%
\pgfpathmoveto{\pgfqpoint{2.550282in}{1.630331in}}%
\pgfpathcurveto{\pgfqpoint{2.558518in}{1.630331in}}{\pgfqpoint{2.566418in}{1.633603in}}{\pgfqpoint{2.572242in}{1.639427in}}%
\pgfpathcurveto{\pgfqpoint{2.578066in}{1.645251in}}{\pgfqpoint{2.581338in}{1.653151in}}{\pgfqpoint{2.581338in}{1.661387in}}%
\pgfpathcurveto{\pgfqpoint{2.581338in}{1.669624in}}{\pgfqpoint{2.578066in}{1.677524in}}{\pgfqpoint{2.572242in}{1.683348in}}%
\pgfpathcurveto{\pgfqpoint{2.566418in}{1.689171in}}{\pgfqpoint{2.558518in}{1.692444in}}{\pgfqpoint{2.550282in}{1.692444in}}%
\pgfpathcurveto{\pgfqpoint{2.542045in}{1.692444in}}{\pgfqpoint{2.534145in}{1.689171in}}{\pgfqpoint{2.528322in}{1.683348in}}%
\pgfpathcurveto{\pgfqpoint{2.522498in}{1.677524in}}{\pgfqpoint{2.519225in}{1.669624in}}{\pgfqpoint{2.519225in}{1.661387in}}%
\pgfpathcurveto{\pgfqpoint{2.519225in}{1.653151in}}{\pgfqpoint{2.522498in}{1.645251in}}{\pgfqpoint{2.528322in}{1.639427in}}%
\pgfpathcurveto{\pgfqpoint{2.534145in}{1.633603in}}{\pgfqpoint{2.542045in}{1.630331in}}{\pgfqpoint{2.550282in}{1.630331in}}%
\pgfpathclose%
\pgfusepath{stroke,fill}%
\end{pgfscope}%
\begin{pgfscope}%
\pgfpathrectangle{\pgfqpoint{0.100000in}{0.212622in}}{\pgfqpoint{3.696000in}{3.696000in}}%
\pgfusepath{clip}%
\pgfsetbuttcap%
\pgfsetroundjoin%
\definecolor{currentfill}{rgb}{0.121569,0.466667,0.705882}%
\pgfsetfillcolor{currentfill}%
\pgfsetfillopacity{0.943868}%
\pgfsetlinewidth{1.003750pt}%
\definecolor{currentstroke}{rgb}{0.121569,0.466667,0.705882}%
\pgfsetstrokecolor{currentstroke}%
\pgfsetstrokeopacity{0.943868}%
\pgfsetdash{}{0pt}%
\pgfpathmoveto{\pgfqpoint{1.824432in}{1.838536in}}%
\pgfpathcurveto{\pgfqpoint{1.832668in}{1.838536in}}{\pgfqpoint{1.840568in}{1.841809in}}{\pgfqpoint{1.846392in}{1.847633in}}%
\pgfpathcurveto{\pgfqpoint{1.852216in}{1.853457in}}{\pgfqpoint{1.855488in}{1.861357in}}{\pgfqpoint{1.855488in}{1.869593in}}%
\pgfpathcurveto{\pgfqpoint{1.855488in}{1.877829in}}{\pgfqpoint{1.852216in}{1.885729in}}{\pgfqpoint{1.846392in}{1.891553in}}%
\pgfpathcurveto{\pgfqpoint{1.840568in}{1.897377in}}{\pgfqpoint{1.832668in}{1.900649in}}{\pgfqpoint{1.824432in}{1.900649in}}%
\pgfpathcurveto{\pgfqpoint{1.816196in}{1.900649in}}{\pgfqpoint{1.808296in}{1.897377in}}{\pgfqpoint{1.802472in}{1.891553in}}%
\pgfpathcurveto{\pgfqpoint{1.796648in}{1.885729in}}{\pgfqpoint{1.793375in}{1.877829in}}{\pgfqpoint{1.793375in}{1.869593in}}%
\pgfpathcurveto{\pgfqpoint{1.793375in}{1.861357in}}{\pgfqpoint{1.796648in}{1.853457in}}{\pgfqpoint{1.802472in}{1.847633in}}%
\pgfpathcurveto{\pgfqpoint{1.808296in}{1.841809in}}{\pgfqpoint{1.816196in}{1.838536in}}{\pgfqpoint{1.824432in}{1.838536in}}%
\pgfpathclose%
\pgfusepath{stroke,fill}%
\end{pgfscope}%
\begin{pgfscope}%
\pgfpathrectangle{\pgfqpoint{0.100000in}{0.212622in}}{\pgfqpoint{3.696000in}{3.696000in}}%
\pgfusepath{clip}%
\pgfsetbuttcap%
\pgfsetroundjoin%
\definecolor{currentfill}{rgb}{0.121569,0.466667,0.705882}%
\pgfsetfillcolor{currentfill}%
\pgfsetfillopacity{0.945408}%
\pgfsetlinewidth{1.003750pt}%
\definecolor{currentstroke}{rgb}{0.121569,0.466667,0.705882}%
\pgfsetstrokecolor{currentstroke}%
\pgfsetstrokeopacity{0.945408}%
\pgfsetdash{}{0pt}%
\pgfpathmoveto{\pgfqpoint{2.543232in}{1.624178in}}%
\pgfpathcurveto{\pgfqpoint{2.551468in}{1.624178in}}{\pgfqpoint{2.559368in}{1.627450in}}{\pgfqpoint{2.565192in}{1.633274in}}%
\pgfpathcurveto{\pgfqpoint{2.571016in}{1.639098in}}{\pgfqpoint{2.574288in}{1.646998in}}{\pgfqpoint{2.574288in}{1.655234in}}%
\pgfpathcurveto{\pgfqpoint{2.574288in}{1.663470in}}{\pgfqpoint{2.571016in}{1.671370in}}{\pgfqpoint{2.565192in}{1.677194in}}%
\pgfpathcurveto{\pgfqpoint{2.559368in}{1.683018in}}{\pgfqpoint{2.551468in}{1.686291in}}{\pgfqpoint{2.543232in}{1.686291in}}%
\pgfpathcurveto{\pgfqpoint{2.534995in}{1.686291in}}{\pgfqpoint{2.527095in}{1.683018in}}{\pgfqpoint{2.521271in}{1.677194in}}%
\pgfpathcurveto{\pgfqpoint{2.515447in}{1.671370in}}{\pgfqpoint{2.512175in}{1.663470in}}{\pgfqpoint{2.512175in}{1.655234in}}%
\pgfpathcurveto{\pgfqpoint{2.512175in}{1.646998in}}{\pgfqpoint{2.515447in}{1.639098in}}{\pgfqpoint{2.521271in}{1.633274in}}%
\pgfpathcurveto{\pgfqpoint{2.527095in}{1.627450in}}{\pgfqpoint{2.534995in}{1.624178in}}{\pgfqpoint{2.543232in}{1.624178in}}%
\pgfpathclose%
\pgfusepath{stroke,fill}%
\end{pgfscope}%
\begin{pgfscope}%
\pgfpathrectangle{\pgfqpoint{0.100000in}{0.212622in}}{\pgfqpoint{3.696000in}{3.696000in}}%
\pgfusepath{clip}%
\pgfsetbuttcap%
\pgfsetroundjoin%
\definecolor{currentfill}{rgb}{0.121569,0.466667,0.705882}%
\pgfsetfillcolor{currentfill}%
\pgfsetfillopacity{0.946371}%
\pgfsetlinewidth{1.003750pt}%
\definecolor{currentstroke}{rgb}{0.121569,0.466667,0.705882}%
\pgfsetstrokecolor{currentstroke}%
\pgfsetstrokeopacity{0.946371}%
\pgfsetdash{}{0pt}%
\pgfpathmoveto{\pgfqpoint{1.837825in}{1.841063in}}%
\pgfpathcurveto{\pgfqpoint{1.846061in}{1.841063in}}{\pgfqpoint{1.853961in}{1.844336in}}{\pgfqpoint{1.859785in}{1.850159in}}%
\pgfpathcurveto{\pgfqpoint{1.865609in}{1.855983in}}{\pgfqpoint{1.868881in}{1.863883in}}{\pgfqpoint{1.868881in}{1.872120in}}%
\pgfpathcurveto{\pgfqpoint{1.868881in}{1.880356in}}{\pgfqpoint{1.865609in}{1.888256in}}{\pgfqpoint{1.859785in}{1.894080in}}%
\pgfpathcurveto{\pgfqpoint{1.853961in}{1.899904in}}{\pgfqpoint{1.846061in}{1.903176in}}{\pgfqpoint{1.837825in}{1.903176in}}%
\pgfpathcurveto{\pgfqpoint{1.829588in}{1.903176in}}{\pgfqpoint{1.821688in}{1.899904in}}{\pgfqpoint{1.815864in}{1.894080in}}%
\pgfpathcurveto{\pgfqpoint{1.810040in}{1.888256in}}{\pgfqpoint{1.806768in}{1.880356in}}{\pgfqpoint{1.806768in}{1.872120in}}%
\pgfpathcurveto{\pgfqpoint{1.806768in}{1.863883in}}{\pgfqpoint{1.810040in}{1.855983in}}{\pgfqpoint{1.815864in}{1.850159in}}%
\pgfpathcurveto{\pgfqpoint{1.821688in}{1.844336in}}{\pgfqpoint{1.829588in}{1.841063in}}{\pgfqpoint{1.837825in}{1.841063in}}%
\pgfpathclose%
\pgfusepath{stroke,fill}%
\end{pgfscope}%
\begin{pgfscope}%
\pgfpathrectangle{\pgfqpoint{0.100000in}{0.212622in}}{\pgfqpoint{3.696000in}{3.696000in}}%
\pgfusepath{clip}%
\pgfsetbuttcap%
\pgfsetroundjoin%
\definecolor{currentfill}{rgb}{0.121569,0.466667,0.705882}%
\pgfsetfillcolor{currentfill}%
\pgfsetfillopacity{0.946971}%
\pgfsetlinewidth{1.003750pt}%
\definecolor{currentstroke}{rgb}{0.121569,0.466667,0.705882}%
\pgfsetstrokecolor{currentstroke}%
\pgfsetstrokeopacity{0.946971}%
\pgfsetdash{}{0pt}%
\pgfpathmoveto{\pgfqpoint{2.538125in}{1.625171in}}%
\pgfpathcurveto{\pgfqpoint{2.546361in}{1.625171in}}{\pgfqpoint{2.554261in}{1.628443in}}{\pgfqpoint{2.560085in}{1.634267in}}%
\pgfpathcurveto{\pgfqpoint{2.565909in}{1.640091in}}{\pgfqpoint{2.569182in}{1.647991in}}{\pgfqpoint{2.569182in}{1.656227in}}%
\pgfpathcurveto{\pgfqpoint{2.569182in}{1.664463in}}{\pgfqpoint{2.565909in}{1.672363in}}{\pgfqpoint{2.560085in}{1.678187in}}%
\pgfpathcurveto{\pgfqpoint{2.554261in}{1.684011in}}{\pgfqpoint{2.546361in}{1.687284in}}{\pgfqpoint{2.538125in}{1.687284in}}%
\pgfpathcurveto{\pgfqpoint{2.529889in}{1.687284in}}{\pgfqpoint{2.521989in}{1.684011in}}{\pgfqpoint{2.516165in}{1.678187in}}%
\pgfpathcurveto{\pgfqpoint{2.510341in}{1.672363in}}{\pgfqpoint{2.507069in}{1.664463in}}{\pgfqpoint{2.507069in}{1.656227in}}%
\pgfpathcurveto{\pgfqpoint{2.507069in}{1.647991in}}{\pgfqpoint{2.510341in}{1.640091in}}{\pgfqpoint{2.516165in}{1.634267in}}%
\pgfpathcurveto{\pgfqpoint{2.521989in}{1.628443in}}{\pgfqpoint{2.529889in}{1.625171in}}{\pgfqpoint{2.538125in}{1.625171in}}%
\pgfpathclose%
\pgfusepath{stroke,fill}%
\end{pgfscope}%
\begin{pgfscope}%
\pgfpathrectangle{\pgfqpoint{0.100000in}{0.212622in}}{\pgfqpoint{3.696000in}{3.696000in}}%
\pgfusepath{clip}%
\pgfsetbuttcap%
\pgfsetroundjoin%
\definecolor{currentfill}{rgb}{0.121569,0.466667,0.705882}%
\pgfsetfillcolor{currentfill}%
\pgfsetfillopacity{0.947227}%
\pgfsetlinewidth{1.003750pt}%
\definecolor{currentstroke}{rgb}{0.121569,0.466667,0.705882}%
\pgfsetstrokecolor{currentstroke}%
\pgfsetstrokeopacity{0.947227}%
\pgfsetdash{}{0pt}%
\pgfpathmoveto{\pgfqpoint{1.849021in}{1.829890in}}%
\pgfpathcurveto{\pgfqpoint{1.857257in}{1.829890in}}{\pgfqpoint{1.865157in}{1.833162in}}{\pgfqpoint{1.870981in}{1.838986in}}%
\pgfpathcurveto{\pgfqpoint{1.876805in}{1.844810in}}{\pgfqpoint{1.880078in}{1.852710in}}{\pgfqpoint{1.880078in}{1.860947in}}%
\pgfpathcurveto{\pgfqpoint{1.880078in}{1.869183in}}{\pgfqpoint{1.876805in}{1.877083in}}{\pgfqpoint{1.870981in}{1.882907in}}%
\pgfpathcurveto{\pgfqpoint{1.865157in}{1.888731in}}{\pgfqpoint{1.857257in}{1.892003in}}{\pgfqpoint{1.849021in}{1.892003in}}%
\pgfpathcurveto{\pgfqpoint{1.840785in}{1.892003in}}{\pgfqpoint{1.832885in}{1.888731in}}{\pgfqpoint{1.827061in}{1.882907in}}%
\pgfpathcurveto{\pgfqpoint{1.821237in}{1.877083in}}{\pgfqpoint{1.817965in}{1.869183in}}{\pgfqpoint{1.817965in}{1.860947in}}%
\pgfpathcurveto{\pgfqpoint{1.817965in}{1.852710in}}{\pgfqpoint{1.821237in}{1.844810in}}{\pgfqpoint{1.827061in}{1.838986in}}%
\pgfpathcurveto{\pgfqpoint{1.832885in}{1.833162in}}{\pgfqpoint{1.840785in}{1.829890in}}{\pgfqpoint{1.849021in}{1.829890in}}%
\pgfpathclose%
\pgfusepath{stroke,fill}%
\end{pgfscope}%
\begin{pgfscope}%
\pgfpathrectangle{\pgfqpoint{0.100000in}{0.212622in}}{\pgfqpoint{3.696000in}{3.696000in}}%
\pgfusepath{clip}%
\pgfsetbuttcap%
\pgfsetroundjoin%
\definecolor{currentfill}{rgb}{0.121569,0.466667,0.705882}%
\pgfsetfillcolor{currentfill}%
\pgfsetfillopacity{0.948721}%
\pgfsetlinewidth{1.003750pt}%
\definecolor{currentstroke}{rgb}{0.121569,0.466667,0.705882}%
\pgfsetstrokecolor{currentstroke}%
\pgfsetstrokeopacity{0.948721}%
\pgfsetdash{}{0pt}%
\pgfpathmoveto{\pgfqpoint{1.859325in}{1.828614in}}%
\pgfpathcurveto{\pgfqpoint{1.867561in}{1.828614in}}{\pgfqpoint{1.875461in}{1.831886in}}{\pgfqpoint{1.881285in}{1.837710in}}%
\pgfpathcurveto{\pgfqpoint{1.887109in}{1.843534in}}{\pgfqpoint{1.890381in}{1.851434in}}{\pgfqpoint{1.890381in}{1.859670in}}%
\pgfpathcurveto{\pgfqpoint{1.890381in}{1.867907in}}{\pgfqpoint{1.887109in}{1.875807in}}{\pgfqpoint{1.881285in}{1.881631in}}%
\pgfpathcurveto{\pgfqpoint{1.875461in}{1.887455in}}{\pgfqpoint{1.867561in}{1.890727in}}{\pgfqpoint{1.859325in}{1.890727in}}%
\pgfpathcurveto{\pgfqpoint{1.851089in}{1.890727in}}{\pgfqpoint{1.843188in}{1.887455in}}{\pgfqpoint{1.837365in}{1.881631in}}%
\pgfpathcurveto{\pgfqpoint{1.831541in}{1.875807in}}{\pgfqpoint{1.828268in}{1.867907in}}{\pgfqpoint{1.828268in}{1.859670in}}%
\pgfpathcurveto{\pgfqpoint{1.828268in}{1.851434in}}{\pgfqpoint{1.831541in}{1.843534in}}{\pgfqpoint{1.837365in}{1.837710in}}%
\pgfpathcurveto{\pgfqpoint{1.843188in}{1.831886in}}{\pgfqpoint{1.851089in}{1.828614in}}{\pgfqpoint{1.859325in}{1.828614in}}%
\pgfpathclose%
\pgfusepath{stroke,fill}%
\end{pgfscope}%
\begin{pgfscope}%
\pgfpathrectangle{\pgfqpoint{0.100000in}{0.212622in}}{\pgfqpoint{3.696000in}{3.696000in}}%
\pgfusepath{clip}%
\pgfsetbuttcap%
\pgfsetroundjoin%
\definecolor{currentfill}{rgb}{0.121569,0.466667,0.705882}%
\pgfsetfillcolor{currentfill}%
\pgfsetfillopacity{0.949246}%
\pgfsetlinewidth{1.003750pt}%
\definecolor{currentstroke}{rgb}{0.121569,0.466667,0.705882}%
\pgfsetstrokecolor{currentstroke}%
\pgfsetstrokeopacity{0.949246}%
\pgfsetdash{}{0pt}%
\pgfpathmoveto{\pgfqpoint{2.535454in}{1.623529in}}%
\pgfpathcurveto{\pgfqpoint{2.543690in}{1.623529in}}{\pgfqpoint{2.551590in}{1.626801in}}{\pgfqpoint{2.557414in}{1.632625in}}%
\pgfpathcurveto{\pgfqpoint{2.563238in}{1.638449in}}{\pgfqpoint{2.566511in}{1.646349in}}{\pgfqpoint{2.566511in}{1.654585in}}%
\pgfpathcurveto{\pgfqpoint{2.566511in}{1.662822in}}{\pgfqpoint{2.563238in}{1.670722in}}{\pgfqpoint{2.557414in}{1.676546in}}%
\pgfpathcurveto{\pgfqpoint{2.551590in}{1.682370in}}{\pgfqpoint{2.543690in}{1.685642in}}{\pgfqpoint{2.535454in}{1.685642in}}%
\pgfpathcurveto{\pgfqpoint{2.527218in}{1.685642in}}{\pgfqpoint{2.519318in}{1.682370in}}{\pgfqpoint{2.513494in}{1.676546in}}%
\pgfpathcurveto{\pgfqpoint{2.507670in}{1.670722in}}{\pgfqpoint{2.504398in}{1.662822in}}{\pgfqpoint{2.504398in}{1.654585in}}%
\pgfpathcurveto{\pgfqpoint{2.504398in}{1.646349in}}{\pgfqpoint{2.507670in}{1.638449in}}{\pgfqpoint{2.513494in}{1.632625in}}%
\pgfpathcurveto{\pgfqpoint{2.519318in}{1.626801in}}{\pgfqpoint{2.527218in}{1.623529in}}{\pgfqpoint{2.535454in}{1.623529in}}%
\pgfpathclose%
\pgfusepath{stroke,fill}%
\end{pgfscope}%
\begin{pgfscope}%
\pgfpathrectangle{\pgfqpoint{0.100000in}{0.212622in}}{\pgfqpoint{3.696000in}{3.696000in}}%
\pgfusepath{clip}%
\pgfsetbuttcap%
\pgfsetroundjoin%
\definecolor{currentfill}{rgb}{0.121569,0.466667,0.705882}%
\pgfsetfillcolor{currentfill}%
\pgfsetfillopacity{0.950724}%
\pgfsetlinewidth{1.003750pt}%
\definecolor{currentstroke}{rgb}{0.121569,0.466667,0.705882}%
\pgfsetstrokecolor{currentstroke}%
\pgfsetstrokeopacity{0.950724}%
\pgfsetdash{}{0pt}%
\pgfpathmoveto{\pgfqpoint{2.531190in}{1.616462in}}%
\pgfpathcurveto{\pgfqpoint{2.539427in}{1.616462in}}{\pgfqpoint{2.547327in}{1.619735in}}{\pgfqpoint{2.553151in}{1.625559in}}%
\pgfpathcurveto{\pgfqpoint{2.558974in}{1.631383in}}{\pgfqpoint{2.562247in}{1.639283in}}{\pgfqpoint{2.562247in}{1.647519in}}%
\pgfpathcurveto{\pgfqpoint{2.562247in}{1.655755in}}{\pgfqpoint{2.558974in}{1.663655in}}{\pgfqpoint{2.553151in}{1.669479in}}%
\pgfpathcurveto{\pgfqpoint{2.547327in}{1.675303in}}{\pgfqpoint{2.539427in}{1.678575in}}{\pgfqpoint{2.531190in}{1.678575in}}%
\pgfpathcurveto{\pgfqpoint{2.522954in}{1.678575in}}{\pgfqpoint{2.515054in}{1.675303in}}{\pgfqpoint{2.509230in}{1.669479in}}%
\pgfpathcurveto{\pgfqpoint{2.503406in}{1.663655in}}{\pgfqpoint{2.500134in}{1.655755in}}{\pgfqpoint{2.500134in}{1.647519in}}%
\pgfpathcurveto{\pgfqpoint{2.500134in}{1.639283in}}{\pgfqpoint{2.503406in}{1.631383in}}{\pgfqpoint{2.509230in}{1.625559in}}%
\pgfpathcurveto{\pgfqpoint{2.515054in}{1.619735in}}{\pgfqpoint{2.522954in}{1.616462in}}{\pgfqpoint{2.531190in}{1.616462in}}%
\pgfpathclose%
\pgfusepath{stroke,fill}%
\end{pgfscope}%
\begin{pgfscope}%
\pgfpathrectangle{\pgfqpoint{0.100000in}{0.212622in}}{\pgfqpoint{3.696000in}{3.696000in}}%
\pgfusepath{clip}%
\pgfsetbuttcap%
\pgfsetroundjoin%
\definecolor{currentfill}{rgb}{0.121569,0.466667,0.705882}%
\pgfsetfillcolor{currentfill}%
\pgfsetfillopacity{0.951376}%
\pgfsetlinewidth{1.003750pt}%
\definecolor{currentstroke}{rgb}{0.121569,0.466667,0.705882}%
\pgfsetstrokecolor{currentstroke}%
\pgfsetstrokeopacity{0.951376}%
\pgfsetdash{}{0pt}%
\pgfpathmoveto{\pgfqpoint{1.877598in}{1.824193in}}%
\pgfpathcurveto{\pgfqpoint{1.885835in}{1.824193in}}{\pgfqpoint{1.893735in}{1.827465in}}{\pgfqpoint{1.899559in}{1.833289in}}%
\pgfpathcurveto{\pgfqpoint{1.905382in}{1.839113in}}{\pgfqpoint{1.908655in}{1.847013in}}{\pgfqpoint{1.908655in}{1.855249in}}%
\pgfpathcurveto{\pgfqpoint{1.908655in}{1.863486in}}{\pgfqpoint{1.905382in}{1.871386in}}{\pgfqpoint{1.899559in}{1.877210in}}%
\pgfpathcurveto{\pgfqpoint{1.893735in}{1.883033in}}{\pgfqpoint{1.885835in}{1.886306in}}{\pgfqpoint{1.877598in}{1.886306in}}%
\pgfpathcurveto{\pgfqpoint{1.869362in}{1.886306in}}{\pgfqpoint{1.861462in}{1.883033in}}{\pgfqpoint{1.855638in}{1.877210in}}%
\pgfpathcurveto{\pgfqpoint{1.849814in}{1.871386in}}{\pgfqpoint{1.846542in}{1.863486in}}{\pgfqpoint{1.846542in}{1.855249in}}%
\pgfpathcurveto{\pgfqpoint{1.846542in}{1.847013in}}{\pgfqpoint{1.849814in}{1.839113in}}{\pgfqpoint{1.855638in}{1.833289in}}%
\pgfpathcurveto{\pgfqpoint{1.861462in}{1.827465in}}{\pgfqpoint{1.869362in}{1.824193in}}{\pgfqpoint{1.877598in}{1.824193in}}%
\pgfpathclose%
\pgfusepath{stroke,fill}%
\end{pgfscope}%
\begin{pgfscope}%
\pgfpathrectangle{\pgfqpoint{0.100000in}{0.212622in}}{\pgfqpoint{3.696000in}{3.696000in}}%
\pgfusepath{clip}%
\pgfsetbuttcap%
\pgfsetroundjoin%
\definecolor{currentfill}{rgb}{0.121569,0.466667,0.705882}%
\pgfsetfillcolor{currentfill}%
\pgfsetfillopacity{0.953060}%
\pgfsetlinewidth{1.003750pt}%
\definecolor{currentstroke}{rgb}{0.121569,0.466667,0.705882}%
\pgfsetstrokecolor{currentstroke}%
\pgfsetstrokeopacity{0.953060}%
\pgfsetdash{}{0pt}%
\pgfpathmoveto{\pgfqpoint{1.894581in}{1.811883in}}%
\pgfpathcurveto{\pgfqpoint{1.902817in}{1.811883in}}{\pgfqpoint{1.910717in}{1.815155in}}{\pgfqpoint{1.916541in}{1.820979in}}%
\pgfpathcurveto{\pgfqpoint{1.922365in}{1.826803in}}{\pgfqpoint{1.925637in}{1.834703in}}{\pgfqpoint{1.925637in}{1.842940in}}%
\pgfpathcurveto{\pgfqpoint{1.925637in}{1.851176in}}{\pgfqpoint{1.922365in}{1.859076in}}{\pgfqpoint{1.916541in}{1.864900in}}%
\pgfpathcurveto{\pgfqpoint{1.910717in}{1.870724in}}{\pgfqpoint{1.902817in}{1.873996in}}{\pgfqpoint{1.894581in}{1.873996in}}%
\pgfpathcurveto{\pgfqpoint{1.886344in}{1.873996in}}{\pgfqpoint{1.878444in}{1.870724in}}{\pgfqpoint{1.872620in}{1.864900in}}%
\pgfpathcurveto{\pgfqpoint{1.866796in}{1.859076in}}{\pgfqpoint{1.863524in}{1.851176in}}{\pgfqpoint{1.863524in}{1.842940in}}%
\pgfpathcurveto{\pgfqpoint{1.863524in}{1.834703in}}{\pgfqpoint{1.866796in}{1.826803in}}{\pgfqpoint{1.872620in}{1.820979in}}%
\pgfpathcurveto{\pgfqpoint{1.878444in}{1.815155in}}{\pgfqpoint{1.886344in}{1.811883in}}{\pgfqpoint{1.894581in}{1.811883in}}%
\pgfpathclose%
\pgfusepath{stroke,fill}%
\end{pgfscope}%
\begin{pgfscope}%
\pgfpathrectangle{\pgfqpoint{0.100000in}{0.212622in}}{\pgfqpoint{3.696000in}{3.696000in}}%
\pgfusepath{clip}%
\pgfsetbuttcap%
\pgfsetroundjoin%
\definecolor{currentfill}{rgb}{0.121569,0.466667,0.705882}%
\pgfsetfillcolor{currentfill}%
\pgfsetfillopacity{0.954120}%
\pgfsetlinewidth{1.003750pt}%
\definecolor{currentstroke}{rgb}{0.121569,0.466667,0.705882}%
\pgfsetstrokecolor{currentstroke}%
\pgfsetstrokeopacity{0.954120}%
\pgfsetdash{}{0pt}%
\pgfpathmoveto{\pgfqpoint{2.522603in}{1.623294in}}%
\pgfpathcurveto{\pgfqpoint{2.530839in}{1.623294in}}{\pgfqpoint{2.538739in}{1.626567in}}{\pgfqpoint{2.544563in}{1.632391in}}%
\pgfpathcurveto{\pgfqpoint{2.550387in}{1.638215in}}{\pgfqpoint{2.553659in}{1.646115in}}{\pgfqpoint{2.553659in}{1.654351in}}%
\pgfpathcurveto{\pgfqpoint{2.553659in}{1.662587in}}{\pgfqpoint{2.550387in}{1.670487in}}{\pgfqpoint{2.544563in}{1.676311in}}%
\pgfpathcurveto{\pgfqpoint{2.538739in}{1.682135in}}{\pgfqpoint{2.530839in}{1.685407in}}{\pgfqpoint{2.522603in}{1.685407in}}%
\pgfpathcurveto{\pgfqpoint{2.514366in}{1.685407in}}{\pgfqpoint{2.506466in}{1.682135in}}{\pgfqpoint{2.500642in}{1.676311in}}%
\pgfpathcurveto{\pgfqpoint{2.494818in}{1.670487in}}{\pgfqpoint{2.491546in}{1.662587in}}{\pgfqpoint{2.491546in}{1.654351in}}%
\pgfpathcurveto{\pgfqpoint{2.491546in}{1.646115in}}{\pgfqpoint{2.494818in}{1.638215in}}{\pgfqpoint{2.500642in}{1.632391in}}%
\pgfpathcurveto{\pgfqpoint{2.506466in}{1.626567in}}{\pgfqpoint{2.514366in}{1.623294in}}{\pgfqpoint{2.522603in}{1.623294in}}%
\pgfpathclose%
\pgfusepath{stroke,fill}%
\end{pgfscope}%
\begin{pgfscope}%
\pgfpathrectangle{\pgfqpoint{0.100000in}{0.212622in}}{\pgfqpoint{3.696000in}{3.696000in}}%
\pgfusepath{clip}%
\pgfsetbuttcap%
\pgfsetroundjoin%
\definecolor{currentfill}{rgb}{0.121569,0.466667,0.705882}%
\pgfsetfillcolor{currentfill}%
\pgfsetfillopacity{0.955323}%
\pgfsetlinewidth{1.003750pt}%
\definecolor{currentstroke}{rgb}{0.121569,0.466667,0.705882}%
\pgfsetstrokecolor{currentstroke}%
\pgfsetstrokeopacity{0.955323}%
\pgfsetdash{}{0pt}%
\pgfpathmoveto{\pgfqpoint{1.910676in}{1.808814in}}%
\pgfpathcurveto{\pgfqpoint{1.918912in}{1.808814in}}{\pgfqpoint{1.926812in}{1.812087in}}{\pgfqpoint{1.932636in}{1.817910in}}%
\pgfpathcurveto{\pgfqpoint{1.938460in}{1.823734in}}{\pgfqpoint{1.941732in}{1.831634in}}{\pgfqpoint{1.941732in}{1.839871in}}%
\pgfpathcurveto{\pgfqpoint{1.941732in}{1.848107in}}{\pgfqpoint{1.938460in}{1.856007in}}{\pgfqpoint{1.932636in}{1.861831in}}%
\pgfpathcurveto{\pgfqpoint{1.926812in}{1.867655in}}{\pgfqpoint{1.918912in}{1.870927in}}{\pgfqpoint{1.910676in}{1.870927in}}%
\pgfpathcurveto{\pgfqpoint{1.902440in}{1.870927in}}{\pgfqpoint{1.894540in}{1.867655in}}{\pgfqpoint{1.888716in}{1.861831in}}%
\pgfpathcurveto{\pgfqpoint{1.882892in}{1.856007in}}{\pgfqpoint{1.879619in}{1.848107in}}{\pgfqpoint{1.879619in}{1.839871in}}%
\pgfpathcurveto{\pgfqpoint{1.879619in}{1.831634in}}{\pgfqpoint{1.882892in}{1.823734in}}{\pgfqpoint{1.888716in}{1.817910in}}%
\pgfpathcurveto{\pgfqpoint{1.894540in}{1.812087in}}{\pgfqpoint{1.902440in}{1.808814in}}{\pgfqpoint{1.910676in}{1.808814in}}%
\pgfpathclose%
\pgfusepath{stroke,fill}%
\end{pgfscope}%
\begin{pgfscope}%
\pgfpathrectangle{\pgfqpoint{0.100000in}{0.212622in}}{\pgfqpoint{3.696000in}{3.696000in}}%
\pgfusepath{clip}%
\pgfsetbuttcap%
\pgfsetroundjoin%
\definecolor{currentfill}{rgb}{0.121569,0.466667,0.705882}%
\pgfsetfillcolor{currentfill}%
\pgfsetfillopacity{0.955422}%
\pgfsetlinewidth{1.003750pt}%
\definecolor{currentstroke}{rgb}{0.121569,0.466667,0.705882}%
\pgfsetstrokecolor{currentstroke}%
\pgfsetstrokeopacity{0.955422}%
\pgfsetdash{}{0pt}%
\pgfpathmoveto{\pgfqpoint{2.521013in}{1.619285in}}%
\pgfpathcurveto{\pgfqpoint{2.529249in}{1.619285in}}{\pgfqpoint{2.537149in}{1.622557in}}{\pgfqpoint{2.542973in}{1.628381in}}%
\pgfpathcurveto{\pgfqpoint{2.548797in}{1.634205in}}{\pgfqpoint{2.552070in}{1.642105in}}{\pgfqpoint{2.552070in}{1.650341in}}%
\pgfpathcurveto{\pgfqpoint{2.552070in}{1.658578in}}{\pgfqpoint{2.548797in}{1.666478in}}{\pgfqpoint{2.542973in}{1.672302in}}%
\pgfpathcurveto{\pgfqpoint{2.537149in}{1.678126in}}{\pgfqpoint{2.529249in}{1.681398in}}{\pgfqpoint{2.521013in}{1.681398in}}%
\pgfpathcurveto{\pgfqpoint{2.512777in}{1.681398in}}{\pgfqpoint{2.504877in}{1.678126in}}{\pgfqpoint{2.499053in}{1.672302in}}%
\pgfpathcurveto{\pgfqpoint{2.493229in}{1.666478in}}{\pgfqpoint{2.489957in}{1.658578in}}{\pgfqpoint{2.489957in}{1.650341in}}%
\pgfpathcurveto{\pgfqpoint{2.489957in}{1.642105in}}{\pgfqpoint{2.493229in}{1.634205in}}{\pgfqpoint{2.499053in}{1.628381in}}%
\pgfpathcurveto{\pgfqpoint{2.504877in}{1.622557in}}{\pgfqpoint{2.512777in}{1.619285in}}{\pgfqpoint{2.521013in}{1.619285in}}%
\pgfpathclose%
\pgfusepath{stroke,fill}%
\end{pgfscope}%
\begin{pgfscope}%
\pgfpathrectangle{\pgfqpoint{0.100000in}{0.212622in}}{\pgfqpoint{3.696000in}{3.696000in}}%
\pgfusepath{clip}%
\pgfsetbuttcap%
\pgfsetroundjoin%
\definecolor{currentfill}{rgb}{0.121569,0.466667,0.705882}%
\pgfsetfillcolor{currentfill}%
\pgfsetfillopacity{0.956087}%
\pgfsetlinewidth{1.003750pt}%
\definecolor{currentstroke}{rgb}{0.121569,0.466667,0.705882}%
\pgfsetstrokecolor{currentstroke}%
\pgfsetstrokeopacity{0.956087}%
\pgfsetdash{}{0pt}%
\pgfpathmoveto{\pgfqpoint{1.926200in}{1.796225in}}%
\pgfpathcurveto{\pgfqpoint{1.934437in}{1.796225in}}{\pgfqpoint{1.942337in}{1.799498in}}{\pgfqpoint{1.948161in}{1.805322in}}%
\pgfpathcurveto{\pgfqpoint{1.953985in}{1.811146in}}{\pgfqpoint{1.957257in}{1.819046in}}{\pgfqpoint{1.957257in}{1.827282in}}%
\pgfpathcurveto{\pgfqpoint{1.957257in}{1.835518in}}{\pgfqpoint{1.953985in}{1.843418in}}{\pgfqpoint{1.948161in}{1.849242in}}%
\pgfpathcurveto{\pgfqpoint{1.942337in}{1.855066in}}{\pgfqpoint{1.934437in}{1.858338in}}{\pgfqpoint{1.926200in}{1.858338in}}%
\pgfpathcurveto{\pgfqpoint{1.917964in}{1.858338in}}{\pgfqpoint{1.910064in}{1.855066in}}{\pgfqpoint{1.904240in}{1.849242in}}%
\pgfpathcurveto{\pgfqpoint{1.898416in}{1.843418in}}{\pgfqpoint{1.895144in}{1.835518in}}{\pgfqpoint{1.895144in}{1.827282in}}%
\pgfpathcurveto{\pgfqpoint{1.895144in}{1.819046in}}{\pgfqpoint{1.898416in}{1.811146in}}{\pgfqpoint{1.904240in}{1.805322in}}%
\pgfpathcurveto{\pgfqpoint{1.910064in}{1.799498in}}{\pgfqpoint{1.917964in}{1.796225in}}{\pgfqpoint{1.926200in}{1.796225in}}%
\pgfpathclose%
\pgfusepath{stroke,fill}%
\end{pgfscope}%
\begin{pgfscope}%
\pgfpathrectangle{\pgfqpoint{0.100000in}{0.212622in}}{\pgfqpoint{3.696000in}{3.696000in}}%
\pgfusepath{clip}%
\pgfsetbuttcap%
\pgfsetroundjoin%
\definecolor{currentfill}{rgb}{0.121569,0.466667,0.705882}%
\pgfsetfillcolor{currentfill}%
\pgfsetfillopacity{0.956117}%
\pgfsetlinewidth{1.003750pt}%
\definecolor{currentstroke}{rgb}{0.121569,0.466667,0.705882}%
\pgfsetstrokecolor{currentstroke}%
\pgfsetstrokeopacity{0.956117}%
\pgfsetdash{}{0pt}%
\pgfpathmoveto{\pgfqpoint{2.520014in}{1.617009in}}%
\pgfpathcurveto{\pgfqpoint{2.528250in}{1.617009in}}{\pgfqpoint{2.536150in}{1.620282in}}{\pgfqpoint{2.541974in}{1.626106in}}%
\pgfpathcurveto{\pgfqpoint{2.547798in}{1.631930in}}{\pgfqpoint{2.551070in}{1.639830in}}{\pgfqpoint{2.551070in}{1.648066in}}%
\pgfpathcurveto{\pgfqpoint{2.551070in}{1.656302in}}{\pgfqpoint{2.547798in}{1.664202in}}{\pgfqpoint{2.541974in}{1.670026in}}%
\pgfpathcurveto{\pgfqpoint{2.536150in}{1.675850in}}{\pgfqpoint{2.528250in}{1.679122in}}{\pgfqpoint{2.520014in}{1.679122in}}%
\pgfpathcurveto{\pgfqpoint{2.511777in}{1.679122in}}{\pgfqpoint{2.503877in}{1.675850in}}{\pgfqpoint{2.498053in}{1.670026in}}%
\pgfpathcurveto{\pgfqpoint{2.492230in}{1.664202in}}{\pgfqpoint{2.488957in}{1.656302in}}{\pgfqpoint{2.488957in}{1.648066in}}%
\pgfpathcurveto{\pgfqpoint{2.488957in}{1.639830in}}{\pgfqpoint{2.492230in}{1.631930in}}{\pgfqpoint{2.498053in}{1.626106in}}%
\pgfpathcurveto{\pgfqpoint{2.503877in}{1.620282in}}{\pgfqpoint{2.511777in}{1.617009in}}{\pgfqpoint{2.520014in}{1.617009in}}%
\pgfpathclose%
\pgfusepath{stroke,fill}%
\end{pgfscope}%
\begin{pgfscope}%
\pgfpathrectangle{\pgfqpoint{0.100000in}{0.212622in}}{\pgfqpoint{3.696000in}{3.696000in}}%
\pgfusepath{clip}%
\pgfsetbuttcap%
\pgfsetroundjoin%
\definecolor{currentfill}{rgb}{0.121569,0.466667,0.705882}%
\pgfsetfillcolor{currentfill}%
\pgfsetfillopacity{0.957343}%
\pgfsetlinewidth{1.003750pt}%
\definecolor{currentstroke}{rgb}{0.121569,0.466667,0.705882}%
\pgfsetstrokecolor{currentstroke}%
\pgfsetstrokeopacity{0.957343}%
\pgfsetdash{}{0pt}%
\pgfpathmoveto{\pgfqpoint{2.516946in}{1.614821in}}%
\pgfpathcurveto{\pgfqpoint{2.525182in}{1.614821in}}{\pgfqpoint{2.533082in}{1.618094in}}{\pgfqpoint{2.538906in}{1.623917in}}%
\pgfpathcurveto{\pgfqpoint{2.544730in}{1.629741in}}{\pgfqpoint{2.548002in}{1.637641in}}{\pgfqpoint{2.548002in}{1.645878in}}%
\pgfpathcurveto{\pgfqpoint{2.548002in}{1.654114in}}{\pgfqpoint{2.544730in}{1.662014in}}{\pgfqpoint{2.538906in}{1.667838in}}%
\pgfpathcurveto{\pgfqpoint{2.533082in}{1.673662in}}{\pgfqpoint{2.525182in}{1.676934in}}{\pgfqpoint{2.516946in}{1.676934in}}%
\pgfpathcurveto{\pgfqpoint{2.508709in}{1.676934in}}{\pgfqpoint{2.500809in}{1.673662in}}{\pgfqpoint{2.494985in}{1.667838in}}%
\pgfpathcurveto{\pgfqpoint{2.489162in}{1.662014in}}{\pgfqpoint{2.485889in}{1.654114in}}{\pgfqpoint{2.485889in}{1.645878in}}%
\pgfpathcurveto{\pgfqpoint{2.485889in}{1.637641in}}{\pgfqpoint{2.489162in}{1.629741in}}{\pgfqpoint{2.494985in}{1.623917in}}%
\pgfpathcurveto{\pgfqpoint{2.500809in}{1.618094in}}{\pgfqpoint{2.508709in}{1.614821in}}{\pgfqpoint{2.516946in}{1.614821in}}%
\pgfpathclose%
\pgfusepath{stroke,fill}%
\end{pgfscope}%
\begin{pgfscope}%
\pgfpathrectangle{\pgfqpoint{0.100000in}{0.212622in}}{\pgfqpoint{3.696000in}{3.696000in}}%
\pgfusepath{clip}%
\pgfsetbuttcap%
\pgfsetroundjoin%
\definecolor{currentfill}{rgb}{0.121569,0.466667,0.705882}%
\pgfsetfillcolor{currentfill}%
\pgfsetfillopacity{0.957444}%
\pgfsetlinewidth{1.003750pt}%
\definecolor{currentstroke}{rgb}{0.121569,0.466667,0.705882}%
\pgfsetstrokecolor{currentstroke}%
\pgfsetstrokeopacity{0.957444}%
\pgfsetdash{}{0pt}%
\pgfpathmoveto{\pgfqpoint{1.941347in}{1.787748in}}%
\pgfpathcurveto{\pgfqpoint{1.949583in}{1.787748in}}{\pgfqpoint{1.957484in}{1.791021in}}{\pgfqpoint{1.963307in}{1.796845in}}%
\pgfpathcurveto{\pgfqpoint{1.969131in}{1.802669in}}{\pgfqpoint{1.972404in}{1.810569in}}{\pgfqpoint{1.972404in}{1.818805in}}%
\pgfpathcurveto{\pgfqpoint{1.972404in}{1.827041in}}{\pgfqpoint{1.969131in}{1.834941in}}{\pgfqpoint{1.963307in}{1.840765in}}%
\pgfpathcurveto{\pgfqpoint{1.957484in}{1.846589in}}{\pgfqpoint{1.949583in}{1.849861in}}{\pgfqpoint{1.941347in}{1.849861in}}%
\pgfpathcurveto{\pgfqpoint{1.933111in}{1.849861in}}{\pgfqpoint{1.925211in}{1.846589in}}{\pgfqpoint{1.919387in}{1.840765in}}%
\pgfpathcurveto{\pgfqpoint{1.913563in}{1.834941in}}{\pgfqpoint{1.910291in}{1.827041in}}{\pgfqpoint{1.910291in}{1.818805in}}%
\pgfpathcurveto{\pgfqpoint{1.910291in}{1.810569in}}{\pgfqpoint{1.913563in}{1.802669in}}{\pgfqpoint{1.919387in}{1.796845in}}%
\pgfpathcurveto{\pgfqpoint{1.925211in}{1.791021in}}{\pgfqpoint{1.933111in}{1.787748in}}{\pgfqpoint{1.941347in}{1.787748in}}%
\pgfpathclose%
\pgfusepath{stroke,fill}%
\end{pgfscope}%
\begin{pgfscope}%
\pgfpathrectangle{\pgfqpoint{0.100000in}{0.212622in}}{\pgfqpoint{3.696000in}{3.696000in}}%
\pgfusepath{clip}%
\pgfsetbuttcap%
\pgfsetroundjoin%
\definecolor{currentfill}{rgb}{0.121569,0.466667,0.705882}%
\pgfsetfillcolor{currentfill}%
\pgfsetfillopacity{0.958094}%
\pgfsetlinewidth{1.003750pt}%
\definecolor{currentstroke}{rgb}{0.121569,0.466667,0.705882}%
\pgfsetstrokecolor{currentstroke}%
\pgfsetstrokeopacity{0.958094}%
\pgfsetdash{}{0pt}%
\pgfpathmoveto{\pgfqpoint{2.514752in}{1.614915in}}%
\pgfpathcurveto{\pgfqpoint{2.522989in}{1.614915in}}{\pgfqpoint{2.530889in}{1.618187in}}{\pgfqpoint{2.536713in}{1.624011in}}%
\pgfpathcurveto{\pgfqpoint{2.542537in}{1.629835in}}{\pgfqpoint{2.545809in}{1.637735in}}{\pgfqpoint{2.545809in}{1.645971in}}%
\pgfpathcurveto{\pgfqpoint{2.545809in}{1.654208in}}{\pgfqpoint{2.542537in}{1.662108in}}{\pgfqpoint{2.536713in}{1.667932in}}%
\pgfpathcurveto{\pgfqpoint{2.530889in}{1.673756in}}{\pgfqpoint{2.522989in}{1.677028in}}{\pgfqpoint{2.514752in}{1.677028in}}%
\pgfpathcurveto{\pgfqpoint{2.506516in}{1.677028in}}{\pgfqpoint{2.498616in}{1.673756in}}{\pgfqpoint{2.492792in}{1.667932in}}%
\pgfpathcurveto{\pgfqpoint{2.486968in}{1.662108in}}{\pgfqpoint{2.483696in}{1.654208in}}{\pgfqpoint{2.483696in}{1.645971in}}%
\pgfpathcurveto{\pgfqpoint{2.483696in}{1.637735in}}{\pgfqpoint{2.486968in}{1.629835in}}{\pgfqpoint{2.492792in}{1.624011in}}%
\pgfpathcurveto{\pgfqpoint{2.498616in}{1.618187in}}{\pgfqpoint{2.506516in}{1.614915in}}{\pgfqpoint{2.514752in}{1.614915in}}%
\pgfpathclose%
\pgfusepath{stroke,fill}%
\end{pgfscope}%
\begin{pgfscope}%
\pgfpathrectangle{\pgfqpoint{0.100000in}{0.212622in}}{\pgfqpoint{3.696000in}{3.696000in}}%
\pgfusepath{clip}%
\pgfsetbuttcap%
\pgfsetroundjoin%
\definecolor{currentfill}{rgb}{0.121569,0.466667,0.705882}%
\pgfsetfillcolor{currentfill}%
\pgfsetfillopacity{0.958571}%
\pgfsetlinewidth{1.003750pt}%
\definecolor{currentstroke}{rgb}{0.121569,0.466667,0.705882}%
\pgfsetstrokecolor{currentstroke}%
\pgfsetstrokeopacity{0.958571}%
\pgfsetdash{}{0pt}%
\pgfpathmoveto{\pgfqpoint{1.954091in}{1.782215in}}%
\pgfpathcurveto{\pgfqpoint{1.962327in}{1.782215in}}{\pgfqpoint{1.970227in}{1.785487in}}{\pgfqpoint{1.976051in}{1.791311in}}%
\pgfpathcurveto{\pgfqpoint{1.981875in}{1.797135in}}{\pgfqpoint{1.985148in}{1.805035in}}{\pgfqpoint{1.985148in}{1.813272in}}%
\pgfpathcurveto{\pgfqpoint{1.985148in}{1.821508in}}{\pgfqpoint{1.981875in}{1.829408in}}{\pgfqpoint{1.976051in}{1.835232in}}%
\pgfpathcurveto{\pgfqpoint{1.970227in}{1.841056in}}{\pgfqpoint{1.962327in}{1.844328in}}{\pgfqpoint{1.954091in}{1.844328in}}%
\pgfpathcurveto{\pgfqpoint{1.945855in}{1.844328in}}{\pgfqpoint{1.937955in}{1.841056in}}{\pgfqpoint{1.932131in}{1.835232in}}%
\pgfpathcurveto{\pgfqpoint{1.926307in}{1.829408in}}{\pgfqpoint{1.923035in}{1.821508in}}{\pgfqpoint{1.923035in}{1.813272in}}%
\pgfpathcurveto{\pgfqpoint{1.923035in}{1.805035in}}{\pgfqpoint{1.926307in}{1.797135in}}{\pgfqpoint{1.932131in}{1.791311in}}%
\pgfpathcurveto{\pgfqpoint{1.937955in}{1.785487in}}{\pgfqpoint{1.945855in}{1.782215in}}{\pgfqpoint{1.954091in}{1.782215in}}%
\pgfpathclose%
\pgfusepath{stroke,fill}%
\end{pgfscope}%
\begin{pgfscope}%
\pgfpathrectangle{\pgfqpoint{0.100000in}{0.212622in}}{\pgfqpoint{3.696000in}{3.696000in}}%
\pgfusepath{clip}%
\pgfsetbuttcap%
\pgfsetroundjoin%
\definecolor{currentfill}{rgb}{0.121569,0.466667,0.705882}%
\pgfsetfillcolor{currentfill}%
\pgfsetfillopacity{0.959121}%
\pgfsetlinewidth{1.003750pt}%
\definecolor{currentstroke}{rgb}{0.121569,0.466667,0.705882}%
\pgfsetstrokecolor{currentstroke}%
\pgfsetstrokeopacity{0.959121}%
\pgfsetdash{}{0pt}%
\pgfpathmoveto{\pgfqpoint{1.965370in}{1.772567in}}%
\pgfpathcurveto{\pgfqpoint{1.973606in}{1.772567in}}{\pgfqpoint{1.981507in}{1.775839in}}{\pgfqpoint{1.987330in}{1.781663in}}%
\pgfpathcurveto{\pgfqpoint{1.993154in}{1.787487in}}{\pgfqpoint{1.996427in}{1.795387in}}{\pgfqpoint{1.996427in}{1.803624in}}%
\pgfpathcurveto{\pgfqpoint{1.996427in}{1.811860in}}{\pgfqpoint{1.993154in}{1.819760in}}{\pgfqpoint{1.987330in}{1.825584in}}%
\pgfpathcurveto{\pgfqpoint{1.981507in}{1.831408in}}{\pgfqpoint{1.973606in}{1.834680in}}{\pgfqpoint{1.965370in}{1.834680in}}%
\pgfpathcurveto{\pgfqpoint{1.957134in}{1.834680in}}{\pgfqpoint{1.949234in}{1.831408in}}{\pgfqpoint{1.943410in}{1.825584in}}%
\pgfpathcurveto{\pgfqpoint{1.937586in}{1.819760in}}{\pgfqpoint{1.934314in}{1.811860in}}{\pgfqpoint{1.934314in}{1.803624in}}%
\pgfpathcurveto{\pgfqpoint{1.934314in}{1.795387in}}{\pgfqpoint{1.937586in}{1.787487in}}{\pgfqpoint{1.943410in}{1.781663in}}%
\pgfpathcurveto{\pgfqpoint{1.949234in}{1.775839in}}{\pgfqpoint{1.957134in}{1.772567in}}{\pgfqpoint{1.965370in}{1.772567in}}%
\pgfpathclose%
\pgfusepath{stroke,fill}%
\end{pgfscope}%
\begin{pgfscope}%
\pgfpathrectangle{\pgfqpoint{0.100000in}{0.212622in}}{\pgfqpoint{3.696000in}{3.696000in}}%
\pgfusepath{clip}%
\pgfsetbuttcap%
\pgfsetroundjoin%
\definecolor{currentfill}{rgb}{0.121569,0.466667,0.705882}%
\pgfsetfillcolor{currentfill}%
\pgfsetfillopacity{0.959393}%
\pgfsetlinewidth{1.003750pt}%
\definecolor{currentstroke}{rgb}{0.121569,0.466667,0.705882}%
\pgfsetstrokecolor{currentstroke}%
\pgfsetstrokeopacity{0.959393}%
\pgfsetdash{}{0pt}%
\pgfpathmoveto{\pgfqpoint{2.513257in}{1.613743in}}%
\pgfpathcurveto{\pgfqpoint{2.521493in}{1.613743in}}{\pgfqpoint{2.529393in}{1.617015in}}{\pgfqpoint{2.535217in}{1.622839in}}%
\pgfpathcurveto{\pgfqpoint{2.541041in}{1.628663in}}{\pgfqpoint{2.544314in}{1.636563in}}{\pgfqpoint{2.544314in}{1.644799in}}%
\pgfpathcurveto{\pgfqpoint{2.544314in}{1.653036in}}{\pgfqpoint{2.541041in}{1.660936in}}{\pgfqpoint{2.535217in}{1.666760in}}%
\pgfpathcurveto{\pgfqpoint{2.529393in}{1.672584in}}{\pgfqpoint{2.521493in}{1.675856in}}{\pgfqpoint{2.513257in}{1.675856in}}%
\pgfpathcurveto{\pgfqpoint{2.505021in}{1.675856in}}{\pgfqpoint{2.497121in}{1.672584in}}{\pgfqpoint{2.491297in}{1.666760in}}%
\pgfpathcurveto{\pgfqpoint{2.485473in}{1.660936in}}{\pgfqpoint{2.482201in}{1.653036in}}{\pgfqpoint{2.482201in}{1.644799in}}%
\pgfpathcurveto{\pgfqpoint{2.482201in}{1.636563in}}{\pgfqpoint{2.485473in}{1.628663in}}{\pgfqpoint{2.491297in}{1.622839in}}%
\pgfpathcurveto{\pgfqpoint{2.497121in}{1.617015in}}{\pgfqpoint{2.505021in}{1.613743in}}{\pgfqpoint{2.513257in}{1.613743in}}%
\pgfpathclose%
\pgfusepath{stroke,fill}%
\end{pgfscope}%
\begin{pgfscope}%
\pgfpathrectangle{\pgfqpoint{0.100000in}{0.212622in}}{\pgfqpoint{3.696000in}{3.696000in}}%
\pgfusepath{clip}%
\pgfsetbuttcap%
\pgfsetroundjoin%
\definecolor{currentfill}{rgb}{0.121569,0.466667,0.705882}%
\pgfsetfillcolor{currentfill}%
\pgfsetfillopacity{0.959783}%
\pgfsetlinewidth{1.003750pt}%
\definecolor{currentstroke}{rgb}{0.121569,0.466667,0.705882}%
\pgfsetstrokecolor{currentstroke}%
\pgfsetstrokeopacity{0.959783}%
\pgfsetdash{}{0pt}%
\pgfpathmoveto{\pgfqpoint{1.976696in}{1.765362in}}%
\pgfpathcurveto{\pgfqpoint{1.984932in}{1.765362in}}{\pgfqpoint{1.992832in}{1.768634in}}{\pgfqpoint{1.998656in}{1.774458in}}%
\pgfpathcurveto{\pgfqpoint{2.004480in}{1.780282in}}{\pgfqpoint{2.007753in}{1.788182in}}{\pgfqpoint{2.007753in}{1.796418in}}%
\pgfpathcurveto{\pgfqpoint{2.007753in}{1.804654in}}{\pgfqpoint{2.004480in}{1.812554in}}{\pgfqpoint{1.998656in}{1.818378in}}%
\pgfpathcurveto{\pgfqpoint{1.992832in}{1.824202in}}{\pgfqpoint{1.984932in}{1.827475in}}{\pgfqpoint{1.976696in}{1.827475in}}%
\pgfpathcurveto{\pgfqpoint{1.968460in}{1.827475in}}{\pgfqpoint{1.960560in}{1.824202in}}{\pgfqpoint{1.954736in}{1.818378in}}%
\pgfpathcurveto{\pgfqpoint{1.948912in}{1.812554in}}{\pgfqpoint{1.945640in}{1.804654in}}{\pgfqpoint{1.945640in}{1.796418in}}%
\pgfpathcurveto{\pgfqpoint{1.945640in}{1.788182in}}{\pgfqpoint{1.948912in}{1.780282in}}{\pgfqpoint{1.954736in}{1.774458in}}%
\pgfpathcurveto{\pgfqpoint{1.960560in}{1.768634in}}{\pgfqpoint{1.968460in}{1.765362in}}{\pgfqpoint{1.976696in}{1.765362in}}%
\pgfpathclose%
\pgfusepath{stroke,fill}%
\end{pgfscope}%
\begin{pgfscope}%
\pgfpathrectangle{\pgfqpoint{0.100000in}{0.212622in}}{\pgfqpoint{3.696000in}{3.696000in}}%
\pgfusepath{clip}%
\pgfsetbuttcap%
\pgfsetroundjoin%
\definecolor{currentfill}{rgb}{0.121569,0.466667,0.705882}%
\pgfsetfillcolor{currentfill}%
\pgfsetfillopacity{0.960249}%
\pgfsetlinewidth{1.003750pt}%
\definecolor{currentstroke}{rgb}{0.121569,0.466667,0.705882}%
\pgfsetstrokecolor{currentstroke}%
\pgfsetstrokeopacity{0.960249}%
\pgfsetdash{}{0pt}%
\pgfpathmoveto{\pgfqpoint{1.985149in}{1.760037in}}%
\pgfpathcurveto{\pgfqpoint{1.993385in}{1.760037in}}{\pgfqpoint{2.001285in}{1.763310in}}{\pgfqpoint{2.007109in}{1.769134in}}%
\pgfpathcurveto{\pgfqpoint{2.012933in}{1.774958in}}{\pgfqpoint{2.016205in}{1.782858in}}{\pgfqpoint{2.016205in}{1.791094in}}%
\pgfpathcurveto{\pgfqpoint{2.016205in}{1.799330in}}{\pgfqpoint{2.012933in}{1.807230in}}{\pgfqpoint{2.007109in}{1.813054in}}%
\pgfpathcurveto{\pgfqpoint{2.001285in}{1.818878in}}{\pgfqpoint{1.993385in}{1.822150in}}{\pgfqpoint{1.985149in}{1.822150in}}%
\pgfpathcurveto{\pgfqpoint{1.976912in}{1.822150in}}{\pgfqpoint{1.969012in}{1.818878in}}{\pgfqpoint{1.963188in}{1.813054in}}%
\pgfpathcurveto{\pgfqpoint{1.957365in}{1.807230in}}{\pgfqpoint{1.954092in}{1.799330in}}{\pgfqpoint{1.954092in}{1.791094in}}%
\pgfpathcurveto{\pgfqpoint{1.954092in}{1.782858in}}{\pgfqpoint{1.957365in}{1.774958in}}{\pgfqpoint{1.963188in}{1.769134in}}%
\pgfpathcurveto{\pgfqpoint{1.969012in}{1.763310in}}{\pgfqpoint{1.976912in}{1.760037in}}{\pgfqpoint{1.985149in}{1.760037in}}%
\pgfpathclose%
\pgfusepath{stroke,fill}%
\end{pgfscope}%
\begin{pgfscope}%
\pgfpathrectangle{\pgfqpoint{0.100000in}{0.212622in}}{\pgfqpoint{3.696000in}{3.696000in}}%
\pgfusepath{clip}%
\pgfsetbuttcap%
\pgfsetroundjoin%
\definecolor{currentfill}{rgb}{0.121569,0.466667,0.705882}%
\pgfsetfillcolor{currentfill}%
\pgfsetfillopacity{0.960645}%
\pgfsetlinewidth{1.003750pt}%
\definecolor{currentstroke}{rgb}{0.121569,0.466667,0.705882}%
\pgfsetstrokecolor{currentstroke}%
\pgfsetstrokeopacity{0.960645}%
\pgfsetdash{}{0pt}%
\pgfpathmoveto{\pgfqpoint{2.510941in}{1.611590in}}%
\pgfpathcurveto{\pgfqpoint{2.519178in}{1.611590in}}{\pgfqpoint{2.527078in}{1.614862in}}{\pgfqpoint{2.532901in}{1.620686in}}%
\pgfpathcurveto{\pgfqpoint{2.538725in}{1.626510in}}{\pgfqpoint{2.541998in}{1.634410in}}{\pgfqpoint{2.541998in}{1.642646in}}%
\pgfpathcurveto{\pgfqpoint{2.541998in}{1.650882in}}{\pgfqpoint{2.538725in}{1.658782in}}{\pgfqpoint{2.532901in}{1.664606in}}%
\pgfpathcurveto{\pgfqpoint{2.527078in}{1.670430in}}{\pgfqpoint{2.519178in}{1.673703in}}{\pgfqpoint{2.510941in}{1.673703in}}%
\pgfpathcurveto{\pgfqpoint{2.502705in}{1.673703in}}{\pgfqpoint{2.494805in}{1.670430in}}{\pgfqpoint{2.488981in}{1.664606in}}%
\pgfpathcurveto{\pgfqpoint{2.483157in}{1.658782in}}{\pgfqpoint{2.479885in}{1.650882in}}{\pgfqpoint{2.479885in}{1.642646in}}%
\pgfpathcurveto{\pgfqpoint{2.479885in}{1.634410in}}{\pgfqpoint{2.483157in}{1.626510in}}{\pgfqpoint{2.488981in}{1.620686in}}%
\pgfpathcurveto{\pgfqpoint{2.494805in}{1.614862in}}{\pgfqpoint{2.502705in}{1.611590in}}{\pgfqpoint{2.510941in}{1.611590in}}%
\pgfpathclose%
\pgfusepath{stroke,fill}%
\end{pgfscope}%
\begin{pgfscope}%
\pgfpathrectangle{\pgfqpoint{0.100000in}{0.212622in}}{\pgfqpoint{3.696000in}{3.696000in}}%
\pgfusepath{clip}%
\pgfsetbuttcap%
\pgfsetroundjoin%
\definecolor{currentfill}{rgb}{0.121569,0.466667,0.705882}%
\pgfsetfillcolor{currentfill}%
\pgfsetfillopacity{0.960882}%
\pgfsetlinewidth{1.003750pt}%
\definecolor{currentstroke}{rgb}{0.121569,0.466667,0.705882}%
\pgfsetstrokecolor{currentstroke}%
\pgfsetstrokeopacity{0.960882}%
\pgfsetdash{}{0pt}%
\pgfpathmoveto{\pgfqpoint{1.992598in}{1.754849in}}%
\pgfpathcurveto{\pgfqpoint{2.000834in}{1.754849in}}{\pgfqpoint{2.008734in}{1.758122in}}{\pgfqpoint{2.014558in}{1.763946in}}%
\pgfpathcurveto{\pgfqpoint{2.020382in}{1.769770in}}{\pgfqpoint{2.023654in}{1.777670in}}{\pgfqpoint{2.023654in}{1.785906in}}%
\pgfpathcurveto{\pgfqpoint{2.023654in}{1.794142in}}{\pgfqpoint{2.020382in}{1.802042in}}{\pgfqpoint{2.014558in}{1.807866in}}%
\pgfpathcurveto{\pgfqpoint{2.008734in}{1.813690in}}{\pgfqpoint{2.000834in}{1.816962in}}{\pgfqpoint{1.992598in}{1.816962in}}%
\pgfpathcurveto{\pgfqpoint{1.984361in}{1.816962in}}{\pgfqpoint{1.976461in}{1.813690in}}{\pgfqpoint{1.970637in}{1.807866in}}%
\pgfpathcurveto{\pgfqpoint{1.964813in}{1.802042in}}{\pgfqpoint{1.961541in}{1.794142in}}{\pgfqpoint{1.961541in}{1.785906in}}%
\pgfpathcurveto{\pgfqpoint{1.961541in}{1.777670in}}{\pgfqpoint{1.964813in}{1.769770in}}{\pgfqpoint{1.970637in}{1.763946in}}%
\pgfpathcurveto{\pgfqpoint{1.976461in}{1.758122in}}{\pgfqpoint{1.984361in}{1.754849in}}{\pgfqpoint{1.992598in}{1.754849in}}%
\pgfpathclose%
\pgfusepath{stroke,fill}%
\end{pgfscope}%
\begin{pgfscope}%
\pgfpathrectangle{\pgfqpoint{0.100000in}{0.212622in}}{\pgfqpoint{3.696000in}{3.696000in}}%
\pgfusepath{clip}%
\pgfsetbuttcap%
\pgfsetroundjoin%
\definecolor{currentfill}{rgb}{0.121569,0.466667,0.705882}%
\pgfsetfillcolor{currentfill}%
\pgfsetfillopacity{0.961818}%
\pgfsetlinewidth{1.003750pt}%
\definecolor{currentstroke}{rgb}{0.121569,0.466667,0.705882}%
\pgfsetstrokecolor{currentstroke}%
\pgfsetstrokeopacity{0.961818}%
\pgfsetdash{}{0pt}%
\pgfpathmoveto{\pgfqpoint{1.999697in}{1.752209in}}%
\pgfpathcurveto{\pgfqpoint{2.007933in}{1.752209in}}{\pgfqpoint{2.015833in}{1.755481in}}{\pgfqpoint{2.021657in}{1.761305in}}%
\pgfpathcurveto{\pgfqpoint{2.027481in}{1.767129in}}{\pgfqpoint{2.030753in}{1.775029in}}{\pgfqpoint{2.030753in}{1.783265in}}%
\pgfpathcurveto{\pgfqpoint{2.030753in}{1.791502in}}{\pgfqpoint{2.027481in}{1.799402in}}{\pgfqpoint{2.021657in}{1.805226in}}%
\pgfpathcurveto{\pgfqpoint{2.015833in}{1.811050in}}{\pgfqpoint{2.007933in}{1.814322in}}{\pgfqpoint{1.999697in}{1.814322in}}%
\pgfpathcurveto{\pgfqpoint{1.991460in}{1.814322in}}{\pgfqpoint{1.983560in}{1.811050in}}{\pgfqpoint{1.977737in}{1.805226in}}%
\pgfpathcurveto{\pgfqpoint{1.971913in}{1.799402in}}{\pgfqpoint{1.968640in}{1.791502in}}{\pgfqpoint{1.968640in}{1.783265in}}%
\pgfpathcurveto{\pgfqpoint{1.968640in}{1.775029in}}{\pgfqpoint{1.971913in}{1.767129in}}{\pgfqpoint{1.977737in}{1.761305in}}%
\pgfpathcurveto{\pgfqpoint{1.983560in}{1.755481in}}{\pgfqpoint{1.991460in}{1.752209in}}{\pgfqpoint{1.999697in}{1.752209in}}%
\pgfpathclose%
\pgfusepath{stroke,fill}%
\end{pgfscope}%
\begin{pgfscope}%
\pgfpathrectangle{\pgfqpoint{0.100000in}{0.212622in}}{\pgfqpoint{3.696000in}{3.696000in}}%
\pgfusepath{clip}%
\pgfsetbuttcap%
\pgfsetroundjoin%
\definecolor{currentfill}{rgb}{0.121569,0.466667,0.705882}%
\pgfsetfillcolor{currentfill}%
\pgfsetfillopacity{0.962454}%
\pgfsetlinewidth{1.003750pt}%
\definecolor{currentstroke}{rgb}{0.121569,0.466667,0.705882}%
\pgfsetstrokecolor{currentstroke}%
\pgfsetstrokeopacity{0.962454}%
\pgfsetdash{}{0pt}%
\pgfpathmoveto{\pgfqpoint{2.004289in}{1.750306in}}%
\pgfpathcurveto{\pgfqpoint{2.012525in}{1.750306in}}{\pgfqpoint{2.020426in}{1.753578in}}{\pgfqpoint{2.026249in}{1.759402in}}%
\pgfpathcurveto{\pgfqpoint{2.032073in}{1.765226in}}{\pgfqpoint{2.035346in}{1.773126in}}{\pgfqpoint{2.035346in}{1.781362in}}%
\pgfpathcurveto{\pgfqpoint{2.035346in}{1.789599in}}{\pgfqpoint{2.032073in}{1.797499in}}{\pgfqpoint{2.026249in}{1.803323in}}%
\pgfpathcurveto{\pgfqpoint{2.020426in}{1.809146in}}{\pgfqpoint{2.012525in}{1.812419in}}{\pgfqpoint{2.004289in}{1.812419in}}%
\pgfpathcurveto{\pgfqpoint{1.996053in}{1.812419in}}{\pgfqpoint{1.988153in}{1.809146in}}{\pgfqpoint{1.982329in}{1.803323in}}%
\pgfpathcurveto{\pgfqpoint{1.976505in}{1.797499in}}{\pgfqpoint{1.973233in}{1.789599in}}{\pgfqpoint{1.973233in}{1.781362in}}%
\pgfpathcurveto{\pgfqpoint{1.973233in}{1.773126in}}{\pgfqpoint{1.976505in}{1.765226in}}{\pgfqpoint{1.982329in}{1.759402in}}%
\pgfpathcurveto{\pgfqpoint{1.988153in}{1.753578in}}{\pgfqpoint{1.996053in}{1.750306in}}{\pgfqpoint{2.004289in}{1.750306in}}%
\pgfpathclose%
\pgfusepath{stroke,fill}%
\end{pgfscope}%
\begin{pgfscope}%
\pgfpathrectangle{\pgfqpoint{0.100000in}{0.212622in}}{\pgfqpoint{3.696000in}{3.696000in}}%
\pgfusepath{clip}%
\pgfsetbuttcap%
\pgfsetroundjoin%
\definecolor{currentfill}{rgb}{0.121569,0.466667,0.705882}%
\pgfsetfillcolor{currentfill}%
\pgfsetfillopacity{0.962487}%
\pgfsetlinewidth{1.003750pt}%
\definecolor{currentstroke}{rgb}{0.121569,0.466667,0.705882}%
\pgfsetstrokecolor{currentstroke}%
\pgfsetstrokeopacity{0.962487}%
\pgfsetdash{}{0pt}%
\pgfpathmoveto{\pgfqpoint{2.505693in}{1.611029in}}%
\pgfpathcurveto{\pgfqpoint{2.513929in}{1.611029in}}{\pgfqpoint{2.521830in}{1.614301in}}{\pgfqpoint{2.527653in}{1.620125in}}%
\pgfpathcurveto{\pgfqpoint{2.533477in}{1.625949in}}{\pgfqpoint{2.536750in}{1.633849in}}{\pgfqpoint{2.536750in}{1.642085in}}%
\pgfpathcurveto{\pgfqpoint{2.536750in}{1.650322in}}{\pgfqpoint{2.533477in}{1.658222in}}{\pgfqpoint{2.527653in}{1.664046in}}%
\pgfpathcurveto{\pgfqpoint{2.521830in}{1.669870in}}{\pgfqpoint{2.513929in}{1.673142in}}{\pgfqpoint{2.505693in}{1.673142in}}%
\pgfpathcurveto{\pgfqpoint{2.497457in}{1.673142in}}{\pgfqpoint{2.489557in}{1.669870in}}{\pgfqpoint{2.483733in}{1.664046in}}%
\pgfpathcurveto{\pgfqpoint{2.477909in}{1.658222in}}{\pgfqpoint{2.474637in}{1.650322in}}{\pgfqpoint{2.474637in}{1.642085in}}%
\pgfpathcurveto{\pgfqpoint{2.474637in}{1.633849in}}{\pgfqpoint{2.477909in}{1.625949in}}{\pgfqpoint{2.483733in}{1.620125in}}%
\pgfpathcurveto{\pgfqpoint{2.489557in}{1.614301in}}{\pgfqpoint{2.497457in}{1.611029in}}{\pgfqpoint{2.505693in}{1.611029in}}%
\pgfpathclose%
\pgfusepath{stroke,fill}%
\end{pgfscope}%
\begin{pgfscope}%
\pgfpathrectangle{\pgfqpoint{0.100000in}{0.212622in}}{\pgfqpoint{3.696000in}{3.696000in}}%
\pgfusepath{clip}%
\pgfsetbuttcap%
\pgfsetroundjoin%
\definecolor{currentfill}{rgb}{0.121569,0.466667,0.705882}%
\pgfsetfillcolor{currentfill}%
\pgfsetfillopacity{0.963384}%
\pgfsetlinewidth{1.003750pt}%
\definecolor{currentstroke}{rgb}{0.121569,0.466667,0.705882}%
\pgfsetstrokecolor{currentstroke}%
\pgfsetstrokeopacity{0.963384}%
\pgfsetdash{}{0pt}%
\pgfpathmoveto{\pgfqpoint{2.503853in}{1.608581in}}%
\pgfpathcurveto{\pgfqpoint{2.512089in}{1.608581in}}{\pgfqpoint{2.519989in}{1.611853in}}{\pgfqpoint{2.525813in}{1.617677in}}%
\pgfpathcurveto{\pgfqpoint{2.531637in}{1.623501in}}{\pgfqpoint{2.534910in}{1.631401in}}{\pgfqpoint{2.534910in}{1.639637in}}%
\pgfpathcurveto{\pgfqpoint{2.534910in}{1.647873in}}{\pgfqpoint{2.531637in}{1.655773in}}{\pgfqpoint{2.525813in}{1.661597in}}%
\pgfpathcurveto{\pgfqpoint{2.519989in}{1.667421in}}{\pgfqpoint{2.512089in}{1.670694in}}{\pgfqpoint{2.503853in}{1.670694in}}%
\pgfpathcurveto{\pgfqpoint{2.495617in}{1.670694in}}{\pgfqpoint{2.487717in}{1.667421in}}{\pgfqpoint{2.481893in}{1.661597in}}%
\pgfpathcurveto{\pgfqpoint{2.476069in}{1.655773in}}{\pgfqpoint{2.472797in}{1.647873in}}{\pgfqpoint{2.472797in}{1.639637in}}%
\pgfpathcurveto{\pgfqpoint{2.472797in}{1.631401in}}{\pgfqpoint{2.476069in}{1.623501in}}{\pgfqpoint{2.481893in}{1.617677in}}%
\pgfpathcurveto{\pgfqpoint{2.487717in}{1.611853in}}{\pgfqpoint{2.495617in}{1.608581in}}{\pgfqpoint{2.503853in}{1.608581in}}%
\pgfpathclose%
\pgfusepath{stroke,fill}%
\end{pgfscope}%
\begin{pgfscope}%
\pgfpathrectangle{\pgfqpoint{0.100000in}{0.212622in}}{\pgfqpoint{3.696000in}{3.696000in}}%
\pgfusepath{clip}%
\pgfsetbuttcap%
\pgfsetroundjoin%
\definecolor{currentfill}{rgb}{0.121569,0.466667,0.705882}%
\pgfsetfillcolor{currentfill}%
\pgfsetfillopacity{0.964031}%
\pgfsetlinewidth{1.003750pt}%
\definecolor{currentstroke}{rgb}{0.121569,0.466667,0.705882}%
\pgfsetstrokecolor{currentstroke}%
\pgfsetstrokeopacity{0.964031}%
\pgfsetdash{}{0pt}%
\pgfpathmoveto{\pgfqpoint{2.012737in}{1.749898in}}%
\pgfpathcurveto{\pgfqpoint{2.020973in}{1.749898in}}{\pgfqpoint{2.028873in}{1.753170in}}{\pgfqpoint{2.034697in}{1.758994in}}%
\pgfpathcurveto{\pgfqpoint{2.040521in}{1.764818in}}{\pgfqpoint{2.043793in}{1.772718in}}{\pgfqpoint{2.043793in}{1.780954in}}%
\pgfpathcurveto{\pgfqpoint{2.043793in}{1.789190in}}{\pgfqpoint{2.040521in}{1.797090in}}{\pgfqpoint{2.034697in}{1.802914in}}%
\pgfpathcurveto{\pgfqpoint{2.028873in}{1.808738in}}{\pgfqpoint{2.020973in}{1.812011in}}{\pgfqpoint{2.012737in}{1.812011in}}%
\pgfpathcurveto{\pgfqpoint{2.004500in}{1.812011in}}{\pgfqpoint{1.996600in}{1.808738in}}{\pgfqpoint{1.990776in}{1.802914in}}%
\pgfpathcurveto{\pgfqpoint{1.984953in}{1.797090in}}{\pgfqpoint{1.981680in}{1.789190in}}{\pgfqpoint{1.981680in}{1.780954in}}%
\pgfpathcurveto{\pgfqpoint{1.981680in}{1.772718in}}{\pgfqpoint{1.984953in}{1.764818in}}{\pgfqpoint{1.990776in}{1.758994in}}%
\pgfpathcurveto{\pgfqpoint{1.996600in}{1.753170in}}{\pgfqpoint{2.004500in}{1.749898in}}{\pgfqpoint{2.012737in}{1.749898in}}%
\pgfpathclose%
\pgfusepath{stroke,fill}%
\end{pgfscope}%
\begin{pgfscope}%
\pgfpathrectangle{\pgfqpoint{0.100000in}{0.212622in}}{\pgfqpoint{3.696000in}{3.696000in}}%
\pgfusepath{clip}%
\pgfsetbuttcap%
\pgfsetroundjoin%
\definecolor{currentfill}{rgb}{0.121569,0.466667,0.705882}%
\pgfsetfillcolor{currentfill}%
\pgfsetfillopacity{0.964033}%
\pgfsetlinewidth{1.003750pt}%
\definecolor{currentstroke}{rgb}{0.121569,0.466667,0.705882}%
\pgfsetstrokecolor{currentstroke}%
\pgfsetstrokeopacity{0.964033}%
\pgfsetdash{}{0pt}%
\pgfpathmoveto{\pgfqpoint{2.503246in}{1.607974in}}%
\pgfpathcurveto{\pgfqpoint{2.511482in}{1.607974in}}{\pgfqpoint{2.519382in}{1.611247in}}{\pgfqpoint{2.525206in}{1.617071in}}%
\pgfpathcurveto{\pgfqpoint{2.531030in}{1.622894in}}{\pgfqpoint{2.534302in}{1.630794in}}{\pgfqpoint{2.534302in}{1.639031in}}%
\pgfpathcurveto{\pgfqpoint{2.534302in}{1.647267in}}{\pgfqpoint{2.531030in}{1.655167in}}{\pgfqpoint{2.525206in}{1.660991in}}%
\pgfpathcurveto{\pgfqpoint{2.519382in}{1.666815in}}{\pgfqpoint{2.511482in}{1.670087in}}{\pgfqpoint{2.503246in}{1.670087in}}%
\pgfpathcurveto{\pgfqpoint{2.495009in}{1.670087in}}{\pgfqpoint{2.487109in}{1.666815in}}{\pgfqpoint{2.481285in}{1.660991in}}%
\pgfpathcurveto{\pgfqpoint{2.475462in}{1.655167in}}{\pgfqpoint{2.472189in}{1.647267in}}{\pgfqpoint{2.472189in}{1.639031in}}%
\pgfpathcurveto{\pgfqpoint{2.472189in}{1.630794in}}{\pgfqpoint{2.475462in}{1.622894in}}{\pgfqpoint{2.481285in}{1.617071in}}%
\pgfpathcurveto{\pgfqpoint{2.487109in}{1.611247in}}{\pgfqpoint{2.495009in}{1.607974in}}{\pgfqpoint{2.503246in}{1.607974in}}%
\pgfpathclose%
\pgfusepath{stroke,fill}%
\end{pgfscope}%
\begin{pgfscope}%
\pgfpathrectangle{\pgfqpoint{0.100000in}{0.212622in}}{\pgfqpoint{3.696000in}{3.696000in}}%
\pgfusepath{clip}%
\pgfsetbuttcap%
\pgfsetroundjoin%
\definecolor{currentfill}{rgb}{0.121569,0.466667,0.705882}%
\pgfsetfillcolor{currentfill}%
\pgfsetfillopacity{0.964780}%
\pgfsetlinewidth{1.003750pt}%
\definecolor{currentstroke}{rgb}{0.121569,0.466667,0.705882}%
\pgfsetstrokecolor{currentstroke}%
\pgfsetstrokeopacity{0.964780}%
\pgfsetdash{}{0pt}%
\pgfpathmoveto{\pgfqpoint{2.500286in}{1.605488in}}%
\pgfpathcurveto{\pgfqpoint{2.508522in}{1.605488in}}{\pgfqpoint{2.516422in}{1.608760in}}{\pgfqpoint{2.522246in}{1.614584in}}%
\pgfpathcurveto{\pgfqpoint{2.528070in}{1.620408in}}{\pgfqpoint{2.531342in}{1.628308in}}{\pgfqpoint{2.531342in}{1.636544in}}%
\pgfpathcurveto{\pgfqpoint{2.531342in}{1.644780in}}{\pgfqpoint{2.528070in}{1.652681in}}{\pgfqpoint{2.522246in}{1.658504in}}%
\pgfpathcurveto{\pgfqpoint{2.516422in}{1.664328in}}{\pgfqpoint{2.508522in}{1.667601in}}{\pgfqpoint{2.500286in}{1.667601in}}%
\pgfpathcurveto{\pgfqpoint{2.492049in}{1.667601in}}{\pgfqpoint{2.484149in}{1.664328in}}{\pgfqpoint{2.478325in}{1.658504in}}%
\pgfpathcurveto{\pgfqpoint{2.472502in}{1.652681in}}{\pgfqpoint{2.469229in}{1.644780in}}{\pgfqpoint{2.469229in}{1.636544in}}%
\pgfpathcurveto{\pgfqpoint{2.469229in}{1.628308in}}{\pgfqpoint{2.472502in}{1.620408in}}{\pgfqpoint{2.478325in}{1.614584in}}%
\pgfpathcurveto{\pgfqpoint{2.484149in}{1.608760in}}{\pgfqpoint{2.492049in}{1.605488in}}{\pgfqpoint{2.500286in}{1.605488in}}%
\pgfpathclose%
\pgfusepath{stroke,fill}%
\end{pgfscope}%
\begin{pgfscope}%
\pgfpathrectangle{\pgfqpoint{0.100000in}{0.212622in}}{\pgfqpoint{3.696000in}{3.696000in}}%
\pgfusepath{clip}%
\pgfsetbuttcap%
\pgfsetroundjoin%
\definecolor{currentfill}{rgb}{0.121569,0.466667,0.705882}%
\pgfsetfillcolor{currentfill}%
\pgfsetfillopacity{0.964988}%
\pgfsetlinewidth{1.003750pt}%
\definecolor{currentstroke}{rgb}{0.121569,0.466667,0.705882}%
\pgfsetstrokecolor{currentstroke}%
\pgfsetstrokeopacity{0.964988}%
\pgfsetdash{}{0pt}%
\pgfpathmoveto{\pgfqpoint{2.020786in}{1.746805in}}%
\pgfpathcurveto{\pgfqpoint{2.029022in}{1.746805in}}{\pgfqpoint{2.036922in}{1.750078in}}{\pgfqpoint{2.042746in}{1.755902in}}%
\pgfpathcurveto{\pgfqpoint{2.048570in}{1.761726in}}{\pgfqpoint{2.051842in}{1.769626in}}{\pgfqpoint{2.051842in}{1.777862in}}%
\pgfpathcurveto{\pgfqpoint{2.051842in}{1.786098in}}{\pgfqpoint{2.048570in}{1.793998in}}{\pgfqpoint{2.042746in}{1.799822in}}%
\pgfpathcurveto{\pgfqpoint{2.036922in}{1.805646in}}{\pgfqpoint{2.029022in}{1.808918in}}{\pgfqpoint{2.020786in}{1.808918in}}%
\pgfpathcurveto{\pgfqpoint{2.012549in}{1.808918in}}{\pgfqpoint{2.004649in}{1.805646in}}{\pgfqpoint{1.998825in}{1.799822in}}%
\pgfpathcurveto{\pgfqpoint{1.993001in}{1.793998in}}{\pgfqpoint{1.989729in}{1.786098in}}{\pgfqpoint{1.989729in}{1.777862in}}%
\pgfpathcurveto{\pgfqpoint{1.989729in}{1.769626in}}{\pgfqpoint{1.993001in}{1.761726in}}{\pgfqpoint{1.998825in}{1.755902in}}%
\pgfpathcurveto{\pgfqpoint{2.004649in}{1.750078in}}{\pgfqpoint{2.012549in}{1.746805in}}{\pgfqpoint{2.020786in}{1.746805in}}%
\pgfpathclose%
\pgfusepath{stroke,fill}%
\end{pgfscope}%
\begin{pgfscope}%
\pgfpathrectangle{\pgfqpoint{0.100000in}{0.212622in}}{\pgfqpoint{3.696000in}{3.696000in}}%
\pgfusepath{clip}%
\pgfsetbuttcap%
\pgfsetroundjoin%
\definecolor{currentfill}{rgb}{0.121569,0.466667,0.705882}%
\pgfsetfillcolor{currentfill}%
\pgfsetfillopacity{0.965620}%
\pgfsetlinewidth{1.003750pt}%
\definecolor{currentstroke}{rgb}{0.121569,0.466667,0.705882}%
\pgfsetstrokecolor{currentstroke}%
\pgfsetstrokeopacity{0.965620}%
\pgfsetdash{}{0pt}%
\pgfpathmoveto{\pgfqpoint{2.026749in}{1.743832in}}%
\pgfpathcurveto{\pgfqpoint{2.034985in}{1.743832in}}{\pgfqpoint{2.042885in}{1.747104in}}{\pgfqpoint{2.048709in}{1.752928in}}%
\pgfpathcurveto{\pgfqpoint{2.054533in}{1.758752in}}{\pgfqpoint{2.057805in}{1.766652in}}{\pgfqpoint{2.057805in}{1.774889in}}%
\pgfpathcurveto{\pgfqpoint{2.057805in}{1.783125in}}{\pgfqpoint{2.054533in}{1.791025in}}{\pgfqpoint{2.048709in}{1.796849in}}%
\pgfpathcurveto{\pgfqpoint{2.042885in}{1.802673in}}{\pgfqpoint{2.034985in}{1.805945in}}{\pgfqpoint{2.026749in}{1.805945in}}%
\pgfpathcurveto{\pgfqpoint{2.018513in}{1.805945in}}{\pgfqpoint{2.010613in}{1.802673in}}{\pgfqpoint{2.004789in}{1.796849in}}%
\pgfpathcurveto{\pgfqpoint{1.998965in}{1.791025in}}{\pgfqpoint{1.995692in}{1.783125in}}{\pgfqpoint{1.995692in}{1.774889in}}%
\pgfpathcurveto{\pgfqpoint{1.995692in}{1.766652in}}{\pgfqpoint{1.998965in}{1.758752in}}{\pgfqpoint{2.004789in}{1.752928in}}%
\pgfpathcurveto{\pgfqpoint{2.010613in}{1.747104in}}{\pgfqpoint{2.018513in}{1.743832in}}{\pgfqpoint{2.026749in}{1.743832in}}%
\pgfpathclose%
\pgfusepath{stroke,fill}%
\end{pgfscope}%
\begin{pgfscope}%
\pgfpathrectangle{\pgfqpoint{0.100000in}{0.212622in}}{\pgfqpoint{3.696000in}{3.696000in}}%
\pgfusepath{clip}%
\pgfsetbuttcap%
\pgfsetroundjoin%
\definecolor{currentfill}{rgb}{0.121569,0.466667,0.705882}%
\pgfsetfillcolor{currentfill}%
\pgfsetfillopacity{0.966198}%
\pgfsetlinewidth{1.003750pt}%
\definecolor{currentstroke}{rgb}{0.121569,0.466667,0.705882}%
\pgfsetstrokecolor{currentstroke}%
\pgfsetstrokeopacity{0.966198}%
\pgfsetdash{}{0pt}%
\pgfpathmoveto{\pgfqpoint{2.496407in}{1.606388in}}%
\pgfpathcurveto{\pgfqpoint{2.504643in}{1.606388in}}{\pgfqpoint{2.512543in}{1.609661in}}{\pgfqpoint{2.518367in}{1.615485in}}%
\pgfpathcurveto{\pgfqpoint{2.524191in}{1.621309in}}{\pgfqpoint{2.527463in}{1.629209in}}{\pgfqpoint{2.527463in}{1.637445in}}%
\pgfpathcurveto{\pgfqpoint{2.527463in}{1.645681in}}{\pgfqpoint{2.524191in}{1.653581in}}{\pgfqpoint{2.518367in}{1.659405in}}%
\pgfpathcurveto{\pgfqpoint{2.512543in}{1.665229in}}{\pgfqpoint{2.504643in}{1.668501in}}{\pgfqpoint{2.496407in}{1.668501in}}%
\pgfpathcurveto{\pgfqpoint{2.488171in}{1.668501in}}{\pgfqpoint{2.480271in}{1.665229in}}{\pgfqpoint{2.474447in}{1.659405in}}%
\pgfpathcurveto{\pgfqpoint{2.468623in}{1.653581in}}{\pgfqpoint{2.465350in}{1.645681in}}{\pgfqpoint{2.465350in}{1.637445in}}%
\pgfpathcurveto{\pgfqpoint{2.465350in}{1.629209in}}{\pgfqpoint{2.468623in}{1.621309in}}{\pgfqpoint{2.474447in}{1.615485in}}%
\pgfpathcurveto{\pgfqpoint{2.480271in}{1.609661in}}{\pgfqpoint{2.488171in}{1.606388in}}{\pgfqpoint{2.496407in}{1.606388in}}%
\pgfpathclose%
\pgfusepath{stroke,fill}%
\end{pgfscope}%
\begin{pgfscope}%
\pgfpathrectangle{\pgfqpoint{0.100000in}{0.212622in}}{\pgfqpoint{3.696000in}{3.696000in}}%
\pgfusepath{clip}%
\pgfsetbuttcap%
\pgfsetroundjoin%
\definecolor{currentfill}{rgb}{0.121569,0.466667,0.705882}%
\pgfsetfillcolor{currentfill}%
\pgfsetfillopacity{0.966843}%
\pgfsetlinewidth{1.003750pt}%
\definecolor{currentstroke}{rgb}{0.121569,0.466667,0.705882}%
\pgfsetstrokecolor{currentstroke}%
\pgfsetstrokeopacity{0.966843}%
\pgfsetdash{}{0pt}%
\pgfpathmoveto{\pgfqpoint{2.038005in}{1.740375in}}%
\pgfpathcurveto{\pgfqpoint{2.046241in}{1.740375in}}{\pgfqpoint{2.054141in}{1.743647in}}{\pgfqpoint{2.059965in}{1.749471in}}%
\pgfpathcurveto{\pgfqpoint{2.065789in}{1.755295in}}{\pgfqpoint{2.069061in}{1.763195in}}{\pgfqpoint{2.069061in}{1.771431in}}%
\pgfpathcurveto{\pgfqpoint{2.069061in}{1.779668in}}{\pgfqpoint{2.065789in}{1.787568in}}{\pgfqpoint{2.059965in}{1.793392in}}%
\pgfpathcurveto{\pgfqpoint{2.054141in}{1.799216in}}{\pgfqpoint{2.046241in}{1.802488in}}{\pgfqpoint{2.038005in}{1.802488in}}%
\pgfpathcurveto{\pgfqpoint{2.029769in}{1.802488in}}{\pgfqpoint{2.021869in}{1.799216in}}{\pgfqpoint{2.016045in}{1.793392in}}%
\pgfpathcurveto{\pgfqpoint{2.010221in}{1.787568in}}{\pgfqpoint{2.006948in}{1.779668in}}{\pgfqpoint{2.006948in}{1.771431in}}%
\pgfpathcurveto{\pgfqpoint{2.006948in}{1.763195in}}{\pgfqpoint{2.010221in}{1.755295in}}{\pgfqpoint{2.016045in}{1.749471in}}%
\pgfpathcurveto{\pgfqpoint{2.021869in}{1.743647in}}{\pgfqpoint{2.029769in}{1.740375in}}{\pgfqpoint{2.038005in}{1.740375in}}%
\pgfpathclose%
\pgfusepath{stroke,fill}%
\end{pgfscope}%
\begin{pgfscope}%
\pgfpathrectangle{\pgfqpoint{0.100000in}{0.212622in}}{\pgfqpoint{3.696000in}{3.696000in}}%
\pgfusepath{clip}%
\pgfsetbuttcap%
\pgfsetroundjoin%
\definecolor{currentfill}{rgb}{0.121569,0.466667,0.705882}%
\pgfsetfillcolor{currentfill}%
\pgfsetfillopacity{0.967466}%
\pgfsetlinewidth{1.003750pt}%
\definecolor{currentstroke}{rgb}{0.121569,0.466667,0.705882}%
\pgfsetstrokecolor{currentstroke}%
\pgfsetstrokeopacity{0.967466}%
\pgfsetdash{}{0pt}%
\pgfpathmoveto{\pgfqpoint{2.494463in}{1.601052in}}%
\pgfpathcurveto{\pgfqpoint{2.502700in}{1.601052in}}{\pgfqpoint{2.510600in}{1.604324in}}{\pgfqpoint{2.516424in}{1.610148in}}%
\pgfpathcurveto{\pgfqpoint{2.522248in}{1.615972in}}{\pgfqpoint{2.525520in}{1.623872in}}{\pgfqpoint{2.525520in}{1.632108in}}%
\pgfpathcurveto{\pgfqpoint{2.525520in}{1.640345in}}{\pgfqpoint{2.522248in}{1.648245in}}{\pgfqpoint{2.516424in}{1.654069in}}%
\pgfpathcurveto{\pgfqpoint{2.510600in}{1.659893in}}{\pgfqpoint{2.502700in}{1.663165in}}{\pgfqpoint{2.494463in}{1.663165in}}%
\pgfpathcurveto{\pgfqpoint{2.486227in}{1.663165in}}{\pgfqpoint{2.478327in}{1.659893in}}{\pgfqpoint{2.472503in}{1.654069in}}%
\pgfpathcurveto{\pgfqpoint{2.466679in}{1.648245in}}{\pgfqpoint{2.463407in}{1.640345in}}{\pgfqpoint{2.463407in}{1.632108in}}%
\pgfpathcurveto{\pgfqpoint{2.463407in}{1.623872in}}{\pgfqpoint{2.466679in}{1.615972in}}{\pgfqpoint{2.472503in}{1.610148in}}%
\pgfpathcurveto{\pgfqpoint{2.478327in}{1.604324in}}{\pgfqpoint{2.486227in}{1.601052in}}{\pgfqpoint{2.494463in}{1.601052in}}%
\pgfpathclose%
\pgfusepath{stroke,fill}%
\end{pgfscope}%
\begin{pgfscope}%
\pgfpathrectangle{\pgfqpoint{0.100000in}{0.212622in}}{\pgfqpoint{3.696000in}{3.696000in}}%
\pgfusepath{clip}%
\pgfsetbuttcap%
\pgfsetroundjoin%
\definecolor{currentfill}{rgb}{0.121569,0.466667,0.705882}%
\pgfsetfillcolor{currentfill}%
\pgfsetfillopacity{0.967488}%
\pgfsetlinewidth{1.003750pt}%
\definecolor{currentstroke}{rgb}{0.121569,0.466667,0.705882}%
\pgfsetstrokecolor{currentstroke}%
\pgfsetstrokeopacity{0.967488}%
\pgfsetdash{}{0pt}%
\pgfpathmoveto{\pgfqpoint{2.047872in}{1.731890in}}%
\pgfpathcurveto{\pgfqpoint{2.056108in}{1.731890in}}{\pgfqpoint{2.064008in}{1.735162in}}{\pgfqpoint{2.069832in}{1.740986in}}%
\pgfpathcurveto{\pgfqpoint{2.075656in}{1.746810in}}{\pgfqpoint{2.078928in}{1.754710in}}{\pgfqpoint{2.078928in}{1.762946in}}%
\pgfpathcurveto{\pgfqpoint{2.078928in}{1.771182in}}{\pgfqpoint{2.075656in}{1.779083in}}{\pgfqpoint{2.069832in}{1.784906in}}%
\pgfpathcurveto{\pgfqpoint{2.064008in}{1.790730in}}{\pgfqpoint{2.056108in}{1.794003in}}{\pgfqpoint{2.047872in}{1.794003in}}%
\pgfpathcurveto{\pgfqpoint{2.039636in}{1.794003in}}{\pgfqpoint{2.031735in}{1.790730in}}{\pgfqpoint{2.025912in}{1.784906in}}%
\pgfpathcurveto{\pgfqpoint{2.020088in}{1.779083in}}{\pgfqpoint{2.016815in}{1.771182in}}{\pgfqpoint{2.016815in}{1.762946in}}%
\pgfpathcurveto{\pgfqpoint{2.016815in}{1.754710in}}{\pgfqpoint{2.020088in}{1.746810in}}{\pgfqpoint{2.025912in}{1.740986in}}%
\pgfpathcurveto{\pgfqpoint{2.031735in}{1.735162in}}{\pgfqpoint{2.039636in}{1.731890in}}{\pgfqpoint{2.047872in}{1.731890in}}%
\pgfpathclose%
\pgfusepath{stroke,fill}%
\end{pgfscope}%
\begin{pgfscope}%
\pgfpathrectangle{\pgfqpoint{0.100000in}{0.212622in}}{\pgfqpoint{3.696000in}{3.696000in}}%
\pgfusepath{clip}%
\pgfsetbuttcap%
\pgfsetroundjoin%
\definecolor{currentfill}{rgb}{0.121569,0.466667,0.705882}%
\pgfsetfillcolor{currentfill}%
\pgfsetfillopacity{0.968500}%
\pgfsetlinewidth{1.003750pt}%
\definecolor{currentstroke}{rgb}{0.121569,0.466667,0.705882}%
\pgfsetstrokecolor{currentstroke}%
\pgfsetstrokeopacity{0.968500}%
\pgfsetdash{}{0pt}%
\pgfpathmoveto{\pgfqpoint{2.056625in}{1.730120in}}%
\pgfpathcurveto{\pgfqpoint{2.064862in}{1.730120in}}{\pgfqpoint{2.072762in}{1.733392in}}{\pgfqpoint{2.078586in}{1.739216in}}%
\pgfpathcurveto{\pgfqpoint{2.084409in}{1.745040in}}{\pgfqpoint{2.087682in}{1.752940in}}{\pgfqpoint{2.087682in}{1.761177in}}%
\pgfpathcurveto{\pgfqpoint{2.087682in}{1.769413in}}{\pgfqpoint{2.084409in}{1.777313in}}{\pgfqpoint{2.078586in}{1.783137in}}%
\pgfpathcurveto{\pgfqpoint{2.072762in}{1.788961in}}{\pgfqpoint{2.064862in}{1.792233in}}{\pgfqpoint{2.056625in}{1.792233in}}%
\pgfpathcurveto{\pgfqpoint{2.048389in}{1.792233in}}{\pgfqpoint{2.040489in}{1.788961in}}{\pgfqpoint{2.034665in}{1.783137in}}%
\pgfpathcurveto{\pgfqpoint{2.028841in}{1.777313in}}{\pgfqpoint{2.025569in}{1.769413in}}{\pgfqpoint{2.025569in}{1.761177in}}%
\pgfpathcurveto{\pgfqpoint{2.025569in}{1.752940in}}{\pgfqpoint{2.028841in}{1.745040in}}{\pgfqpoint{2.034665in}{1.739216in}}%
\pgfpathcurveto{\pgfqpoint{2.040489in}{1.733392in}}{\pgfqpoint{2.048389in}{1.730120in}}{\pgfqpoint{2.056625in}{1.730120in}}%
\pgfpathclose%
\pgfusepath{stroke,fill}%
\end{pgfscope}%
\begin{pgfscope}%
\pgfpathrectangle{\pgfqpoint{0.100000in}{0.212622in}}{\pgfqpoint{3.696000in}{3.696000in}}%
\pgfusepath{clip}%
\pgfsetbuttcap%
\pgfsetroundjoin%
\definecolor{currentfill}{rgb}{0.121569,0.466667,0.705882}%
\pgfsetfillcolor{currentfill}%
\pgfsetfillopacity{0.969178}%
\pgfsetlinewidth{1.003750pt}%
\definecolor{currentstroke}{rgb}{0.121569,0.466667,0.705882}%
\pgfsetstrokecolor{currentstroke}%
\pgfsetstrokeopacity{0.969178}%
\pgfsetdash{}{0pt}%
\pgfpathmoveto{\pgfqpoint{2.064603in}{1.725561in}}%
\pgfpathcurveto{\pgfqpoint{2.072839in}{1.725561in}}{\pgfqpoint{2.080739in}{1.728833in}}{\pgfqpoint{2.086563in}{1.734657in}}%
\pgfpathcurveto{\pgfqpoint{2.092387in}{1.740481in}}{\pgfqpoint{2.095659in}{1.748381in}}{\pgfqpoint{2.095659in}{1.756617in}}%
\pgfpathcurveto{\pgfqpoint{2.095659in}{1.764854in}}{\pgfqpoint{2.092387in}{1.772754in}}{\pgfqpoint{2.086563in}{1.778578in}}%
\pgfpathcurveto{\pgfqpoint{2.080739in}{1.784402in}}{\pgfqpoint{2.072839in}{1.787674in}}{\pgfqpoint{2.064603in}{1.787674in}}%
\pgfpathcurveto{\pgfqpoint{2.056367in}{1.787674in}}{\pgfqpoint{2.048466in}{1.784402in}}{\pgfqpoint{2.042643in}{1.778578in}}%
\pgfpathcurveto{\pgfqpoint{2.036819in}{1.772754in}}{\pgfqpoint{2.033546in}{1.764854in}}{\pgfqpoint{2.033546in}{1.756617in}}%
\pgfpathcurveto{\pgfqpoint{2.033546in}{1.748381in}}{\pgfqpoint{2.036819in}{1.740481in}}{\pgfqpoint{2.042643in}{1.734657in}}%
\pgfpathcurveto{\pgfqpoint{2.048466in}{1.728833in}}{\pgfqpoint{2.056367in}{1.725561in}}{\pgfqpoint{2.064603in}{1.725561in}}%
\pgfpathclose%
\pgfusepath{stroke,fill}%
\end{pgfscope}%
\begin{pgfscope}%
\pgfpathrectangle{\pgfqpoint{0.100000in}{0.212622in}}{\pgfqpoint{3.696000in}{3.696000in}}%
\pgfusepath{clip}%
\pgfsetbuttcap%
\pgfsetroundjoin%
\definecolor{currentfill}{rgb}{0.121569,0.466667,0.705882}%
\pgfsetfillcolor{currentfill}%
\pgfsetfillopacity{0.969283}%
\pgfsetlinewidth{1.003750pt}%
\definecolor{currentstroke}{rgb}{0.121569,0.466667,0.705882}%
\pgfsetstrokecolor{currentstroke}%
\pgfsetstrokeopacity{0.969283}%
\pgfsetdash{}{0pt}%
\pgfpathmoveto{\pgfqpoint{2.493060in}{1.597961in}}%
\pgfpathcurveto{\pgfqpoint{2.501296in}{1.597961in}}{\pgfqpoint{2.509196in}{1.601233in}}{\pgfqpoint{2.515020in}{1.607057in}}%
\pgfpathcurveto{\pgfqpoint{2.520844in}{1.612881in}}{\pgfqpoint{2.524116in}{1.620781in}}{\pgfqpoint{2.524116in}{1.629017in}}%
\pgfpathcurveto{\pgfqpoint{2.524116in}{1.637254in}}{\pgfqpoint{2.520844in}{1.645154in}}{\pgfqpoint{2.515020in}{1.650978in}}%
\pgfpathcurveto{\pgfqpoint{2.509196in}{1.656801in}}{\pgfqpoint{2.501296in}{1.660074in}}{\pgfqpoint{2.493060in}{1.660074in}}%
\pgfpathcurveto{\pgfqpoint{2.484824in}{1.660074in}}{\pgfqpoint{2.476924in}{1.656801in}}{\pgfqpoint{2.471100in}{1.650978in}}%
\pgfpathcurveto{\pgfqpoint{2.465276in}{1.645154in}}{\pgfqpoint{2.462003in}{1.637254in}}{\pgfqpoint{2.462003in}{1.629017in}}%
\pgfpathcurveto{\pgfqpoint{2.462003in}{1.620781in}}{\pgfqpoint{2.465276in}{1.612881in}}{\pgfqpoint{2.471100in}{1.607057in}}%
\pgfpathcurveto{\pgfqpoint{2.476924in}{1.601233in}}{\pgfqpoint{2.484824in}{1.597961in}}{\pgfqpoint{2.493060in}{1.597961in}}%
\pgfpathclose%
\pgfusepath{stroke,fill}%
\end{pgfscope}%
\begin{pgfscope}%
\pgfpathrectangle{\pgfqpoint{0.100000in}{0.212622in}}{\pgfqpoint{3.696000in}{3.696000in}}%
\pgfusepath{clip}%
\pgfsetbuttcap%
\pgfsetroundjoin%
\definecolor{currentfill}{rgb}{0.121569,0.466667,0.705882}%
\pgfsetfillcolor{currentfill}%
\pgfsetfillopacity{0.969925}%
\pgfsetlinewidth{1.003750pt}%
\definecolor{currentstroke}{rgb}{0.121569,0.466667,0.705882}%
\pgfsetstrokecolor{currentstroke}%
\pgfsetstrokeopacity{0.969925}%
\pgfsetdash{}{0pt}%
\pgfpathmoveto{\pgfqpoint{2.079202in}{1.714430in}}%
\pgfpathcurveto{\pgfqpoint{2.087439in}{1.714430in}}{\pgfqpoint{2.095339in}{1.717702in}}{\pgfqpoint{2.101163in}{1.723526in}}%
\pgfpathcurveto{\pgfqpoint{2.106986in}{1.729350in}}{\pgfqpoint{2.110259in}{1.737250in}}{\pgfqpoint{2.110259in}{1.745486in}}%
\pgfpathcurveto{\pgfqpoint{2.110259in}{1.753722in}}{\pgfqpoint{2.106986in}{1.761623in}}{\pgfqpoint{2.101163in}{1.767446in}}%
\pgfpathcurveto{\pgfqpoint{2.095339in}{1.773270in}}{\pgfqpoint{2.087439in}{1.776543in}}{\pgfqpoint{2.079202in}{1.776543in}}%
\pgfpathcurveto{\pgfqpoint{2.070966in}{1.776543in}}{\pgfqpoint{2.063066in}{1.773270in}}{\pgfqpoint{2.057242in}{1.767446in}}%
\pgfpathcurveto{\pgfqpoint{2.051418in}{1.761623in}}{\pgfqpoint{2.048146in}{1.753722in}}{\pgfqpoint{2.048146in}{1.745486in}}%
\pgfpathcurveto{\pgfqpoint{2.048146in}{1.737250in}}{\pgfqpoint{2.051418in}{1.729350in}}{\pgfqpoint{2.057242in}{1.723526in}}%
\pgfpathcurveto{\pgfqpoint{2.063066in}{1.717702in}}{\pgfqpoint{2.070966in}{1.714430in}}{\pgfqpoint{2.079202in}{1.714430in}}%
\pgfpathclose%
\pgfusepath{stroke,fill}%
\end{pgfscope}%
\begin{pgfscope}%
\pgfpathrectangle{\pgfqpoint{0.100000in}{0.212622in}}{\pgfqpoint{3.696000in}{3.696000in}}%
\pgfusepath{clip}%
\pgfsetbuttcap%
\pgfsetroundjoin%
\definecolor{currentfill}{rgb}{0.121569,0.466667,0.705882}%
\pgfsetfillcolor{currentfill}%
\pgfsetfillopacity{0.971041}%
\pgfsetlinewidth{1.003750pt}%
\definecolor{currentstroke}{rgb}{0.121569,0.466667,0.705882}%
\pgfsetstrokecolor{currentstroke}%
\pgfsetstrokeopacity{0.971041}%
\pgfsetdash{}{0pt}%
\pgfpathmoveto{\pgfqpoint{2.091231in}{1.707704in}}%
\pgfpathcurveto{\pgfqpoint{2.099467in}{1.707704in}}{\pgfqpoint{2.107367in}{1.710976in}}{\pgfqpoint{2.113191in}{1.716800in}}%
\pgfpathcurveto{\pgfqpoint{2.119015in}{1.722624in}}{\pgfqpoint{2.122287in}{1.730524in}}{\pgfqpoint{2.122287in}{1.738760in}}%
\pgfpathcurveto{\pgfqpoint{2.122287in}{1.746996in}}{\pgfqpoint{2.119015in}{1.754897in}}{\pgfqpoint{2.113191in}{1.760720in}}%
\pgfpathcurveto{\pgfqpoint{2.107367in}{1.766544in}}{\pgfqpoint{2.099467in}{1.769817in}}{\pgfqpoint{2.091231in}{1.769817in}}%
\pgfpathcurveto{\pgfqpoint{2.082995in}{1.769817in}}{\pgfqpoint{2.075095in}{1.766544in}}{\pgfqpoint{2.069271in}{1.760720in}}%
\pgfpathcurveto{\pgfqpoint{2.063447in}{1.754897in}}{\pgfqpoint{2.060174in}{1.746996in}}{\pgfqpoint{2.060174in}{1.738760in}}%
\pgfpathcurveto{\pgfqpoint{2.060174in}{1.730524in}}{\pgfqpoint{2.063447in}{1.722624in}}{\pgfqpoint{2.069271in}{1.716800in}}%
\pgfpathcurveto{\pgfqpoint{2.075095in}{1.710976in}}{\pgfqpoint{2.082995in}{1.707704in}}{\pgfqpoint{2.091231in}{1.707704in}}%
\pgfpathclose%
\pgfusepath{stroke,fill}%
\end{pgfscope}%
\begin{pgfscope}%
\pgfpathrectangle{\pgfqpoint{0.100000in}{0.212622in}}{\pgfqpoint{3.696000in}{3.696000in}}%
\pgfusepath{clip}%
\pgfsetbuttcap%
\pgfsetroundjoin%
\definecolor{currentfill}{rgb}{0.121569,0.466667,0.705882}%
\pgfsetfillcolor{currentfill}%
\pgfsetfillopacity{0.971535}%
\pgfsetlinewidth{1.003750pt}%
\definecolor{currentstroke}{rgb}{0.121569,0.466667,0.705882}%
\pgfsetstrokecolor{currentstroke}%
\pgfsetstrokeopacity{0.971535}%
\pgfsetdash{}{0pt}%
\pgfpathmoveto{\pgfqpoint{2.486154in}{1.596693in}}%
\pgfpathcurveto{\pgfqpoint{2.494390in}{1.596693in}}{\pgfqpoint{2.502290in}{1.599965in}}{\pgfqpoint{2.508114in}{1.605789in}}%
\pgfpathcurveto{\pgfqpoint{2.513938in}{1.611613in}}{\pgfqpoint{2.517210in}{1.619513in}}{\pgfqpoint{2.517210in}{1.627750in}}%
\pgfpathcurveto{\pgfqpoint{2.517210in}{1.635986in}}{\pgfqpoint{2.513938in}{1.643886in}}{\pgfqpoint{2.508114in}{1.649710in}}%
\pgfpathcurveto{\pgfqpoint{2.502290in}{1.655534in}}{\pgfqpoint{2.494390in}{1.658806in}}{\pgfqpoint{2.486154in}{1.658806in}}%
\pgfpathcurveto{\pgfqpoint{2.477918in}{1.658806in}}{\pgfqpoint{2.470017in}{1.655534in}}{\pgfqpoint{2.464194in}{1.649710in}}%
\pgfpathcurveto{\pgfqpoint{2.458370in}{1.643886in}}{\pgfqpoint{2.455097in}{1.635986in}}{\pgfqpoint{2.455097in}{1.627750in}}%
\pgfpathcurveto{\pgfqpoint{2.455097in}{1.619513in}}{\pgfqpoint{2.458370in}{1.611613in}}{\pgfqpoint{2.464194in}{1.605789in}}%
\pgfpathcurveto{\pgfqpoint{2.470017in}{1.599965in}}{\pgfqpoint{2.477918in}{1.596693in}}{\pgfqpoint{2.486154in}{1.596693in}}%
\pgfpathclose%
\pgfusepath{stroke,fill}%
\end{pgfscope}%
\begin{pgfscope}%
\pgfpathrectangle{\pgfqpoint{0.100000in}{0.212622in}}{\pgfqpoint{3.696000in}{3.696000in}}%
\pgfusepath{clip}%
\pgfsetbuttcap%
\pgfsetroundjoin%
\definecolor{currentfill}{rgb}{0.121569,0.466667,0.705882}%
\pgfsetfillcolor{currentfill}%
\pgfsetfillopacity{0.971755}%
\pgfsetlinewidth{1.003750pt}%
\definecolor{currentstroke}{rgb}{0.121569,0.466667,0.705882}%
\pgfsetstrokecolor{currentstroke}%
\pgfsetstrokeopacity{0.971755}%
\pgfsetdash{}{0pt}%
\pgfpathmoveto{\pgfqpoint{2.101878in}{1.701503in}}%
\pgfpathcurveto{\pgfqpoint{2.110114in}{1.701503in}}{\pgfqpoint{2.118014in}{1.704775in}}{\pgfqpoint{2.123838in}{1.710599in}}%
\pgfpathcurveto{\pgfqpoint{2.129662in}{1.716423in}}{\pgfqpoint{2.132934in}{1.724323in}}{\pgfqpoint{2.132934in}{1.732559in}}%
\pgfpathcurveto{\pgfqpoint{2.132934in}{1.740795in}}{\pgfqpoint{2.129662in}{1.748695in}}{\pgfqpoint{2.123838in}{1.754519in}}%
\pgfpathcurveto{\pgfqpoint{2.118014in}{1.760343in}}{\pgfqpoint{2.110114in}{1.763616in}}{\pgfqpoint{2.101878in}{1.763616in}}%
\pgfpathcurveto{\pgfqpoint{2.093641in}{1.763616in}}{\pgfqpoint{2.085741in}{1.760343in}}{\pgfqpoint{2.079917in}{1.754519in}}%
\pgfpathcurveto{\pgfqpoint{2.074093in}{1.748695in}}{\pgfqpoint{2.070821in}{1.740795in}}{\pgfqpoint{2.070821in}{1.732559in}}%
\pgfpathcurveto{\pgfqpoint{2.070821in}{1.724323in}}{\pgfqpoint{2.074093in}{1.716423in}}{\pgfqpoint{2.079917in}{1.710599in}}%
\pgfpathcurveto{\pgfqpoint{2.085741in}{1.704775in}}{\pgfqpoint{2.093641in}{1.701503in}}{\pgfqpoint{2.101878in}{1.701503in}}%
\pgfpathclose%
\pgfusepath{stroke,fill}%
\end{pgfscope}%
\begin{pgfscope}%
\pgfpathrectangle{\pgfqpoint{0.100000in}{0.212622in}}{\pgfqpoint{3.696000in}{3.696000in}}%
\pgfusepath{clip}%
\pgfsetbuttcap%
\pgfsetroundjoin%
\definecolor{currentfill}{rgb}{0.121569,0.466667,0.705882}%
\pgfsetfillcolor{currentfill}%
\pgfsetfillopacity{0.971916}%
\pgfsetlinewidth{1.003750pt}%
\definecolor{currentstroke}{rgb}{0.121569,0.466667,0.705882}%
\pgfsetstrokecolor{currentstroke}%
\pgfsetstrokeopacity{0.971916}%
\pgfsetdash{}{0pt}%
\pgfpathmoveto{\pgfqpoint{2.112129in}{1.692056in}}%
\pgfpathcurveto{\pgfqpoint{2.120366in}{1.692056in}}{\pgfqpoint{2.128266in}{1.695328in}}{\pgfqpoint{2.134090in}{1.701152in}}%
\pgfpathcurveto{\pgfqpoint{2.139913in}{1.706976in}}{\pgfqpoint{2.143186in}{1.714876in}}{\pgfqpoint{2.143186in}{1.723112in}}%
\pgfpathcurveto{\pgfqpoint{2.143186in}{1.731348in}}{\pgfqpoint{2.139913in}{1.739249in}}{\pgfqpoint{2.134090in}{1.745072in}}%
\pgfpathcurveto{\pgfqpoint{2.128266in}{1.750896in}}{\pgfqpoint{2.120366in}{1.754169in}}{\pgfqpoint{2.112129in}{1.754169in}}%
\pgfpathcurveto{\pgfqpoint{2.103893in}{1.754169in}}{\pgfqpoint{2.095993in}{1.750896in}}{\pgfqpoint{2.090169in}{1.745072in}}%
\pgfpathcurveto{\pgfqpoint{2.084345in}{1.739249in}}{\pgfqpoint{2.081073in}{1.731348in}}{\pgfqpoint{2.081073in}{1.723112in}}%
\pgfpathcurveto{\pgfqpoint{2.081073in}{1.714876in}}{\pgfqpoint{2.084345in}{1.706976in}}{\pgfqpoint{2.090169in}{1.701152in}}%
\pgfpathcurveto{\pgfqpoint{2.095993in}{1.695328in}}{\pgfqpoint{2.103893in}{1.692056in}}{\pgfqpoint{2.112129in}{1.692056in}}%
\pgfpathclose%
\pgfusepath{stroke,fill}%
\end{pgfscope}%
\begin{pgfscope}%
\pgfpathrectangle{\pgfqpoint{0.100000in}{0.212622in}}{\pgfqpoint{3.696000in}{3.696000in}}%
\pgfusepath{clip}%
\pgfsetbuttcap%
\pgfsetroundjoin%
\definecolor{currentfill}{rgb}{0.121569,0.466667,0.705882}%
\pgfsetfillcolor{currentfill}%
\pgfsetfillopacity{0.972278}%
\pgfsetlinewidth{1.003750pt}%
\definecolor{currentstroke}{rgb}{0.121569,0.466667,0.705882}%
\pgfsetstrokecolor{currentstroke}%
\pgfsetstrokeopacity{0.972278}%
\pgfsetdash{}{0pt}%
\pgfpathmoveto{\pgfqpoint{2.120611in}{1.688187in}}%
\pgfpathcurveto{\pgfqpoint{2.128847in}{1.688187in}}{\pgfqpoint{2.136747in}{1.691459in}}{\pgfqpoint{2.142571in}{1.697283in}}%
\pgfpathcurveto{\pgfqpoint{2.148395in}{1.703107in}}{\pgfqpoint{2.151667in}{1.711007in}}{\pgfqpoint{2.151667in}{1.719243in}}%
\pgfpathcurveto{\pgfqpoint{2.151667in}{1.727480in}}{\pgfqpoint{2.148395in}{1.735380in}}{\pgfqpoint{2.142571in}{1.741204in}}%
\pgfpathcurveto{\pgfqpoint{2.136747in}{1.747028in}}{\pgfqpoint{2.128847in}{1.750300in}}{\pgfqpoint{2.120611in}{1.750300in}}%
\pgfpathcurveto{\pgfqpoint{2.112375in}{1.750300in}}{\pgfqpoint{2.104474in}{1.747028in}}{\pgfqpoint{2.098651in}{1.741204in}}%
\pgfpathcurveto{\pgfqpoint{2.092827in}{1.735380in}}{\pgfqpoint{2.089554in}{1.727480in}}{\pgfqpoint{2.089554in}{1.719243in}}%
\pgfpathcurveto{\pgfqpoint{2.089554in}{1.711007in}}{\pgfqpoint{2.092827in}{1.703107in}}{\pgfqpoint{2.098651in}{1.697283in}}%
\pgfpathcurveto{\pgfqpoint{2.104474in}{1.691459in}}{\pgfqpoint{2.112375in}{1.688187in}}{\pgfqpoint{2.120611in}{1.688187in}}%
\pgfpathclose%
\pgfusepath{stroke,fill}%
\end{pgfscope}%
\begin{pgfscope}%
\pgfpathrectangle{\pgfqpoint{0.100000in}{0.212622in}}{\pgfqpoint{3.696000in}{3.696000in}}%
\pgfusepath{clip}%
\pgfsetbuttcap%
\pgfsetroundjoin%
\definecolor{currentfill}{rgb}{0.121569,0.466667,0.705882}%
\pgfsetfillcolor{currentfill}%
\pgfsetfillopacity{0.972465}%
\pgfsetlinewidth{1.003750pt}%
\definecolor{currentstroke}{rgb}{0.121569,0.466667,0.705882}%
\pgfsetstrokecolor{currentstroke}%
\pgfsetstrokeopacity{0.972465}%
\pgfsetdash{}{0pt}%
\pgfpathmoveto{\pgfqpoint{2.126508in}{1.683579in}}%
\pgfpathcurveto{\pgfqpoint{2.134745in}{1.683579in}}{\pgfqpoint{2.142645in}{1.686851in}}{\pgfqpoint{2.148469in}{1.692675in}}%
\pgfpathcurveto{\pgfqpoint{2.154293in}{1.698499in}}{\pgfqpoint{2.157565in}{1.706399in}}{\pgfqpoint{2.157565in}{1.714635in}}%
\pgfpathcurveto{\pgfqpoint{2.157565in}{1.722872in}}{\pgfqpoint{2.154293in}{1.730772in}}{\pgfqpoint{2.148469in}{1.736596in}}%
\pgfpathcurveto{\pgfqpoint{2.142645in}{1.742420in}}{\pgfqpoint{2.134745in}{1.745692in}}{\pgfqpoint{2.126508in}{1.745692in}}%
\pgfpathcurveto{\pgfqpoint{2.118272in}{1.745692in}}{\pgfqpoint{2.110372in}{1.742420in}}{\pgfqpoint{2.104548in}{1.736596in}}%
\pgfpathcurveto{\pgfqpoint{2.098724in}{1.730772in}}{\pgfqpoint{2.095452in}{1.722872in}}{\pgfqpoint{2.095452in}{1.714635in}}%
\pgfpathcurveto{\pgfqpoint{2.095452in}{1.706399in}}{\pgfqpoint{2.098724in}{1.698499in}}{\pgfqpoint{2.104548in}{1.692675in}}%
\pgfpathcurveto{\pgfqpoint{2.110372in}{1.686851in}}{\pgfqpoint{2.118272in}{1.683579in}}{\pgfqpoint{2.126508in}{1.683579in}}%
\pgfpathclose%
\pgfusepath{stroke,fill}%
\end{pgfscope}%
\begin{pgfscope}%
\pgfpathrectangle{\pgfqpoint{0.100000in}{0.212622in}}{\pgfqpoint{3.696000in}{3.696000in}}%
\pgfusepath{clip}%
\pgfsetbuttcap%
\pgfsetroundjoin%
\definecolor{currentfill}{rgb}{0.121569,0.466667,0.705882}%
\pgfsetfillcolor{currentfill}%
\pgfsetfillopacity{0.972885}%
\pgfsetlinewidth{1.003750pt}%
\definecolor{currentstroke}{rgb}{0.121569,0.466667,0.705882}%
\pgfsetstrokecolor{currentstroke}%
\pgfsetstrokeopacity{0.972885}%
\pgfsetdash{}{0pt}%
\pgfpathmoveto{\pgfqpoint{2.483379in}{1.595242in}}%
\pgfpathcurveto{\pgfqpoint{2.491615in}{1.595242in}}{\pgfqpoint{2.499515in}{1.598514in}}{\pgfqpoint{2.505339in}{1.604338in}}%
\pgfpathcurveto{\pgfqpoint{2.511163in}{1.610162in}}{\pgfqpoint{2.514435in}{1.618062in}}{\pgfqpoint{2.514435in}{1.626298in}}%
\pgfpathcurveto{\pgfqpoint{2.514435in}{1.634535in}}{\pgfqpoint{2.511163in}{1.642435in}}{\pgfqpoint{2.505339in}{1.648259in}}%
\pgfpathcurveto{\pgfqpoint{2.499515in}{1.654083in}}{\pgfqpoint{2.491615in}{1.657355in}}{\pgfqpoint{2.483379in}{1.657355in}}%
\pgfpathcurveto{\pgfqpoint{2.475142in}{1.657355in}}{\pgfqpoint{2.467242in}{1.654083in}}{\pgfqpoint{2.461418in}{1.648259in}}%
\pgfpathcurveto{\pgfqpoint{2.455594in}{1.642435in}}{\pgfqpoint{2.452322in}{1.634535in}}{\pgfqpoint{2.452322in}{1.626298in}}%
\pgfpathcurveto{\pgfqpoint{2.452322in}{1.618062in}}{\pgfqpoint{2.455594in}{1.610162in}}{\pgfqpoint{2.461418in}{1.604338in}}%
\pgfpathcurveto{\pgfqpoint{2.467242in}{1.598514in}}{\pgfqpoint{2.475142in}{1.595242in}}{\pgfqpoint{2.483379in}{1.595242in}}%
\pgfpathclose%
\pgfusepath{stroke,fill}%
\end{pgfscope}%
\begin{pgfscope}%
\pgfpathrectangle{\pgfqpoint{0.100000in}{0.212622in}}{\pgfqpoint{3.696000in}{3.696000in}}%
\pgfusepath{clip}%
\pgfsetbuttcap%
\pgfsetroundjoin%
\definecolor{currentfill}{rgb}{0.121569,0.466667,0.705882}%
\pgfsetfillcolor{currentfill}%
\pgfsetfillopacity{0.973403}%
\pgfsetlinewidth{1.003750pt}%
\definecolor{currentstroke}{rgb}{0.121569,0.466667,0.705882}%
\pgfsetstrokecolor{currentstroke}%
\pgfsetstrokeopacity{0.973403}%
\pgfsetdash{}{0pt}%
\pgfpathmoveto{\pgfqpoint{2.137286in}{1.679264in}}%
\pgfpathcurveto{\pgfqpoint{2.145522in}{1.679264in}}{\pgfqpoint{2.153422in}{1.682536in}}{\pgfqpoint{2.159246in}{1.688360in}}%
\pgfpathcurveto{\pgfqpoint{2.165070in}{1.694184in}}{\pgfqpoint{2.168342in}{1.702084in}}{\pgfqpoint{2.168342in}{1.710320in}}%
\pgfpathcurveto{\pgfqpoint{2.168342in}{1.718557in}}{\pgfqpoint{2.165070in}{1.726457in}}{\pgfqpoint{2.159246in}{1.732281in}}%
\pgfpathcurveto{\pgfqpoint{2.153422in}{1.738105in}}{\pgfqpoint{2.145522in}{1.741377in}}{\pgfqpoint{2.137286in}{1.741377in}}%
\pgfpathcurveto{\pgfqpoint{2.129050in}{1.741377in}}{\pgfqpoint{2.121150in}{1.738105in}}{\pgfqpoint{2.115326in}{1.732281in}}%
\pgfpathcurveto{\pgfqpoint{2.109502in}{1.726457in}}{\pgfqpoint{2.106229in}{1.718557in}}{\pgfqpoint{2.106229in}{1.710320in}}%
\pgfpathcurveto{\pgfqpoint{2.106229in}{1.702084in}}{\pgfqpoint{2.109502in}{1.694184in}}{\pgfqpoint{2.115326in}{1.688360in}}%
\pgfpathcurveto{\pgfqpoint{2.121150in}{1.682536in}}{\pgfqpoint{2.129050in}{1.679264in}}{\pgfqpoint{2.137286in}{1.679264in}}%
\pgfpathclose%
\pgfusepath{stroke,fill}%
\end{pgfscope}%
\begin{pgfscope}%
\pgfpathrectangle{\pgfqpoint{0.100000in}{0.212622in}}{\pgfqpoint{3.696000in}{3.696000in}}%
\pgfusepath{clip}%
\pgfsetbuttcap%
\pgfsetroundjoin%
\definecolor{currentfill}{rgb}{0.121569,0.466667,0.705882}%
\pgfsetfillcolor{currentfill}%
\pgfsetfillopacity{0.973930}%
\pgfsetlinewidth{1.003750pt}%
\definecolor{currentstroke}{rgb}{0.121569,0.466667,0.705882}%
\pgfsetstrokecolor{currentstroke}%
\pgfsetstrokeopacity{0.973930}%
\pgfsetdash{}{0pt}%
\pgfpathmoveto{\pgfqpoint{2.146072in}{1.674529in}}%
\pgfpathcurveto{\pgfqpoint{2.154309in}{1.674529in}}{\pgfqpoint{2.162209in}{1.677801in}}{\pgfqpoint{2.168033in}{1.683625in}}%
\pgfpathcurveto{\pgfqpoint{2.173856in}{1.689449in}}{\pgfqpoint{2.177129in}{1.697349in}}{\pgfqpoint{2.177129in}{1.705585in}}%
\pgfpathcurveto{\pgfqpoint{2.177129in}{1.713822in}}{\pgfqpoint{2.173856in}{1.721722in}}{\pgfqpoint{2.168033in}{1.727546in}}%
\pgfpathcurveto{\pgfqpoint{2.162209in}{1.733370in}}{\pgfqpoint{2.154309in}{1.736642in}}{\pgfqpoint{2.146072in}{1.736642in}}%
\pgfpathcurveto{\pgfqpoint{2.137836in}{1.736642in}}{\pgfqpoint{2.129936in}{1.733370in}}{\pgfqpoint{2.124112in}{1.727546in}}%
\pgfpathcurveto{\pgfqpoint{2.118288in}{1.721722in}}{\pgfqpoint{2.115016in}{1.713822in}}{\pgfqpoint{2.115016in}{1.705585in}}%
\pgfpathcurveto{\pgfqpoint{2.115016in}{1.697349in}}{\pgfqpoint{2.118288in}{1.689449in}}{\pgfqpoint{2.124112in}{1.683625in}}%
\pgfpathcurveto{\pgfqpoint{2.129936in}{1.677801in}}{\pgfqpoint{2.137836in}{1.674529in}}{\pgfqpoint{2.146072in}{1.674529in}}%
\pgfpathclose%
\pgfusepath{stroke,fill}%
\end{pgfscope}%
\begin{pgfscope}%
\pgfpathrectangle{\pgfqpoint{0.100000in}{0.212622in}}{\pgfqpoint{3.696000in}{3.696000in}}%
\pgfusepath{clip}%
\pgfsetbuttcap%
\pgfsetroundjoin%
\definecolor{currentfill}{rgb}{0.121569,0.466667,0.705882}%
\pgfsetfillcolor{currentfill}%
\pgfsetfillopacity{0.973945}%
\pgfsetlinewidth{1.003750pt}%
\definecolor{currentstroke}{rgb}{0.121569,0.466667,0.705882}%
\pgfsetstrokecolor{currentstroke}%
\pgfsetstrokeopacity{0.973945}%
\pgfsetdash{}{0pt}%
\pgfpathmoveto{\pgfqpoint{2.151301in}{1.668308in}}%
\pgfpathcurveto{\pgfqpoint{2.159537in}{1.668308in}}{\pgfqpoint{2.167437in}{1.671580in}}{\pgfqpoint{2.173261in}{1.677404in}}%
\pgfpathcurveto{\pgfqpoint{2.179085in}{1.683228in}}{\pgfqpoint{2.182357in}{1.691128in}}{\pgfqpoint{2.182357in}{1.699364in}}%
\pgfpathcurveto{\pgfqpoint{2.182357in}{1.707600in}}{\pgfqpoint{2.179085in}{1.715500in}}{\pgfqpoint{2.173261in}{1.721324in}}%
\pgfpathcurveto{\pgfqpoint{2.167437in}{1.727148in}}{\pgfqpoint{2.159537in}{1.730421in}}{\pgfqpoint{2.151301in}{1.730421in}}%
\pgfpathcurveto{\pgfqpoint{2.143064in}{1.730421in}}{\pgfqpoint{2.135164in}{1.727148in}}{\pgfqpoint{2.129340in}{1.721324in}}%
\pgfpathcurveto{\pgfqpoint{2.123516in}{1.715500in}}{\pgfqpoint{2.120244in}{1.707600in}}{\pgfqpoint{2.120244in}{1.699364in}}%
\pgfpathcurveto{\pgfqpoint{2.120244in}{1.691128in}}{\pgfqpoint{2.123516in}{1.683228in}}{\pgfqpoint{2.129340in}{1.677404in}}%
\pgfpathcurveto{\pgfqpoint{2.135164in}{1.671580in}}{\pgfqpoint{2.143064in}{1.668308in}}{\pgfqpoint{2.151301in}{1.668308in}}%
\pgfpathclose%
\pgfusepath{stroke,fill}%
\end{pgfscope}%
\begin{pgfscope}%
\pgfpathrectangle{\pgfqpoint{0.100000in}{0.212622in}}{\pgfqpoint{3.696000in}{3.696000in}}%
\pgfusepath{clip}%
\pgfsetbuttcap%
\pgfsetroundjoin%
\definecolor{currentfill}{rgb}{0.121569,0.466667,0.705882}%
\pgfsetfillcolor{currentfill}%
\pgfsetfillopacity{0.974379}%
\pgfsetlinewidth{1.003750pt}%
\definecolor{currentstroke}{rgb}{0.121569,0.466667,0.705882}%
\pgfsetstrokecolor{currentstroke}%
\pgfsetstrokeopacity{0.974379}%
\pgfsetdash{}{0pt}%
\pgfpathmoveto{\pgfqpoint{2.479511in}{1.592348in}}%
\pgfpathcurveto{\pgfqpoint{2.487748in}{1.592348in}}{\pgfqpoint{2.495648in}{1.595620in}}{\pgfqpoint{2.501472in}{1.601444in}}%
\pgfpathcurveto{\pgfqpoint{2.507296in}{1.607268in}}{\pgfqpoint{2.510568in}{1.615168in}}{\pgfqpoint{2.510568in}{1.623404in}}%
\pgfpathcurveto{\pgfqpoint{2.510568in}{1.631641in}}{\pgfqpoint{2.507296in}{1.639541in}}{\pgfqpoint{2.501472in}{1.645365in}}%
\pgfpathcurveto{\pgfqpoint{2.495648in}{1.651189in}}{\pgfqpoint{2.487748in}{1.654461in}}{\pgfqpoint{2.479511in}{1.654461in}}%
\pgfpathcurveto{\pgfqpoint{2.471275in}{1.654461in}}{\pgfqpoint{2.463375in}{1.651189in}}{\pgfqpoint{2.457551in}{1.645365in}}%
\pgfpathcurveto{\pgfqpoint{2.451727in}{1.639541in}}{\pgfqpoint{2.448455in}{1.631641in}}{\pgfqpoint{2.448455in}{1.623404in}}%
\pgfpathcurveto{\pgfqpoint{2.448455in}{1.615168in}}{\pgfqpoint{2.451727in}{1.607268in}}{\pgfqpoint{2.457551in}{1.601444in}}%
\pgfpathcurveto{\pgfqpoint{2.463375in}{1.595620in}}{\pgfqpoint{2.471275in}{1.592348in}}{\pgfqpoint{2.479511in}{1.592348in}}%
\pgfpathclose%
\pgfusepath{stroke,fill}%
\end{pgfscope}%
\begin{pgfscope}%
\pgfpathrectangle{\pgfqpoint{0.100000in}{0.212622in}}{\pgfqpoint{3.696000in}{3.696000in}}%
\pgfusepath{clip}%
\pgfsetbuttcap%
\pgfsetroundjoin%
\definecolor{currentfill}{rgb}{0.121569,0.466667,0.705882}%
\pgfsetfillcolor{currentfill}%
\pgfsetfillopacity{0.974442}%
\pgfsetlinewidth{1.003750pt}%
\definecolor{currentstroke}{rgb}{0.121569,0.466667,0.705882}%
\pgfsetstrokecolor{currentstroke}%
\pgfsetstrokeopacity{0.974442}%
\pgfsetdash{}{0pt}%
\pgfpathmoveto{\pgfqpoint{2.160547in}{1.659197in}}%
\pgfpathcurveto{\pgfqpoint{2.168783in}{1.659197in}}{\pgfqpoint{2.176683in}{1.662469in}}{\pgfqpoint{2.182507in}{1.668293in}}%
\pgfpathcurveto{\pgfqpoint{2.188331in}{1.674117in}}{\pgfqpoint{2.191604in}{1.682017in}}{\pgfqpoint{2.191604in}{1.690253in}}%
\pgfpathcurveto{\pgfqpoint{2.191604in}{1.698489in}}{\pgfqpoint{2.188331in}{1.706389in}}{\pgfqpoint{2.182507in}{1.712213in}}%
\pgfpathcurveto{\pgfqpoint{2.176683in}{1.718037in}}{\pgfqpoint{2.168783in}{1.721310in}}{\pgfqpoint{2.160547in}{1.721310in}}%
\pgfpathcurveto{\pgfqpoint{2.152311in}{1.721310in}}{\pgfqpoint{2.144411in}{1.718037in}}{\pgfqpoint{2.138587in}{1.712213in}}%
\pgfpathcurveto{\pgfqpoint{2.132763in}{1.706389in}}{\pgfqpoint{2.129491in}{1.698489in}}{\pgfqpoint{2.129491in}{1.690253in}}%
\pgfpathcurveto{\pgfqpoint{2.129491in}{1.682017in}}{\pgfqpoint{2.132763in}{1.674117in}}{\pgfqpoint{2.138587in}{1.668293in}}%
\pgfpathcurveto{\pgfqpoint{2.144411in}{1.662469in}}{\pgfqpoint{2.152311in}{1.659197in}}{\pgfqpoint{2.160547in}{1.659197in}}%
\pgfpathclose%
\pgfusepath{stroke,fill}%
\end{pgfscope}%
\begin{pgfscope}%
\pgfpathrectangle{\pgfqpoint{0.100000in}{0.212622in}}{\pgfqpoint{3.696000in}{3.696000in}}%
\pgfusepath{clip}%
\pgfsetbuttcap%
\pgfsetroundjoin%
\definecolor{currentfill}{rgb}{0.121569,0.466667,0.705882}%
\pgfsetfillcolor{currentfill}%
\pgfsetfillopacity{0.975172}%
\pgfsetlinewidth{1.003750pt}%
\definecolor{currentstroke}{rgb}{0.121569,0.466667,0.705882}%
\pgfsetstrokecolor{currentstroke}%
\pgfsetstrokeopacity{0.975172}%
\pgfsetdash{}{0pt}%
\pgfpathmoveto{\pgfqpoint{2.169089in}{1.657335in}}%
\pgfpathcurveto{\pgfqpoint{2.177326in}{1.657335in}}{\pgfqpoint{2.185226in}{1.660608in}}{\pgfqpoint{2.191050in}{1.666432in}}%
\pgfpathcurveto{\pgfqpoint{2.196874in}{1.672256in}}{\pgfqpoint{2.200146in}{1.680156in}}{\pgfqpoint{2.200146in}{1.688392in}}%
\pgfpathcurveto{\pgfqpoint{2.200146in}{1.696628in}}{\pgfqpoint{2.196874in}{1.704528in}}{\pgfqpoint{2.191050in}{1.710352in}}%
\pgfpathcurveto{\pgfqpoint{2.185226in}{1.716176in}}{\pgfqpoint{2.177326in}{1.719448in}}{\pgfqpoint{2.169089in}{1.719448in}}%
\pgfpathcurveto{\pgfqpoint{2.160853in}{1.719448in}}{\pgfqpoint{2.152953in}{1.716176in}}{\pgfqpoint{2.147129in}{1.710352in}}%
\pgfpathcurveto{\pgfqpoint{2.141305in}{1.704528in}}{\pgfqpoint{2.138033in}{1.696628in}}{\pgfqpoint{2.138033in}{1.688392in}}%
\pgfpathcurveto{\pgfqpoint{2.138033in}{1.680156in}}{\pgfqpoint{2.141305in}{1.672256in}}{\pgfqpoint{2.147129in}{1.666432in}}%
\pgfpathcurveto{\pgfqpoint{2.152953in}{1.660608in}}{\pgfqpoint{2.160853in}{1.657335in}}{\pgfqpoint{2.169089in}{1.657335in}}%
\pgfpathclose%
\pgfusepath{stroke,fill}%
\end{pgfscope}%
\begin{pgfscope}%
\pgfpathrectangle{\pgfqpoint{0.100000in}{0.212622in}}{\pgfqpoint{3.696000in}{3.696000in}}%
\pgfusepath{clip}%
\pgfsetbuttcap%
\pgfsetroundjoin%
\definecolor{currentfill}{rgb}{0.121569,0.466667,0.705882}%
\pgfsetfillcolor{currentfill}%
\pgfsetfillopacity{0.975358}%
\pgfsetlinewidth{1.003750pt}%
\definecolor{currentstroke}{rgb}{0.121569,0.466667,0.705882}%
\pgfsetstrokecolor{currentstroke}%
\pgfsetstrokeopacity{0.975358}%
\pgfsetdash{}{0pt}%
\pgfpathmoveto{\pgfqpoint{2.174121in}{1.652288in}}%
\pgfpathcurveto{\pgfqpoint{2.182357in}{1.652288in}}{\pgfqpoint{2.190257in}{1.655560in}}{\pgfqpoint{2.196081in}{1.661384in}}%
\pgfpathcurveto{\pgfqpoint{2.201905in}{1.667208in}}{\pgfqpoint{2.205178in}{1.675108in}}{\pgfqpoint{2.205178in}{1.683345in}}%
\pgfpathcurveto{\pgfqpoint{2.205178in}{1.691581in}}{\pgfqpoint{2.201905in}{1.699481in}}{\pgfqpoint{2.196081in}{1.705305in}}%
\pgfpathcurveto{\pgfqpoint{2.190257in}{1.711129in}}{\pgfqpoint{2.182357in}{1.714401in}}{\pgfqpoint{2.174121in}{1.714401in}}%
\pgfpathcurveto{\pgfqpoint{2.165885in}{1.714401in}}{\pgfqpoint{2.157985in}{1.711129in}}{\pgfqpoint{2.152161in}{1.705305in}}%
\pgfpathcurveto{\pgfqpoint{2.146337in}{1.699481in}}{\pgfqpoint{2.143065in}{1.691581in}}{\pgfqpoint{2.143065in}{1.683345in}}%
\pgfpathcurveto{\pgfqpoint{2.143065in}{1.675108in}}{\pgfqpoint{2.146337in}{1.667208in}}{\pgfqpoint{2.152161in}{1.661384in}}%
\pgfpathcurveto{\pgfqpoint{2.157985in}{1.655560in}}{\pgfqpoint{2.165885in}{1.652288in}}{\pgfqpoint{2.174121in}{1.652288in}}%
\pgfpathclose%
\pgfusepath{stroke,fill}%
\end{pgfscope}%
\begin{pgfscope}%
\pgfpathrectangle{\pgfqpoint{0.100000in}{0.212622in}}{\pgfqpoint{3.696000in}{3.696000in}}%
\pgfusepath{clip}%
\pgfsetbuttcap%
\pgfsetroundjoin%
\definecolor{currentfill}{rgb}{0.121569,0.466667,0.705882}%
\pgfsetfillcolor{currentfill}%
\pgfsetfillopacity{0.976959}%
\pgfsetlinewidth{1.003750pt}%
\definecolor{currentstroke}{rgb}{0.121569,0.466667,0.705882}%
\pgfsetstrokecolor{currentstroke}%
\pgfsetstrokeopacity{0.976959}%
\pgfsetdash{}{0pt}%
\pgfpathmoveto{\pgfqpoint{2.184030in}{1.654025in}}%
\pgfpathcurveto{\pgfqpoint{2.192267in}{1.654025in}}{\pgfqpoint{2.200167in}{1.657298in}}{\pgfqpoint{2.205991in}{1.663122in}}%
\pgfpathcurveto{\pgfqpoint{2.211814in}{1.668946in}}{\pgfqpoint{2.215087in}{1.676846in}}{\pgfqpoint{2.215087in}{1.685082in}}%
\pgfpathcurveto{\pgfqpoint{2.215087in}{1.693318in}}{\pgfqpoint{2.211814in}{1.701218in}}{\pgfqpoint{2.205991in}{1.707042in}}%
\pgfpathcurveto{\pgfqpoint{2.200167in}{1.712866in}}{\pgfqpoint{2.192267in}{1.716138in}}{\pgfqpoint{2.184030in}{1.716138in}}%
\pgfpathcurveto{\pgfqpoint{2.175794in}{1.716138in}}{\pgfqpoint{2.167894in}{1.712866in}}{\pgfqpoint{2.162070in}{1.707042in}}%
\pgfpathcurveto{\pgfqpoint{2.156246in}{1.701218in}}{\pgfqpoint{2.152974in}{1.693318in}}{\pgfqpoint{2.152974in}{1.685082in}}%
\pgfpathcurveto{\pgfqpoint{2.152974in}{1.676846in}}{\pgfqpoint{2.156246in}{1.668946in}}{\pgfqpoint{2.162070in}{1.663122in}}%
\pgfpathcurveto{\pgfqpoint{2.167894in}{1.657298in}}{\pgfqpoint{2.175794in}{1.654025in}}{\pgfqpoint{2.184030in}{1.654025in}}%
\pgfpathclose%
\pgfusepath{stroke,fill}%
\end{pgfscope}%
\begin{pgfscope}%
\pgfpathrectangle{\pgfqpoint{0.100000in}{0.212622in}}{\pgfqpoint{3.696000in}{3.696000in}}%
\pgfusepath{clip}%
\pgfsetbuttcap%
\pgfsetroundjoin%
\definecolor{currentfill}{rgb}{0.121569,0.466667,0.705882}%
\pgfsetfillcolor{currentfill}%
\pgfsetfillopacity{0.976967}%
\pgfsetlinewidth{1.003750pt}%
\definecolor{currentstroke}{rgb}{0.121569,0.466667,0.705882}%
\pgfsetstrokecolor{currentstroke}%
\pgfsetstrokeopacity{0.976967}%
\pgfsetdash{}{0pt}%
\pgfpathmoveto{\pgfqpoint{2.477520in}{1.593574in}}%
\pgfpathcurveto{\pgfqpoint{2.485757in}{1.593574in}}{\pgfqpoint{2.493657in}{1.596846in}}{\pgfqpoint{2.499481in}{1.602670in}}%
\pgfpathcurveto{\pgfqpoint{2.505305in}{1.608494in}}{\pgfqpoint{2.508577in}{1.616394in}}{\pgfqpoint{2.508577in}{1.624630in}}%
\pgfpathcurveto{\pgfqpoint{2.508577in}{1.632866in}}{\pgfqpoint{2.505305in}{1.640766in}}{\pgfqpoint{2.499481in}{1.646590in}}%
\pgfpathcurveto{\pgfqpoint{2.493657in}{1.652414in}}{\pgfqpoint{2.485757in}{1.655687in}}{\pgfqpoint{2.477520in}{1.655687in}}%
\pgfpathcurveto{\pgfqpoint{2.469284in}{1.655687in}}{\pgfqpoint{2.461384in}{1.652414in}}{\pgfqpoint{2.455560in}{1.646590in}}%
\pgfpathcurveto{\pgfqpoint{2.449736in}{1.640766in}}{\pgfqpoint{2.446464in}{1.632866in}}{\pgfqpoint{2.446464in}{1.624630in}}%
\pgfpathcurveto{\pgfqpoint{2.446464in}{1.616394in}}{\pgfqpoint{2.449736in}{1.608494in}}{\pgfqpoint{2.455560in}{1.602670in}}%
\pgfpathcurveto{\pgfqpoint{2.461384in}{1.596846in}}{\pgfqpoint{2.469284in}{1.593574in}}{\pgfqpoint{2.477520in}{1.593574in}}%
\pgfpathclose%
\pgfusepath{stroke,fill}%
\end{pgfscope}%
\begin{pgfscope}%
\pgfpathrectangle{\pgfqpoint{0.100000in}{0.212622in}}{\pgfqpoint{3.696000in}{3.696000in}}%
\pgfusepath{clip}%
\pgfsetbuttcap%
\pgfsetroundjoin%
\definecolor{currentfill}{rgb}{0.121569,0.466667,0.705882}%
\pgfsetfillcolor{currentfill}%
\pgfsetfillopacity{0.977707}%
\pgfsetlinewidth{1.003750pt}%
\definecolor{currentstroke}{rgb}{0.121569,0.466667,0.705882}%
\pgfsetstrokecolor{currentstroke}%
\pgfsetstrokeopacity{0.977707}%
\pgfsetdash{}{0pt}%
\pgfpathmoveto{\pgfqpoint{2.193266in}{1.649616in}}%
\pgfpathcurveto{\pgfqpoint{2.201502in}{1.649616in}}{\pgfqpoint{2.209402in}{1.652888in}}{\pgfqpoint{2.215226in}{1.658712in}}%
\pgfpathcurveto{\pgfqpoint{2.221050in}{1.664536in}}{\pgfqpoint{2.224322in}{1.672436in}}{\pgfqpoint{2.224322in}{1.680672in}}%
\pgfpathcurveto{\pgfqpoint{2.224322in}{1.688908in}}{\pgfqpoint{2.221050in}{1.696809in}}{\pgfqpoint{2.215226in}{1.702632in}}%
\pgfpathcurveto{\pgfqpoint{2.209402in}{1.708456in}}{\pgfqpoint{2.201502in}{1.711729in}}{\pgfqpoint{2.193266in}{1.711729in}}%
\pgfpathcurveto{\pgfqpoint{2.185029in}{1.711729in}}{\pgfqpoint{2.177129in}{1.708456in}}{\pgfqpoint{2.171305in}{1.702632in}}%
\pgfpathcurveto{\pgfqpoint{2.165481in}{1.696809in}}{\pgfqpoint{2.162209in}{1.688908in}}{\pgfqpoint{2.162209in}{1.680672in}}%
\pgfpathcurveto{\pgfqpoint{2.162209in}{1.672436in}}{\pgfqpoint{2.165481in}{1.664536in}}{\pgfqpoint{2.171305in}{1.658712in}}%
\pgfpathcurveto{\pgfqpoint{2.177129in}{1.652888in}}{\pgfqpoint{2.185029in}{1.649616in}}{\pgfqpoint{2.193266in}{1.649616in}}%
\pgfpathclose%
\pgfusepath{stroke,fill}%
\end{pgfscope}%
\begin{pgfscope}%
\pgfpathrectangle{\pgfqpoint{0.100000in}{0.212622in}}{\pgfqpoint{3.696000in}{3.696000in}}%
\pgfusepath{clip}%
\pgfsetbuttcap%
\pgfsetroundjoin%
\definecolor{currentfill}{rgb}{0.121569,0.466667,0.705882}%
\pgfsetfillcolor{currentfill}%
\pgfsetfillopacity{0.978260}%
\pgfsetlinewidth{1.003750pt}%
\definecolor{currentstroke}{rgb}{0.121569,0.466667,0.705882}%
\pgfsetstrokecolor{currentstroke}%
\pgfsetstrokeopacity{0.978260}%
\pgfsetdash{}{0pt}%
\pgfpathmoveto{\pgfqpoint{2.200267in}{1.644543in}}%
\pgfpathcurveto{\pgfqpoint{2.208504in}{1.644543in}}{\pgfqpoint{2.216404in}{1.647815in}}{\pgfqpoint{2.222228in}{1.653639in}}%
\pgfpathcurveto{\pgfqpoint{2.228052in}{1.659463in}}{\pgfqpoint{2.231324in}{1.667363in}}{\pgfqpoint{2.231324in}{1.675600in}}%
\pgfpathcurveto{\pgfqpoint{2.231324in}{1.683836in}}{\pgfqpoint{2.228052in}{1.691736in}}{\pgfqpoint{2.222228in}{1.697560in}}%
\pgfpathcurveto{\pgfqpoint{2.216404in}{1.703384in}}{\pgfqpoint{2.208504in}{1.706656in}}{\pgfqpoint{2.200267in}{1.706656in}}%
\pgfpathcurveto{\pgfqpoint{2.192031in}{1.706656in}}{\pgfqpoint{2.184131in}{1.703384in}}{\pgfqpoint{2.178307in}{1.697560in}}%
\pgfpathcurveto{\pgfqpoint{2.172483in}{1.691736in}}{\pgfqpoint{2.169211in}{1.683836in}}{\pgfqpoint{2.169211in}{1.675600in}}%
\pgfpathcurveto{\pgfqpoint{2.169211in}{1.667363in}}{\pgfqpoint{2.172483in}{1.659463in}}{\pgfqpoint{2.178307in}{1.653639in}}%
\pgfpathcurveto{\pgfqpoint{2.184131in}{1.647815in}}{\pgfqpoint{2.192031in}{1.644543in}}{\pgfqpoint{2.200267in}{1.644543in}}%
\pgfpathclose%
\pgfusepath{stroke,fill}%
\end{pgfscope}%
\begin{pgfscope}%
\pgfpathrectangle{\pgfqpoint{0.100000in}{0.212622in}}{\pgfqpoint{3.696000in}{3.696000in}}%
\pgfusepath{clip}%
\pgfsetbuttcap%
\pgfsetroundjoin%
\definecolor{currentfill}{rgb}{0.121569,0.466667,0.705882}%
\pgfsetfillcolor{currentfill}%
\pgfsetfillopacity{0.978658}%
\pgfsetlinewidth{1.003750pt}%
\definecolor{currentstroke}{rgb}{0.121569,0.466667,0.705882}%
\pgfsetstrokecolor{currentstroke}%
\pgfsetstrokeopacity{0.978658}%
\pgfsetdash{}{0pt}%
\pgfpathmoveto{\pgfqpoint{2.470924in}{1.585352in}}%
\pgfpathcurveto{\pgfqpoint{2.479160in}{1.585352in}}{\pgfqpoint{2.487060in}{1.588624in}}{\pgfqpoint{2.492884in}{1.594448in}}%
\pgfpathcurveto{\pgfqpoint{2.498708in}{1.600272in}}{\pgfqpoint{2.501980in}{1.608172in}}{\pgfqpoint{2.501980in}{1.616408in}}%
\pgfpathcurveto{\pgfqpoint{2.501980in}{1.624645in}}{\pgfqpoint{2.498708in}{1.632545in}}{\pgfqpoint{2.492884in}{1.638369in}}%
\pgfpathcurveto{\pgfqpoint{2.487060in}{1.644193in}}{\pgfqpoint{2.479160in}{1.647465in}}{\pgfqpoint{2.470924in}{1.647465in}}%
\pgfpathcurveto{\pgfqpoint{2.462688in}{1.647465in}}{\pgfqpoint{2.454787in}{1.644193in}}{\pgfqpoint{2.448964in}{1.638369in}}%
\pgfpathcurveto{\pgfqpoint{2.443140in}{1.632545in}}{\pgfqpoint{2.439867in}{1.624645in}}{\pgfqpoint{2.439867in}{1.616408in}}%
\pgfpathcurveto{\pgfqpoint{2.439867in}{1.608172in}}{\pgfqpoint{2.443140in}{1.600272in}}{\pgfqpoint{2.448964in}{1.594448in}}%
\pgfpathcurveto{\pgfqpoint{2.454787in}{1.588624in}}{\pgfqpoint{2.462688in}{1.585352in}}{\pgfqpoint{2.470924in}{1.585352in}}%
\pgfpathclose%
\pgfusepath{stroke,fill}%
\end{pgfscope}%
\begin{pgfscope}%
\pgfpathrectangle{\pgfqpoint{0.100000in}{0.212622in}}{\pgfqpoint{3.696000in}{3.696000in}}%
\pgfusepath{clip}%
\pgfsetbuttcap%
\pgfsetroundjoin%
\definecolor{currentfill}{rgb}{0.121569,0.466667,0.705882}%
\pgfsetfillcolor{currentfill}%
\pgfsetfillopacity{0.979009}%
\pgfsetlinewidth{1.003750pt}%
\definecolor{currentstroke}{rgb}{0.121569,0.466667,0.705882}%
\pgfsetstrokecolor{currentstroke}%
\pgfsetstrokeopacity{0.979009}%
\pgfsetdash{}{0pt}%
\pgfpathmoveto{\pgfqpoint{2.206520in}{1.640782in}}%
\pgfpathcurveto{\pgfqpoint{2.214756in}{1.640782in}}{\pgfqpoint{2.222656in}{1.644054in}}{\pgfqpoint{2.228480in}{1.649878in}}%
\pgfpathcurveto{\pgfqpoint{2.234304in}{1.655702in}}{\pgfqpoint{2.237577in}{1.663602in}}{\pgfqpoint{2.237577in}{1.671838in}}%
\pgfpathcurveto{\pgfqpoint{2.237577in}{1.680075in}}{\pgfqpoint{2.234304in}{1.687975in}}{\pgfqpoint{2.228480in}{1.693799in}}%
\pgfpathcurveto{\pgfqpoint{2.222656in}{1.699623in}}{\pgfqpoint{2.214756in}{1.702895in}}{\pgfqpoint{2.206520in}{1.702895in}}%
\pgfpathcurveto{\pgfqpoint{2.198284in}{1.702895in}}{\pgfqpoint{2.190384in}{1.699623in}}{\pgfqpoint{2.184560in}{1.693799in}}%
\pgfpathcurveto{\pgfqpoint{2.178736in}{1.687975in}}{\pgfqpoint{2.175464in}{1.680075in}}{\pgfqpoint{2.175464in}{1.671838in}}%
\pgfpathcurveto{\pgfqpoint{2.175464in}{1.663602in}}{\pgfqpoint{2.178736in}{1.655702in}}{\pgfqpoint{2.184560in}{1.649878in}}%
\pgfpathcurveto{\pgfqpoint{2.190384in}{1.644054in}}{\pgfqpoint{2.198284in}{1.640782in}}{\pgfqpoint{2.206520in}{1.640782in}}%
\pgfpathclose%
\pgfusepath{stroke,fill}%
\end{pgfscope}%
\begin{pgfscope}%
\pgfpathrectangle{\pgfqpoint{0.100000in}{0.212622in}}{\pgfqpoint{3.696000in}{3.696000in}}%
\pgfusepath{clip}%
\pgfsetbuttcap%
\pgfsetroundjoin%
\definecolor{currentfill}{rgb}{0.121569,0.466667,0.705882}%
\pgfsetfillcolor{currentfill}%
\pgfsetfillopacity{0.980783}%
\pgfsetlinewidth{1.003750pt}%
\definecolor{currentstroke}{rgb}{0.121569,0.466667,0.705882}%
\pgfsetstrokecolor{currentstroke}%
\pgfsetstrokeopacity{0.980783}%
\pgfsetdash{}{0pt}%
\pgfpathmoveto{\pgfqpoint{2.462802in}{1.581366in}}%
\pgfpathcurveto{\pgfqpoint{2.471038in}{1.581366in}}{\pgfqpoint{2.478938in}{1.584638in}}{\pgfqpoint{2.484762in}{1.590462in}}%
\pgfpathcurveto{\pgfqpoint{2.490586in}{1.596286in}}{\pgfqpoint{2.493858in}{1.604186in}}{\pgfqpoint{2.493858in}{1.612422in}}%
\pgfpathcurveto{\pgfqpoint{2.493858in}{1.620659in}}{\pgfqpoint{2.490586in}{1.628559in}}{\pgfqpoint{2.484762in}{1.634383in}}%
\pgfpathcurveto{\pgfqpoint{2.478938in}{1.640207in}}{\pgfqpoint{2.471038in}{1.643479in}}{\pgfqpoint{2.462802in}{1.643479in}}%
\pgfpathcurveto{\pgfqpoint{2.454565in}{1.643479in}}{\pgfqpoint{2.446665in}{1.640207in}}{\pgfqpoint{2.440841in}{1.634383in}}%
\pgfpathcurveto{\pgfqpoint{2.435017in}{1.628559in}}{\pgfqpoint{2.431745in}{1.620659in}}{\pgfqpoint{2.431745in}{1.612422in}}%
\pgfpathcurveto{\pgfqpoint{2.431745in}{1.604186in}}{\pgfqpoint{2.435017in}{1.596286in}}{\pgfqpoint{2.440841in}{1.590462in}}%
\pgfpathcurveto{\pgfqpoint{2.446665in}{1.584638in}}{\pgfqpoint{2.454565in}{1.581366in}}{\pgfqpoint{2.462802in}{1.581366in}}%
\pgfpathclose%
\pgfusepath{stroke,fill}%
\end{pgfscope}%
\begin{pgfscope}%
\pgfpathrectangle{\pgfqpoint{0.100000in}{0.212622in}}{\pgfqpoint{3.696000in}{3.696000in}}%
\pgfusepath{clip}%
\pgfsetbuttcap%
\pgfsetroundjoin%
\definecolor{currentfill}{rgb}{0.121569,0.466667,0.705882}%
\pgfsetfillcolor{currentfill}%
\pgfsetfillopacity{0.981203}%
\pgfsetlinewidth{1.003750pt}%
\definecolor{currentstroke}{rgb}{0.121569,0.466667,0.705882}%
\pgfsetstrokecolor{currentstroke}%
\pgfsetstrokeopacity{0.981203}%
\pgfsetdash{}{0pt}%
\pgfpathmoveto{\pgfqpoint{2.219104in}{1.643452in}}%
\pgfpathcurveto{\pgfqpoint{2.227340in}{1.643452in}}{\pgfqpoint{2.235240in}{1.646724in}}{\pgfqpoint{2.241064in}{1.652548in}}%
\pgfpathcurveto{\pgfqpoint{2.246888in}{1.658372in}}{\pgfqpoint{2.250160in}{1.666272in}}{\pgfqpoint{2.250160in}{1.674508in}}%
\pgfpathcurveto{\pgfqpoint{2.250160in}{1.682744in}}{\pgfqpoint{2.246888in}{1.690644in}}{\pgfqpoint{2.241064in}{1.696468in}}%
\pgfpathcurveto{\pgfqpoint{2.235240in}{1.702292in}}{\pgfqpoint{2.227340in}{1.705565in}}{\pgfqpoint{2.219104in}{1.705565in}}%
\pgfpathcurveto{\pgfqpoint{2.210868in}{1.705565in}}{\pgfqpoint{2.202968in}{1.702292in}}{\pgfqpoint{2.197144in}{1.696468in}}%
\pgfpathcurveto{\pgfqpoint{2.191320in}{1.690644in}}{\pgfqpoint{2.188047in}{1.682744in}}{\pgfqpoint{2.188047in}{1.674508in}}%
\pgfpathcurveto{\pgfqpoint{2.188047in}{1.666272in}}{\pgfqpoint{2.191320in}{1.658372in}}{\pgfqpoint{2.197144in}{1.652548in}}%
\pgfpathcurveto{\pgfqpoint{2.202968in}{1.646724in}}{\pgfqpoint{2.210868in}{1.643452in}}{\pgfqpoint{2.219104in}{1.643452in}}%
\pgfpathclose%
\pgfusepath{stroke,fill}%
\end{pgfscope}%
\begin{pgfscope}%
\pgfpathrectangle{\pgfqpoint{0.100000in}{0.212622in}}{\pgfqpoint{3.696000in}{3.696000in}}%
\pgfusepath{clip}%
\pgfsetbuttcap%
\pgfsetroundjoin%
\definecolor{currentfill}{rgb}{0.121569,0.466667,0.705882}%
\pgfsetfillcolor{currentfill}%
\pgfsetfillopacity{0.981612}%
\pgfsetlinewidth{1.003750pt}%
\definecolor{currentstroke}{rgb}{0.121569,0.466667,0.705882}%
\pgfsetstrokecolor{currentstroke}%
\pgfsetstrokeopacity{0.981612}%
\pgfsetdash{}{0pt}%
\pgfpathmoveto{\pgfqpoint{2.230847in}{1.635142in}}%
\pgfpathcurveto{\pgfqpoint{2.239083in}{1.635142in}}{\pgfqpoint{2.246983in}{1.638414in}}{\pgfqpoint{2.252807in}{1.644238in}}%
\pgfpathcurveto{\pgfqpoint{2.258631in}{1.650062in}}{\pgfqpoint{2.261903in}{1.657962in}}{\pgfqpoint{2.261903in}{1.666199in}}%
\pgfpathcurveto{\pgfqpoint{2.261903in}{1.674435in}}{\pgfqpoint{2.258631in}{1.682335in}}{\pgfqpoint{2.252807in}{1.688159in}}%
\pgfpathcurveto{\pgfqpoint{2.246983in}{1.693983in}}{\pgfqpoint{2.239083in}{1.697255in}}{\pgfqpoint{2.230847in}{1.697255in}}%
\pgfpathcurveto{\pgfqpoint{2.222611in}{1.697255in}}{\pgfqpoint{2.214711in}{1.693983in}}{\pgfqpoint{2.208887in}{1.688159in}}%
\pgfpathcurveto{\pgfqpoint{2.203063in}{1.682335in}}{\pgfqpoint{2.199790in}{1.674435in}}{\pgfqpoint{2.199790in}{1.666199in}}%
\pgfpathcurveto{\pgfqpoint{2.199790in}{1.657962in}}{\pgfqpoint{2.203063in}{1.650062in}}{\pgfqpoint{2.208887in}{1.644238in}}%
\pgfpathcurveto{\pgfqpoint{2.214711in}{1.638414in}}{\pgfqpoint{2.222611in}{1.635142in}}{\pgfqpoint{2.230847in}{1.635142in}}%
\pgfpathclose%
\pgfusepath{stroke,fill}%
\end{pgfscope}%
\begin{pgfscope}%
\pgfpathrectangle{\pgfqpoint{0.100000in}{0.212622in}}{\pgfqpoint{3.696000in}{3.696000in}}%
\pgfusepath{clip}%
\pgfsetbuttcap%
\pgfsetroundjoin%
\definecolor{currentfill}{rgb}{0.121569,0.466667,0.705882}%
\pgfsetfillcolor{currentfill}%
\pgfsetfillopacity{0.982293}%
\pgfsetlinewidth{1.003750pt}%
\definecolor{currentstroke}{rgb}{0.121569,0.466667,0.705882}%
\pgfsetstrokecolor{currentstroke}%
\pgfsetstrokeopacity{0.982293}%
\pgfsetdash{}{0pt}%
\pgfpathmoveto{\pgfqpoint{2.240563in}{1.631194in}}%
\pgfpathcurveto{\pgfqpoint{2.248799in}{1.631194in}}{\pgfqpoint{2.256699in}{1.634466in}}{\pgfqpoint{2.262523in}{1.640290in}}%
\pgfpathcurveto{\pgfqpoint{2.268347in}{1.646114in}}{\pgfqpoint{2.271620in}{1.654014in}}{\pgfqpoint{2.271620in}{1.662250in}}%
\pgfpathcurveto{\pgfqpoint{2.271620in}{1.670486in}}{\pgfqpoint{2.268347in}{1.678386in}}{\pgfqpoint{2.262523in}{1.684210in}}%
\pgfpathcurveto{\pgfqpoint{2.256699in}{1.690034in}}{\pgfqpoint{2.248799in}{1.693307in}}{\pgfqpoint{2.240563in}{1.693307in}}%
\pgfpathcurveto{\pgfqpoint{2.232327in}{1.693307in}}{\pgfqpoint{2.224427in}{1.690034in}}{\pgfqpoint{2.218603in}{1.684210in}}%
\pgfpathcurveto{\pgfqpoint{2.212779in}{1.678386in}}{\pgfqpoint{2.209507in}{1.670486in}}{\pgfqpoint{2.209507in}{1.662250in}}%
\pgfpathcurveto{\pgfqpoint{2.209507in}{1.654014in}}{\pgfqpoint{2.212779in}{1.646114in}}{\pgfqpoint{2.218603in}{1.640290in}}%
\pgfpathcurveto{\pgfqpoint{2.224427in}{1.634466in}}{\pgfqpoint{2.232327in}{1.631194in}}{\pgfqpoint{2.240563in}{1.631194in}}%
\pgfpathclose%
\pgfusepath{stroke,fill}%
\end{pgfscope}%
\begin{pgfscope}%
\pgfpathrectangle{\pgfqpoint{0.100000in}{0.212622in}}{\pgfqpoint{3.696000in}{3.696000in}}%
\pgfusepath{clip}%
\pgfsetbuttcap%
\pgfsetroundjoin%
\definecolor{currentfill}{rgb}{0.121569,0.466667,0.705882}%
\pgfsetfillcolor{currentfill}%
\pgfsetfillopacity{0.982685}%
\pgfsetlinewidth{1.003750pt}%
\definecolor{currentstroke}{rgb}{0.121569,0.466667,0.705882}%
\pgfsetstrokecolor{currentstroke}%
\pgfsetstrokeopacity{0.982685}%
\pgfsetdash{}{0pt}%
\pgfpathmoveto{\pgfqpoint{2.248513in}{1.626170in}}%
\pgfpathcurveto{\pgfqpoint{2.256749in}{1.626170in}}{\pgfqpoint{2.264649in}{1.629443in}}{\pgfqpoint{2.270473in}{1.635267in}}%
\pgfpathcurveto{\pgfqpoint{2.276297in}{1.641090in}}{\pgfqpoint{2.279570in}{1.648991in}}{\pgfqpoint{2.279570in}{1.657227in}}%
\pgfpathcurveto{\pgfqpoint{2.279570in}{1.665463in}}{\pgfqpoint{2.276297in}{1.673363in}}{\pgfqpoint{2.270473in}{1.679187in}}%
\pgfpathcurveto{\pgfqpoint{2.264649in}{1.685011in}}{\pgfqpoint{2.256749in}{1.688283in}}{\pgfqpoint{2.248513in}{1.688283in}}%
\pgfpathcurveto{\pgfqpoint{2.240277in}{1.688283in}}{\pgfqpoint{2.232377in}{1.685011in}}{\pgfqpoint{2.226553in}{1.679187in}}%
\pgfpathcurveto{\pgfqpoint{2.220729in}{1.673363in}}{\pgfqpoint{2.217457in}{1.665463in}}{\pgfqpoint{2.217457in}{1.657227in}}%
\pgfpathcurveto{\pgfqpoint{2.217457in}{1.648991in}}{\pgfqpoint{2.220729in}{1.641090in}}{\pgfqpoint{2.226553in}{1.635267in}}%
\pgfpathcurveto{\pgfqpoint{2.232377in}{1.629443in}}{\pgfqpoint{2.240277in}{1.626170in}}{\pgfqpoint{2.248513in}{1.626170in}}%
\pgfpathclose%
\pgfusepath{stroke,fill}%
\end{pgfscope}%
\begin{pgfscope}%
\pgfpathrectangle{\pgfqpoint{0.100000in}{0.212622in}}{\pgfqpoint{3.696000in}{3.696000in}}%
\pgfusepath{clip}%
\pgfsetbuttcap%
\pgfsetroundjoin%
\definecolor{currentfill}{rgb}{0.121569,0.466667,0.705882}%
\pgfsetfillcolor{currentfill}%
\pgfsetfillopacity{0.984405}%
\pgfsetlinewidth{1.003750pt}%
\definecolor{currentstroke}{rgb}{0.121569,0.466667,0.705882}%
\pgfsetstrokecolor{currentstroke}%
\pgfsetstrokeopacity{0.984405}%
\pgfsetdash{}{0pt}%
\pgfpathmoveto{\pgfqpoint{2.459266in}{1.578840in}}%
\pgfpathcurveto{\pgfqpoint{2.467502in}{1.578840in}}{\pgfqpoint{2.475402in}{1.582112in}}{\pgfqpoint{2.481226in}{1.587936in}}%
\pgfpathcurveto{\pgfqpoint{2.487050in}{1.593760in}}{\pgfqpoint{2.490322in}{1.601660in}}{\pgfqpoint{2.490322in}{1.609897in}}%
\pgfpathcurveto{\pgfqpoint{2.490322in}{1.618133in}}{\pgfqpoint{2.487050in}{1.626033in}}{\pgfqpoint{2.481226in}{1.631857in}}%
\pgfpathcurveto{\pgfqpoint{2.475402in}{1.637681in}}{\pgfqpoint{2.467502in}{1.640953in}}{\pgfqpoint{2.459266in}{1.640953in}}%
\pgfpathcurveto{\pgfqpoint{2.451030in}{1.640953in}}{\pgfqpoint{2.443130in}{1.637681in}}{\pgfqpoint{2.437306in}{1.631857in}}%
\pgfpathcurveto{\pgfqpoint{2.431482in}{1.626033in}}{\pgfqpoint{2.428209in}{1.618133in}}{\pgfqpoint{2.428209in}{1.609897in}}%
\pgfpathcurveto{\pgfqpoint{2.428209in}{1.601660in}}{\pgfqpoint{2.431482in}{1.593760in}}{\pgfqpoint{2.437306in}{1.587936in}}%
\pgfpathcurveto{\pgfqpoint{2.443130in}{1.582112in}}{\pgfqpoint{2.451030in}{1.578840in}}{\pgfqpoint{2.459266in}{1.578840in}}%
\pgfpathclose%
\pgfusepath{stroke,fill}%
\end{pgfscope}%
\begin{pgfscope}%
\pgfpathrectangle{\pgfqpoint{0.100000in}{0.212622in}}{\pgfqpoint{3.696000in}{3.696000in}}%
\pgfusepath{clip}%
\pgfsetbuttcap%
\pgfsetroundjoin%
\definecolor{currentfill}{rgb}{0.121569,0.466667,0.705882}%
\pgfsetfillcolor{currentfill}%
\pgfsetfillopacity{0.984527}%
\pgfsetlinewidth{1.003750pt}%
\definecolor{currentstroke}{rgb}{0.121569,0.466667,0.705882}%
\pgfsetstrokecolor{currentstroke}%
\pgfsetstrokeopacity{0.984527}%
\pgfsetdash{}{0pt}%
\pgfpathmoveto{\pgfqpoint{2.262984in}{1.624199in}}%
\pgfpathcurveto{\pgfqpoint{2.271221in}{1.624199in}}{\pgfqpoint{2.279121in}{1.627471in}}{\pgfqpoint{2.284945in}{1.633295in}}%
\pgfpathcurveto{\pgfqpoint{2.290769in}{1.639119in}}{\pgfqpoint{2.294041in}{1.647019in}}{\pgfqpoint{2.294041in}{1.655255in}}%
\pgfpathcurveto{\pgfqpoint{2.294041in}{1.663492in}}{\pgfqpoint{2.290769in}{1.671392in}}{\pgfqpoint{2.284945in}{1.677216in}}%
\pgfpathcurveto{\pgfqpoint{2.279121in}{1.683039in}}{\pgfqpoint{2.271221in}{1.686312in}}{\pgfqpoint{2.262984in}{1.686312in}}%
\pgfpathcurveto{\pgfqpoint{2.254748in}{1.686312in}}{\pgfqpoint{2.246848in}{1.683039in}}{\pgfqpoint{2.241024in}{1.677216in}}%
\pgfpathcurveto{\pgfqpoint{2.235200in}{1.671392in}}{\pgfqpoint{2.231928in}{1.663492in}}{\pgfqpoint{2.231928in}{1.655255in}}%
\pgfpathcurveto{\pgfqpoint{2.231928in}{1.647019in}}{\pgfqpoint{2.235200in}{1.639119in}}{\pgfqpoint{2.241024in}{1.633295in}}%
\pgfpathcurveto{\pgfqpoint{2.246848in}{1.627471in}}{\pgfqpoint{2.254748in}{1.624199in}}{\pgfqpoint{2.262984in}{1.624199in}}%
\pgfpathclose%
\pgfusepath{stroke,fill}%
\end{pgfscope}%
\begin{pgfscope}%
\pgfpathrectangle{\pgfqpoint{0.100000in}{0.212622in}}{\pgfqpoint{3.696000in}{3.696000in}}%
\pgfusepath{clip}%
\pgfsetbuttcap%
\pgfsetroundjoin%
\definecolor{currentfill}{rgb}{0.121569,0.466667,0.705882}%
\pgfsetfillcolor{currentfill}%
\pgfsetfillopacity{0.985766}%
\pgfsetlinewidth{1.003750pt}%
\definecolor{currentstroke}{rgb}{0.121569,0.466667,0.705882}%
\pgfsetstrokecolor{currentstroke}%
\pgfsetstrokeopacity{0.985766}%
\pgfsetdash{}{0pt}%
\pgfpathmoveto{\pgfqpoint{2.274931in}{1.619752in}}%
\pgfpathcurveto{\pgfqpoint{2.283167in}{1.619752in}}{\pgfqpoint{2.291067in}{1.623024in}}{\pgfqpoint{2.296891in}{1.628848in}}%
\pgfpathcurveto{\pgfqpoint{2.302715in}{1.634672in}}{\pgfqpoint{2.305987in}{1.642572in}}{\pgfqpoint{2.305987in}{1.650808in}}%
\pgfpathcurveto{\pgfqpoint{2.305987in}{1.659045in}}{\pgfqpoint{2.302715in}{1.666945in}}{\pgfqpoint{2.296891in}{1.672769in}}%
\pgfpathcurveto{\pgfqpoint{2.291067in}{1.678592in}}{\pgfqpoint{2.283167in}{1.681865in}}{\pgfqpoint{2.274931in}{1.681865in}}%
\pgfpathcurveto{\pgfqpoint{2.266695in}{1.681865in}}{\pgfqpoint{2.258795in}{1.678592in}}{\pgfqpoint{2.252971in}{1.672769in}}%
\pgfpathcurveto{\pgfqpoint{2.247147in}{1.666945in}}{\pgfqpoint{2.243874in}{1.659045in}}{\pgfqpoint{2.243874in}{1.650808in}}%
\pgfpathcurveto{\pgfqpoint{2.243874in}{1.642572in}}{\pgfqpoint{2.247147in}{1.634672in}}{\pgfqpoint{2.252971in}{1.628848in}}%
\pgfpathcurveto{\pgfqpoint{2.258795in}{1.623024in}}{\pgfqpoint{2.266695in}{1.619752in}}{\pgfqpoint{2.274931in}{1.619752in}}%
\pgfpathclose%
\pgfusepath{stroke,fill}%
\end{pgfscope}%
\begin{pgfscope}%
\pgfpathrectangle{\pgfqpoint{0.100000in}{0.212622in}}{\pgfqpoint{3.696000in}{3.696000in}}%
\pgfusepath{clip}%
\pgfsetbuttcap%
\pgfsetroundjoin%
\definecolor{currentfill}{rgb}{0.121569,0.466667,0.705882}%
\pgfsetfillcolor{currentfill}%
\pgfsetfillopacity{0.985877}%
\pgfsetlinewidth{1.003750pt}%
\definecolor{currentstroke}{rgb}{0.121569,0.466667,0.705882}%
\pgfsetstrokecolor{currentstroke}%
\pgfsetstrokeopacity{0.985877}%
\pgfsetdash{}{0pt}%
\pgfpathmoveto{\pgfqpoint{2.455519in}{1.575589in}}%
\pgfpathcurveto{\pgfqpoint{2.463755in}{1.575589in}}{\pgfqpoint{2.471655in}{1.578862in}}{\pgfqpoint{2.477479in}{1.584686in}}%
\pgfpathcurveto{\pgfqpoint{2.483303in}{1.590510in}}{\pgfqpoint{2.486576in}{1.598410in}}{\pgfqpoint{2.486576in}{1.606646in}}%
\pgfpathcurveto{\pgfqpoint{2.486576in}{1.614882in}}{\pgfqpoint{2.483303in}{1.622782in}}{\pgfqpoint{2.477479in}{1.628606in}}%
\pgfpathcurveto{\pgfqpoint{2.471655in}{1.634430in}}{\pgfqpoint{2.463755in}{1.637702in}}{\pgfqpoint{2.455519in}{1.637702in}}%
\pgfpathcurveto{\pgfqpoint{2.447283in}{1.637702in}}{\pgfqpoint{2.439383in}{1.634430in}}{\pgfqpoint{2.433559in}{1.628606in}}%
\pgfpathcurveto{\pgfqpoint{2.427735in}{1.622782in}}{\pgfqpoint{2.424463in}{1.614882in}}{\pgfqpoint{2.424463in}{1.606646in}}%
\pgfpathcurveto{\pgfqpoint{2.424463in}{1.598410in}}{\pgfqpoint{2.427735in}{1.590510in}}{\pgfqpoint{2.433559in}{1.584686in}}%
\pgfpathcurveto{\pgfqpoint{2.439383in}{1.578862in}}{\pgfqpoint{2.447283in}{1.575589in}}{\pgfqpoint{2.455519in}{1.575589in}}%
\pgfpathclose%
\pgfusepath{stroke,fill}%
\end{pgfscope}%
\begin{pgfscope}%
\pgfpathrectangle{\pgfqpoint{0.100000in}{0.212622in}}{\pgfqpoint{3.696000in}{3.696000in}}%
\pgfusepath{clip}%
\pgfsetbuttcap%
\pgfsetroundjoin%
\definecolor{currentfill}{rgb}{0.121569,0.466667,0.705882}%
\pgfsetfillcolor{currentfill}%
\pgfsetfillopacity{0.985961}%
\pgfsetlinewidth{1.003750pt}%
\definecolor{currentstroke}{rgb}{0.121569,0.466667,0.705882}%
\pgfsetstrokecolor{currentstroke}%
\pgfsetstrokeopacity{0.985961}%
\pgfsetdash{}{0pt}%
\pgfpathmoveto{\pgfqpoint{2.283860in}{1.612077in}}%
\pgfpathcurveto{\pgfqpoint{2.292096in}{1.612077in}}{\pgfqpoint{2.299996in}{1.615350in}}{\pgfqpoint{2.305820in}{1.621174in}}%
\pgfpathcurveto{\pgfqpoint{2.311644in}{1.626998in}}{\pgfqpoint{2.314916in}{1.634898in}}{\pgfqpoint{2.314916in}{1.643134in}}%
\pgfpathcurveto{\pgfqpoint{2.314916in}{1.651370in}}{\pgfqpoint{2.311644in}{1.659270in}}{\pgfqpoint{2.305820in}{1.665094in}}%
\pgfpathcurveto{\pgfqpoint{2.299996in}{1.670918in}}{\pgfqpoint{2.292096in}{1.674190in}}{\pgfqpoint{2.283860in}{1.674190in}}%
\pgfpathcurveto{\pgfqpoint{2.275624in}{1.674190in}}{\pgfqpoint{2.267723in}{1.670918in}}{\pgfqpoint{2.261900in}{1.665094in}}%
\pgfpathcurveto{\pgfqpoint{2.256076in}{1.659270in}}{\pgfqpoint{2.252803in}{1.651370in}}{\pgfqpoint{2.252803in}{1.643134in}}%
\pgfpathcurveto{\pgfqpoint{2.252803in}{1.634898in}}{\pgfqpoint{2.256076in}{1.626998in}}{\pgfqpoint{2.261900in}{1.621174in}}%
\pgfpathcurveto{\pgfqpoint{2.267723in}{1.615350in}}{\pgfqpoint{2.275624in}{1.612077in}}{\pgfqpoint{2.283860in}{1.612077in}}%
\pgfpathclose%
\pgfusepath{stroke,fill}%
\end{pgfscope}%
\begin{pgfscope}%
\pgfpathrectangle{\pgfqpoint{0.100000in}{0.212622in}}{\pgfqpoint{3.696000in}{3.696000in}}%
\pgfusepath{clip}%
\pgfsetbuttcap%
\pgfsetroundjoin%
\definecolor{currentfill}{rgb}{0.121569,0.466667,0.705882}%
\pgfsetfillcolor{currentfill}%
\pgfsetfillopacity{0.986953}%
\pgfsetlinewidth{1.003750pt}%
\definecolor{currentstroke}{rgb}{0.121569,0.466667,0.705882}%
\pgfsetstrokecolor{currentstroke}%
\pgfsetstrokeopacity{0.986953}%
\pgfsetdash{}{0pt}%
\pgfpathmoveto{\pgfqpoint{2.292509in}{1.609746in}}%
\pgfpathcurveto{\pgfqpoint{2.300745in}{1.609746in}}{\pgfqpoint{2.308645in}{1.613018in}}{\pgfqpoint{2.314469in}{1.618842in}}%
\pgfpathcurveto{\pgfqpoint{2.320293in}{1.624666in}}{\pgfqpoint{2.323565in}{1.632566in}}{\pgfqpoint{2.323565in}{1.640802in}}%
\pgfpathcurveto{\pgfqpoint{2.323565in}{1.649039in}}{\pgfqpoint{2.320293in}{1.656939in}}{\pgfqpoint{2.314469in}{1.662762in}}%
\pgfpathcurveto{\pgfqpoint{2.308645in}{1.668586in}}{\pgfqpoint{2.300745in}{1.671859in}}{\pgfqpoint{2.292509in}{1.671859in}}%
\pgfpathcurveto{\pgfqpoint{2.284272in}{1.671859in}}{\pgfqpoint{2.276372in}{1.668586in}}{\pgfqpoint{2.270548in}{1.662762in}}%
\pgfpathcurveto{\pgfqpoint{2.264724in}{1.656939in}}{\pgfqpoint{2.261452in}{1.649039in}}{\pgfqpoint{2.261452in}{1.640802in}}%
\pgfpathcurveto{\pgfqpoint{2.261452in}{1.632566in}}{\pgfqpoint{2.264724in}{1.624666in}}{\pgfqpoint{2.270548in}{1.618842in}}%
\pgfpathcurveto{\pgfqpoint{2.276372in}{1.613018in}}{\pgfqpoint{2.284272in}{1.609746in}}{\pgfqpoint{2.292509in}{1.609746in}}%
\pgfpathclose%
\pgfusepath{stroke,fill}%
\end{pgfscope}%
\begin{pgfscope}%
\pgfpathrectangle{\pgfqpoint{0.100000in}{0.212622in}}{\pgfqpoint{3.696000in}{3.696000in}}%
\pgfusepath{clip}%
\pgfsetbuttcap%
\pgfsetroundjoin%
\definecolor{currentfill}{rgb}{0.121569,0.466667,0.705882}%
\pgfsetfillcolor{currentfill}%
\pgfsetfillopacity{0.987495}%
\pgfsetlinewidth{1.003750pt}%
\definecolor{currentstroke}{rgb}{0.121569,0.466667,0.705882}%
\pgfsetstrokecolor{currentstroke}%
\pgfsetstrokeopacity{0.987495}%
\pgfsetdash{}{0pt}%
\pgfpathmoveto{\pgfqpoint{2.451830in}{1.572321in}}%
\pgfpathcurveto{\pgfqpoint{2.460066in}{1.572321in}}{\pgfqpoint{2.467966in}{1.575593in}}{\pgfqpoint{2.473790in}{1.581417in}}%
\pgfpathcurveto{\pgfqpoint{2.479614in}{1.587241in}}{\pgfqpoint{2.482886in}{1.595141in}}{\pgfqpoint{2.482886in}{1.603377in}}%
\pgfpathcurveto{\pgfqpoint{2.482886in}{1.611614in}}{\pgfqpoint{2.479614in}{1.619514in}}{\pgfqpoint{2.473790in}{1.625338in}}%
\pgfpathcurveto{\pgfqpoint{2.467966in}{1.631162in}}{\pgfqpoint{2.460066in}{1.634434in}}{\pgfqpoint{2.451830in}{1.634434in}}%
\pgfpathcurveto{\pgfqpoint{2.443594in}{1.634434in}}{\pgfqpoint{2.435694in}{1.631162in}}{\pgfqpoint{2.429870in}{1.625338in}}%
\pgfpathcurveto{\pgfqpoint{2.424046in}{1.619514in}}{\pgfqpoint{2.420773in}{1.611614in}}{\pgfqpoint{2.420773in}{1.603377in}}%
\pgfpathcurveto{\pgfqpoint{2.420773in}{1.595141in}}{\pgfqpoint{2.424046in}{1.587241in}}{\pgfqpoint{2.429870in}{1.581417in}}%
\pgfpathcurveto{\pgfqpoint{2.435694in}{1.575593in}}{\pgfqpoint{2.443594in}{1.572321in}}{\pgfqpoint{2.451830in}{1.572321in}}%
\pgfpathclose%
\pgfusepath{stroke,fill}%
\end{pgfscope}%
\begin{pgfscope}%
\pgfpathrectangle{\pgfqpoint{0.100000in}{0.212622in}}{\pgfqpoint{3.696000in}{3.696000in}}%
\pgfusepath{clip}%
\pgfsetbuttcap%
\pgfsetroundjoin%
\definecolor{currentfill}{rgb}{0.121569,0.466667,0.705882}%
\pgfsetfillcolor{currentfill}%
\pgfsetfillopacity{0.987761}%
\pgfsetlinewidth{1.003750pt}%
\definecolor{currentstroke}{rgb}{0.121569,0.466667,0.705882}%
\pgfsetstrokecolor{currentstroke}%
\pgfsetstrokeopacity{0.987761}%
\pgfsetdash{}{0pt}%
\pgfpathmoveto{\pgfqpoint{2.300391in}{1.607863in}}%
\pgfpathcurveto{\pgfqpoint{2.308627in}{1.607863in}}{\pgfqpoint{2.316527in}{1.611136in}}{\pgfqpoint{2.322351in}{1.616960in}}%
\pgfpathcurveto{\pgfqpoint{2.328175in}{1.622783in}}{\pgfqpoint{2.331447in}{1.630683in}}{\pgfqpoint{2.331447in}{1.638920in}}%
\pgfpathcurveto{\pgfqpoint{2.331447in}{1.647156in}}{\pgfqpoint{2.328175in}{1.655056in}}{\pgfqpoint{2.322351in}{1.660880in}}%
\pgfpathcurveto{\pgfqpoint{2.316527in}{1.666704in}}{\pgfqpoint{2.308627in}{1.669976in}}{\pgfqpoint{2.300391in}{1.669976in}}%
\pgfpathcurveto{\pgfqpoint{2.292155in}{1.669976in}}{\pgfqpoint{2.284255in}{1.666704in}}{\pgfqpoint{2.278431in}{1.660880in}}%
\pgfpathcurveto{\pgfqpoint{2.272607in}{1.655056in}}{\pgfqpoint{2.269334in}{1.647156in}}{\pgfqpoint{2.269334in}{1.638920in}}%
\pgfpathcurveto{\pgfqpoint{2.269334in}{1.630683in}}{\pgfqpoint{2.272607in}{1.622783in}}{\pgfqpoint{2.278431in}{1.616960in}}%
\pgfpathcurveto{\pgfqpoint{2.284255in}{1.611136in}}{\pgfqpoint{2.292155in}{1.607863in}}{\pgfqpoint{2.300391in}{1.607863in}}%
\pgfpathclose%
\pgfusepath{stroke,fill}%
\end{pgfscope}%
\begin{pgfscope}%
\pgfpathrectangle{\pgfqpoint{0.100000in}{0.212622in}}{\pgfqpoint{3.696000in}{3.696000in}}%
\pgfusepath{clip}%
\pgfsetbuttcap%
\pgfsetroundjoin%
\definecolor{currentfill}{rgb}{0.121569,0.466667,0.705882}%
\pgfsetfillcolor{currentfill}%
\pgfsetfillopacity{0.987833}%
\pgfsetlinewidth{1.003750pt}%
\definecolor{currentstroke}{rgb}{0.121569,0.466667,0.705882}%
\pgfsetstrokecolor{currentstroke}%
\pgfsetstrokeopacity{0.987833}%
\pgfsetdash{}{0pt}%
\pgfpathmoveto{\pgfqpoint{2.305076in}{1.603237in}}%
\pgfpathcurveto{\pgfqpoint{2.313312in}{1.603237in}}{\pgfqpoint{2.321212in}{1.606509in}}{\pgfqpoint{2.327036in}{1.612333in}}%
\pgfpathcurveto{\pgfqpoint{2.332860in}{1.618157in}}{\pgfqpoint{2.336133in}{1.626057in}}{\pgfqpoint{2.336133in}{1.634293in}}%
\pgfpathcurveto{\pgfqpoint{2.336133in}{1.642530in}}{\pgfqpoint{2.332860in}{1.650430in}}{\pgfqpoint{2.327036in}{1.656254in}}%
\pgfpathcurveto{\pgfqpoint{2.321212in}{1.662078in}}{\pgfqpoint{2.313312in}{1.665350in}}{\pgfqpoint{2.305076in}{1.665350in}}%
\pgfpathcurveto{\pgfqpoint{2.296840in}{1.665350in}}{\pgfqpoint{2.288940in}{1.662078in}}{\pgfqpoint{2.283116in}{1.656254in}}%
\pgfpathcurveto{\pgfqpoint{2.277292in}{1.650430in}}{\pgfqpoint{2.274020in}{1.642530in}}{\pgfqpoint{2.274020in}{1.634293in}}%
\pgfpathcurveto{\pgfqpoint{2.274020in}{1.626057in}}{\pgfqpoint{2.277292in}{1.618157in}}{\pgfqpoint{2.283116in}{1.612333in}}%
\pgfpathcurveto{\pgfqpoint{2.288940in}{1.606509in}}{\pgfqpoint{2.296840in}{1.603237in}}{\pgfqpoint{2.305076in}{1.603237in}}%
\pgfpathclose%
\pgfusepath{stroke,fill}%
\end{pgfscope}%
\begin{pgfscope}%
\pgfpathrectangle{\pgfqpoint{0.100000in}{0.212622in}}{\pgfqpoint{3.696000in}{3.696000in}}%
\pgfusepath{clip}%
\pgfsetbuttcap%
\pgfsetroundjoin%
\definecolor{currentfill}{rgb}{0.121569,0.466667,0.705882}%
\pgfsetfillcolor{currentfill}%
\pgfsetfillopacity{0.988352}%
\pgfsetlinewidth{1.003750pt}%
\definecolor{currentstroke}{rgb}{0.121569,0.466667,0.705882}%
\pgfsetstrokecolor{currentstroke}%
\pgfsetstrokeopacity{0.988352}%
\pgfsetdash{}{0pt}%
\pgfpathmoveto{\pgfqpoint{2.449664in}{1.570463in}}%
\pgfpathcurveto{\pgfqpoint{2.457900in}{1.570463in}}{\pgfqpoint{2.465800in}{1.573735in}}{\pgfqpoint{2.471624in}{1.579559in}}%
\pgfpathcurveto{\pgfqpoint{2.477448in}{1.585383in}}{\pgfqpoint{2.480720in}{1.593283in}}{\pgfqpoint{2.480720in}{1.601519in}}%
\pgfpathcurveto{\pgfqpoint{2.480720in}{1.609756in}}{\pgfqpoint{2.477448in}{1.617656in}}{\pgfqpoint{2.471624in}{1.623479in}}%
\pgfpathcurveto{\pgfqpoint{2.465800in}{1.629303in}}{\pgfqpoint{2.457900in}{1.632576in}}{\pgfqpoint{2.449664in}{1.632576in}}%
\pgfpathcurveto{\pgfqpoint{2.441427in}{1.632576in}}{\pgfqpoint{2.433527in}{1.629303in}}{\pgfqpoint{2.427704in}{1.623479in}}%
\pgfpathcurveto{\pgfqpoint{2.421880in}{1.617656in}}{\pgfqpoint{2.418607in}{1.609756in}}{\pgfqpoint{2.418607in}{1.601519in}}%
\pgfpathcurveto{\pgfqpoint{2.418607in}{1.593283in}}{\pgfqpoint{2.421880in}{1.585383in}}{\pgfqpoint{2.427704in}{1.579559in}}%
\pgfpathcurveto{\pgfqpoint{2.433527in}{1.573735in}}{\pgfqpoint{2.441427in}{1.570463in}}{\pgfqpoint{2.449664in}{1.570463in}}%
\pgfpathclose%
\pgfusepath{stroke,fill}%
\end{pgfscope}%
\begin{pgfscope}%
\pgfpathrectangle{\pgfqpoint{0.100000in}{0.212622in}}{\pgfqpoint{3.696000in}{3.696000in}}%
\pgfusepath{clip}%
\pgfsetbuttcap%
\pgfsetroundjoin%
\definecolor{currentfill}{rgb}{0.121569,0.466667,0.705882}%
\pgfsetfillcolor{currentfill}%
\pgfsetfillopacity{0.988398}%
\pgfsetlinewidth{1.003750pt}%
\definecolor{currentstroke}{rgb}{0.121569,0.466667,0.705882}%
\pgfsetstrokecolor{currentstroke}%
\pgfsetstrokeopacity{0.988398}%
\pgfsetdash{}{0pt}%
\pgfpathmoveto{\pgfqpoint{2.309397in}{1.603086in}}%
\pgfpathcurveto{\pgfqpoint{2.317634in}{1.603086in}}{\pgfqpoint{2.325534in}{1.606358in}}{\pgfqpoint{2.331358in}{1.612182in}}%
\pgfpathcurveto{\pgfqpoint{2.337181in}{1.618006in}}{\pgfqpoint{2.340454in}{1.625906in}}{\pgfqpoint{2.340454in}{1.634143in}}%
\pgfpathcurveto{\pgfqpoint{2.340454in}{1.642379in}}{\pgfqpoint{2.337181in}{1.650279in}}{\pgfqpoint{2.331358in}{1.656103in}}%
\pgfpathcurveto{\pgfqpoint{2.325534in}{1.661927in}}{\pgfqpoint{2.317634in}{1.665199in}}{\pgfqpoint{2.309397in}{1.665199in}}%
\pgfpathcurveto{\pgfqpoint{2.301161in}{1.665199in}}{\pgfqpoint{2.293261in}{1.661927in}}{\pgfqpoint{2.287437in}{1.656103in}}%
\pgfpathcurveto{\pgfqpoint{2.281613in}{1.650279in}}{\pgfqpoint{2.278341in}{1.642379in}}{\pgfqpoint{2.278341in}{1.634143in}}%
\pgfpathcurveto{\pgfqpoint{2.278341in}{1.625906in}}{\pgfqpoint{2.281613in}{1.618006in}}{\pgfqpoint{2.287437in}{1.612182in}}%
\pgfpathcurveto{\pgfqpoint{2.293261in}{1.606358in}}{\pgfqpoint{2.301161in}{1.603086in}}{\pgfqpoint{2.309397in}{1.603086in}}%
\pgfpathclose%
\pgfusepath{stroke,fill}%
\end{pgfscope}%
\begin{pgfscope}%
\pgfpathrectangle{\pgfqpoint{0.100000in}{0.212622in}}{\pgfqpoint{3.696000in}{3.696000in}}%
\pgfusepath{clip}%
\pgfsetbuttcap%
\pgfsetroundjoin%
\definecolor{currentfill}{rgb}{0.121569,0.466667,0.705882}%
\pgfsetfillcolor{currentfill}%
\pgfsetfillopacity{0.988737}%
\pgfsetlinewidth{1.003750pt}%
\definecolor{currentstroke}{rgb}{0.121569,0.466667,0.705882}%
\pgfsetstrokecolor{currentstroke}%
\pgfsetstrokeopacity{0.988737}%
\pgfsetdash{}{0pt}%
\pgfpathmoveto{\pgfqpoint{2.313365in}{1.601696in}}%
\pgfpathcurveto{\pgfqpoint{2.321602in}{1.601696in}}{\pgfqpoint{2.329502in}{1.604968in}}{\pgfqpoint{2.335326in}{1.610792in}}%
\pgfpathcurveto{\pgfqpoint{2.341149in}{1.616616in}}{\pgfqpoint{2.344422in}{1.624516in}}{\pgfqpoint{2.344422in}{1.632752in}}%
\pgfpathcurveto{\pgfqpoint{2.344422in}{1.640988in}}{\pgfqpoint{2.341149in}{1.648888in}}{\pgfqpoint{2.335326in}{1.654712in}}%
\pgfpathcurveto{\pgfqpoint{2.329502in}{1.660536in}}{\pgfqpoint{2.321602in}{1.663809in}}{\pgfqpoint{2.313365in}{1.663809in}}%
\pgfpathcurveto{\pgfqpoint{2.305129in}{1.663809in}}{\pgfqpoint{2.297229in}{1.660536in}}{\pgfqpoint{2.291405in}{1.654712in}}%
\pgfpathcurveto{\pgfqpoint{2.285581in}{1.648888in}}{\pgfqpoint{2.282309in}{1.640988in}}{\pgfqpoint{2.282309in}{1.632752in}}%
\pgfpathcurveto{\pgfqpoint{2.282309in}{1.624516in}}{\pgfqpoint{2.285581in}{1.616616in}}{\pgfqpoint{2.291405in}{1.610792in}}%
\pgfpathcurveto{\pgfqpoint{2.297229in}{1.604968in}}{\pgfqpoint{2.305129in}{1.601696in}}{\pgfqpoint{2.313365in}{1.601696in}}%
\pgfpathclose%
\pgfusepath{stroke,fill}%
\end{pgfscope}%
\begin{pgfscope}%
\pgfpathrectangle{\pgfqpoint{0.100000in}{0.212622in}}{\pgfqpoint{3.696000in}{3.696000in}}%
\pgfusepath{clip}%
\pgfsetbuttcap%
\pgfsetroundjoin%
\definecolor{currentfill}{rgb}{0.121569,0.466667,0.705882}%
\pgfsetfillcolor{currentfill}%
\pgfsetfillopacity{0.988862}%
\pgfsetlinewidth{1.003750pt}%
\definecolor{currentstroke}{rgb}{0.121569,0.466667,0.705882}%
\pgfsetstrokecolor{currentstroke}%
\pgfsetstrokeopacity{0.988862}%
\pgfsetdash{}{0pt}%
\pgfpathmoveto{\pgfqpoint{2.448605in}{1.569541in}}%
\pgfpathcurveto{\pgfqpoint{2.456841in}{1.569541in}}{\pgfqpoint{2.464741in}{1.572813in}}{\pgfqpoint{2.470565in}{1.578637in}}%
\pgfpathcurveto{\pgfqpoint{2.476389in}{1.584461in}}{\pgfqpoint{2.479662in}{1.592361in}}{\pgfqpoint{2.479662in}{1.600597in}}%
\pgfpathcurveto{\pgfqpoint{2.479662in}{1.608834in}}{\pgfqpoint{2.476389in}{1.616734in}}{\pgfqpoint{2.470565in}{1.622558in}}%
\pgfpathcurveto{\pgfqpoint{2.464741in}{1.628382in}}{\pgfqpoint{2.456841in}{1.631654in}}{\pgfqpoint{2.448605in}{1.631654in}}%
\pgfpathcurveto{\pgfqpoint{2.440369in}{1.631654in}}{\pgfqpoint{2.432469in}{1.628382in}}{\pgfqpoint{2.426645in}{1.622558in}}%
\pgfpathcurveto{\pgfqpoint{2.420821in}{1.616734in}}{\pgfqpoint{2.417549in}{1.608834in}}{\pgfqpoint{2.417549in}{1.600597in}}%
\pgfpathcurveto{\pgfqpoint{2.417549in}{1.592361in}}{\pgfqpoint{2.420821in}{1.584461in}}{\pgfqpoint{2.426645in}{1.578637in}}%
\pgfpathcurveto{\pgfqpoint{2.432469in}{1.572813in}}{\pgfqpoint{2.440369in}{1.569541in}}{\pgfqpoint{2.448605in}{1.569541in}}%
\pgfpathclose%
\pgfusepath{stroke,fill}%
\end{pgfscope}%
\begin{pgfscope}%
\pgfpathrectangle{\pgfqpoint{0.100000in}{0.212622in}}{\pgfqpoint{3.696000in}{3.696000in}}%
\pgfusepath{clip}%
\pgfsetbuttcap%
\pgfsetroundjoin%
\definecolor{currentfill}{rgb}{0.121569,0.466667,0.705882}%
\pgfsetfillcolor{currentfill}%
\pgfsetfillopacity{0.988878}%
\pgfsetlinewidth{1.003750pt}%
\definecolor{currentstroke}{rgb}{0.121569,0.466667,0.705882}%
\pgfsetstrokecolor{currentstroke}%
\pgfsetstrokeopacity{0.988878}%
\pgfsetdash{}{0pt}%
\pgfpathmoveto{\pgfqpoint{2.315844in}{1.599996in}}%
\pgfpathcurveto{\pgfqpoint{2.324080in}{1.599996in}}{\pgfqpoint{2.331980in}{1.603268in}}{\pgfqpoint{2.337804in}{1.609092in}}%
\pgfpathcurveto{\pgfqpoint{2.343628in}{1.614916in}}{\pgfqpoint{2.346900in}{1.622816in}}{\pgfqpoint{2.346900in}{1.631052in}}%
\pgfpathcurveto{\pgfqpoint{2.346900in}{1.639288in}}{\pgfqpoint{2.343628in}{1.647188in}}{\pgfqpoint{2.337804in}{1.653012in}}%
\pgfpathcurveto{\pgfqpoint{2.331980in}{1.658836in}}{\pgfqpoint{2.324080in}{1.662109in}}{\pgfqpoint{2.315844in}{1.662109in}}%
\pgfpathcurveto{\pgfqpoint{2.307608in}{1.662109in}}{\pgfqpoint{2.299708in}{1.658836in}}{\pgfqpoint{2.293884in}{1.653012in}}%
\pgfpathcurveto{\pgfqpoint{2.288060in}{1.647188in}}{\pgfqpoint{2.284787in}{1.639288in}}{\pgfqpoint{2.284787in}{1.631052in}}%
\pgfpathcurveto{\pgfqpoint{2.284787in}{1.622816in}}{\pgfqpoint{2.288060in}{1.614916in}}{\pgfqpoint{2.293884in}{1.609092in}}%
\pgfpathcurveto{\pgfqpoint{2.299708in}{1.603268in}}{\pgfqpoint{2.307608in}{1.599996in}}{\pgfqpoint{2.315844in}{1.599996in}}%
\pgfpathclose%
\pgfusepath{stroke,fill}%
\end{pgfscope}%
\begin{pgfscope}%
\pgfpathrectangle{\pgfqpoint{0.100000in}{0.212622in}}{\pgfqpoint{3.696000in}{3.696000in}}%
\pgfusepath{clip}%
\pgfsetbuttcap%
\pgfsetroundjoin%
\definecolor{currentfill}{rgb}{0.121569,0.466667,0.705882}%
\pgfsetfillcolor{currentfill}%
\pgfsetfillopacity{0.989254}%
\pgfsetlinewidth{1.003750pt}%
\definecolor{currentstroke}{rgb}{0.121569,0.466667,0.705882}%
\pgfsetstrokecolor{currentstroke}%
\pgfsetstrokeopacity{0.989254}%
\pgfsetdash{}{0pt}%
\pgfpathmoveto{\pgfqpoint{2.320122in}{1.596880in}}%
\pgfpathcurveto{\pgfqpoint{2.328358in}{1.596880in}}{\pgfqpoint{2.336258in}{1.600152in}}{\pgfqpoint{2.342082in}{1.605976in}}%
\pgfpathcurveto{\pgfqpoint{2.347906in}{1.611800in}}{\pgfqpoint{2.351178in}{1.619700in}}{\pgfqpoint{2.351178in}{1.627936in}}%
\pgfpathcurveto{\pgfqpoint{2.351178in}{1.636173in}}{\pgfqpoint{2.347906in}{1.644073in}}{\pgfqpoint{2.342082in}{1.649897in}}%
\pgfpathcurveto{\pgfqpoint{2.336258in}{1.655721in}}{\pgfqpoint{2.328358in}{1.658993in}}{\pgfqpoint{2.320122in}{1.658993in}}%
\pgfpathcurveto{\pgfqpoint{2.311885in}{1.658993in}}{\pgfqpoint{2.303985in}{1.655721in}}{\pgfqpoint{2.298161in}{1.649897in}}%
\pgfpathcurveto{\pgfqpoint{2.292338in}{1.644073in}}{\pgfqpoint{2.289065in}{1.636173in}}{\pgfqpoint{2.289065in}{1.627936in}}%
\pgfpathcurveto{\pgfqpoint{2.289065in}{1.619700in}}{\pgfqpoint{2.292338in}{1.611800in}}{\pgfqpoint{2.298161in}{1.605976in}}%
\pgfpathcurveto{\pgfqpoint{2.303985in}{1.600152in}}{\pgfqpoint{2.311885in}{1.596880in}}{\pgfqpoint{2.320122in}{1.596880in}}%
\pgfpathclose%
\pgfusepath{stroke,fill}%
\end{pgfscope}%
\begin{pgfscope}%
\pgfpathrectangle{\pgfqpoint{0.100000in}{0.212622in}}{\pgfqpoint{3.696000in}{3.696000in}}%
\pgfusepath{clip}%
\pgfsetbuttcap%
\pgfsetroundjoin%
\definecolor{currentfill}{rgb}{0.121569,0.466667,0.705882}%
\pgfsetfillcolor{currentfill}%
\pgfsetfillopacity{0.989469}%
\pgfsetlinewidth{1.003750pt}%
\definecolor{currentstroke}{rgb}{0.121569,0.466667,0.705882}%
\pgfsetstrokecolor{currentstroke}%
\pgfsetstrokeopacity{0.989469}%
\pgfsetdash{}{0pt}%
\pgfpathmoveto{\pgfqpoint{2.447087in}{1.568839in}}%
\pgfpathcurveto{\pgfqpoint{2.455323in}{1.568839in}}{\pgfqpoint{2.463223in}{1.572111in}}{\pgfqpoint{2.469047in}{1.577935in}}%
\pgfpathcurveto{\pgfqpoint{2.474871in}{1.583759in}}{\pgfqpoint{2.478144in}{1.591659in}}{\pgfqpoint{2.478144in}{1.599895in}}%
\pgfpathcurveto{\pgfqpoint{2.478144in}{1.608131in}}{\pgfqpoint{2.474871in}{1.616031in}}{\pgfqpoint{2.469047in}{1.621855in}}%
\pgfpathcurveto{\pgfqpoint{2.463223in}{1.627679in}}{\pgfqpoint{2.455323in}{1.630952in}}{\pgfqpoint{2.447087in}{1.630952in}}%
\pgfpathcurveto{\pgfqpoint{2.438851in}{1.630952in}}{\pgfqpoint{2.430951in}{1.627679in}}{\pgfqpoint{2.425127in}{1.621855in}}%
\pgfpathcurveto{\pgfqpoint{2.419303in}{1.616031in}}{\pgfqpoint{2.416031in}{1.608131in}}{\pgfqpoint{2.416031in}{1.599895in}}%
\pgfpathcurveto{\pgfqpoint{2.416031in}{1.591659in}}{\pgfqpoint{2.419303in}{1.583759in}}{\pgfqpoint{2.425127in}{1.577935in}}%
\pgfpathcurveto{\pgfqpoint{2.430951in}{1.572111in}}{\pgfqpoint{2.438851in}{1.568839in}}{\pgfqpoint{2.447087in}{1.568839in}}%
\pgfpathclose%
\pgfusepath{stroke,fill}%
\end{pgfscope}%
\begin{pgfscope}%
\pgfpathrectangle{\pgfqpoint{0.100000in}{0.212622in}}{\pgfqpoint{3.696000in}{3.696000in}}%
\pgfusepath{clip}%
\pgfsetbuttcap%
\pgfsetroundjoin%
\definecolor{currentfill}{rgb}{0.121569,0.466667,0.705882}%
\pgfsetfillcolor{currentfill}%
\pgfsetfillopacity{0.990077}%
\pgfsetlinewidth{1.003750pt}%
\definecolor{currentstroke}{rgb}{0.121569,0.466667,0.705882}%
\pgfsetstrokecolor{currentstroke}%
\pgfsetstrokeopacity{0.990077}%
\pgfsetdash{}{0pt}%
\pgfpathmoveto{\pgfqpoint{2.445876in}{1.566704in}}%
\pgfpathcurveto{\pgfqpoint{2.454112in}{1.566704in}}{\pgfqpoint{2.462012in}{1.569976in}}{\pgfqpoint{2.467836in}{1.575800in}}%
\pgfpathcurveto{\pgfqpoint{2.473660in}{1.581624in}}{\pgfqpoint{2.476932in}{1.589524in}}{\pgfqpoint{2.476932in}{1.597760in}}%
\pgfpathcurveto{\pgfqpoint{2.476932in}{1.605996in}}{\pgfqpoint{2.473660in}{1.613896in}}{\pgfqpoint{2.467836in}{1.619720in}}%
\pgfpathcurveto{\pgfqpoint{2.462012in}{1.625544in}}{\pgfqpoint{2.454112in}{1.628817in}}{\pgfqpoint{2.445876in}{1.628817in}}%
\pgfpathcurveto{\pgfqpoint{2.437639in}{1.628817in}}{\pgfqpoint{2.429739in}{1.625544in}}{\pgfqpoint{2.423915in}{1.619720in}}%
\pgfpathcurveto{\pgfqpoint{2.418091in}{1.613896in}}{\pgfqpoint{2.414819in}{1.605996in}}{\pgfqpoint{2.414819in}{1.597760in}}%
\pgfpathcurveto{\pgfqpoint{2.414819in}{1.589524in}}{\pgfqpoint{2.418091in}{1.581624in}}{\pgfqpoint{2.423915in}{1.575800in}}%
\pgfpathcurveto{\pgfqpoint{2.429739in}{1.569976in}}{\pgfqpoint{2.437639in}{1.566704in}}{\pgfqpoint{2.445876in}{1.566704in}}%
\pgfpathclose%
\pgfusepath{stroke,fill}%
\end{pgfscope}%
\begin{pgfscope}%
\pgfpathrectangle{\pgfqpoint{0.100000in}{0.212622in}}{\pgfqpoint{3.696000in}{3.696000in}}%
\pgfusepath{clip}%
\pgfsetbuttcap%
\pgfsetroundjoin%
\definecolor{currentfill}{rgb}{0.121569,0.466667,0.705882}%
\pgfsetfillcolor{currentfill}%
\pgfsetfillopacity{0.990549}%
\pgfsetlinewidth{1.003750pt}%
\definecolor{currentstroke}{rgb}{0.121569,0.466667,0.705882}%
\pgfsetstrokecolor{currentstroke}%
\pgfsetstrokeopacity{0.990549}%
\pgfsetdash{}{0pt}%
\pgfpathmoveto{\pgfqpoint{2.445428in}{1.566279in}}%
\pgfpathcurveto{\pgfqpoint{2.453665in}{1.566279in}}{\pgfqpoint{2.461565in}{1.569552in}}{\pgfqpoint{2.467389in}{1.575376in}}%
\pgfpathcurveto{\pgfqpoint{2.473213in}{1.581200in}}{\pgfqpoint{2.476485in}{1.589100in}}{\pgfqpoint{2.476485in}{1.597336in}}%
\pgfpathcurveto{\pgfqpoint{2.476485in}{1.605572in}}{\pgfqpoint{2.473213in}{1.613472in}}{\pgfqpoint{2.467389in}{1.619296in}}%
\pgfpathcurveto{\pgfqpoint{2.461565in}{1.625120in}}{\pgfqpoint{2.453665in}{1.628392in}}{\pgfqpoint{2.445428in}{1.628392in}}%
\pgfpathcurveto{\pgfqpoint{2.437192in}{1.628392in}}{\pgfqpoint{2.429292in}{1.625120in}}{\pgfqpoint{2.423468in}{1.619296in}}%
\pgfpathcurveto{\pgfqpoint{2.417644in}{1.613472in}}{\pgfqpoint{2.414372in}{1.605572in}}{\pgfqpoint{2.414372in}{1.597336in}}%
\pgfpathcurveto{\pgfqpoint{2.414372in}{1.589100in}}{\pgfqpoint{2.417644in}{1.581200in}}{\pgfqpoint{2.423468in}{1.575376in}}%
\pgfpathcurveto{\pgfqpoint{2.429292in}{1.569552in}}{\pgfqpoint{2.437192in}{1.566279in}}{\pgfqpoint{2.445428in}{1.566279in}}%
\pgfpathclose%
\pgfusepath{stroke,fill}%
\end{pgfscope}%
\begin{pgfscope}%
\pgfpathrectangle{\pgfqpoint{0.100000in}{0.212622in}}{\pgfqpoint{3.696000in}{3.696000in}}%
\pgfusepath{clip}%
\pgfsetbuttcap%
\pgfsetroundjoin%
\definecolor{currentfill}{rgb}{0.121569,0.466667,0.705882}%
\pgfsetfillcolor{currentfill}%
\pgfsetfillopacity{0.990651}%
\pgfsetlinewidth{1.003750pt}%
\definecolor{currentstroke}{rgb}{0.121569,0.466667,0.705882}%
\pgfsetstrokecolor{currentstroke}%
\pgfsetstrokeopacity{0.990651}%
\pgfsetdash{}{0pt}%
\pgfpathmoveto{\pgfqpoint{2.328596in}{1.598184in}}%
\pgfpathcurveto{\pgfqpoint{2.336832in}{1.598184in}}{\pgfqpoint{2.344732in}{1.601456in}}{\pgfqpoint{2.350556in}{1.607280in}}%
\pgfpathcurveto{\pgfqpoint{2.356380in}{1.613104in}}{\pgfqpoint{2.359652in}{1.621004in}}{\pgfqpoint{2.359652in}{1.629240in}}%
\pgfpathcurveto{\pgfqpoint{2.359652in}{1.637476in}}{\pgfqpoint{2.356380in}{1.645376in}}{\pgfqpoint{2.350556in}{1.651200in}}%
\pgfpathcurveto{\pgfqpoint{2.344732in}{1.657024in}}{\pgfqpoint{2.336832in}{1.660297in}}{\pgfqpoint{2.328596in}{1.660297in}}%
\pgfpathcurveto{\pgfqpoint{2.320360in}{1.660297in}}{\pgfqpoint{2.312459in}{1.657024in}}{\pgfqpoint{2.306636in}{1.651200in}}%
\pgfpathcurveto{\pgfqpoint{2.300812in}{1.645376in}}{\pgfqpoint{2.297539in}{1.637476in}}{\pgfqpoint{2.297539in}{1.629240in}}%
\pgfpathcurveto{\pgfqpoint{2.297539in}{1.621004in}}{\pgfqpoint{2.300812in}{1.613104in}}{\pgfqpoint{2.306636in}{1.607280in}}%
\pgfpathcurveto{\pgfqpoint{2.312459in}{1.601456in}}{\pgfqpoint{2.320360in}{1.598184in}}{\pgfqpoint{2.328596in}{1.598184in}}%
\pgfpathclose%
\pgfusepath{stroke,fill}%
\end{pgfscope}%
\begin{pgfscope}%
\pgfpathrectangle{\pgfqpoint{0.100000in}{0.212622in}}{\pgfqpoint{3.696000in}{3.696000in}}%
\pgfusepath{clip}%
\pgfsetbuttcap%
\pgfsetroundjoin%
\definecolor{currentfill}{rgb}{0.121569,0.466667,0.705882}%
\pgfsetfillcolor{currentfill}%
\pgfsetfillopacity{0.990993}%
\pgfsetlinewidth{1.003750pt}%
\definecolor{currentstroke}{rgb}{0.121569,0.466667,0.705882}%
\pgfsetstrokecolor{currentstroke}%
\pgfsetstrokeopacity{0.990993}%
\pgfsetdash{}{0pt}%
\pgfpathmoveto{\pgfqpoint{2.335289in}{1.591932in}}%
\pgfpathcurveto{\pgfqpoint{2.343525in}{1.591932in}}{\pgfqpoint{2.351425in}{1.595205in}}{\pgfqpoint{2.357249in}{1.601029in}}%
\pgfpathcurveto{\pgfqpoint{2.363073in}{1.606853in}}{\pgfqpoint{2.366345in}{1.614753in}}{\pgfqpoint{2.366345in}{1.622989in}}%
\pgfpathcurveto{\pgfqpoint{2.366345in}{1.631225in}}{\pgfqpoint{2.363073in}{1.639125in}}{\pgfqpoint{2.357249in}{1.644949in}}%
\pgfpathcurveto{\pgfqpoint{2.351425in}{1.650773in}}{\pgfqpoint{2.343525in}{1.654045in}}{\pgfqpoint{2.335289in}{1.654045in}}%
\pgfpathcurveto{\pgfqpoint{2.327052in}{1.654045in}}{\pgfqpoint{2.319152in}{1.650773in}}{\pgfqpoint{2.313328in}{1.644949in}}%
\pgfpathcurveto{\pgfqpoint{2.307505in}{1.639125in}}{\pgfqpoint{2.304232in}{1.631225in}}{\pgfqpoint{2.304232in}{1.622989in}}%
\pgfpathcurveto{\pgfqpoint{2.304232in}{1.614753in}}{\pgfqpoint{2.307505in}{1.606853in}}{\pgfqpoint{2.313328in}{1.601029in}}%
\pgfpathcurveto{\pgfqpoint{2.319152in}{1.595205in}}{\pgfqpoint{2.327052in}{1.591932in}}{\pgfqpoint{2.335289in}{1.591932in}}%
\pgfpathclose%
\pgfusepath{stroke,fill}%
\end{pgfscope}%
\begin{pgfscope}%
\pgfpathrectangle{\pgfqpoint{0.100000in}{0.212622in}}{\pgfqpoint{3.696000in}{3.696000in}}%
\pgfusepath{clip}%
\pgfsetbuttcap%
\pgfsetroundjoin%
\definecolor{currentfill}{rgb}{0.121569,0.466667,0.705882}%
\pgfsetfillcolor{currentfill}%
\pgfsetfillopacity{0.991884}%
\pgfsetlinewidth{1.003750pt}%
\definecolor{currentstroke}{rgb}{0.121569,0.466667,0.705882}%
\pgfsetstrokecolor{currentstroke}%
\pgfsetstrokeopacity{0.991884}%
\pgfsetdash{}{0pt}%
\pgfpathmoveto{\pgfqpoint{2.442050in}{1.568065in}}%
\pgfpathcurveto{\pgfqpoint{2.450287in}{1.568065in}}{\pgfqpoint{2.458187in}{1.571338in}}{\pgfqpoint{2.464011in}{1.577162in}}%
\pgfpathcurveto{\pgfqpoint{2.469835in}{1.582986in}}{\pgfqpoint{2.473107in}{1.590886in}}{\pgfqpoint{2.473107in}{1.599122in}}%
\pgfpathcurveto{\pgfqpoint{2.473107in}{1.607358in}}{\pgfqpoint{2.469835in}{1.615258in}}{\pgfqpoint{2.464011in}{1.621082in}}%
\pgfpathcurveto{\pgfqpoint{2.458187in}{1.626906in}}{\pgfqpoint{2.450287in}{1.630178in}}{\pgfqpoint{2.442050in}{1.630178in}}%
\pgfpathcurveto{\pgfqpoint{2.433814in}{1.630178in}}{\pgfqpoint{2.425914in}{1.626906in}}{\pgfqpoint{2.420090in}{1.621082in}}%
\pgfpathcurveto{\pgfqpoint{2.414266in}{1.615258in}}{\pgfqpoint{2.410994in}{1.607358in}}{\pgfqpoint{2.410994in}{1.599122in}}%
\pgfpathcurveto{\pgfqpoint{2.410994in}{1.590886in}}{\pgfqpoint{2.414266in}{1.582986in}}{\pgfqpoint{2.420090in}{1.577162in}}%
\pgfpathcurveto{\pgfqpoint{2.425914in}{1.571338in}}{\pgfqpoint{2.433814in}{1.568065in}}{\pgfqpoint{2.442050in}{1.568065in}}%
\pgfpathclose%
\pgfusepath{stroke,fill}%
\end{pgfscope}%
\begin{pgfscope}%
\pgfpathrectangle{\pgfqpoint{0.100000in}{0.212622in}}{\pgfqpoint{3.696000in}{3.696000in}}%
\pgfusepath{clip}%
\pgfsetbuttcap%
\pgfsetroundjoin%
\definecolor{currentfill}{rgb}{0.121569,0.466667,0.705882}%
\pgfsetfillcolor{currentfill}%
\pgfsetfillopacity{0.991921}%
\pgfsetlinewidth{1.003750pt}%
\definecolor{currentstroke}{rgb}{0.121569,0.466667,0.705882}%
\pgfsetstrokecolor{currentstroke}%
\pgfsetstrokeopacity{0.991921}%
\pgfsetdash{}{0pt}%
\pgfpathmoveto{\pgfqpoint{2.341860in}{1.592062in}}%
\pgfpathcurveto{\pgfqpoint{2.350096in}{1.592062in}}{\pgfqpoint{2.357996in}{1.595335in}}{\pgfqpoint{2.363820in}{1.601159in}}%
\pgfpathcurveto{\pgfqpoint{2.369644in}{1.606982in}}{\pgfqpoint{2.372917in}{1.614883in}}{\pgfqpoint{2.372917in}{1.623119in}}%
\pgfpathcurveto{\pgfqpoint{2.372917in}{1.631355in}}{\pgfqpoint{2.369644in}{1.639255in}}{\pgfqpoint{2.363820in}{1.645079in}}%
\pgfpathcurveto{\pgfqpoint{2.357996in}{1.650903in}}{\pgfqpoint{2.350096in}{1.654175in}}{\pgfqpoint{2.341860in}{1.654175in}}%
\pgfpathcurveto{\pgfqpoint{2.333624in}{1.654175in}}{\pgfqpoint{2.325724in}{1.650903in}}{\pgfqpoint{2.319900in}{1.645079in}}%
\pgfpathcurveto{\pgfqpoint{2.314076in}{1.639255in}}{\pgfqpoint{2.310804in}{1.631355in}}{\pgfqpoint{2.310804in}{1.623119in}}%
\pgfpathcurveto{\pgfqpoint{2.310804in}{1.614883in}}{\pgfqpoint{2.314076in}{1.606982in}}{\pgfqpoint{2.319900in}{1.601159in}}%
\pgfpathcurveto{\pgfqpoint{2.325724in}{1.595335in}}{\pgfqpoint{2.333624in}{1.592062in}}{\pgfqpoint{2.341860in}{1.592062in}}%
\pgfpathclose%
\pgfusepath{stroke,fill}%
\end{pgfscope}%
\begin{pgfscope}%
\pgfpathrectangle{\pgfqpoint{0.100000in}{0.212622in}}{\pgfqpoint{3.696000in}{3.696000in}}%
\pgfusepath{clip}%
\pgfsetbuttcap%
\pgfsetroundjoin%
\definecolor{currentfill}{rgb}{0.121569,0.466667,0.705882}%
\pgfsetfillcolor{currentfill}%
\pgfsetfillopacity{0.993110}%
\pgfsetlinewidth{1.003750pt}%
\definecolor{currentstroke}{rgb}{0.121569,0.466667,0.705882}%
\pgfsetstrokecolor{currentstroke}%
\pgfsetstrokeopacity{0.993110}%
\pgfsetdash{}{0pt}%
\pgfpathmoveto{\pgfqpoint{2.440118in}{1.566154in}}%
\pgfpathcurveto{\pgfqpoint{2.448354in}{1.566154in}}{\pgfqpoint{2.456254in}{1.569426in}}{\pgfqpoint{2.462078in}{1.575250in}}%
\pgfpathcurveto{\pgfqpoint{2.467902in}{1.581074in}}{\pgfqpoint{2.471174in}{1.588974in}}{\pgfqpoint{2.471174in}{1.597211in}}%
\pgfpathcurveto{\pgfqpoint{2.471174in}{1.605447in}}{\pgfqpoint{2.467902in}{1.613347in}}{\pgfqpoint{2.462078in}{1.619171in}}%
\pgfpathcurveto{\pgfqpoint{2.456254in}{1.624995in}}{\pgfqpoint{2.448354in}{1.628267in}}{\pgfqpoint{2.440118in}{1.628267in}}%
\pgfpathcurveto{\pgfqpoint{2.431881in}{1.628267in}}{\pgfqpoint{2.423981in}{1.624995in}}{\pgfqpoint{2.418157in}{1.619171in}}%
\pgfpathcurveto{\pgfqpoint{2.412334in}{1.613347in}}{\pgfqpoint{2.409061in}{1.605447in}}{\pgfqpoint{2.409061in}{1.597211in}}%
\pgfpathcurveto{\pgfqpoint{2.409061in}{1.588974in}}{\pgfqpoint{2.412334in}{1.581074in}}{\pgfqpoint{2.418157in}{1.575250in}}%
\pgfpathcurveto{\pgfqpoint{2.423981in}{1.569426in}}{\pgfqpoint{2.431881in}{1.566154in}}{\pgfqpoint{2.440118in}{1.566154in}}%
\pgfpathclose%
\pgfusepath{stroke,fill}%
\end{pgfscope}%
\begin{pgfscope}%
\pgfpathrectangle{\pgfqpoint{0.100000in}{0.212622in}}{\pgfqpoint{3.696000in}{3.696000in}}%
\pgfusepath{clip}%
\pgfsetbuttcap%
\pgfsetroundjoin%
\definecolor{currentfill}{rgb}{0.121569,0.466667,0.705882}%
\pgfsetfillcolor{currentfill}%
\pgfsetfillopacity{0.993211}%
\pgfsetlinewidth{1.003750pt}%
\definecolor{currentstroke}{rgb}{0.121569,0.466667,0.705882}%
\pgfsetstrokecolor{currentstroke}%
\pgfsetstrokeopacity{0.993211}%
\pgfsetdash{}{0pt}%
\pgfpathmoveto{\pgfqpoint{2.352942in}{1.586580in}}%
\pgfpathcurveto{\pgfqpoint{2.361178in}{1.586580in}}{\pgfqpoint{2.369078in}{1.589852in}}{\pgfqpoint{2.374902in}{1.595676in}}%
\pgfpathcurveto{\pgfqpoint{2.380726in}{1.601500in}}{\pgfqpoint{2.383998in}{1.609400in}}{\pgfqpoint{2.383998in}{1.617637in}}%
\pgfpathcurveto{\pgfqpoint{2.383998in}{1.625873in}}{\pgfqpoint{2.380726in}{1.633773in}}{\pgfqpoint{2.374902in}{1.639597in}}%
\pgfpathcurveto{\pgfqpoint{2.369078in}{1.645421in}}{\pgfqpoint{2.361178in}{1.648693in}}{\pgfqpoint{2.352942in}{1.648693in}}%
\pgfpathcurveto{\pgfqpoint{2.344705in}{1.648693in}}{\pgfqpoint{2.336805in}{1.645421in}}{\pgfqpoint{2.330982in}{1.639597in}}%
\pgfpathcurveto{\pgfqpoint{2.325158in}{1.633773in}}{\pgfqpoint{2.321885in}{1.625873in}}{\pgfqpoint{2.321885in}{1.617637in}}%
\pgfpathcurveto{\pgfqpoint{2.321885in}{1.609400in}}{\pgfqpoint{2.325158in}{1.601500in}}{\pgfqpoint{2.330982in}{1.595676in}}%
\pgfpathcurveto{\pgfqpoint{2.336805in}{1.589852in}}{\pgfqpoint{2.344705in}{1.586580in}}{\pgfqpoint{2.352942in}{1.586580in}}%
\pgfpathclose%
\pgfusepath{stroke,fill}%
\end{pgfscope}%
\begin{pgfscope}%
\pgfpathrectangle{\pgfqpoint{0.100000in}{0.212622in}}{\pgfqpoint{3.696000in}{3.696000in}}%
\pgfusepath{clip}%
\pgfsetbuttcap%
\pgfsetroundjoin%
\definecolor{currentfill}{rgb}{0.121569,0.466667,0.705882}%
\pgfsetfillcolor{currentfill}%
\pgfsetfillopacity{0.994185}%
\pgfsetlinewidth{1.003750pt}%
\definecolor{currentstroke}{rgb}{0.121569,0.466667,0.705882}%
\pgfsetstrokecolor{currentstroke}%
\pgfsetstrokeopacity{0.994185}%
\pgfsetdash{}{0pt}%
\pgfpathmoveto{\pgfqpoint{2.363961in}{1.580298in}}%
\pgfpathcurveto{\pgfqpoint{2.372197in}{1.580298in}}{\pgfqpoint{2.380098in}{1.583570in}}{\pgfqpoint{2.385921in}{1.589394in}}%
\pgfpathcurveto{\pgfqpoint{2.391745in}{1.595218in}}{\pgfqpoint{2.395018in}{1.603118in}}{\pgfqpoint{2.395018in}{1.611354in}}%
\pgfpathcurveto{\pgfqpoint{2.395018in}{1.619591in}}{\pgfqpoint{2.391745in}{1.627491in}}{\pgfqpoint{2.385921in}{1.633315in}}%
\pgfpathcurveto{\pgfqpoint{2.380098in}{1.639138in}}{\pgfqpoint{2.372197in}{1.642411in}}{\pgfqpoint{2.363961in}{1.642411in}}%
\pgfpathcurveto{\pgfqpoint{2.355725in}{1.642411in}}{\pgfqpoint{2.347825in}{1.639138in}}{\pgfqpoint{2.342001in}{1.633315in}}%
\pgfpathcurveto{\pgfqpoint{2.336177in}{1.627491in}}{\pgfqpoint{2.332905in}{1.619591in}}{\pgfqpoint{2.332905in}{1.611354in}}%
\pgfpathcurveto{\pgfqpoint{2.332905in}{1.603118in}}{\pgfqpoint{2.336177in}{1.595218in}}{\pgfqpoint{2.342001in}{1.589394in}}%
\pgfpathcurveto{\pgfqpoint{2.347825in}{1.583570in}}{\pgfqpoint{2.355725in}{1.580298in}}{\pgfqpoint{2.363961in}{1.580298in}}%
\pgfpathclose%
\pgfusepath{stroke,fill}%
\end{pgfscope}%
\begin{pgfscope}%
\pgfpathrectangle{\pgfqpoint{0.100000in}{0.212622in}}{\pgfqpoint{3.696000in}{3.696000in}}%
\pgfusepath{clip}%
\pgfsetbuttcap%
\pgfsetroundjoin%
\definecolor{currentfill}{rgb}{0.121569,0.466667,0.705882}%
\pgfsetfillcolor{currentfill}%
\pgfsetfillopacity{0.994317}%
\pgfsetlinewidth{1.003750pt}%
\definecolor{currentstroke}{rgb}{0.121569,0.466667,0.705882}%
\pgfsetstrokecolor{currentstroke}%
\pgfsetstrokeopacity{0.994317}%
\pgfsetdash{}{0pt}%
\pgfpathmoveto{\pgfqpoint{2.437819in}{1.563709in}}%
\pgfpathcurveto{\pgfqpoint{2.446056in}{1.563709in}}{\pgfqpoint{2.453956in}{1.566981in}}{\pgfqpoint{2.459780in}{1.572805in}}%
\pgfpathcurveto{\pgfqpoint{2.465604in}{1.578629in}}{\pgfqpoint{2.468876in}{1.586529in}}{\pgfqpoint{2.468876in}{1.594765in}}%
\pgfpathcurveto{\pgfqpoint{2.468876in}{1.603002in}}{\pgfqpoint{2.465604in}{1.610902in}}{\pgfqpoint{2.459780in}{1.616726in}}%
\pgfpathcurveto{\pgfqpoint{2.453956in}{1.622550in}}{\pgfqpoint{2.446056in}{1.625822in}}{\pgfqpoint{2.437819in}{1.625822in}}%
\pgfpathcurveto{\pgfqpoint{2.429583in}{1.625822in}}{\pgfqpoint{2.421683in}{1.622550in}}{\pgfqpoint{2.415859in}{1.616726in}}%
\pgfpathcurveto{\pgfqpoint{2.410035in}{1.610902in}}{\pgfqpoint{2.406763in}{1.603002in}}{\pgfqpoint{2.406763in}{1.594765in}}%
\pgfpathcurveto{\pgfqpoint{2.406763in}{1.586529in}}{\pgfqpoint{2.410035in}{1.578629in}}{\pgfqpoint{2.415859in}{1.572805in}}%
\pgfpathcurveto{\pgfqpoint{2.421683in}{1.566981in}}{\pgfqpoint{2.429583in}{1.563709in}}{\pgfqpoint{2.437819in}{1.563709in}}%
\pgfpathclose%
\pgfusepath{stroke,fill}%
\end{pgfscope}%
\begin{pgfscope}%
\pgfpathrectangle{\pgfqpoint{0.100000in}{0.212622in}}{\pgfqpoint{3.696000in}{3.696000in}}%
\pgfusepath{clip}%
\pgfsetbuttcap%
\pgfsetroundjoin%
\definecolor{currentfill}{rgb}{0.121569,0.466667,0.705882}%
\pgfsetfillcolor{currentfill}%
\pgfsetfillopacity{0.995147}%
\pgfsetlinewidth{1.003750pt}%
\definecolor{currentstroke}{rgb}{0.121569,0.466667,0.705882}%
\pgfsetstrokecolor{currentstroke}%
\pgfsetstrokeopacity{0.995147}%
\pgfsetdash{}{0pt}%
\pgfpathmoveto{\pgfqpoint{2.373679in}{1.577862in}}%
\pgfpathcurveto{\pgfqpoint{2.381916in}{1.577862in}}{\pgfqpoint{2.389816in}{1.581134in}}{\pgfqpoint{2.395640in}{1.586958in}}%
\pgfpathcurveto{\pgfqpoint{2.401463in}{1.592782in}}{\pgfqpoint{2.404736in}{1.600682in}}{\pgfqpoint{2.404736in}{1.608919in}}%
\pgfpathcurveto{\pgfqpoint{2.404736in}{1.617155in}}{\pgfqpoint{2.401463in}{1.625055in}}{\pgfqpoint{2.395640in}{1.630879in}}%
\pgfpathcurveto{\pgfqpoint{2.389816in}{1.636703in}}{\pgfqpoint{2.381916in}{1.639975in}}{\pgfqpoint{2.373679in}{1.639975in}}%
\pgfpathcurveto{\pgfqpoint{2.365443in}{1.639975in}}{\pgfqpoint{2.357543in}{1.636703in}}{\pgfqpoint{2.351719in}{1.630879in}}%
\pgfpathcurveto{\pgfqpoint{2.345895in}{1.625055in}}{\pgfqpoint{2.342623in}{1.617155in}}{\pgfqpoint{2.342623in}{1.608919in}}%
\pgfpathcurveto{\pgfqpoint{2.342623in}{1.600682in}}{\pgfqpoint{2.345895in}{1.592782in}}{\pgfqpoint{2.351719in}{1.586958in}}%
\pgfpathcurveto{\pgfqpoint{2.357543in}{1.581134in}}{\pgfqpoint{2.365443in}{1.577862in}}{\pgfqpoint{2.373679in}{1.577862in}}%
\pgfpathclose%
\pgfusepath{stroke,fill}%
\end{pgfscope}%
\begin{pgfscope}%
\pgfpathrectangle{\pgfqpoint{0.100000in}{0.212622in}}{\pgfqpoint{3.696000in}{3.696000in}}%
\pgfusepath{clip}%
\pgfsetbuttcap%
\pgfsetroundjoin%
\definecolor{currentfill}{rgb}{0.121569,0.466667,0.705882}%
\pgfsetfillcolor{currentfill}%
\pgfsetfillopacity{0.995532}%
\pgfsetlinewidth{1.003750pt}%
\definecolor{currentstroke}{rgb}{0.121569,0.466667,0.705882}%
\pgfsetstrokecolor{currentstroke}%
\pgfsetstrokeopacity{0.995532}%
\pgfsetdash{}{0pt}%
\pgfpathmoveto{\pgfqpoint{2.379857in}{1.571869in}}%
\pgfpathcurveto{\pgfqpoint{2.388094in}{1.571869in}}{\pgfqpoint{2.395994in}{1.575141in}}{\pgfqpoint{2.401818in}{1.580965in}}%
\pgfpathcurveto{\pgfqpoint{2.407641in}{1.586789in}}{\pgfqpoint{2.410914in}{1.594689in}}{\pgfqpoint{2.410914in}{1.602925in}}%
\pgfpathcurveto{\pgfqpoint{2.410914in}{1.611161in}}{\pgfqpoint{2.407641in}{1.619061in}}{\pgfqpoint{2.401818in}{1.624885in}}%
\pgfpathcurveto{\pgfqpoint{2.395994in}{1.630709in}}{\pgfqpoint{2.388094in}{1.633982in}}{\pgfqpoint{2.379857in}{1.633982in}}%
\pgfpathcurveto{\pgfqpoint{2.371621in}{1.633982in}}{\pgfqpoint{2.363721in}{1.630709in}}{\pgfqpoint{2.357897in}{1.624885in}}%
\pgfpathcurveto{\pgfqpoint{2.352073in}{1.619061in}}{\pgfqpoint{2.348801in}{1.611161in}}{\pgfqpoint{2.348801in}{1.602925in}}%
\pgfpathcurveto{\pgfqpoint{2.348801in}{1.594689in}}{\pgfqpoint{2.352073in}{1.586789in}}{\pgfqpoint{2.357897in}{1.580965in}}%
\pgfpathcurveto{\pgfqpoint{2.363721in}{1.575141in}}{\pgfqpoint{2.371621in}{1.571869in}}{\pgfqpoint{2.379857in}{1.571869in}}%
\pgfpathclose%
\pgfusepath{stroke,fill}%
\end{pgfscope}%
\begin{pgfscope}%
\pgfpathrectangle{\pgfqpoint{0.100000in}{0.212622in}}{\pgfqpoint{3.696000in}{3.696000in}}%
\pgfusepath{clip}%
\pgfsetbuttcap%
\pgfsetroundjoin%
\definecolor{currentfill}{rgb}{0.121569,0.466667,0.705882}%
\pgfsetfillcolor{currentfill}%
\pgfsetfillopacity{0.995554}%
\pgfsetlinewidth{1.003750pt}%
\definecolor{currentstroke}{rgb}{0.121569,0.466667,0.705882}%
\pgfsetstrokecolor{currentstroke}%
\pgfsetstrokeopacity{0.995554}%
\pgfsetdash{}{0pt}%
\pgfpathmoveto{\pgfqpoint{2.434355in}{1.561489in}}%
\pgfpathcurveto{\pgfqpoint{2.442592in}{1.561489in}}{\pgfqpoint{2.450492in}{1.564761in}}{\pgfqpoint{2.456316in}{1.570585in}}%
\pgfpathcurveto{\pgfqpoint{2.462140in}{1.576409in}}{\pgfqpoint{2.465412in}{1.584309in}}{\pgfqpoint{2.465412in}{1.592545in}}%
\pgfpathcurveto{\pgfqpoint{2.465412in}{1.600781in}}{\pgfqpoint{2.462140in}{1.608681in}}{\pgfqpoint{2.456316in}{1.614505in}}%
\pgfpathcurveto{\pgfqpoint{2.450492in}{1.620329in}}{\pgfqpoint{2.442592in}{1.623602in}}{\pgfqpoint{2.434355in}{1.623602in}}%
\pgfpathcurveto{\pgfqpoint{2.426119in}{1.623602in}}{\pgfqpoint{2.418219in}{1.620329in}}{\pgfqpoint{2.412395in}{1.614505in}}%
\pgfpathcurveto{\pgfqpoint{2.406571in}{1.608681in}}{\pgfqpoint{2.403299in}{1.600781in}}{\pgfqpoint{2.403299in}{1.592545in}}%
\pgfpathcurveto{\pgfqpoint{2.403299in}{1.584309in}}{\pgfqpoint{2.406571in}{1.576409in}}{\pgfqpoint{2.412395in}{1.570585in}}%
\pgfpathcurveto{\pgfqpoint{2.418219in}{1.564761in}}{\pgfqpoint{2.426119in}{1.561489in}}{\pgfqpoint{2.434355in}{1.561489in}}%
\pgfpathclose%
\pgfusepath{stroke,fill}%
\end{pgfscope}%
\begin{pgfscope}%
\pgfpathrectangle{\pgfqpoint{0.100000in}{0.212622in}}{\pgfqpoint{3.696000in}{3.696000in}}%
\pgfusepath{clip}%
\pgfsetbuttcap%
\pgfsetroundjoin%
\definecolor{currentfill}{rgb}{0.121569,0.466667,0.705882}%
\pgfsetfillcolor{currentfill}%
\pgfsetfillopacity{0.996510}%
\pgfsetlinewidth{1.003750pt}%
\definecolor{currentstroke}{rgb}{0.121569,0.466667,0.705882}%
\pgfsetstrokecolor{currentstroke}%
\pgfsetstrokeopacity{0.996510}%
\pgfsetdash{}{0pt}%
\pgfpathmoveto{\pgfqpoint{2.385570in}{1.571154in}}%
\pgfpathcurveto{\pgfqpoint{2.393806in}{1.571154in}}{\pgfqpoint{2.401706in}{1.574426in}}{\pgfqpoint{2.407530in}{1.580250in}}%
\pgfpathcurveto{\pgfqpoint{2.413354in}{1.586074in}}{\pgfqpoint{2.416627in}{1.593974in}}{\pgfqpoint{2.416627in}{1.602210in}}%
\pgfpathcurveto{\pgfqpoint{2.416627in}{1.610447in}}{\pgfqpoint{2.413354in}{1.618347in}}{\pgfqpoint{2.407530in}{1.624171in}}%
\pgfpathcurveto{\pgfqpoint{2.401706in}{1.629995in}}{\pgfqpoint{2.393806in}{1.633267in}}{\pgfqpoint{2.385570in}{1.633267in}}%
\pgfpathcurveto{\pgfqpoint{2.377334in}{1.633267in}}{\pgfqpoint{2.369434in}{1.629995in}}{\pgfqpoint{2.363610in}{1.624171in}}%
\pgfpathcurveto{\pgfqpoint{2.357786in}{1.618347in}}{\pgfqpoint{2.354514in}{1.610447in}}{\pgfqpoint{2.354514in}{1.602210in}}%
\pgfpathcurveto{\pgfqpoint{2.354514in}{1.593974in}}{\pgfqpoint{2.357786in}{1.586074in}}{\pgfqpoint{2.363610in}{1.580250in}}%
\pgfpathcurveto{\pgfqpoint{2.369434in}{1.574426in}}{\pgfqpoint{2.377334in}{1.571154in}}{\pgfqpoint{2.385570in}{1.571154in}}%
\pgfpathclose%
\pgfusepath{stroke,fill}%
\end{pgfscope}%
\begin{pgfscope}%
\pgfpathrectangle{\pgfqpoint{0.100000in}{0.212622in}}{\pgfqpoint{3.696000in}{3.696000in}}%
\pgfusepath{clip}%
\pgfsetbuttcap%
\pgfsetroundjoin%
\definecolor{currentfill}{rgb}{0.121569,0.466667,0.705882}%
\pgfsetfillcolor{currentfill}%
\pgfsetfillopacity{0.996989}%
\pgfsetlinewidth{1.003750pt}%
\definecolor{currentstroke}{rgb}{0.121569,0.466667,0.705882}%
\pgfsetstrokecolor{currentstroke}%
\pgfsetstrokeopacity{0.996989}%
\pgfsetdash{}{0pt}%
\pgfpathmoveto{\pgfqpoint{2.431444in}{1.558502in}}%
\pgfpathcurveto{\pgfqpoint{2.439680in}{1.558502in}}{\pgfqpoint{2.447580in}{1.561774in}}{\pgfqpoint{2.453404in}{1.567598in}}%
\pgfpathcurveto{\pgfqpoint{2.459228in}{1.573422in}}{\pgfqpoint{2.462500in}{1.581322in}}{\pgfqpoint{2.462500in}{1.589559in}}%
\pgfpathcurveto{\pgfqpoint{2.462500in}{1.597795in}}{\pgfqpoint{2.459228in}{1.605695in}}{\pgfqpoint{2.453404in}{1.611519in}}%
\pgfpathcurveto{\pgfqpoint{2.447580in}{1.617343in}}{\pgfqpoint{2.439680in}{1.620615in}}{\pgfqpoint{2.431444in}{1.620615in}}%
\pgfpathcurveto{\pgfqpoint{2.423208in}{1.620615in}}{\pgfqpoint{2.415307in}{1.617343in}}{\pgfqpoint{2.409484in}{1.611519in}}%
\pgfpathcurveto{\pgfqpoint{2.403660in}{1.605695in}}{\pgfqpoint{2.400387in}{1.597795in}}{\pgfqpoint{2.400387in}{1.589559in}}%
\pgfpathcurveto{\pgfqpoint{2.400387in}{1.581322in}}{\pgfqpoint{2.403660in}{1.573422in}}{\pgfqpoint{2.409484in}{1.567598in}}%
\pgfpathcurveto{\pgfqpoint{2.415307in}{1.561774in}}{\pgfqpoint{2.423208in}{1.558502in}}{\pgfqpoint{2.431444in}{1.558502in}}%
\pgfpathclose%
\pgfusepath{stroke,fill}%
\end{pgfscope}%
\begin{pgfscope}%
\pgfpathrectangle{\pgfqpoint{0.100000in}{0.212622in}}{\pgfqpoint{3.696000in}{3.696000in}}%
\pgfusepath{clip}%
\pgfsetbuttcap%
\pgfsetroundjoin%
\definecolor{currentfill}{rgb}{0.121569,0.466667,0.705882}%
\pgfsetfillcolor{currentfill}%
\pgfsetfillopacity{0.997308}%
\pgfsetlinewidth{1.003750pt}%
\definecolor{currentstroke}{rgb}{0.121569,0.466667,0.705882}%
\pgfsetstrokecolor{currentstroke}%
\pgfsetstrokeopacity{0.997308}%
\pgfsetdash{}{0pt}%
\pgfpathmoveto{\pgfqpoint{2.391054in}{1.570066in}}%
\pgfpathcurveto{\pgfqpoint{2.399290in}{1.570066in}}{\pgfqpoint{2.407190in}{1.573338in}}{\pgfqpoint{2.413014in}{1.579162in}}%
\pgfpathcurveto{\pgfqpoint{2.418838in}{1.584986in}}{\pgfqpoint{2.422110in}{1.592886in}}{\pgfqpoint{2.422110in}{1.601122in}}%
\pgfpathcurveto{\pgfqpoint{2.422110in}{1.609359in}}{\pgfqpoint{2.418838in}{1.617259in}}{\pgfqpoint{2.413014in}{1.623083in}}%
\pgfpathcurveto{\pgfqpoint{2.407190in}{1.628907in}}{\pgfqpoint{2.399290in}{1.632179in}}{\pgfqpoint{2.391054in}{1.632179in}}%
\pgfpathcurveto{\pgfqpoint{2.382817in}{1.632179in}}{\pgfqpoint{2.374917in}{1.628907in}}{\pgfqpoint{2.369093in}{1.623083in}}%
\pgfpathcurveto{\pgfqpoint{2.363269in}{1.617259in}}{\pgfqpoint{2.359997in}{1.609359in}}{\pgfqpoint{2.359997in}{1.601122in}}%
\pgfpathcurveto{\pgfqpoint{2.359997in}{1.592886in}}{\pgfqpoint{2.363269in}{1.584986in}}{\pgfqpoint{2.369093in}{1.579162in}}%
\pgfpathcurveto{\pgfqpoint{2.374917in}{1.573338in}}{\pgfqpoint{2.382817in}{1.570066in}}{\pgfqpoint{2.391054in}{1.570066in}}%
\pgfpathclose%
\pgfusepath{stroke,fill}%
\end{pgfscope}%
\begin{pgfscope}%
\pgfpathrectangle{\pgfqpoint{0.100000in}{0.212622in}}{\pgfqpoint{3.696000in}{3.696000in}}%
\pgfusepath{clip}%
\pgfsetbuttcap%
\pgfsetroundjoin%
\definecolor{currentfill}{rgb}{0.121569,0.466667,0.705882}%
\pgfsetfillcolor{currentfill}%
\pgfsetfillopacity{0.997668}%
\pgfsetlinewidth{1.003750pt}%
\definecolor{currentstroke}{rgb}{0.121569,0.466667,0.705882}%
\pgfsetstrokecolor{currentstroke}%
\pgfsetstrokeopacity{0.997668}%
\pgfsetdash{}{0pt}%
\pgfpathmoveto{\pgfqpoint{2.395870in}{1.566059in}}%
\pgfpathcurveto{\pgfqpoint{2.404107in}{1.566059in}}{\pgfqpoint{2.412007in}{1.569331in}}{\pgfqpoint{2.417831in}{1.575155in}}%
\pgfpathcurveto{\pgfqpoint{2.423655in}{1.580979in}}{\pgfqpoint{2.426927in}{1.588879in}}{\pgfqpoint{2.426927in}{1.597115in}}%
\pgfpathcurveto{\pgfqpoint{2.426927in}{1.605351in}}{\pgfqpoint{2.423655in}{1.613252in}}{\pgfqpoint{2.417831in}{1.619075in}}%
\pgfpathcurveto{\pgfqpoint{2.412007in}{1.624899in}}{\pgfqpoint{2.404107in}{1.628172in}}{\pgfqpoint{2.395870in}{1.628172in}}%
\pgfpathcurveto{\pgfqpoint{2.387634in}{1.628172in}}{\pgfqpoint{2.379734in}{1.624899in}}{\pgfqpoint{2.373910in}{1.619075in}}%
\pgfpathcurveto{\pgfqpoint{2.368086in}{1.613252in}}{\pgfqpoint{2.364814in}{1.605351in}}{\pgfqpoint{2.364814in}{1.597115in}}%
\pgfpathcurveto{\pgfqpoint{2.364814in}{1.588879in}}{\pgfqpoint{2.368086in}{1.580979in}}{\pgfqpoint{2.373910in}{1.575155in}}%
\pgfpathcurveto{\pgfqpoint{2.379734in}{1.569331in}}{\pgfqpoint{2.387634in}{1.566059in}}{\pgfqpoint{2.395870in}{1.566059in}}%
\pgfpathclose%
\pgfusepath{stroke,fill}%
\end{pgfscope}%
\begin{pgfscope}%
\pgfpathrectangle{\pgfqpoint{0.100000in}{0.212622in}}{\pgfqpoint{3.696000in}{3.696000in}}%
\pgfusepath{clip}%
\pgfsetbuttcap%
\pgfsetroundjoin%
\definecolor{currentfill}{rgb}{0.121569,0.466667,0.705882}%
\pgfsetfillcolor{currentfill}%
\pgfsetfillopacity{0.997742}%
\pgfsetlinewidth{1.003750pt}%
\definecolor{currentstroke}{rgb}{0.121569,0.466667,0.705882}%
\pgfsetstrokecolor{currentstroke}%
\pgfsetstrokeopacity{0.997742}%
\pgfsetdash{}{0pt}%
\pgfpathmoveto{\pgfqpoint{2.429226in}{1.557382in}}%
\pgfpathcurveto{\pgfqpoint{2.437462in}{1.557382in}}{\pgfqpoint{2.445362in}{1.560655in}}{\pgfqpoint{2.451186in}{1.566478in}}%
\pgfpathcurveto{\pgfqpoint{2.457010in}{1.572302in}}{\pgfqpoint{2.460282in}{1.580202in}}{\pgfqpoint{2.460282in}{1.588439in}}%
\pgfpathcurveto{\pgfqpoint{2.460282in}{1.596675in}}{\pgfqpoint{2.457010in}{1.604575in}}{\pgfqpoint{2.451186in}{1.610399in}}%
\pgfpathcurveto{\pgfqpoint{2.445362in}{1.616223in}}{\pgfqpoint{2.437462in}{1.619495in}}{\pgfqpoint{2.429226in}{1.619495in}}%
\pgfpathcurveto{\pgfqpoint{2.420989in}{1.619495in}}{\pgfqpoint{2.413089in}{1.616223in}}{\pgfqpoint{2.407265in}{1.610399in}}%
\pgfpathcurveto{\pgfqpoint{2.401441in}{1.604575in}}{\pgfqpoint{2.398169in}{1.596675in}}{\pgfqpoint{2.398169in}{1.588439in}}%
\pgfpathcurveto{\pgfqpoint{2.398169in}{1.580202in}}{\pgfqpoint{2.401441in}{1.572302in}}{\pgfqpoint{2.407265in}{1.566478in}}%
\pgfpathcurveto{\pgfqpoint{2.413089in}{1.560655in}}{\pgfqpoint{2.420989in}{1.557382in}}{\pgfqpoint{2.429226in}{1.557382in}}%
\pgfpathclose%
\pgfusepath{stroke,fill}%
\end{pgfscope}%
\begin{pgfscope}%
\pgfpathrectangle{\pgfqpoint{0.100000in}{0.212622in}}{\pgfqpoint{3.696000in}{3.696000in}}%
\pgfusepath{clip}%
\pgfsetbuttcap%
\pgfsetroundjoin%
\definecolor{currentfill}{rgb}{0.121569,0.466667,0.705882}%
\pgfsetfillcolor{currentfill}%
\pgfsetfillopacity{0.998034}%
\pgfsetlinewidth{1.003750pt}%
\definecolor{currentstroke}{rgb}{0.121569,0.466667,0.705882}%
\pgfsetstrokecolor{currentstroke}%
\pgfsetstrokeopacity{0.998034}%
\pgfsetdash{}{0pt}%
\pgfpathmoveto{\pgfqpoint{2.398828in}{1.565794in}}%
\pgfpathcurveto{\pgfqpoint{2.407065in}{1.565794in}}{\pgfqpoint{2.414965in}{1.569066in}}{\pgfqpoint{2.420789in}{1.574890in}}%
\pgfpathcurveto{\pgfqpoint{2.426613in}{1.580714in}}{\pgfqpoint{2.429885in}{1.588614in}}{\pgfqpoint{2.429885in}{1.596850in}}%
\pgfpathcurveto{\pgfqpoint{2.429885in}{1.605087in}}{\pgfqpoint{2.426613in}{1.612987in}}{\pgfqpoint{2.420789in}{1.618811in}}%
\pgfpathcurveto{\pgfqpoint{2.414965in}{1.624635in}}{\pgfqpoint{2.407065in}{1.627907in}}{\pgfqpoint{2.398828in}{1.627907in}}%
\pgfpathcurveto{\pgfqpoint{2.390592in}{1.627907in}}{\pgfqpoint{2.382692in}{1.624635in}}{\pgfqpoint{2.376868in}{1.618811in}}%
\pgfpathcurveto{\pgfqpoint{2.371044in}{1.612987in}}{\pgfqpoint{2.367772in}{1.605087in}}{\pgfqpoint{2.367772in}{1.596850in}}%
\pgfpathcurveto{\pgfqpoint{2.367772in}{1.588614in}}{\pgfqpoint{2.371044in}{1.580714in}}{\pgfqpoint{2.376868in}{1.574890in}}%
\pgfpathcurveto{\pgfqpoint{2.382692in}{1.569066in}}{\pgfqpoint{2.390592in}{1.565794in}}{\pgfqpoint{2.398828in}{1.565794in}}%
\pgfpathclose%
\pgfusepath{stroke,fill}%
\end{pgfscope}%
\begin{pgfscope}%
\pgfpathrectangle{\pgfqpoint{0.100000in}{0.212622in}}{\pgfqpoint{3.696000in}{3.696000in}}%
\pgfusepath{clip}%
\pgfsetbuttcap%
\pgfsetroundjoin%
\definecolor{currentfill}{rgb}{0.121569,0.466667,0.705882}%
\pgfsetfillcolor{currentfill}%
\pgfsetfillopacity{0.998075}%
\pgfsetlinewidth{1.003750pt}%
\definecolor{currentstroke}{rgb}{0.121569,0.466667,0.705882}%
\pgfsetstrokecolor{currentstroke}%
\pgfsetstrokeopacity{0.998075}%
\pgfsetdash{}{0pt}%
\pgfpathmoveto{\pgfqpoint{2.400045in}{1.564269in}}%
\pgfpathcurveto{\pgfqpoint{2.408281in}{1.564269in}}{\pgfqpoint{2.416181in}{1.567542in}}{\pgfqpoint{2.422005in}{1.573366in}}%
\pgfpathcurveto{\pgfqpoint{2.427829in}{1.579190in}}{\pgfqpoint{2.431101in}{1.587090in}}{\pgfqpoint{2.431101in}{1.595326in}}%
\pgfpathcurveto{\pgfqpoint{2.431101in}{1.603562in}}{\pgfqpoint{2.427829in}{1.611462in}}{\pgfqpoint{2.422005in}{1.617286in}}%
\pgfpathcurveto{\pgfqpoint{2.416181in}{1.623110in}}{\pgfqpoint{2.408281in}{1.626382in}}{\pgfqpoint{2.400045in}{1.626382in}}%
\pgfpathcurveto{\pgfqpoint{2.391808in}{1.626382in}}{\pgfqpoint{2.383908in}{1.623110in}}{\pgfqpoint{2.378084in}{1.617286in}}%
\pgfpathcurveto{\pgfqpoint{2.372261in}{1.611462in}}{\pgfqpoint{2.368988in}{1.603562in}}{\pgfqpoint{2.368988in}{1.595326in}}%
\pgfpathcurveto{\pgfqpoint{2.368988in}{1.587090in}}{\pgfqpoint{2.372261in}{1.579190in}}{\pgfqpoint{2.378084in}{1.573366in}}%
\pgfpathcurveto{\pgfqpoint{2.383908in}{1.567542in}}{\pgfqpoint{2.391808in}{1.564269in}}{\pgfqpoint{2.400045in}{1.564269in}}%
\pgfpathclose%
\pgfusepath{stroke,fill}%
\end{pgfscope}%
\begin{pgfscope}%
\pgfpathrectangle{\pgfqpoint{0.100000in}{0.212622in}}{\pgfqpoint{3.696000in}{3.696000in}}%
\pgfusepath{clip}%
\pgfsetbuttcap%
\pgfsetroundjoin%
\definecolor{currentfill}{rgb}{0.121569,0.466667,0.705882}%
\pgfsetfillcolor{currentfill}%
\pgfsetfillopacity{0.998138}%
\pgfsetlinewidth{1.003750pt}%
\definecolor{currentstroke}{rgb}{0.121569,0.466667,0.705882}%
\pgfsetstrokecolor{currentstroke}%
\pgfsetstrokeopacity{0.998138}%
\pgfsetdash{}{0pt}%
\pgfpathmoveto{\pgfqpoint{2.400306in}{1.564464in}}%
\pgfpathcurveto{\pgfqpoint{2.408542in}{1.564464in}}{\pgfqpoint{2.416442in}{1.567736in}}{\pgfqpoint{2.422266in}{1.573560in}}%
\pgfpathcurveto{\pgfqpoint{2.428090in}{1.579384in}}{\pgfqpoint{2.431362in}{1.587284in}}{\pgfqpoint{2.431362in}{1.595521in}}%
\pgfpathcurveto{\pgfqpoint{2.431362in}{1.603757in}}{\pgfqpoint{2.428090in}{1.611657in}}{\pgfqpoint{2.422266in}{1.617481in}}%
\pgfpathcurveto{\pgfqpoint{2.416442in}{1.623305in}}{\pgfqpoint{2.408542in}{1.626577in}}{\pgfqpoint{2.400306in}{1.626577in}}%
\pgfpathcurveto{\pgfqpoint{2.392069in}{1.626577in}}{\pgfqpoint{2.384169in}{1.623305in}}{\pgfqpoint{2.378345in}{1.617481in}}%
\pgfpathcurveto{\pgfqpoint{2.372522in}{1.611657in}}{\pgfqpoint{2.369249in}{1.603757in}}{\pgfqpoint{2.369249in}{1.595521in}}%
\pgfpathcurveto{\pgfqpoint{2.369249in}{1.587284in}}{\pgfqpoint{2.372522in}{1.579384in}}{\pgfqpoint{2.378345in}{1.573560in}}%
\pgfpathcurveto{\pgfqpoint{2.384169in}{1.567736in}}{\pgfqpoint{2.392069in}{1.564464in}}{\pgfqpoint{2.400306in}{1.564464in}}%
\pgfpathclose%
\pgfusepath{stroke,fill}%
\end{pgfscope}%
\begin{pgfscope}%
\pgfpathrectangle{\pgfqpoint{0.100000in}{0.212622in}}{\pgfqpoint{3.696000in}{3.696000in}}%
\pgfusepath{clip}%
\pgfsetbuttcap%
\pgfsetroundjoin%
\definecolor{currentfill}{rgb}{0.121569,0.466667,0.705882}%
\pgfsetfillcolor{currentfill}%
\pgfsetfillopacity{0.998181}%
\pgfsetlinewidth{1.003750pt}%
\definecolor{currentstroke}{rgb}{0.121569,0.466667,0.705882}%
\pgfsetstrokecolor{currentstroke}%
\pgfsetstrokeopacity{0.998181}%
\pgfsetdash{}{0pt}%
\pgfpathmoveto{\pgfqpoint{2.400698in}{1.564093in}}%
\pgfpathcurveto{\pgfqpoint{2.408934in}{1.564093in}}{\pgfqpoint{2.416834in}{1.567365in}}{\pgfqpoint{2.422658in}{1.573189in}}%
\pgfpathcurveto{\pgfqpoint{2.428482in}{1.579013in}}{\pgfqpoint{2.431754in}{1.586913in}}{\pgfqpoint{2.431754in}{1.595150in}}%
\pgfpathcurveto{\pgfqpoint{2.431754in}{1.603386in}}{\pgfqpoint{2.428482in}{1.611286in}}{\pgfqpoint{2.422658in}{1.617110in}}%
\pgfpathcurveto{\pgfqpoint{2.416834in}{1.622934in}}{\pgfqpoint{2.408934in}{1.626206in}}{\pgfqpoint{2.400698in}{1.626206in}}%
\pgfpathcurveto{\pgfqpoint{2.392462in}{1.626206in}}{\pgfqpoint{2.384562in}{1.622934in}}{\pgfqpoint{2.378738in}{1.617110in}}%
\pgfpathcurveto{\pgfqpoint{2.372914in}{1.611286in}}{\pgfqpoint{2.369641in}{1.603386in}}{\pgfqpoint{2.369641in}{1.595150in}}%
\pgfpathcurveto{\pgfqpoint{2.369641in}{1.586913in}}{\pgfqpoint{2.372914in}{1.579013in}}{\pgfqpoint{2.378738in}{1.573189in}}%
\pgfpathcurveto{\pgfqpoint{2.384562in}{1.567365in}}{\pgfqpoint{2.392462in}{1.564093in}}{\pgfqpoint{2.400698in}{1.564093in}}%
\pgfpathclose%
\pgfusepath{stroke,fill}%
\end{pgfscope}%
\begin{pgfscope}%
\pgfpathrectangle{\pgfqpoint{0.100000in}{0.212622in}}{\pgfqpoint{3.696000in}{3.696000in}}%
\pgfusepath{clip}%
\pgfsetbuttcap%
\pgfsetroundjoin%
\definecolor{currentfill}{rgb}{0.121569,0.466667,0.705882}%
\pgfsetfillcolor{currentfill}%
\pgfsetfillopacity{0.998208}%
\pgfsetlinewidth{1.003750pt}%
\definecolor{currentstroke}{rgb}{0.121569,0.466667,0.705882}%
\pgfsetstrokecolor{currentstroke}%
\pgfsetstrokeopacity{0.998208}%
\pgfsetdash{}{0pt}%
\pgfpathmoveto{\pgfqpoint{2.400804in}{1.564136in}}%
\pgfpathcurveto{\pgfqpoint{2.409041in}{1.564136in}}{\pgfqpoint{2.416941in}{1.567409in}}{\pgfqpoint{2.422765in}{1.573232in}}%
\pgfpathcurveto{\pgfqpoint{2.428589in}{1.579056in}}{\pgfqpoint{2.431861in}{1.586956in}}{\pgfqpoint{2.431861in}{1.595193in}}%
\pgfpathcurveto{\pgfqpoint{2.431861in}{1.603429in}}{\pgfqpoint{2.428589in}{1.611329in}}{\pgfqpoint{2.422765in}{1.617153in}}%
\pgfpathcurveto{\pgfqpoint{2.416941in}{1.622977in}}{\pgfqpoint{2.409041in}{1.626249in}}{\pgfqpoint{2.400804in}{1.626249in}}%
\pgfpathcurveto{\pgfqpoint{2.392568in}{1.626249in}}{\pgfqpoint{2.384668in}{1.622977in}}{\pgfqpoint{2.378844in}{1.617153in}}%
\pgfpathcurveto{\pgfqpoint{2.373020in}{1.611329in}}{\pgfqpoint{2.369748in}{1.603429in}}{\pgfqpoint{2.369748in}{1.595193in}}%
\pgfpathcurveto{\pgfqpoint{2.369748in}{1.586956in}}{\pgfqpoint{2.373020in}{1.579056in}}{\pgfqpoint{2.378844in}{1.573232in}}%
\pgfpathcurveto{\pgfqpoint{2.384668in}{1.567409in}}{\pgfqpoint{2.392568in}{1.564136in}}{\pgfqpoint{2.400804in}{1.564136in}}%
\pgfpathclose%
\pgfusepath{stroke,fill}%
\end{pgfscope}%
\begin{pgfscope}%
\pgfpathrectangle{\pgfqpoint{0.100000in}{0.212622in}}{\pgfqpoint{3.696000in}{3.696000in}}%
\pgfusepath{clip}%
\pgfsetbuttcap%
\pgfsetroundjoin%
\definecolor{currentfill}{rgb}{0.121569,0.466667,0.705882}%
\pgfsetfillcolor{currentfill}%
\pgfsetfillopacity{0.998240}%
\pgfsetlinewidth{1.003750pt}%
\definecolor{currentstroke}{rgb}{0.121569,0.466667,0.705882}%
\pgfsetstrokecolor{currentstroke}%
\pgfsetstrokeopacity{0.998240}%
\pgfsetdash{}{0pt}%
\pgfpathmoveto{\pgfqpoint{2.400966in}{1.564011in}}%
\pgfpathcurveto{\pgfqpoint{2.409202in}{1.564011in}}{\pgfqpoint{2.417102in}{1.567284in}}{\pgfqpoint{2.422926in}{1.573107in}}%
\pgfpathcurveto{\pgfqpoint{2.428750in}{1.578931in}}{\pgfqpoint{2.432022in}{1.586831in}}{\pgfqpoint{2.432022in}{1.595068in}}%
\pgfpathcurveto{\pgfqpoint{2.432022in}{1.603304in}}{\pgfqpoint{2.428750in}{1.611204in}}{\pgfqpoint{2.422926in}{1.617028in}}%
\pgfpathcurveto{\pgfqpoint{2.417102in}{1.622852in}}{\pgfqpoint{2.409202in}{1.626124in}}{\pgfqpoint{2.400966in}{1.626124in}}%
\pgfpathcurveto{\pgfqpoint{2.392729in}{1.626124in}}{\pgfqpoint{2.384829in}{1.622852in}}{\pgfqpoint{2.379005in}{1.617028in}}%
\pgfpathcurveto{\pgfqpoint{2.373181in}{1.611204in}}{\pgfqpoint{2.369909in}{1.603304in}}{\pgfqpoint{2.369909in}{1.595068in}}%
\pgfpathcurveto{\pgfqpoint{2.369909in}{1.586831in}}{\pgfqpoint{2.373181in}{1.578931in}}{\pgfqpoint{2.379005in}{1.573107in}}%
\pgfpathcurveto{\pgfqpoint{2.384829in}{1.567284in}}{\pgfqpoint{2.392729in}{1.564011in}}{\pgfqpoint{2.400966in}{1.564011in}}%
\pgfpathclose%
\pgfusepath{stroke,fill}%
\end{pgfscope}%
\begin{pgfscope}%
\pgfpathrectangle{\pgfqpoint{0.100000in}{0.212622in}}{\pgfqpoint{3.696000in}{3.696000in}}%
\pgfusepath{clip}%
\pgfsetbuttcap%
\pgfsetroundjoin%
\definecolor{currentfill}{rgb}{0.121569,0.466667,0.705882}%
\pgfsetfillcolor{currentfill}%
\pgfsetfillopacity{0.998327}%
\pgfsetlinewidth{1.003750pt}%
\definecolor{currentstroke}{rgb}{0.121569,0.466667,0.705882}%
\pgfsetstrokecolor{currentstroke}%
\pgfsetstrokeopacity{0.998327}%
\pgfsetdash{}{0pt}%
\pgfpathmoveto{\pgfqpoint{2.401330in}{1.564159in}}%
\pgfpathcurveto{\pgfqpoint{2.409566in}{1.564159in}}{\pgfqpoint{2.417466in}{1.567432in}}{\pgfqpoint{2.423290in}{1.573256in}}%
\pgfpathcurveto{\pgfqpoint{2.429114in}{1.579080in}}{\pgfqpoint{2.432386in}{1.586980in}}{\pgfqpoint{2.432386in}{1.595216in}}%
\pgfpathcurveto{\pgfqpoint{2.432386in}{1.603452in}}{\pgfqpoint{2.429114in}{1.611352in}}{\pgfqpoint{2.423290in}{1.617176in}}%
\pgfpathcurveto{\pgfqpoint{2.417466in}{1.623000in}}{\pgfqpoint{2.409566in}{1.626272in}}{\pgfqpoint{2.401330in}{1.626272in}}%
\pgfpathcurveto{\pgfqpoint{2.393093in}{1.626272in}}{\pgfqpoint{2.385193in}{1.623000in}}{\pgfqpoint{2.379369in}{1.617176in}}%
\pgfpathcurveto{\pgfqpoint{2.373545in}{1.611352in}}{\pgfqpoint{2.370273in}{1.603452in}}{\pgfqpoint{2.370273in}{1.595216in}}%
\pgfpathcurveto{\pgfqpoint{2.370273in}{1.586980in}}{\pgfqpoint{2.373545in}{1.579080in}}{\pgfqpoint{2.379369in}{1.573256in}}%
\pgfpathcurveto{\pgfqpoint{2.385193in}{1.567432in}}{\pgfqpoint{2.393093in}{1.564159in}}{\pgfqpoint{2.401330in}{1.564159in}}%
\pgfpathclose%
\pgfusepath{stroke,fill}%
\end{pgfscope}%
\begin{pgfscope}%
\pgfpathrectangle{\pgfqpoint{0.100000in}{0.212622in}}{\pgfqpoint{3.696000in}{3.696000in}}%
\pgfusepath{clip}%
\pgfsetbuttcap%
\pgfsetroundjoin%
\definecolor{currentfill}{rgb}{0.121569,0.466667,0.705882}%
\pgfsetfillcolor{currentfill}%
\pgfsetfillopacity{0.998335}%
\pgfsetlinewidth{1.003750pt}%
\definecolor{currentstroke}{rgb}{0.121569,0.466667,0.705882}%
\pgfsetstrokecolor{currentstroke}%
\pgfsetstrokeopacity{0.998335}%
\pgfsetdash{}{0pt}%
\pgfpathmoveto{\pgfqpoint{2.401812in}{1.563057in}}%
\pgfpathcurveto{\pgfqpoint{2.410048in}{1.563057in}}{\pgfqpoint{2.417948in}{1.566329in}}{\pgfqpoint{2.423772in}{1.572153in}}%
\pgfpathcurveto{\pgfqpoint{2.429596in}{1.577977in}}{\pgfqpoint{2.432869in}{1.585877in}}{\pgfqpoint{2.432869in}{1.594114in}}%
\pgfpathcurveto{\pgfqpoint{2.432869in}{1.602350in}}{\pgfqpoint{2.429596in}{1.610250in}}{\pgfqpoint{2.423772in}{1.616074in}}%
\pgfpathcurveto{\pgfqpoint{2.417948in}{1.621898in}}{\pgfqpoint{2.410048in}{1.625170in}}{\pgfqpoint{2.401812in}{1.625170in}}%
\pgfpathcurveto{\pgfqpoint{2.393576in}{1.625170in}}{\pgfqpoint{2.385676in}{1.621898in}}{\pgfqpoint{2.379852in}{1.616074in}}%
\pgfpathcurveto{\pgfqpoint{2.374028in}{1.610250in}}{\pgfqpoint{2.370756in}{1.602350in}}{\pgfqpoint{2.370756in}{1.594114in}}%
\pgfpathcurveto{\pgfqpoint{2.370756in}{1.585877in}}{\pgfqpoint{2.374028in}{1.577977in}}{\pgfqpoint{2.379852in}{1.572153in}}%
\pgfpathcurveto{\pgfqpoint{2.385676in}{1.566329in}}{\pgfqpoint{2.393576in}{1.563057in}}{\pgfqpoint{2.401812in}{1.563057in}}%
\pgfpathclose%
\pgfusepath{stroke,fill}%
\end{pgfscope}%
\begin{pgfscope}%
\pgfpathrectangle{\pgfqpoint{0.100000in}{0.212622in}}{\pgfqpoint{3.696000in}{3.696000in}}%
\pgfusepath{clip}%
\pgfsetbuttcap%
\pgfsetroundjoin%
\definecolor{currentfill}{rgb}{0.121569,0.466667,0.705882}%
\pgfsetfillcolor{currentfill}%
\pgfsetfillopacity{0.998575}%
\pgfsetlinewidth{1.003750pt}%
\definecolor{currentstroke}{rgb}{0.121569,0.466667,0.705882}%
\pgfsetstrokecolor{currentstroke}%
\pgfsetstrokeopacity{0.998575}%
\pgfsetdash{}{0pt}%
\pgfpathmoveto{\pgfqpoint{2.403041in}{1.563340in}}%
\pgfpathcurveto{\pgfqpoint{2.411277in}{1.563340in}}{\pgfqpoint{2.419177in}{1.566613in}}{\pgfqpoint{2.425001in}{1.572437in}}%
\pgfpathcurveto{\pgfqpoint{2.430825in}{1.578261in}}{\pgfqpoint{2.434097in}{1.586161in}}{\pgfqpoint{2.434097in}{1.594397in}}%
\pgfpathcurveto{\pgfqpoint{2.434097in}{1.602633in}}{\pgfqpoint{2.430825in}{1.610533in}}{\pgfqpoint{2.425001in}{1.616357in}}%
\pgfpathcurveto{\pgfqpoint{2.419177in}{1.622181in}}{\pgfqpoint{2.411277in}{1.625453in}}{\pgfqpoint{2.403041in}{1.625453in}}%
\pgfpathcurveto{\pgfqpoint{2.394805in}{1.625453in}}{\pgfqpoint{2.386905in}{1.622181in}}{\pgfqpoint{2.381081in}{1.616357in}}%
\pgfpathcurveto{\pgfqpoint{2.375257in}{1.610533in}}{\pgfqpoint{2.371984in}{1.602633in}}{\pgfqpoint{2.371984in}{1.594397in}}%
\pgfpathcurveto{\pgfqpoint{2.371984in}{1.586161in}}{\pgfqpoint{2.375257in}{1.578261in}}{\pgfqpoint{2.381081in}{1.572437in}}%
\pgfpathcurveto{\pgfqpoint{2.386905in}{1.566613in}}{\pgfqpoint{2.394805in}{1.563340in}}{\pgfqpoint{2.403041in}{1.563340in}}%
\pgfpathclose%
\pgfusepath{stroke,fill}%
\end{pgfscope}%
\begin{pgfscope}%
\pgfpathrectangle{\pgfqpoint{0.100000in}{0.212622in}}{\pgfqpoint{3.696000in}{3.696000in}}%
\pgfusepath{clip}%
\pgfsetbuttcap%
\pgfsetroundjoin%
\definecolor{currentfill}{rgb}{0.121569,0.466667,0.705882}%
\pgfsetfillcolor{currentfill}%
\pgfsetfillopacity{0.998592}%
\pgfsetlinewidth{1.003750pt}%
\definecolor{currentstroke}{rgb}{0.121569,0.466667,0.705882}%
\pgfsetstrokecolor{currentstroke}%
\pgfsetstrokeopacity{0.998592}%
\pgfsetdash{}{0pt}%
\pgfpathmoveto{\pgfqpoint{2.427174in}{1.555536in}}%
\pgfpathcurveto{\pgfqpoint{2.435410in}{1.555536in}}{\pgfqpoint{2.443311in}{1.558808in}}{\pgfqpoint{2.449134in}{1.564632in}}%
\pgfpathcurveto{\pgfqpoint{2.454958in}{1.570456in}}{\pgfqpoint{2.458231in}{1.578356in}}{\pgfqpoint{2.458231in}{1.586592in}}%
\pgfpathcurveto{\pgfqpoint{2.458231in}{1.594829in}}{\pgfqpoint{2.454958in}{1.602729in}}{\pgfqpoint{2.449134in}{1.608553in}}%
\pgfpathcurveto{\pgfqpoint{2.443311in}{1.614377in}}{\pgfqpoint{2.435410in}{1.617649in}}{\pgfqpoint{2.427174in}{1.617649in}}%
\pgfpathcurveto{\pgfqpoint{2.418938in}{1.617649in}}{\pgfqpoint{2.411038in}{1.614377in}}{\pgfqpoint{2.405214in}{1.608553in}}%
\pgfpathcurveto{\pgfqpoint{2.399390in}{1.602729in}}{\pgfqpoint{2.396118in}{1.594829in}}{\pgfqpoint{2.396118in}{1.586592in}}%
\pgfpathcurveto{\pgfqpoint{2.396118in}{1.578356in}}{\pgfqpoint{2.399390in}{1.570456in}}{\pgfqpoint{2.405214in}{1.564632in}}%
\pgfpathcurveto{\pgfqpoint{2.411038in}{1.558808in}}{\pgfqpoint{2.418938in}{1.555536in}}{\pgfqpoint{2.427174in}{1.555536in}}%
\pgfpathclose%
\pgfusepath{stroke,fill}%
\end{pgfscope}%
\begin{pgfscope}%
\pgfpathrectangle{\pgfqpoint{0.100000in}{0.212622in}}{\pgfqpoint{3.696000in}{3.696000in}}%
\pgfusepath{clip}%
\pgfsetbuttcap%
\pgfsetroundjoin%
\definecolor{currentfill}{rgb}{0.121569,0.466667,0.705882}%
\pgfsetfillcolor{currentfill}%
\pgfsetfillopacity{0.998714}%
\pgfsetlinewidth{1.003750pt}%
\definecolor{currentstroke}{rgb}{0.121569,0.466667,0.705882}%
\pgfsetstrokecolor{currentstroke}%
\pgfsetstrokeopacity{0.998714}%
\pgfsetdash{}{0pt}%
\pgfpathmoveto{\pgfqpoint{2.404999in}{1.561150in}}%
\pgfpathcurveto{\pgfqpoint{2.413236in}{1.561150in}}{\pgfqpoint{2.421136in}{1.564422in}}{\pgfqpoint{2.426960in}{1.570246in}}%
\pgfpathcurveto{\pgfqpoint{2.432783in}{1.576070in}}{\pgfqpoint{2.436056in}{1.583970in}}{\pgfqpoint{2.436056in}{1.592206in}}%
\pgfpathcurveto{\pgfqpoint{2.436056in}{1.600443in}}{\pgfqpoint{2.432783in}{1.608343in}}{\pgfqpoint{2.426960in}{1.614167in}}%
\pgfpathcurveto{\pgfqpoint{2.421136in}{1.619991in}}{\pgfqpoint{2.413236in}{1.623263in}}{\pgfqpoint{2.404999in}{1.623263in}}%
\pgfpathcurveto{\pgfqpoint{2.396763in}{1.623263in}}{\pgfqpoint{2.388863in}{1.619991in}}{\pgfqpoint{2.383039in}{1.614167in}}%
\pgfpathcurveto{\pgfqpoint{2.377215in}{1.608343in}}{\pgfqpoint{2.373943in}{1.600443in}}{\pgfqpoint{2.373943in}{1.592206in}}%
\pgfpathcurveto{\pgfqpoint{2.373943in}{1.583970in}}{\pgfqpoint{2.377215in}{1.576070in}}{\pgfqpoint{2.383039in}{1.570246in}}%
\pgfpathcurveto{\pgfqpoint{2.388863in}{1.564422in}}{\pgfqpoint{2.396763in}{1.561150in}}{\pgfqpoint{2.404999in}{1.561150in}}%
\pgfpathclose%
\pgfusepath{stroke,fill}%
\end{pgfscope}%
\begin{pgfscope}%
\pgfpathrectangle{\pgfqpoint{0.100000in}{0.212622in}}{\pgfqpoint{3.696000in}{3.696000in}}%
\pgfusepath{clip}%
\pgfsetbuttcap%
\pgfsetroundjoin%
\definecolor{currentfill}{rgb}{0.121569,0.466667,0.705882}%
\pgfsetfillcolor{currentfill}%
\pgfsetfillopacity{0.999044}%
\pgfsetlinewidth{1.003750pt}%
\definecolor{currentstroke}{rgb}{0.121569,0.466667,0.705882}%
\pgfsetstrokecolor{currentstroke}%
\pgfsetstrokeopacity{0.999044}%
\pgfsetdash{}{0pt}%
\pgfpathmoveto{\pgfqpoint{2.425893in}{1.554614in}}%
\pgfpathcurveto{\pgfqpoint{2.434129in}{1.554614in}}{\pgfqpoint{2.442029in}{1.557886in}}{\pgfqpoint{2.447853in}{1.563710in}}%
\pgfpathcurveto{\pgfqpoint{2.453677in}{1.569534in}}{\pgfqpoint{2.456949in}{1.577434in}}{\pgfqpoint{2.456949in}{1.585671in}}%
\pgfpathcurveto{\pgfqpoint{2.456949in}{1.593907in}}{\pgfqpoint{2.453677in}{1.601807in}}{\pgfqpoint{2.447853in}{1.607631in}}%
\pgfpathcurveto{\pgfqpoint{2.442029in}{1.613455in}}{\pgfqpoint{2.434129in}{1.616727in}}{\pgfqpoint{2.425893in}{1.616727in}}%
\pgfpathcurveto{\pgfqpoint{2.417656in}{1.616727in}}{\pgfqpoint{2.409756in}{1.613455in}}{\pgfqpoint{2.403932in}{1.607631in}}%
\pgfpathcurveto{\pgfqpoint{2.398108in}{1.601807in}}{\pgfqpoint{2.394836in}{1.593907in}}{\pgfqpoint{2.394836in}{1.585671in}}%
\pgfpathcurveto{\pgfqpoint{2.394836in}{1.577434in}}{\pgfqpoint{2.398108in}{1.569534in}}{\pgfqpoint{2.403932in}{1.563710in}}%
\pgfpathcurveto{\pgfqpoint{2.409756in}{1.557886in}}{\pgfqpoint{2.417656in}{1.554614in}}{\pgfqpoint{2.425893in}{1.554614in}}%
\pgfpathclose%
\pgfusepath{stroke,fill}%
\end{pgfscope}%
\begin{pgfscope}%
\pgfpathrectangle{\pgfqpoint{0.100000in}{0.212622in}}{\pgfqpoint{3.696000in}{3.696000in}}%
\pgfusepath{clip}%
\pgfsetbuttcap%
\pgfsetroundjoin%
\definecolor{currentfill}{rgb}{0.121569,0.466667,0.705882}%
\pgfsetfillcolor{currentfill}%
\pgfsetfillopacity{0.999047}%
\pgfsetlinewidth{1.003750pt}%
\definecolor{currentstroke}{rgb}{0.121569,0.466667,0.705882}%
\pgfsetstrokecolor{currentstroke}%
\pgfsetstrokeopacity{0.999047}%
\pgfsetdash{}{0pt}%
\pgfpathmoveto{\pgfqpoint{2.406796in}{1.561064in}}%
\pgfpathcurveto{\pgfqpoint{2.415033in}{1.561064in}}{\pgfqpoint{2.422933in}{1.564337in}}{\pgfqpoint{2.428757in}{1.570160in}}%
\pgfpathcurveto{\pgfqpoint{2.434581in}{1.575984in}}{\pgfqpoint{2.437853in}{1.583884in}}{\pgfqpoint{2.437853in}{1.592121in}}%
\pgfpathcurveto{\pgfqpoint{2.437853in}{1.600357in}}{\pgfqpoint{2.434581in}{1.608257in}}{\pgfqpoint{2.428757in}{1.614081in}}%
\pgfpathcurveto{\pgfqpoint{2.422933in}{1.619905in}}{\pgfqpoint{2.415033in}{1.623177in}}{\pgfqpoint{2.406796in}{1.623177in}}%
\pgfpathcurveto{\pgfqpoint{2.398560in}{1.623177in}}{\pgfqpoint{2.390660in}{1.619905in}}{\pgfqpoint{2.384836in}{1.614081in}}%
\pgfpathcurveto{\pgfqpoint{2.379012in}{1.608257in}}{\pgfqpoint{2.375740in}{1.600357in}}{\pgfqpoint{2.375740in}{1.592121in}}%
\pgfpathcurveto{\pgfqpoint{2.375740in}{1.583884in}}{\pgfqpoint{2.379012in}{1.575984in}}{\pgfqpoint{2.384836in}{1.570160in}}%
\pgfpathcurveto{\pgfqpoint{2.390660in}{1.564337in}}{\pgfqpoint{2.398560in}{1.561064in}}{\pgfqpoint{2.406796in}{1.561064in}}%
\pgfpathclose%
\pgfusepath{stroke,fill}%
\end{pgfscope}%
\begin{pgfscope}%
\pgfpathrectangle{\pgfqpoint{0.100000in}{0.212622in}}{\pgfqpoint{3.696000in}{3.696000in}}%
\pgfusepath{clip}%
\pgfsetbuttcap%
\pgfsetroundjoin%
\definecolor{currentfill}{rgb}{0.121569,0.466667,0.705882}%
\pgfsetfillcolor{currentfill}%
\pgfsetfillopacity{0.999201}%
\pgfsetlinewidth{1.003750pt}%
\definecolor{currentstroke}{rgb}{0.121569,0.466667,0.705882}%
\pgfsetstrokecolor{currentstroke}%
\pgfsetstrokeopacity{0.999201}%
\pgfsetdash{}{0pt}%
\pgfpathmoveto{\pgfqpoint{2.408072in}{1.559993in}}%
\pgfpathcurveto{\pgfqpoint{2.416308in}{1.559993in}}{\pgfqpoint{2.424208in}{1.563265in}}{\pgfqpoint{2.430032in}{1.569089in}}%
\pgfpathcurveto{\pgfqpoint{2.435856in}{1.574913in}}{\pgfqpoint{2.439128in}{1.582813in}}{\pgfqpoint{2.439128in}{1.591049in}}%
\pgfpathcurveto{\pgfqpoint{2.439128in}{1.599285in}}{\pgfqpoint{2.435856in}{1.607185in}}{\pgfqpoint{2.430032in}{1.613009in}}%
\pgfpathcurveto{\pgfqpoint{2.424208in}{1.618833in}}{\pgfqpoint{2.416308in}{1.622106in}}{\pgfqpoint{2.408072in}{1.622106in}}%
\pgfpathcurveto{\pgfqpoint{2.399835in}{1.622106in}}{\pgfqpoint{2.391935in}{1.618833in}}{\pgfqpoint{2.386111in}{1.613009in}}%
\pgfpathcurveto{\pgfqpoint{2.380287in}{1.607185in}}{\pgfqpoint{2.377015in}{1.599285in}}{\pgfqpoint{2.377015in}{1.591049in}}%
\pgfpathcurveto{\pgfqpoint{2.377015in}{1.582813in}}{\pgfqpoint{2.380287in}{1.574913in}}{\pgfqpoint{2.386111in}{1.569089in}}%
\pgfpathcurveto{\pgfqpoint{2.391935in}{1.563265in}}{\pgfqpoint{2.399835in}{1.559993in}}{\pgfqpoint{2.408072in}{1.559993in}}%
\pgfpathclose%
\pgfusepath{stroke,fill}%
\end{pgfscope}%
\begin{pgfscope}%
\pgfpathrectangle{\pgfqpoint{0.100000in}{0.212622in}}{\pgfqpoint{3.696000in}{3.696000in}}%
\pgfusepath{clip}%
\pgfsetbuttcap%
\pgfsetroundjoin%
\definecolor{currentfill}{rgb}{0.121569,0.466667,0.705882}%
\pgfsetfillcolor{currentfill}%
\pgfsetfillopacity{0.999302}%
\pgfsetlinewidth{1.003750pt}%
\definecolor{currentstroke}{rgb}{0.121569,0.466667,0.705882}%
\pgfsetstrokecolor{currentstroke}%
\pgfsetstrokeopacity{0.999302}%
\pgfsetdash{}{0pt}%
\pgfpathmoveto{\pgfqpoint{2.425270in}{1.554061in}}%
\pgfpathcurveto{\pgfqpoint{2.433506in}{1.554061in}}{\pgfqpoint{2.441406in}{1.557333in}}{\pgfqpoint{2.447230in}{1.563157in}}%
\pgfpathcurveto{\pgfqpoint{2.453054in}{1.568981in}}{\pgfqpoint{2.456326in}{1.576881in}}{\pgfqpoint{2.456326in}{1.585117in}}%
\pgfpathcurveto{\pgfqpoint{2.456326in}{1.593353in}}{\pgfqpoint{2.453054in}{1.601253in}}{\pgfqpoint{2.447230in}{1.607077in}}%
\pgfpathcurveto{\pgfqpoint{2.441406in}{1.612901in}}{\pgfqpoint{2.433506in}{1.616174in}}{\pgfqpoint{2.425270in}{1.616174in}}%
\pgfpathcurveto{\pgfqpoint{2.417033in}{1.616174in}}{\pgfqpoint{2.409133in}{1.612901in}}{\pgfqpoint{2.403309in}{1.607077in}}%
\pgfpathcurveto{\pgfqpoint{2.397486in}{1.601253in}}{\pgfqpoint{2.394213in}{1.593353in}}{\pgfqpoint{2.394213in}{1.585117in}}%
\pgfpathcurveto{\pgfqpoint{2.394213in}{1.576881in}}{\pgfqpoint{2.397486in}{1.568981in}}{\pgfqpoint{2.403309in}{1.563157in}}%
\pgfpathcurveto{\pgfqpoint{2.409133in}{1.557333in}}{\pgfqpoint{2.417033in}{1.554061in}}{\pgfqpoint{2.425270in}{1.554061in}}%
\pgfpathclose%
\pgfusepath{stroke,fill}%
\end{pgfscope}%
\begin{pgfscope}%
\pgfpathrectangle{\pgfqpoint{0.100000in}{0.212622in}}{\pgfqpoint{3.696000in}{3.696000in}}%
\pgfusepath{clip}%
\pgfsetbuttcap%
\pgfsetroundjoin%
\definecolor{currentfill}{rgb}{0.121569,0.466667,0.705882}%
\pgfsetfillcolor{currentfill}%
\pgfsetfillopacity{0.999440}%
\pgfsetlinewidth{1.003750pt}%
\definecolor{currentstroke}{rgb}{0.121569,0.466667,0.705882}%
\pgfsetstrokecolor{currentstroke}%
\pgfsetstrokeopacity{0.999440}%
\pgfsetdash{}{0pt}%
\pgfpathmoveto{\pgfqpoint{2.424886in}{1.553787in}}%
\pgfpathcurveto{\pgfqpoint{2.433122in}{1.553787in}}{\pgfqpoint{2.441022in}{1.557060in}}{\pgfqpoint{2.446846in}{1.562884in}}%
\pgfpathcurveto{\pgfqpoint{2.452670in}{1.568708in}}{\pgfqpoint{2.455943in}{1.576608in}}{\pgfqpoint{2.455943in}{1.584844in}}%
\pgfpathcurveto{\pgfqpoint{2.455943in}{1.593080in}}{\pgfqpoint{2.452670in}{1.600980in}}{\pgfqpoint{2.446846in}{1.606804in}}%
\pgfpathcurveto{\pgfqpoint{2.441022in}{1.612628in}}{\pgfqpoint{2.433122in}{1.615900in}}{\pgfqpoint{2.424886in}{1.615900in}}%
\pgfpathcurveto{\pgfqpoint{2.416650in}{1.615900in}}{\pgfqpoint{2.408750in}{1.612628in}}{\pgfqpoint{2.402926in}{1.606804in}}%
\pgfpathcurveto{\pgfqpoint{2.397102in}{1.600980in}}{\pgfqpoint{2.393830in}{1.593080in}}{\pgfqpoint{2.393830in}{1.584844in}}%
\pgfpathcurveto{\pgfqpoint{2.393830in}{1.576608in}}{\pgfqpoint{2.397102in}{1.568708in}}{\pgfqpoint{2.402926in}{1.562884in}}%
\pgfpathcurveto{\pgfqpoint{2.408750in}{1.557060in}}{\pgfqpoint{2.416650in}{1.553787in}}{\pgfqpoint{2.424886in}{1.553787in}}%
\pgfpathclose%
\pgfusepath{stroke,fill}%
\end{pgfscope}%
\begin{pgfscope}%
\pgfpathrectangle{\pgfqpoint{0.100000in}{0.212622in}}{\pgfqpoint{3.696000in}{3.696000in}}%
\pgfusepath{clip}%
\pgfsetbuttcap%
\pgfsetroundjoin%
\definecolor{currentfill}{rgb}{0.121569,0.466667,0.705882}%
\pgfsetfillcolor{currentfill}%
\pgfsetfillopacity{0.999504}%
\pgfsetlinewidth{1.003750pt}%
\definecolor{currentstroke}{rgb}{0.121569,0.466667,0.705882}%
\pgfsetstrokecolor{currentstroke}%
\pgfsetstrokeopacity{0.999504}%
\pgfsetdash{}{0pt}%
\pgfpathmoveto{\pgfqpoint{2.424641in}{1.553608in}}%
\pgfpathcurveto{\pgfqpoint{2.432877in}{1.553608in}}{\pgfqpoint{2.440777in}{1.556881in}}{\pgfqpoint{2.446601in}{1.562704in}}%
\pgfpathcurveto{\pgfqpoint{2.452425in}{1.568528in}}{\pgfqpoint{2.455697in}{1.576428in}}{\pgfqpoint{2.455697in}{1.584665in}}%
\pgfpathcurveto{\pgfqpoint{2.455697in}{1.592901in}}{\pgfqpoint{2.452425in}{1.600801in}}{\pgfqpoint{2.446601in}{1.606625in}}%
\pgfpathcurveto{\pgfqpoint{2.440777in}{1.612449in}}{\pgfqpoint{2.432877in}{1.615721in}}{\pgfqpoint{2.424641in}{1.615721in}}%
\pgfpathcurveto{\pgfqpoint{2.416404in}{1.615721in}}{\pgfqpoint{2.408504in}{1.612449in}}{\pgfqpoint{2.402680in}{1.606625in}}%
\pgfpathcurveto{\pgfqpoint{2.396856in}{1.600801in}}{\pgfqpoint{2.393584in}{1.592901in}}{\pgfqpoint{2.393584in}{1.584665in}}%
\pgfpathcurveto{\pgfqpoint{2.393584in}{1.576428in}}{\pgfqpoint{2.396856in}{1.568528in}}{\pgfqpoint{2.402680in}{1.562704in}}%
\pgfpathcurveto{\pgfqpoint{2.408504in}{1.556881in}}{\pgfqpoint{2.416404in}{1.553608in}}{\pgfqpoint{2.424641in}{1.553608in}}%
\pgfpathclose%
\pgfusepath{stroke,fill}%
\end{pgfscope}%
\begin{pgfscope}%
\pgfpathrectangle{\pgfqpoint{0.100000in}{0.212622in}}{\pgfqpoint{3.696000in}{3.696000in}}%
\pgfusepath{clip}%
\pgfsetbuttcap%
\pgfsetroundjoin%
\definecolor{currentfill}{rgb}{0.121569,0.466667,0.705882}%
\pgfsetfillcolor{currentfill}%
\pgfsetfillopacity{0.999537}%
\pgfsetlinewidth{1.003750pt}%
\definecolor{currentstroke}{rgb}{0.121569,0.466667,0.705882}%
\pgfsetstrokecolor{currentstroke}%
\pgfsetstrokeopacity{0.999537}%
\pgfsetdash{}{0pt}%
\pgfpathmoveto{\pgfqpoint{2.424489in}{1.553527in}}%
\pgfpathcurveto{\pgfqpoint{2.432726in}{1.553527in}}{\pgfqpoint{2.440626in}{1.556800in}}{\pgfqpoint{2.446450in}{1.562624in}}%
\pgfpathcurveto{\pgfqpoint{2.452274in}{1.568448in}}{\pgfqpoint{2.455546in}{1.576348in}}{\pgfqpoint{2.455546in}{1.584584in}}%
\pgfpathcurveto{\pgfqpoint{2.455546in}{1.592820in}}{\pgfqpoint{2.452274in}{1.600720in}}{\pgfqpoint{2.446450in}{1.606544in}}%
\pgfpathcurveto{\pgfqpoint{2.440626in}{1.612368in}}{\pgfqpoint{2.432726in}{1.615640in}}{\pgfqpoint{2.424489in}{1.615640in}}%
\pgfpathcurveto{\pgfqpoint{2.416253in}{1.615640in}}{\pgfqpoint{2.408353in}{1.612368in}}{\pgfqpoint{2.402529in}{1.606544in}}%
\pgfpathcurveto{\pgfqpoint{2.396705in}{1.600720in}}{\pgfqpoint{2.393433in}{1.592820in}}{\pgfqpoint{2.393433in}{1.584584in}}%
\pgfpathcurveto{\pgfqpoint{2.393433in}{1.576348in}}{\pgfqpoint{2.396705in}{1.568448in}}{\pgfqpoint{2.402529in}{1.562624in}}%
\pgfpathcurveto{\pgfqpoint{2.408353in}{1.556800in}}{\pgfqpoint{2.416253in}{1.553527in}}{\pgfqpoint{2.424489in}{1.553527in}}%
\pgfpathclose%
\pgfusepath{stroke,fill}%
\end{pgfscope}%
\begin{pgfscope}%
\pgfpathrectangle{\pgfqpoint{0.100000in}{0.212622in}}{\pgfqpoint{3.696000in}{3.696000in}}%
\pgfusepath{clip}%
\pgfsetbuttcap%
\pgfsetroundjoin%
\definecolor{currentfill}{rgb}{0.121569,0.466667,0.705882}%
\pgfsetfillcolor{currentfill}%
\pgfsetfillopacity{0.999563}%
\pgfsetlinewidth{1.003750pt}%
\definecolor{currentstroke}{rgb}{0.121569,0.466667,0.705882}%
\pgfsetstrokecolor{currentstroke}%
\pgfsetstrokeopacity{0.999563}%
\pgfsetdash{}{0pt}%
\pgfpathmoveto{\pgfqpoint{2.410724in}{1.559563in}}%
\pgfpathcurveto{\pgfqpoint{2.418960in}{1.559563in}}{\pgfqpoint{2.426860in}{1.562835in}}{\pgfqpoint{2.432684in}{1.568659in}}%
\pgfpathcurveto{\pgfqpoint{2.438508in}{1.574483in}}{\pgfqpoint{2.441780in}{1.582383in}}{\pgfqpoint{2.441780in}{1.590619in}}%
\pgfpathcurveto{\pgfqpoint{2.441780in}{1.598856in}}{\pgfqpoint{2.438508in}{1.606756in}}{\pgfqpoint{2.432684in}{1.612580in}}%
\pgfpathcurveto{\pgfqpoint{2.426860in}{1.618404in}}{\pgfqpoint{2.418960in}{1.621676in}}{\pgfqpoint{2.410724in}{1.621676in}}%
\pgfpathcurveto{\pgfqpoint{2.402488in}{1.621676in}}{\pgfqpoint{2.394588in}{1.618404in}}{\pgfqpoint{2.388764in}{1.612580in}}%
\pgfpathcurveto{\pgfqpoint{2.382940in}{1.606756in}}{\pgfqpoint{2.379667in}{1.598856in}}{\pgfqpoint{2.379667in}{1.590619in}}%
\pgfpathcurveto{\pgfqpoint{2.379667in}{1.582383in}}{\pgfqpoint{2.382940in}{1.574483in}}{\pgfqpoint{2.388764in}{1.568659in}}%
\pgfpathcurveto{\pgfqpoint{2.394588in}{1.562835in}}{\pgfqpoint{2.402488in}{1.559563in}}{\pgfqpoint{2.410724in}{1.559563in}}%
\pgfpathclose%
\pgfusepath{stroke,fill}%
\end{pgfscope}%
\begin{pgfscope}%
\pgfpathrectangle{\pgfqpoint{0.100000in}{0.212622in}}{\pgfqpoint{3.696000in}{3.696000in}}%
\pgfusepath{clip}%
\pgfsetbuttcap%
\pgfsetroundjoin%
\definecolor{currentfill}{rgb}{0.121569,0.466667,0.705882}%
\pgfsetfillcolor{currentfill}%
\pgfsetfillopacity{0.999590}%
\pgfsetlinewidth{1.003750pt}%
\definecolor{currentstroke}{rgb}{0.121569,0.466667,0.705882}%
\pgfsetstrokecolor{currentstroke}%
\pgfsetstrokeopacity{0.999590}%
\pgfsetdash{}{0pt}%
\pgfpathmoveto{\pgfqpoint{2.412968in}{1.557069in}}%
\pgfpathcurveto{\pgfqpoint{2.421205in}{1.557069in}}{\pgfqpoint{2.429105in}{1.560342in}}{\pgfqpoint{2.434928in}{1.566166in}}%
\pgfpathcurveto{\pgfqpoint{2.440752in}{1.571990in}}{\pgfqpoint{2.444025in}{1.579890in}}{\pgfqpoint{2.444025in}{1.588126in}}%
\pgfpathcurveto{\pgfqpoint{2.444025in}{1.596362in}}{\pgfqpoint{2.440752in}{1.604262in}}{\pgfqpoint{2.434928in}{1.610086in}}%
\pgfpathcurveto{\pgfqpoint{2.429105in}{1.615910in}}{\pgfqpoint{2.421205in}{1.619182in}}{\pgfqpoint{2.412968in}{1.619182in}}%
\pgfpathcurveto{\pgfqpoint{2.404732in}{1.619182in}}{\pgfqpoint{2.396832in}{1.615910in}}{\pgfqpoint{2.391008in}{1.610086in}}%
\pgfpathcurveto{\pgfqpoint{2.385184in}{1.604262in}}{\pgfqpoint{2.381912in}{1.596362in}}{\pgfqpoint{2.381912in}{1.588126in}}%
\pgfpathcurveto{\pgfqpoint{2.381912in}{1.579890in}}{\pgfqpoint{2.385184in}{1.571990in}}{\pgfqpoint{2.391008in}{1.566166in}}%
\pgfpathcurveto{\pgfqpoint{2.396832in}{1.560342in}}{\pgfqpoint{2.404732in}{1.557069in}}{\pgfqpoint{2.412968in}{1.557069in}}%
\pgfpathclose%
\pgfusepath{stroke,fill}%
\end{pgfscope}%
\begin{pgfscope}%
\pgfpathrectangle{\pgfqpoint{0.100000in}{0.212622in}}{\pgfqpoint{3.696000in}{3.696000in}}%
\pgfusepath{clip}%
\pgfsetbuttcap%
\pgfsetroundjoin%
\definecolor{currentfill}{rgb}{0.121569,0.466667,0.705882}%
\pgfsetfillcolor{currentfill}%
\pgfsetfillopacity{0.999654}%
\pgfsetlinewidth{1.003750pt}%
\definecolor{currentstroke}{rgb}{0.121569,0.466667,0.705882}%
\pgfsetstrokecolor{currentstroke}%
\pgfsetstrokeopacity{0.999654}%
\pgfsetdash{}{0pt}%
\pgfpathmoveto{\pgfqpoint{2.423925in}{1.553597in}}%
\pgfpathcurveto{\pgfqpoint{2.432161in}{1.553597in}}{\pgfqpoint{2.440061in}{1.556870in}}{\pgfqpoint{2.445885in}{1.562693in}}%
\pgfpathcurveto{\pgfqpoint{2.451709in}{1.568517in}}{\pgfqpoint{2.454981in}{1.576417in}}{\pgfqpoint{2.454981in}{1.584654in}}%
\pgfpathcurveto{\pgfqpoint{2.454981in}{1.592890in}}{\pgfqpoint{2.451709in}{1.600790in}}{\pgfqpoint{2.445885in}{1.606614in}}%
\pgfpathcurveto{\pgfqpoint{2.440061in}{1.612438in}}{\pgfqpoint{2.432161in}{1.615710in}}{\pgfqpoint{2.423925in}{1.615710in}}%
\pgfpathcurveto{\pgfqpoint{2.415688in}{1.615710in}}{\pgfqpoint{2.407788in}{1.612438in}}{\pgfqpoint{2.401964in}{1.606614in}}%
\pgfpathcurveto{\pgfqpoint{2.396140in}{1.600790in}}{\pgfqpoint{2.392868in}{1.592890in}}{\pgfqpoint{2.392868in}{1.584654in}}%
\pgfpathcurveto{\pgfqpoint{2.392868in}{1.576417in}}{\pgfqpoint{2.396140in}{1.568517in}}{\pgfqpoint{2.401964in}{1.562693in}}%
\pgfpathcurveto{\pgfqpoint{2.407788in}{1.556870in}}{\pgfqpoint{2.415688in}{1.553597in}}{\pgfqpoint{2.423925in}{1.553597in}}%
\pgfpathclose%
\pgfusepath{stroke,fill}%
\end{pgfscope}%
\begin{pgfscope}%
\pgfpathrectangle{\pgfqpoint{0.100000in}{0.212622in}}{\pgfqpoint{3.696000in}{3.696000in}}%
\pgfusepath{clip}%
\pgfsetbuttcap%
\pgfsetroundjoin%
\definecolor{currentfill}{rgb}{0.121569,0.466667,0.705882}%
\pgfsetfillcolor{currentfill}%
\pgfsetfillopacity{0.999824}%
\pgfsetlinewidth{1.003750pt}%
\definecolor{currentstroke}{rgb}{0.121569,0.466667,0.705882}%
\pgfsetstrokecolor{currentstroke}%
\pgfsetstrokeopacity{0.999824}%
\pgfsetdash{}{0pt}%
\pgfpathmoveto{\pgfqpoint{2.422770in}{1.554043in}}%
\pgfpathcurveto{\pgfqpoint{2.431007in}{1.554043in}}{\pgfqpoint{2.438907in}{1.557315in}}{\pgfqpoint{2.444731in}{1.563139in}}%
\pgfpathcurveto{\pgfqpoint{2.450554in}{1.568963in}}{\pgfqpoint{2.453827in}{1.576863in}}{\pgfqpoint{2.453827in}{1.585100in}}%
\pgfpathcurveto{\pgfqpoint{2.453827in}{1.593336in}}{\pgfqpoint{2.450554in}{1.601236in}}{\pgfqpoint{2.444731in}{1.607060in}}%
\pgfpathcurveto{\pgfqpoint{2.438907in}{1.612884in}}{\pgfqpoint{2.431007in}{1.616156in}}{\pgfqpoint{2.422770in}{1.616156in}}%
\pgfpathcurveto{\pgfqpoint{2.414534in}{1.616156in}}{\pgfqpoint{2.406634in}{1.612884in}}{\pgfqpoint{2.400810in}{1.607060in}}%
\pgfpathcurveto{\pgfqpoint{2.394986in}{1.601236in}}{\pgfqpoint{2.391714in}{1.593336in}}{\pgfqpoint{2.391714in}{1.585100in}}%
\pgfpathcurveto{\pgfqpoint{2.391714in}{1.576863in}}{\pgfqpoint{2.394986in}{1.568963in}}{\pgfqpoint{2.400810in}{1.563139in}}%
\pgfpathcurveto{\pgfqpoint{2.406634in}{1.557315in}}{\pgfqpoint{2.414534in}{1.554043in}}{\pgfqpoint{2.422770in}{1.554043in}}%
\pgfpathclose%
\pgfusepath{stroke,fill}%
\end{pgfscope}%
\begin{pgfscope}%
\pgfpathrectangle{\pgfqpoint{0.100000in}{0.212622in}}{\pgfqpoint{3.696000in}{3.696000in}}%
\pgfusepath{clip}%
\pgfsetbuttcap%
\pgfsetroundjoin%
\definecolor{currentfill}{rgb}{0.121569,0.466667,0.705882}%
\pgfsetfillcolor{currentfill}%
\pgfsetfillopacity{0.999938}%
\pgfsetlinewidth{1.003750pt}%
\definecolor{currentstroke}{rgb}{0.121569,0.466667,0.705882}%
\pgfsetstrokecolor{currentstroke}%
\pgfsetstrokeopacity{0.999938}%
\pgfsetdash{}{0pt}%
\pgfpathmoveto{\pgfqpoint{2.417416in}{1.555855in}}%
\pgfpathcurveto{\pgfqpoint{2.425653in}{1.555855in}}{\pgfqpoint{2.433553in}{1.559128in}}{\pgfqpoint{2.439377in}{1.564952in}}%
\pgfpathcurveto{\pgfqpoint{2.445201in}{1.570775in}}{\pgfqpoint{2.448473in}{1.578676in}}{\pgfqpoint{2.448473in}{1.586912in}}%
\pgfpathcurveto{\pgfqpoint{2.448473in}{1.595148in}}{\pgfqpoint{2.445201in}{1.603048in}}{\pgfqpoint{2.439377in}{1.608872in}}%
\pgfpathcurveto{\pgfqpoint{2.433553in}{1.614696in}}{\pgfqpoint{2.425653in}{1.617968in}}{\pgfqpoint{2.417416in}{1.617968in}}%
\pgfpathcurveto{\pgfqpoint{2.409180in}{1.617968in}}{\pgfqpoint{2.401280in}{1.614696in}}{\pgfqpoint{2.395456in}{1.608872in}}%
\pgfpathcurveto{\pgfqpoint{2.389632in}{1.603048in}}{\pgfqpoint{2.386360in}{1.595148in}}{\pgfqpoint{2.386360in}{1.586912in}}%
\pgfpathcurveto{\pgfqpoint{2.386360in}{1.578676in}}{\pgfqpoint{2.389632in}{1.570775in}}{\pgfqpoint{2.395456in}{1.564952in}}%
\pgfpathcurveto{\pgfqpoint{2.401280in}{1.559128in}}{\pgfqpoint{2.409180in}{1.555855in}}{\pgfqpoint{2.417416in}{1.555855in}}%
\pgfpathclose%
\pgfusepath{stroke,fill}%
\end{pgfscope}%
\begin{pgfscope}%
\pgfpathrectangle{\pgfqpoint{0.100000in}{0.212622in}}{\pgfqpoint{3.696000in}{3.696000in}}%
\pgfusepath{clip}%
\pgfsetbuttcap%
\pgfsetroundjoin%
\definecolor{currentfill}{rgb}{0.121569,0.466667,0.705882}%
\pgfsetfillcolor{currentfill}%
\pgfsetlinewidth{1.003750pt}%
\definecolor{currentstroke}{rgb}{0.121569,0.466667,0.705882}%
\pgfsetstrokecolor{currentstroke}%
\pgfsetdash{}{0pt}%
\pgfpathmoveto{\pgfqpoint{2.420596in}{1.554484in}}%
\pgfpathcurveto{\pgfqpoint{2.428833in}{1.554484in}}{\pgfqpoint{2.436733in}{1.557757in}}{\pgfqpoint{2.442557in}{1.563580in}}%
\pgfpathcurveto{\pgfqpoint{2.448381in}{1.569404in}}{\pgfqpoint{2.451653in}{1.577304in}}{\pgfqpoint{2.451653in}{1.585541in}}%
\pgfpathcurveto{\pgfqpoint{2.451653in}{1.593777in}}{\pgfqpoint{2.448381in}{1.601677in}}{\pgfqpoint{2.442557in}{1.607501in}}%
\pgfpathcurveto{\pgfqpoint{2.436733in}{1.613325in}}{\pgfqpoint{2.428833in}{1.616597in}}{\pgfqpoint{2.420596in}{1.616597in}}%
\pgfpathcurveto{\pgfqpoint{2.412360in}{1.616597in}}{\pgfqpoint{2.404460in}{1.613325in}}{\pgfqpoint{2.398636in}{1.607501in}}%
\pgfpathcurveto{\pgfqpoint{2.392812in}{1.601677in}}{\pgfqpoint{2.389540in}{1.593777in}}{\pgfqpoint{2.389540in}{1.585541in}}%
\pgfpathcurveto{\pgfqpoint{2.389540in}{1.577304in}}{\pgfqpoint{2.392812in}{1.569404in}}{\pgfqpoint{2.398636in}{1.563580in}}%
\pgfpathcurveto{\pgfqpoint{2.404460in}{1.557757in}}{\pgfqpoint{2.412360in}{1.554484in}}{\pgfqpoint{2.420596in}{1.554484in}}%
\pgfpathclose%
\pgfusepath{stroke,fill}%
\end{pgfscope}%
\begin{pgfscope}%
\pgfsetbuttcap%
\pgfsetmiterjoin%
\definecolor{currentfill}{rgb}{1.000000,1.000000,1.000000}%
\pgfsetfillcolor{currentfill}%
\pgfsetfillopacity{0.800000}%
\pgfsetlinewidth{1.003750pt}%
\definecolor{currentstroke}{rgb}{0.800000,0.800000,0.800000}%
\pgfsetstrokecolor{currentstroke}%
\pgfsetstrokeopacity{0.800000}%
\pgfsetdash{}{0pt}%
\pgfpathmoveto{\pgfqpoint{2.104889in}{3.216678in}}%
\pgfpathlineto{\pgfqpoint{3.698778in}{3.216678in}}%
\pgfpathquadraticcurveto{\pgfqpoint{3.726556in}{3.216678in}}{\pgfqpoint{3.726556in}{3.244456in}}%
\pgfpathlineto{\pgfqpoint{3.726556in}{3.811400in}}%
\pgfpathquadraticcurveto{\pgfqpoint{3.726556in}{3.839178in}}{\pgfqpoint{3.698778in}{3.839178in}}%
\pgfpathlineto{\pgfqpoint{2.104889in}{3.839178in}}%
\pgfpathquadraticcurveto{\pgfqpoint{2.077111in}{3.839178in}}{\pgfqpoint{2.077111in}{3.811400in}}%
\pgfpathlineto{\pgfqpoint{2.077111in}{3.244456in}}%
\pgfpathquadraticcurveto{\pgfqpoint{2.077111in}{3.216678in}}{\pgfqpoint{2.104889in}{3.216678in}}%
\pgfpathclose%
\pgfusepath{stroke,fill}%
\end{pgfscope}%
\begin{pgfscope}%
\pgfsetrectcap%
\pgfsetroundjoin%
\pgfsetlinewidth{1.505625pt}%
\definecolor{currentstroke}{rgb}{0.121569,0.466667,0.705882}%
\pgfsetstrokecolor{currentstroke}%
\pgfsetdash{}{0pt}%
\pgfpathmoveto{\pgfqpoint{2.132667in}{3.735011in}}%
\pgfpathlineto{\pgfqpoint{2.410444in}{3.735011in}}%
\pgfusepath{stroke}%
\end{pgfscope}%
\begin{pgfscope}%
\definecolor{textcolor}{rgb}{0.000000,0.000000,0.000000}%
\pgfsetstrokecolor{textcolor}%
\pgfsetfillcolor{textcolor}%
\pgftext[x=2.521555in,y=3.686400in,left,base]{\color{textcolor}\rmfamily\fontsize{10.000000}{12.000000}\selectfont Ground truth}%
\end{pgfscope}%
\begin{pgfscope}%
\pgfsetbuttcap%
\pgfsetroundjoin%
\definecolor{currentfill}{rgb}{0.121569,0.466667,0.705882}%
\pgfsetfillcolor{currentfill}%
\pgfsetlinewidth{1.003750pt}%
\definecolor{currentstroke}{rgb}{0.121569,0.466667,0.705882}%
\pgfsetstrokecolor{currentstroke}%
\pgfsetdash{}{0pt}%
\pgfsys@defobject{currentmarker}{\pgfqpoint{-0.031056in}{-0.031056in}}{\pgfqpoint{0.031056in}{0.031056in}}{%
\pgfpathmoveto{\pgfqpoint{0.000000in}{-0.031056in}}%
\pgfpathcurveto{\pgfqpoint{0.008236in}{-0.031056in}}{\pgfqpoint{0.016136in}{-0.027784in}}{\pgfqpoint{0.021960in}{-0.021960in}}%
\pgfpathcurveto{\pgfqpoint{0.027784in}{-0.016136in}}{\pgfqpoint{0.031056in}{-0.008236in}}{\pgfqpoint{0.031056in}{0.000000in}}%
\pgfpathcurveto{\pgfqpoint{0.031056in}{0.008236in}}{\pgfqpoint{0.027784in}{0.016136in}}{\pgfqpoint{0.021960in}{0.021960in}}%
\pgfpathcurveto{\pgfqpoint{0.016136in}{0.027784in}}{\pgfqpoint{0.008236in}{0.031056in}}{\pgfqpoint{0.000000in}{0.031056in}}%
\pgfpathcurveto{\pgfqpoint{-0.008236in}{0.031056in}}{\pgfqpoint{-0.016136in}{0.027784in}}{\pgfqpoint{-0.021960in}{0.021960in}}%
\pgfpathcurveto{\pgfqpoint{-0.027784in}{0.016136in}}{\pgfqpoint{-0.031056in}{0.008236in}}{\pgfqpoint{-0.031056in}{0.000000in}}%
\pgfpathcurveto{\pgfqpoint{-0.031056in}{-0.008236in}}{\pgfqpoint{-0.027784in}{-0.016136in}}{\pgfqpoint{-0.021960in}{-0.021960in}}%
\pgfpathcurveto{\pgfqpoint{-0.016136in}{-0.027784in}}{\pgfqpoint{-0.008236in}{-0.031056in}}{\pgfqpoint{0.000000in}{-0.031056in}}%
\pgfpathclose%
\pgfusepath{stroke,fill}%
}%
\begin{pgfscope}%
\pgfsys@transformshift{2.271555in}{3.529248in}%
\pgfsys@useobject{currentmarker}{}%
\end{pgfscope}%
\end{pgfscope}%
\begin{pgfscope}%
\definecolor{textcolor}{rgb}{0.000000,0.000000,0.000000}%
\pgfsetstrokecolor{textcolor}%
\pgfsetfillcolor{textcolor}%
\pgftext[x=2.521555in,y=3.492789in,left,base]{\color{textcolor}\rmfamily\fontsize{10.000000}{12.000000}\selectfont Estimated position}%
\end{pgfscope}%
\begin{pgfscope}%
\pgfsetbuttcap%
\pgfsetroundjoin%
\definecolor{currentfill}{rgb}{1.000000,0.498039,0.054902}%
\pgfsetfillcolor{currentfill}%
\pgfsetlinewidth{1.003750pt}%
\definecolor{currentstroke}{rgb}{1.000000,0.498039,0.054902}%
\pgfsetstrokecolor{currentstroke}%
\pgfsetdash{}{0pt}%
\pgfsys@defobject{currentmarker}{\pgfqpoint{-0.031056in}{-0.031056in}}{\pgfqpoint{0.031056in}{0.031056in}}{%
\pgfpathmoveto{\pgfqpoint{0.000000in}{-0.031056in}}%
\pgfpathcurveto{\pgfqpoint{0.008236in}{-0.031056in}}{\pgfqpoint{0.016136in}{-0.027784in}}{\pgfqpoint{0.021960in}{-0.021960in}}%
\pgfpathcurveto{\pgfqpoint{0.027784in}{-0.016136in}}{\pgfqpoint{0.031056in}{-0.008236in}}{\pgfqpoint{0.031056in}{0.000000in}}%
\pgfpathcurveto{\pgfqpoint{0.031056in}{0.008236in}}{\pgfqpoint{0.027784in}{0.016136in}}{\pgfqpoint{0.021960in}{0.021960in}}%
\pgfpathcurveto{\pgfqpoint{0.016136in}{0.027784in}}{\pgfqpoint{0.008236in}{0.031056in}}{\pgfqpoint{0.000000in}{0.031056in}}%
\pgfpathcurveto{\pgfqpoint{-0.008236in}{0.031056in}}{\pgfqpoint{-0.016136in}{0.027784in}}{\pgfqpoint{-0.021960in}{0.021960in}}%
\pgfpathcurveto{\pgfqpoint{-0.027784in}{0.016136in}}{\pgfqpoint{-0.031056in}{0.008236in}}{\pgfqpoint{-0.031056in}{0.000000in}}%
\pgfpathcurveto{\pgfqpoint{-0.031056in}{-0.008236in}}{\pgfqpoint{-0.027784in}{-0.016136in}}{\pgfqpoint{-0.021960in}{-0.021960in}}%
\pgfpathcurveto{\pgfqpoint{-0.016136in}{-0.027784in}}{\pgfqpoint{-0.008236in}{-0.031056in}}{\pgfqpoint{0.000000in}{-0.031056in}}%
\pgfpathclose%
\pgfusepath{stroke,fill}%
}%
\begin{pgfscope}%
\pgfsys@transformshift{2.271555in}{3.335637in}%
\pgfsys@useobject{currentmarker}{}%
\end{pgfscope}%
\end{pgfscope}%
\begin{pgfscope}%
\definecolor{textcolor}{rgb}{0.000000,0.000000,0.000000}%
\pgfsetstrokecolor{textcolor}%
\pgfsetfillcolor{textcolor}%
\pgftext[x=2.521555in,y=3.299178in,left,base]{\color{textcolor}\rmfamily\fontsize{10.000000}{12.000000}\selectfont Estimated turn}%
\end{pgfscope}%
\end{pgfpicture}%
\makeatother%
\endgroup%
}
%         \caption{EKF's 3D position estimation had the lowest turn error for the 28-meter  side square experiment.}
%         \label{fig:square283D}
%     \end{subfigure}
%     \caption{Position estimation by the best performing algorithms in the 28-meter side square experiment.}
%     \label{fig:square28}
% \end{figure}

% \subsection{Spiral}

% For the spiral experiment, the Davenport algorithm which had the lowest displacement error with an average of 2.64 meters (1.78\% of error margin), and ROLEQ with an average of 2.46 meters of turn error (1.66\% of error margin).

% \begin{figure}[!h]
%     \centering
%     \begin{table}[H]
    \begin{center}
        \resizebox{1\linewidth}{!}{


            \begin{tabular}[t]{lcccc}
                \hline
                Algorithm        & Displacement Error[$m$] & Displacement Error[\%] & Turn Error[$m$] & Turn Error[\%] \\
                \hline
                AngularRate      & 41.51                   & 28.05                  & 61.37           & 41.47          \\
                \acrshort{aqua}  & 3.34                    & 2.26                   & 6.23            & 4.21           \\
                Complementary    & 2.86                    & 1.93                   & 6.92            & 4.67           \\
                Davenport        & 2.64                    & 1.78                   & 3.02            & 2.04           \\
                \acrshort{ekf}   & 3.01                    & 2.03                   & 3.03            & 2.05           \\
                \acrshort{famc}  & 39.96                   & 27.00                  & 48.09           & 32.49          \\
                \acrshort{flae}  & 3.18                    & 2.15                   & 3.71            & 2.51           \\
                Fourati          & 41.45                   & 28.00                  & 61.91           & 41.83          \\
                Madgwick         & 2.74                    & 1.85                   & 2.46            & 1.66           \\
                Mahony           & 2.81                    & 1.90                   & 3.47            & 2.34           \\
                \acrshort{oleq}  & 3.26                    & 2.20                   & 3.06            & 2.07           \\
                \acrshort{quest} & 19.97                   & 13.50                  & 35.23           & 23.80          \\
                \acrshort{roleq} & 3.39                    & 2.29                   & 3.17            & 2.14           \\
                \acrshort{saam}  & 2.73                    & 1.84                   & 3.29            & 2.23           \\
                Tilt             & 2.73                    & 1.84                   & 3.29            & 2.23           \\

                \hline
                Average          & 11.70                   & 7.91                   & 16.55           & 11.18
            \end{tabular}
        }
        \caption{Spiral position estimation error (displacement and turn) of the sensor fusion algorithms. }
        \label{tab:spiral}
    \end{center}
\end{table}
% \end{figure}

% \begin{figure}[!h]
%     \centering
%     \begin{subfigure}{0.49\textwidth}
%         \centering
%         \resizebox{1\linewidth}{!}{%% Creator: Matplotlib, PGF backend
%%
%% To include the figure in your LaTeX document, write
%%   \input{<filename>.pgf}
%%
%% Make sure the required packages are loaded in your preamble
%%   \usepackage{pgf}
%%
%% and, on pdftex
%%   \usepackage[utf8]{inputenc}\DeclareUnicodeCharacter{2212}{-}
%%
%% or, on luatex and xetex
%%   \usepackage{unicode-math}
%%
%% Figures using additional raster images can only be included by \input if
%% they are in the same directory as the main LaTeX file. For loading figures
%% from other directories you can use the `import` package
%%   \usepackage{import}
%%
%% and then include the figures with
%%   \import{<path to file>}{<filename>.pgf}
%%
%% Matplotlib used the following preamble
%%   \usepackage{fontspec}
%%
\begingroup%
\makeatletter%
\begin{pgfpicture}%
\pgfpathrectangle{\pgfpointorigin}{\pgfqpoint{4.342355in}{4.207622in}}%
\pgfusepath{use as bounding box, clip}%
\begin{pgfscope}%
\pgfsetbuttcap%
\pgfsetmiterjoin%
\definecolor{currentfill}{rgb}{1.000000,1.000000,1.000000}%
\pgfsetfillcolor{currentfill}%
\pgfsetlinewidth{0.000000pt}%
\definecolor{currentstroke}{rgb}{1.000000,1.000000,1.000000}%
\pgfsetstrokecolor{currentstroke}%
\pgfsetdash{}{0pt}%
\pgfpathmoveto{\pgfqpoint{0.000000in}{0.000000in}}%
\pgfpathlineto{\pgfqpoint{4.342355in}{0.000000in}}%
\pgfpathlineto{\pgfqpoint{4.342355in}{4.207622in}}%
\pgfpathlineto{\pgfqpoint{0.000000in}{4.207622in}}%
\pgfpathclose%
\pgfusepath{fill}%
\end{pgfscope}%
\begin{pgfscope}%
\pgfsetbuttcap%
\pgfsetmiterjoin%
\definecolor{currentfill}{rgb}{1.000000,1.000000,1.000000}%
\pgfsetfillcolor{currentfill}%
\pgfsetlinewidth{0.000000pt}%
\definecolor{currentstroke}{rgb}{0.000000,0.000000,0.000000}%
\pgfsetstrokecolor{currentstroke}%
\pgfsetstrokeopacity{0.000000}%
\pgfsetdash{}{0pt}%
\pgfpathmoveto{\pgfqpoint{0.100000in}{0.212622in}}%
\pgfpathlineto{\pgfqpoint{3.796000in}{0.212622in}}%
\pgfpathlineto{\pgfqpoint{3.796000in}{3.908622in}}%
\pgfpathlineto{\pgfqpoint{0.100000in}{3.908622in}}%
\pgfpathclose%
\pgfusepath{fill}%
\end{pgfscope}%
\begin{pgfscope}%
\pgfsetbuttcap%
\pgfsetmiterjoin%
\definecolor{currentfill}{rgb}{0.950000,0.950000,0.950000}%
\pgfsetfillcolor{currentfill}%
\pgfsetfillopacity{0.500000}%
\pgfsetlinewidth{1.003750pt}%
\definecolor{currentstroke}{rgb}{0.950000,0.950000,0.950000}%
\pgfsetstrokecolor{currentstroke}%
\pgfsetstrokeopacity{0.500000}%
\pgfsetdash{}{0pt}%
\pgfpathmoveto{\pgfqpoint{0.379073in}{1.123938in}}%
\pgfpathlineto{\pgfqpoint{1.599613in}{2.147018in}}%
\pgfpathlineto{\pgfqpoint{1.582647in}{3.622484in}}%
\pgfpathlineto{\pgfqpoint{0.303698in}{2.689165in}}%
\pgfusepath{stroke,fill}%
\end{pgfscope}%
\begin{pgfscope}%
\pgfsetbuttcap%
\pgfsetmiterjoin%
\definecolor{currentfill}{rgb}{0.900000,0.900000,0.900000}%
\pgfsetfillcolor{currentfill}%
\pgfsetfillopacity{0.500000}%
\pgfsetlinewidth{1.003750pt}%
\definecolor{currentstroke}{rgb}{0.900000,0.900000,0.900000}%
\pgfsetstrokecolor{currentstroke}%
\pgfsetstrokeopacity{0.500000}%
\pgfsetdash{}{0pt}%
\pgfpathmoveto{\pgfqpoint{1.599613in}{2.147018in}}%
\pgfpathlineto{\pgfqpoint{3.558144in}{1.577751in}}%
\pgfpathlineto{\pgfqpoint{3.628038in}{3.104037in}}%
\pgfpathlineto{\pgfqpoint{1.582647in}{3.622484in}}%
\pgfusepath{stroke,fill}%
\end{pgfscope}%
\begin{pgfscope}%
\pgfsetbuttcap%
\pgfsetmiterjoin%
\definecolor{currentfill}{rgb}{0.925000,0.925000,0.925000}%
\pgfsetfillcolor{currentfill}%
\pgfsetfillopacity{0.500000}%
\pgfsetlinewidth{1.003750pt}%
\definecolor{currentstroke}{rgb}{0.925000,0.925000,0.925000}%
\pgfsetstrokecolor{currentstroke}%
\pgfsetstrokeopacity{0.500000}%
\pgfsetdash{}{0pt}%
\pgfpathmoveto{\pgfqpoint{0.379073in}{1.123938in}}%
\pgfpathlineto{\pgfqpoint{2.455212in}{0.445871in}}%
\pgfpathlineto{\pgfqpoint{3.558144in}{1.577751in}}%
\pgfpathlineto{\pgfqpoint{1.599613in}{2.147018in}}%
\pgfusepath{stroke,fill}%
\end{pgfscope}%
\begin{pgfscope}%
\pgfsetrectcap%
\pgfsetroundjoin%
\pgfsetlinewidth{0.803000pt}%
\definecolor{currentstroke}{rgb}{0.000000,0.000000,0.000000}%
\pgfsetstrokecolor{currentstroke}%
\pgfsetdash{}{0pt}%
\pgfpathmoveto{\pgfqpoint{0.379073in}{1.123938in}}%
\pgfpathlineto{\pgfqpoint{2.455212in}{0.445871in}}%
\pgfusepath{stroke}%
\end{pgfscope}%
\begin{pgfscope}%
\definecolor{textcolor}{rgb}{0.000000,0.000000,0.000000}%
\pgfsetstrokecolor{textcolor}%
\pgfsetfillcolor{textcolor}%
\pgftext[x=0.730374in, y=0.408886in, left, base,rotate=341.912962]{\color{textcolor}\rmfamily\fontsize{10.000000}{12.000000}\selectfont Position X [\(\displaystyle m\)]}%
\end{pgfscope}%
\begin{pgfscope}%
\pgfsetbuttcap%
\pgfsetroundjoin%
\pgfsetlinewidth{0.803000pt}%
\definecolor{currentstroke}{rgb}{0.690196,0.690196,0.690196}%
\pgfsetstrokecolor{currentstroke}%
\pgfsetdash{}{0pt}%
\pgfpathmoveto{\pgfqpoint{0.851850in}{0.969529in}}%
\pgfpathlineto{\pgfqpoint{2.047131in}{2.016943in}}%
\pgfpathlineto{\pgfqpoint{2.049251in}{3.504213in}}%
\pgfusepath{stroke}%
\end{pgfscope}%
\begin{pgfscope}%
\pgfsetbuttcap%
\pgfsetroundjoin%
\pgfsetlinewidth{0.803000pt}%
\definecolor{currentstroke}{rgb}{0.690196,0.690196,0.690196}%
\pgfsetstrokecolor{currentstroke}%
\pgfsetdash{}{0pt}%
\pgfpathmoveto{\pgfqpoint{1.310499in}{0.819734in}}%
\pgfpathlineto{\pgfqpoint{2.480415in}{1.891004in}}%
\pgfpathlineto{\pgfqpoint{2.501443in}{3.389596in}}%
\pgfusepath{stroke}%
\end{pgfscope}%
\begin{pgfscope}%
\pgfsetbuttcap%
\pgfsetroundjoin%
\pgfsetlinewidth{0.803000pt}%
\definecolor{currentstroke}{rgb}{0.690196,0.690196,0.690196}%
\pgfsetstrokecolor{currentstroke}%
\pgfsetdash{}{0pt}%
\pgfpathmoveto{\pgfqpoint{1.780117in}{0.666357in}}%
\pgfpathlineto{\pgfqpoint{2.923187in}{1.762308in}}%
\pgfpathlineto{\pgfqpoint{2.963974in}{3.272358in}}%
\pgfusepath{stroke}%
\end{pgfscope}%
\begin{pgfscope}%
\pgfsetbuttcap%
\pgfsetroundjoin%
\pgfsetlinewidth{0.803000pt}%
\definecolor{currentstroke}{rgb}{0.690196,0.690196,0.690196}%
\pgfsetstrokecolor{currentstroke}%
\pgfsetdash{}{0pt}%
\pgfpathmoveto{\pgfqpoint{2.261103in}{0.509267in}}%
\pgfpathlineto{\pgfqpoint{3.375762in}{1.630762in}}%
\pgfpathlineto{\pgfqpoint{3.437202in}{3.152408in}}%
\pgfusepath{stroke}%
\end{pgfscope}%
\begin{pgfscope}%
\pgfsetrectcap%
\pgfsetroundjoin%
\pgfsetlinewidth{0.803000pt}%
\definecolor{currentstroke}{rgb}{0.000000,0.000000,0.000000}%
\pgfsetstrokecolor{currentstroke}%
\pgfsetdash{}{0pt}%
\pgfpathmoveto{\pgfqpoint{0.862266in}{0.978656in}}%
\pgfpathlineto{\pgfqpoint{0.830973in}{0.951234in}}%
\pgfusepath{stroke}%
\end{pgfscope}%
\begin{pgfscope}%
\definecolor{textcolor}{rgb}{0.000000,0.000000,0.000000}%
\pgfsetstrokecolor{textcolor}%
\pgfsetfillcolor{textcolor}%
\pgftext[x=0.747635in,y=0.748923in,,top]{\color{textcolor}\rmfamily\fontsize{10.000000}{12.000000}\selectfont \(\displaystyle {-10}\)}%
\end{pgfscope}%
\begin{pgfscope}%
\pgfsetrectcap%
\pgfsetroundjoin%
\pgfsetlinewidth{0.803000pt}%
\definecolor{currentstroke}{rgb}{0.000000,0.000000,0.000000}%
\pgfsetstrokecolor{currentstroke}%
\pgfsetdash{}{0pt}%
\pgfpathmoveto{\pgfqpoint{1.320703in}{0.829079in}}%
\pgfpathlineto{\pgfqpoint{1.290044in}{0.801004in}}%
\pgfusepath{stroke}%
\end{pgfscope}%
\begin{pgfscope}%
\definecolor{textcolor}{rgb}{0.000000,0.000000,0.000000}%
\pgfsetstrokecolor{textcolor}%
\pgfsetfillcolor{textcolor}%
\pgftext[x=1.206788in,y=0.595943in,,top]{\color{textcolor}\rmfamily\fontsize{10.000000}{12.000000}\selectfont \(\displaystyle {0}\)}%
\end{pgfscope}%
\begin{pgfscope}%
\pgfsetrectcap%
\pgfsetroundjoin%
\pgfsetlinewidth{0.803000pt}%
\definecolor{currentstroke}{rgb}{0.000000,0.000000,0.000000}%
\pgfsetstrokecolor{currentstroke}%
\pgfsetdash{}{0pt}%
\pgfpathmoveto{\pgfqpoint{1.790097in}{0.675926in}}%
\pgfpathlineto{\pgfqpoint{1.760111in}{0.647176in}}%
\pgfusepath{stroke}%
\end{pgfscope}%
\begin{pgfscope}%
\definecolor{textcolor}{rgb}{0.000000,0.000000,0.000000}%
\pgfsetstrokecolor{textcolor}%
\pgfsetfillcolor{textcolor}%
\pgftext[x=1.676964in,y=0.439290in,,top]{\color{textcolor}\rmfamily\fontsize{10.000000}{12.000000}\selectfont \(\displaystyle {10}\)}%
\end{pgfscope}%
\begin{pgfscope}%
\pgfsetrectcap%
\pgfsetroundjoin%
\pgfsetlinewidth{0.803000pt}%
\definecolor{currentstroke}{rgb}{0.000000,0.000000,0.000000}%
\pgfsetstrokecolor{currentstroke}%
\pgfsetdash{}{0pt}%
\pgfpathmoveto{\pgfqpoint{2.270845in}{0.519069in}}%
\pgfpathlineto{\pgfqpoint{2.241574in}{0.489619in}}%
\pgfusepath{stroke}%
\end{pgfscope}%
\begin{pgfscope}%
\definecolor{textcolor}{rgb}{0.000000,0.000000,0.000000}%
\pgfsetstrokecolor{textcolor}%
\pgfsetfillcolor{textcolor}%
\pgftext[x=2.158567in,y=0.278830in,,top]{\color{textcolor}\rmfamily\fontsize{10.000000}{12.000000}\selectfont \(\displaystyle {20}\)}%
\end{pgfscope}%
\begin{pgfscope}%
\pgfsetrectcap%
\pgfsetroundjoin%
\pgfsetlinewidth{0.803000pt}%
\definecolor{currentstroke}{rgb}{0.000000,0.000000,0.000000}%
\pgfsetstrokecolor{currentstroke}%
\pgfsetdash{}{0pt}%
\pgfpathmoveto{\pgfqpoint{3.558144in}{1.577751in}}%
\pgfpathlineto{\pgfqpoint{2.455212in}{0.445871in}}%
\pgfusepath{stroke}%
\end{pgfscope}%
\begin{pgfscope}%
\definecolor{textcolor}{rgb}{0.000000,0.000000,0.000000}%
\pgfsetstrokecolor{textcolor}%
\pgfsetfillcolor{textcolor}%
\pgftext[x=3.120747in, y=0.305657in, left, base,rotate=45.742112]{\color{textcolor}\rmfamily\fontsize{10.000000}{12.000000}\selectfont Position Y [\(\displaystyle m\)]}%
\end{pgfscope}%
\begin{pgfscope}%
\pgfsetbuttcap%
\pgfsetroundjoin%
\pgfsetlinewidth{0.803000pt}%
\definecolor{currentstroke}{rgb}{0.690196,0.690196,0.690196}%
\pgfsetstrokecolor{currentstroke}%
\pgfsetdash{}{0pt}%
\pgfpathmoveto{\pgfqpoint{0.477773in}{2.816197in}}%
\pgfpathlineto{\pgfqpoint{0.544655in}{1.262731in}}%
\pgfpathlineto{\pgfqpoint{2.605410in}{0.600011in}}%
\pgfusepath{stroke}%
\end{pgfscope}%
\begin{pgfscope}%
\pgfsetbuttcap%
\pgfsetroundjoin%
\pgfsetlinewidth{0.803000pt}%
\definecolor{currentstroke}{rgb}{0.690196,0.690196,0.690196}%
\pgfsetstrokecolor{currentstroke}%
\pgfsetdash{}{0pt}%
\pgfpathmoveto{\pgfqpoint{0.687617in}{2.969332in}}%
\pgfpathlineto{\pgfqpoint{0.744487in}{1.430234in}}%
\pgfpathlineto{\pgfqpoint{2.786438in}{0.785790in}}%
\pgfusepath{stroke}%
\end{pgfscope}%
\begin{pgfscope}%
\pgfsetbuttcap%
\pgfsetroundjoin%
\pgfsetlinewidth{0.803000pt}%
\definecolor{currentstroke}{rgb}{0.690196,0.690196,0.690196}%
\pgfsetstrokecolor{currentstroke}%
\pgfsetdash{}{0pt}%
\pgfpathmoveto{\pgfqpoint{0.891609in}{3.118196in}}%
\pgfpathlineto{\pgfqpoint{0.938984in}{1.593266in}}%
\pgfpathlineto{\pgfqpoint{2.962381in}{0.966351in}}%
\pgfusepath{stroke}%
\end{pgfscope}%
\begin{pgfscope}%
\pgfsetbuttcap%
\pgfsetroundjoin%
\pgfsetlinewidth{0.803000pt}%
\definecolor{currentstroke}{rgb}{0.690196,0.690196,0.690196}%
\pgfsetstrokecolor{currentstroke}%
\pgfsetdash{}{0pt}%
\pgfpathmoveto{\pgfqpoint{1.089990in}{3.262966in}}%
\pgfpathlineto{\pgfqpoint{1.128358in}{1.752003in}}%
\pgfpathlineto{\pgfqpoint{3.133452in}{1.141912in}}%
\pgfusepath{stroke}%
\end{pgfscope}%
\begin{pgfscope}%
\pgfsetbuttcap%
\pgfsetroundjoin%
\pgfsetlinewidth{0.803000pt}%
\definecolor{currentstroke}{rgb}{0.690196,0.690196,0.690196}%
\pgfsetstrokecolor{currentstroke}%
\pgfsetdash{}{0pt}%
\pgfpathmoveto{\pgfqpoint{1.282989in}{3.403807in}}%
\pgfpathlineto{\pgfqpoint{1.312809in}{1.906613in}}%
\pgfpathlineto{\pgfqpoint{3.299850in}{1.312677in}}%
\pgfusepath{stroke}%
\end{pgfscope}%
\begin{pgfscope}%
\pgfsetbuttcap%
\pgfsetroundjoin%
\pgfsetlinewidth{0.803000pt}%
\definecolor{currentstroke}{rgb}{0.690196,0.690196,0.690196}%
\pgfsetstrokecolor{currentstroke}%
\pgfsetdash{}{0pt}%
\pgfpathmoveto{\pgfqpoint{1.470822in}{3.540879in}}%
\pgfpathlineto{\pgfqpoint{1.492525in}{2.057255in}}%
\pgfpathlineto{\pgfqpoint{3.461764in}{1.478841in}}%
\pgfusepath{stroke}%
\end{pgfscope}%
\begin{pgfscope}%
\pgfsetrectcap%
\pgfsetroundjoin%
\pgfsetlinewidth{0.803000pt}%
\definecolor{currentstroke}{rgb}{0.000000,0.000000,0.000000}%
\pgfsetstrokecolor{currentstroke}%
\pgfsetdash{}{0pt}%
\pgfpathmoveto{\pgfqpoint{2.588049in}{0.605594in}}%
\pgfpathlineto{\pgfqpoint{2.640177in}{0.588831in}}%
\pgfusepath{stroke}%
\end{pgfscope}%
\begin{pgfscope}%
\definecolor{textcolor}{rgb}{0.000000,0.000000,0.000000}%
\pgfsetstrokecolor{textcolor}%
\pgfsetfillcolor{textcolor}%
\pgftext[x=2.783318in,y=0.414700in,,top]{\color{textcolor}\rmfamily\fontsize{10.000000}{12.000000}\selectfont \(\displaystyle {-10}\)}%
\end{pgfscope}%
\begin{pgfscope}%
\pgfsetrectcap%
\pgfsetroundjoin%
\pgfsetlinewidth{0.803000pt}%
\definecolor{currentstroke}{rgb}{0.000000,0.000000,0.000000}%
\pgfsetstrokecolor{currentstroke}%
\pgfsetdash{}{0pt}%
\pgfpathmoveto{\pgfqpoint{2.769247in}{0.791215in}}%
\pgfpathlineto{\pgfqpoint{2.820862in}{0.774925in}}%
\pgfusepath{stroke}%
\end{pgfscope}%
\begin{pgfscope}%
\definecolor{textcolor}{rgb}{0.000000,0.000000,0.000000}%
\pgfsetstrokecolor{textcolor}%
\pgfsetfillcolor{textcolor}%
\pgftext[x=2.961916in,y=0.603228in,,top]{\color{textcolor}\rmfamily\fontsize{10.000000}{12.000000}\selectfont \(\displaystyle {-5}\)}%
\end{pgfscope}%
\begin{pgfscope}%
\pgfsetrectcap%
\pgfsetroundjoin%
\pgfsetlinewidth{0.803000pt}%
\definecolor{currentstroke}{rgb}{0.000000,0.000000,0.000000}%
\pgfsetstrokecolor{currentstroke}%
\pgfsetdash{}{0pt}%
\pgfpathmoveto{\pgfqpoint{2.945359in}{0.971625in}}%
\pgfpathlineto{\pgfqpoint{2.996468in}{0.955790in}}%
\pgfusepath{stroke}%
\end{pgfscope}%
\begin{pgfscope}%
\definecolor{textcolor}{rgb}{0.000000,0.000000,0.000000}%
\pgfsetstrokecolor{textcolor}%
\pgfsetfillcolor{textcolor}%
\pgftext[x=3.135496in,y=0.786458in,,top]{\color{textcolor}\rmfamily\fontsize{10.000000}{12.000000}\selectfont \(\displaystyle {0}\)}%
\end{pgfscope}%
\begin{pgfscope}%
\pgfsetrectcap%
\pgfsetroundjoin%
\pgfsetlinewidth{0.803000pt}%
\definecolor{currentstroke}{rgb}{0.000000,0.000000,0.000000}%
\pgfsetstrokecolor{currentstroke}%
\pgfsetdash{}{0pt}%
\pgfpathmoveto{\pgfqpoint{3.116595in}{1.147041in}}%
\pgfpathlineto{\pgfqpoint{3.167207in}{1.131641in}}%
\pgfusepath{stroke}%
\end{pgfscope}%
\begin{pgfscope}%
\definecolor{textcolor}{rgb}{0.000000,0.000000,0.000000}%
\pgfsetstrokecolor{textcolor}%
\pgfsetfillcolor{textcolor}%
\pgftext[x=3.304266in,y=0.964612in,,top]{\color{textcolor}\rmfamily\fontsize{10.000000}{12.000000}\selectfont \(\displaystyle {5}\)}%
\end{pgfscope}%
\begin{pgfscope}%
\pgfsetrectcap%
\pgfsetroundjoin%
\pgfsetlinewidth{0.803000pt}%
\definecolor{currentstroke}{rgb}{0.000000,0.000000,0.000000}%
\pgfsetstrokecolor{currentstroke}%
\pgfsetdash{}{0pt}%
\pgfpathmoveto{\pgfqpoint{3.283156in}{1.317667in}}%
\pgfpathlineto{\pgfqpoint{3.333278in}{1.302685in}}%
\pgfusepath{stroke}%
\end{pgfscope}%
\begin{pgfscope}%
\definecolor{textcolor}{rgb}{0.000000,0.000000,0.000000}%
\pgfsetstrokecolor{textcolor}%
\pgfsetfillcolor{textcolor}%
\pgftext[x=3.468423in,y=1.137896in,,top]{\color{textcolor}\rmfamily\fontsize{10.000000}{12.000000}\selectfont \(\displaystyle {10}\)}%
\end{pgfscope}%
\begin{pgfscope}%
\pgfsetrectcap%
\pgfsetroundjoin%
\pgfsetlinewidth{0.803000pt}%
\definecolor{currentstroke}{rgb}{0.000000,0.000000,0.000000}%
\pgfsetstrokecolor{currentstroke}%
\pgfsetdash{}{0pt}%
\pgfpathmoveto{\pgfqpoint{3.445230in}{1.483697in}}%
\pgfpathlineto{\pgfqpoint{3.494871in}{1.469116in}}%
\pgfusepath{stroke}%
\end{pgfscope}%
\begin{pgfscope}%
\definecolor{textcolor}{rgb}{0.000000,0.000000,0.000000}%
\pgfsetstrokecolor{textcolor}%
\pgfsetfillcolor{textcolor}%
\pgftext[x=3.628155in,y=1.306508in,,top]{\color{textcolor}\rmfamily\fontsize{10.000000}{12.000000}\selectfont \(\displaystyle {15}\)}%
\end{pgfscope}%
\begin{pgfscope}%
\pgfsetrectcap%
\pgfsetroundjoin%
\pgfsetlinewidth{0.803000pt}%
\definecolor{currentstroke}{rgb}{0.000000,0.000000,0.000000}%
\pgfsetstrokecolor{currentstroke}%
\pgfsetdash{}{0pt}%
\pgfpathmoveto{\pgfqpoint{3.558144in}{1.577751in}}%
\pgfpathlineto{\pgfqpoint{3.628038in}{3.104037in}}%
\pgfusepath{stroke}%
\end{pgfscope}%
\begin{pgfscope}%
\definecolor{textcolor}{rgb}{0.000000,0.000000,0.000000}%
\pgfsetstrokecolor{textcolor}%
\pgfsetfillcolor{textcolor}%
\pgftext[x=4.167903in, y=1.963517in, left, base,rotate=87.378092]{\color{textcolor}\rmfamily\fontsize{10.000000}{12.000000}\selectfont Position Z [\(\displaystyle m\)]}%
\end{pgfscope}%
\begin{pgfscope}%
\pgfsetbuttcap%
\pgfsetroundjoin%
\pgfsetlinewidth{0.803000pt}%
\definecolor{currentstroke}{rgb}{0.690196,0.690196,0.690196}%
\pgfsetstrokecolor{currentstroke}%
\pgfsetdash{}{0pt}%
\pgfpathmoveto{\pgfqpoint{3.564692in}{1.720734in}}%
\pgfpathlineto{\pgfqpoint{1.598021in}{2.285506in}}%
\pgfpathlineto{\pgfqpoint{0.372023in}{1.270345in}}%
\pgfusepath{stroke}%
\end{pgfscope}%
\begin{pgfscope}%
\pgfsetbuttcap%
\pgfsetroundjoin%
\pgfsetlinewidth{0.803000pt}%
\definecolor{currentstroke}{rgb}{0.690196,0.690196,0.690196}%
\pgfsetstrokecolor{currentstroke}%
\pgfsetdash{}{0pt}%
\pgfpathmoveto{\pgfqpoint{3.574313in}{1.930841in}}%
\pgfpathlineto{\pgfqpoint{1.595682in}{2.488906in}}%
\pgfpathlineto{\pgfqpoint{0.361659in}{1.485567in}}%
\pgfusepath{stroke}%
\end{pgfscope}%
\begin{pgfscope}%
\pgfsetbuttcap%
\pgfsetroundjoin%
\pgfsetlinewidth{0.803000pt}%
\definecolor{currentstroke}{rgb}{0.690196,0.690196,0.690196}%
\pgfsetstrokecolor{currentstroke}%
\pgfsetdash{}{0pt}%
\pgfpathmoveto{\pgfqpoint{3.584054in}{2.143544in}}%
\pgfpathlineto{\pgfqpoint{1.593316in}{2.694699in}}%
\pgfpathlineto{\pgfqpoint{0.351161in}{1.703551in}}%
\pgfusepath{stroke}%
\end{pgfscope}%
\begin{pgfscope}%
\pgfsetbuttcap%
\pgfsetroundjoin%
\pgfsetlinewidth{0.803000pt}%
\definecolor{currentstroke}{rgb}{0.690196,0.690196,0.690196}%
\pgfsetstrokecolor{currentstroke}%
\pgfsetdash{}{0pt}%
\pgfpathmoveto{\pgfqpoint{3.593915in}{2.358892in}}%
\pgfpathlineto{\pgfqpoint{1.590921in}{2.902928in}}%
\pgfpathlineto{\pgfqpoint{0.340528in}{1.924349in}}%
\pgfusepath{stroke}%
\end{pgfscope}%
\begin{pgfscope}%
\pgfsetbuttcap%
\pgfsetroundjoin%
\pgfsetlinewidth{0.803000pt}%
\definecolor{currentstroke}{rgb}{0.690196,0.690196,0.690196}%
\pgfsetstrokecolor{currentstroke}%
\pgfsetdash{}{0pt}%
\pgfpathmoveto{\pgfqpoint{3.603900in}{2.576935in}}%
\pgfpathlineto{\pgfqpoint{1.588498in}{3.113636in}}%
\pgfpathlineto{\pgfqpoint{0.329757in}{2.148018in}}%
\pgfusepath{stroke}%
\end{pgfscope}%
\begin{pgfscope}%
\pgfsetbuttcap%
\pgfsetroundjoin%
\pgfsetlinewidth{0.803000pt}%
\definecolor{currentstroke}{rgb}{0.690196,0.690196,0.690196}%
\pgfsetstrokecolor{currentstroke}%
\pgfsetdash{}{0pt}%
\pgfpathmoveto{\pgfqpoint{3.614011in}{2.797723in}}%
\pgfpathlineto{\pgfqpoint{1.586046in}{3.326868in}}%
\pgfpathlineto{\pgfqpoint{0.318845in}{2.374612in}}%
\pgfusepath{stroke}%
\end{pgfscope}%
\begin{pgfscope}%
\pgfsetbuttcap%
\pgfsetroundjoin%
\pgfsetlinewidth{0.803000pt}%
\definecolor{currentstroke}{rgb}{0.690196,0.690196,0.690196}%
\pgfsetstrokecolor{currentstroke}%
\pgfsetdash{}{0pt}%
\pgfpathmoveto{\pgfqpoint{3.624249in}{3.021308in}}%
\pgfpathlineto{\pgfqpoint{1.583565in}{3.542669in}}%
\pgfpathlineto{\pgfqpoint{0.307790in}{2.604190in}}%
\pgfusepath{stroke}%
\end{pgfscope}%
\begin{pgfscope}%
\pgfsetrectcap%
\pgfsetroundjoin%
\pgfsetlinewidth{0.803000pt}%
\definecolor{currentstroke}{rgb}{0.000000,0.000000,0.000000}%
\pgfsetstrokecolor{currentstroke}%
\pgfsetdash{}{0pt}%
\pgfpathmoveto{\pgfqpoint{3.548183in}{1.725475in}}%
\pgfpathlineto{\pgfqpoint{3.597750in}{1.711241in}}%
\pgfusepath{stroke}%
\end{pgfscope}%
\begin{pgfscope}%
\definecolor{textcolor}{rgb}{0.000000,0.000000,0.000000}%
\pgfsetstrokecolor{textcolor}%
\pgfsetfillcolor{textcolor}%
\pgftext[x=3.819186in,y=1.756636in,,top]{\color{textcolor}\rmfamily\fontsize{10.000000}{12.000000}\selectfont \(\displaystyle {-2.0}\)}%
\end{pgfscope}%
\begin{pgfscope}%
\pgfsetrectcap%
\pgfsetroundjoin%
\pgfsetlinewidth{0.803000pt}%
\definecolor{currentstroke}{rgb}{0.000000,0.000000,0.000000}%
\pgfsetstrokecolor{currentstroke}%
\pgfsetdash{}{0pt}%
\pgfpathmoveto{\pgfqpoint{3.557699in}{1.935527in}}%
\pgfpathlineto{\pgfqpoint{3.607582in}{1.921458in}}%
\pgfusepath{stroke}%
\end{pgfscope}%
\begin{pgfscope}%
\definecolor{textcolor}{rgb}{0.000000,0.000000,0.000000}%
\pgfsetstrokecolor{textcolor}%
\pgfsetfillcolor{textcolor}%
\pgftext[x=3.830337in,y=1.966326in,,top]{\color{textcolor}\rmfamily\fontsize{10.000000}{12.000000}\selectfont \(\displaystyle {-1.5}\)}%
\end{pgfscope}%
\begin{pgfscope}%
\pgfsetrectcap%
\pgfsetroundjoin%
\pgfsetlinewidth{0.803000pt}%
\definecolor{currentstroke}{rgb}{0.000000,0.000000,0.000000}%
\pgfsetstrokecolor{currentstroke}%
\pgfsetdash{}{0pt}%
\pgfpathmoveto{\pgfqpoint{3.567333in}{2.148173in}}%
\pgfpathlineto{\pgfqpoint{3.617536in}{2.134274in}}%
\pgfusepath{stroke}%
\end{pgfscope}%
\begin{pgfscope}%
\definecolor{textcolor}{rgb}{0.000000,0.000000,0.000000}%
\pgfsetstrokecolor{textcolor}%
\pgfsetfillcolor{textcolor}%
\pgftext[x=3.841624in,y=2.178599in,,top]{\color{textcolor}\rmfamily\fontsize{10.000000}{12.000000}\selectfont \(\displaystyle {-1.0}\)}%
\end{pgfscope}%
\begin{pgfscope}%
\pgfsetrectcap%
\pgfsetroundjoin%
\pgfsetlinewidth{0.803000pt}%
\definecolor{currentstroke}{rgb}{0.000000,0.000000,0.000000}%
\pgfsetstrokecolor{currentstroke}%
\pgfsetdash{}{0pt}%
\pgfpathmoveto{\pgfqpoint{3.577087in}{2.363463in}}%
\pgfpathlineto{\pgfqpoint{3.627613in}{2.349739in}}%
\pgfusepath{stroke}%
\end{pgfscope}%
\begin{pgfscope}%
\definecolor{textcolor}{rgb}{0.000000,0.000000,0.000000}%
\pgfsetstrokecolor{textcolor}%
\pgfsetfillcolor{textcolor}%
\pgftext[x=3.853052in,y=2.393504in,,top]{\color{textcolor}\rmfamily\fontsize{10.000000}{12.000000}\selectfont \(\displaystyle {-0.5}\)}%
\end{pgfscope}%
\begin{pgfscope}%
\pgfsetrectcap%
\pgfsetroundjoin%
\pgfsetlinewidth{0.803000pt}%
\definecolor{currentstroke}{rgb}{0.000000,0.000000,0.000000}%
\pgfsetstrokecolor{currentstroke}%
\pgfsetdash{}{0pt}%
\pgfpathmoveto{\pgfqpoint{3.586962in}{2.581445in}}%
\pgfpathlineto{\pgfqpoint{3.637817in}{2.567902in}}%
\pgfusepath{stroke}%
\end{pgfscope}%
\begin{pgfscope}%
\definecolor{textcolor}{rgb}{0.000000,0.000000,0.000000}%
\pgfsetstrokecolor{textcolor}%
\pgfsetfillcolor{textcolor}%
\pgftext[x=3.864622in,y=2.611090in,,top]{\color{textcolor}\rmfamily\fontsize{10.000000}{12.000000}\selectfont \(\displaystyle {0.0}\)}%
\end{pgfscope}%
\begin{pgfscope}%
\pgfsetrectcap%
\pgfsetroundjoin%
\pgfsetlinewidth{0.803000pt}%
\definecolor{currentstroke}{rgb}{0.000000,0.000000,0.000000}%
\pgfsetstrokecolor{currentstroke}%
\pgfsetdash{}{0pt}%
\pgfpathmoveto{\pgfqpoint{3.596962in}{2.802171in}}%
\pgfpathlineto{\pgfqpoint{3.648150in}{2.788815in}}%
\pgfusepath{stroke}%
\end{pgfscope}%
\begin{pgfscope}%
\definecolor{textcolor}{rgb}{0.000000,0.000000,0.000000}%
\pgfsetstrokecolor{textcolor}%
\pgfsetfillcolor{textcolor}%
\pgftext[x=3.876338in,y=2.831406in,,top]{\color{textcolor}\rmfamily\fontsize{10.000000}{12.000000}\selectfont \(\displaystyle {0.5}\)}%
\end{pgfscope}%
\begin{pgfscope}%
\pgfsetrectcap%
\pgfsetroundjoin%
\pgfsetlinewidth{0.803000pt}%
\definecolor{currentstroke}{rgb}{0.000000,0.000000,0.000000}%
\pgfsetstrokecolor{currentstroke}%
\pgfsetdash{}{0pt}%
\pgfpathmoveto{\pgfqpoint{3.607088in}{3.025693in}}%
\pgfpathlineto{\pgfqpoint{3.658614in}{3.012529in}}%
\pgfusepath{stroke}%
\end{pgfscope}%
\begin{pgfscope}%
\definecolor{textcolor}{rgb}{0.000000,0.000000,0.000000}%
\pgfsetstrokecolor{textcolor}%
\pgfsetfillcolor{textcolor}%
\pgftext[x=3.888201in,y=3.054506in,,top]{\color{textcolor}\rmfamily\fontsize{10.000000}{12.000000}\selectfont \(\displaystyle {1.0}\)}%
\end{pgfscope}%
\begin{pgfscope}%
\pgfpathrectangle{\pgfqpoint{0.100000in}{0.212622in}}{\pgfqpoint{3.696000in}{3.696000in}}%
\pgfusepath{clip}%
\pgfsetrectcap%
\pgfsetroundjoin%
\pgfsetlinewidth{1.505625pt}%
\definecolor{currentstroke}{rgb}{0.121569,0.466667,0.705882}%
\pgfsetstrokecolor{currentstroke}%
\pgfsetdash{}{0pt}%
\pgfpathmoveto{\pgfqpoint{1.843258in}{2.325591in}}%
\pgfpathlineto{\pgfqpoint{1.689889in}{2.196742in}}%
\pgfpathlineto{\pgfqpoint{2.068980in}{2.087282in}}%
\pgfpathlineto{\pgfqpoint{2.510483in}{2.472461in}}%
\pgfpathlineto{\pgfqpoint{1.775978in}{2.675241in}}%
\pgfpathlineto{\pgfqpoint{0.994920in}{2.042484in}}%
\pgfpathlineto{\pgfqpoint{2.149844in}{1.698482in}}%
\pgfpathlineto{\pgfqpoint{3.025826in}{2.493272in}}%
\pgfpathlineto{\pgfqpoint{1.567254in}{2.890146in}}%
\pgfusepath{stroke}%
\end{pgfscope}%
\begin{pgfscope}%
\pgfpathrectangle{\pgfqpoint{0.100000in}{0.212622in}}{\pgfqpoint{3.696000in}{3.696000in}}%
\pgfusepath{clip}%
\pgfsetrectcap%
\pgfsetroundjoin%
\pgfsetlinewidth{1.505625pt}%
\definecolor{currentstroke}{rgb}{1.000000,0.000000,0.000000}%
\pgfsetstrokecolor{currentstroke}%
\pgfsetdash{}{0pt}%
\pgfpathmoveto{\pgfqpoint{1.843258in}{2.325591in}}%
\pgfpathlineto{\pgfqpoint{1.843258in}{2.325591in}}%
\pgfusepath{stroke}%
\end{pgfscope}%
\begin{pgfscope}%
\pgfpathrectangle{\pgfqpoint{0.100000in}{0.212622in}}{\pgfqpoint{3.696000in}{3.696000in}}%
\pgfusepath{clip}%
\pgfsetrectcap%
\pgfsetroundjoin%
\pgfsetlinewidth{1.505625pt}%
\definecolor{currentstroke}{rgb}{1.000000,0.000000,0.000000}%
\pgfsetstrokecolor{currentstroke}%
\pgfsetdash{}{0pt}%
\pgfpathmoveto{\pgfqpoint{1.843258in}{2.325591in}}%
\pgfpathlineto{\pgfqpoint{1.843258in}{2.325591in}}%
\pgfusepath{stroke}%
\end{pgfscope}%
\begin{pgfscope}%
\pgfpathrectangle{\pgfqpoint{0.100000in}{0.212622in}}{\pgfqpoint{3.696000in}{3.696000in}}%
\pgfusepath{clip}%
\pgfsetrectcap%
\pgfsetroundjoin%
\pgfsetlinewidth{1.505625pt}%
\definecolor{currentstroke}{rgb}{1.000000,0.000000,0.000000}%
\pgfsetstrokecolor{currentstroke}%
\pgfsetdash{}{0pt}%
\pgfpathmoveto{\pgfqpoint{1.843258in}{2.325591in}}%
\pgfpathlineto{\pgfqpoint{1.843258in}{2.325591in}}%
\pgfusepath{stroke}%
\end{pgfscope}%
\begin{pgfscope}%
\pgfpathrectangle{\pgfqpoint{0.100000in}{0.212622in}}{\pgfqpoint{3.696000in}{3.696000in}}%
\pgfusepath{clip}%
\pgfsetrectcap%
\pgfsetroundjoin%
\pgfsetlinewidth{1.505625pt}%
\definecolor{currentstroke}{rgb}{1.000000,0.000000,0.000000}%
\pgfsetstrokecolor{currentstroke}%
\pgfsetdash{}{0pt}%
\pgfpathmoveto{\pgfqpoint{1.843258in}{2.325591in}}%
\pgfpathlineto{\pgfqpoint{1.843258in}{2.325591in}}%
\pgfusepath{stroke}%
\end{pgfscope}%
\begin{pgfscope}%
\pgfpathrectangle{\pgfqpoint{0.100000in}{0.212622in}}{\pgfqpoint{3.696000in}{3.696000in}}%
\pgfusepath{clip}%
\pgfsetrectcap%
\pgfsetroundjoin%
\pgfsetlinewidth{1.505625pt}%
\definecolor{currentstroke}{rgb}{1.000000,0.000000,0.000000}%
\pgfsetstrokecolor{currentstroke}%
\pgfsetdash{}{0pt}%
\pgfpathmoveto{\pgfqpoint{1.843258in}{2.325591in}}%
\pgfpathlineto{\pgfqpoint{1.843258in}{2.325591in}}%
\pgfusepath{stroke}%
\end{pgfscope}%
\begin{pgfscope}%
\pgfpathrectangle{\pgfqpoint{0.100000in}{0.212622in}}{\pgfqpoint{3.696000in}{3.696000in}}%
\pgfusepath{clip}%
\pgfsetrectcap%
\pgfsetroundjoin%
\pgfsetlinewidth{1.505625pt}%
\definecolor{currentstroke}{rgb}{1.000000,0.000000,0.000000}%
\pgfsetstrokecolor{currentstroke}%
\pgfsetdash{}{0pt}%
\pgfpathmoveto{\pgfqpoint{1.843258in}{2.325591in}}%
\pgfpathlineto{\pgfqpoint{1.843258in}{2.325591in}}%
\pgfusepath{stroke}%
\end{pgfscope}%
\begin{pgfscope}%
\pgfpathrectangle{\pgfqpoint{0.100000in}{0.212622in}}{\pgfqpoint{3.696000in}{3.696000in}}%
\pgfusepath{clip}%
\pgfsetrectcap%
\pgfsetroundjoin%
\pgfsetlinewidth{1.505625pt}%
\definecolor{currentstroke}{rgb}{1.000000,0.000000,0.000000}%
\pgfsetstrokecolor{currentstroke}%
\pgfsetdash{}{0pt}%
\pgfpathmoveto{\pgfqpoint{1.843258in}{2.325591in}}%
\pgfpathlineto{\pgfqpoint{1.843258in}{2.325591in}}%
\pgfusepath{stroke}%
\end{pgfscope}%
\begin{pgfscope}%
\pgfpathrectangle{\pgfqpoint{0.100000in}{0.212622in}}{\pgfqpoint{3.696000in}{3.696000in}}%
\pgfusepath{clip}%
\pgfsetrectcap%
\pgfsetroundjoin%
\pgfsetlinewidth{1.505625pt}%
\definecolor{currentstroke}{rgb}{1.000000,0.000000,0.000000}%
\pgfsetstrokecolor{currentstroke}%
\pgfsetdash{}{0pt}%
\pgfpathmoveto{\pgfqpoint{1.843258in}{2.325591in}}%
\pgfpathlineto{\pgfqpoint{1.843258in}{2.325591in}}%
\pgfusepath{stroke}%
\end{pgfscope}%
\begin{pgfscope}%
\pgfpathrectangle{\pgfqpoint{0.100000in}{0.212622in}}{\pgfqpoint{3.696000in}{3.696000in}}%
\pgfusepath{clip}%
\pgfsetrectcap%
\pgfsetroundjoin%
\pgfsetlinewidth{1.505625pt}%
\definecolor{currentstroke}{rgb}{1.000000,0.000000,0.000000}%
\pgfsetstrokecolor{currentstroke}%
\pgfsetdash{}{0pt}%
\pgfpathmoveto{\pgfqpoint{1.843258in}{2.325591in}}%
\pgfpathlineto{\pgfqpoint{1.843258in}{2.325591in}}%
\pgfusepath{stroke}%
\end{pgfscope}%
\begin{pgfscope}%
\pgfpathrectangle{\pgfqpoint{0.100000in}{0.212622in}}{\pgfqpoint{3.696000in}{3.696000in}}%
\pgfusepath{clip}%
\pgfsetrectcap%
\pgfsetroundjoin%
\pgfsetlinewidth{1.505625pt}%
\definecolor{currentstroke}{rgb}{1.000000,0.000000,0.000000}%
\pgfsetstrokecolor{currentstroke}%
\pgfsetdash{}{0pt}%
\pgfpathmoveto{\pgfqpoint{1.843258in}{2.325591in}}%
\pgfpathlineto{\pgfqpoint{1.843258in}{2.325591in}}%
\pgfusepath{stroke}%
\end{pgfscope}%
\begin{pgfscope}%
\pgfpathrectangle{\pgfqpoint{0.100000in}{0.212622in}}{\pgfqpoint{3.696000in}{3.696000in}}%
\pgfusepath{clip}%
\pgfsetrectcap%
\pgfsetroundjoin%
\pgfsetlinewidth{1.505625pt}%
\definecolor{currentstroke}{rgb}{1.000000,0.000000,0.000000}%
\pgfsetstrokecolor{currentstroke}%
\pgfsetdash{}{0pt}%
\pgfpathmoveto{\pgfqpoint{1.843258in}{2.325591in}}%
\pgfpathlineto{\pgfqpoint{1.843258in}{2.325591in}}%
\pgfusepath{stroke}%
\end{pgfscope}%
\begin{pgfscope}%
\pgfpathrectangle{\pgfqpoint{0.100000in}{0.212622in}}{\pgfqpoint{3.696000in}{3.696000in}}%
\pgfusepath{clip}%
\pgfsetrectcap%
\pgfsetroundjoin%
\pgfsetlinewidth{1.505625pt}%
\definecolor{currentstroke}{rgb}{1.000000,0.000000,0.000000}%
\pgfsetstrokecolor{currentstroke}%
\pgfsetdash{}{0pt}%
\pgfpathmoveto{\pgfqpoint{1.843258in}{2.325591in}}%
\pgfpathlineto{\pgfqpoint{1.843258in}{2.325591in}}%
\pgfusepath{stroke}%
\end{pgfscope}%
\begin{pgfscope}%
\pgfpathrectangle{\pgfqpoint{0.100000in}{0.212622in}}{\pgfqpoint{3.696000in}{3.696000in}}%
\pgfusepath{clip}%
\pgfsetrectcap%
\pgfsetroundjoin%
\pgfsetlinewidth{1.505625pt}%
\definecolor{currentstroke}{rgb}{1.000000,0.000000,0.000000}%
\pgfsetstrokecolor{currentstroke}%
\pgfsetdash{}{0pt}%
\pgfpathmoveto{\pgfqpoint{1.843258in}{2.325591in}}%
\pgfpathlineto{\pgfqpoint{1.843258in}{2.325591in}}%
\pgfusepath{stroke}%
\end{pgfscope}%
\begin{pgfscope}%
\pgfpathrectangle{\pgfqpoint{0.100000in}{0.212622in}}{\pgfqpoint{3.696000in}{3.696000in}}%
\pgfusepath{clip}%
\pgfsetrectcap%
\pgfsetroundjoin%
\pgfsetlinewidth{1.505625pt}%
\definecolor{currentstroke}{rgb}{1.000000,0.000000,0.000000}%
\pgfsetstrokecolor{currentstroke}%
\pgfsetdash{}{0pt}%
\pgfpathmoveto{\pgfqpoint{1.843258in}{2.325591in}}%
\pgfpathlineto{\pgfqpoint{1.843258in}{2.325591in}}%
\pgfusepath{stroke}%
\end{pgfscope}%
\begin{pgfscope}%
\pgfpathrectangle{\pgfqpoint{0.100000in}{0.212622in}}{\pgfqpoint{3.696000in}{3.696000in}}%
\pgfusepath{clip}%
\pgfsetrectcap%
\pgfsetroundjoin%
\pgfsetlinewidth{1.505625pt}%
\definecolor{currentstroke}{rgb}{1.000000,0.000000,0.000000}%
\pgfsetstrokecolor{currentstroke}%
\pgfsetdash{}{0pt}%
\pgfpathmoveto{\pgfqpoint{1.843258in}{2.325591in}}%
\pgfpathlineto{\pgfqpoint{1.843258in}{2.325591in}}%
\pgfusepath{stroke}%
\end{pgfscope}%
\begin{pgfscope}%
\pgfpathrectangle{\pgfqpoint{0.100000in}{0.212622in}}{\pgfqpoint{3.696000in}{3.696000in}}%
\pgfusepath{clip}%
\pgfsetrectcap%
\pgfsetroundjoin%
\pgfsetlinewidth{1.505625pt}%
\definecolor{currentstroke}{rgb}{1.000000,0.000000,0.000000}%
\pgfsetstrokecolor{currentstroke}%
\pgfsetdash{}{0pt}%
\pgfpathmoveto{\pgfqpoint{1.843258in}{2.325591in}}%
\pgfpathlineto{\pgfqpoint{1.843258in}{2.325591in}}%
\pgfusepath{stroke}%
\end{pgfscope}%
\begin{pgfscope}%
\pgfpathrectangle{\pgfqpoint{0.100000in}{0.212622in}}{\pgfqpoint{3.696000in}{3.696000in}}%
\pgfusepath{clip}%
\pgfsetrectcap%
\pgfsetroundjoin%
\pgfsetlinewidth{1.505625pt}%
\definecolor{currentstroke}{rgb}{1.000000,0.000000,0.000000}%
\pgfsetstrokecolor{currentstroke}%
\pgfsetdash{}{0pt}%
\pgfpathmoveto{\pgfqpoint{1.843258in}{2.325591in}}%
\pgfpathlineto{\pgfqpoint{1.843258in}{2.325591in}}%
\pgfusepath{stroke}%
\end{pgfscope}%
\begin{pgfscope}%
\pgfpathrectangle{\pgfqpoint{0.100000in}{0.212622in}}{\pgfqpoint{3.696000in}{3.696000in}}%
\pgfusepath{clip}%
\pgfsetrectcap%
\pgfsetroundjoin%
\pgfsetlinewidth{1.505625pt}%
\definecolor{currentstroke}{rgb}{1.000000,0.000000,0.000000}%
\pgfsetstrokecolor{currentstroke}%
\pgfsetdash{}{0pt}%
\pgfpathmoveto{\pgfqpoint{1.843258in}{2.325591in}}%
\pgfpathlineto{\pgfqpoint{1.843258in}{2.325591in}}%
\pgfusepath{stroke}%
\end{pgfscope}%
\begin{pgfscope}%
\pgfpathrectangle{\pgfqpoint{0.100000in}{0.212622in}}{\pgfqpoint{3.696000in}{3.696000in}}%
\pgfusepath{clip}%
\pgfsetrectcap%
\pgfsetroundjoin%
\pgfsetlinewidth{1.505625pt}%
\definecolor{currentstroke}{rgb}{1.000000,0.000000,0.000000}%
\pgfsetstrokecolor{currentstroke}%
\pgfsetdash{}{0pt}%
\pgfpathmoveto{\pgfqpoint{1.843258in}{2.325591in}}%
\pgfpathlineto{\pgfqpoint{1.843258in}{2.325591in}}%
\pgfusepath{stroke}%
\end{pgfscope}%
\begin{pgfscope}%
\pgfpathrectangle{\pgfqpoint{0.100000in}{0.212622in}}{\pgfqpoint{3.696000in}{3.696000in}}%
\pgfusepath{clip}%
\pgfsetrectcap%
\pgfsetroundjoin%
\pgfsetlinewidth{1.505625pt}%
\definecolor{currentstroke}{rgb}{1.000000,0.000000,0.000000}%
\pgfsetstrokecolor{currentstroke}%
\pgfsetdash{}{0pt}%
\pgfpathmoveto{\pgfqpoint{1.843258in}{2.325591in}}%
\pgfpathlineto{\pgfqpoint{1.843258in}{2.325591in}}%
\pgfusepath{stroke}%
\end{pgfscope}%
\begin{pgfscope}%
\pgfpathrectangle{\pgfqpoint{0.100000in}{0.212622in}}{\pgfqpoint{3.696000in}{3.696000in}}%
\pgfusepath{clip}%
\pgfsetrectcap%
\pgfsetroundjoin%
\pgfsetlinewidth{1.505625pt}%
\definecolor{currentstroke}{rgb}{1.000000,0.000000,0.000000}%
\pgfsetstrokecolor{currentstroke}%
\pgfsetdash{}{0pt}%
\pgfpathmoveto{\pgfqpoint{1.843258in}{2.325591in}}%
\pgfpathlineto{\pgfqpoint{1.843258in}{2.325591in}}%
\pgfusepath{stroke}%
\end{pgfscope}%
\begin{pgfscope}%
\pgfpathrectangle{\pgfqpoint{0.100000in}{0.212622in}}{\pgfqpoint{3.696000in}{3.696000in}}%
\pgfusepath{clip}%
\pgfsetrectcap%
\pgfsetroundjoin%
\pgfsetlinewidth{1.505625pt}%
\definecolor{currentstroke}{rgb}{1.000000,0.000000,0.000000}%
\pgfsetstrokecolor{currentstroke}%
\pgfsetdash{}{0pt}%
\pgfpathmoveto{\pgfqpoint{1.843258in}{2.325591in}}%
\pgfpathlineto{\pgfqpoint{1.843258in}{2.325591in}}%
\pgfusepath{stroke}%
\end{pgfscope}%
\begin{pgfscope}%
\pgfpathrectangle{\pgfqpoint{0.100000in}{0.212622in}}{\pgfqpoint{3.696000in}{3.696000in}}%
\pgfusepath{clip}%
\pgfsetrectcap%
\pgfsetroundjoin%
\pgfsetlinewidth{1.505625pt}%
\definecolor{currentstroke}{rgb}{1.000000,0.000000,0.000000}%
\pgfsetstrokecolor{currentstroke}%
\pgfsetdash{}{0pt}%
\pgfpathmoveto{\pgfqpoint{1.843258in}{2.325591in}}%
\pgfpathlineto{\pgfqpoint{1.843258in}{2.325591in}}%
\pgfusepath{stroke}%
\end{pgfscope}%
\begin{pgfscope}%
\pgfpathrectangle{\pgfqpoint{0.100000in}{0.212622in}}{\pgfqpoint{3.696000in}{3.696000in}}%
\pgfusepath{clip}%
\pgfsetrectcap%
\pgfsetroundjoin%
\pgfsetlinewidth{1.505625pt}%
\definecolor{currentstroke}{rgb}{1.000000,0.000000,0.000000}%
\pgfsetstrokecolor{currentstroke}%
\pgfsetdash{}{0pt}%
\pgfpathmoveto{\pgfqpoint{1.843258in}{2.325591in}}%
\pgfpathlineto{\pgfqpoint{1.843258in}{2.325591in}}%
\pgfusepath{stroke}%
\end{pgfscope}%
\begin{pgfscope}%
\pgfpathrectangle{\pgfqpoint{0.100000in}{0.212622in}}{\pgfqpoint{3.696000in}{3.696000in}}%
\pgfusepath{clip}%
\pgfsetrectcap%
\pgfsetroundjoin%
\pgfsetlinewidth{1.505625pt}%
\definecolor{currentstroke}{rgb}{1.000000,0.000000,0.000000}%
\pgfsetstrokecolor{currentstroke}%
\pgfsetdash{}{0pt}%
\pgfpathmoveto{\pgfqpoint{1.843258in}{2.325591in}}%
\pgfpathlineto{\pgfqpoint{1.843258in}{2.325591in}}%
\pgfusepath{stroke}%
\end{pgfscope}%
\begin{pgfscope}%
\pgfpathrectangle{\pgfqpoint{0.100000in}{0.212622in}}{\pgfqpoint{3.696000in}{3.696000in}}%
\pgfusepath{clip}%
\pgfsetrectcap%
\pgfsetroundjoin%
\pgfsetlinewidth{1.505625pt}%
\definecolor{currentstroke}{rgb}{1.000000,0.000000,0.000000}%
\pgfsetstrokecolor{currentstroke}%
\pgfsetdash{}{0pt}%
\pgfpathmoveto{\pgfqpoint{1.843258in}{2.325591in}}%
\pgfpathlineto{\pgfqpoint{1.843258in}{2.325591in}}%
\pgfusepath{stroke}%
\end{pgfscope}%
\begin{pgfscope}%
\pgfpathrectangle{\pgfqpoint{0.100000in}{0.212622in}}{\pgfqpoint{3.696000in}{3.696000in}}%
\pgfusepath{clip}%
\pgfsetrectcap%
\pgfsetroundjoin%
\pgfsetlinewidth{1.505625pt}%
\definecolor{currentstroke}{rgb}{1.000000,0.000000,0.000000}%
\pgfsetstrokecolor{currentstroke}%
\pgfsetdash{}{0pt}%
\pgfpathmoveto{\pgfqpoint{1.843258in}{2.325591in}}%
\pgfpathlineto{\pgfqpoint{1.843258in}{2.325591in}}%
\pgfusepath{stroke}%
\end{pgfscope}%
\begin{pgfscope}%
\pgfpathrectangle{\pgfqpoint{0.100000in}{0.212622in}}{\pgfqpoint{3.696000in}{3.696000in}}%
\pgfusepath{clip}%
\pgfsetrectcap%
\pgfsetroundjoin%
\pgfsetlinewidth{1.505625pt}%
\definecolor{currentstroke}{rgb}{1.000000,0.000000,0.000000}%
\pgfsetstrokecolor{currentstroke}%
\pgfsetdash{}{0pt}%
\pgfpathmoveto{\pgfqpoint{1.843258in}{2.325591in}}%
\pgfpathlineto{\pgfqpoint{1.843258in}{2.325591in}}%
\pgfusepath{stroke}%
\end{pgfscope}%
\begin{pgfscope}%
\pgfpathrectangle{\pgfqpoint{0.100000in}{0.212622in}}{\pgfqpoint{3.696000in}{3.696000in}}%
\pgfusepath{clip}%
\pgfsetrectcap%
\pgfsetroundjoin%
\pgfsetlinewidth{1.505625pt}%
\definecolor{currentstroke}{rgb}{1.000000,0.000000,0.000000}%
\pgfsetstrokecolor{currentstroke}%
\pgfsetdash{}{0pt}%
\pgfpathmoveto{\pgfqpoint{1.843258in}{2.325591in}}%
\pgfpathlineto{\pgfqpoint{1.843258in}{2.325591in}}%
\pgfusepath{stroke}%
\end{pgfscope}%
\begin{pgfscope}%
\pgfpathrectangle{\pgfqpoint{0.100000in}{0.212622in}}{\pgfqpoint{3.696000in}{3.696000in}}%
\pgfusepath{clip}%
\pgfsetrectcap%
\pgfsetroundjoin%
\pgfsetlinewidth{1.505625pt}%
\definecolor{currentstroke}{rgb}{1.000000,0.000000,0.000000}%
\pgfsetstrokecolor{currentstroke}%
\pgfsetdash{}{0pt}%
\pgfpathmoveto{\pgfqpoint{1.843258in}{2.325591in}}%
\pgfpathlineto{\pgfqpoint{1.843258in}{2.325591in}}%
\pgfusepath{stroke}%
\end{pgfscope}%
\begin{pgfscope}%
\pgfpathrectangle{\pgfqpoint{0.100000in}{0.212622in}}{\pgfqpoint{3.696000in}{3.696000in}}%
\pgfusepath{clip}%
\pgfsetrectcap%
\pgfsetroundjoin%
\pgfsetlinewidth{1.505625pt}%
\definecolor{currentstroke}{rgb}{1.000000,0.000000,0.000000}%
\pgfsetstrokecolor{currentstroke}%
\pgfsetdash{}{0pt}%
\pgfpathmoveto{\pgfqpoint{1.843258in}{2.325591in}}%
\pgfpathlineto{\pgfqpoint{1.843258in}{2.325591in}}%
\pgfusepath{stroke}%
\end{pgfscope}%
\begin{pgfscope}%
\pgfpathrectangle{\pgfqpoint{0.100000in}{0.212622in}}{\pgfqpoint{3.696000in}{3.696000in}}%
\pgfusepath{clip}%
\pgfsetrectcap%
\pgfsetroundjoin%
\pgfsetlinewidth{1.505625pt}%
\definecolor{currentstroke}{rgb}{1.000000,0.000000,0.000000}%
\pgfsetstrokecolor{currentstroke}%
\pgfsetdash{}{0pt}%
\pgfpathmoveto{\pgfqpoint{1.843258in}{2.325591in}}%
\pgfpathlineto{\pgfqpoint{1.843258in}{2.325591in}}%
\pgfusepath{stroke}%
\end{pgfscope}%
\begin{pgfscope}%
\pgfpathrectangle{\pgfqpoint{0.100000in}{0.212622in}}{\pgfqpoint{3.696000in}{3.696000in}}%
\pgfusepath{clip}%
\pgfsetrectcap%
\pgfsetroundjoin%
\pgfsetlinewidth{1.505625pt}%
\definecolor{currentstroke}{rgb}{1.000000,0.000000,0.000000}%
\pgfsetstrokecolor{currentstroke}%
\pgfsetdash{}{0pt}%
\pgfpathmoveto{\pgfqpoint{1.843258in}{2.325591in}}%
\pgfpathlineto{\pgfqpoint{1.843258in}{2.325591in}}%
\pgfusepath{stroke}%
\end{pgfscope}%
\begin{pgfscope}%
\pgfpathrectangle{\pgfqpoint{0.100000in}{0.212622in}}{\pgfqpoint{3.696000in}{3.696000in}}%
\pgfusepath{clip}%
\pgfsetrectcap%
\pgfsetroundjoin%
\pgfsetlinewidth{1.505625pt}%
\definecolor{currentstroke}{rgb}{1.000000,0.000000,0.000000}%
\pgfsetstrokecolor{currentstroke}%
\pgfsetdash{}{0pt}%
\pgfpathmoveto{\pgfqpoint{1.843258in}{2.325591in}}%
\pgfpathlineto{\pgfqpoint{1.843258in}{2.325591in}}%
\pgfusepath{stroke}%
\end{pgfscope}%
\begin{pgfscope}%
\pgfpathrectangle{\pgfqpoint{0.100000in}{0.212622in}}{\pgfqpoint{3.696000in}{3.696000in}}%
\pgfusepath{clip}%
\pgfsetrectcap%
\pgfsetroundjoin%
\pgfsetlinewidth{1.505625pt}%
\definecolor{currentstroke}{rgb}{1.000000,0.000000,0.000000}%
\pgfsetstrokecolor{currentstroke}%
\pgfsetdash{}{0pt}%
\pgfpathmoveto{\pgfqpoint{1.843258in}{2.325591in}}%
\pgfpathlineto{\pgfqpoint{1.843258in}{2.325591in}}%
\pgfusepath{stroke}%
\end{pgfscope}%
\begin{pgfscope}%
\pgfpathrectangle{\pgfqpoint{0.100000in}{0.212622in}}{\pgfqpoint{3.696000in}{3.696000in}}%
\pgfusepath{clip}%
\pgfsetrectcap%
\pgfsetroundjoin%
\pgfsetlinewidth{1.505625pt}%
\definecolor{currentstroke}{rgb}{1.000000,0.000000,0.000000}%
\pgfsetstrokecolor{currentstroke}%
\pgfsetdash{}{0pt}%
\pgfpathmoveto{\pgfqpoint{1.843258in}{2.325591in}}%
\pgfpathlineto{\pgfqpoint{1.843258in}{2.325591in}}%
\pgfusepath{stroke}%
\end{pgfscope}%
\begin{pgfscope}%
\pgfpathrectangle{\pgfqpoint{0.100000in}{0.212622in}}{\pgfqpoint{3.696000in}{3.696000in}}%
\pgfusepath{clip}%
\pgfsetrectcap%
\pgfsetroundjoin%
\pgfsetlinewidth{1.505625pt}%
\definecolor{currentstroke}{rgb}{1.000000,0.000000,0.000000}%
\pgfsetstrokecolor{currentstroke}%
\pgfsetdash{}{0pt}%
\pgfpathmoveto{\pgfqpoint{1.843258in}{2.325591in}}%
\pgfpathlineto{\pgfqpoint{1.843258in}{2.325591in}}%
\pgfusepath{stroke}%
\end{pgfscope}%
\begin{pgfscope}%
\pgfpathrectangle{\pgfqpoint{0.100000in}{0.212622in}}{\pgfqpoint{3.696000in}{3.696000in}}%
\pgfusepath{clip}%
\pgfsetrectcap%
\pgfsetroundjoin%
\pgfsetlinewidth{1.505625pt}%
\definecolor{currentstroke}{rgb}{1.000000,0.000000,0.000000}%
\pgfsetstrokecolor{currentstroke}%
\pgfsetdash{}{0pt}%
\pgfpathmoveto{\pgfqpoint{1.843098in}{2.325166in}}%
\pgfpathlineto{\pgfqpoint{1.843258in}{2.325591in}}%
\pgfusepath{stroke}%
\end{pgfscope}%
\begin{pgfscope}%
\pgfpathrectangle{\pgfqpoint{0.100000in}{0.212622in}}{\pgfqpoint{3.696000in}{3.696000in}}%
\pgfusepath{clip}%
\pgfsetrectcap%
\pgfsetroundjoin%
\pgfsetlinewidth{1.505625pt}%
\definecolor{currentstroke}{rgb}{1.000000,0.000000,0.000000}%
\pgfsetstrokecolor{currentstroke}%
\pgfsetdash{}{0pt}%
\pgfpathmoveto{\pgfqpoint{1.842982in}{2.324956in}}%
\pgfpathlineto{\pgfqpoint{1.843258in}{2.325591in}}%
\pgfusepath{stroke}%
\end{pgfscope}%
\begin{pgfscope}%
\pgfpathrectangle{\pgfqpoint{0.100000in}{0.212622in}}{\pgfqpoint{3.696000in}{3.696000in}}%
\pgfusepath{clip}%
\pgfsetrectcap%
\pgfsetroundjoin%
\pgfsetlinewidth{1.505625pt}%
\definecolor{currentstroke}{rgb}{1.000000,0.000000,0.000000}%
\pgfsetstrokecolor{currentstroke}%
\pgfsetdash{}{0pt}%
\pgfpathmoveto{\pgfqpoint{1.842944in}{2.324826in}}%
\pgfpathlineto{\pgfqpoint{1.843258in}{2.325591in}}%
\pgfusepath{stroke}%
\end{pgfscope}%
\begin{pgfscope}%
\pgfpathrectangle{\pgfqpoint{0.100000in}{0.212622in}}{\pgfqpoint{3.696000in}{3.696000in}}%
\pgfusepath{clip}%
\pgfsetrectcap%
\pgfsetroundjoin%
\pgfsetlinewidth{1.505625pt}%
\definecolor{currentstroke}{rgb}{1.000000,0.000000,0.000000}%
\pgfsetstrokecolor{currentstroke}%
\pgfsetdash{}{0pt}%
\pgfpathmoveto{\pgfqpoint{1.842909in}{2.324768in}}%
\pgfpathlineto{\pgfqpoint{1.843258in}{2.325591in}}%
\pgfusepath{stroke}%
\end{pgfscope}%
\begin{pgfscope}%
\pgfpathrectangle{\pgfqpoint{0.100000in}{0.212622in}}{\pgfqpoint{3.696000in}{3.696000in}}%
\pgfusepath{clip}%
\pgfsetrectcap%
\pgfsetroundjoin%
\pgfsetlinewidth{1.505625pt}%
\definecolor{currentstroke}{rgb}{1.000000,0.000000,0.000000}%
\pgfsetstrokecolor{currentstroke}%
\pgfsetdash{}{0pt}%
\pgfpathmoveto{\pgfqpoint{1.842894in}{2.324731in}}%
\pgfpathlineto{\pgfqpoint{1.843258in}{2.325591in}}%
\pgfusepath{stroke}%
\end{pgfscope}%
\begin{pgfscope}%
\pgfpathrectangle{\pgfqpoint{0.100000in}{0.212622in}}{\pgfqpoint{3.696000in}{3.696000in}}%
\pgfusepath{clip}%
\pgfsetrectcap%
\pgfsetroundjoin%
\pgfsetlinewidth{1.505625pt}%
\definecolor{currentstroke}{rgb}{1.000000,0.000000,0.000000}%
\pgfsetstrokecolor{currentstroke}%
\pgfsetdash{}{0pt}%
\pgfpathmoveto{\pgfqpoint{1.842886in}{2.324710in}}%
\pgfpathlineto{\pgfqpoint{1.843258in}{2.325591in}}%
\pgfusepath{stroke}%
\end{pgfscope}%
\begin{pgfscope}%
\pgfpathrectangle{\pgfqpoint{0.100000in}{0.212622in}}{\pgfqpoint{3.696000in}{3.696000in}}%
\pgfusepath{clip}%
\pgfsetrectcap%
\pgfsetroundjoin%
\pgfsetlinewidth{1.505625pt}%
\definecolor{currentstroke}{rgb}{1.000000,0.000000,0.000000}%
\pgfsetstrokecolor{currentstroke}%
\pgfsetdash{}{0pt}%
\pgfpathmoveto{\pgfqpoint{1.842881in}{2.324699in}}%
\pgfpathlineto{\pgfqpoint{1.843258in}{2.325591in}}%
\pgfusepath{stroke}%
\end{pgfscope}%
\begin{pgfscope}%
\pgfpathrectangle{\pgfqpoint{0.100000in}{0.212622in}}{\pgfqpoint{3.696000in}{3.696000in}}%
\pgfusepath{clip}%
\pgfsetrectcap%
\pgfsetroundjoin%
\pgfsetlinewidth{1.505625pt}%
\definecolor{currentstroke}{rgb}{1.000000,0.000000,0.000000}%
\pgfsetstrokecolor{currentstroke}%
\pgfsetdash{}{0pt}%
\pgfpathmoveto{\pgfqpoint{1.842879in}{2.324693in}}%
\pgfpathlineto{\pgfqpoint{1.843258in}{2.325591in}}%
\pgfusepath{stroke}%
\end{pgfscope}%
\begin{pgfscope}%
\pgfpathrectangle{\pgfqpoint{0.100000in}{0.212622in}}{\pgfqpoint{3.696000in}{3.696000in}}%
\pgfusepath{clip}%
\pgfsetrectcap%
\pgfsetroundjoin%
\pgfsetlinewidth{1.505625pt}%
\definecolor{currentstroke}{rgb}{1.000000,0.000000,0.000000}%
\pgfsetstrokecolor{currentstroke}%
\pgfsetdash{}{0pt}%
\pgfpathmoveto{\pgfqpoint{1.842877in}{2.324690in}}%
\pgfpathlineto{\pgfqpoint{1.843258in}{2.325591in}}%
\pgfusepath{stroke}%
\end{pgfscope}%
\begin{pgfscope}%
\pgfpathrectangle{\pgfqpoint{0.100000in}{0.212622in}}{\pgfqpoint{3.696000in}{3.696000in}}%
\pgfusepath{clip}%
\pgfsetrectcap%
\pgfsetroundjoin%
\pgfsetlinewidth{1.505625pt}%
\definecolor{currentstroke}{rgb}{1.000000,0.000000,0.000000}%
\pgfsetstrokecolor{currentstroke}%
\pgfsetdash{}{0pt}%
\pgfpathmoveto{\pgfqpoint{1.842876in}{2.324688in}}%
\pgfpathlineto{\pgfqpoint{1.843258in}{2.325591in}}%
\pgfusepath{stroke}%
\end{pgfscope}%
\begin{pgfscope}%
\pgfpathrectangle{\pgfqpoint{0.100000in}{0.212622in}}{\pgfqpoint{3.696000in}{3.696000in}}%
\pgfusepath{clip}%
\pgfsetrectcap%
\pgfsetroundjoin%
\pgfsetlinewidth{1.505625pt}%
\definecolor{currentstroke}{rgb}{1.000000,0.000000,0.000000}%
\pgfsetstrokecolor{currentstroke}%
\pgfsetdash{}{0pt}%
\pgfpathmoveto{\pgfqpoint{1.842876in}{2.324687in}}%
\pgfpathlineto{\pgfqpoint{1.843258in}{2.325591in}}%
\pgfusepath{stroke}%
\end{pgfscope}%
\begin{pgfscope}%
\pgfpathrectangle{\pgfqpoint{0.100000in}{0.212622in}}{\pgfqpoint{3.696000in}{3.696000in}}%
\pgfusepath{clip}%
\pgfsetrectcap%
\pgfsetroundjoin%
\pgfsetlinewidth{1.505625pt}%
\definecolor{currentstroke}{rgb}{1.000000,0.000000,0.000000}%
\pgfsetstrokecolor{currentstroke}%
\pgfsetdash{}{0pt}%
\pgfpathmoveto{\pgfqpoint{1.842695in}{2.324331in}}%
\pgfpathlineto{\pgfqpoint{1.843258in}{2.325591in}}%
\pgfusepath{stroke}%
\end{pgfscope}%
\begin{pgfscope}%
\pgfpathrectangle{\pgfqpoint{0.100000in}{0.212622in}}{\pgfqpoint{3.696000in}{3.696000in}}%
\pgfusepath{clip}%
\pgfsetrectcap%
\pgfsetroundjoin%
\pgfsetlinewidth{1.505625pt}%
\definecolor{currentstroke}{rgb}{1.000000,0.000000,0.000000}%
\pgfsetstrokecolor{currentstroke}%
\pgfsetdash{}{0pt}%
\pgfpathmoveto{\pgfqpoint{1.842510in}{2.323373in}}%
\pgfpathlineto{\pgfqpoint{1.841726in}{2.324304in}}%
\pgfusepath{stroke}%
\end{pgfscope}%
\begin{pgfscope}%
\pgfpathrectangle{\pgfqpoint{0.100000in}{0.212622in}}{\pgfqpoint{3.696000in}{3.696000in}}%
\pgfusepath{clip}%
\pgfsetrectcap%
\pgfsetroundjoin%
\pgfsetlinewidth{1.505625pt}%
\definecolor{currentstroke}{rgb}{1.000000,0.000000,0.000000}%
\pgfsetstrokecolor{currentstroke}%
\pgfsetdash{}{0pt}%
\pgfpathmoveto{\pgfqpoint{1.841068in}{2.320936in}}%
\pgfpathlineto{\pgfqpoint{1.840194in}{2.323017in}}%
\pgfusepath{stroke}%
\end{pgfscope}%
\begin{pgfscope}%
\pgfpathrectangle{\pgfqpoint{0.100000in}{0.212622in}}{\pgfqpoint{3.696000in}{3.696000in}}%
\pgfusepath{clip}%
\pgfsetrectcap%
\pgfsetroundjoin%
\pgfsetlinewidth{1.505625pt}%
\definecolor{currentstroke}{rgb}{1.000000,0.000000,0.000000}%
\pgfsetstrokecolor{currentstroke}%
\pgfsetdash{}{0pt}%
\pgfpathmoveto{\pgfqpoint{1.840534in}{2.319435in}}%
\pgfpathlineto{\pgfqpoint{1.840194in}{2.323017in}}%
\pgfusepath{stroke}%
\end{pgfscope}%
\begin{pgfscope}%
\pgfpathrectangle{\pgfqpoint{0.100000in}{0.212622in}}{\pgfqpoint{3.696000in}{3.696000in}}%
\pgfusepath{clip}%
\pgfsetrectcap%
\pgfsetroundjoin%
\pgfsetlinewidth{1.505625pt}%
\definecolor{currentstroke}{rgb}{1.000000,0.000000,0.000000}%
\pgfsetstrokecolor{currentstroke}%
\pgfsetdash{}{0pt}%
\pgfpathmoveto{\pgfqpoint{1.840359in}{2.318604in}}%
\pgfpathlineto{\pgfqpoint{1.840194in}{2.323017in}}%
\pgfusepath{stroke}%
\end{pgfscope}%
\begin{pgfscope}%
\pgfpathrectangle{\pgfqpoint{0.100000in}{0.212622in}}{\pgfqpoint{3.696000in}{3.696000in}}%
\pgfusepath{clip}%
\pgfsetrectcap%
\pgfsetroundjoin%
\pgfsetlinewidth{1.505625pt}%
\definecolor{currentstroke}{rgb}{1.000000,0.000000,0.000000}%
\pgfsetstrokecolor{currentstroke}%
\pgfsetdash{}{0pt}%
\pgfpathmoveto{\pgfqpoint{1.839772in}{2.317465in}}%
\pgfpathlineto{\pgfqpoint{1.838661in}{2.321729in}}%
\pgfusepath{stroke}%
\end{pgfscope}%
\begin{pgfscope}%
\pgfpathrectangle{\pgfqpoint{0.100000in}{0.212622in}}{\pgfqpoint{3.696000in}{3.696000in}}%
\pgfusepath{clip}%
\pgfsetrectcap%
\pgfsetroundjoin%
\pgfsetlinewidth{1.505625pt}%
\definecolor{currentstroke}{rgb}{1.000000,0.000000,0.000000}%
\pgfsetstrokecolor{currentstroke}%
\pgfsetdash{}{0pt}%
\pgfpathmoveto{\pgfqpoint{1.839141in}{2.315215in}}%
\pgfpathlineto{\pgfqpoint{1.837128in}{2.320441in}}%
\pgfusepath{stroke}%
\end{pgfscope}%
\begin{pgfscope}%
\pgfpathrectangle{\pgfqpoint{0.100000in}{0.212622in}}{\pgfqpoint{3.696000in}{3.696000in}}%
\pgfusepath{clip}%
\pgfsetrectcap%
\pgfsetroundjoin%
\pgfsetlinewidth{1.505625pt}%
\definecolor{currentstroke}{rgb}{1.000000,0.000000,0.000000}%
\pgfsetstrokecolor{currentstroke}%
\pgfsetdash{}{0pt}%
\pgfpathmoveto{\pgfqpoint{1.838387in}{2.312842in}}%
\pgfpathlineto{\pgfqpoint{1.837128in}{2.320441in}}%
\pgfusepath{stroke}%
\end{pgfscope}%
\begin{pgfscope}%
\pgfpathrectangle{\pgfqpoint{0.100000in}{0.212622in}}{\pgfqpoint{3.696000in}{3.696000in}}%
\pgfusepath{clip}%
\pgfsetrectcap%
\pgfsetroundjoin%
\pgfsetlinewidth{1.505625pt}%
\definecolor{currentstroke}{rgb}{1.000000,0.000000,0.000000}%
\pgfsetstrokecolor{currentstroke}%
\pgfsetdash{}{0pt}%
\pgfpathmoveto{\pgfqpoint{1.837667in}{2.311528in}}%
\pgfpathlineto{\pgfqpoint{1.835595in}{2.319153in}}%
\pgfusepath{stroke}%
\end{pgfscope}%
\begin{pgfscope}%
\pgfpathrectangle{\pgfqpoint{0.100000in}{0.212622in}}{\pgfqpoint{3.696000in}{3.696000in}}%
\pgfusepath{clip}%
\pgfsetrectcap%
\pgfsetroundjoin%
\pgfsetlinewidth{1.505625pt}%
\definecolor{currentstroke}{rgb}{1.000000,0.000000,0.000000}%
\pgfsetstrokecolor{currentstroke}%
\pgfsetdash{}{0pt}%
\pgfpathmoveto{\pgfqpoint{1.837147in}{2.308657in}}%
\pgfpathlineto{\pgfqpoint{1.834061in}{2.317865in}}%
\pgfusepath{stroke}%
\end{pgfscope}%
\begin{pgfscope}%
\pgfpathrectangle{\pgfqpoint{0.100000in}{0.212622in}}{\pgfqpoint{3.696000in}{3.696000in}}%
\pgfusepath{clip}%
\pgfsetrectcap%
\pgfsetroundjoin%
\pgfsetlinewidth{1.505625pt}%
\definecolor{currentstroke}{rgb}{1.000000,0.000000,0.000000}%
\pgfsetstrokecolor{currentstroke}%
\pgfsetdash{}{0pt}%
\pgfpathmoveto{\pgfqpoint{1.836791in}{2.307092in}}%
\pgfpathlineto{\pgfqpoint{1.834061in}{2.317865in}}%
\pgfusepath{stroke}%
\end{pgfscope}%
\begin{pgfscope}%
\pgfpathrectangle{\pgfqpoint{0.100000in}{0.212622in}}{\pgfqpoint{3.696000in}{3.696000in}}%
\pgfusepath{clip}%
\pgfsetrectcap%
\pgfsetroundjoin%
\pgfsetlinewidth{1.505625pt}%
\definecolor{currentstroke}{rgb}{1.000000,0.000000,0.000000}%
\pgfsetstrokecolor{currentstroke}%
\pgfsetdash{}{0pt}%
\pgfpathmoveto{\pgfqpoint{1.836065in}{2.304644in}}%
\pgfpathlineto{\pgfqpoint{1.832527in}{2.316576in}}%
\pgfusepath{stroke}%
\end{pgfscope}%
\begin{pgfscope}%
\pgfpathrectangle{\pgfqpoint{0.100000in}{0.212622in}}{\pgfqpoint{3.696000in}{3.696000in}}%
\pgfusepath{clip}%
\pgfsetrectcap%
\pgfsetroundjoin%
\pgfsetlinewidth{1.505625pt}%
\definecolor{currentstroke}{rgb}{1.000000,0.000000,0.000000}%
\pgfsetstrokecolor{currentstroke}%
\pgfsetdash{}{0pt}%
\pgfpathmoveto{\pgfqpoint{1.835606in}{2.300880in}}%
\pgfpathlineto{\pgfqpoint{1.830993in}{2.315287in}}%
\pgfusepath{stroke}%
\end{pgfscope}%
\begin{pgfscope}%
\pgfpathrectangle{\pgfqpoint{0.100000in}{0.212622in}}{\pgfqpoint{3.696000in}{3.696000in}}%
\pgfusepath{clip}%
\pgfsetrectcap%
\pgfsetroundjoin%
\pgfsetlinewidth{1.505625pt}%
\definecolor{currentstroke}{rgb}{1.000000,0.000000,0.000000}%
\pgfsetstrokecolor{currentstroke}%
\pgfsetdash{}{0pt}%
\pgfpathmoveto{\pgfqpoint{1.833803in}{2.296213in}}%
\pgfpathlineto{\pgfqpoint{1.827923in}{2.312708in}}%
\pgfusepath{stroke}%
\end{pgfscope}%
\begin{pgfscope}%
\pgfpathrectangle{\pgfqpoint{0.100000in}{0.212622in}}{\pgfqpoint{3.696000in}{3.696000in}}%
\pgfusepath{clip}%
\pgfsetrectcap%
\pgfsetroundjoin%
\pgfsetlinewidth{1.505625pt}%
\definecolor{currentstroke}{rgb}{1.000000,0.000000,0.000000}%
\pgfsetstrokecolor{currentstroke}%
\pgfsetdash{}{0pt}%
\pgfpathmoveto{\pgfqpoint{1.832277in}{2.293849in}}%
\pgfpathlineto{\pgfqpoint{1.826388in}{2.311418in}}%
\pgfusepath{stroke}%
\end{pgfscope}%
\begin{pgfscope}%
\pgfpathrectangle{\pgfqpoint{0.100000in}{0.212622in}}{\pgfqpoint{3.696000in}{3.696000in}}%
\pgfusepath{clip}%
\pgfsetrectcap%
\pgfsetroundjoin%
\pgfsetlinewidth{1.505625pt}%
\definecolor{currentstroke}{rgb}{1.000000,0.000000,0.000000}%
\pgfsetstrokecolor{currentstroke}%
\pgfsetdash{}{0pt}%
\pgfpathmoveto{\pgfqpoint{1.830809in}{2.290863in}}%
\pgfpathlineto{\pgfqpoint{1.824852in}{2.310128in}}%
\pgfusepath{stroke}%
\end{pgfscope}%
\begin{pgfscope}%
\pgfpathrectangle{\pgfqpoint{0.100000in}{0.212622in}}{\pgfqpoint{3.696000in}{3.696000in}}%
\pgfusepath{clip}%
\pgfsetrectcap%
\pgfsetroundjoin%
\pgfsetlinewidth{1.505625pt}%
\definecolor{currentstroke}{rgb}{1.000000,0.000000,0.000000}%
\pgfsetstrokecolor{currentstroke}%
\pgfsetdash{}{0pt}%
\pgfpathmoveto{\pgfqpoint{1.829266in}{2.287415in}}%
\pgfpathlineto{\pgfqpoint{1.823316in}{2.308837in}}%
\pgfusepath{stroke}%
\end{pgfscope}%
\begin{pgfscope}%
\pgfpathrectangle{\pgfqpoint{0.100000in}{0.212622in}}{\pgfqpoint{3.696000in}{3.696000in}}%
\pgfusepath{clip}%
\pgfsetrectcap%
\pgfsetroundjoin%
\pgfsetlinewidth{1.505625pt}%
\definecolor{currentstroke}{rgb}{1.000000,0.000000,0.000000}%
\pgfsetstrokecolor{currentstroke}%
\pgfsetdash{}{0pt}%
\pgfpathmoveto{\pgfqpoint{1.826790in}{2.283853in}}%
\pgfpathlineto{\pgfqpoint{1.821780in}{2.307546in}}%
\pgfusepath{stroke}%
\end{pgfscope}%
\begin{pgfscope}%
\pgfpathrectangle{\pgfqpoint{0.100000in}{0.212622in}}{\pgfqpoint{3.696000in}{3.696000in}}%
\pgfusepath{clip}%
\pgfsetrectcap%
\pgfsetroundjoin%
\pgfsetlinewidth{1.505625pt}%
\definecolor{currentstroke}{rgb}{1.000000,0.000000,0.000000}%
\pgfsetstrokecolor{currentstroke}%
\pgfsetdash{}{0pt}%
\pgfpathmoveto{\pgfqpoint{1.824517in}{2.279261in}}%
\pgfpathlineto{\pgfqpoint{1.818706in}{2.304964in}}%
\pgfusepath{stroke}%
\end{pgfscope}%
\begin{pgfscope}%
\pgfpathrectangle{\pgfqpoint{0.100000in}{0.212622in}}{\pgfqpoint{3.696000in}{3.696000in}}%
\pgfusepath{clip}%
\pgfsetrectcap%
\pgfsetroundjoin%
\pgfsetlinewidth{1.505625pt}%
\definecolor{currentstroke}{rgb}{1.000000,0.000000,0.000000}%
\pgfsetstrokecolor{currentstroke}%
\pgfsetdash{}{0pt}%
\pgfpathmoveto{\pgfqpoint{1.823262in}{2.276641in}}%
\pgfpathlineto{\pgfqpoint{1.818706in}{2.304964in}}%
\pgfusepath{stroke}%
\end{pgfscope}%
\begin{pgfscope}%
\pgfpathrectangle{\pgfqpoint{0.100000in}{0.212622in}}{\pgfqpoint{3.696000in}{3.696000in}}%
\pgfusepath{clip}%
\pgfsetrectcap%
\pgfsetroundjoin%
\pgfsetlinewidth{1.505625pt}%
\definecolor{currentstroke}{rgb}{1.000000,0.000000,0.000000}%
\pgfsetstrokecolor{currentstroke}%
\pgfsetdash{}{0pt}%
\pgfpathmoveto{\pgfqpoint{1.822385in}{2.275218in}}%
\pgfpathlineto{\pgfqpoint{1.817168in}{2.303672in}}%
\pgfusepath{stroke}%
\end{pgfscope}%
\begin{pgfscope}%
\pgfpathrectangle{\pgfqpoint{0.100000in}{0.212622in}}{\pgfqpoint{3.696000in}{3.696000in}}%
\pgfusepath{clip}%
\pgfsetrectcap%
\pgfsetroundjoin%
\pgfsetlinewidth{1.505625pt}%
\definecolor{currentstroke}{rgb}{1.000000,0.000000,0.000000}%
\pgfsetstrokecolor{currentstroke}%
\pgfsetdash{}{0pt}%
\pgfpathmoveto{\pgfqpoint{1.821068in}{2.272342in}}%
\pgfpathlineto{\pgfqpoint{1.815630in}{2.302380in}}%
\pgfusepath{stroke}%
\end{pgfscope}%
\begin{pgfscope}%
\pgfpathrectangle{\pgfqpoint{0.100000in}{0.212622in}}{\pgfqpoint{3.696000in}{3.696000in}}%
\pgfusepath{clip}%
\pgfsetrectcap%
\pgfsetroundjoin%
\pgfsetlinewidth{1.505625pt}%
\definecolor{currentstroke}{rgb}{1.000000,0.000000,0.000000}%
\pgfsetstrokecolor{currentstroke}%
\pgfsetdash{}{0pt}%
\pgfpathmoveto{\pgfqpoint{1.820384in}{2.270602in}}%
\pgfpathlineto{\pgfqpoint{1.815630in}{2.302380in}}%
\pgfusepath{stroke}%
\end{pgfscope}%
\begin{pgfscope}%
\pgfpathrectangle{\pgfqpoint{0.100000in}{0.212622in}}{\pgfqpoint{3.696000in}{3.696000in}}%
\pgfusepath{clip}%
\pgfsetrectcap%
\pgfsetroundjoin%
\pgfsetlinewidth{1.505625pt}%
\definecolor{currentstroke}{rgb}{1.000000,0.000000,0.000000}%
\pgfsetstrokecolor{currentstroke}%
\pgfsetdash{}{0pt}%
\pgfpathmoveto{\pgfqpoint{1.819848in}{2.269751in}}%
\pgfpathlineto{\pgfqpoint{1.814092in}{2.301088in}}%
\pgfusepath{stroke}%
\end{pgfscope}%
\begin{pgfscope}%
\pgfpathrectangle{\pgfqpoint{0.100000in}{0.212622in}}{\pgfqpoint{3.696000in}{3.696000in}}%
\pgfusepath{clip}%
\pgfsetrectcap%
\pgfsetroundjoin%
\pgfsetlinewidth{1.505625pt}%
\definecolor{currentstroke}{rgb}{1.000000,0.000000,0.000000}%
\pgfsetstrokecolor{currentstroke}%
\pgfsetdash{}{0pt}%
\pgfpathmoveto{\pgfqpoint{1.818889in}{2.267644in}}%
\pgfpathlineto{\pgfqpoint{1.814092in}{2.301088in}}%
\pgfusepath{stroke}%
\end{pgfscope}%
\begin{pgfscope}%
\pgfpathrectangle{\pgfqpoint{0.100000in}{0.212622in}}{\pgfqpoint{3.696000in}{3.696000in}}%
\pgfusepath{clip}%
\pgfsetrectcap%
\pgfsetroundjoin%
\pgfsetlinewidth{1.505625pt}%
\definecolor{currentstroke}{rgb}{1.000000,0.000000,0.000000}%
\pgfsetstrokecolor{currentstroke}%
\pgfsetdash{}{0pt}%
\pgfpathmoveto{\pgfqpoint{1.818484in}{2.266312in}}%
\pgfpathlineto{\pgfqpoint{1.812554in}{2.299795in}}%
\pgfusepath{stroke}%
\end{pgfscope}%
\begin{pgfscope}%
\pgfpathrectangle{\pgfqpoint{0.100000in}{0.212622in}}{\pgfqpoint{3.696000in}{3.696000in}}%
\pgfusepath{clip}%
\pgfsetrectcap%
\pgfsetroundjoin%
\pgfsetlinewidth{1.505625pt}%
\definecolor{currentstroke}{rgb}{1.000000,0.000000,0.000000}%
\pgfsetstrokecolor{currentstroke}%
\pgfsetdash{}{0pt}%
\pgfpathmoveto{\pgfqpoint{1.817248in}{2.264369in}}%
\pgfpathlineto{\pgfqpoint{1.812554in}{2.299795in}}%
\pgfusepath{stroke}%
\end{pgfscope}%
\begin{pgfscope}%
\pgfpathrectangle{\pgfqpoint{0.100000in}{0.212622in}}{\pgfqpoint{3.696000in}{3.696000in}}%
\pgfusepath{clip}%
\pgfsetrectcap%
\pgfsetroundjoin%
\pgfsetlinewidth{1.505625pt}%
\definecolor{currentstroke}{rgb}{1.000000,0.000000,0.000000}%
\pgfsetstrokecolor{currentstroke}%
\pgfsetdash{}{0pt}%
\pgfpathmoveto{\pgfqpoint{1.815948in}{2.261366in}}%
\pgfpathlineto{\pgfqpoint{1.811015in}{2.298502in}}%
\pgfusepath{stroke}%
\end{pgfscope}%
\begin{pgfscope}%
\pgfpathrectangle{\pgfqpoint{0.100000in}{0.212622in}}{\pgfqpoint{3.696000in}{3.696000in}}%
\pgfusepath{clip}%
\pgfsetrectcap%
\pgfsetroundjoin%
\pgfsetlinewidth{1.505625pt}%
\definecolor{currentstroke}{rgb}{1.000000,0.000000,0.000000}%
\pgfsetstrokecolor{currentstroke}%
\pgfsetdash{}{0pt}%
\pgfpathmoveto{\pgfqpoint{1.815114in}{2.257503in}}%
\pgfpathlineto{\pgfqpoint{1.809475in}{2.297209in}}%
\pgfusepath{stroke}%
\end{pgfscope}%
\begin{pgfscope}%
\pgfpathrectangle{\pgfqpoint{0.100000in}{0.212622in}}{\pgfqpoint{3.696000in}{3.696000in}}%
\pgfusepath{clip}%
\pgfsetrectcap%
\pgfsetroundjoin%
\pgfsetlinewidth{1.505625pt}%
\definecolor{currentstroke}{rgb}{1.000000,0.000000,0.000000}%
\pgfsetstrokecolor{currentstroke}%
\pgfsetdash{}{0pt}%
\pgfpathmoveto{\pgfqpoint{1.812306in}{2.253081in}}%
\pgfpathlineto{\pgfqpoint{1.806396in}{2.294622in}}%
\pgfusepath{stroke}%
\end{pgfscope}%
\begin{pgfscope}%
\pgfpathrectangle{\pgfqpoint{0.100000in}{0.212622in}}{\pgfqpoint{3.696000in}{3.696000in}}%
\pgfusepath{clip}%
\pgfsetrectcap%
\pgfsetroundjoin%
\pgfsetlinewidth{1.505625pt}%
\definecolor{currentstroke}{rgb}{1.000000,0.000000,0.000000}%
\pgfsetstrokecolor{currentstroke}%
\pgfsetdash{}{0pt}%
\pgfpathmoveto{\pgfqpoint{1.810826in}{2.250636in}}%
\pgfpathlineto{\pgfqpoint{1.804855in}{2.293328in}}%
\pgfusepath{stroke}%
\end{pgfscope}%
\begin{pgfscope}%
\pgfpathrectangle{\pgfqpoint{0.100000in}{0.212622in}}{\pgfqpoint{3.696000in}{3.696000in}}%
\pgfusepath{clip}%
\pgfsetrectcap%
\pgfsetroundjoin%
\pgfsetlinewidth{1.505625pt}%
\definecolor{currentstroke}{rgb}{1.000000,0.000000,0.000000}%
\pgfsetstrokecolor{currentstroke}%
\pgfsetdash{}{0pt}%
\pgfpathmoveto{\pgfqpoint{1.810043in}{2.246241in}}%
\pgfpathlineto{\pgfqpoint{1.803315in}{2.292033in}}%
\pgfusepath{stroke}%
\end{pgfscope}%
\begin{pgfscope}%
\pgfpathrectangle{\pgfqpoint{0.100000in}{0.212622in}}{\pgfqpoint{3.696000in}{3.696000in}}%
\pgfusepath{clip}%
\pgfsetrectcap%
\pgfsetroundjoin%
\pgfsetlinewidth{1.505625pt}%
\definecolor{currentstroke}{rgb}{1.000000,0.000000,0.000000}%
\pgfsetstrokecolor{currentstroke}%
\pgfsetdash{}{0pt}%
\pgfpathmoveto{\pgfqpoint{1.808112in}{2.242029in}}%
\pgfpathlineto{\pgfqpoint{1.800232in}{2.289444in}}%
\pgfusepath{stroke}%
\end{pgfscope}%
\begin{pgfscope}%
\pgfpathrectangle{\pgfqpoint{0.100000in}{0.212622in}}{\pgfqpoint{3.696000in}{3.696000in}}%
\pgfusepath{clip}%
\pgfsetrectcap%
\pgfsetroundjoin%
\pgfsetlinewidth{1.505625pt}%
\definecolor{currentstroke}{rgb}{1.000000,0.000000,0.000000}%
\pgfsetstrokecolor{currentstroke}%
\pgfsetdash{}{0pt}%
\pgfpathmoveto{\pgfqpoint{1.805276in}{2.237524in}}%
\pgfpathlineto{\pgfqpoint{1.798691in}{2.288149in}}%
\pgfusepath{stroke}%
\end{pgfscope}%
\begin{pgfscope}%
\pgfpathrectangle{\pgfqpoint{0.100000in}{0.212622in}}{\pgfqpoint{3.696000in}{3.696000in}}%
\pgfusepath{clip}%
\pgfsetrectcap%
\pgfsetroundjoin%
\pgfsetlinewidth{1.505625pt}%
\definecolor{currentstroke}{rgb}{1.000000,0.000000,0.000000}%
\pgfsetstrokecolor{currentstroke}%
\pgfsetdash{}{0pt}%
\pgfpathmoveto{\pgfqpoint{1.803157in}{2.230277in}}%
\pgfpathlineto{\pgfqpoint{1.795606in}{2.285557in}}%
\pgfusepath{stroke}%
\end{pgfscope}%
\begin{pgfscope}%
\pgfpathrectangle{\pgfqpoint{0.100000in}{0.212622in}}{\pgfqpoint{3.696000in}{3.696000in}}%
\pgfusepath{clip}%
\pgfsetrectcap%
\pgfsetroundjoin%
\pgfsetlinewidth{1.505625pt}%
\definecolor{currentstroke}{rgb}{1.000000,0.000000,0.000000}%
\pgfsetstrokecolor{currentstroke}%
\pgfsetdash{}{0pt}%
\pgfpathmoveto{\pgfqpoint{1.800771in}{2.222678in}}%
\pgfpathlineto{\pgfqpoint{1.790977in}{2.281668in}}%
\pgfusepath{stroke}%
\end{pgfscope}%
\begin{pgfscope}%
\pgfpathrectangle{\pgfqpoint{0.100000in}{0.212622in}}{\pgfqpoint{3.696000in}{3.696000in}}%
\pgfusepath{clip}%
\pgfsetrectcap%
\pgfsetroundjoin%
\pgfsetlinewidth{1.505625pt}%
\definecolor{currentstroke}{rgb}{1.000000,0.000000,0.000000}%
\pgfsetstrokecolor{currentstroke}%
\pgfsetdash{}{0pt}%
\pgfpathmoveto{\pgfqpoint{1.798473in}{2.218650in}}%
\pgfpathlineto{\pgfqpoint{1.789433in}{2.280371in}}%
\pgfusepath{stroke}%
\end{pgfscope}%
\begin{pgfscope}%
\pgfpathrectangle{\pgfqpoint{0.100000in}{0.212622in}}{\pgfqpoint{3.696000in}{3.696000in}}%
\pgfusepath{clip}%
\pgfsetrectcap%
\pgfsetroundjoin%
\pgfsetlinewidth{1.505625pt}%
\definecolor{currentstroke}{rgb}{1.000000,0.000000,0.000000}%
\pgfsetstrokecolor{currentstroke}%
\pgfsetdash{}{0pt}%
\pgfpathmoveto{\pgfqpoint{1.796496in}{2.213205in}}%
\pgfpathlineto{\pgfqpoint{1.786344in}{2.277776in}}%
\pgfusepath{stroke}%
\end{pgfscope}%
\begin{pgfscope}%
\pgfpathrectangle{\pgfqpoint{0.100000in}{0.212622in}}{\pgfqpoint{3.696000in}{3.696000in}}%
\pgfusepath{clip}%
\pgfsetrectcap%
\pgfsetroundjoin%
\pgfsetlinewidth{1.505625pt}%
\definecolor{currentstroke}{rgb}{1.000000,0.000000,0.000000}%
\pgfsetstrokecolor{currentstroke}%
\pgfsetdash{}{0pt}%
\pgfpathmoveto{\pgfqpoint{1.795499in}{2.206338in}}%
\pgfpathlineto{\pgfqpoint{1.783254in}{2.275180in}}%
\pgfusepath{stroke}%
\end{pgfscope}%
\begin{pgfscope}%
\pgfpathrectangle{\pgfqpoint{0.100000in}{0.212622in}}{\pgfqpoint{3.696000in}{3.696000in}}%
\pgfusepath{clip}%
\pgfsetrectcap%
\pgfsetroundjoin%
\pgfsetlinewidth{1.505625pt}%
\definecolor{currentstroke}{rgb}{1.000000,0.000000,0.000000}%
\pgfsetstrokecolor{currentstroke}%
\pgfsetdash{}{0pt}%
\pgfpathmoveto{\pgfqpoint{1.791817in}{2.199577in}}%
\pgfpathlineto{\pgfqpoint{1.780163in}{2.272583in}}%
\pgfusepath{stroke}%
\end{pgfscope}%
\begin{pgfscope}%
\pgfpathrectangle{\pgfqpoint{0.100000in}{0.212622in}}{\pgfqpoint{3.696000in}{3.696000in}}%
\pgfusepath{clip}%
\pgfsetrectcap%
\pgfsetroundjoin%
\pgfsetlinewidth{1.505625pt}%
\definecolor{currentstroke}{rgb}{1.000000,0.000000,0.000000}%
\pgfsetstrokecolor{currentstroke}%
\pgfsetdash{}{0pt}%
\pgfpathmoveto{\pgfqpoint{1.787990in}{2.192768in}}%
\pgfpathlineto{\pgfqpoint{1.777070in}{2.269984in}}%
\pgfusepath{stroke}%
\end{pgfscope}%
\begin{pgfscope}%
\pgfpathrectangle{\pgfqpoint{0.100000in}{0.212622in}}{\pgfqpoint{3.696000in}{3.696000in}}%
\pgfusepath{clip}%
\pgfsetrectcap%
\pgfsetroundjoin%
\pgfsetlinewidth{1.505625pt}%
\definecolor{currentstroke}{rgb}{1.000000,0.000000,0.000000}%
\pgfsetstrokecolor{currentstroke}%
\pgfsetdash{}{0pt}%
\pgfpathmoveto{\pgfqpoint{1.786242in}{2.183661in}}%
\pgfpathlineto{\pgfqpoint{1.772428in}{2.266084in}}%
\pgfusepath{stroke}%
\end{pgfscope}%
\begin{pgfscope}%
\pgfpathrectangle{\pgfqpoint{0.100000in}{0.212622in}}{\pgfqpoint{3.696000in}{3.696000in}}%
\pgfusepath{clip}%
\pgfsetrectcap%
\pgfsetroundjoin%
\pgfsetlinewidth{1.505625pt}%
\definecolor{currentstroke}{rgb}{1.000000,0.000000,0.000000}%
\pgfsetstrokecolor{currentstroke}%
\pgfsetdash{}{0pt}%
\pgfpathmoveto{\pgfqpoint{1.782335in}{2.175006in}}%
\pgfpathlineto{\pgfqpoint{1.767782in}{2.262182in}}%
\pgfusepath{stroke}%
\end{pgfscope}%
\begin{pgfscope}%
\pgfpathrectangle{\pgfqpoint{0.100000in}{0.212622in}}{\pgfqpoint{3.696000in}{3.696000in}}%
\pgfusepath{clip}%
\pgfsetrectcap%
\pgfsetroundjoin%
\pgfsetlinewidth{1.505625pt}%
\definecolor{currentstroke}{rgb}{1.000000,0.000000,0.000000}%
\pgfsetstrokecolor{currentstroke}%
\pgfsetdash{}{0pt}%
\pgfpathmoveto{\pgfqpoint{1.779514in}{2.170057in}}%
\pgfpathlineto{\pgfqpoint{1.766233in}{2.260880in}}%
\pgfusepath{stroke}%
\end{pgfscope}%
\begin{pgfscope}%
\pgfpathrectangle{\pgfqpoint{0.100000in}{0.212622in}}{\pgfqpoint{3.696000in}{3.696000in}}%
\pgfusepath{clip}%
\pgfsetrectcap%
\pgfsetroundjoin%
\pgfsetlinewidth{1.505625pt}%
\definecolor{currentstroke}{rgb}{1.000000,0.000000,0.000000}%
\pgfsetstrokecolor{currentstroke}%
\pgfsetdash{}{0pt}%
\pgfpathmoveto{\pgfqpoint{1.776965in}{2.163614in}}%
\pgfpathlineto{\pgfqpoint{1.763134in}{2.258276in}}%
\pgfusepath{stroke}%
\end{pgfscope}%
\begin{pgfscope}%
\pgfpathrectangle{\pgfqpoint{0.100000in}{0.212622in}}{\pgfqpoint{3.696000in}{3.696000in}}%
\pgfusepath{clip}%
\pgfsetrectcap%
\pgfsetroundjoin%
\pgfsetlinewidth{1.505625pt}%
\definecolor{currentstroke}{rgb}{1.000000,0.000000,0.000000}%
\pgfsetstrokecolor{currentstroke}%
\pgfsetdash{}{0pt}%
\pgfpathmoveto{\pgfqpoint{1.774312in}{2.156352in}}%
\pgfpathlineto{\pgfqpoint{1.758482in}{2.254369in}}%
\pgfusepath{stroke}%
\end{pgfscope}%
\begin{pgfscope}%
\pgfpathrectangle{\pgfqpoint{0.100000in}{0.212622in}}{\pgfqpoint{3.696000in}{3.696000in}}%
\pgfusepath{clip}%
\pgfsetrectcap%
\pgfsetroundjoin%
\pgfsetlinewidth{1.505625pt}%
\definecolor{currentstroke}{rgb}{1.000000,0.000000,0.000000}%
\pgfsetstrokecolor{currentstroke}%
\pgfsetdash{}{0pt}%
\pgfpathmoveto{\pgfqpoint{1.769968in}{2.149189in}}%
\pgfpathlineto{\pgfqpoint{1.755379in}{2.251762in}}%
\pgfusepath{stroke}%
\end{pgfscope}%
\begin{pgfscope}%
\pgfpathrectangle{\pgfqpoint{0.100000in}{0.212622in}}{\pgfqpoint{3.696000in}{3.696000in}}%
\pgfusepath{clip}%
\pgfsetrectcap%
\pgfsetroundjoin%
\pgfsetlinewidth{1.505625pt}%
\definecolor{currentstroke}{rgb}{1.000000,0.000000,0.000000}%
\pgfsetstrokecolor{currentstroke}%
\pgfsetdash{}{0pt}%
\pgfpathmoveto{\pgfqpoint{1.767973in}{2.145257in}}%
\pgfpathlineto{\pgfqpoint{1.753827in}{2.250458in}}%
\pgfusepath{stroke}%
\end{pgfscope}%
\begin{pgfscope}%
\pgfpathrectangle{\pgfqpoint{0.100000in}{0.212622in}}{\pgfqpoint{3.696000in}{3.696000in}}%
\pgfusepath{clip}%
\pgfsetrectcap%
\pgfsetroundjoin%
\pgfsetlinewidth{1.505625pt}%
\definecolor{currentstroke}{rgb}{1.000000,0.000000,0.000000}%
\pgfsetstrokecolor{currentstroke}%
\pgfsetdash{}{0pt}%
\pgfpathmoveto{\pgfqpoint{1.766491in}{2.139917in}}%
\pgfpathlineto{\pgfqpoint{1.750722in}{2.247849in}}%
\pgfusepath{stroke}%
\end{pgfscope}%
\begin{pgfscope}%
\pgfpathrectangle{\pgfqpoint{0.100000in}{0.212622in}}{\pgfqpoint{3.696000in}{3.696000in}}%
\pgfusepath{clip}%
\pgfsetrectcap%
\pgfsetroundjoin%
\pgfsetlinewidth{1.505625pt}%
\definecolor{currentstroke}{rgb}{1.000000,0.000000,0.000000}%
\pgfsetstrokecolor{currentstroke}%
\pgfsetdash{}{0pt}%
\pgfpathmoveto{\pgfqpoint{1.763134in}{2.134407in}}%
\pgfpathlineto{\pgfqpoint{1.747616in}{2.245240in}}%
\pgfusepath{stroke}%
\end{pgfscope}%
\begin{pgfscope}%
\pgfpathrectangle{\pgfqpoint{0.100000in}{0.212622in}}{\pgfqpoint{3.696000in}{3.696000in}}%
\pgfusepath{clip}%
\pgfsetrectcap%
\pgfsetroundjoin%
\pgfsetlinewidth{1.505625pt}%
\definecolor{currentstroke}{rgb}{1.000000,0.000000,0.000000}%
\pgfsetstrokecolor{currentstroke}%
\pgfsetdash{}{0pt}%
\pgfpathmoveto{\pgfqpoint{1.761280in}{2.131591in}}%
\pgfpathlineto{\pgfqpoint{1.746062in}{2.243934in}}%
\pgfusepath{stroke}%
\end{pgfscope}%
\begin{pgfscope}%
\pgfpathrectangle{\pgfqpoint{0.100000in}{0.212622in}}{\pgfqpoint{3.696000in}{3.696000in}}%
\pgfusepath{clip}%
\pgfsetrectcap%
\pgfsetroundjoin%
\pgfsetlinewidth{1.505625pt}%
\definecolor{currentstroke}{rgb}{1.000000,0.000000,0.000000}%
\pgfsetstrokecolor{currentstroke}%
\pgfsetdash{}{0pt}%
\pgfpathmoveto{\pgfqpoint{1.759848in}{2.126617in}}%
\pgfpathlineto{\pgfqpoint{1.744508in}{2.242629in}}%
\pgfusepath{stroke}%
\end{pgfscope}%
\begin{pgfscope}%
\pgfpathrectangle{\pgfqpoint{0.100000in}{0.212622in}}{\pgfqpoint{3.696000in}{3.696000in}}%
\pgfusepath{clip}%
\pgfsetrectcap%
\pgfsetroundjoin%
\pgfsetlinewidth{1.505625pt}%
\definecolor{currentstroke}{rgb}{1.000000,0.000000,0.000000}%
\pgfsetstrokecolor{currentstroke}%
\pgfsetdash{}{0pt}%
\pgfpathmoveto{\pgfqpoint{1.758530in}{2.124220in}}%
\pgfpathlineto{\pgfqpoint{1.742954in}{2.241323in}}%
\pgfusepath{stroke}%
\end{pgfscope}%
\begin{pgfscope}%
\pgfpathrectangle{\pgfqpoint{0.100000in}{0.212622in}}{\pgfqpoint{3.696000in}{3.696000in}}%
\pgfusepath{clip}%
\pgfsetrectcap%
\pgfsetroundjoin%
\pgfsetlinewidth{1.505625pt}%
\definecolor{currentstroke}{rgb}{1.000000,0.000000,0.000000}%
\pgfsetstrokecolor{currentstroke}%
\pgfsetdash{}{0pt}%
\pgfpathmoveto{\pgfqpoint{1.757723in}{2.123013in}}%
\pgfpathlineto{\pgfqpoint{1.741399in}{2.240016in}}%
\pgfusepath{stroke}%
\end{pgfscope}%
\begin{pgfscope}%
\pgfpathrectangle{\pgfqpoint{0.100000in}{0.212622in}}{\pgfqpoint{3.696000in}{3.696000in}}%
\pgfusepath{clip}%
\pgfsetrectcap%
\pgfsetroundjoin%
\pgfsetlinewidth{1.505625pt}%
\definecolor{currentstroke}{rgb}{1.000000,0.000000,0.000000}%
\pgfsetstrokecolor{currentstroke}%
\pgfsetdash{}{0pt}%
\pgfpathmoveto{\pgfqpoint{1.756676in}{2.120266in}}%
\pgfpathlineto{\pgfqpoint{1.741399in}{2.240016in}}%
\pgfusepath{stroke}%
\end{pgfscope}%
\begin{pgfscope}%
\pgfpathrectangle{\pgfqpoint{0.100000in}{0.212622in}}{\pgfqpoint{3.696000in}{3.696000in}}%
\pgfusepath{clip}%
\pgfsetrectcap%
\pgfsetroundjoin%
\pgfsetlinewidth{1.505625pt}%
\definecolor{currentstroke}{rgb}{1.000000,0.000000,0.000000}%
\pgfsetstrokecolor{currentstroke}%
\pgfsetdash{}{0pt}%
\pgfpathmoveto{\pgfqpoint{1.755596in}{2.116998in}}%
\pgfpathlineto{\pgfqpoint{1.739844in}{2.238710in}}%
\pgfusepath{stroke}%
\end{pgfscope}%
\begin{pgfscope}%
\pgfpathrectangle{\pgfqpoint{0.100000in}{0.212622in}}{\pgfqpoint{3.696000in}{3.696000in}}%
\pgfusepath{clip}%
\pgfsetrectcap%
\pgfsetroundjoin%
\pgfsetlinewidth{1.505625pt}%
\definecolor{currentstroke}{rgb}{1.000000,0.000000,0.000000}%
\pgfsetstrokecolor{currentstroke}%
\pgfsetdash{}{0pt}%
\pgfpathmoveto{\pgfqpoint{1.753442in}{2.113232in}}%
\pgfpathlineto{\pgfqpoint{1.736732in}{2.236096in}}%
\pgfusepath{stroke}%
\end{pgfscope}%
\begin{pgfscope}%
\pgfpathrectangle{\pgfqpoint{0.100000in}{0.212622in}}{\pgfqpoint{3.696000in}{3.696000in}}%
\pgfusepath{clip}%
\pgfsetrectcap%
\pgfsetroundjoin%
\pgfsetlinewidth{1.505625pt}%
\definecolor{currentstroke}{rgb}{1.000000,0.000000,0.000000}%
\pgfsetstrokecolor{currentstroke}%
\pgfsetdash{}{0pt}%
\pgfpathmoveto{\pgfqpoint{1.752381in}{2.111349in}}%
\pgfpathlineto{\pgfqpoint{1.736732in}{2.236096in}}%
\pgfusepath{stroke}%
\end{pgfscope}%
\begin{pgfscope}%
\pgfpathrectangle{\pgfqpoint{0.100000in}{0.212622in}}{\pgfqpoint{3.696000in}{3.696000in}}%
\pgfusepath{clip}%
\pgfsetrectcap%
\pgfsetroundjoin%
\pgfsetlinewidth{1.505625pt}%
\definecolor{currentstroke}{rgb}{1.000000,0.000000,0.000000}%
\pgfsetstrokecolor{currentstroke}%
\pgfsetdash{}{0pt}%
\pgfpathmoveto{\pgfqpoint{1.752114in}{2.110032in}}%
\pgfpathlineto{\pgfqpoint{1.735176in}{2.234788in}}%
\pgfusepath{stroke}%
\end{pgfscope}%
\begin{pgfscope}%
\pgfpathrectangle{\pgfqpoint{0.100000in}{0.212622in}}{\pgfqpoint{3.696000in}{3.696000in}}%
\pgfusepath{clip}%
\pgfsetrectcap%
\pgfsetroundjoin%
\pgfsetlinewidth{1.505625pt}%
\definecolor{currentstroke}{rgb}{1.000000,0.000000,0.000000}%
\pgfsetstrokecolor{currentstroke}%
\pgfsetdash{}{0pt}%
\pgfpathmoveto{\pgfqpoint{1.750886in}{2.107907in}}%
\pgfpathlineto{\pgfqpoint{1.735176in}{2.234788in}}%
\pgfusepath{stroke}%
\end{pgfscope}%
\begin{pgfscope}%
\pgfpathrectangle{\pgfqpoint{0.100000in}{0.212622in}}{\pgfqpoint{3.696000in}{3.696000in}}%
\pgfusepath{clip}%
\pgfsetrectcap%
\pgfsetroundjoin%
\pgfsetlinewidth{1.505625pt}%
\definecolor{currentstroke}{rgb}{1.000000,0.000000,0.000000}%
\pgfsetstrokecolor{currentstroke}%
\pgfsetdash{}{0pt}%
\pgfpathmoveto{\pgfqpoint{1.750168in}{2.106745in}}%
\pgfpathlineto{\pgfqpoint{1.733619in}{2.233481in}}%
\pgfusepath{stroke}%
\end{pgfscope}%
\begin{pgfscope}%
\pgfpathrectangle{\pgfqpoint{0.100000in}{0.212622in}}{\pgfqpoint{3.696000in}{3.696000in}}%
\pgfusepath{clip}%
\pgfsetrectcap%
\pgfsetroundjoin%
\pgfsetlinewidth{1.505625pt}%
\definecolor{currentstroke}{rgb}{1.000000,0.000000,0.000000}%
\pgfsetstrokecolor{currentstroke}%
\pgfsetdash{}{0pt}%
\pgfpathmoveto{\pgfqpoint{1.749210in}{2.103785in}}%
\pgfpathlineto{\pgfqpoint{1.732062in}{2.232173in}}%
\pgfusepath{stroke}%
\end{pgfscope}%
\begin{pgfscope}%
\pgfpathrectangle{\pgfqpoint{0.100000in}{0.212622in}}{\pgfqpoint{3.696000in}{3.696000in}}%
\pgfusepath{clip}%
\pgfsetrectcap%
\pgfsetroundjoin%
\pgfsetlinewidth{1.505625pt}%
\definecolor{currentstroke}{rgb}{1.000000,0.000000,0.000000}%
\pgfsetstrokecolor{currentstroke}%
\pgfsetdash{}{0pt}%
\pgfpathmoveto{\pgfqpoint{1.748530in}{2.102299in}}%
\pgfpathlineto{\pgfqpoint{1.732062in}{2.232173in}}%
\pgfusepath{stroke}%
\end{pgfscope}%
\begin{pgfscope}%
\pgfpathrectangle{\pgfqpoint{0.100000in}{0.212622in}}{\pgfqpoint{3.696000in}{3.696000in}}%
\pgfusepath{clip}%
\pgfsetrectcap%
\pgfsetroundjoin%
\pgfsetlinewidth{1.505625pt}%
\definecolor{currentstroke}{rgb}{1.000000,0.000000,0.000000}%
\pgfsetstrokecolor{currentstroke}%
\pgfsetdash{}{0pt}%
\pgfpathmoveto{\pgfqpoint{1.748045in}{2.101512in}}%
\pgfpathlineto{\pgfqpoint{1.730505in}{2.230864in}}%
\pgfusepath{stroke}%
\end{pgfscope}%
\begin{pgfscope}%
\pgfpathrectangle{\pgfqpoint{0.100000in}{0.212622in}}{\pgfqpoint{3.696000in}{3.696000in}}%
\pgfusepath{clip}%
\pgfsetrectcap%
\pgfsetroundjoin%
\pgfsetlinewidth{1.505625pt}%
\definecolor{currentstroke}{rgb}{1.000000,0.000000,0.000000}%
\pgfsetstrokecolor{currentstroke}%
\pgfsetdash{}{0pt}%
\pgfpathmoveto{\pgfqpoint{1.747253in}{2.099415in}}%
\pgfpathlineto{\pgfqpoint{1.730505in}{2.230864in}}%
\pgfusepath{stroke}%
\end{pgfscope}%
\begin{pgfscope}%
\pgfpathrectangle{\pgfqpoint{0.100000in}{0.212622in}}{\pgfqpoint{3.696000in}{3.696000in}}%
\pgfusepath{clip}%
\pgfsetrectcap%
\pgfsetroundjoin%
\pgfsetlinewidth{1.505625pt}%
\definecolor{currentstroke}{rgb}{1.000000,0.000000,0.000000}%
\pgfsetstrokecolor{currentstroke}%
\pgfsetdash{}{0pt}%
\pgfpathmoveto{\pgfqpoint{1.746302in}{2.096884in}}%
\pgfpathlineto{\pgfqpoint{1.728947in}{2.229556in}}%
\pgfusepath{stroke}%
\end{pgfscope}%
\begin{pgfscope}%
\pgfpathrectangle{\pgfqpoint{0.100000in}{0.212622in}}{\pgfqpoint{3.696000in}{3.696000in}}%
\pgfusepath{clip}%
\pgfsetrectcap%
\pgfsetroundjoin%
\pgfsetlinewidth{1.505625pt}%
\definecolor{currentstroke}{rgb}{1.000000,0.000000,0.000000}%
\pgfsetstrokecolor{currentstroke}%
\pgfsetdash{}{0pt}%
\pgfpathmoveto{\pgfqpoint{1.745541in}{2.095648in}}%
\pgfpathlineto{\pgfqpoint{1.728947in}{2.229556in}}%
\pgfusepath{stroke}%
\end{pgfscope}%
\begin{pgfscope}%
\pgfpathrectangle{\pgfqpoint{0.100000in}{0.212622in}}{\pgfqpoint{3.696000in}{3.696000in}}%
\pgfusepath{clip}%
\pgfsetrectcap%
\pgfsetroundjoin%
\pgfsetlinewidth{1.505625pt}%
\definecolor{currentstroke}{rgb}{1.000000,0.000000,0.000000}%
\pgfsetstrokecolor{currentstroke}%
\pgfsetdash{}{0pt}%
\pgfpathmoveto{\pgfqpoint{1.745139in}{2.094984in}}%
\pgfpathlineto{\pgfqpoint{1.727389in}{2.228247in}}%
\pgfusepath{stroke}%
\end{pgfscope}%
\begin{pgfscope}%
\pgfpathrectangle{\pgfqpoint{0.100000in}{0.212622in}}{\pgfqpoint{3.696000in}{3.696000in}}%
\pgfusepath{clip}%
\pgfsetrectcap%
\pgfsetroundjoin%
\pgfsetlinewidth{1.505625pt}%
\definecolor{currentstroke}{rgb}{1.000000,0.000000,0.000000}%
\pgfsetstrokecolor{currentstroke}%
\pgfsetdash{}{0pt}%
\pgfpathmoveto{\pgfqpoint{1.745020in}{2.094530in}}%
\pgfpathlineto{\pgfqpoint{1.727389in}{2.228247in}}%
\pgfusepath{stroke}%
\end{pgfscope}%
\begin{pgfscope}%
\pgfpathrectangle{\pgfqpoint{0.100000in}{0.212622in}}{\pgfqpoint{3.696000in}{3.696000in}}%
\pgfusepath{clip}%
\pgfsetrectcap%
\pgfsetroundjoin%
\pgfsetlinewidth{1.505625pt}%
\definecolor{currentstroke}{rgb}{1.000000,0.000000,0.000000}%
\pgfsetstrokecolor{currentstroke}%
\pgfsetdash{}{0pt}%
\pgfpathmoveto{\pgfqpoint{1.744426in}{2.093506in}}%
\pgfpathlineto{\pgfqpoint{1.727389in}{2.228247in}}%
\pgfusepath{stroke}%
\end{pgfscope}%
\begin{pgfscope}%
\pgfpathrectangle{\pgfqpoint{0.100000in}{0.212622in}}{\pgfqpoint{3.696000in}{3.696000in}}%
\pgfusepath{clip}%
\pgfsetrectcap%
\pgfsetroundjoin%
\pgfsetlinewidth{1.505625pt}%
\definecolor{currentstroke}{rgb}{1.000000,0.000000,0.000000}%
\pgfsetstrokecolor{currentstroke}%
\pgfsetdash{}{0pt}%
\pgfpathmoveto{\pgfqpoint{1.744061in}{2.092953in}}%
\pgfpathlineto{\pgfqpoint{1.727389in}{2.228247in}}%
\pgfusepath{stroke}%
\end{pgfscope}%
\begin{pgfscope}%
\pgfpathrectangle{\pgfqpoint{0.100000in}{0.212622in}}{\pgfqpoint{3.696000in}{3.696000in}}%
\pgfusepath{clip}%
\pgfsetrectcap%
\pgfsetroundjoin%
\pgfsetlinewidth{1.505625pt}%
\definecolor{currentstroke}{rgb}{1.000000,0.000000,0.000000}%
\pgfsetstrokecolor{currentstroke}%
\pgfsetdash{}{0pt}%
\pgfpathmoveto{\pgfqpoint{1.743316in}{2.091262in}}%
\pgfpathlineto{\pgfqpoint{1.725831in}{2.226938in}}%
\pgfusepath{stroke}%
\end{pgfscope}%
\begin{pgfscope}%
\pgfpathrectangle{\pgfqpoint{0.100000in}{0.212622in}}{\pgfqpoint{3.696000in}{3.696000in}}%
\pgfusepath{clip}%
\pgfsetrectcap%
\pgfsetroundjoin%
\pgfsetlinewidth{1.505625pt}%
\definecolor{currentstroke}{rgb}{1.000000,0.000000,0.000000}%
\pgfsetstrokecolor{currentstroke}%
\pgfsetdash{}{0pt}%
\pgfpathmoveto{\pgfqpoint{1.742872in}{2.090332in}}%
\pgfpathlineto{\pgfqpoint{1.725831in}{2.226938in}}%
\pgfusepath{stroke}%
\end{pgfscope}%
\begin{pgfscope}%
\pgfpathrectangle{\pgfqpoint{0.100000in}{0.212622in}}{\pgfqpoint{3.696000in}{3.696000in}}%
\pgfusepath{clip}%
\pgfsetrectcap%
\pgfsetroundjoin%
\pgfsetlinewidth{1.505625pt}%
\definecolor{currentstroke}{rgb}{1.000000,0.000000,0.000000}%
\pgfsetstrokecolor{currentstroke}%
\pgfsetdash{}{0pt}%
\pgfpathmoveto{\pgfqpoint{1.742556in}{2.089870in}}%
\pgfpathlineto{\pgfqpoint{1.725831in}{2.226938in}}%
\pgfusepath{stroke}%
\end{pgfscope}%
\begin{pgfscope}%
\pgfpathrectangle{\pgfqpoint{0.100000in}{0.212622in}}{\pgfqpoint{3.696000in}{3.696000in}}%
\pgfusepath{clip}%
\pgfsetrectcap%
\pgfsetroundjoin%
\pgfsetlinewidth{1.505625pt}%
\definecolor{currentstroke}{rgb}{1.000000,0.000000,0.000000}%
\pgfsetstrokecolor{currentstroke}%
\pgfsetdash{}{0pt}%
\pgfpathmoveto{\pgfqpoint{1.742401in}{2.089602in}}%
\pgfpathlineto{\pgfqpoint{1.725831in}{2.226938in}}%
\pgfusepath{stroke}%
\end{pgfscope}%
\begin{pgfscope}%
\pgfpathrectangle{\pgfqpoint{0.100000in}{0.212622in}}{\pgfqpoint{3.696000in}{3.696000in}}%
\pgfusepath{clip}%
\pgfsetrectcap%
\pgfsetroundjoin%
\pgfsetlinewidth{1.505625pt}%
\definecolor{currentstroke}{rgb}{1.000000,0.000000,0.000000}%
\pgfsetstrokecolor{currentstroke}%
\pgfsetdash{}{0pt}%
\pgfpathmoveto{\pgfqpoint{1.742339in}{2.089431in}}%
\pgfpathlineto{\pgfqpoint{1.725831in}{2.226938in}}%
\pgfusepath{stroke}%
\end{pgfscope}%
\begin{pgfscope}%
\pgfpathrectangle{\pgfqpoint{0.100000in}{0.212622in}}{\pgfqpoint{3.696000in}{3.696000in}}%
\pgfusepath{clip}%
\pgfsetrectcap%
\pgfsetroundjoin%
\pgfsetlinewidth{1.505625pt}%
\definecolor{currentstroke}{rgb}{1.000000,0.000000,0.000000}%
\pgfsetstrokecolor{currentstroke}%
\pgfsetdash{}{0pt}%
\pgfpathmoveto{\pgfqpoint{1.741963in}{2.088757in}}%
\pgfpathlineto{\pgfqpoint{1.724272in}{2.225628in}}%
\pgfusepath{stroke}%
\end{pgfscope}%
\begin{pgfscope}%
\pgfpathrectangle{\pgfqpoint{0.100000in}{0.212622in}}{\pgfqpoint{3.696000in}{3.696000in}}%
\pgfusepath{clip}%
\pgfsetrectcap%
\pgfsetroundjoin%
\pgfsetlinewidth{1.505625pt}%
\definecolor{currentstroke}{rgb}{1.000000,0.000000,0.000000}%
\pgfsetstrokecolor{currentstroke}%
\pgfsetdash{}{0pt}%
\pgfpathmoveto{\pgfqpoint{1.741737in}{2.088410in}}%
\pgfpathlineto{\pgfqpoint{1.724272in}{2.225628in}}%
\pgfusepath{stroke}%
\end{pgfscope}%
\begin{pgfscope}%
\pgfpathrectangle{\pgfqpoint{0.100000in}{0.212622in}}{\pgfqpoint{3.696000in}{3.696000in}}%
\pgfusepath{clip}%
\pgfsetrectcap%
\pgfsetroundjoin%
\pgfsetlinewidth{1.505625pt}%
\definecolor{currentstroke}{rgb}{1.000000,0.000000,0.000000}%
\pgfsetstrokecolor{currentstroke}%
\pgfsetdash{}{0pt}%
\pgfpathmoveto{\pgfqpoint{1.741179in}{2.087399in}}%
\pgfpathlineto{\pgfqpoint{1.724272in}{2.225628in}}%
\pgfusepath{stroke}%
\end{pgfscope}%
\begin{pgfscope}%
\pgfpathrectangle{\pgfqpoint{0.100000in}{0.212622in}}{\pgfqpoint{3.696000in}{3.696000in}}%
\pgfusepath{clip}%
\pgfsetrectcap%
\pgfsetroundjoin%
\pgfsetlinewidth{1.505625pt}%
\definecolor{currentstroke}{rgb}{1.000000,0.000000,0.000000}%
\pgfsetstrokecolor{currentstroke}%
\pgfsetdash{}{0pt}%
\pgfpathmoveto{\pgfqpoint{1.740556in}{2.085854in}}%
\pgfpathlineto{\pgfqpoint{1.722713in}{2.224318in}}%
\pgfusepath{stroke}%
\end{pgfscope}%
\begin{pgfscope}%
\pgfpathrectangle{\pgfqpoint{0.100000in}{0.212622in}}{\pgfqpoint{3.696000in}{3.696000in}}%
\pgfusepath{clip}%
\pgfsetrectcap%
\pgfsetroundjoin%
\pgfsetlinewidth{1.505625pt}%
\definecolor{currentstroke}{rgb}{1.000000,0.000000,0.000000}%
\pgfsetstrokecolor{currentstroke}%
\pgfsetdash{}{0pt}%
\pgfpathmoveto{\pgfqpoint{1.740127in}{2.085096in}}%
\pgfpathlineto{\pgfqpoint{1.722713in}{2.224318in}}%
\pgfusepath{stroke}%
\end{pgfscope}%
\begin{pgfscope}%
\pgfpathrectangle{\pgfqpoint{0.100000in}{0.212622in}}{\pgfqpoint{3.696000in}{3.696000in}}%
\pgfusepath{clip}%
\pgfsetrectcap%
\pgfsetroundjoin%
\pgfsetlinewidth{1.505625pt}%
\definecolor{currentstroke}{rgb}{1.000000,0.000000,0.000000}%
\pgfsetstrokecolor{currentstroke}%
\pgfsetdash{}{0pt}%
\pgfpathmoveto{\pgfqpoint{1.739855in}{2.084704in}}%
\pgfpathlineto{\pgfqpoint{1.722713in}{2.224318in}}%
\pgfusepath{stroke}%
\end{pgfscope}%
\begin{pgfscope}%
\pgfpathrectangle{\pgfqpoint{0.100000in}{0.212622in}}{\pgfqpoint{3.696000in}{3.696000in}}%
\pgfusepath{clip}%
\pgfsetrectcap%
\pgfsetroundjoin%
\pgfsetlinewidth{1.505625pt}%
\definecolor{currentstroke}{rgb}{1.000000,0.000000,0.000000}%
\pgfsetstrokecolor{currentstroke}%
\pgfsetdash{}{0pt}%
\pgfpathmoveto{\pgfqpoint{1.739267in}{2.083573in}}%
\pgfpathlineto{\pgfqpoint{1.722713in}{2.224318in}}%
\pgfusepath{stroke}%
\end{pgfscope}%
\begin{pgfscope}%
\pgfpathrectangle{\pgfqpoint{0.100000in}{0.212622in}}{\pgfqpoint{3.696000in}{3.696000in}}%
\pgfusepath{clip}%
\pgfsetrectcap%
\pgfsetroundjoin%
\pgfsetlinewidth{1.505625pt}%
\definecolor{currentstroke}{rgb}{1.000000,0.000000,0.000000}%
\pgfsetstrokecolor{currentstroke}%
\pgfsetdash{}{0pt}%
\pgfpathmoveto{\pgfqpoint{1.738654in}{2.081868in}}%
\pgfpathlineto{\pgfqpoint{1.721154in}{2.223008in}}%
\pgfusepath{stroke}%
\end{pgfscope}%
\begin{pgfscope}%
\pgfpathrectangle{\pgfqpoint{0.100000in}{0.212622in}}{\pgfqpoint{3.696000in}{3.696000in}}%
\pgfusepath{clip}%
\pgfsetrectcap%
\pgfsetroundjoin%
\pgfsetlinewidth{1.505625pt}%
\definecolor{currentstroke}{rgb}{1.000000,0.000000,0.000000}%
\pgfsetstrokecolor{currentstroke}%
\pgfsetdash{}{0pt}%
\pgfpathmoveto{\pgfqpoint{1.737515in}{2.080040in}}%
\pgfpathlineto{\pgfqpoint{1.719594in}{2.221698in}}%
\pgfusepath{stroke}%
\end{pgfscope}%
\begin{pgfscope}%
\pgfpathrectangle{\pgfqpoint{0.100000in}{0.212622in}}{\pgfqpoint{3.696000in}{3.696000in}}%
\pgfusepath{clip}%
\pgfsetrectcap%
\pgfsetroundjoin%
\pgfsetlinewidth{1.505625pt}%
\definecolor{currentstroke}{rgb}{1.000000,0.000000,0.000000}%
\pgfsetstrokecolor{currentstroke}%
\pgfsetdash{}{0pt}%
\pgfpathmoveto{\pgfqpoint{1.736833in}{2.079054in}}%
\pgfpathlineto{\pgfqpoint{1.719594in}{2.221698in}}%
\pgfusepath{stroke}%
\end{pgfscope}%
\begin{pgfscope}%
\pgfpathrectangle{\pgfqpoint{0.100000in}{0.212622in}}{\pgfqpoint{3.696000in}{3.696000in}}%
\pgfusepath{clip}%
\pgfsetrectcap%
\pgfsetroundjoin%
\pgfsetlinewidth{1.505625pt}%
\definecolor{currentstroke}{rgb}{1.000000,0.000000,0.000000}%
\pgfsetstrokecolor{currentstroke}%
\pgfsetdash{}{0pt}%
\pgfpathmoveto{\pgfqpoint{1.735922in}{2.076731in}}%
\pgfpathlineto{\pgfqpoint{1.718034in}{2.220387in}}%
\pgfusepath{stroke}%
\end{pgfscope}%
\begin{pgfscope}%
\pgfpathrectangle{\pgfqpoint{0.100000in}{0.212622in}}{\pgfqpoint{3.696000in}{3.696000in}}%
\pgfusepath{clip}%
\pgfsetrectcap%
\pgfsetroundjoin%
\pgfsetlinewidth{1.505625pt}%
\definecolor{currentstroke}{rgb}{1.000000,0.000000,0.000000}%
\pgfsetstrokecolor{currentstroke}%
\pgfsetdash{}{0pt}%
\pgfpathmoveto{\pgfqpoint{1.734710in}{2.074322in}}%
\pgfpathlineto{\pgfqpoint{1.718034in}{2.220387in}}%
\pgfusepath{stroke}%
\end{pgfscope}%
\begin{pgfscope}%
\pgfpathrectangle{\pgfqpoint{0.100000in}{0.212622in}}{\pgfqpoint{3.696000in}{3.696000in}}%
\pgfusepath{clip}%
\pgfsetrectcap%
\pgfsetroundjoin%
\pgfsetlinewidth{1.505625pt}%
\definecolor{currentstroke}{rgb}{1.000000,0.000000,0.000000}%
\pgfsetstrokecolor{currentstroke}%
\pgfsetdash{}{0pt}%
\pgfpathmoveto{\pgfqpoint{1.733901in}{2.073094in}}%
\pgfpathlineto{\pgfqpoint{1.716473in}{2.219076in}}%
\pgfusepath{stroke}%
\end{pgfscope}%
\begin{pgfscope}%
\pgfpathrectangle{\pgfqpoint{0.100000in}{0.212622in}}{\pgfqpoint{3.696000in}{3.696000in}}%
\pgfusepath{clip}%
\pgfsetrectcap%
\pgfsetroundjoin%
\pgfsetlinewidth{1.505625pt}%
\definecolor{currentstroke}{rgb}{1.000000,0.000000,0.000000}%
\pgfsetstrokecolor{currentstroke}%
\pgfsetdash{}{0pt}%
\pgfpathmoveto{\pgfqpoint{1.732686in}{2.071245in}}%
\pgfpathlineto{\pgfqpoint{1.714912in}{2.217764in}}%
\pgfusepath{stroke}%
\end{pgfscope}%
\begin{pgfscope}%
\pgfpathrectangle{\pgfqpoint{0.100000in}{0.212622in}}{\pgfqpoint{3.696000in}{3.696000in}}%
\pgfusepath{clip}%
\pgfsetrectcap%
\pgfsetroundjoin%
\pgfsetlinewidth{1.505625pt}%
\definecolor{currentstroke}{rgb}{1.000000,0.000000,0.000000}%
\pgfsetstrokecolor{currentstroke}%
\pgfsetdash{}{0pt}%
\pgfpathmoveto{\pgfqpoint{1.732193in}{2.069929in}}%
\pgfpathlineto{\pgfqpoint{1.714912in}{2.217764in}}%
\pgfusepath{stroke}%
\end{pgfscope}%
\begin{pgfscope}%
\pgfpathrectangle{\pgfqpoint{0.100000in}{0.212622in}}{\pgfqpoint{3.696000in}{3.696000in}}%
\pgfusepath{clip}%
\pgfsetrectcap%
\pgfsetroundjoin%
\pgfsetlinewidth{1.505625pt}%
\definecolor{currentstroke}{rgb}{1.000000,0.000000,0.000000}%
\pgfsetstrokecolor{currentstroke}%
\pgfsetdash{}{0pt}%
\pgfpathmoveto{\pgfqpoint{1.731843in}{2.069320in}}%
\pgfpathlineto{\pgfqpoint{1.714912in}{2.217764in}}%
\pgfusepath{stroke}%
\end{pgfscope}%
\begin{pgfscope}%
\pgfpathrectangle{\pgfqpoint{0.100000in}{0.212622in}}{\pgfqpoint{3.696000in}{3.696000in}}%
\pgfusepath{clip}%
\pgfsetrectcap%
\pgfsetroundjoin%
\pgfsetlinewidth{1.505625pt}%
\definecolor{currentstroke}{rgb}{1.000000,0.000000,0.000000}%
\pgfsetstrokecolor{currentstroke}%
\pgfsetdash{}{0pt}%
\pgfpathmoveto{\pgfqpoint{1.731629in}{2.068998in}}%
\pgfpathlineto{\pgfqpoint{1.714912in}{2.217764in}}%
\pgfusepath{stroke}%
\end{pgfscope}%
\begin{pgfscope}%
\pgfpathrectangle{\pgfqpoint{0.100000in}{0.212622in}}{\pgfqpoint{3.696000in}{3.696000in}}%
\pgfusepath{clip}%
\pgfsetrectcap%
\pgfsetroundjoin%
\pgfsetlinewidth{1.505625pt}%
\definecolor{currentstroke}{rgb}{1.000000,0.000000,0.000000}%
\pgfsetstrokecolor{currentstroke}%
\pgfsetdash{}{0pt}%
\pgfpathmoveto{\pgfqpoint{1.731051in}{2.067757in}}%
\pgfpathlineto{\pgfqpoint{1.713351in}{2.216453in}}%
\pgfusepath{stroke}%
\end{pgfscope}%
\begin{pgfscope}%
\pgfpathrectangle{\pgfqpoint{0.100000in}{0.212622in}}{\pgfqpoint{3.696000in}{3.696000in}}%
\pgfusepath{clip}%
\pgfsetrectcap%
\pgfsetroundjoin%
\pgfsetlinewidth{1.505625pt}%
\definecolor{currentstroke}{rgb}{1.000000,0.000000,0.000000}%
\pgfsetstrokecolor{currentstroke}%
\pgfsetdash{}{0pt}%
\pgfpathmoveto{\pgfqpoint{1.730800in}{2.066981in}}%
\pgfpathlineto{\pgfqpoint{1.713351in}{2.216453in}}%
\pgfusepath{stroke}%
\end{pgfscope}%
\begin{pgfscope}%
\pgfpathrectangle{\pgfqpoint{0.100000in}{0.212622in}}{\pgfqpoint{3.696000in}{3.696000in}}%
\pgfusepath{clip}%
\pgfsetrectcap%
\pgfsetroundjoin%
\pgfsetlinewidth{1.505625pt}%
\definecolor{currentstroke}{rgb}{1.000000,0.000000,0.000000}%
\pgfsetstrokecolor{currentstroke}%
\pgfsetdash{}{0pt}%
\pgfpathmoveto{\pgfqpoint{1.730588in}{2.066626in}}%
\pgfpathlineto{\pgfqpoint{1.713351in}{2.216453in}}%
\pgfusepath{stroke}%
\end{pgfscope}%
\begin{pgfscope}%
\pgfpathrectangle{\pgfqpoint{0.100000in}{0.212622in}}{\pgfqpoint{3.696000in}{3.696000in}}%
\pgfusepath{clip}%
\pgfsetrectcap%
\pgfsetroundjoin%
\pgfsetlinewidth{1.505625pt}%
\definecolor{currentstroke}{rgb}{1.000000,0.000000,0.000000}%
\pgfsetstrokecolor{currentstroke}%
\pgfsetdash{}{0pt}%
\pgfpathmoveto{\pgfqpoint{1.730465in}{2.066434in}}%
\pgfpathlineto{\pgfqpoint{1.713351in}{2.216453in}}%
\pgfusepath{stroke}%
\end{pgfscope}%
\begin{pgfscope}%
\pgfpathrectangle{\pgfqpoint{0.100000in}{0.212622in}}{\pgfqpoint{3.696000in}{3.696000in}}%
\pgfusepath{clip}%
\pgfsetrectcap%
\pgfsetroundjoin%
\pgfsetlinewidth{1.505625pt}%
\definecolor{currentstroke}{rgb}{1.000000,0.000000,0.000000}%
\pgfsetstrokecolor{currentstroke}%
\pgfsetdash{}{0pt}%
\pgfpathmoveto{\pgfqpoint{1.730405in}{2.066323in}}%
\pgfpathlineto{\pgfqpoint{1.713351in}{2.216453in}}%
\pgfusepath{stroke}%
\end{pgfscope}%
\begin{pgfscope}%
\pgfpathrectangle{\pgfqpoint{0.100000in}{0.212622in}}{\pgfqpoint{3.696000in}{3.696000in}}%
\pgfusepath{clip}%
\pgfsetrectcap%
\pgfsetroundjoin%
\pgfsetlinewidth{1.505625pt}%
\definecolor{currentstroke}{rgb}{1.000000,0.000000,0.000000}%
\pgfsetstrokecolor{currentstroke}%
\pgfsetdash{}{0pt}%
\pgfpathmoveto{\pgfqpoint{1.730374in}{2.066263in}}%
\pgfpathlineto{\pgfqpoint{1.713351in}{2.216453in}}%
\pgfusepath{stroke}%
\end{pgfscope}%
\begin{pgfscope}%
\pgfpathrectangle{\pgfqpoint{0.100000in}{0.212622in}}{\pgfqpoint{3.696000in}{3.696000in}}%
\pgfusepath{clip}%
\pgfsetrectcap%
\pgfsetroundjoin%
\pgfsetlinewidth{1.505625pt}%
\definecolor{currentstroke}{rgb}{1.000000,0.000000,0.000000}%
\pgfsetstrokecolor{currentstroke}%
\pgfsetdash{}{0pt}%
\pgfpathmoveto{\pgfqpoint{1.730357in}{2.066228in}}%
\pgfpathlineto{\pgfqpoint{1.713351in}{2.216453in}}%
\pgfusepath{stroke}%
\end{pgfscope}%
\begin{pgfscope}%
\pgfpathrectangle{\pgfqpoint{0.100000in}{0.212622in}}{\pgfqpoint{3.696000in}{3.696000in}}%
\pgfusepath{clip}%
\pgfsetrectcap%
\pgfsetroundjoin%
\pgfsetlinewidth{1.505625pt}%
\definecolor{currentstroke}{rgb}{1.000000,0.000000,0.000000}%
\pgfsetstrokecolor{currentstroke}%
\pgfsetdash{}{0pt}%
\pgfpathmoveto{\pgfqpoint{1.730348in}{2.066210in}}%
\pgfpathlineto{\pgfqpoint{1.713351in}{2.216453in}}%
\pgfusepath{stroke}%
\end{pgfscope}%
\begin{pgfscope}%
\pgfpathrectangle{\pgfqpoint{0.100000in}{0.212622in}}{\pgfqpoint{3.696000in}{3.696000in}}%
\pgfusepath{clip}%
\pgfsetrectcap%
\pgfsetroundjoin%
\pgfsetlinewidth{1.505625pt}%
\definecolor{currentstroke}{rgb}{1.000000,0.000000,0.000000}%
\pgfsetstrokecolor{currentstroke}%
\pgfsetdash{}{0pt}%
\pgfpathmoveto{\pgfqpoint{1.730342in}{2.066200in}}%
\pgfpathlineto{\pgfqpoint{1.713351in}{2.216453in}}%
\pgfusepath{stroke}%
\end{pgfscope}%
\begin{pgfscope}%
\pgfpathrectangle{\pgfqpoint{0.100000in}{0.212622in}}{\pgfqpoint{3.696000in}{3.696000in}}%
\pgfusepath{clip}%
\pgfsetrectcap%
\pgfsetroundjoin%
\pgfsetlinewidth{1.505625pt}%
\definecolor{currentstroke}{rgb}{1.000000,0.000000,0.000000}%
\pgfsetstrokecolor{currentstroke}%
\pgfsetdash{}{0pt}%
\pgfpathmoveto{\pgfqpoint{1.730340in}{2.066195in}}%
\pgfpathlineto{\pgfqpoint{1.713351in}{2.216453in}}%
\pgfusepath{stroke}%
\end{pgfscope}%
\begin{pgfscope}%
\pgfpathrectangle{\pgfqpoint{0.100000in}{0.212622in}}{\pgfqpoint{3.696000in}{3.696000in}}%
\pgfusepath{clip}%
\pgfsetrectcap%
\pgfsetroundjoin%
\pgfsetlinewidth{1.505625pt}%
\definecolor{currentstroke}{rgb}{1.000000,0.000000,0.000000}%
\pgfsetstrokecolor{currentstroke}%
\pgfsetdash{}{0pt}%
\pgfpathmoveto{\pgfqpoint{1.730338in}{2.066192in}}%
\pgfpathlineto{\pgfqpoint{1.713351in}{2.216453in}}%
\pgfusepath{stroke}%
\end{pgfscope}%
\begin{pgfscope}%
\pgfpathrectangle{\pgfqpoint{0.100000in}{0.212622in}}{\pgfqpoint{3.696000in}{3.696000in}}%
\pgfusepath{clip}%
\pgfsetrectcap%
\pgfsetroundjoin%
\pgfsetlinewidth{1.505625pt}%
\definecolor{currentstroke}{rgb}{1.000000,0.000000,0.000000}%
\pgfsetstrokecolor{currentstroke}%
\pgfsetdash{}{0pt}%
\pgfpathmoveto{\pgfqpoint{1.730338in}{2.066190in}}%
\pgfpathlineto{\pgfqpoint{1.713351in}{2.216453in}}%
\pgfusepath{stroke}%
\end{pgfscope}%
\begin{pgfscope}%
\pgfpathrectangle{\pgfqpoint{0.100000in}{0.212622in}}{\pgfqpoint{3.696000in}{3.696000in}}%
\pgfusepath{clip}%
\pgfsetrectcap%
\pgfsetroundjoin%
\pgfsetlinewidth{1.505625pt}%
\definecolor{currentstroke}{rgb}{1.000000,0.000000,0.000000}%
\pgfsetstrokecolor{currentstroke}%
\pgfsetdash{}{0pt}%
\pgfpathmoveto{\pgfqpoint{1.730337in}{2.066189in}}%
\pgfpathlineto{\pgfqpoint{1.713351in}{2.216453in}}%
\pgfusepath{stroke}%
\end{pgfscope}%
\begin{pgfscope}%
\pgfpathrectangle{\pgfqpoint{0.100000in}{0.212622in}}{\pgfqpoint{3.696000in}{3.696000in}}%
\pgfusepath{clip}%
\pgfsetrectcap%
\pgfsetroundjoin%
\pgfsetlinewidth{1.505625pt}%
\definecolor{currentstroke}{rgb}{1.000000,0.000000,0.000000}%
\pgfsetstrokecolor{currentstroke}%
\pgfsetdash{}{0pt}%
\pgfpathmoveto{\pgfqpoint{1.730130in}{2.065701in}}%
\pgfpathlineto{\pgfqpoint{1.713351in}{2.216453in}}%
\pgfusepath{stroke}%
\end{pgfscope}%
\begin{pgfscope}%
\pgfpathrectangle{\pgfqpoint{0.100000in}{0.212622in}}{\pgfqpoint{3.696000in}{3.696000in}}%
\pgfusepath{clip}%
\pgfsetrectcap%
\pgfsetroundjoin%
\pgfsetlinewidth{1.505625pt}%
\definecolor{currentstroke}{rgb}{1.000000,0.000000,0.000000}%
\pgfsetstrokecolor{currentstroke}%
\pgfsetdash{}{0pt}%
\pgfpathmoveto{\pgfqpoint{1.729800in}{2.064853in}}%
\pgfpathlineto{\pgfqpoint{1.711789in}{2.215141in}}%
\pgfusepath{stroke}%
\end{pgfscope}%
\begin{pgfscope}%
\pgfpathrectangle{\pgfqpoint{0.100000in}{0.212622in}}{\pgfqpoint{3.696000in}{3.696000in}}%
\pgfusepath{clip}%
\pgfsetrectcap%
\pgfsetroundjoin%
\pgfsetlinewidth{1.505625pt}%
\definecolor{currentstroke}{rgb}{1.000000,0.000000,0.000000}%
\pgfsetstrokecolor{currentstroke}%
\pgfsetdash{}{0pt}%
\pgfpathmoveto{\pgfqpoint{1.729265in}{2.063380in}}%
\pgfpathlineto{\pgfqpoint{1.711789in}{2.215141in}}%
\pgfusepath{stroke}%
\end{pgfscope}%
\begin{pgfscope}%
\pgfpathrectangle{\pgfqpoint{0.100000in}{0.212622in}}{\pgfqpoint{3.696000in}{3.696000in}}%
\pgfusepath{clip}%
\pgfsetrectcap%
\pgfsetroundjoin%
\pgfsetlinewidth{1.505625pt}%
\definecolor{currentstroke}{rgb}{1.000000,0.000000,0.000000}%
\pgfsetstrokecolor{currentstroke}%
\pgfsetdash{}{0pt}%
\pgfpathmoveto{\pgfqpoint{1.728534in}{2.061224in}}%
\pgfpathlineto{\pgfqpoint{1.710228in}{2.213829in}}%
\pgfusepath{stroke}%
\end{pgfscope}%
\begin{pgfscope}%
\pgfpathrectangle{\pgfqpoint{0.100000in}{0.212622in}}{\pgfqpoint{3.696000in}{3.696000in}}%
\pgfusepath{clip}%
\pgfsetrectcap%
\pgfsetroundjoin%
\pgfsetlinewidth{1.505625pt}%
\definecolor{currentstroke}{rgb}{1.000000,0.000000,0.000000}%
\pgfsetstrokecolor{currentstroke}%
\pgfsetdash{}{0pt}%
\pgfpathmoveto{\pgfqpoint{1.727651in}{2.058573in}}%
\pgfpathlineto{\pgfqpoint{1.708665in}{2.212516in}}%
\pgfusepath{stroke}%
\end{pgfscope}%
\begin{pgfscope}%
\pgfpathrectangle{\pgfqpoint{0.100000in}{0.212622in}}{\pgfqpoint{3.696000in}{3.696000in}}%
\pgfusepath{clip}%
\pgfsetrectcap%
\pgfsetroundjoin%
\pgfsetlinewidth{1.505625pt}%
\definecolor{currentstroke}{rgb}{1.000000,0.000000,0.000000}%
\pgfsetstrokecolor{currentstroke}%
\pgfsetdash{}{0pt}%
\pgfpathmoveto{\pgfqpoint{1.727265in}{2.057049in}}%
\pgfpathlineto{\pgfqpoint{1.708665in}{2.212516in}}%
\pgfusepath{stroke}%
\end{pgfscope}%
\begin{pgfscope}%
\pgfpathrectangle{\pgfqpoint{0.100000in}{0.212622in}}{\pgfqpoint{3.696000in}{3.696000in}}%
\pgfusepath{clip}%
\pgfsetrectcap%
\pgfsetroundjoin%
\pgfsetlinewidth{1.505625pt}%
\definecolor{currentstroke}{rgb}{1.000000,0.000000,0.000000}%
\pgfsetstrokecolor{currentstroke}%
\pgfsetdash{}{0pt}%
\pgfpathmoveto{\pgfqpoint{1.726808in}{2.054958in}}%
\pgfpathlineto{\pgfqpoint{1.707103in}{2.211203in}}%
\pgfusepath{stroke}%
\end{pgfscope}%
\begin{pgfscope}%
\pgfpathrectangle{\pgfqpoint{0.100000in}{0.212622in}}{\pgfqpoint{3.696000in}{3.696000in}}%
\pgfusepath{clip}%
\pgfsetrectcap%
\pgfsetroundjoin%
\pgfsetlinewidth{1.505625pt}%
\definecolor{currentstroke}{rgb}{1.000000,0.000000,0.000000}%
\pgfsetstrokecolor{currentstroke}%
\pgfsetdash{}{0pt}%
\pgfpathmoveto{\pgfqpoint{1.726278in}{2.052116in}}%
\pgfpathlineto{\pgfqpoint{1.705539in}{2.209890in}}%
\pgfusepath{stroke}%
\end{pgfscope}%
\begin{pgfscope}%
\pgfpathrectangle{\pgfqpoint{0.100000in}{0.212622in}}{\pgfqpoint{3.696000in}{3.696000in}}%
\pgfusepath{clip}%
\pgfsetrectcap%
\pgfsetroundjoin%
\pgfsetlinewidth{1.505625pt}%
\definecolor{currentstroke}{rgb}{1.000000,0.000000,0.000000}%
\pgfsetstrokecolor{currentstroke}%
\pgfsetdash{}{0pt}%
\pgfpathmoveto{\pgfqpoint{1.726263in}{2.048897in}}%
\pgfpathlineto{\pgfqpoint{1.703976in}{2.208577in}}%
\pgfusepath{stroke}%
\end{pgfscope}%
\begin{pgfscope}%
\pgfpathrectangle{\pgfqpoint{0.100000in}{0.212622in}}{\pgfqpoint{3.696000in}{3.696000in}}%
\pgfusepath{clip}%
\pgfsetrectcap%
\pgfsetroundjoin%
\pgfsetlinewidth{1.505625pt}%
\definecolor{currentstroke}{rgb}{1.000000,0.000000,0.000000}%
\pgfsetstrokecolor{currentstroke}%
\pgfsetdash{}{0pt}%
\pgfpathmoveto{\pgfqpoint{1.726868in}{2.045400in}}%
\pgfpathlineto{\pgfqpoint{1.703976in}{2.208577in}}%
\pgfusepath{stroke}%
\end{pgfscope}%
\begin{pgfscope}%
\pgfpathrectangle{\pgfqpoint{0.100000in}{0.212622in}}{\pgfqpoint{3.696000in}{3.696000in}}%
\pgfusepath{clip}%
\pgfsetrectcap%
\pgfsetroundjoin%
\pgfsetlinewidth{1.505625pt}%
\definecolor{currentstroke}{rgb}{1.000000,0.000000,0.000000}%
\pgfsetstrokecolor{currentstroke}%
\pgfsetdash{}{0pt}%
\pgfpathmoveto{\pgfqpoint{1.727864in}{2.041710in}}%
\pgfpathlineto{\pgfqpoint{1.702412in}{2.207263in}}%
\pgfusepath{stroke}%
\end{pgfscope}%
\begin{pgfscope}%
\pgfpathrectangle{\pgfqpoint{0.100000in}{0.212622in}}{\pgfqpoint{3.696000in}{3.696000in}}%
\pgfusepath{clip}%
\pgfsetrectcap%
\pgfsetroundjoin%
\pgfsetlinewidth{1.505625pt}%
\definecolor{currentstroke}{rgb}{1.000000,0.000000,0.000000}%
\pgfsetstrokecolor{currentstroke}%
\pgfsetdash{}{0pt}%
\pgfpathmoveto{\pgfqpoint{1.729645in}{2.038504in}}%
\pgfpathlineto{\pgfqpoint{1.700848in}{2.205949in}}%
\pgfusepath{stroke}%
\end{pgfscope}%
\begin{pgfscope}%
\pgfpathrectangle{\pgfqpoint{0.100000in}{0.212622in}}{\pgfqpoint{3.696000in}{3.696000in}}%
\pgfusepath{clip}%
\pgfsetrectcap%
\pgfsetroundjoin%
\pgfsetlinewidth{1.505625pt}%
\definecolor{currentstroke}{rgb}{1.000000,0.000000,0.000000}%
\pgfsetstrokecolor{currentstroke}%
\pgfsetdash{}{0pt}%
\pgfpathmoveto{\pgfqpoint{1.732112in}{2.035274in}}%
\pgfpathlineto{\pgfqpoint{1.699284in}{2.204635in}}%
\pgfusepath{stroke}%
\end{pgfscope}%
\begin{pgfscope}%
\pgfpathrectangle{\pgfqpoint{0.100000in}{0.212622in}}{\pgfqpoint{3.696000in}{3.696000in}}%
\pgfusepath{clip}%
\pgfsetrectcap%
\pgfsetroundjoin%
\pgfsetlinewidth{1.505625pt}%
\definecolor{currentstroke}{rgb}{1.000000,0.000000,0.000000}%
\pgfsetstrokecolor{currentstroke}%
\pgfsetdash{}{0pt}%
\pgfpathmoveto{\pgfqpoint{1.735367in}{2.032715in}}%
\pgfpathlineto{\pgfqpoint{1.699284in}{2.204635in}}%
\pgfusepath{stroke}%
\end{pgfscope}%
\begin{pgfscope}%
\pgfpathrectangle{\pgfqpoint{0.100000in}{0.212622in}}{\pgfqpoint{3.696000in}{3.696000in}}%
\pgfusepath{clip}%
\pgfsetrectcap%
\pgfsetroundjoin%
\pgfsetlinewidth{1.505625pt}%
\definecolor{currentstroke}{rgb}{1.000000,0.000000,0.000000}%
\pgfsetstrokecolor{currentstroke}%
\pgfsetdash{}{0pt}%
\pgfpathmoveto{\pgfqpoint{1.739070in}{2.030174in}}%
\pgfpathlineto{\pgfqpoint{1.699284in}{2.204635in}}%
\pgfusepath{stroke}%
\end{pgfscope}%
\begin{pgfscope}%
\pgfpathrectangle{\pgfqpoint{0.100000in}{0.212622in}}{\pgfqpoint{3.696000in}{3.696000in}}%
\pgfusepath{clip}%
\pgfsetrectcap%
\pgfsetroundjoin%
\pgfsetlinewidth{1.505625pt}%
\definecolor{currentstroke}{rgb}{1.000000,0.000000,0.000000}%
\pgfsetstrokecolor{currentstroke}%
\pgfsetdash{}{0pt}%
\pgfpathmoveto{\pgfqpoint{1.743511in}{2.028365in}}%
\pgfpathlineto{\pgfqpoint{1.697719in}{2.203320in}}%
\pgfusepath{stroke}%
\end{pgfscope}%
\begin{pgfscope}%
\pgfpathrectangle{\pgfqpoint{0.100000in}{0.212622in}}{\pgfqpoint{3.696000in}{3.696000in}}%
\pgfusepath{clip}%
\pgfsetrectcap%
\pgfsetroundjoin%
\pgfsetlinewidth{1.505625pt}%
\definecolor{currentstroke}{rgb}{1.000000,0.000000,0.000000}%
\pgfsetstrokecolor{currentstroke}%
\pgfsetdash{}{0pt}%
\pgfpathmoveto{\pgfqpoint{1.748944in}{2.027382in}}%
\pgfpathlineto{\pgfqpoint{1.699284in}{2.204635in}}%
\pgfusepath{stroke}%
\end{pgfscope}%
\begin{pgfscope}%
\pgfpathrectangle{\pgfqpoint{0.100000in}{0.212622in}}{\pgfqpoint{3.696000in}{3.696000in}}%
\pgfusepath{clip}%
\pgfsetrectcap%
\pgfsetroundjoin%
\pgfsetlinewidth{1.505625pt}%
\definecolor{currentstroke}{rgb}{1.000000,0.000000,0.000000}%
\pgfsetstrokecolor{currentstroke}%
\pgfsetdash{}{0pt}%
\pgfpathmoveto{\pgfqpoint{1.755744in}{2.028530in}}%
\pgfpathlineto{\pgfqpoint{1.699284in}{2.204635in}}%
\pgfusepath{stroke}%
\end{pgfscope}%
\begin{pgfscope}%
\pgfpathrectangle{\pgfqpoint{0.100000in}{0.212622in}}{\pgfqpoint{3.696000in}{3.696000in}}%
\pgfusepath{clip}%
\pgfsetrectcap%
\pgfsetroundjoin%
\pgfsetlinewidth{1.505625pt}%
\definecolor{currentstroke}{rgb}{1.000000,0.000000,0.000000}%
\pgfsetstrokecolor{currentstroke}%
\pgfsetdash{}{0pt}%
\pgfpathmoveto{\pgfqpoint{1.763196in}{2.030400in}}%
\pgfpathlineto{\pgfqpoint{1.700848in}{2.205949in}}%
\pgfusepath{stroke}%
\end{pgfscope}%
\begin{pgfscope}%
\pgfpathrectangle{\pgfqpoint{0.100000in}{0.212622in}}{\pgfqpoint{3.696000in}{3.696000in}}%
\pgfusepath{clip}%
\pgfsetrectcap%
\pgfsetroundjoin%
\pgfsetlinewidth{1.505625pt}%
\definecolor{currentstroke}{rgb}{1.000000,0.000000,0.000000}%
\pgfsetstrokecolor{currentstroke}%
\pgfsetdash{}{0pt}%
\pgfpathmoveto{\pgfqpoint{1.767171in}{2.031405in}}%
\pgfpathlineto{\pgfqpoint{1.702412in}{2.207263in}}%
\pgfusepath{stroke}%
\end{pgfscope}%
\begin{pgfscope}%
\pgfpathrectangle{\pgfqpoint{0.100000in}{0.212622in}}{\pgfqpoint{3.696000in}{3.696000in}}%
\pgfusepath{clip}%
\pgfsetrectcap%
\pgfsetroundjoin%
\pgfsetlinewidth{1.505625pt}%
\definecolor{currentstroke}{rgb}{1.000000,0.000000,0.000000}%
\pgfsetstrokecolor{currentstroke}%
\pgfsetdash{}{0pt}%
\pgfpathmoveto{\pgfqpoint{1.769034in}{2.030973in}}%
\pgfpathlineto{\pgfqpoint{1.702412in}{2.207263in}}%
\pgfusepath{stroke}%
\end{pgfscope}%
\begin{pgfscope}%
\pgfpathrectangle{\pgfqpoint{0.100000in}{0.212622in}}{\pgfqpoint{3.696000in}{3.696000in}}%
\pgfusepath{clip}%
\pgfsetrectcap%
\pgfsetroundjoin%
\pgfsetlinewidth{1.505625pt}%
\definecolor{currentstroke}{rgb}{1.000000,0.000000,0.000000}%
\pgfsetstrokecolor{currentstroke}%
\pgfsetdash{}{0pt}%
\pgfpathmoveto{\pgfqpoint{1.770030in}{2.030472in}}%
\pgfpathlineto{\pgfqpoint{1.702412in}{2.207263in}}%
\pgfusepath{stroke}%
\end{pgfscope}%
\begin{pgfscope}%
\pgfpathrectangle{\pgfqpoint{0.100000in}{0.212622in}}{\pgfqpoint{3.696000in}{3.696000in}}%
\pgfusepath{clip}%
\pgfsetrectcap%
\pgfsetroundjoin%
\pgfsetlinewidth{1.505625pt}%
\definecolor{currentstroke}{rgb}{1.000000,0.000000,0.000000}%
\pgfsetstrokecolor{currentstroke}%
\pgfsetdash{}{0pt}%
\pgfpathmoveto{\pgfqpoint{1.770475in}{2.030041in}}%
\pgfpathlineto{\pgfqpoint{1.702412in}{2.207263in}}%
\pgfusepath{stroke}%
\end{pgfscope}%
\begin{pgfscope}%
\pgfpathrectangle{\pgfqpoint{0.100000in}{0.212622in}}{\pgfqpoint{3.696000in}{3.696000in}}%
\pgfusepath{clip}%
\pgfsetrectcap%
\pgfsetroundjoin%
\pgfsetlinewidth{1.505625pt}%
\definecolor{currentstroke}{rgb}{1.000000,0.000000,0.000000}%
\pgfsetstrokecolor{currentstroke}%
\pgfsetdash{}{0pt}%
\pgfpathmoveto{\pgfqpoint{1.771411in}{2.029085in}}%
\pgfpathlineto{\pgfqpoint{1.700848in}{2.205949in}}%
\pgfusepath{stroke}%
\end{pgfscope}%
\begin{pgfscope}%
\pgfpathrectangle{\pgfqpoint{0.100000in}{0.212622in}}{\pgfqpoint{3.696000in}{3.696000in}}%
\pgfusepath{clip}%
\pgfsetrectcap%
\pgfsetroundjoin%
\pgfsetlinewidth{1.505625pt}%
\definecolor{currentstroke}{rgb}{1.000000,0.000000,0.000000}%
\pgfsetstrokecolor{currentstroke}%
\pgfsetdash{}{0pt}%
\pgfpathmoveto{\pgfqpoint{1.772805in}{2.028050in}}%
\pgfpathlineto{\pgfqpoint{1.700848in}{2.205949in}}%
\pgfusepath{stroke}%
\end{pgfscope}%
\begin{pgfscope}%
\pgfpathrectangle{\pgfqpoint{0.100000in}{0.212622in}}{\pgfqpoint{3.696000in}{3.696000in}}%
\pgfusepath{clip}%
\pgfsetrectcap%
\pgfsetroundjoin%
\pgfsetlinewidth{1.505625pt}%
\definecolor{currentstroke}{rgb}{1.000000,0.000000,0.000000}%
\pgfsetstrokecolor{currentstroke}%
\pgfsetdash{}{0pt}%
\pgfpathmoveto{\pgfqpoint{1.774881in}{2.027361in}}%
\pgfpathlineto{\pgfqpoint{1.700848in}{2.205949in}}%
\pgfusepath{stroke}%
\end{pgfscope}%
\begin{pgfscope}%
\pgfpathrectangle{\pgfqpoint{0.100000in}{0.212622in}}{\pgfqpoint{3.696000in}{3.696000in}}%
\pgfusepath{clip}%
\pgfsetrectcap%
\pgfsetroundjoin%
\pgfsetlinewidth{1.505625pt}%
\definecolor{currentstroke}{rgb}{1.000000,0.000000,0.000000}%
\pgfsetstrokecolor{currentstroke}%
\pgfsetdash{}{0pt}%
\pgfpathmoveto{\pgfqpoint{1.777673in}{2.026695in}}%
\pgfpathlineto{\pgfqpoint{1.700848in}{2.205949in}}%
\pgfusepath{stroke}%
\end{pgfscope}%
\begin{pgfscope}%
\pgfpathrectangle{\pgfqpoint{0.100000in}{0.212622in}}{\pgfqpoint{3.696000in}{3.696000in}}%
\pgfusepath{clip}%
\pgfsetrectcap%
\pgfsetroundjoin%
\pgfsetlinewidth{1.505625pt}%
\definecolor{currentstroke}{rgb}{1.000000,0.000000,0.000000}%
\pgfsetstrokecolor{currentstroke}%
\pgfsetdash{}{0pt}%
\pgfpathmoveto{\pgfqpoint{1.780954in}{2.025909in}}%
\pgfpathlineto{\pgfqpoint{1.702412in}{2.207263in}}%
\pgfusepath{stroke}%
\end{pgfscope}%
\begin{pgfscope}%
\pgfpathrectangle{\pgfqpoint{0.100000in}{0.212622in}}{\pgfqpoint{3.696000in}{3.696000in}}%
\pgfusepath{clip}%
\pgfsetrectcap%
\pgfsetroundjoin%
\pgfsetlinewidth{1.505625pt}%
\definecolor{currentstroke}{rgb}{1.000000,0.000000,0.000000}%
\pgfsetstrokecolor{currentstroke}%
\pgfsetdash{}{0pt}%
\pgfpathmoveto{\pgfqpoint{1.784935in}{2.024120in}}%
\pgfpathlineto{\pgfqpoint{1.702412in}{2.207263in}}%
\pgfusepath{stroke}%
\end{pgfscope}%
\begin{pgfscope}%
\pgfpathrectangle{\pgfqpoint{0.100000in}{0.212622in}}{\pgfqpoint{3.696000in}{3.696000in}}%
\pgfusepath{clip}%
\pgfsetrectcap%
\pgfsetroundjoin%
\pgfsetlinewidth{1.505625pt}%
\definecolor{currentstroke}{rgb}{1.000000,0.000000,0.000000}%
\pgfsetstrokecolor{currentstroke}%
\pgfsetdash{}{0pt}%
\pgfpathmoveto{\pgfqpoint{1.789182in}{2.021969in}}%
\pgfpathlineto{\pgfqpoint{1.700848in}{2.205949in}}%
\pgfusepath{stroke}%
\end{pgfscope}%
\begin{pgfscope}%
\pgfpathrectangle{\pgfqpoint{0.100000in}{0.212622in}}{\pgfqpoint{3.696000in}{3.696000in}}%
\pgfusepath{clip}%
\pgfsetrectcap%
\pgfsetroundjoin%
\pgfsetlinewidth{1.505625pt}%
\definecolor{currentstroke}{rgb}{1.000000,0.000000,0.000000}%
\pgfsetstrokecolor{currentstroke}%
\pgfsetdash{}{0pt}%
\pgfpathmoveto{\pgfqpoint{1.793513in}{2.019333in}}%
\pgfpathlineto{\pgfqpoint{1.700848in}{2.205949in}}%
\pgfusepath{stroke}%
\end{pgfscope}%
\begin{pgfscope}%
\pgfpathrectangle{\pgfqpoint{0.100000in}{0.212622in}}{\pgfqpoint{3.696000in}{3.696000in}}%
\pgfusepath{clip}%
\pgfsetrectcap%
\pgfsetroundjoin%
\pgfsetlinewidth{1.505625pt}%
\definecolor{currentstroke}{rgb}{1.000000,0.000000,0.000000}%
\pgfsetstrokecolor{currentstroke}%
\pgfsetdash{}{0pt}%
\pgfpathmoveto{\pgfqpoint{1.798214in}{2.015054in}}%
\pgfpathlineto{\pgfqpoint{1.700848in}{2.205949in}}%
\pgfusepath{stroke}%
\end{pgfscope}%
\begin{pgfscope}%
\pgfpathrectangle{\pgfqpoint{0.100000in}{0.212622in}}{\pgfqpoint{3.696000in}{3.696000in}}%
\pgfusepath{clip}%
\pgfsetrectcap%
\pgfsetroundjoin%
\pgfsetlinewidth{1.505625pt}%
\definecolor{currentstroke}{rgb}{1.000000,0.000000,0.000000}%
\pgfsetstrokecolor{currentstroke}%
\pgfsetdash{}{0pt}%
\pgfpathmoveto{\pgfqpoint{1.803675in}{2.011670in}}%
\pgfpathlineto{\pgfqpoint{1.700848in}{2.205949in}}%
\pgfusepath{stroke}%
\end{pgfscope}%
\begin{pgfscope}%
\pgfpathrectangle{\pgfqpoint{0.100000in}{0.212622in}}{\pgfqpoint{3.696000in}{3.696000in}}%
\pgfusepath{clip}%
\pgfsetrectcap%
\pgfsetroundjoin%
\pgfsetlinewidth{1.505625pt}%
\definecolor{currentstroke}{rgb}{1.000000,0.000000,0.000000}%
\pgfsetstrokecolor{currentstroke}%
\pgfsetdash{}{0pt}%
\pgfpathmoveto{\pgfqpoint{1.809527in}{2.008029in}}%
\pgfpathlineto{\pgfqpoint{1.699284in}{2.204635in}}%
\pgfusepath{stroke}%
\end{pgfscope}%
\begin{pgfscope}%
\pgfpathrectangle{\pgfqpoint{0.100000in}{0.212622in}}{\pgfqpoint{3.696000in}{3.696000in}}%
\pgfusepath{clip}%
\pgfsetrectcap%
\pgfsetroundjoin%
\pgfsetlinewidth{1.505625pt}%
\definecolor{currentstroke}{rgb}{1.000000,0.000000,0.000000}%
\pgfsetstrokecolor{currentstroke}%
\pgfsetdash{}{0pt}%
\pgfpathmoveto{\pgfqpoint{1.816134in}{2.002818in}}%
\pgfpathlineto{\pgfqpoint{1.699284in}{2.204635in}}%
\pgfusepath{stroke}%
\end{pgfscope}%
\begin{pgfscope}%
\pgfpathrectangle{\pgfqpoint{0.100000in}{0.212622in}}{\pgfqpoint{3.696000in}{3.696000in}}%
\pgfusepath{clip}%
\pgfsetrectcap%
\pgfsetroundjoin%
\pgfsetlinewidth{1.505625pt}%
\definecolor{currentstroke}{rgb}{1.000000,0.000000,0.000000}%
\pgfsetstrokecolor{currentstroke}%
\pgfsetdash{}{0pt}%
\pgfpathmoveto{\pgfqpoint{1.823006in}{1.998245in}}%
\pgfpathlineto{\pgfqpoint{1.697719in}{2.203320in}}%
\pgfusepath{stroke}%
\end{pgfscope}%
\begin{pgfscope}%
\pgfpathrectangle{\pgfqpoint{0.100000in}{0.212622in}}{\pgfqpoint{3.696000in}{3.696000in}}%
\pgfusepath{clip}%
\pgfsetrectcap%
\pgfsetroundjoin%
\pgfsetlinewidth{1.505625pt}%
\definecolor{currentstroke}{rgb}{1.000000,0.000000,0.000000}%
\pgfsetstrokecolor{currentstroke}%
\pgfsetdash{}{0pt}%
\pgfpathmoveto{\pgfqpoint{1.831290in}{1.994424in}}%
\pgfpathlineto{\pgfqpoint{1.697719in}{2.203320in}}%
\pgfusepath{stroke}%
\end{pgfscope}%
\begin{pgfscope}%
\pgfpathrectangle{\pgfqpoint{0.100000in}{0.212622in}}{\pgfqpoint{3.696000in}{3.696000in}}%
\pgfusepath{clip}%
\pgfsetrectcap%
\pgfsetroundjoin%
\pgfsetlinewidth{1.505625pt}%
\definecolor{currentstroke}{rgb}{1.000000,0.000000,0.000000}%
\pgfsetstrokecolor{currentstroke}%
\pgfsetdash{}{0pt}%
\pgfpathmoveto{\pgfqpoint{1.839316in}{1.988367in}}%
\pgfpathlineto{\pgfqpoint{1.697719in}{2.203320in}}%
\pgfusepath{stroke}%
\end{pgfscope}%
\begin{pgfscope}%
\pgfpathrectangle{\pgfqpoint{0.100000in}{0.212622in}}{\pgfqpoint{3.696000in}{3.696000in}}%
\pgfusepath{clip}%
\pgfsetrectcap%
\pgfsetroundjoin%
\pgfsetlinewidth{1.505625pt}%
\definecolor{currentstroke}{rgb}{1.000000,0.000000,0.000000}%
\pgfsetstrokecolor{currentstroke}%
\pgfsetdash{}{0pt}%
\pgfpathmoveto{\pgfqpoint{1.848621in}{1.984868in}}%
\pgfpathlineto{\pgfqpoint{1.696154in}{2.202005in}}%
\pgfusepath{stroke}%
\end{pgfscope}%
\begin{pgfscope}%
\pgfpathrectangle{\pgfqpoint{0.100000in}{0.212622in}}{\pgfqpoint{3.696000in}{3.696000in}}%
\pgfusepath{clip}%
\pgfsetrectcap%
\pgfsetroundjoin%
\pgfsetlinewidth{1.505625pt}%
\definecolor{currentstroke}{rgb}{1.000000,0.000000,0.000000}%
\pgfsetstrokecolor{currentstroke}%
\pgfsetdash{}{0pt}%
\pgfpathmoveto{\pgfqpoint{1.858852in}{1.977746in}}%
\pgfpathlineto{\pgfqpoint{1.696154in}{2.202005in}}%
\pgfusepath{stroke}%
\end{pgfscope}%
\begin{pgfscope}%
\pgfpathrectangle{\pgfqpoint{0.100000in}{0.212622in}}{\pgfqpoint{3.696000in}{3.696000in}}%
\pgfusepath{clip}%
\pgfsetrectcap%
\pgfsetroundjoin%
\pgfsetlinewidth{1.505625pt}%
\definecolor{currentstroke}{rgb}{1.000000,0.000000,0.000000}%
\pgfsetstrokecolor{currentstroke}%
\pgfsetdash{}{0pt}%
\pgfpathmoveto{\pgfqpoint{1.869817in}{1.971244in}}%
\pgfpathlineto{\pgfqpoint{1.696154in}{2.202005in}}%
\pgfusepath{stroke}%
\end{pgfscope}%
\begin{pgfscope}%
\pgfpathrectangle{\pgfqpoint{0.100000in}{0.212622in}}{\pgfqpoint{3.696000in}{3.696000in}}%
\pgfusepath{clip}%
\pgfsetrectcap%
\pgfsetroundjoin%
\pgfsetlinewidth{1.505625pt}%
\definecolor{currentstroke}{rgb}{1.000000,0.000000,0.000000}%
\pgfsetstrokecolor{currentstroke}%
\pgfsetdash{}{0pt}%
\pgfpathmoveto{\pgfqpoint{1.875784in}{1.968150in}}%
\pgfpathlineto{\pgfqpoint{1.694588in}{2.200690in}}%
\pgfusepath{stroke}%
\end{pgfscope}%
\begin{pgfscope}%
\pgfpathrectangle{\pgfqpoint{0.100000in}{0.212622in}}{\pgfqpoint{3.696000in}{3.696000in}}%
\pgfusepath{clip}%
\pgfsetrectcap%
\pgfsetroundjoin%
\pgfsetlinewidth{1.505625pt}%
\definecolor{currentstroke}{rgb}{1.000000,0.000000,0.000000}%
\pgfsetstrokecolor{currentstroke}%
\pgfsetdash{}{0pt}%
\pgfpathmoveto{\pgfqpoint{1.878925in}{1.965489in}}%
\pgfpathlineto{\pgfqpoint{1.694588in}{2.200690in}}%
\pgfusepath{stroke}%
\end{pgfscope}%
\begin{pgfscope}%
\pgfpathrectangle{\pgfqpoint{0.100000in}{0.212622in}}{\pgfqpoint{3.696000in}{3.696000in}}%
\pgfusepath{clip}%
\pgfsetrectcap%
\pgfsetroundjoin%
\pgfsetlinewidth{1.505625pt}%
\definecolor{currentstroke}{rgb}{1.000000,0.000000,0.000000}%
\pgfsetstrokecolor{currentstroke}%
\pgfsetdash{}{0pt}%
\pgfpathmoveto{\pgfqpoint{1.880678in}{1.964220in}}%
\pgfpathlineto{\pgfqpoint{1.694588in}{2.200690in}}%
\pgfusepath{stroke}%
\end{pgfscope}%
\begin{pgfscope}%
\pgfpathrectangle{\pgfqpoint{0.100000in}{0.212622in}}{\pgfqpoint{3.696000in}{3.696000in}}%
\pgfusepath{clip}%
\pgfsetrectcap%
\pgfsetroundjoin%
\pgfsetlinewidth{1.505625pt}%
\definecolor{currentstroke}{rgb}{1.000000,0.000000,0.000000}%
\pgfsetstrokecolor{currentstroke}%
\pgfsetdash{}{0pt}%
\pgfpathmoveto{\pgfqpoint{1.882651in}{1.962952in}}%
\pgfpathlineto{\pgfqpoint{1.694588in}{2.200690in}}%
\pgfusepath{stroke}%
\end{pgfscope}%
\begin{pgfscope}%
\pgfpathrectangle{\pgfqpoint{0.100000in}{0.212622in}}{\pgfqpoint{3.696000in}{3.696000in}}%
\pgfusepath{clip}%
\pgfsetrectcap%
\pgfsetroundjoin%
\pgfsetlinewidth{1.505625pt}%
\definecolor{currentstroke}{rgb}{1.000000,0.000000,0.000000}%
\pgfsetstrokecolor{currentstroke}%
\pgfsetdash{}{0pt}%
\pgfpathmoveto{\pgfqpoint{1.884949in}{1.961062in}}%
\pgfpathlineto{\pgfqpoint{2.073592in}{2.091305in}}%
\pgfusepath{stroke}%
\end{pgfscope}%
\begin{pgfscope}%
\pgfpathrectangle{\pgfqpoint{0.100000in}{0.212622in}}{\pgfqpoint{3.696000in}{3.696000in}}%
\pgfusepath{clip}%
\pgfsetrectcap%
\pgfsetroundjoin%
\pgfsetlinewidth{1.505625pt}%
\definecolor{currentstroke}{rgb}{1.000000,0.000000,0.000000}%
\pgfsetstrokecolor{currentstroke}%
\pgfsetdash{}{0pt}%
\pgfpathmoveto{\pgfqpoint{1.886212in}{1.960091in}}%
\pgfpathlineto{\pgfqpoint{2.073592in}{2.091305in}}%
\pgfusepath{stroke}%
\end{pgfscope}%
\begin{pgfscope}%
\pgfpathrectangle{\pgfqpoint{0.100000in}{0.212622in}}{\pgfqpoint{3.696000in}{3.696000in}}%
\pgfusepath{clip}%
\pgfsetrectcap%
\pgfsetroundjoin%
\pgfsetlinewidth{1.505625pt}%
\definecolor{currentstroke}{rgb}{1.000000,0.000000,0.000000}%
\pgfsetstrokecolor{currentstroke}%
\pgfsetdash{}{0pt}%
\pgfpathmoveto{\pgfqpoint{1.888112in}{1.958823in}}%
\pgfpathlineto{\pgfqpoint{2.073592in}{2.091305in}}%
\pgfusepath{stroke}%
\end{pgfscope}%
\begin{pgfscope}%
\pgfpathrectangle{\pgfqpoint{0.100000in}{0.212622in}}{\pgfqpoint{3.696000in}{3.696000in}}%
\pgfusepath{clip}%
\pgfsetrectcap%
\pgfsetroundjoin%
\pgfsetlinewidth{1.505625pt}%
\definecolor{currentstroke}{rgb}{1.000000,0.000000,0.000000}%
\pgfsetstrokecolor{currentstroke}%
\pgfsetdash{}{0pt}%
\pgfpathmoveto{\pgfqpoint{1.891882in}{1.956454in}}%
\pgfpathlineto{\pgfqpoint{2.073592in}{2.091305in}}%
\pgfusepath{stroke}%
\end{pgfscope}%
\begin{pgfscope}%
\pgfpathrectangle{\pgfqpoint{0.100000in}{0.212622in}}{\pgfqpoint{3.696000in}{3.696000in}}%
\pgfusepath{clip}%
\pgfsetrectcap%
\pgfsetroundjoin%
\pgfsetlinewidth{1.505625pt}%
\definecolor{currentstroke}{rgb}{1.000000,0.000000,0.000000}%
\pgfsetstrokecolor{currentstroke}%
\pgfsetdash{}{0pt}%
\pgfpathmoveto{\pgfqpoint{1.896513in}{1.953864in}}%
\pgfpathlineto{\pgfqpoint{2.073592in}{2.091305in}}%
\pgfusepath{stroke}%
\end{pgfscope}%
\begin{pgfscope}%
\pgfpathrectangle{\pgfqpoint{0.100000in}{0.212622in}}{\pgfqpoint{3.696000in}{3.696000in}}%
\pgfusepath{clip}%
\pgfsetrectcap%
\pgfsetroundjoin%
\pgfsetlinewidth{1.505625pt}%
\definecolor{currentstroke}{rgb}{1.000000,0.000000,0.000000}%
\pgfsetstrokecolor{currentstroke}%
\pgfsetdash{}{0pt}%
\pgfpathmoveto{\pgfqpoint{1.899108in}{1.952850in}}%
\pgfpathlineto{\pgfqpoint{2.073592in}{2.091305in}}%
\pgfusepath{stroke}%
\end{pgfscope}%
\begin{pgfscope}%
\pgfpathrectangle{\pgfqpoint{0.100000in}{0.212622in}}{\pgfqpoint{3.696000in}{3.696000in}}%
\pgfusepath{clip}%
\pgfsetrectcap%
\pgfsetroundjoin%
\pgfsetlinewidth{1.505625pt}%
\definecolor{currentstroke}{rgb}{1.000000,0.000000,0.000000}%
\pgfsetstrokecolor{currentstroke}%
\pgfsetdash{}{0pt}%
\pgfpathmoveto{\pgfqpoint{1.900452in}{1.951844in}}%
\pgfpathlineto{\pgfqpoint{2.073592in}{2.091305in}}%
\pgfusepath{stroke}%
\end{pgfscope}%
\begin{pgfscope}%
\pgfpathrectangle{\pgfqpoint{0.100000in}{0.212622in}}{\pgfqpoint{3.696000in}{3.696000in}}%
\pgfusepath{clip}%
\pgfsetrectcap%
\pgfsetroundjoin%
\pgfsetlinewidth{1.505625pt}%
\definecolor{currentstroke}{rgb}{1.000000,0.000000,0.000000}%
\pgfsetstrokecolor{currentstroke}%
\pgfsetdash{}{0pt}%
\pgfpathmoveto{\pgfqpoint{1.902250in}{1.950642in}}%
\pgfpathlineto{\pgfqpoint{2.073592in}{2.091305in}}%
\pgfusepath{stroke}%
\end{pgfscope}%
\begin{pgfscope}%
\pgfpathrectangle{\pgfqpoint{0.100000in}{0.212622in}}{\pgfqpoint{3.696000in}{3.696000in}}%
\pgfusepath{clip}%
\pgfsetrectcap%
\pgfsetroundjoin%
\pgfsetlinewidth{1.505625pt}%
\definecolor{currentstroke}{rgb}{1.000000,0.000000,0.000000}%
\pgfsetstrokecolor{currentstroke}%
\pgfsetdash{}{0pt}%
\pgfpathmoveto{\pgfqpoint{1.904953in}{1.949288in}}%
\pgfpathlineto{\pgfqpoint{2.073592in}{2.091305in}}%
\pgfusepath{stroke}%
\end{pgfscope}%
\begin{pgfscope}%
\pgfpathrectangle{\pgfqpoint{0.100000in}{0.212622in}}{\pgfqpoint{3.696000in}{3.696000in}}%
\pgfusepath{clip}%
\pgfsetrectcap%
\pgfsetroundjoin%
\pgfsetlinewidth{1.505625pt}%
\definecolor{currentstroke}{rgb}{1.000000,0.000000,0.000000}%
\pgfsetstrokecolor{currentstroke}%
\pgfsetdash{}{0pt}%
\pgfpathmoveto{\pgfqpoint{1.908472in}{1.946379in}}%
\pgfpathlineto{\pgfqpoint{2.073592in}{2.091305in}}%
\pgfusepath{stroke}%
\end{pgfscope}%
\begin{pgfscope}%
\pgfpathrectangle{\pgfqpoint{0.100000in}{0.212622in}}{\pgfqpoint{3.696000in}{3.696000in}}%
\pgfusepath{clip}%
\pgfsetrectcap%
\pgfsetroundjoin%
\pgfsetlinewidth{1.505625pt}%
\definecolor{currentstroke}{rgb}{1.000000,0.000000,0.000000}%
\pgfsetstrokecolor{currentstroke}%
\pgfsetdash{}{0pt}%
\pgfpathmoveto{\pgfqpoint{1.913093in}{1.943886in}}%
\pgfpathlineto{\pgfqpoint{2.073592in}{2.091305in}}%
\pgfusepath{stroke}%
\end{pgfscope}%
\begin{pgfscope}%
\pgfpathrectangle{\pgfqpoint{0.100000in}{0.212622in}}{\pgfqpoint{3.696000in}{3.696000in}}%
\pgfusepath{clip}%
\pgfsetrectcap%
\pgfsetroundjoin%
\pgfsetlinewidth{1.505625pt}%
\definecolor{currentstroke}{rgb}{1.000000,0.000000,0.000000}%
\pgfsetstrokecolor{currentstroke}%
\pgfsetdash{}{0pt}%
\pgfpathmoveto{\pgfqpoint{1.917755in}{1.941321in}}%
\pgfpathlineto{\pgfqpoint{2.068980in}{2.087282in}}%
\pgfusepath{stroke}%
\end{pgfscope}%
\begin{pgfscope}%
\pgfpathrectangle{\pgfqpoint{0.100000in}{0.212622in}}{\pgfqpoint{3.696000in}{3.696000in}}%
\pgfusepath{clip}%
\pgfsetrectcap%
\pgfsetroundjoin%
\pgfsetlinewidth{1.505625pt}%
\definecolor{currentstroke}{rgb}{1.000000,0.000000,0.000000}%
\pgfsetstrokecolor{currentstroke}%
\pgfsetdash{}{0pt}%
\pgfpathmoveto{\pgfqpoint{1.923489in}{1.938234in}}%
\pgfpathlineto{\pgfqpoint{2.068980in}{2.087282in}}%
\pgfusepath{stroke}%
\end{pgfscope}%
\begin{pgfscope}%
\pgfpathrectangle{\pgfqpoint{0.100000in}{0.212622in}}{\pgfqpoint{3.696000in}{3.696000in}}%
\pgfusepath{clip}%
\pgfsetrectcap%
\pgfsetroundjoin%
\pgfsetlinewidth{1.505625pt}%
\definecolor{currentstroke}{rgb}{1.000000,0.000000,0.000000}%
\pgfsetstrokecolor{currentstroke}%
\pgfsetdash{}{0pt}%
\pgfpathmoveto{\pgfqpoint{1.929610in}{1.933522in}}%
\pgfpathlineto{\pgfqpoint{2.068980in}{2.087282in}}%
\pgfusepath{stroke}%
\end{pgfscope}%
\begin{pgfscope}%
\pgfpathrectangle{\pgfqpoint{0.100000in}{0.212622in}}{\pgfqpoint{3.696000in}{3.696000in}}%
\pgfusepath{clip}%
\pgfsetrectcap%
\pgfsetroundjoin%
\pgfsetlinewidth{1.505625pt}%
\definecolor{currentstroke}{rgb}{1.000000,0.000000,0.000000}%
\pgfsetstrokecolor{currentstroke}%
\pgfsetdash{}{0pt}%
\pgfpathmoveto{\pgfqpoint{1.936867in}{1.929080in}}%
\pgfpathlineto{\pgfqpoint{2.068980in}{2.087282in}}%
\pgfusepath{stroke}%
\end{pgfscope}%
\begin{pgfscope}%
\pgfpathrectangle{\pgfqpoint{0.100000in}{0.212622in}}{\pgfqpoint{3.696000in}{3.696000in}}%
\pgfusepath{clip}%
\pgfsetrectcap%
\pgfsetroundjoin%
\pgfsetlinewidth{1.505625pt}%
\definecolor{currentstroke}{rgb}{1.000000,0.000000,0.000000}%
\pgfsetstrokecolor{currentstroke}%
\pgfsetdash{}{0pt}%
\pgfpathmoveto{\pgfqpoint{1.945382in}{1.925834in}}%
\pgfpathlineto{\pgfqpoint{2.068980in}{2.087282in}}%
\pgfusepath{stroke}%
\end{pgfscope}%
\begin{pgfscope}%
\pgfpathrectangle{\pgfqpoint{0.100000in}{0.212622in}}{\pgfqpoint{3.696000in}{3.696000in}}%
\pgfusepath{clip}%
\pgfsetrectcap%
\pgfsetroundjoin%
\pgfsetlinewidth{1.505625pt}%
\definecolor{currentstroke}{rgb}{1.000000,0.000000,0.000000}%
\pgfsetstrokecolor{currentstroke}%
\pgfsetdash{}{0pt}%
\pgfpathmoveto{\pgfqpoint{1.949891in}{1.923069in}}%
\pgfpathlineto{\pgfqpoint{2.068980in}{2.087282in}}%
\pgfusepath{stroke}%
\end{pgfscope}%
\begin{pgfscope}%
\pgfpathrectangle{\pgfqpoint{0.100000in}{0.212622in}}{\pgfqpoint{3.696000in}{3.696000in}}%
\pgfusepath{clip}%
\pgfsetrectcap%
\pgfsetroundjoin%
\pgfsetlinewidth{1.505625pt}%
\definecolor{currentstroke}{rgb}{1.000000,0.000000,0.000000}%
\pgfsetstrokecolor{currentstroke}%
\pgfsetdash{}{0pt}%
\pgfpathmoveto{\pgfqpoint{1.954948in}{1.920937in}}%
\pgfpathlineto{\pgfqpoint{2.068980in}{2.087282in}}%
\pgfusepath{stroke}%
\end{pgfscope}%
\begin{pgfscope}%
\pgfpathrectangle{\pgfqpoint{0.100000in}{0.212622in}}{\pgfqpoint{3.696000in}{3.696000in}}%
\pgfusepath{clip}%
\pgfsetrectcap%
\pgfsetroundjoin%
\pgfsetlinewidth{1.505625pt}%
\definecolor{currentstroke}{rgb}{1.000000,0.000000,0.000000}%
\pgfsetstrokecolor{currentstroke}%
\pgfsetdash{}{0pt}%
\pgfpathmoveto{\pgfqpoint{1.961094in}{1.919101in}}%
\pgfpathlineto{\pgfqpoint{2.068980in}{2.087282in}}%
\pgfusepath{stroke}%
\end{pgfscope}%
\begin{pgfscope}%
\pgfpathrectangle{\pgfqpoint{0.100000in}{0.212622in}}{\pgfqpoint{3.696000in}{3.696000in}}%
\pgfusepath{clip}%
\pgfsetrectcap%
\pgfsetroundjoin%
\pgfsetlinewidth{1.505625pt}%
\definecolor{currentstroke}{rgb}{1.000000,0.000000,0.000000}%
\pgfsetstrokecolor{currentstroke}%
\pgfsetdash{}{0pt}%
\pgfpathmoveto{\pgfqpoint{1.967535in}{1.917458in}}%
\pgfpathlineto{\pgfqpoint{2.068980in}{2.087282in}}%
\pgfusepath{stroke}%
\end{pgfscope}%
\begin{pgfscope}%
\pgfpathrectangle{\pgfqpoint{0.100000in}{0.212622in}}{\pgfqpoint{3.696000in}{3.696000in}}%
\pgfusepath{clip}%
\pgfsetrectcap%
\pgfsetroundjoin%
\pgfsetlinewidth{1.505625pt}%
\definecolor{currentstroke}{rgb}{1.000000,0.000000,0.000000}%
\pgfsetstrokecolor{currentstroke}%
\pgfsetdash{}{0pt}%
\pgfpathmoveto{\pgfqpoint{1.971010in}{1.916117in}}%
\pgfpathlineto{\pgfqpoint{2.068980in}{2.087282in}}%
\pgfusepath{stroke}%
\end{pgfscope}%
\begin{pgfscope}%
\pgfpathrectangle{\pgfqpoint{0.100000in}{0.212622in}}{\pgfqpoint{3.696000in}{3.696000in}}%
\pgfusepath{clip}%
\pgfsetrectcap%
\pgfsetroundjoin%
\pgfsetlinewidth{1.505625pt}%
\definecolor{currentstroke}{rgb}{1.000000,0.000000,0.000000}%
\pgfsetstrokecolor{currentstroke}%
\pgfsetdash{}{0pt}%
\pgfpathmoveto{\pgfqpoint{1.975482in}{1.914446in}}%
\pgfpathlineto{\pgfqpoint{2.068980in}{2.087282in}}%
\pgfusepath{stroke}%
\end{pgfscope}%
\begin{pgfscope}%
\pgfpathrectangle{\pgfqpoint{0.100000in}{0.212622in}}{\pgfqpoint{3.696000in}{3.696000in}}%
\pgfusepath{clip}%
\pgfsetrectcap%
\pgfsetroundjoin%
\pgfsetlinewidth{1.505625pt}%
\definecolor{currentstroke}{rgb}{1.000000,0.000000,0.000000}%
\pgfsetstrokecolor{currentstroke}%
\pgfsetdash{}{0pt}%
\pgfpathmoveto{\pgfqpoint{1.980296in}{1.913275in}}%
\pgfpathlineto{\pgfqpoint{2.068980in}{2.087282in}}%
\pgfusepath{stroke}%
\end{pgfscope}%
\begin{pgfscope}%
\pgfpathrectangle{\pgfqpoint{0.100000in}{0.212622in}}{\pgfqpoint{3.696000in}{3.696000in}}%
\pgfusepath{clip}%
\pgfsetrectcap%
\pgfsetroundjoin%
\pgfsetlinewidth{1.505625pt}%
\definecolor{currentstroke}{rgb}{1.000000,0.000000,0.000000}%
\pgfsetstrokecolor{currentstroke}%
\pgfsetdash{}{0pt}%
\pgfpathmoveto{\pgfqpoint{1.983067in}{1.913024in}}%
\pgfpathlineto{\pgfqpoint{2.068980in}{2.087282in}}%
\pgfusepath{stroke}%
\end{pgfscope}%
\begin{pgfscope}%
\pgfpathrectangle{\pgfqpoint{0.100000in}{0.212622in}}{\pgfqpoint{3.696000in}{3.696000in}}%
\pgfusepath{clip}%
\pgfsetrectcap%
\pgfsetroundjoin%
\pgfsetlinewidth{1.505625pt}%
\definecolor{currentstroke}{rgb}{1.000000,0.000000,0.000000}%
\pgfsetstrokecolor{currentstroke}%
\pgfsetdash{}{0pt}%
\pgfpathmoveto{\pgfqpoint{1.986646in}{1.912065in}}%
\pgfpathlineto{\pgfqpoint{2.068980in}{2.087282in}}%
\pgfusepath{stroke}%
\end{pgfscope}%
\begin{pgfscope}%
\pgfpathrectangle{\pgfqpoint{0.100000in}{0.212622in}}{\pgfqpoint{3.696000in}{3.696000in}}%
\pgfusepath{clip}%
\pgfsetrectcap%
\pgfsetroundjoin%
\pgfsetlinewidth{1.505625pt}%
\definecolor{currentstroke}{rgb}{1.000000,0.000000,0.000000}%
\pgfsetstrokecolor{currentstroke}%
\pgfsetdash{}{0pt}%
\pgfpathmoveto{\pgfqpoint{1.991438in}{1.911531in}}%
\pgfpathlineto{\pgfqpoint{2.073592in}{2.091305in}}%
\pgfusepath{stroke}%
\end{pgfscope}%
\begin{pgfscope}%
\pgfpathrectangle{\pgfqpoint{0.100000in}{0.212622in}}{\pgfqpoint{3.696000in}{3.696000in}}%
\pgfusepath{clip}%
\pgfsetrectcap%
\pgfsetroundjoin%
\pgfsetlinewidth{1.505625pt}%
\definecolor{currentstroke}{rgb}{1.000000,0.000000,0.000000}%
\pgfsetstrokecolor{currentstroke}%
\pgfsetdash{}{0pt}%
\pgfpathmoveto{\pgfqpoint{1.997238in}{1.912670in}}%
\pgfpathlineto{\pgfqpoint{2.073592in}{2.091305in}}%
\pgfusepath{stroke}%
\end{pgfscope}%
\begin{pgfscope}%
\pgfpathrectangle{\pgfqpoint{0.100000in}{0.212622in}}{\pgfqpoint{3.696000in}{3.696000in}}%
\pgfusepath{clip}%
\pgfsetrectcap%
\pgfsetroundjoin%
\pgfsetlinewidth{1.505625pt}%
\definecolor{currentstroke}{rgb}{1.000000,0.000000,0.000000}%
\pgfsetstrokecolor{currentstroke}%
\pgfsetdash{}{0pt}%
\pgfpathmoveto{\pgfqpoint{2.003249in}{1.911292in}}%
\pgfpathlineto{\pgfqpoint{2.073592in}{2.091305in}}%
\pgfusepath{stroke}%
\end{pgfscope}%
\begin{pgfscope}%
\pgfpathrectangle{\pgfqpoint{0.100000in}{0.212622in}}{\pgfqpoint{3.696000in}{3.696000in}}%
\pgfusepath{clip}%
\pgfsetrectcap%
\pgfsetroundjoin%
\pgfsetlinewidth{1.505625pt}%
\definecolor{currentstroke}{rgb}{1.000000,0.000000,0.000000}%
\pgfsetstrokecolor{currentstroke}%
\pgfsetdash{}{0pt}%
\pgfpathmoveto{\pgfqpoint{2.006451in}{1.909689in}}%
\pgfpathlineto{\pgfqpoint{2.073592in}{2.091305in}}%
\pgfusepath{stroke}%
\end{pgfscope}%
\begin{pgfscope}%
\pgfpathrectangle{\pgfqpoint{0.100000in}{0.212622in}}{\pgfqpoint{3.696000in}{3.696000in}}%
\pgfusepath{clip}%
\pgfsetrectcap%
\pgfsetroundjoin%
\pgfsetlinewidth{1.505625pt}%
\definecolor{currentstroke}{rgb}{1.000000,0.000000,0.000000}%
\pgfsetstrokecolor{currentstroke}%
\pgfsetdash{}{0pt}%
\pgfpathmoveto{\pgfqpoint{2.008157in}{1.908804in}}%
\pgfpathlineto{\pgfqpoint{2.073592in}{2.091305in}}%
\pgfusepath{stroke}%
\end{pgfscope}%
\begin{pgfscope}%
\pgfpathrectangle{\pgfqpoint{0.100000in}{0.212622in}}{\pgfqpoint{3.696000in}{3.696000in}}%
\pgfusepath{clip}%
\pgfsetrectcap%
\pgfsetroundjoin%
\pgfsetlinewidth{1.505625pt}%
\definecolor{currentstroke}{rgb}{1.000000,0.000000,0.000000}%
\pgfsetstrokecolor{currentstroke}%
\pgfsetdash{}{0pt}%
\pgfpathmoveto{\pgfqpoint{2.010950in}{1.907805in}}%
\pgfpathlineto{\pgfqpoint{2.073592in}{2.091305in}}%
\pgfusepath{stroke}%
\end{pgfscope}%
\begin{pgfscope}%
\pgfpathrectangle{\pgfqpoint{0.100000in}{0.212622in}}{\pgfqpoint{3.696000in}{3.696000in}}%
\pgfusepath{clip}%
\pgfsetrectcap%
\pgfsetroundjoin%
\pgfsetlinewidth{1.505625pt}%
\definecolor{currentstroke}{rgb}{1.000000,0.000000,0.000000}%
\pgfsetstrokecolor{currentstroke}%
\pgfsetdash{}{0pt}%
\pgfpathmoveto{\pgfqpoint{2.012412in}{1.906850in}}%
\pgfpathlineto{\pgfqpoint{2.073592in}{2.091305in}}%
\pgfusepath{stroke}%
\end{pgfscope}%
\begin{pgfscope}%
\pgfpathrectangle{\pgfqpoint{0.100000in}{0.212622in}}{\pgfqpoint{3.696000in}{3.696000in}}%
\pgfusepath{clip}%
\pgfsetrectcap%
\pgfsetroundjoin%
\pgfsetlinewidth{1.505625pt}%
\definecolor{currentstroke}{rgb}{1.000000,0.000000,0.000000}%
\pgfsetstrokecolor{currentstroke}%
\pgfsetdash{}{0pt}%
\pgfpathmoveto{\pgfqpoint{2.013258in}{1.906464in}}%
\pgfpathlineto{\pgfqpoint{2.073592in}{2.091305in}}%
\pgfusepath{stroke}%
\end{pgfscope}%
\begin{pgfscope}%
\pgfpathrectangle{\pgfqpoint{0.100000in}{0.212622in}}{\pgfqpoint{3.696000in}{3.696000in}}%
\pgfusepath{clip}%
\pgfsetrectcap%
\pgfsetroundjoin%
\pgfsetlinewidth{1.505625pt}%
\definecolor{currentstroke}{rgb}{1.000000,0.000000,0.000000}%
\pgfsetstrokecolor{currentstroke}%
\pgfsetdash{}{0pt}%
\pgfpathmoveto{\pgfqpoint{2.013713in}{1.906241in}}%
\pgfpathlineto{\pgfqpoint{2.073592in}{2.091305in}}%
\pgfusepath{stroke}%
\end{pgfscope}%
\begin{pgfscope}%
\pgfpathrectangle{\pgfqpoint{0.100000in}{0.212622in}}{\pgfqpoint{3.696000in}{3.696000in}}%
\pgfusepath{clip}%
\pgfsetrectcap%
\pgfsetroundjoin%
\pgfsetlinewidth{1.505625pt}%
\definecolor{currentstroke}{rgb}{1.000000,0.000000,0.000000}%
\pgfsetstrokecolor{currentstroke}%
\pgfsetdash{}{0pt}%
\pgfpathmoveto{\pgfqpoint{2.014709in}{1.905639in}}%
\pgfpathlineto{\pgfqpoint{2.073592in}{2.091305in}}%
\pgfusepath{stroke}%
\end{pgfscope}%
\begin{pgfscope}%
\pgfpathrectangle{\pgfqpoint{0.100000in}{0.212622in}}{\pgfqpoint{3.696000in}{3.696000in}}%
\pgfusepath{clip}%
\pgfsetrectcap%
\pgfsetroundjoin%
\pgfsetlinewidth{1.505625pt}%
\definecolor{currentstroke}{rgb}{1.000000,0.000000,0.000000}%
\pgfsetstrokecolor{currentstroke}%
\pgfsetdash{}{0pt}%
\pgfpathmoveto{\pgfqpoint{2.016124in}{1.904926in}}%
\pgfpathlineto{\pgfqpoint{2.073592in}{2.091305in}}%
\pgfusepath{stroke}%
\end{pgfscope}%
\begin{pgfscope}%
\pgfpathrectangle{\pgfqpoint{0.100000in}{0.212622in}}{\pgfqpoint{3.696000in}{3.696000in}}%
\pgfusepath{clip}%
\pgfsetrectcap%
\pgfsetroundjoin%
\pgfsetlinewidth{1.505625pt}%
\definecolor{currentstroke}{rgb}{1.000000,0.000000,0.000000}%
\pgfsetstrokecolor{currentstroke}%
\pgfsetdash{}{0pt}%
\pgfpathmoveto{\pgfqpoint{2.016879in}{1.904478in}}%
\pgfpathlineto{\pgfqpoint{2.073592in}{2.091305in}}%
\pgfusepath{stroke}%
\end{pgfscope}%
\begin{pgfscope}%
\pgfpathrectangle{\pgfqpoint{0.100000in}{0.212622in}}{\pgfqpoint{3.696000in}{3.696000in}}%
\pgfusepath{clip}%
\pgfsetrectcap%
\pgfsetroundjoin%
\pgfsetlinewidth{1.505625pt}%
\definecolor{currentstroke}{rgb}{1.000000,0.000000,0.000000}%
\pgfsetstrokecolor{currentstroke}%
\pgfsetdash{}{0pt}%
\pgfpathmoveto{\pgfqpoint{2.018200in}{1.904038in}}%
\pgfpathlineto{\pgfqpoint{2.073592in}{2.091305in}}%
\pgfusepath{stroke}%
\end{pgfscope}%
\begin{pgfscope}%
\pgfpathrectangle{\pgfqpoint{0.100000in}{0.212622in}}{\pgfqpoint{3.696000in}{3.696000in}}%
\pgfusepath{clip}%
\pgfsetrectcap%
\pgfsetroundjoin%
\pgfsetlinewidth{1.505625pt}%
\definecolor{currentstroke}{rgb}{1.000000,0.000000,0.000000}%
\pgfsetstrokecolor{currentstroke}%
\pgfsetdash{}{0pt}%
\pgfpathmoveto{\pgfqpoint{2.019764in}{1.903231in}}%
\pgfpathlineto{\pgfqpoint{2.073592in}{2.091305in}}%
\pgfusepath{stroke}%
\end{pgfscope}%
\begin{pgfscope}%
\pgfpathrectangle{\pgfqpoint{0.100000in}{0.212622in}}{\pgfqpoint{3.696000in}{3.696000in}}%
\pgfusepath{clip}%
\pgfsetrectcap%
\pgfsetroundjoin%
\pgfsetlinewidth{1.505625pt}%
\definecolor{currentstroke}{rgb}{1.000000,0.000000,0.000000}%
\pgfsetstrokecolor{currentstroke}%
\pgfsetdash{}{0pt}%
\pgfpathmoveto{\pgfqpoint{2.021894in}{1.901830in}}%
\pgfpathlineto{\pgfqpoint{2.073592in}{2.091305in}}%
\pgfusepath{stroke}%
\end{pgfscope}%
\begin{pgfscope}%
\pgfpathrectangle{\pgfqpoint{0.100000in}{0.212622in}}{\pgfqpoint{3.696000in}{3.696000in}}%
\pgfusepath{clip}%
\pgfsetrectcap%
\pgfsetroundjoin%
\pgfsetlinewidth{1.505625pt}%
\definecolor{currentstroke}{rgb}{1.000000,0.000000,0.000000}%
\pgfsetstrokecolor{currentstroke}%
\pgfsetdash{}{0pt}%
\pgfpathmoveto{\pgfqpoint{2.023069in}{1.901136in}}%
\pgfpathlineto{\pgfqpoint{2.073592in}{2.091305in}}%
\pgfusepath{stroke}%
\end{pgfscope}%
\begin{pgfscope}%
\pgfpathrectangle{\pgfqpoint{0.100000in}{0.212622in}}{\pgfqpoint{3.696000in}{3.696000in}}%
\pgfusepath{clip}%
\pgfsetrectcap%
\pgfsetroundjoin%
\pgfsetlinewidth{1.505625pt}%
\definecolor{currentstroke}{rgb}{1.000000,0.000000,0.000000}%
\pgfsetstrokecolor{currentstroke}%
\pgfsetdash{}{0pt}%
\pgfpathmoveto{\pgfqpoint{2.025016in}{1.900378in}}%
\pgfpathlineto{\pgfqpoint{2.073592in}{2.091305in}}%
\pgfusepath{stroke}%
\end{pgfscope}%
\begin{pgfscope}%
\pgfpathrectangle{\pgfqpoint{0.100000in}{0.212622in}}{\pgfqpoint{3.696000in}{3.696000in}}%
\pgfusepath{clip}%
\pgfsetrectcap%
\pgfsetroundjoin%
\pgfsetlinewidth{1.505625pt}%
\definecolor{currentstroke}{rgb}{1.000000,0.000000,0.000000}%
\pgfsetstrokecolor{currentstroke}%
\pgfsetdash{}{0pt}%
\pgfpathmoveto{\pgfqpoint{2.027369in}{1.899817in}}%
\pgfpathlineto{\pgfqpoint{2.073592in}{2.091305in}}%
\pgfusepath{stroke}%
\end{pgfscope}%
\begin{pgfscope}%
\pgfpathrectangle{\pgfqpoint{0.100000in}{0.212622in}}{\pgfqpoint{3.696000in}{3.696000in}}%
\pgfusepath{clip}%
\pgfsetrectcap%
\pgfsetroundjoin%
\pgfsetlinewidth{1.505625pt}%
\definecolor{currentstroke}{rgb}{1.000000,0.000000,0.000000}%
\pgfsetstrokecolor{currentstroke}%
\pgfsetdash{}{0pt}%
\pgfpathmoveto{\pgfqpoint{2.029962in}{1.899046in}}%
\pgfpathlineto{\pgfqpoint{2.073592in}{2.091305in}}%
\pgfusepath{stroke}%
\end{pgfscope}%
\begin{pgfscope}%
\pgfpathrectangle{\pgfqpoint{0.100000in}{0.212622in}}{\pgfqpoint{3.696000in}{3.696000in}}%
\pgfusepath{clip}%
\pgfsetrectcap%
\pgfsetroundjoin%
\pgfsetlinewidth{1.505625pt}%
\definecolor{currentstroke}{rgb}{1.000000,0.000000,0.000000}%
\pgfsetstrokecolor{currentstroke}%
\pgfsetdash{}{0pt}%
\pgfpathmoveto{\pgfqpoint{2.031384in}{1.898543in}}%
\pgfpathlineto{\pgfqpoint{2.073592in}{2.091305in}}%
\pgfusepath{stroke}%
\end{pgfscope}%
\begin{pgfscope}%
\pgfpathrectangle{\pgfqpoint{0.100000in}{0.212622in}}{\pgfqpoint{3.696000in}{3.696000in}}%
\pgfusepath{clip}%
\pgfsetrectcap%
\pgfsetroundjoin%
\pgfsetlinewidth{1.505625pt}%
\definecolor{currentstroke}{rgb}{1.000000,0.000000,0.000000}%
\pgfsetstrokecolor{currentstroke}%
\pgfsetdash{}{0pt}%
\pgfpathmoveto{\pgfqpoint{2.033593in}{1.897503in}}%
\pgfpathlineto{\pgfqpoint{2.073592in}{2.091305in}}%
\pgfusepath{stroke}%
\end{pgfscope}%
\begin{pgfscope}%
\pgfpathrectangle{\pgfqpoint{0.100000in}{0.212622in}}{\pgfqpoint{3.696000in}{3.696000in}}%
\pgfusepath{clip}%
\pgfsetrectcap%
\pgfsetroundjoin%
\pgfsetlinewidth{1.505625pt}%
\definecolor{currentstroke}{rgb}{1.000000,0.000000,0.000000}%
\pgfsetstrokecolor{currentstroke}%
\pgfsetdash{}{0pt}%
\pgfpathmoveto{\pgfqpoint{2.034831in}{1.896884in}}%
\pgfpathlineto{\pgfqpoint{2.073592in}{2.091305in}}%
\pgfusepath{stroke}%
\end{pgfscope}%
\begin{pgfscope}%
\pgfpathrectangle{\pgfqpoint{0.100000in}{0.212622in}}{\pgfqpoint{3.696000in}{3.696000in}}%
\pgfusepath{clip}%
\pgfsetrectcap%
\pgfsetroundjoin%
\pgfsetlinewidth{1.505625pt}%
\definecolor{currentstroke}{rgb}{1.000000,0.000000,0.000000}%
\pgfsetstrokecolor{currentstroke}%
\pgfsetdash{}{0pt}%
\pgfpathmoveto{\pgfqpoint{2.036403in}{1.896455in}}%
\pgfpathlineto{\pgfqpoint{2.073592in}{2.091305in}}%
\pgfusepath{stroke}%
\end{pgfscope}%
\begin{pgfscope}%
\pgfpathrectangle{\pgfqpoint{0.100000in}{0.212622in}}{\pgfqpoint{3.696000in}{3.696000in}}%
\pgfusepath{clip}%
\pgfsetrectcap%
\pgfsetroundjoin%
\pgfsetlinewidth{1.505625pt}%
\definecolor{currentstroke}{rgb}{1.000000,0.000000,0.000000}%
\pgfsetstrokecolor{currentstroke}%
\pgfsetdash{}{0pt}%
\pgfpathmoveto{\pgfqpoint{2.037247in}{1.896081in}}%
\pgfpathlineto{\pgfqpoint{2.073592in}{2.091305in}}%
\pgfusepath{stroke}%
\end{pgfscope}%
\begin{pgfscope}%
\pgfpathrectangle{\pgfqpoint{0.100000in}{0.212622in}}{\pgfqpoint{3.696000in}{3.696000in}}%
\pgfusepath{clip}%
\pgfsetrectcap%
\pgfsetroundjoin%
\pgfsetlinewidth{1.505625pt}%
\definecolor{currentstroke}{rgb}{1.000000,0.000000,0.000000}%
\pgfsetstrokecolor{currentstroke}%
\pgfsetdash{}{0pt}%
\pgfpathmoveto{\pgfqpoint{2.038605in}{1.895263in}}%
\pgfpathlineto{\pgfqpoint{2.073592in}{2.091305in}}%
\pgfusepath{stroke}%
\end{pgfscope}%
\begin{pgfscope}%
\pgfpathrectangle{\pgfqpoint{0.100000in}{0.212622in}}{\pgfqpoint{3.696000in}{3.696000in}}%
\pgfusepath{clip}%
\pgfsetrectcap%
\pgfsetroundjoin%
\pgfsetlinewidth{1.505625pt}%
\definecolor{currentstroke}{rgb}{1.000000,0.000000,0.000000}%
\pgfsetstrokecolor{currentstroke}%
\pgfsetdash{}{0pt}%
\pgfpathmoveto{\pgfqpoint{2.040231in}{1.894390in}}%
\pgfpathlineto{\pgfqpoint{2.073592in}{2.091305in}}%
\pgfusepath{stroke}%
\end{pgfscope}%
\begin{pgfscope}%
\pgfpathrectangle{\pgfqpoint{0.100000in}{0.212622in}}{\pgfqpoint{3.696000in}{3.696000in}}%
\pgfusepath{clip}%
\pgfsetrectcap%
\pgfsetroundjoin%
\pgfsetlinewidth{1.505625pt}%
\definecolor{currentstroke}{rgb}{1.000000,0.000000,0.000000}%
\pgfsetstrokecolor{currentstroke}%
\pgfsetdash{}{0pt}%
\pgfpathmoveto{\pgfqpoint{2.042374in}{1.893706in}}%
\pgfpathlineto{\pgfqpoint{2.073592in}{2.091305in}}%
\pgfusepath{stroke}%
\end{pgfscope}%
\begin{pgfscope}%
\pgfpathrectangle{\pgfqpoint{0.100000in}{0.212622in}}{\pgfqpoint{3.696000in}{3.696000in}}%
\pgfusepath{clip}%
\pgfsetrectcap%
\pgfsetroundjoin%
\pgfsetlinewidth{1.505625pt}%
\definecolor{currentstroke}{rgb}{1.000000,0.000000,0.000000}%
\pgfsetstrokecolor{currentstroke}%
\pgfsetdash{}{0pt}%
\pgfpathmoveto{\pgfqpoint{2.043455in}{1.893149in}}%
\pgfpathlineto{\pgfqpoint{2.073592in}{2.091305in}}%
\pgfusepath{stroke}%
\end{pgfscope}%
\begin{pgfscope}%
\pgfpathrectangle{\pgfqpoint{0.100000in}{0.212622in}}{\pgfqpoint{3.696000in}{3.696000in}}%
\pgfusepath{clip}%
\pgfsetrectcap%
\pgfsetroundjoin%
\pgfsetlinewidth{1.505625pt}%
\definecolor{currentstroke}{rgb}{1.000000,0.000000,0.000000}%
\pgfsetstrokecolor{currentstroke}%
\pgfsetdash{}{0pt}%
\pgfpathmoveto{\pgfqpoint{2.045081in}{1.892472in}}%
\pgfpathlineto{\pgfqpoint{2.073592in}{2.091305in}}%
\pgfusepath{stroke}%
\end{pgfscope}%
\begin{pgfscope}%
\pgfpathrectangle{\pgfqpoint{0.100000in}{0.212622in}}{\pgfqpoint{3.696000in}{3.696000in}}%
\pgfusepath{clip}%
\pgfsetrectcap%
\pgfsetroundjoin%
\pgfsetlinewidth{1.505625pt}%
\definecolor{currentstroke}{rgb}{1.000000,0.000000,0.000000}%
\pgfsetstrokecolor{currentstroke}%
\pgfsetdash{}{0pt}%
\pgfpathmoveto{\pgfqpoint{2.047559in}{1.891417in}}%
\pgfpathlineto{\pgfqpoint{2.073592in}{2.091305in}}%
\pgfusepath{stroke}%
\end{pgfscope}%
\begin{pgfscope}%
\pgfpathrectangle{\pgfqpoint{0.100000in}{0.212622in}}{\pgfqpoint{3.696000in}{3.696000in}}%
\pgfusepath{clip}%
\pgfsetrectcap%
\pgfsetroundjoin%
\pgfsetlinewidth{1.505625pt}%
\definecolor{currentstroke}{rgb}{1.000000,0.000000,0.000000}%
\pgfsetstrokecolor{currentstroke}%
\pgfsetdash{}{0pt}%
\pgfpathmoveto{\pgfqpoint{2.050357in}{1.890451in}}%
\pgfpathlineto{\pgfqpoint{2.073592in}{2.091305in}}%
\pgfusepath{stroke}%
\end{pgfscope}%
\begin{pgfscope}%
\pgfpathrectangle{\pgfqpoint{0.100000in}{0.212622in}}{\pgfqpoint{3.696000in}{3.696000in}}%
\pgfusepath{clip}%
\pgfsetrectcap%
\pgfsetroundjoin%
\pgfsetlinewidth{1.505625pt}%
\definecolor{currentstroke}{rgb}{1.000000,0.000000,0.000000}%
\pgfsetstrokecolor{currentstroke}%
\pgfsetdash{}{0pt}%
\pgfpathmoveto{\pgfqpoint{2.051919in}{1.890061in}}%
\pgfpathlineto{\pgfqpoint{2.073592in}{2.091305in}}%
\pgfusepath{stroke}%
\end{pgfscope}%
\begin{pgfscope}%
\pgfpathrectangle{\pgfqpoint{0.100000in}{0.212622in}}{\pgfqpoint{3.696000in}{3.696000in}}%
\pgfusepath{clip}%
\pgfsetrectcap%
\pgfsetroundjoin%
\pgfsetlinewidth{1.505625pt}%
\definecolor{currentstroke}{rgb}{1.000000,0.000000,0.000000}%
\pgfsetstrokecolor{currentstroke}%
\pgfsetdash{}{0pt}%
\pgfpathmoveto{\pgfqpoint{2.054506in}{1.889487in}}%
\pgfpathlineto{\pgfqpoint{2.073592in}{2.091305in}}%
\pgfusepath{stroke}%
\end{pgfscope}%
\begin{pgfscope}%
\pgfpathrectangle{\pgfqpoint{0.100000in}{0.212622in}}{\pgfqpoint{3.696000in}{3.696000in}}%
\pgfusepath{clip}%
\pgfsetrectcap%
\pgfsetroundjoin%
\pgfsetlinewidth{1.505625pt}%
\definecolor{currentstroke}{rgb}{1.000000,0.000000,0.000000}%
\pgfsetstrokecolor{currentstroke}%
\pgfsetdash{}{0pt}%
\pgfpathmoveto{\pgfqpoint{2.056011in}{1.889528in}}%
\pgfpathlineto{\pgfqpoint{2.073592in}{2.091305in}}%
\pgfusepath{stroke}%
\end{pgfscope}%
\begin{pgfscope}%
\pgfpathrectangle{\pgfqpoint{0.100000in}{0.212622in}}{\pgfqpoint{3.696000in}{3.696000in}}%
\pgfusepath{clip}%
\pgfsetrectcap%
\pgfsetroundjoin%
\pgfsetlinewidth{1.505625pt}%
\definecolor{currentstroke}{rgb}{1.000000,0.000000,0.000000}%
\pgfsetstrokecolor{currentstroke}%
\pgfsetdash{}{0pt}%
\pgfpathmoveto{\pgfqpoint{2.057917in}{1.889843in}}%
\pgfpathlineto{\pgfqpoint{2.073592in}{2.091305in}}%
\pgfusepath{stroke}%
\end{pgfscope}%
\begin{pgfscope}%
\pgfpathrectangle{\pgfqpoint{0.100000in}{0.212622in}}{\pgfqpoint{3.696000in}{3.696000in}}%
\pgfusepath{clip}%
\pgfsetrectcap%
\pgfsetroundjoin%
\pgfsetlinewidth{1.505625pt}%
\definecolor{currentstroke}{rgb}{1.000000,0.000000,0.000000}%
\pgfsetstrokecolor{currentstroke}%
\pgfsetdash{}{0pt}%
\pgfpathmoveto{\pgfqpoint{2.060533in}{1.891125in}}%
\pgfpathlineto{\pgfqpoint{2.073592in}{2.091305in}}%
\pgfusepath{stroke}%
\end{pgfscope}%
\begin{pgfscope}%
\pgfpathrectangle{\pgfqpoint{0.100000in}{0.212622in}}{\pgfqpoint{3.696000in}{3.696000in}}%
\pgfusepath{clip}%
\pgfsetrectcap%
\pgfsetroundjoin%
\pgfsetlinewidth{1.505625pt}%
\definecolor{currentstroke}{rgb}{1.000000,0.000000,0.000000}%
\pgfsetstrokecolor{currentstroke}%
\pgfsetdash{}{0pt}%
\pgfpathmoveto{\pgfqpoint{2.061996in}{1.892075in}}%
\pgfpathlineto{\pgfqpoint{2.073592in}{2.091305in}}%
\pgfusepath{stroke}%
\end{pgfscope}%
\begin{pgfscope}%
\pgfpathrectangle{\pgfqpoint{0.100000in}{0.212622in}}{\pgfqpoint{3.696000in}{3.696000in}}%
\pgfusepath{clip}%
\pgfsetrectcap%
\pgfsetroundjoin%
\pgfsetlinewidth{1.505625pt}%
\definecolor{currentstroke}{rgb}{1.000000,0.000000,0.000000}%
\pgfsetstrokecolor{currentstroke}%
\pgfsetdash{}{0pt}%
\pgfpathmoveto{\pgfqpoint{2.063896in}{1.893298in}}%
\pgfpathlineto{\pgfqpoint{2.073592in}{2.091305in}}%
\pgfusepath{stroke}%
\end{pgfscope}%
\begin{pgfscope}%
\pgfpathrectangle{\pgfqpoint{0.100000in}{0.212622in}}{\pgfqpoint{3.696000in}{3.696000in}}%
\pgfusepath{clip}%
\pgfsetrectcap%
\pgfsetroundjoin%
\pgfsetlinewidth{1.505625pt}%
\definecolor{currentstroke}{rgb}{1.000000,0.000000,0.000000}%
\pgfsetstrokecolor{currentstroke}%
\pgfsetdash{}{0pt}%
\pgfpathmoveto{\pgfqpoint{2.066559in}{1.895706in}}%
\pgfpathlineto{\pgfqpoint{2.078200in}{2.095326in}}%
\pgfusepath{stroke}%
\end{pgfscope}%
\begin{pgfscope}%
\pgfpathrectangle{\pgfqpoint{0.100000in}{0.212622in}}{\pgfqpoint{3.696000in}{3.696000in}}%
\pgfusepath{clip}%
\pgfsetrectcap%
\pgfsetroundjoin%
\pgfsetlinewidth{1.505625pt}%
\definecolor{currentstroke}{rgb}{1.000000,0.000000,0.000000}%
\pgfsetstrokecolor{currentstroke}%
\pgfsetdash{}{0pt}%
\pgfpathmoveto{\pgfqpoint{2.069330in}{1.899177in}}%
\pgfpathlineto{\pgfqpoint{2.078200in}{2.095326in}}%
\pgfusepath{stroke}%
\end{pgfscope}%
\begin{pgfscope}%
\pgfpathrectangle{\pgfqpoint{0.100000in}{0.212622in}}{\pgfqpoint{3.696000in}{3.696000in}}%
\pgfusepath{clip}%
\pgfsetrectcap%
\pgfsetroundjoin%
\pgfsetlinewidth{1.505625pt}%
\definecolor{currentstroke}{rgb}{1.000000,0.000000,0.000000}%
\pgfsetstrokecolor{currentstroke}%
\pgfsetdash{}{0pt}%
\pgfpathmoveto{\pgfqpoint{2.070626in}{1.901146in}}%
\pgfpathlineto{\pgfqpoint{2.078200in}{2.095326in}}%
\pgfusepath{stroke}%
\end{pgfscope}%
\begin{pgfscope}%
\pgfpathrectangle{\pgfqpoint{0.100000in}{0.212622in}}{\pgfqpoint{3.696000in}{3.696000in}}%
\pgfusepath{clip}%
\pgfsetrectcap%
\pgfsetroundjoin%
\pgfsetlinewidth{1.505625pt}%
\definecolor{currentstroke}{rgb}{1.000000,0.000000,0.000000}%
\pgfsetstrokecolor{currentstroke}%
\pgfsetdash{}{0pt}%
\pgfpathmoveto{\pgfqpoint{2.072264in}{1.903222in}}%
\pgfpathlineto{\pgfqpoint{2.078200in}{2.095326in}}%
\pgfusepath{stroke}%
\end{pgfscope}%
\begin{pgfscope}%
\pgfpathrectangle{\pgfqpoint{0.100000in}{0.212622in}}{\pgfqpoint{3.696000in}{3.696000in}}%
\pgfusepath{clip}%
\pgfsetrectcap%
\pgfsetroundjoin%
\pgfsetlinewidth{1.505625pt}%
\definecolor{currentstroke}{rgb}{1.000000,0.000000,0.000000}%
\pgfsetstrokecolor{currentstroke}%
\pgfsetdash{}{0pt}%
\pgfpathmoveto{\pgfqpoint{2.073822in}{1.906468in}}%
\pgfpathlineto{\pgfqpoint{2.082806in}{2.099344in}}%
\pgfusepath{stroke}%
\end{pgfscope}%
\begin{pgfscope}%
\pgfpathrectangle{\pgfqpoint{0.100000in}{0.212622in}}{\pgfqpoint{3.696000in}{3.696000in}}%
\pgfusepath{clip}%
\pgfsetrectcap%
\pgfsetroundjoin%
\pgfsetlinewidth{1.505625pt}%
\definecolor{currentstroke}{rgb}{1.000000,0.000000,0.000000}%
\pgfsetstrokecolor{currentstroke}%
\pgfsetdash{}{0pt}%
\pgfpathmoveto{\pgfqpoint{2.076075in}{1.910311in}}%
\pgfpathlineto{\pgfqpoint{2.082806in}{2.099344in}}%
\pgfusepath{stroke}%
\end{pgfscope}%
\begin{pgfscope}%
\pgfpathrectangle{\pgfqpoint{0.100000in}{0.212622in}}{\pgfqpoint{3.696000in}{3.696000in}}%
\pgfusepath{clip}%
\pgfsetrectcap%
\pgfsetroundjoin%
\pgfsetlinewidth{1.505625pt}%
\definecolor{currentstroke}{rgb}{1.000000,0.000000,0.000000}%
\pgfsetstrokecolor{currentstroke}%
\pgfsetdash{}{0pt}%
\pgfpathmoveto{\pgfqpoint{2.077125in}{1.912242in}}%
\pgfpathlineto{\pgfqpoint{2.087408in}{2.103359in}}%
\pgfusepath{stroke}%
\end{pgfscope}%
\begin{pgfscope}%
\pgfpathrectangle{\pgfqpoint{0.100000in}{0.212622in}}{\pgfqpoint{3.696000in}{3.696000in}}%
\pgfusepath{clip}%
\pgfsetrectcap%
\pgfsetroundjoin%
\pgfsetlinewidth{1.505625pt}%
\definecolor{currentstroke}{rgb}{1.000000,0.000000,0.000000}%
\pgfsetstrokecolor{currentstroke}%
\pgfsetdash{}{0pt}%
\pgfpathmoveto{\pgfqpoint{2.077704in}{1.913359in}}%
\pgfpathlineto{\pgfqpoint{2.087408in}{2.103359in}}%
\pgfusepath{stroke}%
\end{pgfscope}%
\begin{pgfscope}%
\pgfpathrectangle{\pgfqpoint{0.100000in}{0.212622in}}{\pgfqpoint{3.696000in}{3.696000in}}%
\pgfusepath{clip}%
\pgfsetrectcap%
\pgfsetroundjoin%
\pgfsetlinewidth{1.505625pt}%
\definecolor{currentstroke}{rgb}{1.000000,0.000000,0.000000}%
\pgfsetstrokecolor{currentstroke}%
\pgfsetdash{}{0pt}%
\pgfpathmoveto{\pgfqpoint{2.078028in}{1.913976in}}%
\pgfpathlineto{\pgfqpoint{2.087408in}{2.103359in}}%
\pgfusepath{stroke}%
\end{pgfscope}%
\begin{pgfscope}%
\pgfpathrectangle{\pgfqpoint{0.100000in}{0.212622in}}{\pgfqpoint{3.696000in}{3.696000in}}%
\pgfusepath{clip}%
\pgfsetrectcap%
\pgfsetroundjoin%
\pgfsetlinewidth{1.505625pt}%
\definecolor{currentstroke}{rgb}{1.000000,0.000000,0.000000}%
\pgfsetstrokecolor{currentstroke}%
\pgfsetdash{}{0pt}%
\pgfpathmoveto{\pgfqpoint{2.078204in}{1.914304in}}%
\pgfpathlineto{\pgfqpoint{2.087408in}{2.103359in}}%
\pgfusepath{stroke}%
\end{pgfscope}%
\begin{pgfscope}%
\pgfpathrectangle{\pgfqpoint{0.100000in}{0.212622in}}{\pgfqpoint{3.696000in}{3.696000in}}%
\pgfusepath{clip}%
\pgfsetrectcap%
\pgfsetroundjoin%
\pgfsetlinewidth{1.505625pt}%
\definecolor{currentstroke}{rgb}{1.000000,0.000000,0.000000}%
\pgfsetstrokecolor{currentstroke}%
\pgfsetdash{}{0pt}%
\pgfpathmoveto{\pgfqpoint{2.078300in}{1.914483in}}%
\pgfpathlineto{\pgfqpoint{2.087408in}{2.103359in}}%
\pgfusepath{stroke}%
\end{pgfscope}%
\begin{pgfscope}%
\pgfpathrectangle{\pgfqpoint{0.100000in}{0.212622in}}{\pgfqpoint{3.696000in}{3.696000in}}%
\pgfusepath{clip}%
\pgfsetrectcap%
\pgfsetroundjoin%
\pgfsetlinewidth{1.505625pt}%
\definecolor{currentstroke}{rgb}{1.000000,0.000000,0.000000}%
\pgfsetstrokecolor{currentstroke}%
\pgfsetdash{}{0pt}%
\pgfpathmoveto{\pgfqpoint{2.078353in}{1.914583in}}%
\pgfpathlineto{\pgfqpoint{2.087408in}{2.103359in}}%
\pgfusepath{stroke}%
\end{pgfscope}%
\begin{pgfscope}%
\pgfpathrectangle{\pgfqpoint{0.100000in}{0.212622in}}{\pgfqpoint{3.696000in}{3.696000in}}%
\pgfusepath{clip}%
\pgfsetrectcap%
\pgfsetroundjoin%
\pgfsetlinewidth{1.505625pt}%
\definecolor{currentstroke}{rgb}{1.000000,0.000000,0.000000}%
\pgfsetstrokecolor{currentstroke}%
\pgfsetdash{}{0pt}%
\pgfpathmoveto{\pgfqpoint{2.078383in}{1.914635in}}%
\pgfpathlineto{\pgfqpoint{2.087408in}{2.103359in}}%
\pgfusepath{stroke}%
\end{pgfscope}%
\begin{pgfscope}%
\pgfpathrectangle{\pgfqpoint{0.100000in}{0.212622in}}{\pgfqpoint{3.696000in}{3.696000in}}%
\pgfusepath{clip}%
\pgfsetrectcap%
\pgfsetroundjoin%
\pgfsetlinewidth{1.505625pt}%
\definecolor{currentstroke}{rgb}{1.000000,0.000000,0.000000}%
\pgfsetstrokecolor{currentstroke}%
\pgfsetdash{}{0pt}%
\pgfpathmoveto{\pgfqpoint{2.078400in}{1.914666in}}%
\pgfpathlineto{\pgfqpoint{2.087408in}{2.103359in}}%
\pgfusepath{stroke}%
\end{pgfscope}%
\begin{pgfscope}%
\pgfpathrectangle{\pgfqpoint{0.100000in}{0.212622in}}{\pgfqpoint{3.696000in}{3.696000in}}%
\pgfusepath{clip}%
\pgfsetrectcap%
\pgfsetroundjoin%
\pgfsetlinewidth{1.505625pt}%
\definecolor{currentstroke}{rgb}{1.000000,0.000000,0.000000}%
\pgfsetstrokecolor{currentstroke}%
\pgfsetdash{}{0pt}%
\pgfpathmoveto{\pgfqpoint{2.078409in}{1.914681in}}%
\pgfpathlineto{\pgfqpoint{2.087408in}{2.103359in}}%
\pgfusepath{stroke}%
\end{pgfscope}%
\begin{pgfscope}%
\pgfpathrectangle{\pgfqpoint{0.100000in}{0.212622in}}{\pgfqpoint{3.696000in}{3.696000in}}%
\pgfusepath{clip}%
\pgfsetrectcap%
\pgfsetroundjoin%
\pgfsetlinewidth{1.505625pt}%
\definecolor{currentstroke}{rgb}{1.000000,0.000000,0.000000}%
\pgfsetstrokecolor{currentstroke}%
\pgfsetdash{}{0pt}%
\pgfpathmoveto{\pgfqpoint{2.078414in}{1.914689in}}%
\pgfpathlineto{\pgfqpoint{2.087408in}{2.103359in}}%
\pgfusepath{stroke}%
\end{pgfscope}%
\begin{pgfscope}%
\pgfpathrectangle{\pgfqpoint{0.100000in}{0.212622in}}{\pgfqpoint{3.696000in}{3.696000in}}%
\pgfusepath{clip}%
\pgfsetrectcap%
\pgfsetroundjoin%
\pgfsetlinewidth{1.505625pt}%
\definecolor{currentstroke}{rgb}{1.000000,0.000000,0.000000}%
\pgfsetstrokecolor{currentstroke}%
\pgfsetdash{}{0pt}%
\pgfpathmoveto{\pgfqpoint{2.078417in}{1.914694in}}%
\pgfpathlineto{\pgfqpoint{2.087408in}{2.103359in}}%
\pgfusepath{stroke}%
\end{pgfscope}%
\begin{pgfscope}%
\pgfpathrectangle{\pgfqpoint{0.100000in}{0.212622in}}{\pgfqpoint{3.696000in}{3.696000in}}%
\pgfusepath{clip}%
\pgfsetrectcap%
\pgfsetroundjoin%
\pgfsetlinewidth{1.505625pt}%
\definecolor{currentstroke}{rgb}{1.000000,0.000000,0.000000}%
\pgfsetstrokecolor{currentstroke}%
\pgfsetdash{}{0pt}%
\pgfpathmoveto{\pgfqpoint{2.078418in}{1.914697in}}%
\pgfpathlineto{\pgfqpoint{2.087408in}{2.103359in}}%
\pgfusepath{stroke}%
\end{pgfscope}%
\begin{pgfscope}%
\pgfpathrectangle{\pgfqpoint{0.100000in}{0.212622in}}{\pgfqpoint{3.696000in}{3.696000in}}%
\pgfusepath{clip}%
\pgfsetrectcap%
\pgfsetroundjoin%
\pgfsetlinewidth{1.505625pt}%
\definecolor{currentstroke}{rgb}{1.000000,0.000000,0.000000}%
\pgfsetstrokecolor{currentstroke}%
\pgfsetdash{}{0pt}%
\pgfpathmoveto{\pgfqpoint{2.078419in}{1.914699in}}%
\pgfpathlineto{\pgfqpoint{2.087408in}{2.103359in}}%
\pgfusepath{stroke}%
\end{pgfscope}%
\begin{pgfscope}%
\pgfpathrectangle{\pgfqpoint{0.100000in}{0.212622in}}{\pgfqpoint{3.696000in}{3.696000in}}%
\pgfusepath{clip}%
\pgfsetrectcap%
\pgfsetroundjoin%
\pgfsetlinewidth{1.505625pt}%
\definecolor{currentstroke}{rgb}{1.000000,0.000000,0.000000}%
\pgfsetstrokecolor{currentstroke}%
\pgfsetdash{}{0pt}%
\pgfpathmoveto{\pgfqpoint{2.078420in}{1.914699in}}%
\pgfpathlineto{\pgfqpoint{2.087408in}{2.103359in}}%
\pgfusepath{stroke}%
\end{pgfscope}%
\begin{pgfscope}%
\pgfpathrectangle{\pgfqpoint{0.100000in}{0.212622in}}{\pgfqpoint{3.696000in}{3.696000in}}%
\pgfusepath{clip}%
\pgfsetrectcap%
\pgfsetroundjoin%
\pgfsetlinewidth{1.505625pt}%
\definecolor{currentstroke}{rgb}{1.000000,0.000000,0.000000}%
\pgfsetstrokecolor{currentstroke}%
\pgfsetdash{}{0pt}%
\pgfpathmoveto{\pgfqpoint{2.078420in}{1.914700in}}%
\pgfpathlineto{\pgfqpoint{2.087408in}{2.103359in}}%
\pgfusepath{stroke}%
\end{pgfscope}%
\begin{pgfscope}%
\pgfpathrectangle{\pgfqpoint{0.100000in}{0.212622in}}{\pgfqpoint{3.696000in}{3.696000in}}%
\pgfusepath{clip}%
\pgfsetrectcap%
\pgfsetroundjoin%
\pgfsetlinewidth{1.505625pt}%
\definecolor{currentstroke}{rgb}{1.000000,0.000000,0.000000}%
\pgfsetstrokecolor{currentstroke}%
\pgfsetdash{}{0pt}%
\pgfpathmoveto{\pgfqpoint{2.078420in}{1.914700in}}%
\pgfpathlineto{\pgfqpoint{2.087408in}{2.103359in}}%
\pgfusepath{stroke}%
\end{pgfscope}%
\begin{pgfscope}%
\pgfpathrectangle{\pgfqpoint{0.100000in}{0.212622in}}{\pgfqpoint{3.696000in}{3.696000in}}%
\pgfusepath{clip}%
\pgfsetrectcap%
\pgfsetroundjoin%
\pgfsetlinewidth{1.505625pt}%
\definecolor{currentstroke}{rgb}{1.000000,0.000000,0.000000}%
\pgfsetstrokecolor{currentstroke}%
\pgfsetdash{}{0pt}%
\pgfpathmoveto{\pgfqpoint{2.078420in}{1.914700in}}%
\pgfpathlineto{\pgfqpoint{2.087408in}{2.103359in}}%
\pgfusepath{stroke}%
\end{pgfscope}%
\begin{pgfscope}%
\pgfpathrectangle{\pgfqpoint{0.100000in}{0.212622in}}{\pgfqpoint{3.696000in}{3.696000in}}%
\pgfusepath{clip}%
\pgfsetrectcap%
\pgfsetroundjoin%
\pgfsetlinewidth{1.505625pt}%
\definecolor{currentstroke}{rgb}{1.000000,0.000000,0.000000}%
\pgfsetstrokecolor{currentstroke}%
\pgfsetdash{}{0pt}%
\pgfpathmoveto{\pgfqpoint{2.078420in}{1.914700in}}%
\pgfpathlineto{\pgfqpoint{2.087408in}{2.103359in}}%
\pgfusepath{stroke}%
\end{pgfscope}%
\begin{pgfscope}%
\pgfpathrectangle{\pgfqpoint{0.100000in}{0.212622in}}{\pgfqpoint{3.696000in}{3.696000in}}%
\pgfusepath{clip}%
\pgfsetrectcap%
\pgfsetroundjoin%
\pgfsetlinewidth{1.505625pt}%
\definecolor{currentstroke}{rgb}{1.000000,0.000000,0.000000}%
\pgfsetstrokecolor{currentstroke}%
\pgfsetdash{}{0pt}%
\pgfpathmoveto{\pgfqpoint{2.078420in}{1.914700in}}%
\pgfpathlineto{\pgfqpoint{2.087408in}{2.103359in}}%
\pgfusepath{stroke}%
\end{pgfscope}%
\begin{pgfscope}%
\pgfpathrectangle{\pgfqpoint{0.100000in}{0.212622in}}{\pgfqpoint{3.696000in}{3.696000in}}%
\pgfusepath{clip}%
\pgfsetrectcap%
\pgfsetroundjoin%
\pgfsetlinewidth{1.505625pt}%
\definecolor{currentstroke}{rgb}{1.000000,0.000000,0.000000}%
\pgfsetstrokecolor{currentstroke}%
\pgfsetdash{}{0pt}%
\pgfpathmoveto{\pgfqpoint{2.078420in}{1.914700in}}%
\pgfpathlineto{\pgfqpoint{2.087408in}{2.103359in}}%
\pgfusepath{stroke}%
\end{pgfscope}%
\begin{pgfscope}%
\pgfpathrectangle{\pgfqpoint{0.100000in}{0.212622in}}{\pgfqpoint{3.696000in}{3.696000in}}%
\pgfusepath{clip}%
\pgfsetrectcap%
\pgfsetroundjoin%
\pgfsetlinewidth{1.505625pt}%
\definecolor{currentstroke}{rgb}{1.000000,0.000000,0.000000}%
\pgfsetstrokecolor{currentstroke}%
\pgfsetdash{}{0pt}%
\pgfpathmoveto{\pgfqpoint{2.078420in}{1.914700in}}%
\pgfpathlineto{\pgfqpoint{2.087408in}{2.103359in}}%
\pgfusepath{stroke}%
\end{pgfscope}%
\begin{pgfscope}%
\pgfpathrectangle{\pgfqpoint{0.100000in}{0.212622in}}{\pgfqpoint{3.696000in}{3.696000in}}%
\pgfusepath{clip}%
\pgfsetrectcap%
\pgfsetroundjoin%
\pgfsetlinewidth{1.505625pt}%
\definecolor{currentstroke}{rgb}{1.000000,0.000000,0.000000}%
\pgfsetstrokecolor{currentstroke}%
\pgfsetdash{}{0pt}%
\pgfpathmoveto{\pgfqpoint{2.078420in}{1.914700in}}%
\pgfpathlineto{\pgfqpoint{2.087408in}{2.103359in}}%
\pgfusepath{stroke}%
\end{pgfscope}%
\begin{pgfscope}%
\pgfpathrectangle{\pgfqpoint{0.100000in}{0.212622in}}{\pgfqpoint{3.696000in}{3.696000in}}%
\pgfusepath{clip}%
\pgfsetrectcap%
\pgfsetroundjoin%
\pgfsetlinewidth{1.505625pt}%
\definecolor{currentstroke}{rgb}{1.000000,0.000000,0.000000}%
\pgfsetstrokecolor{currentstroke}%
\pgfsetdash{}{0pt}%
\pgfpathmoveto{\pgfqpoint{2.078420in}{1.914700in}}%
\pgfpathlineto{\pgfqpoint{2.087408in}{2.103359in}}%
\pgfusepath{stroke}%
\end{pgfscope}%
\begin{pgfscope}%
\pgfpathrectangle{\pgfqpoint{0.100000in}{0.212622in}}{\pgfqpoint{3.696000in}{3.696000in}}%
\pgfusepath{clip}%
\pgfsetrectcap%
\pgfsetroundjoin%
\pgfsetlinewidth{1.505625pt}%
\definecolor{currentstroke}{rgb}{1.000000,0.000000,0.000000}%
\pgfsetstrokecolor{currentstroke}%
\pgfsetdash{}{0pt}%
\pgfpathmoveto{\pgfqpoint{2.078420in}{1.914700in}}%
\pgfpathlineto{\pgfqpoint{2.087408in}{2.103359in}}%
\pgfusepath{stroke}%
\end{pgfscope}%
\begin{pgfscope}%
\pgfpathrectangle{\pgfqpoint{0.100000in}{0.212622in}}{\pgfqpoint{3.696000in}{3.696000in}}%
\pgfusepath{clip}%
\pgfsetrectcap%
\pgfsetroundjoin%
\pgfsetlinewidth{1.505625pt}%
\definecolor{currentstroke}{rgb}{1.000000,0.000000,0.000000}%
\pgfsetstrokecolor{currentstroke}%
\pgfsetdash{}{0pt}%
\pgfpathmoveto{\pgfqpoint{2.078420in}{1.914700in}}%
\pgfpathlineto{\pgfqpoint{2.087408in}{2.103359in}}%
\pgfusepath{stroke}%
\end{pgfscope}%
\begin{pgfscope}%
\pgfpathrectangle{\pgfqpoint{0.100000in}{0.212622in}}{\pgfqpoint{3.696000in}{3.696000in}}%
\pgfusepath{clip}%
\pgfsetrectcap%
\pgfsetroundjoin%
\pgfsetlinewidth{1.505625pt}%
\definecolor{currentstroke}{rgb}{1.000000,0.000000,0.000000}%
\pgfsetstrokecolor{currentstroke}%
\pgfsetdash{}{0pt}%
\pgfpathmoveto{\pgfqpoint{2.078420in}{1.914700in}}%
\pgfpathlineto{\pgfqpoint{2.087408in}{2.103359in}}%
\pgfusepath{stroke}%
\end{pgfscope}%
\begin{pgfscope}%
\pgfpathrectangle{\pgfqpoint{0.100000in}{0.212622in}}{\pgfqpoint{3.696000in}{3.696000in}}%
\pgfusepath{clip}%
\pgfsetrectcap%
\pgfsetroundjoin%
\pgfsetlinewidth{1.505625pt}%
\definecolor{currentstroke}{rgb}{1.000000,0.000000,0.000000}%
\pgfsetstrokecolor{currentstroke}%
\pgfsetdash{}{0pt}%
\pgfpathmoveto{\pgfqpoint{2.078420in}{1.914700in}}%
\pgfpathlineto{\pgfqpoint{2.087408in}{2.103359in}}%
\pgfusepath{stroke}%
\end{pgfscope}%
\begin{pgfscope}%
\pgfpathrectangle{\pgfqpoint{0.100000in}{0.212622in}}{\pgfqpoint{3.696000in}{3.696000in}}%
\pgfusepath{clip}%
\pgfsetrectcap%
\pgfsetroundjoin%
\pgfsetlinewidth{1.505625pt}%
\definecolor{currentstroke}{rgb}{1.000000,0.000000,0.000000}%
\pgfsetstrokecolor{currentstroke}%
\pgfsetdash{}{0pt}%
\pgfpathmoveto{\pgfqpoint{2.078420in}{1.914700in}}%
\pgfpathlineto{\pgfqpoint{2.087408in}{2.103359in}}%
\pgfusepath{stroke}%
\end{pgfscope}%
\begin{pgfscope}%
\pgfpathrectangle{\pgfqpoint{0.100000in}{0.212622in}}{\pgfqpoint{3.696000in}{3.696000in}}%
\pgfusepath{clip}%
\pgfsetrectcap%
\pgfsetroundjoin%
\pgfsetlinewidth{1.505625pt}%
\definecolor{currentstroke}{rgb}{1.000000,0.000000,0.000000}%
\pgfsetstrokecolor{currentstroke}%
\pgfsetdash{}{0pt}%
\pgfpathmoveto{\pgfqpoint{2.078420in}{1.914700in}}%
\pgfpathlineto{\pgfqpoint{2.087408in}{2.103359in}}%
\pgfusepath{stroke}%
\end{pgfscope}%
\begin{pgfscope}%
\pgfpathrectangle{\pgfqpoint{0.100000in}{0.212622in}}{\pgfqpoint{3.696000in}{3.696000in}}%
\pgfusepath{clip}%
\pgfsetrectcap%
\pgfsetroundjoin%
\pgfsetlinewidth{1.505625pt}%
\definecolor{currentstroke}{rgb}{1.000000,0.000000,0.000000}%
\pgfsetstrokecolor{currentstroke}%
\pgfsetdash{}{0pt}%
\pgfpathmoveto{\pgfqpoint{2.078420in}{1.914700in}}%
\pgfpathlineto{\pgfqpoint{2.087408in}{2.103359in}}%
\pgfusepath{stroke}%
\end{pgfscope}%
\begin{pgfscope}%
\pgfpathrectangle{\pgfqpoint{0.100000in}{0.212622in}}{\pgfqpoint{3.696000in}{3.696000in}}%
\pgfusepath{clip}%
\pgfsetrectcap%
\pgfsetroundjoin%
\pgfsetlinewidth{1.505625pt}%
\definecolor{currentstroke}{rgb}{1.000000,0.000000,0.000000}%
\pgfsetstrokecolor{currentstroke}%
\pgfsetdash{}{0pt}%
\pgfpathmoveto{\pgfqpoint{2.078420in}{1.914700in}}%
\pgfpathlineto{\pgfqpoint{2.087408in}{2.103359in}}%
\pgfusepath{stroke}%
\end{pgfscope}%
\begin{pgfscope}%
\pgfpathrectangle{\pgfqpoint{0.100000in}{0.212622in}}{\pgfqpoint{3.696000in}{3.696000in}}%
\pgfusepath{clip}%
\pgfsetrectcap%
\pgfsetroundjoin%
\pgfsetlinewidth{1.505625pt}%
\definecolor{currentstroke}{rgb}{1.000000,0.000000,0.000000}%
\pgfsetstrokecolor{currentstroke}%
\pgfsetdash{}{0pt}%
\pgfpathmoveto{\pgfqpoint{2.078420in}{1.914700in}}%
\pgfpathlineto{\pgfqpoint{2.087408in}{2.103359in}}%
\pgfusepath{stroke}%
\end{pgfscope}%
\begin{pgfscope}%
\pgfpathrectangle{\pgfqpoint{0.100000in}{0.212622in}}{\pgfqpoint{3.696000in}{3.696000in}}%
\pgfusepath{clip}%
\pgfsetrectcap%
\pgfsetroundjoin%
\pgfsetlinewidth{1.505625pt}%
\definecolor{currentstroke}{rgb}{1.000000,0.000000,0.000000}%
\pgfsetstrokecolor{currentstroke}%
\pgfsetdash{}{0pt}%
\pgfpathmoveto{\pgfqpoint{2.078420in}{1.914700in}}%
\pgfpathlineto{\pgfqpoint{2.087408in}{2.103359in}}%
\pgfusepath{stroke}%
\end{pgfscope}%
\begin{pgfscope}%
\pgfpathrectangle{\pgfqpoint{0.100000in}{0.212622in}}{\pgfqpoint{3.696000in}{3.696000in}}%
\pgfusepath{clip}%
\pgfsetrectcap%
\pgfsetroundjoin%
\pgfsetlinewidth{1.505625pt}%
\definecolor{currentstroke}{rgb}{1.000000,0.000000,0.000000}%
\pgfsetstrokecolor{currentstroke}%
\pgfsetdash{}{0pt}%
\pgfpathmoveto{\pgfqpoint{2.078420in}{1.914700in}}%
\pgfpathlineto{\pgfqpoint{2.087408in}{2.103359in}}%
\pgfusepath{stroke}%
\end{pgfscope}%
\begin{pgfscope}%
\pgfpathrectangle{\pgfqpoint{0.100000in}{0.212622in}}{\pgfqpoint{3.696000in}{3.696000in}}%
\pgfusepath{clip}%
\pgfsetrectcap%
\pgfsetroundjoin%
\pgfsetlinewidth{1.505625pt}%
\definecolor{currentstroke}{rgb}{1.000000,0.000000,0.000000}%
\pgfsetstrokecolor{currentstroke}%
\pgfsetdash{}{0pt}%
\pgfpathmoveto{\pgfqpoint{2.078420in}{1.914700in}}%
\pgfpathlineto{\pgfqpoint{2.087408in}{2.103359in}}%
\pgfusepath{stroke}%
\end{pgfscope}%
\begin{pgfscope}%
\pgfpathrectangle{\pgfqpoint{0.100000in}{0.212622in}}{\pgfqpoint{3.696000in}{3.696000in}}%
\pgfusepath{clip}%
\pgfsetrectcap%
\pgfsetroundjoin%
\pgfsetlinewidth{1.505625pt}%
\definecolor{currentstroke}{rgb}{1.000000,0.000000,0.000000}%
\pgfsetstrokecolor{currentstroke}%
\pgfsetdash{}{0pt}%
\pgfpathmoveto{\pgfqpoint{2.078420in}{1.914700in}}%
\pgfpathlineto{\pgfqpoint{2.087408in}{2.103359in}}%
\pgfusepath{stroke}%
\end{pgfscope}%
\begin{pgfscope}%
\pgfpathrectangle{\pgfqpoint{0.100000in}{0.212622in}}{\pgfqpoint{3.696000in}{3.696000in}}%
\pgfusepath{clip}%
\pgfsetrectcap%
\pgfsetroundjoin%
\pgfsetlinewidth{1.505625pt}%
\definecolor{currentstroke}{rgb}{1.000000,0.000000,0.000000}%
\pgfsetstrokecolor{currentstroke}%
\pgfsetdash{}{0pt}%
\pgfpathmoveto{\pgfqpoint{2.078420in}{1.914700in}}%
\pgfpathlineto{\pgfqpoint{2.087408in}{2.103359in}}%
\pgfusepath{stroke}%
\end{pgfscope}%
\begin{pgfscope}%
\pgfpathrectangle{\pgfqpoint{0.100000in}{0.212622in}}{\pgfqpoint{3.696000in}{3.696000in}}%
\pgfusepath{clip}%
\pgfsetrectcap%
\pgfsetroundjoin%
\pgfsetlinewidth{1.505625pt}%
\definecolor{currentstroke}{rgb}{1.000000,0.000000,0.000000}%
\pgfsetstrokecolor{currentstroke}%
\pgfsetdash{}{0pt}%
\pgfpathmoveto{\pgfqpoint{2.078420in}{1.914700in}}%
\pgfpathlineto{\pgfqpoint{2.087408in}{2.103359in}}%
\pgfusepath{stroke}%
\end{pgfscope}%
\begin{pgfscope}%
\pgfpathrectangle{\pgfqpoint{0.100000in}{0.212622in}}{\pgfqpoint{3.696000in}{3.696000in}}%
\pgfusepath{clip}%
\pgfsetrectcap%
\pgfsetroundjoin%
\pgfsetlinewidth{1.505625pt}%
\definecolor{currentstroke}{rgb}{1.000000,0.000000,0.000000}%
\pgfsetstrokecolor{currentstroke}%
\pgfsetdash{}{0pt}%
\pgfpathmoveto{\pgfqpoint{2.078420in}{1.914700in}}%
\pgfpathlineto{\pgfqpoint{2.087408in}{2.103359in}}%
\pgfusepath{stroke}%
\end{pgfscope}%
\begin{pgfscope}%
\pgfpathrectangle{\pgfqpoint{0.100000in}{0.212622in}}{\pgfqpoint{3.696000in}{3.696000in}}%
\pgfusepath{clip}%
\pgfsetrectcap%
\pgfsetroundjoin%
\pgfsetlinewidth{1.505625pt}%
\definecolor{currentstroke}{rgb}{1.000000,0.000000,0.000000}%
\pgfsetstrokecolor{currentstroke}%
\pgfsetdash{}{0pt}%
\pgfpathmoveto{\pgfqpoint{2.078420in}{1.914700in}}%
\pgfpathlineto{\pgfqpoint{2.087408in}{2.103359in}}%
\pgfusepath{stroke}%
\end{pgfscope}%
\begin{pgfscope}%
\pgfpathrectangle{\pgfqpoint{0.100000in}{0.212622in}}{\pgfqpoint{3.696000in}{3.696000in}}%
\pgfusepath{clip}%
\pgfsetrectcap%
\pgfsetroundjoin%
\pgfsetlinewidth{1.505625pt}%
\definecolor{currentstroke}{rgb}{1.000000,0.000000,0.000000}%
\pgfsetstrokecolor{currentstroke}%
\pgfsetdash{}{0pt}%
\pgfpathmoveto{\pgfqpoint{2.079332in}{1.916111in}}%
\pgfpathlineto{\pgfqpoint{2.087408in}{2.103359in}}%
\pgfusepath{stroke}%
\end{pgfscope}%
\begin{pgfscope}%
\pgfpathrectangle{\pgfqpoint{0.100000in}{0.212622in}}{\pgfqpoint{3.696000in}{3.696000in}}%
\pgfusepath{clip}%
\pgfsetrectcap%
\pgfsetroundjoin%
\pgfsetlinewidth{1.505625pt}%
\definecolor{currentstroke}{rgb}{1.000000,0.000000,0.000000}%
\pgfsetstrokecolor{currentstroke}%
\pgfsetdash{}{0pt}%
\pgfpathmoveto{\pgfqpoint{2.079716in}{1.916943in}}%
\pgfpathlineto{\pgfqpoint{2.087408in}{2.103359in}}%
\pgfusepath{stroke}%
\end{pgfscope}%
\begin{pgfscope}%
\pgfpathrectangle{\pgfqpoint{0.100000in}{0.212622in}}{\pgfqpoint{3.696000in}{3.696000in}}%
\pgfusepath{clip}%
\pgfsetrectcap%
\pgfsetroundjoin%
\pgfsetlinewidth{1.505625pt}%
\definecolor{currentstroke}{rgb}{1.000000,0.000000,0.000000}%
\pgfsetstrokecolor{currentstroke}%
\pgfsetdash{}{0pt}%
\pgfpathmoveto{\pgfqpoint{2.079979in}{1.917374in}}%
\pgfpathlineto{\pgfqpoint{2.087408in}{2.103359in}}%
\pgfusepath{stroke}%
\end{pgfscope}%
\begin{pgfscope}%
\pgfpathrectangle{\pgfqpoint{0.100000in}{0.212622in}}{\pgfqpoint{3.696000in}{3.696000in}}%
\pgfusepath{clip}%
\pgfsetrectcap%
\pgfsetroundjoin%
\pgfsetlinewidth{1.505625pt}%
\definecolor{currentstroke}{rgb}{1.000000,0.000000,0.000000}%
\pgfsetstrokecolor{currentstroke}%
\pgfsetdash{}{0pt}%
\pgfpathmoveto{\pgfqpoint{2.080110in}{1.917626in}}%
\pgfpathlineto{\pgfqpoint{2.087408in}{2.103359in}}%
\pgfusepath{stroke}%
\end{pgfscope}%
\begin{pgfscope}%
\pgfpathrectangle{\pgfqpoint{0.100000in}{0.212622in}}{\pgfqpoint{3.696000in}{3.696000in}}%
\pgfusepath{clip}%
\pgfsetrectcap%
\pgfsetroundjoin%
\pgfsetlinewidth{1.505625pt}%
\definecolor{currentstroke}{rgb}{1.000000,0.000000,0.000000}%
\pgfsetstrokecolor{currentstroke}%
\pgfsetdash{}{0pt}%
\pgfpathmoveto{\pgfqpoint{2.080178in}{1.917764in}}%
\pgfpathlineto{\pgfqpoint{2.087408in}{2.103359in}}%
\pgfusepath{stroke}%
\end{pgfscope}%
\begin{pgfscope}%
\pgfpathrectangle{\pgfqpoint{0.100000in}{0.212622in}}{\pgfqpoint{3.696000in}{3.696000in}}%
\pgfusepath{clip}%
\pgfsetrectcap%
\pgfsetroundjoin%
\pgfsetlinewidth{1.505625pt}%
\definecolor{currentstroke}{rgb}{1.000000,0.000000,0.000000}%
\pgfsetstrokecolor{currentstroke}%
\pgfsetdash{}{0pt}%
\pgfpathmoveto{\pgfqpoint{2.080224in}{1.917837in}}%
\pgfpathlineto{\pgfqpoint{2.087408in}{2.103359in}}%
\pgfusepath{stroke}%
\end{pgfscope}%
\begin{pgfscope}%
\pgfpathrectangle{\pgfqpoint{0.100000in}{0.212622in}}{\pgfqpoint{3.696000in}{3.696000in}}%
\pgfusepath{clip}%
\pgfsetrectcap%
\pgfsetroundjoin%
\pgfsetlinewidth{1.505625pt}%
\definecolor{currentstroke}{rgb}{1.000000,0.000000,0.000000}%
\pgfsetstrokecolor{currentstroke}%
\pgfsetdash{}{0pt}%
\pgfpathmoveto{\pgfqpoint{2.080244in}{1.917877in}}%
\pgfpathlineto{\pgfqpoint{2.087408in}{2.103359in}}%
\pgfusepath{stroke}%
\end{pgfscope}%
\begin{pgfscope}%
\pgfpathrectangle{\pgfqpoint{0.100000in}{0.212622in}}{\pgfqpoint{3.696000in}{3.696000in}}%
\pgfusepath{clip}%
\pgfsetrectcap%
\pgfsetroundjoin%
\pgfsetlinewidth{1.505625pt}%
\definecolor{currentstroke}{rgb}{1.000000,0.000000,0.000000}%
\pgfsetstrokecolor{currentstroke}%
\pgfsetdash{}{0pt}%
\pgfpathmoveto{\pgfqpoint{2.080258in}{1.917900in}}%
\pgfpathlineto{\pgfqpoint{2.087408in}{2.103359in}}%
\pgfusepath{stroke}%
\end{pgfscope}%
\begin{pgfscope}%
\pgfpathrectangle{\pgfqpoint{0.100000in}{0.212622in}}{\pgfqpoint{3.696000in}{3.696000in}}%
\pgfusepath{clip}%
\pgfsetrectcap%
\pgfsetroundjoin%
\pgfsetlinewidth{1.505625pt}%
\definecolor{currentstroke}{rgb}{1.000000,0.000000,0.000000}%
\pgfsetstrokecolor{currentstroke}%
\pgfsetdash{}{0pt}%
\pgfpathmoveto{\pgfqpoint{2.080463in}{1.918293in}}%
\pgfpathlineto{\pgfqpoint{2.087408in}{2.103359in}}%
\pgfusepath{stroke}%
\end{pgfscope}%
\begin{pgfscope}%
\pgfpathrectangle{\pgfqpoint{0.100000in}{0.212622in}}{\pgfqpoint{3.696000in}{3.696000in}}%
\pgfusepath{clip}%
\pgfsetrectcap%
\pgfsetroundjoin%
\pgfsetlinewidth{1.505625pt}%
\definecolor{currentstroke}{rgb}{1.000000,0.000000,0.000000}%
\pgfsetstrokecolor{currentstroke}%
\pgfsetdash{}{0pt}%
\pgfpathmoveto{\pgfqpoint{2.081146in}{1.919374in}}%
\pgfpathlineto{\pgfqpoint{2.087408in}{2.103359in}}%
\pgfusepath{stroke}%
\end{pgfscope}%
\begin{pgfscope}%
\pgfpathrectangle{\pgfqpoint{0.100000in}{0.212622in}}{\pgfqpoint{3.696000in}{3.696000in}}%
\pgfusepath{clip}%
\pgfsetrectcap%
\pgfsetroundjoin%
\pgfsetlinewidth{1.505625pt}%
\definecolor{currentstroke}{rgb}{1.000000,0.000000,0.000000}%
\pgfsetstrokecolor{currentstroke}%
\pgfsetdash{}{0pt}%
\pgfpathmoveto{\pgfqpoint{2.081870in}{1.920928in}}%
\pgfpathlineto{\pgfqpoint{2.092007in}{2.107371in}}%
\pgfusepath{stroke}%
\end{pgfscope}%
\begin{pgfscope}%
\pgfpathrectangle{\pgfqpoint{0.100000in}{0.212622in}}{\pgfqpoint{3.696000in}{3.696000in}}%
\pgfusepath{clip}%
\pgfsetrectcap%
\pgfsetroundjoin%
\pgfsetlinewidth{1.505625pt}%
\definecolor{currentstroke}{rgb}{1.000000,0.000000,0.000000}%
\pgfsetstrokecolor{currentstroke}%
\pgfsetdash{}{0pt}%
\pgfpathmoveto{\pgfqpoint{2.082341in}{1.921828in}}%
\pgfpathlineto{\pgfqpoint{2.092007in}{2.107371in}}%
\pgfusepath{stroke}%
\end{pgfscope}%
\begin{pgfscope}%
\pgfpathrectangle{\pgfqpoint{0.100000in}{0.212622in}}{\pgfqpoint{3.696000in}{3.696000in}}%
\pgfusepath{clip}%
\pgfsetrectcap%
\pgfsetroundjoin%
\pgfsetlinewidth{1.505625pt}%
\definecolor{currentstroke}{rgb}{1.000000,0.000000,0.000000}%
\pgfsetstrokecolor{currentstroke}%
\pgfsetdash{}{0pt}%
\pgfpathmoveto{\pgfqpoint{2.082615in}{1.922295in}}%
\pgfpathlineto{\pgfqpoint{2.092007in}{2.107371in}}%
\pgfusepath{stroke}%
\end{pgfscope}%
\begin{pgfscope}%
\pgfpathrectangle{\pgfqpoint{0.100000in}{0.212622in}}{\pgfqpoint{3.696000in}{3.696000in}}%
\pgfusepath{clip}%
\pgfsetrectcap%
\pgfsetroundjoin%
\pgfsetlinewidth{1.505625pt}%
\definecolor{currentstroke}{rgb}{1.000000,0.000000,0.000000}%
\pgfsetstrokecolor{currentstroke}%
\pgfsetdash{}{0pt}%
\pgfpathmoveto{\pgfqpoint{2.082731in}{1.922564in}}%
\pgfpathlineto{\pgfqpoint{2.092007in}{2.107371in}}%
\pgfusepath{stroke}%
\end{pgfscope}%
\begin{pgfscope}%
\pgfpathrectangle{\pgfqpoint{0.100000in}{0.212622in}}{\pgfqpoint{3.696000in}{3.696000in}}%
\pgfusepath{clip}%
\pgfsetrectcap%
\pgfsetroundjoin%
\pgfsetlinewidth{1.505625pt}%
\definecolor{currentstroke}{rgb}{1.000000,0.000000,0.000000}%
\pgfsetstrokecolor{currentstroke}%
\pgfsetdash{}{0pt}%
\pgfpathmoveto{\pgfqpoint{2.083165in}{1.923297in}}%
\pgfpathlineto{\pgfqpoint{2.092007in}{2.107371in}}%
\pgfusepath{stroke}%
\end{pgfscope}%
\begin{pgfscope}%
\pgfpathrectangle{\pgfqpoint{0.100000in}{0.212622in}}{\pgfqpoint{3.696000in}{3.696000in}}%
\pgfusepath{clip}%
\pgfsetrectcap%
\pgfsetroundjoin%
\pgfsetlinewidth{1.505625pt}%
\definecolor{currentstroke}{rgb}{1.000000,0.000000,0.000000}%
\pgfsetstrokecolor{currentstroke}%
\pgfsetdash{}{0pt}%
\pgfpathmoveto{\pgfqpoint{2.083715in}{1.924656in}}%
\pgfpathlineto{\pgfqpoint{2.092007in}{2.107371in}}%
\pgfusepath{stroke}%
\end{pgfscope}%
\begin{pgfscope}%
\pgfpathrectangle{\pgfqpoint{0.100000in}{0.212622in}}{\pgfqpoint{3.696000in}{3.696000in}}%
\pgfusepath{clip}%
\pgfsetrectcap%
\pgfsetroundjoin%
\pgfsetlinewidth{1.505625pt}%
\definecolor{currentstroke}{rgb}{1.000000,0.000000,0.000000}%
\pgfsetstrokecolor{currentstroke}%
\pgfsetdash{}{0pt}%
\pgfpathmoveto{\pgfqpoint{2.084906in}{1.926649in}}%
\pgfpathlineto{\pgfqpoint{2.092007in}{2.107371in}}%
\pgfusepath{stroke}%
\end{pgfscope}%
\begin{pgfscope}%
\pgfpathrectangle{\pgfqpoint{0.100000in}{0.212622in}}{\pgfqpoint{3.696000in}{3.696000in}}%
\pgfusepath{clip}%
\pgfsetrectcap%
\pgfsetroundjoin%
\pgfsetlinewidth{1.505625pt}%
\definecolor{currentstroke}{rgb}{1.000000,0.000000,0.000000}%
\pgfsetstrokecolor{currentstroke}%
\pgfsetdash{}{0pt}%
\pgfpathmoveto{\pgfqpoint{2.085984in}{1.929145in}}%
\pgfpathlineto{\pgfqpoint{2.096603in}{2.111380in}}%
\pgfusepath{stroke}%
\end{pgfscope}%
\begin{pgfscope}%
\pgfpathrectangle{\pgfqpoint{0.100000in}{0.212622in}}{\pgfqpoint{3.696000in}{3.696000in}}%
\pgfusepath{clip}%
\pgfsetrectcap%
\pgfsetroundjoin%
\pgfsetlinewidth{1.505625pt}%
\definecolor{currentstroke}{rgb}{1.000000,0.000000,0.000000}%
\pgfsetstrokecolor{currentstroke}%
\pgfsetdash{}{0pt}%
\pgfpathmoveto{\pgfqpoint{2.086719in}{1.930471in}}%
\pgfpathlineto{\pgfqpoint{2.096603in}{2.111380in}}%
\pgfusepath{stroke}%
\end{pgfscope}%
\begin{pgfscope}%
\pgfpathrectangle{\pgfqpoint{0.100000in}{0.212622in}}{\pgfqpoint{3.696000in}{3.696000in}}%
\pgfusepath{clip}%
\pgfsetrectcap%
\pgfsetroundjoin%
\pgfsetlinewidth{1.505625pt}%
\definecolor{currentstroke}{rgb}{1.000000,0.000000,0.000000}%
\pgfsetstrokecolor{currentstroke}%
\pgfsetdash{}{0pt}%
\pgfpathmoveto{\pgfqpoint{2.087112in}{1.931187in}}%
\pgfpathlineto{\pgfqpoint{2.096603in}{2.111380in}}%
\pgfusepath{stroke}%
\end{pgfscope}%
\begin{pgfscope}%
\pgfpathrectangle{\pgfqpoint{0.100000in}{0.212622in}}{\pgfqpoint{3.696000in}{3.696000in}}%
\pgfusepath{clip}%
\pgfsetrectcap%
\pgfsetroundjoin%
\pgfsetlinewidth{1.505625pt}%
\definecolor{currentstroke}{rgb}{1.000000,0.000000,0.000000}%
\pgfsetstrokecolor{currentstroke}%
\pgfsetdash{}{0pt}%
\pgfpathmoveto{\pgfqpoint{2.087260in}{1.931601in}}%
\pgfpathlineto{\pgfqpoint{2.096603in}{2.111380in}}%
\pgfusepath{stroke}%
\end{pgfscope}%
\begin{pgfscope}%
\pgfpathrectangle{\pgfqpoint{0.100000in}{0.212622in}}{\pgfqpoint{3.696000in}{3.696000in}}%
\pgfusepath{clip}%
\pgfsetrectcap%
\pgfsetroundjoin%
\pgfsetlinewidth{1.505625pt}%
\definecolor{currentstroke}{rgb}{1.000000,0.000000,0.000000}%
\pgfsetstrokecolor{currentstroke}%
\pgfsetdash{}{0pt}%
\pgfpathmoveto{\pgfqpoint{2.087932in}{1.932748in}}%
\pgfpathlineto{\pgfqpoint{2.096603in}{2.111380in}}%
\pgfusepath{stroke}%
\end{pgfscope}%
\begin{pgfscope}%
\pgfpathrectangle{\pgfqpoint{0.100000in}{0.212622in}}{\pgfqpoint{3.696000in}{3.696000in}}%
\pgfusepath{clip}%
\pgfsetrectcap%
\pgfsetroundjoin%
\pgfsetlinewidth{1.505625pt}%
\definecolor{currentstroke}{rgb}{1.000000,0.000000,0.000000}%
\pgfsetstrokecolor{currentstroke}%
\pgfsetdash{}{0pt}%
\pgfpathmoveto{\pgfqpoint{2.088634in}{1.934406in}}%
\pgfpathlineto{\pgfqpoint{2.096603in}{2.111380in}}%
\pgfusepath{stroke}%
\end{pgfscope}%
\begin{pgfscope}%
\pgfpathrectangle{\pgfqpoint{0.100000in}{0.212622in}}{\pgfqpoint{3.696000in}{3.696000in}}%
\pgfusepath{clip}%
\pgfsetrectcap%
\pgfsetroundjoin%
\pgfsetlinewidth{1.505625pt}%
\definecolor{currentstroke}{rgb}{1.000000,0.000000,0.000000}%
\pgfsetstrokecolor{currentstroke}%
\pgfsetdash{}{0pt}%
\pgfpathmoveto{\pgfqpoint{2.089111in}{1.935229in}}%
\pgfpathlineto{\pgfqpoint{2.096603in}{2.111380in}}%
\pgfusepath{stroke}%
\end{pgfscope}%
\begin{pgfscope}%
\pgfpathrectangle{\pgfqpoint{0.100000in}{0.212622in}}{\pgfqpoint{3.696000in}{3.696000in}}%
\pgfusepath{clip}%
\pgfsetrectcap%
\pgfsetroundjoin%
\pgfsetlinewidth{1.505625pt}%
\definecolor{currentstroke}{rgb}{1.000000,0.000000,0.000000}%
\pgfsetstrokecolor{currentstroke}%
\pgfsetdash{}{0pt}%
\pgfpathmoveto{\pgfqpoint{2.089389in}{1.935702in}}%
\pgfpathlineto{\pgfqpoint{2.096603in}{2.111380in}}%
\pgfusepath{stroke}%
\end{pgfscope}%
\begin{pgfscope}%
\pgfpathrectangle{\pgfqpoint{0.100000in}{0.212622in}}{\pgfqpoint{3.696000in}{3.696000in}}%
\pgfusepath{clip}%
\pgfsetrectcap%
\pgfsetroundjoin%
\pgfsetlinewidth{1.505625pt}%
\definecolor{currentstroke}{rgb}{1.000000,0.000000,0.000000}%
\pgfsetstrokecolor{currentstroke}%
\pgfsetdash{}{0pt}%
\pgfpathmoveto{\pgfqpoint{2.089487in}{1.935973in}}%
\pgfpathlineto{\pgfqpoint{2.096603in}{2.111380in}}%
\pgfusepath{stroke}%
\end{pgfscope}%
\begin{pgfscope}%
\pgfpathrectangle{\pgfqpoint{0.100000in}{0.212622in}}{\pgfqpoint{3.696000in}{3.696000in}}%
\pgfusepath{clip}%
\pgfsetrectcap%
\pgfsetroundjoin%
\pgfsetlinewidth{1.505625pt}%
\definecolor{currentstroke}{rgb}{1.000000,0.000000,0.000000}%
\pgfsetstrokecolor{currentstroke}%
\pgfsetdash{}{0pt}%
\pgfpathmoveto{\pgfqpoint{2.089573in}{1.936122in}}%
\pgfpathlineto{\pgfqpoint{2.096603in}{2.111380in}}%
\pgfusepath{stroke}%
\end{pgfscope}%
\begin{pgfscope}%
\pgfpathrectangle{\pgfqpoint{0.100000in}{0.212622in}}{\pgfqpoint{3.696000in}{3.696000in}}%
\pgfusepath{clip}%
\pgfsetrectcap%
\pgfsetroundjoin%
\pgfsetlinewidth{1.505625pt}%
\definecolor{currentstroke}{rgb}{1.000000,0.000000,0.000000}%
\pgfsetstrokecolor{currentstroke}%
\pgfsetdash{}{0pt}%
\pgfpathmoveto{\pgfqpoint{2.089863in}{1.937123in}}%
\pgfpathlineto{\pgfqpoint{2.096603in}{2.111380in}}%
\pgfusepath{stroke}%
\end{pgfscope}%
\begin{pgfscope}%
\pgfpathrectangle{\pgfqpoint{0.100000in}{0.212622in}}{\pgfqpoint{3.696000in}{3.696000in}}%
\pgfusepath{clip}%
\pgfsetrectcap%
\pgfsetroundjoin%
\pgfsetlinewidth{1.505625pt}%
\definecolor{currentstroke}{rgb}{1.000000,0.000000,0.000000}%
\pgfsetstrokecolor{currentstroke}%
\pgfsetdash{}{0pt}%
\pgfpathmoveto{\pgfqpoint{2.091846in}{1.940089in}}%
\pgfpathlineto{\pgfqpoint{2.101195in}{2.115387in}}%
\pgfusepath{stroke}%
\end{pgfscope}%
\begin{pgfscope}%
\pgfpathrectangle{\pgfqpoint{0.100000in}{0.212622in}}{\pgfqpoint{3.696000in}{3.696000in}}%
\pgfusepath{clip}%
\pgfsetrectcap%
\pgfsetroundjoin%
\pgfsetlinewidth{1.505625pt}%
\definecolor{currentstroke}{rgb}{1.000000,0.000000,0.000000}%
\pgfsetstrokecolor{currentstroke}%
\pgfsetdash{}{0pt}%
\pgfpathmoveto{\pgfqpoint{2.093580in}{1.944012in}}%
\pgfpathlineto{\pgfqpoint{2.101195in}{2.115387in}}%
\pgfusepath{stroke}%
\end{pgfscope}%
\begin{pgfscope}%
\pgfpathrectangle{\pgfqpoint{0.100000in}{0.212622in}}{\pgfqpoint{3.696000in}{3.696000in}}%
\pgfusepath{clip}%
\pgfsetrectcap%
\pgfsetroundjoin%
\pgfsetlinewidth{1.505625pt}%
\definecolor{currentstroke}{rgb}{1.000000,0.000000,0.000000}%
\pgfsetstrokecolor{currentstroke}%
\pgfsetdash{}{0pt}%
\pgfpathmoveto{\pgfqpoint{2.096060in}{1.948343in}}%
\pgfpathlineto{\pgfqpoint{2.105784in}{2.119391in}}%
\pgfusepath{stroke}%
\end{pgfscope}%
\begin{pgfscope}%
\pgfpathrectangle{\pgfqpoint{0.100000in}{0.212622in}}{\pgfqpoint{3.696000in}{3.696000in}}%
\pgfusepath{clip}%
\pgfsetrectcap%
\pgfsetroundjoin%
\pgfsetlinewidth{1.505625pt}%
\definecolor{currentstroke}{rgb}{1.000000,0.000000,0.000000}%
\pgfsetstrokecolor{currentstroke}%
\pgfsetdash{}{0pt}%
\pgfpathmoveto{\pgfqpoint{2.098715in}{1.953229in}}%
\pgfpathlineto{\pgfqpoint{2.105784in}{2.119391in}}%
\pgfusepath{stroke}%
\end{pgfscope}%
\begin{pgfscope}%
\pgfpathrectangle{\pgfqpoint{0.100000in}{0.212622in}}{\pgfqpoint{3.696000in}{3.696000in}}%
\pgfusepath{clip}%
\pgfsetrectcap%
\pgfsetroundjoin%
\pgfsetlinewidth{1.505625pt}%
\definecolor{currentstroke}{rgb}{1.000000,0.000000,0.000000}%
\pgfsetstrokecolor{currentstroke}%
\pgfsetdash{}{0pt}%
\pgfpathmoveto{\pgfqpoint{2.099832in}{1.955985in}}%
\pgfpathlineto{\pgfqpoint{2.110371in}{2.123392in}}%
\pgfusepath{stroke}%
\end{pgfscope}%
\begin{pgfscope}%
\pgfpathrectangle{\pgfqpoint{0.100000in}{0.212622in}}{\pgfqpoint{3.696000in}{3.696000in}}%
\pgfusepath{clip}%
\pgfsetrectcap%
\pgfsetroundjoin%
\pgfsetlinewidth{1.505625pt}%
\definecolor{currentstroke}{rgb}{1.000000,0.000000,0.000000}%
\pgfsetstrokecolor{currentstroke}%
\pgfsetdash{}{0pt}%
\pgfpathmoveto{\pgfqpoint{2.100705in}{1.957448in}}%
\pgfpathlineto{\pgfqpoint{2.110371in}{2.123392in}}%
\pgfusepath{stroke}%
\end{pgfscope}%
\begin{pgfscope}%
\pgfpathrectangle{\pgfqpoint{0.100000in}{0.212622in}}{\pgfqpoint{3.696000in}{3.696000in}}%
\pgfusepath{clip}%
\pgfsetrectcap%
\pgfsetroundjoin%
\pgfsetlinewidth{1.505625pt}%
\definecolor{currentstroke}{rgb}{1.000000,0.000000,0.000000}%
\pgfsetstrokecolor{currentstroke}%
\pgfsetdash{}{0pt}%
\pgfpathmoveto{\pgfqpoint{2.101418in}{1.959313in}}%
\pgfpathlineto{\pgfqpoint{2.110371in}{2.123392in}}%
\pgfusepath{stroke}%
\end{pgfscope}%
\begin{pgfscope}%
\pgfpathrectangle{\pgfqpoint{0.100000in}{0.212622in}}{\pgfqpoint{3.696000in}{3.696000in}}%
\pgfusepath{clip}%
\pgfsetrectcap%
\pgfsetroundjoin%
\pgfsetlinewidth{1.505625pt}%
\definecolor{currentstroke}{rgb}{1.000000,0.000000,0.000000}%
\pgfsetstrokecolor{currentstroke}%
\pgfsetdash{}{0pt}%
\pgfpathmoveto{\pgfqpoint{2.102829in}{1.961418in}}%
\pgfpathlineto{\pgfqpoint{2.110371in}{2.123392in}}%
\pgfusepath{stroke}%
\end{pgfscope}%
\begin{pgfscope}%
\pgfpathrectangle{\pgfqpoint{0.100000in}{0.212622in}}{\pgfqpoint{3.696000in}{3.696000in}}%
\pgfusepath{clip}%
\pgfsetrectcap%
\pgfsetroundjoin%
\pgfsetlinewidth{1.505625pt}%
\definecolor{currentstroke}{rgb}{1.000000,0.000000,0.000000}%
\pgfsetstrokecolor{currentstroke}%
\pgfsetdash{}{0pt}%
\pgfpathmoveto{\pgfqpoint{2.103982in}{1.964350in}}%
\pgfpathlineto{\pgfqpoint{2.114954in}{2.127390in}}%
\pgfusepath{stroke}%
\end{pgfscope}%
\begin{pgfscope}%
\pgfpathrectangle{\pgfqpoint{0.100000in}{0.212622in}}{\pgfqpoint{3.696000in}{3.696000in}}%
\pgfusepath{clip}%
\pgfsetrectcap%
\pgfsetroundjoin%
\pgfsetlinewidth{1.505625pt}%
\definecolor{currentstroke}{rgb}{1.000000,0.000000,0.000000}%
\pgfsetstrokecolor{currentstroke}%
\pgfsetdash{}{0pt}%
\pgfpathmoveto{\pgfqpoint{2.106484in}{1.968020in}}%
\pgfpathlineto{\pgfqpoint{2.114954in}{2.127390in}}%
\pgfusepath{stroke}%
\end{pgfscope}%
\begin{pgfscope}%
\pgfpathrectangle{\pgfqpoint{0.100000in}{0.212622in}}{\pgfqpoint{3.696000in}{3.696000in}}%
\pgfusepath{clip}%
\pgfsetrectcap%
\pgfsetroundjoin%
\pgfsetlinewidth{1.505625pt}%
\definecolor{currentstroke}{rgb}{1.000000,0.000000,0.000000}%
\pgfsetstrokecolor{currentstroke}%
\pgfsetdash{}{0pt}%
\pgfpathmoveto{\pgfqpoint{2.108438in}{1.972649in}}%
\pgfpathlineto{\pgfqpoint{2.119533in}{2.131386in}}%
\pgfusepath{stroke}%
\end{pgfscope}%
\begin{pgfscope}%
\pgfpathrectangle{\pgfqpoint{0.100000in}{0.212622in}}{\pgfqpoint{3.696000in}{3.696000in}}%
\pgfusepath{clip}%
\pgfsetrectcap%
\pgfsetroundjoin%
\pgfsetlinewidth{1.505625pt}%
\definecolor{currentstroke}{rgb}{1.000000,0.000000,0.000000}%
\pgfsetstrokecolor{currentstroke}%
\pgfsetdash{}{0pt}%
\pgfpathmoveto{\pgfqpoint{2.109862in}{1.975054in}}%
\pgfpathlineto{\pgfqpoint{2.119533in}{2.131386in}}%
\pgfusepath{stroke}%
\end{pgfscope}%
\begin{pgfscope}%
\pgfpathrectangle{\pgfqpoint{0.100000in}{0.212622in}}{\pgfqpoint{3.696000in}{3.696000in}}%
\pgfusepath{clip}%
\pgfsetrectcap%
\pgfsetroundjoin%
\pgfsetlinewidth{1.505625pt}%
\definecolor{currentstroke}{rgb}{1.000000,0.000000,0.000000}%
\pgfsetstrokecolor{currentstroke}%
\pgfsetdash{}{0pt}%
\pgfpathmoveto{\pgfqpoint{2.110547in}{1.976476in}}%
\pgfpathlineto{\pgfqpoint{2.119533in}{2.131386in}}%
\pgfusepath{stroke}%
\end{pgfscope}%
\begin{pgfscope}%
\pgfpathrectangle{\pgfqpoint{0.100000in}{0.212622in}}{\pgfqpoint{3.696000in}{3.696000in}}%
\pgfusepath{clip}%
\pgfsetrectcap%
\pgfsetroundjoin%
\pgfsetlinewidth{1.505625pt}%
\definecolor{currentstroke}{rgb}{1.000000,0.000000,0.000000}%
\pgfsetstrokecolor{currentstroke}%
\pgfsetdash{}{0pt}%
\pgfpathmoveto{\pgfqpoint{2.111624in}{1.978265in}}%
\pgfpathlineto{\pgfqpoint{2.119533in}{2.131386in}}%
\pgfusepath{stroke}%
\end{pgfscope}%
\begin{pgfscope}%
\pgfpathrectangle{\pgfqpoint{0.100000in}{0.212622in}}{\pgfqpoint{3.696000in}{3.696000in}}%
\pgfusepath{clip}%
\pgfsetrectcap%
\pgfsetroundjoin%
\pgfsetlinewidth{1.505625pt}%
\definecolor{currentstroke}{rgb}{1.000000,0.000000,0.000000}%
\pgfsetstrokecolor{currentstroke}%
\pgfsetdash{}{0pt}%
\pgfpathmoveto{\pgfqpoint{2.112237in}{1.979258in}}%
\pgfpathlineto{\pgfqpoint{2.119533in}{2.131386in}}%
\pgfusepath{stroke}%
\end{pgfscope}%
\begin{pgfscope}%
\pgfpathrectangle{\pgfqpoint{0.100000in}{0.212622in}}{\pgfqpoint{3.696000in}{3.696000in}}%
\pgfusepath{clip}%
\pgfsetrectcap%
\pgfsetroundjoin%
\pgfsetlinewidth{1.505625pt}%
\definecolor{currentstroke}{rgb}{1.000000,0.000000,0.000000}%
\pgfsetstrokecolor{currentstroke}%
\pgfsetdash{}{0pt}%
\pgfpathmoveto{\pgfqpoint{2.112501in}{1.979860in}}%
\pgfpathlineto{\pgfqpoint{2.119533in}{2.131386in}}%
\pgfusepath{stroke}%
\end{pgfscope}%
\begin{pgfscope}%
\pgfpathrectangle{\pgfqpoint{0.100000in}{0.212622in}}{\pgfqpoint{3.696000in}{3.696000in}}%
\pgfusepath{clip}%
\pgfsetrectcap%
\pgfsetroundjoin%
\pgfsetlinewidth{1.505625pt}%
\definecolor{currentstroke}{rgb}{1.000000,0.000000,0.000000}%
\pgfsetstrokecolor{currentstroke}%
\pgfsetdash{}{0pt}%
\pgfpathmoveto{\pgfqpoint{2.113526in}{1.981406in}}%
\pgfpathlineto{\pgfqpoint{2.124110in}{2.135379in}}%
\pgfusepath{stroke}%
\end{pgfscope}%
\begin{pgfscope}%
\pgfpathrectangle{\pgfqpoint{0.100000in}{0.212622in}}{\pgfqpoint{3.696000in}{3.696000in}}%
\pgfusepath{clip}%
\pgfsetrectcap%
\pgfsetroundjoin%
\pgfsetlinewidth{1.505625pt}%
\definecolor{currentstroke}{rgb}{1.000000,0.000000,0.000000}%
\pgfsetstrokecolor{currentstroke}%
\pgfsetdash{}{0pt}%
\pgfpathmoveto{\pgfqpoint{2.114884in}{1.983307in}}%
\pgfpathlineto{\pgfqpoint{2.124110in}{2.135379in}}%
\pgfusepath{stroke}%
\end{pgfscope}%
\begin{pgfscope}%
\pgfpathrectangle{\pgfqpoint{0.100000in}{0.212622in}}{\pgfqpoint{3.696000in}{3.696000in}}%
\pgfusepath{clip}%
\pgfsetrectcap%
\pgfsetroundjoin%
\pgfsetlinewidth{1.505625pt}%
\definecolor{currentstroke}{rgb}{1.000000,0.000000,0.000000}%
\pgfsetstrokecolor{currentstroke}%
\pgfsetdash{}{0pt}%
\pgfpathmoveto{\pgfqpoint{2.115502in}{1.984263in}}%
\pgfpathlineto{\pgfqpoint{2.124110in}{2.135379in}}%
\pgfusepath{stroke}%
\end{pgfscope}%
\begin{pgfscope}%
\pgfpathrectangle{\pgfqpoint{0.100000in}{0.212622in}}{\pgfqpoint{3.696000in}{3.696000in}}%
\pgfusepath{clip}%
\pgfsetrectcap%
\pgfsetroundjoin%
\pgfsetlinewidth{1.505625pt}%
\definecolor{currentstroke}{rgb}{1.000000,0.000000,0.000000}%
\pgfsetstrokecolor{currentstroke}%
\pgfsetdash{}{0pt}%
\pgfpathmoveto{\pgfqpoint{2.115881in}{1.984867in}}%
\pgfpathlineto{\pgfqpoint{2.124110in}{2.135379in}}%
\pgfusepath{stroke}%
\end{pgfscope}%
\begin{pgfscope}%
\pgfpathrectangle{\pgfqpoint{0.100000in}{0.212622in}}{\pgfqpoint{3.696000in}{3.696000in}}%
\pgfusepath{clip}%
\pgfsetrectcap%
\pgfsetroundjoin%
\pgfsetlinewidth{1.505625pt}%
\definecolor{currentstroke}{rgb}{1.000000,0.000000,0.000000}%
\pgfsetstrokecolor{currentstroke}%
\pgfsetdash{}{0pt}%
\pgfpathmoveto{\pgfqpoint{2.116459in}{1.986125in}}%
\pgfpathlineto{\pgfqpoint{2.124110in}{2.135379in}}%
\pgfusepath{stroke}%
\end{pgfscope}%
\begin{pgfscope}%
\pgfpathrectangle{\pgfqpoint{0.100000in}{0.212622in}}{\pgfqpoint{3.696000in}{3.696000in}}%
\pgfusepath{clip}%
\pgfsetrectcap%
\pgfsetroundjoin%
\pgfsetlinewidth{1.505625pt}%
\definecolor{currentstroke}{rgb}{1.000000,0.000000,0.000000}%
\pgfsetstrokecolor{currentstroke}%
\pgfsetdash{}{0pt}%
\pgfpathmoveto{\pgfqpoint{2.117666in}{1.987957in}}%
\pgfpathlineto{\pgfqpoint{2.124110in}{2.135379in}}%
\pgfusepath{stroke}%
\end{pgfscope}%
\begin{pgfscope}%
\pgfpathrectangle{\pgfqpoint{0.100000in}{0.212622in}}{\pgfqpoint{3.696000in}{3.696000in}}%
\pgfusepath{clip}%
\pgfsetrectcap%
\pgfsetroundjoin%
\pgfsetlinewidth{1.505625pt}%
\definecolor{currentstroke}{rgb}{1.000000,0.000000,0.000000}%
\pgfsetstrokecolor{currentstroke}%
\pgfsetdash{}{0pt}%
\pgfpathmoveto{\pgfqpoint{2.118232in}{1.989120in}}%
\pgfpathlineto{\pgfqpoint{2.128684in}{2.139369in}}%
\pgfusepath{stroke}%
\end{pgfscope}%
\begin{pgfscope}%
\pgfpathrectangle{\pgfqpoint{0.100000in}{0.212622in}}{\pgfqpoint{3.696000in}{3.696000in}}%
\pgfusepath{clip}%
\pgfsetrectcap%
\pgfsetroundjoin%
\pgfsetlinewidth{1.505625pt}%
\definecolor{currentstroke}{rgb}{1.000000,0.000000,0.000000}%
\pgfsetstrokecolor{currentstroke}%
\pgfsetdash{}{0pt}%
\pgfpathmoveto{\pgfqpoint{2.118522in}{1.989664in}}%
\pgfpathlineto{\pgfqpoint{2.128684in}{2.139369in}}%
\pgfusepath{stroke}%
\end{pgfscope}%
\begin{pgfscope}%
\pgfpathrectangle{\pgfqpoint{0.100000in}{0.212622in}}{\pgfqpoint{3.696000in}{3.696000in}}%
\pgfusepath{clip}%
\pgfsetrectcap%
\pgfsetroundjoin%
\pgfsetlinewidth{1.505625pt}%
\definecolor{currentstroke}{rgb}{1.000000,0.000000,0.000000}%
\pgfsetstrokecolor{currentstroke}%
\pgfsetdash{}{0pt}%
\pgfpathmoveto{\pgfqpoint{2.118727in}{1.989986in}}%
\pgfpathlineto{\pgfqpoint{2.128684in}{2.139369in}}%
\pgfusepath{stroke}%
\end{pgfscope}%
\begin{pgfscope}%
\pgfpathrectangle{\pgfqpoint{0.100000in}{0.212622in}}{\pgfqpoint{3.696000in}{3.696000in}}%
\pgfusepath{clip}%
\pgfsetrectcap%
\pgfsetroundjoin%
\pgfsetlinewidth{1.505625pt}%
\definecolor{currentstroke}{rgb}{1.000000,0.000000,0.000000}%
\pgfsetstrokecolor{currentstroke}%
\pgfsetdash{}{0pt}%
\pgfpathmoveto{\pgfqpoint{2.118812in}{1.990161in}}%
\pgfpathlineto{\pgfqpoint{2.128684in}{2.139369in}}%
\pgfusepath{stroke}%
\end{pgfscope}%
\begin{pgfscope}%
\pgfpathrectangle{\pgfqpoint{0.100000in}{0.212622in}}{\pgfqpoint{3.696000in}{3.696000in}}%
\pgfusepath{clip}%
\pgfsetrectcap%
\pgfsetroundjoin%
\pgfsetlinewidth{1.505625pt}%
\definecolor{currentstroke}{rgb}{1.000000,0.000000,0.000000}%
\pgfsetstrokecolor{currentstroke}%
\pgfsetdash{}{0pt}%
\pgfpathmoveto{\pgfqpoint{2.119237in}{1.990880in}}%
\pgfpathlineto{\pgfqpoint{2.128684in}{2.139369in}}%
\pgfusepath{stroke}%
\end{pgfscope}%
\begin{pgfscope}%
\pgfpathrectangle{\pgfqpoint{0.100000in}{0.212622in}}{\pgfqpoint{3.696000in}{3.696000in}}%
\pgfusepath{clip}%
\pgfsetrectcap%
\pgfsetroundjoin%
\pgfsetlinewidth{1.505625pt}%
\definecolor{currentstroke}{rgb}{1.000000,0.000000,0.000000}%
\pgfsetstrokecolor{currentstroke}%
\pgfsetdash{}{0pt}%
\pgfpathmoveto{\pgfqpoint{2.120144in}{1.992213in}}%
\pgfpathlineto{\pgfqpoint{2.128684in}{2.139369in}}%
\pgfusepath{stroke}%
\end{pgfscope}%
\begin{pgfscope}%
\pgfpathrectangle{\pgfqpoint{0.100000in}{0.212622in}}{\pgfqpoint{3.696000in}{3.696000in}}%
\pgfusepath{clip}%
\pgfsetrectcap%
\pgfsetroundjoin%
\pgfsetlinewidth{1.505625pt}%
\definecolor{currentstroke}{rgb}{1.000000,0.000000,0.000000}%
\pgfsetstrokecolor{currentstroke}%
\pgfsetdash{}{0pt}%
\pgfpathmoveto{\pgfqpoint{2.121851in}{1.995990in}}%
\pgfpathlineto{\pgfqpoint{2.128684in}{2.139369in}}%
\pgfusepath{stroke}%
\end{pgfscope}%
\begin{pgfscope}%
\pgfpathrectangle{\pgfqpoint{0.100000in}{0.212622in}}{\pgfqpoint{3.696000in}{3.696000in}}%
\pgfusepath{clip}%
\pgfsetrectcap%
\pgfsetroundjoin%
\pgfsetlinewidth{1.505625pt}%
\definecolor{currentstroke}{rgb}{1.000000,0.000000,0.000000}%
\pgfsetstrokecolor{currentstroke}%
\pgfsetdash{}{0pt}%
\pgfpathmoveto{\pgfqpoint{2.122636in}{1.998123in}}%
\pgfpathlineto{\pgfqpoint{2.133254in}{2.143356in}}%
\pgfusepath{stroke}%
\end{pgfscope}%
\begin{pgfscope}%
\pgfpathrectangle{\pgfqpoint{0.100000in}{0.212622in}}{\pgfqpoint{3.696000in}{3.696000in}}%
\pgfusepath{clip}%
\pgfsetrectcap%
\pgfsetroundjoin%
\pgfsetlinewidth{1.505625pt}%
\definecolor{currentstroke}{rgb}{1.000000,0.000000,0.000000}%
\pgfsetstrokecolor{currentstroke}%
\pgfsetdash{}{0pt}%
\pgfpathmoveto{\pgfqpoint{2.125049in}{2.001882in}}%
\pgfpathlineto{\pgfqpoint{2.133254in}{2.143356in}}%
\pgfusepath{stroke}%
\end{pgfscope}%
\begin{pgfscope}%
\pgfpathrectangle{\pgfqpoint{0.100000in}{0.212622in}}{\pgfqpoint{3.696000in}{3.696000in}}%
\pgfusepath{clip}%
\pgfsetrectcap%
\pgfsetroundjoin%
\pgfsetlinewidth{1.505625pt}%
\definecolor{currentstroke}{rgb}{1.000000,0.000000,0.000000}%
\pgfsetstrokecolor{currentstroke}%
\pgfsetdash{}{0pt}%
\pgfpathmoveto{\pgfqpoint{2.126059in}{2.003931in}}%
\pgfpathlineto{\pgfqpoint{2.133254in}{2.143356in}}%
\pgfusepath{stroke}%
\end{pgfscope}%
\begin{pgfscope}%
\pgfpathrectangle{\pgfqpoint{0.100000in}{0.212622in}}{\pgfqpoint{3.696000in}{3.696000in}}%
\pgfusepath{clip}%
\pgfsetrectcap%
\pgfsetroundjoin%
\pgfsetlinewidth{1.505625pt}%
\definecolor{currentstroke}{rgb}{1.000000,0.000000,0.000000}%
\pgfsetstrokecolor{currentstroke}%
\pgfsetdash{}{0pt}%
\pgfpathmoveto{\pgfqpoint{2.126918in}{2.006587in}}%
\pgfpathlineto{\pgfqpoint{2.133254in}{2.143356in}}%
\pgfusepath{stroke}%
\end{pgfscope}%
\begin{pgfscope}%
\pgfpathrectangle{\pgfqpoint{0.100000in}{0.212622in}}{\pgfqpoint{3.696000in}{3.696000in}}%
\pgfusepath{clip}%
\pgfsetrectcap%
\pgfsetroundjoin%
\pgfsetlinewidth{1.505625pt}%
\definecolor{currentstroke}{rgb}{1.000000,0.000000,0.000000}%
\pgfsetstrokecolor{currentstroke}%
\pgfsetdash{}{0pt}%
\pgfpathmoveto{\pgfqpoint{2.128999in}{2.010586in}}%
\pgfpathlineto{\pgfqpoint{2.137821in}{2.147340in}}%
\pgfusepath{stroke}%
\end{pgfscope}%
\begin{pgfscope}%
\pgfpathrectangle{\pgfqpoint{0.100000in}{0.212622in}}{\pgfqpoint{3.696000in}{3.696000in}}%
\pgfusepath{clip}%
\pgfsetrectcap%
\pgfsetroundjoin%
\pgfsetlinewidth{1.505625pt}%
\definecolor{currentstroke}{rgb}{1.000000,0.000000,0.000000}%
\pgfsetstrokecolor{currentstroke}%
\pgfsetdash{}{0pt}%
\pgfpathmoveto{\pgfqpoint{2.130440in}{2.012507in}}%
\pgfpathlineto{\pgfqpoint{2.137821in}{2.147340in}}%
\pgfusepath{stroke}%
\end{pgfscope}%
\begin{pgfscope}%
\pgfpathrectangle{\pgfqpoint{0.100000in}{0.212622in}}{\pgfqpoint{3.696000in}{3.696000in}}%
\pgfusepath{clip}%
\pgfsetrectcap%
\pgfsetroundjoin%
\pgfsetlinewidth{1.505625pt}%
\definecolor{currentstroke}{rgb}{1.000000,0.000000,0.000000}%
\pgfsetstrokecolor{currentstroke}%
\pgfsetdash{}{0pt}%
\pgfpathmoveto{\pgfqpoint{2.132081in}{2.017328in}}%
\pgfpathlineto{\pgfqpoint{2.142385in}{2.151322in}}%
\pgfusepath{stroke}%
\end{pgfscope}%
\begin{pgfscope}%
\pgfpathrectangle{\pgfqpoint{0.100000in}{0.212622in}}{\pgfqpoint{3.696000in}{3.696000in}}%
\pgfusepath{clip}%
\pgfsetrectcap%
\pgfsetroundjoin%
\pgfsetlinewidth{1.505625pt}%
\definecolor{currentstroke}{rgb}{1.000000,0.000000,0.000000}%
\pgfsetstrokecolor{currentstroke}%
\pgfsetdash{}{0pt}%
\pgfpathmoveto{\pgfqpoint{2.133115in}{2.019841in}}%
\pgfpathlineto{\pgfqpoint{2.142385in}{2.151322in}}%
\pgfusepath{stroke}%
\end{pgfscope}%
\begin{pgfscope}%
\pgfpathrectangle{\pgfqpoint{0.100000in}{0.212622in}}{\pgfqpoint{3.696000in}{3.696000in}}%
\pgfusepath{clip}%
\pgfsetrectcap%
\pgfsetroundjoin%
\pgfsetlinewidth{1.505625pt}%
\definecolor{currentstroke}{rgb}{1.000000,0.000000,0.000000}%
\pgfsetstrokecolor{currentstroke}%
\pgfsetdash{}{0pt}%
\pgfpathmoveto{\pgfqpoint{2.133988in}{2.021115in}}%
\pgfpathlineto{\pgfqpoint{2.142385in}{2.151322in}}%
\pgfusepath{stroke}%
\end{pgfscope}%
\begin{pgfscope}%
\pgfpathrectangle{\pgfqpoint{0.100000in}{0.212622in}}{\pgfqpoint{3.696000in}{3.696000in}}%
\pgfusepath{clip}%
\pgfsetrectcap%
\pgfsetroundjoin%
\pgfsetlinewidth{1.505625pt}%
\definecolor{currentstroke}{rgb}{1.000000,0.000000,0.000000}%
\pgfsetstrokecolor{currentstroke}%
\pgfsetdash{}{0pt}%
\pgfpathmoveto{\pgfqpoint{2.135304in}{2.024622in}}%
\pgfpathlineto{\pgfqpoint{2.142385in}{2.151322in}}%
\pgfusepath{stroke}%
\end{pgfscope}%
\begin{pgfscope}%
\pgfpathrectangle{\pgfqpoint{0.100000in}{0.212622in}}{\pgfqpoint{3.696000in}{3.696000in}}%
\pgfusepath{clip}%
\pgfsetrectcap%
\pgfsetroundjoin%
\pgfsetlinewidth{1.505625pt}%
\definecolor{currentstroke}{rgb}{1.000000,0.000000,0.000000}%
\pgfsetstrokecolor{currentstroke}%
\pgfsetdash{}{0pt}%
\pgfpathmoveto{\pgfqpoint{2.135741in}{2.026691in}}%
\pgfpathlineto{\pgfqpoint{2.146946in}{2.155301in}}%
\pgfusepath{stroke}%
\end{pgfscope}%
\begin{pgfscope}%
\pgfpathrectangle{\pgfqpoint{0.100000in}{0.212622in}}{\pgfqpoint{3.696000in}{3.696000in}}%
\pgfusepath{clip}%
\pgfsetrectcap%
\pgfsetroundjoin%
\pgfsetlinewidth{1.505625pt}%
\definecolor{currentstroke}{rgb}{1.000000,0.000000,0.000000}%
\pgfsetstrokecolor{currentstroke}%
\pgfsetdash{}{0pt}%
\pgfpathmoveto{\pgfqpoint{2.138249in}{2.030659in}}%
\pgfpathlineto{\pgfqpoint{2.146946in}{2.155301in}}%
\pgfusepath{stroke}%
\end{pgfscope}%
\begin{pgfscope}%
\pgfpathrectangle{\pgfqpoint{0.100000in}{0.212622in}}{\pgfqpoint{3.696000in}{3.696000in}}%
\pgfusepath{clip}%
\pgfsetrectcap%
\pgfsetroundjoin%
\pgfsetlinewidth{1.505625pt}%
\definecolor{currentstroke}{rgb}{1.000000,0.000000,0.000000}%
\pgfsetstrokecolor{currentstroke}%
\pgfsetdash{}{0pt}%
\pgfpathmoveto{\pgfqpoint{2.139590in}{2.032742in}}%
\pgfpathlineto{\pgfqpoint{2.146946in}{2.155301in}}%
\pgfusepath{stroke}%
\end{pgfscope}%
\begin{pgfscope}%
\pgfpathrectangle{\pgfqpoint{0.100000in}{0.212622in}}{\pgfqpoint{3.696000in}{3.696000in}}%
\pgfusepath{clip}%
\pgfsetrectcap%
\pgfsetroundjoin%
\pgfsetlinewidth{1.505625pt}%
\definecolor{currentstroke}{rgb}{1.000000,0.000000,0.000000}%
\pgfsetstrokecolor{currentstroke}%
\pgfsetdash{}{0pt}%
\pgfpathmoveto{\pgfqpoint{2.140329in}{2.036245in}}%
\pgfpathlineto{\pgfqpoint{2.151503in}{2.159277in}}%
\pgfusepath{stroke}%
\end{pgfscope}%
\begin{pgfscope}%
\pgfpathrectangle{\pgfqpoint{0.100000in}{0.212622in}}{\pgfqpoint{3.696000in}{3.696000in}}%
\pgfusepath{clip}%
\pgfsetrectcap%
\pgfsetroundjoin%
\pgfsetlinewidth{1.505625pt}%
\definecolor{currentstroke}{rgb}{1.000000,0.000000,0.000000}%
\pgfsetstrokecolor{currentstroke}%
\pgfsetdash{}{0pt}%
\pgfpathmoveto{\pgfqpoint{2.142929in}{2.040918in}}%
\pgfpathlineto{\pgfqpoint{2.151503in}{2.159277in}}%
\pgfusepath{stroke}%
\end{pgfscope}%
\begin{pgfscope}%
\pgfpathrectangle{\pgfqpoint{0.100000in}{0.212622in}}{\pgfqpoint{3.696000in}{3.696000in}}%
\pgfusepath{clip}%
\pgfsetrectcap%
\pgfsetroundjoin%
\pgfsetlinewidth{1.505625pt}%
\definecolor{currentstroke}{rgb}{1.000000,0.000000,0.000000}%
\pgfsetstrokecolor{currentstroke}%
\pgfsetdash{}{0pt}%
\pgfpathmoveto{\pgfqpoint{2.144408in}{2.043067in}}%
\pgfpathlineto{\pgfqpoint{2.151503in}{2.159277in}}%
\pgfusepath{stroke}%
\end{pgfscope}%
\begin{pgfscope}%
\pgfpathrectangle{\pgfqpoint{0.100000in}{0.212622in}}{\pgfqpoint{3.696000in}{3.696000in}}%
\pgfusepath{clip}%
\pgfsetrectcap%
\pgfsetroundjoin%
\pgfsetlinewidth{1.505625pt}%
\definecolor{currentstroke}{rgb}{1.000000,0.000000,0.000000}%
\pgfsetstrokecolor{currentstroke}%
\pgfsetdash{}{0pt}%
\pgfpathmoveto{\pgfqpoint{2.145567in}{2.048338in}}%
\pgfpathlineto{\pgfqpoint{2.156058in}{2.163251in}}%
\pgfusepath{stroke}%
\end{pgfscope}%
\begin{pgfscope}%
\pgfpathrectangle{\pgfqpoint{0.100000in}{0.212622in}}{\pgfqpoint{3.696000in}{3.696000in}}%
\pgfusepath{clip}%
\pgfsetrectcap%
\pgfsetroundjoin%
\pgfsetlinewidth{1.505625pt}%
\definecolor{currentstroke}{rgb}{1.000000,0.000000,0.000000}%
\pgfsetstrokecolor{currentstroke}%
\pgfsetdash{}{0pt}%
\pgfpathmoveto{\pgfqpoint{2.147986in}{2.053030in}}%
\pgfpathlineto{\pgfqpoint{2.160609in}{2.167222in}}%
\pgfusepath{stroke}%
\end{pgfscope}%
\begin{pgfscope}%
\pgfpathrectangle{\pgfqpoint{0.100000in}{0.212622in}}{\pgfqpoint{3.696000in}{3.696000in}}%
\pgfusepath{clip}%
\pgfsetrectcap%
\pgfsetroundjoin%
\pgfsetlinewidth{1.505625pt}%
\definecolor{currentstroke}{rgb}{1.000000,0.000000,0.000000}%
\pgfsetstrokecolor{currentstroke}%
\pgfsetdash{}{0pt}%
\pgfpathmoveto{\pgfqpoint{2.149745in}{2.055664in}}%
\pgfpathlineto{\pgfqpoint{2.160609in}{2.167222in}}%
\pgfusepath{stroke}%
\end{pgfscope}%
\begin{pgfscope}%
\pgfpathrectangle{\pgfqpoint{0.100000in}{0.212622in}}{\pgfqpoint{3.696000in}{3.696000in}}%
\pgfusepath{clip}%
\pgfsetrectcap%
\pgfsetroundjoin%
\pgfsetlinewidth{1.505625pt}%
\definecolor{currentstroke}{rgb}{1.000000,0.000000,0.000000}%
\pgfsetstrokecolor{currentstroke}%
\pgfsetdash{}{0pt}%
\pgfpathmoveto{\pgfqpoint{2.151542in}{2.059666in}}%
\pgfpathlineto{\pgfqpoint{2.160609in}{2.167222in}}%
\pgfusepath{stroke}%
\end{pgfscope}%
\begin{pgfscope}%
\pgfpathrectangle{\pgfqpoint{0.100000in}{0.212622in}}{\pgfqpoint{3.696000in}{3.696000in}}%
\pgfusepath{clip}%
\pgfsetrectcap%
\pgfsetroundjoin%
\pgfsetlinewidth{1.505625pt}%
\definecolor{currentstroke}{rgb}{1.000000,0.000000,0.000000}%
\pgfsetstrokecolor{currentstroke}%
\pgfsetdash{}{0pt}%
\pgfpathmoveto{\pgfqpoint{2.152074in}{2.065078in}}%
\pgfpathlineto{\pgfqpoint{2.165158in}{2.171189in}}%
\pgfusepath{stroke}%
\end{pgfscope}%
\begin{pgfscope}%
\pgfpathrectangle{\pgfqpoint{0.100000in}{0.212622in}}{\pgfqpoint{3.696000in}{3.696000in}}%
\pgfusepath{clip}%
\pgfsetrectcap%
\pgfsetroundjoin%
\pgfsetlinewidth{1.505625pt}%
\definecolor{currentstroke}{rgb}{1.000000,0.000000,0.000000}%
\pgfsetstrokecolor{currentstroke}%
\pgfsetdash{}{0pt}%
\pgfpathmoveto{\pgfqpoint{2.155304in}{2.071304in}}%
\pgfpathlineto{\pgfqpoint{2.169703in}{2.175155in}}%
\pgfusepath{stroke}%
\end{pgfscope}%
\begin{pgfscope}%
\pgfpathrectangle{\pgfqpoint{0.100000in}{0.212622in}}{\pgfqpoint{3.696000in}{3.696000in}}%
\pgfusepath{clip}%
\pgfsetrectcap%
\pgfsetroundjoin%
\pgfsetlinewidth{1.505625pt}%
\definecolor{currentstroke}{rgb}{1.000000,0.000000,0.000000}%
\pgfsetstrokecolor{currentstroke}%
\pgfsetdash{}{0pt}%
\pgfpathmoveto{\pgfqpoint{2.157477in}{2.074547in}}%
\pgfpathlineto{\pgfqpoint{2.169703in}{2.175155in}}%
\pgfusepath{stroke}%
\end{pgfscope}%
\begin{pgfscope}%
\pgfpathrectangle{\pgfqpoint{0.100000in}{0.212622in}}{\pgfqpoint{3.696000in}{3.696000in}}%
\pgfusepath{clip}%
\pgfsetrectcap%
\pgfsetroundjoin%
\pgfsetlinewidth{1.505625pt}%
\definecolor{currentstroke}{rgb}{1.000000,0.000000,0.000000}%
\pgfsetstrokecolor{currentstroke}%
\pgfsetdash{}{0pt}%
\pgfpathmoveto{\pgfqpoint{2.159550in}{2.079981in}}%
\pgfpathlineto{\pgfqpoint{2.174245in}{2.179117in}}%
\pgfusepath{stroke}%
\end{pgfscope}%
\begin{pgfscope}%
\pgfpathrectangle{\pgfqpoint{0.100000in}{0.212622in}}{\pgfqpoint{3.696000in}{3.696000in}}%
\pgfusepath{clip}%
\pgfsetrectcap%
\pgfsetroundjoin%
\pgfsetlinewidth{1.505625pt}%
\definecolor{currentstroke}{rgb}{1.000000,0.000000,0.000000}%
\pgfsetstrokecolor{currentstroke}%
\pgfsetdash{}{0pt}%
\pgfpathmoveto{\pgfqpoint{2.160220in}{2.083300in}}%
\pgfpathlineto{\pgfqpoint{2.174245in}{2.179117in}}%
\pgfusepath{stroke}%
\end{pgfscope}%
\begin{pgfscope}%
\pgfpathrectangle{\pgfqpoint{0.100000in}{0.212622in}}{\pgfqpoint{3.696000in}{3.696000in}}%
\pgfusepath{clip}%
\pgfsetrectcap%
\pgfsetroundjoin%
\pgfsetlinewidth{1.505625pt}%
\definecolor{currentstroke}{rgb}{1.000000,0.000000,0.000000}%
\pgfsetstrokecolor{currentstroke}%
\pgfsetdash{}{0pt}%
\pgfpathmoveto{\pgfqpoint{2.163469in}{2.088376in}}%
\pgfpathlineto{\pgfqpoint{2.174245in}{2.179117in}}%
\pgfusepath{stroke}%
\end{pgfscope}%
\begin{pgfscope}%
\pgfpathrectangle{\pgfqpoint{0.100000in}{0.212622in}}{\pgfqpoint{3.696000in}{3.696000in}}%
\pgfusepath{clip}%
\pgfsetrectcap%
\pgfsetroundjoin%
\pgfsetlinewidth{1.505625pt}%
\definecolor{currentstroke}{rgb}{1.000000,0.000000,0.000000}%
\pgfsetstrokecolor{currentstroke}%
\pgfsetdash{}{0pt}%
\pgfpathmoveto{\pgfqpoint{2.165293in}{2.090821in}}%
\pgfpathlineto{\pgfqpoint{2.178783in}{2.183077in}}%
\pgfusepath{stroke}%
\end{pgfscope}%
\begin{pgfscope}%
\pgfpathrectangle{\pgfqpoint{0.100000in}{0.212622in}}{\pgfqpoint{3.696000in}{3.696000in}}%
\pgfusepath{clip}%
\pgfsetrectcap%
\pgfsetroundjoin%
\pgfsetlinewidth{1.505625pt}%
\definecolor{currentstroke}{rgb}{1.000000,0.000000,0.000000}%
\pgfsetstrokecolor{currentstroke}%
\pgfsetdash{}{0pt}%
\pgfpathmoveto{\pgfqpoint{2.167317in}{2.096424in}}%
\pgfpathlineto{\pgfqpoint{2.178783in}{2.183077in}}%
\pgfusepath{stroke}%
\end{pgfscope}%
\begin{pgfscope}%
\pgfpathrectangle{\pgfqpoint{0.100000in}{0.212622in}}{\pgfqpoint{3.696000in}{3.696000in}}%
\pgfusepath{clip}%
\pgfsetrectcap%
\pgfsetroundjoin%
\pgfsetlinewidth{1.505625pt}%
\definecolor{currentstroke}{rgb}{1.000000,0.000000,0.000000}%
\pgfsetstrokecolor{currentstroke}%
\pgfsetdash{}{0pt}%
\pgfpathmoveto{\pgfqpoint{2.168466in}{2.099389in}}%
\pgfpathlineto{\pgfqpoint{2.183319in}{2.187034in}}%
\pgfusepath{stroke}%
\end{pgfscope}%
\begin{pgfscope}%
\pgfpathrectangle{\pgfqpoint{0.100000in}{0.212622in}}{\pgfqpoint{3.696000in}{3.696000in}}%
\pgfusepath{clip}%
\pgfsetrectcap%
\pgfsetroundjoin%
\pgfsetlinewidth{1.505625pt}%
\definecolor{currentstroke}{rgb}{1.000000,0.000000,0.000000}%
\pgfsetstrokecolor{currentstroke}%
\pgfsetdash{}{0pt}%
\pgfpathmoveto{\pgfqpoint{2.170505in}{2.102225in}}%
\pgfpathlineto{\pgfqpoint{2.183319in}{2.187034in}}%
\pgfusepath{stroke}%
\end{pgfscope}%
\begin{pgfscope}%
\pgfpathrectangle{\pgfqpoint{0.100000in}{0.212622in}}{\pgfqpoint{3.696000in}{3.696000in}}%
\pgfusepath{clip}%
\pgfsetrectcap%
\pgfsetroundjoin%
\pgfsetlinewidth{1.505625pt}%
\definecolor{currentstroke}{rgb}{1.000000,0.000000,0.000000}%
\pgfsetstrokecolor{currentstroke}%
\pgfsetdash{}{0pt}%
\pgfpathmoveto{\pgfqpoint{2.172783in}{2.105742in}}%
\pgfpathlineto{\pgfqpoint{2.183319in}{2.187034in}}%
\pgfusepath{stroke}%
\end{pgfscope}%
\begin{pgfscope}%
\pgfpathrectangle{\pgfqpoint{0.100000in}{0.212622in}}{\pgfqpoint{3.696000in}{3.696000in}}%
\pgfusepath{clip}%
\pgfsetrectcap%
\pgfsetroundjoin%
\pgfsetlinewidth{1.505625pt}%
\definecolor{currentstroke}{rgb}{1.000000,0.000000,0.000000}%
\pgfsetstrokecolor{currentstroke}%
\pgfsetdash{}{0pt}%
\pgfpathmoveto{\pgfqpoint{2.175029in}{2.109788in}}%
\pgfpathlineto{\pgfqpoint{2.187851in}{2.190988in}}%
\pgfusepath{stroke}%
\end{pgfscope}%
\begin{pgfscope}%
\pgfpathrectangle{\pgfqpoint{0.100000in}{0.212622in}}{\pgfqpoint{3.696000in}{3.696000in}}%
\pgfusepath{clip}%
\pgfsetrectcap%
\pgfsetroundjoin%
\pgfsetlinewidth{1.505625pt}%
\definecolor{currentstroke}{rgb}{1.000000,0.000000,0.000000}%
\pgfsetstrokecolor{currentstroke}%
\pgfsetdash{}{0pt}%
\pgfpathmoveto{\pgfqpoint{2.176070in}{2.112285in}}%
\pgfpathlineto{\pgfqpoint{2.187851in}{2.190988in}}%
\pgfusepath{stroke}%
\end{pgfscope}%
\begin{pgfscope}%
\pgfpathrectangle{\pgfqpoint{0.100000in}{0.212622in}}{\pgfqpoint{3.696000in}{3.696000in}}%
\pgfusepath{clip}%
\pgfsetrectcap%
\pgfsetroundjoin%
\pgfsetlinewidth{1.505625pt}%
\definecolor{currentstroke}{rgb}{1.000000,0.000000,0.000000}%
\pgfsetstrokecolor{currentstroke}%
\pgfsetdash{}{0pt}%
\pgfpathmoveto{\pgfqpoint{2.177842in}{2.115308in}}%
\pgfpathlineto{\pgfqpoint{2.192381in}{2.194940in}}%
\pgfusepath{stroke}%
\end{pgfscope}%
\begin{pgfscope}%
\pgfpathrectangle{\pgfqpoint{0.100000in}{0.212622in}}{\pgfqpoint{3.696000in}{3.696000in}}%
\pgfusepath{clip}%
\pgfsetrectcap%
\pgfsetroundjoin%
\pgfsetlinewidth{1.505625pt}%
\definecolor{currentstroke}{rgb}{1.000000,0.000000,0.000000}%
\pgfsetstrokecolor{currentstroke}%
\pgfsetdash{}{0pt}%
\pgfpathmoveto{\pgfqpoint{2.178988in}{2.117068in}}%
\pgfpathlineto{\pgfqpoint{2.192381in}{2.194940in}}%
\pgfusepath{stroke}%
\end{pgfscope}%
\begin{pgfscope}%
\pgfpathrectangle{\pgfqpoint{0.100000in}{0.212622in}}{\pgfqpoint{3.696000in}{3.696000in}}%
\pgfusepath{clip}%
\pgfsetrectcap%
\pgfsetroundjoin%
\pgfsetlinewidth{1.505625pt}%
\definecolor{currentstroke}{rgb}{1.000000,0.000000,0.000000}%
\pgfsetstrokecolor{currentstroke}%
\pgfsetdash{}{0pt}%
\pgfpathmoveto{\pgfqpoint{2.179638in}{2.117925in}}%
\pgfpathlineto{\pgfqpoint{2.192381in}{2.194940in}}%
\pgfusepath{stroke}%
\end{pgfscope}%
\begin{pgfscope}%
\pgfpathrectangle{\pgfqpoint{0.100000in}{0.212622in}}{\pgfqpoint{3.696000in}{3.696000in}}%
\pgfusepath{clip}%
\pgfsetrectcap%
\pgfsetroundjoin%
\pgfsetlinewidth{1.505625pt}%
\definecolor{currentstroke}{rgb}{1.000000,0.000000,0.000000}%
\pgfsetstrokecolor{currentstroke}%
\pgfsetdash{}{0pt}%
\pgfpathmoveto{\pgfqpoint{2.180426in}{2.119617in}}%
\pgfpathlineto{\pgfqpoint{2.192381in}{2.194940in}}%
\pgfusepath{stroke}%
\end{pgfscope}%
\begin{pgfscope}%
\pgfpathrectangle{\pgfqpoint{0.100000in}{0.212622in}}{\pgfqpoint{3.696000in}{3.696000in}}%
\pgfusepath{clip}%
\pgfsetrectcap%
\pgfsetroundjoin%
\pgfsetlinewidth{1.505625pt}%
\definecolor{currentstroke}{rgb}{1.000000,0.000000,0.000000}%
\pgfsetstrokecolor{currentstroke}%
\pgfsetdash{}{0pt}%
\pgfpathmoveto{\pgfqpoint{2.180845in}{2.120562in}}%
\pgfpathlineto{\pgfqpoint{2.192381in}{2.194940in}}%
\pgfusepath{stroke}%
\end{pgfscope}%
\begin{pgfscope}%
\pgfpathrectangle{\pgfqpoint{0.100000in}{0.212622in}}{\pgfqpoint{3.696000in}{3.696000in}}%
\pgfusepath{clip}%
\pgfsetrectcap%
\pgfsetroundjoin%
\pgfsetlinewidth{1.505625pt}%
\definecolor{currentstroke}{rgb}{1.000000,0.000000,0.000000}%
\pgfsetstrokecolor{currentstroke}%
\pgfsetdash{}{0pt}%
\pgfpathmoveto{\pgfqpoint{2.181845in}{2.122018in}}%
\pgfpathlineto{\pgfqpoint{2.192381in}{2.194940in}}%
\pgfusepath{stroke}%
\end{pgfscope}%
\begin{pgfscope}%
\pgfpathrectangle{\pgfqpoint{0.100000in}{0.212622in}}{\pgfqpoint{3.696000in}{3.696000in}}%
\pgfusepath{clip}%
\pgfsetrectcap%
\pgfsetroundjoin%
\pgfsetlinewidth{1.505625pt}%
\definecolor{currentstroke}{rgb}{1.000000,0.000000,0.000000}%
\pgfsetstrokecolor{currentstroke}%
\pgfsetdash{}{0pt}%
\pgfpathmoveto{\pgfqpoint{2.182410in}{2.122781in}}%
\pgfpathlineto{\pgfqpoint{2.192381in}{2.194940in}}%
\pgfusepath{stroke}%
\end{pgfscope}%
\begin{pgfscope}%
\pgfpathrectangle{\pgfqpoint{0.100000in}{0.212622in}}{\pgfqpoint{3.696000in}{3.696000in}}%
\pgfusepath{clip}%
\pgfsetrectcap%
\pgfsetroundjoin%
\pgfsetlinewidth{1.505625pt}%
\definecolor{currentstroke}{rgb}{1.000000,0.000000,0.000000}%
\pgfsetstrokecolor{currentstroke}%
\pgfsetdash{}{0pt}%
\pgfpathmoveto{\pgfqpoint{2.183395in}{2.124829in}}%
\pgfpathlineto{\pgfqpoint{2.196907in}{2.198889in}}%
\pgfusepath{stroke}%
\end{pgfscope}%
\begin{pgfscope}%
\pgfpathrectangle{\pgfqpoint{0.100000in}{0.212622in}}{\pgfqpoint{3.696000in}{3.696000in}}%
\pgfusepath{clip}%
\pgfsetrectcap%
\pgfsetroundjoin%
\pgfsetlinewidth{1.505625pt}%
\definecolor{currentstroke}{rgb}{1.000000,0.000000,0.000000}%
\pgfsetstrokecolor{currentstroke}%
\pgfsetdash{}{0pt}%
\pgfpathmoveto{\pgfqpoint{2.183960in}{2.125966in}}%
\pgfpathlineto{\pgfqpoint{2.196907in}{2.198889in}}%
\pgfusepath{stroke}%
\end{pgfscope}%
\begin{pgfscope}%
\pgfpathrectangle{\pgfqpoint{0.100000in}{0.212622in}}{\pgfqpoint{3.696000in}{3.696000in}}%
\pgfusepath{clip}%
\pgfsetrectcap%
\pgfsetroundjoin%
\pgfsetlinewidth{1.505625pt}%
\definecolor{currentstroke}{rgb}{1.000000,0.000000,0.000000}%
\pgfsetstrokecolor{currentstroke}%
\pgfsetdash{}{0pt}%
\pgfpathmoveto{\pgfqpoint{2.184368in}{2.126538in}}%
\pgfpathlineto{\pgfqpoint{2.196907in}{2.198889in}}%
\pgfusepath{stroke}%
\end{pgfscope}%
\begin{pgfscope}%
\pgfpathrectangle{\pgfqpoint{0.100000in}{0.212622in}}{\pgfqpoint{3.696000in}{3.696000in}}%
\pgfusepath{clip}%
\pgfsetrectcap%
\pgfsetroundjoin%
\pgfsetlinewidth{1.505625pt}%
\definecolor{currentstroke}{rgb}{1.000000,0.000000,0.000000}%
\pgfsetstrokecolor{currentstroke}%
\pgfsetdash{}{0pt}%
\pgfpathmoveto{\pgfqpoint{2.184570in}{2.126823in}}%
\pgfpathlineto{\pgfqpoint{2.196907in}{2.198889in}}%
\pgfusepath{stroke}%
\end{pgfscope}%
\begin{pgfscope}%
\pgfpathrectangle{\pgfqpoint{0.100000in}{0.212622in}}{\pgfqpoint{3.696000in}{3.696000in}}%
\pgfusepath{clip}%
\pgfsetrectcap%
\pgfsetroundjoin%
\pgfsetlinewidth{1.505625pt}%
\definecolor{currentstroke}{rgb}{1.000000,0.000000,0.000000}%
\pgfsetstrokecolor{currentstroke}%
\pgfsetdash{}{0pt}%
\pgfpathmoveto{\pgfqpoint{2.184961in}{2.127884in}}%
\pgfpathlineto{\pgfqpoint{2.196907in}{2.198889in}}%
\pgfusepath{stroke}%
\end{pgfscope}%
\begin{pgfscope}%
\pgfpathrectangle{\pgfqpoint{0.100000in}{0.212622in}}{\pgfqpoint{3.696000in}{3.696000in}}%
\pgfusepath{clip}%
\pgfsetrectcap%
\pgfsetroundjoin%
\pgfsetlinewidth{1.505625pt}%
\definecolor{currentstroke}{rgb}{1.000000,0.000000,0.000000}%
\pgfsetstrokecolor{currentstroke}%
\pgfsetdash{}{0pt}%
\pgfpathmoveto{\pgfqpoint{2.185712in}{2.129274in}}%
\pgfpathlineto{\pgfqpoint{2.196907in}{2.198889in}}%
\pgfusepath{stroke}%
\end{pgfscope}%
\begin{pgfscope}%
\pgfpathrectangle{\pgfqpoint{0.100000in}{0.212622in}}{\pgfqpoint{3.696000in}{3.696000in}}%
\pgfusepath{clip}%
\pgfsetrectcap%
\pgfsetroundjoin%
\pgfsetlinewidth{1.505625pt}%
\definecolor{currentstroke}{rgb}{1.000000,0.000000,0.000000}%
\pgfsetstrokecolor{currentstroke}%
\pgfsetdash{}{0pt}%
\pgfpathmoveto{\pgfqpoint{2.186204in}{2.129978in}}%
\pgfpathlineto{\pgfqpoint{2.196907in}{2.198889in}}%
\pgfusepath{stroke}%
\end{pgfscope}%
\begin{pgfscope}%
\pgfpathrectangle{\pgfqpoint{0.100000in}{0.212622in}}{\pgfqpoint{3.696000in}{3.696000in}}%
\pgfusepath{clip}%
\pgfsetrectcap%
\pgfsetroundjoin%
\pgfsetlinewidth{1.505625pt}%
\definecolor{currentstroke}{rgb}{1.000000,0.000000,0.000000}%
\pgfsetstrokecolor{currentstroke}%
\pgfsetdash{}{0pt}%
\pgfpathmoveto{\pgfqpoint{2.186474in}{2.130355in}}%
\pgfpathlineto{\pgfqpoint{2.196907in}{2.198889in}}%
\pgfusepath{stroke}%
\end{pgfscope}%
\begin{pgfscope}%
\pgfpathrectangle{\pgfqpoint{0.100000in}{0.212622in}}{\pgfqpoint{3.696000in}{3.696000in}}%
\pgfusepath{clip}%
\pgfsetrectcap%
\pgfsetroundjoin%
\pgfsetlinewidth{1.505625pt}%
\definecolor{currentstroke}{rgb}{1.000000,0.000000,0.000000}%
\pgfsetstrokecolor{currentstroke}%
\pgfsetdash{}{0pt}%
\pgfpathmoveto{\pgfqpoint{2.186894in}{2.131197in}}%
\pgfpathlineto{\pgfqpoint{2.196907in}{2.198889in}}%
\pgfusepath{stroke}%
\end{pgfscope}%
\begin{pgfscope}%
\pgfpathrectangle{\pgfqpoint{0.100000in}{0.212622in}}{\pgfqpoint{3.696000in}{3.696000in}}%
\pgfusepath{clip}%
\pgfsetrectcap%
\pgfsetroundjoin%
\pgfsetlinewidth{1.505625pt}%
\definecolor{currentstroke}{rgb}{1.000000,0.000000,0.000000}%
\pgfsetstrokecolor{currentstroke}%
\pgfsetdash{}{0pt}%
\pgfpathmoveto{\pgfqpoint{2.187467in}{2.132546in}}%
\pgfpathlineto{\pgfqpoint{2.201430in}{2.202835in}}%
\pgfusepath{stroke}%
\end{pgfscope}%
\begin{pgfscope}%
\pgfpathrectangle{\pgfqpoint{0.100000in}{0.212622in}}{\pgfqpoint{3.696000in}{3.696000in}}%
\pgfusepath{clip}%
\pgfsetrectcap%
\pgfsetroundjoin%
\pgfsetlinewidth{1.505625pt}%
\definecolor{currentstroke}{rgb}{1.000000,0.000000,0.000000}%
\pgfsetstrokecolor{currentstroke}%
\pgfsetdash{}{0pt}%
\pgfpathmoveto{\pgfqpoint{2.188701in}{2.134278in}}%
\pgfpathlineto{\pgfqpoint{2.201430in}{2.202835in}}%
\pgfusepath{stroke}%
\end{pgfscope}%
\begin{pgfscope}%
\pgfpathrectangle{\pgfqpoint{0.100000in}{0.212622in}}{\pgfqpoint{3.696000in}{3.696000in}}%
\pgfusepath{clip}%
\pgfsetrectcap%
\pgfsetroundjoin%
\pgfsetlinewidth{1.505625pt}%
\definecolor{currentstroke}{rgb}{1.000000,0.000000,0.000000}%
\pgfsetstrokecolor{currentstroke}%
\pgfsetdash{}{0pt}%
\pgfpathmoveto{\pgfqpoint{2.189376in}{2.135234in}}%
\pgfpathlineto{\pgfqpoint{2.201430in}{2.202835in}}%
\pgfusepath{stroke}%
\end{pgfscope}%
\begin{pgfscope}%
\pgfpathrectangle{\pgfqpoint{0.100000in}{0.212622in}}{\pgfqpoint{3.696000in}{3.696000in}}%
\pgfusepath{clip}%
\pgfsetrectcap%
\pgfsetroundjoin%
\pgfsetlinewidth{1.505625pt}%
\definecolor{currentstroke}{rgb}{1.000000,0.000000,0.000000}%
\pgfsetstrokecolor{currentstroke}%
\pgfsetdash{}{0pt}%
\pgfpathmoveto{\pgfqpoint{2.190553in}{2.137271in}}%
\pgfpathlineto{\pgfqpoint{2.201430in}{2.202835in}}%
\pgfusepath{stroke}%
\end{pgfscope}%
\begin{pgfscope}%
\pgfpathrectangle{\pgfqpoint{0.100000in}{0.212622in}}{\pgfqpoint{3.696000in}{3.696000in}}%
\pgfusepath{clip}%
\pgfsetrectcap%
\pgfsetroundjoin%
\pgfsetlinewidth{1.505625pt}%
\definecolor{currentstroke}{rgb}{1.000000,0.000000,0.000000}%
\pgfsetstrokecolor{currentstroke}%
\pgfsetdash{}{0pt}%
\pgfpathmoveto{\pgfqpoint{2.191681in}{2.140010in}}%
\pgfpathlineto{\pgfqpoint{2.201430in}{2.202835in}}%
\pgfusepath{stroke}%
\end{pgfscope}%
\begin{pgfscope}%
\pgfpathrectangle{\pgfqpoint{0.100000in}{0.212622in}}{\pgfqpoint{3.696000in}{3.696000in}}%
\pgfusepath{clip}%
\pgfsetrectcap%
\pgfsetroundjoin%
\pgfsetlinewidth{1.505625pt}%
\definecolor{currentstroke}{rgb}{1.000000,0.000000,0.000000}%
\pgfsetstrokecolor{currentstroke}%
\pgfsetdash{}{0pt}%
\pgfpathmoveto{\pgfqpoint{2.193353in}{2.142715in}}%
\pgfpathlineto{\pgfqpoint{2.205950in}{2.206778in}}%
\pgfusepath{stroke}%
\end{pgfscope}%
\begin{pgfscope}%
\pgfpathrectangle{\pgfqpoint{0.100000in}{0.212622in}}{\pgfqpoint{3.696000in}{3.696000in}}%
\pgfusepath{clip}%
\pgfsetrectcap%
\pgfsetroundjoin%
\pgfsetlinewidth{1.505625pt}%
\definecolor{currentstroke}{rgb}{1.000000,0.000000,0.000000}%
\pgfsetstrokecolor{currentstroke}%
\pgfsetdash{}{0pt}%
\pgfpathmoveto{\pgfqpoint{2.194419in}{2.144173in}}%
\pgfpathlineto{\pgfqpoint{2.205950in}{2.206778in}}%
\pgfusepath{stroke}%
\end{pgfscope}%
\begin{pgfscope}%
\pgfpathrectangle{\pgfqpoint{0.100000in}{0.212622in}}{\pgfqpoint{3.696000in}{3.696000in}}%
\pgfusepath{clip}%
\pgfsetrectcap%
\pgfsetroundjoin%
\pgfsetlinewidth{1.505625pt}%
\definecolor{currentstroke}{rgb}{1.000000,0.000000,0.000000}%
\pgfsetstrokecolor{currentstroke}%
\pgfsetdash{}{0pt}%
\pgfpathmoveto{\pgfqpoint{2.195020in}{2.144935in}}%
\pgfpathlineto{\pgfqpoint{2.205950in}{2.206778in}}%
\pgfusepath{stroke}%
\end{pgfscope}%
\begin{pgfscope}%
\pgfpathrectangle{\pgfqpoint{0.100000in}{0.212622in}}{\pgfqpoint{3.696000in}{3.696000in}}%
\pgfusepath{clip}%
\pgfsetrectcap%
\pgfsetroundjoin%
\pgfsetlinewidth{1.505625pt}%
\definecolor{currentstroke}{rgb}{1.000000,0.000000,0.000000}%
\pgfsetstrokecolor{currentstroke}%
\pgfsetdash{}{0pt}%
\pgfpathmoveto{\pgfqpoint{2.195331in}{2.145377in}}%
\pgfpathlineto{\pgfqpoint{2.205950in}{2.206778in}}%
\pgfusepath{stroke}%
\end{pgfscope}%
\begin{pgfscope}%
\pgfpathrectangle{\pgfqpoint{0.100000in}{0.212622in}}{\pgfqpoint{3.696000in}{3.696000in}}%
\pgfusepath{clip}%
\pgfsetrectcap%
\pgfsetroundjoin%
\pgfsetlinewidth{1.505625pt}%
\definecolor{currentstroke}{rgb}{1.000000,0.000000,0.000000}%
\pgfsetstrokecolor{currentstroke}%
\pgfsetdash{}{0pt}%
\pgfpathmoveto{\pgfqpoint{2.195486in}{2.145637in}}%
\pgfpathlineto{\pgfqpoint{2.205950in}{2.206778in}}%
\pgfusepath{stroke}%
\end{pgfscope}%
\begin{pgfscope}%
\pgfpathrectangle{\pgfqpoint{0.100000in}{0.212622in}}{\pgfqpoint{3.696000in}{3.696000in}}%
\pgfusepath{clip}%
\pgfsetrectcap%
\pgfsetroundjoin%
\pgfsetlinewidth{1.505625pt}%
\definecolor{currentstroke}{rgb}{1.000000,0.000000,0.000000}%
\pgfsetstrokecolor{currentstroke}%
\pgfsetdash{}{0pt}%
\pgfpathmoveto{\pgfqpoint{2.195918in}{2.146556in}}%
\pgfpathlineto{\pgfqpoint{2.205950in}{2.206778in}}%
\pgfusepath{stroke}%
\end{pgfscope}%
\begin{pgfscope}%
\pgfpathrectangle{\pgfqpoint{0.100000in}{0.212622in}}{\pgfqpoint{3.696000in}{3.696000in}}%
\pgfusepath{clip}%
\pgfsetrectcap%
\pgfsetroundjoin%
\pgfsetlinewidth{1.505625pt}%
\definecolor{currentstroke}{rgb}{1.000000,0.000000,0.000000}%
\pgfsetstrokecolor{currentstroke}%
\pgfsetdash{}{0pt}%
\pgfpathmoveto{\pgfqpoint{2.196668in}{2.147755in}}%
\pgfpathlineto{\pgfqpoint{2.205950in}{2.206778in}}%
\pgfusepath{stroke}%
\end{pgfscope}%
\begin{pgfscope}%
\pgfpathrectangle{\pgfqpoint{0.100000in}{0.212622in}}{\pgfqpoint{3.696000in}{3.696000in}}%
\pgfusepath{clip}%
\pgfsetrectcap%
\pgfsetroundjoin%
\pgfsetlinewidth{1.505625pt}%
\definecolor{currentstroke}{rgb}{1.000000,0.000000,0.000000}%
\pgfsetstrokecolor{currentstroke}%
\pgfsetdash{}{0pt}%
\pgfpathmoveto{\pgfqpoint{2.197150in}{2.148392in}}%
\pgfpathlineto{\pgfqpoint{2.210467in}{2.210719in}}%
\pgfusepath{stroke}%
\end{pgfscope}%
\begin{pgfscope}%
\pgfpathrectangle{\pgfqpoint{0.100000in}{0.212622in}}{\pgfqpoint{3.696000in}{3.696000in}}%
\pgfusepath{clip}%
\pgfsetrectcap%
\pgfsetroundjoin%
\pgfsetlinewidth{1.505625pt}%
\definecolor{currentstroke}{rgb}{1.000000,0.000000,0.000000}%
\pgfsetstrokecolor{currentstroke}%
\pgfsetdash{}{0pt}%
\pgfpathmoveto{\pgfqpoint{2.197405in}{2.148729in}}%
\pgfpathlineto{\pgfqpoint{2.210467in}{2.210719in}}%
\pgfusepath{stroke}%
\end{pgfscope}%
\begin{pgfscope}%
\pgfpathrectangle{\pgfqpoint{0.100000in}{0.212622in}}{\pgfqpoint{3.696000in}{3.696000in}}%
\pgfusepath{clip}%
\pgfsetrectcap%
\pgfsetroundjoin%
\pgfsetlinewidth{1.505625pt}%
\definecolor{currentstroke}{rgb}{1.000000,0.000000,0.000000}%
\pgfsetstrokecolor{currentstroke}%
\pgfsetdash{}{0pt}%
\pgfpathmoveto{\pgfqpoint{2.197829in}{2.149616in}}%
\pgfpathlineto{\pgfqpoint{2.210467in}{2.210719in}}%
\pgfusepath{stroke}%
\end{pgfscope}%
\begin{pgfscope}%
\pgfpathrectangle{\pgfqpoint{0.100000in}{0.212622in}}{\pgfqpoint{3.696000in}{3.696000in}}%
\pgfusepath{clip}%
\pgfsetrectcap%
\pgfsetroundjoin%
\pgfsetlinewidth{1.505625pt}%
\definecolor{currentstroke}{rgb}{1.000000,0.000000,0.000000}%
\pgfsetstrokecolor{currentstroke}%
\pgfsetdash{}{0pt}%
\pgfpathmoveto{\pgfqpoint{2.198074in}{2.150082in}}%
\pgfpathlineto{\pgfqpoint{2.210467in}{2.210719in}}%
\pgfusepath{stroke}%
\end{pgfscope}%
\begin{pgfscope}%
\pgfpathrectangle{\pgfqpoint{0.100000in}{0.212622in}}{\pgfqpoint{3.696000in}{3.696000in}}%
\pgfusepath{clip}%
\pgfsetrectcap%
\pgfsetroundjoin%
\pgfsetlinewidth{1.505625pt}%
\definecolor{currentstroke}{rgb}{1.000000,0.000000,0.000000}%
\pgfsetstrokecolor{currentstroke}%
\pgfsetdash{}{0pt}%
\pgfpathmoveto{\pgfqpoint{2.198786in}{2.151106in}}%
\pgfpathlineto{\pgfqpoint{2.210467in}{2.210719in}}%
\pgfusepath{stroke}%
\end{pgfscope}%
\begin{pgfscope}%
\pgfpathrectangle{\pgfqpoint{0.100000in}{0.212622in}}{\pgfqpoint{3.696000in}{3.696000in}}%
\pgfusepath{clip}%
\pgfsetrectcap%
\pgfsetroundjoin%
\pgfsetlinewidth{1.505625pt}%
\definecolor{currentstroke}{rgb}{1.000000,0.000000,0.000000}%
\pgfsetstrokecolor{currentstroke}%
\pgfsetdash{}{0pt}%
\pgfpathmoveto{\pgfqpoint{2.199195in}{2.151591in}}%
\pgfpathlineto{\pgfqpoint{2.210467in}{2.210719in}}%
\pgfusepath{stroke}%
\end{pgfscope}%
\begin{pgfscope}%
\pgfpathrectangle{\pgfqpoint{0.100000in}{0.212622in}}{\pgfqpoint{3.696000in}{3.696000in}}%
\pgfusepath{clip}%
\pgfsetrectcap%
\pgfsetroundjoin%
\pgfsetlinewidth{1.505625pt}%
\definecolor{currentstroke}{rgb}{1.000000,0.000000,0.000000}%
\pgfsetstrokecolor{currentstroke}%
\pgfsetdash{}{0pt}%
\pgfpathmoveto{\pgfqpoint{2.200680in}{2.154251in}}%
\pgfpathlineto{\pgfqpoint{2.210467in}{2.210719in}}%
\pgfusepath{stroke}%
\end{pgfscope}%
\begin{pgfscope}%
\pgfpathrectangle{\pgfqpoint{0.100000in}{0.212622in}}{\pgfqpoint{3.696000in}{3.696000in}}%
\pgfusepath{clip}%
\pgfsetrectcap%
\pgfsetroundjoin%
\pgfsetlinewidth{1.505625pt}%
\definecolor{currentstroke}{rgb}{1.000000,0.000000,0.000000}%
\pgfsetstrokecolor{currentstroke}%
\pgfsetdash{}{0pt}%
\pgfpathmoveto{\pgfqpoint{2.201712in}{2.158301in}}%
\pgfpathlineto{\pgfqpoint{2.214981in}{2.214657in}}%
\pgfusepath{stroke}%
\end{pgfscope}%
\begin{pgfscope}%
\pgfpathrectangle{\pgfqpoint{0.100000in}{0.212622in}}{\pgfqpoint{3.696000in}{3.696000in}}%
\pgfusepath{clip}%
\pgfsetrectcap%
\pgfsetroundjoin%
\pgfsetlinewidth{1.505625pt}%
\definecolor{currentstroke}{rgb}{1.000000,0.000000,0.000000}%
\pgfsetstrokecolor{currentstroke}%
\pgfsetdash{}{0pt}%
\pgfpathmoveto{\pgfqpoint{2.203935in}{2.162096in}}%
\pgfpathlineto{\pgfqpoint{2.214981in}{2.214657in}}%
\pgfusepath{stroke}%
\end{pgfscope}%
\begin{pgfscope}%
\pgfpathrectangle{\pgfqpoint{0.100000in}{0.212622in}}{\pgfqpoint{3.696000in}{3.696000in}}%
\pgfusepath{clip}%
\pgfsetrectcap%
\pgfsetroundjoin%
\pgfsetlinewidth{1.505625pt}%
\definecolor{currentstroke}{rgb}{1.000000,0.000000,0.000000}%
\pgfsetstrokecolor{currentstroke}%
\pgfsetdash{}{0pt}%
\pgfpathmoveto{\pgfqpoint{2.205459in}{2.163995in}}%
\pgfpathlineto{\pgfqpoint{2.214981in}{2.214657in}}%
\pgfusepath{stroke}%
\end{pgfscope}%
\begin{pgfscope}%
\pgfpathrectangle{\pgfqpoint{0.100000in}{0.212622in}}{\pgfqpoint{3.696000in}{3.696000in}}%
\pgfusepath{clip}%
\pgfsetrectcap%
\pgfsetroundjoin%
\pgfsetlinewidth{1.505625pt}%
\definecolor{currentstroke}{rgb}{1.000000,0.000000,0.000000}%
\pgfsetstrokecolor{currentstroke}%
\pgfsetdash{}{0pt}%
\pgfpathmoveto{\pgfqpoint{2.206253in}{2.164999in}}%
\pgfpathlineto{\pgfqpoint{2.214981in}{2.214657in}}%
\pgfusepath{stroke}%
\end{pgfscope}%
\begin{pgfscope}%
\pgfpathrectangle{\pgfqpoint{0.100000in}{0.212622in}}{\pgfqpoint{3.696000in}{3.696000in}}%
\pgfusepath{clip}%
\pgfsetrectcap%
\pgfsetroundjoin%
\pgfsetlinewidth{1.505625pt}%
\definecolor{currentstroke}{rgb}{1.000000,0.000000,0.000000}%
\pgfsetstrokecolor{currentstroke}%
\pgfsetdash{}{0pt}%
\pgfpathmoveto{\pgfqpoint{2.207107in}{2.167296in}}%
\pgfpathlineto{\pgfqpoint{2.219492in}{2.218592in}}%
\pgfusepath{stroke}%
\end{pgfscope}%
\begin{pgfscope}%
\pgfpathrectangle{\pgfqpoint{0.100000in}{0.212622in}}{\pgfqpoint{3.696000in}{3.696000in}}%
\pgfusepath{clip}%
\pgfsetrectcap%
\pgfsetroundjoin%
\pgfsetlinewidth{1.505625pt}%
\definecolor{currentstroke}{rgb}{1.000000,0.000000,0.000000}%
\pgfsetstrokecolor{currentstroke}%
\pgfsetdash{}{0pt}%
\pgfpathmoveto{\pgfqpoint{2.207557in}{2.168547in}}%
\pgfpathlineto{\pgfqpoint{2.219492in}{2.218592in}}%
\pgfusepath{stroke}%
\end{pgfscope}%
\begin{pgfscope}%
\pgfpathrectangle{\pgfqpoint{0.100000in}{0.212622in}}{\pgfqpoint{3.696000in}{3.696000in}}%
\pgfusepath{clip}%
\pgfsetrectcap%
\pgfsetroundjoin%
\pgfsetlinewidth{1.505625pt}%
\definecolor{currentstroke}{rgb}{1.000000,0.000000,0.000000}%
\pgfsetstrokecolor{currentstroke}%
\pgfsetdash{}{0pt}%
\pgfpathmoveto{\pgfqpoint{2.209002in}{2.170476in}}%
\pgfpathlineto{\pgfqpoint{2.219492in}{2.218592in}}%
\pgfusepath{stroke}%
\end{pgfscope}%
\begin{pgfscope}%
\pgfpathrectangle{\pgfqpoint{0.100000in}{0.212622in}}{\pgfqpoint{3.696000in}{3.696000in}}%
\pgfusepath{clip}%
\pgfsetrectcap%
\pgfsetroundjoin%
\pgfsetlinewidth{1.505625pt}%
\definecolor{currentstroke}{rgb}{1.000000,0.000000,0.000000}%
\pgfsetstrokecolor{currentstroke}%
\pgfsetdash{}{0pt}%
\pgfpathmoveto{\pgfqpoint{2.209859in}{2.171437in}}%
\pgfpathlineto{\pgfqpoint{2.219492in}{2.218592in}}%
\pgfusepath{stroke}%
\end{pgfscope}%
\begin{pgfscope}%
\pgfpathrectangle{\pgfqpoint{0.100000in}{0.212622in}}{\pgfqpoint{3.696000in}{3.696000in}}%
\pgfusepath{clip}%
\pgfsetrectcap%
\pgfsetroundjoin%
\pgfsetlinewidth{1.505625pt}%
\definecolor{currentstroke}{rgb}{1.000000,0.000000,0.000000}%
\pgfsetstrokecolor{currentstroke}%
\pgfsetdash{}{0pt}%
\pgfpathmoveto{\pgfqpoint{2.211266in}{2.174375in}}%
\pgfpathlineto{\pgfqpoint{2.223999in}{2.222525in}}%
\pgfusepath{stroke}%
\end{pgfscope}%
\begin{pgfscope}%
\pgfpathrectangle{\pgfqpoint{0.100000in}{0.212622in}}{\pgfqpoint{3.696000in}{3.696000in}}%
\pgfusepath{clip}%
\pgfsetrectcap%
\pgfsetroundjoin%
\pgfsetlinewidth{1.505625pt}%
\definecolor{currentstroke}{rgb}{1.000000,0.000000,0.000000}%
\pgfsetstrokecolor{currentstroke}%
\pgfsetdash{}{0pt}%
\pgfpathmoveto{\pgfqpoint{2.211948in}{2.178735in}}%
\pgfpathlineto{\pgfqpoint{2.223999in}{2.222525in}}%
\pgfusepath{stroke}%
\end{pgfscope}%
\begin{pgfscope}%
\pgfpathrectangle{\pgfqpoint{0.100000in}{0.212622in}}{\pgfqpoint{3.696000in}{3.696000in}}%
\pgfusepath{clip}%
\pgfsetrectcap%
\pgfsetroundjoin%
\pgfsetlinewidth{1.505625pt}%
\definecolor{currentstroke}{rgb}{1.000000,0.000000,0.000000}%
\pgfsetstrokecolor{currentstroke}%
\pgfsetdash{}{0pt}%
\pgfpathmoveto{\pgfqpoint{2.214841in}{2.183137in}}%
\pgfpathlineto{\pgfqpoint{2.228504in}{2.226454in}}%
\pgfusepath{stroke}%
\end{pgfscope}%
\begin{pgfscope}%
\pgfpathrectangle{\pgfqpoint{0.100000in}{0.212622in}}{\pgfqpoint{3.696000in}{3.696000in}}%
\pgfusepath{clip}%
\pgfsetrectcap%
\pgfsetroundjoin%
\pgfsetlinewidth{1.505625pt}%
\definecolor{currentstroke}{rgb}{1.000000,0.000000,0.000000}%
\pgfsetstrokecolor{currentstroke}%
\pgfsetdash{}{0pt}%
\pgfpathmoveto{\pgfqpoint{2.216699in}{2.185268in}}%
\pgfpathlineto{\pgfqpoint{2.228504in}{2.226454in}}%
\pgfusepath{stroke}%
\end{pgfscope}%
\begin{pgfscope}%
\pgfpathrectangle{\pgfqpoint{0.100000in}{0.212622in}}{\pgfqpoint{3.696000in}{3.696000in}}%
\pgfusepath{clip}%
\pgfsetrectcap%
\pgfsetroundjoin%
\pgfsetlinewidth{1.505625pt}%
\definecolor{currentstroke}{rgb}{1.000000,0.000000,0.000000}%
\pgfsetstrokecolor{currentstroke}%
\pgfsetdash{}{0pt}%
\pgfpathmoveto{\pgfqpoint{2.218835in}{2.189175in}}%
\pgfpathlineto{\pgfqpoint{2.228504in}{2.226454in}}%
\pgfusepath{stroke}%
\end{pgfscope}%
\begin{pgfscope}%
\pgfpathrectangle{\pgfqpoint{0.100000in}{0.212622in}}{\pgfqpoint{3.696000in}{3.696000in}}%
\pgfusepath{clip}%
\pgfsetrectcap%
\pgfsetroundjoin%
\pgfsetlinewidth{1.505625pt}%
\definecolor{currentstroke}{rgb}{1.000000,0.000000,0.000000}%
\pgfsetstrokecolor{currentstroke}%
\pgfsetdash{}{0pt}%
\pgfpathmoveto{\pgfqpoint{2.219943in}{2.194690in}}%
\pgfpathlineto{\pgfqpoint{2.233005in}{2.230382in}}%
\pgfusepath{stroke}%
\end{pgfscope}%
\begin{pgfscope}%
\pgfpathrectangle{\pgfqpoint{0.100000in}{0.212622in}}{\pgfqpoint{3.696000in}{3.696000in}}%
\pgfusepath{clip}%
\pgfsetrectcap%
\pgfsetroundjoin%
\pgfsetlinewidth{1.505625pt}%
\definecolor{currentstroke}{rgb}{1.000000,0.000000,0.000000}%
\pgfsetstrokecolor{currentstroke}%
\pgfsetdash{}{0pt}%
\pgfpathmoveto{\pgfqpoint{2.223019in}{2.200016in}}%
\pgfpathlineto{\pgfqpoint{2.233005in}{2.230382in}}%
\pgfusepath{stroke}%
\end{pgfscope}%
\begin{pgfscope}%
\pgfpathrectangle{\pgfqpoint{0.100000in}{0.212622in}}{\pgfqpoint{3.696000in}{3.696000in}}%
\pgfusepath{clip}%
\pgfsetrectcap%
\pgfsetroundjoin%
\pgfsetlinewidth{1.505625pt}%
\definecolor{currentstroke}{rgb}{1.000000,0.000000,0.000000}%
\pgfsetstrokecolor{currentstroke}%
\pgfsetdash{}{0pt}%
\pgfpathmoveto{\pgfqpoint{2.227076in}{2.205000in}}%
\pgfpathlineto{\pgfqpoint{2.237504in}{2.234306in}}%
\pgfusepath{stroke}%
\end{pgfscope}%
\begin{pgfscope}%
\pgfpathrectangle{\pgfqpoint{0.100000in}{0.212622in}}{\pgfqpoint{3.696000in}{3.696000in}}%
\pgfusepath{clip}%
\pgfsetrectcap%
\pgfsetroundjoin%
\pgfsetlinewidth{1.505625pt}%
\definecolor{currentstroke}{rgb}{1.000000,0.000000,0.000000}%
\pgfsetstrokecolor{currentstroke}%
\pgfsetdash{}{0pt}%
\pgfpathmoveto{\pgfqpoint{2.230837in}{2.210916in}}%
\pgfpathlineto{\pgfqpoint{2.241999in}{2.238228in}}%
\pgfusepath{stroke}%
\end{pgfscope}%
\begin{pgfscope}%
\pgfpathrectangle{\pgfqpoint{0.100000in}{0.212622in}}{\pgfqpoint{3.696000in}{3.696000in}}%
\pgfusepath{clip}%
\pgfsetrectcap%
\pgfsetroundjoin%
\pgfsetlinewidth{1.505625pt}%
\definecolor{currentstroke}{rgb}{1.000000,0.000000,0.000000}%
\pgfsetstrokecolor{currentstroke}%
\pgfsetdash{}{0pt}%
\pgfpathmoveto{\pgfqpoint{2.233161in}{2.219072in}}%
\pgfpathlineto{\pgfqpoint{2.246491in}{2.242147in}}%
\pgfusepath{stroke}%
\end{pgfscope}%
\begin{pgfscope}%
\pgfpathrectangle{\pgfqpoint{0.100000in}{0.212622in}}{\pgfqpoint{3.696000in}{3.696000in}}%
\pgfusepath{clip}%
\pgfsetrectcap%
\pgfsetroundjoin%
\pgfsetlinewidth{1.505625pt}%
\definecolor{currentstroke}{rgb}{1.000000,0.000000,0.000000}%
\pgfsetstrokecolor{currentstroke}%
\pgfsetdash{}{0pt}%
\pgfpathmoveto{\pgfqpoint{2.234804in}{2.223331in}}%
\pgfpathlineto{\pgfqpoint{2.246491in}{2.242147in}}%
\pgfusepath{stroke}%
\end{pgfscope}%
\begin{pgfscope}%
\pgfpathrectangle{\pgfqpoint{0.100000in}{0.212622in}}{\pgfqpoint{3.696000in}{3.696000in}}%
\pgfusepath{clip}%
\pgfsetrectcap%
\pgfsetroundjoin%
\pgfsetlinewidth{1.505625pt}%
\definecolor{currentstroke}{rgb}{1.000000,0.000000,0.000000}%
\pgfsetstrokecolor{currentstroke}%
\pgfsetdash{}{0pt}%
\pgfpathmoveto{\pgfqpoint{2.236267in}{2.225322in}}%
\pgfpathlineto{\pgfqpoint{2.246491in}{2.242147in}}%
\pgfusepath{stroke}%
\end{pgfscope}%
\begin{pgfscope}%
\pgfpathrectangle{\pgfqpoint{0.100000in}{0.212622in}}{\pgfqpoint{3.696000in}{3.696000in}}%
\pgfusepath{clip}%
\pgfsetrectcap%
\pgfsetroundjoin%
\pgfsetlinewidth{1.505625pt}%
\definecolor{currentstroke}{rgb}{1.000000,0.000000,0.000000}%
\pgfsetstrokecolor{currentstroke}%
\pgfsetdash{}{0pt}%
\pgfpathmoveto{\pgfqpoint{2.237060in}{2.226369in}}%
\pgfpathlineto{\pgfqpoint{2.250980in}{2.246063in}}%
\pgfusepath{stroke}%
\end{pgfscope}%
\begin{pgfscope}%
\pgfpathrectangle{\pgfqpoint{0.100000in}{0.212622in}}{\pgfqpoint{3.696000in}{3.696000in}}%
\pgfusepath{clip}%
\pgfsetrectcap%
\pgfsetroundjoin%
\pgfsetlinewidth{1.505625pt}%
\definecolor{currentstroke}{rgb}{1.000000,0.000000,0.000000}%
\pgfsetstrokecolor{currentstroke}%
\pgfsetdash{}{0pt}%
\pgfpathmoveto{\pgfqpoint{2.237743in}{2.228790in}}%
\pgfpathlineto{\pgfqpoint{2.250980in}{2.246063in}}%
\pgfusepath{stroke}%
\end{pgfscope}%
\begin{pgfscope}%
\pgfpathrectangle{\pgfqpoint{0.100000in}{0.212622in}}{\pgfqpoint{3.696000in}{3.696000in}}%
\pgfusepath{clip}%
\pgfsetrectcap%
\pgfsetroundjoin%
\pgfsetlinewidth{1.505625pt}%
\definecolor{currentstroke}{rgb}{1.000000,0.000000,0.000000}%
\pgfsetstrokecolor{currentstroke}%
\pgfsetdash{}{0pt}%
\pgfpathmoveto{\pgfqpoint{2.238120in}{2.230028in}}%
\pgfpathlineto{\pgfqpoint{2.250980in}{2.246063in}}%
\pgfusepath{stroke}%
\end{pgfscope}%
\begin{pgfscope}%
\pgfpathrectangle{\pgfqpoint{0.100000in}{0.212622in}}{\pgfqpoint{3.696000in}{3.696000in}}%
\pgfusepath{clip}%
\pgfsetrectcap%
\pgfsetroundjoin%
\pgfsetlinewidth{1.505625pt}%
\definecolor{currentstroke}{rgb}{1.000000,0.000000,0.000000}%
\pgfsetstrokecolor{currentstroke}%
\pgfsetdash{}{0pt}%
\pgfpathmoveto{\pgfqpoint{2.238526in}{2.230603in}}%
\pgfpathlineto{\pgfqpoint{2.250980in}{2.246063in}}%
\pgfusepath{stroke}%
\end{pgfscope}%
\begin{pgfscope}%
\pgfpathrectangle{\pgfqpoint{0.100000in}{0.212622in}}{\pgfqpoint{3.696000in}{3.696000in}}%
\pgfusepath{clip}%
\pgfsetrectcap%
\pgfsetroundjoin%
\pgfsetlinewidth{1.505625pt}%
\definecolor{currentstroke}{rgb}{1.000000,0.000000,0.000000}%
\pgfsetstrokecolor{currentstroke}%
\pgfsetdash{}{0pt}%
\pgfpathmoveto{\pgfqpoint{2.238761in}{2.230923in}}%
\pgfpathlineto{\pgfqpoint{2.250980in}{2.246063in}}%
\pgfusepath{stroke}%
\end{pgfscope}%
\begin{pgfscope}%
\pgfpathrectangle{\pgfqpoint{0.100000in}{0.212622in}}{\pgfqpoint{3.696000in}{3.696000in}}%
\pgfusepath{clip}%
\pgfsetrectcap%
\pgfsetroundjoin%
\pgfsetlinewidth{1.505625pt}%
\definecolor{currentstroke}{rgb}{1.000000,0.000000,0.000000}%
\pgfsetstrokecolor{currentstroke}%
\pgfsetdash{}{0pt}%
\pgfpathmoveto{\pgfqpoint{2.239639in}{2.232883in}}%
\pgfpathlineto{\pgfqpoint{2.250980in}{2.246063in}}%
\pgfusepath{stroke}%
\end{pgfscope}%
\begin{pgfscope}%
\pgfpathrectangle{\pgfqpoint{0.100000in}{0.212622in}}{\pgfqpoint{3.696000in}{3.696000in}}%
\pgfusepath{clip}%
\pgfsetrectcap%
\pgfsetroundjoin%
\pgfsetlinewidth{1.505625pt}%
\definecolor{currentstroke}{rgb}{1.000000,0.000000,0.000000}%
\pgfsetstrokecolor{currentstroke}%
\pgfsetdash{}{0pt}%
\pgfpathmoveto{\pgfqpoint{2.239841in}{2.234128in}}%
\pgfpathlineto{\pgfqpoint{2.250980in}{2.246063in}}%
\pgfusepath{stroke}%
\end{pgfscope}%
\begin{pgfscope}%
\pgfpathrectangle{\pgfqpoint{0.100000in}{0.212622in}}{\pgfqpoint{3.696000in}{3.696000in}}%
\pgfusepath{clip}%
\pgfsetrectcap%
\pgfsetroundjoin%
\pgfsetlinewidth{1.505625pt}%
\definecolor{currentstroke}{rgb}{1.000000,0.000000,0.000000}%
\pgfsetstrokecolor{currentstroke}%
\pgfsetdash{}{0pt}%
\pgfpathmoveto{\pgfqpoint{2.241089in}{2.236400in}}%
\pgfpathlineto{\pgfqpoint{2.255466in}{2.249977in}}%
\pgfusepath{stroke}%
\end{pgfscope}%
\begin{pgfscope}%
\pgfpathrectangle{\pgfqpoint{0.100000in}{0.212622in}}{\pgfqpoint{3.696000in}{3.696000in}}%
\pgfusepath{clip}%
\pgfsetrectcap%
\pgfsetroundjoin%
\pgfsetlinewidth{1.505625pt}%
\definecolor{currentstroke}{rgb}{1.000000,0.000000,0.000000}%
\pgfsetstrokecolor{currentstroke}%
\pgfsetdash{}{0pt}%
\pgfpathmoveto{\pgfqpoint{2.241942in}{2.237472in}}%
\pgfpathlineto{\pgfqpoint{2.255466in}{2.249977in}}%
\pgfusepath{stroke}%
\end{pgfscope}%
\begin{pgfscope}%
\pgfpathrectangle{\pgfqpoint{0.100000in}{0.212622in}}{\pgfqpoint{3.696000in}{3.696000in}}%
\pgfusepath{clip}%
\pgfsetrectcap%
\pgfsetroundjoin%
\pgfsetlinewidth{1.505625pt}%
\definecolor{currentstroke}{rgb}{1.000000,0.000000,0.000000}%
\pgfsetstrokecolor{currentstroke}%
\pgfsetdash{}{0pt}%
\pgfpathmoveto{\pgfqpoint{2.243511in}{2.240566in}}%
\pgfpathlineto{\pgfqpoint{2.255466in}{2.249977in}}%
\pgfusepath{stroke}%
\end{pgfscope}%
\begin{pgfscope}%
\pgfpathrectangle{\pgfqpoint{0.100000in}{0.212622in}}{\pgfqpoint{3.696000in}{3.696000in}}%
\pgfusepath{clip}%
\pgfsetrectcap%
\pgfsetroundjoin%
\pgfsetlinewidth{1.505625pt}%
\definecolor{currentstroke}{rgb}{1.000000,0.000000,0.000000}%
\pgfsetstrokecolor{currentstroke}%
\pgfsetdash{}{0pt}%
\pgfpathmoveto{\pgfqpoint{2.243962in}{2.244846in}}%
\pgfpathlineto{\pgfqpoint{2.259949in}{2.253888in}}%
\pgfusepath{stroke}%
\end{pgfscope}%
\begin{pgfscope}%
\pgfpathrectangle{\pgfqpoint{0.100000in}{0.212622in}}{\pgfqpoint{3.696000in}{3.696000in}}%
\pgfusepath{clip}%
\pgfsetrectcap%
\pgfsetroundjoin%
\pgfsetlinewidth{1.505625pt}%
\definecolor{currentstroke}{rgb}{1.000000,0.000000,0.000000}%
\pgfsetstrokecolor{currentstroke}%
\pgfsetdash{}{0pt}%
\pgfpathmoveto{\pgfqpoint{2.246328in}{2.249474in}}%
\pgfpathlineto{\pgfqpoint{2.259949in}{2.253888in}}%
\pgfusepath{stroke}%
\end{pgfscope}%
\begin{pgfscope}%
\pgfpathrectangle{\pgfqpoint{0.100000in}{0.212622in}}{\pgfqpoint{3.696000in}{3.696000in}}%
\pgfusepath{clip}%
\pgfsetrectcap%
\pgfsetroundjoin%
\pgfsetlinewidth{1.505625pt}%
\definecolor{currentstroke}{rgb}{1.000000,0.000000,0.000000}%
\pgfsetstrokecolor{currentstroke}%
\pgfsetdash{}{0pt}%
\pgfpathmoveto{\pgfqpoint{2.248037in}{2.251602in}}%
\pgfpathlineto{\pgfqpoint{2.259949in}{2.253888in}}%
\pgfusepath{stroke}%
\end{pgfscope}%
\begin{pgfscope}%
\pgfpathrectangle{\pgfqpoint{0.100000in}{0.212622in}}{\pgfqpoint{3.696000in}{3.696000in}}%
\pgfusepath{clip}%
\pgfsetrectcap%
\pgfsetroundjoin%
\pgfsetlinewidth{1.505625pt}%
\definecolor{currentstroke}{rgb}{1.000000,0.000000,0.000000}%
\pgfsetstrokecolor{currentstroke}%
\pgfsetdash{}{0pt}%
\pgfpathmoveto{\pgfqpoint{2.250117in}{2.254934in}}%
\pgfpathlineto{\pgfqpoint{2.264428in}{2.257796in}}%
\pgfusepath{stroke}%
\end{pgfscope}%
\begin{pgfscope}%
\pgfpathrectangle{\pgfqpoint{0.100000in}{0.212622in}}{\pgfqpoint{3.696000in}{3.696000in}}%
\pgfusepath{clip}%
\pgfsetrectcap%
\pgfsetroundjoin%
\pgfsetlinewidth{1.505625pt}%
\definecolor{currentstroke}{rgb}{1.000000,0.000000,0.000000}%
\pgfsetstrokecolor{currentstroke}%
\pgfsetdash{}{0pt}%
\pgfpathmoveto{\pgfqpoint{2.251004in}{2.260614in}}%
\pgfpathlineto{\pgfqpoint{2.264428in}{2.257796in}}%
\pgfusepath{stroke}%
\end{pgfscope}%
\begin{pgfscope}%
\pgfpathrectangle{\pgfqpoint{0.100000in}{0.212622in}}{\pgfqpoint{3.696000in}{3.696000in}}%
\pgfusepath{clip}%
\pgfsetrectcap%
\pgfsetroundjoin%
\pgfsetlinewidth{1.505625pt}%
\definecolor{currentstroke}{rgb}{1.000000,0.000000,0.000000}%
\pgfsetstrokecolor{currentstroke}%
\pgfsetdash{}{0pt}%
\pgfpathmoveto{\pgfqpoint{2.251985in}{2.263559in}}%
\pgfpathlineto{\pgfqpoint{2.268905in}{2.261702in}}%
\pgfusepath{stroke}%
\end{pgfscope}%
\begin{pgfscope}%
\pgfpathrectangle{\pgfqpoint{0.100000in}{0.212622in}}{\pgfqpoint{3.696000in}{3.696000in}}%
\pgfusepath{clip}%
\pgfsetrectcap%
\pgfsetroundjoin%
\pgfsetlinewidth{1.505625pt}%
\definecolor{currentstroke}{rgb}{1.000000,0.000000,0.000000}%
\pgfsetstrokecolor{currentstroke}%
\pgfsetdash{}{0pt}%
\pgfpathmoveto{\pgfqpoint{2.254255in}{2.266819in}}%
\pgfpathlineto{\pgfqpoint{2.268905in}{2.261702in}}%
\pgfusepath{stroke}%
\end{pgfscope}%
\begin{pgfscope}%
\pgfpathrectangle{\pgfqpoint{0.100000in}{0.212622in}}{\pgfqpoint{3.696000in}{3.696000in}}%
\pgfusepath{clip}%
\pgfsetrectcap%
\pgfsetroundjoin%
\pgfsetlinewidth{1.505625pt}%
\definecolor{currentstroke}{rgb}{1.000000,0.000000,0.000000}%
\pgfsetstrokecolor{currentstroke}%
\pgfsetdash{}{0pt}%
\pgfpathmoveto{\pgfqpoint{2.255525in}{2.268455in}}%
\pgfpathlineto{\pgfqpoint{2.268905in}{2.261702in}}%
\pgfusepath{stroke}%
\end{pgfscope}%
\begin{pgfscope}%
\pgfpathrectangle{\pgfqpoint{0.100000in}{0.212622in}}{\pgfqpoint{3.696000in}{3.696000in}}%
\pgfusepath{clip}%
\pgfsetrectcap%
\pgfsetroundjoin%
\pgfsetlinewidth{1.505625pt}%
\definecolor{currentstroke}{rgb}{1.000000,0.000000,0.000000}%
\pgfsetstrokecolor{currentstroke}%
\pgfsetdash{}{0pt}%
\pgfpathmoveto{\pgfqpoint{2.256913in}{2.272681in}}%
\pgfpathlineto{\pgfqpoint{2.273379in}{2.265605in}}%
\pgfusepath{stroke}%
\end{pgfscope}%
\begin{pgfscope}%
\pgfpathrectangle{\pgfqpoint{0.100000in}{0.212622in}}{\pgfqpoint{3.696000in}{3.696000in}}%
\pgfusepath{clip}%
\pgfsetrectcap%
\pgfsetroundjoin%
\pgfsetlinewidth{1.505625pt}%
\definecolor{currentstroke}{rgb}{1.000000,0.000000,0.000000}%
\pgfsetstrokecolor{currentstroke}%
\pgfsetdash{}{0pt}%
\pgfpathmoveto{\pgfqpoint{2.257542in}{2.275182in}}%
\pgfpathlineto{\pgfqpoint{2.273379in}{2.265605in}}%
\pgfusepath{stroke}%
\end{pgfscope}%
\begin{pgfscope}%
\pgfpathrectangle{\pgfqpoint{0.100000in}{0.212622in}}{\pgfqpoint{3.696000in}{3.696000in}}%
\pgfusepath{clip}%
\pgfsetrectcap%
\pgfsetroundjoin%
\pgfsetlinewidth{1.505625pt}%
\definecolor{currentstroke}{rgb}{1.000000,0.000000,0.000000}%
\pgfsetstrokecolor{currentstroke}%
\pgfsetdash{}{0pt}%
\pgfpathmoveto{\pgfqpoint{2.259876in}{2.278642in}}%
\pgfpathlineto{\pgfqpoint{2.273379in}{2.265605in}}%
\pgfusepath{stroke}%
\end{pgfscope}%
\begin{pgfscope}%
\pgfpathrectangle{\pgfqpoint{0.100000in}{0.212622in}}{\pgfqpoint{3.696000in}{3.696000in}}%
\pgfusepath{clip}%
\pgfsetrectcap%
\pgfsetroundjoin%
\pgfsetlinewidth{1.505625pt}%
\definecolor{currentstroke}{rgb}{1.000000,0.000000,0.000000}%
\pgfsetstrokecolor{currentstroke}%
\pgfsetdash{}{0pt}%
\pgfpathmoveto{\pgfqpoint{2.262582in}{2.282210in}}%
\pgfpathlineto{\pgfqpoint{2.277849in}{2.269505in}}%
\pgfusepath{stroke}%
\end{pgfscope}%
\begin{pgfscope}%
\pgfpathrectangle{\pgfqpoint{0.100000in}{0.212622in}}{\pgfqpoint{3.696000in}{3.696000in}}%
\pgfusepath{clip}%
\pgfsetrectcap%
\pgfsetroundjoin%
\pgfsetlinewidth{1.505625pt}%
\definecolor{currentstroke}{rgb}{1.000000,0.000000,0.000000}%
\pgfsetstrokecolor{currentstroke}%
\pgfsetdash{}{0pt}%
\pgfpathmoveto{\pgfqpoint{2.265051in}{2.288636in}}%
\pgfpathlineto{\pgfqpoint{2.277849in}{2.269505in}}%
\pgfusepath{stroke}%
\end{pgfscope}%
\begin{pgfscope}%
\pgfpathrectangle{\pgfqpoint{0.100000in}{0.212622in}}{\pgfqpoint{3.696000in}{3.696000in}}%
\pgfusepath{clip}%
\pgfsetrectcap%
\pgfsetroundjoin%
\pgfsetlinewidth{1.505625pt}%
\definecolor{currentstroke}{rgb}{1.000000,0.000000,0.000000}%
\pgfsetstrokecolor{currentstroke}%
\pgfsetdash{}{0pt}%
\pgfpathmoveto{\pgfqpoint{2.265614in}{2.292652in}}%
\pgfpathlineto{\pgfqpoint{2.282317in}{2.273402in}}%
\pgfusepath{stroke}%
\end{pgfscope}%
\begin{pgfscope}%
\pgfpathrectangle{\pgfqpoint{0.100000in}{0.212622in}}{\pgfqpoint{3.696000in}{3.696000in}}%
\pgfusepath{clip}%
\pgfsetrectcap%
\pgfsetroundjoin%
\pgfsetlinewidth{1.505625pt}%
\definecolor{currentstroke}{rgb}{1.000000,0.000000,0.000000}%
\pgfsetstrokecolor{currentstroke}%
\pgfsetdash{}{0pt}%
\pgfpathmoveto{\pgfqpoint{2.268459in}{2.297524in}}%
\pgfpathlineto{\pgfqpoint{2.282317in}{2.273402in}}%
\pgfusepath{stroke}%
\end{pgfscope}%
\begin{pgfscope}%
\pgfpathrectangle{\pgfqpoint{0.100000in}{0.212622in}}{\pgfqpoint{3.696000in}{3.696000in}}%
\pgfusepath{clip}%
\pgfsetrectcap%
\pgfsetroundjoin%
\pgfsetlinewidth{1.505625pt}%
\definecolor{currentstroke}{rgb}{1.000000,0.000000,0.000000}%
\pgfsetstrokecolor{currentstroke}%
\pgfsetdash{}{0pt}%
\pgfpathmoveto{\pgfqpoint{2.270257in}{2.299910in}}%
\pgfpathlineto{\pgfqpoint{2.286781in}{2.277297in}}%
\pgfusepath{stroke}%
\end{pgfscope}%
\begin{pgfscope}%
\pgfpathrectangle{\pgfqpoint{0.100000in}{0.212622in}}{\pgfqpoint{3.696000in}{3.696000in}}%
\pgfusepath{clip}%
\pgfsetrectcap%
\pgfsetroundjoin%
\pgfsetlinewidth{1.505625pt}%
\definecolor{currentstroke}{rgb}{1.000000,0.000000,0.000000}%
\pgfsetstrokecolor{currentstroke}%
\pgfsetdash{}{0pt}%
\pgfpathmoveto{\pgfqpoint{2.272119in}{2.304291in}}%
\pgfpathlineto{\pgfqpoint{2.286781in}{2.277297in}}%
\pgfusepath{stroke}%
\end{pgfscope}%
\begin{pgfscope}%
\pgfpathrectangle{\pgfqpoint{0.100000in}{0.212622in}}{\pgfqpoint{3.696000in}{3.696000in}}%
\pgfusepath{clip}%
\pgfsetrectcap%
\pgfsetroundjoin%
\pgfsetlinewidth{1.505625pt}%
\definecolor{currentstroke}{rgb}{1.000000,0.000000,0.000000}%
\pgfsetstrokecolor{currentstroke}%
\pgfsetdash{}{0pt}%
\pgfpathmoveto{\pgfqpoint{2.272528in}{2.307111in}}%
\pgfpathlineto{\pgfqpoint{2.286781in}{2.277297in}}%
\pgfusepath{stroke}%
\end{pgfscope}%
\begin{pgfscope}%
\pgfpathrectangle{\pgfqpoint{0.100000in}{0.212622in}}{\pgfqpoint{3.696000in}{3.696000in}}%
\pgfusepath{clip}%
\pgfsetrectcap%
\pgfsetroundjoin%
\pgfsetlinewidth{1.505625pt}%
\definecolor{currentstroke}{rgb}{1.000000,0.000000,0.000000}%
\pgfsetstrokecolor{currentstroke}%
\pgfsetdash{}{0pt}%
\pgfpathmoveto{\pgfqpoint{2.275038in}{2.310916in}}%
\pgfpathlineto{\pgfqpoint{2.291243in}{2.281190in}}%
\pgfusepath{stroke}%
\end{pgfscope}%
\begin{pgfscope}%
\pgfpathrectangle{\pgfqpoint{0.100000in}{0.212622in}}{\pgfqpoint{3.696000in}{3.696000in}}%
\pgfusepath{clip}%
\pgfsetrectcap%
\pgfsetroundjoin%
\pgfsetlinewidth{1.505625pt}%
\definecolor{currentstroke}{rgb}{1.000000,0.000000,0.000000}%
\pgfsetstrokecolor{currentstroke}%
\pgfsetdash{}{0pt}%
\pgfpathmoveto{\pgfqpoint{2.276530in}{2.312868in}}%
\pgfpathlineto{\pgfqpoint{2.291243in}{2.281190in}}%
\pgfusepath{stroke}%
\end{pgfscope}%
\begin{pgfscope}%
\pgfpathrectangle{\pgfqpoint{0.100000in}{0.212622in}}{\pgfqpoint{3.696000in}{3.696000in}}%
\pgfusepath{clip}%
\pgfsetrectcap%
\pgfsetroundjoin%
\pgfsetlinewidth{1.505625pt}%
\definecolor{currentstroke}{rgb}{1.000000,0.000000,0.000000}%
\pgfsetstrokecolor{currentstroke}%
\pgfsetdash{}{0pt}%
\pgfpathmoveto{\pgfqpoint{2.278285in}{2.317165in}}%
\pgfpathlineto{\pgfqpoint{2.291243in}{2.281190in}}%
\pgfusepath{stroke}%
\end{pgfscope}%
\begin{pgfscope}%
\pgfpathrectangle{\pgfqpoint{0.100000in}{0.212622in}}{\pgfqpoint{3.696000in}{3.696000in}}%
\pgfusepath{clip}%
\pgfsetrectcap%
\pgfsetroundjoin%
\pgfsetlinewidth{1.505625pt}%
\definecolor{currentstroke}{rgb}{1.000000,0.000000,0.000000}%
\pgfsetstrokecolor{currentstroke}%
\pgfsetdash{}{0pt}%
\pgfpathmoveto{\pgfqpoint{2.278875in}{2.319994in}}%
\pgfpathlineto{\pgfqpoint{2.295701in}{2.285079in}}%
\pgfusepath{stroke}%
\end{pgfscope}%
\begin{pgfscope}%
\pgfpathrectangle{\pgfqpoint{0.100000in}{0.212622in}}{\pgfqpoint{3.696000in}{3.696000in}}%
\pgfusepath{clip}%
\pgfsetrectcap%
\pgfsetroundjoin%
\pgfsetlinewidth{1.505625pt}%
\definecolor{currentstroke}{rgb}{1.000000,0.000000,0.000000}%
\pgfsetstrokecolor{currentstroke}%
\pgfsetdash{}{0pt}%
\pgfpathmoveto{\pgfqpoint{2.281212in}{2.323632in}}%
\pgfpathlineto{\pgfqpoint{2.295701in}{2.285079in}}%
\pgfusepath{stroke}%
\end{pgfscope}%
\begin{pgfscope}%
\pgfpathrectangle{\pgfqpoint{0.100000in}{0.212622in}}{\pgfqpoint{3.696000in}{3.696000in}}%
\pgfusepath{clip}%
\pgfsetrectcap%
\pgfsetroundjoin%
\pgfsetlinewidth{1.505625pt}%
\definecolor{currentstroke}{rgb}{1.000000,0.000000,0.000000}%
\pgfsetstrokecolor{currentstroke}%
\pgfsetdash{}{0pt}%
\pgfpathmoveto{\pgfqpoint{2.283865in}{2.327354in}}%
\pgfpathlineto{\pgfqpoint{2.300156in}{2.288966in}}%
\pgfusepath{stroke}%
\end{pgfscope}%
\begin{pgfscope}%
\pgfpathrectangle{\pgfqpoint{0.100000in}{0.212622in}}{\pgfqpoint{3.696000in}{3.696000in}}%
\pgfusepath{clip}%
\pgfsetrectcap%
\pgfsetroundjoin%
\pgfsetlinewidth{1.505625pt}%
\definecolor{currentstroke}{rgb}{1.000000,0.000000,0.000000}%
\pgfsetstrokecolor{currentstroke}%
\pgfsetdash{}{0pt}%
\pgfpathmoveto{\pgfqpoint{2.286622in}{2.334085in}}%
\pgfpathlineto{\pgfqpoint{2.300156in}{2.288966in}}%
\pgfusepath{stroke}%
\end{pgfscope}%
\begin{pgfscope}%
\pgfpathrectangle{\pgfqpoint{0.100000in}{0.212622in}}{\pgfqpoint{3.696000in}{3.696000in}}%
\pgfusepath{clip}%
\pgfsetrectcap%
\pgfsetroundjoin%
\pgfsetlinewidth{1.505625pt}%
\definecolor{currentstroke}{rgb}{1.000000,0.000000,0.000000}%
\pgfsetstrokecolor{currentstroke}%
\pgfsetdash{}{0pt}%
\pgfpathmoveto{\pgfqpoint{2.287625in}{2.338086in}}%
\pgfpathlineto{\pgfqpoint{2.304609in}{2.292850in}}%
\pgfusepath{stroke}%
\end{pgfscope}%
\begin{pgfscope}%
\pgfpathrectangle{\pgfqpoint{0.100000in}{0.212622in}}{\pgfqpoint{3.696000in}{3.696000in}}%
\pgfusepath{clip}%
\pgfsetrectcap%
\pgfsetroundjoin%
\pgfsetlinewidth{1.505625pt}%
\definecolor{currentstroke}{rgb}{1.000000,0.000000,0.000000}%
\pgfsetstrokecolor{currentstroke}%
\pgfsetdash{}{0pt}%
\pgfpathmoveto{\pgfqpoint{2.291032in}{2.343145in}}%
\pgfpathlineto{\pgfqpoint{2.304609in}{2.292850in}}%
\pgfusepath{stroke}%
\end{pgfscope}%
\begin{pgfscope}%
\pgfpathrectangle{\pgfqpoint{0.100000in}{0.212622in}}{\pgfqpoint{3.696000in}{3.696000in}}%
\pgfusepath{clip}%
\pgfsetrectcap%
\pgfsetroundjoin%
\pgfsetlinewidth{1.505625pt}%
\definecolor{currentstroke}{rgb}{1.000000,0.000000,0.000000}%
\pgfsetstrokecolor{currentstroke}%
\pgfsetdash{}{0pt}%
\pgfpathmoveto{\pgfqpoint{2.292994in}{2.345577in}}%
\pgfpathlineto{\pgfqpoint{2.309058in}{2.296732in}}%
\pgfusepath{stroke}%
\end{pgfscope}%
\begin{pgfscope}%
\pgfpathrectangle{\pgfqpoint{0.100000in}{0.212622in}}{\pgfqpoint{3.696000in}{3.696000in}}%
\pgfusepath{clip}%
\pgfsetrectcap%
\pgfsetroundjoin%
\pgfsetlinewidth{1.505625pt}%
\definecolor{currentstroke}{rgb}{1.000000,0.000000,0.000000}%
\pgfsetstrokecolor{currentstroke}%
\pgfsetdash{}{0pt}%
\pgfpathmoveto{\pgfqpoint{2.294530in}{2.350239in}}%
\pgfpathlineto{\pgfqpoint{2.309058in}{2.296732in}}%
\pgfusepath{stroke}%
\end{pgfscope}%
\begin{pgfscope}%
\pgfpathrectangle{\pgfqpoint{0.100000in}{0.212622in}}{\pgfqpoint{3.696000in}{3.696000in}}%
\pgfusepath{clip}%
\pgfsetrectcap%
\pgfsetroundjoin%
\pgfsetlinewidth{1.505625pt}%
\definecolor{currentstroke}{rgb}{1.000000,0.000000,0.000000}%
\pgfsetstrokecolor{currentstroke}%
\pgfsetdash{}{0pt}%
\pgfpathmoveto{\pgfqpoint{2.294882in}{2.353132in}}%
\pgfpathlineto{\pgfqpoint{2.309058in}{2.296732in}}%
\pgfusepath{stroke}%
\end{pgfscope}%
\begin{pgfscope}%
\pgfpathrectangle{\pgfqpoint{0.100000in}{0.212622in}}{\pgfqpoint{3.696000in}{3.696000in}}%
\pgfusepath{clip}%
\pgfsetrectcap%
\pgfsetroundjoin%
\pgfsetlinewidth{1.505625pt}%
\definecolor{currentstroke}{rgb}{1.000000,0.000000,0.000000}%
\pgfsetstrokecolor{currentstroke}%
\pgfsetdash{}{0pt}%
\pgfpathmoveto{\pgfqpoint{2.297739in}{2.357224in}}%
\pgfpathlineto{\pgfqpoint{2.313504in}{2.300611in}}%
\pgfusepath{stroke}%
\end{pgfscope}%
\begin{pgfscope}%
\pgfpathrectangle{\pgfqpoint{0.100000in}{0.212622in}}{\pgfqpoint{3.696000in}{3.696000in}}%
\pgfusepath{clip}%
\pgfsetrectcap%
\pgfsetroundjoin%
\pgfsetlinewidth{1.505625pt}%
\definecolor{currentstroke}{rgb}{1.000000,0.000000,0.000000}%
\pgfsetstrokecolor{currentstroke}%
\pgfsetdash{}{0pt}%
\pgfpathmoveto{\pgfqpoint{2.300962in}{2.361506in}}%
\pgfpathlineto{\pgfqpoint{2.313504in}{2.300611in}}%
\pgfusepath{stroke}%
\end{pgfscope}%
\begin{pgfscope}%
\pgfpathrectangle{\pgfqpoint{0.100000in}{0.212622in}}{\pgfqpoint{3.696000in}{3.696000in}}%
\pgfusepath{clip}%
\pgfsetrectcap%
\pgfsetroundjoin%
\pgfsetlinewidth{1.505625pt}%
\definecolor{currentstroke}{rgb}{1.000000,0.000000,0.000000}%
\pgfsetstrokecolor{currentstroke}%
\pgfsetdash{}{0pt}%
\pgfpathmoveto{\pgfqpoint{2.304034in}{2.368374in}}%
\pgfpathlineto{\pgfqpoint{2.317947in}{2.304487in}}%
\pgfusepath{stroke}%
\end{pgfscope}%
\begin{pgfscope}%
\pgfpathrectangle{\pgfqpoint{0.100000in}{0.212622in}}{\pgfqpoint{3.696000in}{3.696000in}}%
\pgfusepath{clip}%
\pgfsetrectcap%
\pgfsetroundjoin%
\pgfsetlinewidth{1.505625pt}%
\definecolor{currentstroke}{rgb}{1.000000,0.000000,0.000000}%
\pgfsetstrokecolor{currentstroke}%
\pgfsetdash{}{0pt}%
\pgfpathmoveto{\pgfqpoint{2.304575in}{2.372985in}}%
\pgfpathlineto{\pgfqpoint{2.322387in}{2.308361in}}%
\pgfusepath{stroke}%
\end{pgfscope}%
\begin{pgfscope}%
\pgfpathrectangle{\pgfqpoint{0.100000in}{0.212622in}}{\pgfqpoint{3.696000in}{3.696000in}}%
\pgfusepath{clip}%
\pgfsetrectcap%
\pgfsetroundjoin%
\pgfsetlinewidth{1.505625pt}%
\definecolor{currentstroke}{rgb}{1.000000,0.000000,0.000000}%
\pgfsetstrokecolor{currentstroke}%
\pgfsetdash{}{0pt}%
\pgfpathmoveto{\pgfqpoint{2.308197in}{2.378686in}}%
\pgfpathlineto{\pgfqpoint{2.322387in}{2.308361in}}%
\pgfusepath{stroke}%
\end{pgfscope}%
\begin{pgfscope}%
\pgfpathrectangle{\pgfqpoint{0.100000in}{0.212622in}}{\pgfqpoint{3.696000in}{3.696000in}}%
\pgfusepath{clip}%
\pgfsetrectcap%
\pgfsetroundjoin%
\pgfsetlinewidth{1.505625pt}%
\definecolor{currentstroke}{rgb}{1.000000,0.000000,0.000000}%
\pgfsetstrokecolor{currentstroke}%
\pgfsetdash{}{0pt}%
\pgfpathmoveto{\pgfqpoint{2.310292in}{2.381316in}}%
\pgfpathlineto{\pgfqpoint{2.326824in}{2.312232in}}%
\pgfusepath{stroke}%
\end{pgfscope}%
\begin{pgfscope}%
\pgfpathrectangle{\pgfqpoint{0.100000in}{0.212622in}}{\pgfqpoint{3.696000in}{3.696000in}}%
\pgfusepath{clip}%
\pgfsetrectcap%
\pgfsetroundjoin%
\pgfsetlinewidth{1.505625pt}%
\definecolor{currentstroke}{rgb}{1.000000,0.000000,0.000000}%
\pgfsetstrokecolor{currentstroke}%
\pgfsetdash{}{0pt}%
\pgfpathmoveto{\pgfqpoint{2.312395in}{2.387717in}}%
\pgfpathlineto{\pgfqpoint{2.326824in}{2.312232in}}%
\pgfusepath{stroke}%
\end{pgfscope}%
\begin{pgfscope}%
\pgfpathrectangle{\pgfqpoint{0.100000in}{0.212622in}}{\pgfqpoint{3.696000in}{3.696000in}}%
\pgfusepath{clip}%
\pgfsetrectcap%
\pgfsetroundjoin%
\pgfsetlinewidth{1.505625pt}%
\definecolor{currentstroke}{rgb}{1.000000,0.000000,0.000000}%
\pgfsetstrokecolor{currentstroke}%
\pgfsetdash{}{0pt}%
\pgfpathmoveto{\pgfqpoint{2.313345in}{2.391544in}}%
\pgfpathlineto{\pgfqpoint{2.331258in}{2.316100in}}%
\pgfusepath{stroke}%
\end{pgfscope}%
\begin{pgfscope}%
\pgfpathrectangle{\pgfqpoint{0.100000in}{0.212622in}}{\pgfqpoint{3.696000in}{3.696000in}}%
\pgfusepath{clip}%
\pgfsetrectcap%
\pgfsetroundjoin%
\pgfsetlinewidth{1.505625pt}%
\definecolor{currentstroke}{rgb}{1.000000,0.000000,0.000000}%
\pgfsetstrokecolor{currentstroke}%
\pgfsetdash{}{0pt}%
\pgfpathmoveto{\pgfqpoint{2.316305in}{2.396850in}}%
\pgfpathlineto{\pgfqpoint{2.331258in}{2.316100in}}%
\pgfusepath{stroke}%
\end{pgfscope}%
\begin{pgfscope}%
\pgfpathrectangle{\pgfqpoint{0.100000in}{0.212622in}}{\pgfqpoint{3.696000in}{3.696000in}}%
\pgfusepath{clip}%
\pgfsetrectcap%
\pgfsetroundjoin%
\pgfsetlinewidth{1.505625pt}%
\definecolor{currentstroke}{rgb}{1.000000,0.000000,0.000000}%
\pgfsetstrokecolor{currentstroke}%
\pgfsetdash{}{0pt}%
\pgfpathmoveto{\pgfqpoint{2.318208in}{2.399389in}}%
\pgfpathlineto{\pgfqpoint{2.335689in}{2.319966in}}%
\pgfusepath{stroke}%
\end{pgfscope}%
\begin{pgfscope}%
\pgfpathrectangle{\pgfqpoint{0.100000in}{0.212622in}}{\pgfqpoint{3.696000in}{3.696000in}}%
\pgfusepath{clip}%
\pgfsetrectcap%
\pgfsetroundjoin%
\pgfsetlinewidth{1.505625pt}%
\definecolor{currentstroke}{rgb}{1.000000,0.000000,0.000000}%
\pgfsetstrokecolor{currentstroke}%
\pgfsetdash{}{0pt}%
\pgfpathmoveto{\pgfqpoint{2.319617in}{2.406155in}}%
\pgfpathlineto{\pgfqpoint{2.335689in}{2.319966in}}%
\pgfusepath{stroke}%
\end{pgfscope}%
\begin{pgfscope}%
\pgfpathrectangle{\pgfqpoint{0.100000in}{0.212622in}}{\pgfqpoint{3.696000in}{3.696000in}}%
\pgfusepath{clip}%
\pgfsetrectcap%
\pgfsetroundjoin%
\pgfsetlinewidth{1.505625pt}%
\definecolor{currentstroke}{rgb}{1.000000,0.000000,0.000000}%
\pgfsetstrokecolor{currentstroke}%
\pgfsetdash{}{0pt}%
\pgfpathmoveto{\pgfqpoint{2.320784in}{2.410089in}}%
\pgfpathlineto{\pgfqpoint{2.340117in}{2.323829in}}%
\pgfusepath{stroke}%
\end{pgfscope}%
\begin{pgfscope}%
\pgfpathrectangle{\pgfqpoint{0.100000in}{0.212622in}}{\pgfqpoint{3.696000in}{3.696000in}}%
\pgfusepath{clip}%
\pgfsetrectcap%
\pgfsetroundjoin%
\pgfsetlinewidth{1.505625pt}%
\definecolor{currentstroke}{rgb}{1.000000,0.000000,0.000000}%
\pgfsetstrokecolor{currentstroke}%
\pgfsetdash{}{0pt}%
\pgfpathmoveto{\pgfqpoint{2.323855in}{2.414856in}}%
\pgfpathlineto{\pgfqpoint{2.340117in}{2.323829in}}%
\pgfusepath{stroke}%
\end{pgfscope}%
\begin{pgfscope}%
\pgfpathrectangle{\pgfqpoint{0.100000in}{0.212622in}}{\pgfqpoint{3.696000in}{3.696000in}}%
\pgfusepath{clip}%
\pgfsetrectcap%
\pgfsetroundjoin%
\pgfsetlinewidth{1.505625pt}%
\definecolor{currentstroke}{rgb}{1.000000,0.000000,0.000000}%
\pgfsetstrokecolor{currentstroke}%
\pgfsetdash{}{0pt}%
\pgfpathmoveto{\pgfqpoint{2.327274in}{2.419252in}}%
\pgfpathlineto{\pgfqpoint{2.344542in}{2.327690in}}%
\pgfusepath{stroke}%
\end{pgfscope}%
\begin{pgfscope}%
\pgfpathrectangle{\pgfqpoint{0.100000in}{0.212622in}}{\pgfqpoint{3.696000in}{3.696000in}}%
\pgfusepath{clip}%
\pgfsetrectcap%
\pgfsetroundjoin%
\pgfsetlinewidth{1.505625pt}%
\definecolor{currentstroke}{rgb}{1.000000,0.000000,0.000000}%
\pgfsetstrokecolor{currentstroke}%
\pgfsetdash{}{0pt}%
\pgfpathmoveto{\pgfqpoint{2.329163in}{2.428298in}}%
\pgfpathlineto{\pgfqpoint{2.344542in}{2.327690in}}%
\pgfusepath{stroke}%
\end{pgfscope}%
\begin{pgfscope}%
\pgfpathrectangle{\pgfqpoint{0.100000in}{0.212622in}}{\pgfqpoint{3.696000in}{3.696000in}}%
\pgfusepath{clip}%
\pgfsetrectcap%
\pgfsetroundjoin%
\pgfsetlinewidth{1.505625pt}%
\definecolor{currentstroke}{rgb}{1.000000,0.000000,0.000000}%
\pgfsetstrokecolor{currentstroke}%
\pgfsetdash{}{0pt}%
\pgfpathmoveto{\pgfqpoint{2.330863in}{2.432449in}}%
\pgfpathlineto{\pgfqpoint{2.348964in}{2.331547in}}%
\pgfusepath{stroke}%
\end{pgfscope}%
\begin{pgfscope}%
\pgfpathrectangle{\pgfqpoint{0.100000in}{0.212622in}}{\pgfqpoint{3.696000in}{3.696000in}}%
\pgfusepath{clip}%
\pgfsetrectcap%
\pgfsetroundjoin%
\pgfsetlinewidth{1.505625pt}%
\definecolor{currentstroke}{rgb}{1.000000,0.000000,0.000000}%
\pgfsetstrokecolor{currentstroke}%
\pgfsetdash{}{0pt}%
\pgfpathmoveto{\pgfqpoint{2.334193in}{2.437138in}}%
\pgfpathlineto{\pgfqpoint{2.348964in}{2.331547in}}%
\pgfusepath{stroke}%
\end{pgfscope}%
\begin{pgfscope}%
\pgfpathrectangle{\pgfqpoint{0.100000in}{0.212622in}}{\pgfqpoint{3.696000in}{3.696000in}}%
\pgfusepath{clip}%
\pgfsetrectcap%
\pgfsetroundjoin%
\pgfsetlinewidth{1.505625pt}%
\definecolor{currentstroke}{rgb}{1.000000,0.000000,0.000000}%
\pgfsetstrokecolor{currentstroke}%
\pgfsetdash{}{0pt}%
\pgfpathmoveto{\pgfqpoint{2.335851in}{2.439571in}}%
\pgfpathlineto{\pgfqpoint{2.353383in}{2.335403in}}%
\pgfusepath{stroke}%
\end{pgfscope}%
\begin{pgfscope}%
\pgfpathrectangle{\pgfqpoint{0.100000in}{0.212622in}}{\pgfqpoint{3.696000in}{3.696000in}}%
\pgfusepath{clip}%
\pgfsetrectcap%
\pgfsetroundjoin%
\pgfsetlinewidth{1.505625pt}%
\definecolor{currentstroke}{rgb}{1.000000,0.000000,0.000000}%
\pgfsetstrokecolor{currentstroke}%
\pgfsetdash{}{0pt}%
\pgfpathmoveto{\pgfqpoint{2.336690in}{2.445157in}}%
\pgfpathlineto{\pgfqpoint{2.353383in}{2.335403in}}%
\pgfusepath{stroke}%
\end{pgfscope}%
\begin{pgfscope}%
\pgfpathrectangle{\pgfqpoint{0.100000in}{0.212622in}}{\pgfqpoint{3.696000in}{3.696000in}}%
\pgfusepath{clip}%
\pgfsetrectcap%
\pgfsetroundjoin%
\pgfsetlinewidth{1.505625pt}%
\definecolor{currentstroke}{rgb}{1.000000,0.000000,0.000000}%
\pgfsetstrokecolor{currentstroke}%
\pgfsetdash{}{0pt}%
\pgfpathmoveto{\pgfqpoint{2.339132in}{2.450762in}}%
\pgfpathlineto{\pgfqpoint{2.357799in}{2.339255in}}%
\pgfusepath{stroke}%
\end{pgfscope}%
\begin{pgfscope}%
\pgfpathrectangle{\pgfqpoint{0.100000in}{0.212622in}}{\pgfqpoint{3.696000in}{3.696000in}}%
\pgfusepath{clip}%
\pgfsetrectcap%
\pgfsetroundjoin%
\pgfsetlinewidth{1.505625pt}%
\definecolor{currentstroke}{rgb}{1.000000,0.000000,0.000000}%
\pgfsetstrokecolor{currentstroke}%
\pgfsetdash{}{0pt}%
\pgfpathmoveto{\pgfqpoint{2.343344in}{2.456238in}}%
\pgfpathlineto{\pgfqpoint{2.357799in}{2.339255in}}%
\pgfusepath{stroke}%
\end{pgfscope}%
\begin{pgfscope}%
\pgfpathrectangle{\pgfqpoint{0.100000in}{0.212622in}}{\pgfqpoint{3.696000in}{3.696000in}}%
\pgfusepath{clip}%
\pgfsetrectcap%
\pgfsetroundjoin%
\pgfsetlinewidth{1.505625pt}%
\definecolor{currentstroke}{rgb}{1.000000,0.000000,0.000000}%
\pgfsetstrokecolor{currentstroke}%
\pgfsetdash{}{0pt}%
\pgfpathmoveto{\pgfqpoint{2.347098in}{2.462294in}}%
\pgfpathlineto{\pgfqpoint{2.362212in}{2.343105in}}%
\pgfusepath{stroke}%
\end{pgfscope}%
\begin{pgfscope}%
\pgfpathrectangle{\pgfqpoint{0.100000in}{0.212622in}}{\pgfqpoint{3.696000in}{3.696000in}}%
\pgfusepath{clip}%
\pgfsetrectcap%
\pgfsetroundjoin%
\pgfsetlinewidth{1.505625pt}%
\definecolor{currentstroke}{rgb}{1.000000,0.000000,0.000000}%
\pgfsetstrokecolor{currentstroke}%
\pgfsetdash{}{0pt}%
\pgfpathmoveto{\pgfqpoint{2.348861in}{2.472252in}}%
\pgfpathlineto{\pgfqpoint{2.366622in}{2.346952in}}%
\pgfusepath{stroke}%
\end{pgfscope}%
\begin{pgfscope}%
\pgfpathrectangle{\pgfqpoint{0.100000in}{0.212622in}}{\pgfqpoint{3.696000in}{3.696000in}}%
\pgfusepath{clip}%
\pgfsetrectcap%
\pgfsetroundjoin%
\pgfsetlinewidth{1.505625pt}%
\definecolor{currentstroke}{rgb}{1.000000,0.000000,0.000000}%
\pgfsetstrokecolor{currentstroke}%
\pgfsetdash{}{0pt}%
\pgfpathmoveto{\pgfqpoint{2.351080in}{2.476869in}}%
\pgfpathlineto{\pgfqpoint{2.371029in}{2.350797in}}%
\pgfusepath{stroke}%
\end{pgfscope}%
\begin{pgfscope}%
\pgfpathrectangle{\pgfqpoint{0.100000in}{0.212622in}}{\pgfqpoint{3.696000in}{3.696000in}}%
\pgfusepath{clip}%
\pgfsetrectcap%
\pgfsetroundjoin%
\pgfsetlinewidth{1.505625pt}%
\definecolor{currentstroke}{rgb}{1.000000,0.000000,0.000000}%
\pgfsetstrokecolor{currentstroke}%
\pgfsetdash{}{0pt}%
\pgfpathmoveto{\pgfqpoint{2.354620in}{2.481822in}}%
\pgfpathlineto{\pgfqpoint{2.371029in}{2.350797in}}%
\pgfusepath{stroke}%
\end{pgfscope}%
\begin{pgfscope}%
\pgfpathrectangle{\pgfqpoint{0.100000in}{0.212622in}}{\pgfqpoint{3.696000in}{3.696000in}}%
\pgfusepath{clip}%
\pgfsetrectcap%
\pgfsetroundjoin%
\pgfsetlinewidth{1.505625pt}%
\definecolor{currentstroke}{rgb}{1.000000,0.000000,0.000000}%
\pgfsetstrokecolor{currentstroke}%
\pgfsetdash{}{0pt}%
\pgfpathmoveto{\pgfqpoint{2.356363in}{2.484681in}}%
\pgfpathlineto{\pgfqpoint{2.371029in}{2.350797in}}%
\pgfusepath{stroke}%
\end{pgfscope}%
\begin{pgfscope}%
\pgfpathrectangle{\pgfqpoint{0.100000in}{0.212622in}}{\pgfqpoint{3.696000in}{3.696000in}}%
\pgfusepath{clip}%
\pgfsetrectcap%
\pgfsetroundjoin%
\pgfsetlinewidth{1.505625pt}%
\definecolor{currentstroke}{rgb}{1.000000,0.000000,0.000000}%
\pgfsetstrokecolor{currentstroke}%
\pgfsetdash{}{0pt}%
\pgfpathmoveto{\pgfqpoint{2.357460in}{2.490397in}}%
\pgfpathlineto{\pgfqpoint{2.375433in}{2.354639in}}%
\pgfusepath{stroke}%
\end{pgfscope}%
\begin{pgfscope}%
\pgfpathrectangle{\pgfqpoint{0.100000in}{0.212622in}}{\pgfqpoint{3.696000in}{3.696000in}}%
\pgfusepath{clip}%
\pgfsetrectcap%
\pgfsetroundjoin%
\pgfsetlinewidth{1.505625pt}%
\definecolor{currentstroke}{rgb}{1.000000,0.000000,0.000000}%
\pgfsetstrokecolor{currentstroke}%
\pgfsetdash{}{0pt}%
\pgfpathmoveto{\pgfqpoint{2.359593in}{2.495894in}}%
\pgfpathlineto{\pgfqpoint{2.379834in}{2.358479in}}%
\pgfusepath{stroke}%
\end{pgfscope}%
\begin{pgfscope}%
\pgfpathrectangle{\pgfqpoint{0.100000in}{0.212622in}}{\pgfqpoint{3.696000in}{3.696000in}}%
\pgfusepath{clip}%
\pgfsetrectcap%
\pgfsetroundjoin%
\pgfsetlinewidth{1.505625pt}%
\definecolor{currentstroke}{rgb}{1.000000,0.000000,0.000000}%
\pgfsetstrokecolor{currentstroke}%
\pgfsetdash{}{0pt}%
\pgfpathmoveto{\pgfqpoint{2.363687in}{2.501686in}}%
\pgfpathlineto{\pgfqpoint{2.379834in}{2.358479in}}%
\pgfusepath{stroke}%
\end{pgfscope}%
\begin{pgfscope}%
\pgfpathrectangle{\pgfqpoint{0.100000in}{0.212622in}}{\pgfqpoint{3.696000in}{3.696000in}}%
\pgfusepath{clip}%
\pgfsetrectcap%
\pgfsetroundjoin%
\pgfsetlinewidth{1.505625pt}%
\definecolor{currentstroke}{rgb}{1.000000,0.000000,0.000000}%
\pgfsetstrokecolor{currentstroke}%
\pgfsetdash{}{0pt}%
\pgfpathmoveto{\pgfqpoint{2.365879in}{2.504586in}}%
\pgfpathlineto{\pgfqpoint{2.384232in}{2.362315in}}%
\pgfusepath{stroke}%
\end{pgfscope}%
\begin{pgfscope}%
\pgfpathrectangle{\pgfqpoint{0.100000in}{0.212622in}}{\pgfqpoint{3.696000in}{3.696000in}}%
\pgfusepath{clip}%
\pgfsetrectcap%
\pgfsetroundjoin%
\pgfsetlinewidth{1.505625pt}%
\definecolor{currentstroke}{rgb}{1.000000,0.000000,0.000000}%
\pgfsetstrokecolor{currentstroke}%
\pgfsetdash{}{0pt}%
\pgfpathmoveto{\pgfqpoint{2.367745in}{2.511495in}}%
\pgfpathlineto{\pgfqpoint{2.384232in}{2.362315in}}%
\pgfusepath{stroke}%
\end{pgfscope}%
\begin{pgfscope}%
\pgfpathrectangle{\pgfqpoint{0.100000in}{0.212622in}}{\pgfqpoint{3.696000in}{3.696000in}}%
\pgfusepath{clip}%
\pgfsetrectcap%
\pgfsetroundjoin%
\pgfsetlinewidth{1.505625pt}%
\definecolor{currentstroke}{rgb}{1.000000,0.000000,0.000000}%
\pgfsetstrokecolor{currentstroke}%
\pgfsetdash{}{0pt}%
\pgfpathmoveto{\pgfqpoint{2.369005in}{2.514992in}}%
\pgfpathlineto{\pgfqpoint{2.388626in}{2.366150in}}%
\pgfusepath{stroke}%
\end{pgfscope}%
\begin{pgfscope}%
\pgfpathrectangle{\pgfqpoint{0.100000in}{0.212622in}}{\pgfqpoint{3.696000in}{3.696000in}}%
\pgfusepath{clip}%
\pgfsetrectcap%
\pgfsetroundjoin%
\pgfsetlinewidth{1.505625pt}%
\definecolor{currentstroke}{rgb}{1.000000,0.000000,0.000000}%
\pgfsetstrokecolor{currentstroke}%
\pgfsetdash{}{0pt}%
\pgfpathmoveto{\pgfqpoint{2.371626in}{2.518838in}}%
\pgfpathlineto{\pgfqpoint{2.388626in}{2.366150in}}%
\pgfusepath{stroke}%
\end{pgfscope}%
\begin{pgfscope}%
\pgfpathrectangle{\pgfqpoint{0.100000in}{0.212622in}}{\pgfqpoint{3.696000in}{3.696000in}}%
\pgfusepath{clip}%
\pgfsetrectcap%
\pgfsetroundjoin%
\pgfsetlinewidth{1.505625pt}%
\definecolor{currentstroke}{rgb}{1.000000,0.000000,0.000000}%
\pgfsetstrokecolor{currentstroke}%
\pgfsetdash{}{0pt}%
\pgfpathmoveto{\pgfqpoint{2.373079in}{2.520842in}}%
\pgfpathlineto{\pgfqpoint{2.388626in}{2.366150in}}%
\pgfusepath{stroke}%
\end{pgfscope}%
\begin{pgfscope}%
\pgfpathrectangle{\pgfqpoint{0.100000in}{0.212622in}}{\pgfqpoint{3.696000in}{3.696000in}}%
\pgfusepath{clip}%
\pgfsetrectcap%
\pgfsetroundjoin%
\pgfsetlinewidth{1.505625pt}%
\definecolor{currentstroke}{rgb}{1.000000,0.000000,0.000000}%
\pgfsetstrokecolor{currentstroke}%
\pgfsetdash{}{0pt}%
\pgfpathmoveto{\pgfqpoint{2.374363in}{2.525890in}}%
\pgfpathlineto{\pgfqpoint{2.393018in}{2.369981in}}%
\pgfusepath{stroke}%
\end{pgfscope}%
\begin{pgfscope}%
\pgfpathrectangle{\pgfqpoint{0.100000in}{0.212622in}}{\pgfqpoint{3.696000in}{3.696000in}}%
\pgfusepath{clip}%
\pgfsetrectcap%
\pgfsetroundjoin%
\pgfsetlinewidth{1.505625pt}%
\definecolor{currentstroke}{rgb}{1.000000,0.000000,0.000000}%
\pgfsetstrokecolor{currentstroke}%
\pgfsetdash{}{0pt}%
\pgfpathmoveto{\pgfqpoint{2.375365in}{2.528418in}}%
\pgfpathlineto{\pgfqpoint{2.393018in}{2.369981in}}%
\pgfusepath{stroke}%
\end{pgfscope}%
\begin{pgfscope}%
\pgfpathrectangle{\pgfqpoint{0.100000in}{0.212622in}}{\pgfqpoint{3.696000in}{3.696000in}}%
\pgfusepath{clip}%
\pgfsetrectcap%
\pgfsetroundjoin%
\pgfsetlinewidth{1.505625pt}%
\definecolor{currentstroke}{rgb}{1.000000,0.000000,0.000000}%
\pgfsetstrokecolor{currentstroke}%
\pgfsetdash{}{0pt}%
\pgfpathmoveto{\pgfqpoint{2.377383in}{2.531285in}}%
\pgfpathlineto{\pgfqpoint{2.393018in}{2.369981in}}%
\pgfusepath{stroke}%
\end{pgfscope}%
\begin{pgfscope}%
\pgfpathrectangle{\pgfqpoint{0.100000in}{0.212622in}}{\pgfqpoint{3.696000in}{3.696000in}}%
\pgfusepath{clip}%
\pgfsetrectcap%
\pgfsetroundjoin%
\pgfsetlinewidth{1.505625pt}%
\definecolor{currentstroke}{rgb}{1.000000,0.000000,0.000000}%
\pgfsetstrokecolor{currentstroke}%
\pgfsetdash{}{0pt}%
\pgfpathmoveto{\pgfqpoint{2.379963in}{2.535341in}}%
\pgfpathlineto{\pgfqpoint{2.397407in}{2.373810in}}%
\pgfusepath{stroke}%
\end{pgfscope}%
\begin{pgfscope}%
\pgfpathrectangle{\pgfqpoint{0.100000in}{0.212622in}}{\pgfqpoint{3.696000in}{3.696000in}}%
\pgfusepath{clip}%
\pgfsetrectcap%
\pgfsetroundjoin%
\pgfsetlinewidth{1.505625pt}%
\definecolor{currentstroke}{rgb}{1.000000,0.000000,0.000000}%
\pgfsetstrokecolor{currentstroke}%
\pgfsetdash{}{0pt}%
\pgfpathmoveto{\pgfqpoint{2.383269in}{2.540533in}}%
\pgfpathlineto{\pgfqpoint{2.397407in}{2.373810in}}%
\pgfusepath{stroke}%
\end{pgfscope}%
\begin{pgfscope}%
\pgfpathrectangle{\pgfqpoint{0.100000in}{0.212622in}}{\pgfqpoint{3.696000in}{3.696000in}}%
\pgfusepath{clip}%
\pgfsetrectcap%
\pgfsetroundjoin%
\pgfsetlinewidth{1.505625pt}%
\definecolor{currentstroke}{rgb}{1.000000,0.000000,0.000000}%
\pgfsetstrokecolor{currentstroke}%
\pgfsetdash{}{0pt}%
\pgfpathmoveto{\pgfqpoint{2.385351in}{2.549930in}}%
\pgfpathlineto{\pgfqpoint{2.401793in}{2.377637in}}%
\pgfusepath{stroke}%
\end{pgfscope}%
\begin{pgfscope}%
\pgfpathrectangle{\pgfqpoint{0.100000in}{0.212622in}}{\pgfqpoint{3.696000in}{3.696000in}}%
\pgfusepath{clip}%
\pgfsetrectcap%
\pgfsetroundjoin%
\pgfsetlinewidth{1.505625pt}%
\definecolor{currentstroke}{rgb}{1.000000,0.000000,0.000000}%
\pgfsetstrokecolor{currentstroke}%
\pgfsetdash{}{0pt}%
\pgfpathmoveto{\pgfqpoint{2.386975in}{2.554756in}}%
\pgfpathlineto{\pgfqpoint{2.406176in}{2.381461in}}%
\pgfusepath{stroke}%
\end{pgfscope}%
\begin{pgfscope}%
\pgfpathrectangle{\pgfqpoint{0.100000in}{0.212622in}}{\pgfqpoint{3.696000in}{3.696000in}}%
\pgfusepath{clip}%
\pgfsetrectcap%
\pgfsetroundjoin%
\pgfsetlinewidth{1.505625pt}%
\definecolor{currentstroke}{rgb}{1.000000,0.000000,0.000000}%
\pgfsetstrokecolor{currentstroke}%
\pgfsetdash{}{0pt}%
\pgfpathmoveto{\pgfqpoint{2.390709in}{2.560310in}}%
\pgfpathlineto{\pgfqpoint{2.406176in}{2.381461in}}%
\pgfusepath{stroke}%
\end{pgfscope}%
\begin{pgfscope}%
\pgfpathrectangle{\pgfqpoint{0.100000in}{0.212622in}}{\pgfqpoint{3.696000in}{3.696000in}}%
\pgfusepath{clip}%
\pgfsetrectcap%
\pgfsetroundjoin%
\pgfsetlinewidth{1.505625pt}%
\definecolor{currentstroke}{rgb}{1.000000,0.000000,0.000000}%
\pgfsetstrokecolor{currentstroke}%
\pgfsetdash{}{0pt}%
\pgfpathmoveto{\pgfqpoint{2.392671in}{2.563376in}}%
\pgfpathlineto{\pgfqpoint{2.410556in}{2.385282in}}%
\pgfusepath{stroke}%
\end{pgfscope}%
\begin{pgfscope}%
\pgfpathrectangle{\pgfqpoint{0.100000in}{0.212622in}}{\pgfqpoint{3.696000in}{3.696000in}}%
\pgfusepath{clip}%
\pgfsetrectcap%
\pgfsetroundjoin%
\pgfsetlinewidth{1.505625pt}%
\definecolor{currentstroke}{rgb}{1.000000,0.000000,0.000000}%
\pgfsetstrokecolor{currentstroke}%
\pgfsetdash{}{0pt}%
\pgfpathmoveto{\pgfqpoint{2.393844in}{2.569122in}}%
\pgfpathlineto{\pgfqpoint{2.410556in}{2.385282in}}%
\pgfusepath{stroke}%
\end{pgfscope}%
\begin{pgfscope}%
\pgfpathrectangle{\pgfqpoint{0.100000in}{0.212622in}}{\pgfqpoint{3.696000in}{3.696000in}}%
\pgfusepath{clip}%
\pgfsetrectcap%
\pgfsetroundjoin%
\pgfsetlinewidth{1.505625pt}%
\definecolor{currentstroke}{rgb}{1.000000,0.000000,0.000000}%
\pgfsetstrokecolor{currentstroke}%
\pgfsetdash{}{0pt}%
\pgfpathmoveto{\pgfqpoint{2.395965in}{2.574916in}}%
\pgfpathlineto{\pgfqpoint{2.414933in}{2.389101in}}%
\pgfusepath{stroke}%
\end{pgfscope}%
\begin{pgfscope}%
\pgfpathrectangle{\pgfqpoint{0.100000in}{0.212622in}}{\pgfqpoint{3.696000in}{3.696000in}}%
\pgfusepath{clip}%
\pgfsetrectcap%
\pgfsetroundjoin%
\pgfsetlinewidth{1.505625pt}%
\definecolor{currentstroke}{rgb}{1.000000,0.000000,0.000000}%
\pgfsetstrokecolor{currentstroke}%
\pgfsetdash{}{0pt}%
\pgfpathmoveto{\pgfqpoint{2.400597in}{2.581726in}}%
\pgfpathlineto{\pgfqpoint{2.419308in}{2.392917in}}%
\pgfusepath{stroke}%
\end{pgfscope}%
\begin{pgfscope}%
\pgfpathrectangle{\pgfqpoint{0.100000in}{0.212622in}}{\pgfqpoint{3.696000in}{3.696000in}}%
\pgfusepath{clip}%
\pgfsetrectcap%
\pgfsetroundjoin%
\pgfsetlinewidth{1.505625pt}%
\definecolor{currentstroke}{rgb}{1.000000,0.000000,0.000000}%
\pgfsetstrokecolor{currentstroke}%
\pgfsetdash{}{0pt}%
\pgfpathmoveto{\pgfqpoint{2.405035in}{2.589322in}}%
\pgfpathlineto{\pgfqpoint{2.419308in}{2.392917in}}%
\pgfusepath{stroke}%
\end{pgfscope}%
\begin{pgfscope}%
\pgfpathrectangle{\pgfqpoint{0.100000in}{0.212622in}}{\pgfqpoint{3.696000in}{3.696000in}}%
\pgfusepath{clip}%
\pgfsetrectcap%
\pgfsetroundjoin%
\pgfsetlinewidth{1.505625pt}%
\definecolor{currentstroke}{rgb}{1.000000,0.000000,0.000000}%
\pgfsetstrokecolor{currentstroke}%
\pgfsetdash{}{0pt}%
\pgfpathmoveto{\pgfqpoint{2.406337in}{2.600043in}}%
\pgfpathlineto{\pgfqpoint{2.423679in}{2.396730in}}%
\pgfusepath{stroke}%
\end{pgfscope}%
\begin{pgfscope}%
\pgfpathrectangle{\pgfqpoint{0.100000in}{0.212622in}}{\pgfqpoint{3.696000in}{3.696000in}}%
\pgfusepath{clip}%
\pgfsetrectcap%
\pgfsetroundjoin%
\pgfsetlinewidth{1.505625pt}%
\definecolor{currentstroke}{rgb}{1.000000,0.000000,0.000000}%
\pgfsetstrokecolor{currentstroke}%
\pgfsetdash{}{0pt}%
\pgfpathmoveto{\pgfqpoint{2.410543in}{2.610537in}}%
\pgfpathlineto{\pgfqpoint{2.432412in}{2.404349in}}%
\pgfusepath{stroke}%
\end{pgfscope}%
\begin{pgfscope}%
\pgfpathrectangle{\pgfqpoint{0.100000in}{0.212622in}}{\pgfqpoint{3.696000in}{3.696000in}}%
\pgfusepath{clip}%
\pgfsetrectcap%
\pgfsetroundjoin%
\pgfsetlinewidth{1.505625pt}%
\definecolor{currentstroke}{rgb}{1.000000,0.000000,0.000000}%
\pgfsetstrokecolor{currentstroke}%
\pgfsetdash{}{0pt}%
\pgfpathmoveto{\pgfqpoint{2.417771in}{2.621416in}}%
\pgfpathlineto{\pgfqpoint{2.436774in}{2.408155in}}%
\pgfusepath{stroke}%
\end{pgfscope}%
\begin{pgfscope}%
\pgfpathrectangle{\pgfqpoint{0.100000in}{0.212622in}}{\pgfqpoint{3.696000in}{3.696000in}}%
\pgfusepath{clip}%
\pgfsetrectcap%
\pgfsetroundjoin%
\pgfsetlinewidth{1.505625pt}%
\definecolor{currentstroke}{rgb}{1.000000,0.000000,0.000000}%
\pgfsetstrokecolor{currentstroke}%
\pgfsetdash{}{0pt}%
\pgfpathmoveto{\pgfqpoint{2.421523in}{2.627799in}}%
\pgfpathlineto{\pgfqpoint{2.436774in}{2.408155in}}%
\pgfusepath{stroke}%
\end{pgfscope}%
\begin{pgfscope}%
\pgfpathrectangle{\pgfqpoint{0.100000in}{0.212622in}}{\pgfqpoint{3.696000in}{3.696000in}}%
\pgfusepath{clip}%
\pgfsetrectcap%
\pgfsetroundjoin%
\pgfsetlinewidth{1.505625pt}%
\definecolor{currentstroke}{rgb}{1.000000,0.000000,0.000000}%
\pgfsetstrokecolor{currentstroke}%
\pgfsetdash{}{0pt}%
\pgfpathmoveto{\pgfqpoint{2.424105in}{2.637037in}}%
\pgfpathlineto{\pgfqpoint{2.441134in}{2.411958in}}%
\pgfusepath{stroke}%
\end{pgfscope}%
\begin{pgfscope}%
\pgfpathrectangle{\pgfqpoint{0.100000in}{0.212622in}}{\pgfqpoint{3.696000in}{3.696000in}}%
\pgfusepath{clip}%
\pgfsetrectcap%
\pgfsetroundjoin%
\pgfsetlinewidth{1.505625pt}%
\definecolor{currentstroke}{rgb}{1.000000,0.000000,0.000000}%
\pgfsetstrokecolor{currentstroke}%
\pgfsetdash{}{0pt}%
\pgfpathmoveto{\pgfqpoint{2.425755in}{2.641724in}}%
\pgfpathlineto{\pgfqpoint{2.445490in}{2.415759in}}%
\pgfusepath{stroke}%
\end{pgfscope}%
\begin{pgfscope}%
\pgfpathrectangle{\pgfqpoint{0.100000in}{0.212622in}}{\pgfqpoint{3.696000in}{3.696000in}}%
\pgfusepath{clip}%
\pgfsetrectcap%
\pgfsetroundjoin%
\pgfsetlinewidth{1.505625pt}%
\definecolor{currentstroke}{rgb}{1.000000,0.000000,0.000000}%
\pgfsetstrokecolor{currentstroke}%
\pgfsetdash{}{0pt}%
\pgfpathmoveto{\pgfqpoint{2.429910in}{2.648061in}}%
\pgfpathlineto{\pgfqpoint{2.445490in}{2.415759in}}%
\pgfusepath{stroke}%
\end{pgfscope}%
\begin{pgfscope}%
\pgfpathrectangle{\pgfqpoint{0.100000in}{0.212622in}}{\pgfqpoint{3.696000in}{3.696000in}}%
\pgfusepath{clip}%
\pgfsetrectcap%
\pgfsetroundjoin%
\pgfsetlinewidth{1.505625pt}%
\definecolor{currentstroke}{rgb}{1.000000,0.000000,0.000000}%
\pgfsetstrokecolor{currentstroke}%
\pgfsetdash{}{0pt}%
\pgfpathmoveto{\pgfqpoint{2.432167in}{2.651240in}}%
\pgfpathlineto{\pgfqpoint{2.449843in}{2.419557in}}%
\pgfusepath{stroke}%
\end{pgfscope}%
\begin{pgfscope}%
\pgfpathrectangle{\pgfqpoint{0.100000in}{0.212622in}}{\pgfqpoint{3.696000in}{3.696000in}}%
\pgfusepath{clip}%
\pgfsetrectcap%
\pgfsetroundjoin%
\pgfsetlinewidth{1.505625pt}%
\definecolor{currentstroke}{rgb}{1.000000,0.000000,0.000000}%
\pgfsetstrokecolor{currentstroke}%
\pgfsetdash{}{0pt}%
\pgfpathmoveto{\pgfqpoint{2.433800in}{2.658511in}}%
\pgfpathlineto{\pgfqpoint{2.449843in}{2.419557in}}%
\pgfusepath{stroke}%
\end{pgfscope}%
\begin{pgfscope}%
\pgfpathrectangle{\pgfqpoint{0.100000in}{0.212622in}}{\pgfqpoint{3.696000in}{3.696000in}}%
\pgfusepath{clip}%
\pgfsetrectcap%
\pgfsetroundjoin%
\pgfsetlinewidth{1.505625pt}%
\definecolor{currentstroke}{rgb}{1.000000,0.000000,0.000000}%
\pgfsetstrokecolor{currentstroke}%
\pgfsetdash{}{0pt}%
\pgfpathmoveto{\pgfqpoint{2.435058in}{2.662431in}}%
\pgfpathlineto{\pgfqpoint{2.454194in}{2.423352in}}%
\pgfusepath{stroke}%
\end{pgfscope}%
\begin{pgfscope}%
\pgfpathrectangle{\pgfqpoint{0.100000in}{0.212622in}}{\pgfqpoint{3.696000in}{3.696000in}}%
\pgfusepath{clip}%
\pgfsetrectcap%
\pgfsetroundjoin%
\pgfsetlinewidth{1.505625pt}%
\definecolor{currentstroke}{rgb}{1.000000,0.000000,0.000000}%
\pgfsetstrokecolor{currentstroke}%
\pgfsetdash{}{0pt}%
\pgfpathmoveto{\pgfqpoint{2.437888in}{2.667242in}}%
\pgfpathlineto{\pgfqpoint{2.454194in}{2.423352in}}%
\pgfusepath{stroke}%
\end{pgfscope}%
\begin{pgfscope}%
\pgfpathrectangle{\pgfqpoint{0.100000in}{0.212622in}}{\pgfqpoint{3.696000in}{3.696000in}}%
\pgfusepath{clip}%
\pgfsetrectcap%
\pgfsetroundjoin%
\pgfsetlinewidth{1.505625pt}%
\definecolor{currentstroke}{rgb}{1.000000,0.000000,0.000000}%
\pgfsetstrokecolor{currentstroke}%
\pgfsetdash{}{0pt}%
\pgfpathmoveto{\pgfqpoint{2.441155in}{2.671739in}}%
\pgfpathlineto{\pgfqpoint{2.458541in}{2.427145in}}%
\pgfusepath{stroke}%
\end{pgfscope}%
\begin{pgfscope}%
\pgfpathrectangle{\pgfqpoint{0.100000in}{0.212622in}}{\pgfqpoint{3.696000in}{3.696000in}}%
\pgfusepath{clip}%
\pgfsetrectcap%
\pgfsetroundjoin%
\pgfsetlinewidth{1.505625pt}%
\definecolor{currentstroke}{rgb}{1.000000,0.000000,0.000000}%
\pgfsetstrokecolor{currentstroke}%
\pgfsetdash{}{0pt}%
\pgfpathmoveto{\pgfqpoint{2.444116in}{2.680378in}}%
\pgfpathlineto{\pgfqpoint{2.462886in}{2.430936in}}%
\pgfusepath{stroke}%
\end{pgfscope}%
\begin{pgfscope}%
\pgfpathrectangle{\pgfqpoint{0.100000in}{0.212622in}}{\pgfqpoint{3.696000in}{3.696000in}}%
\pgfusepath{clip}%
\pgfsetrectcap%
\pgfsetroundjoin%
\pgfsetlinewidth{1.505625pt}%
\definecolor{currentstroke}{rgb}{1.000000,0.000000,0.000000}%
\pgfsetstrokecolor{currentstroke}%
\pgfsetdash{}{0pt}%
\pgfpathmoveto{\pgfqpoint{2.445290in}{2.685346in}}%
\pgfpathlineto{\pgfqpoint{2.462886in}{2.430936in}}%
\pgfusepath{stroke}%
\end{pgfscope}%
\begin{pgfscope}%
\pgfpathrectangle{\pgfqpoint{0.100000in}{0.212622in}}{\pgfqpoint{3.696000in}{3.696000in}}%
\pgfusepath{clip}%
\pgfsetrectcap%
\pgfsetroundjoin%
\pgfsetlinewidth{1.505625pt}%
\definecolor{currentstroke}{rgb}{1.000000,0.000000,0.000000}%
\pgfsetstrokecolor{currentstroke}%
\pgfsetdash{}{0pt}%
\pgfpathmoveto{\pgfqpoint{2.449460in}{2.691675in}}%
\pgfpathlineto{\pgfqpoint{2.467228in}{2.434723in}}%
\pgfusepath{stroke}%
\end{pgfscope}%
\begin{pgfscope}%
\pgfpathrectangle{\pgfqpoint{0.100000in}{0.212622in}}{\pgfqpoint{3.696000in}{3.696000in}}%
\pgfusepath{clip}%
\pgfsetrectcap%
\pgfsetroundjoin%
\pgfsetlinewidth{1.505625pt}%
\definecolor{currentstroke}{rgb}{1.000000,0.000000,0.000000}%
\pgfsetstrokecolor{currentstroke}%
\pgfsetdash{}{0pt}%
\pgfpathmoveto{\pgfqpoint{2.454064in}{2.697902in}}%
\pgfpathlineto{\pgfqpoint{2.471566in}{2.438509in}}%
\pgfusepath{stroke}%
\end{pgfscope}%
\begin{pgfscope}%
\pgfpathrectangle{\pgfqpoint{0.100000in}{0.212622in}}{\pgfqpoint{3.696000in}{3.696000in}}%
\pgfusepath{clip}%
\pgfsetrectcap%
\pgfsetroundjoin%
\pgfsetlinewidth{1.505625pt}%
\definecolor{currentstroke}{rgb}{1.000000,0.000000,0.000000}%
\pgfsetstrokecolor{currentstroke}%
\pgfsetdash{}{0pt}%
\pgfpathmoveto{\pgfqpoint{2.458018in}{2.708843in}}%
\pgfpathlineto{\pgfqpoint{2.475902in}{2.442291in}}%
\pgfusepath{stroke}%
\end{pgfscope}%
\begin{pgfscope}%
\pgfpathrectangle{\pgfqpoint{0.100000in}{0.212622in}}{\pgfqpoint{3.696000in}{3.696000in}}%
\pgfusepath{clip}%
\pgfsetrectcap%
\pgfsetroundjoin%
\pgfsetlinewidth{1.505625pt}%
\definecolor{currentstroke}{rgb}{1.000000,0.000000,0.000000}%
\pgfsetstrokecolor{currentstroke}%
\pgfsetdash{}{0pt}%
\pgfpathmoveto{\pgfqpoint{2.459693in}{2.714879in}}%
\pgfpathlineto{\pgfqpoint{2.475902in}{2.442291in}}%
\pgfusepath{stroke}%
\end{pgfscope}%
\begin{pgfscope}%
\pgfpathrectangle{\pgfqpoint{0.100000in}{0.212622in}}{\pgfqpoint{3.696000in}{3.696000in}}%
\pgfusepath{clip}%
\pgfsetrectcap%
\pgfsetroundjoin%
\pgfsetlinewidth{1.505625pt}%
\definecolor{currentstroke}{rgb}{1.000000,0.000000,0.000000}%
\pgfsetstrokecolor{currentstroke}%
\pgfsetdash{}{0pt}%
\pgfpathmoveto{\pgfqpoint{2.464718in}{2.722227in}}%
\pgfpathlineto{\pgfqpoint{2.480235in}{2.446071in}}%
\pgfusepath{stroke}%
\end{pgfscope}%
\begin{pgfscope}%
\pgfpathrectangle{\pgfqpoint{0.100000in}{0.212622in}}{\pgfqpoint{3.696000in}{3.696000in}}%
\pgfusepath{clip}%
\pgfsetrectcap%
\pgfsetroundjoin%
\pgfsetlinewidth{1.505625pt}%
\definecolor{currentstroke}{rgb}{1.000000,0.000000,0.000000}%
\pgfsetstrokecolor{currentstroke}%
\pgfsetdash{}{0pt}%
\pgfpathmoveto{\pgfqpoint{2.469834in}{2.729628in}}%
\pgfpathlineto{\pgfqpoint{2.484565in}{2.449849in}}%
\pgfusepath{stroke}%
\end{pgfscope}%
\begin{pgfscope}%
\pgfpathrectangle{\pgfqpoint{0.100000in}{0.212622in}}{\pgfqpoint{3.696000in}{3.696000in}}%
\pgfusepath{clip}%
\pgfsetrectcap%
\pgfsetroundjoin%
\pgfsetlinewidth{1.505625pt}%
\definecolor{currentstroke}{rgb}{1.000000,0.000000,0.000000}%
\pgfsetstrokecolor{currentstroke}%
\pgfsetdash{}{0pt}%
\pgfpathmoveto{\pgfqpoint{2.473197in}{2.742064in}}%
\pgfpathlineto{\pgfqpoint{2.488892in}{2.453624in}}%
\pgfusepath{stroke}%
\end{pgfscope}%
\begin{pgfscope}%
\pgfpathrectangle{\pgfqpoint{0.100000in}{0.212622in}}{\pgfqpoint{3.696000in}{3.696000in}}%
\pgfusepath{clip}%
\pgfsetrectcap%
\pgfsetroundjoin%
\pgfsetlinewidth{1.505625pt}%
\definecolor{currentstroke}{rgb}{1.000000,0.000000,0.000000}%
\pgfsetstrokecolor{currentstroke}%
\pgfsetdash{}{0pt}%
\pgfpathmoveto{\pgfqpoint{2.475308in}{2.748513in}}%
\pgfpathlineto{\pgfqpoint{2.493216in}{2.457396in}}%
\pgfusepath{stroke}%
\end{pgfscope}%
\begin{pgfscope}%
\pgfpathrectangle{\pgfqpoint{0.100000in}{0.212622in}}{\pgfqpoint{3.696000in}{3.696000in}}%
\pgfusepath{clip}%
\pgfsetrectcap%
\pgfsetroundjoin%
\pgfsetlinewidth{1.505625pt}%
\definecolor{currentstroke}{rgb}{1.000000,0.000000,0.000000}%
\pgfsetstrokecolor{currentstroke}%
\pgfsetdash{}{0pt}%
\pgfpathmoveto{\pgfqpoint{2.479605in}{2.756072in}}%
\pgfpathlineto{\pgfqpoint{2.487827in}{2.478716in}}%
\pgfusepath{stroke}%
\end{pgfscope}%
\begin{pgfscope}%
\pgfpathrectangle{\pgfqpoint{0.100000in}{0.212622in}}{\pgfqpoint{3.696000in}{3.696000in}}%
\pgfusepath{clip}%
\pgfsetrectcap%
\pgfsetroundjoin%
\pgfsetlinewidth{1.505625pt}%
\definecolor{currentstroke}{rgb}{1.000000,0.000000,0.000000}%
\pgfsetstrokecolor{currentstroke}%
\pgfsetdash{}{0pt}%
\pgfpathmoveto{\pgfqpoint{2.482160in}{2.759832in}}%
\pgfpathlineto{\pgfqpoint{2.493488in}{2.477153in}}%
\pgfusepath{stroke}%
\end{pgfscope}%
\begin{pgfscope}%
\pgfpathrectangle{\pgfqpoint{0.100000in}{0.212622in}}{\pgfqpoint{3.696000in}{3.696000in}}%
\pgfusepath{clip}%
\pgfsetrectcap%
\pgfsetroundjoin%
\pgfsetlinewidth{1.505625pt}%
\definecolor{currentstroke}{rgb}{1.000000,0.000000,0.000000}%
\pgfsetstrokecolor{currentstroke}%
\pgfsetdash{}{0pt}%
\pgfpathmoveto{\pgfqpoint{2.483490in}{2.768771in}}%
\pgfpathlineto{\pgfqpoint{2.487827in}{2.478716in}}%
\pgfusepath{stroke}%
\end{pgfscope}%
\begin{pgfscope}%
\pgfpathrectangle{\pgfqpoint{0.100000in}{0.212622in}}{\pgfqpoint{3.696000in}{3.696000in}}%
\pgfusepath{clip}%
\pgfsetrectcap%
\pgfsetroundjoin%
\pgfsetlinewidth{1.505625pt}%
\definecolor{currentstroke}{rgb}{1.000000,0.000000,0.000000}%
\pgfsetstrokecolor{currentstroke}%
\pgfsetdash{}{0pt}%
\pgfpathmoveto{\pgfqpoint{2.484531in}{2.773380in}}%
\pgfpathlineto{\pgfqpoint{2.487827in}{2.478716in}}%
\pgfusepath{stroke}%
\end{pgfscope}%
\begin{pgfscope}%
\pgfpathrectangle{\pgfqpoint{0.100000in}{0.212622in}}{\pgfqpoint{3.696000in}{3.696000in}}%
\pgfusepath{clip}%
\pgfsetrectcap%
\pgfsetroundjoin%
\pgfsetlinewidth{1.505625pt}%
\definecolor{currentstroke}{rgb}{1.000000,0.000000,0.000000}%
\pgfsetstrokecolor{currentstroke}%
\pgfsetdash{}{0pt}%
\pgfpathmoveto{\pgfqpoint{2.487572in}{2.778552in}}%
\pgfpathlineto{\pgfqpoint{2.487827in}{2.478716in}}%
\pgfusepath{stroke}%
\end{pgfscope}%
\begin{pgfscope}%
\pgfpathrectangle{\pgfqpoint{0.100000in}{0.212622in}}{\pgfqpoint{3.696000in}{3.696000in}}%
\pgfusepath{clip}%
\pgfsetrectcap%
\pgfsetroundjoin%
\pgfsetlinewidth{1.505625pt}%
\definecolor{currentstroke}{rgb}{1.000000,0.000000,0.000000}%
\pgfsetstrokecolor{currentstroke}%
\pgfsetdash{}{0pt}%
\pgfpathmoveto{\pgfqpoint{2.490953in}{2.783877in}}%
\pgfpathlineto{\pgfqpoint{2.487827in}{2.478716in}}%
\pgfusepath{stroke}%
\end{pgfscope}%
\begin{pgfscope}%
\pgfpathrectangle{\pgfqpoint{0.100000in}{0.212622in}}{\pgfqpoint{3.696000in}{3.696000in}}%
\pgfusepath{clip}%
\pgfsetrectcap%
\pgfsetroundjoin%
\pgfsetlinewidth{1.505625pt}%
\definecolor{currentstroke}{rgb}{1.000000,0.000000,0.000000}%
\pgfsetstrokecolor{currentstroke}%
\pgfsetdash{}{0pt}%
\pgfpathmoveto{\pgfqpoint{2.493340in}{2.794667in}}%
\pgfpathlineto{\pgfqpoint{2.487827in}{2.478716in}}%
\pgfusepath{stroke}%
\end{pgfscope}%
\begin{pgfscope}%
\pgfpathrectangle{\pgfqpoint{0.100000in}{0.212622in}}{\pgfqpoint{3.696000in}{3.696000in}}%
\pgfusepath{clip}%
\pgfsetrectcap%
\pgfsetroundjoin%
\pgfsetlinewidth{1.505625pt}%
\definecolor{currentstroke}{rgb}{1.000000,0.000000,0.000000}%
\pgfsetstrokecolor{currentstroke}%
\pgfsetdash{}{0pt}%
\pgfpathmoveto{\pgfqpoint{2.494157in}{2.800132in}}%
\pgfpathlineto{\pgfqpoint{2.487827in}{2.478716in}}%
\pgfusepath{stroke}%
\end{pgfscope}%
\begin{pgfscope}%
\pgfpathrectangle{\pgfqpoint{0.100000in}{0.212622in}}{\pgfqpoint{3.696000in}{3.696000in}}%
\pgfusepath{clip}%
\pgfsetrectcap%
\pgfsetroundjoin%
\pgfsetlinewidth{1.505625pt}%
\definecolor{currentstroke}{rgb}{1.000000,0.000000,0.000000}%
\pgfsetstrokecolor{currentstroke}%
\pgfsetdash{}{0pt}%
\pgfpathmoveto{\pgfqpoint{2.498399in}{2.807980in}}%
\pgfpathlineto{\pgfqpoint{2.487827in}{2.478716in}}%
\pgfusepath{stroke}%
\end{pgfscope}%
\begin{pgfscope}%
\pgfpathrectangle{\pgfqpoint{0.100000in}{0.212622in}}{\pgfqpoint{3.696000in}{3.696000in}}%
\pgfusepath{clip}%
\pgfsetrectcap%
\pgfsetroundjoin%
\pgfsetlinewidth{1.505625pt}%
\definecolor{currentstroke}{rgb}{1.000000,0.000000,0.000000}%
\pgfsetstrokecolor{currentstroke}%
\pgfsetdash{}{0pt}%
\pgfpathmoveto{\pgfqpoint{2.500561in}{2.812029in}}%
\pgfpathlineto{\pgfqpoint{2.487827in}{2.478716in}}%
\pgfusepath{stroke}%
\end{pgfscope}%
\begin{pgfscope}%
\pgfpathrectangle{\pgfqpoint{0.100000in}{0.212622in}}{\pgfqpoint{3.696000in}{3.696000in}}%
\pgfusepath{clip}%
\pgfsetrectcap%
\pgfsetroundjoin%
\pgfsetlinewidth{1.505625pt}%
\definecolor{currentstroke}{rgb}{1.000000,0.000000,0.000000}%
\pgfsetstrokecolor{currentstroke}%
\pgfsetdash{}{0pt}%
\pgfpathmoveto{\pgfqpoint{2.501178in}{2.819793in}}%
\pgfpathlineto{\pgfqpoint{2.482167in}{2.480278in}}%
\pgfusepath{stroke}%
\end{pgfscope}%
\begin{pgfscope}%
\pgfpathrectangle{\pgfqpoint{0.100000in}{0.212622in}}{\pgfqpoint{3.696000in}{3.696000in}}%
\pgfusepath{clip}%
\pgfsetrectcap%
\pgfsetroundjoin%
\pgfsetlinewidth{1.505625pt}%
\definecolor{currentstroke}{rgb}{1.000000,0.000000,0.000000}%
\pgfsetstrokecolor{currentstroke}%
\pgfsetdash{}{0pt}%
\pgfpathmoveto{\pgfqpoint{2.502256in}{2.823768in}}%
\pgfpathlineto{\pgfqpoint{2.482167in}{2.480278in}}%
\pgfusepath{stroke}%
\end{pgfscope}%
\begin{pgfscope}%
\pgfpathrectangle{\pgfqpoint{0.100000in}{0.212622in}}{\pgfqpoint{3.696000in}{3.696000in}}%
\pgfusepath{clip}%
\pgfsetrectcap%
\pgfsetroundjoin%
\pgfsetlinewidth{1.505625pt}%
\definecolor{currentstroke}{rgb}{1.000000,0.000000,0.000000}%
\pgfsetstrokecolor{currentstroke}%
\pgfsetdash{}{0pt}%
\pgfpathmoveto{\pgfqpoint{2.505226in}{2.829447in}}%
\pgfpathlineto{\pgfqpoint{2.482167in}{2.480278in}}%
\pgfusepath{stroke}%
\end{pgfscope}%
\begin{pgfscope}%
\pgfpathrectangle{\pgfqpoint{0.100000in}{0.212622in}}{\pgfqpoint{3.696000in}{3.696000in}}%
\pgfusepath{clip}%
\pgfsetrectcap%
\pgfsetroundjoin%
\pgfsetlinewidth{1.505625pt}%
\definecolor{currentstroke}{rgb}{1.000000,0.000000,0.000000}%
\pgfsetstrokecolor{currentstroke}%
\pgfsetdash{}{0pt}%
\pgfpathmoveto{\pgfqpoint{2.506805in}{2.832483in}}%
\pgfpathlineto{\pgfqpoint{2.482167in}{2.480278in}}%
\pgfusepath{stroke}%
\end{pgfscope}%
\begin{pgfscope}%
\pgfpathrectangle{\pgfqpoint{0.100000in}{0.212622in}}{\pgfqpoint{3.696000in}{3.696000in}}%
\pgfusepath{clip}%
\pgfsetrectcap%
\pgfsetroundjoin%
\pgfsetlinewidth{1.505625pt}%
\definecolor{currentstroke}{rgb}{1.000000,0.000000,0.000000}%
\pgfsetstrokecolor{currentstroke}%
\pgfsetdash{}{0pt}%
\pgfpathmoveto{\pgfqpoint{2.507508in}{2.838122in}}%
\pgfpathlineto{\pgfqpoint{2.482167in}{2.480278in}}%
\pgfusepath{stroke}%
\end{pgfscope}%
\begin{pgfscope}%
\pgfpathrectangle{\pgfqpoint{0.100000in}{0.212622in}}{\pgfqpoint{3.696000in}{3.696000in}}%
\pgfusepath{clip}%
\pgfsetrectcap%
\pgfsetroundjoin%
\pgfsetlinewidth{1.505625pt}%
\definecolor{currentstroke}{rgb}{1.000000,0.000000,0.000000}%
\pgfsetstrokecolor{currentstroke}%
\pgfsetdash{}{0pt}%
\pgfpathmoveto{\pgfqpoint{2.508260in}{2.841073in}}%
\pgfpathlineto{\pgfqpoint{2.482167in}{2.480278in}}%
\pgfusepath{stroke}%
\end{pgfscope}%
\begin{pgfscope}%
\pgfpathrectangle{\pgfqpoint{0.100000in}{0.212622in}}{\pgfqpoint{3.696000in}{3.696000in}}%
\pgfusepath{clip}%
\pgfsetrectcap%
\pgfsetroundjoin%
\pgfsetlinewidth{1.505625pt}%
\definecolor{currentstroke}{rgb}{1.000000,0.000000,0.000000}%
\pgfsetstrokecolor{currentstroke}%
\pgfsetdash{}{0pt}%
\pgfpathmoveto{\pgfqpoint{2.510937in}{2.845736in}}%
\pgfpathlineto{\pgfqpoint{2.482167in}{2.480278in}}%
\pgfusepath{stroke}%
\end{pgfscope}%
\begin{pgfscope}%
\pgfpathrectangle{\pgfqpoint{0.100000in}{0.212622in}}{\pgfqpoint{3.696000in}{3.696000in}}%
\pgfusepath{clip}%
\pgfsetrectcap%
\pgfsetroundjoin%
\pgfsetlinewidth{1.505625pt}%
\definecolor{currentstroke}{rgb}{1.000000,0.000000,0.000000}%
\pgfsetstrokecolor{currentstroke}%
\pgfsetdash{}{0pt}%
\pgfpathmoveto{\pgfqpoint{2.512423in}{2.848294in}}%
\pgfpathlineto{\pgfqpoint{2.482167in}{2.480278in}}%
\pgfusepath{stroke}%
\end{pgfscope}%
\begin{pgfscope}%
\pgfpathrectangle{\pgfqpoint{0.100000in}{0.212622in}}{\pgfqpoint{3.696000in}{3.696000in}}%
\pgfusepath{clip}%
\pgfsetrectcap%
\pgfsetroundjoin%
\pgfsetlinewidth{1.505625pt}%
\definecolor{currentstroke}{rgb}{1.000000,0.000000,0.000000}%
\pgfsetstrokecolor{currentstroke}%
\pgfsetdash{}{0pt}%
\pgfpathmoveto{\pgfqpoint{2.513709in}{2.851908in}}%
\pgfpathlineto{\pgfqpoint{2.482167in}{2.480278in}}%
\pgfusepath{stroke}%
\end{pgfscope}%
\begin{pgfscope}%
\pgfpathrectangle{\pgfqpoint{0.100000in}{0.212622in}}{\pgfqpoint{3.696000in}{3.696000in}}%
\pgfusepath{clip}%
\pgfsetrectcap%
\pgfsetroundjoin%
\pgfsetlinewidth{1.505625pt}%
\definecolor{currentstroke}{rgb}{1.000000,0.000000,0.000000}%
\pgfsetstrokecolor{currentstroke}%
\pgfsetdash{}{0pt}%
\pgfpathmoveto{\pgfqpoint{2.515626in}{2.855960in}}%
\pgfpathlineto{\pgfqpoint{2.482167in}{2.480278in}}%
\pgfusepath{stroke}%
\end{pgfscope}%
\begin{pgfscope}%
\pgfpathrectangle{\pgfqpoint{0.100000in}{0.212622in}}{\pgfqpoint{3.696000in}{3.696000in}}%
\pgfusepath{clip}%
\pgfsetrectcap%
\pgfsetroundjoin%
\pgfsetlinewidth{1.505625pt}%
\definecolor{currentstroke}{rgb}{1.000000,0.000000,0.000000}%
\pgfsetstrokecolor{currentstroke}%
\pgfsetdash{}{0pt}%
\pgfpathmoveto{\pgfqpoint{2.518705in}{2.860912in}}%
\pgfpathlineto{\pgfqpoint{2.482167in}{2.480278in}}%
\pgfusepath{stroke}%
\end{pgfscope}%
\begin{pgfscope}%
\pgfpathrectangle{\pgfqpoint{0.100000in}{0.212622in}}{\pgfqpoint{3.696000in}{3.696000in}}%
\pgfusepath{clip}%
\pgfsetrectcap%
\pgfsetroundjoin%
\pgfsetlinewidth{1.505625pt}%
\definecolor{currentstroke}{rgb}{1.000000,0.000000,0.000000}%
\pgfsetstrokecolor{currentstroke}%
\pgfsetdash{}{0pt}%
\pgfpathmoveto{\pgfqpoint{2.521832in}{2.866381in}}%
\pgfpathlineto{\pgfqpoint{2.482167in}{2.480278in}}%
\pgfusepath{stroke}%
\end{pgfscope}%
\begin{pgfscope}%
\pgfpathrectangle{\pgfqpoint{0.100000in}{0.212622in}}{\pgfqpoint{3.696000in}{3.696000in}}%
\pgfusepath{clip}%
\pgfsetrectcap%
\pgfsetroundjoin%
\pgfsetlinewidth{1.505625pt}%
\definecolor{currentstroke}{rgb}{1.000000,0.000000,0.000000}%
\pgfsetstrokecolor{currentstroke}%
\pgfsetdash{}{0pt}%
\pgfpathmoveto{\pgfqpoint{2.523753in}{2.873466in}}%
\pgfpathlineto{\pgfqpoint{2.482167in}{2.480278in}}%
\pgfusepath{stroke}%
\end{pgfscope}%
\begin{pgfscope}%
\pgfpathrectangle{\pgfqpoint{0.100000in}{0.212622in}}{\pgfqpoint{3.696000in}{3.696000in}}%
\pgfusepath{clip}%
\pgfsetrectcap%
\pgfsetroundjoin%
\pgfsetlinewidth{1.505625pt}%
\definecolor{currentstroke}{rgb}{1.000000,0.000000,0.000000}%
\pgfsetstrokecolor{currentstroke}%
\pgfsetdash{}{0pt}%
\pgfpathmoveto{\pgfqpoint{2.527847in}{2.881494in}}%
\pgfpathlineto{\pgfqpoint{2.482167in}{2.480278in}}%
\pgfusepath{stroke}%
\end{pgfscope}%
\begin{pgfscope}%
\pgfpathrectangle{\pgfqpoint{0.100000in}{0.212622in}}{\pgfqpoint{3.696000in}{3.696000in}}%
\pgfusepath{clip}%
\pgfsetrectcap%
\pgfsetroundjoin%
\pgfsetlinewidth{1.505625pt}%
\definecolor{currentstroke}{rgb}{1.000000,0.000000,0.000000}%
\pgfsetstrokecolor{currentstroke}%
\pgfsetdash{}{0pt}%
\pgfpathmoveto{\pgfqpoint{2.533282in}{2.890105in}}%
\pgfpathlineto{\pgfqpoint{2.482167in}{2.480278in}}%
\pgfusepath{stroke}%
\end{pgfscope}%
\begin{pgfscope}%
\pgfpathrectangle{\pgfqpoint{0.100000in}{0.212622in}}{\pgfqpoint{3.696000in}{3.696000in}}%
\pgfusepath{clip}%
\pgfsetrectcap%
\pgfsetroundjoin%
\pgfsetlinewidth{1.505625pt}%
\definecolor{currentstroke}{rgb}{1.000000,0.000000,0.000000}%
\pgfsetstrokecolor{currentstroke}%
\pgfsetdash{}{0pt}%
\pgfpathmoveto{\pgfqpoint{2.538734in}{2.898411in}}%
\pgfpathlineto{\pgfqpoint{2.487827in}{2.478716in}}%
\pgfusepath{stroke}%
\end{pgfscope}%
\begin{pgfscope}%
\pgfpathrectangle{\pgfqpoint{0.100000in}{0.212622in}}{\pgfqpoint{3.696000in}{3.696000in}}%
\pgfusepath{clip}%
\pgfsetrectcap%
\pgfsetroundjoin%
\pgfsetlinewidth{1.505625pt}%
\definecolor{currentstroke}{rgb}{1.000000,0.000000,0.000000}%
\pgfsetstrokecolor{currentstroke}%
\pgfsetdash{}{0pt}%
\pgfpathmoveto{\pgfqpoint{2.542376in}{2.910530in}}%
\pgfpathlineto{\pgfqpoint{2.482167in}{2.480278in}}%
\pgfusepath{stroke}%
\end{pgfscope}%
\begin{pgfscope}%
\pgfpathrectangle{\pgfqpoint{0.100000in}{0.212622in}}{\pgfqpoint{3.696000in}{3.696000in}}%
\pgfusepath{clip}%
\pgfsetrectcap%
\pgfsetroundjoin%
\pgfsetlinewidth{1.505625pt}%
\definecolor{currentstroke}{rgb}{1.000000,0.000000,0.000000}%
\pgfsetstrokecolor{currentstroke}%
\pgfsetdash{}{0pt}%
\pgfpathmoveto{\pgfqpoint{2.544461in}{2.917117in}}%
\pgfpathlineto{\pgfqpoint{2.482167in}{2.480278in}}%
\pgfusepath{stroke}%
\end{pgfscope}%
\begin{pgfscope}%
\pgfpathrectangle{\pgfqpoint{0.100000in}{0.212622in}}{\pgfqpoint{3.696000in}{3.696000in}}%
\pgfusepath{clip}%
\pgfsetrectcap%
\pgfsetroundjoin%
\pgfsetlinewidth{1.505625pt}%
\definecolor{currentstroke}{rgb}{1.000000,0.000000,0.000000}%
\pgfsetstrokecolor{currentstroke}%
\pgfsetdash{}{0pt}%
\pgfpathmoveto{\pgfqpoint{2.548711in}{2.924284in}}%
\pgfpathlineto{\pgfqpoint{2.482167in}{2.480278in}}%
\pgfusepath{stroke}%
\end{pgfscope}%
\begin{pgfscope}%
\pgfpathrectangle{\pgfqpoint{0.100000in}{0.212622in}}{\pgfqpoint{3.696000in}{3.696000in}}%
\pgfusepath{clip}%
\pgfsetrectcap%
\pgfsetroundjoin%
\pgfsetlinewidth{1.505625pt}%
\definecolor{currentstroke}{rgb}{1.000000,0.000000,0.000000}%
\pgfsetstrokecolor{currentstroke}%
\pgfsetdash{}{0pt}%
\pgfpathmoveto{\pgfqpoint{2.551196in}{2.927667in}}%
\pgfpathlineto{\pgfqpoint{2.487827in}{2.478716in}}%
\pgfusepath{stroke}%
\end{pgfscope}%
\begin{pgfscope}%
\pgfpathrectangle{\pgfqpoint{0.100000in}{0.212622in}}{\pgfqpoint{3.696000in}{3.696000in}}%
\pgfusepath{clip}%
\pgfsetrectcap%
\pgfsetroundjoin%
\pgfsetlinewidth{1.505625pt}%
\definecolor{currentstroke}{rgb}{1.000000,0.000000,0.000000}%
\pgfsetstrokecolor{currentstroke}%
\pgfsetdash{}{0pt}%
\pgfpathmoveto{\pgfqpoint{2.552485in}{2.929748in}}%
\pgfpathlineto{\pgfqpoint{2.487827in}{2.478716in}}%
\pgfusepath{stroke}%
\end{pgfscope}%
\begin{pgfscope}%
\pgfpathrectangle{\pgfqpoint{0.100000in}{0.212622in}}{\pgfqpoint{3.696000in}{3.696000in}}%
\pgfusepath{clip}%
\pgfsetrectcap%
\pgfsetroundjoin%
\pgfsetlinewidth{1.505625pt}%
\definecolor{currentstroke}{rgb}{1.000000,0.000000,0.000000}%
\pgfsetstrokecolor{currentstroke}%
\pgfsetdash{}{0pt}%
\pgfpathmoveto{\pgfqpoint{2.553877in}{2.933385in}}%
\pgfpathlineto{\pgfqpoint{2.487827in}{2.478716in}}%
\pgfusepath{stroke}%
\end{pgfscope}%
\begin{pgfscope}%
\pgfpathrectangle{\pgfqpoint{0.100000in}{0.212622in}}{\pgfqpoint{3.696000in}{3.696000in}}%
\pgfusepath{clip}%
\pgfsetrectcap%
\pgfsetroundjoin%
\pgfsetlinewidth{1.505625pt}%
\definecolor{currentstroke}{rgb}{1.000000,0.000000,0.000000}%
\pgfsetstrokecolor{currentstroke}%
\pgfsetdash{}{0pt}%
\pgfpathmoveto{\pgfqpoint{2.554973in}{2.937589in}}%
\pgfpathlineto{\pgfqpoint{2.482167in}{2.480278in}}%
\pgfusepath{stroke}%
\end{pgfscope}%
\begin{pgfscope}%
\pgfpathrectangle{\pgfqpoint{0.100000in}{0.212622in}}{\pgfqpoint{3.696000in}{3.696000in}}%
\pgfusepath{clip}%
\pgfsetrectcap%
\pgfsetroundjoin%
\pgfsetlinewidth{1.505625pt}%
\definecolor{currentstroke}{rgb}{1.000000,0.000000,0.000000}%
\pgfsetstrokecolor{currentstroke}%
\pgfsetdash{}{0pt}%
\pgfpathmoveto{\pgfqpoint{2.557750in}{2.943100in}}%
\pgfpathlineto{\pgfqpoint{2.487827in}{2.478716in}}%
\pgfusepath{stroke}%
\end{pgfscope}%
\begin{pgfscope}%
\pgfpathrectangle{\pgfqpoint{0.100000in}{0.212622in}}{\pgfqpoint{3.696000in}{3.696000in}}%
\pgfusepath{clip}%
\pgfsetrectcap%
\pgfsetroundjoin%
\pgfsetlinewidth{1.505625pt}%
\definecolor{currentstroke}{rgb}{1.000000,0.000000,0.000000}%
\pgfsetstrokecolor{currentstroke}%
\pgfsetdash{}{0pt}%
\pgfpathmoveto{\pgfqpoint{2.561831in}{2.948574in}}%
\pgfpathlineto{\pgfqpoint{2.487827in}{2.478716in}}%
\pgfusepath{stroke}%
\end{pgfscope}%
\begin{pgfscope}%
\pgfpathrectangle{\pgfqpoint{0.100000in}{0.212622in}}{\pgfqpoint{3.696000in}{3.696000in}}%
\pgfusepath{clip}%
\pgfsetrectcap%
\pgfsetroundjoin%
\pgfsetlinewidth{1.505625pt}%
\definecolor{currentstroke}{rgb}{1.000000,0.000000,0.000000}%
\pgfsetstrokecolor{currentstroke}%
\pgfsetdash{}{0pt}%
\pgfpathmoveto{\pgfqpoint{2.565662in}{2.955293in}}%
\pgfpathlineto{\pgfqpoint{2.487827in}{2.478716in}}%
\pgfusepath{stroke}%
\end{pgfscope}%
\begin{pgfscope}%
\pgfpathrectangle{\pgfqpoint{0.100000in}{0.212622in}}{\pgfqpoint{3.696000in}{3.696000in}}%
\pgfusepath{clip}%
\pgfsetrectcap%
\pgfsetroundjoin%
\pgfsetlinewidth{1.505625pt}%
\definecolor{currentstroke}{rgb}{1.000000,0.000000,0.000000}%
\pgfsetstrokecolor{currentstroke}%
\pgfsetdash{}{0pt}%
\pgfpathmoveto{\pgfqpoint{2.568288in}{2.964647in}}%
\pgfpathlineto{\pgfqpoint{2.487827in}{2.478716in}}%
\pgfusepath{stroke}%
\end{pgfscope}%
\begin{pgfscope}%
\pgfpathrectangle{\pgfqpoint{0.100000in}{0.212622in}}{\pgfqpoint{3.696000in}{3.696000in}}%
\pgfusepath{clip}%
\pgfsetrectcap%
\pgfsetroundjoin%
\pgfsetlinewidth{1.505625pt}%
\definecolor{currentstroke}{rgb}{1.000000,0.000000,0.000000}%
\pgfsetstrokecolor{currentstroke}%
\pgfsetdash{}{0pt}%
\pgfpathmoveto{\pgfqpoint{2.569918in}{2.969710in}}%
\pgfpathlineto{\pgfqpoint{2.487827in}{2.478716in}}%
\pgfusepath{stroke}%
\end{pgfscope}%
\begin{pgfscope}%
\pgfpathrectangle{\pgfqpoint{0.100000in}{0.212622in}}{\pgfqpoint{3.696000in}{3.696000in}}%
\pgfusepath{clip}%
\pgfsetrectcap%
\pgfsetroundjoin%
\pgfsetlinewidth{1.505625pt}%
\definecolor{currentstroke}{rgb}{1.000000,0.000000,0.000000}%
\pgfsetstrokecolor{currentstroke}%
\pgfsetdash{}{0pt}%
\pgfpathmoveto{\pgfqpoint{2.571058in}{2.972213in}}%
\pgfpathlineto{\pgfqpoint{2.487827in}{2.478716in}}%
\pgfusepath{stroke}%
\end{pgfscope}%
\begin{pgfscope}%
\pgfpathrectangle{\pgfqpoint{0.100000in}{0.212622in}}{\pgfqpoint{3.696000in}{3.696000in}}%
\pgfusepath{clip}%
\pgfsetrectcap%
\pgfsetroundjoin%
\pgfsetlinewidth{1.505625pt}%
\definecolor{currentstroke}{rgb}{1.000000,0.000000,0.000000}%
\pgfsetstrokecolor{currentstroke}%
\pgfsetdash{}{0pt}%
\pgfpathmoveto{\pgfqpoint{2.571713in}{2.973563in}}%
\pgfpathlineto{\pgfqpoint{2.487827in}{2.478716in}}%
\pgfusepath{stroke}%
\end{pgfscope}%
\begin{pgfscope}%
\pgfpathrectangle{\pgfqpoint{0.100000in}{0.212622in}}{\pgfqpoint{3.696000in}{3.696000in}}%
\pgfusepath{clip}%
\pgfsetrectcap%
\pgfsetroundjoin%
\pgfsetlinewidth{1.505625pt}%
\definecolor{currentstroke}{rgb}{1.000000,0.000000,0.000000}%
\pgfsetstrokecolor{currentstroke}%
\pgfsetdash{}{0pt}%
\pgfpathmoveto{\pgfqpoint{2.572083in}{2.974269in}}%
\pgfpathlineto{\pgfqpoint{2.487827in}{2.478716in}}%
\pgfusepath{stroke}%
\end{pgfscope}%
\begin{pgfscope}%
\pgfpathrectangle{\pgfqpoint{0.100000in}{0.212622in}}{\pgfqpoint{3.696000in}{3.696000in}}%
\pgfusepath{clip}%
\pgfsetrectcap%
\pgfsetroundjoin%
\pgfsetlinewidth{1.505625pt}%
\definecolor{currentstroke}{rgb}{1.000000,0.000000,0.000000}%
\pgfsetstrokecolor{currentstroke}%
\pgfsetdash{}{0pt}%
\pgfpathmoveto{\pgfqpoint{2.572262in}{2.974679in}}%
\pgfpathlineto{\pgfqpoint{2.487827in}{2.478716in}}%
\pgfusepath{stroke}%
\end{pgfscope}%
\begin{pgfscope}%
\pgfpathrectangle{\pgfqpoint{0.100000in}{0.212622in}}{\pgfqpoint{3.696000in}{3.696000in}}%
\pgfusepath{clip}%
\pgfsetrectcap%
\pgfsetroundjoin%
\pgfsetlinewidth{1.505625pt}%
\definecolor{currentstroke}{rgb}{1.000000,0.000000,0.000000}%
\pgfsetstrokecolor{currentstroke}%
\pgfsetdash{}{0pt}%
\pgfpathmoveto{\pgfqpoint{2.572360in}{2.974900in}}%
\pgfpathlineto{\pgfqpoint{2.487827in}{2.478716in}}%
\pgfusepath{stroke}%
\end{pgfscope}%
\begin{pgfscope}%
\pgfpathrectangle{\pgfqpoint{0.100000in}{0.212622in}}{\pgfqpoint{3.696000in}{3.696000in}}%
\pgfusepath{clip}%
\pgfsetrectcap%
\pgfsetroundjoin%
\pgfsetlinewidth{1.505625pt}%
\definecolor{currentstroke}{rgb}{1.000000,0.000000,0.000000}%
\pgfsetstrokecolor{currentstroke}%
\pgfsetdash{}{0pt}%
\pgfpathmoveto{\pgfqpoint{2.572408in}{2.975010in}}%
\pgfpathlineto{\pgfqpoint{2.487827in}{2.478716in}}%
\pgfusepath{stroke}%
\end{pgfscope}%
\begin{pgfscope}%
\pgfpathrectangle{\pgfqpoint{0.100000in}{0.212622in}}{\pgfqpoint{3.696000in}{3.696000in}}%
\pgfusepath{clip}%
\pgfsetrectcap%
\pgfsetroundjoin%
\pgfsetlinewidth{1.505625pt}%
\definecolor{currentstroke}{rgb}{1.000000,0.000000,0.000000}%
\pgfsetstrokecolor{currentstroke}%
\pgfsetdash{}{0pt}%
\pgfpathmoveto{\pgfqpoint{2.572437in}{2.975074in}}%
\pgfpathlineto{\pgfqpoint{2.487827in}{2.478716in}}%
\pgfusepath{stroke}%
\end{pgfscope}%
\begin{pgfscope}%
\pgfpathrectangle{\pgfqpoint{0.100000in}{0.212622in}}{\pgfqpoint{3.696000in}{3.696000in}}%
\pgfusepath{clip}%
\pgfsetrectcap%
\pgfsetroundjoin%
\pgfsetlinewidth{1.505625pt}%
\definecolor{currentstroke}{rgb}{1.000000,0.000000,0.000000}%
\pgfsetstrokecolor{currentstroke}%
\pgfsetdash{}{0pt}%
\pgfpathmoveto{\pgfqpoint{2.572452in}{2.975109in}}%
\pgfpathlineto{\pgfqpoint{2.487827in}{2.478716in}}%
\pgfusepath{stroke}%
\end{pgfscope}%
\begin{pgfscope}%
\pgfpathrectangle{\pgfqpoint{0.100000in}{0.212622in}}{\pgfqpoint{3.696000in}{3.696000in}}%
\pgfusepath{clip}%
\pgfsetrectcap%
\pgfsetroundjoin%
\pgfsetlinewidth{1.505625pt}%
\definecolor{currentstroke}{rgb}{1.000000,0.000000,0.000000}%
\pgfsetstrokecolor{currentstroke}%
\pgfsetdash{}{0pt}%
\pgfpathmoveto{\pgfqpoint{2.572460in}{2.975128in}}%
\pgfpathlineto{\pgfqpoint{2.487827in}{2.478716in}}%
\pgfusepath{stroke}%
\end{pgfscope}%
\begin{pgfscope}%
\pgfpathrectangle{\pgfqpoint{0.100000in}{0.212622in}}{\pgfqpoint{3.696000in}{3.696000in}}%
\pgfusepath{clip}%
\pgfsetrectcap%
\pgfsetroundjoin%
\pgfsetlinewidth{1.505625pt}%
\definecolor{currentstroke}{rgb}{1.000000,0.000000,0.000000}%
\pgfsetstrokecolor{currentstroke}%
\pgfsetdash{}{0pt}%
\pgfpathmoveto{\pgfqpoint{2.572465in}{2.975139in}}%
\pgfpathlineto{\pgfqpoint{2.487827in}{2.478716in}}%
\pgfusepath{stroke}%
\end{pgfscope}%
\begin{pgfscope}%
\pgfpathrectangle{\pgfqpoint{0.100000in}{0.212622in}}{\pgfqpoint{3.696000in}{3.696000in}}%
\pgfusepath{clip}%
\pgfsetrectcap%
\pgfsetroundjoin%
\pgfsetlinewidth{1.505625pt}%
\definecolor{currentstroke}{rgb}{1.000000,0.000000,0.000000}%
\pgfsetstrokecolor{currentstroke}%
\pgfsetdash{}{0pt}%
\pgfpathmoveto{\pgfqpoint{2.572468in}{2.975145in}}%
\pgfpathlineto{\pgfqpoint{2.487827in}{2.478716in}}%
\pgfusepath{stroke}%
\end{pgfscope}%
\begin{pgfscope}%
\pgfpathrectangle{\pgfqpoint{0.100000in}{0.212622in}}{\pgfqpoint{3.696000in}{3.696000in}}%
\pgfusepath{clip}%
\pgfsetrectcap%
\pgfsetroundjoin%
\pgfsetlinewidth{1.505625pt}%
\definecolor{currentstroke}{rgb}{1.000000,0.000000,0.000000}%
\pgfsetstrokecolor{currentstroke}%
\pgfsetdash{}{0pt}%
\pgfpathmoveto{\pgfqpoint{2.572469in}{2.975148in}}%
\pgfpathlineto{\pgfqpoint{2.487827in}{2.478716in}}%
\pgfusepath{stroke}%
\end{pgfscope}%
\begin{pgfscope}%
\pgfpathrectangle{\pgfqpoint{0.100000in}{0.212622in}}{\pgfqpoint{3.696000in}{3.696000in}}%
\pgfusepath{clip}%
\pgfsetrectcap%
\pgfsetroundjoin%
\pgfsetlinewidth{1.505625pt}%
\definecolor{currentstroke}{rgb}{1.000000,0.000000,0.000000}%
\pgfsetstrokecolor{currentstroke}%
\pgfsetdash{}{0pt}%
\pgfpathmoveto{\pgfqpoint{2.572470in}{2.975150in}}%
\pgfpathlineto{\pgfqpoint{2.487827in}{2.478716in}}%
\pgfusepath{stroke}%
\end{pgfscope}%
\begin{pgfscope}%
\pgfpathrectangle{\pgfqpoint{0.100000in}{0.212622in}}{\pgfqpoint{3.696000in}{3.696000in}}%
\pgfusepath{clip}%
\pgfsetrectcap%
\pgfsetroundjoin%
\pgfsetlinewidth{1.505625pt}%
\definecolor{currentstroke}{rgb}{1.000000,0.000000,0.000000}%
\pgfsetstrokecolor{currentstroke}%
\pgfsetdash{}{0pt}%
\pgfpathmoveto{\pgfqpoint{2.572470in}{2.975151in}}%
\pgfpathlineto{\pgfqpoint{2.487827in}{2.478716in}}%
\pgfusepath{stroke}%
\end{pgfscope}%
\begin{pgfscope}%
\pgfpathrectangle{\pgfqpoint{0.100000in}{0.212622in}}{\pgfqpoint{3.696000in}{3.696000in}}%
\pgfusepath{clip}%
\pgfsetrectcap%
\pgfsetroundjoin%
\pgfsetlinewidth{1.505625pt}%
\definecolor{currentstroke}{rgb}{1.000000,0.000000,0.000000}%
\pgfsetstrokecolor{currentstroke}%
\pgfsetdash{}{0pt}%
\pgfpathmoveto{\pgfqpoint{2.572470in}{2.975152in}}%
\pgfpathlineto{\pgfqpoint{2.487827in}{2.478716in}}%
\pgfusepath{stroke}%
\end{pgfscope}%
\begin{pgfscope}%
\pgfpathrectangle{\pgfqpoint{0.100000in}{0.212622in}}{\pgfqpoint{3.696000in}{3.696000in}}%
\pgfusepath{clip}%
\pgfsetrectcap%
\pgfsetroundjoin%
\pgfsetlinewidth{1.505625pt}%
\definecolor{currentstroke}{rgb}{1.000000,0.000000,0.000000}%
\pgfsetstrokecolor{currentstroke}%
\pgfsetdash{}{0pt}%
\pgfpathmoveto{\pgfqpoint{2.572471in}{2.975152in}}%
\pgfpathlineto{\pgfqpoint{2.487827in}{2.478716in}}%
\pgfusepath{stroke}%
\end{pgfscope}%
\begin{pgfscope}%
\pgfpathrectangle{\pgfqpoint{0.100000in}{0.212622in}}{\pgfqpoint{3.696000in}{3.696000in}}%
\pgfusepath{clip}%
\pgfsetrectcap%
\pgfsetroundjoin%
\pgfsetlinewidth{1.505625pt}%
\definecolor{currentstroke}{rgb}{1.000000,0.000000,0.000000}%
\pgfsetstrokecolor{currentstroke}%
\pgfsetdash{}{0pt}%
\pgfpathmoveto{\pgfqpoint{2.572471in}{2.975152in}}%
\pgfpathlineto{\pgfqpoint{2.487827in}{2.478716in}}%
\pgfusepath{stroke}%
\end{pgfscope}%
\begin{pgfscope}%
\pgfpathrectangle{\pgfqpoint{0.100000in}{0.212622in}}{\pgfqpoint{3.696000in}{3.696000in}}%
\pgfusepath{clip}%
\pgfsetrectcap%
\pgfsetroundjoin%
\pgfsetlinewidth{1.505625pt}%
\definecolor{currentstroke}{rgb}{1.000000,0.000000,0.000000}%
\pgfsetstrokecolor{currentstroke}%
\pgfsetdash{}{0pt}%
\pgfpathmoveto{\pgfqpoint{2.572611in}{2.975707in}}%
\pgfpathlineto{\pgfqpoint{2.487827in}{2.478716in}}%
\pgfusepath{stroke}%
\end{pgfscope}%
\begin{pgfscope}%
\pgfpathrectangle{\pgfqpoint{0.100000in}{0.212622in}}{\pgfqpoint{3.696000in}{3.696000in}}%
\pgfusepath{clip}%
\pgfsetrectcap%
\pgfsetroundjoin%
\pgfsetlinewidth{1.505625pt}%
\definecolor{currentstroke}{rgb}{1.000000,0.000000,0.000000}%
\pgfsetstrokecolor{currentstroke}%
\pgfsetdash{}{0pt}%
\pgfpathmoveto{\pgfqpoint{2.572622in}{2.976493in}}%
\pgfpathlineto{\pgfqpoint{2.482167in}{2.480278in}}%
\pgfusepath{stroke}%
\end{pgfscope}%
\begin{pgfscope}%
\pgfpathrectangle{\pgfqpoint{0.100000in}{0.212622in}}{\pgfqpoint{3.696000in}{3.696000in}}%
\pgfusepath{clip}%
\pgfsetrectcap%
\pgfsetroundjoin%
\pgfsetlinewidth{1.505625pt}%
\definecolor{currentstroke}{rgb}{1.000000,0.000000,0.000000}%
\pgfsetstrokecolor{currentstroke}%
\pgfsetdash{}{0pt}%
\pgfpathmoveto{\pgfqpoint{2.572523in}{2.976788in}}%
\pgfpathlineto{\pgfqpoint{2.482167in}{2.480278in}}%
\pgfusepath{stroke}%
\end{pgfscope}%
\begin{pgfscope}%
\pgfpathrectangle{\pgfqpoint{0.100000in}{0.212622in}}{\pgfqpoint{3.696000in}{3.696000in}}%
\pgfusepath{clip}%
\pgfsetrectcap%
\pgfsetroundjoin%
\pgfsetlinewidth{1.505625pt}%
\definecolor{currentstroke}{rgb}{1.000000,0.000000,0.000000}%
\pgfsetstrokecolor{currentstroke}%
\pgfsetdash{}{0pt}%
\pgfpathmoveto{\pgfqpoint{2.572383in}{2.976876in}}%
\pgfpathlineto{\pgfqpoint{2.482167in}{2.480278in}}%
\pgfusepath{stroke}%
\end{pgfscope}%
\begin{pgfscope}%
\pgfpathrectangle{\pgfqpoint{0.100000in}{0.212622in}}{\pgfqpoint{3.696000in}{3.696000in}}%
\pgfusepath{clip}%
\pgfsetrectcap%
\pgfsetroundjoin%
\pgfsetlinewidth{1.505625pt}%
\definecolor{currentstroke}{rgb}{1.000000,0.000000,0.000000}%
\pgfsetstrokecolor{currentstroke}%
\pgfsetdash{}{0pt}%
\pgfpathmoveto{\pgfqpoint{2.572288in}{2.976912in}}%
\pgfpathlineto{\pgfqpoint{2.482167in}{2.480278in}}%
\pgfusepath{stroke}%
\end{pgfscope}%
\begin{pgfscope}%
\pgfpathrectangle{\pgfqpoint{0.100000in}{0.212622in}}{\pgfqpoint{3.696000in}{3.696000in}}%
\pgfusepath{clip}%
\pgfsetrectcap%
\pgfsetroundjoin%
\pgfsetlinewidth{1.505625pt}%
\definecolor{currentstroke}{rgb}{1.000000,0.000000,0.000000}%
\pgfsetstrokecolor{currentstroke}%
\pgfsetdash{}{0pt}%
\pgfpathmoveto{\pgfqpoint{2.571875in}{2.976955in}}%
\pgfpathlineto{\pgfqpoint{2.482167in}{2.480278in}}%
\pgfusepath{stroke}%
\end{pgfscope}%
\begin{pgfscope}%
\pgfpathrectangle{\pgfqpoint{0.100000in}{0.212622in}}{\pgfqpoint{3.696000in}{3.696000in}}%
\pgfusepath{clip}%
\pgfsetrectcap%
\pgfsetroundjoin%
\pgfsetlinewidth{1.505625pt}%
\definecolor{currentstroke}{rgb}{1.000000,0.000000,0.000000}%
\pgfsetstrokecolor{currentstroke}%
\pgfsetdash{}{0pt}%
\pgfpathmoveto{\pgfqpoint{2.571647in}{2.976998in}}%
\pgfpathlineto{\pgfqpoint{2.482167in}{2.480278in}}%
\pgfusepath{stroke}%
\end{pgfscope}%
\begin{pgfscope}%
\pgfpathrectangle{\pgfqpoint{0.100000in}{0.212622in}}{\pgfqpoint{3.696000in}{3.696000in}}%
\pgfusepath{clip}%
\pgfsetrectcap%
\pgfsetroundjoin%
\pgfsetlinewidth{1.505625pt}%
\definecolor{currentstroke}{rgb}{1.000000,0.000000,0.000000}%
\pgfsetstrokecolor{currentstroke}%
\pgfsetdash{}{0pt}%
\pgfpathmoveto{\pgfqpoint{2.571511in}{2.976994in}}%
\pgfpathlineto{\pgfqpoint{2.482167in}{2.480278in}}%
\pgfusepath{stroke}%
\end{pgfscope}%
\begin{pgfscope}%
\pgfpathrectangle{\pgfqpoint{0.100000in}{0.212622in}}{\pgfqpoint{3.696000in}{3.696000in}}%
\pgfusepath{clip}%
\pgfsetrectcap%
\pgfsetroundjoin%
\pgfsetlinewidth{1.505625pt}%
\definecolor{currentstroke}{rgb}{1.000000,0.000000,0.000000}%
\pgfsetstrokecolor{currentstroke}%
\pgfsetdash{}{0pt}%
\pgfpathmoveto{\pgfqpoint{2.571443in}{2.976992in}}%
\pgfpathlineto{\pgfqpoint{2.482167in}{2.480278in}}%
\pgfusepath{stroke}%
\end{pgfscope}%
\begin{pgfscope}%
\pgfpathrectangle{\pgfqpoint{0.100000in}{0.212622in}}{\pgfqpoint{3.696000in}{3.696000in}}%
\pgfusepath{clip}%
\pgfsetrectcap%
\pgfsetroundjoin%
\pgfsetlinewidth{1.505625pt}%
\definecolor{currentstroke}{rgb}{1.000000,0.000000,0.000000}%
\pgfsetstrokecolor{currentstroke}%
\pgfsetdash{}{0pt}%
\pgfpathmoveto{\pgfqpoint{2.571402in}{2.976992in}}%
\pgfpathlineto{\pgfqpoint{2.482167in}{2.480278in}}%
\pgfusepath{stroke}%
\end{pgfscope}%
\begin{pgfscope}%
\pgfpathrectangle{\pgfqpoint{0.100000in}{0.212622in}}{\pgfqpoint{3.696000in}{3.696000in}}%
\pgfusepath{clip}%
\pgfsetrectcap%
\pgfsetroundjoin%
\pgfsetlinewidth{1.505625pt}%
\definecolor{currentstroke}{rgb}{1.000000,0.000000,0.000000}%
\pgfsetstrokecolor{currentstroke}%
\pgfsetdash{}{0pt}%
\pgfpathmoveto{\pgfqpoint{2.571378in}{2.976991in}}%
\pgfpathlineto{\pgfqpoint{2.482167in}{2.480278in}}%
\pgfusepath{stroke}%
\end{pgfscope}%
\begin{pgfscope}%
\pgfpathrectangle{\pgfqpoint{0.100000in}{0.212622in}}{\pgfqpoint{3.696000in}{3.696000in}}%
\pgfusepath{clip}%
\pgfsetrectcap%
\pgfsetroundjoin%
\pgfsetlinewidth{1.505625pt}%
\definecolor{currentstroke}{rgb}{1.000000,0.000000,0.000000}%
\pgfsetstrokecolor{currentstroke}%
\pgfsetdash{}{0pt}%
\pgfpathmoveto{\pgfqpoint{2.571366in}{2.976992in}}%
\pgfpathlineto{\pgfqpoint{2.482167in}{2.480278in}}%
\pgfusepath{stroke}%
\end{pgfscope}%
\begin{pgfscope}%
\pgfpathrectangle{\pgfqpoint{0.100000in}{0.212622in}}{\pgfqpoint{3.696000in}{3.696000in}}%
\pgfusepath{clip}%
\pgfsetrectcap%
\pgfsetroundjoin%
\pgfsetlinewidth{1.505625pt}%
\definecolor{currentstroke}{rgb}{1.000000,0.000000,0.000000}%
\pgfsetstrokecolor{currentstroke}%
\pgfsetdash{}{0pt}%
\pgfpathmoveto{\pgfqpoint{2.571359in}{2.976992in}}%
\pgfpathlineto{\pgfqpoint{2.482167in}{2.480278in}}%
\pgfusepath{stroke}%
\end{pgfscope}%
\begin{pgfscope}%
\pgfpathrectangle{\pgfqpoint{0.100000in}{0.212622in}}{\pgfqpoint{3.696000in}{3.696000in}}%
\pgfusepath{clip}%
\pgfsetrectcap%
\pgfsetroundjoin%
\pgfsetlinewidth{1.505625pt}%
\definecolor{currentstroke}{rgb}{1.000000,0.000000,0.000000}%
\pgfsetstrokecolor{currentstroke}%
\pgfsetdash{}{0pt}%
\pgfpathmoveto{\pgfqpoint{2.571355in}{2.976991in}}%
\pgfpathlineto{\pgfqpoint{2.482167in}{2.480278in}}%
\pgfusepath{stroke}%
\end{pgfscope}%
\begin{pgfscope}%
\pgfpathrectangle{\pgfqpoint{0.100000in}{0.212622in}}{\pgfqpoint{3.696000in}{3.696000in}}%
\pgfusepath{clip}%
\pgfsetrectcap%
\pgfsetroundjoin%
\pgfsetlinewidth{1.505625pt}%
\definecolor{currentstroke}{rgb}{1.000000,0.000000,0.000000}%
\pgfsetstrokecolor{currentstroke}%
\pgfsetdash{}{0pt}%
\pgfpathmoveto{\pgfqpoint{2.571352in}{2.976991in}}%
\pgfpathlineto{\pgfqpoint{2.482167in}{2.480278in}}%
\pgfusepath{stroke}%
\end{pgfscope}%
\begin{pgfscope}%
\pgfpathrectangle{\pgfqpoint{0.100000in}{0.212622in}}{\pgfqpoint{3.696000in}{3.696000in}}%
\pgfusepath{clip}%
\pgfsetrectcap%
\pgfsetroundjoin%
\pgfsetlinewidth{1.505625pt}%
\definecolor{currentstroke}{rgb}{1.000000,0.000000,0.000000}%
\pgfsetstrokecolor{currentstroke}%
\pgfsetdash{}{0pt}%
\pgfpathmoveto{\pgfqpoint{2.571351in}{2.976991in}}%
\pgfpathlineto{\pgfqpoint{2.482167in}{2.480278in}}%
\pgfusepath{stroke}%
\end{pgfscope}%
\begin{pgfscope}%
\pgfpathrectangle{\pgfqpoint{0.100000in}{0.212622in}}{\pgfqpoint{3.696000in}{3.696000in}}%
\pgfusepath{clip}%
\pgfsetrectcap%
\pgfsetroundjoin%
\pgfsetlinewidth{1.505625pt}%
\definecolor{currentstroke}{rgb}{1.000000,0.000000,0.000000}%
\pgfsetstrokecolor{currentstroke}%
\pgfsetdash{}{0pt}%
\pgfpathmoveto{\pgfqpoint{2.571350in}{2.976991in}}%
\pgfpathlineto{\pgfqpoint{2.482167in}{2.480278in}}%
\pgfusepath{stroke}%
\end{pgfscope}%
\begin{pgfscope}%
\pgfpathrectangle{\pgfqpoint{0.100000in}{0.212622in}}{\pgfqpoint{3.696000in}{3.696000in}}%
\pgfusepath{clip}%
\pgfsetrectcap%
\pgfsetroundjoin%
\pgfsetlinewidth{1.505625pt}%
\definecolor{currentstroke}{rgb}{1.000000,0.000000,0.000000}%
\pgfsetstrokecolor{currentstroke}%
\pgfsetdash{}{0pt}%
\pgfpathmoveto{\pgfqpoint{2.571350in}{2.976991in}}%
\pgfpathlineto{\pgfqpoint{2.482167in}{2.480278in}}%
\pgfusepath{stroke}%
\end{pgfscope}%
\begin{pgfscope}%
\pgfpathrectangle{\pgfqpoint{0.100000in}{0.212622in}}{\pgfqpoint{3.696000in}{3.696000in}}%
\pgfusepath{clip}%
\pgfsetrectcap%
\pgfsetroundjoin%
\pgfsetlinewidth{1.505625pt}%
\definecolor{currentstroke}{rgb}{1.000000,0.000000,0.000000}%
\pgfsetstrokecolor{currentstroke}%
\pgfsetdash{}{0pt}%
\pgfpathmoveto{\pgfqpoint{2.571350in}{2.976991in}}%
\pgfpathlineto{\pgfqpoint{2.482167in}{2.480278in}}%
\pgfusepath{stroke}%
\end{pgfscope}%
\begin{pgfscope}%
\pgfpathrectangle{\pgfqpoint{0.100000in}{0.212622in}}{\pgfqpoint{3.696000in}{3.696000in}}%
\pgfusepath{clip}%
\pgfsetrectcap%
\pgfsetroundjoin%
\pgfsetlinewidth{1.505625pt}%
\definecolor{currentstroke}{rgb}{1.000000,0.000000,0.000000}%
\pgfsetstrokecolor{currentstroke}%
\pgfsetdash{}{0pt}%
\pgfpathmoveto{\pgfqpoint{2.571350in}{2.976991in}}%
\pgfpathlineto{\pgfqpoint{2.482167in}{2.480278in}}%
\pgfusepath{stroke}%
\end{pgfscope}%
\begin{pgfscope}%
\pgfpathrectangle{\pgfqpoint{0.100000in}{0.212622in}}{\pgfqpoint{3.696000in}{3.696000in}}%
\pgfusepath{clip}%
\pgfsetrectcap%
\pgfsetroundjoin%
\pgfsetlinewidth{1.505625pt}%
\definecolor{currentstroke}{rgb}{1.000000,0.000000,0.000000}%
\pgfsetstrokecolor{currentstroke}%
\pgfsetdash{}{0pt}%
\pgfpathmoveto{\pgfqpoint{2.571040in}{2.976941in}}%
\pgfpathlineto{\pgfqpoint{2.482167in}{2.480278in}}%
\pgfusepath{stroke}%
\end{pgfscope}%
\begin{pgfscope}%
\pgfpathrectangle{\pgfqpoint{0.100000in}{0.212622in}}{\pgfqpoint{3.696000in}{3.696000in}}%
\pgfusepath{clip}%
\pgfsetrectcap%
\pgfsetroundjoin%
\pgfsetlinewidth{1.505625pt}%
\definecolor{currentstroke}{rgb}{1.000000,0.000000,0.000000}%
\pgfsetstrokecolor{currentstroke}%
\pgfsetdash{}{0pt}%
\pgfpathmoveto{\pgfqpoint{2.569127in}{2.976725in}}%
\pgfpathlineto{\pgfqpoint{2.482167in}{2.480278in}}%
\pgfusepath{stroke}%
\end{pgfscope}%
\begin{pgfscope}%
\pgfpathrectangle{\pgfqpoint{0.100000in}{0.212622in}}{\pgfqpoint{3.696000in}{3.696000in}}%
\pgfusepath{clip}%
\pgfsetrectcap%
\pgfsetroundjoin%
\pgfsetlinewidth{1.505625pt}%
\definecolor{currentstroke}{rgb}{1.000000,0.000000,0.000000}%
\pgfsetstrokecolor{currentstroke}%
\pgfsetdash{}{0pt}%
\pgfpathmoveto{\pgfqpoint{2.568076in}{2.976733in}}%
\pgfpathlineto{\pgfqpoint{2.482167in}{2.480278in}}%
\pgfusepath{stroke}%
\end{pgfscope}%
\begin{pgfscope}%
\pgfpathrectangle{\pgfqpoint{0.100000in}{0.212622in}}{\pgfqpoint{3.696000in}{3.696000in}}%
\pgfusepath{clip}%
\pgfsetrectcap%
\pgfsetroundjoin%
\pgfsetlinewidth{1.505625pt}%
\definecolor{currentstroke}{rgb}{1.000000,0.000000,0.000000}%
\pgfsetstrokecolor{currentstroke}%
\pgfsetdash{}{0pt}%
\pgfpathmoveto{\pgfqpoint{2.566177in}{2.976456in}}%
\pgfpathlineto{\pgfqpoint{2.476508in}{2.481841in}}%
\pgfusepath{stroke}%
\end{pgfscope}%
\begin{pgfscope}%
\pgfpathrectangle{\pgfqpoint{0.100000in}{0.212622in}}{\pgfqpoint{3.696000in}{3.696000in}}%
\pgfusepath{clip}%
\pgfsetrectcap%
\pgfsetroundjoin%
\pgfsetlinewidth{1.505625pt}%
\definecolor{currentstroke}{rgb}{1.000000,0.000000,0.000000}%
\pgfsetstrokecolor{currentstroke}%
\pgfsetdash{}{0pt}%
\pgfpathmoveto{\pgfqpoint{2.563788in}{2.976205in}}%
\pgfpathlineto{\pgfqpoint{2.476508in}{2.481841in}}%
\pgfusepath{stroke}%
\end{pgfscope}%
\begin{pgfscope}%
\pgfpathrectangle{\pgfqpoint{0.100000in}{0.212622in}}{\pgfqpoint{3.696000in}{3.696000in}}%
\pgfusepath{clip}%
\pgfsetrectcap%
\pgfsetroundjoin%
\pgfsetlinewidth{1.505625pt}%
\definecolor{currentstroke}{rgb}{1.000000,0.000000,0.000000}%
\pgfsetstrokecolor{currentstroke}%
\pgfsetdash{}{0pt}%
\pgfpathmoveto{\pgfqpoint{2.560600in}{2.976090in}}%
\pgfpathlineto{\pgfqpoint{2.476508in}{2.481841in}}%
\pgfusepath{stroke}%
\end{pgfscope}%
\begin{pgfscope}%
\pgfpathrectangle{\pgfqpoint{0.100000in}{0.212622in}}{\pgfqpoint{3.696000in}{3.696000in}}%
\pgfusepath{clip}%
\pgfsetrectcap%
\pgfsetroundjoin%
\pgfsetlinewidth{1.505625pt}%
\definecolor{currentstroke}{rgb}{1.000000,0.000000,0.000000}%
\pgfsetstrokecolor{currentstroke}%
\pgfsetdash{}{0pt}%
\pgfpathmoveto{\pgfqpoint{2.556538in}{2.975765in}}%
\pgfpathlineto{\pgfqpoint{2.470851in}{2.483402in}}%
\pgfusepath{stroke}%
\end{pgfscope}%
\begin{pgfscope}%
\pgfpathrectangle{\pgfqpoint{0.100000in}{0.212622in}}{\pgfqpoint{3.696000in}{3.696000in}}%
\pgfusepath{clip}%
\pgfsetrectcap%
\pgfsetroundjoin%
\pgfsetlinewidth{1.505625pt}%
\definecolor{currentstroke}{rgb}{1.000000,0.000000,0.000000}%
\pgfsetstrokecolor{currentstroke}%
\pgfsetdash{}{0pt}%
\pgfpathmoveto{\pgfqpoint{2.554308in}{2.975428in}}%
\pgfpathlineto{\pgfqpoint{2.470851in}{2.483402in}}%
\pgfusepath{stroke}%
\end{pgfscope}%
\begin{pgfscope}%
\pgfpathrectangle{\pgfqpoint{0.100000in}{0.212622in}}{\pgfqpoint{3.696000in}{3.696000in}}%
\pgfusepath{clip}%
\pgfsetrectcap%
\pgfsetroundjoin%
\pgfsetlinewidth{1.505625pt}%
\definecolor{currentstroke}{rgb}{1.000000,0.000000,0.000000}%
\pgfsetstrokecolor{currentstroke}%
\pgfsetdash{}{0pt}%
\pgfpathmoveto{\pgfqpoint{2.550603in}{2.975698in}}%
\pgfpathlineto{\pgfqpoint{2.465196in}{2.484964in}}%
\pgfusepath{stroke}%
\end{pgfscope}%
\begin{pgfscope}%
\pgfpathrectangle{\pgfqpoint{0.100000in}{0.212622in}}{\pgfqpoint{3.696000in}{3.696000in}}%
\pgfusepath{clip}%
\pgfsetrectcap%
\pgfsetroundjoin%
\pgfsetlinewidth{1.505625pt}%
\definecolor{currentstroke}{rgb}{1.000000,0.000000,0.000000}%
\pgfsetstrokecolor{currentstroke}%
\pgfsetdash{}{0pt}%
\pgfpathmoveto{\pgfqpoint{2.548592in}{2.975721in}}%
\pgfpathlineto{\pgfqpoint{2.465196in}{2.484964in}}%
\pgfusepath{stroke}%
\end{pgfscope}%
\begin{pgfscope}%
\pgfpathrectangle{\pgfqpoint{0.100000in}{0.212622in}}{\pgfqpoint{3.696000in}{3.696000in}}%
\pgfusepath{clip}%
\pgfsetrectcap%
\pgfsetroundjoin%
\pgfsetlinewidth{1.505625pt}%
\definecolor{currentstroke}{rgb}{1.000000,0.000000,0.000000}%
\pgfsetstrokecolor{currentstroke}%
\pgfsetdash{}{0pt}%
\pgfpathmoveto{\pgfqpoint{2.546115in}{2.975105in}}%
\pgfpathlineto{\pgfqpoint{2.459542in}{2.486525in}}%
\pgfusepath{stroke}%
\end{pgfscope}%
\begin{pgfscope}%
\pgfpathrectangle{\pgfqpoint{0.100000in}{0.212622in}}{\pgfqpoint{3.696000in}{3.696000in}}%
\pgfusepath{clip}%
\pgfsetrectcap%
\pgfsetroundjoin%
\pgfsetlinewidth{1.505625pt}%
\definecolor{currentstroke}{rgb}{1.000000,0.000000,0.000000}%
\pgfsetstrokecolor{currentstroke}%
\pgfsetdash{}{0pt}%
\pgfpathmoveto{\pgfqpoint{2.542910in}{2.974308in}}%
\pgfpathlineto{\pgfqpoint{2.459542in}{2.486525in}}%
\pgfusepath{stroke}%
\end{pgfscope}%
\begin{pgfscope}%
\pgfpathrectangle{\pgfqpoint{0.100000in}{0.212622in}}{\pgfqpoint{3.696000in}{3.696000in}}%
\pgfusepath{clip}%
\pgfsetrectcap%
\pgfsetroundjoin%
\pgfsetlinewidth{1.505625pt}%
\definecolor{currentstroke}{rgb}{1.000000,0.000000,0.000000}%
\pgfsetstrokecolor{currentstroke}%
\pgfsetdash{}{0pt}%
\pgfpathmoveto{\pgfqpoint{2.538979in}{2.974906in}}%
\pgfpathlineto{\pgfqpoint{2.453889in}{2.488085in}}%
\pgfusepath{stroke}%
\end{pgfscope}%
\begin{pgfscope}%
\pgfpathrectangle{\pgfqpoint{0.100000in}{0.212622in}}{\pgfqpoint{3.696000in}{3.696000in}}%
\pgfusepath{clip}%
\pgfsetrectcap%
\pgfsetroundjoin%
\pgfsetlinewidth{1.505625pt}%
\definecolor{currentstroke}{rgb}{1.000000,0.000000,0.000000}%
\pgfsetstrokecolor{currentstroke}%
\pgfsetdash{}{0pt}%
\pgfpathmoveto{\pgfqpoint{2.534425in}{2.974358in}}%
\pgfpathlineto{\pgfqpoint{2.448239in}{2.489645in}}%
\pgfusepath{stroke}%
\end{pgfscope}%
\begin{pgfscope}%
\pgfpathrectangle{\pgfqpoint{0.100000in}{0.212622in}}{\pgfqpoint{3.696000in}{3.696000in}}%
\pgfusepath{clip}%
\pgfsetrectcap%
\pgfsetroundjoin%
\pgfsetlinewidth{1.505625pt}%
\definecolor{currentstroke}{rgb}{1.000000,0.000000,0.000000}%
\pgfsetstrokecolor{currentstroke}%
\pgfsetdash{}{0pt}%
\pgfpathmoveto{\pgfqpoint{2.531950in}{2.973835in}}%
\pgfpathlineto{\pgfqpoint{2.448239in}{2.489645in}}%
\pgfusepath{stroke}%
\end{pgfscope}%
\begin{pgfscope}%
\pgfpathrectangle{\pgfqpoint{0.100000in}{0.212622in}}{\pgfqpoint{3.696000in}{3.696000in}}%
\pgfusepath{clip}%
\pgfsetrectcap%
\pgfsetroundjoin%
\pgfsetlinewidth{1.505625pt}%
\definecolor{currentstroke}{rgb}{1.000000,0.000000,0.000000}%
\pgfsetstrokecolor{currentstroke}%
\pgfsetdash{}{0pt}%
\pgfpathmoveto{\pgfqpoint{2.528695in}{2.974401in}}%
\pgfpathlineto{\pgfqpoint{2.442590in}{2.491205in}}%
\pgfusepath{stroke}%
\end{pgfscope}%
\begin{pgfscope}%
\pgfpathrectangle{\pgfqpoint{0.100000in}{0.212622in}}{\pgfqpoint{3.696000in}{3.696000in}}%
\pgfusepath{clip}%
\pgfsetrectcap%
\pgfsetroundjoin%
\pgfsetlinewidth{1.505625pt}%
\definecolor{currentstroke}{rgb}{1.000000,0.000000,0.000000}%
\pgfsetstrokecolor{currentstroke}%
\pgfsetdash{}{0pt}%
\pgfpathmoveto{\pgfqpoint{2.525656in}{2.975318in}}%
\pgfpathlineto{\pgfqpoint{2.442590in}{2.491205in}}%
\pgfusepath{stroke}%
\end{pgfscope}%
\begin{pgfscope}%
\pgfpathrectangle{\pgfqpoint{0.100000in}{0.212622in}}{\pgfqpoint{3.696000in}{3.696000in}}%
\pgfusepath{clip}%
\pgfsetrectcap%
\pgfsetroundjoin%
\pgfsetlinewidth{1.505625pt}%
\definecolor{currentstroke}{rgb}{1.000000,0.000000,0.000000}%
\pgfsetstrokecolor{currentstroke}%
\pgfsetdash{}{0pt}%
\pgfpathmoveto{\pgfqpoint{2.521892in}{2.976316in}}%
\pgfpathlineto{\pgfqpoint{2.436942in}{2.492764in}}%
\pgfusepath{stroke}%
\end{pgfscope}%
\begin{pgfscope}%
\pgfpathrectangle{\pgfqpoint{0.100000in}{0.212622in}}{\pgfqpoint{3.696000in}{3.696000in}}%
\pgfusepath{clip}%
\pgfsetrectcap%
\pgfsetroundjoin%
\pgfsetlinewidth{1.505625pt}%
\definecolor{currentstroke}{rgb}{1.000000,0.000000,0.000000}%
\pgfsetstrokecolor{currentstroke}%
\pgfsetdash{}{0pt}%
\pgfpathmoveto{\pgfqpoint{2.517818in}{2.977662in}}%
\pgfpathlineto{\pgfqpoint{2.431296in}{2.494322in}}%
\pgfusepath{stroke}%
\end{pgfscope}%
\begin{pgfscope}%
\pgfpathrectangle{\pgfqpoint{0.100000in}{0.212622in}}{\pgfqpoint{3.696000in}{3.696000in}}%
\pgfusepath{clip}%
\pgfsetrectcap%
\pgfsetroundjoin%
\pgfsetlinewidth{1.505625pt}%
\definecolor{currentstroke}{rgb}{1.000000,0.000000,0.000000}%
\pgfsetstrokecolor{currentstroke}%
\pgfsetdash{}{0pt}%
\pgfpathmoveto{\pgfqpoint{2.513501in}{2.979033in}}%
\pgfpathlineto{\pgfqpoint{2.431296in}{2.494322in}}%
\pgfusepath{stroke}%
\end{pgfscope}%
\begin{pgfscope}%
\pgfpathrectangle{\pgfqpoint{0.100000in}{0.212622in}}{\pgfqpoint{3.696000in}{3.696000in}}%
\pgfusepath{clip}%
\pgfsetrectcap%
\pgfsetroundjoin%
\pgfsetlinewidth{1.505625pt}%
\definecolor{currentstroke}{rgb}{1.000000,0.000000,0.000000}%
\pgfsetstrokecolor{currentstroke}%
\pgfsetdash{}{0pt}%
\pgfpathmoveto{\pgfqpoint{2.508319in}{2.981372in}}%
\pgfpathlineto{\pgfqpoint{2.425652in}{2.495881in}}%
\pgfusepath{stroke}%
\end{pgfscope}%
\begin{pgfscope}%
\pgfpathrectangle{\pgfqpoint{0.100000in}{0.212622in}}{\pgfqpoint{3.696000in}{3.696000in}}%
\pgfusepath{clip}%
\pgfsetrectcap%
\pgfsetroundjoin%
\pgfsetlinewidth{1.505625pt}%
\definecolor{currentstroke}{rgb}{1.000000,0.000000,0.000000}%
\pgfsetstrokecolor{currentstroke}%
\pgfsetdash{}{0pt}%
\pgfpathmoveto{\pgfqpoint{2.505547in}{2.982636in}}%
\pgfpathlineto{\pgfqpoint{2.420009in}{2.497439in}}%
\pgfusepath{stroke}%
\end{pgfscope}%
\begin{pgfscope}%
\pgfpathrectangle{\pgfqpoint{0.100000in}{0.212622in}}{\pgfqpoint{3.696000in}{3.696000in}}%
\pgfusepath{clip}%
\pgfsetrectcap%
\pgfsetroundjoin%
\pgfsetlinewidth{1.505625pt}%
\definecolor{currentstroke}{rgb}{1.000000,0.000000,0.000000}%
\pgfsetstrokecolor{currentstroke}%
\pgfsetdash{}{0pt}%
\pgfpathmoveto{\pgfqpoint{2.501720in}{2.983960in}}%
\pgfpathlineto{\pgfqpoint{2.414368in}{2.498996in}}%
\pgfusepath{stroke}%
\end{pgfscope}%
\begin{pgfscope}%
\pgfpathrectangle{\pgfqpoint{0.100000in}{0.212622in}}{\pgfqpoint{3.696000in}{3.696000in}}%
\pgfusepath{clip}%
\pgfsetrectcap%
\pgfsetroundjoin%
\pgfsetlinewidth{1.505625pt}%
\definecolor{currentstroke}{rgb}{1.000000,0.000000,0.000000}%
\pgfsetstrokecolor{currentstroke}%
\pgfsetdash{}{0pt}%
\pgfpathmoveto{\pgfqpoint{2.499582in}{2.984561in}}%
\pgfpathlineto{\pgfqpoint{2.414368in}{2.498996in}}%
\pgfusepath{stroke}%
\end{pgfscope}%
\begin{pgfscope}%
\pgfpathrectangle{\pgfqpoint{0.100000in}{0.212622in}}{\pgfqpoint{3.696000in}{3.696000in}}%
\pgfusepath{clip}%
\pgfsetrectcap%
\pgfsetroundjoin%
\pgfsetlinewidth{1.505625pt}%
\definecolor{currentstroke}{rgb}{1.000000,0.000000,0.000000}%
\pgfsetstrokecolor{currentstroke}%
\pgfsetdash{}{0pt}%
\pgfpathmoveto{\pgfqpoint{2.496482in}{2.985465in}}%
\pgfpathlineto{\pgfqpoint{2.408728in}{2.500553in}}%
\pgfusepath{stroke}%
\end{pgfscope}%
\begin{pgfscope}%
\pgfpathrectangle{\pgfqpoint{0.100000in}{0.212622in}}{\pgfqpoint{3.696000in}{3.696000in}}%
\pgfusepath{clip}%
\pgfsetrectcap%
\pgfsetroundjoin%
\pgfsetlinewidth{1.505625pt}%
\definecolor{currentstroke}{rgb}{1.000000,0.000000,0.000000}%
\pgfsetstrokecolor{currentstroke}%
\pgfsetdash{}{0pt}%
\pgfpathmoveto{\pgfqpoint{2.492907in}{2.986047in}}%
\pgfpathlineto{\pgfqpoint{2.408728in}{2.500553in}}%
\pgfusepath{stroke}%
\end{pgfscope}%
\begin{pgfscope}%
\pgfpathrectangle{\pgfqpoint{0.100000in}{0.212622in}}{\pgfqpoint{3.696000in}{3.696000in}}%
\pgfusepath{clip}%
\pgfsetrectcap%
\pgfsetroundjoin%
\pgfsetlinewidth{1.505625pt}%
\definecolor{currentstroke}{rgb}{1.000000,0.000000,0.000000}%
\pgfsetstrokecolor{currentstroke}%
\pgfsetdash{}{0pt}%
\pgfpathmoveto{\pgfqpoint{2.488824in}{2.986939in}}%
\pgfpathlineto{\pgfqpoint{2.403090in}{2.502109in}}%
\pgfusepath{stroke}%
\end{pgfscope}%
\begin{pgfscope}%
\pgfpathrectangle{\pgfqpoint{0.100000in}{0.212622in}}{\pgfqpoint{3.696000in}{3.696000in}}%
\pgfusepath{clip}%
\pgfsetrectcap%
\pgfsetroundjoin%
\pgfsetlinewidth{1.505625pt}%
\definecolor{currentstroke}{rgb}{1.000000,0.000000,0.000000}%
\pgfsetstrokecolor{currentstroke}%
\pgfsetdash{}{0pt}%
\pgfpathmoveto{\pgfqpoint{2.483948in}{2.988213in}}%
\pgfpathlineto{\pgfqpoint{2.397454in}{2.503666in}}%
\pgfusepath{stroke}%
\end{pgfscope}%
\begin{pgfscope}%
\pgfpathrectangle{\pgfqpoint{0.100000in}{0.212622in}}{\pgfqpoint{3.696000in}{3.696000in}}%
\pgfusepath{clip}%
\pgfsetrectcap%
\pgfsetroundjoin%
\pgfsetlinewidth{1.505625pt}%
\definecolor{currentstroke}{rgb}{1.000000,0.000000,0.000000}%
\pgfsetstrokecolor{currentstroke}%
\pgfsetdash{}{0pt}%
\pgfpathmoveto{\pgfqpoint{2.481194in}{2.988268in}}%
\pgfpathlineto{\pgfqpoint{2.391819in}{2.505221in}}%
\pgfusepath{stroke}%
\end{pgfscope}%
\begin{pgfscope}%
\pgfpathrectangle{\pgfqpoint{0.100000in}{0.212622in}}{\pgfqpoint{3.696000in}{3.696000in}}%
\pgfusepath{clip}%
\pgfsetrectcap%
\pgfsetroundjoin%
\pgfsetlinewidth{1.505625pt}%
\definecolor{currentstroke}{rgb}{1.000000,0.000000,0.000000}%
\pgfsetstrokecolor{currentstroke}%
\pgfsetdash{}{0pt}%
\pgfpathmoveto{\pgfqpoint{2.477212in}{2.988054in}}%
\pgfpathlineto{\pgfqpoint{2.391819in}{2.505221in}}%
\pgfusepath{stroke}%
\end{pgfscope}%
\begin{pgfscope}%
\pgfpathrectangle{\pgfqpoint{0.100000in}{0.212622in}}{\pgfqpoint{3.696000in}{3.696000in}}%
\pgfusepath{clip}%
\pgfsetrectcap%
\pgfsetroundjoin%
\pgfsetlinewidth{1.505625pt}%
\definecolor{currentstroke}{rgb}{1.000000,0.000000,0.000000}%
\pgfsetstrokecolor{currentstroke}%
\pgfsetdash{}{0pt}%
\pgfpathmoveto{\pgfqpoint{2.475125in}{2.988528in}}%
\pgfpathlineto{\pgfqpoint{2.386186in}{2.506776in}}%
\pgfusepath{stroke}%
\end{pgfscope}%
\begin{pgfscope}%
\pgfpathrectangle{\pgfqpoint{0.100000in}{0.212622in}}{\pgfqpoint{3.696000in}{3.696000in}}%
\pgfusepath{clip}%
\pgfsetrectcap%
\pgfsetroundjoin%
\pgfsetlinewidth{1.505625pt}%
\definecolor{currentstroke}{rgb}{1.000000,0.000000,0.000000}%
\pgfsetstrokecolor{currentstroke}%
\pgfsetdash{}{0pt}%
\pgfpathmoveto{\pgfqpoint{2.471811in}{2.989392in}}%
\pgfpathlineto{\pgfqpoint{2.386186in}{2.506776in}}%
\pgfusepath{stroke}%
\end{pgfscope}%
\begin{pgfscope}%
\pgfpathrectangle{\pgfqpoint{0.100000in}{0.212622in}}{\pgfqpoint{3.696000in}{3.696000in}}%
\pgfusepath{clip}%
\pgfsetrectcap%
\pgfsetroundjoin%
\pgfsetlinewidth{1.505625pt}%
\definecolor{currentstroke}{rgb}{1.000000,0.000000,0.000000}%
\pgfsetstrokecolor{currentstroke}%
\pgfsetdash{}{0pt}%
\pgfpathmoveto{\pgfqpoint{2.468349in}{2.989668in}}%
\pgfpathlineto{\pgfqpoint{2.380554in}{2.508331in}}%
\pgfusepath{stroke}%
\end{pgfscope}%
\begin{pgfscope}%
\pgfpathrectangle{\pgfqpoint{0.100000in}{0.212622in}}{\pgfqpoint{3.696000in}{3.696000in}}%
\pgfusepath{clip}%
\pgfsetrectcap%
\pgfsetroundjoin%
\pgfsetlinewidth{1.505625pt}%
\definecolor{currentstroke}{rgb}{1.000000,0.000000,0.000000}%
\pgfsetstrokecolor{currentstroke}%
\pgfsetdash{}{0pt}%
\pgfpathmoveto{\pgfqpoint{2.463314in}{2.990094in}}%
\pgfpathlineto{\pgfqpoint{2.374924in}{2.509886in}}%
\pgfusepath{stroke}%
\end{pgfscope}%
\begin{pgfscope}%
\pgfpathrectangle{\pgfqpoint{0.100000in}{0.212622in}}{\pgfqpoint{3.696000in}{3.696000in}}%
\pgfusepath{clip}%
\pgfsetrectcap%
\pgfsetroundjoin%
\pgfsetlinewidth{1.505625pt}%
\definecolor{currentstroke}{rgb}{1.000000,0.000000,0.000000}%
\pgfsetstrokecolor{currentstroke}%
\pgfsetdash{}{0pt}%
\pgfpathmoveto{\pgfqpoint{2.460548in}{2.990581in}}%
\pgfpathlineto{\pgfqpoint{2.374924in}{2.509886in}}%
\pgfusepath{stroke}%
\end{pgfscope}%
\begin{pgfscope}%
\pgfpathrectangle{\pgfqpoint{0.100000in}{0.212622in}}{\pgfqpoint{3.696000in}{3.696000in}}%
\pgfusepath{clip}%
\pgfsetrectcap%
\pgfsetroundjoin%
\pgfsetlinewidth{1.505625pt}%
\definecolor{currentstroke}{rgb}{1.000000,0.000000,0.000000}%
\pgfsetstrokecolor{currentstroke}%
\pgfsetdash{}{0pt}%
\pgfpathmoveto{\pgfqpoint{2.456458in}{2.991556in}}%
\pgfpathlineto{\pgfqpoint{2.369295in}{2.511440in}}%
\pgfusepath{stroke}%
\end{pgfscope}%
\begin{pgfscope}%
\pgfpathrectangle{\pgfqpoint{0.100000in}{0.212622in}}{\pgfqpoint{3.696000in}{3.696000in}}%
\pgfusepath{clip}%
\pgfsetrectcap%
\pgfsetroundjoin%
\pgfsetlinewidth{1.505625pt}%
\definecolor{currentstroke}{rgb}{1.000000,0.000000,0.000000}%
\pgfsetstrokecolor{currentstroke}%
\pgfsetdash{}{0pt}%
\pgfpathmoveto{\pgfqpoint{2.454098in}{2.991552in}}%
\pgfpathlineto{\pgfqpoint{2.369295in}{2.511440in}}%
\pgfusepath{stroke}%
\end{pgfscope}%
\begin{pgfscope}%
\pgfpathrectangle{\pgfqpoint{0.100000in}{0.212622in}}{\pgfqpoint{3.696000in}{3.696000in}}%
\pgfusepath{clip}%
\pgfsetrectcap%
\pgfsetroundjoin%
\pgfsetlinewidth{1.505625pt}%
\definecolor{currentstroke}{rgb}{1.000000,0.000000,0.000000}%
\pgfsetstrokecolor{currentstroke}%
\pgfsetdash{}{0pt}%
\pgfpathmoveto{\pgfqpoint{2.450672in}{2.992839in}}%
\pgfpathlineto{\pgfqpoint{2.363668in}{2.512993in}}%
\pgfusepath{stroke}%
\end{pgfscope}%
\begin{pgfscope}%
\pgfpathrectangle{\pgfqpoint{0.100000in}{0.212622in}}{\pgfqpoint{3.696000in}{3.696000in}}%
\pgfusepath{clip}%
\pgfsetrectcap%
\pgfsetroundjoin%
\pgfsetlinewidth{1.505625pt}%
\definecolor{currentstroke}{rgb}{1.000000,0.000000,0.000000}%
\pgfsetstrokecolor{currentstroke}%
\pgfsetdash{}{0pt}%
\pgfpathmoveto{\pgfqpoint{2.448753in}{2.993105in}}%
\pgfpathlineto{\pgfqpoint{2.363668in}{2.512993in}}%
\pgfusepath{stroke}%
\end{pgfscope}%
\begin{pgfscope}%
\pgfpathrectangle{\pgfqpoint{0.100000in}{0.212622in}}{\pgfqpoint{3.696000in}{3.696000in}}%
\pgfusepath{clip}%
\pgfsetrectcap%
\pgfsetroundjoin%
\pgfsetlinewidth{1.505625pt}%
\definecolor{currentstroke}{rgb}{1.000000,0.000000,0.000000}%
\pgfsetstrokecolor{currentstroke}%
\pgfsetdash{}{0pt}%
\pgfpathmoveto{\pgfqpoint{2.445791in}{2.993339in}}%
\pgfpathlineto{\pgfqpoint{2.358043in}{2.514546in}}%
\pgfusepath{stroke}%
\end{pgfscope}%
\begin{pgfscope}%
\pgfpathrectangle{\pgfqpoint{0.100000in}{0.212622in}}{\pgfqpoint{3.696000in}{3.696000in}}%
\pgfusepath{clip}%
\pgfsetrectcap%
\pgfsetroundjoin%
\pgfsetlinewidth{1.505625pt}%
\definecolor{currentstroke}{rgb}{1.000000,0.000000,0.000000}%
\pgfsetstrokecolor{currentstroke}%
\pgfsetdash{}{0pt}%
\pgfpathmoveto{\pgfqpoint{2.442396in}{2.993776in}}%
\pgfpathlineto{\pgfqpoint{2.358043in}{2.514546in}}%
\pgfusepath{stroke}%
\end{pgfscope}%
\begin{pgfscope}%
\pgfpathrectangle{\pgfqpoint{0.100000in}{0.212622in}}{\pgfqpoint{3.696000in}{3.696000in}}%
\pgfusepath{clip}%
\pgfsetrectcap%
\pgfsetroundjoin%
\pgfsetlinewidth{1.505625pt}%
\definecolor{currentstroke}{rgb}{1.000000,0.000000,0.000000}%
\pgfsetstrokecolor{currentstroke}%
\pgfsetdash{}{0pt}%
\pgfpathmoveto{\pgfqpoint{2.438407in}{2.995115in}}%
\pgfpathlineto{\pgfqpoint{2.352419in}{2.516099in}}%
\pgfusepath{stroke}%
\end{pgfscope}%
\begin{pgfscope}%
\pgfpathrectangle{\pgfqpoint{0.100000in}{0.212622in}}{\pgfqpoint{3.696000in}{3.696000in}}%
\pgfusepath{clip}%
\pgfsetrectcap%
\pgfsetroundjoin%
\pgfsetlinewidth{1.505625pt}%
\definecolor{currentstroke}{rgb}{1.000000,0.000000,0.000000}%
\pgfsetstrokecolor{currentstroke}%
\pgfsetdash{}{0pt}%
\pgfpathmoveto{\pgfqpoint{2.433777in}{2.995790in}}%
\pgfpathlineto{\pgfqpoint{2.346796in}{2.517651in}}%
\pgfusepath{stroke}%
\end{pgfscope}%
\begin{pgfscope}%
\pgfpathrectangle{\pgfqpoint{0.100000in}{0.212622in}}{\pgfqpoint{3.696000in}{3.696000in}}%
\pgfusepath{clip}%
\pgfsetrectcap%
\pgfsetroundjoin%
\pgfsetlinewidth{1.505625pt}%
\definecolor{currentstroke}{rgb}{1.000000,0.000000,0.000000}%
\pgfsetstrokecolor{currentstroke}%
\pgfsetdash{}{0pt}%
\pgfpathmoveto{\pgfqpoint{2.428299in}{2.996256in}}%
\pgfpathlineto{\pgfqpoint{2.341176in}{2.519203in}}%
\pgfusepath{stroke}%
\end{pgfscope}%
\begin{pgfscope}%
\pgfpathrectangle{\pgfqpoint{0.100000in}{0.212622in}}{\pgfqpoint{3.696000in}{3.696000in}}%
\pgfusepath{clip}%
\pgfsetrectcap%
\pgfsetroundjoin%
\pgfsetlinewidth{1.505625pt}%
\definecolor{currentstroke}{rgb}{1.000000,0.000000,0.000000}%
\pgfsetstrokecolor{currentstroke}%
\pgfsetdash{}{0pt}%
\pgfpathmoveto{\pgfqpoint{2.422566in}{2.996555in}}%
\pgfpathlineto{\pgfqpoint{2.335556in}{2.520754in}}%
\pgfusepath{stroke}%
\end{pgfscope}%
\begin{pgfscope}%
\pgfpathrectangle{\pgfqpoint{0.100000in}{0.212622in}}{\pgfqpoint{3.696000in}{3.696000in}}%
\pgfusepath{clip}%
\pgfsetrectcap%
\pgfsetroundjoin%
\pgfsetlinewidth{1.505625pt}%
\definecolor{currentstroke}{rgb}{1.000000,0.000000,0.000000}%
\pgfsetstrokecolor{currentstroke}%
\pgfsetdash{}{0pt}%
\pgfpathmoveto{\pgfqpoint{2.415504in}{2.998112in}}%
\pgfpathlineto{\pgfqpoint{2.329939in}{2.522305in}}%
\pgfusepath{stroke}%
\end{pgfscope}%
\begin{pgfscope}%
\pgfpathrectangle{\pgfqpoint{0.100000in}{0.212622in}}{\pgfqpoint{3.696000in}{3.696000in}}%
\pgfusepath{clip}%
\pgfsetrectcap%
\pgfsetroundjoin%
\pgfsetlinewidth{1.505625pt}%
\definecolor{currentstroke}{rgb}{1.000000,0.000000,0.000000}%
\pgfsetstrokecolor{currentstroke}%
\pgfsetdash{}{0pt}%
\pgfpathmoveto{\pgfqpoint{2.411721in}{2.998951in}}%
\pgfpathlineto{\pgfqpoint{2.324323in}{2.523855in}}%
\pgfusepath{stroke}%
\end{pgfscope}%
\begin{pgfscope}%
\pgfpathrectangle{\pgfqpoint{0.100000in}{0.212622in}}{\pgfqpoint{3.696000in}{3.696000in}}%
\pgfusepath{clip}%
\pgfsetrectcap%
\pgfsetroundjoin%
\pgfsetlinewidth{1.505625pt}%
\definecolor{currentstroke}{rgb}{1.000000,0.000000,0.000000}%
\pgfsetstrokecolor{currentstroke}%
\pgfsetdash{}{0pt}%
\pgfpathmoveto{\pgfqpoint{2.406711in}{3.000082in}}%
\pgfpathlineto{\pgfqpoint{2.318708in}{2.525405in}}%
\pgfusepath{stroke}%
\end{pgfscope}%
\begin{pgfscope}%
\pgfpathrectangle{\pgfqpoint{0.100000in}{0.212622in}}{\pgfqpoint{3.696000in}{3.696000in}}%
\pgfusepath{clip}%
\pgfsetrectcap%
\pgfsetroundjoin%
\pgfsetlinewidth{1.505625pt}%
\definecolor{currentstroke}{rgb}{1.000000,0.000000,0.000000}%
\pgfsetstrokecolor{currentstroke}%
\pgfsetdash{}{0pt}%
\pgfpathmoveto{\pgfqpoint{2.401736in}{3.001413in}}%
\pgfpathlineto{\pgfqpoint{2.313095in}{2.526955in}}%
\pgfusepath{stroke}%
\end{pgfscope}%
\begin{pgfscope}%
\pgfpathrectangle{\pgfqpoint{0.100000in}{0.212622in}}{\pgfqpoint{3.696000in}{3.696000in}}%
\pgfusepath{clip}%
\pgfsetrectcap%
\pgfsetroundjoin%
\pgfsetlinewidth{1.505625pt}%
\definecolor{currentstroke}{rgb}{1.000000,0.000000,0.000000}%
\pgfsetstrokecolor{currentstroke}%
\pgfsetdash{}{0pt}%
\pgfpathmoveto{\pgfqpoint{2.395735in}{3.002991in}}%
\pgfpathlineto{\pgfqpoint{2.307484in}{2.528504in}}%
\pgfusepath{stroke}%
\end{pgfscope}%
\begin{pgfscope}%
\pgfpathrectangle{\pgfqpoint{0.100000in}{0.212622in}}{\pgfqpoint{3.696000in}{3.696000in}}%
\pgfusepath{clip}%
\pgfsetrectcap%
\pgfsetroundjoin%
\pgfsetlinewidth{1.505625pt}%
\definecolor{currentstroke}{rgb}{1.000000,0.000000,0.000000}%
\pgfsetstrokecolor{currentstroke}%
\pgfsetdash{}{0pt}%
\pgfpathmoveto{\pgfqpoint{2.389072in}{3.005092in}}%
\pgfpathlineto{\pgfqpoint{2.301874in}{2.530053in}}%
\pgfusepath{stroke}%
\end{pgfscope}%
\begin{pgfscope}%
\pgfpathrectangle{\pgfqpoint{0.100000in}{0.212622in}}{\pgfqpoint{3.696000in}{3.696000in}}%
\pgfusepath{clip}%
\pgfsetrectcap%
\pgfsetroundjoin%
\pgfsetlinewidth{1.505625pt}%
\definecolor{currentstroke}{rgb}{1.000000,0.000000,0.000000}%
\pgfsetstrokecolor{currentstroke}%
\pgfsetdash{}{0pt}%
\pgfpathmoveto{\pgfqpoint{2.381552in}{3.006518in}}%
\pgfpathlineto{\pgfqpoint{2.296266in}{2.531601in}}%
\pgfusepath{stroke}%
\end{pgfscope}%
\begin{pgfscope}%
\pgfpathrectangle{\pgfqpoint{0.100000in}{0.212622in}}{\pgfqpoint{3.696000in}{3.696000in}}%
\pgfusepath{clip}%
\pgfsetrectcap%
\pgfsetroundjoin%
\pgfsetlinewidth{1.505625pt}%
\definecolor{currentstroke}{rgb}{1.000000,0.000000,0.000000}%
\pgfsetstrokecolor{currentstroke}%
\pgfsetdash{}{0pt}%
\pgfpathmoveto{\pgfqpoint{2.373289in}{3.007693in}}%
\pgfpathlineto{\pgfqpoint{2.285054in}{2.534696in}}%
\pgfusepath{stroke}%
\end{pgfscope}%
\begin{pgfscope}%
\pgfpathrectangle{\pgfqpoint{0.100000in}{0.212622in}}{\pgfqpoint{3.696000in}{3.696000in}}%
\pgfusepath{clip}%
\pgfsetrectcap%
\pgfsetroundjoin%
\pgfsetlinewidth{1.505625pt}%
\definecolor{currentstroke}{rgb}{1.000000,0.000000,0.000000}%
\pgfsetstrokecolor{currentstroke}%
\pgfsetdash{}{0pt}%
\pgfpathmoveto{\pgfqpoint{2.368563in}{3.008315in}}%
\pgfpathlineto{\pgfqpoint{2.279451in}{2.536244in}}%
\pgfusepath{stroke}%
\end{pgfscope}%
\begin{pgfscope}%
\pgfpathrectangle{\pgfqpoint{0.100000in}{0.212622in}}{\pgfqpoint{3.696000in}{3.696000in}}%
\pgfusepath{clip}%
\pgfsetrectcap%
\pgfsetroundjoin%
\pgfsetlinewidth{1.505625pt}%
\definecolor{currentstroke}{rgb}{1.000000,0.000000,0.000000}%
\pgfsetstrokecolor{currentstroke}%
\pgfsetdash{}{0pt}%
\pgfpathmoveto{\pgfqpoint{2.363983in}{3.009824in}}%
\pgfpathlineto{\pgfqpoint{2.279451in}{2.536244in}}%
\pgfusepath{stroke}%
\end{pgfscope}%
\begin{pgfscope}%
\pgfpathrectangle{\pgfqpoint{0.100000in}{0.212622in}}{\pgfqpoint{3.696000in}{3.696000in}}%
\pgfusepath{clip}%
\pgfsetrectcap%
\pgfsetroundjoin%
\pgfsetlinewidth{1.505625pt}%
\definecolor{currentstroke}{rgb}{1.000000,0.000000,0.000000}%
\pgfsetstrokecolor{currentstroke}%
\pgfsetdash{}{0pt}%
\pgfpathmoveto{\pgfqpoint{2.358765in}{3.010758in}}%
\pgfpathlineto{\pgfqpoint{2.273849in}{2.537790in}}%
\pgfusepath{stroke}%
\end{pgfscope}%
\begin{pgfscope}%
\pgfpathrectangle{\pgfqpoint{0.100000in}{0.212622in}}{\pgfqpoint{3.696000in}{3.696000in}}%
\pgfusepath{clip}%
\pgfsetrectcap%
\pgfsetroundjoin%
\pgfsetlinewidth{1.505625pt}%
\definecolor{currentstroke}{rgb}{1.000000,0.000000,0.000000}%
\pgfsetstrokecolor{currentstroke}%
\pgfsetdash{}{0pt}%
\pgfpathmoveto{\pgfqpoint{2.355720in}{3.011158in}}%
\pgfpathlineto{\pgfqpoint{2.268248in}{2.539336in}}%
\pgfusepath{stroke}%
\end{pgfscope}%
\begin{pgfscope}%
\pgfpathrectangle{\pgfqpoint{0.100000in}{0.212622in}}{\pgfqpoint{3.696000in}{3.696000in}}%
\pgfusepath{clip}%
\pgfsetrectcap%
\pgfsetroundjoin%
\pgfsetlinewidth{1.505625pt}%
\definecolor{currentstroke}{rgb}{1.000000,0.000000,0.000000}%
\pgfsetstrokecolor{currentstroke}%
\pgfsetdash{}{0pt}%
\pgfpathmoveto{\pgfqpoint{2.352017in}{3.011936in}}%
\pgfpathlineto{\pgfqpoint{2.262649in}{2.540882in}}%
\pgfusepath{stroke}%
\end{pgfscope}%
\begin{pgfscope}%
\pgfpathrectangle{\pgfqpoint{0.100000in}{0.212622in}}{\pgfqpoint{3.696000in}{3.696000in}}%
\pgfusepath{clip}%
\pgfsetrectcap%
\pgfsetroundjoin%
\pgfsetlinewidth{1.505625pt}%
\definecolor{currentstroke}{rgb}{1.000000,0.000000,0.000000}%
\pgfsetstrokecolor{currentstroke}%
\pgfsetdash{}{0pt}%
\pgfpathmoveto{\pgfqpoint{2.347757in}{3.012312in}}%
\pgfpathlineto{\pgfqpoint{2.262649in}{2.540882in}}%
\pgfusepath{stroke}%
\end{pgfscope}%
\begin{pgfscope}%
\pgfpathrectangle{\pgfqpoint{0.100000in}{0.212622in}}{\pgfqpoint{3.696000in}{3.696000in}}%
\pgfusepath{clip}%
\pgfsetrectcap%
\pgfsetroundjoin%
\pgfsetlinewidth{1.505625pt}%
\definecolor{currentstroke}{rgb}{1.000000,0.000000,0.000000}%
\pgfsetstrokecolor{currentstroke}%
\pgfsetdash{}{0pt}%
\pgfpathmoveto{\pgfqpoint{2.342088in}{3.013717in}}%
\pgfpathlineto{\pgfqpoint{2.257052in}{2.542427in}}%
\pgfusepath{stroke}%
\end{pgfscope}%
\begin{pgfscope}%
\pgfpathrectangle{\pgfqpoint{0.100000in}{0.212622in}}{\pgfqpoint{3.696000in}{3.696000in}}%
\pgfusepath{clip}%
\pgfsetrectcap%
\pgfsetroundjoin%
\pgfsetlinewidth{1.505625pt}%
\definecolor{currentstroke}{rgb}{1.000000,0.000000,0.000000}%
\pgfsetstrokecolor{currentstroke}%
\pgfsetdash{}{0pt}%
\pgfpathmoveto{\pgfqpoint{2.336324in}{3.014424in}}%
\pgfpathlineto{\pgfqpoint{2.251456in}{2.543972in}}%
\pgfusepath{stroke}%
\end{pgfscope}%
\begin{pgfscope}%
\pgfpathrectangle{\pgfqpoint{0.100000in}{0.212622in}}{\pgfqpoint{3.696000in}{3.696000in}}%
\pgfusepath{clip}%
\pgfsetrectcap%
\pgfsetroundjoin%
\pgfsetlinewidth{1.505625pt}%
\definecolor{currentstroke}{rgb}{1.000000,0.000000,0.000000}%
\pgfsetstrokecolor{currentstroke}%
\pgfsetdash{}{0pt}%
\pgfpathmoveto{\pgfqpoint{2.329274in}{3.014629in}}%
\pgfpathlineto{\pgfqpoint{2.240269in}{2.547061in}}%
\pgfusepath{stroke}%
\end{pgfscope}%
\begin{pgfscope}%
\pgfpathrectangle{\pgfqpoint{0.100000in}{0.212622in}}{\pgfqpoint{3.696000in}{3.696000in}}%
\pgfusepath{clip}%
\pgfsetrectcap%
\pgfsetroundjoin%
\pgfsetlinewidth{1.505625pt}%
\definecolor{currentstroke}{rgb}{1.000000,0.000000,0.000000}%
\pgfsetstrokecolor{currentstroke}%
\pgfsetdash{}{0pt}%
\pgfpathmoveto{\pgfqpoint{2.322311in}{3.015040in}}%
\pgfpathlineto{\pgfqpoint{2.234678in}{2.548604in}}%
\pgfusepath{stroke}%
\end{pgfscope}%
\begin{pgfscope}%
\pgfpathrectangle{\pgfqpoint{0.100000in}{0.212622in}}{\pgfqpoint{3.696000in}{3.696000in}}%
\pgfusepath{clip}%
\pgfsetrectcap%
\pgfsetroundjoin%
\pgfsetlinewidth{1.505625pt}%
\definecolor{currentstroke}{rgb}{1.000000,0.000000,0.000000}%
\pgfsetstrokecolor{currentstroke}%
\pgfsetdash{}{0pt}%
\pgfpathmoveto{\pgfqpoint{2.313958in}{3.016545in}}%
\pgfpathlineto{\pgfqpoint{2.229089in}{2.550147in}}%
\pgfusepath{stroke}%
\end{pgfscope}%
\begin{pgfscope}%
\pgfpathrectangle{\pgfqpoint{0.100000in}{0.212622in}}{\pgfqpoint{3.696000in}{3.696000in}}%
\pgfusepath{clip}%
\pgfsetrectcap%
\pgfsetroundjoin%
\pgfsetlinewidth{1.505625pt}%
\definecolor{currentstroke}{rgb}{1.000000,0.000000,0.000000}%
\pgfsetstrokecolor{currentstroke}%
\pgfsetdash{}{0pt}%
\pgfpathmoveto{\pgfqpoint{2.309407in}{3.017207in}}%
\pgfpathlineto{\pgfqpoint{2.223501in}{2.551690in}}%
\pgfusepath{stroke}%
\end{pgfscope}%
\begin{pgfscope}%
\pgfpathrectangle{\pgfqpoint{0.100000in}{0.212622in}}{\pgfqpoint{3.696000in}{3.696000in}}%
\pgfusepath{clip}%
\pgfsetrectcap%
\pgfsetroundjoin%
\pgfsetlinewidth{1.505625pt}%
\definecolor{currentstroke}{rgb}{1.000000,0.000000,0.000000}%
\pgfsetstrokecolor{currentstroke}%
\pgfsetdash{}{0pt}%
\pgfpathmoveto{\pgfqpoint{2.303714in}{3.018254in}}%
\pgfpathlineto{\pgfqpoint{2.217914in}{2.553232in}}%
\pgfusepath{stroke}%
\end{pgfscope}%
\begin{pgfscope}%
\pgfpathrectangle{\pgfqpoint{0.100000in}{0.212622in}}{\pgfqpoint{3.696000in}{3.696000in}}%
\pgfusepath{clip}%
\pgfsetrectcap%
\pgfsetroundjoin%
\pgfsetlinewidth{1.505625pt}%
\definecolor{currentstroke}{rgb}{1.000000,0.000000,0.000000}%
\pgfsetstrokecolor{currentstroke}%
\pgfsetdash{}{0pt}%
\pgfpathmoveto{\pgfqpoint{2.300537in}{3.017947in}}%
\pgfpathlineto{\pgfqpoint{2.212330in}{2.554774in}}%
\pgfusepath{stroke}%
\end{pgfscope}%
\begin{pgfscope}%
\pgfpathrectangle{\pgfqpoint{0.100000in}{0.212622in}}{\pgfqpoint{3.696000in}{3.696000in}}%
\pgfusepath{clip}%
\pgfsetrectcap%
\pgfsetroundjoin%
\pgfsetlinewidth{1.505625pt}%
\definecolor{currentstroke}{rgb}{1.000000,0.000000,0.000000}%
\pgfsetstrokecolor{currentstroke}%
\pgfsetdash{}{0pt}%
\pgfpathmoveto{\pgfqpoint{2.295661in}{3.018203in}}%
\pgfpathlineto{\pgfqpoint{2.212330in}{2.554774in}}%
\pgfusepath{stroke}%
\end{pgfscope}%
\begin{pgfscope}%
\pgfpathrectangle{\pgfqpoint{0.100000in}{0.212622in}}{\pgfqpoint{3.696000in}{3.696000in}}%
\pgfusepath{clip}%
\pgfsetrectcap%
\pgfsetroundjoin%
\pgfsetlinewidth{1.505625pt}%
\definecolor{currentstroke}{rgb}{1.000000,0.000000,0.000000}%
\pgfsetstrokecolor{currentstroke}%
\pgfsetdash{}{0pt}%
\pgfpathmoveto{\pgfqpoint{2.293212in}{3.018511in}}%
\pgfpathlineto{\pgfqpoint{2.206746in}{2.556316in}}%
\pgfusepath{stroke}%
\end{pgfscope}%
\begin{pgfscope}%
\pgfpathrectangle{\pgfqpoint{0.100000in}{0.212622in}}{\pgfqpoint{3.696000in}{3.696000in}}%
\pgfusepath{clip}%
\pgfsetrectcap%
\pgfsetroundjoin%
\pgfsetlinewidth{1.505625pt}%
\definecolor{currentstroke}{rgb}{1.000000,0.000000,0.000000}%
\pgfsetstrokecolor{currentstroke}%
\pgfsetdash{}{0pt}%
\pgfpathmoveto{\pgfqpoint{2.289959in}{3.018969in}}%
\pgfpathlineto{\pgfqpoint{2.206746in}{2.556316in}}%
\pgfusepath{stroke}%
\end{pgfscope}%
\begin{pgfscope}%
\pgfpathrectangle{\pgfqpoint{0.100000in}{0.212622in}}{\pgfqpoint{3.696000in}{3.696000in}}%
\pgfusepath{clip}%
\pgfsetrectcap%
\pgfsetroundjoin%
\pgfsetlinewidth{1.505625pt}%
\definecolor{currentstroke}{rgb}{1.000000,0.000000,0.000000}%
\pgfsetstrokecolor{currentstroke}%
\pgfsetdash{}{0pt}%
\pgfpathmoveto{\pgfqpoint{2.288101in}{3.019183in}}%
\pgfpathlineto{\pgfqpoint{2.201164in}{2.557857in}}%
\pgfusepath{stroke}%
\end{pgfscope}%
\begin{pgfscope}%
\pgfpathrectangle{\pgfqpoint{0.100000in}{0.212622in}}{\pgfqpoint{3.696000in}{3.696000in}}%
\pgfusepath{clip}%
\pgfsetrectcap%
\pgfsetroundjoin%
\pgfsetlinewidth{1.505625pt}%
\definecolor{currentstroke}{rgb}{1.000000,0.000000,0.000000}%
\pgfsetstrokecolor{currentstroke}%
\pgfsetdash{}{0pt}%
\pgfpathmoveto{\pgfqpoint{2.286020in}{3.019677in}}%
\pgfpathlineto{\pgfqpoint{2.201164in}{2.557857in}}%
\pgfusepath{stroke}%
\end{pgfscope}%
\begin{pgfscope}%
\pgfpathrectangle{\pgfqpoint{0.100000in}{0.212622in}}{\pgfqpoint{3.696000in}{3.696000in}}%
\pgfusepath{clip}%
\pgfsetrectcap%
\pgfsetroundjoin%
\pgfsetlinewidth{1.505625pt}%
\definecolor{currentstroke}{rgb}{1.000000,0.000000,0.000000}%
\pgfsetstrokecolor{currentstroke}%
\pgfsetdash{}{0pt}%
\pgfpathmoveto{\pgfqpoint{2.282020in}{3.020239in}}%
\pgfpathlineto{\pgfqpoint{2.195584in}{2.559397in}}%
\pgfusepath{stroke}%
\end{pgfscope}%
\begin{pgfscope}%
\pgfpathrectangle{\pgfqpoint{0.100000in}{0.212622in}}{\pgfqpoint{3.696000in}{3.696000in}}%
\pgfusepath{clip}%
\pgfsetrectcap%
\pgfsetroundjoin%
\pgfsetlinewidth{1.505625pt}%
\definecolor{currentstroke}{rgb}{1.000000,0.000000,0.000000}%
\pgfsetstrokecolor{currentstroke}%
\pgfsetdash{}{0pt}%
\pgfpathmoveto{\pgfqpoint{2.277549in}{3.021121in}}%
\pgfpathlineto{\pgfqpoint{2.190006in}{2.560937in}}%
\pgfusepath{stroke}%
\end{pgfscope}%
\begin{pgfscope}%
\pgfpathrectangle{\pgfqpoint{0.100000in}{0.212622in}}{\pgfqpoint{3.696000in}{3.696000in}}%
\pgfusepath{clip}%
\pgfsetrectcap%
\pgfsetroundjoin%
\pgfsetlinewidth{1.505625pt}%
\definecolor{currentstroke}{rgb}{1.000000,0.000000,0.000000}%
\pgfsetstrokecolor{currentstroke}%
\pgfsetdash{}{0pt}%
\pgfpathmoveto{\pgfqpoint{2.271772in}{3.022513in}}%
\pgfpathlineto{\pgfqpoint{2.184428in}{2.562477in}}%
\pgfusepath{stroke}%
\end{pgfscope}%
\begin{pgfscope}%
\pgfpathrectangle{\pgfqpoint{0.100000in}{0.212622in}}{\pgfqpoint{3.696000in}{3.696000in}}%
\pgfusepath{clip}%
\pgfsetrectcap%
\pgfsetroundjoin%
\pgfsetlinewidth{1.505625pt}%
\definecolor{currentstroke}{rgb}{1.000000,0.000000,0.000000}%
\pgfsetstrokecolor{currentstroke}%
\pgfsetdash{}{0pt}%
\pgfpathmoveto{\pgfqpoint{2.265483in}{3.021851in}}%
\pgfpathlineto{\pgfqpoint{2.178853in}{2.564016in}}%
\pgfusepath{stroke}%
\end{pgfscope}%
\begin{pgfscope}%
\pgfpathrectangle{\pgfqpoint{0.100000in}{0.212622in}}{\pgfqpoint{3.696000in}{3.696000in}}%
\pgfusepath{clip}%
\pgfsetrectcap%
\pgfsetroundjoin%
\pgfsetlinewidth{1.505625pt}%
\definecolor{currentstroke}{rgb}{1.000000,0.000000,0.000000}%
\pgfsetstrokecolor{currentstroke}%
\pgfsetdash{}{0pt}%
\pgfpathmoveto{\pgfqpoint{2.257971in}{3.022481in}}%
\pgfpathlineto{\pgfqpoint{2.173279in}{2.565555in}}%
\pgfusepath{stroke}%
\end{pgfscope}%
\begin{pgfscope}%
\pgfpathrectangle{\pgfqpoint{0.100000in}{0.212622in}}{\pgfqpoint{3.696000in}{3.696000in}}%
\pgfusepath{clip}%
\pgfsetrectcap%
\pgfsetroundjoin%
\pgfsetlinewidth{1.505625pt}%
\definecolor{currentstroke}{rgb}{1.000000,0.000000,0.000000}%
\pgfsetstrokecolor{currentstroke}%
\pgfsetdash{}{0pt}%
\pgfpathmoveto{\pgfqpoint{2.250098in}{3.024501in}}%
\pgfpathlineto{\pgfqpoint{2.167706in}{2.567093in}}%
\pgfusepath{stroke}%
\end{pgfscope}%
\begin{pgfscope}%
\pgfpathrectangle{\pgfqpoint{0.100000in}{0.212622in}}{\pgfqpoint{3.696000in}{3.696000in}}%
\pgfusepath{clip}%
\pgfsetrectcap%
\pgfsetroundjoin%
\pgfsetlinewidth{1.505625pt}%
\definecolor{currentstroke}{rgb}{1.000000,0.000000,0.000000}%
\pgfsetstrokecolor{currentstroke}%
\pgfsetdash{}{0pt}%
\pgfpathmoveto{\pgfqpoint{2.241712in}{3.026672in}}%
\pgfpathlineto{\pgfqpoint{2.156566in}{2.570169in}}%
\pgfusepath{stroke}%
\end{pgfscope}%
\begin{pgfscope}%
\pgfpathrectangle{\pgfqpoint{0.100000in}{0.212622in}}{\pgfqpoint{3.696000in}{3.696000in}}%
\pgfusepath{clip}%
\pgfsetrectcap%
\pgfsetroundjoin%
\pgfsetlinewidth{1.505625pt}%
\definecolor{currentstroke}{rgb}{1.000000,0.000000,0.000000}%
\pgfsetstrokecolor{currentstroke}%
\pgfsetdash{}{0pt}%
\pgfpathmoveto{\pgfqpoint{2.232950in}{3.027995in}}%
\pgfpathlineto{\pgfqpoint{2.150998in}{2.571706in}}%
\pgfusepath{stroke}%
\end{pgfscope}%
\begin{pgfscope}%
\pgfpathrectangle{\pgfqpoint{0.100000in}{0.212622in}}{\pgfqpoint{3.696000in}{3.696000in}}%
\pgfusepath{clip}%
\pgfsetrectcap%
\pgfsetroundjoin%
\pgfsetlinewidth{1.505625pt}%
\definecolor{currentstroke}{rgb}{1.000000,0.000000,0.000000}%
\pgfsetstrokecolor{currentstroke}%
\pgfsetdash{}{0pt}%
\pgfpathmoveto{\pgfqpoint{2.223379in}{3.027401in}}%
\pgfpathlineto{\pgfqpoint{2.139867in}{2.574779in}}%
\pgfusepath{stroke}%
\end{pgfscope}%
\begin{pgfscope}%
\pgfpathrectangle{\pgfqpoint{0.100000in}{0.212622in}}{\pgfqpoint{3.696000in}{3.696000in}}%
\pgfusepath{clip}%
\pgfsetrectcap%
\pgfsetroundjoin%
\pgfsetlinewidth{1.505625pt}%
\definecolor{currentstroke}{rgb}{1.000000,0.000000,0.000000}%
\pgfsetstrokecolor{currentstroke}%
\pgfsetdash{}{0pt}%
\pgfpathmoveto{\pgfqpoint{2.211667in}{3.030169in}}%
\pgfpathlineto{\pgfqpoint{2.128742in}{2.577851in}}%
\pgfusepath{stroke}%
\end{pgfscope}%
\begin{pgfscope}%
\pgfpathrectangle{\pgfqpoint{0.100000in}{0.212622in}}{\pgfqpoint{3.696000in}{3.696000in}}%
\pgfusepath{clip}%
\pgfsetrectcap%
\pgfsetroundjoin%
\pgfsetlinewidth{1.505625pt}%
\definecolor{currentstroke}{rgb}{1.000000,0.000000,0.000000}%
\pgfsetstrokecolor{currentstroke}%
\pgfsetdash{}{0pt}%
\pgfpathmoveto{\pgfqpoint{2.205387in}{3.030970in}}%
\pgfpathlineto{\pgfqpoint{2.123182in}{2.579386in}}%
\pgfusepath{stroke}%
\end{pgfscope}%
\begin{pgfscope}%
\pgfpathrectangle{\pgfqpoint{0.100000in}{0.212622in}}{\pgfqpoint{3.696000in}{3.696000in}}%
\pgfusepath{clip}%
\pgfsetrectcap%
\pgfsetroundjoin%
\pgfsetlinewidth{1.505625pt}%
\definecolor{currentstroke}{rgb}{1.000000,0.000000,0.000000}%
\pgfsetstrokecolor{currentstroke}%
\pgfsetdash{}{0pt}%
\pgfpathmoveto{\pgfqpoint{2.198212in}{3.032931in}}%
\pgfpathlineto{\pgfqpoint{2.117624in}{2.580920in}}%
\pgfusepath{stroke}%
\end{pgfscope}%
\begin{pgfscope}%
\pgfpathrectangle{\pgfqpoint{0.100000in}{0.212622in}}{\pgfqpoint{3.696000in}{3.696000in}}%
\pgfusepath{clip}%
\pgfsetrectcap%
\pgfsetroundjoin%
\pgfsetlinewidth{1.505625pt}%
\definecolor{currentstroke}{rgb}{1.000000,0.000000,0.000000}%
\pgfsetstrokecolor{currentstroke}%
\pgfsetdash{}{0pt}%
\pgfpathmoveto{\pgfqpoint{2.190894in}{3.032526in}}%
\pgfpathlineto{\pgfqpoint{2.106511in}{2.583988in}}%
\pgfusepath{stroke}%
\end{pgfscope}%
\begin{pgfscope}%
\pgfpathrectangle{\pgfqpoint{0.100000in}{0.212622in}}{\pgfqpoint{3.696000in}{3.696000in}}%
\pgfusepath{clip}%
\pgfsetrectcap%
\pgfsetroundjoin%
\pgfsetlinewidth{1.505625pt}%
\definecolor{currentstroke}{rgb}{1.000000,0.000000,0.000000}%
\pgfsetstrokecolor{currentstroke}%
\pgfsetdash{}{0pt}%
\pgfpathmoveto{\pgfqpoint{2.182185in}{3.033628in}}%
\pgfpathlineto{\pgfqpoint{2.100957in}{2.585521in}}%
\pgfusepath{stroke}%
\end{pgfscope}%
\begin{pgfscope}%
\pgfpathrectangle{\pgfqpoint{0.100000in}{0.212622in}}{\pgfqpoint{3.696000in}{3.696000in}}%
\pgfusepath{clip}%
\pgfsetrectcap%
\pgfsetroundjoin%
\pgfsetlinewidth{1.505625pt}%
\definecolor{currentstroke}{rgb}{1.000000,0.000000,0.000000}%
\pgfsetstrokecolor{currentstroke}%
\pgfsetdash{}{0pt}%
\pgfpathmoveto{\pgfqpoint{2.173233in}{3.036391in}}%
\pgfpathlineto{\pgfqpoint{2.089854in}{2.588587in}}%
\pgfusepath{stroke}%
\end{pgfscope}%
\begin{pgfscope}%
\pgfpathrectangle{\pgfqpoint{0.100000in}{0.212622in}}{\pgfqpoint{3.696000in}{3.696000in}}%
\pgfusepath{clip}%
\pgfsetrectcap%
\pgfsetroundjoin%
\pgfsetlinewidth{1.505625pt}%
\definecolor{currentstroke}{rgb}{1.000000,0.000000,0.000000}%
\pgfsetstrokecolor{currentstroke}%
\pgfsetdash{}{0pt}%
\pgfpathmoveto{\pgfqpoint{2.163411in}{3.038622in}}%
\pgfpathlineto{\pgfqpoint{2.078757in}{2.591650in}}%
\pgfusepath{stroke}%
\end{pgfscope}%
\begin{pgfscope}%
\pgfpathrectangle{\pgfqpoint{0.100000in}{0.212622in}}{\pgfqpoint{3.696000in}{3.696000in}}%
\pgfusepath{clip}%
\pgfsetrectcap%
\pgfsetroundjoin%
\pgfsetlinewidth{1.505625pt}%
\definecolor{currentstroke}{rgb}{1.000000,0.000000,0.000000}%
\pgfsetstrokecolor{currentstroke}%
\pgfsetdash{}{0pt}%
\pgfpathmoveto{\pgfqpoint{2.152952in}{3.039633in}}%
\pgfpathlineto{\pgfqpoint{2.067666in}{2.594712in}}%
\pgfusepath{stroke}%
\end{pgfscope}%
\begin{pgfscope}%
\pgfpathrectangle{\pgfqpoint{0.100000in}{0.212622in}}{\pgfqpoint{3.696000in}{3.696000in}}%
\pgfusepath{clip}%
\pgfsetrectcap%
\pgfsetroundjoin%
\pgfsetlinewidth{1.505625pt}%
\definecolor{currentstroke}{rgb}{1.000000,0.000000,0.000000}%
\pgfsetstrokecolor{currentstroke}%
\pgfsetdash{}{0pt}%
\pgfpathmoveto{\pgfqpoint{2.141996in}{3.040442in}}%
\pgfpathlineto{\pgfqpoint{2.056581in}{2.597773in}}%
\pgfusepath{stroke}%
\end{pgfscope}%
\begin{pgfscope}%
\pgfpathrectangle{\pgfqpoint{0.100000in}{0.212622in}}{\pgfqpoint{3.696000in}{3.696000in}}%
\pgfusepath{clip}%
\pgfsetrectcap%
\pgfsetroundjoin%
\pgfsetlinewidth{1.505625pt}%
\definecolor{currentstroke}{rgb}{1.000000,0.000000,0.000000}%
\pgfsetstrokecolor{currentstroke}%
\pgfsetdash{}{0pt}%
\pgfpathmoveto{\pgfqpoint{2.129331in}{3.043294in}}%
\pgfpathlineto{\pgfqpoint{2.045502in}{2.600831in}}%
\pgfusepath{stroke}%
\end{pgfscope}%
\begin{pgfscope}%
\pgfpathrectangle{\pgfqpoint{0.100000in}{0.212622in}}{\pgfqpoint{3.696000in}{3.696000in}}%
\pgfusepath{clip}%
\pgfsetrectcap%
\pgfsetroundjoin%
\pgfsetlinewidth{1.505625pt}%
\definecolor{currentstroke}{rgb}{1.000000,0.000000,0.000000}%
\pgfsetstrokecolor{currentstroke}%
\pgfsetdash{}{0pt}%
\pgfpathmoveto{\pgfqpoint{2.122215in}{3.044438in}}%
\pgfpathlineto{\pgfqpoint{2.039965in}{2.602360in}}%
\pgfusepath{stroke}%
\end{pgfscope}%
\begin{pgfscope}%
\pgfpathrectangle{\pgfqpoint{0.100000in}{0.212622in}}{\pgfqpoint{3.696000in}{3.696000in}}%
\pgfusepath{clip}%
\pgfsetrectcap%
\pgfsetroundjoin%
\pgfsetlinewidth{1.505625pt}%
\definecolor{currentstroke}{rgb}{1.000000,0.000000,0.000000}%
\pgfsetstrokecolor{currentstroke}%
\pgfsetdash{}{0pt}%
\pgfpathmoveto{\pgfqpoint{2.114798in}{3.046547in}}%
\pgfpathlineto{\pgfqpoint{2.034430in}{2.603888in}}%
\pgfusepath{stroke}%
\end{pgfscope}%
\begin{pgfscope}%
\pgfpathrectangle{\pgfqpoint{0.100000in}{0.212622in}}{\pgfqpoint{3.696000in}{3.696000in}}%
\pgfusepath{clip}%
\pgfsetrectcap%
\pgfsetroundjoin%
\pgfsetlinewidth{1.505625pt}%
\definecolor{currentstroke}{rgb}{1.000000,0.000000,0.000000}%
\pgfsetstrokecolor{currentstroke}%
\pgfsetdash{}{0pt}%
\pgfpathmoveto{\pgfqpoint{2.106665in}{3.046255in}}%
\pgfpathlineto{\pgfqpoint{2.023363in}{2.606943in}}%
\pgfusepath{stroke}%
\end{pgfscope}%
\begin{pgfscope}%
\pgfpathrectangle{\pgfqpoint{0.100000in}{0.212622in}}{\pgfqpoint{3.696000in}{3.696000in}}%
\pgfusepath{clip}%
\pgfsetrectcap%
\pgfsetroundjoin%
\pgfsetlinewidth{1.505625pt}%
\definecolor{currentstroke}{rgb}{1.000000,0.000000,0.000000}%
\pgfsetstrokecolor{currentstroke}%
\pgfsetdash{}{0pt}%
\pgfpathmoveto{\pgfqpoint{2.097885in}{3.045324in}}%
\pgfpathlineto{\pgfqpoint{2.017832in}{2.608470in}}%
\pgfusepath{stroke}%
\end{pgfscope}%
\begin{pgfscope}%
\pgfpathrectangle{\pgfqpoint{0.100000in}{0.212622in}}{\pgfqpoint{3.696000in}{3.696000in}}%
\pgfusepath{clip}%
\pgfsetrectcap%
\pgfsetroundjoin%
\pgfsetlinewidth{1.505625pt}%
\definecolor{currentstroke}{rgb}{1.000000,0.000000,0.000000}%
\pgfsetstrokecolor{currentstroke}%
\pgfsetdash{}{0pt}%
\pgfpathmoveto{\pgfqpoint{2.086918in}{3.047019in}}%
\pgfpathlineto{\pgfqpoint{2.006775in}{2.611523in}}%
\pgfusepath{stroke}%
\end{pgfscope}%
\begin{pgfscope}%
\pgfpathrectangle{\pgfqpoint{0.100000in}{0.212622in}}{\pgfqpoint{3.696000in}{3.696000in}}%
\pgfusepath{clip}%
\pgfsetrectcap%
\pgfsetroundjoin%
\pgfsetlinewidth{1.505625pt}%
\definecolor{currentstroke}{rgb}{1.000000,0.000000,0.000000}%
\pgfsetstrokecolor{currentstroke}%
\pgfsetdash{}{0pt}%
\pgfpathmoveto{\pgfqpoint{2.076405in}{3.048847in}}%
\pgfpathlineto{\pgfqpoint{1.995724in}{2.614574in}}%
\pgfusepath{stroke}%
\end{pgfscope}%
\begin{pgfscope}%
\pgfpathrectangle{\pgfqpoint{0.100000in}{0.212622in}}{\pgfqpoint{3.696000in}{3.696000in}}%
\pgfusepath{clip}%
\pgfsetrectcap%
\pgfsetroundjoin%
\pgfsetlinewidth{1.505625pt}%
\definecolor{currentstroke}{rgb}{1.000000,0.000000,0.000000}%
\pgfsetstrokecolor{currentstroke}%
\pgfsetdash{}{0pt}%
\pgfpathmoveto{\pgfqpoint{2.064585in}{3.051483in}}%
\pgfpathlineto{\pgfqpoint{1.984679in}{2.617623in}}%
\pgfusepath{stroke}%
\end{pgfscope}%
\begin{pgfscope}%
\pgfpathrectangle{\pgfqpoint{0.100000in}{0.212622in}}{\pgfqpoint{3.696000in}{3.696000in}}%
\pgfusepath{clip}%
\pgfsetrectcap%
\pgfsetroundjoin%
\pgfsetlinewidth{1.505625pt}%
\definecolor{currentstroke}{rgb}{1.000000,0.000000,0.000000}%
\pgfsetstrokecolor{currentstroke}%
\pgfsetdash{}{0pt}%
\pgfpathmoveto{\pgfqpoint{2.058112in}{3.050840in}}%
\pgfpathlineto{\pgfqpoint{1.979159in}{2.619147in}}%
\pgfusepath{stroke}%
\end{pgfscope}%
\begin{pgfscope}%
\pgfpathrectangle{\pgfqpoint{0.100000in}{0.212622in}}{\pgfqpoint{3.696000in}{3.696000in}}%
\pgfusepath{clip}%
\pgfsetrectcap%
\pgfsetroundjoin%
\pgfsetlinewidth{1.505625pt}%
\definecolor{currentstroke}{rgb}{1.000000,0.000000,0.000000}%
\pgfsetstrokecolor{currentstroke}%
\pgfsetdash{}{0pt}%
\pgfpathmoveto{\pgfqpoint{2.049256in}{3.052210in}}%
\pgfpathlineto{\pgfqpoint{1.968123in}{2.622194in}}%
\pgfusepath{stroke}%
\end{pgfscope}%
\begin{pgfscope}%
\pgfpathrectangle{\pgfqpoint{0.100000in}{0.212622in}}{\pgfqpoint{3.696000in}{3.696000in}}%
\pgfusepath{clip}%
\pgfsetrectcap%
\pgfsetroundjoin%
\pgfsetlinewidth{1.505625pt}%
\definecolor{currentstroke}{rgb}{1.000000,0.000000,0.000000}%
\pgfsetstrokecolor{currentstroke}%
\pgfsetdash{}{0pt}%
\pgfpathmoveto{\pgfqpoint{2.044560in}{3.052916in}}%
\pgfpathlineto{\pgfqpoint{1.968123in}{2.622194in}}%
\pgfusepath{stroke}%
\end{pgfscope}%
\begin{pgfscope}%
\pgfpathrectangle{\pgfqpoint{0.100000in}{0.212622in}}{\pgfqpoint{3.696000in}{3.696000in}}%
\pgfusepath{clip}%
\pgfsetrectcap%
\pgfsetroundjoin%
\pgfsetlinewidth{1.505625pt}%
\definecolor{currentstroke}{rgb}{1.000000,0.000000,0.000000}%
\pgfsetstrokecolor{currentstroke}%
\pgfsetdash{}{0pt}%
\pgfpathmoveto{\pgfqpoint{2.038299in}{3.053630in}}%
\pgfpathlineto{\pgfqpoint{1.962607in}{2.623717in}}%
\pgfusepath{stroke}%
\end{pgfscope}%
\begin{pgfscope}%
\pgfpathrectangle{\pgfqpoint{0.100000in}{0.212622in}}{\pgfqpoint{3.696000in}{3.696000in}}%
\pgfusepath{clip}%
\pgfsetrectcap%
\pgfsetroundjoin%
\pgfsetlinewidth{1.505625pt}%
\definecolor{currentstroke}{rgb}{1.000000,0.000000,0.000000}%
\pgfsetstrokecolor{currentstroke}%
\pgfsetdash{}{0pt}%
\pgfpathmoveto{\pgfqpoint{2.031982in}{3.052665in}}%
\pgfpathlineto{\pgfqpoint{1.957093in}{2.625239in}}%
\pgfusepath{stroke}%
\end{pgfscope}%
\begin{pgfscope}%
\pgfpathrectangle{\pgfqpoint{0.100000in}{0.212622in}}{\pgfqpoint{3.696000in}{3.696000in}}%
\pgfusepath{clip}%
\pgfsetrectcap%
\pgfsetroundjoin%
\pgfsetlinewidth{1.505625pt}%
\definecolor{currentstroke}{rgb}{1.000000,0.000000,0.000000}%
\pgfsetstrokecolor{currentstroke}%
\pgfsetdash{}{0pt}%
\pgfpathmoveto{\pgfqpoint{2.024717in}{3.052982in}}%
\pgfpathlineto{\pgfqpoint{1.957093in}{2.625239in}}%
\pgfusepath{stroke}%
\end{pgfscope}%
\begin{pgfscope}%
\pgfpathrectangle{\pgfqpoint{0.100000in}{0.212622in}}{\pgfqpoint{3.696000in}{3.696000in}}%
\pgfusepath{clip}%
\pgfsetrectcap%
\pgfsetroundjoin%
\pgfsetlinewidth{1.505625pt}%
\definecolor{currentstroke}{rgb}{1.000000,0.000000,0.000000}%
\pgfsetstrokecolor{currentstroke}%
\pgfsetdash{}{0pt}%
\pgfpathmoveto{\pgfqpoint{2.016120in}{3.055509in}}%
\pgfpathlineto{\pgfqpoint{1.957093in}{2.625239in}}%
\pgfusepath{stroke}%
\end{pgfscope}%
\begin{pgfscope}%
\pgfpathrectangle{\pgfqpoint{0.100000in}{0.212622in}}{\pgfqpoint{3.696000in}{3.696000in}}%
\pgfusepath{clip}%
\pgfsetrectcap%
\pgfsetroundjoin%
\pgfsetlinewidth{1.505625pt}%
\definecolor{currentstroke}{rgb}{1.000000,0.000000,0.000000}%
\pgfsetstrokecolor{currentstroke}%
\pgfsetdash{}{0pt}%
\pgfpathmoveto{\pgfqpoint{2.006866in}{3.056284in}}%
\pgfpathlineto{\pgfqpoint{1.957093in}{2.625239in}}%
\pgfusepath{stroke}%
\end{pgfscope}%
\begin{pgfscope}%
\pgfpathrectangle{\pgfqpoint{0.100000in}{0.212622in}}{\pgfqpoint{3.696000in}{3.696000in}}%
\pgfusepath{clip}%
\pgfsetrectcap%
\pgfsetroundjoin%
\pgfsetlinewidth{1.505625pt}%
\definecolor{currentstroke}{rgb}{1.000000,0.000000,0.000000}%
\pgfsetstrokecolor{currentstroke}%
\pgfsetdash{}{0pt}%
\pgfpathmoveto{\pgfqpoint{1.996705in}{3.057753in}}%
\pgfpathlineto{\pgfqpoint{1.957093in}{2.625239in}}%
\pgfusepath{stroke}%
\end{pgfscope}%
\begin{pgfscope}%
\pgfpathrectangle{\pgfqpoint{0.100000in}{0.212622in}}{\pgfqpoint{3.696000in}{3.696000in}}%
\pgfusepath{clip}%
\pgfsetrectcap%
\pgfsetroundjoin%
\pgfsetlinewidth{1.505625pt}%
\definecolor{currentstroke}{rgb}{1.000000,0.000000,0.000000}%
\pgfsetstrokecolor{currentstroke}%
\pgfsetdash{}{0pt}%
\pgfpathmoveto{\pgfqpoint{1.991045in}{3.057233in}}%
\pgfpathlineto{\pgfqpoint{1.957093in}{2.625239in}}%
\pgfusepath{stroke}%
\end{pgfscope}%
\begin{pgfscope}%
\pgfpathrectangle{\pgfqpoint{0.100000in}{0.212622in}}{\pgfqpoint{3.696000in}{3.696000in}}%
\pgfusepath{clip}%
\pgfsetrectcap%
\pgfsetroundjoin%
\pgfsetlinewidth{1.505625pt}%
\definecolor{currentstroke}{rgb}{1.000000,0.000000,0.000000}%
\pgfsetstrokecolor{currentstroke}%
\pgfsetdash{}{0pt}%
\pgfpathmoveto{\pgfqpoint{1.983835in}{3.059350in}}%
\pgfpathlineto{\pgfqpoint{1.957093in}{2.625239in}}%
\pgfusepath{stroke}%
\end{pgfscope}%
\begin{pgfscope}%
\pgfpathrectangle{\pgfqpoint{0.100000in}{0.212622in}}{\pgfqpoint{3.696000in}{3.696000in}}%
\pgfusepath{clip}%
\pgfsetrectcap%
\pgfsetroundjoin%
\pgfsetlinewidth{1.505625pt}%
\definecolor{currentstroke}{rgb}{1.000000,0.000000,0.000000}%
\pgfsetstrokecolor{currentstroke}%
\pgfsetdash{}{0pt}%
\pgfpathmoveto{\pgfqpoint{1.980385in}{3.060581in}}%
\pgfpathlineto{\pgfqpoint{1.957093in}{2.625239in}}%
\pgfusepath{stroke}%
\end{pgfscope}%
\begin{pgfscope}%
\pgfpathrectangle{\pgfqpoint{0.100000in}{0.212622in}}{\pgfqpoint{3.696000in}{3.696000in}}%
\pgfusepath{clip}%
\pgfsetrectcap%
\pgfsetroundjoin%
\pgfsetlinewidth{1.505625pt}%
\definecolor{currentstroke}{rgb}{1.000000,0.000000,0.000000}%
\pgfsetstrokecolor{currentstroke}%
\pgfsetdash{}{0pt}%
\pgfpathmoveto{\pgfqpoint{1.975734in}{3.062014in}}%
\pgfpathlineto{\pgfqpoint{1.957093in}{2.625239in}}%
\pgfusepath{stroke}%
\end{pgfscope}%
\begin{pgfscope}%
\pgfpathrectangle{\pgfqpoint{0.100000in}{0.212622in}}{\pgfqpoint{3.696000in}{3.696000in}}%
\pgfusepath{clip}%
\pgfsetrectcap%
\pgfsetroundjoin%
\pgfsetlinewidth{1.505625pt}%
\definecolor{currentstroke}{rgb}{1.000000,0.000000,0.000000}%
\pgfsetstrokecolor{currentstroke}%
\pgfsetdash{}{0pt}%
\pgfpathmoveto{\pgfqpoint{1.973090in}{3.062240in}}%
\pgfpathlineto{\pgfqpoint{1.957093in}{2.625239in}}%
\pgfusepath{stroke}%
\end{pgfscope}%
\begin{pgfscope}%
\pgfpathrectangle{\pgfqpoint{0.100000in}{0.212622in}}{\pgfqpoint{3.696000in}{3.696000in}}%
\pgfusepath{clip}%
\pgfsetrectcap%
\pgfsetroundjoin%
\pgfsetlinewidth{1.505625pt}%
\definecolor{currentstroke}{rgb}{1.000000,0.000000,0.000000}%
\pgfsetstrokecolor{currentstroke}%
\pgfsetdash{}{0pt}%
\pgfpathmoveto{\pgfqpoint{1.969067in}{3.063605in}}%
\pgfpathlineto{\pgfqpoint{1.957093in}{2.625239in}}%
\pgfusepath{stroke}%
\end{pgfscope}%
\begin{pgfscope}%
\pgfpathrectangle{\pgfqpoint{0.100000in}{0.212622in}}{\pgfqpoint{3.696000in}{3.696000in}}%
\pgfusepath{clip}%
\pgfsetrectcap%
\pgfsetroundjoin%
\pgfsetlinewidth{1.505625pt}%
\definecolor{currentstroke}{rgb}{1.000000,0.000000,0.000000}%
\pgfsetstrokecolor{currentstroke}%
\pgfsetdash{}{0pt}%
\pgfpathmoveto{\pgfqpoint{1.966744in}{3.064039in}}%
\pgfpathlineto{\pgfqpoint{1.957093in}{2.625239in}}%
\pgfusepath{stroke}%
\end{pgfscope}%
\begin{pgfscope}%
\pgfpathrectangle{\pgfqpoint{0.100000in}{0.212622in}}{\pgfqpoint{3.696000in}{3.696000in}}%
\pgfusepath{clip}%
\pgfsetrectcap%
\pgfsetroundjoin%
\pgfsetlinewidth{1.505625pt}%
\definecolor{currentstroke}{rgb}{1.000000,0.000000,0.000000}%
\pgfsetstrokecolor{currentstroke}%
\pgfsetdash{}{0pt}%
\pgfpathmoveto{\pgfqpoint{1.963071in}{3.064913in}}%
\pgfpathlineto{\pgfqpoint{1.957093in}{2.625239in}}%
\pgfusepath{stroke}%
\end{pgfscope}%
\begin{pgfscope}%
\pgfpathrectangle{\pgfqpoint{0.100000in}{0.212622in}}{\pgfqpoint{3.696000in}{3.696000in}}%
\pgfusepath{clip}%
\pgfsetrectcap%
\pgfsetroundjoin%
\pgfsetlinewidth{1.505625pt}%
\definecolor{currentstroke}{rgb}{1.000000,0.000000,0.000000}%
\pgfsetstrokecolor{currentstroke}%
\pgfsetdash{}{0pt}%
\pgfpathmoveto{\pgfqpoint{1.961073in}{3.064966in}}%
\pgfpathlineto{\pgfqpoint{1.957093in}{2.625239in}}%
\pgfusepath{stroke}%
\end{pgfscope}%
\begin{pgfscope}%
\pgfpathrectangle{\pgfqpoint{0.100000in}{0.212622in}}{\pgfqpoint{3.696000in}{3.696000in}}%
\pgfusepath{clip}%
\pgfsetrectcap%
\pgfsetroundjoin%
\pgfsetlinewidth{1.505625pt}%
\definecolor{currentstroke}{rgb}{1.000000,0.000000,0.000000}%
\pgfsetstrokecolor{currentstroke}%
\pgfsetdash{}{0pt}%
\pgfpathmoveto{\pgfqpoint{1.957884in}{3.065514in}}%
\pgfpathlineto{\pgfqpoint{1.957093in}{2.625239in}}%
\pgfusepath{stroke}%
\end{pgfscope}%
\begin{pgfscope}%
\pgfpathrectangle{\pgfqpoint{0.100000in}{0.212622in}}{\pgfqpoint{3.696000in}{3.696000in}}%
\pgfusepath{clip}%
\pgfsetrectcap%
\pgfsetroundjoin%
\pgfsetlinewidth{1.505625pt}%
\definecolor{currentstroke}{rgb}{1.000000,0.000000,0.000000}%
\pgfsetstrokecolor{currentstroke}%
\pgfsetdash{}{0pt}%
\pgfpathmoveto{\pgfqpoint{1.956148in}{3.066047in}}%
\pgfpathlineto{\pgfqpoint{1.957093in}{2.625239in}}%
\pgfusepath{stroke}%
\end{pgfscope}%
\begin{pgfscope}%
\pgfpathrectangle{\pgfqpoint{0.100000in}{0.212622in}}{\pgfqpoint{3.696000in}{3.696000in}}%
\pgfusepath{clip}%
\pgfsetrectcap%
\pgfsetroundjoin%
\pgfsetlinewidth{1.505625pt}%
\definecolor{currentstroke}{rgb}{1.000000,0.000000,0.000000}%
\pgfsetstrokecolor{currentstroke}%
\pgfsetdash{}{0pt}%
\pgfpathmoveto{\pgfqpoint{1.952892in}{3.066646in}}%
\pgfpathlineto{\pgfqpoint{1.957093in}{2.625239in}}%
\pgfusepath{stroke}%
\end{pgfscope}%
\begin{pgfscope}%
\pgfpathrectangle{\pgfqpoint{0.100000in}{0.212622in}}{\pgfqpoint{3.696000in}{3.696000in}}%
\pgfusepath{clip}%
\pgfsetrectcap%
\pgfsetroundjoin%
\pgfsetlinewidth{1.505625pt}%
\definecolor{currentstroke}{rgb}{1.000000,0.000000,0.000000}%
\pgfsetstrokecolor{currentstroke}%
\pgfsetdash{}{0pt}%
\pgfpathmoveto{\pgfqpoint{1.951143in}{3.066274in}}%
\pgfpathlineto{\pgfqpoint{1.957093in}{2.625239in}}%
\pgfusepath{stroke}%
\end{pgfscope}%
\begin{pgfscope}%
\pgfpathrectangle{\pgfqpoint{0.100000in}{0.212622in}}{\pgfqpoint{3.696000in}{3.696000in}}%
\pgfusepath{clip}%
\pgfsetrectcap%
\pgfsetroundjoin%
\pgfsetlinewidth{1.505625pt}%
\definecolor{currentstroke}{rgb}{1.000000,0.000000,0.000000}%
\pgfsetstrokecolor{currentstroke}%
\pgfsetdash{}{0pt}%
\pgfpathmoveto{\pgfqpoint{1.947527in}{3.066995in}}%
\pgfpathlineto{\pgfqpoint{1.957093in}{2.625239in}}%
\pgfusepath{stroke}%
\end{pgfscope}%
\begin{pgfscope}%
\pgfpathrectangle{\pgfqpoint{0.100000in}{0.212622in}}{\pgfqpoint{3.696000in}{3.696000in}}%
\pgfusepath{clip}%
\pgfsetrectcap%
\pgfsetroundjoin%
\pgfsetlinewidth{1.505625pt}%
\definecolor{currentstroke}{rgb}{1.000000,0.000000,0.000000}%
\pgfsetstrokecolor{currentstroke}%
\pgfsetdash{}{0pt}%
\pgfpathmoveto{\pgfqpoint{1.945519in}{3.067204in}}%
\pgfpathlineto{\pgfqpoint{1.957093in}{2.625239in}}%
\pgfusepath{stroke}%
\end{pgfscope}%
\begin{pgfscope}%
\pgfpathrectangle{\pgfqpoint{0.100000in}{0.212622in}}{\pgfqpoint{3.696000in}{3.696000in}}%
\pgfusepath{clip}%
\pgfsetrectcap%
\pgfsetroundjoin%
\pgfsetlinewidth{1.505625pt}%
\definecolor{currentstroke}{rgb}{1.000000,0.000000,0.000000}%
\pgfsetstrokecolor{currentstroke}%
\pgfsetdash{}{0pt}%
\pgfpathmoveto{\pgfqpoint{1.942052in}{3.067533in}}%
\pgfpathlineto{\pgfqpoint{1.775978in}{2.675241in}}%
\pgfusepath{stroke}%
\end{pgfscope}%
\begin{pgfscope}%
\pgfpathrectangle{\pgfqpoint{0.100000in}{0.212622in}}{\pgfqpoint{3.696000in}{3.696000in}}%
\pgfusepath{clip}%
\pgfsetrectcap%
\pgfsetroundjoin%
\pgfsetlinewidth{1.505625pt}%
\definecolor{currentstroke}{rgb}{1.000000,0.000000,0.000000}%
\pgfsetstrokecolor{currentstroke}%
\pgfsetdash{}{0pt}%
\pgfpathmoveto{\pgfqpoint{1.938243in}{3.067561in}}%
\pgfpathlineto{\pgfqpoint{1.775978in}{2.675241in}}%
\pgfusepath{stroke}%
\end{pgfscope}%
\begin{pgfscope}%
\pgfpathrectangle{\pgfqpoint{0.100000in}{0.212622in}}{\pgfqpoint{3.696000in}{3.696000in}}%
\pgfusepath{clip}%
\pgfsetrectcap%
\pgfsetroundjoin%
\pgfsetlinewidth{1.505625pt}%
\definecolor{currentstroke}{rgb}{1.000000,0.000000,0.000000}%
\pgfsetstrokecolor{currentstroke}%
\pgfsetdash{}{0pt}%
\pgfpathmoveto{\pgfqpoint{1.932804in}{3.067998in}}%
\pgfpathlineto{\pgfqpoint{1.775978in}{2.675241in}}%
\pgfusepath{stroke}%
\end{pgfscope}%
\begin{pgfscope}%
\pgfpathrectangle{\pgfqpoint{0.100000in}{0.212622in}}{\pgfqpoint{3.696000in}{3.696000in}}%
\pgfusepath{clip}%
\pgfsetrectcap%
\pgfsetroundjoin%
\pgfsetlinewidth{1.505625pt}%
\definecolor{currentstroke}{rgb}{1.000000,0.000000,0.000000}%
\pgfsetstrokecolor{currentstroke}%
\pgfsetdash{}{0pt}%
\pgfpathmoveto{\pgfqpoint{1.927108in}{3.069625in}}%
\pgfpathlineto{\pgfqpoint{1.775978in}{2.675241in}}%
\pgfusepath{stroke}%
\end{pgfscope}%
\begin{pgfscope}%
\pgfpathrectangle{\pgfqpoint{0.100000in}{0.212622in}}{\pgfqpoint{3.696000in}{3.696000in}}%
\pgfusepath{clip}%
\pgfsetrectcap%
\pgfsetroundjoin%
\pgfsetlinewidth{1.505625pt}%
\definecolor{currentstroke}{rgb}{1.000000,0.000000,0.000000}%
\pgfsetstrokecolor{currentstroke}%
\pgfsetdash{}{0pt}%
\pgfpathmoveto{\pgfqpoint{1.919797in}{3.070917in}}%
\pgfpathlineto{\pgfqpoint{1.775978in}{2.675241in}}%
\pgfusepath{stroke}%
\end{pgfscope}%
\begin{pgfscope}%
\pgfpathrectangle{\pgfqpoint{0.100000in}{0.212622in}}{\pgfqpoint{3.696000in}{3.696000in}}%
\pgfusepath{clip}%
\pgfsetrectcap%
\pgfsetroundjoin%
\pgfsetlinewidth{1.505625pt}%
\definecolor{currentstroke}{rgb}{1.000000,0.000000,0.000000}%
\pgfsetstrokecolor{currentstroke}%
\pgfsetdash{}{0pt}%
\pgfpathmoveto{\pgfqpoint{1.912141in}{3.070212in}}%
\pgfpathlineto{\pgfqpoint{1.775978in}{2.675241in}}%
\pgfusepath{stroke}%
\end{pgfscope}%
\begin{pgfscope}%
\pgfpathrectangle{\pgfqpoint{0.100000in}{0.212622in}}{\pgfqpoint{3.696000in}{3.696000in}}%
\pgfusepath{clip}%
\pgfsetrectcap%
\pgfsetroundjoin%
\pgfsetlinewidth{1.505625pt}%
\definecolor{currentstroke}{rgb}{1.000000,0.000000,0.000000}%
\pgfsetstrokecolor{currentstroke}%
\pgfsetdash{}{0pt}%
\pgfpathmoveto{\pgfqpoint{1.902383in}{3.070383in}}%
\pgfpathlineto{\pgfqpoint{1.775978in}{2.675241in}}%
\pgfusepath{stroke}%
\end{pgfscope}%
\begin{pgfscope}%
\pgfpathrectangle{\pgfqpoint{0.100000in}{0.212622in}}{\pgfqpoint{3.696000in}{3.696000in}}%
\pgfusepath{clip}%
\pgfsetrectcap%
\pgfsetroundjoin%
\pgfsetlinewidth{1.505625pt}%
\definecolor{currentstroke}{rgb}{1.000000,0.000000,0.000000}%
\pgfsetstrokecolor{currentstroke}%
\pgfsetdash{}{0pt}%
\pgfpathmoveto{\pgfqpoint{1.892033in}{3.072331in}}%
\pgfpathlineto{\pgfqpoint{1.775978in}{2.675241in}}%
\pgfusepath{stroke}%
\end{pgfscope}%
\begin{pgfscope}%
\pgfpathrectangle{\pgfqpoint{0.100000in}{0.212622in}}{\pgfqpoint{3.696000in}{3.696000in}}%
\pgfusepath{clip}%
\pgfsetrectcap%
\pgfsetroundjoin%
\pgfsetlinewidth{1.505625pt}%
\definecolor{currentstroke}{rgb}{1.000000,0.000000,0.000000}%
\pgfsetstrokecolor{currentstroke}%
\pgfsetdash{}{0pt}%
\pgfpathmoveto{\pgfqpoint{1.881184in}{3.075068in}}%
\pgfpathlineto{\pgfqpoint{1.775978in}{2.675241in}}%
\pgfusepath{stroke}%
\end{pgfscope}%
\begin{pgfscope}%
\pgfpathrectangle{\pgfqpoint{0.100000in}{0.212622in}}{\pgfqpoint{3.696000in}{3.696000in}}%
\pgfusepath{clip}%
\pgfsetrectcap%
\pgfsetroundjoin%
\pgfsetlinewidth{1.505625pt}%
\definecolor{currentstroke}{rgb}{1.000000,0.000000,0.000000}%
\pgfsetstrokecolor{currentstroke}%
\pgfsetdash{}{0pt}%
\pgfpathmoveto{\pgfqpoint{1.869810in}{3.073925in}}%
\pgfpathlineto{\pgfqpoint{1.775978in}{2.675241in}}%
\pgfusepath{stroke}%
\end{pgfscope}%
\begin{pgfscope}%
\pgfpathrectangle{\pgfqpoint{0.100000in}{0.212622in}}{\pgfqpoint{3.696000in}{3.696000in}}%
\pgfusepath{clip}%
\pgfsetrectcap%
\pgfsetroundjoin%
\pgfsetlinewidth{1.505625pt}%
\definecolor{currentstroke}{rgb}{1.000000,0.000000,0.000000}%
\pgfsetstrokecolor{currentstroke}%
\pgfsetdash{}{0pt}%
\pgfpathmoveto{\pgfqpoint{1.858077in}{3.073761in}}%
\pgfpathlineto{\pgfqpoint{1.775978in}{2.675241in}}%
\pgfusepath{stroke}%
\end{pgfscope}%
\begin{pgfscope}%
\pgfpathrectangle{\pgfqpoint{0.100000in}{0.212622in}}{\pgfqpoint{3.696000in}{3.696000in}}%
\pgfusepath{clip}%
\pgfsetrectcap%
\pgfsetroundjoin%
\pgfsetlinewidth{1.505625pt}%
\definecolor{currentstroke}{rgb}{1.000000,0.000000,0.000000}%
\pgfsetstrokecolor{currentstroke}%
\pgfsetdash{}{0pt}%
\pgfpathmoveto{\pgfqpoint{1.845005in}{3.077728in}}%
\pgfpathlineto{\pgfqpoint{1.775978in}{2.675241in}}%
\pgfusepath{stroke}%
\end{pgfscope}%
\begin{pgfscope}%
\pgfpathrectangle{\pgfqpoint{0.100000in}{0.212622in}}{\pgfqpoint{3.696000in}{3.696000in}}%
\pgfusepath{clip}%
\pgfsetrectcap%
\pgfsetroundjoin%
\pgfsetlinewidth{1.505625pt}%
\definecolor{currentstroke}{rgb}{1.000000,0.000000,0.000000}%
\pgfsetstrokecolor{currentstroke}%
\pgfsetdash{}{0pt}%
\pgfpathmoveto{\pgfqpoint{1.831756in}{3.080641in}}%
\pgfpathlineto{\pgfqpoint{1.775978in}{2.675241in}}%
\pgfusepath{stroke}%
\end{pgfscope}%
\begin{pgfscope}%
\pgfpathrectangle{\pgfqpoint{0.100000in}{0.212622in}}{\pgfqpoint{3.696000in}{3.696000in}}%
\pgfusepath{clip}%
\pgfsetrectcap%
\pgfsetroundjoin%
\pgfsetlinewidth{1.505625pt}%
\definecolor{currentstroke}{rgb}{1.000000,0.000000,0.000000}%
\pgfsetstrokecolor{currentstroke}%
\pgfsetdash{}{0pt}%
\pgfpathmoveto{\pgfqpoint{1.817321in}{3.082479in}}%
\pgfpathlineto{\pgfqpoint{1.775978in}{2.675241in}}%
\pgfusepath{stroke}%
\end{pgfscope}%
\begin{pgfscope}%
\pgfpathrectangle{\pgfqpoint{0.100000in}{0.212622in}}{\pgfqpoint{3.696000in}{3.696000in}}%
\pgfusepath{clip}%
\pgfsetrectcap%
\pgfsetroundjoin%
\pgfsetlinewidth{1.505625pt}%
\definecolor{currentstroke}{rgb}{1.000000,0.000000,0.000000}%
\pgfsetstrokecolor{currentstroke}%
\pgfsetdash{}{0pt}%
\pgfpathmoveto{\pgfqpoint{1.802774in}{3.084451in}}%
\pgfpathlineto{\pgfqpoint{1.775978in}{2.675241in}}%
\pgfusepath{stroke}%
\end{pgfscope}%
\begin{pgfscope}%
\pgfpathrectangle{\pgfqpoint{0.100000in}{0.212622in}}{\pgfqpoint{3.696000in}{3.696000in}}%
\pgfusepath{clip}%
\pgfsetrectcap%
\pgfsetroundjoin%
\pgfsetlinewidth{1.505625pt}%
\definecolor{currentstroke}{rgb}{1.000000,0.000000,0.000000}%
\pgfsetstrokecolor{currentstroke}%
\pgfsetdash{}{0pt}%
\pgfpathmoveto{\pgfqpoint{1.787455in}{3.088920in}}%
\pgfpathlineto{\pgfqpoint{1.775978in}{2.675241in}}%
\pgfusepath{stroke}%
\end{pgfscope}%
\begin{pgfscope}%
\pgfpathrectangle{\pgfqpoint{0.100000in}{0.212622in}}{\pgfqpoint{3.696000in}{3.696000in}}%
\pgfusepath{clip}%
\pgfsetrectcap%
\pgfsetroundjoin%
\pgfsetlinewidth{1.505625pt}%
\definecolor{currentstroke}{rgb}{1.000000,0.000000,0.000000}%
\pgfsetstrokecolor{currentstroke}%
\pgfsetdash{}{0pt}%
\pgfpathmoveto{\pgfqpoint{1.772057in}{3.092445in}}%
\pgfpathlineto{\pgfqpoint{1.775978in}{2.675241in}}%
\pgfusepath{stroke}%
\end{pgfscope}%
\begin{pgfscope}%
\pgfpathrectangle{\pgfqpoint{0.100000in}{0.212622in}}{\pgfqpoint{3.696000in}{3.696000in}}%
\pgfusepath{clip}%
\pgfsetrectcap%
\pgfsetroundjoin%
\pgfsetlinewidth{1.505625pt}%
\definecolor{currentstroke}{rgb}{1.000000,0.000000,0.000000}%
\pgfsetstrokecolor{currentstroke}%
\pgfsetdash{}{0pt}%
\pgfpathmoveto{\pgfqpoint{1.755477in}{3.095973in}}%
\pgfpathlineto{\pgfqpoint{1.775978in}{2.675241in}}%
\pgfusepath{stroke}%
\end{pgfscope}%
\begin{pgfscope}%
\pgfpathrectangle{\pgfqpoint{0.100000in}{0.212622in}}{\pgfqpoint{3.696000in}{3.696000in}}%
\pgfusepath{clip}%
\pgfsetrectcap%
\pgfsetroundjoin%
\pgfsetlinewidth{1.505625pt}%
\definecolor{currentstroke}{rgb}{1.000000,0.000000,0.000000}%
\pgfsetstrokecolor{currentstroke}%
\pgfsetdash{}{0pt}%
\pgfpathmoveto{\pgfqpoint{1.739213in}{3.099397in}}%
\pgfpathlineto{\pgfqpoint{1.775978in}{2.675241in}}%
\pgfusepath{stroke}%
\end{pgfscope}%
\begin{pgfscope}%
\pgfpathrectangle{\pgfqpoint{0.100000in}{0.212622in}}{\pgfqpoint{3.696000in}{3.696000in}}%
\pgfusepath{clip}%
\pgfsetrectcap%
\pgfsetroundjoin%
\pgfsetlinewidth{1.505625pt}%
\definecolor{currentstroke}{rgb}{1.000000,0.000000,0.000000}%
\pgfsetstrokecolor{currentstroke}%
\pgfsetdash{}{0pt}%
\pgfpathmoveto{\pgfqpoint{1.729956in}{3.101455in}}%
\pgfpathlineto{\pgfqpoint{1.775978in}{2.675241in}}%
\pgfusepath{stroke}%
\end{pgfscope}%
\begin{pgfscope}%
\pgfpathrectangle{\pgfqpoint{0.100000in}{0.212622in}}{\pgfqpoint{3.696000in}{3.696000in}}%
\pgfusepath{clip}%
\pgfsetrectcap%
\pgfsetroundjoin%
\pgfsetlinewidth{1.505625pt}%
\definecolor{currentstroke}{rgb}{1.000000,0.000000,0.000000}%
\pgfsetstrokecolor{currentstroke}%
\pgfsetdash{}{0pt}%
\pgfpathmoveto{\pgfqpoint{1.724985in}{3.102643in}}%
\pgfpathlineto{\pgfqpoint{1.775978in}{2.675241in}}%
\pgfusepath{stroke}%
\end{pgfscope}%
\begin{pgfscope}%
\pgfpathrectangle{\pgfqpoint{0.100000in}{0.212622in}}{\pgfqpoint{3.696000in}{3.696000in}}%
\pgfusepath{clip}%
\pgfsetrectcap%
\pgfsetroundjoin%
\pgfsetlinewidth{1.505625pt}%
\definecolor{currentstroke}{rgb}{1.000000,0.000000,0.000000}%
\pgfsetstrokecolor{currentstroke}%
\pgfsetdash{}{0pt}%
\pgfpathmoveto{\pgfqpoint{1.722264in}{3.103388in}}%
\pgfpathlineto{\pgfqpoint{1.775978in}{2.675241in}}%
\pgfusepath{stroke}%
\end{pgfscope}%
\begin{pgfscope}%
\pgfpathrectangle{\pgfqpoint{0.100000in}{0.212622in}}{\pgfqpoint{3.696000in}{3.696000in}}%
\pgfusepath{clip}%
\pgfsetrectcap%
\pgfsetroundjoin%
\pgfsetlinewidth{1.505625pt}%
\definecolor{currentstroke}{rgb}{1.000000,0.000000,0.000000}%
\pgfsetstrokecolor{currentstroke}%
\pgfsetdash{}{0pt}%
\pgfpathmoveto{\pgfqpoint{1.720755in}{3.103750in}}%
\pgfpathlineto{\pgfqpoint{1.775978in}{2.675241in}}%
\pgfusepath{stroke}%
\end{pgfscope}%
\begin{pgfscope}%
\pgfpathrectangle{\pgfqpoint{0.100000in}{0.212622in}}{\pgfqpoint{3.696000in}{3.696000in}}%
\pgfusepath{clip}%
\pgfsetrectcap%
\pgfsetroundjoin%
\pgfsetlinewidth{1.505625pt}%
\definecolor{currentstroke}{rgb}{1.000000,0.000000,0.000000}%
\pgfsetstrokecolor{currentstroke}%
\pgfsetdash{}{0pt}%
\pgfpathmoveto{\pgfqpoint{1.719913in}{3.103918in}}%
\pgfpathlineto{\pgfqpoint{1.775978in}{2.675241in}}%
\pgfusepath{stroke}%
\end{pgfscope}%
\begin{pgfscope}%
\pgfpathrectangle{\pgfqpoint{0.100000in}{0.212622in}}{\pgfqpoint{3.696000in}{3.696000in}}%
\pgfusepath{clip}%
\pgfsetrectcap%
\pgfsetroundjoin%
\pgfsetlinewidth{1.505625pt}%
\definecolor{currentstroke}{rgb}{1.000000,0.000000,0.000000}%
\pgfsetstrokecolor{currentstroke}%
\pgfsetdash{}{0pt}%
\pgfpathmoveto{\pgfqpoint{1.719474in}{3.104037in}}%
\pgfpathlineto{\pgfqpoint{1.775978in}{2.675241in}}%
\pgfusepath{stroke}%
\end{pgfscope}%
\begin{pgfscope}%
\pgfpathrectangle{\pgfqpoint{0.100000in}{0.212622in}}{\pgfqpoint{3.696000in}{3.696000in}}%
\pgfusepath{clip}%
\pgfsetrectcap%
\pgfsetroundjoin%
\pgfsetlinewidth{1.505625pt}%
\definecolor{currentstroke}{rgb}{1.000000,0.000000,0.000000}%
\pgfsetstrokecolor{currentstroke}%
\pgfsetdash{}{0pt}%
\pgfpathmoveto{\pgfqpoint{1.719217in}{3.104078in}}%
\pgfpathlineto{\pgfqpoint{1.775978in}{2.675241in}}%
\pgfusepath{stroke}%
\end{pgfscope}%
\begin{pgfscope}%
\pgfpathrectangle{\pgfqpoint{0.100000in}{0.212622in}}{\pgfqpoint{3.696000in}{3.696000in}}%
\pgfusepath{clip}%
\pgfsetrectcap%
\pgfsetroundjoin%
\pgfsetlinewidth{1.505625pt}%
\definecolor{currentstroke}{rgb}{1.000000,0.000000,0.000000}%
\pgfsetstrokecolor{currentstroke}%
\pgfsetdash{}{0pt}%
\pgfpathmoveto{\pgfqpoint{1.719082in}{3.104113in}}%
\pgfpathlineto{\pgfqpoint{1.775978in}{2.675241in}}%
\pgfusepath{stroke}%
\end{pgfscope}%
\begin{pgfscope}%
\pgfpathrectangle{\pgfqpoint{0.100000in}{0.212622in}}{\pgfqpoint{3.696000in}{3.696000in}}%
\pgfusepath{clip}%
\pgfsetrectcap%
\pgfsetroundjoin%
\pgfsetlinewidth{1.505625pt}%
\definecolor{currentstroke}{rgb}{1.000000,0.000000,0.000000}%
\pgfsetstrokecolor{currentstroke}%
\pgfsetdash{}{0pt}%
\pgfpathmoveto{\pgfqpoint{1.719006in}{3.104131in}}%
\pgfpathlineto{\pgfqpoint{1.775978in}{2.675241in}}%
\pgfusepath{stroke}%
\end{pgfscope}%
\begin{pgfscope}%
\pgfpathrectangle{\pgfqpoint{0.100000in}{0.212622in}}{\pgfqpoint{3.696000in}{3.696000in}}%
\pgfusepath{clip}%
\pgfsetrectcap%
\pgfsetroundjoin%
\pgfsetlinewidth{1.505625pt}%
\definecolor{currentstroke}{rgb}{1.000000,0.000000,0.000000}%
\pgfsetstrokecolor{currentstroke}%
\pgfsetdash{}{0pt}%
\pgfpathmoveto{\pgfqpoint{1.718966in}{3.104144in}}%
\pgfpathlineto{\pgfqpoint{1.775978in}{2.675241in}}%
\pgfusepath{stroke}%
\end{pgfscope}%
\begin{pgfscope}%
\pgfpathrectangle{\pgfqpoint{0.100000in}{0.212622in}}{\pgfqpoint{3.696000in}{3.696000in}}%
\pgfusepath{clip}%
\pgfsetrectcap%
\pgfsetroundjoin%
\pgfsetlinewidth{1.505625pt}%
\definecolor{currentstroke}{rgb}{1.000000,0.000000,0.000000}%
\pgfsetstrokecolor{currentstroke}%
\pgfsetdash{}{0pt}%
\pgfpathmoveto{\pgfqpoint{1.718942in}{3.104148in}}%
\pgfpathlineto{\pgfqpoint{1.775978in}{2.675241in}}%
\pgfusepath{stroke}%
\end{pgfscope}%
\begin{pgfscope}%
\pgfpathrectangle{\pgfqpoint{0.100000in}{0.212622in}}{\pgfqpoint{3.696000in}{3.696000in}}%
\pgfusepath{clip}%
\pgfsetrectcap%
\pgfsetroundjoin%
\pgfsetlinewidth{1.505625pt}%
\definecolor{currentstroke}{rgb}{1.000000,0.000000,0.000000}%
\pgfsetstrokecolor{currentstroke}%
\pgfsetdash{}{0pt}%
\pgfpathmoveto{\pgfqpoint{1.718929in}{3.104150in}}%
\pgfpathlineto{\pgfqpoint{1.775978in}{2.675241in}}%
\pgfusepath{stroke}%
\end{pgfscope}%
\begin{pgfscope}%
\pgfpathrectangle{\pgfqpoint{0.100000in}{0.212622in}}{\pgfqpoint{3.696000in}{3.696000in}}%
\pgfusepath{clip}%
\pgfsetrectcap%
\pgfsetroundjoin%
\pgfsetlinewidth{1.505625pt}%
\definecolor{currentstroke}{rgb}{1.000000,0.000000,0.000000}%
\pgfsetstrokecolor{currentstroke}%
\pgfsetdash{}{0pt}%
\pgfpathmoveto{\pgfqpoint{1.718922in}{3.104150in}}%
\pgfpathlineto{\pgfqpoint{1.775978in}{2.675241in}}%
\pgfusepath{stroke}%
\end{pgfscope}%
\begin{pgfscope}%
\pgfpathrectangle{\pgfqpoint{0.100000in}{0.212622in}}{\pgfqpoint{3.696000in}{3.696000in}}%
\pgfusepath{clip}%
\pgfsetrectcap%
\pgfsetroundjoin%
\pgfsetlinewidth{1.505625pt}%
\definecolor{currentstroke}{rgb}{1.000000,0.000000,0.000000}%
\pgfsetstrokecolor{currentstroke}%
\pgfsetdash{}{0pt}%
\pgfpathmoveto{\pgfqpoint{1.718918in}{3.104151in}}%
\pgfpathlineto{\pgfqpoint{1.775978in}{2.675241in}}%
\pgfusepath{stroke}%
\end{pgfscope}%
\begin{pgfscope}%
\pgfpathrectangle{\pgfqpoint{0.100000in}{0.212622in}}{\pgfqpoint{3.696000in}{3.696000in}}%
\pgfusepath{clip}%
\pgfsetrectcap%
\pgfsetroundjoin%
\pgfsetlinewidth{1.505625pt}%
\definecolor{currentstroke}{rgb}{1.000000,0.000000,0.000000}%
\pgfsetstrokecolor{currentstroke}%
\pgfsetdash{}{0pt}%
\pgfpathmoveto{\pgfqpoint{1.718916in}{3.104152in}}%
\pgfpathlineto{\pgfqpoint{1.775978in}{2.675241in}}%
\pgfusepath{stroke}%
\end{pgfscope}%
\begin{pgfscope}%
\pgfpathrectangle{\pgfqpoint{0.100000in}{0.212622in}}{\pgfqpoint{3.696000in}{3.696000in}}%
\pgfusepath{clip}%
\pgfsetrectcap%
\pgfsetroundjoin%
\pgfsetlinewidth{1.505625pt}%
\definecolor{currentstroke}{rgb}{1.000000,0.000000,0.000000}%
\pgfsetstrokecolor{currentstroke}%
\pgfsetdash{}{0pt}%
\pgfpathmoveto{\pgfqpoint{1.718915in}{3.104152in}}%
\pgfpathlineto{\pgfqpoint{1.775978in}{2.675241in}}%
\pgfusepath{stroke}%
\end{pgfscope}%
\begin{pgfscope}%
\pgfpathrectangle{\pgfqpoint{0.100000in}{0.212622in}}{\pgfqpoint{3.696000in}{3.696000in}}%
\pgfusepath{clip}%
\pgfsetrectcap%
\pgfsetroundjoin%
\pgfsetlinewidth{1.505625pt}%
\definecolor{currentstroke}{rgb}{1.000000,0.000000,0.000000}%
\pgfsetstrokecolor{currentstroke}%
\pgfsetdash{}{0pt}%
\pgfpathmoveto{\pgfqpoint{1.718914in}{3.104152in}}%
\pgfpathlineto{\pgfqpoint{1.775978in}{2.675241in}}%
\pgfusepath{stroke}%
\end{pgfscope}%
\begin{pgfscope}%
\pgfpathrectangle{\pgfqpoint{0.100000in}{0.212622in}}{\pgfqpoint{3.696000in}{3.696000in}}%
\pgfusepath{clip}%
\pgfsetrectcap%
\pgfsetroundjoin%
\pgfsetlinewidth{1.505625pt}%
\definecolor{currentstroke}{rgb}{1.000000,0.000000,0.000000}%
\pgfsetstrokecolor{currentstroke}%
\pgfsetdash{}{0pt}%
\pgfpathmoveto{\pgfqpoint{1.718914in}{3.104152in}}%
\pgfpathlineto{\pgfqpoint{1.775978in}{2.675241in}}%
\pgfusepath{stroke}%
\end{pgfscope}%
\begin{pgfscope}%
\pgfpathrectangle{\pgfqpoint{0.100000in}{0.212622in}}{\pgfqpoint{3.696000in}{3.696000in}}%
\pgfusepath{clip}%
\pgfsetrectcap%
\pgfsetroundjoin%
\pgfsetlinewidth{1.505625pt}%
\definecolor{currentstroke}{rgb}{1.000000,0.000000,0.000000}%
\pgfsetstrokecolor{currentstroke}%
\pgfsetdash{}{0pt}%
\pgfpathmoveto{\pgfqpoint{1.718913in}{3.104152in}}%
\pgfpathlineto{\pgfqpoint{1.775978in}{2.675241in}}%
\pgfusepath{stroke}%
\end{pgfscope}%
\begin{pgfscope}%
\pgfpathrectangle{\pgfqpoint{0.100000in}{0.212622in}}{\pgfqpoint{3.696000in}{3.696000in}}%
\pgfusepath{clip}%
\pgfsetrectcap%
\pgfsetroundjoin%
\pgfsetlinewidth{1.505625pt}%
\definecolor{currentstroke}{rgb}{1.000000,0.000000,0.000000}%
\pgfsetstrokecolor{currentstroke}%
\pgfsetdash{}{0pt}%
\pgfpathmoveto{\pgfqpoint{1.718913in}{3.104152in}}%
\pgfpathlineto{\pgfqpoint{1.775978in}{2.675241in}}%
\pgfusepath{stroke}%
\end{pgfscope}%
\begin{pgfscope}%
\pgfpathrectangle{\pgfqpoint{0.100000in}{0.212622in}}{\pgfqpoint{3.696000in}{3.696000in}}%
\pgfusepath{clip}%
\pgfsetrectcap%
\pgfsetroundjoin%
\pgfsetlinewidth{1.505625pt}%
\definecolor{currentstroke}{rgb}{1.000000,0.000000,0.000000}%
\pgfsetstrokecolor{currentstroke}%
\pgfsetdash{}{0pt}%
\pgfpathmoveto{\pgfqpoint{1.718913in}{3.104152in}}%
\pgfpathlineto{\pgfqpoint{1.775978in}{2.675241in}}%
\pgfusepath{stroke}%
\end{pgfscope}%
\begin{pgfscope}%
\pgfpathrectangle{\pgfqpoint{0.100000in}{0.212622in}}{\pgfqpoint{3.696000in}{3.696000in}}%
\pgfusepath{clip}%
\pgfsetrectcap%
\pgfsetroundjoin%
\pgfsetlinewidth{1.505625pt}%
\definecolor{currentstroke}{rgb}{1.000000,0.000000,0.000000}%
\pgfsetstrokecolor{currentstroke}%
\pgfsetdash{}{0pt}%
\pgfpathmoveto{\pgfqpoint{1.718913in}{3.104152in}}%
\pgfpathlineto{\pgfqpoint{1.775978in}{2.675241in}}%
\pgfusepath{stroke}%
\end{pgfscope}%
\begin{pgfscope}%
\pgfpathrectangle{\pgfqpoint{0.100000in}{0.212622in}}{\pgfqpoint{3.696000in}{3.696000in}}%
\pgfusepath{clip}%
\pgfsetrectcap%
\pgfsetroundjoin%
\pgfsetlinewidth{1.505625pt}%
\definecolor{currentstroke}{rgb}{1.000000,0.000000,0.000000}%
\pgfsetstrokecolor{currentstroke}%
\pgfsetdash{}{0pt}%
\pgfpathmoveto{\pgfqpoint{1.718913in}{3.104152in}}%
\pgfpathlineto{\pgfqpoint{1.775978in}{2.675241in}}%
\pgfusepath{stroke}%
\end{pgfscope}%
\begin{pgfscope}%
\pgfpathrectangle{\pgfqpoint{0.100000in}{0.212622in}}{\pgfqpoint{3.696000in}{3.696000in}}%
\pgfusepath{clip}%
\pgfsetrectcap%
\pgfsetroundjoin%
\pgfsetlinewidth{1.505625pt}%
\definecolor{currentstroke}{rgb}{1.000000,0.000000,0.000000}%
\pgfsetstrokecolor{currentstroke}%
\pgfsetdash{}{0pt}%
\pgfpathmoveto{\pgfqpoint{1.718913in}{3.104152in}}%
\pgfpathlineto{\pgfqpoint{1.775978in}{2.675241in}}%
\pgfusepath{stroke}%
\end{pgfscope}%
\begin{pgfscope}%
\pgfpathrectangle{\pgfqpoint{0.100000in}{0.212622in}}{\pgfqpoint{3.696000in}{3.696000in}}%
\pgfusepath{clip}%
\pgfsetrectcap%
\pgfsetroundjoin%
\pgfsetlinewidth{1.505625pt}%
\definecolor{currentstroke}{rgb}{1.000000,0.000000,0.000000}%
\pgfsetstrokecolor{currentstroke}%
\pgfsetdash{}{0pt}%
\pgfpathmoveto{\pgfqpoint{1.718913in}{3.104152in}}%
\pgfpathlineto{\pgfqpoint{1.775978in}{2.675241in}}%
\pgfusepath{stroke}%
\end{pgfscope}%
\begin{pgfscope}%
\pgfpathrectangle{\pgfqpoint{0.100000in}{0.212622in}}{\pgfqpoint{3.696000in}{3.696000in}}%
\pgfusepath{clip}%
\pgfsetrectcap%
\pgfsetroundjoin%
\pgfsetlinewidth{1.505625pt}%
\definecolor{currentstroke}{rgb}{1.000000,0.000000,0.000000}%
\pgfsetstrokecolor{currentstroke}%
\pgfsetdash{}{0pt}%
\pgfpathmoveto{\pgfqpoint{1.718913in}{3.104152in}}%
\pgfpathlineto{\pgfqpoint{1.775978in}{2.675241in}}%
\pgfusepath{stroke}%
\end{pgfscope}%
\begin{pgfscope}%
\pgfpathrectangle{\pgfqpoint{0.100000in}{0.212622in}}{\pgfqpoint{3.696000in}{3.696000in}}%
\pgfusepath{clip}%
\pgfsetrectcap%
\pgfsetroundjoin%
\pgfsetlinewidth{1.505625pt}%
\definecolor{currentstroke}{rgb}{1.000000,0.000000,0.000000}%
\pgfsetstrokecolor{currentstroke}%
\pgfsetdash{}{0pt}%
\pgfpathmoveto{\pgfqpoint{1.718913in}{3.104152in}}%
\pgfpathlineto{\pgfqpoint{1.775978in}{2.675241in}}%
\pgfusepath{stroke}%
\end{pgfscope}%
\begin{pgfscope}%
\pgfpathrectangle{\pgfqpoint{0.100000in}{0.212622in}}{\pgfqpoint{3.696000in}{3.696000in}}%
\pgfusepath{clip}%
\pgfsetrectcap%
\pgfsetroundjoin%
\pgfsetlinewidth{1.505625pt}%
\definecolor{currentstroke}{rgb}{1.000000,0.000000,0.000000}%
\pgfsetstrokecolor{currentstroke}%
\pgfsetdash{}{0pt}%
\pgfpathmoveto{\pgfqpoint{1.718913in}{3.104152in}}%
\pgfpathlineto{\pgfqpoint{1.775978in}{2.675241in}}%
\pgfusepath{stroke}%
\end{pgfscope}%
\begin{pgfscope}%
\pgfpathrectangle{\pgfqpoint{0.100000in}{0.212622in}}{\pgfqpoint{3.696000in}{3.696000in}}%
\pgfusepath{clip}%
\pgfsetrectcap%
\pgfsetroundjoin%
\pgfsetlinewidth{1.505625pt}%
\definecolor{currentstroke}{rgb}{1.000000,0.000000,0.000000}%
\pgfsetstrokecolor{currentstroke}%
\pgfsetdash{}{0pt}%
\pgfpathmoveto{\pgfqpoint{1.718913in}{3.104152in}}%
\pgfpathlineto{\pgfqpoint{1.775978in}{2.675241in}}%
\pgfusepath{stroke}%
\end{pgfscope}%
\begin{pgfscope}%
\pgfpathrectangle{\pgfqpoint{0.100000in}{0.212622in}}{\pgfqpoint{3.696000in}{3.696000in}}%
\pgfusepath{clip}%
\pgfsetrectcap%
\pgfsetroundjoin%
\pgfsetlinewidth{1.505625pt}%
\definecolor{currentstroke}{rgb}{1.000000,0.000000,0.000000}%
\pgfsetstrokecolor{currentstroke}%
\pgfsetdash{}{0pt}%
\pgfpathmoveto{\pgfqpoint{1.718913in}{3.104152in}}%
\pgfpathlineto{\pgfqpoint{1.775978in}{2.675241in}}%
\pgfusepath{stroke}%
\end{pgfscope}%
\begin{pgfscope}%
\pgfpathrectangle{\pgfqpoint{0.100000in}{0.212622in}}{\pgfqpoint{3.696000in}{3.696000in}}%
\pgfusepath{clip}%
\pgfsetrectcap%
\pgfsetroundjoin%
\pgfsetlinewidth{1.505625pt}%
\definecolor{currentstroke}{rgb}{1.000000,0.000000,0.000000}%
\pgfsetstrokecolor{currentstroke}%
\pgfsetdash{}{0pt}%
\pgfpathmoveto{\pgfqpoint{1.718913in}{3.104152in}}%
\pgfpathlineto{\pgfqpoint{1.775978in}{2.675241in}}%
\pgfusepath{stroke}%
\end{pgfscope}%
\begin{pgfscope}%
\pgfpathrectangle{\pgfqpoint{0.100000in}{0.212622in}}{\pgfqpoint{3.696000in}{3.696000in}}%
\pgfusepath{clip}%
\pgfsetrectcap%
\pgfsetroundjoin%
\pgfsetlinewidth{1.505625pt}%
\definecolor{currentstroke}{rgb}{1.000000,0.000000,0.000000}%
\pgfsetstrokecolor{currentstroke}%
\pgfsetdash{}{0pt}%
\pgfpathmoveto{\pgfqpoint{1.718913in}{3.104152in}}%
\pgfpathlineto{\pgfqpoint{1.775978in}{2.675241in}}%
\pgfusepath{stroke}%
\end{pgfscope}%
\begin{pgfscope}%
\pgfpathrectangle{\pgfqpoint{0.100000in}{0.212622in}}{\pgfqpoint{3.696000in}{3.696000in}}%
\pgfusepath{clip}%
\pgfsetrectcap%
\pgfsetroundjoin%
\pgfsetlinewidth{1.505625pt}%
\definecolor{currentstroke}{rgb}{1.000000,0.000000,0.000000}%
\pgfsetstrokecolor{currentstroke}%
\pgfsetdash{}{0pt}%
\pgfpathmoveto{\pgfqpoint{1.718623in}{3.103730in}}%
\pgfpathlineto{\pgfqpoint{1.775978in}{2.675241in}}%
\pgfusepath{stroke}%
\end{pgfscope}%
\begin{pgfscope}%
\pgfpathrectangle{\pgfqpoint{0.100000in}{0.212622in}}{\pgfqpoint{3.696000in}{3.696000in}}%
\pgfusepath{clip}%
\pgfsetrectcap%
\pgfsetroundjoin%
\pgfsetlinewidth{1.505625pt}%
\definecolor{currentstroke}{rgb}{1.000000,0.000000,0.000000}%
\pgfsetstrokecolor{currentstroke}%
\pgfsetdash{}{0pt}%
\pgfpathmoveto{\pgfqpoint{1.718454in}{3.103475in}}%
\pgfpathlineto{\pgfqpoint{1.775978in}{2.675241in}}%
\pgfusepath{stroke}%
\end{pgfscope}%
\begin{pgfscope}%
\pgfpathrectangle{\pgfqpoint{0.100000in}{0.212622in}}{\pgfqpoint{3.696000in}{3.696000in}}%
\pgfusepath{clip}%
\pgfsetrectcap%
\pgfsetroundjoin%
\pgfsetlinewidth{1.505625pt}%
\definecolor{currentstroke}{rgb}{1.000000,0.000000,0.000000}%
\pgfsetstrokecolor{currentstroke}%
\pgfsetdash{}{0pt}%
\pgfpathmoveto{\pgfqpoint{1.718367in}{3.103335in}}%
\pgfpathlineto{\pgfqpoint{1.775978in}{2.675241in}}%
\pgfusepath{stroke}%
\end{pgfscope}%
\begin{pgfscope}%
\pgfpathrectangle{\pgfqpoint{0.100000in}{0.212622in}}{\pgfqpoint{3.696000in}{3.696000in}}%
\pgfusepath{clip}%
\pgfsetrectcap%
\pgfsetroundjoin%
\pgfsetlinewidth{1.505625pt}%
\definecolor{currentstroke}{rgb}{1.000000,0.000000,0.000000}%
\pgfsetstrokecolor{currentstroke}%
\pgfsetdash{}{0pt}%
\pgfpathmoveto{\pgfqpoint{1.718326in}{3.103253in}}%
\pgfpathlineto{\pgfqpoint{1.775978in}{2.675241in}}%
\pgfusepath{stroke}%
\end{pgfscope}%
\begin{pgfscope}%
\pgfpathrectangle{\pgfqpoint{0.100000in}{0.212622in}}{\pgfqpoint{3.696000in}{3.696000in}}%
\pgfusepath{clip}%
\pgfsetrectcap%
\pgfsetroundjoin%
\pgfsetlinewidth{1.505625pt}%
\definecolor{currentstroke}{rgb}{1.000000,0.000000,0.000000}%
\pgfsetstrokecolor{currentstroke}%
\pgfsetdash{}{0pt}%
\pgfpathmoveto{\pgfqpoint{1.718307in}{3.103210in}}%
\pgfpathlineto{\pgfqpoint{1.775978in}{2.675241in}}%
\pgfusepath{stroke}%
\end{pgfscope}%
\begin{pgfscope}%
\pgfpathrectangle{\pgfqpoint{0.100000in}{0.212622in}}{\pgfqpoint{3.696000in}{3.696000in}}%
\pgfusepath{clip}%
\pgfsetrectcap%
\pgfsetroundjoin%
\pgfsetlinewidth{1.505625pt}%
\definecolor{currentstroke}{rgb}{1.000000,0.000000,0.000000}%
\pgfsetstrokecolor{currentstroke}%
\pgfsetdash{}{0pt}%
\pgfpathmoveto{\pgfqpoint{1.718297in}{3.103187in}}%
\pgfpathlineto{\pgfqpoint{1.775978in}{2.675241in}}%
\pgfusepath{stroke}%
\end{pgfscope}%
\begin{pgfscope}%
\pgfpathrectangle{\pgfqpoint{0.100000in}{0.212622in}}{\pgfqpoint{3.696000in}{3.696000in}}%
\pgfusepath{clip}%
\pgfsetrectcap%
\pgfsetroundjoin%
\pgfsetlinewidth{1.505625pt}%
\definecolor{currentstroke}{rgb}{1.000000,0.000000,0.000000}%
\pgfsetstrokecolor{currentstroke}%
\pgfsetdash{}{0pt}%
\pgfpathmoveto{\pgfqpoint{1.718291in}{3.103175in}}%
\pgfpathlineto{\pgfqpoint{1.775978in}{2.675241in}}%
\pgfusepath{stroke}%
\end{pgfscope}%
\begin{pgfscope}%
\pgfpathrectangle{\pgfqpoint{0.100000in}{0.212622in}}{\pgfqpoint{3.696000in}{3.696000in}}%
\pgfusepath{clip}%
\pgfsetrectcap%
\pgfsetroundjoin%
\pgfsetlinewidth{1.505625pt}%
\definecolor{currentstroke}{rgb}{1.000000,0.000000,0.000000}%
\pgfsetstrokecolor{currentstroke}%
\pgfsetdash{}{0pt}%
\pgfpathmoveto{\pgfqpoint{1.718288in}{3.103168in}}%
\pgfpathlineto{\pgfqpoint{1.775978in}{2.675241in}}%
\pgfusepath{stroke}%
\end{pgfscope}%
\begin{pgfscope}%
\pgfpathrectangle{\pgfqpoint{0.100000in}{0.212622in}}{\pgfqpoint{3.696000in}{3.696000in}}%
\pgfusepath{clip}%
\pgfsetrectcap%
\pgfsetroundjoin%
\pgfsetlinewidth{1.505625pt}%
\definecolor{currentstroke}{rgb}{1.000000,0.000000,0.000000}%
\pgfsetstrokecolor{currentstroke}%
\pgfsetdash{}{0pt}%
\pgfpathmoveto{\pgfqpoint{1.718287in}{3.103164in}}%
\pgfpathlineto{\pgfqpoint{1.775978in}{2.675241in}}%
\pgfusepath{stroke}%
\end{pgfscope}%
\begin{pgfscope}%
\pgfpathrectangle{\pgfqpoint{0.100000in}{0.212622in}}{\pgfqpoint{3.696000in}{3.696000in}}%
\pgfusepath{clip}%
\pgfsetrectcap%
\pgfsetroundjoin%
\pgfsetlinewidth{1.505625pt}%
\definecolor{currentstroke}{rgb}{1.000000,0.000000,0.000000}%
\pgfsetstrokecolor{currentstroke}%
\pgfsetdash{}{0pt}%
\pgfpathmoveto{\pgfqpoint{1.718286in}{3.103162in}}%
\pgfpathlineto{\pgfqpoint{1.775978in}{2.675241in}}%
\pgfusepath{stroke}%
\end{pgfscope}%
\begin{pgfscope}%
\pgfpathrectangle{\pgfqpoint{0.100000in}{0.212622in}}{\pgfqpoint{3.696000in}{3.696000in}}%
\pgfusepath{clip}%
\pgfsetrectcap%
\pgfsetroundjoin%
\pgfsetlinewidth{1.505625pt}%
\definecolor{currentstroke}{rgb}{1.000000,0.000000,0.000000}%
\pgfsetstrokecolor{currentstroke}%
\pgfsetdash{}{0pt}%
\pgfpathmoveto{\pgfqpoint{1.718285in}{3.103161in}}%
\pgfpathlineto{\pgfqpoint{1.775978in}{2.675241in}}%
\pgfusepath{stroke}%
\end{pgfscope}%
\begin{pgfscope}%
\pgfpathrectangle{\pgfqpoint{0.100000in}{0.212622in}}{\pgfqpoint{3.696000in}{3.696000in}}%
\pgfusepath{clip}%
\pgfsetrectcap%
\pgfsetroundjoin%
\pgfsetlinewidth{1.505625pt}%
\definecolor{currentstroke}{rgb}{1.000000,0.000000,0.000000}%
\pgfsetstrokecolor{currentstroke}%
\pgfsetdash{}{0pt}%
\pgfpathmoveto{\pgfqpoint{1.718285in}{3.103160in}}%
\pgfpathlineto{\pgfqpoint{1.775978in}{2.675241in}}%
\pgfusepath{stroke}%
\end{pgfscope}%
\begin{pgfscope}%
\pgfpathrectangle{\pgfqpoint{0.100000in}{0.212622in}}{\pgfqpoint{3.696000in}{3.696000in}}%
\pgfusepath{clip}%
\pgfsetrectcap%
\pgfsetroundjoin%
\pgfsetlinewidth{1.505625pt}%
\definecolor{currentstroke}{rgb}{1.000000,0.000000,0.000000}%
\pgfsetstrokecolor{currentstroke}%
\pgfsetdash{}{0pt}%
\pgfpathmoveto{\pgfqpoint{1.718285in}{3.103160in}}%
\pgfpathlineto{\pgfqpoint{1.775978in}{2.675241in}}%
\pgfusepath{stroke}%
\end{pgfscope}%
\begin{pgfscope}%
\pgfpathrectangle{\pgfqpoint{0.100000in}{0.212622in}}{\pgfqpoint{3.696000in}{3.696000in}}%
\pgfusepath{clip}%
\pgfsetrectcap%
\pgfsetroundjoin%
\pgfsetlinewidth{1.505625pt}%
\definecolor{currentstroke}{rgb}{1.000000,0.000000,0.000000}%
\pgfsetstrokecolor{currentstroke}%
\pgfsetdash{}{0pt}%
\pgfpathmoveto{\pgfqpoint{1.718285in}{3.103160in}}%
\pgfpathlineto{\pgfqpoint{1.775978in}{2.675241in}}%
\pgfusepath{stroke}%
\end{pgfscope}%
\begin{pgfscope}%
\pgfpathrectangle{\pgfqpoint{0.100000in}{0.212622in}}{\pgfqpoint{3.696000in}{3.696000in}}%
\pgfusepath{clip}%
\pgfsetrectcap%
\pgfsetroundjoin%
\pgfsetlinewidth{1.505625pt}%
\definecolor{currentstroke}{rgb}{1.000000,0.000000,0.000000}%
\pgfsetstrokecolor{currentstroke}%
\pgfsetdash{}{0pt}%
\pgfpathmoveto{\pgfqpoint{1.718285in}{3.103160in}}%
\pgfpathlineto{\pgfqpoint{1.775978in}{2.675241in}}%
\pgfusepath{stroke}%
\end{pgfscope}%
\begin{pgfscope}%
\pgfpathrectangle{\pgfqpoint{0.100000in}{0.212622in}}{\pgfqpoint{3.696000in}{3.696000in}}%
\pgfusepath{clip}%
\pgfsetrectcap%
\pgfsetroundjoin%
\pgfsetlinewidth{1.505625pt}%
\definecolor{currentstroke}{rgb}{1.000000,0.000000,0.000000}%
\pgfsetstrokecolor{currentstroke}%
\pgfsetdash{}{0pt}%
\pgfpathmoveto{\pgfqpoint{1.718285in}{3.103160in}}%
\pgfpathlineto{\pgfqpoint{1.775978in}{2.675241in}}%
\pgfusepath{stroke}%
\end{pgfscope}%
\begin{pgfscope}%
\pgfpathrectangle{\pgfqpoint{0.100000in}{0.212622in}}{\pgfqpoint{3.696000in}{3.696000in}}%
\pgfusepath{clip}%
\pgfsetrectcap%
\pgfsetroundjoin%
\pgfsetlinewidth{1.505625pt}%
\definecolor{currentstroke}{rgb}{1.000000,0.000000,0.000000}%
\pgfsetstrokecolor{currentstroke}%
\pgfsetdash{}{0pt}%
\pgfpathmoveto{\pgfqpoint{1.718285in}{3.103160in}}%
\pgfpathlineto{\pgfqpoint{1.775978in}{2.675241in}}%
\pgfusepath{stroke}%
\end{pgfscope}%
\begin{pgfscope}%
\pgfpathrectangle{\pgfqpoint{0.100000in}{0.212622in}}{\pgfqpoint{3.696000in}{3.696000in}}%
\pgfusepath{clip}%
\pgfsetrectcap%
\pgfsetroundjoin%
\pgfsetlinewidth{1.505625pt}%
\definecolor{currentstroke}{rgb}{1.000000,0.000000,0.000000}%
\pgfsetstrokecolor{currentstroke}%
\pgfsetdash{}{0pt}%
\pgfpathmoveto{\pgfqpoint{1.718285in}{3.103160in}}%
\pgfpathlineto{\pgfqpoint{1.775978in}{2.675241in}}%
\pgfusepath{stroke}%
\end{pgfscope}%
\begin{pgfscope}%
\pgfpathrectangle{\pgfqpoint{0.100000in}{0.212622in}}{\pgfqpoint{3.696000in}{3.696000in}}%
\pgfusepath{clip}%
\pgfsetrectcap%
\pgfsetroundjoin%
\pgfsetlinewidth{1.505625pt}%
\definecolor{currentstroke}{rgb}{1.000000,0.000000,0.000000}%
\pgfsetstrokecolor{currentstroke}%
\pgfsetdash{}{0pt}%
\pgfpathmoveto{\pgfqpoint{1.718285in}{3.103160in}}%
\pgfpathlineto{\pgfqpoint{1.775978in}{2.675241in}}%
\pgfusepath{stroke}%
\end{pgfscope}%
\begin{pgfscope}%
\pgfpathrectangle{\pgfqpoint{0.100000in}{0.212622in}}{\pgfqpoint{3.696000in}{3.696000in}}%
\pgfusepath{clip}%
\pgfsetrectcap%
\pgfsetroundjoin%
\pgfsetlinewidth{1.505625pt}%
\definecolor{currentstroke}{rgb}{1.000000,0.000000,0.000000}%
\pgfsetstrokecolor{currentstroke}%
\pgfsetdash{}{0pt}%
\pgfpathmoveto{\pgfqpoint{1.718285in}{3.103160in}}%
\pgfpathlineto{\pgfqpoint{1.775978in}{2.675241in}}%
\pgfusepath{stroke}%
\end{pgfscope}%
\begin{pgfscope}%
\pgfpathrectangle{\pgfqpoint{0.100000in}{0.212622in}}{\pgfqpoint{3.696000in}{3.696000in}}%
\pgfusepath{clip}%
\pgfsetrectcap%
\pgfsetroundjoin%
\pgfsetlinewidth{1.505625pt}%
\definecolor{currentstroke}{rgb}{1.000000,0.000000,0.000000}%
\pgfsetstrokecolor{currentstroke}%
\pgfsetdash{}{0pt}%
\pgfpathmoveto{\pgfqpoint{1.718285in}{3.103159in}}%
\pgfpathlineto{\pgfqpoint{1.775978in}{2.675241in}}%
\pgfusepath{stroke}%
\end{pgfscope}%
\begin{pgfscope}%
\pgfpathrectangle{\pgfqpoint{0.100000in}{0.212622in}}{\pgfqpoint{3.696000in}{3.696000in}}%
\pgfusepath{clip}%
\pgfsetrectcap%
\pgfsetroundjoin%
\pgfsetlinewidth{1.505625pt}%
\definecolor{currentstroke}{rgb}{1.000000,0.000000,0.000000}%
\pgfsetstrokecolor{currentstroke}%
\pgfsetdash{}{0pt}%
\pgfpathmoveto{\pgfqpoint{1.718285in}{3.103159in}}%
\pgfpathlineto{\pgfqpoint{1.775978in}{2.675241in}}%
\pgfusepath{stroke}%
\end{pgfscope}%
\begin{pgfscope}%
\pgfpathrectangle{\pgfqpoint{0.100000in}{0.212622in}}{\pgfqpoint{3.696000in}{3.696000in}}%
\pgfusepath{clip}%
\pgfsetrectcap%
\pgfsetroundjoin%
\pgfsetlinewidth{1.505625pt}%
\definecolor{currentstroke}{rgb}{1.000000,0.000000,0.000000}%
\pgfsetstrokecolor{currentstroke}%
\pgfsetdash{}{0pt}%
\pgfpathmoveto{\pgfqpoint{1.718285in}{3.103159in}}%
\pgfpathlineto{\pgfqpoint{1.775978in}{2.675241in}}%
\pgfusepath{stroke}%
\end{pgfscope}%
\begin{pgfscope}%
\pgfpathrectangle{\pgfqpoint{0.100000in}{0.212622in}}{\pgfqpoint{3.696000in}{3.696000in}}%
\pgfusepath{clip}%
\pgfsetrectcap%
\pgfsetroundjoin%
\pgfsetlinewidth{1.505625pt}%
\definecolor{currentstroke}{rgb}{1.000000,0.000000,0.000000}%
\pgfsetstrokecolor{currentstroke}%
\pgfsetdash{}{0pt}%
\pgfpathmoveto{\pgfqpoint{1.718285in}{3.103159in}}%
\pgfpathlineto{\pgfqpoint{1.775978in}{2.675241in}}%
\pgfusepath{stroke}%
\end{pgfscope}%
\begin{pgfscope}%
\pgfpathrectangle{\pgfqpoint{0.100000in}{0.212622in}}{\pgfqpoint{3.696000in}{3.696000in}}%
\pgfusepath{clip}%
\pgfsetrectcap%
\pgfsetroundjoin%
\pgfsetlinewidth{1.505625pt}%
\definecolor{currentstroke}{rgb}{1.000000,0.000000,0.000000}%
\pgfsetstrokecolor{currentstroke}%
\pgfsetdash{}{0pt}%
\pgfpathmoveto{\pgfqpoint{1.718285in}{3.103159in}}%
\pgfpathlineto{\pgfqpoint{1.775978in}{2.675241in}}%
\pgfusepath{stroke}%
\end{pgfscope}%
\begin{pgfscope}%
\pgfpathrectangle{\pgfqpoint{0.100000in}{0.212622in}}{\pgfqpoint{3.696000in}{3.696000in}}%
\pgfusepath{clip}%
\pgfsetrectcap%
\pgfsetroundjoin%
\pgfsetlinewidth{1.505625pt}%
\definecolor{currentstroke}{rgb}{1.000000,0.000000,0.000000}%
\pgfsetstrokecolor{currentstroke}%
\pgfsetdash{}{0pt}%
\pgfpathmoveto{\pgfqpoint{1.718285in}{3.103159in}}%
\pgfpathlineto{\pgfqpoint{1.775978in}{2.675241in}}%
\pgfusepath{stroke}%
\end{pgfscope}%
\begin{pgfscope}%
\pgfpathrectangle{\pgfqpoint{0.100000in}{0.212622in}}{\pgfqpoint{3.696000in}{3.696000in}}%
\pgfusepath{clip}%
\pgfsetrectcap%
\pgfsetroundjoin%
\pgfsetlinewidth{1.505625pt}%
\definecolor{currentstroke}{rgb}{1.000000,0.000000,0.000000}%
\pgfsetstrokecolor{currentstroke}%
\pgfsetdash{}{0pt}%
\pgfpathmoveto{\pgfqpoint{1.718285in}{3.103159in}}%
\pgfpathlineto{\pgfqpoint{1.775978in}{2.675241in}}%
\pgfusepath{stroke}%
\end{pgfscope}%
\begin{pgfscope}%
\pgfpathrectangle{\pgfqpoint{0.100000in}{0.212622in}}{\pgfqpoint{3.696000in}{3.696000in}}%
\pgfusepath{clip}%
\pgfsetrectcap%
\pgfsetroundjoin%
\pgfsetlinewidth{1.505625pt}%
\definecolor{currentstroke}{rgb}{1.000000,0.000000,0.000000}%
\pgfsetstrokecolor{currentstroke}%
\pgfsetdash{}{0pt}%
\pgfpathmoveto{\pgfqpoint{1.718285in}{3.103159in}}%
\pgfpathlineto{\pgfqpoint{1.775978in}{2.675241in}}%
\pgfusepath{stroke}%
\end{pgfscope}%
\begin{pgfscope}%
\pgfpathrectangle{\pgfqpoint{0.100000in}{0.212622in}}{\pgfqpoint{3.696000in}{3.696000in}}%
\pgfusepath{clip}%
\pgfsetrectcap%
\pgfsetroundjoin%
\pgfsetlinewidth{1.505625pt}%
\definecolor{currentstroke}{rgb}{1.000000,0.000000,0.000000}%
\pgfsetstrokecolor{currentstroke}%
\pgfsetdash{}{0pt}%
\pgfpathmoveto{\pgfqpoint{1.718285in}{3.103159in}}%
\pgfpathlineto{\pgfqpoint{1.775978in}{2.675241in}}%
\pgfusepath{stroke}%
\end{pgfscope}%
\begin{pgfscope}%
\pgfpathrectangle{\pgfqpoint{0.100000in}{0.212622in}}{\pgfqpoint{3.696000in}{3.696000in}}%
\pgfusepath{clip}%
\pgfsetrectcap%
\pgfsetroundjoin%
\pgfsetlinewidth{1.505625pt}%
\definecolor{currentstroke}{rgb}{1.000000,0.000000,0.000000}%
\pgfsetstrokecolor{currentstroke}%
\pgfsetdash{}{0pt}%
\pgfpathmoveto{\pgfqpoint{1.718285in}{3.103159in}}%
\pgfpathlineto{\pgfqpoint{1.775978in}{2.675241in}}%
\pgfusepath{stroke}%
\end{pgfscope}%
\begin{pgfscope}%
\pgfpathrectangle{\pgfqpoint{0.100000in}{0.212622in}}{\pgfqpoint{3.696000in}{3.696000in}}%
\pgfusepath{clip}%
\pgfsetrectcap%
\pgfsetroundjoin%
\pgfsetlinewidth{1.505625pt}%
\definecolor{currentstroke}{rgb}{1.000000,0.000000,0.000000}%
\pgfsetstrokecolor{currentstroke}%
\pgfsetdash{}{0pt}%
\pgfpathmoveto{\pgfqpoint{1.718285in}{3.103159in}}%
\pgfpathlineto{\pgfqpoint{1.775978in}{2.675241in}}%
\pgfusepath{stroke}%
\end{pgfscope}%
\begin{pgfscope}%
\pgfpathrectangle{\pgfqpoint{0.100000in}{0.212622in}}{\pgfqpoint{3.696000in}{3.696000in}}%
\pgfusepath{clip}%
\pgfsetrectcap%
\pgfsetroundjoin%
\pgfsetlinewidth{1.505625pt}%
\definecolor{currentstroke}{rgb}{1.000000,0.000000,0.000000}%
\pgfsetstrokecolor{currentstroke}%
\pgfsetdash{}{0pt}%
\pgfpathmoveto{\pgfqpoint{1.718285in}{3.103159in}}%
\pgfpathlineto{\pgfqpoint{1.775978in}{2.675241in}}%
\pgfusepath{stroke}%
\end{pgfscope}%
\begin{pgfscope}%
\pgfpathrectangle{\pgfqpoint{0.100000in}{0.212622in}}{\pgfqpoint{3.696000in}{3.696000in}}%
\pgfusepath{clip}%
\pgfsetrectcap%
\pgfsetroundjoin%
\pgfsetlinewidth{1.505625pt}%
\definecolor{currentstroke}{rgb}{1.000000,0.000000,0.000000}%
\pgfsetstrokecolor{currentstroke}%
\pgfsetdash{}{0pt}%
\pgfpathmoveto{\pgfqpoint{1.718285in}{3.103159in}}%
\pgfpathlineto{\pgfqpoint{1.775978in}{2.675241in}}%
\pgfusepath{stroke}%
\end{pgfscope}%
\begin{pgfscope}%
\pgfpathrectangle{\pgfqpoint{0.100000in}{0.212622in}}{\pgfqpoint{3.696000in}{3.696000in}}%
\pgfusepath{clip}%
\pgfsetrectcap%
\pgfsetroundjoin%
\pgfsetlinewidth{1.505625pt}%
\definecolor{currentstroke}{rgb}{1.000000,0.000000,0.000000}%
\pgfsetstrokecolor{currentstroke}%
\pgfsetdash{}{0pt}%
\pgfpathmoveto{\pgfqpoint{1.718285in}{3.103159in}}%
\pgfpathlineto{\pgfqpoint{1.775978in}{2.675241in}}%
\pgfusepath{stroke}%
\end{pgfscope}%
\begin{pgfscope}%
\pgfpathrectangle{\pgfqpoint{0.100000in}{0.212622in}}{\pgfqpoint{3.696000in}{3.696000in}}%
\pgfusepath{clip}%
\pgfsetrectcap%
\pgfsetroundjoin%
\pgfsetlinewidth{1.505625pt}%
\definecolor{currentstroke}{rgb}{1.000000,0.000000,0.000000}%
\pgfsetstrokecolor{currentstroke}%
\pgfsetdash{}{0pt}%
\pgfpathmoveto{\pgfqpoint{1.718285in}{3.103159in}}%
\pgfpathlineto{\pgfqpoint{1.775978in}{2.675241in}}%
\pgfusepath{stroke}%
\end{pgfscope}%
\begin{pgfscope}%
\pgfpathrectangle{\pgfqpoint{0.100000in}{0.212622in}}{\pgfqpoint{3.696000in}{3.696000in}}%
\pgfusepath{clip}%
\pgfsetrectcap%
\pgfsetroundjoin%
\pgfsetlinewidth{1.505625pt}%
\definecolor{currentstroke}{rgb}{1.000000,0.000000,0.000000}%
\pgfsetstrokecolor{currentstroke}%
\pgfsetdash{}{0pt}%
\pgfpathmoveto{\pgfqpoint{1.718285in}{3.103159in}}%
\pgfpathlineto{\pgfqpoint{1.775978in}{2.675241in}}%
\pgfusepath{stroke}%
\end{pgfscope}%
\begin{pgfscope}%
\pgfpathrectangle{\pgfqpoint{0.100000in}{0.212622in}}{\pgfqpoint{3.696000in}{3.696000in}}%
\pgfusepath{clip}%
\pgfsetrectcap%
\pgfsetroundjoin%
\pgfsetlinewidth{1.505625pt}%
\definecolor{currentstroke}{rgb}{1.000000,0.000000,0.000000}%
\pgfsetstrokecolor{currentstroke}%
\pgfsetdash{}{0pt}%
\pgfpathmoveto{\pgfqpoint{1.718285in}{3.103159in}}%
\pgfpathlineto{\pgfqpoint{1.775978in}{2.675241in}}%
\pgfusepath{stroke}%
\end{pgfscope}%
\begin{pgfscope}%
\pgfpathrectangle{\pgfqpoint{0.100000in}{0.212622in}}{\pgfqpoint{3.696000in}{3.696000in}}%
\pgfusepath{clip}%
\pgfsetrectcap%
\pgfsetroundjoin%
\pgfsetlinewidth{1.505625pt}%
\definecolor{currentstroke}{rgb}{1.000000,0.000000,0.000000}%
\pgfsetstrokecolor{currentstroke}%
\pgfsetdash{}{0pt}%
\pgfpathmoveto{\pgfqpoint{1.718285in}{3.103159in}}%
\pgfpathlineto{\pgfqpoint{1.775978in}{2.675241in}}%
\pgfusepath{stroke}%
\end{pgfscope}%
\begin{pgfscope}%
\pgfpathrectangle{\pgfqpoint{0.100000in}{0.212622in}}{\pgfqpoint{3.696000in}{3.696000in}}%
\pgfusepath{clip}%
\pgfsetrectcap%
\pgfsetroundjoin%
\pgfsetlinewidth{1.505625pt}%
\definecolor{currentstroke}{rgb}{1.000000,0.000000,0.000000}%
\pgfsetstrokecolor{currentstroke}%
\pgfsetdash{}{0pt}%
\pgfpathmoveto{\pgfqpoint{1.718285in}{3.103159in}}%
\pgfpathlineto{\pgfqpoint{1.775978in}{2.675241in}}%
\pgfusepath{stroke}%
\end{pgfscope}%
\begin{pgfscope}%
\pgfpathrectangle{\pgfqpoint{0.100000in}{0.212622in}}{\pgfqpoint{3.696000in}{3.696000in}}%
\pgfusepath{clip}%
\pgfsetrectcap%
\pgfsetroundjoin%
\pgfsetlinewidth{1.505625pt}%
\definecolor{currentstroke}{rgb}{1.000000,0.000000,0.000000}%
\pgfsetstrokecolor{currentstroke}%
\pgfsetdash{}{0pt}%
\pgfpathmoveto{\pgfqpoint{1.718125in}{3.102792in}}%
\pgfpathlineto{\pgfqpoint{1.775978in}{2.675241in}}%
\pgfusepath{stroke}%
\end{pgfscope}%
\begin{pgfscope}%
\pgfpathrectangle{\pgfqpoint{0.100000in}{0.212622in}}{\pgfqpoint{3.696000in}{3.696000in}}%
\pgfusepath{clip}%
\pgfsetrectcap%
\pgfsetroundjoin%
\pgfsetlinewidth{1.505625pt}%
\definecolor{currentstroke}{rgb}{1.000000,0.000000,0.000000}%
\pgfsetstrokecolor{currentstroke}%
\pgfsetdash{}{0pt}%
\pgfpathmoveto{\pgfqpoint{1.717769in}{3.102138in}}%
\pgfpathlineto{\pgfqpoint{1.775978in}{2.675241in}}%
\pgfusepath{stroke}%
\end{pgfscope}%
\begin{pgfscope}%
\pgfpathrectangle{\pgfqpoint{0.100000in}{0.212622in}}{\pgfqpoint{3.696000in}{3.696000in}}%
\pgfusepath{clip}%
\pgfsetrectcap%
\pgfsetroundjoin%
\pgfsetlinewidth{1.505625pt}%
\definecolor{currentstroke}{rgb}{1.000000,0.000000,0.000000}%
\pgfsetstrokecolor{currentstroke}%
\pgfsetdash{}{0pt}%
\pgfpathmoveto{\pgfqpoint{1.717548in}{3.101735in}}%
\pgfpathlineto{\pgfqpoint{1.775978in}{2.675241in}}%
\pgfusepath{stroke}%
\end{pgfscope}%
\begin{pgfscope}%
\pgfpathrectangle{\pgfqpoint{0.100000in}{0.212622in}}{\pgfqpoint{3.696000in}{3.696000in}}%
\pgfusepath{clip}%
\pgfsetrectcap%
\pgfsetroundjoin%
\pgfsetlinewidth{1.505625pt}%
\definecolor{currentstroke}{rgb}{1.000000,0.000000,0.000000}%
\pgfsetstrokecolor{currentstroke}%
\pgfsetdash{}{0pt}%
\pgfpathmoveto{\pgfqpoint{1.717475in}{3.101513in}}%
\pgfpathlineto{\pgfqpoint{1.775978in}{2.675241in}}%
\pgfusepath{stroke}%
\end{pgfscope}%
\begin{pgfscope}%
\pgfpathrectangle{\pgfqpoint{0.100000in}{0.212622in}}{\pgfqpoint{3.696000in}{3.696000in}}%
\pgfusepath{clip}%
\pgfsetrectcap%
\pgfsetroundjoin%
\pgfsetlinewidth{1.505625pt}%
\definecolor{currentstroke}{rgb}{1.000000,0.000000,0.000000}%
\pgfsetstrokecolor{currentstroke}%
\pgfsetdash{}{0pt}%
\pgfpathmoveto{\pgfqpoint{1.717436in}{3.101390in}}%
\pgfpathlineto{\pgfqpoint{1.775978in}{2.675241in}}%
\pgfusepath{stroke}%
\end{pgfscope}%
\begin{pgfscope}%
\pgfpathrectangle{\pgfqpoint{0.100000in}{0.212622in}}{\pgfqpoint{3.696000in}{3.696000in}}%
\pgfusepath{clip}%
\pgfsetrectcap%
\pgfsetroundjoin%
\pgfsetlinewidth{1.505625pt}%
\definecolor{currentstroke}{rgb}{1.000000,0.000000,0.000000}%
\pgfsetstrokecolor{currentstroke}%
\pgfsetdash{}{0pt}%
\pgfpathmoveto{\pgfqpoint{1.716986in}{3.100691in}}%
\pgfpathlineto{\pgfqpoint{1.775978in}{2.675241in}}%
\pgfusepath{stroke}%
\end{pgfscope}%
\begin{pgfscope}%
\pgfpathrectangle{\pgfqpoint{0.100000in}{0.212622in}}{\pgfqpoint{3.696000in}{3.696000in}}%
\pgfusepath{clip}%
\pgfsetrectcap%
\pgfsetroundjoin%
\pgfsetlinewidth{1.505625pt}%
\definecolor{currentstroke}{rgb}{1.000000,0.000000,0.000000}%
\pgfsetstrokecolor{currentstroke}%
\pgfsetdash{}{0pt}%
\pgfpathmoveto{\pgfqpoint{1.716782in}{3.100277in}}%
\pgfpathlineto{\pgfqpoint{1.775978in}{2.675241in}}%
\pgfusepath{stroke}%
\end{pgfscope}%
\begin{pgfscope}%
\pgfpathrectangle{\pgfqpoint{0.100000in}{0.212622in}}{\pgfqpoint{3.696000in}{3.696000in}}%
\pgfusepath{clip}%
\pgfsetrectcap%
\pgfsetroundjoin%
\pgfsetlinewidth{1.505625pt}%
\definecolor{currentstroke}{rgb}{1.000000,0.000000,0.000000}%
\pgfsetstrokecolor{currentstroke}%
\pgfsetdash{}{0pt}%
\pgfpathmoveto{\pgfqpoint{1.716512in}{3.099199in}}%
\pgfpathlineto{\pgfqpoint{1.775978in}{2.675241in}}%
\pgfusepath{stroke}%
\end{pgfscope}%
\begin{pgfscope}%
\pgfpathrectangle{\pgfqpoint{0.100000in}{0.212622in}}{\pgfqpoint{3.696000in}{3.696000in}}%
\pgfusepath{clip}%
\pgfsetrectcap%
\pgfsetroundjoin%
\pgfsetlinewidth{1.505625pt}%
\definecolor{currentstroke}{rgb}{1.000000,0.000000,0.000000}%
\pgfsetstrokecolor{currentstroke}%
\pgfsetdash{}{0pt}%
\pgfpathmoveto{\pgfqpoint{1.715139in}{3.096310in}}%
\pgfpathlineto{\pgfqpoint{1.775978in}{2.675241in}}%
\pgfusepath{stroke}%
\end{pgfscope}%
\begin{pgfscope}%
\pgfpathrectangle{\pgfqpoint{0.100000in}{0.212622in}}{\pgfqpoint{3.696000in}{3.696000in}}%
\pgfusepath{clip}%
\pgfsetrectcap%
\pgfsetroundjoin%
\pgfsetlinewidth{1.505625pt}%
\definecolor{currentstroke}{rgb}{1.000000,0.000000,0.000000}%
\pgfsetstrokecolor{currentstroke}%
\pgfsetdash{}{0pt}%
\pgfpathmoveto{\pgfqpoint{1.714450in}{3.094800in}}%
\pgfpathlineto{\pgfqpoint{1.775978in}{2.675241in}}%
\pgfusepath{stroke}%
\end{pgfscope}%
\begin{pgfscope}%
\pgfpathrectangle{\pgfqpoint{0.100000in}{0.212622in}}{\pgfqpoint{3.696000in}{3.696000in}}%
\pgfusepath{clip}%
\pgfsetrectcap%
\pgfsetroundjoin%
\pgfsetlinewidth{1.505625pt}%
\definecolor{currentstroke}{rgb}{1.000000,0.000000,0.000000}%
\pgfsetstrokecolor{currentstroke}%
\pgfsetdash{}{0pt}%
\pgfpathmoveto{\pgfqpoint{1.714097in}{3.092372in}}%
\pgfpathlineto{\pgfqpoint{1.775978in}{2.675241in}}%
\pgfusepath{stroke}%
\end{pgfscope}%
\begin{pgfscope}%
\pgfpathrectangle{\pgfqpoint{0.100000in}{0.212622in}}{\pgfqpoint{3.696000in}{3.696000in}}%
\pgfusepath{clip}%
\pgfsetrectcap%
\pgfsetroundjoin%
\pgfsetlinewidth{1.505625pt}%
\definecolor{currentstroke}{rgb}{1.000000,0.000000,0.000000}%
\pgfsetstrokecolor{currentstroke}%
\pgfsetdash{}{0pt}%
\pgfpathmoveto{\pgfqpoint{1.713016in}{3.089008in}}%
\pgfpathlineto{\pgfqpoint{1.775978in}{2.675241in}}%
\pgfusepath{stroke}%
\end{pgfscope}%
\begin{pgfscope}%
\pgfpathrectangle{\pgfqpoint{0.100000in}{0.212622in}}{\pgfqpoint{3.696000in}{3.696000in}}%
\pgfusepath{clip}%
\pgfsetrectcap%
\pgfsetroundjoin%
\pgfsetlinewidth{1.505625pt}%
\definecolor{currentstroke}{rgb}{1.000000,0.000000,0.000000}%
\pgfsetstrokecolor{currentstroke}%
\pgfsetdash{}{0pt}%
\pgfpathmoveto{\pgfqpoint{1.711128in}{3.085358in}}%
\pgfpathlineto{\pgfqpoint{1.775978in}{2.675241in}}%
\pgfusepath{stroke}%
\end{pgfscope}%
\begin{pgfscope}%
\pgfpathrectangle{\pgfqpoint{0.100000in}{0.212622in}}{\pgfqpoint{3.696000in}{3.696000in}}%
\pgfusepath{clip}%
\pgfsetrectcap%
\pgfsetroundjoin%
\pgfsetlinewidth{1.505625pt}%
\definecolor{currentstroke}{rgb}{1.000000,0.000000,0.000000}%
\pgfsetstrokecolor{currentstroke}%
\pgfsetdash{}{0pt}%
\pgfpathmoveto{\pgfqpoint{1.710109in}{3.078434in}}%
\pgfpathlineto{\pgfqpoint{1.775978in}{2.675241in}}%
\pgfusepath{stroke}%
\end{pgfscope}%
\begin{pgfscope}%
\pgfpathrectangle{\pgfqpoint{0.100000in}{0.212622in}}{\pgfqpoint{3.696000in}{3.696000in}}%
\pgfusepath{clip}%
\pgfsetrectcap%
\pgfsetroundjoin%
\pgfsetlinewidth{1.505625pt}%
\definecolor{currentstroke}{rgb}{1.000000,0.000000,0.000000}%
\pgfsetstrokecolor{currentstroke}%
\pgfsetdash{}{0pt}%
\pgfpathmoveto{\pgfqpoint{1.707695in}{3.071551in}}%
\pgfpathlineto{\pgfqpoint{1.775978in}{2.675241in}}%
\pgfusepath{stroke}%
\end{pgfscope}%
\begin{pgfscope}%
\pgfpathrectangle{\pgfqpoint{0.100000in}{0.212622in}}{\pgfqpoint{3.696000in}{3.696000in}}%
\pgfusepath{clip}%
\pgfsetrectcap%
\pgfsetroundjoin%
\pgfsetlinewidth{1.505625pt}%
\definecolor{currentstroke}{rgb}{1.000000,0.000000,0.000000}%
\pgfsetstrokecolor{currentstroke}%
\pgfsetdash{}{0pt}%
\pgfpathmoveto{\pgfqpoint{1.704099in}{3.064537in}}%
\pgfpathlineto{\pgfqpoint{1.775978in}{2.675241in}}%
\pgfusepath{stroke}%
\end{pgfscope}%
\begin{pgfscope}%
\pgfpathrectangle{\pgfqpoint{0.100000in}{0.212622in}}{\pgfqpoint{3.696000in}{3.696000in}}%
\pgfusepath{clip}%
\pgfsetrectcap%
\pgfsetroundjoin%
\pgfsetlinewidth{1.505625pt}%
\definecolor{currentstroke}{rgb}{1.000000,0.000000,0.000000}%
\pgfsetstrokecolor{currentstroke}%
\pgfsetdash{}{0pt}%
\pgfpathmoveto{\pgfqpoint{1.701292in}{3.055375in}}%
\pgfpathlineto{\pgfqpoint{1.775978in}{2.675241in}}%
\pgfusepath{stroke}%
\end{pgfscope}%
\begin{pgfscope}%
\pgfpathrectangle{\pgfqpoint{0.100000in}{0.212622in}}{\pgfqpoint{3.696000in}{3.696000in}}%
\pgfusepath{clip}%
\pgfsetrectcap%
\pgfsetroundjoin%
\pgfsetlinewidth{1.505625pt}%
\definecolor{currentstroke}{rgb}{1.000000,0.000000,0.000000}%
\pgfsetstrokecolor{currentstroke}%
\pgfsetdash{}{0pt}%
\pgfpathmoveto{\pgfqpoint{1.698702in}{3.045686in}}%
\pgfpathlineto{\pgfqpoint{1.775978in}{2.675241in}}%
\pgfusepath{stroke}%
\end{pgfscope}%
\begin{pgfscope}%
\pgfpathrectangle{\pgfqpoint{0.100000in}{0.212622in}}{\pgfqpoint{3.696000in}{3.696000in}}%
\pgfusepath{clip}%
\pgfsetrectcap%
\pgfsetroundjoin%
\pgfsetlinewidth{1.505625pt}%
\definecolor{currentstroke}{rgb}{1.000000,0.000000,0.000000}%
\pgfsetstrokecolor{currentstroke}%
\pgfsetdash{}{0pt}%
\pgfpathmoveto{\pgfqpoint{1.693179in}{3.035157in}}%
\pgfpathlineto{\pgfqpoint{1.775978in}{2.675241in}}%
\pgfusepath{stroke}%
\end{pgfscope}%
\begin{pgfscope}%
\pgfpathrectangle{\pgfqpoint{0.100000in}{0.212622in}}{\pgfqpoint{3.696000in}{3.696000in}}%
\pgfusepath{clip}%
\pgfsetrectcap%
\pgfsetroundjoin%
\pgfsetlinewidth{1.505625pt}%
\definecolor{currentstroke}{rgb}{1.000000,0.000000,0.000000}%
\pgfsetstrokecolor{currentstroke}%
\pgfsetdash{}{0pt}%
\pgfpathmoveto{\pgfqpoint{1.689725in}{3.022190in}}%
\pgfpathlineto{\pgfqpoint{1.775978in}{2.675241in}}%
\pgfusepath{stroke}%
\end{pgfscope}%
\begin{pgfscope}%
\pgfpathrectangle{\pgfqpoint{0.100000in}{0.212622in}}{\pgfqpoint{3.696000in}{3.696000in}}%
\pgfusepath{clip}%
\pgfsetrectcap%
\pgfsetroundjoin%
\pgfsetlinewidth{1.505625pt}%
\definecolor{currentstroke}{rgb}{1.000000,0.000000,0.000000}%
\pgfsetstrokecolor{currentstroke}%
\pgfsetdash{}{0pt}%
\pgfpathmoveto{\pgfqpoint{1.687793in}{3.007867in}}%
\pgfpathlineto{\pgfqpoint{1.775978in}{2.675241in}}%
\pgfusepath{stroke}%
\end{pgfscope}%
\begin{pgfscope}%
\pgfpathrectangle{\pgfqpoint{0.100000in}{0.212622in}}{\pgfqpoint{3.696000in}{3.696000in}}%
\pgfusepath{clip}%
\pgfsetrectcap%
\pgfsetroundjoin%
\pgfsetlinewidth{1.505625pt}%
\definecolor{currentstroke}{rgb}{1.000000,0.000000,0.000000}%
\pgfsetstrokecolor{currentstroke}%
\pgfsetdash{}{0pt}%
\pgfpathmoveto{\pgfqpoint{1.682005in}{2.994970in}}%
\pgfpathlineto{\pgfqpoint{1.775978in}{2.675241in}}%
\pgfusepath{stroke}%
\end{pgfscope}%
\begin{pgfscope}%
\pgfpathrectangle{\pgfqpoint{0.100000in}{0.212622in}}{\pgfqpoint{3.696000in}{3.696000in}}%
\pgfusepath{clip}%
\pgfsetrectcap%
\pgfsetroundjoin%
\pgfsetlinewidth{1.505625pt}%
\definecolor{currentstroke}{rgb}{1.000000,0.000000,0.000000}%
\pgfsetstrokecolor{currentstroke}%
\pgfsetdash{}{0pt}%
\pgfpathmoveto{\pgfqpoint{1.677998in}{2.988055in}}%
\pgfpathlineto{\pgfqpoint{1.775978in}{2.675241in}}%
\pgfusepath{stroke}%
\end{pgfscope}%
\begin{pgfscope}%
\pgfpathrectangle{\pgfqpoint{0.100000in}{0.212622in}}{\pgfqpoint{3.696000in}{3.696000in}}%
\pgfusepath{clip}%
\pgfsetrectcap%
\pgfsetroundjoin%
\pgfsetlinewidth{1.505625pt}%
\definecolor{currentstroke}{rgb}{1.000000,0.000000,0.000000}%
\pgfsetstrokecolor{currentstroke}%
\pgfsetdash{}{0pt}%
\pgfpathmoveto{\pgfqpoint{1.676489in}{2.983511in}}%
\pgfpathlineto{\pgfqpoint{1.775978in}{2.675241in}}%
\pgfusepath{stroke}%
\end{pgfscope}%
\begin{pgfscope}%
\pgfpathrectangle{\pgfqpoint{0.100000in}{0.212622in}}{\pgfqpoint{3.696000in}{3.696000in}}%
\pgfusepath{clip}%
\pgfsetrectcap%
\pgfsetroundjoin%
\pgfsetlinewidth{1.505625pt}%
\definecolor{currentstroke}{rgb}{1.000000,0.000000,0.000000}%
\pgfsetstrokecolor{currentstroke}%
\pgfsetdash{}{0pt}%
\pgfpathmoveto{\pgfqpoint{1.675402in}{2.981219in}}%
\pgfpathlineto{\pgfqpoint{1.775978in}{2.675241in}}%
\pgfusepath{stroke}%
\end{pgfscope}%
\begin{pgfscope}%
\pgfpathrectangle{\pgfqpoint{0.100000in}{0.212622in}}{\pgfqpoint{3.696000in}{3.696000in}}%
\pgfusepath{clip}%
\pgfsetrectcap%
\pgfsetroundjoin%
\pgfsetlinewidth{1.505625pt}%
\definecolor{currentstroke}{rgb}{1.000000,0.000000,0.000000}%
\pgfsetstrokecolor{currentstroke}%
\pgfsetdash{}{0pt}%
\pgfpathmoveto{\pgfqpoint{1.673831in}{2.978347in}}%
\pgfpathlineto{\pgfqpoint{1.775978in}{2.675241in}}%
\pgfusepath{stroke}%
\end{pgfscope}%
\begin{pgfscope}%
\pgfpathrectangle{\pgfqpoint{0.100000in}{0.212622in}}{\pgfqpoint{3.696000in}{3.696000in}}%
\pgfusepath{clip}%
\pgfsetrectcap%
\pgfsetroundjoin%
\pgfsetlinewidth{1.505625pt}%
\definecolor{currentstroke}{rgb}{1.000000,0.000000,0.000000}%
\pgfsetstrokecolor{currentstroke}%
\pgfsetdash{}{0pt}%
\pgfpathmoveto{\pgfqpoint{1.672173in}{2.975156in}}%
\pgfpathlineto{\pgfqpoint{1.775978in}{2.675241in}}%
\pgfusepath{stroke}%
\end{pgfscope}%
\begin{pgfscope}%
\pgfpathrectangle{\pgfqpoint{0.100000in}{0.212622in}}{\pgfqpoint{3.696000in}{3.696000in}}%
\pgfusepath{clip}%
\pgfsetrectcap%
\pgfsetroundjoin%
\pgfsetlinewidth{1.505625pt}%
\definecolor{currentstroke}{rgb}{1.000000,0.000000,0.000000}%
\pgfsetstrokecolor{currentstroke}%
\pgfsetdash{}{0pt}%
\pgfpathmoveto{\pgfqpoint{1.671711in}{2.973025in}}%
\pgfpathlineto{\pgfqpoint{1.775978in}{2.675241in}}%
\pgfusepath{stroke}%
\end{pgfscope}%
\begin{pgfscope}%
\pgfpathrectangle{\pgfqpoint{0.100000in}{0.212622in}}{\pgfqpoint{3.696000in}{3.696000in}}%
\pgfusepath{clip}%
\pgfsetrectcap%
\pgfsetroundjoin%
\pgfsetlinewidth{1.505625pt}%
\definecolor{currentstroke}{rgb}{1.000000,0.000000,0.000000}%
\pgfsetstrokecolor{currentstroke}%
\pgfsetdash{}{0pt}%
\pgfpathmoveto{\pgfqpoint{1.669674in}{2.969258in}}%
\pgfpathlineto{\pgfqpoint{1.775978in}{2.675241in}}%
\pgfusepath{stroke}%
\end{pgfscope}%
\begin{pgfscope}%
\pgfpathrectangle{\pgfqpoint{0.100000in}{0.212622in}}{\pgfqpoint{3.696000in}{3.696000in}}%
\pgfusepath{clip}%
\pgfsetrectcap%
\pgfsetroundjoin%
\pgfsetlinewidth{1.505625pt}%
\definecolor{currentstroke}{rgb}{1.000000,0.000000,0.000000}%
\pgfsetstrokecolor{currentstroke}%
\pgfsetdash{}{0pt}%
\pgfpathmoveto{\pgfqpoint{1.668456in}{2.967352in}}%
\pgfpathlineto{\pgfqpoint{1.775978in}{2.675241in}}%
\pgfusepath{stroke}%
\end{pgfscope}%
\begin{pgfscope}%
\pgfpathrectangle{\pgfqpoint{0.100000in}{0.212622in}}{\pgfqpoint{3.696000in}{3.696000in}}%
\pgfusepath{clip}%
\pgfsetrectcap%
\pgfsetroundjoin%
\pgfsetlinewidth{1.505625pt}%
\definecolor{currentstroke}{rgb}{1.000000,0.000000,0.000000}%
\pgfsetstrokecolor{currentstroke}%
\pgfsetdash{}{0pt}%
\pgfpathmoveto{\pgfqpoint{1.666852in}{2.963106in}}%
\pgfpathlineto{\pgfqpoint{1.775978in}{2.675241in}}%
\pgfusepath{stroke}%
\end{pgfscope}%
\begin{pgfscope}%
\pgfpathrectangle{\pgfqpoint{0.100000in}{0.212622in}}{\pgfqpoint{3.696000in}{3.696000in}}%
\pgfusepath{clip}%
\pgfsetrectcap%
\pgfsetroundjoin%
\pgfsetlinewidth{1.505625pt}%
\definecolor{currentstroke}{rgb}{1.000000,0.000000,0.000000}%
\pgfsetstrokecolor{currentstroke}%
\pgfsetdash{}{0pt}%
\pgfpathmoveto{\pgfqpoint{1.665256in}{2.958249in}}%
\pgfpathlineto{\pgfqpoint{1.775978in}{2.675241in}}%
\pgfusepath{stroke}%
\end{pgfscope}%
\begin{pgfscope}%
\pgfpathrectangle{\pgfqpoint{0.100000in}{0.212622in}}{\pgfqpoint{3.696000in}{3.696000in}}%
\pgfusepath{clip}%
\pgfsetrectcap%
\pgfsetroundjoin%
\pgfsetlinewidth{1.505625pt}%
\definecolor{currentstroke}{rgb}{1.000000,0.000000,0.000000}%
\pgfsetstrokecolor{currentstroke}%
\pgfsetdash{}{0pt}%
\pgfpathmoveto{\pgfqpoint{1.661954in}{2.952723in}}%
\pgfpathlineto{\pgfqpoint{1.775978in}{2.675241in}}%
\pgfusepath{stroke}%
\end{pgfscope}%
\begin{pgfscope}%
\pgfpathrectangle{\pgfqpoint{0.100000in}{0.212622in}}{\pgfqpoint{3.696000in}{3.696000in}}%
\pgfusepath{clip}%
\pgfsetrectcap%
\pgfsetroundjoin%
\pgfsetlinewidth{1.505625pt}%
\definecolor{currentstroke}{rgb}{1.000000,0.000000,0.000000}%
\pgfsetstrokecolor{currentstroke}%
\pgfsetdash{}{0pt}%
\pgfpathmoveto{\pgfqpoint{1.658458in}{2.946579in}}%
\pgfpathlineto{\pgfqpoint{1.775978in}{2.675241in}}%
\pgfusepath{stroke}%
\end{pgfscope}%
\begin{pgfscope}%
\pgfpathrectangle{\pgfqpoint{0.100000in}{0.212622in}}{\pgfqpoint{3.696000in}{3.696000in}}%
\pgfusepath{clip}%
\pgfsetrectcap%
\pgfsetroundjoin%
\pgfsetlinewidth{1.505625pt}%
\definecolor{currentstroke}{rgb}{1.000000,0.000000,0.000000}%
\pgfsetstrokecolor{currentstroke}%
\pgfsetdash{}{0pt}%
\pgfpathmoveto{\pgfqpoint{1.656505in}{2.943273in}}%
\pgfpathlineto{\pgfqpoint{1.775978in}{2.675241in}}%
\pgfusepath{stroke}%
\end{pgfscope}%
\begin{pgfscope}%
\pgfpathrectangle{\pgfqpoint{0.100000in}{0.212622in}}{\pgfqpoint{3.696000in}{3.696000in}}%
\pgfusepath{clip}%
\pgfsetrectcap%
\pgfsetroundjoin%
\pgfsetlinewidth{1.505625pt}%
\definecolor{currentstroke}{rgb}{1.000000,0.000000,0.000000}%
\pgfsetstrokecolor{currentstroke}%
\pgfsetdash{}{0pt}%
\pgfpathmoveto{\pgfqpoint{1.654922in}{2.937992in}}%
\pgfpathlineto{\pgfqpoint{1.775978in}{2.675241in}}%
\pgfusepath{stroke}%
\end{pgfscope}%
\begin{pgfscope}%
\pgfpathrectangle{\pgfqpoint{0.100000in}{0.212622in}}{\pgfqpoint{3.696000in}{3.696000in}}%
\pgfusepath{clip}%
\pgfsetrectcap%
\pgfsetroundjoin%
\pgfsetlinewidth{1.505625pt}%
\definecolor{currentstroke}{rgb}{1.000000,0.000000,0.000000}%
\pgfsetstrokecolor{currentstroke}%
\pgfsetdash{}{0pt}%
\pgfpathmoveto{\pgfqpoint{1.651981in}{2.931779in}}%
\pgfpathlineto{\pgfqpoint{1.775978in}{2.675241in}}%
\pgfusepath{stroke}%
\end{pgfscope}%
\begin{pgfscope}%
\pgfpathrectangle{\pgfqpoint{0.100000in}{0.212622in}}{\pgfqpoint{3.696000in}{3.696000in}}%
\pgfusepath{clip}%
\pgfsetrectcap%
\pgfsetroundjoin%
\pgfsetlinewidth{1.505625pt}%
\definecolor{currentstroke}{rgb}{1.000000,0.000000,0.000000}%
\pgfsetstrokecolor{currentstroke}%
\pgfsetdash{}{0pt}%
\pgfpathmoveto{\pgfqpoint{1.650099in}{2.928487in}}%
\pgfpathlineto{\pgfqpoint{1.770008in}{2.670404in}}%
\pgfusepath{stroke}%
\end{pgfscope}%
\begin{pgfscope}%
\pgfpathrectangle{\pgfqpoint{0.100000in}{0.212622in}}{\pgfqpoint{3.696000in}{3.696000in}}%
\pgfusepath{clip}%
\pgfsetrectcap%
\pgfsetroundjoin%
\pgfsetlinewidth{1.505625pt}%
\definecolor{currentstroke}{rgb}{1.000000,0.000000,0.000000}%
\pgfsetstrokecolor{currentstroke}%
\pgfsetdash{}{0pt}%
\pgfpathmoveto{\pgfqpoint{1.647891in}{2.923579in}}%
\pgfpathlineto{\pgfqpoint{1.770008in}{2.670404in}}%
\pgfusepath{stroke}%
\end{pgfscope}%
\begin{pgfscope}%
\pgfpathrectangle{\pgfqpoint{0.100000in}{0.212622in}}{\pgfqpoint{3.696000in}{3.696000in}}%
\pgfusepath{clip}%
\pgfsetrectcap%
\pgfsetroundjoin%
\pgfsetlinewidth{1.505625pt}%
\definecolor{currentstroke}{rgb}{1.000000,0.000000,0.000000}%
\pgfsetstrokecolor{currentstroke}%
\pgfsetdash{}{0pt}%
\pgfpathmoveto{\pgfqpoint{1.647055in}{2.920680in}}%
\pgfpathlineto{\pgfqpoint{1.770008in}{2.670404in}}%
\pgfusepath{stroke}%
\end{pgfscope}%
\begin{pgfscope}%
\pgfpathrectangle{\pgfqpoint{0.100000in}{0.212622in}}{\pgfqpoint{3.696000in}{3.696000in}}%
\pgfusepath{clip}%
\pgfsetrectcap%
\pgfsetroundjoin%
\pgfsetlinewidth{1.505625pt}%
\definecolor{currentstroke}{rgb}{1.000000,0.000000,0.000000}%
\pgfsetstrokecolor{currentstroke}%
\pgfsetdash{}{0pt}%
\pgfpathmoveto{\pgfqpoint{1.644882in}{2.916706in}}%
\pgfpathlineto{\pgfqpoint{1.764033in}{2.665563in}}%
\pgfusepath{stroke}%
\end{pgfscope}%
\begin{pgfscope}%
\pgfpathrectangle{\pgfqpoint{0.100000in}{0.212622in}}{\pgfqpoint{3.696000in}{3.696000in}}%
\pgfusepath{clip}%
\pgfsetrectcap%
\pgfsetroundjoin%
\pgfsetlinewidth{1.505625pt}%
\definecolor{currentstroke}{rgb}{1.000000,0.000000,0.000000}%
\pgfsetstrokecolor{currentstroke}%
\pgfsetdash{}{0pt}%
\pgfpathmoveto{\pgfqpoint{1.643585in}{2.914458in}}%
\pgfpathlineto{\pgfqpoint{1.764033in}{2.665563in}}%
\pgfusepath{stroke}%
\end{pgfscope}%
\begin{pgfscope}%
\pgfpathrectangle{\pgfqpoint{0.100000in}{0.212622in}}{\pgfqpoint{3.696000in}{3.696000in}}%
\pgfusepath{clip}%
\pgfsetrectcap%
\pgfsetroundjoin%
\pgfsetlinewidth{1.505625pt}%
\definecolor{currentstroke}{rgb}{1.000000,0.000000,0.000000}%
\pgfsetstrokecolor{currentstroke}%
\pgfsetdash{}{0pt}%
\pgfpathmoveto{\pgfqpoint{1.641774in}{2.910462in}}%
\pgfpathlineto{\pgfqpoint{1.764033in}{2.665563in}}%
\pgfusepath{stroke}%
\end{pgfscope}%
\begin{pgfscope}%
\pgfpathrectangle{\pgfqpoint{0.100000in}{0.212622in}}{\pgfqpoint{3.696000in}{3.696000in}}%
\pgfusepath{clip}%
\pgfsetrectcap%
\pgfsetroundjoin%
\pgfsetlinewidth{1.505625pt}%
\definecolor{currentstroke}{rgb}{1.000000,0.000000,0.000000}%
\pgfsetstrokecolor{currentstroke}%
\pgfsetdash{}{0pt}%
\pgfpathmoveto{\pgfqpoint{1.640622in}{2.905112in}}%
\pgfpathlineto{\pgfqpoint{1.764033in}{2.665563in}}%
\pgfusepath{stroke}%
\end{pgfscope}%
\begin{pgfscope}%
\pgfpathrectangle{\pgfqpoint{0.100000in}{0.212622in}}{\pgfqpoint{3.696000in}{3.696000in}}%
\pgfusepath{clip}%
\pgfsetrectcap%
\pgfsetroundjoin%
\pgfsetlinewidth{1.505625pt}%
\definecolor{currentstroke}{rgb}{1.000000,0.000000,0.000000}%
\pgfsetstrokecolor{currentstroke}%
\pgfsetdash{}{0pt}%
\pgfpathmoveto{\pgfqpoint{1.637830in}{2.898964in}}%
\pgfpathlineto{\pgfqpoint{1.758052in}{2.660718in}}%
\pgfusepath{stroke}%
\end{pgfscope}%
\begin{pgfscope}%
\pgfpathrectangle{\pgfqpoint{0.100000in}{0.212622in}}{\pgfqpoint{3.696000in}{3.696000in}}%
\pgfusepath{clip}%
\pgfsetrectcap%
\pgfsetroundjoin%
\pgfsetlinewidth{1.505625pt}%
\definecolor{currentstroke}{rgb}{1.000000,0.000000,0.000000}%
\pgfsetstrokecolor{currentstroke}%
\pgfsetdash{}{0pt}%
\pgfpathmoveto{\pgfqpoint{1.636091in}{2.895904in}}%
\pgfpathlineto{\pgfqpoint{1.758052in}{2.660718in}}%
\pgfusepath{stroke}%
\end{pgfscope}%
\begin{pgfscope}%
\pgfpathrectangle{\pgfqpoint{0.100000in}{0.212622in}}{\pgfqpoint{3.696000in}{3.696000in}}%
\pgfusepath{clip}%
\pgfsetrectcap%
\pgfsetroundjoin%
\pgfsetlinewidth{1.505625pt}%
\definecolor{currentstroke}{rgb}{1.000000,0.000000,0.000000}%
\pgfsetstrokecolor{currentstroke}%
\pgfsetdash{}{0pt}%
\pgfpathmoveto{\pgfqpoint{1.635128in}{2.894197in}}%
\pgfpathlineto{\pgfqpoint{1.758052in}{2.660718in}}%
\pgfusepath{stroke}%
\end{pgfscope}%
\begin{pgfscope}%
\pgfpathrectangle{\pgfqpoint{0.100000in}{0.212622in}}{\pgfqpoint{3.696000in}{3.696000in}}%
\pgfusepath{clip}%
\pgfsetrectcap%
\pgfsetroundjoin%
\pgfsetlinewidth{1.505625pt}%
\definecolor{currentstroke}{rgb}{1.000000,0.000000,0.000000}%
\pgfsetstrokecolor{currentstroke}%
\pgfsetdash{}{0pt}%
\pgfpathmoveto{\pgfqpoint{1.634246in}{2.891202in}}%
\pgfpathlineto{\pgfqpoint{1.758052in}{2.660718in}}%
\pgfusepath{stroke}%
\end{pgfscope}%
\begin{pgfscope}%
\pgfpathrectangle{\pgfqpoint{0.100000in}{0.212622in}}{\pgfqpoint{3.696000in}{3.696000in}}%
\pgfusepath{clip}%
\pgfsetrectcap%
\pgfsetroundjoin%
\pgfsetlinewidth{1.505625pt}%
\definecolor{currentstroke}{rgb}{1.000000,0.000000,0.000000}%
\pgfsetstrokecolor{currentstroke}%
\pgfsetdash{}{0pt}%
\pgfpathmoveto{\pgfqpoint{1.633295in}{2.887369in}}%
\pgfpathlineto{\pgfqpoint{1.752066in}{2.655869in}}%
\pgfusepath{stroke}%
\end{pgfscope}%
\begin{pgfscope}%
\pgfpathrectangle{\pgfqpoint{0.100000in}{0.212622in}}{\pgfqpoint{3.696000in}{3.696000in}}%
\pgfusepath{clip}%
\pgfsetrectcap%
\pgfsetroundjoin%
\pgfsetlinewidth{1.505625pt}%
\definecolor{currentstroke}{rgb}{1.000000,0.000000,0.000000}%
\pgfsetstrokecolor{currentstroke}%
\pgfsetdash{}{0pt}%
\pgfpathmoveto{\pgfqpoint{1.630898in}{2.882595in}}%
\pgfpathlineto{\pgfqpoint{1.752066in}{2.655869in}}%
\pgfusepath{stroke}%
\end{pgfscope}%
\begin{pgfscope}%
\pgfpathrectangle{\pgfqpoint{0.100000in}{0.212622in}}{\pgfqpoint{3.696000in}{3.696000in}}%
\pgfusepath{clip}%
\pgfsetrectcap%
\pgfsetroundjoin%
\pgfsetlinewidth{1.505625pt}%
\definecolor{currentstroke}{rgb}{1.000000,0.000000,0.000000}%
\pgfsetstrokecolor{currentstroke}%
\pgfsetdash{}{0pt}%
\pgfpathmoveto{\pgfqpoint{1.629474in}{2.880288in}}%
\pgfpathlineto{\pgfqpoint{1.752066in}{2.655869in}}%
\pgfusepath{stroke}%
\end{pgfscope}%
\begin{pgfscope}%
\pgfpathrectangle{\pgfqpoint{0.100000in}{0.212622in}}{\pgfqpoint{3.696000in}{3.696000in}}%
\pgfusepath{clip}%
\pgfsetrectcap%
\pgfsetroundjoin%
\pgfsetlinewidth{1.505625pt}%
\definecolor{currentstroke}{rgb}{1.000000,0.000000,0.000000}%
\pgfsetstrokecolor{currentstroke}%
\pgfsetdash{}{0pt}%
\pgfpathmoveto{\pgfqpoint{1.627804in}{2.874897in}}%
\pgfpathlineto{\pgfqpoint{1.746075in}{2.651016in}}%
\pgfusepath{stroke}%
\end{pgfscope}%
\begin{pgfscope}%
\pgfpathrectangle{\pgfqpoint{0.100000in}{0.212622in}}{\pgfqpoint{3.696000in}{3.696000in}}%
\pgfusepath{clip}%
\pgfsetrectcap%
\pgfsetroundjoin%
\pgfsetlinewidth{1.505625pt}%
\definecolor{currentstroke}{rgb}{1.000000,0.000000,0.000000}%
\pgfsetstrokecolor{currentstroke}%
\pgfsetdash{}{0pt}%
\pgfpathmoveto{\pgfqpoint{1.626791in}{2.871726in}}%
\pgfpathlineto{\pgfqpoint{1.746075in}{2.651016in}}%
\pgfusepath{stroke}%
\end{pgfscope}%
\begin{pgfscope}%
\pgfpathrectangle{\pgfqpoint{0.100000in}{0.212622in}}{\pgfqpoint{3.696000in}{3.696000in}}%
\pgfusepath{clip}%
\pgfsetrectcap%
\pgfsetroundjoin%
\pgfsetlinewidth{1.505625pt}%
\definecolor{currentstroke}{rgb}{1.000000,0.000000,0.000000}%
\pgfsetstrokecolor{currentstroke}%
\pgfsetdash{}{0pt}%
\pgfpathmoveto{\pgfqpoint{1.624673in}{2.867852in}}%
\pgfpathlineto{\pgfqpoint{1.746075in}{2.651016in}}%
\pgfusepath{stroke}%
\end{pgfscope}%
\begin{pgfscope}%
\pgfpathrectangle{\pgfqpoint{0.100000in}{0.212622in}}{\pgfqpoint{3.696000in}{3.696000in}}%
\pgfusepath{clip}%
\pgfsetrectcap%
\pgfsetroundjoin%
\pgfsetlinewidth{1.505625pt}%
\definecolor{currentstroke}{rgb}{1.000000,0.000000,0.000000}%
\pgfsetstrokecolor{currentstroke}%
\pgfsetdash{}{0pt}%
\pgfpathmoveto{\pgfqpoint{1.622203in}{2.864020in}}%
\pgfpathlineto{\pgfqpoint{1.740079in}{2.646158in}}%
\pgfusepath{stroke}%
\end{pgfscope}%
\begin{pgfscope}%
\pgfpathrectangle{\pgfqpoint{0.100000in}{0.212622in}}{\pgfqpoint{3.696000in}{3.696000in}}%
\pgfusepath{clip}%
\pgfsetrectcap%
\pgfsetroundjoin%
\pgfsetlinewidth{1.505625pt}%
\definecolor{currentstroke}{rgb}{1.000000,0.000000,0.000000}%
\pgfsetstrokecolor{currentstroke}%
\pgfsetdash{}{0pt}%
\pgfpathmoveto{\pgfqpoint{1.619554in}{2.857427in}}%
\pgfpathlineto{\pgfqpoint{1.740079in}{2.646158in}}%
\pgfusepath{stroke}%
\end{pgfscope}%
\begin{pgfscope}%
\pgfpathrectangle{\pgfqpoint{0.100000in}{0.212622in}}{\pgfqpoint{3.696000in}{3.696000in}}%
\pgfusepath{clip}%
\pgfsetrectcap%
\pgfsetroundjoin%
\pgfsetlinewidth{1.505625pt}%
\definecolor{currentstroke}{rgb}{1.000000,0.000000,0.000000}%
\pgfsetstrokecolor{currentstroke}%
\pgfsetdash{}{0pt}%
\pgfpathmoveto{\pgfqpoint{1.618575in}{2.853519in}}%
\pgfpathlineto{\pgfqpoint{1.740079in}{2.646158in}}%
\pgfusepath{stroke}%
\end{pgfscope}%
\begin{pgfscope}%
\pgfpathrectangle{\pgfqpoint{0.100000in}{0.212622in}}{\pgfqpoint{3.696000in}{3.696000in}}%
\pgfusepath{clip}%
\pgfsetrectcap%
\pgfsetroundjoin%
\pgfsetlinewidth{1.505625pt}%
\definecolor{currentstroke}{rgb}{1.000000,0.000000,0.000000}%
\pgfsetstrokecolor{currentstroke}%
\pgfsetdash{}{0pt}%
\pgfpathmoveto{\pgfqpoint{1.616651in}{2.849791in}}%
\pgfpathlineto{\pgfqpoint{1.734077in}{2.641296in}}%
\pgfusepath{stroke}%
\end{pgfscope}%
\begin{pgfscope}%
\pgfpathrectangle{\pgfqpoint{0.100000in}{0.212622in}}{\pgfqpoint{3.696000in}{3.696000in}}%
\pgfusepath{clip}%
\pgfsetrectcap%
\pgfsetroundjoin%
\pgfsetlinewidth{1.505625pt}%
\definecolor{currentstroke}{rgb}{1.000000,0.000000,0.000000}%
\pgfsetstrokecolor{currentstroke}%
\pgfsetdash{}{0pt}%
\pgfpathmoveto{\pgfqpoint{1.615534in}{2.848134in}}%
\pgfpathlineto{\pgfqpoint{1.734077in}{2.641296in}}%
\pgfusepath{stroke}%
\end{pgfscope}%
\begin{pgfscope}%
\pgfpathrectangle{\pgfqpoint{0.100000in}{0.212622in}}{\pgfqpoint{3.696000in}{3.696000in}}%
\pgfusepath{clip}%
\pgfsetrectcap%
\pgfsetroundjoin%
\pgfsetlinewidth{1.505625pt}%
\definecolor{currentstroke}{rgb}{1.000000,0.000000,0.000000}%
\pgfsetstrokecolor{currentstroke}%
\pgfsetdash{}{0pt}%
\pgfpathmoveto{\pgfqpoint{1.615137in}{2.845074in}}%
\pgfpathlineto{\pgfqpoint{1.734077in}{2.641296in}}%
\pgfusepath{stroke}%
\end{pgfscope}%
\begin{pgfscope}%
\pgfpathrectangle{\pgfqpoint{0.100000in}{0.212622in}}{\pgfqpoint{3.696000in}{3.696000in}}%
\pgfusepath{clip}%
\pgfsetrectcap%
\pgfsetroundjoin%
\pgfsetlinewidth{1.505625pt}%
\definecolor{currentstroke}{rgb}{1.000000,0.000000,0.000000}%
\pgfsetstrokecolor{currentstroke}%
\pgfsetdash{}{0pt}%
\pgfpathmoveto{\pgfqpoint{1.615049in}{2.841214in}}%
\pgfpathlineto{\pgfqpoint{1.734077in}{2.641296in}}%
\pgfusepath{stroke}%
\end{pgfscope}%
\begin{pgfscope}%
\pgfpathrectangle{\pgfqpoint{0.100000in}{0.212622in}}{\pgfqpoint{3.696000in}{3.696000in}}%
\pgfusepath{clip}%
\pgfsetrectcap%
\pgfsetroundjoin%
\pgfsetlinewidth{1.505625pt}%
\definecolor{currentstroke}{rgb}{1.000000,0.000000,0.000000}%
\pgfsetstrokecolor{currentstroke}%
\pgfsetdash{}{0pt}%
\pgfpathmoveto{\pgfqpoint{1.613054in}{2.836530in}}%
\pgfpathlineto{\pgfqpoint{1.728070in}{2.636429in}}%
\pgfusepath{stroke}%
\end{pgfscope}%
\begin{pgfscope}%
\pgfpathrectangle{\pgfqpoint{0.100000in}{0.212622in}}{\pgfqpoint{3.696000in}{3.696000in}}%
\pgfusepath{clip}%
\pgfsetrectcap%
\pgfsetroundjoin%
\pgfsetlinewidth{1.505625pt}%
\definecolor{currentstroke}{rgb}{1.000000,0.000000,0.000000}%
\pgfsetstrokecolor{currentstroke}%
\pgfsetdash{}{0pt}%
\pgfpathmoveto{\pgfqpoint{1.610728in}{2.831674in}}%
\pgfpathlineto{\pgfqpoint{1.728070in}{2.636429in}}%
\pgfusepath{stroke}%
\end{pgfscope}%
\begin{pgfscope}%
\pgfpathrectangle{\pgfqpoint{0.100000in}{0.212622in}}{\pgfqpoint{3.696000in}{3.696000in}}%
\pgfusepath{clip}%
\pgfsetrectcap%
\pgfsetroundjoin%
\pgfsetlinewidth{1.505625pt}%
\definecolor{currentstroke}{rgb}{1.000000,0.000000,0.000000}%
\pgfsetstrokecolor{currentstroke}%
\pgfsetdash{}{0pt}%
\pgfpathmoveto{\pgfqpoint{1.609939in}{2.824380in}}%
\pgfpathlineto{\pgfqpoint{1.722058in}{2.631558in}}%
\pgfusepath{stroke}%
\end{pgfscope}%
\begin{pgfscope}%
\pgfpathrectangle{\pgfqpoint{0.100000in}{0.212622in}}{\pgfqpoint{3.696000in}{3.696000in}}%
\pgfusepath{clip}%
\pgfsetrectcap%
\pgfsetroundjoin%
\pgfsetlinewidth{1.505625pt}%
\definecolor{currentstroke}{rgb}{1.000000,0.000000,0.000000}%
\pgfsetstrokecolor{currentstroke}%
\pgfsetdash{}{0pt}%
\pgfpathmoveto{\pgfqpoint{1.608509in}{2.820196in}}%
\pgfpathlineto{\pgfqpoint{1.722058in}{2.631558in}}%
\pgfusepath{stroke}%
\end{pgfscope}%
\begin{pgfscope}%
\pgfpathrectangle{\pgfqpoint{0.100000in}{0.212622in}}{\pgfqpoint{3.696000in}{3.696000in}}%
\pgfusepath{clip}%
\pgfsetrectcap%
\pgfsetroundjoin%
\pgfsetlinewidth{1.505625pt}%
\definecolor{currentstroke}{rgb}{1.000000,0.000000,0.000000}%
\pgfsetstrokecolor{currentstroke}%
\pgfsetdash{}{0pt}%
\pgfpathmoveto{\pgfqpoint{1.606017in}{2.815555in}}%
\pgfpathlineto{\pgfqpoint{1.722058in}{2.631558in}}%
\pgfusepath{stroke}%
\end{pgfscope}%
\begin{pgfscope}%
\pgfpathrectangle{\pgfqpoint{0.100000in}{0.212622in}}{\pgfqpoint{3.696000in}{3.696000in}}%
\pgfusepath{clip}%
\pgfsetrectcap%
\pgfsetroundjoin%
\pgfsetlinewidth{1.505625pt}%
\definecolor{currentstroke}{rgb}{1.000000,0.000000,0.000000}%
\pgfsetstrokecolor{currentstroke}%
\pgfsetdash{}{0pt}%
\pgfpathmoveto{\pgfqpoint{1.604549in}{2.813190in}}%
\pgfpathlineto{\pgfqpoint{1.716040in}{2.626683in}}%
\pgfusepath{stroke}%
\end{pgfscope}%
\begin{pgfscope}%
\pgfpathrectangle{\pgfqpoint{0.100000in}{0.212622in}}{\pgfqpoint{3.696000in}{3.696000in}}%
\pgfusepath{clip}%
\pgfsetrectcap%
\pgfsetroundjoin%
\pgfsetlinewidth{1.505625pt}%
\definecolor{currentstroke}{rgb}{1.000000,0.000000,0.000000}%
\pgfsetstrokecolor{currentstroke}%
\pgfsetdash{}{0pt}%
\pgfpathmoveto{\pgfqpoint{1.603075in}{2.809894in}}%
\pgfpathlineto{\pgfqpoint{1.716040in}{2.626683in}}%
\pgfusepath{stroke}%
\end{pgfscope}%
\begin{pgfscope}%
\pgfpathrectangle{\pgfqpoint{0.100000in}{0.212622in}}{\pgfqpoint{3.696000in}{3.696000in}}%
\pgfusepath{clip}%
\pgfsetrectcap%
\pgfsetroundjoin%
\pgfsetlinewidth{1.505625pt}%
\definecolor{currentstroke}{rgb}{1.000000,0.000000,0.000000}%
\pgfsetstrokecolor{currentstroke}%
\pgfsetdash{}{0pt}%
\pgfpathmoveto{\pgfqpoint{1.602322in}{2.807971in}}%
\pgfpathlineto{\pgfqpoint{1.716040in}{2.626683in}}%
\pgfusepath{stroke}%
\end{pgfscope}%
\begin{pgfscope}%
\pgfpathrectangle{\pgfqpoint{0.100000in}{0.212622in}}{\pgfqpoint{3.696000in}{3.696000in}}%
\pgfusepath{clip}%
\pgfsetrectcap%
\pgfsetroundjoin%
\pgfsetlinewidth{1.505625pt}%
\definecolor{currentstroke}{rgb}{1.000000,0.000000,0.000000}%
\pgfsetstrokecolor{currentstroke}%
\pgfsetdash{}{0pt}%
\pgfpathmoveto{\pgfqpoint{1.600348in}{2.804783in}}%
\pgfpathlineto{\pgfqpoint{1.716040in}{2.626683in}}%
\pgfusepath{stroke}%
\end{pgfscope}%
\begin{pgfscope}%
\pgfpathrectangle{\pgfqpoint{0.100000in}{0.212622in}}{\pgfqpoint{3.696000in}{3.696000in}}%
\pgfusepath{clip}%
\pgfsetrectcap%
\pgfsetroundjoin%
\pgfsetlinewidth{1.505625pt}%
\definecolor{currentstroke}{rgb}{1.000000,0.000000,0.000000}%
\pgfsetstrokecolor{currentstroke}%
\pgfsetdash{}{0pt}%
\pgfpathmoveto{\pgfqpoint{1.598082in}{2.801175in}}%
\pgfpathlineto{\pgfqpoint{1.716040in}{2.626683in}}%
\pgfusepath{stroke}%
\end{pgfscope}%
\begin{pgfscope}%
\pgfpathrectangle{\pgfqpoint{0.100000in}{0.212622in}}{\pgfqpoint{3.696000in}{3.696000in}}%
\pgfusepath{clip}%
\pgfsetrectcap%
\pgfsetroundjoin%
\pgfsetlinewidth{1.505625pt}%
\definecolor{currentstroke}{rgb}{1.000000,0.000000,0.000000}%
\pgfsetstrokecolor{currentstroke}%
\pgfsetdash{}{0pt}%
\pgfpathmoveto{\pgfqpoint{1.595594in}{2.794825in}}%
\pgfpathlineto{\pgfqpoint{1.710017in}{2.621804in}}%
\pgfusepath{stroke}%
\end{pgfscope}%
\begin{pgfscope}%
\pgfpathrectangle{\pgfqpoint{0.100000in}{0.212622in}}{\pgfqpoint{3.696000in}{3.696000in}}%
\pgfusepath{clip}%
\pgfsetrectcap%
\pgfsetroundjoin%
\pgfsetlinewidth{1.505625pt}%
\definecolor{currentstroke}{rgb}{1.000000,0.000000,0.000000}%
\pgfsetstrokecolor{currentstroke}%
\pgfsetdash{}{0pt}%
\pgfpathmoveto{\pgfqpoint{1.594629in}{2.791107in}}%
\pgfpathlineto{\pgfqpoint{1.710017in}{2.621804in}}%
\pgfusepath{stroke}%
\end{pgfscope}%
\begin{pgfscope}%
\pgfpathrectangle{\pgfqpoint{0.100000in}{0.212622in}}{\pgfqpoint{3.696000in}{3.696000in}}%
\pgfusepath{clip}%
\pgfsetrectcap%
\pgfsetroundjoin%
\pgfsetlinewidth{1.505625pt}%
\definecolor{currentstroke}{rgb}{1.000000,0.000000,0.000000}%
\pgfsetstrokecolor{currentstroke}%
\pgfsetdash{}{0pt}%
\pgfpathmoveto{\pgfqpoint{1.591947in}{2.786207in}}%
\pgfpathlineto{\pgfqpoint{1.703989in}{2.616920in}}%
\pgfusepath{stroke}%
\end{pgfscope}%
\begin{pgfscope}%
\pgfpathrectangle{\pgfqpoint{0.100000in}{0.212622in}}{\pgfqpoint{3.696000in}{3.696000in}}%
\pgfusepath{clip}%
\pgfsetrectcap%
\pgfsetroundjoin%
\pgfsetlinewidth{1.505625pt}%
\definecolor{currentstroke}{rgb}{1.000000,0.000000,0.000000}%
\pgfsetstrokecolor{currentstroke}%
\pgfsetdash{}{0pt}%
\pgfpathmoveto{\pgfqpoint{1.590292in}{2.783657in}}%
\pgfpathlineto{\pgfqpoint{1.703989in}{2.616920in}}%
\pgfusepath{stroke}%
\end{pgfscope}%
\begin{pgfscope}%
\pgfpathrectangle{\pgfqpoint{0.100000in}{0.212622in}}{\pgfqpoint{3.696000in}{3.696000in}}%
\pgfusepath{clip}%
\pgfsetrectcap%
\pgfsetroundjoin%
\pgfsetlinewidth{1.505625pt}%
\definecolor{currentstroke}{rgb}{1.000000,0.000000,0.000000}%
\pgfsetstrokecolor{currentstroke}%
\pgfsetdash{}{0pt}%
\pgfpathmoveto{\pgfqpoint{1.588033in}{2.779183in}}%
\pgfpathlineto{\pgfqpoint{1.703989in}{2.616920in}}%
\pgfusepath{stroke}%
\end{pgfscope}%
\begin{pgfscope}%
\pgfpathrectangle{\pgfqpoint{0.100000in}{0.212622in}}{\pgfqpoint{3.696000in}{3.696000in}}%
\pgfusepath{clip}%
\pgfsetrectcap%
\pgfsetroundjoin%
\pgfsetlinewidth{1.505625pt}%
\definecolor{currentstroke}{rgb}{1.000000,0.000000,0.000000}%
\pgfsetstrokecolor{currentstroke}%
\pgfsetdash{}{0pt}%
\pgfpathmoveto{\pgfqpoint{1.586270in}{2.772677in}}%
\pgfpathlineto{\pgfqpoint{1.703989in}{2.616920in}}%
\pgfusepath{stroke}%
\end{pgfscope}%
\begin{pgfscope}%
\pgfpathrectangle{\pgfqpoint{0.100000in}{0.212622in}}{\pgfqpoint{3.696000in}{3.696000in}}%
\pgfusepath{clip}%
\pgfsetrectcap%
\pgfsetroundjoin%
\pgfsetlinewidth{1.505625pt}%
\definecolor{currentstroke}{rgb}{1.000000,0.000000,0.000000}%
\pgfsetstrokecolor{currentstroke}%
\pgfsetdash{}{0pt}%
\pgfpathmoveto{\pgfqpoint{1.584751in}{2.769439in}}%
\pgfpathlineto{\pgfqpoint{1.697955in}{2.612032in}}%
\pgfusepath{stroke}%
\end{pgfscope}%
\begin{pgfscope}%
\pgfpathrectangle{\pgfqpoint{0.100000in}{0.212622in}}{\pgfqpoint{3.696000in}{3.696000in}}%
\pgfusepath{clip}%
\pgfsetrectcap%
\pgfsetroundjoin%
\pgfsetlinewidth{1.505625pt}%
\definecolor{currentstroke}{rgb}{1.000000,0.000000,0.000000}%
\pgfsetstrokecolor{currentstroke}%
\pgfsetdash{}{0pt}%
\pgfpathmoveto{\pgfqpoint{1.583741in}{2.767798in}}%
\pgfpathlineto{\pgfqpoint{1.697955in}{2.612032in}}%
\pgfusepath{stroke}%
\end{pgfscope}%
\begin{pgfscope}%
\pgfpathrectangle{\pgfqpoint{0.100000in}{0.212622in}}{\pgfqpoint{3.696000in}{3.696000in}}%
\pgfusepath{clip}%
\pgfsetrectcap%
\pgfsetroundjoin%
\pgfsetlinewidth{1.505625pt}%
\definecolor{currentstroke}{rgb}{1.000000,0.000000,0.000000}%
\pgfsetstrokecolor{currentstroke}%
\pgfsetdash{}{0pt}%
\pgfpathmoveto{\pgfqpoint{1.582423in}{2.765612in}}%
\pgfpathlineto{\pgfqpoint{1.697955in}{2.612032in}}%
\pgfusepath{stroke}%
\end{pgfscope}%
\begin{pgfscope}%
\pgfpathrectangle{\pgfqpoint{0.100000in}{0.212622in}}{\pgfqpoint{3.696000in}{3.696000in}}%
\pgfusepath{clip}%
\pgfsetrectcap%
\pgfsetroundjoin%
\pgfsetlinewidth{1.505625pt}%
\definecolor{currentstroke}{rgb}{1.000000,0.000000,0.000000}%
\pgfsetstrokecolor{currentstroke}%
\pgfsetdash{}{0pt}%
\pgfpathmoveto{\pgfqpoint{1.581088in}{2.761107in}}%
\pgfpathlineto{\pgfqpoint{1.697955in}{2.612032in}}%
\pgfusepath{stroke}%
\end{pgfscope}%
\begin{pgfscope}%
\pgfpathrectangle{\pgfqpoint{0.100000in}{0.212622in}}{\pgfqpoint{3.696000in}{3.696000in}}%
\pgfusepath{clip}%
\pgfsetrectcap%
\pgfsetroundjoin%
\pgfsetlinewidth{1.505625pt}%
\definecolor{currentstroke}{rgb}{1.000000,0.000000,0.000000}%
\pgfsetstrokecolor{currentstroke}%
\pgfsetdash{}{0pt}%
\pgfpathmoveto{\pgfqpoint{1.580147in}{2.758707in}}%
\pgfpathlineto{\pgfqpoint{1.697955in}{2.612032in}}%
\pgfusepath{stroke}%
\end{pgfscope}%
\begin{pgfscope}%
\pgfpathrectangle{\pgfqpoint{0.100000in}{0.212622in}}{\pgfqpoint{3.696000in}{3.696000in}}%
\pgfusepath{clip}%
\pgfsetrectcap%
\pgfsetroundjoin%
\pgfsetlinewidth{1.505625pt}%
\definecolor{currentstroke}{rgb}{1.000000,0.000000,0.000000}%
\pgfsetstrokecolor{currentstroke}%
\pgfsetdash{}{0pt}%
\pgfpathmoveto{\pgfqpoint{1.578535in}{2.756415in}}%
\pgfpathlineto{\pgfqpoint{1.691916in}{2.607140in}}%
\pgfusepath{stroke}%
\end{pgfscope}%
\begin{pgfscope}%
\pgfpathrectangle{\pgfqpoint{0.100000in}{0.212622in}}{\pgfqpoint{3.696000in}{3.696000in}}%
\pgfusepath{clip}%
\pgfsetrectcap%
\pgfsetroundjoin%
\pgfsetlinewidth{1.505625pt}%
\definecolor{currentstroke}{rgb}{1.000000,0.000000,0.000000}%
\pgfsetstrokecolor{currentstroke}%
\pgfsetdash{}{0pt}%
\pgfpathmoveto{\pgfqpoint{1.576664in}{2.753607in}}%
\pgfpathlineto{\pgfqpoint{1.691916in}{2.607140in}}%
\pgfusepath{stroke}%
\end{pgfscope}%
\begin{pgfscope}%
\pgfpathrectangle{\pgfqpoint{0.100000in}{0.212622in}}{\pgfqpoint{3.696000in}{3.696000in}}%
\pgfusepath{clip}%
\pgfsetrectcap%
\pgfsetroundjoin%
\pgfsetlinewidth{1.505625pt}%
\definecolor{currentstroke}{rgb}{1.000000,0.000000,0.000000}%
\pgfsetstrokecolor{currentstroke}%
\pgfsetdash{}{0pt}%
\pgfpathmoveto{\pgfqpoint{1.574596in}{2.748380in}}%
\pgfpathlineto{\pgfqpoint{1.691916in}{2.607140in}}%
\pgfusepath{stroke}%
\end{pgfscope}%
\begin{pgfscope}%
\pgfpathrectangle{\pgfqpoint{0.100000in}{0.212622in}}{\pgfqpoint{3.696000in}{3.696000in}}%
\pgfusepath{clip}%
\pgfsetrectcap%
\pgfsetroundjoin%
\pgfsetlinewidth{1.505625pt}%
\definecolor{currentstroke}{rgb}{1.000000,0.000000,0.000000}%
\pgfsetstrokecolor{currentstroke}%
\pgfsetdash{}{0pt}%
\pgfpathmoveto{\pgfqpoint{1.573522in}{2.745253in}}%
\pgfpathlineto{\pgfqpoint{1.685872in}{2.602243in}}%
\pgfusepath{stroke}%
\end{pgfscope}%
\begin{pgfscope}%
\pgfpathrectangle{\pgfqpoint{0.100000in}{0.212622in}}{\pgfqpoint{3.696000in}{3.696000in}}%
\pgfusepath{clip}%
\pgfsetrectcap%
\pgfsetroundjoin%
\pgfsetlinewidth{1.505625pt}%
\definecolor{currentstroke}{rgb}{1.000000,0.000000,0.000000}%
\pgfsetstrokecolor{currentstroke}%
\pgfsetdash{}{0pt}%
\pgfpathmoveto{\pgfqpoint{1.571110in}{2.741271in}}%
\pgfpathlineto{\pgfqpoint{1.685872in}{2.602243in}}%
\pgfusepath{stroke}%
\end{pgfscope}%
\begin{pgfscope}%
\pgfpathrectangle{\pgfqpoint{0.100000in}{0.212622in}}{\pgfqpoint{3.696000in}{3.696000in}}%
\pgfusepath{clip}%
\pgfsetrectcap%
\pgfsetroundjoin%
\pgfsetlinewidth{1.505625pt}%
\definecolor{currentstroke}{rgb}{1.000000,0.000000,0.000000}%
\pgfsetstrokecolor{currentstroke}%
\pgfsetdash{}{0pt}%
\pgfpathmoveto{\pgfqpoint{1.569735in}{2.739212in}}%
\pgfpathlineto{\pgfqpoint{1.685872in}{2.602243in}}%
\pgfusepath{stroke}%
\end{pgfscope}%
\begin{pgfscope}%
\pgfpathrectangle{\pgfqpoint{0.100000in}{0.212622in}}{\pgfqpoint{3.696000in}{3.696000in}}%
\pgfusepath{clip}%
\pgfsetrectcap%
\pgfsetroundjoin%
\pgfsetlinewidth{1.505625pt}%
\definecolor{currentstroke}{rgb}{1.000000,0.000000,0.000000}%
\pgfsetstrokecolor{currentstroke}%
\pgfsetdash{}{0pt}%
\pgfpathmoveto{\pgfqpoint{1.567916in}{2.735025in}}%
\pgfpathlineto{\pgfqpoint{1.685872in}{2.602243in}}%
\pgfusepath{stroke}%
\end{pgfscope}%
\begin{pgfscope}%
\pgfpathrectangle{\pgfqpoint{0.100000in}{0.212622in}}{\pgfqpoint{3.696000in}{3.696000in}}%
\pgfusepath{clip}%
\pgfsetrectcap%
\pgfsetroundjoin%
\pgfsetlinewidth{1.505625pt}%
\definecolor{currentstroke}{rgb}{1.000000,0.000000,0.000000}%
\pgfsetstrokecolor{currentstroke}%
\pgfsetdash{}{0pt}%
\pgfpathmoveto{\pgfqpoint{1.566232in}{2.729347in}}%
\pgfpathlineto{\pgfqpoint{1.679822in}{2.597342in}}%
\pgfusepath{stroke}%
\end{pgfscope}%
\begin{pgfscope}%
\pgfpathrectangle{\pgfqpoint{0.100000in}{0.212622in}}{\pgfqpoint{3.696000in}{3.696000in}}%
\pgfusepath{clip}%
\pgfsetrectcap%
\pgfsetroundjoin%
\pgfsetlinewidth{1.505625pt}%
\definecolor{currentstroke}{rgb}{1.000000,0.000000,0.000000}%
\pgfsetstrokecolor{currentstroke}%
\pgfsetdash{}{0pt}%
\pgfpathmoveto{\pgfqpoint{1.563111in}{2.723531in}}%
\pgfpathlineto{\pgfqpoint{1.679822in}{2.597342in}}%
\pgfusepath{stroke}%
\end{pgfscope}%
\begin{pgfscope}%
\pgfpathrectangle{\pgfqpoint{0.100000in}{0.212622in}}{\pgfqpoint{3.696000in}{3.696000in}}%
\pgfusepath{clip}%
\pgfsetrectcap%
\pgfsetroundjoin%
\pgfsetlinewidth{1.505625pt}%
\definecolor{currentstroke}{rgb}{1.000000,0.000000,0.000000}%
\pgfsetstrokecolor{currentstroke}%
\pgfsetdash{}{0pt}%
\pgfpathmoveto{\pgfqpoint{1.561138in}{2.720670in}}%
\pgfpathlineto{\pgfqpoint{1.679822in}{2.597342in}}%
\pgfusepath{stroke}%
\end{pgfscope}%
\begin{pgfscope}%
\pgfpathrectangle{\pgfqpoint{0.100000in}{0.212622in}}{\pgfqpoint{3.696000in}{3.696000in}}%
\pgfusepath{clip}%
\pgfsetrectcap%
\pgfsetroundjoin%
\pgfsetlinewidth{1.505625pt}%
\definecolor{currentstroke}{rgb}{1.000000,0.000000,0.000000}%
\pgfsetstrokecolor{currentstroke}%
\pgfsetdash{}{0pt}%
\pgfpathmoveto{\pgfqpoint{1.558890in}{2.717113in}}%
\pgfpathlineto{\pgfqpoint{1.673767in}{2.592437in}}%
\pgfusepath{stroke}%
\end{pgfscope}%
\begin{pgfscope}%
\pgfpathrectangle{\pgfqpoint{0.100000in}{0.212622in}}{\pgfqpoint{3.696000in}{3.696000in}}%
\pgfusepath{clip}%
\pgfsetrectcap%
\pgfsetroundjoin%
\pgfsetlinewidth{1.505625pt}%
\definecolor{currentstroke}{rgb}{1.000000,0.000000,0.000000}%
\pgfsetstrokecolor{currentstroke}%
\pgfsetdash{}{0pt}%
\pgfpathmoveto{\pgfqpoint{1.556986in}{2.710449in}}%
\pgfpathlineto{\pgfqpoint{1.673767in}{2.592437in}}%
\pgfusepath{stroke}%
\end{pgfscope}%
\begin{pgfscope}%
\pgfpathrectangle{\pgfqpoint{0.100000in}{0.212622in}}{\pgfqpoint{3.696000in}{3.696000in}}%
\pgfusepath{clip}%
\pgfsetrectcap%
\pgfsetroundjoin%
\pgfsetlinewidth{1.505625pt}%
\definecolor{currentstroke}{rgb}{1.000000,0.000000,0.000000}%
\pgfsetstrokecolor{currentstroke}%
\pgfsetdash{}{0pt}%
\pgfpathmoveto{\pgfqpoint{1.555791in}{2.706639in}}%
\pgfpathlineto{\pgfqpoint{1.673767in}{2.592437in}}%
\pgfusepath{stroke}%
\end{pgfscope}%
\begin{pgfscope}%
\pgfpathrectangle{\pgfqpoint{0.100000in}{0.212622in}}{\pgfqpoint{3.696000in}{3.696000in}}%
\pgfusepath{clip}%
\pgfsetrectcap%
\pgfsetroundjoin%
\pgfsetlinewidth{1.505625pt}%
\definecolor{currentstroke}{rgb}{1.000000,0.000000,0.000000}%
\pgfsetstrokecolor{currentstroke}%
\pgfsetdash{}{0pt}%
\pgfpathmoveto{\pgfqpoint{1.553439in}{2.702501in}}%
\pgfpathlineto{\pgfqpoint{1.667706in}{2.587527in}}%
\pgfusepath{stroke}%
\end{pgfscope}%
\begin{pgfscope}%
\pgfpathrectangle{\pgfqpoint{0.100000in}{0.212622in}}{\pgfqpoint{3.696000in}{3.696000in}}%
\pgfusepath{clip}%
\pgfsetrectcap%
\pgfsetroundjoin%
\pgfsetlinewidth{1.505625pt}%
\definecolor{currentstroke}{rgb}{1.000000,0.000000,0.000000}%
\pgfsetstrokecolor{currentstroke}%
\pgfsetdash{}{0pt}%
\pgfpathmoveto{\pgfqpoint{1.550609in}{2.698461in}}%
\pgfpathlineto{\pgfqpoint{1.667706in}{2.587527in}}%
\pgfusepath{stroke}%
\end{pgfscope}%
\begin{pgfscope}%
\pgfpathrectangle{\pgfqpoint{0.100000in}{0.212622in}}{\pgfqpoint{3.696000in}{3.696000in}}%
\pgfusepath{clip}%
\pgfsetrectcap%
\pgfsetroundjoin%
\pgfsetlinewidth{1.505625pt}%
\definecolor{currentstroke}{rgb}{1.000000,0.000000,0.000000}%
\pgfsetstrokecolor{currentstroke}%
\pgfsetdash{}{0pt}%
\pgfpathmoveto{\pgfqpoint{1.547185in}{2.691858in}}%
\pgfpathlineto{\pgfqpoint{1.661640in}{2.582613in}}%
\pgfusepath{stroke}%
\end{pgfscope}%
\begin{pgfscope}%
\pgfpathrectangle{\pgfqpoint{0.100000in}{0.212622in}}{\pgfqpoint{3.696000in}{3.696000in}}%
\pgfusepath{clip}%
\pgfsetrectcap%
\pgfsetroundjoin%
\pgfsetlinewidth{1.505625pt}%
\definecolor{currentstroke}{rgb}{1.000000,0.000000,0.000000}%
\pgfsetstrokecolor{currentstroke}%
\pgfsetdash{}{0pt}%
\pgfpathmoveto{\pgfqpoint{1.544148in}{2.682510in}}%
\pgfpathlineto{\pgfqpoint{1.661640in}{2.582613in}}%
\pgfusepath{stroke}%
\end{pgfscope}%
\begin{pgfscope}%
\pgfpathrectangle{\pgfqpoint{0.100000in}{0.212622in}}{\pgfqpoint{3.696000in}{3.696000in}}%
\pgfusepath{clip}%
\pgfsetrectcap%
\pgfsetroundjoin%
\pgfsetlinewidth{1.505625pt}%
\definecolor{currentstroke}{rgb}{1.000000,0.000000,0.000000}%
\pgfsetstrokecolor{currentstroke}%
\pgfsetdash{}{0pt}%
\pgfpathmoveto{\pgfqpoint{1.542047in}{2.677846in}}%
\pgfpathlineto{\pgfqpoint{1.661640in}{2.582613in}}%
\pgfusepath{stroke}%
\end{pgfscope}%
\begin{pgfscope}%
\pgfpathrectangle{\pgfqpoint{0.100000in}{0.212622in}}{\pgfqpoint{3.696000in}{3.696000in}}%
\pgfusepath{clip}%
\pgfsetrectcap%
\pgfsetroundjoin%
\pgfsetlinewidth{1.505625pt}%
\definecolor{currentstroke}{rgb}{1.000000,0.000000,0.000000}%
\pgfsetstrokecolor{currentstroke}%
\pgfsetdash{}{0pt}%
\pgfpathmoveto{\pgfqpoint{1.540648in}{2.675621in}}%
\pgfpathlineto{\pgfqpoint{1.655569in}{2.577694in}}%
\pgfusepath{stroke}%
\end{pgfscope}%
\begin{pgfscope}%
\pgfpathrectangle{\pgfqpoint{0.100000in}{0.212622in}}{\pgfqpoint{3.696000in}{3.696000in}}%
\pgfusepath{clip}%
\pgfsetrectcap%
\pgfsetroundjoin%
\pgfsetlinewidth{1.505625pt}%
\definecolor{currentstroke}{rgb}{1.000000,0.000000,0.000000}%
\pgfsetstrokecolor{currentstroke}%
\pgfsetdash{}{0pt}%
\pgfpathmoveto{\pgfqpoint{1.538641in}{2.672670in}}%
\pgfpathlineto{\pgfqpoint{1.655569in}{2.577694in}}%
\pgfusepath{stroke}%
\end{pgfscope}%
\begin{pgfscope}%
\pgfpathrectangle{\pgfqpoint{0.100000in}{0.212622in}}{\pgfqpoint{3.696000in}{3.696000in}}%
\pgfusepath{clip}%
\pgfsetrectcap%
\pgfsetroundjoin%
\pgfsetlinewidth{1.505625pt}%
\definecolor{currentstroke}{rgb}{1.000000,0.000000,0.000000}%
\pgfsetstrokecolor{currentstroke}%
\pgfsetdash{}{0pt}%
\pgfpathmoveto{\pgfqpoint{1.535890in}{2.667177in}}%
\pgfpathlineto{\pgfqpoint{1.655569in}{2.577694in}}%
\pgfusepath{stroke}%
\end{pgfscope}%
\begin{pgfscope}%
\pgfpathrectangle{\pgfqpoint{0.100000in}{0.212622in}}{\pgfqpoint{3.696000in}{3.696000in}}%
\pgfusepath{clip}%
\pgfsetrectcap%
\pgfsetroundjoin%
\pgfsetlinewidth{1.505625pt}%
\definecolor{currentstroke}{rgb}{1.000000,0.000000,0.000000}%
\pgfsetstrokecolor{currentstroke}%
\pgfsetdash{}{0pt}%
\pgfpathmoveto{\pgfqpoint{1.533883in}{2.659601in}}%
\pgfpathlineto{\pgfqpoint{1.649492in}{2.572771in}}%
\pgfusepath{stroke}%
\end{pgfscope}%
\begin{pgfscope}%
\pgfpathrectangle{\pgfqpoint{0.100000in}{0.212622in}}{\pgfqpoint{3.696000in}{3.696000in}}%
\pgfusepath{clip}%
\pgfsetrectcap%
\pgfsetroundjoin%
\pgfsetlinewidth{1.505625pt}%
\definecolor{currentstroke}{rgb}{1.000000,0.000000,0.000000}%
\pgfsetstrokecolor{currentstroke}%
\pgfsetdash{}{0pt}%
\pgfpathmoveto{\pgfqpoint{1.532267in}{2.655974in}}%
\pgfpathlineto{\pgfqpoint{1.649492in}{2.572771in}}%
\pgfusepath{stroke}%
\end{pgfscope}%
\begin{pgfscope}%
\pgfpathrectangle{\pgfqpoint{0.100000in}{0.212622in}}{\pgfqpoint{3.696000in}{3.696000in}}%
\pgfusepath{clip}%
\pgfsetrectcap%
\pgfsetroundjoin%
\pgfsetlinewidth{1.505625pt}%
\definecolor{currentstroke}{rgb}{1.000000,0.000000,0.000000}%
\pgfsetstrokecolor{currentstroke}%
\pgfsetdash{}{0pt}%
\pgfpathmoveto{\pgfqpoint{1.531050in}{2.654141in}}%
\pgfpathlineto{\pgfqpoint{1.649492in}{2.572771in}}%
\pgfusepath{stroke}%
\end{pgfscope}%
\begin{pgfscope}%
\pgfpathrectangle{\pgfqpoint{0.100000in}{0.212622in}}{\pgfqpoint{3.696000in}{3.696000in}}%
\pgfusepath{clip}%
\pgfsetrectcap%
\pgfsetroundjoin%
\pgfsetlinewidth{1.505625pt}%
\definecolor{currentstroke}{rgb}{1.000000,0.000000,0.000000}%
\pgfsetstrokecolor{currentstroke}%
\pgfsetdash{}{0pt}%
\pgfpathmoveto{\pgfqpoint{1.529576in}{2.651980in}}%
\pgfpathlineto{\pgfqpoint{1.649492in}{2.572771in}}%
\pgfusepath{stroke}%
\end{pgfscope}%
\begin{pgfscope}%
\pgfpathrectangle{\pgfqpoint{0.100000in}{0.212622in}}{\pgfqpoint{3.696000in}{3.696000in}}%
\pgfusepath{clip}%
\pgfsetrectcap%
\pgfsetroundjoin%
\pgfsetlinewidth{1.505625pt}%
\definecolor{currentstroke}{rgb}{1.000000,0.000000,0.000000}%
\pgfsetstrokecolor{currentstroke}%
\pgfsetdash{}{0pt}%
\pgfpathmoveto{\pgfqpoint{1.527522in}{2.647488in}}%
\pgfpathlineto{\pgfqpoint{1.643410in}{2.567844in}}%
\pgfusepath{stroke}%
\end{pgfscope}%
\begin{pgfscope}%
\pgfpathrectangle{\pgfqpoint{0.100000in}{0.212622in}}{\pgfqpoint{3.696000in}{3.696000in}}%
\pgfusepath{clip}%
\pgfsetrectcap%
\pgfsetroundjoin%
\pgfsetlinewidth{1.505625pt}%
\definecolor{currentstroke}{rgb}{1.000000,0.000000,0.000000}%
\pgfsetstrokecolor{currentstroke}%
\pgfsetdash{}{0pt}%
\pgfpathmoveto{\pgfqpoint{1.525816in}{2.642287in}}%
\pgfpathlineto{\pgfqpoint{1.643410in}{2.567844in}}%
\pgfusepath{stroke}%
\end{pgfscope}%
\begin{pgfscope}%
\pgfpathrectangle{\pgfqpoint{0.100000in}{0.212622in}}{\pgfqpoint{3.696000in}{3.696000in}}%
\pgfusepath{clip}%
\pgfsetrectcap%
\pgfsetroundjoin%
\pgfsetlinewidth{1.505625pt}%
\definecolor{currentstroke}{rgb}{1.000000,0.000000,0.000000}%
\pgfsetstrokecolor{currentstroke}%
\pgfsetdash{}{0pt}%
\pgfpathmoveto{\pgfqpoint{1.522704in}{2.636667in}}%
\pgfpathlineto{\pgfqpoint{1.643410in}{2.567844in}}%
\pgfusepath{stroke}%
\end{pgfscope}%
\begin{pgfscope}%
\pgfpathrectangle{\pgfqpoint{0.100000in}{0.212622in}}{\pgfqpoint{3.696000in}{3.696000in}}%
\pgfusepath{clip}%
\pgfsetrectcap%
\pgfsetroundjoin%
\pgfsetlinewidth{1.505625pt}%
\definecolor{currentstroke}{rgb}{1.000000,0.000000,0.000000}%
\pgfsetstrokecolor{currentstroke}%
\pgfsetdash{}{0pt}%
\pgfpathmoveto{\pgfqpoint{1.518536in}{2.630951in}}%
\pgfpathlineto{\pgfqpoint{1.637323in}{2.562912in}}%
\pgfusepath{stroke}%
\end{pgfscope}%
\begin{pgfscope}%
\pgfpathrectangle{\pgfqpoint{0.100000in}{0.212622in}}{\pgfqpoint{3.696000in}{3.696000in}}%
\pgfusepath{clip}%
\pgfsetrectcap%
\pgfsetroundjoin%
\pgfsetlinewidth{1.505625pt}%
\definecolor{currentstroke}{rgb}{1.000000,0.000000,0.000000}%
\pgfsetstrokecolor{currentstroke}%
\pgfsetdash{}{0pt}%
\pgfpathmoveto{\pgfqpoint{1.514272in}{2.625267in}}%
\pgfpathlineto{\pgfqpoint{1.637323in}{2.562912in}}%
\pgfusepath{stroke}%
\end{pgfscope}%
\begin{pgfscope}%
\pgfpathrectangle{\pgfqpoint{0.100000in}{0.212622in}}{\pgfqpoint{3.696000in}{3.696000in}}%
\pgfusepath{clip}%
\pgfsetrectcap%
\pgfsetroundjoin%
\pgfsetlinewidth{1.505625pt}%
\definecolor{currentstroke}{rgb}{1.000000,0.000000,0.000000}%
\pgfsetstrokecolor{currentstroke}%
\pgfsetdash{}{0pt}%
\pgfpathmoveto{\pgfqpoint{1.510796in}{2.615074in}}%
\pgfpathlineto{\pgfqpoint{1.631230in}{2.557976in}}%
\pgfusepath{stroke}%
\end{pgfscope}%
\begin{pgfscope}%
\pgfpathrectangle{\pgfqpoint{0.100000in}{0.212622in}}{\pgfqpoint{3.696000in}{3.696000in}}%
\pgfusepath{clip}%
\pgfsetrectcap%
\pgfsetroundjoin%
\pgfsetlinewidth{1.505625pt}%
\definecolor{currentstroke}{rgb}{1.000000,0.000000,0.000000}%
\pgfsetstrokecolor{currentstroke}%
\pgfsetdash{}{0pt}%
\pgfpathmoveto{\pgfqpoint{1.508473in}{2.609571in}}%
\pgfpathlineto{\pgfqpoint{1.625132in}{2.553036in}}%
\pgfusepath{stroke}%
\end{pgfscope}%
\begin{pgfscope}%
\pgfpathrectangle{\pgfqpoint{0.100000in}{0.212622in}}{\pgfqpoint{3.696000in}{3.696000in}}%
\pgfusepath{clip}%
\pgfsetrectcap%
\pgfsetroundjoin%
\pgfsetlinewidth{1.505625pt}%
\definecolor{currentstroke}{rgb}{1.000000,0.000000,0.000000}%
\pgfsetstrokecolor{currentstroke}%
\pgfsetdash{}{0pt}%
\pgfpathmoveto{\pgfqpoint{1.504772in}{2.604136in}}%
\pgfpathlineto{\pgfqpoint{1.625132in}{2.553036in}}%
\pgfusepath{stroke}%
\end{pgfscope}%
\begin{pgfscope}%
\pgfpathrectangle{\pgfqpoint{0.100000in}{0.212622in}}{\pgfqpoint{3.696000in}{3.696000in}}%
\pgfusepath{clip}%
\pgfsetrectcap%
\pgfsetroundjoin%
\pgfsetlinewidth{1.505625pt}%
\definecolor{currentstroke}{rgb}{1.000000,0.000000,0.000000}%
\pgfsetstrokecolor{currentstroke}%
\pgfsetdash{}{0pt}%
\pgfpathmoveto{\pgfqpoint{1.500647in}{2.598524in}}%
\pgfpathlineto{\pgfqpoint{1.625132in}{2.553036in}}%
\pgfusepath{stroke}%
\end{pgfscope}%
\begin{pgfscope}%
\pgfpathrectangle{\pgfqpoint{0.100000in}{0.212622in}}{\pgfqpoint{3.696000in}{3.696000in}}%
\pgfusepath{clip}%
\pgfsetrectcap%
\pgfsetroundjoin%
\pgfsetlinewidth{1.505625pt}%
\definecolor{currentstroke}{rgb}{1.000000,0.000000,0.000000}%
\pgfsetstrokecolor{currentstroke}%
\pgfsetdash{}{0pt}%
\pgfpathmoveto{\pgfqpoint{1.496652in}{2.588463in}}%
\pgfpathlineto{\pgfqpoint{1.619028in}{2.548091in}}%
\pgfusepath{stroke}%
\end{pgfscope}%
\begin{pgfscope}%
\pgfpathrectangle{\pgfqpoint{0.100000in}{0.212622in}}{\pgfqpoint{3.696000in}{3.696000in}}%
\pgfusepath{clip}%
\pgfsetrectcap%
\pgfsetroundjoin%
\pgfsetlinewidth{1.505625pt}%
\definecolor{currentstroke}{rgb}{1.000000,0.000000,0.000000}%
\pgfsetstrokecolor{currentstroke}%
\pgfsetdash{}{0pt}%
\pgfpathmoveto{\pgfqpoint{1.494530in}{2.575129in}}%
\pgfpathlineto{\pgfqpoint{1.612919in}{2.543142in}}%
\pgfusepath{stroke}%
\end{pgfscope}%
\begin{pgfscope}%
\pgfpathrectangle{\pgfqpoint{0.100000in}{0.212622in}}{\pgfqpoint{3.696000in}{3.696000in}}%
\pgfusepath{clip}%
\pgfsetrectcap%
\pgfsetroundjoin%
\pgfsetlinewidth{1.505625pt}%
\definecolor{currentstroke}{rgb}{1.000000,0.000000,0.000000}%
\pgfsetstrokecolor{currentstroke}%
\pgfsetdash{}{0pt}%
\pgfpathmoveto{\pgfqpoint{1.491495in}{2.568607in}}%
\pgfpathlineto{\pgfqpoint{1.612919in}{2.543142in}}%
\pgfusepath{stroke}%
\end{pgfscope}%
\begin{pgfscope}%
\pgfpathrectangle{\pgfqpoint{0.100000in}{0.212622in}}{\pgfqpoint{3.696000in}{3.696000in}}%
\pgfusepath{clip}%
\pgfsetrectcap%
\pgfsetroundjoin%
\pgfsetlinewidth{1.505625pt}%
\definecolor{currentstroke}{rgb}{1.000000,0.000000,0.000000}%
\pgfsetstrokecolor{currentstroke}%
\pgfsetdash{}{0pt}%
\pgfpathmoveto{\pgfqpoint{1.489479in}{2.565506in}}%
\pgfpathlineto{\pgfqpoint{1.606804in}{2.538188in}}%
\pgfusepath{stroke}%
\end{pgfscope}%
\begin{pgfscope}%
\pgfpathrectangle{\pgfqpoint{0.100000in}{0.212622in}}{\pgfqpoint{3.696000in}{3.696000in}}%
\pgfusepath{clip}%
\pgfsetrectcap%
\pgfsetroundjoin%
\pgfsetlinewidth{1.505625pt}%
\definecolor{currentstroke}{rgb}{1.000000,0.000000,0.000000}%
\pgfsetstrokecolor{currentstroke}%
\pgfsetdash{}{0pt}%
\pgfpathmoveto{\pgfqpoint{1.486872in}{2.562203in}}%
\pgfpathlineto{\pgfqpoint{1.606804in}{2.538188in}}%
\pgfusepath{stroke}%
\end{pgfscope}%
\begin{pgfscope}%
\pgfpathrectangle{\pgfqpoint{0.100000in}{0.212622in}}{\pgfqpoint{3.696000in}{3.696000in}}%
\pgfusepath{clip}%
\pgfsetrectcap%
\pgfsetroundjoin%
\pgfsetlinewidth{1.505625pt}%
\definecolor{currentstroke}{rgb}{1.000000,0.000000,0.000000}%
\pgfsetstrokecolor{currentstroke}%
\pgfsetdash{}{0pt}%
\pgfpathmoveto{\pgfqpoint{1.483576in}{2.556120in}}%
\pgfpathlineto{\pgfqpoint{1.606804in}{2.538188in}}%
\pgfusepath{stroke}%
\end{pgfscope}%
\begin{pgfscope}%
\pgfpathrectangle{\pgfqpoint{0.100000in}{0.212622in}}{\pgfqpoint{3.696000in}{3.696000in}}%
\pgfusepath{clip}%
\pgfsetrectcap%
\pgfsetroundjoin%
\pgfsetlinewidth{1.505625pt}%
\definecolor{currentstroke}{rgb}{1.000000,0.000000,0.000000}%
\pgfsetstrokecolor{currentstroke}%
\pgfsetdash{}{0pt}%
\pgfpathmoveto{\pgfqpoint{1.480933in}{2.547511in}}%
\pgfpathlineto{\pgfqpoint{1.600684in}{2.533230in}}%
\pgfusepath{stroke}%
\end{pgfscope}%
\begin{pgfscope}%
\pgfpathrectangle{\pgfqpoint{0.100000in}{0.212622in}}{\pgfqpoint{3.696000in}{3.696000in}}%
\pgfusepath{clip}%
\pgfsetrectcap%
\pgfsetroundjoin%
\pgfsetlinewidth{1.505625pt}%
\definecolor{currentstroke}{rgb}{1.000000,0.000000,0.000000}%
\pgfsetstrokecolor{currentstroke}%
\pgfsetdash{}{0pt}%
\pgfpathmoveto{\pgfqpoint{1.479061in}{2.542953in}}%
\pgfpathlineto{\pgfqpoint{1.600684in}{2.533230in}}%
\pgfusepath{stroke}%
\end{pgfscope}%
\begin{pgfscope}%
\pgfpathrectangle{\pgfqpoint{0.100000in}{0.212622in}}{\pgfqpoint{3.696000in}{3.696000in}}%
\pgfusepath{clip}%
\pgfsetrectcap%
\pgfsetroundjoin%
\pgfsetlinewidth{1.505625pt}%
\definecolor{currentstroke}{rgb}{1.000000,0.000000,0.000000}%
\pgfsetstrokecolor{currentstroke}%
\pgfsetdash{}{0pt}%
\pgfpathmoveto{\pgfqpoint{1.475702in}{2.538249in}}%
\pgfpathlineto{\pgfqpoint{1.594558in}{2.528267in}}%
\pgfusepath{stroke}%
\end{pgfscope}%
\begin{pgfscope}%
\pgfpathrectangle{\pgfqpoint{0.100000in}{0.212622in}}{\pgfqpoint{3.696000in}{3.696000in}}%
\pgfusepath{clip}%
\pgfsetrectcap%
\pgfsetroundjoin%
\pgfsetlinewidth{1.505625pt}%
\definecolor{currentstroke}{rgb}{1.000000,0.000000,0.000000}%
\pgfsetstrokecolor{currentstroke}%
\pgfsetdash{}{0pt}%
\pgfpathmoveto{\pgfqpoint{1.471924in}{2.533430in}}%
\pgfpathlineto{\pgfqpoint{1.594558in}{2.528267in}}%
\pgfusepath{stroke}%
\end{pgfscope}%
\begin{pgfscope}%
\pgfpathrectangle{\pgfqpoint{0.100000in}{0.212622in}}{\pgfqpoint{3.696000in}{3.696000in}}%
\pgfusepath{clip}%
\pgfsetrectcap%
\pgfsetroundjoin%
\pgfsetlinewidth{1.505625pt}%
\definecolor{currentstroke}{rgb}{1.000000,0.000000,0.000000}%
\pgfsetstrokecolor{currentstroke}%
\pgfsetdash{}{0pt}%
\pgfpathmoveto{\pgfqpoint{1.467892in}{2.526235in}}%
\pgfpathlineto{\pgfqpoint{1.588427in}{2.523300in}}%
\pgfusepath{stroke}%
\end{pgfscope}%
\begin{pgfscope}%
\pgfpathrectangle{\pgfqpoint{0.100000in}{0.212622in}}{\pgfqpoint{3.696000in}{3.696000in}}%
\pgfusepath{clip}%
\pgfsetrectcap%
\pgfsetroundjoin%
\pgfsetlinewidth{1.505625pt}%
\definecolor{currentstroke}{rgb}{1.000000,0.000000,0.000000}%
\pgfsetstrokecolor{currentstroke}%
\pgfsetdash{}{0pt}%
\pgfpathmoveto{\pgfqpoint{1.465261in}{2.515357in}}%
\pgfpathlineto{\pgfqpoint{1.582290in}{2.518329in}}%
\pgfusepath{stroke}%
\end{pgfscope}%
\begin{pgfscope}%
\pgfpathrectangle{\pgfqpoint{0.100000in}{0.212622in}}{\pgfqpoint{3.696000in}{3.696000in}}%
\pgfusepath{clip}%
\pgfsetrectcap%
\pgfsetroundjoin%
\pgfsetlinewidth{1.505625pt}%
\definecolor{currentstroke}{rgb}{1.000000,0.000000,0.000000}%
\pgfsetstrokecolor{currentstroke}%
\pgfsetdash{}{0pt}%
\pgfpathmoveto{\pgfqpoint{1.462694in}{2.510195in}}%
\pgfpathlineto{\pgfqpoint{1.582290in}{2.518329in}}%
\pgfusepath{stroke}%
\end{pgfscope}%
\begin{pgfscope}%
\pgfpathrectangle{\pgfqpoint{0.100000in}{0.212622in}}{\pgfqpoint{3.696000in}{3.696000in}}%
\pgfusepath{clip}%
\pgfsetrectcap%
\pgfsetroundjoin%
\pgfsetlinewidth{1.505625pt}%
\definecolor{currentstroke}{rgb}{1.000000,0.000000,0.000000}%
\pgfsetstrokecolor{currentstroke}%
\pgfsetdash{}{0pt}%
\pgfpathmoveto{\pgfqpoint{1.459250in}{2.505288in}}%
\pgfpathlineto{\pgfqpoint{1.582290in}{2.518329in}}%
\pgfusepath{stroke}%
\end{pgfscope}%
\begin{pgfscope}%
\pgfpathrectangle{\pgfqpoint{0.100000in}{0.212622in}}{\pgfqpoint{3.696000in}{3.696000in}}%
\pgfusepath{clip}%
\pgfsetrectcap%
\pgfsetroundjoin%
\pgfsetlinewidth{1.505625pt}%
\definecolor{currentstroke}{rgb}{1.000000,0.000000,0.000000}%
\pgfsetstrokecolor{currentstroke}%
\pgfsetdash{}{0pt}%
\pgfpathmoveto{\pgfqpoint{1.455808in}{2.499987in}}%
\pgfpathlineto{\pgfqpoint{1.576148in}{2.513353in}}%
\pgfusepath{stroke}%
\end{pgfscope}%
\begin{pgfscope}%
\pgfpathrectangle{\pgfqpoint{0.100000in}{0.212622in}}{\pgfqpoint{3.696000in}{3.696000in}}%
\pgfusepath{clip}%
\pgfsetrectcap%
\pgfsetroundjoin%
\pgfsetlinewidth{1.505625pt}%
\definecolor{currentstroke}{rgb}{1.000000,0.000000,0.000000}%
\pgfsetstrokecolor{currentstroke}%
\pgfsetdash{}{0pt}%
\pgfpathmoveto{\pgfqpoint{1.452611in}{2.491374in}}%
\pgfpathlineto{\pgfqpoint{1.576148in}{2.513353in}}%
\pgfusepath{stroke}%
\end{pgfscope}%
\begin{pgfscope}%
\pgfpathrectangle{\pgfqpoint{0.100000in}{0.212622in}}{\pgfqpoint{3.696000in}{3.696000in}}%
\pgfusepath{clip}%
\pgfsetrectcap%
\pgfsetroundjoin%
\pgfsetlinewidth{1.505625pt}%
\definecolor{currentstroke}{rgb}{1.000000,0.000000,0.000000}%
\pgfsetstrokecolor{currentstroke}%
\pgfsetdash{}{0pt}%
\pgfpathmoveto{\pgfqpoint{1.450853in}{2.486394in}}%
\pgfpathlineto{\pgfqpoint{1.570001in}{2.508373in}}%
\pgfusepath{stroke}%
\end{pgfscope}%
\begin{pgfscope}%
\pgfpathrectangle{\pgfqpoint{0.100000in}{0.212622in}}{\pgfqpoint{3.696000in}{3.696000in}}%
\pgfusepath{clip}%
\pgfsetrectcap%
\pgfsetroundjoin%
\pgfsetlinewidth{1.505625pt}%
\definecolor{currentstroke}{rgb}{1.000000,0.000000,0.000000}%
\pgfsetstrokecolor{currentstroke}%
\pgfsetdash{}{0pt}%
\pgfpathmoveto{\pgfqpoint{1.447317in}{2.480721in}}%
\pgfpathlineto{\pgfqpoint{1.570001in}{2.508373in}}%
\pgfusepath{stroke}%
\end{pgfscope}%
\begin{pgfscope}%
\pgfpathrectangle{\pgfqpoint{0.100000in}{0.212622in}}{\pgfqpoint{3.696000in}{3.696000in}}%
\pgfusepath{clip}%
\pgfsetrectcap%
\pgfsetroundjoin%
\pgfsetlinewidth{1.505625pt}%
\definecolor{currentstroke}{rgb}{1.000000,0.000000,0.000000}%
\pgfsetstrokecolor{currentstroke}%
\pgfsetdash{}{0pt}%
\pgfpathmoveto{\pgfqpoint{1.443206in}{2.475252in}}%
\pgfpathlineto{\pgfqpoint{1.563848in}{2.503388in}}%
\pgfusepath{stroke}%
\end{pgfscope}%
\begin{pgfscope}%
\pgfpathrectangle{\pgfqpoint{0.100000in}{0.212622in}}{\pgfqpoint{3.696000in}{3.696000in}}%
\pgfusepath{clip}%
\pgfsetrectcap%
\pgfsetroundjoin%
\pgfsetlinewidth{1.505625pt}%
\definecolor{currentstroke}{rgb}{1.000000,0.000000,0.000000}%
\pgfsetstrokecolor{currentstroke}%
\pgfsetdash{}{0pt}%
\pgfpathmoveto{\pgfqpoint{1.441101in}{2.472105in}}%
\pgfpathlineto{\pgfqpoint{1.563848in}{2.503388in}}%
\pgfusepath{stroke}%
\end{pgfscope}%
\begin{pgfscope}%
\pgfpathrectangle{\pgfqpoint{0.100000in}{0.212622in}}{\pgfqpoint{3.696000in}{3.696000in}}%
\pgfusepath{clip}%
\pgfsetrectcap%
\pgfsetroundjoin%
\pgfsetlinewidth{1.505625pt}%
\definecolor{currentstroke}{rgb}{1.000000,0.000000,0.000000}%
\pgfsetstrokecolor{currentstroke}%
\pgfsetdash{}{0pt}%
\pgfpathmoveto{\pgfqpoint{1.438924in}{2.466888in}}%
\pgfpathlineto{\pgfqpoint{1.563848in}{2.503388in}}%
\pgfusepath{stroke}%
\end{pgfscope}%
\begin{pgfscope}%
\pgfpathrectangle{\pgfqpoint{0.100000in}{0.212622in}}{\pgfqpoint{3.696000in}{3.696000in}}%
\pgfusepath{clip}%
\pgfsetrectcap%
\pgfsetroundjoin%
\pgfsetlinewidth{1.505625pt}%
\definecolor{currentstroke}{rgb}{1.000000,0.000000,0.000000}%
\pgfsetstrokecolor{currentstroke}%
\pgfsetdash{}{0pt}%
\pgfpathmoveto{\pgfqpoint{1.437814in}{2.463787in}}%
\pgfpathlineto{\pgfqpoint{1.557689in}{2.498399in}}%
\pgfusepath{stroke}%
\end{pgfscope}%
\begin{pgfscope}%
\pgfpathrectangle{\pgfqpoint{0.100000in}{0.212622in}}{\pgfqpoint{3.696000in}{3.696000in}}%
\pgfusepath{clip}%
\pgfsetrectcap%
\pgfsetroundjoin%
\pgfsetlinewidth{1.505625pt}%
\definecolor{currentstroke}{rgb}{1.000000,0.000000,0.000000}%
\pgfsetstrokecolor{currentstroke}%
\pgfsetdash{}{0pt}%
\pgfpathmoveto{\pgfqpoint{1.435552in}{2.459815in}}%
\pgfpathlineto{\pgfqpoint{1.557689in}{2.498399in}}%
\pgfusepath{stroke}%
\end{pgfscope}%
\begin{pgfscope}%
\pgfpathrectangle{\pgfqpoint{0.100000in}{0.212622in}}{\pgfqpoint{3.696000in}{3.696000in}}%
\pgfusepath{clip}%
\pgfsetrectcap%
\pgfsetroundjoin%
\pgfsetlinewidth{1.505625pt}%
\definecolor{currentstroke}{rgb}{1.000000,0.000000,0.000000}%
\pgfsetstrokecolor{currentstroke}%
\pgfsetdash{}{0pt}%
\pgfpathmoveto{\pgfqpoint{1.432765in}{2.455931in}}%
\pgfpathlineto{\pgfqpoint{1.557689in}{2.498399in}}%
\pgfusepath{stroke}%
\end{pgfscope}%
\begin{pgfscope}%
\pgfpathrectangle{\pgfqpoint{0.100000in}{0.212622in}}{\pgfqpoint{3.696000in}{3.696000in}}%
\pgfusepath{clip}%
\pgfsetrectcap%
\pgfsetroundjoin%
\pgfsetlinewidth{1.505625pt}%
\definecolor{currentstroke}{rgb}{1.000000,0.000000,0.000000}%
\pgfsetstrokecolor{currentstroke}%
\pgfsetdash{}{0pt}%
\pgfpathmoveto{\pgfqpoint{1.431345in}{2.453990in}}%
\pgfpathlineto{\pgfqpoint{1.557689in}{2.498399in}}%
\pgfusepath{stroke}%
\end{pgfscope}%
\begin{pgfscope}%
\pgfpathrectangle{\pgfqpoint{0.100000in}{0.212622in}}{\pgfqpoint{3.696000in}{3.696000in}}%
\pgfusepath{clip}%
\pgfsetrectcap%
\pgfsetroundjoin%
\pgfsetlinewidth{1.505625pt}%
\definecolor{currentstroke}{rgb}{1.000000,0.000000,0.000000}%
\pgfsetstrokecolor{currentstroke}%
\pgfsetdash{}{0pt}%
\pgfpathmoveto{\pgfqpoint{1.429555in}{2.449509in}}%
\pgfpathlineto{\pgfqpoint{1.551525in}{2.493405in}}%
\pgfusepath{stroke}%
\end{pgfscope}%
\begin{pgfscope}%
\pgfpathrectangle{\pgfqpoint{0.100000in}{0.212622in}}{\pgfqpoint{3.696000in}{3.696000in}}%
\pgfusepath{clip}%
\pgfsetrectcap%
\pgfsetroundjoin%
\pgfsetlinewidth{1.505625pt}%
\definecolor{currentstroke}{rgb}{1.000000,0.000000,0.000000}%
\pgfsetstrokecolor{currentstroke}%
\pgfsetdash{}{0pt}%
\pgfpathmoveto{\pgfqpoint{1.428627in}{2.446930in}}%
\pgfpathlineto{\pgfqpoint{1.551525in}{2.493405in}}%
\pgfusepath{stroke}%
\end{pgfscope}%
\begin{pgfscope}%
\pgfpathrectangle{\pgfqpoint{0.100000in}{0.212622in}}{\pgfqpoint{3.696000in}{3.696000in}}%
\pgfusepath{clip}%
\pgfsetrectcap%
\pgfsetroundjoin%
\pgfsetlinewidth{1.505625pt}%
\definecolor{currentstroke}{rgb}{1.000000,0.000000,0.000000}%
\pgfsetstrokecolor{currentstroke}%
\pgfsetdash{}{0pt}%
\pgfpathmoveto{\pgfqpoint{1.426011in}{2.442843in}}%
\pgfpathlineto{\pgfqpoint{1.551525in}{2.493405in}}%
\pgfusepath{stroke}%
\end{pgfscope}%
\begin{pgfscope}%
\pgfpathrectangle{\pgfqpoint{0.100000in}{0.212622in}}{\pgfqpoint{3.696000in}{3.696000in}}%
\pgfusepath{clip}%
\pgfsetrectcap%
\pgfsetroundjoin%
\pgfsetlinewidth{1.505625pt}%
\definecolor{currentstroke}{rgb}{1.000000,0.000000,0.000000}%
\pgfsetstrokecolor{currentstroke}%
\pgfsetdash{}{0pt}%
\pgfpathmoveto{\pgfqpoint{1.424543in}{2.440758in}}%
\pgfpathlineto{\pgfqpoint{1.551525in}{2.493405in}}%
\pgfusepath{stroke}%
\end{pgfscope}%
\begin{pgfscope}%
\pgfpathrectangle{\pgfqpoint{0.100000in}{0.212622in}}{\pgfqpoint{3.696000in}{3.696000in}}%
\pgfusepath{clip}%
\pgfsetrectcap%
\pgfsetroundjoin%
\pgfsetlinewidth{1.505625pt}%
\definecolor{currentstroke}{rgb}{1.000000,0.000000,0.000000}%
\pgfsetstrokecolor{currentstroke}%
\pgfsetdash{}{0pt}%
\pgfpathmoveto{\pgfqpoint{1.422449in}{2.437559in}}%
\pgfpathlineto{\pgfqpoint{1.545356in}{2.488407in}}%
\pgfusepath{stroke}%
\end{pgfscope}%
\begin{pgfscope}%
\pgfpathrectangle{\pgfqpoint{0.100000in}{0.212622in}}{\pgfqpoint{3.696000in}{3.696000in}}%
\pgfusepath{clip}%
\pgfsetrectcap%
\pgfsetroundjoin%
\pgfsetlinewidth{1.505625pt}%
\definecolor{currentstroke}{rgb}{1.000000,0.000000,0.000000}%
\pgfsetstrokecolor{currentstroke}%
\pgfsetdash{}{0pt}%
\pgfpathmoveto{\pgfqpoint{1.420575in}{2.431954in}}%
\pgfpathlineto{\pgfqpoint{1.545356in}{2.488407in}}%
\pgfusepath{stroke}%
\end{pgfscope}%
\begin{pgfscope}%
\pgfpathrectangle{\pgfqpoint{0.100000in}{0.212622in}}{\pgfqpoint{3.696000in}{3.696000in}}%
\pgfusepath{clip}%
\pgfsetrectcap%
\pgfsetroundjoin%
\pgfsetlinewidth{1.505625pt}%
\definecolor{currentstroke}{rgb}{1.000000,0.000000,0.000000}%
\pgfsetstrokecolor{currentstroke}%
\pgfsetdash{}{0pt}%
\pgfpathmoveto{\pgfqpoint{1.419249in}{2.429205in}}%
\pgfpathlineto{\pgfqpoint{1.545356in}{2.488407in}}%
\pgfusepath{stroke}%
\end{pgfscope}%
\begin{pgfscope}%
\pgfpathrectangle{\pgfqpoint{0.100000in}{0.212622in}}{\pgfqpoint{3.696000in}{3.696000in}}%
\pgfusepath{clip}%
\pgfsetrectcap%
\pgfsetroundjoin%
\pgfsetlinewidth{1.505625pt}%
\definecolor{currentstroke}{rgb}{1.000000,0.000000,0.000000}%
\pgfsetstrokecolor{currentstroke}%
\pgfsetdash{}{0pt}%
\pgfpathmoveto{\pgfqpoint{1.417067in}{2.426058in}}%
\pgfpathlineto{\pgfqpoint{1.545356in}{2.488407in}}%
\pgfusepath{stroke}%
\end{pgfscope}%
\begin{pgfscope}%
\pgfpathrectangle{\pgfqpoint{0.100000in}{0.212622in}}{\pgfqpoint{3.696000in}{3.696000in}}%
\pgfusepath{clip}%
\pgfsetrectcap%
\pgfsetroundjoin%
\pgfsetlinewidth{1.505625pt}%
\definecolor{currentstroke}{rgb}{1.000000,0.000000,0.000000}%
\pgfsetstrokecolor{currentstroke}%
\pgfsetdash{}{0pt}%
\pgfpathmoveto{\pgfqpoint{1.414426in}{2.422651in}}%
\pgfpathlineto{\pgfqpoint{1.539180in}{2.483404in}}%
\pgfusepath{stroke}%
\end{pgfscope}%
\begin{pgfscope}%
\pgfpathrectangle{\pgfqpoint{0.100000in}{0.212622in}}{\pgfqpoint{3.696000in}{3.696000in}}%
\pgfusepath{clip}%
\pgfsetrectcap%
\pgfsetroundjoin%
\pgfsetlinewidth{1.505625pt}%
\definecolor{currentstroke}{rgb}{1.000000,0.000000,0.000000}%
\pgfsetstrokecolor{currentstroke}%
\pgfsetdash{}{0pt}%
\pgfpathmoveto{\pgfqpoint{1.411292in}{2.416930in}}%
\pgfpathlineto{\pgfqpoint{1.539180in}{2.483404in}}%
\pgfusepath{stroke}%
\end{pgfscope}%
\begin{pgfscope}%
\pgfpathrectangle{\pgfqpoint{0.100000in}{0.212622in}}{\pgfqpoint{3.696000in}{3.696000in}}%
\pgfusepath{clip}%
\pgfsetrectcap%
\pgfsetroundjoin%
\pgfsetlinewidth{1.505625pt}%
\definecolor{currentstroke}{rgb}{1.000000,0.000000,0.000000}%
\pgfsetstrokecolor{currentstroke}%
\pgfsetdash{}{0pt}%
\pgfpathmoveto{\pgfqpoint{1.408810in}{2.408504in}}%
\pgfpathlineto{\pgfqpoint{1.533000in}{2.478397in}}%
\pgfusepath{stroke}%
\end{pgfscope}%
\begin{pgfscope}%
\pgfpathrectangle{\pgfqpoint{0.100000in}{0.212622in}}{\pgfqpoint{3.696000in}{3.696000in}}%
\pgfusepath{clip}%
\pgfsetrectcap%
\pgfsetroundjoin%
\pgfsetlinewidth{1.505625pt}%
\definecolor{currentstroke}{rgb}{1.000000,0.000000,0.000000}%
\pgfsetstrokecolor{currentstroke}%
\pgfsetdash{}{0pt}%
\pgfpathmoveto{\pgfqpoint{1.406963in}{2.404026in}}%
\pgfpathlineto{\pgfqpoint{1.533000in}{2.478397in}}%
\pgfusepath{stroke}%
\end{pgfscope}%
\begin{pgfscope}%
\pgfpathrectangle{\pgfqpoint{0.100000in}{0.212622in}}{\pgfqpoint{3.696000in}{3.696000in}}%
\pgfusepath{clip}%
\pgfsetrectcap%
\pgfsetroundjoin%
\pgfsetlinewidth{1.505625pt}%
\definecolor{currentstroke}{rgb}{1.000000,0.000000,0.000000}%
\pgfsetstrokecolor{currentstroke}%
\pgfsetdash{}{0pt}%
\pgfpathmoveto{\pgfqpoint{1.405612in}{2.402017in}}%
\pgfpathlineto{\pgfqpoint{1.533000in}{2.478397in}}%
\pgfusepath{stroke}%
\end{pgfscope}%
\begin{pgfscope}%
\pgfpathrectangle{\pgfqpoint{0.100000in}{0.212622in}}{\pgfqpoint{3.696000in}{3.696000in}}%
\pgfusepath{clip}%
\pgfsetrectcap%
\pgfsetroundjoin%
\pgfsetlinewidth{1.505625pt}%
\definecolor{currentstroke}{rgb}{1.000000,0.000000,0.000000}%
\pgfsetstrokecolor{currentstroke}%
\pgfsetdash{}{0pt}%
\pgfpathmoveto{\pgfqpoint{1.403836in}{2.399599in}}%
\pgfpathlineto{\pgfqpoint{1.526814in}{2.473386in}}%
\pgfusepath{stroke}%
\end{pgfscope}%
\begin{pgfscope}%
\pgfpathrectangle{\pgfqpoint{0.100000in}{0.212622in}}{\pgfqpoint{3.696000in}{3.696000in}}%
\pgfusepath{clip}%
\pgfsetrectcap%
\pgfsetroundjoin%
\pgfsetlinewidth{1.505625pt}%
\definecolor{currentstroke}{rgb}{1.000000,0.000000,0.000000}%
\pgfsetstrokecolor{currentstroke}%
\pgfsetdash{}{0pt}%
\pgfpathmoveto{\pgfqpoint{1.401576in}{2.395674in}}%
\pgfpathlineto{\pgfqpoint{1.526814in}{2.473386in}}%
\pgfusepath{stroke}%
\end{pgfscope}%
\begin{pgfscope}%
\pgfpathrectangle{\pgfqpoint{0.100000in}{0.212622in}}{\pgfqpoint{3.696000in}{3.696000in}}%
\pgfusepath{clip}%
\pgfsetrectcap%
\pgfsetroundjoin%
\pgfsetlinewidth{1.505625pt}%
\definecolor{currentstroke}{rgb}{1.000000,0.000000,0.000000}%
\pgfsetstrokecolor{currentstroke}%
\pgfsetdash{}{0pt}%
\pgfpathmoveto{\pgfqpoint{1.399568in}{2.389598in}}%
\pgfpathlineto{\pgfqpoint{1.526814in}{2.473386in}}%
\pgfusepath{stroke}%
\end{pgfscope}%
\begin{pgfscope}%
\pgfpathrectangle{\pgfqpoint{0.100000in}{0.212622in}}{\pgfqpoint{3.696000in}{3.696000in}}%
\pgfusepath{clip}%
\pgfsetrectcap%
\pgfsetroundjoin%
\pgfsetlinewidth{1.505625pt}%
\definecolor{currentstroke}{rgb}{1.000000,0.000000,0.000000}%
\pgfsetstrokecolor{currentstroke}%
\pgfsetdash{}{0pt}%
\pgfpathmoveto{\pgfqpoint{1.398412in}{2.386343in}}%
\pgfpathlineto{\pgfqpoint{1.526814in}{2.473386in}}%
\pgfusepath{stroke}%
\end{pgfscope}%
\begin{pgfscope}%
\pgfpathrectangle{\pgfqpoint{0.100000in}{0.212622in}}{\pgfqpoint{3.696000in}{3.696000in}}%
\pgfusepath{clip}%
\pgfsetrectcap%
\pgfsetroundjoin%
\pgfsetlinewidth{1.505625pt}%
\definecolor{currentstroke}{rgb}{1.000000,0.000000,0.000000}%
\pgfsetstrokecolor{currentstroke}%
\pgfsetdash{}{0pt}%
\pgfpathmoveto{\pgfqpoint{1.396497in}{2.382865in}}%
\pgfpathlineto{\pgfqpoint{1.520622in}{2.468370in}}%
\pgfusepath{stroke}%
\end{pgfscope}%
\begin{pgfscope}%
\pgfpathrectangle{\pgfqpoint{0.100000in}{0.212622in}}{\pgfqpoint{3.696000in}{3.696000in}}%
\pgfusepath{clip}%
\pgfsetrectcap%
\pgfsetroundjoin%
\pgfsetlinewidth{1.505625pt}%
\definecolor{currentstroke}{rgb}{1.000000,0.000000,0.000000}%
\pgfsetstrokecolor{currentstroke}%
\pgfsetdash{}{0pt}%
\pgfpathmoveto{\pgfqpoint{1.395323in}{2.381142in}}%
\pgfpathlineto{\pgfqpoint{1.520622in}{2.468370in}}%
\pgfusepath{stroke}%
\end{pgfscope}%
\begin{pgfscope}%
\pgfpathrectangle{\pgfqpoint{0.100000in}{0.212622in}}{\pgfqpoint{3.696000in}{3.696000in}}%
\pgfusepath{clip}%
\pgfsetrectcap%
\pgfsetroundjoin%
\pgfsetlinewidth{1.505625pt}%
\definecolor{currentstroke}{rgb}{1.000000,0.000000,0.000000}%
\pgfsetstrokecolor{currentstroke}%
\pgfsetdash{}{0pt}%
\pgfpathmoveto{\pgfqpoint{1.394670in}{2.380259in}}%
\pgfpathlineto{\pgfqpoint{1.520622in}{2.468370in}}%
\pgfusepath{stroke}%
\end{pgfscope}%
\begin{pgfscope}%
\pgfpathrectangle{\pgfqpoint{0.100000in}{0.212622in}}{\pgfqpoint{3.696000in}{3.696000in}}%
\pgfusepath{clip}%
\pgfsetrectcap%
\pgfsetroundjoin%
\pgfsetlinewidth{1.505625pt}%
\definecolor{currentstroke}{rgb}{1.000000,0.000000,0.000000}%
\pgfsetstrokecolor{currentstroke}%
\pgfsetdash{}{0pt}%
\pgfpathmoveto{\pgfqpoint{1.393544in}{2.377641in}}%
\pgfpathlineto{\pgfqpoint{1.520622in}{2.468370in}}%
\pgfusepath{stroke}%
\end{pgfscope}%
\begin{pgfscope}%
\pgfpathrectangle{\pgfqpoint{0.100000in}{0.212622in}}{\pgfqpoint{3.696000in}{3.696000in}}%
\pgfusepath{clip}%
\pgfsetrectcap%
\pgfsetroundjoin%
\pgfsetlinewidth{1.505625pt}%
\definecolor{currentstroke}{rgb}{1.000000,0.000000,0.000000}%
\pgfsetstrokecolor{currentstroke}%
\pgfsetdash{}{0pt}%
\pgfpathmoveto{\pgfqpoint{1.392768in}{2.374269in}}%
\pgfpathlineto{\pgfqpoint{1.520622in}{2.468370in}}%
\pgfusepath{stroke}%
\end{pgfscope}%
\begin{pgfscope}%
\pgfpathrectangle{\pgfqpoint{0.100000in}{0.212622in}}{\pgfqpoint{3.696000in}{3.696000in}}%
\pgfusepath{clip}%
\pgfsetrectcap%
\pgfsetroundjoin%
\pgfsetlinewidth{1.505625pt}%
\definecolor{currentstroke}{rgb}{1.000000,0.000000,0.000000}%
\pgfsetstrokecolor{currentstroke}%
\pgfsetdash{}{0pt}%
\pgfpathmoveto{\pgfqpoint{1.390896in}{2.370524in}}%
\pgfpathlineto{\pgfqpoint{1.514424in}{2.463349in}}%
\pgfusepath{stroke}%
\end{pgfscope}%
\begin{pgfscope}%
\pgfpathrectangle{\pgfqpoint{0.100000in}{0.212622in}}{\pgfqpoint{3.696000in}{3.696000in}}%
\pgfusepath{clip}%
\pgfsetrectcap%
\pgfsetroundjoin%
\pgfsetlinewidth{1.505625pt}%
\definecolor{currentstroke}{rgb}{1.000000,0.000000,0.000000}%
\pgfsetstrokecolor{currentstroke}%
\pgfsetdash{}{0pt}%
\pgfpathmoveto{\pgfqpoint{1.388051in}{2.366417in}}%
\pgfpathlineto{\pgfqpoint{1.514424in}{2.463349in}}%
\pgfusepath{stroke}%
\end{pgfscope}%
\begin{pgfscope}%
\pgfpathrectangle{\pgfqpoint{0.100000in}{0.212622in}}{\pgfqpoint{3.696000in}{3.696000in}}%
\pgfusepath{clip}%
\pgfsetrectcap%
\pgfsetroundjoin%
\pgfsetlinewidth{1.505625pt}%
\definecolor{currentstroke}{rgb}{1.000000,0.000000,0.000000}%
\pgfsetstrokecolor{currentstroke}%
\pgfsetdash{}{0pt}%
\pgfpathmoveto{\pgfqpoint{1.384813in}{2.362084in}}%
\pgfpathlineto{\pgfqpoint{1.514424in}{2.463349in}}%
\pgfusepath{stroke}%
\end{pgfscope}%
\begin{pgfscope}%
\pgfpathrectangle{\pgfqpoint{0.100000in}{0.212622in}}{\pgfqpoint{3.696000in}{3.696000in}}%
\pgfusepath{clip}%
\pgfsetrectcap%
\pgfsetroundjoin%
\pgfsetlinewidth{1.505625pt}%
\definecolor{currentstroke}{rgb}{1.000000,0.000000,0.000000}%
\pgfsetstrokecolor{currentstroke}%
\pgfsetdash{}{0pt}%
\pgfpathmoveto{\pgfqpoint{1.381266in}{2.354609in}}%
\pgfpathlineto{\pgfqpoint{1.508221in}{2.458324in}}%
\pgfusepath{stroke}%
\end{pgfscope}%
\begin{pgfscope}%
\pgfpathrectangle{\pgfqpoint{0.100000in}{0.212622in}}{\pgfqpoint{3.696000in}{3.696000in}}%
\pgfusepath{clip}%
\pgfsetrectcap%
\pgfsetroundjoin%
\pgfsetlinewidth{1.505625pt}%
\definecolor{currentstroke}{rgb}{1.000000,0.000000,0.000000}%
\pgfsetstrokecolor{currentstroke}%
\pgfsetdash{}{0pt}%
\pgfpathmoveto{\pgfqpoint{1.378401in}{2.344876in}}%
\pgfpathlineto{\pgfqpoint{1.502013in}{2.453294in}}%
\pgfusepath{stroke}%
\end{pgfscope}%
\begin{pgfscope}%
\pgfpathrectangle{\pgfqpoint{0.100000in}{0.212622in}}{\pgfqpoint{3.696000in}{3.696000in}}%
\pgfusepath{clip}%
\pgfsetrectcap%
\pgfsetroundjoin%
\pgfsetlinewidth{1.505625pt}%
\definecolor{currentstroke}{rgb}{1.000000,0.000000,0.000000}%
\pgfsetstrokecolor{currentstroke}%
\pgfsetdash{}{0pt}%
\pgfpathmoveto{\pgfqpoint{1.376375in}{2.339818in}}%
\pgfpathlineto{\pgfqpoint{1.502013in}{2.453294in}}%
\pgfusepath{stroke}%
\end{pgfscope}%
\begin{pgfscope}%
\pgfpathrectangle{\pgfqpoint{0.100000in}{0.212622in}}{\pgfqpoint{3.696000in}{3.696000in}}%
\pgfusepath{clip}%
\pgfsetrectcap%
\pgfsetroundjoin%
\pgfsetlinewidth{1.505625pt}%
\definecolor{currentstroke}{rgb}{1.000000,0.000000,0.000000}%
\pgfsetstrokecolor{currentstroke}%
\pgfsetdash{}{0pt}%
\pgfpathmoveto{\pgfqpoint{1.373219in}{2.335216in}}%
\pgfpathlineto{\pgfqpoint{1.502013in}{2.453294in}}%
\pgfusepath{stroke}%
\end{pgfscope}%
\begin{pgfscope}%
\pgfpathrectangle{\pgfqpoint{0.100000in}{0.212622in}}{\pgfqpoint{3.696000in}{3.696000in}}%
\pgfusepath{clip}%
\pgfsetrectcap%
\pgfsetroundjoin%
\pgfsetlinewidth{1.505625pt}%
\definecolor{currentstroke}{rgb}{1.000000,0.000000,0.000000}%
\pgfsetstrokecolor{currentstroke}%
\pgfsetdash{}{0pt}%
\pgfpathmoveto{\pgfqpoint{1.369566in}{2.330705in}}%
\pgfpathlineto{\pgfqpoint{1.495799in}{2.448260in}}%
\pgfusepath{stroke}%
\end{pgfscope}%
\begin{pgfscope}%
\pgfpathrectangle{\pgfqpoint{0.100000in}{0.212622in}}{\pgfqpoint{3.696000in}{3.696000in}}%
\pgfusepath{clip}%
\pgfsetrectcap%
\pgfsetroundjoin%
\pgfsetlinewidth{1.505625pt}%
\definecolor{currentstroke}{rgb}{1.000000,0.000000,0.000000}%
\pgfsetstrokecolor{currentstroke}%
\pgfsetdash{}{0pt}%
\pgfpathmoveto{\pgfqpoint{1.365272in}{2.322841in}}%
\pgfpathlineto{\pgfqpoint{1.495799in}{2.448260in}}%
\pgfusepath{stroke}%
\end{pgfscope}%
\begin{pgfscope}%
\pgfpathrectangle{\pgfqpoint{0.100000in}{0.212622in}}{\pgfqpoint{3.696000in}{3.696000in}}%
\pgfusepath{clip}%
\pgfsetrectcap%
\pgfsetroundjoin%
\pgfsetlinewidth{1.505625pt}%
\definecolor{currentstroke}{rgb}{1.000000,0.000000,0.000000}%
\pgfsetstrokecolor{currentstroke}%
\pgfsetdash{}{0pt}%
\pgfpathmoveto{\pgfqpoint{1.362461in}{2.312369in}}%
\pgfpathlineto{\pgfqpoint{1.489579in}{2.443221in}}%
\pgfusepath{stroke}%
\end{pgfscope}%
\begin{pgfscope}%
\pgfpathrectangle{\pgfqpoint{0.100000in}{0.212622in}}{\pgfqpoint{3.696000in}{3.696000in}}%
\pgfusepath{clip}%
\pgfsetrectcap%
\pgfsetroundjoin%
\pgfsetlinewidth{1.505625pt}%
\definecolor{currentstroke}{rgb}{1.000000,0.000000,0.000000}%
\pgfsetstrokecolor{currentstroke}%
\pgfsetdash{}{0pt}%
\pgfpathmoveto{\pgfqpoint{1.360626in}{2.306378in}}%
\pgfpathlineto{\pgfqpoint{1.489579in}{2.443221in}}%
\pgfusepath{stroke}%
\end{pgfscope}%
\begin{pgfscope}%
\pgfpathrectangle{\pgfqpoint{0.100000in}{0.212622in}}{\pgfqpoint{3.696000in}{3.696000in}}%
\pgfusepath{clip}%
\pgfsetrectcap%
\pgfsetroundjoin%
\pgfsetlinewidth{1.505625pt}%
\definecolor{currentstroke}{rgb}{1.000000,0.000000,0.000000}%
\pgfsetstrokecolor{currentstroke}%
\pgfsetdash{}{0pt}%
\pgfpathmoveto{\pgfqpoint{1.357290in}{2.300713in}}%
\pgfpathlineto{\pgfqpoint{1.483354in}{2.438178in}}%
\pgfusepath{stroke}%
\end{pgfscope}%
\begin{pgfscope}%
\pgfpathrectangle{\pgfqpoint{0.100000in}{0.212622in}}{\pgfqpoint{3.696000in}{3.696000in}}%
\pgfusepath{clip}%
\pgfsetrectcap%
\pgfsetroundjoin%
\pgfsetlinewidth{1.505625pt}%
\definecolor{currentstroke}{rgb}{1.000000,0.000000,0.000000}%
\pgfsetstrokecolor{currentstroke}%
\pgfsetdash{}{0pt}%
\pgfpathmoveto{\pgfqpoint{1.355240in}{2.298116in}}%
\pgfpathlineto{\pgfqpoint{1.483354in}{2.438178in}}%
\pgfusepath{stroke}%
\end{pgfscope}%
\begin{pgfscope}%
\pgfpathrectangle{\pgfqpoint{0.100000in}{0.212622in}}{\pgfqpoint{3.696000in}{3.696000in}}%
\pgfusepath{clip}%
\pgfsetrectcap%
\pgfsetroundjoin%
\pgfsetlinewidth{1.505625pt}%
\definecolor{currentstroke}{rgb}{1.000000,0.000000,0.000000}%
\pgfsetstrokecolor{currentstroke}%
\pgfsetdash{}{0pt}%
\pgfpathmoveto{\pgfqpoint{1.352868in}{2.294419in}}%
\pgfpathlineto{\pgfqpoint{1.483354in}{2.438178in}}%
\pgfusepath{stroke}%
\end{pgfscope}%
\begin{pgfscope}%
\pgfpathrectangle{\pgfqpoint{0.100000in}{0.212622in}}{\pgfqpoint{3.696000in}{3.696000in}}%
\pgfusepath{clip}%
\pgfsetrectcap%
\pgfsetroundjoin%
\pgfsetlinewidth{1.505625pt}%
\definecolor{currentstroke}{rgb}{1.000000,0.000000,0.000000}%
\pgfsetstrokecolor{currentstroke}%
\pgfsetdash{}{0pt}%
\pgfpathmoveto{\pgfqpoint{1.350383in}{2.287690in}}%
\pgfpathlineto{\pgfqpoint{1.477123in}{2.433130in}}%
\pgfusepath{stroke}%
\end{pgfscope}%
\begin{pgfscope}%
\pgfpathrectangle{\pgfqpoint{0.100000in}{0.212622in}}{\pgfqpoint{3.696000in}{3.696000in}}%
\pgfusepath{clip}%
\pgfsetrectcap%
\pgfsetroundjoin%
\pgfsetlinewidth{1.505625pt}%
\definecolor{currentstroke}{rgb}{1.000000,0.000000,0.000000}%
\pgfsetstrokecolor{currentstroke}%
\pgfsetdash{}{0pt}%
\pgfpathmoveto{\pgfqpoint{1.349244in}{2.283989in}}%
\pgfpathlineto{\pgfqpoint{1.477123in}{2.433130in}}%
\pgfusepath{stroke}%
\end{pgfscope}%
\begin{pgfscope}%
\pgfpathrectangle{\pgfqpoint{0.100000in}{0.212622in}}{\pgfqpoint{3.696000in}{3.696000in}}%
\pgfusepath{clip}%
\pgfsetrectcap%
\pgfsetroundjoin%
\pgfsetlinewidth{1.505625pt}%
\definecolor{currentstroke}{rgb}{1.000000,0.000000,0.000000}%
\pgfsetstrokecolor{currentstroke}%
\pgfsetdash{}{0pt}%
\pgfpathmoveto{\pgfqpoint{1.346466in}{2.279700in}}%
\pgfpathlineto{\pgfqpoint{1.470887in}{2.428078in}}%
\pgfusepath{stroke}%
\end{pgfscope}%
\begin{pgfscope}%
\pgfpathrectangle{\pgfqpoint{0.100000in}{0.212622in}}{\pgfqpoint{3.696000in}{3.696000in}}%
\pgfusepath{clip}%
\pgfsetrectcap%
\pgfsetroundjoin%
\pgfsetlinewidth{1.505625pt}%
\definecolor{currentstroke}{rgb}{1.000000,0.000000,0.000000}%
\pgfsetstrokecolor{currentstroke}%
\pgfsetdash{}{0pt}%
\pgfpathmoveto{\pgfqpoint{1.344875in}{2.277441in}}%
\pgfpathlineto{\pgfqpoint{1.470887in}{2.428078in}}%
\pgfusepath{stroke}%
\end{pgfscope}%
\begin{pgfscope}%
\pgfpathrectangle{\pgfqpoint{0.100000in}{0.212622in}}{\pgfqpoint{3.696000in}{3.696000in}}%
\pgfusepath{clip}%
\pgfsetrectcap%
\pgfsetroundjoin%
\pgfsetlinewidth{1.505625pt}%
\definecolor{currentstroke}{rgb}{1.000000,0.000000,0.000000}%
\pgfsetstrokecolor{currentstroke}%
\pgfsetdash{}{0pt}%
\pgfpathmoveto{\pgfqpoint{1.342571in}{2.273036in}}%
\pgfpathlineto{\pgfqpoint{1.470887in}{2.428078in}}%
\pgfusepath{stroke}%
\end{pgfscope}%
\begin{pgfscope}%
\pgfpathrectangle{\pgfqpoint{0.100000in}{0.212622in}}{\pgfqpoint{3.696000in}{3.696000in}}%
\pgfusepath{clip}%
\pgfsetrectcap%
\pgfsetroundjoin%
\pgfsetlinewidth{1.505625pt}%
\definecolor{currentstroke}{rgb}{1.000000,0.000000,0.000000}%
\pgfsetstrokecolor{currentstroke}%
\pgfsetdash{}{0pt}%
\pgfpathmoveto{\pgfqpoint{1.340571in}{2.266236in}}%
\pgfpathlineto{\pgfqpoint{1.464644in}{2.423021in}}%
\pgfusepath{stroke}%
\end{pgfscope}%
\begin{pgfscope}%
\pgfpathrectangle{\pgfqpoint{0.100000in}{0.212622in}}{\pgfqpoint{3.696000in}{3.696000in}}%
\pgfusepath{clip}%
\pgfsetrectcap%
\pgfsetroundjoin%
\pgfsetlinewidth{1.505625pt}%
\definecolor{currentstroke}{rgb}{1.000000,0.000000,0.000000}%
\pgfsetstrokecolor{currentstroke}%
\pgfsetdash{}{0pt}%
\pgfpathmoveto{\pgfqpoint{1.339317in}{2.262769in}}%
\pgfpathlineto{\pgfqpoint{1.464644in}{2.423021in}}%
\pgfusepath{stroke}%
\end{pgfscope}%
\begin{pgfscope}%
\pgfpathrectangle{\pgfqpoint{0.100000in}{0.212622in}}{\pgfqpoint{3.696000in}{3.696000in}}%
\pgfusepath{clip}%
\pgfsetrectcap%
\pgfsetroundjoin%
\pgfsetlinewidth{1.505625pt}%
\definecolor{currentstroke}{rgb}{1.000000,0.000000,0.000000}%
\pgfsetstrokecolor{currentstroke}%
\pgfsetdash{}{0pt}%
\pgfpathmoveto{\pgfqpoint{1.336830in}{2.258489in}}%
\pgfpathlineto{\pgfqpoint{1.464644in}{2.423021in}}%
\pgfusepath{stroke}%
\end{pgfscope}%
\begin{pgfscope}%
\pgfpathrectangle{\pgfqpoint{0.100000in}{0.212622in}}{\pgfqpoint{3.696000in}{3.696000in}}%
\pgfusepath{clip}%
\pgfsetrectcap%
\pgfsetroundjoin%
\pgfsetlinewidth{1.505625pt}%
\definecolor{currentstroke}{rgb}{1.000000,0.000000,0.000000}%
\pgfsetstrokecolor{currentstroke}%
\pgfsetdash{}{0pt}%
\pgfpathmoveto{\pgfqpoint{1.335338in}{2.256382in}}%
\pgfpathlineto{\pgfqpoint{1.464644in}{2.423021in}}%
\pgfusepath{stroke}%
\end{pgfscope}%
\begin{pgfscope}%
\pgfpathrectangle{\pgfqpoint{0.100000in}{0.212622in}}{\pgfqpoint{3.696000in}{3.696000in}}%
\pgfusepath{clip}%
\pgfsetrectcap%
\pgfsetroundjoin%
\pgfsetlinewidth{1.505625pt}%
\definecolor{currentstroke}{rgb}{1.000000,0.000000,0.000000}%
\pgfsetstrokecolor{currentstroke}%
\pgfsetdash{}{0pt}%
\pgfpathmoveto{\pgfqpoint{1.333628in}{2.253964in}}%
\pgfpathlineto{\pgfqpoint{1.458397in}{2.417959in}}%
\pgfusepath{stroke}%
\end{pgfscope}%
\begin{pgfscope}%
\pgfpathrectangle{\pgfqpoint{0.100000in}{0.212622in}}{\pgfqpoint{3.696000in}{3.696000in}}%
\pgfusepath{clip}%
\pgfsetrectcap%
\pgfsetroundjoin%
\pgfsetlinewidth{1.505625pt}%
\definecolor{currentstroke}{rgb}{1.000000,0.000000,0.000000}%
\pgfsetstrokecolor{currentstroke}%
\pgfsetdash{}{0pt}%
\pgfpathmoveto{\pgfqpoint{1.331640in}{2.248631in}}%
\pgfpathlineto{\pgfqpoint{1.458397in}{2.417959in}}%
\pgfusepath{stroke}%
\end{pgfscope}%
\begin{pgfscope}%
\pgfpathrectangle{\pgfqpoint{0.100000in}{0.212622in}}{\pgfqpoint{3.696000in}{3.696000in}}%
\pgfusepath{clip}%
\pgfsetrectcap%
\pgfsetroundjoin%
\pgfsetlinewidth{1.505625pt}%
\definecolor{currentstroke}{rgb}{1.000000,0.000000,0.000000}%
\pgfsetstrokecolor{currentstroke}%
\pgfsetdash{}{0pt}%
\pgfpathmoveto{\pgfqpoint{1.330651in}{2.245725in}}%
\pgfpathlineto{\pgfqpoint{1.458397in}{2.417959in}}%
\pgfusepath{stroke}%
\end{pgfscope}%
\begin{pgfscope}%
\pgfpathrectangle{\pgfqpoint{0.100000in}{0.212622in}}{\pgfqpoint{3.696000in}{3.696000in}}%
\pgfusepath{clip}%
\pgfsetrectcap%
\pgfsetroundjoin%
\pgfsetlinewidth{1.505625pt}%
\definecolor{currentstroke}{rgb}{1.000000,0.000000,0.000000}%
\pgfsetstrokecolor{currentstroke}%
\pgfsetdash{}{0pt}%
\pgfpathmoveto{\pgfqpoint{1.328320in}{2.241692in}}%
\pgfpathlineto{\pgfqpoint{1.458397in}{2.417959in}}%
\pgfusepath{stroke}%
\end{pgfscope}%
\begin{pgfscope}%
\pgfpathrectangle{\pgfqpoint{0.100000in}{0.212622in}}{\pgfqpoint{3.696000in}{3.696000in}}%
\pgfusepath{clip}%
\pgfsetrectcap%
\pgfsetroundjoin%
\pgfsetlinewidth{1.505625pt}%
\definecolor{currentstroke}{rgb}{1.000000,0.000000,0.000000}%
\pgfsetstrokecolor{currentstroke}%
\pgfsetdash{}{0pt}%
\pgfpathmoveto{\pgfqpoint{1.326873in}{2.239690in}}%
\pgfpathlineto{\pgfqpoint{1.452143in}{2.412893in}}%
\pgfusepath{stroke}%
\end{pgfscope}%
\begin{pgfscope}%
\pgfpathrectangle{\pgfqpoint{0.100000in}{0.212622in}}{\pgfqpoint{3.696000in}{3.696000in}}%
\pgfusepath{clip}%
\pgfsetrectcap%
\pgfsetroundjoin%
\pgfsetlinewidth{1.505625pt}%
\definecolor{currentstroke}{rgb}{1.000000,0.000000,0.000000}%
\pgfsetstrokecolor{currentstroke}%
\pgfsetdash{}{0pt}%
\pgfpathmoveto{\pgfqpoint{1.325139in}{2.236961in}}%
\pgfpathlineto{\pgfqpoint{1.452143in}{2.412893in}}%
\pgfusepath{stroke}%
\end{pgfscope}%
\begin{pgfscope}%
\pgfpathrectangle{\pgfqpoint{0.100000in}{0.212622in}}{\pgfqpoint{3.696000in}{3.696000in}}%
\pgfusepath{clip}%
\pgfsetrectcap%
\pgfsetroundjoin%
\pgfsetlinewidth{1.505625pt}%
\definecolor{currentstroke}{rgb}{1.000000,0.000000,0.000000}%
\pgfsetstrokecolor{currentstroke}%
\pgfsetdash{}{0pt}%
\pgfpathmoveto{\pgfqpoint{1.323206in}{2.231683in}}%
\pgfpathlineto{\pgfqpoint{1.452143in}{2.412893in}}%
\pgfusepath{stroke}%
\end{pgfscope}%
\begin{pgfscope}%
\pgfpathrectangle{\pgfqpoint{0.100000in}{0.212622in}}{\pgfqpoint{3.696000in}{3.696000in}}%
\pgfusepath{clip}%
\pgfsetrectcap%
\pgfsetroundjoin%
\pgfsetlinewidth{1.505625pt}%
\definecolor{currentstroke}{rgb}{1.000000,0.000000,0.000000}%
\pgfsetstrokecolor{currentstroke}%
\pgfsetdash{}{0pt}%
\pgfpathmoveto{\pgfqpoint{1.322395in}{2.228583in}}%
\pgfpathlineto{\pgfqpoint{1.452143in}{2.412893in}}%
\pgfusepath{stroke}%
\end{pgfscope}%
\begin{pgfscope}%
\pgfpathrectangle{\pgfqpoint{0.100000in}{0.212622in}}{\pgfqpoint{3.696000in}{3.696000in}}%
\pgfusepath{clip}%
\pgfsetrectcap%
\pgfsetroundjoin%
\pgfsetlinewidth{1.505625pt}%
\definecolor{currentstroke}{rgb}{1.000000,0.000000,0.000000}%
\pgfsetstrokecolor{currentstroke}%
\pgfsetdash{}{0pt}%
\pgfpathmoveto{\pgfqpoint{1.320479in}{2.224447in}}%
\pgfpathlineto{\pgfqpoint{1.445884in}{2.407823in}}%
\pgfusepath{stroke}%
\end{pgfscope}%
\begin{pgfscope}%
\pgfpathrectangle{\pgfqpoint{0.100000in}{0.212622in}}{\pgfqpoint{3.696000in}{3.696000in}}%
\pgfusepath{clip}%
\pgfsetrectcap%
\pgfsetroundjoin%
\pgfsetlinewidth{1.505625pt}%
\definecolor{currentstroke}{rgb}{1.000000,0.000000,0.000000}%
\pgfsetstrokecolor{currentstroke}%
\pgfsetdash{}{0pt}%
\pgfpathmoveto{\pgfqpoint{1.317609in}{2.219944in}}%
\pgfpathlineto{\pgfqpoint{1.445884in}{2.407823in}}%
\pgfusepath{stroke}%
\end{pgfscope}%
\begin{pgfscope}%
\pgfpathrectangle{\pgfqpoint{0.100000in}{0.212622in}}{\pgfqpoint{3.696000in}{3.696000in}}%
\pgfusepath{clip}%
\pgfsetrectcap%
\pgfsetroundjoin%
\pgfsetlinewidth{1.505625pt}%
\definecolor{currentstroke}{rgb}{1.000000,0.000000,0.000000}%
\pgfsetstrokecolor{currentstroke}%
\pgfsetdash{}{0pt}%
\pgfpathmoveto{\pgfqpoint{1.314295in}{2.215222in}}%
\pgfpathlineto{\pgfqpoint{1.445884in}{2.407823in}}%
\pgfusepath{stroke}%
\end{pgfscope}%
\begin{pgfscope}%
\pgfpathrectangle{\pgfqpoint{0.100000in}{0.212622in}}{\pgfqpoint{3.696000in}{3.696000in}}%
\pgfusepath{clip}%
\pgfsetrectcap%
\pgfsetroundjoin%
\pgfsetlinewidth{1.505625pt}%
\definecolor{currentstroke}{rgb}{1.000000,0.000000,0.000000}%
\pgfsetstrokecolor{currentstroke}%
\pgfsetdash{}{0pt}%
\pgfpathmoveto{\pgfqpoint{1.310293in}{2.209301in}}%
\pgfpathlineto{\pgfqpoint{1.439620in}{2.402747in}}%
\pgfusepath{stroke}%
\end{pgfscope}%
\begin{pgfscope}%
\pgfpathrectangle{\pgfqpoint{0.100000in}{0.212622in}}{\pgfqpoint{3.696000in}{3.696000in}}%
\pgfusepath{clip}%
\pgfsetrectcap%
\pgfsetroundjoin%
\pgfsetlinewidth{1.505625pt}%
\definecolor{currentstroke}{rgb}{1.000000,0.000000,0.000000}%
\pgfsetstrokecolor{currentstroke}%
\pgfsetdash{}{0pt}%
\pgfpathmoveto{\pgfqpoint{1.306952in}{2.199754in}}%
\pgfpathlineto{\pgfqpoint{1.433349in}{2.397668in}}%
\pgfusepath{stroke}%
\end{pgfscope}%
\begin{pgfscope}%
\pgfpathrectangle{\pgfqpoint{0.100000in}{0.212622in}}{\pgfqpoint{3.696000in}{3.696000in}}%
\pgfusepath{clip}%
\pgfsetrectcap%
\pgfsetroundjoin%
\pgfsetlinewidth{1.505625pt}%
\definecolor{currentstroke}{rgb}{1.000000,0.000000,0.000000}%
\pgfsetstrokecolor{currentstroke}%
\pgfsetdash{}{0pt}%
\pgfpathmoveto{\pgfqpoint{1.304367in}{2.188903in}}%
\pgfpathlineto{\pgfqpoint{1.433349in}{2.397668in}}%
\pgfusepath{stroke}%
\end{pgfscope}%
\begin{pgfscope}%
\pgfpathrectangle{\pgfqpoint{0.100000in}{0.212622in}}{\pgfqpoint{3.696000in}{3.696000in}}%
\pgfusepath{clip}%
\pgfsetrectcap%
\pgfsetroundjoin%
\pgfsetlinewidth{1.505625pt}%
\definecolor{currentstroke}{rgb}{1.000000,0.000000,0.000000}%
\pgfsetstrokecolor{currentstroke}%
\pgfsetdash{}{0pt}%
\pgfpathmoveto{\pgfqpoint{1.298727in}{2.177461in}}%
\pgfpathlineto{\pgfqpoint{1.427073in}{2.392583in}}%
\pgfusepath{stroke}%
\end{pgfscope}%
\begin{pgfscope}%
\pgfpathrectangle{\pgfqpoint{0.100000in}{0.212622in}}{\pgfqpoint{3.696000in}{3.696000in}}%
\pgfusepath{clip}%
\pgfsetrectcap%
\pgfsetroundjoin%
\pgfsetlinewidth{1.505625pt}%
\definecolor{currentstroke}{rgb}{1.000000,0.000000,0.000000}%
\pgfsetstrokecolor{currentstroke}%
\pgfsetdash{}{0pt}%
\pgfpathmoveto{\pgfqpoint{1.291331in}{2.167127in}}%
\pgfpathlineto{\pgfqpoint{1.420792in}{2.387494in}}%
\pgfusepath{stroke}%
\end{pgfscope}%
\begin{pgfscope}%
\pgfpathrectangle{\pgfqpoint{0.100000in}{0.212622in}}{\pgfqpoint{3.696000in}{3.696000in}}%
\pgfusepath{clip}%
\pgfsetrectcap%
\pgfsetroundjoin%
\pgfsetlinewidth{1.505625pt}%
\definecolor{currentstroke}{rgb}{1.000000,0.000000,0.000000}%
\pgfsetstrokecolor{currentstroke}%
\pgfsetdash{}{0pt}%
\pgfpathmoveto{\pgfqpoint{1.283851in}{2.156166in}}%
\pgfpathlineto{\pgfqpoint{1.414504in}{2.382401in}}%
\pgfusepath{stroke}%
\end{pgfscope}%
\begin{pgfscope}%
\pgfpathrectangle{\pgfqpoint{0.100000in}{0.212622in}}{\pgfqpoint{3.696000in}{3.696000in}}%
\pgfusepath{clip}%
\pgfsetrectcap%
\pgfsetroundjoin%
\pgfsetlinewidth{1.505625pt}%
\definecolor{currentstroke}{rgb}{1.000000,0.000000,0.000000}%
\pgfsetstrokecolor{currentstroke}%
\pgfsetdash{}{0pt}%
\pgfpathmoveto{\pgfqpoint{1.277015in}{2.141284in}}%
\pgfpathlineto{\pgfqpoint{1.408211in}{2.377303in}}%
\pgfusepath{stroke}%
\end{pgfscope}%
\begin{pgfscope}%
\pgfpathrectangle{\pgfqpoint{0.100000in}{0.212622in}}{\pgfqpoint{3.696000in}{3.696000in}}%
\pgfusepath{clip}%
\pgfsetrectcap%
\pgfsetroundjoin%
\pgfsetlinewidth{1.505625pt}%
\definecolor{currentstroke}{rgb}{1.000000,0.000000,0.000000}%
\pgfsetstrokecolor{currentstroke}%
\pgfsetdash{}{0pt}%
\pgfpathmoveto{\pgfqpoint{1.273204in}{2.123592in}}%
\pgfpathlineto{\pgfqpoint{1.401912in}{2.372200in}}%
\pgfusepath{stroke}%
\end{pgfscope}%
\begin{pgfscope}%
\pgfpathrectangle{\pgfqpoint{0.100000in}{0.212622in}}{\pgfqpoint{3.696000in}{3.696000in}}%
\pgfusepath{clip}%
\pgfsetrectcap%
\pgfsetroundjoin%
\pgfsetlinewidth{1.505625pt}%
\definecolor{currentstroke}{rgb}{1.000000,0.000000,0.000000}%
\pgfsetstrokecolor{currentstroke}%
\pgfsetdash{}{0pt}%
\pgfpathmoveto{\pgfqpoint{1.267012in}{2.107021in}}%
\pgfpathlineto{\pgfqpoint{1.395608in}{2.367092in}}%
\pgfusepath{stroke}%
\end{pgfscope}%
\begin{pgfscope}%
\pgfpathrectangle{\pgfqpoint{0.100000in}{0.212622in}}{\pgfqpoint{3.696000in}{3.696000in}}%
\pgfusepath{clip}%
\pgfsetrectcap%
\pgfsetroundjoin%
\pgfsetlinewidth{1.505625pt}%
\definecolor{currentstroke}{rgb}{1.000000,0.000000,0.000000}%
\pgfsetstrokecolor{currentstroke}%
\pgfsetdash{}{0pt}%
\pgfpathmoveto{\pgfqpoint{1.257554in}{2.092324in}}%
\pgfpathlineto{\pgfqpoint{1.389298in}{2.361980in}}%
\pgfusepath{stroke}%
\end{pgfscope}%
\begin{pgfscope}%
\pgfpathrectangle{\pgfqpoint{0.100000in}{0.212622in}}{\pgfqpoint{3.696000in}{3.696000in}}%
\pgfusepath{clip}%
\pgfsetrectcap%
\pgfsetroundjoin%
\pgfsetlinewidth{1.505625pt}%
\definecolor{currentstroke}{rgb}{1.000000,0.000000,0.000000}%
\pgfsetstrokecolor{currentstroke}%
\pgfsetdash{}{0pt}%
\pgfpathmoveto{\pgfqpoint{1.247272in}{2.078710in}}%
\pgfpathlineto{\pgfqpoint{1.376660in}{2.351742in}}%
\pgfusepath{stroke}%
\end{pgfscope}%
\begin{pgfscope}%
\pgfpathrectangle{\pgfqpoint{0.100000in}{0.212622in}}{\pgfqpoint{3.696000in}{3.696000in}}%
\pgfusepath{clip}%
\pgfsetrectcap%
\pgfsetroundjoin%
\pgfsetlinewidth{1.505625pt}%
\definecolor{currentstroke}{rgb}{1.000000,0.000000,0.000000}%
\pgfsetstrokecolor{currentstroke}%
\pgfsetdash{}{0pt}%
\pgfpathmoveto{\pgfqpoint{1.236731in}{2.063109in}}%
\pgfpathlineto{\pgfqpoint{1.370333in}{2.346616in}}%
\pgfusepath{stroke}%
\end{pgfscope}%
\begin{pgfscope}%
\pgfpathrectangle{\pgfqpoint{0.100000in}{0.212622in}}{\pgfqpoint{3.696000in}{3.696000in}}%
\pgfusepath{clip}%
\pgfsetrectcap%
\pgfsetroundjoin%
\pgfsetlinewidth{1.505625pt}%
\definecolor{currentstroke}{rgb}{1.000000,0.000000,0.000000}%
\pgfsetstrokecolor{currentstroke}%
\pgfsetdash{}{0pt}%
\pgfpathmoveto{\pgfqpoint{1.227120in}{2.042604in}}%
\pgfpathlineto{\pgfqpoint{1.357661in}{2.336350in}}%
\pgfusepath{stroke}%
\end{pgfscope}%
\begin{pgfscope}%
\pgfpathrectangle{\pgfqpoint{0.100000in}{0.212622in}}{\pgfqpoint{3.696000in}{3.696000in}}%
\pgfusepath{clip}%
\pgfsetrectcap%
\pgfsetroundjoin%
\pgfsetlinewidth{1.505625pt}%
\definecolor{currentstroke}{rgb}{1.000000,0.000000,0.000000}%
\pgfsetstrokecolor{currentstroke}%
\pgfsetdash{}{0pt}%
\pgfpathmoveto{\pgfqpoint{1.215313in}{2.023103in}}%
\pgfpathlineto{\pgfqpoint{1.351316in}{2.331210in}}%
\pgfusepath{stroke}%
\end{pgfscope}%
\begin{pgfscope}%
\pgfpathrectangle{\pgfqpoint{0.100000in}{0.212622in}}{\pgfqpoint{3.696000in}{3.696000in}}%
\pgfusepath{clip}%
\pgfsetrectcap%
\pgfsetroundjoin%
\pgfsetlinewidth{1.505625pt}%
\definecolor{currentstroke}{rgb}{1.000000,0.000000,0.000000}%
\pgfsetstrokecolor{currentstroke}%
\pgfsetdash{}{0pt}%
\pgfpathmoveto{\pgfqpoint{1.204700in}{2.000172in}}%
\pgfpathlineto{\pgfqpoint{1.338609in}{2.320916in}}%
\pgfusepath{stroke}%
\end{pgfscope}%
\begin{pgfscope}%
\pgfpathrectangle{\pgfqpoint{0.100000in}{0.212622in}}{\pgfqpoint{3.696000in}{3.696000in}}%
\pgfusepath{clip}%
\pgfsetrectcap%
\pgfsetroundjoin%
\pgfsetlinewidth{1.505625pt}%
\definecolor{currentstroke}{rgb}{1.000000,0.000000,0.000000}%
\pgfsetstrokecolor{currentstroke}%
\pgfsetdash{}{0pt}%
\pgfpathmoveto{\pgfqpoint{1.200711in}{1.985731in}}%
\pgfpathlineto{\pgfqpoint{1.332247in}{2.315762in}}%
\pgfusepath{stroke}%
\end{pgfscope}%
\begin{pgfscope}%
\pgfpathrectangle{\pgfqpoint{0.100000in}{0.212622in}}{\pgfqpoint{3.696000in}{3.696000in}}%
\pgfusepath{clip}%
\pgfsetrectcap%
\pgfsetroundjoin%
\pgfsetlinewidth{1.505625pt}%
\definecolor{currentstroke}{rgb}{1.000000,0.000000,0.000000}%
\pgfsetstrokecolor{currentstroke}%
\pgfsetdash{}{0pt}%
\pgfpathmoveto{\pgfqpoint{1.195516in}{1.971561in}}%
\pgfpathlineto{\pgfqpoint{1.325880in}{2.310604in}}%
\pgfusepath{stroke}%
\end{pgfscope}%
\begin{pgfscope}%
\pgfpathrectangle{\pgfqpoint{0.100000in}{0.212622in}}{\pgfqpoint{3.696000in}{3.696000in}}%
\pgfusepath{clip}%
\pgfsetrectcap%
\pgfsetroundjoin%
\pgfsetlinewidth{1.505625pt}%
\definecolor{currentstroke}{rgb}{1.000000,0.000000,0.000000}%
\pgfsetstrokecolor{currentstroke}%
\pgfsetdash{}{0pt}%
\pgfpathmoveto{\pgfqpoint{1.187926in}{1.958281in}}%
\pgfpathlineto{\pgfqpoint{1.319506in}{2.305440in}}%
\pgfusepath{stroke}%
\end{pgfscope}%
\begin{pgfscope}%
\pgfpathrectangle{\pgfqpoint{0.100000in}{0.212622in}}{\pgfqpoint{3.696000in}{3.696000in}}%
\pgfusepath{clip}%
\pgfsetrectcap%
\pgfsetroundjoin%
\pgfsetlinewidth{1.505625pt}%
\definecolor{currentstroke}{rgb}{1.000000,0.000000,0.000000}%
\pgfsetstrokecolor{currentstroke}%
\pgfsetdash{}{0pt}%
\pgfpathmoveto{\pgfqpoint{1.183466in}{1.952129in}}%
\pgfpathlineto{\pgfqpoint{1.319506in}{2.305440in}}%
\pgfusepath{stroke}%
\end{pgfscope}%
\begin{pgfscope}%
\pgfpathrectangle{\pgfqpoint{0.100000in}{0.212622in}}{\pgfqpoint{3.696000in}{3.696000in}}%
\pgfusepath{clip}%
\pgfsetrectcap%
\pgfsetroundjoin%
\pgfsetlinewidth{1.505625pt}%
\definecolor{currentstroke}{rgb}{1.000000,0.000000,0.000000}%
\pgfsetstrokecolor{currentstroke}%
\pgfsetdash{}{0pt}%
\pgfpathmoveto{\pgfqpoint{1.178733in}{1.945547in}}%
\pgfpathlineto{\pgfqpoint{1.313127in}{2.300272in}}%
\pgfusepath{stroke}%
\end{pgfscope}%
\begin{pgfscope}%
\pgfpathrectangle{\pgfqpoint{0.100000in}{0.212622in}}{\pgfqpoint{3.696000in}{3.696000in}}%
\pgfusepath{clip}%
\pgfsetrectcap%
\pgfsetroundjoin%
\pgfsetlinewidth{1.505625pt}%
\definecolor{currentstroke}{rgb}{1.000000,0.000000,0.000000}%
\pgfsetstrokecolor{currentstroke}%
\pgfsetdash{}{0pt}%
\pgfpathmoveto{\pgfqpoint{1.173762in}{1.934835in}}%
\pgfpathlineto{\pgfqpoint{1.306741in}{2.295099in}}%
\pgfusepath{stroke}%
\end{pgfscope}%
\begin{pgfscope}%
\pgfpathrectangle{\pgfqpoint{0.100000in}{0.212622in}}{\pgfqpoint{3.696000in}{3.696000in}}%
\pgfusepath{clip}%
\pgfsetrectcap%
\pgfsetroundjoin%
\pgfsetlinewidth{1.505625pt}%
\definecolor{currentstroke}{rgb}{1.000000,0.000000,0.000000}%
\pgfsetstrokecolor{currentstroke}%
\pgfsetdash{}{0pt}%
\pgfpathmoveto{\pgfqpoint{1.172066in}{1.928450in}}%
\pgfpathlineto{\pgfqpoint{1.306741in}{2.295099in}}%
\pgfusepath{stroke}%
\end{pgfscope}%
\begin{pgfscope}%
\pgfpathrectangle{\pgfqpoint{0.100000in}{0.212622in}}{\pgfqpoint{3.696000in}{3.696000in}}%
\pgfusepath{clip}%
\pgfsetrectcap%
\pgfsetroundjoin%
\pgfsetlinewidth{1.505625pt}%
\definecolor{currentstroke}{rgb}{1.000000,0.000000,0.000000}%
\pgfsetstrokecolor{currentstroke}%
\pgfsetdash{}{0pt}%
\pgfpathmoveto{\pgfqpoint{1.169334in}{1.922195in}}%
\pgfpathlineto{\pgfqpoint{1.306741in}{2.295099in}}%
\pgfusepath{stroke}%
\end{pgfscope}%
\begin{pgfscope}%
\pgfpathrectangle{\pgfqpoint{0.100000in}{0.212622in}}{\pgfqpoint{3.696000in}{3.696000in}}%
\pgfusepath{clip}%
\pgfsetrectcap%
\pgfsetroundjoin%
\pgfsetlinewidth{1.505625pt}%
\definecolor{currentstroke}{rgb}{1.000000,0.000000,0.000000}%
\pgfsetstrokecolor{currentstroke}%
\pgfsetdash{}{0pt}%
\pgfpathmoveto{\pgfqpoint{1.167559in}{1.919032in}}%
\pgfpathlineto{\pgfqpoint{1.300350in}{2.289922in}}%
\pgfusepath{stroke}%
\end{pgfscope}%
\begin{pgfscope}%
\pgfpathrectangle{\pgfqpoint{0.100000in}{0.212622in}}{\pgfqpoint{3.696000in}{3.696000in}}%
\pgfusepath{clip}%
\pgfsetrectcap%
\pgfsetroundjoin%
\pgfsetlinewidth{1.505625pt}%
\definecolor{currentstroke}{rgb}{1.000000,0.000000,0.000000}%
\pgfsetstrokecolor{currentstroke}%
\pgfsetdash{}{0pt}%
\pgfpathmoveto{\pgfqpoint{1.165338in}{1.915714in}}%
\pgfpathlineto{\pgfqpoint{1.300350in}{2.289922in}}%
\pgfusepath{stroke}%
\end{pgfscope}%
\begin{pgfscope}%
\pgfpathrectangle{\pgfqpoint{0.100000in}{0.212622in}}{\pgfqpoint{3.696000in}{3.696000in}}%
\pgfusepath{clip}%
\pgfsetrectcap%
\pgfsetroundjoin%
\pgfsetlinewidth{1.505625pt}%
\definecolor{currentstroke}{rgb}{1.000000,0.000000,0.000000}%
\pgfsetstrokecolor{currentstroke}%
\pgfsetdash{}{0pt}%
\pgfpathmoveto{\pgfqpoint{1.162840in}{1.912007in}}%
\pgfpathlineto{\pgfqpoint{1.300350in}{2.289922in}}%
\pgfusepath{stroke}%
\end{pgfscope}%
\begin{pgfscope}%
\pgfpathrectangle{\pgfqpoint{0.100000in}{0.212622in}}{\pgfqpoint{3.696000in}{3.696000in}}%
\pgfusepath{clip}%
\pgfsetrectcap%
\pgfsetroundjoin%
\pgfsetlinewidth{1.505625pt}%
\definecolor{currentstroke}{rgb}{1.000000,0.000000,0.000000}%
\pgfsetstrokecolor{currentstroke}%
\pgfsetdash{}{0pt}%
\pgfpathmoveto{\pgfqpoint{1.160109in}{1.906042in}}%
\pgfpathlineto{\pgfqpoint{1.293954in}{2.284739in}}%
\pgfusepath{stroke}%
\end{pgfscope}%
\begin{pgfscope}%
\pgfpathrectangle{\pgfqpoint{0.100000in}{0.212622in}}{\pgfqpoint{3.696000in}{3.696000in}}%
\pgfusepath{clip}%
\pgfsetrectcap%
\pgfsetroundjoin%
\pgfsetlinewidth{1.505625pt}%
\definecolor{currentstroke}{rgb}{1.000000,0.000000,0.000000}%
\pgfsetstrokecolor{currentstroke}%
\pgfsetdash{}{0pt}%
\pgfpathmoveto{\pgfqpoint{1.159035in}{1.902508in}}%
\pgfpathlineto{\pgfqpoint{1.293954in}{2.284739in}}%
\pgfusepath{stroke}%
\end{pgfscope}%
\begin{pgfscope}%
\pgfpathrectangle{\pgfqpoint{0.100000in}{0.212622in}}{\pgfqpoint{3.696000in}{3.696000in}}%
\pgfusepath{clip}%
\pgfsetrectcap%
\pgfsetroundjoin%
\pgfsetlinewidth{1.505625pt}%
\definecolor{currentstroke}{rgb}{1.000000,0.000000,0.000000}%
\pgfsetstrokecolor{currentstroke}%
\pgfsetdash{}{0pt}%
\pgfpathmoveto{\pgfqpoint{1.157057in}{1.898451in}}%
\pgfpathlineto{\pgfqpoint{1.293954in}{2.284739in}}%
\pgfusepath{stroke}%
\end{pgfscope}%
\begin{pgfscope}%
\pgfpathrectangle{\pgfqpoint{0.100000in}{0.212622in}}{\pgfqpoint{3.696000in}{3.696000in}}%
\pgfusepath{clip}%
\pgfsetrectcap%
\pgfsetroundjoin%
\pgfsetlinewidth{1.505625pt}%
\definecolor{currentstroke}{rgb}{1.000000,0.000000,0.000000}%
\pgfsetstrokecolor{currentstroke}%
\pgfsetdash{}{0pt}%
\pgfpathmoveto{\pgfqpoint{1.154216in}{1.893913in}}%
\pgfpathlineto{\pgfqpoint{1.287551in}{2.279552in}}%
\pgfusepath{stroke}%
\end{pgfscope}%
\begin{pgfscope}%
\pgfpathrectangle{\pgfqpoint{0.100000in}{0.212622in}}{\pgfqpoint{3.696000in}{3.696000in}}%
\pgfusepath{clip}%
\pgfsetrectcap%
\pgfsetroundjoin%
\pgfsetlinewidth{1.505625pt}%
\definecolor{currentstroke}{rgb}{1.000000,0.000000,0.000000}%
\pgfsetstrokecolor{currentstroke}%
\pgfsetdash{}{0pt}%
\pgfpathmoveto{\pgfqpoint{1.150703in}{1.889359in}}%
\pgfpathlineto{\pgfqpoint{1.287551in}{2.279552in}}%
\pgfusepath{stroke}%
\end{pgfscope}%
\begin{pgfscope}%
\pgfpathrectangle{\pgfqpoint{0.100000in}{0.212622in}}{\pgfqpoint{3.696000in}{3.696000in}}%
\pgfusepath{clip}%
\pgfsetrectcap%
\pgfsetroundjoin%
\pgfsetlinewidth{1.505625pt}%
\definecolor{currentstroke}{rgb}{1.000000,0.000000,0.000000}%
\pgfsetstrokecolor{currentstroke}%
\pgfsetdash{}{0pt}%
\pgfpathmoveto{\pgfqpoint{1.146764in}{1.882259in}}%
\pgfpathlineto{\pgfqpoint{1.008861in}{2.218616in}}%
\pgfusepath{stroke}%
\end{pgfscope}%
\begin{pgfscope}%
\pgfpathrectangle{\pgfqpoint{0.100000in}{0.212622in}}{\pgfqpoint{3.696000in}{3.696000in}}%
\pgfusepath{clip}%
\pgfsetrectcap%
\pgfsetroundjoin%
\pgfsetlinewidth{1.505625pt}%
\definecolor{currentstroke}{rgb}{1.000000,0.000000,0.000000}%
\pgfsetstrokecolor{currentstroke}%
\pgfsetdash{}{0pt}%
\pgfpathmoveto{\pgfqpoint{1.143224in}{1.872214in}}%
\pgfpathlineto{\pgfqpoint{1.008861in}{2.218616in}}%
\pgfusepath{stroke}%
\end{pgfscope}%
\begin{pgfscope}%
\pgfpathrectangle{\pgfqpoint{0.100000in}{0.212622in}}{\pgfqpoint{3.696000in}{3.696000in}}%
\pgfusepath{clip}%
\pgfsetrectcap%
\pgfsetroundjoin%
\pgfsetlinewidth{1.505625pt}%
\definecolor{currentstroke}{rgb}{1.000000,0.000000,0.000000}%
\pgfsetstrokecolor{currentstroke}%
\pgfsetdash{}{0pt}%
\pgfpathmoveto{\pgfqpoint{1.140074in}{1.860854in}}%
\pgfpathlineto{\pgfqpoint{1.010721in}{2.218070in}}%
\pgfusepath{stroke}%
\end{pgfscope}%
\begin{pgfscope}%
\pgfpathrectangle{\pgfqpoint{0.100000in}{0.212622in}}{\pgfqpoint{3.696000in}{3.696000in}}%
\pgfusepath{clip}%
\pgfsetrectcap%
\pgfsetroundjoin%
\pgfsetlinewidth{1.505625pt}%
\definecolor{currentstroke}{rgb}{1.000000,0.000000,0.000000}%
\pgfsetstrokecolor{currentstroke}%
\pgfsetdash{}{0pt}%
\pgfpathmoveto{\pgfqpoint{1.134758in}{1.849824in}}%
\pgfpathlineto{\pgfqpoint{1.010721in}{2.218070in}}%
\pgfusepath{stroke}%
\end{pgfscope}%
\begin{pgfscope}%
\pgfpathrectangle{\pgfqpoint{0.100000in}{0.212622in}}{\pgfqpoint{3.696000in}{3.696000in}}%
\pgfusepath{clip}%
\pgfsetrectcap%
\pgfsetroundjoin%
\pgfsetlinewidth{1.505625pt}%
\definecolor{currentstroke}{rgb}{1.000000,0.000000,0.000000}%
\pgfsetstrokecolor{currentstroke}%
\pgfsetdash{}{0pt}%
\pgfpathmoveto{\pgfqpoint{1.128426in}{1.839481in}}%
\pgfpathlineto{\pgfqpoint{1.008861in}{2.218616in}}%
\pgfusepath{stroke}%
\end{pgfscope}%
\begin{pgfscope}%
\pgfpathrectangle{\pgfqpoint{0.100000in}{0.212622in}}{\pgfqpoint{3.696000in}{3.696000in}}%
\pgfusepath{clip}%
\pgfsetrectcap%
\pgfsetroundjoin%
\pgfsetlinewidth{1.505625pt}%
\definecolor{currentstroke}{rgb}{1.000000,0.000000,0.000000}%
\pgfsetstrokecolor{currentstroke}%
\pgfsetdash{}{0pt}%
\pgfpathmoveto{\pgfqpoint{1.121362in}{1.829891in}}%
\pgfpathlineto{\pgfqpoint{1.007000in}{2.219161in}}%
\pgfusepath{stroke}%
\end{pgfscope}%
\begin{pgfscope}%
\pgfpathrectangle{\pgfqpoint{0.100000in}{0.212622in}}{\pgfqpoint{3.696000in}{3.696000in}}%
\pgfusepath{clip}%
\pgfsetrectcap%
\pgfsetroundjoin%
\pgfsetlinewidth{1.505625pt}%
\definecolor{currentstroke}{rgb}{1.000000,0.000000,0.000000}%
\pgfsetstrokecolor{currentstroke}%
\pgfsetdash{}{0pt}%
\pgfpathmoveto{\pgfqpoint{1.114038in}{1.817025in}}%
\pgfpathlineto{\pgfqpoint{1.007000in}{2.219161in}}%
\pgfusepath{stroke}%
\end{pgfscope}%
\begin{pgfscope}%
\pgfpathrectangle{\pgfqpoint{0.100000in}{0.212622in}}{\pgfqpoint{3.696000in}{3.696000in}}%
\pgfusepath{clip}%
\pgfsetrectcap%
\pgfsetroundjoin%
\pgfsetlinewidth{1.505625pt}%
\definecolor{currentstroke}{rgb}{1.000000,0.000000,0.000000}%
\pgfsetstrokecolor{currentstroke}%
\pgfsetdash{}{0pt}%
\pgfpathmoveto{\pgfqpoint{1.108519in}{1.798812in}}%
\pgfpathlineto{\pgfqpoint{1.008861in}{2.218616in}}%
\pgfusepath{stroke}%
\end{pgfscope}%
\begin{pgfscope}%
\pgfpathrectangle{\pgfqpoint{0.100000in}{0.212622in}}{\pgfqpoint{3.696000in}{3.696000in}}%
\pgfusepath{clip}%
\pgfsetrectcap%
\pgfsetroundjoin%
\pgfsetlinewidth{1.505625pt}%
\definecolor{currentstroke}{rgb}{1.000000,0.000000,0.000000}%
\pgfsetstrokecolor{currentstroke}%
\pgfsetdash{}{0pt}%
\pgfpathmoveto{\pgfqpoint{1.103392in}{1.778580in}}%
\pgfpathlineto{\pgfqpoint{1.010721in}{2.218070in}}%
\pgfusepath{stroke}%
\end{pgfscope}%
\begin{pgfscope}%
\pgfpathrectangle{\pgfqpoint{0.100000in}{0.212622in}}{\pgfqpoint{3.696000in}{3.696000in}}%
\pgfusepath{clip}%
\pgfsetrectcap%
\pgfsetroundjoin%
\pgfsetlinewidth{1.505625pt}%
\definecolor{currentstroke}{rgb}{1.000000,0.000000,0.000000}%
\pgfsetstrokecolor{currentstroke}%
\pgfsetdash{}{0pt}%
\pgfpathmoveto{\pgfqpoint{1.094580in}{1.759995in}}%
\pgfpathlineto{\pgfqpoint{1.010721in}{2.218070in}}%
\pgfusepath{stroke}%
\end{pgfscope}%
\begin{pgfscope}%
\pgfpathrectangle{\pgfqpoint{0.100000in}{0.212622in}}{\pgfqpoint{3.696000in}{3.696000in}}%
\pgfusepath{clip}%
\pgfsetrectcap%
\pgfsetroundjoin%
\pgfsetlinewidth{1.505625pt}%
\definecolor{currentstroke}{rgb}{1.000000,0.000000,0.000000}%
\pgfsetstrokecolor{currentstroke}%
\pgfsetdash{}{0pt}%
\pgfpathmoveto{\pgfqpoint{1.083699in}{1.741887in}}%
\pgfpathlineto{\pgfqpoint{1.008861in}{2.218616in}}%
\pgfusepath{stroke}%
\end{pgfscope}%
\begin{pgfscope}%
\pgfpathrectangle{\pgfqpoint{0.100000in}{0.212622in}}{\pgfqpoint{3.696000in}{3.696000in}}%
\pgfusepath{clip}%
\pgfsetrectcap%
\pgfsetroundjoin%
\pgfsetlinewidth{1.505625pt}%
\definecolor{currentstroke}{rgb}{1.000000,0.000000,0.000000}%
\pgfsetstrokecolor{currentstroke}%
\pgfsetdash{}{0pt}%
\pgfpathmoveto{\pgfqpoint{1.071528in}{1.724445in}}%
\pgfpathlineto{\pgfqpoint{1.005140in}{2.219707in}}%
\pgfusepath{stroke}%
\end{pgfscope}%
\begin{pgfscope}%
\pgfpathrectangle{\pgfqpoint{0.100000in}{0.212622in}}{\pgfqpoint{3.696000in}{3.696000in}}%
\pgfusepath{clip}%
\pgfsetrectcap%
\pgfsetroundjoin%
\pgfsetlinewidth{1.505625pt}%
\definecolor{currentstroke}{rgb}{1.000000,0.000000,0.000000}%
\pgfsetstrokecolor{currentstroke}%
\pgfsetdash{}{0pt}%
\pgfpathmoveto{\pgfqpoint{1.060090in}{1.705340in}}%
\pgfpathlineto{\pgfqpoint{1.003280in}{2.220252in}}%
\pgfusepath{stroke}%
\end{pgfscope}%
\begin{pgfscope}%
\pgfpathrectangle{\pgfqpoint{0.100000in}{0.212622in}}{\pgfqpoint{3.696000in}{3.696000in}}%
\pgfusepath{clip}%
\pgfsetrectcap%
\pgfsetroundjoin%
\pgfsetlinewidth{1.505625pt}%
\definecolor{currentstroke}{rgb}{1.000000,0.000000,0.000000}%
\pgfsetstrokecolor{currentstroke}%
\pgfsetdash{}{0pt}%
\pgfpathmoveto{\pgfqpoint{1.050435in}{1.680641in}}%
\pgfpathlineto{\pgfqpoint{1.003280in}{2.220252in}}%
\pgfusepath{stroke}%
\end{pgfscope}%
\begin{pgfscope}%
\pgfpathrectangle{\pgfqpoint{0.100000in}{0.212622in}}{\pgfqpoint{3.696000in}{3.696000in}}%
\pgfusepath{clip}%
\pgfsetrectcap%
\pgfsetroundjoin%
\pgfsetlinewidth{1.505625pt}%
\definecolor{currentstroke}{rgb}{1.000000,0.000000,0.000000}%
\pgfsetstrokecolor{currentstroke}%
\pgfsetdash{}{0pt}%
\pgfpathmoveto{\pgfqpoint{1.046127in}{1.666445in}}%
\pgfpathlineto{\pgfqpoint{1.005140in}{2.219707in}}%
\pgfusepath{stroke}%
\end{pgfscope}%
\begin{pgfscope}%
\pgfpathrectangle{\pgfqpoint{0.100000in}{0.212622in}}{\pgfqpoint{3.696000in}{3.696000in}}%
\pgfusepath{clip}%
\pgfsetrectcap%
\pgfsetroundjoin%
\pgfsetlinewidth{1.505625pt}%
\definecolor{currentstroke}{rgb}{1.000000,0.000000,0.000000}%
\pgfsetstrokecolor{currentstroke}%
\pgfsetdash{}{0pt}%
\pgfpathmoveto{\pgfqpoint{1.043807in}{1.658920in}}%
\pgfpathlineto{\pgfqpoint{1.005140in}{2.219707in}}%
\pgfusepath{stroke}%
\end{pgfscope}%
\begin{pgfscope}%
\pgfpathrectangle{\pgfqpoint{0.100000in}{0.212622in}}{\pgfqpoint{3.696000in}{3.696000in}}%
\pgfusepath{clip}%
\pgfsetrectcap%
\pgfsetroundjoin%
\pgfsetlinewidth{1.505625pt}%
\definecolor{currentstroke}{rgb}{1.000000,0.000000,0.000000}%
\pgfsetstrokecolor{currentstroke}%
\pgfsetdash{}{0pt}%
\pgfpathmoveto{\pgfqpoint{1.040304in}{1.651459in}}%
\pgfpathlineto{\pgfqpoint{1.005140in}{2.219707in}}%
\pgfusepath{stroke}%
\end{pgfscope}%
\begin{pgfscope}%
\pgfpathrectangle{\pgfqpoint{0.100000in}{0.212622in}}{\pgfqpoint{3.696000in}{3.696000in}}%
\pgfusepath{clip}%
\pgfsetrectcap%
\pgfsetroundjoin%
\pgfsetlinewidth{1.505625pt}%
\definecolor{currentstroke}{rgb}{1.000000,0.000000,0.000000}%
\pgfsetstrokecolor{currentstroke}%
\pgfsetdash{}{0pt}%
\pgfpathmoveto{\pgfqpoint{1.036404in}{1.643512in}}%
\pgfpathlineto{\pgfqpoint{1.005140in}{2.219707in}}%
\pgfusepath{stroke}%
\end{pgfscope}%
\begin{pgfscope}%
\pgfpathrectangle{\pgfqpoint{0.100000in}{0.212622in}}{\pgfqpoint{3.696000in}{3.696000in}}%
\pgfusepath{clip}%
\pgfsetrectcap%
\pgfsetroundjoin%
\pgfsetlinewidth{1.505625pt}%
\definecolor{currentstroke}{rgb}{1.000000,0.000000,0.000000}%
\pgfsetstrokecolor{currentstroke}%
\pgfsetdash{}{0pt}%
\pgfpathmoveto{\pgfqpoint{1.031578in}{1.635153in}}%
\pgfpathlineto{\pgfqpoint{1.005140in}{2.219707in}}%
\pgfusepath{stroke}%
\end{pgfscope}%
\begin{pgfscope}%
\pgfpathrectangle{\pgfqpoint{0.100000in}{0.212622in}}{\pgfqpoint{3.696000in}{3.696000in}}%
\pgfusepath{clip}%
\pgfsetrectcap%
\pgfsetroundjoin%
\pgfsetlinewidth{1.505625pt}%
\definecolor{currentstroke}{rgb}{1.000000,0.000000,0.000000}%
\pgfsetstrokecolor{currentstroke}%
\pgfsetdash{}{0pt}%
\pgfpathmoveto{\pgfqpoint{1.025636in}{1.627034in}}%
\pgfpathlineto{\pgfqpoint{1.003280in}{2.220252in}}%
\pgfusepath{stroke}%
\end{pgfscope}%
\begin{pgfscope}%
\pgfpathrectangle{\pgfqpoint{0.100000in}{0.212622in}}{\pgfqpoint{3.696000in}{3.696000in}}%
\pgfusepath{clip}%
\pgfsetrectcap%
\pgfsetroundjoin%
\pgfsetlinewidth{1.505625pt}%
\definecolor{currentstroke}{rgb}{1.000000,0.000000,0.000000}%
\pgfsetstrokecolor{currentstroke}%
\pgfsetdash{}{0pt}%
\pgfpathmoveto{\pgfqpoint{1.022590in}{1.622879in}}%
\pgfpathlineto{\pgfqpoint{1.001421in}{2.220797in}}%
\pgfusepath{stroke}%
\end{pgfscope}%
\begin{pgfscope}%
\pgfpathrectangle{\pgfqpoint{0.100000in}{0.212622in}}{\pgfqpoint{3.696000in}{3.696000in}}%
\pgfusepath{clip}%
\pgfsetrectcap%
\pgfsetroundjoin%
\pgfsetlinewidth{1.505625pt}%
\definecolor{currentstroke}{rgb}{1.000000,0.000000,0.000000}%
\pgfsetstrokecolor{currentstroke}%
\pgfsetdash{}{0pt}%
\pgfpathmoveto{\pgfqpoint{1.018963in}{1.615918in}}%
\pgfpathlineto{\pgfqpoint{1.001421in}{2.220797in}}%
\pgfusepath{stroke}%
\end{pgfscope}%
\begin{pgfscope}%
\pgfpathrectangle{\pgfqpoint{0.100000in}{0.212622in}}{\pgfqpoint{3.696000in}{3.696000in}}%
\pgfusepath{clip}%
\pgfsetrectcap%
\pgfsetroundjoin%
\pgfsetlinewidth{1.505625pt}%
\definecolor{currentstroke}{rgb}{1.000000,0.000000,0.000000}%
\pgfsetstrokecolor{currentstroke}%
\pgfsetdash{}{0pt}%
\pgfpathmoveto{\pgfqpoint{1.016938in}{1.612000in}}%
\pgfpathlineto{\pgfqpoint{1.001421in}{2.220797in}}%
\pgfusepath{stroke}%
\end{pgfscope}%
\begin{pgfscope}%
\pgfpathrectangle{\pgfqpoint{0.100000in}{0.212622in}}{\pgfqpoint{3.696000in}{3.696000in}}%
\pgfusepath{clip}%
\pgfsetrectcap%
\pgfsetroundjoin%
\pgfsetlinewidth{1.505625pt}%
\definecolor{currentstroke}{rgb}{1.000000,0.000000,0.000000}%
\pgfsetstrokecolor{currentstroke}%
\pgfsetdash{}{0pt}%
\pgfpathmoveto{\pgfqpoint{1.015872in}{1.609793in}}%
\pgfpathlineto{\pgfqpoint{1.001421in}{2.220797in}}%
\pgfusepath{stroke}%
\end{pgfscope}%
\begin{pgfscope}%
\pgfpathrectangle{\pgfqpoint{0.100000in}{0.212622in}}{\pgfqpoint{3.696000in}{3.696000in}}%
\pgfusepath{clip}%
\pgfsetrectcap%
\pgfsetroundjoin%
\pgfsetlinewidth{1.505625pt}%
\definecolor{currentstroke}{rgb}{1.000000,0.000000,0.000000}%
\pgfsetstrokecolor{currentstroke}%
\pgfsetdash{}{0pt}%
\pgfpathmoveto{\pgfqpoint{1.015231in}{1.608635in}}%
\pgfpathlineto{\pgfqpoint{1.001421in}{2.220797in}}%
\pgfusepath{stroke}%
\end{pgfscope}%
\begin{pgfscope}%
\pgfpathrectangle{\pgfqpoint{0.100000in}{0.212622in}}{\pgfqpoint{3.696000in}{3.696000in}}%
\pgfusepath{clip}%
\pgfsetrectcap%
\pgfsetroundjoin%
\pgfsetlinewidth{1.505625pt}%
\definecolor{currentstroke}{rgb}{1.000000,0.000000,0.000000}%
\pgfsetstrokecolor{currentstroke}%
\pgfsetdash{}{0pt}%
\pgfpathmoveto{\pgfqpoint{1.014835in}{1.608044in}}%
\pgfpathlineto{\pgfqpoint{1.001421in}{2.220797in}}%
\pgfusepath{stroke}%
\end{pgfscope}%
\begin{pgfscope}%
\pgfpathrectangle{\pgfqpoint{0.100000in}{0.212622in}}{\pgfqpoint{3.696000in}{3.696000in}}%
\pgfusepath{clip}%
\pgfsetrectcap%
\pgfsetroundjoin%
\pgfsetlinewidth{1.505625pt}%
\definecolor{currentstroke}{rgb}{1.000000,0.000000,0.000000}%
\pgfsetstrokecolor{currentstroke}%
\pgfsetdash{}{0pt}%
\pgfpathmoveto{\pgfqpoint{1.013913in}{1.606732in}}%
\pgfpathlineto{\pgfqpoint{1.001421in}{2.220797in}}%
\pgfusepath{stroke}%
\end{pgfscope}%
\begin{pgfscope}%
\pgfpathrectangle{\pgfqpoint{0.100000in}{0.212622in}}{\pgfqpoint{3.696000in}{3.696000in}}%
\pgfusepath{clip}%
\pgfsetrectcap%
\pgfsetroundjoin%
\pgfsetlinewidth{1.505625pt}%
\definecolor{currentstroke}{rgb}{1.000000,0.000000,0.000000}%
\pgfsetstrokecolor{currentstroke}%
\pgfsetdash{}{0pt}%
\pgfpathmoveto{\pgfqpoint{1.012704in}{1.605041in}}%
\pgfpathlineto{\pgfqpoint{1.001421in}{2.220797in}}%
\pgfusepath{stroke}%
\end{pgfscope}%
\begin{pgfscope}%
\pgfpathrectangle{\pgfqpoint{0.100000in}{0.212622in}}{\pgfqpoint{3.696000in}{3.696000in}}%
\pgfusepath{clip}%
\pgfsetrectcap%
\pgfsetroundjoin%
\pgfsetlinewidth{1.505625pt}%
\definecolor{currentstroke}{rgb}{1.000000,0.000000,0.000000}%
\pgfsetstrokecolor{currentstroke}%
\pgfsetdash{}{0pt}%
\pgfpathmoveto{\pgfqpoint{1.011255in}{1.601705in}}%
\pgfpathlineto{\pgfqpoint{1.001421in}{2.220797in}}%
\pgfusepath{stroke}%
\end{pgfscope}%
\begin{pgfscope}%
\pgfpathrectangle{\pgfqpoint{0.100000in}{0.212622in}}{\pgfqpoint{3.696000in}{3.696000in}}%
\pgfusepath{clip}%
\pgfsetrectcap%
\pgfsetroundjoin%
\pgfsetlinewidth{1.505625pt}%
\definecolor{currentstroke}{rgb}{1.000000,0.000000,0.000000}%
\pgfsetstrokecolor{currentstroke}%
\pgfsetdash{}{0pt}%
\pgfpathmoveto{\pgfqpoint{1.009694in}{1.597585in}}%
\pgfpathlineto{\pgfqpoint{1.001421in}{2.220797in}}%
\pgfusepath{stroke}%
\end{pgfscope}%
\begin{pgfscope}%
\pgfpathrectangle{\pgfqpoint{0.100000in}{0.212622in}}{\pgfqpoint{3.696000in}{3.696000in}}%
\pgfusepath{clip}%
\pgfsetrectcap%
\pgfsetroundjoin%
\pgfsetlinewidth{1.505625pt}%
\definecolor{currentstroke}{rgb}{1.000000,0.000000,0.000000}%
\pgfsetstrokecolor{currentstroke}%
\pgfsetdash{}{0pt}%
\pgfpathmoveto{\pgfqpoint{1.008371in}{1.592317in}}%
\pgfpathlineto{\pgfqpoint{1.001421in}{2.220797in}}%
\pgfusepath{stroke}%
\end{pgfscope}%
\begin{pgfscope}%
\pgfpathrectangle{\pgfqpoint{0.100000in}{0.212622in}}{\pgfqpoint{3.696000in}{3.696000in}}%
\pgfusepath{clip}%
\pgfsetrectcap%
\pgfsetroundjoin%
\pgfsetlinewidth{1.505625pt}%
\definecolor{currentstroke}{rgb}{1.000000,0.000000,0.000000}%
\pgfsetstrokecolor{currentstroke}%
\pgfsetdash{}{0pt}%
\pgfpathmoveto{\pgfqpoint{1.006067in}{1.585934in}}%
\pgfpathlineto{\pgfqpoint{1.001421in}{2.220797in}}%
\pgfusepath{stroke}%
\end{pgfscope}%
\begin{pgfscope}%
\pgfpathrectangle{\pgfqpoint{0.100000in}{0.212622in}}{\pgfqpoint{3.696000in}{3.696000in}}%
\pgfusepath{clip}%
\pgfsetrectcap%
\pgfsetroundjoin%
\pgfsetlinewidth{1.505625pt}%
\definecolor{currentstroke}{rgb}{1.000000,0.000000,0.000000}%
\pgfsetstrokecolor{currentstroke}%
\pgfsetdash{}{0pt}%
\pgfpathmoveto{\pgfqpoint{1.002391in}{1.579805in}}%
\pgfpathlineto{\pgfqpoint{1.001421in}{2.220797in}}%
\pgfusepath{stroke}%
\end{pgfscope}%
\begin{pgfscope}%
\pgfpathrectangle{\pgfqpoint{0.100000in}{0.212622in}}{\pgfqpoint{3.696000in}{3.696000in}}%
\pgfusepath{clip}%
\pgfsetrectcap%
\pgfsetroundjoin%
\pgfsetlinewidth{1.505625pt}%
\definecolor{currentstroke}{rgb}{1.000000,0.000000,0.000000}%
\pgfsetstrokecolor{currentstroke}%
\pgfsetdash{}{0pt}%
\pgfpathmoveto{\pgfqpoint{1.000390in}{1.576611in}}%
\pgfpathlineto{\pgfqpoint{1.001421in}{2.220797in}}%
\pgfusepath{stroke}%
\end{pgfscope}%
\begin{pgfscope}%
\pgfpathrectangle{\pgfqpoint{0.100000in}{0.212622in}}{\pgfqpoint{3.696000in}{3.696000in}}%
\pgfusepath{clip}%
\pgfsetrectcap%
\pgfsetroundjoin%
\pgfsetlinewidth{1.505625pt}%
\definecolor{currentstroke}{rgb}{1.000000,0.000000,0.000000}%
\pgfsetstrokecolor{currentstroke}%
\pgfsetdash{}{0pt}%
\pgfpathmoveto{\pgfqpoint{0.999269in}{1.574997in}}%
\pgfpathlineto{\pgfqpoint{1.001421in}{2.220797in}}%
\pgfusepath{stroke}%
\end{pgfscope}%
\begin{pgfscope}%
\pgfpathrectangle{\pgfqpoint{0.100000in}{0.212622in}}{\pgfqpoint{3.696000in}{3.696000in}}%
\pgfusepath{clip}%
\pgfsetrectcap%
\pgfsetroundjoin%
\pgfsetlinewidth{1.505625pt}%
\definecolor{currentstroke}{rgb}{1.000000,0.000000,0.000000}%
\pgfsetstrokecolor{currentstroke}%
\pgfsetdash{}{0pt}%
\pgfpathmoveto{\pgfqpoint{0.997817in}{1.572267in}}%
\pgfpathlineto{\pgfqpoint{1.001421in}{2.220797in}}%
\pgfusepath{stroke}%
\end{pgfscope}%
\begin{pgfscope}%
\pgfpathrectangle{\pgfqpoint{0.100000in}{0.212622in}}{\pgfqpoint{3.696000in}{3.696000in}}%
\pgfusepath{clip}%
\pgfsetrectcap%
\pgfsetroundjoin%
\pgfsetlinewidth{1.505625pt}%
\definecolor{currentstroke}{rgb}{1.000000,0.000000,0.000000}%
\pgfsetstrokecolor{currentstroke}%
\pgfsetdash{}{0pt}%
\pgfpathmoveto{\pgfqpoint{0.996340in}{1.568278in}}%
\pgfpathlineto{\pgfqpoint{1.001421in}{2.220797in}}%
\pgfusepath{stroke}%
\end{pgfscope}%
\begin{pgfscope}%
\pgfpathrectangle{\pgfqpoint{0.100000in}{0.212622in}}{\pgfqpoint{3.696000in}{3.696000in}}%
\pgfusepath{clip}%
\pgfsetrectcap%
\pgfsetroundjoin%
\pgfsetlinewidth{1.505625pt}%
\definecolor{currentstroke}{rgb}{1.000000,0.000000,0.000000}%
\pgfsetstrokecolor{currentstroke}%
\pgfsetdash{}{0pt}%
\pgfpathmoveto{\pgfqpoint{0.995465in}{1.563337in}}%
\pgfpathlineto{\pgfqpoint{1.001421in}{2.220797in}}%
\pgfusepath{stroke}%
\end{pgfscope}%
\begin{pgfscope}%
\pgfpathrectangle{\pgfqpoint{0.100000in}{0.212622in}}{\pgfqpoint{3.696000in}{3.696000in}}%
\pgfusepath{clip}%
\pgfsetrectcap%
\pgfsetroundjoin%
\pgfsetlinewidth{1.505625pt}%
\definecolor{currentstroke}{rgb}{1.000000,0.000000,0.000000}%
\pgfsetstrokecolor{currentstroke}%
\pgfsetdash{}{0pt}%
\pgfpathmoveto{\pgfqpoint{0.994118in}{1.558019in}}%
\pgfpathlineto{\pgfqpoint{1.003280in}{2.220252in}}%
\pgfusepath{stroke}%
\end{pgfscope}%
\begin{pgfscope}%
\pgfpathrectangle{\pgfqpoint{0.100000in}{0.212622in}}{\pgfqpoint{3.696000in}{3.696000in}}%
\pgfusepath{clip}%
\pgfsetrectcap%
\pgfsetroundjoin%
\pgfsetlinewidth{1.505625pt}%
\definecolor{currentstroke}{rgb}{1.000000,0.000000,0.000000}%
\pgfsetstrokecolor{currentstroke}%
\pgfsetdash{}{0pt}%
\pgfpathmoveto{\pgfqpoint{0.992839in}{1.555401in}}%
\pgfpathlineto{\pgfqpoint{1.003280in}{2.220252in}}%
\pgfusepath{stroke}%
\end{pgfscope}%
\begin{pgfscope}%
\pgfpathrectangle{\pgfqpoint{0.100000in}{0.212622in}}{\pgfqpoint{3.696000in}{3.696000in}}%
\pgfusepath{clip}%
\pgfsetrectcap%
\pgfsetroundjoin%
\pgfsetlinewidth{1.505625pt}%
\definecolor{currentstroke}{rgb}{1.000000,0.000000,0.000000}%
\pgfsetstrokecolor{currentstroke}%
\pgfsetdash{}{0pt}%
\pgfpathmoveto{\pgfqpoint{0.991241in}{1.552521in}}%
\pgfpathlineto{\pgfqpoint{1.003280in}{2.220252in}}%
\pgfusepath{stroke}%
\end{pgfscope}%
\begin{pgfscope}%
\pgfpathrectangle{\pgfqpoint{0.100000in}{0.212622in}}{\pgfqpoint{3.696000in}{3.696000in}}%
\pgfusepath{clip}%
\pgfsetrectcap%
\pgfsetroundjoin%
\pgfsetlinewidth{1.505625pt}%
\definecolor{currentstroke}{rgb}{1.000000,0.000000,0.000000}%
\pgfsetstrokecolor{currentstroke}%
\pgfsetdash{}{0pt}%
\pgfpathmoveto{\pgfqpoint{0.989305in}{1.548865in}}%
\pgfpathlineto{\pgfqpoint{1.001421in}{2.220797in}}%
\pgfusepath{stroke}%
\end{pgfscope}%
\begin{pgfscope}%
\pgfpathrectangle{\pgfqpoint{0.100000in}{0.212622in}}{\pgfqpoint{3.696000in}{3.696000in}}%
\pgfusepath{clip}%
\pgfsetrectcap%
\pgfsetroundjoin%
\pgfsetlinewidth{1.505625pt}%
\definecolor{currentstroke}{rgb}{1.000000,0.000000,0.000000}%
\pgfsetstrokecolor{currentstroke}%
\pgfsetdash{}{0pt}%
\pgfpathmoveto{\pgfqpoint{0.988258in}{1.546873in}}%
\pgfpathlineto{\pgfqpoint{1.001421in}{2.220797in}}%
\pgfusepath{stroke}%
\end{pgfscope}%
\begin{pgfscope}%
\pgfpathrectangle{\pgfqpoint{0.100000in}{0.212622in}}{\pgfqpoint{3.696000in}{3.696000in}}%
\pgfusepath{clip}%
\pgfsetrectcap%
\pgfsetroundjoin%
\pgfsetlinewidth{1.505625pt}%
\definecolor{currentstroke}{rgb}{1.000000,0.000000,0.000000}%
\pgfsetstrokecolor{currentstroke}%
\pgfsetdash{}{0pt}%
\pgfpathmoveto{\pgfqpoint{0.987706in}{1.545729in}}%
\pgfpathlineto{\pgfqpoint{1.001421in}{2.220797in}}%
\pgfusepath{stroke}%
\end{pgfscope}%
\begin{pgfscope}%
\pgfpathrectangle{\pgfqpoint{0.100000in}{0.212622in}}{\pgfqpoint{3.696000in}{3.696000in}}%
\pgfusepath{clip}%
\pgfsetrectcap%
\pgfsetroundjoin%
\pgfsetlinewidth{1.505625pt}%
\definecolor{currentstroke}{rgb}{1.000000,0.000000,0.000000}%
\pgfsetstrokecolor{currentstroke}%
\pgfsetdash{}{0pt}%
\pgfpathmoveto{\pgfqpoint{0.987403in}{1.545088in}}%
\pgfpathlineto{\pgfqpoint{1.001421in}{2.220797in}}%
\pgfusepath{stroke}%
\end{pgfscope}%
\begin{pgfscope}%
\pgfpathrectangle{\pgfqpoint{0.100000in}{0.212622in}}{\pgfqpoint{3.696000in}{3.696000in}}%
\pgfusepath{clip}%
\pgfsetrectcap%
\pgfsetroundjoin%
\pgfsetlinewidth{1.505625pt}%
\definecolor{currentstroke}{rgb}{1.000000,0.000000,0.000000}%
\pgfsetstrokecolor{currentstroke}%
\pgfsetdash{}{0pt}%
\pgfpathmoveto{\pgfqpoint{0.987253in}{1.544736in}}%
\pgfpathlineto{\pgfqpoint{1.001421in}{2.220797in}}%
\pgfusepath{stroke}%
\end{pgfscope}%
\begin{pgfscope}%
\pgfpathrectangle{\pgfqpoint{0.100000in}{0.212622in}}{\pgfqpoint{3.696000in}{3.696000in}}%
\pgfusepath{clip}%
\pgfsetrectcap%
\pgfsetroundjoin%
\pgfsetlinewidth{1.505625pt}%
\definecolor{currentstroke}{rgb}{1.000000,0.000000,0.000000}%
\pgfsetstrokecolor{currentstroke}%
\pgfsetdash{}{0pt}%
\pgfpathmoveto{\pgfqpoint{0.987164in}{1.544547in}}%
\pgfpathlineto{\pgfqpoint{1.001421in}{2.220797in}}%
\pgfusepath{stroke}%
\end{pgfscope}%
\begin{pgfscope}%
\pgfpathrectangle{\pgfqpoint{0.100000in}{0.212622in}}{\pgfqpoint{3.696000in}{3.696000in}}%
\pgfusepath{clip}%
\pgfsetrectcap%
\pgfsetroundjoin%
\pgfsetlinewidth{1.505625pt}%
\definecolor{currentstroke}{rgb}{1.000000,0.000000,0.000000}%
\pgfsetstrokecolor{currentstroke}%
\pgfsetdash{}{0pt}%
\pgfpathmoveto{\pgfqpoint{0.987113in}{1.544446in}}%
\pgfpathlineto{\pgfqpoint{1.001421in}{2.220797in}}%
\pgfusepath{stroke}%
\end{pgfscope}%
\begin{pgfscope}%
\pgfpathrectangle{\pgfqpoint{0.100000in}{0.212622in}}{\pgfqpoint{3.696000in}{3.696000in}}%
\pgfusepath{clip}%
\pgfsetrectcap%
\pgfsetroundjoin%
\pgfsetlinewidth{1.505625pt}%
\definecolor{currentstroke}{rgb}{1.000000,0.000000,0.000000}%
\pgfsetstrokecolor{currentstroke}%
\pgfsetdash{}{0pt}%
\pgfpathmoveto{\pgfqpoint{0.987084in}{1.544390in}}%
\pgfpathlineto{\pgfqpoint{1.001421in}{2.220797in}}%
\pgfusepath{stroke}%
\end{pgfscope}%
\begin{pgfscope}%
\pgfpathrectangle{\pgfqpoint{0.100000in}{0.212622in}}{\pgfqpoint{3.696000in}{3.696000in}}%
\pgfusepath{clip}%
\pgfsetrectcap%
\pgfsetroundjoin%
\pgfsetlinewidth{1.505625pt}%
\definecolor{currentstroke}{rgb}{1.000000,0.000000,0.000000}%
\pgfsetstrokecolor{currentstroke}%
\pgfsetdash{}{0pt}%
\pgfpathmoveto{\pgfqpoint{0.987069in}{1.544359in}}%
\pgfpathlineto{\pgfqpoint{1.001421in}{2.220797in}}%
\pgfusepath{stroke}%
\end{pgfscope}%
\begin{pgfscope}%
\pgfpathrectangle{\pgfqpoint{0.100000in}{0.212622in}}{\pgfqpoint{3.696000in}{3.696000in}}%
\pgfusepath{clip}%
\pgfsetrectcap%
\pgfsetroundjoin%
\pgfsetlinewidth{1.505625pt}%
\definecolor{currentstroke}{rgb}{1.000000,0.000000,0.000000}%
\pgfsetstrokecolor{currentstroke}%
\pgfsetdash{}{0pt}%
\pgfpathmoveto{\pgfqpoint{0.987061in}{1.544342in}}%
\pgfpathlineto{\pgfqpoint{1.001421in}{2.220797in}}%
\pgfusepath{stroke}%
\end{pgfscope}%
\begin{pgfscope}%
\pgfpathrectangle{\pgfqpoint{0.100000in}{0.212622in}}{\pgfqpoint{3.696000in}{3.696000in}}%
\pgfusepath{clip}%
\pgfsetrectcap%
\pgfsetroundjoin%
\pgfsetlinewidth{1.505625pt}%
\definecolor{currentstroke}{rgb}{1.000000,0.000000,0.000000}%
\pgfsetstrokecolor{currentstroke}%
\pgfsetdash{}{0pt}%
\pgfpathmoveto{\pgfqpoint{0.987056in}{1.544333in}}%
\pgfpathlineto{\pgfqpoint{1.001421in}{2.220797in}}%
\pgfusepath{stroke}%
\end{pgfscope}%
\begin{pgfscope}%
\pgfpathrectangle{\pgfqpoint{0.100000in}{0.212622in}}{\pgfqpoint{3.696000in}{3.696000in}}%
\pgfusepath{clip}%
\pgfsetrectcap%
\pgfsetroundjoin%
\pgfsetlinewidth{1.505625pt}%
\definecolor{currentstroke}{rgb}{1.000000,0.000000,0.000000}%
\pgfsetstrokecolor{currentstroke}%
\pgfsetdash{}{0pt}%
\pgfpathmoveto{\pgfqpoint{0.987054in}{1.544328in}}%
\pgfpathlineto{\pgfqpoint{1.001421in}{2.220797in}}%
\pgfusepath{stroke}%
\end{pgfscope}%
\begin{pgfscope}%
\pgfpathrectangle{\pgfqpoint{0.100000in}{0.212622in}}{\pgfqpoint{3.696000in}{3.696000in}}%
\pgfusepath{clip}%
\pgfsetrectcap%
\pgfsetroundjoin%
\pgfsetlinewidth{1.505625pt}%
\definecolor{currentstroke}{rgb}{1.000000,0.000000,0.000000}%
\pgfsetstrokecolor{currentstroke}%
\pgfsetdash{}{0pt}%
\pgfpathmoveto{\pgfqpoint{0.987052in}{1.544325in}}%
\pgfpathlineto{\pgfqpoint{1.001421in}{2.220797in}}%
\pgfusepath{stroke}%
\end{pgfscope}%
\begin{pgfscope}%
\pgfpathrectangle{\pgfqpoint{0.100000in}{0.212622in}}{\pgfqpoint{3.696000in}{3.696000in}}%
\pgfusepath{clip}%
\pgfsetrectcap%
\pgfsetroundjoin%
\pgfsetlinewidth{1.505625pt}%
\definecolor{currentstroke}{rgb}{1.000000,0.000000,0.000000}%
\pgfsetstrokecolor{currentstroke}%
\pgfsetdash{}{0pt}%
\pgfpathmoveto{\pgfqpoint{0.987051in}{1.544323in}}%
\pgfpathlineto{\pgfqpoint{1.001421in}{2.220797in}}%
\pgfusepath{stroke}%
\end{pgfscope}%
\begin{pgfscope}%
\pgfpathrectangle{\pgfqpoint{0.100000in}{0.212622in}}{\pgfqpoint{3.696000in}{3.696000in}}%
\pgfusepath{clip}%
\pgfsetrectcap%
\pgfsetroundjoin%
\pgfsetlinewidth{1.505625pt}%
\definecolor{currentstroke}{rgb}{1.000000,0.000000,0.000000}%
\pgfsetstrokecolor{currentstroke}%
\pgfsetdash{}{0pt}%
\pgfpathmoveto{\pgfqpoint{0.987051in}{1.544322in}}%
\pgfpathlineto{\pgfqpoint{1.001421in}{2.220797in}}%
\pgfusepath{stroke}%
\end{pgfscope}%
\begin{pgfscope}%
\pgfpathrectangle{\pgfqpoint{0.100000in}{0.212622in}}{\pgfqpoint{3.696000in}{3.696000in}}%
\pgfusepath{clip}%
\pgfsetrectcap%
\pgfsetroundjoin%
\pgfsetlinewidth{1.505625pt}%
\definecolor{currentstroke}{rgb}{1.000000,0.000000,0.000000}%
\pgfsetstrokecolor{currentstroke}%
\pgfsetdash{}{0pt}%
\pgfpathmoveto{\pgfqpoint{0.987051in}{1.544322in}}%
\pgfpathlineto{\pgfqpoint{1.001421in}{2.220797in}}%
\pgfusepath{stroke}%
\end{pgfscope}%
\begin{pgfscope}%
\pgfpathrectangle{\pgfqpoint{0.100000in}{0.212622in}}{\pgfqpoint{3.696000in}{3.696000in}}%
\pgfusepath{clip}%
\pgfsetrectcap%
\pgfsetroundjoin%
\pgfsetlinewidth{1.505625pt}%
\definecolor{currentstroke}{rgb}{1.000000,0.000000,0.000000}%
\pgfsetstrokecolor{currentstroke}%
\pgfsetdash{}{0pt}%
\pgfpathmoveto{\pgfqpoint{0.987051in}{1.544322in}}%
\pgfpathlineto{\pgfqpoint{1.001421in}{2.220797in}}%
\pgfusepath{stroke}%
\end{pgfscope}%
\begin{pgfscope}%
\pgfpathrectangle{\pgfqpoint{0.100000in}{0.212622in}}{\pgfqpoint{3.696000in}{3.696000in}}%
\pgfusepath{clip}%
\pgfsetrectcap%
\pgfsetroundjoin%
\pgfsetlinewidth{1.505625pt}%
\definecolor{currentstroke}{rgb}{1.000000,0.000000,0.000000}%
\pgfsetstrokecolor{currentstroke}%
\pgfsetdash{}{0pt}%
\pgfpathmoveto{\pgfqpoint{0.987051in}{1.544321in}}%
\pgfpathlineto{\pgfqpoint{1.001421in}{2.220797in}}%
\pgfusepath{stroke}%
\end{pgfscope}%
\begin{pgfscope}%
\pgfpathrectangle{\pgfqpoint{0.100000in}{0.212622in}}{\pgfqpoint{3.696000in}{3.696000in}}%
\pgfusepath{clip}%
\pgfsetrectcap%
\pgfsetroundjoin%
\pgfsetlinewidth{1.505625pt}%
\definecolor{currentstroke}{rgb}{1.000000,0.000000,0.000000}%
\pgfsetstrokecolor{currentstroke}%
\pgfsetdash{}{0pt}%
\pgfpathmoveto{\pgfqpoint{0.987051in}{1.544321in}}%
\pgfpathlineto{\pgfqpoint{1.001421in}{2.220797in}}%
\pgfusepath{stroke}%
\end{pgfscope}%
\begin{pgfscope}%
\pgfpathrectangle{\pgfqpoint{0.100000in}{0.212622in}}{\pgfqpoint{3.696000in}{3.696000in}}%
\pgfusepath{clip}%
\pgfsetrectcap%
\pgfsetroundjoin%
\pgfsetlinewidth{1.505625pt}%
\definecolor{currentstroke}{rgb}{1.000000,0.000000,0.000000}%
\pgfsetstrokecolor{currentstroke}%
\pgfsetdash{}{0pt}%
\pgfpathmoveto{\pgfqpoint{0.987051in}{1.544321in}}%
\pgfpathlineto{\pgfqpoint{1.001421in}{2.220797in}}%
\pgfusepath{stroke}%
\end{pgfscope}%
\begin{pgfscope}%
\pgfpathrectangle{\pgfqpoint{0.100000in}{0.212622in}}{\pgfqpoint{3.696000in}{3.696000in}}%
\pgfusepath{clip}%
\pgfsetrectcap%
\pgfsetroundjoin%
\pgfsetlinewidth{1.505625pt}%
\definecolor{currentstroke}{rgb}{1.000000,0.000000,0.000000}%
\pgfsetstrokecolor{currentstroke}%
\pgfsetdash{}{0pt}%
\pgfpathmoveto{\pgfqpoint{0.987050in}{1.544321in}}%
\pgfpathlineto{\pgfqpoint{1.001421in}{2.220797in}}%
\pgfusepath{stroke}%
\end{pgfscope}%
\begin{pgfscope}%
\pgfpathrectangle{\pgfqpoint{0.100000in}{0.212622in}}{\pgfqpoint{3.696000in}{3.696000in}}%
\pgfusepath{clip}%
\pgfsetrectcap%
\pgfsetroundjoin%
\pgfsetlinewidth{1.505625pt}%
\definecolor{currentstroke}{rgb}{1.000000,0.000000,0.000000}%
\pgfsetstrokecolor{currentstroke}%
\pgfsetdash{}{0pt}%
\pgfpathmoveto{\pgfqpoint{0.987050in}{1.544321in}}%
\pgfpathlineto{\pgfqpoint{1.001421in}{2.220797in}}%
\pgfusepath{stroke}%
\end{pgfscope}%
\begin{pgfscope}%
\pgfpathrectangle{\pgfqpoint{0.100000in}{0.212622in}}{\pgfqpoint{3.696000in}{3.696000in}}%
\pgfusepath{clip}%
\pgfsetrectcap%
\pgfsetroundjoin%
\pgfsetlinewidth{1.505625pt}%
\definecolor{currentstroke}{rgb}{1.000000,0.000000,0.000000}%
\pgfsetstrokecolor{currentstroke}%
\pgfsetdash{}{0pt}%
\pgfpathmoveto{\pgfqpoint{0.987050in}{1.544321in}}%
\pgfpathlineto{\pgfqpoint{1.001421in}{2.220797in}}%
\pgfusepath{stroke}%
\end{pgfscope}%
\begin{pgfscope}%
\pgfpathrectangle{\pgfqpoint{0.100000in}{0.212622in}}{\pgfqpoint{3.696000in}{3.696000in}}%
\pgfusepath{clip}%
\pgfsetrectcap%
\pgfsetroundjoin%
\pgfsetlinewidth{1.505625pt}%
\definecolor{currentstroke}{rgb}{1.000000,0.000000,0.000000}%
\pgfsetstrokecolor{currentstroke}%
\pgfsetdash{}{0pt}%
\pgfpathmoveto{\pgfqpoint{0.987050in}{1.544321in}}%
\pgfpathlineto{\pgfqpoint{1.001421in}{2.220797in}}%
\pgfusepath{stroke}%
\end{pgfscope}%
\begin{pgfscope}%
\pgfpathrectangle{\pgfqpoint{0.100000in}{0.212622in}}{\pgfqpoint{3.696000in}{3.696000in}}%
\pgfusepath{clip}%
\pgfsetrectcap%
\pgfsetroundjoin%
\pgfsetlinewidth{1.505625pt}%
\definecolor{currentstroke}{rgb}{1.000000,0.000000,0.000000}%
\pgfsetstrokecolor{currentstroke}%
\pgfsetdash{}{0pt}%
\pgfpathmoveto{\pgfqpoint{0.987050in}{1.544321in}}%
\pgfpathlineto{\pgfqpoint{1.001421in}{2.220797in}}%
\pgfusepath{stroke}%
\end{pgfscope}%
\begin{pgfscope}%
\pgfpathrectangle{\pgfqpoint{0.100000in}{0.212622in}}{\pgfqpoint{3.696000in}{3.696000in}}%
\pgfusepath{clip}%
\pgfsetrectcap%
\pgfsetroundjoin%
\pgfsetlinewidth{1.505625pt}%
\definecolor{currentstroke}{rgb}{1.000000,0.000000,0.000000}%
\pgfsetstrokecolor{currentstroke}%
\pgfsetdash{}{0pt}%
\pgfpathmoveto{\pgfqpoint{0.987050in}{1.544321in}}%
\pgfpathlineto{\pgfqpoint{1.001421in}{2.220797in}}%
\pgfusepath{stroke}%
\end{pgfscope}%
\begin{pgfscope}%
\pgfpathrectangle{\pgfqpoint{0.100000in}{0.212622in}}{\pgfqpoint{3.696000in}{3.696000in}}%
\pgfusepath{clip}%
\pgfsetrectcap%
\pgfsetroundjoin%
\pgfsetlinewidth{1.505625pt}%
\definecolor{currentstroke}{rgb}{1.000000,0.000000,0.000000}%
\pgfsetstrokecolor{currentstroke}%
\pgfsetdash{}{0pt}%
\pgfpathmoveto{\pgfqpoint{0.987050in}{1.544321in}}%
\pgfpathlineto{\pgfqpoint{1.001421in}{2.220797in}}%
\pgfusepath{stroke}%
\end{pgfscope}%
\begin{pgfscope}%
\pgfpathrectangle{\pgfqpoint{0.100000in}{0.212622in}}{\pgfqpoint{3.696000in}{3.696000in}}%
\pgfusepath{clip}%
\pgfsetrectcap%
\pgfsetroundjoin%
\pgfsetlinewidth{1.505625pt}%
\definecolor{currentstroke}{rgb}{1.000000,0.000000,0.000000}%
\pgfsetstrokecolor{currentstroke}%
\pgfsetdash{}{0pt}%
\pgfpathmoveto{\pgfqpoint{0.987050in}{1.544321in}}%
\pgfpathlineto{\pgfqpoint{1.001421in}{2.220797in}}%
\pgfusepath{stroke}%
\end{pgfscope}%
\begin{pgfscope}%
\pgfpathrectangle{\pgfqpoint{0.100000in}{0.212622in}}{\pgfqpoint{3.696000in}{3.696000in}}%
\pgfusepath{clip}%
\pgfsetrectcap%
\pgfsetroundjoin%
\pgfsetlinewidth{1.505625pt}%
\definecolor{currentstroke}{rgb}{1.000000,0.000000,0.000000}%
\pgfsetstrokecolor{currentstroke}%
\pgfsetdash{}{0pt}%
\pgfpathmoveto{\pgfqpoint{0.987050in}{1.544321in}}%
\pgfpathlineto{\pgfqpoint{1.001421in}{2.220797in}}%
\pgfusepath{stroke}%
\end{pgfscope}%
\begin{pgfscope}%
\pgfpathrectangle{\pgfqpoint{0.100000in}{0.212622in}}{\pgfqpoint{3.696000in}{3.696000in}}%
\pgfusepath{clip}%
\pgfsetrectcap%
\pgfsetroundjoin%
\pgfsetlinewidth{1.505625pt}%
\definecolor{currentstroke}{rgb}{1.000000,0.000000,0.000000}%
\pgfsetstrokecolor{currentstroke}%
\pgfsetdash{}{0pt}%
\pgfpathmoveto{\pgfqpoint{0.987050in}{1.544321in}}%
\pgfpathlineto{\pgfqpoint{1.001421in}{2.220797in}}%
\pgfusepath{stroke}%
\end{pgfscope}%
\begin{pgfscope}%
\pgfpathrectangle{\pgfqpoint{0.100000in}{0.212622in}}{\pgfqpoint{3.696000in}{3.696000in}}%
\pgfusepath{clip}%
\pgfsetrectcap%
\pgfsetroundjoin%
\pgfsetlinewidth{1.505625pt}%
\definecolor{currentstroke}{rgb}{1.000000,0.000000,0.000000}%
\pgfsetstrokecolor{currentstroke}%
\pgfsetdash{}{0pt}%
\pgfpathmoveto{\pgfqpoint{0.987050in}{1.544321in}}%
\pgfpathlineto{\pgfqpoint{1.001421in}{2.220797in}}%
\pgfusepath{stroke}%
\end{pgfscope}%
\begin{pgfscope}%
\pgfpathrectangle{\pgfqpoint{0.100000in}{0.212622in}}{\pgfqpoint{3.696000in}{3.696000in}}%
\pgfusepath{clip}%
\pgfsetrectcap%
\pgfsetroundjoin%
\pgfsetlinewidth{1.505625pt}%
\definecolor{currentstroke}{rgb}{1.000000,0.000000,0.000000}%
\pgfsetstrokecolor{currentstroke}%
\pgfsetdash{}{0pt}%
\pgfpathmoveto{\pgfqpoint{0.987050in}{1.544321in}}%
\pgfpathlineto{\pgfqpoint{1.001421in}{2.220797in}}%
\pgfusepath{stroke}%
\end{pgfscope}%
\begin{pgfscope}%
\pgfpathrectangle{\pgfqpoint{0.100000in}{0.212622in}}{\pgfqpoint{3.696000in}{3.696000in}}%
\pgfusepath{clip}%
\pgfsetrectcap%
\pgfsetroundjoin%
\pgfsetlinewidth{1.505625pt}%
\definecolor{currentstroke}{rgb}{1.000000,0.000000,0.000000}%
\pgfsetstrokecolor{currentstroke}%
\pgfsetdash{}{0pt}%
\pgfpathmoveto{\pgfqpoint{0.987050in}{1.544321in}}%
\pgfpathlineto{\pgfqpoint{1.001421in}{2.220797in}}%
\pgfusepath{stroke}%
\end{pgfscope}%
\begin{pgfscope}%
\pgfpathrectangle{\pgfqpoint{0.100000in}{0.212622in}}{\pgfqpoint{3.696000in}{3.696000in}}%
\pgfusepath{clip}%
\pgfsetrectcap%
\pgfsetroundjoin%
\pgfsetlinewidth{1.505625pt}%
\definecolor{currentstroke}{rgb}{1.000000,0.000000,0.000000}%
\pgfsetstrokecolor{currentstroke}%
\pgfsetdash{}{0pt}%
\pgfpathmoveto{\pgfqpoint{0.987050in}{1.544321in}}%
\pgfpathlineto{\pgfqpoint{1.001421in}{2.220797in}}%
\pgfusepath{stroke}%
\end{pgfscope}%
\begin{pgfscope}%
\pgfpathrectangle{\pgfqpoint{0.100000in}{0.212622in}}{\pgfqpoint{3.696000in}{3.696000in}}%
\pgfusepath{clip}%
\pgfsetrectcap%
\pgfsetroundjoin%
\pgfsetlinewidth{1.505625pt}%
\definecolor{currentstroke}{rgb}{1.000000,0.000000,0.000000}%
\pgfsetstrokecolor{currentstroke}%
\pgfsetdash{}{0pt}%
\pgfpathmoveto{\pgfqpoint{0.987050in}{1.544321in}}%
\pgfpathlineto{\pgfqpoint{1.001421in}{2.220797in}}%
\pgfusepath{stroke}%
\end{pgfscope}%
\begin{pgfscope}%
\pgfpathrectangle{\pgfqpoint{0.100000in}{0.212622in}}{\pgfqpoint{3.696000in}{3.696000in}}%
\pgfusepath{clip}%
\pgfsetrectcap%
\pgfsetroundjoin%
\pgfsetlinewidth{1.505625pt}%
\definecolor{currentstroke}{rgb}{1.000000,0.000000,0.000000}%
\pgfsetstrokecolor{currentstroke}%
\pgfsetdash{}{0pt}%
\pgfpathmoveto{\pgfqpoint{0.987050in}{1.544321in}}%
\pgfpathlineto{\pgfqpoint{1.001421in}{2.220797in}}%
\pgfusepath{stroke}%
\end{pgfscope}%
\begin{pgfscope}%
\pgfpathrectangle{\pgfqpoint{0.100000in}{0.212622in}}{\pgfqpoint{3.696000in}{3.696000in}}%
\pgfusepath{clip}%
\pgfsetrectcap%
\pgfsetroundjoin%
\pgfsetlinewidth{1.505625pt}%
\definecolor{currentstroke}{rgb}{1.000000,0.000000,0.000000}%
\pgfsetstrokecolor{currentstroke}%
\pgfsetdash{}{0pt}%
\pgfpathmoveto{\pgfqpoint{0.987050in}{1.544321in}}%
\pgfpathlineto{\pgfqpoint{1.001421in}{2.220797in}}%
\pgfusepath{stroke}%
\end{pgfscope}%
\begin{pgfscope}%
\pgfpathrectangle{\pgfqpoint{0.100000in}{0.212622in}}{\pgfqpoint{3.696000in}{3.696000in}}%
\pgfusepath{clip}%
\pgfsetrectcap%
\pgfsetroundjoin%
\pgfsetlinewidth{1.505625pt}%
\definecolor{currentstroke}{rgb}{1.000000,0.000000,0.000000}%
\pgfsetstrokecolor{currentstroke}%
\pgfsetdash{}{0pt}%
\pgfpathmoveto{\pgfqpoint{0.987050in}{1.544321in}}%
\pgfpathlineto{\pgfqpoint{1.001421in}{2.220797in}}%
\pgfusepath{stroke}%
\end{pgfscope}%
\begin{pgfscope}%
\pgfpathrectangle{\pgfqpoint{0.100000in}{0.212622in}}{\pgfqpoint{3.696000in}{3.696000in}}%
\pgfusepath{clip}%
\pgfsetrectcap%
\pgfsetroundjoin%
\pgfsetlinewidth{1.505625pt}%
\definecolor{currentstroke}{rgb}{1.000000,0.000000,0.000000}%
\pgfsetstrokecolor{currentstroke}%
\pgfsetdash{}{0pt}%
\pgfpathmoveto{\pgfqpoint{0.987050in}{1.544321in}}%
\pgfpathlineto{\pgfqpoint{1.001421in}{2.220797in}}%
\pgfusepath{stroke}%
\end{pgfscope}%
\begin{pgfscope}%
\pgfpathrectangle{\pgfqpoint{0.100000in}{0.212622in}}{\pgfqpoint{3.696000in}{3.696000in}}%
\pgfusepath{clip}%
\pgfsetrectcap%
\pgfsetroundjoin%
\pgfsetlinewidth{1.505625pt}%
\definecolor{currentstroke}{rgb}{1.000000,0.000000,0.000000}%
\pgfsetstrokecolor{currentstroke}%
\pgfsetdash{}{0pt}%
\pgfpathmoveto{\pgfqpoint{0.987050in}{1.544321in}}%
\pgfpathlineto{\pgfqpoint{1.001421in}{2.220797in}}%
\pgfusepath{stroke}%
\end{pgfscope}%
\begin{pgfscope}%
\pgfpathrectangle{\pgfqpoint{0.100000in}{0.212622in}}{\pgfqpoint{3.696000in}{3.696000in}}%
\pgfusepath{clip}%
\pgfsetrectcap%
\pgfsetroundjoin%
\pgfsetlinewidth{1.505625pt}%
\definecolor{currentstroke}{rgb}{1.000000,0.000000,0.000000}%
\pgfsetstrokecolor{currentstroke}%
\pgfsetdash{}{0pt}%
\pgfpathmoveto{\pgfqpoint{0.987050in}{1.544321in}}%
\pgfpathlineto{\pgfqpoint{1.001421in}{2.220797in}}%
\pgfusepath{stroke}%
\end{pgfscope}%
\begin{pgfscope}%
\pgfpathrectangle{\pgfqpoint{0.100000in}{0.212622in}}{\pgfqpoint{3.696000in}{3.696000in}}%
\pgfusepath{clip}%
\pgfsetrectcap%
\pgfsetroundjoin%
\pgfsetlinewidth{1.505625pt}%
\definecolor{currentstroke}{rgb}{1.000000,0.000000,0.000000}%
\pgfsetstrokecolor{currentstroke}%
\pgfsetdash{}{0pt}%
\pgfpathmoveto{\pgfqpoint{0.987050in}{1.544321in}}%
\pgfpathlineto{\pgfqpoint{1.001421in}{2.220797in}}%
\pgfusepath{stroke}%
\end{pgfscope}%
\begin{pgfscope}%
\pgfpathrectangle{\pgfqpoint{0.100000in}{0.212622in}}{\pgfqpoint{3.696000in}{3.696000in}}%
\pgfusepath{clip}%
\pgfsetrectcap%
\pgfsetroundjoin%
\pgfsetlinewidth{1.505625pt}%
\definecolor{currentstroke}{rgb}{1.000000,0.000000,0.000000}%
\pgfsetstrokecolor{currentstroke}%
\pgfsetdash{}{0pt}%
\pgfpathmoveto{\pgfqpoint{0.987050in}{1.544321in}}%
\pgfpathlineto{\pgfqpoint{1.001421in}{2.220797in}}%
\pgfusepath{stroke}%
\end{pgfscope}%
\begin{pgfscope}%
\pgfpathrectangle{\pgfqpoint{0.100000in}{0.212622in}}{\pgfqpoint{3.696000in}{3.696000in}}%
\pgfusepath{clip}%
\pgfsetrectcap%
\pgfsetroundjoin%
\pgfsetlinewidth{1.505625pt}%
\definecolor{currentstroke}{rgb}{1.000000,0.000000,0.000000}%
\pgfsetstrokecolor{currentstroke}%
\pgfsetdash{}{0pt}%
\pgfpathmoveto{\pgfqpoint{0.987050in}{1.544321in}}%
\pgfpathlineto{\pgfqpoint{1.001421in}{2.220797in}}%
\pgfusepath{stroke}%
\end{pgfscope}%
\begin{pgfscope}%
\pgfpathrectangle{\pgfqpoint{0.100000in}{0.212622in}}{\pgfqpoint{3.696000in}{3.696000in}}%
\pgfusepath{clip}%
\pgfsetrectcap%
\pgfsetroundjoin%
\pgfsetlinewidth{1.505625pt}%
\definecolor{currentstroke}{rgb}{1.000000,0.000000,0.000000}%
\pgfsetstrokecolor{currentstroke}%
\pgfsetdash{}{0pt}%
\pgfpathmoveto{\pgfqpoint{0.987050in}{1.544321in}}%
\pgfpathlineto{\pgfqpoint{1.001421in}{2.220797in}}%
\pgfusepath{stroke}%
\end{pgfscope}%
\begin{pgfscope}%
\pgfpathrectangle{\pgfqpoint{0.100000in}{0.212622in}}{\pgfqpoint{3.696000in}{3.696000in}}%
\pgfusepath{clip}%
\pgfsetrectcap%
\pgfsetroundjoin%
\pgfsetlinewidth{1.505625pt}%
\definecolor{currentstroke}{rgb}{1.000000,0.000000,0.000000}%
\pgfsetstrokecolor{currentstroke}%
\pgfsetdash{}{0pt}%
\pgfpathmoveto{\pgfqpoint{0.987050in}{1.544321in}}%
\pgfpathlineto{\pgfqpoint{1.001421in}{2.220797in}}%
\pgfusepath{stroke}%
\end{pgfscope}%
\begin{pgfscope}%
\pgfpathrectangle{\pgfqpoint{0.100000in}{0.212622in}}{\pgfqpoint{3.696000in}{3.696000in}}%
\pgfusepath{clip}%
\pgfsetrectcap%
\pgfsetroundjoin%
\pgfsetlinewidth{1.505625pt}%
\definecolor{currentstroke}{rgb}{1.000000,0.000000,0.000000}%
\pgfsetstrokecolor{currentstroke}%
\pgfsetdash{}{0pt}%
\pgfpathmoveto{\pgfqpoint{0.987050in}{1.544321in}}%
\pgfpathlineto{\pgfqpoint{1.001421in}{2.220797in}}%
\pgfusepath{stroke}%
\end{pgfscope}%
\begin{pgfscope}%
\pgfpathrectangle{\pgfqpoint{0.100000in}{0.212622in}}{\pgfqpoint{3.696000in}{3.696000in}}%
\pgfusepath{clip}%
\pgfsetrectcap%
\pgfsetroundjoin%
\pgfsetlinewidth{1.505625pt}%
\definecolor{currentstroke}{rgb}{1.000000,0.000000,0.000000}%
\pgfsetstrokecolor{currentstroke}%
\pgfsetdash{}{0pt}%
\pgfpathmoveto{\pgfqpoint{0.986982in}{1.543898in}}%
\pgfpathlineto{\pgfqpoint{1.003280in}{2.220252in}}%
\pgfusepath{stroke}%
\end{pgfscope}%
\begin{pgfscope}%
\pgfpathrectangle{\pgfqpoint{0.100000in}{0.212622in}}{\pgfqpoint{3.696000in}{3.696000in}}%
\pgfusepath{clip}%
\pgfsetrectcap%
\pgfsetroundjoin%
\pgfsetlinewidth{1.505625pt}%
\definecolor{currentstroke}{rgb}{1.000000,0.000000,0.000000}%
\pgfsetstrokecolor{currentstroke}%
\pgfsetdash{}{0pt}%
\pgfpathmoveto{\pgfqpoint{0.986974in}{1.543696in}}%
\pgfpathlineto{\pgfqpoint{1.003280in}{2.220252in}}%
\pgfusepath{stroke}%
\end{pgfscope}%
\begin{pgfscope}%
\pgfpathrectangle{\pgfqpoint{0.100000in}{0.212622in}}{\pgfqpoint{3.696000in}{3.696000in}}%
\pgfusepath{clip}%
\pgfsetrectcap%
\pgfsetroundjoin%
\pgfsetlinewidth{1.505625pt}%
\definecolor{currentstroke}{rgb}{1.000000,0.000000,0.000000}%
\pgfsetstrokecolor{currentstroke}%
\pgfsetdash{}{0pt}%
\pgfpathmoveto{\pgfqpoint{0.987089in}{1.543225in}}%
\pgfpathlineto{\pgfqpoint{1.003280in}{2.220252in}}%
\pgfusepath{stroke}%
\end{pgfscope}%
\begin{pgfscope}%
\pgfpathrectangle{\pgfqpoint{0.100000in}{0.212622in}}{\pgfqpoint{3.696000in}{3.696000in}}%
\pgfusepath{clip}%
\pgfsetrectcap%
\pgfsetroundjoin%
\pgfsetlinewidth{1.505625pt}%
\definecolor{currentstroke}{rgb}{1.000000,0.000000,0.000000}%
\pgfsetstrokecolor{currentstroke}%
\pgfsetdash{}{0pt}%
\pgfpathmoveto{\pgfqpoint{0.987289in}{1.543080in}}%
\pgfpathlineto{\pgfqpoint{1.003280in}{2.220252in}}%
\pgfusepath{stroke}%
\end{pgfscope}%
\begin{pgfscope}%
\pgfpathrectangle{\pgfqpoint{0.100000in}{0.212622in}}{\pgfqpoint{3.696000in}{3.696000in}}%
\pgfusepath{clip}%
\pgfsetrectcap%
\pgfsetroundjoin%
\pgfsetlinewidth{1.505625pt}%
\definecolor{currentstroke}{rgb}{1.000000,0.000000,0.000000}%
\pgfsetstrokecolor{currentstroke}%
\pgfsetdash{}{0pt}%
\pgfpathmoveto{\pgfqpoint{0.987420in}{1.543028in}}%
\pgfpathlineto{\pgfqpoint{1.003280in}{2.220252in}}%
\pgfusepath{stroke}%
\end{pgfscope}%
\begin{pgfscope}%
\pgfpathrectangle{\pgfqpoint{0.100000in}{0.212622in}}{\pgfqpoint{3.696000in}{3.696000in}}%
\pgfusepath{clip}%
\pgfsetrectcap%
\pgfsetroundjoin%
\pgfsetlinewidth{1.505625pt}%
\definecolor{currentstroke}{rgb}{1.000000,0.000000,0.000000}%
\pgfsetstrokecolor{currentstroke}%
\pgfsetdash{}{0pt}%
\pgfpathmoveto{\pgfqpoint{0.987911in}{1.542818in}}%
\pgfpathlineto{\pgfqpoint{1.003280in}{2.220252in}}%
\pgfusepath{stroke}%
\end{pgfscope}%
\begin{pgfscope}%
\pgfpathrectangle{\pgfqpoint{0.100000in}{0.212622in}}{\pgfqpoint{3.696000in}{3.696000in}}%
\pgfusepath{clip}%
\pgfsetrectcap%
\pgfsetroundjoin%
\pgfsetlinewidth{1.505625pt}%
\definecolor{currentstroke}{rgb}{1.000000,0.000000,0.000000}%
\pgfsetstrokecolor{currentstroke}%
\pgfsetdash{}{0pt}%
\pgfpathmoveto{\pgfqpoint{0.989015in}{1.542426in}}%
\pgfpathlineto{\pgfqpoint{1.005140in}{2.219707in}}%
\pgfusepath{stroke}%
\end{pgfscope}%
\begin{pgfscope}%
\pgfpathrectangle{\pgfqpoint{0.100000in}{0.212622in}}{\pgfqpoint{3.696000in}{3.696000in}}%
\pgfusepath{clip}%
\pgfsetrectcap%
\pgfsetroundjoin%
\pgfsetlinewidth{1.505625pt}%
\definecolor{currentstroke}{rgb}{1.000000,0.000000,0.000000}%
\pgfsetstrokecolor{currentstroke}%
\pgfsetdash{}{0pt}%
\pgfpathmoveto{\pgfqpoint{0.990485in}{1.541930in}}%
\pgfpathlineto{\pgfqpoint{1.007000in}{2.219161in}}%
\pgfusepath{stroke}%
\end{pgfscope}%
\begin{pgfscope}%
\pgfpathrectangle{\pgfqpoint{0.100000in}{0.212622in}}{\pgfqpoint{3.696000in}{3.696000in}}%
\pgfusepath{clip}%
\pgfsetrectcap%
\pgfsetroundjoin%
\pgfsetlinewidth{1.505625pt}%
\definecolor{currentstroke}{rgb}{1.000000,0.000000,0.000000}%
\pgfsetstrokecolor{currentstroke}%
\pgfsetdash{}{0pt}%
\pgfpathmoveto{\pgfqpoint{0.992113in}{1.541189in}}%
\pgfpathlineto{\pgfqpoint{1.008861in}{2.218616in}}%
\pgfusepath{stroke}%
\end{pgfscope}%
\begin{pgfscope}%
\pgfpathrectangle{\pgfqpoint{0.100000in}{0.212622in}}{\pgfqpoint{3.696000in}{3.696000in}}%
\pgfusepath{clip}%
\pgfsetrectcap%
\pgfsetroundjoin%
\pgfsetlinewidth{1.505625pt}%
\definecolor{currentstroke}{rgb}{1.000000,0.000000,0.000000}%
\pgfsetstrokecolor{currentstroke}%
\pgfsetdash{}{0pt}%
\pgfpathmoveto{\pgfqpoint{0.994169in}{1.540287in}}%
\pgfpathlineto{\pgfqpoint{1.010721in}{2.218070in}}%
\pgfusepath{stroke}%
\end{pgfscope}%
\begin{pgfscope}%
\pgfpathrectangle{\pgfqpoint{0.100000in}{0.212622in}}{\pgfqpoint{3.696000in}{3.696000in}}%
\pgfusepath{clip}%
\pgfsetrectcap%
\pgfsetroundjoin%
\pgfsetlinewidth{1.505625pt}%
\definecolor{currentstroke}{rgb}{1.000000,0.000000,0.000000}%
\pgfsetstrokecolor{currentstroke}%
\pgfsetdash{}{0pt}%
\pgfpathmoveto{\pgfqpoint{0.997861in}{1.538995in}}%
\pgfpathlineto{\pgfqpoint{1.014442in}{2.216979in}}%
\pgfusepath{stroke}%
\end{pgfscope}%
\begin{pgfscope}%
\pgfpathrectangle{\pgfqpoint{0.100000in}{0.212622in}}{\pgfqpoint{3.696000in}{3.696000in}}%
\pgfusepath{clip}%
\pgfsetrectcap%
\pgfsetroundjoin%
\pgfsetlinewidth{1.505625pt}%
\definecolor{currentstroke}{rgb}{1.000000,0.000000,0.000000}%
\pgfsetstrokecolor{currentstroke}%
\pgfsetdash{}{0pt}%
\pgfpathmoveto{\pgfqpoint{1.001789in}{1.539058in}}%
\pgfpathlineto{\pgfqpoint{1.018164in}{2.215888in}}%
\pgfusepath{stroke}%
\end{pgfscope}%
\begin{pgfscope}%
\pgfpathrectangle{\pgfqpoint{0.100000in}{0.212622in}}{\pgfqpoint{3.696000in}{3.696000in}}%
\pgfusepath{clip}%
\pgfsetrectcap%
\pgfsetroundjoin%
\pgfsetlinewidth{1.505625pt}%
\definecolor{currentstroke}{rgb}{1.000000,0.000000,0.000000}%
\pgfsetstrokecolor{currentstroke}%
\pgfsetdash{}{0pt}%
\pgfpathmoveto{\pgfqpoint{1.009845in}{1.536229in}}%
\pgfpathlineto{\pgfqpoint{1.025611in}{2.213704in}}%
\pgfusepath{stroke}%
\end{pgfscope}%
\begin{pgfscope}%
\pgfpathrectangle{\pgfqpoint{0.100000in}{0.212622in}}{\pgfqpoint{3.696000in}{3.696000in}}%
\pgfusepath{clip}%
\pgfsetrectcap%
\pgfsetroundjoin%
\pgfsetlinewidth{1.505625pt}%
\definecolor{currentstroke}{rgb}{1.000000,0.000000,0.000000}%
\pgfsetstrokecolor{currentstroke}%
\pgfsetdash{}{0pt}%
\pgfpathmoveto{\pgfqpoint{1.019082in}{1.531756in}}%
\pgfpathlineto{\pgfqpoint{1.034923in}{2.210974in}}%
\pgfusepath{stroke}%
\end{pgfscope}%
\begin{pgfscope}%
\pgfpathrectangle{\pgfqpoint{0.100000in}{0.212622in}}{\pgfqpoint{3.696000in}{3.696000in}}%
\pgfusepath{clip}%
\pgfsetrectcap%
\pgfsetroundjoin%
\pgfsetlinewidth{1.505625pt}%
\definecolor{currentstroke}{rgb}{1.000000,0.000000,0.000000}%
\pgfsetstrokecolor{currentstroke}%
\pgfsetdash{}{0pt}%
\pgfpathmoveto{\pgfqpoint{1.024389in}{1.528377in}}%
\pgfpathlineto{\pgfqpoint{1.040512in}{2.209335in}}%
\pgfusepath{stroke}%
\end{pgfscope}%
\begin{pgfscope}%
\pgfpathrectangle{\pgfqpoint{0.100000in}{0.212622in}}{\pgfqpoint{3.696000in}{3.696000in}}%
\pgfusepath{clip}%
\pgfsetrectcap%
\pgfsetroundjoin%
\pgfsetlinewidth{1.505625pt}%
\definecolor{currentstroke}{rgb}{1.000000,0.000000,0.000000}%
\pgfsetstrokecolor{currentstroke}%
\pgfsetdash{}{0pt}%
\pgfpathmoveto{\pgfqpoint{1.030188in}{1.528032in}}%
\pgfpathlineto{\pgfqpoint{1.046103in}{2.207695in}}%
\pgfusepath{stroke}%
\end{pgfscope}%
\begin{pgfscope}%
\pgfpathrectangle{\pgfqpoint{0.100000in}{0.212622in}}{\pgfqpoint{3.696000in}{3.696000in}}%
\pgfusepath{clip}%
\pgfsetrectcap%
\pgfsetroundjoin%
\pgfsetlinewidth{1.505625pt}%
\definecolor{currentstroke}{rgb}{1.000000,0.000000,0.000000}%
\pgfsetstrokecolor{currentstroke}%
\pgfsetdash{}{0pt}%
\pgfpathmoveto{\pgfqpoint{1.038071in}{1.529520in}}%
\pgfpathlineto{\pgfqpoint{1.053560in}{2.205509in}}%
\pgfusepath{stroke}%
\end{pgfscope}%
\begin{pgfscope}%
\pgfpathrectangle{\pgfqpoint{0.100000in}{0.212622in}}{\pgfqpoint{3.696000in}{3.696000in}}%
\pgfusepath{clip}%
\pgfsetrectcap%
\pgfsetroundjoin%
\pgfsetlinewidth{1.505625pt}%
\definecolor{currentstroke}{rgb}{1.000000,0.000000,0.000000}%
\pgfsetstrokecolor{currentstroke}%
\pgfsetdash{}{0pt}%
\pgfpathmoveto{\pgfqpoint{1.046178in}{1.521968in}}%
\pgfpathlineto{\pgfqpoint{1.061020in}{2.203321in}}%
\pgfusepath{stroke}%
\end{pgfscope}%
\begin{pgfscope}%
\pgfpathrectangle{\pgfqpoint{0.100000in}{0.212622in}}{\pgfqpoint{3.696000in}{3.696000in}}%
\pgfusepath{clip}%
\pgfsetrectcap%
\pgfsetroundjoin%
\pgfsetlinewidth{1.505625pt}%
\definecolor{currentstroke}{rgb}{1.000000,0.000000,0.000000}%
\pgfsetstrokecolor{currentstroke}%
\pgfsetdash{}{0pt}%
\pgfpathmoveto{\pgfqpoint{1.050537in}{1.519237in}}%
\pgfpathlineto{\pgfqpoint{1.066616in}{2.201680in}}%
\pgfusepath{stroke}%
\end{pgfscope}%
\begin{pgfscope}%
\pgfpathrectangle{\pgfqpoint{0.100000in}{0.212622in}}{\pgfqpoint{3.696000in}{3.696000in}}%
\pgfusepath{clip}%
\pgfsetrectcap%
\pgfsetroundjoin%
\pgfsetlinewidth{1.505625pt}%
\definecolor{currentstroke}{rgb}{1.000000,0.000000,0.000000}%
\pgfsetstrokecolor{currentstroke}%
\pgfsetdash{}{0pt}%
\pgfpathmoveto{\pgfqpoint{1.056920in}{1.515424in}}%
\pgfpathlineto{\pgfqpoint{1.072215in}{2.200039in}}%
\pgfusepath{stroke}%
\end{pgfscope}%
\begin{pgfscope}%
\pgfpathrectangle{\pgfqpoint{0.100000in}{0.212622in}}{\pgfqpoint{3.696000in}{3.696000in}}%
\pgfusepath{clip}%
\pgfsetrectcap%
\pgfsetroundjoin%
\pgfsetlinewidth{1.505625pt}%
\definecolor{currentstroke}{rgb}{1.000000,0.000000,0.000000}%
\pgfsetstrokecolor{currentstroke}%
\pgfsetdash{}{0pt}%
\pgfpathmoveto{\pgfqpoint{1.064850in}{1.514287in}}%
\pgfpathlineto{\pgfqpoint{1.079682in}{2.197849in}}%
\pgfusepath{stroke}%
\end{pgfscope}%
\begin{pgfscope}%
\pgfpathrectangle{\pgfqpoint{0.100000in}{0.212622in}}{\pgfqpoint{3.696000in}{3.696000in}}%
\pgfusepath{clip}%
\pgfsetrectcap%
\pgfsetroundjoin%
\pgfsetlinewidth{1.505625pt}%
\definecolor{currentstroke}{rgb}{1.000000,0.000000,0.000000}%
\pgfsetstrokecolor{currentstroke}%
\pgfsetdash{}{0pt}%
\pgfpathmoveto{\pgfqpoint{1.073673in}{1.510404in}}%
\pgfpathlineto{\pgfqpoint{1.089019in}{2.195111in}}%
\pgfusepath{stroke}%
\end{pgfscope}%
\begin{pgfscope}%
\pgfpathrectangle{\pgfqpoint{0.100000in}{0.212622in}}{\pgfqpoint{3.696000in}{3.696000in}}%
\pgfusepath{clip}%
\pgfsetrectcap%
\pgfsetroundjoin%
\pgfsetlinewidth{1.505625pt}%
\definecolor{currentstroke}{rgb}{1.000000,0.000000,0.000000}%
\pgfsetstrokecolor{currentstroke}%
\pgfsetdash{}{0pt}%
\pgfpathmoveto{\pgfqpoint{1.083182in}{1.505113in}}%
\pgfpathlineto{\pgfqpoint{1.100231in}{2.191824in}}%
\pgfusepath{stroke}%
\end{pgfscope}%
\begin{pgfscope}%
\pgfpathrectangle{\pgfqpoint{0.100000in}{0.212622in}}{\pgfqpoint{3.696000in}{3.696000in}}%
\pgfusepath{clip}%
\pgfsetrectcap%
\pgfsetroundjoin%
\pgfsetlinewidth{1.505625pt}%
\definecolor{currentstroke}{rgb}{1.000000,0.000000,0.000000}%
\pgfsetstrokecolor{currentstroke}%
\pgfsetdash{}{0pt}%
\pgfpathmoveto{\pgfqpoint{1.088466in}{1.501886in}}%
\pgfpathlineto{\pgfqpoint{1.103969in}{2.190727in}}%
\pgfusepath{stroke}%
\end{pgfscope}%
\begin{pgfscope}%
\pgfpathrectangle{\pgfqpoint{0.100000in}{0.212622in}}{\pgfqpoint{3.696000in}{3.696000in}}%
\pgfusepath{clip}%
\pgfsetrectcap%
\pgfsetroundjoin%
\pgfsetlinewidth{1.505625pt}%
\definecolor{currentstroke}{rgb}{1.000000,0.000000,0.000000}%
\pgfsetstrokecolor{currentstroke}%
\pgfsetdash{}{0pt}%
\pgfpathmoveto{\pgfqpoint{1.095494in}{1.498836in}}%
\pgfpathlineto{\pgfqpoint{1.111448in}{2.188534in}}%
\pgfusepath{stroke}%
\end{pgfscope}%
\begin{pgfscope}%
\pgfpathrectangle{\pgfqpoint{0.100000in}{0.212622in}}{\pgfqpoint{3.696000in}{3.696000in}}%
\pgfusepath{clip}%
\pgfsetrectcap%
\pgfsetroundjoin%
\pgfsetlinewidth{1.505625pt}%
\definecolor{currentstroke}{rgb}{1.000000,0.000000,0.000000}%
\pgfsetstrokecolor{currentstroke}%
\pgfsetdash{}{0pt}%
\pgfpathmoveto{\pgfqpoint{1.103521in}{1.499058in}}%
\pgfpathlineto{\pgfqpoint{1.118930in}{2.186340in}}%
\pgfusepath{stroke}%
\end{pgfscope}%
\begin{pgfscope}%
\pgfpathrectangle{\pgfqpoint{0.100000in}{0.212622in}}{\pgfqpoint{3.696000in}{3.696000in}}%
\pgfusepath{clip}%
\pgfsetrectcap%
\pgfsetroundjoin%
\pgfsetlinewidth{1.505625pt}%
\definecolor{currentstroke}{rgb}{1.000000,0.000000,0.000000}%
\pgfsetstrokecolor{currentstroke}%
\pgfsetdash{}{0pt}%
\pgfpathmoveto{\pgfqpoint{1.113837in}{1.497154in}}%
\pgfpathlineto{\pgfqpoint{1.125561in}{2.148320in}}%
\pgfusepath{stroke}%
\end{pgfscope}%
\begin{pgfscope}%
\pgfpathrectangle{\pgfqpoint{0.100000in}{0.212622in}}{\pgfqpoint{3.696000in}{3.696000in}}%
\pgfusepath{clip}%
\pgfsetrectcap%
\pgfsetroundjoin%
\pgfsetlinewidth{1.505625pt}%
\definecolor{currentstroke}{rgb}{1.000000,0.000000,0.000000}%
\pgfsetstrokecolor{currentstroke}%
\pgfsetdash{}{0pt}%
\pgfpathmoveto{\pgfqpoint{1.123749in}{1.493376in}}%
\pgfpathlineto{\pgfqpoint{1.125561in}{2.148320in}}%
\pgfusepath{stroke}%
\end{pgfscope}%
\begin{pgfscope}%
\pgfpathrectangle{\pgfqpoint{0.100000in}{0.212622in}}{\pgfqpoint{3.696000in}{3.696000in}}%
\pgfusepath{clip}%
\pgfsetrectcap%
\pgfsetroundjoin%
\pgfsetlinewidth{1.505625pt}%
\definecolor{currentstroke}{rgb}{1.000000,0.000000,0.000000}%
\pgfsetstrokecolor{currentstroke}%
\pgfsetdash{}{0pt}%
\pgfpathmoveto{\pgfqpoint{1.133899in}{1.486615in}}%
\pgfpathlineto{\pgfqpoint{1.125561in}{2.148320in}}%
\pgfusepath{stroke}%
\end{pgfscope}%
\begin{pgfscope}%
\pgfpathrectangle{\pgfqpoint{0.100000in}{0.212622in}}{\pgfqpoint{3.696000in}{3.696000in}}%
\pgfusepath{clip}%
\pgfsetrectcap%
\pgfsetroundjoin%
\pgfsetlinewidth{1.505625pt}%
\definecolor{currentstroke}{rgb}{1.000000,0.000000,0.000000}%
\pgfsetstrokecolor{currentstroke}%
\pgfsetdash{}{0pt}%
\pgfpathmoveto{\pgfqpoint{1.139602in}{1.484684in}}%
\pgfpathlineto{\pgfqpoint{1.125561in}{2.148320in}}%
\pgfusepath{stroke}%
\end{pgfscope}%
\begin{pgfscope}%
\pgfpathrectangle{\pgfqpoint{0.100000in}{0.212622in}}{\pgfqpoint{3.696000in}{3.696000in}}%
\pgfusepath{clip}%
\pgfsetrectcap%
\pgfsetroundjoin%
\pgfsetlinewidth{1.505625pt}%
\definecolor{currentstroke}{rgb}{1.000000,0.000000,0.000000}%
\pgfsetstrokecolor{currentstroke}%
\pgfsetdash{}{0pt}%
\pgfpathmoveto{\pgfqpoint{1.146078in}{1.485022in}}%
\pgfpathlineto{\pgfqpoint{1.127199in}{2.149647in}}%
\pgfusepath{stroke}%
\end{pgfscope}%
\begin{pgfscope}%
\pgfpathrectangle{\pgfqpoint{0.100000in}{0.212622in}}{\pgfqpoint{3.696000in}{3.696000in}}%
\pgfusepath{clip}%
\pgfsetrectcap%
\pgfsetroundjoin%
\pgfsetlinewidth{1.505625pt}%
\definecolor{currentstroke}{rgb}{1.000000,0.000000,0.000000}%
\pgfsetstrokecolor{currentstroke}%
\pgfsetdash{}{0pt}%
\pgfpathmoveto{\pgfqpoint{1.152866in}{1.482442in}}%
\pgfpathlineto{\pgfqpoint{1.127199in}{2.149647in}}%
\pgfusepath{stroke}%
\end{pgfscope}%
\begin{pgfscope}%
\pgfpathrectangle{\pgfqpoint{0.100000in}{0.212622in}}{\pgfqpoint{3.696000in}{3.696000in}}%
\pgfusepath{clip}%
\pgfsetrectcap%
\pgfsetroundjoin%
\pgfsetlinewidth{1.505625pt}%
\definecolor{currentstroke}{rgb}{1.000000,0.000000,0.000000}%
\pgfsetstrokecolor{currentstroke}%
\pgfsetdash{}{0pt}%
\pgfpathmoveto{\pgfqpoint{1.159787in}{1.479423in}}%
\pgfpathlineto{\pgfqpoint{1.127199in}{2.149647in}}%
\pgfusepath{stroke}%
\end{pgfscope}%
\begin{pgfscope}%
\pgfpathrectangle{\pgfqpoint{0.100000in}{0.212622in}}{\pgfqpoint{3.696000in}{3.696000in}}%
\pgfusepath{clip}%
\pgfsetrectcap%
\pgfsetroundjoin%
\pgfsetlinewidth{1.505625pt}%
\definecolor{currentstroke}{rgb}{1.000000,0.000000,0.000000}%
\pgfsetstrokecolor{currentstroke}%
\pgfsetdash{}{0pt}%
\pgfpathmoveto{\pgfqpoint{1.168060in}{1.475118in}}%
\pgfpathlineto{\pgfqpoint{1.127199in}{2.149647in}}%
\pgfusepath{stroke}%
\end{pgfscope}%
\begin{pgfscope}%
\pgfpathrectangle{\pgfqpoint{0.100000in}{0.212622in}}{\pgfqpoint{3.696000in}{3.696000in}}%
\pgfusepath{clip}%
\pgfsetrectcap%
\pgfsetroundjoin%
\pgfsetlinewidth{1.505625pt}%
\definecolor{currentstroke}{rgb}{1.000000,0.000000,0.000000}%
\pgfsetstrokecolor{currentstroke}%
\pgfsetdash{}{0pt}%
\pgfpathmoveto{\pgfqpoint{1.176981in}{1.469466in}}%
\pgfpathlineto{\pgfqpoint{1.127199in}{2.149647in}}%
\pgfusepath{stroke}%
\end{pgfscope}%
\begin{pgfscope}%
\pgfpathrectangle{\pgfqpoint{0.100000in}{0.212622in}}{\pgfqpoint{3.696000in}{3.696000in}}%
\pgfusepath{clip}%
\pgfsetrectcap%
\pgfsetroundjoin%
\pgfsetlinewidth{1.505625pt}%
\definecolor{currentstroke}{rgb}{1.000000,0.000000,0.000000}%
\pgfsetstrokecolor{currentstroke}%
\pgfsetdash{}{0pt}%
\pgfpathmoveto{\pgfqpoint{1.185887in}{1.465318in}}%
\pgfpathlineto{\pgfqpoint{1.203300in}{2.161600in}}%
\pgfusepath{stroke}%
\end{pgfscope}%
\begin{pgfscope}%
\pgfpathrectangle{\pgfqpoint{0.100000in}{0.212622in}}{\pgfqpoint{3.696000in}{3.696000in}}%
\pgfusepath{clip}%
\pgfsetrectcap%
\pgfsetroundjoin%
\pgfsetlinewidth{1.505625pt}%
\definecolor{currentstroke}{rgb}{1.000000,0.000000,0.000000}%
\pgfsetstrokecolor{currentstroke}%
\pgfsetdash{}{0pt}%
\pgfpathmoveto{\pgfqpoint{1.190298in}{1.461892in}}%
\pgfpathlineto{\pgfqpoint{1.210818in}{2.159396in}}%
\pgfusepath{stroke}%
\end{pgfscope}%
\begin{pgfscope}%
\pgfpathrectangle{\pgfqpoint{0.100000in}{0.212622in}}{\pgfqpoint{3.696000in}{3.696000in}}%
\pgfusepath{clip}%
\pgfsetrectcap%
\pgfsetroundjoin%
\pgfsetlinewidth{1.505625pt}%
\definecolor{currentstroke}{rgb}{1.000000,0.000000,0.000000}%
\pgfsetstrokecolor{currentstroke}%
\pgfsetdash{}{0pt}%
\pgfpathmoveto{\pgfqpoint{1.195314in}{1.458877in}}%
\pgfpathlineto{\pgfqpoint{1.210818in}{2.159396in}}%
\pgfusepath{stroke}%
\end{pgfscope}%
\begin{pgfscope}%
\pgfpathrectangle{\pgfqpoint{0.100000in}{0.212622in}}{\pgfqpoint{3.696000in}{3.696000in}}%
\pgfusepath{clip}%
\pgfsetrectcap%
\pgfsetroundjoin%
\pgfsetlinewidth{1.505625pt}%
\definecolor{currentstroke}{rgb}{1.000000,0.000000,0.000000}%
\pgfsetstrokecolor{currentstroke}%
\pgfsetdash{}{0pt}%
\pgfpathmoveto{\pgfqpoint{1.198433in}{1.456820in}}%
\pgfpathlineto{\pgfqpoint{1.218338in}{2.157191in}}%
\pgfusepath{stroke}%
\end{pgfscope}%
\begin{pgfscope}%
\pgfpathrectangle{\pgfqpoint{0.100000in}{0.212622in}}{\pgfqpoint{3.696000in}{3.696000in}}%
\pgfusepath{clip}%
\pgfsetrectcap%
\pgfsetroundjoin%
\pgfsetlinewidth{1.505625pt}%
\definecolor{currentstroke}{rgb}{1.000000,0.000000,0.000000}%
\pgfsetstrokecolor{currentstroke}%
\pgfsetdash{}{0pt}%
\pgfpathmoveto{\pgfqpoint{1.200019in}{1.455450in}}%
\pgfpathlineto{\pgfqpoint{1.218338in}{2.157191in}}%
\pgfusepath{stroke}%
\end{pgfscope}%
\begin{pgfscope}%
\pgfpathrectangle{\pgfqpoint{0.100000in}{0.212622in}}{\pgfqpoint{3.696000in}{3.696000in}}%
\pgfusepath{clip}%
\pgfsetrectcap%
\pgfsetroundjoin%
\pgfsetlinewidth{1.505625pt}%
\definecolor{currentstroke}{rgb}{1.000000,0.000000,0.000000}%
\pgfsetstrokecolor{currentstroke}%
\pgfsetdash{}{0pt}%
\pgfpathmoveto{\pgfqpoint{1.200958in}{1.455024in}}%
\pgfpathlineto{\pgfqpoint{1.218338in}{2.157191in}}%
\pgfusepath{stroke}%
\end{pgfscope}%
\begin{pgfscope}%
\pgfpathrectangle{\pgfqpoint{0.100000in}{0.212622in}}{\pgfqpoint{3.696000in}{3.696000in}}%
\pgfusepath{clip}%
\pgfsetrectcap%
\pgfsetroundjoin%
\pgfsetlinewidth{1.505625pt}%
\definecolor{currentstroke}{rgb}{1.000000,0.000000,0.000000}%
\pgfsetstrokecolor{currentstroke}%
\pgfsetdash{}{0pt}%
\pgfpathmoveto{\pgfqpoint{1.201420in}{1.454600in}}%
\pgfpathlineto{\pgfqpoint{1.218338in}{2.157191in}}%
\pgfusepath{stroke}%
\end{pgfscope}%
\begin{pgfscope}%
\pgfpathrectangle{\pgfqpoint{0.100000in}{0.212622in}}{\pgfqpoint{3.696000in}{3.696000in}}%
\pgfusepath{clip}%
\pgfsetrectcap%
\pgfsetroundjoin%
\pgfsetlinewidth{1.505625pt}%
\definecolor{currentstroke}{rgb}{1.000000,0.000000,0.000000}%
\pgfsetstrokecolor{currentstroke}%
\pgfsetdash{}{0pt}%
\pgfpathmoveto{\pgfqpoint{1.202255in}{1.453798in}}%
\pgfpathlineto{\pgfqpoint{1.218338in}{2.157191in}}%
\pgfusepath{stroke}%
\end{pgfscope}%
\begin{pgfscope}%
\pgfpathrectangle{\pgfqpoint{0.100000in}{0.212622in}}{\pgfqpoint{3.696000in}{3.696000in}}%
\pgfusepath{clip}%
\pgfsetrectcap%
\pgfsetroundjoin%
\pgfsetlinewidth{1.505625pt}%
\definecolor{currentstroke}{rgb}{1.000000,0.000000,0.000000}%
\pgfsetstrokecolor{currentstroke}%
\pgfsetdash{}{0pt}%
\pgfpathmoveto{\pgfqpoint{1.203265in}{1.452536in}}%
\pgfpathlineto{\pgfqpoint{1.225861in}{2.154985in}}%
\pgfusepath{stroke}%
\end{pgfscope}%
\begin{pgfscope}%
\pgfpathrectangle{\pgfqpoint{0.100000in}{0.212622in}}{\pgfqpoint{3.696000in}{3.696000in}}%
\pgfusepath{clip}%
\pgfsetrectcap%
\pgfsetroundjoin%
\pgfsetlinewidth{1.505625pt}%
\definecolor{currentstroke}{rgb}{1.000000,0.000000,0.000000}%
\pgfsetstrokecolor{currentstroke}%
\pgfsetdash{}{0pt}%
\pgfpathmoveto{\pgfqpoint{1.204916in}{1.450309in}}%
\pgfpathlineto{\pgfqpoint{1.225861in}{2.154985in}}%
\pgfusepath{stroke}%
\end{pgfscope}%
\begin{pgfscope}%
\pgfpathrectangle{\pgfqpoint{0.100000in}{0.212622in}}{\pgfqpoint{3.696000in}{3.696000in}}%
\pgfusepath{clip}%
\pgfsetrectcap%
\pgfsetroundjoin%
\pgfsetlinewidth{1.505625pt}%
\definecolor{currentstroke}{rgb}{1.000000,0.000000,0.000000}%
\pgfsetstrokecolor{currentstroke}%
\pgfsetdash{}{0pt}%
\pgfpathmoveto{\pgfqpoint{1.208552in}{1.447069in}}%
\pgfpathlineto{\pgfqpoint{1.225861in}{2.154985in}}%
\pgfusepath{stroke}%
\end{pgfscope}%
\begin{pgfscope}%
\pgfpathrectangle{\pgfqpoint{0.100000in}{0.212622in}}{\pgfqpoint{3.696000in}{3.696000in}}%
\pgfusepath{clip}%
\pgfsetrectcap%
\pgfsetroundjoin%
\pgfsetlinewidth{1.505625pt}%
\definecolor{currentstroke}{rgb}{1.000000,0.000000,0.000000}%
\pgfsetstrokecolor{currentstroke}%
\pgfsetdash{}{0pt}%
\pgfpathmoveto{\pgfqpoint{1.213173in}{1.443279in}}%
\pgfpathlineto{\pgfqpoint{1.233387in}{2.152778in}}%
\pgfusepath{stroke}%
\end{pgfscope}%
\begin{pgfscope}%
\pgfpathrectangle{\pgfqpoint{0.100000in}{0.212622in}}{\pgfqpoint{3.696000in}{3.696000in}}%
\pgfusepath{clip}%
\pgfsetrectcap%
\pgfsetroundjoin%
\pgfsetlinewidth{1.505625pt}%
\definecolor{currentstroke}{rgb}{1.000000,0.000000,0.000000}%
\pgfsetstrokecolor{currentstroke}%
\pgfsetdash{}{0pt}%
\pgfpathmoveto{\pgfqpoint{1.217989in}{1.439111in}}%
\pgfpathlineto{\pgfqpoint{1.240915in}{2.150571in}}%
\pgfusepath{stroke}%
\end{pgfscope}%
\begin{pgfscope}%
\pgfpathrectangle{\pgfqpoint{0.100000in}{0.212622in}}{\pgfqpoint{3.696000in}{3.696000in}}%
\pgfusepath{clip}%
\pgfsetrectcap%
\pgfsetroundjoin%
\pgfsetlinewidth{1.505625pt}%
\definecolor{currentstroke}{rgb}{1.000000,0.000000,0.000000}%
\pgfsetstrokecolor{currentstroke}%
\pgfsetdash{}{0pt}%
\pgfpathmoveto{\pgfqpoint{1.220824in}{1.436633in}}%
\pgfpathlineto{\pgfqpoint{1.240915in}{2.150571in}}%
\pgfusepath{stroke}%
\end{pgfscope}%
\begin{pgfscope}%
\pgfpathrectangle{\pgfqpoint{0.100000in}{0.212622in}}{\pgfqpoint{3.696000in}{3.696000in}}%
\pgfusepath{clip}%
\pgfsetrectcap%
\pgfsetroundjoin%
\pgfsetlinewidth{1.505625pt}%
\definecolor{currentstroke}{rgb}{1.000000,0.000000,0.000000}%
\pgfsetstrokecolor{currentstroke}%
\pgfsetdash{}{0pt}%
\pgfpathmoveto{\pgfqpoint{1.224165in}{1.432573in}}%
\pgfpathlineto{\pgfqpoint{1.248447in}{2.148362in}}%
\pgfusepath{stroke}%
\end{pgfscope}%
\begin{pgfscope}%
\pgfpathrectangle{\pgfqpoint{0.100000in}{0.212622in}}{\pgfqpoint{3.696000in}{3.696000in}}%
\pgfusepath{clip}%
\pgfsetrectcap%
\pgfsetroundjoin%
\pgfsetlinewidth{1.505625pt}%
\definecolor{currentstroke}{rgb}{1.000000,0.000000,0.000000}%
\pgfsetstrokecolor{currentstroke}%
\pgfsetdash{}{0pt}%
\pgfpathmoveto{\pgfqpoint{1.229245in}{1.427003in}}%
\pgfpathlineto{\pgfqpoint{1.255982in}{2.146153in}}%
\pgfusepath{stroke}%
\end{pgfscope}%
\begin{pgfscope}%
\pgfpathrectangle{\pgfqpoint{0.100000in}{0.212622in}}{\pgfqpoint{3.696000in}{3.696000in}}%
\pgfusepath{clip}%
\pgfsetrectcap%
\pgfsetroundjoin%
\pgfsetlinewidth{1.505625pt}%
\definecolor{currentstroke}{rgb}{1.000000,0.000000,0.000000}%
\pgfsetstrokecolor{currentstroke}%
\pgfsetdash{}{0pt}%
\pgfpathmoveto{\pgfqpoint{1.235989in}{1.421254in}}%
\pgfpathlineto{\pgfqpoint{1.263519in}{2.143943in}}%
\pgfusepath{stroke}%
\end{pgfscope}%
\begin{pgfscope}%
\pgfpathrectangle{\pgfqpoint{0.100000in}{0.212622in}}{\pgfqpoint{3.696000in}{3.696000in}}%
\pgfusepath{clip}%
\pgfsetrectcap%
\pgfsetroundjoin%
\pgfsetlinewidth{1.505625pt}%
\definecolor{currentstroke}{rgb}{1.000000,0.000000,0.000000}%
\pgfsetstrokecolor{currentstroke}%
\pgfsetdash{}{0pt}%
\pgfpathmoveto{\pgfqpoint{1.243720in}{1.414685in}}%
\pgfpathlineto{\pgfqpoint{1.271059in}{2.141731in}}%
\pgfusepath{stroke}%
\end{pgfscope}%
\begin{pgfscope}%
\pgfpathrectangle{\pgfqpoint{0.100000in}{0.212622in}}{\pgfqpoint{3.696000in}{3.696000in}}%
\pgfusepath{clip}%
\pgfsetrectcap%
\pgfsetroundjoin%
\pgfsetlinewidth{1.505625pt}%
\definecolor{currentstroke}{rgb}{1.000000,0.000000,0.000000}%
\pgfsetstrokecolor{currentstroke}%
\pgfsetdash{}{0pt}%
\pgfpathmoveto{\pgfqpoint{1.252037in}{1.409106in}}%
\pgfpathlineto{\pgfqpoint{1.278603in}{2.139520in}}%
\pgfusepath{stroke}%
\end{pgfscope}%
\begin{pgfscope}%
\pgfpathrectangle{\pgfqpoint{0.100000in}{0.212622in}}{\pgfqpoint{3.696000in}{3.696000in}}%
\pgfusepath{clip}%
\pgfsetrectcap%
\pgfsetroundjoin%
\pgfsetlinewidth{1.505625pt}%
\definecolor{currentstroke}{rgb}{1.000000,0.000000,0.000000}%
\pgfsetstrokecolor{currentstroke}%
\pgfsetdash{}{0pt}%
\pgfpathmoveto{\pgfqpoint{1.257170in}{1.408828in}}%
\pgfpathlineto{\pgfqpoint{1.286149in}{2.137307in}}%
\pgfusepath{stroke}%
\end{pgfscope}%
\begin{pgfscope}%
\pgfpathrectangle{\pgfqpoint{0.100000in}{0.212622in}}{\pgfqpoint{3.696000in}{3.696000in}}%
\pgfusepath{clip}%
\pgfsetrectcap%
\pgfsetroundjoin%
\pgfsetlinewidth{1.505625pt}%
\definecolor{currentstroke}{rgb}{1.000000,0.000000,0.000000}%
\pgfsetstrokecolor{currentstroke}%
\pgfsetdash{}{0pt}%
\pgfpathmoveto{\pgfqpoint{1.262873in}{1.405952in}}%
\pgfpathlineto{\pgfqpoint{1.293698in}{2.135093in}}%
\pgfusepath{stroke}%
\end{pgfscope}%
\begin{pgfscope}%
\pgfpathrectangle{\pgfqpoint{0.100000in}{0.212622in}}{\pgfqpoint{3.696000in}{3.696000in}}%
\pgfusepath{clip}%
\pgfsetrectcap%
\pgfsetroundjoin%
\pgfsetlinewidth{1.505625pt}%
\definecolor{currentstroke}{rgb}{1.000000,0.000000,0.000000}%
\pgfsetstrokecolor{currentstroke}%
\pgfsetdash{}{0pt}%
\pgfpathmoveto{\pgfqpoint{1.269404in}{1.401720in}}%
\pgfpathlineto{\pgfqpoint{1.301250in}{2.132879in}}%
\pgfusepath{stroke}%
\end{pgfscope}%
\begin{pgfscope}%
\pgfpathrectangle{\pgfqpoint{0.100000in}{0.212622in}}{\pgfqpoint{3.696000in}{3.696000in}}%
\pgfusepath{clip}%
\pgfsetrectcap%
\pgfsetroundjoin%
\pgfsetlinewidth{1.505625pt}%
\definecolor{currentstroke}{rgb}{1.000000,0.000000,0.000000}%
\pgfsetstrokecolor{currentstroke}%
\pgfsetdash{}{0pt}%
\pgfpathmoveto{\pgfqpoint{1.278243in}{1.396576in}}%
\pgfpathlineto{\pgfqpoint{1.308805in}{2.130663in}}%
\pgfusepath{stroke}%
\end{pgfscope}%
\begin{pgfscope}%
\pgfpathrectangle{\pgfqpoint{0.100000in}{0.212622in}}{\pgfqpoint{3.696000in}{3.696000in}}%
\pgfusepath{clip}%
\pgfsetrectcap%
\pgfsetroundjoin%
\pgfsetlinewidth{1.505625pt}%
\definecolor{currentstroke}{rgb}{1.000000,0.000000,0.000000}%
\pgfsetstrokecolor{currentstroke}%
\pgfsetdash{}{0pt}%
\pgfpathmoveto{\pgfqpoint{1.288430in}{1.392574in}}%
\pgfpathlineto{\pgfqpoint{1.316362in}{2.128447in}}%
\pgfusepath{stroke}%
\end{pgfscope}%
\begin{pgfscope}%
\pgfpathrectangle{\pgfqpoint{0.100000in}{0.212622in}}{\pgfqpoint{3.696000in}{3.696000in}}%
\pgfusepath{clip}%
\pgfsetrectcap%
\pgfsetroundjoin%
\pgfsetlinewidth{1.505625pt}%
\definecolor{currentstroke}{rgb}{1.000000,0.000000,0.000000}%
\pgfsetstrokecolor{currentstroke}%
\pgfsetdash{}{0pt}%
\pgfpathmoveto{\pgfqpoint{1.298780in}{1.384891in}}%
\pgfpathlineto{\pgfqpoint{1.331487in}{2.124012in}}%
\pgfusepath{stroke}%
\end{pgfscope}%
\begin{pgfscope}%
\pgfpathrectangle{\pgfqpoint{0.100000in}{0.212622in}}{\pgfqpoint{3.696000in}{3.696000in}}%
\pgfusepath{clip}%
\pgfsetrectcap%
\pgfsetroundjoin%
\pgfsetlinewidth{1.505625pt}%
\definecolor{currentstroke}{rgb}{1.000000,0.000000,0.000000}%
\pgfsetstrokecolor{currentstroke}%
\pgfsetdash{}{0pt}%
\pgfpathmoveto{\pgfqpoint{1.309065in}{1.376892in}}%
\pgfpathlineto{\pgfqpoint{1.339053in}{2.121794in}}%
\pgfusepath{stroke}%
\end{pgfscope}%
\begin{pgfscope}%
\pgfpathrectangle{\pgfqpoint{0.100000in}{0.212622in}}{\pgfqpoint{3.696000in}{3.696000in}}%
\pgfusepath{clip}%
\pgfsetrectcap%
\pgfsetroundjoin%
\pgfsetlinewidth{1.505625pt}%
\definecolor{currentstroke}{rgb}{1.000000,0.000000,0.000000}%
\pgfsetstrokecolor{currentstroke}%
\pgfsetdash{}{0pt}%
\pgfpathmoveto{\pgfqpoint{1.314829in}{1.373404in}}%
\pgfpathlineto{\pgfqpoint{1.346623in}{2.119574in}}%
\pgfusepath{stroke}%
\end{pgfscope}%
\begin{pgfscope}%
\pgfpathrectangle{\pgfqpoint{0.100000in}{0.212622in}}{\pgfqpoint{3.696000in}{3.696000in}}%
\pgfusepath{clip}%
\pgfsetrectcap%
\pgfsetroundjoin%
\pgfsetlinewidth{1.505625pt}%
\definecolor{currentstroke}{rgb}{1.000000,0.000000,0.000000}%
\pgfsetstrokecolor{currentstroke}%
\pgfsetdash{}{0pt}%
\pgfpathmoveto{\pgfqpoint{1.322032in}{1.371190in}}%
\pgfpathlineto{\pgfqpoint{1.354195in}{2.117354in}}%
\pgfusepath{stroke}%
\end{pgfscope}%
\begin{pgfscope}%
\pgfpathrectangle{\pgfqpoint{0.100000in}{0.212622in}}{\pgfqpoint{3.696000in}{3.696000in}}%
\pgfusepath{clip}%
\pgfsetrectcap%
\pgfsetroundjoin%
\pgfsetlinewidth{1.505625pt}%
\definecolor{currentstroke}{rgb}{1.000000,0.000000,0.000000}%
\pgfsetstrokecolor{currentstroke}%
\pgfsetdash{}{0pt}%
\pgfpathmoveto{\pgfqpoint{1.329636in}{1.369919in}}%
\pgfpathlineto{\pgfqpoint{1.361770in}{2.115132in}}%
\pgfusepath{stroke}%
\end{pgfscope}%
\begin{pgfscope}%
\pgfpathrectangle{\pgfqpoint{0.100000in}{0.212622in}}{\pgfqpoint{3.696000in}{3.696000in}}%
\pgfusepath{clip}%
\pgfsetrectcap%
\pgfsetroundjoin%
\pgfsetlinewidth{1.505625pt}%
\definecolor{currentstroke}{rgb}{1.000000,0.000000,0.000000}%
\pgfsetstrokecolor{currentstroke}%
\pgfsetdash{}{0pt}%
\pgfpathmoveto{\pgfqpoint{1.338734in}{1.366324in}}%
\pgfpathlineto{\pgfqpoint{1.369348in}{2.112910in}}%
\pgfusepath{stroke}%
\end{pgfscope}%
\begin{pgfscope}%
\pgfpathrectangle{\pgfqpoint{0.100000in}{0.212622in}}{\pgfqpoint{3.696000in}{3.696000in}}%
\pgfusepath{clip}%
\pgfsetrectcap%
\pgfsetroundjoin%
\pgfsetlinewidth{1.505625pt}%
\definecolor{currentstroke}{rgb}{1.000000,0.000000,0.000000}%
\pgfsetstrokecolor{currentstroke}%
\pgfsetdash{}{0pt}%
\pgfpathmoveto{\pgfqpoint{1.347926in}{1.357696in}}%
\pgfpathlineto{\pgfqpoint{1.376929in}{2.110687in}}%
\pgfusepath{stroke}%
\end{pgfscope}%
\begin{pgfscope}%
\pgfpathrectangle{\pgfqpoint{0.100000in}{0.212622in}}{\pgfqpoint{3.696000in}{3.696000in}}%
\pgfusepath{clip}%
\pgfsetrectcap%
\pgfsetroundjoin%
\pgfsetlinewidth{1.505625pt}%
\definecolor{currentstroke}{rgb}{1.000000,0.000000,0.000000}%
\pgfsetstrokecolor{currentstroke}%
\pgfsetdash{}{0pt}%
\pgfpathmoveto{\pgfqpoint{1.358508in}{1.351418in}}%
\pgfpathlineto{\pgfqpoint{1.392100in}{2.106239in}}%
\pgfusepath{stroke}%
\end{pgfscope}%
\begin{pgfscope}%
\pgfpathrectangle{\pgfqpoint{0.100000in}{0.212622in}}{\pgfqpoint{3.696000in}{3.696000in}}%
\pgfusepath{clip}%
\pgfsetrectcap%
\pgfsetroundjoin%
\pgfsetlinewidth{1.505625pt}%
\definecolor{currentstroke}{rgb}{1.000000,0.000000,0.000000}%
\pgfsetstrokecolor{currentstroke}%
\pgfsetdash{}{0pt}%
\pgfpathmoveto{\pgfqpoint{1.363901in}{1.347307in}}%
\pgfpathlineto{\pgfqpoint{1.399690in}{2.104013in}}%
\pgfusepath{stroke}%
\end{pgfscope}%
\begin{pgfscope}%
\pgfpathrectangle{\pgfqpoint{0.100000in}{0.212622in}}{\pgfqpoint{3.696000in}{3.696000in}}%
\pgfusepath{clip}%
\pgfsetrectcap%
\pgfsetroundjoin%
\pgfsetlinewidth{1.505625pt}%
\definecolor{currentstroke}{rgb}{1.000000,0.000000,0.000000}%
\pgfsetstrokecolor{currentstroke}%
\pgfsetdash{}{0pt}%
\pgfpathmoveto{\pgfqpoint{1.371418in}{1.347201in}}%
\pgfpathlineto{\pgfqpoint{1.407283in}{2.101787in}}%
\pgfusepath{stroke}%
\end{pgfscope}%
\begin{pgfscope}%
\pgfpathrectangle{\pgfqpoint{0.100000in}{0.212622in}}{\pgfqpoint{3.696000in}{3.696000in}}%
\pgfusepath{clip}%
\pgfsetrectcap%
\pgfsetroundjoin%
\pgfsetlinewidth{1.505625pt}%
\definecolor{currentstroke}{rgb}{1.000000,0.000000,0.000000}%
\pgfsetstrokecolor{currentstroke}%
\pgfsetdash{}{0pt}%
\pgfpathmoveto{\pgfqpoint{1.381113in}{1.340607in}}%
\pgfpathlineto{\pgfqpoint{1.414879in}{2.099559in}}%
\pgfusepath{stroke}%
\end{pgfscope}%
\begin{pgfscope}%
\pgfpathrectangle{\pgfqpoint{0.100000in}{0.212622in}}{\pgfqpoint{3.696000in}{3.696000in}}%
\pgfusepath{clip}%
\pgfsetrectcap%
\pgfsetroundjoin%
\pgfsetlinewidth{1.505625pt}%
\definecolor{currentstroke}{rgb}{1.000000,0.000000,0.000000}%
\pgfsetstrokecolor{currentstroke}%
\pgfsetdash{}{0pt}%
\pgfpathmoveto{\pgfqpoint{1.390825in}{1.332832in}}%
\pgfpathlineto{\pgfqpoint{1.422478in}{2.097331in}}%
\pgfusepath{stroke}%
\end{pgfscope}%
\begin{pgfscope}%
\pgfpathrectangle{\pgfqpoint{0.100000in}{0.212622in}}{\pgfqpoint{3.696000in}{3.696000in}}%
\pgfusepath{clip}%
\pgfsetrectcap%
\pgfsetroundjoin%
\pgfsetlinewidth{1.505625pt}%
\definecolor{currentstroke}{rgb}{1.000000,0.000000,0.000000}%
\pgfsetstrokecolor{currentstroke}%
\pgfsetdash{}{0pt}%
\pgfpathmoveto{\pgfqpoint{1.396302in}{1.327575in}}%
\pgfpathlineto{\pgfqpoint{1.430079in}{2.095102in}}%
\pgfusepath{stroke}%
\end{pgfscope}%
\begin{pgfscope}%
\pgfpathrectangle{\pgfqpoint{0.100000in}{0.212622in}}{\pgfqpoint{3.696000in}{3.696000in}}%
\pgfusepath{clip}%
\pgfsetrectcap%
\pgfsetroundjoin%
\pgfsetlinewidth{1.505625pt}%
\definecolor{currentstroke}{rgb}{1.000000,0.000000,0.000000}%
\pgfsetstrokecolor{currentstroke}%
\pgfsetdash{}{0pt}%
\pgfpathmoveto{\pgfqpoint{1.402858in}{1.323376in}}%
\pgfpathlineto{\pgfqpoint{1.437684in}{2.092872in}}%
\pgfusepath{stroke}%
\end{pgfscope}%
\begin{pgfscope}%
\pgfpathrectangle{\pgfqpoint{0.100000in}{0.212622in}}{\pgfqpoint{3.696000in}{3.696000in}}%
\pgfusepath{clip}%
\pgfsetrectcap%
\pgfsetroundjoin%
\pgfsetlinewidth{1.505625pt}%
\definecolor{currentstroke}{rgb}{1.000000,0.000000,0.000000}%
\pgfsetstrokecolor{currentstroke}%
\pgfsetdash{}{0pt}%
\pgfpathmoveto{\pgfqpoint{1.410529in}{1.318109in}}%
\pgfpathlineto{\pgfqpoint{1.445291in}{2.090642in}}%
\pgfusepath{stroke}%
\end{pgfscope}%
\begin{pgfscope}%
\pgfpathrectangle{\pgfqpoint{0.100000in}{0.212622in}}{\pgfqpoint{3.696000in}{3.696000in}}%
\pgfusepath{clip}%
\pgfsetrectcap%
\pgfsetroundjoin%
\pgfsetlinewidth{1.505625pt}%
\definecolor{currentstroke}{rgb}{1.000000,0.000000,0.000000}%
\pgfsetstrokecolor{currentstroke}%
\pgfsetdash{}{0pt}%
\pgfpathmoveto{\pgfqpoint{1.419227in}{1.315486in}}%
\pgfpathlineto{\pgfqpoint{1.452902in}{2.088410in}}%
\pgfusepath{stroke}%
\end{pgfscope}%
\begin{pgfscope}%
\pgfpathrectangle{\pgfqpoint{0.100000in}{0.212622in}}{\pgfqpoint{3.696000in}{3.696000in}}%
\pgfusepath{clip}%
\pgfsetrectcap%
\pgfsetroundjoin%
\pgfsetlinewidth{1.505625pt}%
\definecolor{currentstroke}{rgb}{1.000000,0.000000,0.000000}%
\pgfsetstrokecolor{currentstroke}%
\pgfsetdash{}{0pt}%
\pgfpathmoveto{\pgfqpoint{1.424048in}{1.311662in}}%
\pgfpathlineto{\pgfqpoint{1.460515in}{2.086177in}}%
\pgfusepath{stroke}%
\end{pgfscope}%
\begin{pgfscope}%
\pgfpathrectangle{\pgfqpoint{0.100000in}{0.212622in}}{\pgfqpoint{3.696000in}{3.696000in}}%
\pgfusepath{clip}%
\pgfsetrectcap%
\pgfsetroundjoin%
\pgfsetlinewidth{1.505625pt}%
\definecolor{currentstroke}{rgb}{1.000000,0.000000,0.000000}%
\pgfsetstrokecolor{currentstroke}%
\pgfsetdash{}{0pt}%
\pgfpathmoveto{\pgfqpoint{1.429508in}{1.305872in}}%
\pgfpathlineto{\pgfqpoint{1.468131in}{2.083944in}}%
\pgfusepath{stroke}%
\end{pgfscope}%
\begin{pgfscope}%
\pgfpathrectangle{\pgfqpoint{0.100000in}{0.212622in}}{\pgfqpoint{3.696000in}{3.696000in}}%
\pgfusepath{clip}%
\pgfsetrectcap%
\pgfsetroundjoin%
\pgfsetlinewidth{1.505625pt}%
\definecolor{currentstroke}{rgb}{1.000000,0.000000,0.000000}%
\pgfsetstrokecolor{currentstroke}%
\pgfsetdash{}{0pt}%
\pgfpathmoveto{\pgfqpoint{1.437260in}{1.300072in}}%
\pgfpathlineto{\pgfqpoint{1.475750in}{2.081710in}}%
\pgfusepath{stroke}%
\end{pgfscope}%
\begin{pgfscope}%
\pgfpathrectangle{\pgfqpoint{0.100000in}{0.212622in}}{\pgfqpoint{3.696000in}{3.696000in}}%
\pgfusepath{clip}%
\pgfsetrectcap%
\pgfsetroundjoin%
\pgfsetlinewidth{1.505625pt}%
\definecolor{currentstroke}{rgb}{1.000000,0.000000,0.000000}%
\pgfsetstrokecolor{currentstroke}%
\pgfsetdash{}{0pt}%
\pgfpathmoveto{\pgfqpoint{1.448662in}{1.295384in}}%
\pgfpathlineto{\pgfqpoint{1.483373in}{2.079475in}}%
\pgfusepath{stroke}%
\end{pgfscope}%
\begin{pgfscope}%
\pgfpathrectangle{\pgfqpoint{0.100000in}{0.212622in}}{\pgfqpoint{3.696000in}{3.696000in}}%
\pgfusepath{clip}%
\pgfsetrectcap%
\pgfsetroundjoin%
\pgfsetlinewidth{1.505625pt}%
\definecolor{currentstroke}{rgb}{1.000000,0.000000,0.000000}%
\pgfsetstrokecolor{currentstroke}%
\pgfsetdash{}{0pt}%
\pgfpathmoveto{\pgfqpoint{1.461059in}{1.293280in}}%
\pgfpathlineto{\pgfqpoint{1.498626in}{2.075002in}}%
\pgfusepath{stroke}%
\end{pgfscope}%
\begin{pgfscope}%
\pgfpathrectangle{\pgfqpoint{0.100000in}{0.212622in}}{\pgfqpoint{3.696000in}{3.696000in}}%
\pgfusepath{clip}%
\pgfsetrectcap%
\pgfsetroundjoin%
\pgfsetlinewidth{1.505625pt}%
\definecolor{currentstroke}{rgb}{1.000000,0.000000,0.000000}%
\pgfsetstrokecolor{currentstroke}%
\pgfsetdash{}{0pt}%
\pgfpathmoveto{\pgfqpoint{1.467271in}{1.287176in}}%
\pgfpathlineto{\pgfqpoint{1.506257in}{2.072765in}}%
\pgfusepath{stroke}%
\end{pgfscope}%
\begin{pgfscope}%
\pgfpathrectangle{\pgfqpoint{0.100000in}{0.212622in}}{\pgfqpoint{3.696000in}{3.696000in}}%
\pgfusepath{clip}%
\pgfsetrectcap%
\pgfsetroundjoin%
\pgfsetlinewidth{1.505625pt}%
\definecolor{currentstroke}{rgb}{1.000000,0.000000,0.000000}%
\pgfsetstrokecolor{currentstroke}%
\pgfsetdash{}{0pt}%
\pgfpathmoveto{\pgfqpoint{1.473796in}{1.279636in}}%
\pgfpathlineto{\pgfqpoint{1.513891in}{2.070526in}}%
\pgfusepath{stroke}%
\end{pgfscope}%
\begin{pgfscope}%
\pgfpathrectangle{\pgfqpoint{0.100000in}{0.212622in}}{\pgfqpoint{3.696000in}{3.696000in}}%
\pgfusepath{clip}%
\pgfsetrectcap%
\pgfsetroundjoin%
\pgfsetlinewidth{1.505625pt}%
\definecolor{currentstroke}{rgb}{1.000000,0.000000,0.000000}%
\pgfsetstrokecolor{currentstroke}%
\pgfsetdash{}{0pt}%
\pgfpathmoveto{\pgfqpoint{1.483739in}{1.273054in}}%
\pgfpathlineto{\pgfqpoint{1.521528in}{2.068287in}}%
\pgfusepath{stroke}%
\end{pgfscope}%
\begin{pgfscope}%
\pgfpathrectangle{\pgfqpoint{0.100000in}{0.212622in}}{\pgfqpoint{3.696000in}{3.696000in}}%
\pgfusepath{clip}%
\pgfsetrectcap%
\pgfsetroundjoin%
\pgfsetlinewidth{1.505625pt}%
\definecolor{currentstroke}{rgb}{1.000000,0.000000,0.000000}%
\pgfsetstrokecolor{currentstroke}%
\pgfsetdash{}{0pt}%
\pgfpathmoveto{\pgfqpoint{1.496776in}{1.270883in}}%
\pgfpathlineto{\pgfqpoint{1.536810in}{2.063805in}}%
\pgfusepath{stroke}%
\end{pgfscope}%
\begin{pgfscope}%
\pgfpathrectangle{\pgfqpoint{0.100000in}{0.212622in}}{\pgfqpoint{3.696000in}{3.696000in}}%
\pgfusepath{clip}%
\pgfsetrectcap%
\pgfsetroundjoin%
\pgfsetlinewidth{1.505625pt}%
\definecolor{currentstroke}{rgb}{1.000000,0.000000,0.000000}%
\pgfsetstrokecolor{currentstroke}%
\pgfsetdash{}{0pt}%
\pgfpathmoveto{\pgfqpoint{1.510722in}{1.269137in}}%
\pgfpathlineto{\pgfqpoint{1.552105in}{2.059321in}}%
\pgfusepath{stroke}%
\end{pgfscope}%
\begin{pgfscope}%
\pgfpathrectangle{\pgfqpoint{0.100000in}{0.212622in}}{\pgfqpoint{3.696000in}{3.696000in}}%
\pgfusepath{clip}%
\pgfsetrectcap%
\pgfsetroundjoin%
\pgfsetlinewidth{1.505625pt}%
\definecolor{currentstroke}{rgb}{1.000000,0.000000,0.000000}%
\pgfsetstrokecolor{currentstroke}%
\pgfsetdash{}{0pt}%
\pgfpathmoveto{\pgfqpoint{1.525428in}{1.260219in}}%
\pgfpathlineto{\pgfqpoint{1.567411in}{2.054832in}}%
\pgfusepath{stroke}%
\end{pgfscope}%
\begin{pgfscope}%
\pgfpathrectangle{\pgfqpoint{0.100000in}{0.212622in}}{\pgfqpoint{3.696000in}{3.696000in}}%
\pgfusepath{clip}%
\pgfsetrectcap%
\pgfsetroundjoin%
\pgfsetlinewidth{1.505625pt}%
\definecolor{currentstroke}{rgb}{1.000000,0.000000,0.000000}%
\pgfsetstrokecolor{currentstroke}%
\pgfsetdash{}{0pt}%
\pgfpathmoveto{\pgfqpoint{1.540774in}{1.248807in}}%
\pgfpathlineto{\pgfqpoint{1.582730in}{2.050340in}}%
\pgfusepath{stroke}%
\end{pgfscope}%
\begin{pgfscope}%
\pgfpathrectangle{\pgfqpoint{0.100000in}{0.212622in}}{\pgfqpoint{3.696000in}{3.696000in}}%
\pgfusepath{clip}%
\pgfsetrectcap%
\pgfsetroundjoin%
\pgfsetlinewidth{1.505625pt}%
\definecolor{currentstroke}{rgb}{1.000000,0.000000,0.000000}%
\pgfsetstrokecolor{currentstroke}%
\pgfsetdash{}{0pt}%
\pgfpathmoveto{\pgfqpoint{1.557556in}{1.237299in}}%
\pgfpathlineto{\pgfqpoint{1.598060in}{2.045845in}}%
\pgfusepath{stroke}%
\end{pgfscope}%
\begin{pgfscope}%
\pgfpathrectangle{\pgfqpoint{0.100000in}{0.212622in}}{\pgfqpoint{3.696000in}{3.696000in}}%
\pgfusepath{clip}%
\pgfsetrectcap%
\pgfsetroundjoin%
\pgfsetlinewidth{1.505625pt}%
\definecolor{currentstroke}{rgb}{1.000000,0.000000,0.000000}%
\pgfsetstrokecolor{currentstroke}%
\pgfsetdash{}{0pt}%
\pgfpathmoveto{\pgfqpoint{1.573677in}{1.230911in}}%
\pgfpathlineto{\pgfqpoint{1.613402in}{2.041346in}}%
\pgfusepath{stroke}%
\end{pgfscope}%
\begin{pgfscope}%
\pgfpathrectangle{\pgfqpoint{0.100000in}{0.212622in}}{\pgfqpoint{3.696000in}{3.696000in}}%
\pgfusepath{clip}%
\pgfsetrectcap%
\pgfsetroundjoin%
\pgfsetlinewidth{1.505625pt}%
\definecolor{currentstroke}{rgb}{1.000000,0.000000,0.000000}%
\pgfsetstrokecolor{currentstroke}%
\pgfsetdash{}{0pt}%
\pgfpathmoveto{\pgfqpoint{1.582836in}{1.228206in}}%
\pgfpathlineto{\pgfqpoint{1.621077in}{2.039096in}}%
\pgfusepath{stroke}%
\end{pgfscope}%
\begin{pgfscope}%
\pgfpathrectangle{\pgfqpoint{0.100000in}{0.212622in}}{\pgfqpoint{3.696000in}{3.696000in}}%
\pgfusepath{clip}%
\pgfsetrectcap%
\pgfsetroundjoin%
\pgfsetlinewidth{1.505625pt}%
\definecolor{currentstroke}{rgb}{1.000000,0.000000,0.000000}%
\pgfsetstrokecolor{currentstroke}%
\pgfsetdash{}{0pt}%
\pgfpathmoveto{\pgfqpoint{1.594548in}{1.225607in}}%
\pgfpathlineto{\pgfqpoint{1.628756in}{2.036844in}}%
\pgfusepath{stroke}%
\end{pgfscope}%
\begin{pgfscope}%
\pgfpathrectangle{\pgfqpoint{0.100000in}{0.212622in}}{\pgfqpoint{3.696000in}{3.696000in}}%
\pgfusepath{clip}%
\pgfsetrectcap%
\pgfsetroundjoin%
\pgfsetlinewidth{1.505625pt}%
\definecolor{currentstroke}{rgb}{1.000000,0.000000,0.000000}%
\pgfsetstrokecolor{currentstroke}%
\pgfsetdash{}{0pt}%
\pgfpathmoveto{\pgfqpoint{1.607183in}{1.217529in}}%
\pgfpathlineto{\pgfqpoint{1.644122in}{2.032339in}}%
\pgfusepath{stroke}%
\end{pgfscope}%
\begin{pgfscope}%
\pgfpathrectangle{\pgfqpoint{0.100000in}{0.212622in}}{\pgfqpoint{3.696000in}{3.696000in}}%
\pgfusepath{clip}%
\pgfsetrectcap%
\pgfsetroundjoin%
\pgfsetlinewidth{1.505625pt}%
\definecolor{currentstroke}{rgb}{1.000000,0.000000,0.000000}%
\pgfsetstrokecolor{currentstroke}%
\pgfsetdash{}{0pt}%
\pgfpathmoveto{\pgfqpoint{1.618869in}{1.205538in}}%
\pgfpathlineto{\pgfqpoint{1.659500in}{2.027829in}}%
\pgfusepath{stroke}%
\end{pgfscope}%
\begin{pgfscope}%
\pgfpathrectangle{\pgfqpoint{0.100000in}{0.212622in}}{\pgfqpoint{3.696000in}{3.696000in}}%
\pgfusepath{clip}%
\pgfsetrectcap%
\pgfsetroundjoin%
\pgfsetlinewidth{1.505625pt}%
\definecolor{currentstroke}{rgb}{1.000000,0.000000,0.000000}%
\pgfsetstrokecolor{currentstroke}%
\pgfsetdash{}{0pt}%
\pgfpathmoveto{\pgfqpoint{1.631725in}{1.193155in}}%
\pgfpathlineto{\pgfqpoint{1.667193in}{2.025573in}}%
\pgfusepath{stroke}%
\end{pgfscope}%
\begin{pgfscope}%
\pgfpathrectangle{\pgfqpoint{0.100000in}{0.212622in}}{\pgfqpoint{3.696000in}{3.696000in}}%
\pgfusepath{clip}%
\pgfsetrectcap%
\pgfsetroundjoin%
\pgfsetlinewidth{1.505625pt}%
\definecolor{currentstroke}{rgb}{1.000000,0.000000,0.000000}%
\pgfsetstrokecolor{currentstroke}%
\pgfsetdash{}{0pt}%
\pgfpathmoveto{\pgfqpoint{1.639078in}{1.189062in}}%
\pgfpathlineto{\pgfqpoint{1.674889in}{2.023317in}}%
\pgfusepath{stroke}%
\end{pgfscope}%
\begin{pgfscope}%
\pgfpathrectangle{\pgfqpoint{0.100000in}{0.212622in}}{\pgfqpoint{3.696000in}{3.696000in}}%
\pgfusepath{clip}%
\pgfsetrectcap%
\pgfsetroundjoin%
\pgfsetlinewidth{1.505625pt}%
\definecolor{currentstroke}{rgb}{1.000000,0.000000,0.000000}%
\pgfsetstrokecolor{currentstroke}%
\pgfsetdash{}{0pt}%
\pgfpathmoveto{\pgfqpoint{1.647483in}{1.187376in}}%
\pgfpathlineto{\pgfqpoint{1.682589in}{2.021059in}}%
\pgfusepath{stroke}%
\end{pgfscope}%
\begin{pgfscope}%
\pgfpathrectangle{\pgfqpoint{0.100000in}{0.212622in}}{\pgfqpoint{3.696000in}{3.696000in}}%
\pgfusepath{clip}%
\pgfsetrectcap%
\pgfsetroundjoin%
\pgfsetlinewidth{1.505625pt}%
\definecolor{currentstroke}{rgb}{1.000000,0.000000,0.000000}%
\pgfsetstrokecolor{currentstroke}%
\pgfsetdash{}{0pt}%
\pgfpathmoveto{\pgfqpoint{1.657897in}{1.183862in}}%
\pgfpathlineto{\pgfqpoint{1.697996in}{2.016541in}}%
\pgfusepath{stroke}%
\end{pgfscope}%
\begin{pgfscope}%
\pgfpathrectangle{\pgfqpoint{0.100000in}{0.212622in}}{\pgfqpoint{3.696000in}{3.696000in}}%
\pgfusepath{clip}%
\pgfsetrectcap%
\pgfsetroundjoin%
\pgfsetlinewidth{1.505625pt}%
\definecolor{currentstroke}{rgb}{1.000000,0.000000,0.000000}%
\pgfsetstrokecolor{currentstroke}%
\pgfsetdash{}{0pt}%
\pgfpathmoveto{\pgfqpoint{1.668592in}{1.174140in}}%
\pgfpathlineto{\pgfqpoint{1.705705in}{2.014281in}}%
\pgfusepath{stroke}%
\end{pgfscope}%
\begin{pgfscope}%
\pgfpathrectangle{\pgfqpoint{0.100000in}{0.212622in}}{\pgfqpoint{3.696000in}{3.696000in}}%
\pgfusepath{clip}%
\pgfsetrectcap%
\pgfsetroundjoin%
\pgfsetlinewidth{1.505625pt}%
\definecolor{currentstroke}{rgb}{1.000000,0.000000,0.000000}%
\pgfsetstrokecolor{currentstroke}%
\pgfsetdash{}{0pt}%
\pgfpathmoveto{\pgfqpoint{1.680427in}{1.164307in}}%
\pgfpathlineto{\pgfqpoint{1.721130in}{2.009757in}}%
\pgfusepath{stroke}%
\end{pgfscope}%
\begin{pgfscope}%
\pgfpathrectangle{\pgfqpoint{0.100000in}{0.212622in}}{\pgfqpoint{3.696000in}{3.696000in}}%
\pgfusepath{clip}%
\pgfsetrectcap%
\pgfsetroundjoin%
\pgfsetlinewidth{1.505625pt}%
\definecolor{currentstroke}{rgb}{1.000000,0.000000,0.000000}%
\pgfsetstrokecolor{currentstroke}%
\pgfsetdash{}{0pt}%
\pgfpathmoveto{\pgfqpoint{1.693426in}{1.154417in}}%
\pgfpathlineto{\pgfqpoint{1.736568in}{2.005230in}}%
\pgfusepath{stroke}%
\end{pgfscope}%
\begin{pgfscope}%
\pgfpathrectangle{\pgfqpoint{0.100000in}{0.212622in}}{\pgfqpoint{3.696000in}{3.696000in}}%
\pgfusepath{clip}%
\pgfsetrectcap%
\pgfsetroundjoin%
\pgfsetlinewidth{1.505625pt}%
\definecolor{currentstroke}{rgb}{1.000000,0.000000,0.000000}%
\pgfsetstrokecolor{currentstroke}%
\pgfsetdash{}{0pt}%
\pgfpathmoveto{\pgfqpoint{1.710332in}{1.156843in}}%
\pgfpathlineto{\pgfqpoint{1.744291in}{2.002966in}}%
\pgfusepath{stroke}%
\end{pgfscope}%
\begin{pgfscope}%
\pgfpathrectangle{\pgfqpoint{0.100000in}{0.212622in}}{\pgfqpoint{3.696000in}{3.696000in}}%
\pgfusepath{clip}%
\pgfsetrectcap%
\pgfsetroundjoin%
\pgfsetlinewidth{1.505625pt}%
\definecolor{currentstroke}{rgb}{1.000000,0.000000,0.000000}%
\pgfsetstrokecolor{currentstroke}%
\pgfsetdash{}{0pt}%
\pgfpathmoveto{\pgfqpoint{1.726822in}{1.153621in}}%
\pgfpathlineto{\pgfqpoint{1.767480in}{1.996166in}}%
\pgfusepath{stroke}%
\end{pgfscope}%
\begin{pgfscope}%
\pgfpathrectangle{\pgfqpoint{0.100000in}{0.212622in}}{\pgfqpoint{3.696000in}{3.696000in}}%
\pgfusepath{clip}%
\pgfsetrectcap%
\pgfsetroundjoin%
\pgfsetlinewidth{1.505625pt}%
\definecolor{currentstroke}{rgb}{1.000000,0.000000,0.000000}%
\pgfsetstrokecolor{currentstroke}%
\pgfsetdash{}{0pt}%
\pgfpathmoveto{\pgfqpoint{1.744824in}{1.140628in}}%
\pgfpathlineto{\pgfqpoint{1.782953in}{1.991629in}}%
\pgfusepath{stroke}%
\end{pgfscope}%
\begin{pgfscope}%
\pgfpathrectangle{\pgfqpoint{0.100000in}{0.212622in}}{\pgfqpoint{3.696000in}{3.696000in}}%
\pgfusepath{clip}%
\pgfsetrectcap%
\pgfsetroundjoin%
\pgfsetlinewidth{1.505625pt}%
\definecolor{currentstroke}{rgb}{1.000000,0.000000,0.000000}%
\pgfsetstrokecolor{currentstroke}%
\pgfsetdash{}{0pt}%
\pgfpathmoveto{\pgfqpoint{1.765631in}{1.127310in}}%
\pgfpathlineto{\pgfqpoint{1.806187in}{1.984816in}}%
\pgfusepath{stroke}%
\end{pgfscope}%
\begin{pgfscope}%
\pgfpathrectangle{\pgfqpoint{0.100000in}{0.212622in}}{\pgfqpoint{3.696000in}{3.696000in}}%
\pgfusepath{clip}%
\pgfsetrectcap%
\pgfsetroundjoin%
\pgfsetlinewidth{1.505625pt}%
\definecolor{currentstroke}{rgb}{1.000000,0.000000,0.000000}%
\pgfsetstrokecolor{currentstroke}%
\pgfsetdash{}{0pt}%
\pgfpathmoveto{\pgfqpoint{1.786427in}{1.111694in}}%
\pgfpathlineto{\pgfqpoint{1.821691in}{1.980270in}}%
\pgfusepath{stroke}%
\end{pgfscope}%
\begin{pgfscope}%
\pgfpathrectangle{\pgfqpoint{0.100000in}{0.212622in}}{\pgfqpoint{3.696000in}{3.696000in}}%
\pgfusepath{clip}%
\pgfsetrectcap%
\pgfsetroundjoin%
\pgfsetlinewidth{1.505625pt}%
\definecolor{currentstroke}{rgb}{1.000000,0.000000,0.000000}%
\pgfsetstrokecolor{currentstroke}%
\pgfsetdash{}{0pt}%
\pgfpathmoveto{\pgfqpoint{1.798295in}{1.108618in}}%
\pgfpathlineto{\pgfqpoint{1.837207in}{1.975720in}}%
\pgfusepath{stroke}%
\end{pgfscope}%
\begin{pgfscope}%
\pgfpathrectangle{\pgfqpoint{0.100000in}{0.212622in}}{\pgfqpoint{3.696000in}{3.696000in}}%
\pgfusepath{clip}%
\pgfsetrectcap%
\pgfsetroundjoin%
\pgfsetlinewidth{1.505625pt}%
\definecolor{currentstroke}{rgb}{1.000000,0.000000,0.000000}%
\pgfsetstrokecolor{currentstroke}%
\pgfsetdash{}{0pt}%
\pgfpathmoveto{\pgfqpoint{1.805102in}{1.109094in}}%
\pgfpathlineto{\pgfqpoint{1.844969in}{1.973444in}}%
\pgfusepath{stroke}%
\end{pgfscope}%
\begin{pgfscope}%
\pgfpathrectangle{\pgfqpoint{0.100000in}{0.212622in}}{\pgfqpoint{3.696000in}{3.696000in}}%
\pgfusepath{clip}%
\pgfsetrectcap%
\pgfsetroundjoin%
\pgfsetlinewidth{1.505625pt}%
\definecolor{currentstroke}{rgb}{1.000000,0.000000,0.000000}%
\pgfsetstrokecolor{currentstroke}%
\pgfsetdash{}{0pt}%
\pgfpathmoveto{\pgfqpoint{1.815126in}{1.105125in}}%
\pgfpathlineto{\pgfqpoint{1.852735in}{1.971167in}}%
\pgfusepath{stroke}%
\end{pgfscope}%
\begin{pgfscope}%
\pgfpathrectangle{\pgfqpoint{0.100000in}{0.212622in}}{\pgfqpoint{3.696000in}{3.696000in}}%
\pgfusepath{clip}%
\pgfsetrectcap%
\pgfsetroundjoin%
\pgfsetlinewidth{1.505625pt}%
\definecolor{currentstroke}{rgb}{1.000000,0.000000,0.000000}%
\pgfsetstrokecolor{currentstroke}%
\pgfsetdash{}{0pt}%
\pgfpathmoveto{\pgfqpoint{1.827315in}{1.096806in}}%
\pgfpathlineto{\pgfqpoint{1.860503in}{1.968889in}}%
\pgfusepath{stroke}%
\end{pgfscope}%
\begin{pgfscope}%
\pgfpathrectangle{\pgfqpoint{0.100000in}{0.212622in}}{\pgfqpoint{3.696000in}{3.696000in}}%
\pgfusepath{clip}%
\pgfsetrectcap%
\pgfsetroundjoin%
\pgfsetlinewidth{1.505625pt}%
\definecolor{currentstroke}{rgb}{1.000000,0.000000,0.000000}%
\pgfsetstrokecolor{currentstroke}%
\pgfsetdash{}{0pt}%
\pgfpathmoveto{\pgfqpoint{1.839541in}{1.085952in}}%
\pgfpathlineto{\pgfqpoint{1.876050in}{1.964330in}}%
\pgfusepath{stroke}%
\end{pgfscope}%
\begin{pgfscope}%
\pgfpathrectangle{\pgfqpoint{0.100000in}{0.212622in}}{\pgfqpoint{3.696000in}{3.696000in}}%
\pgfusepath{clip}%
\pgfsetrectcap%
\pgfsetroundjoin%
\pgfsetlinewidth{1.505625pt}%
\definecolor{currentstroke}{rgb}{1.000000,0.000000,0.000000}%
\pgfsetstrokecolor{currentstroke}%
\pgfsetdash{}{0pt}%
\pgfpathmoveto{\pgfqpoint{1.853529in}{1.080354in}}%
\pgfpathlineto{\pgfqpoint{1.891608in}{1.959768in}}%
\pgfusepath{stroke}%
\end{pgfscope}%
\begin{pgfscope}%
\pgfpathrectangle{\pgfqpoint{0.100000in}{0.212622in}}{\pgfqpoint{3.696000in}{3.696000in}}%
\pgfusepath{clip}%
\pgfsetrectcap%
\pgfsetroundjoin%
\pgfsetlinewidth{1.505625pt}%
\definecolor{currentstroke}{rgb}{1.000000,0.000000,0.000000}%
\pgfsetstrokecolor{currentstroke}%
\pgfsetdash{}{0pt}%
\pgfpathmoveto{\pgfqpoint{1.870324in}{1.081172in}}%
\pgfpathlineto{\pgfqpoint{1.907179in}{1.955202in}}%
\pgfusepath{stroke}%
\end{pgfscope}%
\begin{pgfscope}%
\pgfpathrectangle{\pgfqpoint{0.100000in}{0.212622in}}{\pgfqpoint{3.696000in}{3.696000in}}%
\pgfusepath{clip}%
\pgfsetrectcap%
\pgfsetroundjoin%
\pgfsetlinewidth{1.505625pt}%
\definecolor{currentstroke}{rgb}{1.000000,0.000000,0.000000}%
\pgfsetstrokecolor{currentstroke}%
\pgfsetdash{}{0pt}%
\pgfpathmoveto{\pgfqpoint{1.888420in}{1.088688in}}%
\pgfpathlineto{\pgfqpoint{1.914969in}{1.952918in}}%
\pgfusepath{stroke}%
\end{pgfscope}%
\begin{pgfscope}%
\pgfpathrectangle{\pgfqpoint{0.100000in}{0.212622in}}{\pgfqpoint{3.696000in}{3.696000in}}%
\pgfusepath{clip}%
\pgfsetrectcap%
\pgfsetroundjoin%
\pgfsetlinewidth{1.505625pt}%
\definecolor{currentstroke}{rgb}{1.000000,0.000000,0.000000}%
\pgfsetstrokecolor{currentstroke}%
\pgfsetdash{}{0pt}%
\pgfpathmoveto{\pgfqpoint{1.908227in}{1.083936in}}%
\pgfpathlineto{\pgfqpoint{1.914969in}{1.952918in}}%
\pgfusepath{stroke}%
\end{pgfscope}%
\begin{pgfscope}%
\pgfpathrectangle{\pgfqpoint{0.100000in}{0.212622in}}{\pgfqpoint{3.696000in}{3.696000in}}%
\pgfusepath{clip}%
\pgfsetrectcap%
\pgfsetroundjoin%
\pgfsetlinewidth{1.505625pt}%
\definecolor{currentstroke}{rgb}{1.000000,0.000000,0.000000}%
\pgfsetstrokecolor{currentstroke}%
\pgfsetdash{}{0pt}%
\pgfpathmoveto{\pgfqpoint{1.930927in}{1.071407in}}%
\pgfpathlineto{\pgfqpoint{1.914969in}{1.952918in}}%
\pgfusepath{stroke}%
\end{pgfscope}%
\begin{pgfscope}%
\pgfpathrectangle{\pgfqpoint{0.100000in}{0.212622in}}{\pgfqpoint{3.696000in}{3.696000in}}%
\pgfusepath{clip}%
\pgfsetrectcap%
\pgfsetroundjoin%
\pgfsetlinewidth{1.505625pt}%
\definecolor{currentstroke}{rgb}{1.000000,0.000000,0.000000}%
\pgfsetstrokecolor{currentstroke}%
\pgfsetdash{}{0pt}%
\pgfpathmoveto{\pgfqpoint{1.943288in}{1.063867in}}%
\pgfpathlineto{\pgfqpoint{1.914969in}{1.952918in}}%
\pgfusepath{stroke}%
\end{pgfscope}%
\begin{pgfscope}%
\pgfpathrectangle{\pgfqpoint{0.100000in}{0.212622in}}{\pgfqpoint{3.696000in}{3.696000in}}%
\pgfusepath{clip}%
\pgfsetrectcap%
\pgfsetroundjoin%
\pgfsetlinewidth{1.505625pt}%
\definecolor{currentstroke}{rgb}{1.000000,0.000000,0.000000}%
\pgfsetstrokecolor{currentstroke}%
\pgfsetdash{}{0pt}%
\pgfpathmoveto{\pgfqpoint{1.956467in}{1.054471in}}%
\pgfpathlineto{\pgfqpoint{1.914969in}{1.952918in}}%
\pgfusepath{stroke}%
\end{pgfscope}%
\begin{pgfscope}%
\pgfpathrectangle{\pgfqpoint{0.100000in}{0.212622in}}{\pgfqpoint{3.696000in}{3.696000in}}%
\pgfusepath{clip}%
\pgfsetrectcap%
\pgfsetroundjoin%
\pgfsetlinewidth{1.505625pt}%
\definecolor{currentstroke}{rgb}{1.000000,0.000000,0.000000}%
\pgfsetstrokecolor{currentstroke}%
\pgfsetdash{}{0pt}%
\pgfpathmoveto{\pgfqpoint{1.971095in}{1.053949in}}%
\pgfpathlineto{\pgfqpoint{1.914969in}{1.952918in}}%
\pgfusepath{stroke}%
\end{pgfscope}%
\begin{pgfscope}%
\pgfpathrectangle{\pgfqpoint{0.100000in}{0.212622in}}{\pgfqpoint{3.696000in}{3.696000in}}%
\pgfusepath{clip}%
\pgfsetrectcap%
\pgfsetroundjoin%
\pgfsetlinewidth{1.505625pt}%
\definecolor{currentstroke}{rgb}{1.000000,0.000000,0.000000}%
\pgfsetstrokecolor{currentstroke}%
\pgfsetdash{}{0pt}%
\pgfpathmoveto{\pgfqpoint{1.978879in}{1.054356in}}%
\pgfpathlineto{\pgfqpoint{1.914969in}{1.952918in}}%
\pgfusepath{stroke}%
\end{pgfscope}%
\begin{pgfscope}%
\pgfpathrectangle{\pgfqpoint{0.100000in}{0.212622in}}{\pgfqpoint{3.696000in}{3.696000in}}%
\pgfusepath{clip}%
\pgfsetrectcap%
\pgfsetroundjoin%
\pgfsetlinewidth{1.505625pt}%
\definecolor{currentstroke}{rgb}{1.000000,0.000000,0.000000}%
\pgfsetstrokecolor{currentstroke}%
\pgfsetdash{}{0pt}%
\pgfpathmoveto{\pgfqpoint{1.988565in}{1.056429in}}%
\pgfpathlineto{\pgfqpoint{1.914969in}{1.952918in}}%
\pgfusepath{stroke}%
\end{pgfscope}%
\begin{pgfscope}%
\pgfpathrectangle{\pgfqpoint{0.100000in}{0.212622in}}{\pgfqpoint{3.696000in}{3.696000in}}%
\pgfusepath{clip}%
\pgfsetrectcap%
\pgfsetroundjoin%
\pgfsetlinewidth{1.505625pt}%
\definecolor{currentstroke}{rgb}{1.000000,0.000000,0.000000}%
\pgfsetstrokecolor{currentstroke}%
\pgfsetdash{}{0pt}%
\pgfpathmoveto{\pgfqpoint{1.999741in}{1.050713in}}%
\pgfpathlineto{\pgfqpoint{1.914969in}{1.952918in}}%
\pgfusepath{stroke}%
\end{pgfscope}%
\begin{pgfscope}%
\pgfpathrectangle{\pgfqpoint{0.100000in}{0.212622in}}{\pgfqpoint{3.696000in}{3.696000in}}%
\pgfusepath{clip}%
\pgfsetrectcap%
\pgfsetroundjoin%
\pgfsetlinewidth{1.505625pt}%
\definecolor{currentstroke}{rgb}{1.000000,0.000000,0.000000}%
\pgfsetstrokecolor{currentstroke}%
\pgfsetdash{}{0pt}%
\pgfpathmoveto{\pgfqpoint{2.010006in}{1.041597in}}%
\pgfpathlineto{\pgfqpoint{1.914969in}{1.952918in}}%
\pgfusepath{stroke}%
\end{pgfscope}%
\begin{pgfscope}%
\pgfpathrectangle{\pgfqpoint{0.100000in}{0.212622in}}{\pgfqpoint{3.696000in}{3.696000in}}%
\pgfusepath{clip}%
\pgfsetrectcap%
\pgfsetroundjoin%
\pgfsetlinewidth{1.505625pt}%
\definecolor{currentstroke}{rgb}{1.000000,0.000000,0.000000}%
\pgfsetstrokecolor{currentstroke}%
\pgfsetdash{}{0pt}%
\pgfpathmoveto{\pgfqpoint{2.015681in}{1.036222in}}%
\pgfpathlineto{\pgfqpoint{1.914969in}{1.952918in}}%
\pgfusepath{stroke}%
\end{pgfscope}%
\begin{pgfscope}%
\pgfpathrectangle{\pgfqpoint{0.100000in}{0.212622in}}{\pgfqpoint{3.696000in}{3.696000in}}%
\pgfusepath{clip}%
\pgfsetrectcap%
\pgfsetroundjoin%
\pgfsetlinewidth{1.505625pt}%
\definecolor{currentstroke}{rgb}{1.000000,0.000000,0.000000}%
\pgfsetstrokecolor{currentstroke}%
\pgfsetdash{}{0pt}%
\pgfpathmoveto{\pgfqpoint{2.023671in}{1.037199in}}%
\pgfpathlineto{\pgfqpoint{1.914969in}{1.952918in}}%
\pgfusepath{stroke}%
\end{pgfscope}%
\begin{pgfscope}%
\pgfpathrectangle{\pgfqpoint{0.100000in}{0.212622in}}{\pgfqpoint{3.696000in}{3.696000in}}%
\pgfusepath{clip}%
\pgfsetrectcap%
\pgfsetroundjoin%
\pgfsetlinewidth{1.505625pt}%
\definecolor{currentstroke}{rgb}{1.000000,0.000000,0.000000}%
\pgfsetstrokecolor{currentstroke}%
\pgfsetdash{}{0pt}%
\pgfpathmoveto{\pgfqpoint{2.028096in}{1.038124in}}%
\pgfpathlineto{\pgfqpoint{1.914969in}{1.952918in}}%
\pgfusepath{stroke}%
\end{pgfscope}%
\begin{pgfscope}%
\pgfpathrectangle{\pgfqpoint{0.100000in}{0.212622in}}{\pgfqpoint{3.696000in}{3.696000in}}%
\pgfusepath{clip}%
\pgfsetrectcap%
\pgfsetroundjoin%
\pgfsetlinewidth{1.505625pt}%
\definecolor{currentstroke}{rgb}{1.000000,0.000000,0.000000}%
\pgfsetstrokecolor{currentstroke}%
\pgfsetdash{}{0pt}%
\pgfpathmoveto{\pgfqpoint{2.035672in}{1.036614in}}%
\pgfpathlineto{\pgfqpoint{1.914969in}{1.952918in}}%
\pgfusepath{stroke}%
\end{pgfscope}%
\begin{pgfscope}%
\pgfpathrectangle{\pgfqpoint{0.100000in}{0.212622in}}{\pgfqpoint{3.696000in}{3.696000in}}%
\pgfusepath{clip}%
\pgfsetrectcap%
\pgfsetroundjoin%
\pgfsetlinewidth{1.505625pt}%
\definecolor{currentstroke}{rgb}{1.000000,0.000000,0.000000}%
\pgfsetstrokecolor{currentstroke}%
\pgfsetdash{}{0pt}%
\pgfpathmoveto{\pgfqpoint{2.045013in}{1.030228in}}%
\pgfpathlineto{\pgfqpoint{1.914969in}{1.952918in}}%
\pgfusepath{stroke}%
\end{pgfscope}%
\begin{pgfscope}%
\pgfpathrectangle{\pgfqpoint{0.100000in}{0.212622in}}{\pgfqpoint{3.696000in}{3.696000in}}%
\pgfusepath{clip}%
\pgfsetrectcap%
\pgfsetroundjoin%
\pgfsetlinewidth{1.505625pt}%
\definecolor{currentstroke}{rgb}{1.000000,0.000000,0.000000}%
\pgfsetstrokecolor{currentstroke}%
\pgfsetdash{}{0pt}%
\pgfpathmoveto{\pgfqpoint{2.054733in}{1.021519in}}%
\pgfpathlineto{\pgfqpoint{1.914969in}{1.952918in}}%
\pgfusepath{stroke}%
\end{pgfscope}%
\begin{pgfscope}%
\pgfpathrectangle{\pgfqpoint{0.100000in}{0.212622in}}{\pgfqpoint{3.696000in}{3.696000in}}%
\pgfusepath{clip}%
\pgfsetrectcap%
\pgfsetroundjoin%
\pgfsetlinewidth{1.505625pt}%
\definecolor{currentstroke}{rgb}{1.000000,0.000000,0.000000}%
\pgfsetstrokecolor{currentstroke}%
\pgfsetdash{}{0pt}%
\pgfpathmoveto{\pgfqpoint{2.065499in}{1.010631in}}%
\pgfpathlineto{\pgfqpoint{1.914969in}{1.952918in}}%
\pgfusepath{stroke}%
\end{pgfscope}%
\begin{pgfscope}%
\pgfpathrectangle{\pgfqpoint{0.100000in}{0.212622in}}{\pgfqpoint{3.696000in}{3.696000in}}%
\pgfusepath{clip}%
\pgfsetrectcap%
\pgfsetroundjoin%
\pgfsetlinewidth{1.505625pt}%
\definecolor{currentstroke}{rgb}{1.000000,0.000000,0.000000}%
\pgfsetstrokecolor{currentstroke}%
\pgfsetdash{}{0pt}%
\pgfpathmoveto{\pgfqpoint{2.078487in}{1.010207in}}%
\pgfpathlineto{\pgfqpoint{1.914969in}{1.952918in}}%
\pgfusepath{stroke}%
\end{pgfscope}%
\begin{pgfscope}%
\pgfpathrectangle{\pgfqpoint{0.100000in}{0.212622in}}{\pgfqpoint{3.696000in}{3.696000in}}%
\pgfusepath{clip}%
\pgfsetrectcap%
\pgfsetroundjoin%
\pgfsetlinewidth{1.505625pt}%
\definecolor{currentstroke}{rgb}{1.000000,0.000000,0.000000}%
\pgfsetstrokecolor{currentstroke}%
\pgfsetdash{}{0pt}%
\pgfpathmoveto{\pgfqpoint{2.091851in}{1.009672in}}%
\pgfpathlineto{\pgfqpoint{2.281225in}{1.817686in}}%
\pgfusepath{stroke}%
\end{pgfscope}%
\begin{pgfscope}%
\pgfpathrectangle{\pgfqpoint{0.100000in}{0.212622in}}{\pgfqpoint{3.696000in}{3.696000in}}%
\pgfusepath{clip}%
\pgfsetrectcap%
\pgfsetroundjoin%
\pgfsetlinewidth{1.505625pt}%
\definecolor{currentstroke}{rgb}{1.000000,0.000000,0.000000}%
\pgfsetstrokecolor{currentstroke}%
\pgfsetdash{}{0pt}%
\pgfpathmoveto{\pgfqpoint{2.106047in}{1.007821in}}%
\pgfpathlineto{\pgfqpoint{2.281225in}{1.817686in}}%
\pgfusepath{stroke}%
\end{pgfscope}%
\begin{pgfscope}%
\pgfpathrectangle{\pgfqpoint{0.100000in}{0.212622in}}{\pgfqpoint{3.696000in}{3.696000in}}%
\pgfusepath{clip}%
\pgfsetrectcap%
\pgfsetroundjoin%
\pgfsetlinewidth{1.505625pt}%
\definecolor{currentstroke}{rgb}{1.000000,0.000000,0.000000}%
\pgfsetstrokecolor{currentstroke}%
\pgfsetdash{}{0pt}%
\pgfpathmoveto{\pgfqpoint{2.122119in}{0.999218in}}%
\pgfpathlineto{\pgfqpoint{2.281225in}{1.817686in}}%
\pgfusepath{stroke}%
\end{pgfscope}%
\begin{pgfscope}%
\pgfpathrectangle{\pgfqpoint{0.100000in}{0.212622in}}{\pgfqpoint{3.696000in}{3.696000in}}%
\pgfusepath{clip}%
\pgfsetrectcap%
\pgfsetroundjoin%
\pgfsetlinewidth{1.505625pt}%
\definecolor{currentstroke}{rgb}{1.000000,0.000000,0.000000}%
\pgfsetstrokecolor{currentstroke}%
\pgfsetdash{}{0pt}%
\pgfpathmoveto{\pgfqpoint{2.130719in}{0.993245in}}%
\pgfpathlineto{\pgfqpoint{2.281225in}{1.817686in}}%
\pgfusepath{stroke}%
\end{pgfscope}%
\begin{pgfscope}%
\pgfpathrectangle{\pgfqpoint{0.100000in}{0.212622in}}{\pgfqpoint{3.696000in}{3.696000in}}%
\pgfusepath{clip}%
\pgfsetrectcap%
\pgfsetroundjoin%
\pgfsetlinewidth{1.505625pt}%
\definecolor{currentstroke}{rgb}{1.000000,0.000000,0.000000}%
\pgfsetstrokecolor{currentstroke}%
\pgfsetdash{}{0pt}%
\pgfpathmoveto{\pgfqpoint{2.134962in}{0.988552in}}%
\pgfpathlineto{\pgfqpoint{2.281225in}{1.817686in}}%
\pgfusepath{stroke}%
\end{pgfscope}%
\begin{pgfscope}%
\pgfpathrectangle{\pgfqpoint{0.100000in}{0.212622in}}{\pgfqpoint{3.696000in}{3.696000in}}%
\pgfusepath{clip}%
\pgfsetrectcap%
\pgfsetroundjoin%
\pgfsetlinewidth{1.505625pt}%
\definecolor{currentstroke}{rgb}{1.000000,0.000000,0.000000}%
\pgfsetstrokecolor{currentstroke}%
\pgfsetdash{}{0pt}%
\pgfpathmoveto{\pgfqpoint{2.139600in}{0.984793in}}%
\pgfpathlineto{\pgfqpoint{2.281225in}{1.817686in}}%
\pgfusepath{stroke}%
\end{pgfscope}%
\begin{pgfscope}%
\pgfpathrectangle{\pgfqpoint{0.100000in}{0.212622in}}{\pgfqpoint{3.696000in}{3.696000in}}%
\pgfusepath{clip}%
\pgfsetrectcap%
\pgfsetroundjoin%
\pgfsetlinewidth{1.505625pt}%
\definecolor{currentstroke}{rgb}{1.000000,0.000000,0.000000}%
\pgfsetstrokecolor{currentstroke}%
\pgfsetdash{}{0pt}%
\pgfpathmoveto{\pgfqpoint{2.145460in}{0.983713in}}%
\pgfpathlineto{\pgfqpoint{2.281225in}{1.817686in}}%
\pgfusepath{stroke}%
\end{pgfscope}%
\begin{pgfscope}%
\pgfpathrectangle{\pgfqpoint{0.100000in}{0.212622in}}{\pgfqpoint{3.696000in}{3.696000in}}%
\pgfusepath{clip}%
\pgfsetrectcap%
\pgfsetroundjoin%
\pgfsetlinewidth{1.505625pt}%
\definecolor{currentstroke}{rgb}{1.000000,0.000000,0.000000}%
\pgfsetstrokecolor{currentstroke}%
\pgfsetdash{}{0pt}%
\pgfpathmoveto{\pgfqpoint{2.153350in}{0.982074in}}%
\pgfpathlineto{\pgfqpoint{2.281225in}{1.817686in}}%
\pgfusepath{stroke}%
\end{pgfscope}%
\begin{pgfscope}%
\pgfpathrectangle{\pgfqpoint{0.100000in}{0.212622in}}{\pgfqpoint{3.696000in}{3.696000in}}%
\pgfusepath{clip}%
\pgfsetrectcap%
\pgfsetroundjoin%
\pgfsetlinewidth{1.505625pt}%
\definecolor{currentstroke}{rgb}{1.000000,0.000000,0.000000}%
\pgfsetstrokecolor{currentstroke}%
\pgfsetdash{}{0pt}%
\pgfpathmoveto{\pgfqpoint{2.162399in}{0.974919in}}%
\pgfpathlineto{\pgfqpoint{2.281225in}{1.817686in}}%
\pgfusepath{stroke}%
\end{pgfscope}%
\begin{pgfscope}%
\pgfpathrectangle{\pgfqpoint{0.100000in}{0.212622in}}{\pgfqpoint{3.696000in}{3.696000in}}%
\pgfusepath{clip}%
\pgfsetrectcap%
\pgfsetroundjoin%
\pgfsetlinewidth{1.505625pt}%
\definecolor{currentstroke}{rgb}{1.000000,0.000000,0.000000}%
\pgfsetstrokecolor{currentstroke}%
\pgfsetdash{}{0pt}%
\pgfpathmoveto{\pgfqpoint{2.167136in}{0.970272in}}%
\pgfpathlineto{\pgfqpoint{2.281225in}{1.817686in}}%
\pgfusepath{stroke}%
\end{pgfscope}%
\begin{pgfscope}%
\pgfpathrectangle{\pgfqpoint{0.100000in}{0.212622in}}{\pgfqpoint{3.696000in}{3.696000in}}%
\pgfusepath{clip}%
\pgfsetrectcap%
\pgfsetroundjoin%
\pgfsetlinewidth{1.505625pt}%
\definecolor{currentstroke}{rgb}{1.000000,0.000000,0.000000}%
\pgfsetstrokecolor{currentstroke}%
\pgfsetdash{}{0pt}%
\pgfpathmoveto{\pgfqpoint{2.172520in}{0.966152in}}%
\pgfpathlineto{\pgfqpoint{2.281225in}{1.817686in}}%
\pgfusepath{stroke}%
\end{pgfscope}%
\begin{pgfscope}%
\pgfpathrectangle{\pgfqpoint{0.100000in}{0.212622in}}{\pgfqpoint{3.696000in}{3.696000in}}%
\pgfusepath{clip}%
\pgfsetrectcap%
\pgfsetroundjoin%
\pgfsetlinewidth{1.505625pt}%
\definecolor{currentstroke}{rgb}{1.000000,0.000000,0.000000}%
\pgfsetstrokecolor{currentstroke}%
\pgfsetdash{}{0pt}%
\pgfpathmoveto{\pgfqpoint{2.180083in}{0.963466in}}%
\pgfpathlineto{\pgfqpoint{2.281225in}{1.817686in}}%
\pgfusepath{stroke}%
\end{pgfscope}%
\begin{pgfscope}%
\pgfpathrectangle{\pgfqpoint{0.100000in}{0.212622in}}{\pgfqpoint{3.696000in}{3.696000in}}%
\pgfusepath{clip}%
\pgfsetrectcap%
\pgfsetroundjoin%
\pgfsetlinewidth{1.505625pt}%
\definecolor{currentstroke}{rgb}{1.000000,0.000000,0.000000}%
\pgfsetstrokecolor{currentstroke}%
\pgfsetdash{}{0pt}%
\pgfpathmoveto{\pgfqpoint{2.188584in}{0.963627in}}%
\pgfpathlineto{\pgfqpoint{2.292055in}{1.827512in}}%
\pgfusepath{stroke}%
\end{pgfscope}%
\begin{pgfscope}%
\pgfpathrectangle{\pgfqpoint{0.100000in}{0.212622in}}{\pgfqpoint{3.696000in}{3.696000in}}%
\pgfusepath{clip}%
\pgfsetrectcap%
\pgfsetroundjoin%
\pgfsetlinewidth{1.505625pt}%
\definecolor{currentstroke}{rgb}{1.000000,0.000000,0.000000}%
\pgfsetstrokecolor{currentstroke}%
\pgfsetdash{}{0pt}%
\pgfpathmoveto{\pgfqpoint{2.192786in}{0.962917in}}%
\pgfpathlineto{\pgfqpoint{2.292055in}{1.827512in}}%
\pgfusepath{stroke}%
\end{pgfscope}%
\begin{pgfscope}%
\pgfpathrectangle{\pgfqpoint{0.100000in}{0.212622in}}{\pgfqpoint{3.696000in}{3.696000in}}%
\pgfusepath{clip}%
\pgfsetrectcap%
\pgfsetroundjoin%
\pgfsetlinewidth{1.505625pt}%
\definecolor{currentstroke}{rgb}{1.000000,0.000000,0.000000}%
\pgfsetstrokecolor{currentstroke}%
\pgfsetdash{}{0pt}%
\pgfpathmoveto{\pgfqpoint{2.198830in}{0.958614in}}%
\pgfpathlineto{\pgfqpoint{2.292055in}{1.827512in}}%
\pgfusepath{stroke}%
\end{pgfscope}%
\begin{pgfscope}%
\pgfpathrectangle{\pgfqpoint{0.100000in}{0.212622in}}{\pgfqpoint{3.696000in}{3.696000in}}%
\pgfusepath{clip}%
\pgfsetrectcap%
\pgfsetroundjoin%
\pgfsetlinewidth{1.505625pt}%
\definecolor{currentstroke}{rgb}{1.000000,0.000000,0.000000}%
\pgfsetstrokecolor{currentstroke}%
\pgfsetdash{}{0pt}%
\pgfpathmoveto{\pgfqpoint{2.206029in}{0.952392in}}%
\pgfpathlineto{\pgfqpoint{2.281225in}{1.817686in}}%
\pgfusepath{stroke}%
\end{pgfscope}%
\begin{pgfscope}%
\pgfpathrectangle{\pgfqpoint{0.100000in}{0.212622in}}{\pgfqpoint{3.696000in}{3.696000in}}%
\pgfusepath{clip}%
\pgfsetrectcap%
\pgfsetroundjoin%
\pgfsetlinewidth{1.505625pt}%
\definecolor{currentstroke}{rgb}{1.000000,0.000000,0.000000}%
\pgfsetstrokecolor{currentstroke}%
\pgfsetdash{}{0pt}%
\pgfpathmoveto{\pgfqpoint{2.214533in}{0.944373in}}%
\pgfpathlineto{\pgfqpoint{2.281225in}{1.817686in}}%
\pgfusepath{stroke}%
\end{pgfscope}%
\begin{pgfscope}%
\pgfpathrectangle{\pgfqpoint{0.100000in}{0.212622in}}{\pgfqpoint{3.696000in}{3.696000in}}%
\pgfusepath{clip}%
\pgfsetrectcap%
\pgfsetroundjoin%
\pgfsetlinewidth{1.505625pt}%
\definecolor{currentstroke}{rgb}{1.000000,0.000000,0.000000}%
\pgfsetstrokecolor{currentstroke}%
\pgfsetdash{}{0pt}%
\pgfpathmoveto{\pgfqpoint{2.225383in}{0.938282in}}%
\pgfpathlineto{\pgfqpoint{2.281225in}{1.817686in}}%
\pgfusepath{stroke}%
\end{pgfscope}%
\begin{pgfscope}%
\pgfpathrectangle{\pgfqpoint{0.100000in}{0.212622in}}{\pgfqpoint{3.696000in}{3.696000in}}%
\pgfusepath{clip}%
\pgfsetrectcap%
\pgfsetroundjoin%
\pgfsetlinewidth{1.505625pt}%
\definecolor{currentstroke}{rgb}{1.000000,0.000000,0.000000}%
\pgfsetstrokecolor{currentstroke}%
\pgfsetdash{}{0pt}%
\pgfpathmoveto{\pgfqpoint{2.238864in}{0.932812in}}%
\pgfpathlineto{\pgfqpoint{2.281225in}{1.817686in}}%
\pgfusepath{stroke}%
\end{pgfscope}%
\begin{pgfscope}%
\pgfpathrectangle{\pgfqpoint{0.100000in}{0.212622in}}{\pgfqpoint{3.696000in}{3.696000in}}%
\pgfusepath{clip}%
\pgfsetrectcap%
\pgfsetroundjoin%
\pgfsetlinewidth{1.505625pt}%
\definecolor{currentstroke}{rgb}{1.000000,0.000000,0.000000}%
\pgfsetstrokecolor{currentstroke}%
\pgfsetdash{}{0pt}%
\pgfpathmoveto{\pgfqpoint{2.246523in}{0.931378in}}%
\pgfpathlineto{\pgfqpoint{2.292055in}{1.827512in}}%
\pgfusepath{stroke}%
\end{pgfscope}%
\begin{pgfscope}%
\pgfpathrectangle{\pgfqpoint{0.100000in}{0.212622in}}{\pgfqpoint{3.696000in}{3.696000in}}%
\pgfusepath{clip}%
\pgfsetrectcap%
\pgfsetroundjoin%
\pgfsetlinewidth{1.505625pt}%
\definecolor{currentstroke}{rgb}{1.000000,0.000000,0.000000}%
\pgfsetstrokecolor{currentstroke}%
\pgfsetdash{}{0pt}%
\pgfpathmoveto{\pgfqpoint{2.254805in}{0.924912in}}%
\pgfpathlineto{\pgfqpoint{2.292055in}{1.827512in}}%
\pgfusepath{stroke}%
\end{pgfscope}%
\begin{pgfscope}%
\pgfpathrectangle{\pgfqpoint{0.100000in}{0.212622in}}{\pgfqpoint{3.696000in}{3.696000in}}%
\pgfusepath{clip}%
\pgfsetrectcap%
\pgfsetroundjoin%
\pgfsetlinewidth{1.505625pt}%
\definecolor{currentstroke}{rgb}{1.000000,0.000000,0.000000}%
\pgfsetstrokecolor{currentstroke}%
\pgfsetdash{}{0pt}%
\pgfpathmoveto{\pgfqpoint{2.264197in}{0.920762in}}%
\pgfpathlineto{\pgfqpoint{2.292055in}{1.827512in}}%
\pgfusepath{stroke}%
\end{pgfscope}%
\begin{pgfscope}%
\pgfpathrectangle{\pgfqpoint{0.100000in}{0.212622in}}{\pgfqpoint{3.696000in}{3.696000in}}%
\pgfusepath{clip}%
\pgfsetrectcap%
\pgfsetroundjoin%
\pgfsetlinewidth{1.505625pt}%
\definecolor{currentstroke}{rgb}{1.000000,0.000000,0.000000}%
\pgfsetstrokecolor{currentstroke}%
\pgfsetdash{}{0pt}%
\pgfpathmoveto{\pgfqpoint{2.268975in}{0.917085in}}%
\pgfpathlineto{\pgfqpoint{2.292055in}{1.827512in}}%
\pgfusepath{stroke}%
\end{pgfscope}%
\begin{pgfscope}%
\pgfpathrectangle{\pgfqpoint{0.100000in}{0.212622in}}{\pgfqpoint{3.696000in}{3.696000in}}%
\pgfusepath{clip}%
\pgfsetrectcap%
\pgfsetroundjoin%
\pgfsetlinewidth{1.505625pt}%
\definecolor{currentstroke}{rgb}{1.000000,0.000000,0.000000}%
\pgfsetstrokecolor{currentstroke}%
\pgfsetdash{}{0pt}%
\pgfpathmoveto{\pgfqpoint{2.275281in}{0.915538in}}%
\pgfpathlineto{\pgfqpoint{2.292055in}{1.827512in}}%
\pgfusepath{stroke}%
\end{pgfscope}%
\begin{pgfscope}%
\pgfpathrectangle{\pgfqpoint{0.100000in}{0.212622in}}{\pgfqpoint{3.696000in}{3.696000in}}%
\pgfusepath{clip}%
\pgfsetrectcap%
\pgfsetroundjoin%
\pgfsetlinewidth{1.505625pt}%
\definecolor{currentstroke}{rgb}{1.000000,0.000000,0.000000}%
\pgfsetstrokecolor{currentstroke}%
\pgfsetdash{}{0pt}%
\pgfpathmoveto{\pgfqpoint{2.278743in}{0.915535in}}%
\pgfpathlineto{\pgfqpoint{2.292055in}{1.827512in}}%
\pgfusepath{stroke}%
\end{pgfscope}%
\begin{pgfscope}%
\pgfpathrectangle{\pgfqpoint{0.100000in}{0.212622in}}{\pgfqpoint{3.696000in}{3.696000in}}%
\pgfusepath{clip}%
\pgfsetrectcap%
\pgfsetroundjoin%
\pgfsetlinewidth{1.505625pt}%
\definecolor{currentstroke}{rgb}{1.000000,0.000000,0.000000}%
\pgfsetstrokecolor{currentstroke}%
\pgfsetdash{}{0pt}%
\pgfpathmoveto{\pgfqpoint{2.282980in}{0.915941in}}%
\pgfpathlineto{\pgfqpoint{2.292055in}{1.827512in}}%
\pgfusepath{stroke}%
\end{pgfscope}%
\begin{pgfscope}%
\pgfpathrectangle{\pgfqpoint{0.100000in}{0.212622in}}{\pgfqpoint{3.696000in}{3.696000in}}%
\pgfusepath{clip}%
\pgfsetrectcap%
\pgfsetroundjoin%
\pgfsetlinewidth{1.505625pt}%
\definecolor{currentstroke}{rgb}{1.000000,0.000000,0.000000}%
\pgfsetstrokecolor{currentstroke}%
\pgfsetdash{}{0pt}%
\pgfpathmoveto{\pgfqpoint{2.288217in}{0.914289in}}%
\pgfpathlineto{\pgfqpoint{2.292055in}{1.827512in}}%
\pgfusepath{stroke}%
\end{pgfscope}%
\begin{pgfscope}%
\pgfpathrectangle{\pgfqpoint{0.100000in}{0.212622in}}{\pgfqpoint{3.696000in}{3.696000in}}%
\pgfusepath{clip}%
\pgfsetrectcap%
\pgfsetroundjoin%
\pgfsetlinewidth{1.505625pt}%
\definecolor{currentstroke}{rgb}{1.000000,0.000000,0.000000}%
\pgfsetstrokecolor{currentstroke}%
\pgfsetdash{}{0pt}%
\pgfpathmoveto{\pgfqpoint{2.290747in}{0.911767in}}%
\pgfpathlineto{\pgfqpoint{2.292055in}{1.827512in}}%
\pgfusepath{stroke}%
\end{pgfscope}%
\begin{pgfscope}%
\pgfpathrectangle{\pgfqpoint{0.100000in}{0.212622in}}{\pgfqpoint{3.696000in}{3.696000in}}%
\pgfusepath{clip}%
\pgfsetrectcap%
\pgfsetroundjoin%
\pgfsetlinewidth{1.505625pt}%
\definecolor{currentstroke}{rgb}{1.000000,0.000000,0.000000}%
\pgfsetstrokecolor{currentstroke}%
\pgfsetdash{}{0pt}%
\pgfpathmoveto{\pgfqpoint{2.294328in}{0.908399in}}%
\pgfpathlineto{\pgfqpoint{2.292055in}{1.827512in}}%
\pgfusepath{stroke}%
\end{pgfscope}%
\begin{pgfscope}%
\pgfpathrectangle{\pgfqpoint{0.100000in}{0.212622in}}{\pgfqpoint{3.696000in}{3.696000in}}%
\pgfusepath{clip}%
\pgfsetrectcap%
\pgfsetroundjoin%
\pgfsetlinewidth{1.505625pt}%
\definecolor{currentstroke}{rgb}{1.000000,0.000000,0.000000}%
\pgfsetstrokecolor{currentstroke}%
\pgfsetdash{}{0pt}%
\pgfpathmoveto{\pgfqpoint{2.296441in}{0.907127in}}%
\pgfpathlineto{\pgfqpoint{2.292055in}{1.827512in}}%
\pgfusepath{stroke}%
\end{pgfscope}%
\begin{pgfscope}%
\pgfpathrectangle{\pgfqpoint{0.100000in}{0.212622in}}{\pgfqpoint{3.696000in}{3.696000in}}%
\pgfusepath{clip}%
\pgfsetrectcap%
\pgfsetroundjoin%
\pgfsetlinewidth{1.505625pt}%
\definecolor{currentstroke}{rgb}{1.000000,0.000000,0.000000}%
\pgfsetstrokecolor{currentstroke}%
\pgfsetdash{}{0pt}%
\pgfpathmoveto{\pgfqpoint{2.299082in}{0.906573in}}%
\pgfpathlineto{\pgfqpoint{2.292055in}{1.827512in}}%
\pgfusepath{stroke}%
\end{pgfscope}%
\begin{pgfscope}%
\pgfpathrectangle{\pgfqpoint{0.100000in}{0.212622in}}{\pgfqpoint{3.696000in}{3.696000in}}%
\pgfusepath{clip}%
\pgfsetrectcap%
\pgfsetroundjoin%
\pgfsetlinewidth{1.505625pt}%
\definecolor{currentstroke}{rgb}{1.000000,0.000000,0.000000}%
\pgfsetstrokecolor{currentstroke}%
\pgfsetdash{}{0pt}%
\pgfpathmoveto{\pgfqpoint{2.302244in}{0.905934in}}%
\pgfpathlineto{\pgfqpoint{2.292055in}{1.827512in}}%
\pgfusepath{stroke}%
\end{pgfscope}%
\begin{pgfscope}%
\pgfpathrectangle{\pgfqpoint{0.100000in}{0.212622in}}{\pgfqpoint{3.696000in}{3.696000in}}%
\pgfusepath{clip}%
\pgfsetrectcap%
\pgfsetroundjoin%
\pgfsetlinewidth{1.505625pt}%
\definecolor{currentstroke}{rgb}{1.000000,0.000000,0.000000}%
\pgfsetstrokecolor{currentstroke}%
\pgfsetdash{}{0pt}%
\pgfpathmoveto{\pgfqpoint{2.308160in}{0.904818in}}%
\pgfpathlineto{\pgfqpoint{2.292055in}{1.827512in}}%
\pgfusepath{stroke}%
\end{pgfscope}%
\begin{pgfscope}%
\pgfpathrectangle{\pgfqpoint{0.100000in}{0.212622in}}{\pgfqpoint{3.696000in}{3.696000in}}%
\pgfusepath{clip}%
\pgfsetrectcap%
\pgfsetroundjoin%
\pgfsetlinewidth{1.505625pt}%
\definecolor{currentstroke}{rgb}{1.000000,0.000000,0.000000}%
\pgfsetstrokecolor{currentstroke}%
\pgfsetdash{}{0pt}%
\pgfpathmoveto{\pgfqpoint{2.315622in}{0.898226in}}%
\pgfpathlineto{\pgfqpoint{2.292055in}{1.827512in}}%
\pgfusepath{stroke}%
\end{pgfscope}%
\begin{pgfscope}%
\pgfpathrectangle{\pgfqpoint{0.100000in}{0.212622in}}{\pgfqpoint{3.696000in}{3.696000in}}%
\pgfusepath{clip}%
\pgfsetrectcap%
\pgfsetroundjoin%
\pgfsetlinewidth{1.505625pt}%
\definecolor{currentstroke}{rgb}{1.000000,0.000000,0.000000}%
\pgfsetstrokecolor{currentstroke}%
\pgfsetdash{}{0pt}%
\pgfpathmoveto{\pgfqpoint{2.324217in}{0.892329in}}%
\pgfpathlineto{\pgfqpoint{2.292055in}{1.827512in}}%
\pgfusepath{stroke}%
\end{pgfscope}%
\begin{pgfscope}%
\pgfpathrectangle{\pgfqpoint{0.100000in}{0.212622in}}{\pgfqpoint{3.696000in}{3.696000in}}%
\pgfusepath{clip}%
\pgfsetrectcap%
\pgfsetroundjoin%
\pgfsetlinewidth{1.505625pt}%
\definecolor{currentstroke}{rgb}{1.000000,0.000000,0.000000}%
\pgfsetstrokecolor{currentstroke}%
\pgfsetdash{}{0pt}%
\pgfpathmoveto{\pgfqpoint{2.333455in}{0.885423in}}%
\pgfpathlineto{\pgfqpoint{2.292055in}{1.827512in}}%
\pgfusepath{stroke}%
\end{pgfscope}%
\begin{pgfscope}%
\pgfpathrectangle{\pgfqpoint{0.100000in}{0.212622in}}{\pgfqpoint{3.696000in}{3.696000in}}%
\pgfusepath{clip}%
\pgfsetrectcap%
\pgfsetroundjoin%
\pgfsetlinewidth{1.505625pt}%
\definecolor{currentstroke}{rgb}{1.000000,0.000000,0.000000}%
\pgfsetstrokecolor{currentstroke}%
\pgfsetdash{}{0pt}%
\pgfpathmoveto{\pgfqpoint{2.343090in}{0.881483in}}%
\pgfpathlineto{\pgfqpoint{2.292055in}{1.827512in}}%
\pgfusepath{stroke}%
\end{pgfscope}%
\begin{pgfscope}%
\pgfpathrectangle{\pgfqpoint{0.100000in}{0.212622in}}{\pgfqpoint{3.696000in}{3.696000in}}%
\pgfusepath{clip}%
\pgfsetrectcap%
\pgfsetroundjoin%
\pgfsetlinewidth{1.505625pt}%
\definecolor{currentstroke}{rgb}{1.000000,0.000000,0.000000}%
\pgfsetstrokecolor{currentstroke}%
\pgfsetdash{}{0pt}%
\pgfpathmoveto{\pgfqpoint{2.348845in}{0.881444in}}%
\pgfpathlineto{\pgfqpoint{2.292055in}{1.827512in}}%
\pgfusepath{stroke}%
\end{pgfscope}%
\begin{pgfscope}%
\pgfpathrectangle{\pgfqpoint{0.100000in}{0.212622in}}{\pgfqpoint{3.696000in}{3.696000in}}%
\pgfusepath{clip}%
\pgfsetrectcap%
\pgfsetroundjoin%
\pgfsetlinewidth{1.505625pt}%
\definecolor{currentstroke}{rgb}{1.000000,0.000000,0.000000}%
\pgfsetstrokecolor{currentstroke}%
\pgfsetdash{}{0pt}%
\pgfpathmoveto{\pgfqpoint{2.357829in}{0.881494in}}%
\pgfpathlineto{\pgfqpoint{2.292055in}{1.827512in}}%
\pgfusepath{stroke}%
\end{pgfscope}%
\begin{pgfscope}%
\pgfpathrectangle{\pgfqpoint{0.100000in}{0.212622in}}{\pgfqpoint{3.696000in}{3.696000in}}%
\pgfusepath{clip}%
\pgfsetrectcap%
\pgfsetroundjoin%
\pgfsetlinewidth{1.505625pt}%
\definecolor{currentstroke}{rgb}{1.000000,0.000000,0.000000}%
\pgfsetstrokecolor{currentstroke}%
\pgfsetdash{}{0pt}%
\pgfpathmoveto{\pgfqpoint{2.367709in}{0.873022in}}%
\pgfpathlineto{\pgfqpoint{2.292055in}{1.827512in}}%
\pgfusepath{stroke}%
\end{pgfscope}%
\begin{pgfscope}%
\pgfpathrectangle{\pgfqpoint{0.100000in}{0.212622in}}{\pgfqpoint{3.696000in}{3.696000in}}%
\pgfusepath{clip}%
\pgfsetrectcap%
\pgfsetroundjoin%
\pgfsetlinewidth{1.505625pt}%
\definecolor{currentstroke}{rgb}{1.000000,0.000000,0.000000}%
\pgfsetstrokecolor{currentstroke}%
\pgfsetdash{}{0pt}%
\pgfpathmoveto{\pgfqpoint{2.373158in}{0.867865in}}%
\pgfpathlineto{\pgfqpoint{2.292055in}{1.827512in}}%
\pgfusepath{stroke}%
\end{pgfscope}%
\begin{pgfscope}%
\pgfpathrectangle{\pgfqpoint{0.100000in}{0.212622in}}{\pgfqpoint{3.696000in}{3.696000in}}%
\pgfusepath{clip}%
\pgfsetrectcap%
\pgfsetroundjoin%
\pgfsetlinewidth{1.505625pt}%
\definecolor{currentstroke}{rgb}{1.000000,0.000000,0.000000}%
\pgfsetstrokecolor{currentstroke}%
\pgfsetdash{}{0pt}%
\pgfpathmoveto{\pgfqpoint{2.379410in}{0.864651in}}%
\pgfpathlineto{\pgfqpoint{2.292055in}{1.827512in}}%
\pgfusepath{stroke}%
\end{pgfscope}%
\begin{pgfscope}%
\pgfpathrectangle{\pgfqpoint{0.100000in}{0.212622in}}{\pgfqpoint{3.696000in}{3.696000in}}%
\pgfusepath{clip}%
\pgfsetrectcap%
\pgfsetroundjoin%
\pgfsetlinewidth{1.505625pt}%
\definecolor{currentstroke}{rgb}{1.000000,0.000000,0.000000}%
\pgfsetstrokecolor{currentstroke}%
\pgfsetdash{}{0pt}%
\pgfpathmoveto{\pgfqpoint{2.386321in}{0.861941in}}%
\pgfpathlineto{\pgfqpoint{2.292055in}{1.827512in}}%
\pgfusepath{stroke}%
\end{pgfscope}%
\begin{pgfscope}%
\pgfpathrectangle{\pgfqpoint{0.100000in}{0.212622in}}{\pgfqpoint{3.696000in}{3.696000in}}%
\pgfusepath{clip}%
\pgfsetrectcap%
\pgfsetroundjoin%
\pgfsetlinewidth{1.505625pt}%
\definecolor{currentstroke}{rgb}{1.000000,0.000000,0.000000}%
\pgfsetstrokecolor{currentstroke}%
\pgfsetdash{}{0pt}%
\pgfpathmoveto{\pgfqpoint{2.390273in}{0.862434in}}%
\pgfpathlineto{\pgfqpoint{2.292055in}{1.827512in}}%
\pgfusepath{stroke}%
\end{pgfscope}%
\begin{pgfscope}%
\pgfpathrectangle{\pgfqpoint{0.100000in}{0.212622in}}{\pgfqpoint{3.696000in}{3.696000in}}%
\pgfusepath{clip}%
\pgfsetrectcap%
\pgfsetroundjoin%
\pgfsetlinewidth{1.505625pt}%
\definecolor{currentstroke}{rgb}{1.000000,0.000000,0.000000}%
\pgfsetstrokecolor{currentstroke}%
\pgfsetdash{}{0pt}%
\pgfpathmoveto{\pgfqpoint{2.395199in}{0.862608in}}%
\pgfpathlineto{\pgfqpoint{2.292055in}{1.827512in}}%
\pgfusepath{stroke}%
\end{pgfscope}%
\begin{pgfscope}%
\pgfpathrectangle{\pgfqpoint{0.100000in}{0.212622in}}{\pgfqpoint{3.696000in}{3.696000in}}%
\pgfusepath{clip}%
\pgfsetrectcap%
\pgfsetroundjoin%
\pgfsetlinewidth{1.505625pt}%
\definecolor{currentstroke}{rgb}{1.000000,0.000000,0.000000}%
\pgfsetstrokecolor{currentstroke}%
\pgfsetdash{}{0pt}%
\pgfpathmoveto{\pgfqpoint{2.400462in}{0.860771in}}%
\pgfpathlineto{\pgfqpoint{2.292055in}{1.827512in}}%
\pgfusepath{stroke}%
\end{pgfscope}%
\begin{pgfscope}%
\pgfpathrectangle{\pgfqpoint{0.100000in}{0.212622in}}{\pgfqpoint{3.696000in}{3.696000in}}%
\pgfusepath{clip}%
\pgfsetrectcap%
\pgfsetroundjoin%
\pgfsetlinewidth{1.505625pt}%
\definecolor{currentstroke}{rgb}{1.000000,0.000000,0.000000}%
\pgfsetstrokecolor{currentstroke}%
\pgfsetdash{}{0pt}%
\pgfpathmoveto{\pgfqpoint{2.405992in}{0.856112in}}%
\pgfpathlineto{\pgfqpoint{2.292055in}{1.827512in}}%
\pgfusepath{stroke}%
\end{pgfscope}%
\begin{pgfscope}%
\pgfpathrectangle{\pgfqpoint{0.100000in}{0.212622in}}{\pgfqpoint{3.696000in}{3.696000in}}%
\pgfusepath{clip}%
\pgfsetrectcap%
\pgfsetroundjoin%
\pgfsetlinewidth{1.505625pt}%
\definecolor{currentstroke}{rgb}{1.000000,0.000000,0.000000}%
\pgfsetstrokecolor{currentstroke}%
\pgfsetdash{}{0pt}%
\pgfpathmoveto{\pgfqpoint{2.412195in}{0.851791in}}%
\pgfpathlineto{\pgfqpoint{2.292055in}{1.827512in}}%
\pgfusepath{stroke}%
\end{pgfscope}%
\begin{pgfscope}%
\pgfpathrectangle{\pgfqpoint{0.100000in}{0.212622in}}{\pgfqpoint{3.696000in}{3.696000in}}%
\pgfusepath{clip}%
\pgfsetrectcap%
\pgfsetroundjoin%
\pgfsetlinewidth{1.505625pt}%
\definecolor{currentstroke}{rgb}{1.000000,0.000000,0.000000}%
\pgfsetstrokecolor{currentstroke}%
\pgfsetdash{}{0pt}%
\pgfpathmoveto{\pgfqpoint{2.419804in}{0.847833in}}%
\pgfpathlineto{\pgfqpoint{2.292055in}{1.827512in}}%
\pgfusepath{stroke}%
\end{pgfscope}%
\begin{pgfscope}%
\pgfpathrectangle{\pgfqpoint{0.100000in}{0.212622in}}{\pgfqpoint{3.696000in}{3.696000in}}%
\pgfusepath{clip}%
\pgfsetrectcap%
\pgfsetroundjoin%
\pgfsetlinewidth{1.505625pt}%
\definecolor{currentstroke}{rgb}{1.000000,0.000000,0.000000}%
\pgfsetstrokecolor{currentstroke}%
\pgfsetdash{}{0pt}%
\pgfpathmoveto{\pgfqpoint{2.427958in}{0.844197in}}%
\pgfpathlineto{\pgfqpoint{2.292055in}{1.827512in}}%
\pgfusepath{stroke}%
\end{pgfscope}%
\begin{pgfscope}%
\pgfpathrectangle{\pgfqpoint{0.100000in}{0.212622in}}{\pgfqpoint{3.696000in}{3.696000in}}%
\pgfusepath{clip}%
\pgfsetrectcap%
\pgfsetroundjoin%
\pgfsetlinewidth{1.505625pt}%
\definecolor{currentstroke}{rgb}{1.000000,0.000000,0.000000}%
\pgfsetstrokecolor{currentstroke}%
\pgfsetdash{}{0pt}%
\pgfpathmoveto{\pgfqpoint{2.436580in}{0.838471in}}%
\pgfpathlineto{\pgfqpoint{2.292055in}{1.827512in}}%
\pgfusepath{stroke}%
\end{pgfscope}%
\begin{pgfscope}%
\pgfpathrectangle{\pgfqpoint{0.100000in}{0.212622in}}{\pgfqpoint{3.696000in}{3.696000in}}%
\pgfusepath{clip}%
\pgfsetrectcap%
\pgfsetroundjoin%
\pgfsetlinewidth{1.505625pt}%
\definecolor{currentstroke}{rgb}{1.000000,0.000000,0.000000}%
\pgfsetstrokecolor{currentstroke}%
\pgfsetdash{}{0pt}%
\pgfpathmoveto{\pgfqpoint{2.446251in}{0.834415in}}%
\pgfpathlineto{\pgfqpoint{2.292055in}{1.827512in}}%
\pgfusepath{stroke}%
\end{pgfscope}%
\begin{pgfscope}%
\pgfpathrectangle{\pgfqpoint{0.100000in}{0.212622in}}{\pgfqpoint{3.696000in}{3.696000in}}%
\pgfusepath{clip}%
\pgfsetrectcap%
\pgfsetroundjoin%
\pgfsetlinewidth{1.505625pt}%
\definecolor{currentstroke}{rgb}{1.000000,0.000000,0.000000}%
\pgfsetstrokecolor{currentstroke}%
\pgfsetdash{}{0pt}%
\pgfpathmoveto{\pgfqpoint{2.457113in}{0.829959in}}%
\pgfpathlineto{\pgfqpoint{2.292055in}{1.827512in}}%
\pgfusepath{stroke}%
\end{pgfscope}%
\begin{pgfscope}%
\pgfpathrectangle{\pgfqpoint{0.100000in}{0.212622in}}{\pgfqpoint{3.696000in}{3.696000in}}%
\pgfusepath{clip}%
\pgfsetrectcap%
\pgfsetroundjoin%
\pgfsetlinewidth{1.505625pt}%
\definecolor{currentstroke}{rgb}{1.000000,0.000000,0.000000}%
\pgfsetstrokecolor{currentstroke}%
\pgfsetdash{}{0pt}%
\pgfpathmoveto{\pgfqpoint{2.468141in}{0.825114in}}%
\pgfpathlineto{\pgfqpoint{2.292055in}{1.827512in}}%
\pgfusepath{stroke}%
\end{pgfscope}%
\begin{pgfscope}%
\pgfpathrectangle{\pgfqpoint{0.100000in}{0.212622in}}{\pgfqpoint{3.696000in}{3.696000in}}%
\pgfusepath{clip}%
\pgfsetrectcap%
\pgfsetroundjoin%
\pgfsetlinewidth{1.505625pt}%
\definecolor{currentstroke}{rgb}{1.000000,0.000000,0.000000}%
\pgfsetstrokecolor{currentstroke}%
\pgfsetdash{}{0pt}%
\pgfpathmoveto{\pgfqpoint{2.480863in}{0.822053in}}%
\pgfpathlineto{\pgfqpoint{2.292055in}{1.827512in}}%
\pgfusepath{stroke}%
\end{pgfscope}%
\begin{pgfscope}%
\pgfpathrectangle{\pgfqpoint{0.100000in}{0.212622in}}{\pgfqpoint{3.696000in}{3.696000in}}%
\pgfusepath{clip}%
\pgfsetrectcap%
\pgfsetroundjoin%
\pgfsetlinewidth{1.505625pt}%
\definecolor{currentstroke}{rgb}{1.000000,0.000000,0.000000}%
\pgfsetstrokecolor{currentstroke}%
\pgfsetdash{}{0pt}%
\pgfpathmoveto{\pgfqpoint{2.493979in}{0.819180in}}%
\pgfpathlineto{\pgfqpoint{2.292055in}{1.827512in}}%
\pgfusepath{stroke}%
\end{pgfscope}%
\begin{pgfscope}%
\pgfpathrectangle{\pgfqpoint{0.100000in}{0.212622in}}{\pgfqpoint{3.696000in}{3.696000in}}%
\pgfusepath{clip}%
\pgfsetrectcap%
\pgfsetroundjoin%
\pgfsetlinewidth{1.505625pt}%
\definecolor{currentstroke}{rgb}{1.000000,0.000000,0.000000}%
\pgfsetstrokecolor{currentstroke}%
\pgfsetdash{}{0pt}%
\pgfpathmoveto{\pgfqpoint{2.508662in}{0.818941in}}%
\pgfpathlineto{\pgfqpoint{2.292055in}{1.827512in}}%
\pgfusepath{stroke}%
\end{pgfscope}%
\begin{pgfscope}%
\pgfpathrectangle{\pgfqpoint{0.100000in}{0.212622in}}{\pgfqpoint{3.696000in}{3.696000in}}%
\pgfusepath{clip}%
\pgfsetrectcap%
\pgfsetroundjoin%
\pgfsetlinewidth{1.505625pt}%
\definecolor{currentstroke}{rgb}{1.000000,0.000000,0.000000}%
\pgfsetstrokecolor{currentstroke}%
\pgfsetdash{}{0pt}%
\pgfpathmoveto{\pgfqpoint{2.524291in}{0.818630in}}%
\pgfpathlineto{\pgfqpoint{2.302867in}{1.837322in}}%
\pgfusepath{stroke}%
\end{pgfscope}%
\begin{pgfscope}%
\pgfpathrectangle{\pgfqpoint{0.100000in}{0.212622in}}{\pgfqpoint{3.696000in}{3.696000in}}%
\pgfusepath{clip}%
\pgfsetrectcap%
\pgfsetroundjoin%
\pgfsetlinewidth{1.505625pt}%
\definecolor{currentstroke}{rgb}{1.000000,0.000000,0.000000}%
\pgfsetstrokecolor{currentstroke}%
\pgfsetdash{}{0pt}%
\pgfpathmoveto{\pgfqpoint{2.533474in}{0.824607in}}%
\pgfpathlineto{\pgfqpoint{2.302867in}{1.837322in}}%
\pgfusepath{stroke}%
\end{pgfscope}%
\begin{pgfscope}%
\pgfpathrectangle{\pgfqpoint{0.100000in}{0.212622in}}{\pgfqpoint{3.696000in}{3.696000in}}%
\pgfusepath{clip}%
\pgfsetrectcap%
\pgfsetroundjoin%
\pgfsetlinewidth{1.505625pt}%
\definecolor{currentstroke}{rgb}{1.000000,0.000000,0.000000}%
\pgfsetstrokecolor{currentstroke}%
\pgfsetdash{}{0pt}%
\pgfpathmoveto{\pgfqpoint{2.542610in}{0.837454in}}%
\pgfpathlineto{\pgfqpoint{2.313662in}{1.847116in}}%
\pgfusepath{stroke}%
\end{pgfscope}%
\begin{pgfscope}%
\pgfpathrectangle{\pgfqpoint{0.100000in}{0.212622in}}{\pgfqpoint{3.696000in}{3.696000in}}%
\pgfusepath{clip}%
\pgfsetrectcap%
\pgfsetroundjoin%
\pgfsetlinewidth{1.505625pt}%
\definecolor{currentstroke}{rgb}{1.000000,0.000000,0.000000}%
\pgfsetstrokecolor{currentstroke}%
\pgfsetdash{}{0pt}%
\pgfpathmoveto{\pgfqpoint{2.546316in}{0.844980in}}%
\pgfpathlineto{\pgfqpoint{2.313662in}{1.847116in}}%
\pgfusepath{stroke}%
\end{pgfscope}%
\begin{pgfscope}%
\pgfpathrectangle{\pgfqpoint{0.100000in}{0.212622in}}{\pgfqpoint{3.696000in}{3.696000in}}%
\pgfusepath{clip}%
\pgfsetrectcap%
\pgfsetroundjoin%
\pgfsetlinewidth{1.505625pt}%
\definecolor{currentstroke}{rgb}{1.000000,0.000000,0.000000}%
\pgfsetstrokecolor{currentstroke}%
\pgfsetdash{}{0pt}%
\pgfpathmoveto{\pgfqpoint{2.548638in}{0.848912in}}%
\pgfpathlineto{\pgfqpoint{2.313662in}{1.847116in}}%
\pgfusepath{stroke}%
\end{pgfscope}%
\begin{pgfscope}%
\pgfpathrectangle{\pgfqpoint{0.100000in}{0.212622in}}{\pgfqpoint{3.696000in}{3.696000in}}%
\pgfusepath{clip}%
\pgfsetrectcap%
\pgfsetroundjoin%
\pgfsetlinewidth{1.505625pt}%
\definecolor{currentstroke}{rgb}{1.000000,0.000000,0.000000}%
\pgfsetstrokecolor{currentstroke}%
\pgfsetdash{}{0pt}%
\pgfpathmoveto{\pgfqpoint{2.549957in}{0.851109in}}%
\pgfpathlineto{\pgfqpoint{2.313662in}{1.847116in}}%
\pgfusepath{stroke}%
\end{pgfscope}%
\begin{pgfscope}%
\pgfpathrectangle{\pgfqpoint{0.100000in}{0.212622in}}{\pgfqpoint{3.696000in}{3.696000in}}%
\pgfusepath{clip}%
\pgfsetrectcap%
\pgfsetroundjoin%
\pgfsetlinewidth{1.505625pt}%
\definecolor{currentstroke}{rgb}{1.000000,0.000000,0.000000}%
\pgfsetstrokecolor{currentstroke}%
\pgfsetdash{}{0pt}%
\pgfpathmoveto{\pgfqpoint{2.550643in}{0.852219in}}%
\pgfpathlineto{\pgfqpoint{2.313662in}{1.847116in}}%
\pgfusepath{stroke}%
\end{pgfscope}%
\begin{pgfscope}%
\pgfpathrectangle{\pgfqpoint{0.100000in}{0.212622in}}{\pgfqpoint{3.696000in}{3.696000in}}%
\pgfusepath{clip}%
\pgfsetrectcap%
\pgfsetroundjoin%
\pgfsetlinewidth{1.505625pt}%
\definecolor{currentstroke}{rgb}{1.000000,0.000000,0.000000}%
\pgfsetstrokecolor{currentstroke}%
\pgfsetdash{}{0pt}%
\pgfpathmoveto{\pgfqpoint{2.550988in}{0.852834in}}%
\pgfpathlineto{\pgfqpoint{2.313662in}{1.847116in}}%
\pgfusepath{stroke}%
\end{pgfscope}%
\begin{pgfscope}%
\pgfpathrectangle{\pgfqpoint{0.100000in}{0.212622in}}{\pgfqpoint{3.696000in}{3.696000in}}%
\pgfusepath{clip}%
\pgfsetrectcap%
\pgfsetroundjoin%
\pgfsetlinewidth{1.505625pt}%
\definecolor{currentstroke}{rgb}{1.000000,0.000000,0.000000}%
\pgfsetstrokecolor{currentstroke}%
\pgfsetdash{}{0pt}%
\pgfpathmoveto{\pgfqpoint{2.551158in}{0.853207in}}%
\pgfpathlineto{\pgfqpoint{2.313662in}{1.847116in}}%
\pgfusepath{stroke}%
\end{pgfscope}%
\begin{pgfscope}%
\pgfpathrectangle{\pgfqpoint{0.100000in}{0.212622in}}{\pgfqpoint{3.696000in}{3.696000in}}%
\pgfusepath{clip}%
\pgfsetrectcap%
\pgfsetroundjoin%
\pgfsetlinewidth{1.505625pt}%
\definecolor{currentstroke}{rgb}{1.000000,0.000000,0.000000}%
\pgfsetstrokecolor{currentstroke}%
\pgfsetdash{}{0pt}%
\pgfpathmoveto{\pgfqpoint{2.551255in}{0.853400in}}%
\pgfpathlineto{\pgfqpoint{2.313662in}{1.847116in}}%
\pgfusepath{stroke}%
\end{pgfscope}%
\begin{pgfscope}%
\pgfpathrectangle{\pgfqpoint{0.100000in}{0.212622in}}{\pgfqpoint{3.696000in}{3.696000in}}%
\pgfusepath{clip}%
\pgfsetrectcap%
\pgfsetroundjoin%
\pgfsetlinewidth{1.505625pt}%
\definecolor{currentstroke}{rgb}{1.000000,0.000000,0.000000}%
\pgfsetstrokecolor{currentstroke}%
\pgfsetdash{}{0pt}%
\pgfpathmoveto{\pgfqpoint{2.551302in}{0.853513in}}%
\pgfpathlineto{\pgfqpoint{2.313662in}{1.847116in}}%
\pgfusepath{stroke}%
\end{pgfscope}%
\begin{pgfscope}%
\pgfpathrectangle{\pgfqpoint{0.100000in}{0.212622in}}{\pgfqpoint{3.696000in}{3.696000in}}%
\pgfusepath{clip}%
\pgfsetrectcap%
\pgfsetroundjoin%
\pgfsetlinewidth{1.505625pt}%
\definecolor{currentstroke}{rgb}{1.000000,0.000000,0.000000}%
\pgfsetstrokecolor{currentstroke}%
\pgfsetdash{}{0pt}%
\pgfpathmoveto{\pgfqpoint{2.551327in}{0.853572in}}%
\pgfpathlineto{\pgfqpoint{2.313662in}{1.847116in}}%
\pgfusepath{stroke}%
\end{pgfscope}%
\begin{pgfscope}%
\pgfpathrectangle{\pgfqpoint{0.100000in}{0.212622in}}{\pgfqpoint{3.696000in}{3.696000in}}%
\pgfusepath{clip}%
\pgfsetrectcap%
\pgfsetroundjoin%
\pgfsetlinewidth{1.505625pt}%
\definecolor{currentstroke}{rgb}{1.000000,0.000000,0.000000}%
\pgfsetstrokecolor{currentstroke}%
\pgfsetdash{}{0pt}%
\pgfpathmoveto{\pgfqpoint{2.551341in}{0.853604in}}%
\pgfpathlineto{\pgfqpoint{2.313662in}{1.847116in}}%
\pgfusepath{stroke}%
\end{pgfscope}%
\begin{pgfscope}%
\pgfpathrectangle{\pgfqpoint{0.100000in}{0.212622in}}{\pgfqpoint{3.696000in}{3.696000in}}%
\pgfusepath{clip}%
\pgfsetrectcap%
\pgfsetroundjoin%
\pgfsetlinewidth{1.505625pt}%
\definecolor{currentstroke}{rgb}{1.000000,0.000000,0.000000}%
\pgfsetstrokecolor{currentstroke}%
\pgfsetdash{}{0pt}%
\pgfpathmoveto{\pgfqpoint{2.551349in}{0.853620in}}%
\pgfpathlineto{\pgfqpoint{2.313662in}{1.847116in}}%
\pgfusepath{stroke}%
\end{pgfscope}%
\begin{pgfscope}%
\pgfpathrectangle{\pgfqpoint{0.100000in}{0.212622in}}{\pgfqpoint{3.696000in}{3.696000in}}%
\pgfusepath{clip}%
\pgfsetrectcap%
\pgfsetroundjoin%
\pgfsetlinewidth{1.505625pt}%
\definecolor{currentstroke}{rgb}{1.000000,0.000000,0.000000}%
\pgfsetstrokecolor{currentstroke}%
\pgfsetdash{}{0pt}%
\pgfpathmoveto{\pgfqpoint{2.551354in}{0.853629in}}%
\pgfpathlineto{\pgfqpoint{2.313662in}{1.847116in}}%
\pgfusepath{stroke}%
\end{pgfscope}%
\begin{pgfscope}%
\pgfpathrectangle{\pgfqpoint{0.100000in}{0.212622in}}{\pgfqpoint{3.696000in}{3.696000in}}%
\pgfusepath{clip}%
\pgfsetrectcap%
\pgfsetroundjoin%
\pgfsetlinewidth{1.505625pt}%
\definecolor{currentstroke}{rgb}{1.000000,0.000000,0.000000}%
\pgfsetstrokecolor{currentstroke}%
\pgfsetdash{}{0pt}%
\pgfpathmoveto{\pgfqpoint{2.551356in}{0.853634in}}%
\pgfpathlineto{\pgfqpoint{2.313662in}{1.847116in}}%
\pgfusepath{stroke}%
\end{pgfscope}%
\begin{pgfscope}%
\pgfpathrectangle{\pgfqpoint{0.100000in}{0.212622in}}{\pgfqpoint{3.696000in}{3.696000in}}%
\pgfusepath{clip}%
\pgfsetrectcap%
\pgfsetroundjoin%
\pgfsetlinewidth{1.505625pt}%
\definecolor{currentstroke}{rgb}{1.000000,0.000000,0.000000}%
\pgfsetstrokecolor{currentstroke}%
\pgfsetdash{}{0pt}%
\pgfpathmoveto{\pgfqpoint{2.551358in}{0.853637in}}%
\pgfpathlineto{\pgfqpoint{2.313662in}{1.847116in}}%
\pgfusepath{stroke}%
\end{pgfscope}%
\begin{pgfscope}%
\pgfpathrectangle{\pgfqpoint{0.100000in}{0.212622in}}{\pgfqpoint{3.696000in}{3.696000in}}%
\pgfusepath{clip}%
\pgfsetrectcap%
\pgfsetroundjoin%
\pgfsetlinewidth{1.505625pt}%
\definecolor{currentstroke}{rgb}{1.000000,0.000000,0.000000}%
\pgfsetstrokecolor{currentstroke}%
\pgfsetdash{}{0pt}%
\pgfpathmoveto{\pgfqpoint{2.551359in}{0.853639in}}%
\pgfpathlineto{\pgfqpoint{2.313662in}{1.847116in}}%
\pgfusepath{stroke}%
\end{pgfscope}%
\begin{pgfscope}%
\pgfpathrectangle{\pgfqpoint{0.100000in}{0.212622in}}{\pgfqpoint{3.696000in}{3.696000in}}%
\pgfusepath{clip}%
\pgfsetrectcap%
\pgfsetroundjoin%
\pgfsetlinewidth{1.505625pt}%
\definecolor{currentstroke}{rgb}{1.000000,0.000000,0.000000}%
\pgfsetstrokecolor{currentstroke}%
\pgfsetdash{}{0pt}%
\pgfpathmoveto{\pgfqpoint{2.551359in}{0.853640in}}%
\pgfpathlineto{\pgfqpoint{2.313662in}{1.847116in}}%
\pgfusepath{stroke}%
\end{pgfscope}%
\begin{pgfscope}%
\pgfpathrectangle{\pgfqpoint{0.100000in}{0.212622in}}{\pgfqpoint{3.696000in}{3.696000in}}%
\pgfusepath{clip}%
\pgfsetrectcap%
\pgfsetroundjoin%
\pgfsetlinewidth{1.505625pt}%
\definecolor{currentstroke}{rgb}{1.000000,0.000000,0.000000}%
\pgfsetstrokecolor{currentstroke}%
\pgfsetdash{}{0pt}%
\pgfpathmoveto{\pgfqpoint{2.551359in}{0.853640in}}%
\pgfpathlineto{\pgfqpoint{2.313662in}{1.847116in}}%
\pgfusepath{stroke}%
\end{pgfscope}%
\begin{pgfscope}%
\pgfpathrectangle{\pgfqpoint{0.100000in}{0.212622in}}{\pgfqpoint{3.696000in}{3.696000in}}%
\pgfusepath{clip}%
\pgfsetrectcap%
\pgfsetroundjoin%
\pgfsetlinewidth{1.505625pt}%
\definecolor{currentstroke}{rgb}{1.000000,0.000000,0.000000}%
\pgfsetstrokecolor{currentstroke}%
\pgfsetdash{}{0pt}%
\pgfpathmoveto{\pgfqpoint{2.551359in}{0.853640in}}%
\pgfpathlineto{\pgfqpoint{2.313662in}{1.847116in}}%
\pgfusepath{stroke}%
\end{pgfscope}%
\begin{pgfscope}%
\pgfpathrectangle{\pgfqpoint{0.100000in}{0.212622in}}{\pgfqpoint{3.696000in}{3.696000in}}%
\pgfusepath{clip}%
\pgfsetrectcap%
\pgfsetroundjoin%
\pgfsetlinewidth{1.505625pt}%
\definecolor{currentstroke}{rgb}{1.000000,0.000000,0.000000}%
\pgfsetstrokecolor{currentstroke}%
\pgfsetdash{}{0pt}%
\pgfpathmoveto{\pgfqpoint{2.551359in}{0.853640in}}%
\pgfpathlineto{\pgfqpoint{2.313662in}{1.847116in}}%
\pgfusepath{stroke}%
\end{pgfscope}%
\begin{pgfscope}%
\pgfpathrectangle{\pgfqpoint{0.100000in}{0.212622in}}{\pgfqpoint{3.696000in}{3.696000in}}%
\pgfusepath{clip}%
\pgfsetrectcap%
\pgfsetroundjoin%
\pgfsetlinewidth{1.505625pt}%
\definecolor{currentstroke}{rgb}{1.000000,0.000000,0.000000}%
\pgfsetstrokecolor{currentstroke}%
\pgfsetdash{}{0pt}%
\pgfpathmoveto{\pgfqpoint{2.551359in}{0.853640in}}%
\pgfpathlineto{\pgfqpoint{2.313662in}{1.847116in}}%
\pgfusepath{stroke}%
\end{pgfscope}%
\begin{pgfscope}%
\pgfpathrectangle{\pgfqpoint{0.100000in}{0.212622in}}{\pgfqpoint{3.696000in}{3.696000in}}%
\pgfusepath{clip}%
\pgfsetrectcap%
\pgfsetroundjoin%
\pgfsetlinewidth{1.505625pt}%
\definecolor{currentstroke}{rgb}{1.000000,0.000000,0.000000}%
\pgfsetstrokecolor{currentstroke}%
\pgfsetdash{}{0pt}%
\pgfpathmoveto{\pgfqpoint{2.551359in}{0.853641in}}%
\pgfpathlineto{\pgfqpoint{2.313662in}{1.847116in}}%
\pgfusepath{stroke}%
\end{pgfscope}%
\begin{pgfscope}%
\pgfpathrectangle{\pgfqpoint{0.100000in}{0.212622in}}{\pgfqpoint{3.696000in}{3.696000in}}%
\pgfusepath{clip}%
\pgfsetrectcap%
\pgfsetroundjoin%
\pgfsetlinewidth{1.505625pt}%
\definecolor{currentstroke}{rgb}{1.000000,0.000000,0.000000}%
\pgfsetstrokecolor{currentstroke}%
\pgfsetdash{}{0pt}%
\pgfpathmoveto{\pgfqpoint{2.551359in}{0.853641in}}%
\pgfpathlineto{\pgfqpoint{2.313662in}{1.847116in}}%
\pgfusepath{stroke}%
\end{pgfscope}%
\begin{pgfscope}%
\pgfpathrectangle{\pgfqpoint{0.100000in}{0.212622in}}{\pgfqpoint{3.696000in}{3.696000in}}%
\pgfusepath{clip}%
\pgfsetrectcap%
\pgfsetroundjoin%
\pgfsetlinewidth{1.505625pt}%
\definecolor{currentstroke}{rgb}{1.000000,0.000000,0.000000}%
\pgfsetstrokecolor{currentstroke}%
\pgfsetdash{}{0pt}%
\pgfpathmoveto{\pgfqpoint{2.551359in}{0.853641in}}%
\pgfpathlineto{\pgfqpoint{2.313662in}{1.847116in}}%
\pgfusepath{stroke}%
\end{pgfscope}%
\begin{pgfscope}%
\pgfpathrectangle{\pgfqpoint{0.100000in}{0.212622in}}{\pgfqpoint{3.696000in}{3.696000in}}%
\pgfusepath{clip}%
\pgfsetrectcap%
\pgfsetroundjoin%
\pgfsetlinewidth{1.505625pt}%
\definecolor{currentstroke}{rgb}{1.000000,0.000000,0.000000}%
\pgfsetstrokecolor{currentstroke}%
\pgfsetdash{}{0pt}%
\pgfpathmoveto{\pgfqpoint{2.551359in}{0.853641in}}%
\pgfpathlineto{\pgfqpoint{2.313662in}{1.847116in}}%
\pgfusepath{stroke}%
\end{pgfscope}%
\begin{pgfscope}%
\pgfpathrectangle{\pgfqpoint{0.100000in}{0.212622in}}{\pgfqpoint{3.696000in}{3.696000in}}%
\pgfusepath{clip}%
\pgfsetrectcap%
\pgfsetroundjoin%
\pgfsetlinewidth{1.505625pt}%
\definecolor{currentstroke}{rgb}{1.000000,0.000000,0.000000}%
\pgfsetstrokecolor{currentstroke}%
\pgfsetdash{}{0pt}%
\pgfpathmoveto{\pgfqpoint{2.551359in}{0.853641in}}%
\pgfpathlineto{\pgfqpoint{2.313662in}{1.847116in}}%
\pgfusepath{stroke}%
\end{pgfscope}%
\begin{pgfscope}%
\pgfpathrectangle{\pgfqpoint{0.100000in}{0.212622in}}{\pgfqpoint{3.696000in}{3.696000in}}%
\pgfusepath{clip}%
\pgfsetrectcap%
\pgfsetroundjoin%
\pgfsetlinewidth{1.505625pt}%
\definecolor{currentstroke}{rgb}{1.000000,0.000000,0.000000}%
\pgfsetstrokecolor{currentstroke}%
\pgfsetdash{}{0pt}%
\pgfpathmoveto{\pgfqpoint{2.551359in}{0.853641in}}%
\pgfpathlineto{\pgfqpoint{2.313662in}{1.847116in}}%
\pgfusepath{stroke}%
\end{pgfscope}%
\begin{pgfscope}%
\pgfpathrectangle{\pgfqpoint{0.100000in}{0.212622in}}{\pgfqpoint{3.696000in}{3.696000in}}%
\pgfusepath{clip}%
\pgfsetrectcap%
\pgfsetroundjoin%
\pgfsetlinewidth{1.505625pt}%
\definecolor{currentstroke}{rgb}{1.000000,0.000000,0.000000}%
\pgfsetstrokecolor{currentstroke}%
\pgfsetdash{}{0pt}%
\pgfpathmoveto{\pgfqpoint{2.551359in}{0.853641in}}%
\pgfpathlineto{\pgfqpoint{2.313662in}{1.847116in}}%
\pgfusepath{stroke}%
\end{pgfscope}%
\begin{pgfscope}%
\pgfpathrectangle{\pgfqpoint{0.100000in}{0.212622in}}{\pgfqpoint{3.696000in}{3.696000in}}%
\pgfusepath{clip}%
\pgfsetrectcap%
\pgfsetroundjoin%
\pgfsetlinewidth{1.505625pt}%
\definecolor{currentstroke}{rgb}{1.000000,0.000000,0.000000}%
\pgfsetstrokecolor{currentstroke}%
\pgfsetdash{}{0pt}%
\pgfpathmoveto{\pgfqpoint{2.551359in}{0.853641in}}%
\pgfpathlineto{\pgfqpoint{2.313662in}{1.847116in}}%
\pgfusepath{stroke}%
\end{pgfscope}%
\begin{pgfscope}%
\pgfpathrectangle{\pgfqpoint{0.100000in}{0.212622in}}{\pgfqpoint{3.696000in}{3.696000in}}%
\pgfusepath{clip}%
\pgfsetrectcap%
\pgfsetroundjoin%
\pgfsetlinewidth{1.505625pt}%
\definecolor{currentstroke}{rgb}{1.000000,0.000000,0.000000}%
\pgfsetstrokecolor{currentstroke}%
\pgfsetdash{}{0pt}%
\pgfpathmoveto{\pgfqpoint{2.551359in}{0.853641in}}%
\pgfpathlineto{\pgfqpoint{2.313662in}{1.847116in}}%
\pgfusepath{stroke}%
\end{pgfscope}%
\begin{pgfscope}%
\pgfpathrectangle{\pgfqpoint{0.100000in}{0.212622in}}{\pgfqpoint{3.696000in}{3.696000in}}%
\pgfusepath{clip}%
\pgfsetrectcap%
\pgfsetroundjoin%
\pgfsetlinewidth{1.505625pt}%
\definecolor{currentstroke}{rgb}{1.000000,0.000000,0.000000}%
\pgfsetstrokecolor{currentstroke}%
\pgfsetdash{}{0pt}%
\pgfpathmoveto{\pgfqpoint{2.551359in}{0.853641in}}%
\pgfpathlineto{\pgfqpoint{2.313662in}{1.847116in}}%
\pgfusepath{stroke}%
\end{pgfscope}%
\begin{pgfscope}%
\pgfpathrectangle{\pgfqpoint{0.100000in}{0.212622in}}{\pgfqpoint{3.696000in}{3.696000in}}%
\pgfusepath{clip}%
\pgfsetrectcap%
\pgfsetroundjoin%
\pgfsetlinewidth{1.505625pt}%
\definecolor{currentstroke}{rgb}{1.000000,0.000000,0.000000}%
\pgfsetstrokecolor{currentstroke}%
\pgfsetdash{}{0pt}%
\pgfpathmoveto{\pgfqpoint{2.551359in}{0.853641in}}%
\pgfpathlineto{\pgfqpoint{2.313662in}{1.847116in}}%
\pgfusepath{stroke}%
\end{pgfscope}%
\begin{pgfscope}%
\pgfpathrectangle{\pgfqpoint{0.100000in}{0.212622in}}{\pgfqpoint{3.696000in}{3.696000in}}%
\pgfusepath{clip}%
\pgfsetrectcap%
\pgfsetroundjoin%
\pgfsetlinewidth{1.505625pt}%
\definecolor{currentstroke}{rgb}{1.000000,0.000000,0.000000}%
\pgfsetstrokecolor{currentstroke}%
\pgfsetdash{}{0pt}%
\pgfpathmoveto{\pgfqpoint{2.551359in}{0.853641in}}%
\pgfpathlineto{\pgfqpoint{2.313662in}{1.847116in}}%
\pgfusepath{stroke}%
\end{pgfscope}%
\begin{pgfscope}%
\pgfpathrectangle{\pgfqpoint{0.100000in}{0.212622in}}{\pgfqpoint{3.696000in}{3.696000in}}%
\pgfusepath{clip}%
\pgfsetrectcap%
\pgfsetroundjoin%
\pgfsetlinewidth{1.505625pt}%
\definecolor{currentstroke}{rgb}{1.000000,0.000000,0.000000}%
\pgfsetstrokecolor{currentstroke}%
\pgfsetdash{}{0pt}%
\pgfpathmoveto{\pgfqpoint{2.551359in}{0.853641in}}%
\pgfpathlineto{\pgfqpoint{2.313662in}{1.847116in}}%
\pgfusepath{stroke}%
\end{pgfscope}%
\begin{pgfscope}%
\pgfpathrectangle{\pgfqpoint{0.100000in}{0.212622in}}{\pgfqpoint{3.696000in}{3.696000in}}%
\pgfusepath{clip}%
\pgfsetrectcap%
\pgfsetroundjoin%
\pgfsetlinewidth{1.505625pt}%
\definecolor{currentstroke}{rgb}{1.000000,0.000000,0.000000}%
\pgfsetstrokecolor{currentstroke}%
\pgfsetdash{}{0pt}%
\pgfpathmoveto{\pgfqpoint{2.551359in}{0.853641in}}%
\pgfpathlineto{\pgfqpoint{2.313662in}{1.847116in}}%
\pgfusepath{stroke}%
\end{pgfscope}%
\begin{pgfscope}%
\pgfpathrectangle{\pgfqpoint{0.100000in}{0.212622in}}{\pgfqpoint{3.696000in}{3.696000in}}%
\pgfusepath{clip}%
\pgfsetrectcap%
\pgfsetroundjoin%
\pgfsetlinewidth{1.505625pt}%
\definecolor{currentstroke}{rgb}{1.000000,0.000000,0.000000}%
\pgfsetstrokecolor{currentstroke}%
\pgfsetdash{}{0pt}%
\pgfpathmoveto{\pgfqpoint{2.551359in}{0.853641in}}%
\pgfpathlineto{\pgfqpoint{2.313662in}{1.847116in}}%
\pgfusepath{stroke}%
\end{pgfscope}%
\begin{pgfscope}%
\pgfpathrectangle{\pgfqpoint{0.100000in}{0.212622in}}{\pgfqpoint{3.696000in}{3.696000in}}%
\pgfusepath{clip}%
\pgfsetrectcap%
\pgfsetroundjoin%
\pgfsetlinewidth{1.505625pt}%
\definecolor{currentstroke}{rgb}{1.000000,0.000000,0.000000}%
\pgfsetstrokecolor{currentstroke}%
\pgfsetdash{}{0pt}%
\pgfpathmoveto{\pgfqpoint{2.551359in}{0.853641in}}%
\pgfpathlineto{\pgfqpoint{2.313662in}{1.847116in}}%
\pgfusepath{stroke}%
\end{pgfscope}%
\begin{pgfscope}%
\pgfpathrectangle{\pgfqpoint{0.100000in}{0.212622in}}{\pgfqpoint{3.696000in}{3.696000in}}%
\pgfusepath{clip}%
\pgfsetrectcap%
\pgfsetroundjoin%
\pgfsetlinewidth{1.505625pt}%
\definecolor{currentstroke}{rgb}{1.000000,0.000000,0.000000}%
\pgfsetstrokecolor{currentstroke}%
\pgfsetdash{}{0pt}%
\pgfpathmoveto{\pgfqpoint{2.551359in}{0.853641in}}%
\pgfpathlineto{\pgfqpoint{2.313662in}{1.847116in}}%
\pgfusepath{stroke}%
\end{pgfscope}%
\begin{pgfscope}%
\pgfpathrectangle{\pgfqpoint{0.100000in}{0.212622in}}{\pgfqpoint{3.696000in}{3.696000in}}%
\pgfusepath{clip}%
\pgfsetrectcap%
\pgfsetroundjoin%
\pgfsetlinewidth{1.505625pt}%
\definecolor{currentstroke}{rgb}{1.000000,0.000000,0.000000}%
\pgfsetstrokecolor{currentstroke}%
\pgfsetdash{}{0pt}%
\pgfpathmoveto{\pgfqpoint{2.551359in}{0.853641in}}%
\pgfpathlineto{\pgfqpoint{2.313662in}{1.847116in}}%
\pgfusepath{stroke}%
\end{pgfscope}%
\begin{pgfscope}%
\pgfpathrectangle{\pgfqpoint{0.100000in}{0.212622in}}{\pgfqpoint{3.696000in}{3.696000in}}%
\pgfusepath{clip}%
\pgfsetrectcap%
\pgfsetroundjoin%
\pgfsetlinewidth{1.505625pt}%
\definecolor{currentstroke}{rgb}{1.000000,0.000000,0.000000}%
\pgfsetstrokecolor{currentstroke}%
\pgfsetdash{}{0pt}%
\pgfpathmoveto{\pgfqpoint{2.551359in}{0.853641in}}%
\pgfpathlineto{\pgfqpoint{2.313662in}{1.847116in}}%
\pgfusepath{stroke}%
\end{pgfscope}%
\begin{pgfscope}%
\pgfpathrectangle{\pgfqpoint{0.100000in}{0.212622in}}{\pgfqpoint{3.696000in}{3.696000in}}%
\pgfusepath{clip}%
\pgfsetrectcap%
\pgfsetroundjoin%
\pgfsetlinewidth{1.505625pt}%
\definecolor{currentstroke}{rgb}{1.000000,0.000000,0.000000}%
\pgfsetstrokecolor{currentstroke}%
\pgfsetdash{}{0pt}%
\pgfpathmoveto{\pgfqpoint{2.551359in}{0.853641in}}%
\pgfpathlineto{\pgfqpoint{2.313662in}{1.847116in}}%
\pgfusepath{stroke}%
\end{pgfscope}%
\begin{pgfscope}%
\pgfpathrectangle{\pgfqpoint{0.100000in}{0.212622in}}{\pgfqpoint{3.696000in}{3.696000in}}%
\pgfusepath{clip}%
\pgfsetrectcap%
\pgfsetroundjoin%
\pgfsetlinewidth{1.505625pt}%
\definecolor{currentstroke}{rgb}{1.000000,0.000000,0.000000}%
\pgfsetstrokecolor{currentstroke}%
\pgfsetdash{}{0pt}%
\pgfpathmoveto{\pgfqpoint{2.551359in}{0.853641in}}%
\pgfpathlineto{\pgfqpoint{2.313662in}{1.847116in}}%
\pgfusepath{stroke}%
\end{pgfscope}%
\begin{pgfscope}%
\pgfpathrectangle{\pgfqpoint{0.100000in}{0.212622in}}{\pgfqpoint{3.696000in}{3.696000in}}%
\pgfusepath{clip}%
\pgfsetrectcap%
\pgfsetroundjoin%
\pgfsetlinewidth{1.505625pt}%
\definecolor{currentstroke}{rgb}{1.000000,0.000000,0.000000}%
\pgfsetstrokecolor{currentstroke}%
\pgfsetdash{}{0pt}%
\pgfpathmoveto{\pgfqpoint{2.551359in}{0.853641in}}%
\pgfpathlineto{\pgfqpoint{2.313662in}{1.847116in}}%
\pgfusepath{stroke}%
\end{pgfscope}%
\begin{pgfscope}%
\pgfpathrectangle{\pgfqpoint{0.100000in}{0.212622in}}{\pgfqpoint{3.696000in}{3.696000in}}%
\pgfusepath{clip}%
\pgfsetrectcap%
\pgfsetroundjoin%
\pgfsetlinewidth{1.505625pt}%
\definecolor{currentstroke}{rgb}{1.000000,0.000000,0.000000}%
\pgfsetstrokecolor{currentstroke}%
\pgfsetdash{}{0pt}%
\pgfpathmoveto{\pgfqpoint{2.551359in}{0.853641in}}%
\pgfpathlineto{\pgfqpoint{2.313662in}{1.847116in}}%
\pgfusepath{stroke}%
\end{pgfscope}%
\begin{pgfscope}%
\pgfpathrectangle{\pgfqpoint{0.100000in}{0.212622in}}{\pgfqpoint{3.696000in}{3.696000in}}%
\pgfusepath{clip}%
\pgfsetrectcap%
\pgfsetroundjoin%
\pgfsetlinewidth{1.505625pt}%
\definecolor{currentstroke}{rgb}{1.000000,0.000000,0.000000}%
\pgfsetstrokecolor{currentstroke}%
\pgfsetdash{}{0pt}%
\pgfpathmoveto{\pgfqpoint{2.551359in}{0.853641in}}%
\pgfpathlineto{\pgfqpoint{2.313662in}{1.847116in}}%
\pgfusepath{stroke}%
\end{pgfscope}%
\begin{pgfscope}%
\pgfpathrectangle{\pgfqpoint{0.100000in}{0.212622in}}{\pgfqpoint{3.696000in}{3.696000in}}%
\pgfusepath{clip}%
\pgfsetrectcap%
\pgfsetroundjoin%
\pgfsetlinewidth{1.505625pt}%
\definecolor{currentstroke}{rgb}{1.000000,0.000000,0.000000}%
\pgfsetstrokecolor{currentstroke}%
\pgfsetdash{}{0pt}%
\pgfpathmoveto{\pgfqpoint{2.551642in}{0.854379in}}%
\pgfpathlineto{\pgfqpoint{2.313662in}{1.847116in}}%
\pgfusepath{stroke}%
\end{pgfscope}%
\begin{pgfscope}%
\pgfpathrectangle{\pgfqpoint{0.100000in}{0.212622in}}{\pgfqpoint{3.696000in}{3.696000in}}%
\pgfusepath{clip}%
\pgfsetrectcap%
\pgfsetroundjoin%
\pgfsetlinewidth{1.505625pt}%
\definecolor{currentstroke}{rgb}{1.000000,0.000000,0.000000}%
\pgfsetstrokecolor{currentstroke}%
\pgfsetdash{}{0pt}%
\pgfpathmoveto{\pgfqpoint{2.551731in}{0.854814in}}%
\pgfpathlineto{\pgfqpoint{2.313662in}{1.847116in}}%
\pgfusepath{stroke}%
\end{pgfscope}%
\begin{pgfscope}%
\pgfpathrectangle{\pgfqpoint{0.100000in}{0.212622in}}{\pgfqpoint{3.696000in}{3.696000in}}%
\pgfusepath{clip}%
\pgfsetrectcap%
\pgfsetroundjoin%
\pgfsetlinewidth{1.505625pt}%
\definecolor{currentstroke}{rgb}{1.000000,0.000000,0.000000}%
\pgfsetstrokecolor{currentstroke}%
\pgfsetdash{}{0pt}%
\pgfpathmoveto{\pgfqpoint{2.552917in}{0.856939in}}%
\pgfpathlineto{\pgfqpoint{2.313662in}{1.847116in}}%
\pgfusepath{stroke}%
\end{pgfscope}%
\begin{pgfscope}%
\pgfpathrectangle{\pgfqpoint{0.100000in}{0.212622in}}{\pgfqpoint{3.696000in}{3.696000in}}%
\pgfusepath{clip}%
\pgfsetrectcap%
\pgfsetroundjoin%
\pgfsetlinewidth{1.505625pt}%
\definecolor{currentstroke}{rgb}{1.000000,0.000000,0.000000}%
\pgfsetstrokecolor{currentstroke}%
\pgfsetdash{}{0pt}%
\pgfpathmoveto{\pgfqpoint{2.555401in}{0.860219in}}%
\pgfpathlineto{\pgfqpoint{2.313662in}{1.847116in}}%
\pgfusepath{stroke}%
\end{pgfscope}%
\begin{pgfscope}%
\pgfpathrectangle{\pgfqpoint{0.100000in}{0.212622in}}{\pgfqpoint{3.696000in}{3.696000in}}%
\pgfusepath{clip}%
\pgfsetrectcap%
\pgfsetroundjoin%
\pgfsetlinewidth{1.505625pt}%
\definecolor{currentstroke}{rgb}{1.000000,0.000000,0.000000}%
\pgfsetstrokecolor{currentstroke}%
\pgfsetdash{}{0pt}%
\pgfpathmoveto{\pgfqpoint{2.558007in}{0.865664in}}%
\pgfpathlineto{\pgfqpoint{2.324438in}{1.856893in}}%
\pgfusepath{stroke}%
\end{pgfscope}%
\begin{pgfscope}%
\pgfpathrectangle{\pgfqpoint{0.100000in}{0.212622in}}{\pgfqpoint{3.696000in}{3.696000in}}%
\pgfusepath{clip}%
\pgfsetrectcap%
\pgfsetroundjoin%
\pgfsetlinewidth{1.505625pt}%
\definecolor{currentstroke}{rgb}{1.000000,0.000000,0.000000}%
\pgfsetstrokecolor{currentstroke}%
\pgfsetdash{}{0pt}%
\pgfpathmoveto{\pgfqpoint{2.559590in}{0.872828in}}%
\pgfpathlineto{\pgfqpoint{2.324438in}{1.856893in}}%
\pgfusepath{stroke}%
\end{pgfscope}%
\begin{pgfscope}%
\pgfpathrectangle{\pgfqpoint{0.100000in}{0.212622in}}{\pgfqpoint{3.696000in}{3.696000in}}%
\pgfusepath{clip}%
\pgfsetrectcap%
\pgfsetroundjoin%
\pgfsetlinewidth{1.505625pt}%
\definecolor{currentstroke}{rgb}{1.000000,0.000000,0.000000}%
\pgfsetstrokecolor{currentstroke}%
\pgfsetdash{}{0pt}%
\pgfpathmoveto{\pgfqpoint{2.562082in}{0.879266in}}%
\pgfpathlineto{\pgfqpoint{2.324438in}{1.856893in}}%
\pgfusepath{stroke}%
\end{pgfscope}%
\begin{pgfscope}%
\pgfpathrectangle{\pgfqpoint{0.100000in}{0.212622in}}{\pgfqpoint{3.696000in}{3.696000in}}%
\pgfusepath{clip}%
\pgfsetrectcap%
\pgfsetroundjoin%
\pgfsetlinewidth{1.505625pt}%
\definecolor{currentstroke}{rgb}{1.000000,0.000000,0.000000}%
\pgfsetstrokecolor{currentstroke}%
\pgfsetdash{}{0pt}%
\pgfpathmoveto{\pgfqpoint{2.567082in}{0.887170in}}%
\pgfpathlineto{\pgfqpoint{2.335196in}{1.866655in}}%
\pgfusepath{stroke}%
\end{pgfscope}%
\begin{pgfscope}%
\pgfpathrectangle{\pgfqpoint{0.100000in}{0.212622in}}{\pgfqpoint{3.696000in}{3.696000in}}%
\pgfusepath{clip}%
\pgfsetrectcap%
\pgfsetroundjoin%
\pgfsetlinewidth{1.505625pt}%
\definecolor{currentstroke}{rgb}{1.000000,0.000000,0.000000}%
\pgfsetstrokecolor{currentstroke}%
\pgfsetdash{}{0pt}%
\pgfpathmoveto{\pgfqpoint{2.570005in}{0.890982in}}%
\pgfpathlineto{\pgfqpoint{2.335196in}{1.866655in}}%
\pgfusepath{stroke}%
\end{pgfscope}%
\begin{pgfscope}%
\pgfpathrectangle{\pgfqpoint{0.100000in}{0.212622in}}{\pgfqpoint{3.696000in}{3.696000in}}%
\pgfusepath{clip}%
\pgfsetrectcap%
\pgfsetroundjoin%
\pgfsetlinewidth{1.505625pt}%
\definecolor{currentstroke}{rgb}{1.000000,0.000000,0.000000}%
\pgfsetstrokecolor{currentstroke}%
\pgfsetdash{}{0pt}%
\pgfpathmoveto{\pgfqpoint{2.573036in}{0.898290in}}%
\pgfpathlineto{\pgfqpoint{2.335196in}{1.866655in}}%
\pgfusepath{stroke}%
\end{pgfscope}%
\begin{pgfscope}%
\pgfpathrectangle{\pgfqpoint{0.100000in}{0.212622in}}{\pgfqpoint{3.696000in}{3.696000in}}%
\pgfusepath{clip}%
\pgfsetrectcap%
\pgfsetroundjoin%
\pgfsetlinewidth{1.505625pt}%
\definecolor{currentstroke}{rgb}{1.000000,0.000000,0.000000}%
\pgfsetstrokecolor{currentstroke}%
\pgfsetdash{}{0pt}%
\pgfpathmoveto{\pgfqpoint{2.574406in}{0.902509in}}%
\pgfpathlineto{\pgfqpoint{2.335196in}{1.866655in}}%
\pgfusepath{stroke}%
\end{pgfscope}%
\begin{pgfscope}%
\pgfpathrectangle{\pgfqpoint{0.100000in}{0.212622in}}{\pgfqpoint{3.696000in}{3.696000in}}%
\pgfusepath{clip}%
\pgfsetrectcap%
\pgfsetroundjoin%
\pgfsetlinewidth{1.505625pt}%
\definecolor{currentstroke}{rgb}{1.000000,0.000000,0.000000}%
\pgfsetstrokecolor{currentstroke}%
\pgfsetdash{}{0pt}%
\pgfpathmoveto{\pgfqpoint{2.576624in}{0.908012in}}%
\pgfpathlineto{\pgfqpoint{2.345937in}{1.876400in}}%
\pgfusepath{stroke}%
\end{pgfscope}%
\begin{pgfscope}%
\pgfpathrectangle{\pgfqpoint{0.100000in}{0.212622in}}{\pgfqpoint{3.696000in}{3.696000in}}%
\pgfusepath{clip}%
\pgfsetrectcap%
\pgfsetroundjoin%
\pgfsetlinewidth{1.505625pt}%
\definecolor{currentstroke}{rgb}{1.000000,0.000000,0.000000}%
\pgfsetstrokecolor{currentstroke}%
\pgfsetdash{}{0pt}%
\pgfpathmoveto{\pgfqpoint{2.580660in}{0.912737in}}%
\pgfpathlineto{\pgfqpoint{2.345937in}{1.876400in}}%
\pgfusepath{stroke}%
\end{pgfscope}%
\begin{pgfscope}%
\pgfpathrectangle{\pgfqpoint{0.100000in}{0.212622in}}{\pgfqpoint{3.696000in}{3.696000in}}%
\pgfusepath{clip}%
\pgfsetrectcap%
\pgfsetroundjoin%
\pgfsetlinewidth{1.505625pt}%
\definecolor{currentstroke}{rgb}{1.000000,0.000000,0.000000}%
\pgfsetstrokecolor{currentstroke}%
\pgfsetdash{}{0pt}%
\pgfpathmoveto{\pgfqpoint{2.584954in}{0.919401in}}%
\pgfpathlineto{\pgfqpoint{2.345937in}{1.876400in}}%
\pgfusepath{stroke}%
\end{pgfscope}%
\begin{pgfscope}%
\pgfpathrectangle{\pgfqpoint{0.100000in}{0.212622in}}{\pgfqpoint{3.696000in}{3.696000in}}%
\pgfusepath{clip}%
\pgfsetrectcap%
\pgfsetroundjoin%
\pgfsetlinewidth{1.505625pt}%
\definecolor{currentstroke}{rgb}{1.000000,0.000000,0.000000}%
\pgfsetstrokecolor{currentstroke}%
\pgfsetdash{}{0pt}%
\pgfpathmoveto{\pgfqpoint{2.587552in}{0.922569in}}%
\pgfpathlineto{\pgfqpoint{2.345937in}{1.876400in}}%
\pgfusepath{stroke}%
\end{pgfscope}%
\begin{pgfscope}%
\pgfpathrectangle{\pgfqpoint{0.100000in}{0.212622in}}{\pgfqpoint{3.696000in}{3.696000in}}%
\pgfusepath{clip}%
\pgfsetrectcap%
\pgfsetroundjoin%
\pgfsetlinewidth{1.505625pt}%
\definecolor{currentstroke}{rgb}{1.000000,0.000000,0.000000}%
\pgfsetstrokecolor{currentstroke}%
\pgfsetdash{}{0pt}%
\pgfpathmoveto{\pgfqpoint{2.590688in}{0.927021in}}%
\pgfpathlineto{\pgfqpoint{2.356660in}{1.886129in}}%
\pgfusepath{stroke}%
\end{pgfscope}%
\begin{pgfscope}%
\pgfpathrectangle{\pgfqpoint{0.100000in}{0.212622in}}{\pgfqpoint{3.696000in}{3.696000in}}%
\pgfusepath{clip}%
\pgfsetrectcap%
\pgfsetroundjoin%
\pgfsetlinewidth{1.505625pt}%
\definecolor{currentstroke}{rgb}{1.000000,0.000000,0.000000}%
\pgfsetstrokecolor{currentstroke}%
\pgfsetdash{}{0pt}%
\pgfpathmoveto{\pgfqpoint{2.592864in}{0.933233in}}%
\pgfpathlineto{\pgfqpoint{2.356660in}{1.886129in}}%
\pgfusepath{stroke}%
\end{pgfscope}%
\begin{pgfscope}%
\pgfpathrectangle{\pgfqpoint{0.100000in}{0.212622in}}{\pgfqpoint{3.696000in}{3.696000in}}%
\pgfusepath{clip}%
\pgfsetrectcap%
\pgfsetroundjoin%
\pgfsetlinewidth{1.505625pt}%
\definecolor{currentstroke}{rgb}{1.000000,0.000000,0.000000}%
\pgfsetstrokecolor{currentstroke}%
\pgfsetdash{}{0pt}%
\pgfpathmoveto{\pgfqpoint{2.593262in}{0.937009in}}%
\pgfpathlineto{\pgfqpoint{2.356660in}{1.886129in}}%
\pgfusepath{stroke}%
\end{pgfscope}%
\begin{pgfscope}%
\pgfpathrectangle{\pgfqpoint{0.100000in}{0.212622in}}{\pgfqpoint{3.696000in}{3.696000in}}%
\pgfusepath{clip}%
\pgfsetrectcap%
\pgfsetroundjoin%
\pgfsetlinewidth{1.505625pt}%
\definecolor{currentstroke}{rgb}{1.000000,0.000000,0.000000}%
\pgfsetstrokecolor{currentstroke}%
\pgfsetdash{}{0pt}%
\pgfpathmoveto{\pgfqpoint{2.593124in}{0.941414in}}%
\pgfpathlineto{\pgfqpoint{2.356660in}{1.886129in}}%
\pgfusepath{stroke}%
\end{pgfscope}%
\begin{pgfscope}%
\pgfpathrectangle{\pgfqpoint{0.100000in}{0.212622in}}{\pgfqpoint{3.696000in}{3.696000in}}%
\pgfusepath{clip}%
\pgfsetrectcap%
\pgfsetroundjoin%
\pgfsetlinewidth{1.505625pt}%
\definecolor{currentstroke}{rgb}{1.000000,0.000000,0.000000}%
\pgfsetstrokecolor{currentstroke}%
\pgfsetdash{}{0pt}%
\pgfpathmoveto{\pgfqpoint{2.593354in}{0.943782in}}%
\pgfpathlineto{\pgfqpoint{2.356660in}{1.886129in}}%
\pgfusepath{stroke}%
\end{pgfscope}%
\begin{pgfscope}%
\pgfpathrectangle{\pgfqpoint{0.100000in}{0.212622in}}{\pgfqpoint{3.696000in}{3.696000in}}%
\pgfusepath{clip}%
\pgfsetrectcap%
\pgfsetroundjoin%
\pgfsetlinewidth{1.505625pt}%
\definecolor{currentstroke}{rgb}{1.000000,0.000000,0.000000}%
\pgfsetstrokecolor{currentstroke}%
\pgfsetdash{}{0pt}%
\pgfpathmoveto{\pgfqpoint{2.594563in}{0.947147in}}%
\pgfpathlineto{\pgfqpoint{2.367365in}{1.895842in}}%
\pgfusepath{stroke}%
\end{pgfscope}%
\begin{pgfscope}%
\pgfpathrectangle{\pgfqpoint{0.100000in}{0.212622in}}{\pgfqpoint{3.696000in}{3.696000in}}%
\pgfusepath{clip}%
\pgfsetrectcap%
\pgfsetroundjoin%
\pgfsetlinewidth{1.505625pt}%
\definecolor{currentstroke}{rgb}{1.000000,0.000000,0.000000}%
\pgfsetstrokecolor{currentstroke}%
\pgfsetdash{}{0pt}%
\pgfpathmoveto{\pgfqpoint{2.597109in}{0.950184in}}%
\pgfpathlineto{\pgfqpoint{2.367365in}{1.895842in}}%
\pgfusepath{stroke}%
\end{pgfscope}%
\begin{pgfscope}%
\pgfpathrectangle{\pgfqpoint{0.100000in}{0.212622in}}{\pgfqpoint{3.696000in}{3.696000in}}%
\pgfusepath{clip}%
\pgfsetrectcap%
\pgfsetroundjoin%
\pgfsetlinewidth{1.505625pt}%
\definecolor{currentstroke}{rgb}{1.000000,0.000000,0.000000}%
\pgfsetstrokecolor{currentstroke}%
\pgfsetdash{}{0pt}%
\pgfpathmoveto{\pgfqpoint{2.598457in}{0.951737in}}%
\pgfpathlineto{\pgfqpoint{2.367365in}{1.895842in}}%
\pgfusepath{stroke}%
\end{pgfscope}%
\begin{pgfscope}%
\pgfpathrectangle{\pgfqpoint{0.100000in}{0.212622in}}{\pgfqpoint{3.696000in}{3.696000in}}%
\pgfusepath{clip}%
\pgfsetrectcap%
\pgfsetroundjoin%
\pgfsetlinewidth{1.505625pt}%
\definecolor{currentstroke}{rgb}{1.000000,0.000000,0.000000}%
\pgfsetstrokecolor{currentstroke}%
\pgfsetdash{}{0pt}%
\pgfpathmoveto{\pgfqpoint{2.599984in}{0.954503in}}%
\pgfpathlineto{\pgfqpoint{2.367365in}{1.895842in}}%
\pgfusepath{stroke}%
\end{pgfscope}%
\begin{pgfscope}%
\pgfpathrectangle{\pgfqpoint{0.100000in}{0.212622in}}{\pgfqpoint{3.696000in}{3.696000in}}%
\pgfusepath{clip}%
\pgfsetrectcap%
\pgfsetroundjoin%
\pgfsetlinewidth{1.505625pt}%
\definecolor{currentstroke}{rgb}{1.000000,0.000000,0.000000}%
\pgfsetstrokecolor{currentstroke}%
\pgfsetdash{}{0pt}%
\pgfpathmoveto{\pgfqpoint{2.600407in}{0.956503in}}%
\pgfpathlineto{\pgfqpoint{2.367365in}{1.895842in}}%
\pgfusepath{stroke}%
\end{pgfscope}%
\begin{pgfscope}%
\pgfpathrectangle{\pgfqpoint{0.100000in}{0.212622in}}{\pgfqpoint{3.696000in}{3.696000in}}%
\pgfusepath{clip}%
\pgfsetrectcap%
\pgfsetroundjoin%
\pgfsetlinewidth{1.505625pt}%
\definecolor{currentstroke}{rgb}{1.000000,0.000000,0.000000}%
\pgfsetstrokecolor{currentstroke}%
\pgfsetdash{}{0pt}%
\pgfpathmoveto{\pgfqpoint{2.600558in}{0.959267in}}%
\pgfpathlineto{\pgfqpoint{2.367365in}{1.895842in}}%
\pgfusepath{stroke}%
\end{pgfscope}%
\begin{pgfscope}%
\pgfpathrectangle{\pgfqpoint{0.100000in}{0.212622in}}{\pgfqpoint{3.696000in}{3.696000in}}%
\pgfusepath{clip}%
\pgfsetrectcap%
\pgfsetroundjoin%
\pgfsetlinewidth{1.505625pt}%
\definecolor{currentstroke}{rgb}{1.000000,0.000000,0.000000}%
\pgfsetstrokecolor{currentstroke}%
\pgfsetdash{}{0pt}%
\pgfpathmoveto{\pgfqpoint{2.600715in}{0.960595in}}%
\pgfpathlineto{\pgfqpoint{2.367365in}{1.895842in}}%
\pgfusepath{stroke}%
\end{pgfscope}%
\begin{pgfscope}%
\pgfpathrectangle{\pgfqpoint{0.100000in}{0.212622in}}{\pgfqpoint{3.696000in}{3.696000in}}%
\pgfusepath{clip}%
\pgfsetrectcap%
\pgfsetroundjoin%
\pgfsetlinewidth{1.505625pt}%
\definecolor{currentstroke}{rgb}{1.000000,0.000000,0.000000}%
\pgfsetstrokecolor{currentstroke}%
\pgfsetdash{}{0pt}%
\pgfpathmoveto{\pgfqpoint{2.602215in}{0.963612in}}%
\pgfpathlineto{\pgfqpoint{2.367365in}{1.895842in}}%
\pgfusepath{stroke}%
\end{pgfscope}%
\begin{pgfscope}%
\pgfpathrectangle{\pgfqpoint{0.100000in}{0.212622in}}{\pgfqpoint{3.696000in}{3.696000in}}%
\pgfusepath{clip}%
\pgfsetrectcap%
\pgfsetroundjoin%
\pgfsetlinewidth{1.505625pt}%
\definecolor{currentstroke}{rgb}{1.000000,0.000000,0.000000}%
\pgfsetstrokecolor{currentstroke}%
\pgfsetdash{}{0pt}%
\pgfpathmoveto{\pgfqpoint{2.605085in}{0.968208in}}%
\pgfpathlineto{\pgfqpoint{2.378053in}{1.905539in}}%
\pgfusepath{stroke}%
\end{pgfscope}%
\begin{pgfscope}%
\pgfpathrectangle{\pgfqpoint{0.100000in}{0.212622in}}{\pgfqpoint{3.696000in}{3.696000in}}%
\pgfusepath{clip}%
\pgfsetrectcap%
\pgfsetroundjoin%
\pgfsetlinewidth{1.505625pt}%
\definecolor{currentstroke}{rgb}{1.000000,0.000000,0.000000}%
\pgfsetstrokecolor{currentstroke}%
\pgfsetdash{}{0pt}%
\pgfpathmoveto{\pgfqpoint{2.606754in}{0.970253in}}%
\pgfpathlineto{\pgfqpoint{2.378053in}{1.905539in}}%
\pgfusepath{stroke}%
\end{pgfscope}%
\begin{pgfscope}%
\pgfpathrectangle{\pgfqpoint{0.100000in}{0.212622in}}{\pgfqpoint{3.696000in}{3.696000in}}%
\pgfusepath{clip}%
\pgfsetrectcap%
\pgfsetroundjoin%
\pgfsetlinewidth{1.505625pt}%
\definecolor{currentstroke}{rgb}{1.000000,0.000000,0.000000}%
\pgfsetstrokecolor{currentstroke}%
\pgfsetdash{}{0pt}%
\pgfpathmoveto{\pgfqpoint{2.609890in}{0.974891in}}%
\pgfpathlineto{\pgfqpoint{2.378053in}{1.905539in}}%
\pgfusepath{stroke}%
\end{pgfscope}%
\begin{pgfscope}%
\pgfpathrectangle{\pgfqpoint{0.100000in}{0.212622in}}{\pgfqpoint{3.696000in}{3.696000in}}%
\pgfusepath{clip}%
\pgfsetrectcap%
\pgfsetroundjoin%
\pgfsetlinewidth{1.505625pt}%
\definecolor{currentstroke}{rgb}{1.000000,0.000000,0.000000}%
\pgfsetstrokecolor{currentstroke}%
\pgfsetdash{}{0pt}%
\pgfpathmoveto{\pgfqpoint{2.611269in}{0.983837in}}%
\pgfpathlineto{\pgfqpoint{2.378053in}{1.905539in}}%
\pgfusepath{stroke}%
\end{pgfscope}%
\begin{pgfscope}%
\pgfpathrectangle{\pgfqpoint{0.100000in}{0.212622in}}{\pgfqpoint{3.696000in}{3.696000in}}%
\pgfusepath{clip}%
\pgfsetrectcap%
\pgfsetroundjoin%
\pgfsetlinewidth{1.505625pt}%
\definecolor{currentstroke}{rgb}{1.000000,0.000000,0.000000}%
\pgfsetstrokecolor{currentstroke}%
\pgfsetdash{}{0pt}%
\pgfpathmoveto{\pgfqpoint{2.611895in}{0.988715in}}%
\pgfpathlineto{\pgfqpoint{2.378053in}{1.905539in}}%
\pgfusepath{stroke}%
\end{pgfscope}%
\begin{pgfscope}%
\pgfpathrectangle{\pgfqpoint{0.100000in}{0.212622in}}{\pgfqpoint{3.696000in}{3.696000in}}%
\pgfusepath{clip}%
\pgfsetrectcap%
\pgfsetroundjoin%
\pgfsetlinewidth{1.505625pt}%
\definecolor{currentstroke}{rgb}{1.000000,0.000000,0.000000}%
\pgfsetstrokecolor{currentstroke}%
\pgfsetdash{}{0pt}%
\pgfpathmoveto{\pgfqpoint{2.613384in}{0.994215in}}%
\pgfpathlineto{\pgfqpoint{2.388723in}{1.915220in}}%
\pgfusepath{stroke}%
\end{pgfscope}%
\begin{pgfscope}%
\pgfpathrectangle{\pgfqpoint{0.100000in}{0.212622in}}{\pgfqpoint{3.696000in}{3.696000in}}%
\pgfusepath{clip}%
\pgfsetrectcap%
\pgfsetroundjoin%
\pgfsetlinewidth{1.505625pt}%
\definecolor{currentstroke}{rgb}{1.000000,0.000000,0.000000}%
\pgfsetstrokecolor{currentstroke}%
\pgfsetdash{}{0pt}%
\pgfpathmoveto{\pgfqpoint{2.617680in}{1.000469in}}%
\pgfpathlineto{\pgfqpoint{2.388723in}{1.915220in}}%
\pgfusepath{stroke}%
\end{pgfscope}%
\begin{pgfscope}%
\pgfpathrectangle{\pgfqpoint{0.100000in}{0.212622in}}{\pgfqpoint{3.696000in}{3.696000in}}%
\pgfusepath{clip}%
\pgfsetrectcap%
\pgfsetroundjoin%
\pgfsetlinewidth{1.505625pt}%
\definecolor{currentstroke}{rgb}{1.000000,0.000000,0.000000}%
\pgfsetstrokecolor{currentstroke}%
\pgfsetdash{}{0pt}%
\pgfpathmoveto{\pgfqpoint{2.620165in}{1.003556in}}%
\pgfpathlineto{\pgfqpoint{2.388723in}{1.915220in}}%
\pgfusepath{stroke}%
\end{pgfscope}%
\begin{pgfscope}%
\pgfpathrectangle{\pgfqpoint{0.100000in}{0.212622in}}{\pgfqpoint{3.696000in}{3.696000in}}%
\pgfusepath{clip}%
\pgfsetrectcap%
\pgfsetroundjoin%
\pgfsetlinewidth{1.505625pt}%
\definecolor{currentstroke}{rgb}{1.000000,0.000000,0.000000}%
\pgfsetstrokecolor{currentstroke}%
\pgfsetdash{}{0pt}%
\pgfpathmoveto{\pgfqpoint{2.623030in}{1.007136in}}%
\pgfpathlineto{\pgfqpoint{2.388723in}{1.915220in}}%
\pgfusepath{stroke}%
\end{pgfscope}%
\begin{pgfscope}%
\pgfpathrectangle{\pgfqpoint{0.100000in}{0.212622in}}{\pgfqpoint{3.696000in}{3.696000in}}%
\pgfusepath{clip}%
\pgfsetrectcap%
\pgfsetroundjoin%
\pgfsetlinewidth{1.505625pt}%
\definecolor{currentstroke}{rgb}{1.000000,0.000000,0.000000}%
\pgfsetstrokecolor{currentstroke}%
\pgfsetdash{}{0pt}%
\pgfpathmoveto{\pgfqpoint{2.625708in}{1.013072in}}%
\pgfpathlineto{\pgfqpoint{2.399375in}{1.924885in}}%
\pgfusepath{stroke}%
\end{pgfscope}%
\begin{pgfscope}%
\pgfpathrectangle{\pgfqpoint{0.100000in}{0.212622in}}{\pgfqpoint{3.696000in}{3.696000in}}%
\pgfusepath{clip}%
\pgfsetrectcap%
\pgfsetroundjoin%
\pgfsetlinewidth{1.505625pt}%
\definecolor{currentstroke}{rgb}{1.000000,0.000000,0.000000}%
\pgfsetstrokecolor{currentstroke}%
\pgfsetdash{}{0pt}%
\pgfpathmoveto{\pgfqpoint{2.626893in}{1.020808in}}%
\pgfpathlineto{\pgfqpoint{2.399375in}{1.924885in}}%
\pgfusepath{stroke}%
\end{pgfscope}%
\begin{pgfscope}%
\pgfpathrectangle{\pgfqpoint{0.100000in}{0.212622in}}{\pgfqpoint{3.696000in}{3.696000in}}%
\pgfusepath{clip}%
\pgfsetrectcap%
\pgfsetroundjoin%
\pgfsetlinewidth{1.505625pt}%
\definecolor{currentstroke}{rgb}{1.000000,0.000000,0.000000}%
\pgfsetstrokecolor{currentstroke}%
\pgfsetdash{}{0pt}%
\pgfpathmoveto{\pgfqpoint{2.627651in}{1.024934in}}%
\pgfpathlineto{\pgfqpoint{2.399375in}{1.924885in}}%
\pgfusepath{stroke}%
\end{pgfscope}%
\begin{pgfscope}%
\pgfpathrectangle{\pgfqpoint{0.100000in}{0.212622in}}{\pgfqpoint{3.696000in}{3.696000in}}%
\pgfusepath{clip}%
\pgfsetrectcap%
\pgfsetroundjoin%
\pgfsetlinewidth{1.505625pt}%
\definecolor{currentstroke}{rgb}{1.000000,0.000000,0.000000}%
\pgfsetstrokecolor{currentstroke}%
\pgfsetdash{}{0pt}%
\pgfpathmoveto{\pgfqpoint{2.630126in}{1.029833in}}%
\pgfpathlineto{\pgfqpoint{2.399375in}{1.924885in}}%
\pgfusepath{stroke}%
\end{pgfscope}%
\begin{pgfscope}%
\pgfpathrectangle{\pgfqpoint{0.100000in}{0.212622in}}{\pgfqpoint{3.696000in}{3.696000in}}%
\pgfusepath{clip}%
\pgfsetrectcap%
\pgfsetroundjoin%
\pgfsetlinewidth{1.505625pt}%
\definecolor{currentstroke}{rgb}{1.000000,0.000000,0.000000}%
\pgfsetstrokecolor{currentstroke}%
\pgfsetdash{}{0pt}%
\pgfpathmoveto{\pgfqpoint{2.633773in}{1.035595in}}%
\pgfpathlineto{\pgfqpoint{2.410010in}{1.934534in}}%
\pgfusepath{stroke}%
\end{pgfscope}%
\begin{pgfscope}%
\pgfpathrectangle{\pgfqpoint{0.100000in}{0.212622in}}{\pgfqpoint{3.696000in}{3.696000in}}%
\pgfusepath{clip}%
\pgfsetrectcap%
\pgfsetroundjoin%
\pgfsetlinewidth{1.505625pt}%
\definecolor{currentstroke}{rgb}{1.000000,0.000000,0.000000}%
\pgfsetstrokecolor{currentstroke}%
\pgfsetdash{}{0pt}%
\pgfpathmoveto{\pgfqpoint{2.638075in}{1.041156in}}%
\pgfpathlineto{\pgfqpoint{2.410010in}{1.934534in}}%
\pgfusepath{stroke}%
\end{pgfscope}%
\begin{pgfscope}%
\pgfpathrectangle{\pgfqpoint{0.100000in}{0.212622in}}{\pgfqpoint{3.696000in}{3.696000in}}%
\pgfusepath{clip}%
\pgfsetrectcap%
\pgfsetroundjoin%
\pgfsetlinewidth{1.505625pt}%
\definecolor{currentstroke}{rgb}{1.000000,0.000000,0.000000}%
\pgfsetstrokecolor{currentstroke}%
\pgfsetdash{}{0pt}%
\pgfpathmoveto{\pgfqpoint{2.642664in}{1.049929in}}%
\pgfpathlineto{\pgfqpoint{2.410010in}{1.934534in}}%
\pgfusepath{stroke}%
\end{pgfscope}%
\begin{pgfscope}%
\pgfpathrectangle{\pgfqpoint{0.100000in}{0.212622in}}{\pgfqpoint{3.696000in}{3.696000in}}%
\pgfusepath{clip}%
\pgfsetrectcap%
\pgfsetroundjoin%
\pgfsetlinewidth{1.505625pt}%
\definecolor{currentstroke}{rgb}{1.000000,0.000000,0.000000}%
\pgfsetstrokecolor{currentstroke}%
\pgfsetdash{}{0pt}%
\pgfpathmoveto{\pgfqpoint{2.645129in}{1.063872in}}%
\pgfpathlineto{\pgfqpoint{2.420627in}{1.944167in}}%
\pgfusepath{stroke}%
\end{pgfscope}%
\begin{pgfscope}%
\pgfpathrectangle{\pgfqpoint{0.100000in}{0.212622in}}{\pgfqpoint{3.696000in}{3.696000in}}%
\pgfusepath{clip}%
\pgfsetrectcap%
\pgfsetroundjoin%
\pgfsetlinewidth{1.505625pt}%
\definecolor{currentstroke}{rgb}{1.000000,0.000000,0.000000}%
\pgfsetstrokecolor{currentstroke}%
\pgfsetdash{}{0pt}%
\pgfpathmoveto{\pgfqpoint{2.646579in}{1.071390in}}%
\pgfpathlineto{\pgfqpoint{2.420627in}{1.944167in}}%
\pgfusepath{stroke}%
\end{pgfscope}%
\begin{pgfscope}%
\pgfpathrectangle{\pgfqpoint{0.100000in}{0.212622in}}{\pgfqpoint{3.696000in}{3.696000in}}%
\pgfusepath{clip}%
\pgfsetrectcap%
\pgfsetroundjoin%
\pgfsetlinewidth{1.505625pt}%
\definecolor{currentstroke}{rgb}{1.000000,0.000000,0.000000}%
\pgfsetstrokecolor{currentstroke}%
\pgfsetdash{}{0pt}%
\pgfpathmoveto{\pgfqpoint{2.651080in}{1.080641in}}%
\pgfpathlineto{\pgfqpoint{2.431227in}{1.953785in}}%
\pgfusepath{stroke}%
\end{pgfscope}%
\begin{pgfscope}%
\pgfpathrectangle{\pgfqpoint{0.100000in}{0.212622in}}{\pgfqpoint{3.696000in}{3.696000in}}%
\pgfusepath{clip}%
\pgfsetrectcap%
\pgfsetroundjoin%
\pgfsetlinewidth{1.505625pt}%
\definecolor{currentstroke}{rgb}{1.000000,0.000000,0.000000}%
\pgfsetstrokecolor{currentstroke}%
\pgfsetdash{}{0pt}%
\pgfpathmoveto{\pgfqpoint{2.658094in}{1.089743in}}%
\pgfpathlineto{\pgfqpoint{2.431227in}{1.953785in}}%
\pgfusepath{stroke}%
\end{pgfscope}%
\begin{pgfscope}%
\pgfpathrectangle{\pgfqpoint{0.100000in}{0.212622in}}{\pgfqpoint{3.696000in}{3.696000in}}%
\pgfusepath{clip}%
\pgfsetrectcap%
\pgfsetroundjoin%
\pgfsetlinewidth{1.505625pt}%
\definecolor{currentstroke}{rgb}{1.000000,0.000000,0.000000}%
\pgfsetstrokecolor{currentstroke}%
\pgfsetdash{}{0pt}%
\pgfpathmoveto{\pgfqpoint{2.661888in}{1.094127in}}%
\pgfpathlineto{\pgfqpoint{2.431227in}{1.953785in}}%
\pgfusepath{stroke}%
\end{pgfscope}%
\begin{pgfscope}%
\pgfpathrectangle{\pgfqpoint{0.100000in}{0.212622in}}{\pgfqpoint{3.696000in}{3.696000in}}%
\pgfusepath{clip}%
\pgfsetrectcap%
\pgfsetroundjoin%
\pgfsetlinewidth{1.505625pt}%
\definecolor{currentstroke}{rgb}{1.000000,0.000000,0.000000}%
\pgfsetstrokecolor{currentstroke}%
\pgfsetdash{}{0pt}%
\pgfpathmoveto{\pgfqpoint{2.665387in}{1.100905in}}%
\pgfpathlineto{\pgfqpoint{2.441809in}{1.963386in}}%
\pgfusepath{stroke}%
\end{pgfscope}%
\begin{pgfscope}%
\pgfpathrectangle{\pgfqpoint{0.100000in}{0.212622in}}{\pgfqpoint{3.696000in}{3.696000in}}%
\pgfusepath{clip}%
\pgfsetrectcap%
\pgfsetroundjoin%
\pgfsetlinewidth{1.505625pt}%
\definecolor{currentstroke}{rgb}{1.000000,0.000000,0.000000}%
\pgfsetstrokecolor{currentstroke}%
\pgfsetdash{}{0pt}%
\pgfpathmoveto{\pgfqpoint{2.666387in}{1.105079in}}%
\pgfpathlineto{\pgfqpoint{2.441809in}{1.963386in}}%
\pgfusepath{stroke}%
\end{pgfscope}%
\begin{pgfscope}%
\pgfpathrectangle{\pgfqpoint{0.100000in}{0.212622in}}{\pgfqpoint{3.696000in}{3.696000in}}%
\pgfusepath{clip}%
\pgfsetrectcap%
\pgfsetroundjoin%
\pgfsetlinewidth{1.505625pt}%
\definecolor{currentstroke}{rgb}{1.000000,0.000000,0.000000}%
\pgfsetstrokecolor{currentstroke}%
\pgfsetdash{}{0pt}%
\pgfpathmoveto{\pgfqpoint{2.667920in}{1.109724in}}%
\pgfpathlineto{\pgfqpoint{2.441809in}{1.963386in}}%
\pgfusepath{stroke}%
\end{pgfscope}%
\begin{pgfscope}%
\pgfpathrectangle{\pgfqpoint{0.100000in}{0.212622in}}{\pgfqpoint{3.696000in}{3.696000in}}%
\pgfusepath{clip}%
\pgfsetrectcap%
\pgfsetroundjoin%
\pgfsetlinewidth{1.505625pt}%
\definecolor{currentstroke}{rgb}{1.000000,0.000000,0.000000}%
\pgfsetstrokecolor{currentstroke}%
\pgfsetdash{}{0pt}%
\pgfpathmoveto{\pgfqpoint{2.671716in}{1.115485in}}%
\pgfpathlineto{\pgfqpoint{2.441809in}{1.963386in}}%
\pgfusepath{stroke}%
\end{pgfscope}%
\begin{pgfscope}%
\pgfpathrectangle{\pgfqpoint{0.100000in}{0.212622in}}{\pgfqpoint{3.696000in}{3.696000in}}%
\pgfusepath{clip}%
\pgfsetrectcap%
\pgfsetroundjoin%
\pgfsetlinewidth{1.505625pt}%
\definecolor{currentstroke}{rgb}{1.000000,0.000000,0.000000}%
\pgfsetstrokecolor{currentstroke}%
\pgfsetdash{}{0pt}%
\pgfpathmoveto{\pgfqpoint{2.676063in}{1.120919in}}%
\pgfpathlineto{\pgfqpoint{2.452374in}{1.972972in}}%
\pgfusepath{stroke}%
\end{pgfscope}%
\begin{pgfscope}%
\pgfpathrectangle{\pgfqpoint{0.100000in}{0.212622in}}{\pgfqpoint{3.696000in}{3.696000in}}%
\pgfusepath{clip}%
\pgfsetrectcap%
\pgfsetroundjoin%
\pgfsetlinewidth{1.505625pt}%
\definecolor{currentstroke}{rgb}{1.000000,0.000000,0.000000}%
\pgfsetstrokecolor{currentstroke}%
\pgfsetdash{}{0pt}%
\pgfpathmoveto{\pgfqpoint{2.679708in}{1.129775in}}%
\pgfpathlineto{\pgfqpoint{2.452374in}{1.972972in}}%
\pgfusepath{stroke}%
\end{pgfscope}%
\begin{pgfscope}%
\pgfpathrectangle{\pgfqpoint{0.100000in}{0.212622in}}{\pgfqpoint{3.696000in}{3.696000in}}%
\pgfusepath{clip}%
\pgfsetrectcap%
\pgfsetroundjoin%
\pgfsetlinewidth{1.505625pt}%
\definecolor{currentstroke}{rgb}{1.000000,0.000000,0.000000}%
\pgfsetstrokecolor{currentstroke}%
\pgfsetdash{}{0pt}%
\pgfpathmoveto{\pgfqpoint{2.679935in}{1.142339in}}%
\pgfpathlineto{\pgfqpoint{2.462922in}{1.982542in}}%
\pgfusepath{stroke}%
\end{pgfscope}%
\begin{pgfscope}%
\pgfpathrectangle{\pgfqpoint{0.100000in}{0.212622in}}{\pgfqpoint{3.696000in}{3.696000in}}%
\pgfusepath{clip}%
\pgfsetrectcap%
\pgfsetroundjoin%
\pgfsetlinewidth{1.505625pt}%
\definecolor{currentstroke}{rgb}{1.000000,0.000000,0.000000}%
\pgfsetstrokecolor{currentstroke}%
\pgfsetdash{}{0pt}%
\pgfpathmoveto{\pgfqpoint{2.682501in}{1.154119in}}%
\pgfpathlineto{\pgfqpoint{2.462922in}{1.982542in}}%
\pgfusepath{stroke}%
\end{pgfscope}%
\begin{pgfscope}%
\pgfpathrectangle{\pgfqpoint{0.100000in}{0.212622in}}{\pgfqpoint{3.696000in}{3.696000in}}%
\pgfusepath{clip}%
\pgfsetrectcap%
\pgfsetroundjoin%
\pgfsetlinewidth{1.505625pt}%
\definecolor{currentstroke}{rgb}{1.000000,0.000000,0.000000}%
\pgfsetstrokecolor{currentstroke}%
\pgfsetdash{}{0pt}%
\pgfpathmoveto{\pgfqpoint{2.690410in}{1.165975in}}%
\pgfpathlineto{\pgfqpoint{2.473453in}{1.992096in}}%
\pgfusepath{stroke}%
\end{pgfscope}%
\begin{pgfscope}%
\pgfpathrectangle{\pgfqpoint{0.100000in}{0.212622in}}{\pgfqpoint{3.696000in}{3.696000in}}%
\pgfusepath{clip}%
\pgfsetrectcap%
\pgfsetroundjoin%
\pgfsetlinewidth{1.505625pt}%
\definecolor{currentstroke}{rgb}{1.000000,0.000000,0.000000}%
\pgfsetstrokecolor{currentstroke}%
\pgfsetdash{}{0pt}%
\pgfpathmoveto{\pgfqpoint{2.699771in}{1.177817in}}%
\pgfpathlineto{\pgfqpoint{2.473453in}{1.992096in}}%
\pgfusepath{stroke}%
\end{pgfscope}%
\begin{pgfscope}%
\pgfpathrectangle{\pgfqpoint{0.100000in}{0.212622in}}{\pgfqpoint{3.696000in}{3.696000in}}%
\pgfusepath{clip}%
\pgfsetrectcap%
\pgfsetroundjoin%
\pgfsetlinewidth{1.505625pt}%
\definecolor{currentstroke}{rgb}{1.000000,0.000000,0.000000}%
\pgfsetstrokecolor{currentstroke}%
\pgfsetdash{}{0pt}%
\pgfpathmoveto{\pgfqpoint{2.709312in}{1.191537in}}%
\pgfpathlineto{\pgfqpoint{2.483966in}{2.001635in}}%
\pgfusepath{stroke}%
\end{pgfscope}%
\begin{pgfscope}%
\pgfpathrectangle{\pgfqpoint{0.100000in}{0.212622in}}{\pgfqpoint{3.696000in}{3.696000in}}%
\pgfusepath{clip}%
\pgfsetrectcap%
\pgfsetroundjoin%
\pgfsetlinewidth{1.505625pt}%
\definecolor{currentstroke}{rgb}{1.000000,0.000000,0.000000}%
\pgfsetstrokecolor{currentstroke}%
\pgfsetdash{}{0pt}%
\pgfpathmoveto{\pgfqpoint{2.714357in}{1.209353in}}%
\pgfpathlineto{\pgfqpoint{2.494462in}{2.011158in}}%
\pgfusepath{stroke}%
\end{pgfscope}%
\begin{pgfscope}%
\pgfpathrectangle{\pgfqpoint{0.100000in}{0.212622in}}{\pgfqpoint{3.696000in}{3.696000in}}%
\pgfusepath{clip}%
\pgfsetrectcap%
\pgfsetroundjoin%
\pgfsetlinewidth{1.505625pt}%
\definecolor{currentstroke}{rgb}{1.000000,0.000000,0.000000}%
\pgfsetstrokecolor{currentstroke}%
\pgfsetdash{}{0pt}%
\pgfpathmoveto{\pgfqpoint{2.717221in}{1.228826in}}%
\pgfpathlineto{\pgfqpoint{2.504941in}{2.020666in}}%
\pgfusepath{stroke}%
\end{pgfscope}%
\begin{pgfscope}%
\pgfpathrectangle{\pgfqpoint{0.100000in}{0.212622in}}{\pgfqpoint{3.696000in}{3.696000in}}%
\pgfusepath{clip}%
\pgfsetrectcap%
\pgfsetroundjoin%
\pgfsetlinewidth{1.505625pt}%
\definecolor{currentstroke}{rgb}{1.000000,0.000000,0.000000}%
\pgfsetstrokecolor{currentstroke}%
\pgfsetdash{}{0pt}%
\pgfpathmoveto{\pgfqpoint{2.726236in}{1.249720in}}%
\pgfpathlineto{\pgfqpoint{2.515403in}{2.030158in}}%
\pgfusepath{stroke}%
\end{pgfscope}%
\begin{pgfscope}%
\pgfpathrectangle{\pgfqpoint{0.100000in}{0.212622in}}{\pgfqpoint{3.696000in}{3.696000in}}%
\pgfusepath{clip}%
\pgfsetrectcap%
\pgfsetroundjoin%
\pgfsetlinewidth{1.505625pt}%
\definecolor{currentstroke}{rgb}{1.000000,0.000000,0.000000}%
\pgfsetstrokecolor{currentstroke}%
\pgfsetdash{}{0pt}%
\pgfpathmoveto{\pgfqpoint{2.740603in}{1.269971in}}%
\pgfpathlineto{\pgfqpoint{2.525847in}{2.039635in}}%
\pgfusepath{stroke}%
\end{pgfscope}%
\begin{pgfscope}%
\pgfpathrectangle{\pgfqpoint{0.100000in}{0.212622in}}{\pgfqpoint{3.696000in}{3.696000in}}%
\pgfusepath{clip}%
\pgfsetrectcap%
\pgfsetroundjoin%
\pgfsetlinewidth{1.505625pt}%
\definecolor{currentstroke}{rgb}{1.000000,0.000000,0.000000}%
\pgfsetstrokecolor{currentstroke}%
\pgfsetdash{}{0pt}%
\pgfpathmoveto{\pgfqpoint{2.756031in}{1.288873in}}%
\pgfpathlineto{\pgfqpoint{2.536275in}{2.049096in}}%
\pgfusepath{stroke}%
\end{pgfscope}%
\begin{pgfscope}%
\pgfpathrectangle{\pgfqpoint{0.100000in}{0.212622in}}{\pgfqpoint{3.696000in}{3.696000in}}%
\pgfusepath{clip}%
\pgfsetrectcap%
\pgfsetroundjoin%
\pgfsetlinewidth{1.505625pt}%
\definecolor{currentstroke}{rgb}{1.000000,0.000000,0.000000}%
\pgfsetstrokecolor{currentstroke}%
\pgfsetdash{}{0pt}%
\pgfpathmoveto{\pgfqpoint{2.767873in}{1.314442in}}%
\pgfpathlineto{\pgfqpoint{2.546686in}{2.058542in}}%
\pgfusepath{stroke}%
\end{pgfscope}%
\begin{pgfscope}%
\pgfpathrectangle{\pgfqpoint{0.100000in}{0.212622in}}{\pgfqpoint{3.696000in}{3.696000in}}%
\pgfusepath{clip}%
\pgfsetrectcap%
\pgfsetroundjoin%
\pgfsetlinewidth{1.505625pt}%
\definecolor{currentstroke}{rgb}{1.000000,0.000000,0.000000}%
\pgfsetstrokecolor{currentstroke}%
\pgfsetdash{}{0pt}%
\pgfpathmoveto{\pgfqpoint{2.775511in}{1.345270in}}%
\pgfpathlineto{\pgfqpoint{2.557079in}{2.067972in}}%
\pgfusepath{stroke}%
\end{pgfscope}%
\begin{pgfscope}%
\pgfpathrectangle{\pgfqpoint{0.100000in}{0.212622in}}{\pgfqpoint{3.696000in}{3.696000in}}%
\pgfusepath{clip}%
\pgfsetrectcap%
\pgfsetroundjoin%
\pgfsetlinewidth{1.505625pt}%
\definecolor{currentstroke}{rgb}{1.000000,0.000000,0.000000}%
\pgfsetstrokecolor{currentstroke}%
\pgfsetdash{}{0pt}%
\pgfpathmoveto{\pgfqpoint{2.780078in}{1.376380in}}%
\pgfpathlineto{\pgfqpoint{2.567456in}{2.077387in}}%
\pgfusepath{stroke}%
\end{pgfscope}%
\begin{pgfscope}%
\pgfpathrectangle{\pgfqpoint{0.100000in}{0.212622in}}{\pgfqpoint{3.696000in}{3.696000in}}%
\pgfusepath{clip}%
\pgfsetrectcap%
\pgfsetroundjoin%
\pgfsetlinewidth{1.505625pt}%
\definecolor{currentstroke}{rgb}{1.000000,0.000000,0.000000}%
\pgfsetstrokecolor{currentstroke}%
\pgfsetdash{}{0pt}%
\pgfpathmoveto{\pgfqpoint{2.792666in}{1.407907in}}%
\pgfpathlineto{\pgfqpoint{2.588159in}{2.096171in}}%
\pgfusepath{stroke}%
\end{pgfscope}%
\begin{pgfscope}%
\pgfpathrectangle{\pgfqpoint{0.100000in}{0.212622in}}{\pgfqpoint{3.696000in}{3.696000in}}%
\pgfusepath{clip}%
\pgfsetrectcap%
\pgfsetroundjoin%
\pgfsetlinewidth{1.505625pt}%
\definecolor{currentstroke}{rgb}{1.000000,0.000000,0.000000}%
\pgfsetstrokecolor{currentstroke}%
\pgfsetdash{}{0pt}%
\pgfpathmoveto{\pgfqpoint{2.811173in}{1.437200in}}%
\pgfpathlineto{\pgfqpoint{2.598485in}{2.105540in}}%
\pgfusepath{stroke}%
\end{pgfscope}%
\begin{pgfscope}%
\pgfpathrectangle{\pgfqpoint{0.100000in}{0.212622in}}{\pgfqpoint{3.696000in}{3.696000in}}%
\pgfusepath{clip}%
\pgfsetrectcap%
\pgfsetroundjoin%
\pgfsetlinewidth{1.505625pt}%
\definecolor{currentstroke}{rgb}{1.000000,0.000000,0.000000}%
\pgfsetstrokecolor{currentstroke}%
\pgfsetdash{}{0pt}%
\pgfpathmoveto{\pgfqpoint{2.831378in}{1.464832in}}%
\pgfpathlineto{\pgfqpoint{2.619088in}{2.124233in}}%
\pgfusepath{stroke}%
\end{pgfscope}%
\begin{pgfscope}%
\pgfpathrectangle{\pgfqpoint{0.100000in}{0.212622in}}{\pgfqpoint{3.696000in}{3.696000in}}%
\pgfusepath{clip}%
\pgfsetrectcap%
\pgfsetroundjoin%
\pgfsetlinewidth{1.505625pt}%
\definecolor{currentstroke}{rgb}{1.000000,0.000000,0.000000}%
\pgfsetstrokecolor{currentstroke}%
\pgfsetdash{}{0pt}%
\pgfpathmoveto{\pgfqpoint{2.849065in}{1.497632in}}%
\pgfpathlineto{\pgfqpoint{2.629364in}{2.133557in}}%
\pgfusepath{stroke}%
\end{pgfscope}%
\begin{pgfscope}%
\pgfpathrectangle{\pgfqpoint{0.100000in}{0.212622in}}{\pgfqpoint{3.696000in}{3.696000in}}%
\pgfusepath{clip}%
\pgfsetrectcap%
\pgfsetroundjoin%
\pgfsetlinewidth{1.505625pt}%
\definecolor{currentstroke}{rgb}{1.000000,0.000000,0.000000}%
\pgfsetstrokecolor{currentstroke}%
\pgfsetdash{}{0pt}%
\pgfpathmoveto{\pgfqpoint{2.855531in}{1.540086in}}%
\pgfpathlineto{\pgfqpoint{2.649867in}{2.152159in}}%
\pgfusepath{stroke}%
\end{pgfscope}%
\begin{pgfscope}%
\pgfpathrectangle{\pgfqpoint{0.100000in}{0.212622in}}{\pgfqpoint{3.696000in}{3.696000in}}%
\pgfusepath{clip}%
\pgfsetrectcap%
\pgfsetroundjoin%
\pgfsetlinewidth{1.505625pt}%
\definecolor{currentstroke}{rgb}{1.000000,0.000000,0.000000}%
\pgfsetstrokecolor{currentstroke}%
\pgfsetdash{}{0pt}%
\pgfpathmoveto{\pgfqpoint{2.862691in}{1.578836in}}%
\pgfpathlineto{\pgfqpoint{2.660093in}{2.161438in}}%
\pgfusepath{stroke}%
\end{pgfscope}%
\begin{pgfscope}%
\pgfpathrectangle{\pgfqpoint{0.100000in}{0.212622in}}{\pgfqpoint{3.696000in}{3.696000in}}%
\pgfusepath{clip}%
\pgfsetrectcap%
\pgfsetroundjoin%
\pgfsetlinewidth{1.505625pt}%
\definecolor{currentstroke}{rgb}{1.000000,0.000000,0.000000}%
\pgfsetstrokecolor{currentstroke}%
\pgfsetdash{}{0pt}%
\pgfpathmoveto{\pgfqpoint{2.884458in}{1.613128in}}%
\pgfpathlineto{\pgfqpoint{2.680497in}{2.179950in}}%
\pgfusepath{stroke}%
\end{pgfscope}%
\begin{pgfscope}%
\pgfpathrectangle{\pgfqpoint{0.100000in}{0.212622in}}{\pgfqpoint{3.696000in}{3.696000in}}%
\pgfusepath{clip}%
\pgfsetrectcap%
\pgfsetroundjoin%
\pgfsetlinewidth{1.505625pt}%
\definecolor{currentstroke}{rgb}{1.000000,0.000000,0.000000}%
\pgfsetstrokecolor{currentstroke}%
\pgfsetdash{}{0pt}%
\pgfpathmoveto{\pgfqpoint{2.908129in}{1.647370in}}%
\pgfpathlineto{\pgfqpoint{2.700835in}{2.198403in}}%
\pgfusepath{stroke}%
\end{pgfscope}%
\begin{pgfscope}%
\pgfpathrectangle{\pgfqpoint{0.100000in}{0.212622in}}{\pgfqpoint{3.696000in}{3.696000in}}%
\pgfusepath{clip}%
\pgfsetrectcap%
\pgfsetroundjoin%
\pgfsetlinewidth{1.505625pt}%
\definecolor{currentstroke}{rgb}{1.000000,0.000000,0.000000}%
\pgfsetstrokecolor{currentstroke}%
\pgfsetdash{}{0pt}%
\pgfpathmoveto{\pgfqpoint{2.932927in}{1.676742in}}%
\pgfpathlineto{\pgfqpoint{2.721107in}{2.216797in}}%
\pgfusepath{stroke}%
\end{pgfscope}%
\begin{pgfscope}%
\pgfpathrectangle{\pgfqpoint{0.100000in}{0.212622in}}{\pgfqpoint{3.696000in}{3.696000in}}%
\pgfusepath{clip}%
\pgfsetrectcap%
\pgfsetroundjoin%
\pgfsetlinewidth{1.505625pt}%
\definecolor{currentstroke}{rgb}{1.000000,0.000000,0.000000}%
\pgfsetstrokecolor{currentstroke}%
\pgfsetdash{}{0pt}%
\pgfpathmoveto{\pgfqpoint{2.941517in}{1.721623in}}%
\pgfpathlineto{\pgfqpoint{2.741315in}{2.235131in}}%
\pgfusepath{stroke}%
\end{pgfscope}%
\begin{pgfscope}%
\pgfpathrectangle{\pgfqpoint{0.100000in}{0.212622in}}{\pgfqpoint{3.696000in}{3.696000in}}%
\pgfusepath{clip}%
\pgfsetrectcap%
\pgfsetroundjoin%
\pgfsetlinewidth{1.505625pt}%
\definecolor{currentstroke}{rgb}{1.000000,0.000000,0.000000}%
\pgfsetstrokecolor{currentstroke}%
\pgfsetdash{}{0pt}%
\pgfpathmoveto{\pgfqpoint{2.943565in}{1.747162in}}%
\pgfpathlineto{\pgfqpoint{2.751394in}{2.244276in}}%
\pgfusepath{stroke}%
\end{pgfscope}%
\begin{pgfscope}%
\pgfpathrectangle{\pgfqpoint{0.100000in}{0.212622in}}{\pgfqpoint{3.696000in}{3.696000in}}%
\pgfusepath{clip}%
\pgfsetrectcap%
\pgfsetroundjoin%
\pgfsetlinewidth{1.505625pt}%
\definecolor{currentstroke}{rgb}{1.000000,0.000000,0.000000}%
\pgfsetstrokecolor{currentstroke}%
\pgfsetdash{}{0pt}%
\pgfpathmoveto{\pgfqpoint{2.950294in}{1.771386in}}%
\pgfpathlineto{\pgfqpoint{2.761458in}{2.253407in}}%
\pgfusepath{stroke}%
\end{pgfscope}%
\begin{pgfscope}%
\pgfpathrectangle{\pgfqpoint{0.100000in}{0.212622in}}{\pgfqpoint{3.696000in}{3.696000in}}%
\pgfusepath{clip}%
\pgfsetrectcap%
\pgfsetroundjoin%
\pgfsetlinewidth{1.505625pt}%
\definecolor{currentstroke}{rgb}{1.000000,0.000000,0.000000}%
\pgfsetstrokecolor{currentstroke}%
\pgfsetdash{}{0pt}%
\pgfpathmoveto{\pgfqpoint{2.964640in}{1.794914in}}%
\pgfpathlineto{\pgfqpoint{2.771505in}{2.262523in}}%
\pgfusepath{stroke}%
\end{pgfscope}%
\begin{pgfscope}%
\pgfpathrectangle{\pgfqpoint{0.100000in}{0.212622in}}{\pgfqpoint{3.696000in}{3.696000in}}%
\pgfusepath{clip}%
\pgfsetrectcap%
\pgfsetroundjoin%
\pgfsetlinewidth{1.505625pt}%
\definecolor{currentstroke}{rgb}{1.000000,0.000000,0.000000}%
\pgfsetstrokecolor{currentstroke}%
\pgfsetdash{}{0pt}%
\pgfpathmoveto{\pgfqpoint{2.973271in}{1.806840in}}%
\pgfpathlineto{\pgfqpoint{2.781536in}{2.271624in}}%
\pgfusepath{stroke}%
\end{pgfscope}%
\begin{pgfscope}%
\pgfpathrectangle{\pgfqpoint{0.100000in}{0.212622in}}{\pgfqpoint{3.696000in}{3.696000in}}%
\pgfusepath{clip}%
\pgfsetrectcap%
\pgfsetroundjoin%
\pgfsetlinewidth{1.505625pt}%
\definecolor{currentstroke}{rgb}{1.000000,0.000000,0.000000}%
\pgfsetstrokecolor{currentstroke}%
\pgfsetdash{}{0pt}%
\pgfpathmoveto{\pgfqpoint{2.981417in}{1.821007in}}%
\pgfpathlineto{\pgfqpoint{2.781536in}{2.271624in}}%
\pgfusepath{stroke}%
\end{pgfscope}%
\begin{pgfscope}%
\pgfpathrectangle{\pgfqpoint{0.100000in}{0.212622in}}{\pgfqpoint{3.696000in}{3.696000in}}%
\pgfusepath{clip}%
\pgfsetrectcap%
\pgfsetroundjoin%
\pgfsetlinewidth{1.505625pt}%
\definecolor{currentstroke}{rgb}{1.000000,0.000000,0.000000}%
\pgfsetstrokecolor{currentstroke}%
\pgfsetdash{}{0pt}%
\pgfpathmoveto{\pgfqpoint{2.986000in}{1.840544in}}%
\pgfpathlineto{\pgfqpoint{2.791551in}{2.280711in}}%
\pgfusepath{stroke}%
\end{pgfscope}%
\begin{pgfscope}%
\pgfpathrectangle{\pgfqpoint{0.100000in}{0.212622in}}{\pgfqpoint{3.696000in}{3.696000in}}%
\pgfusepath{clip}%
\pgfsetrectcap%
\pgfsetroundjoin%
\pgfsetlinewidth{1.505625pt}%
\definecolor{currentstroke}{rgb}{1.000000,0.000000,0.000000}%
\pgfsetstrokecolor{currentstroke}%
\pgfsetdash{}{0pt}%
\pgfpathmoveto{\pgfqpoint{2.989797in}{1.859271in}}%
\pgfpathlineto{\pgfqpoint{2.801550in}{2.289783in}}%
\pgfusepath{stroke}%
\end{pgfscope}%
\begin{pgfscope}%
\pgfpathrectangle{\pgfqpoint{0.100000in}{0.212622in}}{\pgfqpoint{3.696000in}{3.696000in}}%
\pgfusepath{clip}%
\pgfsetrectcap%
\pgfsetroundjoin%
\pgfsetlinewidth{1.505625pt}%
\definecolor{currentstroke}{rgb}{1.000000,0.000000,0.000000}%
\pgfsetstrokecolor{currentstroke}%
\pgfsetdash{}{0pt}%
\pgfpathmoveto{\pgfqpoint{2.998274in}{1.877615in}}%
\pgfpathlineto{\pgfqpoint{2.811533in}{2.298841in}}%
\pgfusepath{stroke}%
\end{pgfscope}%
\begin{pgfscope}%
\pgfpathrectangle{\pgfqpoint{0.100000in}{0.212622in}}{\pgfqpoint{3.696000in}{3.696000in}}%
\pgfusepath{clip}%
\pgfsetrectcap%
\pgfsetroundjoin%
\pgfsetlinewidth{1.505625pt}%
\definecolor{currentstroke}{rgb}{1.000000,0.000000,0.000000}%
\pgfsetstrokecolor{currentstroke}%
\pgfsetdash{}{0pt}%
\pgfpathmoveto{\pgfqpoint{3.011187in}{1.896891in}}%
\pgfpathlineto{\pgfqpoint{2.821500in}{2.307885in}}%
\pgfusepath{stroke}%
\end{pgfscope}%
\begin{pgfscope}%
\pgfpathrectangle{\pgfqpoint{0.100000in}{0.212622in}}{\pgfqpoint{3.696000in}{3.696000in}}%
\pgfusepath{clip}%
\pgfsetrectcap%
\pgfsetroundjoin%
\pgfsetlinewidth{1.505625pt}%
\definecolor{currentstroke}{rgb}{1.000000,0.000000,0.000000}%
\pgfsetstrokecolor{currentstroke}%
\pgfsetdash{}{0pt}%
\pgfpathmoveto{\pgfqpoint{3.024821in}{1.915111in}}%
\pgfpathlineto{\pgfqpoint{2.831452in}{2.316914in}}%
\pgfusepath{stroke}%
\end{pgfscope}%
\begin{pgfscope}%
\pgfpathrectangle{\pgfqpoint{0.100000in}{0.212622in}}{\pgfqpoint{3.696000in}{3.696000in}}%
\pgfusepath{clip}%
\pgfsetrectcap%
\pgfsetroundjoin%
\pgfsetlinewidth{1.505625pt}%
\definecolor{currentstroke}{rgb}{1.000000,0.000000,0.000000}%
\pgfsetstrokecolor{currentstroke}%
\pgfsetdash{}{0pt}%
\pgfpathmoveto{\pgfqpoint{3.034709in}{1.939090in}}%
\pgfpathlineto{\pgfqpoint{2.841387in}{2.325928in}}%
\pgfusepath{stroke}%
\end{pgfscope}%
\begin{pgfscope}%
\pgfpathrectangle{\pgfqpoint{0.100000in}{0.212622in}}{\pgfqpoint{3.696000in}{3.696000in}}%
\pgfusepath{clip}%
\pgfsetrectcap%
\pgfsetroundjoin%
\pgfsetlinewidth{1.505625pt}%
\definecolor{currentstroke}{rgb}{1.000000,0.000000,0.000000}%
\pgfsetstrokecolor{currentstroke}%
\pgfsetdash{}{0pt}%
\pgfpathmoveto{\pgfqpoint{3.039701in}{1.965841in}}%
\pgfpathlineto{\pgfqpoint{2.851307in}{2.334929in}}%
\pgfusepath{stroke}%
\end{pgfscope}%
\begin{pgfscope}%
\pgfpathrectangle{\pgfqpoint{0.100000in}{0.212622in}}{\pgfqpoint{3.696000in}{3.696000in}}%
\pgfusepath{clip}%
\pgfsetrectcap%
\pgfsetroundjoin%
\pgfsetlinewidth{1.505625pt}%
\definecolor{currentstroke}{rgb}{1.000000,0.000000,0.000000}%
\pgfsetstrokecolor{currentstroke}%
\pgfsetdash{}{0pt}%
\pgfpathmoveto{\pgfqpoint{3.048871in}{1.989724in}}%
\pgfpathlineto{\pgfqpoint{2.861211in}{2.343915in}}%
\pgfusepath{stroke}%
\end{pgfscope}%
\begin{pgfscope}%
\pgfpathrectangle{\pgfqpoint{0.100000in}{0.212622in}}{\pgfqpoint{3.696000in}{3.696000in}}%
\pgfusepath{clip}%
\pgfsetrectcap%
\pgfsetroundjoin%
\pgfsetlinewidth{1.505625pt}%
\definecolor{currentstroke}{rgb}{1.000000,0.000000,0.000000}%
\pgfsetstrokecolor{currentstroke}%
\pgfsetdash{}{0pt}%
\pgfpathmoveto{\pgfqpoint{3.063580in}{2.014717in}}%
\pgfpathlineto{\pgfqpoint{2.871099in}{2.352886in}}%
\pgfusepath{stroke}%
\end{pgfscope}%
\begin{pgfscope}%
\pgfpathrectangle{\pgfqpoint{0.100000in}{0.212622in}}{\pgfqpoint{3.696000in}{3.696000in}}%
\pgfusepath{clip}%
\pgfsetrectcap%
\pgfsetroundjoin%
\pgfsetlinewidth{1.505625pt}%
\definecolor{currentstroke}{rgb}{1.000000,0.000000,0.000000}%
\pgfsetstrokecolor{currentstroke}%
\pgfsetdash{}{0pt}%
\pgfpathmoveto{\pgfqpoint{3.080026in}{2.036454in}}%
\pgfpathlineto{\pgfqpoint{2.880972in}{2.361844in}}%
\pgfusepath{stroke}%
\end{pgfscope}%
\begin{pgfscope}%
\pgfpathrectangle{\pgfqpoint{0.100000in}{0.212622in}}{\pgfqpoint{3.696000in}{3.696000in}}%
\pgfusepath{clip}%
\pgfsetrectcap%
\pgfsetroundjoin%
\pgfsetlinewidth{1.505625pt}%
\definecolor{currentstroke}{rgb}{1.000000,0.000000,0.000000}%
\pgfsetstrokecolor{currentstroke}%
\pgfsetdash{}{0pt}%
\pgfpathmoveto{\pgfqpoint{3.095022in}{2.059887in}}%
\pgfpathlineto{\pgfqpoint{2.900670in}{2.379716in}}%
\pgfusepath{stroke}%
\end{pgfscope}%
\begin{pgfscope}%
\pgfpathrectangle{\pgfqpoint{0.100000in}{0.212622in}}{\pgfqpoint{3.696000in}{3.696000in}}%
\pgfusepath{clip}%
\pgfsetrectcap%
\pgfsetroundjoin%
\pgfsetlinewidth{1.505625pt}%
\definecolor{currentstroke}{rgb}{1.000000,0.000000,0.000000}%
\pgfsetstrokecolor{currentstroke}%
\pgfsetdash{}{0pt}%
\pgfpathmoveto{\pgfqpoint{3.103900in}{2.088635in}}%
\pgfpathlineto{\pgfqpoint{2.910496in}{2.388631in}}%
\pgfusepath{stroke}%
\end{pgfscope}%
\begin{pgfscope}%
\pgfpathrectangle{\pgfqpoint{0.100000in}{0.212622in}}{\pgfqpoint{3.696000in}{3.696000in}}%
\pgfusepath{clip}%
\pgfsetrectcap%
\pgfsetroundjoin%
\pgfsetlinewidth{1.505625pt}%
\definecolor{currentstroke}{rgb}{1.000000,0.000000,0.000000}%
\pgfsetstrokecolor{currentstroke}%
\pgfsetdash{}{0pt}%
\pgfpathmoveto{\pgfqpoint{3.112211in}{2.119765in}}%
\pgfpathlineto{\pgfqpoint{2.920306in}{2.397532in}}%
\pgfusepath{stroke}%
\end{pgfscope}%
\begin{pgfscope}%
\pgfpathrectangle{\pgfqpoint{0.100000in}{0.212622in}}{\pgfqpoint{3.696000in}{3.696000in}}%
\pgfusepath{clip}%
\pgfsetrectcap%
\pgfsetroundjoin%
\pgfsetlinewidth{1.505625pt}%
\definecolor{currentstroke}{rgb}{1.000000,0.000000,0.000000}%
\pgfsetstrokecolor{currentstroke}%
\pgfsetdash{}{0pt}%
\pgfpathmoveto{\pgfqpoint{3.118120in}{2.134498in}}%
\pgfpathlineto{\pgfqpoint{2.930100in}{2.406419in}}%
\pgfusepath{stroke}%
\end{pgfscope}%
\begin{pgfscope}%
\pgfpathrectangle{\pgfqpoint{0.100000in}{0.212622in}}{\pgfqpoint{3.696000in}{3.696000in}}%
\pgfusepath{clip}%
\pgfsetrectcap%
\pgfsetroundjoin%
\pgfsetlinewidth{1.505625pt}%
\definecolor{currentstroke}{rgb}{1.000000,0.000000,0.000000}%
\pgfsetstrokecolor{currentstroke}%
\pgfsetdash{}{0pt}%
\pgfpathmoveto{\pgfqpoint{3.122082in}{2.142581in}}%
\pgfpathlineto{\pgfqpoint{2.930100in}{2.406419in}}%
\pgfusepath{stroke}%
\end{pgfscope}%
\begin{pgfscope}%
\pgfpathrectangle{\pgfqpoint{0.100000in}{0.212622in}}{\pgfqpoint{3.696000in}{3.696000in}}%
\pgfusepath{clip}%
\pgfsetrectcap%
\pgfsetroundjoin%
\pgfsetlinewidth{1.505625pt}%
\definecolor{currentstroke}{rgb}{1.000000,0.000000,0.000000}%
\pgfsetstrokecolor{currentstroke}%
\pgfsetdash{}{0pt}%
\pgfpathmoveto{\pgfqpoint{3.124316in}{2.146876in}}%
\pgfpathlineto{\pgfqpoint{2.939880in}{2.415292in}}%
\pgfusepath{stroke}%
\end{pgfscope}%
\begin{pgfscope}%
\pgfpathrectangle{\pgfqpoint{0.100000in}{0.212622in}}{\pgfqpoint{3.696000in}{3.696000in}}%
\pgfusepath{clip}%
\pgfsetrectcap%
\pgfsetroundjoin%
\pgfsetlinewidth{1.505625pt}%
\definecolor{currentstroke}{rgb}{1.000000,0.000000,0.000000}%
\pgfsetstrokecolor{currentstroke}%
\pgfsetdash{}{0pt}%
\pgfpathmoveto{\pgfqpoint{3.125592in}{2.149296in}}%
\pgfpathlineto{\pgfqpoint{2.939880in}{2.415292in}}%
\pgfusepath{stroke}%
\end{pgfscope}%
\begin{pgfscope}%
\pgfpathrectangle{\pgfqpoint{0.100000in}{0.212622in}}{\pgfqpoint{3.696000in}{3.696000in}}%
\pgfusepath{clip}%
\pgfsetrectcap%
\pgfsetroundjoin%
\pgfsetlinewidth{1.505625pt}%
\definecolor{currentstroke}{rgb}{1.000000,0.000000,0.000000}%
\pgfsetstrokecolor{currentstroke}%
\pgfsetdash{}{0pt}%
\pgfpathmoveto{\pgfqpoint{3.127181in}{2.151947in}}%
\pgfpathlineto{\pgfqpoint{2.939880in}{2.415292in}}%
\pgfusepath{stroke}%
\end{pgfscope}%
\begin{pgfscope}%
\pgfpathrectangle{\pgfqpoint{0.100000in}{0.212622in}}{\pgfqpoint{3.696000in}{3.696000in}}%
\pgfusepath{clip}%
\pgfsetrectcap%
\pgfsetroundjoin%
\pgfsetlinewidth{1.505625pt}%
\definecolor{currentstroke}{rgb}{1.000000,0.000000,0.000000}%
\pgfsetstrokecolor{currentstroke}%
\pgfsetdash{}{0pt}%
\pgfpathmoveto{\pgfqpoint{3.128109in}{2.153406in}}%
\pgfpathlineto{\pgfqpoint{2.939880in}{2.415292in}}%
\pgfusepath{stroke}%
\end{pgfscope}%
\begin{pgfscope}%
\pgfpathrectangle{\pgfqpoint{0.100000in}{0.212622in}}{\pgfqpoint{3.696000in}{3.696000in}}%
\pgfusepath{clip}%
\pgfsetrectcap%
\pgfsetroundjoin%
\pgfsetlinewidth{1.505625pt}%
\definecolor{currentstroke}{rgb}{1.000000,0.000000,0.000000}%
\pgfsetstrokecolor{currentstroke}%
\pgfsetdash{}{0pt}%
\pgfpathmoveto{\pgfqpoint{3.129590in}{2.155714in}}%
\pgfpathlineto{\pgfqpoint{2.939880in}{2.415292in}}%
\pgfusepath{stroke}%
\end{pgfscope}%
\begin{pgfscope}%
\pgfpathrectangle{\pgfqpoint{0.100000in}{0.212622in}}{\pgfqpoint{3.696000in}{3.696000in}}%
\pgfusepath{clip}%
\pgfsetrectcap%
\pgfsetroundjoin%
\pgfsetlinewidth{1.505625pt}%
\definecolor{currentstroke}{rgb}{1.000000,0.000000,0.000000}%
\pgfsetstrokecolor{currentstroke}%
\pgfsetdash{}{0pt}%
\pgfpathmoveto{\pgfqpoint{3.131434in}{2.158590in}}%
\pgfpathlineto{\pgfqpoint{2.939880in}{2.415292in}}%
\pgfusepath{stroke}%
\end{pgfscope}%
\begin{pgfscope}%
\pgfpathrectangle{\pgfqpoint{0.100000in}{0.212622in}}{\pgfqpoint{3.696000in}{3.696000in}}%
\pgfusepath{clip}%
\pgfsetrectcap%
\pgfsetroundjoin%
\pgfsetlinewidth{1.505625pt}%
\definecolor{currentstroke}{rgb}{1.000000,0.000000,0.000000}%
\pgfsetstrokecolor{currentstroke}%
\pgfsetdash{}{0pt}%
\pgfpathmoveto{\pgfqpoint{3.133117in}{2.162515in}}%
\pgfpathlineto{\pgfqpoint{2.939880in}{2.415292in}}%
\pgfusepath{stroke}%
\end{pgfscope}%
\begin{pgfscope}%
\pgfpathrectangle{\pgfqpoint{0.100000in}{0.212622in}}{\pgfqpoint{3.696000in}{3.696000in}}%
\pgfusepath{clip}%
\pgfsetrectcap%
\pgfsetroundjoin%
\pgfsetlinewidth{1.505625pt}%
\definecolor{currentstroke}{rgb}{1.000000,0.000000,0.000000}%
\pgfsetstrokecolor{currentstroke}%
\pgfsetdash{}{0pt}%
\pgfpathmoveto{\pgfqpoint{3.133769in}{2.164866in}}%
\pgfpathlineto{\pgfqpoint{2.949643in}{2.424151in}}%
\pgfusepath{stroke}%
\end{pgfscope}%
\begin{pgfscope}%
\pgfpathrectangle{\pgfqpoint{0.100000in}{0.212622in}}{\pgfqpoint{3.696000in}{3.696000in}}%
\pgfusepath{clip}%
\pgfsetrectcap%
\pgfsetroundjoin%
\pgfsetlinewidth{1.505625pt}%
\definecolor{currentstroke}{rgb}{1.000000,0.000000,0.000000}%
\pgfsetstrokecolor{currentstroke}%
\pgfsetdash{}{0pt}%
\pgfpathmoveto{\pgfqpoint{3.134204in}{2.166011in}}%
\pgfpathlineto{\pgfqpoint{2.949643in}{2.424151in}}%
\pgfusepath{stroke}%
\end{pgfscope}%
\begin{pgfscope}%
\pgfpathrectangle{\pgfqpoint{0.100000in}{0.212622in}}{\pgfqpoint{3.696000in}{3.696000in}}%
\pgfusepath{clip}%
\pgfsetrectcap%
\pgfsetroundjoin%
\pgfsetlinewidth{1.505625pt}%
\definecolor{currentstroke}{rgb}{1.000000,0.000000,0.000000}%
\pgfsetstrokecolor{currentstroke}%
\pgfsetdash{}{0pt}%
\pgfpathmoveto{\pgfqpoint{3.135217in}{2.167995in}}%
\pgfpathlineto{\pgfqpoint{2.949643in}{2.424151in}}%
\pgfusepath{stroke}%
\end{pgfscope}%
\begin{pgfscope}%
\pgfpathrectangle{\pgfqpoint{0.100000in}{0.212622in}}{\pgfqpoint{3.696000in}{3.696000in}}%
\pgfusepath{clip}%
\pgfsetrectcap%
\pgfsetroundjoin%
\pgfsetlinewidth{1.505625pt}%
\definecolor{currentstroke}{rgb}{1.000000,0.000000,0.000000}%
\pgfsetstrokecolor{currentstroke}%
\pgfsetdash{}{0pt}%
\pgfpathmoveto{\pgfqpoint{3.136661in}{2.169944in}}%
\pgfpathlineto{\pgfqpoint{2.949643in}{2.424151in}}%
\pgfusepath{stroke}%
\end{pgfscope}%
\begin{pgfscope}%
\pgfpathrectangle{\pgfqpoint{0.100000in}{0.212622in}}{\pgfqpoint{3.696000in}{3.696000in}}%
\pgfusepath{clip}%
\pgfsetrectcap%
\pgfsetroundjoin%
\pgfsetlinewidth{1.505625pt}%
\definecolor{currentstroke}{rgb}{1.000000,0.000000,0.000000}%
\pgfsetstrokecolor{currentstroke}%
\pgfsetdash{}{0pt}%
\pgfpathmoveto{\pgfqpoint{3.137441in}{2.171028in}}%
\pgfpathlineto{\pgfqpoint{2.949643in}{2.424151in}}%
\pgfusepath{stroke}%
\end{pgfscope}%
\begin{pgfscope}%
\pgfpathrectangle{\pgfqpoint{0.100000in}{0.212622in}}{\pgfqpoint{3.696000in}{3.696000in}}%
\pgfusepath{clip}%
\pgfsetrectcap%
\pgfsetroundjoin%
\pgfsetlinewidth{1.505625pt}%
\definecolor{currentstroke}{rgb}{1.000000,0.000000,0.000000}%
\pgfsetstrokecolor{currentstroke}%
\pgfsetdash{}{0pt}%
\pgfpathmoveto{\pgfqpoint{3.138822in}{2.173701in}}%
\pgfpathlineto{\pgfqpoint{2.949643in}{2.424151in}}%
\pgfusepath{stroke}%
\end{pgfscope}%
\begin{pgfscope}%
\pgfpathrectangle{\pgfqpoint{0.100000in}{0.212622in}}{\pgfqpoint{3.696000in}{3.696000in}}%
\pgfusepath{clip}%
\pgfsetrectcap%
\pgfsetroundjoin%
\pgfsetlinewidth{1.505625pt}%
\definecolor{currentstroke}{rgb}{1.000000,0.000000,0.000000}%
\pgfsetstrokecolor{currentstroke}%
\pgfsetdash{}{0pt}%
\pgfpathmoveto{\pgfqpoint{3.139638in}{2.177467in}}%
\pgfpathlineto{\pgfqpoint{2.949643in}{2.424151in}}%
\pgfusepath{stroke}%
\end{pgfscope}%
\begin{pgfscope}%
\pgfpathrectangle{\pgfqpoint{0.100000in}{0.212622in}}{\pgfqpoint{3.696000in}{3.696000in}}%
\pgfusepath{clip}%
\pgfsetrectcap%
\pgfsetroundjoin%
\pgfsetlinewidth{1.505625pt}%
\definecolor{currentstroke}{rgb}{1.000000,0.000000,0.000000}%
\pgfsetstrokecolor{currentstroke}%
\pgfsetdash{}{0pt}%
\pgfpathmoveto{\pgfqpoint{3.140554in}{2.181689in}}%
\pgfpathlineto{\pgfqpoint{2.949643in}{2.424151in}}%
\pgfusepath{stroke}%
\end{pgfscope}%
\begin{pgfscope}%
\pgfpathrectangle{\pgfqpoint{0.100000in}{0.212622in}}{\pgfqpoint{3.696000in}{3.696000in}}%
\pgfusepath{clip}%
\pgfsetrectcap%
\pgfsetroundjoin%
\pgfsetlinewidth{1.505625pt}%
\definecolor{currentstroke}{rgb}{1.000000,0.000000,0.000000}%
\pgfsetstrokecolor{currentstroke}%
\pgfsetdash{}{0pt}%
\pgfpathmoveto{\pgfqpoint{3.143063in}{2.187293in}}%
\pgfpathlineto{\pgfqpoint{2.959392in}{2.432995in}}%
\pgfusepath{stroke}%
\end{pgfscope}%
\begin{pgfscope}%
\pgfpathrectangle{\pgfqpoint{0.100000in}{0.212622in}}{\pgfqpoint{3.696000in}{3.696000in}}%
\pgfusepath{clip}%
\pgfsetrectcap%
\pgfsetroundjoin%
\pgfsetlinewidth{1.505625pt}%
\definecolor{currentstroke}{rgb}{1.000000,0.000000,0.000000}%
\pgfsetstrokecolor{currentstroke}%
\pgfsetdash{}{0pt}%
\pgfpathmoveto{\pgfqpoint{3.147345in}{2.193220in}}%
\pgfpathlineto{\pgfqpoint{2.959392in}{2.432995in}}%
\pgfusepath{stroke}%
\end{pgfscope}%
\begin{pgfscope}%
\pgfpathrectangle{\pgfqpoint{0.100000in}{0.212622in}}{\pgfqpoint{3.696000in}{3.696000in}}%
\pgfusepath{clip}%
\pgfsetrectcap%
\pgfsetroundjoin%
\pgfsetlinewidth{1.505625pt}%
\definecolor{currentstroke}{rgb}{1.000000,0.000000,0.000000}%
\pgfsetstrokecolor{currentstroke}%
\pgfsetdash{}{0pt}%
\pgfpathmoveto{\pgfqpoint{3.149714in}{2.196528in}}%
\pgfpathlineto{\pgfqpoint{2.959392in}{2.432995in}}%
\pgfusepath{stroke}%
\end{pgfscope}%
\begin{pgfscope}%
\pgfpathrectangle{\pgfqpoint{0.100000in}{0.212622in}}{\pgfqpoint{3.696000in}{3.696000in}}%
\pgfusepath{clip}%
\pgfsetrectcap%
\pgfsetroundjoin%
\pgfsetlinewidth{1.505625pt}%
\definecolor{currentstroke}{rgb}{1.000000,0.000000,0.000000}%
\pgfsetstrokecolor{currentstroke}%
\pgfsetdash{}{0pt}%
\pgfpathmoveto{\pgfqpoint{3.151489in}{2.202447in}}%
\pgfpathlineto{\pgfqpoint{2.959392in}{2.432995in}}%
\pgfusepath{stroke}%
\end{pgfscope}%
\begin{pgfscope}%
\pgfpathrectangle{\pgfqpoint{0.100000in}{0.212622in}}{\pgfqpoint{3.696000in}{3.696000in}}%
\pgfusepath{clip}%
\pgfsetrectcap%
\pgfsetroundjoin%
\pgfsetlinewidth{1.505625pt}%
\definecolor{currentstroke}{rgb}{1.000000,0.000000,0.000000}%
\pgfsetstrokecolor{currentstroke}%
\pgfsetdash{}{0pt}%
\pgfpathmoveto{\pgfqpoint{3.152038in}{2.205789in}}%
\pgfpathlineto{\pgfqpoint{2.969125in}{2.441826in}}%
\pgfusepath{stroke}%
\end{pgfscope}%
\begin{pgfscope}%
\pgfpathrectangle{\pgfqpoint{0.100000in}{0.212622in}}{\pgfqpoint{3.696000in}{3.696000in}}%
\pgfusepath{clip}%
\pgfsetrectcap%
\pgfsetroundjoin%
\pgfsetlinewidth{1.505625pt}%
\definecolor{currentstroke}{rgb}{1.000000,0.000000,0.000000}%
\pgfsetstrokecolor{currentstroke}%
\pgfsetdash{}{0pt}%
\pgfpathmoveto{\pgfqpoint{3.153150in}{2.209374in}}%
\pgfpathlineto{\pgfqpoint{2.969125in}{2.441826in}}%
\pgfusepath{stroke}%
\end{pgfscope}%
\begin{pgfscope}%
\pgfpathrectangle{\pgfqpoint{0.100000in}{0.212622in}}{\pgfqpoint{3.696000in}{3.696000in}}%
\pgfusepath{clip}%
\pgfsetrectcap%
\pgfsetroundjoin%
\pgfsetlinewidth{1.505625pt}%
\definecolor{currentstroke}{rgb}{1.000000,0.000000,0.000000}%
\pgfsetstrokecolor{currentstroke}%
\pgfsetdash{}{0pt}%
\pgfpathmoveto{\pgfqpoint{3.155743in}{2.214019in}}%
\pgfpathlineto{\pgfqpoint{2.969125in}{2.441826in}}%
\pgfusepath{stroke}%
\end{pgfscope}%
\begin{pgfscope}%
\pgfpathrectangle{\pgfqpoint{0.100000in}{0.212622in}}{\pgfqpoint{3.696000in}{3.696000in}}%
\pgfusepath{clip}%
\pgfsetrectcap%
\pgfsetroundjoin%
\pgfsetlinewidth{1.505625pt}%
\definecolor{currentstroke}{rgb}{1.000000,0.000000,0.000000}%
\pgfsetstrokecolor{currentstroke}%
\pgfsetdash{}{0pt}%
\pgfpathmoveto{\pgfqpoint{3.157465in}{2.216237in}}%
\pgfpathlineto{\pgfqpoint{2.969125in}{2.441826in}}%
\pgfusepath{stroke}%
\end{pgfscope}%
\begin{pgfscope}%
\pgfpathrectangle{\pgfqpoint{0.100000in}{0.212622in}}{\pgfqpoint{3.696000in}{3.696000in}}%
\pgfusepath{clip}%
\pgfsetrectcap%
\pgfsetroundjoin%
\pgfsetlinewidth{1.505625pt}%
\definecolor{currentstroke}{rgb}{1.000000,0.000000,0.000000}%
\pgfsetstrokecolor{currentstroke}%
\pgfsetdash{}{0pt}%
\pgfpathmoveto{\pgfqpoint{3.159835in}{2.219478in}}%
\pgfpathlineto{\pgfqpoint{2.969125in}{2.441826in}}%
\pgfusepath{stroke}%
\end{pgfscope}%
\begin{pgfscope}%
\pgfpathrectangle{\pgfqpoint{0.100000in}{0.212622in}}{\pgfqpoint{3.696000in}{3.696000in}}%
\pgfusepath{clip}%
\pgfsetrectcap%
\pgfsetroundjoin%
\pgfsetlinewidth{1.505625pt}%
\definecolor{currentstroke}{rgb}{1.000000,0.000000,0.000000}%
\pgfsetstrokecolor{currentstroke}%
\pgfsetdash{}{0pt}%
\pgfpathmoveto{\pgfqpoint{3.161454in}{2.226466in}}%
\pgfpathlineto{\pgfqpoint{2.978843in}{2.450643in}}%
\pgfusepath{stroke}%
\end{pgfscope}%
\begin{pgfscope}%
\pgfpathrectangle{\pgfqpoint{0.100000in}{0.212622in}}{\pgfqpoint{3.696000in}{3.696000in}}%
\pgfusepath{clip}%
\pgfsetrectcap%
\pgfsetroundjoin%
\pgfsetlinewidth{1.505625pt}%
\definecolor{currentstroke}{rgb}{1.000000,0.000000,0.000000}%
\pgfsetstrokecolor{currentstroke}%
\pgfsetdash{}{0pt}%
\pgfpathmoveto{\pgfqpoint{3.162805in}{2.234504in}}%
\pgfpathlineto{\pgfqpoint{2.978843in}{2.450643in}}%
\pgfusepath{stroke}%
\end{pgfscope}%
\begin{pgfscope}%
\pgfpathrectangle{\pgfqpoint{0.100000in}{0.212622in}}{\pgfqpoint{3.696000in}{3.696000in}}%
\pgfusepath{clip}%
\pgfsetrectcap%
\pgfsetroundjoin%
\pgfsetlinewidth{1.505625pt}%
\definecolor{currentstroke}{rgb}{1.000000,0.000000,0.000000}%
\pgfsetstrokecolor{currentstroke}%
\pgfsetdash{}{0pt}%
\pgfpathmoveto{\pgfqpoint{3.164747in}{2.242783in}}%
\pgfpathlineto{\pgfqpoint{2.978843in}{2.450643in}}%
\pgfusepath{stroke}%
\end{pgfscope}%
\begin{pgfscope}%
\pgfpathrectangle{\pgfqpoint{0.100000in}{0.212622in}}{\pgfqpoint{3.696000in}{3.696000in}}%
\pgfusepath{clip}%
\pgfsetrectcap%
\pgfsetroundjoin%
\pgfsetlinewidth{1.505625pt}%
\definecolor{currentstroke}{rgb}{1.000000,0.000000,0.000000}%
\pgfsetstrokecolor{currentstroke}%
\pgfsetdash{}{0pt}%
\pgfpathmoveto{\pgfqpoint{3.169977in}{2.251523in}}%
\pgfpathlineto{\pgfqpoint{2.988545in}{2.459446in}}%
\pgfusepath{stroke}%
\end{pgfscope}%
\begin{pgfscope}%
\pgfpathrectangle{\pgfqpoint{0.100000in}{0.212622in}}{\pgfqpoint{3.696000in}{3.696000in}}%
\pgfusepath{clip}%
\pgfsetrectcap%
\pgfsetroundjoin%
\pgfsetlinewidth{1.505625pt}%
\definecolor{currentstroke}{rgb}{1.000000,0.000000,0.000000}%
\pgfsetstrokecolor{currentstroke}%
\pgfsetdash{}{0pt}%
\pgfpathmoveto{\pgfqpoint{3.176190in}{2.259266in}}%
\pgfpathlineto{\pgfqpoint{2.988545in}{2.459446in}}%
\pgfusepath{stroke}%
\end{pgfscope}%
\begin{pgfscope}%
\pgfpathrectangle{\pgfqpoint{0.100000in}{0.212622in}}{\pgfqpoint{3.696000in}{3.696000in}}%
\pgfusepath{clip}%
\pgfsetrectcap%
\pgfsetroundjoin%
\pgfsetlinewidth{1.505625pt}%
\definecolor{currentstroke}{rgb}{1.000000,0.000000,0.000000}%
\pgfsetstrokecolor{currentstroke}%
\pgfsetdash{}{0pt}%
\pgfpathmoveto{\pgfqpoint{3.182489in}{2.270544in}}%
\pgfpathlineto{\pgfqpoint{2.998232in}{2.468236in}}%
\pgfusepath{stroke}%
\end{pgfscope}%
\begin{pgfscope}%
\pgfpathrectangle{\pgfqpoint{0.100000in}{0.212622in}}{\pgfqpoint{3.696000in}{3.696000in}}%
\pgfusepath{clip}%
\pgfsetrectcap%
\pgfsetroundjoin%
\pgfsetlinewidth{1.505625pt}%
\definecolor{currentstroke}{rgb}{1.000000,0.000000,0.000000}%
\pgfsetstrokecolor{currentstroke}%
\pgfsetdash{}{0pt}%
\pgfpathmoveto{\pgfqpoint{3.186344in}{2.285439in}}%
\pgfpathlineto{\pgfqpoint{2.998232in}{2.468236in}}%
\pgfusepath{stroke}%
\end{pgfscope}%
\begin{pgfscope}%
\pgfpathrectangle{\pgfqpoint{0.100000in}{0.212622in}}{\pgfqpoint{3.696000in}{3.696000in}}%
\pgfusepath{clip}%
\pgfsetrectcap%
\pgfsetroundjoin%
\pgfsetlinewidth{1.505625pt}%
\definecolor{currentstroke}{rgb}{1.000000,0.000000,0.000000}%
\pgfsetstrokecolor{currentstroke}%
\pgfsetdash{}{0pt}%
\pgfpathmoveto{\pgfqpoint{3.187131in}{2.294174in}}%
\pgfpathlineto{\pgfqpoint{3.007904in}{2.477011in}}%
\pgfusepath{stroke}%
\end{pgfscope}%
\begin{pgfscope}%
\pgfpathrectangle{\pgfqpoint{0.100000in}{0.212622in}}{\pgfqpoint{3.696000in}{3.696000in}}%
\pgfusepath{clip}%
\pgfsetrectcap%
\pgfsetroundjoin%
\pgfsetlinewidth{1.505625pt}%
\definecolor{currentstroke}{rgb}{1.000000,0.000000,0.000000}%
\pgfsetstrokecolor{currentstroke}%
\pgfsetdash{}{0pt}%
\pgfpathmoveto{\pgfqpoint{3.190314in}{2.303190in}}%
\pgfpathlineto{\pgfqpoint{3.007904in}{2.477011in}}%
\pgfusepath{stroke}%
\end{pgfscope}%
\begin{pgfscope}%
\pgfpathrectangle{\pgfqpoint{0.100000in}{0.212622in}}{\pgfqpoint{3.696000in}{3.696000in}}%
\pgfusepath{clip}%
\pgfsetrectcap%
\pgfsetroundjoin%
\pgfsetlinewidth{1.505625pt}%
\definecolor{currentstroke}{rgb}{1.000000,0.000000,0.000000}%
\pgfsetstrokecolor{currentstroke}%
\pgfsetdash{}{0pt}%
\pgfpathmoveto{\pgfqpoint{3.195390in}{2.312475in}}%
\pgfpathlineto{\pgfqpoint{3.007904in}{2.477011in}}%
\pgfusepath{stroke}%
\end{pgfscope}%
\begin{pgfscope}%
\pgfpathrectangle{\pgfqpoint{0.100000in}{0.212622in}}{\pgfqpoint{3.696000in}{3.696000in}}%
\pgfusepath{clip}%
\pgfsetrectcap%
\pgfsetroundjoin%
\pgfsetlinewidth{1.505625pt}%
\definecolor{currentstroke}{rgb}{1.000000,0.000000,0.000000}%
\pgfsetstrokecolor{currentstroke}%
\pgfsetdash{}{0pt}%
\pgfpathmoveto{\pgfqpoint{3.198612in}{2.317157in}}%
\pgfpathlineto{\pgfqpoint{3.017561in}{2.485773in}}%
\pgfusepath{stroke}%
\end{pgfscope}%
\begin{pgfscope}%
\pgfpathrectangle{\pgfqpoint{0.100000in}{0.212622in}}{\pgfqpoint{3.696000in}{3.696000in}}%
\pgfusepath{clip}%
\pgfsetrectcap%
\pgfsetroundjoin%
\pgfsetlinewidth{1.505625pt}%
\definecolor{currentstroke}{rgb}{1.000000,0.000000,0.000000}%
\pgfsetstrokecolor{currentstroke}%
\pgfsetdash{}{0pt}%
\pgfpathmoveto{\pgfqpoint{3.202707in}{2.324307in}}%
\pgfpathlineto{\pgfqpoint{3.017561in}{2.485773in}}%
\pgfusepath{stroke}%
\end{pgfscope}%
\begin{pgfscope}%
\pgfpathrectangle{\pgfqpoint{0.100000in}{0.212622in}}{\pgfqpoint{3.696000in}{3.696000in}}%
\pgfusepath{clip}%
\pgfsetrectcap%
\pgfsetroundjoin%
\pgfsetlinewidth{1.505625pt}%
\definecolor{currentstroke}{rgb}{1.000000,0.000000,0.000000}%
\pgfsetstrokecolor{currentstroke}%
\pgfsetdash{}{0pt}%
\pgfpathmoveto{\pgfqpoint{3.204622in}{2.334866in}}%
\pgfpathlineto{\pgfqpoint{3.017561in}{2.485773in}}%
\pgfusepath{stroke}%
\end{pgfscope}%
\begin{pgfscope}%
\pgfpathrectangle{\pgfqpoint{0.100000in}{0.212622in}}{\pgfqpoint{3.696000in}{3.696000in}}%
\pgfusepath{clip}%
\pgfsetrectcap%
\pgfsetroundjoin%
\pgfsetlinewidth{1.505625pt}%
\definecolor{currentstroke}{rgb}{1.000000,0.000000,0.000000}%
\pgfsetstrokecolor{currentstroke}%
\pgfsetdash{}{0pt}%
\pgfpathmoveto{\pgfqpoint{3.205228in}{2.346342in}}%
\pgfpathlineto{\pgfqpoint{3.025826in}{2.493272in}}%
\pgfusepath{stroke}%
\end{pgfscope}%
\begin{pgfscope}%
\pgfpathrectangle{\pgfqpoint{0.100000in}{0.212622in}}{\pgfqpoint{3.696000in}{3.696000in}}%
\pgfusepath{clip}%
\pgfsetrectcap%
\pgfsetroundjoin%
\pgfsetlinewidth{1.505625pt}%
\definecolor{currentstroke}{rgb}{1.000000,0.000000,0.000000}%
\pgfsetstrokecolor{currentstroke}%
\pgfsetdash{}{0pt}%
\pgfpathmoveto{\pgfqpoint{3.208070in}{2.358911in}}%
\pgfpathlineto{\pgfqpoint{3.029954in}{2.497018in}}%
\pgfusepath{stroke}%
\end{pgfscope}%
\begin{pgfscope}%
\pgfpathrectangle{\pgfqpoint{0.100000in}{0.212622in}}{\pgfqpoint{3.696000in}{3.696000in}}%
\pgfusepath{clip}%
\pgfsetrectcap%
\pgfsetroundjoin%
\pgfsetlinewidth{1.505625pt}%
\definecolor{currentstroke}{rgb}{1.000000,0.000000,0.000000}%
\pgfsetstrokecolor{currentstroke}%
\pgfsetdash{}{0pt}%
\pgfpathmoveto{\pgfqpoint{3.215049in}{2.371183in}}%
\pgfpathlineto{\pgfqpoint{3.035455in}{2.502008in}}%
\pgfusepath{stroke}%
\end{pgfscope}%
\begin{pgfscope}%
\pgfpathrectangle{\pgfqpoint{0.100000in}{0.212622in}}{\pgfqpoint{3.696000in}{3.696000in}}%
\pgfusepath{clip}%
\pgfsetrectcap%
\pgfsetroundjoin%
\pgfsetlinewidth{1.505625pt}%
\definecolor{currentstroke}{rgb}{1.000000,0.000000,0.000000}%
\pgfsetstrokecolor{currentstroke}%
\pgfsetdash{}{0pt}%
\pgfpathmoveto{\pgfqpoint{3.219293in}{2.376867in}}%
\pgfpathlineto{\pgfqpoint{3.038203in}{2.504502in}}%
\pgfusepath{stroke}%
\end{pgfscope}%
\begin{pgfscope}%
\pgfpathrectangle{\pgfqpoint{0.100000in}{0.212622in}}{\pgfqpoint{3.696000in}{3.696000in}}%
\pgfusepath{clip}%
\pgfsetrectcap%
\pgfsetroundjoin%
\pgfsetlinewidth{1.505625pt}%
\definecolor{currentstroke}{rgb}{1.000000,0.000000,0.000000}%
\pgfsetstrokecolor{currentstroke}%
\pgfsetdash{}{0pt}%
\pgfpathmoveto{\pgfqpoint{3.223835in}{2.383132in}}%
\pgfpathlineto{\pgfqpoint{3.042323in}{2.508240in}}%
\pgfusepath{stroke}%
\end{pgfscope}%
\begin{pgfscope}%
\pgfpathrectangle{\pgfqpoint{0.100000in}{0.212622in}}{\pgfqpoint{3.696000in}{3.696000in}}%
\pgfusepath{clip}%
\pgfsetrectcap%
\pgfsetroundjoin%
\pgfsetlinewidth{1.505625pt}%
\definecolor{currentstroke}{rgb}{1.000000,0.000000,0.000000}%
\pgfsetstrokecolor{currentstroke}%
\pgfsetdash{}{0pt}%
\pgfpathmoveto{\pgfqpoint{3.227200in}{2.393205in}}%
\pgfpathlineto{\pgfqpoint{3.046440in}{2.511976in}}%
\pgfusepath{stroke}%
\end{pgfscope}%
\begin{pgfscope}%
\pgfpathrectangle{\pgfqpoint{0.100000in}{0.212622in}}{\pgfqpoint{3.696000in}{3.696000in}}%
\pgfusepath{clip}%
\pgfsetrectcap%
\pgfsetroundjoin%
\pgfsetlinewidth{1.505625pt}%
\definecolor{currentstroke}{rgb}{1.000000,0.000000,0.000000}%
\pgfsetstrokecolor{currentstroke}%
\pgfsetdash{}{0pt}%
\pgfpathmoveto{\pgfqpoint{3.228416in}{2.405928in}}%
\pgfpathlineto{\pgfqpoint{3.050555in}{2.515709in}}%
\pgfusepath{stroke}%
\end{pgfscope}%
\begin{pgfscope}%
\pgfpathrectangle{\pgfqpoint{0.100000in}{0.212622in}}{\pgfqpoint{3.696000in}{3.696000in}}%
\pgfusepath{clip}%
\pgfsetrectcap%
\pgfsetroundjoin%
\pgfsetlinewidth{1.505625pt}%
\definecolor{currentstroke}{rgb}{1.000000,0.000000,0.000000}%
\pgfsetstrokecolor{currentstroke}%
\pgfsetdash{}{0pt}%
\pgfpathmoveto{\pgfqpoint{3.229485in}{2.412242in}}%
\pgfpathlineto{\pgfqpoint{3.053297in}{2.518196in}}%
\pgfusepath{stroke}%
\end{pgfscope}%
\begin{pgfscope}%
\pgfpathrectangle{\pgfqpoint{0.100000in}{0.212622in}}{\pgfqpoint{3.696000in}{3.696000in}}%
\pgfusepath{clip}%
\pgfsetrectcap%
\pgfsetroundjoin%
\pgfsetlinewidth{1.505625pt}%
\definecolor{currentstroke}{rgb}{1.000000,0.000000,0.000000}%
\pgfsetstrokecolor{currentstroke}%
\pgfsetdash{}{0pt}%
\pgfpathmoveto{\pgfqpoint{3.233258in}{2.419281in}}%
\pgfpathlineto{\pgfqpoint{3.056037in}{2.520683in}}%
\pgfusepath{stroke}%
\end{pgfscope}%
\begin{pgfscope}%
\pgfpathrectangle{\pgfqpoint{0.100000in}{0.212622in}}{\pgfqpoint{3.696000in}{3.696000in}}%
\pgfusepath{clip}%
\pgfsetrectcap%
\pgfsetroundjoin%
\pgfsetlinewidth{1.505625pt}%
\definecolor{currentstroke}{rgb}{1.000000,0.000000,0.000000}%
\pgfsetstrokecolor{currentstroke}%
\pgfsetdash{}{0pt}%
\pgfpathmoveto{\pgfqpoint{3.238491in}{2.426525in}}%
\pgfpathlineto{\pgfqpoint{3.060145in}{2.524410in}}%
\pgfusepath{stroke}%
\end{pgfscope}%
\begin{pgfscope}%
\pgfpathrectangle{\pgfqpoint{0.100000in}{0.212622in}}{\pgfqpoint{3.696000in}{3.696000in}}%
\pgfusepath{clip}%
\pgfsetrectcap%
\pgfsetroundjoin%
\pgfsetlinewidth{1.505625pt}%
\definecolor{currentstroke}{rgb}{1.000000,0.000000,0.000000}%
\pgfsetstrokecolor{currentstroke}%
\pgfsetdash{}{0pt}%
\pgfpathmoveto{\pgfqpoint{3.241377in}{2.430642in}}%
\pgfpathlineto{\pgfqpoint{3.061514in}{2.525652in}}%
\pgfusepath{stroke}%
\end{pgfscope}%
\begin{pgfscope}%
\pgfpathrectangle{\pgfqpoint{0.100000in}{0.212622in}}{\pgfqpoint{3.696000in}{3.696000in}}%
\pgfusepath{clip}%
\pgfsetrectcap%
\pgfsetroundjoin%
\pgfsetlinewidth{1.505625pt}%
\definecolor{currentstroke}{rgb}{1.000000,0.000000,0.000000}%
\pgfsetstrokecolor{currentstroke}%
\pgfsetdash{}{0pt}%
\pgfpathmoveto{\pgfqpoint{3.243524in}{2.438040in}}%
\pgfpathlineto{\pgfqpoint{3.065618in}{2.529376in}}%
\pgfusepath{stroke}%
\end{pgfscope}%
\begin{pgfscope}%
\pgfpathrectangle{\pgfqpoint{0.100000in}{0.212622in}}{\pgfqpoint{3.696000in}{3.696000in}}%
\pgfusepath{clip}%
\pgfsetrectcap%
\pgfsetroundjoin%
\pgfsetlinewidth{1.505625pt}%
\definecolor{currentstroke}{rgb}{1.000000,0.000000,0.000000}%
\pgfsetstrokecolor{currentstroke}%
\pgfsetdash{}{0pt}%
\pgfpathmoveto{\pgfqpoint{3.244931in}{2.446924in}}%
\pgfpathlineto{\pgfqpoint{3.068353in}{2.531857in}}%
\pgfusepath{stroke}%
\end{pgfscope}%
\begin{pgfscope}%
\pgfpathrectangle{\pgfqpoint{0.100000in}{0.212622in}}{\pgfqpoint{3.696000in}{3.696000in}}%
\pgfusepath{clip}%
\pgfsetrectcap%
\pgfsetroundjoin%
\pgfsetlinewidth{1.505625pt}%
\definecolor{currentstroke}{rgb}{1.000000,0.000000,0.000000}%
\pgfsetstrokecolor{currentstroke}%
\pgfsetdash{}{0pt}%
\pgfpathmoveto{\pgfqpoint{3.246896in}{2.455624in}}%
\pgfpathlineto{\pgfqpoint{3.071087in}{2.534338in}}%
\pgfusepath{stroke}%
\end{pgfscope}%
\begin{pgfscope}%
\pgfpathrectangle{\pgfqpoint{0.100000in}{0.212622in}}{\pgfqpoint{3.696000in}{3.696000in}}%
\pgfusepath{clip}%
\pgfsetrectcap%
\pgfsetroundjoin%
\pgfsetlinewidth{1.505625pt}%
\definecolor{currentstroke}{rgb}{1.000000,0.000000,0.000000}%
\pgfsetstrokecolor{currentstroke}%
\pgfsetdash{}{0pt}%
\pgfpathmoveto{\pgfqpoint{3.251022in}{2.465590in}}%
\pgfpathlineto{\pgfqpoint{3.075185in}{2.538056in}}%
\pgfusepath{stroke}%
\end{pgfscope}%
\begin{pgfscope}%
\pgfpathrectangle{\pgfqpoint{0.100000in}{0.212622in}}{\pgfqpoint{3.696000in}{3.696000in}}%
\pgfusepath{clip}%
\pgfsetrectcap%
\pgfsetroundjoin%
\pgfsetlinewidth{1.505625pt}%
\definecolor{currentstroke}{rgb}{1.000000,0.000000,0.000000}%
\pgfsetstrokecolor{currentstroke}%
\pgfsetdash{}{0pt}%
\pgfpathmoveto{\pgfqpoint{3.256897in}{2.476187in}}%
\pgfpathlineto{\pgfqpoint{3.080645in}{2.543010in}}%
\pgfusepath{stroke}%
\end{pgfscope}%
\begin{pgfscope}%
\pgfpathrectangle{\pgfqpoint{0.100000in}{0.212622in}}{\pgfqpoint{3.696000in}{3.696000in}}%
\pgfusepath{clip}%
\pgfsetrectcap%
\pgfsetroundjoin%
\pgfsetlinewidth{1.505625pt}%
\definecolor{currentstroke}{rgb}{1.000000,0.000000,0.000000}%
\pgfsetstrokecolor{currentstroke}%
\pgfsetdash{}{0pt}%
\pgfpathmoveto{\pgfqpoint{3.264085in}{2.485679in}}%
\pgfpathlineto{\pgfqpoint{3.084736in}{2.546722in}}%
\pgfusepath{stroke}%
\end{pgfscope}%
\begin{pgfscope}%
\pgfpathrectangle{\pgfqpoint{0.100000in}{0.212622in}}{\pgfqpoint{3.696000in}{3.696000in}}%
\pgfusepath{clip}%
\pgfsetrectcap%
\pgfsetroundjoin%
\pgfsetlinewidth{1.505625pt}%
\definecolor{currentstroke}{rgb}{1.000000,0.000000,0.000000}%
\pgfsetstrokecolor{currentstroke}%
\pgfsetdash{}{0pt}%
\pgfpathmoveto{\pgfqpoint{3.270884in}{2.498877in}}%
\pgfpathlineto{\pgfqpoint{3.090188in}{2.551668in}}%
\pgfusepath{stroke}%
\end{pgfscope}%
\begin{pgfscope}%
\pgfpathrectangle{\pgfqpoint{0.100000in}{0.212622in}}{\pgfqpoint{3.696000in}{3.696000in}}%
\pgfusepath{clip}%
\pgfsetrectcap%
\pgfsetroundjoin%
\pgfsetlinewidth{1.505625pt}%
\definecolor{currentstroke}{rgb}{1.000000,0.000000,0.000000}%
\pgfsetstrokecolor{currentstroke}%
\pgfsetdash{}{0pt}%
\pgfpathmoveto{\pgfqpoint{3.273995in}{2.518507in}}%
\pgfpathlineto{\pgfqpoint{3.096995in}{2.557845in}}%
\pgfusepath{stroke}%
\end{pgfscope}%
\begin{pgfscope}%
\pgfpathrectangle{\pgfqpoint{0.100000in}{0.212622in}}{\pgfqpoint{3.696000in}{3.696000in}}%
\pgfusepath{clip}%
\pgfsetrectcap%
\pgfsetroundjoin%
\pgfsetlinewidth{1.505625pt}%
\definecolor{currentstroke}{rgb}{1.000000,0.000000,0.000000}%
\pgfsetstrokecolor{currentstroke}%
\pgfsetdash{}{0pt}%
\pgfpathmoveto{\pgfqpoint{3.275482in}{2.528798in}}%
\pgfpathlineto{\pgfqpoint{3.101076in}{2.561547in}}%
\pgfusepath{stroke}%
\end{pgfscope}%
\begin{pgfscope}%
\pgfpathrectangle{\pgfqpoint{0.100000in}{0.212622in}}{\pgfqpoint{3.696000in}{3.696000in}}%
\pgfusepath{clip}%
\pgfsetrectcap%
\pgfsetroundjoin%
\pgfsetlinewidth{1.505625pt}%
\definecolor{currentstroke}{rgb}{1.000000,0.000000,0.000000}%
\pgfsetstrokecolor{currentstroke}%
\pgfsetdash{}{0pt}%
\pgfpathmoveto{\pgfqpoint{3.280244in}{2.539114in}}%
\pgfpathlineto{\pgfqpoint{3.105154in}{2.565247in}}%
\pgfusepath{stroke}%
\end{pgfscope}%
\begin{pgfscope}%
\pgfpathrectangle{\pgfqpoint{0.100000in}{0.212622in}}{\pgfqpoint{3.696000in}{3.696000in}}%
\pgfusepath{clip}%
\pgfsetrectcap%
\pgfsetroundjoin%
\pgfsetlinewidth{1.505625pt}%
\definecolor{currentstroke}{rgb}{1.000000,0.000000,0.000000}%
\pgfsetstrokecolor{currentstroke}%
\pgfsetdash{}{0pt}%
\pgfpathmoveto{\pgfqpoint{3.287355in}{2.549874in}}%
\pgfpathlineto{\pgfqpoint{3.110587in}{2.570177in}}%
\pgfusepath{stroke}%
\end{pgfscope}%
\begin{pgfscope}%
\pgfpathrectangle{\pgfqpoint{0.100000in}{0.212622in}}{\pgfqpoint{3.696000in}{3.696000in}}%
\pgfusepath{clip}%
\pgfsetrectcap%
\pgfsetroundjoin%
\pgfsetlinewidth{1.505625pt}%
\definecolor{currentstroke}{rgb}{1.000000,0.000000,0.000000}%
\pgfsetstrokecolor{currentstroke}%
\pgfsetdash{}{0pt}%
\pgfpathmoveto{\pgfqpoint{3.295108in}{2.561494in}}%
\pgfpathlineto{\pgfqpoint{3.116015in}{2.575102in}}%
\pgfusepath{stroke}%
\end{pgfscope}%
\begin{pgfscope}%
\pgfpathrectangle{\pgfqpoint{0.100000in}{0.212622in}}{\pgfqpoint{3.696000in}{3.696000in}}%
\pgfusepath{clip}%
\pgfsetrectcap%
\pgfsetroundjoin%
\pgfsetlinewidth{1.505625pt}%
\definecolor{currentstroke}{rgb}{1.000000,0.000000,0.000000}%
\pgfsetstrokecolor{currentstroke}%
\pgfsetdash{}{0pt}%
\pgfpathmoveto{\pgfqpoint{3.301978in}{2.575954in}}%
\pgfpathlineto{\pgfqpoint{3.122794in}{2.581252in}}%
\pgfusepath{stroke}%
\end{pgfscope}%
\begin{pgfscope}%
\pgfpathrectangle{\pgfqpoint{0.100000in}{0.212622in}}{\pgfqpoint{3.696000in}{3.696000in}}%
\pgfusepath{clip}%
\pgfsetrectcap%
\pgfsetroundjoin%
\pgfsetlinewidth{1.505625pt}%
\definecolor{currentstroke}{rgb}{1.000000,0.000000,0.000000}%
\pgfsetstrokecolor{currentstroke}%
\pgfsetdash{}{0pt}%
\pgfpathmoveto{\pgfqpoint{3.305052in}{2.593643in}}%
\pgfpathlineto{\pgfqpoint{3.129565in}{2.587396in}}%
\pgfusepath{stroke}%
\end{pgfscope}%
\begin{pgfscope}%
\pgfpathrectangle{\pgfqpoint{0.100000in}{0.212622in}}{\pgfqpoint{3.696000in}{3.696000in}}%
\pgfusepath{clip}%
\pgfsetrectcap%
\pgfsetroundjoin%
\pgfsetlinewidth{1.505625pt}%
\definecolor{currentstroke}{rgb}{1.000000,0.000000,0.000000}%
\pgfsetstrokecolor{currentstroke}%
\pgfsetdash{}{0pt}%
\pgfpathmoveto{\pgfqpoint{3.308452in}{2.611606in}}%
\pgfpathlineto{\pgfqpoint{3.134977in}{2.592306in}}%
\pgfusepath{stroke}%
\end{pgfscope}%
\begin{pgfscope}%
\pgfpathrectangle{\pgfqpoint{0.100000in}{0.212622in}}{\pgfqpoint{3.696000in}{3.696000in}}%
\pgfusepath{clip}%
\pgfsetrectcap%
\pgfsetroundjoin%
\pgfsetlinewidth{1.505625pt}%
\definecolor{currentstroke}{rgb}{1.000000,0.000000,0.000000}%
\pgfsetstrokecolor{currentstroke}%
\pgfsetdash{}{0pt}%
\pgfpathmoveto{\pgfqpoint{3.311897in}{2.621173in}}%
\pgfpathlineto{\pgfqpoint{3.139032in}{2.595985in}}%
\pgfusepath{stroke}%
\end{pgfscope}%
\begin{pgfscope}%
\pgfpathrectangle{\pgfqpoint{0.100000in}{0.212622in}}{\pgfqpoint{3.696000in}{3.696000in}}%
\pgfusepath{clip}%
\pgfsetrectcap%
\pgfsetroundjoin%
\pgfsetlinewidth{1.505625pt}%
\definecolor{currentstroke}{rgb}{1.000000,0.000000,0.000000}%
\pgfsetstrokecolor{currentstroke}%
\pgfsetdash{}{0pt}%
\pgfpathmoveto{\pgfqpoint{3.313807in}{2.625717in}}%
\pgfpathlineto{\pgfqpoint{3.141734in}{2.598437in}}%
\pgfusepath{stroke}%
\end{pgfscope}%
\begin{pgfscope}%
\pgfpathrectangle{\pgfqpoint{0.100000in}{0.212622in}}{\pgfqpoint{3.696000in}{3.696000in}}%
\pgfusepath{clip}%
\pgfsetrectcap%
\pgfsetroundjoin%
\pgfsetlinewidth{1.505625pt}%
\definecolor{currentstroke}{rgb}{1.000000,0.000000,0.000000}%
\pgfsetstrokecolor{currentstroke}%
\pgfsetdash{}{0pt}%
\pgfpathmoveto{\pgfqpoint{3.316100in}{2.630414in}}%
\pgfpathlineto{\pgfqpoint{3.144435in}{2.600887in}}%
\pgfusepath{stroke}%
\end{pgfscope}%
\begin{pgfscope}%
\pgfpathrectangle{\pgfqpoint{0.100000in}{0.212622in}}{\pgfqpoint{3.696000in}{3.696000in}}%
\pgfusepath{clip}%
\pgfsetrectcap%
\pgfsetroundjoin%
\pgfsetlinewidth{1.505625pt}%
\definecolor{currentstroke}{rgb}{1.000000,0.000000,0.000000}%
\pgfsetstrokecolor{currentstroke}%
\pgfsetdash{}{0pt}%
\pgfpathmoveto{\pgfqpoint{3.317512in}{2.632988in}}%
\pgfpathlineto{\pgfqpoint{3.145785in}{2.602112in}}%
\pgfusepath{stroke}%
\end{pgfscope}%
\begin{pgfscope}%
\pgfpathrectangle{\pgfqpoint{0.100000in}{0.212622in}}{\pgfqpoint{3.696000in}{3.696000in}}%
\pgfusepath{clip}%
\pgfsetrectcap%
\pgfsetroundjoin%
\pgfsetlinewidth{1.505625pt}%
\definecolor{currentstroke}{rgb}{1.000000,0.000000,0.000000}%
\pgfsetstrokecolor{currentstroke}%
\pgfsetdash{}{0pt}%
\pgfpathmoveto{\pgfqpoint{3.319428in}{2.635729in}}%
\pgfpathlineto{\pgfqpoint{3.147135in}{2.603337in}}%
\pgfusepath{stroke}%
\end{pgfscope}%
\begin{pgfscope}%
\pgfpathrectangle{\pgfqpoint{0.100000in}{0.212622in}}{\pgfqpoint{3.696000in}{3.696000in}}%
\pgfusepath{clip}%
\pgfsetrectcap%
\pgfsetroundjoin%
\pgfsetlinewidth{1.505625pt}%
\definecolor{currentstroke}{rgb}{1.000000,0.000000,0.000000}%
\pgfsetstrokecolor{currentstroke}%
\pgfsetdash{}{0pt}%
\pgfpathmoveto{\pgfqpoint{3.320164in}{2.637374in}}%
\pgfpathlineto{\pgfqpoint{3.147135in}{2.603337in}}%
\pgfusepath{stroke}%
\end{pgfscope}%
\begin{pgfscope}%
\pgfpathrectangle{\pgfqpoint{0.100000in}{0.212622in}}{\pgfqpoint{3.696000in}{3.696000in}}%
\pgfusepath{clip}%
\pgfsetrectcap%
\pgfsetroundjoin%
\pgfsetlinewidth{1.505625pt}%
\definecolor{currentstroke}{rgb}{1.000000,0.000000,0.000000}%
\pgfsetstrokecolor{currentstroke}%
\pgfsetdash{}{0pt}%
\pgfpathmoveto{\pgfqpoint{3.320542in}{2.638244in}}%
\pgfpathlineto{\pgfqpoint{3.148484in}{2.604561in}}%
\pgfusepath{stroke}%
\end{pgfscope}%
\begin{pgfscope}%
\pgfpathrectangle{\pgfqpoint{0.100000in}{0.212622in}}{\pgfqpoint{3.696000in}{3.696000in}}%
\pgfusepath{clip}%
\pgfsetrectcap%
\pgfsetroundjoin%
\pgfsetlinewidth{1.505625pt}%
\definecolor{currentstroke}{rgb}{1.000000,0.000000,0.000000}%
\pgfsetstrokecolor{currentstroke}%
\pgfsetdash{}{0pt}%
\pgfpathmoveto{\pgfqpoint{3.320702in}{2.638728in}}%
\pgfpathlineto{\pgfqpoint{3.148484in}{2.604561in}}%
\pgfusepath{stroke}%
\end{pgfscope}%
\begin{pgfscope}%
\pgfpathrectangle{\pgfqpoint{0.100000in}{0.212622in}}{\pgfqpoint{3.696000in}{3.696000in}}%
\pgfusepath{clip}%
\pgfsetrectcap%
\pgfsetroundjoin%
\pgfsetlinewidth{1.505625pt}%
\definecolor{currentstroke}{rgb}{1.000000,0.000000,0.000000}%
\pgfsetstrokecolor{currentstroke}%
\pgfsetdash{}{0pt}%
\pgfpathmoveto{\pgfqpoint{3.320626in}{2.638913in}}%
\pgfpathlineto{\pgfqpoint{3.148484in}{2.604561in}}%
\pgfusepath{stroke}%
\end{pgfscope}%
\begin{pgfscope}%
\pgfpathrectangle{\pgfqpoint{0.100000in}{0.212622in}}{\pgfqpoint{3.696000in}{3.696000in}}%
\pgfusepath{clip}%
\pgfsetrectcap%
\pgfsetroundjoin%
\pgfsetlinewidth{1.505625pt}%
\definecolor{currentstroke}{rgb}{1.000000,0.000000,0.000000}%
\pgfsetstrokecolor{currentstroke}%
\pgfsetdash{}{0pt}%
\pgfpathmoveto{\pgfqpoint{3.320062in}{2.639127in}}%
\pgfpathlineto{\pgfqpoint{3.148484in}{2.604561in}}%
\pgfusepath{stroke}%
\end{pgfscope}%
\begin{pgfscope}%
\pgfpathrectangle{\pgfqpoint{0.100000in}{0.212622in}}{\pgfqpoint{3.696000in}{3.696000in}}%
\pgfusepath{clip}%
\pgfsetrectcap%
\pgfsetroundjoin%
\pgfsetlinewidth{1.505625pt}%
\definecolor{currentstroke}{rgb}{1.000000,0.000000,0.000000}%
\pgfsetstrokecolor{currentstroke}%
\pgfsetdash{}{0pt}%
\pgfpathmoveto{\pgfqpoint{3.318870in}{2.639331in}}%
\pgfpathlineto{\pgfqpoint{3.148484in}{2.604561in}}%
\pgfusepath{stroke}%
\end{pgfscope}%
\begin{pgfscope}%
\pgfpathrectangle{\pgfqpoint{0.100000in}{0.212622in}}{\pgfqpoint{3.696000in}{3.696000in}}%
\pgfusepath{clip}%
\pgfsetrectcap%
\pgfsetroundjoin%
\pgfsetlinewidth{1.505625pt}%
\definecolor{currentstroke}{rgb}{1.000000,0.000000,0.000000}%
\pgfsetstrokecolor{currentstroke}%
\pgfsetdash{}{0pt}%
\pgfpathmoveto{\pgfqpoint{3.317141in}{2.639365in}}%
\pgfpathlineto{\pgfqpoint{3.148484in}{2.604561in}}%
\pgfusepath{stroke}%
\end{pgfscope}%
\begin{pgfscope}%
\pgfpathrectangle{\pgfqpoint{0.100000in}{0.212622in}}{\pgfqpoint{3.696000in}{3.696000in}}%
\pgfusepath{clip}%
\pgfsetrectcap%
\pgfsetroundjoin%
\pgfsetlinewidth{1.505625pt}%
\definecolor{currentstroke}{rgb}{1.000000,0.000000,0.000000}%
\pgfsetstrokecolor{currentstroke}%
\pgfsetdash{}{0pt}%
\pgfpathmoveto{\pgfqpoint{3.315100in}{2.639437in}}%
\pgfpathlineto{\pgfqpoint{3.148484in}{2.604561in}}%
\pgfusepath{stroke}%
\end{pgfscope}%
\begin{pgfscope}%
\pgfpathrectangle{\pgfqpoint{0.100000in}{0.212622in}}{\pgfqpoint{3.696000in}{3.696000in}}%
\pgfusepath{clip}%
\pgfsetrectcap%
\pgfsetroundjoin%
\pgfsetlinewidth{1.505625pt}%
\definecolor{currentstroke}{rgb}{1.000000,0.000000,0.000000}%
\pgfsetstrokecolor{currentstroke}%
\pgfsetdash{}{0pt}%
\pgfpathmoveto{\pgfqpoint{3.314007in}{2.639575in}}%
\pgfpathlineto{\pgfqpoint{3.148484in}{2.604561in}}%
\pgfusepath{stroke}%
\end{pgfscope}%
\begin{pgfscope}%
\pgfpathrectangle{\pgfqpoint{0.100000in}{0.212622in}}{\pgfqpoint{3.696000in}{3.696000in}}%
\pgfusepath{clip}%
\pgfsetrectcap%
\pgfsetroundjoin%
\pgfsetlinewidth{1.505625pt}%
\definecolor{currentstroke}{rgb}{1.000000,0.000000,0.000000}%
\pgfsetstrokecolor{currentstroke}%
\pgfsetdash{}{0pt}%
\pgfpathmoveto{\pgfqpoint{3.313411in}{2.639701in}}%
\pgfpathlineto{\pgfqpoint{3.148484in}{2.604561in}}%
\pgfusepath{stroke}%
\end{pgfscope}%
\begin{pgfscope}%
\pgfpathrectangle{\pgfqpoint{0.100000in}{0.212622in}}{\pgfqpoint{3.696000in}{3.696000in}}%
\pgfusepath{clip}%
\pgfsetrectcap%
\pgfsetroundjoin%
\pgfsetlinewidth{1.505625pt}%
\definecolor{currentstroke}{rgb}{1.000000,0.000000,0.000000}%
\pgfsetstrokecolor{currentstroke}%
\pgfsetdash{}{0pt}%
\pgfpathmoveto{\pgfqpoint{3.313086in}{2.639765in}}%
\pgfpathlineto{\pgfqpoint{3.148484in}{2.604561in}}%
\pgfusepath{stroke}%
\end{pgfscope}%
\begin{pgfscope}%
\pgfpathrectangle{\pgfqpoint{0.100000in}{0.212622in}}{\pgfqpoint{3.696000in}{3.696000in}}%
\pgfusepath{clip}%
\pgfsetrectcap%
\pgfsetroundjoin%
\pgfsetlinewidth{1.505625pt}%
\definecolor{currentstroke}{rgb}{1.000000,0.000000,0.000000}%
\pgfsetstrokecolor{currentstroke}%
\pgfsetdash{}{0pt}%
\pgfpathmoveto{\pgfqpoint{3.312900in}{2.639799in}}%
\pgfpathlineto{\pgfqpoint{3.148484in}{2.604561in}}%
\pgfusepath{stroke}%
\end{pgfscope}%
\begin{pgfscope}%
\pgfpathrectangle{\pgfqpoint{0.100000in}{0.212622in}}{\pgfqpoint{3.696000in}{3.696000in}}%
\pgfusepath{clip}%
\pgfsetrectcap%
\pgfsetroundjoin%
\pgfsetlinewidth{1.505625pt}%
\definecolor{currentstroke}{rgb}{1.000000,0.000000,0.000000}%
\pgfsetstrokecolor{currentstroke}%
\pgfsetdash{}{0pt}%
\pgfpathmoveto{\pgfqpoint{3.312805in}{2.639811in}}%
\pgfpathlineto{\pgfqpoint{3.148484in}{2.604561in}}%
\pgfusepath{stroke}%
\end{pgfscope}%
\begin{pgfscope}%
\pgfpathrectangle{\pgfqpoint{0.100000in}{0.212622in}}{\pgfqpoint{3.696000in}{3.696000in}}%
\pgfusepath{clip}%
\pgfsetrectcap%
\pgfsetroundjoin%
\pgfsetlinewidth{1.505625pt}%
\definecolor{currentstroke}{rgb}{1.000000,0.000000,0.000000}%
\pgfsetstrokecolor{currentstroke}%
\pgfsetdash{}{0pt}%
\pgfpathmoveto{\pgfqpoint{3.312755in}{2.639824in}}%
\pgfpathlineto{\pgfqpoint{3.148484in}{2.604561in}}%
\pgfusepath{stroke}%
\end{pgfscope}%
\begin{pgfscope}%
\pgfpathrectangle{\pgfqpoint{0.100000in}{0.212622in}}{\pgfqpoint{3.696000in}{3.696000in}}%
\pgfusepath{clip}%
\pgfsetrectcap%
\pgfsetroundjoin%
\pgfsetlinewidth{1.505625pt}%
\definecolor{currentstroke}{rgb}{1.000000,0.000000,0.000000}%
\pgfsetstrokecolor{currentstroke}%
\pgfsetdash{}{0pt}%
\pgfpathmoveto{\pgfqpoint{3.312726in}{2.639831in}}%
\pgfpathlineto{\pgfqpoint{3.148484in}{2.604561in}}%
\pgfusepath{stroke}%
\end{pgfscope}%
\begin{pgfscope}%
\pgfpathrectangle{\pgfqpoint{0.100000in}{0.212622in}}{\pgfqpoint{3.696000in}{3.696000in}}%
\pgfusepath{clip}%
\pgfsetrectcap%
\pgfsetroundjoin%
\pgfsetlinewidth{1.505625pt}%
\definecolor{currentstroke}{rgb}{1.000000,0.000000,0.000000}%
\pgfsetstrokecolor{currentstroke}%
\pgfsetdash{}{0pt}%
\pgfpathmoveto{\pgfqpoint{3.312711in}{2.639835in}}%
\pgfpathlineto{\pgfqpoint{3.148484in}{2.604561in}}%
\pgfusepath{stroke}%
\end{pgfscope}%
\begin{pgfscope}%
\pgfpathrectangle{\pgfqpoint{0.100000in}{0.212622in}}{\pgfqpoint{3.696000in}{3.696000in}}%
\pgfusepath{clip}%
\pgfsetrectcap%
\pgfsetroundjoin%
\pgfsetlinewidth{1.505625pt}%
\definecolor{currentstroke}{rgb}{1.000000,0.000000,0.000000}%
\pgfsetstrokecolor{currentstroke}%
\pgfsetdash{}{0pt}%
\pgfpathmoveto{\pgfqpoint{3.312702in}{2.639836in}}%
\pgfpathlineto{\pgfqpoint{3.148484in}{2.604561in}}%
\pgfusepath{stroke}%
\end{pgfscope}%
\begin{pgfscope}%
\pgfpathrectangle{\pgfqpoint{0.100000in}{0.212622in}}{\pgfqpoint{3.696000in}{3.696000in}}%
\pgfusepath{clip}%
\pgfsetrectcap%
\pgfsetroundjoin%
\pgfsetlinewidth{1.505625pt}%
\definecolor{currentstroke}{rgb}{1.000000,0.000000,0.000000}%
\pgfsetstrokecolor{currentstroke}%
\pgfsetdash{}{0pt}%
\pgfpathmoveto{\pgfqpoint{3.312697in}{2.639837in}}%
\pgfpathlineto{\pgfqpoint{3.148484in}{2.604561in}}%
\pgfusepath{stroke}%
\end{pgfscope}%
\begin{pgfscope}%
\pgfpathrectangle{\pgfqpoint{0.100000in}{0.212622in}}{\pgfqpoint{3.696000in}{3.696000in}}%
\pgfusepath{clip}%
\pgfsetrectcap%
\pgfsetroundjoin%
\pgfsetlinewidth{1.505625pt}%
\definecolor{currentstroke}{rgb}{1.000000,0.000000,0.000000}%
\pgfsetstrokecolor{currentstroke}%
\pgfsetdash{}{0pt}%
\pgfpathmoveto{\pgfqpoint{3.312695in}{2.639839in}}%
\pgfpathlineto{\pgfqpoint{3.148484in}{2.604561in}}%
\pgfusepath{stroke}%
\end{pgfscope}%
\begin{pgfscope}%
\pgfpathrectangle{\pgfqpoint{0.100000in}{0.212622in}}{\pgfqpoint{3.696000in}{3.696000in}}%
\pgfusepath{clip}%
\pgfsetrectcap%
\pgfsetroundjoin%
\pgfsetlinewidth{1.505625pt}%
\definecolor{currentstroke}{rgb}{1.000000,0.000000,0.000000}%
\pgfsetstrokecolor{currentstroke}%
\pgfsetdash{}{0pt}%
\pgfpathmoveto{\pgfqpoint{3.312694in}{2.639839in}}%
\pgfpathlineto{\pgfqpoint{3.148484in}{2.604561in}}%
\pgfusepath{stroke}%
\end{pgfscope}%
\begin{pgfscope}%
\pgfpathrectangle{\pgfqpoint{0.100000in}{0.212622in}}{\pgfqpoint{3.696000in}{3.696000in}}%
\pgfusepath{clip}%
\pgfsetrectcap%
\pgfsetroundjoin%
\pgfsetlinewidth{1.505625pt}%
\definecolor{currentstroke}{rgb}{1.000000,0.000000,0.000000}%
\pgfsetstrokecolor{currentstroke}%
\pgfsetdash{}{0pt}%
\pgfpathmoveto{\pgfqpoint{3.312693in}{2.639839in}}%
\pgfpathlineto{\pgfqpoint{3.148484in}{2.604561in}}%
\pgfusepath{stroke}%
\end{pgfscope}%
\begin{pgfscope}%
\pgfpathrectangle{\pgfqpoint{0.100000in}{0.212622in}}{\pgfqpoint{3.696000in}{3.696000in}}%
\pgfusepath{clip}%
\pgfsetrectcap%
\pgfsetroundjoin%
\pgfsetlinewidth{1.505625pt}%
\definecolor{currentstroke}{rgb}{1.000000,0.000000,0.000000}%
\pgfsetstrokecolor{currentstroke}%
\pgfsetdash{}{0pt}%
\pgfpathmoveto{\pgfqpoint{3.312692in}{2.639839in}}%
\pgfpathlineto{\pgfqpoint{3.148484in}{2.604561in}}%
\pgfusepath{stroke}%
\end{pgfscope}%
\begin{pgfscope}%
\pgfpathrectangle{\pgfqpoint{0.100000in}{0.212622in}}{\pgfqpoint{3.696000in}{3.696000in}}%
\pgfusepath{clip}%
\pgfsetrectcap%
\pgfsetroundjoin%
\pgfsetlinewidth{1.505625pt}%
\definecolor{currentstroke}{rgb}{1.000000,0.000000,0.000000}%
\pgfsetstrokecolor{currentstroke}%
\pgfsetdash{}{0pt}%
\pgfpathmoveto{\pgfqpoint{3.312692in}{2.639840in}}%
\pgfpathlineto{\pgfqpoint{3.148484in}{2.604561in}}%
\pgfusepath{stroke}%
\end{pgfscope}%
\begin{pgfscope}%
\pgfpathrectangle{\pgfqpoint{0.100000in}{0.212622in}}{\pgfqpoint{3.696000in}{3.696000in}}%
\pgfusepath{clip}%
\pgfsetrectcap%
\pgfsetroundjoin%
\pgfsetlinewidth{1.505625pt}%
\definecolor{currentstroke}{rgb}{1.000000,0.000000,0.000000}%
\pgfsetstrokecolor{currentstroke}%
\pgfsetdash{}{0pt}%
\pgfpathmoveto{\pgfqpoint{3.312692in}{2.639840in}}%
\pgfpathlineto{\pgfqpoint{3.148484in}{2.604561in}}%
\pgfusepath{stroke}%
\end{pgfscope}%
\begin{pgfscope}%
\pgfpathrectangle{\pgfqpoint{0.100000in}{0.212622in}}{\pgfqpoint{3.696000in}{3.696000in}}%
\pgfusepath{clip}%
\pgfsetrectcap%
\pgfsetroundjoin%
\pgfsetlinewidth{1.505625pt}%
\definecolor{currentstroke}{rgb}{1.000000,0.000000,0.000000}%
\pgfsetstrokecolor{currentstroke}%
\pgfsetdash{}{0pt}%
\pgfpathmoveto{\pgfqpoint{3.312692in}{2.639840in}}%
\pgfpathlineto{\pgfqpoint{3.148484in}{2.604561in}}%
\pgfusepath{stroke}%
\end{pgfscope}%
\begin{pgfscope}%
\pgfpathrectangle{\pgfqpoint{0.100000in}{0.212622in}}{\pgfqpoint{3.696000in}{3.696000in}}%
\pgfusepath{clip}%
\pgfsetrectcap%
\pgfsetroundjoin%
\pgfsetlinewidth{1.505625pt}%
\definecolor{currentstroke}{rgb}{1.000000,0.000000,0.000000}%
\pgfsetstrokecolor{currentstroke}%
\pgfsetdash{}{0pt}%
\pgfpathmoveto{\pgfqpoint{3.312692in}{2.639840in}}%
\pgfpathlineto{\pgfqpoint{3.148484in}{2.604561in}}%
\pgfusepath{stroke}%
\end{pgfscope}%
\begin{pgfscope}%
\pgfpathrectangle{\pgfqpoint{0.100000in}{0.212622in}}{\pgfqpoint{3.696000in}{3.696000in}}%
\pgfusepath{clip}%
\pgfsetrectcap%
\pgfsetroundjoin%
\pgfsetlinewidth{1.505625pt}%
\definecolor{currentstroke}{rgb}{1.000000,0.000000,0.000000}%
\pgfsetstrokecolor{currentstroke}%
\pgfsetdash{}{0pt}%
\pgfpathmoveto{\pgfqpoint{3.312692in}{2.639840in}}%
\pgfpathlineto{\pgfqpoint{3.148484in}{2.604561in}}%
\pgfusepath{stroke}%
\end{pgfscope}%
\begin{pgfscope}%
\pgfpathrectangle{\pgfqpoint{0.100000in}{0.212622in}}{\pgfqpoint{3.696000in}{3.696000in}}%
\pgfusepath{clip}%
\pgfsetrectcap%
\pgfsetroundjoin%
\pgfsetlinewidth{1.505625pt}%
\definecolor{currentstroke}{rgb}{1.000000,0.000000,0.000000}%
\pgfsetstrokecolor{currentstroke}%
\pgfsetdash{}{0pt}%
\pgfpathmoveto{\pgfqpoint{3.312692in}{2.639840in}}%
\pgfpathlineto{\pgfqpoint{3.148484in}{2.604561in}}%
\pgfusepath{stroke}%
\end{pgfscope}%
\begin{pgfscope}%
\pgfpathrectangle{\pgfqpoint{0.100000in}{0.212622in}}{\pgfqpoint{3.696000in}{3.696000in}}%
\pgfusepath{clip}%
\pgfsetrectcap%
\pgfsetroundjoin%
\pgfsetlinewidth{1.505625pt}%
\definecolor{currentstroke}{rgb}{1.000000,0.000000,0.000000}%
\pgfsetstrokecolor{currentstroke}%
\pgfsetdash{}{0pt}%
\pgfpathmoveto{\pgfqpoint{3.312692in}{2.639840in}}%
\pgfpathlineto{\pgfqpoint{3.148484in}{2.604561in}}%
\pgfusepath{stroke}%
\end{pgfscope}%
\begin{pgfscope}%
\pgfpathrectangle{\pgfqpoint{0.100000in}{0.212622in}}{\pgfqpoint{3.696000in}{3.696000in}}%
\pgfusepath{clip}%
\pgfsetrectcap%
\pgfsetroundjoin%
\pgfsetlinewidth{1.505625pt}%
\definecolor{currentstroke}{rgb}{1.000000,0.000000,0.000000}%
\pgfsetstrokecolor{currentstroke}%
\pgfsetdash{}{0pt}%
\pgfpathmoveto{\pgfqpoint{3.312692in}{2.639840in}}%
\pgfpathlineto{\pgfqpoint{3.148484in}{2.604561in}}%
\pgfusepath{stroke}%
\end{pgfscope}%
\begin{pgfscope}%
\pgfpathrectangle{\pgfqpoint{0.100000in}{0.212622in}}{\pgfqpoint{3.696000in}{3.696000in}}%
\pgfusepath{clip}%
\pgfsetrectcap%
\pgfsetroundjoin%
\pgfsetlinewidth{1.505625pt}%
\definecolor{currentstroke}{rgb}{1.000000,0.000000,0.000000}%
\pgfsetstrokecolor{currentstroke}%
\pgfsetdash{}{0pt}%
\pgfpathmoveto{\pgfqpoint{3.312692in}{2.639840in}}%
\pgfpathlineto{\pgfqpoint{3.148484in}{2.604561in}}%
\pgfusepath{stroke}%
\end{pgfscope}%
\begin{pgfscope}%
\pgfpathrectangle{\pgfqpoint{0.100000in}{0.212622in}}{\pgfqpoint{3.696000in}{3.696000in}}%
\pgfusepath{clip}%
\pgfsetrectcap%
\pgfsetroundjoin%
\pgfsetlinewidth{1.505625pt}%
\definecolor{currentstroke}{rgb}{1.000000,0.000000,0.000000}%
\pgfsetstrokecolor{currentstroke}%
\pgfsetdash{}{0pt}%
\pgfpathmoveto{\pgfqpoint{3.312692in}{2.639840in}}%
\pgfpathlineto{\pgfqpoint{3.148484in}{2.604561in}}%
\pgfusepath{stroke}%
\end{pgfscope}%
\begin{pgfscope}%
\pgfpathrectangle{\pgfqpoint{0.100000in}{0.212622in}}{\pgfqpoint{3.696000in}{3.696000in}}%
\pgfusepath{clip}%
\pgfsetrectcap%
\pgfsetroundjoin%
\pgfsetlinewidth{1.505625pt}%
\definecolor{currentstroke}{rgb}{1.000000,0.000000,0.000000}%
\pgfsetstrokecolor{currentstroke}%
\pgfsetdash{}{0pt}%
\pgfpathmoveto{\pgfqpoint{3.312692in}{2.639840in}}%
\pgfpathlineto{\pgfqpoint{3.148484in}{2.604561in}}%
\pgfusepath{stroke}%
\end{pgfscope}%
\begin{pgfscope}%
\pgfpathrectangle{\pgfqpoint{0.100000in}{0.212622in}}{\pgfqpoint{3.696000in}{3.696000in}}%
\pgfusepath{clip}%
\pgfsetrectcap%
\pgfsetroundjoin%
\pgfsetlinewidth{1.505625pt}%
\definecolor{currentstroke}{rgb}{1.000000,0.000000,0.000000}%
\pgfsetstrokecolor{currentstroke}%
\pgfsetdash{}{0pt}%
\pgfpathmoveto{\pgfqpoint{3.312692in}{2.639840in}}%
\pgfpathlineto{\pgfqpoint{3.148484in}{2.604561in}}%
\pgfusepath{stroke}%
\end{pgfscope}%
\begin{pgfscope}%
\pgfpathrectangle{\pgfqpoint{0.100000in}{0.212622in}}{\pgfqpoint{3.696000in}{3.696000in}}%
\pgfusepath{clip}%
\pgfsetrectcap%
\pgfsetroundjoin%
\pgfsetlinewidth{1.505625pt}%
\definecolor{currentstroke}{rgb}{1.000000,0.000000,0.000000}%
\pgfsetstrokecolor{currentstroke}%
\pgfsetdash{}{0pt}%
\pgfpathmoveto{\pgfqpoint{3.312692in}{2.639840in}}%
\pgfpathlineto{\pgfqpoint{3.148484in}{2.604561in}}%
\pgfusepath{stroke}%
\end{pgfscope}%
\begin{pgfscope}%
\pgfpathrectangle{\pgfqpoint{0.100000in}{0.212622in}}{\pgfqpoint{3.696000in}{3.696000in}}%
\pgfusepath{clip}%
\pgfsetrectcap%
\pgfsetroundjoin%
\pgfsetlinewidth{1.505625pt}%
\definecolor{currentstroke}{rgb}{1.000000,0.000000,0.000000}%
\pgfsetstrokecolor{currentstroke}%
\pgfsetdash{}{0pt}%
\pgfpathmoveto{\pgfqpoint{3.312692in}{2.639840in}}%
\pgfpathlineto{\pgfqpoint{3.148484in}{2.604561in}}%
\pgfusepath{stroke}%
\end{pgfscope}%
\begin{pgfscope}%
\pgfpathrectangle{\pgfqpoint{0.100000in}{0.212622in}}{\pgfqpoint{3.696000in}{3.696000in}}%
\pgfusepath{clip}%
\pgfsetrectcap%
\pgfsetroundjoin%
\pgfsetlinewidth{1.505625pt}%
\definecolor{currentstroke}{rgb}{1.000000,0.000000,0.000000}%
\pgfsetstrokecolor{currentstroke}%
\pgfsetdash{}{0pt}%
\pgfpathmoveto{\pgfqpoint{3.312692in}{2.639840in}}%
\pgfpathlineto{\pgfqpoint{3.148484in}{2.604561in}}%
\pgfusepath{stroke}%
\end{pgfscope}%
\begin{pgfscope}%
\pgfpathrectangle{\pgfqpoint{0.100000in}{0.212622in}}{\pgfqpoint{3.696000in}{3.696000in}}%
\pgfusepath{clip}%
\pgfsetrectcap%
\pgfsetroundjoin%
\pgfsetlinewidth{1.505625pt}%
\definecolor{currentstroke}{rgb}{1.000000,0.000000,0.000000}%
\pgfsetstrokecolor{currentstroke}%
\pgfsetdash{}{0pt}%
\pgfpathmoveto{\pgfqpoint{3.312692in}{2.639840in}}%
\pgfpathlineto{\pgfqpoint{3.148484in}{2.604561in}}%
\pgfusepath{stroke}%
\end{pgfscope}%
\begin{pgfscope}%
\pgfpathrectangle{\pgfqpoint{0.100000in}{0.212622in}}{\pgfqpoint{3.696000in}{3.696000in}}%
\pgfusepath{clip}%
\pgfsetrectcap%
\pgfsetroundjoin%
\pgfsetlinewidth{1.505625pt}%
\definecolor{currentstroke}{rgb}{1.000000,0.000000,0.000000}%
\pgfsetstrokecolor{currentstroke}%
\pgfsetdash{}{0pt}%
\pgfpathmoveto{\pgfqpoint{3.312692in}{2.639840in}}%
\pgfpathlineto{\pgfqpoint{3.148484in}{2.604561in}}%
\pgfusepath{stroke}%
\end{pgfscope}%
\begin{pgfscope}%
\pgfpathrectangle{\pgfqpoint{0.100000in}{0.212622in}}{\pgfqpoint{3.696000in}{3.696000in}}%
\pgfusepath{clip}%
\pgfsetrectcap%
\pgfsetroundjoin%
\pgfsetlinewidth{1.505625pt}%
\definecolor{currentstroke}{rgb}{1.000000,0.000000,0.000000}%
\pgfsetstrokecolor{currentstroke}%
\pgfsetdash{}{0pt}%
\pgfpathmoveto{\pgfqpoint{3.312692in}{2.639840in}}%
\pgfpathlineto{\pgfqpoint{3.148484in}{2.604561in}}%
\pgfusepath{stroke}%
\end{pgfscope}%
\begin{pgfscope}%
\pgfpathrectangle{\pgfqpoint{0.100000in}{0.212622in}}{\pgfqpoint{3.696000in}{3.696000in}}%
\pgfusepath{clip}%
\pgfsetrectcap%
\pgfsetroundjoin%
\pgfsetlinewidth{1.505625pt}%
\definecolor{currentstroke}{rgb}{1.000000,0.000000,0.000000}%
\pgfsetstrokecolor{currentstroke}%
\pgfsetdash{}{0pt}%
\pgfpathmoveto{\pgfqpoint{3.312692in}{2.639840in}}%
\pgfpathlineto{\pgfqpoint{3.148484in}{2.604561in}}%
\pgfusepath{stroke}%
\end{pgfscope}%
\begin{pgfscope}%
\pgfpathrectangle{\pgfqpoint{0.100000in}{0.212622in}}{\pgfqpoint{3.696000in}{3.696000in}}%
\pgfusepath{clip}%
\pgfsetrectcap%
\pgfsetroundjoin%
\pgfsetlinewidth{1.505625pt}%
\definecolor{currentstroke}{rgb}{1.000000,0.000000,0.000000}%
\pgfsetstrokecolor{currentstroke}%
\pgfsetdash{}{0pt}%
\pgfpathmoveto{\pgfqpoint{3.312692in}{2.639840in}}%
\pgfpathlineto{\pgfqpoint{3.148484in}{2.604561in}}%
\pgfusepath{stroke}%
\end{pgfscope}%
\begin{pgfscope}%
\pgfpathrectangle{\pgfqpoint{0.100000in}{0.212622in}}{\pgfqpoint{3.696000in}{3.696000in}}%
\pgfusepath{clip}%
\pgfsetrectcap%
\pgfsetroundjoin%
\pgfsetlinewidth{1.505625pt}%
\definecolor{currentstroke}{rgb}{1.000000,0.000000,0.000000}%
\pgfsetstrokecolor{currentstroke}%
\pgfsetdash{}{0pt}%
\pgfpathmoveto{\pgfqpoint{3.312692in}{2.639840in}}%
\pgfpathlineto{\pgfqpoint{3.148484in}{2.604561in}}%
\pgfusepath{stroke}%
\end{pgfscope}%
\begin{pgfscope}%
\pgfpathrectangle{\pgfqpoint{0.100000in}{0.212622in}}{\pgfqpoint{3.696000in}{3.696000in}}%
\pgfusepath{clip}%
\pgfsetrectcap%
\pgfsetroundjoin%
\pgfsetlinewidth{1.505625pt}%
\definecolor{currentstroke}{rgb}{1.000000,0.000000,0.000000}%
\pgfsetstrokecolor{currentstroke}%
\pgfsetdash{}{0pt}%
\pgfpathmoveto{\pgfqpoint{3.312692in}{2.639840in}}%
\pgfpathlineto{\pgfqpoint{3.148484in}{2.604561in}}%
\pgfusepath{stroke}%
\end{pgfscope}%
\begin{pgfscope}%
\pgfpathrectangle{\pgfqpoint{0.100000in}{0.212622in}}{\pgfqpoint{3.696000in}{3.696000in}}%
\pgfusepath{clip}%
\pgfsetrectcap%
\pgfsetroundjoin%
\pgfsetlinewidth{1.505625pt}%
\definecolor{currentstroke}{rgb}{1.000000,0.000000,0.000000}%
\pgfsetstrokecolor{currentstroke}%
\pgfsetdash{}{0pt}%
\pgfpathmoveto{\pgfqpoint{3.312692in}{2.639840in}}%
\pgfpathlineto{\pgfqpoint{3.148484in}{2.604561in}}%
\pgfusepath{stroke}%
\end{pgfscope}%
\begin{pgfscope}%
\pgfpathrectangle{\pgfqpoint{0.100000in}{0.212622in}}{\pgfqpoint{3.696000in}{3.696000in}}%
\pgfusepath{clip}%
\pgfsetrectcap%
\pgfsetroundjoin%
\pgfsetlinewidth{1.505625pt}%
\definecolor{currentstroke}{rgb}{1.000000,0.000000,0.000000}%
\pgfsetstrokecolor{currentstroke}%
\pgfsetdash{}{0pt}%
\pgfpathmoveto{\pgfqpoint{3.312692in}{2.639840in}}%
\pgfpathlineto{\pgfqpoint{3.148484in}{2.604561in}}%
\pgfusepath{stroke}%
\end{pgfscope}%
\begin{pgfscope}%
\pgfpathrectangle{\pgfqpoint{0.100000in}{0.212622in}}{\pgfqpoint{3.696000in}{3.696000in}}%
\pgfusepath{clip}%
\pgfsetrectcap%
\pgfsetroundjoin%
\pgfsetlinewidth{1.505625pt}%
\definecolor{currentstroke}{rgb}{1.000000,0.000000,0.000000}%
\pgfsetstrokecolor{currentstroke}%
\pgfsetdash{}{0pt}%
\pgfpathmoveto{\pgfqpoint{3.312692in}{2.639840in}}%
\pgfpathlineto{\pgfqpoint{3.148484in}{2.604561in}}%
\pgfusepath{stroke}%
\end{pgfscope}%
\begin{pgfscope}%
\pgfpathrectangle{\pgfqpoint{0.100000in}{0.212622in}}{\pgfqpoint{3.696000in}{3.696000in}}%
\pgfusepath{clip}%
\pgfsetrectcap%
\pgfsetroundjoin%
\pgfsetlinewidth{1.505625pt}%
\definecolor{currentstroke}{rgb}{1.000000,0.000000,0.000000}%
\pgfsetstrokecolor{currentstroke}%
\pgfsetdash{}{0pt}%
\pgfpathmoveto{\pgfqpoint{3.312692in}{2.639840in}}%
\pgfpathlineto{\pgfqpoint{3.148484in}{2.604561in}}%
\pgfusepath{stroke}%
\end{pgfscope}%
\begin{pgfscope}%
\pgfpathrectangle{\pgfqpoint{0.100000in}{0.212622in}}{\pgfqpoint{3.696000in}{3.696000in}}%
\pgfusepath{clip}%
\pgfsetrectcap%
\pgfsetroundjoin%
\pgfsetlinewidth{1.505625pt}%
\definecolor{currentstroke}{rgb}{1.000000,0.000000,0.000000}%
\pgfsetstrokecolor{currentstroke}%
\pgfsetdash{}{0pt}%
\pgfpathmoveto{\pgfqpoint{3.312692in}{2.639840in}}%
\pgfpathlineto{\pgfqpoint{3.148484in}{2.604561in}}%
\pgfusepath{stroke}%
\end{pgfscope}%
\begin{pgfscope}%
\pgfpathrectangle{\pgfqpoint{0.100000in}{0.212622in}}{\pgfqpoint{3.696000in}{3.696000in}}%
\pgfusepath{clip}%
\pgfsetrectcap%
\pgfsetroundjoin%
\pgfsetlinewidth{1.505625pt}%
\definecolor{currentstroke}{rgb}{1.000000,0.000000,0.000000}%
\pgfsetstrokecolor{currentstroke}%
\pgfsetdash{}{0pt}%
\pgfpathmoveto{\pgfqpoint{3.312692in}{2.639840in}}%
\pgfpathlineto{\pgfqpoint{3.148484in}{2.604561in}}%
\pgfusepath{stroke}%
\end{pgfscope}%
\begin{pgfscope}%
\pgfpathrectangle{\pgfqpoint{0.100000in}{0.212622in}}{\pgfqpoint{3.696000in}{3.696000in}}%
\pgfusepath{clip}%
\pgfsetrectcap%
\pgfsetroundjoin%
\pgfsetlinewidth{1.505625pt}%
\definecolor{currentstroke}{rgb}{1.000000,0.000000,0.000000}%
\pgfsetstrokecolor{currentstroke}%
\pgfsetdash{}{0pt}%
\pgfpathmoveto{\pgfqpoint{3.312064in}{2.639853in}}%
\pgfpathlineto{\pgfqpoint{3.148484in}{2.604561in}}%
\pgfusepath{stroke}%
\end{pgfscope}%
\begin{pgfscope}%
\pgfpathrectangle{\pgfqpoint{0.100000in}{0.212622in}}{\pgfqpoint{3.696000in}{3.696000in}}%
\pgfusepath{clip}%
\pgfsetrectcap%
\pgfsetroundjoin%
\pgfsetlinewidth{1.505625pt}%
\definecolor{currentstroke}{rgb}{1.000000,0.000000,0.000000}%
\pgfsetstrokecolor{currentstroke}%
\pgfsetdash{}{0pt}%
\pgfpathmoveto{\pgfqpoint{3.310528in}{2.640087in}}%
\pgfpathlineto{\pgfqpoint{3.148484in}{2.604561in}}%
\pgfusepath{stroke}%
\end{pgfscope}%
\begin{pgfscope}%
\pgfpathrectangle{\pgfqpoint{0.100000in}{0.212622in}}{\pgfqpoint{3.696000in}{3.696000in}}%
\pgfusepath{clip}%
\pgfsetrectcap%
\pgfsetroundjoin%
\pgfsetlinewidth{1.505625pt}%
\definecolor{currentstroke}{rgb}{1.000000,0.000000,0.000000}%
\pgfsetstrokecolor{currentstroke}%
\pgfsetdash{}{0pt}%
\pgfpathmoveto{\pgfqpoint{3.308794in}{2.640902in}}%
\pgfpathlineto{\pgfqpoint{3.148484in}{2.604561in}}%
\pgfusepath{stroke}%
\end{pgfscope}%
\begin{pgfscope}%
\pgfpathrectangle{\pgfqpoint{0.100000in}{0.212622in}}{\pgfqpoint{3.696000in}{3.696000in}}%
\pgfusepath{clip}%
\pgfsetrectcap%
\pgfsetroundjoin%
\pgfsetlinewidth{1.505625pt}%
\definecolor{currentstroke}{rgb}{1.000000,0.000000,0.000000}%
\pgfsetstrokecolor{currentstroke}%
\pgfsetdash{}{0pt}%
\pgfpathmoveto{\pgfqpoint{3.305735in}{2.641205in}}%
\pgfpathlineto{\pgfqpoint{3.148484in}{2.604561in}}%
\pgfusepath{stroke}%
\end{pgfscope}%
\begin{pgfscope}%
\pgfpathrectangle{\pgfqpoint{0.100000in}{0.212622in}}{\pgfqpoint{3.696000in}{3.696000in}}%
\pgfusepath{clip}%
\pgfsetrectcap%
\pgfsetroundjoin%
\pgfsetlinewidth{1.505625pt}%
\definecolor{currentstroke}{rgb}{1.000000,0.000000,0.000000}%
\pgfsetstrokecolor{currentstroke}%
\pgfsetdash{}{0pt}%
\pgfpathmoveto{\pgfqpoint{3.301453in}{2.640984in}}%
\pgfpathlineto{\pgfqpoint{3.148484in}{2.604561in}}%
\pgfusepath{stroke}%
\end{pgfscope}%
\begin{pgfscope}%
\pgfpathrectangle{\pgfqpoint{0.100000in}{0.212622in}}{\pgfqpoint{3.696000in}{3.696000in}}%
\pgfusepath{clip}%
\pgfsetrectcap%
\pgfsetroundjoin%
\pgfsetlinewidth{1.505625pt}%
\definecolor{currentstroke}{rgb}{1.000000,0.000000,0.000000}%
\pgfsetstrokecolor{currentstroke}%
\pgfsetdash{}{0pt}%
\pgfpathmoveto{\pgfqpoint{3.296535in}{2.641453in}}%
\pgfpathlineto{\pgfqpoint{3.149833in}{2.605785in}}%
\pgfusepath{stroke}%
\end{pgfscope}%
\begin{pgfscope}%
\pgfpathrectangle{\pgfqpoint{0.100000in}{0.212622in}}{\pgfqpoint{3.696000in}{3.696000in}}%
\pgfusepath{clip}%
\pgfsetrectcap%
\pgfsetroundjoin%
\pgfsetlinewidth{1.505625pt}%
\definecolor{currentstroke}{rgb}{1.000000,0.000000,0.000000}%
\pgfsetstrokecolor{currentstroke}%
\pgfsetdash{}{0pt}%
\pgfpathmoveto{\pgfqpoint{3.290717in}{2.643567in}}%
\pgfpathlineto{\pgfqpoint{3.149833in}{2.605785in}}%
\pgfusepath{stroke}%
\end{pgfscope}%
\begin{pgfscope}%
\pgfpathrectangle{\pgfqpoint{0.100000in}{0.212622in}}{\pgfqpoint{3.696000in}{3.696000in}}%
\pgfusepath{clip}%
\pgfsetrectcap%
\pgfsetroundjoin%
\pgfsetlinewidth{1.505625pt}%
\definecolor{currentstroke}{rgb}{1.000000,0.000000,0.000000}%
\pgfsetstrokecolor{currentstroke}%
\pgfsetdash{}{0pt}%
\pgfpathmoveto{\pgfqpoint{3.287398in}{2.644427in}}%
\pgfpathlineto{\pgfqpoint{3.149833in}{2.605785in}}%
\pgfusepath{stroke}%
\end{pgfscope}%
\begin{pgfscope}%
\pgfpathrectangle{\pgfqpoint{0.100000in}{0.212622in}}{\pgfqpoint{3.696000in}{3.696000in}}%
\pgfusepath{clip}%
\pgfsetrectcap%
\pgfsetroundjoin%
\pgfsetlinewidth{1.505625pt}%
\definecolor{currentstroke}{rgb}{1.000000,0.000000,0.000000}%
\pgfsetstrokecolor{currentstroke}%
\pgfsetdash{}{0pt}%
\pgfpathmoveto{\pgfqpoint{3.283014in}{2.645321in}}%
\pgfpathlineto{\pgfqpoint{3.149833in}{2.605785in}}%
\pgfusepath{stroke}%
\end{pgfscope}%
\begin{pgfscope}%
\pgfpathrectangle{\pgfqpoint{0.100000in}{0.212622in}}{\pgfqpoint{3.696000in}{3.696000in}}%
\pgfusepath{clip}%
\pgfsetrectcap%
\pgfsetroundjoin%
\pgfsetlinewidth{1.505625pt}%
\definecolor{currentstroke}{rgb}{1.000000,0.000000,0.000000}%
\pgfsetstrokecolor{currentstroke}%
\pgfsetdash{}{0pt}%
\pgfpathmoveto{\pgfqpoint{3.277456in}{2.645439in}}%
\pgfpathlineto{\pgfqpoint{3.149833in}{2.605785in}}%
\pgfusepath{stroke}%
\end{pgfscope}%
\begin{pgfscope}%
\pgfpathrectangle{\pgfqpoint{0.100000in}{0.212622in}}{\pgfqpoint{3.696000in}{3.696000in}}%
\pgfusepath{clip}%
\pgfsetrectcap%
\pgfsetroundjoin%
\pgfsetlinewidth{1.505625pt}%
\definecolor{currentstroke}{rgb}{1.000000,0.000000,0.000000}%
\pgfsetstrokecolor{currentstroke}%
\pgfsetdash{}{0pt}%
\pgfpathmoveto{\pgfqpoint{3.274183in}{2.645083in}}%
\pgfpathlineto{\pgfqpoint{3.149833in}{2.605785in}}%
\pgfusepath{stroke}%
\end{pgfscope}%
\begin{pgfscope}%
\pgfpathrectangle{\pgfqpoint{0.100000in}{0.212622in}}{\pgfqpoint{3.696000in}{3.696000in}}%
\pgfusepath{clip}%
\pgfsetrectcap%
\pgfsetroundjoin%
\pgfsetlinewidth{1.505625pt}%
\definecolor{currentstroke}{rgb}{1.000000,0.000000,0.000000}%
\pgfsetstrokecolor{currentstroke}%
\pgfsetdash{}{0pt}%
\pgfpathmoveto{\pgfqpoint{3.268845in}{2.645411in}}%
\pgfpathlineto{\pgfqpoint{3.149833in}{2.605785in}}%
\pgfusepath{stroke}%
\end{pgfscope}%
\begin{pgfscope}%
\pgfpathrectangle{\pgfqpoint{0.100000in}{0.212622in}}{\pgfqpoint{3.696000in}{3.696000in}}%
\pgfusepath{clip}%
\pgfsetrectcap%
\pgfsetroundjoin%
\pgfsetlinewidth{1.505625pt}%
\definecolor{currentstroke}{rgb}{1.000000,0.000000,0.000000}%
\pgfsetstrokecolor{currentstroke}%
\pgfsetdash{}{0pt}%
\pgfpathmoveto{\pgfqpoint{3.262923in}{2.646971in}}%
\pgfpathlineto{\pgfqpoint{3.151182in}{2.607009in}}%
\pgfusepath{stroke}%
\end{pgfscope}%
\begin{pgfscope}%
\pgfpathrectangle{\pgfqpoint{0.100000in}{0.212622in}}{\pgfqpoint{3.696000in}{3.696000in}}%
\pgfusepath{clip}%
\pgfsetrectcap%
\pgfsetroundjoin%
\pgfsetlinewidth{1.505625pt}%
\definecolor{currentstroke}{rgb}{1.000000,0.000000,0.000000}%
\pgfsetstrokecolor{currentstroke}%
\pgfsetdash{}{0pt}%
\pgfpathmoveto{\pgfqpoint{3.256620in}{2.648875in}}%
\pgfpathlineto{\pgfqpoint{3.151182in}{2.607009in}}%
\pgfusepath{stroke}%
\end{pgfscope}%
\begin{pgfscope}%
\pgfpathrectangle{\pgfqpoint{0.100000in}{0.212622in}}{\pgfqpoint{3.696000in}{3.696000in}}%
\pgfusepath{clip}%
\pgfsetrectcap%
\pgfsetroundjoin%
\pgfsetlinewidth{1.505625pt}%
\definecolor{currentstroke}{rgb}{1.000000,0.000000,0.000000}%
\pgfsetstrokecolor{currentstroke}%
\pgfsetdash{}{0pt}%
\pgfpathmoveto{\pgfqpoint{3.248508in}{2.649630in}}%
\pgfpathlineto{\pgfqpoint{3.151182in}{2.607009in}}%
\pgfusepath{stroke}%
\end{pgfscope}%
\begin{pgfscope}%
\pgfpathrectangle{\pgfqpoint{0.100000in}{0.212622in}}{\pgfqpoint{3.696000in}{3.696000in}}%
\pgfusepath{clip}%
\pgfsetrectcap%
\pgfsetroundjoin%
\pgfsetlinewidth{1.505625pt}%
\definecolor{currentstroke}{rgb}{1.000000,0.000000,0.000000}%
\pgfsetstrokecolor{currentstroke}%
\pgfsetdash{}{0pt}%
\pgfpathmoveto{\pgfqpoint{3.238847in}{2.647986in}}%
\pgfpathlineto{\pgfqpoint{3.151182in}{2.607009in}}%
\pgfusepath{stroke}%
\end{pgfscope}%
\begin{pgfscope}%
\pgfpathrectangle{\pgfqpoint{0.100000in}{0.212622in}}{\pgfqpoint{3.696000in}{3.696000in}}%
\pgfusepath{clip}%
\pgfsetrectcap%
\pgfsetroundjoin%
\pgfsetlinewidth{1.505625pt}%
\definecolor{currentstroke}{rgb}{1.000000,0.000000,0.000000}%
\pgfsetstrokecolor{currentstroke}%
\pgfsetdash{}{0pt}%
\pgfpathmoveto{\pgfqpoint{3.229262in}{2.646822in}}%
\pgfpathlineto{\pgfqpoint{3.151182in}{2.607009in}}%
\pgfusepath{stroke}%
\end{pgfscope}%
\begin{pgfscope}%
\pgfpathrectangle{\pgfqpoint{0.100000in}{0.212622in}}{\pgfqpoint{3.696000in}{3.696000in}}%
\pgfusepath{clip}%
\pgfsetrectcap%
\pgfsetroundjoin%
\pgfsetlinewidth{1.505625pt}%
\definecolor{currentstroke}{rgb}{1.000000,0.000000,0.000000}%
\pgfsetstrokecolor{currentstroke}%
\pgfsetdash{}{0pt}%
\pgfpathmoveto{\pgfqpoint{3.217622in}{2.647549in}}%
\pgfpathlineto{\pgfqpoint{3.151182in}{2.607009in}}%
\pgfusepath{stroke}%
\end{pgfscope}%
\begin{pgfscope}%
\pgfpathrectangle{\pgfqpoint{0.100000in}{0.212622in}}{\pgfqpoint{3.696000in}{3.696000in}}%
\pgfusepath{clip}%
\pgfsetrectcap%
\pgfsetroundjoin%
\pgfsetlinewidth{1.505625pt}%
\definecolor{currentstroke}{rgb}{1.000000,0.000000,0.000000}%
\pgfsetstrokecolor{currentstroke}%
\pgfsetdash{}{0pt}%
\pgfpathmoveto{\pgfqpoint{3.204740in}{2.649888in}}%
\pgfpathlineto{\pgfqpoint{3.151182in}{2.607009in}}%
\pgfusepath{stroke}%
\end{pgfscope}%
\begin{pgfscope}%
\pgfpathrectangle{\pgfqpoint{0.100000in}{0.212622in}}{\pgfqpoint{3.696000in}{3.696000in}}%
\pgfusepath{clip}%
\pgfsetrectcap%
\pgfsetroundjoin%
\pgfsetlinewidth{1.505625pt}%
\definecolor{currentstroke}{rgb}{1.000000,0.000000,0.000000}%
\pgfsetstrokecolor{currentstroke}%
\pgfsetdash{}{0pt}%
\pgfpathmoveto{\pgfqpoint{3.192445in}{2.652207in}}%
\pgfpathlineto{\pgfqpoint{3.152531in}{2.608233in}}%
\pgfusepath{stroke}%
\end{pgfscope}%
\begin{pgfscope}%
\pgfpathrectangle{\pgfqpoint{0.100000in}{0.212622in}}{\pgfqpoint{3.696000in}{3.696000in}}%
\pgfusepath{clip}%
\pgfsetrectcap%
\pgfsetroundjoin%
\pgfsetlinewidth{1.505625pt}%
\definecolor{currentstroke}{rgb}{1.000000,0.000000,0.000000}%
\pgfsetstrokecolor{currentstroke}%
\pgfsetdash{}{0pt}%
\pgfpathmoveto{\pgfqpoint{3.178341in}{2.653767in}}%
\pgfpathlineto{\pgfqpoint{3.152531in}{2.608233in}}%
\pgfusepath{stroke}%
\end{pgfscope}%
\begin{pgfscope}%
\pgfpathrectangle{\pgfqpoint{0.100000in}{0.212622in}}{\pgfqpoint{3.696000in}{3.696000in}}%
\pgfusepath{clip}%
\pgfsetrectcap%
\pgfsetroundjoin%
\pgfsetlinewidth{1.505625pt}%
\definecolor{currentstroke}{rgb}{1.000000,0.000000,0.000000}%
\pgfsetstrokecolor{currentstroke}%
\pgfsetdash{}{0pt}%
\pgfpathmoveto{\pgfqpoint{3.162631in}{2.654013in}}%
\pgfpathlineto{\pgfqpoint{3.152531in}{2.608233in}}%
\pgfusepath{stroke}%
\end{pgfscope}%
\begin{pgfscope}%
\pgfpathrectangle{\pgfqpoint{0.100000in}{0.212622in}}{\pgfqpoint{3.696000in}{3.696000in}}%
\pgfusepath{clip}%
\pgfsetrectcap%
\pgfsetroundjoin%
\pgfsetlinewidth{1.505625pt}%
\definecolor{currentstroke}{rgb}{1.000000,0.000000,0.000000}%
\pgfsetstrokecolor{currentstroke}%
\pgfsetdash{}{0pt}%
\pgfpathmoveto{\pgfqpoint{3.146198in}{2.651489in}}%
\pgfpathlineto{\pgfqpoint{3.152531in}{2.608233in}}%
\pgfusepath{stroke}%
\end{pgfscope}%
\begin{pgfscope}%
\pgfpathrectangle{\pgfqpoint{0.100000in}{0.212622in}}{\pgfqpoint{3.696000in}{3.696000in}}%
\pgfusepath{clip}%
\pgfsetrectcap%
\pgfsetroundjoin%
\pgfsetlinewidth{1.505625pt}%
\definecolor{currentstroke}{rgb}{1.000000,0.000000,0.000000}%
\pgfsetstrokecolor{currentstroke}%
\pgfsetdash{}{0pt}%
\pgfpathmoveto{\pgfqpoint{3.138019in}{2.651658in}}%
\pgfpathlineto{\pgfqpoint{3.145462in}{2.619633in}}%
\pgfusepath{stroke}%
\end{pgfscope}%
\begin{pgfscope}%
\pgfpathrectangle{\pgfqpoint{0.100000in}{0.212622in}}{\pgfqpoint{3.696000in}{3.696000in}}%
\pgfusepath{clip}%
\pgfsetrectcap%
\pgfsetroundjoin%
\pgfsetlinewidth{1.505625pt}%
\definecolor{currentstroke}{rgb}{1.000000,0.000000,0.000000}%
\pgfsetstrokecolor{currentstroke}%
\pgfsetdash{}{0pt}%
\pgfpathmoveto{\pgfqpoint{3.127405in}{2.651206in}}%
\pgfpathlineto{\pgfqpoint{3.134105in}{2.622679in}}%
\pgfusepath{stroke}%
\end{pgfscope}%
\begin{pgfscope}%
\pgfpathrectangle{\pgfqpoint{0.100000in}{0.212622in}}{\pgfqpoint{3.696000in}{3.696000in}}%
\pgfusepath{clip}%
\pgfsetrectcap%
\pgfsetroundjoin%
\pgfsetlinewidth{1.505625pt}%
\definecolor{currentstroke}{rgb}{1.000000,0.000000,0.000000}%
\pgfsetstrokecolor{currentstroke}%
\pgfsetdash{}{0pt}%
\pgfpathmoveto{\pgfqpoint{3.113526in}{2.651058in}}%
\pgfpathlineto{\pgfqpoint{3.120862in}{2.626231in}}%
\pgfusepath{stroke}%
\end{pgfscope}%
\begin{pgfscope}%
\pgfpathrectangle{\pgfqpoint{0.100000in}{0.212622in}}{\pgfqpoint{3.696000in}{3.696000in}}%
\pgfusepath{clip}%
\pgfsetrectcap%
\pgfsetroundjoin%
\pgfsetlinewidth{1.505625pt}%
\definecolor{currentstroke}{rgb}{1.000000,0.000000,0.000000}%
\pgfsetstrokecolor{currentstroke}%
\pgfsetdash{}{0pt}%
\pgfpathmoveto{\pgfqpoint{3.097321in}{2.651893in}}%
\pgfpathlineto{\pgfqpoint{3.107628in}{2.629781in}}%
\pgfusepath{stroke}%
\end{pgfscope}%
\begin{pgfscope}%
\pgfpathrectangle{\pgfqpoint{0.100000in}{0.212622in}}{\pgfqpoint{3.696000in}{3.696000in}}%
\pgfusepath{clip}%
\pgfsetrectcap%
\pgfsetroundjoin%
\pgfsetlinewidth{1.505625pt}%
\definecolor{currentstroke}{rgb}{1.000000,0.000000,0.000000}%
\pgfsetstrokecolor{currentstroke}%
\pgfsetdash{}{0pt}%
\pgfpathmoveto{\pgfqpoint{3.089728in}{2.653250in}}%
\pgfpathlineto{\pgfqpoint{3.100069in}{2.631809in}}%
\pgfusepath{stroke}%
\end{pgfscope}%
\begin{pgfscope}%
\pgfpathrectangle{\pgfqpoint{0.100000in}{0.212622in}}{\pgfqpoint{3.696000in}{3.696000in}}%
\pgfusepath{clip}%
\pgfsetrectcap%
\pgfsetroundjoin%
\pgfsetlinewidth{1.505625pt}%
\definecolor{currentstroke}{rgb}{1.000000,0.000000,0.000000}%
\pgfsetstrokecolor{currentstroke}%
\pgfsetdash{}{0pt}%
\pgfpathmoveto{\pgfqpoint{3.080780in}{2.654350in}}%
\pgfpathlineto{\pgfqpoint{3.092513in}{2.633835in}}%
\pgfusepath{stroke}%
\end{pgfscope}%
\begin{pgfscope}%
\pgfpathrectangle{\pgfqpoint{0.100000in}{0.212622in}}{\pgfqpoint{3.696000in}{3.696000in}}%
\pgfusepath{clip}%
\pgfsetrectcap%
\pgfsetroundjoin%
\pgfsetlinewidth{1.505625pt}%
\definecolor{currentstroke}{rgb}{1.000000,0.000000,0.000000}%
\pgfsetstrokecolor{currentstroke}%
\pgfsetdash{}{0pt}%
\pgfpathmoveto{\pgfqpoint{3.073195in}{2.656827in}}%
\pgfpathlineto{\pgfqpoint{3.083073in}{2.636368in}}%
\pgfusepath{stroke}%
\end{pgfscope}%
\begin{pgfscope}%
\pgfpathrectangle{\pgfqpoint{0.100000in}{0.212622in}}{\pgfqpoint{3.696000in}{3.696000in}}%
\pgfusepath{clip}%
\pgfsetrectcap%
\pgfsetroundjoin%
\pgfsetlinewidth{1.505625pt}%
\definecolor{currentstroke}{rgb}{1.000000,0.000000,0.000000}%
\pgfsetstrokecolor{currentstroke}%
\pgfsetdash{}{0pt}%
\pgfpathmoveto{\pgfqpoint{3.063409in}{2.656869in}}%
\pgfpathlineto{\pgfqpoint{3.073636in}{2.638899in}}%
\pgfusepath{stroke}%
\end{pgfscope}%
\begin{pgfscope}%
\pgfpathrectangle{\pgfqpoint{0.100000in}{0.212622in}}{\pgfqpoint{3.696000in}{3.696000in}}%
\pgfusepath{clip}%
\pgfsetrectcap%
\pgfsetroundjoin%
\pgfsetlinewidth{1.505625pt}%
\definecolor{currentstroke}{rgb}{1.000000,0.000000,0.000000}%
\pgfsetstrokecolor{currentstroke}%
\pgfsetdash{}{0pt}%
\pgfpathmoveto{\pgfqpoint{3.052182in}{2.657388in}}%
\pgfpathlineto{\pgfqpoint{3.062318in}{2.641935in}}%
\pgfusepath{stroke}%
\end{pgfscope}%
\begin{pgfscope}%
\pgfpathrectangle{\pgfqpoint{0.100000in}{0.212622in}}{\pgfqpoint{3.696000in}{3.696000in}}%
\pgfusepath{clip}%
\pgfsetrectcap%
\pgfsetroundjoin%
\pgfsetlinewidth{1.505625pt}%
\definecolor{currentstroke}{rgb}{1.000000,0.000000,0.000000}%
\pgfsetstrokecolor{currentstroke}%
\pgfsetdash{}{0pt}%
\pgfpathmoveto{\pgfqpoint{3.041273in}{2.656943in}}%
\pgfpathlineto{\pgfqpoint{3.051007in}{2.644969in}}%
\pgfusepath{stroke}%
\end{pgfscope}%
\begin{pgfscope}%
\pgfpathrectangle{\pgfqpoint{0.100000in}{0.212622in}}{\pgfqpoint{3.696000in}{3.696000in}}%
\pgfusepath{clip}%
\pgfsetrectcap%
\pgfsetroundjoin%
\pgfsetlinewidth{1.505625pt}%
\definecolor{currentstroke}{rgb}{1.000000,0.000000,0.000000}%
\pgfsetstrokecolor{currentstroke}%
\pgfsetdash{}{0pt}%
\pgfpathmoveto{\pgfqpoint{3.029601in}{2.657654in}}%
\pgfpathlineto{\pgfqpoint{3.039701in}{2.648001in}}%
\pgfusepath{stroke}%
\end{pgfscope}%
\begin{pgfscope}%
\pgfpathrectangle{\pgfqpoint{0.100000in}{0.212622in}}{\pgfqpoint{3.696000in}{3.696000in}}%
\pgfusepath{clip}%
\pgfsetrectcap%
\pgfsetroundjoin%
\pgfsetlinewidth{1.505625pt}%
\definecolor{currentstroke}{rgb}{1.000000,0.000000,0.000000}%
\pgfsetstrokecolor{currentstroke}%
\pgfsetdash{}{0pt}%
\pgfpathmoveto{\pgfqpoint{3.016862in}{2.657982in}}%
\pgfpathlineto{\pgfqpoint{3.026519in}{2.651537in}}%
\pgfusepath{stroke}%
\end{pgfscope}%
\begin{pgfscope}%
\pgfpathrectangle{\pgfqpoint{0.100000in}{0.212622in}}{\pgfqpoint{3.696000in}{3.696000in}}%
\pgfusepath{clip}%
\pgfsetrectcap%
\pgfsetroundjoin%
\pgfsetlinewidth{1.505625pt}%
\definecolor{currentstroke}{rgb}{1.000000,0.000000,0.000000}%
\pgfsetstrokecolor{currentstroke}%
\pgfsetdash{}{0pt}%
\pgfpathmoveto{\pgfqpoint{3.002184in}{2.659620in}}%
\pgfpathlineto{\pgfqpoint{3.011464in}{2.655575in}}%
\pgfusepath{stroke}%
\end{pgfscope}%
\begin{pgfscope}%
\pgfpathrectangle{\pgfqpoint{0.100000in}{0.212622in}}{\pgfqpoint{3.696000in}{3.696000in}}%
\pgfusepath{clip}%
\pgfsetrectcap%
\pgfsetroundjoin%
\pgfsetlinewidth{1.505625pt}%
\definecolor{currentstroke}{rgb}{1.000000,0.000000,0.000000}%
\pgfsetstrokecolor{currentstroke}%
\pgfsetdash{}{0pt}%
\pgfpathmoveto{\pgfqpoint{2.986253in}{2.660724in}}%
\pgfpathlineto{\pgfqpoint{2.994541in}{2.660115in}}%
\pgfusepath{stroke}%
\end{pgfscope}%
\begin{pgfscope}%
\pgfpathrectangle{\pgfqpoint{0.100000in}{0.212622in}}{\pgfqpoint{3.696000in}{3.696000in}}%
\pgfusepath{clip}%
\pgfsetrectcap%
\pgfsetroundjoin%
\pgfsetlinewidth{1.505625pt}%
\definecolor{currentstroke}{rgb}{1.000000,0.000000,0.000000}%
\pgfsetstrokecolor{currentstroke}%
\pgfsetdash{}{0pt}%
\pgfpathmoveto{\pgfqpoint{2.978539in}{2.664562in}}%
\pgfpathlineto{\pgfqpoint{2.985145in}{2.662635in}}%
\pgfusepath{stroke}%
\end{pgfscope}%
\begin{pgfscope}%
\pgfpathrectangle{\pgfqpoint{0.100000in}{0.212622in}}{\pgfqpoint{3.696000in}{3.696000in}}%
\pgfusepath{clip}%
\pgfsetrectcap%
\pgfsetroundjoin%
\pgfsetlinewidth{1.505625pt}%
\definecolor{currentstroke}{rgb}{1.000000,0.000000,0.000000}%
\pgfsetstrokecolor{currentstroke}%
\pgfsetdash{}{0pt}%
\pgfpathmoveto{\pgfqpoint{2.967168in}{2.665773in}}%
\pgfpathlineto{\pgfqpoint{2.975754in}{2.665154in}}%
\pgfusepath{stroke}%
\end{pgfscope}%
\begin{pgfscope}%
\pgfpathrectangle{\pgfqpoint{0.100000in}{0.212622in}}{\pgfqpoint{3.696000in}{3.696000in}}%
\pgfusepath{clip}%
\pgfsetrectcap%
\pgfsetroundjoin%
\pgfsetlinewidth{1.505625pt}%
\definecolor{currentstroke}{rgb}{1.000000,0.000000,0.000000}%
\pgfsetstrokecolor{currentstroke}%
\pgfsetdash{}{0pt}%
\pgfpathmoveto{\pgfqpoint{2.956769in}{2.667584in}}%
\pgfpathlineto{\pgfqpoint{2.962613in}{2.668679in}}%
\pgfusepath{stroke}%
\end{pgfscope}%
\begin{pgfscope}%
\pgfpathrectangle{\pgfqpoint{0.100000in}{0.212622in}}{\pgfqpoint{3.696000in}{3.696000in}}%
\pgfusepath{clip}%
\pgfsetrectcap%
\pgfsetroundjoin%
\pgfsetlinewidth{1.505625pt}%
\definecolor{currentstroke}{rgb}{1.000000,0.000000,0.000000}%
\pgfsetstrokecolor{currentstroke}%
\pgfsetdash{}{0pt}%
\pgfpathmoveto{\pgfqpoint{2.946073in}{2.670196in}}%
\pgfpathlineto{\pgfqpoint{2.949481in}{2.672201in}}%
\pgfusepath{stroke}%
\end{pgfscope}%
\begin{pgfscope}%
\pgfpathrectangle{\pgfqpoint{0.100000in}{0.212622in}}{\pgfqpoint{3.696000in}{3.696000in}}%
\pgfusepath{clip}%
\pgfsetrectcap%
\pgfsetroundjoin%
\pgfsetlinewidth{1.505625pt}%
\definecolor{currentstroke}{rgb}{1.000000,0.000000,0.000000}%
\pgfsetstrokecolor{currentstroke}%
\pgfsetdash{}{0pt}%
\pgfpathmoveto{\pgfqpoint{2.934006in}{2.673333in}}%
\pgfpathlineto{\pgfqpoint{2.936357in}{2.675721in}}%
\pgfusepath{stroke}%
\end{pgfscope}%
\begin{pgfscope}%
\pgfpathrectangle{\pgfqpoint{0.100000in}{0.212622in}}{\pgfqpoint{3.696000in}{3.696000in}}%
\pgfusepath{clip}%
\pgfsetrectcap%
\pgfsetroundjoin%
\pgfsetlinewidth{1.505625pt}%
\definecolor{currentstroke}{rgb}{1.000000,0.000000,0.000000}%
\pgfsetstrokecolor{currentstroke}%
\pgfsetdash{}{0pt}%
\pgfpathmoveto{\pgfqpoint{2.920977in}{2.672061in}}%
\pgfpathlineto{\pgfqpoint{2.923241in}{2.679239in}}%
\pgfusepath{stroke}%
\end{pgfscope}%
\begin{pgfscope}%
\pgfpathrectangle{\pgfqpoint{0.100000in}{0.212622in}}{\pgfqpoint{3.696000in}{3.696000in}}%
\pgfusepath{clip}%
\pgfsetrectcap%
\pgfsetroundjoin%
\pgfsetlinewidth{1.505625pt}%
\definecolor{currentstroke}{rgb}{1.000000,0.000000,0.000000}%
\pgfsetstrokecolor{currentstroke}%
\pgfsetdash{}{0pt}%
\pgfpathmoveto{\pgfqpoint{2.907502in}{2.666491in}}%
\pgfpathlineto{\pgfqpoint{2.910134in}{2.682755in}}%
\pgfusepath{stroke}%
\end{pgfscope}%
\begin{pgfscope}%
\pgfpathrectangle{\pgfqpoint{0.100000in}{0.212622in}}{\pgfqpoint{3.696000in}{3.696000in}}%
\pgfusepath{clip}%
\pgfsetrectcap%
\pgfsetroundjoin%
\pgfsetlinewidth{1.505625pt}%
\definecolor{currentstroke}{rgb}{1.000000,0.000000,0.000000}%
\pgfsetstrokecolor{currentstroke}%
\pgfsetdash{}{0pt}%
\pgfpathmoveto{\pgfqpoint{2.892676in}{2.661330in}}%
\pgfpathlineto{\pgfqpoint{2.897036in}{2.686269in}}%
\pgfusepath{stroke}%
\end{pgfscope}%
\begin{pgfscope}%
\pgfpathrectangle{\pgfqpoint{0.100000in}{0.212622in}}{\pgfqpoint{3.696000in}{3.696000in}}%
\pgfusepath{clip}%
\pgfsetrectcap%
\pgfsetroundjoin%
\pgfsetlinewidth{1.505625pt}%
\definecolor{currentstroke}{rgb}{1.000000,0.000000,0.000000}%
\pgfsetstrokecolor{currentstroke}%
\pgfsetdash{}{0pt}%
\pgfpathmoveto{\pgfqpoint{2.874542in}{2.660879in}}%
\pgfpathlineto{\pgfqpoint{2.883945in}{2.689780in}}%
\pgfusepath{stroke}%
\end{pgfscope}%
\begin{pgfscope}%
\pgfpathrectangle{\pgfqpoint{0.100000in}{0.212622in}}{\pgfqpoint{3.696000in}{3.696000in}}%
\pgfusepath{clip}%
\pgfsetrectcap%
\pgfsetroundjoin%
\pgfsetlinewidth{1.505625pt}%
\definecolor{currentstroke}{rgb}{1.000000,0.000000,0.000000}%
\pgfsetstrokecolor{currentstroke}%
\pgfsetdash{}{0pt}%
\pgfpathmoveto{\pgfqpoint{2.854781in}{2.669889in}}%
\pgfpathlineto{\pgfqpoint{2.857790in}{2.696796in}}%
\pgfusepath{stroke}%
\end{pgfscope}%
\begin{pgfscope}%
\pgfpathrectangle{\pgfqpoint{0.100000in}{0.212622in}}{\pgfqpoint{3.696000in}{3.696000in}}%
\pgfusepath{clip}%
\pgfsetrectcap%
\pgfsetroundjoin%
\pgfsetlinewidth{1.505625pt}%
\definecolor{currentstroke}{rgb}{1.000000,0.000000,0.000000}%
\pgfsetstrokecolor{currentstroke}%
\pgfsetdash{}{0pt}%
\pgfpathmoveto{\pgfqpoint{2.844247in}{2.672441in}}%
\pgfpathlineto{\pgfqpoint{2.857790in}{2.696796in}}%
\pgfusepath{stroke}%
\end{pgfscope}%
\begin{pgfscope}%
\pgfpathrectangle{\pgfqpoint{0.100000in}{0.212622in}}{\pgfqpoint{3.696000in}{3.696000in}}%
\pgfusepath{clip}%
\pgfsetrectcap%
\pgfsetroundjoin%
\pgfsetlinewidth{1.505625pt}%
\definecolor{currentstroke}{rgb}{1.000000,0.000000,0.000000}%
\pgfsetstrokecolor{currentstroke}%
\pgfsetdash{}{0pt}%
\pgfpathmoveto{\pgfqpoint{2.832764in}{2.675440in}}%
\pgfpathlineto{\pgfqpoint{2.844724in}{2.700300in}}%
\pgfusepath{stroke}%
\end{pgfscope}%
\begin{pgfscope}%
\pgfpathrectangle{\pgfqpoint{0.100000in}{0.212622in}}{\pgfqpoint{3.696000in}{3.696000in}}%
\pgfusepath{clip}%
\pgfsetrectcap%
\pgfsetroundjoin%
\pgfsetlinewidth{1.505625pt}%
\definecolor{currentstroke}{rgb}{1.000000,0.000000,0.000000}%
\pgfsetstrokecolor{currentstroke}%
\pgfsetdash{}{0pt}%
\pgfpathmoveto{\pgfqpoint{2.826425in}{2.676842in}}%
\pgfpathlineto{\pgfqpoint{2.831668in}{2.703802in}}%
\pgfusepath{stroke}%
\end{pgfscope}%
\begin{pgfscope}%
\pgfpathrectangle{\pgfqpoint{0.100000in}{0.212622in}}{\pgfqpoint{3.696000in}{3.696000in}}%
\pgfusepath{clip}%
\pgfsetrectcap%
\pgfsetroundjoin%
\pgfsetlinewidth{1.505625pt}%
\definecolor{currentstroke}{rgb}{1.000000,0.000000,0.000000}%
\pgfsetstrokecolor{currentstroke}%
\pgfsetdash{}{0pt}%
\pgfpathmoveto{\pgfqpoint{2.818674in}{2.678299in}}%
\pgfpathlineto{\pgfqpoint{2.831668in}{2.703802in}}%
\pgfusepath{stroke}%
\end{pgfscope}%
\begin{pgfscope}%
\pgfpathrectangle{\pgfqpoint{0.100000in}{0.212622in}}{\pgfqpoint{3.696000in}{3.696000in}}%
\pgfusepath{clip}%
\pgfsetrectcap%
\pgfsetroundjoin%
\pgfsetlinewidth{1.505625pt}%
\definecolor{currentstroke}{rgb}{1.000000,0.000000,0.000000}%
\pgfsetstrokecolor{currentstroke}%
\pgfsetdash{}{0pt}%
\pgfpathmoveto{\pgfqpoint{2.809359in}{2.678941in}}%
\pgfpathlineto{\pgfqpoint{2.818619in}{2.707302in}}%
\pgfusepath{stroke}%
\end{pgfscope}%
\begin{pgfscope}%
\pgfpathrectangle{\pgfqpoint{0.100000in}{0.212622in}}{\pgfqpoint{3.696000in}{3.696000in}}%
\pgfusepath{clip}%
\pgfsetrectcap%
\pgfsetroundjoin%
\pgfsetlinewidth{1.505625pt}%
\definecolor{currentstroke}{rgb}{1.000000,0.000000,0.000000}%
\pgfsetstrokecolor{currentstroke}%
\pgfsetdash{}{0pt}%
\pgfpathmoveto{\pgfqpoint{2.800166in}{2.677033in}}%
\pgfpathlineto{\pgfqpoint{2.805579in}{2.710800in}}%
\pgfusepath{stroke}%
\end{pgfscope}%
\begin{pgfscope}%
\pgfpathrectangle{\pgfqpoint{0.100000in}{0.212622in}}{\pgfqpoint{3.696000in}{3.696000in}}%
\pgfusepath{clip}%
\pgfsetrectcap%
\pgfsetroundjoin%
\pgfsetlinewidth{1.505625pt}%
\definecolor{currentstroke}{rgb}{1.000000,0.000000,0.000000}%
\pgfsetstrokecolor{currentstroke}%
\pgfsetdash{}{0pt}%
\pgfpathmoveto{\pgfqpoint{2.790248in}{2.673508in}}%
\pgfpathlineto{\pgfqpoint{2.805579in}{2.710800in}}%
\pgfusepath{stroke}%
\end{pgfscope}%
\begin{pgfscope}%
\pgfpathrectangle{\pgfqpoint{0.100000in}{0.212622in}}{\pgfqpoint{3.696000in}{3.696000in}}%
\pgfusepath{clip}%
\pgfsetrectcap%
\pgfsetroundjoin%
\pgfsetlinewidth{1.505625pt}%
\definecolor{currentstroke}{rgb}{1.000000,0.000000,0.000000}%
\pgfsetstrokecolor{currentstroke}%
\pgfsetdash{}{0pt}%
\pgfpathmoveto{\pgfqpoint{2.779520in}{2.673082in}}%
\pgfpathlineto{\pgfqpoint{2.792547in}{2.714296in}}%
\pgfusepath{stroke}%
\end{pgfscope}%
\begin{pgfscope}%
\pgfpathrectangle{\pgfqpoint{0.100000in}{0.212622in}}{\pgfqpoint{3.696000in}{3.696000in}}%
\pgfusepath{clip}%
\pgfsetrectcap%
\pgfsetroundjoin%
\pgfsetlinewidth{1.505625pt}%
\definecolor{currentstroke}{rgb}{1.000000,0.000000,0.000000}%
\pgfsetstrokecolor{currentstroke}%
\pgfsetdash{}{0pt}%
\pgfpathmoveto{\pgfqpoint{2.762838in}{2.671563in}}%
\pgfpathlineto{\pgfqpoint{2.766509in}{2.721280in}}%
\pgfusepath{stroke}%
\end{pgfscope}%
\begin{pgfscope}%
\pgfpathrectangle{\pgfqpoint{0.100000in}{0.212622in}}{\pgfqpoint{3.696000in}{3.696000in}}%
\pgfusepath{clip}%
\pgfsetrectcap%
\pgfsetroundjoin%
\pgfsetlinewidth{1.505625pt}%
\definecolor{currentstroke}{rgb}{1.000000,0.000000,0.000000}%
\pgfsetstrokecolor{currentstroke}%
\pgfsetdash{}{0pt}%
\pgfpathmoveto{\pgfqpoint{2.747283in}{2.678172in}}%
\pgfpathlineto{\pgfqpoint{2.753502in}{2.724769in}}%
\pgfusepath{stroke}%
\end{pgfscope}%
\begin{pgfscope}%
\pgfpathrectangle{\pgfqpoint{0.100000in}{0.212622in}}{\pgfqpoint{3.696000in}{3.696000in}}%
\pgfusepath{clip}%
\pgfsetrectcap%
\pgfsetroundjoin%
\pgfsetlinewidth{1.505625pt}%
\definecolor{currentstroke}{rgb}{1.000000,0.000000,0.000000}%
\pgfsetstrokecolor{currentstroke}%
\pgfsetdash{}{0pt}%
\pgfpathmoveto{\pgfqpoint{2.731264in}{2.683005in}}%
\pgfpathlineto{\pgfqpoint{2.740503in}{2.728256in}}%
\pgfusepath{stroke}%
\end{pgfscope}%
\begin{pgfscope}%
\pgfpathrectangle{\pgfqpoint{0.100000in}{0.212622in}}{\pgfqpoint{3.696000in}{3.696000in}}%
\pgfusepath{clip}%
\pgfsetrectcap%
\pgfsetroundjoin%
\pgfsetlinewidth{1.505625pt}%
\definecolor{currentstroke}{rgb}{1.000000,0.000000,0.000000}%
\pgfsetstrokecolor{currentstroke}%
\pgfsetdash{}{0pt}%
\pgfpathmoveto{\pgfqpoint{2.715723in}{2.686064in}}%
\pgfpathlineto{\pgfqpoint{2.727513in}{2.731740in}}%
\pgfusepath{stroke}%
\end{pgfscope}%
\begin{pgfscope}%
\pgfpathrectangle{\pgfqpoint{0.100000in}{0.212622in}}{\pgfqpoint{3.696000in}{3.696000in}}%
\pgfusepath{clip}%
\pgfsetrectcap%
\pgfsetroundjoin%
\pgfsetlinewidth{1.505625pt}%
\definecolor{currentstroke}{rgb}{1.000000,0.000000,0.000000}%
\pgfsetstrokecolor{currentstroke}%
\pgfsetdash{}{0pt}%
\pgfpathmoveto{\pgfqpoint{2.697893in}{2.687754in}}%
\pgfpathlineto{\pgfqpoint{2.701557in}{2.738702in}}%
\pgfusepath{stroke}%
\end{pgfscope}%
\begin{pgfscope}%
\pgfpathrectangle{\pgfqpoint{0.100000in}{0.212622in}}{\pgfqpoint{3.696000in}{3.696000in}}%
\pgfusepath{clip}%
\pgfsetrectcap%
\pgfsetroundjoin%
\pgfsetlinewidth{1.505625pt}%
\definecolor{currentstroke}{rgb}{1.000000,0.000000,0.000000}%
\pgfsetstrokecolor{currentstroke}%
\pgfsetdash{}{0pt}%
\pgfpathmoveto{\pgfqpoint{2.677663in}{2.683910in}}%
\pgfpathlineto{\pgfqpoint{2.688592in}{2.742180in}}%
\pgfusepath{stroke}%
\end{pgfscope}%
\begin{pgfscope}%
\pgfpathrectangle{\pgfqpoint{0.100000in}{0.212622in}}{\pgfqpoint{3.696000in}{3.696000in}}%
\pgfusepath{clip}%
\pgfsetrectcap%
\pgfsetroundjoin%
\pgfsetlinewidth{1.505625pt}%
\definecolor{currentstroke}{rgb}{1.000000,0.000000,0.000000}%
\pgfsetstrokecolor{currentstroke}%
\pgfsetdash{}{0pt}%
\pgfpathmoveto{\pgfqpoint{2.658445in}{2.676309in}}%
\pgfpathlineto{\pgfqpoint{2.662686in}{2.749129in}}%
\pgfusepath{stroke}%
\end{pgfscope}%
\begin{pgfscope}%
\pgfpathrectangle{\pgfqpoint{0.100000in}{0.212622in}}{\pgfqpoint{3.696000in}{3.696000in}}%
\pgfusepath{clip}%
\pgfsetrectcap%
\pgfsetroundjoin%
\pgfsetlinewidth{1.505625pt}%
\definecolor{currentstroke}{rgb}{1.000000,0.000000,0.000000}%
\pgfsetstrokecolor{currentstroke}%
\pgfsetdash{}{0pt}%
\pgfpathmoveto{\pgfqpoint{2.639834in}{2.669797in}}%
\pgfpathlineto{\pgfqpoint{2.649746in}{2.752600in}}%
\pgfusepath{stroke}%
\end{pgfscope}%
\begin{pgfscope}%
\pgfpathrectangle{\pgfqpoint{0.100000in}{0.212622in}}{\pgfqpoint{3.696000in}{3.696000in}}%
\pgfusepath{clip}%
\pgfsetrectcap%
\pgfsetroundjoin%
\pgfsetlinewidth{1.505625pt}%
\definecolor{currentstroke}{rgb}{1.000000,0.000000,0.000000}%
\pgfsetstrokecolor{currentstroke}%
\pgfsetdash{}{0pt}%
\pgfpathmoveto{\pgfqpoint{2.618326in}{2.665511in}}%
\pgfpathlineto{\pgfqpoint{2.623889in}{2.759535in}}%
\pgfusepath{stroke}%
\end{pgfscope}%
\begin{pgfscope}%
\pgfpathrectangle{\pgfqpoint{0.100000in}{0.212622in}}{\pgfqpoint{3.696000in}{3.696000in}}%
\pgfusepath{clip}%
\pgfsetrectcap%
\pgfsetroundjoin%
\pgfsetlinewidth{1.505625pt}%
\definecolor{currentstroke}{rgb}{1.000000,0.000000,0.000000}%
\pgfsetstrokecolor{currentstroke}%
\pgfsetdash{}{0pt}%
\pgfpathmoveto{\pgfqpoint{2.591938in}{2.671776in}}%
\pgfpathlineto{\pgfqpoint{2.598065in}{2.766462in}}%
\pgfusepath{stroke}%
\end{pgfscope}%
\begin{pgfscope}%
\pgfpathrectangle{\pgfqpoint{0.100000in}{0.212622in}}{\pgfqpoint{3.696000in}{3.696000in}}%
\pgfusepath{clip}%
\pgfsetrectcap%
\pgfsetroundjoin%
\pgfsetlinewidth{1.505625pt}%
\definecolor{currentstroke}{rgb}{1.000000,0.000000,0.000000}%
\pgfsetstrokecolor{currentstroke}%
\pgfsetdash{}{0pt}%
\pgfpathmoveto{\pgfqpoint{2.564965in}{2.676684in}}%
\pgfpathlineto{\pgfqpoint{2.572275in}{2.773380in}}%
\pgfusepath{stroke}%
\end{pgfscope}%
\begin{pgfscope}%
\pgfpathrectangle{\pgfqpoint{0.100000in}{0.212622in}}{\pgfqpoint{3.696000in}{3.696000in}}%
\pgfusepath{clip}%
\pgfsetrectcap%
\pgfsetroundjoin%
\pgfsetlinewidth{1.505625pt}%
\definecolor{currentstroke}{rgb}{1.000000,0.000000,0.000000}%
\pgfsetstrokecolor{currentstroke}%
\pgfsetdash{}{0pt}%
\pgfpathmoveto{\pgfqpoint{2.535487in}{2.680802in}}%
\pgfpathlineto{\pgfqpoint{2.546517in}{2.780289in}}%
\pgfusepath{stroke}%
\end{pgfscope}%
\begin{pgfscope}%
\pgfpathrectangle{\pgfqpoint{0.100000in}{0.212622in}}{\pgfqpoint{3.696000in}{3.696000in}}%
\pgfusepath{clip}%
\pgfsetrectcap%
\pgfsetroundjoin%
\pgfsetlinewidth{1.505625pt}%
\definecolor{currentstroke}{rgb}{1.000000,0.000000,0.000000}%
\pgfsetstrokecolor{currentstroke}%
\pgfsetdash{}{0pt}%
\pgfpathmoveto{\pgfqpoint{2.509711in}{2.687470in}}%
\pgfpathlineto{\pgfqpoint{2.520791in}{2.787189in}}%
\pgfusepath{stroke}%
\end{pgfscope}%
\begin{pgfscope}%
\pgfpathrectangle{\pgfqpoint{0.100000in}{0.212622in}}{\pgfqpoint{3.696000in}{3.696000in}}%
\pgfusepath{clip}%
\pgfsetrectcap%
\pgfsetroundjoin%
\pgfsetlinewidth{1.505625pt}%
\definecolor{currentstroke}{rgb}{1.000000,0.000000,0.000000}%
\pgfsetstrokecolor{currentstroke}%
\pgfsetdash{}{0pt}%
\pgfpathmoveto{\pgfqpoint{2.479023in}{2.686456in}}%
\pgfpathlineto{\pgfqpoint{2.495099in}{2.794081in}}%
\pgfusepath{stroke}%
\end{pgfscope}%
\begin{pgfscope}%
\pgfpathrectangle{\pgfqpoint{0.100000in}{0.212622in}}{\pgfqpoint{3.696000in}{3.696000in}}%
\pgfusepath{clip}%
\pgfsetrectcap%
\pgfsetroundjoin%
\pgfsetlinewidth{1.505625pt}%
\definecolor{currentstroke}{rgb}{1.000000,0.000000,0.000000}%
\pgfsetstrokecolor{currentstroke}%
\pgfsetdash{}{0pt}%
\pgfpathmoveto{\pgfqpoint{2.451657in}{2.684237in}}%
\pgfpathlineto{\pgfqpoint{2.469439in}{2.800964in}}%
\pgfusepath{stroke}%
\end{pgfscope}%
\begin{pgfscope}%
\pgfpathrectangle{\pgfqpoint{0.100000in}{0.212622in}}{\pgfqpoint{3.696000in}{3.696000in}}%
\pgfusepath{clip}%
\pgfsetrectcap%
\pgfsetroundjoin%
\pgfsetlinewidth{1.505625pt}%
\definecolor{currentstroke}{rgb}{1.000000,0.000000,0.000000}%
\pgfsetstrokecolor{currentstroke}%
\pgfsetdash{}{0pt}%
\pgfpathmoveto{\pgfqpoint{2.423573in}{2.674220in}}%
\pgfpathlineto{\pgfqpoint{2.431010in}{2.811271in}}%
\pgfusepath{stroke}%
\end{pgfscope}%
\begin{pgfscope}%
\pgfpathrectangle{\pgfqpoint{0.100000in}{0.212622in}}{\pgfqpoint{3.696000in}{3.696000in}}%
\pgfusepath{clip}%
\pgfsetrectcap%
\pgfsetroundjoin%
\pgfsetlinewidth{1.505625pt}%
\definecolor{currentstroke}{rgb}{1.000000,0.000000,0.000000}%
\pgfsetstrokecolor{currentstroke}%
\pgfsetdash{}{0pt}%
\pgfpathmoveto{\pgfqpoint{2.391385in}{2.666816in}}%
\pgfpathlineto{\pgfqpoint{2.405431in}{2.818132in}}%
\pgfusepath{stroke}%
\end{pgfscope}%
\begin{pgfscope}%
\pgfpathrectangle{\pgfqpoint{0.100000in}{0.212622in}}{\pgfqpoint{3.696000in}{3.696000in}}%
\pgfusepath{clip}%
\pgfsetrectcap%
\pgfsetroundjoin%
\pgfsetlinewidth{1.505625pt}%
\definecolor{currentstroke}{rgb}{1.000000,0.000000,0.000000}%
\pgfsetstrokecolor{currentstroke}%
\pgfsetdash{}{0pt}%
\pgfpathmoveto{\pgfqpoint{2.352305in}{2.666974in}}%
\pgfpathlineto{\pgfqpoint{2.367124in}{2.828408in}}%
\pgfusepath{stroke}%
\end{pgfscope}%
\begin{pgfscope}%
\pgfpathrectangle{\pgfqpoint{0.100000in}{0.212622in}}{\pgfqpoint{3.696000in}{3.696000in}}%
\pgfusepath{clip}%
\pgfsetrectcap%
\pgfsetroundjoin%
\pgfsetlinewidth{1.505625pt}%
\definecolor{currentstroke}{rgb}{1.000000,0.000000,0.000000}%
\pgfsetstrokecolor{currentstroke}%
\pgfsetdash{}{0pt}%
\pgfpathmoveto{\pgfqpoint{2.312580in}{2.672634in}}%
\pgfpathlineto{\pgfqpoint{2.341626in}{2.835247in}}%
\pgfusepath{stroke}%
\end{pgfscope}%
\begin{pgfscope}%
\pgfpathrectangle{\pgfqpoint{0.100000in}{0.212622in}}{\pgfqpoint{3.696000in}{3.696000in}}%
\pgfusepath{clip}%
\pgfsetrectcap%
\pgfsetroundjoin%
\pgfsetlinewidth{1.505625pt}%
\definecolor{currentstroke}{rgb}{1.000000,0.000000,0.000000}%
\pgfsetstrokecolor{currentstroke}%
\pgfsetdash{}{0pt}%
\pgfpathmoveto{\pgfqpoint{2.280532in}{2.681693in}}%
\pgfpathlineto{\pgfqpoint{2.303439in}{2.845490in}}%
\pgfusepath{stroke}%
\end{pgfscope}%
\begin{pgfscope}%
\pgfpathrectangle{\pgfqpoint{0.100000in}{0.212622in}}{\pgfqpoint{3.696000in}{3.696000in}}%
\pgfusepath{clip}%
\pgfsetrectcap%
\pgfsetroundjoin%
\pgfsetlinewidth{1.505625pt}%
\definecolor{currentstroke}{rgb}{1.000000,0.000000,0.000000}%
\pgfsetstrokecolor{currentstroke}%
\pgfsetdash{}{0pt}%
\pgfpathmoveto{\pgfqpoint{2.245058in}{2.683241in}}%
\pgfpathlineto{\pgfqpoint{2.265325in}{2.855713in}}%
\pgfusepath{stroke}%
\end{pgfscope}%
\begin{pgfscope}%
\pgfpathrectangle{\pgfqpoint{0.100000in}{0.212622in}}{\pgfqpoint{3.696000in}{3.696000in}}%
\pgfusepath{clip}%
\pgfsetrectcap%
\pgfsetroundjoin%
\pgfsetlinewidth{1.505625pt}%
\definecolor{currentstroke}{rgb}{1.000000,0.000000,0.000000}%
\pgfsetstrokecolor{currentstroke}%
\pgfsetdash{}{0pt}%
\pgfpathmoveto{\pgfqpoint{2.207361in}{2.679775in}}%
\pgfpathlineto{\pgfqpoint{2.227283in}{2.865917in}}%
\pgfusepath{stroke}%
\end{pgfscope}%
\begin{pgfscope}%
\pgfpathrectangle{\pgfqpoint{0.100000in}{0.212622in}}{\pgfqpoint{3.696000in}{3.696000in}}%
\pgfusepath{clip}%
\pgfsetrectcap%
\pgfsetroundjoin%
\pgfsetlinewidth{1.505625pt}%
\definecolor{currentstroke}{rgb}{1.000000,0.000000,0.000000}%
\pgfsetstrokecolor{currentstroke}%
\pgfsetdash{}{0pt}%
\pgfpathmoveto{\pgfqpoint{2.168103in}{2.670744in}}%
\pgfpathlineto{\pgfqpoint{2.189313in}{2.876102in}}%
\pgfusepath{stroke}%
\end{pgfscope}%
\begin{pgfscope}%
\pgfpathrectangle{\pgfqpoint{0.100000in}{0.212622in}}{\pgfqpoint{3.696000in}{3.696000in}}%
\pgfusepath{clip}%
\pgfsetrectcap%
\pgfsetroundjoin%
\pgfsetlinewidth{1.505625pt}%
\definecolor{currentstroke}{rgb}{1.000000,0.000000,0.000000}%
\pgfsetstrokecolor{currentstroke}%
\pgfsetdash{}{0pt}%
\pgfpathmoveto{\pgfqpoint{2.146988in}{2.664140in}}%
\pgfpathlineto{\pgfqpoint{2.176672in}{2.879493in}}%
\pgfusepath{stroke}%
\end{pgfscope}%
\begin{pgfscope}%
\pgfpathrectangle{\pgfqpoint{0.100000in}{0.212622in}}{\pgfqpoint{3.696000in}{3.696000in}}%
\pgfusepath{clip}%
\pgfsetrectcap%
\pgfsetroundjoin%
\pgfsetlinewidth{1.505625pt}%
\definecolor{currentstroke}{rgb}{1.000000,0.000000,0.000000}%
\pgfsetstrokecolor{currentstroke}%
\pgfsetdash{}{0pt}%
\pgfpathmoveto{\pgfqpoint{2.124524in}{2.667804in}}%
\pgfpathlineto{\pgfqpoint{2.151414in}{2.886268in}}%
\pgfusepath{stroke}%
\end{pgfscope}%
\begin{pgfscope}%
\pgfpathrectangle{\pgfqpoint{0.100000in}{0.212622in}}{\pgfqpoint{3.696000in}{3.696000in}}%
\pgfusepath{clip}%
\pgfsetrectcap%
\pgfsetroundjoin%
\pgfsetlinewidth{1.505625pt}%
\definecolor{currentstroke}{rgb}{1.000000,0.000000,0.000000}%
\pgfsetstrokecolor{currentstroke}%
\pgfsetdash{}{0pt}%
\pgfpathmoveto{\pgfqpoint{2.099227in}{2.673263in}}%
\pgfpathlineto{\pgfqpoint{2.126188in}{2.893034in}}%
\pgfusepath{stroke}%
\end{pgfscope}%
\begin{pgfscope}%
\pgfpathrectangle{\pgfqpoint{0.100000in}{0.212622in}}{\pgfqpoint{3.696000in}{3.696000in}}%
\pgfusepath{clip}%
\pgfsetrectcap%
\pgfsetroundjoin%
\pgfsetlinewidth{1.505625pt}%
\definecolor{currentstroke}{rgb}{1.000000,0.000000,0.000000}%
\pgfsetstrokecolor{currentstroke}%
\pgfsetdash{}{0pt}%
\pgfpathmoveto{\pgfqpoint{2.072557in}{2.686738in}}%
\pgfpathlineto{\pgfqpoint{2.100994in}{2.899792in}}%
\pgfusepath{stroke}%
\end{pgfscope}%
\begin{pgfscope}%
\pgfpathrectangle{\pgfqpoint{0.100000in}{0.212622in}}{\pgfqpoint{3.696000in}{3.696000in}}%
\pgfusepath{clip}%
\pgfsetrectcap%
\pgfsetroundjoin%
\pgfsetlinewidth{1.505625pt}%
\definecolor{currentstroke}{rgb}{1.000000,0.000000,0.000000}%
\pgfsetstrokecolor{currentstroke}%
\pgfsetdash{}{0pt}%
\pgfpathmoveto{\pgfqpoint{2.043996in}{2.689895in}}%
\pgfpathlineto{\pgfqpoint{2.075831in}{2.906541in}}%
\pgfusepath{stroke}%
\end{pgfscope}%
\begin{pgfscope}%
\pgfpathrectangle{\pgfqpoint{0.100000in}{0.212622in}}{\pgfqpoint{3.696000in}{3.696000in}}%
\pgfusepath{clip}%
\pgfsetrectcap%
\pgfsetroundjoin%
\pgfsetlinewidth{1.505625pt}%
\definecolor{currentstroke}{rgb}{1.000000,0.000000,0.000000}%
\pgfsetstrokecolor{currentstroke}%
\pgfsetdash{}{0pt}%
\pgfpathmoveto{\pgfqpoint{2.017925in}{2.695065in}}%
\pgfpathlineto{\pgfqpoint{2.038146in}{2.916650in}}%
\pgfusepath{stroke}%
\end{pgfscope}%
\begin{pgfscope}%
\pgfpathrectangle{\pgfqpoint{0.100000in}{0.212622in}}{\pgfqpoint{3.696000in}{3.696000in}}%
\pgfusepath{clip}%
\pgfsetrectcap%
\pgfsetroundjoin%
\pgfsetlinewidth{1.505625pt}%
\definecolor{currentstroke}{rgb}{1.000000,0.000000,0.000000}%
\pgfsetstrokecolor{currentstroke}%
\pgfsetdash{}{0pt}%
\pgfpathmoveto{\pgfqpoint{1.991343in}{2.697071in}}%
\pgfpathlineto{\pgfqpoint{2.013063in}{2.923378in}}%
\pgfusepath{stroke}%
\end{pgfscope}%
\begin{pgfscope}%
\pgfpathrectangle{\pgfqpoint{0.100000in}{0.212622in}}{\pgfqpoint{3.696000in}{3.696000in}}%
\pgfusepath{clip}%
\pgfsetrectcap%
\pgfsetroundjoin%
\pgfsetlinewidth{1.505625pt}%
\definecolor{currentstroke}{rgb}{1.000000,0.000000,0.000000}%
\pgfsetstrokecolor{currentstroke}%
\pgfsetdash{}{0pt}%
\pgfpathmoveto{\pgfqpoint{1.963196in}{2.691012in}}%
\pgfpathlineto{\pgfqpoint{1.988010in}{2.930098in}}%
\pgfusepath{stroke}%
\end{pgfscope}%
\begin{pgfscope}%
\pgfpathrectangle{\pgfqpoint{0.100000in}{0.212622in}}{\pgfqpoint{3.696000in}{3.696000in}}%
\pgfusepath{clip}%
\pgfsetrectcap%
\pgfsetroundjoin%
\pgfsetlinewidth{1.505625pt}%
\definecolor{currentstroke}{rgb}{1.000000,0.000000,0.000000}%
\pgfsetstrokecolor{currentstroke}%
\pgfsetdash{}{0pt}%
\pgfpathmoveto{\pgfqpoint{1.933837in}{2.680783in}}%
\pgfpathlineto{\pgfqpoint{1.950491in}{2.940162in}}%
\pgfusepath{stroke}%
\end{pgfscope}%
\begin{pgfscope}%
\pgfpathrectangle{\pgfqpoint{0.100000in}{0.212622in}}{\pgfqpoint{3.696000in}{3.696000in}}%
\pgfusepath{clip}%
\pgfsetrectcap%
\pgfsetroundjoin%
\pgfsetlinewidth{1.505625pt}%
\definecolor{currentstroke}{rgb}{1.000000,0.000000,0.000000}%
\pgfsetstrokecolor{currentstroke}%
\pgfsetdash{}{0pt}%
\pgfpathmoveto{\pgfqpoint{1.900244in}{2.672830in}}%
\pgfpathlineto{\pgfqpoint{1.925517in}{2.946861in}}%
\pgfusepath{stroke}%
\end{pgfscope}%
\begin{pgfscope}%
\pgfpathrectangle{\pgfqpoint{0.100000in}{0.212622in}}{\pgfqpoint{3.696000in}{3.696000in}}%
\pgfusepath{clip}%
\pgfsetrectcap%
\pgfsetroundjoin%
\pgfsetlinewidth{1.505625pt}%
\definecolor{currentstroke}{rgb}{1.000000,0.000000,0.000000}%
\pgfsetstrokecolor{currentstroke}%
\pgfsetdash{}{0pt}%
\pgfpathmoveto{\pgfqpoint{1.863816in}{2.679804in}}%
\pgfpathlineto{\pgfqpoint{1.888114in}{2.956893in}}%
\pgfusepath{stroke}%
\end{pgfscope}%
\begin{pgfscope}%
\pgfpathrectangle{\pgfqpoint{0.100000in}{0.212622in}}{\pgfqpoint{3.696000in}{3.696000in}}%
\pgfusepath{clip}%
\pgfsetrectcap%
\pgfsetroundjoin%
\pgfsetlinewidth{1.505625pt}%
\definecolor{currentstroke}{rgb}{1.000000,0.000000,0.000000}%
\pgfsetstrokecolor{currentstroke}%
\pgfsetdash{}{0pt}%
\pgfpathmoveto{\pgfqpoint{1.845465in}{2.685631in}}%
\pgfpathlineto{\pgfqpoint{1.888114in}{2.956893in}}%
\pgfusepath{stroke}%
\end{pgfscope}%
\begin{pgfscope}%
\pgfpathrectangle{\pgfqpoint{0.100000in}{0.212622in}}{\pgfqpoint{3.696000in}{3.696000in}}%
\pgfusepath{clip}%
\pgfsetrectcap%
\pgfsetroundjoin%
\pgfsetlinewidth{1.505625pt}%
\definecolor{currentstroke}{rgb}{1.000000,0.000000,0.000000}%
\pgfsetstrokecolor{currentstroke}%
\pgfsetdash{}{0pt}%
\pgfpathmoveto{\pgfqpoint{1.827371in}{2.687956in}}%
\pgfpathlineto{\pgfqpoint{1.888114in}{2.956893in}}%
\pgfusepath{stroke}%
\end{pgfscope}%
\begin{pgfscope}%
\pgfpathrectangle{\pgfqpoint{0.100000in}{0.212622in}}{\pgfqpoint{3.696000in}{3.696000in}}%
\pgfusepath{clip}%
\pgfsetrectcap%
\pgfsetroundjoin%
\pgfsetlinewidth{1.505625pt}%
\definecolor{currentstroke}{rgb}{1.000000,0.000000,0.000000}%
\pgfsetstrokecolor{currentstroke}%
\pgfsetdash{}{0pt}%
\pgfpathmoveto{\pgfqpoint{1.808968in}{2.690817in}}%
\pgfpathlineto{\pgfqpoint{1.888114in}{2.956893in}}%
\pgfusepath{stroke}%
\end{pgfscope}%
\begin{pgfscope}%
\pgfpathrectangle{\pgfqpoint{0.100000in}{0.212622in}}{\pgfqpoint{3.696000in}{3.696000in}}%
\pgfusepath{clip}%
\pgfsetrectcap%
\pgfsetroundjoin%
\pgfsetlinewidth{1.505625pt}%
\definecolor{currentstroke}{rgb}{1.000000,0.000000,0.000000}%
\pgfsetstrokecolor{currentstroke}%
\pgfsetdash{}{0pt}%
\pgfpathmoveto{\pgfqpoint{1.791511in}{2.693533in}}%
\pgfpathlineto{\pgfqpoint{1.888114in}{2.956893in}}%
\pgfusepath{stroke}%
\end{pgfscope}%
\begin{pgfscope}%
\pgfpathrectangle{\pgfqpoint{0.100000in}{0.212622in}}{\pgfqpoint{3.696000in}{3.696000in}}%
\pgfusepath{clip}%
\pgfsetrectcap%
\pgfsetroundjoin%
\pgfsetlinewidth{1.505625pt}%
\definecolor{currentstroke}{rgb}{1.000000,0.000000,0.000000}%
\pgfsetstrokecolor{currentstroke}%
\pgfsetdash{}{0pt}%
\pgfpathmoveto{\pgfqpoint{1.772449in}{2.693916in}}%
\pgfpathlineto{\pgfqpoint{1.888114in}{2.956893in}}%
\pgfusepath{stroke}%
\end{pgfscope}%
\begin{pgfscope}%
\pgfpathrectangle{\pgfqpoint{0.100000in}{0.212622in}}{\pgfqpoint{3.696000in}{3.696000in}}%
\pgfusepath{clip}%
\pgfsetrectcap%
\pgfsetroundjoin%
\pgfsetlinewidth{1.505625pt}%
\definecolor{currentstroke}{rgb}{1.000000,0.000000,0.000000}%
\pgfsetstrokecolor{currentstroke}%
\pgfsetdash{}{0pt}%
\pgfpathmoveto{\pgfqpoint{1.754399in}{2.695613in}}%
\pgfpathlineto{\pgfqpoint{1.888114in}{2.956893in}}%
\pgfusepath{stroke}%
\end{pgfscope}%
\begin{pgfscope}%
\pgfpathrectangle{\pgfqpoint{0.100000in}{0.212622in}}{\pgfqpoint{3.696000in}{3.696000in}}%
\pgfusepath{clip}%
\pgfsetrectcap%
\pgfsetroundjoin%
\pgfsetlinewidth{1.505625pt}%
\definecolor{currentstroke}{rgb}{1.000000,0.000000,0.000000}%
\pgfsetstrokecolor{currentstroke}%
\pgfsetdash{}{0pt}%
\pgfpathmoveto{\pgfqpoint{1.736833in}{2.697629in}}%
\pgfpathlineto{\pgfqpoint{1.888114in}{2.956893in}}%
\pgfusepath{stroke}%
\end{pgfscope}%
\begin{pgfscope}%
\pgfpathrectangle{\pgfqpoint{0.100000in}{0.212622in}}{\pgfqpoint{3.696000in}{3.696000in}}%
\pgfusepath{clip}%
\pgfsetrectcap%
\pgfsetroundjoin%
\pgfsetlinewidth{1.505625pt}%
\definecolor{currentstroke}{rgb}{1.000000,0.000000,0.000000}%
\pgfsetstrokecolor{currentstroke}%
\pgfsetdash{}{0pt}%
\pgfpathmoveto{\pgfqpoint{1.717824in}{2.699459in}}%
\pgfpathlineto{\pgfqpoint{1.567254in}{2.890146in}}%
\pgfusepath{stroke}%
\end{pgfscope}%
\begin{pgfscope}%
\pgfpathrectangle{\pgfqpoint{0.100000in}{0.212622in}}{\pgfqpoint{3.696000in}{3.696000in}}%
\pgfusepath{clip}%
\pgfsetrectcap%
\pgfsetroundjoin%
\pgfsetlinewidth{1.505625pt}%
\definecolor{currentstroke}{rgb}{1.000000,0.000000,0.000000}%
\pgfsetstrokecolor{currentstroke}%
\pgfsetdash{}{0pt}%
\pgfpathmoveto{\pgfqpoint{1.699465in}{2.701090in}}%
\pgfpathlineto{\pgfqpoint{1.567254in}{2.890146in}}%
\pgfusepath{stroke}%
\end{pgfscope}%
\begin{pgfscope}%
\pgfpathrectangle{\pgfqpoint{0.100000in}{0.212622in}}{\pgfqpoint{3.696000in}{3.696000in}}%
\pgfusepath{clip}%
\pgfsetrectcap%
\pgfsetroundjoin%
\pgfsetlinewidth{1.505625pt}%
\definecolor{currentstroke}{rgb}{1.000000,0.000000,0.000000}%
\pgfsetstrokecolor{currentstroke}%
\pgfsetdash{}{0pt}%
\pgfpathmoveto{\pgfqpoint{1.679984in}{2.701838in}}%
\pgfpathlineto{\pgfqpoint{1.567254in}{2.890146in}}%
\pgfusepath{stroke}%
\end{pgfscope}%
\begin{pgfscope}%
\pgfpathrectangle{\pgfqpoint{0.100000in}{0.212622in}}{\pgfqpoint{3.696000in}{3.696000in}}%
\pgfusepath{clip}%
\pgfsetrectcap%
\pgfsetroundjoin%
\pgfsetlinewidth{1.505625pt}%
\definecolor{currentstroke}{rgb}{1.000000,0.000000,0.000000}%
\pgfsetstrokecolor{currentstroke}%
\pgfsetdash{}{0pt}%
\pgfpathmoveto{\pgfqpoint{1.669099in}{2.702086in}}%
\pgfpathlineto{\pgfqpoint{1.567254in}{2.890146in}}%
\pgfusepath{stroke}%
\end{pgfscope}%
\begin{pgfscope}%
\pgfpathrectangle{\pgfqpoint{0.100000in}{0.212622in}}{\pgfqpoint{3.696000in}{3.696000in}}%
\pgfusepath{clip}%
\pgfsetrectcap%
\pgfsetroundjoin%
\pgfsetlinewidth{1.505625pt}%
\definecolor{currentstroke}{rgb}{1.000000,0.000000,0.000000}%
\pgfsetstrokecolor{currentstroke}%
\pgfsetdash{}{0pt}%
\pgfpathmoveto{\pgfqpoint{1.658076in}{2.702397in}}%
\pgfpathlineto{\pgfqpoint{1.567254in}{2.890146in}}%
\pgfusepath{stroke}%
\end{pgfscope}%
\begin{pgfscope}%
\pgfpathrectangle{\pgfqpoint{0.100000in}{0.212622in}}{\pgfqpoint{3.696000in}{3.696000in}}%
\pgfusepath{clip}%
\pgfsetrectcap%
\pgfsetroundjoin%
\pgfsetlinewidth{1.505625pt}%
\definecolor{currentstroke}{rgb}{1.000000,0.000000,0.000000}%
\pgfsetstrokecolor{currentstroke}%
\pgfsetdash{}{0pt}%
\pgfpathmoveto{\pgfqpoint{1.646549in}{2.702318in}}%
\pgfpathlineto{\pgfqpoint{1.567254in}{2.890146in}}%
\pgfusepath{stroke}%
\end{pgfscope}%
\begin{pgfscope}%
\pgfpathrectangle{\pgfqpoint{0.100000in}{0.212622in}}{\pgfqpoint{3.696000in}{3.696000in}}%
\pgfusepath{clip}%
\pgfsetrectcap%
\pgfsetroundjoin%
\pgfsetlinewidth{1.505625pt}%
\definecolor{currentstroke}{rgb}{1.000000,0.000000,0.000000}%
\pgfsetstrokecolor{currentstroke}%
\pgfsetdash{}{0pt}%
\pgfpathmoveto{\pgfqpoint{1.640557in}{2.702568in}}%
\pgfpathlineto{\pgfqpoint{1.567254in}{2.890146in}}%
\pgfusepath{stroke}%
\end{pgfscope}%
\begin{pgfscope}%
\pgfpathrectangle{\pgfqpoint{0.100000in}{0.212622in}}{\pgfqpoint{3.696000in}{3.696000in}}%
\pgfusepath{clip}%
\pgfsetrectcap%
\pgfsetroundjoin%
\pgfsetlinewidth{1.505625pt}%
\definecolor{currentstroke}{rgb}{1.000000,0.000000,0.000000}%
\pgfsetstrokecolor{currentstroke}%
\pgfsetdash{}{0pt}%
\pgfpathmoveto{\pgfqpoint{1.637645in}{2.702645in}}%
\pgfpathlineto{\pgfqpoint{1.567254in}{2.890146in}}%
\pgfusepath{stroke}%
\end{pgfscope}%
\begin{pgfscope}%
\pgfpathrectangle{\pgfqpoint{0.100000in}{0.212622in}}{\pgfqpoint{3.696000in}{3.696000in}}%
\pgfusepath{clip}%
\pgfsetrectcap%
\pgfsetroundjoin%
\pgfsetlinewidth{1.505625pt}%
\definecolor{currentstroke}{rgb}{1.000000,0.000000,0.000000}%
\pgfsetstrokecolor{currentstroke}%
\pgfsetdash{}{0pt}%
\pgfpathmoveto{\pgfqpoint{1.635892in}{2.702541in}}%
\pgfpathlineto{\pgfqpoint{1.567254in}{2.890146in}}%
\pgfusepath{stroke}%
\end{pgfscope}%
\begin{pgfscope}%
\pgfpathrectangle{\pgfqpoint{0.100000in}{0.212622in}}{\pgfqpoint{3.696000in}{3.696000in}}%
\pgfusepath{clip}%
\pgfsetrectcap%
\pgfsetroundjoin%
\pgfsetlinewidth{1.505625pt}%
\definecolor{currentstroke}{rgb}{1.000000,0.000000,0.000000}%
\pgfsetstrokecolor{currentstroke}%
\pgfsetdash{}{0pt}%
\pgfpathmoveto{\pgfqpoint{1.635045in}{2.702601in}}%
\pgfpathlineto{\pgfqpoint{1.567254in}{2.890146in}}%
\pgfusepath{stroke}%
\end{pgfscope}%
\begin{pgfscope}%
\pgfpathrectangle{\pgfqpoint{0.100000in}{0.212622in}}{\pgfqpoint{3.696000in}{3.696000in}}%
\pgfusepath{clip}%
\pgfsetrectcap%
\pgfsetroundjoin%
\pgfsetlinewidth{1.505625pt}%
\definecolor{currentstroke}{rgb}{1.000000,0.000000,0.000000}%
\pgfsetstrokecolor{currentstroke}%
\pgfsetdash{}{0pt}%
\pgfpathmoveto{\pgfqpoint{1.634526in}{2.702585in}}%
\pgfpathlineto{\pgfqpoint{1.567254in}{2.890146in}}%
\pgfusepath{stroke}%
\end{pgfscope}%
\begin{pgfscope}%
\pgfpathrectangle{\pgfqpoint{0.100000in}{0.212622in}}{\pgfqpoint{3.696000in}{3.696000in}}%
\pgfusepath{clip}%
\pgfsetrectcap%
\pgfsetroundjoin%
\pgfsetlinewidth{1.505625pt}%
\definecolor{currentstroke}{rgb}{1.000000,0.000000,0.000000}%
\pgfsetstrokecolor{currentstroke}%
\pgfsetdash{}{0pt}%
\pgfpathmoveto{\pgfqpoint{1.634291in}{2.702627in}}%
\pgfpathlineto{\pgfqpoint{1.567254in}{2.890146in}}%
\pgfusepath{stroke}%
\end{pgfscope}%
\begin{pgfscope}%
\pgfpathrectangle{\pgfqpoint{0.100000in}{0.212622in}}{\pgfqpoint{3.696000in}{3.696000in}}%
\pgfusepath{clip}%
\pgfsetrectcap%
\pgfsetroundjoin%
\pgfsetlinewidth{1.505625pt}%
\definecolor{currentstroke}{rgb}{1.000000,0.000000,0.000000}%
\pgfsetstrokecolor{currentstroke}%
\pgfsetdash{}{0pt}%
\pgfpathmoveto{\pgfqpoint{1.633785in}{2.702612in}}%
\pgfpathlineto{\pgfqpoint{1.567254in}{2.890146in}}%
\pgfusepath{stroke}%
\end{pgfscope}%
\begin{pgfscope}%
\pgfpathrectangle{\pgfqpoint{0.100000in}{0.212622in}}{\pgfqpoint{3.696000in}{3.696000in}}%
\pgfusepath{clip}%
\pgfsetrectcap%
\pgfsetroundjoin%
\pgfsetlinewidth{1.505625pt}%
\definecolor{currentstroke}{rgb}{1.000000,0.000000,0.000000}%
\pgfsetstrokecolor{currentstroke}%
\pgfsetdash{}{0pt}%
\pgfpathmoveto{\pgfqpoint{1.633557in}{2.702629in}}%
\pgfpathlineto{\pgfqpoint{1.567254in}{2.890146in}}%
\pgfusepath{stroke}%
\end{pgfscope}%
\begin{pgfscope}%
\pgfpathrectangle{\pgfqpoint{0.100000in}{0.212622in}}{\pgfqpoint{3.696000in}{3.696000in}}%
\pgfusepath{clip}%
\pgfsetrectcap%
\pgfsetroundjoin%
\pgfsetlinewidth{1.505625pt}%
\definecolor{currentstroke}{rgb}{1.000000,0.000000,0.000000}%
\pgfsetstrokecolor{currentstroke}%
\pgfsetdash{}{0pt}%
\pgfpathmoveto{\pgfqpoint{1.633423in}{2.702635in}}%
\pgfpathlineto{\pgfqpoint{1.567254in}{2.890146in}}%
\pgfusepath{stroke}%
\end{pgfscope}%
\begin{pgfscope}%
\pgfpathrectangle{\pgfqpoint{0.100000in}{0.212622in}}{\pgfqpoint{3.696000in}{3.696000in}}%
\pgfusepath{clip}%
\pgfsetrectcap%
\pgfsetroundjoin%
\pgfsetlinewidth{1.505625pt}%
\definecolor{currentstroke}{rgb}{1.000000,0.000000,0.000000}%
\pgfsetstrokecolor{currentstroke}%
\pgfsetdash{}{0pt}%
\pgfpathmoveto{\pgfqpoint{1.633346in}{2.702638in}}%
\pgfpathlineto{\pgfqpoint{1.567254in}{2.890146in}}%
\pgfusepath{stroke}%
\end{pgfscope}%
\begin{pgfscope}%
\pgfpathrectangle{\pgfqpoint{0.100000in}{0.212622in}}{\pgfqpoint{3.696000in}{3.696000in}}%
\pgfusepath{clip}%
\pgfsetrectcap%
\pgfsetroundjoin%
\pgfsetlinewidth{1.505625pt}%
\definecolor{currentstroke}{rgb}{1.000000,0.000000,0.000000}%
\pgfsetstrokecolor{currentstroke}%
\pgfsetdash{}{0pt}%
\pgfpathmoveto{\pgfqpoint{1.633306in}{2.702640in}}%
\pgfpathlineto{\pgfqpoint{1.567254in}{2.890146in}}%
\pgfusepath{stroke}%
\end{pgfscope}%
\begin{pgfscope}%
\pgfpathrectangle{\pgfqpoint{0.100000in}{0.212622in}}{\pgfqpoint{3.696000in}{3.696000in}}%
\pgfusepath{clip}%
\pgfsetrectcap%
\pgfsetroundjoin%
\pgfsetlinewidth{1.505625pt}%
\definecolor{currentstroke}{rgb}{1.000000,0.000000,0.000000}%
\pgfsetstrokecolor{currentstroke}%
\pgfsetdash{}{0pt}%
\pgfpathmoveto{\pgfqpoint{1.633283in}{2.702641in}}%
\pgfpathlineto{\pgfqpoint{1.567254in}{2.890146in}}%
\pgfusepath{stroke}%
\end{pgfscope}%
\begin{pgfscope}%
\pgfpathrectangle{\pgfqpoint{0.100000in}{0.212622in}}{\pgfqpoint{3.696000in}{3.696000in}}%
\pgfusepath{clip}%
\pgfsetrectcap%
\pgfsetroundjoin%
\pgfsetlinewidth{1.505625pt}%
\definecolor{currentstroke}{rgb}{1.000000,0.000000,0.000000}%
\pgfsetstrokecolor{currentstroke}%
\pgfsetdash{}{0pt}%
\pgfpathmoveto{\pgfqpoint{1.633271in}{2.702642in}}%
\pgfpathlineto{\pgfqpoint{1.567254in}{2.890146in}}%
\pgfusepath{stroke}%
\end{pgfscope}%
\begin{pgfscope}%
\pgfpathrectangle{\pgfqpoint{0.100000in}{0.212622in}}{\pgfqpoint{3.696000in}{3.696000in}}%
\pgfusepath{clip}%
\pgfsetrectcap%
\pgfsetroundjoin%
\pgfsetlinewidth{1.505625pt}%
\definecolor{currentstroke}{rgb}{1.000000,0.000000,0.000000}%
\pgfsetstrokecolor{currentstroke}%
\pgfsetdash{}{0pt}%
\pgfpathmoveto{\pgfqpoint{1.633264in}{2.702642in}}%
\pgfpathlineto{\pgfqpoint{1.567254in}{2.890146in}}%
\pgfusepath{stroke}%
\end{pgfscope}%
\begin{pgfscope}%
\pgfpathrectangle{\pgfqpoint{0.100000in}{0.212622in}}{\pgfqpoint{3.696000in}{3.696000in}}%
\pgfusepath{clip}%
\pgfsetrectcap%
\pgfsetroundjoin%
\pgfsetlinewidth{1.505625pt}%
\definecolor{currentstroke}{rgb}{1.000000,0.000000,0.000000}%
\pgfsetstrokecolor{currentstroke}%
\pgfsetdash{}{0pt}%
\pgfpathmoveto{\pgfqpoint{1.633260in}{2.702642in}}%
\pgfpathlineto{\pgfqpoint{1.567254in}{2.890146in}}%
\pgfusepath{stroke}%
\end{pgfscope}%
\begin{pgfscope}%
\pgfpathrectangle{\pgfqpoint{0.100000in}{0.212622in}}{\pgfqpoint{3.696000in}{3.696000in}}%
\pgfusepath{clip}%
\pgfsetrectcap%
\pgfsetroundjoin%
\pgfsetlinewidth{1.505625pt}%
\definecolor{currentstroke}{rgb}{1.000000,0.000000,0.000000}%
\pgfsetstrokecolor{currentstroke}%
\pgfsetdash{}{0pt}%
\pgfpathmoveto{\pgfqpoint{1.633258in}{2.702642in}}%
\pgfpathlineto{\pgfqpoint{1.567254in}{2.890146in}}%
\pgfusepath{stroke}%
\end{pgfscope}%
\begin{pgfscope}%
\pgfpathrectangle{\pgfqpoint{0.100000in}{0.212622in}}{\pgfqpoint{3.696000in}{3.696000in}}%
\pgfusepath{clip}%
\pgfsetrectcap%
\pgfsetroundjoin%
\pgfsetlinewidth{1.505625pt}%
\definecolor{currentstroke}{rgb}{1.000000,0.000000,0.000000}%
\pgfsetstrokecolor{currentstroke}%
\pgfsetdash{}{0pt}%
\pgfpathmoveto{\pgfqpoint{1.633257in}{2.702642in}}%
\pgfpathlineto{\pgfqpoint{1.567254in}{2.890146in}}%
\pgfusepath{stroke}%
\end{pgfscope}%
\begin{pgfscope}%
\pgfpathrectangle{\pgfqpoint{0.100000in}{0.212622in}}{\pgfqpoint{3.696000in}{3.696000in}}%
\pgfusepath{clip}%
\pgfsetrectcap%
\pgfsetroundjoin%
\pgfsetlinewidth{1.505625pt}%
\definecolor{currentstroke}{rgb}{1.000000,0.000000,0.000000}%
\pgfsetstrokecolor{currentstroke}%
\pgfsetdash{}{0pt}%
\pgfpathmoveto{\pgfqpoint{1.633256in}{2.702642in}}%
\pgfpathlineto{\pgfqpoint{1.567254in}{2.890146in}}%
\pgfusepath{stroke}%
\end{pgfscope}%
\begin{pgfscope}%
\pgfpathrectangle{\pgfqpoint{0.100000in}{0.212622in}}{\pgfqpoint{3.696000in}{3.696000in}}%
\pgfusepath{clip}%
\pgfsetrectcap%
\pgfsetroundjoin%
\pgfsetlinewidth{1.505625pt}%
\definecolor{currentstroke}{rgb}{1.000000,0.000000,0.000000}%
\pgfsetstrokecolor{currentstroke}%
\pgfsetdash{}{0pt}%
\pgfpathmoveto{\pgfqpoint{1.633256in}{2.702642in}}%
\pgfpathlineto{\pgfqpoint{1.567254in}{2.890146in}}%
\pgfusepath{stroke}%
\end{pgfscope}%
\begin{pgfscope}%
\pgfpathrectangle{\pgfqpoint{0.100000in}{0.212622in}}{\pgfqpoint{3.696000in}{3.696000in}}%
\pgfusepath{clip}%
\pgfsetrectcap%
\pgfsetroundjoin%
\pgfsetlinewidth{1.505625pt}%
\definecolor{currentstroke}{rgb}{1.000000,0.000000,0.000000}%
\pgfsetstrokecolor{currentstroke}%
\pgfsetdash{}{0pt}%
\pgfpathmoveto{\pgfqpoint{1.633256in}{2.702642in}}%
\pgfpathlineto{\pgfqpoint{1.567254in}{2.890146in}}%
\pgfusepath{stroke}%
\end{pgfscope}%
\begin{pgfscope}%
\pgfpathrectangle{\pgfqpoint{0.100000in}{0.212622in}}{\pgfqpoint{3.696000in}{3.696000in}}%
\pgfusepath{clip}%
\pgfsetrectcap%
\pgfsetroundjoin%
\pgfsetlinewidth{1.505625pt}%
\definecolor{currentstroke}{rgb}{1.000000,0.000000,0.000000}%
\pgfsetstrokecolor{currentstroke}%
\pgfsetdash{}{0pt}%
\pgfpathmoveto{\pgfqpoint{1.633256in}{2.702642in}}%
\pgfpathlineto{\pgfqpoint{1.567254in}{2.890146in}}%
\pgfusepath{stroke}%
\end{pgfscope}%
\begin{pgfscope}%
\pgfpathrectangle{\pgfqpoint{0.100000in}{0.212622in}}{\pgfqpoint{3.696000in}{3.696000in}}%
\pgfusepath{clip}%
\pgfsetrectcap%
\pgfsetroundjoin%
\pgfsetlinewidth{1.505625pt}%
\definecolor{currentstroke}{rgb}{1.000000,0.000000,0.000000}%
\pgfsetstrokecolor{currentstroke}%
\pgfsetdash{}{0pt}%
\pgfpathmoveto{\pgfqpoint{1.633256in}{2.702642in}}%
\pgfpathlineto{\pgfqpoint{1.567254in}{2.890146in}}%
\pgfusepath{stroke}%
\end{pgfscope}%
\begin{pgfscope}%
\pgfpathrectangle{\pgfqpoint{0.100000in}{0.212622in}}{\pgfqpoint{3.696000in}{3.696000in}}%
\pgfusepath{clip}%
\pgfsetrectcap%
\pgfsetroundjoin%
\pgfsetlinewidth{1.505625pt}%
\definecolor{currentstroke}{rgb}{1.000000,0.000000,0.000000}%
\pgfsetstrokecolor{currentstroke}%
\pgfsetdash{}{0pt}%
\pgfpathmoveto{\pgfqpoint{1.633082in}{2.702647in}}%
\pgfpathlineto{\pgfqpoint{1.567254in}{2.890146in}}%
\pgfusepath{stroke}%
\end{pgfscope}%
\begin{pgfscope}%
\pgfpathrectangle{\pgfqpoint{0.100000in}{0.212622in}}{\pgfqpoint{3.696000in}{3.696000in}}%
\pgfusepath{clip}%
\pgfsetbuttcap%
\pgfsetroundjoin%
\definecolor{currentfill}{rgb}{0.121569,0.466667,0.705882}%
\pgfsetfillcolor{currentfill}%
\pgfsetfillopacity{0.300000}%
\pgfsetlinewidth{1.003750pt}%
\definecolor{currentstroke}{rgb}{0.121569,0.466667,0.705882}%
\pgfsetstrokecolor{currentstroke}%
\pgfsetstrokeopacity{0.300000}%
\pgfsetdash{}{0pt}%
\pgfpathmoveto{\pgfqpoint{1.633082in}{2.671591in}}%
\pgfpathcurveto{\pgfqpoint{1.641318in}{2.671591in}}{\pgfqpoint{1.649218in}{2.674863in}}{\pgfqpoint{1.655042in}{2.680687in}}%
\pgfpathcurveto{\pgfqpoint{1.660866in}{2.686511in}}{\pgfqpoint{1.664139in}{2.694411in}}{\pgfqpoint{1.664139in}{2.702647in}}%
\pgfpathcurveto{\pgfqpoint{1.664139in}{2.710883in}}{\pgfqpoint{1.660866in}{2.718783in}}{\pgfqpoint{1.655042in}{2.724607in}}%
\pgfpathcurveto{\pgfqpoint{1.649218in}{2.730431in}}{\pgfqpoint{1.641318in}{2.733704in}}{\pgfqpoint{1.633082in}{2.733704in}}%
\pgfpathcurveto{\pgfqpoint{1.624846in}{2.733704in}}{\pgfqpoint{1.616946in}{2.730431in}}{\pgfqpoint{1.611122in}{2.724607in}}%
\pgfpathcurveto{\pgfqpoint{1.605298in}{2.718783in}}{\pgfqpoint{1.602026in}{2.710883in}}{\pgfqpoint{1.602026in}{2.702647in}}%
\pgfpathcurveto{\pgfqpoint{1.602026in}{2.694411in}}{\pgfqpoint{1.605298in}{2.686511in}}{\pgfqpoint{1.611122in}{2.680687in}}%
\pgfpathcurveto{\pgfqpoint{1.616946in}{2.674863in}}{\pgfqpoint{1.624846in}{2.671591in}}{\pgfqpoint{1.633082in}{2.671591in}}%
\pgfpathclose%
\pgfusepath{stroke,fill}%
\end{pgfscope}%
\begin{pgfscope}%
\pgfpathrectangle{\pgfqpoint{0.100000in}{0.212622in}}{\pgfqpoint{3.696000in}{3.696000in}}%
\pgfusepath{clip}%
\pgfsetbuttcap%
\pgfsetroundjoin%
\definecolor{currentfill}{rgb}{0.121569,0.466667,0.705882}%
\pgfsetfillcolor{currentfill}%
\pgfsetfillopacity{0.300078}%
\pgfsetlinewidth{1.003750pt}%
\definecolor{currentstroke}{rgb}{0.121569,0.466667,0.705882}%
\pgfsetstrokecolor{currentstroke}%
\pgfsetstrokeopacity{0.300078}%
\pgfsetdash{}{0pt}%
\pgfpathmoveto{\pgfqpoint{1.633256in}{2.671586in}}%
\pgfpathcurveto{\pgfqpoint{1.641492in}{2.671586in}}{\pgfqpoint{1.649392in}{2.674858in}}{\pgfqpoint{1.655216in}{2.680682in}}%
\pgfpathcurveto{\pgfqpoint{1.661040in}{2.686506in}}{\pgfqpoint{1.664312in}{2.694406in}}{\pgfqpoint{1.664312in}{2.702642in}}%
\pgfpathcurveto{\pgfqpoint{1.664312in}{2.710879in}}{\pgfqpoint{1.661040in}{2.718779in}}{\pgfqpoint{1.655216in}{2.724603in}}%
\pgfpathcurveto{\pgfqpoint{1.649392in}{2.730427in}}{\pgfqpoint{1.641492in}{2.733699in}}{\pgfqpoint{1.633256in}{2.733699in}}%
\pgfpathcurveto{\pgfqpoint{1.625020in}{2.733699in}}{\pgfqpoint{1.617119in}{2.730427in}}{\pgfqpoint{1.611296in}{2.724603in}}%
\pgfpathcurveto{\pgfqpoint{1.605472in}{2.718779in}}{\pgfqpoint{1.602199in}{2.710879in}}{\pgfqpoint{1.602199in}{2.702642in}}%
\pgfpathcurveto{\pgfqpoint{1.602199in}{2.694406in}}{\pgfqpoint{1.605472in}{2.686506in}}{\pgfqpoint{1.611296in}{2.680682in}}%
\pgfpathcurveto{\pgfqpoint{1.617119in}{2.674858in}}{\pgfqpoint{1.625020in}{2.671586in}}{\pgfqpoint{1.633256in}{2.671586in}}%
\pgfpathclose%
\pgfusepath{stroke,fill}%
\end{pgfscope}%
\begin{pgfscope}%
\pgfpathrectangle{\pgfqpoint{0.100000in}{0.212622in}}{\pgfqpoint{3.696000in}{3.696000in}}%
\pgfusepath{clip}%
\pgfsetbuttcap%
\pgfsetroundjoin%
\definecolor{currentfill}{rgb}{0.121569,0.466667,0.705882}%
\pgfsetfillcolor{currentfill}%
\pgfsetfillopacity{0.300078}%
\pgfsetlinewidth{1.003750pt}%
\definecolor{currentstroke}{rgb}{0.121569,0.466667,0.705882}%
\pgfsetstrokecolor{currentstroke}%
\pgfsetstrokeopacity{0.300078}%
\pgfsetdash{}{0pt}%
\pgfpathmoveto{\pgfqpoint{1.633256in}{2.671586in}}%
\pgfpathcurveto{\pgfqpoint{1.641492in}{2.671586in}}{\pgfqpoint{1.649392in}{2.674858in}}{\pgfqpoint{1.655216in}{2.680682in}}%
\pgfpathcurveto{\pgfqpoint{1.661040in}{2.686506in}}{\pgfqpoint{1.664312in}{2.694406in}}{\pgfqpoint{1.664312in}{2.702642in}}%
\pgfpathcurveto{\pgfqpoint{1.664312in}{2.710879in}}{\pgfqpoint{1.661040in}{2.718779in}}{\pgfqpoint{1.655216in}{2.724603in}}%
\pgfpathcurveto{\pgfqpoint{1.649392in}{2.730426in}}{\pgfqpoint{1.641492in}{2.733699in}}{\pgfqpoint{1.633256in}{2.733699in}}%
\pgfpathcurveto{\pgfqpoint{1.625020in}{2.733699in}}{\pgfqpoint{1.617120in}{2.730426in}}{\pgfqpoint{1.611296in}{2.724603in}}%
\pgfpathcurveto{\pgfqpoint{1.605472in}{2.718779in}}{\pgfqpoint{1.602199in}{2.710879in}}{\pgfqpoint{1.602199in}{2.702642in}}%
\pgfpathcurveto{\pgfqpoint{1.602199in}{2.694406in}}{\pgfqpoint{1.605472in}{2.686506in}}{\pgfqpoint{1.611296in}{2.680682in}}%
\pgfpathcurveto{\pgfqpoint{1.617120in}{2.674858in}}{\pgfqpoint{1.625020in}{2.671586in}}{\pgfqpoint{1.633256in}{2.671586in}}%
\pgfpathclose%
\pgfusepath{stroke,fill}%
\end{pgfscope}%
\begin{pgfscope}%
\pgfpathrectangle{\pgfqpoint{0.100000in}{0.212622in}}{\pgfqpoint{3.696000in}{3.696000in}}%
\pgfusepath{clip}%
\pgfsetbuttcap%
\pgfsetroundjoin%
\definecolor{currentfill}{rgb}{0.121569,0.466667,0.705882}%
\pgfsetfillcolor{currentfill}%
\pgfsetfillopacity{0.300078}%
\pgfsetlinewidth{1.003750pt}%
\definecolor{currentstroke}{rgb}{0.121569,0.466667,0.705882}%
\pgfsetstrokecolor{currentstroke}%
\pgfsetstrokeopacity{0.300078}%
\pgfsetdash{}{0pt}%
\pgfpathmoveto{\pgfqpoint{1.633256in}{2.671586in}}%
\pgfpathcurveto{\pgfqpoint{1.641492in}{2.671586in}}{\pgfqpoint{1.649392in}{2.674858in}}{\pgfqpoint{1.655216in}{2.680682in}}%
\pgfpathcurveto{\pgfqpoint{1.661040in}{2.686506in}}{\pgfqpoint{1.664312in}{2.694406in}}{\pgfqpoint{1.664312in}{2.702642in}}%
\pgfpathcurveto{\pgfqpoint{1.664312in}{2.710879in}}{\pgfqpoint{1.661040in}{2.718779in}}{\pgfqpoint{1.655216in}{2.724603in}}%
\pgfpathcurveto{\pgfqpoint{1.649392in}{2.730426in}}{\pgfqpoint{1.641492in}{2.733699in}}{\pgfqpoint{1.633256in}{2.733699in}}%
\pgfpathcurveto{\pgfqpoint{1.625020in}{2.733699in}}{\pgfqpoint{1.617120in}{2.730426in}}{\pgfqpoint{1.611296in}{2.724603in}}%
\pgfpathcurveto{\pgfqpoint{1.605472in}{2.718779in}}{\pgfqpoint{1.602199in}{2.710879in}}{\pgfqpoint{1.602199in}{2.702642in}}%
\pgfpathcurveto{\pgfqpoint{1.602199in}{2.694406in}}{\pgfqpoint{1.605472in}{2.686506in}}{\pgfqpoint{1.611296in}{2.680682in}}%
\pgfpathcurveto{\pgfqpoint{1.617120in}{2.674858in}}{\pgfqpoint{1.625020in}{2.671586in}}{\pgfqpoint{1.633256in}{2.671586in}}%
\pgfpathclose%
\pgfusepath{stroke,fill}%
\end{pgfscope}%
\begin{pgfscope}%
\pgfpathrectangle{\pgfqpoint{0.100000in}{0.212622in}}{\pgfqpoint{3.696000in}{3.696000in}}%
\pgfusepath{clip}%
\pgfsetbuttcap%
\pgfsetroundjoin%
\definecolor{currentfill}{rgb}{0.121569,0.466667,0.705882}%
\pgfsetfillcolor{currentfill}%
\pgfsetfillopacity{0.300079}%
\pgfsetlinewidth{1.003750pt}%
\definecolor{currentstroke}{rgb}{0.121569,0.466667,0.705882}%
\pgfsetstrokecolor{currentstroke}%
\pgfsetstrokeopacity{0.300079}%
\pgfsetdash{}{0pt}%
\pgfpathmoveto{\pgfqpoint{1.633256in}{2.671586in}}%
\pgfpathcurveto{\pgfqpoint{1.641492in}{2.671586in}}{\pgfqpoint{1.649392in}{2.674858in}}{\pgfqpoint{1.655216in}{2.680682in}}%
\pgfpathcurveto{\pgfqpoint{1.661040in}{2.686506in}}{\pgfqpoint{1.664313in}{2.694406in}}{\pgfqpoint{1.664313in}{2.702642in}}%
\pgfpathcurveto{\pgfqpoint{1.664313in}{2.710879in}}{\pgfqpoint{1.661040in}{2.718779in}}{\pgfqpoint{1.655216in}{2.724603in}}%
\pgfpathcurveto{\pgfqpoint{1.649392in}{2.730426in}}{\pgfqpoint{1.641492in}{2.733699in}}{\pgfqpoint{1.633256in}{2.733699in}}%
\pgfpathcurveto{\pgfqpoint{1.625020in}{2.733699in}}{\pgfqpoint{1.617120in}{2.730426in}}{\pgfqpoint{1.611296in}{2.724603in}}%
\pgfpathcurveto{\pgfqpoint{1.605472in}{2.718779in}}{\pgfqpoint{1.602200in}{2.710879in}}{\pgfqpoint{1.602200in}{2.702642in}}%
\pgfpathcurveto{\pgfqpoint{1.602200in}{2.694406in}}{\pgfqpoint{1.605472in}{2.686506in}}{\pgfqpoint{1.611296in}{2.680682in}}%
\pgfpathcurveto{\pgfqpoint{1.617120in}{2.674858in}}{\pgfqpoint{1.625020in}{2.671586in}}{\pgfqpoint{1.633256in}{2.671586in}}%
\pgfpathclose%
\pgfusepath{stroke,fill}%
\end{pgfscope}%
\begin{pgfscope}%
\pgfpathrectangle{\pgfqpoint{0.100000in}{0.212622in}}{\pgfqpoint{3.696000in}{3.696000in}}%
\pgfusepath{clip}%
\pgfsetbuttcap%
\pgfsetroundjoin%
\definecolor{currentfill}{rgb}{0.121569,0.466667,0.705882}%
\pgfsetfillcolor{currentfill}%
\pgfsetfillopacity{0.300079}%
\pgfsetlinewidth{1.003750pt}%
\definecolor{currentstroke}{rgb}{0.121569,0.466667,0.705882}%
\pgfsetstrokecolor{currentstroke}%
\pgfsetstrokeopacity{0.300079}%
\pgfsetdash{}{0pt}%
\pgfpathmoveto{\pgfqpoint{1.633256in}{2.671586in}}%
\pgfpathcurveto{\pgfqpoint{1.641493in}{2.671586in}}{\pgfqpoint{1.649393in}{2.674858in}}{\pgfqpoint{1.655217in}{2.680682in}}%
\pgfpathcurveto{\pgfqpoint{1.661041in}{2.686506in}}{\pgfqpoint{1.664313in}{2.694406in}}{\pgfqpoint{1.664313in}{2.702642in}}%
\pgfpathcurveto{\pgfqpoint{1.664313in}{2.710879in}}{\pgfqpoint{1.661041in}{2.718779in}}{\pgfqpoint{1.655217in}{2.724602in}}%
\pgfpathcurveto{\pgfqpoint{1.649393in}{2.730426in}}{\pgfqpoint{1.641493in}{2.733699in}}{\pgfqpoint{1.633256in}{2.733699in}}%
\pgfpathcurveto{\pgfqpoint{1.625020in}{2.733699in}}{\pgfqpoint{1.617120in}{2.730426in}}{\pgfqpoint{1.611296in}{2.724602in}}%
\pgfpathcurveto{\pgfqpoint{1.605472in}{2.718779in}}{\pgfqpoint{1.602200in}{2.710879in}}{\pgfqpoint{1.602200in}{2.702642in}}%
\pgfpathcurveto{\pgfqpoint{1.602200in}{2.694406in}}{\pgfqpoint{1.605472in}{2.686506in}}{\pgfqpoint{1.611296in}{2.680682in}}%
\pgfpathcurveto{\pgfqpoint{1.617120in}{2.674858in}}{\pgfqpoint{1.625020in}{2.671586in}}{\pgfqpoint{1.633256in}{2.671586in}}%
\pgfpathclose%
\pgfusepath{stroke,fill}%
\end{pgfscope}%
\begin{pgfscope}%
\pgfpathrectangle{\pgfqpoint{0.100000in}{0.212622in}}{\pgfqpoint{3.696000in}{3.696000in}}%
\pgfusepath{clip}%
\pgfsetbuttcap%
\pgfsetroundjoin%
\definecolor{currentfill}{rgb}{0.121569,0.466667,0.705882}%
\pgfsetfillcolor{currentfill}%
\pgfsetfillopacity{0.300079}%
\pgfsetlinewidth{1.003750pt}%
\definecolor{currentstroke}{rgb}{0.121569,0.466667,0.705882}%
\pgfsetstrokecolor{currentstroke}%
\pgfsetstrokeopacity{0.300079}%
\pgfsetdash{}{0pt}%
\pgfpathmoveto{\pgfqpoint{1.633257in}{2.671586in}}%
\pgfpathcurveto{\pgfqpoint{1.641493in}{2.671586in}}{\pgfqpoint{1.649393in}{2.674858in}}{\pgfqpoint{1.655217in}{2.680682in}}%
\pgfpathcurveto{\pgfqpoint{1.661041in}{2.686506in}}{\pgfqpoint{1.664314in}{2.694406in}}{\pgfqpoint{1.664314in}{2.702642in}}%
\pgfpathcurveto{\pgfqpoint{1.664314in}{2.710878in}}{\pgfqpoint{1.661041in}{2.718779in}}{\pgfqpoint{1.655217in}{2.724602in}}%
\pgfpathcurveto{\pgfqpoint{1.649393in}{2.730426in}}{\pgfqpoint{1.641493in}{2.733699in}}{\pgfqpoint{1.633257in}{2.733699in}}%
\pgfpathcurveto{\pgfqpoint{1.625021in}{2.733699in}}{\pgfqpoint{1.617121in}{2.730426in}}{\pgfqpoint{1.611297in}{2.724602in}}%
\pgfpathcurveto{\pgfqpoint{1.605473in}{2.718779in}}{\pgfqpoint{1.602201in}{2.710878in}}{\pgfqpoint{1.602201in}{2.702642in}}%
\pgfpathcurveto{\pgfqpoint{1.602201in}{2.694406in}}{\pgfqpoint{1.605473in}{2.686506in}}{\pgfqpoint{1.611297in}{2.680682in}}%
\pgfpathcurveto{\pgfqpoint{1.617121in}{2.674858in}}{\pgfqpoint{1.625021in}{2.671586in}}{\pgfqpoint{1.633257in}{2.671586in}}%
\pgfpathclose%
\pgfusepath{stroke,fill}%
\end{pgfscope}%
\begin{pgfscope}%
\pgfpathrectangle{\pgfqpoint{0.100000in}{0.212622in}}{\pgfqpoint{3.696000in}{3.696000in}}%
\pgfusepath{clip}%
\pgfsetbuttcap%
\pgfsetroundjoin%
\definecolor{currentfill}{rgb}{0.121569,0.466667,0.705882}%
\pgfsetfillcolor{currentfill}%
\pgfsetfillopacity{0.300079}%
\pgfsetlinewidth{1.003750pt}%
\definecolor{currentstroke}{rgb}{0.121569,0.466667,0.705882}%
\pgfsetstrokecolor{currentstroke}%
\pgfsetstrokeopacity{0.300079}%
\pgfsetdash{}{0pt}%
\pgfpathmoveto{\pgfqpoint{1.633258in}{2.671586in}}%
\pgfpathcurveto{\pgfqpoint{1.641494in}{2.671586in}}{\pgfqpoint{1.649395in}{2.674858in}}{\pgfqpoint{1.655218in}{2.680682in}}%
\pgfpathcurveto{\pgfqpoint{1.661042in}{2.686506in}}{\pgfqpoint{1.664315in}{2.694406in}}{\pgfqpoint{1.664315in}{2.702642in}}%
\pgfpathcurveto{\pgfqpoint{1.664315in}{2.710878in}}{\pgfqpoint{1.661042in}{2.718778in}}{\pgfqpoint{1.655218in}{2.724602in}}%
\pgfpathcurveto{\pgfqpoint{1.649395in}{2.730426in}}{\pgfqpoint{1.641494in}{2.733699in}}{\pgfqpoint{1.633258in}{2.733699in}}%
\pgfpathcurveto{\pgfqpoint{1.625022in}{2.733699in}}{\pgfqpoint{1.617122in}{2.730426in}}{\pgfqpoint{1.611298in}{2.724602in}}%
\pgfpathcurveto{\pgfqpoint{1.605474in}{2.718778in}}{\pgfqpoint{1.602202in}{2.710878in}}{\pgfqpoint{1.602202in}{2.702642in}}%
\pgfpathcurveto{\pgfqpoint{1.602202in}{2.694406in}}{\pgfqpoint{1.605474in}{2.686506in}}{\pgfqpoint{1.611298in}{2.680682in}}%
\pgfpathcurveto{\pgfqpoint{1.617122in}{2.674858in}}{\pgfqpoint{1.625022in}{2.671586in}}{\pgfqpoint{1.633258in}{2.671586in}}%
\pgfpathclose%
\pgfusepath{stroke,fill}%
\end{pgfscope}%
\begin{pgfscope}%
\pgfpathrectangle{\pgfqpoint{0.100000in}{0.212622in}}{\pgfqpoint{3.696000in}{3.696000in}}%
\pgfusepath{clip}%
\pgfsetbuttcap%
\pgfsetroundjoin%
\definecolor{currentfill}{rgb}{0.121569,0.466667,0.705882}%
\pgfsetfillcolor{currentfill}%
\pgfsetfillopacity{0.300080}%
\pgfsetlinewidth{1.003750pt}%
\definecolor{currentstroke}{rgb}{0.121569,0.466667,0.705882}%
\pgfsetstrokecolor{currentstroke}%
\pgfsetstrokeopacity{0.300080}%
\pgfsetdash{}{0pt}%
\pgfpathmoveto{\pgfqpoint{1.633260in}{2.671586in}}%
\pgfpathcurveto{\pgfqpoint{1.641496in}{2.671586in}}{\pgfqpoint{1.649396in}{2.674858in}}{\pgfqpoint{1.655220in}{2.680682in}}%
\pgfpathcurveto{\pgfqpoint{1.661044in}{2.686506in}}{\pgfqpoint{1.664317in}{2.694406in}}{\pgfqpoint{1.664317in}{2.702642in}}%
\pgfpathcurveto{\pgfqpoint{1.664317in}{2.710878in}}{\pgfqpoint{1.661044in}{2.718778in}}{\pgfqpoint{1.655220in}{2.724602in}}%
\pgfpathcurveto{\pgfqpoint{1.649396in}{2.730426in}}{\pgfqpoint{1.641496in}{2.733698in}}{\pgfqpoint{1.633260in}{2.733698in}}%
\pgfpathcurveto{\pgfqpoint{1.625024in}{2.733698in}}{\pgfqpoint{1.617124in}{2.730426in}}{\pgfqpoint{1.611300in}{2.724602in}}%
\pgfpathcurveto{\pgfqpoint{1.605476in}{2.718778in}}{\pgfqpoint{1.602204in}{2.710878in}}{\pgfqpoint{1.602204in}{2.702642in}}%
\pgfpathcurveto{\pgfqpoint{1.602204in}{2.694406in}}{\pgfqpoint{1.605476in}{2.686506in}}{\pgfqpoint{1.611300in}{2.680682in}}%
\pgfpathcurveto{\pgfqpoint{1.617124in}{2.674858in}}{\pgfqpoint{1.625024in}{2.671586in}}{\pgfqpoint{1.633260in}{2.671586in}}%
\pgfpathclose%
\pgfusepath{stroke,fill}%
\end{pgfscope}%
\begin{pgfscope}%
\pgfpathrectangle{\pgfqpoint{0.100000in}{0.212622in}}{\pgfqpoint{3.696000in}{3.696000in}}%
\pgfusepath{clip}%
\pgfsetbuttcap%
\pgfsetroundjoin%
\definecolor{currentfill}{rgb}{0.121569,0.466667,0.705882}%
\pgfsetfillcolor{currentfill}%
\pgfsetfillopacity{0.300082}%
\pgfsetlinewidth{1.003750pt}%
\definecolor{currentstroke}{rgb}{0.121569,0.466667,0.705882}%
\pgfsetstrokecolor{currentstroke}%
\pgfsetstrokeopacity{0.300082}%
\pgfsetdash{}{0pt}%
\pgfpathmoveto{\pgfqpoint{1.633264in}{2.671585in}}%
\pgfpathcurveto{\pgfqpoint{1.641500in}{2.671585in}}{\pgfqpoint{1.649400in}{2.674858in}}{\pgfqpoint{1.655224in}{2.680682in}}%
\pgfpathcurveto{\pgfqpoint{1.661048in}{2.686506in}}{\pgfqpoint{1.664320in}{2.694406in}}{\pgfqpoint{1.664320in}{2.702642in}}%
\pgfpathcurveto{\pgfqpoint{1.664320in}{2.710878in}}{\pgfqpoint{1.661048in}{2.718778in}}{\pgfqpoint{1.655224in}{2.724602in}}%
\pgfpathcurveto{\pgfqpoint{1.649400in}{2.730426in}}{\pgfqpoint{1.641500in}{2.733698in}}{\pgfqpoint{1.633264in}{2.733698in}}%
\pgfpathcurveto{\pgfqpoint{1.625028in}{2.733698in}}{\pgfqpoint{1.617128in}{2.730426in}}{\pgfqpoint{1.611304in}{2.724602in}}%
\pgfpathcurveto{\pgfqpoint{1.605480in}{2.718778in}}{\pgfqpoint{1.602207in}{2.710878in}}{\pgfqpoint{1.602207in}{2.702642in}}%
\pgfpathcurveto{\pgfqpoint{1.602207in}{2.694406in}}{\pgfqpoint{1.605480in}{2.686506in}}{\pgfqpoint{1.611304in}{2.680682in}}%
\pgfpathcurveto{\pgfqpoint{1.617128in}{2.674858in}}{\pgfqpoint{1.625028in}{2.671585in}}{\pgfqpoint{1.633264in}{2.671585in}}%
\pgfpathclose%
\pgfusepath{stroke,fill}%
\end{pgfscope}%
\begin{pgfscope}%
\pgfpathrectangle{\pgfqpoint{0.100000in}{0.212622in}}{\pgfqpoint{3.696000in}{3.696000in}}%
\pgfusepath{clip}%
\pgfsetbuttcap%
\pgfsetroundjoin%
\definecolor{currentfill}{rgb}{0.121569,0.466667,0.705882}%
\pgfsetfillcolor{currentfill}%
\pgfsetfillopacity{0.300085}%
\pgfsetlinewidth{1.003750pt}%
\definecolor{currentstroke}{rgb}{0.121569,0.466667,0.705882}%
\pgfsetstrokecolor{currentstroke}%
\pgfsetstrokeopacity{0.300085}%
\pgfsetdash{}{0pt}%
\pgfpathmoveto{\pgfqpoint{1.633271in}{2.671586in}}%
\pgfpathcurveto{\pgfqpoint{1.641508in}{2.671586in}}{\pgfqpoint{1.649408in}{2.674858in}}{\pgfqpoint{1.655232in}{2.680682in}}%
\pgfpathcurveto{\pgfqpoint{1.661056in}{2.686506in}}{\pgfqpoint{1.664328in}{2.694406in}}{\pgfqpoint{1.664328in}{2.702642in}}%
\pgfpathcurveto{\pgfqpoint{1.664328in}{2.710878in}}{\pgfqpoint{1.661056in}{2.718778in}}{\pgfqpoint{1.655232in}{2.724602in}}%
\pgfpathcurveto{\pgfqpoint{1.649408in}{2.730426in}}{\pgfqpoint{1.641508in}{2.733699in}}{\pgfqpoint{1.633271in}{2.733699in}}%
\pgfpathcurveto{\pgfqpoint{1.625035in}{2.733699in}}{\pgfqpoint{1.617135in}{2.730426in}}{\pgfqpoint{1.611311in}{2.724602in}}%
\pgfpathcurveto{\pgfqpoint{1.605487in}{2.718778in}}{\pgfqpoint{1.602215in}{2.710878in}}{\pgfqpoint{1.602215in}{2.702642in}}%
\pgfpathcurveto{\pgfqpoint{1.602215in}{2.694406in}}{\pgfqpoint{1.605487in}{2.686506in}}{\pgfqpoint{1.611311in}{2.680682in}}%
\pgfpathcurveto{\pgfqpoint{1.617135in}{2.674858in}}{\pgfqpoint{1.625035in}{2.671586in}}{\pgfqpoint{1.633271in}{2.671586in}}%
\pgfpathclose%
\pgfusepath{stroke,fill}%
\end{pgfscope}%
\begin{pgfscope}%
\pgfpathrectangle{\pgfqpoint{0.100000in}{0.212622in}}{\pgfqpoint{3.696000in}{3.696000in}}%
\pgfusepath{clip}%
\pgfsetbuttcap%
\pgfsetroundjoin%
\definecolor{currentfill}{rgb}{0.121569,0.466667,0.705882}%
\pgfsetfillcolor{currentfill}%
\pgfsetfillopacity{0.300090}%
\pgfsetlinewidth{1.003750pt}%
\definecolor{currentstroke}{rgb}{0.121569,0.466667,0.705882}%
\pgfsetstrokecolor{currentstroke}%
\pgfsetstrokeopacity{0.300090}%
\pgfsetdash{}{0pt}%
\pgfpathmoveto{\pgfqpoint{1.633283in}{2.671584in}}%
\pgfpathcurveto{\pgfqpoint{1.641520in}{2.671584in}}{\pgfqpoint{1.649420in}{2.674857in}}{\pgfqpoint{1.655244in}{2.680681in}}%
\pgfpathcurveto{\pgfqpoint{1.661067in}{2.686504in}}{\pgfqpoint{1.664340in}{2.694405in}}{\pgfqpoint{1.664340in}{2.702641in}}%
\pgfpathcurveto{\pgfqpoint{1.664340in}{2.710877in}}{\pgfqpoint{1.661067in}{2.718777in}}{\pgfqpoint{1.655244in}{2.724601in}}%
\pgfpathcurveto{\pgfqpoint{1.649420in}{2.730425in}}{\pgfqpoint{1.641520in}{2.733697in}}{\pgfqpoint{1.633283in}{2.733697in}}%
\pgfpathcurveto{\pgfqpoint{1.625047in}{2.733697in}}{\pgfqpoint{1.617147in}{2.730425in}}{\pgfqpoint{1.611323in}{2.724601in}}%
\pgfpathcurveto{\pgfqpoint{1.605499in}{2.718777in}}{\pgfqpoint{1.602227in}{2.710877in}}{\pgfqpoint{1.602227in}{2.702641in}}%
\pgfpathcurveto{\pgfqpoint{1.602227in}{2.694405in}}{\pgfqpoint{1.605499in}{2.686504in}}{\pgfqpoint{1.611323in}{2.680681in}}%
\pgfpathcurveto{\pgfqpoint{1.617147in}{2.674857in}}{\pgfqpoint{1.625047in}{2.671584in}}{\pgfqpoint{1.633283in}{2.671584in}}%
\pgfpathclose%
\pgfusepath{stroke,fill}%
\end{pgfscope}%
\begin{pgfscope}%
\pgfpathrectangle{\pgfqpoint{0.100000in}{0.212622in}}{\pgfqpoint{3.696000in}{3.696000in}}%
\pgfusepath{clip}%
\pgfsetbuttcap%
\pgfsetroundjoin%
\definecolor{currentfill}{rgb}{0.121569,0.466667,0.705882}%
\pgfsetfillcolor{currentfill}%
\pgfsetfillopacity{0.300100}%
\pgfsetlinewidth{1.003750pt}%
\definecolor{currentstroke}{rgb}{0.121569,0.466667,0.705882}%
\pgfsetstrokecolor{currentstroke}%
\pgfsetstrokeopacity{0.300100}%
\pgfsetdash{}{0pt}%
\pgfpathmoveto{\pgfqpoint{1.633306in}{2.671584in}}%
\pgfpathcurveto{\pgfqpoint{1.641543in}{2.671584in}}{\pgfqpoint{1.649443in}{2.674856in}}{\pgfqpoint{1.655267in}{2.680680in}}%
\pgfpathcurveto{\pgfqpoint{1.661091in}{2.686504in}}{\pgfqpoint{1.664363in}{2.694404in}}{\pgfqpoint{1.664363in}{2.702640in}}%
\pgfpathcurveto{\pgfqpoint{1.664363in}{2.710877in}}{\pgfqpoint{1.661091in}{2.718777in}}{\pgfqpoint{1.655267in}{2.724601in}}%
\pgfpathcurveto{\pgfqpoint{1.649443in}{2.730425in}}{\pgfqpoint{1.641543in}{2.733697in}}{\pgfqpoint{1.633306in}{2.733697in}}%
\pgfpathcurveto{\pgfqpoint{1.625070in}{2.733697in}}{\pgfqpoint{1.617170in}{2.730425in}}{\pgfqpoint{1.611346in}{2.724601in}}%
\pgfpathcurveto{\pgfqpoint{1.605522in}{2.718777in}}{\pgfqpoint{1.602250in}{2.710877in}}{\pgfqpoint{1.602250in}{2.702640in}}%
\pgfpathcurveto{\pgfqpoint{1.602250in}{2.694404in}}{\pgfqpoint{1.605522in}{2.686504in}}{\pgfqpoint{1.611346in}{2.680680in}}%
\pgfpathcurveto{\pgfqpoint{1.617170in}{2.674856in}}{\pgfqpoint{1.625070in}{2.671584in}}{\pgfqpoint{1.633306in}{2.671584in}}%
\pgfpathclose%
\pgfusepath{stroke,fill}%
\end{pgfscope}%
\begin{pgfscope}%
\pgfpathrectangle{\pgfqpoint{0.100000in}{0.212622in}}{\pgfqpoint{3.696000in}{3.696000in}}%
\pgfusepath{clip}%
\pgfsetbuttcap%
\pgfsetroundjoin%
\definecolor{currentfill}{rgb}{0.121569,0.466667,0.705882}%
\pgfsetfillcolor{currentfill}%
\pgfsetfillopacity{0.300119}%
\pgfsetlinewidth{1.003750pt}%
\definecolor{currentstroke}{rgb}{0.121569,0.466667,0.705882}%
\pgfsetstrokecolor{currentstroke}%
\pgfsetstrokeopacity{0.300119}%
\pgfsetdash{}{0pt}%
\pgfpathmoveto{\pgfqpoint{1.633346in}{2.671581in}}%
\pgfpathcurveto{\pgfqpoint{1.641582in}{2.671581in}}{\pgfqpoint{1.649482in}{2.674854in}}{\pgfqpoint{1.655306in}{2.680678in}}%
\pgfpathcurveto{\pgfqpoint{1.661130in}{2.686502in}}{\pgfqpoint{1.664402in}{2.694402in}}{\pgfqpoint{1.664402in}{2.702638in}}%
\pgfpathcurveto{\pgfqpoint{1.664402in}{2.710874in}}{\pgfqpoint{1.661130in}{2.718774in}}{\pgfqpoint{1.655306in}{2.724598in}}%
\pgfpathcurveto{\pgfqpoint{1.649482in}{2.730422in}}{\pgfqpoint{1.641582in}{2.733694in}}{\pgfqpoint{1.633346in}{2.733694in}}%
\pgfpathcurveto{\pgfqpoint{1.625109in}{2.733694in}}{\pgfqpoint{1.617209in}{2.730422in}}{\pgfqpoint{1.611385in}{2.724598in}}%
\pgfpathcurveto{\pgfqpoint{1.605561in}{2.718774in}}{\pgfqpoint{1.602289in}{2.710874in}}{\pgfqpoint{1.602289in}{2.702638in}}%
\pgfpathcurveto{\pgfqpoint{1.602289in}{2.694402in}}{\pgfqpoint{1.605561in}{2.686502in}}{\pgfqpoint{1.611385in}{2.680678in}}%
\pgfpathcurveto{\pgfqpoint{1.617209in}{2.674854in}}{\pgfqpoint{1.625109in}{2.671581in}}{\pgfqpoint{1.633346in}{2.671581in}}%
\pgfpathclose%
\pgfusepath{stroke,fill}%
\end{pgfscope}%
\begin{pgfscope}%
\pgfpathrectangle{\pgfqpoint{0.100000in}{0.212622in}}{\pgfqpoint{3.696000in}{3.696000in}}%
\pgfusepath{clip}%
\pgfsetbuttcap%
\pgfsetroundjoin%
\definecolor{currentfill}{rgb}{0.121569,0.466667,0.705882}%
\pgfsetfillcolor{currentfill}%
\pgfsetfillopacity{0.300150}%
\pgfsetlinewidth{1.003750pt}%
\definecolor{currentstroke}{rgb}{0.121569,0.466667,0.705882}%
\pgfsetstrokecolor{currentstroke}%
\pgfsetstrokeopacity{0.300150}%
\pgfsetdash{}{0pt}%
\pgfpathmoveto{\pgfqpoint{1.633423in}{2.671578in}}%
\pgfpathcurveto{\pgfqpoint{1.641660in}{2.671578in}}{\pgfqpoint{1.649560in}{2.674851in}}{\pgfqpoint{1.655384in}{2.680675in}}%
\pgfpathcurveto{\pgfqpoint{1.661208in}{2.686499in}}{\pgfqpoint{1.664480in}{2.694399in}}{\pgfqpoint{1.664480in}{2.702635in}}%
\pgfpathcurveto{\pgfqpoint{1.664480in}{2.710871in}}{\pgfqpoint{1.661208in}{2.718771in}}{\pgfqpoint{1.655384in}{2.724595in}}%
\pgfpathcurveto{\pgfqpoint{1.649560in}{2.730419in}}{\pgfqpoint{1.641660in}{2.733691in}}{\pgfqpoint{1.633423in}{2.733691in}}%
\pgfpathcurveto{\pgfqpoint{1.625187in}{2.733691in}}{\pgfqpoint{1.617287in}{2.730419in}}{\pgfqpoint{1.611463in}{2.724595in}}%
\pgfpathcurveto{\pgfqpoint{1.605639in}{2.718771in}}{\pgfqpoint{1.602367in}{2.710871in}}{\pgfqpoint{1.602367in}{2.702635in}}%
\pgfpathcurveto{\pgfqpoint{1.602367in}{2.694399in}}{\pgfqpoint{1.605639in}{2.686499in}}{\pgfqpoint{1.611463in}{2.680675in}}%
\pgfpathcurveto{\pgfqpoint{1.617287in}{2.674851in}}{\pgfqpoint{1.625187in}{2.671578in}}{\pgfqpoint{1.633423in}{2.671578in}}%
\pgfpathclose%
\pgfusepath{stroke,fill}%
\end{pgfscope}%
\begin{pgfscope}%
\pgfpathrectangle{\pgfqpoint{0.100000in}{0.212622in}}{\pgfqpoint{3.696000in}{3.696000in}}%
\pgfusepath{clip}%
\pgfsetbuttcap%
\pgfsetroundjoin%
\definecolor{currentfill}{rgb}{0.121569,0.466667,0.705882}%
\pgfsetfillcolor{currentfill}%
\pgfsetfillopacity{0.300211}%
\pgfsetlinewidth{1.003750pt}%
\definecolor{currentstroke}{rgb}{0.121569,0.466667,0.705882}%
\pgfsetstrokecolor{currentstroke}%
\pgfsetstrokeopacity{0.300211}%
\pgfsetdash{}{0pt}%
\pgfpathmoveto{\pgfqpoint{1.633557in}{2.671573in}}%
\pgfpathcurveto{\pgfqpoint{1.641793in}{2.671573in}}{\pgfqpoint{1.649693in}{2.674845in}}{\pgfqpoint{1.655517in}{2.680669in}}%
\pgfpathcurveto{\pgfqpoint{1.661341in}{2.686493in}}{\pgfqpoint{1.664613in}{2.694393in}}{\pgfqpoint{1.664613in}{2.702629in}}%
\pgfpathcurveto{\pgfqpoint{1.664613in}{2.710866in}}{\pgfqpoint{1.661341in}{2.718766in}}{\pgfqpoint{1.655517in}{2.724590in}}%
\pgfpathcurveto{\pgfqpoint{1.649693in}{2.730413in}}{\pgfqpoint{1.641793in}{2.733686in}}{\pgfqpoint{1.633557in}{2.733686in}}%
\pgfpathcurveto{\pgfqpoint{1.625321in}{2.733686in}}{\pgfqpoint{1.617421in}{2.730413in}}{\pgfqpoint{1.611597in}{2.724590in}}%
\pgfpathcurveto{\pgfqpoint{1.605773in}{2.718766in}}{\pgfqpoint{1.602500in}{2.710866in}}{\pgfqpoint{1.602500in}{2.702629in}}%
\pgfpathcurveto{\pgfqpoint{1.602500in}{2.694393in}}{\pgfqpoint{1.605773in}{2.686493in}}{\pgfqpoint{1.611597in}{2.680669in}}%
\pgfpathcurveto{\pgfqpoint{1.617421in}{2.674845in}}{\pgfqpoint{1.625321in}{2.671573in}}{\pgfqpoint{1.633557in}{2.671573in}}%
\pgfpathclose%
\pgfusepath{stroke,fill}%
\end{pgfscope}%
\begin{pgfscope}%
\pgfpathrectangle{\pgfqpoint{0.100000in}{0.212622in}}{\pgfqpoint{3.696000in}{3.696000in}}%
\pgfusepath{clip}%
\pgfsetbuttcap%
\pgfsetroundjoin%
\definecolor{currentfill}{rgb}{0.121569,0.466667,0.705882}%
\pgfsetfillcolor{currentfill}%
\pgfsetfillopacity{0.300327}%
\pgfsetlinewidth{1.003750pt}%
\definecolor{currentstroke}{rgb}{0.121569,0.466667,0.705882}%
\pgfsetstrokecolor{currentstroke}%
\pgfsetstrokeopacity{0.300327}%
\pgfsetdash{}{0pt}%
\pgfpathmoveto{\pgfqpoint{1.633785in}{2.671555in}}%
\pgfpathcurveto{\pgfqpoint{1.642021in}{2.671555in}}{\pgfqpoint{1.649921in}{2.674828in}}{\pgfqpoint{1.655745in}{2.680652in}}%
\pgfpathcurveto{\pgfqpoint{1.661569in}{2.686475in}}{\pgfqpoint{1.664841in}{2.694375in}}{\pgfqpoint{1.664841in}{2.702612in}}%
\pgfpathcurveto{\pgfqpoint{1.664841in}{2.710848in}}{\pgfqpoint{1.661569in}{2.718748in}}{\pgfqpoint{1.655745in}{2.724572in}}%
\pgfpathcurveto{\pgfqpoint{1.649921in}{2.730396in}}{\pgfqpoint{1.642021in}{2.733668in}}{\pgfqpoint{1.633785in}{2.733668in}}%
\pgfpathcurveto{\pgfqpoint{1.625548in}{2.733668in}}{\pgfqpoint{1.617648in}{2.730396in}}{\pgfqpoint{1.611824in}{2.724572in}}%
\pgfpathcurveto{\pgfqpoint{1.606001in}{2.718748in}}{\pgfqpoint{1.602728in}{2.710848in}}{\pgfqpoint{1.602728in}{2.702612in}}%
\pgfpathcurveto{\pgfqpoint{1.602728in}{2.694375in}}{\pgfqpoint{1.606001in}{2.686475in}}{\pgfqpoint{1.611824in}{2.680652in}}%
\pgfpathcurveto{\pgfqpoint{1.617648in}{2.674828in}}{\pgfqpoint{1.625548in}{2.671555in}}{\pgfqpoint{1.633785in}{2.671555in}}%
\pgfpathclose%
\pgfusepath{stroke,fill}%
\end{pgfscope}%
\begin{pgfscope}%
\pgfpathrectangle{\pgfqpoint{0.100000in}{0.212622in}}{\pgfqpoint{3.696000in}{3.696000in}}%
\pgfusepath{clip}%
\pgfsetbuttcap%
\pgfsetroundjoin%
\definecolor{currentfill}{rgb}{0.121569,0.466667,0.705882}%
\pgfsetfillcolor{currentfill}%
\pgfsetfillopacity{0.300501}%
\pgfsetlinewidth{1.003750pt}%
\definecolor{currentstroke}{rgb}{0.121569,0.466667,0.705882}%
\pgfsetstrokecolor{currentstroke}%
\pgfsetstrokeopacity{0.300501}%
\pgfsetdash{}{0pt}%
\pgfpathmoveto{\pgfqpoint{1.634291in}{2.671571in}}%
\pgfpathcurveto{\pgfqpoint{1.642527in}{2.671571in}}{\pgfqpoint{1.650427in}{2.674843in}}{\pgfqpoint{1.656251in}{2.680667in}}%
\pgfpathcurveto{\pgfqpoint{1.662075in}{2.686491in}}{\pgfqpoint{1.665348in}{2.694391in}}{\pgfqpoint{1.665348in}{2.702627in}}%
\pgfpathcurveto{\pgfqpoint{1.665348in}{2.710864in}}{\pgfqpoint{1.662075in}{2.718764in}}{\pgfqpoint{1.656251in}{2.724588in}}%
\pgfpathcurveto{\pgfqpoint{1.650427in}{2.730411in}}{\pgfqpoint{1.642527in}{2.733684in}}{\pgfqpoint{1.634291in}{2.733684in}}%
\pgfpathcurveto{\pgfqpoint{1.626055in}{2.733684in}}{\pgfqpoint{1.618155in}{2.730411in}}{\pgfqpoint{1.612331in}{2.724588in}}%
\pgfpathcurveto{\pgfqpoint{1.606507in}{2.718764in}}{\pgfqpoint{1.603235in}{2.710864in}}{\pgfqpoint{1.603235in}{2.702627in}}%
\pgfpathcurveto{\pgfqpoint{1.603235in}{2.694391in}}{\pgfqpoint{1.606507in}{2.686491in}}{\pgfqpoint{1.612331in}{2.680667in}}%
\pgfpathcurveto{\pgfqpoint{1.618155in}{2.674843in}}{\pgfqpoint{1.626055in}{2.671571in}}{\pgfqpoint{1.634291in}{2.671571in}}%
\pgfpathclose%
\pgfusepath{stroke,fill}%
\end{pgfscope}%
\begin{pgfscope}%
\pgfpathrectangle{\pgfqpoint{0.100000in}{0.212622in}}{\pgfqpoint{3.696000in}{3.696000in}}%
\pgfusepath{clip}%
\pgfsetbuttcap%
\pgfsetroundjoin%
\definecolor{currentfill}{rgb}{0.121569,0.466667,0.705882}%
\pgfsetfillcolor{currentfill}%
\pgfsetfillopacity{0.300622}%
\pgfsetlinewidth{1.003750pt}%
\definecolor{currentstroke}{rgb}{0.121569,0.466667,0.705882}%
\pgfsetstrokecolor{currentstroke}%
\pgfsetstrokeopacity{0.300622}%
\pgfsetdash{}{0pt}%
\pgfpathmoveto{\pgfqpoint{1.634526in}{2.671529in}}%
\pgfpathcurveto{\pgfqpoint{1.642762in}{2.671529in}}{\pgfqpoint{1.650662in}{2.674801in}}{\pgfqpoint{1.656486in}{2.680625in}}%
\pgfpathcurveto{\pgfqpoint{1.662310in}{2.686449in}}{\pgfqpoint{1.665582in}{2.694349in}}{\pgfqpoint{1.665582in}{2.702585in}}%
\pgfpathcurveto{\pgfqpoint{1.665582in}{2.710821in}}{\pgfqpoint{1.662310in}{2.718721in}}{\pgfqpoint{1.656486in}{2.724545in}}%
\pgfpathcurveto{\pgfqpoint{1.650662in}{2.730369in}}{\pgfqpoint{1.642762in}{2.733642in}}{\pgfqpoint{1.634526in}{2.733642in}}%
\pgfpathcurveto{\pgfqpoint{1.626290in}{2.733642in}}{\pgfqpoint{1.618390in}{2.730369in}}{\pgfqpoint{1.612566in}{2.724545in}}%
\pgfpathcurveto{\pgfqpoint{1.606742in}{2.718721in}}{\pgfqpoint{1.603469in}{2.710821in}}{\pgfqpoint{1.603469in}{2.702585in}}%
\pgfpathcurveto{\pgfqpoint{1.603469in}{2.694349in}}{\pgfqpoint{1.606742in}{2.686449in}}{\pgfqpoint{1.612566in}{2.680625in}}%
\pgfpathcurveto{\pgfqpoint{1.618390in}{2.674801in}}{\pgfqpoint{1.626290in}{2.671529in}}{\pgfqpoint{1.634526in}{2.671529in}}%
\pgfpathclose%
\pgfusepath{stroke,fill}%
\end{pgfscope}%
\begin{pgfscope}%
\pgfpathrectangle{\pgfqpoint{0.100000in}{0.212622in}}{\pgfqpoint{3.696000in}{3.696000in}}%
\pgfusepath{clip}%
\pgfsetbuttcap%
\pgfsetroundjoin%
\definecolor{currentfill}{rgb}{0.121569,0.466667,0.705882}%
\pgfsetfillcolor{currentfill}%
\pgfsetfillopacity{0.300817}%
\pgfsetlinewidth{1.003750pt}%
\definecolor{currentstroke}{rgb}{0.121569,0.466667,0.705882}%
\pgfsetstrokecolor{currentstroke}%
\pgfsetstrokeopacity{0.300817}%
\pgfsetdash{}{0pt}%
\pgfpathmoveto{\pgfqpoint{1.635045in}{2.671544in}}%
\pgfpathcurveto{\pgfqpoint{1.643282in}{2.671544in}}{\pgfqpoint{1.651182in}{2.674817in}}{\pgfqpoint{1.657006in}{2.680641in}}%
\pgfpathcurveto{\pgfqpoint{1.662830in}{2.686465in}}{\pgfqpoint{1.666102in}{2.694365in}}{\pgfqpoint{1.666102in}{2.702601in}}%
\pgfpathcurveto{\pgfqpoint{1.666102in}{2.710837in}}{\pgfqpoint{1.662830in}{2.718737in}}{\pgfqpoint{1.657006in}{2.724561in}}%
\pgfpathcurveto{\pgfqpoint{1.651182in}{2.730385in}}{\pgfqpoint{1.643282in}{2.733657in}}{\pgfqpoint{1.635045in}{2.733657in}}%
\pgfpathcurveto{\pgfqpoint{1.626809in}{2.733657in}}{\pgfqpoint{1.618909in}{2.730385in}}{\pgfqpoint{1.613085in}{2.724561in}}%
\pgfpathcurveto{\pgfqpoint{1.607261in}{2.718737in}}{\pgfqpoint{1.603989in}{2.710837in}}{\pgfqpoint{1.603989in}{2.702601in}}%
\pgfpathcurveto{\pgfqpoint{1.603989in}{2.694365in}}{\pgfqpoint{1.607261in}{2.686465in}}{\pgfqpoint{1.613085in}{2.680641in}}%
\pgfpathcurveto{\pgfqpoint{1.618909in}{2.674817in}}{\pgfqpoint{1.626809in}{2.671544in}}{\pgfqpoint{1.635045in}{2.671544in}}%
\pgfpathclose%
\pgfusepath{stroke,fill}%
\end{pgfscope}%
\begin{pgfscope}%
\pgfpathrectangle{\pgfqpoint{0.100000in}{0.212622in}}{\pgfqpoint{3.696000in}{3.696000in}}%
\pgfusepath{clip}%
\pgfsetbuttcap%
\pgfsetroundjoin%
\definecolor{currentfill}{rgb}{0.121569,0.466667,0.705882}%
\pgfsetfillcolor{currentfill}%
\pgfsetfillopacity{0.301204}%
\pgfsetlinewidth{1.003750pt}%
\definecolor{currentstroke}{rgb}{0.121569,0.466667,0.705882}%
\pgfsetstrokecolor{currentstroke}%
\pgfsetstrokeopacity{0.301204}%
\pgfsetdash{}{0pt}%
\pgfpathmoveto{\pgfqpoint{1.635892in}{2.671485in}}%
\pgfpathcurveto{\pgfqpoint{1.644129in}{2.671485in}}{\pgfqpoint{1.652029in}{2.674757in}}{\pgfqpoint{1.657853in}{2.680581in}}%
\pgfpathcurveto{\pgfqpoint{1.663676in}{2.686405in}}{\pgfqpoint{1.666949in}{2.694305in}}{\pgfqpoint{1.666949in}{2.702541in}}%
\pgfpathcurveto{\pgfqpoint{1.666949in}{2.710778in}}{\pgfqpoint{1.663676in}{2.718678in}}{\pgfqpoint{1.657853in}{2.724502in}}%
\pgfpathcurveto{\pgfqpoint{1.652029in}{2.730326in}}{\pgfqpoint{1.644129in}{2.733598in}}{\pgfqpoint{1.635892in}{2.733598in}}%
\pgfpathcurveto{\pgfqpoint{1.627656in}{2.733598in}}{\pgfqpoint{1.619756in}{2.730326in}}{\pgfqpoint{1.613932in}{2.724502in}}%
\pgfpathcurveto{\pgfqpoint{1.608108in}{2.718678in}}{\pgfqpoint{1.604836in}{2.710778in}}{\pgfqpoint{1.604836in}{2.702541in}}%
\pgfpathcurveto{\pgfqpoint{1.604836in}{2.694305in}}{\pgfqpoint{1.608108in}{2.686405in}}{\pgfqpoint{1.613932in}{2.680581in}}%
\pgfpathcurveto{\pgfqpoint{1.619756in}{2.674757in}}{\pgfqpoint{1.627656in}{2.671485in}}{\pgfqpoint{1.635892in}{2.671485in}}%
\pgfpathclose%
\pgfusepath{stroke,fill}%
\end{pgfscope}%
\begin{pgfscope}%
\pgfpathrectangle{\pgfqpoint{0.100000in}{0.212622in}}{\pgfqpoint{3.696000in}{3.696000in}}%
\pgfusepath{clip}%
\pgfsetbuttcap%
\pgfsetroundjoin%
\definecolor{currentfill}{rgb}{0.121569,0.466667,0.705882}%
\pgfsetfillcolor{currentfill}%
\pgfsetfillopacity{0.301838}%
\pgfsetlinewidth{1.003750pt}%
\definecolor{currentstroke}{rgb}{0.121569,0.466667,0.705882}%
\pgfsetstrokecolor{currentstroke}%
\pgfsetstrokeopacity{0.301838}%
\pgfsetdash{}{0pt}%
\pgfpathmoveto{\pgfqpoint{1.637645in}{2.671589in}}%
\pgfpathcurveto{\pgfqpoint{1.645881in}{2.671589in}}{\pgfqpoint{1.653781in}{2.674861in}}{\pgfqpoint{1.659605in}{2.680685in}}%
\pgfpathcurveto{\pgfqpoint{1.665429in}{2.686509in}}{\pgfqpoint{1.668702in}{2.694409in}}{\pgfqpoint{1.668702in}{2.702645in}}%
\pgfpathcurveto{\pgfqpoint{1.668702in}{2.710881in}}{\pgfqpoint{1.665429in}{2.718781in}}{\pgfqpoint{1.659605in}{2.724605in}}%
\pgfpathcurveto{\pgfqpoint{1.653781in}{2.730429in}}{\pgfqpoint{1.645881in}{2.733702in}}{\pgfqpoint{1.637645in}{2.733702in}}%
\pgfpathcurveto{\pgfqpoint{1.629409in}{2.733702in}}{\pgfqpoint{1.621509in}{2.730429in}}{\pgfqpoint{1.615685in}{2.724605in}}%
\pgfpathcurveto{\pgfqpoint{1.609861in}{2.718781in}}{\pgfqpoint{1.606589in}{2.710881in}}{\pgfqpoint{1.606589in}{2.702645in}}%
\pgfpathcurveto{\pgfqpoint{1.606589in}{2.694409in}}{\pgfqpoint{1.609861in}{2.686509in}}{\pgfqpoint{1.615685in}{2.680685in}}%
\pgfpathcurveto{\pgfqpoint{1.621509in}{2.674861in}}{\pgfqpoint{1.629409in}{2.671589in}}{\pgfqpoint{1.637645in}{2.671589in}}%
\pgfpathclose%
\pgfusepath{stroke,fill}%
\end{pgfscope}%
\begin{pgfscope}%
\pgfpathrectangle{\pgfqpoint{0.100000in}{0.212622in}}{\pgfqpoint{3.696000in}{3.696000in}}%
\pgfusepath{clip}%
\pgfsetbuttcap%
\pgfsetroundjoin%
\definecolor{currentfill}{rgb}{0.121569,0.466667,0.705882}%
\pgfsetfillcolor{currentfill}%
\pgfsetfillopacity{0.303087}%
\pgfsetlinewidth{1.003750pt}%
\definecolor{currentstroke}{rgb}{0.121569,0.466667,0.705882}%
\pgfsetstrokecolor{currentstroke}%
\pgfsetstrokeopacity{0.303087}%
\pgfsetdash{}{0pt}%
\pgfpathmoveto{\pgfqpoint{1.640557in}{2.671512in}}%
\pgfpathcurveto{\pgfqpoint{1.648793in}{2.671512in}}{\pgfqpoint{1.656693in}{2.674784in}}{\pgfqpoint{1.662517in}{2.680608in}}%
\pgfpathcurveto{\pgfqpoint{1.668341in}{2.686432in}}{\pgfqpoint{1.671613in}{2.694332in}}{\pgfqpoint{1.671613in}{2.702568in}}%
\pgfpathcurveto{\pgfqpoint{1.671613in}{2.710805in}}{\pgfqpoint{1.668341in}{2.718705in}}{\pgfqpoint{1.662517in}{2.724529in}}%
\pgfpathcurveto{\pgfqpoint{1.656693in}{2.730353in}}{\pgfqpoint{1.648793in}{2.733625in}}{\pgfqpoint{1.640557in}{2.733625in}}%
\pgfpathcurveto{\pgfqpoint{1.632320in}{2.733625in}}{\pgfqpoint{1.624420in}{2.730353in}}{\pgfqpoint{1.618596in}{2.724529in}}%
\pgfpathcurveto{\pgfqpoint{1.612772in}{2.718705in}}{\pgfqpoint{1.609500in}{2.710805in}}{\pgfqpoint{1.609500in}{2.702568in}}%
\pgfpathcurveto{\pgfqpoint{1.609500in}{2.694332in}}{\pgfqpoint{1.612772in}{2.686432in}}{\pgfqpoint{1.618596in}{2.680608in}}%
\pgfpathcurveto{\pgfqpoint{1.624420in}{2.674784in}}{\pgfqpoint{1.632320in}{2.671512in}}{\pgfqpoint{1.640557in}{2.671512in}}%
\pgfpathclose%
\pgfusepath{stroke,fill}%
\end{pgfscope}%
\begin{pgfscope}%
\pgfpathrectangle{\pgfqpoint{0.100000in}{0.212622in}}{\pgfqpoint{3.696000in}{3.696000in}}%
\pgfusepath{clip}%
\pgfsetbuttcap%
\pgfsetroundjoin%
\definecolor{currentfill}{rgb}{0.121569,0.466667,0.705882}%
\pgfsetfillcolor{currentfill}%
\pgfsetfillopacity{0.304954}%
\pgfsetlinewidth{1.003750pt}%
\definecolor{currentstroke}{rgb}{0.121569,0.466667,0.705882}%
\pgfsetstrokecolor{currentstroke}%
\pgfsetstrokeopacity{0.304954}%
\pgfsetdash{}{0pt}%
\pgfpathmoveto{\pgfqpoint{1.646549in}{2.671262in}}%
\pgfpathcurveto{\pgfqpoint{1.654785in}{2.671262in}}{\pgfqpoint{1.662686in}{2.674534in}}{\pgfqpoint{1.668509in}{2.680358in}}%
\pgfpathcurveto{\pgfqpoint{1.674333in}{2.686182in}}{\pgfqpoint{1.677606in}{2.694082in}}{\pgfqpoint{1.677606in}{2.702318in}}%
\pgfpathcurveto{\pgfqpoint{1.677606in}{2.710555in}}{\pgfqpoint{1.674333in}{2.718455in}}{\pgfqpoint{1.668509in}{2.724279in}}%
\pgfpathcurveto{\pgfqpoint{1.662686in}{2.730103in}}{\pgfqpoint{1.654785in}{2.733375in}}{\pgfqpoint{1.646549in}{2.733375in}}%
\pgfpathcurveto{\pgfqpoint{1.638313in}{2.733375in}}{\pgfqpoint{1.630413in}{2.730103in}}{\pgfqpoint{1.624589in}{2.724279in}}%
\pgfpathcurveto{\pgfqpoint{1.618765in}{2.718455in}}{\pgfqpoint{1.615493in}{2.710555in}}{\pgfqpoint{1.615493in}{2.702318in}}%
\pgfpathcurveto{\pgfqpoint{1.615493in}{2.694082in}}{\pgfqpoint{1.618765in}{2.686182in}}{\pgfqpoint{1.624589in}{2.680358in}}%
\pgfpathcurveto{\pgfqpoint{1.630413in}{2.674534in}}{\pgfqpoint{1.638313in}{2.671262in}}{\pgfqpoint{1.646549in}{2.671262in}}%
\pgfpathclose%
\pgfusepath{stroke,fill}%
\end{pgfscope}%
\begin{pgfscope}%
\pgfpathrectangle{\pgfqpoint{0.100000in}{0.212622in}}{\pgfqpoint{3.696000in}{3.696000in}}%
\pgfusepath{clip}%
\pgfsetbuttcap%
\pgfsetroundjoin%
\definecolor{currentfill}{rgb}{0.121569,0.466667,0.705882}%
\pgfsetfillcolor{currentfill}%
\pgfsetfillopacity{0.308072}%
\pgfsetlinewidth{1.003750pt}%
\definecolor{currentstroke}{rgb}{0.121569,0.466667,0.705882}%
\pgfsetstrokecolor{currentstroke}%
\pgfsetstrokeopacity{0.308072}%
\pgfsetdash{}{0pt}%
\pgfpathmoveto{\pgfqpoint{1.658076in}{2.671340in}}%
\pgfpathcurveto{\pgfqpoint{1.666312in}{2.671340in}}{\pgfqpoint{1.674212in}{2.674613in}}{\pgfqpoint{1.680036in}{2.680437in}}%
\pgfpathcurveto{\pgfqpoint{1.685860in}{2.686260in}}{\pgfqpoint{1.689132in}{2.694161in}}{\pgfqpoint{1.689132in}{2.702397in}}%
\pgfpathcurveto{\pgfqpoint{1.689132in}{2.710633in}}{\pgfqpoint{1.685860in}{2.718533in}}{\pgfqpoint{1.680036in}{2.724357in}}%
\pgfpathcurveto{\pgfqpoint{1.674212in}{2.730181in}}{\pgfqpoint{1.666312in}{2.733453in}}{\pgfqpoint{1.658076in}{2.733453in}}%
\pgfpathcurveto{\pgfqpoint{1.649840in}{2.733453in}}{\pgfqpoint{1.641940in}{2.730181in}}{\pgfqpoint{1.636116in}{2.724357in}}%
\pgfpathcurveto{\pgfqpoint{1.630292in}{2.718533in}}{\pgfqpoint{1.627019in}{2.710633in}}{\pgfqpoint{1.627019in}{2.702397in}}%
\pgfpathcurveto{\pgfqpoint{1.627019in}{2.694161in}}{\pgfqpoint{1.630292in}{2.686260in}}{\pgfqpoint{1.636116in}{2.680437in}}%
\pgfpathcurveto{\pgfqpoint{1.641940in}{2.674613in}}{\pgfqpoint{1.649840in}{2.671340in}}{\pgfqpoint{1.658076in}{2.671340in}}%
\pgfpathclose%
\pgfusepath{stroke,fill}%
\end{pgfscope}%
\begin{pgfscope}%
\pgfpathrectangle{\pgfqpoint{0.100000in}{0.212622in}}{\pgfqpoint{3.696000in}{3.696000in}}%
\pgfusepath{clip}%
\pgfsetbuttcap%
\pgfsetroundjoin%
\definecolor{currentfill}{rgb}{0.121569,0.466667,0.705882}%
\pgfsetfillcolor{currentfill}%
\pgfsetfillopacity{0.311249}%
\pgfsetlinewidth{1.003750pt}%
\definecolor{currentstroke}{rgb}{0.121569,0.466667,0.705882}%
\pgfsetstrokecolor{currentstroke}%
\pgfsetstrokeopacity{0.311249}%
\pgfsetdash{}{0pt}%
\pgfpathmoveto{\pgfqpoint{1.669099in}{2.671029in}}%
\pgfpathcurveto{\pgfqpoint{1.677335in}{2.671029in}}{\pgfqpoint{1.685235in}{2.674301in}}{\pgfqpoint{1.691059in}{2.680125in}}%
\pgfpathcurveto{\pgfqpoint{1.696883in}{2.685949in}}{\pgfqpoint{1.700155in}{2.693849in}}{\pgfqpoint{1.700155in}{2.702086in}}%
\pgfpathcurveto{\pgfqpoint{1.700155in}{2.710322in}}{\pgfqpoint{1.696883in}{2.718222in}}{\pgfqpoint{1.691059in}{2.724046in}}%
\pgfpathcurveto{\pgfqpoint{1.685235in}{2.729870in}}{\pgfqpoint{1.677335in}{2.733142in}}{\pgfqpoint{1.669099in}{2.733142in}}%
\pgfpathcurveto{\pgfqpoint{1.660863in}{2.733142in}}{\pgfqpoint{1.652963in}{2.729870in}}{\pgfqpoint{1.647139in}{2.724046in}}%
\pgfpathcurveto{\pgfqpoint{1.641315in}{2.718222in}}{\pgfqpoint{1.638042in}{2.710322in}}{\pgfqpoint{1.638042in}{2.702086in}}%
\pgfpathcurveto{\pgfqpoint{1.638042in}{2.693849in}}{\pgfqpoint{1.641315in}{2.685949in}}{\pgfqpoint{1.647139in}{2.680125in}}%
\pgfpathcurveto{\pgfqpoint{1.652963in}{2.674301in}}{\pgfqpoint{1.660863in}{2.671029in}}{\pgfqpoint{1.669099in}{2.671029in}}%
\pgfpathclose%
\pgfusepath{stroke,fill}%
\end{pgfscope}%
\begin{pgfscope}%
\pgfpathrectangle{\pgfqpoint{0.100000in}{0.212622in}}{\pgfqpoint{3.696000in}{3.696000in}}%
\pgfusepath{clip}%
\pgfsetbuttcap%
\pgfsetroundjoin%
\definecolor{currentfill}{rgb}{0.121569,0.466667,0.705882}%
\pgfsetfillcolor{currentfill}%
\pgfsetfillopacity{0.314347}%
\pgfsetlinewidth{1.003750pt}%
\definecolor{currentstroke}{rgb}{0.121569,0.466667,0.705882}%
\pgfsetstrokecolor{currentstroke}%
\pgfsetstrokeopacity{0.314347}%
\pgfsetdash{}{0pt}%
\pgfpathmoveto{\pgfqpoint{1.679984in}{2.670782in}}%
\pgfpathcurveto{\pgfqpoint{1.688221in}{2.670782in}}{\pgfqpoint{1.696121in}{2.674054in}}{\pgfqpoint{1.701945in}{2.679878in}}%
\pgfpathcurveto{\pgfqpoint{1.707769in}{2.685702in}}{\pgfqpoint{1.711041in}{2.693602in}}{\pgfqpoint{1.711041in}{2.701838in}}%
\pgfpathcurveto{\pgfqpoint{1.711041in}{2.710075in}}{\pgfqpoint{1.707769in}{2.717975in}}{\pgfqpoint{1.701945in}{2.723799in}}%
\pgfpathcurveto{\pgfqpoint{1.696121in}{2.729623in}}{\pgfqpoint{1.688221in}{2.732895in}}{\pgfqpoint{1.679984in}{2.732895in}}%
\pgfpathcurveto{\pgfqpoint{1.671748in}{2.732895in}}{\pgfqpoint{1.663848in}{2.729623in}}{\pgfqpoint{1.658024in}{2.723799in}}%
\pgfpathcurveto{\pgfqpoint{1.652200in}{2.717975in}}{\pgfqpoint{1.648928in}{2.710075in}}{\pgfqpoint{1.648928in}{2.701838in}}%
\pgfpathcurveto{\pgfqpoint{1.648928in}{2.693602in}}{\pgfqpoint{1.652200in}{2.685702in}}{\pgfqpoint{1.658024in}{2.679878in}}%
\pgfpathcurveto{\pgfqpoint{1.663848in}{2.674054in}}{\pgfqpoint{1.671748in}{2.670782in}}{\pgfqpoint{1.679984in}{2.670782in}}%
\pgfpathclose%
\pgfusepath{stroke,fill}%
\end{pgfscope}%
\begin{pgfscope}%
\pgfpathrectangle{\pgfqpoint{0.100000in}{0.212622in}}{\pgfqpoint{3.696000in}{3.696000in}}%
\pgfusepath{clip}%
\pgfsetbuttcap%
\pgfsetroundjoin%
\definecolor{currentfill}{rgb}{0.121569,0.466667,0.705882}%
\pgfsetfillcolor{currentfill}%
\pgfsetfillopacity{0.320133}%
\pgfsetlinewidth{1.003750pt}%
\definecolor{currentstroke}{rgb}{0.121569,0.466667,0.705882}%
\pgfsetstrokecolor{currentstroke}%
\pgfsetstrokeopacity{0.320133}%
\pgfsetdash{}{0pt}%
\pgfpathmoveto{\pgfqpoint{1.699465in}{2.670033in}}%
\pgfpathcurveto{\pgfqpoint{1.707702in}{2.670033in}}{\pgfqpoint{1.715602in}{2.673306in}}{\pgfqpoint{1.721426in}{2.679129in}}%
\pgfpathcurveto{\pgfqpoint{1.727249in}{2.684953in}}{\pgfqpoint{1.730522in}{2.692853in}}{\pgfqpoint{1.730522in}{2.701090in}}%
\pgfpathcurveto{\pgfqpoint{1.730522in}{2.709326in}}{\pgfqpoint{1.727249in}{2.717226in}}{\pgfqpoint{1.721426in}{2.723050in}}%
\pgfpathcurveto{\pgfqpoint{1.715602in}{2.728874in}}{\pgfqpoint{1.707702in}{2.732146in}}{\pgfqpoint{1.699465in}{2.732146in}}%
\pgfpathcurveto{\pgfqpoint{1.691229in}{2.732146in}}{\pgfqpoint{1.683329in}{2.728874in}}{\pgfqpoint{1.677505in}{2.723050in}}%
\pgfpathcurveto{\pgfqpoint{1.671681in}{2.717226in}}{\pgfqpoint{1.668409in}{2.709326in}}{\pgfqpoint{1.668409in}{2.701090in}}%
\pgfpathcurveto{\pgfqpoint{1.668409in}{2.692853in}}{\pgfqpoint{1.671681in}{2.684953in}}{\pgfqpoint{1.677505in}{2.679129in}}%
\pgfpathcurveto{\pgfqpoint{1.683329in}{2.673306in}}{\pgfqpoint{1.691229in}{2.670033in}}{\pgfqpoint{1.699465in}{2.670033in}}%
\pgfpathclose%
\pgfusepath{stroke,fill}%
\end{pgfscope}%
\begin{pgfscope}%
\pgfpathrectangle{\pgfqpoint{0.100000in}{0.212622in}}{\pgfqpoint{3.696000in}{3.696000in}}%
\pgfusepath{clip}%
\pgfsetbuttcap%
\pgfsetroundjoin%
\definecolor{currentfill}{rgb}{0.121569,0.466667,0.705882}%
\pgfsetfillcolor{currentfill}%
\pgfsetfillopacity{0.326249}%
\pgfsetlinewidth{1.003750pt}%
\definecolor{currentstroke}{rgb}{0.121569,0.466667,0.705882}%
\pgfsetstrokecolor{currentstroke}%
\pgfsetstrokeopacity{0.326249}%
\pgfsetdash{}{0pt}%
\pgfpathmoveto{\pgfqpoint{1.717824in}{2.668403in}}%
\pgfpathcurveto{\pgfqpoint{1.726060in}{2.668403in}}{\pgfqpoint{1.733960in}{2.671675in}}{\pgfqpoint{1.739784in}{2.677499in}}%
\pgfpathcurveto{\pgfqpoint{1.745608in}{2.683323in}}{\pgfqpoint{1.748880in}{2.691223in}}{\pgfqpoint{1.748880in}{2.699459in}}%
\pgfpathcurveto{\pgfqpoint{1.748880in}{2.707695in}}{\pgfqpoint{1.745608in}{2.715595in}}{\pgfqpoint{1.739784in}{2.721419in}}%
\pgfpathcurveto{\pgfqpoint{1.733960in}{2.727243in}}{\pgfqpoint{1.726060in}{2.730516in}}{\pgfqpoint{1.717824in}{2.730516in}}%
\pgfpathcurveto{\pgfqpoint{1.709587in}{2.730516in}}{\pgfqpoint{1.701687in}{2.727243in}}{\pgfqpoint{1.695863in}{2.721419in}}%
\pgfpathcurveto{\pgfqpoint{1.690039in}{2.715595in}}{\pgfqpoint{1.686767in}{2.707695in}}{\pgfqpoint{1.686767in}{2.699459in}}%
\pgfpathcurveto{\pgfqpoint{1.686767in}{2.691223in}}{\pgfqpoint{1.690039in}{2.683323in}}{\pgfqpoint{1.695863in}{2.677499in}}%
\pgfpathcurveto{\pgfqpoint{1.701687in}{2.671675in}}{\pgfqpoint{1.709587in}{2.668403in}}{\pgfqpoint{1.717824in}{2.668403in}}%
\pgfpathclose%
\pgfusepath{stroke,fill}%
\end{pgfscope}%
\begin{pgfscope}%
\pgfpathrectangle{\pgfqpoint{0.100000in}{0.212622in}}{\pgfqpoint{3.696000in}{3.696000in}}%
\pgfusepath{clip}%
\pgfsetbuttcap%
\pgfsetroundjoin%
\definecolor{currentfill}{rgb}{0.121569,0.466667,0.705882}%
\pgfsetfillcolor{currentfill}%
\pgfsetfillopacity{0.331784}%
\pgfsetlinewidth{1.003750pt}%
\definecolor{currentstroke}{rgb}{0.121569,0.466667,0.705882}%
\pgfsetstrokecolor{currentstroke}%
\pgfsetstrokeopacity{0.331784}%
\pgfsetdash{}{0pt}%
\pgfpathmoveto{\pgfqpoint{1.736833in}{2.666573in}}%
\pgfpathcurveto{\pgfqpoint{1.745070in}{2.666573in}}{\pgfqpoint{1.752970in}{2.669845in}}{\pgfqpoint{1.758794in}{2.675669in}}%
\pgfpathcurveto{\pgfqpoint{1.764617in}{2.681493in}}{\pgfqpoint{1.767890in}{2.689393in}}{\pgfqpoint{1.767890in}{2.697629in}}%
\pgfpathcurveto{\pgfqpoint{1.767890in}{2.705866in}}{\pgfqpoint{1.764617in}{2.713766in}}{\pgfqpoint{1.758794in}{2.719590in}}%
\pgfpathcurveto{\pgfqpoint{1.752970in}{2.725414in}}{\pgfqpoint{1.745070in}{2.728686in}}{\pgfqpoint{1.736833in}{2.728686in}}%
\pgfpathcurveto{\pgfqpoint{1.728597in}{2.728686in}}{\pgfqpoint{1.720697in}{2.725414in}}{\pgfqpoint{1.714873in}{2.719590in}}%
\pgfpathcurveto{\pgfqpoint{1.709049in}{2.713766in}}{\pgfqpoint{1.705777in}{2.705866in}}{\pgfqpoint{1.705777in}{2.697629in}}%
\pgfpathcurveto{\pgfqpoint{1.705777in}{2.689393in}}{\pgfqpoint{1.709049in}{2.681493in}}{\pgfqpoint{1.714873in}{2.675669in}}%
\pgfpathcurveto{\pgfqpoint{1.720697in}{2.669845in}}{\pgfqpoint{1.728597in}{2.666573in}}{\pgfqpoint{1.736833in}{2.666573in}}%
\pgfpathclose%
\pgfusepath{stroke,fill}%
\end{pgfscope}%
\begin{pgfscope}%
\pgfpathrectangle{\pgfqpoint{0.100000in}{0.212622in}}{\pgfqpoint{3.696000in}{3.696000in}}%
\pgfusepath{clip}%
\pgfsetbuttcap%
\pgfsetroundjoin%
\definecolor{currentfill}{rgb}{0.121569,0.466667,0.705882}%
\pgfsetfillcolor{currentfill}%
\pgfsetfillopacity{0.337943}%
\pgfsetlinewidth{1.003750pt}%
\definecolor{currentstroke}{rgb}{0.121569,0.466667,0.705882}%
\pgfsetstrokecolor{currentstroke}%
\pgfsetstrokeopacity{0.337943}%
\pgfsetdash{}{0pt}%
\pgfpathmoveto{\pgfqpoint{1.754399in}{2.664556in}}%
\pgfpathcurveto{\pgfqpoint{1.762635in}{2.664556in}}{\pgfqpoint{1.770535in}{2.667829in}}{\pgfqpoint{1.776359in}{2.673652in}}%
\pgfpathcurveto{\pgfqpoint{1.782183in}{2.679476in}}{\pgfqpoint{1.785456in}{2.687376in}}{\pgfqpoint{1.785456in}{2.695613in}}%
\pgfpathcurveto{\pgfqpoint{1.785456in}{2.703849in}}{\pgfqpoint{1.782183in}{2.711749in}}{\pgfqpoint{1.776359in}{2.717573in}}%
\pgfpathcurveto{\pgfqpoint{1.770535in}{2.723397in}}{\pgfqpoint{1.762635in}{2.726669in}}{\pgfqpoint{1.754399in}{2.726669in}}%
\pgfpathcurveto{\pgfqpoint{1.746163in}{2.726669in}}{\pgfqpoint{1.738263in}{2.723397in}}{\pgfqpoint{1.732439in}{2.717573in}}%
\pgfpathcurveto{\pgfqpoint{1.726615in}{2.711749in}}{\pgfqpoint{1.723343in}{2.703849in}}{\pgfqpoint{1.723343in}{2.695613in}}%
\pgfpathcurveto{\pgfqpoint{1.723343in}{2.687376in}}{\pgfqpoint{1.726615in}{2.679476in}}{\pgfqpoint{1.732439in}{2.673652in}}%
\pgfpathcurveto{\pgfqpoint{1.738263in}{2.667829in}}{\pgfqpoint{1.746163in}{2.664556in}}{\pgfqpoint{1.754399in}{2.664556in}}%
\pgfpathclose%
\pgfusepath{stroke,fill}%
\end{pgfscope}%
\begin{pgfscope}%
\pgfpathrectangle{\pgfqpoint{0.100000in}{0.212622in}}{\pgfqpoint{3.696000in}{3.696000in}}%
\pgfusepath{clip}%
\pgfsetbuttcap%
\pgfsetroundjoin%
\definecolor{currentfill}{rgb}{0.121569,0.466667,0.705882}%
\pgfsetfillcolor{currentfill}%
\pgfsetfillopacity{0.343718}%
\pgfsetlinewidth{1.003750pt}%
\definecolor{currentstroke}{rgb}{0.121569,0.466667,0.705882}%
\pgfsetstrokecolor{currentstroke}%
\pgfsetstrokeopacity{0.343718}%
\pgfsetdash{}{0pt}%
\pgfpathmoveto{\pgfqpoint{1.772449in}{2.662859in}}%
\pgfpathcurveto{\pgfqpoint{1.780686in}{2.662859in}}{\pgfqpoint{1.788586in}{2.666131in}}{\pgfqpoint{1.794410in}{2.671955in}}%
\pgfpathcurveto{\pgfqpoint{1.800233in}{2.677779in}}{\pgfqpoint{1.803506in}{2.685679in}}{\pgfqpoint{1.803506in}{2.693916in}}%
\pgfpathcurveto{\pgfqpoint{1.803506in}{2.702152in}}{\pgfqpoint{1.800233in}{2.710052in}}{\pgfqpoint{1.794410in}{2.715876in}}%
\pgfpathcurveto{\pgfqpoint{1.788586in}{2.721700in}}{\pgfqpoint{1.780686in}{2.724972in}}{\pgfqpoint{1.772449in}{2.724972in}}%
\pgfpathcurveto{\pgfqpoint{1.764213in}{2.724972in}}{\pgfqpoint{1.756313in}{2.721700in}}{\pgfqpoint{1.750489in}{2.715876in}}%
\pgfpathcurveto{\pgfqpoint{1.744665in}{2.710052in}}{\pgfqpoint{1.741393in}{2.702152in}}{\pgfqpoint{1.741393in}{2.693916in}}%
\pgfpathcurveto{\pgfqpoint{1.741393in}{2.685679in}}{\pgfqpoint{1.744665in}{2.677779in}}{\pgfqpoint{1.750489in}{2.671955in}}%
\pgfpathcurveto{\pgfqpoint{1.756313in}{2.666131in}}{\pgfqpoint{1.764213in}{2.662859in}}{\pgfqpoint{1.772449in}{2.662859in}}%
\pgfpathclose%
\pgfusepath{stroke,fill}%
\end{pgfscope}%
\begin{pgfscope}%
\pgfpathrectangle{\pgfqpoint{0.100000in}{0.212622in}}{\pgfqpoint{3.696000in}{3.696000in}}%
\pgfusepath{clip}%
\pgfsetbuttcap%
\pgfsetroundjoin%
\definecolor{currentfill}{rgb}{0.121569,0.466667,0.705882}%
\pgfsetfillcolor{currentfill}%
\pgfsetfillopacity{0.348951}%
\pgfsetlinewidth{1.003750pt}%
\definecolor{currentstroke}{rgb}{0.121569,0.466667,0.705882}%
\pgfsetstrokecolor{currentstroke}%
\pgfsetstrokeopacity{0.348951}%
\pgfsetdash{}{0pt}%
\pgfpathmoveto{\pgfqpoint{1.791511in}{2.662476in}}%
\pgfpathcurveto{\pgfqpoint{1.799747in}{2.662476in}}{\pgfqpoint{1.807647in}{2.665748in}}{\pgfqpoint{1.813471in}{2.671572in}}%
\pgfpathcurveto{\pgfqpoint{1.819295in}{2.677396in}}{\pgfqpoint{1.822567in}{2.685296in}}{\pgfqpoint{1.822567in}{2.693533in}}%
\pgfpathcurveto{\pgfqpoint{1.822567in}{2.701769in}}{\pgfqpoint{1.819295in}{2.709669in}}{\pgfqpoint{1.813471in}{2.715493in}}%
\pgfpathcurveto{\pgfqpoint{1.807647in}{2.721317in}}{\pgfqpoint{1.799747in}{2.724589in}}{\pgfqpoint{1.791511in}{2.724589in}}%
\pgfpathcurveto{\pgfqpoint{1.783275in}{2.724589in}}{\pgfqpoint{1.775374in}{2.721317in}}{\pgfqpoint{1.769551in}{2.715493in}}%
\pgfpathcurveto{\pgfqpoint{1.763727in}{2.709669in}}{\pgfqpoint{1.760454in}{2.701769in}}{\pgfqpoint{1.760454in}{2.693533in}}%
\pgfpathcurveto{\pgfqpoint{1.760454in}{2.685296in}}{\pgfqpoint{1.763727in}{2.677396in}}{\pgfqpoint{1.769551in}{2.671572in}}%
\pgfpathcurveto{\pgfqpoint{1.775374in}{2.665748in}}{\pgfqpoint{1.783275in}{2.662476in}}{\pgfqpoint{1.791511in}{2.662476in}}%
\pgfpathclose%
\pgfusepath{stroke,fill}%
\end{pgfscope}%
\begin{pgfscope}%
\pgfpathrectangle{\pgfqpoint{0.100000in}{0.212622in}}{\pgfqpoint{3.696000in}{3.696000in}}%
\pgfusepath{clip}%
\pgfsetbuttcap%
\pgfsetroundjoin%
\definecolor{currentfill}{rgb}{0.121569,0.466667,0.705882}%
\pgfsetfillcolor{currentfill}%
\pgfsetfillopacity{0.354501}%
\pgfsetlinewidth{1.003750pt}%
\definecolor{currentstroke}{rgb}{0.121569,0.466667,0.705882}%
\pgfsetstrokecolor{currentstroke}%
\pgfsetstrokeopacity{0.354501}%
\pgfsetdash{}{0pt}%
\pgfpathmoveto{\pgfqpoint{1.808968in}{2.659760in}}%
\pgfpathcurveto{\pgfqpoint{1.817204in}{2.659760in}}{\pgfqpoint{1.825104in}{2.663033in}}{\pgfqpoint{1.830928in}{2.668857in}}%
\pgfpathcurveto{\pgfqpoint{1.836752in}{2.674681in}}{\pgfqpoint{1.840025in}{2.682581in}}{\pgfqpoint{1.840025in}{2.690817in}}%
\pgfpathcurveto{\pgfqpoint{1.840025in}{2.699053in}}{\pgfqpoint{1.836752in}{2.706953in}}{\pgfqpoint{1.830928in}{2.712777in}}%
\pgfpathcurveto{\pgfqpoint{1.825104in}{2.718601in}}{\pgfqpoint{1.817204in}{2.721873in}}{\pgfqpoint{1.808968in}{2.721873in}}%
\pgfpathcurveto{\pgfqpoint{1.800732in}{2.721873in}}{\pgfqpoint{1.792832in}{2.718601in}}{\pgfqpoint{1.787008in}{2.712777in}}%
\pgfpathcurveto{\pgfqpoint{1.781184in}{2.706953in}}{\pgfqpoint{1.777912in}{2.699053in}}{\pgfqpoint{1.777912in}{2.690817in}}%
\pgfpathcurveto{\pgfqpoint{1.777912in}{2.682581in}}{\pgfqpoint{1.781184in}{2.674681in}}{\pgfqpoint{1.787008in}{2.668857in}}%
\pgfpathcurveto{\pgfqpoint{1.792832in}{2.663033in}}{\pgfqpoint{1.800732in}{2.659760in}}{\pgfqpoint{1.808968in}{2.659760in}}%
\pgfpathclose%
\pgfusepath{stroke,fill}%
\end{pgfscope}%
\begin{pgfscope}%
\pgfpathrectangle{\pgfqpoint{0.100000in}{0.212622in}}{\pgfqpoint{3.696000in}{3.696000in}}%
\pgfusepath{clip}%
\pgfsetbuttcap%
\pgfsetroundjoin%
\definecolor{currentfill}{rgb}{0.121569,0.466667,0.705882}%
\pgfsetfillcolor{currentfill}%
\pgfsetfillopacity{0.359244}%
\pgfsetlinewidth{1.003750pt}%
\definecolor{currentstroke}{rgb}{0.121569,0.466667,0.705882}%
\pgfsetstrokecolor{currentstroke}%
\pgfsetstrokeopacity{0.359244}%
\pgfsetdash{}{0pt}%
\pgfpathmoveto{\pgfqpoint{1.827371in}{2.656900in}}%
\pgfpathcurveto{\pgfqpoint{1.835607in}{2.656900in}}{\pgfqpoint{1.843507in}{2.660172in}}{\pgfqpoint{1.849331in}{2.665996in}}%
\pgfpathcurveto{\pgfqpoint{1.855155in}{2.671820in}}{\pgfqpoint{1.858427in}{2.679720in}}{\pgfqpoint{1.858427in}{2.687956in}}%
\pgfpathcurveto{\pgfqpoint{1.858427in}{2.696192in}}{\pgfqpoint{1.855155in}{2.704092in}}{\pgfqpoint{1.849331in}{2.709916in}}%
\pgfpathcurveto{\pgfqpoint{1.843507in}{2.715740in}}{\pgfqpoint{1.835607in}{2.719013in}}{\pgfqpoint{1.827371in}{2.719013in}}%
\pgfpathcurveto{\pgfqpoint{1.819134in}{2.719013in}}{\pgfqpoint{1.811234in}{2.715740in}}{\pgfqpoint{1.805410in}{2.709916in}}%
\pgfpathcurveto{\pgfqpoint{1.799587in}{2.704092in}}{\pgfqpoint{1.796314in}{2.696192in}}{\pgfqpoint{1.796314in}{2.687956in}}%
\pgfpathcurveto{\pgfqpoint{1.796314in}{2.679720in}}{\pgfqpoint{1.799587in}{2.671820in}}{\pgfqpoint{1.805410in}{2.665996in}}%
\pgfpathcurveto{\pgfqpoint{1.811234in}{2.660172in}}{\pgfqpoint{1.819134in}{2.656900in}}{\pgfqpoint{1.827371in}{2.656900in}}%
\pgfpathclose%
\pgfusepath{stroke,fill}%
\end{pgfscope}%
\begin{pgfscope}%
\pgfpathrectangle{\pgfqpoint{0.100000in}{0.212622in}}{\pgfqpoint{3.696000in}{3.696000in}}%
\pgfusepath{clip}%
\pgfsetbuttcap%
\pgfsetroundjoin%
\definecolor{currentfill}{rgb}{0.121569,0.466667,0.705882}%
\pgfsetfillcolor{currentfill}%
\pgfsetfillopacity{0.363953}%
\pgfsetlinewidth{1.003750pt}%
\definecolor{currentstroke}{rgb}{0.121569,0.466667,0.705882}%
\pgfsetstrokecolor{currentstroke}%
\pgfsetstrokeopacity{0.363953}%
\pgfsetdash{}{0pt}%
\pgfpathmoveto{\pgfqpoint{1.845465in}{2.654574in}}%
\pgfpathcurveto{\pgfqpoint{1.853702in}{2.654574in}}{\pgfqpoint{1.861602in}{2.657846in}}{\pgfqpoint{1.867425in}{2.663670in}}%
\pgfpathcurveto{\pgfqpoint{1.873249in}{2.669494in}}{\pgfqpoint{1.876522in}{2.677394in}}{\pgfqpoint{1.876522in}{2.685631in}}%
\pgfpathcurveto{\pgfqpoint{1.876522in}{2.693867in}}{\pgfqpoint{1.873249in}{2.701767in}}{\pgfqpoint{1.867425in}{2.707591in}}%
\pgfpathcurveto{\pgfqpoint{1.861602in}{2.713415in}}{\pgfqpoint{1.853702in}{2.716687in}}{\pgfqpoint{1.845465in}{2.716687in}}%
\pgfpathcurveto{\pgfqpoint{1.837229in}{2.716687in}}{\pgfqpoint{1.829329in}{2.713415in}}{\pgfqpoint{1.823505in}{2.707591in}}%
\pgfpathcurveto{\pgfqpoint{1.817681in}{2.701767in}}{\pgfqpoint{1.814409in}{2.693867in}}{\pgfqpoint{1.814409in}{2.685631in}}%
\pgfpathcurveto{\pgfqpoint{1.814409in}{2.677394in}}{\pgfqpoint{1.817681in}{2.669494in}}{\pgfqpoint{1.823505in}{2.663670in}}%
\pgfpathcurveto{\pgfqpoint{1.829329in}{2.657846in}}{\pgfqpoint{1.837229in}{2.654574in}}{\pgfqpoint{1.845465in}{2.654574in}}%
\pgfpathclose%
\pgfusepath{stroke,fill}%
\end{pgfscope}%
\begin{pgfscope}%
\pgfpathrectangle{\pgfqpoint{0.100000in}{0.212622in}}{\pgfqpoint{3.696000in}{3.696000in}}%
\pgfusepath{clip}%
\pgfsetbuttcap%
\pgfsetroundjoin%
\definecolor{currentfill}{rgb}{0.121569,0.466667,0.705882}%
\pgfsetfillcolor{currentfill}%
\pgfsetfillopacity{0.367265}%
\pgfsetlinewidth{1.003750pt}%
\definecolor{currentstroke}{rgb}{0.121569,0.466667,0.705882}%
\pgfsetstrokecolor{currentstroke}%
\pgfsetstrokeopacity{0.367265}%
\pgfsetdash{}{0pt}%
\pgfpathmoveto{\pgfqpoint{1.863816in}{2.648747in}}%
\pgfpathcurveto{\pgfqpoint{1.872052in}{2.648747in}}{\pgfqpoint{1.879952in}{2.652020in}}{\pgfqpoint{1.885776in}{2.657843in}}%
\pgfpathcurveto{\pgfqpoint{1.891600in}{2.663667in}}{\pgfqpoint{1.894872in}{2.671567in}}{\pgfqpoint{1.894872in}{2.679804in}}%
\pgfpathcurveto{\pgfqpoint{1.894872in}{2.688040in}}{\pgfqpoint{1.891600in}{2.695940in}}{\pgfqpoint{1.885776in}{2.701764in}}%
\pgfpathcurveto{\pgfqpoint{1.879952in}{2.707588in}}{\pgfqpoint{1.872052in}{2.710860in}}{\pgfqpoint{1.863816in}{2.710860in}}%
\pgfpathcurveto{\pgfqpoint{1.855579in}{2.710860in}}{\pgfqpoint{1.847679in}{2.707588in}}{\pgfqpoint{1.841855in}{2.701764in}}%
\pgfpathcurveto{\pgfqpoint{1.836031in}{2.695940in}}{\pgfqpoint{1.832759in}{2.688040in}}{\pgfqpoint{1.832759in}{2.679804in}}%
\pgfpathcurveto{\pgfqpoint{1.832759in}{2.671567in}}{\pgfqpoint{1.836031in}{2.663667in}}{\pgfqpoint{1.841855in}{2.657843in}}%
\pgfpathcurveto{\pgfqpoint{1.847679in}{2.652020in}}{\pgfqpoint{1.855579in}{2.648747in}}{\pgfqpoint{1.863816in}{2.648747in}}%
\pgfpathclose%
\pgfusepath{stroke,fill}%
\end{pgfscope}%
\begin{pgfscope}%
\pgfpathrectangle{\pgfqpoint{0.100000in}{0.212622in}}{\pgfqpoint{3.696000in}{3.696000in}}%
\pgfusepath{clip}%
\pgfsetbuttcap%
\pgfsetroundjoin%
\definecolor{currentfill}{rgb}{0.121569,0.466667,0.705882}%
\pgfsetfillcolor{currentfill}%
\pgfsetfillopacity{0.371695}%
\pgfsetlinewidth{1.003750pt}%
\definecolor{currentstroke}{rgb}{0.121569,0.466667,0.705882}%
\pgfsetstrokecolor{currentstroke}%
\pgfsetstrokeopacity{0.371695}%
\pgfsetdash{}{0pt}%
\pgfpathmoveto{\pgfqpoint{1.900244in}{2.641773in}}%
\pgfpathcurveto{\pgfqpoint{1.908480in}{2.641773in}}{\pgfqpoint{1.916380in}{2.645046in}}{\pgfqpoint{1.922204in}{2.650870in}}%
\pgfpathcurveto{\pgfqpoint{1.928028in}{2.656694in}}{\pgfqpoint{1.931300in}{2.664594in}}{\pgfqpoint{1.931300in}{2.672830in}}%
\pgfpathcurveto{\pgfqpoint{1.931300in}{2.681066in}}{\pgfqpoint{1.928028in}{2.688966in}}{\pgfqpoint{1.922204in}{2.694790in}}%
\pgfpathcurveto{\pgfqpoint{1.916380in}{2.700614in}}{\pgfqpoint{1.908480in}{2.703886in}}{\pgfqpoint{1.900244in}{2.703886in}}%
\pgfpathcurveto{\pgfqpoint{1.892007in}{2.703886in}}{\pgfqpoint{1.884107in}{2.700614in}}{\pgfqpoint{1.878284in}{2.694790in}}%
\pgfpathcurveto{\pgfqpoint{1.872460in}{2.688966in}}{\pgfqpoint{1.869187in}{2.681066in}}{\pgfqpoint{1.869187in}{2.672830in}}%
\pgfpathcurveto{\pgfqpoint{1.869187in}{2.664594in}}{\pgfqpoint{1.872460in}{2.656694in}}{\pgfqpoint{1.878284in}{2.650870in}}%
\pgfpathcurveto{\pgfqpoint{1.884107in}{2.645046in}}{\pgfqpoint{1.892007in}{2.641773in}}{\pgfqpoint{1.900244in}{2.641773in}}%
\pgfpathclose%
\pgfusepath{stroke,fill}%
\end{pgfscope}%
\begin{pgfscope}%
\pgfpathrectangle{\pgfqpoint{0.100000in}{0.212622in}}{\pgfqpoint{3.696000in}{3.696000in}}%
\pgfusepath{clip}%
\pgfsetbuttcap%
\pgfsetroundjoin%
\definecolor{currentfill}{rgb}{0.121569,0.466667,0.705882}%
\pgfsetfillcolor{currentfill}%
\pgfsetfillopacity{0.380293}%
\pgfsetlinewidth{1.003750pt}%
\definecolor{currentstroke}{rgb}{0.121569,0.466667,0.705882}%
\pgfsetstrokecolor{currentstroke}%
\pgfsetstrokeopacity{0.380293}%
\pgfsetdash{}{0pt}%
\pgfpathmoveto{\pgfqpoint{1.933837in}{2.649727in}}%
\pgfpathcurveto{\pgfqpoint{1.942073in}{2.649727in}}{\pgfqpoint{1.949973in}{2.652999in}}{\pgfqpoint{1.955797in}{2.658823in}}%
\pgfpathcurveto{\pgfqpoint{1.961621in}{2.664647in}}{\pgfqpoint{1.964894in}{2.672547in}}{\pgfqpoint{1.964894in}{2.680783in}}%
\pgfpathcurveto{\pgfqpoint{1.964894in}{2.689019in}}{\pgfqpoint{1.961621in}{2.696919in}}{\pgfqpoint{1.955797in}{2.702743in}}%
\pgfpathcurveto{\pgfqpoint{1.949973in}{2.708567in}}{\pgfqpoint{1.942073in}{2.711840in}}{\pgfqpoint{1.933837in}{2.711840in}}%
\pgfpathcurveto{\pgfqpoint{1.925601in}{2.711840in}}{\pgfqpoint{1.917701in}{2.708567in}}{\pgfqpoint{1.911877in}{2.702743in}}%
\pgfpathcurveto{\pgfqpoint{1.906053in}{2.696919in}}{\pgfqpoint{1.902781in}{2.689019in}}{\pgfqpoint{1.902781in}{2.680783in}}%
\pgfpathcurveto{\pgfqpoint{1.902781in}{2.672547in}}{\pgfqpoint{1.906053in}{2.664647in}}{\pgfqpoint{1.911877in}{2.658823in}}%
\pgfpathcurveto{\pgfqpoint{1.917701in}{2.652999in}}{\pgfqpoint{1.925601in}{2.649727in}}{\pgfqpoint{1.933837in}{2.649727in}}%
\pgfpathclose%
\pgfusepath{stroke,fill}%
\end{pgfscope}%
\begin{pgfscope}%
\pgfpathrectangle{\pgfqpoint{0.100000in}{0.212622in}}{\pgfqpoint{3.696000in}{3.696000in}}%
\pgfusepath{clip}%
\pgfsetbuttcap%
\pgfsetroundjoin%
\definecolor{currentfill}{rgb}{0.121569,0.466667,0.705882}%
\pgfsetfillcolor{currentfill}%
\pgfsetfillopacity{0.390492}%
\pgfsetlinewidth{1.003750pt}%
\definecolor{currentstroke}{rgb}{0.121569,0.466667,0.705882}%
\pgfsetstrokecolor{currentstroke}%
\pgfsetstrokeopacity{0.390492}%
\pgfsetdash{}{0pt}%
\pgfpathmoveto{\pgfqpoint{1.963196in}{2.659955in}}%
\pgfpathcurveto{\pgfqpoint{1.971432in}{2.659955in}}{\pgfqpoint{1.979332in}{2.663228in}}{\pgfqpoint{1.985156in}{2.669052in}}%
\pgfpathcurveto{\pgfqpoint{1.990980in}{2.674876in}}{\pgfqpoint{1.994252in}{2.682776in}}{\pgfqpoint{1.994252in}{2.691012in}}%
\pgfpathcurveto{\pgfqpoint{1.994252in}{2.699248in}}{\pgfqpoint{1.990980in}{2.707148in}}{\pgfqpoint{1.985156in}{2.712972in}}%
\pgfpathcurveto{\pgfqpoint{1.979332in}{2.718796in}}{\pgfqpoint{1.971432in}{2.722068in}}{\pgfqpoint{1.963196in}{2.722068in}}%
\pgfpathcurveto{\pgfqpoint{1.954960in}{2.722068in}}{\pgfqpoint{1.947060in}{2.718796in}}{\pgfqpoint{1.941236in}{2.712972in}}%
\pgfpathcurveto{\pgfqpoint{1.935412in}{2.707148in}}{\pgfqpoint{1.932139in}{2.699248in}}{\pgfqpoint{1.932139in}{2.691012in}}%
\pgfpathcurveto{\pgfqpoint{1.932139in}{2.682776in}}{\pgfqpoint{1.935412in}{2.674876in}}{\pgfqpoint{1.941236in}{2.669052in}}%
\pgfpathcurveto{\pgfqpoint{1.947060in}{2.663228in}}{\pgfqpoint{1.954960in}{2.659955in}}{\pgfqpoint{1.963196in}{2.659955in}}%
\pgfpathclose%
\pgfusepath{stroke,fill}%
\end{pgfscope}%
\begin{pgfscope}%
\pgfpathrectangle{\pgfqpoint{0.100000in}{0.212622in}}{\pgfqpoint{3.696000in}{3.696000in}}%
\pgfusepath{clip}%
\pgfsetbuttcap%
\pgfsetroundjoin%
\definecolor{currentfill}{rgb}{0.121569,0.466667,0.705882}%
\pgfsetfillcolor{currentfill}%
\pgfsetfillopacity{0.400213}%
\pgfsetlinewidth{1.003750pt}%
\definecolor{currentstroke}{rgb}{0.121569,0.466667,0.705882}%
\pgfsetstrokecolor{currentstroke}%
\pgfsetstrokeopacity{0.400213}%
\pgfsetdash{}{0pt}%
\pgfpathmoveto{\pgfqpoint{1.991343in}{2.666015in}}%
\pgfpathcurveto{\pgfqpoint{1.999579in}{2.666015in}}{\pgfqpoint{2.007479in}{2.669287in}}{\pgfqpoint{2.013303in}{2.675111in}}%
\pgfpathcurveto{\pgfqpoint{2.019127in}{2.680935in}}{\pgfqpoint{2.022399in}{2.688835in}}{\pgfqpoint{2.022399in}{2.697071in}}%
\pgfpathcurveto{\pgfqpoint{2.022399in}{2.705307in}}{\pgfqpoint{2.019127in}{2.713207in}}{\pgfqpoint{2.013303in}{2.719031in}}%
\pgfpathcurveto{\pgfqpoint{2.007479in}{2.724855in}}{\pgfqpoint{1.999579in}{2.728128in}}{\pgfqpoint{1.991343in}{2.728128in}}%
\pgfpathcurveto{\pgfqpoint{1.983106in}{2.728128in}}{\pgfqpoint{1.975206in}{2.724855in}}{\pgfqpoint{1.969382in}{2.719031in}}%
\pgfpathcurveto{\pgfqpoint{1.963558in}{2.713207in}}{\pgfqpoint{1.960286in}{2.705307in}}{\pgfqpoint{1.960286in}{2.697071in}}%
\pgfpathcurveto{\pgfqpoint{1.960286in}{2.688835in}}{\pgfqpoint{1.963558in}{2.680935in}}{\pgfqpoint{1.969382in}{2.675111in}}%
\pgfpathcurveto{\pgfqpoint{1.975206in}{2.669287in}}{\pgfqpoint{1.983106in}{2.666015in}}{\pgfqpoint{1.991343in}{2.666015in}}%
\pgfpathclose%
\pgfusepath{stroke,fill}%
\end{pgfscope}%
\begin{pgfscope}%
\pgfpathrectangle{\pgfqpoint{0.100000in}{0.212622in}}{\pgfqpoint{3.696000in}{3.696000in}}%
\pgfusepath{clip}%
\pgfsetbuttcap%
\pgfsetroundjoin%
\definecolor{currentfill}{rgb}{0.121569,0.466667,0.705882}%
\pgfsetfillcolor{currentfill}%
\pgfsetfillopacity{0.408175}%
\pgfsetlinewidth{1.003750pt}%
\definecolor{currentstroke}{rgb}{0.121569,0.466667,0.705882}%
\pgfsetstrokecolor{currentstroke}%
\pgfsetstrokeopacity{0.408175}%
\pgfsetdash{}{0pt}%
\pgfpathmoveto{\pgfqpoint{2.017925in}{2.664008in}}%
\pgfpathcurveto{\pgfqpoint{2.026161in}{2.664008in}}{\pgfqpoint{2.034061in}{2.667281in}}{\pgfqpoint{2.039885in}{2.673105in}}%
\pgfpathcurveto{\pgfqpoint{2.045709in}{2.678928in}}{\pgfqpoint{2.048981in}{2.686829in}}{\pgfqpoint{2.048981in}{2.695065in}}%
\pgfpathcurveto{\pgfqpoint{2.048981in}{2.703301in}}{\pgfqpoint{2.045709in}{2.711201in}}{\pgfqpoint{2.039885in}{2.717025in}}%
\pgfpathcurveto{\pgfqpoint{2.034061in}{2.722849in}}{\pgfqpoint{2.026161in}{2.726121in}}{\pgfqpoint{2.017925in}{2.726121in}}%
\pgfpathcurveto{\pgfqpoint{2.009688in}{2.726121in}}{\pgfqpoint{2.001788in}{2.722849in}}{\pgfqpoint{1.995964in}{2.717025in}}%
\pgfpathcurveto{\pgfqpoint{1.990140in}{2.711201in}}{\pgfqpoint{1.986868in}{2.703301in}}{\pgfqpoint{1.986868in}{2.695065in}}%
\pgfpathcurveto{\pgfqpoint{1.986868in}{2.686829in}}{\pgfqpoint{1.990140in}{2.678928in}}{\pgfqpoint{1.995964in}{2.673105in}}%
\pgfpathcurveto{\pgfqpoint{2.001788in}{2.667281in}}{\pgfqpoint{2.009688in}{2.664008in}}{\pgfqpoint{2.017925in}{2.664008in}}%
\pgfpathclose%
\pgfusepath{stroke,fill}%
\end{pgfscope}%
\begin{pgfscope}%
\pgfpathrectangle{\pgfqpoint{0.100000in}{0.212622in}}{\pgfqpoint{3.696000in}{3.696000in}}%
\pgfusepath{clip}%
\pgfsetbuttcap%
\pgfsetroundjoin%
\definecolor{currentfill}{rgb}{0.121569,0.466667,0.705882}%
\pgfsetfillcolor{currentfill}%
\pgfsetfillopacity{0.414658}%
\pgfsetlinewidth{1.003750pt}%
\definecolor{currentstroke}{rgb}{0.121569,0.466667,0.705882}%
\pgfsetstrokecolor{currentstroke}%
\pgfsetstrokeopacity{0.414658}%
\pgfsetdash{}{0pt}%
\pgfpathmoveto{\pgfqpoint{2.043996in}{2.658839in}}%
\pgfpathcurveto{\pgfqpoint{2.052232in}{2.658839in}}{\pgfqpoint{2.060132in}{2.662111in}}{\pgfqpoint{2.065956in}{2.667935in}}%
\pgfpathcurveto{\pgfqpoint{2.071780in}{2.673759in}}{\pgfqpoint{2.075052in}{2.681659in}}{\pgfqpoint{2.075052in}{2.689895in}}%
\pgfpathcurveto{\pgfqpoint{2.075052in}{2.698132in}}{\pgfqpoint{2.071780in}{2.706032in}}{\pgfqpoint{2.065956in}{2.711856in}}%
\pgfpathcurveto{\pgfqpoint{2.060132in}{2.717680in}}{\pgfqpoint{2.052232in}{2.720952in}}{\pgfqpoint{2.043996in}{2.720952in}}%
\pgfpathcurveto{\pgfqpoint{2.035760in}{2.720952in}}{\pgfqpoint{2.027860in}{2.717680in}}{\pgfqpoint{2.022036in}{2.711856in}}%
\pgfpathcurveto{\pgfqpoint{2.016212in}{2.706032in}}{\pgfqpoint{2.012939in}{2.698132in}}{\pgfqpoint{2.012939in}{2.689895in}}%
\pgfpathcurveto{\pgfqpoint{2.012939in}{2.681659in}}{\pgfqpoint{2.016212in}{2.673759in}}{\pgfqpoint{2.022036in}{2.667935in}}%
\pgfpathcurveto{\pgfqpoint{2.027860in}{2.662111in}}{\pgfqpoint{2.035760in}{2.658839in}}{\pgfqpoint{2.043996in}{2.658839in}}%
\pgfpathclose%
\pgfusepath{stroke,fill}%
\end{pgfscope}%
\begin{pgfscope}%
\pgfpathrectangle{\pgfqpoint{0.100000in}{0.212622in}}{\pgfqpoint{3.696000in}{3.696000in}}%
\pgfusepath{clip}%
\pgfsetbuttcap%
\pgfsetroundjoin%
\definecolor{currentfill}{rgb}{0.121569,0.466667,0.705882}%
\pgfsetfillcolor{currentfill}%
\pgfsetfillopacity{0.418975}%
\pgfsetlinewidth{1.003750pt}%
\definecolor{currentstroke}{rgb}{0.121569,0.466667,0.705882}%
\pgfsetstrokecolor{currentstroke}%
\pgfsetstrokeopacity{0.418975}%
\pgfsetdash{}{0pt}%
\pgfpathmoveto{\pgfqpoint{2.072557in}{2.655681in}}%
\pgfpathcurveto{\pgfqpoint{2.080794in}{2.655681in}}{\pgfqpoint{2.088694in}{2.658954in}}{\pgfqpoint{2.094518in}{2.664777in}}%
\pgfpathcurveto{\pgfqpoint{2.100342in}{2.670601in}}{\pgfqpoint{2.103614in}{2.678501in}}{\pgfqpoint{2.103614in}{2.686738in}}%
\pgfpathcurveto{\pgfqpoint{2.103614in}{2.694974in}}{\pgfqpoint{2.100342in}{2.702874in}}{\pgfqpoint{2.094518in}{2.708698in}}%
\pgfpathcurveto{\pgfqpoint{2.088694in}{2.714522in}}{\pgfqpoint{2.080794in}{2.717794in}}{\pgfqpoint{2.072557in}{2.717794in}}%
\pgfpathcurveto{\pgfqpoint{2.064321in}{2.717794in}}{\pgfqpoint{2.056421in}{2.714522in}}{\pgfqpoint{2.050597in}{2.708698in}}%
\pgfpathcurveto{\pgfqpoint{2.044773in}{2.702874in}}{\pgfqpoint{2.041501in}{2.694974in}}{\pgfqpoint{2.041501in}{2.686738in}}%
\pgfpathcurveto{\pgfqpoint{2.041501in}{2.678501in}}{\pgfqpoint{2.044773in}{2.670601in}}{\pgfqpoint{2.050597in}{2.664777in}}%
\pgfpathcurveto{\pgfqpoint{2.056421in}{2.658954in}}{\pgfqpoint{2.064321in}{2.655681in}}{\pgfqpoint{2.072557in}{2.655681in}}%
\pgfpathclose%
\pgfusepath{stroke,fill}%
\end{pgfscope}%
\begin{pgfscope}%
\pgfpathrectangle{\pgfqpoint{0.100000in}{0.212622in}}{\pgfqpoint{3.696000in}{3.696000in}}%
\pgfusepath{clip}%
\pgfsetbuttcap%
\pgfsetroundjoin%
\definecolor{currentfill}{rgb}{0.121569,0.466667,0.705882}%
\pgfsetfillcolor{currentfill}%
\pgfsetfillopacity{0.422090}%
\pgfsetlinewidth{1.003750pt}%
\definecolor{currentstroke}{rgb}{0.121569,0.466667,0.705882}%
\pgfsetstrokecolor{currentstroke}%
\pgfsetstrokeopacity{0.422090}%
\pgfsetdash{}{0pt}%
\pgfpathmoveto{\pgfqpoint{2.099227in}{2.642206in}}%
\pgfpathcurveto{\pgfqpoint{2.107464in}{2.642206in}}{\pgfqpoint{2.115364in}{2.645479in}}{\pgfqpoint{2.121188in}{2.651303in}}%
\pgfpathcurveto{\pgfqpoint{2.127011in}{2.657126in}}{\pgfqpoint{2.130284in}{2.665027in}}{\pgfqpoint{2.130284in}{2.673263in}}%
\pgfpathcurveto{\pgfqpoint{2.130284in}{2.681499in}}{\pgfqpoint{2.127011in}{2.689399in}}{\pgfqpoint{2.121188in}{2.695223in}}%
\pgfpathcurveto{\pgfqpoint{2.115364in}{2.701047in}}{\pgfqpoint{2.107464in}{2.704319in}}{\pgfqpoint{2.099227in}{2.704319in}}%
\pgfpathcurveto{\pgfqpoint{2.090991in}{2.704319in}}{\pgfqpoint{2.083091in}{2.701047in}}{\pgfqpoint{2.077267in}{2.695223in}}%
\pgfpathcurveto{\pgfqpoint{2.071443in}{2.689399in}}{\pgfqpoint{2.068171in}{2.681499in}}{\pgfqpoint{2.068171in}{2.673263in}}%
\pgfpathcurveto{\pgfqpoint{2.068171in}{2.665027in}}{\pgfqpoint{2.071443in}{2.657126in}}{\pgfqpoint{2.077267in}{2.651303in}}%
\pgfpathcurveto{\pgfqpoint{2.083091in}{2.645479in}}{\pgfqpoint{2.090991in}{2.642206in}}{\pgfqpoint{2.099227in}{2.642206in}}%
\pgfpathclose%
\pgfusepath{stroke,fill}%
\end{pgfscope}%
\begin{pgfscope}%
\pgfpathrectangle{\pgfqpoint{0.100000in}{0.212622in}}{\pgfqpoint{3.696000in}{3.696000in}}%
\pgfusepath{clip}%
\pgfsetbuttcap%
\pgfsetroundjoin%
\definecolor{currentfill}{rgb}{0.121569,0.466667,0.705882}%
\pgfsetfillcolor{currentfill}%
\pgfsetfillopacity{0.427067}%
\pgfsetlinewidth{1.003750pt}%
\definecolor{currentstroke}{rgb}{0.121569,0.466667,0.705882}%
\pgfsetstrokecolor{currentstroke}%
\pgfsetstrokeopacity{0.427067}%
\pgfsetdash{}{0pt}%
\pgfpathmoveto{\pgfqpoint{2.124524in}{2.636748in}}%
\pgfpathcurveto{\pgfqpoint{2.132761in}{2.636748in}}{\pgfqpoint{2.140661in}{2.640020in}}{\pgfqpoint{2.146485in}{2.645844in}}%
\pgfpathcurveto{\pgfqpoint{2.152308in}{2.651668in}}{\pgfqpoint{2.155581in}{2.659568in}}{\pgfqpoint{2.155581in}{2.667804in}}%
\pgfpathcurveto{\pgfqpoint{2.155581in}{2.676041in}}{\pgfqpoint{2.152308in}{2.683941in}}{\pgfqpoint{2.146485in}{2.689765in}}%
\pgfpathcurveto{\pgfqpoint{2.140661in}{2.695589in}}{\pgfqpoint{2.132761in}{2.698861in}}{\pgfqpoint{2.124524in}{2.698861in}}%
\pgfpathcurveto{\pgfqpoint{2.116288in}{2.698861in}}{\pgfqpoint{2.108388in}{2.695589in}}{\pgfqpoint{2.102564in}{2.689765in}}%
\pgfpathcurveto{\pgfqpoint{2.096740in}{2.683941in}}{\pgfqpoint{2.093468in}{2.676041in}}{\pgfqpoint{2.093468in}{2.667804in}}%
\pgfpathcurveto{\pgfqpoint{2.093468in}{2.659568in}}{\pgfqpoint{2.096740in}{2.651668in}}{\pgfqpoint{2.102564in}{2.645844in}}%
\pgfpathcurveto{\pgfqpoint{2.108388in}{2.640020in}}{\pgfqpoint{2.116288in}{2.636748in}}{\pgfqpoint{2.124524in}{2.636748in}}%
\pgfpathclose%
\pgfusepath{stroke,fill}%
\end{pgfscope}%
\begin{pgfscope}%
\pgfpathrectangle{\pgfqpoint{0.100000in}{0.212622in}}{\pgfqpoint{3.696000in}{3.696000in}}%
\pgfusepath{clip}%
\pgfsetbuttcap%
\pgfsetroundjoin%
\definecolor{currentfill}{rgb}{0.121569,0.466667,0.705882}%
\pgfsetfillcolor{currentfill}%
\pgfsetfillopacity{0.432409}%
\pgfsetlinewidth{1.003750pt}%
\definecolor{currentstroke}{rgb}{0.121569,0.466667,0.705882}%
\pgfsetstrokecolor{currentstroke}%
\pgfsetstrokeopacity{0.432409}%
\pgfsetdash{}{0pt}%
\pgfpathmoveto{\pgfqpoint{2.146988in}{2.633083in}}%
\pgfpathcurveto{\pgfqpoint{2.155224in}{2.633083in}}{\pgfqpoint{2.163124in}{2.636356in}}{\pgfqpoint{2.168948in}{2.642180in}}%
\pgfpathcurveto{\pgfqpoint{2.174772in}{2.648004in}}{\pgfqpoint{2.178044in}{2.655904in}}{\pgfqpoint{2.178044in}{2.664140in}}%
\pgfpathcurveto{\pgfqpoint{2.178044in}{2.672376in}}{\pgfqpoint{2.174772in}{2.680276in}}{\pgfqpoint{2.168948in}{2.686100in}}%
\pgfpathcurveto{\pgfqpoint{2.163124in}{2.691924in}}{\pgfqpoint{2.155224in}{2.695196in}}{\pgfqpoint{2.146988in}{2.695196in}}%
\pgfpathcurveto{\pgfqpoint{2.138751in}{2.695196in}}{\pgfqpoint{2.130851in}{2.691924in}}{\pgfqpoint{2.125027in}{2.686100in}}%
\pgfpathcurveto{\pgfqpoint{2.119203in}{2.680276in}}{\pgfqpoint{2.115931in}{2.672376in}}{\pgfqpoint{2.115931in}{2.664140in}}%
\pgfpathcurveto{\pgfqpoint{2.115931in}{2.655904in}}{\pgfqpoint{2.119203in}{2.648004in}}{\pgfqpoint{2.125027in}{2.642180in}}%
\pgfpathcurveto{\pgfqpoint{2.130851in}{2.636356in}}{\pgfqpoint{2.138751in}{2.633083in}}{\pgfqpoint{2.146988in}{2.633083in}}%
\pgfpathclose%
\pgfusepath{stroke,fill}%
\end{pgfscope}%
\begin{pgfscope}%
\pgfpathrectangle{\pgfqpoint{0.100000in}{0.212622in}}{\pgfqpoint{3.696000in}{3.696000in}}%
\pgfusepath{clip}%
\pgfsetbuttcap%
\pgfsetroundjoin%
\definecolor{currentfill}{rgb}{0.121569,0.466667,0.705882}%
\pgfsetfillcolor{currentfill}%
\pgfsetfillopacity{0.438619}%
\pgfsetlinewidth{1.003750pt}%
\definecolor{currentstroke}{rgb}{0.121569,0.466667,0.705882}%
\pgfsetstrokecolor{currentstroke}%
\pgfsetstrokeopacity{0.438619}%
\pgfsetdash{}{0pt}%
\pgfpathmoveto{\pgfqpoint{2.168103in}{2.639688in}}%
\pgfpathcurveto{\pgfqpoint{2.176340in}{2.639688in}}{\pgfqpoint{2.184240in}{2.642960in}}{\pgfqpoint{2.190064in}{2.648784in}}%
\pgfpathcurveto{\pgfqpoint{2.195887in}{2.654608in}}{\pgfqpoint{2.199160in}{2.662508in}}{\pgfqpoint{2.199160in}{2.670744in}}%
\pgfpathcurveto{\pgfqpoint{2.199160in}{2.678980in}}{\pgfqpoint{2.195887in}{2.686880in}}{\pgfqpoint{2.190064in}{2.692704in}}%
\pgfpathcurveto{\pgfqpoint{2.184240in}{2.698528in}}{\pgfqpoint{2.176340in}{2.701801in}}{\pgfqpoint{2.168103in}{2.701801in}}%
\pgfpathcurveto{\pgfqpoint{2.159867in}{2.701801in}}{\pgfqpoint{2.151967in}{2.698528in}}{\pgfqpoint{2.146143in}{2.692704in}}%
\pgfpathcurveto{\pgfqpoint{2.140319in}{2.686880in}}{\pgfqpoint{2.137047in}{2.678980in}}{\pgfqpoint{2.137047in}{2.670744in}}%
\pgfpathcurveto{\pgfqpoint{2.137047in}{2.662508in}}{\pgfqpoint{2.140319in}{2.654608in}}{\pgfqpoint{2.146143in}{2.648784in}}%
\pgfpathcurveto{\pgfqpoint{2.151967in}{2.642960in}}{\pgfqpoint{2.159867in}{2.639688in}}{\pgfqpoint{2.168103in}{2.639688in}}%
\pgfpathclose%
\pgfusepath{stroke,fill}%
\end{pgfscope}%
\begin{pgfscope}%
\pgfpathrectangle{\pgfqpoint{0.100000in}{0.212622in}}{\pgfqpoint{3.696000in}{3.696000in}}%
\pgfusepath{clip}%
\pgfsetbuttcap%
\pgfsetroundjoin%
\definecolor{currentfill}{rgb}{0.121569,0.466667,0.705882}%
\pgfsetfillcolor{currentfill}%
\pgfsetfillopacity{0.448688}%
\pgfsetlinewidth{1.003750pt}%
\definecolor{currentstroke}{rgb}{0.121569,0.466667,0.705882}%
\pgfsetstrokecolor{currentstroke}%
\pgfsetstrokeopacity{0.448688}%
\pgfsetdash{}{0pt}%
\pgfpathmoveto{\pgfqpoint{2.207361in}{2.648719in}}%
\pgfpathcurveto{\pgfqpoint{2.215597in}{2.648719in}}{\pgfqpoint{2.223497in}{2.651991in}}{\pgfqpoint{2.229321in}{2.657815in}}%
\pgfpathcurveto{\pgfqpoint{2.235145in}{2.663639in}}{\pgfqpoint{2.238418in}{2.671539in}}{\pgfqpoint{2.238418in}{2.679775in}}%
\pgfpathcurveto{\pgfqpoint{2.238418in}{2.688011in}}{\pgfqpoint{2.235145in}{2.695911in}}{\pgfqpoint{2.229321in}{2.701735in}}%
\pgfpathcurveto{\pgfqpoint{2.223497in}{2.707559in}}{\pgfqpoint{2.215597in}{2.710832in}}{\pgfqpoint{2.207361in}{2.710832in}}%
\pgfpathcurveto{\pgfqpoint{2.199125in}{2.710832in}}{\pgfqpoint{2.191225in}{2.707559in}}{\pgfqpoint{2.185401in}{2.701735in}}%
\pgfpathcurveto{\pgfqpoint{2.179577in}{2.695911in}}{\pgfqpoint{2.176305in}{2.688011in}}{\pgfqpoint{2.176305in}{2.679775in}}%
\pgfpathcurveto{\pgfqpoint{2.176305in}{2.671539in}}{\pgfqpoint{2.179577in}{2.663639in}}{\pgfqpoint{2.185401in}{2.657815in}}%
\pgfpathcurveto{\pgfqpoint{2.191225in}{2.651991in}}{\pgfqpoint{2.199125in}{2.648719in}}{\pgfqpoint{2.207361in}{2.648719in}}%
\pgfpathclose%
\pgfusepath{stroke,fill}%
\end{pgfscope}%
\begin{pgfscope}%
\pgfpathrectangle{\pgfqpoint{0.100000in}{0.212622in}}{\pgfqpoint{3.696000in}{3.696000in}}%
\pgfusepath{clip}%
\pgfsetbuttcap%
\pgfsetroundjoin%
\definecolor{currentfill}{rgb}{0.121569,0.466667,0.705882}%
\pgfsetfillcolor{currentfill}%
\pgfsetfillopacity{0.457948}%
\pgfsetlinewidth{1.003750pt}%
\definecolor{currentstroke}{rgb}{0.121569,0.466667,0.705882}%
\pgfsetstrokecolor{currentstroke}%
\pgfsetstrokeopacity{0.457948}%
\pgfsetdash{}{0pt}%
\pgfpathmoveto{\pgfqpoint{2.245058in}{2.652185in}}%
\pgfpathcurveto{\pgfqpoint{2.253294in}{2.652185in}}{\pgfqpoint{2.261194in}{2.655457in}}{\pgfqpoint{2.267018in}{2.661281in}}%
\pgfpathcurveto{\pgfqpoint{2.272842in}{2.667105in}}{\pgfqpoint{2.276114in}{2.675005in}}{\pgfqpoint{2.276114in}{2.683241in}}%
\pgfpathcurveto{\pgfqpoint{2.276114in}{2.691477in}}{\pgfqpoint{2.272842in}{2.699377in}}{\pgfqpoint{2.267018in}{2.705201in}}%
\pgfpathcurveto{\pgfqpoint{2.261194in}{2.711025in}}{\pgfqpoint{2.253294in}{2.714298in}}{\pgfqpoint{2.245058in}{2.714298in}}%
\pgfpathcurveto{\pgfqpoint{2.236821in}{2.714298in}}{\pgfqpoint{2.228921in}{2.711025in}}{\pgfqpoint{2.223097in}{2.705201in}}%
\pgfpathcurveto{\pgfqpoint{2.217273in}{2.699377in}}{\pgfqpoint{2.214001in}{2.691477in}}{\pgfqpoint{2.214001in}{2.683241in}}%
\pgfpathcurveto{\pgfqpoint{2.214001in}{2.675005in}}{\pgfqpoint{2.217273in}{2.667105in}}{\pgfqpoint{2.223097in}{2.661281in}}%
\pgfpathcurveto{\pgfqpoint{2.228921in}{2.655457in}}{\pgfqpoint{2.236821in}{2.652185in}}{\pgfqpoint{2.245058in}{2.652185in}}%
\pgfpathclose%
\pgfusepath{stroke,fill}%
\end{pgfscope}%
\begin{pgfscope}%
\pgfpathrectangle{\pgfqpoint{0.100000in}{0.212622in}}{\pgfqpoint{3.696000in}{3.696000in}}%
\pgfusepath{clip}%
\pgfsetbuttcap%
\pgfsetroundjoin%
\definecolor{currentfill}{rgb}{0.121569,0.466667,0.705882}%
\pgfsetfillcolor{currentfill}%
\pgfsetfillopacity{0.466486}%
\pgfsetlinewidth{1.003750pt}%
\definecolor{currentstroke}{rgb}{0.121569,0.466667,0.705882}%
\pgfsetstrokecolor{currentstroke}%
\pgfsetstrokeopacity{0.466486}%
\pgfsetdash{}{0pt}%
\pgfpathmoveto{\pgfqpoint{2.280532in}{2.650636in}}%
\pgfpathcurveto{\pgfqpoint{2.288768in}{2.650636in}}{\pgfqpoint{2.296668in}{2.653909in}}{\pgfqpoint{2.302492in}{2.659733in}}%
\pgfpathcurveto{\pgfqpoint{2.308316in}{2.665557in}}{\pgfqpoint{2.311588in}{2.673457in}}{\pgfqpoint{2.311588in}{2.681693in}}%
\pgfpathcurveto{\pgfqpoint{2.311588in}{2.689929in}}{\pgfqpoint{2.308316in}{2.697829in}}{\pgfqpoint{2.302492in}{2.703653in}}%
\pgfpathcurveto{\pgfqpoint{2.296668in}{2.709477in}}{\pgfqpoint{2.288768in}{2.712749in}}{\pgfqpoint{2.280532in}{2.712749in}}%
\pgfpathcurveto{\pgfqpoint{2.272295in}{2.712749in}}{\pgfqpoint{2.264395in}{2.709477in}}{\pgfqpoint{2.258571in}{2.703653in}}%
\pgfpathcurveto{\pgfqpoint{2.252747in}{2.697829in}}{\pgfqpoint{2.249475in}{2.689929in}}{\pgfqpoint{2.249475in}{2.681693in}}%
\pgfpathcurveto{\pgfqpoint{2.249475in}{2.673457in}}{\pgfqpoint{2.252747in}{2.665557in}}{\pgfqpoint{2.258571in}{2.659733in}}%
\pgfpathcurveto{\pgfqpoint{2.264395in}{2.653909in}}{\pgfqpoint{2.272295in}{2.650636in}}{\pgfqpoint{2.280532in}{2.650636in}}%
\pgfpathclose%
\pgfusepath{stroke,fill}%
\end{pgfscope}%
\begin{pgfscope}%
\pgfpathrectangle{\pgfqpoint{0.100000in}{0.212622in}}{\pgfqpoint{3.696000in}{3.696000in}}%
\pgfusepath{clip}%
\pgfsetbuttcap%
\pgfsetroundjoin%
\definecolor{currentfill}{rgb}{0.121569,0.466667,0.705882}%
\pgfsetfillcolor{currentfill}%
\pgfsetfillopacity{0.474955}%
\pgfsetlinewidth{1.003750pt}%
\definecolor{currentstroke}{rgb}{0.121569,0.466667,0.705882}%
\pgfsetstrokecolor{currentstroke}%
\pgfsetstrokeopacity{0.474955}%
\pgfsetdash{}{0pt}%
\pgfpathmoveto{\pgfqpoint{2.312580in}{2.641578in}}%
\pgfpathcurveto{\pgfqpoint{2.320816in}{2.641578in}}{\pgfqpoint{2.328716in}{2.644850in}}{\pgfqpoint{2.334540in}{2.650674in}}%
\pgfpathcurveto{\pgfqpoint{2.340364in}{2.656498in}}{\pgfqpoint{2.343637in}{2.664398in}}{\pgfqpoint{2.343637in}{2.672634in}}%
\pgfpathcurveto{\pgfqpoint{2.343637in}{2.680870in}}{\pgfqpoint{2.340364in}{2.688770in}}{\pgfqpoint{2.334540in}{2.694594in}}%
\pgfpathcurveto{\pgfqpoint{2.328716in}{2.700418in}}{\pgfqpoint{2.320816in}{2.703691in}}{\pgfqpoint{2.312580in}{2.703691in}}%
\pgfpathcurveto{\pgfqpoint{2.304344in}{2.703691in}}{\pgfqpoint{2.296444in}{2.700418in}}{\pgfqpoint{2.290620in}{2.694594in}}%
\pgfpathcurveto{\pgfqpoint{2.284796in}{2.688770in}}{\pgfqpoint{2.281524in}{2.680870in}}{\pgfqpoint{2.281524in}{2.672634in}}%
\pgfpathcurveto{\pgfqpoint{2.281524in}{2.664398in}}{\pgfqpoint{2.284796in}{2.656498in}}{\pgfqpoint{2.290620in}{2.650674in}}%
\pgfpathcurveto{\pgfqpoint{2.296444in}{2.644850in}}{\pgfqpoint{2.304344in}{2.641578in}}{\pgfqpoint{2.312580in}{2.641578in}}%
\pgfpathclose%
\pgfusepath{stroke,fill}%
\end{pgfscope}%
\begin{pgfscope}%
\pgfpathrectangle{\pgfqpoint{0.100000in}{0.212622in}}{\pgfqpoint{3.696000in}{3.696000in}}%
\pgfusepath{clip}%
\pgfsetbuttcap%
\pgfsetroundjoin%
\definecolor{currentfill}{rgb}{0.121569,0.466667,0.705882}%
\pgfsetfillcolor{currentfill}%
\pgfsetfillopacity{0.477201}%
\pgfsetlinewidth{1.003750pt}%
\definecolor{currentstroke}{rgb}{0.121569,0.466667,0.705882}%
\pgfsetstrokecolor{currentstroke}%
\pgfsetstrokeopacity{0.477201}%
\pgfsetdash{}{0pt}%
\pgfpathmoveto{\pgfqpoint{2.352305in}{2.635917in}}%
\pgfpathcurveto{\pgfqpoint{2.360541in}{2.635917in}}{\pgfqpoint{2.368441in}{2.639189in}}{\pgfqpoint{2.374265in}{2.645013in}}%
\pgfpathcurveto{\pgfqpoint{2.380089in}{2.650837in}}{\pgfqpoint{2.383361in}{2.658737in}}{\pgfqpoint{2.383361in}{2.666974in}}%
\pgfpathcurveto{\pgfqpoint{2.383361in}{2.675210in}}{\pgfqpoint{2.380089in}{2.683110in}}{\pgfqpoint{2.374265in}{2.688934in}}%
\pgfpathcurveto{\pgfqpoint{2.368441in}{2.694758in}}{\pgfqpoint{2.360541in}{2.698030in}}{\pgfqpoint{2.352305in}{2.698030in}}%
\pgfpathcurveto{\pgfqpoint{2.344068in}{2.698030in}}{\pgfqpoint{2.336168in}{2.694758in}}{\pgfqpoint{2.330344in}{2.688934in}}%
\pgfpathcurveto{\pgfqpoint{2.324520in}{2.683110in}}{\pgfqpoint{2.321248in}{2.675210in}}{\pgfqpoint{2.321248in}{2.666974in}}%
\pgfpathcurveto{\pgfqpoint{2.321248in}{2.658737in}}{\pgfqpoint{2.324520in}{2.650837in}}{\pgfqpoint{2.330344in}{2.645013in}}%
\pgfpathcurveto{\pgfqpoint{2.336168in}{2.639189in}}{\pgfqpoint{2.344068in}{2.635917in}}{\pgfqpoint{2.352305in}{2.635917in}}%
\pgfpathclose%
\pgfusepath{stroke,fill}%
\end{pgfscope}%
\begin{pgfscope}%
\pgfpathrectangle{\pgfqpoint{0.100000in}{0.212622in}}{\pgfqpoint{3.696000in}{3.696000in}}%
\pgfusepath{clip}%
\pgfsetbuttcap%
\pgfsetroundjoin%
\definecolor{currentfill}{rgb}{0.121569,0.466667,0.705882}%
\pgfsetfillcolor{currentfill}%
\pgfsetfillopacity{0.479926}%
\pgfsetlinewidth{1.003750pt}%
\definecolor{currentstroke}{rgb}{0.121569,0.466667,0.705882}%
\pgfsetstrokecolor{currentstroke}%
\pgfsetstrokeopacity{0.479926}%
\pgfsetdash{}{0pt}%
\pgfpathmoveto{\pgfqpoint{2.391385in}{2.635759in}}%
\pgfpathcurveto{\pgfqpoint{2.399621in}{2.635759in}}{\pgfqpoint{2.407521in}{2.639032in}}{\pgfqpoint{2.413345in}{2.644856in}}%
\pgfpathcurveto{\pgfqpoint{2.419169in}{2.650680in}}{\pgfqpoint{2.422441in}{2.658580in}}{\pgfqpoint{2.422441in}{2.666816in}}%
\pgfpathcurveto{\pgfqpoint{2.422441in}{2.675052in}}{\pgfqpoint{2.419169in}{2.682952in}}{\pgfqpoint{2.413345in}{2.688776in}}%
\pgfpathcurveto{\pgfqpoint{2.407521in}{2.694600in}}{\pgfqpoint{2.399621in}{2.697872in}}{\pgfqpoint{2.391385in}{2.697872in}}%
\pgfpathcurveto{\pgfqpoint{2.383149in}{2.697872in}}{\pgfqpoint{2.375248in}{2.694600in}}{\pgfqpoint{2.369425in}{2.688776in}}%
\pgfpathcurveto{\pgfqpoint{2.363601in}{2.682952in}}{\pgfqpoint{2.360328in}{2.675052in}}{\pgfqpoint{2.360328in}{2.666816in}}%
\pgfpathcurveto{\pgfqpoint{2.360328in}{2.658580in}}{\pgfqpoint{2.363601in}{2.650680in}}{\pgfqpoint{2.369425in}{2.644856in}}%
\pgfpathcurveto{\pgfqpoint{2.375248in}{2.639032in}}{\pgfqpoint{2.383149in}{2.635759in}}{\pgfqpoint{2.391385in}{2.635759in}}%
\pgfpathclose%
\pgfusepath{stroke,fill}%
\end{pgfscope}%
\begin{pgfscope}%
\pgfpathrectangle{\pgfqpoint{0.100000in}{0.212622in}}{\pgfqpoint{3.696000in}{3.696000in}}%
\pgfusepath{clip}%
\pgfsetbuttcap%
\pgfsetroundjoin%
\definecolor{currentfill}{rgb}{0.121569,0.466667,0.705882}%
\pgfsetfillcolor{currentfill}%
\pgfsetfillopacity{0.488296}%
\pgfsetlinewidth{1.003750pt}%
\definecolor{currentstroke}{rgb}{0.121569,0.466667,0.705882}%
\pgfsetstrokecolor{currentstroke}%
\pgfsetstrokeopacity{0.488296}%
\pgfsetdash{}{0pt}%
\pgfpathmoveto{\pgfqpoint{2.423573in}{2.643164in}}%
\pgfpathcurveto{\pgfqpoint{2.431810in}{2.643164in}}{\pgfqpoint{2.439710in}{2.646436in}}{\pgfqpoint{2.445534in}{2.652260in}}%
\pgfpathcurveto{\pgfqpoint{2.451358in}{2.658084in}}{\pgfqpoint{2.454630in}{2.665984in}}{\pgfqpoint{2.454630in}{2.674220in}}%
\pgfpathcurveto{\pgfqpoint{2.454630in}{2.682456in}}{\pgfqpoint{2.451358in}{2.690356in}}{\pgfqpoint{2.445534in}{2.696180in}}%
\pgfpathcurveto{\pgfqpoint{2.439710in}{2.702004in}}{\pgfqpoint{2.431810in}{2.705277in}}{\pgfqpoint{2.423573in}{2.705277in}}%
\pgfpathcurveto{\pgfqpoint{2.415337in}{2.705277in}}{\pgfqpoint{2.407437in}{2.702004in}}{\pgfqpoint{2.401613in}{2.696180in}}%
\pgfpathcurveto{\pgfqpoint{2.395789in}{2.690356in}}{\pgfqpoint{2.392517in}{2.682456in}}{\pgfqpoint{2.392517in}{2.674220in}}%
\pgfpathcurveto{\pgfqpoint{2.392517in}{2.665984in}}{\pgfqpoint{2.395789in}{2.658084in}}{\pgfqpoint{2.401613in}{2.652260in}}%
\pgfpathcurveto{\pgfqpoint{2.407437in}{2.646436in}}{\pgfqpoint{2.415337in}{2.643164in}}{\pgfqpoint{2.423573in}{2.643164in}}%
\pgfpathclose%
\pgfusepath{stroke,fill}%
\end{pgfscope}%
\begin{pgfscope}%
\pgfpathrectangle{\pgfqpoint{0.100000in}{0.212622in}}{\pgfqpoint{3.696000in}{3.696000in}}%
\pgfusepath{clip}%
\pgfsetbuttcap%
\pgfsetroundjoin%
\definecolor{currentfill}{rgb}{0.121569,0.466667,0.705882}%
\pgfsetfillcolor{currentfill}%
\pgfsetfillopacity{0.498612}%
\pgfsetlinewidth{1.003750pt}%
\definecolor{currentstroke}{rgb}{0.121569,0.466667,0.705882}%
\pgfsetstrokecolor{currentstroke}%
\pgfsetstrokeopacity{0.498612}%
\pgfsetdash{}{0pt}%
\pgfpathmoveto{\pgfqpoint{2.451657in}{2.653180in}}%
\pgfpathcurveto{\pgfqpoint{2.459893in}{2.653180in}}{\pgfqpoint{2.467793in}{2.656452in}}{\pgfqpoint{2.473617in}{2.662276in}}%
\pgfpathcurveto{\pgfqpoint{2.479441in}{2.668100in}}{\pgfqpoint{2.482713in}{2.676000in}}{\pgfqpoint{2.482713in}{2.684237in}}%
\pgfpathcurveto{\pgfqpoint{2.482713in}{2.692473in}}{\pgfqpoint{2.479441in}{2.700373in}}{\pgfqpoint{2.473617in}{2.706197in}}%
\pgfpathcurveto{\pgfqpoint{2.467793in}{2.712021in}}{\pgfqpoint{2.459893in}{2.715293in}}{\pgfqpoint{2.451657in}{2.715293in}}%
\pgfpathcurveto{\pgfqpoint{2.443420in}{2.715293in}}{\pgfqpoint{2.435520in}{2.712021in}}{\pgfqpoint{2.429696in}{2.706197in}}%
\pgfpathcurveto{\pgfqpoint{2.423872in}{2.700373in}}{\pgfqpoint{2.420600in}{2.692473in}}{\pgfqpoint{2.420600in}{2.684237in}}%
\pgfpathcurveto{\pgfqpoint{2.420600in}{2.676000in}}{\pgfqpoint{2.423872in}{2.668100in}}{\pgfqpoint{2.429696in}{2.662276in}}%
\pgfpathcurveto{\pgfqpoint{2.435520in}{2.656452in}}{\pgfqpoint{2.443420in}{2.653180in}}{\pgfqpoint{2.451657in}{2.653180in}}%
\pgfpathclose%
\pgfusepath{stroke,fill}%
\end{pgfscope}%
\begin{pgfscope}%
\pgfpathrectangle{\pgfqpoint{0.100000in}{0.212622in}}{\pgfqpoint{3.696000in}{3.696000in}}%
\pgfusepath{clip}%
\pgfsetbuttcap%
\pgfsetroundjoin%
\definecolor{currentfill}{rgb}{0.121569,0.466667,0.705882}%
\pgfsetfillcolor{currentfill}%
\pgfsetfillopacity{0.507406}%
\pgfsetlinewidth{1.003750pt}%
\definecolor{currentstroke}{rgb}{0.121569,0.466667,0.705882}%
\pgfsetstrokecolor{currentstroke}%
\pgfsetstrokeopacity{0.507406}%
\pgfsetdash{}{0pt}%
\pgfpathmoveto{\pgfqpoint{2.479023in}{2.655399in}}%
\pgfpathcurveto{\pgfqpoint{2.487259in}{2.655399in}}{\pgfqpoint{2.495159in}{2.658672in}}{\pgfqpoint{2.500983in}{2.664496in}}%
\pgfpathcurveto{\pgfqpoint{2.506807in}{2.670320in}}{\pgfqpoint{2.510079in}{2.678220in}}{\pgfqpoint{2.510079in}{2.686456in}}%
\pgfpathcurveto{\pgfqpoint{2.510079in}{2.694692in}}{\pgfqpoint{2.506807in}{2.702592in}}{\pgfqpoint{2.500983in}{2.708416in}}%
\pgfpathcurveto{\pgfqpoint{2.495159in}{2.714240in}}{\pgfqpoint{2.487259in}{2.717512in}}{\pgfqpoint{2.479023in}{2.717512in}}%
\pgfpathcurveto{\pgfqpoint{2.470787in}{2.717512in}}{\pgfqpoint{2.462886in}{2.714240in}}{\pgfqpoint{2.457063in}{2.708416in}}%
\pgfpathcurveto{\pgfqpoint{2.451239in}{2.702592in}}{\pgfqpoint{2.447966in}{2.694692in}}{\pgfqpoint{2.447966in}{2.686456in}}%
\pgfpathcurveto{\pgfqpoint{2.447966in}{2.678220in}}{\pgfqpoint{2.451239in}{2.670320in}}{\pgfqpoint{2.457063in}{2.664496in}}%
\pgfpathcurveto{\pgfqpoint{2.462886in}{2.658672in}}{\pgfqpoint{2.470787in}{2.655399in}}{\pgfqpoint{2.479023in}{2.655399in}}%
\pgfpathclose%
\pgfusepath{stroke,fill}%
\end{pgfscope}%
\begin{pgfscope}%
\pgfpathrectangle{\pgfqpoint{0.100000in}{0.212622in}}{\pgfqpoint{3.696000in}{3.696000in}}%
\pgfusepath{clip}%
\pgfsetbuttcap%
\pgfsetroundjoin%
\definecolor{currentfill}{rgb}{0.121569,0.466667,0.705882}%
\pgfsetfillcolor{currentfill}%
\pgfsetfillopacity{0.512107}%
\pgfsetlinewidth{1.003750pt}%
\definecolor{currentstroke}{rgb}{0.121569,0.466667,0.705882}%
\pgfsetstrokecolor{currentstroke}%
\pgfsetstrokeopacity{0.512107}%
\pgfsetdash{}{0pt}%
\pgfpathmoveto{\pgfqpoint{2.509711in}{2.656414in}}%
\pgfpathcurveto{\pgfqpoint{2.517947in}{2.656414in}}{\pgfqpoint{2.525847in}{2.659686in}}{\pgfqpoint{2.531671in}{2.665510in}}%
\pgfpathcurveto{\pgfqpoint{2.537495in}{2.671334in}}{\pgfqpoint{2.540767in}{2.679234in}}{\pgfqpoint{2.540767in}{2.687470in}}%
\pgfpathcurveto{\pgfqpoint{2.540767in}{2.695706in}}{\pgfqpoint{2.537495in}{2.703606in}}{\pgfqpoint{2.531671in}{2.709430in}}%
\pgfpathcurveto{\pgfqpoint{2.525847in}{2.715254in}}{\pgfqpoint{2.517947in}{2.718527in}}{\pgfqpoint{2.509711in}{2.718527in}}%
\pgfpathcurveto{\pgfqpoint{2.501475in}{2.718527in}}{\pgfqpoint{2.493574in}{2.715254in}}{\pgfqpoint{2.487751in}{2.709430in}}%
\pgfpathcurveto{\pgfqpoint{2.481927in}{2.703606in}}{\pgfqpoint{2.478654in}{2.695706in}}{\pgfqpoint{2.478654in}{2.687470in}}%
\pgfpathcurveto{\pgfqpoint{2.478654in}{2.679234in}}{\pgfqpoint{2.481927in}{2.671334in}}{\pgfqpoint{2.487751in}{2.665510in}}%
\pgfpathcurveto{\pgfqpoint{2.493574in}{2.659686in}}{\pgfqpoint{2.501475in}{2.656414in}}{\pgfqpoint{2.509711in}{2.656414in}}%
\pgfpathclose%
\pgfusepath{stroke,fill}%
\end{pgfscope}%
\begin{pgfscope}%
\pgfpathrectangle{\pgfqpoint{0.100000in}{0.212622in}}{\pgfqpoint{3.696000in}{3.696000in}}%
\pgfusepath{clip}%
\pgfsetbuttcap%
\pgfsetroundjoin%
\definecolor{currentfill}{rgb}{0.121569,0.466667,0.705882}%
\pgfsetfillcolor{currentfill}%
\pgfsetfillopacity{0.517788}%
\pgfsetlinewidth{1.003750pt}%
\definecolor{currentstroke}{rgb}{0.121569,0.466667,0.705882}%
\pgfsetstrokecolor{currentstroke}%
\pgfsetstrokeopacity{0.517788}%
\pgfsetdash{}{0pt}%
\pgfpathmoveto{\pgfqpoint{2.535487in}{2.649746in}}%
\pgfpathcurveto{\pgfqpoint{2.543723in}{2.649746in}}{\pgfqpoint{2.551623in}{2.653018in}}{\pgfqpoint{2.557447in}{2.658842in}}%
\pgfpathcurveto{\pgfqpoint{2.563271in}{2.664666in}}{\pgfqpoint{2.566543in}{2.672566in}}{\pgfqpoint{2.566543in}{2.680802in}}%
\pgfpathcurveto{\pgfqpoint{2.566543in}{2.689039in}}{\pgfqpoint{2.563271in}{2.696939in}}{\pgfqpoint{2.557447in}{2.702763in}}%
\pgfpathcurveto{\pgfqpoint{2.551623in}{2.708587in}}{\pgfqpoint{2.543723in}{2.711859in}}{\pgfqpoint{2.535487in}{2.711859in}}%
\pgfpathcurveto{\pgfqpoint{2.527250in}{2.711859in}}{\pgfqpoint{2.519350in}{2.708587in}}{\pgfqpoint{2.513526in}{2.702763in}}%
\pgfpathcurveto{\pgfqpoint{2.507702in}{2.696939in}}{\pgfqpoint{2.504430in}{2.689039in}}{\pgfqpoint{2.504430in}{2.680802in}}%
\pgfpathcurveto{\pgfqpoint{2.504430in}{2.672566in}}{\pgfqpoint{2.507702in}{2.664666in}}{\pgfqpoint{2.513526in}{2.658842in}}%
\pgfpathcurveto{\pgfqpoint{2.519350in}{2.653018in}}{\pgfqpoint{2.527250in}{2.649746in}}{\pgfqpoint{2.535487in}{2.649746in}}%
\pgfpathclose%
\pgfusepath{stroke,fill}%
\end{pgfscope}%
\begin{pgfscope}%
\pgfpathrectangle{\pgfqpoint{0.100000in}{0.212622in}}{\pgfqpoint{3.696000in}{3.696000in}}%
\pgfusepath{clip}%
\pgfsetbuttcap%
\pgfsetroundjoin%
\definecolor{currentfill}{rgb}{0.121569,0.466667,0.705882}%
\pgfsetfillcolor{currentfill}%
\pgfsetfillopacity{0.520351}%
\pgfsetlinewidth{1.003750pt}%
\definecolor{currentstroke}{rgb}{0.121569,0.466667,0.705882}%
\pgfsetstrokecolor{currentstroke}%
\pgfsetstrokeopacity{0.520351}%
\pgfsetdash{}{0pt}%
\pgfpathmoveto{\pgfqpoint{2.564965in}{2.645628in}}%
\pgfpathcurveto{\pgfqpoint{2.573201in}{2.645628in}}{\pgfqpoint{2.581101in}{2.648900in}}{\pgfqpoint{2.586925in}{2.654724in}}%
\pgfpathcurveto{\pgfqpoint{2.592749in}{2.660548in}}{\pgfqpoint{2.596021in}{2.668448in}}{\pgfqpoint{2.596021in}{2.676684in}}%
\pgfpathcurveto{\pgfqpoint{2.596021in}{2.684920in}}{\pgfqpoint{2.592749in}{2.692820in}}{\pgfqpoint{2.586925in}{2.698644in}}%
\pgfpathcurveto{\pgfqpoint{2.581101in}{2.704468in}}{\pgfqpoint{2.573201in}{2.707741in}}{\pgfqpoint{2.564965in}{2.707741in}}%
\pgfpathcurveto{\pgfqpoint{2.556728in}{2.707741in}}{\pgfqpoint{2.548828in}{2.704468in}}{\pgfqpoint{2.543004in}{2.698644in}}%
\pgfpathcurveto{\pgfqpoint{2.537180in}{2.692820in}}{\pgfqpoint{2.533908in}{2.684920in}}{\pgfqpoint{2.533908in}{2.676684in}}%
\pgfpathcurveto{\pgfqpoint{2.533908in}{2.668448in}}{\pgfqpoint{2.537180in}{2.660548in}}{\pgfqpoint{2.543004in}{2.654724in}}%
\pgfpathcurveto{\pgfqpoint{2.548828in}{2.648900in}}{\pgfqpoint{2.556728in}{2.645628in}}{\pgfqpoint{2.564965in}{2.645628in}}%
\pgfpathclose%
\pgfusepath{stroke,fill}%
\end{pgfscope}%
\begin{pgfscope}%
\pgfpathrectangle{\pgfqpoint{0.100000in}{0.212622in}}{\pgfqpoint{3.696000in}{3.696000in}}%
\pgfusepath{clip}%
\pgfsetbuttcap%
\pgfsetroundjoin%
\definecolor{currentfill}{rgb}{0.121569,0.466667,0.705882}%
\pgfsetfillcolor{currentfill}%
\pgfsetfillopacity{0.524635}%
\pgfsetlinewidth{1.003750pt}%
\definecolor{currentstroke}{rgb}{0.121569,0.466667,0.705882}%
\pgfsetstrokecolor{currentstroke}%
\pgfsetstrokeopacity{0.524635}%
\pgfsetdash{}{0pt}%
\pgfpathmoveto{\pgfqpoint{2.591938in}{2.640720in}}%
\pgfpathcurveto{\pgfqpoint{2.600174in}{2.640720in}}{\pgfqpoint{2.608075in}{2.643992in}}{\pgfqpoint{2.613898in}{2.649816in}}%
\pgfpathcurveto{\pgfqpoint{2.619722in}{2.655640in}}{\pgfqpoint{2.622995in}{2.663540in}}{\pgfqpoint{2.622995in}{2.671776in}}%
\pgfpathcurveto{\pgfqpoint{2.622995in}{2.680012in}}{\pgfqpoint{2.619722in}{2.687913in}}{\pgfqpoint{2.613898in}{2.693736in}}%
\pgfpathcurveto{\pgfqpoint{2.608075in}{2.699560in}}{\pgfqpoint{2.600174in}{2.702833in}}{\pgfqpoint{2.591938in}{2.702833in}}%
\pgfpathcurveto{\pgfqpoint{2.583702in}{2.702833in}}{\pgfqpoint{2.575802in}{2.699560in}}{\pgfqpoint{2.569978in}{2.693736in}}%
\pgfpathcurveto{\pgfqpoint{2.564154in}{2.687913in}}{\pgfqpoint{2.560882in}{2.680012in}}{\pgfqpoint{2.560882in}{2.671776in}}%
\pgfpathcurveto{\pgfqpoint{2.560882in}{2.663540in}}{\pgfqpoint{2.564154in}{2.655640in}}{\pgfqpoint{2.569978in}{2.649816in}}%
\pgfpathcurveto{\pgfqpoint{2.575802in}{2.643992in}}{\pgfqpoint{2.583702in}{2.640720in}}{\pgfqpoint{2.591938in}{2.640720in}}%
\pgfpathclose%
\pgfusepath{stroke,fill}%
\end{pgfscope}%
\begin{pgfscope}%
\pgfpathrectangle{\pgfqpoint{0.100000in}{0.212622in}}{\pgfqpoint{3.696000in}{3.696000in}}%
\pgfusepath{clip}%
\pgfsetbuttcap%
\pgfsetroundjoin%
\definecolor{currentfill}{rgb}{0.121569,0.466667,0.705882}%
\pgfsetfillcolor{currentfill}%
\pgfsetfillopacity{0.528151}%
\pgfsetlinewidth{1.003750pt}%
\definecolor{currentstroke}{rgb}{0.121569,0.466667,0.705882}%
\pgfsetstrokecolor{currentstroke}%
\pgfsetstrokeopacity{0.528151}%
\pgfsetdash{}{0pt}%
\pgfpathmoveto{\pgfqpoint{2.618326in}{2.634454in}}%
\pgfpathcurveto{\pgfqpoint{2.626562in}{2.634454in}}{\pgfqpoint{2.634462in}{2.637727in}}{\pgfqpoint{2.640286in}{2.643551in}}%
\pgfpathcurveto{\pgfqpoint{2.646110in}{2.649374in}}{\pgfqpoint{2.649382in}{2.657275in}}{\pgfqpoint{2.649382in}{2.665511in}}%
\pgfpathcurveto{\pgfqpoint{2.649382in}{2.673747in}}{\pgfqpoint{2.646110in}{2.681647in}}{\pgfqpoint{2.640286in}{2.687471in}}%
\pgfpathcurveto{\pgfqpoint{2.634462in}{2.693295in}}{\pgfqpoint{2.626562in}{2.696567in}}{\pgfqpoint{2.618326in}{2.696567in}}%
\pgfpathcurveto{\pgfqpoint{2.610090in}{2.696567in}}{\pgfqpoint{2.602190in}{2.693295in}}{\pgfqpoint{2.596366in}{2.687471in}}%
\pgfpathcurveto{\pgfqpoint{2.590542in}{2.681647in}}{\pgfqpoint{2.587269in}{2.673747in}}{\pgfqpoint{2.587269in}{2.665511in}}%
\pgfpathcurveto{\pgfqpoint{2.587269in}{2.657275in}}{\pgfqpoint{2.590542in}{2.649374in}}{\pgfqpoint{2.596366in}{2.643551in}}%
\pgfpathcurveto{\pgfqpoint{2.602190in}{2.637727in}}{\pgfqpoint{2.610090in}{2.634454in}}{\pgfqpoint{2.618326in}{2.634454in}}%
\pgfpathclose%
\pgfusepath{stroke,fill}%
\end{pgfscope}%
\begin{pgfscope}%
\pgfpathrectangle{\pgfqpoint{0.100000in}{0.212622in}}{\pgfqpoint{3.696000in}{3.696000in}}%
\pgfusepath{clip}%
\pgfsetbuttcap%
\pgfsetroundjoin%
\definecolor{currentfill}{rgb}{0.121569,0.466667,0.705882}%
\pgfsetfillcolor{currentfill}%
\pgfsetfillopacity{0.535397}%
\pgfsetlinewidth{1.003750pt}%
\definecolor{currentstroke}{rgb}{0.121569,0.466667,0.705882}%
\pgfsetstrokecolor{currentstroke}%
\pgfsetstrokeopacity{0.535397}%
\pgfsetdash{}{0pt}%
\pgfpathmoveto{\pgfqpoint{2.639834in}{2.638741in}}%
\pgfpathcurveto{\pgfqpoint{2.648070in}{2.638741in}}{\pgfqpoint{2.655970in}{2.642013in}}{\pgfqpoint{2.661794in}{2.647837in}}%
\pgfpathcurveto{\pgfqpoint{2.667618in}{2.653661in}}{\pgfqpoint{2.670890in}{2.661561in}}{\pgfqpoint{2.670890in}{2.669797in}}%
\pgfpathcurveto{\pgfqpoint{2.670890in}{2.678033in}}{\pgfqpoint{2.667618in}{2.685934in}}{\pgfqpoint{2.661794in}{2.691757in}}%
\pgfpathcurveto{\pgfqpoint{2.655970in}{2.697581in}}{\pgfqpoint{2.648070in}{2.700854in}}{\pgfqpoint{2.639834in}{2.700854in}}%
\pgfpathcurveto{\pgfqpoint{2.631597in}{2.700854in}}{\pgfqpoint{2.623697in}{2.697581in}}{\pgfqpoint{2.617873in}{2.691757in}}%
\pgfpathcurveto{\pgfqpoint{2.612049in}{2.685934in}}{\pgfqpoint{2.608777in}{2.678033in}}{\pgfqpoint{2.608777in}{2.669797in}}%
\pgfpathcurveto{\pgfqpoint{2.608777in}{2.661561in}}{\pgfqpoint{2.612049in}{2.653661in}}{\pgfqpoint{2.617873in}{2.647837in}}%
\pgfpathcurveto{\pgfqpoint{2.623697in}{2.642013in}}{\pgfqpoint{2.631597in}{2.638741in}}{\pgfqpoint{2.639834in}{2.638741in}}%
\pgfpathclose%
\pgfusepath{stroke,fill}%
\end{pgfscope}%
\begin{pgfscope}%
\pgfpathrectangle{\pgfqpoint{0.100000in}{0.212622in}}{\pgfqpoint{3.696000in}{3.696000in}}%
\pgfusepath{clip}%
\pgfsetbuttcap%
\pgfsetroundjoin%
\definecolor{currentfill}{rgb}{0.121569,0.466667,0.705882}%
\pgfsetfillcolor{currentfill}%
\pgfsetfillopacity{0.542604}%
\pgfsetlinewidth{1.003750pt}%
\definecolor{currentstroke}{rgb}{0.121569,0.466667,0.705882}%
\pgfsetstrokecolor{currentstroke}%
\pgfsetstrokeopacity{0.542604}%
\pgfsetdash{}{0pt}%
\pgfpathmoveto{\pgfqpoint{2.658445in}{2.645253in}}%
\pgfpathcurveto{\pgfqpoint{2.666681in}{2.645253in}}{\pgfqpoint{2.674581in}{2.648525in}}{\pgfqpoint{2.680405in}{2.654349in}}%
\pgfpathcurveto{\pgfqpoint{2.686229in}{2.660173in}}{\pgfqpoint{2.689501in}{2.668073in}}{\pgfqpoint{2.689501in}{2.676309in}}%
\pgfpathcurveto{\pgfqpoint{2.689501in}{2.684546in}}{\pgfqpoint{2.686229in}{2.692446in}}{\pgfqpoint{2.680405in}{2.698270in}}%
\pgfpathcurveto{\pgfqpoint{2.674581in}{2.704093in}}{\pgfqpoint{2.666681in}{2.707366in}}{\pgfqpoint{2.658445in}{2.707366in}}%
\pgfpathcurveto{\pgfqpoint{2.650209in}{2.707366in}}{\pgfqpoint{2.642309in}{2.704093in}}{\pgfqpoint{2.636485in}{2.698270in}}%
\pgfpathcurveto{\pgfqpoint{2.630661in}{2.692446in}}{\pgfqpoint{2.627388in}{2.684546in}}{\pgfqpoint{2.627388in}{2.676309in}}%
\pgfpathcurveto{\pgfqpoint{2.627388in}{2.668073in}}{\pgfqpoint{2.630661in}{2.660173in}}{\pgfqpoint{2.636485in}{2.654349in}}%
\pgfpathcurveto{\pgfqpoint{2.642309in}{2.648525in}}{\pgfqpoint{2.650209in}{2.645253in}}{\pgfqpoint{2.658445in}{2.645253in}}%
\pgfpathclose%
\pgfusepath{stroke,fill}%
\end{pgfscope}%
\begin{pgfscope}%
\pgfpathrectangle{\pgfqpoint{0.100000in}{0.212622in}}{\pgfqpoint{3.696000in}{3.696000in}}%
\pgfusepath{clip}%
\pgfsetbuttcap%
\pgfsetroundjoin%
\definecolor{currentfill}{rgb}{0.121569,0.466667,0.705882}%
\pgfsetfillcolor{currentfill}%
\pgfsetfillopacity{0.549051}%
\pgfsetlinewidth{1.003750pt}%
\definecolor{currentstroke}{rgb}{0.121569,0.466667,0.705882}%
\pgfsetstrokecolor{currentstroke}%
\pgfsetstrokeopacity{0.549051}%
\pgfsetdash{}{0pt}%
\pgfpathmoveto{\pgfqpoint{2.677663in}{2.652853in}}%
\pgfpathcurveto{\pgfqpoint{2.685899in}{2.652853in}}{\pgfqpoint{2.693799in}{2.656125in}}{\pgfqpoint{2.699623in}{2.661949in}}%
\pgfpathcurveto{\pgfqpoint{2.705447in}{2.667773in}}{\pgfqpoint{2.708719in}{2.675673in}}{\pgfqpoint{2.708719in}{2.683910in}}%
\pgfpathcurveto{\pgfqpoint{2.708719in}{2.692146in}}{\pgfqpoint{2.705447in}{2.700046in}}{\pgfqpoint{2.699623in}{2.705870in}}%
\pgfpathcurveto{\pgfqpoint{2.693799in}{2.711694in}}{\pgfqpoint{2.685899in}{2.714966in}}{\pgfqpoint{2.677663in}{2.714966in}}%
\pgfpathcurveto{\pgfqpoint{2.669427in}{2.714966in}}{\pgfqpoint{2.661527in}{2.711694in}}{\pgfqpoint{2.655703in}{2.705870in}}%
\pgfpathcurveto{\pgfqpoint{2.649879in}{2.700046in}}{\pgfqpoint{2.646606in}{2.692146in}}{\pgfqpoint{2.646606in}{2.683910in}}%
\pgfpathcurveto{\pgfqpoint{2.646606in}{2.675673in}}{\pgfqpoint{2.649879in}{2.667773in}}{\pgfqpoint{2.655703in}{2.661949in}}%
\pgfpathcurveto{\pgfqpoint{2.661527in}{2.656125in}}{\pgfqpoint{2.669427in}{2.652853in}}{\pgfqpoint{2.677663in}{2.652853in}}%
\pgfpathclose%
\pgfusepath{stroke,fill}%
\end{pgfscope}%
\begin{pgfscope}%
\pgfpathrectangle{\pgfqpoint{0.100000in}{0.212622in}}{\pgfqpoint{3.696000in}{3.696000in}}%
\pgfusepath{clip}%
\pgfsetbuttcap%
\pgfsetroundjoin%
\definecolor{currentfill}{rgb}{0.121569,0.466667,0.705882}%
\pgfsetfillcolor{currentfill}%
\pgfsetfillopacity{0.553497}%
\pgfsetlinewidth{1.003750pt}%
\definecolor{currentstroke}{rgb}{0.121569,0.466667,0.705882}%
\pgfsetstrokecolor{currentstroke}%
\pgfsetstrokeopacity{0.553497}%
\pgfsetdash{}{0pt}%
\pgfpathmoveto{\pgfqpoint{2.697893in}{2.656698in}}%
\pgfpathcurveto{\pgfqpoint{2.706130in}{2.656698in}}{\pgfqpoint{2.714030in}{2.659970in}}{\pgfqpoint{2.719854in}{2.665794in}}%
\pgfpathcurveto{\pgfqpoint{2.725678in}{2.671618in}}{\pgfqpoint{2.728950in}{2.679518in}}{\pgfqpoint{2.728950in}{2.687754in}}%
\pgfpathcurveto{\pgfqpoint{2.728950in}{2.695990in}}{\pgfqpoint{2.725678in}{2.703890in}}{\pgfqpoint{2.719854in}{2.709714in}}%
\pgfpathcurveto{\pgfqpoint{2.714030in}{2.715538in}}{\pgfqpoint{2.706130in}{2.718811in}}{\pgfqpoint{2.697893in}{2.718811in}}%
\pgfpathcurveto{\pgfqpoint{2.689657in}{2.718811in}}{\pgfqpoint{2.681757in}{2.715538in}}{\pgfqpoint{2.675933in}{2.709714in}}%
\pgfpathcurveto{\pgfqpoint{2.670109in}{2.703890in}}{\pgfqpoint{2.666837in}{2.695990in}}{\pgfqpoint{2.666837in}{2.687754in}}%
\pgfpathcurveto{\pgfqpoint{2.666837in}{2.679518in}}{\pgfqpoint{2.670109in}{2.671618in}}{\pgfqpoint{2.675933in}{2.665794in}}%
\pgfpathcurveto{\pgfqpoint{2.681757in}{2.659970in}}{\pgfqpoint{2.689657in}{2.656698in}}{\pgfqpoint{2.697893in}{2.656698in}}%
\pgfpathclose%
\pgfusepath{stroke,fill}%
\end{pgfscope}%
\begin{pgfscope}%
\pgfpathrectangle{\pgfqpoint{0.100000in}{0.212622in}}{\pgfqpoint{3.696000in}{3.696000in}}%
\pgfusepath{clip}%
\pgfsetbuttcap%
\pgfsetroundjoin%
\definecolor{currentfill}{rgb}{0.121569,0.466667,0.705882}%
\pgfsetfillcolor{currentfill}%
\pgfsetfillopacity{0.557497}%
\pgfsetlinewidth{1.003750pt}%
\definecolor{currentstroke}{rgb}{0.121569,0.466667,0.705882}%
\pgfsetstrokecolor{currentstroke}%
\pgfsetstrokeopacity{0.557497}%
\pgfsetdash{}{0pt}%
\pgfpathmoveto{\pgfqpoint{2.715723in}{2.655008in}}%
\pgfpathcurveto{\pgfqpoint{2.723959in}{2.655008in}}{\pgfqpoint{2.731859in}{2.658280in}}{\pgfqpoint{2.737683in}{2.664104in}}%
\pgfpathcurveto{\pgfqpoint{2.743507in}{2.669928in}}{\pgfqpoint{2.746780in}{2.677828in}}{\pgfqpoint{2.746780in}{2.686064in}}%
\pgfpathcurveto{\pgfqpoint{2.746780in}{2.694301in}}{\pgfqpoint{2.743507in}{2.702201in}}{\pgfqpoint{2.737683in}{2.708025in}}%
\pgfpathcurveto{\pgfqpoint{2.731859in}{2.713849in}}{\pgfqpoint{2.723959in}{2.717121in}}{\pgfqpoint{2.715723in}{2.717121in}}%
\pgfpathcurveto{\pgfqpoint{2.707487in}{2.717121in}}{\pgfqpoint{2.699587in}{2.713849in}}{\pgfqpoint{2.693763in}{2.708025in}}%
\pgfpathcurveto{\pgfqpoint{2.687939in}{2.702201in}}{\pgfqpoint{2.684667in}{2.694301in}}{\pgfqpoint{2.684667in}{2.686064in}}%
\pgfpathcurveto{\pgfqpoint{2.684667in}{2.677828in}}{\pgfqpoint{2.687939in}{2.669928in}}{\pgfqpoint{2.693763in}{2.664104in}}%
\pgfpathcurveto{\pgfqpoint{2.699587in}{2.658280in}}{\pgfqpoint{2.707487in}{2.655008in}}{\pgfqpoint{2.715723in}{2.655008in}}%
\pgfpathclose%
\pgfusepath{stroke,fill}%
\end{pgfscope}%
\begin{pgfscope}%
\pgfpathrectangle{\pgfqpoint{0.100000in}{0.212622in}}{\pgfqpoint{3.696000in}{3.696000in}}%
\pgfusepath{clip}%
\pgfsetbuttcap%
\pgfsetroundjoin%
\definecolor{currentfill}{rgb}{0.121569,0.466667,0.705882}%
\pgfsetfillcolor{currentfill}%
\pgfsetfillopacity{0.561405}%
\pgfsetlinewidth{1.003750pt}%
\definecolor{currentstroke}{rgb}{0.121569,0.466667,0.705882}%
\pgfsetstrokecolor{currentstroke}%
\pgfsetstrokeopacity{0.561405}%
\pgfsetdash{}{0pt}%
\pgfpathmoveto{\pgfqpoint{2.731264in}{2.651949in}}%
\pgfpathcurveto{\pgfqpoint{2.739500in}{2.651949in}}{\pgfqpoint{2.747400in}{2.655221in}}{\pgfqpoint{2.753224in}{2.661045in}}%
\pgfpathcurveto{\pgfqpoint{2.759048in}{2.666869in}}{\pgfqpoint{2.762320in}{2.674769in}}{\pgfqpoint{2.762320in}{2.683005in}}%
\pgfpathcurveto{\pgfqpoint{2.762320in}{2.691241in}}{\pgfqpoint{2.759048in}{2.699141in}}{\pgfqpoint{2.753224in}{2.704965in}}%
\pgfpathcurveto{\pgfqpoint{2.747400in}{2.710789in}}{\pgfqpoint{2.739500in}{2.714062in}}{\pgfqpoint{2.731264in}{2.714062in}}%
\pgfpathcurveto{\pgfqpoint{2.723028in}{2.714062in}}{\pgfqpoint{2.715128in}{2.710789in}}{\pgfqpoint{2.709304in}{2.704965in}}%
\pgfpathcurveto{\pgfqpoint{2.703480in}{2.699141in}}{\pgfqpoint{2.700207in}{2.691241in}}{\pgfqpoint{2.700207in}{2.683005in}}%
\pgfpathcurveto{\pgfqpoint{2.700207in}{2.674769in}}{\pgfqpoint{2.703480in}{2.666869in}}{\pgfqpoint{2.709304in}{2.661045in}}%
\pgfpathcurveto{\pgfqpoint{2.715128in}{2.655221in}}{\pgfqpoint{2.723028in}{2.651949in}}{\pgfqpoint{2.731264in}{2.651949in}}%
\pgfpathclose%
\pgfusepath{stroke,fill}%
\end{pgfscope}%
\begin{pgfscope}%
\pgfpathrectangle{\pgfqpoint{0.100000in}{0.212622in}}{\pgfqpoint{3.696000in}{3.696000in}}%
\pgfusepath{clip}%
\pgfsetbuttcap%
\pgfsetroundjoin%
\definecolor{currentfill}{rgb}{0.121569,0.466667,0.705882}%
\pgfsetfillcolor{currentfill}%
\pgfsetfillopacity{0.564123}%
\pgfsetlinewidth{1.003750pt}%
\definecolor{currentstroke}{rgb}{0.121569,0.466667,0.705882}%
\pgfsetstrokecolor{currentstroke}%
\pgfsetstrokeopacity{0.564123}%
\pgfsetdash{}{0pt}%
\pgfpathmoveto{\pgfqpoint{2.747283in}{2.647115in}}%
\pgfpathcurveto{\pgfqpoint{2.755520in}{2.647115in}}{\pgfqpoint{2.763420in}{2.650388in}}{\pgfqpoint{2.769244in}{2.656212in}}%
\pgfpathcurveto{\pgfqpoint{2.775068in}{2.662036in}}{\pgfqpoint{2.778340in}{2.669936in}}{\pgfqpoint{2.778340in}{2.678172in}}%
\pgfpathcurveto{\pgfqpoint{2.778340in}{2.686408in}}{\pgfqpoint{2.775068in}{2.694308in}}{\pgfqpoint{2.769244in}{2.700132in}}%
\pgfpathcurveto{\pgfqpoint{2.763420in}{2.705956in}}{\pgfqpoint{2.755520in}{2.709228in}}{\pgfqpoint{2.747283in}{2.709228in}}%
\pgfpathcurveto{\pgfqpoint{2.739047in}{2.709228in}}{\pgfqpoint{2.731147in}{2.705956in}}{\pgfqpoint{2.725323in}{2.700132in}}%
\pgfpathcurveto{\pgfqpoint{2.719499in}{2.694308in}}{\pgfqpoint{2.716227in}{2.686408in}}{\pgfqpoint{2.716227in}{2.678172in}}%
\pgfpathcurveto{\pgfqpoint{2.716227in}{2.669936in}}{\pgfqpoint{2.719499in}{2.662036in}}{\pgfqpoint{2.725323in}{2.656212in}}%
\pgfpathcurveto{\pgfqpoint{2.731147in}{2.650388in}}{\pgfqpoint{2.739047in}{2.647115in}}{\pgfqpoint{2.747283in}{2.647115in}}%
\pgfpathclose%
\pgfusepath{stroke,fill}%
\end{pgfscope}%
\begin{pgfscope}%
\pgfpathrectangle{\pgfqpoint{0.100000in}{0.212622in}}{\pgfqpoint{3.696000in}{3.696000in}}%
\pgfusepath{clip}%
\pgfsetbuttcap%
\pgfsetroundjoin%
\definecolor{currentfill}{rgb}{0.121569,0.466667,0.705882}%
\pgfsetfillcolor{currentfill}%
\pgfsetfillopacity{0.566354}%
\pgfsetlinewidth{1.003750pt}%
\definecolor{currentstroke}{rgb}{0.121569,0.466667,0.705882}%
\pgfsetstrokecolor{currentstroke}%
\pgfsetstrokeopacity{0.566354}%
\pgfsetdash{}{0pt}%
\pgfpathmoveto{\pgfqpoint{2.762838in}{2.640506in}}%
\pgfpathcurveto{\pgfqpoint{2.771075in}{2.640506in}}{\pgfqpoint{2.778975in}{2.643779in}}{\pgfqpoint{2.784799in}{2.649603in}}%
\pgfpathcurveto{\pgfqpoint{2.790623in}{2.655426in}}{\pgfqpoint{2.793895in}{2.663327in}}{\pgfqpoint{2.793895in}{2.671563in}}%
\pgfpathcurveto{\pgfqpoint{2.793895in}{2.679799in}}{\pgfqpoint{2.790623in}{2.687699in}}{\pgfqpoint{2.784799in}{2.693523in}}%
\pgfpathcurveto{\pgfqpoint{2.778975in}{2.699347in}}{\pgfqpoint{2.771075in}{2.702619in}}{\pgfqpoint{2.762838in}{2.702619in}}%
\pgfpathcurveto{\pgfqpoint{2.754602in}{2.702619in}}{\pgfqpoint{2.746702in}{2.699347in}}{\pgfqpoint{2.740878in}{2.693523in}}%
\pgfpathcurveto{\pgfqpoint{2.735054in}{2.687699in}}{\pgfqpoint{2.731782in}{2.679799in}}{\pgfqpoint{2.731782in}{2.671563in}}%
\pgfpathcurveto{\pgfqpoint{2.731782in}{2.663327in}}{\pgfqpoint{2.735054in}{2.655426in}}{\pgfqpoint{2.740878in}{2.649603in}}%
\pgfpathcurveto{\pgfqpoint{2.746702in}{2.643779in}}{\pgfqpoint{2.754602in}{2.640506in}}{\pgfqpoint{2.762838in}{2.640506in}}%
\pgfpathclose%
\pgfusepath{stroke,fill}%
\end{pgfscope}%
\begin{pgfscope}%
\pgfpathrectangle{\pgfqpoint{0.100000in}{0.212622in}}{\pgfqpoint{3.696000in}{3.696000in}}%
\pgfusepath{clip}%
\pgfsetbuttcap%
\pgfsetroundjoin%
\definecolor{currentfill}{rgb}{0.121569,0.466667,0.705882}%
\pgfsetfillcolor{currentfill}%
\pgfsetfillopacity{0.568512}%
\pgfsetlinewidth{1.003750pt}%
\definecolor{currentstroke}{rgb}{0.121569,0.466667,0.705882}%
\pgfsetstrokecolor{currentstroke}%
\pgfsetstrokeopacity{0.568512}%
\pgfsetdash{}{0pt}%
\pgfpathmoveto{\pgfqpoint{2.779520in}{2.642025in}}%
\pgfpathcurveto{\pgfqpoint{2.787757in}{2.642025in}}{\pgfqpoint{2.795657in}{2.645298in}}{\pgfqpoint{2.801481in}{2.651121in}}%
\pgfpathcurveto{\pgfqpoint{2.807305in}{2.656945in}}{\pgfqpoint{2.810577in}{2.664845in}}{\pgfqpoint{2.810577in}{2.673082in}}%
\pgfpathcurveto{\pgfqpoint{2.810577in}{2.681318in}}{\pgfqpoint{2.807305in}{2.689218in}}{\pgfqpoint{2.801481in}{2.695042in}}%
\pgfpathcurveto{\pgfqpoint{2.795657in}{2.700866in}}{\pgfqpoint{2.787757in}{2.704138in}}{\pgfqpoint{2.779520in}{2.704138in}}%
\pgfpathcurveto{\pgfqpoint{2.771284in}{2.704138in}}{\pgfqpoint{2.763384in}{2.700866in}}{\pgfqpoint{2.757560in}{2.695042in}}%
\pgfpathcurveto{\pgfqpoint{2.751736in}{2.689218in}}{\pgfqpoint{2.748464in}{2.681318in}}{\pgfqpoint{2.748464in}{2.673082in}}%
\pgfpathcurveto{\pgfqpoint{2.748464in}{2.664845in}}{\pgfqpoint{2.751736in}{2.656945in}}{\pgfqpoint{2.757560in}{2.651121in}}%
\pgfpathcurveto{\pgfqpoint{2.763384in}{2.645298in}}{\pgfqpoint{2.771284in}{2.642025in}}{\pgfqpoint{2.779520in}{2.642025in}}%
\pgfpathclose%
\pgfusepath{stroke,fill}%
\end{pgfscope}%
\begin{pgfscope}%
\pgfpathrectangle{\pgfqpoint{0.100000in}{0.212622in}}{\pgfqpoint{3.696000in}{3.696000in}}%
\pgfusepath{clip}%
\pgfsetbuttcap%
\pgfsetroundjoin%
\definecolor{currentfill}{rgb}{0.121569,0.466667,0.705882}%
\pgfsetfillcolor{currentfill}%
\pgfsetfillopacity{0.572845}%
\pgfsetlinewidth{1.003750pt}%
\definecolor{currentstroke}{rgb}{0.121569,0.466667,0.705882}%
\pgfsetstrokecolor{currentstroke}%
\pgfsetstrokeopacity{0.572845}%
\pgfsetdash{}{0pt}%
\pgfpathmoveto{\pgfqpoint{2.790248in}{2.642452in}}%
\pgfpathcurveto{\pgfqpoint{2.798484in}{2.642452in}}{\pgfqpoint{2.806384in}{2.645724in}}{\pgfqpoint{2.812208in}{2.651548in}}%
\pgfpathcurveto{\pgfqpoint{2.818032in}{2.657372in}}{\pgfqpoint{2.821304in}{2.665272in}}{\pgfqpoint{2.821304in}{2.673508in}}%
\pgfpathcurveto{\pgfqpoint{2.821304in}{2.681745in}}{\pgfqpoint{2.818032in}{2.689645in}}{\pgfqpoint{2.812208in}{2.695469in}}%
\pgfpathcurveto{\pgfqpoint{2.806384in}{2.701292in}}{\pgfqpoint{2.798484in}{2.704565in}}{\pgfqpoint{2.790248in}{2.704565in}}%
\pgfpathcurveto{\pgfqpoint{2.782012in}{2.704565in}}{\pgfqpoint{2.774112in}{2.701292in}}{\pgfqpoint{2.768288in}{2.695469in}}%
\pgfpathcurveto{\pgfqpoint{2.762464in}{2.689645in}}{\pgfqpoint{2.759191in}{2.681745in}}{\pgfqpoint{2.759191in}{2.673508in}}%
\pgfpathcurveto{\pgfqpoint{2.759191in}{2.665272in}}{\pgfqpoint{2.762464in}{2.657372in}}{\pgfqpoint{2.768288in}{2.651548in}}%
\pgfpathcurveto{\pgfqpoint{2.774112in}{2.645724in}}{\pgfqpoint{2.782012in}{2.642452in}}{\pgfqpoint{2.790248in}{2.642452in}}%
\pgfpathclose%
\pgfusepath{stroke,fill}%
\end{pgfscope}%
\begin{pgfscope}%
\pgfpathrectangle{\pgfqpoint{0.100000in}{0.212622in}}{\pgfqpoint{3.696000in}{3.696000in}}%
\pgfusepath{clip}%
\pgfsetbuttcap%
\pgfsetroundjoin%
\definecolor{currentfill}{rgb}{0.121569,0.466667,0.705882}%
\pgfsetfillcolor{currentfill}%
\pgfsetfillopacity{0.576313}%
\pgfsetlinewidth{1.003750pt}%
\definecolor{currentstroke}{rgb}{0.121569,0.466667,0.705882}%
\pgfsetstrokecolor{currentstroke}%
\pgfsetstrokeopacity{0.576313}%
\pgfsetdash{}{0pt}%
\pgfpathmoveto{\pgfqpoint{2.800166in}{2.645977in}}%
\pgfpathcurveto{\pgfqpoint{2.808402in}{2.645977in}}{\pgfqpoint{2.816302in}{2.649249in}}{\pgfqpoint{2.822126in}{2.655073in}}%
\pgfpathcurveto{\pgfqpoint{2.827950in}{2.660897in}}{\pgfqpoint{2.831222in}{2.668797in}}{\pgfqpoint{2.831222in}{2.677033in}}%
\pgfpathcurveto{\pgfqpoint{2.831222in}{2.685270in}}{\pgfqpoint{2.827950in}{2.693170in}}{\pgfqpoint{2.822126in}{2.698994in}}%
\pgfpathcurveto{\pgfqpoint{2.816302in}{2.704818in}}{\pgfqpoint{2.808402in}{2.708090in}}{\pgfqpoint{2.800166in}{2.708090in}}%
\pgfpathcurveto{\pgfqpoint{2.791929in}{2.708090in}}{\pgfqpoint{2.784029in}{2.704818in}}{\pgfqpoint{2.778205in}{2.698994in}}%
\pgfpathcurveto{\pgfqpoint{2.772382in}{2.693170in}}{\pgfqpoint{2.769109in}{2.685270in}}{\pgfqpoint{2.769109in}{2.677033in}}%
\pgfpathcurveto{\pgfqpoint{2.769109in}{2.668797in}}{\pgfqpoint{2.772382in}{2.660897in}}{\pgfqpoint{2.778205in}{2.655073in}}%
\pgfpathcurveto{\pgfqpoint{2.784029in}{2.649249in}}{\pgfqpoint{2.791929in}{2.645977in}}{\pgfqpoint{2.800166in}{2.645977in}}%
\pgfpathclose%
\pgfusepath{stroke,fill}%
\end{pgfscope}%
\begin{pgfscope}%
\pgfpathrectangle{\pgfqpoint{0.100000in}{0.212622in}}{\pgfqpoint{3.696000in}{3.696000in}}%
\pgfusepath{clip}%
\pgfsetbuttcap%
\pgfsetroundjoin%
\definecolor{currentfill}{rgb}{0.121569,0.466667,0.705882}%
\pgfsetfillcolor{currentfill}%
\pgfsetfillopacity{0.579286}%
\pgfsetlinewidth{1.003750pt}%
\definecolor{currentstroke}{rgb}{0.121569,0.466667,0.705882}%
\pgfsetstrokecolor{currentstroke}%
\pgfsetstrokeopacity{0.579286}%
\pgfsetdash{}{0pt}%
\pgfpathmoveto{\pgfqpoint{2.809359in}{2.647885in}}%
\pgfpathcurveto{\pgfqpoint{2.817596in}{2.647885in}}{\pgfqpoint{2.825496in}{2.651157in}}{\pgfqpoint{2.831320in}{2.656981in}}%
\pgfpathcurveto{\pgfqpoint{2.837143in}{2.662805in}}{\pgfqpoint{2.840416in}{2.670705in}}{\pgfqpoint{2.840416in}{2.678941in}}%
\pgfpathcurveto{\pgfqpoint{2.840416in}{2.687178in}}{\pgfqpoint{2.837143in}{2.695078in}}{\pgfqpoint{2.831320in}{2.700902in}}%
\pgfpathcurveto{\pgfqpoint{2.825496in}{2.706726in}}{\pgfqpoint{2.817596in}{2.709998in}}{\pgfqpoint{2.809359in}{2.709998in}}%
\pgfpathcurveto{\pgfqpoint{2.801123in}{2.709998in}}{\pgfqpoint{2.793223in}{2.706726in}}{\pgfqpoint{2.787399in}{2.700902in}}%
\pgfpathcurveto{\pgfqpoint{2.781575in}{2.695078in}}{\pgfqpoint{2.778303in}{2.687178in}}{\pgfqpoint{2.778303in}{2.678941in}}%
\pgfpathcurveto{\pgfqpoint{2.778303in}{2.670705in}}{\pgfqpoint{2.781575in}{2.662805in}}{\pgfqpoint{2.787399in}{2.656981in}}%
\pgfpathcurveto{\pgfqpoint{2.793223in}{2.651157in}}{\pgfqpoint{2.801123in}{2.647885in}}{\pgfqpoint{2.809359in}{2.647885in}}%
\pgfpathclose%
\pgfusepath{stroke,fill}%
\end{pgfscope}%
\begin{pgfscope}%
\pgfpathrectangle{\pgfqpoint{0.100000in}{0.212622in}}{\pgfqpoint{3.696000in}{3.696000in}}%
\pgfusepath{clip}%
\pgfsetbuttcap%
\pgfsetroundjoin%
\definecolor{currentfill}{rgb}{0.121569,0.466667,0.705882}%
\pgfsetfillcolor{currentfill}%
\pgfsetfillopacity{0.580974}%
\pgfsetlinewidth{1.003750pt}%
\definecolor{currentstroke}{rgb}{0.121569,0.466667,0.705882}%
\pgfsetstrokecolor{currentstroke}%
\pgfsetstrokeopacity{0.580974}%
\pgfsetdash{}{0pt}%
\pgfpathmoveto{\pgfqpoint{2.818674in}{2.647242in}}%
\pgfpathcurveto{\pgfqpoint{2.826910in}{2.647242in}}{\pgfqpoint{2.834810in}{2.650514in}}{\pgfqpoint{2.840634in}{2.656338in}}%
\pgfpathcurveto{\pgfqpoint{2.846458in}{2.662162in}}{\pgfqpoint{2.849731in}{2.670062in}}{\pgfqpoint{2.849731in}{2.678299in}}%
\pgfpathcurveto{\pgfqpoint{2.849731in}{2.686535in}}{\pgfqpoint{2.846458in}{2.694435in}}{\pgfqpoint{2.840634in}{2.700259in}}%
\pgfpathcurveto{\pgfqpoint{2.834810in}{2.706083in}}{\pgfqpoint{2.826910in}{2.709355in}}{\pgfqpoint{2.818674in}{2.709355in}}%
\pgfpathcurveto{\pgfqpoint{2.810438in}{2.709355in}}{\pgfqpoint{2.802538in}{2.706083in}}{\pgfqpoint{2.796714in}{2.700259in}}%
\pgfpathcurveto{\pgfqpoint{2.790890in}{2.694435in}}{\pgfqpoint{2.787618in}{2.686535in}}{\pgfqpoint{2.787618in}{2.678299in}}%
\pgfpathcurveto{\pgfqpoint{2.787618in}{2.670062in}}{\pgfqpoint{2.790890in}{2.662162in}}{\pgfqpoint{2.796714in}{2.656338in}}%
\pgfpathcurveto{\pgfqpoint{2.802538in}{2.650514in}}{\pgfqpoint{2.810438in}{2.647242in}}{\pgfqpoint{2.818674in}{2.647242in}}%
\pgfpathclose%
\pgfusepath{stroke,fill}%
\end{pgfscope}%
\begin{pgfscope}%
\pgfpathrectangle{\pgfqpoint{0.100000in}{0.212622in}}{\pgfqpoint{3.696000in}{3.696000in}}%
\pgfusepath{clip}%
\pgfsetbuttcap%
\pgfsetroundjoin%
\definecolor{currentfill}{rgb}{0.121569,0.466667,0.705882}%
\pgfsetfillcolor{currentfill}%
\pgfsetfillopacity{0.582331}%
\pgfsetlinewidth{1.003750pt}%
\definecolor{currentstroke}{rgb}{0.121569,0.466667,0.705882}%
\pgfsetstrokecolor{currentstroke}%
\pgfsetstrokeopacity{0.582331}%
\pgfsetdash{}{0pt}%
\pgfpathmoveto{\pgfqpoint{2.826425in}{2.645785in}}%
\pgfpathcurveto{\pgfqpoint{2.834661in}{2.645785in}}{\pgfqpoint{2.842561in}{2.649058in}}{\pgfqpoint{2.848385in}{2.654882in}}%
\pgfpathcurveto{\pgfqpoint{2.854209in}{2.660706in}}{\pgfqpoint{2.857481in}{2.668606in}}{\pgfqpoint{2.857481in}{2.676842in}}%
\pgfpathcurveto{\pgfqpoint{2.857481in}{2.685078in}}{\pgfqpoint{2.854209in}{2.692978in}}{\pgfqpoint{2.848385in}{2.698802in}}%
\pgfpathcurveto{\pgfqpoint{2.842561in}{2.704626in}}{\pgfqpoint{2.834661in}{2.707898in}}{\pgfqpoint{2.826425in}{2.707898in}}%
\pgfpathcurveto{\pgfqpoint{2.818189in}{2.707898in}}{\pgfqpoint{2.810289in}{2.704626in}}{\pgfqpoint{2.804465in}{2.698802in}}%
\pgfpathcurveto{\pgfqpoint{2.798641in}{2.692978in}}{\pgfqpoint{2.795368in}{2.685078in}}{\pgfqpoint{2.795368in}{2.676842in}}%
\pgfpathcurveto{\pgfqpoint{2.795368in}{2.668606in}}{\pgfqpoint{2.798641in}{2.660706in}}{\pgfqpoint{2.804465in}{2.654882in}}%
\pgfpathcurveto{\pgfqpoint{2.810289in}{2.649058in}}{\pgfqpoint{2.818189in}{2.645785in}}{\pgfqpoint{2.826425in}{2.645785in}}%
\pgfpathclose%
\pgfusepath{stroke,fill}%
\end{pgfscope}%
\begin{pgfscope}%
\pgfpathrectangle{\pgfqpoint{0.100000in}{0.212622in}}{\pgfqpoint{3.696000in}{3.696000in}}%
\pgfusepath{clip}%
\pgfsetbuttcap%
\pgfsetroundjoin%
\definecolor{currentfill}{rgb}{0.121569,0.466667,0.705882}%
\pgfsetfillcolor{currentfill}%
\pgfsetfillopacity{0.583260}%
\pgfsetlinewidth{1.003750pt}%
\definecolor{currentstroke}{rgb}{0.121569,0.466667,0.705882}%
\pgfsetstrokecolor{currentstroke}%
\pgfsetstrokeopacity{0.583260}%
\pgfsetdash{}{0pt}%
\pgfpathmoveto{\pgfqpoint{2.832764in}{2.644383in}}%
\pgfpathcurveto{\pgfqpoint{2.841001in}{2.644383in}}{\pgfqpoint{2.848901in}{2.647655in}}{\pgfqpoint{2.854725in}{2.653479in}}%
\pgfpathcurveto{\pgfqpoint{2.860549in}{2.659303in}}{\pgfqpoint{2.863821in}{2.667203in}}{\pgfqpoint{2.863821in}{2.675440in}}%
\pgfpathcurveto{\pgfqpoint{2.863821in}{2.683676in}}{\pgfqpoint{2.860549in}{2.691576in}}{\pgfqpoint{2.854725in}{2.697400in}}%
\pgfpathcurveto{\pgfqpoint{2.848901in}{2.703224in}}{\pgfqpoint{2.841001in}{2.706496in}}{\pgfqpoint{2.832764in}{2.706496in}}%
\pgfpathcurveto{\pgfqpoint{2.824528in}{2.706496in}}{\pgfqpoint{2.816628in}{2.703224in}}{\pgfqpoint{2.810804in}{2.697400in}}%
\pgfpathcurveto{\pgfqpoint{2.804980in}{2.691576in}}{\pgfqpoint{2.801708in}{2.683676in}}{\pgfqpoint{2.801708in}{2.675440in}}%
\pgfpathcurveto{\pgfqpoint{2.801708in}{2.667203in}}{\pgfqpoint{2.804980in}{2.659303in}}{\pgfqpoint{2.810804in}{2.653479in}}%
\pgfpathcurveto{\pgfqpoint{2.816628in}{2.647655in}}{\pgfqpoint{2.824528in}{2.644383in}}{\pgfqpoint{2.832764in}{2.644383in}}%
\pgfpathclose%
\pgfusepath{stroke,fill}%
\end{pgfscope}%
\begin{pgfscope}%
\pgfpathrectangle{\pgfqpoint{0.100000in}{0.212622in}}{\pgfqpoint{3.696000in}{3.696000in}}%
\pgfusepath{clip}%
\pgfsetbuttcap%
\pgfsetroundjoin%
\definecolor{currentfill}{rgb}{0.121569,0.466667,0.705882}%
\pgfsetfillcolor{currentfill}%
\pgfsetfillopacity{0.584887}%
\pgfsetlinewidth{1.003750pt}%
\definecolor{currentstroke}{rgb}{0.121569,0.466667,0.705882}%
\pgfsetstrokecolor{currentstroke}%
\pgfsetstrokeopacity{0.584887}%
\pgfsetdash{}{0pt}%
\pgfpathmoveto{\pgfqpoint{2.844247in}{2.641384in}}%
\pgfpathcurveto{\pgfqpoint{2.852484in}{2.641384in}}{\pgfqpoint{2.860384in}{2.644656in}}{\pgfqpoint{2.866208in}{2.650480in}}%
\pgfpathcurveto{\pgfqpoint{2.872032in}{2.656304in}}{\pgfqpoint{2.875304in}{2.664204in}}{\pgfqpoint{2.875304in}{2.672441in}}%
\pgfpathcurveto{\pgfqpoint{2.875304in}{2.680677in}}{\pgfqpoint{2.872032in}{2.688577in}}{\pgfqpoint{2.866208in}{2.694401in}}%
\pgfpathcurveto{\pgfqpoint{2.860384in}{2.700225in}}{\pgfqpoint{2.852484in}{2.703497in}}{\pgfqpoint{2.844247in}{2.703497in}}%
\pgfpathcurveto{\pgfqpoint{2.836011in}{2.703497in}}{\pgfqpoint{2.828111in}{2.700225in}}{\pgfqpoint{2.822287in}{2.694401in}}%
\pgfpathcurveto{\pgfqpoint{2.816463in}{2.688577in}}{\pgfqpoint{2.813191in}{2.680677in}}{\pgfqpoint{2.813191in}{2.672441in}}%
\pgfpathcurveto{\pgfqpoint{2.813191in}{2.664204in}}{\pgfqpoint{2.816463in}{2.656304in}}{\pgfqpoint{2.822287in}{2.650480in}}%
\pgfpathcurveto{\pgfqpoint{2.828111in}{2.644656in}}{\pgfqpoint{2.836011in}{2.641384in}}{\pgfqpoint{2.844247in}{2.641384in}}%
\pgfpathclose%
\pgfusepath{stroke,fill}%
\end{pgfscope}%
\begin{pgfscope}%
\pgfpathrectangle{\pgfqpoint{0.100000in}{0.212622in}}{\pgfqpoint{3.696000in}{3.696000in}}%
\pgfusepath{clip}%
\pgfsetbuttcap%
\pgfsetroundjoin%
\definecolor{currentfill}{rgb}{0.121569,0.466667,0.705882}%
\pgfsetfillcolor{currentfill}%
\pgfsetfillopacity{0.586592}%
\pgfsetlinewidth{1.003750pt}%
\definecolor{currentstroke}{rgb}{0.121569,0.466667,0.705882}%
\pgfsetstrokecolor{currentstroke}%
\pgfsetstrokeopacity{0.586592}%
\pgfsetdash{}{0pt}%
\pgfpathmoveto{\pgfqpoint{2.854781in}{2.638832in}}%
\pgfpathcurveto{\pgfqpoint{2.863017in}{2.638832in}}{\pgfqpoint{2.870917in}{2.642104in}}{\pgfqpoint{2.876741in}{2.647928in}}%
\pgfpathcurveto{\pgfqpoint{2.882565in}{2.653752in}}{\pgfqpoint{2.885837in}{2.661652in}}{\pgfqpoint{2.885837in}{2.669889in}}%
\pgfpathcurveto{\pgfqpoint{2.885837in}{2.678125in}}{\pgfqpoint{2.882565in}{2.686025in}}{\pgfqpoint{2.876741in}{2.691849in}}%
\pgfpathcurveto{\pgfqpoint{2.870917in}{2.697673in}}{\pgfqpoint{2.863017in}{2.700945in}}{\pgfqpoint{2.854781in}{2.700945in}}%
\pgfpathcurveto{\pgfqpoint{2.846544in}{2.700945in}}{\pgfqpoint{2.838644in}{2.697673in}}{\pgfqpoint{2.832820in}{2.691849in}}%
\pgfpathcurveto{\pgfqpoint{2.826996in}{2.686025in}}{\pgfqpoint{2.823724in}{2.678125in}}{\pgfqpoint{2.823724in}{2.669889in}}%
\pgfpathcurveto{\pgfqpoint{2.823724in}{2.661652in}}{\pgfqpoint{2.826996in}{2.653752in}}{\pgfqpoint{2.832820in}{2.647928in}}%
\pgfpathcurveto{\pgfqpoint{2.838644in}{2.642104in}}{\pgfqpoint{2.846544in}{2.638832in}}{\pgfqpoint{2.854781in}{2.638832in}}%
\pgfpathclose%
\pgfusepath{stroke,fill}%
\end{pgfscope}%
\begin{pgfscope}%
\pgfpathrectangle{\pgfqpoint{0.100000in}{0.212622in}}{\pgfqpoint{3.696000in}{3.696000in}}%
\pgfusepath{clip}%
\pgfsetbuttcap%
\pgfsetroundjoin%
\definecolor{currentfill}{rgb}{0.121569,0.466667,0.705882}%
\pgfsetfillcolor{currentfill}%
\pgfsetfillopacity{0.588110}%
\pgfsetlinewidth{1.003750pt}%
\definecolor{currentstroke}{rgb}{0.121569,0.466667,0.705882}%
\pgfsetstrokecolor{currentstroke}%
\pgfsetstrokeopacity{0.588110}%
\pgfsetdash{}{0pt}%
\pgfpathmoveto{\pgfqpoint{2.874542in}{2.629823in}}%
\pgfpathcurveto{\pgfqpoint{2.882778in}{2.629823in}}{\pgfqpoint{2.890678in}{2.633095in}}{\pgfqpoint{2.896502in}{2.638919in}}%
\pgfpathcurveto{\pgfqpoint{2.902326in}{2.644743in}}{\pgfqpoint{2.905598in}{2.652643in}}{\pgfqpoint{2.905598in}{2.660879in}}%
\pgfpathcurveto{\pgfqpoint{2.905598in}{2.669116in}}{\pgfqpoint{2.902326in}{2.677016in}}{\pgfqpoint{2.896502in}{2.682840in}}%
\pgfpathcurveto{\pgfqpoint{2.890678in}{2.688664in}}{\pgfqpoint{2.882778in}{2.691936in}}{\pgfqpoint{2.874542in}{2.691936in}}%
\pgfpathcurveto{\pgfqpoint{2.866306in}{2.691936in}}{\pgfqpoint{2.858406in}{2.688664in}}{\pgfqpoint{2.852582in}{2.682840in}}%
\pgfpathcurveto{\pgfqpoint{2.846758in}{2.677016in}}{\pgfqpoint{2.843485in}{2.669116in}}{\pgfqpoint{2.843485in}{2.660879in}}%
\pgfpathcurveto{\pgfqpoint{2.843485in}{2.652643in}}{\pgfqpoint{2.846758in}{2.644743in}}{\pgfqpoint{2.852582in}{2.638919in}}%
\pgfpathcurveto{\pgfqpoint{2.858406in}{2.633095in}}{\pgfqpoint{2.866306in}{2.629823in}}{\pgfqpoint{2.874542in}{2.629823in}}%
\pgfpathclose%
\pgfusepath{stroke,fill}%
\end{pgfscope}%
\begin{pgfscope}%
\pgfpathrectangle{\pgfqpoint{0.100000in}{0.212622in}}{\pgfqpoint{3.696000in}{3.696000in}}%
\pgfusepath{clip}%
\pgfsetbuttcap%
\pgfsetroundjoin%
\definecolor{currentfill}{rgb}{0.121569,0.466667,0.705882}%
\pgfsetfillcolor{currentfill}%
\pgfsetfillopacity{0.591818}%
\pgfsetlinewidth{1.003750pt}%
\definecolor{currentstroke}{rgb}{0.121569,0.466667,0.705882}%
\pgfsetstrokecolor{currentstroke}%
\pgfsetstrokeopacity{0.591818}%
\pgfsetdash{}{0pt}%
\pgfpathmoveto{\pgfqpoint{2.892676in}{2.630273in}}%
\pgfpathcurveto{\pgfqpoint{2.900913in}{2.630273in}}{\pgfqpoint{2.908813in}{2.633546in}}{\pgfqpoint{2.914637in}{2.639370in}}%
\pgfpathcurveto{\pgfqpoint{2.920461in}{2.645193in}}{\pgfqpoint{2.923733in}{2.653094in}}{\pgfqpoint{2.923733in}{2.661330in}}%
\pgfpathcurveto{\pgfqpoint{2.923733in}{2.669566in}}{\pgfqpoint{2.920461in}{2.677466in}}{\pgfqpoint{2.914637in}{2.683290in}}%
\pgfpathcurveto{\pgfqpoint{2.908813in}{2.689114in}}{\pgfqpoint{2.900913in}{2.692386in}}{\pgfqpoint{2.892676in}{2.692386in}}%
\pgfpathcurveto{\pgfqpoint{2.884440in}{2.692386in}}{\pgfqpoint{2.876540in}{2.689114in}}{\pgfqpoint{2.870716in}{2.683290in}}%
\pgfpathcurveto{\pgfqpoint{2.864892in}{2.677466in}}{\pgfqpoint{2.861620in}{2.669566in}}{\pgfqpoint{2.861620in}{2.661330in}}%
\pgfpathcurveto{\pgfqpoint{2.861620in}{2.653094in}}{\pgfqpoint{2.864892in}{2.645193in}}{\pgfqpoint{2.870716in}{2.639370in}}%
\pgfpathcurveto{\pgfqpoint{2.876540in}{2.633546in}}{\pgfqpoint{2.884440in}{2.630273in}}{\pgfqpoint{2.892676in}{2.630273in}}%
\pgfpathclose%
\pgfusepath{stroke,fill}%
\end{pgfscope}%
\begin{pgfscope}%
\pgfpathrectangle{\pgfqpoint{0.100000in}{0.212622in}}{\pgfqpoint{3.696000in}{3.696000in}}%
\pgfusepath{clip}%
\pgfsetbuttcap%
\pgfsetroundjoin%
\definecolor{currentfill}{rgb}{0.121569,0.466667,0.705882}%
\pgfsetfillcolor{currentfill}%
\pgfsetfillopacity{0.596956}%
\pgfsetlinewidth{1.003750pt}%
\definecolor{currentstroke}{rgb}{0.121569,0.466667,0.705882}%
\pgfsetstrokecolor{currentstroke}%
\pgfsetstrokeopacity{0.596956}%
\pgfsetdash{}{0pt}%
\pgfpathmoveto{\pgfqpoint{2.907502in}{2.635434in}}%
\pgfpathcurveto{\pgfqpoint{2.915739in}{2.635434in}}{\pgfqpoint{2.923639in}{2.638706in}}{\pgfqpoint{2.929463in}{2.644530in}}%
\pgfpathcurveto{\pgfqpoint{2.935287in}{2.650354in}}{\pgfqpoint{2.938559in}{2.658254in}}{\pgfqpoint{2.938559in}{2.666491in}}%
\pgfpathcurveto{\pgfqpoint{2.938559in}{2.674727in}}{\pgfqpoint{2.935287in}{2.682627in}}{\pgfqpoint{2.929463in}{2.688451in}}%
\pgfpathcurveto{\pgfqpoint{2.923639in}{2.694275in}}{\pgfqpoint{2.915739in}{2.697547in}}{\pgfqpoint{2.907502in}{2.697547in}}%
\pgfpathcurveto{\pgfqpoint{2.899266in}{2.697547in}}{\pgfqpoint{2.891366in}{2.694275in}}{\pgfqpoint{2.885542in}{2.688451in}}%
\pgfpathcurveto{\pgfqpoint{2.879718in}{2.682627in}}{\pgfqpoint{2.876446in}{2.674727in}}{\pgfqpoint{2.876446in}{2.666491in}}%
\pgfpathcurveto{\pgfqpoint{2.876446in}{2.658254in}}{\pgfqpoint{2.879718in}{2.650354in}}{\pgfqpoint{2.885542in}{2.644530in}}%
\pgfpathcurveto{\pgfqpoint{2.891366in}{2.638706in}}{\pgfqpoint{2.899266in}{2.635434in}}{\pgfqpoint{2.907502in}{2.635434in}}%
\pgfpathclose%
\pgfusepath{stroke,fill}%
\end{pgfscope}%
\begin{pgfscope}%
\pgfpathrectangle{\pgfqpoint{0.100000in}{0.212622in}}{\pgfqpoint{3.696000in}{3.696000in}}%
\pgfusepath{clip}%
\pgfsetbuttcap%
\pgfsetroundjoin%
\definecolor{currentfill}{rgb}{0.121569,0.466667,0.705882}%
\pgfsetfillcolor{currentfill}%
\pgfsetfillopacity{0.599388}%
\pgfsetlinewidth{1.003750pt}%
\definecolor{currentstroke}{rgb}{0.121569,0.466667,0.705882}%
\pgfsetstrokecolor{currentstroke}%
\pgfsetstrokeopacity{0.599388}%
\pgfsetdash{}{0pt}%
\pgfpathmoveto{\pgfqpoint{1.718913in}{3.073096in}}%
\pgfpathcurveto{\pgfqpoint{1.727149in}{3.073096in}}{\pgfqpoint{1.735050in}{3.076368in}}{\pgfqpoint{1.740873in}{3.082192in}}%
\pgfpathcurveto{\pgfqpoint{1.746697in}{3.088016in}}{\pgfqpoint{1.749970in}{3.095916in}}{\pgfqpoint{1.749970in}{3.104152in}}%
\pgfpathcurveto{\pgfqpoint{1.749970in}{3.112389in}}{\pgfqpoint{1.746697in}{3.120289in}}{\pgfqpoint{1.740873in}{3.126113in}}%
\pgfpathcurveto{\pgfqpoint{1.735050in}{3.131936in}}{\pgfqpoint{1.727149in}{3.135209in}}{\pgfqpoint{1.718913in}{3.135209in}}%
\pgfpathcurveto{\pgfqpoint{1.710677in}{3.135209in}}{\pgfqpoint{1.702777in}{3.131936in}}{\pgfqpoint{1.696953in}{3.126113in}}%
\pgfpathcurveto{\pgfqpoint{1.691129in}{3.120289in}}{\pgfqpoint{1.687857in}{3.112389in}}{\pgfqpoint{1.687857in}{3.104152in}}%
\pgfpathcurveto{\pgfqpoint{1.687857in}{3.095916in}}{\pgfqpoint{1.691129in}{3.088016in}}{\pgfqpoint{1.696953in}{3.082192in}}%
\pgfpathcurveto{\pgfqpoint{1.702777in}{3.076368in}}{\pgfqpoint{1.710677in}{3.073096in}}{\pgfqpoint{1.718913in}{3.073096in}}%
\pgfpathclose%
\pgfusepath{stroke,fill}%
\end{pgfscope}%
\begin{pgfscope}%
\pgfpathrectangle{\pgfqpoint{0.100000in}{0.212622in}}{\pgfqpoint{3.696000in}{3.696000in}}%
\pgfusepath{clip}%
\pgfsetbuttcap%
\pgfsetroundjoin%
\definecolor{currentfill}{rgb}{0.121569,0.466667,0.705882}%
\pgfsetfillcolor{currentfill}%
\pgfsetfillopacity{0.599388}%
\pgfsetlinewidth{1.003750pt}%
\definecolor{currentstroke}{rgb}{0.121569,0.466667,0.705882}%
\pgfsetstrokecolor{currentstroke}%
\pgfsetstrokeopacity{0.599388}%
\pgfsetdash{}{0pt}%
\pgfpathmoveto{\pgfqpoint{1.718913in}{3.073096in}}%
\pgfpathcurveto{\pgfqpoint{1.727149in}{3.073096in}}{\pgfqpoint{1.735050in}{3.076368in}}{\pgfqpoint{1.740873in}{3.082192in}}%
\pgfpathcurveto{\pgfqpoint{1.746697in}{3.088016in}}{\pgfqpoint{1.749970in}{3.095916in}}{\pgfqpoint{1.749970in}{3.104152in}}%
\pgfpathcurveto{\pgfqpoint{1.749970in}{3.112389in}}{\pgfqpoint{1.746697in}{3.120289in}}{\pgfqpoint{1.740873in}{3.126113in}}%
\pgfpathcurveto{\pgfqpoint{1.735050in}{3.131936in}}{\pgfqpoint{1.727149in}{3.135209in}}{\pgfqpoint{1.718913in}{3.135209in}}%
\pgfpathcurveto{\pgfqpoint{1.710677in}{3.135209in}}{\pgfqpoint{1.702777in}{3.131936in}}{\pgfqpoint{1.696953in}{3.126113in}}%
\pgfpathcurveto{\pgfqpoint{1.691129in}{3.120289in}}{\pgfqpoint{1.687857in}{3.112389in}}{\pgfqpoint{1.687857in}{3.104152in}}%
\pgfpathcurveto{\pgfqpoint{1.687857in}{3.095916in}}{\pgfqpoint{1.691129in}{3.088016in}}{\pgfqpoint{1.696953in}{3.082192in}}%
\pgfpathcurveto{\pgfqpoint{1.702777in}{3.076368in}}{\pgfqpoint{1.710677in}{3.073096in}}{\pgfqpoint{1.718913in}{3.073096in}}%
\pgfpathclose%
\pgfusepath{stroke,fill}%
\end{pgfscope}%
\begin{pgfscope}%
\pgfpathrectangle{\pgfqpoint{0.100000in}{0.212622in}}{\pgfqpoint{3.696000in}{3.696000in}}%
\pgfusepath{clip}%
\pgfsetbuttcap%
\pgfsetroundjoin%
\definecolor{currentfill}{rgb}{0.121569,0.466667,0.705882}%
\pgfsetfillcolor{currentfill}%
\pgfsetfillopacity{0.599388}%
\pgfsetlinewidth{1.003750pt}%
\definecolor{currentstroke}{rgb}{0.121569,0.466667,0.705882}%
\pgfsetstrokecolor{currentstroke}%
\pgfsetstrokeopacity{0.599388}%
\pgfsetdash{}{0pt}%
\pgfpathmoveto{\pgfqpoint{1.718913in}{3.073096in}}%
\pgfpathcurveto{\pgfqpoint{1.727149in}{3.073096in}}{\pgfqpoint{1.735050in}{3.076368in}}{\pgfqpoint{1.740873in}{3.082192in}}%
\pgfpathcurveto{\pgfqpoint{1.746697in}{3.088016in}}{\pgfqpoint{1.749970in}{3.095916in}}{\pgfqpoint{1.749970in}{3.104152in}}%
\pgfpathcurveto{\pgfqpoint{1.749970in}{3.112389in}}{\pgfqpoint{1.746697in}{3.120289in}}{\pgfqpoint{1.740873in}{3.126113in}}%
\pgfpathcurveto{\pgfqpoint{1.735050in}{3.131936in}}{\pgfqpoint{1.727149in}{3.135209in}}{\pgfqpoint{1.718913in}{3.135209in}}%
\pgfpathcurveto{\pgfqpoint{1.710677in}{3.135209in}}{\pgfqpoint{1.702777in}{3.131936in}}{\pgfqpoint{1.696953in}{3.126113in}}%
\pgfpathcurveto{\pgfqpoint{1.691129in}{3.120289in}}{\pgfqpoint{1.687857in}{3.112389in}}{\pgfqpoint{1.687857in}{3.104152in}}%
\pgfpathcurveto{\pgfqpoint{1.687857in}{3.095916in}}{\pgfqpoint{1.691129in}{3.088016in}}{\pgfqpoint{1.696953in}{3.082192in}}%
\pgfpathcurveto{\pgfqpoint{1.702777in}{3.076368in}}{\pgfqpoint{1.710677in}{3.073096in}}{\pgfqpoint{1.718913in}{3.073096in}}%
\pgfpathclose%
\pgfusepath{stroke,fill}%
\end{pgfscope}%
\begin{pgfscope}%
\pgfpathrectangle{\pgfqpoint{0.100000in}{0.212622in}}{\pgfqpoint{3.696000in}{3.696000in}}%
\pgfusepath{clip}%
\pgfsetbuttcap%
\pgfsetroundjoin%
\definecolor{currentfill}{rgb}{0.121569,0.466667,0.705882}%
\pgfsetfillcolor{currentfill}%
\pgfsetfillopacity{0.599388}%
\pgfsetlinewidth{1.003750pt}%
\definecolor{currentstroke}{rgb}{0.121569,0.466667,0.705882}%
\pgfsetstrokecolor{currentstroke}%
\pgfsetstrokeopacity{0.599388}%
\pgfsetdash{}{0pt}%
\pgfpathmoveto{\pgfqpoint{1.718913in}{3.073096in}}%
\pgfpathcurveto{\pgfqpoint{1.727149in}{3.073096in}}{\pgfqpoint{1.735050in}{3.076368in}}{\pgfqpoint{1.740873in}{3.082192in}}%
\pgfpathcurveto{\pgfqpoint{1.746697in}{3.088016in}}{\pgfqpoint{1.749970in}{3.095916in}}{\pgfqpoint{1.749970in}{3.104152in}}%
\pgfpathcurveto{\pgfqpoint{1.749970in}{3.112389in}}{\pgfqpoint{1.746697in}{3.120289in}}{\pgfqpoint{1.740873in}{3.126113in}}%
\pgfpathcurveto{\pgfqpoint{1.735050in}{3.131936in}}{\pgfqpoint{1.727149in}{3.135209in}}{\pgfqpoint{1.718913in}{3.135209in}}%
\pgfpathcurveto{\pgfqpoint{1.710677in}{3.135209in}}{\pgfqpoint{1.702777in}{3.131936in}}{\pgfqpoint{1.696953in}{3.126113in}}%
\pgfpathcurveto{\pgfqpoint{1.691129in}{3.120289in}}{\pgfqpoint{1.687857in}{3.112389in}}{\pgfqpoint{1.687857in}{3.104152in}}%
\pgfpathcurveto{\pgfqpoint{1.687857in}{3.095916in}}{\pgfqpoint{1.691129in}{3.088016in}}{\pgfqpoint{1.696953in}{3.082192in}}%
\pgfpathcurveto{\pgfqpoint{1.702777in}{3.076368in}}{\pgfqpoint{1.710677in}{3.073096in}}{\pgfqpoint{1.718913in}{3.073096in}}%
\pgfpathclose%
\pgfusepath{stroke,fill}%
\end{pgfscope}%
\begin{pgfscope}%
\pgfpathrectangle{\pgfqpoint{0.100000in}{0.212622in}}{\pgfqpoint{3.696000in}{3.696000in}}%
\pgfusepath{clip}%
\pgfsetbuttcap%
\pgfsetroundjoin%
\definecolor{currentfill}{rgb}{0.121569,0.466667,0.705882}%
\pgfsetfillcolor{currentfill}%
\pgfsetfillopacity{0.599388}%
\pgfsetlinewidth{1.003750pt}%
\definecolor{currentstroke}{rgb}{0.121569,0.466667,0.705882}%
\pgfsetstrokecolor{currentstroke}%
\pgfsetstrokeopacity{0.599388}%
\pgfsetdash{}{0pt}%
\pgfpathmoveto{\pgfqpoint{1.718913in}{3.073096in}}%
\pgfpathcurveto{\pgfqpoint{1.727149in}{3.073096in}}{\pgfqpoint{1.735050in}{3.076368in}}{\pgfqpoint{1.740873in}{3.082192in}}%
\pgfpathcurveto{\pgfqpoint{1.746697in}{3.088016in}}{\pgfqpoint{1.749970in}{3.095916in}}{\pgfqpoint{1.749970in}{3.104152in}}%
\pgfpathcurveto{\pgfqpoint{1.749970in}{3.112389in}}{\pgfqpoint{1.746697in}{3.120289in}}{\pgfqpoint{1.740873in}{3.126113in}}%
\pgfpathcurveto{\pgfqpoint{1.735050in}{3.131936in}}{\pgfqpoint{1.727149in}{3.135209in}}{\pgfqpoint{1.718913in}{3.135209in}}%
\pgfpathcurveto{\pgfqpoint{1.710677in}{3.135209in}}{\pgfqpoint{1.702777in}{3.131936in}}{\pgfqpoint{1.696953in}{3.126113in}}%
\pgfpathcurveto{\pgfqpoint{1.691129in}{3.120289in}}{\pgfqpoint{1.687857in}{3.112389in}}{\pgfqpoint{1.687857in}{3.104152in}}%
\pgfpathcurveto{\pgfqpoint{1.687857in}{3.095916in}}{\pgfqpoint{1.691129in}{3.088016in}}{\pgfqpoint{1.696953in}{3.082192in}}%
\pgfpathcurveto{\pgfqpoint{1.702777in}{3.076368in}}{\pgfqpoint{1.710677in}{3.073096in}}{\pgfqpoint{1.718913in}{3.073096in}}%
\pgfpathclose%
\pgfusepath{stroke,fill}%
\end{pgfscope}%
\begin{pgfscope}%
\pgfpathrectangle{\pgfqpoint{0.100000in}{0.212622in}}{\pgfqpoint{3.696000in}{3.696000in}}%
\pgfusepath{clip}%
\pgfsetbuttcap%
\pgfsetroundjoin%
\definecolor{currentfill}{rgb}{0.121569,0.466667,0.705882}%
\pgfsetfillcolor{currentfill}%
\pgfsetfillopacity{0.599388}%
\pgfsetlinewidth{1.003750pt}%
\definecolor{currentstroke}{rgb}{0.121569,0.466667,0.705882}%
\pgfsetstrokecolor{currentstroke}%
\pgfsetstrokeopacity{0.599388}%
\pgfsetdash{}{0pt}%
\pgfpathmoveto{\pgfqpoint{1.718913in}{3.073096in}}%
\pgfpathcurveto{\pgfqpoint{1.727149in}{3.073096in}}{\pgfqpoint{1.735050in}{3.076368in}}{\pgfqpoint{1.740873in}{3.082192in}}%
\pgfpathcurveto{\pgfqpoint{1.746697in}{3.088016in}}{\pgfqpoint{1.749970in}{3.095916in}}{\pgfqpoint{1.749970in}{3.104152in}}%
\pgfpathcurveto{\pgfqpoint{1.749970in}{3.112389in}}{\pgfqpoint{1.746697in}{3.120289in}}{\pgfqpoint{1.740873in}{3.126113in}}%
\pgfpathcurveto{\pgfqpoint{1.735050in}{3.131936in}}{\pgfqpoint{1.727149in}{3.135209in}}{\pgfqpoint{1.718913in}{3.135209in}}%
\pgfpathcurveto{\pgfqpoint{1.710677in}{3.135209in}}{\pgfqpoint{1.702777in}{3.131936in}}{\pgfqpoint{1.696953in}{3.126113in}}%
\pgfpathcurveto{\pgfqpoint{1.691129in}{3.120289in}}{\pgfqpoint{1.687857in}{3.112389in}}{\pgfqpoint{1.687857in}{3.104152in}}%
\pgfpathcurveto{\pgfqpoint{1.687857in}{3.095916in}}{\pgfqpoint{1.691129in}{3.088016in}}{\pgfqpoint{1.696953in}{3.082192in}}%
\pgfpathcurveto{\pgfqpoint{1.702777in}{3.076368in}}{\pgfqpoint{1.710677in}{3.073096in}}{\pgfqpoint{1.718913in}{3.073096in}}%
\pgfpathclose%
\pgfusepath{stroke,fill}%
\end{pgfscope}%
\begin{pgfscope}%
\pgfpathrectangle{\pgfqpoint{0.100000in}{0.212622in}}{\pgfqpoint{3.696000in}{3.696000in}}%
\pgfusepath{clip}%
\pgfsetbuttcap%
\pgfsetroundjoin%
\definecolor{currentfill}{rgb}{0.121569,0.466667,0.705882}%
\pgfsetfillcolor{currentfill}%
\pgfsetfillopacity{0.599388}%
\pgfsetlinewidth{1.003750pt}%
\definecolor{currentstroke}{rgb}{0.121569,0.466667,0.705882}%
\pgfsetstrokecolor{currentstroke}%
\pgfsetstrokeopacity{0.599388}%
\pgfsetdash{}{0pt}%
\pgfpathmoveto{\pgfqpoint{1.718913in}{3.073096in}}%
\pgfpathcurveto{\pgfqpoint{1.727149in}{3.073096in}}{\pgfqpoint{1.735050in}{3.076368in}}{\pgfqpoint{1.740873in}{3.082192in}}%
\pgfpathcurveto{\pgfqpoint{1.746697in}{3.088016in}}{\pgfqpoint{1.749970in}{3.095916in}}{\pgfqpoint{1.749970in}{3.104152in}}%
\pgfpathcurveto{\pgfqpoint{1.749970in}{3.112389in}}{\pgfqpoint{1.746697in}{3.120289in}}{\pgfqpoint{1.740873in}{3.126113in}}%
\pgfpathcurveto{\pgfqpoint{1.735050in}{3.131936in}}{\pgfqpoint{1.727149in}{3.135209in}}{\pgfqpoint{1.718913in}{3.135209in}}%
\pgfpathcurveto{\pgfqpoint{1.710677in}{3.135209in}}{\pgfqpoint{1.702777in}{3.131936in}}{\pgfqpoint{1.696953in}{3.126113in}}%
\pgfpathcurveto{\pgfqpoint{1.691129in}{3.120289in}}{\pgfqpoint{1.687857in}{3.112389in}}{\pgfqpoint{1.687857in}{3.104152in}}%
\pgfpathcurveto{\pgfqpoint{1.687857in}{3.095916in}}{\pgfqpoint{1.691129in}{3.088016in}}{\pgfqpoint{1.696953in}{3.082192in}}%
\pgfpathcurveto{\pgfqpoint{1.702777in}{3.076368in}}{\pgfqpoint{1.710677in}{3.073096in}}{\pgfqpoint{1.718913in}{3.073096in}}%
\pgfpathclose%
\pgfusepath{stroke,fill}%
\end{pgfscope}%
\begin{pgfscope}%
\pgfpathrectangle{\pgfqpoint{0.100000in}{0.212622in}}{\pgfqpoint{3.696000in}{3.696000in}}%
\pgfusepath{clip}%
\pgfsetbuttcap%
\pgfsetroundjoin%
\definecolor{currentfill}{rgb}{0.121569,0.466667,0.705882}%
\pgfsetfillcolor{currentfill}%
\pgfsetfillopacity{0.599388}%
\pgfsetlinewidth{1.003750pt}%
\definecolor{currentstroke}{rgb}{0.121569,0.466667,0.705882}%
\pgfsetstrokecolor{currentstroke}%
\pgfsetstrokeopacity{0.599388}%
\pgfsetdash{}{0pt}%
\pgfpathmoveto{\pgfqpoint{1.718913in}{3.073096in}}%
\pgfpathcurveto{\pgfqpoint{1.727149in}{3.073096in}}{\pgfqpoint{1.735050in}{3.076368in}}{\pgfqpoint{1.740873in}{3.082192in}}%
\pgfpathcurveto{\pgfqpoint{1.746697in}{3.088016in}}{\pgfqpoint{1.749970in}{3.095916in}}{\pgfqpoint{1.749970in}{3.104152in}}%
\pgfpathcurveto{\pgfqpoint{1.749970in}{3.112389in}}{\pgfqpoint{1.746697in}{3.120289in}}{\pgfqpoint{1.740873in}{3.126113in}}%
\pgfpathcurveto{\pgfqpoint{1.735050in}{3.131936in}}{\pgfqpoint{1.727149in}{3.135209in}}{\pgfqpoint{1.718913in}{3.135209in}}%
\pgfpathcurveto{\pgfqpoint{1.710677in}{3.135209in}}{\pgfqpoint{1.702777in}{3.131936in}}{\pgfqpoint{1.696953in}{3.126113in}}%
\pgfpathcurveto{\pgfqpoint{1.691129in}{3.120289in}}{\pgfqpoint{1.687857in}{3.112389in}}{\pgfqpoint{1.687857in}{3.104152in}}%
\pgfpathcurveto{\pgfqpoint{1.687857in}{3.095916in}}{\pgfqpoint{1.691129in}{3.088016in}}{\pgfqpoint{1.696953in}{3.082192in}}%
\pgfpathcurveto{\pgfqpoint{1.702777in}{3.076368in}}{\pgfqpoint{1.710677in}{3.073096in}}{\pgfqpoint{1.718913in}{3.073096in}}%
\pgfpathclose%
\pgfusepath{stroke,fill}%
\end{pgfscope}%
\begin{pgfscope}%
\pgfpathrectangle{\pgfqpoint{0.100000in}{0.212622in}}{\pgfqpoint{3.696000in}{3.696000in}}%
\pgfusepath{clip}%
\pgfsetbuttcap%
\pgfsetroundjoin%
\definecolor{currentfill}{rgb}{0.121569,0.466667,0.705882}%
\pgfsetfillcolor{currentfill}%
\pgfsetfillopacity{0.599388}%
\pgfsetlinewidth{1.003750pt}%
\definecolor{currentstroke}{rgb}{0.121569,0.466667,0.705882}%
\pgfsetstrokecolor{currentstroke}%
\pgfsetstrokeopacity{0.599388}%
\pgfsetdash{}{0pt}%
\pgfpathmoveto{\pgfqpoint{1.718913in}{3.073096in}}%
\pgfpathcurveto{\pgfqpoint{1.727149in}{3.073096in}}{\pgfqpoint{1.735050in}{3.076368in}}{\pgfqpoint{1.740873in}{3.082192in}}%
\pgfpathcurveto{\pgfqpoint{1.746697in}{3.088016in}}{\pgfqpoint{1.749970in}{3.095916in}}{\pgfqpoint{1.749970in}{3.104152in}}%
\pgfpathcurveto{\pgfqpoint{1.749970in}{3.112389in}}{\pgfqpoint{1.746697in}{3.120289in}}{\pgfqpoint{1.740873in}{3.126113in}}%
\pgfpathcurveto{\pgfqpoint{1.735050in}{3.131936in}}{\pgfqpoint{1.727149in}{3.135209in}}{\pgfqpoint{1.718913in}{3.135209in}}%
\pgfpathcurveto{\pgfqpoint{1.710677in}{3.135209in}}{\pgfqpoint{1.702777in}{3.131936in}}{\pgfqpoint{1.696953in}{3.126113in}}%
\pgfpathcurveto{\pgfqpoint{1.691129in}{3.120289in}}{\pgfqpoint{1.687857in}{3.112389in}}{\pgfqpoint{1.687857in}{3.104152in}}%
\pgfpathcurveto{\pgfqpoint{1.687857in}{3.095916in}}{\pgfqpoint{1.691129in}{3.088016in}}{\pgfqpoint{1.696953in}{3.082192in}}%
\pgfpathcurveto{\pgfqpoint{1.702777in}{3.076368in}}{\pgfqpoint{1.710677in}{3.073096in}}{\pgfqpoint{1.718913in}{3.073096in}}%
\pgfpathclose%
\pgfusepath{stroke,fill}%
\end{pgfscope}%
\begin{pgfscope}%
\pgfpathrectangle{\pgfqpoint{0.100000in}{0.212622in}}{\pgfqpoint{3.696000in}{3.696000in}}%
\pgfusepath{clip}%
\pgfsetbuttcap%
\pgfsetroundjoin%
\definecolor{currentfill}{rgb}{0.121569,0.466667,0.705882}%
\pgfsetfillcolor{currentfill}%
\pgfsetfillopacity{0.599388}%
\pgfsetlinewidth{1.003750pt}%
\definecolor{currentstroke}{rgb}{0.121569,0.466667,0.705882}%
\pgfsetstrokecolor{currentstroke}%
\pgfsetstrokeopacity{0.599388}%
\pgfsetdash{}{0pt}%
\pgfpathmoveto{\pgfqpoint{1.718913in}{3.073096in}}%
\pgfpathcurveto{\pgfqpoint{1.727149in}{3.073096in}}{\pgfqpoint{1.735050in}{3.076368in}}{\pgfqpoint{1.740873in}{3.082192in}}%
\pgfpathcurveto{\pgfqpoint{1.746697in}{3.088016in}}{\pgfqpoint{1.749970in}{3.095916in}}{\pgfqpoint{1.749970in}{3.104152in}}%
\pgfpathcurveto{\pgfqpoint{1.749970in}{3.112389in}}{\pgfqpoint{1.746697in}{3.120289in}}{\pgfqpoint{1.740873in}{3.126112in}}%
\pgfpathcurveto{\pgfqpoint{1.735050in}{3.131936in}}{\pgfqpoint{1.727149in}{3.135209in}}{\pgfqpoint{1.718913in}{3.135209in}}%
\pgfpathcurveto{\pgfqpoint{1.710677in}{3.135209in}}{\pgfqpoint{1.702777in}{3.131936in}}{\pgfqpoint{1.696953in}{3.126112in}}%
\pgfpathcurveto{\pgfqpoint{1.691129in}{3.120289in}}{\pgfqpoint{1.687857in}{3.112389in}}{\pgfqpoint{1.687857in}{3.104152in}}%
\pgfpathcurveto{\pgfqpoint{1.687857in}{3.095916in}}{\pgfqpoint{1.691129in}{3.088016in}}{\pgfqpoint{1.696953in}{3.082192in}}%
\pgfpathcurveto{\pgfqpoint{1.702777in}{3.076368in}}{\pgfqpoint{1.710677in}{3.073096in}}{\pgfqpoint{1.718913in}{3.073096in}}%
\pgfpathclose%
\pgfusepath{stroke,fill}%
\end{pgfscope}%
\begin{pgfscope}%
\pgfpathrectangle{\pgfqpoint{0.100000in}{0.212622in}}{\pgfqpoint{3.696000in}{3.696000in}}%
\pgfusepath{clip}%
\pgfsetbuttcap%
\pgfsetroundjoin%
\definecolor{currentfill}{rgb}{0.121569,0.466667,0.705882}%
\pgfsetfillcolor{currentfill}%
\pgfsetfillopacity{0.599388}%
\pgfsetlinewidth{1.003750pt}%
\definecolor{currentstroke}{rgb}{0.121569,0.466667,0.705882}%
\pgfsetstrokecolor{currentstroke}%
\pgfsetstrokeopacity{0.599388}%
\pgfsetdash{}{0pt}%
\pgfpathmoveto{\pgfqpoint{1.718913in}{3.073096in}}%
\pgfpathcurveto{\pgfqpoint{1.727149in}{3.073096in}}{\pgfqpoint{1.735050in}{3.076368in}}{\pgfqpoint{1.740873in}{3.082192in}}%
\pgfpathcurveto{\pgfqpoint{1.746697in}{3.088016in}}{\pgfqpoint{1.749970in}{3.095916in}}{\pgfqpoint{1.749970in}{3.104152in}}%
\pgfpathcurveto{\pgfqpoint{1.749970in}{3.112389in}}{\pgfqpoint{1.746697in}{3.120289in}}{\pgfqpoint{1.740873in}{3.126112in}}%
\pgfpathcurveto{\pgfqpoint{1.735050in}{3.131936in}}{\pgfqpoint{1.727149in}{3.135209in}}{\pgfqpoint{1.718913in}{3.135209in}}%
\pgfpathcurveto{\pgfqpoint{1.710677in}{3.135209in}}{\pgfqpoint{1.702777in}{3.131936in}}{\pgfqpoint{1.696953in}{3.126112in}}%
\pgfpathcurveto{\pgfqpoint{1.691129in}{3.120289in}}{\pgfqpoint{1.687857in}{3.112389in}}{\pgfqpoint{1.687857in}{3.104152in}}%
\pgfpathcurveto{\pgfqpoint{1.687857in}{3.095916in}}{\pgfqpoint{1.691129in}{3.088016in}}{\pgfqpoint{1.696953in}{3.082192in}}%
\pgfpathcurveto{\pgfqpoint{1.702777in}{3.076368in}}{\pgfqpoint{1.710677in}{3.073096in}}{\pgfqpoint{1.718913in}{3.073096in}}%
\pgfpathclose%
\pgfusepath{stroke,fill}%
\end{pgfscope}%
\begin{pgfscope}%
\pgfpathrectangle{\pgfqpoint{0.100000in}{0.212622in}}{\pgfqpoint{3.696000in}{3.696000in}}%
\pgfusepath{clip}%
\pgfsetbuttcap%
\pgfsetroundjoin%
\definecolor{currentfill}{rgb}{0.121569,0.466667,0.705882}%
\pgfsetfillcolor{currentfill}%
\pgfsetfillopacity{0.599388}%
\pgfsetlinewidth{1.003750pt}%
\definecolor{currentstroke}{rgb}{0.121569,0.466667,0.705882}%
\pgfsetstrokecolor{currentstroke}%
\pgfsetstrokeopacity{0.599388}%
\pgfsetdash{}{0pt}%
\pgfpathmoveto{\pgfqpoint{1.718913in}{3.073096in}}%
\pgfpathcurveto{\pgfqpoint{1.727150in}{3.073096in}}{\pgfqpoint{1.735050in}{3.076368in}}{\pgfqpoint{1.740874in}{3.082192in}}%
\pgfpathcurveto{\pgfqpoint{1.746697in}{3.088016in}}{\pgfqpoint{1.749970in}{3.095916in}}{\pgfqpoint{1.749970in}{3.104152in}}%
\pgfpathcurveto{\pgfqpoint{1.749970in}{3.112389in}}{\pgfqpoint{1.746697in}{3.120289in}}{\pgfqpoint{1.740874in}{3.126112in}}%
\pgfpathcurveto{\pgfqpoint{1.735050in}{3.131936in}}{\pgfqpoint{1.727150in}{3.135209in}}{\pgfqpoint{1.718913in}{3.135209in}}%
\pgfpathcurveto{\pgfqpoint{1.710677in}{3.135209in}}{\pgfqpoint{1.702777in}{3.131936in}}{\pgfqpoint{1.696953in}{3.126112in}}%
\pgfpathcurveto{\pgfqpoint{1.691129in}{3.120289in}}{\pgfqpoint{1.687857in}{3.112389in}}{\pgfqpoint{1.687857in}{3.104152in}}%
\pgfpathcurveto{\pgfqpoint{1.687857in}{3.095916in}}{\pgfqpoint{1.691129in}{3.088016in}}{\pgfqpoint{1.696953in}{3.082192in}}%
\pgfpathcurveto{\pgfqpoint{1.702777in}{3.076368in}}{\pgfqpoint{1.710677in}{3.073096in}}{\pgfqpoint{1.718913in}{3.073096in}}%
\pgfpathclose%
\pgfusepath{stroke,fill}%
\end{pgfscope}%
\begin{pgfscope}%
\pgfpathrectangle{\pgfqpoint{0.100000in}{0.212622in}}{\pgfqpoint{3.696000in}{3.696000in}}%
\pgfusepath{clip}%
\pgfsetbuttcap%
\pgfsetroundjoin%
\definecolor{currentfill}{rgb}{0.121569,0.466667,0.705882}%
\pgfsetfillcolor{currentfill}%
\pgfsetfillopacity{0.599388}%
\pgfsetlinewidth{1.003750pt}%
\definecolor{currentstroke}{rgb}{0.121569,0.466667,0.705882}%
\pgfsetstrokecolor{currentstroke}%
\pgfsetstrokeopacity{0.599388}%
\pgfsetdash{}{0pt}%
\pgfpathmoveto{\pgfqpoint{1.718913in}{3.073096in}}%
\pgfpathcurveto{\pgfqpoint{1.727150in}{3.073096in}}{\pgfqpoint{1.735050in}{3.076368in}}{\pgfqpoint{1.740874in}{3.082192in}}%
\pgfpathcurveto{\pgfqpoint{1.746698in}{3.088016in}}{\pgfqpoint{1.749970in}{3.095916in}}{\pgfqpoint{1.749970in}{3.104152in}}%
\pgfpathcurveto{\pgfqpoint{1.749970in}{3.112389in}}{\pgfqpoint{1.746698in}{3.120289in}}{\pgfqpoint{1.740874in}{3.126112in}}%
\pgfpathcurveto{\pgfqpoint{1.735050in}{3.131936in}}{\pgfqpoint{1.727150in}{3.135209in}}{\pgfqpoint{1.718913in}{3.135209in}}%
\pgfpathcurveto{\pgfqpoint{1.710677in}{3.135209in}}{\pgfqpoint{1.702777in}{3.131936in}}{\pgfqpoint{1.696953in}{3.126112in}}%
\pgfpathcurveto{\pgfqpoint{1.691129in}{3.120289in}}{\pgfqpoint{1.687857in}{3.112389in}}{\pgfqpoint{1.687857in}{3.104152in}}%
\pgfpathcurveto{\pgfqpoint{1.687857in}{3.095916in}}{\pgfqpoint{1.691129in}{3.088016in}}{\pgfqpoint{1.696953in}{3.082192in}}%
\pgfpathcurveto{\pgfqpoint{1.702777in}{3.076368in}}{\pgfqpoint{1.710677in}{3.073096in}}{\pgfqpoint{1.718913in}{3.073096in}}%
\pgfpathclose%
\pgfusepath{stroke,fill}%
\end{pgfscope}%
\begin{pgfscope}%
\pgfpathrectangle{\pgfqpoint{0.100000in}{0.212622in}}{\pgfqpoint{3.696000in}{3.696000in}}%
\pgfusepath{clip}%
\pgfsetbuttcap%
\pgfsetroundjoin%
\definecolor{currentfill}{rgb}{0.121569,0.466667,0.705882}%
\pgfsetfillcolor{currentfill}%
\pgfsetfillopacity{0.599388}%
\pgfsetlinewidth{1.003750pt}%
\definecolor{currentstroke}{rgb}{0.121569,0.466667,0.705882}%
\pgfsetstrokecolor{currentstroke}%
\pgfsetstrokeopacity{0.599388}%
\pgfsetdash{}{0pt}%
\pgfpathmoveto{\pgfqpoint{1.718913in}{3.073096in}}%
\pgfpathcurveto{\pgfqpoint{1.727150in}{3.073096in}}{\pgfqpoint{1.735050in}{3.076368in}}{\pgfqpoint{1.740874in}{3.082192in}}%
\pgfpathcurveto{\pgfqpoint{1.746698in}{3.088016in}}{\pgfqpoint{1.749970in}{3.095916in}}{\pgfqpoint{1.749970in}{3.104152in}}%
\pgfpathcurveto{\pgfqpoint{1.749970in}{3.112388in}}{\pgfqpoint{1.746698in}{3.120289in}}{\pgfqpoint{1.740874in}{3.126112in}}%
\pgfpathcurveto{\pgfqpoint{1.735050in}{3.131936in}}{\pgfqpoint{1.727150in}{3.135209in}}{\pgfqpoint{1.718913in}{3.135209in}}%
\pgfpathcurveto{\pgfqpoint{1.710677in}{3.135209in}}{\pgfqpoint{1.702777in}{3.131936in}}{\pgfqpoint{1.696953in}{3.126112in}}%
\pgfpathcurveto{\pgfqpoint{1.691129in}{3.120289in}}{\pgfqpoint{1.687857in}{3.112388in}}{\pgfqpoint{1.687857in}{3.104152in}}%
\pgfpathcurveto{\pgfqpoint{1.687857in}{3.095916in}}{\pgfqpoint{1.691129in}{3.088016in}}{\pgfqpoint{1.696953in}{3.082192in}}%
\pgfpathcurveto{\pgfqpoint{1.702777in}{3.076368in}}{\pgfqpoint{1.710677in}{3.073096in}}{\pgfqpoint{1.718913in}{3.073096in}}%
\pgfpathclose%
\pgfusepath{stroke,fill}%
\end{pgfscope}%
\begin{pgfscope}%
\pgfpathrectangle{\pgfqpoint{0.100000in}{0.212622in}}{\pgfqpoint{3.696000in}{3.696000in}}%
\pgfusepath{clip}%
\pgfsetbuttcap%
\pgfsetroundjoin%
\definecolor{currentfill}{rgb}{0.121569,0.466667,0.705882}%
\pgfsetfillcolor{currentfill}%
\pgfsetfillopacity{0.599388}%
\pgfsetlinewidth{1.003750pt}%
\definecolor{currentstroke}{rgb}{0.121569,0.466667,0.705882}%
\pgfsetstrokecolor{currentstroke}%
\pgfsetstrokeopacity{0.599388}%
\pgfsetdash{}{0pt}%
\pgfpathmoveto{\pgfqpoint{1.718914in}{3.073096in}}%
\pgfpathcurveto{\pgfqpoint{1.727150in}{3.073096in}}{\pgfqpoint{1.735050in}{3.076368in}}{\pgfqpoint{1.740874in}{3.082192in}}%
\pgfpathcurveto{\pgfqpoint{1.746698in}{3.088016in}}{\pgfqpoint{1.749970in}{3.095916in}}{\pgfqpoint{1.749970in}{3.104152in}}%
\pgfpathcurveto{\pgfqpoint{1.749970in}{3.112388in}}{\pgfqpoint{1.746698in}{3.120289in}}{\pgfqpoint{1.740874in}{3.126112in}}%
\pgfpathcurveto{\pgfqpoint{1.735050in}{3.131936in}}{\pgfqpoint{1.727150in}{3.135209in}}{\pgfqpoint{1.718914in}{3.135209in}}%
\pgfpathcurveto{\pgfqpoint{1.710677in}{3.135209in}}{\pgfqpoint{1.702777in}{3.131936in}}{\pgfqpoint{1.696953in}{3.126112in}}%
\pgfpathcurveto{\pgfqpoint{1.691129in}{3.120289in}}{\pgfqpoint{1.687857in}{3.112388in}}{\pgfqpoint{1.687857in}{3.104152in}}%
\pgfpathcurveto{\pgfqpoint{1.687857in}{3.095916in}}{\pgfqpoint{1.691129in}{3.088016in}}{\pgfqpoint{1.696953in}{3.082192in}}%
\pgfpathcurveto{\pgfqpoint{1.702777in}{3.076368in}}{\pgfqpoint{1.710677in}{3.073096in}}{\pgfqpoint{1.718914in}{3.073096in}}%
\pgfpathclose%
\pgfusepath{stroke,fill}%
\end{pgfscope}%
\begin{pgfscope}%
\pgfpathrectangle{\pgfqpoint{0.100000in}{0.212622in}}{\pgfqpoint{3.696000in}{3.696000in}}%
\pgfusepath{clip}%
\pgfsetbuttcap%
\pgfsetroundjoin%
\definecolor{currentfill}{rgb}{0.121569,0.466667,0.705882}%
\pgfsetfillcolor{currentfill}%
\pgfsetfillopacity{0.599388}%
\pgfsetlinewidth{1.003750pt}%
\definecolor{currentstroke}{rgb}{0.121569,0.466667,0.705882}%
\pgfsetstrokecolor{currentstroke}%
\pgfsetstrokeopacity{0.599388}%
\pgfsetdash{}{0pt}%
\pgfpathmoveto{\pgfqpoint{1.718914in}{3.073096in}}%
\pgfpathcurveto{\pgfqpoint{1.727150in}{3.073096in}}{\pgfqpoint{1.735050in}{3.076368in}}{\pgfqpoint{1.740874in}{3.082192in}}%
\pgfpathcurveto{\pgfqpoint{1.746698in}{3.088016in}}{\pgfqpoint{1.749970in}{3.095916in}}{\pgfqpoint{1.749970in}{3.104152in}}%
\pgfpathcurveto{\pgfqpoint{1.749970in}{3.112388in}}{\pgfqpoint{1.746698in}{3.120288in}}{\pgfqpoint{1.740874in}{3.126112in}}%
\pgfpathcurveto{\pgfqpoint{1.735050in}{3.131936in}}{\pgfqpoint{1.727150in}{3.135209in}}{\pgfqpoint{1.718914in}{3.135209in}}%
\pgfpathcurveto{\pgfqpoint{1.710678in}{3.135209in}}{\pgfqpoint{1.702778in}{3.131936in}}{\pgfqpoint{1.696954in}{3.126112in}}%
\pgfpathcurveto{\pgfqpoint{1.691130in}{3.120288in}}{\pgfqpoint{1.687857in}{3.112388in}}{\pgfqpoint{1.687857in}{3.104152in}}%
\pgfpathcurveto{\pgfqpoint{1.687857in}{3.095916in}}{\pgfqpoint{1.691130in}{3.088016in}}{\pgfqpoint{1.696954in}{3.082192in}}%
\pgfpathcurveto{\pgfqpoint{1.702778in}{3.076368in}}{\pgfqpoint{1.710678in}{3.073096in}}{\pgfqpoint{1.718914in}{3.073096in}}%
\pgfpathclose%
\pgfusepath{stroke,fill}%
\end{pgfscope}%
\begin{pgfscope}%
\pgfpathrectangle{\pgfqpoint{0.100000in}{0.212622in}}{\pgfqpoint{3.696000in}{3.696000in}}%
\pgfusepath{clip}%
\pgfsetbuttcap%
\pgfsetroundjoin%
\definecolor{currentfill}{rgb}{0.121569,0.466667,0.705882}%
\pgfsetfillcolor{currentfill}%
\pgfsetfillopacity{0.599388}%
\pgfsetlinewidth{1.003750pt}%
\definecolor{currentstroke}{rgb}{0.121569,0.466667,0.705882}%
\pgfsetstrokecolor{currentstroke}%
\pgfsetstrokeopacity{0.599388}%
\pgfsetdash{}{0pt}%
\pgfpathmoveto{\pgfqpoint{1.718915in}{3.073096in}}%
\pgfpathcurveto{\pgfqpoint{1.727151in}{3.073096in}}{\pgfqpoint{1.735051in}{3.076368in}}{\pgfqpoint{1.740875in}{3.082192in}}%
\pgfpathcurveto{\pgfqpoint{1.746699in}{3.088016in}}{\pgfqpoint{1.749971in}{3.095916in}}{\pgfqpoint{1.749971in}{3.104152in}}%
\pgfpathcurveto{\pgfqpoint{1.749971in}{3.112388in}}{\pgfqpoint{1.746699in}{3.120288in}}{\pgfqpoint{1.740875in}{3.126112in}}%
\pgfpathcurveto{\pgfqpoint{1.735051in}{3.131936in}}{\pgfqpoint{1.727151in}{3.135209in}}{\pgfqpoint{1.718915in}{3.135209in}}%
\pgfpathcurveto{\pgfqpoint{1.710678in}{3.135209in}}{\pgfqpoint{1.702778in}{3.131936in}}{\pgfqpoint{1.696954in}{3.126112in}}%
\pgfpathcurveto{\pgfqpoint{1.691130in}{3.120288in}}{\pgfqpoint{1.687858in}{3.112388in}}{\pgfqpoint{1.687858in}{3.104152in}}%
\pgfpathcurveto{\pgfqpoint{1.687858in}{3.095916in}}{\pgfqpoint{1.691130in}{3.088016in}}{\pgfqpoint{1.696954in}{3.082192in}}%
\pgfpathcurveto{\pgfqpoint{1.702778in}{3.076368in}}{\pgfqpoint{1.710678in}{3.073096in}}{\pgfqpoint{1.718915in}{3.073096in}}%
\pgfpathclose%
\pgfusepath{stroke,fill}%
\end{pgfscope}%
\begin{pgfscope}%
\pgfpathrectangle{\pgfqpoint{0.100000in}{0.212622in}}{\pgfqpoint{3.696000in}{3.696000in}}%
\pgfusepath{clip}%
\pgfsetbuttcap%
\pgfsetroundjoin%
\definecolor{currentfill}{rgb}{0.121569,0.466667,0.705882}%
\pgfsetfillcolor{currentfill}%
\pgfsetfillopacity{0.599388}%
\pgfsetlinewidth{1.003750pt}%
\definecolor{currentstroke}{rgb}{0.121569,0.466667,0.705882}%
\pgfsetstrokecolor{currentstroke}%
\pgfsetstrokeopacity{0.599388}%
\pgfsetdash{}{0pt}%
\pgfpathmoveto{\pgfqpoint{1.718916in}{3.073095in}}%
\pgfpathcurveto{\pgfqpoint{1.727152in}{3.073095in}}{\pgfqpoint{1.735052in}{3.076368in}}{\pgfqpoint{1.740876in}{3.082192in}}%
\pgfpathcurveto{\pgfqpoint{1.746700in}{3.088015in}}{\pgfqpoint{1.749972in}{3.095915in}}{\pgfqpoint{1.749972in}{3.104152in}}%
\pgfpathcurveto{\pgfqpoint{1.749972in}{3.112388in}}{\pgfqpoint{1.746700in}{3.120288in}}{\pgfqpoint{1.740876in}{3.126112in}}%
\pgfpathcurveto{\pgfqpoint{1.735052in}{3.131936in}}{\pgfqpoint{1.727152in}{3.135208in}}{\pgfqpoint{1.718916in}{3.135208in}}%
\pgfpathcurveto{\pgfqpoint{1.710679in}{3.135208in}}{\pgfqpoint{1.702779in}{3.131936in}}{\pgfqpoint{1.696955in}{3.126112in}}%
\pgfpathcurveto{\pgfqpoint{1.691132in}{3.120288in}}{\pgfqpoint{1.687859in}{3.112388in}}{\pgfqpoint{1.687859in}{3.104152in}}%
\pgfpathcurveto{\pgfqpoint{1.687859in}{3.095915in}}{\pgfqpoint{1.691132in}{3.088015in}}{\pgfqpoint{1.696955in}{3.082192in}}%
\pgfpathcurveto{\pgfqpoint{1.702779in}{3.076368in}}{\pgfqpoint{1.710679in}{3.073095in}}{\pgfqpoint{1.718916in}{3.073095in}}%
\pgfpathclose%
\pgfusepath{stroke,fill}%
\end{pgfscope}%
\begin{pgfscope}%
\pgfpathrectangle{\pgfqpoint{0.100000in}{0.212622in}}{\pgfqpoint{3.696000in}{3.696000in}}%
\pgfusepath{clip}%
\pgfsetbuttcap%
\pgfsetroundjoin%
\definecolor{currentfill}{rgb}{0.121569,0.466667,0.705882}%
\pgfsetfillcolor{currentfill}%
\pgfsetfillopacity{0.599389}%
\pgfsetlinewidth{1.003750pt}%
\definecolor{currentstroke}{rgb}{0.121569,0.466667,0.705882}%
\pgfsetstrokecolor{currentstroke}%
\pgfsetstrokeopacity{0.599389}%
\pgfsetdash{}{0pt}%
\pgfpathmoveto{\pgfqpoint{1.718918in}{3.073095in}}%
\pgfpathcurveto{\pgfqpoint{1.727154in}{3.073095in}}{\pgfqpoint{1.735054in}{3.076367in}}{\pgfqpoint{1.740878in}{3.082191in}}%
\pgfpathcurveto{\pgfqpoint{1.746702in}{3.088015in}}{\pgfqpoint{1.749974in}{3.095915in}}{\pgfqpoint{1.749974in}{3.104151in}}%
\pgfpathcurveto{\pgfqpoint{1.749974in}{3.112388in}}{\pgfqpoint{1.746702in}{3.120288in}}{\pgfqpoint{1.740878in}{3.126112in}}%
\pgfpathcurveto{\pgfqpoint{1.735054in}{3.131936in}}{\pgfqpoint{1.727154in}{3.135208in}}{\pgfqpoint{1.718918in}{3.135208in}}%
\pgfpathcurveto{\pgfqpoint{1.710682in}{3.135208in}}{\pgfqpoint{1.702782in}{3.131936in}}{\pgfqpoint{1.696958in}{3.126112in}}%
\pgfpathcurveto{\pgfqpoint{1.691134in}{3.120288in}}{\pgfqpoint{1.687861in}{3.112388in}}{\pgfqpoint{1.687861in}{3.104151in}}%
\pgfpathcurveto{\pgfqpoint{1.687861in}{3.095915in}}{\pgfqpoint{1.691134in}{3.088015in}}{\pgfqpoint{1.696958in}{3.082191in}}%
\pgfpathcurveto{\pgfqpoint{1.702782in}{3.076367in}}{\pgfqpoint{1.710682in}{3.073095in}}{\pgfqpoint{1.718918in}{3.073095in}}%
\pgfpathclose%
\pgfusepath{stroke,fill}%
\end{pgfscope}%
\begin{pgfscope}%
\pgfpathrectangle{\pgfqpoint{0.100000in}{0.212622in}}{\pgfqpoint{3.696000in}{3.696000in}}%
\pgfusepath{clip}%
\pgfsetbuttcap%
\pgfsetroundjoin%
\definecolor{currentfill}{rgb}{0.121569,0.466667,0.705882}%
\pgfsetfillcolor{currentfill}%
\pgfsetfillopacity{0.599390}%
\pgfsetlinewidth{1.003750pt}%
\definecolor{currentstroke}{rgb}{0.121569,0.466667,0.705882}%
\pgfsetstrokecolor{currentstroke}%
\pgfsetstrokeopacity{0.599390}%
\pgfsetdash{}{0pt}%
\pgfpathmoveto{\pgfqpoint{1.718922in}{3.073094in}}%
\pgfpathcurveto{\pgfqpoint{1.727158in}{3.073094in}}{\pgfqpoint{1.735058in}{3.076366in}}{\pgfqpoint{1.740882in}{3.082190in}}%
\pgfpathcurveto{\pgfqpoint{1.746706in}{3.088014in}}{\pgfqpoint{1.749978in}{3.095914in}}{\pgfqpoint{1.749978in}{3.104150in}}%
\pgfpathcurveto{\pgfqpoint{1.749978in}{3.112387in}}{\pgfqpoint{1.746706in}{3.120287in}}{\pgfqpoint{1.740882in}{3.126111in}}%
\pgfpathcurveto{\pgfqpoint{1.735058in}{3.131935in}}{\pgfqpoint{1.727158in}{3.135207in}}{\pgfqpoint{1.718922in}{3.135207in}}%
\pgfpathcurveto{\pgfqpoint{1.710685in}{3.135207in}}{\pgfqpoint{1.702785in}{3.131935in}}{\pgfqpoint{1.696961in}{3.126111in}}%
\pgfpathcurveto{\pgfqpoint{1.691137in}{3.120287in}}{\pgfqpoint{1.687865in}{3.112387in}}{\pgfqpoint{1.687865in}{3.104150in}}%
\pgfpathcurveto{\pgfqpoint{1.687865in}{3.095914in}}{\pgfqpoint{1.691137in}{3.088014in}}{\pgfqpoint{1.696961in}{3.082190in}}%
\pgfpathcurveto{\pgfqpoint{1.702785in}{3.076366in}}{\pgfqpoint{1.710685in}{3.073094in}}{\pgfqpoint{1.718922in}{3.073094in}}%
\pgfpathclose%
\pgfusepath{stroke,fill}%
\end{pgfscope}%
\begin{pgfscope}%
\pgfpathrectangle{\pgfqpoint{0.100000in}{0.212622in}}{\pgfqpoint{3.696000in}{3.696000in}}%
\pgfusepath{clip}%
\pgfsetbuttcap%
\pgfsetroundjoin%
\definecolor{currentfill}{rgb}{0.121569,0.466667,0.705882}%
\pgfsetfillcolor{currentfill}%
\pgfsetfillopacity{0.599391}%
\pgfsetlinewidth{1.003750pt}%
\definecolor{currentstroke}{rgb}{0.121569,0.466667,0.705882}%
\pgfsetstrokecolor{currentstroke}%
\pgfsetstrokeopacity{0.599391}%
\pgfsetdash{}{0pt}%
\pgfpathmoveto{\pgfqpoint{1.718929in}{3.073093in}}%
\pgfpathcurveto{\pgfqpoint{1.727165in}{3.073093in}}{\pgfqpoint{1.735065in}{3.076365in}}{\pgfqpoint{1.740889in}{3.082189in}}%
\pgfpathcurveto{\pgfqpoint{1.746713in}{3.088013in}}{\pgfqpoint{1.749985in}{3.095913in}}{\pgfqpoint{1.749985in}{3.104150in}}%
\pgfpathcurveto{\pgfqpoint{1.749985in}{3.112386in}}{\pgfqpoint{1.746713in}{3.120286in}}{\pgfqpoint{1.740889in}{3.126110in}}%
\pgfpathcurveto{\pgfqpoint{1.735065in}{3.131934in}}{\pgfqpoint{1.727165in}{3.135206in}}{\pgfqpoint{1.718929in}{3.135206in}}%
\pgfpathcurveto{\pgfqpoint{1.710693in}{3.135206in}}{\pgfqpoint{1.702793in}{3.131934in}}{\pgfqpoint{1.696969in}{3.126110in}}%
\pgfpathcurveto{\pgfqpoint{1.691145in}{3.120286in}}{\pgfqpoint{1.687872in}{3.112386in}}{\pgfqpoint{1.687872in}{3.104150in}}%
\pgfpathcurveto{\pgfqpoint{1.687872in}{3.095913in}}{\pgfqpoint{1.691145in}{3.088013in}}{\pgfqpoint{1.696969in}{3.082189in}}%
\pgfpathcurveto{\pgfqpoint{1.702793in}{3.076365in}}{\pgfqpoint{1.710693in}{3.073093in}}{\pgfqpoint{1.718929in}{3.073093in}}%
\pgfpathclose%
\pgfusepath{stroke,fill}%
\end{pgfscope}%
\begin{pgfscope}%
\pgfpathrectangle{\pgfqpoint{0.100000in}{0.212622in}}{\pgfqpoint{3.696000in}{3.696000in}}%
\pgfusepath{clip}%
\pgfsetbuttcap%
\pgfsetroundjoin%
\definecolor{currentfill}{rgb}{0.121569,0.466667,0.705882}%
\pgfsetfillcolor{currentfill}%
\pgfsetfillopacity{0.599394}%
\pgfsetlinewidth{1.003750pt}%
\definecolor{currentstroke}{rgb}{0.121569,0.466667,0.705882}%
\pgfsetstrokecolor{currentstroke}%
\pgfsetstrokeopacity{0.599394}%
\pgfsetdash{}{0pt}%
\pgfpathmoveto{\pgfqpoint{1.718942in}{3.073091in}}%
\pgfpathcurveto{\pgfqpoint{1.727179in}{3.073091in}}{\pgfqpoint{1.735079in}{3.076364in}}{\pgfqpoint{1.740903in}{3.082188in}}%
\pgfpathcurveto{\pgfqpoint{1.746727in}{3.088012in}}{\pgfqpoint{1.749999in}{3.095912in}}{\pgfqpoint{1.749999in}{3.104148in}}%
\pgfpathcurveto{\pgfqpoint{1.749999in}{3.112384in}}{\pgfqpoint{1.746727in}{3.120284in}}{\pgfqpoint{1.740903in}{3.126108in}}%
\pgfpathcurveto{\pgfqpoint{1.735079in}{3.131932in}}{\pgfqpoint{1.727179in}{3.135204in}}{\pgfqpoint{1.718942in}{3.135204in}}%
\pgfpathcurveto{\pgfqpoint{1.710706in}{3.135204in}}{\pgfqpoint{1.702806in}{3.131932in}}{\pgfqpoint{1.696982in}{3.126108in}}%
\pgfpathcurveto{\pgfqpoint{1.691158in}{3.120284in}}{\pgfqpoint{1.687886in}{3.112384in}}{\pgfqpoint{1.687886in}{3.104148in}}%
\pgfpathcurveto{\pgfqpoint{1.687886in}{3.095912in}}{\pgfqpoint{1.691158in}{3.088012in}}{\pgfqpoint{1.696982in}{3.082188in}}%
\pgfpathcurveto{\pgfqpoint{1.702806in}{3.076364in}}{\pgfqpoint{1.710706in}{3.073091in}}{\pgfqpoint{1.718942in}{3.073091in}}%
\pgfpathclose%
\pgfusepath{stroke,fill}%
\end{pgfscope}%
\begin{pgfscope}%
\pgfpathrectangle{\pgfqpoint{0.100000in}{0.212622in}}{\pgfqpoint{3.696000in}{3.696000in}}%
\pgfusepath{clip}%
\pgfsetbuttcap%
\pgfsetroundjoin%
\definecolor{currentfill}{rgb}{0.121569,0.466667,0.705882}%
\pgfsetfillcolor{currentfill}%
\pgfsetfillopacity{0.599399}%
\pgfsetlinewidth{1.003750pt}%
\definecolor{currentstroke}{rgb}{0.121569,0.466667,0.705882}%
\pgfsetstrokecolor{currentstroke}%
\pgfsetstrokeopacity{0.599399}%
\pgfsetdash{}{0pt}%
\pgfpathmoveto{\pgfqpoint{1.718966in}{3.073087in}}%
\pgfpathcurveto{\pgfqpoint{1.727202in}{3.073087in}}{\pgfqpoint{1.735102in}{3.076359in}}{\pgfqpoint{1.740926in}{3.082183in}}%
\pgfpathcurveto{\pgfqpoint{1.746750in}{3.088007in}}{\pgfqpoint{1.750022in}{3.095907in}}{\pgfqpoint{1.750022in}{3.104144in}}%
\pgfpathcurveto{\pgfqpoint{1.750022in}{3.112380in}}{\pgfqpoint{1.746750in}{3.120280in}}{\pgfqpoint{1.740926in}{3.126104in}}%
\pgfpathcurveto{\pgfqpoint{1.735102in}{3.131928in}}{\pgfqpoint{1.727202in}{3.135200in}}{\pgfqpoint{1.718966in}{3.135200in}}%
\pgfpathcurveto{\pgfqpoint{1.710730in}{3.135200in}}{\pgfqpoint{1.702830in}{3.131928in}}{\pgfqpoint{1.697006in}{3.126104in}}%
\pgfpathcurveto{\pgfqpoint{1.691182in}{3.120280in}}{\pgfqpoint{1.687909in}{3.112380in}}{\pgfqpoint{1.687909in}{3.104144in}}%
\pgfpathcurveto{\pgfqpoint{1.687909in}{3.095907in}}{\pgfqpoint{1.691182in}{3.088007in}}{\pgfqpoint{1.697006in}{3.082183in}}%
\pgfpathcurveto{\pgfqpoint{1.702830in}{3.076359in}}{\pgfqpoint{1.710730in}{3.073087in}}{\pgfqpoint{1.718966in}{3.073087in}}%
\pgfpathclose%
\pgfusepath{stroke,fill}%
\end{pgfscope}%
\begin{pgfscope}%
\pgfpathrectangle{\pgfqpoint{0.100000in}{0.212622in}}{\pgfqpoint{3.696000in}{3.696000in}}%
\pgfusepath{clip}%
\pgfsetbuttcap%
\pgfsetroundjoin%
\definecolor{currentfill}{rgb}{0.121569,0.466667,0.705882}%
\pgfsetfillcolor{currentfill}%
\pgfsetfillopacity{0.599409}%
\pgfsetlinewidth{1.003750pt}%
\definecolor{currentstroke}{rgb}{0.121569,0.466667,0.705882}%
\pgfsetstrokecolor{currentstroke}%
\pgfsetstrokeopacity{0.599409}%
\pgfsetdash{}{0pt}%
\pgfpathmoveto{\pgfqpoint{1.719006in}{3.073075in}}%
\pgfpathcurveto{\pgfqpoint{1.727242in}{3.073075in}}{\pgfqpoint{1.735142in}{3.076347in}}{\pgfqpoint{1.740966in}{3.082171in}}%
\pgfpathcurveto{\pgfqpoint{1.746790in}{3.087995in}}{\pgfqpoint{1.750062in}{3.095895in}}{\pgfqpoint{1.750062in}{3.104131in}}%
\pgfpathcurveto{\pgfqpoint{1.750062in}{3.112367in}}{\pgfqpoint{1.746790in}{3.120268in}}{\pgfqpoint{1.740966in}{3.126091in}}%
\pgfpathcurveto{\pgfqpoint{1.735142in}{3.131915in}}{\pgfqpoint{1.727242in}{3.135188in}}{\pgfqpoint{1.719006in}{3.135188in}}%
\pgfpathcurveto{\pgfqpoint{1.710770in}{3.135188in}}{\pgfqpoint{1.702870in}{3.131915in}}{\pgfqpoint{1.697046in}{3.126091in}}%
\pgfpathcurveto{\pgfqpoint{1.691222in}{3.120268in}}{\pgfqpoint{1.687949in}{3.112367in}}{\pgfqpoint{1.687949in}{3.104131in}}%
\pgfpathcurveto{\pgfqpoint{1.687949in}{3.095895in}}{\pgfqpoint{1.691222in}{3.087995in}}{\pgfqpoint{1.697046in}{3.082171in}}%
\pgfpathcurveto{\pgfqpoint{1.702870in}{3.076347in}}{\pgfqpoint{1.710770in}{3.073075in}}{\pgfqpoint{1.719006in}{3.073075in}}%
\pgfpathclose%
\pgfusepath{stroke,fill}%
\end{pgfscope}%
\begin{pgfscope}%
\pgfpathrectangle{\pgfqpoint{0.100000in}{0.212622in}}{\pgfqpoint{3.696000in}{3.696000in}}%
\pgfusepath{clip}%
\pgfsetbuttcap%
\pgfsetroundjoin%
\definecolor{currentfill}{rgb}{0.121569,0.466667,0.705882}%
\pgfsetfillcolor{currentfill}%
\pgfsetfillopacity{0.599426}%
\pgfsetlinewidth{1.003750pt}%
\definecolor{currentstroke}{rgb}{0.121569,0.466667,0.705882}%
\pgfsetstrokecolor{currentstroke}%
\pgfsetstrokeopacity{0.599426}%
\pgfsetdash{}{0pt}%
\pgfpathmoveto{\pgfqpoint{1.719082in}{3.073057in}}%
\pgfpathcurveto{\pgfqpoint{1.727318in}{3.073057in}}{\pgfqpoint{1.735218in}{3.076329in}}{\pgfqpoint{1.741042in}{3.082153in}}%
\pgfpathcurveto{\pgfqpoint{1.746866in}{3.087977in}}{\pgfqpoint{1.750138in}{3.095877in}}{\pgfqpoint{1.750138in}{3.104113in}}%
\pgfpathcurveto{\pgfqpoint{1.750138in}{3.112349in}}{\pgfqpoint{1.746866in}{3.120249in}}{\pgfqpoint{1.741042in}{3.126073in}}%
\pgfpathcurveto{\pgfqpoint{1.735218in}{3.131897in}}{\pgfqpoint{1.727318in}{3.135170in}}{\pgfqpoint{1.719082in}{3.135170in}}%
\pgfpathcurveto{\pgfqpoint{1.710845in}{3.135170in}}{\pgfqpoint{1.702945in}{3.131897in}}{\pgfqpoint{1.697121in}{3.126073in}}%
\pgfpathcurveto{\pgfqpoint{1.691297in}{3.120249in}}{\pgfqpoint{1.688025in}{3.112349in}}{\pgfqpoint{1.688025in}{3.104113in}}%
\pgfpathcurveto{\pgfqpoint{1.688025in}{3.095877in}}{\pgfqpoint{1.691297in}{3.087977in}}{\pgfqpoint{1.697121in}{3.082153in}}%
\pgfpathcurveto{\pgfqpoint{1.702945in}{3.076329in}}{\pgfqpoint{1.710845in}{3.073057in}}{\pgfqpoint{1.719082in}{3.073057in}}%
\pgfpathclose%
\pgfusepath{stroke,fill}%
\end{pgfscope}%
\begin{pgfscope}%
\pgfpathrectangle{\pgfqpoint{0.100000in}{0.212622in}}{\pgfqpoint{3.696000in}{3.696000in}}%
\pgfusepath{clip}%
\pgfsetbuttcap%
\pgfsetroundjoin%
\definecolor{currentfill}{rgb}{0.121569,0.466667,0.705882}%
\pgfsetfillcolor{currentfill}%
\pgfsetfillopacity{0.599457}%
\pgfsetlinewidth{1.003750pt}%
\definecolor{currentstroke}{rgb}{0.121569,0.466667,0.705882}%
\pgfsetstrokecolor{currentstroke}%
\pgfsetstrokeopacity{0.599457}%
\pgfsetdash{}{0pt}%
\pgfpathmoveto{\pgfqpoint{1.718623in}{3.072674in}}%
\pgfpathcurveto{\pgfqpoint{1.726859in}{3.072674in}}{\pgfqpoint{1.734759in}{3.075946in}}{\pgfqpoint{1.740583in}{3.081770in}}%
\pgfpathcurveto{\pgfqpoint{1.746407in}{3.087594in}}{\pgfqpoint{1.749679in}{3.095494in}}{\pgfqpoint{1.749679in}{3.103730in}}%
\pgfpathcurveto{\pgfqpoint{1.749679in}{3.111966in}}{\pgfqpoint{1.746407in}{3.119866in}}{\pgfqpoint{1.740583in}{3.125690in}}%
\pgfpathcurveto{\pgfqpoint{1.734759in}{3.131514in}}{\pgfqpoint{1.726859in}{3.134787in}}{\pgfqpoint{1.718623in}{3.134787in}}%
\pgfpathcurveto{\pgfqpoint{1.710386in}{3.134787in}}{\pgfqpoint{1.702486in}{3.131514in}}{\pgfqpoint{1.696662in}{3.125690in}}%
\pgfpathcurveto{\pgfqpoint{1.690838in}{3.119866in}}{\pgfqpoint{1.687566in}{3.111966in}}{\pgfqpoint{1.687566in}{3.103730in}}%
\pgfpathcurveto{\pgfqpoint{1.687566in}{3.095494in}}{\pgfqpoint{1.690838in}{3.087594in}}{\pgfqpoint{1.696662in}{3.081770in}}%
\pgfpathcurveto{\pgfqpoint{1.702486in}{3.075946in}}{\pgfqpoint{1.710386in}{3.072674in}}{\pgfqpoint{1.718623in}{3.072674in}}%
\pgfpathclose%
\pgfusepath{stroke,fill}%
\end{pgfscope}%
\begin{pgfscope}%
\pgfpathrectangle{\pgfqpoint{0.100000in}{0.212622in}}{\pgfqpoint{3.696000in}{3.696000in}}%
\pgfusepath{clip}%
\pgfsetbuttcap%
\pgfsetroundjoin%
\definecolor{currentfill}{rgb}{0.121569,0.466667,0.705882}%
\pgfsetfillcolor{currentfill}%
\pgfsetfillopacity{0.599458}%
\pgfsetlinewidth{1.003750pt}%
\definecolor{currentstroke}{rgb}{0.121569,0.466667,0.705882}%
\pgfsetstrokecolor{currentstroke}%
\pgfsetstrokeopacity{0.599458}%
\pgfsetdash{}{0pt}%
\pgfpathmoveto{\pgfqpoint{1.719217in}{3.073022in}}%
\pgfpathcurveto{\pgfqpoint{1.727453in}{3.073022in}}{\pgfqpoint{1.735353in}{3.076294in}}{\pgfqpoint{1.741177in}{3.082118in}}%
\pgfpathcurveto{\pgfqpoint{1.747001in}{3.087942in}}{\pgfqpoint{1.750273in}{3.095842in}}{\pgfqpoint{1.750273in}{3.104078in}}%
\pgfpathcurveto{\pgfqpoint{1.750273in}{3.112315in}}{\pgfqpoint{1.747001in}{3.120215in}}{\pgfqpoint{1.741177in}{3.126039in}}%
\pgfpathcurveto{\pgfqpoint{1.735353in}{3.131862in}}{\pgfqpoint{1.727453in}{3.135135in}}{\pgfqpoint{1.719217in}{3.135135in}}%
\pgfpathcurveto{\pgfqpoint{1.710980in}{3.135135in}}{\pgfqpoint{1.703080in}{3.131862in}}{\pgfqpoint{1.697256in}{3.126039in}}%
\pgfpathcurveto{\pgfqpoint{1.691432in}{3.120215in}}{\pgfqpoint{1.688160in}{3.112315in}}{\pgfqpoint{1.688160in}{3.104078in}}%
\pgfpathcurveto{\pgfqpoint{1.688160in}{3.095842in}}{\pgfqpoint{1.691432in}{3.087942in}}{\pgfqpoint{1.697256in}{3.082118in}}%
\pgfpathcurveto{\pgfqpoint{1.703080in}{3.076294in}}{\pgfqpoint{1.710980in}{3.073022in}}{\pgfqpoint{1.719217in}{3.073022in}}%
\pgfpathclose%
\pgfusepath{stroke,fill}%
\end{pgfscope}%
\begin{pgfscope}%
\pgfpathrectangle{\pgfqpoint{0.100000in}{0.212622in}}{\pgfqpoint{3.696000in}{3.696000in}}%
\pgfusepath{clip}%
\pgfsetbuttcap%
\pgfsetroundjoin%
\definecolor{currentfill}{rgb}{0.121569,0.466667,0.705882}%
\pgfsetfillcolor{currentfill}%
\pgfsetfillopacity{0.599485}%
\pgfsetlinewidth{1.003750pt}%
\definecolor{currentstroke}{rgb}{0.121569,0.466667,0.705882}%
\pgfsetstrokecolor{currentstroke}%
\pgfsetstrokeopacity{0.599485}%
\pgfsetdash{}{0pt}%
\pgfpathmoveto{\pgfqpoint{1.718454in}{3.072418in}}%
\pgfpathcurveto{\pgfqpoint{1.726690in}{3.072418in}}{\pgfqpoint{1.734590in}{3.075691in}}{\pgfqpoint{1.740414in}{3.081515in}}%
\pgfpathcurveto{\pgfqpoint{1.746238in}{3.087339in}}{\pgfqpoint{1.749510in}{3.095239in}}{\pgfqpoint{1.749510in}{3.103475in}}%
\pgfpathcurveto{\pgfqpoint{1.749510in}{3.111711in}}{\pgfqpoint{1.746238in}{3.119611in}}{\pgfqpoint{1.740414in}{3.125435in}}%
\pgfpathcurveto{\pgfqpoint{1.734590in}{3.131259in}}{\pgfqpoint{1.726690in}{3.134531in}}{\pgfqpoint{1.718454in}{3.134531in}}%
\pgfpathcurveto{\pgfqpoint{1.710217in}{3.134531in}}{\pgfqpoint{1.702317in}{3.131259in}}{\pgfqpoint{1.696493in}{3.125435in}}%
\pgfpathcurveto{\pgfqpoint{1.690669in}{3.119611in}}{\pgfqpoint{1.687397in}{3.111711in}}{\pgfqpoint{1.687397in}{3.103475in}}%
\pgfpathcurveto{\pgfqpoint{1.687397in}{3.095239in}}{\pgfqpoint{1.690669in}{3.087339in}}{\pgfqpoint{1.696493in}{3.081515in}}%
\pgfpathcurveto{\pgfqpoint{1.702317in}{3.075691in}}{\pgfqpoint{1.710217in}{3.072418in}}{\pgfqpoint{1.718454in}{3.072418in}}%
\pgfpathclose%
\pgfusepath{stroke,fill}%
\end{pgfscope}%
\begin{pgfscope}%
\pgfpathrectangle{\pgfqpoint{0.100000in}{0.212622in}}{\pgfqpoint{3.696000in}{3.696000in}}%
\pgfusepath{clip}%
\pgfsetbuttcap%
\pgfsetroundjoin%
\definecolor{currentfill}{rgb}{0.121569,0.466667,0.705882}%
\pgfsetfillcolor{currentfill}%
\pgfsetfillopacity{0.599504}%
\pgfsetlinewidth{1.003750pt}%
\definecolor{currentstroke}{rgb}{0.121569,0.466667,0.705882}%
\pgfsetstrokecolor{currentstroke}%
\pgfsetstrokeopacity{0.599504}%
\pgfsetdash{}{0pt}%
\pgfpathmoveto{\pgfqpoint{1.718367in}{3.072279in}}%
\pgfpathcurveto{\pgfqpoint{1.726603in}{3.072279in}}{\pgfqpoint{1.734503in}{3.075551in}}{\pgfqpoint{1.740327in}{3.081375in}}%
\pgfpathcurveto{\pgfqpoint{1.746151in}{3.087199in}}{\pgfqpoint{1.749423in}{3.095099in}}{\pgfqpoint{1.749423in}{3.103335in}}%
\pgfpathcurveto{\pgfqpoint{1.749423in}{3.111572in}}{\pgfqpoint{1.746151in}{3.119472in}}{\pgfqpoint{1.740327in}{3.125296in}}%
\pgfpathcurveto{\pgfqpoint{1.734503in}{3.131120in}}{\pgfqpoint{1.726603in}{3.134392in}}{\pgfqpoint{1.718367in}{3.134392in}}%
\pgfpathcurveto{\pgfqpoint{1.710130in}{3.134392in}}{\pgfqpoint{1.702230in}{3.131120in}}{\pgfqpoint{1.696406in}{3.125296in}}%
\pgfpathcurveto{\pgfqpoint{1.690582in}{3.119472in}}{\pgfqpoint{1.687310in}{3.111572in}}{\pgfqpoint{1.687310in}{3.103335in}}%
\pgfpathcurveto{\pgfqpoint{1.687310in}{3.095099in}}{\pgfqpoint{1.690582in}{3.087199in}}{\pgfqpoint{1.696406in}{3.081375in}}%
\pgfpathcurveto{\pgfqpoint{1.702230in}{3.075551in}}{\pgfqpoint{1.710130in}{3.072279in}}{\pgfqpoint{1.718367in}{3.072279in}}%
\pgfpathclose%
\pgfusepath{stroke,fill}%
\end{pgfscope}%
\begin{pgfscope}%
\pgfpathrectangle{\pgfqpoint{0.100000in}{0.212622in}}{\pgfqpoint{3.696000in}{3.696000in}}%
\pgfusepath{clip}%
\pgfsetbuttcap%
\pgfsetroundjoin%
\definecolor{currentfill}{rgb}{0.121569,0.466667,0.705882}%
\pgfsetfillcolor{currentfill}%
\pgfsetfillopacity{0.599514}%
\pgfsetlinewidth{1.003750pt}%
\definecolor{currentstroke}{rgb}{0.121569,0.466667,0.705882}%
\pgfsetstrokecolor{currentstroke}%
\pgfsetstrokeopacity{0.599514}%
\pgfsetdash{}{0pt}%
\pgfpathmoveto{\pgfqpoint{1.719474in}{3.072981in}}%
\pgfpathcurveto{\pgfqpoint{1.727711in}{3.072981in}}{\pgfqpoint{1.735611in}{3.076253in}}{\pgfqpoint{1.741435in}{3.082077in}}%
\pgfpathcurveto{\pgfqpoint{1.747259in}{3.087901in}}{\pgfqpoint{1.750531in}{3.095801in}}{\pgfqpoint{1.750531in}{3.104037in}}%
\pgfpathcurveto{\pgfqpoint{1.750531in}{3.112273in}}{\pgfqpoint{1.747259in}{3.120174in}}{\pgfqpoint{1.741435in}{3.125997in}}%
\pgfpathcurveto{\pgfqpoint{1.735611in}{3.131821in}}{\pgfqpoint{1.727711in}{3.135094in}}{\pgfqpoint{1.719474in}{3.135094in}}%
\pgfpathcurveto{\pgfqpoint{1.711238in}{3.135094in}}{\pgfqpoint{1.703338in}{3.131821in}}{\pgfqpoint{1.697514in}{3.125997in}}%
\pgfpathcurveto{\pgfqpoint{1.691690in}{3.120174in}}{\pgfqpoint{1.688418in}{3.112273in}}{\pgfqpoint{1.688418in}{3.104037in}}%
\pgfpathcurveto{\pgfqpoint{1.688418in}{3.095801in}}{\pgfqpoint{1.691690in}{3.087901in}}{\pgfqpoint{1.697514in}{3.082077in}}%
\pgfpathcurveto{\pgfqpoint{1.703338in}{3.076253in}}{\pgfqpoint{1.711238in}{3.072981in}}{\pgfqpoint{1.719474in}{3.072981in}}%
\pgfpathclose%
\pgfusepath{stroke,fill}%
\end{pgfscope}%
\begin{pgfscope}%
\pgfpathrectangle{\pgfqpoint{0.100000in}{0.212622in}}{\pgfqpoint{3.696000in}{3.696000in}}%
\pgfusepath{clip}%
\pgfsetbuttcap%
\pgfsetroundjoin%
\definecolor{currentfill}{rgb}{0.121569,0.466667,0.705882}%
\pgfsetfillcolor{currentfill}%
\pgfsetfillopacity{0.599515}%
\pgfsetlinewidth{1.003750pt}%
\definecolor{currentstroke}{rgb}{0.121569,0.466667,0.705882}%
\pgfsetstrokecolor{currentstroke}%
\pgfsetstrokeopacity{0.599515}%
\pgfsetdash{}{0pt}%
\pgfpathmoveto{\pgfqpoint{1.718326in}{3.072196in}}%
\pgfpathcurveto{\pgfqpoint{1.726562in}{3.072196in}}{\pgfqpoint{1.734462in}{3.075468in}}{\pgfqpoint{1.740286in}{3.081292in}}%
\pgfpathcurveto{\pgfqpoint{1.746110in}{3.087116in}}{\pgfqpoint{1.749382in}{3.095016in}}{\pgfqpoint{1.749382in}{3.103253in}}%
\pgfpathcurveto{\pgfqpoint{1.749382in}{3.111489in}}{\pgfqpoint{1.746110in}{3.119389in}}{\pgfqpoint{1.740286in}{3.125213in}}%
\pgfpathcurveto{\pgfqpoint{1.734462in}{3.131037in}}{\pgfqpoint{1.726562in}{3.134309in}}{\pgfqpoint{1.718326in}{3.134309in}}%
\pgfpathcurveto{\pgfqpoint{1.710090in}{3.134309in}}{\pgfqpoint{1.702189in}{3.131037in}}{\pgfqpoint{1.696366in}{3.125213in}}%
\pgfpathcurveto{\pgfqpoint{1.690542in}{3.119389in}}{\pgfqpoint{1.687269in}{3.111489in}}{\pgfqpoint{1.687269in}{3.103253in}}%
\pgfpathcurveto{\pgfqpoint{1.687269in}{3.095016in}}{\pgfqpoint{1.690542in}{3.087116in}}{\pgfqpoint{1.696366in}{3.081292in}}%
\pgfpathcurveto{\pgfqpoint{1.702189in}{3.075468in}}{\pgfqpoint{1.710090in}{3.072196in}}{\pgfqpoint{1.718326in}{3.072196in}}%
\pgfpathclose%
\pgfusepath{stroke,fill}%
\end{pgfscope}%
\begin{pgfscope}%
\pgfpathrectangle{\pgfqpoint{0.100000in}{0.212622in}}{\pgfqpoint{3.696000in}{3.696000in}}%
\pgfusepath{clip}%
\pgfsetbuttcap%
\pgfsetroundjoin%
\definecolor{currentfill}{rgb}{0.121569,0.466667,0.705882}%
\pgfsetfillcolor{currentfill}%
\pgfsetfillopacity{0.599523}%
\pgfsetlinewidth{1.003750pt}%
\definecolor{currentstroke}{rgb}{0.121569,0.466667,0.705882}%
\pgfsetstrokecolor{currentstroke}%
\pgfsetstrokeopacity{0.599523}%
\pgfsetdash{}{0pt}%
\pgfpathmoveto{\pgfqpoint{1.718307in}{3.072154in}}%
\pgfpathcurveto{\pgfqpoint{1.726543in}{3.072154in}}{\pgfqpoint{1.734443in}{3.075426in}}{\pgfqpoint{1.740267in}{3.081250in}}%
\pgfpathcurveto{\pgfqpoint{1.746091in}{3.087074in}}{\pgfqpoint{1.749363in}{3.094974in}}{\pgfqpoint{1.749363in}{3.103210in}}%
\pgfpathcurveto{\pgfqpoint{1.749363in}{3.111447in}}{\pgfqpoint{1.746091in}{3.119347in}}{\pgfqpoint{1.740267in}{3.125171in}}%
\pgfpathcurveto{\pgfqpoint{1.734443in}{3.130995in}}{\pgfqpoint{1.726543in}{3.134267in}}{\pgfqpoint{1.718307in}{3.134267in}}%
\pgfpathcurveto{\pgfqpoint{1.710070in}{3.134267in}}{\pgfqpoint{1.702170in}{3.130995in}}{\pgfqpoint{1.696346in}{3.125171in}}%
\pgfpathcurveto{\pgfqpoint{1.690523in}{3.119347in}}{\pgfqpoint{1.687250in}{3.111447in}}{\pgfqpoint{1.687250in}{3.103210in}}%
\pgfpathcurveto{\pgfqpoint{1.687250in}{3.094974in}}{\pgfqpoint{1.690523in}{3.087074in}}{\pgfqpoint{1.696346in}{3.081250in}}%
\pgfpathcurveto{\pgfqpoint{1.702170in}{3.075426in}}{\pgfqpoint{1.710070in}{3.072154in}}{\pgfqpoint{1.718307in}{3.072154in}}%
\pgfpathclose%
\pgfusepath{stroke,fill}%
\end{pgfscope}%
\begin{pgfscope}%
\pgfpathrectangle{\pgfqpoint{0.100000in}{0.212622in}}{\pgfqpoint{3.696000in}{3.696000in}}%
\pgfusepath{clip}%
\pgfsetbuttcap%
\pgfsetroundjoin%
\definecolor{currentfill}{rgb}{0.121569,0.466667,0.705882}%
\pgfsetfillcolor{currentfill}%
\pgfsetfillopacity{0.599527}%
\pgfsetlinewidth{1.003750pt}%
\definecolor{currentstroke}{rgb}{0.121569,0.466667,0.705882}%
\pgfsetstrokecolor{currentstroke}%
\pgfsetstrokeopacity{0.599527}%
\pgfsetdash{}{0pt}%
\pgfpathmoveto{\pgfqpoint{1.718297in}{3.072131in}}%
\pgfpathcurveto{\pgfqpoint{1.726533in}{3.072131in}}{\pgfqpoint{1.734433in}{3.075403in}}{\pgfqpoint{1.740257in}{3.081227in}}%
\pgfpathcurveto{\pgfqpoint{1.746081in}{3.087051in}}{\pgfqpoint{1.749353in}{3.094951in}}{\pgfqpoint{1.749353in}{3.103187in}}%
\pgfpathcurveto{\pgfqpoint{1.749353in}{3.111424in}}{\pgfqpoint{1.746081in}{3.119324in}}{\pgfqpoint{1.740257in}{3.125148in}}%
\pgfpathcurveto{\pgfqpoint{1.734433in}{3.130972in}}{\pgfqpoint{1.726533in}{3.134244in}}{\pgfqpoint{1.718297in}{3.134244in}}%
\pgfpathcurveto{\pgfqpoint{1.710060in}{3.134244in}}{\pgfqpoint{1.702160in}{3.130972in}}{\pgfqpoint{1.696336in}{3.125148in}}%
\pgfpathcurveto{\pgfqpoint{1.690512in}{3.119324in}}{\pgfqpoint{1.687240in}{3.111424in}}{\pgfqpoint{1.687240in}{3.103187in}}%
\pgfpathcurveto{\pgfqpoint{1.687240in}{3.094951in}}{\pgfqpoint{1.690512in}{3.087051in}}{\pgfqpoint{1.696336in}{3.081227in}}%
\pgfpathcurveto{\pgfqpoint{1.702160in}{3.075403in}}{\pgfqpoint{1.710060in}{3.072131in}}{\pgfqpoint{1.718297in}{3.072131in}}%
\pgfpathclose%
\pgfusepath{stroke,fill}%
\end{pgfscope}%
\begin{pgfscope}%
\pgfpathrectangle{\pgfqpoint{0.100000in}{0.212622in}}{\pgfqpoint{3.696000in}{3.696000in}}%
\pgfusepath{clip}%
\pgfsetbuttcap%
\pgfsetroundjoin%
\definecolor{currentfill}{rgb}{0.121569,0.466667,0.705882}%
\pgfsetfillcolor{currentfill}%
\pgfsetfillopacity{0.599529}%
\pgfsetlinewidth{1.003750pt}%
\definecolor{currentstroke}{rgb}{0.121569,0.466667,0.705882}%
\pgfsetstrokecolor{currentstroke}%
\pgfsetstrokeopacity{0.599529}%
\pgfsetdash{}{0pt}%
\pgfpathmoveto{\pgfqpoint{1.718291in}{3.072119in}}%
\pgfpathcurveto{\pgfqpoint{1.726528in}{3.072119in}}{\pgfqpoint{1.734428in}{3.075391in}}{\pgfqpoint{1.740252in}{3.081215in}}%
\pgfpathcurveto{\pgfqpoint{1.746076in}{3.087039in}}{\pgfqpoint{1.749348in}{3.094939in}}{\pgfqpoint{1.749348in}{3.103175in}}%
\pgfpathcurveto{\pgfqpoint{1.749348in}{3.111411in}}{\pgfqpoint{1.746076in}{3.119312in}}{\pgfqpoint{1.740252in}{3.125135in}}%
\pgfpathcurveto{\pgfqpoint{1.734428in}{3.130959in}}{\pgfqpoint{1.726528in}{3.134232in}}{\pgfqpoint{1.718291in}{3.134232in}}%
\pgfpathcurveto{\pgfqpoint{1.710055in}{3.134232in}}{\pgfqpoint{1.702155in}{3.130959in}}{\pgfqpoint{1.696331in}{3.125135in}}%
\pgfpathcurveto{\pgfqpoint{1.690507in}{3.119312in}}{\pgfqpoint{1.687235in}{3.111411in}}{\pgfqpoint{1.687235in}{3.103175in}}%
\pgfpathcurveto{\pgfqpoint{1.687235in}{3.094939in}}{\pgfqpoint{1.690507in}{3.087039in}}{\pgfqpoint{1.696331in}{3.081215in}}%
\pgfpathcurveto{\pgfqpoint{1.702155in}{3.075391in}}{\pgfqpoint{1.710055in}{3.072119in}}{\pgfqpoint{1.718291in}{3.072119in}}%
\pgfpathclose%
\pgfusepath{stroke,fill}%
\end{pgfscope}%
\begin{pgfscope}%
\pgfpathrectangle{\pgfqpoint{0.100000in}{0.212622in}}{\pgfqpoint{3.696000in}{3.696000in}}%
\pgfusepath{clip}%
\pgfsetbuttcap%
\pgfsetroundjoin%
\definecolor{currentfill}{rgb}{0.121569,0.466667,0.705882}%
\pgfsetfillcolor{currentfill}%
\pgfsetfillopacity{0.599531}%
\pgfsetlinewidth{1.003750pt}%
\definecolor{currentstroke}{rgb}{0.121569,0.466667,0.705882}%
\pgfsetstrokecolor{currentstroke}%
\pgfsetstrokeopacity{0.599531}%
\pgfsetdash{}{0pt}%
\pgfpathmoveto{\pgfqpoint{1.718288in}{3.072111in}}%
\pgfpathcurveto{\pgfqpoint{1.726525in}{3.072111in}}{\pgfqpoint{1.734425in}{3.075384in}}{\pgfqpoint{1.740249in}{3.081208in}}%
\pgfpathcurveto{\pgfqpoint{1.746073in}{3.087032in}}{\pgfqpoint{1.749345in}{3.094932in}}{\pgfqpoint{1.749345in}{3.103168in}}%
\pgfpathcurveto{\pgfqpoint{1.749345in}{3.111404in}}{\pgfqpoint{1.746073in}{3.119304in}}{\pgfqpoint{1.740249in}{3.125128in}}%
\pgfpathcurveto{\pgfqpoint{1.734425in}{3.130952in}}{\pgfqpoint{1.726525in}{3.134224in}}{\pgfqpoint{1.718288in}{3.134224in}}%
\pgfpathcurveto{\pgfqpoint{1.710052in}{3.134224in}}{\pgfqpoint{1.702152in}{3.130952in}}{\pgfqpoint{1.696328in}{3.125128in}}%
\pgfpathcurveto{\pgfqpoint{1.690504in}{3.119304in}}{\pgfqpoint{1.687232in}{3.111404in}}{\pgfqpoint{1.687232in}{3.103168in}}%
\pgfpathcurveto{\pgfqpoint{1.687232in}{3.094932in}}{\pgfqpoint{1.690504in}{3.087032in}}{\pgfqpoint{1.696328in}{3.081208in}}%
\pgfpathcurveto{\pgfqpoint{1.702152in}{3.075384in}}{\pgfqpoint{1.710052in}{3.072111in}}{\pgfqpoint{1.718288in}{3.072111in}}%
\pgfpathclose%
\pgfusepath{stroke,fill}%
\end{pgfscope}%
\begin{pgfscope}%
\pgfpathrectangle{\pgfqpoint{0.100000in}{0.212622in}}{\pgfqpoint{3.696000in}{3.696000in}}%
\pgfusepath{clip}%
\pgfsetbuttcap%
\pgfsetroundjoin%
\definecolor{currentfill}{rgb}{0.121569,0.466667,0.705882}%
\pgfsetfillcolor{currentfill}%
\pgfsetfillopacity{0.599531}%
\pgfsetlinewidth{1.003750pt}%
\definecolor{currentstroke}{rgb}{0.121569,0.466667,0.705882}%
\pgfsetstrokecolor{currentstroke}%
\pgfsetstrokeopacity{0.599531}%
\pgfsetdash{}{0pt}%
\pgfpathmoveto{\pgfqpoint{1.718287in}{3.072108in}}%
\pgfpathcurveto{\pgfqpoint{1.726523in}{3.072108in}}{\pgfqpoint{1.734423in}{3.075380in}}{\pgfqpoint{1.740247in}{3.081204in}}%
\pgfpathcurveto{\pgfqpoint{1.746071in}{3.087028in}}{\pgfqpoint{1.749343in}{3.094928in}}{\pgfqpoint{1.749343in}{3.103164in}}%
\pgfpathcurveto{\pgfqpoint{1.749343in}{3.111400in}}{\pgfqpoint{1.746071in}{3.119300in}}{\pgfqpoint{1.740247in}{3.125124in}}%
\pgfpathcurveto{\pgfqpoint{1.734423in}{3.130948in}}{\pgfqpoint{1.726523in}{3.134221in}}{\pgfqpoint{1.718287in}{3.134221in}}%
\pgfpathcurveto{\pgfqpoint{1.710051in}{3.134221in}}{\pgfqpoint{1.702150in}{3.130948in}}{\pgfqpoint{1.696327in}{3.125124in}}%
\pgfpathcurveto{\pgfqpoint{1.690503in}{3.119300in}}{\pgfqpoint{1.687230in}{3.111400in}}{\pgfqpoint{1.687230in}{3.103164in}}%
\pgfpathcurveto{\pgfqpoint{1.687230in}{3.094928in}}{\pgfqpoint{1.690503in}{3.087028in}}{\pgfqpoint{1.696327in}{3.081204in}}%
\pgfpathcurveto{\pgfqpoint{1.702150in}{3.075380in}}{\pgfqpoint{1.710051in}{3.072108in}}{\pgfqpoint{1.718287in}{3.072108in}}%
\pgfpathclose%
\pgfusepath{stroke,fill}%
\end{pgfscope}%
\begin{pgfscope}%
\pgfpathrectangle{\pgfqpoint{0.100000in}{0.212622in}}{\pgfqpoint{3.696000in}{3.696000in}}%
\pgfusepath{clip}%
\pgfsetbuttcap%
\pgfsetroundjoin%
\definecolor{currentfill}{rgb}{0.121569,0.466667,0.705882}%
\pgfsetfillcolor{currentfill}%
\pgfsetfillopacity{0.599532}%
\pgfsetlinewidth{1.003750pt}%
\definecolor{currentstroke}{rgb}{0.121569,0.466667,0.705882}%
\pgfsetstrokecolor{currentstroke}%
\pgfsetstrokeopacity{0.599532}%
\pgfsetdash{}{0pt}%
\pgfpathmoveto{\pgfqpoint{1.718286in}{3.072106in}}%
\pgfpathcurveto{\pgfqpoint{1.726522in}{3.072106in}}{\pgfqpoint{1.734422in}{3.075378in}}{\pgfqpoint{1.740246in}{3.081202in}}%
\pgfpathcurveto{\pgfqpoint{1.746070in}{3.087026in}}{\pgfqpoint{1.749342in}{3.094926in}}{\pgfqpoint{1.749342in}{3.103162in}}%
\pgfpathcurveto{\pgfqpoint{1.749342in}{3.111398in}}{\pgfqpoint{1.746070in}{3.119298in}}{\pgfqpoint{1.740246in}{3.125122in}}%
\pgfpathcurveto{\pgfqpoint{1.734422in}{3.130946in}}{\pgfqpoint{1.726522in}{3.134219in}}{\pgfqpoint{1.718286in}{3.134219in}}%
\pgfpathcurveto{\pgfqpoint{1.710050in}{3.134219in}}{\pgfqpoint{1.702150in}{3.130946in}}{\pgfqpoint{1.696326in}{3.125122in}}%
\pgfpathcurveto{\pgfqpoint{1.690502in}{3.119298in}}{\pgfqpoint{1.687229in}{3.111398in}}{\pgfqpoint{1.687229in}{3.103162in}}%
\pgfpathcurveto{\pgfqpoint{1.687229in}{3.094926in}}{\pgfqpoint{1.690502in}{3.087026in}}{\pgfqpoint{1.696326in}{3.081202in}}%
\pgfpathcurveto{\pgfqpoint{1.702150in}{3.075378in}}{\pgfqpoint{1.710050in}{3.072106in}}{\pgfqpoint{1.718286in}{3.072106in}}%
\pgfpathclose%
\pgfusepath{stroke,fill}%
\end{pgfscope}%
\begin{pgfscope}%
\pgfpathrectangle{\pgfqpoint{0.100000in}{0.212622in}}{\pgfqpoint{3.696000in}{3.696000in}}%
\pgfusepath{clip}%
\pgfsetbuttcap%
\pgfsetroundjoin%
\definecolor{currentfill}{rgb}{0.121569,0.466667,0.705882}%
\pgfsetfillcolor{currentfill}%
\pgfsetfillopacity{0.599532}%
\pgfsetlinewidth{1.003750pt}%
\definecolor{currentstroke}{rgb}{0.121569,0.466667,0.705882}%
\pgfsetstrokecolor{currentstroke}%
\pgfsetstrokeopacity{0.599532}%
\pgfsetdash{}{0pt}%
\pgfpathmoveto{\pgfqpoint{1.718285in}{3.072104in}}%
\pgfpathcurveto{\pgfqpoint{1.726522in}{3.072104in}}{\pgfqpoint{1.734422in}{3.075377in}}{\pgfqpoint{1.740246in}{3.081201in}}%
\pgfpathcurveto{\pgfqpoint{1.746070in}{3.087025in}}{\pgfqpoint{1.749342in}{3.094925in}}{\pgfqpoint{1.749342in}{3.103161in}}%
\pgfpathcurveto{\pgfqpoint{1.749342in}{3.111397in}}{\pgfqpoint{1.746070in}{3.119297in}}{\pgfqpoint{1.740246in}{3.125121in}}%
\pgfpathcurveto{\pgfqpoint{1.734422in}{3.130945in}}{\pgfqpoint{1.726522in}{3.134217in}}{\pgfqpoint{1.718285in}{3.134217in}}%
\pgfpathcurveto{\pgfqpoint{1.710049in}{3.134217in}}{\pgfqpoint{1.702149in}{3.130945in}}{\pgfqpoint{1.696325in}{3.125121in}}%
\pgfpathcurveto{\pgfqpoint{1.690501in}{3.119297in}}{\pgfqpoint{1.687229in}{3.111397in}}{\pgfqpoint{1.687229in}{3.103161in}}%
\pgfpathcurveto{\pgfqpoint{1.687229in}{3.094925in}}{\pgfqpoint{1.690501in}{3.087025in}}{\pgfqpoint{1.696325in}{3.081201in}}%
\pgfpathcurveto{\pgfqpoint{1.702149in}{3.075377in}}{\pgfqpoint{1.710049in}{3.072104in}}{\pgfqpoint{1.718285in}{3.072104in}}%
\pgfpathclose%
\pgfusepath{stroke,fill}%
\end{pgfscope}%
\begin{pgfscope}%
\pgfpathrectangle{\pgfqpoint{0.100000in}{0.212622in}}{\pgfqpoint{3.696000in}{3.696000in}}%
\pgfusepath{clip}%
\pgfsetbuttcap%
\pgfsetroundjoin%
\definecolor{currentfill}{rgb}{0.121569,0.466667,0.705882}%
\pgfsetfillcolor{currentfill}%
\pgfsetfillopacity{0.599532}%
\pgfsetlinewidth{1.003750pt}%
\definecolor{currentstroke}{rgb}{0.121569,0.466667,0.705882}%
\pgfsetstrokecolor{currentstroke}%
\pgfsetstrokeopacity{0.599532}%
\pgfsetdash{}{0pt}%
\pgfpathmoveto{\pgfqpoint{1.718285in}{3.072104in}}%
\pgfpathcurveto{\pgfqpoint{1.726521in}{3.072104in}}{\pgfqpoint{1.734422in}{3.075376in}}{\pgfqpoint{1.740245in}{3.081200in}}%
\pgfpathcurveto{\pgfqpoint{1.746069in}{3.087024in}}{\pgfqpoint{1.749342in}{3.094924in}}{\pgfqpoint{1.749342in}{3.103160in}}%
\pgfpathcurveto{\pgfqpoint{1.749342in}{3.111397in}}{\pgfqpoint{1.746069in}{3.119297in}}{\pgfqpoint{1.740245in}{3.125121in}}%
\pgfpathcurveto{\pgfqpoint{1.734422in}{3.130944in}}{\pgfqpoint{1.726521in}{3.134217in}}{\pgfqpoint{1.718285in}{3.134217in}}%
\pgfpathcurveto{\pgfqpoint{1.710049in}{3.134217in}}{\pgfqpoint{1.702149in}{3.130944in}}{\pgfqpoint{1.696325in}{3.125121in}}%
\pgfpathcurveto{\pgfqpoint{1.690501in}{3.119297in}}{\pgfqpoint{1.687229in}{3.111397in}}{\pgfqpoint{1.687229in}{3.103160in}}%
\pgfpathcurveto{\pgfqpoint{1.687229in}{3.094924in}}{\pgfqpoint{1.690501in}{3.087024in}}{\pgfqpoint{1.696325in}{3.081200in}}%
\pgfpathcurveto{\pgfqpoint{1.702149in}{3.075376in}}{\pgfqpoint{1.710049in}{3.072104in}}{\pgfqpoint{1.718285in}{3.072104in}}%
\pgfpathclose%
\pgfusepath{stroke,fill}%
\end{pgfscope}%
\begin{pgfscope}%
\pgfpathrectangle{\pgfqpoint{0.100000in}{0.212622in}}{\pgfqpoint{3.696000in}{3.696000in}}%
\pgfusepath{clip}%
\pgfsetbuttcap%
\pgfsetroundjoin%
\definecolor{currentfill}{rgb}{0.121569,0.466667,0.705882}%
\pgfsetfillcolor{currentfill}%
\pgfsetfillopacity{0.599532}%
\pgfsetlinewidth{1.003750pt}%
\definecolor{currentstroke}{rgb}{0.121569,0.466667,0.705882}%
\pgfsetstrokecolor{currentstroke}%
\pgfsetstrokeopacity{0.599532}%
\pgfsetdash{}{0pt}%
\pgfpathmoveto{\pgfqpoint{1.718285in}{3.072103in}}%
\pgfpathcurveto{\pgfqpoint{1.726521in}{3.072103in}}{\pgfqpoint{1.734421in}{3.075376in}}{\pgfqpoint{1.740245in}{3.081200in}}%
\pgfpathcurveto{\pgfqpoint{1.746069in}{3.087024in}}{\pgfqpoint{1.749342in}{3.094924in}}{\pgfqpoint{1.749342in}{3.103160in}}%
\pgfpathcurveto{\pgfqpoint{1.749342in}{3.111396in}}{\pgfqpoint{1.746069in}{3.119296in}}{\pgfqpoint{1.740245in}{3.125120in}}%
\pgfpathcurveto{\pgfqpoint{1.734421in}{3.130944in}}{\pgfqpoint{1.726521in}{3.134216in}}{\pgfqpoint{1.718285in}{3.134216in}}%
\pgfpathcurveto{\pgfqpoint{1.710049in}{3.134216in}}{\pgfqpoint{1.702149in}{3.130944in}}{\pgfqpoint{1.696325in}{3.125120in}}%
\pgfpathcurveto{\pgfqpoint{1.690501in}{3.119296in}}{\pgfqpoint{1.687229in}{3.111396in}}{\pgfqpoint{1.687229in}{3.103160in}}%
\pgfpathcurveto{\pgfqpoint{1.687229in}{3.094924in}}{\pgfqpoint{1.690501in}{3.087024in}}{\pgfqpoint{1.696325in}{3.081200in}}%
\pgfpathcurveto{\pgfqpoint{1.702149in}{3.075376in}}{\pgfqpoint{1.710049in}{3.072103in}}{\pgfqpoint{1.718285in}{3.072103in}}%
\pgfpathclose%
\pgfusepath{stroke,fill}%
\end{pgfscope}%
\begin{pgfscope}%
\pgfpathrectangle{\pgfqpoint{0.100000in}{0.212622in}}{\pgfqpoint{3.696000in}{3.696000in}}%
\pgfusepath{clip}%
\pgfsetbuttcap%
\pgfsetroundjoin%
\definecolor{currentfill}{rgb}{0.121569,0.466667,0.705882}%
\pgfsetfillcolor{currentfill}%
\pgfsetfillopacity{0.599532}%
\pgfsetlinewidth{1.003750pt}%
\definecolor{currentstroke}{rgb}{0.121569,0.466667,0.705882}%
\pgfsetstrokecolor{currentstroke}%
\pgfsetstrokeopacity{0.599532}%
\pgfsetdash{}{0pt}%
\pgfpathmoveto{\pgfqpoint{1.718285in}{3.072103in}}%
\pgfpathcurveto{\pgfqpoint{1.726521in}{3.072103in}}{\pgfqpoint{1.734421in}{3.075376in}}{\pgfqpoint{1.740245in}{3.081199in}}%
\pgfpathcurveto{\pgfqpoint{1.746069in}{3.087023in}}{\pgfqpoint{1.749341in}{3.094923in}}{\pgfqpoint{1.749341in}{3.103160in}}%
\pgfpathcurveto{\pgfqpoint{1.749341in}{3.111396in}}{\pgfqpoint{1.746069in}{3.119296in}}{\pgfqpoint{1.740245in}{3.125120in}}%
\pgfpathcurveto{\pgfqpoint{1.734421in}{3.130944in}}{\pgfqpoint{1.726521in}{3.134216in}}{\pgfqpoint{1.718285in}{3.134216in}}%
\pgfpathcurveto{\pgfqpoint{1.710049in}{3.134216in}}{\pgfqpoint{1.702149in}{3.130944in}}{\pgfqpoint{1.696325in}{3.125120in}}%
\pgfpathcurveto{\pgfqpoint{1.690501in}{3.119296in}}{\pgfqpoint{1.687228in}{3.111396in}}{\pgfqpoint{1.687228in}{3.103160in}}%
\pgfpathcurveto{\pgfqpoint{1.687228in}{3.094923in}}{\pgfqpoint{1.690501in}{3.087023in}}{\pgfqpoint{1.696325in}{3.081199in}}%
\pgfpathcurveto{\pgfqpoint{1.702149in}{3.075376in}}{\pgfqpoint{1.710049in}{3.072103in}}{\pgfqpoint{1.718285in}{3.072103in}}%
\pgfpathclose%
\pgfusepath{stroke,fill}%
\end{pgfscope}%
\begin{pgfscope}%
\pgfpathrectangle{\pgfqpoint{0.100000in}{0.212622in}}{\pgfqpoint{3.696000in}{3.696000in}}%
\pgfusepath{clip}%
\pgfsetbuttcap%
\pgfsetroundjoin%
\definecolor{currentfill}{rgb}{0.121569,0.466667,0.705882}%
\pgfsetfillcolor{currentfill}%
\pgfsetfillopacity{0.599532}%
\pgfsetlinewidth{1.003750pt}%
\definecolor{currentstroke}{rgb}{0.121569,0.466667,0.705882}%
\pgfsetstrokecolor{currentstroke}%
\pgfsetstrokeopacity{0.599532}%
\pgfsetdash{}{0pt}%
\pgfpathmoveto{\pgfqpoint{1.718285in}{3.072103in}}%
\pgfpathcurveto{\pgfqpoint{1.726521in}{3.072103in}}{\pgfqpoint{1.734421in}{3.075375in}}{\pgfqpoint{1.740245in}{3.081199in}}%
\pgfpathcurveto{\pgfqpoint{1.746069in}{3.087023in}}{\pgfqpoint{1.749341in}{3.094923in}}{\pgfqpoint{1.749341in}{3.103160in}}%
\pgfpathcurveto{\pgfqpoint{1.749341in}{3.111396in}}{\pgfqpoint{1.746069in}{3.119296in}}{\pgfqpoint{1.740245in}{3.125120in}}%
\pgfpathcurveto{\pgfqpoint{1.734421in}{3.130944in}}{\pgfqpoint{1.726521in}{3.134216in}}{\pgfqpoint{1.718285in}{3.134216in}}%
\pgfpathcurveto{\pgfqpoint{1.710049in}{3.134216in}}{\pgfqpoint{1.702149in}{3.130944in}}{\pgfqpoint{1.696325in}{3.125120in}}%
\pgfpathcurveto{\pgfqpoint{1.690501in}{3.119296in}}{\pgfqpoint{1.687228in}{3.111396in}}{\pgfqpoint{1.687228in}{3.103160in}}%
\pgfpathcurveto{\pgfqpoint{1.687228in}{3.094923in}}{\pgfqpoint{1.690501in}{3.087023in}}{\pgfqpoint{1.696325in}{3.081199in}}%
\pgfpathcurveto{\pgfqpoint{1.702149in}{3.075375in}}{\pgfqpoint{1.710049in}{3.072103in}}{\pgfqpoint{1.718285in}{3.072103in}}%
\pgfpathclose%
\pgfusepath{stroke,fill}%
\end{pgfscope}%
\begin{pgfscope}%
\pgfpathrectangle{\pgfqpoint{0.100000in}{0.212622in}}{\pgfqpoint{3.696000in}{3.696000in}}%
\pgfusepath{clip}%
\pgfsetbuttcap%
\pgfsetroundjoin%
\definecolor{currentfill}{rgb}{0.121569,0.466667,0.705882}%
\pgfsetfillcolor{currentfill}%
\pgfsetfillopacity{0.599532}%
\pgfsetlinewidth{1.003750pt}%
\definecolor{currentstroke}{rgb}{0.121569,0.466667,0.705882}%
\pgfsetstrokecolor{currentstroke}%
\pgfsetstrokeopacity{0.599532}%
\pgfsetdash{}{0pt}%
\pgfpathmoveto{\pgfqpoint{1.718285in}{3.072103in}}%
\pgfpathcurveto{\pgfqpoint{1.726521in}{3.072103in}}{\pgfqpoint{1.734421in}{3.075375in}}{\pgfqpoint{1.740245in}{3.081199in}}%
\pgfpathcurveto{\pgfqpoint{1.746069in}{3.087023in}}{\pgfqpoint{1.749341in}{3.094923in}}{\pgfqpoint{1.749341in}{3.103160in}}%
\pgfpathcurveto{\pgfqpoint{1.749341in}{3.111396in}}{\pgfqpoint{1.746069in}{3.119296in}}{\pgfqpoint{1.740245in}{3.125120in}}%
\pgfpathcurveto{\pgfqpoint{1.734421in}{3.130944in}}{\pgfqpoint{1.726521in}{3.134216in}}{\pgfqpoint{1.718285in}{3.134216in}}%
\pgfpathcurveto{\pgfqpoint{1.710049in}{3.134216in}}{\pgfqpoint{1.702149in}{3.130944in}}{\pgfqpoint{1.696325in}{3.125120in}}%
\pgfpathcurveto{\pgfqpoint{1.690501in}{3.119296in}}{\pgfqpoint{1.687228in}{3.111396in}}{\pgfqpoint{1.687228in}{3.103160in}}%
\pgfpathcurveto{\pgfqpoint{1.687228in}{3.094923in}}{\pgfqpoint{1.690501in}{3.087023in}}{\pgfqpoint{1.696325in}{3.081199in}}%
\pgfpathcurveto{\pgfqpoint{1.702149in}{3.075375in}}{\pgfqpoint{1.710049in}{3.072103in}}{\pgfqpoint{1.718285in}{3.072103in}}%
\pgfpathclose%
\pgfusepath{stroke,fill}%
\end{pgfscope}%
\begin{pgfscope}%
\pgfpathrectangle{\pgfqpoint{0.100000in}{0.212622in}}{\pgfqpoint{3.696000in}{3.696000in}}%
\pgfusepath{clip}%
\pgfsetbuttcap%
\pgfsetroundjoin%
\definecolor{currentfill}{rgb}{0.121569,0.466667,0.705882}%
\pgfsetfillcolor{currentfill}%
\pgfsetfillopacity{0.599532}%
\pgfsetlinewidth{1.003750pt}%
\definecolor{currentstroke}{rgb}{0.121569,0.466667,0.705882}%
\pgfsetstrokecolor{currentstroke}%
\pgfsetstrokeopacity{0.599532}%
\pgfsetdash{}{0pt}%
\pgfpathmoveto{\pgfqpoint{1.718285in}{3.072103in}}%
\pgfpathcurveto{\pgfqpoint{1.726521in}{3.072103in}}{\pgfqpoint{1.734421in}{3.075375in}}{\pgfqpoint{1.740245in}{3.081199in}}%
\pgfpathcurveto{\pgfqpoint{1.746069in}{3.087023in}}{\pgfqpoint{1.749341in}{3.094923in}}{\pgfqpoint{1.749341in}{3.103160in}}%
\pgfpathcurveto{\pgfqpoint{1.749341in}{3.111396in}}{\pgfqpoint{1.746069in}{3.119296in}}{\pgfqpoint{1.740245in}{3.125120in}}%
\pgfpathcurveto{\pgfqpoint{1.734421in}{3.130944in}}{\pgfqpoint{1.726521in}{3.134216in}}{\pgfqpoint{1.718285in}{3.134216in}}%
\pgfpathcurveto{\pgfqpoint{1.710049in}{3.134216in}}{\pgfqpoint{1.702149in}{3.130944in}}{\pgfqpoint{1.696325in}{3.125120in}}%
\pgfpathcurveto{\pgfqpoint{1.690501in}{3.119296in}}{\pgfqpoint{1.687228in}{3.111396in}}{\pgfqpoint{1.687228in}{3.103160in}}%
\pgfpathcurveto{\pgfqpoint{1.687228in}{3.094923in}}{\pgfqpoint{1.690501in}{3.087023in}}{\pgfqpoint{1.696325in}{3.081199in}}%
\pgfpathcurveto{\pgfqpoint{1.702149in}{3.075375in}}{\pgfqpoint{1.710049in}{3.072103in}}{\pgfqpoint{1.718285in}{3.072103in}}%
\pgfpathclose%
\pgfusepath{stroke,fill}%
\end{pgfscope}%
\begin{pgfscope}%
\pgfpathrectangle{\pgfqpoint{0.100000in}{0.212622in}}{\pgfqpoint{3.696000in}{3.696000in}}%
\pgfusepath{clip}%
\pgfsetbuttcap%
\pgfsetroundjoin%
\definecolor{currentfill}{rgb}{0.121569,0.466667,0.705882}%
\pgfsetfillcolor{currentfill}%
\pgfsetfillopacity{0.599532}%
\pgfsetlinewidth{1.003750pt}%
\definecolor{currentstroke}{rgb}{0.121569,0.466667,0.705882}%
\pgfsetstrokecolor{currentstroke}%
\pgfsetstrokeopacity{0.599532}%
\pgfsetdash{}{0pt}%
\pgfpathmoveto{\pgfqpoint{1.718285in}{3.072103in}}%
\pgfpathcurveto{\pgfqpoint{1.726521in}{3.072103in}}{\pgfqpoint{1.734421in}{3.075375in}}{\pgfqpoint{1.740245in}{3.081199in}}%
\pgfpathcurveto{\pgfqpoint{1.746069in}{3.087023in}}{\pgfqpoint{1.749341in}{3.094923in}}{\pgfqpoint{1.749341in}{3.103160in}}%
\pgfpathcurveto{\pgfqpoint{1.749341in}{3.111396in}}{\pgfqpoint{1.746069in}{3.119296in}}{\pgfqpoint{1.740245in}{3.125120in}}%
\pgfpathcurveto{\pgfqpoint{1.734421in}{3.130944in}}{\pgfqpoint{1.726521in}{3.134216in}}{\pgfqpoint{1.718285in}{3.134216in}}%
\pgfpathcurveto{\pgfqpoint{1.710049in}{3.134216in}}{\pgfqpoint{1.702149in}{3.130944in}}{\pgfqpoint{1.696325in}{3.125120in}}%
\pgfpathcurveto{\pgfqpoint{1.690501in}{3.119296in}}{\pgfqpoint{1.687228in}{3.111396in}}{\pgfqpoint{1.687228in}{3.103160in}}%
\pgfpathcurveto{\pgfqpoint{1.687228in}{3.094923in}}{\pgfqpoint{1.690501in}{3.087023in}}{\pgfqpoint{1.696325in}{3.081199in}}%
\pgfpathcurveto{\pgfqpoint{1.702149in}{3.075375in}}{\pgfqpoint{1.710049in}{3.072103in}}{\pgfqpoint{1.718285in}{3.072103in}}%
\pgfpathclose%
\pgfusepath{stroke,fill}%
\end{pgfscope}%
\begin{pgfscope}%
\pgfpathrectangle{\pgfqpoint{0.100000in}{0.212622in}}{\pgfqpoint{3.696000in}{3.696000in}}%
\pgfusepath{clip}%
\pgfsetbuttcap%
\pgfsetroundjoin%
\definecolor{currentfill}{rgb}{0.121569,0.466667,0.705882}%
\pgfsetfillcolor{currentfill}%
\pgfsetfillopacity{0.599532}%
\pgfsetlinewidth{1.003750pt}%
\definecolor{currentstroke}{rgb}{0.121569,0.466667,0.705882}%
\pgfsetstrokecolor{currentstroke}%
\pgfsetstrokeopacity{0.599532}%
\pgfsetdash{}{0pt}%
\pgfpathmoveto{\pgfqpoint{1.718285in}{3.072103in}}%
\pgfpathcurveto{\pgfqpoint{1.726521in}{3.072103in}}{\pgfqpoint{1.734421in}{3.075375in}}{\pgfqpoint{1.740245in}{3.081199in}}%
\pgfpathcurveto{\pgfqpoint{1.746069in}{3.087023in}}{\pgfqpoint{1.749341in}{3.094923in}}{\pgfqpoint{1.749341in}{3.103160in}}%
\pgfpathcurveto{\pgfqpoint{1.749341in}{3.111396in}}{\pgfqpoint{1.746069in}{3.119296in}}{\pgfqpoint{1.740245in}{3.125120in}}%
\pgfpathcurveto{\pgfqpoint{1.734421in}{3.130944in}}{\pgfqpoint{1.726521in}{3.134216in}}{\pgfqpoint{1.718285in}{3.134216in}}%
\pgfpathcurveto{\pgfqpoint{1.710049in}{3.134216in}}{\pgfqpoint{1.702149in}{3.130944in}}{\pgfqpoint{1.696325in}{3.125120in}}%
\pgfpathcurveto{\pgfqpoint{1.690501in}{3.119296in}}{\pgfqpoint{1.687228in}{3.111396in}}{\pgfqpoint{1.687228in}{3.103160in}}%
\pgfpathcurveto{\pgfqpoint{1.687228in}{3.094923in}}{\pgfqpoint{1.690501in}{3.087023in}}{\pgfqpoint{1.696325in}{3.081199in}}%
\pgfpathcurveto{\pgfqpoint{1.702149in}{3.075375in}}{\pgfqpoint{1.710049in}{3.072103in}}{\pgfqpoint{1.718285in}{3.072103in}}%
\pgfpathclose%
\pgfusepath{stroke,fill}%
\end{pgfscope}%
\begin{pgfscope}%
\pgfpathrectangle{\pgfqpoint{0.100000in}{0.212622in}}{\pgfqpoint{3.696000in}{3.696000in}}%
\pgfusepath{clip}%
\pgfsetbuttcap%
\pgfsetroundjoin%
\definecolor{currentfill}{rgb}{0.121569,0.466667,0.705882}%
\pgfsetfillcolor{currentfill}%
\pgfsetfillopacity{0.599532}%
\pgfsetlinewidth{1.003750pt}%
\definecolor{currentstroke}{rgb}{0.121569,0.466667,0.705882}%
\pgfsetstrokecolor{currentstroke}%
\pgfsetstrokeopacity{0.599532}%
\pgfsetdash{}{0pt}%
\pgfpathmoveto{\pgfqpoint{1.718285in}{3.072103in}}%
\pgfpathcurveto{\pgfqpoint{1.726521in}{3.072103in}}{\pgfqpoint{1.734421in}{3.075375in}}{\pgfqpoint{1.740245in}{3.081199in}}%
\pgfpathcurveto{\pgfqpoint{1.746069in}{3.087023in}}{\pgfqpoint{1.749341in}{3.094923in}}{\pgfqpoint{1.749341in}{3.103160in}}%
\pgfpathcurveto{\pgfqpoint{1.749341in}{3.111396in}}{\pgfqpoint{1.746069in}{3.119296in}}{\pgfqpoint{1.740245in}{3.125120in}}%
\pgfpathcurveto{\pgfqpoint{1.734421in}{3.130944in}}{\pgfqpoint{1.726521in}{3.134216in}}{\pgfqpoint{1.718285in}{3.134216in}}%
\pgfpathcurveto{\pgfqpoint{1.710049in}{3.134216in}}{\pgfqpoint{1.702149in}{3.130944in}}{\pgfqpoint{1.696325in}{3.125120in}}%
\pgfpathcurveto{\pgfqpoint{1.690501in}{3.119296in}}{\pgfqpoint{1.687228in}{3.111396in}}{\pgfqpoint{1.687228in}{3.103160in}}%
\pgfpathcurveto{\pgfqpoint{1.687228in}{3.094923in}}{\pgfqpoint{1.690501in}{3.087023in}}{\pgfqpoint{1.696325in}{3.081199in}}%
\pgfpathcurveto{\pgfqpoint{1.702149in}{3.075375in}}{\pgfqpoint{1.710049in}{3.072103in}}{\pgfqpoint{1.718285in}{3.072103in}}%
\pgfpathclose%
\pgfusepath{stroke,fill}%
\end{pgfscope}%
\begin{pgfscope}%
\pgfpathrectangle{\pgfqpoint{0.100000in}{0.212622in}}{\pgfqpoint{3.696000in}{3.696000in}}%
\pgfusepath{clip}%
\pgfsetbuttcap%
\pgfsetroundjoin%
\definecolor{currentfill}{rgb}{0.121569,0.466667,0.705882}%
\pgfsetfillcolor{currentfill}%
\pgfsetfillopacity{0.599532}%
\pgfsetlinewidth{1.003750pt}%
\definecolor{currentstroke}{rgb}{0.121569,0.466667,0.705882}%
\pgfsetstrokecolor{currentstroke}%
\pgfsetstrokeopacity{0.599532}%
\pgfsetdash{}{0pt}%
\pgfpathmoveto{\pgfqpoint{1.718285in}{3.072103in}}%
\pgfpathcurveto{\pgfqpoint{1.726521in}{3.072103in}}{\pgfqpoint{1.734421in}{3.075375in}}{\pgfqpoint{1.740245in}{3.081199in}}%
\pgfpathcurveto{\pgfqpoint{1.746069in}{3.087023in}}{\pgfqpoint{1.749341in}{3.094923in}}{\pgfqpoint{1.749341in}{3.103159in}}%
\pgfpathcurveto{\pgfqpoint{1.749341in}{3.111396in}}{\pgfqpoint{1.746069in}{3.119296in}}{\pgfqpoint{1.740245in}{3.125120in}}%
\pgfpathcurveto{\pgfqpoint{1.734421in}{3.130944in}}{\pgfqpoint{1.726521in}{3.134216in}}{\pgfqpoint{1.718285in}{3.134216in}}%
\pgfpathcurveto{\pgfqpoint{1.710049in}{3.134216in}}{\pgfqpoint{1.702149in}{3.130944in}}{\pgfqpoint{1.696325in}{3.125120in}}%
\pgfpathcurveto{\pgfqpoint{1.690501in}{3.119296in}}{\pgfqpoint{1.687228in}{3.111396in}}{\pgfqpoint{1.687228in}{3.103159in}}%
\pgfpathcurveto{\pgfqpoint{1.687228in}{3.094923in}}{\pgfqpoint{1.690501in}{3.087023in}}{\pgfqpoint{1.696325in}{3.081199in}}%
\pgfpathcurveto{\pgfqpoint{1.702149in}{3.075375in}}{\pgfqpoint{1.710049in}{3.072103in}}{\pgfqpoint{1.718285in}{3.072103in}}%
\pgfpathclose%
\pgfusepath{stroke,fill}%
\end{pgfscope}%
\begin{pgfscope}%
\pgfpathrectangle{\pgfqpoint{0.100000in}{0.212622in}}{\pgfqpoint{3.696000in}{3.696000in}}%
\pgfusepath{clip}%
\pgfsetbuttcap%
\pgfsetroundjoin%
\definecolor{currentfill}{rgb}{0.121569,0.466667,0.705882}%
\pgfsetfillcolor{currentfill}%
\pgfsetfillopacity{0.599532}%
\pgfsetlinewidth{1.003750pt}%
\definecolor{currentstroke}{rgb}{0.121569,0.466667,0.705882}%
\pgfsetstrokecolor{currentstroke}%
\pgfsetstrokeopacity{0.599532}%
\pgfsetdash{}{0pt}%
\pgfpathmoveto{\pgfqpoint{1.718285in}{3.072103in}}%
\pgfpathcurveto{\pgfqpoint{1.726521in}{3.072103in}}{\pgfqpoint{1.734421in}{3.075375in}}{\pgfqpoint{1.740245in}{3.081199in}}%
\pgfpathcurveto{\pgfqpoint{1.746069in}{3.087023in}}{\pgfqpoint{1.749341in}{3.094923in}}{\pgfqpoint{1.749341in}{3.103159in}}%
\pgfpathcurveto{\pgfqpoint{1.749341in}{3.111396in}}{\pgfqpoint{1.746069in}{3.119296in}}{\pgfqpoint{1.740245in}{3.125120in}}%
\pgfpathcurveto{\pgfqpoint{1.734421in}{3.130944in}}{\pgfqpoint{1.726521in}{3.134216in}}{\pgfqpoint{1.718285in}{3.134216in}}%
\pgfpathcurveto{\pgfqpoint{1.710049in}{3.134216in}}{\pgfqpoint{1.702149in}{3.130944in}}{\pgfqpoint{1.696325in}{3.125120in}}%
\pgfpathcurveto{\pgfqpoint{1.690501in}{3.119296in}}{\pgfqpoint{1.687228in}{3.111396in}}{\pgfqpoint{1.687228in}{3.103159in}}%
\pgfpathcurveto{\pgfqpoint{1.687228in}{3.094923in}}{\pgfqpoint{1.690501in}{3.087023in}}{\pgfqpoint{1.696325in}{3.081199in}}%
\pgfpathcurveto{\pgfqpoint{1.702149in}{3.075375in}}{\pgfqpoint{1.710049in}{3.072103in}}{\pgfqpoint{1.718285in}{3.072103in}}%
\pgfpathclose%
\pgfusepath{stroke,fill}%
\end{pgfscope}%
\begin{pgfscope}%
\pgfpathrectangle{\pgfqpoint{0.100000in}{0.212622in}}{\pgfqpoint{3.696000in}{3.696000in}}%
\pgfusepath{clip}%
\pgfsetbuttcap%
\pgfsetroundjoin%
\definecolor{currentfill}{rgb}{0.121569,0.466667,0.705882}%
\pgfsetfillcolor{currentfill}%
\pgfsetfillopacity{0.599532}%
\pgfsetlinewidth{1.003750pt}%
\definecolor{currentstroke}{rgb}{0.121569,0.466667,0.705882}%
\pgfsetstrokecolor{currentstroke}%
\pgfsetstrokeopacity{0.599532}%
\pgfsetdash{}{0pt}%
\pgfpathmoveto{\pgfqpoint{1.718285in}{3.072103in}}%
\pgfpathcurveto{\pgfqpoint{1.726521in}{3.072103in}}{\pgfqpoint{1.734421in}{3.075375in}}{\pgfqpoint{1.740245in}{3.081199in}}%
\pgfpathcurveto{\pgfqpoint{1.746069in}{3.087023in}}{\pgfqpoint{1.749341in}{3.094923in}}{\pgfqpoint{1.749341in}{3.103159in}}%
\pgfpathcurveto{\pgfqpoint{1.749341in}{3.111396in}}{\pgfqpoint{1.746069in}{3.119296in}}{\pgfqpoint{1.740245in}{3.125120in}}%
\pgfpathcurveto{\pgfqpoint{1.734421in}{3.130944in}}{\pgfqpoint{1.726521in}{3.134216in}}{\pgfqpoint{1.718285in}{3.134216in}}%
\pgfpathcurveto{\pgfqpoint{1.710049in}{3.134216in}}{\pgfqpoint{1.702149in}{3.130944in}}{\pgfqpoint{1.696325in}{3.125120in}}%
\pgfpathcurveto{\pgfqpoint{1.690501in}{3.119296in}}{\pgfqpoint{1.687228in}{3.111396in}}{\pgfqpoint{1.687228in}{3.103159in}}%
\pgfpathcurveto{\pgfqpoint{1.687228in}{3.094923in}}{\pgfqpoint{1.690501in}{3.087023in}}{\pgfqpoint{1.696325in}{3.081199in}}%
\pgfpathcurveto{\pgfqpoint{1.702149in}{3.075375in}}{\pgfqpoint{1.710049in}{3.072103in}}{\pgfqpoint{1.718285in}{3.072103in}}%
\pgfpathclose%
\pgfusepath{stroke,fill}%
\end{pgfscope}%
\begin{pgfscope}%
\pgfpathrectangle{\pgfqpoint{0.100000in}{0.212622in}}{\pgfqpoint{3.696000in}{3.696000in}}%
\pgfusepath{clip}%
\pgfsetbuttcap%
\pgfsetroundjoin%
\definecolor{currentfill}{rgb}{0.121569,0.466667,0.705882}%
\pgfsetfillcolor{currentfill}%
\pgfsetfillopacity{0.599532}%
\pgfsetlinewidth{1.003750pt}%
\definecolor{currentstroke}{rgb}{0.121569,0.466667,0.705882}%
\pgfsetstrokecolor{currentstroke}%
\pgfsetstrokeopacity{0.599532}%
\pgfsetdash{}{0pt}%
\pgfpathmoveto{\pgfqpoint{1.718285in}{3.072103in}}%
\pgfpathcurveto{\pgfqpoint{1.726521in}{3.072103in}}{\pgfqpoint{1.734421in}{3.075375in}}{\pgfqpoint{1.740245in}{3.081199in}}%
\pgfpathcurveto{\pgfqpoint{1.746069in}{3.087023in}}{\pgfqpoint{1.749341in}{3.094923in}}{\pgfqpoint{1.749341in}{3.103159in}}%
\pgfpathcurveto{\pgfqpoint{1.749341in}{3.111396in}}{\pgfqpoint{1.746069in}{3.119296in}}{\pgfqpoint{1.740245in}{3.125120in}}%
\pgfpathcurveto{\pgfqpoint{1.734421in}{3.130944in}}{\pgfqpoint{1.726521in}{3.134216in}}{\pgfqpoint{1.718285in}{3.134216in}}%
\pgfpathcurveto{\pgfqpoint{1.710049in}{3.134216in}}{\pgfqpoint{1.702149in}{3.130944in}}{\pgfqpoint{1.696325in}{3.125120in}}%
\pgfpathcurveto{\pgfqpoint{1.690501in}{3.119296in}}{\pgfqpoint{1.687228in}{3.111396in}}{\pgfqpoint{1.687228in}{3.103159in}}%
\pgfpathcurveto{\pgfqpoint{1.687228in}{3.094923in}}{\pgfqpoint{1.690501in}{3.087023in}}{\pgfqpoint{1.696325in}{3.081199in}}%
\pgfpathcurveto{\pgfqpoint{1.702149in}{3.075375in}}{\pgfqpoint{1.710049in}{3.072103in}}{\pgfqpoint{1.718285in}{3.072103in}}%
\pgfpathclose%
\pgfusepath{stroke,fill}%
\end{pgfscope}%
\begin{pgfscope}%
\pgfpathrectangle{\pgfqpoint{0.100000in}{0.212622in}}{\pgfqpoint{3.696000in}{3.696000in}}%
\pgfusepath{clip}%
\pgfsetbuttcap%
\pgfsetroundjoin%
\definecolor{currentfill}{rgb}{0.121569,0.466667,0.705882}%
\pgfsetfillcolor{currentfill}%
\pgfsetfillopacity{0.599532}%
\pgfsetlinewidth{1.003750pt}%
\definecolor{currentstroke}{rgb}{0.121569,0.466667,0.705882}%
\pgfsetstrokecolor{currentstroke}%
\pgfsetstrokeopacity{0.599532}%
\pgfsetdash{}{0pt}%
\pgfpathmoveto{\pgfqpoint{1.718285in}{3.072103in}}%
\pgfpathcurveto{\pgfqpoint{1.726521in}{3.072103in}}{\pgfqpoint{1.734421in}{3.075375in}}{\pgfqpoint{1.740245in}{3.081199in}}%
\pgfpathcurveto{\pgfqpoint{1.746069in}{3.087023in}}{\pgfqpoint{1.749341in}{3.094923in}}{\pgfqpoint{1.749341in}{3.103159in}}%
\pgfpathcurveto{\pgfqpoint{1.749341in}{3.111396in}}{\pgfqpoint{1.746069in}{3.119296in}}{\pgfqpoint{1.740245in}{3.125120in}}%
\pgfpathcurveto{\pgfqpoint{1.734421in}{3.130944in}}{\pgfqpoint{1.726521in}{3.134216in}}{\pgfqpoint{1.718285in}{3.134216in}}%
\pgfpathcurveto{\pgfqpoint{1.710049in}{3.134216in}}{\pgfqpoint{1.702149in}{3.130944in}}{\pgfqpoint{1.696325in}{3.125120in}}%
\pgfpathcurveto{\pgfqpoint{1.690501in}{3.119296in}}{\pgfqpoint{1.687228in}{3.111396in}}{\pgfqpoint{1.687228in}{3.103159in}}%
\pgfpathcurveto{\pgfqpoint{1.687228in}{3.094923in}}{\pgfqpoint{1.690501in}{3.087023in}}{\pgfqpoint{1.696325in}{3.081199in}}%
\pgfpathcurveto{\pgfqpoint{1.702149in}{3.075375in}}{\pgfqpoint{1.710049in}{3.072103in}}{\pgfqpoint{1.718285in}{3.072103in}}%
\pgfpathclose%
\pgfusepath{stroke,fill}%
\end{pgfscope}%
\begin{pgfscope}%
\pgfpathrectangle{\pgfqpoint{0.100000in}{0.212622in}}{\pgfqpoint{3.696000in}{3.696000in}}%
\pgfusepath{clip}%
\pgfsetbuttcap%
\pgfsetroundjoin%
\definecolor{currentfill}{rgb}{0.121569,0.466667,0.705882}%
\pgfsetfillcolor{currentfill}%
\pgfsetfillopacity{0.599532}%
\pgfsetlinewidth{1.003750pt}%
\definecolor{currentstroke}{rgb}{0.121569,0.466667,0.705882}%
\pgfsetstrokecolor{currentstroke}%
\pgfsetstrokeopacity{0.599532}%
\pgfsetdash{}{0pt}%
\pgfpathmoveto{\pgfqpoint{1.718285in}{3.072103in}}%
\pgfpathcurveto{\pgfqpoint{1.726521in}{3.072103in}}{\pgfqpoint{1.734421in}{3.075375in}}{\pgfqpoint{1.740245in}{3.081199in}}%
\pgfpathcurveto{\pgfqpoint{1.746069in}{3.087023in}}{\pgfqpoint{1.749341in}{3.094923in}}{\pgfqpoint{1.749341in}{3.103159in}}%
\pgfpathcurveto{\pgfqpoint{1.749341in}{3.111396in}}{\pgfqpoint{1.746069in}{3.119296in}}{\pgfqpoint{1.740245in}{3.125120in}}%
\pgfpathcurveto{\pgfqpoint{1.734421in}{3.130944in}}{\pgfqpoint{1.726521in}{3.134216in}}{\pgfqpoint{1.718285in}{3.134216in}}%
\pgfpathcurveto{\pgfqpoint{1.710049in}{3.134216in}}{\pgfqpoint{1.702149in}{3.130944in}}{\pgfqpoint{1.696325in}{3.125120in}}%
\pgfpathcurveto{\pgfqpoint{1.690501in}{3.119296in}}{\pgfqpoint{1.687228in}{3.111396in}}{\pgfqpoint{1.687228in}{3.103159in}}%
\pgfpathcurveto{\pgfqpoint{1.687228in}{3.094923in}}{\pgfqpoint{1.690501in}{3.087023in}}{\pgfqpoint{1.696325in}{3.081199in}}%
\pgfpathcurveto{\pgfqpoint{1.702149in}{3.075375in}}{\pgfqpoint{1.710049in}{3.072103in}}{\pgfqpoint{1.718285in}{3.072103in}}%
\pgfpathclose%
\pgfusepath{stroke,fill}%
\end{pgfscope}%
\begin{pgfscope}%
\pgfpathrectangle{\pgfqpoint{0.100000in}{0.212622in}}{\pgfqpoint{3.696000in}{3.696000in}}%
\pgfusepath{clip}%
\pgfsetbuttcap%
\pgfsetroundjoin%
\definecolor{currentfill}{rgb}{0.121569,0.466667,0.705882}%
\pgfsetfillcolor{currentfill}%
\pgfsetfillopacity{0.599532}%
\pgfsetlinewidth{1.003750pt}%
\definecolor{currentstroke}{rgb}{0.121569,0.466667,0.705882}%
\pgfsetstrokecolor{currentstroke}%
\pgfsetstrokeopacity{0.599532}%
\pgfsetdash{}{0pt}%
\pgfpathmoveto{\pgfqpoint{1.718285in}{3.072103in}}%
\pgfpathcurveto{\pgfqpoint{1.726521in}{3.072103in}}{\pgfqpoint{1.734421in}{3.075375in}}{\pgfqpoint{1.740245in}{3.081199in}}%
\pgfpathcurveto{\pgfqpoint{1.746069in}{3.087023in}}{\pgfqpoint{1.749341in}{3.094923in}}{\pgfqpoint{1.749341in}{3.103159in}}%
\pgfpathcurveto{\pgfqpoint{1.749341in}{3.111396in}}{\pgfqpoint{1.746069in}{3.119296in}}{\pgfqpoint{1.740245in}{3.125120in}}%
\pgfpathcurveto{\pgfqpoint{1.734421in}{3.130944in}}{\pgfqpoint{1.726521in}{3.134216in}}{\pgfqpoint{1.718285in}{3.134216in}}%
\pgfpathcurveto{\pgfqpoint{1.710049in}{3.134216in}}{\pgfqpoint{1.702149in}{3.130944in}}{\pgfqpoint{1.696325in}{3.125120in}}%
\pgfpathcurveto{\pgfqpoint{1.690501in}{3.119296in}}{\pgfqpoint{1.687228in}{3.111396in}}{\pgfqpoint{1.687228in}{3.103159in}}%
\pgfpathcurveto{\pgfqpoint{1.687228in}{3.094923in}}{\pgfqpoint{1.690501in}{3.087023in}}{\pgfqpoint{1.696325in}{3.081199in}}%
\pgfpathcurveto{\pgfqpoint{1.702149in}{3.075375in}}{\pgfqpoint{1.710049in}{3.072103in}}{\pgfqpoint{1.718285in}{3.072103in}}%
\pgfpathclose%
\pgfusepath{stroke,fill}%
\end{pgfscope}%
\begin{pgfscope}%
\pgfpathrectangle{\pgfqpoint{0.100000in}{0.212622in}}{\pgfqpoint{3.696000in}{3.696000in}}%
\pgfusepath{clip}%
\pgfsetbuttcap%
\pgfsetroundjoin%
\definecolor{currentfill}{rgb}{0.121569,0.466667,0.705882}%
\pgfsetfillcolor{currentfill}%
\pgfsetfillopacity{0.599532}%
\pgfsetlinewidth{1.003750pt}%
\definecolor{currentstroke}{rgb}{0.121569,0.466667,0.705882}%
\pgfsetstrokecolor{currentstroke}%
\pgfsetstrokeopacity{0.599532}%
\pgfsetdash{}{0pt}%
\pgfpathmoveto{\pgfqpoint{1.718285in}{3.072103in}}%
\pgfpathcurveto{\pgfqpoint{1.726521in}{3.072103in}}{\pgfqpoint{1.734421in}{3.075375in}}{\pgfqpoint{1.740245in}{3.081199in}}%
\pgfpathcurveto{\pgfqpoint{1.746069in}{3.087023in}}{\pgfqpoint{1.749341in}{3.094923in}}{\pgfqpoint{1.749341in}{3.103159in}}%
\pgfpathcurveto{\pgfqpoint{1.749341in}{3.111396in}}{\pgfqpoint{1.746069in}{3.119296in}}{\pgfqpoint{1.740245in}{3.125120in}}%
\pgfpathcurveto{\pgfqpoint{1.734421in}{3.130944in}}{\pgfqpoint{1.726521in}{3.134216in}}{\pgfqpoint{1.718285in}{3.134216in}}%
\pgfpathcurveto{\pgfqpoint{1.710049in}{3.134216in}}{\pgfqpoint{1.702149in}{3.130944in}}{\pgfqpoint{1.696325in}{3.125120in}}%
\pgfpathcurveto{\pgfqpoint{1.690501in}{3.119296in}}{\pgfqpoint{1.687228in}{3.111396in}}{\pgfqpoint{1.687228in}{3.103159in}}%
\pgfpathcurveto{\pgfqpoint{1.687228in}{3.094923in}}{\pgfqpoint{1.690501in}{3.087023in}}{\pgfqpoint{1.696325in}{3.081199in}}%
\pgfpathcurveto{\pgfqpoint{1.702149in}{3.075375in}}{\pgfqpoint{1.710049in}{3.072103in}}{\pgfqpoint{1.718285in}{3.072103in}}%
\pgfpathclose%
\pgfusepath{stroke,fill}%
\end{pgfscope}%
\begin{pgfscope}%
\pgfpathrectangle{\pgfqpoint{0.100000in}{0.212622in}}{\pgfqpoint{3.696000in}{3.696000in}}%
\pgfusepath{clip}%
\pgfsetbuttcap%
\pgfsetroundjoin%
\definecolor{currentfill}{rgb}{0.121569,0.466667,0.705882}%
\pgfsetfillcolor{currentfill}%
\pgfsetfillopacity{0.599532}%
\pgfsetlinewidth{1.003750pt}%
\definecolor{currentstroke}{rgb}{0.121569,0.466667,0.705882}%
\pgfsetstrokecolor{currentstroke}%
\pgfsetstrokeopacity{0.599532}%
\pgfsetdash{}{0pt}%
\pgfpathmoveto{\pgfqpoint{1.718285in}{3.072103in}}%
\pgfpathcurveto{\pgfqpoint{1.726521in}{3.072103in}}{\pgfqpoint{1.734421in}{3.075375in}}{\pgfqpoint{1.740245in}{3.081199in}}%
\pgfpathcurveto{\pgfqpoint{1.746069in}{3.087023in}}{\pgfqpoint{1.749341in}{3.094923in}}{\pgfqpoint{1.749341in}{3.103159in}}%
\pgfpathcurveto{\pgfqpoint{1.749341in}{3.111396in}}{\pgfqpoint{1.746069in}{3.119296in}}{\pgfqpoint{1.740245in}{3.125120in}}%
\pgfpathcurveto{\pgfqpoint{1.734421in}{3.130944in}}{\pgfqpoint{1.726521in}{3.134216in}}{\pgfqpoint{1.718285in}{3.134216in}}%
\pgfpathcurveto{\pgfqpoint{1.710049in}{3.134216in}}{\pgfqpoint{1.702149in}{3.130944in}}{\pgfqpoint{1.696325in}{3.125120in}}%
\pgfpathcurveto{\pgfqpoint{1.690501in}{3.119296in}}{\pgfqpoint{1.687228in}{3.111396in}}{\pgfqpoint{1.687228in}{3.103159in}}%
\pgfpathcurveto{\pgfqpoint{1.687228in}{3.094923in}}{\pgfqpoint{1.690501in}{3.087023in}}{\pgfqpoint{1.696325in}{3.081199in}}%
\pgfpathcurveto{\pgfqpoint{1.702149in}{3.075375in}}{\pgfqpoint{1.710049in}{3.072103in}}{\pgfqpoint{1.718285in}{3.072103in}}%
\pgfpathclose%
\pgfusepath{stroke,fill}%
\end{pgfscope}%
\begin{pgfscope}%
\pgfpathrectangle{\pgfqpoint{0.100000in}{0.212622in}}{\pgfqpoint{3.696000in}{3.696000in}}%
\pgfusepath{clip}%
\pgfsetbuttcap%
\pgfsetroundjoin%
\definecolor{currentfill}{rgb}{0.121569,0.466667,0.705882}%
\pgfsetfillcolor{currentfill}%
\pgfsetfillopacity{0.599532}%
\pgfsetlinewidth{1.003750pt}%
\definecolor{currentstroke}{rgb}{0.121569,0.466667,0.705882}%
\pgfsetstrokecolor{currentstroke}%
\pgfsetstrokeopacity{0.599532}%
\pgfsetdash{}{0pt}%
\pgfpathmoveto{\pgfqpoint{1.718285in}{3.072103in}}%
\pgfpathcurveto{\pgfqpoint{1.726521in}{3.072103in}}{\pgfqpoint{1.734421in}{3.075375in}}{\pgfqpoint{1.740245in}{3.081199in}}%
\pgfpathcurveto{\pgfqpoint{1.746069in}{3.087023in}}{\pgfqpoint{1.749341in}{3.094923in}}{\pgfqpoint{1.749341in}{3.103159in}}%
\pgfpathcurveto{\pgfqpoint{1.749341in}{3.111396in}}{\pgfqpoint{1.746069in}{3.119296in}}{\pgfqpoint{1.740245in}{3.125120in}}%
\pgfpathcurveto{\pgfqpoint{1.734421in}{3.130944in}}{\pgfqpoint{1.726521in}{3.134216in}}{\pgfqpoint{1.718285in}{3.134216in}}%
\pgfpathcurveto{\pgfqpoint{1.710049in}{3.134216in}}{\pgfqpoint{1.702149in}{3.130944in}}{\pgfqpoint{1.696325in}{3.125120in}}%
\pgfpathcurveto{\pgfqpoint{1.690501in}{3.119296in}}{\pgfqpoint{1.687228in}{3.111396in}}{\pgfqpoint{1.687228in}{3.103159in}}%
\pgfpathcurveto{\pgfqpoint{1.687228in}{3.094923in}}{\pgfqpoint{1.690501in}{3.087023in}}{\pgfqpoint{1.696325in}{3.081199in}}%
\pgfpathcurveto{\pgfqpoint{1.702149in}{3.075375in}}{\pgfqpoint{1.710049in}{3.072103in}}{\pgfqpoint{1.718285in}{3.072103in}}%
\pgfpathclose%
\pgfusepath{stroke,fill}%
\end{pgfscope}%
\begin{pgfscope}%
\pgfpathrectangle{\pgfqpoint{0.100000in}{0.212622in}}{\pgfqpoint{3.696000in}{3.696000in}}%
\pgfusepath{clip}%
\pgfsetbuttcap%
\pgfsetroundjoin%
\definecolor{currentfill}{rgb}{0.121569,0.466667,0.705882}%
\pgfsetfillcolor{currentfill}%
\pgfsetfillopacity{0.599532}%
\pgfsetlinewidth{1.003750pt}%
\definecolor{currentstroke}{rgb}{0.121569,0.466667,0.705882}%
\pgfsetstrokecolor{currentstroke}%
\pgfsetstrokeopacity{0.599532}%
\pgfsetdash{}{0pt}%
\pgfpathmoveto{\pgfqpoint{1.718285in}{3.072103in}}%
\pgfpathcurveto{\pgfqpoint{1.726521in}{3.072103in}}{\pgfqpoint{1.734421in}{3.075375in}}{\pgfqpoint{1.740245in}{3.081199in}}%
\pgfpathcurveto{\pgfqpoint{1.746069in}{3.087023in}}{\pgfqpoint{1.749341in}{3.094923in}}{\pgfqpoint{1.749341in}{3.103159in}}%
\pgfpathcurveto{\pgfqpoint{1.749341in}{3.111396in}}{\pgfqpoint{1.746069in}{3.119296in}}{\pgfqpoint{1.740245in}{3.125120in}}%
\pgfpathcurveto{\pgfqpoint{1.734421in}{3.130944in}}{\pgfqpoint{1.726521in}{3.134216in}}{\pgfqpoint{1.718285in}{3.134216in}}%
\pgfpathcurveto{\pgfqpoint{1.710049in}{3.134216in}}{\pgfqpoint{1.702149in}{3.130944in}}{\pgfqpoint{1.696325in}{3.125120in}}%
\pgfpathcurveto{\pgfqpoint{1.690501in}{3.119296in}}{\pgfqpoint{1.687228in}{3.111396in}}{\pgfqpoint{1.687228in}{3.103159in}}%
\pgfpathcurveto{\pgfqpoint{1.687228in}{3.094923in}}{\pgfqpoint{1.690501in}{3.087023in}}{\pgfqpoint{1.696325in}{3.081199in}}%
\pgfpathcurveto{\pgfqpoint{1.702149in}{3.075375in}}{\pgfqpoint{1.710049in}{3.072103in}}{\pgfqpoint{1.718285in}{3.072103in}}%
\pgfpathclose%
\pgfusepath{stroke,fill}%
\end{pgfscope}%
\begin{pgfscope}%
\pgfpathrectangle{\pgfqpoint{0.100000in}{0.212622in}}{\pgfqpoint{3.696000in}{3.696000in}}%
\pgfusepath{clip}%
\pgfsetbuttcap%
\pgfsetroundjoin%
\definecolor{currentfill}{rgb}{0.121569,0.466667,0.705882}%
\pgfsetfillcolor{currentfill}%
\pgfsetfillopacity{0.599532}%
\pgfsetlinewidth{1.003750pt}%
\definecolor{currentstroke}{rgb}{0.121569,0.466667,0.705882}%
\pgfsetstrokecolor{currentstroke}%
\pgfsetstrokeopacity{0.599532}%
\pgfsetdash{}{0pt}%
\pgfpathmoveto{\pgfqpoint{1.718285in}{3.072103in}}%
\pgfpathcurveto{\pgfqpoint{1.726521in}{3.072103in}}{\pgfqpoint{1.734421in}{3.075375in}}{\pgfqpoint{1.740245in}{3.081199in}}%
\pgfpathcurveto{\pgfqpoint{1.746069in}{3.087023in}}{\pgfqpoint{1.749341in}{3.094923in}}{\pgfqpoint{1.749341in}{3.103159in}}%
\pgfpathcurveto{\pgfqpoint{1.749341in}{3.111396in}}{\pgfqpoint{1.746069in}{3.119296in}}{\pgfqpoint{1.740245in}{3.125120in}}%
\pgfpathcurveto{\pgfqpoint{1.734421in}{3.130944in}}{\pgfqpoint{1.726521in}{3.134216in}}{\pgfqpoint{1.718285in}{3.134216in}}%
\pgfpathcurveto{\pgfqpoint{1.710049in}{3.134216in}}{\pgfqpoint{1.702149in}{3.130944in}}{\pgfqpoint{1.696325in}{3.125120in}}%
\pgfpathcurveto{\pgfqpoint{1.690501in}{3.119296in}}{\pgfqpoint{1.687228in}{3.111396in}}{\pgfqpoint{1.687228in}{3.103159in}}%
\pgfpathcurveto{\pgfqpoint{1.687228in}{3.094923in}}{\pgfqpoint{1.690501in}{3.087023in}}{\pgfqpoint{1.696325in}{3.081199in}}%
\pgfpathcurveto{\pgfqpoint{1.702149in}{3.075375in}}{\pgfqpoint{1.710049in}{3.072103in}}{\pgfqpoint{1.718285in}{3.072103in}}%
\pgfpathclose%
\pgfusepath{stroke,fill}%
\end{pgfscope}%
\begin{pgfscope}%
\pgfpathrectangle{\pgfqpoint{0.100000in}{0.212622in}}{\pgfqpoint{3.696000in}{3.696000in}}%
\pgfusepath{clip}%
\pgfsetbuttcap%
\pgfsetroundjoin%
\definecolor{currentfill}{rgb}{0.121569,0.466667,0.705882}%
\pgfsetfillcolor{currentfill}%
\pgfsetfillopacity{0.599532}%
\pgfsetlinewidth{1.003750pt}%
\definecolor{currentstroke}{rgb}{0.121569,0.466667,0.705882}%
\pgfsetstrokecolor{currentstroke}%
\pgfsetstrokeopacity{0.599532}%
\pgfsetdash{}{0pt}%
\pgfpathmoveto{\pgfqpoint{1.718285in}{3.072103in}}%
\pgfpathcurveto{\pgfqpoint{1.726521in}{3.072103in}}{\pgfqpoint{1.734421in}{3.075375in}}{\pgfqpoint{1.740245in}{3.081199in}}%
\pgfpathcurveto{\pgfqpoint{1.746069in}{3.087023in}}{\pgfqpoint{1.749341in}{3.094923in}}{\pgfqpoint{1.749341in}{3.103159in}}%
\pgfpathcurveto{\pgfqpoint{1.749341in}{3.111396in}}{\pgfqpoint{1.746069in}{3.119296in}}{\pgfqpoint{1.740245in}{3.125120in}}%
\pgfpathcurveto{\pgfqpoint{1.734421in}{3.130944in}}{\pgfqpoint{1.726521in}{3.134216in}}{\pgfqpoint{1.718285in}{3.134216in}}%
\pgfpathcurveto{\pgfqpoint{1.710049in}{3.134216in}}{\pgfqpoint{1.702149in}{3.130944in}}{\pgfqpoint{1.696325in}{3.125120in}}%
\pgfpathcurveto{\pgfqpoint{1.690501in}{3.119296in}}{\pgfqpoint{1.687228in}{3.111396in}}{\pgfqpoint{1.687228in}{3.103159in}}%
\pgfpathcurveto{\pgfqpoint{1.687228in}{3.094923in}}{\pgfqpoint{1.690501in}{3.087023in}}{\pgfqpoint{1.696325in}{3.081199in}}%
\pgfpathcurveto{\pgfqpoint{1.702149in}{3.075375in}}{\pgfqpoint{1.710049in}{3.072103in}}{\pgfqpoint{1.718285in}{3.072103in}}%
\pgfpathclose%
\pgfusepath{stroke,fill}%
\end{pgfscope}%
\begin{pgfscope}%
\pgfpathrectangle{\pgfqpoint{0.100000in}{0.212622in}}{\pgfqpoint{3.696000in}{3.696000in}}%
\pgfusepath{clip}%
\pgfsetbuttcap%
\pgfsetroundjoin%
\definecolor{currentfill}{rgb}{0.121569,0.466667,0.705882}%
\pgfsetfillcolor{currentfill}%
\pgfsetfillopacity{0.599532}%
\pgfsetlinewidth{1.003750pt}%
\definecolor{currentstroke}{rgb}{0.121569,0.466667,0.705882}%
\pgfsetstrokecolor{currentstroke}%
\pgfsetstrokeopacity{0.599532}%
\pgfsetdash{}{0pt}%
\pgfpathmoveto{\pgfqpoint{1.718285in}{3.072103in}}%
\pgfpathcurveto{\pgfqpoint{1.726521in}{3.072103in}}{\pgfqpoint{1.734421in}{3.075375in}}{\pgfqpoint{1.740245in}{3.081199in}}%
\pgfpathcurveto{\pgfqpoint{1.746069in}{3.087023in}}{\pgfqpoint{1.749341in}{3.094923in}}{\pgfqpoint{1.749341in}{3.103159in}}%
\pgfpathcurveto{\pgfqpoint{1.749341in}{3.111396in}}{\pgfqpoint{1.746069in}{3.119296in}}{\pgfqpoint{1.740245in}{3.125120in}}%
\pgfpathcurveto{\pgfqpoint{1.734421in}{3.130944in}}{\pgfqpoint{1.726521in}{3.134216in}}{\pgfqpoint{1.718285in}{3.134216in}}%
\pgfpathcurveto{\pgfqpoint{1.710049in}{3.134216in}}{\pgfqpoint{1.702149in}{3.130944in}}{\pgfqpoint{1.696325in}{3.125120in}}%
\pgfpathcurveto{\pgfqpoint{1.690501in}{3.119296in}}{\pgfqpoint{1.687228in}{3.111396in}}{\pgfqpoint{1.687228in}{3.103159in}}%
\pgfpathcurveto{\pgfqpoint{1.687228in}{3.094923in}}{\pgfqpoint{1.690501in}{3.087023in}}{\pgfqpoint{1.696325in}{3.081199in}}%
\pgfpathcurveto{\pgfqpoint{1.702149in}{3.075375in}}{\pgfqpoint{1.710049in}{3.072103in}}{\pgfqpoint{1.718285in}{3.072103in}}%
\pgfpathclose%
\pgfusepath{stroke,fill}%
\end{pgfscope}%
\begin{pgfscope}%
\pgfpathrectangle{\pgfqpoint{0.100000in}{0.212622in}}{\pgfqpoint{3.696000in}{3.696000in}}%
\pgfusepath{clip}%
\pgfsetbuttcap%
\pgfsetroundjoin%
\definecolor{currentfill}{rgb}{0.121569,0.466667,0.705882}%
\pgfsetfillcolor{currentfill}%
\pgfsetfillopacity{0.599532}%
\pgfsetlinewidth{1.003750pt}%
\definecolor{currentstroke}{rgb}{0.121569,0.466667,0.705882}%
\pgfsetstrokecolor{currentstroke}%
\pgfsetstrokeopacity{0.599532}%
\pgfsetdash{}{0pt}%
\pgfpathmoveto{\pgfqpoint{1.718285in}{3.072103in}}%
\pgfpathcurveto{\pgfqpoint{1.726521in}{3.072103in}}{\pgfqpoint{1.734421in}{3.075375in}}{\pgfqpoint{1.740245in}{3.081199in}}%
\pgfpathcurveto{\pgfqpoint{1.746069in}{3.087023in}}{\pgfqpoint{1.749341in}{3.094923in}}{\pgfqpoint{1.749341in}{3.103159in}}%
\pgfpathcurveto{\pgfqpoint{1.749341in}{3.111396in}}{\pgfqpoint{1.746069in}{3.119296in}}{\pgfqpoint{1.740245in}{3.125120in}}%
\pgfpathcurveto{\pgfqpoint{1.734421in}{3.130944in}}{\pgfqpoint{1.726521in}{3.134216in}}{\pgfqpoint{1.718285in}{3.134216in}}%
\pgfpathcurveto{\pgfqpoint{1.710049in}{3.134216in}}{\pgfqpoint{1.702149in}{3.130944in}}{\pgfqpoint{1.696325in}{3.125120in}}%
\pgfpathcurveto{\pgfqpoint{1.690501in}{3.119296in}}{\pgfqpoint{1.687228in}{3.111396in}}{\pgfqpoint{1.687228in}{3.103159in}}%
\pgfpathcurveto{\pgfqpoint{1.687228in}{3.094923in}}{\pgfqpoint{1.690501in}{3.087023in}}{\pgfqpoint{1.696325in}{3.081199in}}%
\pgfpathcurveto{\pgfqpoint{1.702149in}{3.075375in}}{\pgfqpoint{1.710049in}{3.072103in}}{\pgfqpoint{1.718285in}{3.072103in}}%
\pgfpathclose%
\pgfusepath{stroke,fill}%
\end{pgfscope}%
\begin{pgfscope}%
\pgfpathrectangle{\pgfqpoint{0.100000in}{0.212622in}}{\pgfqpoint{3.696000in}{3.696000in}}%
\pgfusepath{clip}%
\pgfsetbuttcap%
\pgfsetroundjoin%
\definecolor{currentfill}{rgb}{0.121569,0.466667,0.705882}%
\pgfsetfillcolor{currentfill}%
\pgfsetfillopacity{0.599532}%
\pgfsetlinewidth{1.003750pt}%
\definecolor{currentstroke}{rgb}{0.121569,0.466667,0.705882}%
\pgfsetstrokecolor{currentstroke}%
\pgfsetstrokeopacity{0.599532}%
\pgfsetdash{}{0pt}%
\pgfpathmoveto{\pgfqpoint{1.718285in}{3.072103in}}%
\pgfpathcurveto{\pgfqpoint{1.726521in}{3.072103in}}{\pgfqpoint{1.734421in}{3.075375in}}{\pgfqpoint{1.740245in}{3.081199in}}%
\pgfpathcurveto{\pgfqpoint{1.746069in}{3.087023in}}{\pgfqpoint{1.749341in}{3.094923in}}{\pgfqpoint{1.749341in}{3.103159in}}%
\pgfpathcurveto{\pgfqpoint{1.749341in}{3.111396in}}{\pgfqpoint{1.746069in}{3.119296in}}{\pgfqpoint{1.740245in}{3.125120in}}%
\pgfpathcurveto{\pgfqpoint{1.734421in}{3.130944in}}{\pgfqpoint{1.726521in}{3.134216in}}{\pgfqpoint{1.718285in}{3.134216in}}%
\pgfpathcurveto{\pgfqpoint{1.710049in}{3.134216in}}{\pgfqpoint{1.702149in}{3.130944in}}{\pgfqpoint{1.696325in}{3.125120in}}%
\pgfpathcurveto{\pgfqpoint{1.690501in}{3.119296in}}{\pgfqpoint{1.687228in}{3.111396in}}{\pgfqpoint{1.687228in}{3.103159in}}%
\pgfpathcurveto{\pgfqpoint{1.687228in}{3.094923in}}{\pgfqpoint{1.690501in}{3.087023in}}{\pgfqpoint{1.696325in}{3.081199in}}%
\pgfpathcurveto{\pgfqpoint{1.702149in}{3.075375in}}{\pgfqpoint{1.710049in}{3.072103in}}{\pgfqpoint{1.718285in}{3.072103in}}%
\pgfpathclose%
\pgfusepath{stroke,fill}%
\end{pgfscope}%
\begin{pgfscope}%
\pgfpathrectangle{\pgfqpoint{0.100000in}{0.212622in}}{\pgfqpoint{3.696000in}{3.696000in}}%
\pgfusepath{clip}%
\pgfsetbuttcap%
\pgfsetroundjoin%
\definecolor{currentfill}{rgb}{0.121569,0.466667,0.705882}%
\pgfsetfillcolor{currentfill}%
\pgfsetfillopacity{0.599532}%
\pgfsetlinewidth{1.003750pt}%
\definecolor{currentstroke}{rgb}{0.121569,0.466667,0.705882}%
\pgfsetstrokecolor{currentstroke}%
\pgfsetstrokeopacity{0.599532}%
\pgfsetdash{}{0pt}%
\pgfpathmoveto{\pgfqpoint{1.718285in}{3.072103in}}%
\pgfpathcurveto{\pgfqpoint{1.726521in}{3.072103in}}{\pgfqpoint{1.734421in}{3.075375in}}{\pgfqpoint{1.740245in}{3.081199in}}%
\pgfpathcurveto{\pgfqpoint{1.746069in}{3.087023in}}{\pgfqpoint{1.749341in}{3.094923in}}{\pgfqpoint{1.749341in}{3.103159in}}%
\pgfpathcurveto{\pgfqpoint{1.749341in}{3.111396in}}{\pgfqpoint{1.746069in}{3.119296in}}{\pgfqpoint{1.740245in}{3.125120in}}%
\pgfpathcurveto{\pgfqpoint{1.734421in}{3.130944in}}{\pgfqpoint{1.726521in}{3.134216in}}{\pgfqpoint{1.718285in}{3.134216in}}%
\pgfpathcurveto{\pgfqpoint{1.710049in}{3.134216in}}{\pgfqpoint{1.702149in}{3.130944in}}{\pgfqpoint{1.696325in}{3.125120in}}%
\pgfpathcurveto{\pgfqpoint{1.690501in}{3.119296in}}{\pgfqpoint{1.687228in}{3.111396in}}{\pgfqpoint{1.687228in}{3.103159in}}%
\pgfpathcurveto{\pgfqpoint{1.687228in}{3.094923in}}{\pgfqpoint{1.690501in}{3.087023in}}{\pgfqpoint{1.696325in}{3.081199in}}%
\pgfpathcurveto{\pgfqpoint{1.702149in}{3.075375in}}{\pgfqpoint{1.710049in}{3.072103in}}{\pgfqpoint{1.718285in}{3.072103in}}%
\pgfpathclose%
\pgfusepath{stroke,fill}%
\end{pgfscope}%
\begin{pgfscope}%
\pgfpathrectangle{\pgfqpoint{0.100000in}{0.212622in}}{\pgfqpoint{3.696000in}{3.696000in}}%
\pgfusepath{clip}%
\pgfsetbuttcap%
\pgfsetroundjoin%
\definecolor{currentfill}{rgb}{0.121569,0.466667,0.705882}%
\pgfsetfillcolor{currentfill}%
\pgfsetfillopacity{0.599532}%
\pgfsetlinewidth{1.003750pt}%
\definecolor{currentstroke}{rgb}{0.121569,0.466667,0.705882}%
\pgfsetstrokecolor{currentstroke}%
\pgfsetstrokeopacity{0.599532}%
\pgfsetdash{}{0pt}%
\pgfpathmoveto{\pgfqpoint{1.718285in}{3.072103in}}%
\pgfpathcurveto{\pgfqpoint{1.726521in}{3.072103in}}{\pgfqpoint{1.734421in}{3.075375in}}{\pgfqpoint{1.740245in}{3.081199in}}%
\pgfpathcurveto{\pgfqpoint{1.746069in}{3.087023in}}{\pgfqpoint{1.749341in}{3.094923in}}{\pgfqpoint{1.749341in}{3.103159in}}%
\pgfpathcurveto{\pgfqpoint{1.749341in}{3.111396in}}{\pgfqpoint{1.746069in}{3.119296in}}{\pgfqpoint{1.740245in}{3.125120in}}%
\pgfpathcurveto{\pgfqpoint{1.734421in}{3.130944in}}{\pgfqpoint{1.726521in}{3.134216in}}{\pgfqpoint{1.718285in}{3.134216in}}%
\pgfpathcurveto{\pgfqpoint{1.710049in}{3.134216in}}{\pgfqpoint{1.702149in}{3.130944in}}{\pgfqpoint{1.696325in}{3.125120in}}%
\pgfpathcurveto{\pgfqpoint{1.690501in}{3.119296in}}{\pgfqpoint{1.687228in}{3.111396in}}{\pgfqpoint{1.687228in}{3.103159in}}%
\pgfpathcurveto{\pgfqpoint{1.687228in}{3.094923in}}{\pgfqpoint{1.690501in}{3.087023in}}{\pgfqpoint{1.696325in}{3.081199in}}%
\pgfpathcurveto{\pgfqpoint{1.702149in}{3.075375in}}{\pgfqpoint{1.710049in}{3.072103in}}{\pgfqpoint{1.718285in}{3.072103in}}%
\pgfpathclose%
\pgfusepath{stroke,fill}%
\end{pgfscope}%
\begin{pgfscope}%
\pgfpathrectangle{\pgfqpoint{0.100000in}{0.212622in}}{\pgfqpoint{3.696000in}{3.696000in}}%
\pgfusepath{clip}%
\pgfsetbuttcap%
\pgfsetroundjoin%
\definecolor{currentfill}{rgb}{0.121569,0.466667,0.705882}%
\pgfsetfillcolor{currentfill}%
\pgfsetfillopacity{0.599532}%
\pgfsetlinewidth{1.003750pt}%
\definecolor{currentstroke}{rgb}{0.121569,0.466667,0.705882}%
\pgfsetstrokecolor{currentstroke}%
\pgfsetstrokeopacity{0.599532}%
\pgfsetdash{}{0pt}%
\pgfpathmoveto{\pgfqpoint{1.718285in}{3.072103in}}%
\pgfpathcurveto{\pgfqpoint{1.726521in}{3.072103in}}{\pgfqpoint{1.734421in}{3.075375in}}{\pgfqpoint{1.740245in}{3.081199in}}%
\pgfpathcurveto{\pgfqpoint{1.746069in}{3.087023in}}{\pgfqpoint{1.749341in}{3.094923in}}{\pgfqpoint{1.749341in}{3.103159in}}%
\pgfpathcurveto{\pgfqpoint{1.749341in}{3.111396in}}{\pgfqpoint{1.746069in}{3.119296in}}{\pgfqpoint{1.740245in}{3.125120in}}%
\pgfpathcurveto{\pgfqpoint{1.734421in}{3.130944in}}{\pgfqpoint{1.726521in}{3.134216in}}{\pgfqpoint{1.718285in}{3.134216in}}%
\pgfpathcurveto{\pgfqpoint{1.710049in}{3.134216in}}{\pgfqpoint{1.702149in}{3.130944in}}{\pgfqpoint{1.696325in}{3.125120in}}%
\pgfpathcurveto{\pgfqpoint{1.690501in}{3.119296in}}{\pgfqpoint{1.687228in}{3.111396in}}{\pgfqpoint{1.687228in}{3.103159in}}%
\pgfpathcurveto{\pgfqpoint{1.687228in}{3.094923in}}{\pgfqpoint{1.690501in}{3.087023in}}{\pgfqpoint{1.696325in}{3.081199in}}%
\pgfpathcurveto{\pgfqpoint{1.702149in}{3.075375in}}{\pgfqpoint{1.710049in}{3.072103in}}{\pgfqpoint{1.718285in}{3.072103in}}%
\pgfpathclose%
\pgfusepath{stroke,fill}%
\end{pgfscope}%
\begin{pgfscope}%
\pgfpathrectangle{\pgfqpoint{0.100000in}{0.212622in}}{\pgfqpoint{3.696000in}{3.696000in}}%
\pgfusepath{clip}%
\pgfsetbuttcap%
\pgfsetroundjoin%
\definecolor{currentfill}{rgb}{0.121569,0.466667,0.705882}%
\pgfsetfillcolor{currentfill}%
\pgfsetfillopacity{0.599625}%
\pgfsetlinewidth{1.003750pt}%
\definecolor{currentstroke}{rgb}{0.121569,0.466667,0.705882}%
\pgfsetstrokecolor{currentstroke}%
\pgfsetstrokeopacity{0.599625}%
\pgfsetdash{}{0pt}%
\pgfpathmoveto{\pgfqpoint{1.719913in}{3.072861in}}%
\pgfpathcurveto{\pgfqpoint{1.728150in}{3.072861in}}{\pgfqpoint{1.736050in}{3.076134in}}{\pgfqpoint{1.741874in}{3.081958in}}%
\pgfpathcurveto{\pgfqpoint{1.747698in}{3.087782in}}{\pgfqpoint{1.750970in}{3.095682in}}{\pgfqpoint{1.750970in}{3.103918in}}%
\pgfpathcurveto{\pgfqpoint{1.750970in}{3.112154in}}{\pgfqpoint{1.747698in}{3.120054in}}{\pgfqpoint{1.741874in}{3.125878in}}%
\pgfpathcurveto{\pgfqpoint{1.736050in}{3.131702in}}{\pgfqpoint{1.728150in}{3.134974in}}{\pgfqpoint{1.719913in}{3.134974in}}%
\pgfpathcurveto{\pgfqpoint{1.711677in}{3.134974in}}{\pgfqpoint{1.703777in}{3.131702in}}{\pgfqpoint{1.697953in}{3.125878in}}%
\pgfpathcurveto{\pgfqpoint{1.692129in}{3.120054in}}{\pgfqpoint{1.688857in}{3.112154in}}{\pgfqpoint{1.688857in}{3.103918in}}%
\pgfpathcurveto{\pgfqpoint{1.688857in}{3.095682in}}{\pgfqpoint{1.692129in}{3.087782in}}{\pgfqpoint{1.697953in}{3.081958in}}%
\pgfpathcurveto{\pgfqpoint{1.703777in}{3.076134in}}{\pgfqpoint{1.711677in}{3.072861in}}{\pgfqpoint{1.719913in}{3.072861in}}%
\pgfpathclose%
\pgfusepath{stroke,fill}%
\end{pgfscope}%
\begin{pgfscope}%
\pgfpathrectangle{\pgfqpoint{0.100000in}{0.212622in}}{\pgfqpoint{3.696000in}{3.696000in}}%
\pgfusepath{clip}%
\pgfsetbuttcap%
\pgfsetroundjoin%
\definecolor{currentfill}{rgb}{0.121569,0.466667,0.705882}%
\pgfsetfillcolor{currentfill}%
\pgfsetfillopacity{0.599627}%
\pgfsetlinewidth{1.003750pt}%
\definecolor{currentstroke}{rgb}{0.121569,0.466667,0.705882}%
\pgfsetstrokecolor{currentstroke}%
\pgfsetstrokeopacity{0.599627}%
\pgfsetdash{}{0pt}%
\pgfpathmoveto{\pgfqpoint{1.718125in}{3.071736in}}%
\pgfpathcurveto{\pgfqpoint{1.726361in}{3.071736in}}{\pgfqpoint{1.734261in}{3.075008in}}{\pgfqpoint{1.740085in}{3.080832in}}%
\pgfpathcurveto{\pgfqpoint{1.745909in}{3.086656in}}{\pgfqpoint{1.749181in}{3.094556in}}{\pgfqpoint{1.749181in}{3.102792in}}%
\pgfpathcurveto{\pgfqpoint{1.749181in}{3.111028in}}{\pgfqpoint{1.745909in}{3.118928in}}{\pgfqpoint{1.740085in}{3.124752in}}%
\pgfpathcurveto{\pgfqpoint{1.734261in}{3.130576in}}{\pgfqpoint{1.726361in}{3.133849in}}{\pgfqpoint{1.718125in}{3.133849in}}%
\pgfpathcurveto{\pgfqpoint{1.709889in}{3.133849in}}{\pgfqpoint{1.701989in}{3.130576in}}{\pgfqpoint{1.696165in}{3.124752in}}%
\pgfpathcurveto{\pgfqpoint{1.690341in}{3.118928in}}{\pgfqpoint{1.687068in}{3.111028in}}{\pgfqpoint{1.687068in}{3.102792in}}%
\pgfpathcurveto{\pgfqpoint{1.687068in}{3.094556in}}{\pgfqpoint{1.690341in}{3.086656in}}{\pgfqpoint{1.696165in}{3.080832in}}%
\pgfpathcurveto{\pgfqpoint{1.701989in}{3.075008in}}{\pgfqpoint{1.709889in}{3.071736in}}{\pgfqpoint{1.718125in}{3.071736in}}%
\pgfpathclose%
\pgfusepath{stroke,fill}%
\end{pgfscope}%
\begin{pgfscope}%
\pgfpathrectangle{\pgfqpoint{0.100000in}{0.212622in}}{\pgfqpoint{3.696000in}{3.696000in}}%
\pgfusepath{clip}%
\pgfsetbuttcap%
\pgfsetroundjoin%
\definecolor{currentfill}{rgb}{0.121569,0.466667,0.705882}%
\pgfsetfillcolor{currentfill}%
\pgfsetfillopacity{0.599800}%
\pgfsetlinewidth{1.003750pt}%
\definecolor{currentstroke}{rgb}{0.121569,0.466667,0.705882}%
\pgfsetstrokecolor{currentstroke}%
\pgfsetstrokeopacity{0.599800}%
\pgfsetdash{}{0pt}%
\pgfpathmoveto{\pgfqpoint{1.717769in}{3.071082in}}%
\pgfpathcurveto{\pgfqpoint{1.726005in}{3.071082in}}{\pgfqpoint{1.733905in}{3.074354in}}{\pgfqpoint{1.739729in}{3.080178in}}%
\pgfpathcurveto{\pgfqpoint{1.745553in}{3.086002in}}{\pgfqpoint{1.748825in}{3.093902in}}{\pgfqpoint{1.748825in}{3.102138in}}%
\pgfpathcurveto{\pgfqpoint{1.748825in}{3.110375in}}{\pgfqpoint{1.745553in}{3.118275in}}{\pgfqpoint{1.739729in}{3.124099in}}%
\pgfpathcurveto{\pgfqpoint{1.733905in}{3.129922in}}{\pgfqpoint{1.726005in}{3.133195in}}{\pgfqpoint{1.717769in}{3.133195in}}%
\pgfpathcurveto{\pgfqpoint{1.709533in}{3.133195in}}{\pgfqpoint{1.701632in}{3.129922in}}{\pgfqpoint{1.695809in}{3.124099in}}%
\pgfpathcurveto{\pgfqpoint{1.689985in}{3.118275in}}{\pgfqpoint{1.686712in}{3.110375in}}{\pgfqpoint{1.686712in}{3.102138in}}%
\pgfpathcurveto{\pgfqpoint{1.686712in}{3.093902in}}{\pgfqpoint{1.689985in}{3.086002in}}{\pgfqpoint{1.695809in}{3.080178in}}%
\pgfpathcurveto{\pgfqpoint{1.701632in}{3.074354in}}{\pgfqpoint{1.709533in}{3.071082in}}{\pgfqpoint{1.717769in}{3.071082in}}%
\pgfpathclose%
\pgfusepath{stroke,fill}%
\end{pgfscope}%
\begin{pgfscope}%
\pgfpathrectangle{\pgfqpoint{0.100000in}{0.212622in}}{\pgfqpoint{3.696000in}{3.696000in}}%
\pgfusepath{clip}%
\pgfsetbuttcap%
\pgfsetroundjoin%
\definecolor{currentfill}{rgb}{0.121569,0.466667,0.705882}%
\pgfsetfillcolor{currentfill}%
\pgfsetfillopacity{0.599808}%
\pgfsetlinewidth{1.003750pt}%
\definecolor{currentstroke}{rgb}{0.121569,0.466667,0.705882}%
\pgfsetstrokecolor{currentstroke}%
\pgfsetstrokeopacity{0.599808}%
\pgfsetdash{}{0pt}%
\pgfpathmoveto{\pgfqpoint{1.720755in}{3.072693in}}%
\pgfpathcurveto{\pgfqpoint{1.728991in}{3.072693in}}{\pgfqpoint{1.736891in}{3.075966in}}{\pgfqpoint{1.742715in}{3.081790in}}%
\pgfpathcurveto{\pgfqpoint{1.748539in}{3.087614in}}{\pgfqpoint{1.751811in}{3.095514in}}{\pgfqpoint{1.751811in}{3.103750in}}%
\pgfpathcurveto{\pgfqpoint{1.751811in}{3.111986in}}{\pgfqpoint{1.748539in}{3.119886in}}{\pgfqpoint{1.742715in}{3.125710in}}%
\pgfpathcurveto{\pgfqpoint{1.736891in}{3.131534in}}{\pgfqpoint{1.728991in}{3.134806in}}{\pgfqpoint{1.720755in}{3.134806in}}%
\pgfpathcurveto{\pgfqpoint{1.712518in}{3.134806in}}{\pgfqpoint{1.704618in}{3.131534in}}{\pgfqpoint{1.698794in}{3.125710in}}%
\pgfpathcurveto{\pgfqpoint{1.692971in}{3.119886in}}{\pgfqpoint{1.689698in}{3.111986in}}{\pgfqpoint{1.689698in}{3.103750in}}%
\pgfpathcurveto{\pgfqpoint{1.689698in}{3.095514in}}{\pgfqpoint{1.692971in}{3.087614in}}{\pgfqpoint{1.698794in}{3.081790in}}%
\pgfpathcurveto{\pgfqpoint{1.704618in}{3.075966in}}{\pgfqpoint{1.712518in}{3.072693in}}{\pgfqpoint{1.720755in}{3.072693in}}%
\pgfpathclose%
\pgfusepath{stroke,fill}%
\end{pgfscope}%
\begin{pgfscope}%
\pgfpathrectangle{\pgfqpoint{0.100000in}{0.212622in}}{\pgfqpoint{3.696000in}{3.696000in}}%
\pgfusepath{clip}%
\pgfsetbuttcap%
\pgfsetroundjoin%
\definecolor{currentfill}{rgb}{0.121569,0.466667,0.705882}%
\pgfsetfillcolor{currentfill}%
\pgfsetfillopacity{0.599881}%
\pgfsetlinewidth{1.003750pt}%
\definecolor{currentstroke}{rgb}{0.121569,0.466667,0.705882}%
\pgfsetstrokecolor{currentstroke}%
\pgfsetstrokeopacity{0.599881}%
\pgfsetdash{}{0pt}%
\pgfpathmoveto{\pgfqpoint{1.717548in}{3.070679in}}%
\pgfpathcurveto{\pgfqpoint{1.725784in}{3.070679in}}{\pgfqpoint{1.733684in}{3.073951in}}{\pgfqpoint{1.739508in}{3.079775in}}%
\pgfpathcurveto{\pgfqpoint{1.745332in}{3.085599in}}{\pgfqpoint{1.748605in}{3.093499in}}{\pgfqpoint{1.748605in}{3.101735in}}%
\pgfpathcurveto{\pgfqpoint{1.748605in}{3.109971in}}{\pgfqpoint{1.745332in}{3.117871in}}{\pgfqpoint{1.739508in}{3.123695in}}%
\pgfpathcurveto{\pgfqpoint{1.733684in}{3.129519in}}{\pgfqpoint{1.725784in}{3.132792in}}{\pgfqpoint{1.717548in}{3.132792in}}%
\pgfpathcurveto{\pgfqpoint{1.709312in}{3.132792in}}{\pgfqpoint{1.701412in}{3.129519in}}{\pgfqpoint{1.695588in}{3.123695in}}%
\pgfpathcurveto{\pgfqpoint{1.689764in}{3.117871in}}{\pgfqpoint{1.686492in}{3.109971in}}{\pgfqpoint{1.686492in}{3.101735in}}%
\pgfpathcurveto{\pgfqpoint{1.686492in}{3.093499in}}{\pgfqpoint{1.689764in}{3.085599in}}{\pgfqpoint{1.695588in}{3.079775in}}%
\pgfpathcurveto{\pgfqpoint{1.701412in}{3.073951in}}{\pgfqpoint{1.709312in}{3.070679in}}{\pgfqpoint{1.717548in}{3.070679in}}%
\pgfpathclose%
\pgfusepath{stroke,fill}%
\end{pgfscope}%
\begin{pgfscope}%
\pgfpathrectangle{\pgfqpoint{0.100000in}{0.212622in}}{\pgfqpoint{3.696000in}{3.696000in}}%
\pgfusepath{clip}%
\pgfsetbuttcap%
\pgfsetroundjoin%
\definecolor{currentfill}{rgb}{0.121569,0.466667,0.705882}%
\pgfsetfillcolor{currentfill}%
\pgfsetfillopacity{0.599931}%
\pgfsetlinewidth{1.003750pt}%
\definecolor{currentstroke}{rgb}{0.121569,0.466667,0.705882}%
\pgfsetstrokecolor{currentstroke}%
\pgfsetstrokeopacity{0.599931}%
\pgfsetdash{}{0pt}%
\pgfpathmoveto{\pgfqpoint{1.717475in}{3.070456in}}%
\pgfpathcurveto{\pgfqpoint{1.725711in}{3.070456in}}{\pgfqpoint{1.733611in}{3.073729in}}{\pgfqpoint{1.739435in}{3.079553in}}%
\pgfpathcurveto{\pgfqpoint{1.745259in}{3.085377in}}{\pgfqpoint{1.748532in}{3.093277in}}{\pgfqpoint{1.748532in}{3.101513in}}%
\pgfpathcurveto{\pgfqpoint{1.748532in}{3.109749in}}{\pgfqpoint{1.745259in}{3.117649in}}{\pgfqpoint{1.739435in}{3.123473in}}%
\pgfpathcurveto{\pgfqpoint{1.733611in}{3.129297in}}{\pgfqpoint{1.725711in}{3.132569in}}{\pgfqpoint{1.717475in}{3.132569in}}%
\pgfpathcurveto{\pgfqpoint{1.709239in}{3.132569in}}{\pgfqpoint{1.701339in}{3.129297in}}{\pgfqpoint{1.695515in}{3.123473in}}%
\pgfpathcurveto{\pgfqpoint{1.689691in}{3.117649in}}{\pgfqpoint{1.686419in}{3.109749in}}{\pgfqpoint{1.686419in}{3.101513in}}%
\pgfpathcurveto{\pgfqpoint{1.686419in}{3.093277in}}{\pgfqpoint{1.689691in}{3.085377in}}{\pgfqpoint{1.695515in}{3.079553in}}%
\pgfpathcurveto{\pgfqpoint{1.701339in}{3.073729in}}{\pgfqpoint{1.709239in}{3.070456in}}{\pgfqpoint{1.717475in}{3.070456in}}%
\pgfpathclose%
\pgfusepath{stroke,fill}%
\end{pgfscope}%
\begin{pgfscope}%
\pgfpathrectangle{\pgfqpoint{0.100000in}{0.212622in}}{\pgfqpoint{3.696000in}{3.696000in}}%
\pgfusepath{clip}%
\pgfsetbuttcap%
\pgfsetroundjoin%
\definecolor{currentfill}{rgb}{0.121569,0.466667,0.705882}%
\pgfsetfillcolor{currentfill}%
\pgfsetfillopacity{0.599959}%
\pgfsetlinewidth{1.003750pt}%
\definecolor{currentstroke}{rgb}{0.121569,0.466667,0.705882}%
\pgfsetstrokecolor{currentstroke}%
\pgfsetstrokeopacity{0.599959}%
\pgfsetdash{}{0pt}%
\pgfpathmoveto{\pgfqpoint{1.717436in}{3.070334in}}%
\pgfpathcurveto{\pgfqpoint{1.725672in}{3.070334in}}{\pgfqpoint{1.733572in}{3.073606in}}{\pgfqpoint{1.739396in}{3.079430in}}%
\pgfpathcurveto{\pgfqpoint{1.745220in}{3.085254in}}{\pgfqpoint{1.748493in}{3.093154in}}{\pgfqpoint{1.748493in}{3.101390in}}%
\pgfpathcurveto{\pgfqpoint{1.748493in}{3.109627in}}{\pgfqpoint{1.745220in}{3.117527in}}{\pgfqpoint{1.739396in}{3.123351in}}%
\pgfpathcurveto{\pgfqpoint{1.733572in}{3.129175in}}{\pgfqpoint{1.725672in}{3.132447in}}{\pgfqpoint{1.717436in}{3.132447in}}%
\pgfpathcurveto{\pgfqpoint{1.709200in}{3.132447in}}{\pgfqpoint{1.701300in}{3.129175in}}{\pgfqpoint{1.695476in}{3.123351in}}%
\pgfpathcurveto{\pgfqpoint{1.689652in}{3.117527in}}{\pgfqpoint{1.686380in}{3.109627in}}{\pgfqpoint{1.686380in}{3.101390in}}%
\pgfpathcurveto{\pgfqpoint{1.686380in}{3.093154in}}{\pgfqpoint{1.689652in}{3.085254in}}{\pgfqpoint{1.695476in}{3.079430in}}%
\pgfpathcurveto{\pgfqpoint{1.701300in}{3.073606in}}{\pgfqpoint{1.709200in}{3.070334in}}{\pgfqpoint{1.717436in}{3.070334in}}%
\pgfpathclose%
\pgfusepath{stroke,fill}%
\end{pgfscope}%
\begin{pgfscope}%
\pgfpathrectangle{\pgfqpoint{0.100000in}{0.212622in}}{\pgfqpoint{3.696000in}{3.696000in}}%
\pgfusepath{clip}%
\pgfsetbuttcap%
\pgfsetroundjoin%
\definecolor{currentfill}{rgb}{0.121569,0.466667,0.705882}%
\pgfsetfillcolor{currentfill}%
\pgfsetfillopacity{0.600119}%
\pgfsetlinewidth{1.003750pt}%
\definecolor{currentstroke}{rgb}{0.121569,0.466667,0.705882}%
\pgfsetstrokecolor{currentstroke}%
\pgfsetstrokeopacity{0.600119}%
\pgfsetdash{}{0pt}%
\pgfpathmoveto{\pgfqpoint{1.716986in}{3.069634in}}%
\pgfpathcurveto{\pgfqpoint{1.725222in}{3.069634in}}{\pgfqpoint{1.733122in}{3.072906in}}{\pgfqpoint{1.738946in}{3.078730in}}%
\pgfpathcurveto{\pgfqpoint{1.744770in}{3.084554in}}{\pgfqpoint{1.748042in}{3.092454in}}{\pgfqpoint{1.748042in}{3.100691in}}%
\pgfpathcurveto{\pgfqpoint{1.748042in}{3.108927in}}{\pgfqpoint{1.744770in}{3.116827in}}{\pgfqpoint{1.738946in}{3.122651in}}%
\pgfpathcurveto{\pgfqpoint{1.733122in}{3.128475in}}{\pgfqpoint{1.725222in}{3.131747in}}{\pgfqpoint{1.716986in}{3.131747in}}%
\pgfpathcurveto{\pgfqpoint{1.708749in}{3.131747in}}{\pgfqpoint{1.700849in}{3.128475in}}{\pgfqpoint{1.695025in}{3.122651in}}%
\pgfpathcurveto{\pgfqpoint{1.689201in}{3.116827in}}{\pgfqpoint{1.685929in}{3.108927in}}{\pgfqpoint{1.685929in}{3.100691in}}%
\pgfpathcurveto{\pgfqpoint{1.685929in}{3.092454in}}{\pgfqpoint{1.689201in}{3.084554in}}{\pgfqpoint{1.695025in}{3.078730in}}%
\pgfpathcurveto{\pgfqpoint{1.700849in}{3.072906in}}{\pgfqpoint{1.708749in}{3.069634in}}{\pgfqpoint{1.716986in}{3.069634in}}%
\pgfpathclose%
\pgfusepath{stroke,fill}%
\end{pgfscope}%
\begin{pgfscope}%
\pgfpathrectangle{\pgfqpoint{0.100000in}{0.212622in}}{\pgfqpoint{3.696000in}{3.696000in}}%
\pgfusepath{clip}%
\pgfsetbuttcap%
\pgfsetroundjoin%
\definecolor{currentfill}{rgb}{0.121569,0.466667,0.705882}%
\pgfsetfillcolor{currentfill}%
\pgfsetfillopacity{0.600144}%
\pgfsetlinewidth{1.003750pt}%
\definecolor{currentstroke}{rgb}{0.121569,0.466667,0.705882}%
\pgfsetstrokecolor{currentstroke}%
\pgfsetstrokeopacity{0.600144}%
\pgfsetdash{}{0pt}%
\pgfpathmoveto{\pgfqpoint{1.722264in}{3.072332in}}%
\pgfpathcurveto{\pgfqpoint{1.730500in}{3.072332in}}{\pgfqpoint{1.738400in}{3.075604in}}{\pgfqpoint{1.744224in}{3.081428in}}%
\pgfpathcurveto{\pgfqpoint{1.750048in}{3.087252in}}{\pgfqpoint{1.753321in}{3.095152in}}{\pgfqpoint{1.753321in}{3.103388in}}%
\pgfpathcurveto{\pgfqpoint{1.753321in}{3.111625in}}{\pgfqpoint{1.750048in}{3.119525in}}{\pgfqpoint{1.744224in}{3.125349in}}%
\pgfpathcurveto{\pgfqpoint{1.738400in}{3.131173in}}{\pgfqpoint{1.730500in}{3.134445in}}{\pgfqpoint{1.722264in}{3.134445in}}%
\pgfpathcurveto{\pgfqpoint{1.714028in}{3.134445in}}{\pgfqpoint{1.706128in}{3.131173in}}{\pgfqpoint{1.700304in}{3.125349in}}%
\pgfpathcurveto{\pgfqpoint{1.694480in}{3.119525in}}{\pgfqpoint{1.691208in}{3.111625in}}{\pgfqpoint{1.691208in}{3.103388in}}%
\pgfpathcurveto{\pgfqpoint{1.691208in}{3.095152in}}{\pgfqpoint{1.694480in}{3.087252in}}{\pgfqpoint{1.700304in}{3.081428in}}%
\pgfpathcurveto{\pgfqpoint{1.706128in}{3.075604in}}{\pgfqpoint{1.714028in}{3.072332in}}{\pgfqpoint{1.722264in}{3.072332in}}%
\pgfpathclose%
\pgfusepath{stroke,fill}%
\end{pgfscope}%
\begin{pgfscope}%
\pgfpathrectangle{\pgfqpoint{0.100000in}{0.212622in}}{\pgfqpoint{3.696000in}{3.696000in}}%
\pgfusepath{clip}%
\pgfsetbuttcap%
\pgfsetroundjoin%
\definecolor{currentfill}{rgb}{0.121569,0.466667,0.705882}%
\pgfsetfillcolor{currentfill}%
\pgfsetfillopacity{0.600211}%
\pgfsetlinewidth{1.003750pt}%
\definecolor{currentstroke}{rgb}{0.121569,0.466667,0.705882}%
\pgfsetstrokecolor{currentstroke}%
\pgfsetstrokeopacity{0.600211}%
\pgfsetdash{}{0pt}%
\pgfpathmoveto{\pgfqpoint{1.716782in}{3.069221in}}%
\pgfpathcurveto{\pgfqpoint{1.725018in}{3.069221in}}{\pgfqpoint{1.732918in}{3.072493in}}{\pgfqpoint{1.738742in}{3.078317in}}%
\pgfpathcurveto{\pgfqpoint{1.744566in}{3.084141in}}{\pgfqpoint{1.747838in}{3.092041in}}{\pgfqpoint{1.747838in}{3.100277in}}%
\pgfpathcurveto{\pgfqpoint{1.747838in}{3.108514in}}{\pgfqpoint{1.744566in}{3.116414in}}{\pgfqpoint{1.738742in}{3.122238in}}%
\pgfpathcurveto{\pgfqpoint{1.732918in}{3.128062in}}{\pgfqpoint{1.725018in}{3.131334in}}{\pgfqpoint{1.716782in}{3.131334in}}%
\pgfpathcurveto{\pgfqpoint{1.708545in}{3.131334in}}{\pgfqpoint{1.700645in}{3.128062in}}{\pgfqpoint{1.694821in}{3.122238in}}%
\pgfpathcurveto{\pgfqpoint{1.688997in}{3.116414in}}{\pgfqpoint{1.685725in}{3.108514in}}{\pgfqpoint{1.685725in}{3.100277in}}%
\pgfpathcurveto{\pgfqpoint{1.685725in}{3.092041in}}{\pgfqpoint{1.688997in}{3.084141in}}{\pgfqpoint{1.694821in}{3.078317in}}%
\pgfpathcurveto{\pgfqpoint{1.700645in}{3.072493in}}{\pgfqpoint{1.708545in}{3.069221in}}{\pgfqpoint{1.716782in}{3.069221in}}%
\pgfpathclose%
\pgfusepath{stroke,fill}%
\end{pgfscope}%
\begin{pgfscope}%
\pgfpathrectangle{\pgfqpoint{0.100000in}{0.212622in}}{\pgfqpoint{3.696000in}{3.696000in}}%
\pgfusepath{clip}%
\pgfsetbuttcap%
\pgfsetroundjoin%
\definecolor{currentfill}{rgb}{0.121569,0.466667,0.705882}%
\pgfsetfillcolor{currentfill}%
\pgfsetfillopacity{0.600414}%
\pgfsetlinewidth{1.003750pt}%
\definecolor{currentstroke}{rgb}{0.121569,0.466667,0.705882}%
\pgfsetstrokecolor{currentstroke}%
\pgfsetstrokeopacity{0.600414}%
\pgfsetdash{}{0pt}%
\pgfpathmoveto{\pgfqpoint{1.716512in}{3.068143in}}%
\pgfpathcurveto{\pgfqpoint{1.724748in}{3.068143in}}{\pgfqpoint{1.732648in}{3.071415in}}{\pgfqpoint{1.738472in}{3.077239in}}%
\pgfpathcurveto{\pgfqpoint{1.744296in}{3.083063in}}{\pgfqpoint{1.747568in}{3.090963in}}{\pgfqpoint{1.747568in}{3.099199in}}%
\pgfpathcurveto{\pgfqpoint{1.747568in}{3.107436in}}{\pgfqpoint{1.744296in}{3.115336in}}{\pgfqpoint{1.738472in}{3.121160in}}%
\pgfpathcurveto{\pgfqpoint{1.732648in}{3.126984in}}{\pgfqpoint{1.724748in}{3.130256in}}{\pgfqpoint{1.716512in}{3.130256in}}%
\pgfpathcurveto{\pgfqpoint{1.708275in}{3.130256in}}{\pgfqpoint{1.700375in}{3.126984in}}{\pgfqpoint{1.694551in}{3.121160in}}%
\pgfpathcurveto{\pgfqpoint{1.688727in}{3.115336in}}{\pgfqpoint{1.685455in}{3.107436in}}{\pgfqpoint{1.685455in}{3.099199in}}%
\pgfpathcurveto{\pgfqpoint{1.685455in}{3.090963in}}{\pgfqpoint{1.688727in}{3.083063in}}{\pgfqpoint{1.694551in}{3.077239in}}%
\pgfpathcurveto{\pgfqpoint{1.700375in}{3.071415in}}{\pgfqpoint{1.708275in}{3.068143in}}{\pgfqpoint{1.716512in}{3.068143in}}%
\pgfpathclose%
\pgfusepath{stroke,fill}%
\end{pgfscope}%
\begin{pgfscope}%
\pgfpathrectangle{\pgfqpoint{0.100000in}{0.212622in}}{\pgfqpoint{3.696000in}{3.696000in}}%
\pgfusepath{clip}%
\pgfsetbuttcap%
\pgfsetroundjoin%
\definecolor{currentfill}{rgb}{0.121569,0.466667,0.705882}%
\pgfsetfillcolor{currentfill}%
\pgfsetfillopacity{0.600754}%
\pgfsetlinewidth{1.003750pt}%
\definecolor{currentstroke}{rgb}{0.121569,0.466667,0.705882}%
\pgfsetstrokecolor{currentstroke}%
\pgfsetstrokeopacity{0.600754}%
\pgfsetdash{}{0pt}%
\pgfpathmoveto{\pgfqpoint{1.724985in}{3.071586in}}%
\pgfpathcurveto{\pgfqpoint{1.733221in}{3.071586in}}{\pgfqpoint{1.741121in}{3.074859in}}{\pgfqpoint{1.746945in}{3.080683in}}%
\pgfpathcurveto{\pgfqpoint{1.752769in}{3.086507in}}{\pgfqpoint{1.756041in}{3.094407in}}{\pgfqpoint{1.756041in}{3.102643in}}%
\pgfpathcurveto{\pgfqpoint{1.756041in}{3.110879in}}{\pgfqpoint{1.752769in}{3.118779in}}{\pgfqpoint{1.746945in}{3.124603in}}%
\pgfpathcurveto{\pgfqpoint{1.741121in}{3.130427in}}{\pgfqpoint{1.733221in}{3.133699in}}{\pgfqpoint{1.724985in}{3.133699in}}%
\pgfpathcurveto{\pgfqpoint{1.716749in}{3.133699in}}{\pgfqpoint{1.708849in}{3.130427in}}{\pgfqpoint{1.703025in}{3.124603in}}%
\pgfpathcurveto{\pgfqpoint{1.697201in}{3.118779in}}{\pgfqpoint{1.693928in}{3.110879in}}{\pgfqpoint{1.693928in}{3.102643in}}%
\pgfpathcurveto{\pgfqpoint{1.693928in}{3.094407in}}{\pgfqpoint{1.697201in}{3.086507in}}{\pgfqpoint{1.703025in}{3.080683in}}%
\pgfpathcurveto{\pgfqpoint{1.708849in}{3.074859in}}{\pgfqpoint{1.716749in}{3.071586in}}{\pgfqpoint{1.724985in}{3.071586in}}%
\pgfpathclose%
\pgfusepath{stroke,fill}%
\end{pgfscope}%
\begin{pgfscope}%
\pgfpathrectangle{\pgfqpoint{0.100000in}{0.212622in}}{\pgfqpoint{3.696000in}{3.696000in}}%
\pgfusepath{clip}%
\pgfsetbuttcap%
\pgfsetroundjoin%
\definecolor{currentfill}{rgb}{0.121569,0.466667,0.705882}%
\pgfsetfillcolor{currentfill}%
\pgfsetfillopacity{0.600877}%
\pgfsetlinewidth{1.003750pt}%
\definecolor{currentstroke}{rgb}{0.121569,0.466667,0.705882}%
\pgfsetstrokecolor{currentstroke}%
\pgfsetstrokeopacity{0.600877}%
\pgfsetdash{}{0pt}%
\pgfpathmoveto{\pgfqpoint{1.715139in}{3.065253in}}%
\pgfpathcurveto{\pgfqpoint{1.723375in}{3.065253in}}{\pgfqpoint{1.731275in}{3.068526in}}{\pgfqpoint{1.737099in}{3.074350in}}%
\pgfpathcurveto{\pgfqpoint{1.742923in}{3.080174in}}{\pgfqpoint{1.746195in}{3.088074in}}{\pgfqpoint{1.746195in}{3.096310in}}%
\pgfpathcurveto{\pgfqpoint{1.746195in}{3.104546in}}{\pgfqpoint{1.742923in}{3.112446in}}{\pgfqpoint{1.737099in}{3.118270in}}%
\pgfpathcurveto{\pgfqpoint{1.731275in}{3.124094in}}{\pgfqpoint{1.723375in}{3.127366in}}{\pgfqpoint{1.715139in}{3.127366in}}%
\pgfpathcurveto{\pgfqpoint{1.706902in}{3.127366in}}{\pgfqpoint{1.699002in}{3.124094in}}{\pgfqpoint{1.693178in}{3.118270in}}%
\pgfpathcurveto{\pgfqpoint{1.687354in}{3.112446in}}{\pgfqpoint{1.684082in}{3.104546in}}{\pgfqpoint{1.684082in}{3.096310in}}%
\pgfpathcurveto{\pgfqpoint{1.684082in}{3.088074in}}{\pgfqpoint{1.687354in}{3.080174in}}{\pgfqpoint{1.693178in}{3.074350in}}%
\pgfpathcurveto{\pgfqpoint{1.699002in}{3.068526in}}{\pgfqpoint{1.706902in}{3.065253in}}{\pgfqpoint{1.715139in}{3.065253in}}%
\pgfpathclose%
\pgfusepath{stroke,fill}%
\end{pgfscope}%
\begin{pgfscope}%
\pgfpathrectangle{\pgfqpoint{0.100000in}{0.212622in}}{\pgfqpoint{3.696000in}{3.696000in}}%
\pgfusepath{clip}%
\pgfsetbuttcap%
\pgfsetroundjoin%
\definecolor{currentfill}{rgb}{0.121569,0.466667,0.705882}%
\pgfsetfillcolor{currentfill}%
\pgfsetfillopacity{0.601161}%
\pgfsetlinewidth{1.003750pt}%
\definecolor{currentstroke}{rgb}{0.121569,0.466667,0.705882}%
\pgfsetstrokecolor{currentstroke}%
\pgfsetstrokeopacity{0.601161}%
\pgfsetdash{}{0pt}%
\pgfpathmoveto{\pgfqpoint{1.714450in}{3.063743in}}%
\pgfpathcurveto{\pgfqpoint{1.722687in}{3.063743in}}{\pgfqpoint{1.730587in}{3.067015in}}{\pgfqpoint{1.736411in}{3.072839in}}%
\pgfpathcurveto{\pgfqpoint{1.742235in}{3.078663in}}{\pgfqpoint{1.745507in}{3.086563in}}{\pgfqpoint{1.745507in}{3.094800in}}%
\pgfpathcurveto{\pgfqpoint{1.745507in}{3.103036in}}{\pgfqpoint{1.742235in}{3.110936in}}{\pgfqpoint{1.736411in}{3.116760in}}%
\pgfpathcurveto{\pgfqpoint{1.730587in}{3.122584in}}{\pgfqpoint{1.722687in}{3.125856in}}{\pgfqpoint{1.714450in}{3.125856in}}%
\pgfpathcurveto{\pgfqpoint{1.706214in}{3.125856in}}{\pgfqpoint{1.698314in}{3.122584in}}{\pgfqpoint{1.692490in}{3.116760in}}%
\pgfpathcurveto{\pgfqpoint{1.686666in}{3.110936in}}{\pgfqpoint{1.683394in}{3.103036in}}{\pgfqpoint{1.683394in}{3.094800in}}%
\pgfpathcurveto{\pgfqpoint{1.683394in}{3.086563in}}{\pgfqpoint{1.686666in}{3.078663in}}{\pgfqpoint{1.692490in}{3.072839in}}%
\pgfpathcurveto{\pgfqpoint{1.698314in}{3.067015in}}{\pgfqpoint{1.706214in}{3.063743in}}{\pgfqpoint{1.714450in}{3.063743in}}%
\pgfpathclose%
\pgfusepath{stroke,fill}%
\end{pgfscope}%
\begin{pgfscope}%
\pgfpathrectangle{\pgfqpoint{0.100000in}{0.212622in}}{\pgfqpoint{3.696000in}{3.696000in}}%
\pgfusepath{clip}%
\pgfsetbuttcap%
\pgfsetroundjoin%
\definecolor{currentfill}{rgb}{0.121569,0.466667,0.705882}%
\pgfsetfillcolor{currentfill}%
\pgfsetfillopacity{0.601502}%
\pgfsetlinewidth{1.003750pt}%
\definecolor{currentstroke}{rgb}{0.121569,0.466667,0.705882}%
\pgfsetstrokecolor{currentstroke}%
\pgfsetstrokeopacity{0.601502}%
\pgfsetdash{}{0pt}%
\pgfpathmoveto{\pgfqpoint{1.714097in}{3.061316in}}%
\pgfpathcurveto{\pgfqpoint{1.722333in}{3.061316in}}{\pgfqpoint{1.730233in}{3.064588in}}{\pgfqpoint{1.736057in}{3.070412in}}%
\pgfpathcurveto{\pgfqpoint{1.741881in}{3.076236in}}{\pgfqpoint{1.745154in}{3.084136in}}{\pgfqpoint{1.745154in}{3.092372in}}%
\pgfpathcurveto{\pgfqpoint{1.745154in}{3.100609in}}{\pgfqpoint{1.741881in}{3.108509in}}{\pgfqpoint{1.736057in}{3.114333in}}%
\pgfpathcurveto{\pgfqpoint{1.730233in}{3.120156in}}{\pgfqpoint{1.722333in}{3.123429in}}{\pgfqpoint{1.714097in}{3.123429in}}%
\pgfpathcurveto{\pgfqpoint{1.705861in}{3.123429in}}{\pgfqpoint{1.697961in}{3.120156in}}{\pgfqpoint{1.692137in}{3.114333in}}%
\pgfpathcurveto{\pgfqpoint{1.686313in}{3.108509in}}{\pgfqpoint{1.683041in}{3.100609in}}{\pgfqpoint{1.683041in}{3.092372in}}%
\pgfpathcurveto{\pgfqpoint{1.683041in}{3.084136in}}{\pgfqpoint{1.686313in}{3.076236in}}{\pgfqpoint{1.692137in}{3.070412in}}%
\pgfpathcurveto{\pgfqpoint{1.697961in}{3.064588in}}{\pgfqpoint{1.705861in}{3.061316in}}{\pgfqpoint{1.714097in}{3.061316in}}%
\pgfpathclose%
\pgfusepath{stroke,fill}%
\end{pgfscope}%
\begin{pgfscope}%
\pgfpathrectangle{\pgfqpoint{0.100000in}{0.212622in}}{\pgfqpoint{3.696000in}{3.696000in}}%
\pgfusepath{clip}%
\pgfsetbuttcap%
\pgfsetroundjoin%
\definecolor{currentfill}{rgb}{0.121569,0.466667,0.705882}%
\pgfsetfillcolor{currentfill}%
\pgfsetfillopacity{0.601881}%
\pgfsetlinewidth{1.003750pt}%
\definecolor{currentstroke}{rgb}{0.121569,0.466667,0.705882}%
\pgfsetstrokecolor{currentstroke}%
\pgfsetstrokeopacity{0.601881}%
\pgfsetdash{}{0pt}%
\pgfpathmoveto{\pgfqpoint{1.729956in}{3.070399in}}%
\pgfpathcurveto{\pgfqpoint{1.738192in}{3.070399in}}{\pgfqpoint{1.746092in}{3.073671in}}{\pgfqpoint{1.751916in}{3.079495in}}%
\pgfpathcurveto{\pgfqpoint{1.757740in}{3.085319in}}{\pgfqpoint{1.761012in}{3.093219in}}{\pgfqpoint{1.761012in}{3.101455in}}%
\pgfpathcurveto{\pgfqpoint{1.761012in}{3.109692in}}{\pgfqpoint{1.757740in}{3.117592in}}{\pgfqpoint{1.751916in}{3.123416in}}%
\pgfpathcurveto{\pgfqpoint{1.746092in}{3.129240in}}{\pgfqpoint{1.738192in}{3.132512in}}{\pgfqpoint{1.729956in}{3.132512in}}%
\pgfpathcurveto{\pgfqpoint{1.721719in}{3.132512in}}{\pgfqpoint{1.713819in}{3.129240in}}{\pgfqpoint{1.707995in}{3.123416in}}%
\pgfpathcurveto{\pgfqpoint{1.702172in}{3.117592in}}{\pgfqpoint{1.698899in}{3.109692in}}{\pgfqpoint{1.698899in}{3.101455in}}%
\pgfpathcurveto{\pgfqpoint{1.698899in}{3.093219in}}{\pgfqpoint{1.702172in}{3.085319in}}{\pgfqpoint{1.707995in}{3.079495in}}%
\pgfpathcurveto{\pgfqpoint{1.713819in}{3.073671in}}{\pgfqpoint{1.721719in}{3.070399in}}{\pgfqpoint{1.729956in}{3.070399in}}%
\pgfpathclose%
\pgfusepath{stroke,fill}%
\end{pgfscope}%
\begin{pgfscope}%
\pgfpathrectangle{\pgfqpoint{0.100000in}{0.212622in}}{\pgfqpoint{3.696000in}{3.696000in}}%
\pgfusepath{clip}%
\pgfsetbuttcap%
\pgfsetroundjoin%
\definecolor{currentfill}{rgb}{0.121569,0.466667,0.705882}%
\pgfsetfillcolor{currentfill}%
\pgfsetfillopacity{0.602077}%
\pgfsetlinewidth{1.003750pt}%
\definecolor{currentstroke}{rgb}{0.121569,0.466667,0.705882}%
\pgfsetstrokecolor{currentstroke}%
\pgfsetstrokeopacity{0.602077}%
\pgfsetdash{}{0pt}%
\pgfpathmoveto{\pgfqpoint{1.713016in}{3.057951in}}%
\pgfpathcurveto{\pgfqpoint{1.721252in}{3.057951in}}{\pgfqpoint{1.729152in}{3.061224in}}{\pgfqpoint{1.734976in}{3.067048in}}%
\pgfpathcurveto{\pgfqpoint{1.740800in}{3.072872in}}{\pgfqpoint{1.744072in}{3.080772in}}{\pgfqpoint{1.744072in}{3.089008in}}%
\pgfpathcurveto{\pgfqpoint{1.744072in}{3.097244in}}{\pgfqpoint{1.740800in}{3.105144in}}{\pgfqpoint{1.734976in}{3.110968in}}%
\pgfpathcurveto{\pgfqpoint{1.729152in}{3.116792in}}{\pgfqpoint{1.721252in}{3.120064in}}{\pgfqpoint{1.713016in}{3.120064in}}%
\pgfpathcurveto{\pgfqpoint{1.704779in}{3.120064in}}{\pgfqpoint{1.696879in}{3.116792in}}{\pgfqpoint{1.691055in}{3.110968in}}%
\pgfpathcurveto{\pgfqpoint{1.685231in}{3.105144in}}{\pgfqpoint{1.681959in}{3.097244in}}{\pgfqpoint{1.681959in}{3.089008in}}%
\pgfpathcurveto{\pgfqpoint{1.681959in}{3.080772in}}{\pgfqpoint{1.685231in}{3.072872in}}{\pgfqpoint{1.691055in}{3.067048in}}%
\pgfpathcurveto{\pgfqpoint{1.696879in}{3.061224in}}{\pgfqpoint{1.704779in}{3.057951in}}{\pgfqpoint{1.713016in}{3.057951in}}%
\pgfpathclose%
\pgfusepath{stroke,fill}%
\end{pgfscope}%
\begin{pgfscope}%
\pgfpathrectangle{\pgfqpoint{0.100000in}{0.212622in}}{\pgfqpoint{3.696000in}{3.696000in}}%
\pgfusepath{clip}%
\pgfsetbuttcap%
\pgfsetroundjoin%
\definecolor{currentfill}{rgb}{0.121569,0.466667,0.705882}%
\pgfsetfillcolor{currentfill}%
\pgfsetfillopacity{0.602100}%
\pgfsetlinewidth{1.003750pt}%
\definecolor{currentstroke}{rgb}{0.121569,0.466667,0.705882}%
\pgfsetstrokecolor{currentstroke}%
\pgfsetstrokeopacity{0.602100}%
\pgfsetdash{}{0pt}%
\pgfpathmoveto{\pgfqpoint{2.920977in}{2.641005in}}%
\pgfpathcurveto{\pgfqpoint{2.929213in}{2.641005in}}{\pgfqpoint{2.937113in}{2.644277in}}{\pgfqpoint{2.942937in}{2.650101in}}%
\pgfpathcurveto{\pgfqpoint{2.948761in}{2.655925in}}{\pgfqpoint{2.952033in}{2.663825in}}{\pgfqpoint{2.952033in}{2.672061in}}%
\pgfpathcurveto{\pgfqpoint{2.952033in}{2.680298in}}{\pgfqpoint{2.948761in}{2.688198in}}{\pgfqpoint{2.942937in}{2.694022in}}%
\pgfpathcurveto{\pgfqpoint{2.937113in}{2.699846in}}{\pgfqpoint{2.929213in}{2.703118in}}{\pgfqpoint{2.920977in}{2.703118in}}%
\pgfpathcurveto{\pgfqpoint{2.912741in}{2.703118in}}{\pgfqpoint{2.904840in}{2.699846in}}{\pgfqpoint{2.899017in}{2.694022in}}%
\pgfpathcurveto{\pgfqpoint{2.893193in}{2.688198in}}{\pgfqpoint{2.889920in}{2.680298in}}{\pgfqpoint{2.889920in}{2.672061in}}%
\pgfpathcurveto{\pgfqpoint{2.889920in}{2.663825in}}{\pgfqpoint{2.893193in}{2.655925in}}{\pgfqpoint{2.899017in}{2.650101in}}%
\pgfpathcurveto{\pgfqpoint{2.904840in}{2.644277in}}{\pgfqpoint{2.912741in}{2.641005in}}{\pgfqpoint{2.920977in}{2.641005in}}%
\pgfpathclose%
\pgfusepath{stroke,fill}%
\end{pgfscope}%
\begin{pgfscope}%
\pgfpathrectangle{\pgfqpoint{0.100000in}{0.212622in}}{\pgfqpoint{3.696000in}{3.696000in}}%
\pgfusepath{clip}%
\pgfsetbuttcap%
\pgfsetroundjoin%
\definecolor{currentfill}{rgb}{0.121569,0.466667,0.705882}%
\pgfsetfillcolor{currentfill}%
\pgfsetfillopacity{0.602681}%
\pgfsetlinewidth{1.003750pt}%
\definecolor{currentstroke}{rgb}{0.121569,0.466667,0.705882}%
\pgfsetstrokecolor{currentstroke}%
\pgfsetstrokeopacity{0.602681}%
\pgfsetdash{}{0pt}%
\pgfpathmoveto{\pgfqpoint{1.711128in}{3.054301in}}%
\pgfpathcurveto{\pgfqpoint{1.719364in}{3.054301in}}{\pgfqpoint{1.727265in}{3.057574in}}{\pgfqpoint{1.733088in}{3.063398in}}%
\pgfpathcurveto{\pgfqpoint{1.738912in}{3.069222in}}{\pgfqpoint{1.742185in}{3.077122in}}{\pgfqpoint{1.742185in}{3.085358in}}%
\pgfpathcurveto{\pgfqpoint{1.742185in}{3.093594in}}{\pgfqpoint{1.738912in}{3.101494in}}{\pgfqpoint{1.733088in}{3.107318in}}%
\pgfpathcurveto{\pgfqpoint{1.727265in}{3.113142in}}{\pgfqpoint{1.719364in}{3.116414in}}{\pgfqpoint{1.711128in}{3.116414in}}%
\pgfpathcurveto{\pgfqpoint{1.702892in}{3.116414in}}{\pgfqpoint{1.694992in}{3.113142in}}{\pgfqpoint{1.689168in}{3.107318in}}%
\pgfpathcurveto{\pgfqpoint{1.683344in}{3.101494in}}{\pgfqpoint{1.680072in}{3.093594in}}{\pgfqpoint{1.680072in}{3.085358in}}%
\pgfpathcurveto{\pgfqpoint{1.680072in}{3.077122in}}{\pgfqpoint{1.683344in}{3.069222in}}{\pgfqpoint{1.689168in}{3.063398in}}%
\pgfpathcurveto{\pgfqpoint{1.694992in}{3.057574in}}{\pgfqpoint{1.702892in}{3.054301in}}{\pgfqpoint{1.711128in}{3.054301in}}%
\pgfpathclose%
\pgfusepath{stroke,fill}%
\end{pgfscope}%
\begin{pgfscope}%
\pgfpathrectangle{\pgfqpoint{0.100000in}{0.212622in}}{\pgfqpoint{3.696000in}{3.696000in}}%
\pgfusepath{clip}%
\pgfsetbuttcap%
\pgfsetroundjoin%
\definecolor{currentfill}{rgb}{0.121569,0.466667,0.705882}%
\pgfsetfillcolor{currentfill}%
\pgfsetfillopacity{0.603706}%
\pgfsetlinewidth{1.003750pt}%
\definecolor{currentstroke}{rgb}{0.121569,0.466667,0.705882}%
\pgfsetstrokecolor{currentstroke}%
\pgfsetstrokeopacity{0.603706}%
\pgfsetdash{}{0pt}%
\pgfpathmoveto{\pgfqpoint{1.710109in}{3.047377in}}%
\pgfpathcurveto{\pgfqpoint{1.718345in}{3.047377in}}{\pgfqpoint{1.726245in}{3.050650in}}{\pgfqpoint{1.732069in}{3.056474in}}%
\pgfpathcurveto{\pgfqpoint{1.737893in}{3.062298in}}{\pgfqpoint{1.741165in}{3.070198in}}{\pgfqpoint{1.741165in}{3.078434in}}%
\pgfpathcurveto{\pgfqpoint{1.741165in}{3.086670in}}{\pgfqpoint{1.737893in}{3.094570in}}{\pgfqpoint{1.732069in}{3.100394in}}%
\pgfpathcurveto{\pgfqpoint{1.726245in}{3.106218in}}{\pgfqpoint{1.718345in}{3.109490in}}{\pgfqpoint{1.710109in}{3.109490in}}%
\pgfpathcurveto{\pgfqpoint{1.701873in}{3.109490in}}{\pgfqpoint{1.693973in}{3.106218in}}{\pgfqpoint{1.688149in}{3.100394in}}%
\pgfpathcurveto{\pgfqpoint{1.682325in}{3.094570in}}{\pgfqpoint{1.679052in}{3.086670in}}{\pgfqpoint{1.679052in}{3.078434in}}%
\pgfpathcurveto{\pgfqpoint{1.679052in}{3.070198in}}{\pgfqpoint{1.682325in}{3.062298in}}{\pgfqpoint{1.688149in}{3.056474in}}%
\pgfpathcurveto{\pgfqpoint{1.693973in}{3.050650in}}{\pgfqpoint{1.701873in}{3.047377in}}{\pgfqpoint{1.710109in}{3.047377in}}%
\pgfpathclose%
\pgfusepath{stroke,fill}%
\end{pgfscope}%
\begin{pgfscope}%
\pgfpathrectangle{\pgfqpoint{0.100000in}{0.212622in}}{\pgfqpoint{3.696000in}{3.696000in}}%
\pgfusepath{clip}%
\pgfsetbuttcap%
\pgfsetroundjoin%
\definecolor{currentfill}{rgb}{0.121569,0.466667,0.705882}%
\pgfsetfillcolor{currentfill}%
\pgfsetfillopacity{0.603804}%
\pgfsetlinewidth{1.003750pt}%
\definecolor{currentstroke}{rgb}{0.121569,0.466667,0.705882}%
\pgfsetstrokecolor{currentstroke}%
\pgfsetstrokeopacity{0.603804}%
\pgfsetdash{}{0pt}%
\pgfpathmoveto{\pgfqpoint{1.739213in}{3.068340in}}%
\pgfpathcurveto{\pgfqpoint{1.747449in}{3.068340in}}{\pgfqpoint{1.755349in}{3.071613in}}{\pgfqpoint{1.761173in}{3.077437in}}%
\pgfpathcurveto{\pgfqpoint{1.766997in}{3.083261in}}{\pgfqpoint{1.770269in}{3.091161in}}{\pgfqpoint{1.770269in}{3.099397in}}%
\pgfpathcurveto{\pgfqpoint{1.770269in}{3.107633in}}{\pgfqpoint{1.766997in}{3.115533in}}{\pgfqpoint{1.761173in}{3.121357in}}%
\pgfpathcurveto{\pgfqpoint{1.755349in}{3.127181in}}{\pgfqpoint{1.747449in}{3.130453in}}{\pgfqpoint{1.739213in}{3.130453in}}%
\pgfpathcurveto{\pgfqpoint{1.730977in}{3.130453in}}{\pgfqpoint{1.723076in}{3.127181in}}{\pgfqpoint{1.717253in}{3.121357in}}%
\pgfpathcurveto{\pgfqpoint{1.711429in}{3.115533in}}{\pgfqpoint{1.708156in}{3.107633in}}{\pgfqpoint{1.708156in}{3.099397in}}%
\pgfpathcurveto{\pgfqpoint{1.708156in}{3.091161in}}{\pgfqpoint{1.711429in}{3.083261in}}{\pgfqpoint{1.717253in}{3.077437in}}%
\pgfpathcurveto{\pgfqpoint{1.723076in}{3.071613in}}{\pgfqpoint{1.730977in}{3.068340in}}{\pgfqpoint{1.739213in}{3.068340in}}%
\pgfpathclose%
\pgfusepath{stroke,fill}%
\end{pgfscope}%
\begin{pgfscope}%
\pgfpathrectangle{\pgfqpoint{0.100000in}{0.212622in}}{\pgfqpoint{3.696000in}{3.696000in}}%
\pgfusepath{clip}%
\pgfsetbuttcap%
\pgfsetroundjoin%
\definecolor{currentfill}{rgb}{0.121569,0.466667,0.705882}%
\pgfsetfillcolor{currentfill}%
\pgfsetfillopacity{0.604869}%
\pgfsetlinewidth{1.003750pt}%
\definecolor{currentstroke}{rgb}{0.121569,0.466667,0.705882}%
\pgfsetstrokecolor{currentstroke}%
\pgfsetstrokeopacity{0.604869}%
\pgfsetdash{}{0pt}%
\pgfpathmoveto{\pgfqpoint{1.707695in}{3.040494in}}%
\pgfpathcurveto{\pgfqpoint{1.715931in}{3.040494in}}{\pgfqpoint{1.723831in}{3.043766in}}{\pgfqpoint{1.729655in}{3.049590in}}%
\pgfpathcurveto{\pgfqpoint{1.735479in}{3.055414in}}{\pgfqpoint{1.738751in}{3.063314in}}{\pgfqpoint{1.738751in}{3.071551in}}%
\pgfpathcurveto{\pgfqpoint{1.738751in}{3.079787in}}{\pgfqpoint{1.735479in}{3.087687in}}{\pgfqpoint{1.729655in}{3.093511in}}%
\pgfpathcurveto{\pgfqpoint{1.723831in}{3.099335in}}{\pgfqpoint{1.715931in}{3.102607in}}{\pgfqpoint{1.707695in}{3.102607in}}%
\pgfpathcurveto{\pgfqpoint{1.699458in}{3.102607in}}{\pgfqpoint{1.691558in}{3.099335in}}{\pgfqpoint{1.685734in}{3.093511in}}%
\pgfpathcurveto{\pgfqpoint{1.679910in}{3.087687in}}{\pgfqpoint{1.676638in}{3.079787in}}{\pgfqpoint{1.676638in}{3.071551in}}%
\pgfpathcurveto{\pgfqpoint{1.676638in}{3.063314in}}{\pgfqpoint{1.679910in}{3.055414in}}{\pgfqpoint{1.685734in}{3.049590in}}%
\pgfpathcurveto{\pgfqpoint{1.691558in}{3.043766in}}{\pgfqpoint{1.699458in}{3.040494in}}{\pgfqpoint{1.707695in}{3.040494in}}%
\pgfpathclose%
\pgfusepath{stroke,fill}%
\end{pgfscope}%
\begin{pgfscope}%
\pgfpathrectangle{\pgfqpoint{0.100000in}{0.212622in}}{\pgfqpoint{3.696000in}{3.696000in}}%
\pgfusepath{clip}%
\pgfsetbuttcap%
\pgfsetroundjoin%
\definecolor{currentfill}{rgb}{0.121569,0.466667,0.705882}%
\pgfsetfillcolor{currentfill}%
\pgfsetfillopacity{0.605989}%
\pgfsetlinewidth{1.003750pt}%
\definecolor{currentstroke}{rgb}{0.121569,0.466667,0.705882}%
\pgfsetstrokecolor{currentstroke}%
\pgfsetstrokeopacity{0.605989}%
\pgfsetdash{}{0pt}%
\pgfpathmoveto{\pgfqpoint{1.704099in}{3.033481in}}%
\pgfpathcurveto{\pgfqpoint{1.712335in}{3.033481in}}{\pgfqpoint{1.720235in}{3.036753in}}{\pgfqpoint{1.726059in}{3.042577in}}%
\pgfpathcurveto{\pgfqpoint{1.731883in}{3.048401in}}{\pgfqpoint{1.735155in}{3.056301in}}{\pgfqpoint{1.735155in}{3.064537in}}%
\pgfpathcurveto{\pgfqpoint{1.735155in}{3.072773in}}{\pgfqpoint{1.731883in}{3.080673in}}{\pgfqpoint{1.726059in}{3.086497in}}%
\pgfpathcurveto{\pgfqpoint{1.720235in}{3.092321in}}{\pgfqpoint{1.712335in}{3.095594in}}{\pgfqpoint{1.704099in}{3.095594in}}%
\pgfpathcurveto{\pgfqpoint{1.695862in}{3.095594in}}{\pgfqpoint{1.687962in}{3.092321in}}{\pgfqpoint{1.682138in}{3.086497in}}%
\pgfpathcurveto{\pgfqpoint{1.676314in}{3.080673in}}{\pgfqpoint{1.673042in}{3.072773in}}{\pgfqpoint{1.673042in}{3.064537in}}%
\pgfpathcurveto{\pgfqpoint{1.673042in}{3.056301in}}{\pgfqpoint{1.676314in}{3.048401in}}{\pgfqpoint{1.682138in}{3.042577in}}%
\pgfpathcurveto{\pgfqpoint{1.687962in}{3.036753in}}{\pgfqpoint{1.695862in}{3.033481in}}{\pgfqpoint{1.704099in}{3.033481in}}%
\pgfpathclose%
\pgfusepath{stroke,fill}%
\end{pgfscope}%
\begin{pgfscope}%
\pgfpathrectangle{\pgfqpoint{0.100000in}{0.212622in}}{\pgfqpoint{3.696000in}{3.696000in}}%
\pgfusepath{clip}%
\pgfsetbuttcap%
\pgfsetroundjoin%
\definecolor{currentfill}{rgb}{0.121569,0.466667,0.705882}%
\pgfsetfillcolor{currentfill}%
\pgfsetfillopacity{0.606218}%
\pgfsetlinewidth{1.003750pt}%
\definecolor{currentstroke}{rgb}{0.121569,0.466667,0.705882}%
\pgfsetstrokecolor{currentstroke}%
\pgfsetstrokeopacity{0.606218}%
\pgfsetdash{}{0pt}%
\pgfpathmoveto{\pgfqpoint{2.934006in}{2.642276in}}%
\pgfpathcurveto{\pgfqpoint{2.942243in}{2.642276in}}{\pgfqpoint{2.950143in}{2.645548in}}{\pgfqpoint{2.955967in}{2.651372in}}%
\pgfpathcurveto{\pgfqpoint{2.961791in}{2.657196in}}{\pgfqpoint{2.965063in}{2.665096in}}{\pgfqpoint{2.965063in}{2.673333in}}%
\pgfpathcurveto{\pgfqpoint{2.965063in}{2.681569in}}{\pgfqpoint{2.961791in}{2.689469in}}{\pgfqpoint{2.955967in}{2.695293in}}%
\pgfpathcurveto{\pgfqpoint{2.950143in}{2.701117in}}{\pgfqpoint{2.942243in}{2.704389in}}{\pgfqpoint{2.934006in}{2.704389in}}%
\pgfpathcurveto{\pgfqpoint{2.925770in}{2.704389in}}{\pgfqpoint{2.917870in}{2.701117in}}{\pgfqpoint{2.912046in}{2.695293in}}%
\pgfpathcurveto{\pgfqpoint{2.906222in}{2.689469in}}{\pgfqpoint{2.902950in}{2.681569in}}{\pgfqpoint{2.902950in}{2.673333in}}%
\pgfpathcurveto{\pgfqpoint{2.902950in}{2.665096in}}{\pgfqpoint{2.906222in}{2.657196in}}{\pgfqpoint{2.912046in}{2.651372in}}%
\pgfpathcurveto{\pgfqpoint{2.917870in}{2.645548in}}{\pgfqpoint{2.925770in}{2.642276in}}{\pgfqpoint{2.934006in}{2.642276in}}%
\pgfpathclose%
\pgfusepath{stroke,fill}%
\end{pgfscope}%
\begin{pgfscope}%
\pgfpathrectangle{\pgfqpoint{0.100000in}{0.212622in}}{\pgfqpoint{3.696000in}{3.696000in}}%
\pgfusepath{clip}%
\pgfsetbuttcap%
\pgfsetroundjoin%
\definecolor{currentfill}{rgb}{0.121569,0.466667,0.705882}%
\pgfsetfillcolor{currentfill}%
\pgfsetfillopacity{0.607629}%
\pgfsetlinewidth{1.003750pt}%
\definecolor{currentstroke}{rgb}{0.121569,0.466667,0.705882}%
\pgfsetstrokecolor{currentstroke}%
\pgfsetstrokeopacity{0.607629}%
\pgfsetdash{}{0pt}%
\pgfpathmoveto{\pgfqpoint{1.701292in}{3.024319in}}%
\pgfpathcurveto{\pgfqpoint{1.709529in}{3.024319in}}{\pgfqpoint{1.717429in}{3.027591in}}{\pgfqpoint{1.723253in}{3.033415in}}%
\pgfpathcurveto{\pgfqpoint{1.729077in}{3.039239in}}{\pgfqpoint{1.732349in}{3.047139in}}{\pgfqpoint{1.732349in}{3.055375in}}%
\pgfpathcurveto{\pgfqpoint{1.732349in}{3.063611in}}{\pgfqpoint{1.729077in}{3.071511in}}{\pgfqpoint{1.723253in}{3.077335in}}%
\pgfpathcurveto{\pgfqpoint{1.717429in}{3.083159in}}{\pgfqpoint{1.709529in}{3.086432in}}{\pgfqpoint{1.701292in}{3.086432in}}%
\pgfpathcurveto{\pgfqpoint{1.693056in}{3.086432in}}{\pgfqpoint{1.685156in}{3.083159in}}{\pgfqpoint{1.679332in}{3.077335in}}%
\pgfpathcurveto{\pgfqpoint{1.673508in}{3.071511in}}{\pgfqpoint{1.670236in}{3.063611in}}{\pgfqpoint{1.670236in}{3.055375in}}%
\pgfpathcurveto{\pgfqpoint{1.670236in}{3.047139in}}{\pgfqpoint{1.673508in}{3.039239in}}{\pgfqpoint{1.679332in}{3.033415in}}%
\pgfpathcurveto{\pgfqpoint{1.685156in}{3.027591in}}{\pgfqpoint{1.693056in}{3.024319in}}{\pgfqpoint{1.701292in}{3.024319in}}%
\pgfpathclose%
\pgfusepath{stroke,fill}%
\end{pgfscope}%
\begin{pgfscope}%
\pgfpathrectangle{\pgfqpoint{0.100000in}{0.212622in}}{\pgfqpoint{3.696000in}{3.696000in}}%
\pgfusepath{clip}%
\pgfsetbuttcap%
\pgfsetroundjoin%
\definecolor{currentfill}{rgb}{0.121569,0.466667,0.705882}%
\pgfsetfillcolor{currentfill}%
\pgfsetfillopacity{0.607765}%
\pgfsetlinewidth{1.003750pt}%
\definecolor{currentstroke}{rgb}{0.121569,0.466667,0.705882}%
\pgfsetstrokecolor{currentstroke}%
\pgfsetstrokeopacity{0.607765}%
\pgfsetdash{}{0pt}%
\pgfpathmoveto{\pgfqpoint{1.755477in}{3.064916in}}%
\pgfpathcurveto{\pgfqpoint{1.763713in}{3.064916in}}{\pgfqpoint{1.771613in}{3.068189in}}{\pgfqpoint{1.777437in}{3.074013in}}%
\pgfpathcurveto{\pgfqpoint{1.783261in}{3.079837in}}{\pgfqpoint{1.786533in}{3.087737in}}{\pgfqpoint{1.786533in}{3.095973in}}%
\pgfpathcurveto{\pgfqpoint{1.786533in}{3.104209in}}{\pgfqpoint{1.783261in}{3.112109in}}{\pgfqpoint{1.777437in}{3.117933in}}%
\pgfpathcurveto{\pgfqpoint{1.771613in}{3.123757in}}{\pgfqpoint{1.763713in}{3.127029in}}{\pgfqpoint{1.755477in}{3.127029in}}%
\pgfpathcurveto{\pgfqpoint{1.747241in}{3.127029in}}{\pgfqpoint{1.739341in}{3.123757in}}{\pgfqpoint{1.733517in}{3.117933in}}%
\pgfpathcurveto{\pgfqpoint{1.727693in}{3.112109in}}{\pgfqpoint{1.724420in}{3.104209in}}{\pgfqpoint{1.724420in}{3.095973in}}%
\pgfpathcurveto{\pgfqpoint{1.724420in}{3.087737in}}{\pgfqpoint{1.727693in}{3.079837in}}{\pgfqpoint{1.733517in}{3.074013in}}%
\pgfpathcurveto{\pgfqpoint{1.739341in}{3.068189in}}{\pgfqpoint{1.747241in}{3.064916in}}{\pgfqpoint{1.755477in}{3.064916in}}%
\pgfpathclose%
\pgfusepath{stroke,fill}%
\end{pgfscope}%
\begin{pgfscope}%
\pgfpathrectangle{\pgfqpoint{0.100000in}{0.212622in}}{\pgfqpoint{3.696000in}{3.696000in}}%
\pgfusepath{clip}%
\pgfsetbuttcap%
\pgfsetroundjoin%
\definecolor{currentfill}{rgb}{0.121569,0.466667,0.705882}%
\pgfsetfillcolor{currentfill}%
\pgfsetfillopacity{0.609250}%
\pgfsetlinewidth{1.003750pt}%
\definecolor{currentstroke}{rgb}{0.121569,0.466667,0.705882}%
\pgfsetstrokecolor{currentstroke}%
\pgfsetstrokeopacity{0.609250}%
\pgfsetdash{}{0pt}%
\pgfpathmoveto{\pgfqpoint{2.946073in}{2.639140in}}%
\pgfpathcurveto{\pgfqpoint{2.954309in}{2.639140in}}{\pgfqpoint{2.962209in}{2.642412in}}{\pgfqpoint{2.968033in}{2.648236in}}%
\pgfpathcurveto{\pgfqpoint{2.973857in}{2.654060in}}{\pgfqpoint{2.977130in}{2.661960in}}{\pgfqpoint{2.977130in}{2.670196in}}%
\pgfpathcurveto{\pgfqpoint{2.977130in}{2.678432in}}{\pgfqpoint{2.973857in}{2.686332in}}{\pgfqpoint{2.968033in}{2.692156in}}%
\pgfpathcurveto{\pgfqpoint{2.962209in}{2.697980in}}{\pgfqpoint{2.954309in}{2.701253in}}{\pgfqpoint{2.946073in}{2.701253in}}%
\pgfpathcurveto{\pgfqpoint{2.937837in}{2.701253in}}{\pgfqpoint{2.929937in}{2.697980in}}{\pgfqpoint{2.924113in}{2.692156in}}%
\pgfpathcurveto{\pgfqpoint{2.918289in}{2.686332in}}{\pgfqpoint{2.915017in}{2.678432in}}{\pgfqpoint{2.915017in}{2.670196in}}%
\pgfpathcurveto{\pgfqpoint{2.915017in}{2.661960in}}{\pgfqpoint{2.918289in}{2.654060in}}{\pgfqpoint{2.924113in}{2.648236in}}%
\pgfpathcurveto{\pgfqpoint{2.929937in}{2.642412in}}{\pgfqpoint{2.937837in}{2.639140in}}{\pgfqpoint{2.946073in}{2.639140in}}%
\pgfpathclose%
\pgfusepath{stroke,fill}%
\end{pgfscope}%
\begin{pgfscope}%
\pgfpathrectangle{\pgfqpoint{0.100000in}{0.212622in}}{\pgfqpoint{3.696000in}{3.696000in}}%
\pgfusepath{clip}%
\pgfsetbuttcap%
\pgfsetroundjoin%
\definecolor{currentfill}{rgb}{0.121569,0.466667,0.705882}%
\pgfsetfillcolor{currentfill}%
\pgfsetfillopacity{0.609333}%
\pgfsetlinewidth{1.003750pt}%
\definecolor{currentstroke}{rgb}{0.121569,0.466667,0.705882}%
\pgfsetstrokecolor{currentstroke}%
\pgfsetstrokeopacity{0.609333}%
\pgfsetdash{}{0pt}%
\pgfpathmoveto{\pgfqpoint{1.698702in}{3.014629in}}%
\pgfpathcurveto{\pgfqpoint{1.706938in}{3.014629in}}{\pgfqpoint{1.714838in}{3.017902in}}{\pgfqpoint{1.720662in}{3.023726in}}%
\pgfpathcurveto{\pgfqpoint{1.726486in}{3.029549in}}{\pgfqpoint{1.729758in}{3.037450in}}{\pgfqpoint{1.729758in}{3.045686in}}%
\pgfpathcurveto{\pgfqpoint{1.729758in}{3.053922in}}{\pgfqpoint{1.726486in}{3.061822in}}{\pgfqpoint{1.720662in}{3.067646in}}%
\pgfpathcurveto{\pgfqpoint{1.714838in}{3.073470in}}{\pgfqpoint{1.706938in}{3.076742in}}{\pgfqpoint{1.698702in}{3.076742in}}%
\pgfpathcurveto{\pgfqpoint{1.690466in}{3.076742in}}{\pgfqpoint{1.682566in}{3.073470in}}{\pgfqpoint{1.676742in}{3.067646in}}%
\pgfpathcurveto{\pgfqpoint{1.670918in}{3.061822in}}{\pgfqpoint{1.667645in}{3.053922in}}{\pgfqpoint{1.667645in}{3.045686in}}%
\pgfpathcurveto{\pgfqpoint{1.667645in}{3.037450in}}{\pgfqpoint{1.670918in}{3.029549in}}{\pgfqpoint{1.676742in}{3.023726in}}%
\pgfpathcurveto{\pgfqpoint{1.682566in}{3.017902in}}{\pgfqpoint{1.690466in}{3.014629in}}{\pgfqpoint{1.698702in}{3.014629in}}%
\pgfpathclose%
\pgfusepath{stroke,fill}%
\end{pgfscope}%
\begin{pgfscope}%
\pgfpathrectangle{\pgfqpoint{0.100000in}{0.212622in}}{\pgfqpoint{3.696000in}{3.696000in}}%
\pgfusepath{clip}%
\pgfsetbuttcap%
\pgfsetroundjoin%
\definecolor{currentfill}{rgb}{0.121569,0.466667,0.705882}%
\pgfsetfillcolor{currentfill}%
\pgfsetfillopacity{0.610976}%
\pgfsetlinewidth{1.003750pt}%
\definecolor{currentstroke}{rgb}{0.121569,0.466667,0.705882}%
\pgfsetstrokecolor{currentstroke}%
\pgfsetstrokeopacity{0.610976}%
\pgfsetdash{}{0pt}%
\pgfpathmoveto{\pgfqpoint{1.693179in}{3.004100in}}%
\pgfpathcurveto{\pgfqpoint{1.701415in}{3.004100in}}{\pgfqpoint{1.709315in}{3.007373in}}{\pgfqpoint{1.715139in}{3.013197in}}%
\pgfpathcurveto{\pgfqpoint{1.720963in}{3.019021in}}{\pgfqpoint{1.724235in}{3.026921in}}{\pgfqpoint{1.724235in}{3.035157in}}%
\pgfpathcurveto{\pgfqpoint{1.724235in}{3.043393in}}{\pgfqpoint{1.720963in}{3.051293in}}{\pgfqpoint{1.715139in}{3.057117in}}%
\pgfpathcurveto{\pgfqpoint{1.709315in}{3.062941in}}{\pgfqpoint{1.701415in}{3.066213in}}{\pgfqpoint{1.693179in}{3.066213in}}%
\pgfpathcurveto{\pgfqpoint{1.684942in}{3.066213in}}{\pgfqpoint{1.677042in}{3.062941in}}{\pgfqpoint{1.671218in}{3.057117in}}%
\pgfpathcurveto{\pgfqpoint{1.665394in}{3.051293in}}{\pgfqpoint{1.662122in}{3.043393in}}{\pgfqpoint{1.662122in}{3.035157in}}%
\pgfpathcurveto{\pgfqpoint{1.662122in}{3.026921in}}{\pgfqpoint{1.665394in}{3.019021in}}{\pgfqpoint{1.671218in}{3.013197in}}%
\pgfpathcurveto{\pgfqpoint{1.677042in}{3.007373in}}{\pgfqpoint{1.684942in}{3.004100in}}{\pgfqpoint{1.693179in}{3.004100in}}%
\pgfpathclose%
\pgfusepath{stroke,fill}%
\end{pgfscope}%
\begin{pgfscope}%
\pgfpathrectangle{\pgfqpoint{0.100000in}{0.212622in}}{\pgfqpoint{3.696000in}{3.696000in}}%
\pgfusepath{clip}%
\pgfsetbuttcap%
\pgfsetroundjoin%
\definecolor{currentfill}{rgb}{0.121569,0.466667,0.705882}%
\pgfsetfillcolor{currentfill}%
\pgfsetfillopacity{0.611289}%
\pgfsetlinewidth{1.003750pt}%
\definecolor{currentstroke}{rgb}{0.121569,0.466667,0.705882}%
\pgfsetstrokecolor{currentstroke}%
\pgfsetstrokeopacity{0.611289}%
\pgfsetdash{}{0pt}%
\pgfpathmoveto{\pgfqpoint{1.772057in}{3.061389in}}%
\pgfpathcurveto{\pgfqpoint{1.780293in}{3.061389in}}{\pgfqpoint{1.788193in}{3.064661in}}{\pgfqpoint{1.794017in}{3.070485in}}%
\pgfpathcurveto{\pgfqpoint{1.799841in}{3.076309in}}{\pgfqpoint{1.803113in}{3.084209in}}{\pgfqpoint{1.803113in}{3.092445in}}%
\pgfpathcurveto{\pgfqpoint{1.803113in}{3.100682in}}{\pgfqpoint{1.799841in}{3.108582in}}{\pgfqpoint{1.794017in}{3.114406in}}%
\pgfpathcurveto{\pgfqpoint{1.788193in}{3.120230in}}{\pgfqpoint{1.780293in}{3.123502in}}{\pgfqpoint{1.772057in}{3.123502in}}%
\pgfpathcurveto{\pgfqpoint{1.763821in}{3.123502in}}{\pgfqpoint{1.755921in}{3.120230in}}{\pgfqpoint{1.750097in}{3.114406in}}%
\pgfpathcurveto{\pgfqpoint{1.744273in}{3.108582in}}{\pgfqpoint{1.741000in}{3.100682in}}{\pgfqpoint{1.741000in}{3.092445in}}%
\pgfpathcurveto{\pgfqpoint{1.741000in}{3.084209in}}{\pgfqpoint{1.744273in}{3.076309in}}{\pgfqpoint{1.750097in}{3.070485in}}%
\pgfpathcurveto{\pgfqpoint{1.755921in}{3.064661in}}{\pgfqpoint{1.763821in}{3.061389in}}{\pgfqpoint{1.772057in}{3.061389in}}%
\pgfpathclose%
\pgfusepath{stroke,fill}%
\end{pgfscope}%
\begin{pgfscope}%
\pgfpathrectangle{\pgfqpoint{0.100000in}{0.212622in}}{\pgfqpoint{3.696000in}{3.696000in}}%
\pgfusepath{clip}%
\pgfsetbuttcap%
\pgfsetroundjoin%
\definecolor{currentfill}{rgb}{0.121569,0.466667,0.705882}%
\pgfsetfillcolor{currentfill}%
\pgfsetfillopacity{0.612093}%
\pgfsetlinewidth{1.003750pt}%
\definecolor{currentstroke}{rgb}{0.121569,0.466667,0.705882}%
\pgfsetstrokecolor{currentstroke}%
\pgfsetstrokeopacity{0.612093}%
\pgfsetdash{}{0pt}%
\pgfpathmoveto{\pgfqpoint{2.956769in}{2.636528in}}%
\pgfpathcurveto{\pgfqpoint{2.965005in}{2.636528in}}{\pgfqpoint{2.972905in}{2.639800in}}{\pgfqpoint{2.978729in}{2.645624in}}%
\pgfpathcurveto{\pgfqpoint{2.984553in}{2.651448in}}{\pgfqpoint{2.987825in}{2.659348in}}{\pgfqpoint{2.987825in}{2.667584in}}%
\pgfpathcurveto{\pgfqpoint{2.987825in}{2.675820in}}{\pgfqpoint{2.984553in}{2.683720in}}{\pgfqpoint{2.978729in}{2.689544in}}%
\pgfpathcurveto{\pgfqpoint{2.972905in}{2.695368in}}{\pgfqpoint{2.965005in}{2.698641in}}{\pgfqpoint{2.956769in}{2.698641in}}%
\pgfpathcurveto{\pgfqpoint{2.948533in}{2.698641in}}{\pgfqpoint{2.940633in}{2.695368in}}{\pgfqpoint{2.934809in}{2.689544in}}%
\pgfpathcurveto{\pgfqpoint{2.928985in}{2.683720in}}{\pgfqpoint{2.925712in}{2.675820in}}{\pgfqpoint{2.925712in}{2.667584in}}%
\pgfpathcurveto{\pgfqpoint{2.925712in}{2.659348in}}{\pgfqpoint{2.928985in}{2.651448in}}{\pgfqpoint{2.934809in}{2.645624in}}%
\pgfpathcurveto{\pgfqpoint{2.940633in}{2.639800in}}{\pgfqpoint{2.948533in}{2.636528in}}{\pgfqpoint{2.956769in}{2.636528in}}%
\pgfpathclose%
\pgfusepath{stroke,fill}%
\end{pgfscope}%
\begin{pgfscope}%
\pgfpathrectangle{\pgfqpoint{0.100000in}{0.212622in}}{\pgfqpoint{3.696000in}{3.696000in}}%
\pgfusepath{clip}%
\pgfsetbuttcap%
\pgfsetroundjoin%
\definecolor{currentfill}{rgb}{0.121569,0.466667,0.705882}%
\pgfsetfillcolor{currentfill}%
\pgfsetfillopacity{0.613008}%
\pgfsetlinewidth{1.003750pt}%
\definecolor{currentstroke}{rgb}{0.121569,0.466667,0.705882}%
\pgfsetstrokecolor{currentstroke}%
\pgfsetstrokeopacity{0.613008}%
\pgfsetdash{}{0pt}%
\pgfpathmoveto{\pgfqpoint{1.689725in}{2.991133in}}%
\pgfpathcurveto{\pgfqpoint{1.697961in}{2.991133in}}{\pgfqpoint{1.705861in}{2.994406in}}{\pgfqpoint{1.711685in}{3.000229in}}%
\pgfpathcurveto{\pgfqpoint{1.717509in}{3.006053in}}{\pgfqpoint{1.720781in}{3.013953in}}{\pgfqpoint{1.720781in}{3.022190in}}%
\pgfpathcurveto{\pgfqpoint{1.720781in}{3.030426in}}{\pgfqpoint{1.717509in}{3.038326in}}{\pgfqpoint{1.711685in}{3.044150in}}%
\pgfpathcurveto{\pgfqpoint{1.705861in}{3.049974in}}{\pgfqpoint{1.697961in}{3.053246in}}{\pgfqpoint{1.689725in}{3.053246in}}%
\pgfpathcurveto{\pgfqpoint{1.681488in}{3.053246in}}{\pgfqpoint{1.673588in}{3.049974in}}{\pgfqpoint{1.667764in}{3.044150in}}%
\pgfpathcurveto{\pgfqpoint{1.661940in}{3.038326in}}{\pgfqpoint{1.658668in}{3.030426in}}{\pgfqpoint{1.658668in}{3.022190in}}%
\pgfpathcurveto{\pgfqpoint{1.658668in}{3.013953in}}{\pgfqpoint{1.661940in}{3.006053in}}{\pgfqpoint{1.667764in}{3.000229in}}%
\pgfpathcurveto{\pgfqpoint{1.673588in}{2.994406in}}{\pgfqpoint{1.681488in}{2.991133in}}{\pgfqpoint{1.689725in}{2.991133in}}%
\pgfpathclose%
\pgfusepath{stroke,fill}%
\end{pgfscope}%
\begin{pgfscope}%
\pgfpathrectangle{\pgfqpoint{0.100000in}{0.212622in}}{\pgfqpoint{3.696000in}{3.696000in}}%
\pgfusepath{clip}%
\pgfsetbuttcap%
\pgfsetroundjoin%
\definecolor{currentfill}{rgb}{0.121569,0.466667,0.705882}%
\pgfsetfillcolor{currentfill}%
\pgfsetfillopacity{0.614044}%
\pgfsetlinewidth{1.003750pt}%
\definecolor{currentstroke}{rgb}{0.121569,0.466667,0.705882}%
\pgfsetstrokecolor{currentstroke}%
\pgfsetstrokeopacity{0.614044}%
\pgfsetdash{}{0pt}%
\pgfpathmoveto{\pgfqpoint{2.967168in}{2.634716in}}%
\pgfpathcurveto{\pgfqpoint{2.975404in}{2.634716in}}{\pgfqpoint{2.983304in}{2.637988in}}{\pgfqpoint{2.989128in}{2.643812in}}%
\pgfpathcurveto{\pgfqpoint{2.994952in}{2.649636in}}{\pgfqpoint{2.998225in}{2.657536in}}{\pgfqpoint{2.998225in}{2.665773in}}%
\pgfpathcurveto{\pgfqpoint{2.998225in}{2.674009in}}{\pgfqpoint{2.994952in}{2.681909in}}{\pgfqpoint{2.989128in}{2.687733in}}%
\pgfpathcurveto{\pgfqpoint{2.983304in}{2.693557in}}{\pgfqpoint{2.975404in}{2.696829in}}{\pgfqpoint{2.967168in}{2.696829in}}%
\pgfpathcurveto{\pgfqpoint{2.958932in}{2.696829in}}{\pgfqpoint{2.951032in}{2.693557in}}{\pgfqpoint{2.945208in}{2.687733in}}%
\pgfpathcurveto{\pgfqpoint{2.939384in}{2.681909in}}{\pgfqpoint{2.936112in}{2.674009in}}{\pgfqpoint{2.936112in}{2.665773in}}%
\pgfpathcurveto{\pgfqpoint{2.936112in}{2.657536in}}{\pgfqpoint{2.939384in}{2.649636in}}{\pgfqpoint{2.945208in}{2.643812in}}%
\pgfpathcurveto{\pgfqpoint{2.951032in}{2.637988in}}{\pgfqpoint{2.958932in}{2.634716in}}{\pgfqpoint{2.967168in}{2.634716in}}%
\pgfpathclose%
\pgfusepath{stroke,fill}%
\end{pgfscope}%
\begin{pgfscope}%
\pgfpathrectangle{\pgfqpoint{0.100000in}{0.212622in}}{\pgfqpoint{3.696000in}{3.696000in}}%
\pgfusepath{clip}%
\pgfsetbuttcap%
\pgfsetroundjoin%
\definecolor{currentfill}{rgb}{0.121569,0.466667,0.705882}%
\pgfsetfillcolor{currentfill}%
\pgfsetfillopacity{0.614512}%
\pgfsetlinewidth{1.003750pt}%
\definecolor{currentstroke}{rgb}{0.121569,0.466667,0.705882}%
\pgfsetstrokecolor{currentstroke}%
\pgfsetstrokeopacity{0.614512}%
\pgfsetdash{}{0pt}%
\pgfpathmoveto{\pgfqpoint{2.978539in}{2.633505in}}%
\pgfpathcurveto{\pgfqpoint{2.986775in}{2.633505in}}{\pgfqpoint{2.994675in}{2.636778in}}{\pgfqpoint{3.000499in}{2.642602in}}%
\pgfpathcurveto{\pgfqpoint{3.006323in}{2.648425in}}{\pgfqpoint{3.009595in}{2.656326in}}{\pgfqpoint{3.009595in}{2.664562in}}%
\pgfpathcurveto{\pgfqpoint{3.009595in}{2.672798in}}{\pgfqpoint{3.006323in}{2.680698in}}{\pgfqpoint{3.000499in}{2.686522in}}%
\pgfpathcurveto{\pgfqpoint{2.994675in}{2.692346in}}{\pgfqpoint{2.986775in}{2.695618in}}{\pgfqpoint{2.978539in}{2.695618in}}%
\pgfpathcurveto{\pgfqpoint{2.970302in}{2.695618in}}{\pgfqpoint{2.962402in}{2.692346in}}{\pgfqpoint{2.956578in}{2.686522in}}%
\pgfpathcurveto{\pgfqpoint{2.950755in}{2.680698in}}{\pgfqpoint{2.947482in}{2.672798in}}{\pgfqpoint{2.947482in}{2.664562in}}%
\pgfpathcurveto{\pgfqpoint{2.947482in}{2.656326in}}{\pgfqpoint{2.950755in}{2.648425in}}{\pgfqpoint{2.956578in}{2.642602in}}%
\pgfpathcurveto{\pgfqpoint{2.962402in}{2.636778in}}{\pgfqpoint{2.970302in}{2.633505in}}{\pgfqpoint{2.978539in}{2.633505in}}%
\pgfpathclose%
\pgfusepath{stroke,fill}%
\end{pgfscope}%
\begin{pgfscope}%
\pgfpathrectangle{\pgfqpoint{0.100000in}{0.212622in}}{\pgfqpoint{3.696000in}{3.696000in}}%
\pgfusepath{clip}%
\pgfsetbuttcap%
\pgfsetroundjoin%
\definecolor{currentfill}{rgb}{0.121569,0.466667,0.705882}%
\pgfsetfillcolor{currentfill}%
\pgfsetfillopacity{0.615064}%
\pgfsetlinewidth{1.003750pt}%
\definecolor{currentstroke}{rgb}{0.121569,0.466667,0.705882}%
\pgfsetstrokecolor{currentstroke}%
\pgfsetstrokeopacity{0.615064}%
\pgfsetdash{}{0pt}%
\pgfpathmoveto{\pgfqpoint{1.687793in}{2.976811in}}%
\pgfpathcurveto{\pgfqpoint{1.696029in}{2.976811in}}{\pgfqpoint{1.703929in}{2.980083in}}{\pgfqpoint{1.709753in}{2.985907in}}%
\pgfpathcurveto{\pgfqpoint{1.715577in}{2.991731in}}{\pgfqpoint{1.718849in}{2.999631in}}{\pgfqpoint{1.718849in}{3.007867in}}%
\pgfpathcurveto{\pgfqpoint{1.718849in}{3.016104in}}{\pgfqpoint{1.715577in}{3.024004in}}{\pgfqpoint{1.709753in}{3.029828in}}%
\pgfpathcurveto{\pgfqpoint{1.703929in}{3.035651in}}{\pgfqpoint{1.696029in}{3.038924in}}{\pgfqpoint{1.687793in}{3.038924in}}%
\pgfpathcurveto{\pgfqpoint{1.679556in}{3.038924in}}{\pgfqpoint{1.671656in}{3.035651in}}{\pgfqpoint{1.665832in}{3.029828in}}%
\pgfpathcurveto{\pgfqpoint{1.660008in}{3.024004in}}{\pgfqpoint{1.656736in}{3.016104in}}{\pgfqpoint{1.656736in}{3.007867in}}%
\pgfpathcurveto{\pgfqpoint{1.656736in}{2.999631in}}{\pgfqpoint{1.660008in}{2.991731in}}{\pgfqpoint{1.665832in}{2.985907in}}%
\pgfpathcurveto{\pgfqpoint{1.671656in}{2.980083in}}{\pgfqpoint{1.679556in}{2.976811in}}{\pgfqpoint{1.687793in}{2.976811in}}%
\pgfpathclose%
\pgfusepath{stroke,fill}%
\end{pgfscope}%
\begin{pgfscope}%
\pgfpathrectangle{\pgfqpoint{0.100000in}{0.212622in}}{\pgfqpoint{3.696000in}{3.696000in}}%
\pgfusepath{clip}%
\pgfsetbuttcap%
\pgfsetroundjoin%
\definecolor{currentfill}{rgb}{0.121569,0.466667,0.705882}%
\pgfsetfillcolor{currentfill}%
\pgfsetfillopacity{0.615398}%
\pgfsetlinewidth{1.003750pt}%
\definecolor{currentstroke}{rgb}{0.121569,0.466667,0.705882}%
\pgfsetstrokecolor{currentstroke}%
\pgfsetstrokeopacity{0.615398}%
\pgfsetdash{}{0pt}%
\pgfpathmoveto{\pgfqpoint{1.787455in}{3.057864in}}%
\pgfpathcurveto{\pgfqpoint{1.795692in}{3.057864in}}{\pgfqpoint{1.803592in}{3.061136in}}{\pgfqpoint{1.809416in}{3.066960in}}%
\pgfpathcurveto{\pgfqpoint{1.815240in}{3.072784in}}{\pgfqpoint{1.818512in}{3.080684in}}{\pgfqpoint{1.818512in}{3.088920in}}%
\pgfpathcurveto{\pgfqpoint{1.818512in}{3.097156in}}{\pgfqpoint{1.815240in}{3.105056in}}{\pgfqpoint{1.809416in}{3.110880in}}%
\pgfpathcurveto{\pgfqpoint{1.803592in}{3.116704in}}{\pgfqpoint{1.795692in}{3.119977in}}{\pgfqpoint{1.787455in}{3.119977in}}%
\pgfpathcurveto{\pgfqpoint{1.779219in}{3.119977in}}{\pgfqpoint{1.771319in}{3.116704in}}{\pgfqpoint{1.765495in}{3.110880in}}%
\pgfpathcurveto{\pgfqpoint{1.759671in}{3.105056in}}{\pgfqpoint{1.756399in}{3.097156in}}{\pgfqpoint{1.756399in}{3.088920in}}%
\pgfpathcurveto{\pgfqpoint{1.756399in}{3.080684in}}{\pgfqpoint{1.759671in}{3.072784in}}{\pgfqpoint{1.765495in}{3.066960in}}%
\pgfpathcurveto{\pgfqpoint{1.771319in}{3.061136in}}{\pgfqpoint{1.779219in}{3.057864in}}{\pgfqpoint{1.787455in}{3.057864in}}%
\pgfpathclose%
\pgfusepath{stroke,fill}%
\end{pgfscope}%
\begin{pgfscope}%
\pgfpathrectangle{\pgfqpoint{0.100000in}{0.212622in}}{\pgfqpoint{3.696000in}{3.696000in}}%
\pgfusepath{clip}%
\pgfsetbuttcap%
\pgfsetroundjoin%
\definecolor{currentfill}{rgb}{0.121569,0.466667,0.705882}%
\pgfsetfillcolor{currentfill}%
\pgfsetfillopacity{0.616908}%
\pgfsetlinewidth{1.003750pt}%
\definecolor{currentstroke}{rgb}{0.121569,0.466667,0.705882}%
\pgfsetstrokecolor{currentstroke}%
\pgfsetstrokeopacity{0.616908}%
\pgfsetdash{}{0pt}%
\pgfpathmoveto{\pgfqpoint{2.986253in}{2.629667in}}%
\pgfpathcurveto{\pgfqpoint{2.994489in}{2.629667in}}{\pgfqpoint{3.002389in}{2.632939in}}{\pgfqpoint{3.008213in}{2.638763in}}%
\pgfpathcurveto{\pgfqpoint{3.014037in}{2.644587in}}{\pgfqpoint{3.017309in}{2.652487in}}{\pgfqpoint{3.017309in}{2.660724in}}%
\pgfpathcurveto{\pgfqpoint{3.017309in}{2.668960in}}{\pgfqpoint{3.014037in}{2.676860in}}{\pgfqpoint{3.008213in}{2.682684in}}%
\pgfpathcurveto{\pgfqpoint{3.002389in}{2.688508in}}{\pgfqpoint{2.994489in}{2.691780in}}{\pgfqpoint{2.986253in}{2.691780in}}%
\pgfpathcurveto{\pgfqpoint{2.978016in}{2.691780in}}{\pgfqpoint{2.970116in}{2.688508in}}{\pgfqpoint{2.964293in}{2.682684in}}%
\pgfpathcurveto{\pgfqpoint{2.958469in}{2.676860in}}{\pgfqpoint{2.955196in}{2.668960in}}{\pgfqpoint{2.955196in}{2.660724in}}%
\pgfpathcurveto{\pgfqpoint{2.955196in}{2.652487in}}{\pgfqpoint{2.958469in}{2.644587in}}{\pgfqpoint{2.964293in}{2.638763in}}%
\pgfpathcurveto{\pgfqpoint{2.970116in}{2.632939in}}{\pgfqpoint{2.978016in}{2.629667in}}{\pgfqpoint{2.986253in}{2.629667in}}%
\pgfpathclose%
\pgfusepath{stroke,fill}%
\end{pgfscope}%
\begin{pgfscope}%
\pgfpathrectangle{\pgfqpoint{0.100000in}{0.212622in}}{\pgfqpoint{3.696000in}{3.696000in}}%
\pgfusepath{clip}%
\pgfsetbuttcap%
\pgfsetroundjoin%
\definecolor{currentfill}{rgb}{0.121569,0.466667,0.705882}%
\pgfsetfillcolor{currentfill}%
\pgfsetfillopacity{0.617687}%
\pgfsetlinewidth{1.003750pt}%
\definecolor{currentstroke}{rgb}{0.121569,0.466667,0.705882}%
\pgfsetstrokecolor{currentstroke}%
\pgfsetstrokeopacity{0.617687}%
\pgfsetdash{}{0pt}%
\pgfpathmoveto{\pgfqpoint{1.682005in}{2.963914in}}%
\pgfpathcurveto{\pgfqpoint{1.690241in}{2.963914in}}{\pgfqpoint{1.698141in}{2.967186in}}{\pgfqpoint{1.703965in}{2.973010in}}%
\pgfpathcurveto{\pgfqpoint{1.709789in}{2.978834in}}{\pgfqpoint{1.713061in}{2.986734in}}{\pgfqpoint{1.713061in}{2.994970in}}%
\pgfpathcurveto{\pgfqpoint{1.713061in}{3.003206in}}{\pgfqpoint{1.709789in}{3.011106in}}{\pgfqpoint{1.703965in}{3.016930in}}%
\pgfpathcurveto{\pgfqpoint{1.698141in}{3.022754in}}{\pgfqpoint{1.690241in}{3.026027in}}{\pgfqpoint{1.682005in}{3.026027in}}%
\pgfpathcurveto{\pgfqpoint{1.673769in}{3.026027in}}{\pgfqpoint{1.665869in}{3.022754in}}{\pgfqpoint{1.660045in}{3.016930in}}%
\pgfpathcurveto{\pgfqpoint{1.654221in}{3.011106in}}{\pgfqpoint{1.650948in}{3.003206in}}{\pgfqpoint{1.650948in}{2.994970in}}%
\pgfpathcurveto{\pgfqpoint{1.650948in}{2.986734in}}{\pgfqpoint{1.654221in}{2.978834in}}{\pgfqpoint{1.660045in}{2.973010in}}%
\pgfpathcurveto{\pgfqpoint{1.665869in}{2.967186in}}{\pgfqpoint{1.673769in}{2.963914in}}{\pgfqpoint{1.682005in}{2.963914in}}%
\pgfpathclose%
\pgfusepath{stroke,fill}%
\end{pgfscope}%
\begin{pgfscope}%
\pgfpathrectangle{\pgfqpoint{0.100000in}{0.212622in}}{\pgfqpoint{3.696000in}{3.696000in}}%
\pgfusepath{clip}%
\pgfsetbuttcap%
\pgfsetroundjoin%
\definecolor{currentfill}{rgb}{0.121569,0.466667,0.705882}%
\pgfsetfillcolor{currentfill}%
\pgfsetfillopacity{0.619001}%
\pgfsetlinewidth{1.003750pt}%
\definecolor{currentstroke}{rgb}{0.121569,0.466667,0.705882}%
\pgfsetstrokecolor{currentstroke}%
\pgfsetstrokeopacity{0.619001}%
\pgfsetdash{}{0pt}%
\pgfpathmoveto{\pgfqpoint{1.677998in}{2.956998in}}%
\pgfpathcurveto{\pgfqpoint{1.686234in}{2.956998in}}{\pgfqpoint{1.694134in}{2.960271in}}{\pgfqpoint{1.699958in}{2.966095in}}%
\pgfpathcurveto{\pgfqpoint{1.705782in}{2.971919in}}{\pgfqpoint{1.709054in}{2.979819in}}{\pgfqpoint{1.709054in}{2.988055in}}%
\pgfpathcurveto{\pgfqpoint{1.709054in}{2.996291in}}{\pgfqpoint{1.705782in}{3.004191in}}{\pgfqpoint{1.699958in}{3.010015in}}%
\pgfpathcurveto{\pgfqpoint{1.694134in}{3.015839in}}{\pgfqpoint{1.686234in}{3.019111in}}{\pgfqpoint{1.677998in}{3.019111in}}%
\pgfpathcurveto{\pgfqpoint{1.669762in}{3.019111in}}{\pgfqpoint{1.661861in}{3.015839in}}{\pgfqpoint{1.656038in}{3.010015in}}%
\pgfpathcurveto{\pgfqpoint{1.650214in}{3.004191in}}{\pgfqpoint{1.646941in}{2.996291in}}{\pgfqpoint{1.646941in}{2.988055in}}%
\pgfpathcurveto{\pgfqpoint{1.646941in}{2.979819in}}{\pgfqpoint{1.650214in}{2.971919in}}{\pgfqpoint{1.656038in}{2.966095in}}%
\pgfpathcurveto{\pgfqpoint{1.661861in}{2.960271in}}{\pgfqpoint{1.669762in}{2.956998in}}{\pgfqpoint{1.677998in}{2.956998in}}%
\pgfpathclose%
\pgfusepath{stroke,fill}%
\end{pgfscope}%
\begin{pgfscope}%
\pgfpathrectangle{\pgfqpoint{0.100000in}{0.212622in}}{\pgfqpoint{3.696000in}{3.696000in}}%
\pgfusepath{clip}%
\pgfsetbuttcap%
\pgfsetroundjoin%
\definecolor{currentfill}{rgb}{0.121569,0.466667,0.705882}%
\pgfsetfillcolor{currentfill}%
\pgfsetfillopacity{0.619090}%
\pgfsetlinewidth{1.003750pt}%
\definecolor{currentstroke}{rgb}{0.121569,0.466667,0.705882}%
\pgfsetstrokecolor{currentstroke}%
\pgfsetstrokeopacity{0.619090}%
\pgfsetdash{}{0pt}%
\pgfpathmoveto{\pgfqpoint{1.802774in}{3.053394in}}%
\pgfpathcurveto{\pgfqpoint{1.811011in}{3.053394in}}{\pgfqpoint{1.818911in}{3.056667in}}{\pgfqpoint{1.824735in}{3.062491in}}%
\pgfpathcurveto{\pgfqpoint{1.830559in}{3.068315in}}{\pgfqpoint{1.833831in}{3.076215in}}{\pgfqpoint{1.833831in}{3.084451in}}%
\pgfpathcurveto{\pgfqpoint{1.833831in}{3.092687in}}{\pgfqpoint{1.830559in}{3.100587in}}{\pgfqpoint{1.824735in}{3.106411in}}%
\pgfpathcurveto{\pgfqpoint{1.818911in}{3.112235in}}{\pgfqpoint{1.811011in}{3.115507in}}{\pgfqpoint{1.802774in}{3.115507in}}%
\pgfpathcurveto{\pgfqpoint{1.794538in}{3.115507in}}{\pgfqpoint{1.786638in}{3.112235in}}{\pgfqpoint{1.780814in}{3.106411in}}%
\pgfpathcurveto{\pgfqpoint{1.774990in}{3.100587in}}{\pgfqpoint{1.771718in}{3.092687in}}{\pgfqpoint{1.771718in}{3.084451in}}%
\pgfpathcurveto{\pgfqpoint{1.771718in}{3.076215in}}{\pgfqpoint{1.774990in}{3.068315in}}{\pgfqpoint{1.780814in}{3.062491in}}%
\pgfpathcurveto{\pgfqpoint{1.786638in}{3.056667in}}{\pgfqpoint{1.794538in}{3.053394in}}{\pgfqpoint{1.802774in}{3.053394in}}%
\pgfpathclose%
\pgfusepath{stroke,fill}%
\end{pgfscope}%
\begin{pgfscope}%
\pgfpathrectangle{\pgfqpoint{0.100000in}{0.212622in}}{\pgfqpoint{3.696000in}{3.696000in}}%
\pgfusepath{clip}%
\pgfsetbuttcap%
\pgfsetroundjoin%
\definecolor{currentfill}{rgb}{0.121569,0.466667,0.705882}%
\pgfsetfillcolor{currentfill}%
\pgfsetfillopacity{0.619688}%
\pgfsetlinewidth{1.003750pt}%
\definecolor{currentstroke}{rgb}{0.121569,0.466667,0.705882}%
\pgfsetstrokecolor{currentstroke}%
\pgfsetstrokeopacity{0.619688}%
\pgfsetdash{}{0pt}%
\pgfpathmoveto{\pgfqpoint{1.676489in}{2.952454in}}%
\pgfpathcurveto{\pgfqpoint{1.684725in}{2.952454in}}{\pgfqpoint{1.692626in}{2.955727in}}{\pgfqpoint{1.698449in}{2.961551in}}%
\pgfpathcurveto{\pgfqpoint{1.704273in}{2.967374in}}{\pgfqpoint{1.707546in}{2.975275in}}{\pgfqpoint{1.707546in}{2.983511in}}%
\pgfpathcurveto{\pgfqpoint{1.707546in}{2.991747in}}{\pgfqpoint{1.704273in}{2.999647in}}{\pgfqpoint{1.698449in}{3.005471in}}%
\pgfpathcurveto{\pgfqpoint{1.692626in}{3.011295in}}{\pgfqpoint{1.684725in}{3.014567in}}{\pgfqpoint{1.676489in}{3.014567in}}%
\pgfpathcurveto{\pgfqpoint{1.668253in}{3.014567in}}{\pgfqpoint{1.660353in}{3.011295in}}{\pgfqpoint{1.654529in}{3.005471in}}%
\pgfpathcurveto{\pgfqpoint{1.648705in}{2.999647in}}{\pgfqpoint{1.645433in}{2.991747in}}{\pgfqpoint{1.645433in}{2.983511in}}%
\pgfpathcurveto{\pgfqpoint{1.645433in}{2.975275in}}{\pgfqpoint{1.648705in}{2.967374in}}{\pgfqpoint{1.654529in}{2.961551in}}%
\pgfpathcurveto{\pgfqpoint{1.660353in}{2.955727in}}{\pgfqpoint{1.668253in}{2.952454in}}{\pgfqpoint{1.676489in}{2.952454in}}%
\pgfpathclose%
\pgfusepath{stroke,fill}%
\end{pgfscope}%
\begin{pgfscope}%
\pgfpathrectangle{\pgfqpoint{0.100000in}{0.212622in}}{\pgfqpoint{3.696000in}{3.696000in}}%
\pgfusepath{clip}%
\pgfsetbuttcap%
\pgfsetroundjoin%
\definecolor{currentfill}{rgb}{0.121569,0.466667,0.705882}%
\pgfsetfillcolor{currentfill}%
\pgfsetfillopacity{0.620073}%
\pgfsetlinewidth{1.003750pt}%
\definecolor{currentstroke}{rgb}{0.121569,0.466667,0.705882}%
\pgfsetstrokecolor{currentstroke}%
\pgfsetstrokeopacity{0.620073}%
\pgfsetdash{}{0pt}%
\pgfpathmoveto{\pgfqpoint{1.675402in}{2.950163in}}%
\pgfpathcurveto{\pgfqpoint{1.683639in}{2.950163in}}{\pgfqpoint{1.691539in}{2.953435in}}{\pgfqpoint{1.697363in}{2.959259in}}%
\pgfpathcurveto{\pgfqpoint{1.703186in}{2.965083in}}{\pgfqpoint{1.706459in}{2.972983in}}{\pgfqpoint{1.706459in}{2.981219in}}%
\pgfpathcurveto{\pgfqpoint{1.706459in}{2.989455in}}{\pgfqpoint{1.703186in}{2.997355in}}{\pgfqpoint{1.697363in}{3.003179in}}%
\pgfpathcurveto{\pgfqpoint{1.691539in}{3.009003in}}{\pgfqpoint{1.683639in}{3.012276in}}{\pgfqpoint{1.675402in}{3.012276in}}%
\pgfpathcurveto{\pgfqpoint{1.667166in}{3.012276in}}{\pgfqpoint{1.659266in}{3.009003in}}{\pgfqpoint{1.653442in}{3.003179in}}%
\pgfpathcurveto{\pgfqpoint{1.647618in}{2.997355in}}{\pgfqpoint{1.644346in}{2.989455in}}{\pgfqpoint{1.644346in}{2.981219in}}%
\pgfpathcurveto{\pgfqpoint{1.644346in}{2.972983in}}{\pgfqpoint{1.647618in}{2.965083in}}{\pgfqpoint{1.653442in}{2.959259in}}%
\pgfpathcurveto{\pgfqpoint{1.659266in}{2.953435in}}{\pgfqpoint{1.667166in}{2.950163in}}{\pgfqpoint{1.675402in}{2.950163in}}%
\pgfpathclose%
\pgfusepath{stroke,fill}%
\end{pgfscope}%
\begin{pgfscope}%
\pgfpathrectangle{\pgfqpoint{0.100000in}{0.212622in}}{\pgfqpoint{3.696000in}{3.696000in}}%
\pgfusepath{clip}%
\pgfsetbuttcap%
\pgfsetroundjoin%
\definecolor{currentfill}{rgb}{0.121569,0.466667,0.705882}%
\pgfsetfillcolor{currentfill}%
\pgfsetfillopacity{0.620397}%
\pgfsetlinewidth{1.003750pt}%
\definecolor{currentstroke}{rgb}{0.121569,0.466667,0.705882}%
\pgfsetstrokecolor{currentstroke}%
\pgfsetstrokeopacity{0.620397}%
\pgfsetdash{}{0pt}%
\pgfpathmoveto{\pgfqpoint{1.673831in}{2.947291in}}%
\pgfpathcurveto{\pgfqpoint{1.682067in}{2.947291in}}{\pgfqpoint{1.689967in}{2.950563in}}{\pgfqpoint{1.695791in}{2.956387in}}%
\pgfpathcurveto{\pgfqpoint{1.701615in}{2.962211in}}{\pgfqpoint{1.704888in}{2.970111in}}{\pgfqpoint{1.704888in}{2.978347in}}%
\pgfpathcurveto{\pgfqpoint{1.704888in}{2.986584in}}{\pgfqpoint{1.701615in}{2.994484in}}{\pgfqpoint{1.695791in}{3.000308in}}%
\pgfpathcurveto{\pgfqpoint{1.689967in}{3.006132in}}{\pgfqpoint{1.682067in}{3.009404in}}{\pgfqpoint{1.673831in}{3.009404in}}%
\pgfpathcurveto{\pgfqpoint{1.665595in}{3.009404in}}{\pgfqpoint{1.657695in}{3.006132in}}{\pgfqpoint{1.651871in}{3.000308in}}%
\pgfpathcurveto{\pgfqpoint{1.646047in}{2.994484in}}{\pgfqpoint{1.642775in}{2.986584in}}{\pgfqpoint{1.642775in}{2.978347in}}%
\pgfpathcurveto{\pgfqpoint{1.642775in}{2.970111in}}{\pgfqpoint{1.646047in}{2.962211in}}{\pgfqpoint{1.651871in}{2.956387in}}%
\pgfpathcurveto{\pgfqpoint{1.657695in}{2.950563in}}{\pgfqpoint{1.665595in}{2.947291in}}{\pgfqpoint{1.673831in}{2.947291in}}%
\pgfpathclose%
\pgfusepath{stroke,fill}%
\end{pgfscope}%
\begin{pgfscope}%
\pgfpathrectangle{\pgfqpoint{0.100000in}{0.212622in}}{\pgfqpoint{3.696000in}{3.696000in}}%
\pgfusepath{clip}%
\pgfsetbuttcap%
\pgfsetroundjoin%
\definecolor{currentfill}{rgb}{0.121569,0.466667,0.705882}%
\pgfsetfillcolor{currentfill}%
\pgfsetfillopacity{0.620867}%
\pgfsetlinewidth{1.003750pt}%
\definecolor{currentstroke}{rgb}{0.121569,0.466667,0.705882}%
\pgfsetstrokecolor{currentstroke}%
\pgfsetstrokeopacity{0.620867}%
\pgfsetdash{}{0pt}%
\pgfpathmoveto{\pgfqpoint{1.672173in}{2.944099in}}%
\pgfpathcurveto{\pgfqpoint{1.680409in}{2.944099in}}{\pgfqpoint{1.688309in}{2.947372in}}{\pgfqpoint{1.694133in}{2.953196in}}%
\pgfpathcurveto{\pgfqpoint{1.699957in}{2.959020in}}{\pgfqpoint{1.703229in}{2.966920in}}{\pgfqpoint{1.703229in}{2.975156in}}%
\pgfpathcurveto{\pgfqpoint{1.703229in}{2.983392in}}{\pgfqpoint{1.699957in}{2.991292in}}{\pgfqpoint{1.694133in}{2.997116in}}%
\pgfpathcurveto{\pgfqpoint{1.688309in}{3.002940in}}{\pgfqpoint{1.680409in}{3.006212in}}{\pgfqpoint{1.672173in}{3.006212in}}%
\pgfpathcurveto{\pgfqpoint{1.663936in}{3.006212in}}{\pgfqpoint{1.656036in}{3.002940in}}{\pgfqpoint{1.650212in}{2.997116in}}%
\pgfpathcurveto{\pgfqpoint{1.644388in}{2.991292in}}{\pgfqpoint{1.641116in}{2.983392in}}{\pgfqpoint{1.641116in}{2.975156in}}%
\pgfpathcurveto{\pgfqpoint{1.641116in}{2.966920in}}{\pgfqpoint{1.644388in}{2.959020in}}{\pgfqpoint{1.650212in}{2.953196in}}%
\pgfpathcurveto{\pgfqpoint{1.656036in}{2.947372in}}{\pgfqpoint{1.663936in}{2.944099in}}{\pgfqpoint{1.672173in}{2.944099in}}%
\pgfpathclose%
\pgfusepath{stroke,fill}%
\end{pgfscope}%
\begin{pgfscope}%
\pgfpathrectangle{\pgfqpoint{0.100000in}{0.212622in}}{\pgfqpoint{3.696000in}{3.696000in}}%
\pgfusepath{clip}%
\pgfsetbuttcap%
\pgfsetroundjoin%
\definecolor{currentfill}{rgb}{0.121569,0.466667,0.705882}%
\pgfsetfillcolor{currentfill}%
\pgfsetfillopacity{0.621107}%
\pgfsetlinewidth{1.003750pt}%
\definecolor{currentstroke}{rgb}{0.121569,0.466667,0.705882}%
\pgfsetstrokecolor{currentstroke}%
\pgfsetstrokeopacity{0.621107}%
\pgfsetdash{}{0pt}%
\pgfpathmoveto{\pgfqpoint{1.671711in}{2.941968in}}%
\pgfpathcurveto{\pgfqpoint{1.679947in}{2.941968in}}{\pgfqpoint{1.687847in}{2.945241in}}{\pgfqpoint{1.693671in}{2.951064in}}%
\pgfpathcurveto{\pgfqpoint{1.699495in}{2.956888in}}{\pgfqpoint{1.702767in}{2.964788in}}{\pgfqpoint{1.702767in}{2.973025in}}%
\pgfpathcurveto{\pgfqpoint{1.702767in}{2.981261in}}{\pgfqpoint{1.699495in}{2.989161in}}{\pgfqpoint{1.693671in}{2.994985in}}%
\pgfpathcurveto{\pgfqpoint{1.687847in}{3.000809in}}{\pgfqpoint{1.679947in}{3.004081in}}{\pgfqpoint{1.671711in}{3.004081in}}%
\pgfpathcurveto{\pgfqpoint{1.663475in}{3.004081in}}{\pgfqpoint{1.655574in}{3.000809in}}{\pgfqpoint{1.649751in}{2.994985in}}%
\pgfpathcurveto{\pgfqpoint{1.643927in}{2.989161in}}{\pgfqpoint{1.640654in}{2.981261in}}{\pgfqpoint{1.640654in}{2.973025in}}%
\pgfpathcurveto{\pgfqpoint{1.640654in}{2.964788in}}{\pgfqpoint{1.643927in}{2.956888in}}{\pgfqpoint{1.649751in}{2.951064in}}%
\pgfpathcurveto{\pgfqpoint{1.655574in}{2.945241in}}{\pgfqpoint{1.663475in}{2.941968in}}{\pgfqpoint{1.671711in}{2.941968in}}%
\pgfpathclose%
\pgfusepath{stroke,fill}%
\end{pgfscope}%
\begin{pgfscope}%
\pgfpathrectangle{\pgfqpoint{0.100000in}{0.212622in}}{\pgfqpoint{3.696000in}{3.696000in}}%
\pgfusepath{clip}%
\pgfsetbuttcap%
\pgfsetroundjoin%
\definecolor{currentfill}{rgb}{0.121569,0.466667,0.705882}%
\pgfsetfillcolor{currentfill}%
\pgfsetfillopacity{0.621404}%
\pgfsetlinewidth{1.003750pt}%
\definecolor{currentstroke}{rgb}{0.121569,0.466667,0.705882}%
\pgfsetstrokecolor{currentstroke}%
\pgfsetstrokeopacity{0.621404}%
\pgfsetdash{}{0pt}%
\pgfpathmoveto{\pgfqpoint{3.002184in}{2.628563in}}%
\pgfpathcurveto{\pgfqpoint{3.010420in}{2.628563in}}{\pgfqpoint{3.018320in}{2.631836in}}{\pgfqpoint{3.024144in}{2.637660in}}%
\pgfpathcurveto{\pgfqpoint{3.029968in}{2.643483in}}{\pgfqpoint{3.033240in}{2.651384in}}{\pgfqpoint{3.033240in}{2.659620in}}%
\pgfpathcurveto{\pgfqpoint{3.033240in}{2.667856in}}{\pgfqpoint{3.029968in}{2.675756in}}{\pgfqpoint{3.024144in}{2.681580in}}%
\pgfpathcurveto{\pgfqpoint{3.018320in}{2.687404in}}{\pgfqpoint{3.010420in}{2.690676in}}{\pgfqpoint{3.002184in}{2.690676in}}%
\pgfpathcurveto{\pgfqpoint{2.993947in}{2.690676in}}{\pgfqpoint{2.986047in}{2.687404in}}{\pgfqpoint{2.980223in}{2.681580in}}%
\pgfpathcurveto{\pgfqpoint{2.974400in}{2.675756in}}{\pgfqpoint{2.971127in}{2.667856in}}{\pgfqpoint{2.971127in}{2.659620in}}%
\pgfpathcurveto{\pgfqpoint{2.971127in}{2.651384in}}{\pgfqpoint{2.974400in}{2.643483in}}{\pgfqpoint{2.980223in}{2.637660in}}%
\pgfpathcurveto{\pgfqpoint{2.986047in}{2.631836in}}{\pgfqpoint{2.993947in}{2.628563in}}{\pgfqpoint{3.002184in}{2.628563in}}%
\pgfpathclose%
\pgfusepath{stroke,fill}%
\end{pgfscope}%
\begin{pgfscope}%
\pgfpathrectangle{\pgfqpoint{0.100000in}{0.212622in}}{\pgfqpoint{3.696000in}{3.696000in}}%
\pgfusepath{clip}%
\pgfsetbuttcap%
\pgfsetroundjoin%
\definecolor{currentfill}{rgb}{0.121569,0.466667,0.705882}%
\pgfsetfillcolor{currentfill}%
\pgfsetfillopacity{0.621545}%
\pgfsetlinewidth{1.003750pt}%
\definecolor{currentstroke}{rgb}{0.121569,0.466667,0.705882}%
\pgfsetstrokecolor{currentstroke}%
\pgfsetstrokeopacity{0.621545}%
\pgfsetdash{}{0pt}%
\pgfpathmoveto{\pgfqpoint{1.669674in}{2.938201in}}%
\pgfpathcurveto{\pgfqpoint{1.677910in}{2.938201in}}{\pgfqpoint{1.685810in}{2.941474in}}{\pgfqpoint{1.691634in}{2.947297in}}%
\pgfpathcurveto{\pgfqpoint{1.697458in}{2.953121in}}{\pgfqpoint{1.700730in}{2.961021in}}{\pgfqpoint{1.700730in}{2.969258in}}%
\pgfpathcurveto{\pgfqpoint{1.700730in}{2.977494in}}{\pgfqpoint{1.697458in}{2.985394in}}{\pgfqpoint{1.691634in}{2.991218in}}%
\pgfpathcurveto{\pgfqpoint{1.685810in}{2.997042in}}{\pgfqpoint{1.677910in}{3.000314in}}{\pgfqpoint{1.669674in}{3.000314in}}%
\pgfpathcurveto{\pgfqpoint{1.661438in}{3.000314in}}{\pgfqpoint{1.653538in}{2.997042in}}{\pgfqpoint{1.647714in}{2.991218in}}%
\pgfpathcurveto{\pgfqpoint{1.641890in}{2.985394in}}{\pgfqpoint{1.638617in}{2.977494in}}{\pgfqpoint{1.638617in}{2.969258in}}%
\pgfpathcurveto{\pgfqpoint{1.638617in}{2.961021in}}{\pgfqpoint{1.641890in}{2.953121in}}{\pgfqpoint{1.647714in}{2.947297in}}%
\pgfpathcurveto{\pgfqpoint{1.653538in}{2.941474in}}{\pgfqpoint{1.661438in}{2.938201in}}{\pgfqpoint{1.669674in}{2.938201in}}%
\pgfpathclose%
\pgfusepath{stroke,fill}%
\end{pgfscope}%
\begin{pgfscope}%
\pgfpathrectangle{\pgfqpoint{0.100000in}{0.212622in}}{\pgfqpoint{3.696000in}{3.696000in}}%
\pgfusepath{clip}%
\pgfsetbuttcap%
\pgfsetroundjoin%
\definecolor{currentfill}{rgb}{0.121569,0.466667,0.705882}%
\pgfsetfillcolor{currentfill}%
\pgfsetfillopacity{0.621782}%
\pgfsetlinewidth{1.003750pt}%
\definecolor{currentstroke}{rgb}{0.121569,0.466667,0.705882}%
\pgfsetstrokecolor{currentstroke}%
\pgfsetstrokeopacity{0.621782}%
\pgfsetdash{}{0pt}%
\pgfpathmoveto{\pgfqpoint{1.668456in}{2.936296in}}%
\pgfpathcurveto{\pgfqpoint{1.676692in}{2.936296in}}{\pgfqpoint{1.684592in}{2.939568in}}{\pgfqpoint{1.690416in}{2.945392in}}%
\pgfpathcurveto{\pgfqpoint{1.696240in}{2.951216in}}{\pgfqpoint{1.699512in}{2.959116in}}{\pgfqpoint{1.699512in}{2.967352in}}%
\pgfpathcurveto{\pgfqpoint{1.699512in}{2.975589in}}{\pgfqpoint{1.696240in}{2.983489in}}{\pgfqpoint{1.690416in}{2.989313in}}%
\pgfpathcurveto{\pgfqpoint{1.684592in}{2.995137in}}{\pgfqpoint{1.676692in}{2.998409in}}{\pgfqpoint{1.668456in}{2.998409in}}%
\pgfpathcurveto{\pgfqpoint{1.660220in}{2.998409in}}{\pgfqpoint{1.652320in}{2.995137in}}{\pgfqpoint{1.646496in}{2.989313in}}%
\pgfpathcurveto{\pgfqpoint{1.640672in}{2.983489in}}{\pgfqpoint{1.637399in}{2.975589in}}{\pgfqpoint{1.637399in}{2.967352in}}%
\pgfpathcurveto{\pgfqpoint{1.637399in}{2.959116in}}{\pgfqpoint{1.640672in}{2.951216in}}{\pgfqpoint{1.646496in}{2.945392in}}%
\pgfpathcurveto{\pgfqpoint{1.652320in}{2.939568in}}{\pgfqpoint{1.660220in}{2.936296in}}{\pgfqpoint{1.668456in}{2.936296in}}%
\pgfpathclose%
\pgfusepath{stroke,fill}%
\end{pgfscope}%
\begin{pgfscope}%
\pgfpathrectangle{\pgfqpoint{0.100000in}{0.212622in}}{\pgfqpoint{3.696000in}{3.696000in}}%
\pgfusepath{clip}%
\pgfsetbuttcap%
\pgfsetroundjoin%
\definecolor{currentfill}{rgb}{0.121569,0.466667,0.705882}%
\pgfsetfillcolor{currentfill}%
\pgfsetfillopacity{0.622330}%
\pgfsetlinewidth{1.003750pt}%
\definecolor{currentstroke}{rgb}{0.121569,0.466667,0.705882}%
\pgfsetstrokecolor{currentstroke}%
\pgfsetstrokeopacity{0.622330}%
\pgfsetdash{}{0pt}%
\pgfpathmoveto{\pgfqpoint{1.666852in}{2.932049in}}%
\pgfpathcurveto{\pgfqpoint{1.675088in}{2.932049in}}{\pgfqpoint{1.682988in}{2.935321in}}{\pgfqpoint{1.688812in}{2.941145in}}%
\pgfpathcurveto{\pgfqpoint{1.694636in}{2.946969in}}{\pgfqpoint{1.697909in}{2.954869in}}{\pgfqpoint{1.697909in}{2.963106in}}%
\pgfpathcurveto{\pgfqpoint{1.697909in}{2.971342in}}{\pgfqpoint{1.694636in}{2.979242in}}{\pgfqpoint{1.688812in}{2.985066in}}%
\pgfpathcurveto{\pgfqpoint{1.682988in}{2.990890in}}{\pgfqpoint{1.675088in}{2.994162in}}{\pgfqpoint{1.666852in}{2.994162in}}%
\pgfpathcurveto{\pgfqpoint{1.658616in}{2.994162in}}{\pgfqpoint{1.650716in}{2.990890in}}{\pgfqpoint{1.644892in}{2.985066in}}%
\pgfpathcurveto{\pgfqpoint{1.639068in}{2.979242in}}{\pgfqpoint{1.635796in}{2.971342in}}{\pgfqpoint{1.635796in}{2.963106in}}%
\pgfpathcurveto{\pgfqpoint{1.635796in}{2.954869in}}{\pgfqpoint{1.639068in}{2.946969in}}{\pgfqpoint{1.644892in}{2.941145in}}%
\pgfpathcurveto{\pgfqpoint{1.650716in}{2.935321in}}{\pgfqpoint{1.658616in}{2.932049in}}{\pgfqpoint{1.666852in}{2.932049in}}%
\pgfpathclose%
\pgfusepath{stroke,fill}%
\end{pgfscope}%
\begin{pgfscope}%
\pgfpathrectangle{\pgfqpoint{0.100000in}{0.212622in}}{\pgfqpoint{3.696000in}{3.696000in}}%
\pgfusepath{clip}%
\pgfsetbuttcap%
\pgfsetroundjoin%
\definecolor{currentfill}{rgb}{0.121569,0.466667,0.705882}%
\pgfsetfillcolor{currentfill}%
\pgfsetfillopacity{0.622828}%
\pgfsetlinewidth{1.003750pt}%
\definecolor{currentstroke}{rgb}{0.121569,0.466667,0.705882}%
\pgfsetstrokecolor{currentstroke}%
\pgfsetstrokeopacity{0.622828}%
\pgfsetdash{}{0pt}%
\pgfpathmoveto{\pgfqpoint{1.817321in}{3.051423in}}%
\pgfpathcurveto{\pgfqpoint{1.825557in}{3.051423in}}{\pgfqpoint{1.833457in}{3.054695in}}{\pgfqpoint{1.839281in}{3.060519in}}%
\pgfpathcurveto{\pgfqpoint{1.845105in}{3.066343in}}{\pgfqpoint{1.848378in}{3.074243in}}{\pgfqpoint{1.848378in}{3.082479in}}%
\pgfpathcurveto{\pgfqpoint{1.848378in}{3.090715in}}{\pgfqpoint{1.845105in}{3.098615in}}{\pgfqpoint{1.839281in}{3.104439in}}%
\pgfpathcurveto{\pgfqpoint{1.833457in}{3.110263in}}{\pgfqpoint{1.825557in}{3.113536in}}{\pgfqpoint{1.817321in}{3.113536in}}%
\pgfpathcurveto{\pgfqpoint{1.809085in}{3.113536in}}{\pgfqpoint{1.801185in}{3.110263in}}{\pgfqpoint{1.795361in}{3.104439in}}%
\pgfpathcurveto{\pgfqpoint{1.789537in}{3.098615in}}{\pgfqpoint{1.786265in}{3.090715in}}{\pgfqpoint{1.786265in}{3.082479in}}%
\pgfpathcurveto{\pgfqpoint{1.786265in}{3.074243in}}{\pgfqpoint{1.789537in}{3.066343in}}{\pgfqpoint{1.795361in}{3.060519in}}%
\pgfpathcurveto{\pgfqpoint{1.801185in}{3.054695in}}{\pgfqpoint{1.809085in}{3.051423in}}{\pgfqpoint{1.817321in}{3.051423in}}%
\pgfpathclose%
\pgfusepath{stroke,fill}%
\end{pgfscope}%
\begin{pgfscope}%
\pgfpathrectangle{\pgfqpoint{0.100000in}{0.212622in}}{\pgfqpoint{3.696000in}{3.696000in}}%
\pgfusepath{clip}%
\pgfsetbuttcap%
\pgfsetroundjoin%
\definecolor{currentfill}{rgb}{0.121569,0.466667,0.705882}%
\pgfsetfillcolor{currentfill}%
\pgfsetfillopacity{0.622912}%
\pgfsetlinewidth{1.003750pt}%
\definecolor{currentstroke}{rgb}{0.121569,0.466667,0.705882}%
\pgfsetstrokecolor{currentstroke}%
\pgfsetstrokeopacity{0.622912}%
\pgfsetdash{}{0pt}%
\pgfpathmoveto{\pgfqpoint{1.665256in}{2.927193in}}%
\pgfpathcurveto{\pgfqpoint{1.673492in}{2.927193in}}{\pgfqpoint{1.681392in}{2.930465in}}{\pgfqpoint{1.687216in}{2.936289in}}%
\pgfpathcurveto{\pgfqpoint{1.693040in}{2.942113in}}{\pgfqpoint{1.696312in}{2.950013in}}{\pgfqpoint{1.696312in}{2.958249in}}%
\pgfpathcurveto{\pgfqpoint{1.696312in}{2.966485in}}{\pgfqpoint{1.693040in}{2.974385in}}{\pgfqpoint{1.687216in}{2.980209in}}%
\pgfpathcurveto{\pgfqpoint{1.681392in}{2.986033in}}{\pgfqpoint{1.673492in}{2.989306in}}{\pgfqpoint{1.665256in}{2.989306in}}%
\pgfpathcurveto{\pgfqpoint{1.657019in}{2.989306in}}{\pgfqpoint{1.649119in}{2.986033in}}{\pgfqpoint{1.643295in}{2.980209in}}%
\pgfpathcurveto{\pgfqpoint{1.637471in}{2.974385in}}{\pgfqpoint{1.634199in}{2.966485in}}{\pgfqpoint{1.634199in}{2.958249in}}%
\pgfpathcurveto{\pgfqpoint{1.634199in}{2.950013in}}{\pgfqpoint{1.637471in}{2.942113in}}{\pgfqpoint{1.643295in}{2.936289in}}%
\pgfpathcurveto{\pgfqpoint{1.649119in}{2.930465in}}{\pgfqpoint{1.657019in}{2.927193in}}{\pgfqpoint{1.665256in}{2.927193in}}%
\pgfpathclose%
\pgfusepath{stroke,fill}%
\end{pgfscope}%
\begin{pgfscope}%
\pgfpathrectangle{\pgfqpoint{0.100000in}{0.212622in}}{\pgfqpoint{3.696000in}{3.696000in}}%
\pgfusepath{clip}%
\pgfsetbuttcap%
\pgfsetroundjoin%
\definecolor{currentfill}{rgb}{0.121569,0.466667,0.705882}%
\pgfsetfillcolor{currentfill}%
\pgfsetfillopacity{0.623552}%
\pgfsetlinewidth{1.003750pt}%
\definecolor{currentstroke}{rgb}{0.121569,0.466667,0.705882}%
\pgfsetstrokecolor{currentstroke}%
\pgfsetstrokeopacity{0.623552}%
\pgfsetdash{}{0pt}%
\pgfpathmoveto{\pgfqpoint{1.661954in}{2.921667in}}%
\pgfpathcurveto{\pgfqpoint{1.670190in}{2.921667in}}{\pgfqpoint{1.678090in}{2.924939in}}{\pgfqpoint{1.683914in}{2.930763in}}%
\pgfpathcurveto{\pgfqpoint{1.689738in}{2.936587in}}{\pgfqpoint{1.693010in}{2.944487in}}{\pgfqpoint{1.693010in}{2.952723in}}%
\pgfpathcurveto{\pgfqpoint{1.693010in}{2.960960in}}{\pgfqpoint{1.689738in}{2.968860in}}{\pgfqpoint{1.683914in}{2.974684in}}%
\pgfpathcurveto{\pgfqpoint{1.678090in}{2.980508in}}{\pgfqpoint{1.670190in}{2.983780in}}{\pgfqpoint{1.661954in}{2.983780in}}%
\pgfpathcurveto{\pgfqpoint{1.653717in}{2.983780in}}{\pgfqpoint{1.645817in}{2.980508in}}{\pgfqpoint{1.639993in}{2.974684in}}%
\pgfpathcurveto{\pgfqpoint{1.634169in}{2.968860in}}{\pgfqpoint{1.630897in}{2.960960in}}{\pgfqpoint{1.630897in}{2.952723in}}%
\pgfpathcurveto{\pgfqpoint{1.630897in}{2.944487in}}{\pgfqpoint{1.634169in}{2.936587in}}{\pgfqpoint{1.639993in}{2.930763in}}%
\pgfpathcurveto{\pgfqpoint{1.645817in}{2.924939in}}{\pgfqpoint{1.653717in}{2.921667in}}{\pgfqpoint{1.661954in}{2.921667in}}%
\pgfpathclose%
\pgfusepath{stroke,fill}%
\end{pgfscope}%
\begin{pgfscope}%
\pgfpathrectangle{\pgfqpoint{0.100000in}{0.212622in}}{\pgfqpoint{3.696000in}{3.696000in}}%
\pgfusepath{clip}%
\pgfsetbuttcap%
\pgfsetroundjoin%
\definecolor{currentfill}{rgb}{0.121569,0.466667,0.705882}%
\pgfsetfillcolor{currentfill}%
\pgfsetfillopacity{0.624316}%
\pgfsetlinewidth{1.003750pt}%
\definecolor{currentstroke}{rgb}{0.121569,0.466667,0.705882}%
\pgfsetstrokecolor{currentstroke}%
\pgfsetstrokeopacity{0.624316}%
\pgfsetdash{}{0pt}%
\pgfpathmoveto{\pgfqpoint{1.658458in}{2.915522in}}%
\pgfpathcurveto{\pgfqpoint{1.666694in}{2.915522in}}{\pgfqpoint{1.674595in}{2.918795in}}{\pgfqpoint{1.680418in}{2.924619in}}%
\pgfpathcurveto{\pgfqpoint{1.686242in}{2.930442in}}{\pgfqpoint{1.689515in}{2.938343in}}{\pgfqpoint{1.689515in}{2.946579in}}%
\pgfpathcurveto{\pgfqpoint{1.689515in}{2.954815in}}{\pgfqpoint{1.686242in}{2.962715in}}{\pgfqpoint{1.680418in}{2.968539in}}%
\pgfpathcurveto{\pgfqpoint{1.674595in}{2.974363in}}{\pgfqpoint{1.666694in}{2.977635in}}{\pgfqpoint{1.658458in}{2.977635in}}%
\pgfpathcurveto{\pgfqpoint{1.650222in}{2.977635in}}{\pgfqpoint{1.642322in}{2.974363in}}{\pgfqpoint{1.636498in}{2.968539in}}%
\pgfpathcurveto{\pgfqpoint{1.630674in}{2.962715in}}{\pgfqpoint{1.627402in}{2.954815in}}{\pgfqpoint{1.627402in}{2.946579in}}%
\pgfpathcurveto{\pgfqpoint{1.627402in}{2.938343in}}{\pgfqpoint{1.630674in}{2.930442in}}{\pgfqpoint{1.636498in}{2.924619in}}%
\pgfpathcurveto{\pgfqpoint{1.642322in}{2.918795in}}{\pgfqpoint{1.650222in}{2.915522in}}{\pgfqpoint{1.658458in}{2.915522in}}%
\pgfpathclose%
\pgfusepath{stroke,fill}%
\end{pgfscope}%
\begin{pgfscope}%
\pgfpathrectangle{\pgfqpoint{0.100000in}{0.212622in}}{\pgfqpoint{3.696000in}{3.696000in}}%
\pgfusepath{clip}%
\pgfsetbuttcap%
\pgfsetroundjoin%
\definecolor{currentfill}{rgb}{0.121569,0.466667,0.705882}%
\pgfsetfillcolor{currentfill}%
\pgfsetfillopacity{0.624741}%
\pgfsetlinewidth{1.003750pt}%
\definecolor{currentstroke}{rgb}{0.121569,0.466667,0.705882}%
\pgfsetstrokecolor{currentstroke}%
\pgfsetstrokeopacity{0.624741}%
\pgfsetdash{}{0pt}%
\pgfpathmoveto{\pgfqpoint{1.656505in}{2.912217in}}%
\pgfpathcurveto{\pgfqpoint{1.664742in}{2.912217in}}{\pgfqpoint{1.672642in}{2.915489in}}{\pgfqpoint{1.678466in}{2.921313in}}%
\pgfpathcurveto{\pgfqpoint{1.684290in}{2.927137in}}{\pgfqpoint{1.687562in}{2.935037in}}{\pgfqpoint{1.687562in}{2.943273in}}%
\pgfpathcurveto{\pgfqpoint{1.687562in}{2.951510in}}{\pgfqpoint{1.684290in}{2.959410in}}{\pgfqpoint{1.678466in}{2.965234in}}%
\pgfpathcurveto{\pgfqpoint{1.672642in}{2.971058in}}{\pgfqpoint{1.664742in}{2.974330in}}{\pgfqpoint{1.656505in}{2.974330in}}%
\pgfpathcurveto{\pgfqpoint{1.648269in}{2.974330in}}{\pgfqpoint{1.640369in}{2.971058in}}{\pgfqpoint{1.634545in}{2.965234in}}%
\pgfpathcurveto{\pgfqpoint{1.628721in}{2.959410in}}{\pgfqpoint{1.625449in}{2.951510in}}{\pgfqpoint{1.625449in}{2.943273in}}%
\pgfpathcurveto{\pgfqpoint{1.625449in}{2.935037in}}{\pgfqpoint{1.628721in}{2.927137in}}{\pgfqpoint{1.634545in}{2.921313in}}%
\pgfpathcurveto{\pgfqpoint{1.640369in}{2.915489in}}{\pgfqpoint{1.648269in}{2.912217in}}{\pgfqpoint{1.656505in}{2.912217in}}%
\pgfpathclose%
\pgfusepath{stroke,fill}%
\end{pgfscope}%
\begin{pgfscope}%
\pgfpathrectangle{\pgfqpoint{0.100000in}{0.212622in}}{\pgfqpoint{3.696000in}{3.696000in}}%
\pgfusepath{clip}%
\pgfsetbuttcap%
\pgfsetroundjoin%
\definecolor{currentfill}{rgb}{0.121569,0.466667,0.705882}%
\pgfsetfillcolor{currentfill}%
\pgfsetfillopacity{0.625354}%
\pgfsetlinewidth{1.003750pt}%
\definecolor{currentstroke}{rgb}{0.121569,0.466667,0.705882}%
\pgfsetstrokecolor{currentstroke}%
\pgfsetstrokeopacity{0.625354}%
\pgfsetdash{}{0pt}%
\pgfpathmoveto{\pgfqpoint{3.016862in}{2.626926in}}%
\pgfpathcurveto{\pgfqpoint{3.025098in}{2.626926in}}{\pgfqpoint{3.032998in}{2.630198in}}{\pgfqpoint{3.038822in}{2.636022in}}%
\pgfpathcurveto{\pgfqpoint{3.044646in}{2.641846in}}{\pgfqpoint{3.047918in}{2.649746in}}{\pgfqpoint{3.047918in}{2.657982in}}%
\pgfpathcurveto{\pgfqpoint{3.047918in}{2.666218in}}{\pgfqpoint{3.044646in}{2.674118in}}{\pgfqpoint{3.038822in}{2.679942in}}%
\pgfpathcurveto{\pgfqpoint{3.032998in}{2.685766in}}{\pgfqpoint{3.025098in}{2.689039in}}{\pgfqpoint{3.016862in}{2.689039in}}%
\pgfpathcurveto{\pgfqpoint{3.008626in}{2.689039in}}{\pgfqpoint{3.000726in}{2.685766in}}{\pgfqpoint{2.994902in}{2.679942in}}%
\pgfpathcurveto{\pgfqpoint{2.989078in}{2.674118in}}{\pgfqpoint{2.985805in}{2.666218in}}{\pgfqpoint{2.985805in}{2.657982in}}%
\pgfpathcurveto{\pgfqpoint{2.985805in}{2.649746in}}{\pgfqpoint{2.989078in}{2.641846in}}{\pgfqpoint{2.994902in}{2.636022in}}%
\pgfpathcurveto{\pgfqpoint{3.000726in}{2.630198in}}{\pgfqpoint{3.008626in}{2.626926in}}{\pgfqpoint{3.016862in}{2.626926in}}%
\pgfpathclose%
\pgfusepath{stroke,fill}%
\end{pgfscope}%
\begin{pgfscope}%
\pgfpathrectangle{\pgfqpoint{0.100000in}{0.212622in}}{\pgfqpoint{3.696000in}{3.696000in}}%
\pgfusepath{clip}%
\pgfsetbuttcap%
\pgfsetroundjoin%
\definecolor{currentfill}{rgb}{0.121569,0.466667,0.705882}%
\pgfsetfillcolor{currentfill}%
\pgfsetfillopacity{0.625489}%
\pgfsetlinewidth{1.003750pt}%
\definecolor{currentstroke}{rgb}{0.121569,0.466667,0.705882}%
\pgfsetstrokecolor{currentstroke}%
\pgfsetstrokeopacity{0.625489}%
\pgfsetdash{}{0pt}%
\pgfpathmoveto{\pgfqpoint{1.654922in}{2.906936in}}%
\pgfpathcurveto{\pgfqpoint{1.663158in}{2.906936in}}{\pgfqpoint{1.671058in}{2.910208in}}{\pgfqpoint{1.676882in}{2.916032in}}%
\pgfpathcurveto{\pgfqpoint{1.682706in}{2.921856in}}{\pgfqpoint{1.685978in}{2.929756in}}{\pgfqpoint{1.685978in}{2.937992in}}%
\pgfpathcurveto{\pgfqpoint{1.685978in}{2.946228in}}{\pgfqpoint{1.682706in}{2.954128in}}{\pgfqpoint{1.676882in}{2.959952in}}%
\pgfpathcurveto{\pgfqpoint{1.671058in}{2.965776in}}{\pgfqpoint{1.663158in}{2.969049in}}{\pgfqpoint{1.654922in}{2.969049in}}%
\pgfpathcurveto{\pgfqpoint{1.646685in}{2.969049in}}{\pgfqpoint{1.638785in}{2.965776in}}{\pgfqpoint{1.632961in}{2.959952in}}%
\pgfpathcurveto{\pgfqpoint{1.627137in}{2.954128in}}{\pgfqpoint{1.623865in}{2.946228in}}{\pgfqpoint{1.623865in}{2.937992in}}%
\pgfpathcurveto{\pgfqpoint{1.623865in}{2.929756in}}{\pgfqpoint{1.627137in}{2.921856in}}{\pgfqpoint{1.632961in}{2.916032in}}%
\pgfpathcurveto{\pgfqpoint{1.638785in}{2.910208in}}{\pgfqpoint{1.646685in}{2.906936in}}{\pgfqpoint{1.654922in}{2.906936in}}%
\pgfpathclose%
\pgfusepath{stroke,fill}%
\end{pgfscope}%
\begin{pgfscope}%
\pgfpathrectangle{\pgfqpoint{0.100000in}{0.212622in}}{\pgfqpoint{3.696000in}{3.696000in}}%
\pgfusepath{clip}%
\pgfsetbuttcap%
\pgfsetroundjoin%
\definecolor{currentfill}{rgb}{0.121569,0.466667,0.705882}%
\pgfsetfillcolor{currentfill}%
\pgfsetfillopacity{0.626148}%
\pgfsetlinewidth{1.003750pt}%
\definecolor{currentstroke}{rgb}{0.121569,0.466667,0.705882}%
\pgfsetstrokecolor{currentstroke}%
\pgfsetstrokeopacity{0.626148}%
\pgfsetdash{}{0pt}%
\pgfpathmoveto{\pgfqpoint{1.831756in}{3.049585in}}%
\pgfpathcurveto{\pgfqpoint{1.839993in}{3.049585in}}{\pgfqpoint{1.847893in}{3.052857in}}{\pgfqpoint{1.853717in}{3.058681in}}%
\pgfpathcurveto{\pgfqpoint{1.859541in}{3.064505in}}{\pgfqpoint{1.862813in}{3.072405in}}{\pgfqpoint{1.862813in}{3.080641in}}%
\pgfpathcurveto{\pgfqpoint{1.862813in}{3.088878in}}{\pgfqpoint{1.859541in}{3.096778in}}{\pgfqpoint{1.853717in}{3.102602in}}%
\pgfpathcurveto{\pgfqpoint{1.847893in}{3.108425in}}{\pgfqpoint{1.839993in}{3.111698in}}{\pgfqpoint{1.831756in}{3.111698in}}%
\pgfpathcurveto{\pgfqpoint{1.823520in}{3.111698in}}{\pgfqpoint{1.815620in}{3.108425in}}{\pgfqpoint{1.809796in}{3.102602in}}%
\pgfpathcurveto{\pgfqpoint{1.803972in}{3.096778in}}{\pgfqpoint{1.800700in}{3.088878in}}{\pgfqpoint{1.800700in}{3.080641in}}%
\pgfpathcurveto{\pgfqpoint{1.800700in}{3.072405in}}{\pgfqpoint{1.803972in}{3.064505in}}{\pgfqpoint{1.809796in}{3.058681in}}%
\pgfpathcurveto{\pgfqpoint{1.815620in}{3.052857in}}{\pgfqpoint{1.823520in}{3.049585in}}{\pgfqpoint{1.831756in}{3.049585in}}%
\pgfpathclose%
\pgfusepath{stroke,fill}%
\end{pgfscope}%
\begin{pgfscope}%
\pgfpathrectangle{\pgfqpoint{0.100000in}{0.212622in}}{\pgfqpoint{3.696000in}{3.696000in}}%
\pgfusepath{clip}%
\pgfsetbuttcap%
\pgfsetroundjoin%
\definecolor{currentfill}{rgb}{0.121569,0.466667,0.705882}%
\pgfsetfillcolor{currentfill}%
\pgfsetfillopacity{0.626252}%
\pgfsetlinewidth{1.003750pt}%
\definecolor{currentstroke}{rgb}{0.121569,0.466667,0.705882}%
\pgfsetstrokecolor{currentstroke}%
\pgfsetstrokeopacity{0.626252}%
\pgfsetdash{}{0pt}%
\pgfpathmoveto{\pgfqpoint{1.651981in}{2.900722in}}%
\pgfpathcurveto{\pgfqpoint{1.660217in}{2.900722in}}{\pgfqpoint{1.668117in}{2.903995in}}{\pgfqpoint{1.673941in}{2.909819in}}%
\pgfpathcurveto{\pgfqpoint{1.679765in}{2.915643in}}{\pgfqpoint{1.683037in}{2.923543in}}{\pgfqpoint{1.683037in}{2.931779in}}%
\pgfpathcurveto{\pgfqpoint{1.683037in}{2.940015in}}{\pgfqpoint{1.679765in}{2.947915in}}{\pgfqpoint{1.673941in}{2.953739in}}%
\pgfpathcurveto{\pgfqpoint{1.668117in}{2.959563in}}{\pgfqpoint{1.660217in}{2.962835in}}{\pgfqpoint{1.651981in}{2.962835in}}%
\pgfpathcurveto{\pgfqpoint{1.643744in}{2.962835in}}{\pgfqpoint{1.635844in}{2.959563in}}{\pgfqpoint{1.630021in}{2.953739in}}%
\pgfpathcurveto{\pgfqpoint{1.624197in}{2.947915in}}{\pgfqpoint{1.620924in}{2.940015in}}{\pgfqpoint{1.620924in}{2.931779in}}%
\pgfpathcurveto{\pgfqpoint{1.620924in}{2.923543in}}{\pgfqpoint{1.624197in}{2.915643in}}{\pgfqpoint{1.630021in}{2.909819in}}%
\pgfpathcurveto{\pgfqpoint{1.635844in}{2.903995in}}{\pgfqpoint{1.643744in}{2.900722in}}{\pgfqpoint{1.651981in}{2.900722in}}%
\pgfpathclose%
\pgfusepath{stroke,fill}%
\end{pgfscope}%
\begin{pgfscope}%
\pgfpathrectangle{\pgfqpoint{0.100000in}{0.212622in}}{\pgfqpoint{3.696000in}{3.696000in}}%
\pgfusepath{clip}%
\pgfsetbuttcap%
\pgfsetroundjoin%
\definecolor{currentfill}{rgb}{0.121569,0.466667,0.705882}%
\pgfsetfillcolor{currentfill}%
\pgfsetfillopacity{0.626613}%
\pgfsetlinewidth{1.003750pt}%
\definecolor{currentstroke}{rgb}{0.121569,0.466667,0.705882}%
\pgfsetstrokecolor{currentstroke}%
\pgfsetstrokeopacity{0.626613}%
\pgfsetdash{}{0pt}%
\pgfpathmoveto{\pgfqpoint{1.650099in}{2.897430in}}%
\pgfpathcurveto{\pgfqpoint{1.658336in}{2.897430in}}{\pgfqpoint{1.666236in}{2.900703in}}{\pgfqpoint{1.672060in}{2.906527in}}%
\pgfpathcurveto{\pgfqpoint{1.677884in}{2.912351in}}{\pgfqpoint{1.681156in}{2.920251in}}{\pgfqpoint{1.681156in}{2.928487in}}%
\pgfpathcurveto{\pgfqpoint{1.681156in}{2.936723in}}{\pgfqpoint{1.677884in}{2.944623in}}{\pgfqpoint{1.672060in}{2.950447in}}%
\pgfpathcurveto{\pgfqpoint{1.666236in}{2.956271in}}{\pgfqpoint{1.658336in}{2.959543in}}{\pgfqpoint{1.650099in}{2.959543in}}%
\pgfpathcurveto{\pgfqpoint{1.641863in}{2.959543in}}{\pgfqpoint{1.633963in}{2.956271in}}{\pgfqpoint{1.628139in}{2.950447in}}%
\pgfpathcurveto{\pgfqpoint{1.622315in}{2.944623in}}{\pgfqpoint{1.619043in}{2.936723in}}{\pgfqpoint{1.619043in}{2.928487in}}%
\pgfpathcurveto{\pgfqpoint{1.619043in}{2.920251in}}{\pgfqpoint{1.622315in}{2.912351in}}{\pgfqpoint{1.628139in}{2.906527in}}%
\pgfpathcurveto{\pgfqpoint{1.633963in}{2.900703in}}{\pgfqpoint{1.641863in}{2.897430in}}{\pgfqpoint{1.650099in}{2.897430in}}%
\pgfpathclose%
\pgfusepath{stroke,fill}%
\end{pgfscope}%
\begin{pgfscope}%
\pgfpathrectangle{\pgfqpoint{0.100000in}{0.212622in}}{\pgfqpoint{3.696000in}{3.696000in}}%
\pgfusepath{clip}%
\pgfsetbuttcap%
\pgfsetroundjoin%
\definecolor{currentfill}{rgb}{0.121569,0.466667,0.705882}%
\pgfsetfillcolor{currentfill}%
\pgfsetfillopacity{0.627294}%
\pgfsetlinewidth{1.003750pt}%
\definecolor{currentstroke}{rgb}{0.121569,0.466667,0.705882}%
\pgfsetstrokecolor{currentstroke}%
\pgfsetstrokeopacity{0.627294}%
\pgfsetdash{}{0pt}%
\pgfpathmoveto{\pgfqpoint{1.647891in}{2.892522in}}%
\pgfpathcurveto{\pgfqpoint{1.656127in}{2.892522in}}{\pgfqpoint{1.664027in}{2.895795in}}{\pgfqpoint{1.669851in}{2.901619in}}%
\pgfpathcurveto{\pgfqpoint{1.675675in}{2.907443in}}{\pgfqpoint{1.678947in}{2.915343in}}{\pgfqpoint{1.678947in}{2.923579in}}%
\pgfpathcurveto{\pgfqpoint{1.678947in}{2.931815in}}{\pgfqpoint{1.675675in}{2.939715in}}{\pgfqpoint{1.669851in}{2.945539in}}%
\pgfpathcurveto{\pgfqpoint{1.664027in}{2.951363in}}{\pgfqpoint{1.656127in}{2.954635in}}{\pgfqpoint{1.647891in}{2.954635in}}%
\pgfpathcurveto{\pgfqpoint{1.639655in}{2.954635in}}{\pgfqpoint{1.631754in}{2.951363in}}{\pgfqpoint{1.625931in}{2.945539in}}%
\pgfpathcurveto{\pgfqpoint{1.620107in}{2.939715in}}{\pgfqpoint{1.616834in}{2.931815in}}{\pgfqpoint{1.616834in}{2.923579in}}%
\pgfpathcurveto{\pgfqpoint{1.616834in}{2.915343in}}{\pgfqpoint{1.620107in}{2.907443in}}{\pgfqpoint{1.625931in}{2.901619in}}%
\pgfpathcurveto{\pgfqpoint{1.631754in}{2.895795in}}{\pgfqpoint{1.639655in}{2.892522in}}{\pgfqpoint{1.647891in}{2.892522in}}%
\pgfpathclose%
\pgfusepath{stroke,fill}%
\end{pgfscope}%
\begin{pgfscope}%
\pgfpathrectangle{\pgfqpoint{0.100000in}{0.212622in}}{\pgfqpoint{3.696000in}{3.696000in}}%
\pgfusepath{clip}%
\pgfsetbuttcap%
\pgfsetroundjoin%
\definecolor{currentfill}{rgb}{0.121569,0.466667,0.705882}%
\pgfsetfillcolor{currentfill}%
\pgfsetfillopacity{0.627679}%
\pgfsetlinewidth{1.003750pt}%
\definecolor{currentstroke}{rgb}{0.121569,0.466667,0.705882}%
\pgfsetstrokecolor{currentstroke}%
\pgfsetstrokeopacity{0.627679}%
\pgfsetdash{}{0pt}%
\pgfpathmoveto{\pgfqpoint{1.647055in}{2.889623in}}%
\pgfpathcurveto{\pgfqpoint{1.655291in}{2.889623in}}{\pgfqpoint{1.663191in}{2.892896in}}{\pgfqpoint{1.669015in}{2.898720in}}%
\pgfpathcurveto{\pgfqpoint{1.674839in}{2.904543in}}{\pgfqpoint{1.678112in}{2.912443in}}{\pgfqpoint{1.678112in}{2.920680in}}%
\pgfpathcurveto{\pgfqpoint{1.678112in}{2.928916in}}{\pgfqpoint{1.674839in}{2.936816in}}{\pgfqpoint{1.669015in}{2.942640in}}%
\pgfpathcurveto{\pgfqpoint{1.663191in}{2.948464in}}{\pgfqpoint{1.655291in}{2.951736in}}{\pgfqpoint{1.647055in}{2.951736in}}%
\pgfpathcurveto{\pgfqpoint{1.638819in}{2.951736in}}{\pgfqpoint{1.630919in}{2.948464in}}{\pgfqpoint{1.625095in}{2.942640in}}%
\pgfpathcurveto{\pgfqpoint{1.619271in}{2.936816in}}{\pgfqpoint{1.615999in}{2.928916in}}{\pgfqpoint{1.615999in}{2.920680in}}%
\pgfpathcurveto{\pgfqpoint{1.615999in}{2.912443in}}{\pgfqpoint{1.619271in}{2.904543in}}{\pgfqpoint{1.625095in}{2.898720in}}%
\pgfpathcurveto{\pgfqpoint{1.630919in}{2.892896in}}{\pgfqpoint{1.638819in}{2.889623in}}{\pgfqpoint{1.647055in}{2.889623in}}%
\pgfpathclose%
\pgfusepath{stroke,fill}%
\end{pgfscope}%
\begin{pgfscope}%
\pgfpathrectangle{\pgfqpoint{0.100000in}{0.212622in}}{\pgfqpoint{3.696000in}{3.696000in}}%
\pgfusepath{clip}%
\pgfsetbuttcap%
\pgfsetroundjoin%
\definecolor{currentfill}{rgb}{0.121569,0.466667,0.705882}%
\pgfsetfillcolor{currentfill}%
\pgfsetfillopacity{0.628222}%
\pgfsetlinewidth{1.003750pt}%
\definecolor{currentstroke}{rgb}{0.121569,0.466667,0.705882}%
\pgfsetstrokecolor{currentstroke}%
\pgfsetstrokeopacity{0.628222}%
\pgfsetdash{}{0pt}%
\pgfpathmoveto{\pgfqpoint{1.644882in}{2.885649in}}%
\pgfpathcurveto{\pgfqpoint{1.653119in}{2.885649in}}{\pgfqpoint{1.661019in}{2.888921in}}{\pgfqpoint{1.666843in}{2.894745in}}%
\pgfpathcurveto{\pgfqpoint{1.672667in}{2.900569in}}{\pgfqpoint{1.675939in}{2.908469in}}{\pgfqpoint{1.675939in}{2.916706in}}%
\pgfpathcurveto{\pgfqpoint{1.675939in}{2.924942in}}{\pgfqpoint{1.672667in}{2.932842in}}{\pgfqpoint{1.666843in}{2.938666in}}%
\pgfpathcurveto{\pgfqpoint{1.661019in}{2.944490in}}{\pgfqpoint{1.653119in}{2.947762in}}{\pgfqpoint{1.644882in}{2.947762in}}%
\pgfpathcurveto{\pgfqpoint{1.636646in}{2.947762in}}{\pgfqpoint{1.628746in}{2.944490in}}{\pgfqpoint{1.622922in}{2.938666in}}%
\pgfpathcurveto{\pgfqpoint{1.617098in}{2.932842in}}{\pgfqpoint{1.613826in}{2.924942in}}{\pgfqpoint{1.613826in}{2.916706in}}%
\pgfpathcurveto{\pgfqpoint{1.613826in}{2.908469in}}{\pgfqpoint{1.617098in}{2.900569in}}{\pgfqpoint{1.622922in}{2.894745in}}%
\pgfpathcurveto{\pgfqpoint{1.628746in}{2.888921in}}{\pgfqpoint{1.636646in}{2.885649in}}{\pgfqpoint{1.644882in}{2.885649in}}%
\pgfpathclose%
\pgfusepath{stroke,fill}%
\end{pgfscope}%
\begin{pgfscope}%
\pgfpathrectangle{\pgfqpoint{0.100000in}{0.212622in}}{\pgfqpoint{3.696000in}{3.696000in}}%
\pgfusepath{clip}%
\pgfsetbuttcap%
\pgfsetroundjoin%
\definecolor{currentfill}{rgb}{0.121569,0.466667,0.705882}%
\pgfsetfillcolor{currentfill}%
\pgfsetfillopacity{0.628470}%
\pgfsetlinewidth{1.003750pt}%
\definecolor{currentstroke}{rgb}{0.121569,0.466667,0.705882}%
\pgfsetstrokecolor{currentstroke}%
\pgfsetstrokeopacity{0.628470}%
\pgfsetdash{}{0pt}%
\pgfpathmoveto{\pgfqpoint{1.643585in}{2.883402in}}%
\pgfpathcurveto{\pgfqpoint{1.651821in}{2.883402in}}{\pgfqpoint{1.659721in}{2.886674in}}{\pgfqpoint{1.665545in}{2.892498in}}%
\pgfpathcurveto{\pgfqpoint{1.671369in}{2.898322in}}{\pgfqpoint{1.674641in}{2.906222in}}{\pgfqpoint{1.674641in}{2.914458in}}%
\pgfpathcurveto{\pgfqpoint{1.674641in}{2.922694in}}{\pgfqpoint{1.671369in}{2.930595in}}{\pgfqpoint{1.665545in}{2.936418in}}%
\pgfpathcurveto{\pgfqpoint{1.659721in}{2.942242in}}{\pgfqpoint{1.651821in}{2.945515in}}{\pgfqpoint{1.643585in}{2.945515in}}%
\pgfpathcurveto{\pgfqpoint{1.635348in}{2.945515in}}{\pgfqpoint{1.627448in}{2.942242in}}{\pgfqpoint{1.621624in}{2.936418in}}%
\pgfpathcurveto{\pgfqpoint{1.615800in}{2.930595in}}{\pgfqpoint{1.612528in}{2.922694in}}{\pgfqpoint{1.612528in}{2.914458in}}%
\pgfpathcurveto{\pgfqpoint{1.612528in}{2.906222in}}{\pgfqpoint{1.615800in}{2.898322in}}{\pgfqpoint{1.621624in}{2.892498in}}%
\pgfpathcurveto{\pgfqpoint{1.627448in}{2.886674in}}{\pgfqpoint{1.635348in}{2.883402in}}{\pgfqpoint{1.643585in}{2.883402in}}%
\pgfpathclose%
\pgfusepath{stroke,fill}%
\end{pgfscope}%
\begin{pgfscope}%
\pgfpathrectangle{\pgfqpoint{0.100000in}{0.212622in}}{\pgfqpoint{3.696000in}{3.696000in}}%
\pgfusepath{clip}%
\pgfsetbuttcap%
\pgfsetroundjoin%
\definecolor{currentfill}{rgb}{0.121569,0.466667,0.705882}%
\pgfsetfillcolor{currentfill}%
\pgfsetfillopacity{0.628799}%
\pgfsetlinewidth{1.003750pt}%
\definecolor{currentstroke}{rgb}{0.121569,0.466667,0.705882}%
\pgfsetstrokecolor{currentstroke}%
\pgfsetstrokeopacity{0.628799}%
\pgfsetdash{}{0pt}%
\pgfpathmoveto{\pgfqpoint{3.029601in}{2.626598in}}%
\pgfpathcurveto{\pgfqpoint{3.037837in}{2.626598in}}{\pgfqpoint{3.045737in}{2.629870in}}{\pgfqpoint{3.051561in}{2.635694in}}%
\pgfpathcurveto{\pgfqpoint{3.057385in}{2.641518in}}{\pgfqpoint{3.060657in}{2.649418in}}{\pgfqpoint{3.060657in}{2.657654in}}%
\pgfpathcurveto{\pgfqpoint{3.060657in}{2.665891in}}{\pgfqpoint{3.057385in}{2.673791in}}{\pgfqpoint{3.051561in}{2.679615in}}%
\pgfpathcurveto{\pgfqpoint{3.045737in}{2.685439in}}{\pgfqpoint{3.037837in}{2.688711in}}{\pgfqpoint{3.029601in}{2.688711in}}%
\pgfpathcurveto{\pgfqpoint{3.021365in}{2.688711in}}{\pgfqpoint{3.013465in}{2.685439in}}{\pgfqpoint{3.007641in}{2.679615in}}%
\pgfpathcurveto{\pgfqpoint{3.001817in}{2.673791in}}{\pgfqpoint{2.998544in}{2.665891in}}{\pgfqpoint{2.998544in}{2.657654in}}%
\pgfpathcurveto{\pgfqpoint{2.998544in}{2.649418in}}{\pgfqpoint{3.001817in}{2.641518in}}{\pgfqpoint{3.007641in}{2.635694in}}%
\pgfpathcurveto{\pgfqpoint{3.013465in}{2.629870in}}{\pgfqpoint{3.021365in}{2.626598in}}{\pgfqpoint{3.029601in}{2.626598in}}%
\pgfpathclose%
\pgfusepath{stroke,fill}%
\end{pgfscope}%
\begin{pgfscope}%
\pgfpathrectangle{\pgfqpoint{0.100000in}{0.212622in}}{\pgfqpoint{3.696000in}{3.696000in}}%
\pgfusepath{clip}%
\pgfsetbuttcap%
\pgfsetroundjoin%
\definecolor{currentfill}{rgb}{0.121569,0.466667,0.705882}%
\pgfsetfillcolor{currentfill}%
\pgfsetfillopacity{0.629025}%
\pgfsetlinewidth{1.003750pt}%
\definecolor{currentstroke}{rgb}{0.121569,0.466667,0.705882}%
\pgfsetstrokecolor{currentstroke}%
\pgfsetstrokeopacity{0.629025}%
\pgfsetdash{}{0pt}%
\pgfpathmoveto{\pgfqpoint{1.641774in}{2.879405in}}%
\pgfpathcurveto{\pgfqpoint{1.650011in}{2.879405in}}{\pgfqpoint{1.657911in}{2.882677in}}{\pgfqpoint{1.663735in}{2.888501in}}%
\pgfpathcurveto{\pgfqpoint{1.669559in}{2.894325in}}{\pgfqpoint{1.672831in}{2.902225in}}{\pgfqpoint{1.672831in}{2.910462in}}%
\pgfpathcurveto{\pgfqpoint{1.672831in}{2.918698in}}{\pgfqpoint{1.669559in}{2.926598in}}{\pgfqpoint{1.663735in}{2.932422in}}%
\pgfpathcurveto{\pgfqpoint{1.657911in}{2.938246in}}{\pgfqpoint{1.650011in}{2.941518in}}{\pgfqpoint{1.641774in}{2.941518in}}%
\pgfpathcurveto{\pgfqpoint{1.633538in}{2.941518in}}{\pgfqpoint{1.625638in}{2.938246in}}{\pgfqpoint{1.619814in}{2.932422in}}%
\pgfpathcurveto{\pgfqpoint{1.613990in}{2.926598in}}{\pgfqpoint{1.610718in}{2.918698in}}{\pgfqpoint{1.610718in}{2.910462in}}%
\pgfpathcurveto{\pgfqpoint{1.610718in}{2.902225in}}{\pgfqpoint{1.613990in}{2.894325in}}{\pgfqpoint{1.619814in}{2.888501in}}%
\pgfpathcurveto{\pgfqpoint{1.625638in}{2.882677in}}{\pgfqpoint{1.633538in}{2.879405in}}{\pgfqpoint{1.641774in}{2.879405in}}%
\pgfpathclose%
\pgfusepath{stroke,fill}%
\end{pgfscope}%
\begin{pgfscope}%
\pgfpathrectangle{\pgfqpoint{0.100000in}{0.212622in}}{\pgfqpoint{3.696000in}{3.696000in}}%
\pgfusepath{clip}%
\pgfsetbuttcap%
\pgfsetroundjoin%
\definecolor{currentfill}{rgb}{0.121569,0.466667,0.705882}%
\pgfsetfillcolor{currentfill}%
\pgfsetfillopacity{0.629278}%
\pgfsetlinewidth{1.003750pt}%
\definecolor{currentstroke}{rgb}{0.121569,0.466667,0.705882}%
\pgfsetstrokecolor{currentstroke}%
\pgfsetstrokeopacity{0.629278}%
\pgfsetdash{}{0pt}%
\pgfpathmoveto{\pgfqpoint{1.845005in}{3.046672in}}%
\pgfpathcurveto{\pgfqpoint{1.853241in}{3.046672in}}{\pgfqpoint{1.861141in}{3.049944in}}{\pgfqpoint{1.866965in}{3.055768in}}%
\pgfpathcurveto{\pgfqpoint{1.872789in}{3.061592in}}{\pgfqpoint{1.876062in}{3.069492in}}{\pgfqpoint{1.876062in}{3.077728in}}%
\pgfpathcurveto{\pgfqpoint{1.876062in}{3.085964in}}{\pgfqpoint{1.872789in}{3.093864in}}{\pgfqpoint{1.866965in}{3.099688in}}%
\pgfpathcurveto{\pgfqpoint{1.861141in}{3.105512in}}{\pgfqpoint{1.853241in}{3.108785in}}{\pgfqpoint{1.845005in}{3.108785in}}%
\pgfpathcurveto{\pgfqpoint{1.836769in}{3.108785in}}{\pgfqpoint{1.828869in}{3.105512in}}{\pgfqpoint{1.823045in}{3.099688in}}%
\pgfpathcurveto{\pgfqpoint{1.817221in}{3.093864in}}{\pgfqpoint{1.813949in}{3.085964in}}{\pgfqpoint{1.813949in}{3.077728in}}%
\pgfpathcurveto{\pgfqpoint{1.813949in}{3.069492in}}{\pgfqpoint{1.817221in}{3.061592in}}{\pgfqpoint{1.823045in}{3.055768in}}%
\pgfpathcurveto{\pgfqpoint{1.828869in}{3.049944in}}{\pgfqpoint{1.836769in}{3.046672in}}{\pgfqpoint{1.845005in}{3.046672in}}%
\pgfpathclose%
\pgfusepath{stroke,fill}%
\end{pgfscope}%
\begin{pgfscope}%
\pgfpathrectangle{\pgfqpoint{0.100000in}{0.212622in}}{\pgfqpoint{3.696000in}{3.696000in}}%
\pgfusepath{clip}%
\pgfsetbuttcap%
\pgfsetroundjoin%
\definecolor{currentfill}{rgb}{0.121569,0.466667,0.705882}%
\pgfsetfillcolor{currentfill}%
\pgfsetfillopacity{0.629699}%
\pgfsetlinewidth{1.003750pt}%
\definecolor{currentstroke}{rgb}{0.121569,0.466667,0.705882}%
\pgfsetstrokecolor{currentstroke}%
\pgfsetstrokeopacity{0.629699}%
\pgfsetdash{}{0pt}%
\pgfpathmoveto{\pgfqpoint{1.640622in}{2.874056in}}%
\pgfpathcurveto{\pgfqpoint{1.648859in}{2.874056in}}{\pgfqpoint{1.656759in}{2.877328in}}{\pgfqpoint{1.662583in}{2.883152in}}%
\pgfpathcurveto{\pgfqpoint{1.668406in}{2.888976in}}{\pgfqpoint{1.671679in}{2.896876in}}{\pgfqpoint{1.671679in}{2.905112in}}%
\pgfpathcurveto{\pgfqpoint{1.671679in}{2.913348in}}{\pgfqpoint{1.668406in}{2.921248in}}{\pgfqpoint{1.662583in}{2.927072in}}%
\pgfpathcurveto{\pgfqpoint{1.656759in}{2.932896in}}{\pgfqpoint{1.648859in}{2.936169in}}{\pgfqpoint{1.640622in}{2.936169in}}%
\pgfpathcurveto{\pgfqpoint{1.632386in}{2.936169in}}{\pgfqpoint{1.624486in}{2.932896in}}{\pgfqpoint{1.618662in}{2.927072in}}%
\pgfpathcurveto{\pgfqpoint{1.612838in}{2.921248in}}{\pgfqpoint{1.609566in}{2.913348in}}{\pgfqpoint{1.609566in}{2.905112in}}%
\pgfpathcurveto{\pgfqpoint{1.609566in}{2.896876in}}{\pgfqpoint{1.612838in}{2.888976in}}{\pgfqpoint{1.618662in}{2.883152in}}%
\pgfpathcurveto{\pgfqpoint{1.624486in}{2.877328in}}{\pgfqpoint{1.632386in}{2.874056in}}{\pgfqpoint{1.640622in}{2.874056in}}%
\pgfpathclose%
\pgfusepath{stroke,fill}%
\end{pgfscope}%
\begin{pgfscope}%
\pgfpathrectangle{\pgfqpoint{0.100000in}{0.212622in}}{\pgfqpoint{3.696000in}{3.696000in}}%
\pgfusepath{clip}%
\pgfsetbuttcap%
\pgfsetroundjoin%
\definecolor{currentfill}{rgb}{0.121569,0.466667,0.705882}%
\pgfsetfillcolor{currentfill}%
\pgfsetfillopacity{0.630562}%
\pgfsetlinewidth{1.003750pt}%
\definecolor{currentstroke}{rgb}{0.121569,0.466667,0.705882}%
\pgfsetstrokecolor{currentstroke}%
\pgfsetstrokeopacity{0.630562}%
\pgfsetdash{}{0pt}%
\pgfpathmoveto{\pgfqpoint{1.637830in}{2.867908in}}%
\pgfpathcurveto{\pgfqpoint{1.646066in}{2.867908in}}{\pgfqpoint{1.653966in}{2.871180in}}{\pgfqpoint{1.659790in}{2.877004in}}%
\pgfpathcurveto{\pgfqpoint{1.665614in}{2.882828in}}{\pgfqpoint{1.668886in}{2.890728in}}{\pgfqpoint{1.668886in}{2.898964in}}%
\pgfpathcurveto{\pgfqpoint{1.668886in}{2.907200in}}{\pgfqpoint{1.665614in}{2.915100in}}{\pgfqpoint{1.659790in}{2.920924in}}%
\pgfpathcurveto{\pgfqpoint{1.653966in}{2.926748in}}{\pgfqpoint{1.646066in}{2.930021in}}{\pgfqpoint{1.637830in}{2.930021in}}%
\pgfpathcurveto{\pgfqpoint{1.629593in}{2.930021in}}{\pgfqpoint{1.621693in}{2.926748in}}{\pgfqpoint{1.615869in}{2.920924in}}%
\pgfpathcurveto{\pgfqpoint{1.610046in}{2.915100in}}{\pgfqpoint{1.606773in}{2.907200in}}{\pgfqpoint{1.606773in}{2.898964in}}%
\pgfpathcurveto{\pgfqpoint{1.606773in}{2.890728in}}{\pgfqpoint{1.610046in}{2.882828in}}{\pgfqpoint{1.615869in}{2.877004in}}%
\pgfpathcurveto{\pgfqpoint{1.621693in}{2.871180in}}{\pgfqpoint{1.629593in}{2.867908in}}{\pgfqpoint{1.637830in}{2.867908in}}%
\pgfpathclose%
\pgfusepath{stroke,fill}%
\end{pgfscope}%
\begin{pgfscope}%
\pgfpathrectangle{\pgfqpoint{0.100000in}{0.212622in}}{\pgfqpoint{3.696000in}{3.696000in}}%
\pgfusepath{clip}%
\pgfsetbuttcap%
\pgfsetroundjoin%
\definecolor{currentfill}{rgb}{0.121569,0.466667,0.705882}%
\pgfsetfillcolor{currentfill}%
\pgfsetfillopacity{0.631054}%
\pgfsetlinewidth{1.003750pt}%
\definecolor{currentstroke}{rgb}{0.121569,0.466667,0.705882}%
\pgfsetstrokecolor{currentstroke}%
\pgfsetstrokeopacity{0.631054}%
\pgfsetdash{}{0pt}%
\pgfpathmoveto{\pgfqpoint{1.636091in}{2.864847in}}%
\pgfpathcurveto{\pgfqpoint{1.644327in}{2.864847in}}{\pgfqpoint{1.652228in}{2.868119in}}{\pgfqpoint{1.658051in}{2.873943in}}%
\pgfpathcurveto{\pgfqpoint{1.663875in}{2.879767in}}{\pgfqpoint{1.667148in}{2.887667in}}{\pgfqpoint{1.667148in}{2.895904in}}%
\pgfpathcurveto{\pgfqpoint{1.667148in}{2.904140in}}{\pgfqpoint{1.663875in}{2.912040in}}{\pgfqpoint{1.658051in}{2.917864in}}%
\pgfpathcurveto{\pgfqpoint{1.652228in}{2.923688in}}{\pgfqpoint{1.644327in}{2.926960in}}{\pgfqpoint{1.636091in}{2.926960in}}%
\pgfpathcurveto{\pgfqpoint{1.627855in}{2.926960in}}{\pgfqpoint{1.619955in}{2.923688in}}{\pgfqpoint{1.614131in}{2.917864in}}%
\pgfpathcurveto{\pgfqpoint{1.608307in}{2.912040in}}{\pgfqpoint{1.605035in}{2.904140in}}{\pgfqpoint{1.605035in}{2.895904in}}%
\pgfpathcurveto{\pgfqpoint{1.605035in}{2.887667in}}{\pgfqpoint{1.608307in}{2.879767in}}{\pgfqpoint{1.614131in}{2.873943in}}%
\pgfpathcurveto{\pgfqpoint{1.619955in}{2.868119in}}{\pgfqpoint{1.627855in}{2.864847in}}{\pgfqpoint{1.636091in}{2.864847in}}%
\pgfpathclose%
\pgfusepath{stroke,fill}%
\end{pgfscope}%
\begin{pgfscope}%
\pgfpathrectangle{\pgfqpoint{0.100000in}{0.212622in}}{\pgfqpoint{3.696000in}{3.696000in}}%
\pgfusepath{clip}%
\pgfsetbuttcap%
\pgfsetroundjoin%
\definecolor{currentfill}{rgb}{0.121569,0.466667,0.705882}%
\pgfsetfillcolor{currentfill}%
\pgfsetfillopacity{0.631317}%
\pgfsetlinewidth{1.003750pt}%
\definecolor{currentstroke}{rgb}{0.121569,0.466667,0.705882}%
\pgfsetstrokecolor{currentstroke}%
\pgfsetstrokeopacity{0.631317}%
\pgfsetdash{}{0pt}%
\pgfpathmoveto{\pgfqpoint{1.635128in}{2.863141in}}%
\pgfpathcurveto{\pgfqpoint{1.643364in}{2.863141in}}{\pgfqpoint{1.651264in}{2.866413in}}{\pgfqpoint{1.657088in}{2.872237in}}%
\pgfpathcurveto{\pgfqpoint{1.662912in}{2.878061in}}{\pgfqpoint{1.666184in}{2.885961in}}{\pgfqpoint{1.666184in}{2.894197in}}%
\pgfpathcurveto{\pgfqpoint{1.666184in}{2.902434in}}{\pgfqpoint{1.662912in}{2.910334in}}{\pgfqpoint{1.657088in}{2.916158in}}%
\pgfpathcurveto{\pgfqpoint{1.651264in}{2.921982in}}{\pgfqpoint{1.643364in}{2.925254in}}{\pgfqpoint{1.635128in}{2.925254in}}%
\pgfpathcurveto{\pgfqpoint{1.626891in}{2.925254in}}{\pgfqpoint{1.618991in}{2.921982in}}{\pgfqpoint{1.613167in}{2.916158in}}%
\pgfpathcurveto{\pgfqpoint{1.607343in}{2.910334in}}{\pgfqpoint{1.604071in}{2.902434in}}{\pgfqpoint{1.604071in}{2.894197in}}%
\pgfpathcurveto{\pgfqpoint{1.604071in}{2.885961in}}{\pgfqpoint{1.607343in}{2.878061in}}{\pgfqpoint{1.613167in}{2.872237in}}%
\pgfpathcurveto{\pgfqpoint{1.618991in}{2.866413in}}{\pgfqpoint{1.626891in}{2.863141in}}{\pgfqpoint{1.635128in}{2.863141in}}%
\pgfpathclose%
\pgfusepath{stroke,fill}%
\end{pgfscope}%
\begin{pgfscope}%
\pgfpathrectangle{\pgfqpoint{0.100000in}{0.212622in}}{\pgfqpoint{3.696000in}{3.696000in}}%
\pgfusepath{clip}%
\pgfsetbuttcap%
\pgfsetroundjoin%
\definecolor{currentfill}{rgb}{0.121569,0.466667,0.705882}%
\pgfsetfillcolor{currentfill}%
\pgfsetfillopacity{0.631694}%
\pgfsetlinewidth{1.003750pt}%
\definecolor{currentstroke}{rgb}{0.121569,0.466667,0.705882}%
\pgfsetstrokecolor{currentstroke}%
\pgfsetstrokeopacity{0.631694}%
\pgfsetdash{}{0pt}%
\pgfpathmoveto{\pgfqpoint{1.634246in}{2.860145in}}%
\pgfpathcurveto{\pgfqpoint{1.642482in}{2.860145in}}{\pgfqpoint{1.650382in}{2.863418in}}{\pgfqpoint{1.656206in}{2.869242in}}%
\pgfpathcurveto{\pgfqpoint{1.662030in}{2.875066in}}{\pgfqpoint{1.665302in}{2.882966in}}{\pgfqpoint{1.665302in}{2.891202in}}%
\pgfpathcurveto{\pgfqpoint{1.665302in}{2.899438in}}{\pgfqpoint{1.662030in}{2.907338in}}{\pgfqpoint{1.656206in}{2.913162in}}%
\pgfpathcurveto{\pgfqpoint{1.650382in}{2.918986in}}{\pgfqpoint{1.642482in}{2.922258in}}{\pgfqpoint{1.634246in}{2.922258in}}%
\pgfpathcurveto{\pgfqpoint{1.626009in}{2.922258in}}{\pgfqpoint{1.618109in}{2.918986in}}{\pgfqpoint{1.612285in}{2.913162in}}%
\pgfpathcurveto{\pgfqpoint{1.606461in}{2.907338in}}{\pgfqpoint{1.603189in}{2.899438in}}{\pgfqpoint{1.603189in}{2.891202in}}%
\pgfpathcurveto{\pgfqpoint{1.603189in}{2.882966in}}{\pgfqpoint{1.606461in}{2.875066in}}{\pgfqpoint{1.612285in}{2.869242in}}%
\pgfpathcurveto{\pgfqpoint{1.618109in}{2.863418in}}{\pgfqpoint{1.626009in}{2.860145in}}{\pgfqpoint{1.634246in}{2.860145in}}%
\pgfpathclose%
\pgfusepath{stroke,fill}%
\end{pgfscope}%
\begin{pgfscope}%
\pgfpathrectangle{\pgfqpoint{0.100000in}{0.212622in}}{\pgfqpoint{3.696000in}{3.696000in}}%
\pgfusepath{clip}%
\pgfsetbuttcap%
\pgfsetroundjoin%
\definecolor{currentfill}{rgb}{0.121569,0.466667,0.705882}%
\pgfsetfillcolor{currentfill}%
\pgfsetfillopacity{0.631884}%
\pgfsetlinewidth{1.003750pt}%
\definecolor{currentstroke}{rgb}{0.121569,0.466667,0.705882}%
\pgfsetstrokecolor{currentstroke}%
\pgfsetstrokeopacity{0.631884}%
\pgfsetdash{}{0pt}%
\pgfpathmoveto{\pgfqpoint{3.041273in}{2.625887in}}%
\pgfpathcurveto{\pgfqpoint{3.049509in}{2.625887in}}{\pgfqpoint{3.057409in}{2.629159in}}{\pgfqpoint{3.063233in}{2.634983in}}%
\pgfpathcurveto{\pgfqpoint{3.069057in}{2.640807in}}{\pgfqpoint{3.072330in}{2.648707in}}{\pgfqpoint{3.072330in}{2.656943in}}%
\pgfpathcurveto{\pgfqpoint{3.072330in}{2.665179in}}{\pgfqpoint{3.069057in}{2.673080in}}{\pgfqpoint{3.063233in}{2.678903in}}%
\pgfpathcurveto{\pgfqpoint{3.057409in}{2.684727in}}{\pgfqpoint{3.049509in}{2.688000in}}{\pgfqpoint{3.041273in}{2.688000in}}%
\pgfpathcurveto{\pgfqpoint{3.033037in}{2.688000in}}{\pgfqpoint{3.025137in}{2.684727in}}{\pgfqpoint{3.019313in}{2.678903in}}%
\pgfpathcurveto{\pgfqpoint{3.013489in}{2.673080in}}{\pgfqpoint{3.010217in}{2.665179in}}{\pgfqpoint{3.010217in}{2.656943in}}%
\pgfpathcurveto{\pgfqpoint{3.010217in}{2.648707in}}{\pgfqpoint{3.013489in}{2.640807in}}{\pgfqpoint{3.019313in}{2.634983in}}%
\pgfpathcurveto{\pgfqpoint{3.025137in}{2.629159in}}{\pgfqpoint{3.033037in}{2.625887in}}{\pgfqpoint{3.041273in}{2.625887in}}%
\pgfpathclose%
\pgfusepath{stroke,fill}%
\end{pgfscope}%
\begin{pgfscope}%
\pgfpathrectangle{\pgfqpoint{0.100000in}{0.212622in}}{\pgfqpoint{3.696000in}{3.696000in}}%
\pgfusepath{clip}%
\pgfsetbuttcap%
\pgfsetroundjoin%
\definecolor{currentfill}{rgb}{0.121569,0.466667,0.705882}%
\pgfsetfillcolor{currentfill}%
\pgfsetfillopacity{0.631989}%
\pgfsetlinewidth{1.003750pt}%
\definecolor{currentstroke}{rgb}{0.121569,0.466667,0.705882}%
\pgfsetstrokecolor{currentstroke}%
\pgfsetstrokeopacity{0.631989}%
\pgfsetdash{}{0pt}%
\pgfpathmoveto{\pgfqpoint{1.858077in}{3.042705in}}%
\pgfpathcurveto{\pgfqpoint{1.866314in}{3.042705in}}{\pgfqpoint{1.874214in}{3.045977in}}{\pgfqpoint{1.880037in}{3.051801in}}%
\pgfpathcurveto{\pgfqpoint{1.885861in}{3.057625in}}{\pgfqpoint{1.889134in}{3.065525in}}{\pgfqpoint{1.889134in}{3.073761in}}%
\pgfpathcurveto{\pgfqpoint{1.889134in}{3.081997in}}{\pgfqpoint{1.885861in}{3.089897in}}{\pgfqpoint{1.880037in}{3.095721in}}%
\pgfpathcurveto{\pgfqpoint{1.874214in}{3.101545in}}{\pgfqpoint{1.866314in}{3.104818in}}{\pgfqpoint{1.858077in}{3.104818in}}%
\pgfpathcurveto{\pgfqpoint{1.849841in}{3.104818in}}{\pgfqpoint{1.841941in}{3.101545in}}{\pgfqpoint{1.836117in}{3.095721in}}%
\pgfpathcurveto{\pgfqpoint{1.830293in}{3.089897in}}{\pgfqpoint{1.827021in}{3.081997in}}{\pgfqpoint{1.827021in}{3.073761in}}%
\pgfpathcurveto{\pgfqpoint{1.827021in}{3.065525in}}{\pgfqpoint{1.830293in}{3.057625in}}{\pgfqpoint{1.836117in}{3.051801in}}%
\pgfpathcurveto{\pgfqpoint{1.841941in}{3.045977in}}{\pgfqpoint{1.849841in}{3.042705in}}{\pgfqpoint{1.858077in}{3.042705in}}%
\pgfpathclose%
\pgfusepath{stroke,fill}%
\end{pgfscope}%
\begin{pgfscope}%
\pgfpathrectangle{\pgfqpoint{0.100000in}{0.212622in}}{\pgfqpoint{3.696000in}{3.696000in}}%
\pgfusepath{clip}%
\pgfsetbuttcap%
\pgfsetroundjoin%
\definecolor{currentfill}{rgb}{0.121569,0.466667,0.705882}%
\pgfsetfillcolor{currentfill}%
\pgfsetfillopacity{0.632143}%
\pgfsetlinewidth{1.003750pt}%
\definecolor{currentstroke}{rgb}{0.121569,0.466667,0.705882}%
\pgfsetstrokecolor{currentstroke}%
\pgfsetstrokeopacity{0.632143}%
\pgfsetdash{}{0pt}%
\pgfpathmoveto{\pgfqpoint{1.633295in}{2.856312in}}%
\pgfpathcurveto{\pgfqpoint{1.641532in}{2.856312in}}{\pgfqpoint{1.649432in}{2.859584in}}{\pgfqpoint{1.655256in}{2.865408in}}%
\pgfpathcurveto{\pgfqpoint{1.661080in}{2.871232in}}{\pgfqpoint{1.664352in}{2.879132in}}{\pgfqpoint{1.664352in}{2.887369in}}%
\pgfpathcurveto{\pgfqpoint{1.664352in}{2.895605in}}{\pgfqpoint{1.661080in}{2.903505in}}{\pgfqpoint{1.655256in}{2.909329in}}%
\pgfpathcurveto{\pgfqpoint{1.649432in}{2.915153in}}{\pgfqpoint{1.641532in}{2.918425in}}{\pgfqpoint{1.633295in}{2.918425in}}%
\pgfpathcurveto{\pgfqpoint{1.625059in}{2.918425in}}{\pgfqpoint{1.617159in}{2.915153in}}{\pgfqpoint{1.611335in}{2.909329in}}%
\pgfpathcurveto{\pgfqpoint{1.605511in}{2.903505in}}{\pgfqpoint{1.602239in}{2.895605in}}{\pgfqpoint{1.602239in}{2.887369in}}%
\pgfpathcurveto{\pgfqpoint{1.602239in}{2.879132in}}{\pgfqpoint{1.605511in}{2.871232in}}{\pgfqpoint{1.611335in}{2.865408in}}%
\pgfpathcurveto{\pgfqpoint{1.617159in}{2.859584in}}{\pgfqpoint{1.625059in}{2.856312in}}{\pgfqpoint{1.633295in}{2.856312in}}%
\pgfpathclose%
\pgfusepath{stroke,fill}%
\end{pgfscope}%
\begin{pgfscope}%
\pgfpathrectangle{\pgfqpoint{0.100000in}{0.212622in}}{\pgfqpoint{3.696000in}{3.696000in}}%
\pgfusepath{clip}%
\pgfsetbuttcap%
\pgfsetroundjoin%
\definecolor{currentfill}{rgb}{0.121569,0.466667,0.705882}%
\pgfsetfillcolor{currentfill}%
\pgfsetfillopacity{0.632852}%
\pgfsetlinewidth{1.003750pt}%
\definecolor{currentstroke}{rgb}{0.121569,0.466667,0.705882}%
\pgfsetstrokecolor{currentstroke}%
\pgfsetstrokeopacity{0.632852}%
\pgfsetdash{}{0pt}%
\pgfpathmoveto{\pgfqpoint{1.630898in}{2.851538in}}%
\pgfpathcurveto{\pgfqpoint{1.639135in}{2.851538in}}{\pgfqpoint{1.647035in}{2.854811in}}{\pgfqpoint{1.652859in}{2.860635in}}%
\pgfpathcurveto{\pgfqpoint{1.658683in}{2.866459in}}{\pgfqpoint{1.661955in}{2.874359in}}{\pgfqpoint{1.661955in}{2.882595in}}%
\pgfpathcurveto{\pgfqpoint{1.661955in}{2.890831in}}{\pgfqpoint{1.658683in}{2.898731in}}{\pgfqpoint{1.652859in}{2.904555in}}%
\pgfpathcurveto{\pgfqpoint{1.647035in}{2.910379in}}{\pgfqpoint{1.639135in}{2.913651in}}{\pgfqpoint{1.630898in}{2.913651in}}%
\pgfpathcurveto{\pgfqpoint{1.622662in}{2.913651in}}{\pgfqpoint{1.614762in}{2.910379in}}{\pgfqpoint{1.608938in}{2.904555in}}%
\pgfpathcurveto{\pgfqpoint{1.603114in}{2.898731in}}{\pgfqpoint{1.599842in}{2.890831in}}{\pgfqpoint{1.599842in}{2.882595in}}%
\pgfpathcurveto{\pgfqpoint{1.599842in}{2.874359in}}{\pgfqpoint{1.603114in}{2.866459in}}{\pgfqpoint{1.608938in}{2.860635in}}%
\pgfpathcurveto{\pgfqpoint{1.614762in}{2.854811in}}{\pgfqpoint{1.622662in}{2.851538in}}{\pgfqpoint{1.630898in}{2.851538in}}%
\pgfpathclose%
\pgfusepath{stroke,fill}%
\end{pgfscope}%
\begin{pgfscope}%
\pgfpathrectangle{\pgfqpoint{0.100000in}{0.212622in}}{\pgfqpoint{3.696000in}{3.696000in}}%
\pgfusepath{clip}%
\pgfsetbuttcap%
\pgfsetroundjoin%
\definecolor{currentfill}{rgb}{0.121569,0.466667,0.705882}%
\pgfsetfillcolor{currentfill}%
\pgfsetfillopacity{0.633281}%
\pgfsetlinewidth{1.003750pt}%
\definecolor{currentstroke}{rgb}{0.121569,0.466667,0.705882}%
\pgfsetstrokecolor{currentstroke}%
\pgfsetstrokeopacity{0.633281}%
\pgfsetdash{}{0pt}%
\pgfpathmoveto{\pgfqpoint{1.629474in}{2.849231in}}%
\pgfpathcurveto{\pgfqpoint{1.637710in}{2.849231in}}{\pgfqpoint{1.645610in}{2.852503in}}{\pgfqpoint{1.651434in}{2.858327in}}%
\pgfpathcurveto{\pgfqpoint{1.657258in}{2.864151in}}{\pgfqpoint{1.660531in}{2.872051in}}{\pgfqpoint{1.660531in}{2.880288in}}%
\pgfpathcurveto{\pgfqpoint{1.660531in}{2.888524in}}{\pgfqpoint{1.657258in}{2.896424in}}{\pgfqpoint{1.651434in}{2.902248in}}%
\pgfpathcurveto{\pgfqpoint{1.645610in}{2.908072in}}{\pgfqpoint{1.637710in}{2.911344in}}{\pgfqpoint{1.629474in}{2.911344in}}%
\pgfpathcurveto{\pgfqpoint{1.621238in}{2.911344in}}{\pgfqpoint{1.613338in}{2.908072in}}{\pgfqpoint{1.607514in}{2.902248in}}%
\pgfpathcurveto{\pgfqpoint{1.601690in}{2.896424in}}{\pgfqpoint{1.598418in}{2.888524in}}{\pgfqpoint{1.598418in}{2.880288in}}%
\pgfpathcurveto{\pgfqpoint{1.598418in}{2.872051in}}{\pgfqpoint{1.601690in}{2.864151in}}{\pgfqpoint{1.607514in}{2.858327in}}%
\pgfpathcurveto{\pgfqpoint{1.613338in}{2.852503in}}{\pgfqpoint{1.621238in}{2.849231in}}{\pgfqpoint{1.629474in}{2.849231in}}%
\pgfpathclose%
\pgfusepath{stroke,fill}%
\end{pgfscope}%
\begin{pgfscope}%
\pgfpathrectangle{\pgfqpoint{0.100000in}{0.212622in}}{\pgfqpoint{3.696000in}{3.696000in}}%
\pgfusepath{clip}%
\pgfsetbuttcap%
\pgfsetroundjoin%
\definecolor{currentfill}{rgb}{0.121569,0.466667,0.705882}%
\pgfsetfillcolor{currentfill}%
\pgfsetfillopacity{0.633965}%
\pgfsetlinewidth{1.003750pt}%
\definecolor{currentstroke}{rgb}{0.121569,0.466667,0.705882}%
\pgfsetstrokecolor{currentstroke}%
\pgfsetstrokeopacity{0.633965}%
\pgfsetdash{}{0pt}%
\pgfpathmoveto{\pgfqpoint{1.627804in}{2.843841in}}%
\pgfpathcurveto{\pgfqpoint{1.636040in}{2.843841in}}{\pgfqpoint{1.643940in}{2.847113in}}{\pgfqpoint{1.649764in}{2.852937in}}%
\pgfpathcurveto{\pgfqpoint{1.655588in}{2.858761in}}{\pgfqpoint{1.658861in}{2.866661in}}{\pgfqpoint{1.658861in}{2.874897in}}%
\pgfpathcurveto{\pgfqpoint{1.658861in}{2.883134in}}{\pgfqpoint{1.655588in}{2.891034in}}{\pgfqpoint{1.649764in}{2.896857in}}%
\pgfpathcurveto{\pgfqpoint{1.643940in}{2.902681in}}{\pgfqpoint{1.636040in}{2.905954in}}{\pgfqpoint{1.627804in}{2.905954in}}%
\pgfpathcurveto{\pgfqpoint{1.619568in}{2.905954in}}{\pgfqpoint{1.611668in}{2.902681in}}{\pgfqpoint{1.605844in}{2.896857in}}%
\pgfpathcurveto{\pgfqpoint{1.600020in}{2.891034in}}{\pgfqpoint{1.596748in}{2.883134in}}{\pgfqpoint{1.596748in}{2.874897in}}%
\pgfpathcurveto{\pgfqpoint{1.596748in}{2.866661in}}{\pgfqpoint{1.600020in}{2.858761in}}{\pgfqpoint{1.605844in}{2.852937in}}%
\pgfpathcurveto{\pgfqpoint{1.611668in}{2.847113in}}{\pgfqpoint{1.619568in}{2.843841in}}{\pgfqpoint{1.627804in}{2.843841in}}%
\pgfpathclose%
\pgfusepath{stroke,fill}%
\end{pgfscope}%
\begin{pgfscope}%
\pgfpathrectangle{\pgfqpoint{0.100000in}{0.212622in}}{\pgfqpoint{3.696000in}{3.696000in}}%
\pgfusepath{clip}%
\pgfsetbuttcap%
\pgfsetroundjoin%
\definecolor{currentfill}{rgb}{0.121569,0.466667,0.705882}%
\pgfsetfillcolor{currentfill}%
\pgfsetfillopacity{0.634286}%
\pgfsetlinewidth{1.003750pt}%
\definecolor{currentstroke}{rgb}{0.121569,0.466667,0.705882}%
\pgfsetstrokecolor{currentstroke}%
\pgfsetstrokeopacity{0.634286}%
\pgfsetdash{}{0pt}%
\pgfpathmoveto{\pgfqpoint{1.626791in}{2.840670in}}%
\pgfpathcurveto{\pgfqpoint{1.635027in}{2.840670in}}{\pgfqpoint{1.642927in}{2.843942in}}{\pgfqpoint{1.648751in}{2.849766in}}%
\pgfpathcurveto{\pgfqpoint{1.654575in}{2.855590in}}{\pgfqpoint{1.657847in}{2.863490in}}{\pgfqpoint{1.657847in}{2.871726in}}%
\pgfpathcurveto{\pgfqpoint{1.657847in}{2.879963in}}{\pgfqpoint{1.654575in}{2.887863in}}{\pgfqpoint{1.648751in}{2.893687in}}%
\pgfpathcurveto{\pgfqpoint{1.642927in}{2.899511in}}{\pgfqpoint{1.635027in}{2.902783in}}{\pgfqpoint{1.626791in}{2.902783in}}%
\pgfpathcurveto{\pgfqpoint{1.618554in}{2.902783in}}{\pgfqpoint{1.610654in}{2.899511in}}{\pgfqpoint{1.604831in}{2.893687in}}%
\pgfpathcurveto{\pgfqpoint{1.599007in}{2.887863in}}{\pgfqpoint{1.595734in}{2.879963in}}{\pgfqpoint{1.595734in}{2.871726in}}%
\pgfpathcurveto{\pgfqpoint{1.595734in}{2.863490in}}{\pgfqpoint{1.599007in}{2.855590in}}{\pgfqpoint{1.604831in}{2.849766in}}%
\pgfpathcurveto{\pgfqpoint{1.610654in}{2.843942in}}{\pgfqpoint{1.618554in}{2.840670in}}{\pgfqpoint{1.626791in}{2.840670in}}%
\pgfpathclose%
\pgfusepath{stroke,fill}%
\end{pgfscope}%
\begin{pgfscope}%
\pgfpathrectangle{\pgfqpoint{0.100000in}{0.212622in}}{\pgfqpoint{3.696000in}{3.696000in}}%
\pgfusepath{clip}%
\pgfsetbuttcap%
\pgfsetroundjoin%
\definecolor{currentfill}{rgb}{0.121569,0.466667,0.705882}%
\pgfsetfillcolor{currentfill}%
\pgfsetfillopacity{0.634694}%
\pgfsetlinewidth{1.003750pt}%
\definecolor{currentstroke}{rgb}{0.121569,0.466667,0.705882}%
\pgfsetstrokecolor{currentstroke}%
\pgfsetstrokeopacity{0.634694}%
\pgfsetdash{}{0pt}%
\pgfpathmoveto{\pgfqpoint{1.624673in}{2.836795in}}%
\pgfpathcurveto{\pgfqpoint{1.632909in}{2.836795in}}{\pgfqpoint{1.640809in}{2.840067in}}{\pgfqpoint{1.646633in}{2.845891in}}%
\pgfpathcurveto{\pgfqpoint{1.652457in}{2.851715in}}{\pgfqpoint{1.655729in}{2.859615in}}{\pgfqpoint{1.655729in}{2.867852in}}%
\pgfpathcurveto{\pgfqpoint{1.655729in}{2.876088in}}{\pgfqpoint{1.652457in}{2.883988in}}{\pgfqpoint{1.646633in}{2.889812in}}%
\pgfpathcurveto{\pgfqpoint{1.640809in}{2.895636in}}{\pgfqpoint{1.632909in}{2.898908in}}{\pgfqpoint{1.624673in}{2.898908in}}%
\pgfpathcurveto{\pgfqpoint{1.616437in}{2.898908in}}{\pgfqpoint{1.608537in}{2.895636in}}{\pgfqpoint{1.602713in}{2.889812in}}%
\pgfpathcurveto{\pgfqpoint{1.596889in}{2.883988in}}{\pgfqpoint{1.593616in}{2.876088in}}{\pgfqpoint{1.593616in}{2.867852in}}%
\pgfpathcurveto{\pgfqpoint{1.593616in}{2.859615in}}{\pgfqpoint{1.596889in}{2.851715in}}{\pgfqpoint{1.602713in}{2.845891in}}%
\pgfpathcurveto{\pgfqpoint{1.608537in}{2.840067in}}{\pgfqpoint{1.616437in}{2.836795in}}{\pgfqpoint{1.624673in}{2.836795in}}%
\pgfpathclose%
\pgfusepath{stroke,fill}%
\end{pgfscope}%
\begin{pgfscope}%
\pgfpathrectangle{\pgfqpoint{0.100000in}{0.212622in}}{\pgfqpoint{3.696000in}{3.696000in}}%
\pgfusepath{clip}%
\pgfsetbuttcap%
\pgfsetroundjoin%
\definecolor{currentfill}{rgb}{0.121569,0.466667,0.705882}%
\pgfsetfillcolor{currentfill}%
\pgfsetfillopacity{0.635187}%
\pgfsetlinewidth{1.003750pt}%
\definecolor{currentstroke}{rgb}{0.121569,0.466667,0.705882}%
\pgfsetstrokecolor{currentstroke}%
\pgfsetstrokeopacity{0.635187}%
\pgfsetdash{}{0pt}%
\pgfpathmoveto{\pgfqpoint{1.622203in}{2.832963in}}%
\pgfpathcurveto{\pgfqpoint{1.630439in}{2.832963in}}{\pgfqpoint{1.638339in}{2.836236in}}{\pgfqpoint{1.644163in}{2.842059in}}%
\pgfpathcurveto{\pgfqpoint{1.649987in}{2.847883in}}{\pgfqpoint{1.653259in}{2.855783in}}{\pgfqpoint{1.653259in}{2.864020in}}%
\pgfpathcurveto{\pgfqpoint{1.653259in}{2.872256in}}{\pgfqpoint{1.649987in}{2.880156in}}{\pgfqpoint{1.644163in}{2.885980in}}%
\pgfpathcurveto{\pgfqpoint{1.638339in}{2.891804in}}{\pgfqpoint{1.630439in}{2.895076in}}{\pgfqpoint{1.622203in}{2.895076in}}%
\pgfpathcurveto{\pgfqpoint{1.613966in}{2.895076in}}{\pgfqpoint{1.606066in}{2.891804in}}{\pgfqpoint{1.600242in}{2.885980in}}%
\pgfpathcurveto{\pgfqpoint{1.594419in}{2.880156in}}{\pgfqpoint{1.591146in}{2.872256in}}{\pgfqpoint{1.591146in}{2.864020in}}%
\pgfpathcurveto{\pgfqpoint{1.591146in}{2.855783in}}{\pgfqpoint{1.594419in}{2.847883in}}{\pgfqpoint{1.600242in}{2.842059in}}%
\pgfpathcurveto{\pgfqpoint{1.606066in}{2.836236in}}{\pgfqpoint{1.613966in}{2.832963in}}{\pgfqpoint{1.622203in}{2.832963in}}%
\pgfpathclose%
\pgfusepath{stroke,fill}%
\end{pgfscope}%
\begin{pgfscope}%
\pgfpathrectangle{\pgfqpoint{0.100000in}{0.212622in}}{\pgfqpoint{3.696000in}{3.696000in}}%
\pgfusepath{clip}%
\pgfsetbuttcap%
\pgfsetroundjoin%
\definecolor{currentfill}{rgb}{0.121569,0.466667,0.705882}%
\pgfsetfillcolor{currentfill}%
\pgfsetfillopacity{0.635243}%
\pgfsetlinewidth{1.003750pt}%
\definecolor{currentstroke}{rgb}{0.121569,0.466667,0.705882}%
\pgfsetstrokecolor{currentstroke}%
\pgfsetstrokeopacity{0.635243}%
\pgfsetdash{}{0pt}%
\pgfpathmoveto{\pgfqpoint{3.052182in}{2.626332in}}%
\pgfpathcurveto{\pgfqpoint{3.060418in}{2.626332in}}{\pgfqpoint{3.068318in}{2.629604in}}{\pgfqpoint{3.074142in}{2.635428in}}%
\pgfpathcurveto{\pgfqpoint{3.079966in}{2.641252in}}{\pgfqpoint{3.083238in}{2.649152in}}{\pgfqpoint{3.083238in}{2.657388in}}%
\pgfpathcurveto{\pgfqpoint{3.083238in}{2.665624in}}{\pgfqpoint{3.079966in}{2.673524in}}{\pgfqpoint{3.074142in}{2.679348in}}%
\pgfpathcurveto{\pgfqpoint{3.068318in}{2.685172in}}{\pgfqpoint{3.060418in}{2.688445in}}{\pgfqpoint{3.052182in}{2.688445in}}%
\pgfpathcurveto{\pgfqpoint{3.043945in}{2.688445in}}{\pgfqpoint{3.036045in}{2.685172in}}{\pgfqpoint{3.030221in}{2.679348in}}%
\pgfpathcurveto{\pgfqpoint{3.024397in}{2.673524in}}{\pgfqpoint{3.021125in}{2.665624in}}{\pgfqpoint{3.021125in}{2.657388in}}%
\pgfpathcurveto{\pgfqpoint{3.021125in}{2.649152in}}{\pgfqpoint{3.024397in}{2.641252in}}{\pgfqpoint{3.030221in}{2.635428in}}%
\pgfpathcurveto{\pgfqpoint{3.036045in}{2.629604in}}{\pgfqpoint{3.043945in}{2.626332in}}{\pgfqpoint{3.052182in}{2.626332in}}%
\pgfpathclose%
\pgfusepath{stroke,fill}%
\end{pgfscope}%
\begin{pgfscope}%
\pgfpathrectangle{\pgfqpoint{0.100000in}{0.212622in}}{\pgfqpoint{3.696000in}{3.696000in}}%
\pgfusepath{clip}%
\pgfsetbuttcap%
\pgfsetroundjoin%
\definecolor{currentfill}{rgb}{0.121569,0.466667,0.705882}%
\pgfsetfillcolor{currentfill}%
\pgfsetfillopacity{0.635341}%
\pgfsetlinewidth{1.003750pt}%
\definecolor{currentstroke}{rgb}{0.121569,0.466667,0.705882}%
\pgfsetstrokecolor{currentstroke}%
\pgfsetstrokeopacity{0.635341}%
\pgfsetdash{}{0pt}%
\pgfpathmoveto{\pgfqpoint{1.869810in}{3.042868in}}%
\pgfpathcurveto{\pgfqpoint{1.878046in}{3.042868in}}{\pgfqpoint{1.885946in}{3.046141in}}{\pgfqpoint{1.891770in}{3.051965in}}%
\pgfpathcurveto{\pgfqpoint{1.897594in}{3.057789in}}{\pgfqpoint{1.900867in}{3.065689in}}{\pgfqpoint{1.900867in}{3.073925in}}%
\pgfpathcurveto{\pgfqpoint{1.900867in}{3.082161in}}{\pgfqpoint{1.897594in}{3.090061in}}{\pgfqpoint{1.891770in}{3.095885in}}%
\pgfpathcurveto{\pgfqpoint{1.885946in}{3.101709in}}{\pgfqpoint{1.878046in}{3.104981in}}{\pgfqpoint{1.869810in}{3.104981in}}%
\pgfpathcurveto{\pgfqpoint{1.861574in}{3.104981in}}{\pgfqpoint{1.853674in}{3.101709in}}{\pgfqpoint{1.847850in}{3.095885in}}%
\pgfpathcurveto{\pgfqpoint{1.842026in}{3.090061in}}{\pgfqpoint{1.838754in}{3.082161in}}{\pgfqpoint{1.838754in}{3.073925in}}%
\pgfpathcurveto{\pgfqpoint{1.838754in}{3.065689in}}{\pgfqpoint{1.842026in}{3.057789in}}{\pgfqpoint{1.847850in}{3.051965in}}%
\pgfpathcurveto{\pgfqpoint{1.853674in}{3.046141in}}{\pgfqpoint{1.861574in}{3.042868in}}{\pgfqpoint{1.869810in}{3.042868in}}%
\pgfpathclose%
\pgfusepath{stroke,fill}%
\end{pgfscope}%
\begin{pgfscope}%
\pgfpathrectangle{\pgfqpoint{0.100000in}{0.212622in}}{\pgfqpoint{3.696000in}{3.696000in}}%
\pgfusepath{clip}%
\pgfsetbuttcap%
\pgfsetroundjoin%
\definecolor{currentfill}{rgb}{0.121569,0.466667,0.705882}%
\pgfsetfillcolor{currentfill}%
\pgfsetfillopacity{0.636020}%
\pgfsetlinewidth{1.003750pt}%
\definecolor{currentstroke}{rgb}{0.121569,0.466667,0.705882}%
\pgfsetstrokecolor{currentstroke}%
\pgfsetstrokeopacity{0.636020}%
\pgfsetdash{}{0pt}%
\pgfpathmoveto{\pgfqpoint{1.619554in}{2.826371in}}%
\pgfpathcurveto{\pgfqpoint{1.627791in}{2.826371in}}{\pgfqpoint{1.635691in}{2.829643in}}{\pgfqpoint{1.641515in}{2.835467in}}%
\pgfpathcurveto{\pgfqpoint{1.647339in}{2.841291in}}{\pgfqpoint{1.650611in}{2.849191in}}{\pgfqpoint{1.650611in}{2.857427in}}%
\pgfpathcurveto{\pgfqpoint{1.650611in}{2.865664in}}{\pgfqpoint{1.647339in}{2.873564in}}{\pgfqpoint{1.641515in}{2.879388in}}%
\pgfpathcurveto{\pgfqpoint{1.635691in}{2.885212in}}{\pgfqpoint{1.627791in}{2.888484in}}{\pgfqpoint{1.619554in}{2.888484in}}%
\pgfpathcurveto{\pgfqpoint{1.611318in}{2.888484in}}{\pgfqpoint{1.603418in}{2.885212in}}{\pgfqpoint{1.597594in}{2.879388in}}%
\pgfpathcurveto{\pgfqpoint{1.591770in}{2.873564in}}{\pgfqpoint{1.588498in}{2.865664in}}{\pgfqpoint{1.588498in}{2.857427in}}%
\pgfpathcurveto{\pgfqpoint{1.588498in}{2.849191in}}{\pgfqpoint{1.591770in}{2.841291in}}{\pgfqpoint{1.597594in}{2.835467in}}%
\pgfpathcurveto{\pgfqpoint{1.603418in}{2.829643in}}{\pgfqpoint{1.611318in}{2.826371in}}{\pgfqpoint{1.619554in}{2.826371in}}%
\pgfpathclose%
\pgfusepath{stroke,fill}%
\end{pgfscope}%
\begin{pgfscope}%
\pgfpathrectangle{\pgfqpoint{0.100000in}{0.212622in}}{\pgfqpoint{3.696000in}{3.696000in}}%
\pgfusepath{clip}%
\pgfsetbuttcap%
\pgfsetroundjoin%
\definecolor{currentfill}{rgb}{0.121569,0.466667,0.705882}%
\pgfsetfillcolor{currentfill}%
\pgfsetfillopacity{0.636473}%
\pgfsetlinewidth{1.003750pt}%
\definecolor{currentstroke}{rgb}{0.121569,0.466667,0.705882}%
\pgfsetstrokecolor{currentstroke}%
\pgfsetstrokeopacity{0.636473}%
\pgfsetdash{}{0pt}%
\pgfpathmoveto{\pgfqpoint{1.618575in}{2.822463in}}%
\pgfpathcurveto{\pgfqpoint{1.626812in}{2.822463in}}{\pgfqpoint{1.634712in}{2.825735in}}{\pgfqpoint{1.640536in}{2.831559in}}%
\pgfpathcurveto{\pgfqpoint{1.646359in}{2.837383in}}{\pgfqpoint{1.649632in}{2.845283in}}{\pgfqpoint{1.649632in}{2.853519in}}%
\pgfpathcurveto{\pgfqpoint{1.649632in}{2.861756in}}{\pgfqpoint{1.646359in}{2.869656in}}{\pgfqpoint{1.640536in}{2.875480in}}%
\pgfpathcurveto{\pgfqpoint{1.634712in}{2.881303in}}{\pgfqpoint{1.626812in}{2.884576in}}{\pgfqpoint{1.618575in}{2.884576in}}%
\pgfpathcurveto{\pgfqpoint{1.610339in}{2.884576in}}{\pgfqpoint{1.602439in}{2.881303in}}{\pgfqpoint{1.596615in}{2.875480in}}%
\pgfpathcurveto{\pgfqpoint{1.590791in}{2.869656in}}{\pgfqpoint{1.587519in}{2.861756in}}{\pgfqpoint{1.587519in}{2.853519in}}%
\pgfpathcurveto{\pgfqpoint{1.587519in}{2.845283in}}{\pgfqpoint{1.590791in}{2.837383in}}{\pgfqpoint{1.596615in}{2.831559in}}%
\pgfpathcurveto{\pgfqpoint{1.602439in}{2.825735in}}{\pgfqpoint{1.610339in}{2.822463in}}{\pgfqpoint{1.618575in}{2.822463in}}%
\pgfpathclose%
\pgfusepath{stroke,fill}%
\end{pgfscope}%
\begin{pgfscope}%
\pgfpathrectangle{\pgfqpoint{0.100000in}{0.212622in}}{\pgfqpoint{3.696000in}{3.696000in}}%
\pgfusepath{clip}%
\pgfsetbuttcap%
\pgfsetroundjoin%
\definecolor{currentfill}{rgb}{0.121569,0.466667,0.705882}%
\pgfsetfillcolor{currentfill}%
\pgfsetfillopacity{0.637096}%
\pgfsetlinewidth{1.003750pt}%
\definecolor{currentstroke}{rgb}{0.121569,0.466667,0.705882}%
\pgfsetstrokecolor{currentstroke}%
\pgfsetstrokeopacity{0.637096}%
\pgfsetdash{}{0pt}%
\pgfpathmoveto{\pgfqpoint{1.616651in}{2.818735in}}%
\pgfpathcurveto{\pgfqpoint{1.624888in}{2.818735in}}{\pgfqpoint{1.632788in}{2.822007in}}{\pgfqpoint{1.638612in}{2.827831in}}%
\pgfpathcurveto{\pgfqpoint{1.644436in}{2.833655in}}{\pgfqpoint{1.647708in}{2.841555in}}{\pgfqpoint{1.647708in}{2.849791in}}%
\pgfpathcurveto{\pgfqpoint{1.647708in}{2.858027in}}{\pgfqpoint{1.644436in}{2.865927in}}{\pgfqpoint{1.638612in}{2.871751in}}%
\pgfpathcurveto{\pgfqpoint{1.632788in}{2.877575in}}{\pgfqpoint{1.624888in}{2.880848in}}{\pgfqpoint{1.616651in}{2.880848in}}%
\pgfpathcurveto{\pgfqpoint{1.608415in}{2.880848in}}{\pgfqpoint{1.600515in}{2.877575in}}{\pgfqpoint{1.594691in}{2.871751in}}%
\pgfpathcurveto{\pgfqpoint{1.588867in}{2.865927in}}{\pgfqpoint{1.585595in}{2.858027in}}{\pgfqpoint{1.585595in}{2.849791in}}%
\pgfpathcurveto{\pgfqpoint{1.585595in}{2.841555in}}{\pgfqpoint{1.588867in}{2.833655in}}{\pgfqpoint{1.594691in}{2.827831in}}%
\pgfpathcurveto{\pgfqpoint{1.600515in}{2.822007in}}{\pgfqpoint{1.608415in}{2.818735in}}{\pgfqpoint{1.616651in}{2.818735in}}%
\pgfpathclose%
\pgfusepath{stroke,fill}%
\end{pgfscope}%
\begin{pgfscope}%
\pgfpathrectangle{\pgfqpoint{0.100000in}{0.212622in}}{\pgfqpoint{3.696000in}{3.696000in}}%
\pgfusepath{clip}%
\pgfsetbuttcap%
\pgfsetroundjoin%
\definecolor{currentfill}{rgb}{0.121569,0.466667,0.705882}%
\pgfsetfillcolor{currentfill}%
\pgfsetfillopacity{0.637508}%
\pgfsetlinewidth{1.003750pt}%
\definecolor{currentstroke}{rgb}{0.121569,0.466667,0.705882}%
\pgfsetstrokecolor{currentstroke}%
\pgfsetstrokeopacity{0.637508}%
\pgfsetdash{}{0pt}%
\pgfpathmoveto{\pgfqpoint{1.615534in}{2.817077in}}%
\pgfpathcurveto{\pgfqpoint{1.623770in}{2.817077in}}{\pgfqpoint{1.631670in}{2.820349in}}{\pgfqpoint{1.637494in}{2.826173in}}%
\pgfpathcurveto{\pgfqpoint{1.643318in}{2.831997in}}{\pgfqpoint{1.646590in}{2.839897in}}{\pgfqpoint{1.646590in}{2.848134in}}%
\pgfpathcurveto{\pgfqpoint{1.646590in}{2.856370in}}{\pgfqpoint{1.643318in}{2.864270in}}{\pgfqpoint{1.637494in}{2.870094in}}%
\pgfpathcurveto{\pgfqpoint{1.631670in}{2.875918in}}{\pgfqpoint{1.623770in}{2.879190in}}{\pgfqpoint{1.615534in}{2.879190in}}%
\pgfpathcurveto{\pgfqpoint{1.607298in}{2.879190in}}{\pgfqpoint{1.599398in}{2.875918in}}{\pgfqpoint{1.593574in}{2.870094in}}%
\pgfpathcurveto{\pgfqpoint{1.587750in}{2.864270in}}{\pgfqpoint{1.584477in}{2.856370in}}{\pgfqpoint{1.584477in}{2.848134in}}%
\pgfpathcurveto{\pgfqpoint{1.584477in}{2.839897in}}{\pgfqpoint{1.587750in}{2.831997in}}{\pgfqpoint{1.593574in}{2.826173in}}%
\pgfpathcurveto{\pgfqpoint{1.599398in}{2.820349in}}{\pgfqpoint{1.607298in}{2.817077in}}{\pgfqpoint{1.615534in}{2.817077in}}%
\pgfpathclose%
\pgfusepath{stroke,fill}%
\end{pgfscope}%
\begin{pgfscope}%
\pgfpathrectangle{\pgfqpoint{0.100000in}{0.212622in}}{\pgfqpoint{3.696000in}{3.696000in}}%
\pgfusepath{clip}%
\pgfsetbuttcap%
\pgfsetroundjoin%
\definecolor{currentfill}{rgb}{0.121569,0.466667,0.705882}%
\pgfsetfillcolor{currentfill}%
\pgfsetfillopacity{0.637842}%
\pgfsetlinewidth{1.003750pt}%
\definecolor{currentstroke}{rgb}{0.121569,0.466667,0.705882}%
\pgfsetstrokecolor{currentstroke}%
\pgfsetstrokeopacity{0.637842}%
\pgfsetdash{}{0pt}%
\pgfpathmoveto{\pgfqpoint{3.063409in}{2.625812in}}%
\pgfpathcurveto{\pgfqpoint{3.071645in}{2.625812in}}{\pgfqpoint{3.079545in}{2.629085in}}{\pgfqpoint{3.085369in}{2.634909in}}%
\pgfpathcurveto{\pgfqpoint{3.091193in}{2.640733in}}{\pgfqpoint{3.094465in}{2.648633in}}{\pgfqpoint{3.094465in}{2.656869in}}%
\pgfpathcurveto{\pgfqpoint{3.094465in}{2.665105in}}{\pgfqpoint{3.091193in}{2.673005in}}{\pgfqpoint{3.085369in}{2.678829in}}%
\pgfpathcurveto{\pgfqpoint{3.079545in}{2.684653in}}{\pgfqpoint{3.071645in}{2.687925in}}{\pgfqpoint{3.063409in}{2.687925in}}%
\pgfpathcurveto{\pgfqpoint{3.055172in}{2.687925in}}{\pgfqpoint{3.047272in}{2.684653in}}{\pgfqpoint{3.041448in}{2.678829in}}%
\pgfpathcurveto{\pgfqpoint{3.035624in}{2.673005in}}{\pgfqpoint{3.032352in}{2.665105in}}{\pgfqpoint{3.032352in}{2.656869in}}%
\pgfpathcurveto{\pgfqpoint{3.032352in}{2.648633in}}{\pgfqpoint{3.035624in}{2.640733in}}{\pgfqpoint{3.041448in}{2.634909in}}%
\pgfpathcurveto{\pgfqpoint{3.047272in}{2.629085in}}{\pgfqpoint{3.055172in}{2.625812in}}{\pgfqpoint{3.063409in}{2.625812in}}%
\pgfpathclose%
\pgfusepath{stroke,fill}%
\end{pgfscope}%
\begin{pgfscope}%
\pgfpathrectangle{\pgfqpoint{0.100000in}{0.212622in}}{\pgfqpoint{3.696000in}{3.696000in}}%
\pgfusepath{clip}%
\pgfsetbuttcap%
\pgfsetroundjoin%
\definecolor{currentfill}{rgb}{0.121569,0.466667,0.705882}%
\pgfsetfillcolor{currentfill}%
\pgfsetfillopacity{0.638126}%
\pgfsetlinewidth{1.003750pt}%
\definecolor{currentstroke}{rgb}{0.121569,0.466667,0.705882}%
\pgfsetstrokecolor{currentstroke}%
\pgfsetstrokeopacity{0.638126}%
\pgfsetdash{}{0pt}%
\pgfpathmoveto{\pgfqpoint{1.615137in}{2.814018in}}%
\pgfpathcurveto{\pgfqpoint{1.623374in}{2.814018in}}{\pgfqpoint{1.631274in}{2.817290in}}{\pgfqpoint{1.637098in}{2.823114in}}%
\pgfpathcurveto{\pgfqpoint{1.642922in}{2.828938in}}{\pgfqpoint{1.646194in}{2.836838in}}{\pgfqpoint{1.646194in}{2.845074in}}%
\pgfpathcurveto{\pgfqpoint{1.646194in}{2.853310in}}{\pgfqpoint{1.642922in}{2.861210in}}{\pgfqpoint{1.637098in}{2.867034in}}%
\pgfpathcurveto{\pgfqpoint{1.631274in}{2.872858in}}{\pgfqpoint{1.623374in}{2.876131in}}{\pgfqpoint{1.615137in}{2.876131in}}%
\pgfpathcurveto{\pgfqpoint{1.606901in}{2.876131in}}{\pgfqpoint{1.599001in}{2.872858in}}{\pgfqpoint{1.593177in}{2.867034in}}%
\pgfpathcurveto{\pgfqpoint{1.587353in}{2.861210in}}{\pgfqpoint{1.584081in}{2.853310in}}{\pgfqpoint{1.584081in}{2.845074in}}%
\pgfpathcurveto{\pgfqpoint{1.584081in}{2.836838in}}{\pgfqpoint{1.587353in}{2.828938in}}{\pgfqpoint{1.593177in}{2.823114in}}%
\pgfpathcurveto{\pgfqpoint{1.599001in}{2.817290in}}{\pgfqpoint{1.606901in}{2.814018in}}{\pgfqpoint{1.615137in}{2.814018in}}%
\pgfpathclose%
\pgfusepath{stroke,fill}%
\end{pgfscope}%
\begin{pgfscope}%
\pgfpathrectangle{\pgfqpoint{0.100000in}{0.212622in}}{\pgfqpoint{3.696000in}{3.696000in}}%
\pgfusepath{clip}%
\pgfsetbuttcap%
\pgfsetroundjoin%
\definecolor{currentfill}{rgb}{0.121569,0.466667,0.705882}%
\pgfsetfillcolor{currentfill}%
\pgfsetfillopacity{0.638426}%
\pgfsetlinewidth{1.003750pt}%
\definecolor{currentstroke}{rgb}{0.121569,0.466667,0.705882}%
\pgfsetstrokecolor{currentstroke}%
\pgfsetstrokeopacity{0.638426}%
\pgfsetdash{}{0pt}%
\pgfpathmoveto{\pgfqpoint{1.881184in}{3.044011in}}%
\pgfpathcurveto{\pgfqpoint{1.889420in}{3.044011in}}{\pgfqpoint{1.897320in}{3.047284in}}{\pgfqpoint{1.903144in}{3.053107in}}%
\pgfpathcurveto{\pgfqpoint{1.908968in}{3.058931in}}{\pgfqpoint{1.912240in}{3.066831in}}{\pgfqpoint{1.912240in}{3.075068in}}%
\pgfpathcurveto{\pgfqpoint{1.912240in}{3.083304in}}{\pgfqpoint{1.908968in}{3.091204in}}{\pgfqpoint{1.903144in}{3.097028in}}%
\pgfpathcurveto{\pgfqpoint{1.897320in}{3.102852in}}{\pgfqpoint{1.889420in}{3.106124in}}{\pgfqpoint{1.881184in}{3.106124in}}%
\pgfpathcurveto{\pgfqpoint{1.872947in}{3.106124in}}{\pgfqpoint{1.865047in}{3.102852in}}{\pgfqpoint{1.859223in}{3.097028in}}%
\pgfpathcurveto{\pgfqpoint{1.853399in}{3.091204in}}{\pgfqpoint{1.850127in}{3.083304in}}{\pgfqpoint{1.850127in}{3.075068in}}%
\pgfpathcurveto{\pgfqpoint{1.850127in}{3.066831in}}{\pgfqpoint{1.853399in}{3.058931in}}{\pgfqpoint{1.859223in}{3.053107in}}%
\pgfpathcurveto{\pgfqpoint{1.865047in}{3.047284in}}{\pgfqpoint{1.872947in}{3.044011in}}{\pgfqpoint{1.881184in}{3.044011in}}%
\pgfpathclose%
\pgfusepath{stroke,fill}%
\end{pgfscope}%
\begin{pgfscope}%
\pgfpathrectangle{\pgfqpoint{0.100000in}{0.212622in}}{\pgfqpoint{3.696000in}{3.696000in}}%
\pgfusepath{clip}%
\pgfsetbuttcap%
\pgfsetroundjoin%
\definecolor{currentfill}{rgb}{0.121569,0.466667,0.705882}%
\pgfsetfillcolor{currentfill}%
\pgfsetfillopacity{0.638760}%
\pgfsetlinewidth{1.003750pt}%
\definecolor{currentstroke}{rgb}{0.121569,0.466667,0.705882}%
\pgfsetstrokecolor{currentstroke}%
\pgfsetstrokeopacity{0.638760}%
\pgfsetdash{}{0pt}%
\pgfpathmoveto{\pgfqpoint{1.615049in}{2.810157in}}%
\pgfpathcurveto{\pgfqpoint{1.623285in}{2.810157in}}{\pgfqpoint{1.631186in}{2.813430in}}{\pgfqpoint{1.637009in}{2.819253in}}%
\pgfpathcurveto{\pgfqpoint{1.642833in}{2.825077in}}{\pgfqpoint{1.646106in}{2.832977in}}{\pgfqpoint{1.646106in}{2.841214in}}%
\pgfpathcurveto{\pgfqpoint{1.646106in}{2.849450in}}{\pgfqpoint{1.642833in}{2.857350in}}{\pgfqpoint{1.637009in}{2.863174in}}%
\pgfpathcurveto{\pgfqpoint{1.631186in}{2.868998in}}{\pgfqpoint{1.623285in}{2.872270in}}{\pgfqpoint{1.615049in}{2.872270in}}%
\pgfpathcurveto{\pgfqpoint{1.606813in}{2.872270in}}{\pgfqpoint{1.598913in}{2.868998in}}{\pgfqpoint{1.593089in}{2.863174in}}%
\pgfpathcurveto{\pgfqpoint{1.587265in}{2.857350in}}{\pgfqpoint{1.583993in}{2.849450in}}{\pgfqpoint{1.583993in}{2.841214in}}%
\pgfpathcurveto{\pgfqpoint{1.583993in}{2.832977in}}{\pgfqpoint{1.587265in}{2.825077in}}{\pgfqpoint{1.593089in}{2.819253in}}%
\pgfpathcurveto{\pgfqpoint{1.598913in}{2.813430in}}{\pgfqpoint{1.606813in}{2.810157in}}{\pgfqpoint{1.615049in}{2.810157in}}%
\pgfpathclose%
\pgfusepath{stroke,fill}%
\end{pgfscope}%
\begin{pgfscope}%
\pgfpathrectangle{\pgfqpoint{0.100000in}{0.212622in}}{\pgfqpoint{3.696000in}{3.696000in}}%
\pgfusepath{clip}%
\pgfsetbuttcap%
\pgfsetroundjoin%
\definecolor{currentfill}{rgb}{0.121569,0.466667,0.705882}%
\pgfsetfillcolor{currentfill}%
\pgfsetfillopacity{0.639832}%
\pgfsetlinewidth{1.003750pt}%
\definecolor{currentstroke}{rgb}{0.121569,0.466667,0.705882}%
\pgfsetstrokecolor{currentstroke}%
\pgfsetstrokeopacity{0.639832}%
\pgfsetdash{}{0pt}%
\pgfpathmoveto{\pgfqpoint{1.613054in}{2.805474in}}%
\pgfpathcurveto{\pgfqpoint{1.621290in}{2.805474in}}{\pgfqpoint{1.629190in}{2.808746in}}{\pgfqpoint{1.635014in}{2.814570in}}%
\pgfpathcurveto{\pgfqpoint{1.640838in}{2.820394in}}{\pgfqpoint{1.644110in}{2.828294in}}{\pgfqpoint{1.644110in}{2.836530in}}%
\pgfpathcurveto{\pgfqpoint{1.644110in}{2.844766in}}{\pgfqpoint{1.640838in}{2.852667in}}{\pgfqpoint{1.635014in}{2.858490in}}%
\pgfpathcurveto{\pgfqpoint{1.629190in}{2.864314in}}{\pgfqpoint{1.621290in}{2.867587in}}{\pgfqpoint{1.613054in}{2.867587in}}%
\pgfpathcurveto{\pgfqpoint{1.604818in}{2.867587in}}{\pgfqpoint{1.596918in}{2.864314in}}{\pgfqpoint{1.591094in}{2.858490in}}%
\pgfpathcurveto{\pgfqpoint{1.585270in}{2.852667in}}{\pgfqpoint{1.581997in}{2.844766in}}{\pgfqpoint{1.581997in}{2.836530in}}%
\pgfpathcurveto{\pgfqpoint{1.581997in}{2.828294in}}{\pgfqpoint{1.585270in}{2.820394in}}{\pgfqpoint{1.591094in}{2.814570in}}%
\pgfpathcurveto{\pgfqpoint{1.596918in}{2.808746in}}{\pgfqpoint{1.604818in}{2.805474in}}{\pgfqpoint{1.613054in}{2.805474in}}%
\pgfpathclose%
\pgfusepath{stroke,fill}%
\end{pgfscope}%
\begin{pgfscope}%
\pgfpathrectangle{\pgfqpoint{0.100000in}{0.212622in}}{\pgfqpoint{3.696000in}{3.696000in}}%
\pgfusepath{clip}%
\pgfsetbuttcap%
\pgfsetroundjoin%
\definecolor{currentfill}{rgb}{0.121569,0.466667,0.705882}%
\pgfsetfillcolor{currentfill}%
\pgfsetfillopacity{0.640100}%
\pgfsetlinewidth{1.003750pt}%
\definecolor{currentstroke}{rgb}{0.121569,0.466667,0.705882}%
\pgfsetstrokecolor{currentstroke}%
\pgfsetstrokeopacity{0.640100}%
\pgfsetdash{}{0pt}%
\pgfpathmoveto{\pgfqpoint{1.892033in}{3.041275in}}%
\pgfpathcurveto{\pgfqpoint{1.900269in}{3.041275in}}{\pgfqpoint{1.908169in}{3.044547in}}{\pgfqpoint{1.913993in}{3.050371in}}%
\pgfpathcurveto{\pgfqpoint{1.919817in}{3.056195in}}{\pgfqpoint{1.923089in}{3.064095in}}{\pgfqpoint{1.923089in}{3.072331in}}%
\pgfpathcurveto{\pgfqpoint{1.923089in}{3.080568in}}{\pgfqpoint{1.919817in}{3.088468in}}{\pgfqpoint{1.913993in}{3.094292in}}%
\pgfpathcurveto{\pgfqpoint{1.908169in}{3.100116in}}{\pgfqpoint{1.900269in}{3.103388in}}{\pgfqpoint{1.892033in}{3.103388in}}%
\pgfpathcurveto{\pgfqpoint{1.883797in}{3.103388in}}{\pgfqpoint{1.875896in}{3.100116in}}{\pgfqpoint{1.870073in}{3.094292in}}%
\pgfpathcurveto{\pgfqpoint{1.864249in}{3.088468in}}{\pgfqpoint{1.860976in}{3.080568in}}{\pgfqpoint{1.860976in}{3.072331in}}%
\pgfpathcurveto{\pgfqpoint{1.860976in}{3.064095in}}{\pgfqpoint{1.864249in}{3.056195in}}{\pgfqpoint{1.870073in}{3.050371in}}%
\pgfpathcurveto{\pgfqpoint{1.875896in}{3.044547in}}{\pgfqpoint{1.883797in}{3.041275in}}{\pgfqpoint{1.892033in}{3.041275in}}%
\pgfpathclose%
\pgfusepath{stroke,fill}%
\end{pgfscope}%
\begin{pgfscope}%
\pgfpathrectangle{\pgfqpoint{0.100000in}{0.212622in}}{\pgfqpoint{3.696000in}{3.696000in}}%
\pgfusepath{clip}%
\pgfsetbuttcap%
\pgfsetroundjoin%
\definecolor{currentfill}{rgb}{0.121569,0.466667,0.705882}%
\pgfsetfillcolor{currentfill}%
\pgfsetfillopacity{0.640728}%
\pgfsetlinewidth{1.003750pt}%
\definecolor{currentstroke}{rgb}{0.121569,0.466667,0.705882}%
\pgfsetstrokecolor{currentstroke}%
\pgfsetstrokeopacity{0.640728}%
\pgfsetdash{}{0pt}%
\pgfpathmoveto{\pgfqpoint{3.073195in}{2.625771in}}%
\pgfpathcurveto{\pgfqpoint{3.081431in}{2.625771in}}{\pgfqpoint{3.089331in}{2.629043in}}{\pgfqpoint{3.095155in}{2.634867in}}%
\pgfpathcurveto{\pgfqpoint{3.100979in}{2.640691in}}{\pgfqpoint{3.104251in}{2.648591in}}{\pgfqpoint{3.104251in}{2.656827in}}%
\pgfpathcurveto{\pgfqpoint{3.104251in}{2.665064in}}{\pgfqpoint{3.100979in}{2.672964in}}{\pgfqpoint{3.095155in}{2.678788in}}%
\pgfpathcurveto{\pgfqpoint{3.089331in}{2.684612in}}{\pgfqpoint{3.081431in}{2.687884in}}{\pgfqpoint{3.073195in}{2.687884in}}%
\pgfpathcurveto{\pgfqpoint{3.064958in}{2.687884in}}{\pgfqpoint{3.057058in}{2.684612in}}{\pgfqpoint{3.051234in}{2.678788in}}%
\pgfpathcurveto{\pgfqpoint{3.045410in}{2.672964in}}{\pgfqpoint{3.042138in}{2.665064in}}{\pgfqpoint{3.042138in}{2.656827in}}%
\pgfpathcurveto{\pgfqpoint{3.042138in}{2.648591in}}{\pgfqpoint{3.045410in}{2.640691in}}{\pgfqpoint{3.051234in}{2.634867in}}%
\pgfpathcurveto{\pgfqpoint{3.057058in}{2.629043in}}{\pgfqpoint{3.064958in}{2.625771in}}{\pgfqpoint{3.073195in}{2.625771in}}%
\pgfpathclose%
\pgfusepath{stroke,fill}%
\end{pgfscope}%
\begin{pgfscope}%
\pgfpathrectangle{\pgfqpoint{0.100000in}{0.212622in}}{\pgfqpoint{3.696000in}{3.696000in}}%
\pgfusepath{clip}%
\pgfsetbuttcap%
\pgfsetroundjoin%
\definecolor{currentfill}{rgb}{0.121569,0.466667,0.705882}%
\pgfsetfillcolor{currentfill}%
\pgfsetfillopacity{0.640998}%
\pgfsetlinewidth{1.003750pt}%
\definecolor{currentstroke}{rgb}{0.121569,0.466667,0.705882}%
\pgfsetstrokecolor{currentstroke}%
\pgfsetstrokeopacity{0.640998}%
\pgfsetdash{}{0pt}%
\pgfpathmoveto{\pgfqpoint{1.610728in}{2.800618in}}%
\pgfpathcurveto{\pgfqpoint{1.618964in}{2.800618in}}{\pgfqpoint{1.626864in}{2.803890in}}{\pgfqpoint{1.632688in}{2.809714in}}%
\pgfpathcurveto{\pgfqpoint{1.638512in}{2.815538in}}{\pgfqpoint{1.641784in}{2.823438in}}{\pgfqpoint{1.641784in}{2.831674in}}%
\pgfpathcurveto{\pgfqpoint{1.641784in}{2.839911in}}{\pgfqpoint{1.638512in}{2.847811in}}{\pgfqpoint{1.632688in}{2.853635in}}%
\pgfpathcurveto{\pgfqpoint{1.626864in}{2.859459in}}{\pgfqpoint{1.618964in}{2.862731in}}{\pgfqpoint{1.610728in}{2.862731in}}%
\pgfpathcurveto{\pgfqpoint{1.602492in}{2.862731in}}{\pgfqpoint{1.594592in}{2.859459in}}{\pgfqpoint{1.588768in}{2.853635in}}%
\pgfpathcurveto{\pgfqpoint{1.582944in}{2.847811in}}{\pgfqpoint{1.579671in}{2.839911in}}{\pgfqpoint{1.579671in}{2.831674in}}%
\pgfpathcurveto{\pgfqpoint{1.579671in}{2.823438in}}{\pgfqpoint{1.582944in}{2.815538in}}{\pgfqpoint{1.588768in}{2.809714in}}%
\pgfpathcurveto{\pgfqpoint{1.594592in}{2.803890in}}{\pgfqpoint{1.602492in}{2.800618in}}{\pgfqpoint{1.610728in}{2.800618in}}%
\pgfpathclose%
\pgfusepath{stroke,fill}%
\end{pgfscope}%
\begin{pgfscope}%
\pgfpathrectangle{\pgfqpoint{0.100000in}{0.212622in}}{\pgfqpoint{3.696000in}{3.696000in}}%
\pgfusepath{clip}%
\pgfsetbuttcap%
\pgfsetroundjoin%
\definecolor{currentfill}{rgb}{0.121569,0.466667,0.705882}%
\pgfsetfillcolor{currentfill}%
\pgfsetfillopacity{0.641187}%
\pgfsetlinewidth{1.003750pt}%
\definecolor{currentstroke}{rgb}{0.121569,0.466667,0.705882}%
\pgfsetstrokecolor{currentstroke}%
\pgfsetstrokeopacity{0.641187}%
\pgfsetdash{}{0pt}%
\pgfpathmoveto{\pgfqpoint{1.902383in}{3.039327in}}%
\pgfpathcurveto{\pgfqpoint{1.910619in}{3.039327in}}{\pgfqpoint{1.918519in}{3.042599in}}{\pgfqpoint{1.924343in}{3.048423in}}%
\pgfpathcurveto{\pgfqpoint{1.930167in}{3.054247in}}{\pgfqpoint{1.933440in}{3.062147in}}{\pgfqpoint{1.933440in}{3.070383in}}%
\pgfpathcurveto{\pgfqpoint{1.933440in}{3.078619in}}{\pgfqpoint{1.930167in}{3.086519in}}{\pgfqpoint{1.924343in}{3.092343in}}%
\pgfpathcurveto{\pgfqpoint{1.918519in}{3.098167in}}{\pgfqpoint{1.910619in}{3.101440in}}{\pgfqpoint{1.902383in}{3.101440in}}%
\pgfpathcurveto{\pgfqpoint{1.894147in}{3.101440in}}{\pgfqpoint{1.886247in}{3.098167in}}{\pgfqpoint{1.880423in}{3.092343in}}%
\pgfpathcurveto{\pgfqpoint{1.874599in}{3.086519in}}{\pgfqpoint{1.871327in}{3.078619in}}{\pgfqpoint{1.871327in}{3.070383in}}%
\pgfpathcurveto{\pgfqpoint{1.871327in}{3.062147in}}{\pgfqpoint{1.874599in}{3.054247in}}{\pgfqpoint{1.880423in}{3.048423in}}%
\pgfpathcurveto{\pgfqpoint{1.886247in}{3.042599in}}{\pgfqpoint{1.894147in}{3.039327in}}{\pgfqpoint{1.902383in}{3.039327in}}%
\pgfpathclose%
\pgfusepath{stroke,fill}%
\end{pgfscope}%
\begin{pgfscope}%
\pgfpathrectangle{\pgfqpoint{0.100000in}{0.212622in}}{\pgfqpoint{3.696000in}{3.696000in}}%
\pgfusepath{clip}%
\pgfsetbuttcap%
\pgfsetroundjoin%
\definecolor{currentfill}{rgb}{0.121569,0.466667,0.705882}%
\pgfsetfillcolor{currentfill}%
\pgfsetfillopacity{0.642058}%
\pgfsetlinewidth{1.003750pt}%
\definecolor{currentstroke}{rgb}{0.121569,0.466667,0.705882}%
\pgfsetstrokecolor{currentstroke}%
\pgfsetstrokeopacity{0.642058}%
\pgfsetdash{}{0pt}%
\pgfpathmoveto{\pgfqpoint{1.609939in}{2.793323in}}%
\pgfpathcurveto{\pgfqpoint{1.618175in}{2.793323in}}{\pgfqpoint{1.626075in}{2.796596in}}{\pgfqpoint{1.631899in}{2.802420in}}%
\pgfpathcurveto{\pgfqpoint{1.637723in}{2.808243in}}{\pgfqpoint{1.640995in}{2.816144in}}{\pgfqpoint{1.640995in}{2.824380in}}%
\pgfpathcurveto{\pgfqpoint{1.640995in}{2.832616in}}{\pgfqpoint{1.637723in}{2.840516in}}{\pgfqpoint{1.631899in}{2.846340in}}%
\pgfpathcurveto{\pgfqpoint{1.626075in}{2.852164in}}{\pgfqpoint{1.618175in}{2.855436in}}{\pgfqpoint{1.609939in}{2.855436in}}%
\pgfpathcurveto{\pgfqpoint{1.601702in}{2.855436in}}{\pgfqpoint{1.593802in}{2.852164in}}{\pgfqpoint{1.587979in}{2.846340in}}%
\pgfpathcurveto{\pgfqpoint{1.582155in}{2.840516in}}{\pgfqpoint{1.578882in}{2.832616in}}{\pgfqpoint{1.578882in}{2.824380in}}%
\pgfpathcurveto{\pgfqpoint{1.578882in}{2.816144in}}{\pgfqpoint{1.582155in}{2.808243in}}{\pgfqpoint{1.587979in}{2.802420in}}%
\pgfpathcurveto{\pgfqpoint{1.593802in}{2.796596in}}{\pgfqpoint{1.601702in}{2.793323in}}{\pgfqpoint{1.609939in}{2.793323in}}%
\pgfpathclose%
\pgfusepath{stroke,fill}%
\end{pgfscope}%
\begin{pgfscope}%
\pgfpathrectangle{\pgfqpoint{0.100000in}{0.212622in}}{\pgfqpoint{3.696000in}{3.696000in}}%
\pgfusepath{clip}%
\pgfsetbuttcap%
\pgfsetroundjoin%
\definecolor{currentfill}{rgb}{0.121569,0.466667,0.705882}%
\pgfsetfillcolor{currentfill}%
\pgfsetfillopacity{0.642582}%
\pgfsetlinewidth{1.003750pt}%
\definecolor{currentstroke}{rgb}{0.121569,0.466667,0.705882}%
\pgfsetstrokecolor{currentstroke}%
\pgfsetstrokeopacity{0.642582}%
\pgfsetdash{}{0pt}%
\pgfpathmoveto{\pgfqpoint{1.608509in}{2.789140in}}%
\pgfpathcurveto{\pgfqpoint{1.616745in}{2.789140in}}{\pgfqpoint{1.624646in}{2.792412in}}{\pgfqpoint{1.630469in}{2.798236in}}%
\pgfpathcurveto{\pgfqpoint{1.636293in}{2.804060in}}{\pgfqpoint{1.639566in}{2.811960in}}{\pgfqpoint{1.639566in}{2.820196in}}%
\pgfpathcurveto{\pgfqpoint{1.639566in}{2.828433in}}{\pgfqpoint{1.636293in}{2.836333in}}{\pgfqpoint{1.630469in}{2.842157in}}%
\pgfpathcurveto{\pgfqpoint{1.624646in}{2.847981in}}{\pgfqpoint{1.616745in}{2.851253in}}{\pgfqpoint{1.608509in}{2.851253in}}%
\pgfpathcurveto{\pgfqpoint{1.600273in}{2.851253in}}{\pgfqpoint{1.592373in}{2.847981in}}{\pgfqpoint{1.586549in}{2.842157in}}%
\pgfpathcurveto{\pgfqpoint{1.580725in}{2.836333in}}{\pgfqpoint{1.577453in}{2.828433in}}{\pgfqpoint{1.577453in}{2.820196in}}%
\pgfpathcurveto{\pgfqpoint{1.577453in}{2.811960in}}{\pgfqpoint{1.580725in}{2.804060in}}{\pgfqpoint{1.586549in}{2.798236in}}%
\pgfpathcurveto{\pgfqpoint{1.592373in}{2.792412in}}{\pgfqpoint{1.600273in}{2.789140in}}{\pgfqpoint{1.608509in}{2.789140in}}%
\pgfpathclose%
\pgfusepath{stroke,fill}%
\end{pgfscope}%
\begin{pgfscope}%
\pgfpathrectangle{\pgfqpoint{0.100000in}{0.212622in}}{\pgfqpoint{3.696000in}{3.696000in}}%
\pgfusepath{clip}%
\pgfsetbuttcap%
\pgfsetroundjoin%
\definecolor{currentfill}{rgb}{0.121569,0.466667,0.705882}%
\pgfsetfillcolor{currentfill}%
\pgfsetfillopacity{0.642815}%
\pgfsetlinewidth{1.003750pt}%
\definecolor{currentstroke}{rgb}{0.121569,0.466667,0.705882}%
\pgfsetstrokecolor{currentstroke}%
\pgfsetstrokeopacity{0.642815}%
\pgfsetdash{}{0pt}%
\pgfpathmoveto{\pgfqpoint{1.912141in}{3.039155in}}%
\pgfpathcurveto{\pgfqpoint{1.920378in}{3.039155in}}{\pgfqpoint{1.928278in}{3.042427in}}{\pgfqpoint{1.934101in}{3.048251in}}%
\pgfpathcurveto{\pgfqpoint{1.939925in}{3.054075in}}{\pgfqpoint{1.943198in}{3.061975in}}{\pgfqpoint{1.943198in}{3.070212in}}%
\pgfpathcurveto{\pgfqpoint{1.943198in}{3.078448in}}{\pgfqpoint{1.939925in}{3.086348in}}{\pgfqpoint{1.934101in}{3.092172in}}%
\pgfpathcurveto{\pgfqpoint{1.928278in}{3.097996in}}{\pgfqpoint{1.920378in}{3.101268in}}{\pgfqpoint{1.912141in}{3.101268in}}%
\pgfpathcurveto{\pgfqpoint{1.903905in}{3.101268in}}{\pgfqpoint{1.896005in}{3.097996in}}{\pgfqpoint{1.890181in}{3.092172in}}%
\pgfpathcurveto{\pgfqpoint{1.884357in}{3.086348in}}{\pgfqpoint{1.881085in}{3.078448in}}{\pgfqpoint{1.881085in}{3.070212in}}%
\pgfpathcurveto{\pgfqpoint{1.881085in}{3.061975in}}{\pgfqpoint{1.884357in}{3.054075in}}{\pgfqpoint{1.890181in}{3.048251in}}%
\pgfpathcurveto{\pgfqpoint{1.896005in}{3.042427in}}{\pgfqpoint{1.903905in}{3.039155in}}{\pgfqpoint{1.912141in}{3.039155in}}%
\pgfpathclose%
\pgfusepath{stroke,fill}%
\end{pgfscope}%
\begin{pgfscope}%
\pgfpathrectangle{\pgfqpoint{0.100000in}{0.212622in}}{\pgfqpoint{3.696000in}{3.696000in}}%
\pgfusepath{clip}%
\pgfsetbuttcap%
\pgfsetroundjoin%
\definecolor{currentfill}{rgb}{0.121569,0.466667,0.705882}%
\pgfsetfillcolor{currentfill}%
\pgfsetfillopacity{0.643033}%
\pgfsetlinewidth{1.003750pt}%
\definecolor{currentstroke}{rgb}{0.121569,0.466667,0.705882}%
\pgfsetstrokecolor{currentstroke}%
\pgfsetstrokeopacity{0.643033}%
\pgfsetdash{}{0pt}%
\pgfpathmoveto{\pgfqpoint{1.606017in}{2.784498in}}%
\pgfpathcurveto{\pgfqpoint{1.614254in}{2.784498in}}{\pgfqpoint{1.622154in}{2.787770in}}{\pgfqpoint{1.627978in}{2.793594in}}%
\pgfpathcurveto{\pgfqpoint{1.633802in}{2.799418in}}{\pgfqpoint{1.637074in}{2.807318in}}{\pgfqpoint{1.637074in}{2.815555in}}%
\pgfpathcurveto{\pgfqpoint{1.637074in}{2.823791in}}{\pgfqpoint{1.633802in}{2.831691in}}{\pgfqpoint{1.627978in}{2.837515in}}%
\pgfpathcurveto{\pgfqpoint{1.622154in}{2.843339in}}{\pgfqpoint{1.614254in}{2.846611in}}{\pgfqpoint{1.606017in}{2.846611in}}%
\pgfpathcurveto{\pgfqpoint{1.597781in}{2.846611in}}{\pgfqpoint{1.589881in}{2.843339in}}{\pgfqpoint{1.584057in}{2.837515in}}%
\pgfpathcurveto{\pgfqpoint{1.578233in}{2.831691in}}{\pgfqpoint{1.574961in}{2.823791in}}{\pgfqpoint{1.574961in}{2.815555in}}%
\pgfpathcurveto{\pgfqpoint{1.574961in}{2.807318in}}{\pgfqpoint{1.578233in}{2.799418in}}{\pgfqpoint{1.584057in}{2.793594in}}%
\pgfpathcurveto{\pgfqpoint{1.589881in}{2.787770in}}{\pgfqpoint{1.597781in}{2.784498in}}{\pgfqpoint{1.606017in}{2.784498in}}%
\pgfpathclose%
\pgfusepath{stroke,fill}%
\end{pgfscope}%
\begin{pgfscope}%
\pgfpathrectangle{\pgfqpoint{0.100000in}{0.212622in}}{\pgfqpoint{3.696000in}{3.696000in}}%
\pgfusepath{clip}%
\pgfsetbuttcap%
\pgfsetroundjoin%
\definecolor{currentfill}{rgb}{0.121569,0.466667,0.705882}%
\pgfsetfillcolor{currentfill}%
\pgfsetfillopacity{0.643174}%
\pgfsetlinewidth{1.003750pt}%
\definecolor{currentstroke}{rgb}{0.121569,0.466667,0.705882}%
\pgfsetstrokecolor{currentstroke}%
\pgfsetstrokeopacity{0.643174}%
\pgfsetdash{}{0pt}%
\pgfpathmoveto{\pgfqpoint{3.080780in}{2.623293in}}%
\pgfpathcurveto{\pgfqpoint{3.089016in}{2.623293in}}{\pgfqpoint{3.096916in}{2.626565in}}{\pgfqpoint{3.102740in}{2.632389in}}%
\pgfpathcurveto{\pgfqpoint{3.108564in}{2.638213in}}{\pgfqpoint{3.111836in}{2.646113in}}{\pgfqpoint{3.111836in}{2.654350in}}%
\pgfpathcurveto{\pgfqpoint{3.111836in}{2.662586in}}{\pgfqpoint{3.108564in}{2.670486in}}{\pgfqpoint{3.102740in}{2.676310in}}%
\pgfpathcurveto{\pgfqpoint{3.096916in}{2.682134in}}{\pgfqpoint{3.089016in}{2.685406in}}{\pgfqpoint{3.080780in}{2.685406in}}%
\pgfpathcurveto{\pgfqpoint{3.072544in}{2.685406in}}{\pgfqpoint{3.064644in}{2.682134in}}{\pgfqpoint{3.058820in}{2.676310in}}%
\pgfpathcurveto{\pgfqpoint{3.052996in}{2.670486in}}{\pgfqpoint{3.049723in}{2.662586in}}{\pgfqpoint{3.049723in}{2.654350in}}%
\pgfpathcurveto{\pgfqpoint{3.049723in}{2.646113in}}{\pgfqpoint{3.052996in}{2.638213in}}{\pgfqpoint{3.058820in}{2.632389in}}%
\pgfpathcurveto{\pgfqpoint{3.064644in}{2.626565in}}{\pgfqpoint{3.072544in}{2.623293in}}{\pgfqpoint{3.080780in}{2.623293in}}%
\pgfpathclose%
\pgfusepath{stroke,fill}%
\end{pgfscope}%
\begin{pgfscope}%
\pgfpathrectangle{\pgfqpoint{0.100000in}{0.212622in}}{\pgfqpoint{3.696000in}{3.696000in}}%
\pgfusepath{clip}%
\pgfsetbuttcap%
\pgfsetroundjoin%
\definecolor{currentfill}{rgb}{0.121569,0.466667,0.705882}%
\pgfsetfillcolor{currentfill}%
\pgfsetfillopacity{0.643279}%
\pgfsetlinewidth{1.003750pt}%
\definecolor{currentstroke}{rgb}{0.121569,0.466667,0.705882}%
\pgfsetstrokecolor{currentstroke}%
\pgfsetstrokeopacity{0.643279}%
\pgfsetdash{}{0pt}%
\pgfpathmoveto{\pgfqpoint{1.604549in}{2.782134in}}%
\pgfpathcurveto{\pgfqpoint{1.612785in}{2.782134in}}{\pgfqpoint{1.620685in}{2.785406in}}{\pgfqpoint{1.626509in}{2.791230in}}%
\pgfpathcurveto{\pgfqpoint{1.632333in}{2.797054in}}{\pgfqpoint{1.635606in}{2.804954in}}{\pgfqpoint{1.635606in}{2.813190in}}%
\pgfpathcurveto{\pgfqpoint{1.635606in}{2.821427in}}{\pgfqpoint{1.632333in}{2.829327in}}{\pgfqpoint{1.626509in}{2.835151in}}%
\pgfpathcurveto{\pgfqpoint{1.620685in}{2.840975in}}{\pgfqpoint{1.612785in}{2.844247in}}{\pgfqpoint{1.604549in}{2.844247in}}%
\pgfpathcurveto{\pgfqpoint{1.596313in}{2.844247in}}{\pgfqpoint{1.588413in}{2.840975in}}{\pgfqpoint{1.582589in}{2.835151in}}%
\pgfpathcurveto{\pgfqpoint{1.576765in}{2.829327in}}{\pgfqpoint{1.573493in}{2.821427in}}{\pgfqpoint{1.573493in}{2.813190in}}%
\pgfpathcurveto{\pgfqpoint{1.573493in}{2.804954in}}{\pgfqpoint{1.576765in}{2.797054in}}{\pgfqpoint{1.582589in}{2.791230in}}%
\pgfpathcurveto{\pgfqpoint{1.588413in}{2.785406in}}{\pgfqpoint{1.596313in}{2.782134in}}{\pgfqpoint{1.604549in}{2.782134in}}%
\pgfpathclose%
\pgfusepath{stroke,fill}%
\end{pgfscope}%
\begin{pgfscope}%
\pgfpathrectangle{\pgfqpoint{0.100000in}{0.212622in}}{\pgfqpoint{3.696000in}{3.696000in}}%
\pgfusepath{clip}%
\pgfsetbuttcap%
\pgfsetroundjoin%
\definecolor{currentfill}{rgb}{0.121569,0.466667,0.705882}%
\pgfsetfillcolor{currentfill}%
\pgfsetfillopacity{0.643617}%
\pgfsetlinewidth{1.003750pt}%
\definecolor{currentstroke}{rgb}{0.121569,0.466667,0.705882}%
\pgfsetstrokecolor{currentstroke}%
\pgfsetstrokeopacity{0.643617}%
\pgfsetdash{}{0pt}%
\pgfpathmoveto{\pgfqpoint{1.603075in}{2.778838in}}%
\pgfpathcurveto{\pgfqpoint{1.611311in}{2.778838in}}{\pgfqpoint{1.619211in}{2.782110in}}{\pgfqpoint{1.625035in}{2.787934in}}%
\pgfpathcurveto{\pgfqpoint{1.630859in}{2.793758in}}{\pgfqpoint{1.634131in}{2.801658in}}{\pgfqpoint{1.634131in}{2.809894in}}%
\pgfpathcurveto{\pgfqpoint{1.634131in}{2.818131in}}{\pgfqpoint{1.630859in}{2.826031in}}{\pgfqpoint{1.625035in}{2.831855in}}%
\pgfpathcurveto{\pgfqpoint{1.619211in}{2.837679in}}{\pgfqpoint{1.611311in}{2.840951in}}{\pgfqpoint{1.603075in}{2.840951in}}%
\pgfpathcurveto{\pgfqpoint{1.594839in}{2.840951in}}{\pgfqpoint{1.586939in}{2.837679in}}{\pgfqpoint{1.581115in}{2.831855in}}%
\pgfpathcurveto{\pgfqpoint{1.575291in}{2.826031in}}{\pgfqpoint{1.572018in}{2.818131in}}{\pgfqpoint{1.572018in}{2.809894in}}%
\pgfpathcurveto{\pgfqpoint{1.572018in}{2.801658in}}{\pgfqpoint{1.575291in}{2.793758in}}{\pgfqpoint{1.581115in}{2.787934in}}%
\pgfpathcurveto{\pgfqpoint{1.586939in}{2.782110in}}{\pgfqpoint{1.594839in}{2.778838in}}{\pgfqpoint{1.603075in}{2.778838in}}%
\pgfpathclose%
\pgfusepath{stroke,fill}%
\end{pgfscope}%
\begin{pgfscope}%
\pgfpathrectangle{\pgfqpoint{0.100000in}{0.212622in}}{\pgfqpoint{3.696000in}{3.696000in}}%
\pgfusepath{clip}%
\pgfsetbuttcap%
\pgfsetroundjoin%
\definecolor{currentfill}{rgb}{0.121569,0.466667,0.705882}%
\pgfsetfillcolor{currentfill}%
\pgfsetfillopacity{0.643793}%
\pgfsetlinewidth{1.003750pt}%
\definecolor{currentstroke}{rgb}{0.121569,0.466667,0.705882}%
\pgfsetstrokecolor{currentstroke}%
\pgfsetstrokeopacity{0.643793}%
\pgfsetdash{}{0pt}%
\pgfpathmoveto{\pgfqpoint{1.602322in}{2.776914in}}%
\pgfpathcurveto{\pgfqpoint{1.610558in}{2.776914in}}{\pgfqpoint{1.618458in}{2.780187in}}{\pgfqpoint{1.624282in}{2.786010in}}%
\pgfpathcurveto{\pgfqpoint{1.630106in}{2.791834in}}{\pgfqpoint{1.633378in}{2.799734in}}{\pgfqpoint{1.633378in}{2.807971in}}%
\pgfpathcurveto{\pgfqpoint{1.633378in}{2.816207in}}{\pgfqpoint{1.630106in}{2.824107in}}{\pgfqpoint{1.624282in}{2.829931in}}%
\pgfpathcurveto{\pgfqpoint{1.618458in}{2.835755in}}{\pgfqpoint{1.610558in}{2.839027in}}{\pgfqpoint{1.602322in}{2.839027in}}%
\pgfpathcurveto{\pgfqpoint{1.594085in}{2.839027in}}{\pgfqpoint{1.586185in}{2.835755in}}{\pgfqpoint{1.580362in}{2.829931in}}%
\pgfpathcurveto{\pgfqpoint{1.574538in}{2.824107in}}{\pgfqpoint{1.571265in}{2.816207in}}{\pgfqpoint{1.571265in}{2.807971in}}%
\pgfpathcurveto{\pgfqpoint{1.571265in}{2.799734in}}{\pgfqpoint{1.574538in}{2.791834in}}{\pgfqpoint{1.580362in}{2.786010in}}%
\pgfpathcurveto{\pgfqpoint{1.586185in}{2.780187in}}{\pgfqpoint{1.594085in}{2.776914in}}{\pgfqpoint{1.602322in}{2.776914in}}%
\pgfpathclose%
\pgfusepath{stroke,fill}%
\end{pgfscope}%
\begin{pgfscope}%
\pgfpathrectangle{\pgfqpoint{0.100000in}{0.212622in}}{\pgfqpoint{3.696000in}{3.696000in}}%
\pgfusepath{clip}%
\pgfsetbuttcap%
\pgfsetroundjoin%
\definecolor{currentfill}{rgb}{0.121569,0.466667,0.705882}%
\pgfsetfillcolor{currentfill}%
\pgfsetfillopacity{0.644045}%
\pgfsetlinewidth{1.003750pt}%
\definecolor{currentstroke}{rgb}{0.121569,0.466667,0.705882}%
\pgfsetstrokecolor{currentstroke}%
\pgfsetstrokeopacity{0.644045}%
\pgfsetdash{}{0pt}%
\pgfpathmoveto{\pgfqpoint{1.600348in}{2.773726in}}%
\pgfpathcurveto{\pgfqpoint{1.608585in}{2.773726in}}{\pgfqpoint{1.616485in}{2.776999in}}{\pgfqpoint{1.622309in}{2.782823in}}%
\pgfpathcurveto{\pgfqpoint{1.628133in}{2.788647in}}{\pgfqpoint{1.631405in}{2.796547in}}{\pgfqpoint{1.631405in}{2.804783in}}%
\pgfpathcurveto{\pgfqpoint{1.631405in}{2.813019in}}{\pgfqpoint{1.628133in}{2.820919in}}{\pgfqpoint{1.622309in}{2.826743in}}%
\pgfpathcurveto{\pgfqpoint{1.616485in}{2.832567in}}{\pgfqpoint{1.608585in}{2.835839in}}{\pgfqpoint{1.600348in}{2.835839in}}%
\pgfpathcurveto{\pgfqpoint{1.592112in}{2.835839in}}{\pgfqpoint{1.584212in}{2.832567in}}{\pgfqpoint{1.578388in}{2.826743in}}%
\pgfpathcurveto{\pgfqpoint{1.572564in}{2.820919in}}{\pgfqpoint{1.569292in}{2.813019in}}{\pgfqpoint{1.569292in}{2.804783in}}%
\pgfpathcurveto{\pgfqpoint{1.569292in}{2.796547in}}{\pgfqpoint{1.572564in}{2.788647in}}{\pgfqpoint{1.578388in}{2.782823in}}%
\pgfpathcurveto{\pgfqpoint{1.584212in}{2.776999in}}{\pgfqpoint{1.592112in}{2.773726in}}{\pgfqpoint{1.600348in}{2.773726in}}%
\pgfpathclose%
\pgfusepath{stroke,fill}%
\end{pgfscope}%
\begin{pgfscope}%
\pgfpathrectangle{\pgfqpoint{0.100000in}{0.212622in}}{\pgfqpoint{3.696000in}{3.696000in}}%
\pgfusepath{clip}%
\pgfsetbuttcap%
\pgfsetroundjoin%
\definecolor{currentfill}{rgb}{0.121569,0.466667,0.705882}%
\pgfsetfillcolor{currentfill}%
\pgfsetfillopacity{0.644329}%
\pgfsetlinewidth{1.003750pt}%
\definecolor{currentstroke}{rgb}{0.121569,0.466667,0.705882}%
\pgfsetstrokecolor{currentstroke}%
\pgfsetstrokeopacity{0.644329}%
\pgfsetdash{}{0pt}%
\pgfpathmoveto{\pgfqpoint{1.598082in}{2.770119in}}%
\pgfpathcurveto{\pgfqpoint{1.606318in}{2.770119in}}{\pgfqpoint{1.614218in}{2.773391in}}{\pgfqpoint{1.620042in}{2.779215in}}%
\pgfpathcurveto{\pgfqpoint{1.625866in}{2.785039in}}{\pgfqpoint{1.629139in}{2.792939in}}{\pgfqpoint{1.629139in}{2.801175in}}%
\pgfpathcurveto{\pgfqpoint{1.629139in}{2.809412in}}{\pgfqpoint{1.625866in}{2.817312in}}{\pgfqpoint{1.620042in}{2.823136in}}%
\pgfpathcurveto{\pgfqpoint{1.614218in}{2.828960in}}{\pgfqpoint{1.606318in}{2.832232in}}{\pgfqpoint{1.598082in}{2.832232in}}%
\pgfpathcurveto{\pgfqpoint{1.589846in}{2.832232in}}{\pgfqpoint{1.581946in}{2.828960in}}{\pgfqpoint{1.576122in}{2.823136in}}%
\pgfpathcurveto{\pgfqpoint{1.570298in}{2.817312in}}{\pgfqpoint{1.567026in}{2.809412in}}{\pgfqpoint{1.567026in}{2.801175in}}%
\pgfpathcurveto{\pgfqpoint{1.567026in}{2.792939in}}{\pgfqpoint{1.570298in}{2.785039in}}{\pgfqpoint{1.576122in}{2.779215in}}%
\pgfpathcurveto{\pgfqpoint{1.581946in}{2.773391in}}{\pgfqpoint{1.589846in}{2.770119in}}{\pgfqpoint{1.598082in}{2.770119in}}%
\pgfpathclose%
\pgfusepath{stroke,fill}%
\end{pgfscope}%
\begin{pgfscope}%
\pgfpathrectangle{\pgfqpoint{0.100000in}{0.212622in}}{\pgfqpoint{3.696000in}{3.696000in}}%
\pgfusepath{clip}%
\pgfsetbuttcap%
\pgfsetroundjoin%
\definecolor{currentfill}{rgb}{0.121569,0.466667,0.705882}%
\pgfsetfillcolor{currentfill}%
\pgfsetfillopacity{0.644344}%
\pgfsetlinewidth{1.003750pt}%
\definecolor{currentstroke}{rgb}{0.121569,0.466667,0.705882}%
\pgfsetstrokecolor{currentstroke}%
\pgfsetstrokeopacity{0.644344}%
\pgfsetdash{}{0pt}%
\pgfpathmoveto{\pgfqpoint{3.089728in}{2.622193in}}%
\pgfpathcurveto{\pgfqpoint{3.097965in}{2.622193in}}{\pgfqpoint{3.105865in}{2.625465in}}{\pgfqpoint{3.111689in}{2.631289in}}%
\pgfpathcurveto{\pgfqpoint{3.117513in}{2.637113in}}{\pgfqpoint{3.120785in}{2.645013in}}{\pgfqpoint{3.120785in}{2.653250in}}%
\pgfpathcurveto{\pgfqpoint{3.120785in}{2.661486in}}{\pgfqpoint{3.117513in}{2.669386in}}{\pgfqpoint{3.111689in}{2.675210in}}%
\pgfpathcurveto{\pgfqpoint{3.105865in}{2.681034in}}{\pgfqpoint{3.097965in}{2.684306in}}{\pgfqpoint{3.089728in}{2.684306in}}%
\pgfpathcurveto{\pgfqpoint{3.081492in}{2.684306in}}{\pgfqpoint{3.073592in}{2.681034in}}{\pgfqpoint{3.067768in}{2.675210in}}%
\pgfpathcurveto{\pgfqpoint{3.061944in}{2.669386in}}{\pgfqpoint{3.058672in}{2.661486in}}{\pgfqpoint{3.058672in}{2.653250in}}%
\pgfpathcurveto{\pgfqpoint{3.058672in}{2.645013in}}{\pgfqpoint{3.061944in}{2.637113in}}{\pgfqpoint{3.067768in}{2.631289in}}%
\pgfpathcurveto{\pgfqpoint{3.073592in}{2.625465in}}{\pgfqpoint{3.081492in}{2.622193in}}{\pgfqpoint{3.089728in}{2.622193in}}%
\pgfpathclose%
\pgfusepath{stroke,fill}%
\end{pgfscope}%
\begin{pgfscope}%
\pgfpathrectangle{\pgfqpoint{0.100000in}{0.212622in}}{\pgfqpoint{3.696000in}{3.696000in}}%
\pgfusepath{clip}%
\pgfsetbuttcap%
\pgfsetroundjoin%
\definecolor{currentfill}{rgb}{0.121569,0.466667,0.705882}%
\pgfsetfillcolor{currentfill}%
\pgfsetfillopacity{0.644621}%
\pgfsetlinewidth{1.003750pt}%
\definecolor{currentstroke}{rgb}{0.121569,0.466667,0.705882}%
\pgfsetstrokecolor{currentstroke}%
\pgfsetstrokeopacity{0.644621}%
\pgfsetdash{}{0pt}%
\pgfpathmoveto{\pgfqpoint{1.919797in}{3.039860in}}%
\pgfpathcurveto{\pgfqpoint{1.928033in}{3.039860in}}{\pgfqpoint{1.935933in}{3.043133in}}{\pgfqpoint{1.941757in}{3.048956in}}%
\pgfpathcurveto{\pgfqpoint{1.947581in}{3.054780in}}{\pgfqpoint{1.950853in}{3.062680in}}{\pgfqpoint{1.950853in}{3.070917in}}%
\pgfpathcurveto{\pgfqpoint{1.950853in}{3.079153in}}{\pgfqpoint{1.947581in}{3.087053in}}{\pgfqpoint{1.941757in}{3.092877in}}%
\pgfpathcurveto{\pgfqpoint{1.935933in}{3.098701in}}{\pgfqpoint{1.928033in}{3.101973in}}{\pgfqpoint{1.919797in}{3.101973in}}%
\pgfpathcurveto{\pgfqpoint{1.911560in}{3.101973in}}{\pgfqpoint{1.903660in}{3.098701in}}{\pgfqpoint{1.897836in}{3.092877in}}%
\pgfpathcurveto{\pgfqpoint{1.892012in}{3.087053in}}{\pgfqpoint{1.888740in}{3.079153in}}{\pgfqpoint{1.888740in}{3.070917in}}%
\pgfpathcurveto{\pgfqpoint{1.888740in}{3.062680in}}{\pgfqpoint{1.892012in}{3.054780in}}{\pgfqpoint{1.897836in}{3.048956in}}%
\pgfpathcurveto{\pgfqpoint{1.903660in}{3.043133in}}{\pgfqpoint{1.911560in}{3.039860in}}{\pgfqpoint{1.919797in}{3.039860in}}%
\pgfpathclose%
\pgfusepath{stroke,fill}%
\end{pgfscope}%
\begin{pgfscope}%
\pgfpathrectangle{\pgfqpoint{0.100000in}{0.212622in}}{\pgfqpoint{3.696000in}{3.696000in}}%
\pgfusepath{clip}%
\pgfsetbuttcap%
\pgfsetroundjoin%
\definecolor{currentfill}{rgb}{0.121569,0.466667,0.705882}%
\pgfsetfillcolor{currentfill}%
\pgfsetfillopacity{0.644950}%
\pgfsetlinewidth{1.003750pt}%
\definecolor{currentstroke}{rgb}{0.121569,0.466667,0.705882}%
\pgfsetstrokecolor{currentstroke}%
\pgfsetstrokeopacity{0.644950}%
\pgfsetdash{}{0pt}%
\pgfpathmoveto{\pgfqpoint{1.595594in}{2.763768in}}%
\pgfpathcurveto{\pgfqpoint{1.603830in}{2.763768in}}{\pgfqpoint{1.611730in}{2.767040in}}{\pgfqpoint{1.617554in}{2.772864in}}%
\pgfpathcurveto{\pgfqpoint{1.623378in}{2.778688in}}{\pgfqpoint{1.626650in}{2.786588in}}{\pgfqpoint{1.626650in}{2.794825in}}%
\pgfpathcurveto{\pgfqpoint{1.626650in}{2.803061in}}{\pgfqpoint{1.623378in}{2.810961in}}{\pgfqpoint{1.617554in}{2.816785in}}%
\pgfpathcurveto{\pgfqpoint{1.611730in}{2.822609in}}{\pgfqpoint{1.603830in}{2.825881in}}{\pgfqpoint{1.595594in}{2.825881in}}%
\pgfpathcurveto{\pgfqpoint{1.587357in}{2.825881in}}{\pgfqpoint{1.579457in}{2.822609in}}{\pgfqpoint{1.573633in}{2.816785in}}%
\pgfpathcurveto{\pgfqpoint{1.567810in}{2.810961in}}{\pgfqpoint{1.564537in}{2.803061in}}{\pgfqpoint{1.564537in}{2.794825in}}%
\pgfpathcurveto{\pgfqpoint{1.564537in}{2.786588in}}{\pgfqpoint{1.567810in}{2.778688in}}{\pgfqpoint{1.573633in}{2.772864in}}%
\pgfpathcurveto{\pgfqpoint{1.579457in}{2.767040in}}{\pgfqpoint{1.587357in}{2.763768in}}{\pgfqpoint{1.595594in}{2.763768in}}%
\pgfpathclose%
\pgfusepath{stroke,fill}%
\end{pgfscope}%
\begin{pgfscope}%
\pgfpathrectangle{\pgfqpoint{0.100000in}{0.212622in}}{\pgfqpoint{3.696000in}{3.696000in}}%
\pgfusepath{clip}%
\pgfsetbuttcap%
\pgfsetroundjoin%
\definecolor{currentfill}{rgb}{0.121569,0.466667,0.705882}%
\pgfsetfillcolor{currentfill}%
\pgfsetfillopacity{0.645301}%
\pgfsetlinewidth{1.003750pt}%
\definecolor{currentstroke}{rgb}{0.121569,0.466667,0.705882}%
\pgfsetstrokecolor{currentstroke}%
\pgfsetstrokeopacity{0.645301}%
\pgfsetdash{}{0pt}%
\pgfpathmoveto{\pgfqpoint{1.594629in}{2.760050in}}%
\pgfpathcurveto{\pgfqpoint{1.602865in}{2.760050in}}{\pgfqpoint{1.610765in}{2.763323in}}{\pgfqpoint{1.616589in}{2.769147in}}%
\pgfpathcurveto{\pgfqpoint{1.622413in}{2.774970in}}{\pgfqpoint{1.625686in}{2.782870in}}{\pgfqpoint{1.625686in}{2.791107in}}%
\pgfpathcurveto{\pgfqpoint{1.625686in}{2.799343in}}{\pgfqpoint{1.622413in}{2.807243in}}{\pgfqpoint{1.616589in}{2.813067in}}%
\pgfpathcurveto{\pgfqpoint{1.610765in}{2.818891in}}{\pgfqpoint{1.602865in}{2.822163in}}{\pgfqpoint{1.594629in}{2.822163in}}%
\pgfpathcurveto{\pgfqpoint{1.586393in}{2.822163in}}{\pgfqpoint{1.578493in}{2.818891in}}{\pgfqpoint{1.572669in}{2.813067in}}%
\pgfpathcurveto{\pgfqpoint{1.566845in}{2.807243in}}{\pgfqpoint{1.563573in}{2.799343in}}{\pgfqpoint{1.563573in}{2.791107in}}%
\pgfpathcurveto{\pgfqpoint{1.563573in}{2.782870in}}{\pgfqpoint{1.566845in}{2.774970in}}{\pgfqpoint{1.572669in}{2.769147in}}%
\pgfpathcurveto{\pgfqpoint{1.578493in}{2.763323in}}{\pgfqpoint{1.586393in}{2.760050in}}{\pgfqpoint{1.594629in}{2.760050in}}%
\pgfpathclose%
\pgfusepath{stroke,fill}%
\end{pgfscope}%
\begin{pgfscope}%
\pgfpathrectangle{\pgfqpoint{0.100000in}{0.212622in}}{\pgfqpoint{3.696000in}{3.696000in}}%
\pgfusepath{clip}%
\pgfsetbuttcap%
\pgfsetroundjoin%
\definecolor{currentfill}{rgb}{0.121569,0.466667,0.705882}%
\pgfsetfillcolor{currentfill}%
\pgfsetfillopacity{0.645765}%
\pgfsetlinewidth{1.003750pt}%
\definecolor{currentstroke}{rgb}{0.121569,0.466667,0.705882}%
\pgfsetstrokecolor{currentstroke}%
\pgfsetstrokeopacity{0.645765}%
\pgfsetdash{}{0pt}%
\pgfpathmoveto{\pgfqpoint{1.591947in}{2.755151in}}%
\pgfpathcurveto{\pgfqpoint{1.600183in}{2.755151in}}{\pgfqpoint{1.608083in}{2.758423in}}{\pgfqpoint{1.613907in}{2.764247in}}%
\pgfpathcurveto{\pgfqpoint{1.619731in}{2.770071in}}{\pgfqpoint{1.623004in}{2.777971in}}{\pgfqpoint{1.623004in}{2.786207in}}%
\pgfpathcurveto{\pgfqpoint{1.623004in}{2.794443in}}{\pgfqpoint{1.619731in}{2.802344in}}{\pgfqpoint{1.613907in}{2.808167in}}%
\pgfpathcurveto{\pgfqpoint{1.608083in}{2.813991in}}{\pgfqpoint{1.600183in}{2.817264in}}{\pgfqpoint{1.591947in}{2.817264in}}%
\pgfpathcurveto{\pgfqpoint{1.583711in}{2.817264in}}{\pgfqpoint{1.575811in}{2.813991in}}{\pgfqpoint{1.569987in}{2.808167in}}%
\pgfpathcurveto{\pgfqpoint{1.564163in}{2.802344in}}{\pgfqpoint{1.560891in}{2.794443in}}{\pgfqpoint{1.560891in}{2.786207in}}%
\pgfpathcurveto{\pgfqpoint{1.560891in}{2.777971in}}{\pgfqpoint{1.564163in}{2.770071in}}{\pgfqpoint{1.569987in}{2.764247in}}%
\pgfpathcurveto{\pgfqpoint{1.575811in}{2.758423in}}{\pgfqpoint{1.583711in}{2.755151in}}{\pgfqpoint{1.591947in}{2.755151in}}%
\pgfpathclose%
\pgfusepath{stroke,fill}%
\end{pgfscope}%
\begin{pgfscope}%
\pgfpathrectangle{\pgfqpoint{0.100000in}{0.212622in}}{\pgfqpoint{3.696000in}{3.696000in}}%
\pgfusepath{clip}%
\pgfsetbuttcap%
\pgfsetroundjoin%
\definecolor{currentfill}{rgb}{0.121569,0.466667,0.705882}%
\pgfsetfillcolor{currentfill}%
\pgfsetfillopacity{0.645822}%
\pgfsetlinewidth{1.003750pt}%
\definecolor{currentstroke}{rgb}{0.121569,0.466667,0.705882}%
\pgfsetstrokecolor{currentstroke}%
\pgfsetstrokeopacity{0.645822}%
\pgfsetdash{}{0pt}%
\pgfpathmoveto{\pgfqpoint{1.927108in}{3.038568in}}%
\pgfpathcurveto{\pgfqpoint{1.935344in}{3.038568in}}{\pgfqpoint{1.943244in}{3.041841in}}{\pgfqpoint{1.949068in}{3.047665in}}%
\pgfpathcurveto{\pgfqpoint{1.954892in}{3.053488in}}{\pgfqpoint{1.958164in}{3.061389in}}{\pgfqpoint{1.958164in}{3.069625in}}%
\pgfpathcurveto{\pgfqpoint{1.958164in}{3.077861in}}{\pgfqpoint{1.954892in}{3.085761in}}{\pgfqpoint{1.949068in}{3.091585in}}%
\pgfpathcurveto{\pgfqpoint{1.943244in}{3.097409in}}{\pgfqpoint{1.935344in}{3.100681in}}{\pgfqpoint{1.927108in}{3.100681in}}%
\pgfpathcurveto{\pgfqpoint{1.918871in}{3.100681in}}{\pgfqpoint{1.910971in}{3.097409in}}{\pgfqpoint{1.905148in}{3.091585in}}%
\pgfpathcurveto{\pgfqpoint{1.899324in}{3.085761in}}{\pgfqpoint{1.896051in}{3.077861in}}{\pgfqpoint{1.896051in}{3.069625in}}%
\pgfpathcurveto{\pgfqpoint{1.896051in}{3.061389in}}{\pgfqpoint{1.899324in}{3.053488in}}{\pgfqpoint{1.905148in}{3.047665in}}%
\pgfpathcurveto{\pgfqpoint{1.910971in}{3.041841in}}{\pgfqpoint{1.918871in}{3.038568in}}{\pgfqpoint{1.927108in}{3.038568in}}%
\pgfpathclose%
\pgfusepath{stroke,fill}%
\end{pgfscope}%
\begin{pgfscope}%
\pgfpathrectangle{\pgfqpoint{0.100000in}{0.212622in}}{\pgfqpoint{3.696000in}{3.696000in}}%
\pgfusepath{clip}%
\pgfsetbuttcap%
\pgfsetroundjoin%
\definecolor{currentfill}{rgb}{0.121569,0.466667,0.705882}%
\pgfsetfillcolor{currentfill}%
\pgfsetfillopacity{0.645880}%
\pgfsetlinewidth{1.003750pt}%
\definecolor{currentstroke}{rgb}{0.121569,0.466667,0.705882}%
\pgfsetstrokecolor{currentstroke}%
\pgfsetstrokeopacity{0.645880}%
\pgfsetdash{}{0pt}%
\pgfpathmoveto{\pgfqpoint{3.097321in}{2.620836in}}%
\pgfpathcurveto{\pgfqpoint{3.105558in}{2.620836in}}{\pgfqpoint{3.113458in}{2.624108in}}{\pgfqpoint{3.119282in}{2.629932in}}%
\pgfpathcurveto{\pgfqpoint{3.125105in}{2.635756in}}{\pgfqpoint{3.128378in}{2.643656in}}{\pgfqpoint{3.128378in}{2.651893in}}%
\pgfpathcurveto{\pgfqpoint{3.128378in}{2.660129in}}{\pgfqpoint{3.125105in}{2.668029in}}{\pgfqpoint{3.119282in}{2.673853in}}%
\pgfpathcurveto{\pgfqpoint{3.113458in}{2.679677in}}{\pgfqpoint{3.105558in}{2.682949in}}{\pgfqpoint{3.097321in}{2.682949in}}%
\pgfpathcurveto{\pgfqpoint{3.089085in}{2.682949in}}{\pgfqpoint{3.081185in}{2.679677in}}{\pgfqpoint{3.075361in}{2.673853in}}%
\pgfpathcurveto{\pgfqpoint{3.069537in}{2.668029in}}{\pgfqpoint{3.066265in}{2.660129in}}{\pgfqpoint{3.066265in}{2.651893in}}%
\pgfpathcurveto{\pgfqpoint{3.066265in}{2.643656in}}{\pgfqpoint{3.069537in}{2.635756in}}{\pgfqpoint{3.075361in}{2.629932in}}%
\pgfpathcurveto{\pgfqpoint{3.081185in}{2.624108in}}{\pgfqpoint{3.089085in}{2.620836in}}{\pgfqpoint{3.097321in}{2.620836in}}%
\pgfpathclose%
\pgfusepath{stroke,fill}%
\end{pgfscope}%
\begin{pgfscope}%
\pgfpathrectangle{\pgfqpoint{0.100000in}{0.212622in}}{\pgfqpoint{3.696000in}{3.696000in}}%
\pgfusepath{clip}%
\pgfsetbuttcap%
\pgfsetroundjoin%
\definecolor{currentfill}{rgb}{0.121569,0.466667,0.705882}%
\pgfsetfillcolor{currentfill}%
\pgfsetfillopacity{0.645961}%
\pgfsetlinewidth{1.003750pt}%
\definecolor{currentstroke}{rgb}{0.121569,0.466667,0.705882}%
\pgfsetstrokecolor{currentstroke}%
\pgfsetstrokeopacity{0.645961}%
\pgfsetdash{}{0pt}%
\pgfpathmoveto{\pgfqpoint{1.590292in}{2.752600in}}%
\pgfpathcurveto{\pgfqpoint{1.598529in}{2.752600in}}{\pgfqpoint{1.606429in}{2.755872in}}{\pgfqpoint{1.612253in}{2.761696in}}%
\pgfpathcurveto{\pgfqpoint{1.618077in}{2.767520in}}{\pgfqpoint{1.621349in}{2.775420in}}{\pgfqpoint{1.621349in}{2.783657in}}%
\pgfpathcurveto{\pgfqpoint{1.621349in}{2.791893in}}{\pgfqpoint{1.618077in}{2.799793in}}{\pgfqpoint{1.612253in}{2.805617in}}%
\pgfpathcurveto{\pgfqpoint{1.606429in}{2.811441in}}{\pgfqpoint{1.598529in}{2.814713in}}{\pgfqpoint{1.590292in}{2.814713in}}%
\pgfpathcurveto{\pgfqpoint{1.582056in}{2.814713in}}{\pgfqpoint{1.574156in}{2.811441in}}{\pgfqpoint{1.568332in}{2.805617in}}%
\pgfpathcurveto{\pgfqpoint{1.562508in}{2.799793in}}{\pgfqpoint{1.559236in}{2.791893in}}{\pgfqpoint{1.559236in}{2.783657in}}%
\pgfpathcurveto{\pgfqpoint{1.559236in}{2.775420in}}{\pgfqpoint{1.562508in}{2.767520in}}{\pgfqpoint{1.568332in}{2.761696in}}%
\pgfpathcurveto{\pgfqpoint{1.574156in}{2.755872in}}{\pgfqpoint{1.582056in}{2.752600in}}{\pgfqpoint{1.590292in}{2.752600in}}%
\pgfpathclose%
\pgfusepath{stroke,fill}%
\end{pgfscope}%
\begin{pgfscope}%
\pgfpathrectangle{\pgfqpoint{0.100000in}{0.212622in}}{\pgfqpoint{3.696000in}{3.696000in}}%
\pgfusepath{clip}%
\pgfsetbuttcap%
\pgfsetroundjoin%
\definecolor{currentfill}{rgb}{0.121569,0.466667,0.705882}%
\pgfsetfillcolor{currentfill}%
\pgfsetfillopacity{0.646253}%
\pgfsetlinewidth{1.003750pt}%
\definecolor{currentstroke}{rgb}{0.121569,0.466667,0.705882}%
\pgfsetstrokecolor{currentstroke}%
\pgfsetstrokeopacity{0.646253}%
\pgfsetdash{}{0pt}%
\pgfpathmoveto{\pgfqpoint{3.113526in}{2.620001in}}%
\pgfpathcurveto{\pgfqpoint{3.121763in}{2.620001in}}{\pgfqpoint{3.129663in}{2.623273in}}{\pgfqpoint{3.135487in}{2.629097in}}%
\pgfpathcurveto{\pgfqpoint{3.141310in}{2.634921in}}{\pgfqpoint{3.144583in}{2.642821in}}{\pgfqpoint{3.144583in}{2.651058in}}%
\pgfpathcurveto{\pgfqpoint{3.144583in}{2.659294in}}{\pgfqpoint{3.141310in}{2.667194in}}{\pgfqpoint{3.135487in}{2.673018in}}%
\pgfpathcurveto{\pgfqpoint{3.129663in}{2.678842in}}{\pgfqpoint{3.121763in}{2.682114in}}{\pgfqpoint{3.113526in}{2.682114in}}%
\pgfpathcurveto{\pgfqpoint{3.105290in}{2.682114in}}{\pgfqpoint{3.097390in}{2.678842in}}{\pgfqpoint{3.091566in}{2.673018in}}%
\pgfpathcurveto{\pgfqpoint{3.085742in}{2.667194in}}{\pgfqpoint{3.082470in}{2.659294in}}{\pgfqpoint{3.082470in}{2.651058in}}%
\pgfpathcurveto{\pgfqpoint{3.082470in}{2.642821in}}{\pgfqpoint{3.085742in}{2.634921in}}{\pgfqpoint{3.091566in}{2.629097in}}%
\pgfpathcurveto{\pgfqpoint{3.097390in}{2.623273in}}{\pgfqpoint{3.105290in}{2.620001in}}{\pgfqpoint{3.113526in}{2.620001in}}%
\pgfpathclose%
\pgfusepath{stroke,fill}%
\end{pgfscope}%
\begin{pgfscope}%
\pgfpathrectangle{\pgfqpoint{0.100000in}{0.212622in}}{\pgfqpoint{3.696000in}{3.696000in}}%
\pgfusepath{clip}%
\pgfsetbuttcap%
\pgfsetroundjoin%
\definecolor{currentfill}{rgb}{0.121569,0.466667,0.705882}%
\pgfsetfillcolor{currentfill}%
\pgfsetfillopacity{0.646411}%
\pgfsetlinewidth{1.003750pt}%
\definecolor{currentstroke}{rgb}{0.121569,0.466667,0.705882}%
\pgfsetstrokecolor{currentstroke}%
\pgfsetstrokeopacity{0.646411}%
\pgfsetdash{}{0pt}%
\pgfpathmoveto{\pgfqpoint{1.588033in}{2.748126in}}%
\pgfpathcurveto{\pgfqpoint{1.596269in}{2.748126in}}{\pgfqpoint{1.604169in}{2.751399in}}{\pgfqpoint{1.609993in}{2.757222in}}%
\pgfpathcurveto{\pgfqpoint{1.615817in}{2.763046in}}{\pgfqpoint{1.619089in}{2.770946in}}{\pgfqpoint{1.619089in}{2.779183in}}%
\pgfpathcurveto{\pgfqpoint{1.619089in}{2.787419in}}{\pgfqpoint{1.615817in}{2.795319in}}{\pgfqpoint{1.609993in}{2.801143in}}%
\pgfpathcurveto{\pgfqpoint{1.604169in}{2.806967in}}{\pgfqpoint{1.596269in}{2.810239in}}{\pgfqpoint{1.588033in}{2.810239in}}%
\pgfpathcurveto{\pgfqpoint{1.579796in}{2.810239in}}{\pgfqpoint{1.571896in}{2.806967in}}{\pgfqpoint{1.566072in}{2.801143in}}%
\pgfpathcurveto{\pgfqpoint{1.560248in}{2.795319in}}{\pgfqpoint{1.556976in}{2.787419in}}{\pgfqpoint{1.556976in}{2.779183in}}%
\pgfpathcurveto{\pgfqpoint{1.556976in}{2.770946in}}{\pgfqpoint{1.560248in}{2.763046in}}{\pgfqpoint{1.566072in}{2.757222in}}%
\pgfpathcurveto{\pgfqpoint{1.571896in}{2.751399in}}{\pgfqpoint{1.579796in}{2.748126in}}{\pgfqpoint{1.588033in}{2.748126in}}%
\pgfpathclose%
\pgfusepath{stroke,fill}%
\end{pgfscope}%
\begin{pgfscope}%
\pgfpathrectangle{\pgfqpoint{0.100000in}{0.212622in}}{\pgfqpoint{3.696000in}{3.696000in}}%
\pgfusepath{clip}%
\pgfsetbuttcap%
\pgfsetroundjoin%
\definecolor{currentfill}{rgb}{0.121569,0.466667,0.705882}%
\pgfsetfillcolor{currentfill}%
\pgfsetfillopacity{0.647061}%
\pgfsetlinewidth{1.003750pt}%
\definecolor{currentstroke}{rgb}{0.121569,0.466667,0.705882}%
\pgfsetstrokecolor{currentstroke}%
\pgfsetstrokeopacity{0.647061}%
\pgfsetdash{}{0pt}%
\pgfpathmoveto{\pgfqpoint{1.932804in}{3.036941in}}%
\pgfpathcurveto{\pgfqpoint{1.941040in}{3.036941in}}{\pgfqpoint{1.948940in}{3.040213in}}{\pgfqpoint{1.954764in}{3.046037in}}%
\pgfpathcurveto{\pgfqpoint{1.960588in}{3.051861in}}{\pgfqpoint{1.963860in}{3.059761in}}{\pgfqpoint{1.963860in}{3.067998in}}%
\pgfpathcurveto{\pgfqpoint{1.963860in}{3.076234in}}{\pgfqpoint{1.960588in}{3.084134in}}{\pgfqpoint{1.954764in}{3.089958in}}%
\pgfpathcurveto{\pgfqpoint{1.948940in}{3.095782in}}{\pgfqpoint{1.941040in}{3.099054in}}{\pgfqpoint{1.932804in}{3.099054in}}%
\pgfpathcurveto{\pgfqpoint{1.924567in}{3.099054in}}{\pgfqpoint{1.916667in}{3.095782in}}{\pgfqpoint{1.910843in}{3.089958in}}%
\pgfpathcurveto{\pgfqpoint{1.905019in}{3.084134in}}{\pgfqpoint{1.901747in}{3.076234in}}{\pgfqpoint{1.901747in}{3.067998in}}%
\pgfpathcurveto{\pgfqpoint{1.901747in}{3.059761in}}{\pgfqpoint{1.905019in}{3.051861in}}{\pgfqpoint{1.910843in}{3.046037in}}%
\pgfpathcurveto{\pgfqpoint{1.916667in}{3.040213in}}{\pgfqpoint{1.924567in}{3.036941in}}{\pgfqpoint{1.932804in}{3.036941in}}%
\pgfpathclose%
\pgfusepath{stroke,fill}%
\end{pgfscope}%
\begin{pgfscope}%
\pgfpathrectangle{\pgfqpoint{0.100000in}{0.212622in}}{\pgfqpoint{3.696000in}{3.696000in}}%
\pgfusepath{clip}%
\pgfsetbuttcap%
\pgfsetroundjoin%
\definecolor{currentfill}{rgb}{0.121569,0.466667,0.705882}%
\pgfsetfillcolor{currentfill}%
\pgfsetfillopacity{0.647088}%
\pgfsetlinewidth{1.003750pt}%
\definecolor{currentstroke}{rgb}{0.121569,0.466667,0.705882}%
\pgfsetstrokecolor{currentstroke}%
\pgfsetstrokeopacity{0.647088}%
\pgfsetdash{}{0pt}%
\pgfpathmoveto{\pgfqpoint{1.586270in}{2.741621in}}%
\pgfpathcurveto{\pgfqpoint{1.594506in}{2.741621in}}{\pgfqpoint{1.602406in}{2.744893in}}{\pgfqpoint{1.608230in}{2.750717in}}%
\pgfpathcurveto{\pgfqpoint{1.614054in}{2.756541in}}{\pgfqpoint{1.617326in}{2.764441in}}{\pgfqpoint{1.617326in}{2.772677in}}%
\pgfpathcurveto{\pgfqpoint{1.617326in}{2.780913in}}{\pgfqpoint{1.614054in}{2.788813in}}{\pgfqpoint{1.608230in}{2.794637in}}%
\pgfpathcurveto{\pgfqpoint{1.602406in}{2.800461in}}{\pgfqpoint{1.594506in}{2.803734in}}{\pgfqpoint{1.586270in}{2.803734in}}%
\pgfpathcurveto{\pgfqpoint{1.578033in}{2.803734in}}{\pgfqpoint{1.570133in}{2.800461in}}{\pgfqpoint{1.564309in}{2.794637in}}%
\pgfpathcurveto{\pgfqpoint{1.558485in}{2.788813in}}{\pgfqpoint{1.555213in}{2.780913in}}{\pgfqpoint{1.555213in}{2.772677in}}%
\pgfpathcurveto{\pgfqpoint{1.555213in}{2.764441in}}{\pgfqpoint{1.558485in}{2.756541in}}{\pgfqpoint{1.564309in}{2.750717in}}%
\pgfpathcurveto{\pgfqpoint{1.570133in}{2.744893in}}{\pgfqpoint{1.578033in}{2.741621in}}{\pgfqpoint{1.586270in}{2.741621in}}%
\pgfpathclose%
\pgfusepath{stroke,fill}%
\end{pgfscope}%
\begin{pgfscope}%
\pgfpathrectangle{\pgfqpoint{0.100000in}{0.212622in}}{\pgfqpoint{3.696000in}{3.696000in}}%
\pgfusepath{clip}%
\pgfsetbuttcap%
\pgfsetroundjoin%
\definecolor{currentfill}{rgb}{0.121569,0.466667,0.705882}%
\pgfsetfillcolor{currentfill}%
\pgfsetfillopacity{0.647440}%
\pgfsetlinewidth{1.003750pt}%
\definecolor{currentstroke}{rgb}{0.121569,0.466667,0.705882}%
\pgfsetstrokecolor{currentstroke}%
\pgfsetstrokeopacity{0.647440}%
\pgfsetdash{}{0pt}%
\pgfpathmoveto{\pgfqpoint{1.584751in}{2.738383in}}%
\pgfpathcurveto{\pgfqpoint{1.592987in}{2.738383in}}{\pgfqpoint{1.600887in}{2.741655in}}{\pgfqpoint{1.606711in}{2.747479in}}%
\pgfpathcurveto{\pgfqpoint{1.612535in}{2.753303in}}{\pgfqpoint{1.615807in}{2.761203in}}{\pgfqpoint{1.615807in}{2.769439in}}%
\pgfpathcurveto{\pgfqpoint{1.615807in}{2.777676in}}{\pgfqpoint{1.612535in}{2.785576in}}{\pgfqpoint{1.606711in}{2.791400in}}%
\pgfpathcurveto{\pgfqpoint{1.600887in}{2.797224in}}{\pgfqpoint{1.592987in}{2.800496in}}{\pgfqpoint{1.584751in}{2.800496in}}%
\pgfpathcurveto{\pgfqpoint{1.576514in}{2.800496in}}{\pgfqpoint{1.568614in}{2.797224in}}{\pgfqpoint{1.562791in}{2.791400in}}%
\pgfpathcurveto{\pgfqpoint{1.556967in}{2.785576in}}{\pgfqpoint{1.553694in}{2.777676in}}{\pgfqpoint{1.553694in}{2.769439in}}%
\pgfpathcurveto{\pgfqpoint{1.553694in}{2.761203in}}{\pgfqpoint{1.556967in}{2.753303in}}{\pgfqpoint{1.562791in}{2.747479in}}%
\pgfpathcurveto{\pgfqpoint{1.568614in}{2.741655in}}{\pgfqpoint{1.576514in}{2.738383in}}{\pgfqpoint{1.584751in}{2.738383in}}%
\pgfpathclose%
\pgfusepath{stroke,fill}%
\end{pgfscope}%
\begin{pgfscope}%
\pgfpathrectangle{\pgfqpoint{0.100000in}{0.212622in}}{\pgfqpoint{3.696000in}{3.696000in}}%
\pgfusepath{clip}%
\pgfsetbuttcap%
\pgfsetroundjoin%
\definecolor{currentfill}{rgb}{0.121569,0.466667,0.705882}%
\pgfsetfillcolor{currentfill}%
\pgfsetfillopacity{0.647597}%
\pgfsetlinewidth{1.003750pt}%
\definecolor{currentstroke}{rgb}{0.121569,0.466667,0.705882}%
\pgfsetstrokecolor{currentstroke}%
\pgfsetstrokeopacity{0.647597}%
\pgfsetdash{}{0pt}%
\pgfpathmoveto{\pgfqpoint{1.583741in}{2.736741in}}%
\pgfpathcurveto{\pgfqpoint{1.591977in}{2.736741in}}{\pgfqpoint{1.599877in}{2.740014in}}{\pgfqpoint{1.605701in}{2.745837in}}%
\pgfpathcurveto{\pgfqpoint{1.611525in}{2.751661in}}{\pgfqpoint{1.614798in}{2.759561in}}{\pgfqpoint{1.614798in}{2.767798in}}%
\pgfpathcurveto{\pgfqpoint{1.614798in}{2.776034in}}{\pgfqpoint{1.611525in}{2.783934in}}{\pgfqpoint{1.605701in}{2.789758in}}%
\pgfpathcurveto{\pgfqpoint{1.599877in}{2.795582in}}{\pgfqpoint{1.591977in}{2.798854in}}{\pgfqpoint{1.583741in}{2.798854in}}%
\pgfpathcurveto{\pgfqpoint{1.575505in}{2.798854in}}{\pgfqpoint{1.567605in}{2.795582in}}{\pgfqpoint{1.561781in}{2.789758in}}%
\pgfpathcurveto{\pgfqpoint{1.555957in}{2.783934in}}{\pgfqpoint{1.552685in}{2.776034in}}{\pgfqpoint{1.552685in}{2.767798in}}%
\pgfpathcurveto{\pgfqpoint{1.552685in}{2.759561in}}{\pgfqpoint{1.555957in}{2.751661in}}{\pgfqpoint{1.561781in}{2.745837in}}%
\pgfpathcurveto{\pgfqpoint{1.567605in}{2.740014in}}{\pgfqpoint{1.575505in}{2.736741in}}{\pgfqpoint{1.583741in}{2.736741in}}%
\pgfpathclose%
\pgfusepath{stroke,fill}%
\end{pgfscope}%
\begin{pgfscope}%
\pgfpathrectangle{\pgfqpoint{0.100000in}{0.212622in}}{\pgfqpoint{3.696000in}{3.696000in}}%
\pgfusepath{clip}%
\pgfsetbuttcap%
\pgfsetroundjoin%
\definecolor{currentfill}{rgb}{0.121569,0.466667,0.705882}%
\pgfsetfillcolor{currentfill}%
\pgfsetfillopacity{0.647846}%
\pgfsetlinewidth{1.003750pt}%
\definecolor{currentstroke}{rgb}{0.121569,0.466667,0.705882}%
\pgfsetstrokecolor{currentstroke}%
\pgfsetstrokeopacity{0.647846}%
\pgfsetdash{}{0pt}%
\pgfpathmoveto{\pgfqpoint{1.582423in}{2.734556in}}%
\pgfpathcurveto{\pgfqpoint{1.590659in}{2.734556in}}{\pgfqpoint{1.598559in}{2.737828in}}{\pgfqpoint{1.604383in}{2.743652in}}%
\pgfpathcurveto{\pgfqpoint{1.610207in}{2.749476in}}{\pgfqpoint{1.613479in}{2.757376in}}{\pgfqpoint{1.613479in}{2.765612in}}%
\pgfpathcurveto{\pgfqpoint{1.613479in}{2.773849in}}{\pgfqpoint{1.610207in}{2.781749in}}{\pgfqpoint{1.604383in}{2.787573in}}%
\pgfpathcurveto{\pgfqpoint{1.598559in}{2.793397in}}{\pgfqpoint{1.590659in}{2.796669in}}{\pgfqpoint{1.582423in}{2.796669in}}%
\pgfpathcurveto{\pgfqpoint{1.574187in}{2.796669in}}{\pgfqpoint{1.566287in}{2.793397in}}{\pgfqpoint{1.560463in}{2.787573in}}%
\pgfpathcurveto{\pgfqpoint{1.554639in}{2.781749in}}{\pgfqpoint{1.551366in}{2.773849in}}{\pgfqpoint{1.551366in}{2.765612in}}%
\pgfpathcurveto{\pgfqpoint{1.551366in}{2.757376in}}{\pgfqpoint{1.554639in}{2.749476in}}{\pgfqpoint{1.560463in}{2.743652in}}%
\pgfpathcurveto{\pgfqpoint{1.566287in}{2.737828in}}{\pgfqpoint{1.574187in}{2.734556in}}{\pgfqpoint{1.582423in}{2.734556in}}%
\pgfpathclose%
\pgfusepath{stroke,fill}%
\end{pgfscope}%
\begin{pgfscope}%
\pgfpathrectangle{\pgfqpoint{0.100000in}{0.212622in}}{\pgfqpoint{3.696000in}{3.696000in}}%
\pgfusepath{clip}%
\pgfsetbuttcap%
\pgfsetroundjoin%
\definecolor{currentfill}{rgb}{0.121569,0.466667,0.705882}%
\pgfsetfillcolor{currentfill}%
\pgfsetfillopacity{0.648312}%
\pgfsetlinewidth{1.003750pt}%
\definecolor{currentstroke}{rgb}{0.121569,0.466667,0.705882}%
\pgfsetstrokecolor{currentstroke}%
\pgfsetstrokeopacity{0.648312}%
\pgfsetdash{}{0pt}%
\pgfpathmoveto{\pgfqpoint{3.127405in}{2.620149in}}%
\pgfpathcurveto{\pgfqpoint{3.135642in}{2.620149in}}{\pgfqpoint{3.143542in}{2.623422in}}{\pgfqpoint{3.149366in}{2.629246in}}%
\pgfpathcurveto{\pgfqpoint{3.155189in}{2.635070in}}{\pgfqpoint{3.158462in}{2.642970in}}{\pgfqpoint{3.158462in}{2.651206in}}%
\pgfpathcurveto{\pgfqpoint{3.158462in}{2.659442in}}{\pgfqpoint{3.155189in}{2.667342in}}{\pgfqpoint{3.149366in}{2.673166in}}%
\pgfpathcurveto{\pgfqpoint{3.143542in}{2.678990in}}{\pgfqpoint{3.135642in}{2.682262in}}{\pgfqpoint{3.127405in}{2.682262in}}%
\pgfpathcurveto{\pgfqpoint{3.119169in}{2.682262in}}{\pgfqpoint{3.111269in}{2.678990in}}{\pgfqpoint{3.105445in}{2.673166in}}%
\pgfpathcurveto{\pgfqpoint{3.099621in}{2.667342in}}{\pgfqpoint{3.096349in}{2.659442in}}{\pgfqpoint{3.096349in}{2.651206in}}%
\pgfpathcurveto{\pgfqpoint{3.096349in}{2.642970in}}{\pgfqpoint{3.099621in}{2.635070in}}{\pgfqpoint{3.105445in}{2.629246in}}%
\pgfpathcurveto{\pgfqpoint{3.111269in}{2.623422in}}{\pgfqpoint{3.119169in}{2.620149in}}{\pgfqpoint{3.127405in}{2.620149in}}%
\pgfpathclose%
\pgfusepath{stroke,fill}%
\end{pgfscope}%
\begin{pgfscope}%
\pgfpathrectangle{\pgfqpoint{0.100000in}{0.212622in}}{\pgfqpoint{3.696000in}{3.696000in}}%
\pgfusepath{clip}%
\pgfsetbuttcap%
\pgfsetroundjoin%
\definecolor{currentfill}{rgb}{0.121569,0.466667,0.705882}%
\pgfsetfillcolor{currentfill}%
\pgfsetfillopacity{0.648318}%
\pgfsetlinewidth{1.003750pt}%
\definecolor{currentstroke}{rgb}{0.121569,0.466667,0.705882}%
\pgfsetstrokecolor{currentstroke}%
\pgfsetstrokeopacity{0.648318}%
\pgfsetdash{}{0pt}%
\pgfpathmoveto{\pgfqpoint{1.581088in}{2.730050in}}%
\pgfpathcurveto{\pgfqpoint{1.589324in}{2.730050in}}{\pgfqpoint{1.597224in}{2.733322in}}{\pgfqpoint{1.603048in}{2.739146in}}%
\pgfpathcurveto{\pgfqpoint{1.608872in}{2.744970in}}{\pgfqpoint{1.612144in}{2.752870in}}{\pgfqpoint{1.612144in}{2.761107in}}%
\pgfpathcurveto{\pgfqpoint{1.612144in}{2.769343in}}{\pgfqpoint{1.608872in}{2.777243in}}{\pgfqpoint{1.603048in}{2.783067in}}%
\pgfpathcurveto{\pgfqpoint{1.597224in}{2.788891in}}{\pgfqpoint{1.589324in}{2.792163in}}{\pgfqpoint{1.581088in}{2.792163in}}%
\pgfpathcurveto{\pgfqpoint{1.572851in}{2.792163in}}{\pgfqpoint{1.564951in}{2.788891in}}{\pgfqpoint{1.559127in}{2.783067in}}%
\pgfpathcurveto{\pgfqpoint{1.553303in}{2.777243in}}{\pgfqpoint{1.550031in}{2.769343in}}{\pgfqpoint{1.550031in}{2.761107in}}%
\pgfpathcurveto{\pgfqpoint{1.550031in}{2.752870in}}{\pgfqpoint{1.553303in}{2.744970in}}{\pgfqpoint{1.559127in}{2.739146in}}%
\pgfpathcurveto{\pgfqpoint{1.564951in}{2.733322in}}{\pgfqpoint{1.572851in}{2.730050in}}{\pgfqpoint{1.581088in}{2.730050in}}%
\pgfpathclose%
\pgfusepath{stroke,fill}%
\end{pgfscope}%
\begin{pgfscope}%
\pgfpathrectangle{\pgfqpoint{0.100000in}{0.212622in}}{\pgfqpoint{3.696000in}{3.696000in}}%
\pgfusepath{clip}%
\pgfsetbuttcap%
\pgfsetroundjoin%
\definecolor{currentfill}{rgb}{0.121569,0.466667,0.705882}%
\pgfsetfillcolor{currentfill}%
\pgfsetfillopacity{0.648461}%
\pgfsetlinewidth{1.003750pt}%
\definecolor{currentstroke}{rgb}{0.121569,0.466667,0.705882}%
\pgfsetstrokecolor{currentstroke}%
\pgfsetstrokeopacity{0.648461}%
\pgfsetdash{}{0pt}%
\pgfpathmoveto{\pgfqpoint{1.938243in}{3.036504in}}%
\pgfpathcurveto{\pgfqpoint{1.946479in}{3.036504in}}{\pgfqpoint{1.954379in}{3.039776in}}{\pgfqpoint{1.960203in}{3.045600in}}%
\pgfpathcurveto{\pgfqpoint{1.966027in}{3.051424in}}{\pgfqpoint{1.969299in}{3.059324in}}{\pgfqpoint{1.969299in}{3.067561in}}%
\pgfpathcurveto{\pgfqpoint{1.969299in}{3.075797in}}{\pgfqpoint{1.966027in}{3.083697in}}{\pgfqpoint{1.960203in}{3.089521in}}%
\pgfpathcurveto{\pgfqpoint{1.954379in}{3.095345in}}{\pgfqpoint{1.946479in}{3.098617in}}{\pgfqpoint{1.938243in}{3.098617in}}%
\pgfpathcurveto{\pgfqpoint{1.930006in}{3.098617in}}{\pgfqpoint{1.922106in}{3.095345in}}{\pgfqpoint{1.916282in}{3.089521in}}%
\pgfpathcurveto{\pgfqpoint{1.910458in}{3.083697in}}{\pgfqpoint{1.907186in}{3.075797in}}{\pgfqpoint{1.907186in}{3.067561in}}%
\pgfpathcurveto{\pgfqpoint{1.907186in}{3.059324in}}{\pgfqpoint{1.910458in}{3.051424in}}{\pgfqpoint{1.916282in}{3.045600in}}%
\pgfpathcurveto{\pgfqpoint{1.922106in}{3.039776in}}{\pgfqpoint{1.930006in}{3.036504in}}{\pgfqpoint{1.938243in}{3.036504in}}%
\pgfpathclose%
\pgfusepath{stroke,fill}%
\end{pgfscope}%
\begin{pgfscope}%
\pgfpathrectangle{\pgfqpoint{0.100000in}{0.212622in}}{\pgfqpoint{3.696000in}{3.696000in}}%
\pgfusepath{clip}%
\pgfsetbuttcap%
\pgfsetroundjoin%
\definecolor{currentfill}{rgb}{0.121569,0.466667,0.705882}%
\pgfsetfillcolor{currentfill}%
\pgfsetfillopacity{0.648563}%
\pgfsetlinewidth{1.003750pt}%
\definecolor{currentstroke}{rgb}{0.121569,0.466667,0.705882}%
\pgfsetstrokecolor{currentstroke}%
\pgfsetstrokeopacity{0.648563}%
\pgfsetdash{}{0pt}%
\pgfpathmoveto{\pgfqpoint{1.580147in}{2.727651in}}%
\pgfpathcurveto{\pgfqpoint{1.588383in}{2.727651in}}{\pgfqpoint{1.596283in}{2.730923in}}{\pgfqpoint{1.602107in}{2.736747in}}%
\pgfpathcurveto{\pgfqpoint{1.607931in}{2.742571in}}{\pgfqpoint{1.611203in}{2.750471in}}{\pgfqpoint{1.611203in}{2.758707in}}%
\pgfpathcurveto{\pgfqpoint{1.611203in}{2.766943in}}{\pgfqpoint{1.607931in}{2.774844in}}{\pgfqpoint{1.602107in}{2.780667in}}%
\pgfpathcurveto{\pgfqpoint{1.596283in}{2.786491in}}{\pgfqpoint{1.588383in}{2.789764in}}{\pgfqpoint{1.580147in}{2.789764in}}%
\pgfpathcurveto{\pgfqpoint{1.571910in}{2.789764in}}{\pgfqpoint{1.564010in}{2.786491in}}{\pgfqpoint{1.558186in}{2.780667in}}%
\pgfpathcurveto{\pgfqpoint{1.552362in}{2.774844in}}{\pgfqpoint{1.549090in}{2.766943in}}{\pgfqpoint{1.549090in}{2.758707in}}%
\pgfpathcurveto{\pgfqpoint{1.549090in}{2.750471in}}{\pgfqpoint{1.552362in}{2.742571in}}{\pgfqpoint{1.558186in}{2.736747in}}%
\pgfpathcurveto{\pgfqpoint{1.564010in}{2.730923in}}{\pgfqpoint{1.571910in}{2.727651in}}{\pgfqpoint{1.580147in}{2.727651in}}%
\pgfpathclose%
\pgfusepath{stroke,fill}%
\end{pgfscope}%
\begin{pgfscope}%
\pgfpathrectangle{\pgfqpoint{0.100000in}{0.212622in}}{\pgfqpoint{3.696000in}{3.696000in}}%
\pgfusepath{clip}%
\pgfsetbuttcap%
\pgfsetroundjoin%
\definecolor{currentfill}{rgb}{0.121569,0.466667,0.705882}%
\pgfsetfillcolor{currentfill}%
\pgfsetfillopacity{0.648802}%
\pgfsetlinewidth{1.003750pt}%
\definecolor{currentstroke}{rgb}{0.121569,0.466667,0.705882}%
\pgfsetstrokecolor{currentstroke}%
\pgfsetstrokeopacity{0.648802}%
\pgfsetdash{}{0pt}%
\pgfpathmoveto{\pgfqpoint{1.578535in}{2.725359in}}%
\pgfpathcurveto{\pgfqpoint{1.586772in}{2.725359in}}{\pgfqpoint{1.594672in}{2.728631in}}{\pgfqpoint{1.600496in}{2.734455in}}%
\pgfpathcurveto{\pgfqpoint{1.606320in}{2.740279in}}{\pgfqpoint{1.609592in}{2.748179in}}{\pgfqpoint{1.609592in}{2.756415in}}%
\pgfpathcurveto{\pgfqpoint{1.609592in}{2.764652in}}{\pgfqpoint{1.606320in}{2.772552in}}{\pgfqpoint{1.600496in}{2.778376in}}%
\pgfpathcurveto{\pgfqpoint{1.594672in}{2.784200in}}{\pgfqpoint{1.586772in}{2.787472in}}{\pgfqpoint{1.578535in}{2.787472in}}%
\pgfpathcurveto{\pgfqpoint{1.570299in}{2.787472in}}{\pgfqpoint{1.562399in}{2.784200in}}{\pgfqpoint{1.556575in}{2.778376in}}%
\pgfpathcurveto{\pgfqpoint{1.550751in}{2.772552in}}{\pgfqpoint{1.547479in}{2.764652in}}{\pgfqpoint{1.547479in}{2.756415in}}%
\pgfpathcurveto{\pgfqpoint{1.547479in}{2.748179in}}{\pgfqpoint{1.550751in}{2.740279in}}{\pgfqpoint{1.556575in}{2.734455in}}%
\pgfpathcurveto{\pgfqpoint{1.562399in}{2.728631in}}{\pgfqpoint{1.570299in}{2.725359in}}{\pgfqpoint{1.578535in}{2.725359in}}%
\pgfpathclose%
\pgfusepath{stroke,fill}%
\end{pgfscope}%
\begin{pgfscope}%
\pgfpathrectangle{\pgfqpoint{0.100000in}{0.212622in}}{\pgfqpoint{3.696000in}{3.696000in}}%
\pgfusepath{clip}%
\pgfsetbuttcap%
\pgfsetroundjoin%
\definecolor{currentfill}{rgb}{0.121569,0.466667,0.705882}%
\pgfsetfillcolor{currentfill}%
\pgfsetfillopacity{0.649146}%
\pgfsetlinewidth{1.003750pt}%
\definecolor{currentstroke}{rgb}{0.121569,0.466667,0.705882}%
\pgfsetstrokecolor{currentstroke}%
\pgfsetstrokeopacity{0.649146}%
\pgfsetdash{}{0pt}%
\pgfpathmoveto{\pgfqpoint{1.576664in}{2.722551in}}%
\pgfpathcurveto{\pgfqpoint{1.584900in}{2.722551in}}{\pgfqpoint{1.592800in}{2.725823in}}{\pgfqpoint{1.598624in}{2.731647in}}%
\pgfpathcurveto{\pgfqpoint{1.604448in}{2.737471in}}{\pgfqpoint{1.607720in}{2.745371in}}{\pgfqpoint{1.607720in}{2.753607in}}%
\pgfpathcurveto{\pgfqpoint{1.607720in}{2.761844in}}{\pgfqpoint{1.604448in}{2.769744in}}{\pgfqpoint{1.598624in}{2.775568in}}%
\pgfpathcurveto{\pgfqpoint{1.592800in}{2.781391in}}{\pgfqpoint{1.584900in}{2.784664in}}{\pgfqpoint{1.576664in}{2.784664in}}%
\pgfpathcurveto{\pgfqpoint{1.568427in}{2.784664in}}{\pgfqpoint{1.560527in}{2.781391in}}{\pgfqpoint{1.554703in}{2.775568in}}%
\pgfpathcurveto{\pgfqpoint{1.548880in}{2.769744in}}{\pgfqpoint{1.545607in}{2.761844in}}{\pgfqpoint{1.545607in}{2.753607in}}%
\pgfpathcurveto{\pgfqpoint{1.545607in}{2.745371in}}{\pgfqpoint{1.548880in}{2.737471in}}{\pgfqpoint{1.554703in}{2.731647in}}%
\pgfpathcurveto{\pgfqpoint{1.560527in}{2.725823in}}{\pgfqpoint{1.568427in}{2.722551in}}{\pgfqpoint{1.576664in}{2.722551in}}%
\pgfpathclose%
\pgfusepath{stroke,fill}%
\end{pgfscope}%
\begin{pgfscope}%
\pgfpathrectangle{\pgfqpoint{0.100000in}{0.212622in}}{\pgfqpoint{3.696000in}{3.696000in}}%
\pgfusepath{clip}%
\pgfsetbuttcap%
\pgfsetroundjoin%
\definecolor{currentfill}{rgb}{0.121569,0.466667,0.705882}%
\pgfsetfillcolor{currentfill}%
\pgfsetfillopacity{0.649435}%
\pgfsetlinewidth{1.003750pt}%
\definecolor{currentstroke}{rgb}{0.121569,0.466667,0.705882}%
\pgfsetstrokecolor{currentstroke}%
\pgfsetstrokeopacity{0.649435}%
\pgfsetdash{}{0pt}%
\pgfpathmoveto{\pgfqpoint{1.942052in}{3.036477in}}%
\pgfpathcurveto{\pgfqpoint{1.950288in}{3.036477in}}{\pgfqpoint{1.958188in}{3.039749in}}{\pgfqpoint{1.964012in}{3.045573in}}%
\pgfpathcurveto{\pgfqpoint{1.969836in}{3.051397in}}{\pgfqpoint{1.973108in}{3.059297in}}{\pgfqpoint{1.973108in}{3.067533in}}%
\pgfpathcurveto{\pgfqpoint{1.973108in}{3.075770in}}{\pgfqpoint{1.969836in}{3.083670in}}{\pgfqpoint{1.964012in}{3.089494in}}%
\pgfpathcurveto{\pgfqpoint{1.958188in}{3.095318in}}{\pgfqpoint{1.950288in}{3.098590in}}{\pgfqpoint{1.942052in}{3.098590in}}%
\pgfpathcurveto{\pgfqpoint{1.933816in}{3.098590in}}{\pgfqpoint{1.925916in}{3.095318in}}{\pgfqpoint{1.920092in}{3.089494in}}%
\pgfpathcurveto{\pgfqpoint{1.914268in}{3.083670in}}{\pgfqpoint{1.910995in}{3.075770in}}{\pgfqpoint{1.910995in}{3.067533in}}%
\pgfpathcurveto{\pgfqpoint{1.910995in}{3.059297in}}{\pgfqpoint{1.914268in}{3.051397in}}{\pgfqpoint{1.920092in}{3.045573in}}%
\pgfpathcurveto{\pgfqpoint{1.925916in}{3.039749in}}{\pgfqpoint{1.933816in}{3.036477in}}{\pgfqpoint{1.942052in}{3.036477in}}%
\pgfpathclose%
\pgfusepath{stroke,fill}%
\end{pgfscope}%
\begin{pgfscope}%
\pgfpathrectangle{\pgfqpoint{0.100000in}{0.212622in}}{\pgfqpoint{3.696000in}{3.696000in}}%
\pgfusepath{clip}%
\pgfsetbuttcap%
\pgfsetroundjoin%
\definecolor{currentfill}{rgb}{0.121569,0.466667,0.705882}%
\pgfsetfillcolor{currentfill}%
\pgfsetfillopacity{0.649738}%
\pgfsetlinewidth{1.003750pt}%
\definecolor{currentstroke}{rgb}{0.121569,0.466667,0.705882}%
\pgfsetstrokecolor{currentstroke}%
\pgfsetstrokeopacity{0.649738}%
\pgfsetdash{}{0pt}%
\pgfpathmoveto{\pgfqpoint{1.574596in}{2.717324in}}%
\pgfpathcurveto{\pgfqpoint{1.582832in}{2.717324in}}{\pgfqpoint{1.590732in}{2.720596in}}{\pgfqpoint{1.596556in}{2.726420in}}%
\pgfpathcurveto{\pgfqpoint{1.602380in}{2.732244in}}{\pgfqpoint{1.605652in}{2.740144in}}{\pgfqpoint{1.605652in}{2.748380in}}%
\pgfpathcurveto{\pgfqpoint{1.605652in}{2.756617in}}{\pgfqpoint{1.602380in}{2.764517in}}{\pgfqpoint{1.596556in}{2.770341in}}%
\pgfpathcurveto{\pgfqpoint{1.590732in}{2.776165in}}{\pgfqpoint{1.582832in}{2.779437in}}{\pgfqpoint{1.574596in}{2.779437in}}%
\pgfpathcurveto{\pgfqpoint{1.566359in}{2.779437in}}{\pgfqpoint{1.558459in}{2.776165in}}{\pgfqpoint{1.552635in}{2.770341in}}%
\pgfpathcurveto{\pgfqpoint{1.546811in}{2.764517in}}{\pgfqpoint{1.543539in}{2.756617in}}{\pgfqpoint{1.543539in}{2.748380in}}%
\pgfpathcurveto{\pgfqpoint{1.543539in}{2.740144in}}{\pgfqpoint{1.546811in}{2.732244in}}{\pgfqpoint{1.552635in}{2.726420in}}%
\pgfpathcurveto{\pgfqpoint{1.558459in}{2.720596in}}{\pgfqpoint{1.566359in}{2.717324in}}{\pgfqpoint{1.574596in}{2.717324in}}%
\pgfpathclose%
\pgfusepath{stroke,fill}%
\end{pgfscope}%
\begin{pgfscope}%
\pgfpathrectangle{\pgfqpoint{0.100000in}{0.212622in}}{\pgfqpoint{3.696000in}{3.696000in}}%
\pgfusepath{clip}%
\pgfsetbuttcap%
\pgfsetroundjoin%
\definecolor{currentfill}{rgb}{0.121569,0.466667,0.705882}%
\pgfsetfillcolor{currentfill}%
\pgfsetfillopacity{0.650020}%
\pgfsetlinewidth{1.003750pt}%
\definecolor{currentstroke}{rgb}{0.121569,0.466667,0.705882}%
\pgfsetstrokecolor{currentstroke}%
\pgfsetstrokeopacity{0.650020}%
\pgfsetdash{}{0pt}%
\pgfpathmoveto{\pgfqpoint{1.573522in}{2.714196in}}%
\pgfpathcurveto{\pgfqpoint{1.581758in}{2.714196in}}{\pgfqpoint{1.589658in}{2.717468in}}{\pgfqpoint{1.595482in}{2.723292in}}%
\pgfpathcurveto{\pgfqpoint{1.601306in}{2.729116in}}{\pgfqpoint{1.604578in}{2.737016in}}{\pgfqpoint{1.604578in}{2.745253in}}%
\pgfpathcurveto{\pgfqpoint{1.604578in}{2.753489in}}{\pgfqpoint{1.601306in}{2.761389in}}{\pgfqpoint{1.595482in}{2.767213in}}%
\pgfpathcurveto{\pgfqpoint{1.589658in}{2.773037in}}{\pgfqpoint{1.581758in}{2.776309in}}{\pgfqpoint{1.573522in}{2.776309in}}%
\pgfpathcurveto{\pgfqpoint{1.565285in}{2.776309in}}{\pgfqpoint{1.557385in}{2.773037in}}{\pgfqpoint{1.551561in}{2.767213in}}%
\pgfpathcurveto{\pgfqpoint{1.545737in}{2.761389in}}{\pgfqpoint{1.542465in}{2.753489in}}{\pgfqpoint{1.542465in}{2.745253in}}%
\pgfpathcurveto{\pgfqpoint{1.542465in}{2.737016in}}{\pgfqpoint{1.545737in}{2.729116in}}{\pgfqpoint{1.551561in}{2.723292in}}%
\pgfpathcurveto{\pgfqpoint{1.557385in}{2.717468in}}{\pgfqpoint{1.565285in}{2.714196in}}{\pgfqpoint{1.573522in}{2.714196in}}%
\pgfpathclose%
\pgfusepath{stroke,fill}%
\end{pgfscope}%
\begin{pgfscope}%
\pgfpathrectangle{\pgfqpoint{0.100000in}{0.212622in}}{\pgfqpoint{3.696000in}{3.696000in}}%
\pgfusepath{clip}%
\pgfsetbuttcap%
\pgfsetroundjoin%
\definecolor{currentfill}{rgb}{0.121569,0.466667,0.705882}%
\pgfsetfillcolor{currentfill}%
\pgfsetfillopacity{0.650117}%
\pgfsetlinewidth{1.003750pt}%
\definecolor{currentstroke}{rgb}{0.121569,0.466667,0.705882}%
\pgfsetstrokecolor{currentstroke}%
\pgfsetstrokeopacity{0.650117}%
\pgfsetdash{}{0pt}%
\pgfpathmoveto{\pgfqpoint{1.945519in}{3.036148in}}%
\pgfpathcurveto{\pgfqpoint{1.953756in}{3.036148in}}{\pgfqpoint{1.961656in}{3.039420in}}{\pgfqpoint{1.967480in}{3.045244in}}%
\pgfpathcurveto{\pgfqpoint{1.973304in}{3.051068in}}{\pgfqpoint{1.976576in}{3.058968in}}{\pgfqpoint{1.976576in}{3.067204in}}%
\pgfpathcurveto{\pgfqpoint{1.976576in}{3.075441in}}{\pgfqpoint{1.973304in}{3.083341in}}{\pgfqpoint{1.967480in}{3.089165in}}%
\pgfpathcurveto{\pgfqpoint{1.961656in}{3.094988in}}{\pgfqpoint{1.953756in}{3.098261in}}{\pgfqpoint{1.945519in}{3.098261in}}%
\pgfpathcurveto{\pgfqpoint{1.937283in}{3.098261in}}{\pgfqpoint{1.929383in}{3.094988in}}{\pgfqpoint{1.923559in}{3.089165in}}%
\pgfpathcurveto{\pgfqpoint{1.917735in}{3.083341in}}{\pgfqpoint{1.914463in}{3.075441in}}{\pgfqpoint{1.914463in}{3.067204in}}%
\pgfpathcurveto{\pgfqpoint{1.914463in}{3.058968in}}{\pgfqpoint{1.917735in}{3.051068in}}{\pgfqpoint{1.923559in}{3.045244in}}%
\pgfpathcurveto{\pgfqpoint{1.929383in}{3.039420in}}{\pgfqpoint{1.937283in}{3.036148in}}{\pgfqpoint{1.945519in}{3.036148in}}%
\pgfpathclose%
\pgfusepath{stroke,fill}%
\end{pgfscope}%
\begin{pgfscope}%
\pgfpathrectangle{\pgfqpoint{0.100000in}{0.212622in}}{\pgfqpoint{3.696000in}{3.696000in}}%
\pgfusepath{clip}%
\pgfsetbuttcap%
\pgfsetroundjoin%
\definecolor{currentfill}{rgb}{0.121569,0.466667,0.705882}%
\pgfsetfillcolor{currentfill}%
\pgfsetfillopacity{0.650392}%
\pgfsetlinewidth{1.003750pt}%
\definecolor{currentstroke}{rgb}{0.121569,0.466667,0.705882}%
\pgfsetstrokecolor{currentstroke}%
\pgfsetstrokeopacity{0.650392}%
\pgfsetdash{}{0pt}%
\pgfpathmoveto{\pgfqpoint{1.571110in}{2.710215in}}%
\pgfpathcurveto{\pgfqpoint{1.579346in}{2.710215in}}{\pgfqpoint{1.587246in}{2.713487in}}{\pgfqpoint{1.593070in}{2.719311in}}%
\pgfpathcurveto{\pgfqpoint{1.598894in}{2.725135in}}{\pgfqpoint{1.602167in}{2.733035in}}{\pgfqpoint{1.602167in}{2.741271in}}%
\pgfpathcurveto{\pgfqpoint{1.602167in}{2.749507in}}{\pgfqpoint{1.598894in}{2.757407in}}{\pgfqpoint{1.593070in}{2.763231in}}%
\pgfpathcurveto{\pgfqpoint{1.587246in}{2.769055in}}{\pgfqpoint{1.579346in}{2.772328in}}{\pgfqpoint{1.571110in}{2.772328in}}%
\pgfpathcurveto{\pgfqpoint{1.562874in}{2.772328in}}{\pgfqpoint{1.554974in}{2.769055in}}{\pgfqpoint{1.549150in}{2.763231in}}%
\pgfpathcurveto{\pgfqpoint{1.543326in}{2.757407in}}{\pgfqpoint{1.540054in}{2.749507in}}{\pgfqpoint{1.540054in}{2.741271in}}%
\pgfpathcurveto{\pgfqpoint{1.540054in}{2.733035in}}{\pgfqpoint{1.543326in}{2.725135in}}{\pgfqpoint{1.549150in}{2.719311in}}%
\pgfpathcurveto{\pgfqpoint{1.554974in}{2.713487in}}{\pgfqpoint{1.562874in}{2.710215in}}{\pgfqpoint{1.571110in}{2.710215in}}%
\pgfpathclose%
\pgfusepath{stroke,fill}%
\end{pgfscope}%
\begin{pgfscope}%
\pgfpathrectangle{\pgfqpoint{0.100000in}{0.212622in}}{\pgfqpoint{3.696000in}{3.696000in}}%
\pgfusepath{clip}%
\pgfsetbuttcap%
\pgfsetroundjoin%
\definecolor{currentfill}{rgb}{0.121569,0.466667,0.705882}%
\pgfsetfillcolor{currentfill}%
\pgfsetfillopacity{0.650403}%
\pgfsetlinewidth{1.003750pt}%
\definecolor{currentstroke}{rgb}{0.121569,0.466667,0.705882}%
\pgfsetstrokecolor{currentstroke}%
\pgfsetstrokeopacity{0.650403}%
\pgfsetdash{}{0pt}%
\pgfpathmoveto{\pgfqpoint{1.947527in}{3.035938in}}%
\pgfpathcurveto{\pgfqpoint{1.955763in}{3.035938in}}{\pgfqpoint{1.963663in}{3.039211in}}{\pgfqpoint{1.969487in}{3.045035in}}%
\pgfpathcurveto{\pgfqpoint{1.975311in}{3.050859in}}{\pgfqpoint{1.978583in}{3.058759in}}{\pgfqpoint{1.978583in}{3.066995in}}%
\pgfpathcurveto{\pgfqpoint{1.978583in}{3.075231in}}{\pgfqpoint{1.975311in}{3.083131in}}{\pgfqpoint{1.969487in}{3.088955in}}%
\pgfpathcurveto{\pgfqpoint{1.963663in}{3.094779in}}{\pgfqpoint{1.955763in}{3.098051in}}{\pgfqpoint{1.947527in}{3.098051in}}%
\pgfpathcurveto{\pgfqpoint{1.939290in}{3.098051in}}{\pgfqpoint{1.931390in}{3.094779in}}{\pgfqpoint{1.925566in}{3.088955in}}%
\pgfpathcurveto{\pgfqpoint{1.919742in}{3.083131in}}{\pgfqpoint{1.916470in}{3.075231in}}{\pgfqpoint{1.916470in}{3.066995in}}%
\pgfpathcurveto{\pgfqpoint{1.916470in}{3.058759in}}{\pgfqpoint{1.919742in}{3.050859in}}{\pgfqpoint{1.925566in}{3.045035in}}%
\pgfpathcurveto{\pgfqpoint{1.931390in}{3.039211in}}{\pgfqpoint{1.939290in}{3.035938in}}{\pgfqpoint{1.947527in}{3.035938in}}%
\pgfpathclose%
\pgfusepath{stroke,fill}%
\end{pgfscope}%
\begin{pgfscope}%
\pgfpathrectangle{\pgfqpoint{0.100000in}{0.212622in}}{\pgfqpoint{3.696000in}{3.696000in}}%
\pgfusepath{clip}%
\pgfsetbuttcap%
\pgfsetroundjoin%
\definecolor{currentfill}{rgb}{0.121569,0.466667,0.705882}%
\pgfsetfillcolor{currentfill}%
\pgfsetfillopacity{0.650599}%
\pgfsetlinewidth{1.003750pt}%
\definecolor{currentstroke}{rgb}{0.121569,0.466667,0.705882}%
\pgfsetstrokecolor{currentstroke}%
\pgfsetstrokeopacity{0.650599}%
\pgfsetdash{}{0pt}%
\pgfpathmoveto{\pgfqpoint{1.569735in}{2.708156in}}%
\pgfpathcurveto{\pgfqpoint{1.577971in}{2.708156in}}{\pgfqpoint{1.585872in}{2.711428in}}{\pgfqpoint{1.591695in}{2.717252in}}%
\pgfpathcurveto{\pgfqpoint{1.597519in}{2.723076in}}{\pgfqpoint{1.600792in}{2.730976in}}{\pgfqpoint{1.600792in}{2.739212in}}%
\pgfpathcurveto{\pgfqpoint{1.600792in}{2.747449in}}{\pgfqpoint{1.597519in}{2.755349in}}{\pgfqpoint{1.591695in}{2.761173in}}%
\pgfpathcurveto{\pgfqpoint{1.585872in}{2.766996in}}{\pgfqpoint{1.577971in}{2.770269in}}{\pgfqpoint{1.569735in}{2.770269in}}%
\pgfpathcurveto{\pgfqpoint{1.561499in}{2.770269in}}{\pgfqpoint{1.553599in}{2.766996in}}{\pgfqpoint{1.547775in}{2.761173in}}%
\pgfpathcurveto{\pgfqpoint{1.541951in}{2.755349in}}{\pgfqpoint{1.538679in}{2.747449in}}{\pgfqpoint{1.538679in}{2.739212in}}%
\pgfpathcurveto{\pgfqpoint{1.538679in}{2.730976in}}{\pgfqpoint{1.541951in}{2.723076in}}{\pgfqpoint{1.547775in}{2.717252in}}%
\pgfpathcurveto{\pgfqpoint{1.553599in}{2.711428in}}{\pgfqpoint{1.561499in}{2.708156in}}{\pgfqpoint{1.569735in}{2.708156in}}%
\pgfpathclose%
\pgfusepath{stroke,fill}%
\end{pgfscope}%
\begin{pgfscope}%
\pgfpathrectangle{\pgfqpoint{0.100000in}{0.212622in}}{\pgfqpoint{3.696000in}{3.696000in}}%
\pgfusepath{clip}%
\pgfsetbuttcap%
\pgfsetroundjoin%
\definecolor{currentfill}{rgb}{0.121569,0.466667,0.705882}%
\pgfsetfillcolor{currentfill}%
\pgfsetfillopacity{0.650879}%
\pgfsetlinewidth{1.003750pt}%
\definecolor{currentstroke}{rgb}{0.121569,0.466667,0.705882}%
\pgfsetstrokecolor{currentstroke}%
\pgfsetstrokeopacity{0.650879}%
\pgfsetdash{}{0pt}%
\pgfpathmoveto{\pgfqpoint{1.951143in}{3.035218in}}%
\pgfpathcurveto{\pgfqpoint{1.959380in}{3.035218in}}{\pgfqpoint{1.967280in}{3.038490in}}{\pgfqpoint{1.973104in}{3.044314in}}%
\pgfpathcurveto{\pgfqpoint{1.978928in}{3.050138in}}{\pgfqpoint{1.982200in}{3.058038in}}{\pgfqpoint{1.982200in}{3.066274in}}%
\pgfpathcurveto{\pgfqpoint{1.982200in}{3.074511in}}{\pgfqpoint{1.978928in}{3.082411in}}{\pgfqpoint{1.973104in}{3.088235in}}%
\pgfpathcurveto{\pgfqpoint{1.967280in}{3.094059in}}{\pgfqpoint{1.959380in}{3.097331in}}{\pgfqpoint{1.951143in}{3.097331in}}%
\pgfpathcurveto{\pgfqpoint{1.942907in}{3.097331in}}{\pgfqpoint{1.935007in}{3.094059in}}{\pgfqpoint{1.929183in}{3.088235in}}%
\pgfpathcurveto{\pgfqpoint{1.923359in}{3.082411in}}{\pgfqpoint{1.920087in}{3.074511in}}{\pgfqpoint{1.920087in}{3.066274in}}%
\pgfpathcurveto{\pgfqpoint{1.920087in}{3.058038in}}{\pgfqpoint{1.923359in}{3.050138in}}{\pgfqpoint{1.929183in}{3.044314in}}%
\pgfpathcurveto{\pgfqpoint{1.935007in}{3.038490in}}{\pgfqpoint{1.942907in}{3.035218in}}{\pgfqpoint{1.951143in}{3.035218in}}%
\pgfpathclose%
\pgfusepath{stroke,fill}%
\end{pgfscope}%
\begin{pgfscope}%
\pgfpathrectangle{\pgfqpoint{0.100000in}{0.212622in}}{\pgfqpoint{3.696000in}{3.696000in}}%
\pgfusepath{clip}%
\pgfsetbuttcap%
\pgfsetroundjoin%
\definecolor{currentfill}{rgb}{0.121569,0.466667,0.705882}%
\pgfsetfillcolor{currentfill}%
\pgfsetfillopacity{0.651056}%
\pgfsetlinewidth{1.003750pt}%
\definecolor{currentstroke}{rgb}{0.121569,0.466667,0.705882}%
\pgfsetstrokecolor{currentstroke}%
\pgfsetstrokeopacity{0.651056}%
\pgfsetdash{}{0pt}%
\pgfpathmoveto{\pgfqpoint{1.567916in}{2.703969in}}%
\pgfpathcurveto{\pgfqpoint{1.576152in}{2.703969in}}{\pgfqpoint{1.584052in}{2.707241in}}{\pgfqpoint{1.589876in}{2.713065in}}%
\pgfpathcurveto{\pgfqpoint{1.595700in}{2.718889in}}{\pgfqpoint{1.598972in}{2.726789in}}{\pgfqpoint{1.598972in}{2.735025in}}%
\pgfpathcurveto{\pgfqpoint{1.598972in}{2.743261in}}{\pgfqpoint{1.595700in}{2.751161in}}{\pgfqpoint{1.589876in}{2.756985in}}%
\pgfpathcurveto{\pgfqpoint{1.584052in}{2.762809in}}{\pgfqpoint{1.576152in}{2.766082in}}{\pgfqpoint{1.567916in}{2.766082in}}%
\pgfpathcurveto{\pgfqpoint{1.559680in}{2.766082in}}{\pgfqpoint{1.551779in}{2.762809in}}{\pgfqpoint{1.545956in}{2.756985in}}%
\pgfpathcurveto{\pgfqpoint{1.540132in}{2.751161in}}{\pgfqpoint{1.536859in}{2.743261in}}{\pgfqpoint{1.536859in}{2.735025in}}%
\pgfpathcurveto{\pgfqpoint{1.536859in}{2.726789in}}{\pgfqpoint{1.540132in}{2.718889in}}{\pgfqpoint{1.545956in}{2.713065in}}%
\pgfpathcurveto{\pgfqpoint{1.551779in}{2.707241in}}{\pgfqpoint{1.559680in}{2.703969in}}{\pgfqpoint{1.567916in}{2.703969in}}%
\pgfpathclose%
\pgfusepath{stroke,fill}%
\end{pgfscope}%
\begin{pgfscope}%
\pgfpathrectangle{\pgfqpoint{0.100000in}{0.212622in}}{\pgfqpoint{3.696000in}{3.696000in}}%
\pgfusepath{clip}%
\pgfsetbuttcap%
\pgfsetroundjoin%
\definecolor{currentfill}{rgb}{0.121569,0.466667,0.705882}%
\pgfsetfillcolor{currentfill}%
\pgfsetfillopacity{0.651284}%
\pgfsetlinewidth{1.003750pt}%
\definecolor{currentstroke}{rgb}{0.121569,0.466667,0.705882}%
\pgfsetstrokecolor{currentstroke}%
\pgfsetstrokeopacity{0.651284}%
\pgfsetdash{}{0pt}%
\pgfpathmoveto{\pgfqpoint{3.138019in}{2.620602in}}%
\pgfpathcurveto{\pgfqpoint{3.146255in}{2.620602in}}{\pgfqpoint{3.154155in}{2.623874in}}{\pgfqpoint{3.159979in}{2.629698in}}%
\pgfpathcurveto{\pgfqpoint{3.165803in}{2.635522in}}{\pgfqpoint{3.169075in}{2.643422in}}{\pgfqpoint{3.169075in}{2.651658in}}%
\pgfpathcurveto{\pgfqpoint{3.169075in}{2.659894in}}{\pgfqpoint{3.165803in}{2.667795in}}{\pgfqpoint{3.159979in}{2.673618in}}%
\pgfpathcurveto{\pgfqpoint{3.154155in}{2.679442in}}{\pgfqpoint{3.146255in}{2.682715in}}{\pgfqpoint{3.138019in}{2.682715in}}%
\pgfpathcurveto{\pgfqpoint{3.129783in}{2.682715in}}{\pgfqpoint{3.121883in}{2.679442in}}{\pgfqpoint{3.116059in}{2.673618in}}%
\pgfpathcurveto{\pgfqpoint{3.110235in}{2.667795in}}{\pgfqpoint{3.106962in}{2.659894in}}{\pgfqpoint{3.106962in}{2.651658in}}%
\pgfpathcurveto{\pgfqpoint{3.106962in}{2.643422in}}{\pgfqpoint{3.110235in}{2.635522in}}{\pgfqpoint{3.116059in}{2.629698in}}%
\pgfpathcurveto{\pgfqpoint{3.121883in}{2.623874in}}{\pgfqpoint{3.129783in}{2.620602in}}{\pgfqpoint{3.138019in}{2.620602in}}%
\pgfpathclose%
\pgfusepath{stroke,fill}%
\end{pgfscope}%
\begin{pgfscope}%
\pgfpathrectangle{\pgfqpoint{0.100000in}{0.212622in}}{\pgfqpoint{3.696000in}{3.696000in}}%
\pgfusepath{clip}%
\pgfsetbuttcap%
\pgfsetroundjoin%
\definecolor{currentfill}{rgb}{0.121569,0.466667,0.705882}%
\pgfsetfillcolor{currentfill}%
\pgfsetfillopacity{0.651414}%
\pgfsetlinewidth{1.003750pt}%
\definecolor{currentstroke}{rgb}{0.121569,0.466667,0.705882}%
\pgfsetstrokecolor{currentstroke}%
\pgfsetstrokeopacity{0.651414}%
\pgfsetdash{}{0pt}%
\pgfpathmoveto{\pgfqpoint{1.952892in}{3.035590in}}%
\pgfpathcurveto{\pgfqpoint{1.961128in}{3.035590in}}{\pgfqpoint{1.969028in}{3.038862in}}{\pgfqpoint{1.974852in}{3.044686in}}%
\pgfpathcurveto{\pgfqpoint{1.980676in}{3.050510in}}{\pgfqpoint{1.983948in}{3.058410in}}{\pgfqpoint{1.983948in}{3.066646in}}%
\pgfpathcurveto{\pgfqpoint{1.983948in}{3.074883in}}{\pgfqpoint{1.980676in}{3.082783in}}{\pgfqpoint{1.974852in}{3.088607in}}%
\pgfpathcurveto{\pgfqpoint{1.969028in}{3.094431in}}{\pgfqpoint{1.961128in}{3.097703in}}{\pgfqpoint{1.952892in}{3.097703in}}%
\pgfpathcurveto{\pgfqpoint{1.944655in}{3.097703in}}{\pgfqpoint{1.936755in}{3.094431in}}{\pgfqpoint{1.930931in}{3.088607in}}%
\pgfpathcurveto{\pgfqpoint{1.925107in}{3.082783in}}{\pgfqpoint{1.921835in}{3.074883in}}{\pgfqpoint{1.921835in}{3.066646in}}%
\pgfpathcurveto{\pgfqpoint{1.921835in}{3.058410in}}{\pgfqpoint{1.925107in}{3.050510in}}{\pgfqpoint{1.930931in}{3.044686in}}%
\pgfpathcurveto{\pgfqpoint{1.936755in}{3.038862in}}{\pgfqpoint{1.944655in}{3.035590in}}{\pgfqpoint{1.952892in}{3.035590in}}%
\pgfpathclose%
\pgfusepath{stroke,fill}%
\end{pgfscope}%
\begin{pgfscope}%
\pgfpathrectangle{\pgfqpoint{0.100000in}{0.212622in}}{\pgfqpoint{3.696000in}{3.696000in}}%
\pgfusepath{clip}%
\pgfsetbuttcap%
\pgfsetroundjoin%
\definecolor{currentfill}{rgb}{0.121569,0.466667,0.705882}%
\pgfsetfillcolor{currentfill}%
\pgfsetfillopacity{0.651610}%
\pgfsetlinewidth{1.003750pt}%
\definecolor{currentstroke}{rgb}{0.121569,0.466667,0.705882}%
\pgfsetstrokecolor{currentstroke}%
\pgfsetstrokeopacity{0.651610}%
\pgfsetdash{}{0pt}%
\pgfpathmoveto{\pgfqpoint{1.566232in}{2.698290in}}%
\pgfpathcurveto{\pgfqpoint{1.574468in}{2.698290in}}{\pgfqpoint{1.582368in}{2.701562in}}{\pgfqpoint{1.588192in}{2.707386in}}%
\pgfpathcurveto{\pgfqpoint{1.594016in}{2.713210in}}{\pgfqpoint{1.597288in}{2.721110in}}{\pgfqpoint{1.597288in}{2.729347in}}%
\pgfpathcurveto{\pgfqpoint{1.597288in}{2.737583in}}{\pgfqpoint{1.594016in}{2.745483in}}{\pgfqpoint{1.588192in}{2.751307in}}%
\pgfpathcurveto{\pgfqpoint{1.582368in}{2.757131in}}{\pgfqpoint{1.574468in}{2.760403in}}{\pgfqpoint{1.566232in}{2.760403in}}%
\pgfpathcurveto{\pgfqpoint{1.557995in}{2.760403in}}{\pgfqpoint{1.550095in}{2.757131in}}{\pgfqpoint{1.544271in}{2.751307in}}%
\pgfpathcurveto{\pgfqpoint{1.538447in}{2.745483in}}{\pgfqpoint{1.535175in}{2.737583in}}{\pgfqpoint{1.535175in}{2.729347in}}%
\pgfpathcurveto{\pgfqpoint{1.535175in}{2.721110in}}{\pgfqpoint{1.538447in}{2.713210in}}{\pgfqpoint{1.544271in}{2.707386in}}%
\pgfpathcurveto{\pgfqpoint{1.550095in}{2.701562in}}{\pgfqpoint{1.557995in}{2.698290in}}{\pgfqpoint{1.566232in}{2.698290in}}%
\pgfpathclose%
\pgfusepath{stroke,fill}%
\end{pgfscope}%
\begin{pgfscope}%
\pgfpathrectangle{\pgfqpoint{0.100000in}{0.212622in}}{\pgfqpoint{3.696000in}{3.696000in}}%
\pgfusepath{clip}%
\pgfsetbuttcap%
\pgfsetroundjoin%
\definecolor{currentfill}{rgb}{0.121569,0.466667,0.705882}%
\pgfsetfillcolor{currentfill}%
\pgfsetfillopacity{0.652059}%
\pgfsetlinewidth{1.003750pt}%
\definecolor{currentstroke}{rgb}{0.121569,0.466667,0.705882}%
\pgfsetstrokecolor{currentstroke}%
\pgfsetstrokeopacity{0.652059}%
\pgfsetdash{}{0pt}%
\pgfpathmoveto{\pgfqpoint{1.956148in}{3.034990in}}%
\pgfpathcurveto{\pgfqpoint{1.964384in}{3.034990in}}{\pgfqpoint{1.972284in}{3.038262in}}{\pgfqpoint{1.978108in}{3.044086in}}%
\pgfpathcurveto{\pgfqpoint{1.983932in}{3.049910in}}{\pgfqpoint{1.987204in}{3.057810in}}{\pgfqpoint{1.987204in}{3.066047in}}%
\pgfpathcurveto{\pgfqpoint{1.987204in}{3.074283in}}{\pgfqpoint{1.983932in}{3.082183in}}{\pgfqpoint{1.978108in}{3.088007in}}%
\pgfpathcurveto{\pgfqpoint{1.972284in}{3.093831in}}{\pgfqpoint{1.964384in}{3.097103in}}{\pgfqpoint{1.956148in}{3.097103in}}%
\pgfpathcurveto{\pgfqpoint{1.947912in}{3.097103in}}{\pgfqpoint{1.940012in}{3.093831in}}{\pgfqpoint{1.934188in}{3.088007in}}%
\pgfpathcurveto{\pgfqpoint{1.928364in}{3.082183in}}{\pgfqpoint{1.925091in}{3.074283in}}{\pgfqpoint{1.925091in}{3.066047in}}%
\pgfpathcurveto{\pgfqpoint{1.925091in}{3.057810in}}{\pgfqpoint{1.928364in}{3.049910in}}{\pgfqpoint{1.934188in}{3.044086in}}%
\pgfpathcurveto{\pgfqpoint{1.940012in}{3.038262in}}{\pgfqpoint{1.947912in}{3.034990in}}{\pgfqpoint{1.956148in}{3.034990in}}%
\pgfpathclose%
\pgfusepath{stroke,fill}%
\end{pgfscope}%
\begin{pgfscope}%
\pgfpathrectangle{\pgfqpoint{0.100000in}{0.212622in}}{\pgfqpoint{3.696000in}{3.696000in}}%
\pgfusepath{clip}%
\pgfsetbuttcap%
\pgfsetroundjoin%
\definecolor{currentfill}{rgb}{0.121569,0.466667,0.705882}%
\pgfsetfillcolor{currentfill}%
\pgfsetfillopacity{0.652198}%
\pgfsetlinewidth{1.003750pt}%
\definecolor{currentstroke}{rgb}{0.121569,0.466667,0.705882}%
\pgfsetstrokecolor{currentstroke}%
\pgfsetstrokeopacity{0.652198}%
\pgfsetdash{}{0pt}%
\pgfpathmoveto{\pgfqpoint{1.563111in}{2.692475in}}%
\pgfpathcurveto{\pgfqpoint{1.571347in}{2.692475in}}{\pgfqpoint{1.579247in}{2.695747in}}{\pgfqpoint{1.585071in}{2.701571in}}%
\pgfpathcurveto{\pgfqpoint{1.590895in}{2.707395in}}{\pgfqpoint{1.594168in}{2.715295in}}{\pgfqpoint{1.594168in}{2.723531in}}%
\pgfpathcurveto{\pgfqpoint{1.594168in}{2.731767in}}{\pgfqpoint{1.590895in}{2.739667in}}{\pgfqpoint{1.585071in}{2.745491in}}%
\pgfpathcurveto{\pgfqpoint{1.579247in}{2.751315in}}{\pgfqpoint{1.571347in}{2.754588in}}{\pgfqpoint{1.563111in}{2.754588in}}%
\pgfpathcurveto{\pgfqpoint{1.554875in}{2.754588in}}{\pgfqpoint{1.546975in}{2.751315in}}{\pgfqpoint{1.541151in}{2.745491in}}%
\pgfpathcurveto{\pgfqpoint{1.535327in}{2.739667in}}{\pgfqpoint{1.532055in}{2.731767in}}{\pgfqpoint{1.532055in}{2.723531in}}%
\pgfpathcurveto{\pgfqpoint{1.532055in}{2.715295in}}{\pgfqpoint{1.535327in}{2.707395in}}{\pgfqpoint{1.541151in}{2.701571in}}%
\pgfpathcurveto{\pgfqpoint{1.546975in}{2.695747in}}{\pgfqpoint{1.554875in}{2.692475in}}{\pgfqpoint{1.563111in}{2.692475in}}%
\pgfpathclose%
\pgfusepath{stroke,fill}%
\end{pgfscope}%
\begin{pgfscope}%
\pgfpathrectangle{\pgfqpoint{0.100000in}{0.212622in}}{\pgfqpoint{3.696000in}{3.696000in}}%
\pgfusepath{clip}%
\pgfsetbuttcap%
\pgfsetroundjoin%
\definecolor{currentfill}{rgb}{0.121569,0.466667,0.705882}%
\pgfsetfillcolor{currentfill}%
\pgfsetfillopacity{0.652469}%
\pgfsetlinewidth{1.003750pt}%
\definecolor{currentstroke}{rgb}{0.121569,0.466667,0.705882}%
\pgfsetstrokecolor{currentstroke}%
\pgfsetstrokeopacity{0.652469}%
\pgfsetdash{}{0pt}%
\pgfpathmoveto{\pgfqpoint{1.561138in}{2.689613in}}%
\pgfpathcurveto{\pgfqpoint{1.569375in}{2.689613in}}{\pgfqpoint{1.577275in}{2.692885in}}{\pgfqpoint{1.583099in}{2.698709in}}%
\pgfpathcurveto{\pgfqpoint{1.588923in}{2.704533in}}{\pgfqpoint{1.592195in}{2.712433in}}{\pgfqpoint{1.592195in}{2.720670in}}%
\pgfpathcurveto{\pgfqpoint{1.592195in}{2.728906in}}{\pgfqpoint{1.588923in}{2.736806in}}{\pgfqpoint{1.583099in}{2.742630in}}%
\pgfpathcurveto{\pgfqpoint{1.577275in}{2.748454in}}{\pgfqpoint{1.569375in}{2.751726in}}{\pgfqpoint{1.561138in}{2.751726in}}%
\pgfpathcurveto{\pgfqpoint{1.552902in}{2.751726in}}{\pgfqpoint{1.545002in}{2.748454in}}{\pgfqpoint{1.539178in}{2.742630in}}%
\pgfpathcurveto{\pgfqpoint{1.533354in}{2.736806in}}{\pgfqpoint{1.530082in}{2.728906in}}{\pgfqpoint{1.530082in}{2.720670in}}%
\pgfpathcurveto{\pgfqpoint{1.530082in}{2.712433in}}{\pgfqpoint{1.533354in}{2.704533in}}{\pgfqpoint{1.539178in}{2.698709in}}%
\pgfpathcurveto{\pgfqpoint{1.545002in}{2.692885in}}{\pgfqpoint{1.552902in}{2.689613in}}{\pgfqpoint{1.561138in}{2.689613in}}%
\pgfpathclose%
\pgfusepath{stroke,fill}%
\end{pgfscope}%
\begin{pgfscope}%
\pgfpathrectangle{\pgfqpoint{0.100000in}{0.212622in}}{\pgfqpoint{3.696000in}{3.696000in}}%
\pgfusepath{clip}%
\pgfsetbuttcap%
\pgfsetroundjoin%
\definecolor{currentfill}{rgb}{0.121569,0.466667,0.705882}%
\pgfsetfillcolor{currentfill}%
\pgfsetfillopacity{0.652508}%
\pgfsetlinewidth{1.003750pt}%
\definecolor{currentstroke}{rgb}{0.121569,0.466667,0.705882}%
\pgfsetstrokecolor{currentstroke}%
\pgfsetstrokeopacity{0.652508}%
\pgfsetdash{}{0pt}%
\pgfpathmoveto{\pgfqpoint{1.957884in}{3.034457in}}%
\pgfpathcurveto{\pgfqpoint{1.966121in}{3.034457in}}{\pgfqpoint{1.974021in}{3.037729in}}{\pgfqpoint{1.979845in}{3.043553in}}%
\pgfpathcurveto{\pgfqpoint{1.985668in}{3.049377in}}{\pgfqpoint{1.988941in}{3.057277in}}{\pgfqpoint{1.988941in}{3.065514in}}%
\pgfpathcurveto{\pgfqpoint{1.988941in}{3.073750in}}{\pgfqpoint{1.985668in}{3.081650in}}{\pgfqpoint{1.979845in}{3.087474in}}%
\pgfpathcurveto{\pgfqpoint{1.974021in}{3.093298in}}{\pgfqpoint{1.966121in}{3.096570in}}{\pgfqpoint{1.957884in}{3.096570in}}%
\pgfpathcurveto{\pgfqpoint{1.949648in}{3.096570in}}{\pgfqpoint{1.941748in}{3.093298in}}{\pgfqpoint{1.935924in}{3.087474in}}%
\pgfpathcurveto{\pgfqpoint{1.930100in}{3.081650in}}{\pgfqpoint{1.926828in}{3.073750in}}{\pgfqpoint{1.926828in}{3.065514in}}%
\pgfpathcurveto{\pgfqpoint{1.926828in}{3.057277in}}{\pgfqpoint{1.930100in}{3.049377in}}{\pgfqpoint{1.935924in}{3.043553in}}%
\pgfpathcurveto{\pgfqpoint{1.941748in}{3.037729in}}{\pgfqpoint{1.949648in}{3.034457in}}{\pgfqpoint{1.957884in}{3.034457in}}%
\pgfpathclose%
\pgfusepath{stroke,fill}%
\end{pgfscope}%
\begin{pgfscope}%
\pgfpathrectangle{\pgfqpoint{0.100000in}{0.212622in}}{\pgfqpoint{3.696000in}{3.696000in}}%
\pgfusepath{clip}%
\pgfsetbuttcap%
\pgfsetroundjoin%
\definecolor{currentfill}{rgb}{0.121569,0.466667,0.705882}%
\pgfsetfillcolor{currentfill}%
\pgfsetfillopacity{0.652974}%
\pgfsetlinewidth{1.003750pt}%
\definecolor{currentstroke}{rgb}{0.121569,0.466667,0.705882}%
\pgfsetstrokecolor{currentstroke}%
\pgfsetstrokeopacity{0.652974}%
\pgfsetdash{}{0pt}%
\pgfpathmoveto{\pgfqpoint{1.558890in}{2.686057in}}%
\pgfpathcurveto{\pgfqpoint{1.567126in}{2.686057in}}{\pgfqpoint{1.575026in}{2.689329in}}{\pgfqpoint{1.580850in}{2.695153in}}%
\pgfpathcurveto{\pgfqpoint{1.586674in}{2.700977in}}{\pgfqpoint{1.589947in}{2.708877in}}{\pgfqpoint{1.589947in}{2.717113in}}%
\pgfpathcurveto{\pgfqpoint{1.589947in}{2.725349in}}{\pgfqpoint{1.586674in}{2.733249in}}{\pgfqpoint{1.580850in}{2.739073in}}%
\pgfpathcurveto{\pgfqpoint{1.575026in}{2.744897in}}{\pgfqpoint{1.567126in}{2.748170in}}{\pgfqpoint{1.558890in}{2.748170in}}%
\pgfpathcurveto{\pgfqpoint{1.550654in}{2.748170in}}{\pgfqpoint{1.542754in}{2.744897in}}{\pgfqpoint{1.536930in}{2.739073in}}%
\pgfpathcurveto{\pgfqpoint{1.531106in}{2.733249in}}{\pgfqpoint{1.527834in}{2.725349in}}{\pgfqpoint{1.527834in}{2.717113in}}%
\pgfpathcurveto{\pgfqpoint{1.527834in}{2.708877in}}{\pgfqpoint{1.531106in}{2.700977in}}{\pgfqpoint{1.536930in}{2.695153in}}%
\pgfpathcurveto{\pgfqpoint{1.542754in}{2.689329in}}{\pgfqpoint{1.550654in}{2.686057in}}{\pgfqpoint{1.558890in}{2.686057in}}%
\pgfpathclose%
\pgfusepath{stroke,fill}%
\end{pgfscope}%
\begin{pgfscope}%
\pgfpathrectangle{\pgfqpoint{0.100000in}{0.212622in}}{\pgfqpoint{3.696000in}{3.696000in}}%
\pgfusepath{clip}%
\pgfsetbuttcap%
\pgfsetroundjoin%
\definecolor{currentfill}{rgb}{0.121569,0.466667,0.705882}%
\pgfsetfillcolor{currentfill}%
\pgfsetfillopacity{0.653394}%
\pgfsetlinewidth{1.003750pt}%
\definecolor{currentstroke}{rgb}{0.121569,0.466667,0.705882}%
\pgfsetstrokecolor{currentstroke}%
\pgfsetstrokeopacity{0.653394}%
\pgfsetdash{}{0pt}%
\pgfpathmoveto{\pgfqpoint{1.961073in}{3.033910in}}%
\pgfpathcurveto{\pgfqpoint{1.969309in}{3.033910in}}{\pgfqpoint{1.977209in}{3.037182in}}{\pgfqpoint{1.983033in}{3.043006in}}%
\pgfpathcurveto{\pgfqpoint{1.988857in}{3.048830in}}{\pgfqpoint{1.992129in}{3.056730in}}{\pgfqpoint{1.992129in}{3.064966in}}%
\pgfpathcurveto{\pgfqpoint{1.992129in}{3.073203in}}{\pgfqpoint{1.988857in}{3.081103in}}{\pgfqpoint{1.983033in}{3.086927in}}%
\pgfpathcurveto{\pgfqpoint{1.977209in}{3.092751in}}{\pgfqpoint{1.969309in}{3.096023in}}{\pgfqpoint{1.961073in}{3.096023in}}%
\pgfpathcurveto{\pgfqpoint{1.952837in}{3.096023in}}{\pgfqpoint{1.944937in}{3.092751in}}{\pgfqpoint{1.939113in}{3.086927in}}%
\pgfpathcurveto{\pgfqpoint{1.933289in}{3.081103in}}{\pgfqpoint{1.930016in}{3.073203in}}{\pgfqpoint{1.930016in}{3.064966in}}%
\pgfpathcurveto{\pgfqpoint{1.930016in}{3.056730in}}{\pgfqpoint{1.933289in}{3.048830in}}{\pgfqpoint{1.939113in}{3.043006in}}%
\pgfpathcurveto{\pgfqpoint{1.944937in}{3.037182in}}{\pgfqpoint{1.952837in}{3.033910in}}{\pgfqpoint{1.961073in}{3.033910in}}%
\pgfpathclose%
\pgfusepath{stroke,fill}%
\end{pgfscope}%
\begin{pgfscope}%
\pgfpathrectangle{\pgfqpoint{0.100000in}{0.212622in}}{\pgfqpoint{3.696000in}{3.696000in}}%
\pgfusepath{clip}%
\pgfsetbuttcap%
\pgfsetroundjoin%
\definecolor{currentfill}{rgb}{0.121569,0.466667,0.705882}%
\pgfsetfillcolor{currentfill}%
\pgfsetfillopacity{0.653659}%
\pgfsetlinewidth{1.003750pt}%
\definecolor{currentstroke}{rgb}{0.121569,0.466667,0.705882}%
\pgfsetstrokecolor{currentstroke}%
\pgfsetstrokeopacity{0.653659}%
\pgfsetdash{}{0pt}%
\pgfpathmoveto{\pgfqpoint{1.556986in}{2.679393in}}%
\pgfpathcurveto{\pgfqpoint{1.565222in}{2.679393in}}{\pgfqpoint{1.573122in}{2.682665in}}{\pgfqpoint{1.578946in}{2.688489in}}%
\pgfpathcurveto{\pgfqpoint{1.584770in}{2.694313in}}{\pgfqpoint{1.588042in}{2.702213in}}{\pgfqpoint{1.588042in}{2.710449in}}%
\pgfpathcurveto{\pgfqpoint{1.588042in}{2.718686in}}{\pgfqpoint{1.584770in}{2.726586in}}{\pgfqpoint{1.578946in}{2.732410in}}%
\pgfpathcurveto{\pgfqpoint{1.573122in}{2.738233in}}{\pgfqpoint{1.565222in}{2.741506in}}{\pgfqpoint{1.556986in}{2.741506in}}%
\pgfpathcurveto{\pgfqpoint{1.548750in}{2.741506in}}{\pgfqpoint{1.540849in}{2.738233in}}{\pgfqpoint{1.535026in}{2.732410in}}%
\pgfpathcurveto{\pgfqpoint{1.529202in}{2.726586in}}{\pgfqpoint{1.525929in}{2.718686in}}{\pgfqpoint{1.525929in}{2.710449in}}%
\pgfpathcurveto{\pgfqpoint{1.525929in}{2.702213in}}{\pgfqpoint{1.529202in}{2.694313in}}{\pgfqpoint{1.535026in}{2.688489in}}%
\pgfpathcurveto{\pgfqpoint{1.540849in}{2.682665in}}{\pgfqpoint{1.548750in}{2.679393in}}{\pgfqpoint{1.556986in}{2.679393in}}%
\pgfpathclose%
\pgfusepath{stroke,fill}%
\end{pgfscope}%
\begin{pgfscope}%
\pgfpathrectangle{\pgfqpoint{0.100000in}{0.212622in}}{\pgfqpoint{3.696000in}{3.696000in}}%
\pgfusepath{clip}%
\pgfsetbuttcap%
\pgfsetroundjoin%
\definecolor{currentfill}{rgb}{0.121569,0.466667,0.705882}%
\pgfsetfillcolor{currentfill}%
\pgfsetfillopacity{0.653950}%
\pgfsetlinewidth{1.003750pt}%
\definecolor{currentstroke}{rgb}{0.121569,0.466667,0.705882}%
\pgfsetstrokecolor{currentstroke}%
\pgfsetstrokeopacity{0.653950}%
\pgfsetdash{}{0pt}%
\pgfpathmoveto{\pgfqpoint{1.963071in}{3.033857in}}%
\pgfpathcurveto{\pgfqpoint{1.971307in}{3.033857in}}{\pgfqpoint{1.979207in}{3.037129in}}{\pgfqpoint{1.985031in}{3.042953in}}%
\pgfpathcurveto{\pgfqpoint{1.990855in}{3.048777in}}{\pgfqpoint{1.994127in}{3.056677in}}{\pgfqpoint{1.994127in}{3.064913in}}%
\pgfpathcurveto{\pgfqpoint{1.994127in}{3.073150in}}{\pgfqpoint{1.990855in}{3.081050in}}{\pgfqpoint{1.985031in}{3.086873in}}%
\pgfpathcurveto{\pgfqpoint{1.979207in}{3.092697in}}{\pgfqpoint{1.971307in}{3.095970in}}{\pgfqpoint{1.963071in}{3.095970in}}%
\pgfpathcurveto{\pgfqpoint{1.954835in}{3.095970in}}{\pgfqpoint{1.946935in}{3.092697in}}{\pgfqpoint{1.941111in}{3.086873in}}%
\pgfpathcurveto{\pgfqpoint{1.935287in}{3.081050in}}{\pgfqpoint{1.932014in}{3.073150in}}{\pgfqpoint{1.932014in}{3.064913in}}%
\pgfpathcurveto{\pgfqpoint{1.932014in}{3.056677in}}{\pgfqpoint{1.935287in}{3.048777in}}{\pgfqpoint{1.941111in}{3.042953in}}%
\pgfpathcurveto{\pgfqpoint{1.946935in}{3.037129in}}{\pgfqpoint{1.954835in}{3.033857in}}{\pgfqpoint{1.963071in}{3.033857in}}%
\pgfpathclose%
\pgfusepath{stroke,fill}%
\end{pgfscope}%
\begin{pgfscope}%
\pgfpathrectangle{\pgfqpoint{0.100000in}{0.212622in}}{\pgfqpoint{3.696000in}{3.696000in}}%
\pgfusepath{clip}%
\pgfsetbuttcap%
\pgfsetroundjoin%
\definecolor{currentfill}{rgb}{0.121569,0.466667,0.705882}%
\pgfsetfillcolor{currentfill}%
\pgfsetfillopacity{0.653987}%
\pgfsetlinewidth{1.003750pt}%
\definecolor{currentstroke}{rgb}{0.121569,0.466667,0.705882}%
\pgfsetstrokecolor{currentstroke}%
\pgfsetstrokeopacity{0.653987}%
\pgfsetdash{}{0pt}%
\pgfpathmoveto{\pgfqpoint{1.555791in}{2.675583in}}%
\pgfpathcurveto{\pgfqpoint{1.564027in}{2.675583in}}{\pgfqpoint{1.571927in}{2.678855in}}{\pgfqpoint{1.577751in}{2.684679in}}%
\pgfpathcurveto{\pgfqpoint{1.583575in}{2.690503in}}{\pgfqpoint{1.586847in}{2.698403in}}{\pgfqpoint{1.586847in}{2.706639in}}%
\pgfpathcurveto{\pgfqpoint{1.586847in}{2.714876in}}{\pgfqpoint{1.583575in}{2.722776in}}{\pgfqpoint{1.577751in}{2.728600in}}%
\pgfpathcurveto{\pgfqpoint{1.571927in}{2.734423in}}{\pgfqpoint{1.564027in}{2.737696in}}{\pgfqpoint{1.555791in}{2.737696in}}%
\pgfpathcurveto{\pgfqpoint{1.547555in}{2.737696in}}{\pgfqpoint{1.539655in}{2.734423in}}{\pgfqpoint{1.533831in}{2.728600in}}%
\pgfpathcurveto{\pgfqpoint{1.528007in}{2.722776in}}{\pgfqpoint{1.524734in}{2.714876in}}{\pgfqpoint{1.524734in}{2.706639in}}%
\pgfpathcurveto{\pgfqpoint{1.524734in}{2.698403in}}{\pgfqpoint{1.528007in}{2.690503in}}{\pgfqpoint{1.533831in}{2.684679in}}%
\pgfpathcurveto{\pgfqpoint{1.539655in}{2.678855in}}{\pgfqpoint{1.547555in}{2.675583in}}{\pgfqpoint{1.555791in}{2.675583in}}%
\pgfpathclose%
\pgfusepath{stroke,fill}%
\end{pgfscope}%
\begin{pgfscope}%
\pgfpathrectangle{\pgfqpoint{0.100000in}{0.212622in}}{\pgfqpoint{3.696000in}{3.696000in}}%
\pgfusepath{clip}%
\pgfsetbuttcap%
\pgfsetroundjoin%
\definecolor{currentfill}{rgb}{0.121569,0.466667,0.705882}%
\pgfsetfillcolor{currentfill}%
\pgfsetfillopacity{0.654009}%
\pgfsetlinewidth{1.003750pt}%
\definecolor{currentstroke}{rgb}{0.121569,0.466667,0.705882}%
\pgfsetstrokecolor{currentstroke}%
\pgfsetstrokeopacity{0.654009}%
\pgfsetdash{}{0pt}%
\pgfpathmoveto{\pgfqpoint{3.146198in}{2.620433in}}%
\pgfpathcurveto{\pgfqpoint{3.154434in}{2.620433in}}{\pgfqpoint{3.162334in}{2.623705in}}{\pgfqpoint{3.168158in}{2.629529in}}%
\pgfpathcurveto{\pgfqpoint{3.173982in}{2.635353in}}{\pgfqpoint{3.177255in}{2.643253in}}{\pgfqpoint{3.177255in}{2.651489in}}%
\pgfpathcurveto{\pgfqpoint{3.177255in}{2.659726in}}{\pgfqpoint{3.173982in}{2.667626in}}{\pgfqpoint{3.168158in}{2.673450in}}%
\pgfpathcurveto{\pgfqpoint{3.162334in}{2.679274in}}{\pgfqpoint{3.154434in}{2.682546in}}{\pgfqpoint{3.146198in}{2.682546in}}%
\pgfpathcurveto{\pgfqpoint{3.137962in}{2.682546in}}{\pgfqpoint{3.130062in}{2.679274in}}{\pgfqpoint{3.124238in}{2.673450in}}%
\pgfpathcurveto{\pgfqpoint{3.118414in}{2.667626in}}{\pgfqpoint{3.115142in}{2.659726in}}{\pgfqpoint{3.115142in}{2.651489in}}%
\pgfpathcurveto{\pgfqpoint{3.115142in}{2.643253in}}{\pgfqpoint{3.118414in}{2.635353in}}{\pgfqpoint{3.124238in}{2.629529in}}%
\pgfpathcurveto{\pgfqpoint{3.130062in}{2.623705in}}{\pgfqpoint{3.137962in}{2.620433in}}{\pgfqpoint{3.146198in}{2.620433in}}%
\pgfpathclose%
\pgfusepath{stroke,fill}%
\end{pgfscope}%
\begin{pgfscope}%
\pgfpathrectangle{\pgfqpoint{0.100000in}{0.212622in}}{\pgfqpoint{3.696000in}{3.696000in}}%
\pgfusepath{clip}%
\pgfsetbuttcap%
\pgfsetroundjoin%
\definecolor{currentfill}{rgb}{0.121569,0.466667,0.705882}%
\pgfsetfillcolor{currentfill}%
\pgfsetfillopacity{0.654343}%
\pgfsetlinewidth{1.003750pt}%
\definecolor{currentstroke}{rgb}{0.121569,0.466667,0.705882}%
\pgfsetstrokecolor{currentstroke}%
\pgfsetstrokeopacity{0.654343}%
\pgfsetdash{}{0pt}%
\pgfpathmoveto{\pgfqpoint{1.553439in}{2.671444in}}%
\pgfpathcurveto{\pgfqpoint{1.561676in}{2.671444in}}{\pgfqpoint{1.569576in}{2.674717in}}{\pgfqpoint{1.575400in}{2.680540in}}%
\pgfpathcurveto{\pgfqpoint{1.581224in}{2.686364in}}{\pgfqpoint{1.584496in}{2.694264in}}{\pgfqpoint{1.584496in}{2.702501in}}%
\pgfpathcurveto{\pgfqpoint{1.584496in}{2.710737in}}{\pgfqpoint{1.581224in}{2.718637in}}{\pgfqpoint{1.575400in}{2.724461in}}%
\pgfpathcurveto{\pgfqpoint{1.569576in}{2.730285in}}{\pgfqpoint{1.561676in}{2.733557in}}{\pgfqpoint{1.553439in}{2.733557in}}%
\pgfpathcurveto{\pgfqpoint{1.545203in}{2.733557in}}{\pgfqpoint{1.537303in}{2.730285in}}{\pgfqpoint{1.531479in}{2.724461in}}%
\pgfpathcurveto{\pgfqpoint{1.525655in}{2.718637in}}{\pgfqpoint{1.522383in}{2.710737in}}{\pgfqpoint{1.522383in}{2.702501in}}%
\pgfpathcurveto{\pgfqpoint{1.522383in}{2.694264in}}{\pgfqpoint{1.525655in}{2.686364in}}{\pgfqpoint{1.531479in}{2.680540in}}%
\pgfpathcurveto{\pgfqpoint{1.537303in}{2.674717in}}{\pgfqpoint{1.545203in}{2.671444in}}{\pgfqpoint{1.553439in}{2.671444in}}%
\pgfpathclose%
\pgfusepath{stroke,fill}%
\end{pgfscope}%
\begin{pgfscope}%
\pgfpathrectangle{\pgfqpoint{0.100000in}{0.212622in}}{\pgfqpoint{3.696000in}{3.696000in}}%
\pgfusepath{clip}%
\pgfsetbuttcap%
\pgfsetroundjoin%
\definecolor{currentfill}{rgb}{0.121569,0.466667,0.705882}%
\pgfsetfillcolor{currentfill}%
\pgfsetfillopacity{0.654767}%
\pgfsetlinewidth{1.003750pt}%
\definecolor{currentstroke}{rgb}{0.121569,0.466667,0.705882}%
\pgfsetstrokecolor{currentstroke}%
\pgfsetstrokeopacity{0.654767}%
\pgfsetdash{}{0pt}%
\pgfpathmoveto{\pgfqpoint{1.966744in}{3.032983in}}%
\pgfpathcurveto{\pgfqpoint{1.974981in}{3.032983in}}{\pgfqpoint{1.982881in}{3.036255in}}{\pgfqpoint{1.988705in}{3.042079in}}%
\pgfpathcurveto{\pgfqpoint{1.994529in}{3.047903in}}{\pgfqpoint{1.997801in}{3.055803in}}{\pgfqpoint{1.997801in}{3.064039in}}%
\pgfpathcurveto{\pgfqpoint{1.997801in}{3.072276in}}{\pgfqpoint{1.994529in}{3.080176in}}{\pgfqpoint{1.988705in}{3.086000in}}%
\pgfpathcurveto{\pgfqpoint{1.982881in}{3.091824in}}{\pgfqpoint{1.974981in}{3.095096in}}{\pgfqpoint{1.966744in}{3.095096in}}%
\pgfpathcurveto{\pgfqpoint{1.958508in}{3.095096in}}{\pgfqpoint{1.950608in}{3.091824in}}{\pgfqpoint{1.944784in}{3.086000in}}%
\pgfpathcurveto{\pgfqpoint{1.938960in}{3.080176in}}{\pgfqpoint{1.935688in}{3.072276in}}{\pgfqpoint{1.935688in}{3.064039in}}%
\pgfpathcurveto{\pgfqpoint{1.935688in}{3.055803in}}{\pgfqpoint{1.938960in}{3.047903in}}{\pgfqpoint{1.944784in}{3.042079in}}%
\pgfpathcurveto{\pgfqpoint{1.950608in}{3.036255in}}{\pgfqpoint{1.958508in}{3.032983in}}{\pgfqpoint{1.966744in}{3.032983in}}%
\pgfpathclose%
\pgfusepath{stroke,fill}%
\end{pgfscope}%
\begin{pgfscope}%
\pgfpathrectangle{\pgfqpoint{0.100000in}{0.212622in}}{\pgfqpoint{3.696000in}{3.696000in}}%
\pgfusepath{clip}%
\pgfsetbuttcap%
\pgfsetroundjoin%
\definecolor{currentfill}{rgb}{0.121569,0.466667,0.705882}%
\pgfsetfillcolor{currentfill}%
\pgfsetfillopacity{0.654850}%
\pgfsetlinewidth{1.003750pt}%
\definecolor{currentstroke}{rgb}{0.121569,0.466667,0.705882}%
\pgfsetstrokecolor{currentstroke}%
\pgfsetstrokeopacity{0.654850}%
\pgfsetdash{}{0pt}%
\pgfpathmoveto{\pgfqpoint{1.550609in}{2.667404in}}%
\pgfpathcurveto{\pgfqpoint{1.558845in}{2.667404in}}{\pgfqpoint{1.566745in}{2.670676in}}{\pgfqpoint{1.572569in}{2.676500in}}%
\pgfpathcurveto{\pgfqpoint{1.578393in}{2.682324in}}{\pgfqpoint{1.581665in}{2.690224in}}{\pgfqpoint{1.581665in}{2.698461in}}%
\pgfpathcurveto{\pgfqpoint{1.581665in}{2.706697in}}{\pgfqpoint{1.578393in}{2.714597in}}{\pgfqpoint{1.572569in}{2.720421in}}%
\pgfpathcurveto{\pgfqpoint{1.566745in}{2.726245in}}{\pgfqpoint{1.558845in}{2.729517in}}{\pgfqpoint{1.550609in}{2.729517in}}%
\pgfpathcurveto{\pgfqpoint{1.542373in}{2.729517in}}{\pgfqpoint{1.534473in}{2.726245in}}{\pgfqpoint{1.528649in}{2.720421in}}%
\pgfpathcurveto{\pgfqpoint{1.522825in}{2.714597in}}{\pgfqpoint{1.519552in}{2.706697in}}{\pgfqpoint{1.519552in}{2.698461in}}%
\pgfpathcurveto{\pgfqpoint{1.519552in}{2.690224in}}{\pgfqpoint{1.522825in}{2.682324in}}{\pgfqpoint{1.528649in}{2.676500in}}%
\pgfpathcurveto{\pgfqpoint{1.534473in}{2.670676in}}{\pgfqpoint{1.542373in}{2.667404in}}{\pgfqpoint{1.550609in}{2.667404in}}%
\pgfpathclose%
\pgfusepath{stroke,fill}%
\end{pgfscope}%
\begin{pgfscope}%
\pgfpathrectangle{\pgfqpoint{0.100000in}{0.212622in}}{\pgfqpoint{3.696000in}{3.696000in}}%
\pgfusepath{clip}%
\pgfsetbuttcap%
\pgfsetroundjoin%
\definecolor{currentfill}{rgb}{0.121569,0.466667,0.705882}%
\pgfsetfillcolor{currentfill}%
\pgfsetfillopacity{0.655301}%
\pgfsetlinewidth{1.003750pt}%
\definecolor{currentstroke}{rgb}{0.121569,0.466667,0.705882}%
\pgfsetstrokecolor{currentstroke}%
\pgfsetstrokeopacity{0.655301}%
\pgfsetdash{}{0pt}%
\pgfpathmoveto{\pgfqpoint{1.969067in}{3.032548in}}%
\pgfpathcurveto{\pgfqpoint{1.977303in}{3.032548in}}{\pgfqpoint{1.985203in}{3.035820in}}{\pgfqpoint{1.991027in}{3.041644in}}%
\pgfpathcurveto{\pgfqpoint{1.996851in}{3.047468in}}{\pgfqpoint{2.000124in}{3.055368in}}{\pgfqpoint{2.000124in}{3.063605in}}%
\pgfpathcurveto{\pgfqpoint{2.000124in}{3.071841in}}{\pgfqpoint{1.996851in}{3.079741in}}{\pgfqpoint{1.991027in}{3.085565in}}%
\pgfpathcurveto{\pgfqpoint{1.985203in}{3.091389in}}{\pgfqpoint{1.977303in}{3.094661in}}{\pgfqpoint{1.969067in}{3.094661in}}%
\pgfpathcurveto{\pgfqpoint{1.960831in}{3.094661in}}{\pgfqpoint{1.952931in}{3.091389in}}{\pgfqpoint{1.947107in}{3.085565in}}%
\pgfpathcurveto{\pgfqpoint{1.941283in}{3.079741in}}{\pgfqpoint{1.938011in}{3.071841in}}{\pgfqpoint{1.938011in}{3.063605in}}%
\pgfpathcurveto{\pgfqpoint{1.938011in}{3.055368in}}{\pgfqpoint{1.941283in}{3.047468in}}{\pgfqpoint{1.947107in}{3.041644in}}%
\pgfpathcurveto{\pgfqpoint{1.952931in}{3.035820in}}{\pgfqpoint{1.960831in}{3.032548in}}{\pgfqpoint{1.969067in}{3.032548in}}%
\pgfpathclose%
\pgfusepath{stroke,fill}%
\end{pgfscope}%
\begin{pgfscope}%
\pgfpathrectangle{\pgfqpoint{0.100000in}{0.212622in}}{\pgfqpoint{3.696000in}{3.696000in}}%
\pgfusepath{clip}%
\pgfsetbuttcap%
\pgfsetroundjoin%
\definecolor{currentfill}{rgb}{0.121569,0.466667,0.705882}%
\pgfsetfillcolor{currentfill}%
\pgfsetfillopacity{0.655591}%
\pgfsetlinewidth{1.003750pt}%
\definecolor{currentstroke}{rgb}{0.121569,0.466667,0.705882}%
\pgfsetstrokecolor{currentstroke}%
\pgfsetstrokeopacity{0.655591}%
\pgfsetdash{}{0pt}%
\pgfpathmoveto{\pgfqpoint{1.547185in}{2.660801in}}%
\pgfpathcurveto{\pgfqpoint{1.555421in}{2.660801in}}{\pgfqpoint{1.563321in}{2.664074in}}{\pgfqpoint{1.569145in}{2.669898in}}%
\pgfpathcurveto{\pgfqpoint{1.574969in}{2.675722in}}{\pgfqpoint{1.578241in}{2.683622in}}{\pgfqpoint{1.578241in}{2.691858in}}%
\pgfpathcurveto{\pgfqpoint{1.578241in}{2.700094in}}{\pgfqpoint{1.574969in}{2.707994in}}{\pgfqpoint{1.569145in}{2.713818in}}%
\pgfpathcurveto{\pgfqpoint{1.563321in}{2.719642in}}{\pgfqpoint{1.555421in}{2.722914in}}{\pgfqpoint{1.547185in}{2.722914in}}%
\pgfpathcurveto{\pgfqpoint{1.538948in}{2.722914in}}{\pgfqpoint{1.531048in}{2.719642in}}{\pgfqpoint{1.525224in}{2.713818in}}%
\pgfpathcurveto{\pgfqpoint{1.519401in}{2.707994in}}{\pgfqpoint{1.516128in}{2.700094in}}{\pgfqpoint{1.516128in}{2.691858in}}%
\pgfpathcurveto{\pgfqpoint{1.516128in}{2.683622in}}{\pgfqpoint{1.519401in}{2.675722in}}{\pgfqpoint{1.525224in}{2.669898in}}%
\pgfpathcurveto{\pgfqpoint{1.531048in}{2.664074in}}{\pgfqpoint{1.538948in}{2.660801in}}{\pgfqpoint{1.547185in}{2.660801in}}%
\pgfpathclose%
\pgfusepath{stroke,fill}%
\end{pgfscope}%
\begin{pgfscope}%
\pgfpathrectangle{\pgfqpoint{0.100000in}{0.212622in}}{\pgfqpoint{3.696000in}{3.696000in}}%
\pgfusepath{clip}%
\pgfsetbuttcap%
\pgfsetroundjoin%
\definecolor{currentfill}{rgb}{0.121569,0.466667,0.705882}%
\pgfsetfillcolor{currentfill}%
\pgfsetfillopacity{0.656285}%
\pgfsetlinewidth{1.003750pt}%
\definecolor{currentstroke}{rgb}{0.121569,0.466667,0.705882}%
\pgfsetstrokecolor{currentstroke}%
\pgfsetstrokeopacity{0.656285}%
\pgfsetdash{}{0pt}%
\pgfpathmoveto{\pgfqpoint{1.973090in}{3.031184in}}%
\pgfpathcurveto{\pgfqpoint{1.981326in}{3.031184in}}{\pgfqpoint{1.989226in}{3.034456in}}{\pgfqpoint{1.995050in}{3.040280in}}%
\pgfpathcurveto{\pgfqpoint{2.000874in}{3.046104in}}{\pgfqpoint{2.004146in}{3.054004in}}{\pgfqpoint{2.004146in}{3.062240in}}%
\pgfpathcurveto{\pgfqpoint{2.004146in}{3.070477in}}{\pgfqpoint{2.000874in}{3.078377in}}{\pgfqpoint{1.995050in}{3.084201in}}%
\pgfpathcurveto{\pgfqpoint{1.989226in}{3.090025in}}{\pgfqpoint{1.981326in}{3.093297in}}{\pgfqpoint{1.973090in}{3.093297in}}%
\pgfpathcurveto{\pgfqpoint{1.964853in}{3.093297in}}{\pgfqpoint{1.956953in}{3.090025in}}{\pgfqpoint{1.951129in}{3.084201in}}%
\pgfpathcurveto{\pgfqpoint{1.945305in}{3.078377in}}{\pgfqpoint{1.942033in}{3.070477in}}{\pgfqpoint{1.942033in}{3.062240in}}%
\pgfpathcurveto{\pgfqpoint{1.942033in}{3.054004in}}{\pgfqpoint{1.945305in}{3.046104in}}{\pgfqpoint{1.951129in}{3.040280in}}%
\pgfpathcurveto{\pgfqpoint{1.956953in}{3.034456in}}{\pgfqpoint{1.964853in}{3.031184in}}{\pgfqpoint{1.973090in}{3.031184in}}%
\pgfpathclose%
\pgfusepath{stroke,fill}%
\end{pgfscope}%
\begin{pgfscope}%
\pgfpathrectangle{\pgfqpoint{0.100000in}{0.212622in}}{\pgfqpoint{3.696000in}{3.696000in}}%
\pgfusepath{clip}%
\pgfsetbuttcap%
\pgfsetroundjoin%
\definecolor{currentfill}{rgb}{0.121569,0.466667,0.705882}%
\pgfsetfillcolor{currentfill}%
\pgfsetfillopacity{0.656398}%
\pgfsetlinewidth{1.003750pt}%
\definecolor{currentstroke}{rgb}{0.121569,0.466667,0.705882}%
\pgfsetstrokecolor{currentstroke}%
\pgfsetstrokeopacity{0.656398}%
\pgfsetdash{}{0pt}%
\pgfpathmoveto{\pgfqpoint{1.544148in}{2.651454in}}%
\pgfpathcurveto{\pgfqpoint{1.552385in}{2.651454in}}{\pgfqpoint{1.560285in}{2.654726in}}{\pgfqpoint{1.566109in}{2.660550in}}%
\pgfpathcurveto{\pgfqpoint{1.571933in}{2.666374in}}{\pgfqpoint{1.575205in}{2.674274in}}{\pgfqpoint{1.575205in}{2.682510in}}%
\pgfpathcurveto{\pgfqpoint{1.575205in}{2.690746in}}{\pgfqpoint{1.571933in}{2.698646in}}{\pgfqpoint{1.566109in}{2.704470in}}%
\pgfpathcurveto{\pgfqpoint{1.560285in}{2.710294in}}{\pgfqpoint{1.552385in}{2.713567in}}{\pgfqpoint{1.544148in}{2.713567in}}%
\pgfpathcurveto{\pgfqpoint{1.535912in}{2.713567in}}{\pgfqpoint{1.528012in}{2.710294in}}{\pgfqpoint{1.522188in}{2.704470in}}%
\pgfpathcurveto{\pgfqpoint{1.516364in}{2.698646in}}{\pgfqpoint{1.513092in}{2.690746in}}{\pgfqpoint{1.513092in}{2.682510in}}%
\pgfpathcurveto{\pgfqpoint{1.513092in}{2.674274in}}{\pgfqpoint{1.516364in}{2.666374in}}{\pgfqpoint{1.522188in}{2.660550in}}%
\pgfpathcurveto{\pgfqpoint{1.528012in}{2.654726in}}{\pgfqpoint{1.535912in}{2.651454in}}{\pgfqpoint{1.544148in}{2.651454in}}%
\pgfpathclose%
\pgfusepath{stroke,fill}%
\end{pgfscope}%
\begin{pgfscope}%
\pgfpathrectangle{\pgfqpoint{0.100000in}{0.212622in}}{\pgfqpoint{3.696000in}{3.696000in}}%
\pgfusepath{clip}%
\pgfsetbuttcap%
\pgfsetroundjoin%
\definecolor{currentfill}{rgb}{0.121569,0.466667,0.705882}%
\pgfsetfillcolor{currentfill}%
\pgfsetfillopacity{0.656853}%
\pgfsetlinewidth{1.003750pt}%
\definecolor{currentstroke}{rgb}{0.121569,0.466667,0.705882}%
\pgfsetstrokecolor{currentstroke}%
\pgfsetstrokeopacity{0.656853}%
\pgfsetdash{}{0pt}%
\pgfpathmoveto{\pgfqpoint{1.542047in}{2.646789in}}%
\pgfpathcurveto{\pgfqpoint{1.550283in}{2.646789in}}{\pgfqpoint{1.558183in}{2.650062in}}{\pgfqpoint{1.564007in}{2.655886in}}%
\pgfpathcurveto{\pgfqpoint{1.569831in}{2.661710in}}{\pgfqpoint{1.573103in}{2.669610in}}{\pgfqpoint{1.573103in}{2.677846in}}%
\pgfpathcurveto{\pgfqpoint{1.573103in}{2.686082in}}{\pgfqpoint{1.569831in}{2.693982in}}{\pgfqpoint{1.564007in}{2.699806in}}%
\pgfpathcurveto{\pgfqpoint{1.558183in}{2.705630in}}{\pgfqpoint{1.550283in}{2.708902in}}{\pgfqpoint{1.542047in}{2.708902in}}%
\pgfpathcurveto{\pgfqpoint{1.533811in}{2.708902in}}{\pgfqpoint{1.525911in}{2.705630in}}{\pgfqpoint{1.520087in}{2.699806in}}%
\pgfpathcurveto{\pgfqpoint{1.514263in}{2.693982in}}{\pgfqpoint{1.510990in}{2.686082in}}{\pgfqpoint{1.510990in}{2.677846in}}%
\pgfpathcurveto{\pgfqpoint{1.510990in}{2.669610in}}{\pgfqpoint{1.514263in}{2.661710in}}{\pgfqpoint{1.520087in}{2.655886in}}%
\pgfpathcurveto{\pgfqpoint{1.525911in}{2.650062in}}{\pgfqpoint{1.533811in}{2.646789in}}{\pgfqpoint{1.542047in}{2.646789in}}%
\pgfpathclose%
\pgfusepath{stroke,fill}%
\end{pgfscope}%
\begin{pgfscope}%
\pgfpathrectangle{\pgfqpoint{0.100000in}{0.212622in}}{\pgfqpoint{3.696000in}{3.696000in}}%
\pgfusepath{clip}%
\pgfsetbuttcap%
\pgfsetroundjoin%
\definecolor{currentfill}{rgb}{0.121569,0.466667,0.705882}%
\pgfsetfillcolor{currentfill}%
\pgfsetfillopacity{0.657017}%
\pgfsetlinewidth{1.003750pt}%
\definecolor{currentstroke}{rgb}{0.121569,0.466667,0.705882}%
\pgfsetstrokecolor{currentstroke}%
\pgfsetstrokeopacity{0.657017}%
\pgfsetdash{}{0pt}%
\pgfpathmoveto{\pgfqpoint{1.975734in}{3.030957in}}%
\pgfpathcurveto{\pgfqpoint{1.983971in}{3.030957in}}{\pgfqpoint{1.991871in}{3.034230in}}{\pgfqpoint{1.997695in}{3.040054in}}%
\pgfpathcurveto{\pgfqpoint{2.003519in}{3.045878in}}{\pgfqpoint{2.006791in}{3.053778in}}{\pgfqpoint{2.006791in}{3.062014in}}%
\pgfpathcurveto{\pgfqpoint{2.006791in}{3.070250in}}{\pgfqpoint{2.003519in}{3.078150in}}{\pgfqpoint{1.997695in}{3.083974in}}%
\pgfpathcurveto{\pgfqpoint{1.991871in}{3.089798in}}{\pgfqpoint{1.983971in}{3.093070in}}{\pgfqpoint{1.975734in}{3.093070in}}%
\pgfpathcurveto{\pgfqpoint{1.967498in}{3.093070in}}{\pgfqpoint{1.959598in}{3.089798in}}{\pgfqpoint{1.953774in}{3.083974in}}%
\pgfpathcurveto{\pgfqpoint{1.947950in}{3.078150in}}{\pgfqpoint{1.944678in}{3.070250in}}{\pgfqpoint{1.944678in}{3.062014in}}%
\pgfpathcurveto{\pgfqpoint{1.944678in}{3.053778in}}{\pgfqpoint{1.947950in}{3.045878in}}{\pgfqpoint{1.953774in}{3.040054in}}%
\pgfpathcurveto{\pgfqpoint{1.959598in}{3.034230in}}{\pgfqpoint{1.967498in}{3.030957in}}{\pgfqpoint{1.975734in}{3.030957in}}%
\pgfpathclose%
\pgfusepath{stroke,fill}%
\end{pgfscope}%
\begin{pgfscope}%
\pgfpathrectangle{\pgfqpoint{0.100000in}{0.212622in}}{\pgfqpoint{3.696000in}{3.696000in}}%
\pgfusepath{clip}%
\pgfsetbuttcap%
\pgfsetroundjoin%
\definecolor{currentfill}{rgb}{0.121569,0.466667,0.705882}%
\pgfsetfillcolor{currentfill}%
\pgfsetfillopacity{0.657087}%
\pgfsetlinewidth{1.003750pt}%
\definecolor{currentstroke}{rgb}{0.121569,0.466667,0.705882}%
\pgfsetstrokecolor{currentstroke}%
\pgfsetstrokeopacity{0.657087}%
\pgfsetdash{}{0pt}%
\pgfpathmoveto{\pgfqpoint{1.540648in}{2.644565in}}%
\pgfpathcurveto{\pgfqpoint{1.548885in}{2.644565in}}{\pgfqpoint{1.556785in}{2.647837in}}{\pgfqpoint{1.562609in}{2.653661in}}%
\pgfpathcurveto{\pgfqpoint{1.568433in}{2.659485in}}{\pgfqpoint{1.571705in}{2.667385in}}{\pgfqpoint{1.571705in}{2.675621in}}%
\pgfpathcurveto{\pgfqpoint{1.571705in}{2.683857in}}{\pgfqpoint{1.568433in}{2.691757in}}{\pgfqpoint{1.562609in}{2.697581in}}%
\pgfpathcurveto{\pgfqpoint{1.556785in}{2.703405in}}{\pgfqpoint{1.548885in}{2.706678in}}{\pgfqpoint{1.540648in}{2.706678in}}%
\pgfpathcurveto{\pgfqpoint{1.532412in}{2.706678in}}{\pgfqpoint{1.524512in}{2.703405in}}{\pgfqpoint{1.518688in}{2.697581in}}%
\pgfpathcurveto{\pgfqpoint{1.512864in}{2.691757in}}{\pgfqpoint{1.509592in}{2.683857in}}{\pgfqpoint{1.509592in}{2.675621in}}%
\pgfpathcurveto{\pgfqpoint{1.509592in}{2.667385in}}{\pgfqpoint{1.512864in}{2.659485in}}{\pgfqpoint{1.518688in}{2.653661in}}%
\pgfpathcurveto{\pgfqpoint{1.524512in}{2.647837in}}{\pgfqpoint{1.532412in}{2.644565in}}{\pgfqpoint{1.540648in}{2.644565in}}%
\pgfpathclose%
\pgfusepath{stroke,fill}%
\end{pgfscope}%
\begin{pgfscope}%
\pgfpathrectangle{\pgfqpoint{0.100000in}{0.212622in}}{\pgfqpoint{3.696000in}{3.696000in}}%
\pgfusepath{clip}%
\pgfsetbuttcap%
\pgfsetroundjoin%
\definecolor{currentfill}{rgb}{0.121569,0.466667,0.705882}%
\pgfsetfillcolor{currentfill}%
\pgfsetfillopacity{0.657334}%
\pgfsetlinewidth{1.003750pt}%
\definecolor{currentstroke}{rgb}{0.121569,0.466667,0.705882}%
\pgfsetstrokecolor{currentstroke}%
\pgfsetstrokeopacity{0.657334}%
\pgfsetdash{}{0pt}%
\pgfpathmoveto{\pgfqpoint{1.538641in}{2.641614in}}%
\pgfpathcurveto{\pgfqpoint{1.546878in}{2.641614in}}{\pgfqpoint{1.554778in}{2.644886in}}{\pgfqpoint{1.560602in}{2.650710in}}%
\pgfpathcurveto{\pgfqpoint{1.566426in}{2.656534in}}{\pgfqpoint{1.569698in}{2.664434in}}{\pgfqpoint{1.569698in}{2.672670in}}%
\pgfpathcurveto{\pgfqpoint{1.569698in}{2.680906in}}{\pgfqpoint{1.566426in}{2.688806in}}{\pgfqpoint{1.560602in}{2.694630in}}%
\pgfpathcurveto{\pgfqpoint{1.554778in}{2.700454in}}{\pgfqpoint{1.546878in}{2.703727in}}{\pgfqpoint{1.538641in}{2.703727in}}%
\pgfpathcurveto{\pgfqpoint{1.530405in}{2.703727in}}{\pgfqpoint{1.522505in}{2.700454in}}{\pgfqpoint{1.516681in}{2.694630in}}%
\pgfpathcurveto{\pgfqpoint{1.510857in}{2.688806in}}{\pgfqpoint{1.507585in}{2.680906in}}{\pgfqpoint{1.507585in}{2.672670in}}%
\pgfpathcurveto{\pgfqpoint{1.507585in}{2.664434in}}{\pgfqpoint{1.510857in}{2.656534in}}{\pgfqpoint{1.516681in}{2.650710in}}%
\pgfpathcurveto{\pgfqpoint{1.522505in}{2.644886in}}{\pgfqpoint{1.530405in}{2.641614in}}{\pgfqpoint{1.538641in}{2.641614in}}%
\pgfpathclose%
\pgfusepath{stroke,fill}%
\end{pgfscope}%
\begin{pgfscope}%
\pgfpathrectangle{\pgfqpoint{0.100000in}{0.212622in}}{\pgfqpoint{3.696000in}{3.696000in}}%
\pgfusepath{clip}%
\pgfsetbuttcap%
\pgfsetroundjoin%
\definecolor{currentfill}{rgb}{0.121569,0.466667,0.705882}%
\pgfsetfillcolor{currentfill}%
\pgfsetfillopacity{0.657886}%
\pgfsetlinewidth{1.003750pt}%
\definecolor{currentstroke}{rgb}{0.121569,0.466667,0.705882}%
\pgfsetstrokecolor{currentstroke}%
\pgfsetstrokeopacity{0.657886}%
\pgfsetdash{}{0pt}%
\pgfpathmoveto{\pgfqpoint{1.535890in}{2.636121in}}%
\pgfpathcurveto{\pgfqpoint{1.544126in}{2.636121in}}{\pgfqpoint{1.552026in}{2.639393in}}{\pgfqpoint{1.557850in}{2.645217in}}%
\pgfpathcurveto{\pgfqpoint{1.563674in}{2.651041in}}{\pgfqpoint{1.566946in}{2.658941in}}{\pgfqpoint{1.566946in}{2.667177in}}%
\pgfpathcurveto{\pgfqpoint{1.566946in}{2.675414in}}{\pgfqpoint{1.563674in}{2.683314in}}{\pgfqpoint{1.557850in}{2.689138in}}%
\pgfpathcurveto{\pgfqpoint{1.552026in}{2.694961in}}{\pgfqpoint{1.544126in}{2.698234in}}{\pgfqpoint{1.535890in}{2.698234in}}%
\pgfpathcurveto{\pgfqpoint{1.527653in}{2.698234in}}{\pgfqpoint{1.519753in}{2.694961in}}{\pgfqpoint{1.513929in}{2.689138in}}%
\pgfpathcurveto{\pgfqpoint{1.508105in}{2.683314in}}{\pgfqpoint{1.504833in}{2.675414in}}{\pgfqpoint{1.504833in}{2.667177in}}%
\pgfpathcurveto{\pgfqpoint{1.504833in}{2.658941in}}{\pgfqpoint{1.508105in}{2.651041in}}{\pgfqpoint{1.513929in}{2.645217in}}%
\pgfpathcurveto{\pgfqpoint{1.519753in}{2.639393in}}{\pgfqpoint{1.527653in}{2.636121in}}{\pgfqpoint{1.535890in}{2.636121in}}%
\pgfpathclose%
\pgfusepath{stroke,fill}%
\end{pgfscope}%
\begin{pgfscope}%
\pgfpathrectangle{\pgfqpoint{0.100000in}{0.212622in}}{\pgfqpoint{3.696000in}{3.696000in}}%
\pgfusepath{clip}%
\pgfsetbuttcap%
\pgfsetroundjoin%
\definecolor{currentfill}{rgb}{0.121569,0.466667,0.705882}%
\pgfsetfillcolor{currentfill}%
\pgfsetfillopacity{0.658235}%
\pgfsetlinewidth{1.003750pt}%
\definecolor{currentstroke}{rgb}{0.121569,0.466667,0.705882}%
\pgfsetstrokecolor{currentstroke}%
\pgfsetstrokeopacity{0.658235}%
\pgfsetdash{}{0pt}%
\pgfpathmoveto{\pgfqpoint{1.980385in}{3.029525in}}%
\pgfpathcurveto{\pgfqpoint{1.988621in}{3.029525in}}{\pgfqpoint{1.996522in}{3.032797in}}{\pgfqpoint{2.002345in}{3.038621in}}%
\pgfpathcurveto{\pgfqpoint{2.008169in}{3.044445in}}{\pgfqpoint{2.011442in}{3.052345in}}{\pgfqpoint{2.011442in}{3.060581in}}%
\pgfpathcurveto{\pgfqpoint{2.011442in}{3.068818in}}{\pgfqpoint{2.008169in}{3.076718in}}{\pgfqpoint{2.002345in}{3.082542in}}%
\pgfpathcurveto{\pgfqpoint{1.996522in}{3.088366in}}{\pgfqpoint{1.988621in}{3.091638in}}{\pgfqpoint{1.980385in}{3.091638in}}%
\pgfpathcurveto{\pgfqpoint{1.972149in}{3.091638in}}{\pgfqpoint{1.964249in}{3.088366in}}{\pgfqpoint{1.958425in}{3.082542in}}%
\pgfpathcurveto{\pgfqpoint{1.952601in}{3.076718in}}{\pgfqpoint{1.949329in}{3.068818in}}{\pgfqpoint{1.949329in}{3.060581in}}%
\pgfpathcurveto{\pgfqpoint{1.949329in}{3.052345in}}{\pgfqpoint{1.952601in}{3.044445in}}{\pgfqpoint{1.958425in}{3.038621in}}%
\pgfpathcurveto{\pgfqpoint{1.964249in}{3.032797in}}{\pgfqpoint{1.972149in}{3.029525in}}{\pgfqpoint{1.980385in}{3.029525in}}%
\pgfpathclose%
\pgfusepath{stroke,fill}%
\end{pgfscope}%
\begin{pgfscope}%
\pgfpathrectangle{\pgfqpoint{0.100000in}{0.212622in}}{\pgfqpoint{3.696000in}{3.696000in}}%
\pgfusepath{clip}%
\pgfsetbuttcap%
\pgfsetroundjoin%
\definecolor{currentfill}{rgb}{0.121569,0.466667,0.705882}%
\pgfsetfillcolor{currentfill}%
\pgfsetfillopacity{0.658479}%
\pgfsetlinewidth{1.003750pt}%
\definecolor{currentstroke}{rgb}{0.121569,0.466667,0.705882}%
\pgfsetstrokecolor{currentstroke}%
\pgfsetstrokeopacity{0.658479}%
\pgfsetdash{}{0pt}%
\pgfpathmoveto{\pgfqpoint{3.162631in}{2.622956in}}%
\pgfpathcurveto{\pgfqpoint{3.170867in}{2.622956in}}{\pgfqpoint{3.178767in}{2.626228in}}{\pgfqpoint{3.184591in}{2.632052in}}%
\pgfpathcurveto{\pgfqpoint{3.190415in}{2.637876in}}{\pgfqpoint{3.193687in}{2.645776in}}{\pgfqpoint{3.193687in}{2.654013in}}%
\pgfpathcurveto{\pgfqpoint{3.193687in}{2.662249in}}{\pgfqpoint{3.190415in}{2.670149in}}{\pgfqpoint{3.184591in}{2.675973in}}%
\pgfpathcurveto{\pgfqpoint{3.178767in}{2.681797in}}{\pgfqpoint{3.170867in}{2.685069in}}{\pgfqpoint{3.162631in}{2.685069in}}%
\pgfpathcurveto{\pgfqpoint{3.154394in}{2.685069in}}{\pgfqpoint{3.146494in}{2.681797in}}{\pgfqpoint{3.140670in}{2.675973in}}%
\pgfpathcurveto{\pgfqpoint{3.134846in}{2.670149in}}{\pgfqpoint{3.131574in}{2.662249in}}{\pgfqpoint{3.131574in}{2.654013in}}%
\pgfpathcurveto{\pgfqpoint{3.131574in}{2.645776in}}{\pgfqpoint{3.134846in}{2.637876in}}{\pgfqpoint{3.140670in}{2.632052in}}%
\pgfpathcurveto{\pgfqpoint{3.146494in}{2.626228in}}{\pgfqpoint{3.154394in}{2.622956in}}{\pgfqpoint{3.162631in}{2.622956in}}%
\pgfpathclose%
\pgfusepath{stroke,fill}%
\end{pgfscope}%
\begin{pgfscope}%
\pgfpathrectangle{\pgfqpoint{0.100000in}{0.212622in}}{\pgfqpoint{3.696000in}{3.696000in}}%
\pgfusepath{clip}%
\pgfsetbuttcap%
\pgfsetroundjoin%
\definecolor{currentfill}{rgb}{0.121569,0.466667,0.705882}%
\pgfsetfillcolor{currentfill}%
\pgfsetfillopacity{0.658606}%
\pgfsetlinewidth{1.003750pt}%
\definecolor{currentstroke}{rgb}{0.121569,0.466667,0.705882}%
\pgfsetstrokecolor{currentstroke}%
\pgfsetstrokeopacity{0.658606}%
\pgfsetdash{}{0pt}%
\pgfpathmoveto{\pgfqpoint{1.533883in}{2.628544in}}%
\pgfpathcurveto{\pgfqpoint{1.542119in}{2.628544in}}{\pgfqpoint{1.550019in}{2.631816in}}{\pgfqpoint{1.555843in}{2.637640in}}%
\pgfpathcurveto{\pgfqpoint{1.561667in}{2.643464in}}{\pgfqpoint{1.564939in}{2.651364in}}{\pgfqpoint{1.564939in}{2.659601in}}%
\pgfpathcurveto{\pgfqpoint{1.564939in}{2.667837in}}{\pgfqpoint{1.561667in}{2.675737in}}{\pgfqpoint{1.555843in}{2.681561in}}%
\pgfpathcurveto{\pgfqpoint{1.550019in}{2.687385in}}{\pgfqpoint{1.542119in}{2.690657in}}{\pgfqpoint{1.533883in}{2.690657in}}%
\pgfpathcurveto{\pgfqpoint{1.525646in}{2.690657in}}{\pgfqpoint{1.517746in}{2.687385in}}{\pgfqpoint{1.511922in}{2.681561in}}%
\pgfpathcurveto{\pgfqpoint{1.506098in}{2.675737in}}{\pgfqpoint{1.502826in}{2.667837in}}{\pgfqpoint{1.502826in}{2.659601in}}%
\pgfpathcurveto{\pgfqpoint{1.502826in}{2.651364in}}{\pgfqpoint{1.506098in}{2.643464in}}{\pgfqpoint{1.511922in}{2.637640in}}%
\pgfpathcurveto{\pgfqpoint{1.517746in}{2.631816in}}{\pgfqpoint{1.525646in}{2.628544in}}{\pgfqpoint{1.533883in}{2.628544in}}%
\pgfpathclose%
\pgfusepath{stroke,fill}%
\end{pgfscope}%
\begin{pgfscope}%
\pgfpathrectangle{\pgfqpoint{0.100000in}{0.212622in}}{\pgfqpoint{3.696000in}{3.696000in}}%
\pgfusepath{clip}%
\pgfsetbuttcap%
\pgfsetroundjoin%
\definecolor{currentfill}{rgb}{0.121569,0.466667,0.705882}%
\pgfsetfillcolor{currentfill}%
\pgfsetfillopacity{0.659040}%
\pgfsetlinewidth{1.003750pt}%
\definecolor{currentstroke}{rgb}{0.121569,0.466667,0.705882}%
\pgfsetstrokecolor{currentstroke}%
\pgfsetstrokeopacity{0.659040}%
\pgfsetdash{}{0pt}%
\pgfpathmoveto{\pgfqpoint{1.532267in}{2.624917in}}%
\pgfpathcurveto{\pgfqpoint{1.540503in}{2.624917in}}{\pgfqpoint{1.548403in}{2.628189in}}{\pgfqpoint{1.554227in}{2.634013in}}%
\pgfpathcurveto{\pgfqpoint{1.560051in}{2.639837in}}{\pgfqpoint{1.563323in}{2.647737in}}{\pgfqpoint{1.563323in}{2.655974in}}%
\pgfpathcurveto{\pgfqpoint{1.563323in}{2.664210in}}{\pgfqpoint{1.560051in}{2.672110in}}{\pgfqpoint{1.554227in}{2.677934in}}%
\pgfpathcurveto{\pgfqpoint{1.548403in}{2.683758in}}{\pgfqpoint{1.540503in}{2.687030in}}{\pgfqpoint{1.532267in}{2.687030in}}%
\pgfpathcurveto{\pgfqpoint{1.524031in}{2.687030in}}{\pgfqpoint{1.516130in}{2.683758in}}{\pgfqpoint{1.510307in}{2.677934in}}%
\pgfpathcurveto{\pgfqpoint{1.504483in}{2.672110in}}{\pgfqpoint{1.501210in}{2.664210in}}{\pgfqpoint{1.501210in}{2.655974in}}%
\pgfpathcurveto{\pgfqpoint{1.501210in}{2.647737in}}{\pgfqpoint{1.504483in}{2.639837in}}{\pgfqpoint{1.510307in}{2.634013in}}%
\pgfpathcurveto{\pgfqpoint{1.516130in}{2.628189in}}{\pgfqpoint{1.524031in}{2.624917in}}{\pgfqpoint{1.532267in}{2.624917in}}%
\pgfpathclose%
\pgfusepath{stroke,fill}%
\end{pgfscope}%
\begin{pgfscope}%
\pgfpathrectangle{\pgfqpoint{0.100000in}{0.212622in}}{\pgfqpoint{3.696000in}{3.696000in}}%
\pgfusepath{clip}%
\pgfsetbuttcap%
\pgfsetroundjoin%
\definecolor{currentfill}{rgb}{0.121569,0.466667,0.705882}%
\pgfsetfillcolor{currentfill}%
\pgfsetfillopacity{0.659082}%
\pgfsetlinewidth{1.003750pt}%
\definecolor{currentstroke}{rgb}{0.121569,0.466667,0.705882}%
\pgfsetstrokecolor{currentstroke}%
\pgfsetstrokeopacity{0.659082}%
\pgfsetdash{}{0pt}%
\pgfpathmoveto{\pgfqpoint{1.983835in}{3.028293in}}%
\pgfpathcurveto{\pgfqpoint{1.992071in}{3.028293in}}{\pgfqpoint{1.999971in}{3.031565in}}{\pgfqpoint{2.005795in}{3.037389in}}%
\pgfpathcurveto{\pgfqpoint{2.011619in}{3.043213in}}{\pgfqpoint{2.014891in}{3.051113in}}{\pgfqpoint{2.014891in}{3.059350in}}%
\pgfpathcurveto{\pgfqpoint{2.014891in}{3.067586in}}{\pgfqpoint{2.011619in}{3.075486in}}{\pgfqpoint{2.005795in}{3.081310in}}%
\pgfpathcurveto{\pgfqpoint{1.999971in}{3.087134in}}{\pgfqpoint{1.992071in}{3.090406in}}{\pgfqpoint{1.983835in}{3.090406in}}%
\pgfpathcurveto{\pgfqpoint{1.975599in}{3.090406in}}{\pgfqpoint{1.967699in}{3.087134in}}{\pgfqpoint{1.961875in}{3.081310in}}%
\pgfpathcurveto{\pgfqpoint{1.956051in}{3.075486in}}{\pgfqpoint{1.952778in}{3.067586in}}{\pgfqpoint{1.952778in}{3.059350in}}%
\pgfpathcurveto{\pgfqpoint{1.952778in}{3.051113in}}{\pgfqpoint{1.956051in}{3.043213in}}{\pgfqpoint{1.961875in}{3.037389in}}%
\pgfpathcurveto{\pgfqpoint{1.967699in}{3.031565in}}{\pgfqpoint{1.975599in}{3.028293in}}{\pgfqpoint{1.983835in}{3.028293in}}%
\pgfpathclose%
\pgfusepath{stroke,fill}%
\end{pgfscope}%
\begin{pgfscope}%
\pgfpathrectangle{\pgfqpoint{0.100000in}{0.212622in}}{\pgfqpoint{3.696000in}{3.696000in}}%
\pgfusepath{clip}%
\pgfsetbuttcap%
\pgfsetroundjoin%
\definecolor{currentfill}{rgb}{0.121569,0.466667,0.705882}%
\pgfsetfillcolor{currentfill}%
\pgfsetfillopacity{0.659180}%
\pgfsetlinewidth{1.003750pt}%
\definecolor{currentstroke}{rgb}{0.121569,0.466667,0.705882}%
\pgfsetstrokecolor{currentstroke}%
\pgfsetstrokeopacity{0.659180}%
\pgfsetdash{}{0pt}%
\pgfpathmoveto{\pgfqpoint{1.531050in}{2.623085in}}%
\pgfpathcurveto{\pgfqpoint{1.539286in}{2.623085in}}{\pgfqpoint{1.547186in}{2.626357in}}{\pgfqpoint{1.553010in}{2.632181in}}%
\pgfpathcurveto{\pgfqpoint{1.558834in}{2.638005in}}{\pgfqpoint{1.562106in}{2.645905in}}{\pgfqpoint{1.562106in}{2.654141in}}%
\pgfpathcurveto{\pgfqpoint{1.562106in}{2.662378in}}{\pgfqpoint{1.558834in}{2.670278in}}{\pgfqpoint{1.553010in}{2.676102in}}%
\pgfpathcurveto{\pgfqpoint{1.547186in}{2.681926in}}{\pgfqpoint{1.539286in}{2.685198in}}{\pgfqpoint{1.531050in}{2.685198in}}%
\pgfpathcurveto{\pgfqpoint{1.522814in}{2.685198in}}{\pgfqpoint{1.514913in}{2.681926in}}{\pgfqpoint{1.509090in}{2.676102in}}%
\pgfpathcurveto{\pgfqpoint{1.503266in}{2.670278in}}{\pgfqpoint{1.499993in}{2.662378in}}{\pgfqpoint{1.499993in}{2.654141in}}%
\pgfpathcurveto{\pgfqpoint{1.499993in}{2.645905in}}{\pgfqpoint{1.503266in}{2.638005in}}{\pgfqpoint{1.509090in}{2.632181in}}%
\pgfpathcurveto{\pgfqpoint{1.514913in}{2.626357in}}{\pgfqpoint{1.522814in}{2.623085in}}{\pgfqpoint{1.531050in}{2.623085in}}%
\pgfpathclose%
\pgfusepath{stroke,fill}%
\end{pgfscope}%
\begin{pgfscope}%
\pgfpathrectangle{\pgfqpoint{0.100000in}{0.212622in}}{\pgfqpoint{3.696000in}{3.696000in}}%
\pgfusepath{clip}%
\pgfsetbuttcap%
\pgfsetroundjoin%
\definecolor{currentfill}{rgb}{0.121569,0.466667,0.705882}%
\pgfsetfillcolor{currentfill}%
\pgfsetfillopacity{0.659365}%
\pgfsetlinewidth{1.003750pt}%
\definecolor{currentstroke}{rgb}{0.121569,0.466667,0.705882}%
\pgfsetstrokecolor{currentstroke}%
\pgfsetstrokeopacity{0.659365}%
\pgfsetdash{}{0pt}%
\pgfpathmoveto{\pgfqpoint{1.529576in}{2.620924in}}%
\pgfpathcurveto{\pgfqpoint{1.537813in}{2.620924in}}{\pgfqpoint{1.545713in}{2.624196in}}{\pgfqpoint{1.551537in}{2.630020in}}%
\pgfpathcurveto{\pgfqpoint{1.557360in}{2.635844in}}{\pgfqpoint{1.560633in}{2.643744in}}{\pgfqpoint{1.560633in}{2.651980in}}%
\pgfpathcurveto{\pgfqpoint{1.560633in}{2.660217in}}{\pgfqpoint{1.557360in}{2.668117in}}{\pgfqpoint{1.551537in}{2.673941in}}%
\pgfpathcurveto{\pgfqpoint{1.545713in}{2.679765in}}{\pgfqpoint{1.537813in}{2.683037in}}{\pgfqpoint{1.529576in}{2.683037in}}%
\pgfpathcurveto{\pgfqpoint{1.521340in}{2.683037in}}{\pgfqpoint{1.513440in}{2.679765in}}{\pgfqpoint{1.507616in}{2.673941in}}%
\pgfpathcurveto{\pgfqpoint{1.501792in}{2.668117in}}{\pgfqpoint{1.498520in}{2.660217in}}{\pgfqpoint{1.498520in}{2.651980in}}%
\pgfpathcurveto{\pgfqpoint{1.498520in}{2.643744in}}{\pgfqpoint{1.501792in}{2.635844in}}{\pgfqpoint{1.507616in}{2.630020in}}%
\pgfpathcurveto{\pgfqpoint{1.513440in}{2.624196in}}{\pgfqpoint{1.521340in}{2.620924in}}{\pgfqpoint{1.529576in}{2.620924in}}%
\pgfpathclose%
\pgfusepath{stroke,fill}%
\end{pgfscope}%
\begin{pgfscope}%
\pgfpathrectangle{\pgfqpoint{0.100000in}{0.212622in}}{\pgfqpoint{3.696000in}{3.696000in}}%
\pgfusepath{clip}%
\pgfsetbuttcap%
\pgfsetroundjoin%
\definecolor{currentfill}{rgb}{0.121569,0.466667,0.705882}%
\pgfsetfillcolor{currentfill}%
\pgfsetfillopacity{0.659770}%
\pgfsetlinewidth{1.003750pt}%
\definecolor{currentstroke}{rgb}{0.121569,0.466667,0.705882}%
\pgfsetstrokecolor{currentstroke}%
\pgfsetstrokeopacity{0.659770}%
\pgfsetdash{}{0pt}%
\pgfpathmoveto{\pgfqpoint{1.527522in}{2.616432in}}%
\pgfpathcurveto{\pgfqpoint{1.535758in}{2.616432in}}{\pgfqpoint{1.543659in}{2.619704in}}{\pgfqpoint{1.549482in}{2.625528in}}%
\pgfpathcurveto{\pgfqpoint{1.555306in}{2.631352in}}{\pgfqpoint{1.558579in}{2.639252in}}{\pgfqpoint{1.558579in}{2.647488in}}%
\pgfpathcurveto{\pgfqpoint{1.558579in}{2.655724in}}{\pgfqpoint{1.555306in}{2.663624in}}{\pgfqpoint{1.549482in}{2.669448in}}%
\pgfpathcurveto{\pgfqpoint{1.543659in}{2.675272in}}{\pgfqpoint{1.535758in}{2.678545in}}{\pgfqpoint{1.527522in}{2.678545in}}%
\pgfpathcurveto{\pgfqpoint{1.519286in}{2.678545in}}{\pgfqpoint{1.511386in}{2.675272in}}{\pgfqpoint{1.505562in}{2.669448in}}%
\pgfpathcurveto{\pgfqpoint{1.499738in}{2.663624in}}{\pgfqpoint{1.496466in}{2.655724in}}{\pgfqpoint{1.496466in}{2.647488in}}%
\pgfpathcurveto{\pgfqpoint{1.496466in}{2.639252in}}{\pgfqpoint{1.499738in}{2.631352in}}{\pgfqpoint{1.505562in}{2.625528in}}%
\pgfpathcurveto{\pgfqpoint{1.511386in}{2.619704in}}{\pgfqpoint{1.519286in}{2.616432in}}{\pgfqpoint{1.527522in}{2.616432in}}%
\pgfpathclose%
\pgfusepath{stroke,fill}%
\end{pgfscope}%
\begin{pgfscope}%
\pgfpathrectangle{\pgfqpoint{0.100000in}{0.212622in}}{\pgfqpoint{3.696000in}{3.696000in}}%
\pgfusepath{clip}%
\pgfsetbuttcap%
\pgfsetroundjoin%
\definecolor{currentfill}{rgb}{0.121569,0.466667,0.705882}%
\pgfsetfillcolor{currentfill}%
\pgfsetfillopacity{0.659958}%
\pgfsetlinewidth{1.003750pt}%
\definecolor{currentstroke}{rgb}{0.121569,0.466667,0.705882}%
\pgfsetstrokecolor{currentstroke}%
\pgfsetstrokeopacity{0.659958}%
\pgfsetdash{}{0pt}%
\pgfpathmoveto{\pgfqpoint{1.991045in}{3.026176in}}%
\pgfpathcurveto{\pgfqpoint{1.999281in}{3.026176in}}{\pgfqpoint{2.007181in}{3.029449in}}{\pgfqpoint{2.013005in}{3.035273in}}%
\pgfpathcurveto{\pgfqpoint{2.018829in}{3.041096in}}{\pgfqpoint{2.022102in}{3.048997in}}{\pgfqpoint{2.022102in}{3.057233in}}%
\pgfpathcurveto{\pgfqpoint{2.022102in}{3.065469in}}{\pgfqpoint{2.018829in}{3.073369in}}{\pgfqpoint{2.013005in}{3.079193in}}%
\pgfpathcurveto{\pgfqpoint{2.007181in}{3.085017in}}{\pgfqpoint{1.999281in}{3.088289in}}{\pgfqpoint{1.991045in}{3.088289in}}%
\pgfpathcurveto{\pgfqpoint{1.982809in}{3.088289in}}{\pgfqpoint{1.974909in}{3.085017in}}{\pgfqpoint{1.969085in}{3.079193in}}%
\pgfpathcurveto{\pgfqpoint{1.963261in}{3.073369in}}{\pgfqpoint{1.959989in}{3.065469in}}{\pgfqpoint{1.959989in}{3.057233in}}%
\pgfpathcurveto{\pgfqpoint{1.959989in}{3.048997in}}{\pgfqpoint{1.963261in}{3.041096in}}{\pgfqpoint{1.969085in}{3.035273in}}%
\pgfpathcurveto{\pgfqpoint{1.974909in}{3.029449in}}{\pgfqpoint{1.982809in}{3.026176in}}{\pgfqpoint{1.991045in}{3.026176in}}%
\pgfpathclose%
\pgfusepath{stroke,fill}%
\end{pgfscope}%
\begin{pgfscope}%
\pgfpathrectangle{\pgfqpoint{0.100000in}{0.212622in}}{\pgfqpoint{3.696000in}{3.696000in}}%
\pgfusepath{clip}%
\pgfsetbuttcap%
\pgfsetroundjoin%
\definecolor{currentfill}{rgb}{0.121569,0.466667,0.705882}%
\pgfsetfillcolor{currentfill}%
\pgfsetfillopacity{0.660282}%
\pgfsetlinewidth{1.003750pt}%
\definecolor{currentstroke}{rgb}{0.121569,0.466667,0.705882}%
\pgfsetstrokecolor{currentstroke}%
\pgfsetstrokeopacity{0.660282}%
\pgfsetdash{}{0pt}%
\pgfpathmoveto{\pgfqpoint{1.525816in}{2.611230in}}%
\pgfpathcurveto{\pgfqpoint{1.534052in}{2.611230in}}{\pgfqpoint{1.541952in}{2.614502in}}{\pgfqpoint{1.547776in}{2.620326in}}%
\pgfpathcurveto{\pgfqpoint{1.553600in}{2.626150in}}{\pgfqpoint{1.556872in}{2.634050in}}{\pgfqpoint{1.556872in}{2.642287in}}%
\pgfpathcurveto{\pgfqpoint{1.556872in}{2.650523in}}{\pgfqpoint{1.553600in}{2.658423in}}{\pgfqpoint{1.547776in}{2.664247in}}%
\pgfpathcurveto{\pgfqpoint{1.541952in}{2.670071in}}{\pgfqpoint{1.534052in}{2.673343in}}{\pgfqpoint{1.525816in}{2.673343in}}%
\pgfpathcurveto{\pgfqpoint{1.517580in}{2.673343in}}{\pgfqpoint{1.509680in}{2.670071in}}{\pgfqpoint{1.503856in}{2.664247in}}%
\pgfpathcurveto{\pgfqpoint{1.498032in}{2.658423in}}{\pgfqpoint{1.494759in}{2.650523in}}{\pgfqpoint{1.494759in}{2.642287in}}%
\pgfpathcurveto{\pgfqpoint{1.494759in}{2.634050in}}{\pgfqpoint{1.498032in}{2.626150in}}{\pgfqpoint{1.503856in}{2.620326in}}%
\pgfpathcurveto{\pgfqpoint{1.509680in}{2.614502in}}{\pgfqpoint{1.517580in}{2.611230in}}{\pgfqpoint{1.525816in}{2.611230in}}%
\pgfpathclose%
\pgfusepath{stroke,fill}%
\end{pgfscope}%
\begin{pgfscope}%
\pgfpathrectangle{\pgfqpoint{0.100000in}{0.212622in}}{\pgfqpoint{3.696000in}{3.696000in}}%
\pgfusepath{clip}%
\pgfsetbuttcap%
\pgfsetroundjoin%
\definecolor{currentfill}{rgb}{0.121569,0.466667,0.705882}%
\pgfsetfillcolor{currentfill}%
\pgfsetfillopacity{0.660785}%
\pgfsetlinewidth{1.003750pt}%
\definecolor{currentstroke}{rgb}{0.121569,0.466667,0.705882}%
\pgfsetstrokecolor{currentstroke}%
\pgfsetstrokeopacity{0.660785}%
\pgfsetdash{}{0pt}%
\pgfpathmoveto{\pgfqpoint{1.522704in}{2.605611in}}%
\pgfpathcurveto{\pgfqpoint{1.530941in}{2.605611in}}{\pgfqpoint{1.538841in}{2.608883in}}{\pgfqpoint{1.544665in}{2.614707in}}%
\pgfpathcurveto{\pgfqpoint{1.550489in}{2.620531in}}{\pgfqpoint{1.553761in}{2.628431in}}{\pgfqpoint{1.553761in}{2.636667in}}%
\pgfpathcurveto{\pgfqpoint{1.553761in}{2.644903in}}{\pgfqpoint{1.550489in}{2.652803in}}{\pgfqpoint{1.544665in}{2.658627in}}%
\pgfpathcurveto{\pgfqpoint{1.538841in}{2.664451in}}{\pgfqpoint{1.530941in}{2.667724in}}{\pgfqpoint{1.522704in}{2.667724in}}%
\pgfpathcurveto{\pgfqpoint{1.514468in}{2.667724in}}{\pgfqpoint{1.506568in}{2.664451in}}{\pgfqpoint{1.500744in}{2.658627in}}%
\pgfpathcurveto{\pgfqpoint{1.494920in}{2.652803in}}{\pgfqpoint{1.491648in}{2.644903in}}{\pgfqpoint{1.491648in}{2.636667in}}%
\pgfpathcurveto{\pgfqpoint{1.491648in}{2.628431in}}{\pgfqpoint{1.494920in}{2.620531in}}{\pgfqpoint{1.500744in}{2.614707in}}%
\pgfpathcurveto{\pgfqpoint{1.506568in}{2.608883in}}{\pgfqpoint{1.514468in}{2.605611in}}{\pgfqpoint{1.522704in}{2.605611in}}%
\pgfpathclose%
\pgfusepath{stroke,fill}%
\end{pgfscope}%
\begin{pgfscope}%
\pgfpathrectangle{\pgfqpoint{0.100000in}{0.212622in}}{\pgfqpoint{3.696000in}{3.696000in}}%
\pgfusepath{clip}%
\pgfsetbuttcap%
\pgfsetroundjoin%
\definecolor{currentfill}{rgb}{0.121569,0.466667,0.705882}%
\pgfsetfillcolor{currentfill}%
\pgfsetfillopacity{0.661277}%
\pgfsetlinewidth{1.003750pt}%
\definecolor{currentstroke}{rgb}{0.121569,0.466667,0.705882}%
\pgfsetstrokecolor{currentstroke}%
\pgfsetstrokeopacity{0.661277}%
\pgfsetdash{}{0pt}%
\pgfpathmoveto{\pgfqpoint{1.996705in}{3.026697in}}%
\pgfpathcurveto{\pgfqpoint{2.004941in}{3.026697in}}{\pgfqpoint{2.012841in}{3.029969in}}{\pgfqpoint{2.018665in}{3.035793in}}%
\pgfpathcurveto{\pgfqpoint{2.024489in}{3.041617in}}{\pgfqpoint{2.027761in}{3.049517in}}{\pgfqpoint{2.027761in}{3.057753in}}%
\pgfpathcurveto{\pgfqpoint{2.027761in}{3.065989in}}{\pgfqpoint{2.024489in}{3.073890in}}{\pgfqpoint{2.018665in}{3.079713in}}%
\pgfpathcurveto{\pgfqpoint{2.012841in}{3.085537in}}{\pgfqpoint{2.004941in}{3.088810in}}{\pgfqpoint{1.996705in}{3.088810in}}%
\pgfpathcurveto{\pgfqpoint{1.988469in}{3.088810in}}{\pgfqpoint{1.980568in}{3.085537in}}{\pgfqpoint{1.974745in}{3.079713in}}%
\pgfpathcurveto{\pgfqpoint{1.968921in}{3.073890in}}{\pgfqpoint{1.965648in}{3.065989in}}{\pgfqpoint{1.965648in}{3.057753in}}%
\pgfpathcurveto{\pgfqpoint{1.965648in}{3.049517in}}{\pgfqpoint{1.968921in}{3.041617in}}{\pgfqpoint{1.974745in}{3.035793in}}%
\pgfpathcurveto{\pgfqpoint{1.980568in}{3.029969in}}{\pgfqpoint{1.988469in}{3.026697in}}{\pgfqpoint{1.996705in}{3.026697in}}%
\pgfpathclose%
\pgfusepath{stroke,fill}%
\end{pgfscope}%
\begin{pgfscope}%
\pgfpathrectangle{\pgfqpoint{0.100000in}{0.212622in}}{\pgfqpoint{3.696000in}{3.696000in}}%
\pgfusepath{clip}%
\pgfsetbuttcap%
\pgfsetroundjoin%
\definecolor{currentfill}{rgb}{0.121569,0.466667,0.705882}%
\pgfsetfillcolor{currentfill}%
\pgfsetfillopacity{0.661342}%
\pgfsetlinewidth{1.003750pt}%
\definecolor{currentstroke}{rgb}{0.121569,0.466667,0.705882}%
\pgfsetstrokecolor{currentstroke}%
\pgfsetstrokeopacity{0.661342}%
\pgfsetdash{}{0pt}%
\pgfpathmoveto{\pgfqpoint{1.518536in}{2.599894in}}%
\pgfpathcurveto{\pgfqpoint{1.526772in}{2.599894in}}{\pgfqpoint{1.534672in}{2.603167in}}{\pgfqpoint{1.540496in}{2.608991in}}%
\pgfpathcurveto{\pgfqpoint{1.546320in}{2.614815in}}{\pgfqpoint{1.549593in}{2.622715in}}{\pgfqpoint{1.549593in}{2.630951in}}%
\pgfpathcurveto{\pgfqpoint{1.549593in}{2.639187in}}{\pgfqpoint{1.546320in}{2.647087in}}{\pgfqpoint{1.540496in}{2.652911in}}%
\pgfpathcurveto{\pgfqpoint{1.534672in}{2.658735in}}{\pgfqpoint{1.526772in}{2.662007in}}{\pgfqpoint{1.518536in}{2.662007in}}%
\pgfpathcurveto{\pgfqpoint{1.510300in}{2.662007in}}{\pgfqpoint{1.502400in}{2.658735in}}{\pgfqpoint{1.496576in}{2.652911in}}%
\pgfpathcurveto{\pgfqpoint{1.490752in}{2.647087in}}{\pgfqpoint{1.487480in}{2.639187in}}{\pgfqpoint{1.487480in}{2.630951in}}%
\pgfpathcurveto{\pgfqpoint{1.487480in}{2.622715in}}{\pgfqpoint{1.490752in}{2.614815in}}{\pgfqpoint{1.496576in}{2.608991in}}%
\pgfpathcurveto{\pgfqpoint{1.502400in}{2.603167in}}{\pgfqpoint{1.510300in}{2.599894in}}{\pgfqpoint{1.518536in}{2.599894in}}%
\pgfpathclose%
\pgfusepath{stroke,fill}%
\end{pgfscope}%
\begin{pgfscope}%
\pgfpathrectangle{\pgfqpoint{0.100000in}{0.212622in}}{\pgfqpoint{3.696000in}{3.696000in}}%
\pgfusepath{clip}%
\pgfsetbuttcap%
\pgfsetroundjoin%
\definecolor{currentfill}{rgb}{0.121569,0.466667,0.705882}%
\pgfsetfillcolor{currentfill}%
\pgfsetfillopacity{0.661834}%
\pgfsetlinewidth{1.003750pt}%
\definecolor{currentstroke}{rgb}{0.121569,0.466667,0.705882}%
\pgfsetstrokecolor{currentstroke}%
\pgfsetstrokeopacity{0.661834}%
\pgfsetdash{}{0pt}%
\pgfpathmoveto{\pgfqpoint{3.178341in}{2.622710in}}%
\pgfpathcurveto{\pgfqpoint{3.186578in}{2.622710in}}{\pgfqpoint{3.194478in}{2.625983in}}{\pgfqpoint{3.200302in}{2.631807in}}%
\pgfpathcurveto{\pgfqpoint{3.206125in}{2.637631in}}{\pgfqpoint{3.209398in}{2.645531in}}{\pgfqpoint{3.209398in}{2.653767in}}%
\pgfpathcurveto{\pgfqpoint{3.209398in}{2.662003in}}{\pgfqpoint{3.206125in}{2.669903in}}{\pgfqpoint{3.200302in}{2.675727in}}%
\pgfpathcurveto{\pgfqpoint{3.194478in}{2.681551in}}{\pgfqpoint{3.186578in}{2.684823in}}{\pgfqpoint{3.178341in}{2.684823in}}%
\pgfpathcurveto{\pgfqpoint{3.170105in}{2.684823in}}{\pgfqpoint{3.162205in}{2.681551in}}{\pgfqpoint{3.156381in}{2.675727in}}%
\pgfpathcurveto{\pgfqpoint{3.150557in}{2.669903in}}{\pgfqpoint{3.147285in}{2.662003in}}{\pgfqpoint{3.147285in}{2.653767in}}%
\pgfpathcurveto{\pgfqpoint{3.147285in}{2.645531in}}{\pgfqpoint{3.150557in}{2.637631in}}{\pgfqpoint{3.156381in}{2.631807in}}%
\pgfpathcurveto{\pgfqpoint{3.162205in}{2.625983in}}{\pgfqpoint{3.170105in}{2.622710in}}{\pgfqpoint{3.178341in}{2.622710in}}%
\pgfpathclose%
\pgfusepath{stroke,fill}%
\end{pgfscope}%
\begin{pgfscope}%
\pgfpathrectangle{\pgfqpoint{0.100000in}{0.212622in}}{\pgfqpoint{3.696000in}{3.696000in}}%
\pgfusepath{clip}%
\pgfsetbuttcap%
\pgfsetroundjoin%
\definecolor{currentfill}{rgb}{0.121569,0.466667,0.705882}%
\pgfsetfillcolor{currentfill}%
\pgfsetfillopacity{0.662135}%
\pgfsetlinewidth{1.003750pt}%
\definecolor{currentstroke}{rgb}{0.121569,0.466667,0.705882}%
\pgfsetstrokecolor{currentstroke}%
\pgfsetstrokeopacity{0.662135}%
\pgfsetdash{}{0pt}%
\pgfpathmoveto{\pgfqpoint{1.514272in}{2.594211in}}%
\pgfpathcurveto{\pgfqpoint{1.522509in}{2.594211in}}{\pgfqpoint{1.530409in}{2.597483in}}{\pgfqpoint{1.536233in}{2.603307in}}%
\pgfpathcurveto{\pgfqpoint{1.542057in}{2.609131in}}{\pgfqpoint{1.545329in}{2.617031in}}{\pgfqpoint{1.545329in}{2.625267in}}%
\pgfpathcurveto{\pgfqpoint{1.545329in}{2.633504in}}{\pgfqpoint{1.542057in}{2.641404in}}{\pgfqpoint{1.536233in}{2.647228in}}%
\pgfpathcurveto{\pgfqpoint{1.530409in}{2.653051in}}{\pgfqpoint{1.522509in}{2.656324in}}{\pgfqpoint{1.514272in}{2.656324in}}%
\pgfpathcurveto{\pgfqpoint{1.506036in}{2.656324in}}{\pgfqpoint{1.498136in}{2.653051in}}{\pgfqpoint{1.492312in}{2.647228in}}%
\pgfpathcurveto{\pgfqpoint{1.486488in}{2.641404in}}{\pgfqpoint{1.483216in}{2.633504in}}{\pgfqpoint{1.483216in}{2.625267in}}%
\pgfpathcurveto{\pgfqpoint{1.483216in}{2.617031in}}{\pgfqpoint{1.486488in}{2.609131in}}{\pgfqpoint{1.492312in}{2.603307in}}%
\pgfpathcurveto{\pgfqpoint{1.498136in}{2.597483in}}{\pgfqpoint{1.506036in}{2.594211in}}{\pgfqpoint{1.514272in}{2.594211in}}%
\pgfpathclose%
\pgfusepath{stroke,fill}%
\end{pgfscope}%
\begin{pgfscope}%
\pgfpathrectangle{\pgfqpoint{0.100000in}{0.212622in}}{\pgfqpoint{3.696000in}{3.696000in}}%
\pgfusepath{clip}%
\pgfsetbuttcap%
\pgfsetroundjoin%
\definecolor{currentfill}{rgb}{0.121569,0.466667,0.705882}%
\pgfsetfillcolor{currentfill}%
\pgfsetfillopacity{0.663218}%
\pgfsetlinewidth{1.003750pt}%
\definecolor{currentstroke}{rgb}{0.121569,0.466667,0.705882}%
\pgfsetstrokecolor{currentstroke}%
\pgfsetstrokeopacity{0.663218}%
\pgfsetdash{}{0pt}%
\pgfpathmoveto{\pgfqpoint{1.510796in}{2.584017in}}%
\pgfpathcurveto{\pgfqpoint{1.519032in}{2.584017in}}{\pgfqpoint{1.526932in}{2.587290in}}{\pgfqpoint{1.532756in}{2.593114in}}%
\pgfpathcurveto{\pgfqpoint{1.538580in}{2.598938in}}{\pgfqpoint{1.541853in}{2.606838in}}{\pgfqpoint{1.541853in}{2.615074in}}%
\pgfpathcurveto{\pgfqpoint{1.541853in}{2.623310in}}{\pgfqpoint{1.538580in}{2.631210in}}{\pgfqpoint{1.532756in}{2.637034in}}%
\pgfpathcurveto{\pgfqpoint{1.526932in}{2.642858in}}{\pgfqpoint{1.519032in}{2.646130in}}{\pgfqpoint{1.510796in}{2.646130in}}%
\pgfpathcurveto{\pgfqpoint{1.502560in}{2.646130in}}{\pgfqpoint{1.494660in}{2.642858in}}{\pgfqpoint{1.488836in}{2.637034in}}%
\pgfpathcurveto{\pgfqpoint{1.483012in}{2.631210in}}{\pgfqpoint{1.479740in}{2.623310in}}{\pgfqpoint{1.479740in}{2.615074in}}%
\pgfpathcurveto{\pgfqpoint{1.479740in}{2.606838in}}{\pgfqpoint{1.483012in}{2.598938in}}{\pgfqpoint{1.488836in}{2.593114in}}%
\pgfpathcurveto{\pgfqpoint{1.494660in}{2.587290in}}{\pgfqpoint{1.502560in}{2.584017in}}{\pgfqpoint{1.510796in}{2.584017in}}%
\pgfpathclose%
\pgfusepath{stroke,fill}%
\end{pgfscope}%
\begin{pgfscope}%
\pgfpathrectangle{\pgfqpoint{0.100000in}{0.212622in}}{\pgfqpoint{3.696000in}{3.696000in}}%
\pgfusepath{clip}%
\pgfsetbuttcap%
\pgfsetroundjoin%
\definecolor{currentfill}{rgb}{0.121569,0.466667,0.705882}%
\pgfsetfillcolor{currentfill}%
\pgfsetfillopacity{0.663265}%
\pgfsetlinewidth{1.003750pt}%
\definecolor{currentstroke}{rgb}{0.121569,0.466667,0.705882}%
\pgfsetstrokecolor{currentstroke}%
\pgfsetstrokeopacity{0.663265}%
\pgfsetdash{}{0pt}%
\pgfpathmoveto{\pgfqpoint{2.006866in}{3.025227in}}%
\pgfpathcurveto{\pgfqpoint{2.015103in}{3.025227in}}{\pgfqpoint{2.023003in}{3.028499in}}{\pgfqpoint{2.028827in}{3.034323in}}%
\pgfpathcurveto{\pgfqpoint{2.034651in}{3.040147in}}{\pgfqpoint{2.037923in}{3.048047in}}{\pgfqpoint{2.037923in}{3.056284in}}%
\pgfpathcurveto{\pgfqpoint{2.037923in}{3.064520in}}{\pgfqpoint{2.034651in}{3.072420in}}{\pgfqpoint{2.028827in}{3.078244in}}%
\pgfpathcurveto{\pgfqpoint{2.023003in}{3.084068in}}{\pgfqpoint{2.015103in}{3.087340in}}{\pgfqpoint{2.006866in}{3.087340in}}%
\pgfpathcurveto{\pgfqpoint{1.998630in}{3.087340in}}{\pgfqpoint{1.990730in}{3.084068in}}{\pgfqpoint{1.984906in}{3.078244in}}%
\pgfpathcurveto{\pgfqpoint{1.979082in}{3.072420in}}{\pgfqpoint{1.975810in}{3.064520in}}{\pgfqpoint{1.975810in}{3.056284in}}%
\pgfpathcurveto{\pgfqpoint{1.975810in}{3.048047in}}{\pgfqpoint{1.979082in}{3.040147in}}{\pgfqpoint{1.984906in}{3.034323in}}%
\pgfpathcurveto{\pgfqpoint{1.990730in}{3.028499in}}{\pgfqpoint{1.998630in}{3.025227in}}{\pgfqpoint{2.006866in}{3.025227in}}%
\pgfpathclose%
\pgfusepath{stroke,fill}%
\end{pgfscope}%
\begin{pgfscope}%
\pgfpathrectangle{\pgfqpoint{0.100000in}{0.212622in}}{\pgfqpoint{3.696000in}{3.696000in}}%
\pgfusepath{clip}%
\pgfsetbuttcap%
\pgfsetroundjoin%
\definecolor{currentfill}{rgb}{0.121569,0.466667,0.705882}%
\pgfsetfillcolor{currentfill}%
\pgfsetfillopacity{0.663756}%
\pgfsetlinewidth{1.003750pt}%
\definecolor{currentstroke}{rgb}{0.121569,0.466667,0.705882}%
\pgfsetstrokecolor{currentstroke}%
\pgfsetstrokeopacity{0.663756}%
\pgfsetdash{}{0pt}%
\pgfpathmoveto{\pgfqpoint{1.508473in}{2.578515in}}%
\pgfpathcurveto{\pgfqpoint{1.516709in}{2.578515in}}{\pgfqpoint{1.524609in}{2.581787in}}{\pgfqpoint{1.530433in}{2.587611in}}%
\pgfpathcurveto{\pgfqpoint{1.536257in}{2.593435in}}{\pgfqpoint{1.539529in}{2.601335in}}{\pgfqpoint{1.539529in}{2.609571in}}%
\pgfpathcurveto{\pgfqpoint{1.539529in}{2.617807in}}{\pgfqpoint{1.536257in}{2.625707in}}{\pgfqpoint{1.530433in}{2.631531in}}%
\pgfpathcurveto{\pgfqpoint{1.524609in}{2.637355in}}{\pgfqpoint{1.516709in}{2.640628in}}{\pgfqpoint{1.508473in}{2.640628in}}%
\pgfpathcurveto{\pgfqpoint{1.500237in}{2.640628in}}{\pgfqpoint{1.492337in}{2.637355in}}{\pgfqpoint{1.486513in}{2.631531in}}%
\pgfpathcurveto{\pgfqpoint{1.480689in}{2.625707in}}{\pgfqpoint{1.477416in}{2.617807in}}{\pgfqpoint{1.477416in}{2.609571in}}%
\pgfpathcurveto{\pgfqpoint{1.477416in}{2.601335in}}{\pgfqpoint{1.480689in}{2.593435in}}{\pgfqpoint{1.486513in}{2.587611in}}%
\pgfpathcurveto{\pgfqpoint{1.492337in}{2.581787in}}{\pgfqpoint{1.500237in}{2.578515in}}{\pgfqpoint{1.508473in}{2.578515in}}%
\pgfpathclose%
\pgfusepath{stroke,fill}%
\end{pgfscope}%
\begin{pgfscope}%
\pgfpathrectangle{\pgfqpoint{0.100000in}{0.212622in}}{\pgfqpoint{3.696000in}{3.696000in}}%
\pgfusepath{clip}%
\pgfsetbuttcap%
\pgfsetroundjoin%
\definecolor{currentfill}{rgb}{0.121569,0.466667,0.705882}%
\pgfsetfillcolor{currentfill}%
\pgfsetfillopacity{0.664263}%
\pgfsetlinewidth{1.003750pt}%
\definecolor{currentstroke}{rgb}{0.121569,0.466667,0.705882}%
\pgfsetstrokecolor{currentstroke}%
\pgfsetstrokeopacity{0.664263}%
\pgfsetdash{}{0pt}%
\pgfpathmoveto{\pgfqpoint{1.504772in}{2.573080in}}%
\pgfpathcurveto{\pgfqpoint{1.513008in}{2.573080in}}{\pgfqpoint{1.520908in}{2.576352in}}{\pgfqpoint{1.526732in}{2.582176in}}%
\pgfpathcurveto{\pgfqpoint{1.532556in}{2.588000in}}{\pgfqpoint{1.535828in}{2.595900in}}{\pgfqpoint{1.535828in}{2.604136in}}%
\pgfpathcurveto{\pgfqpoint{1.535828in}{2.612373in}}{\pgfqpoint{1.532556in}{2.620273in}}{\pgfqpoint{1.526732in}{2.626097in}}%
\pgfpathcurveto{\pgfqpoint{1.520908in}{2.631921in}}{\pgfqpoint{1.513008in}{2.635193in}}{\pgfqpoint{1.504772in}{2.635193in}}%
\pgfpathcurveto{\pgfqpoint{1.496536in}{2.635193in}}{\pgfqpoint{1.488636in}{2.631921in}}{\pgfqpoint{1.482812in}{2.626097in}}%
\pgfpathcurveto{\pgfqpoint{1.476988in}{2.620273in}}{\pgfqpoint{1.473715in}{2.612373in}}{\pgfqpoint{1.473715in}{2.604136in}}%
\pgfpathcurveto{\pgfqpoint{1.473715in}{2.595900in}}{\pgfqpoint{1.476988in}{2.588000in}}{\pgfqpoint{1.482812in}{2.582176in}}%
\pgfpathcurveto{\pgfqpoint{1.488636in}{2.576352in}}{\pgfqpoint{1.496536in}{2.573080in}}{\pgfqpoint{1.504772in}{2.573080in}}%
\pgfpathclose%
\pgfusepath{stroke,fill}%
\end{pgfscope}%
\begin{pgfscope}%
\pgfpathrectangle{\pgfqpoint{0.100000in}{0.212622in}}{\pgfqpoint{3.696000in}{3.696000in}}%
\pgfusepath{clip}%
\pgfsetbuttcap%
\pgfsetroundjoin%
\definecolor{currentfill}{rgb}{0.121569,0.466667,0.705882}%
\pgfsetfillcolor{currentfill}%
\pgfsetfillopacity{0.664560}%
\pgfsetlinewidth{1.003750pt}%
\definecolor{currentstroke}{rgb}{0.121569,0.466667,0.705882}%
\pgfsetstrokecolor{currentstroke}%
\pgfsetstrokeopacity{0.664560}%
\pgfsetdash{}{0pt}%
\pgfpathmoveto{\pgfqpoint{3.192445in}{2.621151in}}%
\pgfpathcurveto{\pgfqpoint{3.200681in}{2.621151in}}{\pgfqpoint{3.208581in}{2.624423in}}{\pgfqpoint{3.214405in}{2.630247in}}%
\pgfpathcurveto{\pgfqpoint{3.220229in}{2.636071in}}{\pgfqpoint{3.223502in}{2.643971in}}{\pgfqpoint{3.223502in}{2.652207in}}%
\pgfpathcurveto{\pgfqpoint{3.223502in}{2.660444in}}{\pgfqpoint{3.220229in}{2.668344in}}{\pgfqpoint{3.214405in}{2.674168in}}%
\pgfpathcurveto{\pgfqpoint{3.208581in}{2.679991in}}{\pgfqpoint{3.200681in}{2.683264in}}{\pgfqpoint{3.192445in}{2.683264in}}%
\pgfpathcurveto{\pgfqpoint{3.184209in}{2.683264in}}{\pgfqpoint{3.176309in}{2.679991in}}{\pgfqpoint{3.170485in}{2.674168in}}%
\pgfpathcurveto{\pgfqpoint{3.164661in}{2.668344in}}{\pgfqpoint{3.161389in}{2.660444in}}{\pgfqpoint{3.161389in}{2.652207in}}%
\pgfpathcurveto{\pgfqpoint{3.161389in}{2.643971in}}{\pgfqpoint{3.164661in}{2.636071in}}{\pgfqpoint{3.170485in}{2.630247in}}%
\pgfpathcurveto{\pgfqpoint{3.176309in}{2.624423in}}{\pgfqpoint{3.184209in}{2.621151in}}{\pgfqpoint{3.192445in}{2.621151in}}%
\pgfpathclose%
\pgfusepath{stroke,fill}%
\end{pgfscope}%
\begin{pgfscope}%
\pgfpathrectangle{\pgfqpoint{0.100000in}{0.212622in}}{\pgfqpoint{3.696000in}{3.696000in}}%
\pgfusepath{clip}%
\pgfsetbuttcap%
\pgfsetroundjoin%
\definecolor{currentfill}{rgb}{0.121569,0.466667,0.705882}%
\pgfsetfillcolor{currentfill}%
\pgfsetfillopacity{0.664643}%
\pgfsetlinewidth{1.003750pt}%
\definecolor{currentstroke}{rgb}{0.121569,0.466667,0.705882}%
\pgfsetstrokecolor{currentstroke}%
\pgfsetstrokeopacity{0.664643}%
\pgfsetdash{}{0pt}%
\pgfpathmoveto{\pgfqpoint{2.016120in}{3.024452in}}%
\pgfpathcurveto{\pgfqpoint{2.024356in}{3.024452in}}{\pgfqpoint{2.032256in}{3.027725in}}{\pgfqpoint{2.038080in}{3.033549in}}%
\pgfpathcurveto{\pgfqpoint{2.043904in}{3.039373in}}{\pgfqpoint{2.047176in}{3.047273in}}{\pgfqpoint{2.047176in}{3.055509in}}%
\pgfpathcurveto{\pgfqpoint{2.047176in}{3.063745in}}{\pgfqpoint{2.043904in}{3.071645in}}{\pgfqpoint{2.038080in}{3.077469in}}%
\pgfpathcurveto{\pgfqpoint{2.032256in}{3.083293in}}{\pgfqpoint{2.024356in}{3.086565in}}{\pgfqpoint{2.016120in}{3.086565in}}%
\pgfpathcurveto{\pgfqpoint{2.007884in}{3.086565in}}{\pgfqpoint{1.999984in}{3.083293in}}{\pgfqpoint{1.994160in}{3.077469in}}%
\pgfpathcurveto{\pgfqpoint{1.988336in}{3.071645in}}{\pgfqpoint{1.985063in}{3.063745in}}{\pgfqpoint{1.985063in}{3.055509in}}%
\pgfpathcurveto{\pgfqpoint{1.985063in}{3.047273in}}{\pgfqpoint{1.988336in}{3.039373in}}{\pgfqpoint{1.994160in}{3.033549in}}%
\pgfpathcurveto{\pgfqpoint{1.999984in}{3.027725in}}{\pgfqpoint{2.007884in}{3.024452in}}{\pgfqpoint{2.016120in}{3.024452in}}%
\pgfpathclose%
\pgfusepath{stroke,fill}%
\end{pgfscope}%
\begin{pgfscope}%
\pgfpathrectangle{\pgfqpoint{0.100000in}{0.212622in}}{\pgfqpoint{3.696000in}{3.696000in}}%
\pgfusepath{clip}%
\pgfsetbuttcap%
\pgfsetroundjoin%
\definecolor{currentfill}{rgb}{0.121569,0.466667,0.705882}%
\pgfsetfillcolor{currentfill}%
\pgfsetfillopacity{0.664962}%
\pgfsetlinewidth{1.003750pt}%
\definecolor{currentstroke}{rgb}{0.121569,0.466667,0.705882}%
\pgfsetstrokecolor{currentstroke}%
\pgfsetstrokeopacity{0.664962}%
\pgfsetdash{}{0pt}%
\pgfpathmoveto{\pgfqpoint{1.500647in}{2.567468in}}%
\pgfpathcurveto{\pgfqpoint{1.508883in}{2.567468in}}{\pgfqpoint{1.516783in}{2.570740in}}{\pgfqpoint{1.522607in}{2.576564in}}%
\pgfpathcurveto{\pgfqpoint{1.528431in}{2.582388in}}{\pgfqpoint{1.531703in}{2.590288in}}{\pgfqpoint{1.531703in}{2.598524in}}%
\pgfpathcurveto{\pgfqpoint{1.531703in}{2.606760in}}{\pgfqpoint{1.528431in}{2.614660in}}{\pgfqpoint{1.522607in}{2.620484in}}%
\pgfpathcurveto{\pgfqpoint{1.516783in}{2.626308in}}{\pgfqpoint{1.508883in}{2.629581in}}{\pgfqpoint{1.500647in}{2.629581in}}%
\pgfpathcurveto{\pgfqpoint{1.492411in}{2.629581in}}{\pgfqpoint{1.484511in}{2.626308in}}{\pgfqpoint{1.478687in}{2.620484in}}%
\pgfpathcurveto{\pgfqpoint{1.472863in}{2.614660in}}{\pgfqpoint{1.469590in}{2.606760in}}{\pgfqpoint{1.469590in}{2.598524in}}%
\pgfpathcurveto{\pgfqpoint{1.469590in}{2.590288in}}{\pgfqpoint{1.472863in}{2.582388in}}{\pgfqpoint{1.478687in}{2.576564in}}%
\pgfpathcurveto{\pgfqpoint{1.484511in}{2.570740in}}{\pgfqpoint{1.492411in}{2.567468in}}{\pgfqpoint{1.500647in}{2.567468in}}%
\pgfpathclose%
\pgfusepath{stroke,fill}%
\end{pgfscope}%
\begin{pgfscope}%
\pgfpathrectangle{\pgfqpoint{0.100000in}{0.212622in}}{\pgfqpoint{3.696000in}{3.696000in}}%
\pgfusepath{clip}%
\pgfsetbuttcap%
\pgfsetroundjoin%
\definecolor{currentfill}{rgb}{0.121569,0.466667,0.705882}%
\pgfsetfillcolor{currentfill}%
\pgfsetfillopacity{0.665758}%
\pgfsetlinewidth{1.003750pt}%
\definecolor{currentstroke}{rgb}{0.121569,0.466667,0.705882}%
\pgfsetstrokecolor{currentstroke}%
\pgfsetstrokeopacity{0.665758}%
\pgfsetdash{}{0pt}%
\pgfpathmoveto{\pgfqpoint{2.024717in}{3.021925in}}%
\pgfpathcurveto{\pgfqpoint{2.032953in}{3.021925in}}{\pgfqpoint{2.040853in}{3.025198in}}{\pgfqpoint{2.046677in}{3.031022in}}%
\pgfpathcurveto{\pgfqpoint{2.052501in}{3.036845in}}{\pgfqpoint{2.055773in}{3.044746in}}{\pgfqpoint{2.055773in}{3.052982in}}%
\pgfpathcurveto{\pgfqpoint{2.055773in}{3.061218in}}{\pgfqpoint{2.052501in}{3.069118in}}{\pgfqpoint{2.046677in}{3.074942in}}%
\pgfpathcurveto{\pgfqpoint{2.040853in}{3.080766in}}{\pgfqpoint{2.032953in}{3.084038in}}{\pgfqpoint{2.024717in}{3.084038in}}%
\pgfpathcurveto{\pgfqpoint{2.016480in}{3.084038in}}{\pgfqpoint{2.008580in}{3.080766in}}{\pgfqpoint{2.002756in}{3.074942in}}%
\pgfpathcurveto{\pgfqpoint{1.996933in}{3.069118in}}{\pgfqpoint{1.993660in}{3.061218in}}{\pgfqpoint{1.993660in}{3.052982in}}%
\pgfpathcurveto{\pgfqpoint{1.993660in}{3.044746in}}{\pgfqpoint{1.996933in}{3.036845in}}{\pgfqpoint{2.002756in}{3.031022in}}%
\pgfpathcurveto{\pgfqpoint{2.008580in}{3.025198in}}{\pgfqpoint{2.016480in}{3.021925in}}{\pgfqpoint{2.024717in}{3.021925in}}%
\pgfpathclose%
\pgfusepath{stroke,fill}%
\end{pgfscope}%
\begin{pgfscope}%
\pgfpathrectangle{\pgfqpoint{0.100000in}{0.212622in}}{\pgfqpoint{3.696000in}{3.696000in}}%
\pgfusepath{clip}%
\pgfsetbuttcap%
\pgfsetroundjoin%
\definecolor{currentfill}{rgb}{0.121569,0.466667,0.705882}%
\pgfsetfillcolor{currentfill}%
\pgfsetfillopacity{0.666127}%
\pgfsetlinewidth{1.003750pt}%
\definecolor{currentstroke}{rgb}{0.121569,0.466667,0.705882}%
\pgfsetstrokecolor{currentstroke}%
\pgfsetstrokeopacity{0.666127}%
\pgfsetdash{}{0pt}%
\pgfpathmoveto{\pgfqpoint{1.496652in}{2.557406in}}%
\pgfpathcurveto{\pgfqpoint{1.504889in}{2.557406in}}{\pgfqpoint{1.512789in}{2.560679in}}{\pgfqpoint{1.518613in}{2.566503in}}%
\pgfpathcurveto{\pgfqpoint{1.524437in}{2.572326in}}{\pgfqpoint{1.527709in}{2.580227in}}{\pgfqpoint{1.527709in}{2.588463in}}%
\pgfpathcurveto{\pgfqpoint{1.527709in}{2.596699in}}{\pgfqpoint{1.524437in}{2.604599in}}{\pgfqpoint{1.518613in}{2.610423in}}%
\pgfpathcurveto{\pgfqpoint{1.512789in}{2.616247in}}{\pgfqpoint{1.504889in}{2.619519in}}{\pgfqpoint{1.496652in}{2.619519in}}%
\pgfpathcurveto{\pgfqpoint{1.488416in}{2.619519in}}{\pgfqpoint{1.480516in}{2.616247in}}{\pgfqpoint{1.474692in}{2.610423in}}%
\pgfpathcurveto{\pgfqpoint{1.468868in}{2.604599in}}{\pgfqpoint{1.465596in}{2.596699in}}{\pgfqpoint{1.465596in}{2.588463in}}%
\pgfpathcurveto{\pgfqpoint{1.465596in}{2.580227in}}{\pgfqpoint{1.468868in}{2.572326in}}{\pgfqpoint{1.474692in}{2.566503in}}%
\pgfpathcurveto{\pgfqpoint{1.480516in}{2.560679in}}{\pgfqpoint{1.488416in}{2.557406in}}{\pgfqpoint{1.496652in}{2.557406in}}%
\pgfpathclose%
\pgfusepath{stroke,fill}%
\end{pgfscope}%
\begin{pgfscope}%
\pgfpathrectangle{\pgfqpoint{0.100000in}{0.212622in}}{\pgfqpoint{3.696000in}{3.696000in}}%
\pgfusepath{clip}%
\pgfsetbuttcap%
\pgfsetroundjoin%
\definecolor{currentfill}{rgb}{0.121569,0.466667,0.705882}%
\pgfsetfillcolor{currentfill}%
\pgfsetfillopacity{0.667332}%
\pgfsetlinewidth{1.003750pt}%
\definecolor{currentstroke}{rgb}{0.121569,0.466667,0.705882}%
\pgfsetstrokecolor{currentstroke}%
\pgfsetstrokeopacity{0.667332}%
\pgfsetdash{}{0pt}%
\pgfpathmoveto{\pgfqpoint{1.494530in}{2.544073in}}%
\pgfpathcurveto{\pgfqpoint{1.502767in}{2.544073in}}{\pgfqpoint{1.510667in}{2.547345in}}{\pgfqpoint{1.516491in}{2.553169in}}%
\pgfpathcurveto{\pgfqpoint{1.522314in}{2.558993in}}{\pgfqpoint{1.525587in}{2.566893in}}{\pgfqpoint{1.525587in}{2.575129in}}%
\pgfpathcurveto{\pgfqpoint{1.525587in}{2.583365in}}{\pgfqpoint{1.522314in}{2.591265in}}{\pgfqpoint{1.516491in}{2.597089in}}%
\pgfpathcurveto{\pgfqpoint{1.510667in}{2.602913in}}{\pgfqpoint{1.502767in}{2.606186in}}{\pgfqpoint{1.494530in}{2.606186in}}%
\pgfpathcurveto{\pgfqpoint{1.486294in}{2.606186in}}{\pgfqpoint{1.478394in}{2.602913in}}{\pgfqpoint{1.472570in}{2.597089in}}%
\pgfpathcurveto{\pgfqpoint{1.466746in}{2.591265in}}{\pgfqpoint{1.463474in}{2.583365in}}{\pgfqpoint{1.463474in}{2.575129in}}%
\pgfpathcurveto{\pgfqpoint{1.463474in}{2.566893in}}{\pgfqpoint{1.466746in}{2.558993in}}{\pgfqpoint{1.472570in}{2.553169in}}%
\pgfpathcurveto{\pgfqpoint{1.478394in}{2.547345in}}{\pgfqpoint{1.486294in}{2.544073in}}{\pgfqpoint{1.494530in}{2.544073in}}%
\pgfpathclose%
\pgfusepath{stroke,fill}%
\end{pgfscope}%
\begin{pgfscope}%
\pgfpathrectangle{\pgfqpoint{0.100000in}{0.212622in}}{\pgfqpoint{3.696000in}{3.696000in}}%
\pgfusepath{clip}%
\pgfsetbuttcap%
\pgfsetroundjoin%
\definecolor{currentfill}{rgb}{0.121569,0.466667,0.705882}%
\pgfsetfillcolor{currentfill}%
\pgfsetfillopacity{0.667547}%
\pgfsetlinewidth{1.003750pt}%
\definecolor{currentstroke}{rgb}{0.121569,0.466667,0.705882}%
\pgfsetstrokecolor{currentstroke}%
\pgfsetstrokeopacity{0.667547}%
\pgfsetdash{}{0pt}%
\pgfpathmoveto{\pgfqpoint{2.031982in}{3.021608in}}%
\pgfpathcurveto{\pgfqpoint{2.040218in}{3.021608in}}{\pgfqpoint{2.048118in}{3.024880in}}{\pgfqpoint{2.053942in}{3.030704in}}%
\pgfpathcurveto{\pgfqpoint{2.059766in}{3.036528in}}{\pgfqpoint{2.063038in}{3.044428in}}{\pgfqpoint{2.063038in}{3.052665in}}%
\pgfpathcurveto{\pgfqpoint{2.063038in}{3.060901in}}{\pgfqpoint{2.059766in}{3.068801in}}{\pgfqpoint{2.053942in}{3.074625in}}%
\pgfpathcurveto{\pgfqpoint{2.048118in}{3.080449in}}{\pgfqpoint{2.040218in}{3.083721in}}{\pgfqpoint{2.031982in}{3.083721in}}%
\pgfpathcurveto{\pgfqpoint{2.023745in}{3.083721in}}{\pgfqpoint{2.015845in}{3.080449in}}{\pgfqpoint{2.010021in}{3.074625in}}%
\pgfpathcurveto{\pgfqpoint{2.004197in}{3.068801in}}{\pgfqpoint{2.000925in}{3.060901in}}{\pgfqpoint{2.000925in}{3.052665in}}%
\pgfpathcurveto{\pgfqpoint{2.000925in}{3.044428in}}{\pgfqpoint{2.004197in}{3.036528in}}{\pgfqpoint{2.010021in}{3.030704in}}%
\pgfpathcurveto{\pgfqpoint{2.015845in}{3.024880in}}{\pgfqpoint{2.023745in}{3.021608in}}{\pgfqpoint{2.031982in}{3.021608in}}%
\pgfpathclose%
\pgfusepath{stroke,fill}%
\end{pgfscope}%
\begin{pgfscope}%
\pgfpathrectangle{\pgfqpoint{0.100000in}{0.212622in}}{\pgfqpoint{3.696000in}{3.696000in}}%
\pgfusepath{clip}%
\pgfsetbuttcap%
\pgfsetroundjoin%
\definecolor{currentfill}{rgb}{0.121569,0.466667,0.705882}%
\pgfsetfillcolor{currentfill}%
\pgfsetfillopacity{0.667936}%
\pgfsetlinewidth{1.003750pt}%
\definecolor{currentstroke}{rgb}{0.121569,0.466667,0.705882}%
\pgfsetstrokecolor{currentstroke}%
\pgfsetstrokeopacity{0.667936}%
\pgfsetdash{}{0pt}%
\pgfpathmoveto{\pgfqpoint{1.491495in}{2.537551in}}%
\pgfpathcurveto{\pgfqpoint{1.499731in}{2.537551in}}{\pgfqpoint{1.507631in}{2.540823in}}{\pgfqpoint{1.513455in}{2.546647in}}%
\pgfpathcurveto{\pgfqpoint{1.519279in}{2.552471in}}{\pgfqpoint{1.522551in}{2.560371in}}{\pgfqpoint{1.522551in}{2.568607in}}%
\pgfpathcurveto{\pgfqpoint{1.522551in}{2.576844in}}{\pgfqpoint{1.519279in}{2.584744in}}{\pgfqpoint{1.513455in}{2.590568in}}%
\pgfpathcurveto{\pgfqpoint{1.507631in}{2.596392in}}{\pgfqpoint{1.499731in}{2.599664in}}{\pgfqpoint{1.491495in}{2.599664in}}%
\pgfpathcurveto{\pgfqpoint{1.483258in}{2.599664in}}{\pgfqpoint{1.475358in}{2.596392in}}{\pgfqpoint{1.469534in}{2.590568in}}%
\pgfpathcurveto{\pgfqpoint{1.463710in}{2.584744in}}{\pgfqpoint{1.460438in}{2.576844in}}{\pgfqpoint{1.460438in}{2.568607in}}%
\pgfpathcurveto{\pgfqpoint{1.460438in}{2.560371in}}{\pgfqpoint{1.463710in}{2.552471in}}{\pgfqpoint{1.469534in}{2.546647in}}%
\pgfpathcurveto{\pgfqpoint{1.475358in}{2.540823in}}{\pgfqpoint{1.483258in}{2.537551in}}{\pgfqpoint{1.491495in}{2.537551in}}%
\pgfpathclose%
\pgfusepath{stroke,fill}%
\end{pgfscope}%
\begin{pgfscope}%
\pgfpathrectangle{\pgfqpoint{0.100000in}{0.212622in}}{\pgfqpoint{3.696000in}{3.696000in}}%
\pgfusepath{clip}%
\pgfsetbuttcap%
\pgfsetroundjoin%
\definecolor{currentfill}{rgb}{0.121569,0.466667,0.705882}%
\pgfsetfillcolor{currentfill}%
\pgfsetfillopacity{0.667961}%
\pgfsetlinewidth{1.003750pt}%
\definecolor{currentstroke}{rgb}{0.121569,0.466667,0.705882}%
\pgfsetstrokecolor{currentstroke}%
\pgfsetstrokeopacity{0.667961}%
\pgfsetdash{}{0pt}%
\pgfpathmoveto{\pgfqpoint{3.204740in}{2.618831in}}%
\pgfpathcurveto{\pgfqpoint{3.212976in}{2.618831in}}{\pgfqpoint{3.220876in}{2.622103in}}{\pgfqpoint{3.226700in}{2.627927in}}%
\pgfpathcurveto{\pgfqpoint{3.232524in}{2.633751in}}{\pgfqpoint{3.235796in}{2.641651in}}{\pgfqpoint{3.235796in}{2.649888in}}%
\pgfpathcurveto{\pgfqpoint{3.235796in}{2.658124in}}{\pgfqpoint{3.232524in}{2.666024in}}{\pgfqpoint{3.226700in}{2.671848in}}%
\pgfpathcurveto{\pgfqpoint{3.220876in}{2.677672in}}{\pgfqpoint{3.212976in}{2.680944in}}{\pgfqpoint{3.204740in}{2.680944in}}%
\pgfpathcurveto{\pgfqpoint{3.196504in}{2.680944in}}{\pgfqpoint{3.188604in}{2.677672in}}{\pgfqpoint{3.182780in}{2.671848in}}%
\pgfpathcurveto{\pgfqpoint{3.176956in}{2.666024in}}{\pgfqpoint{3.173683in}{2.658124in}}{\pgfqpoint{3.173683in}{2.649888in}}%
\pgfpathcurveto{\pgfqpoint{3.173683in}{2.641651in}}{\pgfqpoint{3.176956in}{2.633751in}}{\pgfqpoint{3.182780in}{2.627927in}}%
\pgfpathcurveto{\pgfqpoint{3.188604in}{2.622103in}}{\pgfqpoint{3.196504in}{2.618831in}}{\pgfqpoint{3.204740in}{2.618831in}}%
\pgfpathclose%
\pgfusepath{stroke,fill}%
\end{pgfscope}%
\begin{pgfscope}%
\pgfpathrectangle{\pgfqpoint{0.100000in}{0.212622in}}{\pgfqpoint{3.696000in}{3.696000in}}%
\pgfusepath{clip}%
\pgfsetbuttcap%
\pgfsetroundjoin%
\definecolor{currentfill}{rgb}{0.121569,0.466667,0.705882}%
\pgfsetfillcolor{currentfill}%
\pgfsetfillopacity{0.668230}%
\pgfsetlinewidth{1.003750pt}%
\definecolor{currentstroke}{rgb}{0.121569,0.466667,0.705882}%
\pgfsetstrokecolor{currentstroke}%
\pgfsetstrokeopacity{0.668230}%
\pgfsetdash{}{0pt}%
\pgfpathmoveto{\pgfqpoint{1.489479in}{2.534450in}}%
\pgfpathcurveto{\pgfqpoint{1.497716in}{2.534450in}}{\pgfqpoint{1.505616in}{2.537722in}}{\pgfqpoint{1.511440in}{2.543546in}}%
\pgfpathcurveto{\pgfqpoint{1.517264in}{2.549370in}}{\pgfqpoint{1.520536in}{2.557270in}}{\pgfqpoint{1.520536in}{2.565506in}}%
\pgfpathcurveto{\pgfqpoint{1.520536in}{2.573743in}}{\pgfqpoint{1.517264in}{2.581643in}}{\pgfqpoint{1.511440in}{2.587467in}}%
\pgfpathcurveto{\pgfqpoint{1.505616in}{2.593291in}}{\pgfqpoint{1.497716in}{2.596563in}}{\pgfqpoint{1.489479in}{2.596563in}}%
\pgfpathcurveto{\pgfqpoint{1.481243in}{2.596563in}}{\pgfqpoint{1.473343in}{2.593291in}}{\pgfqpoint{1.467519in}{2.587467in}}%
\pgfpathcurveto{\pgfqpoint{1.461695in}{2.581643in}}{\pgfqpoint{1.458423in}{2.573743in}}{\pgfqpoint{1.458423in}{2.565506in}}%
\pgfpathcurveto{\pgfqpoint{1.458423in}{2.557270in}}{\pgfqpoint{1.461695in}{2.549370in}}{\pgfqpoint{1.467519in}{2.543546in}}%
\pgfpathcurveto{\pgfqpoint{1.473343in}{2.537722in}}{\pgfqpoint{1.481243in}{2.534450in}}{\pgfqpoint{1.489479in}{2.534450in}}%
\pgfpathclose%
\pgfusepath{stroke,fill}%
\end{pgfscope}%
\begin{pgfscope}%
\pgfpathrectangle{\pgfqpoint{0.100000in}{0.212622in}}{\pgfqpoint{3.696000in}{3.696000in}}%
\pgfusepath{clip}%
\pgfsetbuttcap%
\pgfsetroundjoin%
\definecolor{currentfill}{rgb}{0.121569,0.466667,0.705882}%
\pgfsetfillcolor{currentfill}%
\pgfsetfillopacity{0.668666}%
\pgfsetlinewidth{1.003750pt}%
\definecolor{currentstroke}{rgb}{0.121569,0.466667,0.705882}%
\pgfsetstrokecolor{currentstroke}%
\pgfsetstrokeopacity{0.668666}%
\pgfsetdash{}{0pt}%
\pgfpathmoveto{\pgfqpoint{1.486872in}{2.531146in}}%
\pgfpathcurveto{\pgfqpoint{1.495108in}{2.531146in}}{\pgfqpoint{1.503008in}{2.534418in}}{\pgfqpoint{1.508832in}{2.540242in}}%
\pgfpathcurveto{\pgfqpoint{1.514656in}{2.546066in}}{\pgfqpoint{1.517929in}{2.553966in}}{\pgfqpoint{1.517929in}{2.562203in}}%
\pgfpathcurveto{\pgfqpoint{1.517929in}{2.570439in}}{\pgfqpoint{1.514656in}{2.578339in}}{\pgfqpoint{1.508832in}{2.584163in}}%
\pgfpathcurveto{\pgfqpoint{1.503008in}{2.589987in}}{\pgfqpoint{1.495108in}{2.593259in}}{\pgfqpoint{1.486872in}{2.593259in}}%
\pgfpathcurveto{\pgfqpoint{1.478636in}{2.593259in}}{\pgfqpoint{1.470736in}{2.589987in}}{\pgfqpoint{1.464912in}{2.584163in}}%
\pgfpathcurveto{\pgfqpoint{1.459088in}{2.578339in}}{\pgfqpoint{1.455816in}{2.570439in}}{\pgfqpoint{1.455816in}{2.562203in}}%
\pgfpathcurveto{\pgfqpoint{1.455816in}{2.553966in}}{\pgfqpoint{1.459088in}{2.546066in}}{\pgfqpoint{1.464912in}{2.540242in}}%
\pgfpathcurveto{\pgfqpoint{1.470736in}{2.534418in}}{\pgfqpoint{1.478636in}{2.531146in}}{\pgfqpoint{1.486872in}{2.531146in}}%
\pgfpathclose%
\pgfusepath{stroke,fill}%
\end{pgfscope}%
\begin{pgfscope}%
\pgfpathrectangle{\pgfqpoint{0.100000in}{0.212622in}}{\pgfqpoint{3.696000in}{3.696000in}}%
\pgfusepath{clip}%
\pgfsetbuttcap%
\pgfsetroundjoin%
\definecolor{currentfill}{rgb}{0.121569,0.466667,0.705882}%
\pgfsetfillcolor{currentfill}%
\pgfsetfillopacity{0.669071}%
\pgfsetlinewidth{1.003750pt}%
\definecolor{currentstroke}{rgb}{0.121569,0.466667,0.705882}%
\pgfsetstrokecolor{currentstroke}%
\pgfsetstrokeopacity{0.669071}%
\pgfsetdash{}{0pt}%
\pgfpathmoveto{\pgfqpoint{2.038299in}{3.022574in}}%
\pgfpathcurveto{\pgfqpoint{2.046536in}{3.022574in}}{\pgfqpoint{2.054436in}{3.025846in}}{\pgfqpoint{2.060260in}{3.031670in}}%
\pgfpathcurveto{\pgfqpoint{2.066084in}{3.037494in}}{\pgfqpoint{2.069356in}{3.045394in}}{\pgfqpoint{2.069356in}{3.053630in}}%
\pgfpathcurveto{\pgfqpoint{2.069356in}{3.061866in}}{\pgfqpoint{2.066084in}{3.069766in}}{\pgfqpoint{2.060260in}{3.075590in}}%
\pgfpathcurveto{\pgfqpoint{2.054436in}{3.081414in}}{\pgfqpoint{2.046536in}{3.084687in}}{\pgfqpoint{2.038299in}{3.084687in}}%
\pgfpathcurveto{\pgfqpoint{2.030063in}{3.084687in}}{\pgfqpoint{2.022163in}{3.081414in}}{\pgfqpoint{2.016339in}{3.075590in}}%
\pgfpathcurveto{\pgfqpoint{2.010515in}{3.069766in}}{\pgfqpoint{2.007243in}{3.061866in}}{\pgfqpoint{2.007243in}{3.053630in}}%
\pgfpathcurveto{\pgfqpoint{2.007243in}{3.045394in}}{\pgfqpoint{2.010515in}{3.037494in}}{\pgfqpoint{2.016339in}{3.031670in}}%
\pgfpathcurveto{\pgfqpoint{2.022163in}{3.025846in}}{\pgfqpoint{2.030063in}{3.022574in}}{\pgfqpoint{2.038299in}{3.022574in}}%
\pgfpathclose%
\pgfusepath{stroke,fill}%
\end{pgfscope}%
\begin{pgfscope}%
\pgfpathrectangle{\pgfqpoint{0.100000in}{0.212622in}}{\pgfqpoint{3.696000in}{3.696000in}}%
\pgfusepath{clip}%
\pgfsetbuttcap%
\pgfsetroundjoin%
\definecolor{currentfill}{rgb}{0.121569,0.466667,0.705882}%
\pgfsetfillcolor{currentfill}%
\pgfsetfillopacity{0.669385}%
\pgfsetlinewidth{1.003750pt}%
\definecolor{currentstroke}{rgb}{0.121569,0.466667,0.705882}%
\pgfsetstrokecolor{currentstroke}%
\pgfsetstrokeopacity{0.669385}%
\pgfsetdash{}{0pt}%
\pgfpathmoveto{\pgfqpoint{1.483576in}{2.525063in}}%
\pgfpathcurveto{\pgfqpoint{1.491813in}{2.525063in}}{\pgfqpoint{1.499713in}{2.528335in}}{\pgfqpoint{1.505537in}{2.534159in}}%
\pgfpathcurveto{\pgfqpoint{1.511361in}{2.539983in}}{\pgfqpoint{1.514633in}{2.547883in}}{\pgfqpoint{1.514633in}{2.556120in}}%
\pgfpathcurveto{\pgfqpoint{1.514633in}{2.564356in}}{\pgfqpoint{1.511361in}{2.572256in}}{\pgfqpoint{1.505537in}{2.578080in}}%
\pgfpathcurveto{\pgfqpoint{1.499713in}{2.583904in}}{\pgfqpoint{1.491813in}{2.587176in}}{\pgfqpoint{1.483576in}{2.587176in}}%
\pgfpathcurveto{\pgfqpoint{1.475340in}{2.587176in}}{\pgfqpoint{1.467440in}{2.583904in}}{\pgfqpoint{1.461616in}{2.578080in}}%
\pgfpathcurveto{\pgfqpoint{1.455792in}{2.572256in}}{\pgfqpoint{1.452520in}{2.564356in}}{\pgfqpoint{1.452520in}{2.556120in}}%
\pgfpathcurveto{\pgfqpoint{1.452520in}{2.547883in}}{\pgfqpoint{1.455792in}{2.539983in}}{\pgfqpoint{1.461616in}{2.534159in}}%
\pgfpathcurveto{\pgfqpoint{1.467440in}{2.528335in}}{\pgfqpoint{1.475340in}{2.525063in}}{\pgfqpoint{1.483576in}{2.525063in}}%
\pgfpathclose%
\pgfusepath{stroke,fill}%
\end{pgfscope}%
\begin{pgfscope}%
\pgfpathrectangle{\pgfqpoint{0.100000in}{0.212622in}}{\pgfqpoint{3.696000in}{3.696000in}}%
\pgfusepath{clip}%
\pgfsetbuttcap%
\pgfsetroundjoin%
\definecolor{currentfill}{rgb}{0.121569,0.466667,0.705882}%
\pgfsetfillcolor{currentfill}%
\pgfsetfillopacity{0.669815}%
\pgfsetlinewidth{1.003750pt}%
\definecolor{currentstroke}{rgb}{0.121569,0.466667,0.705882}%
\pgfsetstrokecolor{currentstroke}%
\pgfsetstrokeopacity{0.669815}%
\pgfsetdash{}{0pt}%
\pgfpathmoveto{\pgfqpoint{2.044560in}{3.021859in}}%
\pgfpathcurveto{\pgfqpoint{2.052796in}{3.021859in}}{\pgfqpoint{2.060696in}{3.025132in}}{\pgfqpoint{2.066520in}{3.030956in}}%
\pgfpathcurveto{\pgfqpoint{2.072344in}{3.036780in}}{\pgfqpoint{2.075617in}{3.044680in}}{\pgfqpoint{2.075617in}{3.052916in}}%
\pgfpathcurveto{\pgfqpoint{2.075617in}{3.061152in}}{\pgfqpoint{2.072344in}{3.069052in}}{\pgfqpoint{2.066520in}{3.074876in}}%
\pgfpathcurveto{\pgfqpoint{2.060696in}{3.080700in}}{\pgfqpoint{2.052796in}{3.083972in}}{\pgfqpoint{2.044560in}{3.083972in}}%
\pgfpathcurveto{\pgfqpoint{2.036324in}{3.083972in}}{\pgfqpoint{2.028424in}{3.080700in}}{\pgfqpoint{2.022600in}{3.074876in}}%
\pgfpathcurveto{\pgfqpoint{2.016776in}{3.069052in}}{\pgfqpoint{2.013504in}{3.061152in}}{\pgfqpoint{2.013504in}{3.052916in}}%
\pgfpathcurveto{\pgfqpoint{2.013504in}{3.044680in}}{\pgfqpoint{2.016776in}{3.036780in}}{\pgfqpoint{2.022600in}{3.030956in}}%
\pgfpathcurveto{\pgfqpoint{2.028424in}{3.025132in}}{\pgfqpoint{2.036324in}{3.021859in}}{\pgfqpoint{2.044560in}{3.021859in}}%
\pgfpathclose%
\pgfusepath{stroke,fill}%
\end{pgfscope}%
\begin{pgfscope}%
\pgfpathrectangle{\pgfqpoint{0.100000in}{0.212622in}}{\pgfqpoint{3.696000in}{3.696000in}}%
\pgfusepath{clip}%
\pgfsetbuttcap%
\pgfsetroundjoin%
\definecolor{currentfill}{rgb}{0.121569,0.466667,0.705882}%
\pgfsetfillcolor{currentfill}%
\pgfsetfillopacity{0.670390}%
\pgfsetlinewidth{1.003750pt}%
\definecolor{currentstroke}{rgb}{0.121569,0.466667,0.705882}%
\pgfsetstrokecolor{currentstroke}%
\pgfsetstrokeopacity{0.670390}%
\pgfsetdash{}{0pt}%
\pgfpathmoveto{\pgfqpoint{1.480933in}{2.516454in}}%
\pgfpathcurveto{\pgfqpoint{1.489169in}{2.516454in}}{\pgfqpoint{1.497069in}{2.519726in}}{\pgfqpoint{1.502893in}{2.525550in}}%
\pgfpathcurveto{\pgfqpoint{1.508717in}{2.531374in}}{\pgfqpoint{1.511989in}{2.539274in}}{\pgfqpoint{1.511989in}{2.547511in}}%
\pgfpathcurveto{\pgfqpoint{1.511989in}{2.555747in}}{\pgfqpoint{1.508717in}{2.563647in}}{\pgfqpoint{1.502893in}{2.569471in}}%
\pgfpathcurveto{\pgfqpoint{1.497069in}{2.575295in}}{\pgfqpoint{1.489169in}{2.578567in}}{\pgfqpoint{1.480933in}{2.578567in}}%
\pgfpathcurveto{\pgfqpoint{1.472697in}{2.578567in}}{\pgfqpoint{1.464796in}{2.575295in}}{\pgfqpoint{1.458973in}{2.569471in}}%
\pgfpathcurveto{\pgfqpoint{1.453149in}{2.563647in}}{\pgfqpoint{1.449876in}{2.555747in}}{\pgfqpoint{1.449876in}{2.547511in}}%
\pgfpathcurveto{\pgfqpoint{1.449876in}{2.539274in}}{\pgfqpoint{1.453149in}{2.531374in}}{\pgfqpoint{1.458973in}{2.525550in}}%
\pgfpathcurveto{\pgfqpoint{1.464796in}{2.519726in}}{\pgfqpoint{1.472697in}{2.516454in}}{\pgfqpoint{1.480933in}{2.516454in}}%
\pgfpathclose%
\pgfusepath{stroke,fill}%
\end{pgfscope}%
\begin{pgfscope}%
\pgfpathrectangle{\pgfqpoint{0.100000in}{0.212622in}}{\pgfqpoint{3.696000in}{3.696000in}}%
\pgfusepath{clip}%
\pgfsetbuttcap%
\pgfsetroundjoin%
\definecolor{currentfill}{rgb}{0.121569,0.466667,0.705882}%
\pgfsetfillcolor{currentfill}%
\pgfsetfillopacity{0.670584}%
\pgfsetlinewidth{1.003750pt}%
\definecolor{currentstroke}{rgb}{0.121569,0.466667,0.705882}%
\pgfsetstrokecolor{currentstroke}%
\pgfsetstrokeopacity{0.670584}%
\pgfsetdash{}{0pt}%
\pgfpathmoveto{\pgfqpoint{2.049256in}{3.021153in}}%
\pgfpathcurveto{\pgfqpoint{2.057492in}{3.021153in}}{\pgfqpoint{2.065392in}{3.024425in}}{\pgfqpoint{2.071216in}{3.030249in}}%
\pgfpathcurveto{\pgfqpoint{2.077040in}{3.036073in}}{\pgfqpoint{2.080312in}{3.043973in}}{\pgfqpoint{2.080312in}{3.052210in}}%
\pgfpathcurveto{\pgfqpoint{2.080312in}{3.060446in}}{\pgfqpoint{2.077040in}{3.068346in}}{\pgfqpoint{2.071216in}{3.074170in}}%
\pgfpathcurveto{\pgfqpoint{2.065392in}{3.079994in}}{\pgfqpoint{2.057492in}{3.083266in}}{\pgfqpoint{2.049256in}{3.083266in}}%
\pgfpathcurveto{\pgfqpoint{2.041019in}{3.083266in}}{\pgfqpoint{2.033119in}{3.079994in}}{\pgfqpoint{2.027295in}{3.074170in}}%
\pgfpathcurveto{\pgfqpoint{2.021471in}{3.068346in}}{\pgfqpoint{2.018199in}{3.060446in}}{\pgfqpoint{2.018199in}{3.052210in}}%
\pgfpathcurveto{\pgfqpoint{2.018199in}{3.043973in}}{\pgfqpoint{2.021471in}{3.036073in}}{\pgfqpoint{2.027295in}{3.030249in}}%
\pgfpathcurveto{\pgfqpoint{2.033119in}{3.024425in}}{\pgfqpoint{2.041019in}{3.021153in}}{\pgfqpoint{2.049256in}{3.021153in}}%
\pgfpathclose%
\pgfusepath{stroke,fill}%
\end{pgfscope}%
\begin{pgfscope}%
\pgfpathrectangle{\pgfqpoint{0.100000in}{0.212622in}}{\pgfqpoint{3.696000in}{3.696000in}}%
\pgfusepath{clip}%
\pgfsetbuttcap%
\pgfsetroundjoin%
\definecolor{currentfill}{rgb}{0.121569,0.466667,0.705882}%
\pgfsetfillcolor{currentfill}%
\pgfsetfillopacity{0.670611}%
\pgfsetlinewidth{1.003750pt}%
\definecolor{currentstroke}{rgb}{0.121569,0.466667,0.705882}%
\pgfsetstrokecolor{currentstroke}%
\pgfsetstrokeopacity{0.670611}%
\pgfsetdash{}{0pt}%
\pgfpathmoveto{\pgfqpoint{3.217622in}{2.616493in}}%
\pgfpathcurveto{\pgfqpoint{3.225858in}{2.616493in}}{\pgfqpoint{3.233759in}{2.619765in}}{\pgfqpoint{3.239582in}{2.625589in}}%
\pgfpathcurveto{\pgfqpoint{3.245406in}{2.631413in}}{\pgfqpoint{3.248679in}{2.639313in}}{\pgfqpoint{3.248679in}{2.647549in}}%
\pgfpathcurveto{\pgfqpoint{3.248679in}{2.655786in}}{\pgfqpoint{3.245406in}{2.663686in}}{\pgfqpoint{3.239582in}{2.669510in}}%
\pgfpathcurveto{\pgfqpoint{3.233759in}{2.675334in}}{\pgfqpoint{3.225858in}{2.678606in}}{\pgfqpoint{3.217622in}{2.678606in}}%
\pgfpathcurveto{\pgfqpoint{3.209386in}{2.678606in}}{\pgfqpoint{3.201486in}{2.675334in}}{\pgfqpoint{3.195662in}{2.669510in}}%
\pgfpathcurveto{\pgfqpoint{3.189838in}{2.663686in}}{\pgfqpoint{3.186566in}{2.655786in}}{\pgfqpoint{3.186566in}{2.647549in}}%
\pgfpathcurveto{\pgfqpoint{3.186566in}{2.639313in}}{\pgfqpoint{3.189838in}{2.631413in}}{\pgfqpoint{3.195662in}{2.625589in}}%
\pgfpathcurveto{\pgfqpoint{3.201486in}{2.619765in}}{\pgfqpoint{3.209386in}{2.616493in}}{\pgfqpoint{3.217622in}{2.616493in}}%
\pgfpathclose%
\pgfusepath{stroke,fill}%
\end{pgfscope}%
\begin{pgfscope}%
\pgfpathrectangle{\pgfqpoint{0.100000in}{0.212622in}}{\pgfqpoint{3.696000in}{3.696000in}}%
\pgfusepath{clip}%
\pgfsetbuttcap%
\pgfsetroundjoin%
\definecolor{currentfill}{rgb}{0.121569,0.466667,0.705882}%
\pgfsetfillcolor{currentfill}%
\pgfsetfillopacity{0.670918}%
\pgfsetlinewidth{1.003750pt}%
\definecolor{currentstroke}{rgb}{0.121569,0.466667,0.705882}%
\pgfsetstrokecolor{currentstroke}%
\pgfsetstrokeopacity{0.670918}%
\pgfsetdash{}{0pt}%
\pgfpathmoveto{\pgfqpoint{1.479061in}{2.511896in}}%
\pgfpathcurveto{\pgfqpoint{1.487297in}{2.511896in}}{\pgfqpoint{1.495197in}{2.515168in}}{\pgfqpoint{1.501021in}{2.520992in}}%
\pgfpathcurveto{\pgfqpoint{1.506845in}{2.526816in}}{\pgfqpoint{1.510118in}{2.534716in}}{\pgfqpoint{1.510118in}{2.542953in}}%
\pgfpathcurveto{\pgfqpoint{1.510118in}{2.551189in}}{\pgfqpoint{1.506845in}{2.559089in}}{\pgfqpoint{1.501021in}{2.564913in}}%
\pgfpathcurveto{\pgfqpoint{1.495197in}{2.570737in}}{\pgfqpoint{1.487297in}{2.574009in}}{\pgfqpoint{1.479061in}{2.574009in}}%
\pgfpathcurveto{\pgfqpoint{1.470825in}{2.574009in}}{\pgfqpoint{1.462925in}{2.570737in}}{\pgfqpoint{1.457101in}{2.564913in}}%
\pgfpathcurveto{\pgfqpoint{1.451277in}{2.559089in}}{\pgfqpoint{1.448005in}{2.551189in}}{\pgfqpoint{1.448005in}{2.542953in}}%
\pgfpathcurveto{\pgfqpoint{1.448005in}{2.534716in}}{\pgfqpoint{1.451277in}{2.526816in}}{\pgfqpoint{1.457101in}{2.520992in}}%
\pgfpathcurveto{\pgfqpoint{1.462925in}{2.515168in}}{\pgfqpoint{1.470825in}{2.511896in}}{\pgfqpoint{1.479061in}{2.511896in}}%
\pgfpathclose%
\pgfusepath{stroke,fill}%
\end{pgfscope}%
\begin{pgfscope}%
\pgfpathrectangle{\pgfqpoint{0.100000in}{0.212622in}}{\pgfqpoint{3.696000in}{3.696000in}}%
\pgfusepath{clip}%
\pgfsetbuttcap%
\pgfsetroundjoin%
\definecolor{currentfill}{rgb}{0.121569,0.466667,0.705882}%
\pgfsetfillcolor{currentfill}%
\pgfsetfillopacity{0.671439}%
\pgfsetlinewidth{1.003750pt}%
\definecolor{currentstroke}{rgb}{0.121569,0.466667,0.705882}%
\pgfsetstrokecolor{currentstroke}%
\pgfsetstrokeopacity{0.671439}%
\pgfsetdash{}{0pt}%
\pgfpathmoveto{\pgfqpoint{1.475702in}{2.507193in}}%
\pgfpathcurveto{\pgfqpoint{1.483938in}{2.507193in}}{\pgfqpoint{1.491838in}{2.510465in}}{\pgfqpoint{1.497662in}{2.516289in}}%
\pgfpathcurveto{\pgfqpoint{1.503486in}{2.522113in}}{\pgfqpoint{1.506758in}{2.530013in}}{\pgfqpoint{1.506758in}{2.538249in}}%
\pgfpathcurveto{\pgfqpoint{1.506758in}{2.546486in}}{\pgfqpoint{1.503486in}{2.554386in}}{\pgfqpoint{1.497662in}{2.560210in}}%
\pgfpathcurveto{\pgfqpoint{1.491838in}{2.566034in}}{\pgfqpoint{1.483938in}{2.569306in}}{\pgfqpoint{1.475702in}{2.569306in}}%
\pgfpathcurveto{\pgfqpoint{1.467465in}{2.569306in}}{\pgfqpoint{1.459565in}{2.566034in}}{\pgfqpoint{1.453741in}{2.560210in}}%
\pgfpathcurveto{\pgfqpoint{1.447917in}{2.554386in}}{\pgfqpoint{1.444645in}{2.546486in}}{\pgfqpoint{1.444645in}{2.538249in}}%
\pgfpathcurveto{\pgfqpoint{1.444645in}{2.530013in}}{\pgfqpoint{1.447917in}{2.522113in}}{\pgfqpoint{1.453741in}{2.516289in}}%
\pgfpathcurveto{\pgfqpoint{1.459565in}{2.510465in}}{\pgfqpoint{1.467465in}{2.507193in}}{\pgfqpoint{1.475702in}{2.507193in}}%
\pgfpathclose%
\pgfusepath{stroke,fill}%
\end{pgfscope}%
\begin{pgfscope}%
\pgfpathrectangle{\pgfqpoint{0.100000in}{0.212622in}}{\pgfqpoint{3.696000in}{3.696000in}}%
\pgfusepath{clip}%
\pgfsetbuttcap%
\pgfsetroundjoin%
\definecolor{currentfill}{rgb}{0.121569,0.466667,0.705882}%
\pgfsetfillcolor{currentfill}%
\pgfsetfillopacity{0.671686}%
\pgfsetlinewidth{1.003750pt}%
\definecolor{currentstroke}{rgb}{0.121569,0.466667,0.705882}%
\pgfsetstrokecolor{currentstroke}%
\pgfsetstrokeopacity{0.671686}%
\pgfsetdash{}{0pt}%
\pgfpathmoveto{\pgfqpoint{2.058112in}{3.019783in}}%
\pgfpathcurveto{\pgfqpoint{2.066349in}{3.019783in}}{\pgfqpoint{2.074249in}{3.023056in}}{\pgfqpoint{2.080073in}{3.028880in}}%
\pgfpathcurveto{\pgfqpoint{2.085897in}{3.034704in}}{\pgfqpoint{2.089169in}{3.042604in}}{\pgfqpoint{2.089169in}{3.050840in}}%
\pgfpathcurveto{\pgfqpoint{2.089169in}{3.059076in}}{\pgfqpoint{2.085897in}{3.066976in}}{\pgfqpoint{2.080073in}{3.072800in}}%
\pgfpathcurveto{\pgfqpoint{2.074249in}{3.078624in}}{\pgfqpoint{2.066349in}{3.081896in}}{\pgfqpoint{2.058112in}{3.081896in}}%
\pgfpathcurveto{\pgfqpoint{2.049876in}{3.081896in}}{\pgfqpoint{2.041976in}{3.078624in}}{\pgfqpoint{2.036152in}{3.072800in}}%
\pgfpathcurveto{\pgfqpoint{2.030328in}{3.066976in}}{\pgfqpoint{2.027056in}{3.059076in}}{\pgfqpoint{2.027056in}{3.050840in}}%
\pgfpathcurveto{\pgfqpoint{2.027056in}{3.042604in}}{\pgfqpoint{2.030328in}{3.034704in}}{\pgfqpoint{2.036152in}{3.028880in}}%
\pgfpathcurveto{\pgfqpoint{2.041976in}{3.023056in}}{\pgfqpoint{2.049876in}{3.019783in}}{\pgfqpoint{2.058112in}{3.019783in}}%
\pgfpathclose%
\pgfusepath{stroke,fill}%
\end{pgfscope}%
\begin{pgfscope}%
\pgfpathrectangle{\pgfqpoint{0.100000in}{0.212622in}}{\pgfqpoint{3.696000in}{3.696000in}}%
\pgfusepath{clip}%
\pgfsetbuttcap%
\pgfsetroundjoin%
\definecolor{currentfill}{rgb}{0.121569,0.466667,0.705882}%
\pgfsetfillcolor{currentfill}%
\pgfsetfillopacity{0.672136}%
\pgfsetlinewidth{1.003750pt}%
\definecolor{currentstroke}{rgb}{0.121569,0.466667,0.705882}%
\pgfsetstrokecolor{currentstroke}%
\pgfsetstrokeopacity{0.672136}%
\pgfsetdash{}{0pt}%
\pgfpathmoveto{\pgfqpoint{1.471924in}{2.502373in}}%
\pgfpathcurveto{\pgfqpoint{1.480160in}{2.502373in}}{\pgfqpoint{1.488060in}{2.505646in}}{\pgfqpoint{1.493884in}{2.511470in}}%
\pgfpathcurveto{\pgfqpoint{1.499708in}{2.517294in}}{\pgfqpoint{1.502980in}{2.525194in}}{\pgfqpoint{1.502980in}{2.533430in}}%
\pgfpathcurveto{\pgfqpoint{1.502980in}{2.541666in}}{\pgfqpoint{1.499708in}{2.549566in}}{\pgfqpoint{1.493884in}{2.555390in}}%
\pgfpathcurveto{\pgfqpoint{1.488060in}{2.561214in}}{\pgfqpoint{1.480160in}{2.564486in}}{\pgfqpoint{1.471924in}{2.564486in}}%
\pgfpathcurveto{\pgfqpoint{1.463687in}{2.564486in}}{\pgfqpoint{1.455787in}{2.561214in}}{\pgfqpoint{1.449963in}{2.555390in}}%
\pgfpathcurveto{\pgfqpoint{1.444140in}{2.549566in}}{\pgfqpoint{1.440867in}{2.541666in}}{\pgfqpoint{1.440867in}{2.533430in}}%
\pgfpathcurveto{\pgfqpoint{1.440867in}{2.525194in}}{\pgfqpoint{1.444140in}{2.517294in}}{\pgfqpoint{1.449963in}{2.511470in}}%
\pgfpathcurveto{\pgfqpoint{1.455787in}{2.505646in}}{\pgfqpoint{1.463687in}{2.502373in}}{\pgfqpoint{1.471924in}{2.502373in}}%
\pgfpathclose%
\pgfusepath{stroke,fill}%
\end{pgfscope}%
\begin{pgfscope}%
\pgfpathrectangle{\pgfqpoint{0.100000in}{0.212622in}}{\pgfqpoint{3.696000in}{3.696000in}}%
\pgfusepath{clip}%
\pgfsetbuttcap%
\pgfsetroundjoin%
\definecolor{currentfill}{rgb}{0.121569,0.466667,0.705882}%
\pgfsetfillcolor{currentfill}%
\pgfsetfillopacity{0.673257}%
\pgfsetlinewidth{1.003750pt}%
\definecolor{currentstroke}{rgb}{0.121569,0.466667,0.705882}%
\pgfsetstrokecolor{currentstroke}%
\pgfsetstrokeopacity{0.673257}%
\pgfsetdash{}{0pt}%
\pgfpathmoveto{\pgfqpoint{1.467892in}{2.495179in}}%
\pgfpathcurveto{\pgfqpoint{1.476128in}{2.495179in}}{\pgfqpoint{1.484029in}{2.498451in}}{\pgfqpoint{1.489852in}{2.504275in}}%
\pgfpathcurveto{\pgfqpoint{1.495676in}{2.510099in}}{\pgfqpoint{1.498949in}{2.517999in}}{\pgfqpoint{1.498949in}{2.526235in}}%
\pgfpathcurveto{\pgfqpoint{1.498949in}{2.534471in}}{\pgfqpoint{1.495676in}{2.542371in}}{\pgfqpoint{1.489852in}{2.548195in}}%
\pgfpathcurveto{\pgfqpoint{1.484029in}{2.554019in}}{\pgfqpoint{1.476128in}{2.557292in}}{\pgfqpoint{1.467892in}{2.557292in}}%
\pgfpathcurveto{\pgfqpoint{1.459656in}{2.557292in}}{\pgfqpoint{1.451756in}{2.554019in}}{\pgfqpoint{1.445932in}{2.548195in}}%
\pgfpathcurveto{\pgfqpoint{1.440108in}{2.542371in}}{\pgfqpoint{1.436836in}{2.534471in}}{\pgfqpoint{1.436836in}{2.526235in}}%
\pgfpathcurveto{\pgfqpoint{1.436836in}{2.517999in}}{\pgfqpoint{1.440108in}{2.510099in}}{\pgfqpoint{1.445932in}{2.504275in}}%
\pgfpathcurveto{\pgfqpoint{1.451756in}{2.498451in}}{\pgfqpoint{1.459656in}{2.495179in}}{\pgfqpoint{1.467892in}{2.495179in}}%
\pgfpathclose%
\pgfusepath{stroke,fill}%
\end{pgfscope}%
\begin{pgfscope}%
\pgfpathrectangle{\pgfqpoint{0.100000in}{0.212622in}}{\pgfqpoint{3.696000in}{3.696000in}}%
\pgfusepath{clip}%
\pgfsetbuttcap%
\pgfsetroundjoin%
\definecolor{currentfill}{rgb}{0.121569,0.466667,0.705882}%
\pgfsetfillcolor{currentfill}%
\pgfsetfillopacity{0.673381}%
\pgfsetlinewidth{1.003750pt}%
\definecolor{currentstroke}{rgb}{0.121569,0.466667,0.705882}%
\pgfsetstrokecolor{currentstroke}%
\pgfsetstrokeopacity{0.673381}%
\pgfsetdash{}{0pt}%
\pgfpathmoveto{\pgfqpoint{2.064585in}{3.020426in}}%
\pgfpathcurveto{\pgfqpoint{2.072821in}{3.020426in}}{\pgfqpoint{2.080721in}{3.023699in}}{\pgfqpoint{2.086545in}{3.029522in}}%
\pgfpathcurveto{\pgfqpoint{2.092369in}{3.035346in}}{\pgfqpoint{2.095641in}{3.043246in}}{\pgfqpoint{2.095641in}{3.051483in}}%
\pgfpathcurveto{\pgfqpoint{2.095641in}{3.059719in}}{\pgfqpoint{2.092369in}{3.067619in}}{\pgfqpoint{2.086545in}{3.073443in}}%
\pgfpathcurveto{\pgfqpoint{2.080721in}{3.079267in}}{\pgfqpoint{2.072821in}{3.082539in}}{\pgfqpoint{2.064585in}{3.082539in}}%
\pgfpathcurveto{\pgfqpoint{2.056349in}{3.082539in}}{\pgfqpoint{2.048449in}{3.079267in}}{\pgfqpoint{2.042625in}{3.073443in}}%
\pgfpathcurveto{\pgfqpoint{2.036801in}{3.067619in}}{\pgfqpoint{2.033528in}{3.059719in}}{\pgfqpoint{2.033528in}{3.051483in}}%
\pgfpathcurveto{\pgfqpoint{2.033528in}{3.043246in}}{\pgfqpoint{2.036801in}{3.035346in}}{\pgfqpoint{2.042625in}{3.029522in}}%
\pgfpathcurveto{\pgfqpoint{2.048449in}{3.023699in}}{\pgfqpoint{2.056349in}{3.020426in}}{\pgfqpoint{2.064585in}{3.020426in}}%
\pgfpathclose%
\pgfusepath{stroke,fill}%
\end{pgfscope}%
\begin{pgfscope}%
\pgfpathrectangle{\pgfqpoint{0.100000in}{0.212622in}}{\pgfqpoint{3.696000in}{3.696000in}}%
\pgfusepath{clip}%
\pgfsetbuttcap%
\pgfsetroundjoin%
\definecolor{currentfill}{rgb}{0.121569,0.466667,0.705882}%
\pgfsetfillcolor{currentfill}%
\pgfsetfillopacity{0.673572}%
\pgfsetlinewidth{1.003750pt}%
\definecolor{currentstroke}{rgb}{0.121569,0.466667,0.705882}%
\pgfsetstrokecolor{currentstroke}%
\pgfsetstrokeopacity{0.673572}%
\pgfsetdash{}{0pt}%
\pgfpathmoveto{\pgfqpoint{3.229262in}{2.615765in}}%
\pgfpathcurveto{\pgfqpoint{3.237498in}{2.615765in}}{\pgfqpoint{3.245398in}{2.619037in}}{\pgfqpoint{3.251222in}{2.624861in}}%
\pgfpathcurveto{\pgfqpoint{3.257046in}{2.630685in}}{\pgfqpoint{3.260319in}{2.638585in}}{\pgfqpoint{3.260319in}{2.646822in}}%
\pgfpathcurveto{\pgfqpoint{3.260319in}{2.655058in}}{\pgfqpoint{3.257046in}{2.662958in}}{\pgfqpoint{3.251222in}{2.668782in}}%
\pgfpathcurveto{\pgfqpoint{3.245398in}{2.674606in}}{\pgfqpoint{3.237498in}{2.677878in}}{\pgfqpoint{3.229262in}{2.677878in}}%
\pgfpathcurveto{\pgfqpoint{3.221026in}{2.677878in}}{\pgfqpoint{3.213126in}{2.674606in}}{\pgfqpoint{3.207302in}{2.668782in}}%
\pgfpathcurveto{\pgfqpoint{3.201478in}{2.662958in}}{\pgfqpoint{3.198206in}{2.655058in}}{\pgfqpoint{3.198206in}{2.646822in}}%
\pgfpathcurveto{\pgfqpoint{3.198206in}{2.638585in}}{\pgfqpoint{3.201478in}{2.630685in}}{\pgfqpoint{3.207302in}{2.624861in}}%
\pgfpathcurveto{\pgfqpoint{3.213126in}{2.619037in}}{\pgfqpoint{3.221026in}{2.615765in}}{\pgfqpoint{3.229262in}{2.615765in}}%
\pgfpathclose%
\pgfusepath{stroke,fill}%
\end{pgfscope}%
\begin{pgfscope}%
\pgfpathrectangle{\pgfqpoint{0.100000in}{0.212622in}}{\pgfqpoint{3.696000in}{3.696000in}}%
\pgfusepath{clip}%
\pgfsetbuttcap%
\pgfsetroundjoin%
\definecolor{currentfill}{rgb}{0.121569,0.466667,0.705882}%
\pgfsetfillcolor{currentfill}%
\pgfsetfillopacity{0.674349}%
\pgfsetlinewidth{1.003750pt}%
\definecolor{currentstroke}{rgb}{0.121569,0.466667,0.705882}%
\pgfsetstrokecolor{currentstroke}%
\pgfsetstrokeopacity{0.674349}%
\pgfsetdash{}{0pt}%
\pgfpathmoveto{\pgfqpoint{1.465261in}{2.484300in}}%
\pgfpathcurveto{\pgfqpoint{1.473497in}{2.484300in}}{\pgfqpoint{1.481397in}{2.487572in}}{\pgfqpoint{1.487221in}{2.493396in}}%
\pgfpathcurveto{\pgfqpoint{1.493045in}{2.499220in}}{\pgfqpoint{1.496317in}{2.507120in}}{\pgfqpoint{1.496317in}{2.515357in}}%
\pgfpathcurveto{\pgfqpoint{1.496317in}{2.523593in}}{\pgfqpoint{1.493045in}{2.531493in}}{\pgfqpoint{1.487221in}{2.537317in}}%
\pgfpathcurveto{\pgfqpoint{1.481397in}{2.543141in}}{\pgfqpoint{1.473497in}{2.546413in}}{\pgfqpoint{1.465261in}{2.546413in}}%
\pgfpathcurveto{\pgfqpoint{1.457025in}{2.546413in}}{\pgfqpoint{1.449125in}{2.543141in}}{\pgfqpoint{1.443301in}{2.537317in}}%
\pgfpathcurveto{\pgfqpoint{1.437477in}{2.531493in}}{\pgfqpoint{1.434204in}{2.523593in}}{\pgfqpoint{1.434204in}{2.515357in}}%
\pgfpathcurveto{\pgfqpoint{1.434204in}{2.507120in}}{\pgfqpoint{1.437477in}{2.499220in}}{\pgfqpoint{1.443301in}{2.493396in}}%
\pgfpathcurveto{\pgfqpoint{1.449125in}{2.487572in}}{\pgfqpoint{1.457025in}{2.484300in}}{\pgfqpoint{1.465261in}{2.484300in}}%
\pgfpathclose%
\pgfusepath{stroke,fill}%
\end{pgfscope}%
\begin{pgfscope}%
\pgfpathrectangle{\pgfqpoint{0.100000in}{0.212622in}}{\pgfqpoint{3.696000in}{3.696000in}}%
\pgfusepath{clip}%
\pgfsetbuttcap%
\pgfsetroundjoin%
\definecolor{currentfill}{rgb}{0.121569,0.466667,0.705882}%
\pgfsetfillcolor{currentfill}%
\pgfsetfillopacity{0.674942}%
\pgfsetlinewidth{1.003750pt}%
\definecolor{currentstroke}{rgb}{0.121569,0.466667,0.705882}%
\pgfsetstrokecolor{currentstroke}%
\pgfsetstrokeopacity{0.674942}%
\pgfsetdash{}{0pt}%
\pgfpathmoveto{\pgfqpoint{1.462694in}{2.479139in}}%
\pgfpathcurveto{\pgfqpoint{1.470930in}{2.479139in}}{\pgfqpoint{1.478830in}{2.482411in}}{\pgfqpoint{1.484654in}{2.488235in}}%
\pgfpathcurveto{\pgfqpoint{1.490478in}{2.494059in}}{\pgfqpoint{1.493750in}{2.501959in}}{\pgfqpoint{1.493750in}{2.510195in}}%
\pgfpathcurveto{\pgfqpoint{1.493750in}{2.518432in}}{\pgfqpoint{1.490478in}{2.526332in}}{\pgfqpoint{1.484654in}{2.532156in}}%
\pgfpathcurveto{\pgfqpoint{1.478830in}{2.537980in}}{\pgfqpoint{1.470930in}{2.541252in}}{\pgfqpoint{1.462694in}{2.541252in}}%
\pgfpathcurveto{\pgfqpoint{1.454457in}{2.541252in}}{\pgfqpoint{1.446557in}{2.537980in}}{\pgfqpoint{1.440733in}{2.532156in}}%
\pgfpathcurveto{\pgfqpoint{1.434910in}{2.526332in}}{\pgfqpoint{1.431637in}{2.518432in}}{\pgfqpoint{1.431637in}{2.510195in}}%
\pgfpathcurveto{\pgfqpoint{1.431637in}{2.501959in}}{\pgfqpoint{1.434910in}{2.494059in}}{\pgfqpoint{1.440733in}{2.488235in}}%
\pgfpathcurveto{\pgfqpoint{1.446557in}{2.482411in}}{\pgfqpoint{1.454457in}{2.479139in}}{\pgfqpoint{1.462694in}{2.479139in}}%
\pgfpathclose%
\pgfusepath{stroke,fill}%
\end{pgfscope}%
\begin{pgfscope}%
\pgfpathrectangle{\pgfqpoint{0.100000in}{0.212622in}}{\pgfqpoint{3.696000in}{3.696000in}}%
\pgfusepath{clip}%
\pgfsetbuttcap%
\pgfsetroundjoin%
\definecolor{currentfill}{rgb}{0.121569,0.466667,0.705882}%
\pgfsetfillcolor{currentfill}%
\pgfsetfillopacity{0.675514}%
\pgfsetlinewidth{1.003750pt}%
\definecolor{currentstroke}{rgb}{0.121569,0.466667,0.705882}%
\pgfsetstrokecolor{currentstroke}%
\pgfsetstrokeopacity{0.675514}%
\pgfsetdash{}{0pt}%
\pgfpathmoveto{\pgfqpoint{1.459250in}{2.474231in}}%
\pgfpathcurveto{\pgfqpoint{1.467487in}{2.474231in}}{\pgfqpoint{1.475387in}{2.477503in}}{\pgfqpoint{1.481211in}{2.483327in}}%
\pgfpathcurveto{\pgfqpoint{1.487034in}{2.489151in}}{\pgfqpoint{1.490307in}{2.497051in}}{\pgfqpoint{1.490307in}{2.505288in}}%
\pgfpathcurveto{\pgfqpoint{1.490307in}{2.513524in}}{\pgfqpoint{1.487034in}{2.521424in}}{\pgfqpoint{1.481211in}{2.527248in}}%
\pgfpathcurveto{\pgfqpoint{1.475387in}{2.533072in}}{\pgfqpoint{1.467487in}{2.536344in}}{\pgfqpoint{1.459250in}{2.536344in}}%
\pgfpathcurveto{\pgfqpoint{1.451014in}{2.536344in}}{\pgfqpoint{1.443114in}{2.533072in}}{\pgfqpoint{1.437290in}{2.527248in}}%
\pgfpathcurveto{\pgfqpoint{1.431466in}{2.521424in}}{\pgfqpoint{1.428194in}{2.513524in}}{\pgfqpoint{1.428194in}{2.505288in}}%
\pgfpathcurveto{\pgfqpoint{1.428194in}{2.497051in}}{\pgfqpoint{1.431466in}{2.489151in}}{\pgfqpoint{1.437290in}{2.483327in}}%
\pgfpathcurveto{\pgfqpoint{1.443114in}{2.477503in}}{\pgfqpoint{1.451014in}{2.474231in}}{\pgfqpoint{1.459250in}{2.474231in}}%
\pgfpathclose%
\pgfusepath{stroke,fill}%
\end{pgfscope}%
\begin{pgfscope}%
\pgfpathrectangle{\pgfqpoint{0.100000in}{0.212622in}}{\pgfqpoint{3.696000in}{3.696000in}}%
\pgfusepath{clip}%
\pgfsetbuttcap%
\pgfsetroundjoin%
\definecolor{currentfill}{rgb}{0.121569,0.466667,0.705882}%
\pgfsetfillcolor{currentfill}%
\pgfsetfillopacity{0.675612}%
\pgfsetlinewidth{1.003750pt}%
\definecolor{currentstroke}{rgb}{0.121569,0.466667,0.705882}%
\pgfsetstrokecolor{currentstroke}%
\pgfsetstrokeopacity{0.675612}%
\pgfsetdash{}{0pt}%
\pgfpathmoveto{\pgfqpoint{2.076405in}{3.017790in}}%
\pgfpathcurveto{\pgfqpoint{2.084641in}{3.017790in}}{\pgfqpoint{2.092542in}{3.021063in}}{\pgfqpoint{2.098365in}{3.026887in}}%
\pgfpathcurveto{\pgfqpoint{2.104189in}{3.032710in}}{\pgfqpoint{2.107462in}{3.040611in}}{\pgfqpoint{2.107462in}{3.048847in}}%
\pgfpathcurveto{\pgfqpoint{2.107462in}{3.057083in}}{\pgfqpoint{2.104189in}{3.064983in}}{\pgfqpoint{2.098365in}{3.070807in}}%
\pgfpathcurveto{\pgfqpoint{2.092542in}{3.076631in}}{\pgfqpoint{2.084641in}{3.079903in}}{\pgfqpoint{2.076405in}{3.079903in}}%
\pgfpathcurveto{\pgfqpoint{2.068169in}{3.079903in}}{\pgfqpoint{2.060269in}{3.076631in}}{\pgfqpoint{2.054445in}{3.070807in}}%
\pgfpathcurveto{\pgfqpoint{2.048621in}{3.064983in}}{\pgfqpoint{2.045349in}{3.057083in}}{\pgfqpoint{2.045349in}{3.048847in}}%
\pgfpathcurveto{\pgfqpoint{2.045349in}{3.040611in}}{\pgfqpoint{2.048621in}{3.032710in}}{\pgfqpoint{2.054445in}{3.026887in}}%
\pgfpathcurveto{\pgfqpoint{2.060269in}{3.021063in}}{\pgfqpoint{2.068169in}{3.017790in}}{\pgfqpoint{2.076405in}{3.017790in}}%
\pgfpathclose%
\pgfusepath{stroke,fill}%
\end{pgfscope}%
\begin{pgfscope}%
\pgfpathrectangle{\pgfqpoint{0.100000in}{0.212622in}}{\pgfqpoint{3.696000in}{3.696000in}}%
\pgfusepath{clip}%
\pgfsetbuttcap%
\pgfsetroundjoin%
\definecolor{currentfill}{rgb}{0.121569,0.466667,0.705882}%
\pgfsetfillcolor{currentfill}%
\pgfsetfillopacity{0.676309}%
\pgfsetlinewidth{1.003750pt}%
\definecolor{currentstroke}{rgb}{0.121569,0.466667,0.705882}%
\pgfsetstrokecolor{currentstroke}%
\pgfsetstrokeopacity{0.676309}%
\pgfsetdash{}{0pt}%
\pgfpathmoveto{\pgfqpoint{1.455808in}{2.468931in}}%
\pgfpathcurveto{\pgfqpoint{1.464044in}{2.468931in}}{\pgfqpoint{1.471944in}{2.472203in}}{\pgfqpoint{1.477768in}{2.478027in}}%
\pgfpathcurveto{\pgfqpoint{1.483592in}{2.483851in}}{\pgfqpoint{1.486864in}{2.491751in}}{\pgfqpoint{1.486864in}{2.499987in}}%
\pgfpathcurveto{\pgfqpoint{1.486864in}{2.508224in}}{\pgfqpoint{1.483592in}{2.516124in}}{\pgfqpoint{1.477768in}{2.521948in}}%
\pgfpathcurveto{\pgfqpoint{1.471944in}{2.527771in}}{\pgfqpoint{1.464044in}{2.531044in}}{\pgfqpoint{1.455808in}{2.531044in}}%
\pgfpathcurveto{\pgfqpoint{1.447571in}{2.531044in}}{\pgfqpoint{1.439671in}{2.527771in}}{\pgfqpoint{1.433847in}{2.521948in}}%
\pgfpathcurveto{\pgfqpoint{1.428023in}{2.516124in}}{\pgfqpoint{1.424751in}{2.508224in}}{\pgfqpoint{1.424751in}{2.499987in}}%
\pgfpathcurveto{\pgfqpoint{1.424751in}{2.491751in}}{\pgfqpoint{1.428023in}{2.483851in}}{\pgfqpoint{1.433847in}{2.478027in}}%
\pgfpathcurveto{\pgfqpoint{1.439671in}{2.472203in}}{\pgfqpoint{1.447571in}{2.468931in}}{\pgfqpoint{1.455808in}{2.468931in}}%
\pgfpathclose%
\pgfusepath{stroke,fill}%
\end{pgfscope}%
\begin{pgfscope}%
\pgfpathrectangle{\pgfqpoint{0.100000in}{0.212622in}}{\pgfqpoint{3.696000in}{3.696000in}}%
\pgfusepath{clip}%
\pgfsetbuttcap%
\pgfsetroundjoin%
\definecolor{currentfill}{rgb}{0.121569,0.466667,0.705882}%
\pgfsetfillcolor{currentfill}%
\pgfsetfillopacity{0.676596}%
\pgfsetlinewidth{1.003750pt}%
\definecolor{currentstroke}{rgb}{0.121569,0.466667,0.705882}%
\pgfsetstrokecolor{currentstroke}%
\pgfsetstrokeopacity{0.676596}%
\pgfsetdash{}{0pt}%
\pgfpathmoveto{\pgfqpoint{3.238847in}{2.616929in}}%
\pgfpathcurveto{\pgfqpoint{3.247083in}{2.616929in}}{\pgfqpoint{3.254983in}{2.620202in}}{\pgfqpoint{3.260807in}{2.626026in}}%
\pgfpathcurveto{\pgfqpoint{3.266631in}{2.631850in}}{\pgfqpoint{3.269903in}{2.639750in}}{\pgfqpoint{3.269903in}{2.647986in}}%
\pgfpathcurveto{\pgfqpoint{3.269903in}{2.656222in}}{\pgfqpoint{3.266631in}{2.664122in}}{\pgfqpoint{3.260807in}{2.669946in}}%
\pgfpathcurveto{\pgfqpoint{3.254983in}{2.675770in}}{\pgfqpoint{3.247083in}{2.679042in}}{\pgfqpoint{3.238847in}{2.679042in}}%
\pgfpathcurveto{\pgfqpoint{3.230611in}{2.679042in}}{\pgfqpoint{3.222711in}{2.675770in}}{\pgfqpoint{3.216887in}{2.669946in}}%
\pgfpathcurveto{\pgfqpoint{3.211063in}{2.664122in}}{\pgfqpoint{3.207790in}{2.656222in}}{\pgfqpoint{3.207790in}{2.647986in}}%
\pgfpathcurveto{\pgfqpoint{3.207790in}{2.639750in}}{\pgfqpoint{3.211063in}{2.631850in}}{\pgfqpoint{3.216887in}{2.626026in}}%
\pgfpathcurveto{\pgfqpoint{3.222711in}{2.620202in}}{\pgfqpoint{3.230611in}{2.616929in}}{\pgfqpoint{3.238847in}{2.616929in}}%
\pgfpathclose%
\pgfusepath{stroke,fill}%
\end{pgfscope}%
\begin{pgfscope}%
\pgfpathrectangle{\pgfqpoint{0.100000in}{0.212622in}}{\pgfqpoint{3.696000in}{3.696000in}}%
\pgfusepath{clip}%
\pgfsetbuttcap%
\pgfsetroundjoin%
\definecolor{currentfill}{rgb}{0.121569,0.466667,0.705882}%
\pgfsetfillcolor{currentfill}%
\pgfsetfillopacity{0.677242}%
\pgfsetlinewidth{1.003750pt}%
\definecolor{currentstroke}{rgb}{0.121569,0.466667,0.705882}%
\pgfsetstrokecolor{currentstroke}%
\pgfsetstrokeopacity{0.677242}%
\pgfsetdash{}{0pt}%
\pgfpathmoveto{\pgfqpoint{1.452611in}{2.460318in}}%
\pgfpathcurveto{\pgfqpoint{1.460848in}{2.460318in}}{\pgfqpoint{1.468748in}{2.463590in}}{\pgfqpoint{1.474572in}{2.469414in}}%
\pgfpathcurveto{\pgfqpoint{1.480396in}{2.475238in}}{\pgfqpoint{1.483668in}{2.483138in}}{\pgfqpoint{1.483668in}{2.491374in}}%
\pgfpathcurveto{\pgfqpoint{1.483668in}{2.499611in}}{\pgfqpoint{1.480396in}{2.507511in}}{\pgfqpoint{1.474572in}{2.513335in}}%
\pgfpathcurveto{\pgfqpoint{1.468748in}{2.519159in}}{\pgfqpoint{1.460848in}{2.522431in}}{\pgfqpoint{1.452611in}{2.522431in}}%
\pgfpathcurveto{\pgfqpoint{1.444375in}{2.522431in}}{\pgfqpoint{1.436475in}{2.519159in}}{\pgfqpoint{1.430651in}{2.513335in}}%
\pgfpathcurveto{\pgfqpoint{1.424827in}{2.507511in}}{\pgfqpoint{1.421555in}{2.499611in}}{\pgfqpoint{1.421555in}{2.491374in}}%
\pgfpathcurveto{\pgfqpoint{1.421555in}{2.483138in}}{\pgfqpoint{1.424827in}{2.475238in}}{\pgfqpoint{1.430651in}{2.469414in}}%
\pgfpathcurveto{\pgfqpoint{1.436475in}{2.463590in}}{\pgfqpoint{1.444375in}{2.460318in}}{\pgfqpoint{1.452611in}{2.460318in}}%
\pgfpathclose%
\pgfusepath{stroke,fill}%
\end{pgfscope}%
\begin{pgfscope}%
\pgfpathrectangle{\pgfqpoint{0.100000in}{0.212622in}}{\pgfqpoint{3.696000in}{3.696000in}}%
\pgfusepath{clip}%
\pgfsetbuttcap%
\pgfsetroundjoin%
\definecolor{currentfill}{rgb}{0.121569,0.466667,0.705882}%
\pgfsetfillcolor{currentfill}%
\pgfsetfillopacity{0.677480}%
\pgfsetlinewidth{1.003750pt}%
\definecolor{currentstroke}{rgb}{0.121569,0.466667,0.705882}%
\pgfsetstrokecolor{currentstroke}%
\pgfsetstrokeopacity{0.677480}%
\pgfsetdash{}{0pt}%
\pgfpathmoveto{\pgfqpoint{2.086918in}{3.015963in}}%
\pgfpathcurveto{\pgfqpoint{2.095154in}{3.015963in}}{\pgfqpoint{2.103054in}{3.019235in}}{\pgfqpoint{2.108878in}{3.025059in}}%
\pgfpathcurveto{\pgfqpoint{2.114702in}{3.030883in}}{\pgfqpoint{2.117974in}{3.038783in}}{\pgfqpoint{2.117974in}{3.047019in}}%
\pgfpathcurveto{\pgfqpoint{2.117974in}{3.055256in}}{\pgfqpoint{2.114702in}{3.063156in}}{\pgfqpoint{2.108878in}{3.068980in}}%
\pgfpathcurveto{\pgfqpoint{2.103054in}{3.074804in}}{\pgfqpoint{2.095154in}{3.078076in}}{\pgfqpoint{2.086918in}{3.078076in}}%
\pgfpathcurveto{\pgfqpoint{2.078682in}{3.078076in}}{\pgfqpoint{2.070782in}{3.074804in}}{\pgfqpoint{2.064958in}{3.068980in}}%
\pgfpathcurveto{\pgfqpoint{2.059134in}{3.063156in}}{\pgfqpoint{2.055861in}{3.055256in}}{\pgfqpoint{2.055861in}{3.047019in}}%
\pgfpathcurveto{\pgfqpoint{2.055861in}{3.038783in}}{\pgfqpoint{2.059134in}{3.030883in}}{\pgfqpoint{2.064958in}{3.025059in}}%
\pgfpathcurveto{\pgfqpoint{2.070782in}{3.019235in}}{\pgfqpoint{2.078682in}{3.015963in}}{\pgfqpoint{2.086918in}{3.015963in}}%
\pgfpathclose%
\pgfusepath{stroke,fill}%
\end{pgfscope}%
\begin{pgfscope}%
\pgfpathrectangle{\pgfqpoint{0.100000in}{0.212622in}}{\pgfqpoint{3.696000in}{3.696000in}}%
\pgfusepath{clip}%
\pgfsetbuttcap%
\pgfsetroundjoin%
\definecolor{currentfill}{rgb}{0.121569,0.466667,0.705882}%
\pgfsetfillcolor{currentfill}%
\pgfsetfillopacity{0.677702}%
\pgfsetlinewidth{1.003750pt}%
\definecolor{currentstroke}{rgb}{0.121569,0.466667,0.705882}%
\pgfsetstrokecolor{currentstroke}%
\pgfsetstrokeopacity{0.677702}%
\pgfsetdash{}{0pt}%
\pgfpathmoveto{\pgfqpoint{1.450853in}{2.455338in}}%
\pgfpathcurveto{\pgfqpoint{1.459089in}{2.455338in}}{\pgfqpoint{1.466989in}{2.458610in}}{\pgfqpoint{1.472813in}{2.464434in}}%
\pgfpathcurveto{\pgfqpoint{1.478637in}{2.470258in}}{\pgfqpoint{1.481909in}{2.478158in}}{\pgfqpoint{1.481909in}{2.486394in}}%
\pgfpathcurveto{\pgfqpoint{1.481909in}{2.494630in}}{\pgfqpoint{1.478637in}{2.502530in}}{\pgfqpoint{1.472813in}{2.508354in}}%
\pgfpathcurveto{\pgfqpoint{1.466989in}{2.514178in}}{\pgfqpoint{1.459089in}{2.517451in}}{\pgfqpoint{1.450853in}{2.517451in}}%
\pgfpathcurveto{\pgfqpoint{1.442617in}{2.517451in}}{\pgfqpoint{1.434717in}{2.514178in}}{\pgfqpoint{1.428893in}{2.508354in}}%
\pgfpathcurveto{\pgfqpoint{1.423069in}{2.502530in}}{\pgfqpoint{1.419796in}{2.494630in}}{\pgfqpoint{1.419796in}{2.486394in}}%
\pgfpathcurveto{\pgfqpoint{1.419796in}{2.478158in}}{\pgfqpoint{1.423069in}{2.470258in}}{\pgfqpoint{1.428893in}{2.464434in}}%
\pgfpathcurveto{\pgfqpoint{1.434717in}{2.458610in}}{\pgfqpoint{1.442617in}{2.455338in}}{\pgfqpoint{1.450853in}{2.455338in}}%
\pgfpathclose%
\pgfusepath{stroke,fill}%
\end{pgfscope}%
\begin{pgfscope}%
\pgfpathrectangle{\pgfqpoint{0.100000in}{0.212622in}}{\pgfqpoint{3.696000in}{3.696000in}}%
\pgfusepath{clip}%
\pgfsetbuttcap%
\pgfsetroundjoin%
\definecolor{currentfill}{rgb}{0.121569,0.466667,0.705882}%
\pgfsetfillcolor{currentfill}%
\pgfsetfillopacity{0.678256}%
\pgfsetlinewidth{1.003750pt}%
\definecolor{currentstroke}{rgb}{0.121569,0.466667,0.705882}%
\pgfsetstrokecolor{currentstroke}%
\pgfsetstrokeopacity{0.678256}%
\pgfsetdash{}{0pt}%
\pgfpathmoveto{\pgfqpoint{1.447317in}{2.449665in}}%
\pgfpathcurveto{\pgfqpoint{1.455553in}{2.449665in}}{\pgfqpoint{1.463453in}{2.452937in}}{\pgfqpoint{1.469277in}{2.458761in}}%
\pgfpathcurveto{\pgfqpoint{1.475101in}{2.464585in}}{\pgfqpoint{1.478373in}{2.472485in}}{\pgfqpoint{1.478373in}{2.480721in}}%
\pgfpathcurveto{\pgfqpoint{1.478373in}{2.488957in}}{\pgfqpoint{1.475101in}{2.496858in}}{\pgfqpoint{1.469277in}{2.502681in}}%
\pgfpathcurveto{\pgfqpoint{1.463453in}{2.508505in}}{\pgfqpoint{1.455553in}{2.511778in}}{\pgfqpoint{1.447317in}{2.511778in}}%
\pgfpathcurveto{\pgfqpoint{1.439080in}{2.511778in}}{\pgfqpoint{1.431180in}{2.508505in}}{\pgfqpoint{1.425356in}{2.502681in}}%
\pgfpathcurveto{\pgfqpoint{1.419532in}{2.496858in}}{\pgfqpoint{1.416260in}{2.488957in}}{\pgfqpoint{1.416260in}{2.480721in}}%
\pgfpathcurveto{\pgfqpoint{1.416260in}{2.472485in}}{\pgfqpoint{1.419532in}{2.464585in}}{\pgfqpoint{1.425356in}{2.458761in}}%
\pgfpathcurveto{\pgfqpoint{1.431180in}{2.452937in}}{\pgfqpoint{1.439080in}{2.449665in}}{\pgfqpoint{1.447317in}{2.449665in}}%
\pgfpathclose%
\pgfusepath{stroke,fill}%
\end{pgfscope}%
\begin{pgfscope}%
\pgfpathrectangle{\pgfqpoint{0.100000in}{0.212622in}}{\pgfqpoint{3.696000in}{3.696000in}}%
\pgfusepath{clip}%
\pgfsetbuttcap%
\pgfsetroundjoin%
\definecolor{currentfill}{rgb}{0.121569,0.466667,0.705882}%
\pgfsetfillcolor{currentfill}%
\pgfsetfillopacity{0.678581}%
\pgfsetlinewidth{1.003750pt}%
\definecolor{currentstroke}{rgb}{0.121569,0.466667,0.705882}%
\pgfsetstrokecolor{currentstroke}%
\pgfsetstrokeopacity{0.678581}%
\pgfsetdash{}{0pt}%
\pgfpathmoveto{\pgfqpoint{2.097885in}{3.014268in}}%
\pgfpathcurveto{\pgfqpoint{2.106122in}{3.014268in}}{\pgfqpoint{2.114022in}{3.017540in}}{\pgfqpoint{2.119846in}{3.023364in}}%
\pgfpathcurveto{\pgfqpoint{2.125670in}{3.029188in}}{\pgfqpoint{2.128942in}{3.037088in}}{\pgfqpoint{2.128942in}{3.045324in}}%
\pgfpathcurveto{\pgfqpoint{2.128942in}{3.053560in}}{\pgfqpoint{2.125670in}{3.061460in}}{\pgfqpoint{2.119846in}{3.067284in}}%
\pgfpathcurveto{\pgfqpoint{2.114022in}{3.073108in}}{\pgfqpoint{2.106122in}{3.076381in}}{\pgfqpoint{2.097885in}{3.076381in}}%
\pgfpathcurveto{\pgfqpoint{2.089649in}{3.076381in}}{\pgfqpoint{2.081749in}{3.073108in}}{\pgfqpoint{2.075925in}{3.067284in}}%
\pgfpathcurveto{\pgfqpoint{2.070101in}{3.061460in}}{\pgfqpoint{2.066829in}{3.053560in}}{\pgfqpoint{2.066829in}{3.045324in}}%
\pgfpathcurveto{\pgfqpoint{2.066829in}{3.037088in}}{\pgfqpoint{2.070101in}{3.029188in}}{\pgfqpoint{2.075925in}{3.023364in}}%
\pgfpathcurveto{\pgfqpoint{2.081749in}{3.017540in}}{\pgfqpoint{2.089649in}{3.014268in}}{\pgfqpoint{2.097885in}{3.014268in}}%
\pgfpathclose%
\pgfusepath{stroke,fill}%
\end{pgfscope}%
\begin{pgfscope}%
\pgfpathrectangle{\pgfqpoint{0.100000in}{0.212622in}}{\pgfqpoint{3.696000in}{3.696000in}}%
\pgfusepath{clip}%
\pgfsetbuttcap%
\pgfsetroundjoin%
\definecolor{currentfill}{rgb}{0.121569,0.466667,0.705882}%
\pgfsetfillcolor{currentfill}%
\pgfsetfillopacity{0.678865}%
\pgfsetlinewidth{1.003750pt}%
\definecolor{currentstroke}{rgb}{0.121569,0.466667,0.705882}%
\pgfsetstrokecolor{currentstroke}%
\pgfsetstrokeopacity{0.678865}%
\pgfsetdash{}{0pt}%
\pgfpathmoveto{\pgfqpoint{1.443206in}{2.444195in}}%
\pgfpathcurveto{\pgfqpoint{1.451442in}{2.444195in}}{\pgfqpoint{1.459342in}{2.447468in}}{\pgfqpoint{1.465166in}{2.453291in}}%
\pgfpathcurveto{\pgfqpoint{1.470990in}{2.459115in}}{\pgfqpoint{1.474262in}{2.467015in}}{\pgfqpoint{1.474262in}{2.475252in}}%
\pgfpathcurveto{\pgfqpoint{1.474262in}{2.483488in}}{\pgfqpoint{1.470990in}{2.491388in}}{\pgfqpoint{1.465166in}{2.497212in}}%
\pgfpathcurveto{\pgfqpoint{1.459342in}{2.503036in}}{\pgfqpoint{1.451442in}{2.506308in}}{\pgfqpoint{1.443206in}{2.506308in}}%
\pgfpathcurveto{\pgfqpoint{1.434970in}{2.506308in}}{\pgfqpoint{1.427069in}{2.503036in}}{\pgfqpoint{1.421246in}{2.497212in}}%
\pgfpathcurveto{\pgfqpoint{1.415422in}{2.491388in}}{\pgfqpoint{1.412149in}{2.483488in}}{\pgfqpoint{1.412149in}{2.475252in}}%
\pgfpathcurveto{\pgfqpoint{1.412149in}{2.467015in}}{\pgfqpoint{1.415422in}{2.459115in}}{\pgfqpoint{1.421246in}{2.453291in}}%
\pgfpathcurveto{\pgfqpoint{1.427069in}{2.447468in}}{\pgfqpoint{1.434970in}{2.444195in}}{\pgfqpoint{1.443206in}{2.444195in}}%
\pgfpathclose%
\pgfusepath{stroke,fill}%
\end{pgfscope}%
\begin{pgfscope}%
\pgfpathrectangle{\pgfqpoint{0.100000in}{0.212622in}}{\pgfqpoint{3.696000in}{3.696000in}}%
\pgfusepath{clip}%
\pgfsetbuttcap%
\pgfsetroundjoin%
\definecolor{currentfill}{rgb}{0.121569,0.466667,0.705882}%
\pgfsetfillcolor{currentfill}%
\pgfsetfillopacity{0.679229}%
\pgfsetlinewidth{1.003750pt}%
\definecolor{currentstroke}{rgb}{0.121569,0.466667,0.705882}%
\pgfsetstrokecolor{currentstroke}%
\pgfsetstrokeopacity{0.679229}%
\pgfsetdash{}{0pt}%
\pgfpathmoveto{\pgfqpoint{3.248508in}{2.618573in}}%
\pgfpathcurveto{\pgfqpoint{3.256745in}{2.618573in}}{\pgfqpoint{3.264645in}{2.621845in}}{\pgfqpoint{3.270469in}{2.627669in}}%
\pgfpathcurveto{\pgfqpoint{3.276293in}{2.633493in}}{\pgfqpoint{3.279565in}{2.641393in}}{\pgfqpoint{3.279565in}{2.649630in}}%
\pgfpathcurveto{\pgfqpoint{3.279565in}{2.657866in}}{\pgfqpoint{3.276293in}{2.665766in}}{\pgfqpoint{3.270469in}{2.671590in}}%
\pgfpathcurveto{\pgfqpoint{3.264645in}{2.677414in}}{\pgfqpoint{3.256745in}{2.680686in}}{\pgfqpoint{3.248508in}{2.680686in}}%
\pgfpathcurveto{\pgfqpoint{3.240272in}{2.680686in}}{\pgfqpoint{3.232372in}{2.677414in}}{\pgfqpoint{3.226548in}{2.671590in}}%
\pgfpathcurveto{\pgfqpoint{3.220724in}{2.665766in}}{\pgfqpoint{3.217452in}{2.657866in}}{\pgfqpoint{3.217452in}{2.649630in}}%
\pgfpathcurveto{\pgfqpoint{3.217452in}{2.641393in}}{\pgfqpoint{3.220724in}{2.633493in}}{\pgfqpoint{3.226548in}{2.627669in}}%
\pgfpathcurveto{\pgfqpoint{3.232372in}{2.621845in}}{\pgfqpoint{3.240272in}{2.618573in}}{\pgfqpoint{3.248508in}{2.618573in}}%
\pgfpathclose%
\pgfusepath{stroke,fill}%
\end{pgfscope}%
\begin{pgfscope}%
\pgfpathrectangle{\pgfqpoint{0.100000in}{0.212622in}}{\pgfqpoint{3.696000in}{3.696000in}}%
\pgfusepath{clip}%
\pgfsetbuttcap%
\pgfsetroundjoin%
\definecolor{currentfill}{rgb}{0.121569,0.466667,0.705882}%
\pgfsetfillcolor{currentfill}%
\pgfsetfillopacity{0.679262}%
\pgfsetlinewidth{1.003750pt}%
\definecolor{currentstroke}{rgb}{0.121569,0.466667,0.705882}%
\pgfsetstrokecolor{currentstroke}%
\pgfsetstrokeopacity{0.679262}%
\pgfsetdash{}{0pt}%
\pgfpathmoveto{\pgfqpoint{1.441101in}{2.441049in}}%
\pgfpathcurveto{\pgfqpoint{1.449337in}{2.441049in}}{\pgfqpoint{1.457238in}{2.444321in}}{\pgfqpoint{1.463061in}{2.450145in}}%
\pgfpathcurveto{\pgfqpoint{1.468885in}{2.455969in}}{\pgfqpoint{1.472158in}{2.463869in}}{\pgfqpoint{1.472158in}{2.472105in}}%
\pgfpathcurveto{\pgfqpoint{1.472158in}{2.480342in}}{\pgfqpoint{1.468885in}{2.488242in}}{\pgfqpoint{1.463061in}{2.494066in}}%
\pgfpathcurveto{\pgfqpoint{1.457238in}{2.499889in}}{\pgfqpoint{1.449337in}{2.503162in}}{\pgfqpoint{1.441101in}{2.503162in}}%
\pgfpathcurveto{\pgfqpoint{1.432865in}{2.503162in}}{\pgfqpoint{1.424965in}{2.499889in}}{\pgfqpoint{1.419141in}{2.494066in}}%
\pgfpathcurveto{\pgfqpoint{1.413317in}{2.488242in}}{\pgfqpoint{1.410045in}{2.480342in}}{\pgfqpoint{1.410045in}{2.472105in}}%
\pgfpathcurveto{\pgfqpoint{1.410045in}{2.463869in}}{\pgfqpoint{1.413317in}{2.455969in}}{\pgfqpoint{1.419141in}{2.450145in}}%
\pgfpathcurveto{\pgfqpoint{1.424965in}{2.444321in}}{\pgfqpoint{1.432865in}{2.441049in}}{\pgfqpoint{1.441101in}{2.441049in}}%
\pgfpathclose%
\pgfusepath{stroke,fill}%
\end{pgfscope}%
\begin{pgfscope}%
\pgfpathrectangle{\pgfqpoint{0.100000in}{0.212622in}}{\pgfqpoint{3.696000in}{3.696000in}}%
\pgfusepath{clip}%
\pgfsetbuttcap%
\pgfsetroundjoin%
\definecolor{currentfill}{rgb}{0.121569,0.466667,0.705882}%
\pgfsetfillcolor{currentfill}%
\pgfsetfillopacity{0.679885}%
\pgfsetlinewidth{1.003750pt}%
\definecolor{currentstroke}{rgb}{0.121569,0.466667,0.705882}%
\pgfsetstrokecolor{currentstroke}%
\pgfsetstrokeopacity{0.679885}%
\pgfsetdash{}{0pt}%
\pgfpathmoveto{\pgfqpoint{1.438924in}{2.435831in}}%
\pgfpathcurveto{\pgfqpoint{1.447161in}{2.435831in}}{\pgfqpoint{1.455061in}{2.439103in}}{\pgfqpoint{1.460885in}{2.444927in}}%
\pgfpathcurveto{\pgfqpoint{1.466709in}{2.450751in}}{\pgfqpoint{1.469981in}{2.458651in}}{\pgfqpoint{1.469981in}{2.466888in}}%
\pgfpathcurveto{\pgfqpoint{1.469981in}{2.475124in}}{\pgfqpoint{1.466709in}{2.483024in}}{\pgfqpoint{1.460885in}{2.488848in}}%
\pgfpathcurveto{\pgfqpoint{1.455061in}{2.494672in}}{\pgfqpoint{1.447161in}{2.497944in}}{\pgfqpoint{1.438924in}{2.497944in}}%
\pgfpathcurveto{\pgfqpoint{1.430688in}{2.497944in}}{\pgfqpoint{1.422788in}{2.494672in}}{\pgfqpoint{1.416964in}{2.488848in}}%
\pgfpathcurveto{\pgfqpoint{1.411140in}{2.483024in}}{\pgfqpoint{1.407868in}{2.475124in}}{\pgfqpoint{1.407868in}{2.466888in}}%
\pgfpathcurveto{\pgfqpoint{1.407868in}{2.458651in}}{\pgfqpoint{1.411140in}{2.450751in}}{\pgfqpoint{1.416964in}{2.444927in}}%
\pgfpathcurveto{\pgfqpoint{1.422788in}{2.439103in}}{\pgfqpoint{1.430688in}{2.435831in}}{\pgfqpoint{1.438924in}{2.435831in}}%
\pgfpathclose%
\pgfusepath{stroke,fill}%
\end{pgfscope}%
\begin{pgfscope}%
\pgfpathrectangle{\pgfqpoint{0.100000in}{0.212622in}}{\pgfqpoint{3.696000in}{3.696000in}}%
\pgfusepath{clip}%
\pgfsetbuttcap%
\pgfsetroundjoin%
\definecolor{currentfill}{rgb}{0.121569,0.466667,0.705882}%
\pgfsetfillcolor{currentfill}%
\pgfsetfillopacity{0.680194}%
\pgfsetlinewidth{1.003750pt}%
\definecolor{currentstroke}{rgb}{0.121569,0.466667,0.705882}%
\pgfsetstrokecolor{currentstroke}%
\pgfsetstrokeopacity{0.680194}%
\pgfsetdash{}{0pt}%
\pgfpathmoveto{\pgfqpoint{1.437814in}{2.432730in}}%
\pgfpathcurveto{\pgfqpoint{1.446050in}{2.432730in}}{\pgfqpoint{1.453950in}{2.436002in}}{\pgfqpoint{1.459774in}{2.441826in}}%
\pgfpathcurveto{\pgfqpoint{1.465598in}{2.447650in}}{\pgfqpoint{1.468870in}{2.455550in}}{\pgfqpoint{1.468870in}{2.463787in}}%
\pgfpathcurveto{\pgfqpoint{1.468870in}{2.472023in}}{\pgfqpoint{1.465598in}{2.479923in}}{\pgfqpoint{1.459774in}{2.485747in}}%
\pgfpathcurveto{\pgfqpoint{1.453950in}{2.491571in}}{\pgfqpoint{1.446050in}{2.494843in}}{\pgfqpoint{1.437814in}{2.494843in}}%
\pgfpathcurveto{\pgfqpoint{1.429578in}{2.494843in}}{\pgfqpoint{1.421677in}{2.491571in}}{\pgfqpoint{1.415854in}{2.485747in}}%
\pgfpathcurveto{\pgfqpoint{1.410030in}{2.479923in}}{\pgfqpoint{1.406757in}{2.472023in}}{\pgfqpoint{1.406757in}{2.463787in}}%
\pgfpathcurveto{\pgfqpoint{1.406757in}{2.455550in}}{\pgfqpoint{1.410030in}{2.447650in}}{\pgfqpoint{1.415854in}{2.441826in}}%
\pgfpathcurveto{\pgfqpoint{1.421677in}{2.436002in}}{\pgfqpoint{1.429578in}{2.432730in}}{\pgfqpoint{1.437814in}{2.432730in}}%
\pgfpathclose%
\pgfusepath{stroke,fill}%
\end{pgfscope}%
\begin{pgfscope}%
\pgfpathrectangle{\pgfqpoint{0.100000in}{0.212622in}}{\pgfqpoint{3.696000in}{3.696000in}}%
\pgfusepath{clip}%
\pgfsetbuttcap%
\pgfsetroundjoin%
\definecolor{currentfill}{rgb}{0.121569,0.466667,0.705882}%
\pgfsetfillcolor{currentfill}%
\pgfsetfillopacity{0.680561}%
\pgfsetlinewidth{1.003750pt}%
\definecolor{currentstroke}{rgb}{0.121569,0.466667,0.705882}%
\pgfsetstrokecolor{currentstroke}%
\pgfsetstrokeopacity{0.680561}%
\pgfsetdash{}{0pt}%
\pgfpathmoveto{\pgfqpoint{1.435552in}{2.428759in}}%
\pgfpathcurveto{\pgfqpoint{1.443788in}{2.428759in}}{\pgfqpoint{1.451688in}{2.432031in}}{\pgfqpoint{1.457512in}{2.437855in}}%
\pgfpathcurveto{\pgfqpoint{1.463336in}{2.443679in}}{\pgfqpoint{1.466609in}{2.451579in}}{\pgfqpoint{1.466609in}{2.459815in}}%
\pgfpathcurveto{\pgfqpoint{1.466609in}{2.468052in}}{\pgfqpoint{1.463336in}{2.475952in}}{\pgfqpoint{1.457512in}{2.481775in}}%
\pgfpathcurveto{\pgfqpoint{1.451688in}{2.487599in}}{\pgfqpoint{1.443788in}{2.490872in}}{\pgfqpoint{1.435552in}{2.490872in}}%
\pgfpathcurveto{\pgfqpoint{1.427316in}{2.490872in}}{\pgfqpoint{1.419416in}{2.487599in}}{\pgfqpoint{1.413592in}{2.481775in}}%
\pgfpathcurveto{\pgfqpoint{1.407768in}{2.475952in}}{\pgfqpoint{1.404496in}{2.468052in}}{\pgfqpoint{1.404496in}{2.459815in}}%
\pgfpathcurveto{\pgfqpoint{1.404496in}{2.451579in}}{\pgfqpoint{1.407768in}{2.443679in}}{\pgfqpoint{1.413592in}{2.437855in}}%
\pgfpathcurveto{\pgfqpoint{1.419416in}{2.432031in}}{\pgfqpoint{1.427316in}{2.428759in}}{\pgfqpoint{1.435552in}{2.428759in}}%
\pgfpathclose%
\pgfusepath{stroke,fill}%
\end{pgfscope}%
\begin{pgfscope}%
\pgfpathrectangle{\pgfqpoint{0.100000in}{0.212622in}}{\pgfqpoint{3.696000in}{3.696000in}}%
\pgfusepath{clip}%
\pgfsetbuttcap%
\pgfsetroundjoin%
\definecolor{currentfill}{rgb}{0.121569,0.466667,0.705882}%
\pgfsetfillcolor{currentfill}%
\pgfsetfillopacity{0.680926}%
\pgfsetlinewidth{1.003750pt}%
\definecolor{currentstroke}{rgb}{0.121569,0.466667,0.705882}%
\pgfsetstrokecolor{currentstroke}%
\pgfsetstrokeopacity{0.680926}%
\pgfsetdash{}{0pt}%
\pgfpathmoveto{\pgfqpoint{1.432765in}{2.424874in}}%
\pgfpathcurveto{\pgfqpoint{1.441001in}{2.424874in}}{\pgfqpoint{1.448901in}{2.428147in}}{\pgfqpoint{1.454725in}{2.433971in}}%
\pgfpathcurveto{\pgfqpoint{1.460549in}{2.439795in}}{\pgfqpoint{1.463822in}{2.447695in}}{\pgfqpoint{1.463822in}{2.455931in}}%
\pgfpathcurveto{\pgfqpoint{1.463822in}{2.464167in}}{\pgfqpoint{1.460549in}{2.472067in}}{\pgfqpoint{1.454725in}{2.477891in}}%
\pgfpathcurveto{\pgfqpoint{1.448901in}{2.483715in}}{\pgfqpoint{1.441001in}{2.486987in}}{\pgfqpoint{1.432765in}{2.486987in}}%
\pgfpathcurveto{\pgfqpoint{1.424529in}{2.486987in}}{\pgfqpoint{1.416629in}{2.483715in}}{\pgfqpoint{1.410805in}{2.477891in}}%
\pgfpathcurveto{\pgfqpoint{1.404981in}{2.472067in}}{\pgfqpoint{1.401709in}{2.464167in}}{\pgfqpoint{1.401709in}{2.455931in}}%
\pgfpathcurveto{\pgfqpoint{1.401709in}{2.447695in}}{\pgfqpoint{1.404981in}{2.439795in}}{\pgfqpoint{1.410805in}{2.433971in}}%
\pgfpathcurveto{\pgfqpoint{1.416629in}{2.428147in}}{\pgfqpoint{1.424529in}{2.424874in}}{\pgfqpoint{1.432765in}{2.424874in}}%
\pgfpathclose%
\pgfusepath{stroke,fill}%
\end{pgfscope}%
\begin{pgfscope}%
\pgfpathrectangle{\pgfqpoint{0.100000in}{0.212622in}}{\pgfqpoint{3.696000in}{3.696000in}}%
\pgfusepath{clip}%
\pgfsetbuttcap%
\pgfsetroundjoin%
\definecolor{currentfill}{rgb}{0.121569,0.466667,0.705882}%
\pgfsetfillcolor{currentfill}%
\pgfsetfillopacity{0.681076}%
\pgfsetlinewidth{1.003750pt}%
\definecolor{currentstroke}{rgb}{0.121569,0.466667,0.705882}%
\pgfsetstrokecolor{currentstroke}%
\pgfsetstrokeopacity{0.681076}%
\pgfsetdash{}{0pt}%
\pgfpathmoveto{\pgfqpoint{3.256620in}{2.617818in}}%
\pgfpathcurveto{\pgfqpoint{3.264856in}{2.617818in}}{\pgfqpoint{3.272756in}{2.621090in}}{\pgfqpoint{3.278580in}{2.626914in}}%
\pgfpathcurveto{\pgfqpoint{3.284404in}{2.632738in}}{\pgfqpoint{3.287677in}{2.640638in}}{\pgfqpoint{3.287677in}{2.648875in}}%
\pgfpathcurveto{\pgfqpoint{3.287677in}{2.657111in}}{\pgfqpoint{3.284404in}{2.665011in}}{\pgfqpoint{3.278580in}{2.670835in}}%
\pgfpathcurveto{\pgfqpoint{3.272756in}{2.676659in}}{\pgfqpoint{3.264856in}{2.679931in}}{\pgfqpoint{3.256620in}{2.679931in}}%
\pgfpathcurveto{\pgfqpoint{3.248384in}{2.679931in}}{\pgfqpoint{3.240484in}{2.676659in}}{\pgfqpoint{3.234660in}{2.670835in}}%
\pgfpathcurveto{\pgfqpoint{3.228836in}{2.665011in}}{\pgfqpoint{3.225564in}{2.657111in}}{\pgfqpoint{3.225564in}{2.648875in}}%
\pgfpathcurveto{\pgfqpoint{3.225564in}{2.640638in}}{\pgfqpoint{3.228836in}{2.632738in}}{\pgfqpoint{3.234660in}{2.626914in}}%
\pgfpathcurveto{\pgfqpoint{3.240484in}{2.621090in}}{\pgfqpoint{3.248384in}{2.617818in}}{\pgfqpoint{3.256620in}{2.617818in}}%
\pgfpathclose%
\pgfusepath{stroke,fill}%
\end{pgfscope}%
\begin{pgfscope}%
\pgfpathrectangle{\pgfqpoint{0.100000in}{0.212622in}}{\pgfqpoint{3.696000in}{3.696000in}}%
\pgfusepath{clip}%
\pgfsetbuttcap%
\pgfsetroundjoin%
\definecolor{currentfill}{rgb}{0.121569,0.466667,0.705882}%
\pgfsetfillcolor{currentfill}%
\pgfsetfillopacity{0.681117}%
\pgfsetlinewidth{1.003750pt}%
\definecolor{currentstroke}{rgb}{0.121569,0.466667,0.705882}%
\pgfsetstrokecolor{currentstroke}%
\pgfsetstrokeopacity{0.681117}%
\pgfsetdash{}{0pt}%
\pgfpathmoveto{\pgfqpoint{2.106665in}{3.015198in}}%
\pgfpathcurveto{\pgfqpoint{2.114901in}{3.015198in}}{\pgfqpoint{2.122801in}{3.018470in}}{\pgfqpoint{2.128625in}{3.024294in}}%
\pgfpathcurveto{\pgfqpoint{2.134449in}{3.030118in}}{\pgfqpoint{2.137721in}{3.038018in}}{\pgfqpoint{2.137721in}{3.046255in}}%
\pgfpathcurveto{\pgfqpoint{2.137721in}{3.054491in}}{\pgfqpoint{2.134449in}{3.062391in}}{\pgfqpoint{2.128625in}{3.068215in}}%
\pgfpathcurveto{\pgfqpoint{2.122801in}{3.074039in}}{\pgfqpoint{2.114901in}{3.077311in}}{\pgfqpoint{2.106665in}{3.077311in}}%
\pgfpathcurveto{\pgfqpoint{2.098428in}{3.077311in}}{\pgfqpoint{2.090528in}{3.074039in}}{\pgfqpoint{2.084704in}{3.068215in}}%
\pgfpathcurveto{\pgfqpoint{2.078880in}{3.062391in}}{\pgfqpoint{2.075608in}{3.054491in}}{\pgfqpoint{2.075608in}{3.046255in}}%
\pgfpathcurveto{\pgfqpoint{2.075608in}{3.038018in}}{\pgfqpoint{2.078880in}{3.030118in}}{\pgfqpoint{2.084704in}{3.024294in}}%
\pgfpathcurveto{\pgfqpoint{2.090528in}{3.018470in}}{\pgfqpoint{2.098428in}{3.015198in}}{\pgfqpoint{2.106665in}{3.015198in}}%
\pgfpathclose%
\pgfusepath{stroke,fill}%
\end{pgfscope}%
\begin{pgfscope}%
\pgfpathrectangle{\pgfqpoint{0.100000in}{0.212622in}}{\pgfqpoint{3.696000in}{3.696000in}}%
\pgfusepath{clip}%
\pgfsetbuttcap%
\pgfsetroundjoin%
\definecolor{currentfill}{rgb}{0.121569,0.466667,0.705882}%
\pgfsetfillcolor{currentfill}%
\pgfsetfillopacity{0.681238}%
\pgfsetlinewidth{1.003750pt}%
\definecolor{currentstroke}{rgb}{0.121569,0.466667,0.705882}%
\pgfsetstrokecolor{currentstroke}%
\pgfsetstrokeopacity{0.681238}%
\pgfsetdash{}{0pt}%
\pgfpathmoveto{\pgfqpoint{1.431345in}{2.422934in}}%
\pgfpathcurveto{\pgfqpoint{1.439581in}{2.422934in}}{\pgfqpoint{1.447482in}{2.426206in}}{\pgfqpoint{1.453305in}{2.432030in}}%
\pgfpathcurveto{\pgfqpoint{1.459129in}{2.437854in}}{\pgfqpoint{1.462402in}{2.445754in}}{\pgfqpoint{1.462402in}{2.453990in}}%
\pgfpathcurveto{\pgfqpoint{1.462402in}{2.462227in}}{\pgfqpoint{1.459129in}{2.470127in}}{\pgfqpoint{1.453305in}{2.475951in}}%
\pgfpathcurveto{\pgfqpoint{1.447482in}{2.481775in}}{\pgfqpoint{1.439581in}{2.485047in}}{\pgfqpoint{1.431345in}{2.485047in}}%
\pgfpathcurveto{\pgfqpoint{1.423109in}{2.485047in}}{\pgfqpoint{1.415209in}{2.481775in}}{\pgfqpoint{1.409385in}{2.475951in}}%
\pgfpathcurveto{\pgfqpoint{1.403561in}{2.470127in}}{\pgfqpoint{1.400289in}{2.462227in}}{\pgfqpoint{1.400289in}{2.453990in}}%
\pgfpathcurveto{\pgfqpoint{1.400289in}{2.445754in}}{\pgfqpoint{1.403561in}{2.437854in}}{\pgfqpoint{1.409385in}{2.432030in}}%
\pgfpathcurveto{\pgfqpoint{1.415209in}{2.426206in}}{\pgfqpoint{1.423109in}{2.422934in}}{\pgfqpoint{1.431345in}{2.422934in}}%
\pgfpathclose%
\pgfusepath{stroke,fill}%
\end{pgfscope}%
\begin{pgfscope}%
\pgfpathrectangle{\pgfqpoint{0.100000in}{0.212622in}}{\pgfqpoint{3.696000in}{3.696000in}}%
\pgfusepath{clip}%
\pgfsetbuttcap%
\pgfsetroundjoin%
\definecolor{currentfill}{rgb}{0.121569,0.466667,0.705882}%
\pgfsetfillcolor{currentfill}%
\pgfsetfillopacity{0.681735}%
\pgfsetlinewidth{1.003750pt}%
\definecolor{currentstroke}{rgb}{0.121569,0.466667,0.705882}%
\pgfsetstrokecolor{currentstroke}%
\pgfsetstrokeopacity{0.681735}%
\pgfsetdash{}{0pt}%
\pgfpathmoveto{\pgfqpoint{1.429555in}{2.418452in}}%
\pgfpathcurveto{\pgfqpoint{1.437791in}{2.418452in}}{\pgfqpoint{1.445691in}{2.421725in}}{\pgfqpoint{1.451515in}{2.427548in}}%
\pgfpathcurveto{\pgfqpoint{1.457339in}{2.433372in}}{\pgfqpoint{1.460612in}{2.441272in}}{\pgfqpoint{1.460612in}{2.449509in}}%
\pgfpathcurveto{\pgfqpoint{1.460612in}{2.457745in}}{\pgfqpoint{1.457339in}{2.465645in}}{\pgfqpoint{1.451515in}{2.471469in}}%
\pgfpathcurveto{\pgfqpoint{1.445691in}{2.477293in}}{\pgfqpoint{1.437791in}{2.480565in}}{\pgfqpoint{1.429555in}{2.480565in}}%
\pgfpathcurveto{\pgfqpoint{1.421319in}{2.480565in}}{\pgfqpoint{1.413419in}{2.477293in}}{\pgfqpoint{1.407595in}{2.471469in}}%
\pgfpathcurveto{\pgfqpoint{1.401771in}{2.465645in}}{\pgfqpoint{1.398499in}{2.457745in}}{\pgfqpoint{1.398499in}{2.449509in}}%
\pgfpathcurveto{\pgfqpoint{1.398499in}{2.441272in}}{\pgfqpoint{1.401771in}{2.433372in}}{\pgfqpoint{1.407595in}{2.427548in}}%
\pgfpathcurveto{\pgfqpoint{1.413419in}{2.421725in}}{\pgfqpoint{1.421319in}{2.418452in}}{\pgfqpoint{1.429555in}{2.418452in}}%
\pgfpathclose%
\pgfusepath{stroke,fill}%
\end{pgfscope}%
\begin{pgfscope}%
\pgfpathrectangle{\pgfqpoint{0.100000in}{0.212622in}}{\pgfqpoint{3.696000in}{3.696000in}}%
\pgfusepath{clip}%
\pgfsetbuttcap%
\pgfsetroundjoin%
\definecolor{currentfill}{rgb}{0.121569,0.466667,0.705882}%
\pgfsetfillcolor{currentfill}%
\pgfsetfillopacity{0.681994}%
\pgfsetlinewidth{1.003750pt}%
\definecolor{currentstroke}{rgb}{0.121569,0.466667,0.705882}%
\pgfsetstrokecolor{currentstroke}%
\pgfsetstrokeopacity{0.681994}%
\pgfsetdash{}{0pt}%
\pgfpathmoveto{\pgfqpoint{1.428627in}{2.415874in}}%
\pgfpathcurveto{\pgfqpoint{1.436864in}{2.415874in}}{\pgfqpoint{1.444764in}{2.419146in}}{\pgfqpoint{1.450588in}{2.424970in}}%
\pgfpathcurveto{\pgfqpoint{1.456411in}{2.430794in}}{\pgfqpoint{1.459684in}{2.438694in}}{\pgfqpoint{1.459684in}{2.446930in}}%
\pgfpathcurveto{\pgfqpoint{1.459684in}{2.455167in}}{\pgfqpoint{1.456411in}{2.463067in}}{\pgfqpoint{1.450588in}{2.468891in}}%
\pgfpathcurveto{\pgfqpoint{1.444764in}{2.474715in}}{\pgfqpoint{1.436864in}{2.477987in}}{\pgfqpoint{1.428627in}{2.477987in}}%
\pgfpathcurveto{\pgfqpoint{1.420391in}{2.477987in}}{\pgfqpoint{1.412491in}{2.474715in}}{\pgfqpoint{1.406667in}{2.468891in}}%
\pgfpathcurveto{\pgfqpoint{1.400843in}{2.463067in}}{\pgfqpoint{1.397571in}{2.455167in}}{\pgfqpoint{1.397571in}{2.446930in}}%
\pgfpathcurveto{\pgfqpoint{1.397571in}{2.438694in}}{\pgfqpoint{1.400843in}{2.430794in}}{\pgfqpoint{1.406667in}{2.424970in}}%
\pgfpathcurveto{\pgfqpoint{1.412491in}{2.419146in}}{\pgfqpoint{1.420391in}{2.415874in}}{\pgfqpoint{1.428627in}{2.415874in}}%
\pgfpathclose%
\pgfusepath{stroke,fill}%
\end{pgfscope}%
\begin{pgfscope}%
\pgfpathrectangle{\pgfqpoint{0.100000in}{0.212622in}}{\pgfqpoint{3.696000in}{3.696000in}}%
\pgfusepath{clip}%
\pgfsetbuttcap%
\pgfsetroundjoin%
\definecolor{currentfill}{rgb}{0.121569,0.466667,0.705882}%
\pgfsetfillcolor{currentfill}%
\pgfsetfillopacity{0.682387}%
\pgfsetlinewidth{1.003750pt}%
\definecolor{currentstroke}{rgb}{0.121569,0.466667,0.705882}%
\pgfsetstrokecolor{currentstroke}%
\pgfsetstrokeopacity{0.682387}%
\pgfsetdash{}{0pt}%
\pgfpathmoveto{\pgfqpoint{1.426011in}{2.411787in}}%
\pgfpathcurveto{\pgfqpoint{1.434247in}{2.411787in}}{\pgfqpoint{1.442147in}{2.415059in}}{\pgfqpoint{1.447971in}{2.420883in}}%
\pgfpathcurveto{\pgfqpoint{1.453795in}{2.426707in}}{\pgfqpoint{1.457067in}{2.434607in}}{\pgfqpoint{1.457067in}{2.442843in}}%
\pgfpathcurveto{\pgfqpoint{1.457067in}{2.451080in}}{\pgfqpoint{1.453795in}{2.458980in}}{\pgfqpoint{1.447971in}{2.464804in}}%
\pgfpathcurveto{\pgfqpoint{1.442147in}{2.470628in}}{\pgfqpoint{1.434247in}{2.473900in}}{\pgfqpoint{1.426011in}{2.473900in}}%
\pgfpathcurveto{\pgfqpoint{1.417774in}{2.473900in}}{\pgfqpoint{1.409874in}{2.470628in}}{\pgfqpoint{1.404050in}{2.464804in}}%
\pgfpathcurveto{\pgfqpoint{1.398226in}{2.458980in}}{\pgfqpoint{1.394954in}{2.451080in}}{\pgfqpoint{1.394954in}{2.442843in}}%
\pgfpathcurveto{\pgfqpoint{1.394954in}{2.434607in}}{\pgfqpoint{1.398226in}{2.426707in}}{\pgfqpoint{1.404050in}{2.420883in}}%
\pgfpathcurveto{\pgfqpoint{1.409874in}{2.415059in}}{\pgfqpoint{1.417774in}{2.411787in}}{\pgfqpoint{1.426011in}{2.411787in}}%
\pgfpathclose%
\pgfusepath{stroke,fill}%
\end{pgfscope}%
\begin{pgfscope}%
\pgfpathrectangle{\pgfqpoint{0.100000in}{0.212622in}}{\pgfqpoint{3.696000in}{3.696000in}}%
\pgfusepath{clip}%
\pgfsetbuttcap%
\pgfsetroundjoin%
\definecolor{currentfill}{rgb}{0.121569,0.466667,0.705882}%
\pgfsetfillcolor{currentfill}%
\pgfsetfillopacity{0.682437}%
\pgfsetlinewidth{1.003750pt}%
\definecolor{currentstroke}{rgb}{0.121569,0.466667,0.705882}%
\pgfsetstrokecolor{currentstroke}%
\pgfsetstrokeopacity{0.682437}%
\pgfsetdash{}{0pt}%
\pgfpathmoveto{\pgfqpoint{3.262923in}{2.615914in}}%
\pgfpathcurveto{\pgfqpoint{3.271160in}{2.615914in}}{\pgfqpoint{3.279060in}{2.619187in}}{\pgfqpoint{3.284884in}{2.625010in}}%
\pgfpathcurveto{\pgfqpoint{3.290708in}{2.630834in}}{\pgfqpoint{3.293980in}{2.638734in}}{\pgfqpoint{3.293980in}{2.646971in}}%
\pgfpathcurveto{\pgfqpoint{3.293980in}{2.655207in}}{\pgfqpoint{3.290708in}{2.663107in}}{\pgfqpoint{3.284884in}{2.668931in}}%
\pgfpathcurveto{\pgfqpoint{3.279060in}{2.674755in}}{\pgfqpoint{3.271160in}{2.678027in}}{\pgfqpoint{3.262923in}{2.678027in}}%
\pgfpathcurveto{\pgfqpoint{3.254687in}{2.678027in}}{\pgfqpoint{3.246787in}{2.674755in}}{\pgfqpoint{3.240963in}{2.668931in}}%
\pgfpathcurveto{\pgfqpoint{3.235139in}{2.663107in}}{\pgfqpoint{3.231867in}{2.655207in}}{\pgfqpoint{3.231867in}{2.646971in}}%
\pgfpathcurveto{\pgfqpoint{3.231867in}{2.638734in}}{\pgfqpoint{3.235139in}{2.630834in}}{\pgfqpoint{3.240963in}{2.625010in}}%
\pgfpathcurveto{\pgfqpoint{3.246787in}{2.619187in}}{\pgfqpoint{3.254687in}{2.615914in}}{\pgfqpoint{3.262923in}{2.615914in}}%
\pgfpathclose%
\pgfusepath{stroke,fill}%
\end{pgfscope}%
\begin{pgfscope}%
\pgfpathrectangle{\pgfqpoint{0.100000in}{0.212622in}}{\pgfqpoint{3.696000in}{3.696000in}}%
\pgfusepath{clip}%
\pgfsetbuttcap%
\pgfsetroundjoin%
\definecolor{currentfill}{rgb}{0.121569,0.466667,0.705882}%
\pgfsetfillcolor{currentfill}%
\pgfsetfillopacity{0.682623}%
\pgfsetlinewidth{1.003750pt}%
\definecolor{currentstroke}{rgb}{0.121569,0.466667,0.705882}%
\pgfsetstrokecolor{currentstroke}%
\pgfsetstrokeopacity{0.682623}%
\pgfsetdash{}{0pt}%
\pgfpathmoveto{\pgfqpoint{1.424543in}{2.409702in}}%
\pgfpathcurveto{\pgfqpoint{1.432779in}{2.409702in}}{\pgfqpoint{1.440679in}{2.412974in}}{\pgfqpoint{1.446503in}{2.418798in}}%
\pgfpathcurveto{\pgfqpoint{1.452327in}{2.424622in}}{\pgfqpoint{1.455599in}{2.432522in}}{\pgfqpoint{1.455599in}{2.440758in}}%
\pgfpathcurveto{\pgfqpoint{1.455599in}{2.448994in}}{\pgfqpoint{1.452327in}{2.456895in}}{\pgfqpoint{1.446503in}{2.462718in}}%
\pgfpathcurveto{\pgfqpoint{1.440679in}{2.468542in}}{\pgfqpoint{1.432779in}{2.471815in}}{\pgfqpoint{1.424543in}{2.471815in}}%
\pgfpathcurveto{\pgfqpoint{1.416306in}{2.471815in}}{\pgfqpoint{1.408406in}{2.468542in}}{\pgfqpoint{1.402582in}{2.462718in}}%
\pgfpathcurveto{\pgfqpoint{1.396758in}{2.456895in}}{\pgfqpoint{1.393486in}{2.448994in}}{\pgfqpoint{1.393486in}{2.440758in}}%
\pgfpathcurveto{\pgfqpoint{1.393486in}{2.432522in}}{\pgfqpoint{1.396758in}{2.424622in}}{\pgfqpoint{1.402582in}{2.418798in}}%
\pgfpathcurveto{\pgfqpoint{1.408406in}{2.412974in}}{\pgfqpoint{1.416306in}{2.409702in}}{\pgfqpoint{1.424543in}{2.409702in}}%
\pgfpathclose%
\pgfusepath{stroke,fill}%
\end{pgfscope}%
\begin{pgfscope}%
\pgfpathrectangle{\pgfqpoint{0.100000in}{0.212622in}}{\pgfqpoint{3.696000in}{3.696000in}}%
\pgfusepath{clip}%
\pgfsetbuttcap%
\pgfsetroundjoin%
\definecolor{currentfill}{rgb}{0.121569,0.466667,0.705882}%
\pgfsetfillcolor{currentfill}%
\pgfsetfillopacity{0.683093}%
\pgfsetlinewidth{1.003750pt}%
\definecolor{currentstroke}{rgb}{0.121569,0.466667,0.705882}%
\pgfsetstrokecolor{currentstroke}%
\pgfsetstrokeopacity{0.683093}%
\pgfsetdash{}{0pt}%
\pgfpathmoveto{\pgfqpoint{1.422449in}{2.406502in}}%
\pgfpathcurveto{\pgfqpoint{1.430685in}{2.406502in}}{\pgfqpoint{1.438585in}{2.409774in}}{\pgfqpoint{1.444409in}{2.415598in}}%
\pgfpathcurveto{\pgfqpoint{1.450233in}{2.421422in}}{\pgfqpoint{1.453505in}{2.429322in}}{\pgfqpoint{1.453505in}{2.437559in}}%
\pgfpathcurveto{\pgfqpoint{1.453505in}{2.445795in}}{\pgfqpoint{1.450233in}{2.453695in}}{\pgfqpoint{1.444409in}{2.459519in}}%
\pgfpathcurveto{\pgfqpoint{1.438585in}{2.465343in}}{\pgfqpoint{1.430685in}{2.468615in}}{\pgfqpoint{1.422449in}{2.468615in}}%
\pgfpathcurveto{\pgfqpoint{1.414212in}{2.468615in}}{\pgfqpoint{1.406312in}{2.465343in}}{\pgfqpoint{1.400488in}{2.459519in}}%
\pgfpathcurveto{\pgfqpoint{1.394664in}{2.453695in}}{\pgfqpoint{1.391392in}{2.445795in}}{\pgfqpoint{1.391392in}{2.437559in}}%
\pgfpathcurveto{\pgfqpoint{1.391392in}{2.429322in}}{\pgfqpoint{1.394664in}{2.421422in}}{\pgfqpoint{1.400488in}{2.415598in}}%
\pgfpathcurveto{\pgfqpoint{1.406312in}{2.409774in}}{\pgfqpoint{1.414212in}{2.406502in}}{\pgfqpoint{1.422449in}{2.406502in}}%
\pgfpathclose%
\pgfusepath{stroke,fill}%
\end{pgfscope}%
\begin{pgfscope}%
\pgfpathrectangle{\pgfqpoint{0.100000in}{0.212622in}}{\pgfqpoint{3.696000in}{3.696000in}}%
\pgfusepath{clip}%
\pgfsetbuttcap%
\pgfsetroundjoin%
\definecolor{currentfill}{rgb}{0.121569,0.466667,0.705882}%
\pgfsetfillcolor{currentfill}%
\pgfsetfillopacity{0.683255}%
\pgfsetlinewidth{1.003750pt}%
\definecolor{currentstroke}{rgb}{0.121569,0.466667,0.705882}%
\pgfsetstrokecolor{currentstroke}%
\pgfsetstrokeopacity{0.683255}%
\pgfsetdash{}{0pt}%
\pgfpathmoveto{\pgfqpoint{2.114798in}{3.015490in}}%
\pgfpathcurveto{\pgfqpoint{2.123034in}{3.015490in}}{\pgfqpoint{2.130934in}{3.018762in}}{\pgfqpoint{2.136758in}{3.024586in}}%
\pgfpathcurveto{\pgfqpoint{2.142582in}{3.030410in}}{\pgfqpoint{2.145854in}{3.038310in}}{\pgfqpoint{2.145854in}{3.046547in}}%
\pgfpathcurveto{\pgfqpoint{2.145854in}{3.054783in}}{\pgfqpoint{2.142582in}{3.062683in}}{\pgfqpoint{2.136758in}{3.068507in}}%
\pgfpathcurveto{\pgfqpoint{2.130934in}{3.074331in}}{\pgfqpoint{2.123034in}{3.077603in}}{\pgfqpoint{2.114798in}{3.077603in}}%
\pgfpathcurveto{\pgfqpoint{2.106561in}{3.077603in}}{\pgfqpoint{2.098661in}{3.074331in}}{\pgfqpoint{2.092837in}{3.068507in}}%
\pgfpathcurveto{\pgfqpoint{2.087013in}{3.062683in}}{\pgfqpoint{2.083741in}{3.054783in}}{\pgfqpoint{2.083741in}{3.046547in}}%
\pgfpathcurveto{\pgfqpoint{2.083741in}{3.038310in}}{\pgfqpoint{2.087013in}{3.030410in}}{\pgfqpoint{2.092837in}{3.024586in}}%
\pgfpathcurveto{\pgfqpoint{2.098661in}{3.018762in}}{\pgfqpoint{2.106561in}{3.015490in}}{\pgfqpoint{2.114798in}{3.015490in}}%
\pgfpathclose%
\pgfusepath{stroke,fill}%
\end{pgfscope}%
\begin{pgfscope}%
\pgfpathrectangle{\pgfqpoint{0.100000in}{0.212622in}}{\pgfqpoint{3.696000in}{3.696000in}}%
\pgfusepath{clip}%
\pgfsetbuttcap%
\pgfsetroundjoin%
\definecolor{currentfill}{rgb}{0.121569,0.466667,0.705882}%
\pgfsetfillcolor{currentfill}%
\pgfsetfillopacity{0.683674}%
\pgfsetlinewidth{1.003750pt}%
\definecolor{currentstroke}{rgb}{0.121569,0.466667,0.705882}%
\pgfsetstrokecolor{currentstroke}%
\pgfsetstrokeopacity{0.683674}%
\pgfsetdash{}{0pt}%
\pgfpathmoveto{\pgfqpoint{1.420575in}{2.400898in}}%
\pgfpathcurveto{\pgfqpoint{1.428811in}{2.400898in}}{\pgfqpoint{1.436711in}{2.404170in}}{\pgfqpoint{1.442535in}{2.409994in}}%
\pgfpathcurveto{\pgfqpoint{1.448359in}{2.415818in}}{\pgfqpoint{1.451632in}{2.423718in}}{\pgfqpoint{1.451632in}{2.431954in}}%
\pgfpathcurveto{\pgfqpoint{1.451632in}{2.440190in}}{\pgfqpoint{1.448359in}{2.448091in}}{\pgfqpoint{1.442535in}{2.453914in}}%
\pgfpathcurveto{\pgfqpoint{1.436711in}{2.459738in}}{\pgfqpoint{1.428811in}{2.463011in}}{\pgfqpoint{1.420575in}{2.463011in}}%
\pgfpathcurveto{\pgfqpoint{1.412339in}{2.463011in}}{\pgfqpoint{1.404439in}{2.459738in}}{\pgfqpoint{1.398615in}{2.453914in}}%
\pgfpathcurveto{\pgfqpoint{1.392791in}{2.448091in}}{\pgfqpoint{1.389519in}{2.440190in}}{\pgfqpoint{1.389519in}{2.431954in}}%
\pgfpathcurveto{\pgfqpoint{1.389519in}{2.423718in}}{\pgfqpoint{1.392791in}{2.415818in}}{\pgfqpoint{1.398615in}{2.409994in}}%
\pgfpathcurveto{\pgfqpoint{1.404439in}{2.404170in}}{\pgfqpoint{1.412339in}{2.400898in}}{\pgfqpoint{1.420575in}{2.400898in}}%
\pgfpathclose%
\pgfusepath{stroke,fill}%
\end{pgfscope}%
\begin{pgfscope}%
\pgfpathrectangle{\pgfqpoint{0.100000in}{0.212622in}}{\pgfqpoint{3.696000in}{3.696000in}}%
\pgfusepath{clip}%
\pgfsetbuttcap%
\pgfsetroundjoin%
\definecolor{currentfill}{rgb}{0.121569,0.466667,0.705882}%
\pgfsetfillcolor{currentfill}%
\pgfsetfillopacity{0.683917}%
\pgfsetlinewidth{1.003750pt}%
\definecolor{currentstroke}{rgb}{0.121569,0.466667,0.705882}%
\pgfsetstrokecolor{currentstroke}%
\pgfsetstrokeopacity{0.683917}%
\pgfsetdash{}{0pt}%
\pgfpathmoveto{\pgfqpoint{3.268845in}{2.614354in}}%
\pgfpathcurveto{\pgfqpoint{3.277081in}{2.614354in}}{\pgfqpoint{3.284981in}{2.617626in}}{\pgfqpoint{3.290805in}{2.623450in}}%
\pgfpathcurveto{\pgfqpoint{3.296629in}{2.629274in}}{\pgfqpoint{3.299901in}{2.637174in}}{\pgfqpoint{3.299901in}{2.645411in}}%
\pgfpathcurveto{\pgfqpoint{3.299901in}{2.653647in}}{\pgfqpoint{3.296629in}{2.661547in}}{\pgfqpoint{3.290805in}{2.667371in}}%
\pgfpathcurveto{\pgfqpoint{3.284981in}{2.673195in}}{\pgfqpoint{3.277081in}{2.676467in}}{\pgfqpoint{3.268845in}{2.676467in}}%
\pgfpathcurveto{\pgfqpoint{3.260608in}{2.676467in}}{\pgfqpoint{3.252708in}{2.673195in}}{\pgfqpoint{3.246884in}{2.667371in}}%
\pgfpathcurveto{\pgfqpoint{3.241061in}{2.661547in}}{\pgfqpoint{3.237788in}{2.653647in}}{\pgfqpoint{3.237788in}{2.645411in}}%
\pgfpathcurveto{\pgfqpoint{3.237788in}{2.637174in}}{\pgfqpoint{3.241061in}{2.629274in}}{\pgfqpoint{3.246884in}{2.623450in}}%
\pgfpathcurveto{\pgfqpoint{3.252708in}{2.617626in}}{\pgfqpoint{3.260608in}{2.614354in}}{\pgfqpoint{3.268845in}{2.614354in}}%
\pgfpathclose%
\pgfusepath{stroke,fill}%
\end{pgfscope}%
\begin{pgfscope}%
\pgfpathrectangle{\pgfqpoint{0.100000in}{0.212622in}}{\pgfqpoint{3.696000in}{3.696000in}}%
\pgfusepath{clip}%
\pgfsetbuttcap%
\pgfsetroundjoin%
\definecolor{currentfill}{rgb}{0.121569,0.466667,0.705882}%
\pgfsetfillcolor{currentfill}%
\pgfsetfillopacity{0.684006}%
\pgfsetlinewidth{1.003750pt}%
\definecolor{currentstroke}{rgb}{0.121569,0.466667,0.705882}%
\pgfsetstrokecolor{currentstroke}%
\pgfsetstrokeopacity{0.684006}%
\pgfsetdash{}{0pt}%
\pgfpathmoveto{\pgfqpoint{1.419249in}{2.398148in}}%
\pgfpathcurveto{\pgfqpoint{1.427485in}{2.398148in}}{\pgfqpoint{1.435385in}{2.401421in}}{\pgfqpoint{1.441209in}{2.407244in}}%
\pgfpathcurveto{\pgfqpoint{1.447033in}{2.413068in}}{\pgfqpoint{1.450305in}{2.420968in}}{\pgfqpoint{1.450305in}{2.429205in}}%
\pgfpathcurveto{\pgfqpoint{1.450305in}{2.437441in}}{\pgfqpoint{1.447033in}{2.445341in}}{\pgfqpoint{1.441209in}{2.451165in}}%
\pgfpathcurveto{\pgfqpoint{1.435385in}{2.456989in}}{\pgfqpoint{1.427485in}{2.460261in}}{\pgfqpoint{1.419249in}{2.460261in}}%
\pgfpathcurveto{\pgfqpoint{1.411012in}{2.460261in}}{\pgfqpoint{1.403112in}{2.456989in}}{\pgfqpoint{1.397288in}{2.451165in}}%
\pgfpathcurveto{\pgfqpoint{1.391464in}{2.445341in}}{\pgfqpoint{1.388192in}{2.437441in}}{\pgfqpoint{1.388192in}{2.429205in}}%
\pgfpathcurveto{\pgfqpoint{1.388192in}{2.420968in}}{\pgfqpoint{1.391464in}{2.413068in}}{\pgfqpoint{1.397288in}{2.407244in}}%
\pgfpathcurveto{\pgfqpoint{1.403112in}{2.401421in}}{\pgfqpoint{1.411012in}{2.398148in}}{\pgfqpoint{1.419249in}{2.398148in}}%
\pgfpathclose%
\pgfusepath{stroke,fill}%
\end{pgfscope}%
\begin{pgfscope}%
\pgfpathrectangle{\pgfqpoint{0.100000in}{0.212622in}}{\pgfqpoint{3.696000in}{3.696000in}}%
\pgfusepath{clip}%
\pgfsetbuttcap%
\pgfsetroundjoin%
\definecolor{currentfill}{rgb}{0.121569,0.466667,0.705882}%
\pgfsetfillcolor{currentfill}%
\pgfsetfillopacity{0.684333}%
\pgfsetlinewidth{1.003750pt}%
\definecolor{currentstroke}{rgb}{0.121569,0.466667,0.705882}%
\pgfsetstrokecolor{currentstroke}%
\pgfsetstrokeopacity{0.684333}%
\pgfsetdash{}{0pt}%
\pgfpathmoveto{\pgfqpoint{1.417067in}{2.395002in}}%
\pgfpathcurveto{\pgfqpoint{1.425303in}{2.395002in}}{\pgfqpoint{1.433203in}{2.398274in}}{\pgfqpoint{1.439027in}{2.404098in}}%
\pgfpathcurveto{\pgfqpoint{1.444851in}{2.409922in}}{\pgfqpoint{1.448124in}{2.417822in}}{\pgfqpoint{1.448124in}{2.426058in}}%
\pgfpathcurveto{\pgfqpoint{1.448124in}{2.434295in}}{\pgfqpoint{1.444851in}{2.442195in}}{\pgfqpoint{1.439027in}{2.448019in}}%
\pgfpathcurveto{\pgfqpoint{1.433203in}{2.453843in}}{\pgfqpoint{1.425303in}{2.457115in}}{\pgfqpoint{1.417067in}{2.457115in}}%
\pgfpathcurveto{\pgfqpoint{1.408831in}{2.457115in}}{\pgfqpoint{1.400931in}{2.453843in}}{\pgfqpoint{1.395107in}{2.448019in}}%
\pgfpathcurveto{\pgfqpoint{1.389283in}{2.442195in}}{\pgfqpoint{1.386011in}{2.434295in}}{\pgfqpoint{1.386011in}{2.426058in}}%
\pgfpathcurveto{\pgfqpoint{1.386011in}{2.417822in}}{\pgfqpoint{1.389283in}{2.409922in}}{\pgfqpoint{1.395107in}{2.404098in}}%
\pgfpathcurveto{\pgfqpoint{1.400931in}{2.398274in}}{\pgfqpoint{1.408831in}{2.395002in}}{\pgfqpoint{1.417067in}{2.395002in}}%
\pgfpathclose%
\pgfusepath{stroke,fill}%
\end{pgfscope}%
\begin{pgfscope}%
\pgfpathrectangle{\pgfqpoint{0.100000in}{0.212622in}}{\pgfqpoint{3.696000in}{3.696000in}}%
\pgfusepath{clip}%
\pgfsetbuttcap%
\pgfsetroundjoin%
\definecolor{currentfill}{rgb}{0.121569,0.466667,0.705882}%
\pgfsetfillcolor{currentfill}%
\pgfsetfillopacity{0.684513}%
\pgfsetlinewidth{1.003750pt}%
\definecolor{currentstroke}{rgb}{0.121569,0.466667,0.705882}%
\pgfsetstrokecolor{currentstroke}%
\pgfsetstrokeopacity{0.684513}%
\pgfsetdash{}{0pt}%
\pgfpathmoveto{\pgfqpoint{2.122215in}{3.013381in}}%
\pgfpathcurveto{\pgfqpoint{2.130451in}{3.013381in}}{\pgfqpoint{2.138351in}{3.016654in}}{\pgfqpoint{2.144175in}{3.022478in}}%
\pgfpathcurveto{\pgfqpoint{2.149999in}{3.028302in}}{\pgfqpoint{2.153271in}{3.036202in}}{\pgfqpoint{2.153271in}{3.044438in}}%
\pgfpathcurveto{\pgfqpoint{2.153271in}{3.052674in}}{\pgfqpoint{2.149999in}{3.060574in}}{\pgfqpoint{2.144175in}{3.066398in}}%
\pgfpathcurveto{\pgfqpoint{2.138351in}{3.072222in}}{\pgfqpoint{2.130451in}{3.075494in}}{\pgfqpoint{2.122215in}{3.075494in}}%
\pgfpathcurveto{\pgfqpoint{2.113978in}{3.075494in}}{\pgfqpoint{2.106078in}{3.072222in}}{\pgfqpoint{2.100254in}{3.066398in}}%
\pgfpathcurveto{\pgfqpoint{2.094430in}{3.060574in}}{\pgfqpoint{2.091158in}{3.052674in}}{\pgfqpoint{2.091158in}{3.044438in}}%
\pgfpathcurveto{\pgfqpoint{2.091158in}{3.036202in}}{\pgfqpoint{2.094430in}{3.028302in}}{\pgfqpoint{2.100254in}{3.022478in}}%
\pgfpathcurveto{\pgfqpoint{2.106078in}{3.016654in}}{\pgfqpoint{2.113978in}{3.013381in}}{\pgfqpoint{2.122215in}{3.013381in}}%
\pgfpathclose%
\pgfusepath{stroke,fill}%
\end{pgfscope}%
\begin{pgfscope}%
\pgfpathrectangle{\pgfqpoint{0.100000in}{0.212622in}}{\pgfqpoint{3.696000in}{3.696000in}}%
\pgfusepath{clip}%
\pgfsetbuttcap%
\pgfsetroundjoin%
\definecolor{currentfill}{rgb}{0.121569,0.466667,0.705882}%
\pgfsetfillcolor{currentfill}%
\pgfsetfillopacity{0.684762}%
\pgfsetlinewidth{1.003750pt}%
\definecolor{currentstroke}{rgb}{0.121569,0.466667,0.705882}%
\pgfsetstrokecolor{currentstroke}%
\pgfsetstrokeopacity{0.684762}%
\pgfsetdash{}{0pt}%
\pgfpathmoveto{\pgfqpoint{1.414426in}{2.391594in}}%
\pgfpathcurveto{\pgfqpoint{1.422662in}{2.391594in}}{\pgfqpoint{1.430562in}{2.394866in}}{\pgfqpoint{1.436386in}{2.400690in}}%
\pgfpathcurveto{\pgfqpoint{1.442210in}{2.406514in}}{\pgfqpoint{1.445482in}{2.414414in}}{\pgfqpoint{1.445482in}{2.422651in}}%
\pgfpathcurveto{\pgfqpoint{1.445482in}{2.430887in}}{\pgfqpoint{1.442210in}{2.438787in}}{\pgfqpoint{1.436386in}{2.444611in}}%
\pgfpathcurveto{\pgfqpoint{1.430562in}{2.450435in}}{\pgfqpoint{1.422662in}{2.453707in}}{\pgfqpoint{1.414426in}{2.453707in}}%
\pgfpathcurveto{\pgfqpoint{1.406190in}{2.453707in}}{\pgfqpoint{1.398290in}{2.450435in}}{\pgfqpoint{1.392466in}{2.444611in}}%
\pgfpathcurveto{\pgfqpoint{1.386642in}{2.438787in}}{\pgfqpoint{1.383369in}{2.430887in}}{\pgfqpoint{1.383369in}{2.422651in}}%
\pgfpathcurveto{\pgfqpoint{1.383369in}{2.414414in}}{\pgfqpoint{1.386642in}{2.406514in}}{\pgfqpoint{1.392466in}{2.400690in}}%
\pgfpathcurveto{\pgfqpoint{1.398290in}{2.394866in}}{\pgfqpoint{1.406190in}{2.391594in}}{\pgfqpoint{1.414426in}{2.391594in}}%
\pgfpathclose%
\pgfusepath{stroke,fill}%
\end{pgfscope}%
\begin{pgfscope}%
\pgfpathrectangle{\pgfqpoint{0.100000in}{0.212622in}}{\pgfqpoint{3.696000in}{3.696000in}}%
\pgfusepath{clip}%
\pgfsetbuttcap%
\pgfsetroundjoin%
\definecolor{currentfill}{rgb}{0.121569,0.466667,0.705882}%
\pgfsetfillcolor{currentfill}%
\pgfsetfillopacity{0.685055}%
\pgfsetlinewidth{1.003750pt}%
\definecolor{currentstroke}{rgb}{0.121569,0.466667,0.705882}%
\pgfsetstrokecolor{currentstroke}%
\pgfsetstrokeopacity{0.685055}%
\pgfsetdash{}{0pt}%
\pgfpathmoveto{\pgfqpoint{3.274183in}{2.614027in}}%
\pgfpathcurveto{\pgfqpoint{3.282419in}{2.614027in}}{\pgfqpoint{3.290319in}{2.617299in}}{\pgfqpoint{3.296143in}{2.623123in}}%
\pgfpathcurveto{\pgfqpoint{3.301967in}{2.628947in}}{\pgfqpoint{3.305240in}{2.636847in}}{\pgfqpoint{3.305240in}{2.645083in}}%
\pgfpathcurveto{\pgfqpoint{3.305240in}{2.653319in}}{\pgfqpoint{3.301967in}{2.661219in}}{\pgfqpoint{3.296143in}{2.667043in}}%
\pgfpathcurveto{\pgfqpoint{3.290319in}{2.672867in}}{\pgfqpoint{3.282419in}{2.676140in}}{\pgfqpoint{3.274183in}{2.676140in}}%
\pgfpathcurveto{\pgfqpoint{3.265947in}{2.676140in}}{\pgfqpoint{3.258047in}{2.672867in}}{\pgfqpoint{3.252223in}{2.667043in}}%
\pgfpathcurveto{\pgfqpoint{3.246399in}{2.661219in}}{\pgfqpoint{3.243127in}{2.653319in}}{\pgfqpoint{3.243127in}{2.645083in}}%
\pgfpathcurveto{\pgfqpoint{3.243127in}{2.636847in}}{\pgfqpoint{3.246399in}{2.628947in}}{\pgfqpoint{3.252223in}{2.623123in}}%
\pgfpathcurveto{\pgfqpoint{3.258047in}{2.617299in}}{\pgfqpoint{3.265947in}{2.614027in}}{\pgfqpoint{3.274183in}{2.614027in}}%
\pgfpathclose%
\pgfusepath{stroke,fill}%
\end{pgfscope}%
\begin{pgfscope}%
\pgfpathrectangle{\pgfqpoint{0.100000in}{0.212622in}}{\pgfqpoint{3.696000in}{3.696000in}}%
\pgfusepath{clip}%
\pgfsetbuttcap%
\pgfsetroundjoin%
\definecolor{currentfill}{rgb}{0.121569,0.466667,0.705882}%
\pgfsetfillcolor{currentfill}%
\pgfsetfillopacity{0.685500}%
\pgfsetlinewidth{1.003750pt}%
\definecolor{currentstroke}{rgb}{0.121569,0.466667,0.705882}%
\pgfsetstrokecolor{currentstroke}%
\pgfsetstrokeopacity{0.685500}%
\pgfsetdash{}{0pt}%
\pgfpathmoveto{\pgfqpoint{1.411292in}{2.385874in}}%
\pgfpathcurveto{\pgfqpoint{1.419529in}{2.385874in}}{\pgfqpoint{1.427429in}{2.389146in}}{\pgfqpoint{1.433253in}{2.394970in}}%
\pgfpathcurveto{\pgfqpoint{1.439076in}{2.400794in}}{\pgfqpoint{1.442349in}{2.408694in}}{\pgfqpoint{1.442349in}{2.416930in}}%
\pgfpathcurveto{\pgfqpoint{1.442349in}{2.425166in}}{\pgfqpoint{1.439076in}{2.433066in}}{\pgfqpoint{1.433253in}{2.438890in}}%
\pgfpathcurveto{\pgfqpoint{1.427429in}{2.444714in}}{\pgfqpoint{1.419529in}{2.447987in}}{\pgfqpoint{1.411292in}{2.447987in}}%
\pgfpathcurveto{\pgfqpoint{1.403056in}{2.447987in}}{\pgfqpoint{1.395156in}{2.444714in}}{\pgfqpoint{1.389332in}{2.438890in}}%
\pgfpathcurveto{\pgfqpoint{1.383508in}{2.433066in}}{\pgfqpoint{1.380236in}{2.425166in}}{\pgfqpoint{1.380236in}{2.416930in}}%
\pgfpathcurveto{\pgfqpoint{1.380236in}{2.408694in}}{\pgfqpoint{1.383508in}{2.400794in}}{\pgfqpoint{1.389332in}{2.394970in}}%
\pgfpathcurveto{\pgfqpoint{1.395156in}{2.389146in}}{\pgfqpoint{1.403056in}{2.385874in}}{\pgfqpoint{1.411292in}{2.385874in}}%
\pgfpathclose%
\pgfusepath{stroke,fill}%
\end{pgfscope}%
\begin{pgfscope}%
\pgfpathrectangle{\pgfqpoint{0.100000in}{0.212622in}}{\pgfqpoint{3.696000in}{3.696000in}}%
\pgfusepath{clip}%
\pgfsetbuttcap%
\pgfsetroundjoin%
\definecolor{currentfill}{rgb}{0.121569,0.466667,0.705882}%
\pgfsetfillcolor{currentfill}%
\pgfsetfillopacity{0.685551}%
\pgfsetlinewidth{1.003750pt}%
\definecolor{currentstroke}{rgb}{0.121569,0.466667,0.705882}%
\pgfsetstrokecolor{currentstroke}%
\pgfsetstrokeopacity{0.685551}%
\pgfsetdash{}{0pt}%
\pgfpathmoveto{\pgfqpoint{2.129331in}{3.012237in}}%
\pgfpathcurveto{\pgfqpoint{2.137567in}{3.012237in}}{\pgfqpoint{2.145467in}{3.015509in}}{\pgfqpoint{2.151291in}{3.021333in}}%
\pgfpathcurveto{\pgfqpoint{2.157115in}{3.027157in}}{\pgfqpoint{2.160387in}{3.035057in}}{\pgfqpoint{2.160387in}{3.043294in}}%
\pgfpathcurveto{\pgfqpoint{2.160387in}{3.051530in}}{\pgfqpoint{2.157115in}{3.059430in}}{\pgfqpoint{2.151291in}{3.065254in}}%
\pgfpathcurveto{\pgfqpoint{2.145467in}{3.071078in}}{\pgfqpoint{2.137567in}{3.074350in}}{\pgfqpoint{2.129331in}{3.074350in}}%
\pgfpathcurveto{\pgfqpoint{2.121094in}{3.074350in}}{\pgfqpoint{2.113194in}{3.071078in}}{\pgfqpoint{2.107371in}{3.065254in}}%
\pgfpathcurveto{\pgfqpoint{2.101547in}{3.059430in}}{\pgfqpoint{2.098274in}{3.051530in}}{\pgfqpoint{2.098274in}{3.043294in}}%
\pgfpathcurveto{\pgfqpoint{2.098274in}{3.035057in}}{\pgfqpoint{2.101547in}{3.027157in}}{\pgfqpoint{2.107371in}{3.021333in}}%
\pgfpathcurveto{\pgfqpoint{2.113194in}{3.015509in}}{\pgfqpoint{2.121094in}{3.012237in}}{\pgfqpoint{2.129331in}{3.012237in}}%
\pgfpathclose%
\pgfusepath{stroke,fill}%
\end{pgfscope}%
\begin{pgfscope}%
\pgfpathrectangle{\pgfqpoint{0.100000in}{0.212622in}}{\pgfqpoint{3.696000in}{3.696000in}}%
\pgfusepath{clip}%
\pgfsetbuttcap%
\pgfsetroundjoin%
\definecolor{currentfill}{rgb}{0.121569,0.466667,0.705882}%
\pgfsetfillcolor{currentfill}%
\pgfsetfillopacity{0.685975}%
\pgfsetlinewidth{1.003750pt}%
\definecolor{currentstroke}{rgb}{0.121569,0.466667,0.705882}%
\pgfsetstrokecolor{currentstroke}%
\pgfsetstrokeopacity{0.685975}%
\pgfsetdash{}{0pt}%
\pgfpathmoveto{\pgfqpoint{3.277456in}{2.614382in}}%
\pgfpathcurveto{\pgfqpoint{3.285693in}{2.614382in}}{\pgfqpoint{3.293593in}{2.617655in}}{\pgfqpoint{3.299417in}{2.623479in}}%
\pgfpathcurveto{\pgfqpoint{3.305240in}{2.629302in}}{\pgfqpoint{3.308513in}{2.637203in}}{\pgfqpoint{3.308513in}{2.645439in}}%
\pgfpathcurveto{\pgfqpoint{3.308513in}{2.653675in}}{\pgfqpoint{3.305240in}{2.661575in}}{\pgfqpoint{3.299417in}{2.667399in}}%
\pgfpathcurveto{\pgfqpoint{3.293593in}{2.673223in}}{\pgfqpoint{3.285693in}{2.676495in}}{\pgfqpoint{3.277456in}{2.676495in}}%
\pgfpathcurveto{\pgfqpoint{3.269220in}{2.676495in}}{\pgfqpoint{3.261320in}{2.673223in}}{\pgfqpoint{3.255496in}{2.667399in}}%
\pgfpathcurveto{\pgfqpoint{3.249672in}{2.661575in}}{\pgfqpoint{3.246400in}{2.653675in}}{\pgfqpoint{3.246400in}{2.645439in}}%
\pgfpathcurveto{\pgfqpoint{3.246400in}{2.637203in}}{\pgfqpoint{3.249672in}{2.629302in}}{\pgfqpoint{3.255496in}{2.623479in}}%
\pgfpathcurveto{\pgfqpoint{3.261320in}{2.617655in}}{\pgfqpoint{3.269220in}{2.614382in}}{\pgfqpoint{3.277456in}{2.614382in}}%
\pgfpathclose%
\pgfusepath{stroke,fill}%
\end{pgfscope}%
\begin{pgfscope}%
\pgfpathrectangle{\pgfqpoint{0.100000in}{0.212622in}}{\pgfqpoint{3.696000in}{3.696000in}}%
\pgfusepath{clip}%
\pgfsetbuttcap%
\pgfsetroundjoin%
\definecolor{currentfill}{rgb}{0.121569,0.466667,0.705882}%
\pgfsetfillcolor{currentfill}%
\pgfsetfillopacity{0.686395}%
\pgfsetlinewidth{1.003750pt}%
\definecolor{currentstroke}{rgb}{0.121569,0.466667,0.705882}%
\pgfsetstrokecolor{currentstroke}%
\pgfsetstrokeopacity{0.686395}%
\pgfsetdash{}{0pt}%
\pgfpathmoveto{\pgfqpoint{1.408810in}{2.377447in}}%
\pgfpathcurveto{\pgfqpoint{1.417046in}{2.377447in}}{\pgfqpoint{1.424946in}{2.380720in}}{\pgfqpoint{1.430770in}{2.386544in}}%
\pgfpathcurveto{\pgfqpoint{1.436594in}{2.392368in}}{\pgfqpoint{1.439867in}{2.400268in}}{\pgfqpoint{1.439867in}{2.408504in}}%
\pgfpathcurveto{\pgfqpoint{1.439867in}{2.416740in}}{\pgfqpoint{1.436594in}{2.424640in}}{\pgfqpoint{1.430770in}{2.430464in}}%
\pgfpathcurveto{\pgfqpoint{1.424946in}{2.436288in}}{\pgfqpoint{1.417046in}{2.439560in}}{\pgfqpoint{1.408810in}{2.439560in}}%
\pgfpathcurveto{\pgfqpoint{1.400574in}{2.439560in}}{\pgfqpoint{1.392674in}{2.436288in}}{\pgfqpoint{1.386850in}{2.430464in}}%
\pgfpathcurveto{\pgfqpoint{1.381026in}{2.424640in}}{\pgfqpoint{1.377754in}{2.416740in}}{\pgfqpoint{1.377754in}{2.408504in}}%
\pgfpathcurveto{\pgfqpoint{1.377754in}{2.400268in}}{\pgfqpoint{1.381026in}{2.392368in}}{\pgfqpoint{1.386850in}{2.386544in}}%
\pgfpathcurveto{\pgfqpoint{1.392674in}{2.380720in}}{\pgfqpoint{1.400574in}{2.377447in}}{\pgfqpoint{1.408810in}{2.377447in}}%
\pgfpathclose%
\pgfusepath{stroke,fill}%
\end{pgfscope}%
\begin{pgfscope}%
\pgfpathrectangle{\pgfqpoint{0.100000in}{0.212622in}}{\pgfqpoint{3.696000in}{3.696000in}}%
\pgfusepath{clip}%
\pgfsetbuttcap%
\pgfsetroundjoin%
\definecolor{currentfill}{rgb}{0.121569,0.466667,0.705882}%
\pgfsetfillcolor{currentfill}%
\pgfsetfillopacity{0.686842}%
\pgfsetlinewidth{1.003750pt}%
\definecolor{currentstroke}{rgb}{0.121569,0.466667,0.705882}%
\pgfsetstrokecolor{currentstroke}%
\pgfsetstrokeopacity{0.686842}%
\pgfsetdash{}{0pt}%
\pgfpathmoveto{\pgfqpoint{1.406963in}{2.372970in}}%
\pgfpathcurveto{\pgfqpoint{1.415199in}{2.372970in}}{\pgfqpoint{1.423099in}{2.376242in}}{\pgfqpoint{1.428923in}{2.382066in}}%
\pgfpathcurveto{\pgfqpoint{1.434747in}{2.387890in}}{\pgfqpoint{1.438020in}{2.395790in}}{\pgfqpoint{1.438020in}{2.404026in}}%
\pgfpathcurveto{\pgfqpoint{1.438020in}{2.412263in}}{\pgfqpoint{1.434747in}{2.420163in}}{\pgfqpoint{1.428923in}{2.425987in}}%
\pgfpathcurveto{\pgfqpoint{1.423099in}{2.431811in}}{\pgfqpoint{1.415199in}{2.435083in}}{\pgfqpoint{1.406963in}{2.435083in}}%
\pgfpathcurveto{\pgfqpoint{1.398727in}{2.435083in}}{\pgfqpoint{1.390827in}{2.431811in}}{\pgfqpoint{1.385003in}{2.425987in}}%
\pgfpathcurveto{\pgfqpoint{1.379179in}{2.420163in}}{\pgfqpoint{1.375907in}{2.412263in}}{\pgfqpoint{1.375907in}{2.404026in}}%
\pgfpathcurveto{\pgfqpoint{1.375907in}{2.395790in}}{\pgfqpoint{1.379179in}{2.387890in}}{\pgfqpoint{1.385003in}{2.382066in}}%
\pgfpathcurveto{\pgfqpoint{1.390827in}{2.376242in}}{\pgfqpoint{1.398727in}{2.372970in}}{\pgfqpoint{1.406963in}{2.372970in}}%
\pgfpathclose%
\pgfusepath{stroke,fill}%
\end{pgfscope}%
\begin{pgfscope}%
\pgfpathrectangle{\pgfqpoint{0.100000in}{0.212622in}}{\pgfqpoint{3.696000in}{3.696000in}}%
\pgfusepath{clip}%
\pgfsetbuttcap%
\pgfsetroundjoin%
\definecolor{currentfill}{rgb}{0.121569,0.466667,0.705882}%
\pgfsetfillcolor{currentfill}%
\pgfsetfillopacity{0.687064}%
\pgfsetlinewidth{1.003750pt}%
\definecolor{currentstroke}{rgb}{0.121569,0.466667,0.705882}%
\pgfsetstrokecolor{currentstroke}%
\pgfsetstrokeopacity{0.687064}%
\pgfsetdash{}{0pt}%
\pgfpathmoveto{\pgfqpoint{1.405612in}{2.370961in}}%
\pgfpathcurveto{\pgfqpoint{1.413848in}{2.370961in}}{\pgfqpoint{1.421748in}{2.374233in}}{\pgfqpoint{1.427572in}{2.380057in}}%
\pgfpathcurveto{\pgfqpoint{1.433396in}{2.385881in}}{\pgfqpoint{1.436669in}{2.393781in}}{\pgfqpoint{1.436669in}{2.402017in}}%
\pgfpathcurveto{\pgfqpoint{1.436669in}{2.410254in}}{\pgfqpoint{1.433396in}{2.418154in}}{\pgfqpoint{1.427572in}{2.423978in}}%
\pgfpathcurveto{\pgfqpoint{1.421748in}{2.429802in}}{\pgfqpoint{1.413848in}{2.433074in}}{\pgfqpoint{1.405612in}{2.433074in}}%
\pgfpathcurveto{\pgfqpoint{1.397376in}{2.433074in}}{\pgfqpoint{1.389476in}{2.429802in}}{\pgfqpoint{1.383652in}{2.423978in}}%
\pgfpathcurveto{\pgfqpoint{1.377828in}{2.418154in}}{\pgfqpoint{1.374556in}{2.410254in}}{\pgfqpoint{1.374556in}{2.402017in}}%
\pgfpathcurveto{\pgfqpoint{1.374556in}{2.393781in}}{\pgfqpoint{1.377828in}{2.385881in}}{\pgfqpoint{1.383652in}{2.380057in}}%
\pgfpathcurveto{\pgfqpoint{1.389476in}{2.374233in}}{\pgfqpoint{1.397376in}{2.370961in}}{\pgfqpoint{1.405612in}{2.370961in}}%
\pgfpathclose%
\pgfusepath{stroke,fill}%
\end{pgfscope}%
\begin{pgfscope}%
\pgfpathrectangle{\pgfqpoint{0.100000in}{0.212622in}}{\pgfqpoint{3.696000in}{3.696000in}}%
\pgfusepath{clip}%
\pgfsetbuttcap%
\pgfsetroundjoin%
\definecolor{currentfill}{rgb}{0.121569,0.466667,0.705882}%
\pgfsetfillcolor{currentfill}%
\pgfsetfillopacity{0.687386}%
\pgfsetlinewidth{1.003750pt}%
\definecolor{currentstroke}{rgb}{0.121569,0.466667,0.705882}%
\pgfsetstrokecolor{currentstroke}%
\pgfsetstrokeopacity{0.687386}%
\pgfsetdash{}{0pt}%
\pgfpathmoveto{\pgfqpoint{1.403836in}{2.368543in}}%
\pgfpathcurveto{\pgfqpoint{1.412073in}{2.368543in}}{\pgfqpoint{1.419973in}{2.371815in}}{\pgfqpoint{1.425797in}{2.377639in}}%
\pgfpathcurveto{\pgfqpoint{1.431620in}{2.383463in}}{\pgfqpoint{1.434893in}{2.391363in}}{\pgfqpoint{1.434893in}{2.399599in}}%
\pgfpathcurveto{\pgfqpoint{1.434893in}{2.407835in}}{\pgfqpoint{1.431620in}{2.415735in}}{\pgfqpoint{1.425797in}{2.421559in}}%
\pgfpathcurveto{\pgfqpoint{1.419973in}{2.427383in}}{\pgfqpoint{1.412073in}{2.430656in}}{\pgfqpoint{1.403836in}{2.430656in}}%
\pgfpathcurveto{\pgfqpoint{1.395600in}{2.430656in}}{\pgfqpoint{1.387700in}{2.427383in}}{\pgfqpoint{1.381876in}{2.421559in}}%
\pgfpathcurveto{\pgfqpoint{1.376052in}{2.415735in}}{\pgfqpoint{1.372780in}{2.407835in}}{\pgfqpoint{1.372780in}{2.399599in}}%
\pgfpathcurveto{\pgfqpoint{1.372780in}{2.391363in}}{\pgfqpoint{1.376052in}{2.383463in}}{\pgfqpoint{1.381876in}{2.377639in}}%
\pgfpathcurveto{\pgfqpoint{1.387700in}{2.371815in}}{\pgfqpoint{1.395600in}{2.368543in}}{\pgfqpoint{1.403836in}{2.368543in}}%
\pgfpathclose%
\pgfusepath{stroke,fill}%
\end{pgfscope}%
\begin{pgfscope}%
\pgfpathrectangle{\pgfqpoint{0.100000in}{0.212622in}}{\pgfqpoint{3.696000in}{3.696000in}}%
\pgfusepath{clip}%
\pgfsetbuttcap%
\pgfsetroundjoin%
\definecolor{currentfill}{rgb}{0.121569,0.466667,0.705882}%
\pgfsetfillcolor{currentfill}%
\pgfsetfillopacity{0.687508}%
\pgfsetlinewidth{1.003750pt}%
\definecolor{currentstroke}{rgb}{0.121569,0.466667,0.705882}%
\pgfsetstrokecolor{currentstroke}%
\pgfsetstrokeopacity{0.687508}%
\pgfsetdash{}{0pt}%
\pgfpathmoveto{\pgfqpoint{2.141996in}{3.009385in}}%
\pgfpathcurveto{\pgfqpoint{2.150233in}{3.009385in}}{\pgfqpoint{2.158133in}{3.012658in}}{\pgfqpoint{2.163957in}{3.018482in}}%
\pgfpathcurveto{\pgfqpoint{2.169781in}{3.024306in}}{\pgfqpoint{2.173053in}{3.032206in}}{\pgfqpoint{2.173053in}{3.040442in}}%
\pgfpathcurveto{\pgfqpoint{2.173053in}{3.048678in}}{\pgfqpoint{2.169781in}{3.056578in}}{\pgfqpoint{2.163957in}{3.062402in}}%
\pgfpathcurveto{\pgfqpoint{2.158133in}{3.068226in}}{\pgfqpoint{2.150233in}{3.071498in}}{\pgfqpoint{2.141996in}{3.071498in}}%
\pgfpathcurveto{\pgfqpoint{2.133760in}{3.071498in}}{\pgfqpoint{2.125860in}{3.068226in}}{\pgfqpoint{2.120036in}{3.062402in}}%
\pgfpathcurveto{\pgfqpoint{2.114212in}{3.056578in}}{\pgfqpoint{2.110940in}{3.048678in}}{\pgfqpoint{2.110940in}{3.040442in}}%
\pgfpathcurveto{\pgfqpoint{2.110940in}{3.032206in}}{\pgfqpoint{2.114212in}{3.024306in}}{\pgfqpoint{2.120036in}{3.018482in}}%
\pgfpathcurveto{\pgfqpoint{2.125860in}{3.012658in}}{\pgfqpoint{2.133760in}{3.009385in}}{\pgfqpoint{2.141996in}{3.009385in}}%
\pgfpathclose%
\pgfusepath{stroke,fill}%
\end{pgfscope}%
\begin{pgfscope}%
\pgfpathrectangle{\pgfqpoint{0.100000in}{0.212622in}}{\pgfqpoint{3.696000in}{3.696000in}}%
\pgfusepath{clip}%
\pgfsetbuttcap%
\pgfsetroundjoin%
\definecolor{currentfill}{rgb}{0.121569,0.466667,0.705882}%
\pgfsetfillcolor{currentfill}%
\pgfsetfillopacity{0.687762}%
\pgfsetlinewidth{1.003750pt}%
\definecolor{currentstroke}{rgb}{0.121569,0.466667,0.705882}%
\pgfsetstrokecolor{currentstroke}%
\pgfsetstrokeopacity{0.687762}%
\pgfsetdash{}{0pt}%
\pgfpathmoveto{\pgfqpoint{3.283014in}{2.614264in}}%
\pgfpathcurveto{\pgfqpoint{3.291250in}{2.614264in}}{\pgfqpoint{3.299150in}{2.617537in}}{\pgfqpoint{3.304974in}{2.623361in}}%
\pgfpathcurveto{\pgfqpoint{3.310798in}{2.629185in}}{\pgfqpoint{3.314070in}{2.637085in}}{\pgfqpoint{3.314070in}{2.645321in}}%
\pgfpathcurveto{\pgfqpoint{3.314070in}{2.653557in}}{\pgfqpoint{3.310798in}{2.661457in}}{\pgfqpoint{3.304974in}{2.667281in}}%
\pgfpathcurveto{\pgfqpoint{3.299150in}{2.673105in}}{\pgfqpoint{3.291250in}{2.676377in}}{\pgfqpoint{3.283014in}{2.676377in}}%
\pgfpathcurveto{\pgfqpoint{3.274778in}{2.676377in}}{\pgfqpoint{3.266878in}{2.673105in}}{\pgfqpoint{3.261054in}{2.667281in}}%
\pgfpathcurveto{\pgfqpoint{3.255230in}{2.661457in}}{\pgfqpoint{3.251957in}{2.653557in}}{\pgfqpoint{3.251957in}{2.645321in}}%
\pgfpathcurveto{\pgfqpoint{3.251957in}{2.637085in}}{\pgfqpoint{3.255230in}{2.629185in}}{\pgfqpoint{3.261054in}{2.623361in}}%
\pgfpathcurveto{\pgfqpoint{3.266878in}{2.617537in}}{\pgfqpoint{3.274778in}{2.614264in}}{\pgfqpoint{3.283014in}{2.614264in}}%
\pgfpathclose%
\pgfusepath{stroke,fill}%
\end{pgfscope}%
\begin{pgfscope}%
\pgfpathrectangle{\pgfqpoint{0.100000in}{0.212622in}}{\pgfqpoint{3.696000in}{3.696000in}}%
\pgfusepath{clip}%
\pgfsetbuttcap%
\pgfsetroundjoin%
\definecolor{currentfill}{rgb}{0.121569,0.466667,0.705882}%
\pgfsetfillcolor{currentfill}%
\pgfsetfillopacity{0.687861}%
\pgfsetlinewidth{1.003750pt}%
\definecolor{currentstroke}{rgb}{0.121569,0.466667,0.705882}%
\pgfsetstrokecolor{currentstroke}%
\pgfsetstrokeopacity{0.687861}%
\pgfsetdash{}{0pt}%
\pgfpathmoveto{\pgfqpoint{1.401576in}{2.364618in}}%
\pgfpathcurveto{\pgfqpoint{1.409812in}{2.364618in}}{\pgfqpoint{1.417712in}{2.367890in}}{\pgfqpoint{1.423536in}{2.373714in}}%
\pgfpathcurveto{\pgfqpoint{1.429360in}{2.379538in}}{\pgfqpoint{1.432632in}{2.387438in}}{\pgfqpoint{1.432632in}{2.395674in}}%
\pgfpathcurveto{\pgfqpoint{1.432632in}{2.403910in}}{\pgfqpoint{1.429360in}{2.411810in}}{\pgfqpoint{1.423536in}{2.417634in}}%
\pgfpathcurveto{\pgfqpoint{1.417712in}{2.423458in}}{\pgfqpoint{1.409812in}{2.426731in}}{\pgfqpoint{1.401576in}{2.426731in}}%
\pgfpathcurveto{\pgfqpoint{1.393339in}{2.426731in}}{\pgfqpoint{1.385439in}{2.423458in}}{\pgfqpoint{1.379615in}{2.417634in}}%
\pgfpathcurveto{\pgfqpoint{1.373791in}{2.411810in}}{\pgfqpoint{1.370519in}{2.403910in}}{\pgfqpoint{1.370519in}{2.395674in}}%
\pgfpathcurveto{\pgfqpoint{1.370519in}{2.387438in}}{\pgfqpoint{1.373791in}{2.379538in}}{\pgfqpoint{1.379615in}{2.373714in}}%
\pgfpathcurveto{\pgfqpoint{1.385439in}{2.367890in}}{\pgfqpoint{1.393339in}{2.364618in}}{\pgfqpoint{1.401576in}{2.364618in}}%
\pgfpathclose%
\pgfusepath{stroke,fill}%
\end{pgfscope}%
\begin{pgfscope}%
\pgfpathrectangle{\pgfqpoint{0.100000in}{0.212622in}}{\pgfqpoint{3.696000in}{3.696000in}}%
\pgfusepath{clip}%
\pgfsetbuttcap%
\pgfsetroundjoin%
\definecolor{currentfill}{rgb}{0.121569,0.466667,0.705882}%
\pgfsetfillcolor{currentfill}%
\pgfsetfillopacity{0.688506}%
\pgfsetlinewidth{1.003750pt}%
\definecolor{currentstroke}{rgb}{0.121569,0.466667,0.705882}%
\pgfsetstrokecolor{currentstroke}%
\pgfsetstrokeopacity{0.688506}%
\pgfsetdash{}{0pt}%
\pgfpathmoveto{\pgfqpoint{1.399568in}{2.358541in}}%
\pgfpathcurveto{\pgfqpoint{1.407804in}{2.358541in}}{\pgfqpoint{1.415704in}{2.361814in}}{\pgfqpoint{1.421528in}{2.367638in}}%
\pgfpathcurveto{\pgfqpoint{1.427352in}{2.373461in}}{\pgfqpoint{1.430624in}{2.381362in}}{\pgfqpoint{1.430624in}{2.389598in}}%
\pgfpathcurveto{\pgfqpoint{1.430624in}{2.397834in}}{\pgfqpoint{1.427352in}{2.405734in}}{\pgfqpoint{1.421528in}{2.411558in}}%
\pgfpathcurveto{\pgfqpoint{1.415704in}{2.417382in}}{\pgfqpoint{1.407804in}{2.420654in}}{\pgfqpoint{1.399568in}{2.420654in}}%
\pgfpathcurveto{\pgfqpoint{1.391331in}{2.420654in}}{\pgfqpoint{1.383431in}{2.417382in}}{\pgfqpoint{1.377607in}{2.411558in}}%
\pgfpathcurveto{\pgfqpoint{1.371783in}{2.405734in}}{\pgfqpoint{1.368511in}{2.397834in}}{\pgfqpoint{1.368511in}{2.389598in}}%
\pgfpathcurveto{\pgfqpoint{1.368511in}{2.381362in}}{\pgfqpoint{1.371783in}{2.373461in}}{\pgfqpoint{1.377607in}{2.367638in}}%
\pgfpathcurveto{\pgfqpoint{1.383431in}{2.361814in}}{\pgfqpoint{1.391331in}{2.358541in}}{\pgfqpoint{1.399568in}{2.358541in}}%
\pgfpathclose%
\pgfusepath{stroke,fill}%
\end{pgfscope}%
\begin{pgfscope}%
\pgfpathrectangle{\pgfqpoint{0.100000in}{0.212622in}}{\pgfqpoint{3.696000in}{3.696000in}}%
\pgfusepath{clip}%
\pgfsetbuttcap%
\pgfsetroundjoin%
\definecolor{currentfill}{rgb}{0.121569,0.466667,0.705882}%
\pgfsetfillcolor{currentfill}%
\pgfsetfillopacity{0.688832}%
\pgfsetlinewidth{1.003750pt}%
\definecolor{currentstroke}{rgb}{0.121569,0.466667,0.705882}%
\pgfsetstrokecolor{currentstroke}%
\pgfsetstrokeopacity{0.688832}%
\pgfsetdash{}{0pt}%
\pgfpathmoveto{\pgfqpoint{3.287398in}{2.613371in}}%
\pgfpathcurveto{\pgfqpoint{3.295634in}{2.613371in}}{\pgfqpoint{3.303534in}{2.616643in}}{\pgfqpoint{3.309358in}{2.622467in}}%
\pgfpathcurveto{\pgfqpoint{3.315182in}{2.628291in}}{\pgfqpoint{3.318454in}{2.636191in}}{\pgfqpoint{3.318454in}{2.644427in}}%
\pgfpathcurveto{\pgfqpoint{3.318454in}{2.652663in}}{\pgfqpoint{3.315182in}{2.660563in}}{\pgfqpoint{3.309358in}{2.666387in}}%
\pgfpathcurveto{\pgfqpoint{3.303534in}{2.672211in}}{\pgfqpoint{3.295634in}{2.675484in}}{\pgfqpoint{3.287398in}{2.675484in}}%
\pgfpathcurveto{\pgfqpoint{3.279161in}{2.675484in}}{\pgfqpoint{3.271261in}{2.672211in}}{\pgfqpoint{3.265437in}{2.666387in}}%
\pgfpathcurveto{\pgfqpoint{3.259613in}{2.660563in}}{\pgfqpoint{3.256341in}{2.652663in}}{\pgfqpoint{3.256341in}{2.644427in}}%
\pgfpathcurveto{\pgfqpoint{3.256341in}{2.636191in}}{\pgfqpoint{3.259613in}{2.628291in}}{\pgfqpoint{3.265437in}{2.622467in}}%
\pgfpathcurveto{\pgfqpoint{3.271261in}{2.616643in}}{\pgfqpoint{3.279161in}{2.613371in}}{\pgfqpoint{3.287398in}{2.613371in}}%
\pgfpathclose%
\pgfusepath{stroke,fill}%
\end{pgfscope}%
\begin{pgfscope}%
\pgfpathrectangle{\pgfqpoint{0.100000in}{0.212622in}}{\pgfqpoint{3.696000in}{3.696000in}}%
\pgfusepath{clip}%
\pgfsetbuttcap%
\pgfsetroundjoin%
\definecolor{currentfill}{rgb}{0.121569,0.466667,0.705882}%
\pgfsetfillcolor{currentfill}%
\pgfsetfillopacity{0.688873}%
\pgfsetlinewidth{1.003750pt}%
\definecolor{currentstroke}{rgb}{0.121569,0.466667,0.705882}%
\pgfsetstrokecolor{currentstroke}%
\pgfsetstrokeopacity{0.688873}%
\pgfsetdash{}{0pt}%
\pgfpathmoveto{\pgfqpoint{1.398412in}{2.355286in}}%
\pgfpathcurveto{\pgfqpoint{1.406648in}{2.355286in}}{\pgfqpoint{1.414548in}{2.358558in}}{\pgfqpoint{1.420372in}{2.364382in}}%
\pgfpathcurveto{\pgfqpoint{1.426196in}{2.370206in}}{\pgfqpoint{1.429468in}{2.378106in}}{\pgfqpoint{1.429468in}{2.386343in}}%
\pgfpathcurveto{\pgfqpoint{1.429468in}{2.394579in}}{\pgfqpoint{1.426196in}{2.402479in}}{\pgfqpoint{1.420372in}{2.408303in}}%
\pgfpathcurveto{\pgfqpoint{1.414548in}{2.414127in}}{\pgfqpoint{1.406648in}{2.417399in}}{\pgfqpoint{1.398412in}{2.417399in}}%
\pgfpathcurveto{\pgfqpoint{1.390176in}{2.417399in}}{\pgfqpoint{1.382276in}{2.414127in}}{\pgfqpoint{1.376452in}{2.408303in}}%
\pgfpathcurveto{\pgfqpoint{1.370628in}{2.402479in}}{\pgfqpoint{1.367355in}{2.394579in}}{\pgfqpoint{1.367355in}{2.386343in}}%
\pgfpathcurveto{\pgfqpoint{1.367355in}{2.378106in}}{\pgfqpoint{1.370628in}{2.370206in}}{\pgfqpoint{1.376452in}{2.364382in}}%
\pgfpathcurveto{\pgfqpoint{1.382276in}{2.358558in}}{\pgfqpoint{1.390176in}{2.355286in}}{\pgfqpoint{1.398412in}{2.355286in}}%
\pgfpathclose%
\pgfusepath{stroke,fill}%
\end{pgfscope}%
\begin{pgfscope}%
\pgfpathrectangle{\pgfqpoint{0.100000in}{0.212622in}}{\pgfqpoint{3.696000in}{3.696000in}}%
\pgfusepath{clip}%
\pgfsetbuttcap%
\pgfsetroundjoin%
\definecolor{currentfill}{rgb}{0.121569,0.466667,0.705882}%
\pgfsetfillcolor{currentfill}%
\pgfsetfillopacity{0.689284}%
\pgfsetlinewidth{1.003750pt}%
\definecolor{currentstroke}{rgb}{0.121569,0.466667,0.705882}%
\pgfsetstrokecolor{currentstroke}%
\pgfsetstrokeopacity{0.689284}%
\pgfsetdash{}{0pt}%
\pgfpathmoveto{\pgfqpoint{1.396497in}{2.351808in}}%
\pgfpathcurveto{\pgfqpoint{1.404734in}{2.351808in}}{\pgfqpoint{1.412634in}{2.355080in}}{\pgfqpoint{1.418458in}{2.360904in}}%
\pgfpathcurveto{\pgfqpoint{1.424281in}{2.366728in}}{\pgfqpoint{1.427554in}{2.374628in}}{\pgfqpoint{1.427554in}{2.382865in}}%
\pgfpathcurveto{\pgfqpoint{1.427554in}{2.391101in}}{\pgfqpoint{1.424281in}{2.399001in}}{\pgfqpoint{1.418458in}{2.404825in}}%
\pgfpathcurveto{\pgfqpoint{1.412634in}{2.410649in}}{\pgfqpoint{1.404734in}{2.413921in}}{\pgfqpoint{1.396497in}{2.413921in}}%
\pgfpathcurveto{\pgfqpoint{1.388261in}{2.413921in}}{\pgfqpoint{1.380361in}{2.410649in}}{\pgfqpoint{1.374537in}{2.404825in}}%
\pgfpathcurveto{\pgfqpoint{1.368713in}{2.399001in}}{\pgfqpoint{1.365441in}{2.391101in}}{\pgfqpoint{1.365441in}{2.382865in}}%
\pgfpathcurveto{\pgfqpoint{1.365441in}{2.374628in}}{\pgfqpoint{1.368713in}{2.366728in}}{\pgfqpoint{1.374537in}{2.360904in}}%
\pgfpathcurveto{\pgfqpoint{1.380361in}{2.355080in}}{\pgfqpoint{1.388261in}{2.351808in}}{\pgfqpoint{1.396497in}{2.351808in}}%
\pgfpathclose%
\pgfusepath{stroke,fill}%
\end{pgfscope}%
\begin{pgfscope}%
\pgfpathrectangle{\pgfqpoint{0.100000in}{0.212622in}}{\pgfqpoint{3.696000in}{3.696000in}}%
\pgfusepath{clip}%
\pgfsetbuttcap%
\pgfsetroundjoin%
\definecolor{currentfill}{rgb}{0.121569,0.466667,0.705882}%
\pgfsetfillcolor{currentfill}%
\pgfsetfillopacity{0.689499}%
\pgfsetlinewidth{1.003750pt}%
\definecolor{currentstroke}{rgb}{0.121569,0.466667,0.705882}%
\pgfsetstrokecolor{currentstroke}%
\pgfsetstrokeopacity{0.689499}%
\pgfsetdash{}{0pt}%
\pgfpathmoveto{\pgfqpoint{1.395323in}{2.350085in}}%
\pgfpathcurveto{\pgfqpoint{1.403559in}{2.350085in}}{\pgfqpoint{1.411459in}{2.353358in}}{\pgfqpoint{1.417283in}{2.359182in}}%
\pgfpathcurveto{\pgfqpoint{1.423107in}{2.365006in}}{\pgfqpoint{1.426380in}{2.372906in}}{\pgfqpoint{1.426380in}{2.381142in}}%
\pgfpathcurveto{\pgfqpoint{1.426380in}{2.389378in}}{\pgfqpoint{1.423107in}{2.397278in}}{\pgfqpoint{1.417283in}{2.403102in}}%
\pgfpathcurveto{\pgfqpoint{1.411459in}{2.408926in}}{\pgfqpoint{1.403559in}{2.412198in}}{\pgfqpoint{1.395323in}{2.412198in}}%
\pgfpathcurveto{\pgfqpoint{1.387087in}{2.412198in}}{\pgfqpoint{1.379187in}{2.408926in}}{\pgfqpoint{1.373363in}{2.403102in}}%
\pgfpathcurveto{\pgfqpoint{1.367539in}{2.397278in}}{\pgfqpoint{1.364267in}{2.389378in}}{\pgfqpoint{1.364267in}{2.381142in}}%
\pgfpathcurveto{\pgfqpoint{1.364267in}{2.372906in}}{\pgfqpoint{1.367539in}{2.365006in}}{\pgfqpoint{1.373363in}{2.359182in}}%
\pgfpathcurveto{\pgfqpoint{1.379187in}{2.353358in}}{\pgfqpoint{1.387087in}{2.350085in}}{\pgfqpoint{1.395323in}{2.350085in}}%
\pgfpathclose%
\pgfusepath{stroke,fill}%
\end{pgfscope}%
\begin{pgfscope}%
\pgfpathrectangle{\pgfqpoint{0.100000in}{0.212622in}}{\pgfqpoint{3.696000in}{3.696000in}}%
\pgfusepath{clip}%
\pgfsetbuttcap%
\pgfsetroundjoin%
\definecolor{currentfill}{rgb}{0.121569,0.466667,0.705882}%
\pgfsetfillcolor{currentfill}%
\pgfsetfillopacity{0.689575}%
\pgfsetlinewidth{1.003750pt}%
\definecolor{currentstroke}{rgb}{0.121569,0.466667,0.705882}%
\pgfsetstrokecolor{currentstroke}%
\pgfsetstrokeopacity{0.689575}%
\pgfsetdash{}{0pt}%
\pgfpathmoveto{\pgfqpoint{3.290717in}{2.612511in}}%
\pgfpathcurveto{\pgfqpoint{3.298953in}{2.612511in}}{\pgfqpoint{3.306853in}{2.615783in}}{\pgfqpoint{3.312677in}{2.621607in}}%
\pgfpathcurveto{\pgfqpoint{3.318501in}{2.627431in}}{\pgfqpoint{3.321773in}{2.635331in}}{\pgfqpoint{3.321773in}{2.643567in}}%
\pgfpathcurveto{\pgfqpoint{3.321773in}{2.651803in}}{\pgfqpoint{3.318501in}{2.659704in}}{\pgfqpoint{3.312677in}{2.665527in}}%
\pgfpathcurveto{\pgfqpoint{3.306853in}{2.671351in}}{\pgfqpoint{3.298953in}{2.674624in}}{\pgfqpoint{3.290717in}{2.674624in}}%
\pgfpathcurveto{\pgfqpoint{3.282481in}{2.674624in}}{\pgfqpoint{3.274581in}{2.671351in}}{\pgfqpoint{3.268757in}{2.665527in}}%
\pgfpathcurveto{\pgfqpoint{3.262933in}{2.659704in}}{\pgfqpoint{3.259660in}{2.651803in}}{\pgfqpoint{3.259660in}{2.643567in}}%
\pgfpathcurveto{\pgfqpoint{3.259660in}{2.635331in}}{\pgfqpoint{3.262933in}{2.627431in}}{\pgfqpoint{3.268757in}{2.621607in}}%
\pgfpathcurveto{\pgfqpoint{3.274581in}{2.615783in}}{\pgfqpoint{3.282481in}{2.612511in}}{\pgfqpoint{3.290717in}{2.612511in}}%
\pgfpathclose%
\pgfusepath{stroke,fill}%
\end{pgfscope}%
\begin{pgfscope}%
\pgfpathrectangle{\pgfqpoint{0.100000in}{0.212622in}}{\pgfqpoint{3.696000in}{3.696000in}}%
\pgfusepath{clip}%
\pgfsetbuttcap%
\pgfsetroundjoin%
\definecolor{currentfill}{rgb}{0.121569,0.466667,0.705882}%
\pgfsetfillcolor{currentfill}%
\pgfsetfillopacity{0.689628}%
\pgfsetlinewidth{1.003750pt}%
\definecolor{currentstroke}{rgb}{0.121569,0.466667,0.705882}%
\pgfsetstrokecolor{currentstroke}%
\pgfsetstrokeopacity{0.689628}%
\pgfsetdash{}{0pt}%
\pgfpathmoveto{\pgfqpoint{1.394670in}{2.349203in}}%
\pgfpathcurveto{\pgfqpoint{1.402906in}{2.349203in}}{\pgfqpoint{1.410806in}{2.352475in}}{\pgfqpoint{1.416630in}{2.358299in}}%
\pgfpathcurveto{\pgfqpoint{1.422454in}{2.364123in}}{\pgfqpoint{1.425726in}{2.372023in}}{\pgfqpoint{1.425726in}{2.380259in}}%
\pgfpathcurveto{\pgfqpoint{1.425726in}{2.388495in}}{\pgfqpoint{1.422454in}{2.396396in}}{\pgfqpoint{1.416630in}{2.402219in}}%
\pgfpathcurveto{\pgfqpoint{1.410806in}{2.408043in}}{\pgfqpoint{1.402906in}{2.411316in}}{\pgfqpoint{1.394670in}{2.411316in}}%
\pgfpathcurveto{\pgfqpoint{1.386433in}{2.411316in}}{\pgfqpoint{1.378533in}{2.408043in}}{\pgfqpoint{1.372709in}{2.402219in}}%
\pgfpathcurveto{\pgfqpoint{1.366885in}{2.396396in}}{\pgfqpoint{1.363613in}{2.388495in}}{\pgfqpoint{1.363613in}{2.380259in}}%
\pgfpathcurveto{\pgfqpoint{1.363613in}{2.372023in}}{\pgfqpoint{1.366885in}{2.364123in}}{\pgfqpoint{1.372709in}{2.358299in}}%
\pgfpathcurveto{\pgfqpoint{1.378533in}{2.352475in}}{\pgfqpoint{1.386433in}{2.349203in}}{\pgfqpoint{1.394670in}{2.349203in}}%
\pgfpathclose%
\pgfusepath{stroke,fill}%
\end{pgfscope}%
\begin{pgfscope}%
\pgfpathrectangle{\pgfqpoint{0.100000in}{0.212622in}}{\pgfqpoint{3.696000in}{3.696000in}}%
\pgfusepath{clip}%
\pgfsetbuttcap%
\pgfsetroundjoin%
\definecolor{currentfill}{rgb}{0.121569,0.466667,0.705882}%
\pgfsetfillcolor{currentfill}%
\pgfsetfillopacity{0.689924}%
\pgfsetlinewidth{1.003750pt}%
\definecolor{currentstroke}{rgb}{0.121569,0.466667,0.705882}%
\pgfsetstrokecolor{currentstroke}%
\pgfsetstrokeopacity{0.689924}%
\pgfsetdash{}{0pt}%
\pgfpathmoveto{\pgfqpoint{1.393544in}{2.346584in}}%
\pgfpathcurveto{\pgfqpoint{1.401780in}{2.346584in}}{\pgfqpoint{1.409680in}{2.349857in}}{\pgfqpoint{1.415504in}{2.355681in}}%
\pgfpathcurveto{\pgfqpoint{1.421328in}{2.361505in}}{\pgfqpoint{1.424600in}{2.369405in}}{\pgfqpoint{1.424600in}{2.377641in}}%
\pgfpathcurveto{\pgfqpoint{1.424600in}{2.385877in}}{\pgfqpoint{1.421328in}{2.393777in}}{\pgfqpoint{1.415504in}{2.399601in}}%
\pgfpathcurveto{\pgfqpoint{1.409680in}{2.405425in}}{\pgfqpoint{1.401780in}{2.408697in}}{\pgfqpoint{1.393544in}{2.408697in}}%
\pgfpathcurveto{\pgfqpoint{1.385308in}{2.408697in}}{\pgfqpoint{1.377407in}{2.405425in}}{\pgfqpoint{1.371584in}{2.399601in}}%
\pgfpathcurveto{\pgfqpoint{1.365760in}{2.393777in}}{\pgfqpoint{1.362487in}{2.385877in}}{\pgfqpoint{1.362487in}{2.377641in}}%
\pgfpathcurveto{\pgfqpoint{1.362487in}{2.369405in}}{\pgfqpoint{1.365760in}{2.361505in}}{\pgfqpoint{1.371584in}{2.355681in}}%
\pgfpathcurveto{\pgfqpoint{1.377407in}{2.349857in}}{\pgfqpoint{1.385308in}{2.346584in}}{\pgfqpoint{1.393544in}{2.346584in}}%
\pgfpathclose%
\pgfusepath{stroke,fill}%
\end{pgfscope}%
\begin{pgfscope}%
\pgfpathrectangle{\pgfqpoint{0.100000in}{0.212622in}}{\pgfqpoint{3.696000in}{3.696000in}}%
\pgfusepath{clip}%
\pgfsetbuttcap%
\pgfsetroundjoin%
\definecolor{currentfill}{rgb}{0.121569,0.466667,0.705882}%
\pgfsetfillcolor{currentfill}%
\pgfsetfillopacity{0.690054}%
\pgfsetlinewidth{1.003750pt}%
\definecolor{currentstroke}{rgb}{0.121569,0.466667,0.705882}%
\pgfsetstrokecolor{currentstroke}%
\pgfsetstrokeopacity{0.690054}%
\pgfsetdash{}{0pt}%
\pgfpathmoveto{\pgfqpoint{2.152952in}{3.008577in}}%
\pgfpathcurveto{\pgfqpoint{2.161188in}{3.008577in}}{\pgfqpoint{2.169088in}{3.011849in}}{\pgfqpoint{2.174912in}{3.017673in}}%
\pgfpathcurveto{\pgfqpoint{2.180736in}{3.023497in}}{\pgfqpoint{2.184009in}{3.031397in}}{\pgfqpoint{2.184009in}{3.039633in}}%
\pgfpathcurveto{\pgfqpoint{2.184009in}{3.047870in}}{\pgfqpoint{2.180736in}{3.055770in}}{\pgfqpoint{2.174912in}{3.061594in}}%
\pgfpathcurveto{\pgfqpoint{2.169088in}{3.067418in}}{\pgfqpoint{2.161188in}{3.070690in}}{\pgfqpoint{2.152952in}{3.070690in}}%
\pgfpathcurveto{\pgfqpoint{2.144716in}{3.070690in}}{\pgfqpoint{2.136816in}{3.067418in}}{\pgfqpoint{2.130992in}{3.061594in}}%
\pgfpathcurveto{\pgfqpoint{2.125168in}{3.055770in}}{\pgfqpoint{2.121896in}{3.047870in}}{\pgfqpoint{2.121896in}{3.039633in}}%
\pgfpathcurveto{\pgfqpoint{2.121896in}{3.031397in}}{\pgfqpoint{2.125168in}{3.023497in}}{\pgfqpoint{2.130992in}{3.017673in}}%
\pgfpathcurveto{\pgfqpoint{2.136816in}{3.011849in}}{\pgfqpoint{2.144716in}{3.008577in}}{\pgfqpoint{2.152952in}{3.008577in}}%
\pgfpathclose%
\pgfusepath{stroke,fill}%
\end{pgfscope}%
\begin{pgfscope}%
\pgfpathrectangle{\pgfqpoint{0.100000in}{0.212622in}}{\pgfqpoint{3.696000in}{3.696000in}}%
\pgfusepath{clip}%
\pgfsetbuttcap%
\pgfsetroundjoin%
\definecolor{currentfill}{rgb}{0.121569,0.466667,0.705882}%
\pgfsetfillcolor{currentfill}%
\pgfsetfillopacity{0.690266}%
\pgfsetlinewidth{1.003750pt}%
\definecolor{currentstroke}{rgb}{0.121569,0.466667,0.705882}%
\pgfsetstrokecolor{currentstroke}%
\pgfsetstrokeopacity{0.690266}%
\pgfsetdash{}{0pt}%
\pgfpathmoveto{\pgfqpoint{1.392768in}{2.343213in}}%
\pgfpathcurveto{\pgfqpoint{1.401005in}{2.343213in}}{\pgfqpoint{1.408905in}{2.346485in}}{\pgfqpoint{1.414729in}{2.352309in}}%
\pgfpathcurveto{\pgfqpoint{1.420552in}{2.358133in}}{\pgfqpoint{1.423825in}{2.366033in}}{\pgfqpoint{1.423825in}{2.374269in}}%
\pgfpathcurveto{\pgfqpoint{1.423825in}{2.382505in}}{\pgfqpoint{1.420552in}{2.390405in}}{\pgfqpoint{1.414729in}{2.396229in}}%
\pgfpathcurveto{\pgfqpoint{1.408905in}{2.402053in}}{\pgfqpoint{1.401005in}{2.405326in}}{\pgfqpoint{1.392768in}{2.405326in}}%
\pgfpathcurveto{\pgfqpoint{1.384532in}{2.405326in}}{\pgfqpoint{1.376632in}{2.402053in}}{\pgfqpoint{1.370808in}{2.396229in}}%
\pgfpathcurveto{\pgfqpoint{1.364984in}{2.390405in}}{\pgfqpoint{1.361712in}{2.382505in}}{\pgfqpoint{1.361712in}{2.374269in}}%
\pgfpathcurveto{\pgfqpoint{1.361712in}{2.366033in}}{\pgfqpoint{1.364984in}{2.358133in}}{\pgfqpoint{1.370808in}{2.352309in}}%
\pgfpathcurveto{\pgfqpoint{1.376632in}{2.346485in}}{\pgfqpoint{1.384532in}{2.343213in}}{\pgfqpoint{1.392768in}{2.343213in}}%
\pgfpathclose%
\pgfusepath{stroke,fill}%
\end{pgfscope}%
\begin{pgfscope}%
\pgfpathrectangle{\pgfqpoint{0.100000in}{0.212622in}}{\pgfqpoint{3.696000in}{3.696000in}}%
\pgfusepath{clip}%
\pgfsetbuttcap%
\pgfsetroundjoin%
\definecolor{currentfill}{rgb}{0.121569,0.466667,0.705882}%
\pgfsetfillcolor{currentfill}%
\pgfsetfillopacity{0.690717}%
\pgfsetlinewidth{1.003750pt}%
\definecolor{currentstroke}{rgb}{0.121569,0.466667,0.705882}%
\pgfsetstrokecolor{currentstroke}%
\pgfsetstrokeopacity{0.690717}%
\pgfsetdash{}{0pt}%
\pgfpathmoveto{\pgfqpoint{1.390896in}{2.339467in}}%
\pgfpathcurveto{\pgfqpoint{1.399132in}{2.339467in}}{\pgfqpoint{1.407032in}{2.342740in}}{\pgfqpoint{1.412856in}{2.348564in}}%
\pgfpathcurveto{\pgfqpoint{1.418680in}{2.354388in}}{\pgfqpoint{1.421953in}{2.362288in}}{\pgfqpoint{1.421953in}{2.370524in}}%
\pgfpathcurveto{\pgfqpoint{1.421953in}{2.378760in}}{\pgfqpoint{1.418680in}{2.386660in}}{\pgfqpoint{1.412856in}{2.392484in}}%
\pgfpathcurveto{\pgfqpoint{1.407032in}{2.398308in}}{\pgfqpoint{1.399132in}{2.401580in}}{\pgfqpoint{1.390896in}{2.401580in}}%
\pgfpathcurveto{\pgfqpoint{1.382660in}{2.401580in}}{\pgfqpoint{1.374760in}{2.398308in}}{\pgfqpoint{1.368936in}{2.392484in}}%
\pgfpathcurveto{\pgfqpoint{1.363112in}{2.386660in}}{\pgfqpoint{1.359840in}{2.378760in}}{\pgfqpoint{1.359840in}{2.370524in}}%
\pgfpathcurveto{\pgfqpoint{1.359840in}{2.362288in}}{\pgfqpoint{1.363112in}{2.354388in}}{\pgfqpoint{1.368936in}{2.348564in}}%
\pgfpathcurveto{\pgfqpoint{1.374760in}{2.342740in}}{\pgfqpoint{1.382660in}{2.339467in}}{\pgfqpoint{1.390896in}{2.339467in}}%
\pgfpathclose%
\pgfusepath{stroke,fill}%
\end{pgfscope}%
\begin{pgfscope}%
\pgfpathrectangle{\pgfqpoint{0.100000in}{0.212622in}}{\pgfqpoint{3.696000in}{3.696000in}}%
\pgfusepath{clip}%
\pgfsetbuttcap%
\pgfsetroundjoin%
\definecolor{currentfill}{rgb}{0.121569,0.466667,0.705882}%
\pgfsetfillcolor{currentfill}%
\pgfsetfillopacity{0.690957}%
\pgfsetlinewidth{1.003750pt}%
\definecolor{currentstroke}{rgb}{0.121569,0.466667,0.705882}%
\pgfsetstrokecolor{currentstroke}%
\pgfsetstrokeopacity{0.690957}%
\pgfsetdash{}{0pt}%
\pgfpathmoveto{\pgfqpoint{3.296535in}{2.610397in}}%
\pgfpathcurveto{\pgfqpoint{3.304771in}{2.610397in}}{\pgfqpoint{3.312671in}{2.613669in}}{\pgfqpoint{3.318495in}{2.619493in}}%
\pgfpathcurveto{\pgfqpoint{3.324319in}{2.625317in}}{\pgfqpoint{3.327591in}{2.633217in}}{\pgfqpoint{3.327591in}{2.641453in}}%
\pgfpathcurveto{\pgfqpoint{3.327591in}{2.649690in}}{\pgfqpoint{3.324319in}{2.657590in}}{\pgfqpoint{3.318495in}{2.663414in}}%
\pgfpathcurveto{\pgfqpoint{3.312671in}{2.669238in}}{\pgfqpoint{3.304771in}{2.672510in}}{\pgfqpoint{3.296535in}{2.672510in}}%
\pgfpathcurveto{\pgfqpoint{3.288298in}{2.672510in}}{\pgfqpoint{3.280398in}{2.669238in}}{\pgfqpoint{3.274574in}{2.663414in}}%
\pgfpathcurveto{\pgfqpoint{3.268750in}{2.657590in}}{\pgfqpoint{3.265478in}{2.649690in}}{\pgfqpoint{3.265478in}{2.641453in}}%
\pgfpathcurveto{\pgfqpoint{3.265478in}{2.633217in}}{\pgfqpoint{3.268750in}{2.625317in}}{\pgfqpoint{3.274574in}{2.619493in}}%
\pgfpathcurveto{\pgfqpoint{3.280398in}{2.613669in}}{\pgfqpoint{3.288298in}{2.610397in}}{\pgfqpoint{3.296535in}{2.610397in}}%
\pgfpathclose%
\pgfusepath{stroke,fill}%
\end{pgfscope}%
\begin{pgfscope}%
\pgfpathrectangle{\pgfqpoint{0.100000in}{0.212622in}}{\pgfqpoint{3.696000in}{3.696000in}}%
\pgfusepath{clip}%
\pgfsetbuttcap%
\pgfsetroundjoin%
\definecolor{currentfill}{rgb}{0.121569,0.466667,0.705882}%
\pgfsetfillcolor{currentfill}%
\pgfsetfillopacity{0.691231}%
\pgfsetlinewidth{1.003750pt}%
\definecolor{currentstroke}{rgb}{0.121569,0.466667,0.705882}%
\pgfsetstrokecolor{currentstroke}%
\pgfsetstrokeopacity{0.691231}%
\pgfsetdash{}{0pt}%
\pgfpathmoveto{\pgfqpoint{1.388051in}{2.335360in}}%
\pgfpathcurveto{\pgfqpoint{1.396287in}{2.335360in}}{\pgfqpoint{1.404187in}{2.338633in}}{\pgfqpoint{1.410011in}{2.344457in}}%
\pgfpathcurveto{\pgfqpoint{1.415835in}{2.350281in}}{\pgfqpoint{1.419107in}{2.358181in}}{\pgfqpoint{1.419107in}{2.366417in}}%
\pgfpathcurveto{\pgfqpoint{1.419107in}{2.374653in}}{\pgfqpoint{1.415835in}{2.382553in}}{\pgfqpoint{1.410011in}{2.388377in}}%
\pgfpathcurveto{\pgfqpoint{1.404187in}{2.394201in}}{\pgfqpoint{1.396287in}{2.397473in}}{\pgfqpoint{1.388051in}{2.397473in}}%
\pgfpathcurveto{\pgfqpoint{1.379815in}{2.397473in}}{\pgfqpoint{1.371915in}{2.394201in}}{\pgfqpoint{1.366091in}{2.388377in}}%
\pgfpathcurveto{\pgfqpoint{1.360267in}{2.382553in}}{\pgfqpoint{1.356994in}{2.374653in}}{\pgfqpoint{1.356994in}{2.366417in}}%
\pgfpathcurveto{\pgfqpoint{1.356994in}{2.358181in}}{\pgfqpoint{1.360267in}{2.350281in}}{\pgfqpoint{1.366091in}{2.344457in}}%
\pgfpathcurveto{\pgfqpoint{1.371915in}{2.338633in}}{\pgfqpoint{1.379815in}{2.335360in}}{\pgfqpoint{1.388051in}{2.335360in}}%
\pgfpathclose%
\pgfusepath{stroke,fill}%
\end{pgfscope}%
\begin{pgfscope}%
\pgfpathrectangle{\pgfqpoint{0.100000in}{0.212622in}}{\pgfqpoint{3.696000in}{3.696000in}}%
\pgfusepath{clip}%
\pgfsetbuttcap%
\pgfsetroundjoin%
\definecolor{currentfill}{rgb}{0.121569,0.466667,0.705882}%
\pgfsetfillcolor{currentfill}%
\pgfsetfillopacity{0.691860}%
\pgfsetlinewidth{1.003750pt}%
\definecolor{currentstroke}{rgb}{0.121569,0.466667,0.705882}%
\pgfsetstrokecolor{currentstroke}%
\pgfsetstrokeopacity{0.691860}%
\pgfsetdash{}{0pt}%
\pgfpathmoveto{\pgfqpoint{1.384813in}{2.331028in}}%
\pgfpathcurveto{\pgfqpoint{1.393049in}{2.331028in}}{\pgfqpoint{1.400949in}{2.334300in}}{\pgfqpoint{1.406773in}{2.340124in}}%
\pgfpathcurveto{\pgfqpoint{1.412597in}{2.345948in}}{\pgfqpoint{1.415870in}{2.353848in}}{\pgfqpoint{1.415870in}{2.362084in}}%
\pgfpathcurveto{\pgfqpoint{1.415870in}{2.370321in}}{\pgfqpoint{1.412597in}{2.378221in}}{\pgfqpoint{1.406773in}{2.384045in}}%
\pgfpathcurveto{\pgfqpoint{1.400949in}{2.389869in}}{\pgfqpoint{1.393049in}{2.393141in}}{\pgfqpoint{1.384813in}{2.393141in}}%
\pgfpathcurveto{\pgfqpoint{1.376577in}{2.393141in}}{\pgfqpoint{1.368677in}{2.389869in}}{\pgfqpoint{1.362853in}{2.384045in}}%
\pgfpathcurveto{\pgfqpoint{1.357029in}{2.378221in}}{\pgfqpoint{1.353757in}{2.370321in}}{\pgfqpoint{1.353757in}{2.362084in}}%
\pgfpathcurveto{\pgfqpoint{1.353757in}{2.353848in}}{\pgfqpoint{1.357029in}{2.345948in}}{\pgfqpoint{1.362853in}{2.340124in}}%
\pgfpathcurveto{\pgfqpoint{1.368677in}{2.334300in}}{\pgfqpoint{1.376577in}{2.331028in}}{\pgfqpoint{1.384813in}{2.331028in}}%
\pgfpathclose%
\pgfusepath{stroke,fill}%
\end{pgfscope}%
\begin{pgfscope}%
\pgfpathrectangle{\pgfqpoint{0.100000in}{0.212622in}}{\pgfqpoint{3.696000in}{3.696000in}}%
\pgfusepath{clip}%
\pgfsetbuttcap%
\pgfsetroundjoin%
\definecolor{currentfill}{rgb}{0.121569,0.466667,0.705882}%
\pgfsetfillcolor{currentfill}%
\pgfsetfillopacity{0.692215}%
\pgfsetlinewidth{1.003750pt}%
\definecolor{currentstroke}{rgb}{0.121569,0.466667,0.705882}%
\pgfsetstrokecolor{currentstroke}%
\pgfsetstrokeopacity{0.692215}%
\pgfsetdash{}{0pt}%
\pgfpathmoveto{\pgfqpoint{3.301453in}{2.609928in}}%
\pgfpathcurveto{\pgfqpoint{3.309690in}{2.609928in}}{\pgfqpoint{3.317590in}{2.613200in}}{\pgfqpoint{3.323414in}{2.619024in}}%
\pgfpathcurveto{\pgfqpoint{3.329237in}{2.624848in}}{\pgfqpoint{3.332510in}{2.632748in}}{\pgfqpoint{3.332510in}{2.640984in}}%
\pgfpathcurveto{\pgfqpoint{3.332510in}{2.649220in}}{\pgfqpoint{3.329237in}{2.657120in}}{\pgfqpoint{3.323414in}{2.662944in}}%
\pgfpathcurveto{\pgfqpoint{3.317590in}{2.668768in}}{\pgfqpoint{3.309690in}{2.672041in}}{\pgfqpoint{3.301453in}{2.672041in}}%
\pgfpathcurveto{\pgfqpoint{3.293217in}{2.672041in}}{\pgfqpoint{3.285317in}{2.668768in}}{\pgfqpoint{3.279493in}{2.662944in}}%
\pgfpathcurveto{\pgfqpoint{3.273669in}{2.657120in}}{\pgfqpoint{3.270397in}{2.649220in}}{\pgfqpoint{3.270397in}{2.640984in}}%
\pgfpathcurveto{\pgfqpoint{3.270397in}{2.632748in}}{\pgfqpoint{3.273669in}{2.624848in}}{\pgfqpoint{3.279493in}{2.619024in}}%
\pgfpathcurveto{\pgfqpoint{3.285317in}{2.613200in}}{\pgfqpoint{3.293217in}{2.609928in}}{\pgfqpoint{3.301453in}{2.609928in}}%
\pgfpathclose%
\pgfusepath{stroke,fill}%
\end{pgfscope}%
\begin{pgfscope}%
\pgfpathrectangle{\pgfqpoint{0.100000in}{0.212622in}}{\pgfqpoint{3.696000in}{3.696000in}}%
\pgfusepath{clip}%
\pgfsetbuttcap%
\pgfsetroundjoin%
\definecolor{currentfill}{rgb}{0.121569,0.466667,0.705882}%
\pgfsetfillcolor{currentfill}%
\pgfsetfillopacity{0.692666}%
\pgfsetlinewidth{1.003750pt}%
\definecolor{currentstroke}{rgb}{0.121569,0.466667,0.705882}%
\pgfsetstrokecolor{currentstroke}%
\pgfsetstrokeopacity{0.692666}%
\pgfsetdash{}{0pt}%
\pgfpathmoveto{\pgfqpoint{2.163411in}{3.007566in}}%
\pgfpathcurveto{\pgfqpoint{2.171648in}{3.007566in}}{\pgfqpoint{2.179548in}{3.010838in}}{\pgfqpoint{2.185372in}{3.016662in}}%
\pgfpathcurveto{\pgfqpoint{2.191196in}{3.022486in}}{\pgfqpoint{2.194468in}{3.030386in}}{\pgfqpoint{2.194468in}{3.038622in}}%
\pgfpathcurveto{\pgfqpoint{2.194468in}{3.046859in}}{\pgfqpoint{2.191196in}{3.054759in}}{\pgfqpoint{2.185372in}{3.060583in}}%
\pgfpathcurveto{\pgfqpoint{2.179548in}{3.066406in}}{\pgfqpoint{2.171648in}{3.069679in}}{\pgfqpoint{2.163411in}{3.069679in}}%
\pgfpathcurveto{\pgfqpoint{2.155175in}{3.069679in}}{\pgfqpoint{2.147275in}{3.066406in}}{\pgfqpoint{2.141451in}{3.060583in}}%
\pgfpathcurveto{\pgfqpoint{2.135627in}{3.054759in}}{\pgfqpoint{2.132355in}{3.046859in}}{\pgfqpoint{2.132355in}{3.038622in}}%
\pgfpathcurveto{\pgfqpoint{2.132355in}{3.030386in}}{\pgfqpoint{2.135627in}{3.022486in}}{\pgfqpoint{2.141451in}{3.016662in}}%
\pgfpathcurveto{\pgfqpoint{2.147275in}{3.010838in}}{\pgfqpoint{2.155175in}{3.007566in}}{\pgfqpoint{2.163411in}{3.007566in}}%
\pgfpathclose%
\pgfusepath{stroke,fill}%
\end{pgfscope}%
\begin{pgfscope}%
\pgfpathrectangle{\pgfqpoint{0.100000in}{0.212622in}}{\pgfqpoint{3.696000in}{3.696000in}}%
\pgfusepath{clip}%
\pgfsetbuttcap%
\pgfsetroundjoin%
\definecolor{currentfill}{rgb}{0.121569,0.466667,0.705882}%
\pgfsetfillcolor{currentfill}%
\pgfsetfillopacity{0.692790}%
\pgfsetlinewidth{1.003750pt}%
\definecolor{currentstroke}{rgb}{0.121569,0.466667,0.705882}%
\pgfsetstrokecolor{currentstroke}%
\pgfsetstrokeopacity{0.692790}%
\pgfsetdash{}{0pt}%
\pgfpathmoveto{\pgfqpoint{1.381266in}{2.323553in}}%
\pgfpathcurveto{\pgfqpoint{1.389502in}{2.323553in}}{\pgfqpoint{1.397402in}{2.326825in}}{\pgfqpoint{1.403226in}{2.332649in}}%
\pgfpathcurveto{\pgfqpoint{1.409050in}{2.338473in}}{\pgfqpoint{1.412322in}{2.346373in}}{\pgfqpoint{1.412322in}{2.354609in}}%
\pgfpathcurveto{\pgfqpoint{1.412322in}{2.362845in}}{\pgfqpoint{1.409050in}{2.370745in}}{\pgfqpoint{1.403226in}{2.376569in}}%
\pgfpathcurveto{\pgfqpoint{1.397402in}{2.382393in}}{\pgfqpoint{1.389502in}{2.385666in}}{\pgfqpoint{1.381266in}{2.385666in}}%
\pgfpathcurveto{\pgfqpoint{1.373030in}{2.385666in}}{\pgfqpoint{1.365130in}{2.382393in}}{\pgfqpoint{1.359306in}{2.376569in}}%
\pgfpathcurveto{\pgfqpoint{1.353482in}{2.370745in}}{\pgfqpoint{1.350209in}{2.362845in}}{\pgfqpoint{1.350209in}{2.354609in}}%
\pgfpathcurveto{\pgfqpoint{1.350209in}{2.346373in}}{\pgfqpoint{1.353482in}{2.338473in}}{\pgfqpoint{1.359306in}{2.332649in}}%
\pgfpathcurveto{\pgfqpoint{1.365130in}{2.326825in}}{\pgfqpoint{1.373030in}{2.323553in}}{\pgfqpoint{1.381266in}{2.323553in}}%
\pgfpathclose%
\pgfusepath{stroke,fill}%
\end{pgfscope}%
\begin{pgfscope}%
\pgfpathrectangle{\pgfqpoint{0.100000in}{0.212622in}}{\pgfqpoint{3.696000in}{3.696000in}}%
\pgfusepath{clip}%
\pgfsetbuttcap%
\pgfsetroundjoin%
\definecolor{currentfill}{rgb}{0.121569,0.466667,0.705882}%
\pgfsetfillcolor{currentfill}%
\pgfsetfillopacity{0.693403}%
\pgfsetlinewidth{1.003750pt}%
\definecolor{currentstroke}{rgb}{0.121569,0.466667,0.705882}%
\pgfsetstrokecolor{currentstroke}%
\pgfsetstrokeopacity{0.693403}%
\pgfsetdash{}{0pt}%
\pgfpathmoveto{\pgfqpoint{3.305735in}{2.610149in}}%
\pgfpathcurveto{\pgfqpoint{3.313971in}{2.610149in}}{\pgfqpoint{3.321871in}{2.613421in}}{\pgfqpoint{3.327695in}{2.619245in}}%
\pgfpathcurveto{\pgfqpoint{3.333519in}{2.625069in}}{\pgfqpoint{3.336791in}{2.632969in}}{\pgfqpoint{3.336791in}{2.641205in}}%
\pgfpathcurveto{\pgfqpoint{3.336791in}{2.649442in}}{\pgfqpoint{3.333519in}{2.657342in}}{\pgfqpoint{3.327695in}{2.663166in}}%
\pgfpathcurveto{\pgfqpoint{3.321871in}{2.668990in}}{\pgfqpoint{3.313971in}{2.672262in}}{\pgfqpoint{3.305735in}{2.672262in}}%
\pgfpathcurveto{\pgfqpoint{3.297499in}{2.672262in}}{\pgfqpoint{3.289599in}{2.668990in}}{\pgfqpoint{3.283775in}{2.663166in}}%
\pgfpathcurveto{\pgfqpoint{3.277951in}{2.657342in}}{\pgfqpoint{3.274678in}{2.649442in}}{\pgfqpoint{3.274678in}{2.641205in}}%
\pgfpathcurveto{\pgfqpoint{3.274678in}{2.632969in}}{\pgfqpoint{3.277951in}{2.625069in}}{\pgfqpoint{3.283775in}{2.619245in}}%
\pgfpathcurveto{\pgfqpoint{3.289599in}{2.613421in}}{\pgfqpoint{3.297499in}{2.610149in}}{\pgfqpoint{3.305735in}{2.610149in}}%
\pgfpathclose%
\pgfusepath{stroke,fill}%
\end{pgfscope}%
\begin{pgfscope}%
\pgfpathrectangle{\pgfqpoint{0.100000in}{0.212622in}}{\pgfqpoint{3.696000in}{3.696000in}}%
\pgfusepath{clip}%
\pgfsetbuttcap%
\pgfsetroundjoin%
\definecolor{currentfill}{rgb}{0.121569,0.466667,0.705882}%
\pgfsetfillcolor{currentfill}%
\pgfsetfillopacity{0.693840}%
\pgfsetlinewidth{1.003750pt}%
\definecolor{currentstroke}{rgb}{0.121569,0.466667,0.705882}%
\pgfsetstrokecolor{currentstroke}%
\pgfsetstrokeopacity{0.693840}%
\pgfsetdash{}{0pt}%
\pgfpathmoveto{\pgfqpoint{1.378401in}{2.313819in}}%
\pgfpathcurveto{\pgfqpoint{1.386637in}{2.313819in}}{\pgfqpoint{1.394537in}{2.317091in}}{\pgfqpoint{1.400361in}{2.322915in}}%
\pgfpathcurveto{\pgfqpoint{1.406185in}{2.328739in}}{\pgfqpoint{1.409458in}{2.336639in}}{\pgfqpoint{1.409458in}{2.344876in}}%
\pgfpathcurveto{\pgfqpoint{1.409458in}{2.353112in}}{\pgfqpoint{1.406185in}{2.361012in}}{\pgfqpoint{1.400361in}{2.366836in}}%
\pgfpathcurveto{\pgfqpoint{1.394537in}{2.372660in}}{\pgfqpoint{1.386637in}{2.375932in}}{\pgfqpoint{1.378401in}{2.375932in}}%
\pgfpathcurveto{\pgfqpoint{1.370165in}{2.375932in}}{\pgfqpoint{1.362265in}{2.372660in}}{\pgfqpoint{1.356441in}{2.366836in}}%
\pgfpathcurveto{\pgfqpoint{1.350617in}{2.361012in}}{\pgfqpoint{1.347345in}{2.353112in}}{\pgfqpoint{1.347345in}{2.344876in}}%
\pgfpathcurveto{\pgfqpoint{1.347345in}{2.336639in}}{\pgfqpoint{1.350617in}{2.328739in}}{\pgfqpoint{1.356441in}{2.322915in}}%
\pgfpathcurveto{\pgfqpoint{1.362265in}{2.317091in}}{\pgfqpoint{1.370165in}{2.313819in}}{\pgfqpoint{1.378401in}{2.313819in}}%
\pgfpathclose%
\pgfusepath{stroke,fill}%
\end{pgfscope}%
\begin{pgfscope}%
\pgfpathrectangle{\pgfqpoint{0.100000in}{0.212622in}}{\pgfqpoint{3.696000in}{3.696000in}}%
\pgfusepath{clip}%
\pgfsetbuttcap%
\pgfsetroundjoin%
\definecolor{currentfill}{rgb}{0.121569,0.466667,0.705882}%
\pgfsetfillcolor{currentfill}%
\pgfsetfillopacity{0.694180}%
\pgfsetlinewidth{1.003750pt}%
\definecolor{currentstroke}{rgb}{0.121569,0.466667,0.705882}%
\pgfsetstrokecolor{currentstroke}%
\pgfsetstrokeopacity{0.694180}%
\pgfsetdash{}{0pt}%
\pgfpathmoveto{\pgfqpoint{3.308794in}{2.609846in}}%
\pgfpathcurveto{\pgfqpoint{3.317030in}{2.609846in}}{\pgfqpoint{3.324930in}{2.613118in}}{\pgfqpoint{3.330754in}{2.618942in}}%
\pgfpathcurveto{\pgfqpoint{3.336578in}{2.624766in}}{\pgfqpoint{3.339850in}{2.632666in}}{\pgfqpoint{3.339850in}{2.640902in}}%
\pgfpathcurveto{\pgfqpoint{3.339850in}{2.649139in}}{\pgfqpoint{3.336578in}{2.657039in}}{\pgfqpoint{3.330754in}{2.662863in}}%
\pgfpathcurveto{\pgfqpoint{3.324930in}{2.668687in}}{\pgfqpoint{3.317030in}{2.671959in}}{\pgfqpoint{3.308794in}{2.671959in}}%
\pgfpathcurveto{\pgfqpoint{3.300558in}{2.671959in}}{\pgfqpoint{3.292658in}{2.668687in}}{\pgfqpoint{3.286834in}{2.662863in}}%
\pgfpathcurveto{\pgfqpoint{3.281010in}{2.657039in}}{\pgfqpoint{3.277737in}{2.649139in}}{\pgfqpoint{3.277737in}{2.640902in}}%
\pgfpathcurveto{\pgfqpoint{3.277737in}{2.632666in}}{\pgfqpoint{3.281010in}{2.624766in}}{\pgfqpoint{3.286834in}{2.618942in}}%
\pgfpathcurveto{\pgfqpoint{3.292658in}{2.613118in}}{\pgfqpoint{3.300558in}{2.609846in}}{\pgfqpoint{3.308794in}{2.609846in}}%
\pgfpathclose%
\pgfusepath{stroke,fill}%
\end{pgfscope}%
\begin{pgfscope}%
\pgfpathrectangle{\pgfqpoint{0.100000in}{0.212622in}}{\pgfqpoint{3.696000in}{3.696000in}}%
\pgfusepath{clip}%
\pgfsetbuttcap%
\pgfsetroundjoin%
\definecolor{currentfill}{rgb}{0.121569,0.466667,0.705882}%
\pgfsetfillcolor{currentfill}%
\pgfsetfillopacity{0.694344}%
\pgfsetlinewidth{1.003750pt}%
\definecolor{currentstroke}{rgb}{0.121569,0.466667,0.705882}%
\pgfsetstrokecolor{currentstroke}%
\pgfsetstrokeopacity{0.694344}%
\pgfsetdash{}{0pt}%
\pgfpathmoveto{\pgfqpoint{2.173233in}{3.005334in}}%
\pgfpathcurveto{\pgfqpoint{2.181470in}{3.005334in}}{\pgfqpoint{2.189370in}{3.008607in}}{\pgfqpoint{2.195194in}{3.014431in}}%
\pgfpathcurveto{\pgfqpoint{2.201018in}{3.020255in}}{\pgfqpoint{2.204290in}{3.028155in}}{\pgfqpoint{2.204290in}{3.036391in}}%
\pgfpathcurveto{\pgfqpoint{2.204290in}{3.044627in}}{\pgfqpoint{2.201018in}{3.052527in}}{\pgfqpoint{2.195194in}{3.058351in}}%
\pgfpathcurveto{\pgfqpoint{2.189370in}{3.064175in}}{\pgfqpoint{2.181470in}{3.067447in}}{\pgfqpoint{2.173233in}{3.067447in}}%
\pgfpathcurveto{\pgfqpoint{2.164997in}{3.067447in}}{\pgfqpoint{2.157097in}{3.064175in}}{\pgfqpoint{2.151273in}{3.058351in}}%
\pgfpathcurveto{\pgfqpoint{2.145449in}{3.052527in}}{\pgfqpoint{2.142177in}{3.044627in}}{\pgfqpoint{2.142177in}{3.036391in}}%
\pgfpathcurveto{\pgfqpoint{2.142177in}{3.028155in}}{\pgfqpoint{2.145449in}{3.020255in}}{\pgfqpoint{2.151273in}{3.014431in}}%
\pgfpathcurveto{\pgfqpoint{2.157097in}{3.008607in}}{\pgfqpoint{2.164997in}{3.005334in}}{\pgfqpoint{2.173233in}{3.005334in}}%
\pgfpathclose%
\pgfusepath{stroke,fill}%
\end{pgfscope}%
\begin{pgfscope}%
\pgfpathrectangle{\pgfqpoint{0.100000in}{0.212622in}}{\pgfqpoint{3.696000in}{3.696000in}}%
\pgfusepath{clip}%
\pgfsetbuttcap%
\pgfsetroundjoin%
\definecolor{currentfill}{rgb}{0.121569,0.466667,0.705882}%
\pgfsetfillcolor{currentfill}%
\pgfsetfillopacity{0.694417}%
\pgfsetlinewidth{1.003750pt}%
\definecolor{currentstroke}{rgb}{0.121569,0.466667,0.705882}%
\pgfsetstrokecolor{currentstroke}%
\pgfsetstrokeopacity{0.694417}%
\pgfsetdash{}{0pt}%
\pgfpathmoveto{\pgfqpoint{1.376375in}{2.308762in}}%
\pgfpathcurveto{\pgfqpoint{1.384612in}{2.308762in}}{\pgfqpoint{1.392512in}{2.312034in}}{\pgfqpoint{1.398335in}{2.317858in}}%
\pgfpathcurveto{\pgfqpoint{1.404159in}{2.323682in}}{\pgfqpoint{1.407432in}{2.331582in}}{\pgfqpoint{1.407432in}{2.339818in}}%
\pgfpathcurveto{\pgfqpoint{1.407432in}{2.348055in}}{\pgfqpoint{1.404159in}{2.355955in}}{\pgfqpoint{1.398335in}{2.361779in}}%
\pgfpathcurveto{\pgfqpoint{1.392512in}{2.367603in}}{\pgfqpoint{1.384612in}{2.370875in}}{\pgfqpoint{1.376375in}{2.370875in}}%
\pgfpathcurveto{\pgfqpoint{1.368139in}{2.370875in}}{\pgfqpoint{1.360239in}{2.367603in}}{\pgfqpoint{1.354415in}{2.361779in}}%
\pgfpathcurveto{\pgfqpoint{1.348591in}{2.355955in}}{\pgfqpoint{1.345319in}{2.348055in}}{\pgfqpoint{1.345319in}{2.339818in}}%
\pgfpathcurveto{\pgfqpoint{1.345319in}{2.331582in}}{\pgfqpoint{1.348591in}{2.323682in}}{\pgfqpoint{1.354415in}{2.317858in}}%
\pgfpathcurveto{\pgfqpoint{1.360239in}{2.312034in}}{\pgfqpoint{1.368139in}{2.308762in}}{\pgfqpoint{1.376375in}{2.308762in}}%
\pgfpathclose%
\pgfusepath{stroke,fill}%
\end{pgfscope}%
\begin{pgfscope}%
\pgfpathrectangle{\pgfqpoint{0.100000in}{0.212622in}}{\pgfqpoint{3.696000in}{3.696000in}}%
\pgfusepath{clip}%
\pgfsetbuttcap%
\pgfsetroundjoin%
\definecolor{currentfill}{rgb}{0.121569,0.466667,0.705882}%
\pgfsetfillcolor{currentfill}%
\pgfsetfillopacity{0.694819}%
\pgfsetlinewidth{1.003750pt}%
\definecolor{currentstroke}{rgb}{0.121569,0.466667,0.705882}%
\pgfsetstrokecolor{currentstroke}%
\pgfsetstrokeopacity{0.694819}%
\pgfsetdash{}{0pt}%
\pgfpathmoveto{\pgfqpoint{3.310528in}{2.609031in}}%
\pgfpathcurveto{\pgfqpoint{3.318764in}{2.609031in}}{\pgfqpoint{3.326664in}{2.612303in}}{\pgfqpoint{3.332488in}{2.618127in}}%
\pgfpathcurveto{\pgfqpoint{3.338312in}{2.623951in}}{\pgfqpoint{3.341584in}{2.631851in}}{\pgfqpoint{3.341584in}{2.640087in}}%
\pgfpathcurveto{\pgfqpoint{3.341584in}{2.648323in}}{\pgfqpoint{3.338312in}{2.656223in}}{\pgfqpoint{3.332488in}{2.662047in}}%
\pgfpathcurveto{\pgfqpoint{3.326664in}{2.667871in}}{\pgfqpoint{3.318764in}{2.671144in}}{\pgfqpoint{3.310528in}{2.671144in}}%
\pgfpathcurveto{\pgfqpoint{3.302292in}{2.671144in}}{\pgfqpoint{3.294392in}{2.667871in}}{\pgfqpoint{3.288568in}{2.662047in}}%
\pgfpathcurveto{\pgfqpoint{3.282744in}{2.656223in}}{\pgfqpoint{3.279471in}{2.648323in}}{\pgfqpoint{3.279471in}{2.640087in}}%
\pgfpathcurveto{\pgfqpoint{3.279471in}{2.631851in}}{\pgfqpoint{3.282744in}{2.623951in}}{\pgfqpoint{3.288568in}{2.618127in}}%
\pgfpathcurveto{\pgfqpoint{3.294392in}{2.612303in}}{\pgfqpoint{3.302292in}{2.609031in}}{\pgfqpoint{3.310528in}{2.609031in}}%
\pgfpathclose%
\pgfusepath{stroke,fill}%
\end{pgfscope}%
\begin{pgfscope}%
\pgfpathrectangle{\pgfqpoint{0.100000in}{0.212622in}}{\pgfqpoint{3.696000in}{3.696000in}}%
\pgfusepath{clip}%
\pgfsetbuttcap%
\pgfsetroundjoin%
\definecolor{currentfill}{rgb}{0.121569,0.466667,0.705882}%
\pgfsetfillcolor{currentfill}%
\pgfsetfillopacity{0.694949}%
\pgfsetlinewidth{1.003750pt}%
\definecolor{currentstroke}{rgb}{0.121569,0.466667,0.705882}%
\pgfsetstrokecolor{currentstroke}%
\pgfsetstrokeopacity{0.694949}%
\pgfsetdash{}{0pt}%
\pgfpathmoveto{\pgfqpoint{1.373219in}{2.304160in}}%
\pgfpathcurveto{\pgfqpoint{1.381455in}{2.304160in}}{\pgfqpoint{1.389355in}{2.307432in}}{\pgfqpoint{1.395179in}{2.313256in}}%
\pgfpathcurveto{\pgfqpoint{1.401003in}{2.319080in}}{\pgfqpoint{1.404276in}{2.326980in}}{\pgfqpoint{1.404276in}{2.335216in}}%
\pgfpathcurveto{\pgfqpoint{1.404276in}{2.343453in}}{\pgfqpoint{1.401003in}{2.351353in}}{\pgfqpoint{1.395179in}{2.357177in}}%
\pgfpathcurveto{\pgfqpoint{1.389355in}{2.363000in}}{\pgfqpoint{1.381455in}{2.366273in}}{\pgfqpoint{1.373219in}{2.366273in}}%
\pgfpathcurveto{\pgfqpoint{1.364983in}{2.366273in}}{\pgfqpoint{1.357083in}{2.363000in}}{\pgfqpoint{1.351259in}{2.357177in}}%
\pgfpathcurveto{\pgfqpoint{1.345435in}{2.351353in}}{\pgfqpoint{1.342163in}{2.343453in}}{\pgfqpoint{1.342163in}{2.335216in}}%
\pgfpathcurveto{\pgfqpoint{1.342163in}{2.326980in}}{\pgfqpoint{1.345435in}{2.319080in}}{\pgfqpoint{1.351259in}{2.313256in}}%
\pgfpathcurveto{\pgfqpoint{1.357083in}{2.307432in}}{\pgfqpoint{1.364983in}{2.304160in}}{\pgfqpoint{1.373219in}{2.304160in}}%
\pgfpathclose%
\pgfusepath{stroke,fill}%
\end{pgfscope}%
\begin{pgfscope}%
\pgfpathrectangle{\pgfqpoint{0.100000in}{0.212622in}}{\pgfqpoint{3.696000in}{3.696000in}}%
\pgfusepath{clip}%
\pgfsetbuttcap%
\pgfsetroundjoin%
\definecolor{currentfill}{rgb}{0.121569,0.466667,0.705882}%
\pgfsetfillcolor{currentfill}%
\pgfsetfillopacity{0.695262}%
\pgfsetlinewidth{1.003750pt}%
\definecolor{currentstroke}{rgb}{0.121569,0.466667,0.705882}%
\pgfsetstrokecolor{currentstroke}%
\pgfsetstrokeopacity{0.695262}%
\pgfsetdash{}{0pt}%
\pgfpathmoveto{\pgfqpoint{3.312064in}{2.608796in}}%
\pgfpathcurveto{\pgfqpoint{3.320300in}{2.608796in}}{\pgfqpoint{3.328200in}{2.612069in}}{\pgfqpoint{3.334024in}{2.617893in}}%
\pgfpathcurveto{\pgfqpoint{3.339848in}{2.623717in}}{\pgfqpoint{3.343120in}{2.631617in}}{\pgfqpoint{3.343120in}{2.639853in}}%
\pgfpathcurveto{\pgfqpoint{3.343120in}{2.648089in}}{\pgfqpoint{3.339848in}{2.655989in}}{\pgfqpoint{3.334024in}{2.661813in}}%
\pgfpathcurveto{\pgfqpoint{3.328200in}{2.667637in}}{\pgfqpoint{3.320300in}{2.670909in}}{\pgfqpoint{3.312064in}{2.670909in}}%
\pgfpathcurveto{\pgfqpoint{3.303828in}{2.670909in}}{\pgfqpoint{3.295928in}{2.667637in}}{\pgfqpoint{3.290104in}{2.661813in}}%
\pgfpathcurveto{\pgfqpoint{3.284280in}{2.655989in}}{\pgfqpoint{3.281007in}{2.648089in}}{\pgfqpoint{3.281007in}{2.639853in}}%
\pgfpathcurveto{\pgfqpoint{3.281007in}{2.631617in}}{\pgfqpoint{3.284280in}{2.623717in}}{\pgfqpoint{3.290104in}{2.617893in}}%
\pgfpathcurveto{\pgfqpoint{3.295928in}{2.612069in}}{\pgfqpoint{3.303828in}{2.608796in}}{\pgfqpoint{3.312064in}{2.608796in}}%
\pgfpathclose%
\pgfusepath{stroke,fill}%
\end{pgfscope}%
\begin{pgfscope}%
\pgfpathrectangle{\pgfqpoint{0.100000in}{0.212622in}}{\pgfqpoint{3.696000in}{3.696000in}}%
\pgfusepath{clip}%
\pgfsetbuttcap%
\pgfsetroundjoin%
\definecolor{currentfill}{rgb}{0.121569,0.466667,0.705882}%
\pgfsetfillcolor{currentfill}%
\pgfsetfillopacity{0.695421}%
\pgfsetlinewidth{1.003750pt}%
\definecolor{currentstroke}{rgb}{0.121569,0.466667,0.705882}%
\pgfsetstrokecolor{currentstroke}%
\pgfsetstrokeopacity{0.695421}%
\pgfsetdash{}{0pt}%
\pgfpathmoveto{\pgfqpoint{3.312692in}{2.608783in}}%
\pgfpathcurveto{\pgfqpoint{3.320928in}{2.608783in}}{\pgfqpoint{3.328828in}{2.612055in}}{\pgfqpoint{3.334652in}{2.617879in}}%
\pgfpathcurveto{\pgfqpoint{3.340476in}{2.623703in}}{\pgfqpoint{3.343748in}{2.631603in}}{\pgfqpoint{3.343748in}{2.639840in}}%
\pgfpathcurveto{\pgfqpoint{3.343748in}{2.648076in}}{\pgfqpoint{3.340476in}{2.655976in}}{\pgfqpoint{3.334652in}{2.661800in}}%
\pgfpathcurveto{\pgfqpoint{3.328828in}{2.667624in}}{\pgfqpoint{3.320928in}{2.670896in}}{\pgfqpoint{3.312692in}{2.670896in}}%
\pgfpathcurveto{\pgfqpoint{3.304456in}{2.670896in}}{\pgfqpoint{3.296556in}{2.667624in}}{\pgfqpoint{3.290732in}{2.661800in}}%
\pgfpathcurveto{\pgfqpoint{3.284908in}{2.655976in}}{\pgfqpoint{3.281635in}{2.648076in}}{\pgfqpoint{3.281635in}{2.639840in}}%
\pgfpathcurveto{\pgfqpoint{3.281635in}{2.631603in}}{\pgfqpoint{3.284908in}{2.623703in}}{\pgfqpoint{3.290732in}{2.617879in}}%
\pgfpathcurveto{\pgfqpoint{3.296556in}{2.612055in}}{\pgfqpoint{3.304456in}{2.608783in}}{\pgfqpoint{3.312692in}{2.608783in}}%
\pgfpathclose%
\pgfusepath{stroke,fill}%
\end{pgfscope}%
\begin{pgfscope}%
\pgfpathrectangle{\pgfqpoint{0.100000in}{0.212622in}}{\pgfqpoint{3.696000in}{3.696000in}}%
\pgfusepath{clip}%
\pgfsetbuttcap%
\pgfsetroundjoin%
\definecolor{currentfill}{rgb}{0.121569,0.466667,0.705882}%
\pgfsetfillcolor{currentfill}%
\pgfsetfillopacity{0.695421}%
\pgfsetlinewidth{1.003750pt}%
\definecolor{currentstroke}{rgb}{0.121569,0.466667,0.705882}%
\pgfsetstrokecolor{currentstroke}%
\pgfsetstrokeopacity{0.695421}%
\pgfsetdash{}{0pt}%
\pgfpathmoveto{\pgfqpoint{3.312692in}{2.608783in}}%
\pgfpathcurveto{\pgfqpoint{3.320928in}{2.608783in}}{\pgfqpoint{3.328828in}{2.612055in}}{\pgfqpoint{3.334652in}{2.617879in}}%
\pgfpathcurveto{\pgfqpoint{3.340476in}{2.623703in}}{\pgfqpoint{3.343748in}{2.631603in}}{\pgfqpoint{3.343748in}{2.639840in}}%
\pgfpathcurveto{\pgfqpoint{3.343748in}{2.648076in}}{\pgfqpoint{3.340476in}{2.655976in}}{\pgfqpoint{3.334652in}{2.661800in}}%
\pgfpathcurveto{\pgfqpoint{3.328828in}{2.667624in}}{\pgfqpoint{3.320928in}{2.670896in}}{\pgfqpoint{3.312692in}{2.670896in}}%
\pgfpathcurveto{\pgfqpoint{3.304456in}{2.670896in}}{\pgfqpoint{3.296556in}{2.667624in}}{\pgfqpoint{3.290732in}{2.661800in}}%
\pgfpathcurveto{\pgfqpoint{3.284908in}{2.655976in}}{\pgfqpoint{3.281635in}{2.648076in}}{\pgfqpoint{3.281635in}{2.639840in}}%
\pgfpathcurveto{\pgfqpoint{3.281635in}{2.631603in}}{\pgfqpoint{3.284908in}{2.623703in}}{\pgfqpoint{3.290732in}{2.617879in}}%
\pgfpathcurveto{\pgfqpoint{3.296556in}{2.612055in}}{\pgfqpoint{3.304456in}{2.608783in}}{\pgfqpoint{3.312692in}{2.608783in}}%
\pgfpathclose%
\pgfusepath{stroke,fill}%
\end{pgfscope}%
\begin{pgfscope}%
\pgfpathrectangle{\pgfqpoint{0.100000in}{0.212622in}}{\pgfqpoint{3.696000in}{3.696000in}}%
\pgfusepath{clip}%
\pgfsetbuttcap%
\pgfsetroundjoin%
\definecolor{currentfill}{rgb}{0.121569,0.466667,0.705882}%
\pgfsetfillcolor{currentfill}%
\pgfsetfillopacity{0.695421}%
\pgfsetlinewidth{1.003750pt}%
\definecolor{currentstroke}{rgb}{0.121569,0.466667,0.705882}%
\pgfsetstrokecolor{currentstroke}%
\pgfsetstrokeopacity{0.695421}%
\pgfsetdash{}{0pt}%
\pgfpathmoveto{\pgfqpoint{3.312692in}{2.608783in}}%
\pgfpathcurveto{\pgfqpoint{3.320928in}{2.608783in}}{\pgfqpoint{3.328828in}{2.612055in}}{\pgfqpoint{3.334652in}{2.617879in}}%
\pgfpathcurveto{\pgfqpoint{3.340476in}{2.623703in}}{\pgfqpoint{3.343748in}{2.631603in}}{\pgfqpoint{3.343748in}{2.639840in}}%
\pgfpathcurveto{\pgfqpoint{3.343748in}{2.648076in}}{\pgfqpoint{3.340476in}{2.655976in}}{\pgfqpoint{3.334652in}{2.661800in}}%
\pgfpathcurveto{\pgfqpoint{3.328828in}{2.667624in}}{\pgfqpoint{3.320928in}{2.670896in}}{\pgfqpoint{3.312692in}{2.670896in}}%
\pgfpathcurveto{\pgfqpoint{3.304456in}{2.670896in}}{\pgfqpoint{3.296556in}{2.667624in}}{\pgfqpoint{3.290732in}{2.661800in}}%
\pgfpathcurveto{\pgfqpoint{3.284908in}{2.655976in}}{\pgfqpoint{3.281635in}{2.648076in}}{\pgfqpoint{3.281635in}{2.639840in}}%
\pgfpathcurveto{\pgfqpoint{3.281635in}{2.631603in}}{\pgfqpoint{3.284908in}{2.623703in}}{\pgfqpoint{3.290732in}{2.617879in}}%
\pgfpathcurveto{\pgfqpoint{3.296556in}{2.612055in}}{\pgfqpoint{3.304456in}{2.608783in}}{\pgfqpoint{3.312692in}{2.608783in}}%
\pgfpathclose%
\pgfusepath{stroke,fill}%
\end{pgfscope}%
\begin{pgfscope}%
\pgfpathrectangle{\pgfqpoint{0.100000in}{0.212622in}}{\pgfqpoint{3.696000in}{3.696000in}}%
\pgfusepath{clip}%
\pgfsetbuttcap%
\pgfsetroundjoin%
\definecolor{currentfill}{rgb}{0.121569,0.466667,0.705882}%
\pgfsetfillcolor{currentfill}%
\pgfsetfillopacity{0.695421}%
\pgfsetlinewidth{1.003750pt}%
\definecolor{currentstroke}{rgb}{0.121569,0.466667,0.705882}%
\pgfsetstrokecolor{currentstroke}%
\pgfsetstrokeopacity{0.695421}%
\pgfsetdash{}{0pt}%
\pgfpathmoveto{\pgfqpoint{3.312692in}{2.608783in}}%
\pgfpathcurveto{\pgfqpoint{3.320928in}{2.608783in}}{\pgfqpoint{3.328828in}{2.612055in}}{\pgfqpoint{3.334652in}{2.617879in}}%
\pgfpathcurveto{\pgfqpoint{3.340476in}{2.623703in}}{\pgfqpoint{3.343748in}{2.631603in}}{\pgfqpoint{3.343748in}{2.639840in}}%
\pgfpathcurveto{\pgfqpoint{3.343748in}{2.648076in}}{\pgfqpoint{3.340476in}{2.655976in}}{\pgfqpoint{3.334652in}{2.661800in}}%
\pgfpathcurveto{\pgfqpoint{3.328828in}{2.667624in}}{\pgfqpoint{3.320928in}{2.670896in}}{\pgfqpoint{3.312692in}{2.670896in}}%
\pgfpathcurveto{\pgfqpoint{3.304456in}{2.670896in}}{\pgfqpoint{3.296556in}{2.667624in}}{\pgfqpoint{3.290732in}{2.661800in}}%
\pgfpathcurveto{\pgfqpoint{3.284908in}{2.655976in}}{\pgfqpoint{3.281635in}{2.648076in}}{\pgfqpoint{3.281635in}{2.639840in}}%
\pgfpathcurveto{\pgfqpoint{3.281635in}{2.631603in}}{\pgfqpoint{3.284908in}{2.623703in}}{\pgfqpoint{3.290732in}{2.617879in}}%
\pgfpathcurveto{\pgfqpoint{3.296556in}{2.612055in}}{\pgfqpoint{3.304456in}{2.608783in}}{\pgfqpoint{3.312692in}{2.608783in}}%
\pgfpathclose%
\pgfusepath{stroke,fill}%
\end{pgfscope}%
\begin{pgfscope}%
\pgfpathrectangle{\pgfqpoint{0.100000in}{0.212622in}}{\pgfqpoint{3.696000in}{3.696000in}}%
\pgfusepath{clip}%
\pgfsetbuttcap%
\pgfsetroundjoin%
\definecolor{currentfill}{rgb}{0.121569,0.466667,0.705882}%
\pgfsetfillcolor{currentfill}%
\pgfsetfillopacity{0.695421}%
\pgfsetlinewidth{1.003750pt}%
\definecolor{currentstroke}{rgb}{0.121569,0.466667,0.705882}%
\pgfsetstrokecolor{currentstroke}%
\pgfsetstrokeopacity{0.695421}%
\pgfsetdash{}{0pt}%
\pgfpathmoveto{\pgfqpoint{3.312692in}{2.608783in}}%
\pgfpathcurveto{\pgfqpoint{3.320928in}{2.608783in}}{\pgfqpoint{3.328828in}{2.612055in}}{\pgfqpoint{3.334652in}{2.617879in}}%
\pgfpathcurveto{\pgfqpoint{3.340476in}{2.623703in}}{\pgfqpoint{3.343748in}{2.631603in}}{\pgfqpoint{3.343748in}{2.639840in}}%
\pgfpathcurveto{\pgfqpoint{3.343748in}{2.648076in}}{\pgfqpoint{3.340476in}{2.655976in}}{\pgfqpoint{3.334652in}{2.661800in}}%
\pgfpathcurveto{\pgfqpoint{3.328828in}{2.667624in}}{\pgfqpoint{3.320928in}{2.670896in}}{\pgfqpoint{3.312692in}{2.670896in}}%
\pgfpathcurveto{\pgfqpoint{3.304456in}{2.670896in}}{\pgfqpoint{3.296556in}{2.667624in}}{\pgfqpoint{3.290732in}{2.661800in}}%
\pgfpathcurveto{\pgfqpoint{3.284908in}{2.655976in}}{\pgfqpoint{3.281635in}{2.648076in}}{\pgfqpoint{3.281635in}{2.639840in}}%
\pgfpathcurveto{\pgfqpoint{3.281635in}{2.631603in}}{\pgfqpoint{3.284908in}{2.623703in}}{\pgfqpoint{3.290732in}{2.617879in}}%
\pgfpathcurveto{\pgfqpoint{3.296556in}{2.612055in}}{\pgfqpoint{3.304456in}{2.608783in}}{\pgfqpoint{3.312692in}{2.608783in}}%
\pgfpathclose%
\pgfusepath{stroke,fill}%
\end{pgfscope}%
\begin{pgfscope}%
\pgfpathrectangle{\pgfqpoint{0.100000in}{0.212622in}}{\pgfqpoint{3.696000in}{3.696000in}}%
\pgfusepath{clip}%
\pgfsetbuttcap%
\pgfsetroundjoin%
\definecolor{currentfill}{rgb}{0.121569,0.466667,0.705882}%
\pgfsetfillcolor{currentfill}%
\pgfsetfillopacity{0.695421}%
\pgfsetlinewidth{1.003750pt}%
\definecolor{currentstroke}{rgb}{0.121569,0.466667,0.705882}%
\pgfsetstrokecolor{currentstroke}%
\pgfsetstrokeopacity{0.695421}%
\pgfsetdash{}{0pt}%
\pgfpathmoveto{\pgfqpoint{3.312692in}{2.608783in}}%
\pgfpathcurveto{\pgfqpoint{3.320928in}{2.608783in}}{\pgfqpoint{3.328828in}{2.612055in}}{\pgfqpoint{3.334652in}{2.617879in}}%
\pgfpathcurveto{\pgfqpoint{3.340476in}{2.623703in}}{\pgfqpoint{3.343748in}{2.631603in}}{\pgfqpoint{3.343748in}{2.639840in}}%
\pgfpathcurveto{\pgfqpoint{3.343748in}{2.648076in}}{\pgfqpoint{3.340476in}{2.655976in}}{\pgfqpoint{3.334652in}{2.661800in}}%
\pgfpathcurveto{\pgfqpoint{3.328828in}{2.667624in}}{\pgfqpoint{3.320928in}{2.670896in}}{\pgfqpoint{3.312692in}{2.670896in}}%
\pgfpathcurveto{\pgfqpoint{3.304456in}{2.670896in}}{\pgfqpoint{3.296556in}{2.667624in}}{\pgfqpoint{3.290732in}{2.661800in}}%
\pgfpathcurveto{\pgfqpoint{3.284908in}{2.655976in}}{\pgfqpoint{3.281635in}{2.648076in}}{\pgfqpoint{3.281635in}{2.639840in}}%
\pgfpathcurveto{\pgfqpoint{3.281635in}{2.631603in}}{\pgfqpoint{3.284908in}{2.623703in}}{\pgfqpoint{3.290732in}{2.617879in}}%
\pgfpathcurveto{\pgfqpoint{3.296556in}{2.612055in}}{\pgfqpoint{3.304456in}{2.608783in}}{\pgfqpoint{3.312692in}{2.608783in}}%
\pgfpathclose%
\pgfusepath{stroke,fill}%
\end{pgfscope}%
\begin{pgfscope}%
\pgfpathrectangle{\pgfqpoint{0.100000in}{0.212622in}}{\pgfqpoint{3.696000in}{3.696000in}}%
\pgfusepath{clip}%
\pgfsetbuttcap%
\pgfsetroundjoin%
\definecolor{currentfill}{rgb}{0.121569,0.466667,0.705882}%
\pgfsetfillcolor{currentfill}%
\pgfsetfillopacity{0.695421}%
\pgfsetlinewidth{1.003750pt}%
\definecolor{currentstroke}{rgb}{0.121569,0.466667,0.705882}%
\pgfsetstrokecolor{currentstroke}%
\pgfsetstrokeopacity{0.695421}%
\pgfsetdash{}{0pt}%
\pgfpathmoveto{\pgfqpoint{3.312692in}{2.608783in}}%
\pgfpathcurveto{\pgfqpoint{3.320928in}{2.608783in}}{\pgfqpoint{3.328828in}{2.612055in}}{\pgfqpoint{3.334652in}{2.617879in}}%
\pgfpathcurveto{\pgfqpoint{3.340476in}{2.623703in}}{\pgfqpoint{3.343748in}{2.631603in}}{\pgfqpoint{3.343748in}{2.639840in}}%
\pgfpathcurveto{\pgfqpoint{3.343748in}{2.648076in}}{\pgfqpoint{3.340476in}{2.655976in}}{\pgfqpoint{3.334652in}{2.661800in}}%
\pgfpathcurveto{\pgfqpoint{3.328828in}{2.667624in}}{\pgfqpoint{3.320928in}{2.670896in}}{\pgfqpoint{3.312692in}{2.670896in}}%
\pgfpathcurveto{\pgfqpoint{3.304456in}{2.670896in}}{\pgfqpoint{3.296556in}{2.667624in}}{\pgfqpoint{3.290732in}{2.661800in}}%
\pgfpathcurveto{\pgfqpoint{3.284908in}{2.655976in}}{\pgfqpoint{3.281635in}{2.648076in}}{\pgfqpoint{3.281635in}{2.639840in}}%
\pgfpathcurveto{\pgfqpoint{3.281635in}{2.631603in}}{\pgfqpoint{3.284908in}{2.623703in}}{\pgfqpoint{3.290732in}{2.617879in}}%
\pgfpathcurveto{\pgfqpoint{3.296556in}{2.612055in}}{\pgfqpoint{3.304456in}{2.608783in}}{\pgfqpoint{3.312692in}{2.608783in}}%
\pgfpathclose%
\pgfusepath{stroke,fill}%
\end{pgfscope}%
\begin{pgfscope}%
\pgfpathrectangle{\pgfqpoint{0.100000in}{0.212622in}}{\pgfqpoint{3.696000in}{3.696000in}}%
\pgfusepath{clip}%
\pgfsetbuttcap%
\pgfsetroundjoin%
\definecolor{currentfill}{rgb}{0.121569,0.466667,0.705882}%
\pgfsetfillcolor{currentfill}%
\pgfsetfillopacity{0.695421}%
\pgfsetlinewidth{1.003750pt}%
\definecolor{currentstroke}{rgb}{0.121569,0.466667,0.705882}%
\pgfsetstrokecolor{currentstroke}%
\pgfsetstrokeopacity{0.695421}%
\pgfsetdash{}{0pt}%
\pgfpathmoveto{\pgfqpoint{3.312692in}{2.608783in}}%
\pgfpathcurveto{\pgfqpoint{3.320928in}{2.608783in}}{\pgfqpoint{3.328828in}{2.612055in}}{\pgfqpoint{3.334652in}{2.617879in}}%
\pgfpathcurveto{\pgfqpoint{3.340476in}{2.623703in}}{\pgfqpoint{3.343748in}{2.631603in}}{\pgfqpoint{3.343748in}{2.639840in}}%
\pgfpathcurveto{\pgfqpoint{3.343748in}{2.648076in}}{\pgfqpoint{3.340476in}{2.655976in}}{\pgfqpoint{3.334652in}{2.661800in}}%
\pgfpathcurveto{\pgfqpoint{3.328828in}{2.667624in}}{\pgfqpoint{3.320928in}{2.670896in}}{\pgfqpoint{3.312692in}{2.670896in}}%
\pgfpathcurveto{\pgfqpoint{3.304456in}{2.670896in}}{\pgfqpoint{3.296556in}{2.667624in}}{\pgfqpoint{3.290732in}{2.661800in}}%
\pgfpathcurveto{\pgfqpoint{3.284908in}{2.655976in}}{\pgfqpoint{3.281635in}{2.648076in}}{\pgfqpoint{3.281635in}{2.639840in}}%
\pgfpathcurveto{\pgfqpoint{3.281635in}{2.631603in}}{\pgfqpoint{3.284908in}{2.623703in}}{\pgfqpoint{3.290732in}{2.617879in}}%
\pgfpathcurveto{\pgfqpoint{3.296556in}{2.612055in}}{\pgfqpoint{3.304456in}{2.608783in}}{\pgfqpoint{3.312692in}{2.608783in}}%
\pgfpathclose%
\pgfusepath{stroke,fill}%
\end{pgfscope}%
\begin{pgfscope}%
\pgfpathrectangle{\pgfqpoint{0.100000in}{0.212622in}}{\pgfqpoint{3.696000in}{3.696000in}}%
\pgfusepath{clip}%
\pgfsetbuttcap%
\pgfsetroundjoin%
\definecolor{currentfill}{rgb}{0.121569,0.466667,0.705882}%
\pgfsetfillcolor{currentfill}%
\pgfsetfillopacity{0.695421}%
\pgfsetlinewidth{1.003750pt}%
\definecolor{currentstroke}{rgb}{0.121569,0.466667,0.705882}%
\pgfsetstrokecolor{currentstroke}%
\pgfsetstrokeopacity{0.695421}%
\pgfsetdash{}{0pt}%
\pgfpathmoveto{\pgfqpoint{3.312692in}{2.608783in}}%
\pgfpathcurveto{\pgfqpoint{3.320928in}{2.608783in}}{\pgfqpoint{3.328828in}{2.612055in}}{\pgfqpoint{3.334652in}{2.617879in}}%
\pgfpathcurveto{\pgfqpoint{3.340476in}{2.623703in}}{\pgfqpoint{3.343748in}{2.631603in}}{\pgfqpoint{3.343748in}{2.639840in}}%
\pgfpathcurveto{\pgfqpoint{3.343748in}{2.648076in}}{\pgfqpoint{3.340476in}{2.655976in}}{\pgfqpoint{3.334652in}{2.661800in}}%
\pgfpathcurveto{\pgfqpoint{3.328828in}{2.667624in}}{\pgfqpoint{3.320928in}{2.670896in}}{\pgfqpoint{3.312692in}{2.670896in}}%
\pgfpathcurveto{\pgfqpoint{3.304456in}{2.670896in}}{\pgfqpoint{3.296556in}{2.667624in}}{\pgfqpoint{3.290732in}{2.661800in}}%
\pgfpathcurveto{\pgfqpoint{3.284908in}{2.655976in}}{\pgfqpoint{3.281635in}{2.648076in}}{\pgfqpoint{3.281635in}{2.639840in}}%
\pgfpathcurveto{\pgfqpoint{3.281635in}{2.631603in}}{\pgfqpoint{3.284908in}{2.623703in}}{\pgfqpoint{3.290732in}{2.617879in}}%
\pgfpathcurveto{\pgfqpoint{3.296556in}{2.612055in}}{\pgfqpoint{3.304456in}{2.608783in}}{\pgfqpoint{3.312692in}{2.608783in}}%
\pgfpathclose%
\pgfusepath{stroke,fill}%
\end{pgfscope}%
\begin{pgfscope}%
\pgfpathrectangle{\pgfqpoint{0.100000in}{0.212622in}}{\pgfqpoint{3.696000in}{3.696000in}}%
\pgfusepath{clip}%
\pgfsetbuttcap%
\pgfsetroundjoin%
\definecolor{currentfill}{rgb}{0.121569,0.466667,0.705882}%
\pgfsetfillcolor{currentfill}%
\pgfsetfillopacity{0.695421}%
\pgfsetlinewidth{1.003750pt}%
\definecolor{currentstroke}{rgb}{0.121569,0.466667,0.705882}%
\pgfsetstrokecolor{currentstroke}%
\pgfsetstrokeopacity{0.695421}%
\pgfsetdash{}{0pt}%
\pgfpathmoveto{\pgfqpoint{3.312692in}{2.608783in}}%
\pgfpathcurveto{\pgfqpoint{3.320928in}{2.608783in}}{\pgfqpoint{3.328828in}{2.612055in}}{\pgfqpoint{3.334652in}{2.617879in}}%
\pgfpathcurveto{\pgfqpoint{3.340476in}{2.623703in}}{\pgfqpoint{3.343748in}{2.631603in}}{\pgfqpoint{3.343748in}{2.639840in}}%
\pgfpathcurveto{\pgfqpoint{3.343748in}{2.648076in}}{\pgfqpoint{3.340476in}{2.655976in}}{\pgfqpoint{3.334652in}{2.661800in}}%
\pgfpathcurveto{\pgfqpoint{3.328828in}{2.667624in}}{\pgfqpoint{3.320928in}{2.670896in}}{\pgfqpoint{3.312692in}{2.670896in}}%
\pgfpathcurveto{\pgfqpoint{3.304456in}{2.670896in}}{\pgfqpoint{3.296556in}{2.667624in}}{\pgfqpoint{3.290732in}{2.661800in}}%
\pgfpathcurveto{\pgfqpoint{3.284908in}{2.655976in}}{\pgfqpoint{3.281635in}{2.648076in}}{\pgfqpoint{3.281635in}{2.639840in}}%
\pgfpathcurveto{\pgfqpoint{3.281635in}{2.631603in}}{\pgfqpoint{3.284908in}{2.623703in}}{\pgfqpoint{3.290732in}{2.617879in}}%
\pgfpathcurveto{\pgfqpoint{3.296556in}{2.612055in}}{\pgfqpoint{3.304456in}{2.608783in}}{\pgfqpoint{3.312692in}{2.608783in}}%
\pgfpathclose%
\pgfusepath{stroke,fill}%
\end{pgfscope}%
\begin{pgfscope}%
\pgfpathrectangle{\pgfqpoint{0.100000in}{0.212622in}}{\pgfqpoint{3.696000in}{3.696000in}}%
\pgfusepath{clip}%
\pgfsetbuttcap%
\pgfsetroundjoin%
\definecolor{currentfill}{rgb}{0.121569,0.466667,0.705882}%
\pgfsetfillcolor{currentfill}%
\pgfsetfillopacity{0.695421}%
\pgfsetlinewidth{1.003750pt}%
\definecolor{currentstroke}{rgb}{0.121569,0.466667,0.705882}%
\pgfsetstrokecolor{currentstroke}%
\pgfsetstrokeopacity{0.695421}%
\pgfsetdash{}{0pt}%
\pgfpathmoveto{\pgfqpoint{3.312692in}{2.608783in}}%
\pgfpathcurveto{\pgfqpoint{3.320928in}{2.608783in}}{\pgfqpoint{3.328828in}{2.612055in}}{\pgfqpoint{3.334652in}{2.617879in}}%
\pgfpathcurveto{\pgfqpoint{3.340476in}{2.623703in}}{\pgfqpoint{3.343748in}{2.631603in}}{\pgfqpoint{3.343748in}{2.639840in}}%
\pgfpathcurveto{\pgfqpoint{3.343748in}{2.648076in}}{\pgfqpoint{3.340476in}{2.655976in}}{\pgfqpoint{3.334652in}{2.661800in}}%
\pgfpathcurveto{\pgfqpoint{3.328828in}{2.667624in}}{\pgfqpoint{3.320928in}{2.670896in}}{\pgfqpoint{3.312692in}{2.670896in}}%
\pgfpathcurveto{\pgfqpoint{3.304456in}{2.670896in}}{\pgfqpoint{3.296556in}{2.667624in}}{\pgfqpoint{3.290732in}{2.661800in}}%
\pgfpathcurveto{\pgfqpoint{3.284908in}{2.655976in}}{\pgfqpoint{3.281635in}{2.648076in}}{\pgfqpoint{3.281635in}{2.639840in}}%
\pgfpathcurveto{\pgfqpoint{3.281635in}{2.631603in}}{\pgfqpoint{3.284908in}{2.623703in}}{\pgfqpoint{3.290732in}{2.617879in}}%
\pgfpathcurveto{\pgfqpoint{3.296556in}{2.612055in}}{\pgfqpoint{3.304456in}{2.608783in}}{\pgfqpoint{3.312692in}{2.608783in}}%
\pgfpathclose%
\pgfusepath{stroke,fill}%
\end{pgfscope}%
\begin{pgfscope}%
\pgfpathrectangle{\pgfqpoint{0.100000in}{0.212622in}}{\pgfqpoint{3.696000in}{3.696000in}}%
\pgfusepath{clip}%
\pgfsetbuttcap%
\pgfsetroundjoin%
\definecolor{currentfill}{rgb}{0.121569,0.466667,0.705882}%
\pgfsetfillcolor{currentfill}%
\pgfsetfillopacity{0.695421}%
\pgfsetlinewidth{1.003750pt}%
\definecolor{currentstroke}{rgb}{0.121569,0.466667,0.705882}%
\pgfsetstrokecolor{currentstroke}%
\pgfsetstrokeopacity{0.695421}%
\pgfsetdash{}{0pt}%
\pgfpathmoveto{\pgfqpoint{3.312692in}{2.608783in}}%
\pgfpathcurveto{\pgfqpoint{3.320928in}{2.608783in}}{\pgfqpoint{3.328828in}{2.612055in}}{\pgfqpoint{3.334652in}{2.617879in}}%
\pgfpathcurveto{\pgfqpoint{3.340476in}{2.623703in}}{\pgfqpoint{3.343748in}{2.631603in}}{\pgfqpoint{3.343748in}{2.639840in}}%
\pgfpathcurveto{\pgfqpoint{3.343748in}{2.648076in}}{\pgfqpoint{3.340476in}{2.655976in}}{\pgfqpoint{3.334652in}{2.661800in}}%
\pgfpathcurveto{\pgfqpoint{3.328828in}{2.667624in}}{\pgfqpoint{3.320928in}{2.670896in}}{\pgfqpoint{3.312692in}{2.670896in}}%
\pgfpathcurveto{\pgfqpoint{3.304456in}{2.670896in}}{\pgfqpoint{3.296556in}{2.667624in}}{\pgfqpoint{3.290732in}{2.661800in}}%
\pgfpathcurveto{\pgfqpoint{3.284908in}{2.655976in}}{\pgfqpoint{3.281635in}{2.648076in}}{\pgfqpoint{3.281635in}{2.639840in}}%
\pgfpathcurveto{\pgfqpoint{3.281635in}{2.631603in}}{\pgfqpoint{3.284908in}{2.623703in}}{\pgfqpoint{3.290732in}{2.617879in}}%
\pgfpathcurveto{\pgfqpoint{3.296556in}{2.612055in}}{\pgfqpoint{3.304456in}{2.608783in}}{\pgfqpoint{3.312692in}{2.608783in}}%
\pgfpathclose%
\pgfusepath{stroke,fill}%
\end{pgfscope}%
\begin{pgfscope}%
\pgfpathrectangle{\pgfqpoint{0.100000in}{0.212622in}}{\pgfqpoint{3.696000in}{3.696000in}}%
\pgfusepath{clip}%
\pgfsetbuttcap%
\pgfsetroundjoin%
\definecolor{currentfill}{rgb}{0.121569,0.466667,0.705882}%
\pgfsetfillcolor{currentfill}%
\pgfsetfillopacity{0.695421}%
\pgfsetlinewidth{1.003750pt}%
\definecolor{currentstroke}{rgb}{0.121569,0.466667,0.705882}%
\pgfsetstrokecolor{currentstroke}%
\pgfsetstrokeopacity{0.695421}%
\pgfsetdash{}{0pt}%
\pgfpathmoveto{\pgfqpoint{3.312692in}{2.608783in}}%
\pgfpathcurveto{\pgfqpoint{3.320928in}{2.608783in}}{\pgfqpoint{3.328828in}{2.612055in}}{\pgfqpoint{3.334652in}{2.617879in}}%
\pgfpathcurveto{\pgfqpoint{3.340476in}{2.623703in}}{\pgfqpoint{3.343748in}{2.631603in}}{\pgfqpoint{3.343748in}{2.639840in}}%
\pgfpathcurveto{\pgfqpoint{3.343748in}{2.648076in}}{\pgfqpoint{3.340476in}{2.655976in}}{\pgfqpoint{3.334652in}{2.661800in}}%
\pgfpathcurveto{\pgfqpoint{3.328828in}{2.667624in}}{\pgfqpoint{3.320928in}{2.670896in}}{\pgfqpoint{3.312692in}{2.670896in}}%
\pgfpathcurveto{\pgfqpoint{3.304456in}{2.670896in}}{\pgfqpoint{3.296556in}{2.667624in}}{\pgfqpoint{3.290732in}{2.661800in}}%
\pgfpathcurveto{\pgfqpoint{3.284908in}{2.655976in}}{\pgfqpoint{3.281635in}{2.648076in}}{\pgfqpoint{3.281635in}{2.639840in}}%
\pgfpathcurveto{\pgfqpoint{3.281635in}{2.631603in}}{\pgfqpoint{3.284908in}{2.623703in}}{\pgfqpoint{3.290732in}{2.617879in}}%
\pgfpathcurveto{\pgfqpoint{3.296556in}{2.612055in}}{\pgfqpoint{3.304456in}{2.608783in}}{\pgfqpoint{3.312692in}{2.608783in}}%
\pgfpathclose%
\pgfusepath{stroke,fill}%
\end{pgfscope}%
\begin{pgfscope}%
\pgfpathrectangle{\pgfqpoint{0.100000in}{0.212622in}}{\pgfqpoint{3.696000in}{3.696000in}}%
\pgfusepath{clip}%
\pgfsetbuttcap%
\pgfsetroundjoin%
\definecolor{currentfill}{rgb}{0.121569,0.466667,0.705882}%
\pgfsetfillcolor{currentfill}%
\pgfsetfillopacity{0.695421}%
\pgfsetlinewidth{1.003750pt}%
\definecolor{currentstroke}{rgb}{0.121569,0.466667,0.705882}%
\pgfsetstrokecolor{currentstroke}%
\pgfsetstrokeopacity{0.695421}%
\pgfsetdash{}{0pt}%
\pgfpathmoveto{\pgfqpoint{3.312692in}{2.608783in}}%
\pgfpathcurveto{\pgfqpoint{3.320928in}{2.608783in}}{\pgfqpoint{3.328828in}{2.612055in}}{\pgfqpoint{3.334652in}{2.617879in}}%
\pgfpathcurveto{\pgfqpoint{3.340476in}{2.623703in}}{\pgfqpoint{3.343748in}{2.631603in}}{\pgfqpoint{3.343748in}{2.639840in}}%
\pgfpathcurveto{\pgfqpoint{3.343748in}{2.648076in}}{\pgfqpoint{3.340476in}{2.655976in}}{\pgfqpoint{3.334652in}{2.661800in}}%
\pgfpathcurveto{\pgfqpoint{3.328828in}{2.667624in}}{\pgfqpoint{3.320928in}{2.670896in}}{\pgfqpoint{3.312692in}{2.670896in}}%
\pgfpathcurveto{\pgfqpoint{3.304456in}{2.670896in}}{\pgfqpoint{3.296556in}{2.667624in}}{\pgfqpoint{3.290732in}{2.661800in}}%
\pgfpathcurveto{\pgfqpoint{3.284908in}{2.655976in}}{\pgfqpoint{3.281635in}{2.648076in}}{\pgfqpoint{3.281635in}{2.639840in}}%
\pgfpathcurveto{\pgfqpoint{3.281635in}{2.631603in}}{\pgfqpoint{3.284908in}{2.623703in}}{\pgfqpoint{3.290732in}{2.617879in}}%
\pgfpathcurveto{\pgfqpoint{3.296556in}{2.612055in}}{\pgfqpoint{3.304456in}{2.608783in}}{\pgfqpoint{3.312692in}{2.608783in}}%
\pgfpathclose%
\pgfusepath{stroke,fill}%
\end{pgfscope}%
\begin{pgfscope}%
\pgfpathrectangle{\pgfqpoint{0.100000in}{0.212622in}}{\pgfqpoint{3.696000in}{3.696000in}}%
\pgfusepath{clip}%
\pgfsetbuttcap%
\pgfsetroundjoin%
\definecolor{currentfill}{rgb}{0.121569,0.466667,0.705882}%
\pgfsetfillcolor{currentfill}%
\pgfsetfillopacity{0.695421}%
\pgfsetlinewidth{1.003750pt}%
\definecolor{currentstroke}{rgb}{0.121569,0.466667,0.705882}%
\pgfsetstrokecolor{currentstroke}%
\pgfsetstrokeopacity{0.695421}%
\pgfsetdash{}{0pt}%
\pgfpathmoveto{\pgfqpoint{3.312692in}{2.608783in}}%
\pgfpathcurveto{\pgfqpoint{3.320928in}{2.608783in}}{\pgfqpoint{3.328828in}{2.612055in}}{\pgfqpoint{3.334652in}{2.617879in}}%
\pgfpathcurveto{\pgfqpoint{3.340476in}{2.623703in}}{\pgfqpoint{3.343748in}{2.631603in}}{\pgfqpoint{3.343748in}{2.639840in}}%
\pgfpathcurveto{\pgfqpoint{3.343748in}{2.648076in}}{\pgfqpoint{3.340476in}{2.655976in}}{\pgfqpoint{3.334652in}{2.661800in}}%
\pgfpathcurveto{\pgfqpoint{3.328828in}{2.667624in}}{\pgfqpoint{3.320928in}{2.670896in}}{\pgfqpoint{3.312692in}{2.670896in}}%
\pgfpathcurveto{\pgfqpoint{3.304456in}{2.670896in}}{\pgfqpoint{3.296556in}{2.667624in}}{\pgfqpoint{3.290732in}{2.661800in}}%
\pgfpathcurveto{\pgfqpoint{3.284908in}{2.655976in}}{\pgfqpoint{3.281635in}{2.648076in}}{\pgfqpoint{3.281635in}{2.639840in}}%
\pgfpathcurveto{\pgfqpoint{3.281635in}{2.631603in}}{\pgfqpoint{3.284908in}{2.623703in}}{\pgfqpoint{3.290732in}{2.617879in}}%
\pgfpathcurveto{\pgfqpoint{3.296556in}{2.612055in}}{\pgfqpoint{3.304456in}{2.608783in}}{\pgfqpoint{3.312692in}{2.608783in}}%
\pgfpathclose%
\pgfusepath{stroke,fill}%
\end{pgfscope}%
\begin{pgfscope}%
\pgfpathrectangle{\pgfqpoint{0.100000in}{0.212622in}}{\pgfqpoint{3.696000in}{3.696000in}}%
\pgfusepath{clip}%
\pgfsetbuttcap%
\pgfsetroundjoin%
\definecolor{currentfill}{rgb}{0.121569,0.466667,0.705882}%
\pgfsetfillcolor{currentfill}%
\pgfsetfillopacity{0.695421}%
\pgfsetlinewidth{1.003750pt}%
\definecolor{currentstroke}{rgb}{0.121569,0.466667,0.705882}%
\pgfsetstrokecolor{currentstroke}%
\pgfsetstrokeopacity{0.695421}%
\pgfsetdash{}{0pt}%
\pgfpathmoveto{\pgfqpoint{3.312692in}{2.608783in}}%
\pgfpathcurveto{\pgfqpoint{3.320928in}{2.608783in}}{\pgfqpoint{3.328828in}{2.612055in}}{\pgfqpoint{3.334652in}{2.617879in}}%
\pgfpathcurveto{\pgfqpoint{3.340476in}{2.623703in}}{\pgfqpoint{3.343748in}{2.631603in}}{\pgfqpoint{3.343748in}{2.639840in}}%
\pgfpathcurveto{\pgfqpoint{3.343748in}{2.648076in}}{\pgfqpoint{3.340476in}{2.655976in}}{\pgfqpoint{3.334652in}{2.661800in}}%
\pgfpathcurveto{\pgfqpoint{3.328828in}{2.667624in}}{\pgfqpoint{3.320928in}{2.670896in}}{\pgfqpoint{3.312692in}{2.670896in}}%
\pgfpathcurveto{\pgfqpoint{3.304456in}{2.670896in}}{\pgfqpoint{3.296556in}{2.667624in}}{\pgfqpoint{3.290732in}{2.661800in}}%
\pgfpathcurveto{\pgfqpoint{3.284908in}{2.655976in}}{\pgfqpoint{3.281635in}{2.648076in}}{\pgfqpoint{3.281635in}{2.639840in}}%
\pgfpathcurveto{\pgfqpoint{3.281635in}{2.631603in}}{\pgfqpoint{3.284908in}{2.623703in}}{\pgfqpoint{3.290732in}{2.617879in}}%
\pgfpathcurveto{\pgfqpoint{3.296556in}{2.612055in}}{\pgfqpoint{3.304456in}{2.608783in}}{\pgfqpoint{3.312692in}{2.608783in}}%
\pgfpathclose%
\pgfusepath{stroke,fill}%
\end{pgfscope}%
\begin{pgfscope}%
\pgfpathrectangle{\pgfqpoint{0.100000in}{0.212622in}}{\pgfqpoint{3.696000in}{3.696000in}}%
\pgfusepath{clip}%
\pgfsetbuttcap%
\pgfsetroundjoin%
\definecolor{currentfill}{rgb}{0.121569,0.466667,0.705882}%
\pgfsetfillcolor{currentfill}%
\pgfsetfillopacity{0.695421}%
\pgfsetlinewidth{1.003750pt}%
\definecolor{currentstroke}{rgb}{0.121569,0.466667,0.705882}%
\pgfsetstrokecolor{currentstroke}%
\pgfsetstrokeopacity{0.695421}%
\pgfsetdash{}{0pt}%
\pgfpathmoveto{\pgfqpoint{3.312692in}{2.608783in}}%
\pgfpathcurveto{\pgfqpoint{3.320928in}{2.608783in}}{\pgfqpoint{3.328828in}{2.612055in}}{\pgfqpoint{3.334652in}{2.617879in}}%
\pgfpathcurveto{\pgfqpoint{3.340476in}{2.623703in}}{\pgfqpoint{3.343748in}{2.631603in}}{\pgfqpoint{3.343748in}{2.639840in}}%
\pgfpathcurveto{\pgfqpoint{3.343748in}{2.648076in}}{\pgfqpoint{3.340476in}{2.655976in}}{\pgfqpoint{3.334652in}{2.661800in}}%
\pgfpathcurveto{\pgfqpoint{3.328828in}{2.667624in}}{\pgfqpoint{3.320928in}{2.670896in}}{\pgfqpoint{3.312692in}{2.670896in}}%
\pgfpathcurveto{\pgfqpoint{3.304456in}{2.670896in}}{\pgfqpoint{3.296556in}{2.667624in}}{\pgfqpoint{3.290732in}{2.661800in}}%
\pgfpathcurveto{\pgfqpoint{3.284908in}{2.655976in}}{\pgfqpoint{3.281635in}{2.648076in}}{\pgfqpoint{3.281635in}{2.639840in}}%
\pgfpathcurveto{\pgfqpoint{3.281635in}{2.631603in}}{\pgfqpoint{3.284908in}{2.623703in}}{\pgfqpoint{3.290732in}{2.617879in}}%
\pgfpathcurveto{\pgfqpoint{3.296556in}{2.612055in}}{\pgfqpoint{3.304456in}{2.608783in}}{\pgfqpoint{3.312692in}{2.608783in}}%
\pgfpathclose%
\pgfusepath{stroke,fill}%
\end{pgfscope}%
\begin{pgfscope}%
\pgfpathrectangle{\pgfqpoint{0.100000in}{0.212622in}}{\pgfqpoint{3.696000in}{3.696000in}}%
\pgfusepath{clip}%
\pgfsetbuttcap%
\pgfsetroundjoin%
\definecolor{currentfill}{rgb}{0.121569,0.466667,0.705882}%
\pgfsetfillcolor{currentfill}%
\pgfsetfillopacity{0.695421}%
\pgfsetlinewidth{1.003750pt}%
\definecolor{currentstroke}{rgb}{0.121569,0.466667,0.705882}%
\pgfsetstrokecolor{currentstroke}%
\pgfsetstrokeopacity{0.695421}%
\pgfsetdash{}{0pt}%
\pgfpathmoveto{\pgfqpoint{3.312692in}{2.608783in}}%
\pgfpathcurveto{\pgfqpoint{3.320928in}{2.608783in}}{\pgfqpoint{3.328828in}{2.612055in}}{\pgfqpoint{3.334652in}{2.617879in}}%
\pgfpathcurveto{\pgfqpoint{3.340476in}{2.623703in}}{\pgfqpoint{3.343748in}{2.631603in}}{\pgfqpoint{3.343748in}{2.639840in}}%
\pgfpathcurveto{\pgfqpoint{3.343748in}{2.648076in}}{\pgfqpoint{3.340476in}{2.655976in}}{\pgfqpoint{3.334652in}{2.661800in}}%
\pgfpathcurveto{\pgfqpoint{3.328828in}{2.667624in}}{\pgfqpoint{3.320928in}{2.670896in}}{\pgfqpoint{3.312692in}{2.670896in}}%
\pgfpathcurveto{\pgfqpoint{3.304456in}{2.670896in}}{\pgfqpoint{3.296556in}{2.667624in}}{\pgfqpoint{3.290732in}{2.661800in}}%
\pgfpathcurveto{\pgfqpoint{3.284908in}{2.655976in}}{\pgfqpoint{3.281635in}{2.648076in}}{\pgfqpoint{3.281635in}{2.639840in}}%
\pgfpathcurveto{\pgfqpoint{3.281635in}{2.631603in}}{\pgfqpoint{3.284908in}{2.623703in}}{\pgfqpoint{3.290732in}{2.617879in}}%
\pgfpathcurveto{\pgfqpoint{3.296556in}{2.612055in}}{\pgfqpoint{3.304456in}{2.608783in}}{\pgfqpoint{3.312692in}{2.608783in}}%
\pgfpathclose%
\pgfusepath{stroke,fill}%
\end{pgfscope}%
\begin{pgfscope}%
\pgfpathrectangle{\pgfqpoint{0.100000in}{0.212622in}}{\pgfqpoint{3.696000in}{3.696000in}}%
\pgfusepath{clip}%
\pgfsetbuttcap%
\pgfsetroundjoin%
\definecolor{currentfill}{rgb}{0.121569,0.466667,0.705882}%
\pgfsetfillcolor{currentfill}%
\pgfsetfillopacity{0.695421}%
\pgfsetlinewidth{1.003750pt}%
\definecolor{currentstroke}{rgb}{0.121569,0.466667,0.705882}%
\pgfsetstrokecolor{currentstroke}%
\pgfsetstrokeopacity{0.695421}%
\pgfsetdash{}{0pt}%
\pgfpathmoveto{\pgfqpoint{3.312692in}{2.608783in}}%
\pgfpathcurveto{\pgfqpoint{3.320928in}{2.608783in}}{\pgfqpoint{3.328828in}{2.612055in}}{\pgfqpoint{3.334652in}{2.617879in}}%
\pgfpathcurveto{\pgfqpoint{3.340476in}{2.623703in}}{\pgfqpoint{3.343748in}{2.631603in}}{\pgfqpoint{3.343748in}{2.639840in}}%
\pgfpathcurveto{\pgfqpoint{3.343748in}{2.648076in}}{\pgfqpoint{3.340476in}{2.655976in}}{\pgfqpoint{3.334652in}{2.661800in}}%
\pgfpathcurveto{\pgfqpoint{3.328828in}{2.667624in}}{\pgfqpoint{3.320928in}{2.670896in}}{\pgfqpoint{3.312692in}{2.670896in}}%
\pgfpathcurveto{\pgfqpoint{3.304456in}{2.670896in}}{\pgfqpoint{3.296556in}{2.667624in}}{\pgfqpoint{3.290732in}{2.661800in}}%
\pgfpathcurveto{\pgfqpoint{3.284908in}{2.655976in}}{\pgfqpoint{3.281635in}{2.648076in}}{\pgfqpoint{3.281635in}{2.639840in}}%
\pgfpathcurveto{\pgfqpoint{3.281635in}{2.631603in}}{\pgfqpoint{3.284908in}{2.623703in}}{\pgfqpoint{3.290732in}{2.617879in}}%
\pgfpathcurveto{\pgfqpoint{3.296556in}{2.612055in}}{\pgfqpoint{3.304456in}{2.608783in}}{\pgfqpoint{3.312692in}{2.608783in}}%
\pgfpathclose%
\pgfusepath{stroke,fill}%
\end{pgfscope}%
\begin{pgfscope}%
\pgfpathrectangle{\pgfqpoint{0.100000in}{0.212622in}}{\pgfqpoint{3.696000in}{3.696000in}}%
\pgfusepath{clip}%
\pgfsetbuttcap%
\pgfsetroundjoin%
\definecolor{currentfill}{rgb}{0.121569,0.466667,0.705882}%
\pgfsetfillcolor{currentfill}%
\pgfsetfillopacity{0.695421}%
\pgfsetlinewidth{1.003750pt}%
\definecolor{currentstroke}{rgb}{0.121569,0.466667,0.705882}%
\pgfsetstrokecolor{currentstroke}%
\pgfsetstrokeopacity{0.695421}%
\pgfsetdash{}{0pt}%
\pgfpathmoveto{\pgfqpoint{3.312692in}{2.608783in}}%
\pgfpathcurveto{\pgfqpoint{3.320928in}{2.608783in}}{\pgfqpoint{3.328828in}{2.612055in}}{\pgfqpoint{3.334652in}{2.617879in}}%
\pgfpathcurveto{\pgfqpoint{3.340476in}{2.623703in}}{\pgfqpoint{3.343748in}{2.631603in}}{\pgfqpoint{3.343748in}{2.639840in}}%
\pgfpathcurveto{\pgfqpoint{3.343748in}{2.648076in}}{\pgfqpoint{3.340476in}{2.655976in}}{\pgfqpoint{3.334652in}{2.661800in}}%
\pgfpathcurveto{\pgfqpoint{3.328828in}{2.667624in}}{\pgfqpoint{3.320928in}{2.670896in}}{\pgfqpoint{3.312692in}{2.670896in}}%
\pgfpathcurveto{\pgfqpoint{3.304456in}{2.670896in}}{\pgfqpoint{3.296556in}{2.667624in}}{\pgfqpoint{3.290732in}{2.661800in}}%
\pgfpathcurveto{\pgfqpoint{3.284908in}{2.655976in}}{\pgfqpoint{3.281635in}{2.648076in}}{\pgfqpoint{3.281635in}{2.639840in}}%
\pgfpathcurveto{\pgfqpoint{3.281635in}{2.631603in}}{\pgfqpoint{3.284908in}{2.623703in}}{\pgfqpoint{3.290732in}{2.617879in}}%
\pgfpathcurveto{\pgfqpoint{3.296556in}{2.612055in}}{\pgfqpoint{3.304456in}{2.608783in}}{\pgfqpoint{3.312692in}{2.608783in}}%
\pgfpathclose%
\pgfusepath{stroke,fill}%
\end{pgfscope}%
\begin{pgfscope}%
\pgfpathrectangle{\pgfqpoint{0.100000in}{0.212622in}}{\pgfqpoint{3.696000in}{3.696000in}}%
\pgfusepath{clip}%
\pgfsetbuttcap%
\pgfsetroundjoin%
\definecolor{currentfill}{rgb}{0.121569,0.466667,0.705882}%
\pgfsetfillcolor{currentfill}%
\pgfsetfillopacity{0.695421}%
\pgfsetlinewidth{1.003750pt}%
\definecolor{currentstroke}{rgb}{0.121569,0.466667,0.705882}%
\pgfsetstrokecolor{currentstroke}%
\pgfsetstrokeopacity{0.695421}%
\pgfsetdash{}{0pt}%
\pgfpathmoveto{\pgfqpoint{3.312692in}{2.608783in}}%
\pgfpathcurveto{\pgfqpoint{3.320928in}{2.608783in}}{\pgfqpoint{3.328828in}{2.612055in}}{\pgfqpoint{3.334652in}{2.617879in}}%
\pgfpathcurveto{\pgfqpoint{3.340476in}{2.623703in}}{\pgfqpoint{3.343748in}{2.631603in}}{\pgfqpoint{3.343748in}{2.639840in}}%
\pgfpathcurveto{\pgfqpoint{3.343748in}{2.648076in}}{\pgfqpoint{3.340476in}{2.655976in}}{\pgfqpoint{3.334652in}{2.661800in}}%
\pgfpathcurveto{\pgfqpoint{3.328828in}{2.667624in}}{\pgfqpoint{3.320928in}{2.670896in}}{\pgfqpoint{3.312692in}{2.670896in}}%
\pgfpathcurveto{\pgfqpoint{3.304456in}{2.670896in}}{\pgfqpoint{3.296556in}{2.667624in}}{\pgfqpoint{3.290732in}{2.661800in}}%
\pgfpathcurveto{\pgfqpoint{3.284908in}{2.655976in}}{\pgfqpoint{3.281635in}{2.648076in}}{\pgfqpoint{3.281635in}{2.639840in}}%
\pgfpathcurveto{\pgfqpoint{3.281635in}{2.631603in}}{\pgfqpoint{3.284908in}{2.623703in}}{\pgfqpoint{3.290732in}{2.617879in}}%
\pgfpathcurveto{\pgfqpoint{3.296556in}{2.612055in}}{\pgfqpoint{3.304456in}{2.608783in}}{\pgfqpoint{3.312692in}{2.608783in}}%
\pgfpathclose%
\pgfusepath{stroke,fill}%
\end{pgfscope}%
\begin{pgfscope}%
\pgfpathrectangle{\pgfqpoint{0.100000in}{0.212622in}}{\pgfqpoint{3.696000in}{3.696000in}}%
\pgfusepath{clip}%
\pgfsetbuttcap%
\pgfsetroundjoin%
\definecolor{currentfill}{rgb}{0.121569,0.466667,0.705882}%
\pgfsetfillcolor{currentfill}%
\pgfsetfillopacity{0.695421}%
\pgfsetlinewidth{1.003750pt}%
\definecolor{currentstroke}{rgb}{0.121569,0.466667,0.705882}%
\pgfsetstrokecolor{currentstroke}%
\pgfsetstrokeopacity{0.695421}%
\pgfsetdash{}{0pt}%
\pgfpathmoveto{\pgfqpoint{3.312692in}{2.608783in}}%
\pgfpathcurveto{\pgfqpoint{3.320928in}{2.608783in}}{\pgfqpoint{3.328828in}{2.612055in}}{\pgfqpoint{3.334652in}{2.617879in}}%
\pgfpathcurveto{\pgfqpoint{3.340476in}{2.623703in}}{\pgfqpoint{3.343748in}{2.631603in}}{\pgfqpoint{3.343748in}{2.639840in}}%
\pgfpathcurveto{\pgfqpoint{3.343748in}{2.648076in}}{\pgfqpoint{3.340476in}{2.655976in}}{\pgfqpoint{3.334652in}{2.661800in}}%
\pgfpathcurveto{\pgfqpoint{3.328828in}{2.667624in}}{\pgfqpoint{3.320928in}{2.670896in}}{\pgfqpoint{3.312692in}{2.670896in}}%
\pgfpathcurveto{\pgfqpoint{3.304456in}{2.670896in}}{\pgfqpoint{3.296556in}{2.667624in}}{\pgfqpoint{3.290732in}{2.661800in}}%
\pgfpathcurveto{\pgfqpoint{3.284908in}{2.655976in}}{\pgfqpoint{3.281635in}{2.648076in}}{\pgfqpoint{3.281635in}{2.639840in}}%
\pgfpathcurveto{\pgfqpoint{3.281635in}{2.631603in}}{\pgfqpoint{3.284908in}{2.623703in}}{\pgfqpoint{3.290732in}{2.617879in}}%
\pgfpathcurveto{\pgfqpoint{3.296556in}{2.612055in}}{\pgfqpoint{3.304456in}{2.608783in}}{\pgfqpoint{3.312692in}{2.608783in}}%
\pgfpathclose%
\pgfusepath{stroke,fill}%
\end{pgfscope}%
\begin{pgfscope}%
\pgfpathrectangle{\pgfqpoint{0.100000in}{0.212622in}}{\pgfqpoint{3.696000in}{3.696000in}}%
\pgfusepath{clip}%
\pgfsetbuttcap%
\pgfsetroundjoin%
\definecolor{currentfill}{rgb}{0.121569,0.466667,0.705882}%
\pgfsetfillcolor{currentfill}%
\pgfsetfillopacity{0.695421}%
\pgfsetlinewidth{1.003750pt}%
\definecolor{currentstroke}{rgb}{0.121569,0.466667,0.705882}%
\pgfsetstrokecolor{currentstroke}%
\pgfsetstrokeopacity{0.695421}%
\pgfsetdash{}{0pt}%
\pgfpathmoveto{\pgfqpoint{3.312692in}{2.608783in}}%
\pgfpathcurveto{\pgfqpoint{3.320928in}{2.608783in}}{\pgfqpoint{3.328828in}{2.612055in}}{\pgfqpoint{3.334652in}{2.617879in}}%
\pgfpathcurveto{\pgfqpoint{3.340476in}{2.623703in}}{\pgfqpoint{3.343748in}{2.631603in}}{\pgfqpoint{3.343748in}{2.639840in}}%
\pgfpathcurveto{\pgfqpoint{3.343748in}{2.648076in}}{\pgfqpoint{3.340476in}{2.655976in}}{\pgfqpoint{3.334652in}{2.661800in}}%
\pgfpathcurveto{\pgfqpoint{3.328828in}{2.667624in}}{\pgfqpoint{3.320928in}{2.670896in}}{\pgfqpoint{3.312692in}{2.670896in}}%
\pgfpathcurveto{\pgfqpoint{3.304456in}{2.670896in}}{\pgfqpoint{3.296556in}{2.667624in}}{\pgfqpoint{3.290732in}{2.661800in}}%
\pgfpathcurveto{\pgfqpoint{3.284908in}{2.655976in}}{\pgfqpoint{3.281635in}{2.648076in}}{\pgfqpoint{3.281635in}{2.639840in}}%
\pgfpathcurveto{\pgfqpoint{3.281635in}{2.631603in}}{\pgfqpoint{3.284908in}{2.623703in}}{\pgfqpoint{3.290732in}{2.617879in}}%
\pgfpathcurveto{\pgfqpoint{3.296556in}{2.612055in}}{\pgfqpoint{3.304456in}{2.608783in}}{\pgfqpoint{3.312692in}{2.608783in}}%
\pgfpathclose%
\pgfusepath{stroke,fill}%
\end{pgfscope}%
\begin{pgfscope}%
\pgfpathrectangle{\pgfqpoint{0.100000in}{0.212622in}}{\pgfqpoint{3.696000in}{3.696000in}}%
\pgfusepath{clip}%
\pgfsetbuttcap%
\pgfsetroundjoin%
\definecolor{currentfill}{rgb}{0.121569,0.466667,0.705882}%
\pgfsetfillcolor{currentfill}%
\pgfsetfillopacity{0.695421}%
\pgfsetlinewidth{1.003750pt}%
\definecolor{currentstroke}{rgb}{0.121569,0.466667,0.705882}%
\pgfsetstrokecolor{currentstroke}%
\pgfsetstrokeopacity{0.695421}%
\pgfsetdash{}{0pt}%
\pgfpathmoveto{\pgfqpoint{3.312692in}{2.608783in}}%
\pgfpathcurveto{\pgfqpoint{3.320928in}{2.608783in}}{\pgfqpoint{3.328828in}{2.612055in}}{\pgfqpoint{3.334652in}{2.617879in}}%
\pgfpathcurveto{\pgfqpoint{3.340476in}{2.623703in}}{\pgfqpoint{3.343748in}{2.631603in}}{\pgfqpoint{3.343748in}{2.639840in}}%
\pgfpathcurveto{\pgfqpoint{3.343748in}{2.648076in}}{\pgfqpoint{3.340476in}{2.655976in}}{\pgfqpoint{3.334652in}{2.661800in}}%
\pgfpathcurveto{\pgfqpoint{3.328828in}{2.667624in}}{\pgfqpoint{3.320928in}{2.670896in}}{\pgfqpoint{3.312692in}{2.670896in}}%
\pgfpathcurveto{\pgfqpoint{3.304456in}{2.670896in}}{\pgfqpoint{3.296556in}{2.667624in}}{\pgfqpoint{3.290732in}{2.661800in}}%
\pgfpathcurveto{\pgfqpoint{3.284908in}{2.655976in}}{\pgfqpoint{3.281635in}{2.648076in}}{\pgfqpoint{3.281635in}{2.639840in}}%
\pgfpathcurveto{\pgfqpoint{3.281635in}{2.631603in}}{\pgfqpoint{3.284908in}{2.623703in}}{\pgfqpoint{3.290732in}{2.617879in}}%
\pgfpathcurveto{\pgfqpoint{3.296556in}{2.612055in}}{\pgfqpoint{3.304456in}{2.608783in}}{\pgfqpoint{3.312692in}{2.608783in}}%
\pgfpathclose%
\pgfusepath{stroke,fill}%
\end{pgfscope}%
\begin{pgfscope}%
\pgfpathrectangle{\pgfqpoint{0.100000in}{0.212622in}}{\pgfqpoint{3.696000in}{3.696000in}}%
\pgfusepath{clip}%
\pgfsetbuttcap%
\pgfsetroundjoin%
\definecolor{currentfill}{rgb}{0.121569,0.466667,0.705882}%
\pgfsetfillcolor{currentfill}%
\pgfsetfillopacity{0.695421}%
\pgfsetlinewidth{1.003750pt}%
\definecolor{currentstroke}{rgb}{0.121569,0.466667,0.705882}%
\pgfsetstrokecolor{currentstroke}%
\pgfsetstrokeopacity{0.695421}%
\pgfsetdash{}{0pt}%
\pgfpathmoveto{\pgfqpoint{3.312692in}{2.608783in}}%
\pgfpathcurveto{\pgfqpoint{3.320928in}{2.608783in}}{\pgfqpoint{3.328828in}{2.612055in}}{\pgfqpoint{3.334652in}{2.617879in}}%
\pgfpathcurveto{\pgfqpoint{3.340476in}{2.623703in}}{\pgfqpoint{3.343748in}{2.631603in}}{\pgfqpoint{3.343748in}{2.639840in}}%
\pgfpathcurveto{\pgfqpoint{3.343748in}{2.648076in}}{\pgfqpoint{3.340476in}{2.655976in}}{\pgfqpoint{3.334652in}{2.661800in}}%
\pgfpathcurveto{\pgfqpoint{3.328828in}{2.667624in}}{\pgfqpoint{3.320928in}{2.670896in}}{\pgfqpoint{3.312692in}{2.670896in}}%
\pgfpathcurveto{\pgfqpoint{3.304456in}{2.670896in}}{\pgfqpoint{3.296556in}{2.667624in}}{\pgfqpoint{3.290732in}{2.661800in}}%
\pgfpathcurveto{\pgfqpoint{3.284908in}{2.655976in}}{\pgfqpoint{3.281635in}{2.648076in}}{\pgfqpoint{3.281635in}{2.639840in}}%
\pgfpathcurveto{\pgfqpoint{3.281635in}{2.631603in}}{\pgfqpoint{3.284908in}{2.623703in}}{\pgfqpoint{3.290732in}{2.617879in}}%
\pgfpathcurveto{\pgfqpoint{3.296556in}{2.612055in}}{\pgfqpoint{3.304456in}{2.608783in}}{\pgfqpoint{3.312692in}{2.608783in}}%
\pgfpathclose%
\pgfusepath{stroke,fill}%
\end{pgfscope}%
\begin{pgfscope}%
\pgfpathrectangle{\pgfqpoint{0.100000in}{0.212622in}}{\pgfqpoint{3.696000in}{3.696000in}}%
\pgfusepath{clip}%
\pgfsetbuttcap%
\pgfsetroundjoin%
\definecolor{currentfill}{rgb}{0.121569,0.466667,0.705882}%
\pgfsetfillcolor{currentfill}%
\pgfsetfillopacity{0.695421}%
\pgfsetlinewidth{1.003750pt}%
\definecolor{currentstroke}{rgb}{0.121569,0.466667,0.705882}%
\pgfsetstrokecolor{currentstroke}%
\pgfsetstrokeopacity{0.695421}%
\pgfsetdash{}{0pt}%
\pgfpathmoveto{\pgfqpoint{3.312692in}{2.608783in}}%
\pgfpathcurveto{\pgfqpoint{3.320928in}{2.608783in}}{\pgfqpoint{3.328828in}{2.612055in}}{\pgfqpoint{3.334652in}{2.617879in}}%
\pgfpathcurveto{\pgfqpoint{3.340476in}{2.623703in}}{\pgfqpoint{3.343748in}{2.631603in}}{\pgfqpoint{3.343748in}{2.639840in}}%
\pgfpathcurveto{\pgfqpoint{3.343748in}{2.648076in}}{\pgfqpoint{3.340476in}{2.655976in}}{\pgfqpoint{3.334652in}{2.661800in}}%
\pgfpathcurveto{\pgfqpoint{3.328828in}{2.667624in}}{\pgfqpoint{3.320928in}{2.670896in}}{\pgfqpoint{3.312692in}{2.670896in}}%
\pgfpathcurveto{\pgfqpoint{3.304456in}{2.670896in}}{\pgfqpoint{3.296556in}{2.667624in}}{\pgfqpoint{3.290732in}{2.661800in}}%
\pgfpathcurveto{\pgfqpoint{3.284908in}{2.655976in}}{\pgfqpoint{3.281635in}{2.648076in}}{\pgfqpoint{3.281635in}{2.639840in}}%
\pgfpathcurveto{\pgfqpoint{3.281635in}{2.631603in}}{\pgfqpoint{3.284908in}{2.623703in}}{\pgfqpoint{3.290732in}{2.617879in}}%
\pgfpathcurveto{\pgfqpoint{3.296556in}{2.612055in}}{\pgfqpoint{3.304456in}{2.608783in}}{\pgfqpoint{3.312692in}{2.608783in}}%
\pgfpathclose%
\pgfusepath{stroke,fill}%
\end{pgfscope}%
\begin{pgfscope}%
\pgfpathrectangle{\pgfqpoint{0.100000in}{0.212622in}}{\pgfqpoint{3.696000in}{3.696000in}}%
\pgfusepath{clip}%
\pgfsetbuttcap%
\pgfsetroundjoin%
\definecolor{currentfill}{rgb}{0.121569,0.466667,0.705882}%
\pgfsetfillcolor{currentfill}%
\pgfsetfillopacity{0.695421}%
\pgfsetlinewidth{1.003750pt}%
\definecolor{currentstroke}{rgb}{0.121569,0.466667,0.705882}%
\pgfsetstrokecolor{currentstroke}%
\pgfsetstrokeopacity{0.695421}%
\pgfsetdash{}{0pt}%
\pgfpathmoveto{\pgfqpoint{3.312692in}{2.608783in}}%
\pgfpathcurveto{\pgfqpoint{3.320928in}{2.608783in}}{\pgfqpoint{3.328828in}{2.612055in}}{\pgfqpoint{3.334652in}{2.617879in}}%
\pgfpathcurveto{\pgfqpoint{3.340476in}{2.623703in}}{\pgfqpoint{3.343749in}{2.631603in}}{\pgfqpoint{3.343749in}{2.639840in}}%
\pgfpathcurveto{\pgfqpoint{3.343749in}{2.648076in}}{\pgfqpoint{3.340476in}{2.655976in}}{\pgfqpoint{3.334652in}{2.661800in}}%
\pgfpathcurveto{\pgfqpoint{3.328828in}{2.667624in}}{\pgfqpoint{3.320928in}{2.670896in}}{\pgfqpoint{3.312692in}{2.670896in}}%
\pgfpathcurveto{\pgfqpoint{3.304456in}{2.670896in}}{\pgfqpoint{3.296556in}{2.667624in}}{\pgfqpoint{3.290732in}{2.661800in}}%
\pgfpathcurveto{\pgfqpoint{3.284908in}{2.655976in}}{\pgfqpoint{3.281636in}{2.648076in}}{\pgfqpoint{3.281636in}{2.639840in}}%
\pgfpathcurveto{\pgfqpoint{3.281636in}{2.631603in}}{\pgfqpoint{3.284908in}{2.623703in}}{\pgfqpoint{3.290732in}{2.617879in}}%
\pgfpathcurveto{\pgfqpoint{3.296556in}{2.612055in}}{\pgfqpoint{3.304456in}{2.608783in}}{\pgfqpoint{3.312692in}{2.608783in}}%
\pgfpathclose%
\pgfusepath{stroke,fill}%
\end{pgfscope}%
\begin{pgfscope}%
\pgfpathrectangle{\pgfqpoint{0.100000in}{0.212622in}}{\pgfqpoint{3.696000in}{3.696000in}}%
\pgfusepath{clip}%
\pgfsetbuttcap%
\pgfsetroundjoin%
\definecolor{currentfill}{rgb}{0.121569,0.466667,0.705882}%
\pgfsetfillcolor{currentfill}%
\pgfsetfillopacity{0.695421}%
\pgfsetlinewidth{1.003750pt}%
\definecolor{currentstroke}{rgb}{0.121569,0.466667,0.705882}%
\pgfsetstrokecolor{currentstroke}%
\pgfsetstrokeopacity{0.695421}%
\pgfsetdash{}{0pt}%
\pgfpathmoveto{\pgfqpoint{3.312692in}{2.608783in}}%
\pgfpathcurveto{\pgfqpoint{3.320928in}{2.608783in}}{\pgfqpoint{3.328829in}{2.612055in}}{\pgfqpoint{3.334652in}{2.617879in}}%
\pgfpathcurveto{\pgfqpoint{3.340476in}{2.623703in}}{\pgfqpoint{3.343749in}{2.631603in}}{\pgfqpoint{3.343749in}{2.639840in}}%
\pgfpathcurveto{\pgfqpoint{3.343749in}{2.648076in}}{\pgfqpoint{3.340476in}{2.655976in}}{\pgfqpoint{3.334652in}{2.661800in}}%
\pgfpathcurveto{\pgfqpoint{3.328829in}{2.667624in}}{\pgfqpoint{3.320928in}{2.670896in}}{\pgfqpoint{3.312692in}{2.670896in}}%
\pgfpathcurveto{\pgfqpoint{3.304456in}{2.670896in}}{\pgfqpoint{3.296556in}{2.667624in}}{\pgfqpoint{3.290732in}{2.661800in}}%
\pgfpathcurveto{\pgfqpoint{3.284908in}{2.655976in}}{\pgfqpoint{3.281636in}{2.648076in}}{\pgfqpoint{3.281636in}{2.639840in}}%
\pgfpathcurveto{\pgfqpoint{3.281636in}{2.631603in}}{\pgfqpoint{3.284908in}{2.623703in}}{\pgfqpoint{3.290732in}{2.617879in}}%
\pgfpathcurveto{\pgfqpoint{3.296556in}{2.612055in}}{\pgfqpoint{3.304456in}{2.608783in}}{\pgfqpoint{3.312692in}{2.608783in}}%
\pgfpathclose%
\pgfusepath{stroke,fill}%
\end{pgfscope}%
\begin{pgfscope}%
\pgfpathrectangle{\pgfqpoint{0.100000in}{0.212622in}}{\pgfqpoint{3.696000in}{3.696000in}}%
\pgfusepath{clip}%
\pgfsetbuttcap%
\pgfsetroundjoin%
\definecolor{currentfill}{rgb}{0.121569,0.466667,0.705882}%
\pgfsetfillcolor{currentfill}%
\pgfsetfillopacity{0.695421}%
\pgfsetlinewidth{1.003750pt}%
\definecolor{currentstroke}{rgb}{0.121569,0.466667,0.705882}%
\pgfsetstrokecolor{currentstroke}%
\pgfsetstrokeopacity{0.695421}%
\pgfsetdash{}{0pt}%
\pgfpathmoveto{\pgfqpoint{3.312692in}{2.608783in}}%
\pgfpathcurveto{\pgfqpoint{3.320929in}{2.608783in}}{\pgfqpoint{3.328829in}{2.612055in}}{\pgfqpoint{3.334653in}{2.617879in}}%
\pgfpathcurveto{\pgfqpoint{3.340477in}{2.623703in}}{\pgfqpoint{3.343749in}{2.631603in}}{\pgfqpoint{3.343749in}{2.639839in}}%
\pgfpathcurveto{\pgfqpoint{3.343749in}{2.648076in}}{\pgfqpoint{3.340477in}{2.655976in}}{\pgfqpoint{3.334653in}{2.661800in}}%
\pgfpathcurveto{\pgfqpoint{3.328829in}{2.667624in}}{\pgfqpoint{3.320929in}{2.670896in}}{\pgfqpoint{3.312692in}{2.670896in}}%
\pgfpathcurveto{\pgfqpoint{3.304456in}{2.670896in}}{\pgfqpoint{3.296556in}{2.667624in}}{\pgfqpoint{3.290732in}{2.661800in}}%
\pgfpathcurveto{\pgfqpoint{3.284908in}{2.655976in}}{\pgfqpoint{3.281636in}{2.648076in}}{\pgfqpoint{3.281636in}{2.639839in}}%
\pgfpathcurveto{\pgfqpoint{3.281636in}{2.631603in}}{\pgfqpoint{3.284908in}{2.623703in}}{\pgfqpoint{3.290732in}{2.617879in}}%
\pgfpathcurveto{\pgfqpoint{3.296556in}{2.612055in}}{\pgfqpoint{3.304456in}{2.608783in}}{\pgfqpoint{3.312692in}{2.608783in}}%
\pgfpathclose%
\pgfusepath{stroke,fill}%
\end{pgfscope}%
\begin{pgfscope}%
\pgfpathrectangle{\pgfqpoint{0.100000in}{0.212622in}}{\pgfqpoint{3.696000in}{3.696000in}}%
\pgfusepath{clip}%
\pgfsetbuttcap%
\pgfsetroundjoin%
\definecolor{currentfill}{rgb}{0.121569,0.466667,0.705882}%
\pgfsetfillcolor{currentfill}%
\pgfsetfillopacity{0.695421}%
\pgfsetlinewidth{1.003750pt}%
\definecolor{currentstroke}{rgb}{0.121569,0.466667,0.705882}%
\pgfsetstrokecolor{currentstroke}%
\pgfsetstrokeopacity{0.695421}%
\pgfsetdash{}{0pt}%
\pgfpathmoveto{\pgfqpoint{3.312693in}{2.608783in}}%
\pgfpathcurveto{\pgfqpoint{3.320929in}{2.608783in}}{\pgfqpoint{3.328829in}{2.612055in}}{\pgfqpoint{3.334653in}{2.617879in}}%
\pgfpathcurveto{\pgfqpoint{3.340477in}{2.623703in}}{\pgfqpoint{3.343749in}{2.631603in}}{\pgfqpoint{3.343749in}{2.639839in}}%
\pgfpathcurveto{\pgfqpoint{3.343749in}{2.648076in}}{\pgfqpoint{3.340477in}{2.655976in}}{\pgfqpoint{3.334653in}{2.661800in}}%
\pgfpathcurveto{\pgfqpoint{3.328829in}{2.667624in}}{\pgfqpoint{3.320929in}{2.670896in}}{\pgfqpoint{3.312693in}{2.670896in}}%
\pgfpathcurveto{\pgfqpoint{3.304457in}{2.670896in}}{\pgfqpoint{3.296556in}{2.667624in}}{\pgfqpoint{3.290733in}{2.661800in}}%
\pgfpathcurveto{\pgfqpoint{3.284909in}{2.655976in}}{\pgfqpoint{3.281636in}{2.648076in}}{\pgfqpoint{3.281636in}{2.639839in}}%
\pgfpathcurveto{\pgfqpoint{3.281636in}{2.631603in}}{\pgfqpoint{3.284909in}{2.623703in}}{\pgfqpoint{3.290733in}{2.617879in}}%
\pgfpathcurveto{\pgfqpoint{3.296556in}{2.612055in}}{\pgfqpoint{3.304457in}{2.608783in}}{\pgfqpoint{3.312693in}{2.608783in}}%
\pgfpathclose%
\pgfusepath{stroke,fill}%
\end{pgfscope}%
\begin{pgfscope}%
\pgfpathrectangle{\pgfqpoint{0.100000in}{0.212622in}}{\pgfqpoint{3.696000in}{3.696000in}}%
\pgfusepath{clip}%
\pgfsetbuttcap%
\pgfsetroundjoin%
\definecolor{currentfill}{rgb}{0.121569,0.466667,0.705882}%
\pgfsetfillcolor{currentfill}%
\pgfsetfillopacity{0.695422}%
\pgfsetlinewidth{1.003750pt}%
\definecolor{currentstroke}{rgb}{0.121569,0.466667,0.705882}%
\pgfsetstrokecolor{currentstroke}%
\pgfsetstrokeopacity{0.695422}%
\pgfsetdash{}{0pt}%
\pgfpathmoveto{\pgfqpoint{3.312694in}{2.608782in}}%
\pgfpathcurveto{\pgfqpoint{3.320930in}{2.608782in}}{\pgfqpoint{3.328830in}{2.612055in}}{\pgfqpoint{3.334654in}{2.617879in}}%
\pgfpathcurveto{\pgfqpoint{3.340478in}{2.623703in}}{\pgfqpoint{3.343750in}{2.631603in}}{\pgfqpoint{3.343750in}{2.639839in}}%
\pgfpathcurveto{\pgfqpoint{3.343750in}{2.648075in}}{\pgfqpoint{3.340478in}{2.655975in}}{\pgfqpoint{3.334654in}{2.661799in}}%
\pgfpathcurveto{\pgfqpoint{3.328830in}{2.667623in}}{\pgfqpoint{3.320930in}{2.670895in}}{\pgfqpoint{3.312694in}{2.670895in}}%
\pgfpathcurveto{\pgfqpoint{3.304457in}{2.670895in}}{\pgfqpoint{3.296557in}{2.667623in}}{\pgfqpoint{3.290733in}{2.661799in}}%
\pgfpathcurveto{\pgfqpoint{3.284909in}{2.655975in}}{\pgfqpoint{3.281637in}{2.648075in}}{\pgfqpoint{3.281637in}{2.639839in}}%
\pgfpathcurveto{\pgfqpoint{3.281637in}{2.631603in}}{\pgfqpoint{3.284909in}{2.623703in}}{\pgfqpoint{3.290733in}{2.617879in}}%
\pgfpathcurveto{\pgfqpoint{3.296557in}{2.612055in}}{\pgfqpoint{3.304457in}{2.608782in}}{\pgfqpoint{3.312694in}{2.608782in}}%
\pgfpathclose%
\pgfusepath{stroke,fill}%
\end{pgfscope}%
\begin{pgfscope}%
\pgfpathrectangle{\pgfqpoint{0.100000in}{0.212622in}}{\pgfqpoint{3.696000in}{3.696000in}}%
\pgfusepath{clip}%
\pgfsetbuttcap%
\pgfsetroundjoin%
\definecolor{currentfill}{rgb}{0.121569,0.466667,0.705882}%
\pgfsetfillcolor{currentfill}%
\pgfsetfillopacity{0.695422}%
\pgfsetlinewidth{1.003750pt}%
\definecolor{currentstroke}{rgb}{0.121569,0.466667,0.705882}%
\pgfsetstrokecolor{currentstroke}%
\pgfsetstrokeopacity{0.695422}%
\pgfsetdash{}{0pt}%
\pgfpathmoveto{\pgfqpoint{3.312695in}{2.608782in}}%
\pgfpathcurveto{\pgfqpoint{3.320931in}{2.608782in}}{\pgfqpoint{3.328831in}{2.612054in}}{\pgfqpoint{3.334655in}{2.617878in}}%
\pgfpathcurveto{\pgfqpoint{3.340479in}{2.623702in}}{\pgfqpoint{3.343751in}{2.631602in}}{\pgfqpoint{3.343751in}{2.639839in}}%
\pgfpathcurveto{\pgfqpoint{3.343751in}{2.648075in}}{\pgfqpoint{3.340479in}{2.655975in}}{\pgfqpoint{3.334655in}{2.661799in}}%
\pgfpathcurveto{\pgfqpoint{3.328831in}{2.667623in}}{\pgfqpoint{3.320931in}{2.670895in}}{\pgfqpoint{3.312695in}{2.670895in}}%
\pgfpathcurveto{\pgfqpoint{3.304459in}{2.670895in}}{\pgfqpoint{3.296559in}{2.667623in}}{\pgfqpoint{3.290735in}{2.661799in}}%
\pgfpathcurveto{\pgfqpoint{3.284911in}{2.655975in}}{\pgfqpoint{3.281638in}{2.648075in}}{\pgfqpoint{3.281638in}{2.639839in}}%
\pgfpathcurveto{\pgfqpoint{3.281638in}{2.631602in}}{\pgfqpoint{3.284911in}{2.623702in}}{\pgfqpoint{3.290735in}{2.617878in}}%
\pgfpathcurveto{\pgfqpoint{3.296559in}{2.612054in}}{\pgfqpoint{3.304459in}{2.608782in}}{\pgfqpoint{3.312695in}{2.608782in}}%
\pgfpathclose%
\pgfusepath{stroke,fill}%
\end{pgfscope}%
\begin{pgfscope}%
\pgfpathrectangle{\pgfqpoint{0.100000in}{0.212622in}}{\pgfqpoint{3.696000in}{3.696000in}}%
\pgfusepath{clip}%
\pgfsetbuttcap%
\pgfsetroundjoin%
\definecolor{currentfill}{rgb}{0.121569,0.466667,0.705882}%
\pgfsetfillcolor{currentfill}%
\pgfsetfillopacity{0.695423}%
\pgfsetlinewidth{1.003750pt}%
\definecolor{currentstroke}{rgb}{0.121569,0.466667,0.705882}%
\pgfsetstrokecolor{currentstroke}%
\pgfsetstrokeopacity{0.695423}%
\pgfsetdash{}{0pt}%
\pgfpathmoveto{\pgfqpoint{3.312697in}{2.608781in}}%
\pgfpathcurveto{\pgfqpoint{3.320934in}{2.608781in}}{\pgfqpoint{3.328834in}{2.612053in}}{\pgfqpoint{3.334658in}{2.617877in}}%
\pgfpathcurveto{\pgfqpoint{3.340482in}{2.623701in}}{\pgfqpoint{3.343754in}{2.631601in}}{\pgfqpoint{3.343754in}{2.639837in}}%
\pgfpathcurveto{\pgfqpoint{3.343754in}{2.648073in}}{\pgfqpoint{3.340482in}{2.655974in}}{\pgfqpoint{3.334658in}{2.661797in}}%
\pgfpathcurveto{\pgfqpoint{3.328834in}{2.667621in}}{\pgfqpoint{3.320934in}{2.670894in}}{\pgfqpoint{3.312697in}{2.670894in}}%
\pgfpathcurveto{\pgfqpoint{3.304461in}{2.670894in}}{\pgfqpoint{3.296561in}{2.667621in}}{\pgfqpoint{3.290737in}{2.661797in}}%
\pgfpathcurveto{\pgfqpoint{3.284913in}{2.655974in}}{\pgfqpoint{3.281641in}{2.648073in}}{\pgfqpoint{3.281641in}{2.639837in}}%
\pgfpathcurveto{\pgfqpoint{3.281641in}{2.631601in}}{\pgfqpoint{3.284913in}{2.623701in}}{\pgfqpoint{3.290737in}{2.617877in}}%
\pgfpathcurveto{\pgfqpoint{3.296561in}{2.612053in}}{\pgfqpoint{3.304461in}{2.608781in}}{\pgfqpoint{3.312697in}{2.608781in}}%
\pgfpathclose%
\pgfusepath{stroke,fill}%
\end{pgfscope}%
\begin{pgfscope}%
\pgfpathrectangle{\pgfqpoint{0.100000in}{0.212622in}}{\pgfqpoint{3.696000in}{3.696000in}}%
\pgfusepath{clip}%
\pgfsetbuttcap%
\pgfsetroundjoin%
\definecolor{currentfill}{rgb}{0.121569,0.466667,0.705882}%
\pgfsetfillcolor{currentfill}%
\pgfsetfillopacity{0.695424}%
\pgfsetlinewidth{1.003750pt}%
\definecolor{currentstroke}{rgb}{0.121569,0.466667,0.705882}%
\pgfsetstrokecolor{currentstroke}%
\pgfsetstrokeopacity{0.695424}%
\pgfsetdash{}{0pt}%
\pgfpathmoveto{\pgfqpoint{3.312702in}{2.608780in}}%
\pgfpathcurveto{\pgfqpoint{3.320938in}{2.608780in}}{\pgfqpoint{3.328838in}{2.612052in}}{\pgfqpoint{3.334662in}{2.617876in}}%
\pgfpathcurveto{\pgfqpoint{3.340486in}{2.623700in}}{\pgfqpoint{3.343758in}{2.631600in}}{\pgfqpoint{3.343758in}{2.639836in}}%
\pgfpathcurveto{\pgfqpoint{3.343758in}{2.648073in}}{\pgfqpoint{3.340486in}{2.655973in}}{\pgfqpoint{3.334662in}{2.661797in}}%
\pgfpathcurveto{\pgfqpoint{3.328838in}{2.667621in}}{\pgfqpoint{3.320938in}{2.670893in}}{\pgfqpoint{3.312702in}{2.670893in}}%
\pgfpathcurveto{\pgfqpoint{3.304466in}{2.670893in}}{\pgfqpoint{3.296566in}{2.667621in}}{\pgfqpoint{3.290742in}{2.661797in}}%
\pgfpathcurveto{\pgfqpoint{3.284918in}{2.655973in}}{\pgfqpoint{3.281645in}{2.648073in}}{\pgfqpoint{3.281645in}{2.639836in}}%
\pgfpathcurveto{\pgfqpoint{3.281645in}{2.631600in}}{\pgfqpoint{3.284918in}{2.623700in}}{\pgfqpoint{3.290742in}{2.617876in}}%
\pgfpathcurveto{\pgfqpoint{3.296566in}{2.612052in}}{\pgfqpoint{3.304466in}{2.608780in}}{\pgfqpoint{3.312702in}{2.608780in}}%
\pgfpathclose%
\pgfusepath{stroke,fill}%
\end{pgfscope}%
\begin{pgfscope}%
\pgfpathrectangle{\pgfqpoint{0.100000in}{0.212622in}}{\pgfqpoint{3.696000in}{3.696000in}}%
\pgfusepath{clip}%
\pgfsetbuttcap%
\pgfsetroundjoin%
\definecolor{currentfill}{rgb}{0.121569,0.466667,0.705882}%
\pgfsetfillcolor{currentfill}%
\pgfsetfillopacity{0.695426}%
\pgfsetlinewidth{1.003750pt}%
\definecolor{currentstroke}{rgb}{0.121569,0.466667,0.705882}%
\pgfsetstrokecolor{currentstroke}%
\pgfsetstrokeopacity{0.695426}%
\pgfsetdash{}{0pt}%
\pgfpathmoveto{\pgfqpoint{3.312711in}{2.608779in}}%
\pgfpathcurveto{\pgfqpoint{3.320947in}{2.608779in}}{\pgfqpoint{3.328847in}{2.612051in}}{\pgfqpoint{3.334671in}{2.617875in}}%
\pgfpathcurveto{\pgfqpoint{3.340495in}{2.623699in}}{\pgfqpoint{3.343767in}{2.631599in}}{\pgfqpoint{3.343767in}{2.639835in}}%
\pgfpathcurveto{\pgfqpoint{3.343767in}{2.648071in}}{\pgfqpoint{3.340495in}{2.655971in}}{\pgfqpoint{3.334671in}{2.661795in}}%
\pgfpathcurveto{\pgfqpoint{3.328847in}{2.667619in}}{\pgfqpoint{3.320947in}{2.670892in}}{\pgfqpoint{3.312711in}{2.670892in}}%
\pgfpathcurveto{\pgfqpoint{3.304474in}{2.670892in}}{\pgfqpoint{3.296574in}{2.667619in}}{\pgfqpoint{3.290750in}{2.661795in}}%
\pgfpathcurveto{\pgfqpoint{3.284927in}{2.655971in}}{\pgfqpoint{3.281654in}{2.648071in}}{\pgfqpoint{3.281654in}{2.639835in}}%
\pgfpathcurveto{\pgfqpoint{3.281654in}{2.631599in}}{\pgfqpoint{3.284927in}{2.623699in}}{\pgfqpoint{3.290750in}{2.617875in}}%
\pgfpathcurveto{\pgfqpoint{3.296574in}{2.612051in}}{\pgfqpoint{3.304474in}{2.608779in}}{\pgfqpoint{3.312711in}{2.608779in}}%
\pgfpathclose%
\pgfusepath{stroke,fill}%
\end{pgfscope}%
\begin{pgfscope}%
\pgfpathrectangle{\pgfqpoint{0.100000in}{0.212622in}}{\pgfqpoint{3.696000in}{3.696000in}}%
\pgfusepath{clip}%
\pgfsetbuttcap%
\pgfsetroundjoin%
\definecolor{currentfill}{rgb}{0.121569,0.466667,0.705882}%
\pgfsetfillcolor{currentfill}%
\pgfsetfillopacity{0.695429}%
\pgfsetlinewidth{1.003750pt}%
\definecolor{currentstroke}{rgb}{0.121569,0.466667,0.705882}%
\pgfsetstrokecolor{currentstroke}%
\pgfsetstrokeopacity{0.695429}%
\pgfsetdash{}{0pt}%
\pgfpathmoveto{\pgfqpoint{3.312726in}{2.608774in}}%
\pgfpathcurveto{\pgfqpoint{3.320962in}{2.608774in}}{\pgfqpoint{3.328862in}{2.612047in}}{\pgfqpoint{3.334686in}{2.617871in}}%
\pgfpathcurveto{\pgfqpoint{3.340510in}{2.623695in}}{\pgfqpoint{3.343782in}{2.631595in}}{\pgfqpoint{3.343782in}{2.639831in}}%
\pgfpathcurveto{\pgfqpoint{3.343782in}{2.648067in}}{\pgfqpoint{3.340510in}{2.655967in}}{\pgfqpoint{3.334686in}{2.661791in}}%
\pgfpathcurveto{\pgfqpoint{3.328862in}{2.667615in}}{\pgfqpoint{3.320962in}{2.670887in}}{\pgfqpoint{3.312726in}{2.670887in}}%
\pgfpathcurveto{\pgfqpoint{3.304489in}{2.670887in}}{\pgfqpoint{3.296589in}{2.667615in}}{\pgfqpoint{3.290766in}{2.661791in}}%
\pgfpathcurveto{\pgfqpoint{3.284942in}{2.655967in}}{\pgfqpoint{3.281669in}{2.648067in}}{\pgfqpoint{3.281669in}{2.639831in}}%
\pgfpathcurveto{\pgfqpoint{3.281669in}{2.631595in}}{\pgfqpoint{3.284942in}{2.623695in}}{\pgfqpoint{3.290766in}{2.617871in}}%
\pgfpathcurveto{\pgfqpoint{3.296589in}{2.612047in}}{\pgfqpoint{3.304489in}{2.608774in}}{\pgfqpoint{3.312726in}{2.608774in}}%
\pgfpathclose%
\pgfusepath{stroke,fill}%
\end{pgfscope}%
\begin{pgfscope}%
\pgfpathrectangle{\pgfqpoint{0.100000in}{0.212622in}}{\pgfqpoint{3.696000in}{3.696000in}}%
\pgfusepath{clip}%
\pgfsetbuttcap%
\pgfsetroundjoin%
\definecolor{currentfill}{rgb}{0.121569,0.466667,0.705882}%
\pgfsetfillcolor{currentfill}%
\pgfsetfillopacity{0.695435}%
\pgfsetlinewidth{1.003750pt}%
\definecolor{currentstroke}{rgb}{0.121569,0.466667,0.705882}%
\pgfsetstrokecolor{currentstroke}%
\pgfsetstrokeopacity{0.695435}%
\pgfsetdash{}{0pt}%
\pgfpathmoveto{\pgfqpoint{3.312755in}{2.608767in}}%
\pgfpathcurveto{\pgfqpoint{3.320991in}{2.608767in}}{\pgfqpoint{3.328891in}{2.612039in}}{\pgfqpoint{3.334715in}{2.617863in}}%
\pgfpathcurveto{\pgfqpoint{3.340539in}{2.623687in}}{\pgfqpoint{3.343811in}{2.631587in}}{\pgfqpoint{3.343811in}{2.639824in}}%
\pgfpathcurveto{\pgfqpoint{3.343811in}{2.648060in}}{\pgfqpoint{3.340539in}{2.655960in}}{\pgfqpoint{3.334715in}{2.661784in}}%
\pgfpathcurveto{\pgfqpoint{3.328891in}{2.667608in}}{\pgfqpoint{3.320991in}{2.670880in}}{\pgfqpoint{3.312755in}{2.670880in}}%
\pgfpathcurveto{\pgfqpoint{3.304518in}{2.670880in}}{\pgfqpoint{3.296618in}{2.667608in}}{\pgfqpoint{3.290794in}{2.661784in}}%
\pgfpathcurveto{\pgfqpoint{3.284971in}{2.655960in}}{\pgfqpoint{3.281698in}{2.648060in}}{\pgfqpoint{3.281698in}{2.639824in}}%
\pgfpathcurveto{\pgfqpoint{3.281698in}{2.631587in}}{\pgfqpoint{3.284971in}{2.623687in}}{\pgfqpoint{3.290794in}{2.617863in}}%
\pgfpathcurveto{\pgfqpoint{3.296618in}{2.612039in}}{\pgfqpoint{3.304518in}{2.608767in}}{\pgfqpoint{3.312755in}{2.608767in}}%
\pgfpathclose%
\pgfusepath{stroke,fill}%
\end{pgfscope}%
\begin{pgfscope}%
\pgfpathrectangle{\pgfqpoint{0.100000in}{0.212622in}}{\pgfqpoint{3.696000in}{3.696000in}}%
\pgfusepath{clip}%
\pgfsetbuttcap%
\pgfsetroundjoin%
\definecolor{currentfill}{rgb}{0.121569,0.466667,0.705882}%
\pgfsetfillcolor{currentfill}%
\pgfsetfillopacity{0.695448}%
\pgfsetlinewidth{1.003750pt}%
\definecolor{currentstroke}{rgb}{0.121569,0.466667,0.705882}%
\pgfsetstrokecolor{currentstroke}%
\pgfsetstrokeopacity{0.695448}%
\pgfsetdash{}{0pt}%
\pgfpathmoveto{\pgfqpoint{3.312805in}{2.608754in}}%
\pgfpathcurveto{\pgfqpoint{3.321041in}{2.608754in}}{\pgfqpoint{3.328941in}{2.612027in}}{\pgfqpoint{3.334765in}{2.617851in}}%
\pgfpathcurveto{\pgfqpoint{3.340589in}{2.623675in}}{\pgfqpoint{3.343861in}{2.631575in}}{\pgfqpoint{3.343861in}{2.639811in}}%
\pgfpathcurveto{\pgfqpoint{3.343861in}{2.648047in}}{\pgfqpoint{3.340589in}{2.655947in}}{\pgfqpoint{3.334765in}{2.661771in}}%
\pgfpathcurveto{\pgfqpoint{3.328941in}{2.667595in}}{\pgfqpoint{3.321041in}{2.670867in}}{\pgfqpoint{3.312805in}{2.670867in}}%
\pgfpathcurveto{\pgfqpoint{3.304569in}{2.670867in}}{\pgfqpoint{3.296668in}{2.667595in}}{\pgfqpoint{3.290845in}{2.661771in}}%
\pgfpathcurveto{\pgfqpoint{3.285021in}{2.655947in}}{\pgfqpoint{3.281748in}{2.648047in}}{\pgfqpoint{3.281748in}{2.639811in}}%
\pgfpathcurveto{\pgfqpoint{3.281748in}{2.631575in}}{\pgfqpoint{3.285021in}{2.623675in}}{\pgfqpoint{3.290845in}{2.617851in}}%
\pgfpathcurveto{\pgfqpoint{3.296668in}{2.612027in}}{\pgfqpoint{3.304569in}{2.608754in}}{\pgfqpoint{3.312805in}{2.608754in}}%
\pgfpathclose%
\pgfusepath{stroke,fill}%
\end{pgfscope}%
\begin{pgfscope}%
\pgfpathrectangle{\pgfqpoint{0.100000in}{0.212622in}}{\pgfqpoint{3.696000in}{3.696000in}}%
\pgfusepath{clip}%
\pgfsetbuttcap%
\pgfsetroundjoin%
\definecolor{currentfill}{rgb}{0.121569,0.466667,0.705882}%
\pgfsetfillcolor{currentfill}%
\pgfsetfillopacity{0.695470}%
\pgfsetlinewidth{1.003750pt}%
\definecolor{currentstroke}{rgb}{0.121569,0.466667,0.705882}%
\pgfsetstrokecolor{currentstroke}%
\pgfsetstrokeopacity{0.695470}%
\pgfsetdash{}{0pt}%
\pgfpathmoveto{\pgfqpoint{3.312900in}{2.608743in}}%
\pgfpathcurveto{\pgfqpoint{3.321136in}{2.608743in}}{\pgfqpoint{3.329036in}{2.612015in}}{\pgfqpoint{3.334860in}{2.617839in}}%
\pgfpathcurveto{\pgfqpoint{3.340684in}{2.623663in}}{\pgfqpoint{3.343957in}{2.631563in}}{\pgfqpoint{3.343957in}{2.639799in}}%
\pgfpathcurveto{\pgfqpoint{3.343957in}{2.648035in}}{\pgfqpoint{3.340684in}{2.655935in}}{\pgfqpoint{3.334860in}{2.661759in}}%
\pgfpathcurveto{\pgfqpoint{3.329036in}{2.667583in}}{\pgfqpoint{3.321136in}{2.670856in}}{\pgfqpoint{3.312900in}{2.670856in}}%
\pgfpathcurveto{\pgfqpoint{3.304664in}{2.670856in}}{\pgfqpoint{3.296764in}{2.667583in}}{\pgfqpoint{3.290940in}{2.661759in}}%
\pgfpathcurveto{\pgfqpoint{3.285116in}{2.655935in}}{\pgfqpoint{3.281844in}{2.648035in}}{\pgfqpoint{3.281844in}{2.639799in}}%
\pgfpathcurveto{\pgfqpoint{3.281844in}{2.631563in}}{\pgfqpoint{3.285116in}{2.623663in}}{\pgfqpoint{3.290940in}{2.617839in}}%
\pgfpathcurveto{\pgfqpoint{3.296764in}{2.612015in}}{\pgfqpoint{3.304664in}{2.608743in}}{\pgfqpoint{3.312900in}{2.608743in}}%
\pgfpathclose%
\pgfusepath{stroke,fill}%
\end{pgfscope}%
\begin{pgfscope}%
\pgfpathrectangle{\pgfqpoint{0.100000in}{0.212622in}}{\pgfqpoint{3.696000in}{3.696000in}}%
\pgfusepath{clip}%
\pgfsetbuttcap%
\pgfsetroundjoin%
\definecolor{currentfill}{rgb}{0.121569,0.466667,0.705882}%
\pgfsetfillcolor{currentfill}%
\pgfsetfillopacity{0.695496}%
\pgfsetlinewidth{1.003750pt}%
\definecolor{currentstroke}{rgb}{0.121569,0.466667,0.705882}%
\pgfsetstrokecolor{currentstroke}%
\pgfsetstrokeopacity{0.695496}%
\pgfsetdash{}{0pt}%
\pgfpathmoveto{\pgfqpoint{3.313086in}{2.608708in}}%
\pgfpathcurveto{\pgfqpoint{3.321322in}{2.608708in}}{\pgfqpoint{3.329222in}{2.611981in}}{\pgfqpoint{3.335046in}{2.617805in}}%
\pgfpathcurveto{\pgfqpoint{3.340870in}{2.623629in}}{\pgfqpoint{3.344143in}{2.631529in}}{\pgfqpoint{3.344143in}{2.639765in}}%
\pgfpathcurveto{\pgfqpoint{3.344143in}{2.648001in}}{\pgfqpoint{3.340870in}{2.655901in}}{\pgfqpoint{3.335046in}{2.661725in}}%
\pgfpathcurveto{\pgfqpoint{3.329222in}{2.667549in}}{\pgfqpoint{3.321322in}{2.670821in}}{\pgfqpoint{3.313086in}{2.670821in}}%
\pgfpathcurveto{\pgfqpoint{3.304850in}{2.670821in}}{\pgfqpoint{3.296950in}{2.667549in}}{\pgfqpoint{3.291126in}{2.661725in}}%
\pgfpathcurveto{\pgfqpoint{3.285302in}{2.655901in}}{\pgfqpoint{3.282030in}{2.648001in}}{\pgfqpoint{3.282030in}{2.639765in}}%
\pgfpathcurveto{\pgfqpoint{3.282030in}{2.631529in}}{\pgfqpoint{3.285302in}{2.623629in}}{\pgfqpoint{3.291126in}{2.617805in}}%
\pgfpathcurveto{\pgfqpoint{3.296950in}{2.611981in}}{\pgfqpoint{3.304850in}{2.608708in}}{\pgfqpoint{3.313086in}{2.608708in}}%
\pgfpathclose%
\pgfusepath{stroke,fill}%
\end{pgfscope}%
\begin{pgfscope}%
\pgfpathrectangle{\pgfqpoint{0.100000in}{0.212622in}}{\pgfqpoint{3.696000in}{3.696000in}}%
\pgfusepath{clip}%
\pgfsetbuttcap%
\pgfsetroundjoin%
\definecolor{currentfill}{rgb}{0.121569,0.466667,0.705882}%
\pgfsetfillcolor{currentfill}%
\pgfsetfillopacity{0.695555}%
\pgfsetlinewidth{1.003750pt}%
\definecolor{currentstroke}{rgb}{0.121569,0.466667,0.705882}%
\pgfsetstrokecolor{currentstroke}%
\pgfsetstrokeopacity{0.695555}%
\pgfsetdash{}{0pt}%
\pgfpathmoveto{\pgfqpoint{3.313411in}{2.608644in}}%
\pgfpathcurveto{\pgfqpoint{3.321648in}{2.608644in}}{\pgfqpoint{3.329548in}{2.611917in}}{\pgfqpoint{3.335372in}{2.617741in}}%
\pgfpathcurveto{\pgfqpoint{3.341196in}{2.623564in}}{\pgfqpoint{3.344468in}{2.631465in}}{\pgfqpoint{3.344468in}{2.639701in}}%
\pgfpathcurveto{\pgfqpoint{3.344468in}{2.647937in}}{\pgfqpoint{3.341196in}{2.655837in}}{\pgfqpoint{3.335372in}{2.661661in}}%
\pgfpathcurveto{\pgfqpoint{3.329548in}{2.667485in}}{\pgfqpoint{3.321648in}{2.670757in}}{\pgfqpoint{3.313411in}{2.670757in}}%
\pgfpathcurveto{\pgfqpoint{3.305175in}{2.670757in}}{\pgfqpoint{3.297275in}{2.667485in}}{\pgfqpoint{3.291451in}{2.661661in}}%
\pgfpathcurveto{\pgfqpoint{3.285627in}{2.655837in}}{\pgfqpoint{3.282355in}{2.647937in}}{\pgfqpoint{3.282355in}{2.639701in}}%
\pgfpathcurveto{\pgfqpoint{3.282355in}{2.631465in}}{\pgfqpoint{3.285627in}{2.623564in}}{\pgfqpoint{3.291451in}{2.617741in}}%
\pgfpathcurveto{\pgfqpoint{3.297275in}{2.611917in}}{\pgfqpoint{3.305175in}{2.608644in}}{\pgfqpoint{3.313411in}{2.608644in}}%
\pgfpathclose%
\pgfusepath{stroke,fill}%
\end{pgfscope}%
\begin{pgfscope}%
\pgfpathrectangle{\pgfqpoint{0.100000in}{0.212622in}}{\pgfqpoint{3.696000in}{3.696000in}}%
\pgfusepath{clip}%
\pgfsetbuttcap%
\pgfsetroundjoin%
\definecolor{currentfill}{rgb}{0.121569,0.466667,0.705882}%
\pgfsetfillcolor{currentfill}%
\pgfsetfillopacity{0.695656}%
\pgfsetlinewidth{1.003750pt}%
\definecolor{currentstroke}{rgb}{0.121569,0.466667,0.705882}%
\pgfsetstrokecolor{currentstroke}%
\pgfsetstrokeopacity{0.695656}%
\pgfsetdash{}{0pt}%
\pgfpathmoveto{\pgfqpoint{3.314007in}{2.608518in}}%
\pgfpathcurveto{\pgfqpoint{3.322243in}{2.608518in}}{\pgfqpoint{3.330143in}{2.611790in}}{\pgfqpoint{3.335967in}{2.617614in}}%
\pgfpathcurveto{\pgfqpoint{3.341791in}{2.623438in}}{\pgfqpoint{3.345063in}{2.631338in}}{\pgfqpoint{3.345063in}{2.639575in}}%
\pgfpathcurveto{\pgfqpoint{3.345063in}{2.647811in}}{\pgfqpoint{3.341791in}{2.655711in}}{\pgfqpoint{3.335967in}{2.661535in}}%
\pgfpathcurveto{\pgfqpoint{3.330143in}{2.667359in}}{\pgfqpoint{3.322243in}{2.670631in}}{\pgfqpoint{3.314007in}{2.670631in}}%
\pgfpathcurveto{\pgfqpoint{3.305771in}{2.670631in}}{\pgfqpoint{3.297871in}{2.667359in}}{\pgfqpoint{3.292047in}{2.661535in}}%
\pgfpathcurveto{\pgfqpoint{3.286223in}{2.655711in}}{\pgfqpoint{3.282950in}{2.647811in}}{\pgfqpoint{3.282950in}{2.639575in}}%
\pgfpathcurveto{\pgfqpoint{3.282950in}{2.631338in}}{\pgfqpoint{3.286223in}{2.623438in}}{\pgfqpoint{3.292047in}{2.617614in}}%
\pgfpathcurveto{\pgfqpoint{3.297871in}{2.611790in}}{\pgfqpoint{3.305771in}{2.608518in}}{\pgfqpoint{3.314007in}{2.608518in}}%
\pgfpathclose%
\pgfusepath{stroke,fill}%
\end{pgfscope}%
\begin{pgfscope}%
\pgfpathrectangle{\pgfqpoint{0.100000in}{0.212622in}}{\pgfqpoint{3.696000in}{3.696000in}}%
\pgfusepath{clip}%
\pgfsetbuttcap%
\pgfsetroundjoin%
\definecolor{currentfill}{rgb}{0.121569,0.466667,0.705882}%
\pgfsetfillcolor{currentfill}%
\pgfsetfillopacity{0.695673}%
\pgfsetlinewidth{1.003750pt}%
\definecolor{currentstroke}{rgb}{0.121569,0.466667,0.705882}%
\pgfsetstrokecolor{currentstroke}%
\pgfsetstrokeopacity{0.695673}%
\pgfsetdash{}{0pt}%
\pgfpathmoveto{\pgfqpoint{1.369566in}{2.299649in}}%
\pgfpathcurveto{\pgfqpoint{1.377802in}{2.299649in}}{\pgfqpoint{1.385702in}{2.302921in}}{\pgfqpoint{1.391526in}{2.308745in}}%
\pgfpathcurveto{\pgfqpoint{1.397350in}{2.314569in}}{\pgfqpoint{1.400622in}{2.322469in}}{\pgfqpoint{1.400622in}{2.330705in}}%
\pgfpathcurveto{\pgfqpoint{1.400622in}{2.338942in}}{\pgfqpoint{1.397350in}{2.346842in}}{\pgfqpoint{1.391526in}{2.352666in}}%
\pgfpathcurveto{\pgfqpoint{1.385702in}{2.358489in}}{\pgfqpoint{1.377802in}{2.361762in}}{\pgfqpoint{1.369566in}{2.361762in}}%
\pgfpathcurveto{\pgfqpoint{1.361329in}{2.361762in}}{\pgfqpoint{1.353429in}{2.358489in}}{\pgfqpoint{1.347605in}{2.352666in}}%
\pgfpathcurveto{\pgfqpoint{1.341781in}{2.346842in}}{\pgfqpoint{1.338509in}{2.338942in}}{\pgfqpoint{1.338509in}{2.330705in}}%
\pgfpathcurveto{\pgfqpoint{1.338509in}{2.322469in}}{\pgfqpoint{1.341781in}{2.314569in}}{\pgfqpoint{1.347605in}{2.308745in}}%
\pgfpathcurveto{\pgfqpoint{1.353429in}{2.302921in}}{\pgfqpoint{1.361329in}{2.299649in}}{\pgfqpoint{1.369566in}{2.299649in}}%
\pgfpathclose%
\pgfusepath{stroke,fill}%
\end{pgfscope}%
\begin{pgfscope}%
\pgfpathrectangle{\pgfqpoint{0.100000in}{0.212622in}}{\pgfqpoint{3.696000in}{3.696000in}}%
\pgfusepath{clip}%
\pgfsetbuttcap%
\pgfsetroundjoin%
\definecolor{currentfill}{rgb}{0.121569,0.466667,0.705882}%
\pgfsetfillcolor{currentfill}%
\pgfsetfillopacity{0.695854}%
\pgfsetlinewidth{1.003750pt}%
\definecolor{currentstroke}{rgb}{0.121569,0.466667,0.705882}%
\pgfsetstrokecolor{currentstroke}%
\pgfsetstrokeopacity{0.695854}%
\pgfsetdash{}{0pt}%
\pgfpathmoveto{\pgfqpoint{3.315100in}{2.608381in}}%
\pgfpathcurveto{\pgfqpoint{3.323336in}{2.608381in}}{\pgfqpoint{3.331236in}{2.611653in}}{\pgfqpoint{3.337060in}{2.617477in}}%
\pgfpathcurveto{\pgfqpoint{3.342884in}{2.623301in}}{\pgfqpoint{3.346156in}{2.631201in}}{\pgfqpoint{3.346156in}{2.639437in}}%
\pgfpathcurveto{\pgfqpoint{3.346156in}{2.647674in}}{\pgfqpoint{3.342884in}{2.655574in}}{\pgfqpoint{3.337060in}{2.661398in}}%
\pgfpathcurveto{\pgfqpoint{3.331236in}{2.667222in}}{\pgfqpoint{3.323336in}{2.670494in}}{\pgfqpoint{3.315100in}{2.670494in}}%
\pgfpathcurveto{\pgfqpoint{3.306864in}{2.670494in}}{\pgfqpoint{3.298964in}{2.667222in}}{\pgfqpoint{3.293140in}{2.661398in}}%
\pgfpathcurveto{\pgfqpoint{3.287316in}{2.655574in}}{\pgfqpoint{3.284043in}{2.647674in}}{\pgfqpoint{3.284043in}{2.639437in}}%
\pgfpathcurveto{\pgfqpoint{3.284043in}{2.631201in}}{\pgfqpoint{3.287316in}{2.623301in}}{\pgfqpoint{3.293140in}{2.617477in}}%
\pgfpathcurveto{\pgfqpoint{3.298964in}{2.611653in}}{\pgfqpoint{3.306864in}{2.608381in}}{\pgfqpoint{3.315100in}{2.608381in}}%
\pgfpathclose%
\pgfusepath{stroke,fill}%
\end{pgfscope}%
\begin{pgfscope}%
\pgfpathrectangle{\pgfqpoint{0.100000in}{0.212622in}}{\pgfqpoint{3.696000in}{3.696000in}}%
\pgfusepath{clip}%
\pgfsetbuttcap%
\pgfsetroundjoin%
\definecolor{currentfill}{rgb}{0.121569,0.466667,0.705882}%
\pgfsetfillcolor{currentfill}%
\pgfsetfillopacity{0.696167}%
\pgfsetlinewidth{1.003750pt}%
\definecolor{currentstroke}{rgb}{0.121569,0.466667,0.705882}%
\pgfsetstrokecolor{currentstroke}%
\pgfsetstrokeopacity{0.696167}%
\pgfsetdash{}{0pt}%
\pgfpathmoveto{\pgfqpoint{2.182185in}{3.002572in}}%
\pgfpathcurveto{\pgfqpoint{2.190421in}{3.002572in}}{\pgfqpoint{2.198321in}{3.005844in}}{\pgfqpoint{2.204145in}{3.011668in}}%
\pgfpathcurveto{\pgfqpoint{2.209969in}{3.017492in}}{\pgfqpoint{2.213241in}{3.025392in}}{\pgfqpoint{2.213241in}{3.033628in}}%
\pgfpathcurveto{\pgfqpoint{2.213241in}{3.041865in}}{\pgfqpoint{2.209969in}{3.049765in}}{\pgfqpoint{2.204145in}{3.055589in}}%
\pgfpathcurveto{\pgfqpoint{2.198321in}{3.061412in}}{\pgfqpoint{2.190421in}{3.064685in}}{\pgfqpoint{2.182185in}{3.064685in}}%
\pgfpathcurveto{\pgfqpoint{2.173949in}{3.064685in}}{\pgfqpoint{2.166049in}{3.061412in}}{\pgfqpoint{2.160225in}{3.055589in}}%
\pgfpathcurveto{\pgfqpoint{2.154401in}{3.049765in}}{\pgfqpoint{2.151128in}{3.041865in}}{\pgfqpoint{2.151128in}{3.033628in}}%
\pgfpathcurveto{\pgfqpoint{2.151128in}{3.025392in}}{\pgfqpoint{2.154401in}{3.017492in}}{\pgfqpoint{2.160225in}{3.011668in}}%
\pgfpathcurveto{\pgfqpoint{2.166049in}{3.005844in}}{\pgfqpoint{2.173949in}{3.002572in}}{\pgfqpoint{2.182185in}{3.002572in}}%
\pgfpathclose%
\pgfusepath{stroke,fill}%
\end{pgfscope}%
\begin{pgfscope}%
\pgfpathrectangle{\pgfqpoint{0.100000in}{0.212622in}}{\pgfqpoint{3.696000in}{3.696000in}}%
\pgfusepath{clip}%
\pgfsetbuttcap%
\pgfsetroundjoin%
\definecolor{currentfill}{rgb}{0.121569,0.466667,0.705882}%
\pgfsetfillcolor{currentfill}%
\pgfsetfillopacity{0.696205}%
\pgfsetlinewidth{1.003750pt}%
\definecolor{currentstroke}{rgb}{0.121569,0.466667,0.705882}%
\pgfsetstrokecolor{currentstroke}%
\pgfsetstrokeopacity{0.696205}%
\pgfsetdash{}{0pt}%
\pgfpathmoveto{\pgfqpoint{3.317141in}{2.608308in}}%
\pgfpathcurveto{\pgfqpoint{3.325377in}{2.608308in}}{\pgfqpoint{3.333277in}{2.611580in}}{\pgfqpoint{3.339101in}{2.617404in}}%
\pgfpathcurveto{\pgfqpoint{3.344925in}{2.623228in}}{\pgfqpoint{3.348198in}{2.631128in}}{\pgfqpoint{3.348198in}{2.639365in}}%
\pgfpathcurveto{\pgfqpoint{3.348198in}{2.647601in}}{\pgfqpoint{3.344925in}{2.655501in}}{\pgfqpoint{3.339101in}{2.661325in}}%
\pgfpathcurveto{\pgfqpoint{3.333277in}{2.667149in}}{\pgfqpoint{3.325377in}{2.670421in}}{\pgfqpoint{3.317141in}{2.670421in}}%
\pgfpathcurveto{\pgfqpoint{3.308905in}{2.670421in}}{\pgfqpoint{3.301005in}{2.667149in}}{\pgfqpoint{3.295181in}{2.661325in}}%
\pgfpathcurveto{\pgfqpoint{3.289357in}{2.655501in}}{\pgfqpoint{3.286085in}{2.647601in}}{\pgfqpoint{3.286085in}{2.639365in}}%
\pgfpathcurveto{\pgfqpoint{3.286085in}{2.631128in}}{\pgfqpoint{3.289357in}{2.623228in}}{\pgfqpoint{3.295181in}{2.617404in}}%
\pgfpathcurveto{\pgfqpoint{3.301005in}{2.611580in}}{\pgfqpoint{3.308905in}{2.608308in}}{\pgfqpoint{3.317141in}{2.608308in}}%
\pgfpathclose%
\pgfusepath{stroke,fill}%
\end{pgfscope}%
\begin{pgfscope}%
\pgfpathrectangle{\pgfqpoint{0.100000in}{0.212622in}}{\pgfqpoint{3.696000in}{3.696000in}}%
\pgfusepath{clip}%
\pgfsetbuttcap%
\pgfsetroundjoin%
\definecolor{currentfill}{rgb}{0.121569,0.466667,0.705882}%
\pgfsetfillcolor{currentfill}%
\pgfsetfillopacity{0.696525}%
\pgfsetlinewidth{1.003750pt}%
\definecolor{currentstroke}{rgb}{0.121569,0.466667,0.705882}%
\pgfsetstrokecolor{currentstroke}%
\pgfsetstrokeopacity{0.696525}%
\pgfsetdash{}{0pt}%
\pgfpathmoveto{\pgfqpoint{3.318870in}{2.608275in}}%
\pgfpathcurveto{\pgfqpoint{3.327106in}{2.608275in}}{\pgfqpoint{3.335006in}{2.611547in}}{\pgfqpoint{3.340830in}{2.617371in}}%
\pgfpathcurveto{\pgfqpoint{3.346654in}{2.623195in}}{\pgfqpoint{3.349927in}{2.631095in}}{\pgfqpoint{3.349927in}{2.639331in}}%
\pgfpathcurveto{\pgfqpoint{3.349927in}{2.647567in}}{\pgfqpoint{3.346654in}{2.655467in}}{\pgfqpoint{3.340830in}{2.661291in}}%
\pgfpathcurveto{\pgfqpoint{3.335006in}{2.667115in}}{\pgfqpoint{3.327106in}{2.670388in}}{\pgfqpoint{3.318870in}{2.670388in}}%
\pgfpathcurveto{\pgfqpoint{3.310634in}{2.670388in}}{\pgfqpoint{3.302734in}{2.667115in}}{\pgfqpoint{3.296910in}{2.661291in}}%
\pgfpathcurveto{\pgfqpoint{3.291086in}{2.655467in}}{\pgfqpoint{3.287814in}{2.647567in}}{\pgfqpoint{3.287814in}{2.639331in}}%
\pgfpathcurveto{\pgfqpoint{3.287814in}{2.631095in}}{\pgfqpoint{3.291086in}{2.623195in}}{\pgfqpoint{3.296910in}{2.617371in}}%
\pgfpathcurveto{\pgfqpoint{3.302734in}{2.611547in}}{\pgfqpoint{3.310634in}{2.608275in}}{\pgfqpoint{3.318870in}{2.608275in}}%
\pgfpathclose%
\pgfusepath{stroke,fill}%
\end{pgfscope}%
\begin{pgfscope}%
\pgfpathrectangle{\pgfqpoint{0.100000in}{0.212622in}}{\pgfqpoint{3.696000in}{3.696000in}}%
\pgfusepath{clip}%
\pgfsetbuttcap%
\pgfsetroundjoin%
\definecolor{currentfill}{rgb}{0.121569,0.466667,0.705882}%
\pgfsetfillcolor{currentfill}%
\pgfsetfillopacity{0.696566}%
\pgfsetlinewidth{1.003750pt}%
\definecolor{currentstroke}{rgb}{0.121569,0.466667,0.705882}%
\pgfsetstrokecolor{currentstroke}%
\pgfsetstrokeopacity{0.696566}%
\pgfsetdash{}{0pt}%
\pgfpathmoveto{\pgfqpoint{1.365272in}{2.291784in}}%
\pgfpathcurveto{\pgfqpoint{1.373508in}{2.291784in}}{\pgfqpoint{1.381408in}{2.295057in}}{\pgfqpoint{1.387232in}{2.300881in}}%
\pgfpathcurveto{\pgfqpoint{1.393056in}{2.306704in}}{\pgfqpoint{1.396328in}{2.314605in}}{\pgfqpoint{1.396328in}{2.322841in}}%
\pgfpathcurveto{\pgfqpoint{1.396328in}{2.331077in}}{\pgfqpoint{1.393056in}{2.338977in}}{\pgfqpoint{1.387232in}{2.344801in}}%
\pgfpathcurveto{\pgfqpoint{1.381408in}{2.350625in}}{\pgfqpoint{1.373508in}{2.353897in}}{\pgfqpoint{1.365272in}{2.353897in}}%
\pgfpathcurveto{\pgfqpoint{1.357035in}{2.353897in}}{\pgfqpoint{1.349135in}{2.350625in}}{\pgfqpoint{1.343311in}{2.344801in}}%
\pgfpathcurveto{\pgfqpoint{1.337487in}{2.338977in}}{\pgfqpoint{1.334215in}{2.331077in}}{\pgfqpoint{1.334215in}{2.322841in}}%
\pgfpathcurveto{\pgfqpoint{1.334215in}{2.314605in}}{\pgfqpoint{1.337487in}{2.306704in}}{\pgfqpoint{1.343311in}{2.300881in}}%
\pgfpathcurveto{\pgfqpoint{1.349135in}{2.295057in}}{\pgfqpoint{1.357035in}{2.291784in}}{\pgfqpoint{1.365272in}{2.291784in}}%
\pgfpathclose%
\pgfusepath{stroke,fill}%
\end{pgfscope}%
\begin{pgfscope}%
\pgfpathrectangle{\pgfqpoint{0.100000in}{0.212622in}}{\pgfqpoint{3.696000in}{3.696000in}}%
\pgfusepath{clip}%
\pgfsetbuttcap%
\pgfsetroundjoin%
\definecolor{currentfill}{rgb}{0.121569,0.466667,0.705882}%
\pgfsetfillcolor{currentfill}%
\pgfsetfillopacity{0.696764}%
\pgfsetlinewidth{1.003750pt}%
\definecolor{currentstroke}{rgb}{0.121569,0.466667,0.705882}%
\pgfsetstrokecolor{currentstroke}%
\pgfsetstrokeopacity{0.696764}%
\pgfsetdash{}{0pt}%
\pgfpathmoveto{\pgfqpoint{3.320062in}{2.608070in}}%
\pgfpathcurveto{\pgfqpoint{3.328298in}{2.608070in}}{\pgfqpoint{3.336198in}{2.611343in}}{\pgfqpoint{3.342022in}{2.617166in}}%
\pgfpathcurveto{\pgfqpoint{3.347846in}{2.622990in}}{\pgfqpoint{3.351119in}{2.630890in}}{\pgfqpoint{3.351119in}{2.639127in}}%
\pgfpathcurveto{\pgfqpoint{3.351119in}{2.647363in}}{\pgfqpoint{3.347846in}{2.655263in}}{\pgfqpoint{3.342022in}{2.661087in}}%
\pgfpathcurveto{\pgfqpoint{3.336198in}{2.666911in}}{\pgfqpoint{3.328298in}{2.670183in}}{\pgfqpoint{3.320062in}{2.670183in}}%
\pgfpathcurveto{\pgfqpoint{3.311826in}{2.670183in}}{\pgfqpoint{3.303926in}{2.666911in}}{\pgfqpoint{3.298102in}{2.661087in}}%
\pgfpathcurveto{\pgfqpoint{3.292278in}{2.655263in}}{\pgfqpoint{3.289006in}{2.647363in}}{\pgfqpoint{3.289006in}{2.639127in}}%
\pgfpathcurveto{\pgfqpoint{3.289006in}{2.630890in}}{\pgfqpoint{3.292278in}{2.622990in}}{\pgfqpoint{3.298102in}{2.617166in}}%
\pgfpathcurveto{\pgfqpoint{3.303926in}{2.611343in}}{\pgfqpoint{3.311826in}{2.608070in}}{\pgfqpoint{3.320062in}{2.608070in}}%
\pgfpathclose%
\pgfusepath{stroke,fill}%
\end{pgfscope}%
\begin{pgfscope}%
\pgfpathrectangle{\pgfqpoint{0.100000in}{0.212622in}}{\pgfqpoint{3.696000in}{3.696000in}}%
\pgfusepath{clip}%
\pgfsetbuttcap%
\pgfsetroundjoin%
\definecolor{currentfill}{rgb}{0.121569,0.466667,0.705882}%
\pgfsetfillcolor{currentfill}%
\pgfsetfillopacity{0.696934}%
\pgfsetlinewidth{1.003750pt}%
\definecolor{currentstroke}{rgb}{0.121569,0.466667,0.705882}%
\pgfsetstrokecolor{currentstroke}%
\pgfsetstrokeopacity{0.696934}%
\pgfsetdash{}{0pt}%
\pgfpathmoveto{\pgfqpoint{3.320626in}{2.607856in}}%
\pgfpathcurveto{\pgfqpoint{3.328863in}{2.607856in}}{\pgfqpoint{3.336763in}{2.611129in}}{\pgfqpoint{3.342587in}{2.616953in}}%
\pgfpathcurveto{\pgfqpoint{3.348411in}{2.622776in}}{\pgfqpoint{3.351683in}{2.630676in}}{\pgfqpoint{3.351683in}{2.638913in}}%
\pgfpathcurveto{\pgfqpoint{3.351683in}{2.647149in}}{\pgfqpoint{3.348411in}{2.655049in}}{\pgfqpoint{3.342587in}{2.660873in}}%
\pgfpathcurveto{\pgfqpoint{3.336763in}{2.666697in}}{\pgfqpoint{3.328863in}{2.669969in}}{\pgfqpoint{3.320626in}{2.669969in}}%
\pgfpathcurveto{\pgfqpoint{3.312390in}{2.669969in}}{\pgfqpoint{3.304490in}{2.666697in}}{\pgfqpoint{3.298666in}{2.660873in}}%
\pgfpathcurveto{\pgfqpoint{3.292842in}{2.655049in}}{\pgfqpoint{3.289570in}{2.647149in}}{\pgfqpoint{3.289570in}{2.638913in}}%
\pgfpathcurveto{\pgfqpoint{3.289570in}{2.630676in}}{\pgfqpoint{3.292842in}{2.622776in}}{\pgfqpoint{3.298666in}{2.616953in}}%
\pgfpathcurveto{\pgfqpoint{3.304490in}{2.611129in}}{\pgfqpoint{3.312390in}{2.607856in}}{\pgfqpoint{3.320626in}{2.607856in}}%
\pgfpathclose%
\pgfusepath{stroke,fill}%
\end{pgfscope}%
\begin{pgfscope}%
\pgfpathrectangle{\pgfqpoint{0.100000in}{0.212622in}}{\pgfqpoint{3.696000in}{3.696000in}}%
\pgfusepath{clip}%
\pgfsetbuttcap%
\pgfsetroundjoin%
\definecolor{currentfill}{rgb}{0.121569,0.466667,0.705882}%
\pgfsetfillcolor{currentfill}%
\pgfsetfillopacity{0.697003}%
\pgfsetlinewidth{1.003750pt}%
\definecolor{currentstroke}{rgb}{0.121569,0.466667,0.705882}%
\pgfsetstrokecolor{currentstroke}%
\pgfsetstrokeopacity{0.697003}%
\pgfsetdash{}{0pt}%
\pgfpathmoveto{\pgfqpoint{3.320702in}{2.607672in}}%
\pgfpathcurveto{\pgfqpoint{3.328938in}{2.607672in}}{\pgfqpoint{3.336838in}{2.610944in}}{\pgfqpoint{3.342662in}{2.616768in}}%
\pgfpathcurveto{\pgfqpoint{3.348486in}{2.622592in}}{\pgfqpoint{3.351758in}{2.630492in}}{\pgfqpoint{3.351758in}{2.638728in}}%
\pgfpathcurveto{\pgfqpoint{3.351758in}{2.646964in}}{\pgfqpoint{3.348486in}{2.654865in}}{\pgfqpoint{3.342662in}{2.660688in}}%
\pgfpathcurveto{\pgfqpoint{3.336838in}{2.666512in}}{\pgfqpoint{3.328938in}{2.669785in}}{\pgfqpoint{3.320702in}{2.669785in}}%
\pgfpathcurveto{\pgfqpoint{3.312465in}{2.669785in}}{\pgfqpoint{3.304565in}{2.666512in}}{\pgfqpoint{3.298741in}{2.660688in}}%
\pgfpathcurveto{\pgfqpoint{3.292917in}{2.654865in}}{\pgfqpoint{3.289645in}{2.646964in}}{\pgfqpoint{3.289645in}{2.638728in}}%
\pgfpathcurveto{\pgfqpoint{3.289645in}{2.630492in}}{\pgfqpoint{3.292917in}{2.622592in}}{\pgfqpoint{3.298741in}{2.616768in}}%
\pgfpathcurveto{\pgfqpoint{3.304565in}{2.610944in}}{\pgfqpoint{3.312465in}{2.607672in}}{\pgfqpoint{3.320702in}{2.607672in}}%
\pgfpathclose%
\pgfusepath{stroke,fill}%
\end{pgfscope}%
\begin{pgfscope}%
\pgfpathrectangle{\pgfqpoint{0.100000in}{0.212622in}}{\pgfqpoint{3.696000in}{3.696000in}}%
\pgfusepath{clip}%
\pgfsetbuttcap%
\pgfsetroundjoin%
\definecolor{currentfill}{rgb}{0.121569,0.466667,0.705882}%
\pgfsetfillcolor{currentfill}%
\pgfsetfillopacity{0.697128}%
\pgfsetlinewidth{1.003750pt}%
\definecolor{currentstroke}{rgb}{0.121569,0.466667,0.705882}%
\pgfsetstrokecolor{currentstroke}%
\pgfsetstrokeopacity{0.697128}%
\pgfsetdash{}{0pt}%
\pgfpathmoveto{\pgfqpoint{3.320542in}{2.607188in}}%
\pgfpathcurveto{\pgfqpoint{3.328779in}{2.607188in}}{\pgfqpoint{3.336679in}{2.610460in}}{\pgfqpoint{3.342503in}{2.616284in}}%
\pgfpathcurveto{\pgfqpoint{3.348326in}{2.622108in}}{\pgfqpoint{3.351599in}{2.630008in}}{\pgfqpoint{3.351599in}{2.638244in}}%
\pgfpathcurveto{\pgfqpoint{3.351599in}{2.646481in}}{\pgfqpoint{3.348326in}{2.654381in}}{\pgfqpoint{3.342503in}{2.660205in}}%
\pgfpathcurveto{\pgfqpoint{3.336679in}{2.666028in}}{\pgfqpoint{3.328779in}{2.669301in}}{\pgfqpoint{3.320542in}{2.669301in}}%
\pgfpathcurveto{\pgfqpoint{3.312306in}{2.669301in}}{\pgfqpoint{3.304406in}{2.666028in}}{\pgfqpoint{3.298582in}{2.660205in}}%
\pgfpathcurveto{\pgfqpoint{3.292758in}{2.654381in}}{\pgfqpoint{3.289486in}{2.646481in}}{\pgfqpoint{3.289486in}{2.638244in}}%
\pgfpathcurveto{\pgfqpoint{3.289486in}{2.630008in}}{\pgfqpoint{3.292758in}{2.622108in}}{\pgfqpoint{3.298582in}{2.616284in}}%
\pgfpathcurveto{\pgfqpoint{3.304406in}{2.610460in}}{\pgfqpoint{3.312306in}{2.607188in}}{\pgfqpoint{3.320542in}{2.607188in}}%
\pgfpathclose%
\pgfusepath{stroke,fill}%
\end{pgfscope}%
\begin{pgfscope}%
\pgfpathrectangle{\pgfqpoint{0.100000in}{0.212622in}}{\pgfqpoint{3.696000in}{3.696000in}}%
\pgfusepath{clip}%
\pgfsetbuttcap%
\pgfsetroundjoin%
\definecolor{currentfill}{rgb}{0.121569,0.466667,0.705882}%
\pgfsetfillcolor{currentfill}%
\pgfsetfillopacity{0.697347}%
\pgfsetlinewidth{1.003750pt}%
\definecolor{currentstroke}{rgb}{0.121569,0.466667,0.705882}%
\pgfsetstrokecolor{currentstroke}%
\pgfsetstrokeopacity{0.697347}%
\pgfsetdash{}{0pt}%
\pgfpathmoveto{\pgfqpoint{3.320164in}{2.606318in}}%
\pgfpathcurveto{\pgfqpoint{3.328401in}{2.606318in}}{\pgfqpoint{3.336301in}{2.609590in}}{\pgfqpoint{3.342125in}{2.615414in}}%
\pgfpathcurveto{\pgfqpoint{3.347949in}{2.621238in}}{\pgfqpoint{3.351221in}{2.629138in}}{\pgfqpoint{3.351221in}{2.637374in}}%
\pgfpathcurveto{\pgfqpoint{3.351221in}{2.645610in}}{\pgfqpoint{3.347949in}{2.653511in}}{\pgfqpoint{3.342125in}{2.659334in}}%
\pgfpathcurveto{\pgfqpoint{3.336301in}{2.665158in}}{\pgfqpoint{3.328401in}{2.668431in}}{\pgfqpoint{3.320164in}{2.668431in}}%
\pgfpathcurveto{\pgfqpoint{3.311928in}{2.668431in}}{\pgfqpoint{3.304028in}{2.665158in}}{\pgfqpoint{3.298204in}{2.659334in}}%
\pgfpathcurveto{\pgfqpoint{3.292380in}{2.653511in}}{\pgfqpoint{3.289108in}{2.645610in}}{\pgfqpoint{3.289108in}{2.637374in}}%
\pgfpathcurveto{\pgfqpoint{3.289108in}{2.629138in}}{\pgfqpoint{3.292380in}{2.621238in}}{\pgfqpoint{3.298204in}{2.615414in}}%
\pgfpathcurveto{\pgfqpoint{3.304028in}{2.609590in}}{\pgfqpoint{3.311928in}{2.606318in}}{\pgfqpoint{3.320164in}{2.606318in}}%
\pgfpathclose%
\pgfusepath{stroke,fill}%
\end{pgfscope}%
\begin{pgfscope}%
\pgfpathrectangle{\pgfqpoint{0.100000in}{0.212622in}}{\pgfqpoint{3.696000in}{3.696000in}}%
\pgfusepath{clip}%
\pgfsetbuttcap%
\pgfsetroundjoin%
\definecolor{currentfill}{rgb}{0.121569,0.466667,0.705882}%
\pgfsetfillcolor{currentfill}%
\pgfsetfillopacity{0.697726}%
\pgfsetlinewidth{1.003750pt}%
\definecolor{currentstroke}{rgb}{0.121569,0.466667,0.705882}%
\pgfsetstrokecolor{currentstroke}%
\pgfsetstrokeopacity{0.697726}%
\pgfsetdash{}{0pt}%
\pgfpathmoveto{\pgfqpoint{3.319428in}{2.604673in}}%
\pgfpathcurveto{\pgfqpoint{3.327664in}{2.604673in}}{\pgfqpoint{3.335564in}{2.607945in}}{\pgfqpoint{3.341388in}{2.613769in}}%
\pgfpathcurveto{\pgfqpoint{3.347212in}{2.619593in}}{\pgfqpoint{3.350484in}{2.627493in}}{\pgfqpoint{3.350484in}{2.635729in}}%
\pgfpathcurveto{\pgfqpoint{3.350484in}{2.643965in}}{\pgfqpoint{3.347212in}{2.651865in}}{\pgfqpoint{3.341388in}{2.657689in}}%
\pgfpathcurveto{\pgfqpoint{3.335564in}{2.663513in}}{\pgfqpoint{3.327664in}{2.666786in}}{\pgfqpoint{3.319428in}{2.666786in}}%
\pgfpathcurveto{\pgfqpoint{3.311191in}{2.666786in}}{\pgfqpoint{3.303291in}{2.663513in}}{\pgfqpoint{3.297467in}{2.657689in}}%
\pgfpathcurveto{\pgfqpoint{3.291643in}{2.651865in}}{\pgfqpoint{3.288371in}{2.643965in}}{\pgfqpoint{3.288371in}{2.635729in}}%
\pgfpathcurveto{\pgfqpoint{3.288371in}{2.627493in}}{\pgfqpoint{3.291643in}{2.619593in}}{\pgfqpoint{3.297467in}{2.613769in}}%
\pgfpathcurveto{\pgfqpoint{3.303291in}{2.607945in}}{\pgfqpoint{3.311191in}{2.604673in}}{\pgfqpoint{3.319428in}{2.604673in}}%
\pgfpathclose%
\pgfusepath{stroke,fill}%
\end{pgfscope}%
\begin{pgfscope}%
\pgfpathrectangle{\pgfqpoint{0.100000in}{0.212622in}}{\pgfqpoint{3.696000in}{3.696000in}}%
\pgfusepath{clip}%
\pgfsetbuttcap%
\pgfsetroundjoin%
\definecolor{currentfill}{rgb}{0.121569,0.466667,0.705882}%
\pgfsetfillcolor{currentfill}%
\pgfsetfillopacity{0.697731}%
\pgfsetlinewidth{1.003750pt}%
\definecolor{currentstroke}{rgb}{0.121569,0.466667,0.705882}%
\pgfsetstrokecolor{currentstroke}%
\pgfsetstrokeopacity{0.697731}%
\pgfsetdash{}{0pt}%
\pgfpathmoveto{\pgfqpoint{1.362461in}{2.281312in}}%
\pgfpathcurveto{\pgfqpoint{1.370697in}{2.281312in}}{\pgfqpoint{1.378597in}{2.284584in}}{\pgfqpoint{1.384421in}{2.290408in}}%
\pgfpathcurveto{\pgfqpoint{1.390245in}{2.296232in}}{\pgfqpoint{1.393517in}{2.304132in}}{\pgfqpoint{1.393517in}{2.312369in}}%
\pgfpathcurveto{\pgfqpoint{1.393517in}{2.320605in}}{\pgfqpoint{1.390245in}{2.328505in}}{\pgfqpoint{1.384421in}{2.334329in}}%
\pgfpathcurveto{\pgfqpoint{1.378597in}{2.340153in}}{\pgfqpoint{1.370697in}{2.343425in}}{\pgfqpoint{1.362461in}{2.343425in}}%
\pgfpathcurveto{\pgfqpoint{1.354224in}{2.343425in}}{\pgfqpoint{1.346324in}{2.340153in}}{\pgfqpoint{1.340501in}{2.334329in}}%
\pgfpathcurveto{\pgfqpoint{1.334677in}{2.328505in}}{\pgfqpoint{1.331404in}{2.320605in}}{\pgfqpoint{1.331404in}{2.312369in}}%
\pgfpathcurveto{\pgfqpoint{1.331404in}{2.304132in}}{\pgfqpoint{1.334677in}{2.296232in}}{\pgfqpoint{1.340501in}{2.290408in}}%
\pgfpathcurveto{\pgfqpoint{1.346324in}{2.284584in}}{\pgfqpoint{1.354224in}{2.281312in}}{\pgfqpoint{1.362461in}{2.281312in}}%
\pgfpathclose%
\pgfusepath{stroke,fill}%
\end{pgfscope}%
\begin{pgfscope}%
\pgfpathrectangle{\pgfqpoint{0.100000in}{0.212622in}}{\pgfqpoint{3.696000in}{3.696000in}}%
\pgfusepath{clip}%
\pgfsetbuttcap%
\pgfsetroundjoin%
\definecolor{currentfill}{rgb}{0.121569,0.466667,0.705882}%
\pgfsetfillcolor{currentfill}%
\pgfsetfillopacity{0.698172}%
\pgfsetlinewidth{1.003750pt}%
\definecolor{currentstroke}{rgb}{0.121569,0.466667,0.705882}%
\pgfsetstrokecolor{currentstroke}%
\pgfsetstrokeopacity{0.698172}%
\pgfsetdash{}{0pt}%
\pgfpathmoveto{\pgfqpoint{2.190894in}{3.001469in}}%
\pgfpathcurveto{\pgfqpoint{2.199131in}{3.001469in}}{\pgfqpoint{2.207031in}{3.004741in}}{\pgfqpoint{2.212855in}{3.010565in}}%
\pgfpathcurveto{\pgfqpoint{2.218679in}{3.016389in}}{\pgfqpoint{2.221951in}{3.024289in}}{\pgfqpoint{2.221951in}{3.032526in}}%
\pgfpathcurveto{\pgfqpoint{2.221951in}{3.040762in}}{\pgfqpoint{2.218679in}{3.048662in}}{\pgfqpoint{2.212855in}{3.054486in}}%
\pgfpathcurveto{\pgfqpoint{2.207031in}{3.060310in}}{\pgfqpoint{2.199131in}{3.063582in}}{\pgfqpoint{2.190894in}{3.063582in}}%
\pgfpathcurveto{\pgfqpoint{2.182658in}{3.063582in}}{\pgfqpoint{2.174758in}{3.060310in}}{\pgfqpoint{2.168934in}{3.054486in}}%
\pgfpathcurveto{\pgfqpoint{2.163110in}{3.048662in}}{\pgfqpoint{2.159838in}{3.040762in}}{\pgfqpoint{2.159838in}{3.032526in}}%
\pgfpathcurveto{\pgfqpoint{2.159838in}{3.024289in}}{\pgfqpoint{2.163110in}{3.016389in}}{\pgfqpoint{2.168934in}{3.010565in}}%
\pgfpathcurveto{\pgfqpoint{2.174758in}{3.004741in}}{\pgfqpoint{2.182658in}{3.001469in}}{\pgfqpoint{2.190894in}{3.001469in}}%
\pgfpathclose%
\pgfusepath{stroke,fill}%
\end{pgfscope}%
\begin{pgfscope}%
\pgfpathrectangle{\pgfqpoint{0.100000in}{0.212622in}}{\pgfqpoint{3.696000in}{3.696000in}}%
\pgfusepath{clip}%
\pgfsetbuttcap%
\pgfsetroundjoin%
\definecolor{currentfill}{rgb}{0.121569,0.466667,0.705882}%
\pgfsetfillcolor{currentfill}%
\pgfsetfillopacity{0.698293}%
\pgfsetlinewidth{1.003750pt}%
\definecolor{currentstroke}{rgb}{0.121569,0.466667,0.705882}%
\pgfsetstrokecolor{currentstroke}%
\pgfsetstrokeopacity{0.698293}%
\pgfsetdash{}{0pt}%
\pgfpathmoveto{\pgfqpoint{1.360626in}{2.275322in}}%
\pgfpathcurveto{\pgfqpoint{1.368863in}{2.275322in}}{\pgfqpoint{1.376763in}{2.278594in}}{\pgfqpoint{1.382587in}{2.284418in}}%
\pgfpathcurveto{\pgfqpoint{1.388411in}{2.290242in}}{\pgfqpoint{1.391683in}{2.298142in}}{\pgfqpoint{1.391683in}{2.306378in}}%
\pgfpathcurveto{\pgfqpoint{1.391683in}{2.314615in}}{\pgfqpoint{1.388411in}{2.322515in}}{\pgfqpoint{1.382587in}{2.328339in}}%
\pgfpathcurveto{\pgfqpoint{1.376763in}{2.334163in}}{\pgfqpoint{1.368863in}{2.337435in}}{\pgfqpoint{1.360626in}{2.337435in}}%
\pgfpathcurveto{\pgfqpoint{1.352390in}{2.337435in}}{\pgfqpoint{1.344490in}{2.334163in}}{\pgfqpoint{1.338666in}{2.328339in}}%
\pgfpathcurveto{\pgfqpoint{1.332842in}{2.322515in}}{\pgfqpoint{1.329570in}{2.314615in}}{\pgfqpoint{1.329570in}{2.306378in}}%
\pgfpathcurveto{\pgfqpoint{1.329570in}{2.298142in}}{\pgfqpoint{1.332842in}{2.290242in}}{\pgfqpoint{1.338666in}{2.284418in}}%
\pgfpathcurveto{\pgfqpoint{1.344490in}{2.278594in}}{\pgfqpoint{1.352390in}{2.275322in}}{\pgfqpoint{1.360626in}{2.275322in}}%
\pgfpathclose%
\pgfusepath{stroke,fill}%
\end{pgfscope}%
\begin{pgfscope}%
\pgfpathrectangle{\pgfqpoint{0.100000in}{0.212622in}}{\pgfqpoint{3.696000in}{3.696000in}}%
\pgfusepath{clip}%
\pgfsetbuttcap%
\pgfsetroundjoin%
\definecolor{currentfill}{rgb}{0.121569,0.466667,0.705882}%
\pgfsetfillcolor{currentfill}%
\pgfsetfillopacity{0.698319}%
\pgfsetlinewidth{1.003750pt}%
\definecolor{currentstroke}{rgb}{0.121569,0.466667,0.705882}%
\pgfsetstrokecolor{currentstroke}%
\pgfsetstrokeopacity{0.698319}%
\pgfsetdash{}{0pt}%
\pgfpathmoveto{\pgfqpoint{3.317512in}{2.601932in}}%
\pgfpathcurveto{\pgfqpoint{3.325748in}{2.601932in}}{\pgfqpoint{3.333648in}{2.605204in}}{\pgfqpoint{3.339472in}{2.611028in}}%
\pgfpathcurveto{\pgfqpoint{3.345296in}{2.616852in}}{\pgfqpoint{3.348568in}{2.624752in}}{\pgfqpoint{3.348568in}{2.632988in}}%
\pgfpathcurveto{\pgfqpoint{3.348568in}{2.641224in}}{\pgfqpoint{3.345296in}{2.649124in}}{\pgfqpoint{3.339472in}{2.654948in}}%
\pgfpathcurveto{\pgfqpoint{3.333648in}{2.660772in}}{\pgfqpoint{3.325748in}{2.664045in}}{\pgfqpoint{3.317512in}{2.664045in}}%
\pgfpathcurveto{\pgfqpoint{3.309275in}{2.664045in}}{\pgfqpoint{3.301375in}{2.660772in}}{\pgfqpoint{3.295551in}{2.654948in}}%
\pgfpathcurveto{\pgfqpoint{3.289727in}{2.649124in}}{\pgfqpoint{3.286455in}{2.641224in}}{\pgfqpoint{3.286455in}{2.632988in}}%
\pgfpathcurveto{\pgfqpoint{3.286455in}{2.624752in}}{\pgfqpoint{3.289727in}{2.616852in}}{\pgfqpoint{3.295551in}{2.611028in}}%
\pgfpathcurveto{\pgfqpoint{3.301375in}{2.605204in}}{\pgfqpoint{3.309275in}{2.601932in}}{\pgfqpoint{3.317512in}{2.601932in}}%
\pgfpathclose%
\pgfusepath{stroke,fill}%
\end{pgfscope}%
\begin{pgfscope}%
\pgfpathrectangle{\pgfqpoint{0.100000in}{0.212622in}}{\pgfqpoint{3.696000in}{3.696000in}}%
\pgfusepath{clip}%
\pgfsetbuttcap%
\pgfsetroundjoin%
\definecolor{currentfill}{rgb}{0.121569,0.466667,0.705882}%
\pgfsetfillcolor{currentfill}%
\pgfsetfillopacity{0.698825}%
\pgfsetlinewidth{1.003750pt}%
\definecolor{currentstroke}{rgb}{0.121569,0.466667,0.705882}%
\pgfsetstrokecolor{currentstroke}%
\pgfsetstrokeopacity{0.698825}%
\pgfsetdash{}{0pt}%
\pgfpathmoveto{\pgfqpoint{3.316100in}{2.599357in}}%
\pgfpathcurveto{\pgfqpoint{3.324337in}{2.599357in}}{\pgfqpoint{3.332237in}{2.602630in}}{\pgfqpoint{3.338061in}{2.608453in}}%
\pgfpathcurveto{\pgfqpoint{3.343884in}{2.614277in}}{\pgfqpoint{3.347157in}{2.622177in}}{\pgfqpoint{3.347157in}{2.630414in}}%
\pgfpathcurveto{\pgfqpoint{3.347157in}{2.638650in}}{\pgfqpoint{3.343884in}{2.646550in}}{\pgfqpoint{3.338061in}{2.652374in}}%
\pgfpathcurveto{\pgfqpoint{3.332237in}{2.658198in}}{\pgfqpoint{3.324337in}{2.661470in}}{\pgfqpoint{3.316100in}{2.661470in}}%
\pgfpathcurveto{\pgfqpoint{3.307864in}{2.661470in}}{\pgfqpoint{3.299964in}{2.658198in}}{\pgfqpoint{3.294140in}{2.652374in}}%
\pgfpathcurveto{\pgfqpoint{3.288316in}{2.646550in}}{\pgfqpoint{3.285044in}{2.638650in}}{\pgfqpoint{3.285044in}{2.630414in}}%
\pgfpathcurveto{\pgfqpoint{3.285044in}{2.622177in}}{\pgfqpoint{3.288316in}{2.614277in}}{\pgfqpoint{3.294140in}{2.608453in}}%
\pgfpathcurveto{\pgfqpoint{3.299964in}{2.602630in}}{\pgfqpoint{3.307864in}{2.599357in}}{\pgfqpoint{3.316100in}{2.599357in}}%
\pgfpathclose%
\pgfusepath{stroke,fill}%
\end{pgfscope}%
\begin{pgfscope}%
\pgfpathrectangle{\pgfqpoint{0.100000in}{0.212622in}}{\pgfqpoint{3.696000in}{3.696000in}}%
\pgfusepath{clip}%
\pgfsetbuttcap%
\pgfsetroundjoin%
\definecolor{currentfill}{rgb}{0.121569,0.466667,0.705882}%
\pgfsetfillcolor{currentfill}%
\pgfsetfillopacity{0.698912}%
\pgfsetlinewidth{1.003750pt}%
\definecolor{currentstroke}{rgb}{0.121569,0.466667,0.705882}%
\pgfsetstrokecolor{currentstroke}%
\pgfsetstrokeopacity{0.698912}%
\pgfsetdash{}{0pt}%
\pgfpathmoveto{\pgfqpoint{1.357290in}{2.269656in}}%
\pgfpathcurveto{\pgfqpoint{1.365526in}{2.269656in}}{\pgfqpoint{1.373426in}{2.272929in}}{\pgfqpoint{1.379250in}{2.278753in}}%
\pgfpathcurveto{\pgfqpoint{1.385074in}{2.284577in}}{\pgfqpoint{1.388346in}{2.292477in}}{\pgfqpoint{1.388346in}{2.300713in}}%
\pgfpathcurveto{\pgfqpoint{1.388346in}{2.308949in}}{\pgfqpoint{1.385074in}{2.316849in}}{\pgfqpoint{1.379250in}{2.322673in}}%
\pgfpathcurveto{\pgfqpoint{1.373426in}{2.328497in}}{\pgfqpoint{1.365526in}{2.331769in}}{\pgfqpoint{1.357290in}{2.331769in}}%
\pgfpathcurveto{\pgfqpoint{1.349053in}{2.331769in}}{\pgfqpoint{1.341153in}{2.328497in}}{\pgfqpoint{1.335329in}{2.322673in}}%
\pgfpathcurveto{\pgfqpoint{1.329505in}{2.316849in}}{\pgfqpoint{1.326233in}{2.308949in}}{\pgfqpoint{1.326233in}{2.300713in}}%
\pgfpathcurveto{\pgfqpoint{1.326233in}{2.292477in}}{\pgfqpoint{1.329505in}{2.284577in}}{\pgfqpoint{1.335329in}{2.278753in}}%
\pgfpathcurveto{\pgfqpoint{1.341153in}{2.272929in}}{\pgfqpoint{1.349053in}{2.269656in}}{\pgfqpoint{1.357290in}{2.269656in}}%
\pgfpathclose%
\pgfusepath{stroke,fill}%
\end{pgfscope}%
\begin{pgfscope}%
\pgfpathrectangle{\pgfqpoint{0.100000in}{0.212622in}}{\pgfqpoint{3.696000in}{3.696000in}}%
\pgfusepath{clip}%
\pgfsetbuttcap%
\pgfsetroundjoin%
\definecolor{currentfill}{rgb}{0.121569,0.466667,0.705882}%
\pgfsetfillcolor{currentfill}%
\pgfsetfillopacity{0.699253}%
\pgfsetlinewidth{1.003750pt}%
\definecolor{currentstroke}{rgb}{0.121569,0.466667,0.705882}%
\pgfsetstrokecolor{currentstroke}%
\pgfsetstrokeopacity{0.699253}%
\pgfsetdash{}{0pt}%
\pgfpathmoveto{\pgfqpoint{1.355240in}{2.267060in}}%
\pgfpathcurveto{\pgfqpoint{1.363476in}{2.267060in}}{\pgfqpoint{1.371376in}{2.270332in}}{\pgfqpoint{1.377200in}{2.276156in}}%
\pgfpathcurveto{\pgfqpoint{1.383024in}{2.281980in}}{\pgfqpoint{1.386297in}{2.289880in}}{\pgfqpoint{1.386297in}{2.298116in}}%
\pgfpathcurveto{\pgfqpoint{1.386297in}{2.306352in}}{\pgfqpoint{1.383024in}{2.314252in}}{\pgfqpoint{1.377200in}{2.320076in}}%
\pgfpathcurveto{\pgfqpoint{1.371376in}{2.325900in}}{\pgfqpoint{1.363476in}{2.329173in}}{\pgfqpoint{1.355240in}{2.329173in}}%
\pgfpathcurveto{\pgfqpoint{1.347004in}{2.329173in}}{\pgfqpoint{1.339104in}{2.325900in}}{\pgfqpoint{1.333280in}{2.320076in}}%
\pgfpathcurveto{\pgfqpoint{1.327456in}{2.314252in}}{\pgfqpoint{1.324184in}{2.306352in}}{\pgfqpoint{1.324184in}{2.298116in}}%
\pgfpathcurveto{\pgfqpoint{1.324184in}{2.289880in}}{\pgfqpoint{1.327456in}{2.281980in}}{\pgfqpoint{1.333280in}{2.276156in}}%
\pgfpathcurveto{\pgfqpoint{1.339104in}{2.270332in}}{\pgfqpoint{1.347004in}{2.267060in}}{\pgfqpoint{1.355240in}{2.267060in}}%
\pgfpathclose%
\pgfusepath{stroke,fill}%
\end{pgfscope}%
\begin{pgfscope}%
\pgfpathrectangle{\pgfqpoint{0.100000in}{0.212622in}}{\pgfqpoint{3.696000in}{3.696000in}}%
\pgfusepath{clip}%
\pgfsetbuttcap%
\pgfsetroundjoin%
\definecolor{currentfill}{rgb}{0.121569,0.466667,0.705882}%
\pgfsetfillcolor{currentfill}%
\pgfsetfillopacity{0.699723}%
\pgfsetlinewidth{1.003750pt}%
\definecolor{currentstroke}{rgb}{0.121569,0.466667,0.705882}%
\pgfsetstrokecolor{currentstroke}%
\pgfsetstrokeopacity{0.699723}%
\pgfsetdash{}{0pt}%
\pgfpathmoveto{\pgfqpoint{1.352868in}{2.263362in}}%
\pgfpathcurveto{\pgfqpoint{1.361104in}{2.263362in}}{\pgfqpoint{1.369004in}{2.266635in}}{\pgfqpoint{1.374828in}{2.272459in}}%
\pgfpathcurveto{\pgfqpoint{1.380652in}{2.278282in}}{\pgfqpoint{1.383924in}{2.286183in}}{\pgfqpoint{1.383924in}{2.294419in}}%
\pgfpathcurveto{\pgfqpoint{1.383924in}{2.302655in}}{\pgfqpoint{1.380652in}{2.310555in}}{\pgfqpoint{1.374828in}{2.316379in}}%
\pgfpathcurveto{\pgfqpoint{1.369004in}{2.322203in}}{\pgfqpoint{1.361104in}{2.325475in}}{\pgfqpoint{1.352868in}{2.325475in}}%
\pgfpathcurveto{\pgfqpoint{1.344631in}{2.325475in}}{\pgfqpoint{1.336731in}{2.322203in}}{\pgfqpoint{1.330907in}{2.316379in}}%
\pgfpathcurveto{\pgfqpoint{1.325083in}{2.310555in}}{\pgfqpoint{1.321811in}{2.302655in}}{\pgfqpoint{1.321811in}{2.294419in}}%
\pgfpathcurveto{\pgfqpoint{1.321811in}{2.286183in}}{\pgfqpoint{1.325083in}{2.278282in}}{\pgfqpoint{1.330907in}{2.272459in}}%
\pgfpathcurveto{\pgfqpoint{1.336731in}{2.266635in}}{\pgfqpoint{1.344631in}{2.263362in}}{\pgfqpoint{1.352868in}{2.263362in}}%
\pgfpathclose%
\pgfusepath{stroke,fill}%
\end{pgfscope}%
\begin{pgfscope}%
\pgfpathrectangle{\pgfqpoint{0.100000in}{0.212622in}}{\pgfqpoint{3.696000in}{3.696000in}}%
\pgfusepath{clip}%
\pgfsetbuttcap%
\pgfsetroundjoin%
\definecolor{currentfill}{rgb}{0.121569,0.466667,0.705882}%
\pgfsetfillcolor{currentfill}%
\pgfsetfillopacity{0.699805}%
\pgfsetlinewidth{1.003750pt}%
\definecolor{currentstroke}{rgb}{0.121569,0.466667,0.705882}%
\pgfsetstrokecolor{currentstroke}%
\pgfsetstrokeopacity{0.699805}%
\pgfsetdash{}{0pt}%
\pgfpathmoveto{\pgfqpoint{3.313807in}{2.594660in}}%
\pgfpathcurveto{\pgfqpoint{3.322043in}{2.594660in}}{\pgfqpoint{3.329943in}{2.597933in}}{\pgfqpoint{3.335767in}{2.603756in}}%
\pgfpathcurveto{\pgfqpoint{3.341591in}{2.609580in}}{\pgfqpoint{3.344863in}{2.617480in}}{\pgfqpoint{3.344863in}{2.625717in}}%
\pgfpathcurveto{\pgfqpoint{3.344863in}{2.633953in}}{\pgfqpoint{3.341591in}{2.641853in}}{\pgfqpoint{3.335767in}{2.647677in}}%
\pgfpathcurveto{\pgfqpoint{3.329943in}{2.653501in}}{\pgfqpoint{3.322043in}{2.656773in}}{\pgfqpoint{3.313807in}{2.656773in}}%
\pgfpathcurveto{\pgfqpoint{3.305571in}{2.656773in}}{\pgfqpoint{3.297671in}{2.653501in}}{\pgfqpoint{3.291847in}{2.647677in}}%
\pgfpathcurveto{\pgfqpoint{3.286023in}{2.641853in}}{\pgfqpoint{3.282750in}{2.633953in}}{\pgfqpoint{3.282750in}{2.625717in}}%
\pgfpathcurveto{\pgfqpoint{3.282750in}{2.617480in}}{\pgfqpoint{3.286023in}{2.609580in}}{\pgfqpoint{3.291847in}{2.603756in}}%
\pgfpathcurveto{\pgfqpoint{3.297671in}{2.597933in}}{\pgfqpoint{3.305571in}{2.594660in}}{\pgfqpoint{3.313807in}{2.594660in}}%
\pgfpathclose%
\pgfusepath{stroke,fill}%
\end{pgfscope}%
\begin{pgfscope}%
\pgfpathrectangle{\pgfqpoint{0.100000in}{0.212622in}}{\pgfqpoint{3.696000in}{3.696000in}}%
\pgfusepath{clip}%
\pgfsetbuttcap%
\pgfsetroundjoin%
\definecolor{currentfill}{rgb}{0.121569,0.466667,0.705882}%
\pgfsetfillcolor{currentfill}%
\pgfsetfillopacity{0.700135}%
\pgfsetlinewidth{1.003750pt}%
\definecolor{currentstroke}{rgb}{0.121569,0.466667,0.705882}%
\pgfsetstrokecolor{currentstroke}%
\pgfsetstrokeopacity{0.700135}%
\pgfsetdash{}{0pt}%
\pgfpathmoveto{\pgfqpoint{2.198212in}{3.001874in}}%
\pgfpathcurveto{\pgfqpoint{2.206448in}{3.001874in}}{\pgfqpoint{2.214349in}{3.005147in}}{\pgfqpoint{2.220172in}{3.010971in}}%
\pgfpathcurveto{\pgfqpoint{2.225996in}{3.016795in}}{\pgfqpoint{2.229269in}{3.024695in}}{\pgfqpoint{2.229269in}{3.032931in}}%
\pgfpathcurveto{\pgfqpoint{2.229269in}{3.041167in}}{\pgfqpoint{2.225996in}{3.049067in}}{\pgfqpoint{2.220172in}{3.054891in}}%
\pgfpathcurveto{\pgfqpoint{2.214349in}{3.060715in}}{\pgfqpoint{2.206448in}{3.063987in}}{\pgfqpoint{2.198212in}{3.063987in}}%
\pgfpathcurveto{\pgfqpoint{2.189976in}{3.063987in}}{\pgfqpoint{2.182076in}{3.060715in}}{\pgfqpoint{2.176252in}{3.054891in}}%
\pgfpathcurveto{\pgfqpoint{2.170428in}{3.049067in}}{\pgfqpoint{2.167156in}{3.041167in}}{\pgfqpoint{2.167156in}{3.032931in}}%
\pgfpathcurveto{\pgfqpoint{2.167156in}{3.024695in}}{\pgfqpoint{2.170428in}{3.016795in}}{\pgfqpoint{2.176252in}{3.010971in}}%
\pgfpathcurveto{\pgfqpoint{2.182076in}{3.005147in}}{\pgfqpoint{2.189976in}{3.001874in}}{\pgfqpoint{2.198212in}{3.001874in}}%
\pgfpathclose%
\pgfusepath{stroke,fill}%
\end{pgfscope}%
\begin{pgfscope}%
\pgfpathrectangle{\pgfqpoint{0.100000in}{0.212622in}}{\pgfqpoint{3.696000in}{3.696000in}}%
\pgfusepath{clip}%
\pgfsetbuttcap%
\pgfsetroundjoin%
\definecolor{currentfill}{rgb}{0.121569,0.466667,0.705882}%
\pgfsetfillcolor{currentfill}%
\pgfsetfillopacity{0.700428}%
\pgfsetlinewidth{1.003750pt}%
\definecolor{currentstroke}{rgb}{0.121569,0.466667,0.705882}%
\pgfsetstrokecolor{currentstroke}%
\pgfsetstrokeopacity{0.700428}%
\pgfsetdash{}{0pt}%
\pgfpathmoveto{\pgfqpoint{1.350383in}{2.256634in}}%
\pgfpathcurveto{\pgfqpoint{1.358619in}{2.256634in}}{\pgfqpoint{1.366519in}{2.259906in}}{\pgfqpoint{1.372343in}{2.265730in}}%
\pgfpathcurveto{\pgfqpoint{1.378167in}{2.271554in}}{\pgfqpoint{1.381439in}{2.279454in}}{\pgfqpoint{1.381439in}{2.287690in}}%
\pgfpathcurveto{\pgfqpoint{1.381439in}{2.295927in}}{\pgfqpoint{1.378167in}{2.303827in}}{\pgfqpoint{1.372343in}{2.309651in}}%
\pgfpathcurveto{\pgfqpoint{1.366519in}{2.315475in}}{\pgfqpoint{1.358619in}{2.318747in}}{\pgfqpoint{1.350383in}{2.318747in}}%
\pgfpathcurveto{\pgfqpoint{1.342147in}{2.318747in}}{\pgfqpoint{1.334247in}{2.315475in}}{\pgfqpoint{1.328423in}{2.309651in}}%
\pgfpathcurveto{\pgfqpoint{1.322599in}{2.303827in}}{\pgfqpoint{1.319326in}{2.295927in}}{\pgfqpoint{1.319326in}{2.287690in}}%
\pgfpathcurveto{\pgfqpoint{1.319326in}{2.279454in}}{\pgfqpoint{1.322599in}{2.271554in}}{\pgfqpoint{1.328423in}{2.265730in}}%
\pgfpathcurveto{\pgfqpoint{1.334247in}{2.259906in}}{\pgfqpoint{1.342147in}{2.256634in}}{\pgfqpoint{1.350383in}{2.256634in}}%
\pgfpathclose%
\pgfusepath{stroke,fill}%
\end{pgfscope}%
\begin{pgfscope}%
\pgfpathrectangle{\pgfqpoint{0.100000in}{0.212622in}}{\pgfqpoint{3.696000in}{3.696000in}}%
\pgfusepath{clip}%
\pgfsetbuttcap%
\pgfsetroundjoin%
\definecolor{currentfill}{rgb}{0.121569,0.466667,0.705882}%
\pgfsetfillcolor{currentfill}%
\pgfsetfillopacity{0.700674}%
\pgfsetlinewidth{1.003750pt}%
\definecolor{currentstroke}{rgb}{0.121569,0.466667,0.705882}%
\pgfsetstrokecolor{currentstroke}%
\pgfsetstrokeopacity{0.700674}%
\pgfsetdash{}{0pt}%
\pgfpathmoveto{\pgfqpoint{3.311897in}{2.590117in}}%
\pgfpathcurveto{\pgfqpoint{3.320133in}{2.590117in}}{\pgfqpoint{3.328033in}{2.593389in}}{\pgfqpoint{3.333857in}{2.599213in}}%
\pgfpathcurveto{\pgfqpoint{3.339681in}{2.605037in}}{\pgfqpoint{3.342954in}{2.612937in}}{\pgfqpoint{3.342954in}{2.621173in}}%
\pgfpathcurveto{\pgfqpoint{3.342954in}{2.629409in}}{\pgfqpoint{3.339681in}{2.637309in}}{\pgfqpoint{3.333857in}{2.643133in}}%
\pgfpathcurveto{\pgfqpoint{3.328033in}{2.648957in}}{\pgfqpoint{3.320133in}{2.652230in}}{\pgfqpoint{3.311897in}{2.652230in}}%
\pgfpathcurveto{\pgfqpoint{3.303661in}{2.652230in}}{\pgfqpoint{3.295761in}{2.648957in}}{\pgfqpoint{3.289937in}{2.643133in}}%
\pgfpathcurveto{\pgfqpoint{3.284113in}{2.637309in}}{\pgfqpoint{3.280841in}{2.629409in}}{\pgfqpoint{3.280841in}{2.621173in}}%
\pgfpathcurveto{\pgfqpoint{3.280841in}{2.612937in}}{\pgfqpoint{3.284113in}{2.605037in}}{\pgfqpoint{3.289937in}{2.599213in}}%
\pgfpathcurveto{\pgfqpoint{3.295761in}{2.593389in}}{\pgfqpoint{3.303661in}{2.590117in}}{\pgfqpoint{3.311897in}{2.590117in}}%
\pgfpathclose%
\pgfusepath{stroke,fill}%
\end{pgfscope}%
\begin{pgfscope}%
\pgfpathrectangle{\pgfqpoint{0.100000in}{0.212622in}}{\pgfqpoint{3.696000in}{3.696000in}}%
\pgfusepath{clip}%
\pgfsetbuttcap%
\pgfsetroundjoin%
\definecolor{currentfill}{rgb}{0.121569,0.466667,0.705882}%
\pgfsetfillcolor{currentfill}%
\pgfsetfillopacity{0.700849}%
\pgfsetlinewidth{1.003750pt}%
\definecolor{currentstroke}{rgb}{0.121569,0.466667,0.705882}%
\pgfsetstrokecolor{currentstroke}%
\pgfsetstrokeopacity{0.700849}%
\pgfsetdash{}{0pt}%
\pgfpathmoveto{\pgfqpoint{1.349244in}{2.252933in}}%
\pgfpathcurveto{\pgfqpoint{1.357481in}{2.252933in}}{\pgfqpoint{1.365381in}{2.256205in}}{\pgfqpoint{1.371205in}{2.262029in}}%
\pgfpathcurveto{\pgfqpoint{1.377029in}{2.267853in}}{\pgfqpoint{1.380301in}{2.275753in}}{\pgfqpoint{1.380301in}{2.283989in}}%
\pgfpathcurveto{\pgfqpoint{1.380301in}{2.292226in}}{\pgfqpoint{1.377029in}{2.300126in}}{\pgfqpoint{1.371205in}{2.305950in}}%
\pgfpathcurveto{\pgfqpoint{1.365381in}{2.311774in}}{\pgfqpoint{1.357481in}{2.315046in}}{\pgfqpoint{1.349244in}{2.315046in}}%
\pgfpathcurveto{\pgfqpoint{1.341008in}{2.315046in}}{\pgfqpoint{1.333108in}{2.311774in}}{\pgfqpoint{1.327284in}{2.305950in}}%
\pgfpathcurveto{\pgfqpoint{1.321460in}{2.300126in}}{\pgfqpoint{1.318188in}{2.292226in}}{\pgfqpoint{1.318188in}{2.283989in}}%
\pgfpathcurveto{\pgfqpoint{1.318188in}{2.275753in}}{\pgfqpoint{1.321460in}{2.267853in}}{\pgfqpoint{1.327284in}{2.262029in}}%
\pgfpathcurveto{\pgfqpoint{1.333108in}{2.256205in}}{\pgfqpoint{1.341008in}{2.252933in}}{\pgfqpoint{1.349244in}{2.252933in}}%
\pgfpathclose%
\pgfusepath{stroke,fill}%
\end{pgfscope}%
\begin{pgfscope}%
\pgfpathrectangle{\pgfqpoint{0.100000in}{0.212622in}}{\pgfqpoint{3.696000in}{3.696000in}}%
\pgfusepath{clip}%
\pgfsetbuttcap%
\pgfsetroundjoin%
\definecolor{currentfill}{rgb}{0.121569,0.466667,0.705882}%
\pgfsetfillcolor{currentfill}%
\pgfsetfillopacity{0.701383}%
\pgfsetlinewidth{1.003750pt}%
\definecolor{currentstroke}{rgb}{0.121569,0.466667,0.705882}%
\pgfsetstrokecolor{currentstroke}%
\pgfsetstrokeopacity{0.701383}%
\pgfsetdash{}{0pt}%
\pgfpathmoveto{\pgfqpoint{2.205387in}{2.999913in}}%
\pgfpathcurveto{\pgfqpoint{2.213623in}{2.999913in}}{\pgfqpoint{2.221523in}{3.003186in}}{\pgfqpoint{2.227347in}{3.009010in}}%
\pgfpathcurveto{\pgfqpoint{2.233171in}{3.014834in}}{\pgfqpoint{2.236444in}{3.022734in}}{\pgfqpoint{2.236444in}{3.030970in}}%
\pgfpathcurveto{\pgfqpoint{2.236444in}{3.039206in}}{\pgfqpoint{2.233171in}{3.047106in}}{\pgfqpoint{2.227347in}{3.052930in}}%
\pgfpathcurveto{\pgfqpoint{2.221523in}{3.058754in}}{\pgfqpoint{2.213623in}{3.062026in}}{\pgfqpoint{2.205387in}{3.062026in}}%
\pgfpathcurveto{\pgfqpoint{2.197151in}{3.062026in}}{\pgfqpoint{2.189251in}{3.058754in}}{\pgfqpoint{2.183427in}{3.052930in}}%
\pgfpathcurveto{\pgfqpoint{2.177603in}{3.047106in}}{\pgfqpoint{2.174331in}{3.039206in}}{\pgfqpoint{2.174331in}{3.030970in}}%
\pgfpathcurveto{\pgfqpoint{2.174331in}{3.022734in}}{\pgfqpoint{2.177603in}{3.014834in}}{\pgfqpoint{2.183427in}{3.009010in}}%
\pgfpathcurveto{\pgfqpoint{2.189251in}{3.003186in}}{\pgfqpoint{2.197151in}{2.999913in}}{\pgfqpoint{2.205387in}{2.999913in}}%
\pgfpathclose%
\pgfusepath{stroke,fill}%
\end{pgfscope}%
\begin{pgfscope}%
\pgfpathrectangle{\pgfqpoint{0.100000in}{0.212622in}}{\pgfqpoint{3.696000in}{3.696000in}}%
\pgfusepath{clip}%
\pgfsetbuttcap%
\pgfsetroundjoin%
\definecolor{currentfill}{rgb}{0.121569,0.466667,0.705882}%
\pgfsetfillcolor{currentfill}%
\pgfsetfillopacity{0.701407}%
\pgfsetlinewidth{1.003750pt}%
\definecolor{currentstroke}{rgb}{0.121569,0.466667,0.705882}%
\pgfsetstrokecolor{currentstroke}%
\pgfsetstrokeopacity{0.701407}%
\pgfsetdash{}{0pt}%
\pgfpathmoveto{\pgfqpoint{1.346466in}{2.248644in}}%
\pgfpathcurveto{\pgfqpoint{1.354702in}{2.248644in}}{\pgfqpoint{1.362602in}{2.251916in}}{\pgfqpoint{1.368426in}{2.257740in}}%
\pgfpathcurveto{\pgfqpoint{1.374250in}{2.263564in}}{\pgfqpoint{1.377522in}{2.271464in}}{\pgfqpoint{1.377522in}{2.279700in}}%
\pgfpathcurveto{\pgfqpoint{1.377522in}{2.287937in}}{\pgfqpoint{1.374250in}{2.295837in}}{\pgfqpoint{1.368426in}{2.301661in}}%
\pgfpathcurveto{\pgfqpoint{1.362602in}{2.307484in}}{\pgfqpoint{1.354702in}{2.310757in}}{\pgfqpoint{1.346466in}{2.310757in}}%
\pgfpathcurveto{\pgfqpoint{1.338230in}{2.310757in}}{\pgfqpoint{1.330330in}{2.307484in}}{\pgfqpoint{1.324506in}{2.301661in}}%
\pgfpathcurveto{\pgfqpoint{1.318682in}{2.295837in}}{\pgfqpoint{1.315409in}{2.287937in}}{\pgfqpoint{1.315409in}{2.279700in}}%
\pgfpathcurveto{\pgfqpoint{1.315409in}{2.271464in}}{\pgfqpoint{1.318682in}{2.263564in}}{\pgfqpoint{1.324506in}{2.257740in}}%
\pgfpathcurveto{\pgfqpoint{1.330330in}{2.251916in}}{\pgfqpoint{1.338230in}{2.248644in}}{\pgfqpoint{1.346466in}{2.248644in}}%
\pgfpathclose%
\pgfusepath{stroke,fill}%
\end{pgfscope}%
\begin{pgfscope}%
\pgfpathrectangle{\pgfqpoint{0.100000in}{0.212622in}}{\pgfqpoint{3.696000in}{3.696000in}}%
\pgfusepath{clip}%
\pgfsetbuttcap%
\pgfsetroundjoin%
\definecolor{currentfill}{rgb}{0.121569,0.466667,0.705882}%
\pgfsetfillcolor{currentfill}%
\pgfsetfillopacity{0.701704}%
\pgfsetlinewidth{1.003750pt}%
\definecolor{currentstroke}{rgb}{0.121569,0.466667,0.705882}%
\pgfsetstrokecolor{currentstroke}%
\pgfsetstrokeopacity{0.701704}%
\pgfsetdash{}{0pt}%
\pgfpathmoveto{\pgfqpoint{1.344875in}{2.246385in}}%
\pgfpathcurveto{\pgfqpoint{1.353111in}{2.246385in}}{\pgfqpoint{1.361011in}{2.249657in}}{\pgfqpoint{1.366835in}{2.255481in}}%
\pgfpathcurveto{\pgfqpoint{1.372659in}{2.261305in}}{\pgfqpoint{1.375931in}{2.269205in}}{\pgfqpoint{1.375931in}{2.277441in}}%
\pgfpathcurveto{\pgfqpoint{1.375931in}{2.285678in}}{\pgfqpoint{1.372659in}{2.293578in}}{\pgfqpoint{1.366835in}{2.299402in}}%
\pgfpathcurveto{\pgfqpoint{1.361011in}{2.305226in}}{\pgfqpoint{1.353111in}{2.308498in}}{\pgfqpoint{1.344875in}{2.308498in}}%
\pgfpathcurveto{\pgfqpoint{1.336639in}{2.308498in}}{\pgfqpoint{1.328738in}{2.305226in}}{\pgfqpoint{1.322915in}{2.299402in}}%
\pgfpathcurveto{\pgfqpoint{1.317091in}{2.293578in}}{\pgfqpoint{1.313818in}{2.285678in}}{\pgfqpoint{1.313818in}{2.277441in}}%
\pgfpathcurveto{\pgfqpoint{1.313818in}{2.269205in}}{\pgfqpoint{1.317091in}{2.261305in}}{\pgfqpoint{1.322915in}{2.255481in}}%
\pgfpathcurveto{\pgfqpoint{1.328738in}{2.249657in}}{\pgfqpoint{1.336639in}{2.246385in}}{\pgfqpoint{1.344875in}{2.246385in}}%
\pgfpathclose%
\pgfusepath{stroke,fill}%
\end{pgfscope}%
\begin{pgfscope}%
\pgfpathrectangle{\pgfqpoint{0.100000in}{0.212622in}}{\pgfqpoint{3.696000in}{3.696000in}}%
\pgfusepath{clip}%
\pgfsetbuttcap%
\pgfsetroundjoin%
\definecolor{currentfill}{rgb}{0.121569,0.466667,0.705882}%
\pgfsetfillcolor{currentfill}%
\pgfsetfillopacity{0.701984}%
\pgfsetlinewidth{1.003750pt}%
\definecolor{currentstroke}{rgb}{0.121569,0.466667,0.705882}%
\pgfsetstrokecolor{currentstroke}%
\pgfsetstrokeopacity{0.701984}%
\pgfsetdash{}{0pt}%
\pgfpathmoveto{\pgfqpoint{3.308452in}{2.580550in}}%
\pgfpathcurveto{\pgfqpoint{3.316688in}{2.580550in}}{\pgfqpoint{3.324588in}{2.583822in}}{\pgfqpoint{3.330412in}{2.589646in}}%
\pgfpathcurveto{\pgfqpoint{3.336236in}{2.595470in}}{\pgfqpoint{3.339508in}{2.603370in}}{\pgfqpoint{3.339508in}{2.611606in}}%
\pgfpathcurveto{\pgfqpoint{3.339508in}{2.619843in}}{\pgfqpoint{3.336236in}{2.627743in}}{\pgfqpoint{3.330412in}{2.633567in}}%
\pgfpathcurveto{\pgfqpoint{3.324588in}{2.639391in}}{\pgfqpoint{3.316688in}{2.642663in}}{\pgfqpoint{3.308452in}{2.642663in}}%
\pgfpathcurveto{\pgfqpoint{3.300216in}{2.642663in}}{\pgfqpoint{3.292315in}{2.639391in}}{\pgfqpoint{3.286492in}{2.633567in}}%
\pgfpathcurveto{\pgfqpoint{3.280668in}{2.627743in}}{\pgfqpoint{3.277395in}{2.619843in}}{\pgfqpoint{3.277395in}{2.611606in}}%
\pgfpathcurveto{\pgfqpoint{3.277395in}{2.603370in}}{\pgfqpoint{3.280668in}{2.595470in}}{\pgfqpoint{3.286492in}{2.589646in}}%
\pgfpathcurveto{\pgfqpoint{3.292315in}{2.583822in}}{\pgfqpoint{3.300216in}{2.580550in}}{\pgfqpoint{3.308452in}{2.580550in}}%
\pgfpathclose%
\pgfusepath{stroke,fill}%
\end{pgfscope}%
\begin{pgfscope}%
\pgfpathrectangle{\pgfqpoint{0.100000in}{0.212622in}}{\pgfqpoint{3.696000in}{3.696000in}}%
\pgfusepath{clip}%
\pgfsetbuttcap%
\pgfsetroundjoin%
\definecolor{currentfill}{rgb}{0.121569,0.466667,0.705882}%
\pgfsetfillcolor{currentfill}%
\pgfsetfillopacity{0.702264}%
\pgfsetlinewidth{1.003750pt}%
\definecolor{currentstroke}{rgb}{0.121569,0.466667,0.705882}%
\pgfsetstrokecolor{currentstroke}%
\pgfsetstrokeopacity{0.702264}%
\pgfsetdash{}{0pt}%
\pgfpathmoveto{\pgfqpoint{1.342571in}{2.241980in}}%
\pgfpathcurveto{\pgfqpoint{1.350807in}{2.241980in}}{\pgfqpoint{1.358707in}{2.245252in}}{\pgfqpoint{1.364531in}{2.251076in}}%
\pgfpathcurveto{\pgfqpoint{1.370355in}{2.256900in}}{\pgfqpoint{1.373627in}{2.264800in}}{\pgfqpoint{1.373627in}{2.273036in}}%
\pgfpathcurveto{\pgfqpoint{1.373627in}{2.281273in}}{\pgfqpoint{1.370355in}{2.289173in}}{\pgfqpoint{1.364531in}{2.294997in}}%
\pgfpathcurveto{\pgfqpoint{1.358707in}{2.300821in}}{\pgfqpoint{1.350807in}{2.304093in}}{\pgfqpoint{1.342571in}{2.304093in}}%
\pgfpathcurveto{\pgfqpoint{1.334335in}{2.304093in}}{\pgfqpoint{1.326435in}{2.300821in}}{\pgfqpoint{1.320611in}{2.294997in}}%
\pgfpathcurveto{\pgfqpoint{1.314787in}{2.289173in}}{\pgfqpoint{1.311514in}{2.281273in}}{\pgfqpoint{1.311514in}{2.273036in}}%
\pgfpathcurveto{\pgfqpoint{1.311514in}{2.264800in}}{\pgfqpoint{1.314787in}{2.256900in}}{\pgfqpoint{1.320611in}{2.251076in}}%
\pgfpathcurveto{\pgfqpoint{1.326435in}{2.245252in}}{\pgfqpoint{1.334335in}{2.241980in}}{\pgfqpoint{1.342571in}{2.241980in}}%
\pgfpathclose%
\pgfusepath{stroke,fill}%
\end{pgfscope}%
\begin{pgfscope}%
\pgfpathrectangle{\pgfqpoint{0.100000in}{0.212622in}}{\pgfqpoint{3.696000in}{3.696000in}}%
\pgfusepath{clip}%
\pgfsetbuttcap%
\pgfsetroundjoin%
\definecolor{currentfill}{rgb}{0.121569,0.466667,0.705882}%
\pgfsetfillcolor{currentfill}%
\pgfsetfillopacity{0.702452}%
\pgfsetlinewidth{1.003750pt}%
\definecolor{currentstroke}{rgb}{0.121569,0.466667,0.705882}%
\pgfsetstrokecolor{currentstroke}%
\pgfsetstrokeopacity{0.702452}%
\pgfsetdash{}{0pt}%
\pgfpathmoveto{\pgfqpoint{2.211667in}{2.999113in}}%
\pgfpathcurveto{\pgfqpoint{2.219903in}{2.999113in}}{\pgfqpoint{2.227803in}{3.002385in}}{\pgfqpoint{2.233627in}{3.008209in}}%
\pgfpathcurveto{\pgfqpoint{2.239451in}{3.014033in}}{\pgfqpoint{2.242723in}{3.021933in}}{\pgfqpoint{2.242723in}{3.030169in}}%
\pgfpathcurveto{\pgfqpoint{2.242723in}{3.038406in}}{\pgfqpoint{2.239451in}{3.046306in}}{\pgfqpoint{2.233627in}{3.052130in}}%
\pgfpathcurveto{\pgfqpoint{2.227803in}{3.057954in}}{\pgfqpoint{2.219903in}{3.061226in}}{\pgfqpoint{2.211667in}{3.061226in}}%
\pgfpathcurveto{\pgfqpoint{2.203430in}{3.061226in}}{\pgfqpoint{2.195530in}{3.057954in}}{\pgfqpoint{2.189706in}{3.052130in}}%
\pgfpathcurveto{\pgfqpoint{2.183882in}{3.046306in}}{\pgfqpoint{2.180610in}{3.038406in}}{\pgfqpoint{2.180610in}{3.030169in}}%
\pgfpathcurveto{\pgfqpoint{2.180610in}{3.021933in}}{\pgfqpoint{2.183882in}{3.014033in}}{\pgfqpoint{2.189706in}{3.008209in}}%
\pgfpathcurveto{\pgfqpoint{2.195530in}{3.002385in}}{\pgfqpoint{2.203430in}{2.999113in}}{\pgfqpoint{2.211667in}{2.999113in}}%
\pgfpathclose%
\pgfusepath{stroke,fill}%
\end{pgfscope}%
\begin{pgfscope}%
\pgfpathrectangle{\pgfqpoint{0.100000in}{0.212622in}}{\pgfqpoint{3.696000in}{3.696000in}}%
\pgfusepath{clip}%
\pgfsetbuttcap%
\pgfsetroundjoin%
\definecolor{currentfill}{rgb}{0.121569,0.466667,0.705882}%
\pgfsetfillcolor{currentfill}%
\pgfsetfillopacity{0.702912}%
\pgfsetlinewidth{1.003750pt}%
\definecolor{currentstroke}{rgb}{0.121569,0.466667,0.705882}%
\pgfsetstrokecolor{currentstroke}%
\pgfsetstrokeopacity{0.702912}%
\pgfsetdash{}{0pt}%
\pgfpathmoveto{\pgfqpoint{1.340571in}{2.235179in}}%
\pgfpathcurveto{\pgfqpoint{1.348808in}{2.235179in}}{\pgfqpoint{1.356708in}{2.238452in}}{\pgfqpoint{1.362532in}{2.244276in}}%
\pgfpathcurveto{\pgfqpoint{1.368356in}{2.250100in}}{\pgfqpoint{1.371628in}{2.258000in}}{\pgfqpoint{1.371628in}{2.266236in}}%
\pgfpathcurveto{\pgfqpoint{1.371628in}{2.274472in}}{\pgfqpoint{1.368356in}{2.282372in}}{\pgfqpoint{1.362532in}{2.288196in}}%
\pgfpathcurveto{\pgfqpoint{1.356708in}{2.294020in}}{\pgfqpoint{1.348808in}{2.297292in}}{\pgfqpoint{1.340571in}{2.297292in}}%
\pgfpathcurveto{\pgfqpoint{1.332335in}{2.297292in}}{\pgfqpoint{1.324435in}{2.294020in}}{\pgfqpoint{1.318611in}{2.288196in}}%
\pgfpathcurveto{\pgfqpoint{1.312787in}{2.282372in}}{\pgfqpoint{1.309515in}{2.274472in}}{\pgfqpoint{1.309515in}{2.266236in}}%
\pgfpathcurveto{\pgfqpoint{1.309515in}{2.258000in}}{\pgfqpoint{1.312787in}{2.250100in}}{\pgfqpoint{1.318611in}{2.244276in}}%
\pgfpathcurveto{\pgfqpoint{1.324435in}{2.238452in}}{\pgfqpoint{1.332335in}{2.235179in}}{\pgfqpoint{1.340571in}{2.235179in}}%
\pgfpathclose%
\pgfusepath{stroke,fill}%
\end{pgfscope}%
\begin{pgfscope}%
\pgfpathrectangle{\pgfqpoint{0.100000in}{0.212622in}}{\pgfqpoint{3.696000in}{3.696000in}}%
\pgfusepath{clip}%
\pgfsetbuttcap%
\pgfsetroundjoin%
\definecolor{currentfill}{rgb}{0.121569,0.466667,0.705882}%
\pgfsetfillcolor{currentfill}%
\pgfsetfillopacity{0.703308}%
\pgfsetlinewidth{1.003750pt}%
\definecolor{currentstroke}{rgb}{0.121569,0.466667,0.705882}%
\pgfsetstrokecolor{currentstroke}%
\pgfsetstrokeopacity{0.703308}%
\pgfsetdash{}{0pt}%
\pgfpathmoveto{\pgfqpoint{1.339317in}{2.231713in}}%
\pgfpathcurveto{\pgfqpoint{1.347553in}{2.231713in}}{\pgfqpoint{1.355453in}{2.234985in}}{\pgfqpoint{1.361277in}{2.240809in}}%
\pgfpathcurveto{\pgfqpoint{1.367101in}{2.246633in}}{\pgfqpoint{1.370373in}{2.254533in}}{\pgfqpoint{1.370373in}{2.262769in}}%
\pgfpathcurveto{\pgfqpoint{1.370373in}{2.271005in}}{\pgfqpoint{1.367101in}{2.278905in}}{\pgfqpoint{1.361277in}{2.284729in}}%
\pgfpathcurveto{\pgfqpoint{1.355453in}{2.290553in}}{\pgfqpoint{1.347553in}{2.293826in}}{\pgfqpoint{1.339317in}{2.293826in}}%
\pgfpathcurveto{\pgfqpoint{1.331080in}{2.293826in}}{\pgfqpoint{1.323180in}{2.290553in}}{\pgfqpoint{1.317356in}{2.284729in}}%
\pgfpathcurveto{\pgfqpoint{1.311532in}{2.278905in}}{\pgfqpoint{1.308260in}{2.271005in}}{\pgfqpoint{1.308260in}{2.262769in}}%
\pgfpathcurveto{\pgfqpoint{1.308260in}{2.254533in}}{\pgfqpoint{1.311532in}{2.246633in}}{\pgfqpoint{1.317356in}{2.240809in}}%
\pgfpathcurveto{\pgfqpoint{1.323180in}{2.234985in}}{\pgfqpoint{1.331080in}{2.231713in}}{\pgfqpoint{1.339317in}{2.231713in}}%
\pgfpathclose%
\pgfusepath{stroke,fill}%
\end{pgfscope}%
\begin{pgfscope}%
\pgfpathrectangle{\pgfqpoint{0.100000in}{0.212622in}}{\pgfqpoint{3.696000in}{3.696000in}}%
\pgfusepath{clip}%
\pgfsetbuttcap%
\pgfsetroundjoin%
\definecolor{currentfill}{rgb}{0.121569,0.466667,0.705882}%
\pgfsetfillcolor{currentfill}%
\pgfsetfillopacity{0.703717}%
\pgfsetlinewidth{1.003750pt}%
\definecolor{currentstroke}{rgb}{0.121569,0.466667,0.705882}%
\pgfsetstrokecolor{currentstroke}%
\pgfsetstrokeopacity{0.703717}%
\pgfsetdash{}{0pt}%
\pgfpathmoveto{\pgfqpoint{1.336830in}{2.227433in}}%
\pgfpathcurveto{\pgfqpoint{1.345066in}{2.227433in}}{\pgfqpoint{1.352966in}{2.230705in}}{\pgfqpoint{1.358790in}{2.236529in}}%
\pgfpathcurveto{\pgfqpoint{1.364614in}{2.242353in}}{\pgfqpoint{1.367887in}{2.250253in}}{\pgfqpoint{1.367887in}{2.258489in}}%
\pgfpathcurveto{\pgfqpoint{1.367887in}{2.266726in}}{\pgfqpoint{1.364614in}{2.274626in}}{\pgfqpoint{1.358790in}{2.280450in}}%
\pgfpathcurveto{\pgfqpoint{1.352966in}{2.286274in}}{\pgfqpoint{1.345066in}{2.289546in}}{\pgfqpoint{1.336830in}{2.289546in}}%
\pgfpathcurveto{\pgfqpoint{1.328594in}{2.289546in}}{\pgfqpoint{1.320694in}{2.286274in}}{\pgfqpoint{1.314870in}{2.280450in}}%
\pgfpathcurveto{\pgfqpoint{1.309046in}{2.274626in}}{\pgfqpoint{1.305774in}{2.266726in}}{\pgfqpoint{1.305774in}{2.258489in}}%
\pgfpathcurveto{\pgfqpoint{1.305774in}{2.250253in}}{\pgfqpoint{1.309046in}{2.242353in}}{\pgfqpoint{1.314870in}{2.236529in}}%
\pgfpathcurveto{\pgfqpoint{1.320694in}{2.230705in}}{\pgfqpoint{1.328594in}{2.227433in}}{\pgfqpoint{1.336830in}{2.227433in}}%
\pgfpathclose%
\pgfusepath{stroke,fill}%
\end{pgfscope}%
\begin{pgfscope}%
\pgfpathrectangle{\pgfqpoint{0.100000in}{0.212622in}}{\pgfqpoint{3.696000in}{3.696000in}}%
\pgfusepath{clip}%
\pgfsetbuttcap%
\pgfsetroundjoin%
\definecolor{currentfill}{rgb}{0.121569,0.466667,0.705882}%
\pgfsetfillcolor{currentfill}%
\pgfsetfillopacity{0.703849}%
\pgfsetlinewidth{1.003750pt}%
\definecolor{currentstroke}{rgb}{0.121569,0.466667,0.705882}%
\pgfsetstrokecolor{currentstroke}%
\pgfsetstrokeopacity{0.703849}%
\pgfsetdash{}{0pt}%
\pgfpathmoveto{\pgfqpoint{2.223379in}{2.996344in}}%
\pgfpathcurveto{\pgfqpoint{2.231615in}{2.996344in}}{\pgfqpoint{2.239515in}{2.999617in}}{\pgfqpoint{2.245339in}{3.005441in}}%
\pgfpathcurveto{\pgfqpoint{2.251163in}{3.011265in}}{\pgfqpoint{2.254436in}{3.019165in}}{\pgfqpoint{2.254436in}{3.027401in}}%
\pgfpathcurveto{\pgfqpoint{2.254436in}{3.035637in}}{\pgfqpoint{2.251163in}{3.043537in}}{\pgfqpoint{2.245339in}{3.049361in}}%
\pgfpathcurveto{\pgfqpoint{2.239515in}{3.055185in}}{\pgfqpoint{2.231615in}{3.058457in}}{\pgfqpoint{2.223379in}{3.058457in}}%
\pgfpathcurveto{\pgfqpoint{2.215143in}{3.058457in}}{\pgfqpoint{2.207243in}{3.055185in}}{\pgfqpoint{2.201419in}{3.049361in}}%
\pgfpathcurveto{\pgfqpoint{2.195595in}{3.043537in}}{\pgfqpoint{2.192323in}{3.035637in}}{\pgfqpoint{2.192323in}{3.027401in}}%
\pgfpathcurveto{\pgfqpoint{2.192323in}{3.019165in}}{\pgfqpoint{2.195595in}{3.011265in}}{\pgfqpoint{2.201419in}{3.005441in}}%
\pgfpathcurveto{\pgfqpoint{2.207243in}{2.999617in}}{\pgfqpoint{2.215143in}{2.996344in}}{\pgfqpoint{2.223379in}{2.996344in}}%
\pgfpathclose%
\pgfusepath{stroke,fill}%
\end{pgfscope}%
\begin{pgfscope}%
\pgfpathrectangle{\pgfqpoint{0.100000in}{0.212622in}}{\pgfqpoint{3.696000in}{3.696000in}}%
\pgfusepath{clip}%
\pgfsetbuttcap%
\pgfsetroundjoin%
\definecolor{currentfill}{rgb}{0.121569,0.466667,0.705882}%
\pgfsetfillcolor{currentfill}%
\pgfsetfillopacity{0.703928}%
\pgfsetlinewidth{1.003750pt}%
\definecolor{currentstroke}{rgb}{0.121569,0.466667,0.705882}%
\pgfsetstrokecolor{currentstroke}%
\pgfsetstrokeopacity{0.703928}%
\pgfsetdash{}{0pt}%
\pgfpathmoveto{\pgfqpoint{1.335338in}{2.225325in}}%
\pgfpathcurveto{\pgfqpoint{1.343574in}{2.225325in}}{\pgfqpoint{1.351474in}{2.228597in}}{\pgfqpoint{1.357298in}{2.234421in}}%
\pgfpathcurveto{\pgfqpoint{1.363122in}{2.240245in}}{\pgfqpoint{1.366394in}{2.248145in}}{\pgfqpoint{1.366394in}{2.256382in}}%
\pgfpathcurveto{\pgfqpoint{1.366394in}{2.264618in}}{\pgfqpoint{1.363122in}{2.272518in}}{\pgfqpoint{1.357298in}{2.278342in}}%
\pgfpathcurveto{\pgfqpoint{1.351474in}{2.284166in}}{\pgfqpoint{1.343574in}{2.287438in}}{\pgfqpoint{1.335338in}{2.287438in}}%
\pgfpathcurveto{\pgfqpoint{1.327101in}{2.287438in}}{\pgfqpoint{1.319201in}{2.284166in}}{\pgfqpoint{1.313377in}{2.278342in}}%
\pgfpathcurveto{\pgfqpoint{1.307553in}{2.272518in}}{\pgfqpoint{1.304281in}{2.264618in}}{\pgfqpoint{1.304281in}{2.256382in}}%
\pgfpathcurveto{\pgfqpoint{1.304281in}{2.248145in}}{\pgfqpoint{1.307553in}{2.240245in}}{\pgfqpoint{1.313377in}{2.234421in}}%
\pgfpathcurveto{\pgfqpoint{1.319201in}{2.228597in}}{\pgfqpoint{1.327101in}{2.225325in}}{\pgfqpoint{1.335338in}{2.225325in}}%
\pgfpathclose%
\pgfusepath{stroke,fill}%
\end{pgfscope}%
\begin{pgfscope}%
\pgfpathrectangle{\pgfqpoint{0.100000in}{0.212622in}}{\pgfqpoint{3.696000in}{3.696000in}}%
\pgfusepath{clip}%
\pgfsetbuttcap%
\pgfsetroundjoin%
\definecolor{currentfill}{rgb}{0.121569,0.466667,0.705882}%
\pgfsetfillcolor{currentfill}%
\pgfsetfillopacity{0.704265}%
\pgfsetlinewidth{1.003750pt}%
\definecolor{currentstroke}{rgb}{0.121569,0.466667,0.705882}%
\pgfsetstrokecolor{currentstroke}%
\pgfsetstrokeopacity{0.704265}%
\pgfsetdash{}{0pt}%
\pgfpathmoveto{\pgfqpoint{1.333628in}{2.222907in}}%
\pgfpathcurveto{\pgfqpoint{1.341864in}{2.222907in}}{\pgfqpoint{1.349764in}{2.226179in}}{\pgfqpoint{1.355588in}{2.232003in}}%
\pgfpathcurveto{\pgfqpoint{1.361412in}{2.237827in}}{\pgfqpoint{1.364684in}{2.245727in}}{\pgfqpoint{1.364684in}{2.253964in}}%
\pgfpathcurveto{\pgfqpoint{1.364684in}{2.262200in}}{\pgfqpoint{1.361412in}{2.270100in}}{\pgfqpoint{1.355588in}{2.275924in}}%
\pgfpathcurveto{\pgfqpoint{1.349764in}{2.281748in}}{\pgfqpoint{1.341864in}{2.285020in}}{\pgfqpoint{1.333628in}{2.285020in}}%
\pgfpathcurveto{\pgfqpoint{1.325392in}{2.285020in}}{\pgfqpoint{1.317492in}{2.281748in}}{\pgfqpoint{1.311668in}{2.275924in}}%
\pgfpathcurveto{\pgfqpoint{1.305844in}{2.270100in}}{\pgfqpoint{1.302571in}{2.262200in}}{\pgfqpoint{1.302571in}{2.253964in}}%
\pgfpathcurveto{\pgfqpoint{1.302571in}{2.245727in}}{\pgfqpoint{1.305844in}{2.237827in}}{\pgfqpoint{1.311668in}{2.232003in}}%
\pgfpathcurveto{\pgfqpoint{1.317492in}{2.226179in}}{\pgfqpoint{1.325392in}{2.222907in}}{\pgfqpoint{1.333628in}{2.222907in}}%
\pgfpathclose%
\pgfusepath{stroke,fill}%
\end{pgfscope}%
\begin{pgfscope}%
\pgfpathrectangle{\pgfqpoint{0.100000in}{0.212622in}}{\pgfqpoint{3.696000in}{3.696000in}}%
\pgfusepath{clip}%
\pgfsetbuttcap%
\pgfsetroundjoin%
\definecolor{currentfill}{rgb}{0.121569,0.466667,0.705882}%
\pgfsetfillcolor{currentfill}%
\pgfsetfillopacity{0.704490}%
\pgfsetlinewidth{1.003750pt}%
\definecolor{currentstroke}{rgb}{0.121569,0.466667,0.705882}%
\pgfsetstrokecolor{currentstroke}%
\pgfsetstrokeopacity{0.704490}%
\pgfsetdash{}{0pt}%
\pgfpathmoveto{\pgfqpoint{3.305052in}{2.562587in}}%
\pgfpathcurveto{\pgfqpoint{3.313288in}{2.562587in}}{\pgfqpoint{3.321188in}{2.565859in}}{\pgfqpoint{3.327012in}{2.571683in}}%
\pgfpathcurveto{\pgfqpoint{3.332836in}{2.577507in}}{\pgfqpoint{3.336108in}{2.585407in}}{\pgfqpoint{3.336108in}{2.593643in}}%
\pgfpathcurveto{\pgfqpoint{3.336108in}{2.601879in}}{\pgfqpoint{3.332836in}{2.609779in}}{\pgfqpoint{3.327012in}{2.615603in}}%
\pgfpathcurveto{\pgfqpoint{3.321188in}{2.621427in}}{\pgfqpoint{3.313288in}{2.624700in}}{\pgfqpoint{3.305052in}{2.624700in}}%
\pgfpathcurveto{\pgfqpoint{3.296815in}{2.624700in}}{\pgfqpoint{3.288915in}{2.621427in}}{\pgfqpoint{3.283091in}{2.615603in}}%
\pgfpathcurveto{\pgfqpoint{3.277267in}{2.609779in}}{\pgfqpoint{3.273995in}{2.601879in}}{\pgfqpoint{3.273995in}{2.593643in}}%
\pgfpathcurveto{\pgfqpoint{3.273995in}{2.585407in}}{\pgfqpoint{3.277267in}{2.577507in}}{\pgfqpoint{3.283091in}{2.571683in}}%
\pgfpathcurveto{\pgfqpoint{3.288915in}{2.565859in}}{\pgfqpoint{3.296815in}{2.562587in}}{\pgfqpoint{3.305052in}{2.562587in}}%
\pgfpathclose%
\pgfusepath{stroke,fill}%
\end{pgfscope}%
\begin{pgfscope}%
\pgfpathrectangle{\pgfqpoint{0.100000in}{0.212622in}}{\pgfqpoint{3.696000in}{3.696000in}}%
\pgfusepath{clip}%
\pgfsetbuttcap%
\pgfsetroundjoin%
\definecolor{currentfill}{rgb}{0.121569,0.466667,0.705882}%
\pgfsetfillcolor{currentfill}%
\pgfsetfillopacity{0.704768}%
\pgfsetlinewidth{1.003750pt}%
\definecolor{currentstroke}{rgb}{0.121569,0.466667,0.705882}%
\pgfsetstrokecolor{currentstroke}%
\pgfsetstrokeopacity{0.704768}%
\pgfsetdash{}{0pt}%
\pgfpathmoveto{\pgfqpoint{1.331640in}{2.217574in}}%
\pgfpathcurveto{\pgfqpoint{1.339877in}{2.217574in}}{\pgfqpoint{1.347777in}{2.220846in}}{\pgfqpoint{1.353600in}{2.226670in}}%
\pgfpathcurveto{\pgfqpoint{1.359424in}{2.232494in}}{\pgfqpoint{1.362697in}{2.240394in}}{\pgfqpoint{1.362697in}{2.248631in}}%
\pgfpathcurveto{\pgfqpoint{1.362697in}{2.256867in}}{\pgfqpoint{1.359424in}{2.264767in}}{\pgfqpoint{1.353600in}{2.270591in}}%
\pgfpathcurveto{\pgfqpoint{1.347777in}{2.276415in}}{\pgfqpoint{1.339877in}{2.279687in}}{\pgfqpoint{1.331640in}{2.279687in}}%
\pgfpathcurveto{\pgfqpoint{1.323404in}{2.279687in}}{\pgfqpoint{1.315504in}{2.276415in}}{\pgfqpoint{1.309680in}{2.270591in}}%
\pgfpathcurveto{\pgfqpoint{1.303856in}{2.264767in}}{\pgfqpoint{1.300584in}{2.256867in}}{\pgfqpoint{1.300584in}{2.248631in}}%
\pgfpathcurveto{\pgfqpoint{1.300584in}{2.240394in}}{\pgfqpoint{1.303856in}{2.232494in}}{\pgfqpoint{1.309680in}{2.226670in}}%
\pgfpathcurveto{\pgfqpoint{1.315504in}{2.220846in}}{\pgfqpoint{1.323404in}{2.217574in}}{\pgfqpoint{1.331640in}{2.217574in}}%
\pgfpathclose%
\pgfusepath{stroke,fill}%
\end{pgfscope}%
\begin{pgfscope}%
\pgfpathrectangle{\pgfqpoint{0.100000in}{0.212622in}}{\pgfqpoint{3.696000in}{3.696000in}}%
\pgfusepath{clip}%
\pgfsetbuttcap%
\pgfsetroundjoin%
\definecolor{currentfill}{rgb}{0.121569,0.466667,0.705882}%
\pgfsetfillcolor{currentfill}%
\pgfsetfillopacity{0.705069}%
\pgfsetlinewidth{1.003750pt}%
\definecolor{currentstroke}{rgb}{0.121569,0.466667,0.705882}%
\pgfsetstrokecolor{currentstroke}%
\pgfsetstrokeopacity{0.705069}%
\pgfsetdash{}{0pt}%
\pgfpathmoveto{\pgfqpoint{1.330651in}{2.214669in}}%
\pgfpathcurveto{\pgfqpoint{1.338887in}{2.214669in}}{\pgfqpoint{1.346787in}{2.217941in}}{\pgfqpoint{1.352611in}{2.223765in}}%
\pgfpathcurveto{\pgfqpoint{1.358435in}{2.229589in}}{\pgfqpoint{1.361707in}{2.237489in}}{\pgfqpoint{1.361707in}{2.245725in}}%
\pgfpathcurveto{\pgfqpoint{1.361707in}{2.253962in}}{\pgfqpoint{1.358435in}{2.261862in}}{\pgfqpoint{1.352611in}{2.267686in}}%
\pgfpathcurveto{\pgfqpoint{1.346787in}{2.273509in}}{\pgfqpoint{1.338887in}{2.276782in}}{\pgfqpoint{1.330651in}{2.276782in}}%
\pgfpathcurveto{\pgfqpoint{1.322414in}{2.276782in}}{\pgfqpoint{1.314514in}{2.273509in}}{\pgfqpoint{1.308690in}{2.267686in}}%
\pgfpathcurveto{\pgfqpoint{1.302866in}{2.261862in}}{\pgfqpoint{1.299594in}{2.253962in}}{\pgfqpoint{1.299594in}{2.245725in}}%
\pgfpathcurveto{\pgfqpoint{1.299594in}{2.237489in}}{\pgfqpoint{1.302866in}{2.229589in}}{\pgfqpoint{1.308690in}{2.223765in}}%
\pgfpathcurveto{\pgfqpoint{1.314514in}{2.217941in}}{\pgfqpoint{1.322414in}{2.214669in}}{\pgfqpoint{1.330651in}{2.214669in}}%
\pgfpathclose%
\pgfusepath{stroke,fill}%
\end{pgfscope}%
\begin{pgfscope}%
\pgfpathrectangle{\pgfqpoint{0.100000in}{0.212622in}}{\pgfqpoint{3.696000in}{3.696000in}}%
\pgfusepath{clip}%
\pgfsetbuttcap%
\pgfsetroundjoin%
\definecolor{currentfill}{rgb}{0.121569,0.466667,0.705882}%
\pgfsetfillcolor{currentfill}%
\pgfsetfillopacity{0.705562}%
\pgfsetlinewidth{1.003750pt}%
\definecolor{currentstroke}{rgb}{0.121569,0.466667,0.705882}%
\pgfsetstrokecolor{currentstroke}%
\pgfsetstrokeopacity{0.705562}%
\pgfsetdash{}{0pt}%
\pgfpathmoveto{\pgfqpoint{1.328320in}{2.210635in}}%
\pgfpathcurveto{\pgfqpoint{1.336556in}{2.210635in}}{\pgfqpoint{1.344456in}{2.213908in}}{\pgfqpoint{1.350280in}{2.219732in}}%
\pgfpathcurveto{\pgfqpoint{1.356104in}{2.225556in}}{\pgfqpoint{1.359377in}{2.233456in}}{\pgfqpoint{1.359377in}{2.241692in}}%
\pgfpathcurveto{\pgfqpoint{1.359377in}{2.249928in}}{\pgfqpoint{1.356104in}{2.257828in}}{\pgfqpoint{1.350280in}{2.263652in}}%
\pgfpathcurveto{\pgfqpoint{1.344456in}{2.269476in}}{\pgfqpoint{1.336556in}{2.272748in}}{\pgfqpoint{1.328320in}{2.272748in}}%
\pgfpathcurveto{\pgfqpoint{1.320084in}{2.272748in}}{\pgfqpoint{1.312184in}{2.269476in}}{\pgfqpoint{1.306360in}{2.263652in}}%
\pgfpathcurveto{\pgfqpoint{1.300536in}{2.257828in}}{\pgfqpoint{1.297264in}{2.249928in}}{\pgfqpoint{1.297264in}{2.241692in}}%
\pgfpathcurveto{\pgfqpoint{1.297264in}{2.233456in}}{\pgfqpoint{1.300536in}{2.225556in}}{\pgfqpoint{1.306360in}{2.219732in}}%
\pgfpathcurveto{\pgfqpoint{1.312184in}{2.213908in}}{\pgfqpoint{1.320084in}{2.210635in}}{\pgfqpoint{1.328320in}{2.210635in}}%
\pgfpathclose%
\pgfusepath{stroke,fill}%
\end{pgfscope}%
\begin{pgfscope}%
\pgfpathrectangle{\pgfqpoint{0.100000in}{0.212622in}}{\pgfqpoint{3.696000in}{3.696000in}}%
\pgfusepath{clip}%
\pgfsetbuttcap%
\pgfsetroundjoin%
\definecolor{currentfill}{rgb}{0.121569,0.466667,0.705882}%
\pgfsetfillcolor{currentfill}%
\pgfsetfillopacity{0.705800}%
\pgfsetlinewidth{1.003750pt}%
\definecolor{currentstroke}{rgb}{0.121569,0.466667,0.705882}%
\pgfsetstrokecolor{currentstroke}%
\pgfsetstrokeopacity{0.705800}%
\pgfsetdash{}{0pt}%
\pgfpathmoveto{\pgfqpoint{1.326873in}{2.208633in}}%
\pgfpathcurveto{\pgfqpoint{1.335109in}{2.208633in}}{\pgfqpoint{1.343009in}{2.211906in}}{\pgfqpoint{1.348833in}{2.217730in}}%
\pgfpathcurveto{\pgfqpoint{1.354657in}{2.223554in}}{\pgfqpoint{1.357930in}{2.231454in}}{\pgfqpoint{1.357930in}{2.239690in}}%
\pgfpathcurveto{\pgfqpoint{1.357930in}{2.247926in}}{\pgfqpoint{1.354657in}{2.255826in}}{\pgfqpoint{1.348833in}{2.261650in}}%
\pgfpathcurveto{\pgfqpoint{1.343009in}{2.267474in}}{\pgfqpoint{1.335109in}{2.270746in}}{\pgfqpoint{1.326873in}{2.270746in}}%
\pgfpathcurveto{\pgfqpoint{1.318637in}{2.270746in}}{\pgfqpoint{1.310737in}{2.267474in}}{\pgfqpoint{1.304913in}{2.261650in}}%
\pgfpathcurveto{\pgfqpoint{1.299089in}{2.255826in}}{\pgfqpoint{1.295817in}{2.247926in}}{\pgfqpoint{1.295817in}{2.239690in}}%
\pgfpathcurveto{\pgfqpoint{1.295817in}{2.231454in}}{\pgfqpoint{1.299089in}{2.223554in}}{\pgfqpoint{1.304913in}{2.217730in}}%
\pgfpathcurveto{\pgfqpoint{1.310737in}{2.211906in}}{\pgfqpoint{1.318637in}{2.208633in}}{\pgfqpoint{1.326873in}{2.208633in}}%
\pgfpathclose%
\pgfusepath{stroke,fill}%
\end{pgfscope}%
\begin{pgfscope}%
\pgfpathrectangle{\pgfqpoint{0.100000in}{0.212622in}}{\pgfqpoint{3.696000in}{3.696000in}}%
\pgfusepath{clip}%
\pgfsetbuttcap%
\pgfsetroundjoin%
\definecolor{currentfill}{rgb}{0.121569,0.466667,0.705882}%
\pgfsetfillcolor{currentfill}%
\pgfsetfillopacity{0.706156}%
\pgfsetlinewidth{1.003750pt}%
\definecolor{currentstroke}{rgb}{0.121569,0.466667,0.705882}%
\pgfsetstrokecolor{currentstroke}%
\pgfsetstrokeopacity{0.706156}%
\pgfsetdash{}{0pt}%
\pgfpathmoveto{\pgfqpoint{2.232950in}{2.996939in}}%
\pgfpathcurveto{\pgfqpoint{2.241186in}{2.996939in}}{\pgfqpoint{2.249086in}{3.000211in}}{\pgfqpoint{2.254910in}{3.006035in}}%
\pgfpathcurveto{\pgfqpoint{2.260734in}{3.011859in}}{\pgfqpoint{2.264007in}{3.019759in}}{\pgfqpoint{2.264007in}{3.027995in}}%
\pgfpathcurveto{\pgfqpoint{2.264007in}{3.036232in}}{\pgfqpoint{2.260734in}{3.044132in}}{\pgfqpoint{2.254910in}{3.049956in}}%
\pgfpathcurveto{\pgfqpoint{2.249086in}{3.055780in}}{\pgfqpoint{2.241186in}{3.059052in}}{\pgfqpoint{2.232950in}{3.059052in}}%
\pgfpathcurveto{\pgfqpoint{2.224714in}{3.059052in}}{\pgfqpoint{2.216814in}{3.055780in}}{\pgfqpoint{2.210990in}{3.049956in}}%
\pgfpathcurveto{\pgfqpoint{2.205166in}{3.044132in}}{\pgfqpoint{2.201894in}{3.036232in}}{\pgfqpoint{2.201894in}{3.027995in}}%
\pgfpathcurveto{\pgfqpoint{2.201894in}{3.019759in}}{\pgfqpoint{2.205166in}{3.011859in}}{\pgfqpoint{2.210990in}{3.006035in}}%
\pgfpathcurveto{\pgfqpoint{2.216814in}{3.000211in}}{\pgfqpoint{2.224714in}{2.996939in}}{\pgfqpoint{2.232950in}{2.996939in}}%
\pgfpathclose%
\pgfusepath{stroke,fill}%
\end{pgfscope}%
\begin{pgfscope}%
\pgfpathrectangle{\pgfqpoint{0.100000in}{0.212622in}}{\pgfqpoint{3.696000in}{3.696000in}}%
\pgfusepath{clip}%
\pgfsetbuttcap%
\pgfsetroundjoin%
\definecolor{currentfill}{rgb}{0.121569,0.466667,0.705882}%
\pgfsetfillcolor{currentfill}%
\pgfsetfillopacity{0.706175}%
\pgfsetlinewidth{1.003750pt}%
\definecolor{currentstroke}{rgb}{0.121569,0.466667,0.705882}%
\pgfsetstrokecolor{currentstroke}%
\pgfsetstrokeopacity{0.706175}%
\pgfsetdash{}{0pt}%
\pgfpathmoveto{\pgfqpoint{1.325139in}{2.205905in}}%
\pgfpathcurveto{\pgfqpoint{1.333375in}{2.205905in}}{\pgfqpoint{1.341275in}{2.209177in}}{\pgfqpoint{1.347099in}{2.215001in}}%
\pgfpathcurveto{\pgfqpoint{1.352923in}{2.220825in}}{\pgfqpoint{1.356195in}{2.228725in}}{\pgfqpoint{1.356195in}{2.236961in}}%
\pgfpathcurveto{\pgfqpoint{1.356195in}{2.245197in}}{\pgfqpoint{1.352923in}{2.253098in}}{\pgfqpoint{1.347099in}{2.258921in}}%
\pgfpathcurveto{\pgfqpoint{1.341275in}{2.264745in}}{\pgfqpoint{1.333375in}{2.268018in}}{\pgfqpoint{1.325139in}{2.268018in}}%
\pgfpathcurveto{\pgfqpoint{1.316902in}{2.268018in}}{\pgfqpoint{1.309002in}{2.264745in}}{\pgfqpoint{1.303178in}{2.258921in}}%
\pgfpathcurveto{\pgfqpoint{1.297355in}{2.253098in}}{\pgfqpoint{1.294082in}{2.245197in}}{\pgfqpoint{1.294082in}{2.236961in}}%
\pgfpathcurveto{\pgfqpoint{1.294082in}{2.228725in}}{\pgfqpoint{1.297355in}{2.220825in}}{\pgfqpoint{1.303178in}{2.215001in}}%
\pgfpathcurveto{\pgfqpoint{1.309002in}{2.209177in}}{\pgfqpoint{1.316902in}{2.205905in}}{\pgfqpoint{1.325139in}{2.205905in}}%
\pgfpathclose%
\pgfusepath{stroke,fill}%
\end{pgfscope}%
\begin{pgfscope}%
\pgfpathrectangle{\pgfqpoint{0.100000in}{0.212622in}}{\pgfqpoint{3.696000in}{3.696000in}}%
\pgfusepath{clip}%
\pgfsetbuttcap%
\pgfsetroundjoin%
\definecolor{currentfill}{rgb}{0.121569,0.466667,0.705882}%
\pgfsetfillcolor{currentfill}%
\pgfsetfillopacity{0.706718}%
\pgfsetlinewidth{1.003750pt}%
\definecolor{currentstroke}{rgb}{0.121569,0.466667,0.705882}%
\pgfsetstrokecolor{currentstroke}%
\pgfsetstrokeopacity{0.706718}%
\pgfsetdash{}{0pt}%
\pgfpathmoveto{\pgfqpoint{1.323206in}{2.200627in}}%
\pgfpathcurveto{\pgfqpoint{1.331443in}{2.200627in}}{\pgfqpoint{1.339343in}{2.203899in}}{\pgfqpoint{1.345167in}{2.209723in}}%
\pgfpathcurveto{\pgfqpoint{1.350991in}{2.215547in}}{\pgfqpoint{1.354263in}{2.223447in}}{\pgfqpoint{1.354263in}{2.231683in}}%
\pgfpathcurveto{\pgfqpoint{1.354263in}{2.239920in}}{\pgfqpoint{1.350991in}{2.247820in}}{\pgfqpoint{1.345167in}{2.253644in}}%
\pgfpathcurveto{\pgfqpoint{1.339343in}{2.259468in}}{\pgfqpoint{1.331443in}{2.262740in}}{\pgfqpoint{1.323206in}{2.262740in}}%
\pgfpathcurveto{\pgfqpoint{1.314970in}{2.262740in}}{\pgfqpoint{1.307070in}{2.259468in}}{\pgfqpoint{1.301246in}{2.253644in}}%
\pgfpathcurveto{\pgfqpoint{1.295422in}{2.247820in}}{\pgfqpoint{1.292150in}{2.239920in}}{\pgfqpoint{1.292150in}{2.231683in}}%
\pgfpathcurveto{\pgfqpoint{1.292150in}{2.223447in}}{\pgfqpoint{1.295422in}{2.215547in}}{\pgfqpoint{1.301246in}{2.209723in}}%
\pgfpathcurveto{\pgfqpoint{1.307070in}{2.203899in}}{\pgfqpoint{1.314970in}{2.200627in}}{\pgfqpoint{1.323206in}{2.200627in}}%
\pgfpathclose%
\pgfusepath{stroke,fill}%
\end{pgfscope}%
\begin{pgfscope}%
\pgfpathrectangle{\pgfqpoint{0.100000in}{0.212622in}}{\pgfqpoint{3.696000in}{3.696000in}}%
\pgfusepath{clip}%
\pgfsetbuttcap%
\pgfsetroundjoin%
\definecolor{currentfill}{rgb}{0.121569,0.466667,0.705882}%
\pgfsetfillcolor{currentfill}%
\pgfsetfillopacity{0.706879}%
\pgfsetlinewidth{1.003750pt}%
\definecolor{currentstroke}{rgb}{0.121569,0.466667,0.705882}%
\pgfsetstrokecolor{currentstroke}%
\pgfsetstrokeopacity{0.706879}%
\pgfsetdash{}{0pt}%
\pgfpathmoveto{\pgfqpoint{3.301978in}{2.544898in}}%
\pgfpathcurveto{\pgfqpoint{3.310215in}{2.544898in}}{\pgfqpoint{3.318115in}{2.548170in}}{\pgfqpoint{3.323939in}{2.553994in}}%
\pgfpathcurveto{\pgfqpoint{3.329763in}{2.559818in}}{\pgfqpoint{3.333035in}{2.567718in}}{\pgfqpoint{3.333035in}{2.575954in}}%
\pgfpathcurveto{\pgfqpoint{3.333035in}{2.584190in}}{\pgfqpoint{3.329763in}{2.592090in}}{\pgfqpoint{3.323939in}{2.597914in}}%
\pgfpathcurveto{\pgfqpoint{3.318115in}{2.603738in}}{\pgfqpoint{3.310215in}{2.607011in}}{\pgfqpoint{3.301978in}{2.607011in}}%
\pgfpathcurveto{\pgfqpoint{3.293742in}{2.607011in}}{\pgfqpoint{3.285842in}{2.603738in}}{\pgfqpoint{3.280018in}{2.597914in}}%
\pgfpathcurveto{\pgfqpoint{3.274194in}{2.592090in}}{\pgfqpoint{3.270922in}{2.584190in}}{\pgfqpoint{3.270922in}{2.575954in}}%
\pgfpathcurveto{\pgfqpoint{3.270922in}{2.567718in}}{\pgfqpoint{3.274194in}{2.559818in}}{\pgfqpoint{3.280018in}{2.553994in}}%
\pgfpathcurveto{\pgfqpoint{3.285842in}{2.548170in}}{\pgfqpoint{3.293742in}{2.544898in}}{\pgfqpoint{3.301978in}{2.544898in}}%
\pgfpathclose%
\pgfusepath{stroke,fill}%
\end{pgfscope}%
\begin{pgfscope}%
\pgfpathrectangle{\pgfqpoint{0.100000in}{0.212622in}}{\pgfqpoint{3.696000in}{3.696000in}}%
\pgfusepath{clip}%
\pgfsetbuttcap%
\pgfsetroundjoin%
\definecolor{currentfill}{rgb}{0.121569,0.466667,0.705882}%
\pgfsetfillcolor{currentfill}%
\pgfsetfillopacity{0.707006}%
\pgfsetlinewidth{1.003750pt}%
\definecolor{currentstroke}{rgb}{0.121569,0.466667,0.705882}%
\pgfsetstrokecolor{currentstroke}%
\pgfsetstrokeopacity{0.707006}%
\pgfsetdash{}{0pt}%
\pgfpathmoveto{\pgfqpoint{1.322395in}{2.197527in}}%
\pgfpathcurveto{\pgfqpoint{1.330631in}{2.197527in}}{\pgfqpoint{1.338531in}{2.200799in}}{\pgfqpoint{1.344355in}{2.206623in}}%
\pgfpathcurveto{\pgfqpoint{1.350179in}{2.212447in}}{\pgfqpoint{1.353451in}{2.220347in}}{\pgfqpoint{1.353451in}{2.228583in}}%
\pgfpathcurveto{\pgfqpoint{1.353451in}{2.236819in}}{\pgfqpoint{1.350179in}{2.244719in}}{\pgfqpoint{1.344355in}{2.250543in}}%
\pgfpathcurveto{\pgfqpoint{1.338531in}{2.256367in}}{\pgfqpoint{1.330631in}{2.259640in}}{\pgfqpoint{1.322395in}{2.259640in}}%
\pgfpathcurveto{\pgfqpoint{1.314159in}{2.259640in}}{\pgfqpoint{1.306258in}{2.256367in}}{\pgfqpoint{1.300435in}{2.250543in}}%
\pgfpathcurveto{\pgfqpoint{1.294611in}{2.244719in}}{\pgfqpoint{1.291338in}{2.236819in}}{\pgfqpoint{1.291338in}{2.228583in}}%
\pgfpathcurveto{\pgfqpoint{1.291338in}{2.220347in}}{\pgfqpoint{1.294611in}{2.212447in}}{\pgfqpoint{1.300435in}{2.206623in}}%
\pgfpathcurveto{\pgfqpoint{1.306258in}{2.200799in}}{\pgfqpoint{1.314159in}{2.197527in}}{\pgfqpoint{1.322395in}{2.197527in}}%
\pgfpathclose%
\pgfusepath{stroke,fill}%
\end{pgfscope}%
\begin{pgfscope}%
\pgfpathrectangle{\pgfqpoint{0.100000in}{0.212622in}}{\pgfqpoint{3.696000in}{3.696000in}}%
\pgfusepath{clip}%
\pgfsetbuttcap%
\pgfsetroundjoin%
\definecolor{currentfill}{rgb}{0.121569,0.466667,0.705882}%
\pgfsetfillcolor{currentfill}%
\pgfsetfillopacity{0.707437}%
\pgfsetlinewidth{1.003750pt}%
\definecolor{currentstroke}{rgb}{0.121569,0.466667,0.705882}%
\pgfsetstrokecolor{currentstroke}%
\pgfsetstrokeopacity{0.707437}%
\pgfsetdash{}{0pt}%
\pgfpathmoveto{\pgfqpoint{1.320479in}{2.193390in}}%
\pgfpathcurveto{\pgfqpoint{1.328716in}{2.193390in}}{\pgfqpoint{1.336616in}{2.196662in}}{\pgfqpoint{1.342440in}{2.202486in}}%
\pgfpathcurveto{\pgfqpoint{1.348264in}{2.208310in}}{\pgfqpoint{1.351536in}{2.216210in}}{\pgfqpoint{1.351536in}{2.224447in}}%
\pgfpathcurveto{\pgfqpoint{1.351536in}{2.232683in}}{\pgfqpoint{1.348264in}{2.240583in}}{\pgfqpoint{1.342440in}{2.246407in}}%
\pgfpathcurveto{\pgfqpoint{1.336616in}{2.252231in}}{\pgfqpoint{1.328716in}{2.255503in}}{\pgfqpoint{1.320479in}{2.255503in}}%
\pgfpathcurveto{\pgfqpoint{1.312243in}{2.255503in}}{\pgfqpoint{1.304343in}{2.252231in}}{\pgfqpoint{1.298519in}{2.246407in}}%
\pgfpathcurveto{\pgfqpoint{1.292695in}{2.240583in}}{\pgfqpoint{1.289423in}{2.232683in}}{\pgfqpoint{1.289423in}{2.224447in}}%
\pgfpathcurveto{\pgfqpoint{1.289423in}{2.216210in}}{\pgfqpoint{1.292695in}{2.208310in}}{\pgfqpoint{1.298519in}{2.202486in}}%
\pgfpathcurveto{\pgfqpoint{1.304343in}{2.196662in}}{\pgfqpoint{1.312243in}{2.193390in}}{\pgfqpoint{1.320479in}{2.193390in}}%
\pgfpathclose%
\pgfusepath{stroke,fill}%
\end{pgfscope}%
\begin{pgfscope}%
\pgfpathrectangle{\pgfqpoint{0.100000in}{0.212622in}}{\pgfqpoint{3.696000in}{3.696000in}}%
\pgfusepath{clip}%
\pgfsetbuttcap%
\pgfsetroundjoin%
\definecolor{currentfill}{rgb}{0.121569,0.466667,0.705882}%
\pgfsetfillcolor{currentfill}%
\pgfsetfillopacity{0.707945}%
\pgfsetlinewidth{1.003750pt}%
\definecolor{currentstroke}{rgb}{0.121569,0.466667,0.705882}%
\pgfsetstrokecolor{currentstroke}%
\pgfsetstrokeopacity{0.707945}%
\pgfsetdash{}{0pt}%
\pgfpathmoveto{\pgfqpoint{1.317609in}{2.188888in}}%
\pgfpathcurveto{\pgfqpoint{1.325845in}{2.188888in}}{\pgfqpoint{1.333745in}{2.192160in}}{\pgfqpoint{1.339569in}{2.197984in}}%
\pgfpathcurveto{\pgfqpoint{1.345393in}{2.203808in}}{\pgfqpoint{1.348665in}{2.211708in}}{\pgfqpoint{1.348665in}{2.219944in}}%
\pgfpathcurveto{\pgfqpoint{1.348665in}{2.228180in}}{\pgfqpoint{1.345393in}{2.236080in}}{\pgfqpoint{1.339569in}{2.241904in}}%
\pgfpathcurveto{\pgfqpoint{1.333745in}{2.247728in}}{\pgfqpoint{1.325845in}{2.251001in}}{\pgfqpoint{1.317609in}{2.251001in}}%
\pgfpathcurveto{\pgfqpoint{1.309373in}{2.251001in}}{\pgfqpoint{1.301473in}{2.247728in}}{\pgfqpoint{1.295649in}{2.241904in}}%
\pgfpathcurveto{\pgfqpoint{1.289825in}{2.236080in}}{\pgfqpoint{1.286552in}{2.228180in}}{\pgfqpoint{1.286552in}{2.219944in}}%
\pgfpathcurveto{\pgfqpoint{1.286552in}{2.211708in}}{\pgfqpoint{1.289825in}{2.203808in}}{\pgfqpoint{1.295649in}{2.197984in}}%
\pgfpathcurveto{\pgfqpoint{1.301473in}{2.192160in}}{\pgfqpoint{1.309373in}{2.188888in}}{\pgfqpoint{1.317609in}{2.188888in}}%
\pgfpathclose%
\pgfusepath{stroke,fill}%
\end{pgfscope}%
\begin{pgfscope}%
\pgfpathrectangle{\pgfqpoint{0.100000in}{0.212622in}}{\pgfqpoint{3.696000in}{3.696000in}}%
\pgfusepath{clip}%
\pgfsetbuttcap%
\pgfsetroundjoin%
\definecolor{currentfill}{rgb}{0.121569,0.466667,0.705882}%
\pgfsetfillcolor{currentfill}%
\pgfsetfillopacity{0.708311}%
\pgfsetlinewidth{1.003750pt}%
\definecolor{currentstroke}{rgb}{0.121569,0.466667,0.705882}%
\pgfsetstrokecolor{currentstroke}%
\pgfsetstrokeopacity{0.708311}%
\pgfsetdash{}{0pt}%
\pgfpathmoveto{\pgfqpoint{2.241712in}{2.995616in}}%
\pgfpathcurveto{\pgfqpoint{2.249948in}{2.995616in}}{\pgfqpoint{2.257848in}{2.998888in}}{\pgfqpoint{2.263672in}{3.004712in}}%
\pgfpathcurveto{\pgfqpoint{2.269496in}{3.010536in}}{\pgfqpoint{2.272769in}{3.018436in}}{\pgfqpoint{2.272769in}{3.026672in}}%
\pgfpathcurveto{\pgfqpoint{2.272769in}{3.034908in}}{\pgfqpoint{2.269496in}{3.042808in}}{\pgfqpoint{2.263672in}{3.048632in}}%
\pgfpathcurveto{\pgfqpoint{2.257848in}{3.054456in}}{\pgfqpoint{2.249948in}{3.057729in}}{\pgfqpoint{2.241712in}{3.057729in}}%
\pgfpathcurveto{\pgfqpoint{2.233476in}{3.057729in}}{\pgfqpoint{2.225576in}{3.054456in}}{\pgfqpoint{2.219752in}{3.048632in}}%
\pgfpathcurveto{\pgfqpoint{2.213928in}{3.042808in}}{\pgfqpoint{2.210656in}{3.034908in}}{\pgfqpoint{2.210656in}{3.026672in}}%
\pgfpathcurveto{\pgfqpoint{2.210656in}{3.018436in}}{\pgfqpoint{2.213928in}{3.010536in}}{\pgfqpoint{2.219752in}{3.004712in}}%
\pgfpathcurveto{\pgfqpoint{2.225576in}{2.998888in}}{\pgfqpoint{2.233476in}{2.995616in}}{\pgfqpoint{2.241712in}{2.995616in}}%
\pgfpathclose%
\pgfusepath{stroke,fill}%
\end{pgfscope}%
\begin{pgfscope}%
\pgfpathrectangle{\pgfqpoint{0.100000in}{0.212622in}}{\pgfqpoint{3.696000in}{3.696000in}}%
\pgfusepath{clip}%
\pgfsetbuttcap%
\pgfsetroundjoin%
\definecolor{currentfill}{rgb}{0.121569,0.466667,0.705882}%
\pgfsetfillcolor{currentfill}%
\pgfsetfillopacity{0.708512}%
\pgfsetlinewidth{1.003750pt}%
\definecolor{currentstroke}{rgb}{0.121569,0.466667,0.705882}%
\pgfsetstrokecolor{currentstroke}%
\pgfsetstrokeopacity{0.708512}%
\pgfsetdash{}{0pt}%
\pgfpathmoveto{\pgfqpoint{1.314295in}{2.184166in}}%
\pgfpathcurveto{\pgfqpoint{1.322531in}{2.184166in}}{\pgfqpoint{1.330431in}{2.187438in}}{\pgfqpoint{1.336255in}{2.193262in}}%
\pgfpathcurveto{\pgfqpoint{1.342079in}{2.199086in}}{\pgfqpoint{1.345351in}{2.206986in}}{\pgfqpoint{1.345351in}{2.215222in}}%
\pgfpathcurveto{\pgfqpoint{1.345351in}{2.223459in}}{\pgfqpoint{1.342079in}{2.231359in}}{\pgfqpoint{1.336255in}{2.237183in}}%
\pgfpathcurveto{\pgfqpoint{1.330431in}{2.243007in}}{\pgfqpoint{1.322531in}{2.246279in}}{\pgfqpoint{1.314295in}{2.246279in}}%
\pgfpathcurveto{\pgfqpoint{1.306058in}{2.246279in}}{\pgfqpoint{1.298158in}{2.243007in}}{\pgfqpoint{1.292334in}{2.237183in}}%
\pgfpathcurveto{\pgfqpoint{1.286510in}{2.231359in}}{\pgfqpoint{1.283238in}{2.223459in}}{\pgfqpoint{1.283238in}{2.215222in}}%
\pgfpathcurveto{\pgfqpoint{1.283238in}{2.206986in}}{\pgfqpoint{1.286510in}{2.199086in}}{\pgfqpoint{1.292334in}{2.193262in}}%
\pgfpathcurveto{\pgfqpoint{1.298158in}{2.187438in}}{\pgfqpoint{1.306058in}{2.184166in}}{\pgfqpoint{1.314295in}{2.184166in}}%
\pgfpathclose%
\pgfusepath{stroke,fill}%
\end{pgfscope}%
\begin{pgfscope}%
\pgfpathrectangle{\pgfqpoint{0.100000in}{0.212622in}}{\pgfqpoint{3.696000in}{3.696000in}}%
\pgfusepath{clip}%
\pgfsetbuttcap%
\pgfsetroundjoin%
\definecolor{currentfill}{rgb}{0.121569,0.466667,0.705882}%
\pgfsetfillcolor{currentfill}%
\pgfsetfillopacity{0.708924}%
\pgfsetlinewidth{1.003750pt}%
\definecolor{currentstroke}{rgb}{0.121569,0.466667,0.705882}%
\pgfsetstrokecolor{currentstroke}%
\pgfsetstrokeopacity{0.708924}%
\pgfsetdash{}{0pt}%
\pgfpathmoveto{\pgfqpoint{3.295108in}{2.530438in}}%
\pgfpathcurveto{\pgfqpoint{3.303344in}{2.530438in}}{\pgfqpoint{3.311244in}{2.533710in}}{\pgfqpoint{3.317068in}{2.539534in}}%
\pgfpathcurveto{\pgfqpoint{3.322892in}{2.545358in}}{\pgfqpoint{3.326164in}{2.553258in}}{\pgfqpoint{3.326164in}{2.561494in}}%
\pgfpathcurveto{\pgfqpoint{3.326164in}{2.569730in}}{\pgfqpoint{3.322892in}{2.577631in}}{\pgfqpoint{3.317068in}{2.583454in}}%
\pgfpathcurveto{\pgfqpoint{3.311244in}{2.589278in}}{\pgfqpoint{3.303344in}{2.592551in}}{\pgfqpoint{3.295108in}{2.592551in}}%
\pgfpathcurveto{\pgfqpoint{3.286871in}{2.592551in}}{\pgfqpoint{3.278971in}{2.589278in}}{\pgfqpoint{3.273147in}{2.583454in}}%
\pgfpathcurveto{\pgfqpoint{3.267324in}{2.577631in}}{\pgfqpoint{3.264051in}{2.569730in}}{\pgfqpoint{3.264051in}{2.561494in}}%
\pgfpathcurveto{\pgfqpoint{3.264051in}{2.553258in}}{\pgfqpoint{3.267324in}{2.545358in}}{\pgfqpoint{3.273147in}{2.539534in}}%
\pgfpathcurveto{\pgfqpoint{3.278971in}{2.533710in}}{\pgfqpoint{3.286871in}{2.530438in}}{\pgfqpoint{3.295108in}{2.530438in}}%
\pgfpathclose%
\pgfusepath{stroke,fill}%
\end{pgfscope}%
\begin{pgfscope}%
\pgfpathrectangle{\pgfqpoint{0.100000in}{0.212622in}}{\pgfqpoint{3.696000in}{3.696000in}}%
\pgfusepath{clip}%
\pgfsetbuttcap%
\pgfsetroundjoin%
\definecolor{currentfill}{rgb}{0.121569,0.466667,0.705882}%
\pgfsetfillcolor{currentfill}%
\pgfsetfillopacity{0.709332}%
\pgfsetlinewidth{1.003750pt}%
\definecolor{currentstroke}{rgb}{0.121569,0.466667,0.705882}%
\pgfsetstrokecolor{currentstroke}%
\pgfsetstrokeopacity{0.709332}%
\pgfsetdash{}{0pt}%
\pgfpathmoveto{\pgfqpoint{1.310293in}{2.178245in}}%
\pgfpathcurveto{\pgfqpoint{1.318529in}{2.178245in}}{\pgfqpoint{1.326429in}{2.181517in}}{\pgfqpoint{1.332253in}{2.187341in}}%
\pgfpathcurveto{\pgfqpoint{1.338077in}{2.193165in}}{\pgfqpoint{1.341349in}{2.201065in}}{\pgfqpoint{1.341349in}{2.209301in}}%
\pgfpathcurveto{\pgfqpoint{1.341349in}{2.217538in}}{\pgfqpoint{1.338077in}{2.225438in}}{\pgfqpoint{1.332253in}{2.231262in}}%
\pgfpathcurveto{\pgfqpoint{1.326429in}{2.237085in}}{\pgfqpoint{1.318529in}{2.240358in}}{\pgfqpoint{1.310293in}{2.240358in}}%
\pgfpathcurveto{\pgfqpoint{1.302057in}{2.240358in}}{\pgfqpoint{1.294156in}{2.237085in}}{\pgfqpoint{1.288333in}{2.231262in}}%
\pgfpathcurveto{\pgfqpoint{1.282509in}{2.225438in}}{\pgfqpoint{1.279236in}{2.217538in}}{\pgfqpoint{1.279236in}{2.209301in}}%
\pgfpathcurveto{\pgfqpoint{1.279236in}{2.201065in}}{\pgfqpoint{1.282509in}{2.193165in}}{\pgfqpoint{1.288333in}{2.187341in}}%
\pgfpathcurveto{\pgfqpoint{1.294156in}{2.181517in}}{\pgfqpoint{1.302057in}{2.178245in}}{\pgfqpoint{1.310293in}{2.178245in}}%
\pgfpathclose%
\pgfusepath{stroke,fill}%
\end{pgfscope}%
\begin{pgfscope}%
\pgfpathrectangle{\pgfqpoint{0.100000in}{0.212622in}}{\pgfqpoint{3.696000in}{3.696000in}}%
\pgfusepath{clip}%
\pgfsetbuttcap%
\pgfsetroundjoin%
\definecolor{currentfill}{rgb}{0.121569,0.466667,0.705882}%
\pgfsetfillcolor{currentfill}%
\pgfsetfillopacity{0.709855}%
\pgfsetlinewidth{1.003750pt}%
\definecolor{currentstroke}{rgb}{0.121569,0.466667,0.705882}%
\pgfsetstrokecolor{currentstroke}%
\pgfsetstrokeopacity{0.709855}%
\pgfsetdash{}{0pt}%
\pgfpathmoveto{\pgfqpoint{2.250098in}{2.993444in}}%
\pgfpathcurveto{\pgfqpoint{2.258334in}{2.993444in}}{\pgfqpoint{2.266234in}{2.996716in}}{\pgfqpoint{2.272058in}{3.002540in}}%
\pgfpathcurveto{\pgfqpoint{2.277882in}{3.008364in}}{\pgfqpoint{2.281154in}{3.016264in}}{\pgfqpoint{2.281154in}{3.024501in}}%
\pgfpathcurveto{\pgfqpoint{2.281154in}{3.032737in}}{\pgfqpoint{2.277882in}{3.040637in}}{\pgfqpoint{2.272058in}{3.046461in}}%
\pgfpathcurveto{\pgfqpoint{2.266234in}{3.052285in}}{\pgfqpoint{2.258334in}{3.055557in}}{\pgfqpoint{2.250098in}{3.055557in}}%
\pgfpathcurveto{\pgfqpoint{2.241862in}{3.055557in}}{\pgfqpoint{2.233962in}{3.052285in}}{\pgfqpoint{2.228138in}{3.046461in}}%
\pgfpathcurveto{\pgfqpoint{2.222314in}{3.040637in}}{\pgfqpoint{2.219041in}{3.032737in}}{\pgfqpoint{2.219041in}{3.024501in}}%
\pgfpathcurveto{\pgfqpoint{2.219041in}{3.016264in}}{\pgfqpoint{2.222314in}{3.008364in}}{\pgfqpoint{2.228138in}{3.002540in}}%
\pgfpathcurveto{\pgfqpoint{2.233962in}{2.996716in}}{\pgfqpoint{2.241862in}{2.993444in}}{\pgfqpoint{2.250098in}{2.993444in}}%
\pgfpathclose%
\pgfusepath{stroke,fill}%
\end{pgfscope}%
\begin{pgfscope}%
\pgfpathrectangle{\pgfqpoint{0.100000in}{0.212622in}}{\pgfqpoint{3.696000in}{3.696000in}}%
\pgfusepath{clip}%
\pgfsetbuttcap%
\pgfsetroundjoin%
\definecolor{currentfill}{rgb}{0.121569,0.466667,0.705882}%
\pgfsetfillcolor{currentfill}%
\pgfsetfillopacity{0.710208}%
\pgfsetlinewidth{1.003750pt}%
\definecolor{currentstroke}{rgb}{0.121569,0.466667,0.705882}%
\pgfsetstrokecolor{currentstroke}%
\pgfsetstrokeopacity{0.710208}%
\pgfsetdash{}{0pt}%
\pgfpathmoveto{\pgfqpoint{3.287355in}{2.518817in}}%
\pgfpathcurveto{\pgfqpoint{3.295592in}{2.518817in}}{\pgfqpoint{3.303492in}{2.522090in}}{\pgfqpoint{3.309316in}{2.527913in}}%
\pgfpathcurveto{\pgfqpoint{3.315140in}{2.533737in}}{\pgfqpoint{3.318412in}{2.541637in}}{\pgfqpoint{3.318412in}{2.549874in}}%
\pgfpathcurveto{\pgfqpoint{3.318412in}{2.558110in}}{\pgfqpoint{3.315140in}{2.566010in}}{\pgfqpoint{3.309316in}{2.571834in}}%
\pgfpathcurveto{\pgfqpoint{3.303492in}{2.577658in}}{\pgfqpoint{3.295592in}{2.580930in}}{\pgfqpoint{3.287355in}{2.580930in}}%
\pgfpathcurveto{\pgfqpoint{3.279119in}{2.580930in}}{\pgfqpoint{3.271219in}{2.577658in}}{\pgfqpoint{3.265395in}{2.571834in}}%
\pgfpathcurveto{\pgfqpoint{3.259571in}{2.566010in}}{\pgfqpoint{3.256299in}{2.558110in}}{\pgfqpoint{3.256299in}{2.549874in}}%
\pgfpathcurveto{\pgfqpoint{3.256299in}{2.541637in}}{\pgfqpoint{3.259571in}{2.533737in}}{\pgfqpoint{3.265395in}{2.527913in}}%
\pgfpathcurveto{\pgfqpoint{3.271219in}{2.522090in}}{\pgfqpoint{3.279119in}{2.518817in}}{\pgfqpoint{3.287355in}{2.518817in}}%
\pgfpathclose%
\pgfusepath{stroke,fill}%
\end{pgfscope}%
\begin{pgfscope}%
\pgfpathrectangle{\pgfqpoint{0.100000in}{0.212622in}}{\pgfqpoint{3.696000in}{3.696000in}}%
\pgfusepath{clip}%
\pgfsetbuttcap%
\pgfsetroundjoin%
\definecolor{currentfill}{rgb}{0.121569,0.466667,0.705882}%
\pgfsetfillcolor{currentfill}%
\pgfsetfillopacity{0.710509}%
\pgfsetlinewidth{1.003750pt}%
\definecolor{currentstroke}{rgb}{0.121569,0.466667,0.705882}%
\pgfsetstrokecolor{currentstroke}%
\pgfsetstrokeopacity{0.710509}%
\pgfsetdash{}{0pt}%
\pgfpathmoveto{\pgfqpoint{1.306952in}{2.168698in}}%
\pgfpathcurveto{\pgfqpoint{1.315188in}{2.168698in}}{\pgfqpoint{1.323088in}{2.171970in}}{\pgfqpoint{1.328912in}{2.177794in}}%
\pgfpathcurveto{\pgfqpoint{1.334736in}{2.183618in}}{\pgfqpoint{1.338008in}{2.191518in}}{\pgfqpoint{1.338008in}{2.199754in}}%
\pgfpathcurveto{\pgfqpoint{1.338008in}{2.207991in}}{\pgfqpoint{1.334736in}{2.215891in}}{\pgfqpoint{1.328912in}{2.221715in}}%
\pgfpathcurveto{\pgfqpoint{1.323088in}{2.227539in}}{\pgfqpoint{1.315188in}{2.230811in}}{\pgfqpoint{1.306952in}{2.230811in}}%
\pgfpathcurveto{\pgfqpoint{1.298715in}{2.230811in}}{\pgfqpoint{1.290815in}{2.227539in}}{\pgfqpoint{1.284991in}{2.221715in}}%
\pgfpathcurveto{\pgfqpoint{1.279167in}{2.215891in}}{\pgfqpoint{1.275895in}{2.207991in}}{\pgfqpoint{1.275895in}{2.199754in}}%
\pgfpathcurveto{\pgfqpoint{1.275895in}{2.191518in}}{\pgfqpoint{1.279167in}{2.183618in}}{\pgfqpoint{1.284991in}{2.177794in}}%
\pgfpathcurveto{\pgfqpoint{1.290815in}{2.171970in}}{\pgfqpoint{1.298715in}{2.168698in}}{\pgfqpoint{1.306952in}{2.168698in}}%
\pgfpathclose%
\pgfusepath{stroke,fill}%
\end{pgfscope}%
\begin{pgfscope}%
\pgfpathrectangle{\pgfqpoint{0.100000in}{0.212622in}}{\pgfqpoint{3.696000in}{3.696000in}}%
\pgfusepath{clip}%
\pgfsetbuttcap%
\pgfsetroundjoin%
\definecolor{currentfill}{rgb}{0.121569,0.466667,0.705882}%
\pgfsetfillcolor{currentfill}%
\pgfsetfillopacity{0.711173}%
\pgfsetlinewidth{1.003750pt}%
\definecolor{currentstroke}{rgb}{0.121569,0.466667,0.705882}%
\pgfsetstrokecolor{currentstroke}%
\pgfsetstrokeopacity{0.711173}%
\pgfsetdash{}{0pt}%
\pgfpathmoveto{\pgfqpoint{2.257971in}{2.991425in}}%
\pgfpathcurveto{\pgfqpoint{2.266208in}{2.991425in}}{\pgfqpoint{2.274108in}{2.994697in}}{\pgfqpoint{2.279932in}{3.000521in}}%
\pgfpathcurveto{\pgfqpoint{2.285756in}{3.006345in}}{\pgfqpoint{2.289028in}{3.014245in}}{\pgfqpoint{2.289028in}{3.022481in}}%
\pgfpathcurveto{\pgfqpoint{2.289028in}{3.030718in}}{\pgfqpoint{2.285756in}{3.038618in}}{\pgfqpoint{2.279932in}{3.044442in}}%
\pgfpathcurveto{\pgfqpoint{2.274108in}{3.050266in}}{\pgfqpoint{2.266208in}{3.053538in}}{\pgfqpoint{2.257971in}{3.053538in}}%
\pgfpathcurveto{\pgfqpoint{2.249735in}{3.053538in}}{\pgfqpoint{2.241835in}{3.050266in}}{\pgfqpoint{2.236011in}{3.044442in}}%
\pgfpathcurveto{\pgfqpoint{2.230187in}{3.038618in}}{\pgfqpoint{2.226915in}{3.030718in}}{\pgfqpoint{2.226915in}{3.022481in}}%
\pgfpathcurveto{\pgfqpoint{2.226915in}{3.014245in}}{\pgfqpoint{2.230187in}{3.006345in}}{\pgfqpoint{2.236011in}{3.000521in}}%
\pgfpathcurveto{\pgfqpoint{2.241835in}{2.994697in}}{\pgfqpoint{2.249735in}{2.991425in}}{\pgfqpoint{2.257971in}{2.991425in}}%
\pgfpathclose%
\pgfusepath{stroke,fill}%
\end{pgfscope}%
\begin{pgfscope}%
\pgfpathrectangle{\pgfqpoint{0.100000in}{0.212622in}}{\pgfqpoint{3.696000in}{3.696000in}}%
\pgfusepath{clip}%
\pgfsetbuttcap%
\pgfsetroundjoin%
\definecolor{currentfill}{rgb}{0.121569,0.466667,0.705882}%
\pgfsetfillcolor{currentfill}%
\pgfsetfillopacity{0.711640}%
\pgfsetlinewidth{1.003750pt}%
\definecolor{currentstroke}{rgb}{0.121569,0.466667,0.705882}%
\pgfsetstrokecolor{currentstroke}%
\pgfsetstrokeopacity{0.711640}%
\pgfsetdash{}{0pt}%
\pgfpathmoveto{\pgfqpoint{3.280244in}{2.508058in}}%
\pgfpathcurveto{\pgfqpoint{3.288481in}{2.508058in}}{\pgfqpoint{3.296381in}{2.511330in}}{\pgfqpoint{3.302205in}{2.517154in}}%
\pgfpathcurveto{\pgfqpoint{3.308029in}{2.522978in}}{\pgfqpoint{3.311301in}{2.530878in}}{\pgfqpoint{3.311301in}{2.539114in}}%
\pgfpathcurveto{\pgfqpoint{3.311301in}{2.547351in}}{\pgfqpoint{3.308029in}{2.555251in}}{\pgfqpoint{3.302205in}{2.561075in}}%
\pgfpathcurveto{\pgfqpoint{3.296381in}{2.566898in}}{\pgfqpoint{3.288481in}{2.570171in}}{\pgfqpoint{3.280244in}{2.570171in}}%
\pgfpathcurveto{\pgfqpoint{3.272008in}{2.570171in}}{\pgfqpoint{3.264108in}{2.566898in}}{\pgfqpoint{3.258284in}{2.561075in}}%
\pgfpathcurveto{\pgfqpoint{3.252460in}{2.555251in}}{\pgfqpoint{3.249188in}{2.547351in}}{\pgfqpoint{3.249188in}{2.539114in}}%
\pgfpathcurveto{\pgfqpoint{3.249188in}{2.530878in}}{\pgfqpoint{3.252460in}{2.522978in}}{\pgfqpoint{3.258284in}{2.517154in}}%
\pgfpathcurveto{\pgfqpoint{3.264108in}{2.511330in}}{\pgfqpoint{3.272008in}{2.508058in}}{\pgfqpoint{3.280244in}{2.508058in}}%
\pgfpathclose%
\pgfusepath{stroke,fill}%
\end{pgfscope}%
\begin{pgfscope}%
\pgfpathrectangle{\pgfqpoint{0.100000in}{0.212622in}}{\pgfqpoint{3.696000in}{3.696000in}}%
\pgfusepath{clip}%
\pgfsetbuttcap%
\pgfsetroundjoin%
\definecolor{currentfill}{rgb}{0.121569,0.466667,0.705882}%
\pgfsetfillcolor{currentfill}%
\pgfsetfillopacity{0.711644}%
\pgfsetlinewidth{1.003750pt}%
\definecolor{currentstroke}{rgb}{0.121569,0.466667,0.705882}%
\pgfsetstrokecolor{currentstroke}%
\pgfsetstrokeopacity{0.711644}%
\pgfsetdash{}{0pt}%
\pgfpathmoveto{\pgfqpoint{1.304367in}{2.157846in}}%
\pgfpathcurveto{\pgfqpoint{1.312604in}{2.157846in}}{\pgfqpoint{1.320504in}{2.161119in}}{\pgfqpoint{1.326328in}{2.166943in}}%
\pgfpathcurveto{\pgfqpoint{1.332152in}{2.172767in}}{\pgfqpoint{1.335424in}{2.180667in}}{\pgfqpoint{1.335424in}{2.188903in}}%
\pgfpathcurveto{\pgfqpoint{1.335424in}{2.197139in}}{\pgfqpoint{1.332152in}{2.205039in}}{\pgfqpoint{1.326328in}{2.210863in}}%
\pgfpathcurveto{\pgfqpoint{1.320504in}{2.216687in}}{\pgfqpoint{1.312604in}{2.219959in}}{\pgfqpoint{1.304367in}{2.219959in}}%
\pgfpathcurveto{\pgfqpoint{1.296131in}{2.219959in}}{\pgfqpoint{1.288231in}{2.216687in}}{\pgfqpoint{1.282407in}{2.210863in}}%
\pgfpathcurveto{\pgfqpoint{1.276583in}{2.205039in}}{\pgfqpoint{1.273311in}{2.197139in}}{\pgfqpoint{1.273311in}{2.188903in}}%
\pgfpathcurveto{\pgfqpoint{1.273311in}{2.180667in}}{\pgfqpoint{1.276583in}{2.172767in}}{\pgfqpoint{1.282407in}{2.166943in}}%
\pgfpathcurveto{\pgfqpoint{1.288231in}{2.161119in}}{\pgfqpoint{1.296131in}{2.157846in}}{\pgfqpoint{1.304367in}{2.157846in}}%
\pgfpathclose%
\pgfusepath{stroke,fill}%
\end{pgfscope}%
\begin{pgfscope}%
\pgfpathrectangle{\pgfqpoint{0.100000in}{0.212622in}}{\pgfqpoint{3.696000in}{3.696000in}}%
\pgfusepath{clip}%
\pgfsetbuttcap%
\pgfsetroundjoin%
\definecolor{currentfill}{rgb}{0.121569,0.466667,0.705882}%
\pgfsetfillcolor{currentfill}%
\pgfsetfillopacity{0.712732}%
\pgfsetlinewidth{1.003750pt}%
\definecolor{currentstroke}{rgb}{0.121569,0.466667,0.705882}%
\pgfsetstrokecolor{currentstroke}%
\pgfsetstrokeopacity{0.712732}%
\pgfsetdash{}{0pt}%
\pgfpathmoveto{\pgfqpoint{2.265483in}{2.990794in}}%
\pgfpathcurveto{\pgfqpoint{2.273720in}{2.990794in}}{\pgfqpoint{2.281620in}{2.994067in}}{\pgfqpoint{2.287444in}{2.999891in}}%
\pgfpathcurveto{\pgfqpoint{2.293267in}{3.005714in}}{\pgfqpoint{2.296540in}{3.013614in}}{\pgfqpoint{2.296540in}{3.021851in}}%
\pgfpathcurveto{\pgfqpoint{2.296540in}{3.030087in}}{\pgfqpoint{2.293267in}{3.037987in}}{\pgfqpoint{2.287444in}{3.043811in}}%
\pgfpathcurveto{\pgfqpoint{2.281620in}{3.049635in}}{\pgfqpoint{2.273720in}{3.052907in}}{\pgfqpoint{2.265483in}{3.052907in}}%
\pgfpathcurveto{\pgfqpoint{2.257247in}{3.052907in}}{\pgfqpoint{2.249347in}{3.049635in}}{\pgfqpoint{2.243523in}{3.043811in}}%
\pgfpathcurveto{\pgfqpoint{2.237699in}{3.037987in}}{\pgfqpoint{2.234427in}{3.030087in}}{\pgfqpoint{2.234427in}{3.021851in}}%
\pgfpathcurveto{\pgfqpoint{2.234427in}{3.013614in}}{\pgfqpoint{2.237699in}{3.005714in}}{\pgfqpoint{2.243523in}{2.999891in}}%
\pgfpathcurveto{\pgfqpoint{2.249347in}{2.994067in}}{\pgfqpoint{2.257247in}{2.990794in}}{\pgfqpoint{2.265483in}{2.990794in}}%
\pgfpathclose%
\pgfusepath{stroke,fill}%
\end{pgfscope}%
\begin{pgfscope}%
\pgfpathrectangle{\pgfqpoint{0.100000in}{0.212622in}}{\pgfqpoint{3.696000in}{3.696000in}}%
\pgfusepath{clip}%
\pgfsetbuttcap%
\pgfsetroundjoin%
\definecolor{currentfill}{rgb}{0.121569,0.466667,0.705882}%
\pgfsetfillcolor{currentfill}%
\pgfsetfillopacity{0.712837}%
\pgfsetlinewidth{1.003750pt}%
\definecolor{currentstroke}{rgb}{0.121569,0.466667,0.705882}%
\pgfsetstrokecolor{currentstroke}%
\pgfsetstrokeopacity{0.712837}%
\pgfsetdash{}{0pt}%
\pgfpathmoveto{\pgfqpoint{1.298727in}{2.146405in}}%
\pgfpathcurveto{\pgfqpoint{1.306963in}{2.146405in}}{\pgfqpoint{1.314863in}{2.149677in}}{\pgfqpoint{1.320687in}{2.155501in}}%
\pgfpathcurveto{\pgfqpoint{1.326511in}{2.161325in}}{\pgfqpoint{1.329783in}{2.169225in}}{\pgfqpoint{1.329783in}{2.177461in}}%
\pgfpathcurveto{\pgfqpoint{1.329783in}{2.185698in}}{\pgfqpoint{1.326511in}{2.193598in}}{\pgfqpoint{1.320687in}{2.199422in}}%
\pgfpathcurveto{\pgfqpoint{1.314863in}{2.205246in}}{\pgfqpoint{1.306963in}{2.208518in}}{\pgfqpoint{1.298727in}{2.208518in}}%
\pgfpathcurveto{\pgfqpoint{1.290491in}{2.208518in}}{\pgfqpoint{1.282590in}{2.205246in}}{\pgfqpoint{1.276767in}{2.199422in}}%
\pgfpathcurveto{\pgfqpoint{1.270943in}{2.193598in}}{\pgfqpoint{1.267670in}{2.185698in}}{\pgfqpoint{1.267670in}{2.177461in}}%
\pgfpathcurveto{\pgfqpoint{1.267670in}{2.169225in}}{\pgfqpoint{1.270943in}{2.161325in}}{\pgfqpoint{1.276767in}{2.155501in}}%
\pgfpathcurveto{\pgfqpoint{1.282590in}{2.149677in}}{\pgfqpoint{1.290491in}{2.146405in}}{\pgfqpoint{1.298727in}{2.146405in}}%
\pgfpathclose%
\pgfusepath{stroke,fill}%
\end{pgfscope}%
\begin{pgfscope}%
\pgfpathrectangle{\pgfqpoint{0.100000in}{0.212622in}}{\pgfqpoint{3.696000in}{3.696000in}}%
\pgfusepath{clip}%
\pgfsetbuttcap%
\pgfsetroundjoin%
\definecolor{currentfill}{rgb}{0.121569,0.466667,0.705882}%
\pgfsetfillcolor{currentfill}%
\pgfsetfillopacity{0.713199}%
\pgfsetlinewidth{1.003750pt}%
\definecolor{currentstroke}{rgb}{0.121569,0.466667,0.705882}%
\pgfsetstrokecolor{currentstroke}%
\pgfsetstrokeopacity{0.713199}%
\pgfsetdash{}{0pt}%
\pgfpathmoveto{\pgfqpoint{3.275482in}{2.497741in}}%
\pgfpathcurveto{\pgfqpoint{3.283718in}{2.497741in}}{\pgfqpoint{3.291618in}{2.501014in}}{\pgfqpoint{3.297442in}{2.506838in}}%
\pgfpathcurveto{\pgfqpoint{3.303266in}{2.512662in}}{\pgfqpoint{3.306538in}{2.520562in}}{\pgfqpoint{3.306538in}{2.528798in}}%
\pgfpathcurveto{\pgfqpoint{3.306538in}{2.537034in}}{\pgfqpoint{3.303266in}{2.544934in}}{\pgfqpoint{3.297442in}{2.550758in}}%
\pgfpathcurveto{\pgfqpoint{3.291618in}{2.556582in}}{\pgfqpoint{3.283718in}{2.559854in}}{\pgfqpoint{3.275482in}{2.559854in}}%
\pgfpathcurveto{\pgfqpoint{3.267245in}{2.559854in}}{\pgfqpoint{3.259345in}{2.556582in}}{\pgfqpoint{3.253521in}{2.550758in}}%
\pgfpathcurveto{\pgfqpoint{3.247697in}{2.544934in}}{\pgfqpoint{3.244425in}{2.537034in}}{\pgfqpoint{3.244425in}{2.528798in}}%
\pgfpathcurveto{\pgfqpoint{3.244425in}{2.520562in}}{\pgfqpoint{3.247697in}{2.512662in}}{\pgfqpoint{3.253521in}{2.506838in}}%
\pgfpathcurveto{\pgfqpoint{3.259345in}{2.501014in}}{\pgfqpoint{3.267245in}{2.497741in}}{\pgfqpoint{3.275482in}{2.497741in}}%
\pgfpathclose%
\pgfusepath{stroke,fill}%
\end{pgfscope}%
\begin{pgfscope}%
\pgfpathrectangle{\pgfqpoint{0.100000in}{0.212622in}}{\pgfqpoint{3.696000in}{3.696000in}}%
\pgfusepath{clip}%
\pgfsetbuttcap%
\pgfsetroundjoin%
\definecolor{currentfill}{rgb}{0.121569,0.466667,0.705882}%
\pgfsetfillcolor{currentfill}%
\pgfsetfillopacity{0.714022}%
\pgfsetlinewidth{1.003750pt}%
\definecolor{currentstroke}{rgb}{0.121569,0.466667,0.705882}%
\pgfsetstrokecolor{currentstroke}%
\pgfsetstrokeopacity{0.714022}%
\pgfsetdash{}{0pt}%
\pgfpathmoveto{\pgfqpoint{1.291331in}{2.136071in}}%
\pgfpathcurveto{\pgfqpoint{1.299567in}{2.136071in}}{\pgfqpoint{1.307467in}{2.139343in}}{\pgfqpoint{1.313291in}{2.145167in}}%
\pgfpathcurveto{\pgfqpoint{1.319115in}{2.150991in}}{\pgfqpoint{1.322387in}{2.158891in}}{\pgfqpoint{1.322387in}{2.167127in}}%
\pgfpathcurveto{\pgfqpoint{1.322387in}{2.175364in}}{\pgfqpoint{1.319115in}{2.183264in}}{\pgfqpoint{1.313291in}{2.189088in}}%
\pgfpathcurveto{\pgfqpoint{1.307467in}{2.194912in}}{\pgfqpoint{1.299567in}{2.198184in}}{\pgfqpoint{1.291331in}{2.198184in}}%
\pgfpathcurveto{\pgfqpoint{1.283095in}{2.198184in}}{\pgfqpoint{1.275194in}{2.194912in}}{\pgfqpoint{1.269371in}{2.189088in}}%
\pgfpathcurveto{\pgfqpoint{1.263547in}{2.183264in}}{\pgfqpoint{1.260274in}{2.175364in}}{\pgfqpoint{1.260274in}{2.167127in}}%
\pgfpathcurveto{\pgfqpoint{1.260274in}{2.158891in}}{\pgfqpoint{1.263547in}{2.150991in}}{\pgfqpoint{1.269371in}{2.145167in}}%
\pgfpathcurveto{\pgfqpoint{1.275194in}{2.139343in}}{\pgfqpoint{1.283095in}{2.136071in}}{\pgfqpoint{1.291331in}{2.136071in}}%
\pgfpathclose%
\pgfusepath{stroke,fill}%
\end{pgfscope}%
\begin{pgfscope}%
\pgfpathrectangle{\pgfqpoint{0.100000in}{0.212622in}}{\pgfqpoint{3.696000in}{3.696000in}}%
\pgfusepath{clip}%
\pgfsetbuttcap%
\pgfsetroundjoin%
\definecolor{currentfill}{rgb}{0.121569,0.466667,0.705882}%
\pgfsetfillcolor{currentfill}%
\pgfsetfillopacity{0.714175}%
\pgfsetlinewidth{1.003750pt}%
\definecolor{currentstroke}{rgb}{0.121569,0.466667,0.705882}%
\pgfsetstrokecolor{currentstroke}%
\pgfsetstrokeopacity{0.714175}%
\pgfsetdash{}{0pt}%
\pgfpathmoveto{\pgfqpoint{2.271772in}{2.991456in}}%
\pgfpathcurveto{\pgfqpoint{2.280008in}{2.991456in}}{\pgfqpoint{2.287908in}{2.994729in}}{\pgfqpoint{2.293732in}{3.000553in}}%
\pgfpathcurveto{\pgfqpoint{2.299556in}{3.006377in}}{\pgfqpoint{2.302828in}{3.014277in}}{\pgfqpoint{2.302828in}{3.022513in}}%
\pgfpathcurveto{\pgfqpoint{2.302828in}{3.030749in}}{\pgfqpoint{2.299556in}{3.038649in}}{\pgfqpoint{2.293732in}{3.044473in}}%
\pgfpathcurveto{\pgfqpoint{2.287908in}{3.050297in}}{\pgfqpoint{2.280008in}{3.053569in}}{\pgfqpoint{2.271772in}{3.053569in}}%
\pgfpathcurveto{\pgfqpoint{2.263536in}{3.053569in}}{\pgfqpoint{2.255636in}{3.050297in}}{\pgfqpoint{2.249812in}{3.044473in}}%
\pgfpathcurveto{\pgfqpoint{2.243988in}{3.038649in}}{\pgfqpoint{2.240715in}{3.030749in}}{\pgfqpoint{2.240715in}{3.022513in}}%
\pgfpathcurveto{\pgfqpoint{2.240715in}{3.014277in}}{\pgfqpoint{2.243988in}{3.006377in}}{\pgfqpoint{2.249812in}{3.000553in}}%
\pgfpathcurveto{\pgfqpoint{2.255636in}{2.994729in}}{\pgfqpoint{2.263536in}{2.991456in}}{\pgfqpoint{2.271772in}{2.991456in}}%
\pgfpathclose%
\pgfusepath{stroke,fill}%
\end{pgfscope}%
\begin{pgfscope}%
\pgfpathrectangle{\pgfqpoint{0.100000in}{0.212622in}}{\pgfqpoint{3.696000in}{3.696000in}}%
\pgfusepath{clip}%
\pgfsetbuttcap%
\pgfsetroundjoin%
\definecolor{currentfill}{rgb}{0.121569,0.466667,0.705882}%
\pgfsetfillcolor{currentfill}%
\pgfsetfillopacity{0.714555}%
\pgfsetlinewidth{1.003750pt}%
\definecolor{currentstroke}{rgb}{0.121569,0.466667,0.705882}%
\pgfsetstrokecolor{currentstroke}%
\pgfsetstrokeopacity{0.714555}%
\pgfsetdash{}{0pt}%
\pgfpathmoveto{\pgfqpoint{3.273995in}{2.487451in}}%
\pgfpathcurveto{\pgfqpoint{3.282231in}{2.487451in}}{\pgfqpoint{3.290131in}{2.490723in}}{\pgfqpoint{3.295955in}{2.496547in}}%
\pgfpathcurveto{\pgfqpoint{3.301779in}{2.502371in}}{\pgfqpoint{3.305051in}{2.510271in}}{\pgfqpoint{3.305051in}{2.518507in}}%
\pgfpathcurveto{\pgfqpoint{3.305051in}{2.526743in}}{\pgfqpoint{3.301779in}{2.534644in}}{\pgfqpoint{3.295955in}{2.540467in}}%
\pgfpathcurveto{\pgfqpoint{3.290131in}{2.546291in}}{\pgfqpoint{3.282231in}{2.549564in}}{\pgfqpoint{3.273995in}{2.549564in}}%
\pgfpathcurveto{\pgfqpoint{3.265758in}{2.549564in}}{\pgfqpoint{3.257858in}{2.546291in}}{\pgfqpoint{3.252034in}{2.540467in}}%
\pgfpathcurveto{\pgfqpoint{3.246210in}{2.534644in}}{\pgfqpoint{3.242938in}{2.526743in}}{\pgfqpoint{3.242938in}{2.518507in}}%
\pgfpathcurveto{\pgfqpoint{3.242938in}{2.510271in}}{\pgfqpoint{3.246210in}{2.502371in}}{\pgfqpoint{3.252034in}{2.496547in}}%
\pgfpathcurveto{\pgfqpoint{3.257858in}{2.490723in}}{\pgfqpoint{3.265758in}{2.487451in}}{\pgfqpoint{3.273995in}{2.487451in}}%
\pgfpathclose%
\pgfusepath{stroke,fill}%
\end{pgfscope}%
\begin{pgfscope}%
\pgfpathrectangle{\pgfqpoint{0.100000in}{0.212622in}}{\pgfqpoint{3.696000in}{3.696000in}}%
\pgfusepath{clip}%
\pgfsetbuttcap%
\pgfsetroundjoin%
\definecolor{currentfill}{rgb}{0.121569,0.466667,0.705882}%
\pgfsetfillcolor{currentfill}%
\pgfsetfillopacity{0.715109}%
\pgfsetlinewidth{1.003750pt}%
\definecolor{currentstroke}{rgb}{0.121569,0.466667,0.705882}%
\pgfsetstrokecolor{currentstroke}%
\pgfsetstrokeopacity{0.715109}%
\pgfsetdash{}{0pt}%
\pgfpathmoveto{\pgfqpoint{2.277549in}{2.990065in}}%
\pgfpathcurveto{\pgfqpoint{2.285785in}{2.990065in}}{\pgfqpoint{2.293685in}{2.993337in}}{\pgfqpoint{2.299509in}{2.999161in}}%
\pgfpathcurveto{\pgfqpoint{2.305333in}{3.004985in}}{\pgfqpoint{2.308606in}{3.012885in}}{\pgfqpoint{2.308606in}{3.021121in}}%
\pgfpathcurveto{\pgfqpoint{2.308606in}{3.029358in}}{\pgfqpoint{2.305333in}{3.037258in}}{\pgfqpoint{2.299509in}{3.043082in}}%
\pgfpathcurveto{\pgfqpoint{2.293685in}{3.048906in}}{\pgfqpoint{2.285785in}{3.052178in}}{\pgfqpoint{2.277549in}{3.052178in}}%
\pgfpathcurveto{\pgfqpoint{2.269313in}{3.052178in}}{\pgfqpoint{2.261413in}{3.048906in}}{\pgfqpoint{2.255589in}{3.043082in}}%
\pgfpathcurveto{\pgfqpoint{2.249765in}{3.037258in}}{\pgfqpoint{2.246493in}{3.029358in}}{\pgfqpoint{2.246493in}{3.021121in}}%
\pgfpathcurveto{\pgfqpoint{2.246493in}{3.012885in}}{\pgfqpoint{2.249765in}{3.004985in}}{\pgfqpoint{2.255589in}{2.999161in}}%
\pgfpathcurveto{\pgfqpoint{2.261413in}{2.993337in}}{\pgfqpoint{2.269313in}{2.990065in}}{\pgfqpoint{2.277549in}{2.990065in}}%
\pgfpathclose%
\pgfusepath{stroke,fill}%
\end{pgfscope}%
\begin{pgfscope}%
\pgfpathrectangle{\pgfqpoint{0.100000in}{0.212622in}}{\pgfqpoint{3.696000in}{3.696000in}}%
\pgfusepath{clip}%
\pgfsetbuttcap%
\pgfsetroundjoin%
\definecolor{currentfill}{rgb}{0.121569,0.466667,0.705882}%
\pgfsetfillcolor{currentfill}%
\pgfsetfillopacity{0.715294}%
\pgfsetlinewidth{1.003750pt}%
\definecolor{currentstroke}{rgb}{0.121569,0.466667,0.705882}%
\pgfsetstrokecolor{currentstroke}%
\pgfsetstrokeopacity{0.715294}%
\pgfsetdash{}{0pt}%
\pgfpathmoveto{\pgfqpoint{1.283851in}{2.125110in}}%
\pgfpathcurveto{\pgfqpoint{1.292087in}{2.125110in}}{\pgfqpoint{1.299987in}{2.128382in}}{\pgfqpoint{1.305811in}{2.134206in}}%
\pgfpathcurveto{\pgfqpoint{1.311635in}{2.140030in}}{\pgfqpoint{1.314907in}{2.147930in}}{\pgfqpoint{1.314907in}{2.156166in}}%
\pgfpathcurveto{\pgfqpoint{1.314907in}{2.164403in}}{\pgfqpoint{1.311635in}{2.172303in}}{\pgfqpoint{1.305811in}{2.178127in}}%
\pgfpathcurveto{\pgfqpoint{1.299987in}{2.183951in}}{\pgfqpoint{1.292087in}{2.187223in}}{\pgfqpoint{1.283851in}{2.187223in}}%
\pgfpathcurveto{\pgfqpoint{1.275614in}{2.187223in}}{\pgfqpoint{1.267714in}{2.183951in}}{\pgfqpoint{1.261890in}{2.178127in}}%
\pgfpathcurveto{\pgfqpoint{1.256067in}{2.172303in}}{\pgfqpoint{1.252794in}{2.164403in}}{\pgfqpoint{1.252794in}{2.156166in}}%
\pgfpathcurveto{\pgfqpoint{1.252794in}{2.147930in}}{\pgfqpoint{1.256067in}{2.140030in}}{\pgfqpoint{1.261890in}{2.134206in}}%
\pgfpathcurveto{\pgfqpoint{1.267714in}{2.128382in}}{\pgfqpoint{1.275614in}{2.125110in}}{\pgfqpoint{1.283851in}{2.125110in}}%
\pgfpathclose%
\pgfusepath{stroke,fill}%
\end{pgfscope}%
\begin{pgfscope}%
\pgfpathrectangle{\pgfqpoint{0.100000in}{0.212622in}}{\pgfqpoint{3.696000in}{3.696000in}}%
\pgfusepath{clip}%
\pgfsetbuttcap%
\pgfsetroundjoin%
\definecolor{currentfill}{rgb}{0.121569,0.466667,0.705882}%
\pgfsetfillcolor{currentfill}%
\pgfsetfillopacity{0.716077}%
\pgfsetlinewidth{1.003750pt}%
\definecolor{currentstroke}{rgb}{0.121569,0.466667,0.705882}%
\pgfsetstrokecolor{currentstroke}%
\pgfsetstrokeopacity{0.716077}%
\pgfsetdash{}{0pt}%
\pgfpathmoveto{\pgfqpoint{2.282020in}{2.989183in}}%
\pgfpathcurveto{\pgfqpoint{2.290256in}{2.989183in}}{\pgfqpoint{2.298156in}{2.992455in}}{\pgfqpoint{2.303980in}{2.998279in}}%
\pgfpathcurveto{\pgfqpoint{2.309804in}{3.004103in}}{\pgfqpoint{2.313076in}{3.012003in}}{\pgfqpoint{2.313076in}{3.020239in}}%
\pgfpathcurveto{\pgfqpoint{2.313076in}{3.028476in}}{\pgfqpoint{2.309804in}{3.036376in}}{\pgfqpoint{2.303980in}{3.042200in}}%
\pgfpathcurveto{\pgfqpoint{2.298156in}{3.048023in}}{\pgfqpoint{2.290256in}{3.051296in}}{\pgfqpoint{2.282020in}{3.051296in}}%
\pgfpathcurveto{\pgfqpoint{2.273783in}{3.051296in}}{\pgfqpoint{2.265883in}{3.048023in}}{\pgfqpoint{2.260059in}{3.042200in}}%
\pgfpathcurveto{\pgfqpoint{2.254235in}{3.036376in}}{\pgfqpoint{2.250963in}{3.028476in}}{\pgfqpoint{2.250963in}{3.020239in}}%
\pgfpathcurveto{\pgfqpoint{2.250963in}{3.012003in}}{\pgfqpoint{2.254235in}{3.004103in}}{\pgfqpoint{2.260059in}{2.998279in}}%
\pgfpathcurveto{\pgfqpoint{2.265883in}{2.992455in}}{\pgfqpoint{2.273783in}{2.989183in}}{\pgfqpoint{2.282020in}{2.989183in}}%
\pgfpathclose%
\pgfusepath{stroke,fill}%
\end{pgfscope}%
\begin{pgfscope}%
\pgfpathrectangle{\pgfqpoint{0.100000in}{0.212622in}}{\pgfqpoint{3.696000in}{3.696000in}}%
\pgfusepath{clip}%
\pgfsetbuttcap%
\pgfsetroundjoin%
\definecolor{currentfill}{rgb}{0.121569,0.466667,0.705882}%
\pgfsetfillcolor{currentfill}%
\pgfsetfillopacity{0.716828}%
\pgfsetlinewidth{1.003750pt}%
\definecolor{currentstroke}{rgb}{0.121569,0.466667,0.705882}%
\pgfsetstrokecolor{currentstroke}%
\pgfsetstrokeopacity{0.716828}%
\pgfsetdash{}{0pt}%
\pgfpathmoveto{\pgfqpoint{3.270884in}{2.467821in}}%
\pgfpathcurveto{\pgfqpoint{3.279120in}{2.467821in}}{\pgfqpoint{3.287020in}{2.471093in}}{\pgfqpoint{3.292844in}{2.476917in}}%
\pgfpathcurveto{\pgfqpoint{3.298668in}{2.482741in}}{\pgfqpoint{3.301940in}{2.490641in}}{\pgfqpoint{3.301940in}{2.498877in}}%
\pgfpathcurveto{\pgfqpoint{3.301940in}{2.507113in}}{\pgfqpoint{3.298668in}{2.515013in}}{\pgfqpoint{3.292844in}{2.520837in}}%
\pgfpathcurveto{\pgfqpoint{3.287020in}{2.526661in}}{\pgfqpoint{3.279120in}{2.529934in}}{\pgfqpoint{3.270884in}{2.529934in}}%
\pgfpathcurveto{\pgfqpoint{3.262648in}{2.529934in}}{\pgfqpoint{3.254748in}{2.526661in}}{\pgfqpoint{3.248924in}{2.520837in}}%
\pgfpathcurveto{\pgfqpoint{3.243100in}{2.515013in}}{\pgfqpoint{3.239827in}{2.507113in}}{\pgfqpoint{3.239827in}{2.498877in}}%
\pgfpathcurveto{\pgfqpoint{3.239827in}{2.490641in}}{\pgfqpoint{3.243100in}{2.482741in}}{\pgfqpoint{3.248924in}{2.476917in}}%
\pgfpathcurveto{\pgfqpoint{3.254748in}{2.471093in}}{\pgfqpoint{3.262648in}{2.467821in}}{\pgfqpoint{3.270884in}{2.467821in}}%
\pgfpathclose%
\pgfusepath{stroke,fill}%
\end{pgfscope}%
\begin{pgfscope}%
\pgfpathrectangle{\pgfqpoint{0.100000in}{0.212622in}}{\pgfqpoint{3.696000in}{3.696000in}}%
\pgfusepath{clip}%
\pgfsetbuttcap%
\pgfsetroundjoin%
\definecolor{currentfill}{rgb}{0.121569,0.466667,0.705882}%
\pgfsetfillcolor{currentfill}%
\pgfsetfillopacity{0.716947}%
\pgfsetlinewidth{1.003750pt}%
\definecolor{currentstroke}{rgb}{0.121569,0.466667,0.705882}%
\pgfsetstrokecolor{currentstroke}%
\pgfsetstrokeopacity{0.716947}%
\pgfsetdash{}{0pt}%
\pgfpathmoveto{\pgfqpoint{1.277015in}{2.110228in}}%
\pgfpathcurveto{\pgfqpoint{1.285252in}{2.110228in}}{\pgfqpoint{1.293152in}{2.113500in}}{\pgfqpoint{1.298976in}{2.119324in}}%
\pgfpathcurveto{\pgfqpoint{1.304800in}{2.125148in}}{\pgfqpoint{1.308072in}{2.133048in}}{\pgfqpoint{1.308072in}{2.141284in}}%
\pgfpathcurveto{\pgfqpoint{1.308072in}{2.149521in}}{\pgfqpoint{1.304800in}{2.157421in}}{\pgfqpoint{1.298976in}{2.163245in}}%
\pgfpathcurveto{\pgfqpoint{1.293152in}{2.169069in}}{\pgfqpoint{1.285252in}{2.172341in}}{\pgfqpoint{1.277015in}{2.172341in}}%
\pgfpathcurveto{\pgfqpoint{1.268779in}{2.172341in}}{\pgfqpoint{1.260879in}{2.169069in}}{\pgfqpoint{1.255055in}{2.163245in}}%
\pgfpathcurveto{\pgfqpoint{1.249231in}{2.157421in}}{\pgfqpoint{1.245959in}{2.149521in}}{\pgfqpoint{1.245959in}{2.141284in}}%
\pgfpathcurveto{\pgfqpoint{1.245959in}{2.133048in}}{\pgfqpoint{1.249231in}{2.125148in}}{\pgfqpoint{1.255055in}{2.119324in}}%
\pgfpathcurveto{\pgfqpoint{1.260879in}{2.113500in}}{\pgfqpoint{1.268779in}{2.110228in}}{\pgfqpoint{1.277015in}{2.110228in}}%
\pgfpathclose%
\pgfusepath{stroke,fill}%
\end{pgfscope}%
\begin{pgfscope}%
\pgfpathrectangle{\pgfqpoint{0.100000in}{0.212622in}}{\pgfqpoint{3.696000in}{3.696000in}}%
\pgfusepath{clip}%
\pgfsetbuttcap%
\pgfsetroundjoin%
\definecolor{currentfill}{rgb}{0.121569,0.466667,0.705882}%
\pgfsetfillcolor{currentfill}%
\pgfsetfillopacity{0.716958}%
\pgfsetlinewidth{1.003750pt}%
\definecolor{currentstroke}{rgb}{0.121569,0.466667,0.705882}%
\pgfsetstrokecolor{currentstroke}%
\pgfsetstrokeopacity{0.716958}%
\pgfsetdash{}{0pt}%
\pgfpathmoveto{\pgfqpoint{2.286020in}{2.988620in}}%
\pgfpathcurveto{\pgfqpoint{2.294256in}{2.988620in}}{\pgfqpoint{2.302156in}{2.991893in}}{\pgfqpoint{2.307980in}{2.997717in}}%
\pgfpathcurveto{\pgfqpoint{2.313804in}{3.003541in}}{\pgfqpoint{2.317076in}{3.011441in}}{\pgfqpoint{2.317076in}{3.019677in}}%
\pgfpathcurveto{\pgfqpoint{2.317076in}{3.027913in}}{\pgfqpoint{2.313804in}{3.035813in}}{\pgfqpoint{2.307980in}{3.041637in}}%
\pgfpathcurveto{\pgfqpoint{2.302156in}{3.047461in}}{\pgfqpoint{2.294256in}{3.050733in}}{\pgfqpoint{2.286020in}{3.050733in}}%
\pgfpathcurveto{\pgfqpoint{2.277783in}{3.050733in}}{\pgfqpoint{2.269883in}{3.047461in}}{\pgfqpoint{2.264059in}{3.041637in}}%
\pgfpathcurveto{\pgfqpoint{2.258236in}{3.035813in}}{\pgfqpoint{2.254963in}{3.027913in}}{\pgfqpoint{2.254963in}{3.019677in}}%
\pgfpathcurveto{\pgfqpoint{2.254963in}{3.011441in}}{\pgfqpoint{2.258236in}{3.003541in}}{\pgfqpoint{2.264059in}{2.997717in}}%
\pgfpathcurveto{\pgfqpoint{2.269883in}{2.991893in}}{\pgfqpoint{2.277783in}{2.988620in}}{\pgfqpoint{2.286020in}{2.988620in}}%
\pgfpathclose%
\pgfusepath{stroke,fill}%
\end{pgfscope}%
\begin{pgfscope}%
\pgfpathrectangle{\pgfqpoint{0.100000in}{0.212622in}}{\pgfqpoint{3.696000in}{3.696000in}}%
\pgfusepath{clip}%
\pgfsetbuttcap%
\pgfsetroundjoin%
\definecolor{currentfill}{rgb}{0.121569,0.466667,0.705882}%
\pgfsetfillcolor{currentfill}%
\pgfsetfillopacity{0.717580}%
\pgfsetlinewidth{1.003750pt}%
\definecolor{currentstroke}{rgb}{0.121569,0.466667,0.705882}%
\pgfsetstrokecolor{currentstroke}%
\pgfsetstrokeopacity{0.717580}%
\pgfsetdash{}{0pt}%
\pgfpathmoveto{\pgfqpoint{2.288101in}{2.988126in}}%
\pgfpathcurveto{\pgfqpoint{2.296337in}{2.988126in}}{\pgfqpoint{2.304237in}{2.991399in}}{\pgfqpoint{2.310061in}{2.997223in}}%
\pgfpathcurveto{\pgfqpoint{2.315885in}{3.003047in}}{\pgfqpoint{2.319157in}{3.010947in}}{\pgfqpoint{2.319157in}{3.019183in}}%
\pgfpathcurveto{\pgfqpoint{2.319157in}{3.027419in}}{\pgfqpoint{2.315885in}{3.035319in}}{\pgfqpoint{2.310061in}{3.041143in}}%
\pgfpathcurveto{\pgfqpoint{2.304237in}{3.046967in}}{\pgfqpoint{2.296337in}{3.050239in}}{\pgfqpoint{2.288101in}{3.050239in}}%
\pgfpathcurveto{\pgfqpoint{2.279864in}{3.050239in}}{\pgfqpoint{2.271964in}{3.046967in}}{\pgfqpoint{2.266140in}{3.041143in}}%
\pgfpathcurveto{\pgfqpoint{2.260316in}{3.035319in}}{\pgfqpoint{2.257044in}{3.027419in}}{\pgfqpoint{2.257044in}{3.019183in}}%
\pgfpathcurveto{\pgfqpoint{2.257044in}{3.010947in}}{\pgfqpoint{2.260316in}{3.003047in}}{\pgfqpoint{2.266140in}{2.997223in}}%
\pgfpathcurveto{\pgfqpoint{2.271964in}{2.991399in}}{\pgfqpoint{2.279864in}{2.988126in}}{\pgfqpoint{2.288101in}{2.988126in}}%
\pgfpathclose%
\pgfusepath{stroke,fill}%
\end{pgfscope}%
\begin{pgfscope}%
\pgfpathrectangle{\pgfqpoint{0.100000in}{0.212622in}}{\pgfqpoint{3.696000in}{3.696000in}}%
\pgfusepath{clip}%
\pgfsetbuttcap%
\pgfsetroundjoin%
\definecolor{currentfill}{rgb}{0.121569,0.466667,0.705882}%
\pgfsetfillcolor{currentfill}%
\pgfsetfillopacity{0.717943}%
\pgfsetlinewidth{1.003750pt}%
\definecolor{currentstroke}{rgb}{0.121569,0.466667,0.705882}%
\pgfsetstrokecolor{currentstroke}%
\pgfsetstrokeopacity{0.717943}%
\pgfsetdash{}{0pt}%
\pgfpathmoveto{\pgfqpoint{2.289959in}{2.987913in}}%
\pgfpathcurveto{\pgfqpoint{2.298195in}{2.987913in}}{\pgfqpoint{2.306095in}{2.991185in}}{\pgfqpoint{2.311919in}{2.997009in}}%
\pgfpathcurveto{\pgfqpoint{2.317743in}{3.002833in}}{\pgfqpoint{2.321015in}{3.010733in}}{\pgfqpoint{2.321015in}{3.018969in}}%
\pgfpathcurveto{\pgfqpoint{2.321015in}{3.027205in}}{\pgfqpoint{2.317743in}{3.035105in}}{\pgfqpoint{2.311919in}{3.040929in}}%
\pgfpathcurveto{\pgfqpoint{2.306095in}{3.046753in}}{\pgfqpoint{2.298195in}{3.050026in}}{\pgfqpoint{2.289959in}{3.050026in}}%
\pgfpathcurveto{\pgfqpoint{2.281722in}{3.050026in}}{\pgfqpoint{2.273822in}{3.046753in}}{\pgfqpoint{2.267998in}{3.040929in}}%
\pgfpathcurveto{\pgfqpoint{2.262174in}{3.035105in}}{\pgfqpoint{2.258902in}{3.027205in}}{\pgfqpoint{2.258902in}{3.018969in}}%
\pgfpathcurveto{\pgfqpoint{2.258902in}{3.010733in}}{\pgfqpoint{2.262174in}{3.002833in}}{\pgfqpoint{2.267998in}{2.997009in}}%
\pgfpathcurveto{\pgfqpoint{2.273822in}{2.991185in}}{\pgfqpoint{2.281722in}{2.987913in}}{\pgfqpoint{2.289959in}{2.987913in}}%
\pgfpathclose%
\pgfusepath{stroke,fill}%
\end{pgfscope}%
\begin{pgfscope}%
\pgfpathrectangle{\pgfqpoint{0.100000in}{0.212622in}}{\pgfqpoint{3.696000in}{3.696000in}}%
\pgfusepath{clip}%
\pgfsetbuttcap%
\pgfsetroundjoin%
\definecolor{currentfill}{rgb}{0.121569,0.466667,0.705882}%
\pgfsetfillcolor{currentfill}%
\pgfsetfillopacity{0.718623}%
\pgfsetlinewidth{1.003750pt}%
\definecolor{currentstroke}{rgb}{0.121569,0.466667,0.705882}%
\pgfsetstrokecolor{currentstroke}%
\pgfsetstrokeopacity{0.718623}%
\pgfsetdash{}{0pt}%
\pgfpathmoveto{\pgfqpoint{1.273204in}{2.092536in}}%
\pgfpathcurveto{\pgfqpoint{1.281441in}{2.092536in}}{\pgfqpoint{1.289341in}{2.095808in}}{\pgfqpoint{1.295165in}{2.101632in}}%
\pgfpathcurveto{\pgfqpoint{1.300988in}{2.107456in}}{\pgfqpoint{1.304261in}{2.115356in}}{\pgfqpoint{1.304261in}{2.123592in}}%
\pgfpathcurveto{\pgfqpoint{1.304261in}{2.131829in}}{\pgfqpoint{1.300988in}{2.139729in}}{\pgfqpoint{1.295165in}{2.145553in}}%
\pgfpathcurveto{\pgfqpoint{1.289341in}{2.151377in}}{\pgfqpoint{1.281441in}{2.154649in}}{\pgfqpoint{1.273204in}{2.154649in}}%
\pgfpathcurveto{\pgfqpoint{1.264968in}{2.154649in}}{\pgfqpoint{1.257068in}{2.151377in}}{\pgfqpoint{1.251244in}{2.145553in}}%
\pgfpathcurveto{\pgfqpoint{1.245420in}{2.139729in}}{\pgfqpoint{1.242148in}{2.131829in}}{\pgfqpoint{1.242148in}{2.123592in}}%
\pgfpathcurveto{\pgfqpoint{1.242148in}{2.115356in}}{\pgfqpoint{1.245420in}{2.107456in}}{\pgfqpoint{1.251244in}{2.101632in}}%
\pgfpathcurveto{\pgfqpoint{1.257068in}{2.095808in}}{\pgfqpoint{1.264968in}{2.092536in}}{\pgfqpoint{1.273204in}{2.092536in}}%
\pgfpathclose%
\pgfusepath{stroke,fill}%
\end{pgfscope}%
\begin{pgfscope}%
\pgfpathrectangle{\pgfqpoint{0.100000in}{0.212622in}}{\pgfqpoint{3.696000in}{3.696000in}}%
\pgfusepath{clip}%
\pgfsetbuttcap%
\pgfsetroundjoin%
\definecolor{currentfill}{rgb}{0.121569,0.466667,0.705882}%
\pgfsetfillcolor{currentfill}%
\pgfsetfillopacity{0.718686}%
\pgfsetlinewidth{1.003750pt}%
\definecolor{currentstroke}{rgb}{0.121569,0.466667,0.705882}%
\pgfsetstrokecolor{currentstroke}%
\pgfsetstrokeopacity{0.718686}%
\pgfsetdash{}{0pt}%
\pgfpathmoveto{\pgfqpoint{2.293212in}{2.987455in}}%
\pgfpathcurveto{\pgfqpoint{2.301449in}{2.987455in}}{\pgfqpoint{2.309349in}{2.990727in}}{\pgfqpoint{2.315173in}{2.996551in}}%
\pgfpathcurveto{\pgfqpoint{2.320997in}{3.002375in}}{\pgfqpoint{2.324269in}{3.010275in}}{\pgfqpoint{2.324269in}{3.018511in}}%
\pgfpathcurveto{\pgfqpoint{2.324269in}{3.026748in}}{\pgfqpoint{2.320997in}{3.034648in}}{\pgfqpoint{2.315173in}{3.040471in}}%
\pgfpathcurveto{\pgfqpoint{2.309349in}{3.046295in}}{\pgfqpoint{2.301449in}{3.049568in}}{\pgfqpoint{2.293212in}{3.049568in}}%
\pgfpathcurveto{\pgfqpoint{2.284976in}{3.049568in}}{\pgfqpoint{2.277076in}{3.046295in}}{\pgfqpoint{2.271252in}{3.040471in}}%
\pgfpathcurveto{\pgfqpoint{2.265428in}{3.034648in}}{\pgfqpoint{2.262156in}{3.026748in}}{\pgfqpoint{2.262156in}{3.018511in}}%
\pgfpathcurveto{\pgfqpoint{2.262156in}{3.010275in}}{\pgfqpoint{2.265428in}{3.002375in}}{\pgfqpoint{2.271252in}{2.996551in}}%
\pgfpathcurveto{\pgfqpoint{2.277076in}{2.990727in}}{\pgfqpoint{2.284976in}{2.987455in}}{\pgfqpoint{2.293212in}{2.987455in}}%
\pgfpathclose%
\pgfusepath{stroke,fill}%
\end{pgfscope}%
\begin{pgfscope}%
\pgfpathrectangle{\pgfqpoint{0.100000in}{0.212622in}}{\pgfqpoint{3.696000in}{3.696000in}}%
\pgfusepath{clip}%
\pgfsetbuttcap%
\pgfsetroundjoin%
\definecolor{currentfill}{rgb}{0.121569,0.466667,0.705882}%
\pgfsetfillcolor{currentfill}%
\pgfsetfillopacity{0.718809}%
\pgfsetlinewidth{1.003750pt}%
\definecolor{currentstroke}{rgb}{0.121569,0.466667,0.705882}%
\pgfsetstrokecolor{currentstroke}%
\pgfsetstrokeopacity{0.718809}%
\pgfsetdash{}{0pt}%
\pgfpathmoveto{\pgfqpoint{3.264085in}{2.454623in}}%
\pgfpathcurveto{\pgfqpoint{3.272321in}{2.454623in}}{\pgfqpoint{3.280221in}{2.457895in}}{\pgfqpoint{3.286045in}{2.463719in}}%
\pgfpathcurveto{\pgfqpoint{3.291869in}{2.469543in}}{\pgfqpoint{3.295141in}{2.477443in}}{\pgfqpoint{3.295141in}{2.485679in}}%
\pgfpathcurveto{\pgfqpoint{3.295141in}{2.493915in}}{\pgfqpoint{3.291869in}{2.501815in}}{\pgfqpoint{3.286045in}{2.507639in}}%
\pgfpathcurveto{\pgfqpoint{3.280221in}{2.513463in}}{\pgfqpoint{3.272321in}{2.516736in}}{\pgfqpoint{3.264085in}{2.516736in}}%
\pgfpathcurveto{\pgfqpoint{3.255848in}{2.516736in}}{\pgfqpoint{3.247948in}{2.513463in}}{\pgfqpoint{3.242124in}{2.507639in}}%
\pgfpathcurveto{\pgfqpoint{3.236301in}{2.501815in}}{\pgfqpoint{3.233028in}{2.493915in}}{\pgfqpoint{3.233028in}{2.485679in}}%
\pgfpathcurveto{\pgfqpoint{3.233028in}{2.477443in}}{\pgfqpoint{3.236301in}{2.469543in}}{\pgfqpoint{3.242124in}{2.463719in}}%
\pgfpathcurveto{\pgfqpoint{3.247948in}{2.457895in}}{\pgfqpoint{3.255848in}{2.454623in}}{\pgfqpoint{3.264085in}{2.454623in}}%
\pgfpathclose%
\pgfusepath{stroke,fill}%
\end{pgfscope}%
\begin{pgfscope}%
\pgfpathrectangle{\pgfqpoint{0.100000in}{0.212622in}}{\pgfqpoint{3.696000in}{3.696000in}}%
\pgfusepath{clip}%
\pgfsetbuttcap%
\pgfsetroundjoin%
\definecolor{currentfill}{rgb}{0.121569,0.466667,0.705882}%
\pgfsetfillcolor{currentfill}%
\pgfsetfillopacity{0.719265}%
\pgfsetlinewidth{1.003750pt}%
\definecolor{currentstroke}{rgb}{0.121569,0.466667,0.705882}%
\pgfsetstrokecolor{currentstroke}%
\pgfsetstrokeopacity{0.719265}%
\pgfsetdash{}{0pt}%
\pgfpathmoveto{\pgfqpoint{2.295661in}{2.987146in}}%
\pgfpathcurveto{\pgfqpoint{2.303898in}{2.987146in}}{\pgfqpoint{2.311798in}{2.990419in}}{\pgfqpoint{2.317622in}{2.996243in}}%
\pgfpathcurveto{\pgfqpoint{2.323446in}{3.002067in}}{\pgfqpoint{2.326718in}{3.009967in}}{\pgfqpoint{2.326718in}{3.018203in}}%
\pgfpathcurveto{\pgfqpoint{2.326718in}{3.026439in}}{\pgfqpoint{2.323446in}{3.034339in}}{\pgfqpoint{2.317622in}{3.040163in}}%
\pgfpathcurveto{\pgfqpoint{2.311798in}{3.045987in}}{\pgfqpoint{2.303898in}{3.049259in}}{\pgfqpoint{2.295661in}{3.049259in}}%
\pgfpathcurveto{\pgfqpoint{2.287425in}{3.049259in}}{\pgfqpoint{2.279525in}{3.045987in}}{\pgfqpoint{2.273701in}{3.040163in}}%
\pgfpathcurveto{\pgfqpoint{2.267877in}{3.034339in}}{\pgfqpoint{2.264605in}{3.026439in}}{\pgfqpoint{2.264605in}{3.018203in}}%
\pgfpathcurveto{\pgfqpoint{2.264605in}{3.009967in}}{\pgfqpoint{2.267877in}{3.002067in}}{\pgfqpoint{2.273701in}{2.996243in}}%
\pgfpathcurveto{\pgfqpoint{2.279525in}{2.990419in}}{\pgfqpoint{2.287425in}{2.987146in}}{\pgfqpoint{2.295661in}{2.987146in}}%
\pgfpathclose%
\pgfusepath{stroke,fill}%
\end{pgfscope}%
\begin{pgfscope}%
\pgfpathrectangle{\pgfqpoint{0.100000in}{0.212622in}}{\pgfqpoint{3.696000in}{3.696000in}}%
\pgfusepath{clip}%
\pgfsetbuttcap%
\pgfsetroundjoin%
\definecolor{currentfill}{rgb}{0.121569,0.466667,0.705882}%
\pgfsetfillcolor{currentfill}%
\pgfsetfillopacity{0.720023}%
\pgfsetlinewidth{1.003750pt}%
\definecolor{currentstroke}{rgb}{0.121569,0.466667,0.705882}%
\pgfsetstrokecolor{currentstroke}%
\pgfsetstrokeopacity{0.720023}%
\pgfsetdash{}{0pt}%
\pgfpathmoveto{\pgfqpoint{3.256897in}{2.445131in}}%
\pgfpathcurveto{\pgfqpoint{3.265133in}{2.445131in}}{\pgfqpoint{3.273033in}{2.448403in}}{\pgfqpoint{3.278857in}{2.454227in}}%
\pgfpathcurveto{\pgfqpoint{3.284681in}{2.460051in}}{\pgfqpoint{3.287954in}{2.467951in}}{\pgfqpoint{3.287954in}{2.476187in}}%
\pgfpathcurveto{\pgfqpoint{3.287954in}{2.484424in}}{\pgfqpoint{3.284681in}{2.492324in}}{\pgfqpoint{3.278857in}{2.498148in}}%
\pgfpathcurveto{\pgfqpoint{3.273033in}{2.503971in}}{\pgfqpoint{3.265133in}{2.507244in}}{\pgfqpoint{3.256897in}{2.507244in}}%
\pgfpathcurveto{\pgfqpoint{3.248661in}{2.507244in}}{\pgfqpoint{3.240761in}{2.503971in}}{\pgfqpoint{3.234937in}{2.498148in}}%
\pgfpathcurveto{\pgfqpoint{3.229113in}{2.492324in}}{\pgfqpoint{3.225841in}{2.484424in}}{\pgfqpoint{3.225841in}{2.476187in}}%
\pgfpathcurveto{\pgfqpoint{3.225841in}{2.467951in}}{\pgfqpoint{3.229113in}{2.460051in}}{\pgfqpoint{3.234937in}{2.454227in}}%
\pgfpathcurveto{\pgfqpoint{3.240761in}{2.448403in}}{\pgfqpoint{3.248661in}{2.445131in}}{\pgfqpoint{3.256897in}{2.445131in}}%
\pgfpathclose%
\pgfusepath{stroke,fill}%
\end{pgfscope}%
\begin{pgfscope}%
\pgfpathrectangle{\pgfqpoint{0.100000in}{0.212622in}}{\pgfqpoint{3.696000in}{3.696000in}}%
\pgfusepath{clip}%
\pgfsetbuttcap%
\pgfsetroundjoin%
\definecolor{currentfill}{rgb}{0.121569,0.466667,0.705882}%
\pgfsetfillcolor{currentfill}%
\pgfsetfillopacity{0.720036}%
\pgfsetlinewidth{1.003750pt}%
\definecolor{currentstroke}{rgb}{0.121569,0.466667,0.705882}%
\pgfsetstrokecolor{currentstroke}%
\pgfsetstrokeopacity{0.720036}%
\pgfsetdash{}{0pt}%
\pgfpathmoveto{\pgfqpoint{2.300537in}{2.986891in}}%
\pgfpathcurveto{\pgfqpoint{2.308773in}{2.986891in}}{\pgfqpoint{2.316673in}{2.990163in}}{\pgfqpoint{2.322497in}{2.995987in}}%
\pgfpathcurveto{\pgfqpoint{2.328321in}{3.001811in}}{\pgfqpoint{2.331594in}{3.009711in}}{\pgfqpoint{2.331594in}{3.017947in}}%
\pgfpathcurveto{\pgfqpoint{2.331594in}{3.026183in}}{\pgfqpoint{2.328321in}{3.034083in}}{\pgfqpoint{2.322497in}{3.039907in}}%
\pgfpathcurveto{\pgfqpoint{2.316673in}{3.045731in}}{\pgfqpoint{2.308773in}{3.049004in}}{\pgfqpoint{2.300537in}{3.049004in}}%
\pgfpathcurveto{\pgfqpoint{2.292301in}{3.049004in}}{\pgfqpoint{2.284401in}{3.045731in}}{\pgfqpoint{2.278577in}{3.039907in}}%
\pgfpathcurveto{\pgfqpoint{2.272753in}{3.034083in}}{\pgfqpoint{2.269481in}{3.026183in}}{\pgfqpoint{2.269481in}{3.017947in}}%
\pgfpathcurveto{\pgfqpoint{2.269481in}{3.009711in}}{\pgfqpoint{2.272753in}{3.001811in}}{\pgfqpoint{2.278577in}{2.995987in}}%
\pgfpathcurveto{\pgfqpoint{2.284401in}{2.990163in}}{\pgfqpoint{2.292301in}{2.986891in}}{\pgfqpoint{2.300537in}{2.986891in}}%
\pgfpathclose%
\pgfusepath{stroke,fill}%
\end{pgfscope}%
\begin{pgfscope}%
\pgfpathrectangle{\pgfqpoint{0.100000in}{0.212622in}}{\pgfqpoint{3.696000in}{3.696000in}}%
\pgfusepath{clip}%
\pgfsetbuttcap%
\pgfsetroundjoin%
\definecolor{currentfill}{rgb}{0.121569,0.466667,0.705882}%
\pgfsetfillcolor{currentfill}%
\pgfsetfillopacity{0.720515}%
\pgfsetlinewidth{1.003750pt}%
\definecolor{currentstroke}{rgb}{0.121569,0.466667,0.705882}%
\pgfsetstrokecolor{currentstroke}%
\pgfsetstrokeopacity{0.720515}%
\pgfsetdash{}{0pt}%
\pgfpathmoveto{\pgfqpoint{1.267012in}{2.075965in}}%
\pgfpathcurveto{\pgfqpoint{1.275249in}{2.075965in}}{\pgfqpoint{1.283149in}{2.079237in}}{\pgfqpoint{1.288973in}{2.085061in}}%
\pgfpathcurveto{\pgfqpoint{1.294797in}{2.090885in}}{\pgfqpoint{1.298069in}{2.098785in}}{\pgfqpoint{1.298069in}{2.107021in}}%
\pgfpathcurveto{\pgfqpoint{1.298069in}{2.115257in}}{\pgfqpoint{1.294797in}{2.123158in}}{\pgfqpoint{1.288973in}{2.128981in}}%
\pgfpathcurveto{\pgfqpoint{1.283149in}{2.134805in}}{\pgfqpoint{1.275249in}{2.138078in}}{\pgfqpoint{1.267012in}{2.138078in}}%
\pgfpathcurveto{\pgfqpoint{1.258776in}{2.138078in}}{\pgfqpoint{1.250876in}{2.134805in}}{\pgfqpoint{1.245052in}{2.128981in}}%
\pgfpathcurveto{\pgfqpoint{1.239228in}{2.123158in}}{\pgfqpoint{1.235956in}{2.115257in}}{\pgfqpoint{1.235956in}{2.107021in}}%
\pgfpathcurveto{\pgfqpoint{1.235956in}{2.098785in}}{\pgfqpoint{1.239228in}{2.090885in}}{\pgfqpoint{1.245052in}{2.085061in}}%
\pgfpathcurveto{\pgfqpoint{1.250876in}{2.079237in}}{\pgfqpoint{1.258776in}{2.075965in}}{\pgfqpoint{1.267012in}{2.075965in}}%
\pgfpathclose%
\pgfusepath{stroke,fill}%
\end{pgfscope}%
\begin{pgfscope}%
\pgfpathrectangle{\pgfqpoint{0.100000in}{0.212622in}}{\pgfqpoint{3.696000in}{3.696000in}}%
\pgfusepath{clip}%
\pgfsetbuttcap%
\pgfsetroundjoin%
\definecolor{currentfill}{rgb}{0.121569,0.466667,0.705882}%
\pgfsetfillcolor{currentfill}%
\pgfsetfillopacity{0.720831}%
\pgfsetlinewidth{1.003750pt}%
\definecolor{currentstroke}{rgb}{0.121569,0.466667,0.705882}%
\pgfsetstrokecolor{currentstroke}%
\pgfsetstrokeopacity{0.720831}%
\pgfsetdash{}{0pt}%
\pgfpathmoveto{\pgfqpoint{2.303714in}{2.987198in}}%
\pgfpathcurveto{\pgfqpoint{2.311950in}{2.987198in}}{\pgfqpoint{2.319850in}{2.990470in}}{\pgfqpoint{2.325674in}{2.996294in}}%
\pgfpathcurveto{\pgfqpoint{2.331498in}{3.002118in}}{\pgfqpoint{2.334771in}{3.010018in}}{\pgfqpoint{2.334771in}{3.018254in}}%
\pgfpathcurveto{\pgfqpoint{2.334771in}{3.026490in}}{\pgfqpoint{2.331498in}{3.034390in}}{\pgfqpoint{2.325674in}{3.040214in}}%
\pgfpathcurveto{\pgfqpoint{2.319850in}{3.046038in}}{\pgfqpoint{2.311950in}{3.049311in}}{\pgfqpoint{2.303714in}{3.049311in}}%
\pgfpathcurveto{\pgfqpoint{2.295478in}{3.049311in}}{\pgfqpoint{2.287578in}{3.046038in}}{\pgfqpoint{2.281754in}{3.040214in}}%
\pgfpathcurveto{\pgfqpoint{2.275930in}{3.034390in}}{\pgfqpoint{2.272658in}{3.026490in}}{\pgfqpoint{2.272658in}{3.018254in}}%
\pgfpathcurveto{\pgfqpoint{2.272658in}{3.010018in}}{\pgfqpoint{2.275930in}{3.002118in}}{\pgfqpoint{2.281754in}{2.996294in}}%
\pgfpathcurveto{\pgfqpoint{2.287578in}{2.990470in}}{\pgfqpoint{2.295478in}{2.987198in}}{\pgfqpoint{2.303714in}{2.987198in}}%
\pgfpathclose%
\pgfusepath{stroke,fill}%
\end{pgfscope}%
\begin{pgfscope}%
\pgfpathrectangle{\pgfqpoint{0.100000in}{0.212622in}}{\pgfqpoint{3.696000in}{3.696000in}}%
\pgfusepath{clip}%
\pgfsetbuttcap%
\pgfsetroundjoin%
\definecolor{currentfill}{rgb}{0.121569,0.466667,0.705882}%
\pgfsetfillcolor{currentfill}%
\pgfsetfillopacity{0.721132}%
\pgfsetlinewidth{1.003750pt}%
\definecolor{currentstroke}{rgb}{0.121569,0.466667,0.705882}%
\pgfsetstrokecolor{currentstroke}%
\pgfsetstrokeopacity{0.721132}%
\pgfsetdash{}{0pt}%
\pgfpathmoveto{\pgfqpoint{3.251022in}{2.434533in}}%
\pgfpathcurveto{\pgfqpoint{3.259258in}{2.434533in}}{\pgfqpoint{3.267158in}{2.437806in}}{\pgfqpoint{3.272982in}{2.443630in}}%
\pgfpathcurveto{\pgfqpoint{3.278806in}{2.449454in}}{\pgfqpoint{3.282078in}{2.457354in}}{\pgfqpoint{3.282078in}{2.465590in}}%
\pgfpathcurveto{\pgfqpoint{3.282078in}{2.473826in}}{\pgfqpoint{3.278806in}{2.481726in}}{\pgfqpoint{3.272982in}{2.487550in}}%
\pgfpathcurveto{\pgfqpoint{3.267158in}{2.493374in}}{\pgfqpoint{3.259258in}{2.496646in}}{\pgfqpoint{3.251022in}{2.496646in}}%
\pgfpathcurveto{\pgfqpoint{3.242786in}{2.496646in}}{\pgfqpoint{3.234886in}{2.493374in}}{\pgfqpoint{3.229062in}{2.487550in}}%
\pgfpathcurveto{\pgfqpoint{3.223238in}{2.481726in}}{\pgfqpoint{3.219965in}{2.473826in}}{\pgfqpoint{3.219965in}{2.465590in}}%
\pgfpathcurveto{\pgfqpoint{3.219965in}{2.457354in}}{\pgfqpoint{3.223238in}{2.449454in}}{\pgfqpoint{3.229062in}{2.443630in}}%
\pgfpathcurveto{\pgfqpoint{3.234886in}{2.437806in}}{\pgfqpoint{3.242786in}{2.434533in}}{\pgfqpoint{3.251022in}{2.434533in}}%
\pgfpathclose%
\pgfusepath{stroke,fill}%
\end{pgfscope}%
\begin{pgfscope}%
\pgfpathrectangle{\pgfqpoint{0.100000in}{0.212622in}}{\pgfqpoint{3.696000in}{3.696000in}}%
\pgfusepath{clip}%
\pgfsetbuttcap%
\pgfsetroundjoin%
\definecolor{currentfill}{rgb}{0.121569,0.466667,0.705882}%
\pgfsetfillcolor{currentfill}%
\pgfsetfillopacity{0.721986}%
\pgfsetlinewidth{1.003750pt}%
\definecolor{currentstroke}{rgb}{0.121569,0.466667,0.705882}%
\pgfsetstrokecolor{currentstroke}%
\pgfsetstrokeopacity{0.721986}%
\pgfsetdash{}{0pt}%
\pgfpathmoveto{\pgfqpoint{2.309407in}{2.986150in}}%
\pgfpathcurveto{\pgfqpoint{2.317643in}{2.986150in}}{\pgfqpoint{2.325543in}{2.989423in}}{\pgfqpoint{2.331367in}{2.995247in}}%
\pgfpathcurveto{\pgfqpoint{2.337191in}{3.001070in}}{\pgfqpoint{2.340463in}{3.008971in}}{\pgfqpoint{2.340463in}{3.017207in}}%
\pgfpathcurveto{\pgfqpoint{2.340463in}{3.025443in}}{\pgfqpoint{2.337191in}{3.033343in}}{\pgfqpoint{2.331367in}{3.039167in}}%
\pgfpathcurveto{\pgfqpoint{2.325543in}{3.044991in}}{\pgfqpoint{2.317643in}{3.048263in}}{\pgfqpoint{2.309407in}{3.048263in}}%
\pgfpathcurveto{\pgfqpoint{2.301171in}{3.048263in}}{\pgfqpoint{2.293271in}{3.044991in}}{\pgfqpoint{2.287447in}{3.039167in}}%
\pgfpathcurveto{\pgfqpoint{2.281623in}{3.033343in}}{\pgfqpoint{2.278350in}{3.025443in}}{\pgfqpoint{2.278350in}{3.017207in}}%
\pgfpathcurveto{\pgfqpoint{2.278350in}{3.008971in}}{\pgfqpoint{2.281623in}{3.001070in}}{\pgfqpoint{2.287447in}{2.995247in}}%
\pgfpathcurveto{\pgfqpoint{2.293271in}{2.989423in}}{\pgfqpoint{2.301171in}{2.986150in}}{\pgfqpoint{2.309407in}{2.986150in}}%
\pgfpathclose%
\pgfusepath{stroke,fill}%
\end{pgfscope}%
\begin{pgfscope}%
\pgfpathrectangle{\pgfqpoint{0.100000in}{0.212622in}}{\pgfqpoint{3.696000in}{3.696000in}}%
\pgfusepath{clip}%
\pgfsetbuttcap%
\pgfsetroundjoin%
\definecolor{currentfill}{rgb}{0.121569,0.466667,0.705882}%
\pgfsetfillcolor{currentfill}%
\pgfsetfillopacity{0.722219}%
\pgfsetlinewidth{1.003750pt}%
\definecolor{currentstroke}{rgb}{0.121569,0.466667,0.705882}%
\pgfsetstrokecolor{currentstroke}%
\pgfsetstrokeopacity{0.722219}%
\pgfsetdash{}{0pt}%
\pgfpathmoveto{\pgfqpoint{1.257554in}{2.061267in}}%
\pgfpathcurveto{\pgfqpoint{1.265790in}{2.061267in}}{\pgfqpoint{1.273690in}{2.064540in}}{\pgfqpoint{1.279514in}{2.070364in}}%
\pgfpathcurveto{\pgfqpoint{1.285338in}{2.076188in}}{\pgfqpoint{1.288610in}{2.084088in}}{\pgfqpoint{1.288610in}{2.092324in}}%
\pgfpathcurveto{\pgfqpoint{1.288610in}{2.100560in}}{\pgfqpoint{1.285338in}{2.108460in}}{\pgfqpoint{1.279514in}{2.114284in}}%
\pgfpathcurveto{\pgfqpoint{1.273690in}{2.120108in}}{\pgfqpoint{1.265790in}{2.123380in}}{\pgfqpoint{1.257554in}{2.123380in}}%
\pgfpathcurveto{\pgfqpoint{1.249318in}{2.123380in}}{\pgfqpoint{1.241417in}{2.120108in}}{\pgfqpoint{1.235594in}{2.114284in}}%
\pgfpathcurveto{\pgfqpoint{1.229770in}{2.108460in}}{\pgfqpoint{1.226497in}{2.100560in}}{\pgfqpoint{1.226497in}{2.092324in}}%
\pgfpathcurveto{\pgfqpoint{1.226497in}{2.084088in}}{\pgfqpoint{1.229770in}{2.076188in}}{\pgfqpoint{1.235594in}{2.070364in}}%
\pgfpathcurveto{\pgfqpoint{1.241417in}{2.064540in}}{\pgfqpoint{1.249318in}{2.061267in}}{\pgfqpoint{1.257554in}{2.061267in}}%
\pgfpathclose%
\pgfusepath{stroke,fill}%
\end{pgfscope}%
\begin{pgfscope}%
\pgfpathrectangle{\pgfqpoint{0.100000in}{0.212622in}}{\pgfqpoint{3.696000in}{3.696000in}}%
\pgfusepath{clip}%
\pgfsetbuttcap%
\pgfsetroundjoin%
\definecolor{currentfill}{rgb}{0.121569,0.466667,0.705882}%
\pgfsetfillcolor{currentfill}%
\pgfsetfillopacity{0.722378}%
\pgfsetlinewidth{1.003750pt}%
\definecolor{currentstroke}{rgb}{0.121569,0.466667,0.705882}%
\pgfsetstrokecolor{currentstroke}%
\pgfsetstrokeopacity{0.722378}%
\pgfsetdash{}{0pt}%
\pgfpathmoveto{\pgfqpoint{3.246896in}{2.424568in}}%
\pgfpathcurveto{\pgfqpoint{3.255133in}{2.424568in}}{\pgfqpoint{3.263033in}{2.427840in}}{\pgfqpoint{3.268857in}{2.433664in}}%
\pgfpathcurveto{\pgfqpoint{3.274681in}{2.439488in}}{\pgfqpoint{3.277953in}{2.447388in}}{\pgfqpoint{3.277953in}{2.455624in}}%
\pgfpathcurveto{\pgfqpoint{3.277953in}{2.463860in}}{\pgfqpoint{3.274681in}{2.471761in}}{\pgfqpoint{3.268857in}{2.477584in}}%
\pgfpathcurveto{\pgfqpoint{3.263033in}{2.483408in}}{\pgfqpoint{3.255133in}{2.486681in}}{\pgfqpoint{3.246896in}{2.486681in}}%
\pgfpathcurveto{\pgfqpoint{3.238660in}{2.486681in}}{\pgfqpoint{3.230760in}{2.483408in}}{\pgfqpoint{3.224936in}{2.477584in}}%
\pgfpathcurveto{\pgfqpoint{3.219112in}{2.471761in}}{\pgfqpoint{3.215840in}{2.463860in}}{\pgfqpoint{3.215840in}{2.455624in}}%
\pgfpathcurveto{\pgfqpoint{3.215840in}{2.447388in}}{\pgfqpoint{3.219112in}{2.439488in}}{\pgfqpoint{3.224936in}{2.433664in}}%
\pgfpathcurveto{\pgfqpoint{3.230760in}{2.427840in}}{\pgfqpoint{3.238660in}{2.424568in}}{\pgfqpoint{3.246896in}{2.424568in}}%
\pgfpathclose%
\pgfusepath{stroke,fill}%
\end{pgfscope}%
\begin{pgfscope}%
\pgfpathrectangle{\pgfqpoint{0.100000in}{0.212622in}}{\pgfqpoint{3.696000in}{3.696000in}}%
\pgfusepath{clip}%
\pgfsetbuttcap%
\pgfsetroundjoin%
\definecolor{currentfill}{rgb}{0.121569,0.466667,0.705882}%
\pgfsetfillcolor{currentfill}%
\pgfsetfillopacity{0.722933}%
\pgfsetlinewidth{1.003750pt}%
\definecolor{currentstroke}{rgb}{0.121569,0.466667,0.705882}%
\pgfsetstrokecolor{currentstroke}%
\pgfsetstrokeopacity{0.722933}%
\pgfsetdash{}{0pt}%
\pgfpathmoveto{\pgfqpoint{2.313958in}{2.985489in}}%
\pgfpathcurveto{\pgfqpoint{2.322195in}{2.985489in}}{\pgfqpoint{2.330095in}{2.988761in}}{\pgfqpoint{2.335919in}{2.994585in}}%
\pgfpathcurveto{\pgfqpoint{2.341743in}{3.000409in}}{\pgfqpoint{2.345015in}{3.008309in}}{\pgfqpoint{2.345015in}{3.016545in}}%
\pgfpathcurveto{\pgfqpoint{2.345015in}{3.024782in}}{\pgfqpoint{2.341743in}{3.032682in}}{\pgfqpoint{2.335919in}{3.038506in}}%
\pgfpathcurveto{\pgfqpoint{2.330095in}{3.044330in}}{\pgfqpoint{2.322195in}{3.047602in}}{\pgfqpoint{2.313958in}{3.047602in}}%
\pgfpathcurveto{\pgfqpoint{2.305722in}{3.047602in}}{\pgfqpoint{2.297822in}{3.044330in}}{\pgfqpoint{2.291998in}{3.038506in}}%
\pgfpathcurveto{\pgfqpoint{2.286174in}{3.032682in}}{\pgfqpoint{2.282902in}{3.024782in}}{\pgfqpoint{2.282902in}{3.016545in}}%
\pgfpathcurveto{\pgfqpoint{2.282902in}{3.008309in}}{\pgfqpoint{2.286174in}{3.000409in}}{\pgfqpoint{2.291998in}{2.994585in}}%
\pgfpathcurveto{\pgfqpoint{2.297822in}{2.988761in}}{\pgfqpoint{2.305722in}{2.985489in}}{\pgfqpoint{2.313958in}{2.985489in}}%
\pgfpathclose%
\pgfusepath{stroke,fill}%
\end{pgfscope}%
\begin{pgfscope}%
\pgfpathrectangle{\pgfqpoint{0.100000in}{0.212622in}}{\pgfqpoint{3.696000in}{3.696000in}}%
\pgfusepath{clip}%
\pgfsetbuttcap%
\pgfsetroundjoin%
\definecolor{currentfill}{rgb}{0.121569,0.466667,0.705882}%
\pgfsetfillcolor{currentfill}%
\pgfsetfillopacity{0.723651}%
\pgfsetlinewidth{1.003750pt}%
\definecolor{currentstroke}{rgb}{0.121569,0.466667,0.705882}%
\pgfsetstrokecolor{currentstroke}%
\pgfsetstrokeopacity{0.723651}%
\pgfsetdash{}{0pt}%
\pgfpathmoveto{\pgfqpoint{3.244931in}{2.415867in}}%
\pgfpathcurveto{\pgfqpoint{3.253167in}{2.415867in}}{\pgfqpoint{3.261067in}{2.419140in}}{\pgfqpoint{3.266891in}{2.424964in}}%
\pgfpathcurveto{\pgfqpoint{3.272715in}{2.430788in}}{\pgfqpoint{3.275987in}{2.438688in}}{\pgfqpoint{3.275987in}{2.446924in}}%
\pgfpathcurveto{\pgfqpoint{3.275987in}{2.455160in}}{\pgfqpoint{3.272715in}{2.463060in}}{\pgfqpoint{3.266891in}{2.468884in}}%
\pgfpathcurveto{\pgfqpoint{3.261067in}{2.474708in}}{\pgfqpoint{3.253167in}{2.477980in}}{\pgfqpoint{3.244931in}{2.477980in}}%
\pgfpathcurveto{\pgfqpoint{3.236694in}{2.477980in}}{\pgfqpoint{3.228794in}{2.474708in}}{\pgfqpoint{3.222971in}{2.468884in}}%
\pgfpathcurveto{\pgfqpoint{3.217147in}{2.463060in}}{\pgfqpoint{3.213874in}{2.455160in}}{\pgfqpoint{3.213874in}{2.446924in}}%
\pgfpathcurveto{\pgfqpoint{3.213874in}{2.438688in}}{\pgfqpoint{3.217147in}{2.430788in}}{\pgfqpoint{3.222971in}{2.424964in}}%
\pgfpathcurveto{\pgfqpoint{3.228794in}{2.419140in}}{\pgfqpoint{3.236694in}{2.415867in}}{\pgfqpoint{3.244931in}{2.415867in}}%
\pgfpathclose%
\pgfusepath{stroke,fill}%
\end{pgfscope}%
\begin{pgfscope}%
\pgfpathrectangle{\pgfqpoint{0.100000in}{0.212622in}}{\pgfqpoint{3.696000in}{3.696000in}}%
\pgfusepath{clip}%
\pgfsetbuttcap%
\pgfsetroundjoin%
\definecolor{currentfill}{rgb}{0.121569,0.466667,0.705882}%
\pgfsetfillcolor{currentfill}%
\pgfsetfillopacity{0.723987}%
\pgfsetlinewidth{1.003750pt}%
\definecolor{currentstroke}{rgb}{0.121569,0.466667,0.705882}%
\pgfsetstrokecolor{currentstroke}%
\pgfsetstrokeopacity{0.723987}%
\pgfsetdash{}{0pt}%
\pgfpathmoveto{\pgfqpoint{1.247272in}{2.047653in}}%
\pgfpathcurveto{\pgfqpoint{1.255509in}{2.047653in}}{\pgfqpoint{1.263409in}{2.050926in}}{\pgfqpoint{1.269233in}{2.056750in}}%
\pgfpathcurveto{\pgfqpoint{1.275057in}{2.062574in}}{\pgfqpoint{1.278329in}{2.070474in}}{\pgfqpoint{1.278329in}{2.078710in}}%
\pgfpathcurveto{\pgfqpoint{1.278329in}{2.086946in}}{\pgfqpoint{1.275057in}{2.094846in}}{\pgfqpoint{1.269233in}{2.100670in}}%
\pgfpathcurveto{\pgfqpoint{1.263409in}{2.106494in}}{\pgfqpoint{1.255509in}{2.109766in}}{\pgfqpoint{1.247272in}{2.109766in}}%
\pgfpathcurveto{\pgfqpoint{1.239036in}{2.109766in}}{\pgfqpoint{1.231136in}{2.106494in}}{\pgfqpoint{1.225312in}{2.100670in}}%
\pgfpathcurveto{\pgfqpoint{1.219488in}{2.094846in}}{\pgfqpoint{1.216216in}{2.086946in}}{\pgfqpoint{1.216216in}{2.078710in}}%
\pgfpathcurveto{\pgfqpoint{1.216216in}{2.070474in}}{\pgfqpoint{1.219488in}{2.062574in}}{\pgfqpoint{1.225312in}{2.056750in}}%
\pgfpathcurveto{\pgfqpoint{1.231136in}{2.050926in}}{\pgfqpoint{1.239036in}{2.047653in}}{\pgfqpoint{1.247272in}{2.047653in}}%
\pgfpathclose%
\pgfusepath{stroke,fill}%
\end{pgfscope}%
\begin{pgfscope}%
\pgfpathrectangle{\pgfqpoint{0.100000in}{0.212622in}}{\pgfqpoint{3.696000in}{3.696000in}}%
\pgfusepath{clip}%
\pgfsetbuttcap%
\pgfsetroundjoin%
\definecolor{currentfill}{rgb}{0.121569,0.466667,0.705882}%
\pgfsetfillcolor{currentfill}%
\pgfsetfillopacity{0.724531}%
\pgfsetlinewidth{1.003750pt}%
\definecolor{currentstroke}{rgb}{0.121569,0.466667,0.705882}%
\pgfsetstrokecolor{currentstroke}%
\pgfsetstrokeopacity{0.724531}%
\pgfsetdash{}{0pt}%
\pgfpathmoveto{\pgfqpoint{2.322311in}{2.983984in}}%
\pgfpathcurveto{\pgfqpoint{2.330547in}{2.983984in}}{\pgfqpoint{2.338447in}{2.987256in}}{\pgfqpoint{2.344271in}{2.993080in}}%
\pgfpathcurveto{\pgfqpoint{2.350095in}{2.998904in}}{\pgfqpoint{2.353367in}{3.006804in}}{\pgfqpoint{2.353367in}{3.015040in}}%
\pgfpathcurveto{\pgfqpoint{2.353367in}{3.023276in}}{\pgfqpoint{2.350095in}{3.031176in}}{\pgfqpoint{2.344271in}{3.037000in}}%
\pgfpathcurveto{\pgfqpoint{2.338447in}{3.042824in}}{\pgfqpoint{2.330547in}{3.046097in}}{\pgfqpoint{2.322311in}{3.046097in}}%
\pgfpathcurveto{\pgfqpoint{2.314075in}{3.046097in}}{\pgfqpoint{2.306174in}{3.042824in}}{\pgfqpoint{2.300351in}{3.037000in}}%
\pgfpathcurveto{\pgfqpoint{2.294527in}{3.031176in}}{\pgfqpoint{2.291254in}{3.023276in}}{\pgfqpoint{2.291254in}{3.015040in}}%
\pgfpathcurveto{\pgfqpoint{2.291254in}{3.006804in}}{\pgfqpoint{2.294527in}{2.998904in}}{\pgfqpoint{2.300351in}{2.993080in}}%
\pgfpathcurveto{\pgfqpoint{2.306174in}{2.987256in}}{\pgfqpoint{2.314075in}{2.983984in}}{\pgfqpoint{2.322311in}{2.983984in}}%
\pgfpathclose%
\pgfusepath{stroke,fill}%
\end{pgfscope}%
\begin{pgfscope}%
\pgfpathrectangle{\pgfqpoint{0.100000in}{0.212622in}}{\pgfqpoint{3.696000in}{3.696000in}}%
\pgfusepath{clip}%
\pgfsetbuttcap%
\pgfsetroundjoin%
\definecolor{currentfill}{rgb}{0.121569,0.466667,0.705882}%
\pgfsetfillcolor{currentfill}%
\pgfsetfillopacity{0.724742}%
\pgfsetlinewidth{1.003750pt}%
\definecolor{currentstroke}{rgb}{0.121569,0.466667,0.705882}%
\pgfsetstrokecolor{currentstroke}%
\pgfsetstrokeopacity{0.724742}%
\pgfsetdash{}{0pt}%
\pgfpathmoveto{\pgfqpoint{3.243524in}{2.406983in}}%
\pgfpathcurveto{\pgfqpoint{3.251760in}{2.406983in}}{\pgfqpoint{3.259660in}{2.410256in}}{\pgfqpoint{3.265484in}{2.416080in}}%
\pgfpathcurveto{\pgfqpoint{3.271308in}{2.421904in}}{\pgfqpoint{3.274581in}{2.429804in}}{\pgfqpoint{3.274581in}{2.438040in}}%
\pgfpathcurveto{\pgfqpoint{3.274581in}{2.446276in}}{\pgfqpoint{3.271308in}{2.454176in}}{\pgfqpoint{3.265484in}{2.460000in}}%
\pgfpathcurveto{\pgfqpoint{3.259660in}{2.465824in}}{\pgfqpoint{3.251760in}{2.469096in}}{\pgfqpoint{3.243524in}{2.469096in}}%
\pgfpathcurveto{\pgfqpoint{3.235288in}{2.469096in}}{\pgfqpoint{3.227388in}{2.465824in}}{\pgfqpoint{3.221564in}{2.460000in}}%
\pgfpathcurveto{\pgfqpoint{3.215740in}{2.454176in}}{\pgfqpoint{3.212468in}{2.446276in}}{\pgfqpoint{3.212468in}{2.438040in}}%
\pgfpathcurveto{\pgfqpoint{3.212468in}{2.429804in}}{\pgfqpoint{3.215740in}{2.421904in}}{\pgfqpoint{3.221564in}{2.416080in}}%
\pgfpathcurveto{\pgfqpoint{3.227388in}{2.410256in}}{\pgfqpoint{3.235288in}{2.406983in}}{\pgfqpoint{3.243524in}{2.406983in}}%
\pgfpathclose%
\pgfusepath{stroke,fill}%
\end{pgfscope}%
\begin{pgfscope}%
\pgfpathrectangle{\pgfqpoint{0.100000in}{0.212622in}}{\pgfqpoint{3.696000in}{3.696000in}}%
\pgfusepath{clip}%
\pgfsetbuttcap%
\pgfsetroundjoin%
\definecolor{currentfill}{rgb}{0.121569,0.466667,0.705882}%
\pgfsetfillcolor{currentfill}%
\pgfsetfillopacity{0.725796}%
\pgfsetlinewidth{1.003750pt}%
\definecolor{currentstroke}{rgb}{0.121569,0.466667,0.705882}%
\pgfsetstrokecolor{currentstroke}%
\pgfsetstrokeopacity{0.725796}%
\pgfsetdash{}{0pt}%
\pgfpathmoveto{\pgfqpoint{3.241377in}{2.399586in}}%
\pgfpathcurveto{\pgfqpoint{3.249613in}{2.399586in}}{\pgfqpoint{3.257513in}{2.402858in}}{\pgfqpoint{3.263337in}{2.408682in}}%
\pgfpathcurveto{\pgfqpoint{3.269161in}{2.414506in}}{\pgfqpoint{3.272434in}{2.422406in}}{\pgfqpoint{3.272434in}{2.430642in}}%
\pgfpathcurveto{\pgfqpoint{3.272434in}{2.438878in}}{\pgfqpoint{3.269161in}{2.446778in}}{\pgfqpoint{3.263337in}{2.452602in}}%
\pgfpathcurveto{\pgfqpoint{3.257513in}{2.458426in}}{\pgfqpoint{3.249613in}{2.461699in}}{\pgfqpoint{3.241377in}{2.461699in}}%
\pgfpathcurveto{\pgfqpoint{3.233141in}{2.461699in}}{\pgfqpoint{3.225241in}{2.458426in}}{\pgfqpoint{3.219417in}{2.452602in}}%
\pgfpathcurveto{\pgfqpoint{3.213593in}{2.446778in}}{\pgfqpoint{3.210321in}{2.438878in}}{\pgfqpoint{3.210321in}{2.430642in}}%
\pgfpathcurveto{\pgfqpoint{3.210321in}{2.422406in}}{\pgfqpoint{3.213593in}{2.414506in}}{\pgfqpoint{3.219417in}{2.408682in}}%
\pgfpathcurveto{\pgfqpoint{3.225241in}{2.402858in}}{\pgfqpoint{3.233141in}{2.399586in}}{\pgfqpoint{3.241377in}{2.399586in}}%
\pgfpathclose%
\pgfusepath{stroke,fill}%
\end{pgfscope}%
\begin{pgfscope}%
\pgfpathrectangle{\pgfqpoint{0.100000in}{0.212622in}}{\pgfqpoint{3.696000in}{3.696000in}}%
\pgfusepath{clip}%
\pgfsetbuttcap%
\pgfsetroundjoin%
\definecolor{currentfill}{rgb}{0.121569,0.466667,0.705882}%
\pgfsetfillcolor{currentfill}%
\pgfsetfillopacity{0.726132}%
\pgfsetlinewidth{1.003750pt}%
\definecolor{currentstroke}{rgb}{0.121569,0.466667,0.705882}%
\pgfsetstrokecolor{currentstroke}%
\pgfsetstrokeopacity{0.726132}%
\pgfsetdash{}{0pt}%
\pgfpathmoveto{\pgfqpoint{1.236731in}{2.032053in}}%
\pgfpathcurveto{\pgfqpoint{1.244967in}{2.032053in}}{\pgfqpoint{1.252867in}{2.035325in}}{\pgfqpoint{1.258691in}{2.041149in}}%
\pgfpathcurveto{\pgfqpoint{1.264515in}{2.046973in}}{\pgfqpoint{1.267787in}{2.054873in}}{\pgfqpoint{1.267787in}{2.063109in}}%
\pgfpathcurveto{\pgfqpoint{1.267787in}{2.071346in}}{\pgfqpoint{1.264515in}{2.079246in}}{\pgfqpoint{1.258691in}{2.085070in}}%
\pgfpathcurveto{\pgfqpoint{1.252867in}{2.090894in}}{\pgfqpoint{1.244967in}{2.094166in}}{\pgfqpoint{1.236731in}{2.094166in}}%
\pgfpathcurveto{\pgfqpoint{1.228495in}{2.094166in}}{\pgfqpoint{1.220594in}{2.090894in}}{\pgfqpoint{1.214771in}{2.085070in}}%
\pgfpathcurveto{\pgfqpoint{1.208947in}{2.079246in}}{\pgfqpoint{1.205674in}{2.071346in}}{\pgfqpoint{1.205674in}{2.063109in}}%
\pgfpathcurveto{\pgfqpoint{1.205674in}{2.054873in}}{\pgfqpoint{1.208947in}{2.046973in}}{\pgfqpoint{1.214771in}{2.041149in}}%
\pgfpathcurveto{\pgfqpoint{1.220594in}{2.035325in}}{\pgfqpoint{1.228495in}{2.032053in}}{\pgfqpoint{1.236731in}{2.032053in}}%
\pgfpathclose%
\pgfusepath{stroke,fill}%
\end{pgfscope}%
\begin{pgfscope}%
\pgfpathrectangle{\pgfqpoint{0.100000in}{0.212622in}}{\pgfqpoint{3.696000in}{3.696000in}}%
\pgfusepath{clip}%
\pgfsetbuttcap%
\pgfsetroundjoin%
\definecolor{currentfill}{rgb}{0.121569,0.466667,0.705882}%
\pgfsetfillcolor{currentfill}%
\pgfsetfillopacity{0.726204}%
\pgfsetlinewidth{1.003750pt}%
\definecolor{currentstroke}{rgb}{0.121569,0.466667,0.705882}%
\pgfsetstrokecolor{currentstroke}%
\pgfsetstrokeopacity{0.726204}%
\pgfsetdash{}{0pt}%
\pgfpathmoveto{\pgfqpoint{2.329274in}{2.983572in}}%
\pgfpathcurveto{\pgfqpoint{2.337510in}{2.983572in}}{\pgfqpoint{2.345410in}{2.986845in}}{\pgfqpoint{2.351234in}{2.992669in}}%
\pgfpathcurveto{\pgfqpoint{2.357058in}{2.998493in}}{\pgfqpoint{2.360330in}{3.006393in}}{\pgfqpoint{2.360330in}{3.014629in}}%
\pgfpathcurveto{\pgfqpoint{2.360330in}{3.022865in}}{\pgfqpoint{2.357058in}{3.030765in}}{\pgfqpoint{2.351234in}{3.036589in}}%
\pgfpathcurveto{\pgfqpoint{2.345410in}{3.042413in}}{\pgfqpoint{2.337510in}{3.045685in}}{\pgfqpoint{2.329274in}{3.045685in}}%
\pgfpathcurveto{\pgfqpoint{2.321037in}{3.045685in}}{\pgfqpoint{2.313137in}{3.042413in}}{\pgfqpoint{2.307313in}{3.036589in}}%
\pgfpathcurveto{\pgfqpoint{2.301489in}{3.030765in}}{\pgfqpoint{2.298217in}{3.022865in}}{\pgfqpoint{2.298217in}{3.014629in}}%
\pgfpathcurveto{\pgfqpoint{2.298217in}{3.006393in}}{\pgfqpoint{2.301489in}{2.998493in}}{\pgfqpoint{2.307313in}{2.992669in}}%
\pgfpathcurveto{\pgfqpoint{2.313137in}{2.986845in}}{\pgfqpoint{2.321037in}{2.983572in}}{\pgfqpoint{2.329274in}{2.983572in}}%
\pgfpathclose%
\pgfusepath{stroke,fill}%
\end{pgfscope}%
\begin{pgfscope}%
\pgfpathrectangle{\pgfqpoint{0.100000in}{0.212622in}}{\pgfqpoint{3.696000in}{3.696000in}}%
\pgfusepath{clip}%
\pgfsetbuttcap%
\pgfsetroundjoin%
\definecolor{currentfill}{rgb}{0.121569,0.466667,0.705882}%
\pgfsetfillcolor{currentfill}%
\pgfsetfillopacity{0.726377}%
\pgfsetlinewidth{1.003750pt}%
\definecolor{currentstroke}{rgb}{0.121569,0.466667,0.705882}%
\pgfsetstrokecolor{currentstroke}%
\pgfsetstrokeopacity{0.726377}%
\pgfsetdash{}{0pt}%
\pgfpathmoveto{\pgfqpoint{3.238491in}{2.395469in}}%
\pgfpathcurveto{\pgfqpoint{3.246727in}{2.395469in}}{\pgfqpoint{3.254627in}{2.398741in}}{\pgfqpoint{3.260451in}{2.404565in}}%
\pgfpathcurveto{\pgfqpoint{3.266275in}{2.410389in}}{\pgfqpoint{3.269547in}{2.418289in}}{\pgfqpoint{3.269547in}{2.426525in}}%
\pgfpathcurveto{\pgfqpoint{3.269547in}{2.434762in}}{\pgfqpoint{3.266275in}{2.442662in}}{\pgfqpoint{3.260451in}{2.448486in}}%
\pgfpathcurveto{\pgfqpoint{3.254627in}{2.454309in}}{\pgfqpoint{3.246727in}{2.457582in}}{\pgfqpoint{3.238491in}{2.457582in}}%
\pgfpathcurveto{\pgfqpoint{3.230254in}{2.457582in}}{\pgfqpoint{3.222354in}{2.454309in}}{\pgfqpoint{3.216530in}{2.448486in}}%
\pgfpathcurveto{\pgfqpoint{3.210706in}{2.442662in}}{\pgfqpoint{3.207434in}{2.434762in}}{\pgfqpoint{3.207434in}{2.426525in}}%
\pgfpathcurveto{\pgfqpoint{3.207434in}{2.418289in}}{\pgfqpoint{3.210706in}{2.410389in}}{\pgfqpoint{3.216530in}{2.404565in}}%
\pgfpathcurveto{\pgfqpoint{3.222354in}{2.398741in}}{\pgfqpoint{3.230254in}{2.395469in}}{\pgfqpoint{3.238491in}{2.395469in}}%
\pgfpathclose%
\pgfusepath{stroke,fill}%
\end{pgfscope}%
\begin{pgfscope}%
\pgfpathrectangle{\pgfqpoint{0.100000in}{0.212622in}}{\pgfqpoint{3.696000in}{3.696000in}}%
\pgfusepath{clip}%
\pgfsetbuttcap%
\pgfsetroundjoin%
\definecolor{currentfill}{rgb}{0.121569,0.466667,0.705882}%
\pgfsetfillcolor{currentfill}%
\pgfsetfillopacity{0.727491}%
\pgfsetlinewidth{1.003750pt}%
\definecolor{currentstroke}{rgb}{0.121569,0.466667,0.705882}%
\pgfsetstrokecolor{currentstroke}%
\pgfsetstrokeopacity{0.727491}%
\pgfsetdash{}{0pt}%
\pgfpathmoveto{\pgfqpoint{3.233258in}{2.388225in}}%
\pgfpathcurveto{\pgfqpoint{3.241495in}{2.388225in}}{\pgfqpoint{3.249395in}{2.391497in}}{\pgfqpoint{3.255219in}{2.397321in}}%
\pgfpathcurveto{\pgfqpoint{3.261042in}{2.403145in}}{\pgfqpoint{3.264315in}{2.411045in}}{\pgfqpoint{3.264315in}{2.419281in}}%
\pgfpathcurveto{\pgfqpoint{3.264315in}{2.427518in}}{\pgfqpoint{3.261042in}{2.435418in}}{\pgfqpoint{3.255219in}{2.441242in}}%
\pgfpathcurveto{\pgfqpoint{3.249395in}{2.447066in}}{\pgfqpoint{3.241495in}{2.450338in}}{\pgfqpoint{3.233258in}{2.450338in}}%
\pgfpathcurveto{\pgfqpoint{3.225022in}{2.450338in}}{\pgfqpoint{3.217122in}{2.447066in}}{\pgfqpoint{3.211298in}{2.441242in}}%
\pgfpathcurveto{\pgfqpoint{3.205474in}{2.435418in}}{\pgfqpoint{3.202202in}{2.427518in}}{\pgfqpoint{3.202202in}{2.419281in}}%
\pgfpathcurveto{\pgfqpoint{3.202202in}{2.411045in}}{\pgfqpoint{3.205474in}{2.403145in}}{\pgfqpoint{3.211298in}{2.397321in}}%
\pgfpathcurveto{\pgfqpoint{3.217122in}{2.391497in}}{\pgfqpoint{3.225022in}{2.388225in}}{\pgfqpoint{3.233258in}{2.388225in}}%
\pgfpathclose%
\pgfusepath{stroke,fill}%
\end{pgfscope}%
\begin{pgfscope}%
\pgfpathrectangle{\pgfqpoint{0.100000in}{0.212622in}}{\pgfqpoint{3.696000in}{3.696000in}}%
\pgfusepath{clip}%
\pgfsetbuttcap%
\pgfsetroundjoin%
\definecolor{currentfill}{rgb}{0.121569,0.466667,0.705882}%
\pgfsetfillcolor{currentfill}%
\pgfsetfillopacity{0.727542}%
\pgfsetlinewidth{1.003750pt}%
\definecolor{currentstroke}{rgb}{0.121569,0.466667,0.705882}%
\pgfsetstrokecolor{currentstroke}%
\pgfsetstrokeopacity{0.727542}%
\pgfsetdash{}{0pt}%
\pgfpathmoveto{\pgfqpoint{2.336324in}{2.983368in}}%
\pgfpathcurveto{\pgfqpoint{2.344560in}{2.983368in}}{\pgfqpoint{2.352460in}{2.986640in}}{\pgfqpoint{2.358284in}{2.992464in}}%
\pgfpathcurveto{\pgfqpoint{2.364108in}{2.998288in}}{\pgfqpoint{2.367381in}{3.006188in}}{\pgfqpoint{2.367381in}{3.014424in}}%
\pgfpathcurveto{\pgfqpoint{2.367381in}{3.022661in}}{\pgfqpoint{2.364108in}{3.030561in}}{\pgfqpoint{2.358284in}{3.036385in}}%
\pgfpathcurveto{\pgfqpoint{2.352460in}{3.042208in}}{\pgfqpoint{2.344560in}{3.045481in}}{\pgfqpoint{2.336324in}{3.045481in}}%
\pgfpathcurveto{\pgfqpoint{2.328088in}{3.045481in}}{\pgfqpoint{2.320188in}{3.042208in}}{\pgfqpoint{2.314364in}{3.036385in}}%
\pgfpathcurveto{\pgfqpoint{2.308540in}{3.030561in}}{\pgfqpoint{2.305268in}{3.022661in}}{\pgfqpoint{2.305268in}{3.014424in}}%
\pgfpathcurveto{\pgfqpoint{2.305268in}{3.006188in}}{\pgfqpoint{2.308540in}{2.998288in}}{\pgfqpoint{2.314364in}{2.992464in}}%
\pgfpathcurveto{\pgfqpoint{2.320188in}{2.986640in}}{\pgfqpoint{2.328088in}{2.983368in}}{\pgfqpoint{2.336324in}{2.983368in}}%
\pgfpathclose%
\pgfusepath{stroke,fill}%
\end{pgfscope}%
\begin{pgfscope}%
\pgfpathrectangle{\pgfqpoint{0.100000in}{0.212622in}}{\pgfqpoint{3.696000in}{3.696000in}}%
\pgfusepath{clip}%
\pgfsetbuttcap%
\pgfsetroundjoin%
\definecolor{currentfill}{rgb}{0.121569,0.466667,0.705882}%
\pgfsetfillcolor{currentfill}%
\pgfsetfillopacity{0.728459}%
\pgfsetlinewidth{1.003750pt}%
\definecolor{currentstroke}{rgb}{0.121569,0.466667,0.705882}%
\pgfsetstrokecolor{currentstroke}%
\pgfsetstrokeopacity{0.728459}%
\pgfsetdash{}{0pt}%
\pgfpathmoveto{\pgfqpoint{1.227120in}{2.011548in}}%
\pgfpathcurveto{\pgfqpoint{1.235356in}{2.011548in}}{\pgfqpoint{1.243256in}{2.014820in}}{\pgfqpoint{1.249080in}{2.020644in}}%
\pgfpathcurveto{\pgfqpoint{1.254904in}{2.026468in}}{\pgfqpoint{1.258176in}{2.034368in}}{\pgfqpoint{1.258176in}{2.042604in}}%
\pgfpathcurveto{\pgfqpoint{1.258176in}{2.050840in}}{\pgfqpoint{1.254904in}{2.058740in}}{\pgfqpoint{1.249080in}{2.064564in}}%
\pgfpathcurveto{\pgfqpoint{1.243256in}{2.070388in}}{\pgfqpoint{1.235356in}{2.073661in}}{\pgfqpoint{1.227120in}{2.073661in}}%
\pgfpathcurveto{\pgfqpoint{1.218883in}{2.073661in}}{\pgfqpoint{1.210983in}{2.070388in}}{\pgfqpoint{1.205159in}{2.064564in}}%
\pgfpathcurveto{\pgfqpoint{1.199335in}{2.058740in}}{\pgfqpoint{1.196063in}{2.050840in}}{\pgfqpoint{1.196063in}{2.042604in}}%
\pgfpathcurveto{\pgfqpoint{1.196063in}{2.034368in}}{\pgfqpoint{1.199335in}{2.026468in}}{\pgfqpoint{1.205159in}{2.020644in}}%
\pgfpathcurveto{\pgfqpoint{1.210983in}{2.014820in}}{\pgfqpoint{1.218883in}{2.011548in}}{\pgfqpoint{1.227120in}{2.011548in}}%
\pgfpathclose%
\pgfusepath{stroke,fill}%
\end{pgfscope}%
\begin{pgfscope}%
\pgfpathrectangle{\pgfqpoint{0.100000in}{0.212622in}}{\pgfqpoint{3.696000in}{3.696000in}}%
\pgfusepath{clip}%
\pgfsetbuttcap%
\pgfsetroundjoin%
\definecolor{currentfill}{rgb}{0.121569,0.466667,0.705882}%
\pgfsetfillcolor{currentfill}%
\pgfsetfillopacity{0.728582}%
\pgfsetlinewidth{1.003750pt}%
\definecolor{currentstroke}{rgb}{0.121569,0.466667,0.705882}%
\pgfsetstrokecolor{currentstroke}%
\pgfsetstrokeopacity{0.728582}%
\pgfsetdash{}{0pt}%
\pgfpathmoveto{\pgfqpoint{3.229485in}{2.381185in}}%
\pgfpathcurveto{\pgfqpoint{3.237722in}{2.381185in}}{\pgfqpoint{3.245622in}{2.384458in}}{\pgfqpoint{3.251446in}{2.390282in}}%
\pgfpathcurveto{\pgfqpoint{3.257270in}{2.396106in}}{\pgfqpoint{3.260542in}{2.404006in}}{\pgfqpoint{3.260542in}{2.412242in}}%
\pgfpathcurveto{\pgfqpoint{3.260542in}{2.420478in}}{\pgfqpoint{3.257270in}{2.428378in}}{\pgfqpoint{3.251446in}{2.434202in}}%
\pgfpathcurveto{\pgfqpoint{3.245622in}{2.440026in}}{\pgfqpoint{3.237722in}{2.443298in}}{\pgfqpoint{3.229485in}{2.443298in}}%
\pgfpathcurveto{\pgfqpoint{3.221249in}{2.443298in}}{\pgfqpoint{3.213349in}{2.440026in}}{\pgfqpoint{3.207525in}{2.434202in}}%
\pgfpathcurveto{\pgfqpoint{3.201701in}{2.428378in}}{\pgfqpoint{3.198429in}{2.420478in}}{\pgfqpoint{3.198429in}{2.412242in}}%
\pgfpathcurveto{\pgfqpoint{3.198429in}{2.404006in}}{\pgfqpoint{3.201701in}{2.396106in}}{\pgfqpoint{3.207525in}{2.390282in}}%
\pgfpathcurveto{\pgfqpoint{3.213349in}{2.384458in}}{\pgfqpoint{3.221249in}{2.381185in}}{\pgfqpoint{3.229485in}{2.381185in}}%
\pgfpathclose%
\pgfusepath{stroke,fill}%
\end{pgfscope}%
\begin{pgfscope}%
\pgfpathrectangle{\pgfqpoint{0.100000in}{0.212622in}}{\pgfqpoint{3.696000in}{3.696000in}}%
\pgfusepath{clip}%
\pgfsetbuttcap%
\pgfsetroundjoin%
\definecolor{currentfill}{rgb}{0.121569,0.466667,0.705882}%
\pgfsetfillcolor{currentfill}%
\pgfsetfillopacity{0.728745}%
\pgfsetlinewidth{1.003750pt}%
\definecolor{currentstroke}{rgb}{0.121569,0.466667,0.705882}%
\pgfsetstrokecolor{currentstroke}%
\pgfsetstrokeopacity{0.728745}%
\pgfsetdash{}{0pt}%
\pgfpathmoveto{\pgfqpoint{2.342088in}{2.982661in}}%
\pgfpathcurveto{\pgfqpoint{2.350324in}{2.982661in}}{\pgfqpoint{2.358224in}{2.985933in}}{\pgfqpoint{2.364048in}{2.991757in}}%
\pgfpathcurveto{\pgfqpoint{2.369872in}{2.997581in}}{\pgfqpoint{2.373144in}{3.005481in}}{\pgfqpoint{2.373144in}{3.013717in}}%
\pgfpathcurveto{\pgfqpoint{2.373144in}{3.021954in}}{\pgfqpoint{2.369872in}{3.029854in}}{\pgfqpoint{2.364048in}{3.035678in}}%
\pgfpathcurveto{\pgfqpoint{2.358224in}{3.041502in}}{\pgfqpoint{2.350324in}{3.044774in}}{\pgfqpoint{2.342088in}{3.044774in}}%
\pgfpathcurveto{\pgfqpoint{2.333851in}{3.044774in}}{\pgfqpoint{2.325951in}{3.041502in}}{\pgfqpoint{2.320127in}{3.035678in}}%
\pgfpathcurveto{\pgfqpoint{2.314304in}{3.029854in}}{\pgfqpoint{2.311031in}{3.021954in}}{\pgfqpoint{2.311031in}{3.013717in}}%
\pgfpathcurveto{\pgfqpoint{2.311031in}{3.005481in}}{\pgfqpoint{2.314304in}{2.997581in}}{\pgfqpoint{2.320127in}{2.991757in}}%
\pgfpathcurveto{\pgfqpoint{2.325951in}{2.985933in}}{\pgfqpoint{2.333851in}{2.982661in}}{\pgfqpoint{2.342088in}{2.982661in}}%
\pgfpathclose%
\pgfusepath{stroke,fill}%
\end{pgfscope}%
\begin{pgfscope}%
\pgfpathrectangle{\pgfqpoint{0.100000in}{0.212622in}}{\pgfqpoint{3.696000in}{3.696000in}}%
\pgfusepath{clip}%
\pgfsetbuttcap%
\pgfsetroundjoin%
\definecolor{currentfill}{rgb}{0.121569,0.466667,0.705882}%
\pgfsetfillcolor{currentfill}%
\pgfsetfillopacity{0.729594}%
\pgfsetlinewidth{1.003750pt}%
\definecolor{currentstroke}{rgb}{0.121569,0.466667,0.705882}%
\pgfsetstrokecolor{currentstroke}%
\pgfsetstrokeopacity{0.729594}%
\pgfsetdash{}{0pt}%
\pgfpathmoveto{\pgfqpoint{2.347757in}{2.981255in}}%
\pgfpathcurveto{\pgfqpoint{2.355993in}{2.981255in}}{\pgfqpoint{2.363893in}{2.984527in}}{\pgfqpoint{2.369717in}{2.990351in}}%
\pgfpathcurveto{\pgfqpoint{2.375541in}{2.996175in}}{\pgfqpoint{2.378813in}{3.004075in}}{\pgfqpoint{2.378813in}{3.012312in}}%
\pgfpathcurveto{\pgfqpoint{2.378813in}{3.020548in}}{\pgfqpoint{2.375541in}{3.028448in}}{\pgfqpoint{2.369717in}{3.034272in}}%
\pgfpathcurveto{\pgfqpoint{2.363893in}{3.040096in}}{\pgfqpoint{2.355993in}{3.043368in}}{\pgfqpoint{2.347757in}{3.043368in}}%
\pgfpathcurveto{\pgfqpoint{2.339520in}{3.043368in}}{\pgfqpoint{2.331620in}{3.040096in}}{\pgfqpoint{2.325796in}{3.034272in}}%
\pgfpathcurveto{\pgfqpoint{2.319972in}{3.028448in}}{\pgfqpoint{2.316700in}{3.020548in}}{\pgfqpoint{2.316700in}{3.012312in}}%
\pgfpathcurveto{\pgfqpoint{2.316700in}{3.004075in}}{\pgfqpoint{2.319972in}{2.996175in}}{\pgfqpoint{2.325796in}{2.990351in}}%
\pgfpathcurveto{\pgfqpoint{2.331620in}{2.984527in}}{\pgfqpoint{2.339520in}{2.981255in}}{\pgfqpoint{2.347757in}{2.981255in}}%
\pgfpathclose%
\pgfusepath{stroke,fill}%
\end{pgfscope}%
\begin{pgfscope}%
\pgfpathrectangle{\pgfqpoint{0.100000in}{0.212622in}}{\pgfqpoint{3.696000in}{3.696000in}}%
\pgfusepath{clip}%
\pgfsetbuttcap%
\pgfsetroundjoin%
\definecolor{currentfill}{rgb}{0.121569,0.466667,0.705882}%
\pgfsetfillcolor{currentfill}%
\pgfsetfillopacity{0.729599}%
\pgfsetlinewidth{1.003750pt}%
\definecolor{currentstroke}{rgb}{0.121569,0.466667,0.705882}%
\pgfsetstrokecolor{currentstroke}%
\pgfsetstrokeopacity{0.729599}%
\pgfsetdash{}{0pt}%
\pgfpathmoveto{\pgfqpoint{3.228416in}{2.374871in}}%
\pgfpathcurveto{\pgfqpoint{3.236652in}{2.374871in}}{\pgfqpoint{3.244552in}{2.378144in}}{\pgfqpoint{3.250376in}{2.383968in}}%
\pgfpathcurveto{\pgfqpoint{3.256200in}{2.389792in}}{\pgfqpoint{3.259472in}{2.397692in}}{\pgfqpoint{3.259472in}{2.405928in}}%
\pgfpathcurveto{\pgfqpoint{3.259472in}{2.414164in}}{\pgfqpoint{3.256200in}{2.422064in}}{\pgfqpoint{3.250376in}{2.427888in}}%
\pgfpathcurveto{\pgfqpoint{3.244552in}{2.433712in}}{\pgfqpoint{3.236652in}{2.436984in}}{\pgfqpoint{3.228416in}{2.436984in}}%
\pgfpathcurveto{\pgfqpoint{3.220179in}{2.436984in}}{\pgfqpoint{3.212279in}{2.433712in}}{\pgfqpoint{3.206455in}{2.427888in}}%
\pgfpathcurveto{\pgfqpoint{3.200631in}{2.422064in}}{\pgfqpoint{3.197359in}{2.414164in}}{\pgfqpoint{3.197359in}{2.405928in}}%
\pgfpathcurveto{\pgfqpoint{3.197359in}{2.397692in}}{\pgfqpoint{3.200631in}{2.389792in}}{\pgfqpoint{3.206455in}{2.383968in}}%
\pgfpathcurveto{\pgfqpoint{3.212279in}{2.378144in}}{\pgfqpoint{3.220179in}{2.374871in}}{\pgfqpoint{3.228416in}{2.374871in}}%
\pgfpathclose%
\pgfusepath{stroke,fill}%
\end{pgfscope}%
\begin{pgfscope}%
\pgfpathrectangle{\pgfqpoint{0.100000in}{0.212622in}}{\pgfqpoint{3.696000in}{3.696000in}}%
\pgfusepath{clip}%
\pgfsetbuttcap%
\pgfsetroundjoin%
\definecolor{currentfill}{rgb}{0.121569,0.466667,0.705882}%
\pgfsetfillcolor{currentfill}%
\pgfsetfillopacity{0.730548}%
\pgfsetlinewidth{1.003750pt}%
\definecolor{currentstroke}{rgb}{0.121569,0.466667,0.705882}%
\pgfsetstrokecolor{currentstroke}%
\pgfsetstrokeopacity{0.730548}%
\pgfsetdash{}{0pt}%
\pgfpathmoveto{\pgfqpoint{2.352017in}{2.980880in}}%
\pgfpathcurveto{\pgfqpoint{2.360253in}{2.980880in}}{\pgfqpoint{2.368153in}{2.984152in}}{\pgfqpoint{2.373977in}{2.989976in}}%
\pgfpathcurveto{\pgfqpoint{2.379801in}{2.995800in}}{\pgfqpoint{2.383073in}{3.003700in}}{\pgfqpoint{2.383073in}{3.011936in}}%
\pgfpathcurveto{\pgfqpoint{2.383073in}{3.020173in}}{\pgfqpoint{2.379801in}{3.028073in}}{\pgfqpoint{2.373977in}{3.033897in}}%
\pgfpathcurveto{\pgfqpoint{2.368153in}{3.039721in}}{\pgfqpoint{2.360253in}{3.042993in}}{\pgfqpoint{2.352017in}{3.042993in}}%
\pgfpathcurveto{\pgfqpoint{2.343781in}{3.042993in}}{\pgfqpoint{2.335881in}{3.039721in}}{\pgfqpoint{2.330057in}{3.033897in}}%
\pgfpathcurveto{\pgfqpoint{2.324233in}{3.028073in}}{\pgfqpoint{2.320960in}{3.020173in}}{\pgfqpoint{2.320960in}{3.011936in}}%
\pgfpathcurveto{\pgfqpoint{2.320960in}{3.003700in}}{\pgfqpoint{2.324233in}{2.995800in}}{\pgfqpoint{2.330057in}{2.989976in}}%
\pgfpathcurveto{\pgfqpoint{2.335881in}{2.984152in}}{\pgfqpoint{2.343781in}{2.980880in}}{\pgfqpoint{2.352017in}{2.980880in}}%
\pgfpathclose%
\pgfusepath{stroke,fill}%
\end{pgfscope}%
\begin{pgfscope}%
\pgfpathrectangle{\pgfqpoint{0.100000in}{0.212622in}}{\pgfqpoint{3.696000in}{3.696000in}}%
\pgfusepath{clip}%
\pgfsetbuttcap%
\pgfsetroundjoin%
\definecolor{currentfill}{rgb}{0.121569,0.466667,0.705882}%
\pgfsetfillcolor{currentfill}%
\pgfsetfillopacity{0.730698}%
\pgfsetlinewidth{1.003750pt}%
\definecolor{currentstroke}{rgb}{0.121569,0.466667,0.705882}%
\pgfsetstrokecolor{currentstroke}%
\pgfsetstrokeopacity{0.730698}%
\pgfsetdash{}{0pt}%
\pgfpathmoveto{\pgfqpoint{1.215313in}{1.992046in}}%
\pgfpathcurveto{\pgfqpoint{1.223549in}{1.992046in}}{\pgfqpoint{1.231449in}{1.995319in}}{\pgfqpoint{1.237273in}{2.001143in}}%
\pgfpathcurveto{\pgfqpoint{1.243097in}{2.006966in}}{\pgfqpoint{1.246369in}{2.014866in}}{\pgfqpoint{1.246369in}{2.023103in}}%
\pgfpathcurveto{\pgfqpoint{1.246369in}{2.031339in}}{\pgfqpoint{1.243097in}{2.039239in}}{\pgfqpoint{1.237273in}{2.045063in}}%
\pgfpathcurveto{\pgfqpoint{1.231449in}{2.050887in}}{\pgfqpoint{1.223549in}{2.054159in}}{\pgfqpoint{1.215313in}{2.054159in}}%
\pgfpathcurveto{\pgfqpoint{1.207076in}{2.054159in}}{\pgfqpoint{1.199176in}{2.050887in}}{\pgfqpoint{1.193352in}{2.045063in}}%
\pgfpathcurveto{\pgfqpoint{1.187528in}{2.039239in}}{\pgfqpoint{1.184256in}{2.031339in}}{\pgfqpoint{1.184256in}{2.023103in}}%
\pgfpathcurveto{\pgfqpoint{1.184256in}{2.014866in}}{\pgfqpoint{1.187528in}{2.006966in}}{\pgfqpoint{1.193352in}{2.001143in}}%
\pgfpathcurveto{\pgfqpoint{1.199176in}{1.995319in}}{\pgfqpoint{1.207076in}{1.992046in}}{\pgfqpoint{1.215313in}{1.992046in}}%
\pgfpathclose%
\pgfusepath{stroke,fill}%
\end{pgfscope}%
\begin{pgfscope}%
\pgfpathrectangle{\pgfqpoint{0.100000in}{0.212622in}}{\pgfqpoint{3.696000in}{3.696000in}}%
\pgfusepath{clip}%
\pgfsetbuttcap%
\pgfsetroundjoin%
\definecolor{currentfill}{rgb}{0.121569,0.466667,0.705882}%
\pgfsetfillcolor{currentfill}%
\pgfsetfillopacity{0.731151}%
\pgfsetlinewidth{1.003750pt}%
\definecolor{currentstroke}{rgb}{0.121569,0.466667,0.705882}%
\pgfsetstrokecolor{currentstroke}%
\pgfsetstrokeopacity{0.731151}%
\pgfsetdash{}{0pt}%
\pgfpathmoveto{\pgfqpoint{3.227200in}{2.362149in}}%
\pgfpathcurveto{\pgfqpoint{3.235436in}{2.362149in}}{\pgfqpoint{3.243336in}{2.365421in}}{\pgfqpoint{3.249160in}{2.371245in}}%
\pgfpathcurveto{\pgfqpoint{3.254984in}{2.377069in}}{\pgfqpoint{3.258256in}{2.384969in}}{\pgfqpoint{3.258256in}{2.393205in}}%
\pgfpathcurveto{\pgfqpoint{3.258256in}{2.401442in}}{\pgfqpoint{3.254984in}{2.409342in}}{\pgfqpoint{3.249160in}{2.415166in}}%
\pgfpathcurveto{\pgfqpoint{3.243336in}{2.420990in}}{\pgfqpoint{3.235436in}{2.424262in}}{\pgfqpoint{3.227200in}{2.424262in}}%
\pgfpathcurveto{\pgfqpoint{3.218963in}{2.424262in}}{\pgfqpoint{3.211063in}{2.420990in}}{\pgfqpoint{3.205239in}{2.415166in}}%
\pgfpathcurveto{\pgfqpoint{3.199415in}{2.409342in}}{\pgfqpoint{3.196143in}{2.401442in}}{\pgfqpoint{3.196143in}{2.393205in}}%
\pgfpathcurveto{\pgfqpoint{3.196143in}{2.384969in}}{\pgfqpoint{3.199415in}{2.377069in}}{\pgfqpoint{3.205239in}{2.371245in}}%
\pgfpathcurveto{\pgfqpoint{3.211063in}{2.365421in}}{\pgfqpoint{3.218963in}{2.362149in}}{\pgfqpoint{3.227200in}{2.362149in}}%
\pgfpathclose%
\pgfusepath{stroke,fill}%
\end{pgfscope}%
\begin{pgfscope}%
\pgfpathrectangle{\pgfqpoint{0.100000in}{0.212622in}}{\pgfqpoint{3.696000in}{3.696000in}}%
\pgfusepath{clip}%
\pgfsetbuttcap%
\pgfsetroundjoin%
\definecolor{currentfill}{rgb}{0.121569,0.466667,0.705882}%
\pgfsetfillcolor{currentfill}%
\pgfsetfillopacity{0.731277}%
\pgfsetlinewidth{1.003750pt}%
\definecolor{currentstroke}{rgb}{0.121569,0.466667,0.705882}%
\pgfsetstrokecolor{currentstroke}%
\pgfsetstrokeopacity{0.731277}%
\pgfsetdash{}{0pt}%
\pgfpathmoveto{\pgfqpoint{2.355720in}{2.980102in}}%
\pgfpathcurveto{\pgfqpoint{2.363956in}{2.980102in}}{\pgfqpoint{2.371856in}{2.983374in}}{\pgfqpoint{2.377680in}{2.989198in}}%
\pgfpathcurveto{\pgfqpoint{2.383504in}{2.995022in}}{\pgfqpoint{2.386777in}{3.002922in}}{\pgfqpoint{2.386777in}{3.011158in}}%
\pgfpathcurveto{\pgfqpoint{2.386777in}{3.019395in}}{\pgfqpoint{2.383504in}{3.027295in}}{\pgfqpoint{2.377680in}{3.033119in}}%
\pgfpathcurveto{\pgfqpoint{2.371856in}{3.038942in}}{\pgfqpoint{2.363956in}{3.042215in}}{\pgfqpoint{2.355720in}{3.042215in}}%
\pgfpathcurveto{\pgfqpoint{2.347484in}{3.042215in}}{\pgfqpoint{2.339584in}{3.038942in}}{\pgfqpoint{2.333760in}{3.033119in}}%
\pgfpathcurveto{\pgfqpoint{2.327936in}{3.027295in}}{\pgfqpoint{2.324664in}{3.019395in}}{\pgfqpoint{2.324664in}{3.011158in}}%
\pgfpathcurveto{\pgfqpoint{2.324664in}{3.002922in}}{\pgfqpoint{2.327936in}{2.995022in}}{\pgfqpoint{2.333760in}{2.989198in}}%
\pgfpathcurveto{\pgfqpoint{2.339584in}{2.983374in}}{\pgfqpoint{2.347484in}{2.980102in}}{\pgfqpoint{2.355720in}{2.980102in}}%
\pgfpathclose%
\pgfusepath{stroke,fill}%
\end{pgfscope}%
\begin{pgfscope}%
\pgfpathrectangle{\pgfqpoint{0.100000in}{0.212622in}}{\pgfqpoint{3.696000in}{3.696000in}}%
\pgfusepath{clip}%
\pgfsetbuttcap%
\pgfsetroundjoin%
\definecolor{currentfill}{rgb}{0.121569,0.466667,0.705882}%
\pgfsetfillcolor{currentfill}%
\pgfsetfillopacity{0.731785}%
\pgfsetlinewidth{1.003750pt}%
\definecolor{currentstroke}{rgb}{0.121569,0.466667,0.705882}%
\pgfsetstrokecolor{currentstroke}%
\pgfsetstrokeopacity{0.731785}%
\pgfsetdash{}{0pt}%
\pgfpathmoveto{\pgfqpoint{2.358765in}{2.979702in}}%
\pgfpathcurveto{\pgfqpoint{2.367002in}{2.979702in}}{\pgfqpoint{2.374902in}{2.982974in}}{\pgfqpoint{2.380726in}{2.988798in}}%
\pgfpathcurveto{\pgfqpoint{2.386549in}{2.994622in}}{\pgfqpoint{2.389822in}{3.002522in}}{\pgfqpoint{2.389822in}{3.010758in}}%
\pgfpathcurveto{\pgfqpoint{2.389822in}{3.018994in}}{\pgfqpoint{2.386549in}{3.026894in}}{\pgfqpoint{2.380726in}{3.032718in}}%
\pgfpathcurveto{\pgfqpoint{2.374902in}{3.038542in}}{\pgfqpoint{2.367002in}{3.041815in}}{\pgfqpoint{2.358765in}{3.041815in}}%
\pgfpathcurveto{\pgfqpoint{2.350529in}{3.041815in}}{\pgfqpoint{2.342629in}{3.038542in}}{\pgfqpoint{2.336805in}{3.032718in}}%
\pgfpathcurveto{\pgfqpoint{2.330981in}{3.026894in}}{\pgfqpoint{2.327709in}{3.018994in}}{\pgfqpoint{2.327709in}{3.010758in}}%
\pgfpathcurveto{\pgfqpoint{2.327709in}{3.002522in}}{\pgfqpoint{2.330981in}{2.994622in}}{\pgfqpoint{2.336805in}{2.988798in}}%
\pgfpathcurveto{\pgfqpoint{2.342629in}{2.982974in}}{\pgfqpoint{2.350529in}{2.979702in}}{\pgfqpoint{2.358765in}{2.979702in}}%
\pgfpathclose%
\pgfusepath{stroke,fill}%
\end{pgfscope}%
\begin{pgfscope}%
\pgfpathrectangle{\pgfqpoint{0.100000in}{0.212622in}}{\pgfqpoint{3.696000in}{3.696000in}}%
\pgfusepath{clip}%
\pgfsetbuttcap%
\pgfsetroundjoin%
\definecolor{currentfill}{rgb}{0.121569,0.466667,0.705882}%
\pgfsetfillcolor{currentfill}%
\pgfsetfillopacity{0.732812}%
\pgfsetlinewidth{1.003750pt}%
\definecolor{currentstroke}{rgb}{0.121569,0.466667,0.705882}%
\pgfsetstrokecolor{currentstroke}%
\pgfsetstrokeopacity{0.732812}%
\pgfsetdash{}{0pt}%
\pgfpathmoveto{\pgfqpoint{3.223835in}{2.352075in}}%
\pgfpathcurveto{\pgfqpoint{3.232071in}{2.352075in}}{\pgfqpoint{3.239971in}{2.355348in}}{\pgfqpoint{3.245795in}{2.361172in}}%
\pgfpathcurveto{\pgfqpoint{3.251619in}{2.366996in}}{\pgfqpoint{3.254891in}{2.374896in}}{\pgfqpoint{3.254891in}{2.383132in}}%
\pgfpathcurveto{\pgfqpoint{3.254891in}{2.391368in}}{\pgfqpoint{3.251619in}{2.399268in}}{\pgfqpoint{3.245795in}{2.405092in}}%
\pgfpathcurveto{\pgfqpoint{3.239971in}{2.410916in}}{\pgfqpoint{3.232071in}{2.414188in}}{\pgfqpoint{3.223835in}{2.414188in}}%
\pgfpathcurveto{\pgfqpoint{3.215599in}{2.414188in}}{\pgfqpoint{3.207698in}{2.410916in}}{\pgfqpoint{3.201875in}{2.405092in}}%
\pgfpathcurveto{\pgfqpoint{3.196051in}{2.399268in}}{\pgfqpoint{3.192778in}{2.391368in}}{\pgfqpoint{3.192778in}{2.383132in}}%
\pgfpathcurveto{\pgfqpoint{3.192778in}{2.374896in}}{\pgfqpoint{3.196051in}{2.366996in}}{\pgfqpoint{3.201875in}{2.361172in}}%
\pgfpathcurveto{\pgfqpoint{3.207698in}{2.355348in}}{\pgfqpoint{3.215599in}{2.352075in}}{\pgfqpoint{3.223835in}{2.352075in}}%
\pgfpathclose%
\pgfusepath{stroke,fill}%
\end{pgfscope}%
\begin{pgfscope}%
\pgfpathrectangle{\pgfqpoint{0.100000in}{0.212622in}}{\pgfqpoint{3.696000in}{3.696000in}}%
\pgfusepath{clip}%
\pgfsetbuttcap%
\pgfsetroundjoin%
\definecolor{currentfill}{rgb}{0.121569,0.466667,0.705882}%
\pgfsetfillcolor{currentfill}%
\pgfsetfillopacity{0.732923}%
\pgfsetlinewidth{1.003750pt}%
\definecolor{currentstroke}{rgb}{0.121569,0.466667,0.705882}%
\pgfsetstrokecolor{currentstroke}%
\pgfsetstrokeopacity{0.732923}%
\pgfsetdash{}{0pt}%
\pgfpathmoveto{\pgfqpoint{2.363983in}{2.978767in}}%
\pgfpathcurveto{\pgfqpoint{2.372219in}{2.978767in}}{\pgfqpoint{2.380119in}{2.982039in}}{\pgfqpoint{2.385943in}{2.987863in}}%
\pgfpathcurveto{\pgfqpoint{2.391767in}{2.993687in}}{\pgfqpoint{2.395039in}{3.001587in}}{\pgfqpoint{2.395039in}{3.009824in}}%
\pgfpathcurveto{\pgfqpoint{2.395039in}{3.018060in}}{\pgfqpoint{2.391767in}{3.025960in}}{\pgfqpoint{2.385943in}{3.031784in}}%
\pgfpathcurveto{\pgfqpoint{2.380119in}{3.037608in}}{\pgfqpoint{2.372219in}{3.040880in}}{\pgfqpoint{2.363983in}{3.040880in}}%
\pgfpathcurveto{\pgfqpoint{2.355747in}{3.040880in}}{\pgfqpoint{2.347847in}{3.037608in}}{\pgfqpoint{2.342023in}{3.031784in}}%
\pgfpathcurveto{\pgfqpoint{2.336199in}{3.025960in}}{\pgfqpoint{2.332926in}{3.018060in}}{\pgfqpoint{2.332926in}{3.009824in}}%
\pgfpathcurveto{\pgfqpoint{2.332926in}{3.001587in}}{\pgfqpoint{2.336199in}{2.993687in}}{\pgfqpoint{2.342023in}{2.987863in}}%
\pgfpathcurveto{\pgfqpoint{2.347847in}{2.982039in}}{\pgfqpoint{2.355747in}{2.978767in}}{\pgfqpoint{2.363983in}{2.978767in}}%
\pgfpathclose%
\pgfusepath{stroke,fill}%
\end{pgfscope}%
\begin{pgfscope}%
\pgfpathrectangle{\pgfqpoint{0.100000in}{0.212622in}}{\pgfqpoint{3.696000in}{3.696000in}}%
\pgfusepath{clip}%
\pgfsetbuttcap%
\pgfsetroundjoin%
\definecolor{currentfill}{rgb}{0.121569,0.466667,0.705882}%
\pgfsetfillcolor{currentfill}%
\pgfsetfillopacity{0.733305}%
\pgfsetlinewidth{1.003750pt}%
\definecolor{currentstroke}{rgb}{0.121569,0.466667,0.705882}%
\pgfsetstrokecolor{currentstroke}%
\pgfsetstrokeopacity{0.733305}%
\pgfsetdash{}{0pt}%
\pgfpathmoveto{\pgfqpoint{1.204700in}{1.969116in}}%
\pgfpathcurveto{\pgfqpoint{1.212936in}{1.969116in}}{\pgfqpoint{1.220837in}{1.972388in}}{\pgfqpoint{1.226660in}{1.978212in}}%
\pgfpathcurveto{\pgfqpoint{1.232484in}{1.984036in}}{\pgfqpoint{1.235757in}{1.991936in}}{\pgfqpoint{1.235757in}{2.000172in}}%
\pgfpathcurveto{\pgfqpoint{1.235757in}{2.008408in}}{\pgfqpoint{1.232484in}{2.016308in}}{\pgfqpoint{1.226660in}{2.022132in}}%
\pgfpathcurveto{\pgfqpoint{1.220837in}{2.027956in}}{\pgfqpoint{1.212936in}{2.031229in}}{\pgfqpoint{1.204700in}{2.031229in}}%
\pgfpathcurveto{\pgfqpoint{1.196464in}{2.031229in}}{\pgfqpoint{1.188564in}{2.027956in}}{\pgfqpoint{1.182740in}{2.022132in}}%
\pgfpathcurveto{\pgfqpoint{1.176916in}{2.016308in}}{\pgfqpoint{1.173644in}{2.008408in}}{\pgfqpoint{1.173644in}{2.000172in}}%
\pgfpathcurveto{\pgfqpoint{1.173644in}{1.991936in}}{\pgfqpoint{1.176916in}{1.984036in}}{\pgfqpoint{1.182740in}{1.978212in}}%
\pgfpathcurveto{\pgfqpoint{1.188564in}{1.972388in}}{\pgfqpoint{1.196464in}{1.969116in}}{\pgfqpoint{1.204700in}{1.969116in}}%
\pgfpathclose%
\pgfusepath{stroke,fill}%
\end{pgfscope}%
\begin{pgfscope}%
\pgfpathrectangle{\pgfqpoint{0.100000in}{0.212622in}}{\pgfqpoint{3.696000in}{3.696000in}}%
\pgfusepath{clip}%
\pgfsetbuttcap%
\pgfsetroundjoin%
\definecolor{currentfill}{rgb}{0.121569,0.466667,0.705882}%
\pgfsetfillcolor{currentfill}%
\pgfsetfillopacity{0.733859}%
\pgfsetlinewidth{1.003750pt}%
\definecolor{currentstroke}{rgb}{0.121569,0.466667,0.705882}%
\pgfsetstrokecolor{currentstroke}%
\pgfsetstrokeopacity{0.733859}%
\pgfsetdash{}{0pt}%
\pgfpathmoveto{\pgfqpoint{3.219293in}{2.345810in}}%
\pgfpathcurveto{\pgfqpoint{3.227529in}{2.345810in}}{\pgfqpoint{3.235429in}{2.349082in}}{\pgfqpoint{3.241253in}{2.354906in}}%
\pgfpathcurveto{\pgfqpoint{3.247077in}{2.360730in}}{\pgfqpoint{3.250349in}{2.368630in}}{\pgfqpoint{3.250349in}{2.376867in}}%
\pgfpathcurveto{\pgfqpoint{3.250349in}{2.385103in}}{\pgfqpoint{3.247077in}{2.393003in}}{\pgfqpoint{3.241253in}{2.398827in}}%
\pgfpathcurveto{\pgfqpoint{3.235429in}{2.404651in}}{\pgfqpoint{3.227529in}{2.407923in}}{\pgfqpoint{3.219293in}{2.407923in}}%
\pgfpathcurveto{\pgfqpoint{3.211057in}{2.407923in}}{\pgfqpoint{3.203157in}{2.404651in}}{\pgfqpoint{3.197333in}{2.398827in}}%
\pgfpathcurveto{\pgfqpoint{3.191509in}{2.393003in}}{\pgfqpoint{3.188236in}{2.385103in}}{\pgfqpoint{3.188236in}{2.376867in}}%
\pgfpathcurveto{\pgfqpoint{3.188236in}{2.368630in}}{\pgfqpoint{3.191509in}{2.360730in}}{\pgfqpoint{3.197333in}{2.354906in}}%
\pgfpathcurveto{\pgfqpoint{3.203157in}{2.349082in}}{\pgfqpoint{3.211057in}{2.345810in}}{\pgfqpoint{3.219293in}{2.345810in}}%
\pgfpathclose%
\pgfusepath{stroke,fill}%
\end{pgfscope}%
\begin{pgfscope}%
\pgfpathrectangle{\pgfqpoint{0.100000in}{0.212622in}}{\pgfqpoint{3.696000in}{3.696000in}}%
\pgfusepath{clip}%
\pgfsetbuttcap%
\pgfsetroundjoin%
\definecolor{currentfill}{rgb}{0.121569,0.466667,0.705882}%
\pgfsetfillcolor{currentfill}%
\pgfsetfillopacity{0.734157}%
\pgfsetlinewidth{1.003750pt}%
\definecolor{currentstroke}{rgb}{0.121569,0.466667,0.705882}%
\pgfsetstrokecolor{currentstroke}%
\pgfsetstrokeopacity{0.734157}%
\pgfsetdash{}{0pt}%
\pgfpathmoveto{\pgfqpoint{2.368563in}{2.977258in}}%
\pgfpathcurveto{\pgfqpoint{2.376799in}{2.977258in}}{\pgfqpoint{2.384699in}{2.980531in}}{\pgfqpoint{2.390523in}{2.986355in}}%
\pgfpathcurveto{\pgfqpoint{2.396347in}{2.992178in}}{\pgfqpoint{2.399619in}{3.000079in}}{\pgfqpoint{2.399619in}{3.008315in}}%
\pgfpathcurveto{\pgfqpoint{2.399619in}{3.016551in}}{\pgfqpoint{2.396347in}{3.024451in}}{\pgfqpoint{2.390523in}{3.030275in}}%
\pgfpathcurveto{\pgfqpoint{2.384699in}{3.036099in}}{\pgfqpoint{2.376799in}{3.039371in}}{\pgfqpoint{2.368563in}{3.039371in}}%
\pgfpathcurveto{\pgfqpoint{2.360326in}{3.039371in}}{\pgfqpoint{2.352426in}{3.036099in}}{\pgfqpoint{2.346602in}{3.030275in}}%
\pgfpathcurveto{\pgfqpoint{2.340778in}{3.024451in}}{\pgfqpoint{2.337506in}{3.016551in}}{\pgfqpoint{2.337506in}{3.008315in}}%
\pgfpathcurveto{\pgfqpoint{2.337506in}{3.000079in}}{\pgfqpoint{2.340778in}{2.992178in}}{\pgfqpoint{2.346602in}{2.986355in}}%
\pgfpathcurveto{\pgfqpoint{2.352426in}{2.980531in}}{\pgfqpoint{2.360326in}{2.977258in}}{\pgfqpoint{2.368563in}{2.977258in}}%
\pgfpathclose%
\pgfusepath{stroke,fill}%
\end{pgfscope}%
\begin{pgfscope}%
\pgfpathrectangle{\pgfqpoint{0.100000in}{0.212622in}}{\pgfqpoint{3.696000in}{3.696000in}}%
\pgfusepath{clip}%
\pgfsetbuttcap%
\pgfsetroundjoin%
\definecolor{currentfill}{rgb}{0.121569,0.466667,0.705882}%
\pgfsetfillcolor{currentfill}%
\pgfsetfillopacity{0.734661}%
\pgfsetlinewidth{1.003750pt}%
\definecolor{currentstroke}{rgb}{0.121569,0.466667,0.705882}%
\pgfsetstrokecolor{currentstroke}%
\pgfsetstrokeopacity{0.734661}%
\pgfsetdash{}{0pt}%
\pgfpathmoveto{\pgfqpoint{1.200711in}{1.954675in}}%
\pgfpathcurveto{\pgfqpoint{1.208947in}{1.954675in}}{\pgfqpoint{1.216847in}{1.957947in}}{\pgfqpoint{1.222671in}{1.963771in}}%
\pgfpathcurveto{\pgfqpoint{1.228495in}{1.969595in}}{\pgfqpoint{1.231767in}{1.977495in}}{\pgfqpoint{1.231767in}{1.985731in}}%
\pgfpathcurveto{\pgfqpoint{1.231767in}{1.993967in}}{\pgfqpoint{1.228495in}{2.001867in}}{\pgfqpoint{1.222671in}{2.007691in}}%
\pgfpathcurveto{\pgfqpoint{1.216847in}{2.013515in}}{\pgfqpoint{1.208947in}{2.016788in}}{\pgfqpoint{1.200711in}{2.016788in}}%
\pgfpathcurveto{\pgfqpoint{1.192474in}{2.016788in}}{\pgfqpoint{1.184574in}{2.013515in}}{\pgfqpoint{1.178751in}{2.007691in}}%
\pgfpathcurveto{\pgfqpoint{1.172927in}{2.001867in}}{\pgfqpoint{1.169654in}{1.993967in}}{\pgfqpoint{1.169654in}{1.985731in}}%
\pgfpathcurveto{\pgfqpoint{1.169654in}{1.977495in}}{\pgfqpoint{1.172927in}{1.969595in}}{\pgfqpoint{1.178751in}{1.963771in}}%
\pgfpathcurveto{\pgfqpoint{1.184574in}{1.957947in}}{\pgfqpoint{1.192474in}{1.954675in}}{\pgfqpoint{1.200711in}{1.954675in}}%
\pgfpathclose%
\pgfusepath{stroke,fill}%
\end{pgfscope}%
\begin{pgfscope}%
\pgfpathrectangle{\pgfqpoint{0.100000in}{0.212622in}}{\pgfqpoint{3.696000in}{3.696000in}}%
\pgfusepath{clip}%
\pgfsetbuttcap%
\pgfsetroundjoin%
\definecolor{currentfill}{rgb}{0.121569,0.466667,0.705882}%
\pgfsetfillcolor{currentfill}%
\pgfsetfillopacity{0.734774}%
\pgfsetlinewidth{1.003750pt}%
\definecolor{currentstroke}{rgb}{0.121569,0.466667,0.705882}%
\pgfsetstrokecolor{currentstroke}%
\pgfsetstrokeopacity{0.734774}%
\pgfsetdash{}{0pt}%
\pgfpathmoveto{\pgfqpoint{3.215049in}{2.340126in}}%
\pgfpathcurveto{\pgfqpoint{3.223285in}{2.340126in}}{\pgfqpoint{3.231185in}{2.343398in}}{\pgfqpoint{3.237009in}{2.349222in}}%
\pgfpathcurveto{\pgfqpoint{3.242833in}{2.355046in}}{\pgfqpoint{3.246105in}{2.362946in}}{\pgfqpoint{3.246105in}{2.371183in}}%
\pgfpathcurveto{\pgfqpoint{3.246105in}{2.379419in}}{\pgfqpoint{3.242833in}{2.387319in}}{\pgfqpoint{3.237009in}{2.393143in}}%
\pgfpathcurveto{\pgfqpoint{3.231185in}{2.398967in}}{\pgfqpoint{3.223285in}{2.402239in}}{\pgfqpoint{3.215049in}{2.402239in}}%
\pgfpathcurveto{\pgfqpoint{3.206812in}{2.402239in}}{\pgfqpoint{3.198912in}{2.398967in}}{\pgfqpoint{3.193088in}{2.393143in}}%
\pgfpathcurveto{\pgfqpoint{3.187264in}{2.387319in}}{\pgfqpoint{3.183992in}{2.379419in}}{\pgfqpoint{3.183992in}{2.371183in}}%
\pgfpathcurveto{\pgfqpoint{3.183992in}{2.362946in}}{\pgfqpoint{3.187264in}{2.355046in}}{\pgfqpoint{3.193088in}{2.349222in}}%
\pgfpathcurveto{\pgfqpoint{3.198912in}{2.343398in}}{\pgfqpoint{3.206812in}{2.340126in}}{\pgfqpoint{3.215049in}{2.340126in}}%
\pgfpathclose%
\pgfusepath{stroke,fill}%
\end{pgfscope}%
\begin{pgfscope}%
\pgfpathrectangle{\pgfqpoint{0.100000in}{0.212622in}}{\pgfqpoint{3.696000in}{3.696000in}}%
\pgfusepath{clip}%
\pgfsetbuttcap%
\pgfsetroundjoin%
\definecolor{currentfill}{rgb}{0.121569,0.466667,0.705882}%
\pgfsetfillcolor{currentfill}%
\pgfsetfillopacity{0.735043}%
\pgfsetlinewidth{1.003750pt}%
\definecolor{currentstroke}{rgb}{0.121569,0.466667,0.705882}%
\pgfsetstrokecolor{currentstroke}%
\pgfsetstrokeopacity{0.735043}%
\pgfsetdash{}{0pt}%
\pgfpathmoveto{\pgfqpoint{2.373289in}{2.976636in}}%
\pgfpathcurveto{\pgfqpoint{2.381525in}{2.976636in}}{\pgfqpoint{2.389425in}{2.979909in}}{\pgfqpoint{2.395249in}{2.985733in}}%
\pgfpathcurveto{\pgfqpoint{2.401073in}{2.991557in}}{\pgfqpoint{2.404345in}{2.999457in}}{\pgfqpoint{2.404345in}{3.007693in}}%
\pgfpathcurveto{\pgfqpoint{2.404345in}{3.015929in}}{\pgfqpoint{2.401073in}{3.023829in}}{\pgfqpoint{2.395249in}{3.029653in}}%
\pgfpathcurveto{\pgfqpoint{2.389425in}{3.035477in}}{\pgfqpoint{2.381525in}{3.038749in}}{\pgfqpoint{2.373289in}{3.038749in}}%
\pgfpathcurveto{\pgfqpoint{2.365053in}{3.038749in}}{\pgfqpoint{2.357153in}{3.035477in}}{\pgfqpoint{2.351329in}{3.029653in}}%
\pgfpathcurveto{\pgfqpoint{2.345505in}{3.023829in}}{\pgfqpoint{2.342232in}{3.015929in}}{\pgfqpoint{2.342232in}{3.007693in}}%
\pgfpathcurveto{\pgfqpoint{2.342232in}{2.999457in}}{\pgfqpoint{2.345505in}{2.991557in}}{\pgfqpoint{2.351329in}{2.985733in}}%
\pgfpathcurveto{\pgfqpoint{2.357153in}{2.979909in}}{\pgfqpoint{2.365053in}{2.976636in}}{\pgfqpoint{2.373289in}{2.976636in}}%
\pgfpathclose%
\pgfusepath{stroke,fill}%
\end{pgfscope}%
\begin{pgfscope}%
\pgfpathrectangle{\pgfqpoint{0.100000in}{0.212622in}}{\pgfqpoint{3.696000in}{3.696000in}}%
\pgfusepath{clip}%
\pgfsetbuttcap%
\pgfsetroundjoin%
\definecolor{currentfill}{rgb}{0.121569,0.466667,0.705882}%
\pgfsetfillcolor{currentfill}%
\pgfsetfillopacity{0.736150}%
\pgfsetlinewidth{1.003750pt}%
\definecolor{currentstroke}{rgb}{0.121569,0.466667,0.705882}%
\pgfsetstrokecolor{currentstroke}%
\pgfsetstrokeopacity{0.736150}%
\pgfsetdash{}{0pt}%
\pgfpathmoveto{\pgfqpoint{1.195516in}{1.940505in}}%
\pgfpathcurveto{\pgfqpoint{1.203752in}{1.940505in}}{\pgfqpoint{1.211652in}{1.943777in}}{\pgfqpoint{1.217476in}{1.949601in}}%
\pgfpathcurveto{\pgfqpoint{1.223300in}{1.955425in}}{\pgfqpoint{1.226573in}{1.963325in}}{\pgfqpoint{1.226573in}{1.971561in}}%
\pgfpathcurveto{\pgfqpoint{1.226573in}{1.979797in}}{\pgfqpoint{1.223300in}{1.987697in}}{\pgfqpoint{1.217476in}{1.993521in}}%
\pgfpathcurveto{\pgfqpoint{1.211652in}{1.999345in}}{\pgfqpoint{1.203752in}{2.002618in}}{\pgfqpoint{1.195516in}{2.002618in}}%
\pgfpathcurveto{\pgfqpoint{1.187280in}{2.002618in}}{\pgfqpoint{1.179380in}{1.999345in}}{\pgfqpoint{1.173556in}{1.993521in}}%
\pgfpathcurveto{\pgfqpoint{1.167732in}{1.987697in}}{\pgfqpoint{1.164460in}{1.979797in}}{\pgfqpoint{1.164460in}{1.971561in}}%
\pgfpathcurveto{\pgfqpoint{1.164460in}{1.963325in}}{\pgfqpoint{1.167732in}{1.955425in}}{\pgfqpoint{1.173556in}{1.949601in}}%
\pgfpathcurveto{\pgfqpoint{1.179380in}{1.943777in}}{\pgfqpoint{1.187280in}{1.940505in}}{\pgfqpoint{1.195516in}{1.940505in}}%
\pgfpathclose%
\pgfusepath{stroke,fill}%
\end{pgfscope}%
\begin{pgfscope}%
\pgfpathrectangle{\pgfqpoint{0.100000in}{0.212622in}}{\pgfqpoint{3.696000in}{3.696000in}}%
\pgfusepath{clip}%
\pgfsetbuttcap%
\pgfsetroundjoin%
\definecolor{currentfill}{rgb}{0.121569,0.466667,0.705882}%
\pgfsetfillcolor{currentfill}%
\pgfsetfillopacity{0.736368}%
\pgfsetlinewidth{1.003750pt}%
\definecolor{currentstroke}{rgb}{0.121569,0.466667,0.705882}%
\pgfsetstrokecolor{currentstroke}%
\pgfsetstrokeopacity{0.736368}%
\pgfsetdash{}{0pt}%
\pgfpathmoveto{\pgfqpoint{3.208070in}{2.327855in}}%
\pgfpathcurveto{\pgfqpoint{3.216306in}{2.327855in}}{\pgfqpoint{3.224206in}{2.331127in}}{\pgfqpoint{3.230030in}{2.336951in}}%
\pgfpathcurveto{\pgfqpoint{3.235854in}{2.342775in}}{\pgfqpoint{3.239126in}{2.350675in}}{\pgfqpoint{3.239126in}{2.358911in}}%
\pgfpathcurveto{\pgfqpoint{3.239126in}{2.367147in}}{\pgfqpoint{3.235854in}{2.375047in}}{\pgfqpoint{3.230030in}{2.380871in}}%
\pgfpathcurveto{\pgfqpoint{3.224206in}{2.386695in}}{\pgfqpoint{3.216306in}{2.389968in}}{\pgfqpoint{3.208070in}{2.389968in}}%
\pgfpathcurveto{\pgfqpoint{3.199833in}{2.389968in}}{\pgfqpoint{3.191933in}{2.386695in}}{\pgfqpoint{3.186109in}{2.380871in}}%
\pgfpathcurveto{\pgfqpoint{3.180285in}{2.375047in}}{\pgfqpoint{3.177013in}{2.367147in}}{\pgfqpoint{3.177013in}{2.358911in}}%
\pgfpathcurveto{\pgfqpoint{3.177013in}{2.350675in}}{\pgfqpoint{3.180285in}{2.342775in}}{\pgfqpoint{3.186109in}{2.336951in}}%
\pgfpathcurveto{\pgfqpoint{3.191933in}{2.331127in}}{\pgfqpoint{3.199833in}{2.327855in}}{\pgfqpoint{3.208070in}{2.327855in}}%
\pgfpathclose%
\pgfusepath{stroke,fill}%
\end{pgfscope}%
\begin{pgfscope}%
\pgfpathrectangle{\pgfqpoint{0.100000in}{0.212622in}}{\pgfqpoint{3.696000in}{3.696000in}}%
\pgfusepath{clip}%
\pgfsetbuttcap%
\pgfsetroundjoin%
\definecolor{currentfill}{rgb}{0.121569,0.466667,0.705882}%
\pgfsetfillcolor{currentfill}%
\pgfsetfillopacity{0.736911}%
\pgfsetlinewidth{1.003750pt}%
\definecolor{currentstroke}{rgb}{0.121569,0.466667,0.705882}%
\pgfsetstrokecolor{currentstroke}%
\pgfsetstrokeopacity{0.736911}%
\pgfsetdash{}{0pt}%
\pgfpathmoveto{\pgfqpoint{2.381552in}{2.975462in}}%
\pgfpathcurveto{\pgfqpoint{2.389788in}{2.975462in}}{\pgfqpoint{2.397688in}{2.978734in}}{\pgfqpoint{2.403512in}{2.984558in}}%
\pgfpathcurveto{\pgfqpoint{2.409336in}{2.990382in}}{\pgfqpoint{2.412608in}{2.998282in}}{\pgfqpoint{2.412608in}{3.006518in}}%
\pgfpathcurveto{\pgfqpoint{2.412608in}{3.014754in}}{\pgfqpoint{2.409336in}{3.022654in}}{\pgfqpoint{2.403512in}{3.028478in}}%
\pgfpathcurveto{\pgfqpoint{2.397688in}{3.034302in}}{\pgfqpoint{2.389788in}{3.037575in}}{\pgfqpoint{2.381552in}{3.037575in}}%
\pgfpathcurveto{\pgfqpoint{2.373315in}{3.037575in}}{\pgfqpoint{2.365415in}{3.034302in}}{\pgfqpoint{2.359591in}{3.028478in}}%
\pgfpathcurveto{\pgfqpoint{2.353768in}{3.022654in}}{\pgfqpoint{2.350495in}{3.014754in}}{\pgfqpoint{2.350495in}{3.006518in}}%
\pgfpathcurveto{\pgfqpoint{2.350495in}{2.998282in}}{\pgfqpoint{2.353768in}{2.990382in}}{\pgfqpoint{2.359591in}{2.984558in}}%
\pgfpathcurveto{\pgfqpoint{2.365415in}{2.978734in}}{\pgfqpoint{2.373315in}{2.975462in}}{\pgfqpoint{2.381552in}{2.975462in}}%
\pgfpathclose%
\pgfusepath{stroke,fill}%
\end{pgfscope}%
\begin{pgfscope}%
\pgfpathrectangle{\pgfqpoint{0.100000in}{0.212622in}}{\pgfqpoint{3.696000in}{3.696000in}}%
\pgfusepath{clip}%
\pgfsetbuttcap%
\pgfsetroundjoin%
\definecolor{currentfill}{rgb}{0.121569,0.466667,0.705882}%
\pgfsetfillcolor{currentfill}%
\pgfsetfillopacity{0.737405}%
\pgfsetlinewidth{1.003750pt}%
\definecolor{currentstroke}{rgb}{0.121569,0.466667,0.705882}%
\pgfsetstrokecolor{currentstroke}%
\pgfsetstrokeopacity{0.737405}%
\pgfsetdash{}{0pt}%
\pgfpathmoveto{\pgfqpoint{1.187926in}{1.927225in}}%
\pgfpathcurveto{\pgfqpoint{1.196162in}{1.927225in}}{\pgfqpoint{1.204062in}{1.930497in}}{\pgfqpoint{1.209886in}{1.936321in}}%
\pgfpathcurveto{\pgfqpoint{1.215710in}{1.942145in}}{\pgfqpoint{1.218982in}{1.950045in}}{\pgfqpoint{1.218982in}{1.958281in}}%
\pgfpathcurveto{\pgfqpoint{1.218982in}{1.966518in}}{\pgfqpoint{1.215710in}{1.974418in}}{\pgfqpoint{1.209886in}{1.980242in}}%
\pgfpathcurveto{\pgfqpoint{1.204062in}{1.986066in}}{\pgfqpoint{1.196162in}{1.989338in}}{\pgfqpoint{1.187926in}{1.989338in}}%
\pgfpathcurveto{\pgfqpoint{1.179689in}{1.989338in}}{\pgfqpoint{1.171789in}{1.986066in}}{\pgfqpoint{1.165965in}{1.980242in}}%
\pgfpathcurveto{\pgfqpoint{1.160141in}{1.974418in}}{\pgfqpoint{1.156869in}{1.966518in}}{\pgfqpoint{1.156869in}{1.958281in}}%
\pgfpathcurveto{\pgfqpoint{1.156869in}{1.950045in}}{\pgfqpoint{1.160141in}{1.942145in}}{\pgfqpoint{1.165965in}{1.936321in}}%
\pgfpathcurveto{\pgfqpoint{1.171789in}{1.930497in}}{\pgfqpoint{1.179689in}{1.927225in}}{\pgfqpoint{1.187926in}{1.927225in}}%
\pgfpathclose%
\pgfusepath{stroke,fill}%
\end{pgfscope}%
\begin{pgfscope}%
\pgfpathrectangle{\pgfqpoint{0.100000in}{0.212622in}}{\pgfqpoint{3.696000in}{3.696000in}}%
\pgfusepath{clip}%
\pgfsetbuttcap%
\pgfsetroundjoin%
\definecolor{currentfill}{rgb}{0.121569,0.466667,0.705882}%
\pgfsetfillcolor{currentfill}%
\pgfsetfillopacity{0.738049}%
\pgfsetlinewidth{1.003750pt}%
\definecolor{currentstroke}{rgb}{0.121569,0.466667,0.705882}%
\pgfsetstrokecolor{currentstroke}%
\pgfsetstrokeopacity{0.738049}%
\pgfsetdash{}{0pt}%
\pgfpathmoveto{\pgfqpoint{3.205228in}{2.315285in}}%
\pgfpathcurveto{\pgfqpoint{3.213465in}{2.315285in}}{\pgfqpoint{3.221365in}{2.318558in}}{\pgfqpoint{3.227189in}{2.324382in}}%
\pgfpathcurveto{\pgfqpoint{3.233012in}{2.330206in}}{\pgfqpoint{3.236285in}{2.338106in}}{\pgfqpoint{3.236285in}{2.346342in}}%
\pgfpathcurveto{\pgfqpoint{3.236285in}{2.354578in}}{\pgfqpoint{3.233012in}{2.362478in}}{\pgfqpoint{3.227189in}{2.368302in}}%
\pgfpathcurveto{\pgfqpoint{3.221365in}{2.374126in}}{\pgfqpoint{3.213465in}{2.377398in}}{\pgfqpoint{3.205228in}{2.377398in}}%
\pgfpathcurveto{\pgfqpoint{3.196992in}{2.377398in}}{\pgfqpoint{3.189092in}{2.374126in}}{\pgfqpoint{3.183268in}{2.368302in}}%
\pgfpathcurveto{\pgfqpoint{3.177444in}{2.362478in}}{\pgfqpoint{3.174172in}{2.354578in}}{\pgfqpoint{3.174172in}{2.346342in}}%
\pgfpathcurveto{\pgfqpoint{3.174172in}{2.338106in}}{\pgfqpoint{3.177444in}{2.330206in}}{\pgfqpoint{3.183268in}{2.324382in}}%
\pgfpathcurveto{\pgfqpoint{3.189092in}{2.318558in}}{\pgfqpoint{3.196992in}{2.315285in}}{\pgfqpoint{3.205228in}{2.315285in}}%
\pgfpathclose%
\pgfusepath{stroke,fill}%
\end{pgfscope}%
\begin{pgfscope}%
\pgfpathrectangle{\pgfqpoint{0.100000in}{0.212622in}}{\pgfqpoint{3.696000in}{3.696000in}}%
\pgfusepath{clip}%
\pgfsetbuttcap%
\pgfsetroundjoin%
\definecolor{currentfill}{rgb}{0.121569,0.466667,0.705882}%
\pgfsetfillcolor{currentfill}%
\pgfsetfillopacity{0.738214}%
\pgfsetlinewidth{1.003750pt}%
\definecolor{currentstroke}{rgb}{0.121569,0.466667,0.705882}%
\pgfsetstrokecolor{currentstroke}%
\pgfsetstrokeopacity{0.738214}%
\pgfsetdash{}{0pt}%
\pgfpathmoveto{\pgfqpoint{1.183466in}{1.921072in}}%
\pgfpathcurveto{\pgfqpoint{1.191703in}{1.921072in}}{\pgfqpoint{1.199603in}{1.924345in}}{\pgfqpoint{1.205427in}{1.930169in}}%
\pgfpathcurveto{\pgfqpoint{1.211250in}{1.935993in}}{\pgfqpoint{1.214523in}{1.943893in}}{\pgfqpoint{1.214523in}{1.952129in}}%
\pgfpathcurveto{\pgfqpoint{1.214523in}{1.960365in}}{\pgfqpoint{1.211250in}{1.968265in}}{\pgfqpoint{1.205427in}{1.974089in}}%
\pgfpathcurveto{\pgfqpoint{1.199603in}{1.979913in}}{\pgfqpoint{1.191703in}{1.983185in}}{\pgfqpoint{1.183466in}{1.983185in}}%
\pgfpathcurveto{\pgfqpoint{1.175230in}{1.983185in}}{\pgfqpoint{1.167330in}{1.979913in}}{\pgfqpoint{1.161506in}{1.974089in}}%
\pgfpathcurveto{\pgfqpoint{1.155682in}{1.968265in}}{\pgfqpoint{1.152410in}{1.960365in}}{\pgfqpoint{1.152410in}{1.952129in}}%
\pgfpathcurveto{\pgfqpoint{1.152410in}{1.943893in}}{\pgfqpoint{1.155682in}{1.935993in}}{\pgfqpoint{1.161506in}{1.930169in}}%
\pgfpathcurveto{\pgfqpoint{1.167330in}{1.924345in}}{\pgfqpoint{1.175230in}{1.921072in}}{\pgfqpoint{1.183466in}{1.921072in}}%
\pgfpathclose%
\pgfusepath{stroke,fill}%
\end{pgfscope}%
\begin{pgfscope}%
\pgfpathrectangle{\pgfqpoint{0.100000in}{0.212622in}}{\pgfqpoint{3.696000in}{3.696000in}}%
\pgfusepath{clip}%
\pgfsetbuttcap%
\pgfsetroundjoin%
\definecolor{currentfill}{rgb}{0.121569,0.466667,0.705882}%
\pgfsetfillcolor{currentfill}%
\pgfsetfillopacity{0.738450}%
\pgfsetlinewidth{1.003750pt}%
\definecolor{currentstroke}{rgb}{0.121569,0.466667,0.705882}%
\pgfsetstrokecolor{currentstroke}%
\pgfsetstrokeopacity{0.738450}%
\pgfsetdash{}{0pt}%
\pgfpathmoveto{\pgfqpoint{2.389072in}{2.974036in}}%
\pgfpathcurveto{\pgfqpoint{2.397308in}{2.974036in}}{\pgfqpoint{2.405208in}{2.977308in}}{\pgfqpoint{2.411032in}{2.983132in}}%
\pgfpathcurveto{\pgfqpoint{2.416856in}{2.988956in}}{\pgfqpoint{2.420129in}{2.996856in}}{\pgfqpoint{2.420129in}{3.005092in}}%
\pgfpathcurveto{\pgfqpoint{2.420129in}{3.013328in}}{\pgfqpoint{2.416856in}{3.021228in}}{\pgfqpoint{2.411032in}{3.027052in}}%
\pgfpathcurveto{\pgfqpoint{2.405208in}{3.032876in}}{\pgfqpoint{2.397308in}{3.036149in}}{\pgfqpoint{2.389072in}{3.036149in}}%
\pgfpathcurveto{\pgfqpoint{2.380836in}{3.036149in}}{\pgfqpoint{2.372936in}{3.032876in}}{\pgfqpoint{2.367112in}{3.027052in}}%
\pgfpathcurveto{\pgfqpoint{2.361288in}{3.021228in}}{\pgfqpoint{2.358016in}{3.013328in}}{\pgfqpoint{2.358016in}{3.005092in}}%
\pgfpathcurveto{\pgfqpoint{2.358016in}{2.996856in}}{\pgfqpoint{2.361288in}{2.988956in}}{\pgfqpoint{2.367112in}{2.983132in}}%
\pgfpathcurveto{\pgfqpoint{2.372936in}{2.977308in}}{\pgfqpoint{2.380836in}{2.974036in}}{\pgfqpoint{2.389072in}{2.974036in}}%
\pgfpathclose%
\pgfusepath{stroke,fill}%
\end{pgfscope}%
\begin{pgfscope}%
\pgfpathrectangle{\pgfqpoint{0.100000in}{0.212622in}}{\pgfqpoint{3.696000in}{3.696000in}}%
\pgfusepath{clip}%
\pgfsetbuttcap%
\pgfsetroundjoin%
\definecolor{currentfill}{rgb}{0.121569,0.466667,0.705882}%
\pgfsetfillcolor{currentfill}%
\pgfsetfillopacity{0.739216}%
\pgfsetlinewidth{1.003750pt}%
\definecolor{currentstroke}{rgb}{0.121569,0.466667,0.705882}%
\pgfsetstrokecolor{currentstroke}%
\pgfsetstrokeopacity{0.739216}%
\pgfsetdash{}{0pt}%
\pgfpathmoveto{\pgfqpoint{1.178733in}{1.914491in}}%
\pgfpathcurveto{\pgfqpoint{1.186969in}{1.914491in}}{\pgfqpoint{1.194869in}{1.917763in}}{\pgfqpoint{1.200693in}{1.923587in}}%
\pgfpathcurveto{\pgfqpoint{1.206517in}{1.929411in}}{\pgfqpoint{1.209789in}{1.937311in}}{\pgfqpoint{1.209789in}{1.945547in}}%
\pgfpathcurveto{\pgfqpoint{1.209789in}{1.953784in}}{\pgfqpoint{1.206517in}{1.961684in}}{\pgfqpoint{1.200693in}{1.967508in}}%
\pgfpathcurveto{\pgfqpoint{1.194869in}{1.973332in}}{\pgfqpoint{1.186969in}{1.976604in}}{\pgfqpoint{1.178733in}{1.976604in}}%
\pgfpathcurveto{\pgfqpoint{1.170496in}{1.976604in}}{\pgfqpoint{1.162596in}{1.973332in}}{\pgfqpoint{1.156772in}{1.967508in}}%
\pgfpathcurveto{\pgfqpoint{1.150949in}{1.961684in}}{\pgfqpoint{1.147676in}{1.953784in}}{\pgfqpoint{1.147676in}{1.945547in}}%
\pgfpathcurveto{\pgfqpoint{1.147676in}{1.937311in}}{\pgfqpoint{1.150949in}{1.929411in}}{\pgfqpoint{1.156772in}{1.923587in}}%
\pgfpathcurveto{\pgfqpoint{1.162596in}{1.917763in}}{\pgfqpoint{1.170496in}{1.914491in}}{\pgfqpoint{1.178733in}{1.914491in}}%
\pgfpathclose%
\pgfusepath{stroke,fill}%
\end{pgfscope}%
\begin{pgfscope}%
\pgfpathrectangle{\pgfqpoint{0.100000in}{0.212622in}}{\pgfqpoint{3.696000in}{3.696000in}}%
\pgfusepath{clip}%
\pgfsetbuttcap%
\pgfsetroundjoin%
\definecolor{currentfill}{rgb}{0.121569,0.466667,0.705882}%
\pgfsetfillcolor{currentfill}%
\pgfsetfillopacity{0.739584}%
\pgfsetlinewidth{1.003750pt}%
\definecolor{currentstroke}{rgb}{0.121569,0.466667,0.705882}%
\pgfsetstrokecolor{currentstroke}%
\pgfsetstrokeopacity{0.739584}%
\pgfsetdash{}{0pt}%
\pgfpathmoveto{\pgfqpoint{3.204622in}{2.303810in}}%
\pgfpathcurveto{\pgfqpoint{3.212858in}{2.303810in}}{\pgfqpoint{3.220758in}{2.307082in}}{\pgfqpoint{3.226582in}{2.312906in}}%
\pgfpathcurveto{\pgfqpoint{3.232406in}{2.318730in}}{\pgfqpoint{3.235679in}{2.326630in}}{\pgfqpoint{3.235679in}{2.334866in}}%
\pgfpathcurveto{\pgfqpoint{3.235679in}{2.343103in}}{\pgfqpoint{3.232406in}{2.351003in}}{\pgfqpoint{3.226582in}{2.356827in}}%
\pgfpathcurveto{\pgfqpoint{3.220758in}{2.362651in}}{\pgfqpoint{3.212858in}{2.365923in}}{\pgfqpoint{3.204622in}{2.365923in}}%
\pgfpathcurveto{\pgfqpoint{3.196386in}{2.365923in}}{\pgfqpoint{3.188486in}{2.362651in}}{\pgfqpoint{3.182662in}{2.356827in}}%
\pgfpathcurveto{\pgfqpoint{3.176838in}{2.351003in}}{\pgfqpoint{3.173566in}{2.343103in}}{\pgfqpoint{3.173566in}{2.334866in}}%
\pgfpathcurveto{\pgfqpoint{3.173566in}{2.326630in}}{\pgfqpoint{3.176838in}{2.318730in}}{\pgfqpoint{3.182662in}{2.312906in}}%
\pgfpathcurveto{\pgfqpoint{3.188486in}{2.307082in}}{\pgfqpoint{3.196386in}{2.303810in}}{\pgfqpoint{3.204622in}{2.303810in}}%
\pgfpathclose%
\pgfusepath{stroke,fill}%
\end{pgfscope}%
\begin{pgfscope}%
\pgfpathrectangle{\pgfqpoint{0.100000in}{0.212622in}}{\pgfqpoint{3.696000in}{3.696000in}}%
\pgfusepath{clip}%
\pgfsetbuttcap%
\pgfsetroundjoin%
\definecolor{currentfill}{rgb}{0.121569,0.466667,0.705882}%
\pgfsetfillcolor{currentfill}%
\pgfsetfillopacity{0.739887}%
\pgfsetlinewidth{1.003750pt}%
\definecolor{currentstroke}{rgb}{0.121569,0.466667,0.705882}%
\pgfsetstrokecolor{currentstroke}%
\pgfsetstrokeopacity{0.739887}%
\pgfsetdash{}{0pt}%
\pgfpathmoveto{\pgfqpoint{2.395735in}{2.971935in}}%
\pgfpathcurveto{\pgfqpoint{2.403971in}{2.971935in}}{\pgfqpoint{2.411871in}{2.975207in}}{\pgfqpoint{2.417695in}{2.981031in}}%
\pgfpathcurveto{\pgfqpoint{2.423519in}{2.986855in}}{\pgfqpoint{2.426792in}{2.994755in}}{\pgfqpoint{2.426792in}{3.002991in}}%
\pgfpathcurveto{\pgfqpoint{2.426792in}{3.011228in}}{\pgfqpoint{2.423519in}{3.019128in}}{\pgfqpoint{2.417695in}{3.024951in}}%
\pgfpathcurveto{\pgfqpoint{2.411871in}{3.030775in}}{\pgfqpoint{2.403971in}{3.034048in}}{\pgfqpoint{2.395735in}{3.034048in}}%
\pgfpathcurveto{\pgfqpoint{2.387499in}{3.034048in}}{\pgfqpoint{2.379599in}{3.030775in}}{\pgfqpoint{2.373775in}{3.024951in}}%
\pgfpathcurveto{\pgfqpoint{2.367951in}{3.019128in}}{\pgfqpoint{2.364679in}{3.011228in}}{\pgfqpoint{2.364679in}{3.002991in}}%
\pgfpathcurveto{\pgfqpoint{2.364679in}{2.994755in}}{\pgfqpoint{2.367951in}{2.986855in}}{\pgfqpoint{2.373775in}{2.981031in}}%
\pgfpathcurveto{\pgfqpoint{2.379599in}{2.975207in}}{\pgfqpoint{2.387499in}{2.971935in}}{\pgfqpoint{2.395735in}{2.971935in}}%
\pgfpathclose%
\pgfusepath{stroke,fill}%
\end{pgfscope}%
\begin{pgfscope}%
\pgfpathrectangle{\pgfqpoint{0.100000in}{0.212622in}}{\pgfqpoint{3.696000in}{3.696000in}}%
\pgfusepath{clip}%
\pgfsetbuttcap%
\pgfsetroundjoin%
\definecolor{currentfill}{rgb}{0.121569,0.466667,0.705882}%
\pgfsetfillcolor{currentfill}%
\pgfsetfillopacity{0.740100}%
\pgfsetlinewidth{1.003750pt}%
\definecolor{currentstroke}{rgb}{0.121569,0.466667,0.705882}%
\pgfsetstrokecolor{currentstroke}%
\pgfsetstrokeopacity{0.740100}%
\pgfsetdash{}{0pt}%
\pgfpathmoveto{\pgfqpoint{1.173762in}{1.903778in}}%
\pgfpathcurveto{\pgfqpoint{1.181999in}{1.903778in}}{\pgfqpoint{1.189899in}{1.907051in}}{\pgfqpoint{1.195723in}{1.912875in}}%
\pgfpathcurveto{\pgfqpoint{1.201547in}{1.918698in}}{\pgfqpoint{1.204819in}{1.926598in}}{\pgfqpoint{1.204819in}{1.934835in}}%
\pgfpathcurveto{\pgfqpoint{1.204819in}{1.943071in}}{\pgfqpoint{1.201547in}{1.950971in}}{\pgfqpoint{1.195723in}{1.956795in}}%
\pgfpathcurveto{\pgfqpoint{1.189899in}{1.962619in}}{\pgfqpoint{1.181999in}{1.965891in}}{\pgfqpoint{1.173762in}{1.965891in}}%
\pgfpathcurveto{\pgfqpoint{1.165526in}{1.965891in}}{\pgfqpoint{1.157626in}{1.962619in}}{\pgfqpoint{1.151802in}{1.956795in}}%
\pgfpathcurveto{\pgfqpoint{1.145978in}{1.950971in}}{\pgfqpoint{1.142706in}{1.943071in}}{\pgfqpoint{1.142706in}{1.934835in}}%
\pgfpathcurveto{\pgfqpoint{1.142706in}{1.926598in}}{\pgfqpoint{1.145978in}{1.918698in}}{\pgfqpoint{1.151802in}{1.912875in}}%
\pgfpathcurveto{\pgfqpoint{1.157626in}{1.907051in}}{\pgfqpoint{1.165526in}{1.903778in}}{\pgfqpoint{1.173762in}{1.903778in}}%
\pgfpathclose%
\pgfusepath{stroke,fill}%
\end{pgfscope}%
\begin{pgfscope}%
\pgfpathrectangle{\pgfqpoint{0.100000in}{0.212622in}}{\pgfqpoint{3.696000in}{3.696000in}}%
\pgfusepath{clip}%
\pgfsetbuttcap%
\pgfsetroundjoin%
\definecolor{currentfill}{rgb}{0.121569,0.466667,0.705882}%
\pgfsetfillcolor{currentfill}%
\pgfsetfillopacity{0.740684}%
\pgfsetlinewidth{1.003750pt}%
\definecolor{currentstroke}{rgb}{0.121569,0.466667,0.705882}%
\pgfsetstrokecolor{currentstroke}%
\pgfsetstrokeopacity{0.740684}%
\pgfsetdash{}{0pt}%
\pgfpathmoveto{\pgfqpoint{1.172066in}{1.897393in}}%
\pgfpathcurveto{\pgfqpoint{1.180302in}{1.897393in}}{\pgfqpoint{1.188202in}{1.900665in}}{\pgfqpoint{1.194026in}{1.906489in}}%
\pgfpathcurveto{\pgfqpoint{1.199850in}{1.912313in}}{\pgfqpoint{1.203122in}{1.920213in}}{\pgfqpoint{1.203122in}{1.928450in}}%
\pgfpathcurveto{\pgfqpoint{1.203122in}{1.936686in}}{\pgfqpoint{1.199850in}{1.944586in}}{\pgfqpoint{1.194026in}{1.950410in}}%
\pgfpathcurveto{\pgfqpoint{1.188202in}{1.956234in}}{\pgfqpoint{1.180302in}{1.959506in}}{\pgfqpoint{1.172066in}{1.959506in}}%
\pgfpathcurveto{\pgfqpoint{1.163829in}{1.959506in}}{\pgfqpoint{1.155929in}{1.956234in}}{\pgfqpoint{1.150105in}{1.950410in}}%
\pgfpathcurveto{\pgfqpoint{1.144281in}{1.944586in}}{\pgfqpoint{1.141009in}{1.936686in}}{\pgfqpoint{1.141009in}{1.928450in}}%
\pgfpathcurveto{\pgfqpoint{1.141009in}{1.920213in}}{\pgfqpoint{1.144281in}{1.912313in}}{\pgfqpoint{1.150105in}{1.906489in}}%
\pgfpathcurveto{\pgfqpoint{1.155929in}{1.900665in}}{\pgfqpoint{1.163829in}{1.897393in}}{\pgfqpoint{1.172066in}{1.897393in}}%
\pgfpathclose%
\pgfusepath{stroke,fill}%
\end{pgfscope}%
\begin{pgfscope}%
\pgfpathrectangle{\pgfqpoint{0.100000in}{0.212622in}}{\pgfqpoint{3.696000in}{3.696000in}}%
\pgfusepath{clip}%
\pgfsetbuttcap%
\pgfsetroundjoin%
\definecolor{currentfill}{rgb}{0.121569,0.466667,0.705882}%
\pgfsetfillcolor{currentfill}%
\pgfsetfillopacity{0.741168}%
\pgfsetlinewidth{1.003750pt}%
\definecolor{currentstroke}{rgb}{0.121569,0.466667,0.705882}%
\pgfsetstrokecolor{currentstroke}%
\pgfsetstrokeopacity{0.741168}%
\pgfsetdash{}{0pt}%
\pgfpathmoveto{\pgfqpoint{3.202707in}{2.293251in}}%
\pgfpathcurveto{\pgfqpoint{3.210943in}{2.293251in}}{\pgfqpoint{3.218843in}{2.296523in}}{\pgfqpoint{3.224667in}{2.302347in}}%
\pgfpathcurveto{\pgfqpoint{3.230491in}{2.308171in}}{\pgfqpoint{3.233763in}{2.316071in}}{\pgfqpoint{3.233763in}{2.324307in}}%
\pgfpathcurveto{\pgfqpoint{3.233763in}{2.332543in}}{\pgfqpoint{3.230491in}{2.340443in}}{\pgfqpoint{3.224667in}{2.346267in}}%
\pgfpathcurveto{\pgfqpoint{3.218843in}{2.352091in}}{\pgfqpoint{3.210943in}{2.355364in}}{\pgfqpoint{3.202707in}{2.355364in}}%
\pgfpathcurveto{\pgfqpoint{3.194470in}{2.355364in}}{\pgfqpoint{3.186570in}{2.352091in}}{\pgfqpoint{3.180746in}{2.346267in}}%
\pgfpathcurveto{\pgfqpoint{3.174923in}{2.340443in}}{\pgfqpoint{3.171650in}{2.332543in}}{\pgfqpoint{3.171650in}{2.324307in}}%
\pgfpathcurveto{\pgfqpoint{3.171650in}{2.316071in}}{\pgfqpoint{3.174923in}{2.308171in}}{\pgfqpoint{3.180746in}{2.302347in}}%
\pgfpathcurveto{\pgfqpoint{3.186570in}{2.296523in}}{\pgfqpoint{3.194470in}{2.293251in}}{\pgfqpoint{3.202707in}{2.293251in}}%
\pgfpathclose%
\pgfusepath{stroke,fill}%
\end{pgfscope}%
\begin{pgfscope}%
\pgfpathrectangle{\pgfqpoint{0.100000in}{0.212622in}}{\pgfqpoint{3.696000in}{3.696000in}}%
\pgfusepath{clip}%
\pgfsetbuttcap%
\pgfsetroundjoin%
\definecolor{currentfill}{rgb}{0.121569,0.466667,0.705882}%
\pgfsetfillcolor{currentfill}%
\pgfsetfillopacity{0.741337}%
\pgfsetlinewidth{1.003750pt}%
\definecolor{currentstroke}{rgb}{0.121569,0.466667,0.705882}%
\pgfsetstrokecolor{currentstroke}%
\pgfsetstrokeopacity{0.741337}%
\pgfsetdash{}{0pt}%
\pgfpathmoveto{\pgfqpoint{1.169334in}{1.891138in}}%
\pgfpathcurveto{\pgfqpoint{1.177571in}{1.891138in}}{\pgfqpoint{1.185471in}{1.894411in}}{\pgfqpoint{1.191295in}{1.900235in}}%
\pgfpathcurveto{\pgfqpoint{1.197118in}{1.906059in}}{\pgfqpoint{1.200391in}{1.913959in}}{\pgfqpoint{1.200391in}{1.922195in}}%
\pgfpathcurveto{\pgfqpoint{1.200391in}{1.930431in}}{\pgfqpoint{1.197118in}{1.938331in}}{\pgfqpoint{1.191295in}{1.944155in}}%
\pgfpathcurveto{\pgfqpoint{1.185471in}{1.949979in}}{\pgfqpoint{1.177571in}{1.953251in}}{\pgfqpoint{1.169334in}{1.953251in}}%
\pgfpathcurveto{\pgfqpoint{1.161098in}{1.953251in}}{\pgfqpoint{1.153198in}{1.949979in}}{\pgfqpoint{1.147374in}{1.944155in}}%
\pgfpathcurveto{\pgfqpoint{1.141550in}{1.938331in}}{\pgfqpoint{1.138278in}{1.930431in}}{\pgfqpoint{1.138278in}{1.922195in}}%
\pgfpathcurveto{\pgfqpoint{1.138278in}{1.913959in}}{\pgfqpoint{1.141550in}{1.906059in}}{\pgfqpoint{1.147374in}{1.900235in}}%
\pgfpathcurveto{\pgfqpoint{1.153198in}{1.894411in}}{\pgfqpoint{1.161098in}{1.891138in}}{\pgfqpoint{1.169334in}{1.891138in}}%
\pgfpathclose%
\pgfusepath{stroke,fill}%
\end{pgfscope}%
\begin{pgfscope}%
\pgfpathrectangle{\pgfqpoint{0.100000in}{0.212622in}}{\pgfqpoint{3.696000in}{3.696000in}}%
\pgfusepath{clip}%
\pgfsetbuttcap%
\pgfsetroundjoin%
\definecolor{currentfill}{rgb}{0.121569,0.466667,0.705882}%
\pgfsetfillcolor{currentfill}%
\pgfsetfillopacity{0.741426}%
\pgfsetlinewidth{1.003750pt}%
\definecolor{currentstroke}{rgb}{0.121569,0.466667,0.705882}%
\pgfsetstrokecolor{currentstroke}%
\pgfsetstrokeopacity{0.741426}%
\pgfsetdash{}{0pt}%
\pgfpathmoveto{\pgfqpoint{2.401736in}{2.970356in}}%
\pgfpathcurveto{\pgfqpoint{2.409973in}{2.970356in}}{\pgfqpoint{2.417873in}{2.973629in}}{\pgfqpoint{2.423697in}{2.979452in}}%
\pgfpathcurveto{\pgfqpoint{2.429521in}{2.985276in}}{\pgfqpoint{2.432793in}{2.993176in}}{\pgfqpoint{2.432793in}{3.001413in}}%
\pgfpathcurveto{\pgfqpoint{2.432793in}{3.009649in}}{\pgfqpoint{2.429521in}{3.017549in}}{\pgfqpoint{2.423697in}{3.023373in}}%
\pgfpathcurveto{\pgfqpoint{2.417873in}{3.029197in}}{\pgfqpoint{2.409973in}{3.032469in}}{\pgfqpoint{2.401736in}{3.032469in}}%
\pgfpathcurveto{\pgfqpoint{2.393500in}{3.032469in}}{\pgfqpoint{2.385600in}{3.029197in}}{\pgfqpoint{2.379776in}{3.023373in}}%
\pgfpathcurveto{\pgfqpoint{2.373952in}{3.017549in}}{\pgfqpoint{2.370680in}{3.009649in}}{\pgfqpoint{2.370680in}{3.001413in}}%
\pgfpathcurveto{\pgfqpoint{2.370680in}{2.993176in}}{\pgfqpoint{2.373952in}{2.985276in}}{\pgfqpoint{2.379776in}{2.979452in}}%
\pgfpathcurveto{\pgfqpoint{2.385600in}{2.973629in}}{\pgfqpoint{2.393500in}{2.970356in}}{\pgfqpoint{2.401736in}{2.970356in}}%
\pgfpathclose%
\pgfusepath{stroke,fill}%
\end{pgfscope}%
\begin{pgfscope}%
\pgfpathrectangle{\pgfqpoint{0.100000in}{0.212622in}}{\pgfqpoint{3.696000in}{3.696000in}}%
\pgfusepath{clip}%
\pgfsetbuttcap%
\pgfsetroundjoin%
\definecolor{currentfill}{rgb}{0.121569,0.466667,0.705882}%
\pgfsetfillcolor{currentfill}%
\pgfsetfillopacity{0.741666}%
\pgfsetlinewidth{1.003750pt}%
\definecolor{currentstroke}{rgb}{0.121569,0.466667,0.705882}%
\pgfsetstrokecolor{currentstroke}%
\pgfsetstrokeopacity{0.741666}%
\pgfsetdash{}{0pt}%
\pgfpathmoveto{\pgfqpoint{1.167559in}{1.887976in}}%
\pgfpathcurveto{\pgfqpoint{1.175795in}{1.887976in}}{\pgfqpoint{1.183695in}{1.891248in}}{\pgfqpoint{1.189519in}{1.897072in}}%
\pgfpathcurveto{\pgfqpoint{1.195343in}{1.902896in}}{\pgfqpoint{1.198615in}{1.910796in}}{\pgfqpoint{1.198615in}{1.919032in}}%
\pgfpathcurveto{\pgfqpoint{1.198615in}{1.927269in}}{\pgfqpoint{1.195343in}{1.935169in}}{\pgfqpoint{1.189519in}{1.940993in}}%
\pgfpathcurveto{\pgfqpoint{1.183695in}{1.946817in}}{\pgfqpoint{1.175795in}{1.950089in}}{\pgfqpoint{1.167559in}{1.950089in}}%
\pgfpathcurveto{\pgfqpoint{1.159322in}{1.950089in}}{\pgfqpoint{1.151422in}{1.946817in}}{\pgfqpoint{1.145598in}{1.940993in}}%
\pgfpathcurveto{\pgfqpoint{1.139774in}{1.935169in}}{\pgfqpoint{1.136502in}{1.927269in}}{\pgfqpoint{1.136502in}{1.919032in}}%
\pgfpathcurveto{\pgfqpoint{1.136502in}{1.910796in}}{\pgfqpoint{1.139774in}{1.902896in}}{\pgfqpoint{1.145598in}{1.897072in}}%
\pgfpathcurveto{\pgfqpoint{1.151422in}{1.891248in}}{\pgfqpoint{1.159322in}{1.887976in}}{\pgfqpoint{1.167559in}{1.887976in}}%
\pgfpathclose%
\pgfusepath{stroke,fill}%
\end{pgfscope}%
\begin{pgfscope}%
\pgfpathrectangle{\pgfqpoint{0.100000in}{0.212622in}}{\pgfqpoint{3.696000in}{3.696000in}}%
\pgfusepath{clip}%
\pgfsetbuttcap%
\pgfsetroundjoin%
\definecolor{currentfill}{rgb}{0.121569,0.466667,0.705882}%
\pgfsetfillcolor{currentfill}%
\pgfsetfillopacity{0.742008}%
\pgfsetlinewidth{1.003750pt}%
\definecolor{currentstroke}{rgb}{0.121569,0.466667,0.705882}%
\pgfsetstrokecolor{currentstroke}%
\pgfsetstrokeopacity{0.742008}%
\pgfsetdash{}{0pt}%
\pgfpathmoveto{\pgfqpoint{1.165338in}{1.884658in}}%
\pgfpathcurveto{\pgfqpoint{1.173574in}{1.884658in}}{\pgfqpoint{1.181474in}{1.887930in}}{\pgfqpoint{1.187298in}{1.893754in}}%
\pgfpathcurveto{\pgfqpoint{1.193122in}{1.899578in}}{\pgfqpoint{1.196394in}{1.907478in}}{\pgfqpoint{1.196394in}{1.915714in}}%
\pgfpathcurveto{\pgfqpoint{1.196394in}{1.923951in}}{\pgfqpoint{1.193122in}{1.931851in}}{\pgfqpoint{1.187298in}{1.937675in}}%
\pgfpathcurveto{\pgfqpoint{1.181474in}{1.943498in}}{\pgfqpoint{1.173574in}{1.946771in}}{\pgfqpoint{1.165338in}{1.946771in}}%
\pgfpathcurveto{\pgfqpoint{1.157102in}{1.946771in}}{\pgfqpoint{1.149202in}{1.943498in}}{\pgfqpoint{1.143378in}{1.937675in}}%
\pgfpathcurveto{\pgfqpoint{1.137554in}{1.931851in}}{\pgfqpoint{1.134281in}{1.923951in}}{\pgfqpoint{1.134281in}{1.915714in}}%
\pgfpathcurveto{\pgfqpoint{1.134281in}{1.907478in}}{\pgfqpoint{1.137554in}{1.899578in}}{\pgfqpoint{1.143378in}{1.893754in}}%
\pgfpathcurveto{\pgfqpoint{1.149202in}{1.887930in}}{\pgfqpoint{1.157102in}{1.884658in}}{\pgfqpoint{1.165338in}{1.884658in}}%
\pgfpathclose%
\pgfusepath{stroke,fill}%
\end{pgfscope}%
\begin{pgfscope}%
\pgfpathrectangle{\pgfqpoint{0.100000in}{0.212622in}}{\pgfqpoint{3.696000in}{3.696000in}}%
\pgfusepath{clip}%
\pgfsetbuttcap%
\pgfsetroundjoin%
\definecolor{currentfill}{rgb}{0.121569,0.466667,0.705882}%
\pgfsetfillcolor{currentfill}%
\pgfsetfillopacity{0.742242}%
\pgfsetlinewidth{1.003750pt}%
\definecolor{currentstroke}{rgb}{0.121569,0.466667,0.705882}%
\pgfsetstrokecolor{currentstroke}%
\pgfsetstrokeopacity{0.742242}%
\pgfsetdash{}{0pt}%
\pgfpathmoveto{\pgfqpoint{3.198612in}{2.286101in}}%
\pgfpathcurveto{\pgfqpoint{3.206848in}{2.286101in}}{\pgfqpoint{3.214749in}{2.289373in}}{\pgfqpoint{3.220572in}{2.295197in}}%
\pgfpathcurveto{\pgfqpoint{3.226396in}{2.301021in}}{\pgfqpoint{3.229669in}{2.308921in}}{\pgfqpoint{3.229669in}{2.317157in}}%
\pgfpathcurveto{\pgfqpoint{3.229669in}{2.325394in}}{\pgfqpoint{3.226396in}{2.333294in}}{\pgfqpoint{3.220572in}{2.339118in}}%
\pgfpathcurveto{\pgfqpoint{3.214749in}{2.344942in}}{\pgfqpoint{3.206848in}{2.348214in}}{\pgfqpoint{3.198612in}{2.348214in}}%
\pgfpathcurveto{\pgfqpoint{3.190376in}{2.348214in}}{\pgfqpoint{3.182476in}{2.344942in}}{\pgfqpoint{3.176652in}{2.339118in}}%
\pgfpathcurveto{\pgfqpoint{3.170828in}{2.333294in}}{\pgfqpoint{3.167556in}{2.325394in}}{\pgfqpoint{3.167556in}{2.317157in}}%
\pgfpathcurveto{\pgfqpoint{3.167556in}{2.308921in}}{\pgfqpoint{3.170828in}{2.301021in}}{\pgfqpoint{3.176652in}{2.295197in}}%
\pgfpathcurveto{\pgfqpoint{3.182476in}{2.289373in}}{\pgfqpoint{3.190376in}{2.286101in}}{\pgfqpoint{3.198612in}{2.286101in}}%
\pgfpathclose%
\pgfusepath{stroke,fill}%
\end{pgfscope}%
\begin{pgfscope}%
\pgfpathrectangle{\pgfqpoint{0.100000in}{0.212622in}}{\pgfqpoint{3.696000in}{3.696000in}}%
\pgfusepath{clip}%
\pgfsetbuttcap%
\pgfsetroundjoin%
\definecolor{currentfill}{rgb}{0.121569,0.466667,0.705882}%
\pgfsetfillcolor{currentfill}%
\pgfsetfillopacity{0.742482}%
\pgfsetlinewidth{1.003750pt}%
\definecolor{currentstroke}{rgb}{0.121569,0.466667,0.705882}%
\pgfsetstrokecolor{currentstroke}%
\pgfsetstrokeopacity{0.742482}%
\pgfsetdash{}{0pt}%
\pgfpathmoveto{\pgfqpoint{1.162840in}{1.880951in}}%
\pgfpathcurveto{\pgfqpoint{1.171076in}{1.880951in}}{\pgfqpoint{1.178976in}{1.884223in}}{\pgfqpoint{1.184800in}{1.890047in}}%
\pgfpathcurveto{\pgfqpoint{1.190624in}{1.895871in}}{\pgfqpoint{1.193897in}{1.903771in}}{\pgfqpoint{1.193897in}{1.912007in}}%
\pgfpathcurveto{\pgfqpoint{1.193897in}{1.920244in}}{\pgfqpoint{1.190624in}{1.928144in}}{\pgfqpoint{1.184800in}{1.933968in}}%
\pgfpathcurveto{\pgfqpoint{1.178976in}{1.939791in}}{\pgfqpoint{1.171076in}{1.943064in}}{\pgfqpoint{1.162840in}{1.943064in}}%
\pgfpathcurveto{\pgfqpoint{1.154604in}{1.943064in}}{\pgfqpoint{1.146704in}{1.939791in}}{\pgfqpoint{1.140880in}{1.933968in}}%
\pgfpathcurveto{\pgfqpoint{1.135056in}{1.928144in}}{\pgfqpoint{1.131784in}{1.920244in}}{\pgfqpoint{1.131784in}{1.912007in}}%
\pgfpathcurveto{\pgfqpoint{1.131784in}{1.903771in}}{\pgfqpoint{1.135056in}{1.895871in}}{\pgfqpoint{1.140880in}{1.890047in}}%
\pgfpathcurveto{\pgfqpoint{1.146704in}{1.884223in}}{\pgfqpoint{1.154604in}{1.880951in}}{\pgfqpoint{1.162840in}{1.880951in}}%
\pgfpathclose%
\pgfusepath{stroke,fill}%
\end{pgfscope}%
\begin{pgfscope}%
\pgfpathrectangle{\pgfqpoint{0.100000in}{0.212622in}}{\pgfqpoint{3.696000in}{3.696000in}}%
\pgfusepath{clip}%
\pgfsetbuttcap%
\pgfsetroundjoin%
\definecolor{currentfill}{rgb}{0.121569,0.466667,0.705882}%
\pgfsetfillcolor{currentfill}%
\pgfsetfillopacity{0.742743}%
\pgfsetlinewidth{1.003750pt}%
\definecolor{currentstroke}{rgb}{0.121569,0.466667,0.705882}%
\pgfsetstrokecolor{currentstroke}%
\pgfsetstrokeopacity{0.742743}%
\pgfsetdash{}{0pt}%
\pgfpathmoveto{\pgfqpoint{2.406711in}{2.969025in}}%
\pgfpathcurveto{\pgfqpoint{2.414947in}{2.969025in}}{\pgfqpoint{2.422847in}{2.972298in}}{\pgfqpoint{2.428671in}{2.978122in}}%
\pgfpathcurveto{\pgfqpoint{2.434495in}{2.983946in}}{\pgfqpoint{2.437768in}{2.991846in}}{\pgfqpoint{2.437768in}{3.000082in}}%
\pgfpathcurveto{\pgfqpoint{2.437768in}{3.008318in}}{\pgfqpoint{2.434495in}{3.016218in}}{\pgfqpoint{2.428671in}{3.022042in}}%
\pgfpathcurveto{\pgfqpoint{2.422847in}{3.027866in}}{\pgfqpoint{2.414947in}{3.031138in}}{\pgfqpoint{2.406711in}{3.031138in}}%
\pgfpathcurveto{\pgfqpoint{2.398475in}{3.031138in}}{\pgfqpoint{2.390575in}{3.027866in}}{\pgfqpoint{2.384751in}{3.022042in}}%
\pgfpathcurveto{\pgfqpoint{2.378927in}{3.016218in}}{\pgfqpoint{2.375655in}{3.008318in}}{\pgfqpoint{2.375655in}{3.000082in}}%
\pgfpathcurveto{\pgfqpoint{2.375655in}{2.991846in}}{\pgfqpoint{2.378927in}{2.983946in}}{\pgfqpoint{2.384751in}{2.978122in}}%
\pgfpathcurveto{\pgfqpoint{2.390575in}{2.972298in}}{\pgfqpoint{2.398475in}{2.969025in}}{\pgfqpoint{2.406711in}{2.969025in}}%
\pgfpathclose%
\pgfusepath{stroke,fill}%
\end{pgfscope}%
\begin{pgfscope}%
\pgfpathrectangle{\pgfqpoint{0.100000in}{0.212622in}}{\pgfqpoint{3.696000in}{3.696000in}}%
\pgfusepath{clip}%
\pgfsetbuttcap%
\pgfsetroundjoin%
\definecolor{currentfill}{rgb}{0.121569,0.466667,0.705882}%
\pgfsetfillcolor{currentfill}%
\pgfsetfillopacity{0.742886}%
\pgfsetlinewidth{1.003750pt}%
\definecolor{currentstroke}{rgb}{0.121569,0.466667,0.705882}%
\pgfsetstrokecolor{currentstroke}%
\pgfsetstrokeopacity{0.742886}%
\pgfsetdash{}{0pt}%
\pgfpathmoveto{\pgfqpoint{3.195390in}{2.281418in}}%
\pgfpathcurveto{\pgfqpoint{3.203626in}{2.281418in}}{\pgfqpoint{3.211526in}{2.284690in}}{\pgfqpoint{3.217350in}{2.290514in}}%
\pgfpathcurveto{\pgfqpoint{3.223174in}{2.296338in}}{\pgfqpoint{3.226447in}{2.304238in}}{\pgfqpoint{3.226447in}{2.312475in}}%
\pgfpathcurveto{\pgfqpoint{3.226447in}{2.320711in}}{\pgfqpoint{3.223174in}{2.328611in}}{\pgfqpoint{3.217350in}{2.334435in}}%
\pgfpathcurveto{\pgfqpoint{3.211526in}{2.340259in}}{\pgfqpoint{3.203626in}{2.343531in}}{\pgfqpoint{3.195390in}{2.343531in}}%
\pgfpathcurveto{\pgfqpoint{3.187154in}{2.343531in}}{\pgfqpoint{3.179254in}{2.340259in}}{\pgfqpoint{3.173430in}{2.334435in}}%
\pgfpathcurveto{\pgfqpoint{3.167606in}{2.328611in}}{\pgfqpoint{3.164334in}{2.320711in}}{\pgfqpoint{3.164334in}{2.312475in}}%
\pgfpathcurveto{\pgfqpoint{3.164334in}{2.304238in}}{\pgfqpoint{3.167606in}{2.296338in}}{\pgfqpoint{3.173430in}{2.290514in}}%
\pgfpathcurveto{\pgfqpoint{3.179254in}{2.284690in}}{\pgfqpoint{3.187154in}{2.281418in}}{\pgfqpoint{3.195390in}{2.281418in}}%
\pgfpathclose%
\pgfusepath{stroke,fill}%
\end{pgfscope}%
\begin{pgfscope}%
\pgfpathrectangle{\pgfqpoint{0.100000in}{0.212622in}}{\pgfqpoint{3.696000in}{3.696000in}}%
\pgfusepath{clip}%
\pgfsetbuttcap%
\pgfsetroundjoin%
\definecolor{currentfill}{rgb}{0.121569,0.466667,0.705882}%
\pgfsetfillcolor{currentfill}%
\pgfsetfillopacity{0.743115}%
\pgfsetlinewidth{1.003750pt}%
\definecolor{currentstroke}{rgb}{0.121569,0.466667,0.705882}%
\pgfsetstrokecolor{currentstroke}%
\pgfsetstrokeopacity{0.743115}%
\pgfsetdash{}{0pt}%
\pgfpathmoveto{\pgfqpoint{1.160109in}{1.874986in}}%
\pgfpathcurveto{\pgfqpoint{1.168346in}{1.874986in}}{\pgfqpoint{1.176246in}{1.878258in}}{\pgfqpoint{1.182070in}{1.884082in}}%
\pgfpathcurveto{\pgfqpoint{1.187893in}{1.889906in}}{\pgfqpoint{1.191166in}{1.897806in}}{\pgfqpoint{1.191166in}{1.906042in}}%
\pgfpathcurveto{\pgfqpoint{1.191166in}{1.914278in}}{\pgfqpoint{1.187893in}{1.922178in}}{\pgfqpoint{1.182070in}{1.928002in}}%
\pgfpathcurveto{\pgfqpoint{1.176246in}{1.933826in}}{\pgfqpoint{1.168346in}{1.937099in}}{\pgfqpoint{1.160109in}{1.937099in}}%
\pgfpathcurveto{\pgfqpoint{1.151873in}{1.937099in}}{\pgfqpoint{1.143973in}{1.933826in}}{\pgfqpoint{1.138149in}{1.928002in}}%
\pgfpathcurveto{\pgfqpoint{1.132325in}{1.922178in}}{\pgfqpoint{1.129053in}{1.914278in}}{\pgfqpoint{1.129053in}{1.906042in}}%
\pgfpathcurveto{\pgfqpoint{1.129053in}{1.897806in}}{\pgfqpoint{1.132325in}{1.889906in}}{\pgfqpoint{1.138149in}{1.884082in}}%
\pgfpathcurveto{\pgfqpoint{1.143973in}{1.878258in}}{\pgfqpoint{1.151873in}{1.874986in}}{\pgfqpoint{1.160109in}{1.874986in}}%
\pgfpathclose%
\pgfusepath{stroke,fill}%
\end{pgfscope}%
\begin{pgfscope}%
\pgfpathrectangle{\pgfqpoint{0.100000in}{0.212622in}}{\pgfqpoint{3.696000in}{3.696000in}}%
\pgfusepath{clip}%
\pgfsetbuttcap%
\pgfsetroundjoin%
\definecolor{currentfill}{rgb}{0.121569,0.466667,0.705882}%
\pgfsetfillcolor{currentfill}%
\pgfsetfillopacity{0.743489}%
\pgfsetlinewidth{1.003750pt}%
\definecolor{currentstroke}{rgb}{0.121569,0.466667,0.705882}%
\pgfsetstrokecolor{currentstroke}%
\pgfsetstrokeopacity{0.743489}%
\pgfsetdash{}{0pt}%
\pgfpathmoveto{\pgfqpoint{1.159035in}{1.871451in}}%
\pgfpathcurveto{\pgfqpoint{1.167271in}{1.871451in}}{\pgfqpoint{1.175171in}{1.874724in}}{\pgfqpoint{1.180995in}{1.880547in}}%
\pgfpathcurveto{\pgfqpoint{1.186819in}{1.886371in}}{\pgfqpoint{1.190091in}{1.894271in}}{\pgfqpoint{1.190091in}{1.902508in}}%
\pgfpathcurveto{\pgfqpoint{1.190091in}{1.910744in}}{\pgfqpoint{1.186819in}{1.918644in}}{\pgfqpoint{1.180995in}{1.924468in}}%
\pgfpathcurveto{\pgfqpoint{1.175171in}{1.930292in}}{\pgfqpoint{1.167271in}{1.933564in}}{\pgfqpoint{1.159035in}{1.933564in}}%
\pgfpathcurveto{\pgfqpoint{1.150799in}{1.933564in}}{\pgfqpoint{1.142899in}{1.930292in}}{\pgfqpoint{1.137075in}{1.924468in}}%
\pgfpathcurveto{\pgfqpoint{1.131251in}{1.918644in}}{\pgfqpoint{1.127978in}{1.910744in}}{\pgfqpoint{1.127978in}{1.902508in}}%
\pgfpathcurveto{\pgfqpoint{1.127978in}{1.894271in}}{\pgfqpoint{1.131251in}{1.886371in}}{\pgfqpoint{1.137075in}{1.880547in}}%
\pgfpathcurveto{\pgfqpoint{1.142899in}{1.874724in}}{\pgfqpoint{1.150799in}{1.871451in}}{\pgfqpoint{1.159035in}{1.871451in}}%
\pgfpathclose%
\pgfusepath{stroke,fill}%
\end{pgfscope}%
\begin{pgfscope}%
\pgfpathrectangle{\pgfqpoint{0.100000in}{0.212622in}}{\pgfqpoint{3.696000in}{3.696000in}}%
\pgfusepath{clip}%
\pgfsetbuttcap%
\pgfsetroundjoin%
\definecolor{currentfill}{rgb}{0.121569,0.466667,0.705882}%
\pgfsetfillcolor{currentfill}%
\pgfsetfillopacity{0.743692}%
\pgfsetlinewidth{1.003750pt}%
\definecolor{currentstroke}{rgb}{0.121569,0.466667,0.705882}%
\pgfsetstrokecolor{currentstroke}%
\pgfsetstrokeopacity{0.743692}%
\pgfsetdash{}{0pt}%
\pgfpathmoveto{\pgfqpoint{2.411721in}{2.967894in}}%
\pgfpathcurveto{\pgfqpoint{2.419957in}{2.967894in}}{\pgfqpoint{2.427857in}{2.971166in}}{\pgfqpoint{2.433681in}{2.976990in}}%
\pgfpathcurveto{\pgfqpoint{2.439505in}{2.982814in}}{\pgfqpoint{2.442777in}{2.990714in}}{\pgfqpoint{2.442777in}{2.998951in}}%
\pgfpathcurveto{\pgfqpoint{2.442777in}{3.007187in}}{\pgfqpoint{2.439505in}{3.015087in}}{\pgfqpoint{2.433681in}{3.020911in}}%
\pgfpathcurveto{\pgfqpoint{2.427857in}{3.026735in}}{\pgfqpoint{2.419957in}{3.030007in}}{\pgfqpoint{2.411721in}{3.030007in}}%
\pgfpathcurveto{\pgfqpoint{2.403485in}{3.030007in}}{\pgfqpoint{2.395585in}{3.026735in}}{\pgfqpoint{2.389761in}{3.020911in}}%
\pgfpathcurveto{\pgfqpoint{2.383937in}{3.015087in}}{\pgfqpoint{2.380664in}{3.007187in}}{\pgfqpoint{2.380664in}{2.998951in}}%
\pgfpathcurveto{\pgfqpoint{2.380664in}{2.990714in}}{\pgfqpoint{2.383937in}{2.982814in}}{\pgfqpoint{2.389761in}{2.976990in}}%
\pgfpathcurveto{\pgfqpoint{2.395585in}{2.971166in}}{\pgfqpoint{2.403485in}{2.967894in}}{\pgfqpoint{2.411721in}{2.967894in}}%
\pgfpathclose%
\pgfusepath{stroke,fill}%
\end{pgfscope}%
\begin{pgfscope}%
\pgfpathrectangle{\pgfqpoint{0.100000in}{0.212622in}}{\pgfqpoint{3.696000in}{3.696000in}}%
\pgfusepath{clip}%
\pgfsetbuttcap%
\pgfsetroundjoin%
\definecolor{currentfill}{rgb}{0.121569,0.466667,0.705882}%
\pgfsetfillcolor{currentfill}%
\pgfsetfillopacity{0.743973}%
\pgfsetlinewidth{1.003750pt}%
\definecolor{currentstroke}{rgb}{0.121569,0.466667,0.705882}%
\pgfsetstrokecolor{currentstroke}%
\pgfsetstrokeopacity{0.743973}%
\pgfsetdash{}{0pt}%
\pgfpathmoveto{\pgfqpoint{1.157057in}{1.867394in}}%
\pgfpathcurveto{\pgfqpoint{1.165294in}{1.867394in}}{\pgfqpoint{1.173194in}{1.870666in}}{\pgfqpoint{1.179018in}{1.876490in}}%
\pgfpathcurveto{\pgfqpoint{1.184841in}{1.882314in}}{\pgfqpoint{1.188114in}{1.890214in}}{\pgfqpoint{1.188114in}{1.898451in}}%
\pgfpathcurveto{\pgfqpoint{1.188114in}{1.906687in}}{\pgfqpoint{1.184841in}{1.914587in}}{\pgfqpoint{1.179018in}{1.920411in}}%
\pgfpathcurveto{\pgfqpoint{1.173194in}{1.926235in}}{\pgfqpoint{1.165294in}{1.929507in}}{\pgfqpoint{1.157057in}{1.929507in}}%
\pgfpathcurveto{\pgfqpoint{1.148821in}{1.929507in}}{\pgfqpoint{1.140921in}{1.926235in}}{\pgfqpoint{1.135097in}{1.920411in}}%
\pgfpathcurveto{\pgfqpoint{1.129273in}{1.914587in}}{\pgfqpoint{1.126001in}{1.906687in}}{\pgfqpoint{1.126001in}{1.898451in}}%
\pgfpathcurveto{\pgfqpoint{1.126001in}{1.890214in}}{\pgfqpoint{1.129273in}{1.882314in}}{\pgfqpoint{1.135097in}{1.876490in}}%
\pgfpathcurveto{\pgfqpoint{1.140921in}{1.870666in}}{\pgfqpoint{1.148821in}{1.867394in}}{\pgfqpoint{1.157057in}{1.867394in}}%
\pgfpathclose%
\pgfusepath{stroke,fill}%
\end{pgfscope}%
\begin{pgfscope}%
\pgfpathrectangle{\pgfqpoint{0.100000in}{0.212622in}}{\pgfqpoint{3.696000in}{3.696000in}}%
\pgfusepath{clip}%
\pgfsetbuttcap%
\pgfsetroundjoin%
\definecolor{currentfill}{rgb}{0.121569,0.466667,0.705882}%
\pgfsetfillcolor{currentfill}%
\pgfsetfillopacity{0.744218}%
\pgfsetlinewidth{1.003750pt}%
\definecolor{currentstroke}{rgb}{0.121569,0.466667,0.705882}%
\pgfsetstrokecolor{currentstroke}%
\pgfsetstrokeopacity{0.744218}%
\pgfsetdash{}{0pt}%
\pgfpathmoveto{\pgfqpoint{3.190314in}{2.272133in}}%
\pgfpathcurveto{\pgfqpoint{3.198550in}{2.272133in}}{\pgfqpoint{3.206450in}{2.275406in}}{\pgfqpoint{3.212274in}{2.281230in}}%
\pgfpathcurveto{\pgfqpoint{3.218098in}{2.287054in}}{\pgfqpoint{3.221370in}{2.294954in}}{\pgfqpoint{3.221370in}{2.303190in}}%
\pgfpathcurveto{\pgfqpoint{3.221370in}{2.311426in}}{\pgfqpoint{3.218098in}{2.319326in}}{\pgfqpoint{3.212274in}{2.325150in}}%
\pgfpathcurveto{\pgfqpoint{3.206450in}{2.330974in}}{\pgfqpoint{3.198550in}{2.334246in}}{\pgfqpoint{3.190314in}{2.334246in}}%
\pgfpathcurveto{\pgfqpoint{3.182077in}{2.334246in}}{\pgfqpoint{3.174177in}{2.330974in}}{\pgfqpoint{3.168353in}{2.325150in}}%
\pgfpathcurveto{\pgfqpoint{3.162530in}{2.319326in}}{\pgfqpoint{3.159257in}{2.311426in}}{\pgfqpoint{3.159257in}{2.303190in}}%
\pgfpathcurveto{\pgfqpoint{3.159257in}{2.294954in}}{\pgfqpoint{3.162530in}{2.287054in}}{\pgfqpoint{3.168353in}{2.281230in}}%
\pgfpathcurveto{\pgfqpoint{3.174177in}{2.275406in}}{\pgfqpoint{3.182077in}{2.272133in}}{\pgfqpoint{3.190314in}{2.272133in}}%
\pgfpathclose%
\pgfusepath{stroke,fill}%
\end{pgfscope}%
\begin{pgfscope}%
\pgfpathrectangle{\pgfqpoint{0.100000in}{0.212622in}}{\pgfqpoint{3.696000in}{3.696000in}}%
\pgfusepath{clip}%
\pgfsetbuttcap%
\pgfsetroundjoin%
\definecolor{currentfill}{rgb}{0.121569,0.466667,0.705882}%
\pgfsetfillcolor{currentfill}%
\pgfsetfillopacity{0.744453}%
\pgfsetlinewidth{1.003750pt}%
\definecolor{currentstroke}{rgb}{0.121569,0.466667,0.705882}%
\pgfsetstrokecolor{currentstroke}%
\pgfsetstrokeopacity{0.744453}%
\pgfsetdash{}{0pt}%
\pgfpathmoveto{\pgfqpoint{1.154216in}{1.862856in}}%
\pgfpathcurveto{\pgfqpoint{1.162453in}{1.862856in}}{\pgfqpoint{1.170353in}{1.866128in}}{\pgfqpoint{1.176177in}{1.871952in}}%
\pgfpathcurveto{\pgfqpoint{1.182001in}{1.877776in}}{\pgfqpoint{1.185273in}{1.885676in}}{\pgfqpoint{1.185273in}{1.893913in}}%
\pgfpathcurveto{\pgfqpoint{1.185273in}{1.902149in}}{\pgfqpoint{1.182001in}{1.910049in}}{\pgfqpoint{1.176177in}{1.915873in}}%
\pgfpathcurveto{\pgfqpoint{1.170353in}{1.921697in}}{\pgfqpoint{1.162453in}{1.924969in}}{\pgfqpoint{1.154216in}{1.924969in}}%
\pgfpathcurveto{\pgfqpoint{1.145980in}{1.924969in}}{\pgfqpoint{1.138080in}{1.921697in}}{\pgfqpoint{1.132256in}{1.915873in}}%
\pgfpathcurveto{\pgfqpoint{1.126432in}{1.910049in}}{\pgfqpoint{1.123160in}{1.902149in}}{\pgfqpoint{1.123160in}{1.893913in}}%
\pgfpathcurveto{\pgfqpoint{1.123160in}{1.885676in}}{\pgfqpoint{1.126432in}{1.877776in}}{\pgfqpoint{1.132256in}{1.871952in}}%
\pgfpathcurveto{\pgfqpoint{1.138080in}{1.866128in}}{\pgfqpoint{1.145980in}{1.862856in}}{\pgfqpoint{1.154216in}{1.862856in}}%
\pgfpathclose%
\pgfusepath{stroke,fill}%
\end{pgfscope}%
\begin{pgfscope}%
\pgfpathrectangle{\pgfqpoint{0.100000in}{0.212622in}}{\pgfqpoint{3.696000in}{3.696000in}}%
\pgfusepath{clip}%
\pgfsetbuttcap%
\pgfsetroundjoin%
\definecolor{currentfill}{rgb}{0.121569,0.466667,0.705882}%
\pgfsetfillcolor{currentfill}%
\pgfsetfillopacity{0.744620}%
\pgfsetlinewidth{1.003750pt}%
\definecolor{currentstroke}{rgb}{0.121569,0.466667,0.705882}%
\pgfsetstrokecolor{currentstroke}%
\pgfsetstrokeopacity{0.744620}%
\pgfsetdash{}{0pt}%
\pgfpathmoveto{\pgfqpoint{2.415504in}{2.967056in}}%
\pgfpathcurveto{\pgfqpoint{2.423740in}{2.967056in}}{\pgfqpoint{2.431640in}{2.970328in}}{\pgfqpoint{2.437464in}{2.976152in}}%
\pgfpathcurveto{\pgfqpoint{2.443288in}{2.981976in}}{\pgfqpoint{2.446560in}{2.989876in}}{\pgfqpoint{2.446560in}{2.998112in}}%
\pgfpathcurveto{\pgfqpoint{2.446560in}{3.006349in}}{\pgfqpoint{2.443288in}{3.014249in}}{\pgfqpoint{2.437464in}{3.020073in}}%
\pgfpathcurveto{\pgfqpoint{2.431640in}{3.025897in}}{\pgfqpoint{2.423740in}{3.029169in}}{\pgfqpoint{2.415504in}{3.029169in}}%
\pgfpathcurveto{\pgfqpoint{2.407268in}{3.029169in}}{\pgfqpoint{2.399368in}{3.025897in}}{\pgfqpoint{2.393544in}{3.020073in}}%
\pgfpathcurveto{\pgfqpoint{2.387720in}{3.014249in}}{\pgfqpoint{2.384447in}{3.006349in}}{\pgfqpoint{2.384447in}{2.998112in}}%
\pgfpathcurveto{\pgfqpoint{2.384447in}{2.989876in}}{\pgfqpoint{2.387720in}{2.981976in}}{\pgfqpoint{2.393544in}{2.976152in}}%
\pgfpathcurveto{\pgfqpoint{2.399368in}{2.970328in}}{\pgfqpoint{2.407268in}{2.967056in}}{\pgfqpoint{2.415504in}{2.967056in}}%
\pgfpathclose%
\pgfusepath{stroke,fill}%
\end{pgfscope}%
\begin{pgfscope}%
\pgfpathrectangle{\pgfqpoint{0.100000in}{0.212622in}}{\pgfqpoint{3.696000in}{3.696000in}}%
\pgfusepath{clip}%
\pgfsetbuttcap%
\pgfsetroundjoin%
\definecolor{currentfill}{rgb}{0.121569,0.466667,0.705882}%
\pgfsetfillcolor{currentfill}%
\pgfsetfillopacity{0.745072}%
\pgfsetlinewidth{1.003750pt}%
\definecolor{currentstroke}{rgb}{0.121569,0.466667,0.705882}%
\pgfsetstrokecolor{currentstroke}%
\pgfsetstrokeopacity{0.745072}%
\pgfsetdash{}{0pt}%
\pgfpathmoveto{\pgfqpoint{1.150703in}{1.858303in}}%
\pgfpathcurveto{\pgfqpoint{1.158939in}{1.858303in}}{\pgfqpoint{1.166839in}{1.861575in}}{\pgfqpoint{1.172663in}{1.867399in}}%
\pgfpathcurveto{\pgfqpoint{1.178487in}{1.873223in}}{\pgfqpoint{1.181759in}{1.881123in}}{\pgfqpoint{1.181759in}{1.889359in}}%
\pgfpathcurveto{\pgfqpoint{1.181759in}{1.897595in}}{\pgfqpoint{1.178487in}{1.905496in}}{\pgfqpoint{1.172663in}{1.911319in}}%
\pgfpathcurveto{\pgfqpoint{1.166839in}{1.917143in}}{\pgfqpoint{1.158939in}{1.920416in}}{\pgfqpoint{1.150703in}{1.920416in}}%
\pgfpathcurveto{\pgfqpoint{1.142466in}{1.920416in}}{\pgfqpoint{1.134566in}{1.917143in}}{\pgfqpoint{1.128742in}{1.911319in}}%
\pgfpathcurveto{\pgfqpoint{1.122918in}{1.905496in}}{\pgfqpoint{1.119646in}{1.897595in}}{\pgfqpoint{1.119646in}{1.889359in}}%
\pgfpathcurveto{\pgfqpoint{1.119646in}{1.881123in}}{\pgfqpoint{1.122918in}{1.873223in}}{\pgfqpoint{1.128742in}{1.867399in}}%
\pgfpathcurveto{\pgfqpoint{1.134566in}{1.861575in}}{\pgfqpoint{1.142466in}{1.858303in}}{\pgfqpoint{1.150703in}{1.858303in}}%
\pgfpathclose%
\pgfusepath{stroke,fill}%
\end{pgfscope}%
\begin{pgfscope}%
\pgfpathrectangle{\pgfqpoint{0.100000in}{0.212622in}}{\pgfqpoint{3.696000in}{3.696000in}}%
\pgfusepath{clip}%
\pgfsetbuttcap%
\pgfsetroundjoin%
\definecolor{currentfill}{rgb}{0.121569,0.466667,0.705882}%
\pgfsetfillcolor{currentfill}%
\pgfsetfillopacity{0.745604}%
\pgfsetlinewidth{1.003750pt}%
\definecolor{currentstroke}{rgb}{0.121569,0.466667,0.705882}%
\pgfsetstrokecolor{currentstroke}%
\pgfsetstrokeopacity{0.745604}%
\pgfsetdash{}{0pt}%
\pgfpathmoveto{\pgfqpoint{3.187131in}{2.263118in}}%
\pgfpathcurveto{\pgfqpoint{3.195367in}{2.263118in}}{\pgfqpoint{3.203267in}{2.266390in}}{\pgfqpoint{3.209091in}{2.272214in}}%
\pgfpathcurveto{\pgfqpoint{3.214915in}{2.278038in}}{\pgfqpoint{3.218187in}{2.285938in}}{\pgfqpoint{3.218187in}{2.294174in}}%
\pgfpathcurveto{\pgfqpoint{3.218187in}{2.302411in}}{\pgfqpoint{3.214915in}{2.310311in}}{\pgfqpoint{3.209091in}{2.316135in}}%
\pgfpathcurveto{\pgfqpoint{3.203267in}{2.321958in}}{\pgfqpoint{3.195367in}{2.325231in}}{\pgfqpoint{3.187131in}{2.325231in}}%
\pgfpathcurveto{\pgfqpoint{3.178894in}{2.325231in}}{\pgfqpoint{3.170994in}{2.321958in}}{\pgfqpoint{3.165170in}{2.316135in}}%
\pgfpathcurveto{\pgfqpoint{3.159346in}{2.310311in}}{\pgfqpoint{3.156074in}{2.302411in}}{\pgfqpoint{3.156074in}{2.294174in}}%
\pgfpathcurveto{\pgfqpoint{3.156074in}{2.285938in}}{\pgfqpoint{3.159346in}{2.278038in}}{\pgfqpoint{3.165170in}{2.272214in}}%
\pgfpathcurveto{\pgfqpoint{3.170994in}{2.266390in}}{\pgfqpoint{3.178894in}{2.263118in}}{\pgfqpoint{3.187131in}{2.263118in}}%
\pgfpathclose%
\pgfusepath{stroke,fill}%
\end{pgfscope}%
\begin{pgfscope}%
\pgfpathrectangle{\pgfqpoint{0.100000in}{0.212622in}}{\pgfqpoint{3.696000in}{3.696000in}}%
\pgfusepath{clip}%
\pgfsetbuttcap%
\pgfsetroundjoin%
\definecolor{currentfill}{rgb}{0.121569,0.466667,0.705882}%
\pgfsetfillcolor{currentfill}%
\pgfsetfillopacity{0.745895}%
\pgfsetlinewidth{1.003750pt}%
\definecolor{currentstroke}{rgb}{0.121569,0.466667,0.705882}%
\pgfsetstrokecolor{currentstroke}%
\pgfsetstrokeopacity{0.745895}%
\pgfsetdash{}{0pt}%
\pgfpathmoveto{\pgfqpoint{1.146764in}{1.851203in}}%
\pgfpathcurveto{\pgfqpoint{1.155000in}{1.851203in}}{\pgfqpoint{1.162900in}{1.854475in}}{\pgfqpoint{1.168724in}{1.860299in}}%
\pgfpathcurveto{\pgfqpoint{1.174548in}{1.866123in}}{\pgfqpoint{1.177820in}{1.874023in}}{\pgfqpoint{1.177820in}{1.882259in}}%
\pgfpathcurveto{\pgfqpoint{1.177820in}{1.890495in}}{\pgfqpoint{1.174548in}{1.898395in}}{\pgfqpoint{1.168724in}{1.904219in}}%
\pgfpathcurveto{\pgfqpoint{1.162900in}{1.910043in}}{\pgfqpoint{1.155000in}{1.913316in}}{\pgfqpoint{1.146764in}{1.913316in}}%
\pgfpathcurveto{\pgfqpoint{1.138527in}{1.913316in}}{\pgfqpoint{1.130627in}{1.910043in}}{\pgfqpoint{1.124803in}{1.904219in}}%
\pgfpathcurveto{\pgfqpoint{1.118979in}{1.898395in}}{\pgfqpoint{1.115707in}{1.890495in}}{\pgfqpoint{1.115707in}{1.882259in}}%
\pgfpathcurveto{\pgfqpoint{1.115707in}{1.874023in}}{\pgfqpoint{1.118979in}{1.866123in}}{\pgfqpoint{1.124803in}{1.860299in}}%
\pgfpathcurveto{\pgfqpoint{1.130627in}{1.854475in}}{\pgfqpoint{1.138527in}{1.851203in}}{\pgfqpoint{1.146764in}{1.851203in}}%
\pgfpathclose%
\pgfusepath{stroke,fill}%
\end{pgfscope}%
\begin{pgfscope}%
\pgfpathrectangle{\pgfqpoint{0.100000in}{0.212622in}}{\pgfqpoint{3.696000in}{3.696000in}}%
\pgfusepath{clip}%
\pgfsetbuttcap%
\pgfsetroundjoin%
\definecolor{currentfill}{rgb}{0.121569,0.466667,0.705882}%
\pgfsetfillcolor{currentfill}%
\pgfsetfillopacity{0.746167}%
\pgfsetlinewidth{1.003750pt}%
\definecolor{currentstroke}{rgb}{0.121569,0.466667,0.705882}%
\pgfsetstrokecolor{currentstroke}%
\pgfsetstrokeopacity{0.746167}%
\pgfsetdash{}{0pt}%
\pgfpathmoveto{\pgfqpoint{2.422566in}{2.965499in}}%
\pgfpathcurveto{\pgfqpoint{2.430802in}{2.965499in}}{\pgfqpoint{2.438703in}{2.968771in}}{\pgfqpoint{2.444526in}{2.974595in}}%
\pgfpathcurveto{\pgfqpoint{2.450350in}{2.980419in}}{\pgfqpoint{2.453623in}{2.988319in}}{\pgfqpoint{2.453623in}{2.996555in}}%
\pgfpathcurveto{\pgfqpoint{2.453623in}{3.004792in}}{\pgfqpoint{2.450350in}{3.012692in}}{\pgfqpoint{2.444526in}{3.018516in}}%
\pgfpathcurveto{\pgfqpoint{2.438703in}{3.024340in}}{\pgfqpoint{2.430802in}{3.027612in}}{\pgfqpoint{2.422566in}{3.027612in}}%
\pgfpathcurveto{\pgfqpoint{2.414330in}{3.027612in}}{\pgfqpoint{2.406430in}{3.024340in}}{\pgfqpoint{2.400606in}{3.018516in}}%
\pgfpathcurveto{\pgfqpoint{2.394782in}{3.012692in}}{\pgfqpoint{2.391510in}{3.004792in}}{\pgfqpoint{2.391510in}{2.996555in}}%
\pgfpathcurveto{\pgfqpoint{2.391510in}{2.988319in}}{\pgfqpoint{2.394782in}{2.980419in}}{\pgfqpoint{2.400606in}{2.974595in}}%
\pgfpathcurveto{\pgfqpoint{2.406430in}{2.968771in}}{\pgfqpoint{2.414330in}{2.965499in}}{\pgfqpoint{2.422566in}{2.965499in}}%
\pgfpathclose%
\pgfusepath{stroke,fill}%
\end{pgfscope}%
\begin{pgfscope}%
\pgfpathrectangle{\pgfqpoint{0.100000in}{0.212622in}}{\pgfqpoint{3.696000in}{3.696000in}}%
\pgfusepath{clip}%
\pgfsetbuttcap%
\pgfsetroundjoin%
\definecolor{currentfill}{rgb}{0.121569,0.466667,0.705882}%
\pgfsetfillcolor{currentfill}%
\pgfsetfillopacity{0.746694}%
\pgfsetlinewidth{1.003750pt}%
\definecolor{currentstroke}{rgb}{0.121569,0.466667,0.705882}%
\pgfsetstrokecolor{currentstroke}%
\pgfsetstrokeopacity{0.746694}%
\pgfsetdash{}{0pt}%
\pgfpathmoveto{\pgfqpoint{3.186344in}{2.254383in}}%
\pgfpathcurveto{\pgfqpoint{3.194580in}{2.254383in}}{\pgfqpoint{3.202480in}{2.257655in}}{\pgfqpoint{3.208304in}{2.263479in}}%
\pgfpathcurveto{\pgfqpoint{3.214128in}{2.269303in}}{\pgfqpoint{3.217400in}{2.277203in}}{\pgfqpoint{3.217400in}{2.285439in}}%
\pgfpathcurveto{\pgfqpoint{3.217400in}{2.293675in}}{\pgfqpoint{3.214128in}{2.301575in}}{\pgfqpoint{3.208304in}{2.307399in}}%
\pgfpathcurveto{\pgfqpoint{3.202480in}{2.313223in}}{\pgfqpoint{3.194580in}{2.316496in}}{\pgfqpoint{3.186344in}{2.316496in}}%
\pgfpathcurveto{\pgfqpoint{3.178107in}{2.316496in}}{\pgfqpoint{3.170207in}{2.313223in}}{\pgfqpoint{3.164383in}{2.307399in}}%
\pgfpathcurveto{\pgfqpoint{3.158559in}{2.301575in}}{\pgfqpoint{3.155287in}{2.293675in}}{\pgfqpoint{3.155287in}{2.285439in}}%
\pgfpathcurveto{\pgfqpoint{3.155287in}{2.277203in}}{\pgfqpoint{3.158559in}{2.269303in}}{\pgfqpoint{3.164383in}{2.263479in}}%
\pgfpathcurveto{\pgfqpoint{3.170207in}{2.257655in}}{\pgfqpoint{3.178107in}{2.254383in}}{\pgfqpoint{3.186344in}{2.254383in}}%
\pgfpathclose%
\pgfusepath{stroke,fill}%
\end{pgfscope}%
\begin{pgfscope}%
\pgfpathrectangle{\pgfqpoint{0.100000in}{0.212622in}}{\pgfqpoint{3.696000in}{3.696000in}}%
\pgfusepath{clip}%
\pgfsetbuttcap%
\pgfsetroundjoin%
\definecolor{currentfill}{rgb}{0.121569,0.466667,0.705882}%
\pgfsetfillcolor{currentfill}%
\pgfsetfillopacity{0.746937}%
\pgfsetlinewidth{1.003750pt}%
\definecolor{currentstroke}{rgb}{0.121569,0.466667,0.705882}%
\pgfsetstrokecolor{currentstroke}%
\pgfsetstrokeopacity{0.746937}%
\pgfsetdash{}{0pt}%
\pgfpathmoveto{\pgfqpoint{1.143224in}{1.841157in}}%
\pgfpathcurveto{\pgfqpoint{1.151461in}{1.841157in}}{\pgfqpoint{1.159361in}{1.844430in}}{\pgfqpoint{1.165185in}{1.850253in}}%
\pgfpathcurveto{\pgfqpoint{1.171009in}{1.856077in}}{\pgfqpoint{1.174281in}{1.863977in}}{\pgfqpoint{1.174281in}{1.872214in}}%
\pgfpathcurveto{\pgfqpoint{1.174281in}{1.880450in}}{\pgfqpoint{1.171009in}{1.888350in}}{\pgfqpoint{1.165185in}{1.894174in}}%
\pgfpathcurveto{\pgfqpoint{1.159361in}{1.899998in}}{\pgfqpoint{1.151461in}{1.903270in}}{\pgfqpoint{1.143224in}{1.903270in}}%
\pgfpathcurveto{\pgfqpoint{1.134988in}{1.903270in}}{\pgfqpoint{1.127088in}{1.899998in}}{\pgfqpoint{1.121264in}{1.894174in}}%
\pgfpathcurveto{\pgfqpoint{1.115440in}{1.888350in}}{\pgfqpoint{1.112168in}{1.880450in}}{\pgfqpoint{1.112168in}{1.872214in}}%
\pgfpathcurveto{\pgfqpoint{1.112168in}{1.863977in}}{\pgfqpoint{1.115440in}{1.856077in}}{\pgfqpoint{1.121264in}{1.850253in}}%
\pgfpathcurveto{\pgfqpoint{1.127088in}{1.844430in}}{\pgfqpoint{1.134988in}{1.841157in}}{\pgfqpoint{1.143224in}{1.841157in}}%
\pgfpathclose%
\pgfusepath{stroke,fill}%
\end{pgfscope}%
\begin{pgfscope}%
\pgfpathrectangle{\pgfqpoint{0.100000in}{0.212622in}}{\pgfqpoint{3.696000in}{3.696000in}}%
\pgfusepath{clip}%
\pgfsetbuttcap%
\pgfsetroundjoin%
\definecolor{currentfill}{rgb}{0.121569,0.466667,0.705882}%
\pgfsetfillcolor{currentfill}%
\pgfsetfillopacity{0.747510}%
\pgfsetlinewidth{1.003750pt}%
\definecolor{currentstroke}{rgb}{0.121569,0.466667,0.705882}%
\pgfsetstrokecolor{currentstroke}%
\pgfsetstrokeopacity{0.747510}%
\pgfsetdash{}{0pt}%
\pgfpathmoveto{\pgfqpoint{2.428299in}{2.965199in}}%
\pgfpathcurveto{\pgfqpoint{2.436535in}{2.965199in}}{\pgfqpoint{2.444435in}{2.968472in}}{\pgfqpoint{2.450259in}{2.974296in}}%
\pgfpathcurveto{\pgfqpoint{2.456083in}{2.980119in}}{\pgfqpoint{2.459356in}{2.988019in}}{\pgfqpoint{2.459356in}{2.996256in}}%
\pgfpathcurveto{\pgfqpoint{2.459356in}{3.004492in}}{\pgfqpoint{2.456083in}{3.012392in}}{\pgfqpoint{2.450259in}{3.018216in}}%
\pgfpathcurveto{\pgfqpoint{2.444435in}{3.024040in}}{\pgfqpoint{2.436535in}{3.027312in}}{\pgfqpoint{2.428299in}{3.027312in}}%
\pgfpathcurveto{\pgfqpoint{2.420063in}{3.027312in}}{\pgfqpoint{2.412163in}{3.024040in}}{\pgfqpoint{2.406339in}{3.018216in}}%
\pgfpathcurveto{\pgfqpoint{2.400515in}{3.012392in}}{\pgfqpoint{2.397243in}{3.004492in}}{\pgfqpoint{2.397243in}{2.996256in}}%
\pgfpathcurveto{\pgfqpoint{2.397243in}{2.988019in}}{\pgfqpoint{2.400515in}{2.980119in}}{\pgfqpoint{2.406339in}{2.974296in}}%
\pgfpathcurveto{\pgfqpoint{2.412163in}{2.968472in}}{\pgfqpoint{2.420063in}{2.965199in}}{\pgfqpoint{2.428299in}{2.965199in}}%
\pgfpathclose%
\pgfusepath{stroke,fill}%
\end{pgfscope}%
\begin{pgfscope}%
\pgfpathrectangle{\pgfqpoint{0.100000in}{0.212622in}}{\pgfqpoint{3.696000in}{3.696000in}}%
\pgfusepath{clip}%
\pgfsetbuttcap%
\pgfsetroundjoin%
\definecolor{currentfill}{rgb}{0.121569,0.466667,0.705882}%
\pgfsetfillcolor{currentfill}%
\pgfsetfillopacity{0.747947}%
\pgfsetlinewidth{1.003750pt}%
\definecolor{currentstroke}{rgb}{0.121569,0.466667,0.705882}%
\pgfsetstrokecolor{currentstroke}%
\pgfsetstrokeopacity{0.747947}%
\pgfsetdash{}{0pt}%
\pgfpathmoveto{\pgfqpoint{1.140074in}{1.829797in}}%
\pgfpathcurveto{\pgfqpoint{1.148310in}{1.829797in}}{\pgfqpoint{1.156210in}{1.833069in}}{\pgfqpoint{1.162034in}{1.838893in}}%
\pgfpathcurveto{\pgfqpoint{1.167858in}{1.844717in}}{\pgfqpoint{1.171131in}{1.852617in}}{\pgfqpoint{1.171131in}{1.860854in}}%
\pgfpathcurveto{\pgfqpoint{1.171131in}{1.869090in}}{\pgfqpoint{1.167858in}{1.876990in}}{\pgfqpoint{1.162034in}{1.882814in}}%
\pgfpathcurveto{\pgfqpoint{1.156210in}{1.888638in}}{\pgfqpoint{1.148310in}{1.891910in}}{\pgfqpoint{1.140074in}{1.891910in}}%
\pgfpathcurveto{\pgfqpoint{1.131838in}{1.891910in}}{\pgfqpoint{1.123938in}{1.888638in}}{\pgfqpoint{1.118114in}{1.882814in}}%
\pgfpathcurveto{\pgfqpoint{1.112290in}{1.876990in}}{\pgfqpoint{1.109018in}{1.869090in}}{\pgfqpoint{1.109018in}{1.860854in}}%
\pgfpathcurveto{\pgfqpoint{1.109018in}{1.852617in}}{\pgfqpoint{1.112290in}{1.844717in}}{\pgfqpoint{1.118114in}{1.838893in}}%
\pgfpathcurveto{\pgfqpoint{1.123938in}{1.833069in}}{\pgfqpoint{1.131838in}{1.829797in}}{\pgfqpoint{1.140074in}{1.829797in}}%
\pgfpathclose%
\pgfusepath{stroke,fill}%
\end{pgfscope}%
\begin{pgfscope}%
\pgfpathrectangle{\pgfqpoint{0.100000in}{0.212622in}}{\pgfqpoint{3.696000in}{3.696000in}}%
\pgfusepath{clip}%
\pgfsetbuttcap%
\pgfsetroundjoin%
\definecolor{currentfill}{rgb}{0.121569,0.466667,0.705882}%
\pgfsetfillcolor{currentfill}%
\pgfsetfillopacity{0.748755}%
\pgfsetlinewidth{1.003750pt}%
\definecolor{currentstroke}{rgb}{0.121569,0.466667,0.705882}%
\pgfsetstrokecolor{currentstroke}%
\pgfsetstrokeopacity{0.748755}%
\pgfsetdash{}{0pt}%
\pgfpathmoveto{\pgfqpoint{2.433777in}{2.964733in}}%
\pgfpathcurveto{\pgfqpoint{2.442014in}{2.964733in}}{\pgfqpoint{2.449914in}{2.968006in}}{\pgfqpoint{2.455738in}{2.973830in}}%
\pgfpathcurveto{\pgfqpoint{2.461562in}{2.979654in}}{\pgfqpoint{2.464834in}{2.987554in}}{\pgfqpoint{2.464834in}{2.995790in}}%
\pgfpathcurveto{\pgfqpoint{2.464834in}{3.004026in}}{\pgfqpoint{2.461562in}{3.011926in}}{\pgfqpoint{2.455738in}{3.017750in}}%
\pgfpathcurveto{\pgfqpoint{2.449914in}{3.023574in}}{\pgfqpoint{2.442014in}{3.026846in}}{\pgfqpoint{2.433777in}{3.026846in}}%
\pgfpathcurveto{\pgfqpoint{2.425541in}{3.026846in}}{\pgfqpoint{2.417641in}{3.023574in}}{\pgfqpoint{2.411817in}{3.017750in}}%
\pgfpathcurveto{\pgfqpoint{2.405993in}{3.011926in}}{\pgfqpoint{2.402721in}{3.004026in}}{\pgfqpoint{2.402721in}{2.995790in}}%
\pgfpathcurveto{\pgfqpoint{2.402721in}{2.987554in}}{\pgfqpoint{2.405993in}{2.979654in}}{\pgfqpoint{2.411817in}{2.973830in}}%
\pgfpathcurveto{\pgfqpoint{2.417641in}{2.968006in}}{\pgfqpoint{2.425541in}{2.964733in}}{\pgfqpoint{2.433777in}{2.964733in}}%
\pgfpathclose%
\pgfusepath{stroke,fill}%
\end{pgfscope}%
\begin{pgfscope}%
\pgfpathrectangle{\pgfqpoint{0.100000in}{0.212622in}}{\pgfqpoint{3.696000in}{3.696000in}}%
\pgfusepath{clip}%
\pgfsetbuttcap%
\pgfsetroundjoin%
\definecolor{currentfill}{rgb}{0.121569,0.466667,0.705882}%
\pgfsetfillcolor{currentfill}%
\pgfsetfillopacity{0.748892}%
\pgfsetlinewidth{1.003750pt}%
\definecolor{currentstroke}{rgb}{0.121569,0.466667,0.705882}%
\pgfsetstrokecolor{currentstroke}%
\pgfsetstrokeopacity{0.748892}%
\pgfsetdash{}{0pt}%
\pgfpathmoveto{\pgfqpoint{3.182489in}{2.239488in}}%
\pgfpathcurveto{\pgfqpoint{3.190725in}{2.239488in}}{\pgfqpoint{3.198625in}{2.242760in}}{\pgfqpoint{3.204449in}{2.248584in}}%
\pgfpathcurveto{\pgfqpoint{3.210273in}{2.254408in}}{\pgfqpoint{3.213546in}{2.262308in}}{\pgfqpoint{3.213546in}{2.270544in}}%
\pgfpathcurveto{\pgfqpoint{3.213546in}{2.278781in}}{\pgfqpoint{3.210273in}{2.286681in}}{\pgfqpoint{3.204449in}{2.292504in}}%
\pgfpathcurveto{\pgfqpoint{3.198625in}{2.298328in}}{\pgfqpoint{3.190725in}{2.301601in}}{\pgfqpoint{3.182489in}{2.301601in}}%
\pgfpathcurveto{\pgfqpoint{3.174253in}{2.301601in}}{\pgfqpoint{3.166353in}{2.298328in}}{\pgfqpoint{3.160529in}{2.292504in}}%
\pgfpathcurveto{\pgfqpoint{3.154705in}{2.286681in}}{\pgfqpoint{3.151433in}{2.278781in}}{\pgfqpoint{3.151433in}{2.270544in}}%
\pgfpathcurveto{\pgfqpoint{3.151433in}{2.262308in}}{\pgfqpoint{3.154705in}{2.254408in}}{\pgfqpoint{3.160529in}{2.248584in}}%
\pgfpathcurveto{\pgfqpoint{3.166353in}{2.242760in}}{\pgfqpoint{3.174253in}{2.239488in}}{\pgfqpoint{3.182489in}{2.239488in}}%
\pgfpathclose%
\pgfusepath{stroke,fill}%
\end{pgfscope}%
\begin{pgfscope}%
\pgfpathrectangle{\pgfqpoint{0.100000in}{0.212622in}}{\pgfqpoint{3.696000in}{3.696000in}}%
\pgfusepath{clip}%
\pgfsetbuttcap%
\pgfsetroundjoin%
\definecolor{currentfill}{rgb}{0.121569,0.466667,0.705882}%
\pgfsetfillcolor{currentfill}%
\pgfsetfillopacity{0.749119}%
\pgfsetlinewidth{1.003750pt}%
\definecolor{currentstroke}{rgb}{0.121569,0.466667,0.705882}%
\pgfsetstrokecolor{currentstroke}%
\pgfsetstrokeopacity{0.749119}%
\pgfsetdash{}{0pt}%
\pgfpathmoveto{\pgfqpoint{1.134758in}{1.818767in}}%
\pgfpathcurveto{\pgfqpoint{1.142994in}{1.818767in}}{\pgfqpoint{1.150894in}{1.822040in}}{\pgfqpoint{1.156718in}{1.827864in}}%
\pgfpathcurveto{\pgfqpoint{1.162542in}{1.833688in}}{\pgfqpoint{1.165814in}{1.841588in}}{\pgfqpoint{1.165814in}{1.849824in}}%
\pgfpathcurveto{\pgfqpoint{1.165814in}{1.858060in}}{\pgfqpoint{1.162542in}{1.865960in}}{\pgfqpoint{1.156718in}{1.871784in}}%
\pgfpathcurveto{\pgfqpoint{1.150894in}{1.877608in}}{\pgfqpoint{1.142994in}{1.880880in}}{\pgfqpoint{1.134758in}{1.880880in}}%
\pgfpathcurveto{\pgfqpoint{1.126521in}{1.880880in}}{\pgfqpoint{1.118621in}{1.877608in}}{\pgfqpoint{1.112797in}{1.871784in}}%
\pgfpathcurveto{\pgfqpoint{1.106974in}{1.865960in}}{\pgfqpoint{1.103701in}{1.858060in}}{\pgfqpoint{1.103701in}{1.849824in}}%
\pgfpathcurveto{\pgfqpoint{1.103701in}{1.841588in}}{\pgfqpoint{1.106974in}{1.833688in}}{\pgfqpoint{1.112797in}{1.827864in}}%
\pgfpathcurveto{\pgfqpoint{1.118621in}{1.822040in}}{\pgfqpoint{1.126521in}{1.818767in}}{\pgfqpoint{1.134758in}{1.818767in}}%
\pgfpathclose%
\pgfusepath{stroke,fill}%
\end{pgfscope}%
\begin{pgfscope}%
\pgfpathrectangle{\pgfqpoint{0.100000in}{0.212622in}}{\pgfqpoint{3.696000in}{3.696000in}}%
\pgfusepath{clip}%
\pgfsetbuttcap%
\pgfsetroundjoin%
\definecolor{currentfill}{rgb}{0.121569,0.466667,0.705882}%
\pgfsetfillcolor{currentfill}%
\pgfsetfillopacity{0.749624}%
\pgfsetlinewidth{1.003750pt}%
\definecolor{currentstroke}{rgb}{0.121569,0.466667,0.705882}%
\pgfsetstrokecolor{currentstroke}%
\pgfsetstrokeopacity{0.749624}%
\pgfsetdash{}{0pt}%
\pgfpathmoveto{\pgfqpoint{2.438407in}{2.964058in}}%
\pgfpathcurveto{\pgfqpoint{2.446643in}{2.964058in}}{\pgfqpoint{2.454543in}{2.967330in}}{\pgfqpoint{2.460367in}{2.973154in}}%
\pgfpathcurveto{\pgfqpoint{2.466191in}{2.978978in}}{\pgfqpoint{2.469463in}{2.986878in}}{\pgfqpoint{2.469463in}{2.995115in}}%
\pgfpathcurveto{\pgfqpoint{2.469463in}{3.003351in}}{\pgfqpoint{2.466191in}{3.011251in}}{\pgfqpoint{2.460367in}{3.017075in}}%
\pgfpathcurveto{\pgfqpoint{2.454543in}{3.022899in}}{\pgfqpoint{2.446643in}{3.026171in}}{\pgfqpoint{2.438407in}{3.026171in}}%
\pgfpathcurveto{\pgfqpoint{2.430171in}{3.026171in}}{\pgfqpoint{2.422271in}{3.022899in}}{\pgfqpoint{2.416447in}{3.017075in}}%
\pgfpathcurveto{\pgfqpoint{2.410623in}{3.011251in}}{\pgfqpoint{2.407350in}{3.003351in}}{\pgfqpoint{2.407350in}{2.995115in}}%
\pgfpathcurveto{\pgfqpoint{2.407350in}{2.986878in}}{\pgfqpoint{2.410623in}{2.978978in}}{\pgfqpoint{2.416447in}{2.973154in}}%
\pgfpathcurveto{\pgfqpoint{2.422271in}{2.967330in}}{\pgfqpoint{2.430171in}{2.964058in}}{\pgfqpoint{2.438407in}{2.964058in}}%
\pgfpathclose%
\pgfusepath{stroke,fill}%
\end{pgfscope}%
\begin{pgfscope}%
\pgfpathrectangle{\pgfqpoint{0.100000in}{0.212622in}}{\pgfqpoint{3.696000in}{3.696000in}}%
\pgfusepath{clip}%
\pgfsetbuttcap%
\pgfsetroundjoin%
\definecolor{currentfill}{rgb}{0.121569,0.466667,0.705882}%
\pgfsetfillcolor{currentfill}%
\pgfsetfillopacity{0.750285}%
\pgfsetlinewidth{1.003750pt}%
\definecolor{currentstroke}{rgb}{0.121569,0.466667,0.705882}%
\pgfsetstrokecolor{currentstroke}%
\pgfsetstrokeopacity{0.750285}%
\pgfsetdash{}{0pt}%
\pgfpathmoveto{\pgfqpoint{1.128426in}{1.808425in}}%
\pgfpathcurveto{\pgfqpoint{1.136662in}{1.808425in}}{\pgfqpoint{1.144562in}{1.811697in}}{\pgfqpoint{1.150386in}{1.817521in}}%
\pgfpathcurveto{\pgfqpoint{1.156210in}{1.823345in}}{\pgfqpoint{1.159482in}{1.831245in}}{\pgfqpoint{1.159482in}{1.839481in}}%
\pgfpathcurveto{\pgfqpoint{1.159482in}{1.847718in}}{\pgfqpoint{1.156210in}{1.855618in}}{\pgfqpoint{1.150386in}{1.861442in}}%
\pgfpathcurveto{\pgfqpoint{1.144562in}{1.867265in}}{\pgfqpoint{1.136662in}{1.870538in}}{\pgfqpoint{1.128426in}{1.870538in}}%
\pgfpathcurveto{\pgfqpoint{1.120190in}{1.870538in}}{\pgfqpoint{1.112290in}{1.867265in}}{\pgfqpoint{1.106466in}{1.861442in}}%
\pgfpathcurveto{\pgfqpoint{1.100642in}{1.855618in}}{\pgfqpoint{1.097369in}{1.847718in}}{\pgfqpoint{1.097369in}{1.839481in}}%
\pgfpathcurveto{\pgfqpoint{1.097369in}{1.831245in}}{\pgfqpoint{1.100642in}{1.823345in}}{\pgfqpoint{1.106466in}{1.817521in}}%
\pgfpathcurveto{\pgfqpoint{1.112290in}{1.811697in}}{\pgfqpoint{1.120190in}{1.808425in}}{\pgfqpoint{1.128426in}{1.808425in}}%
\pgfpathclose%
\pgfusepath{stroke,fill}%
\end{pgfscope}%
\begin{pgfscope}%
\pgfpathrectangle{\pgfqpoint{0.100000in}{0.212622in}}{\pgfqpoint{3.696000in}{3.696000in}}%
\pgfusepath{clip}%
\pgfsetbuttcap%
\pgfsetroundjoin%
\definecolor{currentfill}{rgb}{0.121569,0.466667,0.705882}%
\pgfsetfillcolor{currentfill}%
\pgfsetfillopacity{0.750458}%
\pgfsetlinewidth{1.003750pt}%
\definecolor{currentstroke}{rgb}{0.121569,0.466667,0.705882}%
\pgfsetstrokecolor{currentstroke}%
\pgfsetstrokeopacity{0.750458}%
\pgfsetdash{}{0pt}%
\pgfpathmoveto{\pgfqpoint{2.442396in}{2.962719in}}%
\pgfpathcurveto{\pgfqpoint{2.450632in}{2.962719in}}{\pgfqpoint{2.458532in}{2.965992in}}{\pgfqpoint{2.464356in}{2.971816in}}%
\pgfpathcurveto{\pgfqpoint{2.470180in}{2.977640in}}{\pgfqpoint{2.473452in}{2.985540in}}{\pgfqpoint{2.473452in}{2.993776in}}%
\pgfpathcurveto{\pgfqpoint{2.473452in}{3.002012in}}{\pgfqpoint{2.470180in}{3.009912in}}{\pgfqpoint{2.464356in}{3.015736in}}%
\pgfpathcurveto{\pgfqpoint{2.458532in}{3.021560in}}{\pgfqpoint{2.450632in}{3.024832in}}{\pgfqpoint{2.442396in}{3.024832in}}%
\pgfpathcurveto{\pgfqpoint{2.434159in}{3.024832in}}{\pgfqpoint{2.426259in}{3.021560in}}{\pgfqpoint{2.420435in}{3.015736in}}%
\pgfpathcurveto{\pgfqpoint{2.414611in}{3.009912in}}{\pgfqpoint{2.411339in}{3.002012in}}{\pgfqpoint{2.411339in}{2.993776in}}%
\pgfpathcurveto{\pgfqpoint{2.411339in}{2.985540in}}{\pgfqpoint{2.414611in}{2.977640in}}{\pgfqpoint{2.420435in}{2.971816in}}%
\pgfpathcurveto{\pgfqpoint{2.426259in}{2.965992in}}{\pgfqpoint{2.434159in}{2.962719in}}{\pgfqpoint{2.442396in}{2.962719in}}%
\pgfpathclose%
\pgfusepath{stroke,fill}%
\end{pgfscope}%
\begin{pgfscope}%
\pgfpathrectangle{\pgfqpoint{0.100000in}{0.212622in}}{\pgfqpoint{3.696000in}{3.696000in}}%
\pgfusepath{clip}%
\pgfsetbuttcap%
\pgfsetroundjoin%
\definecolor{currentfill}{rgb}{0.121569,0.466667,0.705882}%
\pgfsetfillcolor{currentfill}%
\pgfsetfillopacity{0.750611}%
\pgfsetlinewidth{1.003750pt}%
\definecolor{currentstroke}{rgb}{0.121569,0.466667,0.705882}%
\pgfsetstrokecolor{currentstroke}%
\pgfsetstrokeopacity{0.750611}%
\pgfsetdash{}{0pt}%
\pgfpathmoveto{\pgfqpoint{3.176190in}{2.228209in}}%
\pgfpathcurveto{\pgfqpoint{3.184427in}{2.228209in}}{\pgfqpoint{3.192327in}{2.231481in}}{\pgfqpoint{3.198150in}{2.237305in}}%
\pgfpathcurveto{\pgfqpoint{3.203974in}{2.243129in}}{\pgfqpoint{3.207247in}{2.251029in}}{\pgfqpoint{3.207247in}{2.259266in}}%
\pgfpathcurveto{\pgfqpoint{3.207247in}{2.267502in}}{\pgfqpoint{3.203974in}{2.275402in}}{\pgfqpoint{3.198150in}{2.281226in}}%
\pgfpathcurveto{\pgfqpoint{3.192327in}{2.287050in}}{\pgfqpoint{3.184427in}{2.290322in}}{\pgfqpoint{3.176190in}{2.290322in}}%
\pgfpathcurveto{\pgfqpoint{3.167954in}{2.290322in}}{\pgfqpoint{3.160054in}{2.287050in}}{\pgfqpoint{3.154230in}{2.281226in}}%
\pgfpathcurveto{\pgfqpoint{3.148406in}{2.275402in}}{\pgfqpoint{3.145134in}{2.267502in}}{\pgfqpoint{3.145134in}{2.259266in}}%
\pgfpathcurveto{\pgfqpoint{3.145134in}{2.251029in}}{\pgfqpoint{3.148406in}{2.243129in}}{\pgfqpoint{3.154230in}{2.237305in}}%
\pgfpathcurveto{\pgfqpoint{3.160054in}{2.231481in}}{\pgfqpoint{3.167954in}{2.228209in}}{\pgfqpoint{3.176190in}{2.228209in}}%
\pgfpathclose%
\pgfusepath{stroke,fill}%
\end{pgfscope}%
\begin{pgfscope}%
\pgfpathrectangle{\pgfqpoint{0.100000in}{0.212622in}}{\pgfqpoint{3.696000in}{3.696000in}}%
\pgfusepath{clip}%
\pgfsetbuttcap%
\pgfsetroundjoin%
\definecolor{currentfill}{rgb}{0.121569,0.466667,0.705882}%
\pgfsetfillcolor{currentfill}%
\pgfsetfillopacity{0.751218}%
\pgfsetlinewidth{1.003750pt}%
\definecolor{currentstroke}{rgb}{0.121569,0.466667,0.705882}%
\pgfsetstrokecolor{currentstroke}%
\pgfsetstrokeopacity{0.751218}%
\pgfsetdash{}{0pt}%
\pgfpathmoveto{\pgfqpoint{2.445791in}{2.962282in}}%
\pgfpathcurveto{\pgfqpoint{2.454027in}{2.962282in}}{\pgfqpoint{2.461927in}{2.965555in}}{\pgfqpoint{2.467751in}{2.971379in}}%
\pgfpathcurveto{\pgfqpoint{2.473575in}{2.977203in}}{\pgfqpoint{2.476847in}{2.985103in}}{\pgfqpoint{2.476847in}{2.993339in}}%
\pgfpathcurveto{\pgfqpoint{2.476847in}{3.001575in}}{\pgfqpoint{2.473575in}{3.009475in}}{\pgfqpoint{2.467751in}{3.015299in}}%
\pgfpathcurveto{\pgfqpoint{2.461927in}{3.021123in}}{\pgfqpoint{2.454027in}{3.024395in}}{\pgfqpoint{2.445791in}{3.024395in}}%
\pgfpathcurveto{\pgfqpoint{2.437554in}{3.024395in}}{\pgfqpoint{2.429654in}{3.021123in}}{\pgfqpoint{2.423830in}{3.015299in}}%
\pgfpathcurveto{\pgfqpoint{2.418006in}{3.009475in}}{\pgfqpoint{2.414734in}{3.001575in}}{\pgfqpoint{2.414734in}{2.993339in}}%
\pgfpathcurveto{\pgfqpoint{2.414734in}{2.985103in}}{\pgfqpoint{2.418006in}{2.977203in}}{\pgfqpoint{2.423830in}{2.971379in}}%
\pgfpathcurveto{\pgfqpoint{2.429654in}{2.965555in}}{\pgfqpoint{2.437554in}{2.962282in}}{\pgfqpoint{2.445791in}{2.962282in}}%
\pgfpathclose%
\pgfusepath{stroke,fill}%
\end{pgfscope}%
\begin{pgfscope}%
\pgfpathrectangle{\pgfqpoint{0.100000in}{0.212622in}}{\pgfqpoint{3.696000in}{3.696000in}}%
\pgfusepath{clip}%
\pgfsetbuttcap%
\pgfsetroundjoin%
\definecolor{currentfill}{rgb}{0.121569,0.466667,0.705882}%
\pgfsetfillcolor{currentfill}%
\pgfsetfillopacity{0.751760}%
\pgfsetlinewidth{1.003750pt}%
\definecolor{currentstroke}{rgb}{0.121569,0.466667,0.705882}%
\pgfsetstrokecolor{currentstroke}%
\pgfsetstrokeopacity{0.751760}%
\pgfsetdash{}{0pt}%
\pgfpathmoveto{\pgfqpoint{1.121362in}{1.798834in}}%
\pgfpathcurveto{\pgfqpoint{1.129599in}{1.798834in}}{\pgfqpoint{1.137499in}{1.802106in}}{\pgfqpoint{1.143323in}{1.807930in}}%
\pgfpathcurveto{\pgfqpoint{1.149147in}{1.813754in}}{\pgfqpoint{1.152419in}{1.821654in}}{\pgfqpoint{1.152419in}{1.829891in}}%
\pgfpathcurveto{\pgfqpoint{1.152419in}{1.838127in}}{\pgfqpoint{1.149147in}{1.846027in}}{\pgfqpoint{1.143323in}{1.851851in}}%
\pgfpathcurveto{\pgfqpoint{1.137499in}{1.857675in}}{\pgfqpoint{1.129599in}{1.860947in}}{\pgfqpoint{1.121362in}{1.860947in}}%
\pgfpathcurveto{\pgfqpoint{1.113126in}{1.860947in}}{\pgfqpoint{1.105226in}{1.857675in}}{\pgfqpoint{1.099402in}{1.851851in}}%
\pgfpathcurveto{\pgfqpoint{1.093578in}{1.846027in}}{\pgfqpoint{1.090306in}{1.838127in}}{\pgfqpoint{1.090306in}{1.829891in}}%
\pgfpathcurveto{\pgfqpoint{1.090306in}{1.821654in}}{\pgfqpoint{1.093578in}{1.813754in}}{\pgfqpoint{1.099402in}{1.807930in}}%
\pgfpathcurveto{\pgfqpoint{1.105226in}{1.802106in}}{\pgfqpoint{1.113126in}{1.798834in}}{\pgfqpoint{1.121362in}{1.798834in}}%
\pgfpathclose%
\pgfusepath{stroke,fill}%
\end{pgfscope}%
\begin{pgfscope}%
\pgfpathrectangle{\pgfqpoint{0.100000in}{0.212622in}}{\pgfqpoint{3.696000in}{3.696000in}}%
\pgfusepath{clip}%
\pgfsetbuttcap%
\pgfsetroundjoin%
\definecolor{currentfill}{rgb}{0.121569,0.466667,0.705882}%
\pgfsetfillcolor{currentfill}%
\pgfsetfillopacity{0.751809}%
\pgfsetlinewidth{1.003750pt}%
\definecolor{currentstroke}{rgb}{0.121569,0.466667,0.705882}%
\pgfsetstrokecolor{currentstroke}%
\pgfsetstrokeopacity{0.751809}%
\pgfsetdash{}{0pt}%
\pgfpathmoveto{\pgfqpoint{2.448753in}{2.962048in}}%
\pgfpathcurveto{\pgfqpoint{2.456989in}{2.962048in}}{\pgfqpoint{2.464889in}{2.965321in}}{\pgfqpoint{2.470713in}{2.971145in}}%
\pgfpathcurveto{\pgfqpoint{2.476537in}{2.976968in}}{\pgfqpoint{2.479809in}{2.984868in}}{\pgfqpoint{2.479809in}{2.993105in}}%
\pgfpathcurveto{\pgfqpoint{2.479809in}{3.001341in}}{\pgfqpoint{2.476537in}{3.009241in}}{\pgfqpoint{2.470713in}{3.015065in}}%
\pgfpathcurveto{\pgfqpoint{2.464889in}{3.020889in}}{\pgfqpoint{2.456989in}{3.024161in}}{\pgfqpoint{2.448753in}{3.024161in}}%
\pgfpathcurveto{\pgfqpoint{2.440517in}{3.024161in}}{\pgfqpoint{2.432616in}{3.020889in}}{\pgfqpoint{2.426793in}{3.015065in}}%
\pgfpathcurveto{\pgfqpoint{2.420969in}{3.009241in}}{\pgfqpoint{2.417696in}{3.001341in}}{\pgfqpoint{2.417696in}{2.993105in}}%
\pgfpathcurveto{\pgfqpoint{2.417696in}{2.984868in}}{\pgfqpoint{2.420969in}{2.976968in}}{\pgfqpoint{2.426793in}{2.971145in}}%
\pgfpathcurveto{\pgfqpoint{2.432616in}{2.965321in}}{\pgfqpoint{2.440517in}{2.962048in}}{\pgfqpoint{2.448753in}{2.962048in}}%
\pgfpathclose%
\pgfusepath{stroke,fill}%
\end{pgfscope}%
\begin{pgfscope}%
\pgfpathrectangle{\pgfqpoint{0.100000in}{0.212622in}}{\pgfqpoint{3.696000in}{3.696000in}}%
\pgfusepath{clip}%
\pgfsetbuttcap%
\pgfsetroundjoin%
\definecolor{currentfill}{rgb}{0.121569,0.466667,0.705882}%
\pgfsetfillcolor{currentfill}%
\pgfsetfillopacity{0.752140}%
\pgfsetlinewidth{1.003750pt}%
\definecolor{currentstroke}{rgb}{0.121569,0.466667,0.705882}%
\pgfsetstrokecolor{currentstroke}%
\pgfsetstrokeopacity{0.752140}%
\pgfsetdash{}{0pt}%
\pgfpathmoveto{\pgfqpoint{3.169977in}{2.220466in}}%
\pgfpathcurveto{\pgfqpoint{3.178213in}{2.220466in}}{\pgfqpoint{3.186113in}{2.223739in}}{\pgfqpoint{3.191937in}{2.229563in}}%
\pgfpathcurveto{\pgfqpoint{3.197761in}{2.235386in}}{\pgfqpoint{3.201033in}{2.243287in}}{\pgfqpoint{3.201033in}{2.251523in}}%
\pgfpathcurveto{\pgfqpoint{3.201033in}{2.259759in}}{\pgfqpoint{3.197761in}{2.267659in}}{\pgfqpoint{3.191937in}{2.273483in}}%
\pgfpathcurveto{\pgfqpoint{3.186113in}{2.279307in}}{\pgfqpoint{3.178213in}{2.282579in}}{\pgfqpoint{3.169977in}{2.282579in}}%
\pgfpathcurveto{\pgfqpoint{3.161740in}{2.282579in}}{\pgfqpoint{3.153840in}{2.279307in}}{\pgfqpoint{3.148016in}{2.273483in}}%
\pgfpathcurveto{\pgfqpoint{3.142192in}{2.267659in}}{\pgfqpoint{3.138920in}{2.259759in}}{\pgfqpoint{3.138920in}{2.251523in}}%
\pgfpathcurveto{\pgfqpoint{3.138920in}{2.243287in}}{\pgfqpoint{3.142192in}{2.235386in}}{\pgfqpoint{3.148016in}{2.229563in}}%
\pgfpathcurveto{\pgfqpoint{3.153840in}{2.223739in}}{\pgfqpoint{3.161740in}{2.220466in}}{\pgfqpoint{3.169977in}{2.220466in}}%
\pgfpathclose%
\pgfusepath{stroke,fill}%
\end{pgfscope}%
\begin{pgfscope}%
\pgfpathrectangle{\pgfqpoint{0.100000in}{0.212622in}}{\pgfqpoint{3.696000in}{3.696000in}}%
\pgfusepath{clip}%
\pgfsetbuttcap%
\pgfsetroundjoin%
\definecolor{currentfill}{rgb}{0.121569,0.466667,0.705882}%
\pgfsetfillcolor{currentfill}%
\pgfsetfillopacity{0.752239}%
\pgfsetlinewidth{1.003750pt}%
\definecolor{currentstroke}{rgb}{0.121569,0.466667,0.705882}%
\pgfsetstrokecolor{currentstroke}%
\pgfsetstrokeopacity{0.752239}%
\pgfsetdash{}{0pt}%
\pgfpathmoveto{\pgfqpoint{2.450672in}{2.961783in}}%
\pgfpathcurveto{\pgfqpoint{2.458909in}{2.961783in}}{\pgfqpoint{2.466809in}{2.965055in}}{\pgfqpoint{2.472633in}{2.970879in}}%
\pgfpathcurveto{\pgfqpoint{2.478457in}{2.976703in}}{\pgfqpoint{2.481729in}{2.984603in}}{\pgfqpoint{2.481729in}{2.992839in}}%
\pgfpathcurveto{\pgfqpoint{2.481729in}{3.001075in}}{\pgfqpoint{2.478457in}{3.008975in}}{\pgfqpoint{2.472633in}{3.014799in}}%
\pgfpathcurveto{\pgfqpoint{2.466809in}{3.020623in}}{\pgfqpoint{2.458909in}{3.023896in}}{\pgfqpoint{2.450672in}{3.023896in}}%
\pgfpathcurveto{\pgfqpoint{2.442436in}{3.023896in}}{\pgfqpoint{2.434536in}{3.020623in}}{\pgfqpoint{2.428712in}{3.014799in}}%
\pgfpathcurveto{\pgfqpoint{2.422888in}{3.008975in}}{\pgfqpoint{2.419616in}{3.001075in}}{\pgfqpoint{2.419616in}{2.992839in}}%
\pgfpathcurveto{\pgfqpoint{2.419616in}{2.984603in}}{\pgfqpoint{2.422888in}{2.976703in}}{\pgfqpoint{2.428712in}{2.970879in}}%
\pgfpathcurveto{\pgfqpoint{2.434536in}{2.965055in}}{\pgfqpoint{2.442436in}{2.961783in}}{\pgfqpoint{2.450672in}{2.961783in}}%
\pgfpathclose%
\pgfusepath{stroke,fill}%
\end{pgfscope}%
\begin{pgfscope}%
\pgfpathrectangle{\pgfqpoint{0.100000in}{0.212622in}}{\pgfqpoint{3.696000in}{3.696000in}}%
\pgfusepath{clip}%
\pgfsetbuttcap%
\pgfsetroundjoin%
\definecolor{currentfill}{rgb}{0.121569,0.466667,0.705882}%
\pgfsetfillcolor{currentfill}%
\pgfsetfillopacity{0.752887}%
\pgfsetlinewidth{1.003750pt}%
\definecolor{currentstroke}{rgb}{0.121569,0.466667,0.705882}%
\pgfsetstrokecolor{currentstroke}%
\pgfsetstrokeopacity{0.752887}%
\pgfsetdash{}{0pt}%
\pgfpathmoveto{\pgfqpoint{2.454098in}{2.960495in}}%
\pgfpathcurveto{\pgfqpoint{2.462334in}{2.960495in}}{\pgfqpoint{2.470234in}{2.963768in}}{\pgfqpoint{2.476058in}{2.969592in}}%
\pgfpathcurveto{\pgfqpoint{2.481882in}{2.975416in}}{\pgfqpoint{2.485154in}{2.983316in}}{\pgfqpoint{2.485154in}{2.991552in}}%
\pgfpathcurveto{\pgfqpoint{2.485154in}{2.999788in}}{\pgfqpoint{2.481882in}{3.007688in}}{\pgfqpoint{2.476058in}{3.013512in}}%
\pgfpathcurveto{\pgfqpoint{2.470234in}{3.019336in}}{\pgfqpoint{2.462334in}{3.022608in}}{\pgfqpoint{2.454098in}{3.022608in}}%
\pgfpathcurveto{\pgfqpoint{2.445861in}{3.022608in}}{\pgfqpoint{2.437961in}{3.019336in}}{\pgfqpoint{2.432137in}{3.013512in}}%
\pgfpathcurveto{\pgfqpoint{2.426314in}{3.007688in}}{\pgfqpoint{2.423041in}{2.999788in}}{\pgfqpoint{2.423041in}{2.991552in}}%
\pgfpathcurveto{\pgfqpoint{2.423041in}{2.983316in}}{\pgfqpoint{2.426314in}{2.975416in}}{\pgfqpoint{2.432137in}{2.969592in}}%
\pgfpathcurveto{\pgfqpoint{2.437961in}{2.963768in}}{\pgfqpoint{2.445861in}{2.960495in}}{\pgfqpoint{2.454098in}{2.960495in}}%
\pgfpathclose%
\pgfusepath{stroke,fill}%
\end{pgfscope}%
\begin{pgfscope}%
\pgfpathrectangle{\pgfqpoint{0.100000in}{0.212622in}}{\pgfqpoint{3.696000in}{3.696000in}}%
\pgfusepath{clip}%
\pgfsetbuttcap%
\pgfsetroundjoin%
\definecolor{currentfill}{rgb}{0.121569,0.466667,0.705882}%
\pgfsetfillcolor{currentfill}%
\pgfsetfillopacity{0.753366}%
\pgfsetlinewidth{1.003750pt}%
\definecolor{currentstroke}{rgb}{0.121569,0.466667,0.705882}%
\pgfsetstrokecolor{currentstroke}%
\pgfsetstrokeopacity{0.753366}%
\pgfsetdash{}{0pt}%
\pgfpathmoveto{\pgfqpoint{2.456458in}{2.960500in}}%
\pgfpathcurveto{\pgfqpoint{2.464694in}{2.960500in}}{\pgfqpoint{2.472594in}{2.963772in}}{\pgfqpoint{2.478418in}{2.969596in}}%
\pgfpathcurveto{\pgfqpoint{2.484242in}{2.975420in}}{\pgfqpoint{2.487514in}{2.983320in}}{\pgfqpoint{2.487514in}{2.991556in}}%
\pgfpathcurveto{\pgfqpoint{2.487514in}{2.999792in}}{\pgfqpoint{2.484242in}{3.007692in}}{\pgfqpoint{2.478418in}{3.013516in}}%
\pgfpathcurveto{\pgfqpoint{2.472594in}{3.019340in}}{\pgfqpoint{2.464694in}{3.022613in}}{\pgfqpoint{2.456458in}{3.022613in}}%
\pgfpathcurveto{\pgfqpoint{2.448221in}{3.022613in}}{\pgfqpoint{2.440321in}{3.019340in}}{\pgfqpoint{2.434497in}{3.013516in}}%
\pgfpathcurveto{\pgfqpoint{2.428673in}{3.007692in}}{\pgfqpoint{2.425401in}{2.999792in}}{\pgfqpoint{2.425401in}{2.991556in}}%
\pgfpathcurveto{\pgfqpoint{2.425401in}{2.983320in}}{\pgfqpoint{2.428673in}{2.975420in}}{\pgfqpoint{2.434497in}{2.969596in}}%
\pgfpathcurveto{\pgfqpoint{2.440321in}{2.963772in}}{\pgfqpoint{2.448221in}{2.960500in}}{\pgfqpoint{2.456458in}{2.960500in}}%
\pgfpathclose%
\pgfusepath{stroke,fill}%
\end{pgfscope}%
\begin{pgfscope}%
\pgfpathrectangle{\pgfqpoint{0.100000in}{0.212622in}}{\pgfqpoint{3.696000in}{3.696000in}}%
\pgfusepath{clip}%
\pgfsetbuttcap%
\pgfsetroundjoin%
\definecolor{currentfill}{rgb}{0.121569,0.466667,0.705882}%
\pgfsetfillcolor{currentfill}%
\pgfsetfillopacity{0.753547}%
\pgfsetlinewidth{1.003750pt}%
\definecolor{currentstroke}{rgb}{0.121569,0.466667,0.705882}%
\pgfsetstrokecolor{currentstroke}%
\pgfsetstrokeopacity{0.753547}%
\pgfsetdash{}{0pt}%
\pgfpathmoveto{\pgfqpoint{1.114038in}{1.785968in}}%
\pgfpathcurveto{\pgfqpoint{1.122274in}{1.785968in}}{\pgfqpoint{1.130174in}{1.789240in}}{\pgfqpoint{1.135998in}{1.795064in}}%
\pgfpathcurveto{\pgfqpoint{1.141822in}{1.800888in}}{\pgfqpoint{1.145094in}{1.808788in}}{\pgfqpoint{1.145094in}{1.817025in}}%
\pgfpathcurveto{\pgfqpoint{1.145094in}{1.825261in}}{\pgfqpoint{1.141822in}{1.833161in}}{\pgfqpoint{1.135998in}{1.838985in}}%
\pgfpathcurveto{\pgfqpoint{1.130174in}{1.844809in}}{\pgfqpoint{1.122274in}{1.848081in}}{\pgfqpoint{1.114038in}{1.848081in}}%
\pgfpathcurveto{\pgfqpoint{1.105802in}{1.848081in}}{\pgfqpoint{1.097902in}{1.844809in}}{\pgfqpoint{1.092078in}{1.838985in}}%
\pgfpathcurveto{\pgfqpoint{1.086254in}{1.833161in}}{\pgfqpoint{1.082981in}{1.825261in}}{\pgfqpoint{1.082981in}{1.817025in}}%
\pgfpathcurveto{\pgfqpoint{1.082981in}{1.808788in}}{\pgfqpoint{1.086254in}{1.800888in}}{\pgfqpoint{1.092078in}{1.795064in}}%
\pgfpathcurveto{\pgfqpoint{1.097902in}{1.789240in}}{\pgfqpoint{1.105802in}{1.785968in}}{\pgfqpoint{1.114038in}{1.785968in}}%
\pgfpathclose%
\pgfusepath{stroke,fill}%
\end{pgfscope}%
\begin{pgfscope}%
\pgfpathrectangle{\pgfqpoint{0.100000in}{0.212622in}}{\pgfqpoint{3.696000in}{3.696000in}}%
\pgfusepath{clip}%
\pgfsetbuttcap%
\pgfsetroundjoin%
\definecolor{currentfill}{rgb}{0.121569,0.466667,0.705882}%
\pgfsetfillcolor{currentfill}%
\pgfsetfillopacity{0.753582}%
\pgfsetlinewidth{1.003750pt}%
\definecolor{currentstroke}{rgb}{0.121569,0.466667,0.705882}%
\pgfsetstrokecolor{currentstroke}%
\pgfsetstrokeopacity{0.753582}%
\pgfsetdash{}{0pt}%
\pgfpathmoveto{\pgfqpoint{3.164747in}{2.211727in}}%
\pgfpathcurveto{\pgfqpoint{3.172983in}{2.211727in}}{\pgfqpoint{3.180883in}{2.214999in}}{\pgfqpoint{3.186707in}{2.220823in}}%
\pgfpathcurveto{\pgfqpoint{3.192531in}{2.226647in}}{\pgfqpoint{3.195803in}{2.234547in}}{\pgfqpoint{3.195803in}{2.242783in}}%
\pgfpathcurveto{\pgfqpoint{3.195803in}{2.251019in}}{\pgfqpoint{3.192531in}{2.258919in}}{\pgfqpoint{3.186707in}{2.264743in}}%
\pgfpathcurveto{\pgfqpoint{3.180883in}{2.270567in}}{\pgfqpoint{3.172983in}{2.273840in}}{\pgfqpoint{3.164747in}{2.273840in}}%
\pgfpathcurveto{\pgfqpoint{3.156511in}{2.273840in}}{\pgfqpoint{3.148610in}{2.270567in}}{\pgfqpoint{3.142787in}{2.264743in}}%
\pgfpathcurveto{\pgfqpoint{3.136963in}{2.258919in}}{\pgfqpoint{3.133690in}{2.251019in}}{\pgfqpoint{3.133690in}{2.242783in}}%
\pgfpathcurveto{\pgfqpoint{3.133690in}{2.234547in}}{\pgfqpoint{3.136963in}{2.226647in}}{\pgfqpoint{3.142787in}{2.220823in}}%
\pgfpathcurveto{\pgfqpoint{3.148610in}{2.214999in}}{\pgfqpoint{3.156511in}{2.211727in}}{\pgfqpoint{3.164747in}{2.211727in}}%
\pgfpathclose%
\pgfusepath{stroke,fill}%
\end{pgfscope}%
\begin{pgfscope}%
\pgfpathrectangle{\pgfqpoint{0.100000in}{0.212622in}}{\pgfqpoint{3.696000in}{3.696000in}}%
\pgfusepath{clip}%
\pgfsetbuttcap%
\pgfsetroundjoin%
\definecolor{currentfill}{rgb}{0.121569,0.466667,0.705882}%
\pgfsetfillcolor{currentfill}%
\pgfsetfillopacity{0.754180}%
\pgfsetlinewidth{1.003750pt}%
\definecolor{currentstroke}{rgb}{0.121569,0.466667,0.705882}%
\pgfsetstrokecolor{currentstroke}%
\pgfsetstrokeopacity{0.754180}%
\pgfsetdash{}{0pt}%
\pgfpathmoveto{\pgfqpoint{2.460548in}{2.959524in}}%
\pgfpathcurveto{\pgfqpoint{2.468785in}{2.959524in}}{\pgfqpoint{2.476685in}{2.962797in}}{\pgfqpoint{2.482509in}{2.968621in}}%
\pgfpathcurveto{\pgfqpoint{2.488333in}{2.974445in}}{\pgfqpoint{2.491605in}{2.982345in}}{\pgfqpoint{2.491605in}{2.990581in}}%
\pgfpathcurveto{\pgfqpoint{2.491605in}{2.998817in}}{\pgfqpoint{2.488333in}{3.006717in}}{\pgfqpoint{2.482509in}{3.012541in}}%
\pgfpathcurveto{\pgfqpoint{2.476685in}{3.018365in}}{\pgfqpoint{2.468785in}{3.021637in}}{\pgfqpoint{2.460548in}{3.021637in}}%
\pgfpathcurveto{\pgfqpoint{2.452312in}{3.021637in}}{\pgfqpoint{2.444412in}{3.018365in}}{\pgfqpoint{2.438588in}{3.012541in}}%
\pgfpathcurveto{\pgfqpoint{2.432764in}{3.006717in}}{\pgfqpoint{2.429492in}{2.998817in}}{\pgfqpoint{2.429492in}{2.990581in}}%
\pgfpathcurveto{\pgfqpoint{2.429492in}{2.982345in}}{\pgfqpoint{2.432764in}{2.974445in}}{\pgfqpoint{2.438588in}{2.968621in}}%
\pgfpathcurveto{\pgfqpoint{2.444412in}{2.962797in}}{\pgfqpoint{2.452312in}{2.959524in}}{\pgfqpoint{2.460548in}{2.959524in}}%
\pgfpathclose%
\pgfusepath{stroke,fill}%
\end{pgfscope}%
\begin{pgfscope}%
\pgfpathrectangle{\pgfqpoint{0.100000in}{0.212622in}}{\pgfqpoint{3.696000in}{3.696000in}}%
\pgfusepath{clip}%
\pgfsetbuttcap%
\pgfsetroundjoin%
\definecolor{currentfill}{rgb}{0.121569,0.466667,0.705882}%
\pgfsetfillcolor{currentfill}%
\pgfsetfillopacity{0.754747}%
\pgfsetlinewidth{1.003750pt}%
\definecolor{currentstroke}{rgb}{0.121569,0.466667,0.705882}%
\pgfsetstrokecolor{currentstroke}%
\pgfsetstrokeopacity{0.754747}%
\pgfsetdash{}{0pt}%
\pgfpathmoveto{\pgfqpoint{2.463314in}{2.959037in}}%
\pgfpathcurveto{\pgfqpoint{2.471551in}{2.959037in}}{\pgfqpoint{2.479451in}{2.962309in}}{\pgfqpoint{2.485275in}{2.968133in}}%
\pgfpathcurveto{\pgfqpoint{2.491099in}{2.973957in}}{\pgfqpoint{2.494371in}{2.981857in}}{\pgfqpoint{2.494371in}{2.990094in}}%
\pgfpathcurveto{\pgfqpoint{2.494371in}{2.998330in}}{\pgfqpoint{2.491099in}{3.006230in}}{\pgfqpoint{2.485275in}{3.012054in}}%
\pgfpathcurveto{\pgfqpoint{2.479451in}{3.017878in}}{\pgfqpoint{2.471551in}{3.021150in}}{\pgfqpoint{2.463314in}{3.021150in}}%
\pgfpathcurveto{\pgfqpoint{2.455078in}{3.021150in}}{\pgfqpoint{2.447178in}{3.017878in}}{\pgfqpoint{2.441354in}{3.012054in}}%
\pgfpathcurveto{\pgfqpoint{2.435530in}{3.006230in}}{\pgfqpoint{2.432258in}{2.998330in}}{\pgfqpoint{2.432258in}{2.990094in}}%
\pgfpathcurveto{\pgfqpoint{2.432258in}{2.981857in}}{\pgfqpoint{2.435530in}{2.973957in}}{\pgfqpoint{2.441354in}{2.968133in}}%
\pgfpathcurveto{\pgfqpoint{2.447178in}{2.962309in}}{\pgfqpoint{2.455078in}{2.959037in}}{\pgfqpoint{2.463314in}{2.959037in}}%
\pgfpathclose%
\pgfusepath{stroke,fill}%
\end{pgfscope}%
\begin{pgfscope}%
\pgfpathrectangle{\pgfqpoint{0.100000in}{0.212622in}}{\pgfqpoint{3.696000in}{3.696000in}}%
\pgfusepath{clip}%
\pgfsetbuttcap%
\pgfsetroundjoin%
\definecolor{currentfill}{rgb}{0.121569,0.466667,0.705882}%
\pgfsetfillcolor{currentfill}%
\pgfsetfillopacity{0.754932}%
\pgfsetlinewidth{1.003750pt}%
\definecolor{currentstroke}{rgb}{0.121569,0.466667,0.705882}%
\pgfsetstrokecolor{currentstroke}%
\pgfsetstrokeopacity{0.754932}%
\pgfsetdash{}{0pt}%
\pgfpathmoveto{\pgfqpoint{3.162805in}{2.203447in}}%
\pgfpathcurveto{\pgfqpoint{3.171041in}{2.203447in}}{\pgfqpoint{3.178941in}{2.206720in}}{\pgfqpoint{3.184765in}{2.212544in}}%
\pgfpathcurveto{\pgfqpoint{3.190589in}{2.218367in}}{\pgfqpoint{3.193861in}{2.226268in}}{\pgfqpoint{3.193861in}{2.234504in}}%
\pgfpathcurveto{\pgfqpoint{3.193861in}{2.242740in}}{\pgfqpoint{3.190589in}{2.250640in}}{\pgfqpoint{3.184765in}{2.256464in}}%
\pgfpathcurveto{\pgfqpoint{3.178941in}{2.262288in}}{\pgfqpoint{3.171041in}{2.265560in}}{\pgfqpoint{3.162805in}{2.265560in}}%
\pgfpathcurveto{\pgfqpoint{3.154568in}{2.265560in}}{\pgfqpoint{3.146668in}{2.262288in}}{\pgfqpoint{3.140845in}{2.256464in}}%
\pgfpathcurveto{\pgfqpoint{3.135021in}{2.250640in}}{\pgfqpoint{3.131748in}{2.242740in}}{\pgfqpoint{3.131748in}{2.234504in}}%
\pgfpathcurveto{\pgfqpoint{3.131748in}{2.226268in}}{\pgfqpoint{3.135021in}{2.218367in}}{\pgfqpoint{3.140845in}{2.212544in}}%
\pgfpathcurveto{\pgfqpoint{3.146668in}{2.206720in}}{\pgfqpoint{3.154568in}{2.203447in}}{\pgfqpoint{3.162805in}{2.203447in}}%
\pgfpathclose%
\pgfusepath{stroke,fill}%
\end{pgfscope}%
\begin{pgfscope}%
\pgfpathrectangle{\pgfqpoint{0.100000in}{0.212622in}}{\pgfqpoint{3.696000in}{3.696000in}}%
\pgfusepath{clip}%
\pgfsetbuttcap%
\pgfsetroundjoin%
\definecolor{currentfill}{rgb}{0.121569,0.466667,0.705882}%
\pgfsetfillcolor{currentfill}%
\pgfsetfillopacity{0.755660}%
\pgfsetlinewidth{1.003750pt}%
\definecolor{currentstroke}{rgb}{0.121569,0.466667,0.705882}%
\pgfsetstrokecolor{currentstroke}%
\pgfsetstrokeopacity{0.755660}%
\pgfsetdash{}{0pt}%
\pgfpathmoveto{\pgfqpoint{1.108519in}{1.767756in}}%
\pgfpathcurveto{\pgfqpoint{1.116756in}{1.767756in}}{\pgfqpoint{1.124656in}{1.771028in}}{\pgfqpoint{1.130480in}{1.776852in}}%
\pgfpathcurveto{\pgfqpoint{1.136304in}{1.782676in}}{\pgfqpoint{1.139576in}{1.790576in}}{\pgfqpoint{1.139576in}{1.798812in}}%
\pgfpathcurveto{\pgfqpoint{1.139576in}{1.807049in}}{\pgfqpoint{1.136304in}{1.814949in}}{\pgfqpoint{1.130480in}{1.820772in}}%
\pgfpathcurveto{\pgfqpoint{1.124656in}{1.826596in}}{\pgfqpoint{1.116756in}{1.829869in}}{\pgfqpoint{1.108519in}{1.829869in}}%
\pgfpathcurveto{\pgfqpoint{1.100283in}{1.829869in}}{\pgfqpoint{1.092383in}{1.826596in}}{\pgfqpoint{1.086559in}{1.820772in}}%
\pgfpathcurveto{\pgfqpoint{1.080735in}{1.814949in}}{\pgfqpoint{1.077463in}{1.807049in}}{\pgfqpoint{1.077463in}{1.798812in}}%
\pgfpathcurveto{\pgfqpoint{1.077463in}{1.790576in}}{\pgfqpoint{1.080735in}{1.782676in}}{\pgfqpoint{1.086559in}{1.776852in}}%
\pgfpathcurveto{\pgfqpoint{1.092383in}{1.771028in}}{\pgfqpoint{1.100283in}{1.767756in}}{\pgfqpoint{1.108519in}{1.767756in}}%
\pgfpathclose%
\pgfusepath{stroke,fill}%
\end{pgfscope}%
\begin{pgfscope}%
\pgfpathrectangle{\pgfqpoint{0.100000in}{0.212622in}}{\pgfqpoint{3.696000in}{3.696000in}}%
\pgfusepath{clip}%
\pgfsetbuttcap%
\pgfsetroundjoin%
\definecolor{currentfill}{rgb}{0.121569,0.466667,0.705882}%
\pgfsetfillcolor{currentfill}%
\pgfsetfillopacity{0.755882}%
\pgfsetlinewidth{1.003750pt}%
\definecolor{currentstroke}{rgb}{0.121569,0.466667,0.705882}%
\pgfsetstrokecolor{currentstroke}%
\pgfsetstrokeopacity{0.755882}%
\pgfsetdash{}{0pt}%
\pgfpathmoveto{\pgfqpoint{2.468349in}{2.958611in}}%
\pgfpathcurveto{\pgfqpoint{2.476585in}{2.958611in}}{\pgfqpoint{2.484485in}{2.961884in}}{\pgfqpoint{2.490309in}{2.967708in}}%
\pgfpathcurveto{\pgfqpoint{2.496133in}{2.973532in}}{\pgfqpoint{2.499405in}{2.981432in}}{\pgfqpoint{2.499405in}{2.989668in}}%
\pgfpathcurveto{\pgfqpoint{2.499405in}{2.997904in}}{\pgfqpoint{2.496133in}{3.005804in}}{\pgfqpoint{2.490309in}{3.011628in}}%
\pgfpathcurveto{\pgfqpoint{2.484485in}{3.017452in}}{\pgfqpoint{2.476585in}{3.020724in}}{\pgfqpoint{2.468349in}{3.020724in}}%
\pgfpathcurveto{\pgfqpoint{2.460112in}{3.020724in}}{\pgfqpoint{2.452212in}{3.017452in}}{\pgfqpoint{2.446389in}{3.011628in}}%
\pgfpathcurveto{\pgfqpoint{2.440565in}{3.005804in}}{\pgfqpoint{2.437292in}{2.997904in}}{\pgfqpoint{2.437292in}{2.989668in}}%
\pgfpathcurveto{\pgfqpoint{2.437292in}{2.981432in}}{\pgfqpoint{2.440565in}{2.973532in}}{\pgfqpoint{2.446389in}{2.967708in}}%
\pgfpathcurveto{\pgfqpoint{2.452212in}{2.961884in}}{\pgfqpoint{2.460112in}{2.958611in}}{\pgfqpoint{2.468349in}{2.958611in}}%
\pgfpathclose%
\pgfusepath{stroke,fill}%
\end{pgfscope}%
\begin{pgfscope}%
\pgfpathrectangle{\pgfqpoint{0.100000in}{0.212622in}}{\pgfqpoint{3.696000in}{3.696000in}}%
\pgfusepath{clip}%
\pgfsetbuttcap%
\pgfsetroundjoin%
\definecolor{currentfill}{rgb}{0.121569,0.466667,0.705882}%
\pgfsetfillcolor{currentfill}%
\pgfsetfillopacity{0.756083}%
\pgfsetlinewidth{1.003750pt}%
\definecolor{currentstroke}{rgb}{0.121569,0.466667,0.705882}%
\pgfsetstrokecolor{currentstroke}%
\pgfsetstrokeopacity{0.756083}%
\pgfsetdash{}{0pt}%
\pgfpathmoveto{\pgfqpoint{3.161454in}{2.195409in}}%
\pgfpathcurveto{\pgfqpoint{3.169691in}{2.195409in}}{\pgfqpoint{3.177591in}{2.198682in}}{\pgfqpoint{3.183415in}{2.204506in}}%
\pgfpathcurveto{\pgfqpoint{3.189239in}{2.210330in}}{\pgfqpoint{3.192511in}{2.218230in}}{\pgfqpoint{3.192511in}{2.226466in}}%
\pgfpathcurveto{\pgfqpoint{3.192511in}{2.234702in}}{\pgfqpoint{3.189239in}{2.242602in}}{\pgfqpoint{3.183415in}{2.248426in}}%
\pgfpathcurveto{\pgfqpoint{3.177591in}{2.254250in}}{\pgfqpoint{3.169691in}{2.257522in}}{\pgfqpoint{3.161454in}{2.257522in}}%
\pgfpathcurveto{\pgfqpoint{3.153218in}{2.257522in}}{\pgfqpoint{3.145318in}{2.254250in}}{\pgfqpoint{3.139494in}{2.248426in}}%
\pgfpathcurveto{\pgfqpoint{3.133670in}{2.242602in}}{\pgfqpoint{3.130398in}{2.234702in}}{\pgfqpoint{3.130398in}{2.226466in}}%
\pgfpathcurveto{\pgfqpoint{3.130398in}{2.218230in}}{\pgfqpoint{3.133670in}{2.210330in}}{\pgfqpoint{3.139494in}{2.204506in}}%
\pgfpathcurveto{\pgfqpoint{3.145318in}{2.198682in}}{\pgfqpoint{3.153218in}{2.195409in}}{\pgfqpoint{3.161454in}{2.195409in}}%
\pgfpathclose%
\pgfusepath{stroke,fill}%
\end{pgfscope}%
\begin{pgfscope}%
\pgfpathrectangle{\pgfqpoint{0.100000in}{0.212622in}}{\pgfqpoint{3.696000in}{3.696000in}}%
\pgfusepath{clip}%
\pgfsetbuttcap%
\pgfsetroundjoin%
\definecolor{currentfill}{rgb}{0.121569,0.466667,0.705882}%
\pgfsetfillcolor{currentfill}%
\pgfsetfillopacity{0.756854}%
\pgfsetlinewidth{1.003750pt}%
\definecolor{currentstroke}{rgb}{0.121569,0.466667,0.705882}%
\pgfsetstrokecolor{currentstroke}%
\pgfsetstrokeopacity{0.756854}%
\pgfsetdash{}{0pt}%
\pgfpathmoveto{\pgfqpoint{2.471811in}{2.958335in}}%
\pgfpathcurveto{\pgfqpoint{2.480047in}{2.958335in}}{\pgfqpoint{2.487947in}{2.961608in}}{\pgfqpoint{2.493771in}{2.967432in}}%
\pgfpathcurveto{\pgfqpoint{2.499595in}{2.973256in}}{\pgfqpoint{2.502867in}{2.981156in}}{\pgfqpoint{2.502867in}{2.989392in}}%
\pgfpathcurveto{\pgfqpoint{2.502867in}{2.997628in}}{\pgfqpoint{2.499595in}{3.005528in}}{\pgfqpoint{2.493771in}{3.011352in}}%
\pgfpathcurveto{\pgfqpoint{2.487947in}{3.017176in}}{\pgfqpoint{2.480047in}{3.020448in}}{\pgfqpoint{2.471811in}{3.020448in}}%
\pgfpathcurveto{\pgfqpoint{2.463574in}{3.020448in}}{\pgfqpoint{2.455674in}{3.017176in}}{\pgfqpoint{2.449850in}{3.011352in}}%
\pgfpathcurveto{\pgfqpoint{2.444026in}{3.005528in}}{\pgfqpoint{2.440754in}{2.997628in}}{\pgfqpoint{2.440754in}{2.989392in}}%
\pgfpathcurveto{\pgfqpoint{2.440754in}{2.981156in}}{\pgfqpoint{2.444026in}{2.973256in}}{\pgfqpoint{2.449850in}{2.967432in}}%
\pgfpathcurveto{\pgfqpoint{2.455674in}{2.961608in}}{\pgfqpoint{2.463574in}{2.958335in}}{\pgfqpoint{2.471811in}{2.958335in}}%
\pgfpathclose%
\pgfusepath{stroke,fill}%
\end{pgfscope}%
\begin{pgfscope}%
\pgfpathrectangle{\pgfqpoint{0.100000in}{0.212622in}}{\pgfqpoint{3.696000in}{3.696000in}}%
\pgfusepath{clip}%
\pgfsetbuttcap%
\pgfsetroundjoin%
\definecolor{currentfill}{rgb}{0.121569,0.466667,0.705882}%
\pgfsetfillcolor{currentfill}%
\pgfsetfillopacity{0.757219}%
\pgfsetlinewidth{1.003750pt}%
\definecolor{currentstroke}{rgb}{0.121569,0.466667,0.705882}%
\pgfsetstrokecolor{currentstroke}%
\pgfsetstrokeopacity{0.757219}%
\pgfsetdash{}{0pt}%
\pgfpathmoveto{\pgfqpoint{3.159835in}{2.188422in}}%
\pgfpathcurveto{\pgfqpoint{3.168071in}{2.188422in}}{\pgfqpoint{3.175971in}{2.191694in}}{\pgfqpoint{3.181795in}{2.197518in}}%
\pgfpathcurveto{\pgfqpoint{3.187619in}{2.203342in}}{\pgfqpoint{3.190891in}{2.211242in}}{\pgfqpoint{3.190891in}{2.219478in}}%
\pgfpathcurveto{\pgfqpoint{3.190891in}{2.227715in}}{\pgfqpoint{3.187619in}{2.235615in}}{\pgfqpoint{3.181795in}{2.241439in}}%
\pgfpathcurveto{\pgfqpoint{3.175971in}{2.247263in}}{\pgfqpoint{3.168071in}{2.250535in}}{\pgfqpoint{3.159835in}{2.250535in}}%
\pgfpathcurveto{\pgfqpoint{3.151599in}{2.250535in}}{\pgfqpoint{3.143699in}{2.247263in}}{\pgfqpoint{3.137875in}{2.241439in}}%
\pgfpathcurveto{\pgfqpoint{3.132051in}{2.235615in}}{\pgfqpoint{3.128778in}{2.227715in}}{\pgfqpoint{3.128778in}{2.219478in}}%
\pgfpathcurveto{\pgfqpoint{3.128778in}{2.211242in}}{\pgfqpoint{3.132051in}{2.203342in}}{\pgfqpoint{3.137875in}{2.197518in}}%
\pgfpathcurveto{\pgfqpoint{3.143699in}{2.191694in}}{\pgfqpoint{3.151599in}{2.188422in}}{\pgfqpoint{3.159835in}{2.188422in}}%
\pgfpathclose%
\pgfusepath{stroke,fill}%
\end{pgfscope}%
\begin{pgfscope}%
\pgfpathrectangle{\pgfqpoint{0.100000in}{0.212622in}}{\pgfqpoint{3.696000in}{3.696000in}}%
\pgfusepath{clip}%
\pgfsetbuttcap%
\pgfsetroundjoin%
\definecolor{currentfill}{rgb}{0.121569,0.466667,0.705882}%
\pgfsetfillcolor{currentfill}%
\pgfsetfillopacity{0.757558}%
\pgfsetlinewidth{1.003750pt}%
\definecolor{currentstroke}{rgb}{0.121569,0.466667,0.705882}%
\pgfsetstrokecolor{currentstroke}%
\pgfsetstrokeopacity{0.757558}%
\pgfsetdash{}{0pt}%
\pgfpathmoveto{\pgfqpoint{2.475125in}{2.957472in}}%
\pgfpathcurveto{\pgfqpoint{2.483362in}{2.957472in}}{\pgfqpoint{2.491262in}{2.960744in}}{\pgfqpoint{2.497086in}{2.966568in}}%
\pgfpathcurveto{\pgfqpoint{2.502910in}{2.972392in}}{\pgfqpoint{2.506182in}{2.980292in}}{\pgfqpoint{2.506182in}{2.988528in}}%
\pgfpathcurveto{\pgfqpoint{2.506182in}{2.996764in}}{\pgfqpoint{2.502910in}{3.004664in}}{\pgfqpoint{2.497086in}{3.010488in}}%
\pgfpathcurveto{\pgfqpoint{2.491262in}{3.016312in}}{\pgfqpoint{2.483362in}{3.019585in}}{\pgfqpoint{2.475125in}{3.019585in}}%
\pgfpathcurveto{\pgfqpoint{2.466889in}{3.019585in}}{\pgfqpoint{2.458989in}{3.016312in}}{\pgfqpoint{2.453165in}{3.010488in}}%
\pgfpathcurveto{\pgfqpoint{2.447341in}{3.004664in}}{\pgfqpoint{2.444069in}{2.996764in}}{\pgfqpoint{2.444069in}{2.988528in}}%
\pgfpathcurveto{\pgfqpoint{2.444069in}{2.980292in}}{\pgfqpoint{2.447341in}{2.972392in}}{\pgfqpoint{2.453165in}{2.966568in}}%
\pgfpathcurveto{\pgfqpoint{2.458989in}{2.960744in}}{\pgfqpoint{2.466889in}{2.957472in}}{\pgfqpoint{2.475125in}{2.957472in}}%
\pgfpathclose%
\pgfusepath{stroke,fill}%
\end{pgfscope}%
\begin{pgfscope}%
\pgfpathrectangle{\pgfqpoint{0.100000in}{0.212622in}}{\pgfqpoint{3.696000in}{3.696000in}}%
\pgfusepath{clip}%
\pgfsetbuttcap%
\pgfsetroundjoin%
\definecolor{currentfill}{rgb}{0.121569,0.466667,0.705882}%
\pgfsetfillcolor{currentfill}%
\pgfsetfillopacity{0.757580}%
\pgfsetlinewidth{1.003750pt}%
\definecolor{currentstroke}{rgb}{0.121569,0.466667,0.705882}%
\pgfsetstrokecolor{currentstroke}%
\pgfsetstrokeopacity{0.757580}%
\pgfsetdash{}{0pt}%
\pgfpathmoveto{\pgfqpoint{1.103392in}{1.747523in}}%
\pgfpathcurveto{\pgfqpoint{1.111628in}{1.747523in}}{\pgfqpoint{1.119528in}{1.750796in}}{\pgfqpoint{1.125352in}{1.756620in}}%
\pgfpathcurveto{\pgfqpoint{1.131176in}{1.762443in}}{\pgfqpoint{1.134448in}{1.770343in}}{\pgfqpoint{1.134448in}{1.778580in}}%
\pgfpathcurveto{\pgfqpoint{1.134448in}{1.786816in}}{\pgfqpoint{1.131176in}{1.794716in}}{\pgfqpoint{1.125352in}{1.800540in}}%
\pgfpathcurveto{\pgfqpoint{1.119528in}{1.806364in}}{\pgfqpoint{1.111628in}{1.809636in}}{\pgfqpoint{1.103392in}{1.809636in}}%
\pgfpathcurveto{\pgfqpoint{1.095156in}{1.809636in}}{\pgfqpoint{1.087255in}{1.806364in}}{\pgfqpoint{1.081432in}{1.800540in}}%
\pgfpathcurveto{\pgfqpoint{1.075608in}{1.794716in}}{\pgfqpoint{1.072335in}{1.786816in}}{\pgfqpoint{1.072335in}{1.778580in}}%
\pgfpathcurveto{\pgfqpoint{1.072335in}{1.770343in}}{\pgfqpoint{1.075608in}{1.762443in}}{\pgfqpoint{1.081432in}{1.756620in}}%
\pgfpathcurveto{\pgfqpoint{1.087255in}{1.750796in}}{\pgfqpoint{1.095156in}{1.747523in}}{\pgfqpoint{1.103392in}{1.747523in}}%
\pgfpathclose%
\pgfusepath{stroke,fill}%
\end{pgfscope}%
\begin{pgfscope}%
\pgfpathrectangle{\pgfqpoint{0.100000in}{0.212622in}}{\pgfqpoint{3.696000in}{3.696000in}}%
\pgfusepath{clip}%
\pgfsetbuttcap%
\pgfsetroundjoin%
\definecolor{currentfill}{rgb}{0.121569,0.466667,0.705882}%
\pgfsetfillcolor{currentfill}%
\pgfsetfillopacity{0.757909}%
\pgfsetlinewidth{1.003750pt}%
\definecolor{currentstroke}{rgb}{0.121569,0.466667,0.705882}%
\pgfsetstrokecolor{currentstroke}%
\pgfsetstrokeopacity{0.757909}%
\pgfsetdash{}{0pt}%
\pgfpathmoveto{\pgfqpoint{3.157465in}{2.185180in}}%
\pgfpathcurveto{\pgfqpoint{3.165701in}{2.185180in}}{\pgfqpoint{3.173601in}{2.188453in}}{\pgfqpoint{3.179425in}{2.194276in}}%
\pgfpathcurveto{\pgfqpoint{3.185249in}{2.200100in}}{\pgfqpoint{3.188521in}{2.208000in}}{\pgfqpoint{3.188521in}{2.216237in}}%
\pgfpathcurveto{\pgfqpoint{3.188521in}{2.224473in}}{\pgfqpoint{3.185249in}{2.232373in}}{\pgfqpoint{3.179425in}{2.238197in}}%
\pgfpathcurveto{\pgfqpoint{3.173601in}{2.244021in}}{\pgfqpoint{3.165701in}{2.247293in}}{\pgfqpoint{3.157465in}{2.247293in}}%
\pgfpathcurveto{\pgfqpoint{3.149229in}{2.247293in}}{\pgfqpoint{3.141329in}{2.244021in}}{\pgfqpoint{3.135505in}{2.238197in}}%
\pgfpathcurveto{\pgfqpoint{3.129681in}{2.232373in}}{\pgfqpoint{3.126408in}{2.224473in}}{\pgfqpoint{3.126408in}{2.216237in}}%
\pgfpathcurveto{\pgfqpoint{3.126408in}{2.208000in}}{\pgfqpoint{3.129681in}{2.200100in}}{\pgfqpoint{3.135505in}{2.194276in}}%
\pgfpathcurveto{\pgfqpoint{3.141329in}{2.188453in}}{\pgfqpoint{3.149229in}{2.185180in}}{\pgfqpoint{3.157465in}{2.185180in}}%
\pgfpathclose%
\pgfusepath{stroke,fill}%
\end{pgfscope}%
\begin{pgfscope}%
\pgfpathrectangle{\pgfqpoint{0.100000in}{0.212622in}}{\pgfqpoint{3.696000in}{3.696000in}}%
\pgfusepath{clip}%
\pgfsetbuttcap%
\pgfsetroundjoin%
\definecolor{currentfill}{rgb}{0.121569,0.466667,0.705882}%
\pgfsetfillcolor{currentfill}%
\pgfsetfillopacity{0.757991}%
\pgfsetlinewidth{1.003750pt}%
\definecolor{currentstroke}{rgb}{0.121569,0.466667,0.705882}%
\pgfsetstrokecolor{currentstroke}%
\pgfsetstrokeopacity{0.757991}%
\pgfsetdash{}{0pt}%
\pgfpathmoveto{\pgfqpoint{2.477212in}{2.956998in}}%
\pgfpathcurveto{\pgfqpoint{2.485448in}{2.956998in}}{\pgfqpoint{2.493348in}{2.960270in}}{\pgfqpoint{2.499172in}{2.966094in}}%
\pgfpathcurveto{\pgfqpoint{2.504996in}{2.971918in}}{\pgfqpoint{2.508268in}{2.979818in}}{\pgfqpoint{2.508268in}{2.988054in}}%
\pgfpathcurveto{\pgfqpoint{2.508268in}{2.996291in}}{\pgfqpoint{2.504996in}{3.004191in}}{\pgfqpoint{2.499172in}{3.010015in}}%
\pgfpathcurveto{\pgfqpoint{2.493348in}{3.015839in}}{\pgfqpoint{2.485448in}{3.019111in}}{\pgfqpoint{2.477212in}{3.019111in}}%
\pgfpathcurveto{\pgfqpoint{2.468975in}{3.019111in}}{\pgfqpoint{2.461075in}{3.015839in}}{\pgfqpoint{2.455251in}{3.010015in}}%
\pgfpathcurveto{\pgfqpoint{2.449428in}{3.004191in}}{\pgfqpoint{2.446155in}{2.996291in}}{\pgfqpoint{2.446155in}{2.988054in}}%
\pgfpathcurveto{\pgfqpoint{2.446155in}{2.979818in}}{\pgfqpoint{2.449428in}{2.971918in}}{\pgfqpoint{2.455251in}{2.966094in}}%
\pgfpathcurveto{\pgfqpoint{2.461075in}{2.960270in}}{\pgfqpoint{2.468975in}{2.956998in}}{\pgfqpoint{2.477212in}{2.956998in}}%
\pgfpathclose%
\pgfusepath{stroke,fill}%
\end{pgfscope}%
\begin{pgfscope}%
\pgfpathrectangle{\pgfqpoint{0.100000in}{0.212622in}}{\pgfqpoint{3.696000in}{3.696000in}}%
\pgfusepath{clip}%
\pgfsetbuttcap%
\pgfsetroundjoin%
\definecolor{currentfill}{rgb}{0.121569,0.466667,0.705882}%
\pgfsetfillcolor{currentfill}%
\pgfsetfillopacity{0.758279}%
\pgfsetlinewidth{1.003750pt}%
\definecolor{currentstroke}{rgb}{0.121569,0.466667,0.705882}%
\pgfsetstrokecolor{currentstroke}%
\pgfsetstrokeopacity{0.758279}%
\pgfsetdash{}{0pt}%
\pgfpathmoveto{\pgfqpoint{3.155743in}{2.182963in}}%
\pgfpathcurveto{\pgfqpoint{3.163979in}{2.182963in}}{\pgfqpoint{3.171879in}{2.186235in}}{\pgfqpoint{3.177703in}{2.192059in}}%
\pgfpathcurveto{\pgfqpoint{3.183527in}{2.197883in}}{\pgfqpoint{3.186800in}{2.205783in}}{\pgfqpoint{3.186800in}{2.214019in}}%
\pgfpathcurveto{\pgfqpoint{3.186800in}{2.222255in}}{\pgfqpoint{3.183527in}{2.230155in}}{\pgfqpoint{3.177703in}{2.235979in}}%
\pgfpathcurveto{\pgfqpoint{3.171879in}{2.241803in}}{\pgfqpoint{3.163979in}{2.245076in}}{\pgfqpoint{3.155743in}{2.245076in}}%
\pgfpathcurveto{\pgfqpoint{3.147507in}{2.245076in}}{\pgfqpoint{3.139607in}{2.241803in}}{\pgfqpoint{3.133783in}{2.235979in}}%
\pgfpathcurveto{\pgfqpoint{3.127959in}{2.230155in}}{\pgfqpoint{3.124687in}{2.222255in}}{\pgfqpoint{3.124687in}{2.214019in}}%
\pgfpathcurveto{\pgfqpoint{3.124687in}{2.205783in}}{\pgfqpoint{3.127959in}{2.197883in}}{\pgfqpoint{3.133783in}{2.192059in}}%
\pgfpathcurveto{\pgfqpoint{3.139607in}{2.186235in}}{\pgfqpoint{3.147507in}{2.182963in}}{\pgfqpoint{3.155743in}{2.182963in}}%
\pgfpathclose%
\pgfusepath{stroke,fill}%
\end{pgfscope}%
\begin{pgfscope}%
\pgfpathrectangle{\pgfqpoint{0.100000in}{0.212622in}}{\pgfqpoint{3.696000in}{3.696000in}}%
\pgfusepath{clip}%
\pgfsetbuttcap%
\pgfsetroundjoin%
\definecolor{currentfill}{rgb}{0.121569,0.466667,0.705882}%
\pgfsetfillcolor{currentfill}%
\pgfsetfillopacity{0.758872}%
\pgfsetlinewidth{1.003750pt}%
\definecolor{currentstroke}{rgb}{0.121569,0.466667,0.705882}%
\pgfsetstrokecolor{currentstroke}%
\pgfsetstrokeopacity{0.758872}%
\pgfsetdash{}{0pt}%
\pgfpathmoveto{\pgfqpoint{2.481194in}{2.957211in}}%
\pgfpathcurveto{\pgfqpoint{2.489430in}{2.957211in}}{\pgfqpoint{2.497330in}{2.960484in}}{\pgfqpoint{2.503154in}{2.966308in}}%
\pgfpathcurveto{\pgfqpoint{2.508978in}{2.972131in}}{\pgfqpoint{2.512250in}{2.980031in}}{\pgfqpoint{2.512250in}{2.988268in}}%
\pgfpathcurveto{\pgfqpoint{2.512250in}{2.996504in}}{\pgfqpoint{2.508978in}{3.004404in}}{\pgfqpoint{2.503154in}{3.010228in}}%
\pgfpathcurveto{\pgfqpoint{2.497330in}{3.016052in}}{\pgfqpoint{2.489430in}{3.019324in}}{\pgfqpoint{2.481194in}{3.019324in}}%
\pgfpathcurveto{\pgfqpoint{2.472957in}{3.019324in}}{\pgfqpoint{2.465057in}{3.016052in}}{\pgfqpoint{2.459233in}{3.010228in}}%
\pgfpathcurveto{\pgfqpoint{2.453410in}{3.004404in}}{\pgfqpoint{2.450137in}{2.996504in}}{\pgfqpoint{2.450137in}{2.988268in}}%
\pgfpathcurveto{\pgfqpoint{2.450137in}{2.980031in}}{\pgfqpoint{2.453410in}{2.972131in}}{\pgfqpoint{2.459233in}{2.966308in}}%
\pgfpathcurveto{\pgfqpoint{2.465057in}{2.960484in}}{\pgfqpoint{2.472957in}{2.957211in}}{\pgfqpoint{2.481194in}{2.957211in}}%
\pgfpathclose%
\pgfusepath{stroke,fill}%
\end{pgfscope}%
\begin{pgfscope}%
\pgfpathrectangle{\pgfqpoint{0.100000in}{0.212622in}}{\pgfqpoint{3.696000in}{3.696000in}}%
\pgfusepath{clip}%
\pgfsetbuttcap%
\pgfsetroundjoin%
\definecolor{currentfill}{rgb}{0.121569,0.466667,0.705882}%
\pgfsetfillcolor{currentfill}%
\pgfsetfillopacity{0.759048}%
\pgfsetlinewidth{1.003750pt}%
\definecolor{currentstroke}{rgb}{0.121569,0.466667,0.705882}%
\pgfsetstrokecolor{currentstroke}%
\pgfsetstrokeopacity{0.759048}%
\pgfsetdash{}{0pt}%
\pgfpathmoveto{\pgfqpoint{3.153150in}{2.178318in}}%
\pgfpathcurveto{\pgfqpoint{3.161386in}{2.178318in}}{\pgfqpoint{3.169286in}{2.181590in}}{\pgfqpoint{3.175110in}{2.187414in}}%
\pgfpathcurveto{\pgfqpoint{3.180934in}{2.193238in}}{\pgfqpoint{3.184206in}{2.201138in}}{\pgfqpoint{3.184206in}{2.209374in}}%
\pgfpathcurveto{\pgfqpoint{3.184206in}{2.217610in}}{\pgfqpoint{3.180934in}{2.225510in}}{\pgfqpoint{3.175110in}{2.231334in}}%
\pgfpathcurveto{\pgfqpoint{3.169286in}{2.237158in}}{\pgfqpoint{3.161386in}{2.240431in}}{\pgfqpoint{3.153150in}{2.240431in}}%
\pgfpathcurveto{\pgfqpoint{3.144913in}{2.240431in}}{\pgfqpoint{3.137013in}{2.237158in}}{\pgfqpoint{3.131189in}{2.231334in}}%
\pgfpathcurveto{\pgfqpoint{3.125365in}{2.225510in}}{\pgfqpoint{3.122093in}{2.217610in}}{\pgfqpoint{3.122093in}{2.209374in}}%
\pgfpathcurveto{\pgfqpoint{3.122093in}{2.201138in}}{\pgfqpoint{3.125365in}{2.193238in}}{\pgfqpoint{3.131189in}{2.187414in}}%
\pgfpathcurveto{\pgfqpoint{3.137013in}{2.181590in}}{\pgfqpoint{3.144913in}{2.178318in}}{\pgfqpoint{3.153150in}{2.178318in}}%
\pgfpathclose%
\pgfusepath{stroke,fill}%
\end{pgfscope}%
\begin{pgfscope}%
\pgfpathrectangle{\pgfqpoint{0.100000in}{0.212622in}}{\pgfqpoint{3.696000in}{3.696000in}}%
\pgfusepath{clip}%
\pgfsetbuttcap%
\pgfsetroundjoin%
\definecolor{currentfill}{rgb}{0.121569,0.466667,0.705882}%
\pgfsetfillcolor{currentfill}%
\pgfsetfillopacity{0.759556}%
\pgfsetlinewidth{1.003750pt}%
\definecolor{currentstroke}{rgb}{0.121569,0.466667,0.705882}%
\pgfsetstrokecolor{currentstroke}%
\pgfsetstrokeopacity{0.759556}%
\pgfsetdash{}{0pt}%
\pgfpathmoveto{\pgfqpoint{2.483948in}{2.957156in}}%
\pgfpathcurveto{\pgfqpoint{2.492185in}{2.957156in}}{\pgfqpoint{2.500085in}{2.960428in}}{\pgfqpoint{2.505909in}{2.966252in}}%
\pgfpathcurveto{\pgfqpoint{2.511733in}{2.972076in}}{\pgfqpoint{2.515005in}{2.979976in}}{\pgfqpoint{2.515005in}{2.988213in}}%
\pgfpathcurveto{\pgfqpoint{2.515005in}{2.996449in}}{\pgfqpoint{2.511733in}{3.004349in}}{\pgfqpoint{2.505909in}{3.010173in}}%
\pgfpathcurveto{\pgfqpoint{2.500085in}{3.015997in}}{\pgfqpoint{2.492185in}{3.019269in}}{\pgfqpoint{2.483948in}{3.019269in}}%
\pgfpathcurveto{\pgfqpoint{2.475712in}{3.019269in}}{\pgfqpoint{2.467812in}{3.015997in}}{\pgfqpoint{2.461988in}{3.010173in}}%
\pgfpathcurveto{\pgfqpoint{2.456164in}{3.004349in}}{\pgfqpoint{2.452892in}{2.996449in}}{\pgfqpoint{2.452892in}{2.988213in}}%
\pgfpathcurveto{\pgfqpoint{2.452892in}{2.979976in}}{\pgfqpoint{2.456164in}{2.972076in}}{\pgfqpoint{2.461988in}{2.966252in}}%
\pgfpathcurveto{\pgfqpoint{2.467812in}{2.960428in}}{\pgfqpoint{2.475712in}{2.957156in}}{\pgfqpoint{2.483948in}{2.957156in}}%
\pgfpathclose%
\pgfusepath{stroke,fill}%
\end{pgfscope}%
\begin{pgfscope}%
\pgfpathrectangle{\pgfqpoint{0.100000in}{0.212622in}}{\pgfqpoint{3.696000in}{3.696000in}}%
\pgfusepath{clip}%
\pgfsetbuttcap%
\pgfsetroundjoin%
\definecolor{currentfill}{rgb}{0.121569,0.466667,0.705882}%
\pgfsetfillcolor{currentfill}%
\pgfsetfillopacity{0.759718}%
\pgfsetlinewidth{1.003750pt}%
\definecolor{currentstroke}{rgb}{0.121569,0.466667,0.705882}%
\pgfsetstrokecolor{currentstroke}%
\pgfsetstrokeopacity{0.759718}%
\pgfsetdash{}{0pt}%
\pgfpathmoveto{\pgfqpoint{3.152038in}{2.174732in}}%
\pgfpathcurveto{\pgfqpoint{3.160275in}{2.174732in}}{\pgfqpoint{3.168175in}{2.178005in}}{\pgfqpoint{3.173999in}{2.183829in}}%
\pgfpathcurveto{\pgfqpoint{3.179822in}{2.189653in}}{\pgfqpoint{3.183095in}{2.197553in}}{\pgfqpoint{3.183095in}{2.205789in}}%
\pgfpathcurveto{\pgfqpoint{3.183095in}{2.214025in}}{\pgfqpoint{3.179822in}{2.221925in}}{\pgfqpoint{3.173999in}{2.227749in}}%
\pgfpathcurveto{\pgfqpoint{3.168175in}{2.233573in}}{\pgfqpoint{3.160275in}{2.236845in}}{\pgfqpoint{3.152038in}{2.236845in}}%
\pgfpathcurveto{\pgfqpoint{3.143802in}{2.236845in}}{\pgfqpoint{3.135902in}{2.233573in}}{\pgfqpoint{3.130078in}{2.227749in}}%
\pgfpathcurveto{\pgfqpoint{3.124254in}{2.221925in}}{\pgfqpoint{3.120982in}{2.214025in}}{\pgfqpoint{3.120982in}{2.205789in}}%
\pgfpathcurveto{\pgfqpoint{3.120982in}{2.197553in}}{\pgfqpoint{3.124254in}{2.189653in}}{\pgfqpoint{3.130078in}{2.183829in}}%
\pgfpathcurveto{\pgfqpoint{3.135902in}{2.178005in}}{\pgfqpoint{3.143802in}{2.174732in}}{\pgfqpoint{3.152038in}{2.174732in}}%
\pgfpathclose%
\pgfusepath{stroke,fill}%
\end{pgfscope}%
\begin{pgfscope}%
\pgfpathrectangle{\pgfqpoint{0.100000in}{0.212622in}}{\pgfqpoint{3.696000in}{3.696000in}}%
\pgfusepath{clip}%
\pgfsetbuttcap%
\pgfsetroundjoin%
\definecolor{currentfill}{rgb}{0.121569,0.466667,0.705882}%
\pgfsetfillcolor{currentfill}%
\pgfsetfillopacity{0.759739}%
\pgfsetlinewidth{1.003750pt}%
\definecolor{currentstroke}{rgb}{0.121569,0.466667,0.705882}%
\pgfsetstrokecolor{currentstroke}%
\pgfsetstrokeopacity{0.759739}%
\pgfsetdash{}{0pt}%
\pgfpathmoveto{\pgfqpoint{1.094580in}{1.728939in}}%
\pgfpathcurveto{\pgfqpoint{1.102817in}{1.728939in}}{\pgfqpoint{1.110717in}{1.732211in}}{\pgfqpoint{1.116541in}{1.738035in}}%
\pgfpathcurveto{\pgfqpoint{1.122364in}{1.743859in}}{\pgfqpoint{1.125637in}{1.751759in}}{\pgfqpoint{1.125637in}{1.759995in}}%
\pgfpathcurveto{\pgfqpoint{1.125637in}{1.768231in}}{\pgfqpoint{1.122364in}{1.776131in}}{\pgfqpoint{1.116541in}{1.781955in}}%
\pgfpathcurveto{\pgfqpoint{1.110717in}{1.787779in}}{\pgfqpoint{1.102817in}{1.791052in}}{\pgfqpoint{1.094580in}{1.791052in}}%
\pgfpathcurveto{\pgfqpoint{1.086344in}{1.791052in}}{\pgfqpoint{1.078444in}{1.787779in}}{\pgfqpoint{1.072620in}{1.781955in}}%
\pgfpathcurveto{\pgfqpoint{1.066796in}{1.776131in}}{\pgfqpoint{1.063524in}{1.768231in}}{\pgfqpoint{1.063524in}{1.759995in}}%
\pgfpathcurveto{\pgfqpoint{1.063524in}{1.751759in}}{\pgfqpoint{1.066796in}{1.743859in}}{\pgfqpoint{1.072620in}{1.738035in}}%
\pgfpathcurveto{\pgfqpoint{1.078444in}{1.732211in}}{\pgfqpoint{1.086344in}{1.728939in}}{\pgfqpoint{1.094580in}{1.728939in}}%
\pgfpathclose%
\pgfusepath{stroke,fill}%
\end{pgfscope}%
\begin{pgfscope}%
\pgfpathrectangle{\pgfqpoint{0.100000in}{0.212622in}}{\pgfqpoint{3.696000in}{3.696000in}}%
\pgfusepath{clip}%
\pgfsetbuttcap%
\pgfsetroundjoin%
\definecolor{currentfill}{rgb}{0.121569,0.466667,0.705882}%
\pgfsetfillcolor{currentfill}%
\pgfsetfillopacity{0.760288}%
\pgfsetlinewidth{1.003750pt}%
\definecolor{currentstroke}{rgb}{0.121569,0.466667,0.705882}%
\pgfsetstrokecolor{currentstroke}%
\pgfsetstrokeopacity{0.760288}%
\pgfsetdash{}{0pt}%
\pgfpathmoveto{\pgfqpoint{3.151489in}{2.171390in}}%
\pgfpathcurveto{\pgfqpoint{3.159726in}{2.171390in}}{\pgfqpoint{3.167626in}{2.174663in}}{\pgfqpoint{3.173450in}{2.180487in}}%
\pgfpathcurveto{\pgfqpoint{3.179273in}{2.186311in}}{\pgfqpoint{3.182546in}{2.194211in}}{\pgfqpoint{3.182546in}{2.202447in}}%
\pgfpathcurveto{\pgfqpoint{3.182546in}{2.210683in}}{\pgfqpoint{3.179273in}{2.218583in}}{\pgfqpoint{3.173450in}{2.224407in}}%
\pgfpathcurveto{\pgfqpoint{3.167626in}{2.230231in}}{\pgfqpoint{3.159726in}{2.233503in}}{\pgfqpoint{3.151489in}{2.233503in}}%
\pgfpathcurveto{\pgfqpoint{3.143253in}{2.233503in}}{\pgfqpoint{3.135353in}{2.230231in}}{\pgfqpoint{3.129529in}{2.224407in}}%
\pgfpathcurveto{\pgfqpoint{3.123705in}{2.218583in}}{\pgfqpoint{3.120433in}{2.210683in}}{\pgfqpoint{3.120433in}{2.202447in}}%
\pgfpathcurveto{\pgfqpoint{3.120433in}{2.194211in}}{\pgfqpoint{3.123705in}{2.186311in}}{\pgfqpoint{3.129529in}{2.180487in}}%
\pgfpathcurveto{\pgfqpoint{3.135353in}{2.174663in}}{\pgfqpoint{3.143253in}{2.171390in}}{\pgfqpoint{3.151489in}{2.171390in}}%
\pgfpathclose%
\pgfusepath{stroke,fill}%
\end{pgfscope}%
\begin{pgfscope}%
\pgfpathrectangle{\pgfqpoint{0.100000in}{0.212622in}}{\pgfqpoint{3.696000in}{3.696000in}}%
\pgfusepath{clip}%
\pgfsetbuttcap%
\pgfsetroundjoin%
\definecolor{currentfill}{rgb}{0.121569,0.466667,0.705882}%
\pgfsetfillcolor{currentfill}%
\pgfsetfillopacity{0.760634}%
\pgfsetlinewidth{1.003750pt}%
\definecolor{currentstroke}{rgb}{0.121569,0.466667,0.705882}%
\pgfsetstrokecolor{currentstroke}%
\pgfsetstrokeopacity{0.760634}%
\pgfsetdash{}{0pt}%
\pgfpathmoveto{\pgfqpoint{2.488824in}{2.955883in}}%
\pgfpathcurveto{\pgfqpoint{2.497061in}{2.955883in}}{\pgfqpoint{2.504961in}{2.959155in}}{\pgfqpoint{2.510785in}{2.964979in}}%
\pgfpathcurveto{\pgfqpoint{2.516608in}{2.970803in}}{\pgfqpoint{2.519881in}{2.978703in}}{\pgfqpoint{2.519881in}{2.986939in}}%
\pgfpathcurveto{\pgfqpoint{2.519881in}{2.995176in}}{\pgfqpoint{2.516608in}{3.003076in}}{\pgfqpoint{2.510785in}{3.008900in}}%
\pgfpathcurveto{\pgfqpoint{2.504961in}{3.014723in}}{\pgfqpoint{2.497061in}{3.017996in}}{\pgfqpoint{2.488824in}{3.017996in}}%
\pgfpathcurveto{\pgfqpoint{2.480588in}{3.017996in}}{\pgfqpoint{2.472688in}{3.014723in}}{\pgfqpoint{2.466864in}{3.008900in}}%
\pgfpathcurveto{\pgfqpoint{2.461040in}{3.003076in}}{\pgfqpoint{2.457768in}{2.995176in}}{\pgfqpoint{2.457768in}{2.986939in}}%
\pgfpathcurveto{\pgfqpoint{2.457768in}{2.978703in}}{\pgfqpoint{2.461040in}{2.970803in}}{\pgfqpoint{2.466864in}{2.964979in}}%
\pgfpathcurveto{\pgfqpoint{2.472688in}{2.959155in}}{\pgfqpoint{2.480588in}{2.955883in}}{\pgfqpoint{2.488824in}{2.955883in}}%
\pgfpathclose%
\pgfusepath{stroke,fill}%
\end{pgfscope}%
\begin{pgfscope}%
\pgfpathrectangle{\pgfqpoint{0.100000in}{0.212622in}}{\pgfqpoint{3.696000in}{3.696000in}}%
\pgfusepath{clip}%
\pgfsetbuttcap%
\pgfsetroundjoin%
\definecolor{currentfill}{rgb}{0.121569,0.466667,0.705882}%
\pgfsetfillcolor{currentfill}%
\pgfsetfillopacity{0.761332}%
\pgfsetlinewidth{1.003750pt}%
\definecolor{currentstroke}{rgb}{0.121569,0.466667,0.705882}%
\pgfsetstrokecolor{currentstroke}%
\pgfsetstrokeopacity{0.761332}%
\pgfsetdash{}{0pt}%
\pgfpathmoveto{\pgfqpoint{3.149714in}{2.165472in}}%
\pgfpathcurveto{\pgfqpoint{3.157950in}{2.165472in}}{\pgfqpoint{3.165850in}{2.168744in}}{\pgfqpoint{3.171674in}{2.174568in}}%
\pgfpathcurveto{\pgfqpoint{3.177498in}{2.180392in}}{\pgfqpoint{3.180771in}{2.188292in}}{\pgfqpoint{3.180771in}{2.196528in}}%
\pgfpathcurveto{\pgfqpoint{3.180771in}{2.204765in}}{\pgfqpoint{3.177498in}{2.212665in}}{\pgfqpoint{3.171674in}{2.218489in}}%
\pgfpathcurveto{\pgfqpoint{3.165850in}{2.224313in}}{\pgfqpoint{3.157950in}{2.227585in}}{\pgfqpoint{3.149714in}{2.227585in}}%
\pgfpathcurveto{\pgfqpoint{3.141478in}{2.227585in}}{\pgfqpoint{3.133578in}{2.224313in}}{\pgfqpoint{3.127754in}{2.218489in}}%
\pgfpathcurveto{\pgfqpoint{3.121930in}{2.212665in}}{\pgfqpoint{3.118658in}{2.204765in}}{\pgfqpoint{3.118658in}{2.196528in}}%
\pgfpathcurveto{\pgfqpoint{3.118658in}{2.188292in}}{\pgfqpoint{3.121930in}{2.180392in}}{\pgfqpoint{3.127754in}{2.174568in}}%
\pgfpathcurveto{\pgfqpoint{3.133578in}{2.168744in}}{\pgfqpoint{3.141478in}{2.165472in}}{\pgfqpoint{3.149714in}{2.165472in}}%
\pgfpathclose%
\pgfusepath{stroke,fill}%
\end{pgfscope}%
\begin{pgfscope}%
\pgfpathrectangle{\pgfqpoint{0.100000in}{0.212622in}}{\pgfqpoint{3.696000in}{3.696000in}}%
\pgfusepath{clip}%
\pgfsetbuttcap%
\pgfsetroundjoin%
\definecolor{currentfill}{rgb}{0.121569,0.466667,0.705882}%
\pgfsetfillcolor{currentfill}%
\pgfsetfillopacity{0.761561}%
\pgfsetlinewidth{1.003750pt}%
\definecolor{currentstroke}{rgb}{0.121569,0.466667,0.705882}%
\pgfsetstrokecolor{currentstroke}%
\pgfsetstrokeopacity{0.761561}%
\pgfsetdash{}{0pt}%
\pgfpathmoveto{\pgfqpoint{2.492907in}{2.954990in}}%
\pgfpathcurveto{\pgfqpoint{2.501143in}{2.954990in}}{\pgfqpoint{2.509043in}{2.958263in}}{\pgfqpoint{2.514867in}{2.964087in}}%
\pgfpathcurveto{\pgfqpoint{2.520691in}{2.969910in}}{\pgfqpoint{2.523964in}{2.977810in}}{\pgfqpoint{2.523964in}{2.986047in}}%
\pgfpathcurveto{\pgfqpoint{2.523964in}{2.994283in}}{\pgfqpoint{2.520691in}{3.002183in}}{\pgfqpoint{2.514867in}{3.008007in}}%
\pgfpathcurveto{\pgfqpoint{2.509043in}{3.013831in}}{\pgfqpoint{2.501143in}{3.017103in}}{\pgfqpoint{2.492907in}{3.017103in}}%
\pgfpathcurveto{\pgfqpoint{2.484671in}{3.017103in}}{\pgfqpoint{2.476771in}{3.013831in}}{\pgfqpoint{2.470947in}{3.008007in}}%
\pgfpathcurveto{\pgfqpoint{2.465123in}{3.002183in}}{\pgfqpoint{2.461851in}{2.994283in}}{\pgfqpoint{2.461851in}{2.986047in}}%
\pgfpathcurveto{\pgfqpoint{2.461851in}{2.977810in}}{\pgfqpoint{2.465123in}{2.969910in}}{\pgfqpoint{2.470947in}{2.964087in}}%
\pgfpathcurveto{\pgfqpoint{2.476771in}{2.958263in}}{\pgfqpoint{2.484671in}{2.954990in}}{\pgfqpoint{2.492907in}{2.954990in}}%
\pgfpathclose%
\pgfusepath{stroke,fill}%
\end{pgfscope}%
\begin{pgfscope}%
\pgfpathrectangle{\pgfqpoint{0.100000in}{0.212622in}}{\pgfqpoint{3.696000in}{3.696000in}}%
\pgfusepath{clip}%
\pgfsetbuttcap%
\pgfsetroundjoin%
\definecolor{currentfill}{rgb}{0.121569,0.466667,0.705882}%
\pgfsetfillcolor{currentfill}%
\pgfsetfillopacity{0.761727}%
\pgfsetlinewidth{1.003750pt}%
\definecolor{currentstroke}{rgb}{0.121569,0.466667,0.705882}%
\pgfsetstrokecolor{currentstroke}%
\pgfsetstrokeopacity{0.761727}%
\pgfsetdash{}{0pt}%
\pgfpathmoveto{\pgfqpoint{1.083699in}{1.710830in}}%
\pgfpathcurveto{\pgfqpoint{1.091935in}{1.710830in}}{\pgfqpoint{1.099835in}{1.714103in}}{\pgfqpoint{1.105659in}{1.719927in}}%
\pgfpathcurveto{\pgfqpoint{1.111483in}{1.725750in}}{\pgfqpoint{1.114755in}{1.733651in}}{\pgfqpoint{1.114755in}{1.741887in}}%
\pgfpathcurveto{\pgfqpoint{1.114755in}{1.750123in}}{\pgfqpoint{1.111483in}{1.758023in}}{\pgfqpoint{1.105659in}{1.763847in}}%
\pgfpathcurveto{\pgfqpoint{1.099835in}{1.769671in}}{\pgfqpoint{1.091935in}{1.772943in}}{\pgfqpoint{1.083699in}{1.772943in}}%
\pgfpathcurveto{\pgfqpoint{1.075462in}{1.772943in}}{\pgfqpoint{1.067562in}{1.769671in}}{\pgfqpoint{1.061738in}{1.763847in}}%
\pgfpathcurveto{\pgfqpoint{1.055914in}{1.758023in}}{\pgfqpoint{1.052642in}{1.750123in}}{\pgfqpoint{1.052642in}{1.741887in}}%
\pgfpathcurveto{\pgfqpoint{1.052642in}{1.733651in}}{\pgfqpoint{1.055914in}{1.725750in}}{\pgfqpoint{1.061738in}{1.719927in}}%
\pgfpathcurveto{\pgfqpoint{1.067562in}{1.714103in}}{\pgfqpoint{1.075462in}{1.710830in}}{\pgfqpoint{1.083699in}{1.710830in}}%
\pgfpathclose%
\pgfusepath{stroke,fill}%
\end{pgfscope}%
\begin{pgfscope}%
\pgfpathrectangle{\pgfqpoint{0.100000in}{0.212622in}}{\pgfqpoint{3.696000in}{3.696000in}}%
\pgfusepath{clip}%
\pgfsetbuttcap%
\pgfsetroundjoin%
\definecolor{currentfill}{rgb}{0.121569,0.466667,0.705882}%
\pgfsetfillcolor{currentfill}%
\pgfsetfillopacity{0.761964}%
\pgfsetlinewidth{1.003750pt}%
\definecolor{currentstroke}{rgb}{0.121569,0.466667,0.705882}%
\pgfsetstrokecolor{currentstroke}%
\pgfsetstrokeopacity{0.761964}%
\pgfsetdash{}{0pt}%
\pgfpathmoveto{\pgfqpoint{3.147345in}{2.162163in}}%
\pgfpathcurveto{\pgfqpoint{3.155581in}{2.162163in}}{\pgfqpoint{3.163481in}{2.165436in}}{\pgfqpoint{3.169305in}{2.171260in}}%
\pgfpathcurveto{\pgfqpoint{3.175129in}{2.177084in}}{\pgfqpoint{3.178401in}{2.184984in}}{\pgfqpoint{3.178401in}{2.193220in}}%
\pgfpathcurveto{\pgfqpoint{3.178401in}{2.201456in}}{\pgfqpoint{3.175129in}{2.209356in}}{\pgfqpoint{3.169305in}{2.215180in}}%
\pgfpathcurveto{\pgfqpoint{3.163481in}{2.221004in}}{\pgfqpoint{3.155581in}{2.224276in}}{\pgfqpoint{3.147345in}{2.224276in}}%
\pgfpathcurveto{\pgfqpoint{3.139108in}{2.224276in}}{\pgfqpoint{3.131208in}{2.221004in}}{\pgfqpoint{3.125384in}{2.215180in}}%
\pgfpathcurveto{\pgfqpoint{3.119560in}{2.209356in}}{\pgfqpoint{3.116288in}{2.201456in}}{\pgfqpoint{3.116288in}{2.193220in}}%
\pgfpathcurveto{\pgfqpoint{3.116288in}{2.184984in}}{\pgfqpoint{3.119560in}{2.177084in}}{\pgfqpoint{3.125384in}{2.171260in}}%
\pgfpathcurveto{\pgfqpoint{3.131208in}{2.165436in}}{\pgfqpoint{3.139108in}{2.162163in}}{\pgfqpoint{3.147345in}{2.162163in}}%
\pgfpathclose%
\pgfusepath{stroke,fill}%
\end{pgfscope}%
\begin{pgfscope}%
\pgfpathrectangle{\pgfqpoint{0.100000in}{0.212622in}}{\pgfqpoint{3.696000in}{3.696000in}}%
\pgfusepath{clip}%
\pgfsetbuttcap%
\pgfsetroundjoin%
\definecolor{currentfill}{rgb}{0.121569,0.466667,0.705882}%
\pgfsetfillcolor{currentfill}%
\pgfsetfillopacity{0.762617}%
\pgfsetlinewidth{1.003750pt}%
\definecolor{currentstroke}{rgb}{0.121569,0.466667,0.705882}%
\pgfsetstrokecolor{currentstroke}%
\pgfsetstrokeopacity{0.762617}%
\pgfsetdash{}{0pt}%
\pgfpathmoveto{\pgfqpoint{2.496482in}{2.954409in}}%
\pgfpathcurveto{\pgfqpoint{2.504719in}{2.954409in}}{\pgfqpoint{2.512619in}{2.957681in}}{\pgfqpoint{2.518443in}{2.963505in}}%
\pgfpathcurveto{\pgfqpoint{2.524267in}{2.969329in}}{\pgfqpoint{2.527539in}{2.977229in}}{\pgfqpoint{2.527539in}{2.985465in}}%
\pgfpathcurveto{\pgfqpoint{2.527539in}{2.993701in}}{\pgfqpoint{2.524267in}{3.001601in}}{\pgfqpoint{2.518443in}{3.007425in}}%
\pgfpathcurveto{\pgfqpoint{2.512619in}{3.013249in}}{\pgfqpoint{2.504719in}{3.016522in}}{\pgfqpoint{2.496482in}{3.016522in}}%
\pgfpathcurveto{\pgfqpoint{2.488246in}{3.016522in}}{\pgfqpoint{2.480346in}{3.013249in}}{\pgfqpoint{2.474522in}{3.007425in}}%
\pgfpathcurveto{\pgfqpoint{2.468698in}{3.001601in}}{\pgfqpoint{2.465426in}{2.993701in}}{\pgfqpoint{2.465426in}{2.985465in}}%
\pgfpathcurveto{\pgfqpoint{2.465426in}{2.977229in}}{\pgfqpoint{2.468698in}{2.969329in}}{\pgfqpoint{2.474522in}{2.963505in}}%
\pgfpathcurveto{\pgfqpoint{2.480346in}{2.957681in}}{\pgfqpoint{2.488246in}{2.954409in}}{\pgfqpoint{2.496482in}{2.954409in}}%
\pgfpathclose%
\pgfusepath{stroke,fill}%
\end{pgfscope}%
\begin{pgfscope}%
\pgfpathrectangle{\pgfqpoint{0.100000in}{0.212622in}}{\pgfqpoint{3.696000in}{3.696000in}}%
\pgfusepath{clip}%
\pgfsetbuttcap%
\pgfsetroundjoin%
\definecolor{currentfill}{rgb}{0.121569,0.466667,0.705882}%
\pgfsetfillcolor{currentfill}%
\pgfsetfillopacity{0.763145}%
\pgfsetlinewidth{1.003750pt}%
\definecolor{currentstroke}{rgb}{0.121569,0.466667,0.705882}%
\pgfsetstrokecolor{currentstroke}%
\pgfsetstrokeopacity{0.763145}%
\pgfsetdash{}{0pt}%
\pgfpathmoveto{\pgfqpoint{3.143063in}{2.156236in}}%
\pgfpathcurveto{\pgfqpoint{3.151299in}{2.156236in}}{\pgfqpoint{3.159199in}{2.159509in}}{\pgfqpoint{3.165023in}{2.165333in}}%
\pgfpathcurveto{\pgfqpoint{3.170847in}{2.171156in}}{\pgfqpoint{3.174119in}{2.179057in}}{\pgfqpoint{3.174119in}{2.187293in}}%
\pgfpathcurveto{\pgfqpoint{3.174119in}{2.195529in}}{\pgfqpoint{3.170847in}{2.203429in}}{\pgfqpoint{3.165023in}{2.209253in}}%
\pgfpathcurveto{\pgfqpoint{3.159199in}{2.215077in}}{\pgfqpoint{3.151299in}{2.218349in}}{\pgfqpoint{3.143063in}{2.218349in}}%
\pgfpathcurveto{\pgfqpoint{3.134827in}{2.218349in}}{\pgfqpoint{3.126926in}{2.215077in}}{\pgfqpoint{3.121103in}{2.209253in}}%
\pgfpathcurveto{\pgfqpoint{3.115279in}{2.203429in}}{\pgfqpoint{3.112006in}{2.195529in}}{\pgfqpoint{3.112006in}{2.187293in}}%
\pgfpathcurveto{\pgfqpoint{3.112006in}{2.179057in}}{\pgfqpoint{3.115279in}{2.171156in}}{\pgfqpoint{3.121103in}{2.165333in}}%
\pgfpathcurveto{\pgfqpoint{3.126926in}{2.159509in}}{\pgfqpoint{3.134827in}{2.156236in}}{\pgfqpoint{3.143063in}{2.156236in}}%
\pgfpathclose%
\pgfusepath{stroke,fill}%
\end{pgfscope}%
\begin{pgfscope}%
\pgfpathrectangle{\pgfqpoint{0.100000in}{0.212622in}}{\pgfqpoint{3.696000in}{3.696000in}}%
\pgfusepath{clip}%
\pgfsetbuttcap%
\pgfsetroundjoin%
\definecolor{currentfill}{rgb}{0.121569,0.466667,0.705882}%
\pgfsetfillcolor{currentfill}%
\pgfsetfillopacity{0.763417}%
\pgfsetlinewidth{1.003750pt}%
\definecolor{currentstroke}{rgb}{0.121569,0.466667,0.705882}%
\pgfsetstrokecolor{currentstroke}%
\pgfsetstrokeopacity{0.763417}%
\pgfsetdash{}{0pt}%
\pgfpathmoveto{\pgfqpoint{2.499582in}{2.953504in}}%
\pgfpathcurveto{\pgfqpoint{2.507818in}{2.953504in}}{\pgfqpoint{2.515718in}{2.956777in}}{\pgfqpoint{2.521542in}{2.962601in}}%
\pgfpathcurveto{\pgfqpoint{2.527366in}{2.968425in}}{\pgfqpoint{2.530638in}{2.976325in}}{\pgfqpoint{2.530638in}{2.984561in}}%
\pgfpathcurveto{\pgfqpoint{2.530638in}{2.992797in}}{\pgfqpoint{2.527366in}{3.000697in}}{\pgfqpoint{2.521542in}{3.006521in}}%
\pgfpathcurveto{\pgfqpoint{2.515718in}{3.012345in}}{\pgfqpoint{2.507818in}{3.015617in}}{\pgfqpoint{2.499582in}{3.015617in}}%
\pgfpathcurveto{\pgfqpoint{2.491345in}{3.015617in}}{\pgfqpoint{2.483445in}{3.012345in}}{\pgfqpoint{2.477621in}{3.006521in}}%
\pgfpathcurveto{\pgfqpoint{2.471797in}{3.000697in}}{\pgfqpoint{2.468525in}{2.992797in}}{\pgfqpoint{2.468525in}{2.984561in}}%
\pgfpathcurveto{\pgfqpoint{2.468525in}{2.976325in}}{\pgfqpoint{2.471797in}{2.968425in}}{\pgfqpoint{2.477621in}{2.962601in}}%
\pgfpathcurveto{\pgfqpoint{2.483445in}{2.956777in}}{\pgfqpoint{2.491345in}{2.953504in}}{\pgfqpoint{2.499582in}{2.953504in}}%
\pgfpathclose%
\pgfusepath{stroke,fill}%
\end{pgfscope}%
\begin{pgfscope}%
\pgfpathrectangle{\pgfqpoint{0.100000in}{0.212622in}}{\pgfqpoint{3.696000in}{3.696000in}}%
\pgfusepath{clip}%
\pgfsetbuttcap%
\pgfsetroundjoin%
\definecolor{currentfill}{rgb}{0.121569,0.466667,0.705882}%
\pgfsetfillcolor{currentfill}%
\pgfsetfillopacity{0.763661}%
\pgfsetlinewidth{1.003750pt}%
\definecolor{currentstroke}{rgb}{0.121569,0.466667,0.705882}%
\pgfsetstrokecolor{currentstroke}%
\pgfsetstrokeopacity{0.763661}%
\pgfsetdash{}{0pt}%
\pgfpathmoveto{\pgfqpoint{1.071528in}{1.693388in}}%
\pgfpathcurveto{\pgfqpoint{1.079764in}{1.693388in}}{\pgfqpoint{1.087664in}{1.696661in}}{\pgfqpoint{1.093488in}{1.702485in}}%
\pgfpathcurveto{\pgfqpoint{1.099312in}{1.708309in}}{\pgfqpoint{1.102584in}{1.716209in}}{\pgfqpoint{1.102584in}{1.724445in}}%
\pgfpathcurveto{\pgfqpoint{1.102584in}{1.732681in}}{\pgfqpoint{1.099312in}{1.740581in}}{\pgfqpoint{1.093488in}{1.746405in}}%
\pgfpathcurveto{\pgfqpoint{1.087664in}{1.752229in}}{\pgfqpoint{1.079764in}{1.755501in}}{\pgfqpoint{1.071528in}{1.755501in}}%
\pgfpathcurveto{\pgfqpoint{1.063292in}{1.755501in}}{\pgfqpoint{1.055391in}{1.752229in}}{\pgfqpoint{1.049568in}{1.746405in}}%
\pgfpathcurveto{\pgfqpoint{1.043744in}{1.740581in}}{\pgfqpoint{1.040471in}{1.732681in}}{\pgfqpoint{1.040471in}{1.724445in}}%
\pgfpathcurveto{\pgfqpoint{1.040471in}{1.716209in}}{\pgfqpoint{1.043744in}{1.708309in}}{\pgfqpoint{1.049568in}{1.702485in}}%
\pgfpathcurveto{\pgfqpoint{1.055391in}{1.696661in}}{\pgfqpoint{1.063292in}{1.693388in}}{\pgfqpoint{1.071528in}{1.693388in}}%
\pgfpathclose%
\pgfusepath{stroke,fill}%
\end{pgfscope}%
\begin{pgfscope}%
\pgfpathrectangle{\pgfqpoint{0.100000in}{0.212622in}}{\pgfqpoint{3.696000in}{3.696000in}}%
\pgfusepath{clip}%
\pgfsetbuttcap%
\pgfsetroundjoin%
\definecolor{currentfill}{rgb}{0.121569,0.466667,0.705882}%
\pgfsetfillcolor{currentfill}%
\pgfsetfillopacity{0.764042}%
\pgfsetlinewidth{1.003750pt}%
\definecolor{currentstroke}{rgb}{0.121569,0.466667,0.705882}%
\pgfsetstrokecolor{currentstroke}%
\pgfsetstrokeopacity{0.764042}%
\pgfsetdash{}{0pt}%
\pgfpathmoveto{\pgfqpoint{2.501720in}{2.952903in}}%
\pgfpathcurveto{\pgfqpoint{2.509956in}{2.952903in}}{\pgfqpoint{2.517856in}{2.956176in}}{\pgfqpoint{2.523680in}{2.961999in}}%
\pgfpathcurveto{\pgfqpoint{2.529504in}{2.967823in}}{\pgfqpoint{2.532776in}{2.975723in}}{\pgfqpoint{2.532776in}{2.983960in}}%
\pgfpathcurveto{\pgfqpoint{2.532776in}{2.992196in}}{\pgfqpoint{2.529504in}{3.000096in}}{\pgfqpoint{2.523680in}{3.005920in}}%
\pgfpathcurveto{\pgfqpoint{2.517856in}{3.011744in}}{\pgfqpoint{2.509956in}{3.015016in}}{\pgfqpoint{2.501720in}{3.015016in}}%
\pgfpathcurveto{\pgfqpoint{2.493483in}{3.015016in}}{\pgfqpoint{2.485583in}{3.011744in}}{\pgfqpoint{2.479759in}{3.005920in}}%
\pgfpathcurveto{\pgfqpoint{2.473935in}{3.000096in}}{\pgfqpoint{2.470663in}{2.992196in}}{\pgfqpoint{2.470663in}{2.983960in}}%
\pgfpathcurveto{\pgfqpoint{2.470663in}{2.975723in}}{\pgfqpoint{2.473935in}{2.967823in}}{\pgfqpoint{2.479759in}{2.961999in}}%
\pgfpathcurveto{\pgfqpoint{2.485583in}{2.956176in}}{\pgfqpoint{2.493483in}{2.952903in}}{\pgfqpoint{2.501720in}{2.952903in}}%
\pgfpathclose%
\pgfusepath{stroke,fill}%
\end{pgfscope}%
\begin{pgfscope}%
\pgfpathrectangle{\pgfqpoint{0.100000in}{0.212622in}}{\pgfqpoint{3.696000in}{3.696000in}}%
\pgfusepath{clip}%
\pgfsetbuttcap%
\pgfsetroundjoin%
\definecolor{currentfill}{rgb}{0.121569,0.466667,0.705882}%
\pgfsetfillcolor{currentfill}%
\pgfsetfillopacity{0.764155}%
\pgfsetlinewidth{1.003750pt}%
\definecolor{currentstroke}{rgb}{0.121569,0.466667,0.705882}%
\pgfsetstrokecolor{currentstroke}%
\pgfsetstrokeopacity{0.764155}%
\pgfsetdash{}{0pt}%
\pgfpathmoveto{\pgfqpoint{3.140554in}{2.150633in}}%
\pgfpathcurveto{\pgfqpoint{3.148790in}{2.150633in}}{\pgfqpoint{3.156691in}{2.153905in}}{\pgfqpoint{3.162514in}{2.159729in}}%
\pgfpathcurveto{\pgfqpoint{3.168338in}{2.165553in}}{\pgfqpoint{3.171611in}{2.173453in}}{\pgfqpoint{3.171611in}{2.181689in}}%
\pgfpathcurveto{\pgfqpoint{3.171611in}{2.189926in}}{\pgfqpoint{3.168338in}{2.197826in}}{\pgfqpoint{3.162514in}{2.203649in}}%
\pgfpathcurveto{\pgfqpoint{3.156691in}{2.209473in}}{\pgfqpoint{3.148790in}{2.212746in}}{\pgfqpoint{3.140554in}{2.212746in}}%
\pgfpathcurveto{\pgfqpoint{3.132318in}{2.212746in}}{\pgfqpoint{3.124418in}{2.209473in}}{\pgfqpoint{3.118594in}{2.203649in}}%
\pgfpathcurveto{\pgfqpoint{3.112770in}{2.197826in}}{\pgfqpoint{3.109498in}{2.189926in}}{\pgfqpoint{3.109498in}{2.181689in}}%
\pgfpathcurveto{\pgfqpoint{3.109498in}{2.173453in}}{\pgfqpoint{3.112770in}{2.165553in}}{\pgfqpoint{3.118594in}{2.159729in}}%
\pgfpathcurveto{\pgfqpoint{3.124418in}{2.153905in}}{\pgfqpoint{3.132318in}{2.150633in}}{\pgfqpoint{3.140554in}{2.150633in}}%
\pgfpathclose%
\pgfusepath{stroke,fill}%
\end{pgfscope}%
\begin{pgfscope}%
\pgfpathrectangle{\pgfqpoint{0.100000in}{0.212622in}}{\pgfqpoint{3.696000in}{3.696000in}}%
\pgfusepath{clip}%
\pgfsetbuttcap%
\pgfsetroundjoin%
\definecolor{currentfill}{rgb}{0.121569,0.466667,0.705882}%
\pgfsetfillcolor{currentfill}%
\pgfsetfillopacity{0.764923}%
\pgfsetlinewidth{1.003750pt}%
\definecolor{currentstroke}{rgb}{0.121569,0.466667,0.705882}%
\pgfsetstrokecolor{currentstroke}%
\pgfsetstrokeopacity{0.764923}%
\pgfsetdash{}{0pt}%
\pgfpathmoveto{\pgfqpoint{3.139638in}{2.146411in}}%
\pgfpathcurveto{\pgfqpoint{3.147874in}{2.146411in}}{\pgfqpoint{3.155774in}{2.149683in}}{\pgfqpoint{3.161598in}{2.155507in}}%
\pgfpathcurveto{\pgfqpoint{3.167422in}{2.161331in}}{\pgfqpoint{3.170695in}{2.169231in}}{\pgfqpoint{3.170695in}{2.177467in}}%
\pgfpathcurveto{\pgfqpoint{3.170695in}{2.185703in}}{\pgfqpoint{3.167422in}{2.193603in}}{\pgfqpoint{3.161598in}{2.199427in}}%
\pgfpathcurveto{\pgfqpoint{3.155774in}{2.205251in}}{\pgfqpoint{3.147874in}{2.208524in}}{\pgfqpoint{3.139638in}{2.208524in}}%
\pgfpathcurveto{\pgfqpoint{3.131402in}{2.208524in}}{\pgfqpoint{3.123502in}{2.205251in}}{\pgfqpoint{3.117678in}{2.199427in}}%
\pgfpathcurveto{\pgfqpoint{3.111854in}{2.193603in}}{\pgfqpoint{3.108582in}{2.185703in}}{\pgfqpoint{3.108582in}{2.177467in}}%
\pgfpathcurveto{\pgfqpoint{3.108582in}{2.169231in}}{\pgfqpoint{3.111854in}{2.161331in}}{\pgfqpoint{3.117678in}{2.155507in}}%
\pgfpathcurveto{\pgfqpoint{3.123502in}{2.149683in}}{\pgfqpoint{3.131402in}{2.146411in}}{\pgfqpoint{3.139638in}{2.146411in}}%
\pgfpathclose%
\pgfusepath{stroke,fill}%
\end{pgfscope}%
\begin{pgfscope}%
\pgfpathrectangle{\pgfqpoint{0.100000in}{0.212622in}}{\pgfqpoint{3.696000in}{3.696000in}}%
\pgfusepath{clip}%
\pgfsetbuttcap%
\pgfsetroundjoin%
\definecolor{currentfill}{rgb}{0.121569,0.466667,0.705882}%
\pgfsetfillcolor{currentfill}%
\pgfsetfillopacity{0.765164}%
\pgfsetlinewidth{1.003750pt}%
\definecolor{currentstroke}{rgb}{0.121569,0.466667,0.705882}%
\pgfsetstrokecolor{currentstroke}%
\pgfsetstrokeopacity{0.765164}%
\pgfsetdash{}{0pt}%
\pgfpathmoveto{\pgfqpoint{2.505547in}{2.951580in}}%
\pgfpathcurveto{\pgfqpoint{2.513783in}{2.951580in}}{\pgfqpoint{2.521683in}{2.954852in}}{\pgfqpoint{2.527507in}{2.960676in}}%
\pgfpathcurveto{\pgfqpoint{2.533331in}{2.966500in}}{\pgfqpoint{2.536604in}{2.974400in}}{\pgfqpoint{2.536604in}{2.982636in}}%
\pgfpathcurveto{\pgfqpoint{2.536604in}{2.990872in}}{\pgfqpoint{2.533331in}{2.998772in}}{\pgfqpoint{2.527507in}{3.004596in}}%
\pgfpathcurveto{\pgfqpoint{2.521683in}{3.010420in}}{\pgfqpoint{2.513783in}{3.013693in}}{\pgfqpoint{2.505547in}{3.013693in}}%
\pgfpathcurveto{\pgfqpoint{2.497311in}{3.013693in}}{\pgfqpoint{2.489411in}{3.010420in}}{\pgfqpoint{2.483587in}{3.004596in}}%
\pgfpathcurveto{\pgfqpoint{2.477763in}{2.998772in}}{\pgfqpoint{2.474491in}{2.990872in}}{\pgfqpoint{2.474491in}{2.982636in}}%
\pgfpathcurveto{\pgfqpoint{2.474491in}{2.974400in}}{\pgfqpoint{2.477763in}{2.966500in}}{\pgfqpoint{2.483587in}{2.960676in}}%
\pgfpathcurveto{\pgfqpoint{2.489411in}{2.954852in}}{\pgfqpoint{2.497311in}{2.951580in}}{\pgfqpoint{2.505547in}{2.951580in}}%
\pgfpathclose%
\pgfusepath{stroke,fill}%
\end{pgfscope}%
\begin{pgfscope}%
\pgfpathrectangle{\pgfqpoint{0.100000in}{0.212622in}}{\pgfqpoint{3.696000in}{3.696000in}}%
\pgfusepath{clip}%
\pgfsetbuttcap%
\pgfsetroundjoin%
\definecolor{currentfill}{rgb}{0.121569,0.466667,0.705882}%
\pgfsetfillcolor{currentfill}%
\pgfsetfillopacity{0.765555}%
\pgfsetlinewidth{1.003750pt}%
\definecolor{currentstroke}{rgb}{0.121569,0.466667,0.705882}%
\pgfsetstrokecolor{currentstroke}%
\pgfsetstrokeopacity{0.765555}%
\pgfsetdash{}{0pt}%
\pgfpathmoveto{\pgfqpoint{3.138822in}{2.142645in}}%
\pgfpathcurveto{\pgfqpoint{3.147058in}{2.142645in}}{\pgfqpoint{3.154958in}{2.145917in}}{\pgfqpoint{3.160782in}{2.151741in}}%
\pgfpathcurveto{\pgfqpoint{3.166606in}{2.157565in}}{\pgfqpoint{3.169878in}{2.165465in}}{\pgfqpoint{3.169878in}{2.173701in}}%
\pgfpathcurveto{\pgfqpoint{3.169878in}{2.181937in}}{\pgfqpoint{3.166606in}{2.189837in}}{\pgfqpoint{3.160782in}{2.195661in}}%
\pgfpathcurveto{\pgfqpoint{3.154958in}{2.201485in}}{\pgfqpoint{3.147058in}{2.204758in}}{\pgfqpoint{3.138822in}{2.204758in}}%
\pgfpathcurveto{\pgfqpoint{3.130585in}{2.204758in}}{\pgfqpoint{3.122685in}{2.201485in}}{\pgfqpoint{3.116862in}{2.195661in}}%
\pgfpathcurveto{\pgfqpoint{3.111038in}{2.189837in}}{\pgfqpoint{3.107765in}{2.181937in}}{\pgfqpoint{3.107765in}{2.173701in}}%
\pgfpathcurveto{\pgfqpoint{3.107765in}{2.165465in}}{\pgfqpoint{3.111038in}{2.157565in}}{\pgfqpoint{3.116862in}{2.151741in}}%
\pgfpathcurveto{\pgfqpoint{3.122685in}{2.145917in}}{\pgfqpoint{3.130585in}{2.142645in}}{\pgfqpoint{3.138822in}{2.142645in}}%
\pgfpathclose%
\pgfusepath{stroke,fill}%
\end{pgfscope}%
\begin{pgfscope}%
\pgfpathrectangle{\pgfqpoint{0.100000in}{0.212622in}}{\pgfqpoint{3.696000in}{3.696000in}}%
\pgfusepath{clip}%
\pgfsetbuttcap%
\pgfsetroundjoin%
\definecolor{currentfill}{rgb}{0.121569,0.466667,0.705882}%
\pgfsetfillcolor{currentfill}%
\pgfsetfillopacity{0.765960}%
\pgfsetlinewidth{1.003750pt}%
\definecolor{currentstroke}{rgb}{0.121569,0.466667,0.705882}%
\pgfsetstrokecolor{currentstroke}%
\pgfsetstrokeopacity{0.765960}%
\pgfsetdash{}{0pt}%
\pgfpathmoveto{\pgfqpoint{1.060090in}{1.674284in}}%
\pgfpathcurveto{\pgfqpoint{1.068326in}{1.674284in}}{\pgfqpoint{1.076226in}{1.677556in}}{\pgfqpoint{1.082050in}{1.683380in}}%
\pgfpathcurveto{\pgfqpoint{1.087874in}{1.689204in}}{\pgfqpoint{1.091146in}{1.697104in}}{\pgfqpoint{1.091146in}{1.705340in}}%
\pgfpathcurveto{\pgfqpoint{1.091146in}{1.713576in}}{\pgfqpoint{1.087874in}{1.721476in}}{\pgfqpoint{1.082050in}{1.727300in}}%
\pgfpathcurveto{\pgfqpoint{1.076226in}{1.733124in}}{\pgfqpoint{1.068326in}{1.736397in}}{\pgfqpoint{1.060090in}{1.736397in}}%
\pgfpathcurveto{\pgfqpoint{1.051854in}{1.736397in}}{\pgfqpoint{1.043954in}{1.733124in}}{\pgfqpoint{1.038130in}{1.727300in}}%
\pgfpathcurveto{\pgfqpoint{1.032306in}{1.721476in}}{\pgfqpoint{1.029033in}{1.713576in}}{\pgfqpoint{1.029033in}{1.705340in}}%
\pgfpathcurveto{\pgfqpoint{1.029033in}{1.697104in}}{\pgfqpoint{1.032306in}{1.689204in}}{\pgfqpoint{1.038130in}{1.683380in}}%
\pgfpathcurveto{\pgfqpoint{1.043954in}{1.677556in}}{\pgfqpoint{1.051854in}{1.674284in}}{\pgfqpoint{1.060090in}{1.674284in}}%
\pgfpathclose%
\pgfusepath{stroke,fill}%
\end{pgfscope}%
\begin{pgfscope}%
\pgfpathrectangle{\pgfqpoint{0.100000in}{0.212622in}}{\pgfqpoint{3.696000in}{3.696000in}}%
\pgfusepath{clip}%
\pgfsetbuttcap%
\pgfsetroundjoin%
\definecolor{currentfill}{rgb}{0.121569,0.466667,0.705882}%
\pgfsetfillcolor{currentfill}%
\pgfsetfillopacity{0.765970}%
\pgfsetlinewidth{1.003750pt}%
\definecolor{currentstroke}{rgb}{0.121569,0.466667,0.705882}%
\pgfsetstrokecolor{currentstroke}%
\pgfsetstrokeopacity{0.765970}%
\pgfsetdash{}{0pt}%
\pgfpathmoveto{\pgfqpoint{2.508319in}{2.950315in}}%
\pgfpathcurveto{\pgfqpoint{2.516555in}{2.950315in}}{\pgfqpoint{2.524455in}{2.953588in}}{\pgfqpoint{2.530279in}{2.959412in}}%
\pgfpathcurveto{\pgfqpoint{2.536103in}{2.965236in}}{\pgfqpoint{2.539375in}{2.973136in}}{\pgfqpoint{2.539375in}{2.981372in}}%
\pgfpathcurveto{\pgfqpoint{2.539375in}{2.989608in}}{\pgfqpoint{2.536103in}{2.997508in}}{\pgfqpoint{2.530279in}{3.003332in}}%
\pgfpathcurveto{\pgfqpoint{2.524455in}{3.009156in}}{\pgfqpoint{2.516555in}{3.012428in}}{\pgfqpoint{2.508319in}{3.012428in}}%
\pgfpathcurveto{\pgfqpoint{2.500083in}{3.012428in}}{\pgfqpoint{2.492183in}{3.009156in}}{\pgfqpoint{2.486359in}{3.003332in}}%
\pgfpathcurveto{\pgfqpoint{2.480535in}{2.997508in}}{\pgfqpoint{2.477262in}{2.989608in}}{\pgfqpoint{2.477262in}{2.981372in}}%
\pgfpathcurveto{\pgfqpoint{2.477262in}{2.973136in}}{\pgfqpoint{2.480535in}{2.965236in}}{\pgfqpoint{2.486359in}{2.959412in}}%
\pgfpathcurveto{\pgfqpoint{2.492183in}{2.953588in}}{\pgfqpoint{2.500083in}{2.950315in}}{\pgfqpoint{2.508319in}{2.950315in}}%
\pgfpathclose%
\pgfusepath{stroke,fill}%
\end{pgfscope}%
\begin{pgfscope}%
\pgfpathrectangle{\pgfqpoint{0.100000in}{0.212622in}}{\pgfqpoint{3.696000in}{3.696000in}}%
\pgfusepath{clip}%
\pgfsetbuttcap%
\pgfsetroundjoin%
\definecolor{currentfill}{rgb}{0.121569,0.466667,0.705882}%
\pgfsetfillcolor{currentfill}%
\pgfsetfillopacity{0.766052}%
\pgfsetlinewidth{1.003750pt}%
\definecolor{currentstroke}{rgb}{0.121569,0.466667,0.705882}%
\pgfsetstrokecolor{currentstroke}%
\pgfsetstrokeopacity{0.766052}%
\pgfsetdash{}{0pt}%
\pgfpathmoveto{\pgfqpoint{3.137441in}{2.139972in}}%
\pgfpathcurveto{\pgfqpoint{3.145677in}{2.139972in}}{\pgfqpoint{3.153577in}{2.143244in}}{\pgfqpoint{3.159401in}{2.149068in}}%
\pgfpathcurveto{\pgfqpoint{3.165225in}{2.154892in}}{\pgfqpoint{3.168497in}{2.162792in}}{\pgfqpoint{3.168497in}{2.171028in}}%
\pgfpathcurveto{\pgfqpoint{3.168497in}{2.179265in}}{\pgfqpoint{3.165225in}{2.187165in}}{\pgfqpoint{3.159401in}{2.192989in}}%
\pgfpathcurveto{\pgfqpoint{3.153577in}{2.198813in}}{\pgfqpoint{3.145677in}{2.202085in}}{\pgfqpoint{3.137441in}{2.202085in}}%
\pgfpathcurveto{\pgfqpoint{3.129204in}{2.202085in}}{\pgfqpoint{3.121304in}{2.198813in}}{\pgfqpoint{3.115480in}{2.192989in}}%
\pgfpathcurveto{\pgfqpoint{3.109656in}{2.187165in}}{\pgfqpoint{3.106384in}{2.179265in}}{\pgfqpoint{3.106384in}{2.171028in}}%
\pgfpathcurveto{\pgfqpoint{3.106384in}{2.162792in}}{\pgfqpoint{3.109656in}{2.154892in}}{\pgfqpoint{3.115480in}{2.149068in}}%
\pgfpathcurveto{\pgfqpoint{3.121304in}{2.143244in}}{\pgfqpoint{3.129204in}{2.139972in}}{\pgfqpoint{3.137441in}{2.139972in}}%
\pgfpathclose%
\pgfusepath{stroke,fill}%
\end{pgfscope}%
\begin{pgfscope}%
\pgfpathrectangle{\pgfqpoint{0.100000in}{0.212622in}}{\pgfqpoint{3.696000in}{3.696000in}}%
\pgfusepath{clip}%
\pgfsetbuttcap%
\pgfsetroundjoin%
\definecolor{currentfill}{rgb}{0.121569,0.466667,0.705882}%
\pgfsetfillcolor{currentfill}%
\pgfsetfillopacity{0.766242}%
\pgfsetlinewidth{1.003750pt}%
\definecolor{currentstroke}{rgb}{0.121569,0.466667,0.705882}%
\pgfsetstrokecolor{currentstroke}%
\pgfsetstrokeopacity{0.766242}%
\pgfsetdash{}{0pt}%
\pgfpathmoveto{\pgfqpoint{3.136661in}{2.138888in}}%
\pgfpathcurveto{\pgfqpoint{3.144897in}{2.138888in}}{\pgfqpoint{3.152797in}{2.142160in}}{\pgfqpoint{3.158621in}{2.147984in}}%
\pgfpathcurveto{\pgfqpoint{3.164445in}{2.153808in}}{\pgfqpoint{3.167717in}{2.161708in}}{\pgfqpoint{3.167717in}{2.169944in}}%
\pgfpathcurveto{\pgfqpoint{3.167717in}{2.178180in}}{\pgfqpoint{3.164445in}{2.186081in}}{\pgfqpoint{3.158621in}{2.191904in}}%
\pgfpathcurveto{\pgfqpoint{3.152797in}{2.197728in}}{\pgfqpoint{3.144897in}{2.201001in}}{\pgfqpoint{3.136661in}{2.201001in}}%
\pgfpathcurveto{\pgfqpoint{3.128425in}{2.201001in}}{\pgfqpoint{3.120525in}{2.197728in}}{\pgfqpoint{3.114701in}{2.191904in}}%
\pgfpathcurveto{\pgfqpoint{3.108877in}{2.186081in}}{\pgfqpoint{3.105604in}{2.178180in}}{\pgfqpoint{3.105604in}{2.169944in}}%
\pgfpathcurveto{\pgfqpoint{3.105604in}{2.161708in}}{\pgfqpoint{3.108877in}{2.153808in}}{\pgfqpoint{3.114701in}{2.147984in}}%
\pgfpathcurveto{\pgfqpoint{3.120525in}{2.142160in}}{\pgfqpoint{3.128425in}{2.138888in}}{\pgfqpoint{3.136661in}{2.138888in}}%
\pgfpathclose%
\pgfusepath{stroke,fill}%
\end{pgfscope}%
\begin{pgfscope}%
\pgfpathrectangle{\pgfqpoint{0.100000in}{0.212622in}}{\pgfqpoint{3.696000in}{3.696000in}}%
\pgfusepath{clip}%
\pgfsetbuttcap%
\pgfsetroundjoin%
\definecolor{currentfill}{rgb}{0.121569,0.466667,0.705882}%
\pgfsetfillcolor{currentfill}%
\pgfsetfillopacity{0.766580}%
\pgfsetlinewidth{1.003750pt}%
\definecolor{currentstroke}{rgb}{0.121569,0.466667,0.705882}%
\pgfsetstrokecolor{currentstroke}%
\pgfsetstrokeopacity{0.766580}%
\pgfsetdash{}{0pt}%
\pgfpathmoveto{\pgfqpoint{3.135217in}{2.136938in}}%
\pgfpathcurveto{\pgfqpoint{3.143453in}{2.136938in}}{\pgfqpoint{3.151353in}{2.140210in}}{\pgfqpoint{3.157177in}{2.146034in}}%
\pgfpathcurveto{\pgfqpoint{3.163001in}{2.151858in}}{\pgfqpoint{3.166273in}{2.159758in}}{\pgfqpoint{3.166273in}{2.167995in}}%
\pgfpathcurveto{\pgfqpoint{3.166273in}{2.176231in}}{\pgfqpoint{3.163001in}{2.184131in}}{\pgfqpoint{3.157177in}{2.189955in}}%
\pgfpathcurveto{\pgfqpoint{3.151353in}{2.195779in}}{\pgfqpoint{3.143453in}{2.199051in}}{\pgfqpoint{3.135217in}{2.199051in}}%
\pgfpathcurveto{\pgfqpoint{3.126980in}{2.199051in}}{\pgfqpoint{3.119080in}{2.195779in}}{\pgfqpoint{3.113256in}{2.189955in}}%
\pgfpathcurveto{\pgfqpoint{3.107432in}{2.184131in}}{\pgfqpoint{3.104160in}{2.176231in}}{\pgfqpoint{3.104160in}{2.167995in}}%
\pgfpathcurveto{\pgfqpoint{3.104160in}{2.159758in}}{\pgfqpoint{3.107432in}{2.151858in}}{\pgfqpoint{3.113256in}{2.146034in}}%
\pgfpathcurveto{\pgfqpoint{3.119080in}{2.140210in}}{\pgfqpoint{3.126980in}{2.136938in}}{\pgfqpoint{3.135217in}{2.136938in}}%
\pgfpathclose%
\pgfusepath{stroke,fill}%
\end{pgfscope}%
\begin{pgfscope}%
\pgfpathrectangle{\pgfqpoint{0.100000in}{0.212622in}}{\pgfqpoint{3.696000in}{3.696000in}}%
\pgfusepath{clip}%
\pgfsetbuttcap%
\pgfsetroundjoin%
\definecolor{currentfill}{rgb}{0.121569,0.466667,0.705882}%
\pgfsetfillcolor{currentfill}%
\pgfsetfillopacity{0.766853}%
\pgfsetlinewidth{1.003750pt}%
\definecolor{currentstroke}{rgb}{0.121569,0.466667,0.705882}%
\pgfsetstrokecolor{currentstroke}%
\pgfsetstrokeopacity{0.766853}%
\pgfsetdash{}{0pt}%
\pgfpathmoveto{\pgfqpoint{3.134204in}{2.134954in}}%
\pgfpathcurveto{\pgfqpoint{3.142441in}{2.134954in}}{\pgfqpoint{3.150341in}{2.138226in}}{\pgfqpoint{3.156165in}{2.144050in}}%
\pgfpathcurveto{\pgfqpoint{3.161989in}{2.149874in}}{\pgfqpoint{3.165261in}{2.157774in}}{\pgfqpoint{3.165261in}{2.166011in}}%
\pgfpathcurveto{\pgfqpoint{3.165261in}{2.174247in}}{\pgfqpoint{3.161989in}{2.182147in}}{\pgfqpoint{3.156165in}{2.187971in}}%
\pgfpathcurveto{\pgfqpoint{3.150341in}{2.193795in}}{\pgfqpoint{3.142441in}{2.197067in}}{\pgfqpoint{3.134204in}{2.197067in}}%
\pgfpathcurveto{\pgfqpoint{3.125968in}{2.197067in}}{\pgfqpoint{3.118068in}{2.193795in}}{\pgfqpoint{3.112244in}{2.187971in}}%
\pgfpathcurveto{\pgfqpoint{3.106420in}{2.182147in}}{\pgfqpoint{3.103148in}{2.174247in}}{\pgfqpoint{3.103148in}{2.166011in}}%
\pgfpathcurveto{\pgfqpoint{3.103148in}{2.157774in}}{\pgfqpoint{3.106420in}{2.149874in}}{\pgfqpoint{3.112244in}{2.144050in}}%
\pgfpathcurveto{\pgfqpoint{3.118068in}{2.138226in}}{\pgfqpoint{3.125968in}{2.134954in}}{\pgfqpoint{3.134204in}{2.134954in}}%
\pgfpathclose%
\pgfusepath{stroke,fill}%
\end{pgfscope}%
\begin{pgfscope}%
\pgfpathrectangle{\pgfqpoint{0.100000in}{0.212622in}}{\pgfqpoint{3.696000in}{3.696000in}}%
\pgfusepath{clip}%
\pgfsetbuttcap%
\pgfsetroundjoin%
\definecolor{currentfill}{rgb}{0.121569,0.466667,0.705882}%
\pgfsetfillcolor{currentfill}%
\pgfsetfillopacity{0.767079}%
\pgfsetlinewidth{1.003750pt}%
\definecolor{currentstroke}{rgb}{0.121569,0.466667,0.705882}%
\pgfsetstrokecolor{currentstroke}%
\pgfsetstrokeopacity{0.767079}%
\pgfsetdash{}{0pt}%
\pgfpathmoveto{\pgfqpoint{3.133769in}{2.133810in}}%
\pgfpathcurveto{\pgfqpoint{3.142005in}{2.133810in}}{\pgfqpoint{3.149905in}{2.137082in}}{\pgfqpoint{3.155729in}{2.142906in}}%
\pgfpathcurveto{\pgfqpoint{3.161553in}{2.148730in}}{\pgfqpoint{3.164826in}{2.156630in}}{\pgfqpoint{3.164826in}{2.164866in}}%
\pgfpathcurveto{\pgfqpoint{3.164826in}{2.173103in}}{\pgfqpoint{3.161553in}{2.181003in}}{\pgfqpoint{3.155729in}{2.186827in}}%
\pgfpathcurveto{\pgfqpoint{3.149905in}{2.192651in}}{\pgfqpoint{3.142005in}{2.195923in}}{\pgfqpoint{3.133769in}{2.195923in}}%
\pgfpathcurveto{\pgfqpoint{3.125533in}{2.195923in}}{\pgfqpoint{3.117633in}{2.192651in}}{\pgfqpoint{3.111809in}{2.186827in}}%
\pgfpathcurveto{\pgfqpoint{3.105985in}{2.181003in}}{\pgfqpoint{3.102713in}{2.173103in}}{\pgfqpoint{3.102713in}{2.164866in}}%
\pgfpathcurveto{\pgfqpoint{3.102713in}{2.156630in}}{\pgfqpoint{3.105985in}{2.148730in}}{\pgfqpoint{3.111809in}{2.142906in}}%
\pgfpathcurveto{\pgfqpoint{3.117633in}{2.137082in}}{\pgfqpoint{3.125533in}{2.133810in}}{\pgfqpoint{3.133769in}{2.133810in}}%
\pgfpathclose%
\pgfusepath{stroke,fill}%
\end{pgfscope}%
\begin{pgfscope}%
\pgfpathrectangle{\pgfqpoint{0.100000in}{0.212622in}}{\pgfqpoint{3.696000in}{3.696000in}}%
\pgfusepath{clip}%
\pgfsetbuttcap%
\pgfsetroundjoin%
\definecolor{currentfill}{rgb}{0.121569,0.466667,0.705882}%
\pgfsetfillcolor{currentfill}%
\pgfsetfillopacity{0.767339}%
\pgfsetlinewidth{1.003750pt}%
\definecolor{currentstroke}{rgb}{0.121569,0.466667,0.705882}%
\pgfsetstrokecolor{currentstroke}%
\pgfsetstrokeopacity{0.767339}%
\pgfsetdash{}{0pt}%
\pgfpathmoveto{\pgfqpoint{2.513501in}{2.947976in}}%
\pgfpathcurveto{\pgfqpoint{2.521737in}{2.947976in}}{\pgfqpoint{2.529637in}{2.951249in}}{\pgfqpoint{2.535461in}{2.957073in}}%
\pgfpathcurveto{\pgfqpoint{2.541285in}{2.962896in}}{\pgfqpoint{2.544557in}{2.970796in}}{\pgfqpoint{2.544557in}{2.979033in}}%
\pgfpathcurveto{\pgfqpoint{2.544557in}{2.987269in}}{\pgfqpoint{2.541285in}{2.995169in}}{\pgfqpoint{2.535461in}{3.000993in}}%
\pgfpathcurveto{\pgfqpoint{2.529637in}{3.006817in}}{\pgfqpoint{2.521737in}{3.010089in}}{\pgfqpoint{2.513501in}{3.010089in}}%
\pgfpathcurveto{\pgfqpoint{2.505264in}{3.010089in}}{\pgfqpoint{2.497364in}{3.006817in}}{\pgfqpoint{2.491540in}{3.000993in}}%
\pgfpathcurveto{\pgfqpoint{2.485716in}{2.995169in}}{\pgfqpoint{2.482444in}{2.987269in}}{\pgfqpoint{2.482444in}{2.979033in}}%
\pgfpathcurveto{\pgfqpoint{2.482444in}{2.970796in}}{\pgfqpoint{2.485716in}{2.962896in}}{\pgfqpoint{2.491540in}{2.957073in}}%
\pgfpathcurveto{\pgfqpoint{2.497364in}{2.951249in}}{\pgfqpoint{2.505264in}{2.947976in}}{\pgfqpoint{2.513501in}{2.947976in}}%
\pgfpathclose%
\pgfusepath{stroke,fill}%
\end{pgfscope}%
\begin{pgfscope}%
\pgfpathrectangle{\pgfqpoint{0.100000in}{0.212622in}}{\pgfqpoint{3.696000in}{3.696000in}}%
\pgfusepath{clip}%
\pgfsetbuttcap%
\pgfsetroundjoin%
\definecolor{currentfill}{rgb}{0.121569,0.466667,0.705882}%
\pgfsetfillcolor{currentfill}%
\pgfsetfillopacity{0.767445}%
\pgfsetlinewidth{1.003750pt}%
\definecolor{currentstroke}{rgb}{0.121569,0.466667,0.705882}%
\pgfsetstrokecolor{currentstroke}%
\pgfsetstrokeopacity{0.767445}%
\pgfsetdash{}{0pt}%
\pgfpathmoveto{\pgfqpoint{3.133117in}{2.131458in}}%
\pgfpathcurveto{\pgfqpoint{3.141354in}{2.131458in}}{\pgfqpoint{3.149254in}{2.134731in}}{\pgfqpoint{3.155078in}{2.140555in}}%
\pgfpathcurveto{\pgfqpoint{3.160902in}{2.146379in}}{\pgfqpoint{3.164174in}{2.154279in}}{\pgfqpoint{3.164174in}{2.162515in}}%
\pgfpathcurveto{\pgfqpoint{3.164174in}{2.170751in}}{\pgfqpoint{3.160902in}{2.178651in}}{\pgfqpoint{3.155078in}{2.184475in}}%
\pgfpathcurveto{\pgfqpoint{3.149254in}{2.190299in}}{\pgfqpoint{3.141354in}{2.193571in}}{\pgfqpoint{3.133117in}{2.193571in}}%
\pgfpathcurveto{\pgfqpoint{3.124881in}{2.193571in}}{\pgfqpoint{3.116981in}{2.190299in}}{\pgfqpoint{3.111157in}{2.184475in}}%
\pgfpathcurveto{\pgfqpoint{3.105333in}{2.178651in}}{\pgfqpoint{3.102061in}{2.170751in}}{\pgfqpoint{3.102061in}{2.162515in}}%
\pgfpathcurveto{\pgfqpoint{3.102061in}{2.154279in}}{\pgfqpoint{3.105333in}{2.146379in}}{\pgfqpoint{3.111157in}{2.140555in}}%
\pgfpathcurveto{\pgfqpoint{3.116981in}{2.134731in}}{\pgfqpoint{3.124881in}{2.131458in}}{\pgfqpoint{3.133117in}{2.131458in}}%
\pgfpathclose%
\pgfusepath{stroke,fill}%
\end{pgfscope}%
\begin{pgfscope}%
\pgfpathrectangle{\pgfqpoint{0.100000in}{0.212622in}}{\pgfqpoint{3.696000in}{3.696000in}}%
\pgfusepath{clip}%
\pgfsetbuttcap%
\pgfsetroundjoin%
\definecolor{currentfill}{rgb}{0.121569,0.466667,0.705882}%
\pgfsetfillcolor{currentfill}%
\pgfsetfillopacity{0.768120}%
\pgfsetlinewidth{1.003750pt}%
\definecolor{currentstroke}{rgb}{0.121569,0.466667,0.705882}%
\pgfsetstrokecolor{currentstroke}%
\pgfsetstrokeopacity{0.768120}%
\pgfsetdash{}{0pt}%
\pgfpathmoveto{\pgfqpoint{3.131434in}{2.127533in}}%
\pgfpathcurveto{\pgfqpoint{3.139670in}{2.127533in}}{\pgfqpoint{3.147570in}{2.130806in}}{\pgfqpoint{3.153394in}{2.136630in}}%
\pgfpathcurveto{\pgfqpoint{3.159218in}{2.142454in}}{\pgfqpoint{3.162490in}{2.150354in}}{\pgfqpoint{3.162490in}{2.158590in}}%
\pgfpathcurveto{\pgfqpoint{3.162490in}{2.166826in}}{\pgfqpoint{3.159218in}{2.174726in}}{\pgfqpoint{3.153394in}{2.180550in}}%
\pgfpathcurveto{\pgfqpoint{3.147570in}{2.186374in}}{\pgfqpoint{3.139670in}{2.189646in}}{\pgfqpoint{3.131434in}{2.189646in}}%
\pgfpathcurveto{\pgfqpoint{3.123197in}{2.189646in}}{\pgfqpoint{3.115297in}{2.186374in}}{\pgfqpoint{3.109473in}{2.180550in}}%
\pgfpathcurveto{\pgfqpoint{3.103649in}{2.174726in}}{\pgfqpoint{3.100377in}{2.166826in}}{\pgfqpoint{3.100377in}{2.158590in}}%
\pgfpathcurveto{\pgfqpoint{3.100377in}{2.150354in}}{\pgfqpoint{3.103649in}{2.142454in}}{\pgfqpoint{3.109473in}{2.136630in}}%
\pgfpathcurveto{\pgfqpoint{3.115297in}{2.130806in}}{\pgfqpoint{3.123197in}{2.127533in}}{\pgfqpoint{3.131434in}{2.127533in}}%
\pgfpathclose%
\pgfusepath{stroke,fill}%
\end{pgfscope}%
\begin{pgfscope}%
\pgfpathrectangle{\pgfqpoint{0.100000in}{0.212622in}}{\pgfqpoint{3.696000in}{3.696000in}}%
\pgfusepath{clip}%
\pgfsetbuttcap%
\pgfsetroundjoin%
\definecolor{currentfill}{rgb}{0.121569,0.466667,0.705882}%
\pgfsetfillcolor{currentfill}%
\pgfsetfillopacity{0.768316}%
\pgfsetlinewidth{1.003750pt}%
\definecolor{currentstroke}{rgb}{0.121569,0.466667,0.705882}%
\pgfsetstrokecolor{currentstroke}%
\pgfsetstrokeopacity{0.768316}%
\pgfsetdash{}{0pt}%
\pgfpathmoveto{\pgfqpoint{1.050435in}{1.649585in}}%
\pgfpathcurveto{\pgfqpoint{1.058671in}{1.649585in}}{\pgfqpoint{1.066571in}{1.652857in}}{\pgfqpoint{1.072395in}{1.658681in}}%
\pgfpathcurveto{\pgfqpoint{1.078219in}{1.664505in}}{\pgfqpoint{1.081491in}{1.672405in}}{\pgfqpoint{1.081491in}{1.680641in}}%
\pgfpathcurveto{\pgfqpoint{1.081491in}{1.688877in}}{\pgfqpoint{1.078219in}{1.696777in}}{\pgfqpoint{1.072395in}{1.702601in}}%
\pgfpathcurveto{\pgfqpoint{1.066571in}{1.708425in}}{\pgfqpoint{1.058671in}{1.711698in}}{\pgfqpoint{1.050435in}{1.711698in}}%
\pgfpathcurveto{\pgfqpoint{1.042198in}{1.711698in}}{\pgfqpoint{1.034298in}{1.708425in}}{\pgfqpoint{1.028474in}{1.702601in}}%
\pgfpathcurveto{\pgfqpoint{1.022651in}{1.696777in}}{\pgfqpoint{1.019378in}{1.688877in}}{\pgfqpoint{1.019378in}{1.680641in}}%
\pgfpathcurveto{\pgfqpoint{1.019378in}{1.672405in}}{\pgfqpoint{1.022651in}{1.664505in}}{\pgfqpoint{1.028474in}{1.658681in}}%
\pgfpathcurveto{\pgfqpoint{1.034298in}{1.652857in}}{\pgfqpoint{1.042198in}{1.649585in}}{\pgfqpoint{1.050435in}{1.649585in}}%
\pgfpathclose%
\pgfusepath{stroke,fill}%
\end{pgfscope}%
\begin{pgfscope}%
\pgfpathrectangle{\pgfqpoint{0.100000in}{0.212622in}}{\pgfqpoint{3.696000in}{3.696000in}}%
\pgfusepath{clip}%
\pgfsetbuttcap%
\pgfsetroundjoin%
\definecolor{currentfill}{rgb}{0.121569,0.466667,0.705882}%
\pgfsetfillcolor{currentfill}%
\pgfsetfillopacity{0.768598}%
\pgfsetlinewidth{1.003750pt}%
\definecolor{currentstroke}{rgb}{0.121569,0.466667,0.705882}%
\pgfsetstrokecolor{currentstroke}%
\pgfsetstrokeopacity{0.768598}%
\pgfsetdash{}{0pt}%
\pgfpathmoveto{\pgfqpoint{2.517818in}{2.946606in}}%
\pgfpathcurveto{\pgfqpoint{2.526054in}{2.946606in}}{\pgfqpoint{2.533954in}{2.949878in}}{\pgfqpoint{2.539778in}{2.955702in}}%
\pgfpathcurveto{\pgfqpoint{2.545602in}{2.961526in}}{\pgfqpoint{2.548874in}{2.969426in}}{\pgfqpoint{2.548874in}{2.977662in}}%
\pgfpathcurveto{\pgfqpoint{2.548874in}{2.985898in}}{\pgfqpoint{2.545602in}{2.993798in}}{\pgfqpoint{2.539778in}{2.999622in}}%
\pgfpathcurveto{\pgfqpoint{2.533954in}{3.005446in}}{\pgfqpoint{2.526054in}{3.008719in}}{\pgfqpoint{2.517818in}{3.008719in}}%
\pgfpathcurveto{\pgfqpoint{2.509582in}{3.008719in}}{\pgfqpoint{2.501682in}{3.005446in}}{\pgfqpoint{2.495858in}{2.999622in}}%
\pgfpathcurveto{\pgfqpoint{2.490034in}{2.993798in}}{\pgfqpoint{2.486761in}{2.985898in}}{\pgfqpoint{2.486761in}{2.977662in}}%
\pgfpathcurveto{\pgfqpoint{2.486761in}{2.969426in}}{\pgfqpoint{2.490034in}{2.961526in}}{\pgfqpoint{2.495858in}{2.955702in}}%
\pgfpathcurveto{\pgfqpoint{2.501682in}{2.949878in}}{\pgfqpoint{2.509582in}{2.946606in}}{\pgfqpoint{2.517818in}{2.946606in}}%
\pgfpathclose%
\pgfusepath{stroke,fill}%
\end{pgfscope}%
\begin{pgfscope}%
\pgfpathrectangle{\pgfqpoint{0.100000in}{0.212622in}}{\pgfqpoint{3.696000in}{3.696000in}}%
\pgfusepath{clip}%
\pgfsetbuttcap%
\pgfsetroundjoin%
\definecolor{currentfill}{rgb}{0.121569,0.466667,0.705882}%
\pgfsetfillcolor{currentfill}%
\pgfsetfillopacity{0.768606}%
\pgfsetlinewidth{1.003750pt}%
\definecolor{currentstroke}{rgb}{0.121569,0.466667,0.705882}%
\pgfsetstrokecolor{currentstroke}%
\pgfsetstrokeopacity{0.768606}%
\pgfsetdash{}{0pt}%
\pgfpathmoveto{\pgfqpoint{3.129590in}{2.124658in}}%
\pgfpathcurveto{\pgfqpoint{3.137826in}{2.124658in}}{\pgfqpoint{3.145726in}{2.127930in}}{\pgfqpoint{3.151550in}{2.133754in}}%
\pgfpathcurveto{\pgfqpoint{3.157374in}{2.139578in}}{\pgfqpoint{3.160646in}{2.147478in}}{\pgfqpoint{3.160646in}{2.155714in}}%
\pgfpathcurveto{\pgfqpoint{3.160646in}{2.163950in}}{\pgfqpoint{3.157374in}{2.171850in}}{\pgfqpoint{3.151550in}{2.177674in}}%
\pgfpathcurveto{\pgfqpoint{3.145726in}{2.183498in}}{\pgfqpoint{3.137826in}{2.186771in}}{\pgfqpoint{3.129590in}{2.186771in}}%
\pgfpathcurveto{\pgfqpoint{3.121354in}{2.186771in}}{\pgfqpoint{3.113453in}{2.183498in}}{\pgfqpoint{3.107630in}{2.177674in}}%
\pgfpathcurveto{\pgfqpoint{3.101806in}{2.171850in}}{\pgfqpoint{3.098533in}{2.163950in}}{\pgfqpoint{3.098533in}{2.155714in}}%
\pgfpathcurveto{\pgfqpoint{3.098533in}{2.147478in}}{\pgfqpoint{3.101806in}{2.139578in}}{\pgfqpoint{3.107630in}{2.133754in}}%
\pgfpathcurveto{\pgfqpoint{3.113453in}{2.127930in}}{\pgfqpoint{3.121354in}{2.124658in}}{\pgfqpoint{3.129590in}{2.124658in}}%
\pgfpathclose%
\pgfusepath{stroke,fill}%
\end{pgfscope}%
\begin{pgfscope}%
\pgfpathrectangle{\pgfqpoint{0.100000in}{0.212622in}}{\pgfqpoint{3.696000in}{3.696000in}}%
\pgfusepath{clip}%
\pgfsetbuttcap%
\pgfsetroundjoin%
\definecolor{currentfill}{rgb}{0.121569,0.466667,0.705882}%
\pgfsetfillcolor{currentfill}%
\pgfsetfillopacity{0.768961}%
\pgfsetlinewidth{1.003750pt}%
\definecolor{currentstroke}{rgb}{0.121569,0.466667,0.705882}%
\pgfsetstrokecolor{currentstroke}%
\pgfsetstrokeopacity{0.768961}%
\pgfsetdash{}{0pt}%
\pgfpathmoveto{\pgfqpoint{3.128109in}{2.122350in}}%
\pgfpathcurveto{\pgfqpoint{3.136346in}{2.122350in}}{\pgfqpoint{3.144246in}{2.125622in}}{\pgfqpoint{3.150070in}{2.131446in}}%
\pgfpathcurveto{\pgfqpoint{3.155893in}{2.137270in}}{\pgfqpoint{3.159166in}{2.145170in}}{\pgfqpoint{3.159166in}{2.153406in}}%
\pgfpathcurveto{\pgfqpoint{3.159166in}{2.161643in}}{\pgfqpoint{3.155893in}{2.169543in}}{\pgfqpoint{3.150070in}{2.175367in}}%
\pgfpathcurveto{\pgfqpoint{3.144246in}{2.181191in}}{\pgfqpoint{3.136346in}{2.184463in}}{\pgfqpoint{3.128109in}{2.184463in}}%
\pgfpathcurveto{\pgfqpoint{3.119873in}{2.184463in}}{\pgfqpoint{3.111973in}{2.181191in}}{\pgfqpoint{3.106149in}{2.175367in}}%
\pgfpathcurveto{\pgfqpoint{3.100325in}{2.169543in}}{\pgfqpoint{3.097053in}{2.161643in}}{\pgfqpoint{3.097053in}{2.153406in}}%
\pgfpathcurveto{\pgfqpoint{3.097053in}{2.145170in}}{\pgfqpoint{3.100325in}{2.137270in}}{\pgfqpoint{3.106149in}{2.131446in}}%
\pgfpathcurveto{\pgfqpoint{3.111973in}{2.125622in}}{\pgfqpoint{3.119873in}{2.122350in}}{\pgfqpoint{3.128109in}{2.122350in}}%
\pgfpathclose%
\pgfusepath{stroke,fill}%
\end{pgfscope}%
\begin{pgfscope}%
\pgfpathrectangle{\pgfqpoint{0.100000in}{0.212622in}}{\pgfqpoint{3.696000in}{3.696000in}}%
\pgfusepath{clip}%
\pgfsetbuttcap%
\pgfsetroundjoin%
\definecolor{currentfill}{rgb}{0.121569,0.466667,0.705882}%
\pgfsetfillcolor{currentfill}%
\pgfsetfillopacity{0.769219}%
\pgfsetlinewidth{1.003750pt}%
\definecolor{currentstroke}{rgb}{0.121569,0.466667,0.705882}%
\pgfsetstrokecolor{currentstroke}%
\pgfsetstrokeopacity{0.769219}%
\pgfsetdash{}{0pt}%
\pgfpathmoveto{\pgfqpoint{3.127181in}{2.120890in}}%
\pgfpathcurveto{\pgfqpoint{3.135417in}{2.120890in}}{\pgfqpoint{3.143317in}{2.124162in}}{\pgfqpoint{3.149141in}{2.129986in}}%
\pgfpathcurveto{\pgfqpoint{3.154965in}{2.135810in}}{\pgfqpoint{3.158237in}{2.143710in}}{\pgfqpoint{3.158237in}{2.151947in}}%
\pgfpathcurveto{\pgfqpoint{3.158237in}{2.160183in}}{\pgfqpoint{3.154965in}{2.168083in}}{\pgfqpoint{3.149141in}{2.173907in}}%
\pgfpathcurveto{\pgfqpoint{3.143317in}{2.179731in}}{\pgfqpoint{3.135417in}{2.183003in}}{\pgfqpoint{3.127181in}{2.183003in}}%
\pgfpathcurveto{\pgfqpoint{3.118944in}{2.183003in}}{\pgfqpoint{3.111044in}{2.179731in}}{\pgfqpoint{3.105220in}{2.173907in}}%
\pgfpathcurveto{\pgfqpoint{3.099396in}{2.168083in}}{\pgfqpoint{3.096124in}{2.160183in}}{\pgfqpoint{3.096124in}{2.151947in}}%
\pgfpathcurveto{\pgfqpoint{3.096124in}{2.143710in}}{\pgfqpoint{3.099396in}{2.135810in}}{\pgfqpoint{3.105220in}{2.129986in}}%
\pgfpathcurveto{\pgfqpoint{3.111044in}{2.124162in}}{\pgfqpoint{3.118944in}{2.120890in}}{\pgfqpoint{3.127181in}{2.120890in}}%
\pgfpathclose%
\pgfusepath{stroke,fill}%
\end{pgfscope}%
\begin{pgfscope}%
\pgfpathrectangle{\pgfqpoint{0.100000in}{0.212622in}}{\pgfqpoint{3.696000in}{3.696000in}}%
\pgfusepath{clip}%
\pgfsetbuttcap%
\pgfsetroundjoin%
\definecolor{currentfill}{rgb}{0.121569,0.466667,0.705882}%
\pgfsetfillcolor{currentfill}%
\pgfsetfillopacity{0.769649}%
\pgfsetlinewidth{1.003750pt}%
\definecolor{currentstroke}{rgb}{0.121569,0.466667,0.705882}%
\pgfsetstrokecolor{currentstroke}%
\pgfsetstrokeopacity{0.769649}%
\pgfsetdash{}{0pt}%
\pgfpathmoveto{\pgfqpoint{1.046127in}{1.635388in}}%
\pgfpathcurveto{\pgfqpoint{1.054363in}{1.635388in}}{\pgfqpoint{1.062264in}{1.638660in}}{\pgfqpoint{1.068087in}{1.644484in}}%
\pgfpathcurveto{\pgfqpoint{1.073911in}{1.650308in}}{\pgfqpoint{1.077184in}{1.658208in}}{\pgfqpoint{1.077184in}{1.666445in}}%
\pgfpathcurveto{\pgfqpoint{1.077184in}{1.674681in}}{\pgfqpoint{1.073911in}{1.682581in}}{\pgfqpoint{1.068087in}{1.688405in}}%
\pgfpathcurveto{\pgfqpoint{1.062264in}{1.694229in}}{\pgfqpoint{1.054363in}{1.697501in}}{\pgfqpoint{1.046127in}{1.697501in}}%
\pgfpathcurveto{\pgfqpoint{1.037891in}{1.697501in}}{\pgfqpoint{1.029991in}{1.694229in}}{\pgfqpoint{1.024167in}{1.688405in}}%
\pgfpathcurveto{\pgfqpoint{1.018343in}{1.682581in}}{\pgfqpoint{1.015071in}{1.674681in}}{\pgfqpoint{1.015071in}{1.666445in}}%
\pgfpathcurveto{\pgfqpoint{1.015071in}{1.658208in}}{\pgfqpoint{1.018343in}{1.650308in}}{\pgfqpoint{1.024167in}{1.644484in}}%
\pgfpathcurveto{\pgfqpoint{1.029991in}{1.638660in}}{\pgfqpoint{1.037891in}{1.635388in}}{\pgfqpoint{1.046127in}{1.635388in}}%
\pgfpathclose%
\pgfusepath{stroke,fill}%
\end{pgfscope}%
\begin{pgfscope}%
\pgfpathrectangle{\pgfqpoint{0.100000in}{0.212622in}}{\pgfqpoint{3.696000in}{3.696000in}}%
\pgfusepath{clip}%
\pgfsetbuttcap%
\pgfsetroundjoin%
\definecolor{currentfill}{rgb}{0.121569,0.466667,0.705882}%
\pgfsetfillcolor{currentfill}%
\pgfsetfillopacity{0.769724}%
\pgfsetlinewidth{1.003750pt}%
\definecolor{currentstroke}{rgb}{0.121569,0.466667,0.705882}%
\pgfsetstrokecolor{currentstroke}%
\pgfsetstrokeopacity{0.769724}%
\pgfsetdash{}{0pt}%
\pgfpathmoveto{\pgfqpoint{3.125592in}{2.118240in}}%
\pgfpathcurveto{\pgfqpoint{3.133829in}{2.118240in}}{\pgfqpoint{3.141729in}{2.121512in}}{\pgfqpoint{3.147553in}{2.127336in}}%
\pgfpathcurveto{\pgfqpoint{3.153377in}{2.133160in}}{\pgfqpoint{3.156649in}{2.141060in}}{\pgfqpoint{3.156649in}{2.149296in}}%
\pgfpathcurveto{\pgfqpoint{3.156649in}{2.157533in}}{\pgfqpoint{3.153377in}{2.165433in}}{\pgfqpoint{3.147553in}{2.171257in}}%
\pgfpathcurveto{\pgfqpoint{3.141729in}{2.177081in}}{\pgfqpoint{3.133829in}{2.180353in}}{\pgfqpoint{3.125592in}{2.180353in}}%
\pgfpathcurveto{\pgfqpoint{3.117356in}{2.180353in}}{\pgfqpoint{3.109456in}{2.177081in}}{\pgfqpoint{3.103632in}{2.171257in}}%
\pgfpathcurveto{\pgfqpoint{3.097808in}{2.165433in}}{\pgfqpoint{3.094536in}{2.157533in}}{\pgfqpoint{3.094536in}{2.149296in}}%
\pgfpathcurveto{\pgfqpoint{3.094536in}{2.141060in}}{\pgfqpoint{3.097808in}{2.133160in}}{\pgfqpoint{3.103632in}{2.127336in}}%
\pgfpathcurveto{\pgfqpoint{3.109456in}{2.121512in}}{\pgfqpoint{3.117356in}{2.118240in}}{\pgfqpoint{3.125592in}{2.118240in}}%
\pgfpathclose%
\pgfusepath{stroke,fill}%
\end{pgfscope}%
\begin{pgfscope}%
\pgfpathrectangle{\pgfqpoint{0.100000in}{0.212622in}}{\pgfqpoint{3.696000in}{3.696000in}}%
\pgfusepath{clip}%
\pgfsetbuttcap%
\pgfsetroundjoin%
\definecolor{currentfill}{rgb}{0.121569,0.466667,0.705882}%
\pgfsetfillcolor{currentfill}%
\pgfsetfillopacity{0.769753}%
\pgfsetlinewidth{1.003750pt}%
\definecolor{currentstroke}{rgb}{0.121569,0.466667,0.705882}%
\pgfsetstrokecolor{currentstroke}%
\pgfsetstrokeopacity{0.769753}%
\pgfsetdash{}{0pt}%
\pgfpathmoveto{\pgfqpoint{2.521892in}{2.945259in}}%
\pgfpathcurveto{\pgfqpoint{2.530129in}{2.945259in}}{\pgfqpoint{2.538029in}{2.948531in}}{\pgfqpoint{2.543853in}{2.954355in}}%
\pgfpathcurveto{\pgfqpoint{2.549677in}{2.960179in}}{\pgfqpoint{2.552949in}{2.968079in}}{\pgfqpoint{2.552949in}{2.976316in}}%
\pgfpathcurveto{\pgfqpoint{2.552949in}{2.984552in}}{\pgfqpoint{2.549677in}{2.992452in}}{\pgfqpoint{2.543853in}{2.998276in}}%
\pgfpathcurveto{\pgfqpoint{2.538029in}{3.004100in}}{\pgfqpoint{2.530129in}{3.007372in}}{\pgfqpoint{2.521892in}{3.007372in}}%
\pgfpathcurveto{\pgfqpoint{2.513656in}{3.007372in}}{\pgfqpoint{2.505756in}{3.004100in}}{\pgfqpoint{2.499932in}{2.998276in}}%
\pgfpathcurveto{\pgfqpoint{2.494108in}{2.992452in}}{\pgfqpoint{2.490836in}{2.984552in}}{\pgfqpoint{2.490836in}{2.976316in}}%
\pgfpathcurveto{\pgfqpoint{2.490836in}{2.968079in}}{\pgfqpoint{2.494108in}{2.960179in}}{\pgfqpoint{2.499932in}{2.954355in}}%
\pgfpathcurveto{\pgfqpoint{2.505756in}{2.948531in}}{\pgfqpoint{2.513656in}{2.945259in}}{\pgfqpoint{2.521892in}{2.945259in}}%
\pgfpathclose%
\pgfusepath{stroke,fill}%
\end{pgfscope}%
\begin{pgfscope}%
\pgfpathrectangle{\pgfqpoint{0.100000in}{0.212622in}}{\pgfqpoint{3.696000in}{3.696000in}}%
\pgfusepath{clip}%
\pgfsetbuttcap%
\pgfsetroundjoin%
\definecolor{currentfill}{rgb}{0.121569,0.466667,0.705882}%
\pgfsetfillcolor{currentfill}%
\pgfsetfillopacity{0.770168}%
\pgfsetlinewidth{1.003750pt}%
\definecolor{currentstroke}{rgb}{0.121569,0.466667,0.705882}%
\pgfsetstrokecolor{currentstroke}%
\pgfsetstrokeopacity{0.770168}%
\pgfsetdash{}{0pt}%
\pgfpathmoveto{\pgfqpoint{3.124316in}{2.115820in}}%
\pgfpathcurveto{\pgfqpoint{3.132552in}{2.115820in}}{\pgfqpoint{3.140452in}{2.119092in}}{\pgfqpoint{3.146276in}{2.124916in}}%
\pgfpathcurveto{\pgfqpoint{3.152100in}{2.130740in}}{\pgfqpoint{3.155372in}{2.138640in}}{\pgfqpoint{3.155372in}{2.146876in}}%
\pgfpathcurveto{\pgfqpoint{3.155372in}{2.155112in}}{\pgfqpoint{3.152100in}{2.163012in}}{\pgfqpoint{3.146276in}{2.168836in}}%
\pgfpathcurveto{\pgfqpoint{3.140452in}{2.174660in}}{\pgfqpoint{3.132552in}{2.177933in}}{\pgfqpoint{3.124316in}{2.177933in}}%
\pgfpathcurveto{\pgfqpoint{3.116080in}{2.177933in}}{\pgfqpoint{3.108180in}{2.174660in}}{\pgfqpoint{3.102356in}{2.168836in}}%
\pgfpathcurveto{\pgfqpoint{3.096532in}{2.163012in}}{\pgfqpoint{3.093259in}{2.155112in}}{\pgfqpoint{3.093259in}{2.146876in}}%
\pgfpathcurveto{\pgfqpoint{3.093259in}{2.138640in}}{\pgfqpoint{3.096532in}{2.130740in}}{\pgfqpoint{3.102356in}{2.124916in}}%
\pgfpathcurveto{\pgfqpoint{3.108180in}{2.119092in}}{\pgfqpoint{3.116080in}{2.115820in}}{\pgfqpoint{3.124316in}{2.115820in}}%
\pgfpathclose%
\pgfusepath{stroke,fill}%
\end{pgfscope}%
\begin{pgfscope}%
\pgfpathrectangle{\pgfqpoint{0.100000in}{0.212622in}}{\pgfqpoint{3.696000in}{3.696000in}}%
\pgfusepath{clip}%
\pgfsetbuttcap%
\pgfsetroundjoin%
\definecolor{currentfill}{rgb}{0.121569,0.466667,0.705882}%
\pgfsetfillcolor{currentfill}%
\pgfsetfillopacity{0.770454}%
\pgfsetlinewidth{1.003750pt}%
\definecolor{currentstroke}{rgb}{0.121569,0.466667,0.705882}%
\pgfsetstrokecolor{currentstroke}%
\pgfsetstrokeopacity{0.770454}%
\pgfsetdash{}{0pt}%
\pgfpathmoveto{\pgfqpoint{1.043807in}{1.627863in}}%
\pgfpathcurveto{\pgfqpoint{1.052043in}{1.627863in}}{\pgfqpoint{1.059943in}{1.631136in}}{\pgfqpoint{1.065767in}{1.636960in}}%
\pgfpathcurveto{\pgfqpoint{1.071591in}{1.642784in}}{\pgfqpoint{1.074863in}{1.650684in}}{\pgfqpoint{1.074863in}{1.658920in}}%
\pgfpathcurveto{\pgfqpoint{1.074863in}{1.667156in}}{\pgfqpoint{1.071591in}{1.675056in}}{\pgfqpoint{1.065767in}{1.680880in}}%
\pgfpathcurveto{\pgfqpoint{1.059943in}{1.686704in}}{\pgfqpoint{1.052043in}{1.689976in}}{\pgfqpoint{1.043807in}{1.689976in}}%
\pgfpathcurveto{\pgfqpoint{1.035571in}{1.689976in}}{\pgfqpoint{1.027670in}{1.686704in}}{\pgfqpoint{1.021847in}{1.680880in}}%
\pgfpathcurveto{\pgfqpoint{1.016023in}{1.675056in}}{\pgfqpoint{1.012750in}{1.667156in}}{\pgfqpoint{1.012750in}{1.658920in}}%
\pgfpathcurveto{\pgfqpoint{1.012750in}{1.650684in}}{\pgfqpoint{1.016023in}{1.642784in}}{\pgfqpoint{1.021847in}{1.636960in}}%
\pgfpathcurveto{\pgfqpoint{1.027670in}{1.631136in}}{\pgfqpoint{1.035571in}{1.627863in}}{\pgfqpoint{1.043807in}{1.627863in}}%
\pgfpathclose%
\pgfusepath{stroke,fill}%
\end{pgfscope}%
\begin{pgfscope}%
\pgfpathrectangle{\pgfqpoint{0.100000in}{0.212622in}}{\pgfqpoint{3.696000in}{3.696000in}}%
\pgfusepath{clip}%
\pgfsetbuttcap%
\pgfsetroundjoin%
\definecolor{currentfill}{rgb}{0.121569,0.466667,0.705882}%
\pgfsetfillcolor{currentfill}%
\pgfsetfillopacity{0.770701}%
\pgfsetlinewidth{1.003750pt}%
\definecolor{currentstroke}{rgb}{0.121569,0.466667,0.705882}%
\pgfsetstrokecolor{currentstroke}%
\pgfsetstrokeopacity{0.770701}%
\pgfsetdash{}{0pt}%
\pgfpathmoveto{\pgfqpoint{2.525656in}{2.944262in}}%
\pgfpathcurveto{\pgfqpoint{2.533892in}{2.944262in}}{\pgfqpoint{2.541792in}{2.947534in}}{\pgfqpoint{2.547616in}{2.953358in}}%
\pgfpathcurveto{\pgfqpoint{2.553440in}{2.959182in}}{\pgfqpoint{2.556712in}{2.967082in}}{\pgfqpoint{2.556712in}{2.975318in}}%
\pgfpathcurveto{\pgfqpoint{2.556712in}{2.983554in}}{\pgfqpoint{2.553440in}{2.991454in}}{\pgfqpoint{2.547616in}{2.997278in}}%
\pgfpathcurveto{\pgfqpoint{2.541792in}{3.003102in}}{\pgfqpoint{2.533892in}{3.006375in}}{\pgfqpoint{2.525656in}{3.006375in}}%
\pgfpathcurveto{\pgfqpoint{2.517420in}{3.006375in}}{\pgfqpoint{2.509520in}{3.003102in}}{\pgfqpoint{2.503696in}{2.997278in}}%
\pgfpathcurveto{\pgfqpoint{2.497872in}{2.991454in}}{\pgfqpoint{2.494599in}{2.983554in}}{\pgfqpoint{2.494599in}{2.975318in}}%
\pgfpathcurveto{\pgfqpoint{2.494599in}{2.967082in}}{\pgfqpoint{2.497872in}{2.959182in}}{\pgfqpoint{2.503696in}{2.953358in}}%
\pgfpathcurveto{\pgfqpoint{2.509520in}{2.947534in}}{\pgfqpoint{2.517420in}{2.944262in}}{\pgfqpoint{2.525656in}{2.944262in}}%
\pgfpathclose%
\pgfusepath{stroke,fill}%
\end{pgfscope}%
\begin{pgfscope}%
\pgfpathrectangle{\pgfqpoint{0.100000in}{0.212622in}}{\pgfqpoint{3.696000in}{3.696000in}}%
\pgfusepath{clip}%
\pgfsetbuttcap%
\pgfsetroundjoin%
\definecolor{currentfill}{rgb}{0.121569,0.466667,0.705882}%
\pgfsetfillcolor{currentfill}%
\pgfsetfillopacity{0.771020}%
\pgfsetlinewidth{1.003750pt}%
\definecolor{currentstroke}{rgb}{0.121569,0.466667,0.705882}%
\pgfsetstrokecolor{currentstroke}%
\pgfsetstrokeopacity{0.771020}%
\pgfsetdash{}{0pt}%
\pgfpathmoveto{\pgfqpoint{3.122082in}{2.111525in}}%
\pgfpathcurveto{\pgfqpoint{3.130318in}{2.111525in}}{\pgfqpoint{3.138218in}{2.114797in}}{\pgfqpoint{3.144042in}{2.120621in}}%
\pgfpathcurveto{\pgfqpoint{3.149866in}{2.126445in}}{\pgfqpoint{3.153139in}{2.134345in}}{\pgfqpoint{3.153139in}{2.142581in}}%
\pgfpathcurveto{\pgfqpoint{3.153139in}{2.150818in}}{\pgfqpoint{3.149866in}{2.158718in}}{\pgfqpoint{3.144042in}{2.164542in}}%
\pgfpathcurveto{\pgfqpoint{3.138218in}{2.170366in}}{\pgfqpoint{3.130318in}{2.173638in}}{\pgfqpoint{3.122082in}{2.173638in}}%
\pgfpathcurveto{\pgfqpoint{3.113846in}{2.173638in}}{\pgfqpoint{3.105946in}{2.170366in}}{\pgfqpoint{3.100122in}{2.164542in}}%
\pgfpathcurveto{\pgfqpoint{3.094298in}{2.158718in}}{\pgfqpoint{3.091026in}{2.150818in}}{\pgfqpoint{3.091026in}{2.142581in}}%
\pgfpathcurveto{\pgfqpoint{3.091026in}{2.134345in}}{\pgfqpoint{3.094298in}{2.126445in}}{\pgfqpoint{3.100122in}{2.120621in}}%
\pgfpathcurveto{\pgfqpoint{3.105946in}{2.114797in}}{\pgfqpoint{3.113846in}{2.111525in}}{\pgfqpoint{3.122082in}{2.111525in}}%
\pgfpathclose%
\pgfusepath{stroke,fill}%
\end{pgfscope}%
\begin{pgfscope}%
\pgfpathrectangle{\pgfqpoint{0.100000in}{0.212622in}}{\pgfqpoint{3.696000in}{3.696000in}}%
\pgfusepath{clip}%
\pgfsetbuttcap%
\pgfsetroundjoin%
\definecolor{currentfill}{rgb}{0.121569,0.466667,0.705882}%
\pgfsetfillcolor{currentfill}%
\pgfsetfillopacity{0.771306}%
\pgfsetlinewidth{1.003750pt}%
\definecolor{currentstroke}{rgb}{0.121569,0.466667,0.705882}%
\pgfsetstrokecolor{currentstroke}%
\pgfsetstrokeopacity{0.771306}%
\pgfsetdash{}{0pt}%
\pgfpathmoveto{\pgfqpoint{1.040304in}{1.620403in}}%
\pgfpathcurveto{\pgfqpoint{1.048540in}{1.620403in}}{\pgfqpoint{1.056440in}{1.623675in}}{\pgfqpoint{1.062264in}{1.629499in}}%
\pgfpathcurveto{\pgfqpoint{1.068088in}{1.635323in}}{\pgfqpoint{1.071360in}{1.643223in}}{\pgfqpoint{1.071360in}{1.651459in}}%
\pgfpathcurveto{\pgfqpoint{1.071360in}{1.659696in}}{\pgfqpoint{1.068088in}{1.667596in}}{\pgfqpoint{1.062264in}{1.673420in}}%
\pgfpathcurveto{\pgfqpoint{1.056440in}{1.679244in}}{\pgfqpoint{1.048540in}{1.682516in}}{\pgfqpoint{1.040304in}{1.682516in}}%
\pgfpathcurveto{\pgfqpoint{1.032067in}{1.682516in}}{\pgfqpoint{1.024167in}{1.679244in}}{\pgfqpoint{1.018343in}{1.673420in}}%
\pgfpathcurveto{\pgfqpoint{1.012520in}{1.667596in}}{\pgfqpoint{1.009247in}{1.659696in}}{\pgfqpoint{1.009247in}{1.651459in}}%
\pgfpathcurveto{\pgfqpoint{1.009247in}{1.643223in}}{\pgfqpoint{1.012520in}{1.635323in}}{\pgfqpoint{1.018343in}{1.629499in}}%
\pgfpathcurveto{\pgfqpoint{1.024167in}{1.623675in}}{\pgfqpoint{1.032067in}{1.620403in}}{\pgfqpoint{1.040304in}{1.620403in}}%
\pgfpathclose%
\pgfusepath{stroke,fill}%
\end{pgfscope}%
\begin{pgfscope}%
\pgfpathrectangle{\pgfqpoint{0.100000in}{0.212622in}}{\pgfqpoint{3.696000in}{3.696000in}}%
\pgfusepath{clip}%
\pgfsetbuttcap%
\pgfsetroundjoin%
\definecolor{currentfill}{rgb}{0.121569,0.466667,0.705882}%
\pgfsetfillcolor{currentfill}%
\pgfsetfillopacity{0.771525}%
\pgfsetlinewidth{1.003750pt}%
\definecolor{currentstroke}{rgb}{0.121569,0.466667,0.705882}%
\pgfsetstrokecolor{currentstroke}%
\pgfsetstrokeopacity{0.771525}%
\pgfsetdash{}{0pt}%
\pgfpathmoveto{\pgfqpoint{2.528695in}{2.943345in}}%
\pgfpathcurveto{\pgfqpoint{2.536931in}{2.943345in}}{\pgfqpoint{2.544831in}{2.946617in}}{\pgfqpoint{2.550655in}{2.952441in}}%
\pgfpathcurveto{\pgfqpoint{2.556479in}{2.958265in}}{\pgfqpoint{2.559751in}{2.966165in}}{\pgfqpoint{2.559751in}{2.974401in}}%
\pgfpathcurveto{\pgfqpoint{2.559751in}{2.982637in}}{\pgfqpoint{2.556479in}{2.990538in}}{\pgfqpoint{2.550655in}{2.996361in}}%
\pgfpathcurveto{\pgfqpoint{2.544831in}{3.002185in}}{\pgfqpoint{2.536931in}{3.005458in}}{\pgfqpoint{2.528695in}{3.005458in}}%
\pgfpathcurveto{\pgfqpoint{2.520458in}{3.005458in}}{\pgfqpoint{2.512558in}{3.002185in}}{\pgfqpoint{2.506734in}{2.996361in}}%
\pgfpathcurveto{\pgfqpoint{2.500910in}{2.990538in}}{\pgfqpoint{2.497638in}{2.982637in}}{\pgfqpoint{2.497638in}{2.974401in}}%
\pgfpathcurveto{\pgfqpoint{2.497638in}{2.966165in}}{\pgfqpoint{2.500910in}{2.958265in}}{\pgfqpoint{2.506734in}{2.952441in}}%
\pgfpathcurveto{\pgfqpoint{2.512558in}{2.946617in}}{\pgfqpoint{2.520458in}{2.943345in}}{\pgfqpoint{2.528695in}{2.943345in}}%
\pgfpathclose%
\pgfusepath{stroke,fill}%
\end{pgfscope}%
\begin{pgfscope}%
\pgfpathrectangle{\pgfqpoint{0.100000in}{0.212622in}}{\pgfqpoint{3.696000in}{3.696000in}}%
\pgfusepath{clip}%
\pgfsetbuttcap%
\pgfsetroundjoin%
\definecolor{currentfill}{rgb}{0.121569,0.466667,0.705882}%
\pgfsetfillcolor{currentfill}%
\pgfsetfillopacity{0.772019}%
\pgfsetlinewidth{1.003750pt}%
\definecolor{currentstroke}{rgb}{0.121569,0.466667,0.705882}%
\pgfsetstrokecolor{currentstroke}%
\pgfsetstrokeopacity{0.772019}%
\pgfsetdash{}{0pt}%
\pgfpathmoveto{\pgfqpoint{2.531950in}{2.942779in}}%
\pgfpathcurveto{\pgfqpoint{2.540186in}{2.942779in}}{\pgfqpoint{2.548086in}{2.946051in}}{\pgfqpoint{2.553910in}{2.951875in}}%
\pgfpathcurveto{\pgfqpoint{2.559734in}{2.957699in}}{\pgfqpoint{2.563006in}{2.965599in}}{\pgfqpoint{2.563006in}{2.973835in}}%
\pgfpathcurveto{\pgfqpoint{2.563006in}{2.982071in}}{\pgfqpoint{2.559734in}{2.989971in}}{\pgfqpoint{2.553910in}{2.995795in}}%
\pgfpathcurveto{\pgfqpoint{2.548086in}{3.001619in}}{\pgfqpoint{2.540186in}{3.004892in}}{\pgfqpoint{2.531950in}{3.004892in}}%
\pgfpathcurveto{\pgfqpoint{2.523714in}{3.004892in}}{\pgfqpoint{2.515813in}{3.001619in}}{\pgfqpoint{2.509990in}{2.995795in}}%
\pgfpathcurveto{\pgfqpoint{2.504166in}{2.989971in}}{\pgfqpoint{2.500893in}{2.982071in}}{\pgfqpoint{2.500893in}{2.973835in}}%
\pgfpathcurveto{\pgfqpoint{2.500893in}{2.965599in}}{\pgfqpoint{2.504166in}{2.957699in}}{\pgfqpoint{2.509990in}{2.951875in}}%
\pgfpathcurveto{\pgfqpoint{2.515813in}{2.946051in}}{\pgfqpoint{2.523714in}{2.942779in}}{\pgfqpoint{2.531950in}{2.942779in}}%
\pgfpathclose%
\pgfusepath{stroke,fill}%
\end{pgfscope}%
\begin{pgfscope}%
\pgfpathrectangle{\pgfqpoint{0.100000in}{0.212622in}}{\pgfqpoint{3.696000in}{3.696000in}}%
\pgfusepath{clip}%
\pgfsetbuttcap%
\pgfsetroundjoin%
\definecolor{currentfill}{rgb}{0.121569,0.466667,0.705882}%
\pgfsetfillcolor{currentfill}%
\pgfsetfillopacity{0.772204}%
\pgfsetlinewidth{1.003750pt}%
\definecolor{currentstroke}{rgb}{0.121569,0.466667,0.705882}%
\pgfsetstrokecolor{currentstroke}%
\pgfsetstrokeopacity{0.772204}%
\pgfsetdash{}{0pt}%
\pgfpathmoveto{\pgfqpoint{1.036404in}{1.612455in}}%
\pgfpathcurveto{\pgfqpoint{1.044641in}{1.612455in}}{\pgfqpoint{1.052541in}{1.615727in}}{\pgfqpoint{1.058365in}{1.621551in}}%
\pgfpathcurveto{\pgfqpoint{1.064188in}{1.627375in}}{\pgfqpoint{1.067461in}{1.635275in}}{\pgfqpoint{1.067461in}{1.643512in}}%
\pgfpathcurveto{\pgfqpoint{1.067461in}{1.651748in}}{\pgfqpoint{1.064188in}{1.659648in}}{\pgfqpoint{1.058365in}{1.665472in}}%
\pgfpathcurveto{\pgfqpoint{1.052541in}{1.671296in}}{\pgfqpoint{1.044641in}{1.674568in}}{\pgfqpoint{1.036404in}{1.674568in}}%
\pgfpathcurveto{\pgfqpoint{1.028168in}{1.674568in}}{\pgfqpoint{1.020268in}{1.671296in}}{\pgfqpoint{1.014444in}{1.665472in}}%
\pgfpathcurveto{\pgfqpoint{1.008620in}{1.659648in}}{\pgfqpoint{1.005348in}{1.651748in}}{\pgfqpoint{1.005348in}{1.643512in}}%
\pgfpathcurveto{\pgfqpoint{1.005348in}{1.635275in}}{\pgfqpoint{1.008620in}{1.627375in}}{\pgfqpoint{1.014444in}{1.621551in}}%
\pgfpathcurveto{\pgfqpoint{1.020268in}{1.615727in}}{\pgfqpoint{1.028168in}{1.612455in}}{\pgfqpoint{1.036404in}{1.612455in}}%
\pgfpathclose%
\pgfusepath{stroke,fill}%
\end{pgfscope}%
\begin{pgfscope}%
\pgfpathrectangle{\pgfqpoint{0.100000in}{0.212622in}}{\pgfqpoint{3.696000in}{3.696000in}}%
\pgfusepath{clip}%
\pgfsetbuttcap%
\pgfsetroundjoin%
\definecolor{currentfill}{rgb}{0.121569,0.466667,0.705882}%
\pgfsetfillcolor{currentfill}%
\pgfsetfillopacity{0.772532}%
\pgfsetlinewidth{1.003750pt}%
\definecolor{currentstroke}{rgb}{0.121569,0.466667,0.705882}%
\pgfsetstrokecolor{currentstroke}%
\pgfsetstrokeopacity{0.772532}%
\pgfsetdash{}{0pt}%
\pgfpathmoveto{\pgfqpoint{2.534425in}{2.943301in}}%
\pgfpathcurveto{\pgfqpoint{2.542661in}{2.943301in}}{\pgfqpoint{2.550562in}{2.946574in}}{\pgfqpoint{2.556385in}{2.952398in}}%
\pgfpathcurveto{\pgfqpoint{2.562209in}{2.958222in}}{\pgfqpoint{2.565482in}{2.966122in}}{\pgfqpoint{2.565482in}{2.974358in}}%
\pgfpathcurveto{\pgfqpoint{2.565482in}{2.982594in}}{\pgfqpoint{2.562209in}{2.990494in}}{\pgfqpoint{2.556385in}{2.996318in}}%
\pgfpathcurveto{\pgfqpoint{2.550562in}{3.002142in}}{\pgfqpoint{2.542661in}{3.005414in}}{\pgfqpoint{2.534425in}{3.005414in}}%
\pgfpathcurveto{\pgfqpoint{2.526189in}{3.005414in}}{\pgfqpoint{2.518289in}{3.002142in}}{\pgfqpoint{2.512465in}{2.996318in}}%
\pgfpathcurveto{\pgfqpoint{2.506641in}{2.990494in}}{\pgfqpoint{2.503369in}{2.982594in}}{\pgfqpoint{2.503369in}{2.974358in}}%
\pgfpathcurveto{\pgfqpoint{2.503369in}{2.966122in}}{\pgfqpoint{2.506641in}{2.958222in}}{\pgfqpoint{2.512465in}{2.952398in}}%
\pgfpathcurveto{\pgfqpoint{2.518289in}{2.946574in}}{\pgfqpoint{2.526189in}{2.943301in}}{\pgfqpoint{2.534425in}{2.943301in}}%
\pgfpathclose%
\pgfusepath{stroke,fill}%
\end{pgfscope}%
\begin{pgfscope}%
\pgfpathrectangle{\pgfqpoint{0.100000in}{0.212622in}}{\pgfqpoint{3.696000in}{3.696000in}}%
\pgfusepath{clip}%
\pgfsetbuttcap%
\pgfsetroundjoin%
\definecolor{currentfill}{rgb}{0.121569,0.466667,0.705882}%
\pgfsetfillcolor{currentfill}%
\pgfsetfillopacity{0.772536}%
\pgfsetlinewidth{1.003750pt}%
\definecolor{currentstroke}{rgb}{0.121569,0.466667,0.705882}%
\pgfsetstrokecolor{currentstroke}%
\pgfsetstrokeopacity{0.772536}%
\pgfsetdash{}{0pt}%
\pgfpathmoveto{\pgfqpoint{3.118120in}{2.103441in}}%
\pgfpathcurveto{\pgfqpoint{3.126357in}{2.103441in}}{\pgfqpoint{3.134257in}{2.106714in}}{\pgfqpoint{3.140081in}{2.112537in}}%
\pgfpathcurveto{\pgfqpoint{3.145905in}{2.118361in}}{\pgfqpoint{3.149177in}{2.126261in}}{\pgfqpoint{3.149177in}{2.134498in}}%
\pgfpathcurveto{\pgfqpoint{3.149177in}{2.142734in}}{\pgfqpoint{3.145905in}{2.150634in}}{\pgfqpoint{3.140081in}{2.156458in}}%
\pgfpathcurveto{\pgfqpoint{3.134257in}{2.162282in}}{\pgfqpoint{3.126357in}{2.165554in}}{\pgfqpoint{3.118120in}{2.165554in}}%
\pgfpathcurveto{\pgfqpoint{3.109884in}{2.165554in}}{\pgfqpoint{3.101984in}{2.162282in}}{\pgfqpoint{3.096160in}{2.156458in}}%
\pgfpathcurveto{\pgfqpoint{3.090336in}{2.150634in}}{\pgfqpoint{3.087064in}{2.142734in}}{\pgfqpoint{3.087064in}{2.134498in}}%
\pgfpathcurveto{\pgfqpoint{3.087064in}{2.126261in}}{\pgfqpoint{3.090336in}{2.118361in}}{\pgfqpoint{3.096160in}{2.112537in}}%
\pgfpathcurveto{\pgfqpoint{3.101984in}{2.106714in}}{\pgfqpoint{3.109884in}{2.103441in}}{\pgfqpoint{3.118120in}{2.103441in}}%
\pgfpathclose%
\pgfusepath{stroke,fill}%
\end{pgfscope}%
\begin{pgfscope}%
\pgfpathrectangle{\pgfqpoint{0.100000in}{0.212622in}}{\pgfqpoint{3.696000in}{3.696000in}}%
\pgfusepath{clip}%
\pgfsetbuttcap%
\pgfsetroundjoin%
\definecolor{currentfill}{rgb}{0.121569,0.466667,0.705882}%
\pgfsetfillcolor{currentfill}%
\pgfsetfillopacity{0.772893}%
\pgfsetlinewidth{1.003750pt}%
\definecolor{currentstroke}{rgb}{0.121569,0.466667,0.705882}%
\pgfsetstrokecolor{currentstroke}%
\pgfsetstrokeopacity{0.772893}%
\pgfsetdash{}{0pt}%
\pgfpathmoveto{\pgfqpoint{1.031578in}{1.604096in}}%
\pgfpathcurveto{\pgfqpoint{1.039815in}{1.604096in}}{\pgfqpoint{1.047715in}{1.607369in}}{\pgfqpoint{1.053539in}{1.613193in}}%
\pgfpathcurveto{\pgfqpoint{1.059363in}{1.619016in}}{\pgfqpoint{1.062635in}{1.626917in}}{\pgfqpoint{1.062635in}{1.635153in}}%
\pgfpathcurveto{\pgfqpoint{1.062635in}{1.643389in}}{\pgfqpoint{1.059363in}{1.651289in}}{\pgfqpoint{1.053539in}{1.657113in}}%
\pgfpathcurveto{\pgfqpoint{1.047715in}{1.662937in}}{\pgfqpoint{1.039815in}{1.666209in}}{\pgfqpoint{1.031578in}{1.666209in}}%
\pgfpathcurveto{\pgfqpoint{1.023342in}{1.666209in}}{\pgfqpoint{1.015442in}{1.662937in}}{\pgfqpoint{1.009618in}{1.657113in}}%
\pgfpathcurveto{\pgfqpoint{1.003794in}{1.651289in}}{\pgfqpoint{1.000522in}{1.643389in}}{\pgfqpoint{1.000522in}{1.635153in}}%
\pgfpathcurveto{\pgfqpoint{1.000522in}{1.626917in}}{\pgfqpoint{1.003794in}{1.619016in}}{\pgfqpoint{1.009618in}{1.613193in}}%
\pgfpathcurveto{\pgfqpoint{1.015442in}{1.607369in}}{\pgfqpoint{1.023342in}{1.604096in}}{\pgfqpoint{1.031578in}{1.604096in}}%
\pgfpathclose%
\pgfusepath{stroke,fill}%
\end{pgfscope}%
\begin{pgfscope}%
\pgfpathrectangle{\pgfqpoint{0.100000in}{0.212622in}}{\pgfqpoint{3.696000in}{3.696000in}}%
\pgfusepath{clip}%
\pgfsetbuttcap%
\pgfsetroundjoin%
\definecolor{currentfill}{rgb}{0.121569,0.466667,0.705882}%
\pgfsetfillcolor{currentfill}%
\pgfsetfillopacity{0.773319}%
\pgfsetlinewidth{1.003750pt}%
\definecolor{currentstroke}{rgb}{0.121569,0.466667,0.705882}%
\pgfsetstrokecolor{currentstroke}%
\pgfsetstrokeopacity{0.773319}%
\pgfsetdash{}{0pt}%
\pgfpathmoveto{\pgfqpoint{2.538979in}{2.943849in}}%
\pgfpathcurveto{\pgfqpoint{2.547215in}{2.943849in}}{\pgfqpoint{2.555115in}{2.947121in}}{\pgfqpoint{2.560939in}{2.952945in}}%
\pgfpathcurveto{\pgfqpoint{2.566763in}{2.958769in}}{\pgfqpoint{2.570035in}{2.966669in}}{\pgfqpoint{2.570035in}{2.974906in}}%
\pgfpathcurveto{\pgfqpoint{2.570035in}{2.983142in}}{\pgfqpoint{2.566763in}{2.991042in}}{\pgfqpoint{2.560939in}{2.996866in}}%
\pgfpathcurveto{\pgfqpoint{2.555115in}{3.002690in}}{\pgfqpoint{2.547215in}{3.005962in}}{\pgfqpoint{2.538979in}{3.005962in}}%
\pgfpathcurveto{\pgfqpoint{2.530742in}{3.005962in}}{\pgfqpoint{2.522842in}{3.002690in}}{\pgfqpoint{2.517018in}{2.996866in}}%
\pgfpathcurveto{\pgfqpoint{2.511194in}{2.991042in}}{\pgfqpoint{2.507922in}{2.983142in}}{\pgfqpoint{2.507922in}{2.974906in}}%
\pgfpathcurveto{\pgfqpoint{2.507922in}{2.966669in}}{\pgfqpoint{2.511194in}{2.958769in}}{\pgfqpoint{2.517018in}{2.952945in}}%
\pgfpathcurveto{\pgfqpoint{2.522842in}{2.947121in}}{\pgfqpoint{2.530742in}{2.943849in}}{\pgfqpoint{2.538979in}{2.943849in}}%
\pgfpathclose%
\pgfusepath{stroke,fill}%
\end{pgfscope}%
\begin{pgfscope}%
\pgfpathrectangle{\pgfqpoint{0.100000in}{0.212622in}}{\pgfqpoint{3.696000in}{3.696000in}}%
\pgfusepath{clip}%
\pgfsetbuttcap%
\pgfsetroundjoin%
\definecolor{currentfill}{rgb}{0.121569,0.466667,0.705882}%
\pgfsetfillcolor{currentfill}%
\pgfsetfillopacity{0.773552}%
\pgfsetlinewidth{1.003750pt}%
\definecolor{currentstroke}{rgb}{0.121569,0.466667,0.705882}%
\pgfsetstrokecolor{currentstroke}%
\pgfsetstrokeopacity{0.773552}%
\pgfsetdash{}{0pt}%
\pgfpathmoveto{\pgfqpoint{2.542910in}{2.943252in}}%
\pgfpathcurveto{\pgfqpoint{2.551146in}{2.943252in}}{\pgfqpoint{2.559046in}{2.946524in}}{\pgfqpoint{2.564870in}{2.952348in}}%
\pgfpathcurveto{\pgfqpoint{2.570694in}{2.958172in}}{\pgfqpoint{2.573966in}{2.966072in}}{\pgfqpoint{2.573966in}{2.974308in}}%
\pgfpathcurveto{\pgfqpoint{2.573966in}{2.982545in}}{\pgfqpoint{2.570694in}{2.990445in}}{\pgfqpoint{2.564870in}{2.996269in}}%
\pgfpathcurveto{\pgfqpoint{2.559046in}{3.002092in}}{\pgfqpoint{2.551146in}{3.005365in}}{\pgfqpoint{2.542910in}{3.005365in}}%
\pgfpathcurveto{\pgfqpoint{2.534673in}{3.005365in}}{\pgfqpoint{2.526773in}{3.002092in}}{\pgfqpoint{2.520949in}{2.996269in}}%
\pgfpathcurveto{\pgfqpoint{2.515126in}{2.990445in}}{\pgfqpoint{2.511853in}{2.982545in}}{\pgfqpoint{2.511853in}{2.974308in}}%
\pgfpathcurveto{\pgfqpoint{2.511853in}{2.966072in}}{\pgfqpoint{2.515126in}{2.958172in}}{\pgfqpoint{2.520949in}{2.952348in}}%
\pgfpathcurveto{\pgfqpoint{2.526773in}{2.946524in}}{\pgfqpoint{2.534673in}{2.943252in}}{\pgfqpoint{2.542910in}{2.943252in}}%
\pgfpathclose%
\pgfusepath{stroke,fill}%
\end{pgfscope}%
\begin{pgfscope}%
\pgfpathrectangle{\pgfqpoint{0.100000in}{0.212622in}}{\pgfqpoint{3.696000in}{3.696000in}}%
\pgfusepath{clip}%
\pgfsetbuttcap%
\pgfsetroundjoin%
\definecolor{currentfill}{rgb}{0.121569,0.466667,0.705882}%
\pgfsetfillcolor{currentfill}%
\pgfsetfillopacity{0.773669}%
\pgfsetlinewidth{1.003750pt}%
\definecolor{currentstroke}{rgb}{0.121569,0.466667,0.705882}%
\pgfsetstrokecolor{currentstroke}%
\pgfsetstrokeopacity{0.773669}%
\pgfsetdash{}{0pt}%
\pgfpathmoveto{\pgfqpoint{1.025636in}{1.595977in}}%
\pgfpathcurveto{\pgfqpoint{1.033872in}{1.595977in}}{\pgfqpoint{1.041772in}{1.599250in}}{\pgfqpoint{1.047596in}{1.605073in}}%
\pgfpathcurveto{\pgfqpoint{1.053420in}{1.610897in}}{\pgfqpoint{1.056692in}{1.618797in}}{\pgfqpoint{1.056692in}{1.627034in}}%
\pgfpathcurveto{\pgfqpoint{1.056692in}{1.635270in}}{\pgfqpoint{1.053420in}{1.643170in}}{\pgfqpoint{1.047596in}{1.648994in}}%
\pgfpathcurveto{\pgfqpoint{1.041772in}{1.654818in}}{\pgfqpoint{1.033872in}{1.658090in}}{\pgfqpoint{1.025636in}{1.658090in}}%
\pgfpathcurveto{\pgfqpoint{1.017400in}{1.658090in}}{\pgfqpoint{1.009500in}{1.654818in}}{\pgfqpoint{1.003676in}{1.648994in}}%
\pgfpathcurveto{\pgfqpoint{0.997852in}{1.643170in}}{\pgfqpoint{0.994579in}{1.635270in}}{\pgfqpoint{0.994579in}{1.627034in}}%
\pgfpathcurveto{\pgfqpoint{0.994579in}{1.618797in}}{\pgfqpoint{0.997852in}{1.610897in}}{\pgfqpoint{1.003676in}{1.605073in}}%
\pgfpathcurveto{\pgfqpoint{1.009500in}{1.599250in}}{\pgfqpoint{1.017400in}{1.595977in}}{\pgfqpoint{1.025636in}{1.595977in}}%
\pgfpathclose%
\pgfusepath{stroke,fill}%
\end{pgfscope}%
\begin{pgfscope}%
\pgfpathrectangle{\pgfqpoint{0.100000in}{0.212622in}}{\pgfqpoint{3.696000in}{3.696000in}}%
\pgfusepath{clip}%
\pgfsetbuttcap%
\pgfsetroundjoin%
\definecolor{currentfill}{rgb}{0.121569,0.466667,0.705882}%
\pgfsetfillcolor{currentfill}%
\pgfsetfillopacity{0.774132}%
\pgfsetlinewidth{1.003750pt}%
\definecolor{currentstroke}{rgb}{0.121569,0.466667,0.705882}%
\pgfsetstrokecolor{currentstroke}%
\pgfsetstrokeopacity{0.774132}%
\pgfsetdash{}{0pt}%
\pgfpathmoveto{\pgfqpoint{2.546115in}{2.944049in}}%
\pgfpathcurveto{\pgfqpoint{2.554351in}{2.944049in}}{\pgfqpoint{2.562251in}{2.947321in}}{\pgfqpoint{2.568075in}{2.953145in}}%
\pgfpathcurveto{\pgfqpoint{2.573899in}{2.958969in}}{\pgfqpoint{2.577171in}{2.966869in}}{\pgfqpoint{2.577171in}{2.975105in}}%
\pgfpathcurveto{\pgfqpoint{2.577171in}{2.983342in}}{\pgfqpoint{2.573899in}{2.991242in}}{\pgfqpoint{2.568075in}{2.997066in}}%
\pgfpathcurveto{\pgfqpoint{2.562251in}{3.002890in}}{\pgfqpoint{2.554351in}{3.006162in}}{\pgfqpoint{2.546115in}{3.006162in}}%
\pgfpathcurveto{\pgfqpoint{2.537878in}{3.006162in}}{\pgfqpoint{2.529978in}{3.002890in}}{\pgfqpoint{2.524154in}{2.997066in}}%
\pgfpathcurveto{\pgfqpoint{2.518330in}{2.991242in}}{\pgfqpoint{2.515058in}{2.983342in}}{\pgfqpoint{2.515058in}{2.975105in}}%
\pgfpathcurveto{\pgfqpoint{2.515058in}{2.966869in}}{\pgfqpoint{2.518330in}{2.958969in}}{\pgfqpoint{2.524154in}{2.953145in}}%
\pgfpathcurveto{\pgfqpoint{2.529978in}{2.947321in}}{\pgfqpoint{2.537878in}{2.944049in}}{\pgfqpoint{2.546115in}{2.944049in}}%
\pgfpathclose%
\pgfusepath{stroke,fill}%
\end{pgfscope}%
\begin{pgfscope}%
\pgfpathrectangle{\pgfqpoint{0.100000in}{0.212622in}}{\pgfqpoint{3.696000in}{3.696000in}}%
\pgfusepath{clip}%
\pgfsetbuttcap%
\pgfsetroundjoin%
\definecolor{currentfill}{rgb}{0.121569,0.466667,0.705882}%
\pgfsetfillcolor{currentfill}%
\pgfsetfillopacity{0.774319}%
\pgfsetlinewidth{1.003750pt}%
\definecolor{currentstroke}{rgb}{0.121569,0.466667,0.705882}%
\pgfsetstrokecolor{currentstroke}%
\pgfsetstrokeopacity{0.774319}%
\pgfsetdash{}{0pt}%
\pgfpathmoveto{\pgfqpoint{1.022590in}{1.591823in}}%
\pgfpathcurveto{\pgfqpoint{1.030826in}{1.591823in}}{\pgfqpoint{1.038726in}{1.595095in}}{\pgfqpoint{1.044550in}{1.600919in}}%
\pgfpathcurveto{\pgfqpoint{1.050374in}{1.606743in}}{\pgfqpoint{1.053646in}{1.614643in}}{\pgfqpoint{1.053646in}{1.622879in}}%
\pgfpathcurveto{\pgfqpoint{1.053646in}{1.631116in}}{\pgfqpoint{1.050374in}{1.639016in}}{\pgfqpoint{1.044550in}{1.644840in}}%
\pgfpathcurveto{\pgfqpoint{1.038726in}{1.650664in}}{\pgfqpoint{1.030826in}{1.653936in}}{\pgfqpoint{1.022590in}{1.653936in}}%
\pgfpathcurveto{\pgfqpoint{1.014354in}{1.653936in}}{\pgfqpoint{1.006454in}{1.650664in}}{\pgfqpoint{1.000630in}{1.644840in}}%
\pgfpathcurveto{\pgfqpoint{0.994806in}{1.639016in}}{\pgfqpoint{0.991533in}{1.631116in}}{\pgfqpoint{0.991533in}{1.622879in}}%
\pgfpathcurveto{\pgfqpoint{0.991533in}{1.614643in}}{\pgfqpoint{0.994806in}{1.606743in}}{\pgfqpoint{1.000630in}{1.600919in}}%
\pgfpathcurveto{\pgfqpoint{1.006454in}{1.595095in}}{\pgfqpoint{1.014354in}{1.591823in}}{\pgfqpoint{1.022590in}{1.591823in}}%
\pgfpathclose%
\pgfusepath{stroke,fill}%
\end{pgfscope}%
\begin{pgfscope}%
\pgfpathrectangle{\pgfqpoint{0.100000in}{0.212622in}}{\pgfqpoint{3.696000in}{3.696000in}}%
\pgfusepath{clip}%
\pgfsetbuttcap%
\pgfsetroundjoin%
\definecolor{currentfill}{rgb}{0.121569,0.466667,0.705882}%
\pgfsetfillcolor{currentfill}%
\pgfsetfillopacity{0.774661}%
\pgfsetlinewidth{1.003750pt}%
\definecolor{currentstroke}{rgb}{0.121569,0.466667,0.705882}%
\pgfsetstrokecolor{currentstroke}%
\pgfsetstrokeopacity{0.774661}%
\pgfsetdash{}{0pt}%
\pgfpathmoveto{\pgfqpoint{2.548592in}{2.944665in}}%
\pgfpathcurveto{\pgfqpoint{2.556829in}{2.944665in}}{\pgfqpoint{2.564729in}{2.947937in}}{\pgfqpoint{2.570553in}{2.953761in}}%
\pgfpathcurveto{\pgfqpoint{2.576377in}{2.959585in}}{\pgfqpoint{2.579649in}{2.967485in}}{\pgfqpoint{2.579649in}{2.975721in}}%
\pgfpathcurveto{\pgfqpoint{2.579649in}{2.983957in}}{\pgfqpoint{2.576377in}{2.991857in}}{\pgfqpoint{2.570553in}{2.997681in}}%
\pgfpathcurveto{\pgfqpoint{2.564729in}{3.003505in}}{\pgfqpoint{2.556829in}{3.006778in}}{\pgfqpoint{2.548592in}{3.006778in}}%
\pgfpathcurveto{\pgfqpoint{2.540356in}{3.006778in}}{\pgfqpoint{2.532456in}{3.003505in}}{\pgfqpoint{2.526632in}{2.997681in}}%
\pgfpathcurveto{\pgfqpoint{2.520808in}{2.991857in}}{\pgfqpoint{2.517536in}{2.983957in}}{\pgfqpoint{2.517536in}{2.975721in}}%
\pgfpathcurveto{\pgfqpoint{2.517536in}{2.967485in}}{\pgfqpoint{2.520808in}{2.959585in}}{\pgfqpoint{2.526632in}{2.953761in}}%
\pgfpathcurveto{\pgfqpoint{2.532456in}{2.947937in}}{\pgfqpoint{2.540356in}{2.944665in}}{\pgfqpoint{2.548592in}{2.944665in}}%
\pgfpathclose%
\pgfusepath{stroke,fill}%
\end{pgfscope}%
\begin{pgfscope}%
\pgfpathrectangle{\pgfqpoint{0.100000in}{0.212622in}}{\pgfqpoint{3.696000in}{3.696000in}}%
\pgfusepath{clip}%
\pgfsetbuttcap%
\pgfsetroundjoin%
\definecolor{currentfill}{rgb}{0.121569,0.466667,0.705882}%
\pgfsetfillcolor{currentfill}%
\pgfsetfillopacity{0.774917}%
\pgfsetlinewidth{1.003750pt}%
\definecolor{currentstroke}{rgb}{0.121569,0.466667,0.705882}%
\pgfsetstrokecolor{currentstroke}%
\pgfsetstrokeopacity{0.774917}%
\pgfsetdash{}{0pt}%
\pgfpathmoveto{\pgfqpoint{2.550603in}{2.944642in}}%
\pgfpathcurveto{\pgfqpoint{2.558839in}{2.944642in}}{\pgfqpoint{2.566739in}{2.947914in}}{\pgfqpoint{2.572563in}{2.953738in}}%
\pgfpathcurveto{\pgfqpoint{2.578387in}{2.959562in}}{\pgfqpoint{2.581659in}{2.967462in}}{\pgfqpoint{2.581659in}{2.975698in}}%
\pgfpathcurveto{\pgfqpoint{2.581659in}{2.983935in}}{\pgfqpoint{2.578387in}{2.991835in}}{\pgfqpoint{2.572563in}{2.997659in}}%
\pgfpathcurveto{\pgfqpoint{2.566739in}{3.003483in}}{\pgfqpoint{2.558839in}{3.006755in}}{\pgfqpoint{2.550603in}{3.006755in}}%
\pgfpathcurveto{\pgfqpoint{2.542366in}{3.006755in}}{\pgfqpoint{2.534466in}{3.003483in}}{\pgfqpoint{2.528642in}{2.997659in}}%
\pgfpathcurveto{\pgfqpoint{2.522818in}{2.991835in}}{\pgfqpoint{2.519546in}{2.983935in}}{\pgfqpoint{2.519546in}{2.975698in}}%
\pgfpathcurveto{\pgfqpoint{2.519546in}{2.967462in}}{\pgfqpoint{2.522818in}{2.959562in}}{\pgfqpoint{2.528642in}{2.953738in}}%
\pgfpathcurveto{\pgfqpoint{2.534466in}{2.947914in}}{\pgfqpoint{2.542366in}{2.944642in}}{\pgfqpoint{2.550603in}{2.944642in}}%
\pgfpathclose%
\pgfusepath{stroke,fill}%
\end{pgfscope}%
\begin{pgfscope}%
\pgfpathrectangle{\pgfqpoint{0.100000in}{0.212622in}}{\pgfqpoint{3.696000in}{3.696000in}}%
\pgfusepath{clip}%
\pgfsetbuttcap%
\pgfsetroundjoin%
\definecolor{currentfill}{rgb}{0.121569,0.466667,0.705882}%
\pgfsetfillcolor{currentfill}%
\pgfsetfillopacity{0.775265}%
\pgfsetlinewidth{1.003750pt}%
\definecolor{currentstroke}{rgb}{0.121569,0.466667,0.705882}%
\pgfsetstrokecolor{currentstroke}%
\pgfsetstrokeopacity{0.775265}%
\pgfsetdash{}{0pt}%
\pgfpathmoveto{\pgfqpoint{1.018963in}{1.584861in}}%
\pgfpathcurveto{\pgfqpoint{1.027199in}{1.584861in}}{\pgfqpoint{1.035100in}{1.588134in}}{\pgfqpoint{1.040923in}{1.593957in}}%
\pgfpathcurveto{\pgfqpoint{1.046747in}{1.599781in}}{\pgfqpoint{1.050020in}{1.607681in}}{\pgfqpoint{1.050020in}{1.615918in}}%
\pgfpathcurveto{\pgfqpoint{1.050020in}{1.624154in}}{\pgfqpoint{1.046747in}{1.632054in}}{\pgfqpoint{1.040923in}{1.637878in}}%
\pgfpathcurveto{\pgfqpoint{1.035100in}{1.643702in}}{\pgfqpoint{1.027199in}{1.646974in}}{\pgfqpoint{1.018963in}{1.646974in}}%
\pgfpathcurveto{\pgfqpoint{1.010727in}{1.646974in}}{\pgfqpoint{1.002827in}{1.643702in}}{\pgfqpoint{0.997003in}{1.637878in}}%
\pgfpathcurveto{\pgfqpoint{0.991179in}{1.632054in}}{\pgfqpoint{0.987907in}{1.624154in}}{\pgfqpoint{0.987907in}{1.615918in}}%
\pgfpathcurveto{\pgfqpoint{0.987907in}{1.607681in}}{\pgfqpoint{0.991179in}{1.599781in}}{\pgfqpoint{0.997003in}{1.593957in}}%
\pgfpathcurveto{\pgfqpoint{1.002827in}{1.588134in}}{\pgfqpoint{1.010727in}{1.584861in}}{\pgfqpoint{1.018963in}{1.584861in}}%
\pgfpathclose%
\pgfusepath{stroke,fill}%
\end{pgfscope}%
\begin{pgfscope}%
\pgfpathrectangle{\pgfqpoint{0.100000in}{0.212622in}}{\pgfqpoint{3.696000in}{3.696000in}}%
\pgfusepath{clip}%
\pgfsetbuttcap%
\pgfsetroundjoin%
\definecolor{currentfill}{rgb}{0.121569,0.466667,0.705882}%
\pgfsetfillcolor{currentfill}%
\pgfsetfillopacity{0.775272}%
\pgfsetlinewidth{1.003750pt}%
\definecolor{currentstroke}{rgb}{0.121569,0.466667,0.705882}%
\pgfsetstrokecolor{currentstroke}%
\pgfsetstrokeopacity{0.775272}%
\pgfsetdash{}{0pt}%
\pgfpathmoveto{\pgfqpoint{2.554308in}{2.944371in}}%
\pgfpathcurveto{\pgfqpoint{2.562544in}{2.944371in}}{\pgfqpoint{2.570444in}{2.947643in}}{\pgfqpoint{2.576268in}{2.953467in}}%
\pgfpathcurveto{\pgfqpoint{2.582092in}{2.959291in}}{\pgfqpoint{2.585364in}{2.967191in}}{\pgfqpoint{2.585364in}{2.975428in}}%
\pgfpathcurveto{\pgfqpoint{2.585364in}{2.983664in}}{\pgfqpoint{2.582092in}{2.991564in}}{\pgfqpoint{2.576268in}{2.997388in}}%
\pgfpathcurveto{\pgfqpoint{2.570444in}{3.003212in}}{\pgfqpoint{2.562544in}{3.006484in}}{\pgfqpoint{2.554308in}{3.006484in}}%
\pgfpathcurveto{\pgfqpoint{2.546072in}{3.006484in}}{\pgfqpoint{2.538172in}{3.003212in}}{\pgfqpoint{2.532348in}{2.997388in}}%
\pgfpathcurveto{\pgfqpoint{2.526524in}{2.991564in}}{\pgfqpoint{2.523251in}{2.983664in}}{\pgfqpoint{2.523251in}{2.975428in}}%
\pgfpathcurveto{\pgfqpoint{2.523251in}{2.967191in}}{\pgfqpoint{2.526524in}{2.959291in}}{\pgfqpoint{2.532348in}{2.953467in}}%
\pgfpathcurveto{\pgfqpoint{2.538172in}{2.947643in}}{\pgfqpoint{2.546072in}{2.944371in}}{\pgfqpoint{2.554308in}{2.944371in}}%
\pgfpathclose%
\pgfusepath{stroke,fill}%
\end{pgfscope}%
\begin{pgfscope}%
\pgfpathrectangle{\pgfqpoint{0.100000in}{0.212622in}}{\pgfqpoint{3.696000in}{3.696000in}}%
\pgfusepath{clip}%
\pgfsetbuttcap%
\pgfsetroundjoin%
\definecolor{currentfill}{rgb}{0.121569,0.466667,0.705882}%
\pgfsetfillcolor{currentfill}%
\pgfsetfillopacity{0.775535}%
\pgfsetlinewidth{1.003750pt}%
\definecolor{currentstroke}{rgb}{0.121569,0.466667,0.705882}%
\pgfsetstrokecolor{currentstroke}%
\pgfsetstrokeopacity{0.775535}%
\pgfsetdash{}{0pt}%
\pgfpathmoveto{\pgfqpoint{3.112211in}{2.088709in}}%
\pgfpathcurveto{\pgfqpoint{3.120447in}{2.088709in}}{\pgfqpoint{3.128347in}{2.091981in}}{\pgfqpoint{3.134171in}{2.097805in}}%
\pgfpathcurveto{\pgfqpoint{3.139995in}{2.103629in}}{\pgfqpoint{3.143268in}{2.111529in}}{\pgfqpoint{3.143268in}{2.119765in}}%
\pgfpathcurveto{\pgfqpoint{3.143268in}{2.128001in}}{\pgfqpoint{3.139995in}{2.135901in}}{\pgfqpoint{3.134171in}{2.141725in}}%
\pgfpathcurveto{\pgfqpoint{3.128347in}{2.147549in}}{\pgfqpoint{3.120447in}{2.150822in}}{\pgfqpoint{3.112211in}{2.150822in}}%
\pgfpathcurveto{\pgfqpoint{3.103975in}{2.150822in}}{\pgfqpoint{3.096075in}{2.147549in}}{\pgfqpoint{3.090251in}{2.141725in}}%
\pgfpathcurveto{\pgfqpoint{3.084427in}{2.135901in}}{\pgfqpoint{3.081155in}{2.128001in}}{\pgfqpoint{3.081155in}{2.119765in}}%
\pgfpathcurveto{\pgfqpoint{3.081155in}{2.111529in}}{\pgfqpoint{3.084427in}{2.103629in}}{\pgfqpoint{3.090251in}{2.097805in}}%
\pgfpathcurveto{\pgfqpoint{3.096075in}{2.091981in}}{\pgfqpoint{3.103975in}{2.088709in}}{\pgfqpoint{3.112211in}{2.088709in}}%
\pgfpathclose%
\pgfusepath{stroke,fill}%
\end{pgfscope}%
\begin{pgfscope}%
\pgfpathrectangle{\pgfqpoint{0.100000in}{0.212622in}}{\pgfqpoint{3.696000in}{3.696000in}}%
\pgfusepath{clip}%
\pgfsetbuttcap%
\pgfsetroundjoin%
\definecolor{currentfill}{rgb}{0.121569,0.466667,0.705882}%
\pgfsetfillcolor{currentfill}%
\pgfsetfillopacity{0.775720}%
\pgfsetlinewidth{1.003750pt}%
\definecolor{currentstroke}{rgb}{0.121569,0.466667,0.705882}%
\pgfsetstrokecolor{currentstroke}%
\pgfsetstrokeopacity{0.775720}%
\pgfsetdash{}{0pt}%
\pgfpathmoveto{\pgfqpoint{2.556538in}{2.944708in}}%
\pgfpathcurveto{\pgfqpoint{2.564775in}{2.944708in}}{\pgfqpoint{2.572675in}{2.947981in}}{\pgfqpoint{2.578499in}{2.953805in}}%
\pgfpathcurveto{\pgfqpoint{2.584323in}{2.959629in}}{\pgfqpoint{2.587595in}{2.967529in}}{\pgfqpoint{2.587595in}{2.975765in}}%
\pgfpathcurveto{\pgfqpoint{2.587595in}{2.984001in}}{\pgfqpoint{2.584323in}{2.991901in}}{\pgfqpoint{2.578499in}{2.997725in}}%
\pgfpathcurveto{\pgfqpoint{2.572675in}{3.003549in}}{\pgfqpoint{2.564775in}{3.006821in}}{\pgfqpoint{2.556538in}{3.006821in}}%
\pgfpathcurveto{\pgfqpoint{2.548302in}{3.006821in}}{\pgfqpoint{2.540402in}{3.003549in}}{\pgfqpoint{2.534578in}{2.997725in}}%
\pgfpathcurveto{\pgfqpoint{2.528754in}{2.991901in}}{\pgfqpoint{2.525482in}{2.984001in}}{\pgfqpoint{2.525482in}{2.975765in}}%
\pgfpathcurveto{\pgfqpoint{2.525482in}{2.967529in}}{\pgfqpoint{2.528754in}{2.959629in}}{\pgfqpoint{2.534578in}{2.953805in}}%
\pgfpathcurveto{\pgfqpoint{2.540402in}{2.947981in}}{\pgfqpoint{2.548302in}{2.944708in}}{\pgfqpoint{2.556538in}{2.944708in}}%
\pgfpathclose%
\pgfusepath{stroke,fill}%
\end{pgfscope}%
\begin{pgfscope}%
\pgfpathrectangle{\pgfqpoint{0.100000in}{0.212622in}}{\pgfqpoint{3.696000in}{3.696000in}}%
\pgfusepath{clip}%
\pgfsetbuttcap%
\pgfsetroundjoin%
\definecolor{currentfill}{rgb}{0.121569,0.466667,0.705882}%
\pgfsetfillcolor{currentfill}%
\pgfsetfillopacity{0.775755}%
\pgfsetlinewidth{1.003750pt}%
\definecolor{currentstroke}{rgb}{0.121569,0.466667,0.705882}%
\pgfsetstrokecolor{currentstroke}%
\pgfsetstrokeopacity{0.775755}%
\pgfsetdash{}{0pt}%
\pgfpathmoveto{\pgfqpoint{1.016938in}{1.580944in}}%
\pgfpathcurveto{\pgfqpoint{1.025174in}{1.580944in}}{\pgfqpoint{1.033074in}{1.584216in}}{\pgfqpoint{1.038898in}{1.590040in}}%
\pgfpathcurveto{\pgfqpoint{1.044722in}{1.595864in}}{\pgfqpoint{1.047994in}{1.603764in}}{\pgfqpoint{1.047994in}{1.612000in}}%
\pgfpathcurveto{\pgfqpoint{1.047994in}{1.620236in}}{\pgfqpoint{1.044722in}{1.628136in}}{\pgfqpoint{1.038898in}{1.633960in}}%
\pgfpathcurveto{\pgfqpoint{1.033074in}{1.639784in}}{\pgfqpoint{1.025174in}{1.643057in}}{\pgfqpoint{1.016938in}{1.643057in}}%
\pgfpathcurveto{\pgfqpoint{1.008702in}{1.643057in}}{\pgfqpoint{1.000802in}{1.639784in}}{\pgfqpoint{0.994978in}{1.633960in}}%
\pgfpathcurveto{\pgfqpoint{0.989154in}{1.628136in}}{\pgfqpoint{0.985881in}{1.620236in}}{\pgfqpoint{0.985881in}{1.612000in}}%
\pgfpathcurveto{\pgfqpoint{0.985881in}{1.603764in}}{\pgfqpoint{0.989154in}{1.595864in}}{\pgfqpoint{0.994978in}{1.590040in}}%
\pgfpathcurveto{\pgfqpoint{1.000802in}{1.584216in}}{\pgfqpoint{1.008702in}{1.580944in}}{\pgfqpoint{1.016938in}{1.580944in}}%
\pgfpathclose%
\pgfusepath{stroke,fill}%
\end{pgfscope}%
\begin{pgfscope}%
\pgfpathrectangle{\pgfqpoint{0.100000in}{0.212622in}}{\pgfqpoint{3.696000in}{3.696000in}}%
\pgfusepath{clip}%
\pgfsetbuttcap%
\pgfsetroundjoin%
\definecolor{currentfill}{rgb}{0.121569,0.466667,0.705882}%
\pgfsetfillcolor{currentfill}%
\pgfsetfillopacity{0.776027}%
\pgfsetlinewidth{1.003750pt}%
\definecolor{currentstroke}{rgb}{0.121569,0.466667,0.705882}%
\pgfsetstrokecolor{currentstroke}%
\pgfsetstrokeopacity{0.776027}%
\pgfsetdash{}{0pt}%
\pgfpathmoveto{\pgfqpoint{1.015872in}{1.578736in}}%
\pgfpathcurveto{\pgfqpoint{1.024108in}{1.578736in}}{\pgfqpoint{1.032008in}{1.582009in}}{\pgfqpoint{1.037832in}{1.587833in}}%
\pgfpathcurveto{\pgfqpoint{1.043656in}{1.593656in}}{\pgfqpoint{1.046928in}{1.601557in}}{\pgfqpoint{1.046928in}{1.609793in}}%
\pgfpathcurveto{\pgfqpoint{1.046928in}{1.618029in}}{\pgfqpoint{1.043656in}{1.625929in}}{\pgfqpoint{1.037832in}{1.631753in}}%
\pgfpathcurveto{\pgfqpoint{1.032008in}{1.637577in}}{\pgfqpoint{1.024108in}{1.640849in}}{\pgfqpoint{1.015872in}{1.640849in}}%
\pgfpathcurveto{\pgfqpoint{1.007635in}{1.640849in}}{\pgfqpoint{0.999735in}{1.637577in}}{\pgfqpoint{0.993911in}{1.631753in}}%
\pgfpathcurveto{\pgfqpoint{0.988087in}{1.625929in}}{\pgfqpoint{0.984815in}{1.618029in}}{\pgfqpoint{0.984815in}{1.609793in}}%
\pgfpathcurveto{\pgfqpoint{0.984815in}{1.601557in}}{\pgfqpoint{0.988087in}{1.593656in}}{\pgfqpoint{0.993911in}{1.587833in}}%
\pgfpathcurveto{\pgfqpoint{0.999735in}{1.582009in}}{\pgfqpoint{1.007635in}{1.578736in}}{\pgfqpoint{1.015872in}{1.578736in}}%
\pgfpathclose%
\pgfusepath{stroke,fill}%
\end{pgfscope}%
\begin{pgfscope}%
\pgfpathrectangle{\pgfqpoint{0.100000in}{0.212622in}}{\pgfqpoint{3.696000in}{3.696000in}}%
\pgfusepath{clip}%
\pgfsetbuttcap%
\pgfsetroundjoin%
\definecolor{currentfill}{rgb}{0.121569,0.466667,0.705882}%
\pgfsetfillcolor{currentfill}%
\pgfsetfillopacity{0.776172}%
\pgfsetlinewidth{1.003750pt}%
\definecolor{currentstroke}{rgb}{0.121569,0.466667,0.705882}%
\pgfsetstrokecolor{currentstroke}%
\pgfsetstrokeopacity{0.776172}%
\pgfsetdash{}{0pt}%
\pgfpathmoveto{\pgfqpoint{1.015231in}{1.577578in}}%
\pgfpathcurveto{\pgfqpoint{1.023468in}{1.577578in}}{\pgfqpoint{1.031368in}{1.580851in}}{\pgfqpoint{1.037192in}{1.586675in}}%
\pgfpathcurveto{\pgfqpoint{1.043015in}{1.592499in}}{\pgfqpoint{1.046288in}{1.600399in}}{\pgfqpoint{1.046288in}{1.608635in}}%
\pgfpathcurveto{\pgfqpoint{1.046288in}{1.616871in}}{\pgfqpoint{1.043015in}{1.624771in}}{\pgfqpoint{1.037192in}{1.630595in}}%
\pgfpathcurveto{\pgfqpoint{1.031368in}{1.636419in}}{\pgfqpoint{1.023468in}{1.639691in}}{\pgfqpoint{1.015231in}{1.639691in}}%
\pgfpathcurveto{\pgfqpoint{1.006995in}{1.639691in}}{\pgfqpoint{0.999095in}{1.636419in}}{\pgfqpoint{0.993271in}{1.630595in}}%
\pgfpathcurveto{\pgfqpoint{0.987447in}{1.624771in}}{\pgfqpoint{0.984175in}{1.616871in}}{\pgfqpoint{0.984175in}{1.608635in}}%
\pgfpathcurveto{\pgfqpoint{0.984175in}{1.600399in}}{\pgfqpoint{0.987447in}{1.592499in}}{\pgfqpoint{0.993271in}{1.586675in}}%
\pgfpathcurveto{\pgfqpoint{0.999095in}{1.580851in}}{\pgfqpoint{1.006995in}{1.577578in}}{\pgfqpoint{1.015231in}{1.577578in}}%
\pgfpathclose%
\pgfusepath{stroke,fill}%
\end{pgfscope}%
\begin{pgfscope}%
\pgfpathrectangle{\pgfqpoint{0.100000in}{0.212622in}}{\pgfqpoint{3.696000in}{3.696000in}}%
\pgfusepath{clip}%
\pgfsetbuttcap%
\pgfsetroundjoin%
\definecolor{currentfill}{rgb}{0.121569,0.466667,0.705882}%
\pgfsetfillcolor{currentfill}%
\pgfsetfillopacity{0.776242}%
\pgfsetlinewidth{1.003750pt}%
\definecolor{currentstroke}{rgb}{0.121569,0.466667,0.705882}%
\pgfsetstrokecolor{currentstroke}%
\pgfsetstrokeopacity{0.776242}%
\pgfsetdash{}{0pt}%
\pgfpathmoveto{\pgfqpoint{1.014835in}{1.576988in}}%
\pgfpathcurveto{\pgfqpoint{1.023071in}{1.576988in}}{\pgfqpoint{1.030971in}{1.580260in}}{\pgfqpoint{1.036795in}{1.586084in}}%
\pgfpathcurveto{\pgfqpoint{1.042619in}{1.591908in}}{\pgfqpoint{1.045892in}{1.599808in}}{\pgfqpoint{1.045892in}{1.608044in}}%
\pgfpathcurveto{\pgfqpoint{1.045892in}{1.616280in}}{\pgfqpoint{1.042619in}{1.624180in}}{\pgfqpoint{1.036795in}{1.630004in}}%
\pgfpathcurveto{\pgfqpoint{1.030971in}{1.635828in}}{\pgfqpoint{1.023071in}{1.639101in}}{\pgfqpoint{1.014835in}{1.639101in}}%
\pgfpathcurveto{\pgfqpoint{1.006599in}{1.639101in}}{\pgfqpoint{0.998699in}{1.635828in}}{\pgfqpoint{0.992875in}{1.630004in}}%
\pgfpathcurveto{\pgfqpoint{0.987051in}{1.624180in}}{\pgfqpoint{0.983779in}{1.616280in}}{\pgfqpoint{0.983779in}{1.608044in}}%
\pgfpathcurveto{\pgfqpoint{0.983779in}{1.599808in}}{\pgfqpoint{0.987051in}{1.591908in}}{\pgfqpoint{0.992875in}{1.586084in}}%
\pgfpathcurveto{\pgfqpoint{0.998699in}{1.580260in}}{\pgfqpoint{1.006599in}{1.576988in}}{\pgfqpoint{1.014835in}{1.576988in}}%
\pgfpathclose%
\pgfusepath{stroke,fill}%
\end{pgfscope}%
\begin{pgfscope}%
\pgfpathrectangle{\pgfqpoint{0.100000in}{0.212622in}}{\pgfqpoint{3.696000in}{3.696000in}}%
\pgfusepath{clip}%
\pgfsetbuttcap%
\pgfsetroundjoin%
\definecolor{currentfill}{rgb}{0.121569,0.466667,0.705882}%
\pgfsetfillcolor{currentfill}%
\pgfsetfillopacity{0.776422}%
\pgfsetlinewidth{1.003750pt}%
\definecolor{currentstroke}{rgb}{0.121569,0.466667,0.705882}%
\pgfsetstrokecolor{currentstroke}%
\pgfsetstrokeopacity{0.776422}%
\pgfsetdash{}{0pt}%
\pgfpathmoveto{\pgfqpoint{1.013913in}{1.575675in}}%
\pgfpathcurveto{\pgfqpoint{1.022150in}{1.575675in}}{\pgfqpoint{1.030050in}{1.578947in}}{\pgfqpoint{1.035874in}{1.584771in}}%
\pgfpathcurveto{\pgfqpoint{1.041697in}{1.590595in}}{\pgfqpoint{1.044970in}{1.598495in}}{\pgfqpoint{1.044970in}{1.606732in}}%
\pgfpathcurveto{\pgfqpoint{1.044970in}{1.614968in}}{\pgfqpoint{1.041697in}{1.622868in}}{\pgfqpoint{1.035874in}{1.628692in}}%
\pgfpathcurveto{\pgfqpoint{1.030050in}{1.634516in}}{\pgfqpoint{1.022150in}{1.637788in}}{\pgfqpoint{1.013913in}{1.637788in}}%
\pgfpathcurveto{\pgfqpoint{1.005677in}{1.637788in}}{\pgfqpoint{0.997777in}{1.634516in}}{\pgfqpoint{0.991953in}{1.628692in}}%
\pgfpathcurveto{\pgfqpoint{0.986129in}{1.622868in}}{\pgfqpoint{0.982857in}{1.614968in}}{\pgfqpoint{0.982857in}{1.606732in}}%
\pgfpathcurveto{\pgfqpoint{0.982857in}{1.598495in}}{\pgfqpoint{0.986129in}{1.590595in}}{\pgfqpoint{0.991953in}{1.584771in}}%
\pgfpathcurveto{\pgfqpoint{0.997777in}{1.578947in}}{\pgfqpoint{1.005677in}{1.575675in}}{\pgfqpoint{1.013913in}{1.575675in}}%
\pgfpathclose%
\pgfusepath{stroke,fill}%
\end{pgfscope}%
\begin{pgfscope}%
\pgfpathrectangle{\pgfqpoint{0.100000in}{0.212622in}}{\pgfqpoint{3.696000in}{3.696000in}}%
\pgfusepath{clip}%
\pgfsetbuttcap%
\pgfsetroundjoin%
\definecolor{currentfill}{rgb}{0.121569,0.466667,0.705882}%
\pgfsetfillcolor{currentfill}%
\pgfsetfillopacity{0.776461}%
\pgfsetlinewidth{1.003750pt}%
\definecolor{currentstroke}{rgb}{0.121569,0.466667,0.705882}%
\pgfsetstrokecolor{currentstroke}%
\pgfsetstrokeopacity{0.776461}%
\pgfsetdash{}{0pt}%
\pgfpathmoveto{\pgfqpoint{2.560600in}{2.945033in}}%
\pgfpathcurveto{\pgfqpoint{2.568836in}{2.945033in}}{\pgfqpoint{2.576736in}{2.948306in}}{\pgfqpoint{2.582560in}{2.954130in}}%
\pgfpathcurveto{\pgfqpoint{2.588384in}{2.959954in}}{\pgfqpoint{2.591656in}{2.967854in}}{\pgfqpoint{2.591656in}{2.976090in}}%
\pgfpathcurveto{\pgfqpoint{2.591656in}{2.984326in}}{\pgfqpoint{2.588384in}{2.992226in}}{\pgfqpoint{2.582560in}{2.998050in}}%
\pgfpathcurveto{\pgfqpoint{2.576736in}{3.003874in}}{\pgfqpoint{2.568836in}{3.007146in}}{\pgfqpoint{2.560600in}{3.007146in}}%
\pgfpathcurveto{\pgfqpoint{2.552364in}{3.007146in}}{\pgfqpoint{2.544464in}{3.003874in}}{\pgfqpoint{2.538640in}{2.998050in}}%
\pgfpathcurveto{\pgfqpoint{2.532816in}{2.992226in}}{\pgfqpoint{2.529543in}{2.984326in}}{\pgfqpoint{2.529543in}{2.976090in}}%
\pgfpathcurveto{\pgfqpoint{2.529543in}{2.967854in}}{\pgfqpoint{2.532816in}{2.959954in}}{\pgfqpoint{2.538640in}{2.954130in}}%
\pgfpathcurveto{\pgfqpoint{2.544464in}{2.948306in}}{\pgfqpoint{2.552364in}{2.945033in}}{\pgfqpoint{2.560600in}{2.945033in}}%
\pgfpathclose%
\pgfusepath{stroke,fill}%
\end{pgfscope}%
\begin{pgfscope}%
\pgfpathrectangle{\pgfqpoint{0.100000in}{0.212622in}}{\pgfqpoint{3.696000in}{3.696000in}}%
\pgfusepath{clip}%
\pgfsetbuttcap%
\pgfsetroundjoin%
\definecolor{currentfill}{rgb}{0.121569,0.466667,0.705882}%
\pgfsetfillcolor{currentfill}%
\pgfsetfillopacity{0.776711}%
\pgfsetlinewidth{1.003750pt}%
\definecolor{currentstroke}{rgb}{0.121569,0.466667,0.705882}%
\pgfsetstrokecolor{currentstroke}%
\pgfsetstrokeopacity{0.776711}%
\pgfsetdash{}{0pt}%
\pgfpathmoveto{\pgfqpoint{1.012704in}{1.573985in}}%
\pgfpathcurveto{\pgfqpoint{1.020941in}{1.573985in}}{\pgfqpoint{1.028841in}{1.577257in}}{\pgfqpoint{1.034665in}{1.583081in}}%
\pgfpathcurveto{\pgfqpoint{1.040489in}{1.588905in}}{\pgfqpoint{1.043761in}{1.596805in}}{\pgfqpoint{1.043761in}{1.605041in}}%
\pgfpathcurveto{\pgfqpoint{1.043761in}{1.613278in}}{\pgfqpoint{1.040489in}{1.621178in}}{\pgfqpoint{1.034665in}{1.627002in}}%
\pgfpathcurveto{\pgfqpoint{1.028841in}{1.632826in}}{\pgfqpoint{1.020941in}{1.636098in}}{\pgfqpoint{1.012704in}{1.636098in}}%
\pgfpathcurveto{\pgfqpoint{1.004468in}{1.636098in}}{\pgfqpoint{0.996568in}{1.632826in}}{\pgfqpoint{0.990744in}{1.627002in}}%
\pgfpathcurveto{\pgfqpoint{0.984920in}{1.621178in}}{\pgfqpoint{0.981648in}{1.613278in}}{\pgfqpoint{0.981648in}{1.605041in}}%
\pgfpathcurveto{\pgfqpoint{0.981648in}{1.596805in}}{\pgfqpoint{0.984920in}{1.588905in}}{\pgfqpoint{0.990744in}{1.583081in}}%
\pgfpathcurveto{\pgfqpoint{0.996568in}{1.577257in}}{\pgfqpoint{1.004468in}{1.573985in}}{\pgfqpoint{1.012704in}{1.573985in}}%
\pgfpathclose%
\pgfusepath{stroke,fill}%
\end{pgfscope}%
\begin{pgfscope}%
\pgfpathrectangle{\pgfqpoint{0.100000in}{0.212622in}}{\pgfqpoint{3.696000in}{3.696000in}}%
\pgfusepath{clip}%
\pgfsetbuttcap%
\pgfsetroundjoin%
\definecolor{currentfill}{rgb}{0.121569,0.466667,0.705882}%
\pgfsetfillcolor{currentfill}%
\pgfsetfillopacity{0.776788}%
\pgfsetlinewidth{1.003750pt}%
\definecolor{currentstroke}{rgb}{0.121569,0.466667,0.705882}%
\pgfsetstrokecolor{currentstroke}%
\pgfsetstrokeopacity{0.776788}%
\pgfsetdash{}{0pt}%
\pgfpathmoveto{\pgfqpoint{2.563788in}{2.945148in}}%
\pgfpathcurveto{\pgfqpoint{2.572024in}{2.945148in}}{\pgfqpoint{2.579924in}{2.948420in}}{\pgfqpoint{2.585748in}{2.954244in}}%
\pgfpathcurveto{\pgfqpoint{2.591572in}{2.960068in}}{\pgfqpoint{2.594845in}{2.967968in}}{\pgfqpoint{2.594845in}{2.976205in}}%
\pgfpathcurveto{\pgfqpoint{2.594845in}{2.984441in}}{\pgfqpoint{2.591572in}{2.992341in}}{\pgfqpoint{2.585748in}{2.998165in}}%
\pgfpathcurveto{\pgfqpoint{2.579924in}{3.003989in}}{\pgfqpoint{2.572024in}{3.007261in}}{\pgfqpoint{2.563788in}{3.007261in}}%
\pgfpathcurveto{\pgfqpoint{2.555552in}{3.007261in}}{\pgfqpoint{2.547652in}{3.003989in}}{\pgfqpoint{2.541828in}{2.998165in}}%
\pgfpathcurveto{\pgfqpoint{2.536004in}{2.992341in}}{\pgfqpoint{2.532732in}{2.984441in}}{\pgfqpoint{2.532732in}{2.976205in}}%
\pgfpathcurveto{\pgfqpoint{2.532732in}{2.967968in}}{\pgfqpoint{2.536004in}{2.960068in}}{\pgfqpoint{2.541828in}{2.954244in}}%
\pgfpathcurveto{\pgfqpoint{2.547652in}{2.948420in}}{\pgfqpoint{2.555552in}{2.945148in}}{\pgfqpoint{2.563788in}{2.945148in}}%
\pgfpathclose%
\pgfusepath{stroke,fill}%
\end{pgfscope}%
\begin{pgfscope}%
\pgfpathrectangle{\pgfqpoint{0.100000in}{0.212622in}}{\pgfqpoint{3.696000in}{3.696000in}}%
\pgfusepath{clip}%
\pgfsetbuttcap%
\pgfsetroundjoin%
\definecolor{currentfill}{rgb}{0.121569,0.466667,0.705882}%
\pgfsetfillcolor{currentfill}%
\pgfsetfillopacity{0.777044}%
\pgfsetlinewidth{1.003750pt}%
\definecolor{currentstroke}{rgb}{0.121569,0.466667,0.705882}%
\pgfsetstrokecolor{currentstroke}%
\pgfsetstrokeopacity{0.777044}%
\pgfsetdash{}{0pt}%
\pgfpathmoveto{\pgfqpoint{1.011255in}{1.570648in}}%
\pgfpathcurveto{\pgfqpoint{1.019491in}{1.570648in}}{\pgfqpoint{1.027391in}{1.573921in}}{\pgfqpoint{1.033215in}{1.579745in}}%
\pgfpathcurveto{\pgfqpoint{1.039039in}{1.585569in}}{\pgfqpoint{1.042311in}{1.593469in}}{\pgfqpoint{1.042311in}{1.601705in}}%
\pgfpathcurveto{\pgfqpoint{1.042311in}{1.609941in}}{\pgfqpoint{1.039039in}{1.617841in}}{\pgfqpoint{1.033215in}{1.623665in}}%
\pgfpathcurveto{\pgfqpoint{1.027391in}{1.629489in}}{\pgfqpoint{1.019491in}{1.632761in}}{\pgfqpoint{1.011255in}{1.632761in}}%
\pgfpathcurveto{\pgfqpoint{1.003018in}{1.632761in}}{\pgfqpoint{0.995118in}{1.629489in}}{\pgfqpoint{0.989294in}{1.623665in}}%
\pgfpathcurveto{\pgfqpoint{0.983470in}{1.617841in}}{\pgfqpoint{0.980198in}{1.609941in}}{\pgfqpoint{0.980198in}{1.601705in}}%
\pgfpathcurveto{\pgfqpoint{0.980198in}{1.593469in}}{\pgfqpoint{0.983470in}{1.585569in}}{\pgfqpoint{0.989294in}{1.579745in}}%
\pgfpathcurveto{\pgfqpoint{0.995118in}{1.573921in}}{\pgfqpoint{1.003018in}{1.570648in}}{\pgfqpoint{1.011255in}{1.570648in}}%
\pgfpathclose%
\pgfusepath{stroke,fill}%
\end{pgfscope}%
\begin{pgfscope}%
\pgfpathrectangle{\pgfqpoint{0.100000in}{0.212622in}}{\pgfqpoint{3.696000in}{3.696000in}}%
\pgfusepath{clip}%
\pgfsetbuttcap%
\pgfsetroundjoin%
\definecolor{currentfill}{rgb}{0.121569,0.466667,0.705882}%
\pgfsetfillcolor{currentfill}%
\pgfsetfillopacity{0.777200}%
\pgfsetlinewidth{1.003750pt}%
\definecolor{currentstroke}{rgb}{0.121569,0.466667,0.705882}%
\pgfsetstrokecolor{currentstroke}%
\pgfsetstrokeopacity{0.777200}%
\pgfsetdash{}{0pt}%
\pgfpathmoveto{\pgfqpoint{2.566177in}{2.945399in}}%
\pgfpathcurveto{\pgfqpoint{2.574414in}{2.945399in}}{\pgfqpoint{2.582314in}{2.948671in}}{\pgfqpoint{2.588138in}{2.954495in}}%
\pgfpathcurveto{\pgfqpoint{2.593962in}{2.960319in}}{\pgfqpoint{2.597234in}{2.968219in}}{\pgfqpoint{2.597234in}{2.976456in}}%
\pgfpathcurveto{\pgfqpoint{2.597234in}{2.984692in}}{\pgfqpoint{2.593962in}{2.992592in}}{\pgfqpoint{2.588138in}{2.998416in}}%
\pgfpathcurveto{\pgfqpoint{2.582314in}{3.004240in}}{\pgfqpoint{2.574414in}{3.007512in}}{\pgfqpoint{2.566177in}{3.007512in}}%
\pgfpathcurveto{\pgfqpoint{2.557941in}{3.007512in}}{\pgfqpoint{2.550041in}{3.004240in}}{\pgfqpoint{2.544217in}{2.998416in}}%
\pgfpathcurveto{\pgfqpoint{2.538393in}{2.992592in}}{\pgfqpoint{2.535121in}{2.984692in}}{\pgfqpoint{2.535121in}{2.976456in}}%
\pgfpathcurveto{\pgfqpoint{2.535121in}{2.968219in}}{\pgfqpoint{2.538393in}{2.960319in}}{\pgfqpoint{2.544217in}{2.954495in}}%
\pgfpathcurveto{\pgfqpoint{2.550041in}{2.948671in}}{\pgfqpoint{2.557941in}{2.945399in}}{\pgfqpoint{2.566177in}{2.945399in}}%
\pgfpathclose%
\pgfusepath{stroke,fill}%
\end{pgfscope}%
\begin{pgfscope}%
\pgfpathrectangle{\pgfqpoint{0.100000in}{0.212622in}}{\pgfqpoint{3.696000in}{3.696000in}}%
\pgfusepath{clip}%
\pgfsetbuttcap%
\pgfsetroundjoin%
\definecolor{currentfill}{rgb}{0.121569,0.466667,0.705882}%
\pgfsetfillcolor{currentfill}%
\pgfsetfillopacity{0.777505}%
\pgfsetlinewidth{1.003750pt}%
\definecolor{currentstroke}{rgb}{0.121569,0.466667,0.705882}%
\pgfsetstrokecolor{currentstroke}%
\pgfsetstrokeopacity{0.777505}%
\pgfsetdash{}{0pt}%
\pgfpathmoveto{\pgfqpoint{1.009694in}{1.566529in}}%
\pgfpathcurveto{\pgfqpoint{1.017930in}{1.566529in}}{\pgfqpoint{1.025830in}{1.569801in}}{\pgfqpoint{1.031654in}{1.575625in}}%
\pgfpathcurveto{\pgfqpoint{1.037478in}{1.581449in}}{\pgfqpoint{1.040750in}{1.589349in}}{\pgfqpoint{1.040750in}{1.597585in}}%
\pgfpathcurveto{\pgfqpoint{1.040750in}{1.605821in}}{\pgfqpoint{1.037478in}{1.613721in}}{\pgfqpoint{1.031654in}{1.619545in}}%
\pgfpathcurveto{\pgfqpoint{1.025830in}{1.625369in}}{\pgfqpoint{1.017930in}{1.628642in}}{\pgfqpoint{1.009694in}{1.628642in}}%
\pgfpathcurveto{\pgfqpoint{1.001457in}{1.628642in}}{\pgfqpoint{0.993557in}{1.625369in}}{\pgfqpoint{0.987733in}{1.619545in}}%
\pgfpathcurveto{\pgfqpoint{0.981910in}{1.613721in}}{\pgfqpoint{0.978637in}{1.605821in}}{\pgfqpoint{0.978637in}{1.597585in}}%
\pgfpathcurveto{\pgfqpoint{0.978637in}{1.589349in}}{\pgfqpoint{0.981910in}{1.581449in}}{\pgfqpoint{0.987733in}{1.575625in}}%
\pgfpathcurveto{\pgfqpoint{0.993557in}{1.569801in}}{\pgfqpoint{1.001457in}{1.566529in}}{\pgfqpoint{1.009694in}{1.566529in}}%
\pgfpathclose%
\pgfusepath{stroke,fill}%
\end{pgfscope}%
\begin{pgfscope}%
\pgfpathrectangle{\pgfqpoint{0.100000in}{0.212622in}}{\pgfqpoint{3.696000in}{3.696000in}}%
\pgfusepath{clip}%
\pgfsetbuttcap%
\pgfsetroundjoin%
\definecolor{currentfill}{rgb}{0.121569,0.466667,0.705882}%
\pgfsetfillcolor{currentfill}%
\pgfsetfillopacity{0.777607}%
\pgfsetlinewidth{1.003750pt}%
\definecolor{currentstroke}{rgb}{0.121569,0.466667,0.705882}%
\pgfsetstrokecolor{currentstroke}%
\pgfsetstrokeopacity{0.777607}%
\pgfsetdash{}{0pt}%
\pgfpathmoveto{\pgfqpoint{2.568076in}{2.945676in}}%
\pgfpathcurveto{\pgfqpoint{2.576313in}{2.945676in}}{\pgfqpoint{2.584213in}{2.948949in}}{\pgfqpoint{2.590037in}{2.954773in}}%
\pgfpathcurveto{\pgfqpoint{2.595860in}{2.960597in}}{\pgfqpoint{2.599133in}{2.968497in}}{\pgfqpoint{2.599133in}{2.976733in}}%
\pgfpathcurveto{\pgfqpoint{2.599133in}{2.984969in}}{\pgfqpoint{2.595860in}{2.992869in}}{\pgfqpoint{2.590037in}{2.998693in}}%
\pgfpathcurveto{\pgfqpoint{2.584213in}{3.004517in}}{\pgfqpoint{2.576313in}{3.007789in}}{\pgfqpoint{2.568076in}{3.007789in}}%
\pgfpathcurveto{\pgfqpoint{2.559840in}{3.007789in}}{\pgfqpoint{2.551940in}{3.004517in}}{\pgfqpoint{2.546116in}{2.998693in}}%
\pgfpathcurveto{\pgfqpoint{2.540292in}{2.992869in}}{\pgfqpoint{2.537020in}{2.984969in}}{\pgfqpoint{2.537020in}{2.976733in}}%
\pgfpathcurveto{\pgfqpoint{2.537020in}{2.968497in}}{\pgfqpoint{2.540292in}{2.960597in}}{\pgfqpoint{2.546116in}{2.954773in}}%
\pgfpathcurveto{\pgfqpoint{2.551940in}{2.948949in}}{\pgfqpoint{2.559840in}{2.945676in}}{\pgfqpoint{2.568076in}{2.945676in}}%
\pgfpathclose%
\pgfusepath{stroke,fill}%
\end{pgfscope}%
\begin{pgfscope}%
\pgfpathrectangle{\pgfqpoint{0.100000in}{0.212622in}}{\pgfqpoint{3.696000in}{3.696000in}}%
\pgfusepath{clip}%
\pgfsetbuttcap%
\pgfsetroundjoin%
\definecolor{currentfill}{rgb}{0.121569,0.466667,0.705882}%
\pgfsetfillcolor{currentfill}%
\pgfsetfillopacity{0.777748}%
\pgfsetlinewidth{1.003750pt}%
\definecolor{currentstroke}{rgb}{0.121569,0.466667,0.705882}%
\pgfsetstrokecolor{currentstroke}%
\pgfsetstrokeopacity{0.777748}%
\pgfsetdash{}{0pt}%
\pgfpathmoveto{\pgfqpoint{2.569127in}{2.945669in}}%
\pgfpathcurveto{\pgfqpoint{2.577364in}{2.945669in}}{\pgfqpoint{2.585264in}{2.948941in}}{\pgfqpoint{2.591088in}{2.954765in}}%
\pgfpathcurveto{\pgfqpoint{2.596911in}{2.960589in}}{\pgfqpoint{2.600184in}{2.968489in}}{\pgfqpoint{2.600184in}{2.976725in}}%
\pgfpathcurveto{\pgfqpoint{2.600184in}{2.984962in}}{\pgfqpoint{2.596911in}{2.992862in}}{\pgfqpoint{2.591088in}{2.998686in}}%
\pgfpathcurveto{\pgfqpoint{2.585264in}{3.004510in}}{\pgfqpoint{2.577364in}{3.007782in}}{\pgfqpoint{2.569127in}{3.007782in}}%
\pgfpathcurveto{\pgfqpoint{2.560891in}{3.007782in}}{\pgfqpoint{2.552991in}{3.004510in}}{\pgfqpoint{2.547167in}{2.998686in}}%
\pgfpathcurveto{\pgfqpoint{2.541343in}{2.992862in}}{\pgfqpoint{2.538071in}{2.984962in}}{\pgfqpoint{2.538071in}{2.976725in}}%
\pgfpathcurveto{\pgfqpoint{2.538071in}{2.968489in}}{\pgfqpoint{2.541343in}{2.960589in}}{\pgfqpoint{2.547167in}{2.954765in}}%
\pgfpathcurveto{\pgfqpoint{2.552991in}{2.948941in}}{\pgfqpoint{2.560891in}{2.945669in}}{\pgfqpoint{2.569127in}{2.945669in}}%
\pgfpathclose%
\pgfusepath{stroke,fill}%
\end{pgfscope}%
\begin{pgfscope}%
\pgfpathrectangle{\pgfqpoint{0.100000in}{0.212622in}}{\pgfqpoint{3.696000in}{3.696000in}}%
\pgfusepath{clip}%
\pgfsetbuttcap%
\pgfsetroundjoin%
\definecolor{currentfill}{rgb}{0.121569,0.466667,0.705882}%
\pgfsetfillcolor{currentfill}%
\pgfsetfillopacity{0.778023}%
\pgfsetlinewidth{1.003750pt}%
\definecolor{currentstroke}{rgb}{0.121569,0.466667,0.705882}%
\pgfsetstrokecolor{currentstroke}%
\pgfsetstrokeopacity{0.778023}%
\pgfsetdash{}{0pt}%
\pgfpathmoveto{\pgfqpoint{1.008371in}{1.561261in}}%
\pgfpathcurveto{\pgfqpoint{1.016607in}{1.561261in}}{\pgfqpoint{1.024507in}{1.564533in}}{\pgfqpoint{1.030331in}{1.570357in}}%
\pgfpathcurveto{\pgfqpoint{1.036155in}{1.576181in}}{\pgfqpoint{1.039428in}{1.584081in}}{\pgfqpoint{1.039428in}{1.592317in}}%
\pgfpathcurveto{\pgfqpoint{1.039428in}{1.600553in}}{\pgfqpoint{1.036155in}{1.608453in}}{\pgfqpoint{1.030331in}{1.614277in}}%
\pgfpathcurveto{\pgfqpoint{1.024507in}{1.620101in}}{\pgfqpoint{1.016607in}{1.623374in}}{\pgfqpoint{1.008371in}{1.623374in}}%
\pgfpathcurveto{\pgfqpoint{1.000135in}{1.623374in}}{\pgfqpoint{0.992235in}{1.620101in}}{\pgfqpoint{0.986411in}{1.614277in}}%
\pgfpathcurveto{\pgfqpoint{0.980587in}{1.608453in}}{\pgfqpoint{0.977315in}{1.600553in}}{\pgfqpoint{0.977315in}{1.592317in}}%
\pgfpathcurveto{\pgfqpoint{0.977315in}{1.584081in}}{\pgfqpoint{0.980587in}{1.576181in}}{\pgfqpoint{0.986411in}{1.570357in}}%
\pgfpathcurveto{\pgfqpoint{0.992235in}{1.564533in}}{\pgfqpoint{1.000135in}{1.561261in}}{\pgfqpoint{1.008371in}{1.561261in}}%
\pgfpathclose%
\pgfusepath{stroke,fill}%
\end{pgfscope}%
\begin{pgfscope}%
\pgfpathrectangle{\pgfqpoint{0.100000in}{0.212622in}}{\pgfqpoint{3.696000in}{3.696000in}}%
\pgfusepath{clip}%
\pgfsetbuttcap%
\pgfsetroundjoin%
\definecolor{currentfill}{rgb}{0.121569,0.466667,0.705882}%
\pgfsetfillcolor{currentfill}%
\pgfsetfillopacity{0.778056}%
\pgfsetlinewidth{1.003750pt}%
\definecolor{currentstroke}{rgb}{0.121569,0.466667,0.705882}%
\pgfsetstrokecolor{currentstroke}%
\pgfsetstrokeopacity{0.778056}%
\pgfsetdash{}{0pt}%
\pgfpathmoveto{\pgfqpoint{2.571040in}{2.945884in}}%
\pgfpathcurveto{\pgfqpoint{2.579276in}{2.945884in}}{\pgfqpoint{2.587176in}{2.949156in}}{\pgfqpoint{2.593000in}{2.954980in}}%
\pgfpathcurveto{\pgfqpoint{2.598824in}{2.960804in}}{\pgfqpoint{2.602096in}{2.968704in}}{\pgfqpoint{2.602096in}{2.976941in}}%
\pgfpathcurveto{\pgfqpoint{2.602096in}{2.985177in}}{\pgfqpoint{2.598824in}{2.993077in}}{\pgfqpoint{2.593000in}{2.998901in}}%
\pgfpathcurveto{\pgfqpoint{2.587176in}{3.004725in}}{\pgfqpoint{2.579276in}{3.007997in}}{\pgfqpoint{2.571040in}{3.007997in}}%
\pgfpathcurveto{\pgfqpoint{2.562803in}{3.007997in}}{\pgfqpoint{2.554903in}{3.004725in}}{\pgfqpoint{2.549079in}{2.998901in}}%
\pgfpathcurveto{\pgfqpoint{2.543255in}{2.993077in}}{\pgfqpoint{2.539983in}{2.985177in}}{\pgfqpoint{2.539983in}{2.976941in}}%
\pgfpathcurveto{\pgfqpoint{2.539983in}{2.968704in}}{\pgfqpoint{2.543255in}{2.960804in}}{\pgfqpoint{2.549079in}{2.954980in}}%
\pgfpathcurveto{\pgfqpoint{2.554903in}{2.949156in}}{\pgfqpoint{2.562803in}{2.945884in}}{\pgfqpoint{2.571040in}{2.945884in}}%
\pgfpathclose%
\pgfusepath{stroke,fill}%
\end{pgfscope}%
\begin{pgfscope}%
\pgfpathrectangle{\pgfqpoint{0.100000in}{0.212622in}}{\pgfqpoint{3.696000in}{3.696000in}}%
\pgfusepath{clip}%
\pgfsetbuttcap%
\pgfsetroundjoin%
\definecolor{currentfill}{rgb}{0.121569,0.466667,0.705882}%
\pgfsetfillcolor{currentfill}%
\pgfsetfillopacity{0.778120}%
\pgfsetlinewidth{1.003750pt}%
\definecolor{currentstroke}{rgb}{0.121569,0.466667,0.705882}%
\pgfsetstrokecolor{currentstroke}%
\pgfsetstrokeopacity{0.778120}%
\pgfsetdash{}{0pt}%
\pgfpathmoveto{\pgfqpoint{2.571350in}{2.945934in}}%
\pgfpathcurveto{\pgfqpoint{2.579586in}{2.945934in}}{\pgfqpoint{2.587486in}{2.949207in}}{\pgfqpoint{2.593310in}{2.955031in}}%
\pgfpathcurveto{\pgfqpoint{2.599134in}{2.960855in}}{\pgfqpoint{2.602406in}{2.968755in}}{\pgfqpoint{2.602406in}{2.976991in}}%
\pgfpathcurveto{\pgfqpoint{2.602406in}{2.985227in}}{\pgfqpoint{2.599134in}{2.993127in}}{\pgfqpoint{2.593310in}{2.998951in}}%
\pgfpathcurveto{\pgfqpoint{2.587486in}{3.004775in}}{\pgfqpoint{2.579586in}{3.008047in}}{\pgfqpoint{2.571350in}{3.008047in}}%
\pgfpathcurveto{\pgfqpoint{2.563113in}{3.008047in}}{\pgfqpoint{2.555213in}{3.004775in}}{\pgfqpoint{2.549389in}{2.998951in}}%
\pgfpathcurveto{\pgfqpoint{2.543566in}{2.993127in}}{\pgfqpoint{2.540293in}{2.985227in}}{\pgfqpoint{2.540293in}{2.976991in}}%
\pgfpathcurveto{\pgfqpoint{2.540293in}{2.968755in}}{\pgfqpoint{2.543566in}{2.960855in}}{\pgfqpoint{2.549389in}{2.955031in}}%
\pgfpathcurveto{\pgfqpoint{2.555213in}{2.949207in}}{\pgfqpoint{2.563113in}{2.945934in}}{\pgfqpoint{2.571350in}{2.945934in}}%
\pgfpathclose%
\pgfusepath{stroke,fill}%
\end{pgfscope}%
\begin{pgfscope}%
\pgfpathrectangle{\pgfqpoint{0.100000in}{0.212622in}}{\pgfqpoint{3.696000in}{3.696000in}}%
\pgfusepath{clip}%
\pgfsetbuttcap%
\pgfsetroundjoin%
\definecolor{currentfill}{rgb}{0.121569,0.466667,0.705882}%
\pgfsetfillcolor{currentfill}%
\pgfsetfillopacity{0.778120}%
\pgfsetlinewidth{1.003750pt}%
\definecolor{currentstroke}{rgb}{0.121569,0.466667,0.705882}%
\pgfsetstrokecolor{currentstroke}%
\pgfsetstrokeopacity{0.778120}%
\pgfsetdash{}{0pt}%
\pgfpathmoveto{\pgfqpoint{2.571350in}{2.945934in}}%
\pgfpathcurveto{\pgfqpoint{2.579586in}{2.945934in}}{\pgfqpoint{2.587486in}{2.949207in}}{\pgfqpoint{2.593310in}{2.955031in}}%
\pgfpathcurveto{\pgfqpoint{2.599134in}{2.960855in}}{\pgfqpoint{2.602406in}{2.968755in}}{\pgfqpoint{2.602406in}{2.976991in}}%
\pgfpathcurveto{\pgfqpoint{2.602406in}{2.985227in}}{\pgfqpoint{2.599134in}{2.993127in}}{\pgfqpoint{2.593310in}{2.998951in}}%
\pgfpathcurveto{\pgfqpoint{2.587486in}{3.004775in}}{\pgfqpoint{2.579586in}{3.008047in}}{\pgfqpoint{2.571350in}{3.008047in}}%
\pgfpathcurveto{\pgfqpoint{2.563114in}{3.008047in}}{\pgfqpoint{2.555213in}{3.004775in}}{\pgfqpoint{2.549390in}{2.998951in}}%
\pgfpathcurveto{\pgfqpoint{2.543566in}{2.993127in}}{\pgfqpoint{2.540293in}{2.985227in}}{\pgfqpoint{2.540293in}{2.976991in}}%
\pgfpathcurveto{\pgfqpoint{2.540293in}{2.968755in}}{\pgfqpoint{2.543566in}{2.960855in}}{\pgfqpoint{2.549390in}{2.955031in}}%
\pgfpathcurveto{\pgfqpoint{2.555213in}{2.949207in}}{\pgfqpoint{2.563114in}{2.945934in}}{\pgfqpoint{2.571350in}{2.945934in}}%
\pgfpathclose%
\pgfusepath{stroke,fill}%
\end{pgfscope}%
\begin{pgfscope}%
\pgfpathrectangle{\pgfqpoint{0.100000in}{0.212622in}}{\pgfqpoint{3.696000in}{3.696000in}}%
\pgfusepath{clip}%
\pgfsetbuttcap%
\pgfsetroundjoin%
\definecolor{currentfill}{rgb}{0.121569,0.466667,0.705882}%
\pgfsetfillcolor{currentfill}%
\pgfsetfillopacity{0.778120}%
\pgfsetlinewidth{1.003750pt}%
\definecolor{currentstroke}{rgb}{0.121569,0.466667,0.705882}%
\pgfsetstrokecolor{currentstroke}%
\pgfsetstrokeopacity{0.778120}%
\pgfsetdash{}{0pt}%
\pgfpathmoveto{\pgfqpoint{2.571350in}{2.945934in}}%
\pgfpathcurveto{\pgfqpoint{2.579586in}{2.945934in}}{\pgfqpoint{2.587486in}{2.949207in}}{\pgfqpoint{2.593310in}{2.955031in}}%
\pgfpathcurveto{\pgfqpoint{2.599134in}{2.960855in}}{\pgfqpoint{2.602407in}{2.968755in}}{\pgfqpoint{2.602407in}{2.976991in}}%
\pgfpathcurveto{\pgfqpoint{2.602407in}{2.985227in}}{\pgfqpoint{2.599134in}{2.993127in}}{\pgfqpoint{2.593310in}{2.998951in}}%
\pgfpathcurveto{\pgfqpoint{2.587486in}{3.004775in}}{\pgfqpoint{2.579586in}{3.008047in}}{\pgfqpoint{2.571350in}{3.008047in}}%
\pgfpathcurveto{\pgfqpoint{2.563114in}{3.008047in}}{\pgfqpoint{2.555214in}{3.004775in}}{\pgfqpoint{2.549390in}{2.998951in}}%
\pgfpathcurveto{\pgfqpoint{2.543566in}{2.993127in}}{\pgfqpoint{2.540294in}{2.985227in}}{\pgfqpoint{2.540294in}{2.976991in}}%
\pgfpathcurveto{\pgfqpoint{2.540294in}{2.968755in}}{\pgfqpoint{2.543566in}{2.960855in}}{\pgfqpoint{2.549390in}{2.955031in}}%
\pgfpathcurveto{\pgfqpoint{2.555214in}{2.949207in}}{\pgfqpoint{2.563114in}{2.945934in}}{\pgfqpoint{2.571350in}{2.945934in}}%
\pgfpathclose%
\pgfusepath{stroke,fill}%
\end{pgfscope}%
\begin{pgfscope}%
\pgfpathrectangle{\pgfqpoint{0.100000in}{0.212622in}}{\pgfqpoint{3.696000in}{3.696000in}}%
\pgfusepath{clip}%
\pgfsetbuttcap%
\pgfsetroundjoin%
\definecolor{currentfill}{rgb}{0.121569,0.466667,0.705882}%
\pgfsetfillcolor{currentfill}%
\pgfsetfillopacity{0.778120}%
\pgfsetlinewidth{1.003750pt}%
\definecolor{currentstroke}{rgb}{0.121569,0.466667,0.705882}%
\pgfsetstrokecolor{currentstroke}%
\pgfsetstrokeopacity{0.778120}%
\pgfsetdash{}{0pt}%
\pgfpathmoveto{\pgfqpoint{2.571350in}{2.945934in}}%
\pgfpathcurveto{\pgfqpoint{2.579587in}{2.945934in}}{\pgfqpoint{2.587487in}{2.949207in}}{\pgfqpoint{2.593311in}{2.955031in}}%
\pgfpathcurveto{\pgfqpoint{2.599135in}{2.960855in}}{\pgfqpoint{2.602407in}{2.968755in}}{\pgfqpoint{2.602407in}{2.976991in}}%
\pgfpathcurveto{\pgfqpoint{2.602407in}{2.985227in}}{\pgfqpoint{2.599135in}{2.993127in}}{\pgfqpoint{2.593311in}{2.998951in}}%
\pgfpathcurveto{\pgfqpoint{2.587487in}{3.004775in}}{\pgfqpoint{2.579587in}{3.008047in}}{\pgfqpoint{2.571350in}{3.008047in}}%
\pgfpathcurveto{\pgfqpoint{2.563114in}{3.008047in}}{\pgfqpoint{2.555214in}{3.004775in}}{\pgfqpoint{2.549390in}{2.998951in}}%
\pgfpathcurveto{\pgfqpoint{2.543566in}{2.993127in}}{\pgfqpoint{2.540294in}{2.985227in}}{\pgfqpoint{2.540294in}{2.976991in}}%
\pgfpathcurveto{\pgfqpoint{2.540294in}{2.968755in}}{\pgfqpoint{2.543566in}{2.960855in}}{\pgfqpoint{2.549390in}{2.955031in}}%
\pgfpathcurveto{\pgfqpoint{2.555214in}{2.949207in}}{\pgfqpoint{2.563114in}{2.945934in}}{\pgfqpoint{2.571350in}{2.945934in}}%
\pgfpathclose%
\pgfusepath{stroke,fill}%
\end{pgfscope}%
\begin{pgfscope}%
\pgfpathrectangle{\pgfqpoint{0.100000in}{0.212622in}}{\pgfqpoint{3.696000in}{3.696000in}}%
\pgfusepath{clip}%
\pgfsetbuttcap%
\pgfsetroundjoin%
\definecolor{currentfill}{rgb}{0.121569,0.466667,0.705882}%
\pgfsetfillcolor{currentfill}%
\pgfsetfillopacity{0.778120}%
\pgfsetlinewidth{1.003750pt}%
\definecolor{currentstroke}{rgb}{0.121569,0.466667,0.705882}%
\pgfsetstrokecolor{currentstroke}%
\pgfsetstrokeopacity{0.778120}%
\pgfsetdash{}{0pt}%
\pgfpathmoveto{\pgfqpoint{2.571351in}{2.945935in}}%
\pgfpathcurveto{\pgfqpoint{2.579587in}{2.945935in}}{\pgfqpoint{2.587487in}{2.949207in}}{\pgfqpoint{2.593311in}{2.955031in}}%
\pgfpathcurveto{\pgfqpoint{2.599135in}{2.960855in}}{\pgfqpoint{2.602408in}{2.968755in}}{\pgfqpoint{2.602408in}{2.976991in}}%
\pgfpathcurveto{\pgfqpoint{2.602408in}{2.985227in}}{\pgfqpoint{2.599135in}{2.993127in}}{\pgfqpoint{2.593311in}{2.998951in}}%
\pgfpathcurveto{\pgfqpoint{2.587487in}{3.004775in}}{\pgfqpoint{2.579587in}{3.008048in}}{\pgfqpoint{2.571351in}{3.008048in}}%
\pgfpathcurveto{\pgfqpoint{2.563115in}{3.008048in}}{\pgfqpoint{2.555215in}{3.004775in}}{\pgfqpoint{2.549391in}{2.998951in}}%
\pgfpathcurveto{\pgfqpoint{2.543567in}{2.993127in}}{\pgfqpoint{2.540295in}{2.985227in}}{\pgfqpoint{2.540295in}{2.976991in}}%
\pgfpathcurveto{\pgfqpoint{2.540295in}{2.968755in}}{\pgfqpoint{2.543567in}{2.960855in}}{\pgfqpoint{2.549391in}{2.955031in}}%
\pgfpathcurveto{\pgfqpoint{2.555215in}{2.949207in}}{\pgfqpoint{2.563115in}{2.945935in}}{\pgfqpoint{2.571351in}{2.945935in}}%
\pgfpathclose%
\pgfusepath{stroke,fill}%
\end{pgfscope}%
\begin{pgfscope}%
\pgfpathrectangle{\pgfqpoint{0.100000in}{0.212622in}}{\pgfqpoint{3.696000in}{3.696000in}}%
\pgfusepath{clip}%
\pgfsetbuttcap%
\pgfsetroundjoin%
\definecolor{currentfill}{rgb}{0.121569,0.466667,0.705882}%
\pgfsetfillcolor{currentfill}%
\pgfsetfillopacity{0.778120}%
\pgfsetlinewidth{1.003750pt}%
\definecolor{currentstroke}{rgb}{0.121569,0.466667,0.705882}%
\pgfsetstrokecolor{currentstroke}%
\pgfsetstrokeopacity{0.778120}%
\pgfsetdash{}{0pt}%
\pgfpathmoveto{\pgfqpoint{2.571352in}{2.945935in}}%
\pgfpathcurveto{\pgfqpoint{2.579589in}{2.945935in}}{\pgfqpoint{2.587489in}{2.949207in}}{\pgfqpoint{2.593313in}{2.955031in}}%
\pgfpathcurveto{\pgfqpoint{2.599136in}{2.960855in}}{\pgfqpoint{2.602409in}{2.968755in}}{\pgfqpoint{2.602409in}{2.976991in}}%
\pgfpathcurveto{\pgfqpoint{2.602409in}{2.985227in}}{\pgfqpoint{2.599136in}{2.993127in}}{\pgfqpoint{2.593313in}{2.998951in}}%
\pgfpathcurveto{\pgfqpoint{2.587489in}{3.004775in}}{\pgfqpoint{2.579589in}{3.008048in}}{\pgfqpoint{2.571352in}{3.008048in}}%
\pgfpathcurveto{\pgfqpoint{2.563116in}{3.008048in}}{\pgfqpoint{2.555216in}{3.004775in}}{\pgfqpoint{2.549392in}{2.998951in}}%
\pgfpathcurveto{\pgfqpoint{2.543568in}{2.993127in}}{\pgfqpoint{2.540296in}{2.985227in}}{\pgfqpoint{2.540296in}{2.976991in}}%
\pgfpathcurveto{\pgfqpoint{2.540296in}{2.968755in}}{\pgfqpoint{2.543568in}{2.960855in}}{\pgfqpoint{2.549392in}{2.955031in}}%
\pgfpathcurveto{\pgfqpoint{2.555216in}{2.949207in}}{\pgfqpoint{2.563116in}{2.945935in}}{\pgfqpoint{2.571352in}{2.945935in}}%
\pgfpathclose%
\pgfusepath{stroke,fill}%
\end{pgfscope}%
\begin{pgfscope}%
\pgfpathrectangle{\pgfqpoint{0.100000in}{0.212622in}}{\pgfqpoint{3.696000in}{3.696000in}}%
\pgfusepath{clip}%
\pgfsetbuttcap%
\pgfsetroundjoin%
\definecolor{currentfill}{rgb}{0.121569,0.466667,0.705882}%
\pgfsetfillcolor{currentfill}%
\pgfsetfillopacity{0.778120}%
\pgfsetlinewidth{1.003750pt}%
\definecolor{currentstroke}{rgb}{0.121569,0.466667,0.705882}%
\pgfsetstrokecolor{currentstroke}%
\pgfsetstrokeopacity{0.778120}%
\pgfsetdash{}{0pt}%
\pgfpathmoveto{\pgfqpoint{2.571355in}{2.945935in}}%
\pgfpathcurveto{\pgfqpoint{2.579591in}{2.945935in}}{\pgfqpoint{2.587491in}{2.949207in}}{\pgfqpoint{2.593315in}{2.955031in}}%
\pgfpathcurveto{\pgfqpoint{2.599139in}{2.960855in}}{\pgfqpoint{2.602411in}{2.968755in}}{\pgfqpoint{2.602411in}{2.976991in}}%
\pgfpathcurveto{\pgfqpoint{2.602411in}{2.985228in}}{\pgfqpoint{2.599139in}{2.993128in}}{\pgfqpoint{2.593315in}{2.998952in}}%
\pgfpathcurveto{\pgfqpoint{2.587491in}{3.004776in}}{\pgfqpoint{2.579591in}{3.008048in}}{\pgfqpoint{2.571355in}{3.008048in}}%
\pgfpathcurveto{\pgfqpoint{2.563118in}{3.008048in}}{\pgfqpoint{2.555218in}{3.004776in}}{\pgfqpoint{2.549394in}{2.998952in}}%
\pgfpathcurveto{\pgfqpoint{2.543570in}{2.993128in}}{\pgfqpoint{2.540298in}{2.985228in}}{\pgfqpoint{2.540298in}{2.976991in}}%
\pgfpathcurveto{\pgfqpoint{2.540298in}{2.968755in}}{\pgfqpoint{2.543570in}{2.960855in}}{\pgfqpoint{2.549394in}{2.955031in}}%
\pgfpathcurveto{\pgfqpoint{2.555218in}{2.949207in}}{\pgfqpoint{2.563118in}{2.945935in}}{\pgfqpoint{2.571355in}{2.945935in}}%
\pgfpathclose%
\pgfusepath{stroke,fill}%
\end{pgfscope}%
\begin{pgfscope}%
\pgfpathrectangle{\pgfqpoint{0.100000in}{0.212622in}}{\pgfqpoint{3.696000in}{3.696000in}}%
\pgfusepath{clip}%
\pgfsetbuttcap%
\pgfsetroundjoin%
\definecolor{currentfill}{rgb}{0.121569,0.466667,0.705882}%
\pgfsetfillcolor{currentfill}%
\pgfsetfillopacity{0.778121}%
\pgfsetlinewidth{1.003750pt}%
\definecolor{currentstroke}{rgb}{0.121569,0.466667,0.705882}%
\pgfsetstrokecolor{currentstroke}%
\pgfsetstrokeopacity{0.778121}%
\pgfsetdash{}{0pt}%
\pgfpathmoveto{\pgfqpoint{2.571359in}{2.945935in}}%
\pgfpathcurveto{\pgfqpoint{2.579595in}{2.945935in}}{\pgfqpoint{2.587495in}{2.949208in}}{\pgfqpoint{2.593319in}{2.955031in}}%
\pgfpathcurveto{\pgfqpoint{2.599143in}{2.960855in}}{\pgfqpoint{2.602415in}{2.968755in}}{\pgfqpoint{2.602415in}{2.976992in}}%
\pgfpathcurveto{\pgfqpoint{2.602415in}{2.985228in}}{\pgfqpoint{2.599143in}{2.993128in}}{\pgfqpoint{2.593319in}{2.998952in}}%
\pgfpathcurveto{\pgfqpoint{2.587495in}{3.004776in}}{\pgfqpoint{2.579595in}{3.008048in}}{\pgfqpoint{2.571359in}{3.008048in}}%
\pgfpathcurveto{\pgfqpoint{2.563122in}{3.008048in}}{\pgfqpoint{2.555222in}{3.004776in}}{\pgfqpoint{2.549398in}{2.998952in}}%
\pgfpathcurveto{\pgfqpoint{2.543574in}{2.993128in}}{\pgfqpoint{2.540302in}{2.985228in}}{\pgfqpoint{2.540302in}{2.976992in}}%
\pgfpathcurveto{\pgfqpoint{2.540302in}{2.968755in}}{\pgfqpoint{2.543574in}{2.960855in}}{\pgfqpoint{2.549398in}{2.955031in}}%
\pgfpathcurveto{\pgfqpoint{2.555222in}{2.949208in}}{\pgfqpoint{2.563122in}{2.945935in}}{\pgfqpoint{2.571359in}{2.945935in}}%
\pgfpathclose%
\pgfusepath{stroke,fill}%
\end{pgfscope}%
\begin{pgfscope}%
\pgfpathrectangle{\pgfqpoint{0.100000in}{0.212622in}}{\pgfqpoint{3.696000in}{3.696000in}}%
\pgfusepath{clip}%
\pgfsetbuttcap%
\pgfsetroundjoin%
\definecolor{currentfill}{rgb}{0.121569,0.466667,0.705882}%
\pgfsetfillcolor{currentfill}%
\pgfsetfillopacity{0.778122}%
\pgfsetlinewidth{1.003750pt}%
\definecolor{currentstroke}{rgb}{0.121569,0.466667,0.705882}%
\pgfsetstrokecolor{currentstroke}%
\pgfsetstrokeopacity{0.778122}%
\pgfsetdash{}{0pt}%
\pgfpathmoveto{\pgfqpoint{2.571366in}{2.945935in}}%
\pgfpathcurveto{\pgfqpoint{2.579602in}{2.945935in}}{\pgfqpoint{2.587502in}{2.949208in}}{\pgfqpoint{2.593326in}{2.955032in}}%
\pgfpathcurveto{\pgfqpoint{2.599150in}{2.960856in}}{\pgfqpoint{2.602422in}{2.968756in}}{\pgfqpoint{2.602422in}{2.976992in}}%
\pgfpathcurveto{\pgfqpoint{2.602422in}{2.985228in}}{\pgfqpoint{2.599150in}{2.993128in}}{\pgfqpoint{2.593326in}{2.998952in}}%
\pgfpathcurveto{\pgfqpoint{2.587502in}{3.004776in}}{\pgfqpoint{2.579602in}{3.008048in}}{\pgfqpoint{2.571366in}{3.008048in}}%
\pgfpathcurveto{\pgfqpoint{2.563129in}{3.008048in}}{\pgfqpoint{2.555229in}{3.004776in}}{\pgfqpoint{2.549406in}{2.998952in}}%
\pgfpathcurveto{\pgfqpoint{2.543582in}{2.993128in}}{\pgfqpoint{2.540309in}{2.985228in}}{\pgfqpoint{2.540309in}{2.976992in}}%
\pgfpathcurveto{\pgfqpoint{2.540309in}{2.968756in}}{\pgfqpoint{2.543582in}{2.960856in}}{\pgfqpoint{2.549406in}{2.955032in}}%
\pgfpathcurveto{\pgfqpoint{2.555229in}{2.949208in}}{\pgfqpoint{2.563129in}{2.945935in}}{\pgfqpoint{2.571366in}{2.945935in}}%
\pgfpathclose%
\pgfusepath{stroke,fill}%
\end{pgfscope}%
\begin{pgfscope}%
\pgfpathrectangle{\pgfqpoint{0.100000in}{0.212622in}}{\pgfqpoint{3.696000in}{3.696000in}}%
\pgfusepath{clip}%
\pgfsetbuttcap%
\pgfsetroundjoin%
\definecolor{currentfill}{rgb}{0.121569,0.466667,0.705882}%
\pgfsetfillcolor{currentfill}%
\pgfsetfillopacity{0.778125}%
\pgfsetlinewidth{1.003750pt}%
\definecolor{currentstroke}{rgb}{0.121569,0.466667,0.705882}%
\pgfsetstrokecolor{currentstroke}%
\pgfsetstrokeopacity{0.778125}%
\pgfsetdash{}{0pt}%
\pgfpathmoveto{\pgfqpoint{2.571378in}{2.945935in}}%
\pgfpathcurveto{\pgfqpoint{2.579615in}{2.945935in}}{\pgfqpoint{2.587515in}{2.949207in}}{\pgfqpoint{2.593339in}{2.955031in}}%
\pgfpathcurveto{\pgfqpoint{2.599162in}{2.960855in}}{\pgfqpoint{2.602435in}{2.968755in}}{\pgfqpoint{2.602435in}{2.976991in}}%
\pgfpathcurveto{\pgfqpoint{2.602435in}{2.985228in}}{\pgfqpoint{2.599162in}{2.993128in}}{\pgfqpoint{2.593339in}{2.998952in}}%
\pgfpathcurveto{\pgfqpoint{2.587515in}{3.004776in}}{\pgfqpoint{2.579615in}{3.008048in}}{\pgfqpoint{2.571378in}{3.008048in}}%
\pgfpathcurveto{\pgfqpoint{2.563142in}{3.008048in}}{\pgfqpoint{2.555242in}{3.004776in}}{\pgfqpoint{2.549418in}{2.998952in}}%
\pgfpathcurveto{\pgfqpoint{2.543594in}{2.993128in}}{\pgfqpoint{2.540322in}{2.985228in}}{\pgfqpoint{2.540322in}{2.976991in}}%
\pgfpathcurveto{\pgfqpoint{2.540322in}{2.968755in}}{\pgfqpoint{2.543594in}{2.960855in}}{\pgfqpoint{2.549418in}{2.955031in}}%
\pgfpathcurveto{\pgfqpoint{2.555242in}{2.949207in}}{\pgfqpoint{2.563142in}{2.945935in}}{\pgfqpoint{2.571378in}{2.945935in}}%
\pgfpathclose%
\pgfusepath{stroke,fill}%
\end{pgfscope}%
\begin{pgfscope}%
\pgfpathrectangle{\pgfqpoint{0.100000in}{0.212622in}}{\pgfqpoint{3.696000in}{3.696000in}}%
\pgfusepath{clip}%
\pgfsetbuttcap%
\pgfsetroundjoin%
\definecolor{currentfill}{rgb}{0.121569,0.466667,0.705882}%
\pgfsetfillcolor{currentfill}%
\pgfsetfillopacity{0.778130}%
\pgfsetlinewidth{1.003750pt}%
\definecolor{currentstroke}{rgb}{0.121569,0.466667,0.705882}%
\pgfsetstrokecolor{currentstroke}%
\pgfsetstrokeopacity{0.778130}%
\pgfsetdash{}{0pt}%
\pgfpathmoveto{\pgfqpoint{2.571402in}{2.945936in}}%
\pgfpathcurveto{\pgfqpoint{2.579638in}{2.945936in}}{\pgfqpoint{2.587538in}{2.949208in}}{\pgfqpoint{2.593362in}{2.955032in}}%
\pgfpathcurveto{\pgfqpoint{2.599186in}{2.960856in}}{\pgfqpoint{2.602458in}{2.968756in}}{\pgfqpoint{2.602458in}{2.976992in}}%
\pgfpathcurveto{\pgfqpoint{2.602458in}{2.985228in}}{\pgfqpoint{2.599186in}{2.993128in}}{\pgfqpoint{2.593362in}{2.998952in}}%
\pgfpathcurveto{\pgfqpoint{2.587538in}{3.004776in}}{\pgfqpoint{2.579638in}{3.008049in}}{\pgfqpoint{2.571402in}{3.008049in}}%
\pgfpathcurveto{\pgfqpoint{2.563166in}{3.008049in}}{\pgfqpoint{2.555265in}{3.004776in}}{\pgfqpoint{2.549442in}{2.998952in}}%
\pgfpathcurveto{\pgfqpoint{2.543618in}{2.993128in}}{\pgfqpoint{2.540345in}{2.985228in}}{\pgfqpoint{2.540345in}{2.976992in}}%
\pgfpathcurveto{\pgfqpoint{2.540345in}{2.968756in}}{\pgfqpoint{2.543618in}{2.960856in}}{\pgfqpoint{2.549442in}{2.955032in}}%
\pgfpathcurveto{\pgfqpoint{2.555265in}{2.949208in}}{\pgfqpoint{2.563166in}{2.945936in}}{\pgfqpoint{2.571402in}{2.945936in}}%
\pgfpathclose%
\pgfusepath{stroke,fill}%
\end{pgfscope}%
\begin{pgfscope}%
\pgfpathrectangle{\pgfqpoint{0.100000in}{0.212622in}}{\pgfqpoint{3.696000in}{3.696000in}}%
\pgfusepath{clip}%
\pgfsetbuttcap%
\pgfsetroundjoin%
\definecolor{currentfill}{rgb}{0.121569,0.466667,0.705882}%
\pgfsetfillcolor{currentfill}%
\pgfsetfillopacity{0.778139}%
\pgfsetlinewidth{1.003750pt}%
\definecolor{currentstroke}{rgb}{0.121569,0.466667,0.705882}%
\pgfsetstrokecolor{currentstroke}%
\pgfsetstrokeopacity{0.778139}%
\pgfsetdash{}{0pt}%
\pgfpathmoveto{\pgfqpoint{2.571443in}{2.945936in}}%
\pgfpathcurveto{\pgfqpoint{2.579679in}{2.945936in}}{\pgfqpoint{2.587580in}{2.949208in}}{\pgfqpoint{2.593403in}{2.955032in}}%
\pgfpathcurveto{\pgfqpoint{2.599227in}{2.960856in}}{\pgfqpoint{2.602500in}{2.968756in}}{\pgfqpoint{2.602500in}{2.976992in}}%
\pgfpathcurveto{\pgfqpoint{2.602500in}{2.985228in}}{\pgfqpoint{2.599227in}{2.993128in}}{\pgfqpoint{2.593403in}{2.998952in}}%
\pgfpathcurveto{\pgfqpoint{2.587580in}{3.004776in}}{\pgfqpoint{2.579679in}{3.008049in}}{\pgfqpoint{2.571443in}{3.008049in}}%
\pgfpathcurveto{\pgfqpoint{2.563207in}{3.008049in}}{\pgfqpoint{2.555307in}{3.004776in}}{\pgfqpoint{2.549483in}{2.998952in}}%
\pgfpathcurveto{\pgfqpoint{2.543659in}{2.993128in}}{\pgfqpoint{2.540387in}{2.985228in}}{\pgfqpoint{2.540387in}{2.976992in}}%
\pgfpathcurveto{\pgfqpoint{2.540387in}{2.968756in}}{\pgfqpoint{2.543659in}{2.960856in}}{\pgfqpoint{2.549483in}{2.955032in}}%
\pgfpathcurveto{\pgfqpoint{2.555307in}{2.949208in}}{\pgfqpoint{2.563207in}{2.945936in}}{\pgfqpoint{2.571443in}{2.945936in}}%
\pgfpathclose%
\pgfusepath{stroke,fill}%
\end{pgfscope}%
\begin{pgfscope}%
\pgfpathrectangle{\pgfqpoint{0.100000in}{0.212622in}}{\pgfqpoint{3.696000in}{3.696000in}}%
\pgfusepath{clip}%
\pgfsetbuttcap%
\pgfsetroundjoin%
\definecolor{currentfill}{rgb}{0.121569,0.466667,0.705882}%
\pgfsetfillcolor{currentfill}%
\pgfsetfillopacity{0.778161}%
\pgfsetlinewidth{1.003750pt}%
\definecolor{currentstroke}{rgb}{0.121569,0.466667,0.705882}%
\pgfsetstrokecolor{currentstroke}%
\pgfsetstrokeopacity{0.778161}%
\pgfsetdash{}{0pt}%
\pgfpathmoveto{\pgfqpoint{2.571511in}{2.945937in}}%
\pgfpathcurveto{\pgfqpoint{2.579747in}{2.945937in}}{\pgfqpoint{2.587647in}{2.949210in}}{\pgfqpoint{2.593471in}{2.955034in}}%
\pgfpathcurveto{\pgfqpoint{2.599295in}{2.960858in}}{\pgfqpoint{2.602567in}{2.968758in}}{\pgfqpoint{2.602567in}{2.976994in}}%
\pgfpathcurveto{\pgfqpoint{2.602567in}{2.985230in}}{\pgfqpoint{2.599295in}{2.993130in}}{\pgfqpoint{2.593471in}{2.998954in}}%
\pgfpathcurveto{\pgfqpoint{2.587647in}{3.004778in}}{\pgfqpoint{2.579747in}{3.008050in}}{\pgfqpoint{2.571511in}{3.008050in}}%
\pgfpathcurveto{\pgfqpoint{2.563274in}{3.008050in}}{\pgfqpoint{2.555374in}{3.004778in}}{\pgfqpoint{2.549550in}{2.998954in}}%
\pgfpathcurveto{\pgfqpoint{2.543726in}{2.993130in}}{\pgfqpoint{2.540454in}{2.985230in}}{\pgfqpoint{2.540454in}{2.976994in}}%
\pgfpathcurveto{\pgfqpoint{2.540454in}{2.968758in}}{\pgfqpoint{2.543726in}{2.960858in}}{\pgfqpoint{2.549550in}{2.955034in}}%
\pgfpathcurveto{\pgfqpoint{2.555374in}{2.949210in}}{\pgfqpoint{2.563274in}{2.945937in}}{\pgfqpoint{2.571511in}{2.945937in}}%
\pgfpathclose%
\pgfusepath{stroke,fill}%
\end{pgfscope}%
\begin{pgfscope}%
\pgfpathrectangle{\pgfqpoint{0.100000in}{0.212622in}}{\pgfqpoint{3.696000in}{3.696000in}}%
\pgfusepath{clip}%
\pgfsetbuttcap%
\pgfsetroundjoin%
\definecolor{currentfill}{rgb}{0.121569,0.466667,0.705882}%
\pgfsetfillcolor{currentfill}%
\pgfsetfillopacity{0.778193}%
\pgfsetlinewidth{1.003750pt}%
\definecolor{currentstroke}{rgb}{0.121569,0.466667,0.705882}%
\pgfsetstrokecolor{currentstroke}%
\pgfsetstrokeopacity{0.778193}%
\pgfsetdash{}{0pt}%
\pgfpathmoveto{\pgfqpoint{2.571647in}{2.945941in}}%
\pgfpathcurveto{\pgfqpoint{2.579883in}{2.945941in}}{\pgfqpoint{2.587783in}{2.949214in}}{\pgfqpoint{2.593607in}{2.955038in}}%
\pgfpathcurveto{\pgfqpoint{2.599431in}{2.960861in}}{\pgfqpoint{2.602704in}{2.968762in}}{\pgfqpoint{2.602704in}{2.976998in}}%
\pgfpathcurveto{\pgfqpoint{2.602704in}{2.985234in}}{\pgfqpoint{2.599431in}{2.993134in}}{\pgfqpoint{2.593607in}{2.998958in}}%
\pgfpathcurveto{\pgfqpoint{2.587783in}{3.004782in}}{\pgfqpoint{2.579883in}{3.008054in}}{\pgfqpoint{2.571647in}{3.008054in}}%
\pgfpathcurveto{\pgfqpoint{2.563411in}{3.008054in}}{\pgfqpoint{2.555511in}{3.004782in}}{\pgfqpoint{2.549687in}{2.998958in}}%
\pgfpathcurveto{\pgfqpoint{2.543863in}{2.993134in}}{\pgfqpoint{2.540591in}{2.985234in}}{\pgfqpoint{2.540591in}{2.976998in}}%
\pgfpathcurveto{\pgfqpoint{2.540591in}{2.968762in}}{\pgfqpoint{2.543863in}{2.960861in}}{\pgfqpoint{2.549687in}{2.955038in}}%
\pgfpathcurveto{\pgfqpoint{2.555511in}{2.949214in}}{\pgfqpoint{2.563411in}{2.945941in}}{\pgfqpoint{2.571647in}{2.945941in}}%
\pgfpathclose%
\pgfusepath{stroke,fill}%
\end{pgfscope}%
\begin{pgfscope}%
\pgfpathrectangle{\pgfqpoint{0.100000in}{0.212622in}}{\pgfqpoint{3.696000in}{3.696000in}}%
\pgfusepath{clip}%
\pgfsetbuttcap%
\pgfsetroundjoin%
\definecolor{currentfill}{rgb}{0.121569,0.466667,0.705882}%
\pgfsetfillcolor{currentfill}%
\pgfsetfillopacity{0.778254}%
\pgfsetlinewidth{1.003750pt}%
\definecolor{currentstroke}{rgb}{0.121569,0.466667,0.705882}%
\pgfsetstrokecolor{currentstroke}%
\pgfsetstrokeopacity{0.778254}%
\pgfsetdash{}{0pt}%
\pgfpathmoveto{\pgfqpoint{2.571875in}{2.945898in}}%
\pgfpathcurveto{\pgfqpoint{2.580111in}{2.945898in}}{\pgfqpoint{2.588011in}{2.949171in}}{\pgfqpoint{2.593835in}{2.954994in}}%
\pgfpathcurveto{\pgfqpoint{2.599659in}{2.960818in}}{\pgfqpoint{2.602931in}{2.968718in}}{\pgfqpoint{2.602931in}{2.976955in}}%
\pgfpathcurveto{\pgfqpoint{2.602931in}{2.985191in}}{\pgfqpoint{2.599659in}{2.993091in}}{\pgfqpoint{2.593835in}{2.998915in}}%
\pgfpathcurveto{\pgfqpoint{2.588011in}{3.004739in}}{\pgfqpoint{2.580111in}{3.008011in}}{\pgfqpoint{2.571875in}{3.008011in}}%
\pgfpathcurveto{\pgfqpoint{2.563638in}{3.008011in}}{\pgfqpoint{2.555738in}{3.004739in}}{\pgfqpoint{2.549914in}{2.998915in}}%
\pgfpathcurveto{\pgfqpoint{2.544090in}{2.993091in}}{\pgfqpoint{2.540818in}{2.985191in}}{\pgfqpoint{2.540818in}{2.976955in}}%
\pgfpathcurveto{\pgfqpoint{2.540818in}{2.968718in}}{\pgfqpoint{2.544090in}{2.960818in}}{\pgfqpoint{2.549914in}{2.954994in}}%
\pgfpathcurveto{\pgfqpoint{2.555738in}{2.949171in}}{\pgfqpoint{2.563638in}{2.945898in}}{\pgfqpoint{2.571875in}{2.945898in}}%
\pgfpathclose%
\pgfusepath{stroke,fill}%
\end{pgfscope}%
\begin{pgfscope}%
\pgfpathrectangle{\pgfqpoint{0.100000in}{0.212622in}}{\pgfqpoint{3.696000in}{3.696000in}}%
\pgfusepath{clip}%
\pgfsetbuttcap%
\pgfsetroundjoin%
\definecolor{currentfill}{rgb}{0.121569,0.466667,0.705882}%
\pgfsetfillcolor{currentfill}%
\pgfsetfillopacity{0.778373}%
\pgfsetlinewidth{1.003750pt}%
\definecolor{currentstroke}{rgb}{0.121569,0.466667,0.705882}%
\pgfsetstrokecolor{currentstroke}%
\pgfsetstrokeopacity{0.778373}%
\pgfsetdash{}{0pt}%
\pgfpathmoveto{\pgfqpoint{2.572288in}{2.945855in}}%
\pgfpathcurveto{\pgfqpoint{2.580524in}{2.945855in}}{\pgfqpoint{2.588424in}{2.949127in}}{\pgfqpoint{2.594248in}{2.954951in}}%
\pgfpathcurveto{\pgfqpoint{2.600072in}{2.960775in}}{\pgfqpoint{2.603345in}{2.968675in}}{\pgfqpoint{2.603345in}{2.976912in}}%
\pgfpathcurveto{\pgfqpoint{2.603345in}{2.985148in}}{\pgfqpoint{2.600072in}{2.993048in}}{\pgfqpoint{2.594248in}{2.998872in}}%
\pgfpathcurveto{\pgfqpoint{2.588424in}{3.004696in}}{\pgfqpoint{2.580524in}{3.007968in}}{\pgfqpoint{2.572288in}{3.007968in}}%
\pgfpathcurveto{\pgfqpoint{2.564052in}{3.007968in}}{\pgfqpoint{2.556152in}{3.004696in}}{\pgfqpoint{2.550328in}{2.998872in}}%
\pgfpathcurveto{\pgfqpoint{2.544504in}{2.993048in}}{\pgfqpoint{2.541232in}{2.985148in}}{\pgfqpoint{2.541232in}{2.976912in}}%
\pgfpathcurveto{\pgfqpoint{2.541232in}{2.968675in}}{\pgfqpoint{2.544504in}{2.960775in}}{\pgfqpoint{2.550328in}{2.954951in}}%
\pgfpathcurveto{\pgfqpoint{2.556152in}{2.949127in}}{\pgfqpoint{2.564052in}{2.945855in}}{\pgfqpoint{2.572288in}{2.945855in}}%
\pgfpathclose%
\pgfusepath{stroke,fill}%
\end{pgfscope}%
\begin{pgfscope}%
\pgfpathrectangle{\pgfqpoint{0.100000in}{0.212622in}}{\pgfqpoint{3.696000in}{3.696000in}}%
\pgfusepath{clip}%
\pgfsetbuttcap%
\pgfsetroundjoin%
\definecolor{currentfill}{rgb}{0.121569,0.466667,0.705882}%
\pgfsetfillcolor{currentfill}%
\pgfsetfillopacity{0.778408}%
\pgfsetlinewidth{1.003750pt}%
\definecolor{currentstroke}{rgb}{0.121569,0.466667,0.705882}%
\pgfsetstrokecolor{currentstroke}%
\pgfsetstrokeopacity{0.778408}%
\pgfsetdash{}{0pt}%
\pgfpathmoveto{\pgfqpoint{2.572383in}{2.945819in}}%
\pgfpathcurveto{\pgfqpoint{2.580620in}{2.945819in}}{\pgfqpoint{2.588520in}{2.949091in}}{\pgfqpoint{2.594343in}{2.954915in}}%
\pgfpathcurveto{\pgfqpoint{2.600167in}{2.960739in}}{\pgfqpoint{2.603440in}{2.968639in}}{\pgfqpoint{2.603440in}{2.976876in}}%
\pgfpathcurveto{\pgfqpoint{2.603440in}{2.985112in}}{\pgfqpoint{2.600167in}{2.993012in}}{\pgfqpoint{2.594343in}{2.998836in}}%
\pgfpathcurveto{\pgfqpoint{2.588520in}{3.004660in}}{\pgfqpoint{2.580620in}{3.007932in}}{\pgfqpoint{2.572383in}{3.007932in}}%
\pgfpathcurveto{\pgfqpoint{2.564147in}{3.007932in}}{\pgfqpoint{2.556247in}{3.004660in}}{\pgfqpoint{2.550423in}{2.998836in}}%
\pgfpathcurveto{\pgfqpoint{2.544599in}{2.993012in}}{\pgfqpoint{2.541327in}{2.985112in}}{\pgfqpoint{2.541327in}{2.976876in}}%
\pgfpathcurveto{\pgfqpoint{2.541327in}{2.968639in}}{\pgfqpoint{2.544599in}{2.960739in}}{\pgfqpoint{2.550423in}{2.954915in}}%
\pgfpathcurveto{\pgfqpoint{2.556247in}{2.949091in}}{\pgfqpoint{2.564147in}{2.945819in}}{\pgfqpoint{2.572383in}{2.945819in}}%
\pgfpathclose%
\pgfusepath{stroke,fill}%
\end{pgfscope}%
\begin{pgfscope}%
\pgfpathrectangle{\pgfqpoint{0.100000in}{0.212622in}}{\pgfqpoint{3.696000in}{3.696000in}}%
\pgfusepath{clip}%
\pgfsetbuttcap%
\pgfsetroundjoin%
\definecolor{currentfill}{rgb}{0.121569,0.466667,0.705882}%
\pgfsetfillcolor{currentfill}%
\pgfsetfillopacity{0.778484}%
\pgfsetlinewidth{1.003750pt}%
\definecolor{currentstroke}{rgb}{0.121569,0.466667,0.705882}%
\pgfsetstrokecolor{currentstroke}%
\pgfsetstrokeopacity{0.778484}%
\pgfsetdash{}{0pt}%
\pgfpathmoveto{\pgfqpoint{2.572523in}{2.945732in}}%
\pgfpathcurveto{\pgfqpoint{2.580759in}{2.945732in}}{\pgfqpoint{2.588659in}{2.949004in}}{\pgfqpoint{2.594483in}{2.954828in}}%
\pgfpathcurveto{\pgfqpoint{2.600307in}{2.960652in}}{\pgfqpoint{2.603579in}{2.968552in}}{\pgfqpoint{2.603579in}{2.976788in}}%
\pgfpathcurveto{\pgfqpoint{2.603579in}{2.985025in}}{\pgfqpoint{2.600307in}{2.992925in}}{\pgfqpoint{2.594483in}{2.998749in}}%
\pgfpathcurveto{\pgfqpoint{2.588659in}{3.004573in}}{\pgfqpoint{2.580759in}{3.007845in}}{\pgfqpoint{2.572523in}{3.007845in}}%
\pgfpathcurveto{\pgfqpoint{2.564286in}{3.007845in}}{\pgfqpoint{2.556386in}{3.004573in}}{\pgfqpoint{2.550562in}{2.998749in}}%
\pgfpathcurveto{\pgfqpoint{2.544738in}{2.992925in}}{\pgfqpoint{2.541466in}{2.985025in}}{\pgfqpoint{2.541466in}{2.976788in}}%
\pgfpathcurveto{\pgfqpoint{2.541466in}{2.968552in}}{\pgfqpoint{2.544738in}{2.960652in}}{\pgfqpoint{2.550562in}{2.954828in}}%
\pgfpathcurveto{\pgfqpoint{2.556386in}{2.949004in}}{\pgfqpoint{2.564286in}{2.945732in}}{\pgfqpoint{2.572523in}{2.945732in}}%
\pgfpathclose%
\pgfusepath{stroke,fill}%
\end{pgfscope}%
\begin{pgfscope}%
\pgfpathrectangle{\pgfqpoint{0.100000in}{0.212622in}}{\pgfqpoint{3.696000in}{3.696000in}}%
\pgfusepath{clip}%
\pgfsetbuttcap%
\pgfsetroundjoin%
\definecolor{currentfill}{rgb}{0.121569,0.466667,0.705882}%
\pgfsetfillcolor{currentfill}%
\pgfsetfillopacity{0.778643}%
\pgfsetlinewidth{1.003750pt}%
\definecolor{currentstroke}{rgb}{0.121569,0.466667,0.705882}%
\pgfsetstrokecolor{currentstroke}%
\pgfsetstrokeopacity{0.778643}%
\pgfsetdash{}{0pt}%
\pgfpathmoveto{\pgfqpoint{2.572622in}{2.945437in}}%
\pgfpathcurveto{\pgfqpoint{2.580859in}{2.945437in}}{\pgfqpoint{2.588759in}{2.948709in}}{\pgfqpoint{2.594583in}{2.954533in}}%
\pgfpathcurveto{\pgfqpoint{2.600407in}{2.960357in}}{\pgfqpoint{2.603679in}{2.968257in}}{\pgfqpoint{2.603679in}{2.976493in}}%
\pgfpathcurveto{\pgfqpoint{2.603679in}{2.984730in}}{\pgfqpoint{2.600407in}{2.992630in}}{\pgfqpoint{2.594583in}{2.998454in}}%
\pgfpathcurveto{\pgfqpoint{2.588759in}{3.004277in}}{\pgfqpoint{2.580859in}{3.007550in}}{\pgfqpoint{2.572622in}{3.007550in}}%
\pgfpathcurveto{\pgfqpoint{2.564386in}{3.007550in}}{\pgfqpoint{2.556486in}{3.004277in}}{\pgfqpoint{2.550662in}{2.998454in}}%
\pgfpathcurveto{\pgfqpoint{2.544838in}{2.992630in}}{\pgfqpoint{2.541566in}{2.984730in}}{\pgfqpoint{2.541566in}{2.976493in}}%
\pgfpathcurveto{\pgfqpoint{2.541566in}{2.968257in}}{\pgfqpoint{2.544838in}{2.960357in}}{\pgfqpoint{2.550662in}{2.954533in}}%
\pgfpathcurveto{\pgfqpoint{2.556486in}{2.948709in}}{\pgfqpoint{2.564386in}{2.945437in}}{\pgfqpoint{2.572622in}{2.945437in}}%
\pgfpathclose%
\pgfusepath{stroke,fill}%
\end{pgfscope}%
\begin{pgfscope}%
\pgfpathrectangle{\pgfqpoint{0.100000in}{0.212622in}}{\pgfqpoint{3.696000in}{3.696000in}}%
\pgfusepath{clip}%
\pgfsetbuttcap%
\pgfsetroundjoin%
\definecolor{currentfill}{rgb}{0.121569,0.466667,0.705882}%
\pgfsetfillcolor{currentfill}%
\pgfsetfillopacity{0.778718}%
\pgfsetlinewidth{1.003750pt}%
\definecolor{currentstroke}{rgb}{0.121569,0.466667,0.705882}%
\pgfsetstrokecolor{currentstroke}%
\pgfsetstrokeopacity{0.778718}%
\pgfsetdash{}{0pt}%
\pgfpathmoveto{\pgfqpoint{1.006067in}{1.554877in}}%
\pgfpathcurveto{\pgfqpoint{1.014304in}{1.554877in}}{\pgfqpoint{1.022204in}{1.558149in}}{\pgfqpoint{1.028028in}{1.563973in}}%
\pgfpathcurveto{\pgfqpoint{1.033852in}{1.569797in}}{\pgfqpoint{1.037124in}{1.577697in}}{\pgfqpoint{1.037124in}{1.585934in}}%
\pgfpathcurveto{\pgfqpoint{1.037124in}{1.594170in}}{\pgfqpoint{1.033852in}{1.602070in}}{\pgfqpoint{1.028028in}{1.607894in}}%
\pgfpathcurveto{\pgfqpoint{1.022204in}{1.613718in}}{\pgfqpoint{1.014304in}{1.616990in}}{\pgfqpoint{1.006067in}{1.616990in}}%
\pgfpathcurveto{\pgfqpoint{0.997831in}{1.616990in}}{\pgfqpoint{0.989931in}{1.613718in}}{\pgfqpoint{0.984107in}{1.607894in}}%
\pgfpathcurveto{\pgfqpoint{0.978283in}{1.602070in}}{\pgfqpoint{0.975011in}{1.594170in}}{\pgfqpoint{0.975011in}{1.585934in}}%
\pgfpathcurveto{\pgfqpoint{0.975011in}{1.577697in}}{\pgfqpoint{0.978283in}{1.569797in}}{\pgfqpoint{0.984107in}{1.563973in}}%
\pgfpathcurveto{\pgfqpoint{0.989931in}{1.558149in}}{\pgfqpoint{0.997831in}{1.554877in}}{\pgfqpoint{1.006067in}{1.554877in}}%
\pgfpathclose%
\pgfusepath{stroke,fill}%
\end{pgfscope}%
\begin{pgfscope}%
\pgfpathrectangle{\pgfqpoint{0.100000in}{0.212622in}}{\pgfqpoint{3.696000in}{3.696000in}}%
\pgfusepath{clip}%
\pgfsetbuttcap%
\pgfsetroundjoin%
\definecolor{currentfill}{rgb}{0.121569,0.466667,0.705882}%
\pgfsetfillcolor{currentfill}%
\pgfsetfillopacity{0.778911}%
\pgfsetlinewidth{1.003750pt}%
\definecolor{currentstroke}{rgb}{0.121569,0.466667,0.705882}%
\pgfsetstrokecolor{currentstroke}%
\pgfsetstrokeopacity{0.778911}%
\pgfsetdash{}{0pt}%
\pgfpathmoveto{\pgfqpoint{2.572611in}{2.944651in}}%
\pgfpathcurveto{\pgfqpoint{2.580848in}{2.944651in}}{\pgfqpoint{2.588748in}{2.947923in}}{\pgfqpoint{2.594572in}{2.953747in}}%
\pgfpathcurveto{\pgfqpoint{2.600396in}{2.959571in}}{\pgfqpoint{2.603668in}{2.967471in}}{\pgfqpoint{2.603668in}{2.975707in}}%
\pgfpathcurveto{\pgfqpoint{2.603668in}{2.983943in}}{\pgfqpoint{2.600396in}{2.991843in}}{\pgfqpoint{2.594572in}{2.997667in}}%
\pgfpathcurveto{\pgfqpoint{2.588748in}{3.003491in}}{\pgfqpoint{2.580848in}{3.006764in}}{\pgfqpoint{2.572611in}{3.006764in}}%
\pgfpathcurveto{\pgfqpoint{2.564375in}{3.006764in}}{\pgfqpoint{2.556475in}{3.003491in}}{\pgfqpoint{2.550651in}{2.997667in}}%
\pgfpathcurveto{\pgfqpoint{2.544827in}{2.991843in}}{\pgfqpoint{2.541555in}{2.983943in}}{\pgfqpoint{2.541555in}{2.975707in}}%
\pgfpathcurveto{\pgfqpoint{2.541555in}{2.967471in}}{\pgfqpoint{2.544827in}{2.959571in}}{\pgfqpoint{2.550651in}{2.953747in}}%
\pgfpathcurveto{\pgfqpoint{2.556475in}{2.947923in}}{\pgfqpoint{2.564375in}{2.944651in}}{\pgfqpoint{2.572611in}{2.944651in}}%
\pgfpathclose%
\pgfusepath{stroke,fill}%
\end{pgfscope}%
\begin{pgfscope}%
\pgfpathrectangle{\pgfqpoint{0.100000in}{0.212622in}}{\pgfqpoint{3.696000in}{3.696000in}}%
\pgfusepath{clip}%
\pgfsetbuttcap%
\pgfsetroundjoin%
\definecolor{currentfill}{rgb}{0.121569,0.466667,0.705882}%
\pgfsetfillcolor{currentfill}%
\pgfsetfillopacity{0.779078}%
\pgfsetlinewidth{1.003750pt}%
\definecolor{currentstroke}{rgb}{0.121569,0.466667,0.705882}%
\pgfsetstrokecolor{currentstroke}%
\pgfsetstrokeopacity{0.779078}%
\pgfsetdash{}{0pt}%
\pgfpathmoveto{\pgfqpoint{2.572471in}{2.944095in}}%
\pgfpathcurveto{\pgfqpoint{2.580707in}{2.944095in}}{\pgfqpoint{2.588607in}{2.947368in}}{\pgfqpoint{2.594431in}{2.953192in}}%
\pgfpathcurveto{\pgfqpoint{2.600255in}{2.959016in}}{\pgfqpoint{2.603527in}{2.966916in}}{\pgfqpoint{2.603527in}{2.975152in}}%
\pgfpathcurveto{\pgfqpoint{2.603527in}{2.983388in}}{\pgfqpoint{2.600255in}{2.991288in}}{\pgfqpoint{2.594431in}{2.997112in}}%
\pgfpathcurveto{\pgfqpoint{2.588607in}{3.002936in}}{\pgfqpoint{2.580707in}{3.006208in}}{\pgfqpoint{2.572471in}{3.006208in}}%
\pgfpathcurveto{\pgfqpoint{2.564234in}{3.006208in}}{\pgfqpoint{2.556334in}{3.002936in}}{\pgfqpoint{2.550510in}{2.997112in}}%
\pgfpathcurveto{\pgfqpoint{2.544686in}{2.991288in}}{\pgfqpoint{2.541414in}{2.983388in}}{\pgfqpoint{2.541414in}{2.975152in}}%
\pgfpathcurveto{\pgfqpoint{2.541414in}{2.966916in}}{\pgfqpoint{2.544686in}{2.959016in}}{\pgfqpoint{2.550510in}{2.953192in}}%
\pgfpathcurveto{\pgfqpoint{2.556334in}{2.947368in}}{\pgfqpoint{2.564234in}{2.944095in}}{\pgfqpoint{2.572471in}{2.944095in}}%
\pgfpathclose%
\pgfusepath{stroke,fill}%
\end{pgfscope}%
\begin{pgfscope}%
\pgfpathrectangle{\pgfqpoint{0.100000in}{0.212622in}}{\pgfqpoint{3.696000in}{3.696000in}}%
\pgfusepath{clip}%
\pgfsetbuttcap%
\pgfsetroundjoin%
\definecolor{currentfill}{rgb}{0.121569,0.466667,0.705882}%
\pgfsetfillcolor{currentfill}%
\pgfsetfillopacity{0.779078}%
\pgfsetlinewidth{1.003750pt}%
\definecolor{currentstroke}{rgb}{0.121569,0.466667,0.705882}%
\pgfsetstrokecolor{currentstroke}%
\pgfsetstrokeopacity{0.779078}%
\pgfsetdash{}{0pt}%
\pgfpathmoveto{\pgfqpoint{2.572471in}{2.944095in}}%
\pgfpathcurveto{\pgfqpoint{2.580707in}{2.944095in}}{\pgfqpoint{2.588607in}{2.947368in}}{\pgfqpoint{2.594431in}{2.953192in}}%
\pgfpathcurveto{\pgfqpoint{2.600255in}{2.959015in}}{\pgfqpoint{2.603527in}{2.966916in}}{\pgfqpoint{2.603527in}{2.975152in}}%
\pgfpathcurveto{\pgfqpoint{2.603527in}{2.983388in}}{\pgfqpoint{2.600255in}{2.991288in}}{\pgfqpoint{2.594431in}{2.997112in}}%
\pgfpathcurveto{\pgfqpoint{2.588607in}{3.002936in}}{\pgfqpoint{2.580707in}{3.006208in}}{\pgfqpoint{2.572471in}{3.006208in}}%
\pgfpathcurveto{\pgfqpoint{2.564234in}{3.006208in}}{\pgfqpoint{2.556334in}{3.002936in}}{\pgfqpoint{2.550510in}{2.997112in}}%
\pgfpathcurveto{\pgfqpoint{2.544686in}{2.991288in}}{\pgfqpoint{2.541414in}{2.983388in}}{\pgfqpoint{2.541414in}{2.975152in}}%
\pgfpathcurveto{\pgfqpoint{2.541414in}{2.966916in}}{\pgfqpoint{2.544686in}{2.959015in}}{\pgfqpoint{2.550510in}{2.953192in}}%
\pgfpathcurveto{\pgfqpoint{2.556334in}{2.947368in}}{\pgfqpoint{2.564234in}{2.944095in}}{\pgfqpoint{2.572471in}{2.944095in}}%
\pgfpathclose%
\pgfusepath{stroke,fill}%
\end{pgfscope}%
\begin{pgfscope}%
\pgfpathrectangle{\pgfqpoint{0.100000in}{0.212622in}}{\pgfqpoint{3.696000in}{3.696000in}}%
\pgfusepath{clip}%
\pgfsetbuttcap%
\pgfsetroundjoin%
\definecolor{currentfill}{rgb}{0.121569,0.466667,0.705882}%
\pgfsetfillcolor{currentfill}%
\pgfsetfillopacity{0.779078}%
\pgfsetlinewidth{1.003750pt}%
\definecolor{currentstroke}{rgb}{0.121569,0.466667,0.705882}%
\pgfsetstrokecolor{currentstroke}%
\pgfsetstrokeopacity{0.779078}%
\pgfsetdash{}{0pt}%
\pgfpathmoveto{\pgfqpoint{2.572470in}{2.944095in}}%
\pgfpathcurveto{\pgfqpoint{2.580707in}{2.944095in}}{\pgfqpoint{2.588607in}{2.947367in}}{\pgfqpoint{2.594431in}{2.953191in}}%
\pgfpathcurveto{\pgfqpoint{2.600255in}{2.959015in}}{\pgfqpoint{2.603527in}{2.966915in}}{\pgfqpoint{2.603527in}{2.975152in}}%
\pgfpathcurveto{\pgfqpoint{2.603527in}{2.983388in}}{\pgfqpoint{2.600255in}{2.991288in}}{\pgfqpoint{2.594431in}{2.997112in}}%
\pgfpathcurveto{\pgfqpoint{2.588607in}{3.002936in}}{\pgfqpoint{2.580707in}{3.006208in}}{\pgfqpoint{2.572470in}{3.006208in}}%
\pgfpathcurveto{\pgfqpoint{2.564234in}{3.006208in}}{\pgfqpoint{2.556334in}{3.002936in}}{\pgfqpoint{2.550510in}{2.997112in}}%
\pgfpathcurveto{\pgfqpoint{2.544686in}{2.991288in}}{\pgfqpoint{2.541414in}{2.983388in}}{\pgfqpoint{2.541414in}{2.975152in}}%
\pgfpathcurveto{\pgfqpoint{2.541414in}{2.966915in}}{\pgfqpoint{2.544686in}{2.959015in}}{\pgfqpoint{2.550510in}{2.953191in}}%
\pgfpathcurveto{\pgfqpoint{2.556334in}{2.947367in}}{\pgfqpoint{2.564234in}{2.944095in}}{\pgfqpoint{2.572470in}{2.944095in}}%
\pgfpathclose%
\pgfusepath{stroke,fill}%
\end{pgfscope}%
\begin{pgfscope}%
\pgfpathrectangle{\pgfqpoint{0.100000in}{0.212622in}}{\pgfqpoint{3.696000in}{3.696000in}}%
\pgfusepath{clip}%
\pgfsetbuttcap%
\pgfsetroundjoin%
\definecolor{currentfill}{rgb}{0.121569,0.466667,0.705882}%
\pgfsetfillcolor{currentfill}%
\pgfsetfillopacity{0.779078}%
\pgfsetlinewidth{1.003750pt}%
\definecolor{currentstroke}{rgb}{0.121569,0.466667,0.705882}%
\pgfsetstrokecolor{currentstroke}%
\pgfsetstrokeopacity{0.779078}%
\pgfsetdash{}{0pt}%
\pgfpathmoveto{\pgfqpoint{2.572470in}{2.944095in}}%
\pgfpathcurveto{\pgfqpoint{2.580707in}{2.944095in}}{\pgfqpoint{2.588607in}{2.947367in}}{\pgfqpoint{2.594430in}{2.953191in}}%
\pgfpathcurveto{\pgfqpoint{2.600254in}{2.959015in}}{\pgfqpoint{2.603527in}{2.966915in}}{\pgfqpoint{2.603527in}{2.975151in}}%
\pgfpathcurveto{\pgfqpoint{2.603527in}{2.983387in}}{\pgfqpoint{2.600254in}{2.991287in}}{\pgfqpoint{2.594430in}{2.997111in}}%
\pgfpathcurveto{\pgfqpoint{2.588607in}{3.002935in}}{\pgfqpoint{2.580707in}{3.006208in}}{\pgfqpoint{2.572470in}{3.006208in}}%
\pgfpathcurveto{\pgfqpoint{2.564234in}{3.006208in}}{\pgfqpoint{2.556334in}{3.002935in}}{\pgfqpoint{2.550510in}{2.997111in}}%
\pgfpathcurveto{\pgfqpoint{2.544686in}{2.991287in}}{\pgfqpoint{2.541414in}{2.983387in}}{\pgfqpoint{2.541414in}{2.975151in}}%
\pgfpathcurveto{\pgfqpoint{2.541414in}{2.966915in}}{\pgfqpoint{2.544686in}{2.959015in}}{\pgfqpoint{2.550510in}{2.953191in}}%
\pgfpathcurveto{\pgfqpoint{2.556334in}{2.947367in}}{\pgfqpoint{2.564234in}{2.944095in}}{\pgfqpoint{2.572470in}{2.944095in}}%
\pgfpathclose%
\pgfusepath{stroke,fill}%
\end{pgfscope}%
\begin{pgfscope}%
\pgfpathrectangle{\pgfqpoint{0.100000in}{0.212622in}}{\pgfqpoint{3.696000in}{3.696000in}}%
\pgfusepath{clip}%
\pgfsetbuttcap%
\pgfsetroundjoin%
\definecolor{currentfill}{rgb}{0.121569,0.466667,0.705882}%
\pgfsetfillcolor{currentfill}%
\pgfsetfillopacity{0.779079}%
\pgfsetlinewidth{1.003750pt}%
\definecolor{currentstroke}{rgb}{0.121569,0.466667,0.705882}%
\pgfsetstrokecolor{currentstroke}%
\pgfsetstrokeopacity{0.779079}%
\pgfsetdash{}{0pt}%
\pgfpathmoveto{\pgfqpoint{2.572470in}{2.944094in}}%
\pgfpathcurveto{\pgfqpoint{2.580706in}{2.944094in}}{\pgfqpoint{2.588606in}{2.947366in}}{\pgfqpoint{2.594430in}{2.953190in}}%
\pgfpathcurveto{\pgfqpoint{2.600254in}{2.959014in}}{\pgfqpoint{2.603526in}{2.966914in}}{\pgfqpoint{2.603526in}{2.975150in}}%
\pgfpathcurveto{\pgfqpoint{2.603526in}{2.983386in}}{\pgfqpoint{2.600254in}{2.991286in}}{\pgfqpoint{2.594430in}{2.997110in}}%
\pgfpathcurveto{\pgfqpoint{2.588606in}{3.002934in}}{\pgfqpoint{2.580706in}{3.006207in}}{\pgfqpoint{2.572470in}{3.006207in}}%
\pgfpathcurveto{\pgfqpoint{2.564233in}{3.006207in}}{\pgfqpoint{2.556333in}{3.002934in}}{\pgfqpoint{2.550510in}{2.997110in}}%
\pgfpathcurveto{\pgfqpoint{2.544686in}{2.991286in}}{\pgfqpoint{2.541413in}{2.983386in}}{\pgfqpoint{2.541413in}{2.975150in}}%
\pgfpathcurveto{\pgfqpoint{2.541413in}{2.966914in}}{\pgfqpoint{2.544686in}{2.959014in}}{\pgfqpoint{2.550510in}{2.953190in}}%
\pgfpathcurveto{\pgfqpoint{2.556333in}{2.947366in}}{\pgfqpoint{2.564233in}{2.944094in}}{\pgfqpoint{2.572470in}{2.944094in}}%
\pgfpathclose%
\pgfusepath{stroke,fill}%
\end{pgfscope}%
\begin{pgfscope}%
\pgfpathrectangle{\pgfqpoint{0.100000in}{0.212622in}}{\pgfqpoint{3.696000in}{3.696000in}}%
\pgfusepath{clip}%
\pgfsetbuttcap%
\pgfsetroundjoin%
\definecolor{currentfill}{rgb}{0.121569,0.466667,0.705882}%
\pgfsetfillcolor{currentfill}%
\pgfsetfillopacity{0.779079}%
\pgfsetlinewidth{1.003750pt}%
\definecolor{currentstroke}{rgb}{0.121569,0.466667,0.705882}%
\pgfsetstrokecolor{currentstroke}%
\pgfsetstrokeopacity{0.779079}%
\pgfsetdash{}{0pt}%
\pgfpathmoveto{\pgfqpoint{2.572469in}{2.944092in}}%
\pgfpathcurveto{\pgfqpoint{2.580705in}{2.944092in}}{\pgfqpoint{2.588605in}{2.947364in}}{\pgfqpoint{2.594429in}{2.953188in}}%
\pgfpathcurveto{\pgfqpoint{2.600253in}{2.959012in}}{\pgfqpoint{2.603526in}{2.966912in}}{\pgfqpoint{2.603526in}{2.975148in}}%
\pgfpathcurveto{\pgfqpoint{2.603526in}{2.983385in}}{\pgfqpoint{2.600253in}{2.991285in}}{\pgfqpoint{2.594429in}{2.997109in}}%
\pgfpathcurveto{\pgfqpoint{2.588605in}{3.002932in}}{\pgfqpoint{2.580705in}{3.006205in}}{\pgfqpoint{2.572469in}{3.006205in}}%
\pgfpathcurveto{\pgfqpoint{2.564233in}{3.006205in}}{\pgfqpoint{2.556333in}{3.002932in}}{\pgfqpoint{2.550509in}{2.997109in}}%
\pgfpathcurveto{\pgfqpoint{2.544685in}{2.991285in}}{\pgfqpoint{2.541413in}{2.983385in}}{\pgfqpoint{2.541413in}{2.975148in}}%
\pgfpathcurveto{\pgfqpoint{2.541413in}{2.966912in}}{\pgfqpoint{2.544685in}{2.959012in}}{\pgfqpoint{2.550509in}{2.953188in}}%
\pgfpathcurveto{\pgfqpoint{2.556333in}{2.947364in}}{\pgfqpoint{2.564233in}{2.944092in}}{\pgfqpoint{2.572469in}{2.944092in}}%
\pgfpathclose%
\pgfusepath{stroke,fill}%
\end{pgfscope}%
\begin{pgfscope}%
\pgfpathrectangle{\pgfqpoint{0.100000in}{0.212622in}}{\pgfqpoint{3.696000in}{3.696000in}}%
\pgfusepath{clip}%
\pgfsetbuttcap%
\pgfsetroundjoin%
\definecolor{currentfill}{rgb}{0.121569,0.466667,0.705882}%
\pgfsetfillcolor{currentfill}%
\pgfsetfillopacity{0.779079}%
\pgfsetlinewidth{1.003750pt}%
\definecolor{currentstroke}{rgb}{0.121569,0.466667,0.705882}%
\pgfsetstrokecolor{currentstroke}%
\pgfsetstrokeopacity{0.779079}%
\pgfsetdash{}{0pt}%
\pgfpathmoveto{\pgfqpoint{2.572468in}{2.944089in}}%
\pgfpathcurveto{\pgfqpoint{2.580704in}{2.944089in}}{\pgfqpoint{2.588604in}{2.947361in}}{\pgfqpoint{2.594428in}{2.953185in}}%
\pgfpathcurveto{\pgfqpoint{2.600252in}{2.959009in}}{\pgfqpoint{2.603524in}{2.966909in}}{\pgfqpoint{2.603524in}{2.975145in}}%
\pgfpathcurveto{\pgfqpoint{2.603524in}{2.983381in}}{\pgfqpoint{2.600252in}{2.991281in}}{\pgfqpoint{2.594428in}{2.997105in}}%
\pgfpathcurveto{\pgfqpoint{2.588604in}{3.002929in}}{\pgfqpoint{2.580704in}{3.006202in}}{\pgfqpoint{2.572468in}{3.006202in}}%
\pgfpathcurveto{\pgfqpoint{2.564231in}{3.006202in}}{\pgfqpoint{2.556331in}{3.002929in}}{\pgfqpoint{2.550507in}{2.997105in}}%
\pgfpathcurveto{\pgfqpoint{2.544683in}{2.991281in}}{\pgfqpoint{2.541411in}{2.983381in}}{\pgfqpoint{2.541411in}{2.975145in}}%
\pgfpathcurveto{\pgfqpoint{2.541411in}{2.966909in}}{\pgfqpoint{2.544683in}{2.959009in}}{\pgfqpoint{2.550507in}{2.953185in}}%
\pgfpathcurveto{\pgfqpoint{2.556331in}{2.947361in}}{\pgfqpoint{2.564231in}{2.944089in}}{\pgfqpoint{2.572468in}{2.944089in}}%
\pgfpathclose%
\pgfusepath{stroke,fill}%
\end{pgfscope}%
\begin{pgfscope}%
\pgfpathrectangle{\pgfqpoint{0.100000in}{0.212622in}}{\pgfqpoint{3.696000in}{3.696000in}}%
\pgfusepath{clip}%
\pgfsetbuttcap%
\pgfsetroundjoin%
\definecolor{currentfill}{rgb}{0.121569,0.466667,0.705882}%
\pgfsetfillcolor{currentfill}%
\pgfsetfillopacity{0.779081}%
\pgfsetlinewidth{1.003750pt}%
\definecolor{currentstroke}{rgb}{0.121569,0.466667,0.705882}%
\pgfsetstrokecolor{currentstroke}%
\pgfsetstrokeopacity{0.779081}%
\pgfsetdash{}{0pt}%
\pgfpathmoveto{\pgfqpoint{2.572465in}{2.944083in}}%
\pgfpathcurveto{\pgfqpoint{2.580701in}{2.944083in}}{\pgfqpoint{2.588601in}{2.947355in}}{\pgfqpoint{2.594425in}{2.953179in}}%
\pgfpathcurveto{\pgfqpoint{2.600249in}{2.959003in}}{\pgfqpoint{2.603521in}{2.966903in}}{\pgfqpoint{2.603521in}{2.975139in}}%
\pgfpathcurveto{\pgfqpoint{2.603521in}{2.983375in}}{\pgfqpoint{2.600249in}{2.991275in}}{\pgfqpoint{2.594425in}{2.997099in}}%
\pgfpathcurveto{\pgfqpoint{2.588601in}{3.002923in}}{\pgfqpoint{2.580701in}{3.006196in}}{\pgfqpoint{2.572465in}{3.006196in}}%
\pgfpathcurveto{\pgfqpoint{2.564229in}{3.006196in}}{\pgfqpoint{2.556329in}{3.002923in}}{\pgfqpoint{2.550505in}{2.997099in}}%
\pgfpathcurveto{\pgfqpoint{2.544681in}{2.991275in}}{\pgfqpoint{2.541408in}{2.983375in}}{\pgfqpoint{2.541408in}{2.975139in}}%
\pgfpathcurveto{\pgfqpoint{2.541408in}{2.966903in}}{\pgfqpoint{2.544681in}{2.959003in}}{\pgfqpoint{2.550505in}{2.953179in}}%
\pgfpathcurveto{\pgfqpoint{2.556329in}{2.947355in}}{\pgfqpoint{2.564229in}{2.944083in}}{\pgfqpoint{2.572465in}{2.944083in}}%
\pgfpathclose%
\pgfusepath{stroke,fill}%
\end{pgfscope}%
\begin{pgfscope}%
\pgfpathrectangle{\pgfqpoint{0.100000in}{0.212622in}}{\pgfqpoint{3.696000in}{3.696000in}}%
\pgfusepath{clip}%
\pgfsetbuttcap%
\pgfsetroundjoin%
\definecolor{currentfill}{rgb}{0.121569,0.466667,0.705882}%
\pgfsetfillcolor{currentfill}%
\pgfsetfillopacity{0.779082}%
\pgfsetlinewidth{1.003750pt}%
\definecolor{currentstroke}{rgb}{0.121569,0.466667,0.705882}%
\pgfsetstrokecolor{currentstroke}%
\pgfsetstrokeopacity{0.779082}%
\pgfsetdash{}{0pt}%
\pgfpathmoveto{\pgfqpoint{2.572460in}{2.944072in}}%
\pgfpathcurveto{\pgfqpoint{2.580697in}{2.944072in}}{\pgfqpoint{2.588597in}{2.947344in}}{\pgfqpoint{2.594421in}{2.953168in}}%
\pgfpathcurveto{\pgfqpoint{2.600245in}{2.958992in}}{\pgfqpoint{2.603517in}{2.966892in}}{\pgfqpoint{2.603517in}{2.975128in}}%
\pgfpathcurveto{\pgfqpoint{2.603517in}{2.983365in}}{\pgfqpoint{2.600245in}{2.991265in}}{\pgfqpoint{2.594421in}{2.997089in}}%
\pgfpathcurveto{\pgfqpoint{2.588597in}{3.002913in}}{\pgfqpoint{2.580697in}{3.006185in}}{\pgfqpoint{2.572460in}{3.006185in}}%
\pgfpathcurveto{\pgfqpoint{2.564224in}{3.006185in}}{\pgfqpoint{2.556324in}{3.002913in}}{\pgfqpoint{2.550500in}{2.997089in}}%
\pgfpathcurveto{\pgfqpoint{2.544676in}{2.991265in}}{\pgfqpoint{2.541404in}{2.983365in}}{\pgfqpoint{2.541404in}{2.975128in}}%
\pgfpathcurveto{\pgfqpoint{2.541404in}{2.966892in}}{\pgfqpoint{2.544676in}{2.958992in}}{\pgfqpoint{2.550500in}{2.953168in}}%
\pgfpathcurveto{\pgfqpoint{2.556324in}{2.947344in}}{\pgfqpoint{2.564224in}{2.944072in}}{\pgfqpoint{2.572460in}{2.944072in}}%
\pgfpathclose%
\pgfusepath{stroke,fill}%
\end{pgfscope}%
\begin{pgfscope}%
\pgfpathrectangle{\pgfqpoint{0.100000in}{0.212622in}}{\pgfqpoint{3.696000in}{3.696000in}}%
\pgfusepath{clip}%
\pgfsetbuttcap%
\pgfsetroundjoin%
\definecolor{currentfill}{rgb}{0.121569,0.466667,0.705882}%
\pgfsetfillcolor{currentfill}%
\pgfsetfillopacity{0.779086}%
\pgfsetlinewidth{1.003750pt}%
\definecolor{currentstroke}{rgb}{0.121569,0.466667,0.705882}%
\pgfsetstrokecolor{currentstroke}%
\pgfsetstrokeopacity{0.779086}%
\pgfsetdash{}{0pt}%
\pgfpathmoveto{\pgfqpoint{2.572452in}{2.944052in}}%
\pgfpathcurveto{\pgfqpoint{2.580689in}{2.944052in}}{\pgfqpoint{2.588589in}{2.947325in}}{\pgfqpoint{2.594413in}{2.953148in}}%
\pgfpathcurveto{\pgfqpoint{2.600236in}{2.958972in}}{\pgfqpoint{2.603509in}{2.966872in}}{\pgfqpoint{2.603509in}{2.975109in}}%
\pgfpathcurveto{\pgfqpoint{2.603509in}{2.983345in}}{\pgfqpoint{2.600236in}{2.991245in}}{\pgfqpoint{2.594413in}{2.997069in}}%
\pgfpathcurveto{\pgfqpoint{2.588589in}{3.002893in}}{\pgfqpoint{2.580689in}{3.006165in}}{\pgfqpoint{2.572452in}{3.006165in}}%
\pgfpathcurveto{\pgfqpoint{2.564216in}{3.006165in}}{\pgfqpoint{2.556316in}{3.002893in}}{\pgfqpoint{2.550492in}{2.997069in}}%
\pgfpathcurveto{\pgfqpoint{2.544668in}{2.991245in}}{\pgfqpoint{2.541396in}{2.983345in}}{\pgfqpoint{2.541396in}{2.975109in}}%
\pgfpathcurveto{\pgfqpoint{2.541396in}{2.966872in}}{\pgfqpoint{2.544668in}{2.958972in}}{\pgfqpoint{2.550492in}{2.953148in}}%
\pgfpathcurveto{\pgfqpoint{2.556316in}{2.947325in}}{\pgfqpoint{2.564216in}{2.944052in}}{\pgfqpoint{2.572452in}{2.944052in}}%
\pgfpathclose%
\pgfusepath{stroke,fill}%
\end{pgfscope}%
\begin{pgfscope}%
\pgfpathrectangle{\pgfqpoint{0.100000in}{0.212622in}}{\pgfqpoint{3.696000in}{3.696000in}}%
\pgfusepath{clip}%
\pgfsetbuttcap%
\pgfsetroundjoin%
\definecolor{currentfill}{rgb}{0.121569,0.466667,0.705882}%
\pgfsetfillcolor{currentfill}%
\pgfsetfillopacity{0.779092}%
\pgfsetlinewidth{1.003750pt}%
\definecolor{currentstroke}{rgb}{0.121569,0.466667,0.705882}%
\pgfsetstrokecolor{currentstroke}%
\pgfsetstrokeopacity{0.779092}%
\pgfsetdash{}{0pt}%
\pgfpathmoveto{\pgfqpoint{2.572437in}{2.944017in}}%
\pgfpathcurveto{\pgfqpoint{2.580674in}{2.944017in}}{\pgfqpoint{2.588574in}{2.947290in}}{\pgfqpoint{2.594398in}{2.953114in}}%
\pgfpathcurveto{\pgfqpoint{2.600221in}{2.958938in}}{\pgfqpoint{2.603494in}{2.966838in}}{\pgfqpoint{2.603494in}{2.975074in}}%
\pgfpathcurveto{\pgfqpoint{2.603494in}{2.983310in}}{\pgfqpoint{2.600221in}{2.991210in}}{\pgfqpoint{2.594398in}{2.997034in}}%
\pgfpathcurveto{\pgfqpoint{2.588574in}{3.002858in}}{\pgfqpoint{2.580674in}{3.006130in}}{\pgfqpoint{2.572437in}{3.006130in}}%
\pgfpathcurveto{\pgfqpoint{2.564201in}{3.006130in}}{\pgfqpoint{2.556301in}{3.002858in}}{\pgfqpoint{2.550477in}{2.997034in}}%
\pgfpathcurveto{\pgfqpoint{2.544653in}{2.991210in}}{\pgfqpoint{2.541381in}{2.983310in}}{\pgfqpoint{2.541381in}{2.975074in}}%
\pgfpathcurveto{\pgfqpoint{2.541381in}{2.966838in}}{\pgfqpoint{2.544653in}{2.958938in}}{\pgfqpoint{2.550477in}{2.953114in}}%
\pgfpathcurveto{\pgfqpoint{2.556301in}{2.947290in}}{\pgfqpoint{2.564201in}{2.944017in}}{\pgfqpoint{2.572437in}{2.944017in}}%
\pgfpathclose%
\pgfusepath{stroke,fill}%
\end{pgfscope}%
\begin{pgfscope}%
\pgfpathrectangle{\pgfqpoint{0.100000in}{0.212622in}}{\pgfqpoint{3.696000in}{3.696000in}}%
\pgfusepath{clip}%
\pgfsetbuttcap%
\pgfsetroundjoin%
\definecolor{currentfill}{rgb}{0.121569,0.466667,0.705882}%
\pgfsetfillcolor{currentfill}%
\pgfsetfillopacity{0.779104}%
\pgfsetlinewidth{1.003750pt}%
\definecolor{currentstroke}{rgb}{0.121569,0.466667,0.705882}%
\pgfsetstrokecolor{currentstroke}%
\pgfsetstrokeopacity{0.779104}%
\pgfsetdash{}{0pt}%
\pgfpathmoveto{\pgfqpoint{2.572408in}{2.943954in}}%
\pgfpathcurveto{\pgfqpoint{2.580644in}{2.943954in}}{\pgfqpoint{2.588544in}{2.947226in}}{\pgfqpoint{2.594368in}{2.953050in}}%
\pgfpathcurveto{\pgfqpoint{2.600192in}{2.958874in}}{\pgfqpoint{2.603464in}{2.966774in}}{\pgfqpoint{2.603464in}{2.975010in}}%
\pgfpathcurveto{\pgfqpoint{2.603464in}{2.983247in}}{\pgfqpoint{2.600192in}{2.991147in}}{\pgfqpoint{2.594368in}{2.996971in}}%
\pgfpathcurveto{\pgfqpoint{2.588544in}{3.002794in}}{\pgfqpoint{2.580644in}{3.006067in}}{\pgfqpoint{2.572408in}{3.006067in}}%
\pgfpathcurveto{\pgfqpoint{2.564172in}{3.006067in}}{\pgfqpoint{2.556272in}{3.002794in}}{\pgfqpoint{2.550448in}{2.996971in}}%
\pgfpathcurveto{\pgfqpoint{2.544624in}{2.991147in}}{\pgfqpoint{2.541351in}{2.983247in}}{\pgfqpoint{2.541351in}{2.975010in}}%
\pgfpathcurveto{\pgfqpoint{2.541351in}{2.966774in}}{\pgfqpoint{2.544624in}{2.958874in}}{\pgfqpoint{2.550448in}{2.953050in}}%
\pgfpathcurveto{\pgfqpoint{2.556272in}{2.947226in}}{\pgfqpoint{2.564172in}{2.943954in}}{\pgfqpoint{2.572408in}{2.943954in}}%
\pgfpathclose%
\pgfusepath{stroke,fill}%
\end{pgfscope}%
\begin{pgfscope}%
\pgfpathrectangle{\pgfqpoint{0.100000in}{0.212622in}}{\pgfqpoint{3.696000in}{3.696000in}}%
\pgfusepath{clip}%
\pgfsetbuttcap%
\pgfsetroundjoin%
\definecolor{currentfill}{rgb}{0.121569,0.466667,0.705882}%
\pgfsetfillcolor{currentfill}%
\pgfsetfillopacity{0.779126}%
\pgfsetlinewidth{1.003750pt}%
\definecolor{currentstroke}{rgb}{0.121569,0.466667,0.705882}%
\pgfsetstrokecolor{currentstroke}%
\pgfsetstrokeopacity{0.779126}%
\pgfsetdash{}{0pt}%
\pgfpathmoveto{\pgfqpoint{2.572360in}{2.943843in}}%
\pgfpathcurveto{\pgfqpoint{2.580596in}{2.943843in}}{\pgfqpoint{2.588496in}{2.947116in}}{\pgfqpoint{2.594320in}{2.952939in}}%
\pgfpathcurveto{\pgfqpoint{2.600144in}{2.958763in}}{\pgfqpoint{2.603416in}{2.966663in}}{\pgfqpoint{2.603416in}{2.974900in}}%
\pgfpathcurveto{\pgfqpoint{2.603416in}{2.983136in}}{\pgfqpoint{2.600144in}{2.991036in}}{\pgfqpoint{2.594320in}{2.996860in}}%
\pgfpathcurveto{\pgfqpoint{2.588496in}{3.002684in}}{\pgfqpoint{2.580596in}{3.005956in}}{\pgfqpoint{2.572360in}{3.005956in}}%
\pgfpathcurveto{\pgfqpoint{2.564124in}{3.005956in}}{\pgfqpoint{2.556223in}{3.002684in}}{\pgfqpoint{2.550400in}{2.996860in}}%
\pgfpathcurveto{\pgfqpoint{2.544576in}{2.991036in}}{\pgfqpoint{2.541303in}{2.983136in}}{\pgfqpoint{2.541303in}{2.974900in}}%
\pgfpathcurveto{\pgfqpoint{2.541303in}{2.966663in}}{\pgfqpoint{2.544576in}{2.958763in}}{\pgfqpoint{2.550400in}{2.952939in}}%
\pgfpathcurveto{\pgfqpoint{2.556223in}{2.947116in}}{\pgfqpoint{2.564124in}{2.943843in}}{\pgfqpoint{2.572360in}{2.943843in}}%
\pgfpathclose%
\pgfusepath{stroke,fill}%
\end{pgfscope}%
\begin{pgfscope}%
\pgfpathrectangle{\pgfqpoint{0.100000in}{0.212622in}}{\pgfqpoint{3.696000in}{3.696000in}}%
\pgfusepath{clip}%
\pgfsetbuttcap%
\pgfsetroundjoin%
\definecolor{currentfill}{rgb}{0.121569,0.466667,0.705882}%
\pgfsetfillcolor{currentfill}%
\pgfsetfillopacity{0.779161}%
\pgfsetlinewidth{1.003750pt}%
\definecolor{currentstroke}{rgb}{0.121569,0.466667,0.705882}%
\pgfsetstrokecolor{currentstroke}%
\pgfsetstrokeopacity{0.779161}%
\pgfsetdash{}{0pt}%
\pgfpathmoveto{\pgfqpoint{2.572262in}{2.943623in}}%
\pgfpathcurveto{\pgfqpoint{2.580498in}{2.943623in}}{\pgfqpoint{2.588398in}{2.946895in}}{\pgfqpoint{2.594222in}{2.952719in}}%
\pgfpathcurveto{\pgfqpoint{2.600046in}{2.958543in}}{\pgfqpoint{2.603318in}{2.966443in}}{\pgfqpoint{2.603318in}{2.974679in}}%
\pgfpathcurveto{\pgfqpoint{2.603318in}{2.982916in}}{\pgfqpoint{2.600046in}{2.990816in}}{\pgfqpoint{2.594222in}{2.996639in}}%
\pgfpathcurveto{\pgfqpoint{2.588398in}{3.002463in}}{\pgfqpoint{2.580498in}{3.005736in}}{\pgfqpoint{2.572262in}{3.005736in}}%
\pgfpathcurveto{\pgfqpoint{2.564026in}{3.005736in}}{\pgfqpoint{2.556126in}{3.002463in}}{\pgfqpoint{2.550302in}{2.996639in}}%
\pgfpathcurveto{\pgfqpoint{2.544478in}{2.990816in}}{\pgfqpoint{2.541205in}{2.982916in}}{\pgfqpoint{2.541205in}{2.974679in}}%
\pgfpathcurveto{\pgfqpoint{2.541205in}{2.966443in}}{\pgfqpoint{2.544478in}{2.958543in}}{\pgfqpoint{2.550302in}{2.952719in}}%
\pgfpathcurveto{\pgfqpoint{2.556126in}{2.946895in}}{\pgfqpoint{2.564026in}{2.943623in}}{\pgfqpoint{2.572262in}{2.943623in}}%
\pgfpathclose%
\pgfusepath{stroke,fill}%
\end{pgfscope}%
\begin{pgfscope}%
\pgfpathrectangle{\pgfqpoint{0.100000in}{0.212622in}}{\pgfqpoint{3.696000in}{3.696000in}}%
\pgfusepath{clip}%
\pgfsetbuttcap%
\pgfsetroundjoin%
\definecolor{currentfill}{rgb}{0.121569,0.466667,0.705882}%
\pgfsetfillcolor{currentfill}%
\pgfsetfillopacity{0.779223}%
\pgfsetlinewidth{1.003750pt}%
\definecolor{currentstroke}{rgb}{0.121569,0.466667,0.705882}%
\pgfsetstrokecolor{currentstroke}%
\pgfsetstrokeopacity{0.779223}%
\pgfsetdash{}{0pt}%
\pgfpathmoveto{\pgfqpoint{2.572083in}{2.943213in}}%
\pgfpathcurveto{\pgfqpoint{2.580319in}{2.943213in}}{\pgfqpoint{2.588220in}{2.946485in}}{\pgfqpoint{2.594043in}{2.952309in}}%
\pgfpathcurveto{\pgfqpoint{2.599867in}{2.958133in}}{\pgfqpoint{2.603140in}{2.966033in}}{\pgfqpoint{2.603140in}{2.974269in}}%
\pgfpathcurveto{\pgfqpoint{2.603140in}{2.982506in}}{\pgfqpoint{2.599867in}{2.990406in}}{\pgfqpoint{2.594043in}{2.996230in}}%
\pgfpathcurveto{\pgfqpoint{2.588220in}{3.002053in}}{\pgfqpoint{2.580319in}{3.005326in}}{\pgfqpoint{2.572083in}{3.005326in}}%
\pgfpathcurveto{\pgfqpoint{2.563847in}{3.005326in}}{\pgfqpoint{2.555947in}{3.002053in}}{\pgfqpoint{2.550123in}{2.996230in}}%
\pgfpathcurveto{\pgfqpoint{2.544299in}{2.990406in}}{\pgfqpoint{2.541027in}{2.982506in}}{\pgfqpoint{2.541027in}{2.974269in}}%
\pgfpathcurveto{\pgfqpoint{2.541027in}{2.966033in}}{\pgfqpoint{2.544299in}{2.958133in}}{\pgfqpoint{2.550123in}{2.952309in}}%
\pgfpathcurveto{\pgfqpoint{2.555947in}{2.946485in}}{\pgfqpoint{2.563847in}{2.943213in}}{\pgfqpoint{2.572083in}{2.943213in}}%
\pgfpathclose%
\pgfusepath{stroke,fill}%
\end{pgfscope}%
\begin{pgfscope}%
\pgfpathrectangle{\pgfqpoint{0.100000in}{0.212622in}}{\pgfqpoint{3.696000in}{3.696000in}}%
\pgfusepath{clip}%
\pgfsetbuttcap%
\pgfsetroundjoin%
\definecolor{currentfill}{rgb}{0.121569,0.466667,0.705882}%
\pgfsetfillcolor{currentfill}%
\pgfsetfillopacity{0.779332}%
\pgfsetlinewidth{1.003750pt}%
\definecolor{currentstroke}{rgb}{0.121569,0.466667,0.705882}%
\pgfsetstrokecolor{currentstroke}%
\pgfsetstrokeopacity{0.779332}%
\pgfsetdash{}{0pt}%
\pgfpathmoveto{\pgfqpoint{2.571713in}{2.942506in}}%
\pgfpathcurveto{\pgfqpoint{2.579949in}{2.942506in}}{\pgfqpoint{2.587849in}{2.945779in}}{\pgfqpoint{2.593673in}{2.951603in}}%
\pgfpathcurveto{\pgfqpoint{2.599497in}{2.957427in}}{\pgfqpoint{2.602769in}{2.965327in}}{\pgfqpoint{2.602769in}{2.973563in}}%
\pgfpathcurveto{\pgfqpoint{2.602769in}{2.981799in}}{\pgfqpoint{2.599497in}{2.989699in}}{\pgfqpoint{2.593673in}{2.995523in}}%
\pgfpathcurveto{\pgfqpoint{2.587849in}{3.001347in}}{\pgfqpoint{2.579949in}{3.004619in}}{\pgfqpoint{2.571713in}{3.004619in}}%
\pgfpathcurveto{\pgfqpoint{2.563476in}{3.004619in}}{\pgfqpoint{2.555576in}{3.001347in}}{\pgfqpoint{2.549752in}{2.995523in}}%
\pgfpathcurveto{\pgfqpoint{2.543928in}{2.989699in}}{\pgfqpoint{2.540656in}{2.981799in}}{\pgfqpoint{2.540656in}{2.973563in}}%
\pgfpathcurveto{\pgfqpoint{2.540656in}{2.965327in}}{\pgfqpoint{2.543928in}{2.957427in}}{\pgfqpoint{2.549752in}{2.951603in}}%
\pgfpathcurveto{\pgfqpoint{2.555576in}{2.945779in}}{\pgfqpoint{2.563476in}{2.942506in}}{\pgfqpoint{2.571713in}{2.942506in}}%
\pgfpathclose%
\pgfusepath{stroke,fill}%
\end{pgfscope}%
\begin{pgfscope}%
\pgfpathrectangle{\pgfqpoint{0.100000in}{0.212622in}}{\pgfqpoint{3.696000in}{3.696000in}}%
\pgfusepath{clip}%
\pgfsetbuttcap%
\pgfsetroundjoin%
\definecolor{currentfill}{rgb}{0.121569,0.466667,0.705882}%
\pgfsetfillcolor{currentfill}%
\pgfsetfillopacity{0.779408}%
\pgfsetlinewidth{1.003750pt}%
\definecolor{currentstroke}{rgb}{0.121569,0.466667,0.705882}%
\pgfsetstrokecolor{currentstroke}%
\pgfsetstrokeopacity{0.779408}%
\pgfsetdash{}{0pt}%
\pgfpathmoveto{\pgfqpoint{1.002391in}{1.548748in}}%
\pgfpathcurveto{\pgfqpoint{1.010628in}{1.548748in}}{\pgfqpoint{1.018528in}{1.552021in}}{\pgfqpoint{1.024352in}{1.557844in}}%
\pgfpathcurveto{\pgfqpoint{1.030176in}{1.563668in}}{\pgfqpoint{1.033448in}{1.571568in}}{\pgfqpoint{1.033448in}{1.579805in}}%
\pgfpathcurveto{\pgfqpoint{1.033448in}{1.588041in}}{\pgfqpoint{1.030176in}{1.595941in}}{\pgfqpoint{1.024352in}{1.601765in}}%
\pgfpathcurveto{\pgfqpoint{1.018528in}{1.607589in}}{\pgfqpoint{1.010628in}{1.610861in}}{\pgfqpoint{1.002391in}{1.610861in}}%
\pgfpathcurveto{\pgfqpoint{0.994155in}{1.610861in}}{\pgfqpoint{0.986255in}{1.607589in}}{\pgfqpoint{0.980431in}{1.601765in}}%
\pgfpathcurveto{\pgfqpoint{0.974607in}{1.595941in}}{\pgfqpoint{0.971335in}{1.588041in}}{\pgfqpoint{0.971335in}{1.579805in}}%
\pgfpathcurveto{\pgfqpoint{0.971335in}{1.571568in}}{\pgfqpoint{0.974607in}{1.563668in}}{\pgfqpoint{0.980431in}{1.557844in}}%
\pgfpathcurveto{\pgfqpoint{0.986255in}{1.552021in}}{\pgfqpoint{0.994155in}{1.548748in}}{\pgfqpoint{1.002391in}{1.548748in}}%
\pgfpathclose%
\pgfusepath{stroke,fill}%
\end{pgfscope}%
\begin{pgfscope}%
\pgfpathrectangle{\pgfqpoint{0.100000in}{0.212622in}}{\pgfqpoint{3.696000in}{3.696000in}}%
\pgfusepath{clip}%
\pgfsetbuttcap%
\pgfsetroundjoin%
\definecolor{currentfill}{rgb}{0.121569,0.466667,0.705882}%
\pgfsetfillcolor{currentfill}%
\pgfsetfillopacity{0.779521}%
\pgfsetlinewidth{1.003750pt}%
\definecolor{currentstroke}{rgb}{0.121569,0.466667,0.705882}%
\pgfsetstrokecolor{currentstroke}%
\pgfsetstrokeopacity{0.779521}%
\pgfsetdash{}{0pt}%
\pgfpathmoveto{\pgfqpoint{2.571058in}{2.941156in}}%
\pgfpathcurveto{\pgfqpoint{2.579294in}{2.941156in}}{\pgfqpoint{2.587194in}{2.944429in}}{\pgfqpoint{2.593018in}{2.950253in}}%
\pgfpathcurveto{\pgfqpoint{2.598842in}{2.956077in}}{\pgfqpoint{2.602114in}{2.963977in}}{\pgfqpoint{2.602114in}{2.972213in}}%
\pgfpathcurveto{\pgfqpoint{2.602114in}{2.980449in}}{\pgfqpoint{2.598842in}{2.988349in}}{\pgfqpoint{2.593018in}{2.994173in}}%
\pgfpathcurveto{\pgfqpoint{2.587194in}{2.999997in}}{\pgfqpoint{2.579294in}{3.003269in}}{\pgfqpoint{2.571058in}{3.003269in}}%
\pgfpathcurveto{\pgfqpoint{2.562822in}{3.003269in}}{\pgfqpoint{2.554922in}{2.999997in}}{\pgfqpoint{2.549098in}{2.994173in}}%
\pgfpathcurveto{\pgfqpoint{2.543274in}{2.988349in}}{\pgfqpoint{2.540001in}{2.980449in}}{\pgfqpoint{2.540001in}{2.972213in}}%
\pgfpathcurveto{\pgfqpoint{2.540001in}{2.963977in}}{\pgfqpoint{2.543274in}{2.956077in}}{\pgfqpoint{2.549098in}{2.950253in}}%
\pgfpathcurveto{\pgfqpoint{2.554922in}{2.944429in}}{\pgfqpoint{2.562822in}{2.941156in}}{\pgfqpoint{2.571058in}{2.941156in}}%
\pgfpathclose%
\pgfusepath{stroke,fill}%
\end{pgfscope}%
\begin{pgfscope}%
\pgfpathrectangle{\pgfqpoint{0.100000in}{0.212622in}}{\pgfqpoint{3.696000in}{3.696000in}}%
\pgfusepath{clip}%
\pgfsetbuttcap%
\pgfsetroundjoin%
\definecolor{currentfill}{rgb}{0.121569,0.466667,0.705882}%
\pgfsetfillcolor{currentfill}%
\pgfsetfillopacity{0.779839}%
\pgfsetlinewidth{1.003750pt}%
\definecolor{currentstroke}{rgb}{0.121569,0.466667,0.705882}%
\pgfsetstrokecolor{currentstroke}%
\pgfsetstrokeopacity{0.779839}%
\pgfsetdash{}{0pt}%
\pgfpathmoveto{\pgfqpoint{1.000390in}{1.545554in}}%
\pgfpathcurveto{\pgfqpoint{1.008626in}{1.545554in}}{\pgfqpoint{1.016526in}{1.548827in}}{\pgfqpoint{1.022350in}{1.554651in}}%
\pgfpathcurveto{\pgfqpoint{1.028174in}{1.560475in}}{\pgfqpoint{1.031446in}{1.568375in}}{\pgfqpoint{1.031446in}{1.576611in}}%
\pgfpathcurveto{\pgfqpoint{1.031446in}{1.584847in}}{\pgfqpoint{1.028174in}{1.592747in}}{\pgfqpoint{1.022350in}{1.598571in}}%
\pgfpathcurveto{\pgfqpoint{1.016526in}{1.604395in}}{\pgfqpoint{1.008626in}{1.607667in}}{\pgfqpoint{1.000390in}{1.607667in}}%
\pgfpathcurveto{\pgfqpoint{0.992154in}{1.607667in}}{\pgfqpoint{0.984254in}{1.604395in}}{\pgfqpoint{0.978430in}{1.598571in}}%
\pgfpathcurveto{\pgfqpoint{0.972606in}{1.592747in}}{\pgfqpoint{0.969333in}{1.584847in}}{\pgfqpoint{0.969333in}{1.576611in}}%
\pgfpathcurveto{\pgfqpoint{0.969333in}{1.568375in}}{\pgfqpoint{0.972606in}{1.560475in}}{\pgfqpoint{0.978430in}{1.554651in}}%
\pgfpathcurveto{\pgfqpoint{0.984254in}{1.548827in}}{\pgfqpoint{0.992154in}{1.545554in}}{\pgfqpoint{1.000390in}{1.545554in}}%
\pgfpathclose%
\pgfusepath{stroke,fill}%
\end{pgfscope}%
\begin{pgfscope}%
\pgfpathrectangle{\pgfqpoint{0.100000in}{0.212622in}}{\pgfqpoint{3.696000in}{3.696000in}}%
\pgfusepath{clip}%
\pgfsetbuttcap%
\pgfsetroundjoin%
\definecolor{currentfill}{rgb}{0.121569,0.466667,0.705882}%
\pgfsetfillcolor{currentfill}%
\pgfsetfillopacity{0.779869}%
\pgfsetlinewidth{1.003750pt}%
\definecolor{currentstroke}{rgb}{0.121569,0.466667,0.705882}%
\pgfsetstrokecolor{currentstroke}%
\pgfsetstrokeopacity{0.779869}%
\pgfsetdash{}{0pt}%
\pgfpathmoveto{\pgfqpoint{2.569918in}{2.938653in}}%
\pgfpathcurveto{\pgfqpoint{2.578154in}{2.938653in}}{\pgfqpoint{2.586054in}{2.941926in}}{\pgfqpoint{2.591878in}{2.947749in}}%
\pgfpathcurveto{\pgfqpoint{2.597702in}{2.953573in}}{\pgfqpoint{2.600975in}{2.961473in}}{\pgfqpoint{2.600975in}{2.969710in}}%
\pgfpathcurveto{\pgfqpoint{2.600975in}{2.977946in}}{\pgfqpoint{2.597702in}{2.985846in}}{\pgfqpoint{2.591878in}{2.991670in}}%
\pgfpathcurveto{\pgfqpoint{2.586054in}{2.997494in}}{\pgfqpoint{2.578154in}{3.000766in}}{\pgfqpoint{2.569918in}{3.000766in}}%
\pgfpathcurveto{\pgfqpoint{2.561682in}{3.000766in}}{\pgfqpoint{2.553782in}{2.997494in}}{\pgfqpoint{2.547958in}{2.991670in}}%
\pgfpathcurveto{\pgfqpoint{2.542134in}{2.985846in}}{\pgfqpoint{2.538862in}{2.977946in}}{\pgfqpoint{2.538862in}{2.969710in}}%
\pgfpathcurveto{\pgfqpoint{2.538862in}{2.961473in}}{\pgfqpoint{2.542134in}{2.953573in}}{\pgfqpoint{2.547958in}{2.947749in}}%
\pgfpathcurveto{\pgfqpoint{2.553782in}{2.941926in}}{\pgfqpoint{2.561682in}{2.938653in}}{\pgfqpoint{2.569918in}{2.938653in}}%
\pgfpathclose%
\pgfusepath{stroke,fill}%
\end{pgfscope}%
\begin{pgfscope}%
\pgfpathrectangle{\pgfqpoint{0.100000in}{0.212622in}}{\pgfqpoint{3.696000in}{3.696000in}}%
\pgfusepath{clip}%
\pgfsetbuttcap%
\pgfsetroundjoin%
\definecolor{currentfill}{rgb}{0.121569,0.466667,0.705882}%
\pgfsetfillcolor{currentfill}%
\pgfsetfillopacity{0.780102}%
\pgfsetlinewidth{1.003750pt}%
\definecolor{currentstroke}{rgb}{0.121569,0.466667,0.705882}%
\pgfsetstrokecolor{currentstroke}%
\pgfsetstrokeopacity{0.780102}%
\pgfsetdash{}{0pt}%
\pgfpathmoveto{\pgfqpoint{0.999269in}{1.543940in}}%
\pgfpathcurveto{\pgfqpoint{1.007505in}{1.543940in}}{\pgfqpoint{1.015405in}{1.547213in}}{\pgfqpoint{1.021229in}{1.553037in}}%
\pgfpathcurveto{\pgfqpoint{1.027053in}{1.558861in}}{\pgfqpoint{1.030326in}{1.566761in}}{\pgfqpoint{1.030326in}{1.574997in}}%
\pgfpathcurveto{\pgfqpoint{1.030326in}{1.583233in}}{\pgfqpoint{1.027053in}{1.591133in}}{\pgfqpoint{1.021229in}{1.596957in}}%
\pgfpathcurveto{\pgfqpoint{1.015405in}{1.602781in}}{\pgfqpoint{1.007505in}{1.606053in}}{\pgfqpoint{0.999269in}{1.606053in}}%
\pgfpathcurveto{\pgfqpoint{0.991033in}{1.606053in}}{\pgfqpoint{0.983133in}{1.602781in}}{\pgfqpoint{0.977309in}{1.596957in}}%
\pgfpathcurveto{\pgfqpoint{0.971485in}{1.591133in}}{\pgfqpoint{0.968213in}{1.583233in}}{\pgfqpoint{0.968213in}{1.574997in}}%
\pgfpathcurveto{\pgfqpoint{0.968213in}{1.566761in}}{\pgfqpoint{0.971485in}{1.558861in}}{\pgfqpoint{0.977309in}{1.553037in}}%
\pgfpathcurveto{\pgfqpoint{0.983133in}{1.547213in}}{\pgfqpoint{0.991033in}{1.543940in}}{\pgfqpoint{0.999269in}{1.543940in}}%
\pgfpathclose%
\pgfusepath{stroke,fill}%
\end{pgfscope}%
\begin{pgfscope}%
\pgfpathrectangle{\pgfqpoint{0.100000in}{0.212622in}}{\pgfqpoint{3.696000in}{3.696000in}}%
\pgfusepath{clip}%
\pgfsetbuttcap%
\pgfsetroundjoin%
\definecolor{currentfill}{rgb}{0.121569,0.466667,0.705882}%
\pgfsetfillcolor{currentfill}%
\pgfsetfillopacity{0.780317}%
\pgfsetlinewidth{1.003750pt}%
\definecolor{currentstroke}{rgb}{0.121569,0.466667,0.705882}%
\pgfsetstrokecolor{currentstroke}%
\pgfsetstrokeopacity{0.780317}%
\pgfsetdash{}{0pt}%
\pgfpathmoveto{\pgfqpoint{3.103900in}{2.057579in}}%
\pgfpathcurveto{\pgfqpoint{3.112137in}{2.057579in}}{\pgfqpoint{3.120037in}{2.060851in}}{\pgfqpoint{3.125861in}{2.066675in}}%
\pgfpathcurveto{\pgfqpoint{3.131685in}{2.072499in}}{\pgfqpoint{3.134957in}{2.080399in}}{\pgfqpoint{3.134957in}{2.088635in}}%
\pgfpathcurveto{\pgfqpoint{3.134957in}{2.096871in}}{\pgfqpoint{3.131685in}{2.104771in}}{\pgfqpoint{3.125861in}{2.110595in}}%
\pgfpathcurveto{\pgfqpoint{3.120037in}{2.116419in}}{\pgfqpoint{3.112137in}{2.119692in}}{\pgfqpoint{3.103900in}{2.119692in}}%
\pgfpathcurveto{\pgfqpoint{3.095664in}{2.119692in}}{\pgfqpoint{3.087764in}{2.116419in}}{\pgfqpoint{3.081940in}{2.110595in}}%
\pgfpathcurveto{\pgfqpoint{3.076116in}{2.104771in}}{\pgfqpoint{3.072844in}{2.096871in}}{\pgfqpoint{3.072844in}{2.088635in}}%
\pgfpathcurveto{\pgfqpoint{3.072844in}{2.080399in}}{\pgfqpoint{3.076116in}{2.072499in}}{\pgfqpoint{3.081940in}{2.066675in}}%
\pgfpathcurveto{\pgfqpoint{3.087764in}{2.060851in}}{\pgfqpoint{3.095664in}{2.057579in}}{\pgfqpoint{3.103900in}{2.057579in}}%
\pgfpathclose%
\pgfusepath{stroke,fill}%
\end{pgfscope}%
\begin{pgfscope}%
\pgfpathrectangle{\pgfqpoint{0.100000in}{0.212622in}}{\pgfqpoint{3.696000in}{3.696000in}}%
\pgfusepath{clip}%
\pgfsetbuttcap%
\pgfsetroundjoin%
\definecolor{currentfill}{rgb}{0.121569,0.466667,0.705882}%
\pgfsetfillcolor{currentfill}%
\pgfsetfillopacity{0.780471}%
\pgfsetlinewidth{1.003750pt}%
\definecolor{currentstroke}{rgb}{0.121569,0.466667,0.705882}%
\pgfsetstrokecolor{currentstroke}%
\pgfsetstrokeopacity{0.780471}%
\pgfsetdash{}{0pt}%
\pgfpathmoveto{\pgfqpoint{2.568288in}{2.933590in}}%
\pgfpathcurveto{\pgfqpoint{2.576524in}{2.933590in}}{\pgfqpoint{2.584424in}{2.936863in}}{\pgfqpoint{2.590248in}{2.942687in}}%
\pgfpathcurveto{\pgfqpoint{2.596072in}{2.948511in}}{\pgfqpoint{2.599344in}{2.956411in}}{\pgfqpoint{2.599344in}{2.964647in}}%
\pgfpathcurveto{\pgfqpoint{2.599344in}{2.972883in}}{\pgfqpoint{2.596072in}{2.980783in}}{\pgfqpoint{2.590248in}{2.986607in}}%
\pgfpathcurveto{\pgfqpoint{2.584424in}{2.992431in}}{\pgfqpoint{2.576524in}{2.995703in}}{\pgfqpoint{2.568288in}{2.995703in}}%
\pgfpathcurveto{\pgfqpoint{2.560051in}{2.995703in}}{\pgfqpoint{2.552151in}{2.992431in}}{\pgfqpoint{2.546327in}{2.986607in}}%
\pgfpathcurveto{\pgfqpoint{2.540503in}{2.980783in}}{\pgfqpoint{2.537231in}{2.972883in}}{\pgfqpoint{2.537231in}{2.964647in}}%
\pgfpathcurveto{\pgfqpoint{2.537231in}{2.956411in}}{\pgfqpoint{2.540503in}{2.948511in}}{\pgfqpoint{2.546327in}{2.942687in}}%
\pgfpathcurveto{\pgfqpoint{2.552151in}{2.936863in}}{\pgfqpoint{2.560051in}{2.933590in}}{\pgfqpoint{2.568288in}{2.933590in}}%
\pgfpathclose%
\pgfusepath{stroke,fill}%
\end{pgfscope}%
\begin{pgfscope}%
\pgfpathrectangle{\pgfqpoint{0.100000in}{0.212622in}}{\pgfqpoint{3.696000in}{3.696000in}}%
\pgfusepath{clip}%
\pgfsetbuttcap%
\pgfsetroundjoin%
\definecolor{currentfill}{rgb}{0.121569,0.466667,0.705882}%
\pgfsetfillcolor{currentfill}%
\pgfsetfillopacity{0.780475}%
\pgfsetlinewidth{1.003750pt}%
\definecolor{currentstroke}{rgb}{0.121569,0.466667,0.705882}%
\pgfsetstrokecolor{currentstroke}%
\pgfsetstrokeopacity{0.780475}%
\pgfsetdash{}{0pt}%
\pgfpathmoveto{\pgfqpoint{0.997817in}{1.541210in}}%
\pgfpathcurveto{\pgfqpoint{1.006053in}{1.541210in}}{\pgfqpoint{1.013953in}{1.544483in}}{\pgfqpoint{1.019777in}{1.550307in}}%
\pgfpathcurveto{\pgfqpoint{1.025601in}{1.556130in}}{\pgfqpoint{1.028873in}{1.564031in}}{\pgfqpoint{1.028873in}{1.572267in}}%
\pgfpathcurveto{\pgfqpoint{1.028873in}{1.580503in}}{\pgfqpoint{1.025601in}{1.588403in}}{\pgfqpoint{1.019777in}{1.594227in}}%
\pgfpathcurveto{\pgfqpoint{1.013953in}{1.600051in}}{\pgfqpoint{1.006053in}{1.603323in}}{\pgfqpoint{0.997817in}{1.603323in}}%
\pgfpathcurveto{\pgfqpoint{0.989581in}{1.603323in}}{\pgfqpoint{0.981681in}{1.600051in}}{\pgfqpoint{0.975857in}{1.594227in}}%
\pgfpathcurveto{\pgfqpoint{0.970033in}{1.588403in}}{\pgfqpoint{0.966760in}{1.580503in}}{\pgfqpoint{0.966760in}{1.572267in}}%
\pgfpathcurveto{\pgfqpoint{0.966760in}{1.564031in}}{\pgfqpoint{0.970033in}{1.556130in}}{\pgfqpoint{0.975857in}{1.550307in}}%
\pgfpathcurveto{\pgfqpoint{0.981681in}{1.544483in}}{\pgfqpoint{0.989581in}{1.541210in}}{\pgfqpoint{0.997817in}{1.541210in}}%
\pgfpathclose%
\pgfusepath{stroke,fill}%
\end{pgfscope}%
\begin{pgfscope}%
\pgfpathrectangle{\pgfqpoint{0.100000in}{0.212622in}}{\pgfqpoint{3.696000in}{3.696000in}}%
\pgfusepath{clip}%
\pgfsetbuttcap%
\pgfsetroundjoin%
\definecolor{currentfill}{rgb}{0.121569,0.466667,0.705882}%
\pgfsetfillcolor{currentfill}%
\pgfsetfillopacity{0.780982}%
\pgfsetlinewidth{1.003750pt}%
\definecolor{currentstroke}{rgb}{0.121569,0.466667,0.705882}%
\pgfsetstrokecolor{currentstroke}%
\pgfsetstrokeopacity{0.780982}%
\pgfsetdash{}{0pt}%
\pgfpathmoveto{\pgfqpoint{0.996340in}{1.537221in}}%
\pgfpathcurveto{\pgfqpoint{1.004576in}{1.537221in}}{\pgfqpoint{1.012476in}{1.540494in}}{\pgfqpoint{1.018300in}{1.546317in}}%
\pgfpathcurveto{\pgfqpoint{1.024124in}{1.552141in}}{\pgfqpoint{1.027397in}{1.560041in}}{\pgfqpoint{1.027397in}{1.568278in}}%
\pgfpathcurveto{\pgfqpoint{1.027397in}{1.576514in}}{\pgfqpoint{1.024124in}{1.584414in}}{\pgfqpoint{1.018300in}{1.590238in}}%
\pgfpathcurveto{\pgfqpoint{1.012476in}{1.596062in}}{\pgfqpoint{1.004576in}{1.599334in}}{\pgfqpoint{0.996340in}{1.599334in}}%
\pgfpathcurveto{\pgfqpoint{0.988104in}{1.599334in}}{\pgfqpoint{0.980204in}{1.596062in}}{\pgfqpoint{0.974380in}{1.590238in}}%
\pgfpathcurveto{\pgfqpoint{0.968556in}{1.584414in}}{\pgfqpoint{0.965284in}{1.576514in}}{\pgfqpoint{0.965284in}{1.568278in}}%
\pgfpathcurveto{\pgfqpoint{0.965284in}{1.560041in}}{\pgfqpoint{0.968556in}{1.552141in}}{\pgfqpoint{0.974380in}{1.546317in}}%
\pgfpathcurveto{\pgfqpoint{0.980204in}{1.540494in}}{\pgfqpoint{0.988104in}{1.537221in}}{\pgfqpoint{0.996340in}{1.537221in}}%
\pgfpathclose%
\pgfusepath{stroke,fill}%
\end{pgfscope}%
\begin{pgfscope}%
\pgfpathrectangle{\pgfqpoint{0.100000in}{0.212622in}}{\pgfqpoint{3.696000in}{3.696000in}}%
\pgfusepath{clip}%
\pgfsetbuttcap%
\pgfsetroundjoin%
\definecolor{currentfill}{rgb}{0.121569,0.466667,0.705882}%
\pgfsetfillcolor{currentfill}%
\pgfsetfillopacity{0.781497}%
\pgfsetlinewidth{1.003750pt}%
\definecolor{currentstroke}{rgb}{0.121569,0.466667,0.705882}%
\pgfsetstrokecolor{currentstroke}%
\pgfsetstrokeopacity{0.781497}%
\pgfsetdash{}{0pt}%
\pgfpathmoveto{\pgfqpoint{0.995465in}{1.532281in}}%
\pgfpathcurveto{\pgfqpoint{1.003701in}{1.532281in}}{\pgfqpoint{1.011602in}{1.535553in}}{\pgfqpoint{1.017425in}{1.541377in}}%
\pgfpathcurveto{\pgfqpoint{1.023249in}{1.547201in}}{\pgfqpoint{1.026522in}{1.555101in}}{\pgfqpoint{1.026522in}{1.563337in}}%
\pgfpathcurveto{\pgfqpoint{1.026522in}{1.571574in}}{\pgfqpoint{1.023249in}{1.579474in}}{\pgfqpoint{1.017425in}{1.585298in}}%
\pgfpathcurveto{\pgfqpoint{1.011602in}{1.591122in}}{\pgfqpoint{1.003701in}{1.594394in}}{\pgfqpoint{0.995465in}{1.594394in}}%
\pgfpathcurveto{\pgfqpoint{0.987229in}{1.594394in}}{\pgfqpoint{0.979329in}{1.591122in}}{\pgfqpoint{0.973505in}{1.585298in}}%
\pgfpathcurveto{\pgfqpoint{0.967681in}{1.579474in}}{\pgfqpoint{0.964409in}{1.571574in}}{\pgfqpoint{0.964409in}{1.563337in}}%
\pgfpathcurveto{\pgfqpoint{0.964409in}{1.555101in}}{\pgfqpoint{0.967681in}{1.547201in}}{\pgfqpoint{0.973505in}{1.541377in}}%
\pgfpathcurveto{\pgfqpoint{0.979329in}{1.535553in}}{\pgfqpoint{0.987229in}{1.532281in}}{\pgfqpoint{0.995465in}{1.532281in}}%
\pgfpathclose%
\pgfusepath{stroke,fill}%
\end{pgfscope}%
\begin{pgfscope}%
\pgfpathrectangle{\pgfqpoint{0.100000in}{0.212622in}}{\pgfqpoint{3.696000in}{3.696000in}}%
\pgfusepath{clip}%
\pgfsetbuttcap%
\pgfsetroundjoin%
\definecolor{currentfill}{rgb}{0.121569,0.466667,0.705882}%
\pgfsetfillcolor{currentfill}%
\pgfsetfillopacity{0.781573}%
\pgfsetlinewidth{1.003750pt}%
\definecolor{currentstroke}{rgb}{0.121569,0.466667,0.705882}%
\pgfsetstrokecolor{currentstroke}%
\pgfsetstrokeopacity{0.781573}%
\pgfsetdash{}{0pt}%
\pgfpathmoveto{\pgfqpoint{2.565662in}{2.924236in}}%
\pgfpathcurveto{\pgfqpoint{2.573898in}{2.924236in}}{\pgfqpoint{2.581798in}{2.927509in}}{\pgfqpoint{2.587622in}{2.933333in}}%
\pgfpathcurveto{\pgfqpoint{2.593446in}{2.939157in}}{\pgfqpoint{2.596718in}{2.947057in}}{\pgfqpoint{2.596718in}{2.955293in}}%
\pgfpathcurveto{\pgfqpoint{2.596718in}{2.963529in}}{\pgfqpoint{2.593446in}{2.971429in}}{\pgfqpoint{2.587622in}{2.977253in}}%
\pgfpathcurveto{\pgfqpoint{2.581798in}{2.983077in}}{\pgfqpoint{2.573898in}{2.986349in}}{\pgfqpoint{2.565662in}{2.986349in}}%
\pgfpathcurveto{\pgfqpoint{2.557425in}{2.986349in}}{\pgfqpoint{2.549525in}{2.983077in}}{\pgfqpoint{2.543701in}{2.977253in}}%
\pgfpathcurveto{\pgfqpoint{2.537877in}{2.971429in}}{\pgfqpoint{2.534605in}{2.963529in}}{\pgfqpoint{2.534605in}{2.955293in}}%
\pgfpathcurveto{\pgfqpoint{2.534605in}{2.947057in}}{\pgfqpoint{2.537877in}{2.939157in}}{\pgfqpoint{2.543701in}{2.933333in}}%
\pgfpathcurveto{\pgfqpoint{2.549525in}{2.927509in}}{\pgfqpoint{2.557425in}{2.924236in}}{\pgfqpoint{2.565662in}{2.924236in}}%
\pgfpathclose%
\pgfusepath{stroke,fill}%
\end{pgfscope}%
\begin{pgfscope}%
\pgfpathrectangle{\pgfqpoint{0.100000in}{0.212622in}}{\pgfqpoint{3.696000in}{3.696000in}}%
\pgfusepath{clip}%
\pgfsetbuttcap%
\pgfsetroundjoin%
\definecolor{currentfill}{rgb}{0.121569,0.466667,0.705882}%
\pgfsetfillcolor{currentfill}%
\pgfsetfillopacity{0.782086}%
\pgfsetlinewidth{1.003750pt}%
\definecolor{currentstroke}{rgb}{0.121569,0.466667,0.705882}%
\pgfsetstrokecolor{currentstroke}%
\pgfsetstrokeopacity{0.782086}%
\pgfsetdash{}{0pt}%
\pgfpathmoveto{\pgfqpoint{0.994118in}{1.526963in}}%
\pgfpathcurveto{\pgfqpoint{1.002354in}{1.526963in}}{\pgfqpoint{1.010254in}{1.530235in}}{\pgfqpoint{1.016078in}{1.536059in}}%
\pgfpathcurveto{\pgfqpoint{1.021902in}{1.541883in}}{\pgfqpoint{1.025175in}{1.549783in}}{\pgfqpoint{1.025175in}{1.558019in}}%
\pgfpathcurveto{\pgfqpoint{1.025175in}{1.566256in}}{\pgfqpoint{1.021902in}{1.574156in}}{\pgfqpoint{1.016078in}{1.579980in}}%
\pgfpathcurveto{\pgfqpoint{1.010254in}{1.585804in}}{\pgfqpoint{1.002354in}{1.589076in}}{\pgfqpoint{0.994118in}{1.589076in}}%
\pgfpathcurveto{\pgfqpoint{0.985882in}{1.589076in}}{\pgfqpoint{0.977982in}{1.585804in}}{\pgfqpoint{0.972158in}{1.579980in}}%
\pgfpathcurveto{\pgfqpoint{0.966334in}{1.574156in}}{\pgfqpoint{0.963062in}{1.566256in}}{\pgfqpoint{0.963062in}{1.558019in}}%
\pgfpathcurveto{\pgfqpoint{0.963062in}{1.549783in}}{\pgfqpoint{0.966334in}{1.541883in}}{\pgfqpoint{0.972158in}{1.536059in}}%
\pgfpathcurveto{\pgfqpoint{0.977982in}{1.530235in}}{\pgfqpoint{0.985882in}{1.526963in}}{\pgfqpoint{0.994118in}{1.526963in}}%
\pgfpathclose%
\pgfusepath{stroke,fill}%
\end{pgfscope}%
\begin{pgfscope}%
\pgfpathrectangle{\pgfqpoint{0.100000in}{0.212622in}}{\pgfqpoint{3.696000in}{3.696000in}}%
\pgfusepath{clip}%
\pgfsetbuttcap%
\pgfsetroundjoin%
\definecolor{currentfill}{rgb}{0.121569,0.466667,0.705882}%
\pgfsetfillcolor{currentfill}%
\pgfsetfillopacity{0.782388}%
\pgfsetlinewidth{1.003750pt}%
\definecolor{currentstroke}{rgb}{0.121569,0.466667,0.705882}%
\pgfsetstrokecolor{currentstroke}%
\pgfsetstrokeopacity{0.782388}%
\pgfsetdash{}{0pt}%
\pgfpathmoveto{\pgfqpoint{0.992839in}{1.524345in}}%
\pgfpathcurveto{\pgfqpoint{1.001076in}{1.524345in}}{\pgfqpoint{1.008976in}{1.527617in}}{\pgfqpoint{1.014800in}{1.533441in}}%
\pgfpathcurveto{\pgfqpoint{1.020624in}{1.539265in}}{\pgfqpoint{1.023896in}{1.547165in}}{\pgfqpoint{1.023896in}{1.555401in}}%
\pgfpathcurveto{\pgfqpoint{1.023896in}{1.563638in}}{\pgfqpoint{1.020624in}{1.571538in}}{\pgfqpoint{1.014800in}{1.577362in}}%
\pgfpathcurveto{\pgfqpoint{1.008976in}{1.583186in}}{\pgfqpoint{1.001076in}{1.586458in}}{\pgfqpoint{0.992839in}{1.586458in}}%
\pgfpathcurveto{\pgfqpoint{0.984603in}{1.586458in}}{\pgfqpoint{0.976703in}{1.583186in}}{\pgfqpoint{0.970879in}{1.577362in}}%
\pgfpathcurveto{\pgfqpoint{0.965055in}{1.571538in}}{\pgfqpoint{0.961783in}{1.563638in}}{\pgfqpoint{0.961783in}{1.555401in}}%
\pgfpathcurveto{\pgfqpoint{0.961783in}{1.547165in}}{\pgfqpoint{0.965055in}{1.539265in}}{\pgfqpoint{0.970879in}{1.533441in}}%
\pgfpathcurveto{\pgfqpoint{0.976703in}{1.527617in}}{\pgfqpoint{0.984603in}{1.524345in}}{\pgfqpoint{0.992839in}{1.524345in}}%
\pgfpathclose%
\pgfusepath{stroke,fill}%
\end{pgfscope}%
\begin{pgfscope}%
\pgfpathrectangle{\pgfqpoint{0.100000in}{0.212622in}}{\pgfqpoint{3.696000in}{3.696000in}}%
\pgfusepath{clip}%
\pgfsetbuttcap%
\pgfsetroundjoin%
\definecolor{currentfill}{rgb}{0.121569,0.466667,0.705882}%
\pgfsetfillcolor{currentfill}%
\pgfsetfillopacity{0.782485}%
\pgfsetlinewidth{1.003750pt}%
\definecolor{currentstroke}{rgb}{0.121569,0.466667,0.705882}%
\pgfsetstrokecolor{currentstroke}%
\pgfsetstrokeopacity{0.782485}%
\pgfsetdash{}{0pt}%
\pgfpathmoveto{\pgfqpoint{2.561831in}{2.917518in}}%
\pgfpathcurveto{\pgfqpoint{2.570067in}{2.917518in}}{\pgfqpoint{2.577967in}{2.920790in}}{\pgfqpoint{2.583791in}{2.926614in}}%
\pgfpathcurveto{\pgfqpoint{2.589615in}{2.932438in}}{\pgfqpoint{2.592887in}{2.940338in}}{\pgfqpoint{2.592887in}{2.948574in}}%
\pgfpathcurveto{\pgfqpoint{2.592887in}{2.956811in}}{\pgfqpoint{2.589615in}{2.964711in}}{\pgfqpoint{2.583791in}{2.970535in}}%
\pgfpathcurveto{\pgfqpoint{2.577967in}{2.976359in}}{\pgfqpoint{2.570067in}{2.979631in}}{\pgfqpoint{2.561831in}{2.979631in}}%
\pgfpathcurveto{\pgfqpoint{2.553595in}{2.979631in}}{\pgfqpoint{2.545695in}{2.976359in}}{\pgfqpoint{2.539871in}{2.970535in}}%
\pgfpathcurveto{\pgfqpoint{2.534047in}{2.964711in}}{\pgfqpoint{2.530774in}{2.956811in}}{\pgfqpoint{2.530774in}{2.948574in}}%
\pgfpathcurveto{\pgfqpoint{2.530774in}{2.940338in}}{\pgfqpoint{2.534047in}{2.932438in}}{\pgfqpoint{2.539871in}{2.926614in}}%
\pgfpathcurveto{\pgfqpoint{2.545695in}{2.920790in}}{\pgfqpoint{2.553595in}{2.917518in}}{\pgfqpoint{2.561831in}{2.917518in}}%
\pgfpathclose%
\pgfusepath{stroke,fill}%
\end{pgfscope}%
\begin{pgfscope}%
\pgfpathrectangle{\pgfqpoint{0.100000in}{0.212622in}}{\pgfqpoint{3.696000in}{3.696000in}}%
\pgfusepath{clip}%
\pgfsetbuttcap%
\pgfsetroundjoin%
\definecolor{currentfill}{rgb}{0.121569,0.466667,0.705882}%
\pgfsetfillcolor{currentfill}%
\pgfsetfillopacity{0.782758}%
\pgfsetlinewidth{1.003750pt}%
\definecolor{currentstroke}{rgb}{0.121569,0.466667,0.705882}%
\pgfsetstrokecolor{currentstroke}%
\pgfsetstrokeopacity{0.782758}%
\pgfsetdash{}{0pt}%
\pgfpathmoveto{\pgfqpoint{0.991241in}{1.521465in}}%
\pgfpathcurveto{\pgfqpoint{0.999478in}{1.521465in}}{\pgfqpoint{1.007378in}{1.524737in}}{\pgfqpoint{1.013202in}{1.530561in}}%
\pgfpathcurveto{\pgfqpoint{1.019026in}{1.536385in}}{\pgfqpoint{1.022298in}{1.544285in}}{\pgfqpoint{1.022298in}{1.552521in}}%
\pgfpathcurveto{\pgfqpoint{1.022298in}{1.560757in}}{\pgfqpoint{1.019026in}{1.568657in}}{\pgfqpoint{1.013202in}{1.574481in}}%
\pgfpathcurveto{\pgfqpoint{1.007378in}{1.580305in}}{\pgfqpoint{0.999478in}{1.583578in}}{\pgfqpoint{0.991241in}{1.583578in}}%
\pgfpathcurveto{\pgfqpoint{0.983005in}{1.583578in}}{\pgfqpoint{0.975105in}{1.580305in}}{\pgfqpoint{0.969281in}{1.574481in}}%
\pgfpathcurveto{\pgfqpoint{0.963457in}{1.568657in}}{\pgfqpoint{0.960185in}{1.560757in}}{\pgfqpoint{0.960185in}{1.552521in}}%
\pgfpathcurveto{\pgfqpoint{0.960185in}{1.544285in}}{\pgfqpoint{0.963457in}{1.536385in}}{\pgfqpoint{0.969281in}{1.530561in}}%
\pgfpathcurveto{\pgfqpoint{0.975105in}{1.524737in}}{\pgfqpoint{0.983005in}{1.521465in}}{\pgfqpoint{0.991241in}{1.521465in}}%
\pgfpathclose%
\pgfusepath{stroke,fill}%
\end{pgfscope}%
\begin{pgfscope}%
\pgfpathrectangle{\pgfqpoint{0.100000in}{0.212622in}}{\pgfqpoint{3.696000in}{3.696000in}}%
\pgfusepath{clip}%
\pgfsetbuttcap%
\pgfsetroundjoin%
\definecolor{currentfill}{rgb}{0.121569,0.466667,0.705882}%
\pgfsetfillcolor{currentfill}%
\pgfsetfillopacity{0.783077}%
\pgfsetlinewidth{1.003750pt}%
\definecolor{currentstroke}{rgb}{0.121569,0.466667,0.705882}%
\pgfsetstrokecolor{currentstroke}%
\pgfsetstrokeopacity{0.783077}%
\pgfsetdash{}{0pt}%
\pgfpathmoveto{\pgfqpoint{2.557750in}{2.912044in}}%
\pgfpathcurveto{\pgfqpoint{2.565986in}{2.912044in}}{\pgfqpoint{2.573886in}{2.915316in}}{\pgfqpoint{2.579710in}{2.921140in}}%
\pgfpathcurveto{\pgfqpoint{2.585534in}{2.926964in}}{\pgfqpoint{2.588806in}{2.934864in}}{\pgfqpoint{2.588806in}{2.943100in}}%
\pgfpathcurveto{\pgfqpoint{2.588806in}{2.951336in}}{\pgfqpoint{2.585534in}{2.959236in}}{\pgfqpoint{2.579710in}{2.965060in}}%
\pgfpathcurveto{\pgfqpoint{2.573886in}{2.970884in}}{\pgfqpoint{2.565986in}{2.974157in}}{\pgfqpoint{2.557750in}{2.974157in}}%
\pgfpathcurveto{\pgfqpoint{2.549513in}{2.974157in}}{\pgfqpoint{2.541613in}{2.970884in}}{\pgfqpoint{2.535789in}{2.965060in}}%
\pgfpathcurveto{\pgfqpoint{2.529965in}{2.959236in}}{\pgfqpoint{2.526693in}{2.951336in}}{\pgfqpoint{2.526693in}{2.943100in}}%
\pgfpathcurveto{\pgfqpoint{2.526693in}{2.934864in}}{\pgfqpoint{2.529965in}{2.926964in}}{\pgfqpoint{2.535789in}{2.921140in}}%
\pgfpathcurveto{\pgfqpoint{2.541613in}{2.915316in}}{\pgfqpoint{2.549513in}{2.912044in}}{\pgfqpoint{2.557750in}{2.912044in}}%
\pgfpathclose%
\pgfusepath{stroke,fill}%
\end{pgfscope}%
\begin{pgfscope}%
\pgfpathrectangle{\pgfqpoint{0.100000in}{0.212622in}}{\pgfqpoint{3.696000in}{3.696000in}}%
\pgfusepath{clip}%
\pgfsetbuttcap%
\pgfsetroundjoin%
\definecolor{currentfill}{rgb}{0.121569,0.466667,0.705882}%
\pgfsetfillcolor{currentfill}%
\pgfsetfillopacity{0.783232}%
\pgfsetlinewidth{1.003750pt}%
\definecolor{currentstroke}{rgb}{0.121569,0.466667,0.705882}%
\pgfsetstrokecolor{currentstroke}%
\pgfsetstrokeopacity{0.783232}%
\pgfsetdash{}{0pt}%
\pgfpathmoveto{\pgfqpoint{0.989305in}{1.517808in}}%
\pgfpathcurveto{\pgfqpoint{0.997541in}{1.517808in}}{\pgfqpoint{1.005441in}{1.521080in}}{\pgfqpoint{1.011265in}{1.526904in}}%
\pgfpathcurveto{\pgfqpoint{1.017089in}{1.532728in}}{\pgfqpoint{1.020361in}{1.540628in}}{\pgfqpoint{1.020361in}{1.548865in}}%
\pgfpathcurveto{\pgfqpoint{1.020361in}{1.557101in}}{\pgfqpoint{1.017089in}{1.565001in}}{\pgfqpoint{1.011265in}{1.570825in}}%
\pgfpathcurveto{\pgfqpoint{1.005441in}{1.576649in}}{\pgfqpoint{0.997541in}{1.579921in}}{\pgfqpoint{0.989305in}{1.579921in}}%
\pgfpathcurveto{\pgfqpoint{0.981069in}{1.579921in}}{\pgfqpoint{0.973169in}{1.576649in}}{\pgfqpoint{0.967345in}{1.570825in}}%
\pgfpathcurveto{\pgfqpoint{0.961521in}{1.565001in}}{\pgfqpoint{0.958248in}{1.557101in}}{\pgfqpoint{0.958248in}{1.548865in}}%
\pgfpathcurveto{\pgfqpoint{0.958248in}{1.540628in}}{\pgfqpoint{0.961521in}{1.532728in}}{\pgfqpoint{0.967345in}{1.526904in}}%
\pgfpathcurveto{\pgfqpoint{0.973169in}{1.521080in}}{\pgfqpoint{0.981069in}{1.517808in}}{\pgfqpoint{0.989305in}{1.517808in}}%
\pgfpathclose%
\pgfusepath{stroke,fill}%
\end{pgfscope}%
\begin{pgfscope}%
\pgfpathrectangle{\pgfqpoint{0.100000in}{0.212622in}}{\pgfqpoint{3.696000in}{3.696000in}}%
\pgfusepath{clip}%
\pgfsetbuttcap%
\pgfsetroundjoin%
\definecolor{currentfill}{rgb}{0.121569,0.466667,0.705882}%
\pgfsetfillcolor{currentfill}%
\pgfsetfillopacity{0.783503}%
\pgfsetlinewidth{1.003750pt}%
\definecolor{currentstroke}{rgb}{0.121569,0.466667,0.705882}%
\pgfsetstrokecolor{currentstroke}%
\pgfsetstrokeopacity{0.783503}%
\pgfsetdash{}{0pt}%
\pgfpathmoveto{\pgfqpoint{0.988258in}{1.515816in}}%
\pgfpathcurveto{\pgfqpoint{0.996495in}{1.515816in}}{\pgfqpoint{1.004395in}{1.519088in}}{\pgfqpoint{1.010219in}{1.524912in}}%
\pgfpathcurveto{\pgfqpoint{1.016043in}{1.530736in}}{\pgfqpoint{1.019315in}{1.538636in}}{\pgfqpoint{1.019315in}{1.546873in}}%
\pgfpathcurveto{\pgfqpoint{1.019315in}{1.555109in}}{\pgfqpoint{1.016043in}{1.563009in}}{\pgfqpoint{1.010219in}{1.568833in}}%
\pgfpathcurveto{\pgfqpoint{1.004395in}{1.574657in}}{\pgfqpoint{0.996495in}{1.577929in}}{\pgfqpoint{0.988258in}{1.577929in}}%
\pgfpathcurveto{\pgfqpoint{0.980022in}{1.577929in}}{\pgfqpoint{0.972122in}{1.574657in}}{\pgfqpoint{0.966298in}{1.568833in}}%
\pgfpathcurveto{\pgfqpoint{0.960474in}{1.563009in}}{\pgfqpoint{0.957202in}{1.555109in}}{\pgfqpoint{0.957202in}{1.546873in}}%
\pgfpathcurveto{\pgfqpoint{0.957202in}{1.538636in}}{\pgfqpoint{0.960474in}{1.530736in}}{\pgfqpoint{0.966298in}{1.524912in}}%
\pgfpathcurveto{\pgfqpoint{0.972122in}{1.519088in}}{\pgfqpoint{0.980022in}{1.515816in}}{\pgfqpoint{0.988258in}{1.515816in}}%
\pgfpathclose%
\pgfusepath{stroke,fill}%
\end{pgfscope}%
\begin{pgfscope}%
\pgfpathrectangle{\pgfqpoint{0.100000in}{0.212622in}}{\pgfqpoint{3.696000in}{3.696000in}}%
\pgfusepath{clip}%
\pgfsetbuttcap%
\pgfsetroundjoin%
\definecolor{currentfill}{rgb}{0.121569,0.466667,0.705882}%
\pgfsetfillcolor{currentfill}%
\pgfsetfillopacity{0.783649}%
\pgfsetlinewidth{1.003750pt}%
\definecolor{currentstroke}{rgb}{0.121569,0.466667,0.705882}%
\pgfsetstrokecolor{currentstroke}%
\pgfsetstrokeopacity{0.783649}%
\pgfsetdash{}{0pt}%
\pgfpathmoveto{\pgfqpoint{0.987706in}{1.514672in}}%
\pgfpathcurveto{\pgfqpoint{0.995942in}{1.514672in}}{\pgfqpoint{1.003842in}{1.517945in}}{\pgfqpoint{1.009666in}{1.523769in}}%
\pgfpathcurveto{\pgfqpoint{1.015490in}{1.529593in}}{\pgfqpoint{1.018762in}{1.537493in}}{\pgfqpoint{1.018762in}{1.545729in}}%
\pgfpathcurveto{\pgfqpoint{1.018762in}{1.553965in}}{\pgfqpoint{1.015490in}{1.561865in}}{\pgfqpoint{1.009666in}{1.567689in}}%
\pgfpathcurveto{\pgfqpoint{1.003842in}{1.573513in}}{\pgfqpoint{0.995942in}{1.576785in}}{\pgfqpoint{0.987706in}{1.576785in}}%
\pgfpathcurveto{\pgfqpoint{0.979470in}{1.576785in}}{\pgfqpoint{0.971569in}{1.573513in}}{\pgfqpoint{0.965746in}{1.567689in}}%
\pgfpathcurveto{\pgfqpoint{0.959922in}{1.561865in}}{\pgfqpoint{0.956649in}{1.553965in}}{\pgfqpoint{0.956649in}{1.545729in}}%
\pgfpathcurveto{\pgfqpoint{0.956649in}{1.537493in}}{\pgfqpoint{0.959922in}{1.529593in}}{\pgfqpoint{0.965746in}{1.523769in}}%
\pgfpathcurveto{\pgfqpoint{0.971569in}{1.517945in}}{\pgfqpoint{0.979470in}{1.514672in}}{\pgfqpoint{0.987706in}{1.514672in}}%
\pgfpathclose%
\pgfusepath{stroke,fill}%
\end{pgfscope}%
\begin{pgfscope}%
\pgfpathrectangle{\pgfqpoint{0.100000in}{0.212622in}}{\pgfqpoint{3.696000in}{3.696000in}}%
\pgfusepath{clip}%
\pgfsetbuttcap%
\pgfsetroundjoin%
\definecolor{currentfill}{rgb}{0.121569,0.466667,0.705882}%
\pgfsetfillcolor{currentfill}%
\pgfsetfillopacity{0.783726}%
\pgfsetlinewidth{1.003750pt}%
\definecolor{currentstroke}{rgb}{0.121569,0.466667,0.705882}%
\pgfsetstrokecolor{currentstroke}%
\pgfsetstrokeopacity{0.783726}%
\pgfsetdash{}{0pt}%
\pgfpathmoveto{\pgfqpoint{0.987403in}{1.514031in}}%
\pgfpathcurveto{\pgfqpoint{0.995639in}{1.514031in}}{\pgfqpoint{1.003539in}{1.517304in}}{\pgfqpoint{1.009363in}{1.523128in}}%
\pgfpathcurveto{\pgfqpoint{1.015187in}{1.528952in}}{\pgfqpoint{1.018459in}{1.536852in}}{\pgfqpoint{1.018459in}{1.545088in}}%
\pgfpathcurveto{\pgfqpoint{1.018459in}{1.553324in}}{\pgfqpoint{1.015187in}{1.561224in}}{\pgfqpoint{1.009363in}{1.567048in}}%
\pgfpathcurveto{\pgfqpoint{1.003539in}{1.572872in}}{\pgfqpoint{0.995639in}{1.576144in}}{\pgfqpoint{0.987403in}{1.576144in}}%
\pgfpathcurveto{\pgfqpoint{0.979167in}{1.576144in}}{\pgfqpoint{0.971267in}{1.572872in}}{\pgfqpoint{0.965443in}{1.567048in}}%
\pgfpathcurveto{\pgfqpoint{0.959619in}{1.561224in}}{\pgfqpoint{0.956346in}{1.553324in}}{\pgfqpoint{0.956346in}{1.545088in}}%
\pgfpathcurveto{\pgfqpoint{0.956346in}{1.536852in}}{\pgfqpoint{0.959619in}{1.528952in}}{\pgfqpoint{0.965443in}{1.523128in}}%
\pgfpathcurveto{\pgfqpoint{0.971267in}{1.517304in}}{\pgfqpoint{0.979167in}{1.514031in}}{\pgfqpoint{0.987403in}{1.514031in}}%
\pgfpathclose%
\pgfusepath{stroke,fill}%
\end{pgfscope}%
\begin{pgfscope}%
\pgfpathrectangle{\pgfqpoint{0.100000in}{0.212622in}}{\pgfqpoint{3.696000in}{3.696000in}}%
\pgfusepath{clip}%
\pgfsetbuttcap%
\pgfsetroundjoin%
\definecolor{currentfill}{rgb}{0.121569,0.466667,0.705882}%
\pgfsetfillcolor{currentfill}%
\pgfsetfillopacity{0.783772}%
\pgfsetlinewidth{1.003750pt}%
\definecolor{currentstroke}{rgb}{0.121569,0.466667,0.705882}%
\pgfsetstrokecolor{currentstroke}%
\pgfsetstrokeopacity{0.783772}%
\pgfsetdash{}{0pt}%
\pgfpathmoveto{\pgfqpoint{0.987253in}{1.513679in}}%
\pgfpathcurveto{\pgfqpoint{0.995489in}{1.513679in}}{\pgfqpoint{1.003389in}{1.516951in}}{\pgfqpoint{1.009213in}{1.522775in}}%
\pgfpathcurveto{\pgfqpoint{1.015037in}{1.528599in}}{\pgfqpoint{1.018309in}{1.536499in}}{\pgfqpoint{1.018309in}{1.544736in}}%
\pgfpathcurveto{\pgfqpoint{1.018309in}{1.552972in}}{\pgfqpoint{1.015037in}{1.560872in}}{\pgfqpoint{1.009213in}{1.566696in}}%
\pgfpathcurveto{\pgfqpoint{1.003389in}{1.572520in}}{\pgfqpoint{0.995489in}{1.575792in}}{\pgfqpoint{0.987253in}{1.575792in}}%
\pgfpathcurveto{\pgfqpoint{0.979016in}{1.575792in}}{\pgfqpoint{0.971116in}{1.572520in}}{\pgfqpoint{0.965292in}{1.566696in}}%
\pgfpathcurveto{\pgfqpoint{0.959468in}{1.560872in}}{\pgfqpoint{0.956196in}{1.552972in}}{\pgfqpoint{0.956196in}{1.544736in}}%
\pgfpathcurveto{\pgfqpoint{0.956196in}{1.536499in}}{\pgfqpoint{0.959468in}{1.528599in}}{\pgfqpoint{0.965292in}{1.522775in}}%
\pgfpathcurveto{\pgfqpoint{0.971116in}{1.516951in}}{\pgfqpoint{0.979016in}{1.513679in}}{\pgfqpoint{0.987253in}{1.513679in}}%
\pgfpathclose%
\pgfusepath{stroke,fill}%
\end{pgfscope}%
\begin{pgfscope}%
\pgfpathrectangle{\pgfqpoint{0.100000in}{0.212622in}}{\pgfqpoint{3.696000in}{3.696000in}}%
\pgfusepath{clip}%
\pgfsetbuttcap%
\pgfsetroundjoin%
\definecolor{currentfill}{rgb}{0.121569,0.466667,0.705882}%
\pgfsetfillcolor{currentfill}%
\pgfsetfillopacity{0.783798}%
\pgfsetlinewidth{1.003750pt}%
\definecolor{currentstroke}{rgb}{0.121569,0.466667,0.705882}%
\pgfsetstrokecolor{currentstroke}%
\pgfsetstrokeopacity{0.783798}%
\pgfsetdash{}{0pt}%
\pgfpathmoveto{\pgfqpoint{0.987164in}{1.513491in}}%
\pgfpathcurveto{\pgfqpoint{0.995400in}{1.513491in}}{\pgfqpoint{1.003300in}{1.516763in}}{\pgfqpoint{1.009124in}{1.522587in}}%
\pgfpathcurveto{\pgfqpoint{1.014948in}{1.528411in}}{\pgfqpoint{1.018220in}{1.536311in}}{\pgfqpoint{1.018220in}{1.544547in}}%
\pgfpathcurveto{\pgfqpoint{1.018220in}{1.552783in}}{\pgfqpoint{1.014948in}{1.560684in}}{\pgfqpoint{1.009124in}{1.566507in}}%
\pgfpathcurveto{\pgfqpoint{1.003300in}{1.572331in}}{\pgfqpoint{0.995400in}{1.575604in}}{\pgfqpoint{0.987164in}{1.575604in}}%
\pgfpathcurveto{\pgfqpoint{0.978927in}{1.575604in}}{\pgfqpoint{0.971027in}{1.572331in}}{\pgfqpoint{0.965203in}{1.566507in}}%
\pgfpathcurveto{\pgfqpoint{0.959379in}{1.560684in}}{\pgfqpoint{0.956107in}{1.552783in}}{\pgfqpoint{0.956107in}{1.544547in}}%
\pgfpathcurveto{\pgfqpoint{0.956107in}{1.536311in}}{\pgfqpoint{0.959379in}{1.528411in}}{\pgfqpoint{0.965203in}{1.522587in}}%
\pgfpathcurveto{\pgfqpoint{0.971027in}{1.516763in}}{\pgfqpoint{0.978927in}{1.513491in}}{\pgfqpoint{0.987164in}{1.513491in}}%
\pgfpathclose%
\pgfusepath{stroke,fill}%
\end{pgfscope}%
\begin{pgfscope}%
\pgfpathrectangle{\pgfqpoint{0.100000in}{0.212622in}}{\pgfqpoint{3.696000in}{3.696000in}}%
\pgfusepath{clip}%
\pgfsetbuttcap%
\pgfsetroundjoin%
\definecolor{currentfill}{rgb}{0.121569,0.466667,0.705882}%
\pgfsetfillcolor{currentfill}%
\pgfsetfillopacity{0.783812}%
\pgfsetlinewidth{1.003750pt}%
\definecolor{currentstroke}{rgb}{0.121569,0.466667,0.705882}%
\pgfsetstrokecolor{currentstroke}%
\pgfsetstrokeopacity{0.783812}%
\pgfsetdash{}{0pt}%
\pgfpathmoveto{\pgfqpoint{0.987113in}{1.513390in}}%
\pgfpathcurveto{\pgfqpoint{0.995349in}{1.513390in}}{\pgfqpoint{1.003249in}{1.516662in}}{\pgfqpoint{1.009073in}{1.522486in}}%
\pgfpathcurveto{\pgfqpoint{1.014897in}{1.528310in}}{\pgfqpoint{1.018169in}{1.536210in}}{\pgfqpoint{1.018169in}{1.544446in}}%
\pgfpathcurveto{\pgfqpoint{1.018169in}{1.552682in}}{\pgfqpoint{1.014897in}{1.560582in}}{\pgfqpoint{1.009073in}{1.566406in}}%
\pgfpathcurveto{\pgfqpoint{1.003249in}{1.572230in}}{\pgfqpoint{0.995349in}{1.575503in}}{\pgfqpoint{0.987113in}{1.575503in}}%
\pgfpathcurveto{\pgfqpoint{0.978877in}{1.575503in}}{\pgfqpoint{0.970976in}{1.572230in}}{\pgfqpoint{0.965153in}{1.566406in}}%
\pgfpathcurveto{\pgfqpoint{0.959329in}{1.560582in}}{\pgfqpoint{0.956056in}{1.552682in}}{\pgfqpoint{0.956056in}{1.544446in}}%
\pgfpathcurveto{\pgfqpoint{0.956056in}{1.536210in}}{\pgfqpoint{0.959329in}{1.528310in}}{\pgfqpoint{0.965153in}{1.522486in}}%
\pgfpathcurveto{\pgfqpoint{0.970976in}{1.516662in}}{\pgfqpoint{0.978877in}{1.513390in}}{\pgfqpoint{0.987113in}{1.513390in}}%
\pgfpathclose%
\pgfusepath{stroke,fill}%
\end{pgfscope}%
\begin{pgfscope}%
\pgfpathrectangle{\pgfqpoint{0.100000in}{0.212622in}}{\pgfqpoint{3.696000in}{3.696000in}}%
\pgfusepath{clip}%
\pgfsetbuttcap%
\pgfsetroundjoin%
\definecolor{currentfill}{rgb}{0.121569,0.466667,0.705882}%
\pgfsetfillcolor{currentfill}%
\pgfsetfillopacity{0.783819}%
\pgfsetlinewidth{1.003750pt}%
\definecolor{currentstroke}{rgb}{0.121569,0.466667,0.705882}%
\pgfsetstrokecolor{currentstroke}%
\pgfsetstrokeopacity{0.783819}%
\pgfsetdash{}{0pt}%
\pgfpathmoveto{\pgfqpoint{0.987084in}{1.513334in}}%
\pgfpathcurveto{\pgfqpoint{0.995320in}{1.513334in}}{\pgfqpoint{1.003220in}{1.516606in}}{\pgfqpoint{1.009044in}{1.522430in}}%
\pgfpathcurveto{\pgfqpoint{1.014868in}{1.528254in}}{\pgfqpoint{1.018140in}{1.536154in}}{\pgfqpoint{1.018140in}{1.544390in}}%
\pgfpathcurveto{\pgfqpoint{1.018140in}{1.552626in}}{\pgfqpoint{1.014868in}{1.560526in}}{\pgfqpoint{1.009044in}{1.566350in}}%
\pgfpathcurveto{\pgfqpoint{1.003220in}{1.572174in}}{\pgfqpoint{0.995320in}{1.575447in}}{\pgfqpoint{0.987084in}{1.575447in}}%
\pgfpathcurveto{\pgfqpoint{0.978847in}{1.575447in}}{\pgfqpoint{0.970947in}{1.572174in}}{\pgfqpoint{0.965123in}{1.566350in}}%
\pgfpathcurveto{\pgfqpoint{0.959299in}{1.560526in}}{\pgfqpoint{0.956027in}{1.552626in}}{\pgfqpoint{0.956027in}{1.544390in}}%
\pgfpathcurveto{\pgfqpoint{0.956027in}{1.536154in}}{\pgfqpoint{0.959299in}{1.528254in}}{\pgfqpoint{0.965123in}{1.522430in}}%
\pgfpathcurveto{\pgfqpoint{0.970947in}{1.516606in}}{\pgfqpoint{0.978847in}{1.513334in}}{\pgfqpoint{0.987084in}{1.513334in}}%
\pgfpathclose%
\pgfusepath{stroke,fill}%
\end{pgfscope}%
\begin{pgfscope}%
\pgfpathrectangle{\pgfqpoint{0.100000in}{0.212622in}}{\pgfqpoint{3.696000in}{3.696000in}}%
\pgfusepath{clip}%
\pgfsetbuttcap%
\pgfsetroundjoin%
\definecolor{currentfill}{rgb}{0.121569,0.466667,0.705882}%
\pgfsetfillcolor{currentfill}%
\pgfsetfillopacity{0.783823}%
\pgfsetlinewidth{1.003750pt}%
\definecolor{currentstroke}{rgb}{0.121569,0.466667,0.705882}%
\pgfsetstrokecolor{currentstroke}%
\pgfsetstrokeopacity{0.783823}%
\pgfsetdash{}{0pt}%
\pgfpathmoveto{\pgfqpoint{0.987069in}{1.513303in}}%
\pgfpathcurveto{\pgfqpoint{0.995305in}{1.513303in}}{\pgfqpoint{1.003205in}{1.516575in}}{\pgfqpoint{1.009029in}{1.522399in}}%
\pgfpathcurveto{\pgfqpoint{1.014853in}{1.528223in}}{\pgfqpoint{1.018125in}{1.536123in}}{\pgfqpoint{1.018125in}{1.544359in}}%
\pgfpathcurveto{\pgfqpoint{1.018125in}{1.552595in}}{\pgfqpoint{1.014853in}{1.560496in}}{\pgfqpoint{1.009029in}{1.566319in}}%
\pgfpathcurveto{\pgfqpoint{1.003205in}{1.572143in}}{\pgfqpoint{0.995305in}{1.575416in}}{\pgfqpoint{0.987069in}{1.575416in}}%
\pgfpathcurveto{\pgfqpoint{0.978833in}{1.575416in}}{\pgfqpoint{0.970933in}{1.572143in}}{\pgfqpoint{0.965109in}{1.566319in}}%
\pgfpathcurveto{\pgfqpoint{0.959285in}{1.560496in}}{\pgfqpoint{0.956012in}{1.552595in}}{\pgfqpoint{0.956012in}{1.544359in}}%
\pgfpathcurveto{\pgfqpoint{0.956012in}{1.536123in}}{\pgfqpoint{0.959285in}{1.528223in}}{\pgfqpoint{0.965109in}{1.522399in}}%
\pgfpathcurveto{\pgfqpoint{0.970933in}{1.516575in}}{\pgfqpoint{0.978833in}{1.513303in}}{\pgfqpoint{0.987069in}{1.513303in}}%
\pgfpathclose%
\pgfusepath{stroke,fill}%
\end{pgfscope}%
\begin{pgfscope}%
\pgfpathrectangle{\pgfqpoint{0.100000in}{0.212622in}}{\pgfqpoint{3.696000in}{3.696000in}}%
\pgfusepath{clip}%
\pgfsetbuttcap%
\pgfsetroundjoin%
\definecolor{currentfill}{rgb}{0.121569,0.466667,0.705882}%
\pgfsetfillcolor{currentfill}%
\pgfsetfillopacity{0.783826}%
\pgfsetlinewidth{1.003750pt}%
\definecolor{currentstroke}{rgb}{0.121569,0.466667,0.705882}%
\pgfsetstrokecolor{currentstroke}%
\pgfsetstrokeopacity{0.783826}%
\pgfsetdash{}{0pt}%
\pgfpathmoveto{\pgfqpoint{2.554973in}{2.906533in}}%
\pgfpathcurveto{\pgfqpoint{2.563209in}{2.906533in}}{\pgfqpoint{2.571109in}{2.909805in}}{\pgfqpoint{2.576933in}{2.915629in}}%
\pgfpathcurveto{\pgfqpoint{2.582757in}{2.921453in}}{\pgfqpoint{2.586030in}{2.929353in}}{\pgfqpoint{2.586030in}{2.937589in}}%
\pgfpathcurveto{\pgfqpoint{2.586030in}{2.945826in}}{\pgfqpoint{2.582757in}{2.953726in}}{\pgfqpoint{2.576933in}{2.959550in}}%
\pgfpathcurveto{\pgfqpoint{2.571109in}{2.965374in}}{\pgfqpoint{2.563209in}{2.968646in}}{\pgfqpoint{2.554973in}{2.968646in}}%
\pgfpathcurveto{\pgfqpoint{2.546737in}{2.968646in}}{\pgfqpoint{2.538837in}{2.965374in}}{\pgfqpoint{2.533013in}{2.959550in}}%
\pgfpathcurveto{\pgfqpoint{2.527189in}{2.953726in}}{\pgfqpoint{2.523917in}{2.945826in}}{\pgfqpoint{2.523917in}{2.937589in}}%
\pgfpathcurveto{\pgfqpoint{2.523917in}{2.929353in}}{\pgfqpoint{2.527189in}{2.921453in}}{\pgfqpoint{2.533013in}{2.915629in}}%
\pgfpathcurveto{\pgfqpoint{2.538837in}{2.909805in}}{\pgfqpoint{2.546737in}{2.906533in}}{\pgfqpoint{2.554973in}{2.906533in}}%
\pgfpathclose%
\pgfusepath{stroke,fill}%
\end{pgfscope}%
\begin{pgfscope}%
\pgfpathrectangle{\pgfqpoint{0.100000in}{0.212622in}}{\pgfqpoint{3.696000in}{3.696000in}}%
\pgfusepath{clip}%
\pgfsetbuttcap%
\pgfsetroundjoin%
\definecolor{currentfill}{rgb}{0.121569,0.466667,0.705882}%
\pgfsetfillcolor{currentfill}%
\pgfsetfillopacity{0.783826}%
\pgfsetlinewidth{1.003750pt}%
\definecolor{currentstroke}{rgb}{0.121569,0.466667,0.705882}%
\pgfsetstrokecolor{currentstroke}%
\pgfsetstrokeopacity{0.783826}%
\pgfsetdash{}{0pt}%
\pgfpathmoveto{\pgfqpoint{0.987061in}{1.513286in}}%
\pgfpathcurveto{\pgfqpoint{0.995297in}{1.513286in}}{\pgfqpoint{1.003197in}{1.516558in}}{\pgfqpoint{1.009021in}{1.522382in}}%
\pgfpathcurveto{\pgfqpoint{1.014845in}{1.528206in}}{\pgfqpoint{1.018117in}{1.536106in}}{\pgfqpoint{1.018117in}{1.544342in}}%
\pgfpathcurveto{\pgfqpoint{1.018117in}{1.552579in}}{\pgfqpoint{1.014845in}{1.560479in}}{\pgfqpoint{1.009021in}{1.566303in}}%
\pgfpathcurveto{\pgfqpoint{1.003197in}{1.572126in}}{\pgfqpoint{0.995297in}{1.575399in}}{\pgfqpoint{0.987061in}{1.575399in}}%
\pgfpathcurveto{\pgfqpoint{0.978825in}{1.575399in}}{\pgfqpoint{0.970924in}{1.572126in}}{\pgfqpoint{0.965101in}{1.566303in}}%
\pgfpathcurveto{\pgfqpoint{0.959277in}{1.560479in}}{\pgfqpoint{0.956004in}{1.552579in}}{\pgfqpoint{0.956004in}{1.544342in}}%
\pgfpathcurveto{\pgfqpoint{0.956004in}{1.536106in}}{\pgfqpoint{0.959277in}{1.528206in}}{\pgfqpoint{0.965101in}{1.522382in}}%
\pgfpathcurveto{\pgfqpoint{0.970924in}{1.516558in}}{\pgfqpoint{0.978825in}{1.513286in}}{\pgfqpoint{0.987061in}{1.513286in}}%
\pgfpathclose%
\pgfusepath{stroke,fill}%
\end{pgfscope}%
\begin{pgfscope}%
\pgfpathrectangle{\pgfqpoint{0.100000in}{0.212622in}}{\pgfqpoint{3.696000in}{3.696000in}}%
\pgfusepath{clip}%
\pgfsetbuttcap%
\pgfsetroundjoin%
\definecolor{currentfill}{rgb}{0.121569,0.466667,0.705882}%
\pgfsetfillcolor{currentfill}%
\pgfsetfillopacity{0.783827}%
\pgfsetlinewidth{1.003750pt}%
\definecolor{currentstroke}{rgb}{0.121569,0.466667,0.705882}%
\pgfsetstrokecolor{currentstroke}%
\pgfsetstrokeopacity{0.783827}%
\pgfsetdash{}{0pt}%
\pgfpathmoveto{\pgfqpoint{0.987056in}{1.513276in}}%
\pgfpathcurveto{\pgfqpoint{0.995292in}{1.513276in}}{\pgfqpoint{1.003192in}{1.516549in}}{\pgfqpoint{1.009016in}{1.522373in}}%
\pgfpathcurveto{\pgfqpoint{1.014840in}{1.528197in}}{\pgfqpoint{1.018113in}{1.536097in}}{\pgfqpoint{1.018113in}{1.544333in}}%
\pgfpathcurveto{\pgfqpoint{1.018113in}{1.552569in}}{\pgfqpoint{1.014840in}{1.560469in}}{\pgfqpoint{1.009016in}{1.566293in}}%
\pgfpathcurveto{\pgfqpoint{1.003192in}{1.572117in}}{\pgfqpoint{0.995292in}{1.575389in}}{\pgfqpoint{0.987056in}{1.575389in}}%
\pgfpathcurveto{\pgfqpoint{0.978820in}{1.575389in}}{\pgfqpoint{0.970920in}{1.572117in}}{\pgfqpoint{0.965096in}{1.566293in}}%
\pgfpathcurveto{\pgfqpoint{0.959272in}{1.560469in}}{\pgfqpoint{0.956000in}{1.552569in}}{\pgfqpoint{0.956000in}{1.544333in}}%
\pgfpathcurveto{\pgfqpoint{0.956000in}{1.536097in}}{\pgfqpoint{0.959272in}{1.528197in}}{\pgfqpoint{0.965096in}{1.522373in}}%
\pgfpathcurveto{\pgfqpoint{0.970920in}{1.516549in}}{\pgfqpoint{0.978820in}{1.513276in}}{\pgfqpoint{0.987056in}{1.513276in}}%
\pgfpathclose%
\pgfusepath{stroke,fill}%
\end{pgfscope}%
\begin{pgfscope}%
\pgfpathrectangle{\pgfqpoint{0.100000in}{0.212622in}}{\pgfqpoint{3.696000in}{3.696000in}}%
\pgfusepath{clip}%
\pgfsetbuttcap%
\pgfsetroundjoin%
\definecolor{currentfill}{rgb}{0.121569,0.466667,0.705882}%
\pgfsetfillcolor{currentfill}%
\pgfsetfillopacity{0.783828}%
\pgfsetlinewidth{1.003750pt}%
\definecolor{currentstroke}{rgb}{0.121569,0.466667,0.705882}%
\pgfsetstrokecolor{currentstroke}%
\pgfsetstrokeopacity{0.783828}%
\pgfsetdash{}{0pt}%
\pgfpathmoveto{\pgfqpoint{0.987054in}{1.513271in}}%
\pgfpathcurveto{\pgfqpoint{0.995290in}{1.513271in}}{\pgfqpoint{1.003190in}{1.516543in}}{\pgfqpoint{1.009014in}{1.522367in}}%
\pgfpathcurveto{\pgfqpoint{1.014838in}{1.528191in}}{\pgfqpoint{1.018110in}{1.536091in}}{\pgfqpoint{1.018110in}{1.544328in}}%
\pgfpathcurveto{\pgfqpoint{1.018110in}{1.552564in}}{\pgfqpoint{1.014838in}{1.560464in}}{\pgfqpoint{1.009014in}{1.566288in}}%
\pgfpathcurveto{\pgfqpoint{1.003190in}{1.572112in}}{\pgfqpoint{0.995290in}{1.575384in}}{\pgfqpoint{0.987054in}{1.575384in}}%
\pgfpathcurveto{\pgfqpoint{0.978817in}{1.575384in}}{\pgfqpoint{0.970917in}{1.572112in}}{\pgfqpoint{0.965093in}{1.566288in}}%
\pgfpathcurveto{\pgfqpoint{0.959269in}{1.560464in}}{\pgfqpoint{0.955997in}{1.552564in}}{\pgfqpoint{0.955997in}{1.544328in}}%
\pgfpathcurveto{\pgfqpoint{0.955997in}{1.536091in}}{\pgfqpoint{0.959269in}{1.528191in}}{\pgfqpoint{0.965093in}{1.522367in}}%
\pgfpathcurveto{\pgfqpoint{0.970917in}{1.516543in}}{\pgfqpoint{0.978817in}{1.513271in}}{\pgfqpoint{0.987054in}{1.513271in}}%
\pgfpathclose%
\pgfusepath{stroke,fill}%
\end{pgfscope}%
\begin{pgfscope}%
\pgfpathrectangle{\pgfqpoint{0.100000in}{0.212622in}}{\pgfqpoint{3.696000in}{3.696000in}}%
\pgfusepath{clip}%
\pgfsetbuttcap%
\pgfsetroundjoin%
\definecolor{currentfill}{rgb}{0.121569,0.466667,0.705882}%
\pgfsetfillcolor{currentfill}%
\pgfsetfillopacity{0.783828}%
\pgfsetlinewidth{1.003750pt}%
\definecolor{currentstroke}{rgb}{0.121569,0.466667,0.705882}%
\pgfsetstrokecolor{currentstroke}%
\pgfsetstrokeopacity{0.783828}%
\pgfsetdash{}{0pt}%
\pgfpathmoveto{\pgfqpoint{0.987052in}{1.513268in}}%
\pgfpathcurveto{\pgfqpoint{0.995288in}{1.513268in}}{\pgfqpoint{1.003189in}{1.516541in}}{\pgfqpoint{1.009012in}{1.522364in}}%
\pgfpathcurveto{\pgfqpoint{1.014836in}{1.528188in}}{\pgfqpoint{1.018109in}{1.536088in}}{\pgfqpoint{1.018109in}{1.544325in}}%
\pgfpathcurveto{\pgfqpoint{1.018109in}{1.552561in}}{\pgfqpoint{1.014836in}{1.560461in}}{\pgfqpoint{1.009012in}{1.566285in}}%
\pgfpathcurveto{\pgfqpoint{1.003189in}{1.572109in}}{\pgfqpoint{0.995288in}{1.575381in}}{\pgfqpoint{0.987052in}{1.575381in}}%
\pgfpathcurveto{\pgfqpoint{0.978816in}{1.575381in}}{\pgfqpoint{0.970916in}{1.572109in}}{\pgfqpoint{0.965092in}{1.566285in}}%
\pgfpathcurveto{\pgfqpoint{0.959268in}{1.560461in}}{\pgfqpoint{0.955996in}{1.552561in}}{\pgfqpoint{0.955996in}{1.544325in}}%
\pgfpathcurveto{\pgfqpoint{0.955996in}{1.536088in}}{\pgfqpoint{0.959268in}{1.528188in}}{\pgfqpoint{0.965092in}{1.522364in}}%
\pgfpathcurveto{\pgfqpoint{0.970916in}{1.516541in}}{\pgfqpoint{0.978816in}{1.513268in}}{\pgfqpoint{0.987052in}{1.513268in}}%
\pgfpathclose%
\pgfusepath{stroke,fill}%
\end{pgfscope}%
\begin{pgfscope}%
\pgfpathrectangle{\pgfqpoint{0.100000in}{0.212622in}}{\pgfqpoint{3.696000in}{3.696000in}}%
\pgfusepath{clip}%
\pgfsetbuttcap%
\pgfsetroundjoin%
\definecolor{currentfill}{rgb}{0.121569,0.466667,0.705882}%
\pgfsetfillcolor{currentfill}%
\pgfsetfillopacity{0.783828}%
\pgfsetlinewidth{1.003750pt}%
\definecolor{currentstroke}{rgb}{0.121569,0.466667,0.705882}%
\pgfsetstrokecolor{currentstroke}%
\pgfsetstrokeopacity{0.783828}%
\pgfsetdash{}{0pt}%
\pgfpathmoveto{\pgfqpoint{0.987051in}{1.513267in}}%
\pgfpathcurveto{\pgfqpoint{0.995288in}{1.513267in}}{\pgfqpoint{1.003188in}{1.516539in}}{\pgfqpoint{1.009012in}{1.522363in}}%
\pgfpathcurveto{\pgfqpoint{1.014836in}{1.528187in}}{\pgfqpoint{1.018108in}{1.536087in}}{\pgfqpoint{1.018108in}{1.544323in}}%
\pgfpathcurveto{\pgfqpoint{1.018108in}{1.552559in}}{\pgfqpoint{1.014836in}{1.560459in}}{\pgfqpoint{1.009012in}{1.566283in}}%
\pgfpathcurveto{\pgfqpoint{1.003188in}{1.572107in}}{\pgfqpoint{0.995288in}{1.575380in}}{\pgfqpoint{0.987051in}{1.575380in}}%
\pgfpathcurveto{\pgfqpoint{0.978815in}{1.575380in}}{\pgfqpoint{0.970915in}{1.572107in}}{\pgfqpoint{0.965091in}{1.566283in}}%
\pgfpathcurveto{\pgfqpoint{0.959267in}{1.560459in}}{\pgfqpoint{0.955995in}{1.552559in}}{\pgfqpoint{0.955995in}{1.544323in}}%
\pgfpathcurveto{\pgfqpoint{0.955995in}{1.536087in}}{\pgfqpoint{0.959267in}{1.528187in}}{\pgfqpoint{0.965091in}{1.522363in}}%
\pgfpathcurveto{\pgfqpoint{0.970915in}{1.516539in}}{\pgfqpoint{0.978815in}{1.513267in}}{\pgfqpoint{0.987051in}{1.513267in}}%
\pgfpathclose%
\pgfusepath{stroke,fill}%
\end{pgfscope}%
\begin{pgfscope}%
\pgfpathrectangle{\pgfqpoint{0.100000in}{0.212622in}}{\pgfqpoint{3.696000in}{3.696000in}}%
\pgfusepath{clip}%
\pgfsetbuttcap%
\pgfsetroundjoin%
\definecolor{currentfill}{rgb}{0.121569,0.466667,0.705882}%
\pgfsetfillcolor{currentfill}%
\pgfsetfillopacity{0.783828}%
\pgfsetlinewidth{1.003750pt}%
\definecolor{currentstroke}{rgb}{0.121569,0.466667,0.705882}%
\pgfsetstrokecolor{currentstroke}%
\pgfsetstrokeopacity{0.783828}%
\pgfsetdash{}{0pt}%
\pgfpathmoveto{\pgfqpoint{0.987051in}{1.513266in}}%
\pgfpathcurveto{\pgfqpoint{0.995287in}{1.513266in}}{\pgfqpoint{1.003187in}{1.516538in}}{\pgfqpoint{1.009011in}{1.522362in}}%
\pgfpathcurveto{\pgfqpoint{1.014835in}{1.528186in}}{\pgfqpoint{1.018107in}{1.536086in}}{\pgfqpoint{1.018107in}{1.544322in}}%
\pgfpathcurveto{\pgfqpoint{1.018107in}{1.552559in}}{\pgfqpoint{1.014835in}{1.560459in}}{\pgfqpoint{1.009011in}{1.566283in}}%
\pgfpathcurveto{\pgfqpoint{1.003187in}{1.572107in}}{\pgfqpoint{0.995287in}{1.575379in}}{\pgfqpoint{0.987051in}{1.575379in}}%
\pgfpathcurveto{\pgfqpoint{0.978815in}{1.575379in}}{\pgfqpoint{0.970915in}{1.572107in}}{\pgfqpoint{0.965091in}{1.566283in}}%
\pgfpathcurveto{\pgfqpoint{0.959267in}{1.560459in}}{\pgfqpoint{0.955994in}{1.552559in}}{\pgfqpoint{0.955994in}{1.544322in}}%
\pgfpathcurveto{\pgfqpoint{0.955994in}{1.536086in}}{\pgfqpoint{0.959267in}{1.528186in}}{\pgfqpoint{0.965091in}{1.522362in}}%
\pgfpathcurveto{\pgfqpoint{0.970915in}{1.516538in}}{\pgfqpoint{0.978815in}{1.513266in}}{\pgfqpoint{0.987051in}{1.513266in}}%
\pgfpathclose%
\pgfusepath{stroke,fill}%
\end{pgfscope}%
\begin{pgfscope}%
\pgfpathrectangle{\pgfqpoint{0.100000in}{0.212622in}}{\pgfqpoint{3.696000in}{3.696000in}}%
\pgfusepath{clip}%
\pgfsetbuttcap%
\pgfsetroundjoin%
\definecolor{currentfill}{rgb}{0.121569,0.466667,0.705882}%
\pgfsetfillcolor{currentfill}%
\pgfsetfillopacity{0.783829}%
\pgfsetlinewidth{1.003750pt}%
\definecolor{currentstroke}{rgb}{0.121569,0.466667,0.705882}%
\pgfsetstrokecolor{currentstroke}%
\pgfsetstrokeopacity{0.783829}%
\pgfsetdash{}{0pt}%
\pgfpathmoveto{\pgfqpoint{0.987051in}{1.513265in}}%
\pgfpathcurveto{\pgfqpoint{0.995287in}{1.513265in}}{\pgfqpoint{1.003187in}{1.516538in}}{\pgfqpoint{1.009011in}{1.522362in}}%
\pgfpathcurveto{\pgfqpoint{1.014835in}{1.528186in}}{\pgfqpoint{1.018107in}{1.536086in}}{\pgfqpoint{1.018107in}{1.544322in}}%
\pgfpathcurveto{\pgfqpoint{1.018107in}{1.552558in}}{\pgfqpoint{1.014835in}{1.560458in}}{\pgfqpoint{1.009011in}{1.566282in}}%
\pgfpathcurveto{\pgfqpoint{1.003187in}{1.572106in}}{\pgfqpoint{0.995287in}{1.575378in}}{\pgfqpoint{0.987051in}{1.575378in}}%
\pgfpathcurveto{\pgfqpoint{0.978814in}{1.575378in}}{\pgfqpoint{0.970914in}{1.572106in}}{\pgfqpoint{0.965090in}{1.566282in}}%
\pgfpathcurveto{\pgfqpoint{0.959267in}{1.560458in}}{\pgfqpoint{0.955994in}{1.552558in}}{\pgfqpoint{0.955994in}{1.544322in}}%
\pgfpathcurveto{\pgfqpoint{0.955994in}{1.536086in}}{\pgfqpoint{0.959267in}{1.528186in}}{\pgfqpoint{0.965090in}{1.522362in}}%
\pgfpathcurveto{\pgfqpoint{0.970914in}{1.516538in}}{\pgfqpoint{0.978814in}{1.513265in}}{\pgfqpoint{0.987051in}{1.513265in}}%
\pgfpathclose%
\pgfusepath{stroke,fill}%
\end{pgfscope}%
\begin{pgfscope}%
\pgfpathrectangle{\pgfqpoint{0.100000in}{0.212622in}}{\pgfqpoint{3.696000in}{3.696000in}}%
\pgfusepath{clip}%
\pgfsetbuttcap%
\pgfsetroundjoin%
\definecolor{currentfill}{rgb}{0.121569,0.466667,0.705882}%
\pgfsetfillcolor{currentfill}%
\pgfsetfillopacity{0.783829}%
\pgfsetlinewidth{1.003750pt}%
\definecolor{currentstroke}{rgb}{0.121569,0.466667,0.705882}%
\pgfsetstrokecolor{currentstroke}%
\pgfsetstrokeopacity{0.783829}%
\pgfsetdash{}{0pt}%
\pgfpathmoveto{\pgfqpoint{0.987051in}{1.513265in}}%
\pgfpathcurveto{\pgfqpoint{0.995287in}{1.513265in}}{\pgfqpoint{1.003187in}{1.516537in}}{\pgfqpoint{1.009011in}{1.522361in}}%
\pgfpathcurveto{\pgfqpoint{1.014835in}{1.528185in}}{\pgfqpoint{1.018107in}{1.536085in}}{\pgfqpoint{1.018107in}{1.544322in}}%
\pgfpathcurveto{\pgfqpoint{1.018107in}{1.552558in}}{\pgfqpoint{1.014835in}{1.560458in}}{\pgfqpoint{1.009011in}{1.566282in}}%
\pgfpathcurveto{\pgfqpoint{1.003187in}{1.572106in}}{\pgfqpoint{0.995287in}{1.575378in}}{\pgfqpoint{0.987051in}{1.575378in}}%
\pgfpathcurveto{\pgfqpoint{0.978814in}{1.575378in}}{\pgfqpoint{0.970914in}{1.572106in}}{\pgfqpoint{0.965090in}{1.566282in}}%
\pgfpathcurveto{\pgfqpoint{0.959266in}{1.560458in}}{\pgfqpoint{0.955994in}{1.552558in}}{\pgfqpoint{0.955994in}{1.544322in}}%
\pgfpathcurveto{\pgfqpoint{0.955994in}{1.536085in}}{\pgfqpoint{0.959266in}{1.528185in}}{\pgfqpoint{0.965090in}{1.522361in}}%
\pgfpathcurveto{\pgfqpoint{0.970914in}{1.516537in}}{\pgfqpoint{0.978814in}{1.513265in}}{\pgfqpoint{0.987051in}{1.513265in}}%
\pgfpathclose%
\pgfusepath{stroke,fill}%
\end{pgfscope}%
\begin{pgfscope}%
\pgfpathrectangle{\pgfqpoint{0.100000in}{0.212622in}}{\pgfqpoint{3.696000in}{3.696000in}}%
\pgfusepath{clip}%
\pgfsetbuttcap%
\pgfsetroundjoin%
\definecolor{currentfill}{rgb}{0.121569,0.466667,0.705882}%
\pgfsetfillcolor{currentfill}%
\pgfsetfillopacity{0.783829}%
\pgfsetlinewidth{1.003750pt}%
\definecolor{currentstroke}{rgb}{0.121569,0.466667,0.705882}%
\pgfsetstrokecolor{currentstroke}%
\pgfsetstrokeopacity{0.783829}%
\pgfsetdash{}{0pt}%
\pgfpathmoveto{\pgfqpoint{0.987051in}{1.513265in}}%
\pgfpathcurveto{\pgfqpoint{0.995287in}{1.513265in}}{\pgfqpoint{1.003187in}{1.516537in}}{\pgfqpoint{1.009011in}{1.522361in}}%
\pgfpathcurveto{\pgfqpoint{1.014835in}{1.528185in}}{\pgfqpoint{1.018107in}{1.536085in}}{\pgfqpoint{1.018107in}{1.544321in}}%
\pgfpathcurveto{\pgfqpoint{1.018107in}{1.552558in}}{\pgfqpoint{1.014835in}{1.560458in}}{\pgfqpoint{1.009011in}{1.566282in}}%
\pgfpathcurveto{\pgfqpoint{1.003187in}{1.572106in}}{\pgfqpoint{0.995287in}{1.575378in}}{\pgfqpoint{0.987051in}{1.575378in}}%
\pgfpathcurveto{\pgfqpoint{0.978814in}{1.575378in}}{\pgfqpoint{0.970914in}{1.572106in}}{\pgfqpoint{0.965090in}{1.566282in}}%
\pgfpathcurveto{\pgfqpoint{0.959266in}{1.560458in}}{\pgfqpoint{0.955994in}{1.552558in}}{\pgfqpoint{0.955994in}{1.544321in}}%
\pgfpathcurveto{\pgfqpoint{0.955994in}{1.536085in}}{\pgfqpoint{0.959266in}{1.528185in}}{\pgfqpoint{0.965090in}{1.522361in}}%
\pgfpathcurveto{\pgfqpoint{0.970914in}{1.516537in}}{\pgfqpoint{0.978814in}{1.513265in}}{\pgfqpoint{0.987051in}{1.513265in}}%
\pgfpathclose%
\pgfusepath{stroke,fill}%
\end{pgfscope}%
\begin{pgfscope}%
\pgfpathrectangle{\pgfqpoint{0.100000in}{0.212622in}}{\pgfqpoint{3.696000in}{3.696000in}}%
\pgfusepath{clip}%
\pgfsetbuttcap%
\pgfsetroundjoin%
\definecolor{currentfill}{rgb}{0.121569,0.466667,0.705882}%
\pgfsetfillcolor{currentfill}%
\pgfsetfillopacity{0.783829}%
\pgfsetlinewidth{1.003750pt}%
\definecolor{currentstroke}{rgb}{0.121569,0.466667,0.705882}%
\pgfsetstrokecolor{currentstroke}%
\pgfsetstrokeopacity{0.783829}%
\pgfsetdash{}{0pt}%
\pgfpathmoveto{\pgfqpoint{0.987051in}{1.513265in}}%
\pgfpathcurveto{\pgfqpoint{0.995287in}{1.513265in}}{\pgfqpoint{1.003187in}{1.516537in}}{\pgfqpoint{1.009011in}{1.522361in}}%
\pgfpathcurveto{\pgfqpoint{1.014835in}{1.528185in}}{\pgfqpoint{1.018107in}{1.536085in}}{\pgfqpoint{1.018107in}{1.544321in}}%
\pgfpathcurveto{\pgfqpoint{1.018107in}{1.552558in}}{\pgfqpoint{1.014835in}{1.560458in}}{\pgfqpoint{1.009011in}{1.566282in}}%
\pgfpathcurveto{\pgfqpoint{1.003187in}{1.572106in}}{\pgfqpoint{0.995287in}{1.575378in}}{\pgfqpoint{0.987051in}{1.575378in}}%
\pgfpathcurveto{\pgfqpoint{0.978814in}{1.575378in}}{\pgfqpoint{0.970914in}{1.572106in}}{\pgfqpoint{0.965090in}{1.566282in}}%
\pgfpathcurveto{\pgfqpoint{0.959266in}{1.560458in}}{\pgfqpoint{0.955994in}{1.552558in}}{\pgfqpoint{0.955994in}{1.544321in}}%
\pgfpathcurveto{\pgfqpoint{0.955994in}{1.536085in}}{\pgfqpoint{0.959266in}{1.528185in}}{\pgfqpoint{0.965090in}{1.522361in}}%
\pgfpathcurveto{\pgfqpoint{0.970914in}{1.516537in}}{\pgfqpoint{0.978814in}{1.513265in}}{\pgfqpoint{0.987051in}{1.513265in}}%
\pgfpathclose%
\pgfusepath{stroke,fill}%
\end{pgfscope}%
\begin{pgfscope}%
\pgfpathrectangle{\pgfqpoint{0.100000in}{0.212622in}}{\pgfqpoint{3.696000in}{3.696000in}}%
\pgfusepath{clip}%
\pgfsetbuttcap%
\pgfsetroundjoin%
\definecolor{currentfill}{rgb}{0.121569,0.466667,0.705882}%
\pgfsetfillcolor{currentfill}%
\pgfsetfillopacity{0.783829}%
\pgfsetlinewidth{1.003750pt}%
\definecolor{currentstroke}{rgb}{0.121569,0.466667,0.705882}%
\pgfsetstrokecolor{currentstroke}%
\pgfsetstrokeopacity{0.783829}%
\pgfsetdash{}{0pt}%
\pgfpathmoveto{\pgfqpoint{0.987051in}{1.513265in}}%
\pgfpathcurveto{\pgfqpoint{0.995287in}{1.513265in}}{\pgfqpoint{1.003187in}{1.516537in}}{\pgfqpoint{1.009011in}{1.522361in}}%
\pgfpathcurveto{\pgfqpoint{1.014835in}{1.528185in}}{\pgfqpoint{1.018107in}{1.536085in}}{\pgfqpoint{1.018107in}{1.544321in}}%
\pgfpathcurveto{\pgfqpoint{1.018107in}{1.552558in}}{\pgfqpoint{1.014835in}{1.560458in}}{\pgfqpoint{1.009011in}{1.566282in}}%
\pgfpathcurveto{\pgfqpoint{1.003187in}{1.572106in}}{\pgfqpoint{0.995287in}{1.575378in}}{\pgfqpoint{0.987051in}{1.575378in}}%
\pgfpathcurveto{\pgfqpoint{0.978814in}{1.575378in}}{\pgfqpoint{0.970914in}{1.572106in}}{\pgfqpoint{0.965090in}{1.566282in}}%
\pgfpathcurveto{\pgfqpoint{0.959266in}{1.560458in}}{\pgfqpoint{0.955994in}{1.552558in}}{\pgfqpoint{0.955994in}{1.544321in}}%
\pgfpathcurveto{\pgfqpoint{0.955994in}{1.536085in}}{\pgfqpoint{0.959266in}{1.528185in}}{\pgfqpoint{0.965090in}{1.522361in}}%
\pgfpathcurveto{\pgfqpoint{0.970914in}{1.516537in}}{\pgfqpoint{0.978814in}{1.513265in}}{\pgfqpoint{0.987051in}{1.513265in}}%
\pgfpathclose%
\pgfusepath{stroke,fill}%
\end{pgfscope}%
\begin{pgfscope}%
\pgfpathrectangle{\pgfqpoint{0.100000in}{0.212622in}}{\pgfqpoint{3.696000in}{3.696000in}}%
\pgfusepath{clip}%
\pgfsetbuttcap%
\pgfsetroundjoin%
\definecolor{currentfill}{rgb}{0.121569,0.466667,0.705882}%
\pgfsetfillcolor{currentfill}%
\pgfsetfillopacity{0.783829}%
\pgfsetlinewidth{1.003750pt}%
\definecolor{currentstroke}{rgb}{0.121569,0.466667,0.705882}%
\pgfsetstrokecolor{currentstroke}%
\pgfsetstrokeopacity{0.783829}%
\pgfsetdash{}{0pt}%
\pgfpathmoveto{\pgfqpoint{0.987050in}{1.513265in}}%
\pgfpathcurveto{\pgfqpoint{0.995287in}{1.513265in}}{\pgfqpoint{1.003187in}{1.516537in}}{\pgfqpoint{1.009011in}{1.522361in}}%
\pgfpathcurveto{\pgfqpoint{1.014835in}{1.528185in}}{\pgfqpoint{1.018107in}{1.536085in}}{\pgfqpoint{1.018107in}{1.544321in}}%
\pgfpathcurveto{\pgfqpoint{1.018107in}{1.552558in}}{\pgfqpoint{1.014835in}{1.560458in}}{\pgfqpoint{1.009011in}{1.566282in}}%
\pgfpathcurveto{\pgfqpoint{1.003187in}{1.572106in}}{\pgfqpoint{0.995287in}{1.575378in}}{\pgfqpoint{0.987050in}{1.575378in}}%
\pgfpathcurveto{\pgfqpoint{0.978814in}{1.575378in}}{\pgfqpoint{0.970914in}{1.572106in}}{\pgfqpoint{0.965090in}{1.566282in}}%
\pgfpathcurveto{\pgfqpoint{0.959266in}{1.560458in}}{\pgfqpoint{0.955994in}{1.552558in}}{\pgfqpoint{0.955994in}{1.544321in}}%
\pgfpathcurveto{\pgfqpoint{0.955994in}{1.536085in}}{\pgfqpoint{0.959266in}{1.528185in}}{\pgfqpoint{0.965090in}{1.522361in}}%
\pgfpathcurveto{\pgfqpoint{0.970914in}{1.516537in}}{\pgfqpoint{0.978814in}{1.513265in}}{\pgfqpoint{0.987050in}{1.513265in}}%
\pgfpathclose%
\pgfusepath{stroke,fill}%
\end{pgfscope}%
\begin{pgfscope}%
\pgfpathrectangle{\pgfqpoint{0.100000in}{0.212622in}}{\pgfqpoint{3.696000in}{3.696000in}}%
\pgfusepath{clip}%
\pgfsetbuttcap%
\pgfsetroundjoin%
\definecolor{currentfill}{rgb}{0.121569,0.466667,0.705882}%
\pgfsetfillcolor{currentfill}%
\pgfsetfillopacity{0.783829}%
\pgfsetlinewidth{1.003750pt}%
\definecolor{currentstroke}{rgb}{0.121569,0.466667,0.705882}%
\pgfsetstrokecolor{currentstroke}%
\pgfsetstrokeopacity{0.783829}%
\pgfsetdash{}{0pt}%
\pgfpathmoveto{\pgfqpoint{0.987050in}{1.513265in}}%
\pgfpathcurveto{\pgfqpoint{0.995287in}{1.513265in}}{\pgfqpoint{1.003187in}{1.516537in}}{\pgfqpoint{1.009011in}{1.522361in}}%
\pgfpathcurveto{\pgfqpoint{1.014835in}{1.528185in}}{\pgfqpoint{1.018107in}{1.536085in}}{\pgfqpoint{1.018107in}{1.544321in}}%
\pgfpathcurveto{\pgfqpoint{1.018107in}{1.552558in}}{\pgfqpoint{1.014835in}{1.560458in}}{\pgfqpoint{1.009011in}{1.566282in}}%
\pgfpathcurveto{\pgfqpoint{1.003187in}{1.572106in}}{\pgfqpoint{0.995287in}{1.575378in}}{\pgfqpoint{0.987050in}{1.575378in}}%
\pgfpathcurveto{\pgfqpoint{0.978814in}{1.575378in}}{\pgfqpoint{0.970914in}{1.572106in}}{\pgfqpoint{0.965090in}{1.566282in}}%
\pgfpathcurveto{\pgfqpoint{0.959266in}{1.560458in}}{\pgfqpoint{0.955994in}{1.552558in}}{\pgfqpoint{0.955994in}{1.544321in}}%
\pgfpathcurveto{\pgfqpoint{0.955994in}{1.536085in}}{\pgfqpoint{0.959266in}{1.528185in}}{\pgfqpoint{0.965090in}{1.522361in}}%
\pgfpathcurveto{\pgfqpoint{0.970914in}{1.516537in}}{\pgfqpoint{0.978814in}{1.513265in}}{\pgfqpoint{0.987050in}{1.513265in}}%
\pgfpathclose%
\pgfusepath{stroke,fill}%
\end{pgfscope}%
\begin{pgfscope}%
\pgfpathrectangle{\pgfqpoint{0.100000in}{0.212622in}}{\pgfqpoint{3.696000in}{3.696000in}}%
\pgfusepath{clip}%
\pgfsetbuttcap%
\pgfsetroundjoin%
\definecolor{currentfill}{rgb}{0.121569,0.466667,0.705882}%
\pgfsetfillcolor{currentfill}%
\pgfsetfillopacity{0.783829}%
\pgfsetlinewidth{1.003750pt}%
\definecolor{currentstroke}{rgb}{0.121569,0.466667,0.705882}%
\pgfsetstrokecolor{currentstroke}%
\pgfsetstrokeopacity{0.783829}%
\pgfsetdash{}{0pt}%
\pgfpathmoveto{\pgfqpoint{0.987050in}{1.513265in}}%
\pgfpathcurveto{\pgfqpoint{0.995287in}{1.513265in}}{\pgfqpoint{1.003187in}{1.516537in}}{\pgfqpoint{1.009011in}{1.522361in}}%
\pgfpathcurveto{\pgfqpoint{1.014835in}{1.528185in}}{\pgfqpoint{1.018107in}{1.536085in}}{\pgfqpoint{1.018107in}{1.544321in}}%
\pgfpathcurveto{\pgfqpoint{1.018107in}{1.552558in}}{\pgfqpoint{1.014835in}{1.560458in}}{\pgfqpoint{1.009011in}{1.566282in}}%
\pgfpathcurveto{\pgfqpoint{1.003187in}{1.572105in}}{\pgfqpoint{0.995287in}{1.575378in}}{\pgfqpoint{0.987050in}{1.575378in}}%
\pgfpathcurveto{\pgfqpoint{0.978814in}{1.575378in}}{\pgfqpoint{0.970914in}{1.572105in}}{\pgfqpoint{0.965090in}{1.566282in}}%
\pgfpathcurveto{\pgfqpoint{0.959266in}{1.560458in}}{\pgfqpoint{0.955994in}{1.552558in}}{\pgfqpoint{0.955994in}{1.544321in}}%
\pgfpathcurveto{\pgfqpoint{0.955994in}{1.536085in}}{\pgfqpoint{0.959266in}{1.528185in}}{\pgfqpoint{0.965090in}{1.522361in}}%
\pgfpathcurveto{\pgfqpoint{0.970914in}{1.516537in}}{\pgfqpoint{0.978814in}{1.513265in}}{\pgfqpoint{0.987050in}{1.513265in}}%
\pgfpathclose%
\pgfusepath{stroke,fill}%
\end{pgfscope}%
\begin{pgfscope}%
\pgfpathrectangle{\pgfqpoint{0.100000in}{0.212622in}}{\pgfqpoint{3.696000in}{3.696000in}}%
\pgfusepath{clip}%
\pgfsetbuttcap%
\pgfsetroundjoin%
\definecolor{currentfill}{rgb}{0.121569,0.466667,0.705882}%
\pgfsetfillcolor{currentfill}%
\pgfsetfillopacity{0.783829}%
\pgfsetlinewidth{1.003750pt}%
\definecolor{currentstroke}{rgb}{0.121569,0.466667,0.705882}%
\pgfsetstrokecolor{currentstroke}%
\pgfsetstrokeopacity{0.783829}%
\pgfsetdash{}{0pt}%
\pgfpathmoveto{\pgfqpoint{0.987050in}{1.513265in}}%
\pgfpathcurveto{\pgfqpoint{0.995287in}{1.513265in}}{\pgfqpoint{1.003187in}{1.516537in}}{\pgfqpoint{1.009011in}{1.522361in}}%
\pgfpathcurveto{\pgfqpoint{1.014835in}{1.528185in}}{\pgfqpoint{1.018107in}{1.536085in}}{\pgfqpoint{1.018107in}{1.544321in}}%
\pgfpathcurveto{\pgfqpoint{1.018107in}{1.552558in}}{\pgfqpoint{1.014835in}{1.560458in}}{\pgfqpoint{1.009011in}{1.566282in}}%
\pgfpathcurveto{\pgfqpoint{1.003187in}{1.572105in}}{\pgfqpoint{0.995287in}{1.575378in}}{\pgfqpoint{0.987050in}{1.575378in}}%
\pgfpathcurveto{\pgfqpoint{0.978814in}{1.575378in}}{\pgfqpoint{0.970914in}{1.572105in}}{\pgfqpoint{0.965090in}{1.566282in}}%
\pgfpathcurveto{\pgfqpoint{0.959266in}{1.560458in}}{\pgfqpoint{0.955994in}{1.552558in}}{\pgfqpoint{0.955994in}{1.544321in}}%
\pgfpathcurveto{\pgfqpoint{0.955994in}{1.536085in}}{\pgfqpoint{0.959266in}{1.528185in}}{\pgfqpoint{0.965090in}{1.522361in}}%
\pgfpathcurveto{\pgfqpoint{0.970914in}{1.516537in}}{\pgfqpoint{0.978814in}{1.513265in}}{\pgfqpoint{0.987050in}{1.513265in}}%
\pgfpathclose%
\pgfusepath{stroke,fill}%
\end{pgfscope}%
\begin{pgfscope}%
\pgfpathrectangle{\pgfqpoint{0.100000in}{0.212622in}}{\pgfqpoint{3.696000in}{3.696000in}}%
\pgfusepath{clip}%
\pgfsetbuttcap%
\pgfsetroundjoin%
\definecolor{currentfill}{rgb}{0.121569,0.466667,0.705882}%
\pgfsetfillcolor{currentfill}%
\pgfsetfillopacity{0.783829}%
\pgfsetlinewidth{1.003750pt}%
\definecolor{currentstroke}{rgb}{0.121569,0.466667,0.705882}%
\pgfsetstrokecolor{currentstroke}%
\pgfsetstrokeopacity{0.783829}%
\pgfsetdash{}{0pt}%
\pgfpathmoveto{\pgfqpoint{0.987050in}{1.513265in}}%
\pgfpathcurveto{\pgfqpoint{0.995287in}{1.513265in}}{\pgfqpoint{1.003187in}{1.516537in}}{\pgfqpoint{1.009011in}{1.522361in}}%
\pgfpathcurveto{\pgfqpoint{1.014835in}{1.528185in}}{\pgfqpoint{1.018107in}{1.536085in}}{\pgfqpoint{1.018107in}{1.544321in}}%
\pgfpathcurveto{\pgfqpoint{1.018107in}{1.552558in}}{\pgfqpoint{1.014835in}{1.560458in}}{\pgfqpoint{1.009011in}{1.566282in}}%
\pgfpathcurveto{\pgfqpoint{1.003187in}{1.572105in}}{\pgfqpoint{0.995287in}{1.575378in}}{\pgfqpoint{0.987050in}{1.575378in}}%
\pgfpathcurveto{\pgfqpoint{0.978814in}{1.575378in}}{\pgfqpoint{0.970914in}{1.572105in}}{\pgfqpoint{0.965090in}{1.566282in}}%
\pgfpathcurveto{\pgfqpoint{0.959266in}{1.560458in}}{\pgfqpoint{0.955994in}{1.552558in}}{\pgfqpoint{0.955994in}{1.544321in}}%
\pgfpathcurveto{\pgfqpoint{0.955994in}{1.536085in}}{\pgfqpoint{0.959266in}{1.528185in}}{\pgfqpoint{0.965090in}{1.522361in}}%
\pgfpathcurveto{\pgfqpoint{0.970914in}{1.516537in}}{\pgfqpoint{0.978814in}{1.513265in}}{\pgfqpoint{0.987050in}{1.513265in}}%
\pgfpathclose%
\pgfusepath{stroke,fill}%
\end{pgfscope}%
\begin{pgfscope}%
\pgfpathrectangle{\pgfqpoint{0.100000in}{0.212622in}}{\pgfqpoint{3.696000in}{3.696000in}}%
\pgfusepath{clip}%
\pgfsetbuttcap%
\pgfsetroundjoin%
\definecolor{currentfill}{rgb}{0.121569,0.466667,0.705882}%
\pgfsetfillcolor{currentfill}%
\pgfsetfillopacity{0.783829}%
\pgfsetlinewidth{1.003750pt}%
\definecolor{currentstroke}{rgb}{0.121569,0.466667,0.705882}%
\pgfsetstrokecolor{currentstroke}%
\pgfsetstrokeopacity{0.783829}%
\pgfsetdash{}{0pt}%
\pgfpathmoveto{\pgfqpoint{0.987050in}{1.513265in}}%
\pgfpathcurveto{\pgfqpoint{0.995287in}{1.513265in}}{\pgfqpoint{1.003187in}{1.516537in}}{\pgfqpoint{1.009011in}{1.522361in}}%
\pgfpathcurveto{\pgfqpoint{1.014835in}{1.528185in}}{\pgfqpoint{1.018107in}{1.536085in}}{\pgfqpoint{1.018107in}{1.544321in}}%
\pgfpathcurveto{\pgfqpoint{1.018107in}{1.552558in}}{\pgfqpoint{1.014835in}{1.560458in}}{\pgfqpoint{1.009011in}{1.566282in}}%
\pgfpathcurveto{\pgfqpoint{1.003187in}{1.572105in}}{\pgfqpoint{0.995287in}{1.575378in}}{\pgfqpoint{0.987050in}{1.575378in}}%
\pgfpathcurveto{\pgfqpoint{0.978814in}{1.575378in}}{\pgfqpoint{0.970914in}{1.572105in}}{\pgfqpoint{0.965090in}{1.566282in}}%
\pgfpathcurveto{\pgfqpoint{0.959266in}{1.560458in}}{\pgfqpoint{0.955994in}{1.552558in}}{\pgfqpoint{0.955994in}{1.544321in}}%
\pgfpathcurveto{\pgfqpoint{0.955994in}{1.536085in}}{\pgfqpoint{0.959266in}{1.528185in}}{\pgfqpoint{0.965090in}{1.522361in}}%
\pgfpathcurveto{\pgfqpoint{0.970914in}{1.516537in}}{\pgfqpoint{0.978814in}{1.513265in}}{\pgfqpoint{0.987050in}{1.513265in}}%
\pgfpathclose%
\pgfusepath{stroke,fill}%
\end{pgfscope}%
\begin{pgfscope}%
\pgfpathrectangle{\pgfqpoint{0.100000in}{0.212622in}}{\pgfqpoint{3.696000in}{3.696000in}}%
\pgfusepath{clip}%
\pgfsetbuttcap%
\pgfsetroundjoin%
\definecolor{currentfill}{rgb}{0.121569,0.466667,0.705882}%
\pgfsetfillcolor{currentfill}%
\pgfsetfillopacity{0.783829}%
\pgfsetlinewidth{1.003750pt}%
\definecolor{currentstroke}{rgb}{0.121569,0.466667,0.705882}%
\pgfsetstrokecolor{currentstroke}%
\pgfsetstrokeopacity{0.783829}%
\pgfsetdash{}{0pt}%
\pgfpathmoveto{\pgfqpoint{0.987050in}{1.513265in}}%
\pgfpathcurveto{\pgfqpoint{0.995287in}{1.513265in}}{\pgfqpoint{1.003187in}{1.516537in}}{\pgfqpoint{1.009011in}{1.522361in}}%
\pgfpathcurveto{\pgfqpoint{1.014835in}{1.528185in}}{\pgfqpoint{1.018107in}{1.536085in}}{\pgfqpoint{1.018107in}{1.544321in}}%
\pgfpathcurveto{\pgfqpoint{1.018107in}{1.552558in}}{\pgfqpoint{1.014835in}{1.560458in}}{\pgfqpoint{1.009011in}{1.566282in}}%
\pgfpathcurveto{\pgfqpoint{1.003187in}{1.572105in}}{\pgfqpoint{0.995287in}{1.575378in}}{\pgfqpoint{0.987050in}{1.575378in}}%
\pgfpathcurveto{\pgfqpoint{0.978814in}{1.575378in}}{\pgfqpoint{0.970914in}{1.572105in}}{\pgfqpoint{0.965090in}{1.566282in}}%
\pgfpathcurveto{\pgfqpoint{0.959266in}{1.560458in}}{\pgfqpoint{0.955994in}{1.552558in}}{\pgfqpoint{0.955994in}{1.544321in}}%
\pgfpathcurveto{\pgfqpoint{0.955994in}{1.536085in}}{\pgfqpoint{0.959266in}{1.528185in}}{\pgfqpoint{0.965090in}{1.522361in}}%
\pgfpathcurveto{\pgfqpoint{0.970914in}{1.516537in}}{\pgfqpoint{0.978814in}{1.513265in}}{\pgfqpoint{0.987050in}{1.513265in}}%
\pgfpathclose%
\pgfusepath{stroke,fill}%
\end{pgfscope}%
\begin{pgfscope}%
\pgfpathrectangle{\pgfqpoint{0.100000in}{0.212622in}}{\pgfqpoint{3.696000in}{3.696000in}}%
\pgfusepath{clip}%
\pgfsetbuttcap%
\pgfsetroundjoin%
\definecolor{currentfill}{rgb}{0.121569,0.466667,0.705882}%
\pgfsetfillcolor{currentfill}%
\pgfsetfillopacity{0.783829}%
\pgfsetlinewidth{1.003750pt}%
\definecolor{currentstroke}{rgb}{0.121569,0.466667,0.705882}%
\pgfsetstrokecolor{currentstroke}%
\pgfsetstrokeopacity{0.783829}%
\pgfsetdash{}{0pt}%
\pgfpathmoveto{\pgfqpoint{0.987050in}{1.513265in}}%
\pgfpathcurveto{\pgfqpoint{0.995287in}{1.513265in}}{\pgfqpoint{1.003187in}{1.516537in}}{\pgfqpoint{1.009011in}{1.522361in}}%
\pgfpathcurveto{\pgfqpoint{1.014835in}{1.528185in}}{\pgfqpoint{1.018107in}{1.536085in}}{\pgfqpoint{1.018107in}{1.544321in}}%
\pgfpathcurveto{\pgfqpoint{1.018107in}{1.552558in}}{\pgfqpoint{1.014835in}{1.560458in}}{\pgfqpoint{1.009011in}{1.566282in}}%
\pgfpathcurveto{\pgfqpoint{1.003187in}{1.572105in}}{\pgfqpoint{0.995287in}{1.575378in}}{\pgfqpoint{0.987050in}{1.575378in}}%
\pgfpathcurveto{\pgfqpoint{0.978814in}{1.575378in}}{\pgfqpoint{0.970914in}{1.572105in}}{\pgfqpoint{0.965090in}{1.566282in}}%
\pgfpathcurveto{\pgfqpoint{0.959266in}{1.560458in}}{\pgfqpoint{0.955994in}{1.552558in}}{\pgfqpoint{0.955994in}{1.544321in}}%
\pgfpathcurveto{\pgfqpoint{0.955994in}{1.536085in}}{\pgfqpoint{0.959266in}{1.528185in}}{\pgfqpoint{0.965090in}{1.522361in}}%
\pgfpathcurveto{\pgfqpoint{0.970914in}{1.516537in}}{\pgfqpoint{0.978814in}{1.513265in}}{\pgfqpoint{0.987050in}{1.513265in}}%
\pgfpathclose%
\pgfusepath{stroke,fill}%
\end{pgfscope}%
\begin{pgfscope}%
\pgfpathrectangle{\pgfqpoint{0.100000in}{0.212622in}}{\pgfqpoint{3.696000in}{3.696000in}}%
\pgfusepath{clip}%
\pgfsetbuttcap%
\pgfsetroundjoin%
\definecolor{currentfill}{rgb}{0.121569,0.466667,0.705882}%
\pgfsetfillcolor{currentfill}%
\pgfsetfillopacity{0.783829}%
\pgfsetlinewidth{1.003750pt}%
\definecolor{currentstroke}{rgb}{0.121569,0.466667,0.705882}%
\pgfsetstrokecolor{currentstroke}%
\pgfsetstrokeopacity{0.783829}%
\pgfsetdash{}{0pt}%
\pgfpathmoveto{\pgfqpoint{0.987050in}{1.513265in}}%
\pgfpathcurveto{\pgfqpoint{0.995287in}{1.513265in}}{\pgfqpoint{1.003187in}{1.516537in}}{\pgfqpoint{1.009011in}{1.522361in}}%
\pgfpathcurveto{\pgfqpoint{1.014835in}{1.528185in}}{\pgfqpoint{1.018107in}{1.536085in}}{\pgfqpoint{1.018107in}{1.544321in}}%
\pgfpathcurveto{\pgfqpoint{1.018107in}{1.552558in}}{\pgfqpoint{1.014835in}{1.560458in}}{\pgfqpoint{1.009011in}{1.566282in}}%
\pgfpathcurveto{\pgfqpoint{1.003187in}{1.572105in}}{\pgfqpoint{0.995287in}{1.575378in}}{\pgfqpoint{0.987050in}{1.575378in}}%
\pgfpathcurveto{\pgfqpoint{0.978814in}{1.575378in}}{\pgfqpoint{0.970914in}{1.572105in}}{\pgfqpoint{0.965090in}{1.566282in}}%
\pgfpathcurveto{\pgfqpoint{0.959266in}{1.560458in}}{\pgfqpoint{0.955994in}{1.552558in}}{\pgfqpoint{0.955994in}{1.544321in}}%
\pgfpathcurveto{\pgfqpoint{0.955994in}{1.536085in}}{\pgfqpoint{0.959266in}{1.528185in}}{\pgfqpoint{0.965090in}{1.522361in}}%
\pgfpathcurveto{\pgfqpoint{0.970914in}{1.516537in}}{\pgfqpoint{0.978814in}{1.513265in}}{\pgfqpoint{0.987050in}{1.513265in}}%
\pgfpathclose%
\pgfusepath{stroke,fill}%
\end{pgfscope}%
\begin{pgfscope}%
\pgfpathrectangle{\pgfqpoint{0.100000in}{0.212622in}}{\pgfqpoint{3.696000in}{3.696000in}}%
\pgfusepath{clip}%
\pgfsetbuttcap%
\pgfsetroundjoin%
\definecolor{currentfill}{rgb}{0.121569,0.466667,0.705882}%
\pgfsetfillcolor{currentfill}%
\pgfsetfillopacity{0.783829}%
\pgfsetlinewidth{1.003750pt}%
\definecolor{currentstroke}{rgb}{0.121569,0.466667,0.705882}%
\pgfsetstrokecolor{currentstroke}%
\pgfsetstrokeopacity{0.783829}%
\pgfsetdash{}{0pt}%
\pgfpathmoveto{\pgfqpoint{0.987050in}{1.513265in}}%
\pgfpathcurveto{\pgfqpoint{0.995287in}{1.513265in}}{\pgfqpoint{1.003187in}{1.516537in}}{\pgfqpoint{1.009011in}{1.522361in}}%
\pgfpathcurveto{\pgfqpoint{1.014835in}{1.528185in}}{\pgfqpoint{1.018107in}{1.536085in}}{\pgfqpoint{1.018107in}{1.544321in}}%
\pgfpathcurveto{\pgfqpoint{1.018107in}{1.552558in}}{\pgfqpoint{1.014835in}{1.560458in}}{\pgfqpoint{1.009011in}{1.566282in}}%
\pgfpathcurveto{\pgfqpoint{1.003187in}{1.572105in}}{\pgfqpoint{0.995287in}{1.575378in}}{\pgfqpoint{0.987050in}{1.575378in}}%
\pgfpathcurveto{\pgfqpoint{0.978814in}{1.575378in}}{\pgfqpoint{0.970914in}{1.572105in}}{\pgfqpoint{0.965090in}{1.566282in}}%
\pgfpathcurveto{\pgfqpoint{0.959266in}{1.560458in}}{\pgfqpoint{0.955994in}{1.552558in}}{\pgfqpoint{0.955994in}{1.544321in}}%
\pgfpathcurveto{\pgfqpoint{0.955994in}{1.536085in}}{\pgfqpoint{0.959266in}{1.528185in}}{\pgfqpoint{0.965090in}{1.522361in}}%
\pgfpathcurveto{\pgfqpoint{0.970914in}{1.516537in}}{\pgfqpoint{0.978814in}{1.513265in}}{\pgfqpoint{0.987050in}{1.513265in}}%
\pgfpathclose%
\pgfusepath{stroke,fill}%
\end{pgfscope}%
\begin{pgfscope}%
\pgfpathrectangle{\pgfqpoint{0.100000in}{0.212622in}}{\pgfqpoint{3.696000in}{3.696000in}}%
\pgfusepath{clip}%
\pgfsetbuttcap%
\pgfsetroundjoin%
\definecolor{currentfill}{rgb}{0.121569,0.466667,0.705882}%
\pgfsetfillcolor{currentfill}%
\pgfsetfillopacity{0.783829}%
\pgfsetlinewidth{1.003750pt}%
\definecolor{currentstroke}{rgb}{0.121569,0.466667,0.705882}%
\pgfsetstrokecolor{currentstroke}%
\pgfsetstrokeopacity{0.783829}%
\pgfsetdash{}{0pt}%
\pgfpathmoveto{\pgfqpoint{0.987050in}{1.513265in}}%
\pgfpathcurveto{\pgfqpoint{0.995287in}{1.513265in}}{\pgfqpoint{1.003187in}{1.516537in}}{\pgfqpoint{1.009011in}{1.522361in}}%
\pgfpathcurveto{\pgfqpoint{1.014835in}{1.528185in}}{\pgfqpoint{1.018107in}{1.536085in}}{\pgfqpoint{1.018107in}{1.544321in}}%
\pgfpathcurveto{\pgfqpoint{1.018107in}{1.552558in}}{\pgfqpoint{1.014835in}{1.560458in}}{\pgfqpoint{1.009011in}{1.566282in}}%
\pgfpathcurveto{\pgfqpoint{1.003187in}{1.572105in}}{\pgfqpoint{0.995287in}{1.575378in}}{\pgfqpoint{0.987050in}{1.575378in}}%
\pgfpathcurveto{\pgfqpoint{0.978814in}{1.575378in}}{\pgfqpoint{0.970914in}{1.572105in}}{\pgfqpoint{0.965090in}{1.566282in}}%
\pgfpathcurveto{\pgfqpoint{0.959266in}{1.560458in}}{\pgfqpoint{0.955994in}{1.552558in}}{\pgfqpoint{0.955994in}{1.544321in}}%
\pgfpathcurveto{\pgfqpoint{0.955994in}{1.536085in}}{\pgfqpoint{0.959266in}{1.528185in}}{\pgfqpoint{0.965090in}{1.522361in}}%
\pgfpathcurveto{\pgfqpoint{0.970914in}{1.516537in}}{\pgfqpoint{0.978814in}{1.513265in}}{\pgfqpoint{0.987050in}{1.513265in}}%
\pgfpathclose%
\pgfusepath{stroke,fill}%
\end{pgfscope}%
\begin{pgfscope}%
\pgfpathrectangle{\pgfqpoint{0.100000in}{0.212622in}}{\pgfqpoint{3.696000in}{3.696000in}}%
\pgfusepath{clip}%
\pgfsetbuttcap%
\pgfsetroundjoin%
\definecolor{currentfill}{rgb}{0.121569,0.466667,0.705882}%
\pgfsetfillcolor{currentfill}%
\pgfsetfillopacity{0.783829}%
\pgfsetlinewidth{1.003750pt}%
\definecolor{currentstroke}{rgb}{0.121569,0.466667,0.705882}%
\pgfsetstrokecolor{currentstroke}%
\pgfsetstrokeopacity{0.783829}%
\pgfsetdash{}{0pt}%
\pgfpathmoveto{\pgfqpoint{0.987050in}{1.513265in}}%
\pgfpathcurveto{\pgfqpoint{0.995287in}{1.513265in}}{\pgfqpoint{1.003187in}{1.516537in}}{\pgfqpoint{1.009011in}{1.522361in}}%
\pgfpathcurveto{\pgfqpoint{1.014835in}{1.528185in}}{\pgfqpoint{1.018107in}{1.536085in}}{\pgfqpoint{1.018107in}{1.544321in}}%
\pgfpathcurveto{\pgfqpoint{1.018107in}{1.552558in}}{\pgfqpoint{1.014835in}{1.560458in}}{\pgfqpoint{1.009011in}{1.566282in}}%
\pgfpathcurveto{\pgfqpoint{1.003187in}{1.572105in}}{\pgfqpoint{0.995287in}{1.575378in}}{\pgfqpoint{0.987050in}{1.575378in}}%
\pgfpathcurveto{\pgfqpoint{0.978814in}{1.575378in}}{\pgfqpoint{0.970914in}{1.572105in}}{\pgfqpoint{0.965090in}{1.566282in}}%
\pgfpathcurveto{\pgfqpoint{0.959266in}{1.560458in}}{\pgfqpoint{0.955994in}{1.552558in}}{\pgfqpoint{0.955994in}{1.544321in}}%
\pgfpathcurveto{\pgfqpoint{0.955994in}{1.536085in}}{\pgfqpoint{0.959266in}{1.528185in}}{\pgfqpoint{0.965090in}{1.522361in}}%
\pgfpathcurveto{\pgfqpoint{0.970914in}{1.516537in}}{\pgfqpoint{0.978814in}{1.513265in}}{\pgfqpoint{0.987050in}{1.513265in}}%
\pgfpathclose%
\pgfusepath{stroke,fill}%
\end{pgfscope}%
\begin{pgfscope}%
\pgfpathrectangle{\pgfqpoint{0.100000in}{0.212622in}}{\pgfqpoint{3.696000in}{3.696000in}}%
\pgfusepath{clip}%
\pgfsetbuttcap%
\pgfsetroundjoin%
\definecolor{currentfill}{rgb}{0.121569,0.466667,0.705882}%
\pgfsetfillcolor{currentfill}%
\pgfsetfillopacity{0.783829}%
\pgfsetlinewidth{1.003750pt}%
\definecolor{currentstroke}{rgb}{0.121569,0.466667,0.705882}%
\pgfsetstrokecolor{currentstroke}%
\pgfsetstrokeopacity{0.783829}%
\pgfsetdash{}{0pt}%
\pgfpathmoveto{\pgfqpoint{0.987050in}{1.513265in}}%
\pgfpathcurveto{\pgfqpoint{0.995287in}{1.513265in}}{\pgfqpoint{1.003187in}{1.516537in}}{\pgfqpoint{1.009011in}{1.522361in}}%
\pgfpathcurveto{\pgfqpoint{1.014835in}{1.528185in}}{\pgfqpoint{1.018107in}{1.536085in}}{\pgfqpoint{1.018107in}{1.544321in}}%
\pgfpathcurveto{\pgfqpoint{1.018107in}{1.552558in}}{\pgfqpoint{1.014835in}{1.560458in}}{\pgfqpoint{1.009011in}{1.566282in}}%
\pgfpathcurveto{\pgfqpoint{1.003187in}{1.572105in}}{\pgfqpoint{0.995287in}{1.575378in}}{\pgfqpoint{0.987050in}{1.575378in}}%
\pgfpathcurveto{\pgfqpoint{0.978814in}{1.575378in}}{\pgfqpoint{0.970914in}{1.572105in}}{\pgfqpoint{0.965090in}{1.566282in}}%
\pgfpathcurveto{\pgfqpoint{0.959266in}{1.560458in}}{\pgfqpoint{0.955994in}{1.552558in}}{\pgfqpoint{0.955994in}{1.544321in}}%
\pgfpathcurveto{\pgfqpoint{0.955994in}{1.536085in}}{\pgfqpoint{0.959266in}{1.528185in}}{\pgfqpoint{0.965090in}{1.522361in}}%
\pgfpathcurveto{\pgfqpoint{0.970914in}{1.516537in}}{\pgfqpoint{0.978814in}{1.513265in}}{\pgfqpoint{0.987050in}{1.513265in}}%
\pgfpathclose%
\pgfusepath{stroke,fill}%
\end{pgfscope}%
\begin{pgfscope}%
\pgfpathrectangle{\pgfqpoint{0.100000in}{0.212622in}}{\pgfqpoint{3.696000in}{3.696000in}}%
\pgfusepath{clip}%
\pgfsetbuttcap%
\pgfsetroundjoin%
\definecolor{currentfill}{rgb}{0.121569,0.466667,0.705882}%
\pgfsetfillcolor{currentfill}%
\pgfsetfillopacity{0.783829}%
\pgfsetlinewidth{1.003750pt}%
\definecolor{currentstroke}{rgb}{0.121569,0.466667,0.705882}%
\pgfsetstrokecolor{currentstroke}%
\pgfsetstrokeopacity{0.783829}%
\pgfsetdash{}{0pt}%
\pgfpathmoveto{\pgfqpoint{0.987050in}{1.513265in}}%
\pgfpathcurveto{\pgfqpoint{0.995287in}{1.513265in}}{\pgfqpoint{1.003187in}{1.516537in}}{\pgfqpoint{1.009011in}{1.522361in}}%
\pgfpathcurveto{\pgfqpoint{1.014835in}{1.528185in}}{\pgfqpoint{1.018107in}{1.536085in}}{\pgfqpoint{1.018107in}{1.544321in}}%
\pgfpathcurveto{\pgfqpoint{1.018107in}{1.552558in}}{\pgfqpoint{1.014835in}{1.560458in}}{\pgfqpoint{1.009011in}{1.566282in}}%
\pgfpathcurveto{\pgfqpoint{1.003187in}{1.572105in}}{\pgfqpoint{0.995287in}{1.575378in}}{\pgfqpoint{0.987050in}{1.575378in}}%
\pgfpathcurveto{\pgfqpoint{0.978814in}{1.575378in}}{\pgfqpoint{0.970914in}{1.572105in}}{\pgfqpoint{0.965090in}{1.566282in}}%
\pgfpathcurveto{\pgfqpoint{0.959266in}{1.560458in}}{\pgfqpoint{0.955994in}{1.552558in}}{\pgfqpoint{0.955994in}{1.544321in}}%
\pgfpathcurveto{\pgfqpoint{0.955994in}{1.536085in}}{\pgfqpoint{0.959266in}{1.528185in}}{\pgfqpoint{0.965090in}{1.522361in}}%
\pgfpathcurveto{\pgfqpoint{0.970914in}{1.516537in}}{\pgfqpoint{0.978814in}{1.513265in}}{\pgfqpoint{0.987050in}{1.513265in}}%
\pgfpathclose%
\pgfusepath{stroke,fill}%
\end{pgfscope}%
\begin{pgfscope}%
\pgfpathrectangle{\pgfqpoint{0.100000in}{0.212622in}}{\pgfqpoint{3.696000in}{3.696000in}}%
\pgfusepath{clip}%
\pgfsetbuttcap%
\pgfsetroundjoin%
\definecolor{currentfill}{rgb}{0.121569,0.466667,0.705882}%
\pgfsetfillcolor{currentfill}%
\pgfsetfillopacity{0.783829}%
\pgfsetlinewidth{1.003750pt}%
\definecolor{currentstroke}{rgb}{0.121569,0.466667,0.705882}%
\pgfsetstrokecolor{currentstroke}%
\pgfsetstrokeopacity{0.783829}%
\pgfsetdash{}{0pt}%
\pgfpathmoveto{\pgfqpoint{0.987050in}{1.513265in}}%
\pgfpathcurveto{\pgfqpoint{0.995287in}{1.513265in}}{\pgfqpoint{1.003187in}{1.516537in}}{\pgfqpoint{1.009011in}{1.522361in}}%
\pgfpathcurveto{\pgfqpoint{1.014835in}{1.528185in}}{\pgfqpoint{1.018107in}{1.536085in}}{\pgfqpoint{1.018107in}{1.544321in}}%
\pgfpathcurveto{\pgfqpoint{1.018107in}{1.552558in}}{\pgfqpoint{1.014835in}{1.560458in}}{\pgfqpoint{1.009011in}{1.566282in}}%
\pgfpathcurveto{\pgfqpoint{1.003187in}{1.572105in}}{\pgfqpoint{0.995287in}{1.575378in}}{\pgfqpoint{0.987050in}{1.575378in}}%
\pgfpathcurveto{\pgfqpoint{0.978814in}{1.575378in}}{\pgfqpoint{0.970914in}{1.572105in}}{\pgfqpoint{0.965090in}{1.566282in}}%
\pgfpathcurveto{\pgfqpoint{0.959266in}{1.560458in}}{\pgfqpoint{0.955994in}{1.552558in}}{\pgfqpoint{0.955994in}{1.544321in}}%
\pgfpathcurveto{\pgfqpoint{0.955994in}{1.536085in}}{\pgfqpoint{0.959266in}{1.528185in}}{\pgfqpoint{0.965090in}{1.522361in}}%
\pgfpathcurveto{\pgfqpoint{0.970914in}{1.516537in}}{\pgfqpoint{0.978814in}{1.513265in}}{\pgfqpoint{0.987050in}{1.513265in}}%
\pgfpathclose%
\pgfusepath{stroke,fill}%
\end{pgfscope}%
\begin{pgfscope}%
\pgfpathrectangle{\pgfqpoint{0.100000in}{0.212622in}}{\pgfqpoint{3.696000in}{3.696000in}}%
\pgfusepath{clip}%
\pgfsetbuttcap%
\pgfsetroundjoin%
\definecolor{currentfill}{rgb}{0.121569,0.466667,0.705882}%
\pgfsetfillcolor{currentfill}%
\pgfsetfillopacity{0.783829}%
\pgfsetlinewidth{1.003750pt}%
\definecolor{currentstroke}{rgb}{0.121569,0.466667,0.705882}%
\pgfsetstrokecolor{currentstroke}%
\pgfsetstrokeopacity{0.783829}%
\pgfsetdash{}{0pt}%
\pgfpathmoveto{\pgfqpoint{0.987050in}{1.513265in}}%
\pgfpathcurveto{\pgfqpoint{0.995287in}{1.513265in}}{\pgfqpoint{1.003187in}{1.516537in}}{\pgfqpoint{1.009011in}{1.522361in}}%
\pgfpathcurveto{\pgfqpoint{1.014835in}{1.528185in}}{\pgfqpoint{1.018107in}{1.536085in}}{\pgfqpoint{1.018107in}{1.544321in}}%
\pgfpathcurveto{\pgfqpoint{1.018107in}{1.552558in}}{\pgfqpoint{1.014835in}{1.560458in}}{\pgfqpoint{1.009011in}{1.566282in}}%
\pgfpathcurveto{\pgfqpoint{1.003187in}{1.572105in}}{\pgfqpoint{0.995287in}{1.575378in}}{\pgfqpoint{0.987050in}{1.575378in}}%
\pgfpathcurveto{\pgfqpoint{0.978814in}{1.575378in}}{\pgfqpoint{0.970914in}{1.572105in}}{\pgfqpoint{0.965090in}{1.566282in}}%
\pgfpathcurveto{\pgfqpoint{0.959266in}{1.560458in}}{\pgfqpoint{0.955994in}{1.552558in}}{\pgfqpoint{0.955994in}{1.544321in}}%
\pgfpathcurveto{\pgfqpoint{0.955994in}{1.536085in}}{\pgfqpoint{0.959266in}{1.528185in}}{\pgfqpoint{0.965090in}{1.522361in}}%
\pgfpathcurveto{\pgfqpoint{0.970914in}{1.516537in}}{\pgfqpoint{0.978814in}{1.513265in}}{\pgfqpoint{0.987050in}{1.513265in}}%
\pgfpathclose%
\pgfusepath{stroke,fill}%
\end{pgfscope}%
\begin{pgfscope}%
\pgfpathrectangle{\pgfqpoint{0.100000in}{0.212622in}}{\pgfqpoint{3.696000in}{3.696000in}}%
\pgfusepath{clip}%
\pgfsetbuttcap%
\pgfsetroundjoin%
\definecolor{currentfill}{rgb}{0.121569,0.466667,0.705882}%
\pgfsetfillcolor{currentfill}%
\pgfsetfillopacity{0.783829}%
\pgfsetlinewidth{1.003750pt}%
\definecolor{currentstroke}{rgb}{0.121569,0.466667,0.705882}%
\pgfsetstrokecolor{currentstroke}%
\pgfsetstrokeopacity{0.783829}%
\pgfsetdash{}{0pt}%
\pgfpathmoveto{\pgfqpoint{0.987050in}{1.513265in}}%
\pgfpathcurveto{\pgfqpoint{0.995287in}{1.513265in}}{\pgfqpoint{1.003187in}{1.516537in}}{\pgfqpoint{1.009011in}{1.522361in}}%
\pgfpathcurveto{\pgfqpoint{1.014835in}{1.528185in}}{\pgfqpoint{1.018107in}{1.536085in}}{\pgfqpoint{1.018107in}{1.544321in}}%
\pgfpathcurveto{\pgfqpoint{1.018107in}{1.552558in}}{\pgfqpoint{1.014835in}{1.560458in}}{\pgfqpoint{1.009011in}{1.566282in}}%
\pgfpathcurveto{\pgfqpoint{1.003187in}{1.572105in}}{\pgfqpoint{0.995287in}{1.575378in}}{\pgfqpoint{0.987050in}{1.575378in}}%
\pgfpathcurveto{\pgfqpoint{0.978814in}{1.575378in}}{\pgfqpoint{0.970914in}{1.572105in}}{\pgfqpoint{0.965090in}{1.566282in}}%
\pgfpathcurveto{\pgfqpoint{0.959266in}{1.560458in}}{\pgfqpoint{0.955994in}{1.552558in}}{\pgfqpoint{0.955994in}{1.544321in}}%
\pgfpathcurveto{\pgfqpoint{0.955994in}{1.536085in}}{\pgfqpoint{0.959266in}{1.528185in}}{\pgfqpoint{0.965090in}{1.522361in}}%
\pgfpathcurveto{\pgfqpoint{0.970914in}{1.516537in}}{\pgfqpoint{0.978814in}{1.513265in}}{\pgfqpoint{0.987050in}{1.513265in}}%
\pgfpathclose%
\pgfusepath{stroke,fill}%
\end{pgfscope}%
\begin{pgfscope}%
\pgfpathrectangle{\pgfqpoint{0.100000in}{0.212622in}}{\pgfqpoint{3.696000in}{3.696000in}}%
\pgfusepath{clip}%
\pgfsetbuttcap%
\pgfsetroundjoin%
\definecolor{currentfill}{rgb}{0.121569,0.466667,0.705882}%
\pgfsetfillcolor{currentfill}%
\pgfsetfillopacity{0.783829}%
\pgfsetlinewidth{1.003750pt}%
\definecolor{currentstroke}{rgb}{0.121569,0.466667,0.705882}%
\pgfsetstrokecolor{currentstroke}%
\pgfsetstrokeopacity{0.783829}%
\pgfsetdash{}{0pt}%
\pgfpathmoveto{\pgfqpoint{0.987050in}{1.513265in}}%
\pgfpathcurveto{\pgfqpoint{0.995287in}{1.513265in}}{\pgfqpoint{1.003187in}{1.516537in}}{\pgfqpoint{1.009011in}{1.522361in}}%
\pgfpathcurveto{\pgfqpoint{1.014835in}{1.528185in}}{\pgfqpoint{1.018107in}{1.536085in}}{\pgfqpoint{1.018107in}{1.544321in}}%
\pgfpathcurveto{\pgfqpoint{1.018107in}{1.552558in}}{\pgfqpoint{1.014835in}{1.560458in}}{\pgfqpoint{1.009011in}{1.566282in}}%
\pgfpathcurveto{\pgfqpoint{1.003187in}{1.572105in}}{\pgfqpoint{0.995287in}{1.575378in}}{\pgfqpoint{0.987050in}{1.575378in}}%
\pgfpathcurveto{\pgfqpoint{0.978814in}{1.575378in}}{\pgfqpoint{0.970914in}{1.572105in}}{\pgfqpoint{0.965090in}{1.566282in}}%
\pgfpathcurveto{\pgfqpoint{0.959266in}{1.560458in}}{\pgfqpoint{0.955994in}{1.552558in}}{\pgfqpoint{0.955994in}{1.544321in}}%
\pgfpathcurveto{\pgfqpoint{0.955994in}{1.536085in}}{\pgfqpoint{0.959266in}{1.528185in}}{\pgfqpoint{0.965090in}{1.522361in}}%
\pgfpathcurveto{\pgfqpoint{0.970914in}{1.516537in}}{\pgfqpoint{0.978814in}{1.513265in}}{\pgfqpoint{0.987050in}{1.513265in}}%
\pgfpathclose%
\pgfusepath{stroke,fill}%
\end{pgfscope}%
\begin{pgfscope}%
\pgfpathrectangle{\pgfqpoint{0.100000in}{0.212622in}}{\pgfqpoint{3.696000in}{3.696000in}}%
\pgfusepath{clip}%
\pgfsetbuttcap%
\pgfsetroundjoin%
\definecolor{currentfill}{rgb}{0.121569,0.466667,0.705882}%
\pgfsetfillcolor{currentfill}%
\pgfsetfillopacity{0.783829}%
\pgfsetlinewidth{1.003750pt}%
\definecolor{currentstroke}{rgb}{0.121569,0.466667,0.705882}%
\pgfsetstrokecolor{currentstroke}%
\pgfsetstrokeopacity{0.783829}%
\pgfsetdash{}{0pt}%
\pgfpathmoveto{\pgfqpoint{0.987050in}{1.513265in}}%
\pgfpathcurveto{\pgfqpoint{0.995287in}{1.513265in}}{\pgfqpoint{1.003187in}{1.516537in}}{\pgfqpoint{1.009011in}{1.522361in}}%
\pgfpathcurveto{\pgfqpoint{1.014835in}{1.528185in}}{\pgfqpoint{1.018107in}{1.536085in}}{\pgfqpoint{1.018107in}{1.544321in}}%
\pgfpathcurveto{\pgfqpoint{1.018107in}{1.552558in}}{\pgfqpoint{1.014835in}{1.560458in}}{\pgfqpoint{1.009011in}{1.566282in}}%
\pgfpathcurveto{\pgfqpoint{1.003187in}{1.572105in}}{\pgfqpoint{0.995287in}{1.575378in}}{\pgfqpoint{0.987050in}{1.575378in}}%
\pgfpathcurveto{\pgfqpoint{0.978814in}{1.575378in}}{\pgfqpoint{0.970914in}{1.572105in}}{\pgfqpoint{0.965090in}{1.566282in}}%
\pgfpathcurveto{\pgfqpoint{0.959266in}{1.560458in}}{\pgfqpoint{0.955994in}{1.552558in}}{\pgfqpoint{0.955994in}{1.544321in}}%
\pgfpathcurveto{\pgfqpoint{0.955994in}{1.536085in}}{\pgfqpoint{0.959266in}{1.528185in}}{\pgfqpoint{0.965090in}{1.522361in}}%
\pgfpathcurveto{\pgfqpoint{0.970914in}{1.516537in}}{\pgfqpoint{0.978814in}{1.513265in}}{\pgfqpoint{0.987050in}{1.513265in}}%
\pgfpathclose%
\pgfusepath{stroke,fill}%
\end{pgfscope}%
\begin{pgfscope}%
\pgfpathrectangle{\pgfqpoint{0.100000in}{0.212622in}}{\pgfqpoint{3.696000in}{3.696000in}}%
\pgfusepath{clip}%
\pgfsetbuttcap%
\pgfsetroundjoin%
\definecolor{currentfill}{rgb}{0.121569,0.466667,0.705882}%
\pgfsetfillcolor{currentfill}%
\pgfsetfillopacity{0.783829}%
\pgfsetlinewidth{1.003750pt}%
\definecolor{currentstroke}{rgb}{0.121569,0.466667,0.705882}%
\pgfsetstrokecolor{currentstroke}%
\pgfsetstrokeopacity{0.783829}%
\pgfsetdash{}{0pt}%
\pgfpathmoveto{\pgfqpoint{0.987050in}{1.513265in}}%
\pgfpathcurveto{\pgfqpoint{0.995287in}{1.513265in}}{\pgfqpoint{1.003187in}{1.516537in}}{\pgfqpoint{1.009011in}{1.522361in}}%
\pgfpathcurveto{\pgfqpoint{1.014835in}{1.528185in}}{\pgfqpoint{1.018107in}{1.536085in}}{\pgfqpoint{1.018107in}{1.544321in}}%
\pgfpathcurveto{\pgfqpoint{1.018107in}{1.552558in}}{\pgfqpoint{1.014835in}{1.560458in}}{\pgfqpoint{1.009011in}{1.566282in}}%
\pgfpathcurveto{\pgfqpoint{1.003187in}{1.572105in}}{\pgfqpoint{0.995287in}{1.575378in}}{\pgfqpoint{0.987050in}{1.575378in}}%
\pgfpathcurveto{\pgfqpoint{0.978814in}{1.575378in}}{\pgfqpoint{0.970914in}{1.572105in}}{\pgfqpoint{0.965090in}{1.566282in}}%
\pgfpathcurveto{\pgfqpoint{0.959266in}{1.560458in}}{\pgfqpoint{0.955994in}{1.552558in}}{\pgfqpoint{0.955994in}{1.544321in}}%
\pgfpathcurveto{\pgfqpoint{0.955994in}{1.536085in}}{\pgfqpoint{0.959266in}{1.528185in}}{\pgfqpoint{0.965090in}{1.522361in}}%
\pgfpathcurveto{\pgfqpoint{0.970914in}{1.516537in}}{\pgfqpoint{0.978814in}{1.513265in}}{\pgfqpoint{0.987050in}{1.513265in}}%
\pgfpathclose%
\pgfusepath{stroke,fill}%
\end{pgfscope}%
\begin{pgfscope}%
\pgfpathrectangle{\pgfqpoint{0.100000in}{0.212622in}}{\pgfqpoint{3.696000in}{3.696000in}}%
\pgfusepath{clip}%
\pgfsetbuttcap%
\pgfsetroundjoin%
\definecolor{currentfill}{rgb}{0.121569,0.466667,0.705882}%
\pgfsetfillcolor{currentfill}%
\pgfsetfillopacity{0.783829}%
\pgfsetlinewidth{1.003750pt}%
\definecolor{currentstroke}{rgb}{0.121569,0.466667,0.705882}%
\pgfsetstrokecolor{currentstroke}%
\pgfsetstrokeopacity{0.783829}%
\pgfsetdash{}{0pt}%
\pgfpathmoveto{\pgfqpoint{0.987050in}{1.513265in}}%
\pgfpathcurveto{\pgfqpoint{0.995287in}{1.513265in}}{\pgfqpoint{1.003187in}{1.516537in}}{\pgfqpoint{1.009011in}{1.522361in}}%
\pgfpathcurveto{\pgfqpoint{1.014835in}{1.528185in}}{\pgfqpoint{1.018107in}{1.536085in}}{\pgfqpoint{1.018107in}{1.544321in}}%
\pgfpathcurveto{\pgfqpoint{1.018107in}{1.552558in}}{\pgfqpoint{1.014835in}{1.560458in}}{\pgfqpoint{1.009011in}{1.566282in}}%
\pgfpathcurveto{\pgfqpoint{1.003187in}{1.572105in}}{\pgfqpoint{0.995287in}{1.575378in}}{\pgfqpoint{0.987050in}{1.575378in}}%
\pgfpathcurveto{\pgfqpoint{0.978814in}{1.575378in}}{\pgfqpoint{0.970914in}{1.572105in}}{\pgfqpoint{0.965090in}{1.566282in}}%
\pgfpathcurveto{\pgfqpoint{0.959266in}{1.560458in}}{\pgfqpoint{0.955994in}{1.552558in}}{\pgfqpoint{0.955994in}{1.544321in}}%
\pgfpathcurveto{\pgfqpoint{0.955994in}{1.536085in}}{\pgfqpoint{0.959266in}{1.528185in}}{\pgfqpoint{0.965090in}{1.522361in}}%
\pgfpathcurveto{\pgfqpoint{0.970914in}{1.516537in}}{\pgfqpoint{0.978814in}{1.513265in}}{\pgfqpoint{0.987050in}{1.513265in}}%
\pgfpathclose%
\pgfusepath{stroke,fill}%
\end{pgfscope}%
\begin{pgfscope}%
\pgfpathrectangle{\pgfqpoint{0.100000in}{0.212622in}}{\pgfqpoint{3.696000in}{3.696000in}}%
\pgfusepath{clip}%
\pgfsetbuttcap%
\pgfsetroundjoin%
\definecolor{currentfill}{rgb}{0.121569,0.466667,0.705882}%
\pgfsetfillcolor{currentfill}%
\pgfsetfillopacity{0.783829}%
\pgfsetlinewidth{1.003750pt}%
\definecolor{currentstroke}{rgb}{0.121569,0.466667,0.705882}%
\pgfsetstrokecolor{currentstroke}%
\pgfsetstrokeopacity{0.783829}%
\pgfsetdash{}{0pt}%
\pgfpathmoveto{\pgfqpoint{0.987050in}{1.513265in}}%
\pgfpathcurveto{\pgfqpoint{0.995287in}{1.513265in}}{\pgfqpoint{1.003187in}{1.516537in}}{\pgfqpoint{1.009011in}{1.522361in}}%
\pgfpathcurveto{\pgfqpoint{1.014835in}{1.528185in}}{\pgfqpoint{1.018107in}{1.536085in}}{\pgfqpoint{1.018107in}{1.544321in}}%
\pgfpathcurveto{\pgfqpoint{1.018107in}{1.552558in}}{\pgfqpoint{1.014835in}{1.560458in}}{\pgfqpoint{1.009011in}{1.566282in}}%
\pgfpathcurveto{\pgfqpoint{1.003187in}{1.572105in}}{\pgfqpoint{0.995287in}{1.575378in}}{\pgfqpoint{0.987050in}{1.575378in}}%
\pgfpathcurveto{\pgfqpoint{0.978814in}{1.575378in}}{\pgfqpoint{0.970914in}{1.572105in}}{\pgfqpoint{0.965090in}{1.566282in}}%
\pgfpathcurveto{\pgfqpoint{0.959266in}{1.560458in}}{\pgfqpoint{0.955994in}{1.552558in}}{\pgfqpoint{0.955994in}{1.544321in}}%
\pgfpathcurveto{\pgfqpoint{0.955994in}{1.536085in}}{\pgfqpoint{0.959266in}{1.528185in}}{\pgfqpoint{0.965090in}{1.522361in}}%
\pgfpathcurveto{\pgfqpoint{0.970914in}{1.516537in}}{\pgfqpoint{0.978814in}{1.513265in}}{\pgfqpoint{0.987050in}{1.513265in}}%
\pgfpathclose%
\pgfusepath{stroke,fill}%
\end{pgfscope}%
\begin{pgfscope}%
\pgfpathrectangle{\pgfqpoint{0.100000in}{0.212622in}}{\pgfqpoint{3.696000in}{3.696000in}}%
\pgfusepath{clip}%
\pgfsetbuttcap%
\pgfsetroundjoin%
\definecolor{currentfill}{rgb}{0.121569,0.466667,0.705882}%
\pgfsetfillcolor{currentfill}%
\pgfsetfillopacity{0.783829}%
\pgfsetlinewidth{1.003750pt}%
\definecolor{currentstroke}{rgb}{0.121569,0.466667,0.705882}%
\pgfsetstrokecolor{currentstroke}%
\pgfsetstrokeopacity{0.783829}%
\pgfsetdash{}{0pt}%
\pgfpathmoveto{\pgfqpoint{0.987050in}{1.513265in}}%
\pgfpathcurveto{\pgfqpoint{0.995287in}{1.513265in}}{\pgfqpoint{1.003187in}{1.516537in}}{\pgfqpoint{1.009011in}{1.522361in}}%
\pgfpathcurveto{\pgfqpoint{1.014835in}{1.528185in}}{\pgfqpoint{1.018107in}{1.536085in}}{\pgfqpoint{1.018107in}{1.544321in}}%
\pgfpathcurveto{\pgfqpoint{1.018107in}{1.552558in}}{\pgfqpoint{1.014835in}{1.560458in}}{\pgfqpoint{1.009011in}{1.566282in}}%
\pgfpathcurveto{\pgfqpoint{1.003187in}{1.572105in}}{\pgfqpoint{0.995287in}{1.575378in}}{\pgfqpoint{0.987050in}{1.575378in}}%
\pgfpathcurveto{\pgfqpoint{0.978814in}{1.575378in}}{\pgfqpoint{0.970914in}{1.572105in}}{\pgfqpoint{0.965090in}{1.566282in}}%
\pgfpathcurveto{\pgfqpoint{0.959266in}{1.560458in}}{\pgfqpoint{0.955994in}{1.552558in}}{\pgfqpoint{0.955994in}{1.544321in}}%
\pgfpathcurveto{\pgfqpoint{0.955994in}{1.536085in}}{\pgfqpoint{0.959266in}{1.528185in}}{\pgfqpoint{0.965090in}{1.522361in}}%
\pgfpathcurveto{\pgfqpoint{0.970914in}{1.516537in}}{\pgfqpoint{0.978814in}{1.513265in}}{\pgfqpoint{0.987050in}{1.513265in}}%
\pgfpathclose%
\pgfusepath{stroke,fill}%
\end{pgfscope}%
\begin{pgfscope}%
\pgfpathrectangle{\pgfqpoint{0.100000in}{0.212622in}}{\pgfqpoint{3.696000in}{3.696000in}}%
\pgfusepath{clip}%
\pgfsetbuttcap%
\pgfsetroundjoin%
\definecolor{currentfill}{rgb}{0.121569,0.466667,0.705882}%
\pgfsetfillcolor{currentfill}%
\pgfsetfillopacity{0.783829}%
\pgfsetlinewidth{1.003750pt}%
\definecolor{currentstroke}{rgb}{0.121569,0.466667,0.705882}%
\pgfsetstrokecolor{currentstroke}%
\pgfsetstrokeopacity{0.783829}%
\pgfsetdash{}{0pt}%
\pgfpathmoveto{\pgfqpoint{0.987050in}{1.513265in}}%
\pgfpathcurveto{\pgfqpoint{0.995287in}{1.513265in}}{\pgfqpoint{1.003187in}{1.516537in}}{\pgfqpoint{1.009011in}{1.522361in}}%
\pgfpathcurveto{\pgfqpoint{1.014835in}{1.528185in}}{\pgfqpoint{1.018107in}{1.536085in}}{\pgfqpoint{1.018107in}{1.544321in}}%
\pgfpathcurveto{\pgfqpoint{1.018107in}{1.552558in}}{\pgfqpoint{1.014835in}{1.560458in}}{\pgfqpoint{1.009011in}{1.566282in}}%
\pgfpathcurveto{\pgfqpoint{1.003187in}{1.572105in}}{\pgfqpoint{0.995287in}{1.575378in}}{\pgfqpoint{0.987050in}{1.575378in}}%
\pgfpathcurveto{\pgfqpoint{0.978814in}{1.575378in}}{\pgfqpoint{0.970914in}{1.572105in}}{\pgfqpoint{0.965090in}{1.566282in}}%
\pgfpathcurveto{\pgfqpoint{0.959266in}{1.560458in}}{\pgfqpoint{0.955994in}{1.552558in}}{\pgfqpoint{0.955994in}{1.544321in}}%
\pgfpathcurveto{\pgfqpoint{0.955994in}{1.536085in}}{\pgfqpoint{0.959266in}{1.528185in}}{\pgfqpoint{0.965090in}{1.522361in}}%
\pgfpathcurveto{\pgfqpoint{0.970914in}{1.516537in}}{\pgfqpoint{0.978814in}{1.513265in}}{\pgfqpoint{0.987050in}{1.513265in}}%
\pgfpathclose%
\pgfusepath{stroke,fill}%
\end{pgfscope}%
\begin{pgfscope}%
\pgfpathrectangle{\pgfqpoint{0.100000in}{0.212622in}}{\pgfqpoint{3.696000in}{3.696000in}}%
\pgfusepath{clip}%
\pgfsetbuttcap%
\pgfsetroundjoin%
\definecolor{currentfill}{rgb}{0.121569,0.466667,0.705882}%
\pgfsetfillcolor{currentfill}%
\pgfsetfillopacity{0.783829}%
\pgfsetlinewidth{1.003750pt}%
\definecolor{currentstroke}{rgb}{0.121569,0.466667,0.705882}%
\pgfsetstrokecolor{currentstroke}%
\pgfsetstrokeopacity{0.783829}%
\pgfsetdash{}{0pt}%
\pgfpathmoveto{\pgfqpoint{0.987050in}{1.513265in}}%
\pgfpathcurveto{\pgfqpoint{0.995287in}{1.513265in}}{\pgfqpoint{1.003187in}{1.516537in}}{\pgfqpoint{1.009011in}{1.522361in}}%
\pgfpathcurveto{\pgfqpoint{1.014835in}{1.528185in}}{\pgfqpoint{1.018107in}{1.536085in}}{\pgfqpoint{1.018107in}{1.544321in}}%
\pgfpathcurveto{\pgfqpoint{1.018107in}{1.552558in}}{\pgfqpoint{1.014835in}{1.560458in}}{\pgfqpoint{1.009011in}{1.566282in}}%
\pgfpathcurveto{\pgfqpoint{1.003187in}{1.572105in}}{\pgfqpoint{0.995287in}{1.575378in}}{\pgfqpoint{0.987050in}{1.575378in}}%
\pgfpathcurveto{\pgfqpoint{0.978814in}{1.575378in}}{\pgfqpoint{0.970914in}{1.572105in}}{\pgfqpoint{0.965090in}{1.566282in}}%
\pgfpathcurveto{\pgfqpoint{0.959266in}{1.560458in}}{\pgfqpoint{0.955994in}{1.552558in}}{\pgfqpoint{0.955994in}{1.544321in}}%
\pgfpathcurveto{\pgfqpoint{0.955994in}{1.536085in}}{\pgfqpoint{0.959266in}{1.528185in}}{\pgfqpoint{0.965090in}{1.522361in}}%
\pgfpathcurveto{\pgfqpoint{0.970914in}{1.516537in}}{\pgfqpoint{0.978814in}{1.513265in}}{\pgfqpoint{0.987050in}{1.513265in}}%
\pgfpathclose%
\pgfusepath{stroke,fill}%
\end{pgfscope}%
\begin{pgfscope}%
\pgfpathrectangle{\pgfqpoint{0.100000in}{0.212622in}}{\pgfqpoint{3.696000in}{3.696000in}}%
\pgfusepath{clip}%
\pgfsetbuttcap%
\pgfsetroundjoin%
\definecolor{currentfill}{rgb}{0.121569,0.466667,0.705882}%
\pgfsetfillcolor{currentfill}%
\pgfsetfillopacity{0.783829}%
\pgfsetlinewidth{1.003750pt}%
\definecolor{currentstroke}{rgb}{0.121569,0.466667,0.705882}%
\pgfsetstrokecolor{currentstroke}%
\pgfsetstrokeopacity{0.783829}%
\pgfsetdash{}{0pt}%
\pgfpathmoveto{\pgfqpoint{0.987050in}{1.513265in}}%
\pgfpathcurveto{\pgfqpoint{0.995287in}{1.513265in}}{\pgfqpoint{1.003187in}{1.516537in}}{\pgfqpoint{1.009011in}{1.522361in}}%
\pgfpathcurveto{\pgfqpoint{1.014835in}{1.528185in}}{\pgfqpoint{1.018107in}{1.536085in}}{\pgfqpoint{1.018107in}{1.544321in}}%
\pgfpathcurveto{\pgfqpoint{1.018107in}{1.552558in}}{\pgfqpoint{1.014835in}{1.560458in}}{\pgfqpoint{1.009011in}{1.566282in}}%
\pgfpathcurveto{\pgfqpoint{1.003187in}{1.572105in}}{\pgfqpoint{0.995287in}{1.575378in}}{\pgfqpoint{0.987050in}{1.575378in}}%
\pgfpathcurveto{\pgfqpoint{0.978814in}{1.575378in}}{\pgfqpoint{0.970914in}{1.572105in}}{\pgfqpoint{0.965090in}{1.566282in}}%
\pgfpathcurveto{\pgfqpoint{0.959266in}{1.560458in}}{\pgfqpoint{0.955994in}{1.552558in}}{\pgfqpoint{0.955994in}{1.544321in}}%
\pgfpathcurveto{\pgfqpoint{0.955994in}{1.536085in}}{\pgfqpoint{0.959266in}{1.528185in}}{\pgfqpoint{0.965090in}{1.522361in}}%
\pgfpathcurveto{\pgfqpoint{0.970914in}{1.516537in}}{\pgfqpoint{0.978814in}{1.513265in}}{\pgfqpoint{0.987050in}{1.513265in}}%
\pgfpathclose%
\pgfusepath{stroke,fill}%
\end{pgfscope}%
\begin{pgfscope}%
\pgfpathrectangle{\pgfqpoint{0.100000in}{0.212622in}}{\pgfqpoint{3.696000in}{3.696000in}}%
\pgfusepath{clip}%
\pgfsetbuttcap%
\pgfsetroundjoin%
\definecolor{currentfill}{rgb}{0.121569,0.466667,0.705882}%
\pgfsetfillcolor{currentfill}%
\pgfsetfillopacity{0.783829}%
\pgfsetlinewidth{1.003750pt}%
\definecolor{currentstroke}{rgb}{0.121569,0.466667,0.705882}%
\pgfsetstrokecolor{currentstroke}%
\pgfsetstrokeopacity{0.783829}%
\pgfsetdash{}{0pt}%
\pgfpathmoveto{\pgfqpoint{0.987050in}{1.513265in}}%
\pgfpathcurveto{\pgfqpoint{0.995287in}{1.513265in}}{\pgfqpoint{1.003187in}{1.516537in}}{\pgfqpoint{1.009011in}{1.522361in}}%
\pgfpathcurveto{\pgfqpoint{1.014835in}{1.528185in}}{\pgfqpoint{1.018107in}{1.536085in}}{\pgfqpoint{1.018107in}{1.544321in}}%
\pgfpathcurveto{\pgfqpoint{1.018107in}{1.552558in}}{\pgfqpoint{1.014835in}{1.560458in}}{\pgfqpoint{1.009011in}{1.566282in}}%
\pgfpathcurveto{\pgfqpoint{1.003187in}{1.572105in}}{\pgfqpoint{0.995287in}{1.575378in}}{\pgfqpoint{0.987050in}{1.575378in}}%
\pgfpathcurveto{\pgfqpoint{0.978814in}{1.575378in}}{\pgfqpoint{0.970914in}{1.572105in}}{\pgfqpoint{0.965090in}{1.566282in}}%
\pgfpathcurveto{\pgfqpoint{0.959266in}{1.560458in}}{\pgfqpoint{0.955994in}{1.552558in}}{\pgfqpoint{0.955994in}{1.544321in}}%
\pgfpathcurveto{\pgfqpoint{0.955994in}{1.536085in}}{\pgfqpoint{0.959266in}{1.528185in}}{\pgfqpoint{0.965090in}{1.522361in}}%
\pgfpathcurveto{\pgfqpoint{0.970914in}{1.516537in}}{\pgfqpoint{0.978814in}{1.513265in}}{\pgfqpoint{0.987050in}{1.513265in}}%
\pgfpathclose%
\pgfusepath{stroke,fill}%
\end{pgfscope}%
\begin{pgfscope}%
\pgfpathrectangle{\pgfqpoint{0.100000in}{0.212622in}}{\pgfqpoint{3.696000in}{3.696000in}}%
\pgfusepath{clip}%
\pgfsetbuttcap%
\pgfsetroundjoin%
\definecolor{currentfill}{rgb}{0.121569,0.466667,0.705882}%
\pgfsetfillcolor{currentfill}%
\pgfsetfillopacity{0.783945}%
\pgfsetlinewidth{1.003750pt}%
\definecolor{currentstroke}{rgb}{0.121569,0.466667,0.705882}%
\pgfsetstrokecolor{currentstroke}%
\pgfsetstrokeopacity{0.783945}%
\pgfsetdash{}{0pt}%
\pgfpathmoveto{\pgfqpoint{0.986982in}{1.512842in}}%
\pgfpathcurveto{\pgfqpoint{0.995218in}{1.512842in}}{\pgfqpoint{1.003118in}{1.516114in}}{\pgfqpoint{1.008942in}{1.521938in}}%
\pgfpathcurveto{\pgfqpoint{1.014766in}{1.527762in}}{\pgfqpoint{1.018039in}{1.535662in}}{\pgfqpoint{1.018039in}{1.543898in}}%
\pgfpathcurveto{\pgfqpoint{1.018039in}{1.552134in}}{\pgfqpoint{1.014766in}{1.560034in}}{\pgfqpoint{1.008942in}{1.565858in}}%
\pgfpathcurveto{\pgfqpoint{1.003118in}{1.571682in}}{\pgfqpoint{0.995218in}{1.574955in}}{\pgfqpoint{0.986982in}{1.574955in}}%
\pgfpathcurveto{\pgfqpoint{0.978746in}{1.574955in}}{\pgfqpoint{0.970846in}{1.571682in}}{\pgfqpoint{0.965022in}{1.565858in}}%
\pgfpathcurveto{\pgfqpoint{0.959198in}{1.560034in}}{\pgfqpoint{0.955926in}{1.552134in}}{\pgfqpoint{0.955926in}{1.543898in}}%
\pgfpathcurveto{\pgfqpoint{0.955926in}{1.535662in}}{\pgfqpoint{0.959198in}{1.527762in}}{\pgfqpoint{0.965022in}{1.521938in}}%
\pgfpathcurveto{\pgfqpoint{0.970846in}{1.516114in}}{\pgfqpoint{0.978746in}{1.512842in}}{\pgfqpoint{0.986982in}{1.512842in}}%
\pgfpathclose%
\pgfusepath{stroke,fill}%
\end{pgfscope}%
\begin{pgfscope}%
\pgfpathrectangle{\pgfqpoint{0.100000in}{0.212622in}}{\pgfqpoint{3.696000in}{3.696000in}}%
\pgfusepath{clip}%
\pgfsetbuttcap%
\pgfsetroundjoin%
\definecolor{currentfill}{rgb}{0.121569,0.466667,0.705882}%
\pgfsetfillcolor{currentfill}%
\pgfsetfillopacity{0.784012}%
\pgfsetlinewidth{1.003750pt}%
\definecolor{currentstroke}{rgb}{0.121569,0.466667,0.705882}%
\pgfsetstrokecolor{currentstroke}%
\pgfsetstrokeopacity{0.784012}%
\pgfsetdash{}{0pt}%
\pgfpathmoveto{\pgfqpoint{0.986974in}{1.512639in}}%
\pgfpathcurveto{\pgfqpoint{0.995211in}{1.512639in}}{\pgfqpoint{1.003111in}{1.515911in}}{\pgfqpoint{1.008935in}{1.521735in}}%
\pgfpathcurveto{\pgfqpoint{1.014759in}{1.527559in}}{\pgfqpoint{1.018031in}{1.535459in}}{\pgfqpoint{1.018031in}{1.543696in}}%
\pgfpathcurveto{\pgfqpoint{1.018031in}{1.551932in}}{\pgfqpoint{1.014759in}{1.559832in}}{\pgfqpoint{1.008935in}{1.565656in}}%
\pgfpathcurveto{\pgfqpoint{1.003111in}{1.571480in}}{\pgfqpoint{0.995211in}{1.574752in}}{\pgfqpoint{0.986974in}{1.574752in}}%
\pgfpathcurveto{\pgfqpoint{0.978738in}{1.574752in}}{\pgfqpoint{0.970838in}{1.571480in}}{\pgfqpoint{0.965014in}{1.565656in}}%
\pgfpathcurveto{\pgfqpoint{0.959190in}{1.559832in}}{\pgfqpoint{0.955918in}{1.551932in}}{\pgfqpoint{0.955918in}{1.543696in}}%
\pgfpathcurveto{\pgfqpoint{0.955918in}{1.535459in}}{\pgfqpoint{0.959190in}{1.527559in}}{\pgfqpoint{0.965014in}{1.521735in}}%
\pgfpathcurveto{\pgfqpoint{0.970838in}{1.515911in}}{\pgfqpoint{0.978738in}{1.512639in}}{\pgfqpoint{0.986974in}{1.512639in}}%
\pgfpathclose%
\pgfusepath{stroke,fill}%
\end{pgfscope}%
\begin{pgfscope}%
\pgfpathrectangle{\pgfqpoint{0.100000in}{0.212622in}}{\pgfqpoint{3.696000in}{3.696000in}}%
\pgfusepath{clip}%
\pgfsetbuttcap%
\pgfsetroundjoin%
\definecolor{currentfill}{rgb}{0.121569,0.466667,0.705882}%
\pgfsetfillcolor{currentfill}%
\pgfsetfillopacity{0.784189}%
\pgfsetlinewidth{1.003750pt}%
\definecolor{currentstroke}{rgb}{0.121569,0.466667,0.705882}%
\pgfsetstrokecolor{currentstroke}%
\pgfsetstrokeopacity{0.784189}%
\pgfsetdash{}{0pt}%
\pgfpathmoveto{\pgfqpoint{0.987089in}{1.512168in}}%
\pgfpathcurveto{\pgfqpoint{0.995326in}{1.512168in}}{\pgfqpoint{1.003226in}{1.515440in}}{\pgfqpoint{1.009050in}{1.521264in}}%
\pgfpathcurveto{\pgfqpoint{1.014873in}{1.527088in}}{\pgfqpoint{1.018146in}{1.534988in}}{\pgfqpoint{1.018146in}{1.543225in}}%
\pgfpathcurveto{\pgfqpoint{1.018146in}{1.551461in}}{\pgfqpoint{1.014873in}{1.559361in}}{\pgfqpoint{1.009050in}{1.565185in}}%
\pgfpathcurveto{\pgfqpoint{1.003226in}{1.571009in}}{\pgfqpoint{0.995326in}{1.574281in}}{\pgfqpoint{0.987089in}{1.574281in}}%
\pgfpathcurveto{\pgfqpoint{0.978853in}{1.574281in}}{\pgfqpoint{0.970953in}{1.571009in}}{\pgfqpoint{0.965129in}{1.565185in}}%
\pgfpathcurveto{\pgfqpoint{0.959305in}{1.559361in}}{\pgfqpoint{0.956033in}{1.551461in}}{\pgfqpoint{0.956033in}{1.543225in}}%
\pgfpathcurveto{\pgfqpoint{0.956033in}{1.534988in}}{\pgfqpoint{0.959305in}{1.527088in}}{\pgfqpoint{0.965129in}{1.521264in}}%
\pgfpathcurveto{\pgfqpoint{0.970953in}{1.515440in}}{\pgfqpoint{0.978853in}{1.512168in}}{\pgfqpoint{0.987089in}{1.512168in}}%
\pgfpathclose%
\pgfusepath{stroke,fill}%
\end{pgfscope}%
\begin{pgfscope}%
\pgfpathrectangle{\pgfqpoint{0.100000in}{0.212622in}}{\pgfqpoint{3.696000in}{3.696000in}}%
\pgfusepath{clip}%
\pgfsetbuttcap%
\pgfsetroundjoin%
\definecolor{currentfill}{rgb}{0.121569,0.466667,0.705882}%
\pgfsetfillcolor{currentfill}%
\pgfsetfillopacity{0.784261}%
\pgfsetlinewidth{1.003750pt}%
\definecolor{currentstroke}{rgb}{0.121569,0.466667,0.705882}%
\pgfsetstrokecolor{currentstroke}%
\pgfsetstrokeopacity{0.784261}%
\pgfsetdash{}{0pt}%
\pgfpathmoveto{\pgfqpoint{0.987289in}{1.512024in}}%
\pgfpathcurveto{\pgfqpoint{0.995526in}{1.512024in}}{\pgfqpoint{1.003426in}{1.515296in}}{\pgfqpoint{1.009250in}{1.521120in}}%
\pgfpathcurveto{\pgfqpoint{1.015074in}{1.526944in}}{\pgfqpoint{1.018346in}{1.534844in}}{\pgfqpoint{1.018346in}{1.543080in}}%
\pgfpathcurveto{\pgfqpoint{1.018346in}{1.551317in}}{\pgfqpoint{1.015074in}{1.559217in}}{\pgfqpoint{1.009250in}{1.565041in}}%
\pgfpathcurveto{\pgfqpoint{1.003426in}{1.570865in}}{\pgfqpoint{0.995526in}{1.574137in}}{\pgfqpoint{0.987289in}{1.574137in}}%
\pgfpathcurveto{\pgfqpoint{0.979053in}{1.574137in}}{\pgfqpoint{0.971153in}{1.570865in}}{\pgfqpoint{0.965329in}{1.565041in}}%
\pgfpathcurveto{\pgfqpoint{0.959505in}{1.559217in}}{\pgfqpoint{0.956233in}{1.551317in}}{\pgfqpoint{0.956233in}{1.543080in}}%
\pgfpathcurveto{\pgfqpoint{0.956233in}{1.534844in}}{\pgfqpoint{0.959505in}{1.526944in}}{\pgfqpoint{0.965329in}{1.521120in}}%
\pgfpathcurveto{\pgfqpoint{0.971153in}{1.515296in}}{\pgfqpoint{0.979053in}{1.512024in}}{\pgfqpoint{0.987289in}{1.512024in}}%
\pgfpathclose%
\pgfusepath{stroke,fill}%
\end{pgfscope}%
\begin{pgfscope}%
\pgfpathrectangle{\pgfqpoint{0.100000in}{0.212622in}}{\pgfqpoint{3.696000in}{3.696000in}}%
\pgfusepath{clip}%
\pgfsetbuttcap%
\pgfsetroundjoin%
\definecolor{currentfill}{rgb}{0.121569,0.466667,0.705882}%
\pgfsetfillcolor{currentfill}%
\pgfsetfillopacity{0.784294}%
\pgfsetlinewidth{1.003750pt}%
\definecolor{currentstroke}{rgb}{0.121569,0.466667,0.705882}%
\pgfsetstrokecolor{currentstroke}%
\pgfsetstrokeopacity{0.784294}%
\pgfsetdash{}{0pt}%
\pgfpathmoveto{\pgfqpoint{0.987420in}{1.511971in}}%
\pgfpathcurveto{\pgfqpoint{0.995656in}{1.511971in}}{\pgfqpoint{1.003557in}{1.515244in}}{\pgfqpoint{1.009380in}{1.521068in}}%
\pgfpathcurveto{\pgfqpoint{1.015204in}{1.526891in}}{\pgfqpoint{1.018477in}{1.534792in}}{\pgfqpoint{1.018477in}{1.543028in}}%
\pgfpathcurveto{\pgfqpoint{1.018477in}{1.551264in}}{\pgfqpoint{1.015204in}{1.559164in}}{\pgfqpoint{1.009380in}{1.564988in}}%
\pgfpathcurveto{\pgfqpoint{1.003557in}{1.570812in}}{\pgfqpoint{0.995656in}{1.574084in}}{\pgfqpoint{0.987420in}{1.574084in}}%
\pgfpathcurveto{\pgfqpoint{0.979184in}{1.574084in}}{\pgfqpoint{0.971284in}{1.570812in}}{\pgfqpoint{0.965460in}{1.564988in}}%
\pgfpathcurveto{\pgfqpoint{0.959636in}{1.559164in}}{\pgfqpoint{0.956364in}{1.551264in}}{\pgfqpoint{0.956364in}{1.543028in}}%
\pgfpathcurveto{\pgfqpoint{0.956364in}{1.534792in}}{\pgfqpoint{0.959636in}{1.526891in}}{\pgfqpoint{0.965460in}{1.521068in}}%
\pgfpathcurveto{\pgfqpoint{0.971284in}{1.515244in}}{\pgfqpoint{0.979184in}{1.511971in}}{\pgfqpoint{0.987420in}{1.511971in}}%
\pgfpathclose%
\pgfusepath{stroke,fill}%
\end{pgfscope}%
\begin{pgfscope}%
\pgfpathrectangle{\pgfqpoint{0.100000in}{0.212622in}}{\pgfqpoint{3.696000in}{3.696000in}}%
\pgfusepath{clip}%
\pgfsetbuttcap%
\pgfsetroundjoin%
\definecolor{currentfill}{rgb}{0.121569,0.466667,0.705882}%
\pgfsetfillcolor{currentfill}%
\pgfsetfillopacity{0.784403}%
\pgfsetlinewidth{1.003750pt}%
\definecolor{currentstroke}{rgb}{0.121569,0.466667,0.705882}%
\pgfsetstrokecolor{currentstroke}%
\pgfsetstrokeopacity{0.784403}%
\pgfsetdash{}{0pt}%
\pgfpathmoveto{\pgfqpoint{2.553877in}{2.902329in}}%
\pgfpathcurveto{\pgfqpoint{2.562113in}{2.902329in}}{\pgfqpoint{2.570013in}{2.905601in}}{\pgfqpoint{2.575837in}{2.911425in}}%
\pgfpathcurveto{\pgfqpoint{2.581661in}{2.917249in}}{\pgfqpoint{2.584933in}{2.925149in}}{\pgfqpoint{2.584933in}{2.933385in}}%
\pgfpathcurveto{\pgfqpoint{2.584933in}{2.941621in}}{\pgfqpoint{2.581661in}{2.949522in}}{\pgfqpoint{2.575837in}{2.955345in}}%
\pgfpathcurveto{\pgfqpoint{2.570013in}{2.961169in}}{\pgfqpoint{2.562113in}{2.964442in}}{\pgfqpoint{2.553877in}{2.964442in}}%
\pgfpathcurveto{\pgfqpoint{2.545641in}{2.964442in}}{\pgfqpoint{2.537741in}{2.961169in}}{\pgfqpoint{2.531917in}{2.955345in}}%
\pgfpathcurveto{\pgfqpoint{2.526093in}{2.949522in}}{\pgfqpoint{2.522820in}{2.941621in}}{\pgfqpoint{2.522820in}{2.933385in}}%
\pgfpathcurveto{\pgfqpoint{2.522820in}{2.925149in}}{\pgfqpoint{2.526093in}{2.917249in}}{\pgfqpoint{2.531917in}{2.911425in}}%
\pgfpathcurveto{\pgfqpoint{2.537741in}{2.905601in}}{\pgfqpoint{2.545641in}{2.902329in}}{\pgfqpoint{2.553877in}{2.902329in}}%
\pgfpathclose%
\pgfusepath{stroke,fill}%
\end{pgfscope}%
\begin{pgfscope}%
\pgfpathrectangle{\pgfqpoint{0.100000in}{0.212622in}}{\pgfqpoint{3.696000in}{3.696000in}}%
\pgfusepath{clip}%
\pgfsetbuttcap%
\pgfsetroundjoin%
\definecolor{currentfill}{rgb}{0.121569,0.466667,0.705882}%
\pgfsetfillcolor{currentfill}%
\pgfsetfillopacity{0.784422}%
\pgfsetlinewidth{1.003750pt}%
\definecolor{currentstroke}{rgb}{0.121569,0.466667,0.705882}%
\pgfsetstrokecolor{currentstroke}%
\pgfsetstrokeopacity{0.784422}%
\pgfsetdash{}{0pt}%
\pgfpathmoveto{\pgfqpoint{0.987911in}{1.511761in}}%
\pgfpathcurveto{\pgfqpoint{0.996147in}{1.511761in}}{\pgfqpoint{1.004048in}{1.515034in}}{\pgfqpoint{1.009871in}{1.520858in}}%
\pgfpathcurveto{\pgfqpoint{1.015695in}{1.526681in}}{\pgfqpoint{1.018968in}{1.534581in}}{\pgfqpoint{1.018968in}{1.542818in}}%
\pgfpathcurveto{\pgfqpoint{1.018968in}{1.551054in}}{\pgfqpoint{1.015695in}{1.558954in}}{\pgfqpoint{1.009871in}{1.564778in}}%
\pgfpathcurveto{\pgfqpoint{1.004048in}{1.570602in}}{\pgfqpoint{0.996147in}{1.573874in}}{\pgfqpoint{0.987911in}{1.573874in}}%
\pgfpathcurveto{\pgfqpoint{0.979675in}{1.573874in}}{\pgfqpoint{0.971775in}{1.570602in}}{\pgfqpoint{0.965951in}{1.564778in}}%
\pgfpathcurveto{\pgfqpoint{0.960127in}{1.558954in}}{\pgfqpoint{0.956855in}{1.551054in}}{\pgfqpoint{0.956855in}{1.542818in}}%
\pgfpathcurveto{\pgfqpoint{0.956855in}{1.534581in}}{\pgfqpoint{0.960127in}{1.526681in}}{\pgfqpoint{0.965951in}{1.520858in}}%
\pgfpathcurveto{\pgfqpoint{0.971775in}{1.515034in}}{\pgfqpoint{0.979675in}{1.511761in}}{\pgfqpoint{0.987911in}{1.511761in}}%
\pgfpathclose%
\pgfusepath{stroke,fill}%
\end{pgfscope}%
\begin{pgfscope}%
\pgfpathrectangle{\pgfqpoint{0.100000in}{0.212622in}}{\pgfqpoint{3.696000in}{3.696000in}}%
\pgfusepath{clip}%
\pgfsetbuttcap%
\pgfsetroundjoin%
\definecolor{currentfill}{rgb}{0.121569,0.466667,0.705882}%
\pgfsetfillcolor{currentfill}%
\pgfsetfillopacity{0.784688}%
\pgfsetlinewidth{1.003750pt}%
\definecolor{currentstroke}{rgb}{0.121569,0.466667,0.705882}%
\pgfsetstrokecolor{currentstroke}%
\pgfsetstrokeopacity{0.784688}%
\pgfsetdash{}{0pt}%
\pgfpathmoveto{\pgfqpoint{0.989015in}{1.511369in}}%
\pgfpathcurveto{\pgfqpoint{0.997251in}{1.511369in}}{\pgfqpoint{1.005151in}{1.514641in}}{\pgfqpoint{1.010975in}{1.520465in}}%
\pgfpathcurveto{\pgfqpoint{1.016799in}{1.526289in}}{\pgfqpoint{1.020071in}{1.534189in}}{\pgfqpoint{1.020071in}{1.542426in}}%
\pgfpathcurveto{\pgfqpoint{1.020071in}{1.550662in}}{\pgfqpoint{1.016799in}{1.558562in}}{\pgfqpoint{1.010975in}{1.564386in}}%
\pgfpathcurveto{\pgfqpoint{1.005151in}{1.570210in}}{\pgfqpoint{0.997251in}{1.573482in}}{\pgfqpoint{0.989015in}{1.573482in}}%
\pgfpathcurveto{\pgfqpoint{0.980778in}{1.573482in}}{\pgfqpoint{0.972878in}{1.570210in}}{\pgfqpoint{0.967054in}{1.564386in}}%
\pgfpathcurveto{\pgfqpoint{0.961231in}{1.558562in}}{\pgfqpoint{0.957958in}{1.550662in}}{\pgfqpoint{0.957958in}{1.542426in}}%
\pgfpathcurveto{\pgfqpoint{0.957958in}{1.534189in}}{\pgfqpoint{0.961231in}{1.526289in}}{\pgfqpoint{0.967054in}{1.520465in}}%
\pgfpathcurveto{\pgfqpoint{0.972878in}{1.514641in}}{\pgfqpoint{0.980778in}{1.511369in}}{\pgfqpoint{0.989015in}{1.511369in}}%
\pgfpathclose%
\pgfusepath{stroke,fill}%
\end{pgfscope}%
\begin{pgfscope}%
\pgfpathrectangle{\pgfqpoint{0.100000in}{0.212622in}}{\pgfqpoint{3.696000in}{3.696000in}}%
\pgfusepath{clip}%
\pgfsetbuttcap%
\pgfsetroundjoin%
\definecolor{currentfill}{rgb}{0.121569,0.466667,0.705882}%
\pgfsetfillcolor{currentfill}%
\pgfsetfillopacity{0.784873}%
\pgfsetlinewidth{1.003750pt}%
\definecolor{currentstroke}{rgb}{0.121569,0.466667,0.705882}%
\pgfsetstrokecolor{currentstroke}%
\pgfsetstrokeopacity{0.784873}%
\pgfsetdash{}{0pt}%
\pgfpathmoveto{\pgfqpoint{2.552485in}{2.898692in}}%
\pgfpathcurveto{\pgfqpoint{2.560721in}{2.898692in}}{\pgfqpoint{2.568622in}{2.901964in}}{\pgfqpoint{2.574445in}{2.907788in}}%
\pgfpathcurveto{\pgfqpoint{2.580269in}{2.913612in}}{\pgfqpoint{2.583542in}{2.921512in}}{\pgfqpoint{2.583542in}{2.929748in}}%
\pgfpathcurveto{\pgfqpoint{2.583542in}{2.937984in}}{\pgfqpoint{2.580269in}{2.945884in}}{\pgfqpoint{2.574445in}{2.951708in}}%
\pgfpathcurveto{\pgfqpoint{2.568622in}{2.957532in}}{\pgfqpoint{2.560721in}{2.960805in}}{\pgfqpoint{2.552485in}{2.960805in}}%
\pgfpathcurveto{\pgfqpoint{2.544249in}{2.960805in}}{\pgfqpoint{2.536349in}{2.957532in}}{\pgfqpoint{2.530525in}{2.951708in}}%
\pgfpathcurveto{\pgfqpoint{2.524701in}{2.945884in}}{\pgfqpoint{2.521429in}{2.937984in}}{\pgfqpoint{2.521429in}{2.929748in}}%
\pgfpathcurveto{\pgfqpoint{2.521429in}{2.921512in}}{\pgfqpoint{2.524701in}{2.913612in}}{\pgfqpoint{2.530525in}{2.907788in}}%
\pgfpathcurveto{\pgfqpoint{2.536349in}{2.901964in}}{\pgfqpoint{2.544249in}{2.898692in}}{\pgfqpoint{2.552485in}{2.898692in}}%
\pgfpathclose%
\pgfusepath{stroke,fill}%
\end{pgfscope}%
\begin{pgfscope}%
\pgfpathrectangle{\pgfqpoint{0.100000in}{0.212622in}}{\pgfqpoint{3.696000in}{3.696000in}}%
\pgfusepath{clip}%
\pgfsetbuttcap%
\pgfsetroundjoin%
\definecolor{currentfill}{rgb}{0.121569,0.466667,0.705882}%
\pgfsetfillcolor{currentfill}%
\pgfsetfillopacity{0.784983}%
\pgfsetlinewidth{1.003750pt}%
\definecolor{currentstroke}{rgb}{0.121569,0.466667,0.705882}%
\pgfsetstrokecolor{currentstroke}%
\pgfsetstrokeopacity{0.784983}%
\pgfsetdash{}{0pt}%
\pgfpathmoveto{\pgfqpoint{0.990485in}{1.510874in}}%
\pgfpathcurveto{\pgfqpoint{0.998722in}{1.510874in}}{\pgfqpoint{1.006622in}{1.514146in}}{\pgfqpoint{1.012446in}{1.519970in}}%
\pgfpathcurveto{\pgfqpoint{1.018269in}{1.525794in}}{\pgfqpoint{1.021542in}{1.533694in}}{\pgfqpoint{1.021542in}{1.541930in}}%
\pgfpathcurveto{\pgfqpoint{1.021542in}{1.550167in}}{\pgfqpoint{1.018269in}{1.558067in}}{\pgfqpoint{1.012446in}{1.563891in}}%
\pgfpathcurveto{\pgfqpoint{1.006622in}{1.569714in}}{\pgfqpoint{0.998722in}{1.572987in}}{\pgfqpoint{0.990485in}{1.572987in}}%
\pgfpathcurveto{\pgfqpoint{0.982249in}{1.572987in}}{\pgfqpoint{0.974349in}{1.569714in}}{\pgfqpoint{0.968525in}{1.563891in}}%
\pgfpathcurveto{\pgfqpoint{0.962701in}{1.558067in}}{\pgfqpoint{0.959429in}{1.550167in}}{\pgfqpoint{0.959429in}{1.541930in}}%
\pgfpathcurveto{\pgfqpoint{0.959429in}{1.533694in}}{\pgfqpoint{0.962701in}{1.525794in}}{\pgfqpoint{0.968525in}{1.519970in}}%
\pgfpathcurveto{\pgfqpoint{0.974349in}{1.514146in}}{\pgfqpoint{0.982249in}{1.510874in}}{\pgfqpoint{0.990485in}{1.510874in}}%
\pgfpathclose%
\pgfusepath{stroke,fill}%
\end{pgfscope}%
\begin{pgfscope}%
\pgfpathrectangle{\pgfqpoint{0.100000in}{0.212622in}}{\pgfqpoint{3.696000in}{3.696000in}}%
\pgfusepath{clip}%
\pgfsetbuttcap%
\pgfsetroundjoin%
\definecolor{currentfill}{rgb}{0.121569,0.466667,0.705882}%
\pgfsetfillcolor{currentfill}%
\pgfsetfillopacity{0.785147}%
\pgfsetlinewidth{1.003750pt}%
\definecolor{currentstroke}{rgb}{0.121569,0.466667,0.705882}%
\pgfsetstrokecolor{currentstroke}%
\pgfsetstrokeopacity{0.785147}%
\pgfsetdash{}{0pt}%
\pgfpathmoveto{\pgfqpoint{2.551196in}{2.896611in}}%
\pgfpathcurveto{\pgfqpoint{2.559433in}{2.896611in}}{\pgfqpoint{2.567333in}{2.899883in}}{\pgfqpoint{2.573157in}{2.905707in}}%
\pgfpathcurveto{\pgfqpoint{2.578980in}{2.911531in}}{\pgfqpoint{2.582253in}{2.919431in}}{\pgfqpoint{2.582253in}{2.927667in}}%
\pgfpathcurveto{\pgfqpoint{2.582253in}{2.935903in}}{\pgfqpoint{2.578980in}{2.943803in}}{\pgfqpoint{2.573157in}{2.949627in}}%
\pgfpathcurveto{\pgfqpoint{2.567333in}{2.955451in}}{\pgfqpoint{2.559433in}{2.958724in}}{\pgfqpoint{2.551196in}{2.958724in}}%
\pgfpathcurveto{\pgfqpoint{2.542960in}{2.958724in}}{\pgfqpoint{2.535060in}{2.955451in}}{\pgfqpoint{2.529236in}{2.949627in}}%
\pgfpathcurveto{\pgfqpoint{2.523412in}{2.943803in}}{\pgfqpoint{2.520140in}{2.935903in}}{\pgfqpoint{2.520140in}{2.927667in}}%
\pgfpathcurveto{\pgfqpoint{2.520140in}{2.919431in}}{\pgfqpoint{2.523412in}{2.911531in}}{\pgfqpoint{2.529236in}{2.905707in}}%
\pgfpathcurveto{\pgfqpoint{2.535060in}{2.899883in}}{\pgfqpoint{2.542960in}{2.896611in}}{\pgfqpoint{2.551196in}{2.896611in}}%
\pgfpathclose%
\pgfusepath{stroke,fill}%
\end{pgfscope}%
\begin{pgfscope}%
\pgfpathrectangle{\pgfqpoint{0.100000in}{0.212622in}}{\pgfqpoint{3.696000in}{3.696000in}}%
\pgfusepath{clip}%
\pgfsetbuttcap%
\pgfsetroundjoin%
\definecolor{currentfill}{rgb}{0.121569,0.466667,0.705882}%
\pgfsetfillcolor{currentfill}%
\pgfsetfillopacity{0.785404}%
\pgfsetlinewidth{1.003750pt}%
\definecolor{currentstroke}{rgb}{0.121569,0.466667,0.705882}%
\pgfsetstrokecolor{currentstroke}%
\pgfsetstrokeopacity{0.785404}%
\pgfsetdash{}{0pt}%
\pgfpathmoveto{\pgfqpoint{3.095022in}{2.028831in}}%
\pgfpathcurveto{\pgfqpoint{3.103258in}{2.028831in}}{\pgfqpoint{3.111158in}{2.032103in}}{\pgfqpoint{3.116982in}{2.037927in}}%
\pgfpathcurveto{\pgfqpoint{3.122806in}{2.043751in}}{\pgfqpoint{3.126078in}{2.051651in}}{\pgfqpoint{3.126078in}{2.059887in}}%
\pgfpathcurveto{\pgfqpoint{3.126078in}{2.068123in}}{\pgfqpoint{3.122806in}{2.076023in}}{\pgfqpoint{3.116982in}{2.081847in}}%
\pgfpathcurveto{\pgfqpoint{3.111158in}{2.087671in}}{\pgfqpoint{3.103258in}{2.090944in}}{\pgfqpoint{3.095022in}{2.090944in}}%
\pgfpathcurveto{\pgfqpoint{3.086785in}{2.090944in}}{\pgfqpoint{3.078885in}{2.087671in}}{\pgfqpoint{3.073061in}{2.081847in}}%
\pgfpathcurveto{\pgfqpoint{3.067237in}{2.076023in}}{\pgfqpoint{3.063965in}{2.068123in}}{\pgfqpoint{3.063965in}{2.059887in}}%
\pgfpathcurveto{\pgfqpoint{3.063965in}{2.051651in}}{\pgfqpoint{3.067237in}{2.043751in}}{\pgfqpoint{3.073061in}{2.037927in}}%
\pgfpathcurveto{\pgfqpoint{3.078885in}{2.032103in}}{\pgfqpoint{3.086785in}{2.028831in}}{\pgfqpoint{3.095022in}{2.028831in}}%
\pgfpathclose%
\pgfusepath{stroke,fill}%
\end{pgfscope}%
\begin{pgfscope}%
\pgfpathrectangle{\pgfqpoint{0.100000in}{0.212622in}}{\pgfqpoint{3.696000in}{3.696000in}}%
\pgfusepath{clip}%
\pgfsetbuttcap%
\pgfsetroundjoin%
\definecolor{currentfill}{rgb}{0.121569,0.466667,0.705882}%
\pgfsetfillcolor{currentfill}%
\pgfsetfillopacity{0.785443}%
\pgfsetlinewidth{1.003750pt}%
\definecolor{currentstroke}{rgb}{0.121569,0.466667,0.705882}%
\pgfsetstrokecolor{currentstroke}%
\pgfsetstrokeopacity{0.785443}%
\pgfsetdash{}{0pt}%
\pgfpathmoveto{\pgfqpoint{0.992113in}{1.510132in}}%
\pgfpathcurveto{\pgfqpoint{1.000349in}{1.510132in}}{\pgfqpoint{1.008249in}{1.513404in}}{\pgfqpoint{1.014073in}{1.519228in}}%
\pgfpathcurveto{\pgfqpoint{1.019897in}{1.525052in}}{\pgfqpoint{1.023169in}{1.532952in}}{\pgfqpoint{1.023169in}{1.541189in}}%
\pgfpathcurveto{\pgfqpoint{1.023169in}{1.549425in}}{\pgfqpoint{1.019897in}{1.557325in}}{\pgfqpoint{1.014073in}{1.563149in}}%
\pgfpathcurveto{\pgfqpoint{1.008249in}{1.568973in}}{\pgfqpoint{1.000349in}{1.572245in}}{\pgfqpoint{0.992113in}{1.572245in}}%
\pgfpathcurveto{\pgfqpoint{0.983876in}{1.572245in}}{\pgfqpoint{0.975976in}{1.568973in}}{\pgfqpoint{0.970153in}{1.563149in}}%
\pgfpathcurveto{\pgfqpoint{0.964329in}{1.557325in}}{\pgfqpoint{0.961056in}{1.549425in}}{\pgfqpoint{0.961056in}{1.541189in}}%
\pgfpathcurveto{\pgfqpoint{0.961056in}{1.532952in}}{\pgfqpoint{0.964329in}{1.525052in}}{\pgfqpoint{0.970153in}{1.519228in}}%
\pgfpathcurveto{\pgfqpoint{0.975976in}{1.513404in}}{\pgfqpoint{0.983876in}{1.510132in}}{\pgfqpoint{0.992113in}{1.510132in}}%
\pgfpathclose%
\pgfusepath{stroke,fill}%
\end{pgfscope}%
\begin{pgfscope}%
\pgfpathrectangle{\pgfqpoint{0.100000in}{0.212622in}}{\pgfqpoint{3.696000in}{3.696000in}}%
\pgfusepath{clip}%
\pgfsetbuttcap%
\pgfsetroundjoin%
\definecolor{currentfill}{rgb}{0.121569,0.466667,0.705882}%
\pgfsetfillcolor{currentfill}%
\pgfsetfillopacity{0.785662}%
\pgfsetlinewidth{1.003750pt}%
\definecolor{currentstroke}{rgb}{0.121569,0.466667,0.705882}%
\pgfsetstrokecolor{currentstroke}%
\pgfsetstrokeopacity{0.785662}%
\pgfsetdash{}{0pt}%
\pgfpathmoveto{\pgfqpoint{2.548711in}{2.893228in}}%
\pgfpathcurveto{\pgfqpoint{2.556947in}{2.893228in}}{\pgfqpoint{2.564847in}{2.896500in}}{\pgfqpoint{2.570671in}{2.902324in}}%
\pgfpathcurveto{\pgfqpoint{2.576495in}{2.908148in}}{\pgfqpoint{2.579767in}{2.916048in}}{\pgfqpoint{2.579767in}{2.924284in}}%
\pgfpathcurveto{\pgfqpoint{2.579767in}{2.932520in}}{\pgfqpoint{2.576495in}{2.940420in}}{\pgfqpoint{2.570671in}{2.946244in}}%
\pgfpathcurveto{\pgfqpoint{2.564847in}{2.952068in}}{\pgfqpoint{2.556947in}{2.955341in}}{\pgfqpoint{2.548711in}{2.955341in}}%
\pgfpathcurveto{\pgfqpoint{2.540475in}{2.955341in}}{\pgfqpoint{2.532575in}{2.952068in}}{\pgfqpoint{2.526751in}{2.946244in}}%
\pgfpathcurveto{\pgfqpoint{2.520927in}{2.940420in}}{\pgfqpoint{2.517654in}{2.932520in}}{\pgfqpoint{2.517654in}{2.924284in}}%
\pgfpathcurveto{\pgfqpoint{2.517654in}{2.916048in}}{\pgfqpoint{2.520927in}{2.908148in}}{\pgfqpoint{2.526751in}{2.902324in}}%
\pgfpathcurveto{\pgfqpoint{2.532575in}{2.896500in}}{\pgfqpoint{2.540475in}{2.893228in}}{\pgfqpoint{2.548711in}{2.893228in}}%
\pgfpathclose%
\pgfusepath{stroke,fill}%
\end{pgfscope}%
\begin{pgfscope}%
\pgfpathrectangle{\pgfqpoint{0.100000in}{0.212622in}}{\pgfqpoint{3.696000in}{3.696000in}}%
\pgfusepath{clip}%
\pgfsetbuttcap%
\pgfsetroundjoin%
\definecolor{currentfill}{rgb}{0.121569,0.466667,0.705882}%
\pgfsetfillcolor{currentfill}%
\pgfsetfillopacity{0.785867}%
\pgfsetlinewidth{1.003750pt}%
\definecolor{currentstroke}{rgb}{0.121569,0.466667,0.705882}%
\pgfsetstrokecolor{currentstroke}%
\pgfsetstrokeopacity{0.785867}%
\pgfsetdash{}{0pt}%
\pgfpathmoveto{\pgfqpoint{0.994169in}{1.509231in}}%
\pgfpathcurveto{\pgfqpoint{1.002405in}{1.509231in}}{\pgfqpoint{1.010305in}{1.512503in}}{\pgfqpoint{1.016129in}{1.518327in}}%
\pgfpathcurveto{\pgfqpoint{1.021953in}{1.524151in}}{\pgfqpoint{1.025226in}{1.532051in}}{\pgfqpoint{1.025226in}{1.540287in}}%
\pgfpathcurveto{\pgfqpoint{1.025226in}{1.548523in}}{\pgfqpoint{1.021953in}{1.556423in}}{\pgfqpoint{1.016129in}{1.562247in}}%
\pgfpathcurveto{\pgfqpoint{1.010305in}{1.568071in}}{\pgfqpoint{1.002405in}{1.571344in}}{\pgfqpoint{0.994169in}{1.571344in}}%
\pgfpathcurveto{\pgfqpoint{0.985933in}{1.571344in}}{\pgfqpoint{0.978033in}{1.568071in}}{\pgfqpoint{0.972209in}{1.562247in}}%
\pgfpathcurveto{\pgfqpoint{0.966385in}{1.556423in}}{\pgfqpoint{0.963113in}{1.548523in}}{\pgfqpoint{0.963113in}{1.540287in}}%
\pgfpathcurveto{\pgfqpoint{0.963113in}{1.532051in}}{\pgfqpoint{0.966385in}{1.524151in}}{\pgfqpoint{0.972209in}{1.518327in}}%
\pgfpathcurveto{\pgfqpoint{0.978033in}{1.512503in}}{\pgfqpoint{0.985933in}{1.509231in}}{\pgfqpoint{0.994169in}{1.509231in}}%
\pgfpathclose%
\pgfusepath{stroke,fill}%
\end{pgfscope}%
\begin{pgfscope}%
\pgfpathrectangle{\pgfqpoint{0.100000in}{0.212622in}}{\pgfqpoint{3.696000in}{3.696000in}}%
\pgfusepath{clip}%
\pgfsetbuttcap%
\pgfsetroundjoin%
\definecolor{currentfill}{rgb}{0.121569,0.466667,0.705882}%
\pgfsetfillcolor{currentfill}%
\pgfsetfillopacity{0.786366}%
\pgfsetlinewidth{1.003750pt}%
\definecolor{currentstroke}{rgb}{0.121569,0.466667,0.705882}%
\pgfsetstrokecolor{currentstroke}%
\pgfsetstrokeopacity{0.786366}%
\pgfsetdash{}{0pt}%
\pgfpathmoveto{\pgfqpoint{0.997861in}{1.507939in}}%
\pgfpathcurveto{\pgfqpoint{1.006097in}{1.507939in}}{\pgfqpoint{1.013997in}{1.511211in}}{\pgfqpoint{1.019821in}{1.517035in}}%
\pgfpathcurveto{\pgfqpoint{1.025645in}{1.522859in}}{\pgfqpoint{1.028917in}{1.530759in}}{\pgfqpoint{1.028917in}{1.538995in}}%
\pgfpathcurveto{\pgfqpoint{1.028917in}{1.547232in}}{\pgfqpoint{1.025645in}{1.555132in}}{\pgfqpoint{1.019821in}{1.560956in}}%
\pgfpathcurveto{\pgfqpoint{1.013997in}{1.566780in}}{\pgfqpoint{1.006097in}{1.570052in}}{\pgfqpoint{0.997861in}{1.570052in}}%
\pgfpathcurveto{\pgfqpoint{0.989624in}{1.570052in}}{\pgfqpoint{0.981724in}{1.566780in}}{\pgfqpoint{0.975900in}{1.560956in}}%
\pgfpathcurveto{\pgfqpoint{0.970076in}{1.555132in}}{\pgfqpoint{0.966804in}{1.547232in}}{\pgfqpoint{0.966804in}{1.538995in}}%
\pgfpathcurveto{\pgfqpoint{0.966804in}{1.530759in}}{\pgfqpoint{0.970076in}{1.522859in}}{\pgfqpoint{0.975900in}{1.517035in}}%
\pgfpathcurveto{\pgfqpoint{0.981724in}{1.511211in}}{\pgfqpoint{0.989624in}{1.507939in}}{\pgfqpoint{0.997861in}{1.507939in}}%
\pgfpathclose%
\pgfusepath{stroke,fill}%
\end{pgfscope}%
\begin{pgfscope}%
\pgfpathrectangle{\pgfqpoint{0.100000in}{0.212622in}}{\pgfqpoint{3.696000in}{3.696000in}}%
\pgfusepath{clip}%
\pgfsetbuttcap%
\pgfsetroundjoin%
\definecolor{currentfill}{rgb}{0.121569,0.466667,0.705882}%
\pgfsetfillcolor{currentfill}%
\pgfsetfillopacity{0.786512}%
\pgfsetlinewidth{1.003750pt}%
\definecolor{currentstroke}{rgb}{0.121569,0.466667,0.705882}%
\pgfsetstrokecolor{currentstroke}%
\pgfsetstrokeopacity{0.786512}%
\pgfsetdash{}{0pt}%
\pgfpathmoveto{\pgfqpoint{2.544461in}{2.886061in}}%
\pgfpathcurveto{\pgfqpoint{2.552697in}{2.886061in}}{\pgfqpoint{2.560597in}{2.889333in}}{\pgfqpoint{2.566421in}{2.895157in}}%
\pgfpathcurveto{\pgfqpoint{2.572245in}{2.900981in}}{\pgfqpoint{2.575517in}{2.908881in}}{\pgfqpoint{2.575517in}{2.917117in}}%
\pgfpathcurveto{\pgfqpoint{2.575517in}{2.925354in}}{\pgfqpoint{2.572245in}{2.933254in}}{\pgfqpoint{2.566421in}{2.939078in}}%
\pgfpathcurveto{\pgfqpoint{2.560597in}{2.944902in}}{\pgfqpoint{2.552697in}{2.948174in}}{\pgfqpoint{2.544461in}{2.948174in}}%
\pgfpathcurveto{\pgfqpoint{2.536225in}{2.948174in}}{\pgfqpoint{2.528325in}{2.944902in}}{\pgfqpoint{2.522501in}{2.939078in}}%
\pgfpathcurveto{\pgfqpoint{2.516677in}{2.933254in}}{\pgfqpoint{2.513404in}{2.925354in}}{\pgfqpoint{2.513404in}{2.917117in}}%
\pgfpathcurveto{\pgfqpoint{2.513404in}{2.908881in}}{\pgfqpoint{2.516677in}{2.900981in}}{\pgfqpoint{2.522501in}{2.895157in}}%
\pgfpathcurveto{\pgfqpoint{2.528325in}{2.889333in}}{\pgfqpoint{2.536225in}{2.886061in}}{\pgfqpoint{2.544461in}{2.886061in}}%
\pgfpathclose%
\pgfusepath{stroke,fill}%
\end{pgfscope}%
\begin{pgfscope}%
\pgfpathrectangle{\pgfqpoint{0.100000in}{0.212622in}}{\pgfqpoint{3.696000in}{3.696000in}}%
\pgfusepath{clip}%
\pgfsetbuttcap%
\pgfsetroundjoin%
\definecolor{currentfill}{rgb}{0.121569,0.466667,0.705882}%
\pgfsetfillcolor{currentfill}%
\pgfsetfillopacity{0.787328}%
\pgfsetlinewidth{1.003750pt}%
\definecolor{currentstroke}{rgb}{0.121569,0.466667,0.705882}%
\pgfsetstrokecolor{currentstroke}%
\pgfsetstrokeopacity{0.787328}%
\pgfsetdash{}{0pt}%
\pgfpathmoveto{\pgfqpoint{1.001789in}{1.508002in}}%
\pgfpathcurveto{\pgfqpoint{1.010026in}{1.508002in}}{\pgfqpoint{1.017926in}{1.511274in}}{\pgfqpoint{1.023749in}{1.517098in}}%
\pgfpathcurveto{\pgfqpoint{1.029573in}{1.522922in}}{\pgfqpoint{1.032846in}{1.530822in}}{\pgfqpoint{1.032846in}{1.539058in}}%
\pgfpathcurveto{\pgfqpoint{1.032846in}{1.547295in}}{\pgfqpoint{1.029573in}{1.555195in}}{\pgfqpoint{1.023749in}{1.561019in}}%
\pgfpathcurveto{\pgfqpoint{1.017926in}{1.566843in}}{\pgfqpoint{1.010026in}{1.570115in}}{\pgfqpoint{1.001789in}{1.570115in}}%
\pgfpathcurveto{\pgfqpoint{0.993553in}{1.570115in}}{\pgfqpoint{0.985653in}{1.566843in}}{\pgfqpoint{0.979829in}{1.561019in}}%
\pgfpathcurveto{\pgfqpoint{0.974005in}{1.555195in}}{\pgfqpoint{0.970733in}{1.547295in}}{\pgfqpoint{0.970733in}{1.539058in}}%
\pgfpathcurveto{\pgfqpoint{0.970733in}{1.530822in}}{\pgfqpoint{0.974005in}{1.522922in}}{\pgfqpoint{0.979829in}{1.517098in}}%
\pgfpathcurveto{\pgfqpoint{0.985653in}{1.511274in}}{\pgfqpoint{0.993553in}{1.508002in}}{\pgfqpoint{1.001789in}{1.508002in}}%
\pgfpathclose%
\pgfusepath{stroke,fill}%
\end{pgfscope}%
\begin{pgfscope}%
\pgfpathrectangle{\pgfqpoint{0.100000in}{0.212622in}}{\pgfqpoint{3.696000in}{3.696000in}}%
\pgfusepath{clip}%
\pgfsetbuttcap%
\pgfsetroundjoin%
\definecolor{currentfill}{rgb}{0.121569,0.466667,0.705882}%
\pgfsetfillcolor{currentfill}%
\pgfsetfillopacity{0.787370}%
\pgfsetlinewidth{1.003750pt}%
\definecolor{currentstroke}{rgb}{0.121569,0.466667,0.705882}%
\pgfsetstrokecolor{currentstroke}%
\pgfsetstrokeopacity{0.787370}%
\pgfsetdash{}{0pt}%
\pgfpathmoveto{\pgfqpoint{2.542376in}{2.879474in}}%
\pgfpathcurveto{\pgfqpoint{2.550612in}{2.879474in}}{\pgfqpoint{2.558512in}{2.882746in}}{\pgfqpoint{2.564336in}{2.888570in}}%
\pgfpathcurveto{\pgfqpoint{2.570160in}{2.894394in}}{\pgfqpoint{2.573432in}{2.902294in}}{\pgfqpoint{2.573432in}{2.910530in}}%
\pgfpathcurveto{\pgfqpoint{2.573432in}{2.918767in}}{\pgfqpoint{2.570160in}{2.926667in}}{\pgfqpoint{2.564336in}{2.932491in}}%
\pgfpathcurveto{\pgfqpoint{2.558512in}{2.938315in}}{\pgfqpoint{2.550612in}{2.941587in}}{\pgfqpoint{2.542376in}{2.941587in}}%
\pgfpathcurveto{\pgfqpoint{2.534139in}{2.941587in}}{\pgfqpoint{2.526239in}{2.938315in}}{\pgfqpoint{2.520415in}{2.932491in}}%
\pgfpathcurveto{\pgfqpoint{2.514592in}{2.926667in}}{\pgfqpoint{2.511319in}{2.918767in}}{\pgfqpoint{2.511319in}{2.910530in}}%
\pgfpathcurveto{\pgfqpoint{2.511319in}{2.902294in}}{\pgfqpoint{2.514592in}{2.894394in}}{\pgfqpoint{2.520415in}{2.888570in}}%
\pgfpathcurveto{\pgfqpoint{2.526239in}{2.882746in}}{\pgfqpoint{2.534139in}{2.879474in}}{\pgfqpoint{2.542376in}{2.879474in}}%
\pgfpathclose%
\pgfusepath{stroke,fill}%
\end{pgfscope}%
\begin{pgfscope}%
\pgfpathrectangle{\pgfqpoint{0.100000in}{0.212622in}}{\pgfqpoint{3.696000in}{3.696000in}}%
\pgfusepath{clip}%
\pgfsetbuttcap%
\pgfsetroundjoin%
\definecolor{currentfill}{rgb}{0.121569,0.466667,0.705882}%
\pgfsetfillcolor{currentfill}%
\pgfsetfillopacity{0.788381}%
\pgfsetlinewidth{1.003750pt}%
\definecolor{currentstroke}{rgb}{0.121569,0.466667,0.705882}%
\pgfsetstrokecolor{currentstroke}%
\pgfsetstrokeopacity{0.788381}%
\pgfsetdash{}{0pt}%
\pgfpathmoveto{\pgfqpoint{1.009845in}{1.505173in}}%
\pgfpathcurveto{\pgfqpoint{1.018081in}{1.505173in}}{\pgfqpoint{1.025981in}{1.508445in}}{\pgfqpoint{1.031805in}{1.514269in}}%
\pgfpathcurveto{\pgfqpoint{1.037629in}{1.520093in}}{\pgfqpoint{1.040902in}{1.527993in}}{\pgfqpoint{1.040902in}{1.536229in}}%
\pgfpathcurveto{\pgfqpoint{1.040902in}{1.544465in}}{\pgfqpoint{1.037629in}{1.552365in}}{\pgfqpoint{1.031805in}{1.558189in}}%
\pgfpathcurveto{\pgfqpoint{1.025981in}{1.564013in}}{\pgfqpoint{1.018081in}{1.567286in}}{\pgfqpoint{1.009845in}{1.567286in}}%
\pgfpathcurveto{\pgfqpoint{1.001609in}{1.567286in}}{\pgfqpoint{0.993709in}{1.564013in}}{\pgfqpoint{0.987885in}{1.558189in}}%
\pgfpathcurveto{\pgfqpoint{0.982061in}{1.552365in}}{\pgfqpoint{0.978789in}{1.544465in}}{\pgfqpoint{0.978789in}{1.536229in}}%
\pgfpathcurveto{\pgfqpoint{0.978789in}{1.527993in}}{\pgfqpoint{0.982061in}{1.520093in}}{\pgfqpoint{0.987885in}{1.514269in}}%
\pgfpathcurveto{\pgfqpoint{0.993709in}{1.508445in}}{\pgfqpoint{1.001609in}{1.505173in}}{\pgfqpoint{1.009845in}{1.505173in}}%
\pgfpathclose%
\pgfusepath{stroke,fill}%
\end{pgfscope}%
\begin{pgfscope}%
\pgfpathrectangle{\pgfqpoint{0.100000in}{0.212622in}}{\pgfqpoint{3.696000in}{3.696000in}}%
\pgfusepath{clip}%
\pgfsetbuttcap%
\pgfsetroundjoin%
\definecolor{currentfill}{rgb}{0.121569,0.466667,0.705882}%
\pgfsetfillcolor{currentfill}%
\pgfsetfillopacity{0.788917}%
\pgfsetlinewidth{1.003750pt}%
\definecolor{currentstroke}{rgb}{0.121569,0.466667,0.705882}%
\pgfsetstrokecolor{currentstroke}%
\pgfsetstrokeopacity{0.788917}%
\pgfsetdash{}{0pt}%
\pgfpathmoveto{\pgfqpoint{2.538734in}{2.867355in}}%
\pgfpathcurveto{\pgfqpoint{2.546970in}{2.867355in}}{\pgfqpoint{2.554870in}{2.870627in}}{\pgfqpoint{2.560694in}{2.876451in}}%
\pgfpathcurveto{\pgfqpoint{2.566518in}{2.882275in}}{\pgfqpoint{2.569790in}{2.890175in}}{\pgfqpoint{2.569790in}{2.898411in}}%
\pgfpathcurveto{\pgfqpoint{2.569790in}{2.906648in}}{\pgfqpoint{2.566518in}{2.914548in}}{\pgfqpoint{2.560694in}{2.920372in}}%
\pgfpathcurveto{\pgfqpoint{2.554870in}{2.926195in}}{\pgfqpoint{2.546970in}{2.929468in}}{\pgfqpoint{2.538734in}{2.929468in}}%
\pgfpathcurveto{\pgfqpoint{2.530498in}{2.929468in}}{\pgfqpoint{2.522598in}{2.926195in}}{\pgfqpoint{2.516774in}{2.920372in}}%
\pgfpathcurveto{\pgfqpoint{2.510950in}{2.914548in}}{\pgfqpoint{2.507677in}{2.906648in}}{\pgfqpoint{2.507677in}{2.898411in}}%
\pgfpathcurveto{\pgfqpoint{2.507677in}{2.890175in}}{\pgfqpoint{2.510950in}{2.882275in}}{\pgfqpoint{2.516774in}{2.876451in}}%
\pgfpathcurveto{\pgfqpoint{2.522598in}{2.870627in}}{\pgfqpoint{2.530498in}{2.867355in}}{\pgfqpoint{2.538734in}{2.867355in}}%
\pgfpathclose%
\pgfusepath{stroke,fill}%
\end{pgfscope}%
\begin{pgfscope}%
\pgfpathrectangle{\pgfqpoint{0.100000in}{0.212622in}}{\pgfqpoint{3.696000in}{3.696000in}}%
\pgfusepath{clip}%
\pgfsetbuttcap%
\pgfsetroundjoin%
\definecolor{currentfill}{rgb}{0.121569,0.466667,0.705882}%
\pgfsetfillcolor{currentfill}%
\pgfsetfillopacity{0.789730}%
\pgfsetlinewidth{1.003750pt}%
\definecolor{currentstroke}{rgb}{0.121569,0.466667,0.705882}%
\pgfsetstrokecolor{currentstroke}%
\pgfsetstrokeopacity{0.789730}%
\pgfsetdash{}{0pt}%
\pgfpathmoveto{\pgfqpoint{1.019082in}{1.500700in}}%
\pgfpathcurveto{\pgfqpoint{1.027318in}{1.500700in}}{\pgfqpoint{1.035218in}{1.503972in}}{\pgfqpoint{1.041042in}{1.509796in}}%
\pgfpathcurveto{\pgfqpoint{1.046866in}{1.515620in}}{\pgfqpoint{1.050138in}{1.523520in}}{\pgfqpoint{1.050138in}{1.531756in}}%
\pgfpathcurveto{\pgfqpoint{1.050138in}{1.539993in}}{\pgfqpoint{1.046866in}{1.547893in}}{\pgfqpoint{1.041042in}{1.553717in}}%
\pgfpathcurveto{\pgfqpoint{1.035218in}{1.559541in}}{\pgfqpoint{1.027318in}{1.562813in}}{\pgfqpoint{1.019082in}{1.562813in}}%
\pgfpathcurveto{\pgfqpoint{1.010846in}{1.562813in}}{\pgfqpoint{1.002946in}{1.559541in}}{\pgfqpoint{0.997122in}{1.553717in}}%
\pgfpathcurveto{\pgfqpoint{0.991298in}{1.547893in}}{\pgfqpoint{0.988025in}{1.539993in}}{\pgfqpoint{0.988025in}{1.531756in}}%
\pgfpathcurveto{\pgfqpoint{0.988025in}{1.523520in}}{\pgfqpoint{0.991298in}{1.515620in}}{\pgfqpoint{0.997122in}{1.509796in}}%
\pgfpathcurveto{\pgfqpoint{1.002946in}{1.503972in}}{\pgfqpoint{1.010846in}{1.500700in}}{\pgfqpoint{1.019082in}{1.500700in}}%
\pgfpathclose%
\pgfusepath{stroke,fill}%
\end{pgfscope}%
\begin{pgfscope}%
\pgfpathrectangle{\pgfqpoint{0.100000in}{0.212622in}}{\pgfqpoint{3.696000in}{3.696000in}}%
\pgfusepath{clip}%
\pgfsetbuttcap%
\pgfsetroundjoin%
\definecolor{currentfill}{rgb}{0.121569,0.466667,0.705882}%
\pgfsetfillcolor{currentfill}%
\pgfsetfillopacity{0.789779}%
\pgfsetlinewidth{1.003750pt}%
\definecolor{currentstroke}{rgb}{0.121569,0.466667,0.705882}%
\pgfsetstrokecolor{currentstroke}%
\pgfsetstrokeopacity{0.789779}%
\pgfsetdash{}{0pt}%
\pgfpathmoveto{\pgfqpoint{3.080026in}{2.005397in}}%
\pgfpathcurveto{\pgfqpoint{3.088263in}{2.005397in}}{\pgfqpoint{3.096163in}{2.008669in}}{\pgfqpoint{3.101987in}{2.014493in}}%
\pgfpathcurveto{\pgfqpoint{3.107811in}{2.020317in}}{\pgfqpoint{3.111083in}{2.028217in}}{\pgfqpoint{3.111083in}{2.036454in}}%
\pgfpathcurveto{\pgfqpoint{3.111083in}{2.044690in}}{\pgfqpoint{3.107811in}{2.052590in}}{\pgfqpoint{3.101987in}{2.058414in}}%
\pgfpathcurveto{\pgfqpoint{3.096163in}{2.064238in}}{\pgfqpoint{3.088263in}{2.067510in}}{\pgfqpoint{3.080026in}{2.067510in}}%
\pgfpathcurveto{\pgfqpoint{3.071790in}{2.067510in}}{\pgfqpoint{3.063890in}{2.064238in}}{\pgfqpoint{3.058066in}{2.058414in}}%
\pgfpathcurveto{\pgfqpoint{3.052242in}{2.052590in}}{\pgfqpoint{3.048970in}{2.044690in}}{\pgfqpoint{3.048970in}{2.036454in}}%
\pgfpathcurveto{\pgfqpoint{3.048970in}{2.028217in}}{\pgfqpoint{3.052242in}{2.020317in}}{\pgfqpoint{3.058066in}{2.014493in}}%
\pgfpathcurveto{\pgfqpoint{3.063890in}{2.008669in}}{\pgfqpoint{3.071790in}{2.005397in}}{\pgfqpoint{3.080026in}{2.005397in}}%
\pgfpathclose%
\pgfusepath{stroke,fill}%
\end{pgfscope}%
\begin{pgfscope}%
\pgfpathrectangle{\pgfqpoint{0.100000in}{0.212622in}}{\pgfqpoint{3.696000in}{3.696000in}}%
\pgfusepath{clip}%
\pgfsetbuttcap%
\pgfsetroundjoin%
\definecolor{currentfill}{rgb}{0.121569,0.466667,0.705882}%
\pgfsetfillcolor{currentfill}%
\pgfsetfillopacity{0.790021}%
\pgfsetlinewidth{1.003750pt}%
\definecolor{currentstroke}{rgb}{0.121569,0.466667,0.705882}%
\pgfsetstrokecolor{currentstroke}%
\pgfsetstrokeopacity{0.790021}%
\pgfsetdash{}{0pt}%
\pgfpathmoveto{\pgfqpoint{2.533282in}{2.859048in}}%
\pgfpathcurveto{\pgfqpoint{2.541518in}{2.859048in}}{\pgfqpoint{2.549419in}{2.862321in}}{\pgfqpoint{2.555242in}{2.868145in}}%
\pgfpathcurveto{\pgfqpoint{2.561066in}{2.873969in}}{\pgfqpoint{2.564339in}{2.881869in}}{\pgfqpoint{2.564339in}{2.890105in}}%
\pgfpathcurveto{\pgfqpoint{2.564339in}{2.898341in}}{\pgfqpoint{2.561066in}{2.906241in}}{\pgfqpoint{2.555242in}{2.912065in}}%
\pgfpathcurveto{\pgfqpoint{2.549419in}{2.917889in}}{\pgfqpoint{2.541518in}{2.921161in}}{\pgfqpoint{2.533282in}{2.921161in}}%
\pgfpathcurveto{\pgfqpoint{2.525046in}{2.921161in}}{\pgfqpoint{2.517146in}{2.917889in}}{\pgfqpoint{2.511322in}{2.912065in}}%
\pgfpathcurveto{\pgfqpoint{2.505498in}{2.906241in}}{\pgfqpoint{2.502226in}{2.898341in}}{\pgfqpoint{2.502226in}{2.890105in}}%
\pgfpathcurveto{\pgfqpoint{2.502226in}{2.881869in}}{\pgfqpoint{2.505498in}{2.873969in}}{\pgfqpoint{2.511322in}{2.868145in}}%
\pgfpathcurveto{\pgfqpoint{2.517146in}{2.862321in}}{\pgfqpoint{2.525046in}{2.859048in}}{\pgfqpoint{2.533282in}{2.859048in}}%
\pgfpathclose%
\pgfusepath{stroke,fill}%
\end{pgfscope}%
\begin{pgfscope}%
\pgfpathrectangle{\pgfqpoint{0.100000in}{0.212622in}}{\pgfqpoint{3.696000in}{3.696000in}}%
\pgfusepath{clip}%
\pgfsetbuttcap%
\pgfsetroundjoin%
\definecolor{currentfill}{rgb}{0.121569,0.466667,0.705882}%
\pgfsetfillcolor{currentfill}%
\pgfsetfillopacity{0.790114}%
\pgfsetlinewidth{1.003750pt}%
\definecolor{currentstroke}{rgb}{0.121569,0.466667,0.705882}%
\pgfsetstrokecolor{currentstroke}%
\pgfsetstrokeopacity{0.790114}%
\pgfsetdash{}{0pt}%
\pgfpathmoveto{\pgfqpoint{1.024389in}{1.497320in}}%
\pgfpathcurveto{\pgfqpoint{1.032626in}{1.497320in}}{\pgfqpoint{1.040526in}{1.500593in}}{\pgfqpoint{1.046350in}{1.506416in}}%
\pgfpathcurveto{\pgfqpoint{1.052173in}{1.512240in}}{\pgfqpoint{1.055446in}{1.520140in}}{\pgfqpoint{1.055446in}{1.528377in}}%
\pgfpathcurveto{\pgfqpoint{1.055446in}{1.536613in}}{\pgfqpoint{1.052173in}{1.544513in}}{\pgfqpoint{1.046350in}{1.550337in}}%
\pgfpathcurveto{\pgfqpoint{1.040526in}{1.556161in}}{\pgfqpoint{1.032626in}{1.559433in}}{\pgfqpoint{1.024389in}{1.559433in}}%
\pgfpathcurveto{\pgfqpoint{1.016153in}{1.559433in}}{\pgfqpoint{1.008253in}{1.556161in}}{\pgfqpoint{1.002429in}{1.550337in}}%
\pgfpathcurveto{\pgfqpoint{0.996605in}{1.544513in}}{\pgfqpoint{0.993333in}{1.536613in}}{\pgfqpoint{0.993333in}{1.528377in}}%
\pgfpathcurveto{\pgfqpoint{0.993333in}{1.520140in}}{\pgfqpoint{0.996605in}{1.512240in}}{\pgfqpoint{1.002429in}{1.506416in}}%
\pgfpathcurveto{\pgfqpoint{1.008253in}{1.500593in}}{\pgfqpoint{1.016153in}{1.497320in}}{\pgfqpoint{1.024389in}{1.497320in}}%
\pgfpathclose%
\pgfusepath{stroke,fill}%
\end{pgfscope}%
\begin{pgfscope}%
\pgfpathrectangle{\pgfqpoint{0.100000in}{0.212622in}}{\pgfqpoint{3.696000in}{3.696000in}}%
\pgfusepath{clip}%
\pgfsetbuttcap%
\pgfsetroundjoin%
\definecolor{currentfill}{rgb}{0.121569,0.466667,0.705882}%
\pgfsetfillcolor{currentfill}%
\pgfsetfillopacity{0.790691}%
\pgfsetlinewidth{1.003750pt}%
\definecolor{currentstroke}{rgb}{0.121569,0.466667,0.705882}%
\pgfsetstrokecolor{currentstroke}%
\pgfsetstrokeopacity{0.790691}%
\pgfsetdash{}{0pt}%
\pgfpathmoveto{\pgfqpoint{2.527847in}{2.850437in}}%
\pgfpathcurveto{\pgfqpoint{2.536083in}{2.850437in}}{\pgfqpoint{2.543983in}{2.853710in}}{\pgfqpoint{2.549807in}{2.859534in}}%
\pgfpathcurveto{\pgfqpoint{2.555631in}{2.865357in}}{\pgfqpoint{2.558903in}{2.873257in}}{\pgfqpoint{2.558903in}{2.881494in}}%
\pgfpathcurveto{\pgfqpoint{2.558903in}{2.889730in}}{\pgfqpoint{2.555631in}{2.897630in}}{\pgfqpoint{2.549807in}{2.903454in}}%
\pgfpathcurveto{\pgfqpoint{2.543983in}{2.909278in}}{\pgfqpoint{2.536083in}{2.912550in}}{\pgfqpoint{2.527847in}{2.912550in}}%
\pgfpathcurveto{\pgfqpoint{2.519610in}{2.912550in}}{\pgfqpoint{2.511710in}{2.909278in}}{\pgfqpoint{2.505886in}{2.903454in}}%
\pgfpathcurveto{\pgfqpoint{2.500063in}{2.897630in}}{\pgfqpoint{2.496790in}{2.889730in}}{\pgfqpoint{2.496790in}{2.881494in}}%
\pgfpathcurveto{\pgfqpoint{2.496790in}{2.873257in}}{\pgfqpoint{2.500063in}{2.865357in}}{\pgfqpoint{2.505886in}{2.859534in}}%
\pgfpathcurveto{\pgfqpoint{2.511710in}{2.853710in}}{\pgfqpoint{2.519610in}{2.850437in}}{\pgfqpoint{2.527847in}{2.850437in}}%
\pgfpathclose%
\pgfusepath{stroke,fill}%
\end{pgfscope}%
\begin{pgfscope}%
\pgfpathrectangle{\pgfqpoint{0.100000in}{0.212622in}}{\pgfqpoint{3.696000in}{3.696000in}}%
\pgfusepath{clip}%
\pgfsetbuttcap%
\pgfsetroundjoin%
\definecolor{currentfill}{rgb}{0.121569,0.466667,0.705882}%
\pgfsetfillcolor{currentfill}%
\pgfsetfillopacity{0.791077}%
\pgfsetlinewidth{1.003750pt}%
\definecolor{currentstroke}{rgb}{0.121569,0.466667,0.705882}%
\pgfsetstrokecolor{currentstroke}%
\pgfsetstrokeopacity{0.791077}%
\pgfsetdash{}{0pt}%
\pgfpathmoveto{\pgfqpoint{1.030188in}{1.496975in}}%
\pgfpathcurveto{\pgfqpoint{1.038424in}{1.496975in}}{\pgfqpoint{1.046324in}{1.500248in}}{\pgfqpoint{1.052148in}{1.506071in}}%
\pgfpathcurveto{\pgfqpoint{1.057972in}{1.511895in}}{\pgfqpoint{1.061244in}{1.519795in}}{\pgfqpoint{1.061244in}{1.528032in}}%
\pgfpathcurveto{\pgfqpoint{1.061244in}{1.536268in}}{\pgfqpoint{1.057972in}{1.544168in}}{\pgfqpoint{1.052148in}{1.549992in}}%
\pgfpathcurveto{\pgfqpoint{1.046324in}{1.555816in}}{\pgfqpoint{1.038424in}{1.559088in}}{\pgfqpoint{1.030188in}{1.559088in}}%
\pgfpathcurveto{\pgfqpoint{1.021952in}{1.559088in}}{\pgfqpoint{1.014052in}{1.555816in}}{\pgfqpoint{1.008228in}{1.549992in}}%
\pgfpathcurveto{\pgfqpoint{1.002404in}{1.544168in}}{\pgfqpoint{0.999131in}{1.536268in}}{\pgfqpoint{0.999131in}{1.528032in}}%
\pgfpathcurveto{\pgfqpoint{0.999131in}{1.519795in}}{\pgfqpoint{1.002404in}{1.511895in}}{\pgfqpoint{1.008228in}{1.506071in}}%
\pgfpathcurveto{\pgfqpoint{1.014052in}{1.500248in}}{\pgfqpoint{1.021952in}{1.496975in}}{\pgfqpoint{1.030188in}{1.496975in}}%
\pgfpathclose%
\pgfusepath{stroke,fill}%
\end{pgfscope}%
\begin{pgfscope}%
\pgfpathrectangle{\pgfqpoint{0.100000in}{0.212622in}}{\pgfqpoint{3.696000in}{3.696000in}}%
\pgfusepath{clip}%
\pgfsetbuttcap%
\pgfsetroundjoin%
\definecolor{currentfill}{rgb}{0.121569,0.466667,0.705882}%
\pgfsetfillcolor{currentfill}%
\pgfsetfillopacity{0.791536}%
\pgfsetlinewidth{1.003750pt}%
\definecolor{currentstroke}{rgb}{0.121569,0.466667,0.705882}%
\pgfsetstrokecolor{currentstroke}%
\pgfsetstrokeopacity{0.791536}%
\pgfsetdash{}{0pt}%
\pgfpathmoveto{\pgfqpoint{2.523753in}{2.842409in}}%
\pgfpathcurveto{\pgfqpoint{2.531989in}{2.842409in}}{\pgfqpoint{2.539889in}{2.845681in}}{\pgfqpoint{2.545713in}{2.851505in}}%
\pgfpathcurveto{\pgfqpoint{2.551537in}{2.857329in}}{\pgfqpoint{2.554809in}{2.865229in}}{\pgfqpoint{2.554809in}{2.873466in}}%
\pgfpathcurveto{\pgfqpoint{2.554809in}{2.881702in}}{\pgfqpoint{2.551537in}{2.889602in}}{\pgfqpoint{2.545713in}{2.895426in}}%
\pgfpathcurveto{\pgfqpoint{2.539889in}{2.901250in}}{\pgfqpoint{2.531989in}{2.904522in}}{\pgfqpoint{2.523753in}{2.904522in}}%
\pgfpathcurveto{\pgfqpoint{2.515517in}{2.904522in}}{\pgfqpoint{2.507617in}{2.901250in}}{\pgfqpoint{2.501793in}{2.895426in}}%
\pgfpathcurveto{\pgfqpoint{2.495969in}{2.889602in}}{\pgfqpoint{2.492696in}{2.881702in}}{\pgfqpoint{2.492696in}{2.873466in}}%
\pgfpathcurveto{\pgfqpoint{2.492696in}{2.865229in}}{\pgfqpoint{2.495969in}{2.857329in}}{\pgfqpoint{2.501793in}{2.851505in}}%
\pgfpathcurveto{\pgfqpoint{2.507617in}{2.845681in}}{\pgfqpoint{2.515517in}{2.842409in}}{\pgfqpoint{2.523753in}{2.842409in}}%
\pgfpathclose%
\pgfusepath{stroke,fill}%
\end{pgfscope}%
\begin{pgfscope}%
\pgfpathrectangle{\pgfqpoint{0.100000in}{0.212622in}}{\pgfqpoint{3.696000in}{3.696000in}}%
\pgfusepath{clip}%
\pgfsetbuttcap%
\pgfsetroundjoin%
\definecolor{currentfill}{rgb}{0.121569,0.466667,0.705882}%
\pgfsetfillcolor{currentfill}%
\pgfsetfillopacity{0.792369}%
\pgfsetlinewidth{1.003750pt}%
\definecolor{currentstroke}{rgb}{0.121569,0.466667,0.705882}%
\pgfsetstrokecolor{currentstroke}%
\pgfsetstrokeopacity{0.792369}%
\pgfsetdash{}{0pt}%
\pgfpathmoveto{\pgfqpoint{2.521832in}{2.835324in}}%
\pgfpathcurveto{\pgfqpoint{2.530068in}{2.835324in}}{\pgfqpoint{2.537968in}{2.838596in}}{\pgfqpoint{2.543792in}{2.844420in}}%
\pgfpathcurveto{\pgfqpoint{2.549616in}{2.850244in}}{\pgfqpoint{2.552888in}{2.858144in}}{\pgfqpoint{2.552888in}{2.866381in}}%
\pgfpathcurveto{\pgfqpoint{2.552888in}{2.874617in}}{\pgfqpoint{2.549616in}{2.882517in}}{\pgfqpoint{2.543792in}{2.888341in}}%
\pgfpathcurveto{\pgfqpoint{2.537968in}{2.894165in}}{\pgfqpoint{2.530068in}{2.897437in}}{\pgfqpoint{2.521832in}{2.897437in}}%
\pgfpathcurveto{\pgfqpoint{2.513595in}{2.897437in}}{\pgfqpoint{2.505695in}{2.894165in}}{\pgfqpoint{2.499871in}{2.888341in}}%
\pgfpathcurveto{\pgfqpoint{2.494048in}{2.882517in}}{\pgfqpoint{2.490775in}{2.874617in}}{\pgfqpoint{2.490775in}{2.866381in}}%
\pgfpathcurveto{\pgfqpoint{2.490775in}{2.858144in}}{\pgfqpoint{2.494048in}{2.850244in}}{\pgfqpoint{2.499871in}{2.844420in}}%
\pgfpathcurveto{\pgfqpoint{2.505695in}{2.838596in}}{\pgfqpoint{2.513595in}{2.835324in}}{\pgfqpoint{2.521832in}{2.835324in}}%
\pgfpathclose%
\pgfusepath{stroke,fill}%
\end{pgfscope}%
\begin{pgfscope}%
\pgfpathrectangle{\pgfqpoint{0.100000in}{0.212622in}}{\pgfqpoint{3.696000in}{3.696000in}}%
\pgfusepath{clip}%
\pgfsetbuttcap%
\pgfsetroundjoin%
\definecolor{currentfill}{rgb}{0.121569,0.466667,0.705882}%
\pgfsetfillcolor{currentfill}%
\pgfsetfillopacity{0.792769}%
\pgfsetlinewidth{1.003750pt}%
\definecolor{currentstroke}{rgb}{0.121569,0.466667,0.705882}%
\pgfsetstrokecolor{currentstroke}%
\pgfsetstrokeopacity{0.792769}%
\pgfsetdash{}{0pt}%
\pgfpathmoveto{\pgfqpoint{1.038071in}{1.498463in}}%
\pgfpathcurveto{\pgfqpoint{1.046307in}{1.498463in}}{\pgfqpoint{1.054207in}{1.501735in}}{\pgfqpoint{1.060031in}{1.507559in}}%
\pgfpathcurveto{\pgfqpoint{1.065855in}{1.513383in}}{\pgfqpoint{1.069127in}{1.521283in}}{\pgfqpoint{1.069127in}{1.529520in}}%
\pgfpathcurveto{\pgfqpoint{1.069127in}{1.537756in}}{\pgfqpoint{1.065855in}{1.545656in}}{\pgfqpoint{1.060031in}{1.551480in}}%
\pgfpathcurveto{\pgfqpoint{1.054207in}{1.557304in}}{\pgfqpoint{1.046307in}{1.560576in}}{\pgfqpoint{1.038071in}{1.560576in}}%
\pgfpathcurveto{\pgfqpoint{1.029834in}{1.560576in}}{\pgfqpoint{1.021934in}{1.557304in}}{\pgfqpoint{1.016110in}{1.551480in}}%
\pgfpathcurveto{\pgfqpoint{1.010287in}{1.545656in}}{\pgfqpoint{1.007014in}{1.537756in}}{\pgfqpoint{1.007014in}{1.529520in}}%
\pgfpathcurveto{\pgfqpoint{1.007014in}{1.521283in}}{\pgfqpoint{1.010287in}{1.513383in}}{\pgfqpoint{1.016110in}{1.507559in}}%
\pgfpathcurveto{\pgfqpoint{1.021934in}{1.501735in}}{\pgfqpoint{1.029834in}{1.498463in}}{\pgfqpoint{1.038071in}{1.498463in}}%
\pgfpathclose%
\pgfusepath{stroke,fill}%
\end{pgfscope}%
\begin{pgfscope}%
\pgfpathrectangle{\pgfqpoint{0.100000in}{0.212622in}}{\pgfqpoint{3.696000in}{3.696000in}}%
\pgfusepath{clip}%
\pgfsetbuttcap%
\pgfsetroundjoin%
\definecolor{currentfill}{rgb}{0.121569,0.466667,0.705882}%
\pgfsetfillcolor{currentfill}%
\pgfsetfillopacity{0.793091}%
\pgfsetlinewidth{1.003750pt}%
\definecolor{currentstroke}{rgb}{0.121569,0.466667,0.705882}%
\pgfsetstrokecolor{currentstroke}%
\pgfsetstrokeopacity{0.793091}%
\pgfsetdash{}{0pt}%
\pgfpathmoveto{\pgfqpoint{2.518705in}{2.829855in}}%
\pgfpathcurveto{\pgfqpoint{2.526942in}{2.829855in}}{\pgfqpoint{2.534842in}{2.833127in}}{\pgfqpoint{2.540665in}{2.838951in}}%
\pgfpathcurveto{\pgfqpoint{2.546489in}{2.844775in}}{\pgfqpoint{2.549762in}{2.852675in}}{\pgfqpoint{2.549762in}{2.860912in}}%
\pgfpathcurveto{\pgfqpoint{2.549762in}{2.869148in}}{\pgfqpoint{2.546489in}{2.877048in}}{\pgfqpoint{2.540665in}{2.882872in}}%
\pgfpathcurveto{\pgfqpoint{2.534842in}{2.888696in}}{\pgfqpoint{2.526942in}{2.891968in}}{\pgfqpoint{2.518705in}{2.891968in}}%
\pgfpathcurveto{\pgfqpoint{2.510469in}{2.891968in}}{\pgfqpoint{2.502569in}{2.888696in}}{\pgfqpoint{2.496745in}{2.882872in}}%
\pgfpathcurveto{\pgfqpoint{2.490921in}{2.877048in}}{\pgfqpoint{2.487649in}{2.869148in}}{\pgfqpoint{2.487649in}{2.860912in}}%
\pgfpathcurveto{\pgfqpoint{2.487649in}{2.852675in}}{\pgfqpoint{2.490921in}{2.844775in}}{\pgfqpoint{2.496745in}{2.838951in}}%
\pgfpathcurveto{\pgfqpoint{2.502569in}{2.833127in}}{\pgfqpoint{2.510469in}{2.829855in}}{\pgfqpoint{2.518705in}{2.829855in}}%
\pgfpathclose%
\pgfusepath{stroke,fill}%
\end{pgfscope}%
\begin{pgfscope}%
\pgfpathrectangle{\pgfqpoint{0.100000in}{0.212622in}}{\pgfqpoint{3.696000in}{3.696000in}}%
\pgfusepath{clip}%
\pgfsetbuttcap%
\pgfsetroundjoin%
\definecolor{currentfill}{rgb}{0.121569,0.466667,0.705882}%
\pgfsetfillcolor{currentfill}%
\pgfsetfillopacity{0.793421}%
\pgfsetlinewidth{1.003750pt}%
\definecolor{currentstroke}{rgb}{0.121569,0.466667,0.705882}%
\pgfsetstrokecolor{currentstroke}%
\pgfsetstrokeopacity{0.793421}%
\pgfsetdash{}{0pt}%
\pgfpathmoveto{\pgfqpoint{3.063580in}{1.983660in}}%
\pgfpathcurveto{\pgfqpoint{3.071816in}{1.983660in}}{\pgfqpoint{3.079716in}{1.986932in}}{\pgfqpoint{3.085540in}{1.992756in}}%
\pgfpathcurveto{\pgfqpoint{3.091364in}{1.998580in}}{\pgfqpoint{3.094637in}{2.006480in}}{\pgfqpoint{3.094637in}{2.014717in}}%
\pgfpathcurveto{\pgfqpoint{3.094637in}{2.022953in}}{\pgfqpoint{3.091364in}{2.030853in}}{\pgfqpoint{3.085540in}{2.036677in}}%
\pgfpathcurveto{\pgfqpoint{3.079716in}{2.042501in}}{\pgfqpoint{3.071816in}{2.045773in}}{\pgfqpoint{3.063580in}{2.045773in}}%
\pgfpathcurveto{\pgfqpoint{3.055344in}{2.045773in}}{\pgfqpoint{3.047444in}{2.042501in}}{\pgfqpoint{3.041620in}{2.036677in}}%
\pgfpathcurveto{\pgfqpoint{3.035796in}{2.030853in}}{\pgfqpoint{3.032524in}{2.022953in}}{\pgfqpoint{3.032524in}{2.014717in}}%
\pgfpathcurveto{\pgfqpoint{3.032524in}{2.006480in}}{\pgfqpoint{3.035796in}{1.998580in}}{\pgfqpoint{3.041620in}{1.992756in}}%
\pgfpathcurveto{\pgfqpoint{3.047444in}{1.986932in}}{\pgfqpoint{3.055344in}{1.983660in}}{\pgfqpoint{3.063580in}{1.983660in}}%
\pgfpathclose%
\pgfusepath{stroke,fill}%
\end{pgfscope}%
\begin{pgfscope}%
\pgfpathrectangle{\pgfqpoint{0.100000in}{0.212622in}}{\pgfqpoint{3.696000in}{3.696000in}}%
\pgfusepath{clip}%
\pgfsetbuttcap%
\pgfsetroundjoin%
\definecolor{currentfill}{rgb}{0.121569,0.466667,0.705882}%
\pgfsetfillcolor{currentfill}%
\pgfsetfillopacity{0.793480}%
\pgfsetlinewidth{1.003750pt}%
\definecolor{currentstroke}{rgb}{0.121569,0.466667,0.705882}%
\pgfsetstrokecolor{currentstroke}%
\pgfsetstrokeopacity{0.793480}%
\pgfsetdash{}{0pt}%
\pgfpathmoveto{\pgfqpoint{2.515626in}{2.824904in}}%
\pgfpathcurveto{\pgfqpoint{2.523862in}{2.824904in}}{\pgfqpoint{2.531762in}{2.828176in}}{\pgfqpoint{2.537586in}{2.834000in}}%
\pgfpathcurveto{\pgfqpoint{2.543410in}{2.839824in}}{\pgfqpoint{2.546682in}{2.847724in}}{\pgfqpoint{2.546682in}{2.855960in}}%
\pgfpathcurveto{\pgfqpoint{2.546682in}{2.864196in}}{\pgfqpoint{2.543410in}{2.872097in}}{\pgfqpoint{2.537586in}{2.877920in}}%
\pgfpathcurveto{\pgfqpoint{2.531762in}{2.883744in}}{\pgfqpoint{2.523862in}{2.887017in}}{\pgfqpoint{2.515626in}{2.887017in}}%
\pgfpathcurveto{\pgfqpoint{2.507390in}{2.887017in}}{\pgfqpoint{2.499489in}{2.883744in}}{\pgfqpoint{2.493666in}{2.877920in}}%
\pgfpathcurveto{\pgfqpoint{2.487842in}{2.872097in}}{\pgfqpoint{2.484569in}{2.864196in}}{\pgfqpoint{2.484569in}{2.855960in}}%
\pgfpathcurveto{\pgfqpoint{2.484569in}{2.847724in}}{\pgfqpoint{2.487842in}{2.839824in}}{\pgfqpoint{2.493666in}{2.834000in}}%
\pgfpathcurveto{\pgfqpoint{2.499489in}{2.828176in}}{\pgfqpoint{2.507390in}{2.824904in}}{\pgfqpoint{2.515626in}{2.824904in}}%
\pgfpathclose%
\pgfusepath{stroke,fill}%
\end{pgfscope}%
\begin{pgfscope}%
\pgfpathrectangle{\pgfqpoint{0.100000in}{0.212622in}}{\pgfqpoint{3.696000in}{3.696000in}}%
\pgfusepath{clip}%
\pgfsetbuttcap%
\pgfsetroundjoin%
\definecolor{currentfill}{rgb}{0.121569,0.466667,0.705882}%
\pgfsetfillcolor{currentfill}%
\pgfsetfillopacity{0.793552}%
\pgfsetlinewidth{1.003750pt}%
\definecolor{currentstroke}{rgb}{0.121569,0.466667,0.705882}%
\pgfsetstrokecolor{currentstroke}%
\pgfsetstrokeopacity{0.793552}%
\pgfsetdash{}{0pt}%
\pgfpathmoveto{\pgfqpoint{1.046178in}{1.490912in}}%
\pgfpathcurveto{\pgfqpoint{1.054414in}{1.490912in}}{\pgfqpoint{1.062314in}{1.494184in}}{\pgfqpoint{1.068138in}{1.500008in}}%
\pgfpathcurveto{\pgfqpoint{1.073962in}{1.505832in}}{\pgfqpoint{1.077234in}{1.513732in}}{\pgfqpoint{1.077234in}{1.521968in}}%
\pgfpathcurveto{\pgfqpoint{1.077234in}{1.530205in}}{\pgfqpoint{1.073962in}{1.538105in}}{\pgfqpoint{1.068138in}{1.543929in}}%
\pgfpathcurveto{\pgfqpoint{1.062314in}{1.549752in}}{\pgfqpoint{1.054414in}{1.553025in}}{\pgfqpoint{1.046178in}{1.553025in}}%
\pgfpathcurveto{\pgfqpoint{1.037941in}{1.553025in}}{\pgfqpoint{1.030041in}{1.549752in}}{\pgfqpoint{1.024217in}{1.543929in}}%
\pgfpathcurveto{\pgfqpoint{1.018393in}{1.538105in}}{\pgfqpoint{1.015121in}{1.530205in}}{\pgfqpoint{1.015121in}{1.521968in}}%
\pgfpathcurveto{\pgfqpoint{1.015121in}{1.513732in}}{\pgfqpoint{1.018393in}{1.505832in}}{\pgfqpoint{1.024217in}{1.500008in}}%
\pgfpathcurveto{\pgfqpoint{1.030041in}{1.494184in}}{\pgfqpoint{1.037941in}{1.490912in}}{\pgfqpoint{1.046178in}{1.490912in}}%
\pgfpathclose%
\pgfusepath{stroke,fill}%
\end{pgfscope}%
\begin{pgfscope}%
\pgfpathrectangle{\pgfqpoint{0.100000in}{0.212622in}}{\pgfqpoint{3.696000in}{3.696000in}}%
\pgfusepath{clip}%
\pgfsetbuttcap%
\pgfsetroundjoin%
\definecolor{currentfill}{rgb}{0.121569,0.466667,0.705882}%
\pgfsetfillcolor{currentfill}%
\pgfsetfillopacity{0.793982}%
\pgfsetlinewidth{1.003750pt}%
\definecolor{currentstroke}{rgb}{0.121569,0.466667,0.705882}%
\pgfsetstrokecolor{currentstroke}%
\pgfsetstrokeopacity{0.793982}%
\pgfsetdash{}{0pt}%
\pgfpathmoveto{\pgfqpoint{2.513709in}{2.820852in}}%
\pgfpathcurveto{\pgfqpoint{2.521945in}{2.820852in}}{\pgfqpoint{2.529845in}{2.824124in}}{\pgfqpoint{2.535669in}{2.829948in}}%
\pgfpathcurveto{\pgfqpoint{2.541493in}{2.835772in}}{\pgfqpoint{2.544765in}{2.843672in}}{\pgfqpoint{2.544765in}{2.851908in}}%
\pgfpathcurveto{\pgfqpoint{2.544765in}{2.860144in}}{\pgfqpoint{2.541493in}{2.868045in}}{\pgfqpoint{2.535669in}{2.873868in}}%
\pgfpathcurveto{\pgfqpoint{2.529845in}{2.879692in}}{\pgfqpoint{2.521945in}{2.882965in}}{\pgfqpoint{2.513709in}{2.882965in}}%
\pgfpathcurveto{\pgfqpoint{2.505472in}{2.882965in}}{\pgfqpoint{2.497572in}{2.879692in}}{\pgfqpoint{2.491748in}{2.873868in}}%
\pgfpathcurveto{\pgfqpoint{2.485924in}{2.868045in}}{\pgfqpoint{2.482652in}{2.860144in}}{\pgfqpoint{2.482652in}{2.851908in}}%
\pgfpathcurveto{\pgfqpoint{2.482652in}{2.843672in}}{\pgfqpoint{2.485924in}{2.835772in}}{\pgfqpoint{2.491748in}{2.829948in}}%
\pgfpathcurveto{\pgfqpoint{2.497572in}{2.824124in}}{\pgfqpoint{2.505472in}{2.820852in}}{\pgfqpoint{2.513709in}{2.820852in}}%
\pgfpathclose%
\pgfusepath{stroke,fill}%
\end{pgfscope}%
\begin{pgfscope}%
\pgfpathrectangle{\pgfqpoint{0.100000in}{0.212622in}}{\pgfqpoint{3.696000in}{3.696000in}}%
\pgfusepath{clip}%
\pgfsetbuttcap%
\pgfsetroundjoin%
\definecolor{currentfill}{rgb}{0.121569,0.466667,0.705882}%
\pgfsetfillcolor{currentfill}%
\pgfsetfillopacity{0.794345}%
\pgfsetlinewidth{1.003750pt}%
\definecolor{currentstroke}{rgb}{0.121569,0.466667,0.705882}%
\pgfsetstrokecolor{currentstroke}%
\pgfsetstrokeopacity{0.794345}%
\pgfsetdash{}{0pt}%
\pgfpathmoveto{\pgfqpoint{1.050537in}{1.488180in}}%
\pgfpathcurveto{\pgfqpoint{1.058773in}{1.488180in}}{\pgfqpoint{1.066673in}{1.491452in}}{\pgfqpoint{1.072497in}{1.497276in}}%
\pgfpathcurveto{\pgfqpoint{1.078321in}{1.503100in}}{\pgfqpoint{1.081593in}{1.511000in}}{\pgfqpoint{1.081593in}{1.519237in}}%
\pgfpathcurveto{\pgfqpoint{1.081593in}{1.527473in}}{\pgfqpoint{1.078321in}{1.535373in}}{\pgfqpoint{1.072497in}{1.541197in}}%
\pgfpathcurveto{\pgfqpoint{1.066673in}{1.547021in}}{\pgfqpoint{1.058773in}{1.550293in}}{\pgfqpoint{1.050537in}{1.550293in}}%
\pgfpathcurveto{\pgfqpoint{1.042300in}{1.550293in}}{\pgfqpoint{1.034400in}{1.547021in}}{\pgfqpoint{1.028576in}{1.541197in}}%
\pgfpathcurveto{\pgfqpoint{1.022753in}{1.535373in}}{\pgfqpoint{1.019480in}{1.527473in}}{\pgfqpoint{1.019480in}{1.519237in}}%
\pgfpathcurveto{\pgfqpoint{1.019480in}{1.511000in}}{\pgfqpoint{1.022753in}{1.503100in}}{\pgfqpoint{1.028576in}{1.497276in}}%
\pgfpathcurveto{\pgfqpoint{1.034400in}{1.491452in}}{\pgfqpoint{1.042300in}{1.488180in}}{\pgfqpoint{1.050537in}{1.488180in}}%
\pgfpathclose%
\pgfusepath{stroke,fill}%
\end{pgfscope}%
\begin{pgfscope}%
\pgfpathrectangle{\pgfqpoint{0.100000in}{0.212622in}}{\pgfqpoint{3.696000in}{3.696000in}}%
\pgfusepath{clip}%
\pgfsetbuttcap%
\pgfsetroundjoin%
\definecolor{currentfill}{rgb}{0.121569,0.466667,0.705882}%
\pgfsetfillcolor{currentfill}%
\pgfsetfillopacity{0.794404}%
\pgfsetlinewidth{1.003750pt}%
\definecolor{currentstroke}{rgb}{0.121569,0.466667,0.705882}%
\pgfsetstrokecolor{currentstroke}%
\pgfsetstrokeopacity{0.794404}%
\pgfsetdash{}{0pt}%
\pgfpathmoveto{\pgfqpoint{2.512423in}{2.817238in}}%
\pgfpathcurveto{\pgfqpoint{2.520659in}{2.817238in}}{\pgfqpoint{2.528559in}{2.820510in}}{\pgfqpoint{2.534383in}{2.826334in}}%
\pgfpathcurveto{\pgfqpoint{2.540207in}{2.832158in}}{\pgfqpoint{2.543479in}{2.840058in}}{\pgfqpoint{2.543479in}{2.848294in}}%
\pgfpathcurveto{\pgfqpoint{2.543479in}{2.856530in}}{\pgfqpoint{2.540207in}{2.864430in}}{\pgfqpoint{2.534383in}{2.870254in}}%
\pgfpathcurveto{\pgfqpoint{2.528559in}{2.876078in}}{\pgfqpoint{2.520659in}{2.879351in}}{\pgfqpoint{2.512423in}{2.879351in}}%
\pgfpathcurveto{\pgfqpoint{2.504187in}{2.879351in}}{\pgfqpoint{2.496287in}{2.876078in}}{\pgfqpoint{2.490463in}{2.870254in}}%
\pgfpathcurveto{\pgfqpoint{2.484639in}{2.864430in}}{\pgfqpoint{2.481366in}{2.856530in}}{\pgfqpoint{2.481366in}{2.848294in}}%
\pgfpathcurveto{\pgfqpoint{2.481366in}{2.840058in}}{\pgfqpoint{2.484639in}{2.832158in}}{\pgfqpoint{2.490463in}{2.826334in}}%
\pgfpathcurveto{\pgfqpoint{2.496287in}{2.820510in}}{\pgfqpoint{2.504187in}{2.817238in}}{\pgfqpoint{2.512423in}{2.817238in}}%
\pgfpathclose%
\pgfusepath{stroke,fill}%
\end{pgfscope}%
\begin{pgfscope}%
\pgfpathrectangle{\pgfqpoint{0.100000in}{0.212622in}}{\pgfqpoint{3.696000in}{3.696000in}}%
\pgfusepath{clip}%
\pgfsetbuttcap%
\pgfsetroundjoin%
\definecolor{currentfill}{rgb}{0.121569,0.466667,0.705882}%
\pgfsetfillcolor{currentfill}%
\pgfsetfillopacity{0.794739}%
\pgfsetlinewidth{1.003750pt}%
\definecolor{currentstroke}{rgb}{0.121569,0.466667,0.705882}%
\pgfsetstrokecolor{currentstroke}%
\pgfsetstrokeopacity{0.794739}%
\pgfsetdash{}{0pt}%
\pgfpathmoveto{\pgfqpoint{2.510937in}{2.814680in}}%
\pgfpathcurveto{\pgfqpoint{2.519173in}{2.814680in}}{\pgfqpoint{2.527073in}{2.817952in}}{\pgfqpoint{2.532897in}{2.823776in}}%
\pgfpathcurveto{\pgfqpoint{2.538721in}{2.829600in}}{\pgfqpoint{2.541993in}{2.837500in}}{\pgfqpoint{2.541993in}{2.845736in}}%
\pgfpathcurveto{\pgfqpoint{2.541993in}{2.853973in}}{\pgfqpoint{2.538721in}{2.861873in}}{\pgfqpoint{2.532897in}{2.867697in}}%
\pgfpathcurveto{\pgfqpoint{2.527073in}{2.873521in}}{\pgfqpoint{2.519173in}{2.876793in}}{\pgfqpoint{2.510937in}{2.876793in}}%
\pgfpathcurveto{\pgfqpoint{2.502701in}{2.876793in}}{\pgfqpoint{2.494801in}{2.873521in}}{\pgfqpoint{2.488977in}{2.867697in}}%
\pgfpathcurveto{\pgfqpoint{2.483153in}{2.861873in}}{\pgfqpoint{2.479880in}{2.853973in}}{\pgfqpoint{2.479880in}{2.845736in}}%
\pgfpathcurveto{\pgfqpoint{2.479880in}{2.837500in}}{\pgfqpoint{2.483153in}{2.829600in}}{\pgfqpoint{2.488977in}{2.823776in}}%
\pgfpathcurveto{\pgfqpoint{2.494801in}{2.817952in}}{\pgfqpoint{2.502701in}{2.814680in}}{\pgfqpoint{2.510937in}{2.814680in}}%
\pgfpathclose%
\pgfusepath{stroke,fill}%
\end{pgfscope}%
\begin{pgfscope}%
\pgfpathrectangle{\pgfqpoint{0.100000in}{0.212622in}}{\pgfqpoint{3.696000in}{3.696000in}}%
\pgfusepath{clip}%
\pgfsetbuttcap%
\pgfsetroundjoin%
\definecolor{currentfill}{rgb}{0.121569,0.466667,0.705882}%
\pgfsetfillcolor{currentfill}%
\pgfsetfillopacity{0.795356}%
\pgfsetlinewidth{1.003750pt}%
\definecolor{currentstroke}{rgb}{0.121569,0.466667,0.705882}%
\pgfsetstrokecolor{currentstroke}%
\pgfsetstrokeopacity{0.795356}%
\pgfsetdash{}{0pt}%
\pgfpathmoveto{\pgfqpoint{2.508260in}{2.810016in}}%
\pgfpathcurveto{\pgfqpoint{2.516496in}{2.810016in}}{\pgfqpoint{2.524396in}{2.813289in}}{\pgfqpoint{2.530220in}{2.819113in}}%
\pgfpathcurveto{\pgfqpoint{2.536044in}{2.824937in}}{\pgfqpoint{2.539316in}{2.832837in}}{\pgfqpoint{2.539316in}{2.841073in}}%
\pgfpathcurveto{\pgfqpoint{2.539316in}{2.849309in}}{\pgfqpoint{2.536044in}{2.857209in}}{\pgfqpoint{2.530220in}{2.863033in}}%
\pgfpathcurveto{\pgfqpoint{2.524396in}{2.868857in}}{\pgfqpoint{2.516496in}{2.872129in}}{\pgfqpoint{2.508260in}{2.872129in}}%
\pgfpathcurveto{\pgfqpoint{2.500024in}{2.872129in}}{\pgfqpoint{2.492124in}{2.868857in}}{\pgfqpoint{2.486300in}{2.863033in}}%
\pgfpathcurveto{\pgfqpoint{2.480476in}{2.857209in}}{\pgfqpoint{2.477203in}{2.849309in}}{\pgfqpoint{2.477203in}{2.841073in}}%
\pgfpathcurveto{\pgfqpoint{2.477203in}{2.832837in}}{\pgfqpoint{2.480476in}{2.824937in}}{\pgfqpoint{2.486300in}{2.819113in}}%
\pgfpathcurveto{\pgfqpoint{2.492124in}{2.813289in}}{\pgfqpoint{2.500024in}{2.810016in}}{\pgfqpoint{2.508260in}{2.810016in}}%
\pgfpathclose%
\pgfusepath{stroke,fill}%
\end{pgfscope}%
\begin{pgfscope}%
\pgfpathrectangle{\pgfqpoint{0.100000in}{0.212622in}}{\pgfqpoint{3.696000in}{3.696000in}}%
\pgfusepath{clip}%
\pgfsetbuttcap%
\pgfsetroundjoin%
\definecolor{currentfill}{rgb}{0.121569,0.466667,0.705882}%
\pgfsetfillcolor{currentfill}%
\pgfsetfillopacity{0.795479}%
\pgfsetlinewidth{1.003750pt}%
\definecolor{currentstroke}{rgb}{0.121569,0.466667,0.705882}%
\pgfsetstrokecolor{currentstroke}%
\pgfsetstrokeopacity{0.795479}%
\pgfsetdash{}{0pt}%
\pgfpathmoveto{\pgfqpoint{1.056920in}{1.484367in}}%
\pgfpathcurveto{\pgfqpoint{1.065156in}{1.484367in}}{\pgfqpoint{1.073056in}{1.487639in}}{\pgfqpoint{1.078880in}{1.493463in}}%
\pgfpathcurveto{\pgfqpoint{1.084704in}{1.499287in}}{\pgfqpoint{1.087976in}{1.507187in}}{\pgfqpoint{1.087976in}{1.515424in}}%
\pgfpathcurveto{\pgfqpoint{1.087976in}{1.523660in}}{\pgfqpoint{1.084704in}{1.531560in}}{\pgfqpoint{1.078880in}{1.537384in}}%
\pgfpathcurveto{\pgfqpoint{1.073056in}{1.543208in}}{\pgfqpoint{1.065156in}{1.546480in}}{\pgfqpoint{1.056920in}{1.546480in}}%
\pgfpathcurveto{\pgfqpoint{1.048683in}{1.546480in}}{\pgfqpoint{1.040783in}{1.543208in}}{\pgfqpoint{1.034959in}{1.537384in}}%
\pgfpathcurveto{\pgfqpoint{1.029135in}{1.531560in}}{\pgfqpoint{1.025863in}{1.523660in}}{\pgfqpoint{1.025863in}{1.515424in}}%
\pgfpathcurveto{\pgfqpoint{1.025863in}{1.507187in}}{\pgfqpoint{1.029135in}{1.499287in}}{\pgfqpoint{1.034959in}{1.493463in}}%
\pgfpathcurveto{\pgfqpoint{1.040783in}{1.487639in}}{\pgfqpoint{1.048683in}{1.484367in}}{\pgfqpoint{1.056920in}{1.484367in}}%
\pgfpathclose%
\pgfusepath{stroke,fill}%
\end{pgfscope}%
\begin{pgfscope}%
\pgfpathrectangle{\pgfqpoint{0.100000in}{0.212622in}}{\pgfqpoint{3.696000in}{3.696000in}}%
\pgfusepath{clip}%
\pgfsetbuttcap%
\pgfsetroundjoin%
\definecolor{currentfill}{rgb}{0.121569,0.466667,0.705882}%
\pgfsetfillcolor{currentfill}%
\pgfsetfillopacity{0.795817}%
\pgfsetlinewidth{1.003750pt}%
\definecolor{currentstroke}{rgb}{0.121569,0.466667,0.705882}%
\pgfsetstrokecolor{currentstroke}%
\pgfsetstrokeopacity{0.795817}%
\pgfsetdash{}{0pt}%
\pgfpathmoveto{\pgfqpoint{2.507508in}{2.807066in}}%
\pgfpathcurveto{\pgfqpoint{2.515745in}{2.807066in}}{\pgfqpoint{2.523645in}{2.810338in}}{\pgfqpoint{2.529469in}{2.816162in}}%
\pgfpathcurveto{\pgfqpoint{2.535292in}{2.821986in}}{\pgfqpoint{2.538565in}{2.829886in}}{\pgfqpoint{2.538565in}{2.838122in}}%
\pgfpathcurveto{\pgfqpoint{2.538565in}{2.846358in}}{\pgfqpoint{2.535292in}{2.854258in}}{\pgfqpoint{2.529469in}{2.860082in}}%
\pgfpathcurveto{\pgfqpoint{2.523645in}{2.865906in}}{\pgfqpoint{2.515745in}{2.869179in}}{\pgfqpoint{2.507508in}{2.869179in}}%
\pgfpathcurveto{\pgfqpoint{2.499272in}{2.869179in}}{\pgfqpoint{2.491372in}{2.865906in}}{\pgfqpoint{2.485548in}{2.860082in}}%
\pgfpathcurveto{\pgfqpoint{2.479724in}{2.854258in}}{\pgfqpoint{2.476452in}{2.846358in}}{\pgfqpoint{2.476452in}{2.838122in}}%
\pgfpathcurveto{\pgfqpoint{2.476452in}{2.829886in}}{\pgfqpoint{2.479724in}{2.821986in}}{\pgfqpoint{2.485548in}{2.816162in}}%
\pgfpathcurveto{\pgfqpoint{2.491372in}{2.810338in}}{\pgfqpoint{2.499272in}{2.807066in}}{\pgfqpoint{2.507508in}{2.807066in}}%
\pgfpathclose%
\pgfusepath{stroke,fill}%
\end{pgfscope}%
\begin{pgfscope}%
\pgfpathrectangle{\pgfqpoint{0.100000in}{0.212622in}}{\pgfqpoint{3.696000in}{3.696000in}}%
\pgfusepath{clip}%
\pgfsetbuttcap%
\pgfsetroundjoin%
\definecolor{currentfill}{rgb}{0.121569,0.466667,0.705882}%
\pgfsetfillcolor{currentfill}%
\pgfsetfillopacity{0.796595}%
\pgfsetlinewidth{1.003750pt}%
\definecolor{currentstroke}{rgb}{0.121569,0.466667,0.705882}%
\pgfsetstrokecolor{currentstroke}%
\pgfsetstrokeopacity{0.796595}%
\pgfsetdash{}{0pt}%
\pgfpathmoveto{\pgfqpoint{2.506805in}{2.801427in}}%
\pgfpathcurveto{\pgfqpoint{2.515042in}{2.801427in}}{\pgfqpoint{2.522942in}{2.804699in}}{\pgfqpoint{2.528766in}{2.810523in}}%
\pgfpathcurveto{\pgfqpoint{2.534590in}{2.816347in}}{\pgfqpoint{2.537862in}{2.824247in}}{\pgfqpoint{2.537862in}{2.832483in}}%
\pgfpathcurveto{\pgfqpoint{2.537862in}{2.840719in}}{\pgfqpoint{2.534590in}{2.848619in}}{\pgfqpoint{2.528766in}{2.854443in}}%
\pgfpathcurveto{\pgfqpoint{2.522942in}{2.860267in}}{\pgfqpoint{2.515042in}{2.863540in}}{\pgfqpoint{2.506805in}{2.863540in}}%
\pgfpathcurveto{\pgfqpoint{2.498569in}{2.863540in}}{\pgfqpoint{2.490669in}{2.860267in}}{\pgfqpoint{2.484845in}{2.854443in}}%
\pgfpathcurveto{\pgfqpoint{2.479021in}{2.848619in}}{\pgfqpoint{2.475749in}{2.840719in}}{\pgfqpoint{2.475749in}{2.832483in}}%
\pgfpathcurveto{\pgfqpoint{2.475749in}{2.824247in}}{\pgfqpoint{2.479021in}{2.816347in}}{\pgfqpoint{2.484845in}{2.810523in}}%
\pgfpathcurveto{\pgfqpoint{2.490669in}{2.804699in}}{\pgfqpoint{2.498569in}{2.801427in}}{\pgfqpoint{2.506805in}{2.801427in}}%
\pgfpathclose%
\pgfusepath{stroke,fill}%
\end{pgfscope}%
\begin{pgfscope}%
\pgfpathrectangle{\pgfqpoint{0.100000in}{0.212622in}}{\pgfqpoint{3.696000in}{3.696000in}}%
\pgfusepath{clip}%
\pgfsetbuttcap%
\pgfsetroundjoin%
\definecolor{currentfill}{rgb}{0.121569,0.466667,0.705882}%
\pgfsetfillcolor{currentfill}%
\pgfsetfillopacity{0.796852}%
\pgfsetlinewidth{1.003750pt}%
\definecolor{currentstroke}{rgb}{0.121569,0.466667,0.705882}%
\pgfsetstrokecolor{currentstroke}%
\pgfsetstrokeopacity{0.796852}%
\pgfsetdash{}{0pt}%
\pgfpathmoveto{\pgfqpoint{3.048871in}{1.958667in}}%
\pgfpathcurveto{\pgfqpoint{3.057108in}{1.958667in}}{\pgfqpoint{3.065008in}{1.961939in}}{\pgfqpoint{3.070832in}{1.967763in}}%
\pgfpathcurveto{\pgfqpoint{3.076655in}{1.973587in}}{\pgfqpoint{3.079928in}{1.981487in}}{\pgfqpoint{3.079928in}{1.989724in}}%
\pgfpathcurveto{\pgfqpoint{3.079928in}{1.997960in}}{\pgfqpoint{3.076655in}{2.005860in}}{\pgfqpoint{3.070832in}{2.011684in}}%
\pgfpathcurveto{\pgfqpoint{3.065008in}{2.017508in}}{\pgfqpoint{3.057108in}{2.020780in}}{\pgfqpoint{3.048871in}{2.020780in}}%
\pgfpathcurveto{\pgfqpoint{3.040635in}{2.020780in}}{\pgfqpoint{3.032735in}{2.017508in}}{\pgfqpoint{3.026911in}{2.011684in}}%
\pgfpathcurveto{\pgfqpoint{3.021087in}{2.005860in}}{\pgfqpoint{3.017815in}{1.997960in}}{\pgfqpoint{3.017815in}{1.989724in}}%
\pgfpathcurveto{\pgfqpoint{3.017815in}{1.981487in}}{\pgfqpoint{3.021087in}{1.973587in}}{\pgfqpoint{3.026911in}{1.967763in}}%
\pgfpathcurveto{\pgfqpoint{3.032735in}{1.961939in}}{\pgfqpoint{3.040635in}{1.958667in}}{\pgfqpoint{3.048871in}{1.958667in}}%
\pgfpathclose%
\pgfusepath{stroke,fill}%
\end{pgfscope}%
\begin{pgfscope}%
\pgfpathrectangle{\pgfqpoint{0.100000in}{0.212622in}}{\pgfqpoint{3.696000in}{3.696000in}}%
\pgfusepath{clip}%
\pgfsetbuttcap%
\pgfsetroundjoin%
\definecolor{currentfill}{rgb}{0.121569,0.466667,0.705882}%
\pgfsetfillcolor{currentfill}%
\pgfsetfillopacity{0.796931}%
\pgfsetlinewidth{1.003750pt}%
\definecolor{currentstroke}{rgb}{0.121569,0.466667,0.705882}%
\pgfsetstrokecolor{currentstroke}%
\pgfsetstrokeopacity{0.796931}%
\pgfsetdash{}{0pt}%
\pgfpathmoveto{\pgfqpoint{1.064850in}{1.483231in}}%
\pgfpathcurveto{\pgfqpoint{1.073086in}{1.483231in}}{\pgfqpoint{1.080987in}{1.486503in}}{\pgfqpoint{1.086810in}{1.492327in}}%
\pgfpathcurveto{\pgfqpoint{1.092634in}{1.498151in}}{\pgfqpoint{1.095907in}{1.506051in}}{\pgfqpoint{1.095907in}{1.514287in}}%
\pgfpathcurveto{\pgfqpoint{1.095907in}{1.522523in}}{\pgfqpoint{1.092634in}{1.530424in}}{\pgfqpoint{1.086810in}{1.536247in}}%
\pgfpathcurveto{\pgfqpoint{1.080987in}{1.542071in}}{\pgfqpoint{1.073086in}{1.545344in}}{\pgfqpoint{1.064850in}{1.545344in}}%
\pgfpathcurveto{\pgfqpoint{1.056614in}{1.545344in}}{\pgfqpoint{1.048714in}{1.542071in}}{\pgfqpoint{1.042890in}{1.536247in}}%
\pgfpathcurveto{\pgfqpoint{1.037066in}{1.530424in}}{\pgfqpoint{1.033794in}{1.522523in}}{\pgfqpoint{1.033794in}{1.514287in}}%
\pgfpathcurveto{\pgfqpoint{1.033794in}{1.506051in}}{\pgfqpoint{1.037066in}{1.498151in}}{\pgfqpoint{1.042890in}{1.492327in}}%
\pgfpathcurveto{\pgfqpoint{1.048714in}{1.486503in}}{\pgfqpoint{1.056614in}{1.483231in}}{\pgfqpoint{1.064850in}{1.483231in}}%
\pgfpathclose%
\pgfusepath{stroke,fill}%
\end{pgfscope}%
\begin{pgfscope}%
\pgfpathrectangle{\pgfqpoint{0.100000in}{0.212622in}}{\pgfqpoint{3.696000in}{3.696000in}}%
\pgfusepath{clip}%
\pgfsetbuttcap%
\pgfsetroundjoin%
\definecolor{currentfill}{rgb}{0.121569,0.466667,0.705882}%
\pgfsetfillcolor{currentfill}%
\pgfsetfillopacity{0.797121}%
\pgfsetlinewidth{1.003750pt}%
\definecolor{currentstroke}{rgb}{0.121569,0.466667,0.705882}%
\pgfsetstrokecolor{currentstroke}%
\pgfsetstrokeopacity{0.797121}%
\pgfsetdash{}{0pt}%
\pgfpathmoveto{\pgfqpoint{2.505226in}{2.798391in}}%
\pgfpathcurveto{\pgfqpoint{2.513462in}{2.798391in}}{\pgfqpoint{2.521362in}{2.801663in}}{\pgfqpoint{2.527186in}{2.807487in}}%
\pgfpathcurveto{\pgfqpoint{2.533010in}{2.813311in}}{\pgfqpoint{2.536283in}{2.821211in}}{\pgfqpoint{2.536283in}{2.829447in}}%
\pgfpathcurveto{\pgfqpoint{2.536283in}{2.837683in}}{\pgfqpoint{2.533010in}{2.845584in}}{\pgfqpoint{2.527186in}{2.851407in}}%
\pgfpathcurveto{\pgfqpoint{2.521362in}{2.857231in}}{\pgfqpoint{2.513462in}{2.860504in}}{\pgfqpoint{2.505226in}{2.860504in}}%
\pgfpathcurveto{\pgfqpoint{2.496990in}{2.860504in}}{\pgfqpoint{2.489090in}{2.857231in}}{\pgfqpoint{2.483266in}{2.851407in}}%
\pgfpathcurveto{\pgfqpoint{2.477442in}{2.845584in}}{\pgfqpoint{2.474170in}{2.837683in}}{\pgfqpoint{2.474170in}{2.829447in}}%
\pgfpathcurveto{\pgfqpoint{2.474170in}{2.821211in}}{\pgfqpoint{2.477442in}{2.813311in}}{\pgfqpoint{2.483266in}{2.807487in}}%
\pgfpathcurveto{\pgfqpoint{2.489090in}{2.801663in}}{\pgfqpoint{2.496990in}{2.798391in}}{\pgfqpoint{2.505226in}{2.798391in}}%
\pgfpathclose%
\pgfusepath{stroke,fill}%
\end{pgfscope}%
\begin{pgfscope}%
\pgfpathrectangle{\pgfqpoint{0.100000in}{0.212622in}}{\pgfqpoint{3.696000in}{3.696000in}}%
\pgfusepath{clip}%
\pgfsetbuttcap%
\pgfsetroundjoin%
\definecolor{currentfill}{rgb}{0.121569,0.466667,0.705882}%
\pgfsetfillcolor{currentfill}%
\pgfsetfillopacity{0.798018}%
\pgfsetlinewidth{1.003750pt}%
\definecolor{currentstroke}{rgb}{0.121569,0.466667,0.705882}%
\pgfsetstrokecolor{currentstroke}%
\pgfsetstrokeopacity{0.798018}%
\pgfsetdash{}{0pt}%
\pgfpathmoveto{\pgfqpoint{2.502256in}{2.792711in}}%
\pgfpathcurveto{\pgfqpoint{2.510492in}{2.792711in}}{\pgfqpoint{2.518392in}{2.795983in}}{\pgfqpoint{2.524216in}{2.801807in}}%
\pgfpathcurveto{\pgfqpoint{2.530040in}{2.807631in}}{\pgfqpoint{2.533312in}{2.815531in}}{\pgfqpoint{2.533312in}{2.823768in}}%
\pgfpathcurveto{\pgfqpoint{2.533312in}{2.832004in}}{\pgfqpoint{2.530040in}{2.839904in}}{\pgfqpoint{2.524216in}{2.845728in}}%
\pgfpathcurveto{\pgfqpoint{2.518392in}{2.851552in}}{\pgfqpoint{2.510492in}{2.854824in}}{\pgfqpoint{2.502256in}{2.854824in}}%
\pgfpathcurveto{\pgfqpoint{2.494019in}{2.854824in}}{\pgfqpoint{2.486119in}{2.851552in}}{\pgfqpoint{2.480295in}{2.845728in}}%
\pgfpathcurveto{\pgfqpoint{2.474471in}{2.839904in}}{\pgfqpoint{2.471199in}{2.832004in}}{\pgfqpoint{2.471199in}{2.823768in}}%
\pgfpathcurveto{\pgfqpoint{2.471199in}{2.815531in}}{\pgfqpoint{2.474471in}{2.807631in}}{\pgfqpoint{2.480295in}{2.801807in}}%
\pgfpathcurveto{\pgfqpoint{2.486119in}{2.795983in}}{\pgfqpoint{2.494019in}{2.792711in}}{\pgfqpoint{2.502256in}{2.792711in}}%
\pgfpathclose%
\pgfusepath{stroke,fill}%
\end{pgfscope}%
\begin{pgfscope}%
\pgfpathrectangle{\pgfqpoint{0.100000in}{0.212622in}}{\pgfqpoint{3.696000in}{3.696000in}}%
\pgfusepath{clip}%
\pgfsetbuttcap%
\pgfsetroundjoin%
\definecolor{currentfill}{rgb}{0.121569,0.466667,0.705882}%
\pgfsetfillcolor{currentfill}%
\pgfsetfillopacity{0.798671}%
\pgfsetlinewidth{1.003750pt}%
\definecolor{currentstroke}{rgb}{0.121569,0.466667,0.705882}%
\pgfsetstrokecolor{currentstroke}%
\pgfsetstrokeopacity{0.798671}%
\pgfsetdash{}{0pt}%
\pgfpathmoveto{\pgfqpoint{1.073673in}{1.479348in}}%
\pgfpathcurveto{\pgfqpoint{1.081909in}{1.479348in}}{\pgfqpoint{1.089809in}{1.482620in}}{\pgfqpoint{1.095633in}{1.488444in}}%
\pgfpathcurveto{\pgfqpoint{1.101457in}{1.494268in}}{\pgfqpoint{1.104730in}{1.502168in}}{\pgfqpoint{1.104730in}{1.510404in}}%
\pgfpathcurveto{\pgfqpoint{1.104730in}{1.518641in}}{\pgfqpoint{1.101457in}{1.526541in}}{\pgfqpoint{1.095633in}{1.532365in}}%
\pgfpathcurveto{\pgfqpoint{1.089809in}{1.538189in}}{\pgfqpoint{1.081909in}{1.541461in}}{\pgfqpoint{1.073673in}{1.541461in}}%
\pgfpathcurveto{\pgfqpoint{1.065437in}{1.541461in}}{\pgfqpoint{1.057537in}{1.538189in}}{\pgfqpoint{1.051713in}{1.532365in}}%
\pgfpathcurveto{\pgfqpoint{1.045889in}{1.526541in}}{\pgfqpoint{1.042617in}{1.518641in}}{\pgfqpoint{1.042617in}{1.510404in}}%
\pgfpathcurveto{\pgfqpoint{1.042617in}{1.502168in}}{\pgfqpoint{1.045889in}{1.494268in}}{\pgfqpoint{1.051713in}{1.488444in}}%
\pgfpathcurveto{\pgfqpoint{1.057537in}{1.482620in}}{\pgfqpoint{1.065437in}{1.479348in}}{\pgfqpoint{1.073673in}{1.479348in}}%
\pgfpathclose%
\pgfusepath{stroke,fill}%
\end{pgfscope}%
\begin{pgfscope}%
\pgfpathrectangle{\pgfqpoint{0.100000in}{0.212622in}}{\pgfqpoint{3.696000in}{3.696000in}}%
\pgfusepath{clip}%
\pgfsetbuttcap%
\pgfsetroundjoin%
\definecolor{currentfill}{rgb}{0.121569,0.466667,0.705882}%
\pgfsetfillcolor{currentfill}%
\pgfsetfillopacity{0.798713}%
\pgfsetlinewidth{1.003750pt}%
\definecolor{currentstroke}{rgb}{0.121569,0.466667,0.705882}%
\pgfsetstrokecolor{currentstroke}%
\pgfsetstrokeopacity{0.798713}%
\pgfsetdash{}{0pt}%
\pgfpathmoveto{\pgfqpoint{2.501178in}{2.788737in}}%
\pgfpathcurveto{\pgfqpoint{2.509415in}{2.788737in}}{\pgfqpoint{2.517315in}{2.792009in}}{\pgfqpoint{2.523139in}{2.797833in}}%
\pgfpathcurveto{\pgfqpoint{2.528963in}{2.803657in}}{\pgfqpoint{2.532235in}{2.811557in}}{\pgfqpoint{2.532235in}{2.819793in}}%
\pgfpathcurveto{\pgfqpoint{2.532235in}{2.828030in}}{\pgfqpoint{2.528963in}{2.835930in}}{\pgfqpoint{2.523139in}{2.841754in}}%
\pgfpathcurveto{\pgfqpoint{2.517315in}{2.847578in}}{\pgfqpoint{2.509415in}{2.850850in}}{\pgfqpoint{2.501178in}{2.850850in}}%
\pgfpathcurveto{\pgfqpoint{2.492942in}{2.850850in}}{\pgfqpoint{2.485042in}{2.847578in}}{\pgfqpoint{2.479218in}{2.841754in}}%
\pgfpathcurveto{\pgfqpoint{2.473394in}{2.835930in}}{\pgfqpoint{2.470122in}{2.828030in}}{\pgfqpoint{2.470122in}{2.819793in}}%
\pgfpathcurveto{\pgfqpoint{2.470122in}{2.811557in}}{\pgfqpoint{2.473394in}{2.803657in}}{\pgfqpoint{2.479218in}{2.797833in}}%
\pgfpathcurveto{\pgfqpoint{2.485042in}{2.792009in}}{\pgfqpoint{2.492942in}{2.788737in}}{\pgfqpoint{2.501178in}{2.788737in}}%
\pgfpathclose%
\pgfusepath{stroke,fill}%
\end{pgfscope}%
\begin{pgfscope}%
\pgfpathrectangle{\pgfqpoint{0.100000in}{0.212622in}}{\pgfqpoint{3.696000in}{3.696000in}}%
\pgfusepath{clip}%
\pgfsetbuttcap%
\pgfsetroundjoin%
\definecolor{currentfill}{rgb}{0.121569,0.466667,0.705882}%
\pgfsetfillcolor{currentfill}%
\pgfsetfillopacity{0.799836}%
\pgfsetlinewidth{1.003750pt}%
\definecolor{currentstroke}{rgb}{0.121569,0.466667,0.705882}%
\pgfsetstrokecolor{currentstroke}%
\pgfsetstrokeopacity{0.799836}%
\pgfsetdash{}{0pt}%
\pgfpathmoveto{\pgfqpoint{2.500561in}{2.780972in}}%
\pgfpathcurveto{\pgfqpoint{2.508798in}{2.780972in}}{\pgfqpoint{2.516698in}{2.784244in}}{\pgfqpoint{2.522522in}{2.790068in}}%
\pgfpathcurveto{\pgfqpoint{2.528346in}{2.795892in}}{\pgfqpoint{2.531618in}{2.803792in}}{\pgfqpoint{2.531618in}{2.812029in}}%
\pgfpathcurveto{\pgfqpoint{2.531618in}{2.820265in}}{\pgfqpoint{2.528346in}{2.828165in}}{\pgfqpoint{2.522522in}{2.833989in}}%
\pgfpathcurveto{\pgfqpoint{2.516698in}{2.839813in}}{\pgfqpoint{2.508798in}{2.843085in}}{\pgfqpoint{2.500561in}{2.843085in}}%
\pgfpathcurveto{\pgfqpoint{2.492325in}{2.843085in}}{\pgfqpoint{2.484425in}{2.839813in}}{\pgfqpoint{2.478601in}{2.833989in}}%
\pgfpathcurveto{\pgfqpoint{2.472777in}{2.828165in}}{\pgfqpoint{2.469505in}{2.820265in}}{\pgfqpoint{2.469505in}{2.812029in}}%
\pgfpathcurveto{\pgfqpoint{2.469505in}{2.803792in}}{\pgfqpoint{2.472777in}{2.795892in}}{\pgfqpoint{2.478601in}{2.790068in}}%
\pgfpathcurveto{\pgfqpoint{2.484425in}{2.784244in}}{\pgfqpoint{2.492325in}{2.780972in}}{\pgfqpoint{2.500561in}{2.780972in}}%
\pgfpathclose%
\pgfusepath{stroke,fill}%
\end{pgfscope}%
\begin{pgfscope}%
\pgfpathrectangle{\pgfqpoint{0.100000in}{0.212622in}}{\pgfqpoint{3.696000in}{3.696000in}}%
\pgfusepath{clip}%
\pgfsetbuttcap%
\pgfsetroundjoin%
\definecolor{currentfill}{rgb}{0.121569,0.466667,0.705882}%
\pgfsetfillcolor{currentfill}%
\pgfsetfillopacity{0.800322}%
\pgfsetlinewidth{1.003750pt}%
\definecolor{currentstroke}{rgb}{0.121569,0.466667,0.705882}%
\pgfsetstrokecolor{currentstroke}%
\pgfsetstrokeopacity{0.800322}%
\pgfsetdash{}{0pt}%
\pgfpathmoveto{\pgfqpoint{1.083182in}{1.474057in}}%
\pgfpathcurveto{\pgfqpoint{1.091418in}{1.474057in}}{\pgfqpoint{1.099319in}{1.477329in}}{\pgfqpoint{1.105142in}{1.483153in}}%
\pgfpathcurveto{\pgfqpoint{1.110966in}{1.488977in}}{\pgfqpoint{1.114239in}{1.496877in}}{\pgfqpoint{1.114239in}{1.505113in}}%
\pgfpathcurveto{\pgfqpoint{1.114239in}{1.513349in}}{\pgfqpoint{1.110966in}{1.521249in}}{\pgfqpoint{1.105142in}{1.527073in}}%
\pgfpathcurveto{\pgfqpoint{1.099319in}{1.532897in}}{\pgfqpoint{1.091418in}{1.536170in}}{\pgfqpoint{1.083182in}{1.536170in}}%
\pgfpathcurveto{\pgfqpoint{1.074946in}{1.536170in}}{\pgfqpoint{1.067046in}{1.532897in}}{\pgfqpoint{1.061222in}{1.527073in}}%
\pgfpathcurveto{\pgfqpoint{1.055398in}{1.521249in}}{\pgfqpoint{1.052126in}{1.513349in}}{\pgfqpoint{1.052126in}{1.505113in}}%
\pgfpathcurveto{\pgfqpoint{1.052126in}{1.496877in}}{\pgfqpoint{1.055398in}{1.488977in}}{\pgfqpoint{1.061222in}{1.483153in}}%
\pgfpathcurveto{\pgfqpoint{1.067046in}{1.477329in}}{\pgfqpoint{1.074946in}{1.474057in}}{\pgfqpoint{1.083182in}{1.474057in}}%
\pgfpathclose%
\pgfusepath{stroke,fill}%
\end{pgfscope}%
\begin{pgfscope}%
\pgfpathrectangle{\pgfqpoint{0.100000in}{0.212622in}}{\pgfqpoint{3.696000in}{3.696000in}}%
\pgfusepath{clip}%
\pgfsetbuttcap%
\pgfsetroundjoin%
\definecolor{currentfill}{rgb}{0.121569,0.466667,0.705882}%
\pgfsetfillcolor{currentfill}%
\pgfsetfillopacity{0.800603}%
\pgfsetlinewidth{1.003750pt}%
\definecolor{currentstroke}{rgb}{0.121569,0.466667,0.705882}%
\pgfsetstrokecolor{currentstroke}%
\pgfsetstrokeopacity{0.800603}%
\pgfsetdash{}{0pt}%
\pgfpathmoveto{\pgfqpoint{2.498399in}{2.776923in}}%
\pgfpathcurveto{\pgfqpoint{2.506635in}{2.776923in}}{\pgfqpoint{2.514535in}{2.780196in}}{\pgfqpoint{2.520359in}{2.786020in}}%
\pgfpathcurveto{\pgfqpoint{2.526183in}{2.791844in}}{\pgfqpoint{2.529455in}{2.799744in}}{\pgfqpoint{2.529455in}{2.807980in}}%
\pgfpathcurveto{\pgfqpoint{2.529455in}{2.816216in}}{\pgfqpoint{2.526183in}{2.824116in}}{\pgfqpoint{2.520359in}{2.829940in}}%
\pgfpathcurveto{\pgfqpoint{2.514535in}{2.835764in}}{\pgfqpoint{2.506635in}{2.839036in}}{\pgfqpoint{2.498399in}{2.839036in}}%
\pgfpathcurveto{\pgfqpoint{2.490162in}{2.839036in}}{\pgfqpoint{2.482262in}{2.835764in}}{\pgfqpoint{2.476438in}{2.829940in}}%
\pgfpathcurveto{\pgfqpoint{2.470615in}{2.824116in}}{\pgfqpoint{2.467342in}{2.816216in}}{\pgfqpoint{2.467342in}{2.807980in}}%
\pgfpathcurveto{\pgfqpoint{2.467342in}{2.799744in}}{\pgfqpoint{2.470615in}{2.791844in}}{\pgfqpoint{2.476438in}{2.786020in}}%
\pgfpathcurveto{\pgfqpoint{2.482262in}{2.780196in}}{\pgfqpoint{2.490162in}{2.776923in}}{\pgfqpoint{2.498399in}{2.776923in}}%
\pgfpathclose%
\pgfusepath{stroke,fill}%
\end{pgfscope}%
\begin{pgfscope}%
\pgfpathrectangle{\pgfqpoint{0.100000in}{0.212622in}}{\pgfqpoint{3.696000in}{3.696000in}}%
\pgfusepath{clip}%
\pgfsetbuttcap%
\pgfsetroundjoin%
\definecolor{currentfill}{rgb}{0.121569,0.466667,0.705882}%
\pgfsetfillcolor{currentfill}%
\pgfsetfillopacity{0.801129}%
\pgfsetlinewidth{1.003750pt}%
\definecolor{currentstroke}{rgb}{0.121569,0.466667,0.705882}%
\pgfsetstrokecolor{currentstroke}%
\pgfsetstrokeopacity{0.801129}%
\pgfsetdash{}{0pt}%
\pgfpathmoveto{\pgfqpoint{1.088466in}{1.470830in}}%
\pgfpathcurveto{\pgfqpoint{1.096703in}{1.470830in}}{\pgfqpoint{1.104603in}{1.474102in}}{\pgfqpoint{1.110427in}{1.479926in}}%
\pgfpathcurveto{\pgfqpoint{1.116250in}{1.485750in}}{\pgfqpoint{1.119523in}{1.493650in}}{\pgfqpoint{1.119523in}{1.501886in}}%
\pgfpathcurveto{\pgfqpoint{1.119523in}{1.510122in}}{\pgfqpoint{1.116250in}{1.518022in}}{\pgfqpoint{1.110427in}{1.523846in}}%
\pgfpathcurveto{\pgfqpoint{1.104603in}{1.529670in}}{\pgfqpoint{1.096703in}{1.532943in}}{\pgfqpoint{1.088466in}{1.532943in}}%
\pgfpathcurveto{\pgfqpoint{1.080230in}{1.532943in}}{\pgfqpoint{1.072330in}{1.529670in}}{\pgfqpoint{1.066506in}{1.523846in}}%
\pgfpathcurveto{\pgfqpoint{1.060682in}{1.518022in}}{\pgfqpoint{1.057410in}{1.510122in}}{\pgfqpoint{1.057410in}{1.501886in}}%
\pgfpathcurveto{\pgfqpoint{1.057410in}{1.493650in}}{\pgfqpoint{1.060682in}{1.485750in}}{\pgfqpoint{1.066506in}{1.479926in}}%
\pgfpathcurveto{\pgfqpoint{1.072330in}{1.474102in}}{\pgfqpoint{1.080230in}{1.470830in}}{\pgfqpoint{1.088466in}{1.470830in}}%
\pgfpathclose%
\pgfusepath{stroke,fill}%
\end{pgfscope}%
\begin{pgfscope}%
\pgfpathrectangle{\pgfqpoint{0.100000in}{0.212622in}}{\pgfqpoint{3.696000in}{3.696000in}}%
\pgfusepath{clip}%
\pgfsetbuttcap%
\pgfsetroundjoin%
\definecolor{currentfill}{rgb}{0.121569,0.466667,0.705882}%
\pgfsetfillcolor{currentfill}%
\pgfsetfillopacity{0.801660}%
\pgfsetlinewidth{1.003750pt}%
\definecolor{currentstroke}{rgb}{0.121569,0.466667,0.705882}%
\pgfsetstrokecolor{currentstroke}%
\pgfsetstrokeopacity{0.801660}%
\pgfsetdash{}{0pt}%
\pgfpathmoveto{\pgfqpoint{3.039701in}{1.934785in}}%
\pgfpathcurveto{\pgfqpoint{3.047937in}{1.934785in}}{\pgfqpoint{3.055838in}{1.938057in}}{\pgfqpoint{3.061661in}{1.943881in}}%
\pgfpathcurveto{\pgfqpoint{3.067485in}{1.949705in}}{\pgfqpoint{3.070758in}{1.957605in}}{\pgfqpoint{3.070758in}{1.965841in}}%
\pgfpathcurveto{\pgfqpoint{3.070758in}{1.974078in}}{\pgfqpoint{3.067485in}{1.981978in}}{\pgfqpoint{3.061661in}{1.987802in}}%
\pgfpathcurveto{\pgfqpoint{3.055838in}{1.993625in}}{\pgfqpoint{3.047937in}{1.996898in}}{\pgfqpoint{3.039701in}{1.996898in}}%
\pgfpathcurveto{\pgfqpoint{3.031465in}{1.996898in}}{\pgfqpoint{3.023565in}{1.993625in}}{\pgfqpoint{3.017741in}{1.987802in}}%
\pgfpathcurveto{\pgfqpoint{3.011917in}{1.981978in}}{\pgfqpoint{3.008645in}{1.974078in}}{\pgfqpoint{3.008645in}{1.965841in}}%
\pgfpathcurveto{\pgfqpoint{3.008645in}{1.957605in}}{\pgfqpoint{3.011917in}{1.949705in}}{\pgfqpoint{3.017741in}{1.943881in}}%
\pgfpathcurveto{\pgfqpoint{3.023565in}{1.938057in}}{\pgfqpoint{3.031465in}{1.934785in}}{\pgfqpoint{3.039701in}{1.934785in}}%
\pgfpathclose%
\pgfusepath{stroke,fill}%
\end{pgfscope}%
\begin{pgfscope}%
\pgfpathrectangle{\pgfqpoint{0.100000in}{0.212622in}}{\pgfqpoint{3.696000in}{3.696000in}}%
\pgfusepath{clip}%
\pgfsetbuttcap%
\pgfsetroundjoin%
\definecolor{currentfill}{rgb}{0.121569,0.466667,0.705882}%
\pgfsetfillcolor{currentfill}%
\pgfsetfillopacity{0.801806}%
\pgfsetlinewidth{1.003750pt}%
\definecolor{currentstroke}{rgb}{0.121569,0.466667,0.705882}%
\pgfsetstrokecolor{currentstroke}%
\pgfsetstrokeopacity{0.801806}%
\pgfsetdash{}{0pt}%
\pgfpathmoveto{\pgfqpoint{2.494157in}{2.769076in}}%
\pgfpathcurveto{\pgfqpoint{2.502393in}{2.769076in}}{\pgfqpoint{2.510293in}{2.772348in}}{\pgfqpoint{2.516117in}{2.778172in}}%
\pgfpathcurveto{\pgfqpoint{2.521941in}{2.783996in}}{\pgfqpoint{2.525213in}{2.791896in}}{\pgfqpoint{2.525213in}{2.800132in}}%
\pgfpathcurveto{\pgfqpoint{2.525213in}{2.808368in}}{\pgfqpoint{2.521941in}{2.816268in}}{\pgfqpoint{2.516117in}{2.822092in}}%
\pgfpathcurveto{\pgfqpoint{2.510293in}{2.827916in}}{\pgfqpoint{2.502393in}{2.831189in}}{\pgfqpoint{2.494157in}{2.831189in}}%
\pgfpathcurveto{\pgfqpoint{2.485920in}{2.831189in}}{\pgfqpoint{2.478020in}{2.827916in}}{\pgfqpoint{2.472196in}{2.822092in}}%
\pgfpathcurveto{\pgfqpoint{2.466372in}{2.816268in}}{\pgfqpoint{2.463100in}{2.808368in}}{\pgfqpoint{2.463100in}{2.800132in}}%
\pgfpathcurveto{\pgfqpoint{2.463100in}{2.791896in}}{\pgfqpoint{2.466372in}{2.783996in}}{\pgfqpoint{2.472196in}{2.778172in}}%
\pgfpathcurveto{\pgfqpoint{2.478020in}{2.772348in}}{\pgfqpoint{2.485920in}{2.769076in}}{\pgfqpoint{2.494157in}{2.769076in}}%
\pgfpathclose%
\pgfusepath{stroke,fill}%
\end{pgfscope}%
\begin{pgfscope}%
\pgfpathrectangle{\pgfqpoint{0.100000in}{0.212622in}}{\pgfqpoint{3.696000in}{3.696000in}}%
\pgfusepath{clip}%
\pgfsetbuttcap%
\pgfsetroundjoin%
\definecolor{currentfill}{rgb}{0.121569,0.466667,0.705882}%
\pgfsetfillcolor{currentfill}%
\pgfsetfillopacity{0.802122}%
\pgfsetlinewidth{1.003750pt}%
\definecolor{currentstroke}{rgb}{0.121569,0.466667,0.705882}%
\pgfsetstrokecolor{currentstroke}%
\pgfsetstrokeopacity{0.802122}%
\pgfsetdash{}{0pt}%
\pgfpathmoveto{\pgfqpoint{1.095494in}{1.467780in}}%
\pgfpathcurveto{\pgfqpoint{1.103730in}{1.467780in}}{\pgfqpoint{1.111630in}{1.471052in}}{\pgfqpoint{1.117454in}{1.476876in}}%
\pgfpathcurveto{\pgfqpoint{1.123278in}{1.482700in}}{\pgfqpoint{1.126550in}{1.490600in}}{\pgfqpoint{1.126550in}{1.498836in}}%
\pgfpathcurveto{\pgfqpoint{1.126550in}{1.507073in}}{\pgfqpoint{1.123278in}{1.514973in}}{\pgfqpoint{1.117454in}{1.520797in}}%
\pgfpathcurveto{\pgfqpoint{1.111630in}{1.526620in}}{\pgfqpoint{1.103730in}{1.529893in}}{\pgfqpoint{1.095494in}{1.529893in}}%
\pgfpathcurveto{\pgfqpoint{1.087258in}{1.529893in}}{\pgfqpoint{1.079358in}{1.526620in}}{\pgfqpoint{1.073534in}{1.520797in}}%
\pgfpathcurveto{\pgfqpoint{1.067710in}{1.514973in}}{\pgfqpoint{1.064437in}{1.507073in}}{\pgfqpoint{1.064437in}{1.498836in}}%
\pgfpathcurveto{\pgfqpoint{1.064437in}{1.490600in}}{\pgfqpoint{1.067710in}{1.482700in}}{\pgfqpoint{1.073534in}{1.476876in}}%
\pgfpathcurveto{\pgfqpoint{1.079358in}{1.471052in}}{\pgfqpoint{1.087258in}{1.467780in}}{\pgfqpoint{1.095494in}{1.467780in}}%
\pgfpathclose%
\pgfusepath{stroke,fill}%
\end{pgfscope}%
\begin{pgfscope}%
\pgfpathrectangle{\pgfqpoint{0.100000in}{0.212622in}}{\pgfqpoint{3.696000in}{3.696000in}}%
\pgfusepath{clip}%
\pgfsetbuttcap%
\pgfsetroundjoin%
\definecolor{currentfill}{rgb}{0.121569,0.466667,0.705882}%
\pgfsetfillcolor{currentfill}%
\pgfsetfillopacity{0.802708}%
\pgfsetlinewidth{1.003750pt}%
\definecolor{currentstroke}{rgb}{0.121569,0.466667,0.705882}%
\pgfsetstrokecolor{currentstroke}%
\pgfsetstrokeopacity{0.802708}%
\pgfsetdash{}{0pt}%
\pgfpathmoveto{\pgfqpoint{2.493340in}{2.763610in}}%
\pgfpathcurveto{\pgfqpoint{2.501576in}{2.763610in}}{\pgfqpoint{2.509476in}{2.766882in}}{\pgfqpoint{2.515300in}{2.772706in}}%
\pgfpathcurveto{\pgfqpoint{2.521124in}{2.778530in}}{\pgfqpoint{2.524397in}{2.786430in}}{\pgfqpoint{2.524397in}{2.794667in}}%
\pgfpathcurveto{\pgfqpoint{2.524397in}{2.802903in}}{\pgfqpoint{2.521124in}{2.810803in}}{\pgfqpoint{2.515300in}{2.816627in}}%
\pgfpathcurveto{\pgfqpoint{2.509476in}{2.822451in}}{\pgfqpoint{2.501576in}{2.825723in}}{\pgfqpoint{2.493340in}{2.825723in}}%
\pgfpathcurveto{\pgfqpoint{2.485104in}{2.825723in}}{\pgfqpoint{2.477204in}{2.822451in}}{\pgfqpoint{2.471380in}{2.816627in}}%
\pgfpathcurveto{\pgfqpoint{2.465556in}{2.810803in}}{\pgfqpoint{2.462284in}{2.802903in}}{\pgfqpoint{2.462284in}{2.794667in}}%
\pgfpathcurveto{\pgfqpoint{2.462284in}{2.786430in}}{\pgfqpoint{2.465556in}{2.778530in}}{\pgfqpoint{2.471380in}{2.772706in}}%
\pgfpathcurveto{\pgfqpoint{2.477204in}{2.766882in}}{\pgfqpoint{2.485104in}{2.763610in}}{\pgfqpoint{2.493340in}{2.763610in}}%
\pgfpathclose%
\pgfusepath{stroke,fill}%
\end{pgfscope}%
\begin{pgfscope}%
\pgfpathrectangle{\pgfqpoint{0.100000in}{0.212622in}}{\pgfqpoint{3.696000in}{3.696000in}}%
\pgfusepath{clip}%
\pgfsetbuttcap%
\pgfsetroundjoin%
\definecolor{currentfill}{rgb}{0.121569,0.466667,0.705882}%
\pgfsetfillcolor{currentfill}%
\pgfsetfillopacity{0.803664}%
\pgfsetlinewidth{1.003750pt}%
\definecolor{currentstroke}{rgb}{0.121569,0.466667,0.705882}%
\pgfsetstrokecolor{currentstroke}%
\pgfsetstrokeopacity{0.803664}%
\pgfsetdash{}{0pt}%
\pgfpathmoveto{\pgfqpoint{1.103521in}{1.468001in}}%
\pgfpathcurveto{\pgfqpoint{1.111758in}{1.468001in}}{\pgfqpoint{1.119658in}{1.471274in}}{\pgfqpoint{1.125482in}{1.477098in}}%
\pgfpathcurveto{\pgfqpoint{1.131306in}{1.482922in}}{\pgfqpoint{1.134578in}{1.490822in}}{\pgfqpoint{1.134578in}{1.499058in}}%
\pgfpathcurveto{\pgfqpoint{1.134578in}{1.507294in}}{\pgfqpoint{1.131306in}{1.515194in}}{\pgfqpoint{1.125482in}{1.521018in}}%
\pgfpathcurveto{\pgfqpoint{1.119658in}{1.526842in}}{\pgfqpoint{1.111758in}{1.530114in}}{\pgfqpoint{1.103521in}{1.530114in}}%
\pgfpathcurveto{\pgfqpoint{1.095285in}{1.530114in}}{\pgfqpoint{1.087385in}{1.526842in}}{\pgfqpoint{1.081561in}{1.521018in}}%
\pgfpathcurveto{\pgfqpoint{1.075737in}{1.515194in}}{\pgfqpoint{1.072465in}{1.507294in}}{\pgfqpoint{1.072465in}{1.499058in}}%
\pgfpathcurveto{\pgfqpoint{1.072465in}{1.490822in}}{\pgfqpoint{1.075737in}{1.482922in}}{\pgfqpoint{1.081561in}{1.477098in}}%
\pgfpathcurveto{\pgfqpoint{1.087385in}{1.471274in}}{\pgfqpoint{1.095285in}{1.468001in}}{\pgfqpoint{1.103521in}{1.468001in}}%
\pgfpathclose%
\pgfusepath{stroke,fill}%
\end{pgfscope}%
\begin{pgfscope}%
\pgfpathrectangle{\pgfqpoint{0.100000in}{0.212622in}}{\pgfqpoint{3.696000in}{3.696000in}}%
\pgfusepath{clip}%
\pgfsetbuttcap%
\pgfsetroundjoin%
\definecolor{currentfill}{rgb}{0.121569,0.466667,0.705882}%
\pgfsetfillcolor{currentfill}%
\pgfsetfillopacity{0.804165}%
\pgfsetlinewidth{1.003750pt}%
\definecolor{currentstroke}{rgb}{0.121569,0.466667,0.705882}%
\pgfsetstrokecolor{currentstroke}%
\pgfsetstrokeopacity{0.804165}%
\pgfsetdash{}{0pt}%
\pgfpathmoveto{\pgfqpoint{2.490953in}{2.752821in}}%
\pgfpathcurveto{\pgfqpoint{2.499189in}{2.752821in}}{\pgfqpoint{2.507089in}{2.756093in}}{\pgfqpoint{2.512913in}{2.761917in}}%
\pgfpathcurveto{\pgfqpoint{2.518737in}{2.767741in}}{\pgfqpoint{2.522010in}{2.775641in}}{\pgfqpoint{2.522010in}{2.783877in}}%
\pgfpathcurveto{\pgfqpoint{2.522010in}{2.792114in}}{\pgfqpoint{2.518737in}{2.800014in}}{\pgfqpoint{2.512913in}{2.805838in}}%
\pgfpathcurveto{\pgfqpoint{2.507089in}{2.811661in}}{\pgfqpoint{2.499189in}{2.814934in}}{\pgfqpoint{2.490953in}{2.814934in}}%
\pgfpathcurveto{\pgfqpoint{2.482717in}{2.814934in}}{\pgfqpoint{2.474817in}{2.811661in}}{\pgfqpoint{2.468993in}{2.805838in}}%
\pgfpathcurveto{\pgfqpoint{2.463169in}{2.800014in}}{\pgfqpoint{2.459897in}{2.792114in}}{\pgfqpoint{2.459897in}{2.783877in}}%
\pgfpathcurveto{\pgfqpoint{2.459897in}{2.775641in}}{\pgfqpoint{2.463169in}{2.767741in}}{\pgfqpoint{2.468993in}{2.761917in}}%
\pgfpathcurveto{\pgfqpoint{2.474817in}{2.756093in}}{\pgfqpoint{2.482717in}{2.752821in}}{\pgfqpoint{2.490953in}{2.752821in}}%
\pgfpathclose%
\pgfusepath{stroke,fill}%
\end{pgfscope}%
\begin{pgfscope}%
\pgfpathrectangle{\pgfqpoint{0.100000in}{0.212622in}}{\pgfqpoint{3.696000in}{3.696000in}}%
\pgfusepath{clip}%
\pgfsetbuttcap%
\pgfsetroundjoin%
\definecolor{currentfill}{rgb}{0.121569,0.466667,0.705882}%
\pgfsetfillcolor{currentfill}%
\pgfsetfillopacity{0.805101}%
\pgfsetlinewidth{1.003750pt}%
\definecolor{currentstroke}{rgb}{0.121569,0.466667,0.705882}%
\pgfsetstrokecolor{currentstroke}%
\pgfsetstrokeopacity{0.805101}%
\pgfsetdash{}{0pt}%
\pgfpathmoveto{\pgfqpoint{1.113837in}{1.466098in}}%
\pgfpathcurveto{\pgfqpoint{1.122074in}{1.466098in}}{\pgfqpoint{1.129974in}{1.469370in}}{\pgfqpoint{1.135798in}{1.475194in}}%
\pgfpathcurveto{\pgfqpoint{1.141622in}{1.481018in}}{\pgfqpoint{1.144894in}{1.488918in}}{\pgfqpoint{1.144894in}{1.497154in}}%
\pgfpathcurveto{\pgfqpoint{1.144894in}{1.505390in}}{\pgfqpoint{1.141622in}{1.513290in}}{\pgfqpoint{1.135798in}{1.519114in}}%
\pgfpathcurveto{\pgfqpoint{1.129974in}{1.524938in}}{\pgfqpoint{1.122074in}{1.528211in}}{\pgfqpoint{1.113837in}{1.528211in}}%
\pgfpathcurveto{\pgfqpoint{1.105601in}{1.528211in}}{\pgfqpoint{1.097701in}{1.524938in}}{\pgfqpoint{1.091877in}{1.519114in}}%
\pgfpathcurveto{\pgfqpoint{1.086053in}{1.513290in}}{\pgfqpoint{1.082781in}{1.505390in}}{\pgfqpoint{1.082781in}{1.497154in}}%
\pgfpathcurveto{\pgfqpoint{1.082781in}{1.488918in}}{\pgfqpoint{1.086053in}{1.481018in}}{\pgfqpoint{1.091877in}{1.475194in}}%
\pgfpathcurveto{\pgfqpoint{1.097701in}{1.469370in}}{\pgfqpoint{1.105601in}{1.466098in}}{\pgfqpoint{1.113837in}{1.466098in}}%
\pgfpathclose%
\pgfusepath{stroke,fill}%
\end{pgfscope}%
\begin{pgfscope}%
\pgfpathrectangle{\pgfqpoint{0.100000in}{0.212622in}}{\pgfqpoint{3.696000in}{3.696000in}}%
\pgfusepath{clip}%
\pgfsetbuttcap%
\pgfsetroundjoin%
\definecolor{currentfill}{rgb}{0.121569,0.466667,0.705882}%
\pgfsetfillcolor{currentfill}%
\pgfsetfillopacity{0.805162}%
\pgfsetlinewidth{1.003750pt}%
\definecolor{currentstroke}{rgb}{0.121569,0.466667,0.705882}%
\pgfsetstrokecolor{currentstroke}%
\pgfsetstrokeopacity{0.805162}%
\pgfsetdash{}{0pt}%
\pgfpathmoveto{\pgfqpoint{2.487572in}{2.747496in}}%
\pgfpathcurveto{\pgfqpoint{2.495808in}{2.747496in}}{\pgfqpoint{2.503708in}{2.750768in}}{\pgfqpoint{2.509532in}{2.756592in}}%
\pgfpathcurveto{\pgfqpoint{2.515356in}{2.762416in}}{\pgfqpoint{2.518629in}{2.770316in}}{\pgfqpoint{2.518629in}{2.778552in}}%
\pgfpathcurveto{\pgfqpoint{2.518629in}{2.786788in}}{\pgfqpoint{2.515356in}{2.794689in}}{\pgfqpoint{2.509532in}{2.800512in}}%
\pgfpathcurveto{\pgfqpoint{2.503708in}{2.806336in}}{\pgfqpoint{2.495808in}{2.809609in}}{\pgfqpoint{2.487572in}{2.809609in}}%
\pgfpathcurveto{\pgfqpoint{2.479336in}{2.809609in}}{\pgfqpoint{2.471436in}{2.806336in}}{\pgfqpoint{2.465612in}{2.800512in}}%
\pgfpathcurveto{\pgfqpoint{2.459788in}{2.794689in}}{\pgfqpoint{2.456516in}{2.786788in}}{\pgfqpoint{2.456516in}{2.778552in}}%
\pgfpathcurveto{\pgfqpoint{2.456516in}{2.770316in}}{\pgfqpoint{2.459788in}{2.762416in}}{\pgfqpoint{2.465612in}{2.756592in}}%
\pgfpathcurveto{\pgfqpoint{2.471436in}{2.750768in}}{\pgfqpoint{2.479336in}{2.747496in}}{\pgfqpoint{2.487572in}{2.747496in}}%
\pgfpathclose%
\pgfusepath{stroke,fill}%
\end{pgfscope}%
\begin{pgfscope}%
\pgfpathrectangle{\pgfqpoint{0.100000in}{0.212622in}}{\pgfqpoint{3.696000in}{3.696000in}}%
\pgfusepath{clip}%
\pgfsetbuttcap%
\pgfsetroundjoin%
\definecolor{currentfill}{rgb}{0.121569,0.466667,0.705882}%
\pgfsetfillcolor{currentfill}%
\pgfsetfillopacity{0.805962}%
\pgfsetlinewidth{1.003750pt}%
\definecolor{currentstroke}{rgb}{0.121569,0.466667,0.705882}%
\pgfsetstrokecolor{currentstroke}%
\pgfsetstrokeopacity{0.805962}%
\pgfsetdash{}{0pt}%
\pgfpathmoveto{\pgfqpoint{3.034709in}{1.908034in}}%
\pgfpathcurveto{\pgfqpoint{3.042945in}{1.908034in}}{\pgfqpoint{3.050845in}{1.911306in}}{\pgfqpoint{3.056669in}{1.917130in}}%
\pgfpathcurveto{\pgfqpoint{3.062493in}{1.922954in}}{\pgfqpoint{3.065765in}{1.930854in}}{\pgfqpoint{3.065765in}{1.939090in}}%
\pgfpathcurveto{\pgfqpoint{3.065765in}{1.947327in}}{\pgfqpoint{3.062493in}{1.955227in}}{\pgfqpoint{3.056669in}{1.961051in}}%
\pgfpathcurveto{\pgfqpoint{3.050845in}{1.966874in}}{\pgfqpoint{3.042945in}{1.970147in}}{\pgfqpoint{3.034709in}{1.970147in}}%
\pgfpathcurveto{\pgfqpoint{3.026473in}{1.970147in}}{\pgfqpoint{3.018573in}{1.966874in}}{\pgfqpoint{3.012749in}{1.961051in}}%
\pgfpathcurveto{\pgfqpoint{3.006925in}{1.955227in}}{\pgfqpoint{3.003652in}{1.947327in}}{\pgfqpoint{3.003652in}{1.939090in}}%
\pgfpathcurveto{\pgfqpoint{3.003652in}{1.930854in}}{\pgfqpoint{3.006925in}{1.922954in}}{\pgfqpoint{3.012749in}{1.917130in}}%
\pgfpathcurveto{\pgfqpoint{3.018573in}{1.911306in}}{\pgfqpoint{3.026473in}{1.908034in}}{\pgfqpoint{3.034709in}{1.908034in}}%
\pgfpathclose%
\pgfusepath{stroke,fill}%
\end{pgfscope}%
\begin{pgfscope}%
\pgfpathrectangle{\pgfqpoint{0.100000in}{0.212622in}}{\pgfqpoint{3.696000in}{3.696000in}}%
\pgfusepath{clip}%
\pgfsetbuttcap%
\pgfsetroundjoin%
\definecolor{currentfill}{rgb}{0.121569,0.466667,0.705882}%
\pgfsetfillcolor{currentfill}%
\pgfsetfillopacity{0.806032}%
\pgfsetlinewidth{1.003750pt}%
\definecolor{currentstroke}{rgb}{0.121569,0.466667,0.705882}%
\pgfsetstrokecolor{currentstroke}%
\pgfsetstrokeopacity{0.806032}%
\pgfsetdash{}{0pt}%
\pgfpathmoveto{\pgfqpoint{2.484531in}{2.742324in}}%
\pgfpathcurveto{\pgfqpoint{2.492767in}{2.742324in}}{\pgfqpoint{2.500667in}{2.745596in}}{\pgfqpoint{2.506491in}{2.751420in}}%
\pgfpathcurveto{\pgfqpoint{2.512315in}{2.757244in}}{\pgfqpoint{2.515588in}{2.765144in}}{\pgfqpoint{2.515588in}{2.773380in}}%
\pgfpathcurveto{\pgfqpoint{2.515588in}{2.781617in}}{\pgfqpoint{2.512315in}{2.789517in}}{\pgfqpoint{2.506491in}{2.795341in}}%
\pgfpathcurveto{\pgfqpoint{2.500667in}{2.801164in}}{\pgfqpoint{2.492767in}{2.804437in}}{\pgfqpoint{2.484531in}{2.804437in}}%
\pgfpathcurveto{\pgfqpoint{2.476295in}{2.804437in}}{\pgfqpoint{2.468395in}{2.801164in}}{\pgfqpoint{2.462571in}{2.795341in}}%
\pgfpathcurveto{\pgfqpoint{2.456747in}{2.789517in}}{\pgfqpoint{2.453475in}{2.781617in}}{\pgfqpoint{2.453475in}{2.773380in}}%
\pgfpathcurveto{\pgfqpoint{2.453475in}{2.765144in}}{\pgfqpoint{2.456747in}{2.757244in}}{\pgfqpoint{2.462571in}{2.751420in}}%
\pgfpathcurveto{\pgfqpoint{2.468395in}{2.745596in}}{\pgfqpoint{2.476295in}{2.742324in}}{\pgfqpoint{2.484531in}{2.742324in}}%
\pgfpathclose%
\pgfusepath{stroke,fill}%
\end{pgfscope}%
\begin{pgfscope}%
\pgfpathrectangle{\pgfqpoint{0.100000in}{0.212622in}}{\pgfqpoint{3.696000in}{3.696000in}}%
\pgfusepath{clip}%
\pgfsetbuttcap%
\pgfsetroundjoin%
\definecolor{currentfill}{rgb}{0.121569,0.466667,0.705882}%
\pgfsetfillcolor{currentfill}%
\pgfsetfillopacity{0.806629}%
\pgfsetlinewidth{1.003750pt}%
\definecolor{currentstroke}{rgb}{0.121569,0.466667,0.705882}%
\pgfsetstrokecolor{currentstroke}%
\pgfsetstrokeopacity{0.806629}%
\pgfsetdash{}{0pt}%
\pgfpathmoveto{\pgfqpoint{2.483490in}{2.737714in}}%
\pgfpathcurveto{\pgfqpoint{2.491726in}{2.737714in}}{\pgfqpoint{2.499626in}{2.740986in}}{\pgfqpoint{2.505450in}{2.746810in}}%
\pgfpathcurveto{\pgfqpoint{2.511274in}{2.752634in}}{\pgfqpoint{2.514546in}{2.760534in}}{\pgfqpoint{2.514546in}{2.768771in}}%
\pgfpathcurveto{\pgfqpoint{2.514546in}{2.777007in}}{\pgfqpoint{2.511274in}{2.784907in}}{\pgfqpoint{2.505450in}{2.790731in}}%
\pgfpathcurveto{\pgfqpoint{2.499626in}{2.796555in}}{\pgfqpoint{2.491726in}{2.799827in}}{\pgfqpoint{2.483490in}{2.799827in}}%
\pgfpathcurveto{\pgfqpoint{2.475254in}{2.799827in}}{\pgfqpoint{2.467354in}{2.796555in}}{\pgfqpoint{2.461530in}{2.790731in}}%
\pgfpathcurveto{\pgfqpoint{2.455706in}{2.784907in}}{\pgfqpoint{2.452433in}{2.777007in}}{\pgfqpoint{2.452433in}{2.768771in}}%
\pgfpathcurveto{\pgfqpoint{2.452433in}{2.760534in}}{\pgfqpoint{2.455706in}{2.752634in}}{\pgfqpoint{2.461530in}{2.746810in}}%
\pgfpathcurveto{\pgfqpoint{2.467354in}{2.740986in}}{\pgfqpoint{2.475254in}{2.737714in}}{\pgfqpoint{2.483490in}{2.737714in}}%
\pgfpathclose%
\pgfusepath{stroke,fill}%
\end{pgfscope}%
\begin{pgfscope}%
\pgfpathrectangle{\pgfqpoint{0.100000in}{0.212622in}}{\pgfqpoint{3.696000in}{3.696000in}}%
\pgfusepath{clip}%
\pgfsetbuttcap%
\pgfsetroundjoin%
\definecolor{currentfill}{rgb}{0.121569,0.466667,0.705882}%
\pgfsetfillcolor{currentfill}%
\pgfsetfillopacity{0.806917}%
\pgfsetlinewidth{1.003750pt}%
\definecolor{currentstroke}{rgb}{0.121569,0.466667,0.705882}%
\pgfsetstrokecolor{currentstroke}%
\pgfsetstrokeopacity{0.806917}%
\pgfsetdash{}{0pt}%
\pgfpathmoveto{\pgfqpoint{1.123749in}{1.462319in}}%
\pgfpathcurveto{\pgfqpoint{1.131985in}{1.462319in}}{\pgfqpoint{1.139885in}{1.465591in}}{\pgfqpoint{1.145709in}{1.471415in}}%
\pgfpathcurveto{\pgfqpoint{1.151533in}{1.477239in}}{\pgfqpoint{1.154806in}{1.485139in}}{\pgfqpoint{1.154806in}{1.493376in}}%
\pgfpathcurveto{\pgfqpoint{1.154806in}{1.501612in}}{\pgfqpoint{1.151533in}{1.509512in}}{\pgfqpoint{1.145709in}{1.515336in}}%
\pgfpathcurveto{\pgfqpoint{1.139885in}{1.521160in}}{\pgfqpoint{1.131985in}{1.524432in}}{\pgfqpoint{1.123749in}{1.524432in}}%
\pgfpathcurveto{\pgfqpoint{1.115513in}{1.524432in}}{\pgfqpoint{1.107613in}{1.521160in}}{\pgfqpoint{1.101789in}{1.515336in}}%
\pgfpathcurveto{\pgfqpoint{1.095965in}{1.509512in}}{\pgfqpoint{1.092693in}{1.501612in}}{\pgfqpoint{1.092693in}{1.493376in}}%
\pgfpathcurveto{\pgfqpoint{1.092693in}{1.485139in}}{\pgfqpoint{1.095965in}{1.477239in}}{\pgfqpoint{1.101789in}{1.471415in}}%
\pgfpathcurveto{\pgfqpoint{1.107613in}{1.465591in}}{\pgfqpoint{1.115513in}{1.462319in}}{\pgfqpoint{1.123749in}{1.462319in}}%
\pgfpathclose%
\pgfusepath{stroke,fill}%
\end{pgfscope}%
\begin{pgfscope}%
\pgfpathrectangle{\pgfqpoint{0.100000in}{0.212622in}}{\pgfqpoint{3.696000in}{3.696000in}}%
\pgfusepath{clip}%
\pgfsetbuttcap%
\pgfsetroundjoin%
\definecolor{currentfill}{rgb}{0.121569,0.466667,0.705882}%
\pgfsetfillcolor{currentfill}%
\pgfsetfillopacity{0.807598}%
\pgfsetlinewidth{1.003750pt}%
\definecolor{currentstroke}{rgb}{0.121569,0.466667,0.705882}%
\pgfsetstrokecolor{currentstroke}%
\pgfsetstrokeopacity{0.807598}%
\pgfsetdash{}{0pt}%
\pgfpathmoveto{\pgfqpoint{2.482160in}{2.728775in}}%
\pgfpathcurveto{\pgfqpoint{2.490396in}{2.728775in}}{\pgfqpoint{2.498296in}{2.732048in}}{\pgfqpoint{2.504120in}{2.737872in}}%
\pgfpathcurveto{\pgfqpoint{2.509944in}{2.743695in}}{\pgfqpoint{2.513216in}{2.751596in}}{\pgfqpoint{2.513216in}{2.759832in}}%
\pgfpathcurveto{\pgfqpoint{2.513216in}{2.768068in}}{\pgfqpoint{2.509944in}{2.775968in}}{\pgfqpoint{2.504120in}{2.781792in}}%
\pgfpathcurveto{\pgfqpoint{2.498296in}{2.787616in}}{\pgfqpoint{2.490396in}{2.790888in}}{\pgfqpoint{2.482160in}{2.790888in}}%
\pgfpathcurveto{\pgfqpoint{2.473924in}{2.790888in}}{\pgfqpoint{2.466024in}{2.787616in}}{\pgfqpoint{2.460200in}{2.781792in}}%
\pgfpathcurveto{\pgfqpoint{2.454376in}{2.775968in}}{\pgfqpoint{2.451103in}{2.768068in}}{\pgfqpoint{2.451103in}{2.759832in}}%
\pgfpathcurveto{\pgfqpoint{2.451103in}{2.751596in}}{\pgfqpoint{2.454376in}{2.743695in}}{\pgfqpoint{2.460200in}{2.737872in}}%
\pgfpathcurveto{\pgfqpoint{2.466024in}{2.732048in}}{\pgfqpoint{2.473924in}{2.728775in}}{\pgfqpoint{2.482160in}{2.728775in}}%
\pgfpathclose%
\pgfusepath{stroke,fill}%
\end{pgfscope}%
\begin{pgfscope}%
\pgfpathrectangle{\pgfqpoint{0.100000in}{0.212622in}}{\pgfqpoint{3.696000in}{3.696000in}}%
\pgfusepath{clip}%
\pgfsetbuttcap%
\pgfsetroundjoin%
\definecolor{currentfill}{rgb}{0.121569,0.466667,0.705882}%
\pgfsetfillcolor{currentfill}%
\pgfsetfillopacity{0.808082}%
\pgfsetlinewidth{1.003750pt}%
\definecolor{currentstroke}{rgb}{0.121569,0.466667,0.705882}%
\pgfsetstrokecolor{currentstroke}%
\pgfsetstrokeopacity{0.808082}%
\pgfsetdash{}{0pt}%
\pgfpathmoveto{\pgfqpoint{2.479605in}{2.725016in}}%
\pgfpathcurveto{\pgfqpoint{2.487841in}{2.725016in}}{\pgfqpoint{2.495741in}{2.728288in}}{\pgfqpoint{2.501565in}{2.734112in}}%
\pgfpathcurveto{\pgfqpoint{2.507389in}{2.739936in}}{\pgfqpoint{2.510662in}{2.747836in}}{\pgfqpoint{2.510662in}{2.756072in}}%
\pgfpathcurveto{\pgfqpoint{2.510662in}{2.764308in}}{\pgfqpoint{2.507389in}{2.772208in}}{\pgfqpoint{2.501565in}{2.778032in}}%
\pgfpathcurveto{\pgfqpoint{2.495741in}{2.783856in}}{\pgfqpoint{2.487841in}{2.787129in}}{\pgfqpoint{2.479605in}{2.787129in}}%
\pgfpathcurveto{\pgfqpoint{2.471369in}{2.787129in}}{\pgfqpoint{2.463469in}{2.783856in}}{\pgfqpoint{2.457645in}{2.778032in}}%
\pgfpathcurveto{\pgfqpoint{2.451821in}{2.772208in}}{\pgfqpoint{2.448549in}{2.764308in}}{\pgfqpoint{2.448549in}{2.756072in}}%
\pgfpathcurveto{\pgfqpoint{2.448549in}{2.747836in}}{\pgfqpoint{2.451821in}{2.739936in}}{\pgfqpoint{2.457645in}{2.734112in}}%
\pgfpathcurveto{\pgfqpoint{2.463469in}{2.728288in}}{\pgfqpoint{2.471369in}{2.725016in}}{\pgfqpoint{2.479605in}{2.725016in}}%
\pgfpathclose%
\pgfusepath{stroke,fill}%
\end{pgfscope}%
\begin{pgfscope}%
\pgfpathrectangle{\pgfqpoint{0.100000in}{0.212622in}}{\pgfqpoint{3.696000in}{3.696000in}}%
\pgfusepath{clip}%
\pgfsetbuttcap%
\pgfsetroundjoin%
\definecolor{currentfill}{rgb}{0.121569,0.466667,0.705882}%
\pgfsetfillcolor{currentfill}%
\pgfsetfillopacity{0.808197}%
\pgfsetlinewidth{1.003750pt}%
\definecolor{currentstroke}{rgb}{0.121569,0.466667,0.705882}%
\pgfsetstrokecolor{currentstroke}%
\pgfsetstrokeopacity{0.808197}%
\pgfsetdash{}{0pt}%
\pgfpathmoveto{\pgfqpoint{1.133899in}{1.455558in}}%
\pgfpathcurveto{\pgfqpoint{1.142136in}{1.455558in}}{\pgfqpoint{1.150036in}{1.458831in}}{\pgfqpoint{1.155860in}{1.464655in}}%
\pgfpathcurveto{\pgfqpoint{1.161684in}{1.470479in}}{\pgfqpoint{1.164956in}{1.478379in}}{\pgfqpoint{1.164956in}{1.486615in}}%
\pgfpathcurveto{\pgfqpoint{1.164956in}{1.494851in}}{\pgfqpoint{1.161684in}{1.502751in}}{\pgfqpoint{1.155860in}{1.508575in}}%
\pgfpathcurveto{\pgfqpoint{1.150036in}{1.514399in}}{\pgfqpoint{1.142136in}{1.517671in}}{\pgfqpoint{1.133899in}{1.517671in}}%
\pgfpathcurveto{\pgfqpoint{1.125663in}{1.517671in}}{\pgfqpoint{1.117763in}{1.514399in}}{\pgfqpoint{1.111939in}{1.508575in}}%
\pgfpathcurveto{\pgfqpoint{1.106115in}{1.502751in}}{\pgfqpoint{1.102843in}{1.494851in}}{\pgfqpoint{1.102843in}{1.486615in}}%
\pgfpathcurveto{\pgfqpoint{1.102843in}{1.478379in}}{\pgfqpoint{1.106115in}{1.470479in}}{\pgfqpoint{1.111939in}{1.464655in}}%
\pgfpathcurveto{\pgfqpoint{1.117763in}{1.458831in}}{\pgfqpoint{1.125663in}{1.455558in}}{\pgfqpoint{1.133899in}{1.455558in}}%
\pgfpathclose%
\pgfusepath{stroke,fill}%
\end{pgfscope}%
\begin{pgfscope}%
\pgfpathrectangle{\pgfqpoint{0.100000in}{0.212622in}}{\pgfqpoint{3.696000in}{3.696000in}}%
\pgfusepath{clip}%
\pgfsetbuttcap%
\pgfsetroundjoin%
\definecolor{currentfill}{rgb}{0.121569,0.466667,0.705882}%
\pgfsetfillcolor{currentfill}%
\pgfsetfillopacity{0.808968}%
\pgfsetlinewidth{1.003750pt}%
\definecolor{currentstroke}{rgb}{0.121569,0.466667,0.705882}%
\pgfsetstrokecolor{currentstroke}%
\pgfsetstrokeopacity{0.808968}%
\pgfsetdash{}{0pt}%
\pgfpathmoveto{\pgfqpoint{2.475308in}{2.717457in}}%
\pgfpathcurveto{\pgfqpoint{2.483544in}{2.717457in}}{\pgfqpoint{2.491444in}{2.720729in}}{\pgfqpoint{2.497268in}{2.726553in}}%
\pgfpathcurveto{\pgfqpoint{2.503092in}{2.732377in}}{\pgfqpoint{2.506365in}{2.740277in}}{\pgfqpoint{2.506365in}{2.748513in}}%
\pgfpathcurveto{\pgfqpoint{2.506365in}{2.756750in}}{\pgfqpoint{2.503092in}{2.764650in}}{\pgfqpoint{2.497268in}{2.770474in}}%
\pgfpathcurveto{\pgfqpoint{2.491444in}{2.776298in}}{\pgfqpoint{2.483544in}{2.779570in}}{\pgfqpoint{2.475308in}{2.779570in}}%
\pgfpathcurveto{\pgfqpoint{2.467072in}{2.779570in}}{\pgfqpoint{2.459172in}{2.776298in}}{\pgfqpoint{2.453348in}{2.770474in}}%
\pgfpathcurveto{\pgfqpoint{2.447524in}{2.764650in}}{\pgfqpoint{2.444252in}{2.756750in}}{\pgfqpoint{2.444252in}{2.748513in}}%
\pgfpathcurveto{\pgfqpoint{2.444252in}{2.740277in}}{\pgfqpoint{2.447524in}{2.732377in}}{\pgfqpoint{2.453348in}{2.726553in}}%
\pgfpathcurveto{\pgfqpoint{2.459172in}{2.720729in}}{\pgfqpoint{2.467072in}{2.717457in}}{\pgfqpoint{2.475308in}{2.717457in}}%
\pgfpathclose%
\pgfusepath{stroke,fill}%
\end{pgfscope}%
\begin{pgfscope}%
\pgfpathrectangle{\pgfqpoint{0.100000in}{0.212622in}}{\pgfqpoint{3.696000in}{3.696000in}}%
\pgfusepath{clip}%
\pgfsetbuttcap%
\pgfsetroundjoin%
\definecolor{currentfill}{rgb}{0.121569,0.466667,0.705882}%
\pgfsetfillcolor{currentfill}%
\pgfsetfillopacity{0.809189}%
\pgfsetlinewidth{1.003750pt}%
\definecolor{currentstroke}{rgb}{0.121569,0.466667,0.705882}%
\pgfsetstrokecolor{currentstroke}%
\pgfsetstrokeopacity{0.809189}%
\pgfsetdash{}{0pt}%
\pgfpathmoveto{\pgfqpoint{1.139602in}{1.453627in}}%
\pgfpathcurveto{\pgfqpoint{1.147838in}{1.453627in}}{\pgfqpoint{1.155738in}{1.456900in}}{\pgfqpoint{1.161562in}{1.462724in}}%
\pgfpathcurveto{\pgfqpoint{1.167386in}{1.468548in}}{\pgfqpoint{1.170658in}{1.476448in}}{\pgfqpoint{1.170658in}{1.484684in}}%
\pgfpathcurveto{\pgfqpoint{1.170658in}{1.492920in}}{\pgfqpoint{1.167386in}{1.500820in}}{\pgfqpoint{1.161562in}{1.506644in}}%
\pgfpathcurveto{\pgfqpoint{1.155738in}{1.512468in}}{\pgfqpoint{1.147838in}{1.515740in}}{\pgfqpoint{1.139602in}{1.515740in}}%
\pgfpathcurveto{\pgfqpoint{1.131366in}{1.515740in}}{\pgfqpoint{1.123466in}{1.512468in}}{\pgfqpoint{1.117642in}{1.506644in}}%
\pgfpathcurveto{\pgfqpoint{1.111818in}{1.500820in}}{\pgfqpoint{1.108545in}{1.492920in}}{\pgfqpoint{1.108545in}{1.484684in}}%
\pgfpathcurveto{\pgfqpoint{1.108545in}{1.476448in}}{\pgfqpoint{1.111818in}{1.468548in}}{\pgfqpoint{1.117642in}{1.462724in}}%
\pgfpathcurveto{\pgfqpoint{1.123466in}{1.456900in}}{\pgfqpoint{1.131366in}{1.453627in}}{\pgfqpoint{1.139602in}{1.453627in}}%
\pgfpathclose%
\pgfusepath{stroke,fill}%
\end{pgfscope}%
\begin{pgfscope}%
\pgfpathrectangle{\pgfqpoint{0.100000in}{0.212622in}}{\pgfqpoint{3.696000in}{3.696000in}}%
\pgfusepath{clip}%
\pgfsetbuttcap%
\pgfsetroundjoin%
\definecolor{currentfill}{rgb}{0.121569,0.466667,0.705882}%
\pgfsetfillcolor{currentfill}%
\pgfsetfillopacity{0.809916}%
\pgfsetlinewidth{1.003750pt}%
\definecolor{currentstroke}{rgb}{0.121569,0.466667,0.705882}%
\pgfsetstrokecolor{currentstroke}%
\pgfsetstrokeopacity{0.809916}%
\pgfsetdash{}{0pt}%
\pgfpathmoveto{\pgfqpoint{2.473197in}{2.711007in}}%
\pgfpathcurveto{\pgfqpoint{2.481433in}{2.711007in}}{\pgfqpoint{2.489333in}{2.714280in}}{\pgfqpoint{2.495157in}{2.720104in}}%
\pgfpathcurveto{\pgfqpoint{2.500981in}{2.725928in}}{\pgfqpoint{2.504253in}{2.733828in}}{\pgfqpoint{2.504253in}{2.742064in}}%
\pgfpathcurveto{\pgfqpoint{2.504253in}{2.750300in}}{\pgfqpoint{2.500981in}{2.758200in}}{\pgfqpoint{2.495157in}{2.764024in}}%
\pgfpathcurveto{\pgfqpoint{2.489333in}{2.769848in}}{\pgfqpoint{2.481433in}{2.773120in}}{\pgfqpoint{2.473197in}{2.773120in}}%
\pgfpathcurveto{\pgfqpoint{2.464960in}{2.773120in}}{\pgfqpoint{2.457060in}{2.769848in}}{\pgfqpoint{2.451236in}{2.764024in}}%
\pgfpathcurveto{\pgfqpoint{2.445412in}{2.758200in}}{\pgfqpoint{2.442140in}{2.750300in}}{\pgfqpoint{2.442140in}{2.742064in}}%
\pgfpathcurveto{\pgfqpoint{2.442140in}{2.733828in}}{\pgfqpoint{2.445412in}{2.725928in}}{\pgfqpoint{2.451236in}{2.720104in}}%
\pgfpathcurveto{\pgfqpoint{2.457060in}{2.714280in}}{\pgfqpoint{2.464960in}{2.711007in}}{\pgfqpoint{2.473197in}{2.711007in}}%
\pgfpathclose%
\pgfusepath{stroke,fill}%
\end{pgfscope}%
\begin{pgfscope}%
\pgfpathrectangle{\pgfqpoint{0.100000in}{0.212622in}}{\pgfqpoint{3.696000in}{3.696000in}}%
\pgfusepath{clip}%
\pgfsetbuttcap%
\pgfsetroundjoin%
\definecolor{currentfill}{rgb}{0.121569,0.466667,0.705882}%
\pgfsetfillcolor{currentfill}%
\pgfsetfillopacity{0.810246}%
\pgfsetlinewidth{1.003750pt}%
\definecolor{currentstroke}{rgb}{0.121569,0.466667,0.705882}%
\pgfsetstrokecolor{currentstroke}%
\pgfsetstrokeopacity{0.810246}%
\pgfsetdash{}{0pt}%
\pgfpathmoveto{\pgfqpoint{3.024821in}{1.884055in}}%
\pgfpathcurveto{\pgfqpoint{3.033057in}{1.884055in}}{\pgfqpoint{3.040957in}{1.887327in}}{\pgfqpoint{3.046781in}{1.893151in}}%
\pgfpathcurveto{\pgfqpoint{3.052605in}{1.898975in}}{\pgfqpoint{3.055878in}{1.906875in}}{\pgfqpoint{3.055878in}{1.915111in}}%
\pgfpathcurveto{\pgfqpoint{3.055878in}{1.923348in}}{\pgfqpoint{3.052605in}{1.931248in}}{\pgfqpoint{3.046781in}{1.937072in}}%
\pgfpathcurveto{\pgfqpoint{3.040957in}{1.942896in}}{\pgfqpoint{3.033057in}{1.946168in}}{\pgfqpoint{3.024821in}{1.946168in}}%
\pgfpathcurveto{\pgfqpoint{3.016585in}{1.946168in}}{\pgfqpoint{3.008685in}{1.942896in}}{\pgfqpoint{3.002861in}{1.937072in}}%
\pgfpathcurveto{\pgfqpoint{2.997037in}{1.931248in}}{\pgfqpoint{2.993765in}{1.923348in}}{\pgfqpoint{2.993765in}{1.915111in}}%
\pgfpathcurveto{\pgfqpoint{2.993765in}{1.906875in}}{\pgfqpoint{2.997037in}{1.898975in}}{\pgfqpoint{3.002861in}{1.893151in}}%
\pgfpathcurveto{\pgfqpoint{3.008685in}{1.887327in}}{\pgfqpoint{3.016585in}{1.884055in}}{\pgfqpoint{3.024821in}{1.884055in}}%
\pgfpathclose%
\pgfusepath{stroke,fill}%
\end{pgfscope}%
\begin{pgfscope}%
\pgfpathrectangle{\pgfqpoint{0.100000in}{0.212622in}}{\pgfqpoint{3.696000in}{3.696000in}}%
\pgfusepath{clip}%
\pgfsetbuttcap%
\pgfsetroundjoin%
\definecolor{currentfill}{rgb}{0.121569,0.466667,0.705882}%
\pgfsetfillcolor{currentfill}%
\pgfsetfillopacity{0.810446}%
\pgfsetlinewidth{1.003750pt}%
\definecolor{currentstroke}{rgb}{0.121569,0.466667,0.705882}%
\pgfsetstrokecolor{currentstroke}%
\pgfsetstrokeopacity{0.810446}%
\pgfsetdash{}{0pt}%
\pgfpathmoveto{\pgfqpoint{1.146078in}{1.453966in}}%
\pgfpathcurveto{\pgfqpoint{1.154315in}{1.453966in}}{\pgfqpoint{1.162215in}{1.457238in}}{\pgfqpoint{1.168039in}{1.463062in}}%
\pgfpathcurveto{\pgfqpoint{1.173863in}{1.468886in}}{\pgfqpoint{1.177135in}{1.476786in}}{\pgfqpoint{1.177135in}{1.485022in}}%
\pgfpathcurveto{\pgfqpoint{1.177135in}{1.493259in}}{\pgfqpoint{1.173863in}{1.501159in}}{\pgfqpoint{1.168039in}{1.506983in}}%
\pgfpathcurveto{\pgfqpoint{1.162215in}{1.512807in}}{\pgfqpoint{1.154315in}{1.516079in}}{\pgfqpoint{1.146078in}{1.516079in}}%
\pgfpathcurveto{\pgfqpoint{1.137842in}{1.516079in}}{\pgfqpoint{1.129942in}{1.512807in}}{\pgfqpoint{1.124118in}{1.506983in}}%
\pgfpathcurveto{\pgfqpoint{1.118294in}{1.501159in}}{\pgfqpoint{1.115022in}{1.493259in}}{\pgfqpoint{1.115022in}{1.485022in}}%
\pgfpathcurveto{\pgfqpoint{1.115022in}{1.476786in}}{\pgfqpoint{1.118294in}{1.468886in}}{\pgfqpoint{1.124118in}{1.463062in}}%
\pgfpathcurveto{\pgfqpoint{1.129942in}{1.457238in}}{\pgfqpoint{1.137842in}{1.453966in}}{\pgfqpoint{1.146078in}{1.453966in}}%
\pgfpathclose%
\pgfusepath{stroke,fill}%
\end{pgfscope}%
\begin{pgfscope}%
\pgfpathrectangle{\pgfqpoint{0.100000in}{0.212622in}}{\pgfqpoint{3.696000in}{3.696000in}}%
\pgfusepath{clip}%
\pgfsetbuttcap%
\pgfsetroundjoin%
\definecolor{currentfill}{rgb}{0.121569,0.466667,0.705882}%
\pgfsetfillcolor{currentfill}%
\pgfsetfillopacity{0.811370}%
\pgfsetlinewidth{1.003750pt}%
\definecolor{currentstroke}{rgb}{0.121569,0.466667,0.705882}%
\pgfsetstrokecolor{currentstroke}%
\pgfsetstrokeopacity{0.811370}%
\pgfsetdash{}{0pt}%
\pgfpathmoveto{\pgfqpoint{1.152866in}{1.451386in}}%
\pgfpathcurveto{\pgfqpoint{1.161102in}{1.451386in}}{\pgfqpoint{1.169002in}{1.454658in}}{\pgfqpoint{1.174826in}{1.460482in}}%
\pgfpathcurveto{\pgfqpoint{1.180650in}{1.466306in}}{\pgfqpoint{1.183922in}{1.474206in}}{\pgfqpoint{1.183922in}{1.482442in}}%
\pgfpathcurveto{\pgfqpoint{1.183922in}{1.490679in}}{\pgfqpoint{1.180650in}{1.498579in}}{\pgfqpoint{1.174826in}{1.504403in}}%
\pgfpathcurveto{\pgfqpoint{1.169002in}{1.510226in}}{\pgfqpoint{1.161102in}{1.513499in}}{\pgfqpoint{1.152866in}{1.513499in}}%
\pgfpathcurveto{\pgfqpoint{1.144629in}{1.513499in}}{\pgfqpoint{1.136729in}{1.510226in}}{\pgfqpoint{1.130905in}{1.504403in}}%
\pgfpathcurveto{\pgfqpoint{1.125082in}{1.498579in}}{\pgfqpoint{1.121809in}{1.490679in}}{\pgfqpoint{1.121809in}{1.482442in}}%
\pgfpathcurveto{\pgfqpoint{1.121809in}{1.474206in}}{\pgfqpoint{1.125082in}{1.466306in}}{\pgfqpoint{1.130905in}{1.460482in}}%
\pgfpathcurveto{\pgfqpoint{1.136729in}{1.454658in}}{\pgfqpoint{1.144629in}{1.451386in}}{\pgfqpoint{1.152866in}{1.451386in}}%
\pgfpathclose%
\pgfusepath{stroke,fill}%
\end{pgfscope}%
\begin{pgfscope}%
\pgfpathrectangle{\pgfqpoint{0.100000in}{0.212622in}}{\pgfqpoint{3.696000in}{3.696000in}}%
\pgfusepath{clip}%
\pgfsetbuttcap%
\pgfsetroundjoin%
\definecolor{currentfill}{rgb}{0.121569,0.466667,0.705882}%
\pgfsetfillcolor{currentfill}%
\pgfsetfillopacity{0.811532}%
\pgfsetlinewidth{1.003750pt}%
\definecolor{currentstroke}{rgb}{0.121569,0.466667,0.705882}%
\pgfsetstrokecolor{currentstroke}%
\pgfsetstrokeopacity{0.811532}%
\pgfsetdash{}{0pt}%
\pgfpathmoveto{\pgfqpoint{2.469834in}{2.698572in}}%
\pgfpathcurveto{\pgfqpoint{2.478071in}{2.698572in}}{\pgfqpoint{2.485971in}{2.701844in}}{\pgfqpoint{2.491795in}{2.707668in}}%
\pgfpathcurveto{\pgfqpoint{2.497619in}{2.713492in}}{\pgfqpoint{2.500891in}{2.721392in}}{\pgfqpoint{2.500891in}{2.729628in}}%
\pgfpathcurveto{\pgfqpoint{2.500891in}{2.737865in}}{\pgfqpoint{2.497619in}{2.745765in}}{\pgfqpoint{2.491795in}{2.751589in}}%
\pgfpathcurveto{\pgfqpoint{2.485971in}{2.757413in}}{\pgfqpoint{2.478071in}{2.760685in}}{\pgfqpoint{2.469834in}{2.760685in}}%
\pgfpathcurveto{\pgfqpoint{2.461598in}{2.760685in}}{\pgfqpoint{2.453698in}{2.757413in}}{\pgfqpoint{2.447874in}{2.751589in}}%
\pgfpathcurveto{\pgfqpoint{2.442050in}{2.745765in}}{\pgfqpoint{2.438778in}{2.737865in}}{\pgfqpoint{2.438778in}{2.729628in}}%
\pgfpathcurveto{\pgfqpoint{2.438778in}{2.721392in}}{\pgfqpoint{2.442050in}{2.713492in}}{\pgfqpoint{2.447874in}{2.707668in}}%
\pgfpathcurveto{\pgfqpoint{2.453698in}{2.701844in}}{\pgfqpoint{2.461598in}{2.698572in}}{\pgfqpoint{2.469834in}{2.698572in}}%
\pgfpathclose%
\pgfusepath{stroke,fill}%
\end{pgfscope}%
\begin{pgfscope}%
\pgfpathrectangle{\pgfqpoint{0.100000in}{0.212622in}}{\pgfqpoint{3.696000in}{3.696000in}}%
\pgfusepath{clip}%
\pgfsetbuttcap%
\pgfsetroundjoin%
\definecolor{currentfill}{rgb}{0.121569,0.466667,0.705882}%
\pgfsetfillcolor{currentfill}%
\pgfsetfillopacity{0.812574}%
\pgfsetlinewidth{1.003750pt}%
\definecolor{currentstroke}{rgb}{0.121569,0.466667,0.705882}%
\pgfsetstrokecolor{currentstroke}%
\pgfsetstrokeopacity{0.812574}%
\pgfsetdash{}{0pt}%
\pgfpathmoveto{\pgfqpoint{2.464718in}{2.691171in}}%
\pgfpathcurveto{\pgfqpoint{2.472954in}{2.691171in}}{\pgfqpoint{2.480854in}{2.694443in}}{\pgfqpoint{2.486678in}{2.700267in}}%
\pgfpathcurveto{\pgfqpoint{2.492502in}{2.706091in}}{\pgfqpoint{2.495774in}{2.713991in}}{\pgfqpoint{2.495774in}{2.722227in}}%
\pgfpathcurveto{\pgfqpoint{2.495774in}{2.730464in}}{\pgfqpoint{2.492502in}{2.738364in}}{\pgfqpoint{2.486678in}{2.744188in}}%
\pgfpathcurveto{\pgfqpoint{2.480854in}{2.750012in}}{\pgfqpoint{2.472954in}{2.753284in}}{\pgfqpoint{2.464718in}{2.753284in}}%
\pgfpathcurveto{\pgfqpoint{2.456481in}{2.753284in}}{\pgfqpoint{2.448581in}{2.750012in}}{\pgfqpoint{2.442757in}{2.744188in}}%
\pgfpathcurveto{\pgfqpoint{2.436933in}{2.738364in}}{\pgfqpoint{2.433661in}{2.730464in}}{\pgfqpoint{2.433661in}{2.722227in}}%
\pgfpathcurveto{\pgfqpoint{2.433661in}{2.713991in}}{\pgfqpoint{2.436933in}{2.706091in}}{\pgfqpoint{2.442757in}{2.700267in}}%
\pgfpathcurveto{\pgfqpoint{2.448581in}{2.694443in}}{\pgfqpoint{2.456481in}{2.691171in}}{\pgfqpoint{2.464718in}{2.691171in}}%
\pgfpathclose%
\pgfusepath{stroke,fill}%
\end{pgfscope}%
\begin{pgfscope}%
\pgfpathrectangle{\pgfqpoint{0.100000in}{0.212622in}}{\pgfqpoint{3.696000in}{3.696000in}}%
\pgfusepath{clip}%
\pgfsetbuttcap%
\pgfsetroundjoin%
\definecolor{currentfill}{rgb}{0.121569,0.466667,0.705882}%
\pgfsetfillcolor{currentfill}%
\pgfsetfillopacity{0.812760}%
\pgfsetlinewidth{1.003750pt}%
\definecolor{currentstroke}{rgb}{0.121569,0.466667,0.705882}%
\pgfsetstrokecolor{currentstroke}%
\pgfsetstrokeopacity{0.812760}%
\pgfsetdash{}{0pt}%
\pgfpathmoveto{\pgfqpoint{1.159787in}{1.448366in}}%
\pgfpathcurveto{\pgfqpoint{1.168023in}{1.448366in}}{\pgfqpoint{1.175923in}{1.451639in}}{\pgfqpoint{1.181747in}{1.457462in}}%
\pgfpathcurveto{\pgfqpoint{1.187571in}{1.463286in}}{\pgfqpoint{1.190843in}{1.471186in}}{\pgfqpoint{1.190843in}{1.479423in}}%
\pgfpathcurveto{\pgfqpoint{1.190843in}{1.487659in}}{\pgfqpoint{1.187571in}{1.495559in}}{\pgfqpoint{1.181747in}{1.501383in}}%
\pgfpathcurveto{\pgfqpoint{1.175923in}{1.507207in}}{\pgfqpoint{1.168023in}{1.510479in}}{\pgfqpoint{1.159787in}{1.510479in}}%
\pgfpathcurveto{\pgfqpoint{1.151551in}{1.510479in}}{\pgfqpoint{1.143651in}{1.507207in}}{\pgfqpoint{1.137827in}{1.501383in}}%
\pgfpathcurveto{\pgfqpoint{1.132003in}{1.495559in}}{\pgfqpoint{1.128730in}{1.487659in}}{\pgfqpoint{1.128730in}{1.479423in}}%
\pgfpathcurveto{\pgfqpoint{1.128730in}{1.471186in}}{\pgfqpoint{1.132003in}{1.463286in}}{\pgfqpoint{1.137827in}{1.457462in}}%
\pgfpathcurveto{\pgfqpoint{1.143651in}{1.451639in}}{\pgfqpoint{1.151551in}{1.448366in}}{\pgfqpoint{1.159787in}{1.448366in}}%
\pgfpathclose%
\pgfusepath{stroke,fill}%
\end{pgfscope}%
\begin{pgfscope}%
\pgfpathrectangle{\pgfqpoint{0.100000in}{0.212622in}}{\pgfqpoint{3.696000in}{3.696000in}}%
\pgfusepath{clip}%
\pgfsetbuttcap%
\pgfsetroundjoin%
\definecolor{currentfill}{rgb}{0.121569,0.466667,0.705882}%
\pgfsetfillcolor{currentfill}%
\pgfsetfillopacity{0.813360}%
\pgfsetlinewidth{1.003750pt}%
\definecolor{currentstroke}{rgb}{0.121569,0.466667,0.705882}%
\pgfsetstrokecolor{currentstroke}%
\pgfsetstrokeopacity{0.813360}%
\pgfsetdash{}{0pt}%
\pgfpathmoveto{\pgfqpoint{3.011187in}{1.865834in}}%
\pgfpathcurveto{\pgfqpoint{3.019424in}{1.865834in}}{\pgfqpoint{3.027324in}{1.869107in}}{\pgfqpoint{3.033148in}{1.874931in}}%
\pgfpathcurveto{\pgfqpoint{3.038972in}{1.880755in}}{\pgfqpoint{3.042244in}{1.888655in}}{\pgfqpoint{3.042244in}{1.896891in}}%
\pgfpathcurveto{\pgfqpoint{3.042244in}{1.905127in}}{\pgfqpoint{3.038972in}{1.913027in}}{\pgfqpoint{3.033148in}{1.918851in}}%
\pgfpathcurveto{\pgfqpoint{3.027324in}{1.924675in}}{\pgfqpoint{3.019424in}{1.927947in}}{\pgfqpoint{3.011187in}{1.927947in}}%
\pgfpathcurveto{\pgfqpoint{3.002951in}{1.927947in}}{\pgfqpoint{2.995051in}{1.924675in}}{\pgfqpoint{2.989227in}{1.918851in}}%
\pgfpathcurveto{\pgfqpoint{2.983403in}{1.913027in}}{\pgfqpoint{2.980131in}{1.905127in}}{\pgfqpoint{2.980131in}{1.896891in}}%
\pgfpathcurveto{\pgfqpoint{2.980131in}{1.888655in}}{\pgfqpoint{2.983403in}{1.880755in}}{\pgfqpoint{2.989227in}{1.874931in}}%
\pgfpathcurveto{\pgfqpoint{2.995051in}{1.869107in}}{\pgfqpoint{3.002951in}{1.865834in}}{\pgfqpoint{3.011187in}{1.865834in}}%
\pgfpathclose%
\pgfusepath{stroke,fill}%
\end{pgfscope}%
\begin{pgfscope}%
\pgfpathrectangle{\pgfqpoint{0.100000in}{0.212622in}}{\pgfqpoint{3.696000in}{3.696000in}}%
\pgfusepath{clip}%
\pgfsetbuttcap%
\pgfsetroundjoin%
\definecolor{currentfill}{rgb}{0.121569,0.466667,0.705882}%
\pgfsetfillcolor{currentfill}%
\pgfsetfillopacity{0.813424}%
\pgfsetlinewidth{1.003750pt}%
\definecolor{currentstroke}{rgb}{0.121569,0.466667,0.705882}%
\pgfsetstrokecolor{currentstroke}%
\pgfsetstrokeopacity{0.813424}%
\pgfsetdash{}{0pt}%
\pgfpathmoveto{\pgfqpoint{2.459693in}{2.683823in}}%
\pgfpathcurveto{\pgfqpoint{2.467929in}{2.683823in}}{\pgfqpoint{2.475829in}{2.687095in}}{\pgfqpoint{2.481653in}{2.692919in}}%
\pgfpathcurveto{\pgfqpoint{2.487477in}{2.698743in}}{\pgfqpoint{2.490749in}{2.706643in}}{\pgfqpoint{2.490749in}{2.714879in}}%
\pgfpathcurveto{\pgfqpoint{2.490749in}{2.723116in}}{\pgfqpoint{2.487477in}{2.731016in}}{\pgfqpoint{2.481653in}{2.736840in}}%
\pgfpathcurveto{\pgfqpoint{2.475829in}{2.742664in}}{\pgfqpoint{2.467929in}{2.745936in}}{\pgfqpoint{2.459693in}{2.745936in}}%
\pgfpathcurveto{\pgfqpoint{2.451456in}{2.745936in}}{\pgfqpoint{2.443556in}{2.742664in}}{\pgfqpoint{2.437732in}{2.736840in}}%
\pgfpathcurveto{\pgfqpoint{2.431908in}{2.731016in}}{\pgfqpoint{2.428636in}{2.723116in}}{\pgfqpoint{2.428636in}{2.714879in}}%
\pgfpathcurveto{\pgfqpoint{2.428636in}{2.706643in}}{\pgfqpoint{2.431908in}{2.698743in}}{\pgfqpoint{2.437732in}{2.692919in}}%
\pgfpathcurveto{\pgfqpoint{2.443556in}{2.687095in}}{\pgfqpoint{2.451456in}{2.683823in}}{\pgfqpoint{2.459693in}{2.683823in}}%
\pgfpathclose%
\pgfusepath{stroke,fill}%
\end{pgfscope}%
\begin{pgfscope}%
\pgfpathrectangle{\pgfqpoint{0.100000in}{0.212622in}}{\pgfqpoint{3.696000in}{3.696000in}}%
\pgfusepath{clip}%
\pgfsetbuttcap%
\pgfsetroundjoin%
\definecolor{currentfill}{rgb}{0.121569,0.466667,0.705882}%
\pgfsetfillcolor{currentfill}%
\pgfsetfillopacity{0.813838}%
\pgfsetlinewidth{1.003750pt}%
\definecolor{currentstroke}{rgb}{0.121569,0.466667,0.705882}%
\pgfsetstrokecolor{currentstroke}%
\pgfsetstrokeopacity{0.813838}%
\pgfsetdash{}{0pt}%
\pgfpathmoveto{\pgfqpoint{1.168060in}{1.444062in}}%
\pgfpathcurveto{\pgfqpoint{1.176296in}{1.444062in}}{\pgfqpoint{1.184196in}{1.447334in}}{\pgfqpoint{1.190020in}{1.453158in}}%
\pgfpathcurveto{\pgfqpoint{1.195844in}{1.458982in}}{\pgfqpoint{1.199117in}{1.466882in}}{\pgfqpoint{1.199117in}{1.475118in}}%
\pgfpathcurveto{\pgfqpoint{1.199117in}{1.483354in}}{\pgfqpoint{1.195844in}{1.491254in}}{\pgfqpoint{1.190020in}{1.497078in}}%
\pgfpathcurveto{\pgfqpoint{1.184196in}{1.502902in}}{\pgfqpoint{1.176296in}{1.506175in}}{\pgfqpoint{1.168060in}{1.506175in}}%
\pgfpathcurveto{\pgfqpoint{1.159824in}{1.506175in}}{\pgfqpoint{1.151924in}{1.502902in}}{\pgfqpoint{1.146100in}{1.497078in}}%
\pgfpathcurveto{\pgfqpoint{1.140276in}{1.491254in}}{\pgfqpoint{1.137004in}{1.483354in}}{\pgfqpoint{1.137004in}{1.475118in}}%
\pgfpathcurveto{\pgfqpoint{1.137004in}{1.466882in}}{\pgfqpoint{1.140276in}{1.458982in}}{\pgfqpoint{1.146100in}{1.453158in}}%
\pgfpathcurveto{\pgfqpoint{1.151924in}{1.447334in}}{\pgfqpoint{1.159824in}{1.444062in}}{\pgfqpoint{1.168060in}{1.444062in}}%
\pgfpathclose%
\pgfusepath{stroke,fill}%
\end{pgfscope}%
\begin{pgfscope}%
\pgfpathrectangle{\pgfqpoint{0.100000in}{0.212622in}}{\pgfqpoint{3.696000in}{3.696000in}}%
\pgfusepath{clip}%
\pgfsetbuttcap%
\pgfsetroundjoin%
\definecolor{currentfill}{rgb}{0.121569,0.466667,0.705882}%
\pgfsetfillcolor{currentfill}%
\pgfsetfillopacity{0.814315}%
\pgfsetlinewidth{1.003750pt}%
\definecolor{currentstroke}{rgb}{0.121569,0.466667,0.705882}%
\pgfsetstrokecolor{currentstroke}%
\pgfsetstrokeopacity{0.814315}%
\pgfsetdash{}{0pt}%
\pgfpathmoveto{\pgfqpoint{2.458018in}{2.677786in}}%
\pgfpathcurveto{\pgfqpoint{2.466254in}{2.677786in}}{\pgfqpoint{2.474154in}{2.681058in}}{\pgfqpoint{2.479978in}{2.686882in}}%
\pgfpathcurveto{\pgfqpoint{2.485802in}{2.692706in}}{\pgfqpoint{2.489075in}{2.700606in}}{\pgfqpoint{2.489075in}{2.708843in}}%
\pgfpathcurveto{\pgfqpoint{2.489075in}{2.717079in}}{\pgfqpoint{2.485802in}{2.724979in}}{\pgfqpoint{2.479978in}{2.730803in}}%
\pgfpathcurveto{\pgfqpoint{2.474154in}{2.736627in}}{\pgfqpoint{2.466254in}{2.739899in}}{\pgfqpoint{2.458018in}{2.739899in}}%
\pgfpathcurveto{\pgfqpoint{2.449782in}{2.739899in}}{\pgfqpoint{2.441882in}{2.736627in}}{\pgfqpoint{2.436058in}{2.730803in}}%
\pgfpathcurveto{\pgfqpoint{2.430234in}{2.724979in}}{\pgfqpoint{2.426962in}{2.717079in}}{\pgfqpoint{2.426962in}{2.708843in}}%
\pgfpathcurveto{\pgfqpoint{2.426962in}{2.700606in}}{\pgfqpoint{2.430234in}{2.692706in}}{\pgfqpoint{2.436058in}{2.686882in}}%
\pgfpathcurveto{\pgfqpoint{2.441882in}{2.681058in}}{\pgfqpoint{2.449782in}{2.677786in}}{\pgfqpoint{2.458018in}{2.677786in}}%
\pgfpathclose%
\pgfusepath{stroke,fill}%
\end{pgfscope}%
\begin{pgfscope}%
\pgfpathrectangle{\pgfqpoint{0.100000in}{0.212622in}}{\pgfqpoint{3.696000in}{3.696000in}}%
\pgfusepath{clip}%
\pgfsetbuttcap%
\pgfsetroundjoin%
\definecolor{currentfill}{rgb}{0.121569,0.466667,0.705882}%
\pgfsetfillcolor{currentfill}%
\pgfsetfillopacity{0.815289}%
\pgfsetlinewidth{1.003750pt}%
\definecolor{currentstroke}{rgb}{0.121569,0.466667,0.705882}%
\pgfsetstrokecolor{currentstroke}%
\pgfsetstrokeopacity{0.815289}%
\pgfsetdash{}{0pt}%
\pgfpathmoveto{\pgfqpoint{1.176981in}{1.438409in}}%
\pgfpathcurveto{\pgfqpoint{1.185217in}{1.438409in}}{\pgfqpoint{1.193117in}{1.441681in}}{\pgfqpoint{1.198941in}{1.447505in}}%
\pgfpathcurveto{\pgfqpoint{1.204765in}{1.453329in}}{\pgfqpoint{1.208037in}{1.461229in}}{\pgfqpoint{1.208037in}{1.469466in}}%
\pgfpathcurveto{\pgfqpoint{1.208037in}{1.477702in}}{\pgfqpoint{1.204765in}{1.485602in}}{\pgfqpoint{1.198941in}{1.491426in}}%
\pgfpathcurveto{\pgfqpoint{1.193117in}{1.497250in}}{\pgfqpoint{1.185217in}{1.500522in}}{\pgfqpoint{1.176981in}{1.500522in}}%
\pgfpathcurveto{\pgfqpoint{1.168745in}{1.500522in}}{\pgfqpoint{1.160845in}{1.497250in}}{\pgfqpoint{1.155021in}{1.491426in}}%
\pgfpathcurveto{\pgfqpoint{1.149197in}{1.485602in}}{\pgfqpoint{1.145924in}{1.477702in}}{\pgfqpoint{1.145924in}{1.469466in}}%
\pgfpathcurveto{\pgfqpoint{1.145924in}{1.461229in}}{\pgfqpoint{1.149197in}{1.453329in}}{\pgfqpoint{1.155021in}{1.447505in}}%
\pgfpathcurveto{\pgfqpoint{1.160845in}{1.441681in}}{\pgfqpoint{1.168745in}{1.438409in}}{\pgfqpoint{1.176981in}{1.438409in}}%
\pgfpathclose%
\pgfusepath{stroke,fill}%
\end{pgfscope}%
\begin{pgfscope}%
\pgfpathrectangle{\pgfqpoint{0.100000in}{0.212622in}}{\pgfqpoint{3.696000in}{3.696000in}}%
\pgfusepath{clip}%
\pgfsetbuttcap%
\pgfsetroundjoin%
\definecolor{currentfill}{rgb}{0.121569,0.466667,0.705882}%
\pgfsetfillcolor{currentfill}%
\pgfsetfillopacity{0.815844}%
\pgfsetlinewidth{1.003750pt}%
\definecolor{currentstroke}{rgb}{0.121569,0.466667,0.705882}%
\pgfsetstrokecolor{currentstroke}%
\pgfsetstrokeopacity{0.815844}%
\pgfsetdash{}{0pt}%
\pgfpathmoveto{\pgfqpoint{2.454064in}{2.666845in}}%
\pgfpathcurveto{\pgfqpoint{2.462300in}{2.666845in}}{\pgfqpoint{2.470200in}{2.670118in}}{\pgfqpoint{2.476024in}{2.675942in}}%
\pgfpathcurveto{\pgfqpoint{2.481848in}{2.681765in}}{\pgfqpoint{2.485121in}{2.689666in}}{\pgfqpoint{2.485121in}{2.697902in}}%
\pgfpathcurveto{\pgfqpoint{2.485121in}{2.706138in}}{\pgfqpoint{2.481848in}{2.714038in}}{\pgfqpoint{2.476024in}{2.719862in}}%
\pgfpathcurveto{\pgfqpoint{2.470200in}{2.725686in}}{\pgfqpoint{2.462300in}{2.728958in}}{\pgfqpoint{2.454064in}{2.728958in}}%
\pgfpathcurveto{\pgfqpoint{2.445828in}{2.728958in}}{\pgfqpoint{2.437928in}{2.725686in}}{\pgfqpoint{2.432104in}{2.719862in}}%
\pgfpathcurveto{\pgfqpoint{2.426280in}{2.714038in}}{\pgfqpoint{2.423008in}{2.706138in}}{\pgfqpoint{2.423008in}{2.697902in}}%
\pgfpathcurveto{\pgfqpoint{2.423008in}{2.689666in}}{\pgfqpoint{2.426280in}{2.681765in}}{\pgfqpoint{2.432104in}{2.675942in}}%
\pgfpathcurveto{\pgfqpoint{2.437928in}{2.670118in}}{\pgfqpoint{2.445828in}{2.666845in}}{\pgfqpoint{2.454064in}{2.666845in}}%
\pgfpathclose%
\pgfusepath{stroke,fill}%
\end{pgfscope}%
\begin{pgfscope}%
\pgfpathrectangle{\pgfqpoint{0.100000in}{0.212622in}}{\pgfqpoint{3.696000in}{3.696000in}}%
\pgfusepath{clip}%
\pgfsetbuttcap%
\pgfsetroundjoin%
\definecolor{currentfill}{rgb}{0.121569,0.466667,0.705882}%
\pgfsetfillcolor{currentfill}%
\pgfsetfillopacity{0.816287}%
\pgfsetlinewidth{1.003750pt}%
\definecolor{currentstroke}{rgb}{0.121569,0.466667,0.705882}%
\pgfsetstrokecolor{currentstroke}%
\pgfsetstrokeopacity{0.816287}%
\pgfsetdash{}{0pt}%
\pgfpathmoveto{\pgfqpoint{2.998274in}{1.846558in}}%
\pgfpathcurveto{\pgfqpoint{3.006510in}{1.846558in}}{\pgfqpoint{3.014410in}{1.849831in}}{\pgfqpoint{3.020234in}{1.855655in}}%
\pgfpathcurveto{\pgfqpoint{3.026058in}{1.861479in}}{\pgfqpoint{3.029331in}{1.869379in}}{\pgfqpoint{3.029331in}{1.877615in}}%
\pgfpathcurveto{\pgfqpoint{3.029331in}{1.885851in}}{\pgfqpoint{3.026058in}{1.893751in}}{\pgfqpoint{3.020234in}{1.899575in}}%
\pgfpathcurveto{\pgfqpoint{3.014410in}{1.905399in}}{\pgfqpoint{3.006510in}{1.908671in}}{\pgfqpoint{2.998274in}{1.908671in}}%
\pgfpathcurveto{\pgfqpoint{2.990038in}{1.908671in}}{\pgfqpoint{2.982138in}{1.905399in}}{\pgfqpoint{2.976314in}{1.899575in}}%
\pgfpathcurveto{\pgfqpoint{2.970490in}{1.893751in}}{\pgfqpoint{2.967218in}{1.885851in}}{\pgfqpoint{2.967218in}{1.877615in}}%
\pgfpathcurveto{\pgfqpoint{2.967218in}{1.869379in}}{\pgfqpoint{2.970490in}{1.861479in}}{\pgfqpoint{2.976314in}{1.855655in}}%
\pgfpathcurveto{\pgfqpoint{2.982138in}{1.849831in}}{\pgfqpoint{2.990038in}{1.846558in}}{\pgfqpoint{2.998274in}{1.846558in}}%
\pgfpathclose%
\pgfusepath{stroke,fill}%
\end{pgfscope}%
\begin{pgfscope}%
\pgfpathrectangle{\pgfqpoint{0.100000in}{0.212622in}}{\pgfqpoint{3.696000in}{3.696000in}}%
\pgfusepath{clip}%
\pgfsetbuttcap%
\pgfsetroundjoin%
\definecolor{currentfill}{rgb}{0.121569,0.466667,0.705882}%
\pgfsetfillcolor{currentfill}%
\pgfsetfillopacity{0.816643}%
\pgfsetlinewidth{1.003750pt}%
\definecolor{currentstroke}{rgb}{0.121569,0.466667,0.705882}%
\pgfsetstrokecolor{currentstroke}%
\pgfsetstrokeopacity{0.816643}%
\pgfsetdash{}{0pt}%
\pgfpathmoveto{\pgfqpoint{2.449460in}{2.660619in}}%
\pgfpathcurveto{\pgfqpoint{2.457697in}{2.660619in}}{\pgfqpoint{2.465597in}{2.663891in}}{\pgfqpoint{2.471421in}{2.669715in}}%
\pgfpathcurveto{\pgfqpoint{2.477245in}{2.675539in}}{\pgfqpoint{2.480517in}{2.683439in}}{\pgfqpoint{2.480517in}{2.691675in}}%
\pgfpathcurveto{\pgfqpoint{2.480517in}{2.699912in}}{\pgfqpoint{2.477245in}{2.707812in}}{\pgfqpoint{2.471421in}{2.713636in}}%
\pgfpathcurveto{\pgfqpoint{2.465597in}{2.719459in}}{\pgfqpoint{2.457697in}{2.722732in}}{\pgfqpoint{2.449460in}{2.722732in}}%
\pgfpathcurveto{\pgfqpoint{2.441224in}{2.722732in}}{\pgfqpoint{2.433324in}{2.719459in}}{\pgfqpoint{2.427500in}{2.713636in}}%
\pgfpathcurveto{\pgfqpoint{2.421676in}{2.707812in}}{\pgfqpoint{2.418404in}{2.699912in}}{\pgfqpoint{2.418404in}{2.691675in}}%
\pgfpathcurveto{\pgfqpoint{2.418404in}{2.683439in}}{\pgfqpoint{2.421676in}{2.675539in}}{\pgfqpoint{2.427500in}{2.669715in}}%
\pgfpathcurveto{\pgfqpoint{2.433324in}{2.663891in}}{\pgfqpoint{2.441224in}{2.660619in}}{\pgfqpoint{2.449460in}{2.660619in}}%
\pgfpathclose%
\pgfusepath{stroke,fill}%
\end{pgfscope}%
\begin{pgfscope}%
\pgfpathrectangle{\pgfqpoint{0.100000in}{0.212622in}}{\pgfqpoint{3.696000in}{3.696000in}}%
\pgfusepath{clip}%
\pgfsetbuttcap%
\pgfsetroundjoin%
\definecolor{currentfill}{rgb}{0.121569,0.466667,0.705882}%
\pgfsetfillcolor{currentfill}%
\pgfsetfillopacity{0.817408}%
\pgfsetlinewidth{1.003750pt}%
\definecolor{currentstroke}{rgb}{0.121569,0.466667,0.705882}%
\pgfsetstrokecolor{currentstroke}%
\pgfsetstrokeopacity{0.817408}%
\pgfsetdash{}{0pt}%
\pgfpathmoveto{\pgfqpoint{2.445290in}{2.654289in}}%
\pgfpathcurveto{\pgfqpoint{2.453526in}{2.654289in}}{\pgfqpoint{2.461426in}{2.657561in}}{\pgfqpoint{2.467250in}{2.663385in}}%
\pgfpathcurveto{\pgfqpoint{2.473074in}{2.669209in}}{\pgfqpoint{2.476346in}{2.677109in}}{\pgfqpoint{2.476346in}{2.685346in}}%
\pgfpathcurveto{\pgfqpoint{2.476346in}{2.693582in}}{\pgfqpoint{2.473074in}{2.701482in}}{\pgfqpoint{2.467250in}{2.707306in}}%
\pgfpathcurveto{\pgfqpoint{2.461426in}{2.713130in}}{\pgfqpoint{2.453526in}{2.716402in}}{\pgfqpoint{2.445290in}{2.716402in}}%
\pgfpathcurveto{\pgfqpoint{2.437053in}{2.716402in}}{\pgfqpoint{2.429153in}{2.713130in}}{\pgfqpoint{2.423329in}{2.707306in}}%
\pgfpathcurveto{\pgfqpoint{2.417505in}{2.701482in}}{\pgfqpoint{2.414233in}{2.693582in}}{\pgfqpoint{2.414233in}{2.685346in}}%
\pgfpathcurveto{\pgfqpoint{2.414233in}{2.677109in}}{\pgfqpoint{2.417505in}{2.669209in}}{\pgfqpoint{2.423329in}{2.663385in}}%
\pgfpathcurveto{\pgfqpoint{2.429153in}{2.657561in}}{\pgfqpoint{2.437053in}{2.654289in}}{\pgfqpoint{2.445290in}{2.654289in}}%
\pgfpathclose%
\pgfusepath{stroke,fill}%
\end{pgfscope}%
\begin{pgfscope}%
\pgfpathrectangle{\pgfqpoint{0.100000in}{0.212622in}}{\pgfqpoint{3.696000in}{3.696000in}}%
\pgfusepath{clip}%
\pgfsetbuttcap%
\pgfsetroundjoin%
\definecolor{currentfill}{rgb}{0.121569,0.466667,0.705882}%
\pgfsetfillcolor{currentfill}%
\pgfsetfillopacity{0.817434}%
\pgfsetlinewidth{1.003750pt}%
\definecolor{currentstroke}{rgb}{0.121569,0.466667,0.705882}%
\pgfsetstrokecolor{currentstroke}%
\pgfsetstrokeopacity{0.817434}%
\pgfsetdash{}{0pt}%
\pgfpathmoveto{\pgfqpoint{1.185887in}{1.434261in}}%
\pgfpathcurveto{\pgfqpoint{1.194123in}{1.434261in}}{\pgfqpoint{1.202023in}{1.437533in}}{\pgfqpoint{1.207847in}{1.443357in}}%
\pgfpathcurveto{\pgfqpoint{1.213671in}{1.449181in}}{\pgfqpoint{1.216943in}{1.457081in}}{\pgfqpoint{1.216943in}{1.465318in}}%
\pgfpathcurveto{\pgfqpoint{1.216943in}{1.473554in}}{\pgfqpoint{1.213671in}{1.481454in}}{\pgfqpoint{1.207847in}{1.487278in}}%
\pgfpathcurveto{\pgfqpoint{1.202023in}{1.493102in}}{\pgfqpoint{1.194123in}{1.496374in}}{\pgfqpoint{1.185887in}{1.496374in}}%
\pgfpathcurveto{\pgfqpoint{1.177651in}{1.496374in}}{\pgfqpoint{1.169751in}{1.493102in}}{\pgfqpoint{1.163927in}{1.487278in}}%
\pgfpathcurveto{\pgfqpoint{1.158103in}{1.481454in}}{\pgfqpoint{1.154830in}{1.473554in}}{\pgfqpoint{1.154830in}{1.465318in}}%
\pgfpathcurveto{\pgfqpoint{1.154830in}{1.457081in}}{\pgfqpoint{1.158103in}{1.449181in}}{\pgfqpoint{1.163927in}{1.443357in}}%
\pgfpathcurveto{\pgfqpoint{1.169751in}{1.437533in}}{\pgfqpoint{1.177651in}{1.434261in}}{\pgfqpoint{1.185887in}{1.434261in}}%
\pgfpathclose%
\pgfusepath{stroke,fill}%
\end{pgfscope}%
\begin{pgfscope}%
\pgfpathrectangle{\pgfqpoint{0.100000in}{0.212622in}}{\pgfqpoint{3.696000in}{3.696000in}}%
\pgfusepath{clip}%
\pgfsetbuttcap%
\pgfsetroundjoin%
\definecolor{currentfill}{rgb}{0.121569,0.466667,0.705882}%
\pgfsetfillcolor{currentfill}%
\pgfsetfillopacity{0.818087}%
\pgfsetlinewidth{1.003750pt}%
\definecolor{currentstroke}{rgb}{0.121569,0.466667,0.705882}%
\pgfsetstrokecolor{currentstroke}%
\pgfsetstrokeopacity{0.818087}%
\pgfsetdash{}{0pt}%
\pgfpathmoveto{\pgfqpoint{2.444116in}{2.649321in}}%
\pgfpathcurveto{\pgfqpoint{2.452352in}{2.649321in}}{\pgfqpoint{2.460252in}{2.652593in}}{\pgfqpoint{2.466076in}{2.658417in}}%
\pgfpathcurveto{\pgfqpoint{2.471900in}{2.664241in}}{\pgfqpoint{2.475173in}{2.672141in}}{\pgfqpoint{2.475173in}{2.680378in}}%
\pgfpathcurveto{\pgfqpoint{2.475173in}{2.688614in}}{\pgfqpoint{2.471900in}{2.696514in}}{\pgfqpoint{2.466076in}{2.702338in}}%
\pgfpathcurveto{\pgfqpoint{2.460252in}{2.708162in}}{\pgfqpoint{2.452352in}{2.711434in}}{\pgfqpoint{2.444116in}{2.711434in}}%
\pgfpathcurveto{\pgfqpoint{2.435880in}{2.711434in}}{\pgfqpoint{2.427980in}{2.708162in}}{\pgfqpoint{2.422156in}{2.702338in}}%
\pgfpathcurveto{\pgfqpoint{2.416332in}{2.696514in}}{\pgfqpoint{2.413060in}{2.688614in}}{\pgfqpoint{2.413060in}{2.680378in}}%
\pgfpathcurveto{\pgfqpoint{2.413060in}{2.672141in}}{\pgfqpoint{2.416332in}{2.664241in}}{\pgfqpoint{2.422156in}{2.658417in}}%
\pgfpathcurveto{\pgfqpoint{2.427980in}{2.652593in}}{\pgfqpoint{2.435880in}{2.649321in}}{\pgfqpoint{2.444116in}{2.649321in}}%
\pgfpathclose%
\pgfusepath{stroke,fill}%
\end{pgfscope}%
\begin{pgfscope}%
\pgfpathrectangle{\pgfqpoint{0.100000in}{0.212622in}}{\pgfqpoint{3.696000in}{3.696000in}}%
\pgfusepath{clip}%
\pgfsetbuttcap%
\pgfsetroundjoin%
\definecolor{currentfill}{rgb}{0.121569,0.466667,0.705882}%
\pgfsetfillcolor{currentfill}%
\pgfsetfillopacity{0.818664}%
\pgfsetlinewidth{1.003750pt}%
\definecolor{currentstroke}{rgb}{0.121569,0.466667,0.705882}%
\pgfsetstrokecolor{currentstroke}%
\pgfsetstrokeopacity{0.818664}%
\pgfsetdash{}{0pt}%
\pgfpathmoveto{\pgfqpoint{1.190298in}{1.430836in}}%
\pgfpathcurveto{\pgfqpoint{1.198534in}{1.430836in}}{\pgfqpoint{1.206434in}{1.434108in}}{\pgfqpoint{1.212258in}{1.439932in}}%
\pgfpathcurveto{\pgfqpoint{1.218082in}{1.445756in}}{\pgfqpoint{1.221355in}{1.453656in}}{\pgfqpoint{1.221355in}{1.461892in}}%
\pgfpathcurveto{\pgfqpoint{1.221355in}{1.470128in}}{\pgfqpoint{1.218082in}{1.478028in}}{\pgfqpoint{1.212258in}{1.483852in}}%
\pgfpathcurveto{\pgfqpoint{1.206434in}{1.489676in}}{\pgfqpoint{1.198534in}{1.492949in}}{\pgfqpoint{1.190298in}{1.492949in}}%
\pgfpathcurveto{\pgfqpoint{1.182062in}{1.492949in}}{\pgfqpoint{1.174162in}{1.489676in}}{\pgfqpoint{1.168338in}{1.483852in}}%
\pgfpathcurveto{\pgfqpoint{1.162514in}{1.478028in}}{\pgfqpoint{1.159242in}{1.470128in}}{\pgfqpoint{1.159242in}{1.461892in}}%
\pgfpathcurveto{\pgfqpoint{1.159242in}{1.453656in}}{\pgfqpoint{1.162514in}{1.445756in}}{\pgfqpoint{1.168338in}{1.439932in}}%
\pgfpathcurveto{\pgfqpoint{1.174162in}{1.434108in}}{\pgfqpoint{1.182062in}{1.430836in}}{\pgfqpoint{1.190298in}{1.430836in}}%
\pgfpathclose%
\pgfusepath{stroke,fill}%
\end{pgfscope}%
\begin{pgfscope}%
\pgfpathrectangle{\pgfqpoint{0.100000in}{0.212622in}}{\pgfqpoint{3.696000in}{3.696000in}}%
\pgfusepath{clip}%
\pgfsetbuttcap%
\pgfsetroundjoin%
\definecolor{currentfill}{rgb}{0.121569,0.466667,0.705882}%
\pgfsetfillcolor{currentfill}%
\pgfsetfillopacity{0.819335}%
\pgfsetlinewidth{1.003750pt}%
\definecolor{currentstroke}{rgb}{0.121569,0.466667,0.705882}%
\pgfsetstrokecolor{currentstroke}%
\pgfsetstrokeopacity{0.819335}%
\pgfsetdash{}{0pt}%
\pgfpathmoveto{\pgfqpoint{2.441155in}{2.640682in}}%
\pgfpathcurveto{\pgfqpoint{2.449391in}{2.640682in}}{\pgfqpoint{2.457291in}{2.643954in}}{\pgfqpoint{2.463115in}{2.649778in}}%
\pgfpathcurveto{\pgfqpoint{2.468939in}{2.655602in}}{\pgfqpoint{2.472211in}{2.663502in}}{\pgfqpoint{2.472211in}{2.671739in}}%
\pgfpathcurveto{\pgfqpoint{2.472211in}{2.679975in}}{\pgfqpoint{2.468939in}{2.687875in}}{\pgfqpoint{2.463115in}{2.693699in}}%
\pgfpathcurveto{\pgfqpoint{2.457291in}{2.699523in}}{\pgfqpoint{2.449391in}{2.702795in}}{\pgfqpoint{2.441155in}{2.702795in}}%
\pgfpathcurveto{\pgfqpoint{2.432919in}{2.702795in}}{\pgfqpoint{2.425019in}{2.699523in}}{\pgfqpoint{2.419195in}{2.693699in}}%
\pgfpathcurveto{\pgfqpoint{2.413371in}{2.687875in}}{\pgfqpoint{2.410098in}{2.679975in}}{\pgfqpoint{2.410098in}{2.671739in}}%
\pgfpathcurveto{\pgfqpoint{2.410098in}{2.663502in}}{\pgfqpoint{2.413371in}{2.655602in}}{\pgfqpoint{2.419195in}{2.649778in}}%
\pgfpathcurveto{\pgfqpoint{2.425019in}{2.643954in}}{\pgfqpoint{2.432919in}{2.640682in}}{\pgfqpoint{2.441155in}{2.640682in}}%
\pgfpathclose%
\pgfusepath{stroke,fill}%
\end{pgfscope}%
\begin{pgfscope}%
\pgfpathrectangle{\pgfqpoint{0.100000in}{0.212622in}}{\pgfqpoint{3.696000in}{3.696000in}}%
\pgfusepath{clip}%
\pgfsetbuttcap%
\pgfsetroundjoin%
\definecolor{currentfill}{rgb}{0.121569,0.466667,0.705882}%
\pgfsetfillcolor{currentfill}%
\pgfsetfillopacity{0.819793}%
\pgfsetlinewidth{1.003750pt}%
\definecolor{currentstroke}{rgb}{0.121569,0.466667,0.705882}%
\pgfsetstrokecolor{currentstroke}%
\pgfsetstrokeopacity{0.819793}%
\pgfsetdash{}{0pt}%
\pgfpathmoveto{\pgfqpoint{2.989797in}{1.828214in}}%
\pgfpathcurveto{\pgfqpoint{2.998033in}{1.828214in}}{\pgfqpoint{3.005933in}{1.831487in}}{\pgfqpoint{3.011757in}{1.837310in}}%
\pgfpathcurveto{\pgfqpoint{3.017581in}{1.843134in}}{\pgfqpoint{3.020853in}{1.851034in}}{\pgfqpoint{3.020853in}{1.859271in}}%
\pgfpathcurveto{\pgfqpoint{3.020853in}{1.867507in}}{\pgfqpoint{3.017581in}{1.875407in}}{\pgfqpoint{3.011757in}{1.881231in}}%
\pgfpathcurveto{\pgfqpoint{3.005933in}{1.887055in}}{\pgfqpoint{2.998033in}{1.890327in}}{\pgfqpoint{2.989797in}{1.890327in}}%
\pgfpathcurveto{\pgfqpoint{2.981560in}{1.890327in}}{\pgfqpoint{2.973660in}{1.887055in}}{\pgfqpoint{2.967836in}{1.881231in}}%
\pgfpathcurveto{\pgfqpoint{2.962012in}{1.875407in}}{\pgfqpoint{2.958740in}{1.867507in}}{\pgfqpoint{2.958740in}{1.859271in}}%
\pgfpathcurveto{\pgfqpoint{2.958740in}{1.851034in}}{\pgfqpoint{2.962012in}{1.843134in}}{\pgfqpoint{2.967836in}{1.837310in}}%
\pgfpathcurveto{\pgfqpoint{2.973660in}{1.831487in}}{\pgfqpoint{2.981560in}{1.828214in}}{\pgfqpoint{2.989797in}{1.828214in}}%
\pgfpathclose%
\pgfusepath{stroke,fill}%
\end{pgfscope}%
\begin{pgfscope}%
\pgfpathrectangle{\pgfqpoint{0.100000in}{0.212622in}}{\pgfqpoint{3.696000in}{3.696000in}}%
\pgfusepath{clip}%
\pgfsetbuttcap%
\pgfsetroundjoin%
\definecolor{currentfill}{rgb}{0.121569,0.466667,0.705882}%
\pgfsetfillcolor{currentfill}%
\pgfsetfillopacity{0.820017}%
\pgfsetlinewidth{1.003750pt}%
\definecolor{currentstroke}{rgb}{0.121569,0.466667,0.705882}%
\pgfsetstrokecolor{currentstroke}%
\pgfsetstrokeopacity{0.820017}%
\pgfsetdash{}{0pt}%
\pgfpathmoveto{\pgfqpoint{2.437888in}{2.636185in}}%
\pgfpathcurveto{\pgfqpoint{2.446124in}{2.636185in}}{\pgfqpoint{2.454024in}{2.639458in}}{\pgfqpoint{2.459848in}{2.645282in}}%
\pgfpathcurveto{\pgfqpoint{2.465672in}{2.651106in}}{\pgfqpoint{2.468944in}{2.659006in}}{\pgfqpoint{2.468944in}{2.667242in}}%
\pgfpathcurveto{\pgfqpoint{2.468944in}{2.675478in}}{\pgfqpoint{2.465672in}{2.683378in}}{\pgfqpoint{2.459848in}{2.689202in}}%
\pgfpathcurveto{\pgfqpoint{2.454024in}{2.695026in}}{\pgfqpoint{2.446124in}{2.698298in}}{\pgfqpoint{2.437888in}{2.698298in}}%
\pgfpathcurveto{\pgfqpoint{2.429651in}{2.698298in}}{\pgfqpoint{2.421751in}{2.695026in}}{\pgfqpoint{2.415927in}{2.689202in}}%
\pgfpathcurveto{\pgfqpoint{2.410104in}{2.683378in}}{\pgfqpoint{2.406831in}{2.675478in}}{\pgfqpoint{2.406831in}{2.667242in}}%
\pgfpathcurveto{\pgfqpoint{2.406831in}{2.659006in}}{\pgfqpoint{2.410104in}{2.651106in}}{\pgfqpoint{2.415927in}{2.645282in}}%
\pgfpathcurveto{\pgfqpoint{2.421751in}{2.639458in}}{\pgfqpoint{2.429651in}{2.636185in}}{\pgfqpoint{2.437888in}{2.636185in}}%
\pgfpathclose%
\pgfusepath{stroke,fill}%
\end{pgfscope}%
\begin{pgfscope}%
\pgfpathrectangle{\pgfqpoint{0.100000in}{0.212622in}}{\pgfqpoint{3.696000in}{3.696000in}}%
\pgfusepath{clip}%
\pgfsetbuttcap%
\pgfsetroundjoin%
\definecolor{currentfill}{rgb}{0.121569,0.466667,0.705882}%
\pgfsetfillcolor{currentfill}%
\pgfsetfillopacity{0.820154}%
\pgfsetlinewidth{1.003750pt}%
\definecolor{currentstroke}{rgb}{0.121569,0.466667,0.705882}%
\pgfsetstrokecolor{currentstroke}%
\pgfsetstrokeopacity{0.820154}%
\pgfsetdash{}{0pt}%
\pgfpathmoveto{\pgfqpoint{1.195314in}{1.427820in}}%
\pgfpathcurveto{\pgfqpoint{1.203550in}{1.427820in}}{\pgfqpoint{1.211451in}{1.431092in}}{\pgfqpoint{1.217274in}{1.436916in}}%
\pgfpathcurveto{\pgfqpoint{1.223098in}{1.442740in}}{\pgfqpoint{1.226371in}{1.450640in}}{\pgfqpoint{1.226371in}{1.458877in}}%
\pgfpathcurveto{\pgfqpoint{1.226371in}{1.467113in}}{\pgfqpoint{1.223098in}{1.475013in}}{\pgfqpoint{1.217274in}{1.480837in}}%
\pgfpathcurveto{\pgfqpoint{1.211451in}{1.486661in}}{\pgfqpoint{1.203550in}{1.489933in}}{\pgfqpoint{1.195314in}{1.489933in}}%
\pgfpathcurveto{\pgfqpoint{1.187078in}{1.489933in}}{\pgfqpoint{1.179178in}{1.486661in}}{\pgfqpoint{1.173354in}{1.480837in}}%
\pgfpathcurveto{\pgfqpoint{1.167530in}{1.475013in}}{\pgfqpoint{1.164258in}{1.467113in}}{\pgfqpoint{1.164258in}{1.458877in}}%
\pgfpathcurveto{\pgfqpoint{1.164258in}{1.450640in}}{\pgfqpoint{1.167530in}{1.442740in}}{\pgfqpoint{1.173354in}{1.436916in}}%
\pgfpathcurveto{\pgfqpoint{1.179178in}{1.431092in}}{\pgfqpoint{1.187078in}{1.427820in}}{\pgfqpoint{1.195314in}{1.427820in}}%
\pgfpathclose%
\pgfusepath{stroke,fill}%
\end{pgfscope}%
\begin{pgfscope}%
\pgfpathrectangle{\pgfqpoint{0.100000in}{0.212622in}}{\pgfqpoint{3.696000in}{3.696000in}}%
\pgfusepath{clip}%
\pgfsetbuttcap%
\pgfsetroundjoin%
\definecolor{currentfill}{rgb}{0.121569,0.466667,0.705882}%
\pgfsetfillcolor{currentfill}%
\pgfsetfillopacity{0.820571}%
\pgfsetlinewidth{1.003750pt}%
\definecolor{currentstroke}{rgb}{0.121569,0.466667,0.705882}%
\pgfsetstrokecolor{currentstroke}%
\pgfsetstrokeopacity{0.820571}%
\pgfsetdash{}{0pt}%
\pgfpathmoveto{\pgfqpoint{2.435058in}{2.631374in}}%
\pgfpathcurveto{\pgfqpoint{2.443294in}{2.631374in}}{\pgfqpoint{2.451194in}{2.634647in}}{\pgfqpoint{2.457018in}{2.640470in}}%
\pgfpathcurveto{\pgfqpoint{2.462842in}{2.646294in}}{\pgfqpoint{2.466114in}{2.654194in}}{\pgfqpoint{2.466114in}{2.662431in}}%
\pgfpathcurveto{\pgfqpoint{2.466114in}{2.670667in}}{\pgfqpoint{2.462842in}{2.678567in}}{\pgfqpoint{2.457018in}{2.684391in}}%
\pgfpathcurveto{\pgfqpoint{2.451194in}{2.690215in}}{\pgfqpoint{2.443294in}{2.693487in}}{\pgfqpoint{2.435058in}{2.693487in}}%
\pgfpathcurveto{\pgfqpoint{2.426821in}{2.693487in}}{\pgfqpoint{2.418921in}{2.690215in}}{\pgfqpoint{2.413097in}{2.684391in}}%
\pgfpathcurveto{\pgfqpoint{2.407273in}{2.678567in}}{\pgfqpoint{2.404001in}{2.670667in}}{\pgfqpoint{2.404001in}{2.662431in}}%
\pgfpathcurveto{\pgfqpoint{2.404001in}{2.654194in}}{\pgfqpoint{2.407273in}{2.646294in}}{\pgfqpoint{2.413097in}{2.640470in}}%
\pgfpathcurveto{\pgfqpoint{2.418921in}{2.634647in}}{\pgfqpoint{2.426821in}{2.631374in}}{\pgfqpoint{2.435058in}{2.631374in}}%
\pgfpathclose%
\pgfusepath{stroke,fill}%
\end{pgfscope}%
\begin{pgfscope}%
\pgfpathrectangle{\pgfqpoint{0.100000in}{0.212622in}}{\pgfqpoint{3.696000in}{3.696000in}}%
\pgfusepath{clip}%
\pgfsetbuttcap%
\pgfsetroundjoin%
\definecolor{currentfill}{rgb}{0.121569,0.466667,0.705882}%
\pgfsetfillcolor{currentfill}%
\pgfsetfillopacity{0.820665}%
\pgfsetlinewidth{1.003750pt}%
\definecolor{currentstroke}{rgb}{0.121569,0.466667,0.705882}%
\pgfsetstrokecolor{currentstroke}%
\pgfsetstrokeopacity{0.820665}%
\pgfsetdash{}{0pt}%
\pgfpathmoveto{\pgfqpoint{1.198433in}{1.425763in}}%
\pgfpathcurveto{\pgfqpoint{1.206669in}{1.425763in}}{\pgfqpoint{1.214569in}{1.429036in}}{\pgfqpoint{1.220393in}{1.434860in}}%
\pgfpathcurveto{\pgfqpoint{1.226217in}{1.440684in}}{\pgfqpoint{1.229489in}{1.448584in}}{\pgfqpoint{1.229489in}{1.456820in}}%
\pgfpathcurveto{\pgfqpoint{1.229489in}{1.465056in}}{\pgfqpoint{1.226217in}{1.472956in}}{\pgfqpoint{1.220393in}{1.478780in}}%
\pgfpathcurveto{\pgfqpoint{1.214569in}{1.484604in}}{\pgfqpoint{1.206669in}{1.487876in}}{\pgfqpoint{1.198433in}{1.487876in}}%
\pgfpathcurveto{\pgfqpoint{1.190197in}{1.487876in}}{\pgfqpoint{1.182297in}{1.484604in}}{\pgfqpoint{1.176473in}{1.478780in}}%
\pgfpathcurveto{\pgfqpoint{1.170649in}{1.472956in}}{\pgfqpoint{1.167376in}{1.465056in}}{\pgfqpoint{1.167376in}{1.456820in}}%
\pgfpathcurveto{\pgfqpoint{1.167376in}{1.448584in}}{\pgfqpoint{1.170649in}{1.440684in}}{\pgfqpoint{1.176473in}{1.434860in}}%
\pgfpathcurveto{\pgfqpoint{1.182297in}{1.429036in}}{\pgfqpoint{1.190197in}{1.425763in}}{\pgfqpoint{1.198433in}{1.425763in}}%
\pgfpathclose%
\pgfusepath{stroke,fill}%
\end{pgfscope}%
\begin{pgfscope}%
\pgfpathrectangle{\pgfqpoint{0.100000in}{0.212622in}}{\pgfqpoint{3.696000in}{3.696000in}}%
\pgfusepath{clip}%
\pgfsetbuttcap%
\pgfsetroundjoin%
\definecolor{currentfill}{rgb}{0.121569,0.466667,0.705882}%
\pgfsetfillcolor{currentfill}%
\pgfsetfillopacity{0.820977}%
\pgfsetlinewidth{1.003750pt}%
\definecolor{currentstroke}{rgb}{0.121569,0.466667,0.705882}%
\pgfsetstrokecolor{currentstroke}%
\pgfsetstrokeopacity{0.820977}%
\pgfsetdash{}{0pt}%
\pgfpathmoveto{\pgfqpoint{1.200019in}{1.424393in}}%
\pgfpathcurveto{\pgfqpoint{1.208256in}{1.424393in}}{\pgfqpoint{1.216156in}{1.427666in}}{\pgfqpoint{1.221980in}{1.433489in}}%
\pgfpathcurveto{\pgfqpoint{1.227804in}{1.439313in}}{\pgfqpoint{1.231076in}{1.447213in}}{\pgfqpoint{1.231076in}{1.455450in}}%
\pgfpathcurveto{\pgfqpoint{1.231076in}{1.463686in}}{\pgfqpoint{1.227804in}{1.471586in}}{\pgfqpoint{1.221980in}{1.477410in}}%
\pgfpathcurveto{\pgfqpoint{1.216156in}{1.483234in}}{\pgfqpoint{1.208256in}{1.486506in}}{\pgfqpoint{1.200019in}{1.486506in}}%
\pgfpathcurveto{\pgfqpoint{1.191783in}{1.486506in}}{\pgfqpoint{1.183883in}{1.483234in}}{\pgfqpoint{1.178059in}{1.477410in}}%
\pgfpathcurveto{\pgfqpoint{1.172235in}{1.471586in}}{\pgfqpoint{1.168963in}{1.463686in}}{\pgfqpoint{1.168963in}{1.455450in}}%
\pgfpathcurveto{\pgfqpoint{1.168963in}{1.447213in}}{\pgfqpoint{1.172235in}{1.439313in}}{\pgfqpoint{1.178059in}{1.433489in}}%
\pgfpathcurveto{\pgfqpoint{1.183883in}{1.427666in}}{\pgfqpoint{1.191783in}{1.424393in}}{\pgfqpoint{1.200019in}{1.424393in}}%
\pgfpathclose%
\pgfusepath{stroke,fill}%
\end{pgfscope}%
\begin{pgfscope}%
\pgfpathrectangle{\pgfqpoint{0.100000in}{0.212622in}}{\pgfqpoint{3.696000in}{3.696000in}}%
\pgfusepath{clip}%
\pgfsetbuttcap%
\pgfsetroundjoin%
\definecolor{currentfill}{rgb}{0.121569,0.466667,0.705882}%
\pgfsetfillcolor{currentfill}%
\pgfsetfillopacity{0.821127}%
\pgfsetlinewidth{1.003750pt}%
\definecolor{currentstroke}{rgb}{0.121569,0.466667,0.705882}%
\pgfsetstrokecolor{currentstroke}%
\pgfsetstrokeopacity{0.821127}%
\pgfsetdash{}{0pt}%
\pgfpathmoveto{\pgfqpoint{2.433800in}{2.627454in}}%
\pgfpathcurveto{\pgfqpoint{2.442037in}{2.627454in}}{\pgfqpoint{2.449937in}{2.630727in}}{\pgfqpoint{2.455761in}{2.636551in}}%
\pgfpathcurveto{\pgfqpoint{2.461585in}{2.642375in}}{\pgfqpoint{2.464857in}{2.650275in}}{\pgfqpoint{2.464857in}{2.658511in}}%
\pgfpathcurveto{\pgfqpoint{2.464857in}{2.666747in}}{\pgfqpoint{2.461585in}{2.674647in}}{\pgfqpoint{2.455761in}{2.680471in}}%
\pgfpathcurveto{\pgfqpoint{2.449937in}{2.686295in}}{\pgfqpoint{2.442037in}{2.689567in}}{\pgfqpoint{2.433800in}{2.689567in}}%
\pgfpathcurveto{\pgfqpoint{2.425564in}{2.689567in}}{\pgfqpoint{2.417664in}{2.686295in}}{\pgfqpoint{2.411840in}{2.680471in}}%
\pgfpathcurveto{\pgfqpoint{2.406016in}{2.674647in}}{\pgfqpoint{2.402744in}{2.666747in}}{\pgfqpoint{2.402744in}{2.658511in}}%
\pgfpathcurveto{\pgfqpoint{2.402744in}{2.650275in}}{\pgfqpoint{2.406016in}{2.642375in}}{\pgfqpoint{2.411840in}{2.636551in}}%
\pgfpathcurveto{\pgfqpoint{2.417664in}{2.630727in}}{\pgfqpoint{2.425564in}{2.627454in}}{\pgfqpoint{2.433800in}{2.627454in}}%
\pgfpathclose%
\pgfusepath{stroke,fill}%
\end{pgfscope}%
\begin{pgfscope}%
\pgfpathrectangle{\pgfqpoint{0.100000in}{0.212622in}}{\pgfqpoint{3.696000in}{3.696000in}}%
\pgfusepath{clip}%
\pgfsetbuttcap%
\pgfsetroundjoin%
\definecolor{currentfill}{rgb}{0.121569,0.466667,0.705882}%
\pgfsetfillcolor{currentfill}%
\pgfsetfillopacity{0.821177}%
\pgfsetlinewidth{1.003750pt}%
\definecolor{currentstroke}{rgb}{0.121569,0.466667,0.705882}%
\pgfsetstrokecolor{currentstroke}%
\pgfsetstrokeopacity{0.821177}%
\pgfsetdash{}{0pt}%
\pgfpathmoveto{\pgfqpoint{1.200958in}{1.423968in}}%
\pgfpathcurveto{\pgfqpoint{1.209194in}{1.423968in}}{\pgfqpoint{1.217094in}{1.427240in}}{\pgfqpoint{1.222918in}{1.433064in}}%
\pgfpathcurveto{\pgfqpoint{1.228742in}{1.438888in}}{\pgfqpoint{1.232015in}{1.446788in}}{\pgfqpoint{1.232015in}{1.455024in}}%
\pgfpathcurveto{\pgfqpoint{1.232015in}{1.463261in}}{\pgfqpoint{1.228742in}{1.471161in}}{\pgfqpoint{1.222918in}{1.476985in}}%
\pgfpathcurveto{\pgfqpoint{1.217094in}{1.482809in}}{\pgfqpoint{1.209194in}{1.486081in}}{\pgfqpoint{1.200958in}{1.486081in}}%
\pgfpathcurveto{\pgfqpoint{1.192722in}{1.486081in}}{\pgfqpoint{1.184822in}{1.482809in}}{\pgfqpoint{1.178998in}{1.476985in}}%
\pgfpathcurveto{\pgfqpoint{1.173174in}{1.471161in}}{\pgfqpoint{1.169902in}{1.463261in}}{\pgfqpoint{1.169902in}{1.455024in}}%
\pgfpathcurveto{\pgfqpoint{1.169902in}{1.446788in}}{\pgfqpoint{1.173174in}{1.438888in}}{\pgfqpoint{1.178998in}{1.433064in}}%
\pgfpathcurveto{\pgfqpoint{1.184822in}{1.427240in}}{\pgfqpoint{1.192722in}{1.423968in}}{\pgfqpoint{1.200958in}{1.423968in}}%
\pgfpathclose%
\pgfusepath{stroke,fill}%
\end{pgfscope}%
\begin{pgfscope}%
\pgfpathrectangle{\pgfqpoint{0.100000in}{0.212622in}}{\pgfqpoint{3.696000in}{3.696000in}}%
\pgfusepath{clip}%
\pgfsetbuttcap%
\pgfsetroundjoin%
\definecolor{currentfill}{rgb}{0.121569,0.466667,0.705882}%
\pgfsetfillcolor{currentfill}%
\pgfsetfillopacity{0.821280}%
\pgfsetlinewidth{1.003750pt}%
\definecolor{currentstroke}{rgb}{0.121569,0.466667,0.705882}%
\pgfsetstrokecolor{currentstroke}%
\pgfsetstrokeopacity{0.821280}%
\pgfsetdash{}{0pt}%
\pgfpathmoveto{\pgfqpoint{1.201420in}{1.423543in}}%
\pgfpathcurveto{\pgfqpoint{1.209656in}{1.423543in}}{\pgfqpoint{1.217556in}{1.426816in}}{\pgfqpoint{1.223380in}{1.432640in}}%
\pgfpathcurveto{\pgfqpoint{1.229204in}{1.438464in}}{\pgfqpoint{1.232476in}{1.446364in}}{\pgfqpoint{1.232476in}{1.454600in}}%
\pgfpathcurveto{\pgfqpoint{1.232476in}{1.462836in}}{\pgfqpoint{1.229204in}{1.470736in}}{\pgfqpoint{1.223380in}{1.476560in}}%
\pgfpathcurveto{\pgfqpoint{1.217556in}{1.482384in}}{\pgfqpoint{1.209656in}{1.485656in}}{\pgfqpoint{1.201420in}{1.485656in}}%
\pgfpathcurveto{\pgfqpoint{1.193183in}{1.485656in}}{\pgfqpoint{1.185283in}{1.482384in}}{\pgfqpoint{1.179459in}{1.476560in}}%
\pgfpathcurveto{\pgfqpoint{1.173636in}{1.470736in}}{\pgfqpoint{1.170363in}{1.462836in}}{\pgfqpoint{1.170363in}{1.454600in}}%
\pgfpathcurveto{\pgfqpoint{1.170363in}{1.446364in}}{\pgfqpoint{1.173636in}{1.438464in}}{\pgfqpoint{1.179459in}{1.432640in}}%
\pgfpathcurveto{\pgfqpoint{1.185283in}{1.426816in}}{\pgfqpoint{1.193183in}{1.423543in}}{\pgfqpoint{1.201420in}{1.423543in}}%
\pgfpathclose%
\pgfusepath{stroke,fill}%
\end{pgfscope}%
\begin{pgfscope}%
\pgfpathrectangle{\pgfqpoint{0.100000in}{0.212622in}}{\pgfqpoint{3.696000in}{3.696000in}}%
\pgfusepath{clip}%
\pgfsetbuttcap%
\pgfsetroundjoin%
\definecolor{currentfill}{rgb}{0.121569,0.466667,0.705882}%
\pgfsetfillcolor{currentfill}%
\pgfsetfillopacity{0.821544}%
\pgfsetlinewidth{1.003750pt}%
\definecolor{currentstroke}{rgb}{0.121569,0.466667,0.705882}%
\pgfsetstrokecolor{currentstroke}%
\pgfsetstrokeopacity{0.821544}%
\pgfsetdash{}{0pt}%
\pgfpathmoveto{\pgfqpoint{1.202255in}{1.422742in}}%
\pgfpathcurveto{\pgfqpoint{1.210491in}{1.422742in}}{\pgfqpoint{1.218391in}{1.426014in}}{\pgfqpoint{1.224215in}{1.431838in}}%
\pgfpathcurveto{\pgfqpoint{1.230039in}{1.437662in}}{\pgfqpoint{1.233312in}{1.445562in}}{\pgfqpoint{1.233312in}{1.453798in}}%
\pgfpathcurveto{\pgfqpoint{1.233312in}{1.462035in}}{\pgfqpoint{1.230039in}{1.469935in}}{\pgfqpoint{1.224215in}{1.475759in}}%
\pgfpathcurveto{\pgfqpoint{1.218391in}{1.481583in}}{\pgfqpoint{1.210491in}{1.484855in}}{\pgfqpoint{1.202255in}{1.484855in}}%
\pgfpathcurveto{\pgfqpoint{1.194019in}{1.484855in}}{\pgfqpoint{1.186119in}{1.481583in}}{\pgfqpoint{1.180295in}{1.475759in}}%
\pgfpathcurveto{\pgfqpoint{1.174471in}{1.469935in}}{\pgfqpoint{1.171199in}{1.462035in}}{\pgfqpoint{1.171199in}{1.453798in}}%
\pgfpathcurveto{\pgfqpoint{1.171199in}{1.445562in}}{\pgfqpoint{1.174471in}{1.437662in}}{\pgfqpoint{1.180295in}{1.431838in}}%
\pgfpathcurveto{\pgfqpoint{1.186119in}{1.426014in}}{\pgfqpoint{1.194019in}{1.422742in}}{\pgfqpoint{1.202255in}{1.422742in}}%
\pgfpathclose%
\pgfusepath{stroke,fill}%
\end{pgfscope}%
\begin{pgfscope}%
\pgfpathrectangle{\pgfqpoint{0.100000in}{0.212622in}}{\pgfqpoint{3.696000in}{3.696000in}}%
\pgfusepath{clip}%
\pgfsetbuttcap%
\pgfsetroundjoin%
\definecolor{currentfill}{rgb}{0.121569,0.466667,0.705882}%
\pgfsetfillcolor{currentfill}%
\pgfsetfillopacity{0.821806}%
\pgfsetlinewidth{1.003750pt}%
\definecolor{currentstroke}{rgb}{0.121569,0.466667,0.705882}%
\pgfsetstrokecolor{currentstroke}%
\pgfsetstrokeopacity{0.821806}%
\pgfsetdash{}{0pt}%
\pgfpathmoveto{\pgfqpoint{1.203265in}{1.421479in}}%
\pgfpathcurveto{\pgfqpoint{1.211501in}{1.421479in}}{\pgfqpoint{1.219401in}{1.424752in}}{\pgfqpoint{1.225225in}{1.430576in}}%
\pgfpathcurveto{\pgfqpoint{1.231049in}{1.436400in}}{\pgfqpoint{1.234321in}{1.444300in}}{\pgfqpoint{1.234321in}{1.452536in}}%
\pgfpathcurveto{\pgfqpoint{1.234321in}{1.460772in}}{\pgfqpoint{1.231049in}{1.468672in}}{\pgfqpoint{1.225225in}{1.474496in}}%
\pgfpathcurveto{\pgfqpoint{1.219401in}{1.480320in}}{\pgfqpoint{1.211501in}{1.483592in}}{\pgfqpoint{1.203265in}{1.483592in}}%
\pgfpathcurveto{\pgfqpoint{1.195029in}{1.483592in}}{\pgfqpoint{1.187128in}{1.480320in}}{\pgfqpoint{1.181305in}{1.474496in}}%
\pgfpathcurveto{\pgfqpoint{1.175481in}{1.468672in}}{\pgfqpoint{1.172208in}{1.460772in}}{\pgfqpoint{1.172208in}{1.452536in}}%
\pgfpathcurveto{\pgfqpoint{1.172208in}{1.444300in}}{\pgfqpoint{1.175481in}{1.436400in}}{\pgfqpoint{1.181305in}{1.430576in}}%
\pgfpathcurveto{\pgfqpoint{1.187128in}{1.424752in}}{\pgfqpoint{1.195029in}{1.421479in}}{\pgfqpoint{1.203265in}{1.421479in}}%
\pgfpathclose%
\pgfusepath{stroke,fill}%
\end{pgfscope}%
\begin{pgfscope}%
\pgfpathrectangle{\pgfqpoint{0.100000in}{0.212622in}}{\pgfqpoint{3.696000in}{3.696000in}}%
\pgfusepath{clip}%
\pgfsetbuttcap%
\pgfsetroundjoin%
\definecolor{currentfill}{rgb}{0.121569,0.466667,0.705882}%
\pgfsetfillcolor{currentfill}%
\pgfsetfillopacity{0.822132}%
\pgfsetlinewidth{1.003750pt}%
\definecolor{currentstroke}{rgb}{0.121569,0.466667,0.705882}%
\pgfsetstrokecolor{currentstroke}%
\pgfsetstrokeopacity{0.822132}%
\pgfsetdash{}{0pt}%
\pgfpathmoveto{\pgfqpoint{1.204916in}{1.419253in}}%
\pgfpathcurveto{\pgfqpoint{1.213152in}{1.419253in}}{\pgfqpoint{1.221052in}{1.422525in}}{\pgfqpoint{1.226876in}{1.428349in}}%
\pgfpathcurveto{\pgfqpoint{1.232700in}{1.434173in}}{\pgfqpoint{1.235972in}{1.442073in}}{\pgfqpoint{1.235972in}{1.450309in}}%
\pgfpathcurveto{\pgfqpoint{1.235972in}{1.458545in}}{\pgfqpoint{1.232700in}{1.466445in}}{\pgfqpoint{1.226876in}{1.472269in}}%
\pgfpathcurveto{\pgfqpoint{1.221052in}{1.478093in}}{\pgfqpoint{1.213152in}{1.481366in}}{\pgfqpoint{1.204916in}{1.481366in}}%
\pgfpathcurveto{\pgfqpoint{1.196680in}{1.481366in}}{\pgfqpoint{1.188780in}{1.478093in}}{\pgfqpoint{1.182956in}{1.472269in}}%
\pgfpathcurveto{\pgfqpoint{1.177132in}{1.466445in}}{\pgfqpoint{1.173859in}{1.458545in}}{\pgfqpoint{1.173859in}{1.450309in}}%
\pgfpathcurveto{\pgfqpoint{1.173859in}{1.442073in}}{\pgfqpoint{1.177132in}{1.434173in}}{\pgfqpoint{1.182956in}{1.428349in}}%
\pgfpathcurveto{\pgfqpoint{1.188780in}{1.422525in}}{\pgfqpoint{1.196680in}{1.419253in}}{\pgfqpoint{1.204916in}{1.419253in}}%
\pgfpathclose%
\pgfusepath{stroke,fill}%
\end{pgfscope}%
\begin{pgfscope}%
\pgfpathrectangle{\pgfqpoint{0.100000in}{0.212622in}}{\pgfqpoint{3.696000in}{3.696000in}}%
\pgfusepath{clip}%
\pgfsetbuttcap%
\pgfsetroundjoin%
\definecolor{currentfill}{rgb}{0.121569,0.466667,0.705882}%
\pgfsetfillcolor{currentfill}%
\pgfsetfillopacity{0.822153}%
\pgfsetlinewidth{1.003750pt}%
\definecolor{currentstroke}{rgb}{0.121569,0.466667,0.705882}%
\pgfsetstrokecolor{currentstroke}%
\pgfsetstrokeopacity{0.822153}%
\pgfsetdash{}{0pt}%
\pgfpathmoveto{\pgfqpoint{2.432167in}{2.620184in}}%
\pgfpathcurveto{\pgfqpoint{2.440403in}{2.620184in}}{\pgfqpoint{2.448303in}{2.623456in}}{\pgfqpoint{2.454127in}{2.629280in}}%
\pgfpathcurveto{\pgfqpoint{2.459951in}{2.635104in}}{\pgfqpoint{2.463224in}{2.643004in}}{\pgfqpoint{2.463224in}{2.651240in}}%
\pgfpathcurveto{\pgfqpoint{2.463224in}{2.659476in}}{\pgfqpoint{2.459951in}{2.667376in}}{\pgfqpoint{2.454127in}{2.673200in}}%
\pgfpathcurveto{\pgfqpoint{2.448303in}{2.679024in}}{\pgfqpoint{2.440403in}{2.682297in}}{\pgfqpoint{2.432167in}{2.682297in}}%
\pgfpathcurveto{\pgfqpoint{2.423931in}{2.682297in}}{\pgfqpoint{2.416031in}{2.679024in}}{\pgfqpoint{2.410207in}{2.673200in}}%
\pgfpathcurveto{\pgfqpoint{2.404383in}{2.667376in}}{\pgfqpoint{2.401111in}{2.659476in}}{\pgfqpoint{2.401111in}{2.651240in}}%
\pgfpathcurveto{\pgfqpoint{2.401111in}{2.643004in}}{\pgfqpoint{2.404383in}{2.635104in}}{\pgfqpoint{2.410207in}{2.629280in}}%
\pgfpathcurveto{\pgfqpoint{2.416031in}{2.623456in}}{\pgfqpoint{2.423931in}{2.620184in}}{\pgfqpoint{2.432167in}{2.620184in}}%
\pgfpathclose%
\pgfusepath{stroke,fill}%
\end{pgfscope}%
\begin{pgfscope}%
\pgfpathrectangle{\pgfqpoint{0.100000in}{0.212622in}}{\pgfqpoint{3.696000in}{3.696000in}}%
\pgfusepath{clip}%
\pgfsetbuttcap%
\pgfsetroundjoin%
\definecolor{currentfill}{rgb}{0.121569,0.466667,0.705882}%
\pgfsetfillcolor{currentfill}%
\pgfsetfillopacity{0.822652}%
\pgfsetlinewidth{1.003750pt}%
\definecolor{currentstroke}{rgb}{0.121569,0.466667,0.705882}%
\pgfsetstrokecolor{currentstroke}%
\pgfsetstrokeopacity{0.822652}%
\pgfsetdash{}{0pt}%
\pgfpathmoveto{\pgfqpoint{2.429910in}{2.617004in}}%
\pgfpathcurveto{\pgfqpoint{2.438146in}{2.617004in}}{\pgfqpoint{2.446046in}{2.620276in}}{\pgfqpoint{2.451870in}{2.626100in}}%
\pgfpathcurveto{\pgfqpoint{2.457694in}{2.631924in}}{\pgfqpoint{2.460966in}{2.639824in}}{\pgfqpoint{2.460966in}{2.648061in}}%
\pgfpathcurveto{\pgfqpoint{2.460966in}{2.656297in}}{\pgfqpoint{2.457694in}{2.664197in}}{\pgfqpoint{2.451870in}{2.670021in}}%
\pgfpathcurveto{\pgfqpoint{2.446046in}{2.675845in}}{\pgfqpoint{2.438146in}{2.679117in}}{\pgfqpoint{2.429910in}{2.679117in}}%
\pgfpathcurveto{\pgfqpoint{2.421673in}{2.679117in}}{\pgfqpoint{2.413773in}{2.675845in}}{\pgfqpoint{2.407949in}{2.670021in}}%
\pgfpathcurveto{\pgfqpoint{2.402126in}{2.664197in}}{\pgfqpoint{2.398853in}{2.656297in}}{\pgfqpoint{2.398853in}{2.648061in}}%
\pgfpathcurveto{\pgfqpoint{2.398853in}{2.639824in}}{\pgfqpoint{2.402126in}{2.631924in}}{\pgfqpoint{2.407949in}{2.626100in}}%
\pgfpathcurveto{\pgfqpoint{2.413773in}{2.620276in}}{\pgfqpoint{2.421673in}{2.617004in}}{\pgfqpoint{2.429910in}{2.617004in}}%
\pgfpathclose%
\pgfusepath{stroke,fill}%
\end{pgfscope}%
\begin{pgfscope}%
\pgfpathrectangle{\pgfqpoint{0.100000in}{0.212622in}}{\pgfqpoint{3.696000in}{3.696000in}}%
\pgfusepath{clip}%
\pgfsetbuttcap%
\pgfsetroundjoin%
\definecolor{currentfill}{rgb}{0.121569,0.466667,0.705882}%
\pgfsetfillcolor{currentfill}%
\pgfsetfillopacity{0.822958}%
\pgfsetlinewidth{1.003750pt}%
\definecolor{currentstroke}{rgb}{0.121569,0.466667,0.705882}%
\pgfsetstrokecolor{currentstroke}%
\pgfsetstrokeopacity{0.822958}%
\pgfsetdash{}{0pt}%
\pgfpathmoveto{\pgfqpoint{2.986000in}{1.809488in}}%
\pgfpathcurveto{\pgfqpoint{2.994236in}{1.809488in}}{\pgfqpoint{3.002137in}{1.812760in}}{\pgfqpoint{3.007960in}{1.818584in}}%
\pgfpathcurveto{\pgfqpoint{3.013784in}{1.824408in}}{\pgfqpoint{3.017057in}{1.832308in}}{\pgfqpoint{3.017057in}{1.840544in}}%
\pgfpathcurveto{\pgfqpoint{3.017057in}{1.848781in}}{\pgfqpoint{3.013784in}{1.856681in}}{\pgfqpoint{3.007960in}{1.862505in}}%
\pgfpathcurveto{\pgfqpoint{3.002137in}{1.868329in}}{\pgfqpoint{2.994236in}{1.871601in}}{\pgfqpoint{2.986000in}{1.871601in}}%
\pgfpathcurveto{\pgfqpoint{2.977764in}{1.871601in}}{\pgfqpoint{2.969864in}{1.868329in}}{\pgfqpoint{2.964040in}{1.862505in}}%
\pgfpathcurveto{\pgfqpoint{2.958216in}{1.856681in}}{\pgfqpoint{2.954944in}{1.848781in}}{\pgfqpoint{2.954944in}{1.840544in}}%
\pgfpathcurveto{\pgfqpoint{2.954944in}{1.832308in}}{\pgfqpoint{2.958216in}{1.824408in}}{\pgfqpoint{2.964040in}{1.818584in}}%
\pgfpathcurveto{\pgfqpoint{2.969864in}{1.812760in}}{\pgfqpoint{2.977764in}{1.809488in}}{\pgfqpoint{2.986000in}{1.809488in}}%
\pgfpathclose%
\pgfusepath{stroke,fill}%
\end{pgfscope}%
\begin{pgfscope}%
\pgfpathrectangle{\pgfqpoint{0.100000in}{0.212622in}}{\pgfqpoint{3.696000in}{3.696000in}}%
\pgfusepath{clip}%
\pgfsetbuttcap%
\pgfsetroundjoin%
\definecolor{currentfill}{rgb}{0.121569,0.466667,0.705882}%
\pgfsetfillcolor{currentfill}%
\pgfsetfillopacity{0.822994}%
\pgfsetlinewidth{1.003750pt}%
\definecolor{currentstroke}{rgb}{0.121569,0.466667,0.705882}%
\pgfsetstrokecolor{currentstroke}%
\pgfsetstrokeopacity{0.822994}%
\pgfsetdash{}{0pt}%
\pgfpathmoveto{\pgfqpoint{1.208552in}{1.416012in}}%
\pgfpathcurveto{\pgfqpoint{1.216788in}{1.416012in}}{\pgfqpoint{1.224688in}{1.419285in}}{\pgfqpoint{1.230512in}{1.425109in}}%
\pgfpathcurveto{\pgfqpoint{1.236336in}{1.430932in}}{\pgfqpoint{1.239609in}{1.438833in}}{\pgfqpoint{1.239609in}{1.447069in}}%
\pgfpathcurveto{\pgfqpoint{1.239609in}{1.455305in}}{\pgfqpoint{1.236336in}{1.463205in}}{\pgfqpoint{1.230512in}{1.469029in}}%
\pgfpathcurveto{\pgfqpoint{1.224688in}{1.474853in}}{\pgfqpoint{1.216788in}{1.478125in}}{\pgfqpoint{1.208552in}{1.478125in}}%
\pgfpathcurveto{\pgfqpoint{1.200316in}{1.478125in}}{\pgfqpoint{1.192416in}{1.474853in}}{\pgfqpoint{1.186592in}{1.469029in}}%
\pgfpathcurveto{\pgfqpoint{1.180768in}{1.463205in}}{\pgfqpoint{1.177496in}{1.455305in}}{\pgfqpoint{1.177496in}{1.447069in}}%
\pgfpathcurveto{\pgfqpoint{1.177496in}{1.438833in}}{\pgfqpoint{1.180768in}{1.430932in}}{\pgfqpoint{1.186592in}{1.425109in}}%
\pgfpathcurveto{\pgfqpoint{1.192416in}{1.419285in}}{\pgfqpoint{1.200316in}{1.416012in}}{\pgfqpoint{1.208552in}{1.416012in}}%
\pgfpathclose%
\pgfusepath{stroke,fill}%
\end{pgfscope}%
\begin{pgfscope}%
\pgfpathrectangle{\pgfqpoint{0.100000in}{0.212622in}}{\pgfqpoint{3.696000in}{3.696000in}}%
\pgfusepath{clip}%
\pgfsetbuttcap%
\pgfsetroundjoin%
\definecolor{currentfill}{rgb}{0.121569,0.466667,0.705882}%
\pgfsetfillcolor{currentfill}%
\pgfsetfillopacity{0.823416}%
\pgfsetlinewidth{1.003750pt}%
\definecolor{currentstroke}{rgb}{0.121569,0.466667,0.705882}%
\pgfsetstrokecolor{currentstroke}%
\pgfsetstrokeopacity{0.823416}%
\pgfsetdash{}{0pt}%
\pgfpathmoveto{\pgfqpoint{2.425755in}{2.610668in}}%
\pgfpathcurveto{\pgfqpoint{2.433992in}{2.610668in}}{\pgfqpoint{2.441892in}{2.613940in}}{\pgfqpoint{2.447715in}{2.619764in}}%
\pgfpathcurveto{\pgfqpoint{2.453539in}{2.625588in}}{\pgfqpoint{2.456812in}{2.633488in}}{\pgfqpoint{2.456812in}{2.641724in}}%
\pgfpathcurveto{\pgfqpoint{2.456812in}{2.649961in}}{\pgfqpoint{2.453539in}{2.657861in}}{\pgfqpoint{2.447715in}{2.663685in}}%
\pgfpathcurveto{\pgfqpoint{2.441892in}{2.669508in}}{\pgfqpoint{2.433992in}{2.672781in}}{\pgfqpoint{2.425755in}{2.672781in}}%
\pgfpathcurveto{\pgfqpoint{2.417519in}{2.672781in}}{\pgfqpoint{2.409619in}{2.669508in}}{\pgfqpoint{2.403795in}{2.663685in}}%
\pgfpathcurveto{\pgfqpoint{2.397971in}{2.657861in}}{\pgfqpoint{2.394699in}{2.649961in}}{\pgfqpoint{2.394699in}{2.641724in}}%
\pgfpathcurveto{\pgfqpoint{2.394699in}{2.633488in}}{\pgfqpoint{2.397971in}{2.625588in}}{\pgfqpoint{2.403795in}{2.619764in}}%
\pgfpathcurveto{\pgfqpoint{2.409619in}{2.613940in}}{\pgfqpoint{2.417519in}{2.610668in}}{\pgfqpoint{2.425755in}{2.610668in}}%
\pgfpathclose%
\pgfusepath{stroke,fill}%
\end{pgfscope}%
\begin{pgfscope}%
\pgfpathrectangle{\pgfqpoint{0.100000in}{0.212622in}}{\pgfqpoint{3.696000in}{3.696000in}}%
\pgfusepath{clip}%
\pgfsetbuttcap%
\pgfsetroundjoin%
\definecolor{currentfill}{rgb}{0.121569,0.466667,0.705882}%
\pgfsetfillcolor{currentfill}%
\pgfsetfillopacity{0.824054}%
\pgfsetlinewidth{1.003750pt}%
\definecolor{currentstroke}{rgb}{0.121569,0.466667,0.705882}%
\pgfsetstrokecolor{currentstroke}%
\pgfsetstrokeopacity{0.824054}%
\pgfsetdash{}{0pt}%
\pgfpathmoveto{\pgfqpoint{1.213173in}{1.412222in}}%
\pgfpathcurveto{\pgfqpoint{1.221409in}{1.412222in}}{\pgfqpoint{1.229309in}{1.415495in}}{\pgfqpoint{1.235133in}{1.421319in}}%
\pgfpathcurveto{\pgfqpoint{1.240957in}{1.427143in}}{\pgfqpoint{1.244229in}{1.435043in}}{\pgfqpoint{1.244229in}{1.443279in}}%
\pgfpathcurveto{\pgfqpoint{1.244229in}{1.451515in}}{\pgfqpoint{1.240957in}{1.459415in}}{\pgfqpoint{1.235133in}{1.465239in}}%
\pgfpathcurveto{\pgfqpoint{1.229309in}{1.471063in}}{\pgfqpoint{1.221409in}{1.474335in}}{\pgfqpoint{1.213173in}{1.474335in}}%
\pgfpathcurveto{\pgfqpoint{1.204937in}{1.474335in}}{\pgfqpoint{1.197037in}{1.471063in}}{\pgfqpoint{1.191213in}{1.465239in}}%
\pgfpathcurveto{\pgfqpoint{1.185389in}{1.459415in}}{\pgfqpoint{1.182116in}{1.451515in}}{\pgfqpoint{1.182116in}{1.443279in}}%
\pgfpathcurveto{\pgfqpoint{1.182116in}{1.435043in}}{\pgfqpoint{1.185389in}{1.427143in}}{\pgfqpoint{1.191213in}{1.421319in}}%
\pgfpathcurveto{\pgfqpoint{1.197037in}{1.415495in}}{\pgfqpoint{1.204937in}{1.412222in}}{\pgfqpoint{1.213173in}{1.412222in}}%
\pgfpathclose%
\pgfusepath{stroke,fill}%
\end{pgfscope}%
\begin{pgfscope}%
\pgfpathrectangle{\pgfqpoint{0.100000in}{0.212622in}}{\pgfqpoint{3.696000in}{3.696000in}}%
\pgfusepath{clip}%
\pgfsetbuttcap%
\pgfsetroundjoin%
\definecolor{currentfill}{rgb}{0.121569,0.466667,0.705882}%
\pgfsetfillcolor{currentfill}%
\pgfsetfillopacity{0.824131}%
\pgfsetlinewidth{1.003750pt}%
\definecolor{currentstroke}{rgb}{0.121569,0.466667,0.705882}%
\pgfsetstrokecolor{currentstroke}%
\pgfsetstrokeopacity{0.824131}%
\pgfsetdash{}{0pt}%
\pgfpathmoveto{\pgfqpoint{2.424105in}{2.605981in}}%
\pgfpathcurveto{\pgfqpoint{2.432341in}{2.605981in}}{\pgfqpoint{2.440241in}{2.609253in}}{\pgfqpoint{2.446065in}{2.615077in}}%
\pgfpathcurveto{\pgfqpoint{2.451889in}{2.620901in}}{\pgfqpoint{2.455162in}{2.628801in}}{\pgfqpoint{2.455162in}{2.637037in}}%
\pgfpathcurveto{\pgfqpoint{2.455162in}{2.645274in}}{\pgfqpoint{2.451889in}{2.653174in}}{\pgfqpoint{2.446065in}{2.658998in}}%
\pgfpathcurveto{\pgfqpoint{2.440241in}{2.664821in}}{\pgfqpoint{2.432341in}{2.668094in}}{\pgfqpoint{2.424105in}{2.668094in}}%
\pgfpathcurveto{\pgfqpoint{2.415869in}{2.668094in}}{\pgfqpoint{2.407969in}{2.664821in}}{\pgfqpoint{2.402145in}{2.658998in}}%
\pgfpathcurveto{\pgfqpoint{2.396321in}{2.653174in}}{\pgfqpoint{2.393049in}{2.645274in}}{\pgfqpoint{2.393049in}{2.637037in}}%
\pgfpathcurveto{\pgfqpoint{2.393049in}{2.628801in}}{\pgfqpoint{2.396321in}{2.620901in}}{\pgfqpoint{2.402145in}{2.615077in}}%
\pgfpathcurveto{\pgfqpoint{2.407969in}{2.609253in}}{\pgfqpoint{2.415869in}{2.605981in}}{\pgfqpoint{2.424105in}{2.605981in}}%
\pgfpathclose%
\pgfusepath{stroke,fill}%
\end{pgfscope}%
\begin{pgfscope}%
\pgfpathrectangle{\pgfqpoint{0.100000in}{0.212622in}}{\pgfqpoint{3.696000in}{3.696000in}}%
\pgfusepath{clip}%
\pgfsetbuttcap%
\pgfsetroundjoin%
\definecolor{currentfill}{rgb}{0.121569,0.466667,0.705882}%
\pgfsetfillcolor{currentfill}%
\pgfsetfillopacity{0.825290}%
\pgfsetlinewidth{1.003750pt}%
\definecolor{currentstroke}{rgb}{0.121569,0.466667,0.705882}%
\pgfsetstrokecolor{currentstroke}%
\pgfsetstrokeopacity{0.825290}%
\pgfsetdash{}{0pt}%
\pgfpathmoveto{\pgfqpoint{1.217989in}{1.408054in}}%
\pgfpathcurveto{\pgfqpoint{1.226225in}{1.408054in}}{\pgfqpoint{1.234125in}{1.411327in}}{\pgfqpoint{1.239949in}{1.417151in}}%
\pgfpathcurveto{\pgfqpoint{1.245773in}{1.422974in}}{\pgfqpoint{1.249045in}{1.430875in}}{\pgfqpoint{1.249045in}{1.439111in}}%
\pgfpathcurveto{\pgfqpoint{1.249045in}{1.447347in}}{\pgfqpoint{1.245773in}{1.455247in}}{\pgfqpoint{1.239949in}{1.461071in}}%
\pgfpathcurveto{\pgfqpoint{1.234125in}{1.466895in}}{\pgfqpoint{1.226225in}{1.470167in}}{\pgfqpoint{1.217989in}{1.470167in}}%
\pgfpathcurveto{\pgfqpoint{1.209752in}{1.470167in}}{\pgfqpoint{1.201852in}{1.466895in}}{\pgfqpoint{1.196028in}{1.461071in}}%
\pgfpathcurveto{\pgfqpoint{1.190205in}{1.455247in}}{\pgfqpoint{1.186932in}{1.447347in}}{\pgfqpoint{1.186932in}{1.439111in}}%
\pgfpathcurveto{\pgfqpoint{1.186932in}{1.430875in}}{\pgfqpoint{1.190205in}{1.422974in}}{\pgfqpoint{1.196028in}{1.417151in}}%
\pgfpathcurveto{\pgfqpoint{1.201852in}{1.411327in}}{\pgfqpoint{1.209752in}{1.408054in}}{\pgfqpoint{1.217989in}{1.408054in}}%
\pgfpathclose%
\pgfusepath{stroke,fill}%
\end{pgfscope}%
\begin{pgfscope}%
\pgfpathrectangle{\pgfqpoint{0.100000in}{0.212622in}}{\pgfqpoint{3.696000in}{3.696000in}}%
\pgfusepath{clip}%
\pgfsetbuttcap%
\pgfsetroundjoin%
\definecolor{currentfill}{rgb}{0.121569,0.466667,0.705882}%
\pgfsetfillcolor{currentfill}%
\pgfsetfillopacity{0.825320}%
\pgfsetlinewidth{1.003750pt}%
\definecolor{currentstroke}{rgb}{0.121569,0.466667,0.705882}%
\pgfsetstrokecolor{currentstroke}%
\pgfsetstrokeopacity{0.825320}%
\pgfsetdash{}{0pt}%
\pgfpathmoveto{\pgfqpoint{2.421523in}{2.596743in}}%
\pgfpathcurveto{\pgfqpoint{2.429759in}{2.596743in}}{\pgfqpoint{2.437659in}{2.600015in}}{\pgfqpoint{2.443483in}{2.605839in}}%
\pgfpathcurveto{\pgfqpoint{2.449307in}{2.611663in}}{\pgfqpoint{2.452579in}{2.619563in}}{\pgfqpoint{2.452579in}{2.627799in}}%
\pgfpathcurveto{\pgfqpoint{2.452579in}{2.636036in}}{\pgfqpoint{2.449307in}{2.643936in}}{\pgfqpoint{2.443483in}{2.649760in}}%
\pgfpathcurveto{\pgfqpoint{2.437659in}{2.655584in}}{\pgfqpoint{2.429759in}{2.658856in}}{\pgfqpoint{2.421523in}{2.658856in}}%
\pgfpathcurveto{\pgfqpoint{2.413286in}{2.658856in}}{\pgfqpoint{2.405386in}{2.655584in}}{\pgfqpoint{2.399562in}{2.649760in}}%
\pgfpathcurveto{\pgfqpoint{2.393738in}{2.643936in}}{\pgfqpoint{2.390466in}{2.636036in}}{\pgfqpoint{2.390466in}{2.627799in}}%
\pgfpathcurveto{\pgfqpoint{2.390466in}{2.619563in}}{\pgfqpoint{2.393738in}{2.611663in}}{\pgfqpoint{2.399562in}{2.605839in}}%
\pgfpathcurveto{\pgfqpoint{2.405386in}{2.600015in}}{\pgfqpoint{2.413286in}{2.596743in}}{\pgfqpoint{2.421523in}{2.596743in}}%
\pgfpathclose%
\pgfusepath{stroke,fill}%
\end{pgfscope}%
\begin{pgfscope}%
\pgfpathrectangle{\pgfqpoint{0.100000in}{0.212622in}}{\pgfqpoint{3.696000in}{3.696000in}}%
\pgfusepath{clip}%
\pgfsetbuttcap%
\pgfsetroundjoin%
\definecolor{currentfill}{rgb}{0.121569,0.466667,0.705882}%
\pgfsetfillcolor{currentfill}%
\pgfsetfillopacity{0.825679}%
\pgfsetlinewidth{1.003750pt}%
\definecolor{currentstroke}{rgb}{0.121569,0.466667,0.705882}%
\pgfsetstrokecolor{currentstroke}%
\pgfsetstrokeopacity{0.825679}%
\pgfsetdash{}{0pt}%
\pgfpathmoveto{\pgfqpoint{2.981417in}{1.789950in}}%
\pgfpathcurveto{\pgfqpoint{2.989653in}{1.789950in}}{\pgfqpoint{2.997553in}{1.793223in}}{\pgfqpoint{3.003377in}{1.799047in}}%
\pgfpathcurveto{\pgfqpoint{3.009201in}{1.804871in}}{\pgfqpoint{3.012474in}{1.812771in}}{\pgfqpoint{3.012474in}{1.821007in}}%
\pgfpathcurveto{\pgfqpoint{3.012474in}{1.829243in}}{\pgfqpoint{3.009201in}{1.837143in}}{\pgfqpoint{3.003377in}{1.842967in}}%
\pgfpathcurveto{\pgfqpoint{2.997553in}{1.848791in}}{\pgfqpoint{2.989653in}{1.852063in}}{\pgfqpoint{2.981417in}{1.852063in}}%
\pgfpathcurveto{\pgfqpoint{2.973181in}{1.852063in}}{\pgfqpoint{2.965281in}{1.848791in}}{\pgfqpoint{2.959457in}{1.842967in}}%
\pgfpathcurveto{\pgfqpoint{2.953633in}{1.837143in}}{\pgfqpoint{2.950361in}{1.829243in}}{\pgfqpoint{2.950361in}{1.821007in}}%
\pgfpathcurveto{\pgfqpoint{2.950361in}{1.812771in}}{\pgfqpoint{2.953633in}{1.804871in}}{\pgfqpoint{2.959457in}{1.799047in}}%
\pgfpathcurveto{\pgfqpoint{2.965281in}{1.793223in}}{\pgfqpoint{2.973181in}{1.789950in}}{\pgfqpoint{2.981417in}{1.789950in}}%
\pgfpathclose%
\pgfusepath{stroke,fill}%
\end{pgfscope}%
\begin{pgfscope}%
\pgfpathrectangle{\pgfqpoint{0.100000in}{0.212622in}}{\pgfqpoint{3.696000in}{3.696000in}}%
\pgfusepath{clip}%
\pgfsetbuttcap%
\pgfsetroundjoin%
\definecolor{currentfill}{rgb}{0.121569,0.466667,0.705882}%
\pgfsetfillcolor{currentfill}%
\pgfsetfillopacity{0.825821}%
\pgfsetlinewidth{1.003750pt}%
\definecolor{currentstroke}{rgb}{0.121569,0.466667,0.705882}%
\pgfsetstrokecolor{currentstroke}%
\pgfsetstrokeopacity{0.825821}%
\pgfsetdash{}{0pt}%
\pgfpathmoveto{\pgfqpoint{1.220824in}{1.405577in}}%
\pgfpathcurveto{\pgfqpoint{1.229060in}{1.405577in}}{\pgfqpoint{1.236960in}{1.408849in}}{\pgfqpoint{1.242784in}{1.414673in}}%
\pgfpathcurveto{\pgfqpoint{1.248608in}{1.420497in}}{\pgfqpoint{1.251880in}{1.428397in}}{\pgfqpoint{1.251880in}{1.436633in}}%
\pgfpathcurveto{\pgfqpoint{1.251880in}{1.444870in}}{\pgfqpoint{1.248608in}{1.452770in}}{\pgfqpoint{1.242784in}{1.458594in}}%
\pgfpathcurveto{\pgfqpoint{1.236960in}{1.464418in}}{\pgfqpoint{1.229060in}{1.467690in}}{\pgfqpoint{1.220824in}{1.467690in}}%
\pgfpathcurveto{\pgfqpoint{1.212588in}{1.467690in}}{\pgfqpoint{1.204688in}{1.464418in}}{\pgfqpoint{1.198864in}{1.458594in}}%
\pgfpathcurveto{\pgfqpoint{1.193040in}{1.452770in}}{\pgfqpoint{1.189767in}{1.444870in}}{\pgfqpoint{1.189767in}{1.436633in}}%
\pgfpathcurveto{\pgfqpoint{1.189767in}{1.428397in}}{\pgfqpoint{1.193040in}{1.420497in}}{\pgfqpoint{1.198864in}{1.414673in}}%
\pgfpathcurveto{\pgfqpoint{1.204688in}{1.408849in}}{\pgfqpoint{1.212588in}{1.405577in}}{\pgfqpoint{1.220824in}{1.405577in}}%
\pgfpathclose%
\pgfusepath{stroke,fill}%
\end{pgfscope}%
\begin{pgfscope}%
\pgfpathrectangle{\pgfqpoint{0.100000in}{0.212622in}}{\pgfqpoint{3.696000in}{3.696000in}}%
\pgfusepath{clip}%
\pgfsetbuttcap%
\pgfsetroundjoin%
\definecolor{currentfill}{rgb}{0.121569,0.466667,0.705882}%
\pgfsetfillcolor{currentfill}%
\pgfsetfillopacity{0.826094}%
\pgfsetlinewidth{1.003750pt}%
\definecolor{currentstroke}{rgb}{0.121569,0.466667,0.705882}%
\pgfsetstrokecolor{currentstroke}%
\pgfsetstrokeopacity{0.826094}%
\pgfsetdash{}{0pt}%
\pgfpathmoveto{\pgfqpoint{2.417771in}{2.590359in}}%
\pgfpathcurveto{\pgfqpoint{2.426008in}{2.590359in}}{\pgfqpoint{2.433908in}{2.593631in}}{\pgfqpoint{2.439731in}{2.599455in}}%
\pgfpathcurveto{\pgfqpoint{2.445555in}{2.605279in}}{\pgfqpoint{2.448828in}{2.613179in}}{\pgfqpoint{2.448828in}{2.621416in}}%
\pgfpathcurveto{\pgfqpoint{2.448828in}{2.629652in}}{\pgfqpoint{2.445555in}{2.637552in}}{\pgfqpoint{2.439731in}{2.643376in}}%
\pgfpathcurveto{\pgfqpoint{2.433908in}{2.649200in}}{\pgfqpoint{2.426008in}{2.652472in}}{\pgfqpoint{2.417771in}{2.652472in}}%
\pgfpathcurveto{\pgfqpoint{2.409535in}{2.652472in}}{\pgfqpoint{2.401635in}{2.649200in}}{\pgfqpoint{2.395811in}{2.643376in}}%
\pgfpathcurveto{\pgfqpoint{2.389987in}{2.637552in}}{\pgfqpoint{2.386715in}{2.629652in}}{\pgfqpoint{2.386715in}{2.621416in}}%
\pgfpathcurveto{\pgfqpoint{2.386715in}{2.613179in}}{\pgfqpoint{2.389987in}{2.605279in}}{\pgfqpoint{2.395811in}{2.599455in}}%
\pgfpathcurveto{\pgfqpoint{2.401635in}{2.593631in}}{\pgfqpoint{2.409535in}{2.590359in}}{\pgfqpoint{2.417771in}{2.590359in}}%
\pgfpathclose%
\pgfusepath{stroke,fill}%
\end{pgfscope}%
\begin{pgfscope}%
\pgfpathrectangle{\pgfqpoint{0.100000in}{0.212622in}}{\pgfqpoint{3.696000in}{3.696000in}}%
\pgfusepath{clip}%
\pgfsetbuttcap%
\pgfsetroundjoin%
\definecolor{currentfill}{rgb}{0.121569,0.466667,0.705882}%
\pgfsetfillcolor{currentfill}%
\pgfsetfillopacity{0.826444}%
\pgfsetlinewidth{1.003750pt}%
\definecolor{currentstroke}{rgb}{0.121569,0.466667,0.705882}%
\pgfsetstrokecolor{currentstroke}%
\pgfsetstrokeopacity{0.826444}%
\pgfsetdash{}{0pt}%
\pgfpathmoveto{\pgfqpoint{1.224165in}{1.401516in}}%
\pgfpathcurveto{\pgfqpoint{1.232401in}{1.401516in}}{\pgfqpoint{1.240301in}{1.404788in}}{\pgfqpoint{1.246125in}{1.410612in}}%
\pgfpathcurveto{\pgfqpoint{1.251949in}{1.416436in}}{\pgfqpoint{1.255221in}{1.424336in}}{\pgfqpoint{1.255221in}{1.432573in}}%
\pgfpathcurveto{\pgfqpoint{1.255221in}{1.440809in}}{\pgfqpoint{1.251949in}{1.448709in}}{\pgfqpoint{1.246125in}{1.454533in}}%
\pgfpathcurveto{\pgfqpoint{1.240301in}{1.460357in}}{\pgfqpoint{1.232401in}{1.463629in}}{\pgfqpoint{1.224165in}{1.463629in}}%
\pgfpathcurveto{\pgfqpoint{1.215928in}{1.463629in}}{\pgfqpoint{1.208028in}{1.460357in}}{\pgfqpoint{1.202204in}{1.454533in}}%
\pgfpathcurveto{\pgfqpoint{1.196380in}{1.448709in}}{\pgfqpoint{1.193108in}{1.440809in}}{\pgfqpoint{1.193108in}{1.432573in}}%
\pgfpathcurveto{\pgfqpoint{1.193108in}{1.424336in}}{\pgfqpoint{1.196380in}{1.416436in}}{\pgfqpoint{1.202204in}{1.410612in}}%
\pgfpathcurveto{\pgfqpoint{1.208028in}{1.404788in}}{\pgfqpoint{1.215928in}{1.401516in}}{\pgfqpoint{1.224165in}{1.401516in}}%
\pgfpathclose%
\pgfusepath{stroke,fill}%
\end{pgfscope}%
\begin{pgfscope}%
\pgfpathrectangle{\pgfqpoint{0.100000in}{0.212622in}}{\pgfqpoint{3.696000in}{3.696000in}}%
\pgfusepath{clip}%
\pgfsetbuttcap%
\pgfsetroundjoin%
\definecolor{currentfill}{rgb}{0.121569,0.466667,0.705882}%
\pgfsetfillcolor{currentfill}%
\pgfsetfillopacity{0.827473}%
\pgfsetlinewidth{1.003750pt}%
\definecolor{currentstroke}{rgb}{0.121569,0.466667,0.705882}%
\pgfsetstrokecolor{currentstroke}%
\pgfsetstrokeopacity{0.827473}%
\pgfsetdash{}{0pt}%
\pgfpathmoveto{\pgfqpoint{2.410543in}{2.579481in}}%
\pgfpathcurveto{\pgfqpoint{2.418779in}{2.579481in}}{\pgfqpoint{2.426679in}{2.582753in}}{\pgfqpoint{2.432503in}{2.588577in}}%
\pgfpathcurveto{\pgfqpoint{2.438327in}{2.594401in}}{\pgfqpoint{2.441599in}{2.602301in}}{\pgfqpoint{2.441599in}{2.610537in}}%
\pgfpathcurveto{\pgfqpoint{2.441599in}{2.618774in}}{\pgfqpoint{2.438327in}{2.626674in}}{\pgfqpoint{2.432503in}{2.632498in}}%
\pgfpathcurveto{\pgfqpoint{2.426679in}{2.638322in}}{\pgfqpoint{2.418779in}{2.641594in}}{\pgfqpoint{2.410543in}{2.641594in}}%
\pgfpathcurveto{\pgfqpoint{2.402307in}{2.641594in}}{\pgfqpoint{2.394407in}{2.638322in}}{\pgfqpoint{2.388583in}{2.632498in}}%
\pgfpathcurveto{\pgfqpoint{2.382759in}{2.626674in}}{\pgfqpoint{2.379486in}{2.618774in}}{\pgfqpoint{2.379486in}{2.610537in}}%
\pgfpathcurveto{\pgfqpoint{2.379486in}{2.602301in}}{\pgfqpoint{2.382759in}{2.594401in}}{\pgfqpoint{2.388583in}{2.588577in}}%
\pgfpathcurveto{\pgfqpoint{2.394407in}{2.582753in}}{\pgfqpoint{2.402307in}{2.579481in}}{\pgfqpoint{2.410543in}{2.579481in}}%
\pgfpathclose%
\pgfusepath{stroke,fill}%
\end{pgfscope}%
\begin{pgfscope}%
\pgfpathrectangle{\pgfqpoint{0.100000in}{0.212622in}}{\pgfqpoint{3.696000in}{3.696000in}}%
\pgfusepath{clip}%
\pgfsetbuttcap%
\pgfsetroundjoin%
\definecolor{currentfill}{rgb}{0.121569,0.466667,0.705882}%
\pgfsetfillcolor{currentfill}%
\pgfsetfillopacity{0.827828}%
\pgfsetlinewidth{1.003750pt}%
\definecolor{currentstroke}{rgb}{0.121569,0.466667,0.705882}%
\pgfsetstrokecolor{currentstroke}%
\pgfsetstrokeopacity{0.827828}%
\pgfsetdash{}{0pt}%
\pgfpathmoveto{\pgfqpoint{1.229245in}{1.395946in}}%
\pgfpathcurveto{\pgfqpoint{1.237482in}{1.395946in}}{\pgfqpoint{1.245382in}{1.399219in}}{\pgfqpoint{1.251206in}{1.405043in}}%
\pgfpathcurveto{\pgfqpoint{1.257030in}{1.410866in}}{\pgfqpoint{1.260302in}{1.418767in}}{\pgfqpoint{1.260302in}{1.427003in}}%
\pgfpathcurveto{\pgfqpoint{1.260302in}{1.435239in}}{\pgfqpoint{1.257030in}{1.443139in}}{\pgfqpoint{1.251206in}{1.448963in}}%
\pgfpathcurveto{\pgfqpoint{1.245382in}{1.454787in}}{\pgfqpoint{1.237482in}{1.458059in}}{\pgfqpoint{1.229245in}{1.458059in}}%
\pgfpathcurveto{\pgfqpoint{1.221009in}{1.458059in}}{\pgfqpoint{1.213109in}{1.454787in}}{\pgfqpoint{1.207285in}{1.448963in}}%
\pgfpathcurveto{\pgfqpoint{1.201461in}{1.443139in}}{\pgfqpoint{1.198189in}{1.435239in}}{\pgfqpoint{1.198189in}{1.427003in}}%
\pgfpathcurveto{\pgfqpoint{1.198189in}{1.418767in}}{\pgfqpoint{1.201461in}{1.410866in}}{\pgfqpoint{1.207285in}{1.405043in}}%
\pgfpathcurveto{\pgfqpoint{1.213109in}{1.399219in}}{\pgfqpoint{1.221009in}{1.395946in}}{\pgfqpoint{1.229245in}{1.395946in}}%
\pgfpathclose%
\pgfusepath{stroke,fill}%
\end{pgfscope}%
\begin{pgfscope}%
\pgfpathrectangle{\pgfqpoint{0.100000in}{0.212622in}}{\pgfqpoint{3.696000in}{3.696000in}}%
\pgfusepath{clip}%
\pgfsetbuttcap%
\pgfsetroundjoin%
\definecolor{currentfill}{rgb}{0.121569,0.466667,0.705882}%
\pgfsetfillcolor{currentfill}%
\pgfsetfillopacity{0.828305}%
\pgfsetlinewidth{1.003750pt}%
\definecolor{currentstroke}{rgb}{0.121569,0.466667,0.705882}%
\pgfsetstrokecolor{currentstroke}%
\pgfsetstrokeopacity{0.828305}%
\pgfsetdash{}{0pt}%
\pgfpathmoveto{\pgfqpoint{2.973271in}{1.775783in}}%
\pgfpathcurveto{\pgfqpoint{2.981507in}{1.775783in}}{\pgfqpoint{2.989407in}{1.779056in}}{\pgfqpoint{2.995231in}{1.784880in}}%
\pgfpathcurveto{\pgfqpoint{3.001055in}{1.790703in}}{\pgfqpoint{3.004327in}{1.798603in}}{\pgfqpoint{3.004327in}{1.806840in}}%
\pgfpathcurveto{\pgfqpoint{3.004327in}{1.815076in}}{\pgfqpoint{3.001055in}{1.822976in}}{\pgfqpoint{2.995231in}{1.828800in}}%
\pgfpathcurveto{\pgfqpoint{2.989407in}{1.834624in}}{\pgfqpoint{2.981507in}{1.837896in}}{\pgfqpoint{2.973271in}{1.837896in}}%
\pgfpathcurveto{\pgfqpoint{2.965035in}{1.837896in}}{\pgfqpoint{2.957135in}{1.834624in}}{\pgfqpoint{2.951311in}{1.828800in}}%
\pgfpathcurveto{\pgfqpoint{2.945487in}{1.822976in}}{\pgfqpoint{2.942214in}{1.815076in}}{\pgfqpoint{2.942214in}{1.806840in}}%
\pgfpathcurveto{\pgfqpoint{2.942214in}{1.798603in}}{\pgfqpoint{2.945487in}{1.790703in}}{\pgfqpoint{2.951311in}{1.784880in}}%
\pgfpathcurveto{\pgfqpoint{2.957135in}{1.779056in}}{\pgfqpoint{2.965035in}{1.775783in}}{\pgfqpoint{2.973271in}{1.775783in}}%
\pgfpathclose%
\pgfusepath{stroke,fill}%
\end{pgfscope}%
\begin{pgfscope}%
\pgfpathrectangle{\pgfqpoint{0.100000in}{0.212622in}}{\pgfqpoint{3.696000in}{3.696000in}}%
\pgfusepath{clip}%
\pgfsetbuttcap%
\pgfsetroundjoin%
\definecolor{currentfill}{rgb}{0.121569,0.466667,0.705882}%
\pgfsetfillcolor{currentfill}%
\pgfsetfillopacity{0.829239}%
\pgfsetlinewidth{1.003750pt}%
\definecolor{currentstroke}{rgb}{0.121569,0.466667,0.705882}%
\pgfsetstrokecolor{currentstroke}%
\pgfsetstrokeopacity{0.829239}%
\pgfsetdash{}{0pt}%
\pgfpathmoveto{\pgfqpoint{2.406337in}{2.568987in}}%
\pgfpathcurveto{\pgfqpoint{2.414573in}{2.568987in}}{\pgfqpoint{2.422473in}{2.572259in}}{\pgfqpoint{2.428297in}{2.578083in}}%
\pgfpathcurveto{\pgfqpoint{2.434121in}{2.583907in}}{\pgfqpoint{2.437394in}{2.591807in}}{\pgfqpoint{2.437394in}{2.600043in}}%
\pgfpathcurveto{\pgfqpoint{2.437394in}{2.608279in}}{\pgfqpoint{2.434121in}{2.616179in}}{\pgfqpoint{2.428297in}{2.622003in}}%
\pgfpathcurveto{\pgfqpoint{2.422473in}{2.627827in}}{\pgfqpoint{2.414573in}{2.631100in}}{\pgfqpoint{2.406337in}{2.631100in}}%
\pgfpathcurveto{\pgfqpoint{2.398101in}{2.631100in}}{\pgfqpoint{2.390201in}{2.627827in}}{\pgfqpoint{2.384377in}{2.622003in}}%
\pgfpathcurveto{\pgfqpoint{2.378553in}{2.616179in}}{\pgfqpoint{2.375281in}{2.608279in}}{\pgfqpoint{2.375281in}{2.600043in}}%
\pgfpathcurveto{\pgfqpoint{2.375281in}{2.591807in}}{\pgfqpoint{2.378553in}{2.583907in}}{\pgfqpoint{2.384377in}{2.578083in}}%
\pgfpathcurveto{\pgfqpoint{2.390201in}{2.572259in}}{\pgfqpoint{2.398101in}{2.568987in}}{\pgfqpoint{2.406337in}{2.568987in}}%
\pgfpathclose%
\pgfusepath{stroke,fill}%
\end{pgfscope}%
\begin{pgfscope}%
\pgfpathrectangle{\pgfqpoint{0.100000in}{0.212622in}}{\pgfqpoint{3.696000in}{3.696000in}}%
\pgfusepath{clip}%
\pgfsetbuttcap%
\pgfsetroundjoin%
\definecolor{currentfill}{rgb}{0.121569,0.466667,0.705882}%
\pgfsetfillcolor{currentfill}%
\pgfsetfillopacity{0.829610}%
\pgfsetlinewidth{1.003750pt}%
\definecolor{currentstroke}{rgb}{0.121569,0.466667,0.705882}%
\pgfsetstrokecolor{currentstroke}%
\pgfsetstrokeopacity{0.829610}%
\pgfsetdash{}{0pt}%
\pgfpathmoveto{\pgfqpoint{1.235989in}{1.390197in}}%
\pgfpathcurveto{\pgfqpoint{1.244226in}{1.390197in}}{\pgfqpoint{1.252126in}{1.393470in}}{\pgfqpoint{1.257950in}{1.399294in}}%
\pgfpathcurveto{\pgfqpoint{1.263773in}{1.405118in}}{\pgfqpoint{1.267046in}{1.413018in}}{\pgfqpoint{1.267046in}{1.421254in}}%
\pgfpathcurveto{\pgfqpoint{1.267046in}{1.429490in}}{\pgfqpoint{1.263773in}{1.437390in}}{\pgfqpoint{1.257950in}{1.443214in}}%
\pgfpathcurveto{\pgfqpoint{1.252126in}{1.449038in}}{\pgfqpoint{1.244226in}{1.452310in}}{\pgfqpoint{1.235989in}{1.452310in}}%
\pgfpathcurveto{\pgfqpoint{1.227753in}{1.452310in}}{\pgfqpoint{1.219853in}{1.449038in}}{\pgfqpoint{1.214029in}{1.443214in}}%
\pgfpathcurveto{\pgfqpoint{1.208205in}{1.437390in}}{\pgfqpoint{1.204933in}{1.429490in}}{\pgfqpoint{1.204933in}{1.421254in}}%
\pgfpathcurveto{\pgfqpoint{1.204933in}{1.413018in}}{\pgfqpoint{1.208205in}{1.405118in}}{\pgfqpoint{1.214029in}{1.399294in}}%
\pgfpathcurveto{\pgfqpoint{1.219853in}{1.393470in}}{\pgfqpoint{1.227753in}{1.390197in}}{\pgfqpoint{1.235989in}{1.390197in}}%
\pgfpathclose%
\pgfusepath{stroke,fill}%
\end{pgfscope}%
\begin{pgfscope}%
\pgfpathrectangle{\pgfqpoint{0.100000in}{0.212622in}}{\pgfqpoint{3.696000in}{3.696000in}}%
\pgfusepath{clip}%
\pgfsetbuttcap%
\pgfsetroundjoin%
\definecolor{currentfill}{rgb}{0.121569,0.466667,0.705882}%
\pgfsetfillcolor{currentfill}%
\pgfsetfillopacity{0.830581}%
\pgfsetlinewidth{1.003750pt}%
\definecolor{currentstroke}{rgb}{0.121569,0.466667,0.705882}%
\pgfsetstrokecolor{currentstroke}%
\pgfsetstrokeopacity{0.830581}%
\pgfsetdash{}{0pt}%
\pgfpathmoveto{\pgfqpoint{2.964640in}{1.763858in}}%
\pgfpathcurveto{\pgfqpoint{2.972876in}{1.763858in}}{\pgfqpoint{2.980776in}{1.767130in}}{\pgfqpoint{2.986600in}{1.772954in}}%
\pgfpathcurveto{\pgfqpoint{2.992424in}{1.778778in}}{\pgfqpoint{2.995696in}{1.786678in}}{\pgfqpoint{2.995696in}{1.794914in}}%
\pgfpathcurveto{\pgfqpoint{2.995696in}{1.803151in}}{\pgfqpoint{2.992424in}{1.811051in}}{\pgfqpoint{2.986600in}{1.816875in}}%
\pgfpathcurveto{\pgfqpoint{2.980776in}{1.822699in}}{\pgfqpoint{2.972876in}{1.825971in}}{\pgfqpoint{2.964640in}{1.825971in}}%
\pgfpathcurveto{\pgfqpoint{2.956403in}{1.825971in}}{\pgfqpoint{2.948503in}{1.822699in}}{\pgfqpoint{2.942679in}{1.816875in}}%
\pgfpathcurveto{\pgfqpoint{2.936855in}{1.811051in}}{\pgfqpoint{2.933583in}{1.803151in}}{\pgfqpoint{2.933583in}{1.794914in}}%
\pgfpathcurveto{\pgfqpoint{2.933583in}{1.786678in}}{\pgfqpoint{2.936855in}{1.778778in}}{\pgfqpoint{2.942679in}{1.772954in}}%
\pgfpathcurveto{\pgfqpoint{2.948503in}{1.767130in}}{\pgfqpoint{2.956403in}{1.763858in}}{\pgfqpoint{2.964640in}{1.763858in}}%
\pgfpathclose%
\pgfusepath{stroke,fill}%
\end{pgfscope}%
\begin{pgfscope}%
\pgfpathrectangle{\pgfqpoint{0.100000in}{0.212622in}}{\pgfqpoint{3.696000in}{3.696000in}}%
\pgfusepath{clip}%
\pgfsetbuttcap%
\pgfsetroundjoin%
\definecolor{currentfill}{rgb}{0.121569,0.466667,0.705882}%
\pgfsetfillcolor{currentfill}%
\pgfsetfillopacity{0.830819}%
\pgfsetlinewidth{1.003750pt}%
\definecolor{currentstroke}{rgb}{0.121569,0.466667,0.705882}%
\pgfsetstrokecolor{currentstroke}%
\pgfsetstrokeopacity{0.830819}%
\pgfsetdash{}{0pt}%
\pgfpathmoveto{\pgfqpoint{2.405035in}{2.558265in}}%
\pgfpathcurveto{\pgfqpoint{2.413272in}{2.558265in}}{\pgfqpoint{2.421172in}{2.561537in}}{\pgfqpoint{2.426996in}{2.567361in}}%
\pgfpathcurveto{\pgfqpoint{2.432819in}{2.573185in}}{\pgfqpoint{2.436092in}{2.581085in}}{\pgfqpoint{2.436092in}{2.589322in}}%
\pgfpathcurveto{\pgfqpoint{2.436092in}{2.597558in}}{\pgfqpoint{2.432819in}{2.605458in}}{\pgfqpoint{2.426996in}{2.611282in}}%
\pgfpathcurveto{\pgfqpoint{2.421172in}{2.617106in}}{\pgfqpoint{2.413272in}{2.620378in}}{\pgfqpoint{2.405035in}{2.620378in}}%
\pgfpathcurveto{\pgfqpoint{2.396799in}{2.620378in}}{\pgfqpoint{2.388899in}{2.617106in}}{\pgfqpoint{2.383075in}{2.611282in}}%
\pgfpathcurveto{\pgfqpoint{2.377251in}{2.605458in}}{\pgfqpoint{2.373979in}{2.597558in}}{\pgfqpoint{2.373979in}{2.589322in}}%
\pgfpathcurveto{\pgfqpoint{2.373979in}{2.581085in}}{\pgfqpoint{2.377251in}{2.573185in}}{\pgfqpoint{2.383075in}{2.567361in}}%
\pgfpathcurveto{\pgfqpoint{2.388899in}{2.561537in}}{\pgfqpoint{2.396799in}{2.558265in}}{\pgfqpoint{2.405035in}{2.558265in}}%
\pgfpathclose%
\pgfusepath{stroke,fill}%
\end{pgfscope}%
\begin{pgfscope}%
\pgfpathrectangle{\pgfqpoint{0.100000in}{0.212622in}}{\pgfqpoint{3.696000in}{3.696000in}}%
\pgfusepath{clip}%
\pgfsetbuttcap%
\pgfsetroundjoin%
\definecolor{currentfill}{rgb}{0.121569,0.466667,0.705882}%
\pgfsetfillcolor{currentfill}%
\pgfsetfillopacity{0.831195}%
\pgfsetlinewidth{1.003750pt}%
\definecolor{currentstroke}{rgb}{0.121569,0.466667,0.705882}%
\pgfsetstrokecolor{currentstroke}%
\pgfsetstrokeopacity{0.831195}%
\pgfsetdash{}{0pt}%
\pgfpathmoveto{\pgfqpoint{1.243720in}{1.383628in}}%
\pgfpathcurveto{\pgfqpoint{1.251956in}{1.383628in}}{\pgfqpoint{1.259856in}{1.386901in}}{\pgfqpoint{1.265680in}{1.392724in}}%
\pgfpathcurveto{\pgfqpoint{1.271504in}{1.398548in}}{\pgfqpoint{1.274776in}{1.406448in}}{\pgfqpoint{1.274776in}{1.414685in}}%
\pgfpathcurveto{\pgfqpoint{1.274776in}{1.422921in}}{\pgfqpoint{1.271504in}{1.430821in}}{\pgfqpoint{1.265680in}{1.436645in}}%
\pgfpathcurveto{\pgfqpoint{1.259856in}{1.442469in}}{\pgfqpoint{1.251956in}{1.445741in}}{\pgfqpoint{1.243720in}{1.445741in}}%
\pgfpathcurveto{\pgfqpoint{1.235483in}{1.445741in}}{\pgfqpoint{1.227583in}{1.442469in}}{\pgfqpoint{1.221759in}{1.436645in}}%
\pgfpathcurveto{\pgfqpoint{1.215935in}{1.430821in}}{\pgfqpoint{1.212663in}{1.422921in}}{\pgfqpoint{1.212663in}{1.414685in}}%
\pgfpathcurveto{\pgfqpoint{1.212663in}{1.406448in}}{\pgfqpoint{1.215935in}{1.398548in}}{\pgfqpoint{1.221759in}{1.392724in}}%
\pgfpathcurveto{\pgfqpoint{1.227583in}{1.386901in}}{\pgfqpoint{1.235483in}{1.383628in}}{\pgfqpoint{1.243720in}{1.383628in}}%
\pgfpathclose%
\pgfusepath{stroke,fill}%
\end{pgfscope}%
\begin{pgfscope}%
\pgfpathrectangle{\pgfqpoint{0.100000in}{0.212622in}}{\pgfqpoint{3.696000in}{3.696000in}}%
\pgfusepath{clip}%
\pgfsetbuttcap%
\pgfsetroundjoin%
\definecolor{currentfill}{rgb}{0.121569,0.466667,0.705882}%
\pgfsetfillcolor{currentfill}%
\pgfsetfillopacity{0.831967}%
\pgfsetlinewidth{1.003750pt}%
\definecolor{currentstroke}{rgb}{0.121569,0.466667,0.705882}%
\pgfsetstrokecolor{currentstroke}%
\pgfsetstrokeopacity{0.831967}%
\pgfsetdash{}{0pt}%
\pgfpathmoveto{\pgfqpoint{2.400597in}{2.550670in}}%
\pgfpathcurveto{\pgfqpoint{2.408833in}{2.550670in}}{\pgfqpoint{2.416733in}{2.553942in}}{\pgfqpoint{2.422557in}{2.559766in}}%
\pgfpathcurveto{\pgfqpoint{2.428381in}{2.565590in}}{\pgfqpoint{2.431653in}{2.573490in}}{\pgfqpoint{2.431653in}{2.581726in}}%
\pgfpathcurveto{\pgfqpoint{2.431653in}{2.589963in}}{\pgfqpoint{2.428381in}{2.597863in}}{\pgfqpoint{2.422557in}{2.603687in}}%
\pgfpathcurveto{\pgfqpoint{2.416733in}{2.609511in}}{\pgfqpoint{2.408833in}{2.612783in}}{\pgfqpoint{2.400597in}{2.612783in}}%
\pgfpathcurveto{\pgfqpoint{2.392361in}{2.612783in}}{\pgfqpoint{2.384461in}{2.609511in}}{\pgfqpoint{2.378637in}{2.603687in}}%
\pgfpathcurveto{\pgfqpoint{2.372813in}{2.597863in}}{\pgfqpoint{2.369540in}{2.589963in}}{\pgfqpoint{2.369540in}{2.581726in}}%
\pgfpathcurveto{\pgfqpoint{2.369540in}{2.573490in}}{\pgfqpoint{2.372813in}{2.565590in}}{\pgfqpoint{2.378637in}{2.559766in}}%
\pgfpathcurveto{\pgfqpoint{2.384461in}{2.553942in}}{\pgfqpoint{2.392361in}{2.550670in}}{\pgfqpoint{2.400597in}{2.550670in}}%
\pgfpathclose%
\pgfusepath{stroke,fill}%
\end{pgfscope}%
\begin{pgfscope}%
\pgfpathrectangle{\pgfqpoint{0.100000in}{0.212622in}}{\pgfqpoint{3.696000in}{3.696000in}}%
\pgfusepath{clip}%
\pgfsetbuttcap%
\pgfsetroundjoin%
\definecolor{currentfill}{rgb}{0.121569,0.466667,0.705882}%
\pgfsetfillcolor{currentfill}%
\pgfsetfillopacity{0.832802}%
\pgfsetlinewidth{1.003750pt}%
\definecolor{currentstroke}{rgb}{0.121569,0.466667,0.705882}%
\pgfsetstrokecolor{currentstroke}%
\pgfsetstrokeopacity{0.832802}%
\pgfsetdash{}{0pt}%
\pgfpathmoveto{\pgfqpoint{2.395965in}{2.543860in}}%
\pgfpathcurveto{\pgfqpoint{2.404202in}{2.543860in}}{\pgfqpoint{2.412102in}{2.547132in}}{\pgfqpoint{2.417926in}{2.552956in}}%
\pgfpathcurveto{\pgfqpoint{2.423749in}{2.558780in}}{\pgfqpoint{2.427022in}{2.566680in}}{\pgfqpoint{2.427022in}{2.574916in}}%
\pgfpathcurveto{\pgfqpoint{2.427022in}{2.583152in}}{\pgfqpoint{2.423749in}{2.591052in}}{\pgfqpoint{2.417926in}{2.596876in}}%
\pgfpathcurveto{\pgfqpoint{2.412102in}{2.602700in}}{\pgfqpoint{2.404202in}{2.605973in}}{\pgfqpoint{2.395965in}{2.605973in}}%
\pgfpathcurveto{\pgfqpoint{2.387729in}{2.605973in}}{\pgfqpoint{2.379829in}{2.602700in}}{\pgfqpoint{2.374005in}{2.596876in}}%
\pgfpathcurveto{\pgfqpoint{2.368181in}{2.591052in}}{\pgfqpoint{2.364909in}{2.583152in}}{\pgfqpoint{2.364909in}{2.574916in}}%
\pgfpathcurveto{\pgfqpoint{2.364909in}{2.566680in}}{\pgfqpoint{2.368181in}{2.558780in}}{\pgfqpoint{2.374005in}{2.552956in}}%
\pgfpathcurveto{\pgfqpoint{2.379829in}{2.547132in}}{\pgfqpoint{2.387729in}{2.543860in}}{\pgfqpoint{2.395965in}{2.543860in}}%
\pgfpathclose%
\pgfusepath{stroke,fill}%
\end{pgfscope}%
\begin{pgfscope}%
\pgfpathrectangle{\pgfqpoint{0.100000in}{0.212622in}}{\pgfqpoint{3.696000in}{3.696000in}}%
\pgfusepath{clip}%
\pgfsetbuttcap%
\pgfsetroundjoin%
\definecolor{currentfill}{rgb}{0.121569,0.466667,0.705882}%
\pgfsetfillcolor{currentfill}%
\pgfsetfillopacity{0.832902}%
\pgfsetlinewidth{1.003750pt}%
\definecolor{currentstroke}{rgb}{0.121569,0.466667,0.705882}%
\pgfsetstrokecolor{currentstroke}%
\pgfsetstrokeopacity{0.832902}%
\pgfsetdash{}{0pt}%
\pgfpathmoveto{\pgfqpoint{1.252037in}{1.378049in}}%
\pgfpathcurveto{\pgfqpoint{1.260273in}{1.378049in}}{\pgfqpoint{1.268174in}{1.381321in}}{\pgfqpoint{1.273997in}{1.387145in}}%
\pgfpathcurveto{\pgfqpoint{1.279821in}{1.392969in}}{\pgfqpoint{1.283094in}{1.400869in}}{\pgfqpoint{1.283094in}{1.409106in}}%
\pgfpathcurveto{\pgfqpoint{1.283094in}{1.417342in}}{\pgfqpoint{1.279821in}{1.425242in}}{\pgfqpoint{1.273997in}{1.431066in}}%
\pgfpathcurveto{\pgfqpoint{1.268174in}{1.436890in}}{\pgfqpoint{1.260273in}{1.440162in}}{\pgfqpoint{1.252037in}{1.440162in}}%
\pgfpathcurveto{\pgfqpoint{1.243801in}{1.440162in}}{\pgfqpoint{1.235901in}{1.436890in}}{\pgfqpoint{1.230077in}{1.431066in}}%
\pgfpathcurveto{\pgfqpoint{1.224253in}{1.425242in}}{\pgfqpoint{1.220981in}{1.417342in}}{\pgfqpoint{1.220981in}{1.409106in}}%
\pgfpathcurveto{\pgfqpoint{1.220981in}{1.400869in}}{\pgfqpoint{1.224253in}{1.392969in}}{\pgfqpoint{1.230077in}{1.387145in}}%
\pgfpathcurveto{\pgfqpoint{1.235901in}{1.381321in}}{\pgfqpoint{1.243801in}{1.378049in}}{\pgfqpoint{1.252037in}{1.378049in}}%
\pgfpathclose%
\pgfusepath{stroke,fill}%
\end{pgfscope}%
\begin{pgfscope}%
\pgfpathrectangle{\pgfqpoint{0.100000in}{0.212622in}}{\pgfqpoint{3.696000in}{3.696000in}}%
\pgfusepath{clip}%
\pgfsetbuttcap%
\pgfsetroundjoin%
\definecolor{currentfill}{rgb}{0.121569,0.466667,0.705882}%
\pgfsetfillcolor{currentfill}%
\pgfsetfillopacity{0.833720}%
\pgfsetlinewidth{1.003750pt}%
\definecolor{currentstroke}{rgb}{0.121569,0.466667,0.705882}%
\pgfsetstrokecolor{currentstroke}%
\pgfsetstrokeopacity{0.833720}%
\pgfsetdash{}{0pt}%
\pgfpathmoveto{\pgfqpoint{2.393844in}{2.538066in}}%
\pgfpathcurveto{\pgfqpoint{2.402080in}{2.538066in}}{\pgfqpoint{2.409980in}{2.541338in}}{\pgfqpoint{2.415804in}{2.547162in}}%
\pgfpathcurveto{\pgfqpoint{2.421628in}{2.552986in}}{\pgfqpoint{2.424900in}{2.560886in}}{\pgfqpoint{2.424900in}{2.569122in}}%
\pgfpathcurveto{\pgfqpoint{2.424900in}{2.577358in}}{\pgfqpoint{2.421628in}{2.585258in}}{\pgfqpoint{2.415804in}{2.591082in}}%
\pgfpathcurveto{\pgfqpoint{2.409980in}{2.596906in}}{\pgfqpoint{2.402080in}{2.600179in}}{\pgfqpoint{2.393844in}{2.600179in}}%
\pgfpathcurveto{\pgfqpoint{2.385608in}{2.600179in}}{\pgfqpoint{2.377708in}{2.596906in}}{\pgfqpoint{2.371884in}{2.591082in}}%
\pgfpathcurveto{\pgfqpoint{2.366060in}{2.585258in}}{\pgfqpoint{2.362787in}{2.577358in}}{\pgfqpoint{2.362787in}{2.569122in}}%
\pgfpathcurveto{\pgfqpoint{2.362787in}{2.560886in}}{\pgfqpoint{2.366060in}{2.552986in}}{\pgfqpoint{2.371884in}{2.547162in}}%
\pgfpathcurveto{\pgfqpoint{2.377708in}{2.541338in}}{\pgfqpoint{2.385608in}{2.538066in}}{\pgfqpoint{2.393844in}{2.538066in}}%
\pgfpathclose%
\pgfusepath{stroke,fill}%
\end{pgfscope}%
\begin{pgfscope}%
\pgfpathrectangle{\pgfqpoint{0.100000in}{0.212622in}}{\pgfqpoint{3.696000in}{3.696000in}}%
\pgfusepath{clip}%
\pgfsetbuttcap%
\pgfsetroundjoin%
\definecolor{currentfill}{rgb}{0.121569,0.466667,0.705882}%
\pgfsetfillcolor{currentfill}%
\pgfsetfillopacity{0.834043}%
\pgfsetlinewidth{1.003750pt}%
\definecolor{currentstroke}{rgb}{0.121569,0.466667,0.705882}%
\pgfsetstrokecolor{currentstroke}%
\pgfsetstrokeopacity{0.834043}%
\pgfsetdash{}{0pt}%
\pgfpathmoveto{\pgfqpoint{1.257170in}{1.377772in}}%
\pgfpathcurveto{\pgfqpoint{1.265407in}{1.377772in}}{\pgfqpoint{1.273307in}{1.381044in}}{\pgfqpoint{1.279131in}{1.386868in}}%
\pgfpathcurveto{\pgfqpoint{1.284955in}{1.392692in}}{\pgfqpoint{1.288227in}{1.400592in}}{\pgfqpoint{1.288227in}{1.408828in}}%
\pgfpathcurveto{\pgfqpoint{1.288227in}{1.417065in}}{\pgfqpoint{1.284955in}{1.424965in}}{\pgfqpoint{1.279131in}{1.430789in}}%
\pgfpathcurveto{\pgfqpoint{1.273307in}{1.436612in}}{\pgfqpoint{1.265407in}{1.439885in}}{\pgfqpoint{1.257170in}{1.439885in}}%
\pgfpathcurveto{\pgfqpoint{1.248934in}{1.439885in}}{\pgfqpoint{1.241034in}{1.436612in}}{\pgfqpoint{1.235210in}{1.430789in}}%
\pgfpathcurveto{\pgfqpoint{1.229386in}{1.424965in}}{\pgfqpoint{1.226114in}{1.417065in}}{\pgfqpoint{1.226114in}{1.408828in}}%
\pgfpathcurveto{\pgfqpoint{1.226114in}{1.400592in}}{\pgfqpoint{1.229386in}{1.392692in}}{\pgfqpoint{1.235210in}{1.386868in}}%
\pgfpathcurveto{\pgfqpoint{1.241034in}{1.381044in}}{\pgfqpoint{1.248934in}{1.377772in}}{\pgfqpoint{1.257170in}{1.377772in}}%
\pgfpathclose%
\pgfusepath{stroke,fill}%
\end{pgfscope}%
\begin{pgfscope}%
\pgfpathrectangle{\pgfqpoint{0.100000in}{0.212622in}}{\pgfqpoint{3.696000in}{3.696000in}}%
\pgfusepath{clip}%
\pgfsetbuttcap%
\pgfsetroundjoin%
\definecolor{currentfill}{rgb}{0.121569,0.466667,0.705882}%
\pgfsetfillcolor{currentfill}%
\pgfsetfillopacity{0.834507}%
\pgfsetlinewidth{1.003750pt}%
\definecolor{currentstroke}{rgb}{0.121569,0.466667,0.705882}%
\pgfsetstrokecolor{currentstroke}%
\pgfsetstrokeopacity{0.834507}%
\pgfsetdash{}{0pt}%
\pgfpathmoveto{\pgfqpoint{2.392671in}{2.532319in}}%
\pgfpathcurveto{\pgfqpoint{2.400907in}{2.532319in}}{\pgfqpoint{2.408808in}{2.535591in}}{\pgfqpoint{2.414631in}{2.541415in}}%
\pgfpathcurveto{\pgfqpoint{2.420455in}{2.547239in}}{\pgfqpoint{2.423728in}{2.555139in}}{\pgfqpoint{2.423728in}{2.563376in}}%
\pgfpathcurveto{\pgfqpoint{2.423728in}{2.571612in}}{\pgfqpoint{2.420455in}{2.579512in}}{\pgfqpoint{2.414631in}{2.585336in}}%
\pgfpathcurveto{\pgfqpoint{2.408808in}{2.591160in}}{\pgfqpoint{2.400907in}{2.594432in}}{\pgfqpoint{2.392671in}{2.594432in}}%
\pgfpathcurveto{\pgfqpoint{2.384435in}{2.594432in}}{\pgfqpoint{2.376535in}{2.591160in}}{\pgfqpoint{2.370711in}{2.585336in}}%
\pgfpathcurveto{\pgfqpoint{2.364887in}{2.579512in}}{\pgfqpoint{2.361615in}{2.571612in}}{\pgfqpoint{2.361615in}{2.563376in}}%
\pgfpathcurveto{\pgfqpoint{2.361615in}{2.555139in}}{\pgfqpoint{2.364887in}{2.547239in}}{\pgfqpoint{2.370711in}{2.541415in}}%
\pgfpathcurveto{\pgfqpoint{2.376535in}{2.535591in}}{\pgfqpoint{2.384435in}{2.532319in}}{\pgfqpoint{2.392671in}{2.532319in}}%
\pgfpathclose%
\pgfusepath{stroke,fill}%
\end{pgfscope}%
\begin{pgfscope}%
\pgfpathrectangle{\pgfqpoint{0.100000in}{0.212622in}}{\pgfqpoint{3.696000in}{3.696000in}}%
\pgfusepath{clip}%
\pgfsetbuttcap%
\pgfsetroundjoin%
\definecolor{currentfill}{rgb}{0.121569,0.466667,0.705882}%
\pgfsetfillcolor{currentfill}%
\pgfsetfillopacity{0.834879}%
\pgfsetlinewidth{1.003750pt}%
\definecolor{currentstroke}{rgb}{0.121569,0.466667,0.705882}%
\pgfsetstrokecolor{currentstroke}%
\pgfsetstrokeopacity{0.834879}%
\pgfsetdash{}{0pt}%
\pgfpathmoveto{\pgfqpoint{2.950294in}{1.740329in}}%
\pgfpathcurveto{\pgfqpoint{2.958530in}{1.740329in}}{\pgfqpoint{2.966430in}{1.743602in}}{\pgfqpoint{2.972254in}{1.749425in}}%
\pgfpathcurveto{\pgfqpoint{2.978078in}{1.755249in}}{\pgfqpoint{2.981350in}{1.763149in}}{\pgfqpoint{2.981350in}{1.771386in}}%
\pgfpathcurveto{\pgfqpoint{2.981350in}{1.779622in}}{\pgfqpoint{2.978078in}{1.787522in}}{\pgfqpoint{2.972254in}{1.793346in}}%
\pgfpathcurveto{\pgfqpoint{2.966430in}{1.799170in}}{\pgfqpoint{2.958530in}{1.802442in}}{\pgfqpoint{2.950294in}{1.802442in}}%
\pgfpathcurveto{\pgfqpoint{2.942057in}{1.802442in}}{\pgfqpoint{2.934157in}{1.799170in}}{\pgfqpoint{2.928333in}{1.793346in}}%
\pgfpathcurveto{\pgfqpoint{2.922509in}{1.787522in}}{\pgfqpoint{2.919237in}{1.779622in}}{\pgfqpoint{2.919237in}{1.771386in}}%
\pgfpathcurveto{\pgfqpoint{2.919237in}{1.763149in}}{\pgfqpoint{2.922509in}{1.755249in}}{\pgfqpoint{2.928333in}{1.749425in}}%
\pgfpathcurveto{\pgfqpoint{2.934157in}{1.743602in}}{\pgfqpoint{2.942057in}{1.740329in}}{\pgfqpoint{2.950294in}{1.740329in}}%
\pgfpathclose%
\pgfusepath{stroke,fill}%
\end{pgfscope}%
\begin{pgfscope}%
\pgfpathrectangle{\pgfqpoint{0.100000in}{0.212622in}}{\pgfqpoint{3.696000in}{3.696000in}}%
\pgfusepath{clip}%
\pgfsetbuttcap%
\pgfsetroundjoin%
\definecolor{currentfill}{rgb}{0.121569,0.466667,0.705882}%
\pgfsetfillcolor{currentfill}%
\pgfsetfillopacity{0.834944}%
\pgfsetlinewidth{1.003750pt}%
\definecolor{currentstroke}{rgb}{0.121569,0.466667,0.705882}%
\pgfsetstrokecolor{currentstroke}%
\pgfsetstrokeopacity{0.834944}%
\pgfsetdash{}{0pt}%
\pgfpathmoveto{\pgfqpoint{2.390709in}{2.529253in}}%
\pgfpathcurveto{\pgfqpoint{2.398945in}{2.529253in}}{\pgfqpoint{2.406845in}{2.532526in}}{\pgfqpoint{2.412669in}{2.538350in}}%
\pgfpathcurveto{\pgfqpoint{2.418493in}{2.544174in}}{\pgfqpoint{2.421765in}{2.552074in}}{\pgfqpoint{2.421765in}{2.560310in}}%
\pgfpathcurveto{\pgfqpoint{2.421765in}{2.568546in}}{\pgfqpoint{2.418493in}{2.576446in}}{\pgfqpoint{2.412669in}{2.582270in}}%
\pgfpathcurveto{\pgfqpoint{2.406845in}{2.588094in}}{\pgfqpoint{2.398945in}{2.591366in}}{\pgfqpoint{2.390709in}{2.591366in}}%
\pgfpathcurveto{\pgfqpoint{2.382472in}{2.591366in}}{\pgfqpoint{2.374572in}{2.588094in}}{\pgfqpoint{2.368749in}{2.582270in}}%
\pgfpathcurveto{\pgfqpoint{2.362925in}{2.576446in}}{\pgfqpoint{2.359652in}{2.568546in}}{\pgfqpoint{2.359652in}{2.560310in}}%
\pgfpathcurveto{\pgfqpoint{2.359652in}{2.552074in}}{\pgfqpoint{2.362925in}{2.544174in}}{\pgfqpoint{2.368749in}{2.538350in}}%
\pgfpathcurveto{\pgfqpoint{2.374572in}{2.532526in}}{\pgfqpoint{2.382472in}{2.529253in}}{\pgfqpoint{2.390709in}{2.529253in}}%
\pgfpathclose%
\pgfusepath{stroke,fill}%
\end{pgfscope}%
\begin{pgfscope}%
\pgfpathrectangle{\pgfqpoint{0.100000in}{0.212622in}}{\pgfqpoint{3.696000in}{3.696000in}}%
\pgfusepath{clip}%
\pgfsetbuttcap%
\pgfsetroundjoin%
\definecolor{currentfill}{rgb}{0.121569,0.466667,0.705882}%
\pgfsetfillcolor{currentfill}%
\pgfsetfillopacity{0.835066}%
\pgfsetlinewidth{1.003750pt}%
\definecolor{currentstroke}{rgb}{0.121569,0.466667,0.705882}%
\pgfsetstrokecolor{currentstroke}%
\pgfsetstrokeopacity{0.835066}%
\pgfsetdash{}{0pt}%
\pgfpathmoveto{\pgfqpoint{1.262873in}{1.374895in}}%
\pgfpathcurveto{\pgfqpoint{1.271109in}{1.374895in}}{\pgfqpoint{1.279009in}{1.378167in}}{\pgfqpoint{1.284833in}{1.383991in}}%
\pgfpathcurveto{\pgfqpoint{1.290657in}{1.389815in}}{\pgfqpoint{1.293930in}{1.397715in}}{\pgfqpoint{1.293930in}{1.405952in}}%
\pgfpathcurveto{\pgfqpoint{1.293930in}{1.414188in}}{\pgfqpoint{1.290657in}{1.422088in}}{\pgfqpoint{1.284833in}{1.427912in}}%
\pgfpathcurveto{\pgfqpoint{1.279009in}{1.433736in}}{\pgfqpoint{1.271109in}{1.437008in}}{\pgfqpoint{1.262873in}{1.437008in}}%
\pgfpathcurveto{\pgfqpoint{1.254637in}{1.437008in}}{\pgfqpoint{1.246737in}{1.433736in}}{\pgfqpoint{1.240913in}{1.427912in}}%
\pgfpathcurveto{\pgfqpoint{1.235089in}{1.422088in}}{\pgfqpoint{1.231817in}{1.414188in}}{\pgfqpoint{1.231817in}{1.405952in}}%
\pgfpathcurveto{\pgfqpoint{1.231817in}{1.397715in}}{\pgfqpoint{1.235089in}{1.389815in}}{\pgfqpoint{1.240913in}{1.383991in}}%
\pgfpathcurveto{\pgfqpoint{1.246737in}{1.378167in}}{\pgfqpoint{1.254637in}{1.374895in}}{\pgfqpoint{1.262873in}{1.374895in}}%
\pgfpathclose%
\pgfusepath{stroke,fill}%
\end{pgfscope}%
\begin{pgfscope}%
\pgfpathrectangle{\pgfqpoint{0.100000in}{0.212622in}}{\pgfqpoint{3.696000in}{3.696000in}}%
\pgfusepath{clip}%
\pgfsetbuttcap%
\pgfsetroundjoin%
\definecolor{currentfill}{rgb}{0.121569,0.466667,0.705882}%
\pgfsetfillcolor{currentfill}%
\pgfsetfillopacity{0.835664}%
\pgfsetlinewidth{1.003750pt}%
\definecolor{currentstroke}{rgb}{0.121569,0.466667,0.705882}%
\pgfsetstrokecolor{currentstroke}%
\pgfsetstrokeopacity{0.835664}%
\pgfsetdash{}{0pt}%
\pgfpathmoveto{\pgfqpoint{2.386975in}{2.523700in}}%
\pgfpathcurveto{\pgfqpoint{2.395211in}{2.523700in}}{\pgfqpoint{2.403111in}{2.526972in}}{\pgfqpoint{2.408935in}{2.532796in}}%
\pgfpathcurveto{\pgfqpoint{2.414759in}{2.538620in}}{\pgfqpoint{2.418031in}{2.546520in}}{\pgfqpoint{2.418031in}{2.554756in}}%
\pgfpathcurveto{\pgfqpoint{2.418031in}{2.562992in}}{\pgfqpoint{2.414759in}{2.570892in}}{\pgfqpoint{2.408935in}{2.576716in}}%
\pgfpathcurveto{\pgfqpoint{2.403111in}{2.582540in}}{\pgfqpoint{2.395211in}{2.585813in}}{\pgfqpoint{2.386975in}{2.585813in}}%
\pgfpathcurveto{\pgfqpoint{2.378738in}{2.585813in}}{\pgfqpoint{2.370838in}{2.582540in}}{\pgfqpoint{2.365014in}{2.576716in}}%
\pgfpathcurveto{\pgfqpoint{2.359190in}{2.570892in}}{\pgfqpoint{2.355918in}{2.562992in}}{\pgfqpoint{2.355918in}{2.554756in}}%
\pgfpathcurveto{\pgfqpoint{2.355918in}{2.546520in}}{\pgfqpoint{2.359190in}{2.538620in}}{\pgfqpoint{2.365014in}{2.532796in}}%
\pgfpathcurveto{\pgfqpoint{2.370838in}{2.526972in}}{\pgfqpoint{2.378738in}{2.523700in}}{\pgfqpoint{2.386975in}{2.523700in}}%
\pgfpathclose%
\pgfusepath{stroke,fill}%
\end{pgfscope}%
\begin{pgfscope}%
\pgfpathrectangle{\pgfqpoint{0.100000in}{0.212622in}}{\pgfqpoint{3.696000in}{3.696000in}}%
\pgfusepath{clip}%
\pgfsetbuttcap%
\pgfsetroundjoin%
\definecolor{currentfill}{rgb}{0.121569,0.466667,0.705882}%
\pgfsetfillcolor{currentfill}%
\pgfsetfillopacity{0.836206}%
\pgfsetlinewidth{1.003750pt}%
\definecolor{currentstroke}{rgb}{0.121569,0.466667,0.705882}%
\pgfsetstrokecolor{currentstroke}%
\pgfsetstrokeopacity{0.836206}%
\pgfsetdash{}{0pt}%
\pgfpathmoveto{\pgfqpoint{1.269404in}{1.370664in}}%
\pgfpathcurveto{\pgfqpoint{1.277640in}{1.370664in}}{\pgfqpoint{1.285540in}{1.373936in}}{\pgfqpoint{1.291364in}{1.379760in}}%
\pgfpathcurveto{\pgfqpoint{1.297188in}{1.385584in}}{\pgfqpoint{1.300460in}{1.393484in}}{\pgfqpoint{1.300460in}{1.401720in}}%
\pgfpathcurveto{\pgfqpoint{1.300460in}{1.409957in}}{\pgfqpoint{1.297188in}{1.417857in}}{\pgfqpoint{1.291364in}{1.423681in}}%
\pgfpathcurveto{\pgfqpoint{1.285540in}{1.429505in}}{\pgfqpoint{1.277640in}{1.432777in}}{\pgfqpoint{1.269404in}{1.432777in}}%
\pgfpathcurveto{\pgfqpoint{1.261167in}{1.432777in}}{\pgfqpoint{1.253267in}{1.429505in}}{\pgfqpoint{1.247443in}{1.423681in}}%
\pgfpathcurveto{\pgfqpoint{1.241619in}{1.417857in}}{\pgfqpoint{1.238347in}{1.409957in}}{\pgfqpoint{1.238347in}{1.401720in}}%
\pgfpathcurveto{\pgfqpoint{1.238347in}{1.393484in}}{\pgfqpoint{1.241619in}{1.385584in}}{\pgfqpoint{1.247443in}{1.379760in}}%
\pgfpathcurveto{\pgfqpoint{1.253267in}{1.373936in}}{\pgfqpoint{1.261167in}{1.370664in}}{\pgfqpoint{1.269404in}{1.370664in}}%
\pgfpathclose%
\pgfusepath{stroke,fill}%
\end{pgfscope}%
\begin{pgfscope}%
\pgfpathrectangle{\pgfqpoint{0.100000in}{0.212622in}}{\pgfqpoint{3.696000in}{3.696000in}}%
\pgfusepath{clip}%
\pgfsetbuttcap%
\pgfsetroundjoin%
\definecolor{currentfill}{rgb}{0.121569,0.466667,0.705882}%
\pgfsetfillcolor{currentfill}%
\pgfsetfillopacity{0.836501}%
\pgfsetlinewidth{1.003750pt}%
\definecolor{currentstroke}{rgb}{0.121569,0.466667,0.705882}%
\pgfsetstrokecolor{currentstroke}%
\pgfsetstrokeopacity{0.836501}%
\pgfsetdash{}{0pt}%
\pgfpathmoveto{\pgfqpoint{2.385351in}{2.518873in}}%
\pgfpathcurveto{\pgfqpoint{2.393587in}{2.518873in}}{\pgfqpoint{2.401487in}{2.522146in}}{\pgfqpoint{2.407311in}{2.527969in}}%
\pgfpathcurveto{\pgfqpoint{2.413135in}{2.533793in}}{\pgfqpoint{2.416408in}{2.541693in}}{\pgfqpoint{2.416408in}{2.549930in}}%
\pgfpathcurveto{\pgfqpoint{2.416408in}{2.558166in}}{\pgfqpoint{2.413135in}{2.566066in}}{\pgfqpoint{2.407311in}{2.571890in}}%
\pgfpathcurveto{\pgfqpoint{2.401487in}{2.577714in}}{\pgfqpoint{2.393587in}{2.580986in}}{\pgfqpoint{2.385351in}{2.580986in}}%
\pgfpathcurveto{\pgfqpoint{2.377115in}{2.580986in}}{\pgfqpoint{2.369215in}{2.577714in}}{\pgfqpoint{2.363391in}{2.571890in}}%
\pgfpathcurveto{\pgfqpoint{2.357567in}{2.566066in}}{\pgfqpoint{2.354295in}{2.558166in}}{\pgfqpoint{2.354295in}{2.549930in}}%
\pgfpathcurveto{\pgfqpoint{2.354295in}{2.541693in}}{\pgfqpoint{2.357567in}{2.533793in}}{\pgfqpoint{2.363391in}{2.527969in}}%
\pgfpathcurveto{\pgfqpoint{2.369215in}{2.522146in}}{\pgfqpoint{2.377115in}{2.518873in}}{\pgfqpoint{2.385351in}{2.518873in}}%
\pgfpathclose%
\pgfusepath{stroke,fill}%
\end{pgfscope}%
\begin{pgfscope}%
\pgfpathrectangle{\pgfqpoint{0.100000in}{0.212622in}}{\pgfqpoint{3.696000in}{3.696000in}}%
\pgfusepath{clip}%
\pgfsetbuttcap%
\pgfsetroundjoin%
\definecolor{currentfill}{rgb}{0.121569,0.466667,0.705882}%
\pgfsetfillcolor{currentfill}%
\pgfsetfillopacity{0.837581}%
\pgfsetlinewidth{1.003750pt}%
\definecolor{currentstroke}{rgb}{0.121569,0.466667,0.705882}%
\pgfsetstrokecolor{currentstroke}%
\pgfsetstrokeopacity{0.837581}%
\pgfsetdash{}{0pt}%
\pgfpathmoveto{\pgfqpoint{1.278243in}{1.365520in}}%
\pgfpathcurveto{\pgfqpoint{1.286479in}{1.365520in}}{\pgfqpoint{1.294379in}{1.368792in}}{\pgfqpoint{1.300203in}{1.374616in}}%
\pgfpathcurveto{\pgfqpoint{1.306027in}{1.380440in}}{\pgfqpoint{1.309299in}{1.388340in}}{\pgfqpoint{1.309299in}{1.396576in}}%
\pgfpathcurveto{\pgfqpoint{1.309299in}{1.404813in}}{\pgfqpoint{1.306027in}{1.412713in}}{\pgfqpoint{1.300203in}{1.418537in}}%
\pgfpathcurveto{\pgfqpoint{1.294379in}{1.424361in}}{\pgfqpoint{1.286479in}{1.427633in}}{\pgfqpoint{1.278243in}{1.427633in}}%
\pgfpathcurveto{\pgfqpoint{1.270006in}{1.427633in}}{\pgfqpoint{1.262106in}{1.424361in}}{\pgfqpoint{1.256282in}{1.418537in}}%
\pgfpathcurveto{\pgfqpoint{1.250458in}{1.412713in}}{\pgfqpoint{1.247186in}{1.404813in}}{\pgfqpoint{1.247186in}{1.396576in}}%
\pgfpathcurveto{\pgfqpoint{1.247186in}{1.388340in}}{\pgfqpoint{1.250458in}{1.380440in}}{\pgfqpoint{1.256282in}{1.374616in}}%
\pgfpathcurveto{\pgfqpoint{1.262106in}{1.368792in}}{\pgfqpoint{1.270006in}{1.365520in}}{\pgfqpoint{1.278243in}{1.365520in}}%
\pgfpathclose%
\pgfusepath{stroke,fill}%
\end{pgfscope}%
\begin{pgfscope}%
\pgfpathrectangle{\pgfqpoint{0.100000in}{0.212622in}}{\pgfqpoint{3.696000in}{3.696000in}}%
\pgfusepath{clip}%
\pgfsetbuttcap%
\pgfsetroundjoin%
\definecolor{currentfill}{rgb}{0.121569,0.466667,0.705882}%
\pgfsetfillcolor{currentfill}%
\pgfsetfillopacity{0.837941}%
\pgfsetlinewidth{1.003750pt}%
\definecolor{currentstroke}{rgb}{0.121569,0.466667,0.705882}%
\pgfsetstrokecolor{currentstroke}%
\pgfsetstrokeopacity{0.837941}%
\pgfsetdash{}{0pt}%
\pgfpathmoveto{\pgfqpoint{2.383269in}{2.509477in}}%
\pgfpathcurveto{\pgfqpoint{2.391505in}{2.509477in}}{\pgfqpoint{2.399405in}{2.512749in}}{\pgfqpoint{2.405229in}{2.518573in}}%
\pgfpathcurveto{\pgfqpoint{2.411053in}{2.524397in}}{\pgfqpoint{2.414325in}{2.532297in}}{\pgfqpoint{2.414325in}{2.540533in}}%
\pgfpathcurveto{\pgfqpoint{2.414325in}{2.548769in}}{\pgfqpoint{2.411053in}{2.556669in}}{\pgfqpoint{2.405229in}{2.562493in}}%
\pgfpathcurveto{\pgfqpoint{2.399405in}{2.568317in}}{\pgfqpoint{2.391505in}{2.571590in}}{\pgfqpoint{2.383269in}{2.571590in}}%
\pgfpathcurveto{\pgfqpoint{2.375033in}{2.571590in}}{\pgfqpoint{2.367133in}{2.568317in}}{\pgfqpoint{2.361309in}{2.562493in}}%
\pgfpathcurveto{\pgfqpoint{2.355485in}{2.556669in}}{\pgfqpoint{2.352212in}{2.548769in}}{\pgfqpoint{2.352212in}{2.540533in}}%
\pgfpathcurveto{\pgfqpoint{2.352212in}{2.532297in}}{\pgfqpoint{2.355485in}{2.524397in}}{\pgfqpoint{2.361309in}{2.518573in}}%
\pgfpathcurveto{\pgfqpoint{2.367133in}{2.512749in}}{\pgfqpoint{2.375033in}{2.509477in}}{\pgfqpoint{2.383269in}{2.509477in}}%
\pgfpathclose%
\pgfusepath{stroke,fill}%
\end{pgfscope}%
\begin{pgfscope}%
\pgfpathrectangle{\pgfqpoint{0.100000in}{0.212622in}}{\pgfqpoint{3.696000in}{3.696000in}}%
\pgfusepath{clip}%
\pgfsetbuttcap%
\pgfsetroundjoin%
\definecolor{currentfill}{rgb}{0.121569,0.466667,0.705882}%
\pgfsetfillcolor{currentfill}%
\pgfsetfillopacity{0.838847}%
\pgfsetlinewidth{1.003750pt}%
\definecolor{currentstroke}{rgb}{0.121569,0.466667,0.705882}%
\pgfsetstrokecolor{currentstroke}%
\pgfsetstrokeopacity{0.838847}%
\pgfsetdash{}{0pt}%
\pgfpathmoveto{\pgfqpoint{2.379963in}{2.504284in}}%
\pgfpathcurveto{\pgfqpoint{2.388200in}{2.504284in}}{\pgfqpoint{2.396100in}{2.507557in}}{\pgfqpoint{2.401924in}{2.513380in}}%
\pgfpathcurveto{\pgfqpoint{2.407748in}{2.519204in}}{\pgfqpoint{2.411020in}{2.527104in}}{\pgfqpoint{2.411020in}{2.535341in}}%
\pgfpathcurveto{\pgfqpoint{2.411020in}{2.543577in}}{\pgfqpoint{2.407748in}{2.551477in}}{\pgfqpoint{2.401924in}{2.557301in}}%
\pgfpathcurveto{\pgfqpoint{2.396100in}{2.563125in}}{\pgfqpoint{2.388200in}{2.566397in}}{\pgfqpoint{2.379963in}{2.566397in}}%
\pgfpathcurveto{\pgfqpoint{2.371727in}{2.566397in}}{\pgfqpoint{2.363827in}{2.563125in}}{\pgfqpoint{2.358003in}{2.557301in}}%
\pgfpathcurveto{\pgfqpoint{2.352179in}{2.551477in}}{\pgfqpoint{2.348907in}{2.543577in}}{\pgfqpoint{2.348907in}{2.535341in}}%
\pgfpathcurveto{\pgfqpoint{2.348907in}{2.527104in}}{\pgfqpoint{2.352179in}{2.519204in}}{\pgfqpoint{2.358003in}{2.513380in}}%
\pgfpathcurveto{\pgfqpoint{2.363827in}{2.507557in}}{\pgfqpoint{2.371727in}{2.504284in}}{\pgfqpoint{2.379963in}{2.504284in}}%
\pgfpathclose%
\pgfusepath{stroke,fill}%
\end{pgfscope}%
\begin{pgfscope}%
\pgfpathrectangle{\pgfqpoint{0.100000in}{0.212622in}}{\pgfqpoint{3.696000in}{3.696000in}}%
\pgfusepath{clip}%
\pgfsetbuttcap%
\pgfsetroundjoin%
\definecolor{currentfill}{rgb}{0.121569,0.466667,0.705882}%
\pgfsetfillcolor{currentfill}%
\pgfsetfillopacity{0.839157}%
\pgfsetlinewidth{1.003750pt}%
\definecolor{currentstroke}{rgb}{0.121569,0.466667,0.705882}%
\pgfsetstrokecolor{currentstroke}%
\pgfsetstrokeopacity{0.839157}%
\pgfsetdash{}{0pt}%
\pgfpathmoveto{\pgfqpoint{1.288430in}{1.361517in}}%
\pgfpathcurveto{\pgfqpoint{1.296667in}{1.361517in}}{\pgfqpoint{1.304567in}{1.364790in}}{\pgfqpoint{1.310391in}{1.370614in}}%
\pgfpathcurveto{\pgfqpoint{1.316215in}{1.376438in}}{\pgfqpoint{1.319487in}{1.384338in}}{\pgfqpoint{1.319487in}{1.392574in}}%
\pgfpathcurveto{\pgfqpoint{1.319487in}{1.400810in}}{\pgfqpoint{1.316215in}{1.408710in}}{\pgfqpoint{1.310391in}{1.414534in}}%
\pgfpathcurveto{\pgfqpoint{1.304567in}{1.420358in}}{\pgfqpoint{1.296667in}{1.423630in}}{\pgfqpoint{1.288430in}{1.423630in}}%
\pgfpathcurveto{\pgfqpoint{1.280194in}{1.423630in}}{\pgfqpoint{1.272294in}{1.420358in}}{\pgfqpoint{1.266470in}{1.414534in}}%
\pgfpathcurveto{\pgfqpoint{1.260646in}{1.408710in}}{\pgfqpoint{1.257374in}{1.400810in}}{\pgfqpoint{1.257374in}{1.392574in}}%
\pgfpathcurveto{\pgfqpoint{1.257374in}{1.384338in}}{\pgfqpoint{1.260646in}{1.376438in}}{\pgfqpoint{1.266470in}{1.370614in}}%
\pgfpathcurveto{\pgfqpoint{1.272294in}{1.364790in}}{\pgfqpoint{1.280194in}{1.361517in}}{\pgfqpoint{1.288430in}{1.361517in}}%
\pgfpathclose%
\pgfusepath{stroke,fill}%
\end{pgfscope}%
\begin{pgfscope}%
\pgfpathrectangle{\pgfqpoint{0.100000in}{0.212622in}}{\pgfqpoint{3.696000in}{3.696000in}}%
\pgfusepath{clip}%
\pgfsetbuttcap%
\pgfsetroundjoin%
\definecolor{currentfill}{rgb}{0.121569,0.466667,0.705882}%
\pgfsetfillcolor{currentfill}%
\pgfsetfillopacity{0.839420}%
\pgfsetlinewidth{1.003750pt}%
\definecolor{currentstroke}{rgb}{0.121569,0.466667,0.705882}%
\pgfsetstrokecolor{currentstroke}%
\pgfsetstrokeopacity{0.839420}%
\pgfsetdash{}{0pt}%
\pgfpathmoveto{\pgfqpoint{2.377383in}{2.500229in}}%
\pgfpathcurveto{\pgfqpoint{2.385619in}{2.500229in}}{\pgfqpoint{2.393519in}{2.503501in}}{\pgfqpoint{2.399343in}{2.509325in}}%
\pgfpathcurveto{\pgfqpoint{2.405167in}{2.515149in}}{\pgfqpoint{2.408439in}{2.523049in}}{\pgfqpoint{2.408439in}{2.531285in}}%
\pgfpathcurveto{\pgfqpoint{2.408439in}{2.539522in}}{\pgfqpoint{2.405167in}{2.547422in}}{\pgfqpoint{2.399343in}{2.553246in}}%
\pgfpathcurveto{\pgfqpoint{2.393519in}{2.559070in}}{\pgfqpoint{2.385619in}{2.562342in}}{\pgfqpoint{2.377383in}{2.562342in}}%
\pgfpathcurveto{\pgfqpoint{2.369147in}{2.562342in}}{\pgfqpoint{2.361247in}{2.559070in}}{\pgfqpoint{2.355423in}{2.553246in}}%
\pgfpathcurveto{\pgfqpoint{2.349599in}{2.547422in}}{\pgfqpoint{2.346326in}{2.539522in}}{\pgfqpoint{2.346326in}{2.531285in}}%
\pgfpathcurveto{\pgfqpoint{2.346326in}{2.523049in}}{\pgfqpoint{2.349599in}{2.515149in}}{\pgfqpoint{2.355423in}{2.509325in}}%
\pgfpathcurveto{\pgfqpoint{2.361247in}{2.503501in}}{\pgfqpoint{2.369147in}{2.500229in}}{\pgfqpoint{2.377383in}{2.500229in}}%
\pgfpathclose%
\pgfusepath{stroke,fill}%
\end{pgfscope}%
\begin{pgfscope}%
\pgfpathrectangle{\pgfqpoint{0.100000in}{0.212622in}}{\pgfqpoint{3.696000in}{3.696000in}}%
\pgfusepath{clip}%
\pgfsetbuttcap%
\pgfsetroundjoin%
\definecolor{currentfill}{rgb}{0.121569,0.466667,0.705882}%
\pgfsetfillcolor{currentfill}%
\pgfsetfillopacity{0.839611}%
\pgfsetlinewidth{1.003750pt}%
\definecolor{currentstroke}{rgb}{0.121569,0.466667,0.705882}%
\pgfsetstrokecolor{currentstroke}%
\pgfsetstrokeopacity{0.839611}%
\pgfsetdash{}{0pt}%
\pgfpathmoveto{\pgfqpoint{2.943565in}{1.716105in}}%
\pgfpathcurveto{\pgfqpoint{2.951802in}{1.716105in}}{\pgfqpoint{2.959702in}{1.719378in}}{\pgfqpoint{2.965526in}{1.725202in}}%
\pgfpathcurveto{\pgfqpoint{2.971350in}{1.731025in}}{\pgfqpoint{2.974622in}{1.738926in}}{\pgfqpoint{2.974622in}{1.747162in}}%
\pgfpathcurveto{\pgfqpoint{2.974622in}{1.755398in}}{\pgfqpoint{2.971350in}{1.763298in}}{\pgfqpoint{2.965526in}{1.769122in}}%
\pgfpathcurveto{\pgfqpoint{2.959702in}{1.774946in}}{\pgfqpoint{2.951802in}{1.778218in}}{\pgfqpoint{2.943565in}{1.778218in}}%
\pgfpathcurveto{\pgfqpoint{2.935329in}{1.778218in}}{\pgfqpoint{2.927429in}{1.774946in}}{\pgfqpoint{2.921605in}{1.769122in}}%
\pgfpathcurveto{\pgfqpoint{2.915781in}{1.763298in}}{\pgfqpoint{2.912509in}{1.755398in}}{\pgfqpoint{2.912509in}{1.747162in}}%
\pgfpathcurveto{\pgfqpoint{2.912509in}{1.738926in}}{\pgfqpoint{2.915781in}{1.731025in}}{\pgfqpoint{2.921605in}{1.725202in}}%
\pgfpathcurveto{\pgfqpoint{2.927429in}{1.719378in}}{\pgfqpoint{2.935329in}{1.716105in}}{\pgfqpoint{2.943565in}{1.716105in}}%
\pgfpathclose%
\pgfusepath{stroke,fill}%
\end{pgfscope}%
\begin{pgfscope}%
\pgfpathrectangle{\pgfqpoint{0.100000in}{0.212622in}}{\pgfqpoint{3.696000in}{3.696000in}}%
\pgfusepath{clip}%
\pgfsetbuttcap%
\pgfsetroundjoin%
\definecolor{currentfill}{rgb}{0.121569,0.466667,0.705882}%
\pgfsetfillcolor{currentfill}%
\pgfsetfillopacity{0.839848}%
\pgfsetlinewidth{1.003750pt}%
\definecolor{currentstroke}{rgb}{0.121569,0.466667,0.705882}%
\pgfsetstrokecolor{currentstroke}%
\pgfsetstrokeopacity{0.839848}%
\pgfsetdash{}{0pt}%
\pgfpathmoveto{\pgfqpoint{2.375365in}{2.497362in}}%
\pgfpathcurveto{\pgfqpoint{2.383601in}{2.497362in}}{\pgfqpoint{2.391501in}{2.500634in}}{\pgfqpoint{2.397325in}{2.506458in}}%
\pgfpathcurveto{\pgfqpoint{2.403149in}{2.512282in}}{\pgfqpoint{2.406421in}{2.520182in}}{\pgfqpoint{2.406421in}{2.528418in}}%
\pgfpathcurveto{\pgfqpoint{2.406421in}{2.536654in}}{\pgfqpoint{2.403149in}{2.544554in}}{\pgfqpoint{2.397325in}{2.550378in}}%
\pgfpathcurveto{\pgfqpoint{2.391501in}{2.556202in}}{\pgfqpoint{2.383601in}{2.559475in}}{\pgfqpoint{2.375365in}{2.559475in}}%
\pgfpathcurveto{\pgfqpoint{2.367128in}{2.559475in}}{\pgfqpoint{2.359228in}{2.556202in}}{\pgfqpoint{2.353404in}{2.550378in}}%
\pgfpathcurveto{\pgfqpoint{2.347580in}{2.544554in}}{\pgfqpoint{2.344308in}{2.536654in}}{\pgfqpoint{2.344308in}{2.528418in}}%
\pgfpathcurveto{\pgfqpoint{2.344308in}{2.520182in}}{\pgfqpoint{2.347580in}{2.512282in}}{\pgfqpoint{2.353404in}{2.506458in}}%
\pgfpathcurveto{\pgfqpoint{2.359228in}{2.500634in}}{\pgfqpoint{2.367128in}{2.497362in}}{\pgfqpoint{2.375365in}{2.497362in}}%
\pgfpathclose%
\pgfusepath{stroke,fill}%
\end{pgfscope}%
\begin{pgfscope}%
\pgfpathrectangle{\pgfqpoint{0.100000in}{0.212622in}}{\pgfqpoint{3.696000in}{3.696000in}}%
\pgfusepath{clip}%
\pgfsetbuttcap%
\pgfsetroundjoin%
\definecolor{currentfill}{rgb}{0.121569,0.466667,0.705882}%
\pgfsetfillcolor{currentfill}%
\pgfsetfillopacity{0.839888}%
\pgfsetlinewidth{1.003750pt}%
\definecolor{currentstroke}{rgb}{0.121569,0.466667,0.705882}%
\pgfsetstrokecolor{currentstroke}%
\pgfsetstrokeopacity{0.839888}%
\pgfsetdash{}{0pt}%
\pgfpathmoveto{\pgfqpoint{1.843258in}{2.294534in}}%
\pgfpathcurveto{\pgfqpoint{1.851495in}{2.294534in}}{\pgfqpoint{1.859395in}{2.297807in}}{\pgfqpoint{1.865218in}{2.303631in}}%
\pgfpathcurveto{\pgfqpoint{1.871042in}{2.309455in}}{\pgfqpoint{1.874315in}{2.317355in}}{\pgfqpoint{1.874315in}{2.325591in}}%
\pgfpathcurveto{\pgfqpoint{1.874315in}{2.333827in}}{\pgfqpoint{1.871042in}{2.341727in}}{\pgfqpoint{1.865218in}{2.347551in}}%
\pgfpathcurveto{\pgfqpoint{1.859395in}{2.353375in}}{\pgfqpoint{1.851495in}{2.356647in}}{\pgfqpoint{1.843258in}{2.356647in}}%
\pgfpathcurveto{\pgfqpoint{1.835022in}{2.356647in}}{\pgfqpoint{1.827122in}{2.353375in}}{\pgfqpoint{1.821298in}{2.347551in}}%
\pgfpathcurveto{\pgfqpoint{1.815474in}{2.341727in}}{\pgfqpoint{1.812202in}{2.333827in}}{\pgfqpoint{1.812202in}{2.325591in}}%
\pgfpathcurveto{\pgfqpoint{1.812202in}{2.317355in}}{\pgfqpoint{1.815474in}{2.309455in}}{\pgfqpoint{1.821298in}{2.303631in}}%
\pgfpathcurveto{\pgfqpoint{1.827122in}{2.297807in}}{\pgfqpoint{1.835022in}{2.294534in}}{\pgfqpoint{1.843258in}{2.294534in}}%
\pgfpathclose%
\pgfusepath{stroke,fill}%
\end{pgfscope}%
\begin{pgfscope}%
\pgfpathrectangle{\pgfqpoint{0.100000in}{0.212622in}}{\pgfqpoint{3.696000in}{3.696000in}}%
\pgfusepath{clip}%
\pgfsetbuttcap%
\pgfsetroundjoin%
\definecolor{currentfill}{rgb}{0.121569,0.466667,0.705882}%
\pgfsetfillcolor{currentfill}%
\pgfsetfillopacity{0.839888}%
\pgfsetlinewidth{1.003750pt}%
\definecolor{currentstroke}{rgb}{0.121569,0.466667,0.705882}%
\pgfsetstrokecolor{currentstroke}%
\pgfsetstrokeopacity{0.839888}%
\pgfsetdash{}{0pt}%
\pgfpathmoveto{\pgfqpoint{1.843258in}{2.294534in}}%
\pgfpathcurveto{\pgfqpoint{1.851495in}{2.294534in}}{\pgfqpoint{1.859395in}{2.297807in}}{\pgfqpoint{1.865218in}{2.303631in}}%
\pgfpathcurveto{\pgfqpoint{1.871042in}{2.309455in}}{\pgfqpoint{1.874315in}{2.317355in}}{\pgfqpoint{1.874315in}{2.325591in}}%
\pgfpathcurveto{\pgfqpoint{1.874315in}{2.333827in}}{\pgfqpoint{1.871042in}{2.341727in}}{\pgfqpoint{1.865218in}{2.347551in}}%
\pgfpathcurveto{\pgfqpoint{1.859395in}{2.353375in}}{\pgfqpoint{1.851495in}{2.356647in}}{\pgfqpoint{1.843258in}{2.356647in}}%
\pgfpathcurveto{\pgfqpoint{1.835022in}{2.356647in}}{\pgfqpoint{1.827122in}{2.353375in}}{\pgfqpoint{1.821298in}{2.347551in}}%
\pgfpathcurveto{\pgfqpoint{1.815474in}{2.341727in}}{\pgfqpoint{1.812202in}{2.333827in}}{\pgfqpoint{1.812202in}{2.325591in}}%
\pgfpathcurveto{\pgfqpoint{1.812202in}{2.317355in}}{\pgfqpoint{1.815474in}{2.309455in}}{\pgfqpoint{1.821298in}{2.303631in}}%
\pgfpathcurveto{\pgfqpoint{1.827122in}{2.297807in}}{\pgfqpoint{1.835022in}{2.294534in}}{\pgfqpoint{1.843258in}{2.294534in}}%
\pgfpathclose%
\pgfusepath{stroke,fill}%
\end{pgfscope}%
\begin{pgfscope}%
\pgfpathrectangle{\pgfqpoint{0.100000in}{0.212622in}}{\pgfqpoint{3.696000in}{3.696000in}}%
\pgfusepath{clip}%
\pgfsetbuttcap%
\pgfsetroundjoin%
\definecolor{currentfill}{rgb}{0.121569,0.466667,0.705882}%
\pgfsetfillcolor{currentfill}%
\pgfsetfillopacity{0.839888}%
\pgfsetlinewidth{1.003750pt}%
\definecolor{currentstroke}{rgb}{0.121569,0.466667,0.705882}%
\pgfsetstrokecolor{currentstroke}%
\pgfsetstrokeopacity{0.839888}%
\pgfsetdash{}{0pt}%
\pgfpathmoveto{\pgfqpoint{1.843258in}{2.294534in}}%
\pgfpathcurveto{\pgfqpoint{1.851495in}{2.294534in}}{\pgfqpoint{1.859395in}{2.297807in}}{\pgfqpoint{1.865218in}{2.303631in}}%
\pgfpathcurveto{\pgfqpoint{1.871042in}{2.309455in}}{\pgfqpoint{1.874315in}{2.317355in}}{\pgfqpoint{1.874315in}{2.325591in}}%
\pgfpathcurveto{\pgfqpoint{1.874315in}{2.333827in}}{\pgfqpoint{1.871042in}{2.341727in}}{\pgfqpoint{1.865218in}{2.347551in}}%
\pgfpathcurveto{\pgfqpoint{1.859395in}{2.353375in}}{\pgfqpoint{1.851495in}{2.356647in}}{\pgfqpoint{1.843258in}{2.356647in}}%
\pgfpathcurveto{\pgfqpoint{1.835022in}{2.356647in}}{\pgfqpoint{1.827122in}{2.353375in}}{\pgfqpoint{1.821298in}{2.347551in}}%
\pgfpathcurveto{\pgfqpoint{1.815474in}{2.341727in}}{\pgfqpoint{1.812202in}{2.333827in}}{\pgfqpoint{1.812202in}{2.325591in}}%
\pgfpathcurveto{\pgfqpoint{1.812202in}{2.317355in}}{\pgfqpoint{1.815474in}{2.309455in}}{\pgfqpoint{1.821298in}{2.303631in}}%
\pgfpathcurveto{\pgfqpoint{1.827122in}{2.297807in}}{\pgfqpoint{1.835022in}{2.294534in}}{\pgfqpoint{1.843258in}{2.294534in}}%
\pgfpathclose%
\pgfusepath{stroke,fill}%
\end{pgfscope}%
\begin{pgfscope}%
\pgfpathrectangle{\pgfqpoint{0.100000in}{0.212622in}}{\pgfqpoint{3.696000in}{3.696000in}}%
\pgfusepath{clip}%
\pgfsetbuttcap%
\pgfsetroundjoin%
\definecolor{currentfill}{rgb}{0.121569,0.466667,0.705882}%
\pgfsetfillcolor{currentfill}%
\pgfsetfillopacity{0.839888}%
\pgfsetlinewidth{1.003750pt}%
\definecolor{currentstroke}{rgb}{0.121569,0.466667,0.705882}%
\pgfsetstrokecolor{currentstroke}%
\pgfsetstrokeopacity{0.839888}%
\pgfsetdash{}{0pt}%
\pgfpathmoveto{\pgfqpoint{1.843258in}{2.294534in}}%
\pgfpathcurveto{\pgfqpoint{1.851495in}{2.294534in}}{\pgfqpoint{1.859395in}{2.297807in}}{\pgfqpoint{1.865218in}{2.303631in}}%
\pgfpathcurveto{\pgfqpoint{1.871042in}{2.309455in}}{\pgfqpoint{1.874315in}{2.317355in}}{\pgfqpoint{1.874315in}{2.325591in}}%
\pgfpathcurveto{\pgfqpoint{1.874315in}{2.333827in}}{\pgfqpoint{1.871042in}{2.341727in}}{\pgfqpoint{1.865218in}{2.347551in}}%
\pgfpathcurveto{\pgfqpoint{1.859395in}{2.353375in}}{\pgfqpoint{1.851495in}{2.356647in}}{\pgfqpoint{1.843258in}{2.356647in}}%
\pgfpathcurveto{\pgfqpoint{1.835022in}{2.356647in}}{\pgfqpoint{1.827122in}{2.353375in}}{\pgfqpoint{1.821298in}{2.347551in}}%
\pgfpathcurveto{\pgfqpoint{1.815474in}{2.341727in}}{\pgfqpoint{1.812202in}{2.333827in}}{\pgfqpoint{1.812202in}{2.325591in}}%
\pgfpathcurveto{\pgfqpoint{1.812202in}{2.317355in}}{\pgfqpoint{1.815474in}{2.309455in}}{\pgfqpoint{1.821298in}{2.303631in}}%
\pgfpathcurveto{\pgfqpoint{1.827122in}{2.297807in}}{\pgfqpoint{1.835022in}{2.294534in}}{\pgfqpoint{1.843258in}{2.294534in}}%
\pgfpathclose%
\pgfusepath{stroke,fill}%
\end{pgfscope}%
\begin{pgfscope}%
\pgfpathrectangle{\pgfqpoint{0.100000in}{0.212622in}}{\pgfqpoint{3.696000in}{3.696000in}}%
\pgfusepath{clip}%
\pgfsetbuttcap%
\pgfsetroundjoin%
\definecolor{currentfill}{rgb}{0.121569,0.466667,0.705882}%
\pgfsetfillcolor{currentfill}%
\pgfsetfillopacity{0.839888}%
\pgfsetlinewidth{1.003750pt}%
\definecolor{currentstroke}{rgb}{0.121569,0.466667,0.705882}%
\pgfsetstrokecolor{currentstroke}%
\pgfsetstrokeopacity{0.839888}%
\pgfsetdash{}{0pt}%
\pgfpathmoveto{\pgfqpoint{1.843258in}{2.294534in}}%
\pgfpathcurveto{\pgfqpoint{1.851495in}{2.294534in}}{\pgfqpoint{1.859395in}{2.297807in}}{\pgfqpoint{1.865218in}{2.303631in}}%
\pgfpathcurveto{\pgfqpoint{1.871042in}{2.309455in}}{\pgfqpoint{1.874315in}{2.317355in}}{\pgfqpoint{1.874315in}{2.325591in}}%
\pgfpathcurveto{\pgfqpoint{1.874315in}{2.333827in}}{\pgfqpoint{1.871042in}{2.341727in}}{\pgfqpoint{1.865218in}{2.347551in}}%
\pgfpathcurveto{\pgfqpoint{1.859395in}{2.353375in}}{\pgfqpoint{1.851495in}{2.356647in}}{\pgfqpoint{1.843258in}{2.356647in}}%
\pgfpathcurveto{\pgfqpoint{1.835022in}{2.356647in}}{\pgfqpoint{1.827122in}{2.353375in}}{\pgfqpoint{1.821298in}{2.347551in}}%
\pgfpathcurveto{\pgfqpoint{1.815474in}{2.341727in}}{\pgfqpoint{1.812202in}{2.333827in}}{\pgfqpoint{1.812202in}{2.325591in}}%
\pgfpathcurveto{\pgfqpoint{1.812202in}{2.317355in}}{\pgfqpoint{1.815474in}{2.309455in}}{\pgfqpoint{1.821298in}{2.303631in}}%
\pgfpathcurveto{\pgfqpoint{1.827122in}{2.297807in}}{\pgfqpoint{1.835022in}{2.294534in}}{\pgfqpoint{1.843258in}{2.294534in}}%
\pgfpathclose%
\pgfusepath{stroke,fill}%
\end{pgfscope}%
\begin{pgfscope}%
\pgfpathrectangle{\pgfqpoint{0.100000in}{0.212622in}}{\pgfqpoint{3.696000in}{3.696000in}}%
\pgfusepath{clip}%
\pgfsetbuttcap%
\pgfsetroundjoin%
\definecolor{currentfill}{rgb}{0.121569,0.466667,0.705882}%
\pgfsetfillcolor{currentfill}%
\pgfsetfillopacity{0.839888}%
\pgfsetlinewidth{1.003750pt}%
\definecolor{currentstroke}{rgb}{0.121569,0.466667,0.705882}%
\pgfsetstrokecolor{currentstroke}%
\pgfsetstrokeopacity{0.839888}%
\pgfsetdash{}{0pt}%
\pgfpathmoveto{\pgfqpoint{1.843258in}{2.294534in}}%
\pgfpathcurveto{\pgfqpoint{1.851495in}{2.294534in}}{\pgfqpoint{1.859395in}{2.297807in}}{\pgfqpoint{1.865218in}{2.303631in}}%
\pgfpathcurveto{\pgfqpoint{1.871042in}{2.309455in}}{\pgfqpoint{1.874315in}{2.317355in}}{\pgfqpoint{1.874315in}{2.325591in}}%
\pgfpathcurveto{\pgfqpoint{1.874315in}{2.333827in}}{\pgfqpoint{1.871042in}{2.341727in}}{\pgfqpoint{1.865218in}{2.347551in}}%
\pgfpathcurveto{\pgfqpoint{1.859395in}{2.353375in}}{\pgfqpoint{1.851495in}{2.356647in}}{\pgfqpoint{1.843258in}{2.356647in}}%
\pgfpathcurveto{\pgfqpoint{1.835022in}{2.356647in}}{\pgfqpoint{1.827122in}{2.353375in}}{\pgfqpoint{1.821298in}{2.347551in}}%
\pgfpathcurveto{\pgfqpoint{1.815474in}{2.341727in}}{\pgfqpoint{1.812202in}{2.333827in}}{\pgfqpoint{1.812202in}{2.325591in}}%
\pgfpathcurveto{\pgfqpoint{1.812202in}{2.317355in}}{\pgfqpoint{1.815474in}{2.309455in}}{\pgfqpoint{1.821298in}{2.303631in}}%
\pgfpathcurveto{\pgfqpoint{1.827122in}{2.297807in}}{\pgfqpoint{1.835022in}{2.294534in}}{\pgfqpoint{1.843258in}{2.294534in}}%
\pgfpathclose%
\pgfusepath{stroke,fill}%
\end{pgfscope}%
\begin{pgfscope}%
\pgfpathrectangle{\pgfqpoint{0.100000in}{0.212622in}}{\pgfqpoint{3.696000in}{3.696000in}}%
\pgfusepath{clip}%
\pgfsetbuttcap%
\pgfsetroundjoin%
\definecolor{currentfill}{rgb}{0.121569,0.466667,0.705882}%
\pgfsetfillcolor{currentfill}%
\pgfsetfillopacity{0.839888}%
\pgfsetlinewidth{1.003750pt}%
\definecolor{currentstroke}{rgb}{0.121569,0.466667,0.705882}%
\pgfsetstrokecolor{currentstroke}%
\pgfsetstrokeopacity{0.839888}%
\pgfsetdash{}{0pt}%
\pgfpathmoveto{\pgfqpoint{1.843258in}{2.294534in}}%
\pgfpathcurveto{\pgfqpoint{1.851495in}{2.294534in}}{\pgfqpoint{1.859395in}{2.297807in}}{\pgfqpoint{1.865218in}{2.303631in}}%
\pgfpathcurveto{\pgfqpoint{1.871042in}{2.309455in}}{\pgfqpoint{1.874315in}{2.317355in}}{\pgfqpoint{1.874315in}{2.325591in}}%
\pgfpathcurveto{\pgfqpoint{1.874315in}{2.333827in}}{\pgfqpoint{1.871042in}{2.341727in}}{\pgfqpoint{1.865218in}{2.347551in}}%
\pgfpathcurveto{\pgfqpoint{1.859395in}{2.353375in}}{\pgfqpoint{1.851495in}{2.356647in}}{\pgfqpoint{1.843258in}{2.356647in}}%
\pgfpathcurveto{\pgfqpoint{1.835022in}{2.356647in}}{\pgfqpoint{1.827122in}{2.353375in}}{\pgfqpoint{1.821298in}{2.347551in}}%
\pgfpathcurveto{\pgfqpoint{1.815474in}{2.341727in}}{\pgfqpoint{1.812202in}{2.333827in}}{\pgfqpoint{1.812202in}{2.325591in}}%
\pgfpathcurveto{\pgfqpoint{1.812202in}{2.317355in}}{\pgfqpoint{1.815474in}{2.309455in}}{\pgfqpoint{1.821298in}{2.303631in}}%
\pgfpathcurveto{\pgfqpoint{1.827122in}{2.297807in}}{\pgfqpoint{1.835022in}{2.294534in}}{\pgfqpoint{1.843258in}{2.294534in}}%
\pgfpathclose%
\pgfusepath{stroke,fill}%
\end{pgfscope}%
\begin{pgfscope}%
\pgfpathrectangle{\pgfqpoint{0.100000in}{0.212622in}}{\pgfqpoint{3.696000in}{3.696000in}}%
\pgfusepath{clip}%
\pgfsetbuttcap%
\pgfsetroundjoin%
\definecolor{currentfill}{rgb}{0.121569,0.466667,0.705882}%
\pgfsetfillcolor{currentfill}%
\pgfsetfillopacity{0.839888}%
\pgfsetlinewidth{1.003750pt}%
\definecolor{currentstroke}{rgb}{0.121569,0.466667,0.705882}%
\pgfsetstrokecolor{currentstroke}%
\pgfsetstrokeopacity{0.839888}%
\pgfsetdash{}{0pt}%
\pgfpathmoveto{\pgfqpoint{1.843258in}{2.294534in}}%
\pgfpathcurveto{\pgfqpoint{1.851495in}{2.294534in}}{\pgfqpoint{1.859395in}{2.297807in}}{\pgfqpoint{1.865218in}{2.303631in}}%
\pgfpathcurveto{\pgfqpoint{1.871042in}{2.309455in}}{\pgfqpoint{1.874315in}{2.317355in}}{\pgfqpoint{1.874315in}{2.325591in}}%
\pgfpathcurveto{\pgfqpoint{1.874315in}{2.333827in}}{\pgfqpoint{1.871042in}{2.341727in}}{\pgfqpoint{1.865218in}{2.347551in}}%
\pgfpathcurveto{\pgfqpoint{1.859395in}{2.353375in}}{\pgfqpoint{1.851495in}{2.356647in}}{\pgfqpoint{1.843258in}{2.356647in}}%
\pgfpathcurveto{\pgfqpoint{1.835022in}{2.356647in}}{\pgfqpoint{1.827122in}{2.353375in}}{\pgfqpoint{1.821298in}{2.347551in}}%
\pgfpathcurveto{\pgfqpoint{1.815474in}{2.341727in}}{\pgfqpoint{1.812202in}{2.333827in}}{\pgfqpoint{1.812202in}{2.325591in}}%
\pgfpathcurveto{\pgfqpoint{1.812202in}{2.317355in}}{\pgfqpoint{1.815474in}{2.309455in}}{\pgfqpoint{1.821298in}{2.303631in}}%
\pgfpathcurveto{\pgfqpoint{1.827122in}{2.297807in}}{\pgfqpoint{1.835022in}{2.294534in}}{\pgfqpoint{1.843258in}{2.294534in}}%
\pgfpathclose%
\pgfusepath{stroke,fill}%
\end{pgfscope}%
\begin{pgfscope}%
\pgfpathrectangle{\pgfqpoint{0.100000in}{0.212622in}}{\pgfqpoint{3.696000in}{3.696000in}}%
\pgfusepath{clip}%
\pgfsetbuttcap%
\pgfsetroundjoin%
\definecolor{currentfill}{rgb}{0.121569,0.466667,0.705882}%
\pgfsetfillcolor{currentfill}%
\pgfsetfillopacity{0.839888}%
\pgfsetlinewidth{1.003750pt}%
\definecolor{currentstroke}{rgb}{0.121569,0.466667,0.705882}%
\pgfsetstrokecolor{currentstroke}%
\pgfsetstrokeopacity{0.839888}%
\pgfsetdash{}{0pt}%
\pgfpathmoveto{\pgfqpoint{1.843258in}{2.294534in}}%
\pgfpathcurveto{\pgfqpoint{1.851495in}{2.294534in}}{\pgfqpoint{1.859395in}{2.297807in}}{\pgfqpoint{1.865218in}{2.303631in}}%
\pgfpathcurveto{\pgfqpoint{1.871042in}{2.309455in}}{\pgfqpoint{1.874315in}{2.317355in}}{\pgfqpoint{1.874315in}{2.325591in}}%
\pgfpathcurveto{\pgfqpoint{1.874315in}{2.333827in}}{\pgfqpoint{1.871042in}{2.341727in}}{\pgfqpoint{1.865218in}{2.347551in}}%
\pgfpathcurveto{\pgfqpoint{1.859395in}{2.353375in}}{\pgfqpoint{1.851495in}{2.356647in}}{\pgfqpoint{1.843258in}{2.356647in}}%
\pgfpathcurveto{\pgfqpoint{1.835022in}{2.356647in}}{\pgfqpoint{1.827122in}{2.353375in}}{\pgfqpoint{1.821298in}{2.347551in}}%
\pgfpathcurveto{\pgfqpoint{1.815474in}{2.341727in}}{\pgfqpoint{1.812202in}{2.333827in}}{\pgfqpoint{1.812202in}{2.325591in}}%
\pgfpathcurveto{\pgfqpoint{1.812202in}{2.317355in}}{\pgfqpoint{1.815474in}{2.309455in}}{\pgfqpoint{1.821298in}{2.303631in}}%
\pgfpathcurveto{\pgfqpoint{1.827122in}{2.297807in}}{\pgfqpoint{1.835022in}{2.294534in}}{\pgfqpoint{1.843258in}{2.294534in}}%
\pgfpathclose%
\pgfusepath{stroke,fill}%
\end{pgfscope}%
\begin{pgfscope}%
\pgfpathrectangle{\pgfqpoint{0.100000in}{0.212622in}}{\pgfqpoint{3.696000in}{3.696000in}}%
\pgfusepath{clip}%
\pgfsetbuttcap%
\pgfsetroundjoin%
\definecolor{currentfill}{rgb}{0.121569,0.466667,0.705882}%
\pgfsetfillcolor{currentfill}%
\pgfsetfillopacity{0.839888}%
\pgfsetlinewidth{1.003750pt}%
\definecolor{currentstroke}{rgb}{0.121569,0.466667,0.705882}%
\pgfsetstrokecolor{currentstroke}%
\pgfsetstrokeopacity{0.839888}%
\pgfsetdash{}{0pt}%
\pgfpathmoveto{\pgfqpoint{1.843258in}{2.294534in}}%
\pgfpathcurveto{\pgfqpoint{1.851495in}{2.294534in}}{\pgfqpoint{1.859395in}{2.297807in}}{\pgfqpoint{1.865218in}{2.303631in}}%
\pgfpathcurveto{\pgfqpoint{1.871042in}{2.309455in}}{\pgfqpoint{1.874315in}{2.317355in}}{\pgfqpoint{1.874315in}{2.325591in}}%
\pgfpathcurveto{\pgfqpoint{1.874315in}{2.333827in}}{\pgfqpoint{1.871042in}{2.341727in}}{\pgfqpoint{1.865218in}{2.347551in}}%
\pgfpathcurveto{\pgfqpoint{1.859395in}{2.353375in}}{\pgfqpoint{1.851495in}{2.356647in}}{\pgfqpoint{1.843258in}{2.356647in}}%
\pgfpathcurveto{\pgfqpoint{1.835022in}{2.356647in}}{\pgfqpoint{1.827122in}{2.353375in}}{\pgfqpoint{1.821298in}{2.347551in}}%
\pgfpathcurveto{\pgfqpoint{1.815474in}{2.341727in}}{\pgfqpoint{1.812202in}{2.333827in}}{\pgfqpoint{1.812202in}{2.325591in}}%
\pgfpathcurveto{\pgfqpoint{1.812202in}{2.317355in}}{\pgfqpoint{1.815474in}{2.309455in}}{\pgfqpoint{1.821298in}{2.303631in}}%
\pgfpathcurveto{\pgfqpoint{1.827122in}{2.297807in}}{\pgfqpoint{1.835022in}{2.294534in}}{\pgfqpoint{1.843258in}{2.294534in}}%
\pgfpathclose%
\pgfusepath{stroke,fill}%
\end{pgfscope}%
\begin{pgfscope}%
\pgfpathrectangle{\pgfqpoint{0.100000in}{0.212622in}}{\pgfqpoint{3.696000in}{3.696000in}}%
\pgfusepath{clip}%
\pgfsetbuttcap%
\pgfsetroundjoin%
\definecolor{currentfill}{rgb}{0.121569,0.466667,0.705882}%
\pgfsetfillcolor{currentfill}%
\pgfsetfillopacity{0.839888}%
\pgfsetlinewidth{1.003750pt}%
\definecolor{currentstroke}{rgb}{0.121569,0.466667,0.705882}%
\pgfsetstrokecolor{currentstroke}%
\pgfsetstrokeopacity{0.839888}%
\pgfsetdash{}{0pt}%
\pgfpathmoveto{\pgfqpoint{1.843258in}{2.294534in}}%
\pgfpathcurveto{\pgfqpoint{1.851495in}{2.294534in}}{\pgfqpoint{1.859395in}{2.297807in}}{\pgfqpoint{1.865218in}{2.303631in}}%
\pgfpathcurveto{\pgfqpoint{1.871042in}{2.309455in}}{\pgfqpoint{1.874315in}{2.317355in}}{\pgfqpoint{1.874315in}{2.325591in}}%
\pgfpathcurveto{\pgfqpoint{1.874315in}{2.333827in}}{\pgfqpoint{1.871042in}{2.341727in}}{\pgfqpoint{1.865218in}{2.347551in}}%
\pgfpathcurveto{\pgfqpoint{1.859395in}{2.353375in}}{\pgfqpoint{1.851495in}{2.356647in}}{\pgfqpoint{1.843258in}{2.356647in}}%
\pgfpathcurveto{\pgfqpoint{1.835022in}{2.356647in}}{\pgfqpoint{1.827122in}{2.353375in}}{\pgfqpoint{1.821298in}{2.347551in}}%
\pgfpathcurveto{\pgfqpoint{1.815474in}{2.341727in}}{\pgfqpoint{1.812202in}{2.333827in}}{\pgfqpoint{1.812202in}{2.325591in}}%
\pgfpathcurveto{\pgfqpoint{1.812202in}{2.317355in}}{\pgfqpoint{1.815474in}{2.309455in}}{\pgfqpoint{1.821298in}{2.303631in}}%
\pgfpathcurveto{\pgfqpoint{1.827122in}{2.297807in}}{\pgfqpoint{1.835022in}{2.294534in}}{\pgfqpoint{1.843258in}{2.294534in}}%
\pgfpathclose%
\pgfusepath{stroke,fill}%
\end{pgfscope}%
\begin{pgfscope}%
\pgfpathrectangle{\pgfqpoint{0.100000in}{0.212622in}}{\pgfqpoint{3.696000in}{3.696000in}}%
\pgfusepath{clip}%
\pgfsetbuttcap%
\pgfsetroundjoin%
\definecolor{currentfill}{rgb}{0.121569,0.466667,0.705882}%
\pgfsetfillcolor{currentfill}%
\pgfsetfillopacity{0.839888}%
\pgfsetlinewidth{1.003750pt}%
\definecolor{currentstroke}{rgb}{0.121569,0.466667,0.705882}%
\pgfsetstrokecolor{currentstroke}%
\pgfsetstrokeopacity{0.839888}%
\pgfsetdash{}{0pt}%
\pgfpathmoveto{\pgfqpoint{1.843258in}{2.294534in}}%
\pgfpathcurveto{\pgfqpoint{1.851495in}{2.294534in}}{\pgfqpoint{1.859395in}{2.297807in}}{\pgfqpoint{1.865218in}{2.303631in}}%
\pgfpathcurveto{\pgfqpoint{1.871042in}{2.309455in}}{\pgfqpoint{1.874315in}{2.317355in}}{\pgfqpoint{1.874315in}{2.325591in}}%
\pgfpathcurveto{\pgfqpoint{1.874315in}{2.333827in}}{\pgfqpoint{1.871042in}{2.341727in}}{\pgfqpoint{1.865218in}{2.347551in}}%
\pgfpathcurveto{\pgfqpoint{1.859395in}{2.353375in}}{\pgfqpoint{1.851495in}{2.356647in}}{\pgfqpoint{1.843258in}{2.356647in}}%
\pgfpathcurveto{\pgfqpoint{1.835022in}{2.356647in}}{\pgfqpoint{1.827122in}{2.353375in}}{\pgfqpoint{1.821298in}{2.347551in}}%
\pgfpathcurveto{\pgfqpoint{1.815474in}{2.341727in}}{\pgfqpoint{1.812202in}{2.333827in}}{\pgfqpoint{1.812202in}{2.325591in}}%
\pgfpathcurveto{\pgfqpoint{1.812202in}{2.317355in}}{\pgfqpoint{1.815474in}{2.309455in}}{\pgfqpoint{1.821298in}{2.303631in}}%
\pgfpathcurveto{\pgfqpoint{1.827122in}{2.297807in}}{\pgfqpoint{1.835022in}{2.294534in}}{\pgfqpoint{1.843258in}{2.294534in}}%
\pgfpathclose%
\pgfusepath{stroke,fill}%
\end{pgfscope}%
\begin{pgfscope}%
\pgfpathrectangle{\pgfqpoint{0.100000in}{0.212622in}}{\pgfqpoint{3.696000in}{3.696000in}}%
\pgfusepath{clip}%
\pgfsetbuttcap%
\pgfsetroundjoin%
\definecolor{currentfill}{rgb}{0.121569,0.466667,0.705882}%
\pgfsetfillcolor{currentfill}%
\pgfsetfillopacity{0.839888}%
\pgfsetlinewidth{1.003750pt}%
\definecolor{currentstroke}{rgb}{0.121569,0.466667,0.705882}%
\pgfsetstrokecolor{currentstroke}%
\pgfsetstrokeopacity{0.839888}%
\pgfsetdash{}{0pt}%
\pgfpathmoveto{\pgfqpoint{1.843258in}{2.294534in}}%
\pgfpathcurveto{\pgfqpoint{1.851495in}{2.294534in}}{\pgfqpoint{1.859395in}{2.297807in}}{\pgfqpoint{1.865218in}{2.303631in}}%
\pgfpathcurveto{\pgfqpoint{1.871042in}{2.309455in}}{\pgfqpoint{1.874315in}{2.317355in}}{\pgfqpoint{1.874315in}{2.325591in}}%
\pgfpathcurveto{\pgfqpoint{1.874315in}{2.333827in}}{\pgfqpoint{1.871042in}{2.341727in}}{\pgfqpoint{1.865218in}{2.347551in}}%
\pgfpathcurveto{\pgfqpoint{1.859395in}{2.353375in}}{\pgfqpoint{1.851495in}{2.356647in}}{\pgfqpoint{1.843258in}{2.356647in}}%
\pgfpathcurveto{\pgfqpoint{1.835022in}{2.356647in}}{\pgfqpoint{1.827122in}{2.353375in}}{\pgfqpoint{1.821298in}{2.347551in}}%
\pgfpathcurveto{\pgfqpoint{1.815474in}{2.341727in}}{\pgfqpoint{1.812202in}{2.333827in}}{\pgfqpoint{1.812202in}{2.325591in}}%
\pgfpathcurveto{\pgfqpoint{1.812202in}{2.317355in}}{\pgfqpoint{1.815474in}{2.309455in}}{\pgfqpoint{1.821298in}{2.303631in}}%
\pgfpathcurveto{\pgfqpoint{1.827122in}{2.297807in}}{\pgfqpoint{1.835022in}{2.294534in}}{\pgfqpoint{1.843258in}{2.294534in}}%
\pgfpathclose%
\pgfusepath{stroke,fill}%
\end{pgfscope}%
\begin{pgfscope}%
\pgfpathrectangle{\pgfqpoint{0.100000in}{0.212622in}}{\pgfqpoint{3.696000in}{3.696000in}}%
\pgfusepath{clip}%
\pgfsetbuttcap%
\pgfsetroundjoin%
\definecolor{currentfill}{rgb}{0.121569,0.466667,0.705882}%
\pgfsetfillcolor{currentfill}%
\pgfsetfillopacity{0.839888}%
\pgfsetlinewidth{1.003750pt}%
\definecolor{currentstroke}{rgb}{0.121569,0.466667,0.705882}%
\pgfsetstrokecolor{currentstroke}%
\pgfsetstrokeopacity{0.839888}%
\pgfsetdash{}{0pt}%
\pgfpathmoveto{\pgfqpoint{1.843258in}{2.294534in}}%
\pgfpathcurveto{\pgfqpoint{1.851495in}{2.294534in}}{\pgfqpoint{1.859395in}{2.297807in}}{\pgfqpoint{1.865218in}{2.303631in}}%
\pgfpathcurveto{\pgfqpoint{1.871042in}{2.309455in}}{\pgfqpoint{1.874315in}{2.317355in}}{\pgfqpoint{1.874315in}{2.325591in}}%
\pgfpathcurveto{\pgfqpoint{1.874315in}{2.333827in}}{\pgfqpoint{1.871042in}{2.341727in}}{\pgfqpoint{1.865218in}{2.347551in}}%
\pgfpathcurveto{\pgfqpoint{1.859395in}{2.353375in}}{\pgfqpoint{1.851495in}{2.356647in}}{\pgfqpoint{1.843258in}{2.356647in}}%
\pgfpathcurveto{\pgfqpoint{1.835022in}{2.356647in}}{\pgfqpoint{1.827122in}{2.353375in}}{\pgfqpoint{1.821298in}{2.347551in}}%
\pgfpathcurveto{\pgfqpoint{1.815474in}{2.341727in}}{\pgfqpoint{1.812202in}{2.333827in}}{\pgfqpoint{1.812202in}{2.325591in}}%
\pgfpathcurveto{\pgfqpoint{1.812202in}{2.317355in}}{\pgfqpoint{1.815474in}{2.309455in}}{\pgfqpoint{1.821298in}{2.303631in}}%
\pgfpathcurveto{\pgfqpoint{1.827122in}{2.297807in}}{\pgfqpoint{1.835022in}{2.294534in}}{\pgfqpoint{1.843258in}{2.294534in}}%
\pgfpathclose%
\pgfusepath{stroke,fill}%
\end{pgfscope}%
\begin{pgfscope}%
\pgfpathrectangle{\pgfqpoint{0.100000in}{0.212622in}}{\pgfqpoint{3.696000in}{3.696000in}}%
\pgfusepath{clip}%
\pgfsetbuttcap%
\pgfsetroundjoin%
\definecolor{currentfill}{rgb}{0.121569,0.466667,0.705882}%
\pgfsetfillcolor{currentfill}%
\pgfsetfillopacity{0.839888}%
\pgfsetlinewidth{1.003750pt}%
\definecolor{currentstroke}{rgb}{0.121569,0.466667,0.705882}%
\pgfsetstrokecolor{currentstroke}%
\pgfsetstrokeopacity{0.839888}%
\pgfsetdash{}{0pt}%
\pgfpathmoveto{\pgfqpoint{1.843258in}{2.294534in}}%
\pgfpathcurveto{\pgfqpoint{1.851495in}{2.294534in}}{\pgfqpoint{1.859395in}{2.297807in}}{\pgfqpoint{1.865218in}{2.303631in}}%
\pgfpathcurveto{\pgfqpoint{1.871042in}{2.309455in}}{\pgfqpoint{1.874315in}{2.317355in}}{\pgfqpoint{1.874315in}{2.325591in}}%
\pgfpathcurveto{\pgfqpoint{1.874315in}{2.333827in}}{\pgfqpoint{1.871042in}{2.341727in}}{\pgfqpoint{1.865218in}{2.347551in}}%
\pgfpathcurveto{\pgfqpoint{1.859395in}{2.353375in}}{\pgfqpoint{1.851495in}{2.356647in}}{\pgfqpoint{1.843258in}{2.356647in}}%
\pgfpathcurveto{\pgfqpoint{1.835022in}{2.356647in}}{\pgfqpoint{1.827122in}{2.353375in}}{\pgfqpoint{1.821298in}{2.347551in}}%
\pgfpathcurveto{\pgfqpoint{1.815474in}{2.341727in}}{\pgfqpoint{1.812202in}{2.333827in}}{\pgfqpoint{1.812202in}{2.325591in}}%
\pgfpathcurveto{\pgfqpoint{1.812202in}{2.317355in}}{\pgfqpoint{1.815474in}{2.309455in}}{\pgfqpoint{1.821298in}{2.303631in}}%
\pgfpathcurveto{\pgfqpoint{1.827122in}{2.297807in}}{\pgfqpoint{1.835022in}{2.294534in}}{\pgfqpoint{1.843258in}{2.294534in}}%
\pgfpathclose%
\pgfusepath{stroke,fill}%
\end{pgfscope}%
\begin{pgfscope}%
\pgfpathrectangle{\pgfqpoint{0.100000in}{0.212622in}}{\pgfqpoint{3.696000in}{3.696000in}}%
\pgfusepath{clip}%
\pgfsetbuttcap%
\pgfsetroundjoin%
\definecolor{currentfill}{rgb}{0.121569,0.466667,0.705882}%
\pgfsetfillcolor{currentfill}%
\pgfsetfillopacity{0.839888}%
\pgfsetlinewidth{1.003750pt}%
\definecolor{currentstroke}{rgb}{0.121569,0.466667,0.705882}%
\pgfsetstrokecolor{currentstroke}%
\pgfsetstrokeopacity{0.839888}%
\pgfsetdash{}{0pt}%
\pgfpathmoveto{\pgfqpoint{1.843258in}{2.294534in}}%
\pgfpathcurveto{\pgfqpoint{1.851495in}{2.294534in}}{\pgfqpoint{1.859395in}{2.297807in}}{\pgfqpoint{1.865218in}{2.303631in}}%
\pgfpathcurveto{\pgfqpoint{1.871042in}{2.309455in}}{\pgfqpoint{1.874315in}{2.317355in}}{\pgfqpoint{1.874315in}{2.325591in}}%
\pgfpathcurveto{\pgfqpoint{1.874315in}{2.333827in}}{\pgfqpoint{1.871042in}{2.341727in}}{\pgfqpoint{1.865218in}{2.347551in}}%
\pgfpathcurveto{\pgfqpoint{1.859395in}{2.353375in}}{\pgfqpoint{1.851495in}{2.356647in}}{\pgfqpoint{1.843258in}{2.356647in}}%
\pgfpathcurveto{\pgfqpoint{1.835022in}{2.356647in}}{\pgfqpoint{1.827122in}{2.353375in}}{\pgfqpoint{1.821298in}{2.347551in}}%
\pgfpathcurveto{\pgfqpoint{1.815474in}{2.341727in}}{\pgfqpoint{1.812202in}{2.333827in}}{\pgfqpoint{1.812202in}{2.325591in}}%
\pgfpathcurveto{\pgfqpoint{1.812202in}{2.317355in}}{\pgfqpoint{1.815474in}{2.309455in}}{\pgfqpoint{1.821298in}{2.303631in}}%
\pgfpathcurveto{\pgfqpoint{1.827122in}{2.297807in}}{\pgfqpoint{1.835022in}{2.294534in}}{\pgfqpoint{1.843258in}{2.294534in}}%
\pgfpathclose%
\pgfusepath{stroke,fill}%
\end{pgfscope}%
\begin{pgfscope}%
\pgfpathrectangle{\pgfqpoint{0.100000in}{0.212622in}}{\pgfqpoint{3.696000in}{3.696000in}}%
\pgfusepath{clip}%
\pgfsetbuttcap%
\pgfsetroundjoin%
\definecolor{currentfill}{rgb}{0.121569,0.466667,0.705882}%
\pgfsetfillcolor{currentfill}%
\pgfsetfillopacity{0.839888}%
\pgfsetlinewidth{1.003750pt}%
\definecolor{currentstroke}{rgb}{0.121569,0.466667,0.705882}%
\pgfsetstrokecolor{currentstroke}%
\pgfsetstrokeopacity{0.839888}%
\pgfsetdash{}{0pt}%
\pgfpathmoveto{\pgfqpoint{1.843258in}{2.294534in}}%
\pgfpathcurveto{\pgfqpoint{1.851495in}{2.294534in}}{\pgfqpoint{1.859395in}{2.297807in}}{\pgfqpoint{1.865218in}{2.303631in}}%
\pgfpathcurveto{\pgfqpoint{1.871042in}{2.309455in}}{\pgfqpoint{1.874315in}{2.317355in}}{\pgfqpoint{1.874315in}{2.325591in}}%
\pgfpathcurveto{\pgfqpoint{1.874315in}{2.333827in}}{\pgfqpoint{1.871042in}{2.341727in}}{\pgfqpoint{1.865218in}{2.347551in}}%
\pgfpathcurveto{\pgfqpoint{1.859395in}{2.353375in}}{\pgfqpoint{1.851495in}{2.356647in}}{\pgfqpoint{1.843258in}{2.356647in}}%
\pgfpathcurveto{\pgfqpoint{1.835022in}{2.356647in}}{\pgfqpoint{1.827122in}{2.353375in}}{\pgfqpoint{1.821298in}{2.347551in}}%
\pgfpathcurveto{\pgfqpoint{1.815474in}{2.341727in}}{\pgfqpoint{1.812202in}{2.333827in}}{\pgfqpoint{1.812202in}{2.325591in}}%
\pgfpathcurveto{\pgfqpoint{1.812202in}{2.317355in}}{\pgfqpoint{1.815474in}{2.309455in}}{\pgfqpoint{1.821298in}{2.303631in}}%
\pgfpathcurveto{\pgfqpoint{1.827122in}{2.297807in}}{\pgfqpoint{1.835022in}{2.294534in}}{\pgfqpoint{1.843258in}{2.294534in}}%
\pgfpathclose%
\pgfusepath{stroke,fill}%
\end{pgfscope}%
\begin{pgfscope}%
\pgfpathrectangle{\pgfqpoint{0.100000in}{0.212622in}}{\pgfqpoint{3.696000in}{3.696000in}}%
\pgfusepath{clip}%
\pgfsetbuttcap%
\pgfsetroundjoin%
\definecolor{currentfill}{rgb}{0.121569,0.466667,0.705882}%
\pgfsetfillcolor{currentfill}%
\pgfsetfillopacity{0.839888}%
\pgfsetlinewidth{1.003750pt}%
\definecolor{currentstroke}{rgb}{0.121569,0.466667,0.705882}%
\pgfsetstrokecolor{currentstroke}%
\pgfsetstrokeopacity{0.839888}%
\pgfsetdash{}{0pt}%
\pgfpathmoveto{\pgfqpoint{1.843258in}{2.294534in}}%
\pgfpathcurveto{\pgfqpoint{1.851495in}{2.294534in}}{\pgfqpoint{1.859395in}{2.297807in}}{\pgfqpoint{1.865218in}{2.303631in}}%
\pgfpathcurveto{\pgfqpoint{1.871042in}{2.309455in}}{\pgfqpoint{1.874315in}{2.317355in}}{\pgfqpoint{1.874315in}{2.325591in}}%
\pgfpathcurveto{\pgfqpoint{1.874315in}{2.333827in}}{\pgfqpoint{1.871042in}{2.341727in}}{\pgfqpoint{1.865218in}{2.347551in}}%
\pgfpathcurveto{\pgfqpoint{1.859395in}{2.353375in}}{\pgfqpoint{1.851495in}{2.356647in}}{\pgfqpoint{1.843258in}{2.356647in}}%
\pgfpathcurveto{\pgfqpoint{1.835022in}{2.356647in}}{\pgfqpoint{1.827122in}{2.353375in}}{\pgfqpoint{1.821298in}{2.347551in}}%
\pgfpathcurveto{\pgfqpoint{1.815474in}{2.341727in}}{\pgfqpoint{1.812202in}{2.333827in}}{\pgfqpoint{1.812202in}{2.325591in}}%
\pgfpathcurveto{\pgfqpoint{1.812202in}{2.317355in}}{\pgfqpoint{1.815474in}{2.309455in}}{\pgfqpoint{1.821298in}{2.303631in}}%
\pgfpathcurveto{\pgfqpoint{1.827122in}{2.297807in}}{\pgfqpoint{1.835022in}{2.294534in}}{\pgfqpoint{1.843258in}{2.294534in}}%
\pgfpathclose%
\pgfusepath{stroke,fill}%
\end{pgfscope}%
\begin{pgfscope}%
\pgfpathrectangle{\pgfqpoint{0.100000in}{0.212622in}}{\pgfqpoint{3.696000in}{3.696000in}}%
\pgfusepath{clip}%
\pgfsetbuttcap%
\pgfsetroundjoin%
\definecolor{currentfill}{rgb}{0.121569,0.466667,0.705882}%
\pgfsetfillcolor{currentfill}%
\pgfsetfillopacity{0.839888}%
\pgfsetlinewidth{1.003750pt}%
\definecolor{currentstroke}{rgb}{0.121569,0.466667,0.705882}%
\pgfsetstrokecolor{currentstroke}%
\pgfsetstrokeopacity{0.839888}%
\pgfsetdash{}{0pt}%
\pgfpathmoveto{\pgfqpoint{1.843258in}{2.294534in}}%
\pgfpathcurveto{\pgfqpoint{1.851495in}{2.294534in}}{\pgfqpoint{1.859395in}{2.297807in}}{\pgfqpoint{1.865218in}{2.303631in}}%
\pgfpathcurveto{\pgfqpoint{1.871042in}{2.309455in}}{\pgfqpoint{1.874315in}{2.317355in}}{\pgfqpoint{1.874315in}{2.325591in}}%
\pgfpathcurveto{\pgfqpoint{1.874315in}{2.333827in}}{\pgfqpoint{1.871042in}{2.341727in}}{\pgfqpoint{1.865218in}{2.347551in}}%
\pgfpathcurveto{\pgfqpoint{1.859395in}{2.353375in}}{\pgfqpoint{1.851495in}{2.356647in}}{\pgfqpoint{1.843258in}{2.356647in}}%
\pgfpathcurveto{\pgfqpoint{1.835022in}{2.356647in}}{\pgfqpoint{1.827122in}{2.353375in}}{\pgfqpoint{1.821298in}{2.347551in}}%
\pgfpathcurveto{\pgfqpoint{1.815474in}{2.341727in}}{\pgfqpoint{1.812202in}{2.333827in}}{\pgfqpoint{1.812202in}{2.325591in}}%
\pgfpathcurveto{\pgfqpoint{1.812202in}{2.317355in}}{\pgfqpoint{1.815474in}{2.309455in}}{\pgfqpoint{1.821298in}{2.303631in}}%
\pgfpathcurveto{\pgfqpoint{1.827122in}{2.297807in}}{\pgfqpoint{1.835022in}{2.294534in}}{\pgfqpoint{1.843258in}{2.294534in}}%
\pgfpathclose%
\pgfusepath{stroke,fill}%
\end{pgfscope}%
\begin{pgfscope}%
\pgfpathrectangle{\pgfqpoint{0.100000in}{0.212622in}}{\pgfqpoint{3.696000in}{3.696000in}}%
\pgfusepath{clip}%
\pgfsetbuttcap%
\pgfsetroundjoin%
\definecolor{currentfill}{rgb}{0.121569,0.466667,0.705882}%
\pgfsetfillcolor{currentfill}%
\pgfsetfillopacity{0.839888}%
\pgfsetlinewidth{1.003750pt}%
\definecolor{currentstroke}{rgb}{0.121569,0.466667,0.705882}%
\pgfsetstrokecolor{currentstroke}%
\pgfsetstrokeopacity{0.839888}%
\pgfsetdash{}{0pt}%
\pgfpathmoveto{\pgfqpoint{1.843258in}{2.294534in}}%
\pgfpathcurveto{\pgfqpoint{1.851495in}{2.294534in}}{\pgfqpoint{1.859395in}{2.297807in}}{\pgfqpoint{1.865218in}{2.303631in}}%
\pgfpathcurveto{\pgfqpoint{1.871042in}{2.309455in}}{\pgfqpoint{1.874315in}{2.317355in}}{\pgfqpoint{1.874315in}{2.325591in}}%
\pgfpathcurveto{\pgfqpoint{1.874315in}{2.333827in}}{\pgfqpoint{1.871042in}{2.341727in}}{\pgfqpoint{1.865218in}{2.347551in}}%
\pgfpathcurveto{\pgfqpoint{1.859395in}{2.353375in}}{\pgfqpoint{1.851495in}{2.356647in}}{\pgfqpoint{1.843258in}{2.356647in}}%
\pgfpathcurveto{\pgfqpoint{1.835022in}{2.356647in}}{\pgfqpoint{1.827122in}{2.353375in}}{\pgfqpoint{1.821298in}{2.347551in}}%
\pgfpathcurveto{\pgfqpoint{1.815474in}{2.341727in}}{\pgfqpoint{1.812202in}{2.333827in}}{\pgfqpoint{1.812202in}{2.325591in}}%
\pgfpathcurveto{\pgfqpoint{1.812202in}{2.317355in}}{\pgfqpoint{1.815474in}{2.309455in}}{\pgfqpoint{1.821298in}{2.303631in}}%
\pgfpathcurveto{\pgfqpoint{1.827122in}{2.297807in}}{\pgfqpoint{1.835022in}{2.294534in}}{\pgfqpoint{1.843258in}{2.294534in}}%
\pgfpathclose%
\pgfusepath{stroke,fill}%
\end{pgfscope}%
\begin{pgfscope}%
\pgfpathrectangle{\pgfqpoint{0.100000in}{0.212622in}}{\pgfqpoint{3.696000in}{3.696000in}}%
\pgfusepath{clip}%
\pgfsetbuttcap%
\pgfsetroundjoin%
\definecolor{currentfill}{rgb}{0.121569,0.466667,0.705882}%
\pgfsetfillcolor{currentfill}%
\pgfsetfillopacity{0.839888}%
\pgfsetlinewidth{1.003750pt}%
\definecolor{currentstroke}{rgb}{0.121569,0.466667,0.705882}%
\pgfsetstrokecolor{currentstroke}%
\pgfsetstrokeopacity{0.839888}%
\pgfsetdash{}{0pt}%
\pgfpathmoveto{\pgfqpoint{1.843258in}{2.294534in}}%
\pgfpathcurveto{\pgfqpoint{1.851495in}{2.294534in}}{\pgfqpoint{1.859395in}{2.297807in}}{\pgfqpoint{1.865218in}{2.303631in}}%
\pgfpathcurveto{\pgfqpoint{1.871042in}{2.309455in}}{\pgfqpoint{1.874315in}{2.317355in}}{\pgfqpoint{1.874315in}{2.325591in}}%
\pgfpathcurveto{\pgfqpoint{1.874315in}{2.333827in}}{\pgfqpoint{1.871042in}{2.341727in}}{\pgfqpoint{1.865218in}{2.347551in}}%
\pgfpathcurveto{\pgfqpoint{1.859395in}{2.353375in}}{\pgfqpoint{1.851495in}{2.356647in}}{\pgfqpoint{1.843258in}{2.356647in}}%
\pgfpathcurveto{\pgfqpoint{1.835022in}{2.356647in}}{\pgfqpoint{1.827122in}{2.353375in}}{\pgfqpoint{1.821298in}{2.347551in}}%
\pgfpathcurveto{\pgfqpoint{1.815474in}{2.341727in}}{\pgfqpoint{1.812202in}{2.333827in}}{\pgfqpoint{1.812202in}{2.325591in}}%
\pgfpathcurveto{\pgfqpoint{1.812202in}{2.317355in}}{\pgfqpoint{1.815474in}{2.309455in}}{\pgfqpoint{1.821298in}{2.303631in}}%
\pgfpathcurveto{\pgfqpoint{1.827122in}{2.297807in}}{\pgfqpoint{1.835022in}{2.294534in}}{\pgfqpoint{1.843258in}{2.294534in}}%
\pgfpathclose%
\pgfusepath{stroke,fill}%
\end{pgfscope}%
\begin{pgfscope}%
\pgfpathrectangle{\pgfqpoint{0.100000in}{0.212622in}}{\pgfqpoint{3.696000in}{3.696000in}}%
\pgfusepath{clip}%
\pgfsetbuttcap%
\pgfsetroundjoin%
\definecolor{currentfill}{rgb}{0.121569,0.466667,0.705882}%
\pgfsetfillcolor{currentfill}%
\pgfsetfillopacity{0.839888}%
\pgfsetlinewidth{1.003750pt}%
\definecolor{currentstroke}{rgb}{0.121569,0.466667,0.705882}%
\pgfsetstrokecolor{currentstroke}%
\pgfsetstrokeopacity{0.839888}%
\pgfsetdash{}{0pt}%
\pgfpathmoveto{\pgfqpoint{1.843258in}{2.294534in}}%
\pgfpathcurveto{\pgfqpoint{1.851495in}{2.294534in}}{\pgfqpoint{1.859395in}{2.297807in}}{\pgfqpoint{1.865218in}{2.303631in}}%
\pgfpathcurveto{\pgfqpoint{1.871042in}{2.309455in}}{\pgfqpoint{1.874315in}{2.317355in}}{\pgfqpoint{1.874315in}{2.325591in}}%
\pgfpathcurveto{\pgfqpoint{1.874315in}{2.333827in}}{\pgfqpoint{1.871042in}{2.341727in}}{\pgfqpoint{1.865218in}{2.347551in}}%
\pgfpathcurveto{\pgfqpoint{1.859395in}{2.353375in}}{\pgfqpoint{1.851495in}{2.356647in}}{\pgfqpoint{1.843258in}{2.356647in}}%
\pgfpathcurveto{\pgfqpoint{1.835022in}{2.356647in}}{\pgfqpoint{1.827122in}{2.353375in}}{\pgfqpoint{1.821298in}{2.347551in}}%
\pgfpathcurveto{\pgfqpoint{1.815474in}{2.341727in}}{\pgfqpoint{1.812202in}{2.333827in}}{\pgfqpoint{1.812202in}{2.325591in}}%
\pgfpathcurveto{\pgfqpoint{1.812202in}{2.317355in}}{\pgfqpoint{1.815474in}{2.309455in}}{\pgfqpoint{1.821298in}{2.303631in}}%
\pgfpathcurveto{\pgfqpoint{1.827122in}{2.297807in}}{\pgfqpoint{1.835022in}{2.294534in}}{\pgfqpoint{1.843258in}{2.294534in}}%
\pgfpathclose%
\pgfusepath{stroke,fill}%
\end{pgfscope}%
\begin{pgfscope}%
\pgfpathrectangle{\pgfqpoint{0.100000in}{0.212622in}}{\pgfqpoint{3.696000in}{3.696000in}}%
\pgfusepath{clip}%
\pgfsetbuttcap%
\pgfsetroundjoin%
\definecolor{currentfill}{rgb}{0.121569,0.466667,0.705882}%
\pgfsetfillcolor{currentfill}%
\pgfsetfillopacity{0.839888}%
\pgfsetlinewidth{1.003750pt}%
\definecolor{currentstroke}{rgb}{0.121569,0.466667,0.705882}%
\pgfsetstrokecolor{currentstroke}%
\pgfsetstrokeopacity{0.839888}%
\pgfsetdash{}{0pt}%
\pgfpathmoveto{\pgfqpoint{1.843258in}{2.294534in}}%
\pgfpathcurveto{\pgfqpoint{1.851495in}{2.294534in}}{\pgfqpoint{1.859395in}{2.297807in}}{\pgfqpoint{1.865218in}{2.303631in}}%
\pgfpathcurveto{\pgfqpoint{1.871042in}{2.309455in}}{\pgfqpoint{1.874315in}{2.317355in}}{\pgfqpoint{1.874315in}{2.325591in}}%
\pgfpathcurveto{\pgfqpoint{1.874315in}{2.333827in}}{\pgfqpoint{1.871042in}{2.341727in}}{\pgfqpoint{1.865218in}{2.347551in}}%
\pgfpathcurveto{\pgfqpoint{1.859395in}{2.353375in}}{\pgfqpoint{1.851495in}{2.356647in}}{\pgfqpoint{1.843258in}{2.356647in}}%
\pgfpathcurveto{\pgfqpoint{1.835022in}{2.356647in}}{\pgfqpoint{1.827122in}{2.353375in}}{\pgfqpoint{1.821298in}{2.347551in}}%
\pgfpathcurveto{\pgfqpoint{1.815474in}{2.341727in}}{\pgfqpoint{1.812202in}{2.333827in}}{\pgfqpoint{1.812202in}{2.325591in}}%
\pgfpathcurveto{\pgfqpoint{1.812202in}{2.317355in}}{\pgfqpoint{1.815474in}{2.309455in}}{\pgfqpoint{1.821298in}{2.303631in}}%
\pgfpathcurveto{\pgfqpoint{1.827122in}{2.297807in}}{\pgfqpoint{1.835022in}{2.294534in}}{\pgfqpoint{1.843258in}{2.294534in}}%
\pgfpathclose%
\pgfusepath{stroke,fill}%
\end{pgfscope}%
\begin{pgfscope}%
\pgfpathrectangle{\pgfqpoint{0.100000in}{0.212622in}}{\pgfqpoint{3.696000in}{3.696000in}}%
\pgfusepath{clip}%
\pgfsetbuttcap%
\pgfsetroundjoin%
\definecolor{currentfill}{rgb}{0.121569,0.466667,0.705882}%
\pgfsetfillcolor{currentfill}%
\pgfsetfillopacity{0.839888}%
\pgfsetlinewidth{1.003750pt}%
\definecolor{currentstroke}{rgb}{0.121569,0.466667,0.705882}%
\pgfsetstrokecolor{currentstroke}%
\pgfsetstrokeopacity{0.839888}%
\pgfsetdash{}{0pt}%
\pgfpathmoveto{\pgfqpoint{1.843258in}{2.294534in}}%
\pgfpathcurveto{\pgfqpoint{1.851495in}{2.294534in}}{\pgfqpoint{1.859395in}{2.297807in}}{\pgfqpoint{1.865218in}{2.303631in}}%
\pgfpathcurveto{\pgfqpoint{1.871042in}{2.309455in}}{\pgfqpoint{1.874315in}{2.317355in}}{\pgfqpoint{1.874315in}{2.325591in}}%
\pgfpathcurveto{\pgfqpoint{1.874315in}{2.333827in}}{\pgfqpoint{1.871042in}{2.341727in}}{\pgfqpoint{1.865218in}{2.347551in}}%
\pgfpathcurveto{\pgfqpoint{1.859395in}{2.353375in}}{\pgfqpoint{1.851495in}{2.356647in}}{\pgfqpoint{1.843258in}{2.356647in}}%
\pgfpathcurveto{\pgfqpoint{1.835022in}{2.356647in}}{\pgfqpoint{1.827122in}{2.353375in}}{\pgfqpoint{1.821298in}{2.347551in}}%
\pgfpathcurveto{\pgfqpoint{1.815474in}{2.341727in}}{\pgfqpoint{1.812202in}{2.333827in}}{\pgfqpoint{1.812202in}{2.325591in}}%
\pgfpathcurveto{\pgfqpoint{1.812202in}{2.317355in}}{\pgfqpoint{1.815474in}{2.309455in}}{\pgfqpoint{1.821298in}{2.303631in}}%
\pgfpathcurveto{\pgfqpoint{1.827122in}{2.297807in}}{\pgfqpoint{1.835022in}{2.294534in}}{\pgfqpoint{1.843258in}{2.294534in}}%
\pgfpathclose%
\pgfusepath{stroke,fill}%
\end{pgfscope}%
\begin{pgfscope}%
\pgfpathrectangle{\pgfqpoint{0.100000in}{0.212622in}}{\pgfqpoint{3.696000in}{3.696000in}}%
\pgfusepath{clip}%
\pgfsetbuttcap%
\pgfsetroundjoin%
\definecolor{currentfill}{rgb}{0.121569,0.466667,0.705882}%
\pgfsetfillcolor{currentfill}%
\pgfsetfillopacity{0.839888}%
\pgfsetlinewidth{1.003750pt}%
\definecolor{currentstroke}{rgb}{0.121569,0.466667,0.705882}%
\pgfsetstrokecolor{currentstroke}%
\pgfsetstrokeopacity{0.839888}%
\pgfsetdash{}{0pt}%
\pgfpathmoveto{\pgfqpoint{1.843258in}{2.294534in}}%
\pgfpathcurveto{\pgfqpoint{1.851495in}{2.294534in}}{\pgfqpoint{1.859395in}{2.297807in}}{\pgfqpoint{1.865218in}{2.303631in}}%
\pgfpathcurveto{\pgfqpoint{1.871042in}{2.309455in}}{\pgfqpoint{1.874315in}{2.317355in}}{\pgfqpoint{1.874315in}{2.325591in}}%
\pgfpathcurveto{\pgfqpoint{1.874315in}{2.333827in}}{\pgfqpoint{1.871042in}{2.341727in}}{\pgfqpoint{1.865218in}{2.347551in}}%
\pgfpathcurveto{\pgfqpoint{1.859395in}{2.353375in}}{\pgfqpoint{1.851495in}{2.356647in}}{\pgfqpoint{1.843258in}{2.356647in}}%
\pgfpathcurveto{\pgfqpoint{1.835022in}{2.356647in}}{\pgfqpoint{1.827122in}{2.353375in}}{\pgfqpoint{1.821298in}{2.347551in}}%
\pgfpathcurveto{\pgfqpoint{1.815474in}{2.341727in}}{\pgfqpoint{1.812202in}{2.333827in}}{\pgfqpoint{1.812202in}{2.325591in}}%
\pgfpathcurveto{\pgfqpoint{1.812202in}{2.317355in}}{\pgfqpoint{1.815474in}{2.309455in}}{\pgfqpoint{1.821298in}{2.303631in}}%
\pgfpathcurveto{\pgfqpoint{1.827122in}{2.297807in}}{\pgfqpoint{1.835022in}{2.294534in}}{\pgfqpoint{1.843258in}{2.294534in}}%
\pgfpathclose%
\pgfusepath{stroke,fill}%
\end{pgfscope}%
\begin{pgfscope}%
\pgfpathrectangle{\pgfqpoint{0.100000in}{0.212622in}}{\pgfqpoint{3.696000in}{3.696000in}}%
\pgfusepath{clip}%
\pgfsetbuttcap%
\pgfsetroundjoin%
\definecolor{currentfill}{rgb}{0.121569,0.466667,0.705882}%
\pgfsetfillcolor{currentfill}%
\pgfsetfillopacity{0.839888}%
\pgfsetlinewidth{1.003750pt}%
\definecolor{currentstroke}{rgb}{0.121569,0.466667,0.705882}%
\pgfsetstrokecolor{currentstroke}%
\pgfsetstrokeopacity{0.839888}%
\pgfsetdash{}{0pt}%
\pgfpathmoveto{\pgfqpoint{1.843258in}{2.294534in}}%
\pgfpathcurveto{\pgfqpoint{1.851495in}{2.294534in}}{\pgfqpoint{1.859395in}{2.297807in}}{\pgfqpoint{1.865218in}{2.303631in}}%
\pgfpathcurveto{\pgfqpoint{1.871042in}{2.309455in}}{\pgfqpoint{1.874315in}{2.317355in}}{\pgfqpoint{1.874315in}{2.325591in}}%
\pgfpathcurveto{\pgfqpoint{1.874315in}{2.333827in}}{\pgfqpoint{1.871042in}{2.341727in}}{\pgfqpoint{1.865218in}{2.347551in}}%
\pgfpathcurveto{\pgfqpoint{1.859395in}{2.353375in}}{\pgfqpoint{1.851495in}{2.356647in}}{\pgfqpoint{1.843258in}{2.356647in}}%
\pgfpathcurveto{\pgfqpoint{1.835022in}{2.356647in}}{\pgfqpoint{1.827122in}{2.353375in}}{\pgfqpoint{1.821298in}{2.347551in}}%
\pgfpathcurveto{\pgfqpoint{1.815474in}{2.341727in}}{\pgfqpoint{1.812202in}{2.333827in}}{\pgfqpoint{1.812202in}{2.325591in}}%
\pgfpathcurveto{\pgfqpoint{1.812202in}{2.317355in}}{\pgfqpoint{1.815474in}{2.309455in}}{\pgfqpoint{1.821298in}{2.303631in}}%
\pgfpathcurveto{\pgfqpoint{1.827122in}{2.297807in}}{\pgfqpoint{1.835022in}{2.294534in}}{\pgfqpoint{1.843258in}{2.294534in}}%
\pgfpathclose%
\pgfusepath{stroke,fill}%
\end{pgfscope}%
\begin{pgfscope}%
\pgfpathrectangle{\pgfqpoint{0.100000in}{0.212622in}}{\pgfqpoint{3.696000in}{3.696000in}}%
\pgfusepath{clip}%
\pgfsetbuttcap%
\pgfsetroundjoin%
\definecolor{currentfill}{rgb}{0.121569,0.466667,0.705882}%
\pgfsetfillcolor{currentfill}%
\pgfsetfillopacity{0.839888}%
\pgfsetlinewidth{1.003750pt}%
\definecolor{currentstroke}{rgb}{0.121569,0.466667,0.705882}%
\pgfsetstrokecolor{currentstroke}%
\pgfsetstrokeopacity{0.839888}%
\pgfsetdash{}{0pt}%
\pgfpathmoveto{\pgfqpoint{1.843258in}{2.294534in}}%
\pgfpathcurveto{\pgfqpoint{1.851495in}{2.294534in}}{\pgfqpoint{1.859395in}{2.297807in}}{\pgfqpoint{1.865218in}{2.303631in}}%
\pgfpathcurveto{\pgfqpoint{1.871042in}{2.309455in}}{\pgfqpoint{1.874315in}{2.317355in}}{\pgfqpoint{1.874315in}{2.325591in}}%
\pgfpathcurveto{\pgfqpoint{1.874315in}{2.333827in}}{\pgfqpoint{1.871042in}{2.341727in}}{\pgfqpoint{1.865218in}{2.347551in}}%
\pgfpathcurveto{\pgfqpoint{1.859395in}{2.353375in}}{\pgfqpoint{1.851495in}{2.356647in}}{\pgfqpoint{1.843258in}{2.356647in}}%
\pgfpathcurveto{\pgfqpoint{1.835022in}{2.356647in}}{\pgfqpoint{1.827122in}{2.353375in}}{\pgfqpoint{1.821298in}{2.347551in}}%
\pgfpathcurveto{\pgfqpoint{1.815474in}{2.341727in}}{\pgfqpoint{1.812202in}{2.333827in}}{\pgfqpoint{1.812202in}{2.325591in}}%
\pgfpathcurveto{\pgfqpoint{1.812202in}{2.317355in}}{\pgfqpoint{1.815474in}{2.309455in}}{\pgfqpoint{1.821298in}{2.303631in}}%
\pgfpathcurveto{\pgfqpoint{1.827122in}{2.297807in}}{\pgfqpoint{1.835022in}{2.294534in}}{\pgfqpoint{1.843258in}{2.294534in}}%
\pgfpathclose%
\pgfusepath{stroke,fill}%
\end{pgfscope}%
\begin{pgfscope}%
\pgfpathrectangle{\pgfqpoint{0.100000in}{0.212622in}}{\pgfqpoint{3.696000in}{3.696000in}}%
\pgfusepath{clip}%
\pgfsetbuttcap%
\pgfsetroundjoin%
\definecolor{currentfill}{rgb}{0.121569,0.466667,0.705882}%
\pgfsetfillcolor{currentfill}%
\pgfsetfillopacity{0.839888}%
\pgfsetlinewidth{1.003750pt}%
\definecolor{currentstroke}{rgb}{0.121569,0.466667,0.705882}%
\pgfsetstrokecolor{currentstroke}%
\pgfsetstrokeopacity{0.839888}%
\pgfsetdash{}{0pt}%
\pgfpathmoveto{\pgfqpoint{1.843258in}{2.294534in}}%
\pgfpathcurveto{\pgfqpoint{1.851495in}{2.294534in}}{\pgfqpoint{1.859395in}{2.297807in}}{\pgfqpoint{1.865218in}{2.303631in}}%
\pgfpathcurveto{\pgfqpoint{1.871042in}{2.309455in}}{\pgfqpoint{1.874315in}{2.317355in}}{\pgfqpoint{1.874315in}{2.325591in}}%
\pgfpathcurveto{\pgfqpoint{1.874315in}{2.333827in}}{\pgfqpoint{1.871042in}{2.341727in}}{\pgfqpoint{1.865218in}{2.347551in}}%
\pgfpathcurveto{\pgfqpoint{1.859395in}{2.353375in}}{\pgfqpoint{1.851495in}{2.356647in}}{\pgfqpoint{1.843258in}{2.356647in}}%
\pgfpathcurveto{\pgfqpoint{1.835022in}{2.356647in}}{\pgfqpoint{1.827122in}{2.353375in}}{\pgfqpoint{1.821298in}{2.347551in}}%
\pgfpathcurveto{\pgfqpoint{1.815474in}{2.341727in}}{\pgfqpoint{1.812202in}{2.333827in}}{\pgfqpoint{1.812202in}{2.325591in}}%
\pgfpathcurveto{\pgfqpoint{1.812202in}{2.317355in}}{\pgfqpoint{1.815474in}{2.309455in}}{\pgfqpoint{1.821298in}{2.303631in}}%
\pgfpathcurveto{\pgfqpoint{1.827122in}{2.297807in}}{\pgfqpoint{1.835022in}{2.294534in}}{\pgfqpoint{1.843258in}{2.294534in}}%
\pgfpathclose%
\pgfusepath{stroke,fill}%
\end{pgfscope}%
\begin{pgfscope}%
\pgfpathrectangle{\pgfqpoint{0.100000in}{0.212622in}}{\pgfqpoint{3.696000in}{3.696000in}}%
\pgfusepath{clip}%
\pgfsetbuttcap%
\pgfsetroundjoin%
\definecolor{currentfill}{rgb}{0.121569,0.466667,0.705882}%
\pgfsetfillcolor{currentfill}%
\pgfsetfillopacity{0.839888}%
\pgfsetlinewidth{1.003750pt}%
\definecolor{currentstroke}{rgb}{0.121569,0.466667,0.705882}%
\pgfsetstrokecolor{currentstroke}%
\pgfsetstrokeopacity{0.839888}%
\pgfsetdash{}{0pt}%
\pgfpathmoveto{\pgfqpoint{1.843258in}{2.294534in}}%
\pgfpathcurveto{\pgfqpoint{1.851495in}{2.294534in}}{\pgfqpoint{1.859395in}{2.297807in}}{\pgfqpoint{1.865218in}{2.303631in}}%
\pgfpathcurveto{\pgfqpoint{1.871042in}{2.309455in}}{\pgfqpoint{1.874315in}{2.317355in}}{\pgfqpoint{1.874315in}{2.325591in}}%
\pgfpathcurveto{\pgfqpoint{1.874315in}{2.333827in}}{\pgfqpoint{1.871042in}{2.341727in}}{\pgfqpoint{1.865218in}{2.347551in}}%
\pgfpathcurveto{\pgfqpoint{1.859395in}{2.353375in}}{\pgfqpoint{1.851495in}{2.356647in}}{\pgfqpoint{1.843258in}{2.356647in}}%
\pgfpathcurveto{\pgfqpoint{1.835022in}{2.356647in}}{\pgfqpoint{1.827122in}{2.353375in}}{\pgfqpoint{1.821298in}{2.347551in}}%
\pgfpathcurveto{\pgfqpoint{1.815474in}{2.341727in}}{\pgfqpoint{1.812202in}{2.333827in}}{\pgfqpoint{1.812202in}{2.325591in}}%
\pgfpathcurveto{\pgfqpoint{1.812202in}{2.317355in}}{\pgfqpoint{1.815474in}{2.309455in}}{\pgfqpoint{1.821298in}{2.303631in}}%
\pgfpathcurveto{\pgfqpoint{1.827122in}{2.297807in}}{\pgfqpoint{1.835022in}{2.294534in}}{\pgfqpoint{1.843258in}{2.294534in}}%
\pgfpathclose%
\pgfusepath{stroke,fill}%
\end{pgfscope}%
\begin{pgfscope}%
\pgfpathrectangle{\pgfqpoint{0.100000in}{0.212622in}}{\pgfqpoint{3.696000in}{3.696000in}}%
\pgfusepath{clip}%
\pgfsetbuttcap%
\pgfsetroundjoin%
\definecolor{currentfill}{rgb}{0.121569,0.466667,0.705882}%
\pgfsetfillcolor{currentfill}%
\pgfsetfillopacity{0.839888}%
\pgfsetlinewidth{1.003750pt}%
\definecolor{currentstroke}{rgb}{0.121569,0.466667,0.705882}%
\pgfsetstrokecolor{currentstroke}%
\pgfsetstrokeopacity{0.839888}%
\pgfsetdash{}{0pt}%
\pgfpathmoveto{\pgfqpoint{1.843258in}{2.294534in}}%
\pgfpathcurveto{\pgfqpoint{1.851495in}{2.294534in}}{\pgfqpoint{1.859395in}{2.297807in}}{\pgfqpoint{1.865218in}{2.303631in}}%
\pgfpathcurveto{\pgfqpoint{1.871042in}{2.309455in}}{\pgfqpoint{1.874315in}{2.317355in}}{\pgfqpoint{1.874315in}{2.325591in}}%
\pgfpathcurveto{\pgfqpoint{1.874315in}{2.333827in}}{\pgfqpoint{1.871042in}{2.341727in}}{\pgfqpoint{1.865218in}{2.347551in}}%
\pgfpathcurveto{\pgfqpoint{1.859395in}{2.353375in}}{\pgfqpoint{1.851495in}{2.356647in}}{\pgfqpoint{1.843258in}{2.356647in}}%
\pgfpathcurveto{\pgfqpoint{1.835022in}{2.356647in}}{\pgfqpoint{1.827122in}{2.353375in}}{\pgfqpoint{1.821298in}{2.347551in}}%
\pgfpathcurveto{\pgfqpoint{1.815474in}{2.341727in}}{\pgfqpoint{1.812202in}{2.333827in}}{\pgfqpoint{1.812202in}{2.325591in}}%
\pgfpathcurveto{\pgfqpoint{1.812202in}{2.317355in}}{\pgfqpoint{1.815474in}{2.309455in}}{\pgfqpoint{1.821298in}{2.303631in}}%
\pgfpathcurveto{\pgfqpoint{1.827122in}{2.297807in}}{\pgfqpoint{1.835022in}{2.294534in}}{\pgfqpoint{1.843258in}{2.294534in}}%
\pgfpathclose%
\pgfusepath{stroke,fill}%
\end{pgfscope}%
\begin{pgfscope}%
\pgfpathrectangle{\pgfqpoint{0.100000in}{0.212622in}}{\pgfqpoint{3.696000in}{3.696000in}}%
\pgfusepath{clip}%
\pgfsetbuttcap%
\pgfsetroundjoin%
\definecolor{currentfill}{rgb}{0.121569,0.466667,0.705882}%
\pgfsetfillcolor{currentfill}%
\pgfsetfillopacity{0.839888}%
\pgfsetlinewidth{1.003750pt}%
\definecolor{currentstroke}{rgb}{0.121569,0.466667,0.705882}%
\pgfsetstrokecolor{currentstroke}%
\pgfsetstrokeopacity{0.839888}%
\pgfsetdash{}{0pt}%
\pgfpathmoveto{\pgfqpoint{1.843258in}{2.294534in}}%
\pgfpathcurveto{\pgfqpoint{1.851495in}{2.294534in}}{\pgfqpoint{1.859395in}{2.297807in}}{\pgfqpoint{1.865218in}{2.303631in}}%
\pgfpathcurveto{\pgfqpoint{1.871042in}{2.309455in}}{\pgfqpoint{1.874315in}{2.317355in}}{\pgfqpoint{1.874315in}{2.325591in}}%
\pgfpathcurveto{\pgfqpoint{1.874315in}{2.333827in}}{\pgfqpoint{1.871042in}{2.341727in}}{\pgfqpoint{1.865218in}{2.347551in}}%
\pgfpathcurveto{\pgfqpoint{1.859395in}{2.353375in}}{\pgfqpoint{1.851495in}{2.356647in}}{\pgfqpoint{1.843258in}{2.356647in}}%
\pgfpathcurveto{\pgfqpoint{1.835022in}{2.356647in}}{\pgfqpoint{1.827122in}{2.353375in}}{\pgfqpoint{1.821298in}{2.347551in}}%
\pgfpathcurveto{\pgfqpoint{1.815474in}{2.341727in}}{\pgfqpoint{1.812202in}{2.333827in}}{\pgfqpoint{1.812202in}{2.325591in}}%
\pgfpathcurveto{\pgfqpoint{1.812202in}{2.317355in}}{\pgfqpoint{1.815474in}{2.309455in}}{\pgfqpoint{1.821298in}{2.303631in}}%
\pgfpathcurveto{\pgfqpoint{1.827122in}{2.297807in}}{\pgfqpoint{1.835022in}{2.294534in}}{\pgfqpoint{1.843258in}{2.294534in}}%
\pgfpathclose%
\pgfusepath{stroke,fill}%
\end{pgfscope}%
\begin{pgfscope}%
\pgfpathrectangle{\pgfqpoint{0.100000in}{0.212622in}}{\pgfqpoint{3.696000in}{3.696000in}}%
\pgfusepath{clip}%
\pgfsetbuttcap%
\pgfsetroundjoin%
\definecolor{currentfill}{rgb}{0.121569,0.466667,0.705882}%
\pgfsetfillcolor{currentfill}%
\pgfsetfillopacity{0.839888}%
\pgfsetlinewidth{1.003750pt}%
\definecolor{currentstroke}{rgb}{0.121569,0.466667,0.705882}%
\pgfsetstrokecolor{currentstroke}%
\pgfsetstrokeopacity{0.839888}%
\pgfsetdash{}{0pt}%
\pgfpathmoveto{\pgfqpoint{1.843258in}{2.294534in}}%
\pgfpathcurveto{\pgfqpoint{1.851495in}{2.294534in}}{\pgfqpoint{1.859395in}{2.297807in}}{\pgfqpoint{1.865218in}{2.303631in}}%
\pgfpathcurveto{\pgfqpoint{1.871042in}{2.309455in}}{\pgfqpoint{1.874315in}{2.317355in}}{\pgfqpoint{1.874315in}{2.325591in}}%
\pgfpathcurveto{\pgfqpoint{1.874315in}{2.333827in}}{\pgfqpoint{1.871042in}{2.341727in}}{\pgfqpoint{1.865218in}{2.347551in}}%
\pgfpathcurveto{\pgfqpoint{1.859395in}{2.353375in}}{\pgfqpoint{1.851495in}{2.356647in}}{\pgfqpoint{1.843258in}{2.356647in}}%
\pgfpathcurveto{\pgfqpoint{1.835022in}{2.356647in}}{\pgfqpoint{1.827122in}{2.353375in}}{\pgfqpoint{1.821298in}{2.347551in}}%
\pgfpathcurveto{\pgfqpoint{1.815474in}{2.341727in}}{\pgfqpoint{1.812202in}{2.333827in}}{\pgfqpoint{1.812202in}{2.325591in}}%
\pgfpathcurveto{\pgfqpoint{1.812202in}{2.317355in}}{\pgfqpoint{1.815474in}{2.309455in}}{\pgfqpoint{1.821298in}{2.303631in}}%
\pgfpathcurveto{\pgfqpoint{1.827122in}{2.297807in}}{\pgfqpoint{1.835022in}{2.294534in}}{\pgfqpoint{1.843258in}{2.294534in}}%
\pgfpathclose%
\pgfusepath{stroke,fill}%
\end{pgfscope}%
\begin{pgfscope}%
\pgfpathrectangle{\pgfqpoint{0.100000in}{0.212622in}}{\pgfqpoint{3.696000in}{3.696000in}}%
\pgfusepath{clip}%
\pgfsetbuttcap%
\pgfsetroundjoin%
\definecolor{currentfill}{rgb}{0.121569,0.466667,0.705882}%
\pgfsetfillcolor{currentfill}%
\pgfsetfillopacity{0.839888}%
\pgfsetlinewidth{1.003750pt}%
\definecolor{currentstroke}{rgb}{0.121569,0.466667,0.705882}%
\pgfsetstrokecolor{currentstroke}%
\pgfsetstrokeopacity{0.839888}%
\pgfsetdash{}{0pt}%
\pgfpathmoveto{\pgfqpoint{1.843258in}{2.294534in}}%
\pgfpathcurveto{\pgfqpoint{1.851495in}{2.294534in}}{\pgfqpoint{1.859395in}{2.297807in}}{\pgfqpoint{1.865218in}{2.303631in}}%
\pgfpathcurveto{\pgfqpoint{1.871042in}{2.309455in}}{\pgfqpoint{1.874315in}{2.317355in}}{\pgfqpoint{1.874315in}{2.325591in}}%
\pgfpathcurveto{\pgfqpoint{1.874315in}{2.333827in}}{\pgfqpoint{1.871042in}{2.341727in}}{\pgfqpoint{1.865218in}{2.347551in}}%
\pgfpathcurveto{\pgfqpoint{1.859395in}{2.353375in}}{\pgfqpoint{1.851495in}{2.356647in}}{\pgfqpoint{1.843258in}{2.356647in}}%
\pgfpathcurveto{\pgfqpoint{1.835022in}{2.356647in}}{\pgfqpoint{1.827122in}{2.353375in}}{\pgfqpoint{1.821298in}{2.347551in}}%
\pgfpathcurveto{\pgfqpoint{1.815474in}{2.341727in}}{\pgfqpoint{1.812202in}{2.333827in}}{\pgfqpoint{1.812202in}{2.325591in}}%
\pgfpathcurveto{\pgfqpoint{1.812202in}{2.317355in}}{\pgfqpoint{1.815474in}{2.309455in}}{\pgfqpoint{1.821298in}{2.303631in}}%
\pgfpathcurveto{\pgfqpoint{1.827122in}{2.297807in}}{\pgfqpoint{1.835022in}{2.294534in}}{\pgfqpoint{1.843258in}{2.294534in}}%
\pgfpathclose%
\pgfusepath{stroke,fill}%
\end{pgfscope}%
\begin{pgfscope}%
\pgfpathrectangle{\pgfqpoint{0.100000in}{0.212622in}}{\pgfqpoint{3.696000in}{3.696000in}}%
\pgfusepath{clip}%
\pgfsetbuttcap%
\pgfsetroundjoin%
\definecolor{currentfill}{rgb}{0.121569,0.466667,0.705882}%
\pgfsetfillcolor{currentfill}%
\pgfsetfillopacity{0.839888}%
\pgfsetlinewidth{1.003750pt}%
\definecolor{currentstroke}{rgb}{0.121569,0.466667,0.705882}%
\pgfsetstrokecolor{currentstroke}%
\pgfsetstrokeopacity{0.839888}%
\pgfsetdash{}{0pt}%
\pgfpathmoveto{\pgfqpoint{1.843258in}{2.294534in}}%
\pgfpathcurveto{\pgfqpoint{1.851495in}{2.294534in}}{\pgfqpoint{1.859395in}{2.297807in}}{\pgfqpoint{1.865218in}{2.303631in}}%
\pgfpathcurveto{\pgfqpoint{1.871042in}{2.309455in}}{\pgfqpoint{1.874315in}{2.317355in}}{\pgfqpoint{1.874315in}{2.325591in}}%
\pgfpathcurveto{\pgfqpoint{1.874315in}{2.333827in}}{\pgfqpoint{1.871042in}{2.341727in}}{\pgfqpoint{1.865218in}{2.347551in}}%
\pgfpathcurveto{\pgfqpoint{1.859395in}{2.353375in}}{\pgfqpoint{1.851495in}{2.356647in}}{\pgfqpoint{1.843258in}{2.356647in}}%
\pgfpathcurveto{\pgfqpoint{1.835022in}{2.356647in}}{\pgfqpoint{1.827122in}{2.353375in}}{\pgfqpoint{1.821298in}{2.347551in}}%
\pgfpathcurveto{\pgfqpoint{1.815474in}{2.341727in}}{\pgfqpoint{1.812202in}{2.333827in}}{\pgfqpoint{1.812202in}{2.325591in}}%
\pgfpathcurveto{\pgfqpoint{1.812202in}{2.317355in}}{\pgfqpoint{1.815474in}{2.309455in}}{\pgfqpoint{1.821298in}{2.303631in}}%
\pgfpathcurveto{\pgfqpoint{1.827122in}{2.297807in}}{\pgfqpoint{1.835022in}{2.294534in}}{\pgfqpoint{1.843258in}{2.294534in}}%
\pgfpathclose%
\pgfusepath{stroke,fill}%
\end{pgfscope}%
\begin{pgfscope}%
\pgfpathrectangle{\pgfqpoint{0.100000in}{0.212622in}}{\pgfqpoint{3.696000in}{3.696000in}}%
\pgfusepath{clip}%
\pgfsetbuttcap%
\pgfsetroundjoin%
\definecolor{currentfill}{rgb}{0.121569,0.466667,0.705882}%
\pgfsetfillcolor{currentfill}%
\pgfsetfillopacity{0.839888}%
\pgfsetlinewidth{1.003750pt}%
\definecolor{currentstroke}{rgb}{0.121569,0.466667,0.705882}%
\pgfsetstrokecolor{currentstroke}%
\pgfsetstrokeopacity{0.839888}%
\pgfsetdash{}{0pt}%
\pgfpathmoveto{\pgfqpoint{1.843258in}{2.294534in}}%
\pgfpathcurveto{\pgfqpoint{1.851495in}{2.294534in}}{\pgfqpoint{1.859395in}{2.297807in}}{\pgfqpoint{1.865218in}{2.303631in}}%
\pgfpathcurveto{\pgfqpoint{1.871042in}{2.309455in}}{\pgfqpoint{1.874315in}{2.317355in}}{\pgfqpoint{1.874315in}{2.325591in}}%
\pgfpathcurveto{\pgfqpoint{1.874315in}{2.333827in}}{\pgfqpoint{1.871042in}{2.341727in}}{\pgfqpoint{1.865218in}{2.347551in}}%
\pgfpathcurveto{\pgfqpoint{1.859395in}{2.353375in}}{\pgfqpoint{1.851495in}{2.356647in}}{\pgfqpoint{1.843258in}{2.356647in}}%
\pgfpathcurveto{\pgfqpoint{1.835022in}{2.356647in}}{\pgfqpoint{1.827122in}{2.353375in}}{\pgfqpoint{1.821298in}{2.347551in}}%
\pgfpathcurveto{\pgfqpoint{1.815474in}{2.341727in}}{\pgfqpoint{1.812202in}{2.333827in}}{\pgfqpoint{1.812202in}{2.325591in}}%
\pgfpathcurveto{\pgfqpoint{1.812202in}{2.317355in}}{\pgfqpoint{1.815474in}{2.309455in}}{\pgfqpoint{1.821298in}{2.303631in}}%
\pgfpathcurveto{\pgfqpoint{1.827122in}{2.297807in}}{\pgfqpoint{1.835022in}{2.294534in}}{\pgfqpoint{1.843258in}{2.294534in}}%
\pgfpathclose%
\pgfusepath{stroke,fill}%
\end{pgfscope}%
\begin{pgfscope}%
\pgfpathrectangle{\pgfqpoint{0.100000in}{0.212622in}}{\pgfqpoint{3.696000in}{3.696000in}}%
\pgfusepath{clip}%
\pgfsetbuttcap%
\pgfsetroundjoin%
\definecolor{currentfill}{rgb}{0.121569,0.466667,0.705882}%
\pgfsetfillcolor{currentfill}%
\pgfsetfillopacity{0.839888}%
\pgfsetlinewidth{1.003750pt}%
\definecolor{currentstroke}{rgb}{0.121569,0.466667,0.705882}%
\pgfsetstrokecolor{currentstroke}%
\pgfsetstrokeopacity{0.839888}%
\pgfsetdash{}{0pt}%
\pgfpathmoveto{\pgfqpoint{1.843258in}{2.294534in}}%
\pgfpathcurveto{\pgfqpoint{1.851495in}{2.294534in}}{\pgfqpoint{1.859395in}{2.297807in}}{\pgfqpoint{1.865218in}{2.303631in}}%
\pgfpathcurveto{\pgfqpoint{1.871042in}{2.309455in}}{\pgfqpoint{1.874315in}{2.317355in}}{\pgfqpoint{1.874315in}{2.325591in}}%
\pgfpathcurveto{\pgfqpoint{1.874315in}{2.333827in}}{\pgfqpoint{1.871042in}{2.341727in}}{\pgfqpoint{1.865218in}{2.347551in}}%
\pgfpathcurveto{\pgfqpoint{1.859395in}{2.353375in}}{\pgfqpoint{1.851495in}{2.356647in}}{\pgfqpoint{1.843258in}{2.356647in}}%
\pgfpathcurveto{\pgfqpoint{1.835022in}{2.356647in}}{\pgfqpoint{1.827122in}{2.353375in}}{\pgfqpoint{1.821298in}{2.347551in}}%
\pgfpathcurveto{\pgfqpoint{1.815474in}{2.341727in}}{\pgfqpoint{1.812202in}{2.333827in}}{\pgfqpoint{1.812202in}{2.325591in}}%
\pgfpathcurveto{\pgfqpoint{1.812202in}{2.317355in}}{\pgfqpoint{1.815474in}{2.309455in}}{\pgfqpoint{1.821298in}{2.303631in}}%
\pgfpathcurveto{\pgfqpoint{1.827122in}{2.297807in}}{\pgfqpoint{1.835022in}{2.294534in}}{\pgfqpoint{1.843258in}{2.294534in}}%
\pgfpathclose%
\pgfusepath{stroke,fill}%
\end{pgfscope}%
\begin{pgfscope}%
\pgfpathrectangle{\pgfqpoint{0.100000in}{0.212622in}}{\pgfqpoint{3.696000in}{3.696000in}}%
\pgfusepath{clip}%
\pgfsetbuttcap%
\pgfsetroundjoin%
\definecolor{currentfill}{rgb}{0.121569,0.466667,0.705882}%
\pgfsetfillcolor{currentfill}%
\pgfsetfillopacity{0.839888}%
\pgfsetlinewidth{1.003750pt}%
\definecolor{currentstroke}{rgb}{0.121569,0.466667,0.705882}%
\pgfsetstrokecolor{currentstroke}%
\pgfsetstrokeopacity{0.839888}%
\pgfsetdash{}{0pt}%
\pgfpathmoveto{\pgfqpoint{1.843258in}{2.294534in}}%
\pgfpathcurveto{\pgfqpoint{1.851495in}{2.294534in}}{\pgfqpoint{1.859395in}{2.297807in}}{\pgfqpoint{1.865218in}{2.303631in}}%
\pgfpathcurveto{\pgfqpoint{1.871042in}{2.309455in}}{\pgfqpoint{1.874315in}{2.317355in}}{\pgfqpoint{1.874315in}{2.325591in}}%
\pgfpathcurveto{\pgfqpoint{1.874315in}{2.333827in}}{\pgfqpoint{1.871042in}{2.341727in}}{\pgfqpoint{1.865218in}{2.347551in}}%
\pgfpathcurveto{\pgfqpoint{1.859395in}{2.353375in}}{\pgfqpoint{1.851495in}{2.356647in}}{\pgfqpoint{1.843258in}{2.356647in}}%
\pgfpathcurveto{\pgfqpoint{1.835022in}{2.356647in}}{\pgfqpoint{1.827122in}{2.353375in}}{\pgfqpoint{1.821298in}{2.347551in}}%
\pgfpathcurveto{\pgfqpoint{1.815474in}{2.341727in}}{\pgfqpoint{1.812202in}{2.333827in}}{\pgfqpoint{1.812202in}{2.325591in}}%
\pgfpathcurveto{\pgfqpoint{1.812202in}{2.317355in}}{\pgfqpoint{1.815474in}{2.309455in}}{\pgfqpoint{1.821298in}{2.303631in}}%
\pgfpathcurveto{\pgfqpoint{1.827122in}{2.297807in}}{\pgfqpoint{1.835022in}{2.294534in}}{\pgfqpoint{1.843258in}{2.294534in}}%
\pgfpathclose%
\pgfusepath{stroke,fill}%
\end{pgfscope}%
\begin{pgfscope}%
\pgfpathrectangle{\pgfqpoint{0.100000in}{0.212622in}}{\pgfqpoint{3.696000in}{3.696000in}}%
\pgfusepath{clip}%
\pgfsetbuttcap%
\pgfsetroundjoin%
\definecolor{currentfill}{rgb}{0.121569,0.466667,0.705882}%
\pgfsetfillcolor{currentfill}%
\pgfsetfillopacity{0.839972}%
\pgfsetlinewidth{1.003750pt}%
\definecolor{currentstroke}{rgb}{0.121569,0.466667,0.705882}%
\pgfsetstrokecolor{currentstroke}%
\pgfsetstrokeopacity{0.839972}%
\pgfsetdash{}{0pt}%
\pgfpathmoveto{\pgfqpoint{1.843098in}{2.294109in}}%
\pgfpathcurveto{\pgfqpoint{1.851334in}{2.294109in}}{\pgfqpoint{1.859234in}{2.297382in}}{\pgfqpoint{1.865058in}{2.303206in}}%
\pgfpathcurveto{\pgfqpoint{1.870882in}{2.309029in}}{\pgfqpoint{1.874154in}{2.316930in}}{\pgfqpoint{1.874154in}{2.325166in}}%
\pgfpathcurveto{\pgfqpoint{1.874154in}{2.333402in}}{\pgfqpoint{1.870882in}{2.341302in}}{\pgfqpoint{1.865058in}{2.347126in}}%
\pgfpathcurveto{\pgfqpoint{1.859234in}{2.352950in}}{\pgfqpoint{1.851334in}{2.356222in}}{\pgfqpoint{1.843098in}{2.356222in}}%
\pgfpathcurveto{\pgfqpoint{1.834861in}{2.356222in}}{\pgfqpoint{1.826961in}{2.352950in}}{\pgfqpoint{1.821137in}{2.347126in}}%
\pgfpathcurveto{\pgfqpoint{1.815313in}{2.341302in}}{\pgfqpoint{1.812041in}{2.333402in}}{\pgfqpoint{1.812041in}{2.325166in}}%
\pgfpathcurveto{\pgfqpoint{1.812041in}{2.316930in}}{\pgfqpoint{1.815313in}{2.309029in}}{\pgfqpoint{1.821137in}{2.303206in}}%
\pgfpathcurveto{\pgfqpoint{1.826961in}{2.297382in}}{\pgfqpoint{1.834861in}{2.294109in}}{\pgfqpoint{1.843098in}{2.294109in}}%
\pgfpathclose%
\pgfusepath{stroke,fill}%
\end{pgfscope}%
\begin{pgfscope}%
\pgfpathrectangle{\pgfqpoint{0.100000in}{0.212622in}}{\pgfqpoint{3.696000in}{3.696000in}}%
\pgfusepath{clip}%
\pgfsetbuttcap%
\pgfsetroundjoin%
\definecolor{currentfill}{rgb}{0.121569,0.466667,0.705882}%
\pgfsetfillcolor{currentfill}%
\pgfsetfillopacity{0.840019}%
\pgfsetlinewidth{1.003750pt}%
\definecolor{currentstroke}{rgb}{0.121569,0.466667,0.705882}%
\pgfsetstrokecolor{currentstroke}%
\pgfsetstrokeopacity{0.840019}%
\pgfsetdash{}{0pt}%
\pgfpathmoveto{\pgfqpoint{1.842982in}{2.293900in}}%
\pgfpathcurveto{\pgfqpoint{1.851218in}{2.293900in}}{\pgfqpoint{1.859118in}{2.297172in}}{\pgfqpoint{1.864942in}{2.302996in}}%
\pgfpathcurveto{\pgfqpoint{1.870766in}{2.308820in}}{\pgfqpoint{1.874039in}{2.316720in}}{\pgfqpoint{1.874039in}{2.324956in}}%
\pgfpathcurveto{\pgfqpoint{1.874039in}{2.333193in}}{\pgfqpoint{1.870766in}{2.341093in}}{\pgfqpoint{1.864942in}{2.346917in}}%
\pgfpathcurveto{\pgfqpoint{1.859118in}{2.352741in}}{\pgfqpoint{1.851218in}{2.356013in}}{\pgfqpoint{1.842982in}{2.356013in}}%
\pgfpathcurveto{\pgfqpoint{1.834746in}{2.356013in}}{\pgfqpoint{1.826846in}{2.352741in}}{\pgfqpoint{1.821022in}{2.346917in}}%
\pgfpathcurveto{\pgfqpoint{1.815198in}{2.341093in}}{\pgfqpoint{1.811926in}{2.333193in}}{\pgfqpoint{1.811926in}{2.324956in}}%
\pgfpathcurveto{\pgfqpoint{1.811926in}{2.316720in}}{\pgfqpoint{1.815198in}{2.308820in}}{\pgfqpoint{1.821022in}{2.302996in}}%
\pgfpathcurveto{\pgfqpoint{1.826846in}{2.297172in}}{\pgfqpoint{1.834746in}{2.293900in}}{\pgfqpoint{1.842982in}{2.293900in}}%
\pgfpathclose%
\pgfusepath{stroke,fill}%
\end{pgfscope}%
\begin{pgfscope}%
\pgfpathrectangle{\pgfqpoint{0.100000in}{0.212622in}}{\pgfqpoint{3.696000in}{3.696000in}}%
\pgfusepath{clip}%
\pgfsetbuttcap%
\pgfsetroundjoin%
\definecolor{currentfill}{rgb}{0.121569,0.466667,0.705882}%
\pgfsetfillcolor{currentfill}%
\pgfsetfillopacity{0.840045}%
\pgfsetlinewidth{1.003750pt}%
\definecolor{currentstroke}{rgb}{0.121569,0.466667,0.705882}%
\pgfsetstrokecolor{currentstroke}%
\pgfsetstrokeopacity{0.840045}%
\pgfsetdash{}{0pt}%
\pgfpathmoveto{\pgfqpoint{1.842944in}{2.293769in}}%
\pgfpathcurveto{\pgfqpoint{1.851180in}{2.293769in}}{\pgfqpoint{1.859080in}{2.297042in}}{\pgfqpoint{1.864904in}{2.302866in}}%
\pgfpathcurveto{\pgfqpoint{1.870728in}{2.308690in}}{\pgfqpoint{1.874000in}{2.316590in}}{\pgfqpoint{1.874000in}{2.324826in}}%
\pgfpathcurveto{\pgfqpoint{1.874000in}{2.333062in}}{\pgfqpoint{1.870728in}{2.340962in}}{\pgfqpoint{1.864904in}{2.346786in}}%
\pgfpathcurveto{\pgfqpoint{1.859080in}{2.352610in}}{\pgfqpoint{1.851180in}{2.355882in}}{\pgfqpoint{1.842944in}{2.355882in}}%
\pgfpathcurveto{\pgfqpoint{1.834707in}{2.355882in}}{\pgfqpoint{1.826807in}{2.352610in}}{\pgfqpoint{1.820983in}{2.346786in}}%
\pgfpathcurveto{\pgfqpoint{1.815159in}{2.340962in}}{\pgfqpoint{1.811887in}{2.333062in}}{\pgfqpoint{1.811887in}{2.324826in}}%
\pgfpathcurveto{\pgfqpoint{1.811887in}{2.316590in}}{\pgfqpoint{1.815159in}{2.308690in}}{\pgfqpoint{1.820983in}{2.302866in}}%
\pgfpathcurveto{\pgfqpoint{1.826807in}{2.297042in}}{\pgfqpoint{1.834707in}{2.293769in}}{\pgfqpoint{1.842944in}{2.293769in}}%
\pgfpathclose%
\pgfusepath{stroke,fill}%
\end{pgfscope}%
\begin{pgfscope}%
\pgfpathrectangle{\pgfqpoint{0.100000in}{0.212622in}}{\pgfqpoint{3.696000in}{3.696000in}}%
\pgfusepath{clip}%
\pgfsetbuttcap%
\pgfsetroundjoin%
\definecolor{currentfill}{rgb}{0.121569,0.466667,0.705882}%
\pgfsetfillcolor{currentfill}%
\pgfsetfillopacity{0.840060}%
\pgfsetlinewidth{1.003750pt}%
\definecolor{currentstroke}{rgb}{0.121569,0.466667,0.705882}%
\pgfsetstrokecolor{currentstroke}%
\pgfsetstrokeopacity{0.840060}%
\pgfsetdash{}{0pt}%
\pgfpathmoveto{\pgfqpoint{1.842909in}{2.293711in}}%
\pgfpathcurveto{\pgfqpoint{1.851146in}{2.293711in}}{\pgfqpoint{1.859046in}{2.296984in}}{\pgfqpoint{1.864870in}{2.302807in}}%
\pgfpathcurveto{\pgfqpoint{1.870694in}{2.308631in}}{\pgfqpoint{1.873966in}{2.316531in}}{\pgfqpoint{1.873966in}{2.324768in}}%
\pgfpathcurveto{\pgfqpoint{1.873966in}{2.333004in}}{\pgfqpoint{1.870694in}{2.340904in}}{\pgfqpoint{1.864870in}{2.346728in}}%
\pgfpathcurveto{\pgfqpoint{1.859046in}{2.352552in}}{\pgfqpoint{1.851146in}{2.355824in}}{\pgfqpoint{1.842909in}{2.355824in}}%
\pgfpathcurveto{\pgfqpoint{1.834673in}{2.355824in}}{\pgfqpoint{1.826773in}{2.352552in}}{\pgfqpoint{1.820949in}{2.346728in}}%
\pgfpathcurveto{\pgfqpoint{1.815125in}{2.340904in}}{\pgfqpoint{1.811853in}{2.333004in}}{\pgfqpoint{1.811853in}{2.324768in}}%
\pgfpathcurveto{\pgfqpoint{1.811853in}{2.316531in}}{\pgfqpoint{1.815125in}{2.308631in}}{\pgfqpoint{1.820949in}{2.302807in}}%
\pgfpathcurveto{\pgfqpoint{1.826773in}{2.296984in}}{\pgfqpoint{1.834673in}{2.293711in}}{\pgfqpoint{1.842909in}{2.293711in}}%
\pgfpathclose%
\pgfusepath{stroke,fill}%
\end{pgfscope}%
\begin{pgfscope}%
\pgfpathrectangle{\pgfqpoint{0.100000in}{0.212622in}}{\pgfqpoint{3.696000in}{3.696000in}}%
\pgfusepath{clip}%
\pgfsetbuttcap%
\pgfsetroundjoin%
\definecolor{currentfill}{rgb}{0.121569,0.466667,0.705882}%
\pgfsetfillcolor{currentfill}%
\pgfsetfillopacity{0.840068}%
\pgfsetlinewidth{1.003750pt}%
\definecolor{currentstroke}{rgb}{0.121569,0.466667,0.705882}%
\pgfsetstrokecolor{currentstroke}%
\pgfsetstrokeopacity{0.840068}%
\pgfsetdash{}{0pt}%
\pgfpathmoveto{\pgfqpoint{1.842894in}{2.293674in}}%
\pgfpathcurveto{\pgfqpoint{1.851131in}{2.293674in}}{\pgfqpoint{1.859031in}{2.296947in}}{\pgfqpoint{1.864855in}{2.302770in}}%
\pgfpathcurveto{\pgfqpoint{1.870679in}{2.308594in}}{\pgfqpoint{1.873951in}{2.316494in}}{\pgfqpoint{1.873951in}{2.324731in}}%
\pgfpathcurveto{\pgfqpoint{1.873951in}{2.332967in}}{\pgfqpoint{1.870679in}{2.340867in}}{\pgfqpoint{1.864855in}{2.346691in}}%
\pgfpathcurveto{\pgfqpoint{1.859031in}{2.352515in}}{\pgfqpoint{1.851131in}{2.355787in}}{\pgfqpoint{1.842894in}{2.355787in}}%
\pgfpathcurveto{\pgfqpoint{1.834658in}{2.355787in}}{\pgfqpoint{1.826758in}{2.352515in}}{\pgfqpoint{1.820934in}{2.346691in}}%
\pgfpathcurveto{\pgfqpoint{1.815110in}{2.340867in}}{\pgfqpoint{1.811838in}{2.332967in}}{\pgfqpoint{1.811838in}{2.324731in}}%
\pgfpathcurveto{\pgfqpoint{1.811838in}{2.316494in}}{\pgfqpoint{1.815110in}{2.308594in}}{\pgfqpoint{1.820934in}{2.302770in}}%
\pgfpathcurveto{\pgfqpoint{1.826758in}{2.296947in}}{\pgfqpoint{1.834658in}{2.293674in}}{\pgfqpoint{1.842894in}{2.293674in}}%
\pgfpathclose%
\pgfusepath{stroke,fill}%
\end{pgfscope}%
\begin{pgfscope}%
\pgfpathrectangle{\pgfqpoint{0.100000in}{0.212622in}}{\pgfqpoint{3.696000in}{3.696000in}}%
\pgfusepath{clip}%
\pgfsetbuttcap%
\pgfsetroundjoin%
\definecolor{currentfill}{rgb}{0.121569,0.466667,0.705882}%
\pgfsetfillcolor{currentfill}%
\pgfsetfillopacity{0.840073}%
\pgfsetlinewidth{1.003750pt}%
\definecolor{currentstroke}{rgb}{0.121569,0.466667,0.705882}%
\pgfsetstrokecolor{currentstroke}%
\pgfsetstrokeopacity{0.840073}%
\pgfsetdash{}{0pt}%
\pgfpathmoveto{\pgfqpoint{1.842886in}{2.293654in}}%
\pgfpathcurveto{\pgfqpoint{1.851122in}{2.293654in}}{\pgfqpoint{1.859022in}{2.296926in}}{\pgfqpoint{1.864846in}{2.302750in}}%
\pgfpathcurveto{\pgfqpoint{1.870670in}{2.308574in}}{\pgfqpoint{1.873942in}{2.316474in}}{\pgfqpoint{1.873942in}{2.324710in}}%
\pgfpathcurveto{\pgfqpoint{1.873942in}{2.332946in}}{\pgfqpoint{1.870670in}{2.340846in}}{\pgfqpoint{1.864846in}{2.346670in}}%
\pgfpathcurveto{\pgfqpoint{1.859022in}{2.352494in}}{\pgfqpoint{1.851122in}{2.355767in}}{\pgfqpoint{1.842886in}{2.355767in}}%
\pgfpathcurveto{\pgfqpoint{1.834649in}{2.355767in}}{\pgfqpoint{1.826749in}{2.352494in}}{\pgfqpoint{1.820925in}{2.346670in}}%
\pgfpathcurveto{\pgfqpoint{1.815101in}{2.340846in}}{\pgfqpoint{1.811829in}{2.332946in}}{\pgfqpoint{1.811829in}{2.324710in}}%
\pgfpathcurveto{\pgfqpoint{1.811829in}{2.316474in}}{\pgfqpoint{1.815101in}{2.308574in}}{\pgfqpoint{1.820925in}{2.302750in}}%
\pgfpathcurveto{\pgfqpoint{1.826749in}{2.296926in}}{\pgfqpoint{1.834649in}{2.293654in}}{\pgfqpoint{1.842886in}{2.293654in}}%
\pgfpathclose%
\pgfusepath{stroke,fill}%
\end{pgfscope}%
\begin{pgfscope}%
\pgfpathrectangle{\pgfqpoint{0.100000in}{0.212622in}}{\pgfqpoint{3.696000in}{3.696000in}}%
\pgfusepath{clip}%
\pgfsetbuttcap%
\pgfsetroundjoin%
\definecolor{currentfill}{rgb}{0.121569,0.466667,0.705882}%
\pgfsetfillcolor{currentfill}%
\pgfsetfillopacity{0.840075}%
\pgfsetlinewidth{1.003750pt}%
\definecolor{currentstroke}{rgb}{0.121569,0.466667,0.705882}%
\pgfsetstrokecolor{currentstroke}%
\pgfsetstrokeopacity{0.840075}%
\pgfsetdash{}{0pt}%
\pgfpathmoveto{\pgfqpoint{1.842881in}{2.293643in}}%
\pgfpathcurveto{\pgfqpoint{1.851117in}{2.293643in}}{\pgfqpoint{1.859017in}{2.296915in}}{\pgfqpoint{1.864841in}{2.302739in}}%
\pgfpathcurveto{\pgfqpoint{1.870665in}{2.308563in}}{\pgfqpoint{1.873938in}{2.316463in}}{\pgfqpoint{1.873938in}{2.324699in}}%
\pgfpathcurveto{\pgfqpoint{1.873938in}{2.332936in}}{\pgfqpoint{1.870665in}{2.340836in}}{\pgfqpoint{1.864841in}{2.346660in}}%
\pgfpathcurveto{\pgfqpoint{1.859017in}{2.352483in}}{\pgfqpoint{1.851117in}{2.355756in}}{\pgfqpoint{1.842881in}{2.355756in}}%
\pgfpathcurveto{\pgfqpoint{1.834645in}{2.355756in}}{\pgfqpoint{1.826745in}{2.352483in}}{\pgfqpoint{1.820921in}{2.346660in}}%
\pgfpathcurveto{\pgfqpoint{1.815097in}{2.340836in}}{\pgfqpoint{1.811825in}{2.332936in}}{\pgfqpoint{1.811825in}{2.324699in}}%
\pgfpathcurveto{\pgfqpoint{1.811825in}{2.316463in}}{\pgfqpoint{1.815097in}{2.308563in}}{\pgfqpoint{1.820921in}{2.302739in}}%
\pgfpathcurveto{\pgfqpoint{1.826745in}{2.296915in}}{\pgfqpoint{1.834645in}{2.293643in}}{\pgfqpoint{1.842881in}{2.293643in}}%
\pgfpathclose%
\pgfusepath{stroke,fill}%
\end{pgfscope}%
\begin{pgfscope}%
\pgfpathrectangle{\pgfqpoint{0.100000in}{0.212622in}}{\pgfqpoint{3.696000in}{3.696000in}}%
\pgfusepath{clip}%
\pgfsetbuttcap%
\pgfsetroundjoin%
\definecolor{currentfill}{rgb}{0.121569,0.466667,0.705882}%
\pgfsetfillcolor{currentfill}%
\pgfsetfillopacity{0.840077}%
\pgfsetlinewidth{1.003750pt}%
\definecolor{currentstroke}{rgb}{0.121569,0.466667,0.705882}%
\pgfsetstrokecolor{currentstroke}%
\pgfsetstrokeopacity{0.840077}%
\pgfsetdash{}{0pt}%
\pgfpathmoveto{\pgfqpoint{1.842879in}{2.293636in}}%
\pgfpathcurveto{\pgfqpoint{1.851115in}{2.293636in}}{\pgfqpoint{1.859015in}{2.296909in}}{\pgfqpoint{1.864839in}{2.302733in}}%
\pgfpathcurveto{\pgfqpoint{1.870663in}{2.308557in}}{\pgfqpoint{1.873935in}{2.316457in}}{\pgfqpoint{1.873935in}{2.324693in}}%
\pgfpathcurveto{\pgfqpoint{1.873935in}{2.332929in}}{\pgfqpoint{1.870663in}{2.340829in}}{\pgfqpoint{1.864839in}{2.346653in}}%
\pgfpathcurveto{\pgfqpoint{1.859015in}{2.352477in}}{\pgfqpoint{1.851115in}{2.355749in}}{\pgfqpoint{1.842879in}{2.355749in}}%
\pgfpathcurveto{\pgfqpoint{1.834642in}{2.355749in}}{\pgfqpoint{1.826742in}{2.352477in}}{\pgfqpoint{1.820918in}{2.346653in}}%
\pgfpathcurveto{\pgfqpoint{1.815094in}{2.340829in}}{\pgfqpoint{1.811822in}{2.332929in}}{\pgfqpoint{1.811822in}{2.324693in}}%
\pgfpathcurveto{\pgfqpoint{1.811822in}{2.316457in}}{\pgfqpoint{1.815094in}{2.308557in}}{\pgfqpoint{1.820918in}{2.302733in}}%
\pgfpathcurveto{\pgfqpoint{1.826742in}{2.296909in}}{\pgfqpoint{1.834642in}{2.293636in}}{\pgfqpoint{1.842879in}{2.293636in}}%
\pgfpathclose%
\pgfusepath{stroke,fill}%
\end{pgfscope}%
\begin{pgfscope}%
\pgfpathrectangle{\pgfqpoint{0.100000in}{0.212622in}}{\pgfqpoint{3.696000in}{3.696000in}}%
\pgfusepath{clip}%
\pgfsetbuttcap%
\pgfsetroundjoin%
\definecolor{currentfill}{rgb}{0.121569,0.466667,0.705882}%
\pgfsetfillcolor{currentfill}%
\pgfsetfillopacity{0.840077}%
\pgfsetlinewidth{1.003750pt}%
\definecolor{currentstroke}{rgb}{0.121569,0.466667,0.705882}%
\pgfsetstrokecolor{currentstroke}%
\pgfsetstrokeopacity{0.840077}%
\pgfsetdash{}{0pt}%
\pgfpathmoveto{\pgfqpoint{1.842877in}{2.293633in}}%
\pgfpathcurveto{\pgfqpoint{1.851113in}{2.293633in}}{\pgfqpoint{1.859013in}{2.296906in}}{\pgfqpoint{1.864837in}{2.302730in}}%
\pgfpathcurveto{\pgfqpoint{1.870661in}{2.308554in}}{\pgfqpoint{1.873934in}{2.316454in}}{\pgfqpoint{1.873934in}{2.324690in}}%
\pgfpathcurveto{\pgfqpoint{1.873934in}{2.332926in}}{\pgfqpoint{1.870661in}{2.340826in}}{\pgfqpoint{1.864837in}{2.346650in}}%
\pgfpathcurveto{\pgfqpoint{1.859013in}{2.352474in}}{\pgfqpoint{1.851113in}{2.355746in}}{\pgfqpoint{1.842877in}{2.355746in}}%
\pgfpathcurveto{\pgfqpoint{1.834641in}{2.355746in}}{\pgfqpoint{1.826741in}{2.352474in}}{\pgfqpoint{1.820917in}{2.346650in}}%
\pgfpathcurveto{\pgfqpoint{1.815093in}{2.340826in}}{\pgfqpoint{1.811821in}{2.332926in}}{\pgfqpoint{1.811821in}{2.324690in}}%
\pgfpathcurveto{\pgfqpoint{1.811821in}{2.316454in}}{\pgfqpoint{1.815093in}{2.308554in}}{\pgfqpoint{1.820917in}{2.302730in}}%
\pgfpathcurveto{\pgfqpoint{1.826741in}{2.296906in}}{\pgfqpoint{1.834641in}{2.293633in}}{\pgfqpoint{1.842877in}{2.293633in}}%
\pgfpathclose%
\pgfusepath{stroke,fill}%
\end{pgfscope}%
\begin{pgfscope}%
\pgfpathrectangle{\pgfqpoint{0.100000in}{0.212622in}}{\pgfqpoint{3.696000in}{3.696000in}}%
\pgfusepath{clip}%
\pgfsetbuttcap%
\pgfsetroundjoin%
\definecolor{currentfill}{rgb}{0.121569,0.466667,0.705882}%
\pgfsetfillcolor{currentfill}%
\pgfsetfillopacity{0.840078}%
\pgfsetlinewidth{1.003750pt}%
\definecolor{currentstroke}{rgb}{0.121569,0.466667,0.705882}%
\pgfsetstrokecolor{currentstroke}%
\pgfsetstrokeopacity{0.840078}%
\pgfsetdash{}{0pt}%
\pgfpathmoveto{\pgfqpoint{1.842876in}{2.293632in}}%
\pgfpathcurveto{\pgfqpoint{1.851113in}{2.293632in}}{\pgfqpoint{1.859013in}{2.296904in}}{\pgfqpoint{1.864837in}{2.302728in}}%
\pgfpathcurveto{\pgfqpoint{1.870661in}{2.308552in}}{\pgfqpoint{1.873933in}{2.316452in}}{\pgfqpoint{1.873933in}{2.324688in}}%
\pgfpathcurveto{\pgfqpoint{1.873933in}{2.332924in}}{\pgfqpoint{1.870661in}{2.340824in}}{\pgfqpoint{1.864837in}{2.346648in}}%
\pgfpathcurveto{\pgfqpoint{1.859013in}{2.352472in}}{\pgfqpoint{1.851113in}{2.355745in}}{\pgfqpoint{1.842876in}{2.355745in}}%
\pgfpathcurveto{\pgfqpoint{1.834640in}{2.355745in}}{\pgfqpoint{1.826740in}{2.352472in}}{\pgfqpoint{1.820916in}{2.346648in}}%
\pgfpathcurveto{\pgfqpoint{1.815092in}{2.340824in}}{\pgfqpoint{1.811820in}{2.332924in}}{\pgfqpoint{1.811820in}{2.324688in}}%
\pgfpathcurveto{\pgfqpoint{1.811820in}{2.316452in}}{\pgfqpoint{1.815092in}{2.308552in}}{\pgfqpoint{1.820916in}{2.302728in}}%
\pgfpathcurveto{\pgfqpoint{1.826740in}{2.296904in}}{\pgfqpoint{1.834640in}{2.293632in}}{\pgfqpoint{1.842876in}{2.293632in}}%
\pgfpathclose%
\pgfusepath{stroke,fill}%
\end{pgfscope}%
\begin{pgfscope}%
\pgfpathrectangle{\pgfqpoint{0.100000in}{0.212622in}}{\pgfqpoint{3.696000in}{3.696000in}}%
\pgfusepath{clip}%
\pgfsetbuttcap%
\pgfsetroundjoin%
\definecolor{currentfill}{rgb}{0.121569,0.466667,0.705882}%
\pgfsetfillcolor{currentfill}%
\pgfsetfillopacity{0.840078}%
\pgfsetlinewidth{1.003750pt}%
\definecolor{currentstroke}{rgb}{0.121569,0.466667,0.705882}%
\pgfsetstrokecolor{currentstroke}%
\pgfsetstrokeopacity{0.840078}%
\pgfsetdash{}{0pt}%
\pgfpathmoveto{\pgfqpoint{1.842876in}{2.293630in}}%
\pgfpathcurveto{\pgfqpoint{1.851112in}{2.293630in}}{\pgfqpoint{1.859012in}{2.296903in}}{\pgfqpoint{1.864836in}{2.302727in}}%
\pgfpathcurveto{\pgfqpoint{1.870660in}{2.308551in}}{\pgfqpoint{1.873933in}{2.316451in}}{\pgfqpoint{1.873933in}{2.324687in}}%
\pgfpathcurveto{\pgfqpoint{1.873933in}{2.332923in}}{\pgfqpoint{1.870660in}{2.340823in}}{\pgfqpoint{1.864836in}{2.346647in}}%
\pgfpathcurveto{\pgfqpoint{1.859012in}{2.352471in}}{\pgfqpoint{1.851112in}{2.355743in}}{\pgfqpoint{1.842876in}{2.355743in}}%
\pgfpathcurveto{\pgfqpoint{1.834640in}{2.355743in}}{\pgfqpoint{1.826740in}{2.352471in}}{\pgfqpoint{1.820916in}{2.346647in}}%
\pgfpathcurveto{\pgfqpoint{1.815092in}{2.340823in}}{\pgfqpoint{1.811820in}{2.332923in}}{\pgfqpoint{1.811820in}{2.324687in}}%
\pgfpathcurveto{\pgfqpoint{1.811820in}{2.316451in}}{\pgfqpoint{1.815092in}{2.308551in}}{\pgfqpoint{1.820916in}{2.302727in}}%
\pgfpathcurveto{\pgfqpoint{1.826740in}{2.296903in}}{\pgfqpoint{1.834640in}{2.293630in}}{\pgfqpoint{1.842876in}{2.293630in}}%
\pgfpathclose%
\pgfusepath{stroke,fill}%
\end{pgfscope}%
\begin{pgfscope}%
\pgfpathrectangle{\pgfqpoint{0.100000in}{0.212622in}}{\pgfqpoint{3.696000in}{3.696000in}}%
\pgfusepath{clip}%
\pgfsetbuttcap%
\pgfsetroundjoin%
\definecolor{currentfill}{rgb}{0.121569,0.466667,0.705882}%
\pgfsetfillcolor{currentfill}%
\pgfsetfillopacity{0.840149}%
\pgfsetlinewidth{1.003750pt}%
\definecolor{currentstroke}{rgb}{0.121569,0.466667,0.705882}%
\pgfsetstrokecolor{currentstroke}%
\pgfsetstrokeopacity{0.840149}%
\pgfsetdash{}{0pt}%
\pgfpathmoveto{\pgfqpoint{1.842695in}{2.293274in}}%
\pgfpathcurveto{\pgfqpoint{1.850932in}{2.293274in}}{\pgfqpoint{1.858832in}{2.296547in}}{\pgfqpoint{1.864656in}{2.302371in}}%
\pgfpathcurveto{\pgfqpoint{1.870480in}{2.308195in}}{\pgfqpoint{1.873752in}{2.316095in}}{\pgfqpoint{1.873752in}{2.324331in}}%
\pgfpathcurveto{\pgfqpoint{1.873752in}{2.332567in}}{\pgfqpoint{1.870480in}{2.340467in}}{\pgfqpoint{1.864656in}{2.346291in}}%
\pgfpathcurveto{\pgfqpoint{1.858832in}{2.352115in}}{\pgfqpoint{1.850932in}{2.355387in}}{\pgfqpoint{1.842695in}{2.355387in}}%
\pgfpathcurveto{\pgfqpoint{1.834459in}{2.355387in}}{\pgfqpoint{1.826559in}{2.352115in}}{\pgfqpoint{1.820735in}{2.346291in}}%
\pgfpathcurveto{\pgfqpoint{1.814911in}{2.340467in}}{\pgfqpoint{1.811639in}{2.332567in}}{\pgfqpoint{1.811639in}{2.324331in}}%
\pgfpathcurveto{\pgfqpoint{1.811639in}{2.316095in}}{\pgfqpoint{1.814911in}{2.308195in}}{\pgfqpoint{1.820735in}{2.302371in}}%
\pgfpathcurveto{\pgfqpoint{1.826559in}{2.296547in}}{\pgfqpoint{1.834459in}{2.293274in}}{\pgfqpoint{1.842695in}{2.293274in}}%
\pgfpathclose%
\pgfusepath{stroke,fill}%
\end{pgfscope}%
\begin{pgfscope}%
\pgfpathrectangle{\pgfqpoint{0.100000in}{0.212622in}}{\pgfqpoint{3.696000in}{3.696000in}}%
\pgfusepath{clip}%
\pgfsetbuttcap%
\pgfsetroundjoin%
\definecolor{currentfill}{rgb}{0.121569,0.466667,0.705882}%
\pgfsetfillcolor{currentfill}%
\pgfsetfillopacity{0.840259}%
\pgfsetlinewidth{1.003750pt}%
\definecolor{currentstroke}{rgb}{0.121569,0.466667,0.705882}%
\pgfsetstrokecolor{currentstroke}%
\pgfsetstrokeopacity{0.840259}%
\pgfsetdash{}{0pt}%
\pgfpathmoveto{\pgfqpoint{2.374363in}{2.494833in}}%
\pgfpathcurveto{\pgfqpoint{2.382599in}{2.494833in}}{\pgfqpoint{2.390499in}{2.498106in}}{\pgfqpoint{2.396323in}{2.503930in}}%
\pgfpathcurveto{\pgfqpoint{2.402147in}{2.509754in}}{\pgfqpoint{2.405419in}{2.517654in}}{\pgfqpoint{2.405419in}{2.525890in}}%
\pgfpathcurveto{\pgfqpoint{2.405419in}{2.534126in}}{\pgfqpoint{2.402147in}{2.542026in}}{\pgfqpoint{2.396323in}{2.547850in}}%
\pgfpathcurveto{\pgfqpoint{2.390499in}{2.553674in}}{\pgfqpoint{2.382599in}{2.556946in}}{\pgfqpoint{2.374363in}{2.556946in}}%
\pgfpathcurveto{\pgfqpoint{2.366126in}{2.556946in}}{\pgfqpoint{2.358226in}{2.553674in}}{\pgfqpoint{2.352403in}{2.547850in}}%
\pgfpathcurveto{\pgfqpoint{2.346579in}{2.542026in}}{\pgfqpoint{2.343306in}{2.534126in}}{\pgfqpoint{2.343306in}{2.525890in}}%
\pgfpathcurveto{\pgfqpoint{2.343306in}{2.517654in}}{\pgfqpoint{2.346579in}{2.509754in}}{\pgfqpoint{2.352403in}{2.503930in}}%
\pgfpathcurveto{\pgfqpoint{2.358226in}{2.498106in}}{\pgfqpoint{2.366126in}{2.494833in}}{\pgfqpoint{2.374363in}{2.494833in}}%
\pgfpathclose%
\pgfusepath{stroke,fill}%
\end{pgfscope}%
\begin{pgfscope}%
\pgfpathrectangle{\pgfqpoint{0.100000in}{0.212622in}}{\pgfqpoint{3.696000in}{3.696000in}}%
\pgfusepath{clip}%
\pgfsetbuttcap%
\pgfsetroundjoin%
\definecolor{currentfill}{rgb}{0.121569,0.466667,0.705882}%
\pgfsetfillcolor{currentfill}%
\pgfsetfillopacity{0.840328}%
\pgfsetlinewidth{1.003750pt}%
\definecolor{currentstroke}{rgb}{0.121569,0.466667,0.705882}%
\pgfsetstrokecolor{currentstroke}%
\pgfsetstrokeopacity{0.840328}%
\pgfsetdash{}{0pt}%
\pgfpathmoveto{\pgfqpoint{1.842510in}{2.292316in}}%
\pgfpathcurveto{\pgfqpoint{1.850746in}{2.292316in}}{\pgfqpoint{1.858646in}{2.295589in}}{\pgfqpoint{1.864470in}{2.301413in}}%
\pgfpathcurveto{\pgfqpoint{1.870294in}{2.307236in}}{\pgfqpoint{1.873566in}{2.315137in}}{\pgfqpoint{1.873566in}{2.323373in}}%
\pgfpathcurveto{\pgfqpoint{1.873566in}{2.331609in}}{\pgfqpoint{1.870294in}{2.339509in}}{\pgfqpoint{1.864470in}{2.345333in}}%
\pgfpathcurveto{\pgfqpoint{1.858646in}{2.351157in}}{\pgfqpoint{1.850746in}{2.354429in}}{\pgfqpoint{1.842510in}{2.354429in}}%
\pgfpathcurveto{\pgfqpoint{1.834273in}{2.354429in}}{\pgfqpoint{1.826373in}{2.351157in}}{\pgfqpoint{1.820549in}{2.345333in}}%
\pgfpathcurveto{\pgfqpoint{1.814726in}{2.339509in}}{\pgfqpoint{1.811453in}{2.331609in}}{\pgfqpoint{1.811453in}{2.323373in}}%
\pgfpathcurveto{\pgfqpoint{1.811453in}{2.315137in}}{\pgfqpoint{1.814726in}{2.307236in}}{\pgfqpoint{1.820549in}{2.301413in}}%
\pgfpathcurveto{\pgfqpoint{1.826373in}{2.295589in}}{\pgfqpoint{1.834273in}{2.292316in}}{\pgfqpoint{1.842510in}{2.292316in}}%
\pgfpathclose%
\pgfusepath{stroke,fill}%
\end{pgfscope}%
\begin{pgfscope}%
\pgfpathrectangle{\pgfqpoint{0.100000in}{0.212622in}}{\pgfqpoint{3.696000in}{3.696000in}}%
\pgfusepath{clip}%
\pgfsetbuttcap%
\pgfsetroundjoin%
\definecolor{currentfill}{rgb}{0.121569,0.466667,0.705882}%
\pgfsetfillcolor{currentfill}%
\pgfsetfillopacity{0.840760}%
\pgfsetlinewidth{1.003750pt}%
\definecolor{currentstroke}{rgb}{0.121569,0.466667,0.705882}%
\pgfsetstrokecolor{currentstroke}%
\pgfsetstrokeopacity{0.840760}%
\pgfsetdash{}{0pt}%
\pgfpathmoveto{\pgfqpoint{1.298780in}{1.353834in}}%
\pgfpathcurveto{\pgfqpoint{1.307017in}{1.353834in}}{\pgfqpoint{1.314917in}{1.357107in}}{\pgfqpoint{1.320741in}{1.362931in}}%
\pgfpathcurveto{\pgfqpoint{1.326565in}{1.368755in}}{\pgfqpoint{1.329837in}{1.376655in}}{\pgfqpoint{1.329837in}{1.384891in}}%
\pgfpathcurveto{\pgfqpoint{1.329837in}{1.393127in}}{\pgfqpoint{1.326565in}{1.401027in}}{\pgfqpoint{1.320741in}{1.406851in}}%
\pgfpathcurveto{\pgfqpoint{1.314917in}{1.412675in}}{\pgfqpoint{1.307017in}{1.415947in}}{\pgfqpoint{1.298780in}{1.415947in}}%
\pgfpathcurveto{\pgfqpoint{1.290544in}{1.415947in}}{\pgfqpoint{1.282644in}{1.412675in}}{\pgfqpoint{1.276820in}{1.406851in}}%
\pgfpathcurveto{\pgfqpoint{1.270996in}{1.401027in}}{\pgfqpoint{1.267724in}{1.393127in}}{\pgfqpoint{1.267724in}{1.384891in}}%
\pgfpathcurveto{\pgfqpoint{1.267724in}{1.376655in}}{\pgfqpoint{1.270996in}{1.368755in}}{\pgfqpoint{1.276820in}{1.362931in}}%
\pgfpathcurveto{\pgfqpoint{1.282644in}{1.357107in}}{\pgfqpoint{1.290544in}{1.353834in}}{\pgfqpoint{1.298780in}{1.353834in}}%
\pgfpathclose%
\pgfusepath{stroke,fill}%
\end{pgfscope}%
\begin{pgfscope}%
\pgfpathrectangle{\pgfqpoint{0.100000in}{0.212622in}}{\pgfqpoint{3.696000in}{3.696000in}}%
\pgfusepath{clip}%
\pgfsetbuttcap%
\pgfsetroundjoin%
\definecolor{currentfill}{rgb}{0.121569,0.466667,0.705882}%
\pgfsetfillcolor{currentfill}%
\pgfsetfillopacity{0.840874}%
\pgfsetlinewidth{1.003750pt}%
\definecolor{currentstroke}{rgb}{0.121569,0.466667,0.705882}%
\pgfsetstrokecolor{currentstroke}%
\pgfsetstrokeopacity{0.840874}%
\pgfsetdash{}{0pt}%
\pgfpathmoveto{\pgfqpoint{1.841068in}{2.289880in}}%
\pgfpathcurveto{\pgfqpoint{1.849304in}{2.289880in}}{\pgfqpoint{1.857204in}{2.293152in}}{\pgfqpoint{1.863028in}{2.298976in}}%
\pgfpathcurveto{\pgfqpoint{1.868852in}{2.304800in}}{\pgfqpoint{1.872124in}{2.312700in}}{\pgfqpoint{1.872124in}{2.320936in}}%
\pgfpathcurveto{\pgfqpoint{1.872124in}{2.329173in}}{\pgfqpoint{1.868852in}{2.337073in}}{\pgfqpoint{1.863028in}{2.342897in}}%
\pgfpathcurveto{\pgfqpoint{1.857204in}{2.348721in}}{\pgfqpoint{1.849304in}{2.351993in}}{\pgfqpoint{1.841068in}{2.351993in}}%
\pgfpathcurveto{\pgfqpoint{1.832832in}{2.351993in}}{\pgfqpoint{1.824932in}{2.348721in}}{\pgfqpoint{1.819108in}{2.342897in}}%
\pgfpathcurveto{\pgfqpoint{1.813284in}{2.337073in}}{\pgfqpoint{1.810011in}{2.329173in}}{\pgfqpoint{1.810011in}{2.320936in}}%
\pgfpathcurveto{\pgfqpoint{1.810011in}{2.312700in}}{\pgfqpoint{1.813284in}{2.304800in}}{\pgfqpoint{1.819108in}{2.298976in}}%
\pgfpathcurveto{\pgfqpoint{1.824932in}{2.293152in}}{\pgfqpoint{1.832832in}{2.289880in}}{\pgfqpoint{1.841068in}{2.289880in}}%
\pgfpathclose%
\pgfusepath{stroke,fill}%
\end{pgfscope}%
\begin{pgfscope}%
\pgfpathrectangle{\pgfqpoint{0.100000in}{0.212622in}}{\pgfqpoint{3.696000in}{3.696000in}}%
\pgfusepath{clip}%
\pgfsetbuttcap%
\pgfsetroundjoin%
\definecolor{currentfill}{rgb}{0.121569,0.466667,0.705882}%
\pgfsetfillcolor{currentfill}%
\pgfsetfillopacity{0.840964}%
\pgfsetlinewidth{1.003750pt}%
\definecolor{currentstroke}{rgb}{0.121569,0.466667,0.705882}%
\pgfsetstrokecolor{currentstroke}%
\pgfsetstrokeopacity{0.840964}%
\pgfsetdash{}{0pt}%
\pgfpathmoveto{\pgfqpoint{2.373079in}{2.489785in}}%
\pgfpathcurveto{\pgfqpoint{2.381316in}{2.489785in}}{\pgfqpoint{2.389216in}{2.493058in}}{\pgfqpoint{2.395040in}{2.498882in}}%
\pgfpathcurveto{\pgfqpoint{2.400864in}{2.504706in}}{\pgfqpoint{2.404136in}{2.512606in}}{\pgfqpoint{2.404136in}{2.520842in}}%
\pgfpathcurveto{\pgfqpoint{2.404136in}{2.529078in}}{\pgfqpoint{2.400864in}{2.536978in}}{\pgfqpoint{2.395040in}{2.542802in}}%
\pgfpathcurveto{\pgfqpoint{2.389216in}{2.548626in}}{\pgfqpoint{2.381316in}{2.551898in}}{\pgfqpoint{2.373079in}{2.551898in}}%
\pgfpathcurveto{\pgfqpoint{2.364843in}{2.551898in}}{\pgfqpoint{2.356943in}{2.548626in}}{\pgfqpoint{2.351119in}{2.542802in}}%
\pgfpathcurveto{\pgfqpoint{2.345295in}{2.536978in}}{\pgfqpoint{2.342023in}{2.529078in}}{\pgfqpoint{2.342023in}{2.520842in}}%
\pgfpathcurveto{\pgfqpoint{2.342023in}{2.512606in}}{\pgfqpoint{2.345295in}{2.504706in}}{\pgfqpoint{2.351119in}{2.498882in}}%
\pgfpathcurveto{\pgfqpoint{2.356943in}{2.493058in}}{\pgfqpoint{2.364843in}{2.489785in}}{\pgfqpoint{2.373079in}{2.489785in}}%
\pgfpathclose%
\pgfusepath{stroke,fill}%
\end{pgfscope}%
\begin{pgfscope}%
\pgfpathrectangle{\pgfqpoint{0.100000in}{0.212622in}}{\pgfqpoint{3.696000in}{3.696000in}}%
\pgfusepath{clip}%
\pgfsetbuttcap%
\pgfsetroundjoin%
\definecolor{currentfill}{rgb}{0.121569,0.466667,0.705882}%
\pgfsetfillcolor{currentfill}%
\pgfsetfillopacity{0.841183}%
\pgfsetlinewidth{1.003750pt}%
\definecolor{currentstroke}{rgb}{0.121569,0.466667,0.705882}%
\pgfsetstrokecolor{currentstroke}%
\pgfsetstrokeopacity{0.841183}%
\pgfsetdash{}{0pt}%
\pgfpathmoveto{\pgfqpoint{1.840534in}{2.288379in}}%
\pgfpathcurveto{\pgfqpoint{1.848770in}{2.288379in}}{\pgfqpoint{1.856671in}{2.291651in}}{\pgfqpoint{1.862494in}{2.297475in}}%
\pgfpathcurveto{\pgfqpoint{1.868318in}{2.303299in}}{\pgfqpoint{1.871591in}{2.311199in}}{\pgfqpoint{1.871591in}{2.319435in}}%
\pgfpathcurveto{\pgfqpoint{1.871591in}{2.327671in}}{\pgfqpoint{1.868318in}{2.335571in}}{\pgfqpoint{1.862494in}{2.341395in}}%
\pgfpathcurveto{\pgfqpoint{1.856671in}{2.347219in}}{\pgfqpoint{1.848770in}{2.350492in}}{\pgfqpoint{1.840534in}{2.350492in}}%
\pgfpathcurveto{\pgfqpoint{1.832298in}{2.350492in}}{\pgfqpoint{1.824398in}{2.347219in}}{\pgfqpoint{1.818574in}{2.341395in}}%
\pgfpathcurveto{\pgfqpoint{1.812750in}{2.335571in}}{\pgfqpoint{1.809478in}{2.327671in}}{\pgfqpoint{1.809478in}{2.319435in}}%
\pgfpathcurveto{\pgfqpoint{1.809478in}{2.311199in}}{\pgfqpoint{1.812750in}{2.303299in}}{\pgfqpoint{1.818574in}{2.297475in}}%
\pgfpathcurveto{\pgfqpoint{1.824398in}{2.291651in}}{\pgfqpoint{1.832298in}{2.288379in}}{\pgfqpoint{1.840534in}{2.288379in}}%
\pgfpathclose%
\pgfusepath{stroke,fill}%
\end{pgfscope}%
\begin{pgfscope}%
\pgfpathrectangle{\pgfqpoint{0.100000in}{0.212622in}}{\pgfqpoint{3.696000in}{3.696000in}}%
\pgfusepath{clip}%
\pgfsetbuttcap%
\pgfsetroundjoin%
\definecolor{currentfill}{rgb}{0.121569,0.466667,0.705882}%
\pgfsetfillcolor{currentfill}%
\pgfsetfillopacity{0.841285}%
\pgfsetlinewidth{1.003750pt}%
\definecolor{currentstroke}{rgb}{0.121569,0.466667,0.705882}%
\pgfsetstrokecolor{currentstroke}%
\pgfsetstrokeopacity{0.841285}%
\pgfsetdash{}{0pt}%
\pgfpathmoveto{\pgfqpoint{2.371626in}{2.487782in}}%
\pgfpathcurveto{\pgfqpoint{2.379862in}{2.487782in}}{\pgfqpoint{2.387762in}{2.491054in}}{\pgfqpoint{2.393586in}{2.496878in}}%
\pgfpathcurveto{\pgfqpoint{2.399410in}{2.502702in}}{\pgfqpoint{2.402682in}{2.510602in}}{\pgfqpoint{2.402682in}{2.518838in}}%
\pgfpathcurveto{\pgfqpoint{2.402682in}{2.527075in}}{\pgfqpoint{2.399410in}{2.534975in}}{\pgfqpoint{2.393586in}{2.540799in}}%
\pgfpathcurveto{\pgfqpoint{2.387762in}{2.546623in}}{\pgfqpoint{2.379862in}{2.549895in}}{\pgfqpoint{2.371626in}{2.549895in}}%
\pgfpathcurveto{\pgfqpoint{2.363389in}{2.549895in}}{\pgfqpoint{2.355489in}{2.546623in}}{\pgfqpoint{2.349665in}{2.540799in}}%
\pgfpathcurveto{\pgfqpoint{2.343842in}{2.534975in}}{\pgfqpoint{2.340569in}{2.527075in}}{\pgfqpoint{2.340569in}{2.518838in}}%
\pgfpathcurveto{\pgfqpoint{2.340569in}{2.510602in}}{\pgfqpoint{2.343842in}{2.502702in}}{\pgfqpoint{2.349665in}{2.496878in}}%
\pgfpathcurveto{\pgfqpoint{2.355489in}{2.491054in}}{\pgfqpoint{2.363389in}{2.487782in}}{\pgfqpoint{2.371626in}{2.487782in}}%
\pgfpathclose%
\pgfusepath{stroke,fill}%
\end{pgfscope}%
\begin{pgfscope}%
\pgfpathrectangle{\pgfqpoint{0.100000in}{0.212622in}}{\pgfqpoint{3.696000in}{3.696000in}}%
\pgfusepath{clip}%
\pgfsetbuttcap%
\pgfsetroundjoin%
\definecolor{currentfill}{rgb}{0.121569,0.466667,0.705882}%
\pgfsetfillcolor{currentfill}%
\pgfsetfillopacity{0.841355}%
\pgfsetlinewidth{1.003750pt}%
\definecolor{currentstroke}{rgb}{0.121569,0.466667,0.705882}%
\pgfsetstrokecolor{currentstroke}%
\pgfsetstrokeopacity{0.841355}%
\pgfsetdash{}{0pt}%
\pgfpathmoveto{\pgfqpoint{1.840359in}{2.287547in}}%
\pgfpathcurveto{\pgfqpoint{1.848595in}{2.287547in}}{\pgfqpoint{1.856495in}{2.290820in}}{\pgfqpoint{1.862319in}{2.296644in}}%
\pgfpathcurveto{\pgfqpoint{1.868143in}{2.302467in}}{\pgfqpoint{1.871415in}{2.310368in}}{\pgfqpoint{1.871415in}{2.318604in}}%
\pgfpathcurveto{\pgfqpoint{1.871415in}{2.326840in}}{\pgfqpoint{1.868143in}{2.334740in}}{\pgfqpoint{1.862319in}{2.340564in}}%
\pgfpathcurveto{\pgfqpoint{1.856495in}{2.346388in}}{\pgfqpoint{1.848595in}{2.349660in}}{\pgfqpoint{1.840359in}{2.349660in}}%
\pgfpathcurveto{\pgfqpoint{1.832123in}{2.349660in}}{\pgfqpoint{1.824222in}{2.346388in}}{\pgfqpoint{1.818399in}{2.340564in}}%
\pgfpathcurveto{\pgfqpoint{1.812575in}{2.334740in}}{\pgfqpoint{1.809302in}{2.326840in}}{\pgfqpoint{1.809302in}{2.318604in}}%
\pgfpathcurveto{\pgfqpoint{1.809302in}{2.310368in}}{\pgfqpoint{1.812575in}{2.302467in}}{\pgfqpoint{1.818399in}{2.296644in}}%
\pgfpathcurveto{\pgfqpoint{1.824222in}{2.290820in}}{\pgfqpoint{1.832123in}{2.287547in}}{\pgfqpoint{1.840359in}{2.287547in}}%
\pgfpathclose%
\pgfusepath{stroke,fill}%
\end{pgfscope}%
\begin{pgfscope}%
\pgfpathrectangle{\pgfqpoint{0.100000in}{0.212622in}}{\pgfqpoint{3.696000in}{3.696000in}}%
\pgfusepath{clip}%
\pgfsetbuttcap%
\pgfsetroundjoin%
\definecolor{currentfill}{rgb}{0.121569,0.466667,0.705882}%
\pgfsetfillcolor{currentfill}%
\pgfsetfillopacity{0.841587}%
\pgfsetlinewidth{1.003750pt}%
\definecolor{currentstroke}{rgb}{0.121569,0.466667,0.705882}%
\pgfsetstrokecolor{currentstroke}%
\pgfsetstrokeopacity{0.841587}%
\pgfsetdash{}{0pt}%
\pgfpathmoveto{\pgfqpoint{1.839772in}{2.286409in}}%
\pgfpathcurveto{\pgfqpoint{1.848008in}{2.286409in}}{\pgfqpoint{1.855908in}{2.289681in}}{\pgfqpoint{1.861732in}{2.295505in}}%
\pgfpathcurveto{\pgfqpoint{1.867556in}{2.301329in}}{\pgfqpoint{1.870828in}{2.309229in}}{\pgfqpoint{1.870828in}{2.317465in}}%
\pgfpathcurveto{\pgfqpoint{1.870828in}{2.325702in}}{\pgfqpoint{1.867556in}{2.333602in}}{\pgfqpoint{1.861732in}{2.339426in}}%
\pgfpathcurveto{\pgfqpoint{1.855908in}{2.345250in}}{\pgfqpoint{1.848008in}{2.348522in}}{\pgfqpoint{1.839772in}{2.348522in}}%
\pgfpathcurveto{\pgfqpoint{1.831536in}{2.348522in}}{\pgfqpoint{1.823636in}{2.345250in}}{\pgfqpoint{1.817812in}{2.339426in}}%
\pgfpathcurveto{\pgfqpoint{1.811988in}{2.333602in}}{\pgfqpoint{1.808715in}{2.325702in}}{\pgfqpoint{1.808715in}{2.317465in}}%
\pgfpathcurveto{\pgfqpoint{1.808715in}{2.309229in}}{\pgfqpoint{1.811988in}{2.301329in}}{\pgfqpoint{1.817812in}{2.295505in}}%
\pgfpathcurveto{\pgfqpoint{1.823636in}{2.289681in}}{\pgfqpoint{1.831536in}{2.286409in}}{\pgfqpoint{1.839772in}{2.286409in}}%
\pgfpathclose%
\pgfusepath{stroke,fill}%
\end{pgfscope}%
\begin{pgfscope}%
\pgfpathrectangle{\pgfqpoint{0.100000in}{0.212622in}}{\pgfqpoint{3.696000in}{3.696000in}}%
\pgfusepath{clip}%
\pgfsetbuttcap%
\pgfsetroundjoin%
\definecolor{currentfill}{rgb}{0.121569,0.466667,0.705882}%
\pgfsetfillcolor{currentfill}%
\pgfsetfillopacity{0.841837}%
\pgfsetlinewidth{1.003750pt}%
\definecolor{currentstroke}{rgb}{0.121569,0.466667,0.705882}%
\pgfsetstrokecolor{currentstroke}%
\pgfsetstrokeopacity{0.841837}%
\pgfsetdash{}{0pt}%
\pgfpathmoveto{\pgfqpoint{2.369005in}{2.483935in}}%
\pgfpathcurveto{\pgfqpoint{2.377242in}{2.483935in}}{\pgfqpoint{2.385142in}{2.487208in}}{\pgfqpoint{2.390966in}{2.493032in}}%
\pgfpathcurveto{\pgfqpoint{2.396789in}{2.498856in}}{\pgfqpoint{2.400062in}{2.506756in}}{\pgfqpoint{2.400062in}{2.514992in}}%
\pgfpathcurveto{\pgfqpoint{2.400062in}{2.523228in}}{\pgfqpoint{2.396789in}{2.531128in}}{\pgfqpoint{2.390966in}{2.536952in}}%
\pgfpathcurveto{\pgfqpoint{2.385142in}{2.542776in}}{\pgfqpoint{2.377242in}{2.546048in}}{\pgfqpoint{2.369005in}{2.546048in}}%
\pgfpathcurveto{\pgfqpoint{2.360769in}{2.546048in}}{\pgfqpoint{2.352869in}{2.542776in}}{\pgfqpoint{2.347045in}{2.536952in}}%
\pgfpathcurveto{\pgfqpoint{2.341221in}{2.531128in}}{\pgfqpoint{2.337949in}{2.523228in}}{\pgfqpoint{2.337949in}{2.514992in}}%
\pgfpathcurveto{\pgfqpoint{2.337949in}{2.506756in}}{\pgfqpoint{2.341221in}{2.498856in}}{\pgfqpoint{2.347045in}{2.493032in}}%
\pgfpathcurveto{\pgfqpoint{2.352869in}{2.487208in}}{\pgfqpoint{2.360769in}{2.483935in}}{\pgfqpoint{2.369005in}{2.483935in}}%
\pgfpathclose%
\pgfusepath{stroke,fill}%
\end{pgfscope}%
\begin{pgfscope}%
\pgfpathrectangle{\pgfqpoint{0.100000in}{0.212622in}}{\pgfqpoint{3.696000in}{3.696000in}}%
\pgfusepath{clip}%
\pgfsetbuttcap%
\pgfsetroundjoin%
\definecolor{currentfill}{rgb}{0.121569,0.466667,0.705882}%
\pgfsetfillcolor{currentfill}%
\pgfsetfillopacity{0.842018}%
\pgfsetlinewidth{1.003750pt}%
\definecolor{currentstroke}{rgb}{0.121569,0.466667,0.705882}%
\pgfsetstrokecolor{currentstroke}%
\pgfsetstrokeopacity{0.842018}%
\pgfsetdash{}{0pt}%
\pgfpathmoveto{\pgfqpoint{1.839141in}{2.284158in}}%
\pgfpathcurveto{\pgfqpoint{1.847377in}{2.284158in}}{\pgfqpoint{1.855277in}{2.287431in}}{\pgfqpoint{1.861101in}{2.293255in}}%
\pgfpathcurveto{\pgfqpoint{1.866925in}{2.299079in}}{\pgfqpoint{1.870197in}{2.306979in}}{\pgfqpoint{1.870197in}{2.315215in}}%
\pgfpathcurveto{\pgfqpoint{1.870197in}{2.323451in}}{\pgfqpoint{1.866925in}{2.331351in}}{\pgfqpoint{1.861101in}{2.337175in}}%
\pgfpathcurveto{\pgfqpoint{1.855277in}{2.342999in}}{\pgfqpoint{1.847377in}{2.346271in}}{\pgfqpoint{1.839141in}{2.346271in}}%
\pgfpathcurveto{\pgfqpoint{1.830905in}{2.346271in}}{\pgfqpoint{1.823005in}{2.342999in}}{\pgfqpoint{1.817181in}{2.337175in}}%
\pgfpathcurveto{\pgfqpoint{1.811357in}{2.331351in}}{\pgfqpoint{1.808084in}{2.323451in}}{\pgfqpoint{1.808084in}{2.315215in}}%
\pgfpathcurveto{\pgfqpoint{1.808084in}{2.306979in}}{\pgfqpoint{1.811357in}{2.299079in}}{\pgfqpoint{1.817181in}{2.293255in}}%
\pgfpathcurveto{\pgfqpoint{1.823005in}{2.287431in}}{\pgfqpoint{1.830905in}{2.284158in}}{\pgfqpoint{1.839141in}{2.284158in}}%
\pgfpathclose%
\pgfusepath{stroke,fill}%
\end{pgfscope}%
\begin{pgfscope}%
\pgfpathrectangle{\pgfqpoint{0.100000in}{0.212622in}}{\pgfqpoint{3.696000in}{3.696000in}}%
\pgfusepath{clip}%
\pgfsetbuttcap%
\pgfsetroundjoin%
\definecolor{currentfill}{rgb}{0.121569,0.466667,0.705882}%
\pgfsetfillcolor{currentfill}%
\pgfsetfillopacity{0.842354}%
\pgfsetlinewidth{1.003750pt}%
\definecolor{currentstroke}{rgb}{0.121569,0.466667,0.705882}%
\pgfsetstrokecolor{currentstroke}%
\pgfsetstrokeopacity{0.842354}%
\pgfsetdash{}{0pt}%
\pgfpathmoveto{\pgfqpoint{2.367745in}{2.480439in}}%
\pgfpathcurveto{\pgfqpoint{2.375981in}{2.480439in}}{\pgfqpoint{2.383881in}{2.483711in}}{\pgfqpoint{2.389705in}{2.489535in}}%
\pgfpathcurveto{\pgfqpoint{2.395529in}{2.495359in}}{\pgfqpoint{2.398801in}{2.503259in}}{\pgfqpoint{2.398801in}{2.511495in}}%
\pgfpathcurveto{\pgfqpoint{2.398801in}{2.519731in}}{\pgfqpoint{2.395529in}{2.527631in}}{\pgfqpoint{2.389705in}{2.533455in}}%
\pgfpathcurveto{\pgfqpoint{2.383881in}{2.539279in}}{\pgfqpoint{2.375981in}{2.542552in}}{\pgfqpoint{2.367745in}{2.542552in}}%
\pgfpathcurveto{\pgfqpoint{2.359508in}{2.542552in}}{\pgfqpoint{2.351608in}{2.539279in}}{\pgfqpoint{2.345784in}{2.533455in}}%
\pgfpathcurveto{\pgfqpoint{2.339961in}{2.527631in}}{\pgfqpoint{2.336688in}{2.519731in}}{\pgfqpoint{2.336688in}{2.511495in}}%
\pgfpathcurveto{\pgfqpoint{2.336688in}{2.503259in}}{\pgfqpoint{2.339961in}{2.495359in}}{\pgfqpoint{2.345784in}{2.489535in}}%
\pgfpathcurveto{\pgfqpoint{2.351608in}{2.483711in}}{\pgfqpoint{2.359508in}{2.480439in}}{\pgfqpoint{2.367745in}{2.480439in}}%
\pgfpathclose%
\pgfusepath{stroke,fill}%
\end{pgfscope}%
\begin{pgfscope}%
\pgfpathrectangle{\pgfqpoint{0.100000in}{0.212622in}}{\pgfqpoint{3.696000in}{3.696000in}}%
\pgfusepath{clip}%
\pgfsetbuttcap%
\pgfsetroundjoin%
\definecolor{currentfill}{rgb}{0.121569,0.466667,0.705882}%
\pgfsetfillcolor{currentfill}%
\pgfsetfillopacity{0.842586}%
\pgfsetlinewidth{1.003750pt}%
\definecolor{currentstroke}{rgb}{0.121569,0.466667,0.705882}%
\pgfsetstrokecolor{currentstroke}%
\pgfsetstrokeopacity{0.842586}%
\pgfsetdash{}{0pt}%
\pgfpathmoveto{\pgfqpoint{1.309065in}{1.345836in}}%
\pgfpathcurveto{\pgfqpoint{1.317301in}{1.345836in}}{\pgfqpoint{1.325201in}{1.349108in}}{\pgfqpoint{1.331025in}{1.354932in}}%
\pgfpathcurveto{\pgfqpoint{1.336849in}{1.360756in}}{\pgfqpoint{1.340122in}{1.368656in}}{\pgfqpoint{1.340122in}{1.376892in}}%
\pgfpathcurveto{\pgfqpoint{1.340122in}{1.385128in}}{\pgfqpoint{1.336849in}{1.393029in}}{\pgfqpoint{1.331025in}{1.398852in}}%
\pgfpathcurveto{\pgfqpoint{1.325201in}{1.404676in}}{\pgfqpoint{1.317301in}{1.407949in}}{\pgfqpoint{1.309065in}{1.407949in}}%
\pgfpathcurveto{\pgfqpoint{1.300829in}{1.407949in}}{\pgfqpoint{1.292929in}{1.404676in}}{\pgfqpoint{1.287105in}{1.398852in}}%
\pgfpathcurveto{\pgfqpoint{1.281281in}{1.393029in}}{\pgfqpoint{1.278009in}{1.385128in}}{\pgfqpoint{1.278009in}{1.376892in}}%
\pgfpathcurveto{\pgfqpoint{1.278009in}{1.368656in}}{\pgfqpoint{1.281281in}{1.360756in}}{\pgfqpoint{1.287105in}{1.354932in}}%
\pgfpathcurveto{\pgfqpoint{1.292929in}{1.349108in}}{\pgfqpoint{1.300829in}{1.345836in}}{\pgfqpoint{1.309065in}{1.345836in}}%
\pgfpathclose%
\pgfusepath{stroke,fill}%
\end{pgfscope}%
\begin{pgfscope}%
\pgfpathrectangle{\pgfqpoint{0.100000in}{0.212622in}}{\pgfqpoint{3.696000in}{3.696000in}}%
\pgfusepath{clip}%
\pgfsetbuttcap%
\pgfsetroundjoin%
\definecolor{currentfill}{rgb}{0.121569,0.466667,0.705882}%
\pgfsetfillcolor{currentfill}%
\pgfsetfillopacity{0.842592}%
\pgfsetlinewidth{1.003750pt}%
\definecolor{currentstroke}{rgb}{0.121569,0.466667,0.705882}%
\pgfsetstrokecolor{currentstroke}%
\pgfsetstrokeopacity{0.842592}%
\pgfsetdash{}{0pt}%
\pgfpathmoveto{\pgfqpoint{1.838387in}{2.281786in}}%
\pgfpathcurveto{\pgfqpoint{1.846623in}{2.281786in}}{\pgfqpoint{1.854523in}{2.285058in}}{\pgfqpoint{1.860347in}{2.290882in}}%
\pgfpathcurveto{\pgfqpoint{1.866171in}{2.296706in}}{\pgfqpoint{1.869443in}{2.304606in}}{\pgfqpoint{1.869443in}{2.312842in}}%
\pgfpathcurveto{\pgfqpoint{1.869443in}{2.321079in}}{\pgfqpoint{1.866171in}{2.328979in}}{\pgfqpoint{1.860347in}{2.334803in}}%
\pgfpathcurveto{\pgfqpoint{1.854523in}{2.340627in}}{\pgfqpoint{1.846623in}{2.343899in}}{\pgfqpoint{1.838387in}{2.343899in}}%
\pgfpathcurveto{\pgfqpoint{1.830151in}{2.343899in}}{\pgfqpoint{1.822251in}{2.340627in}}{\pgfqpoint{1.816427in}{2.334803in}}%
\pgfpathcurveto{\pgfqpoint{1.810603in}{2.328979in}}{\pgfqpoint{1.807330in}{2.321079in}}{\pgfqpoint{1.807330in}{2.312842in}}%
\pgfpathcurveto{\pgfqpoint{1.807330in}{2.304606in}}{\pgfqpoint{1.810603in}{2.296706in}}{\pgfqpoint{1.816427in}{2.290882in}}%
\pgfpathcurveto{\pgfqpoint{1.822251in}{2.285058in}}{\pgfqpoint{1.830151in}{2.281786in}}{\pgfqpoint{1.838387in}{2.281786in}}%
\pgfpathclose%
\pgfusepath{stroke,fill}%
\end{pgfscope}%
\begin{pgfscope}%
\pgfpathrectangle{\pgfqpoint{0.100000in}{0.212622in}}{\pgfqpoint{3.696000in}{3.696000in}}%
\pgfusepath{clip}%
\pgfsetbuttcap%
\pgfsetroundjoin%
\definecolor{currentfill}{rgb}{0.121569,0.466667,0.705882}%
\pgfsetfillcolor{currentfill}%
\pgfsetfillopacity{0.842865}%
\pgfsetlinewidth{1.003750pt}%
\definecolor{currentstroke}{rgb}{0.121569,0.466667,0.705882}%
\pgfsetstrokecolor{currentstroke}%
\pgfsetstrokeopacity{0.842865}%
\pgfsetdash{}{0pt}%
\pgfpathmoveto{\pgfqpoint{1.837667in}{2.280471in}}%
\pgfpathcurveto{\pgfqpoint{1.845904in}{2.280471in}}{\pgfqpoint{1.853804in}{2.283744in}}{\pgfqpoint{1.859628in}{2.289568in}}%
\pgfpathcurveto{\pgfqpoint{1.865451in}{2.295392in}}{\pgfqpoint{1.868724in}{2.303292in}}{\pgfqpoint{1.868724in}{2.311528in}}%
\pgfpathcurveto{\pgfqpoint{1.868724in}{2.319764in}}{\pgfqpoint{1.865451in}{2.327664in}}{\pgfqpoint{1.859628in}{2.333488in}}%
\pgfpathcurveto{\pgfqpoint{1.853804in}{2.339312in}}{\pgfqpoint{1.845904in}{2.342584in}}{\pgfqpoint{1.837667in}{2.342584in}}%
\pgfpathcurveto{\pgfqpoint{1.829431in}{2.342584in}}{\pgfqpoint{1.821531in}{2.339312in}}{\pgfqpoint{1.815707in}{2.333488in}}%
\pgfpathcurveto{\pgfqpoint{1.809883in}{2.327664in}}{\pgfqpoint{1.806611in}{2.319764in}}{\pgfqpoint{1.806611in}{2.311528in}}%
\pgfpathcurveto{\pgfqpoint{1.806611in}{2.303292in}}{\pgfqpoint{1.809883in}{2.295392in}}{\pgfqpoint{1.815707in}{2.289568in}}%
\pgfpathcurveto{\pgfqpoint{1.821531in}{2.283744in}}{\pgfqpoint{1.829431in}{2.280471in}}{\pgfqpoint{1.837667in}{2.280471in}}%
\pgfpathclose%
\pgfusepath{stroke,fill}%
\end{pgfscope}%
\begin{pgfscope}%
\pgfpathrectangle{\pgfqpoint{0.100000in}{0.212622in}}{\pgfqpoint{3.696000in}{3.696000in}}%
\pgfusepath{clip}%
\pgfsetbuttcap%
\pgfsetroundjoin%
\definecolor{currentfill}{rgb}{0.121569,0.466667,0.705882}%
\pgfsetfillcolor{currentfill}%
\pgfsetfillopacity{0.843221}%
\pgfsetlinewidth{1.003750pt}%
\definecolor{currentstroke}{rgb}{0.121569,0.466667,0.705882}%
\pgfsetstrokecolor{currentstroke}%
\pgfsetstrokeopacity{0.843221}%
\pgfsetdash{}{0pt}%
\pgfpathmoveto{\pgfqpoint{2.365879in}{2.473530in}}%
\pgfpathcurveto{\pgfqpoint{2.374116in}{2.473530in}}{\pgfqpoint{2.382016in}{2.476802in}}{\pgfqpoint{2.387840in}{2.482626in}}%
\pgfpathcurveto{\pgfqpoint{2.393664in}{2.488450in}}{\pgfqpoint{2.396936in}{2.496350in}}{\pgfqpoint{2.396936in}{2.504586in}}%
\pgfpathcurveto{\pgfqpoint{2.396936in}{2.512823in}}{\pgfqpoint{2.393664in}{2.520723in}}{\pgfqpoint{2.387840in}{2.526547in}}%
\pgfpathcurveto{\pgfqpoint{2.382016in}{2.532371in}}{\pgfqpoint{2.374116in}{2.535643in}}{\pgfqpoint{2.365879in}{2.535643in}}%
\pgfpathcurveto{\pgfqpoint{2.357643in}{2.535643in}}{\pgfqpoint{2.349743in}{2.532371in}}{\pgfqpoint{2.343919in}{2.526547in}}%
\pgfpathcurveto{\pgfqpoint{2.338095in}{2.520723in}}{\pgfqpoint{2.334823in}{2.512823in}}{\pgfqpoint{2.334823in}{2.504586in}}%
\pgfpathcurveto{\pgfqpoint{2.334823in}{2.496350in}}{\pgfqpoint{2.338095in}{2.488450in}}{\pgfqpoint{2.343919in}{2.482626in}}%
\pgfpathcurveto{\pgfqpoint{2.349743in}{2.476802in}}{\pgfqpoint{2.357643in}{2.473530in}}{\pgfqpoint{2.365879in}{2.473530in}}%
\pgfpathclose%
\pgfusepath{stroke,fill}%
\end{pgfscope}%
\begin{pgfscope}%
\pgfpathrectangle{\pgfqpoint{0.100000in}{0.212622in}}{\pgfqpoint{3.696000in}{3.696000in}}%
\pgfusepath{clip}%
\pgfsetbuttcap%
\pgfsetroundjoin%
\definecolor{currentfill}{rgb}{0.121569,0.466667,0.705882}%
\pgfsetfillcolor{currentfill}%
\pgfsetfillopacity{0.843497}%
\pgfsetlinewidth{1.003750pt}%
\definecolor{currentstroke}{rgb}{0.121569,0.466667,0.705882}%
\pgfsetstrokecolor{currentstroke}%
\pgfsetstrokeopacity{0.843497}%
\pgfsetdash{}{0pt}%
\pgfpathmoveto{\pgfqpoint{1.837147in}{2.277600in}}%
\pgfpathcurveto{\pgfqpoint{1.845384in}{2.277600in}}{\pgfqpoint{1.853284in}{2.280873in}}{\pgfqpoint{1.859108in}{2.286697in}}%
\pgfpathcurveto{\pgfqpoint{1.864932in}{2.292521in}}{\pgfqpoint{1.868204in}{2.300421in}}{\pgfqpoint{1.868204in}{2.308657in}}%
\pgfpathcurveto{\pgfqpoint{1.868204in}{2.316893in}}{\pgfqpoint{1.864932in}{2.324793in}}{\pgfqpoint{1.859108in}{2.330617in}}%
\pgfpathcurveto{\pgfqpoint{1.853284in}{2.336441in}}{\pgfqpoint{1.845384in}{2.339713in}}{\pgfqpoint{1.837147in}{2.339713in}}%
\pgfpathcurveto{\pgfqpoint{1.828911in}{2.339713in}}{\pgfqpoint{1.821011in}{2.336441in}}{\pgfqpoint{1.815187in}{2.330617in}}%
\pgfpathcurveto{\pgfqpoint{1.809363in}{2.324793in}}{\pgfqpoint{1.806091in}{2.316893in}}{\pgfqpoint{1.806091in}{2.308657in}}%
\pgfpathcurveto{\pgfqpoint{1.806091in}{2.300421in}}{\pgfqpoint{1.809363in}{2.292521in}}{\pgfqpoint{1.815187in}{2.286697in}}%
\pgfpathcurveto{\pgfqpoint{1.821011in}{2.280873in}}{\pgfqpoint{1.828911in}{2.277600in}}{\pgfqpoint{1.837147in}{2.277600in}}%
\pgfpathclose%
\pgfusepath{stroke,fill}%
\end{pgfscope}%
\begin{pgfscope}%
\pgfpathrectangle{\pgfqpoint{0.100000in}{0.212622in}}{\pgfqpoint{3.696000in}{3.696000in}}%
\pgfusepath{clip}%
\pgfsetbuttcap%
\pgfsetroundjoin%
\definecolor{currentfill}{rgb}{0.121569,0.466667,0.705882}%
\pgfsetfillcolor{currentfill}%
\pgfsetfillopacity{0.843664}%
\pgfsetlinewidth{1.003750pt}%
\definecolor{currentstroke}{rgb}{0.121569,0.466667,0.705882}%
\pgfsetstrokecolor{currentstroke}%
\pgfsetstrokeopacity{0.843664}%
\pgfsetdash{}{0pt}%
\pgfpathmoveto{\pgfqpoint{2.941517in}{1.690567in}}%
\pgfpathcurveto{\pgfqpoint{2.949753in}{1.690567in}}{\pgfqpoint{2.957653in}{1.693839in}}{\pgfqpoint{2.963477in}{1.699663in}}%
\pgfpathcurveto{\pgfqpoint{2.969301in}{1.705487in}}{\pgfqpoint{2.972574in}{1.713387in}}{\pgfqpoint{2.972574in}{1.721623in}}%
\pgfpathcurveto{\pgfqpoint{2.972574in}{1.729859in}}{\pgfqpoint{2.969301in}{1.737759in}}{\pgfqpoint{2.963477in}{1.743583in}}%
\pgfpathcurveto{\pgfqpoint{2.957653in}{1.749407in}}{\pgfqpoint{2.949753in}{1.752680in}}{\pgfqpoint{2.941517in}{1.752680in}}%
\pgfpathcurveto{\pgfqpoint{2.933281in}{1.752680in}}{\pgfqpoint{2.925381in}{1.749407in}}{\pgfqpoint{2.919557in}{1.743583in}}%
\pgfpathcurveto{\pgfqpoint{2.913733in}{1.737759in}}{\pgfqpoint{2.910461in}{1.729859in}}{\pgfqpoint{2.910461in}{1.721623in}}%
\pgfpathcurveto{\pgfqpoint{2.910461in}{1.713387in}}{\pgfqpoint{2.913733in}{1.705487in}}{\pgfqpoint{2.919557in}{1.699663in}}%
\pgfpathcurveto{\pgfqpoint{2.925381in}{1.693839in}}{\pgfqpoint{2.933281in}{1.690567in}}{\pgfqpoint{2.941517in}{1.690567in}}%
\pgfpathclose%
\pgfusepath{stroke,fill}%
\end{pgfscope}%
\begin{pgfscope}%
\pgfpathrectangle{\pgfqpoint{0.100000in}{0.212622in}}{\pgfqpoint{3.696000in}{3.696000in}}%
\pgfusepath{clip}%
\pgfsetbuttcap%
\pgfsetroundjoin%
\definecolor{currentfill}{rgb}{0.121569,0.466667,0.705882}%
\pgfsetfillcolor{currentfill}%
\pgfsetfillopacity{0.843718}%
\pgfsetlinewidth{1.003750pt}%
\definecolor{currentstroke}{rgb}{0.121569,0.466667,0.705882}%
\pgfsetstrokecolor{currentstroke}%
\pgfsetstrokeopacity{0.843718}%
\pgfsetdash{}{0pt}%
\pgfpathmoveto{\pgfqpoint{1.314829in}{1.342347in}}%
\pgfpathcurveto{\pgfqpoint{1.323065in}{1.342347in}}{\pgfqpoint{1.330965in}{1.345620in}}{\pgfqpoint{1.336789in}{1.351444in}}%
\pgfpathcurveto{\pgfqpoint{1.342613in}{1.357267in}}{\pgfqpoint{1.345886in}{1.365167in}}{\pgfqpoint{1.345886in}{1.373404in}}%
\pgfpathcurveto{\pgfqpoint{1.345886in}{1.381640in}}{\pgfqpoint{1.342613in}{1.389540in}}{\pgfqpoint{1.336789in}{1.395364in}}%
\pgfpathcurveto{\pgfqpoint{1.330965in}{1.401188in}}{\pgfqpoint{1.323065in}{1.404460in}}{\pgfqpoint{1.314829in}{1.404460in}}%
\pgfpathcurveto{\pgfqpoint{1.306593in}{1.404460in}}{\pgfqpoint{1.298693in}{1.401188in}}{\pgfqpoint{1.292869in}{1.395364in}}%
\pgfpathcurveto{\pgfqpoint{1.287045in}{1.389540in}}{\pgfqpoint{1.283773in}{1.381640in}}{\pgfqpoint{1.283773in}{1.373404in}}%
\pgfpathcurveto{\pgfqpoint{1.283773in}{1.365167in}}{\pgfqpoint{1.287045in}{1.357267in}}{\pgfqpoint{1.292869in}{1.351444in}}%
\pgfpathcurveto{\pgfqpoint{1.298693in}{1.345620in}}{\pgfqpoint{1.306593in}{1.342347in}}{\pgfqpoint{1.314829in}{1.342347in}}%
\pgfpathclose%
\pgfusepath{stroke,fill}%
\end{pgfscope}%
\begin{pgfscope}%
\pgfpathrectangle{\pgfqpoint{0.100000in}{0.212622in}}{\pgfqpoint{3.696000in}{3.696000in}}%
\pgfusepath{clip}%
\pgfsetbuttcap%
\pgfsetroundjoin%
\definecolor{currentfill}{rgb}{0.121569,0.466667,0.705882}%
\pgfsetfillcolor{currentfill}%
\pgfsetfillopacity{0.843740}%
\pgfsetlinewidth{1.003750pt}%
\definecolor{currentstroke}{rgb}{0.121569,0.466667,0.705882}%
\pgfsetstrokecolor{currentstroke}%
\pgfsetstrokeopacity{0.843740}%
\pgfsetdash{}{0pt}%
\pgfpathmoveto{\pgfqpoint{2.363687in}{2.470629in}}%
\pgfpathcurveto{\pgfqpoint{2.371924in}{2.470629in}}{\pgfqpoint{2.379824in}{2.473902in}}{\pgfqpoint{2.385648in}{2.479725in}}%
\pgfpathcurveto{\pgfqpoint{2.391472in}{2.485549in}}{\pgfqpoint{2.394744in}{2.493449in}}{\pgfqpoint{2.394744in}{2.501686in}}%
\pgfpathcurveto{\pgfqpoint{2.394744in}{2.509922in}}{\pgfqpoint{2.391472in}{2.517822in}}{\pgfqpoint{2.385648in}{2.523646in}}%
\pgfpathcurveto{\pgfqpoint{2.379824in}{2.529470in}}{\pgfqpoint{2.371924in}{2.532742in}}{\pgfqpoint{2.363687in}{2.532742in}}%
\pgfpathcurveto{\pgfqpoint{2.355451in}{2.532742in}}{\pgfqpoint{2.347551in}{2.529470in}}{\pgfqpoint{2.341727in}{2.523646in}}%
\pgfpathcurveto{\pgfqpoint{2.335903in}{2.517822in}}{\pgfqpoint{2.332631in}{2.509922in}}{\pgfqpoint{2.332631in}{2.501686in}}%
\pgfpathcurveto{\pgfqpoint{2.332631in}{2.493449in}}{\pgfqpoint{2.335903in}{2.485549in}}{\pgfqpoint{2.341727in}{2.479725in}}%
\pgfpathcurveto{\pgfqpoint{2.347551in}{2.473902in}}{\pgfqpoint{2.355451in}{2.470629in}}{\pgfqpoint{2.363687in}{2.470629in}}%
\pgfpathclose%
\pgfusepath{stroke,fill}%
\end{pgfscope}%
\begin{pgfscope}%
\pgfpathrectangle{\pgfqpoint{0.100000in}{0.212622in}}{\pgfqpoint{3.696000in}{3.696000in}}%
\pgfusepath{clip}%
\pgfsetbuttcap%
\pgfsetroundjoin%
\definecolor{currentfill}{rgb}{0.121569,0.466667,0.705882}%
\pgfsetfillcolor{currentfill}%
\pgfsetfillopacity{0.843849}%
\pgfsetlinewidth{1.003750pt}%
\definecolor{currentstroke}{rgb}{0.121569,0.466667,0.705882}%
\pgfsetstrokecolor{currentstroke}%
\pgfsetstrokeopacity{0.843849}%
\pgfsetdash{}{0pt}%
\pgfpathmoveto{\pgfqpoint{1.836791in}{2.276035in}}%
\pgfpathcurveto{\pgfqpoint{1.845028in}{2.276035in}}{\pgfqpoint{1.852928in}{2.279308in}}{\pgfqpoint{1.858752in}{2.285132in}}%
\pgfpathcurveto{\pgfqpoint{1.864575in}{2.290956in}}{\pgfqpoint{1.867848in}{2.298856in}}{\pgfqpoint{1.867848in}{2.307092in}}%
\pgfpathcurveto{\pgfqpoint{1.867848in}{2.315328in}}{\pgfqpoint{1.864575in}{2.323228in}}{\pgfqpoint{1.858752in}{2.329052in}}%
\pgfpathcurveto{\pgfqpoint{1.852928in}{2.334876in}}{\pgfqpoint{1.845028in}{2.338148in}}{\pgfqpoint{1.836791in}{2.338148in}}%
\pgfpathcurveto{\pgfqpoint{1.828555in}{2.338148in}}{\pgfqpoint{1.820655in}{2.334876in}}{\pgfqpoint{1.814831in}{2.329052in}}%
\pgfpathcurveto{\pgfqpoint{1.809007in}{2.323228in}}{\pgfqpoint{1.805735in}{2.315328in}}{\pgfqpoint{1.805735in}{2.307092in}}%
\pgfpathcurveto{\pgfqpoint{1.805735in}{2.298856in}}{\pgfqpoint{1.809007in}{2.290956in}}{\pgfqpoint{1.814831in}{2.285132in}}%
\pgfpathcurveto{\pgfqpoint{1.820655in}{2.279308in}}{\pgfqpoint{1.828555in}{2.276035in}}{\pgfqpoint{1.836791in}{2.276035in}}%
\pgfpathclose%
\pgfusepath{stroke,fill}%
\end{pgfscope}%
\begin{pgfscope}%
\pgfpathrectangle{\pgfqpoint{0.100000in}{0.212622in}}{\pgfqpoint{3.696000in}{3.696000in}}%
\pgfusepath{clip}%
\pgfsetbuttcap%
\pgfsetroundjoin%
\definecolor{currentfill}{rgb}{0.121569,0.466667,0.705882}%
\pgfsetfillcolor{currentfill}%
\pgfsetfillopacity{0.844387}%
\pgfsetlinewidth{1.003750pt}%
\definecolor{currentstroke}{rgb}{0.121569,0.466667,0.705882}%
\pgfsetstrokecolor{currentstroke}%
\pgfsetstrokeopacity{0.844387}%
\pgfsetdash{}{0pt}%
\pgfpathmoveto{\pgfqpoint{1.836065in}{2.273587in}}%
\pgfpathcurveto{\pgfqpoint{1.844302in}{2.273587in}}{\pgfqpoint{1.852202in}{2.276860in}}{\pgfqpoint{1.858026in}{2.282684in}}%
\pgfpathcurveto{\pgfqpoint{1.863850in}{2.288508in}}{\pgfqpoint{1.867122in}{2.296408in}}{\pgfqpoint{1.867122in}{2.304644in}}%
\pgfpathcurveto{\pgfqpoint{1.867122in}{2.312880in}}{\pgfqpoint{1.863850in}{2.320780in}}{\pgfqpoint{1.858026in}{2.326604in}}%
\pgfpathcurveto{\pgfqpoint{1.852202in}{2.332428in}}{\pgfqpoint{1.844302in}{2.335700in}}{\pgfqpoint{1.836065in}{2.335700in}}%
\pgfpathcurveto{\pgfqpoint{1.827829in}{2.335700in}}{\pgfqpoint{1.819929in}{2.332428in}}{\pgfqpoint{1.814105in}{2.326604in}}%
\pgfpathcurveto{\pgfqpoint{1.808281in}{2.320780in}}{\pgfqpoint{1.805009in}{2.312880in}}{\pgfqpoint{1.805009in}{2.304644in}}%
\pgfpathcurveto{\pgfqpoint{1.805009in}{2.296408in}}{\pgfqpoint{1.808281in}{2.288508in}}{\pgfqpoint{1.814105in}{2.282684in}}%
\pgfpathcurveto{\pgfqpoint{1.819929in}{2.276860in}}{\pgfqpoint{1.827829in}{2.273587in}}{\pgfqpoint{1.836065in}{2.273587in}}%
\pgfpathclose%
\pgfusepath{stroke,fill}%
\end{pgfscope}%
\begin{pgfscope}%
\pgfpathrectangle{\pgfqpoint{0.100000in}{0.212622in}}{\pgfqpoint{3.696000in}{3.696000in}}%
\pgfusepath{clip}%
\pgfsetbuttcap%
\pgfsetroundjoin%
\definecolor{currentfill}{rgb}{0.121569,0.466667,0.705882}%
\pgfsetfillcolor{currentfill}%
\pgfsetfillopacity{0.844517}%
\pgfsetlinewidth{1.003750pt}%
\definecolor{currentstroke}{rgb}{0.121569,0.466667,0.705882}%
\pgfsetstrokecolor{currentstroke}%
\pgfsetstrokeopacity{0.844517}%
\pgfsetdash{}{0pt}%
\pgfpathmoveto{\pgfqpoint{2.359593in}{2.464838in}}%
\pgfpathcurveto{\pgfqpoint{2.367829in}{2.464838in}}{\pgfqpoint{2.375729in}{2.468110in}}{\pgfqpoint{2.381553in}{2.473934in}}%
\pgfpathcurveto{\pgfqpoint{2.387377in}{2.479758in}}{\pgfqpoint{2.390649in}{2.487658in}}{\pgfqpoint{2.390649in}{2.495894in}}%
\pgfpathcurveto{\pgfqpoint{2.390649in}{2.504131in}}{\pgfqpoint{2.387377in}{2.512031in}}{\pgfqpoint{2.381553in}{2.517855in}}%
\pgfpathcurveto{\pgfqpoint{2.375729in}{2.523679in}}{\pgfqpoint{2.367829in}{2.526951in}}{\pgfqpoint{2.359593in}{2.526951in}}%
\pgfpathcurveto{\pgfqpoint{2.351357in}{2.526951in}}{\pgfqpoint{2.343457in}{2.523679in}}{\pgfqpoint{2.337633in}{2.517855in}}%
\pgfpathcurveto{\pgfqpoint{2.331809in}{2.512031in}}{\pgfqpoint{2.328536in}{2.504131in}}{\pgfqpoint{2.328536in}{2.495894in}}%
\pgfpathcurveto{\pgfqpoint{2.328536in}{2.487658in}}{\pgfqpoint{2.331809in}{2.479758in}}{\pgfqpoint{2.337633in}{2.473934in}}%
\pgfpathcurveto{\pgfqpoint{2.343457in}{2.468110in}}{\pgfqpoint{2.351357in}{2.464838in}}{\pgfqpoint{2.359593in}{2.464838in}}%
\pgfpathclose%
\pgfusepath{stroke,fill}%
\end{pgfscope}%
\begin{pgfscope}%
\pgfpathrectangle{\pgfqpoint{0.100000in}{0.212622in}}{\pgfqpoint{3.696000in}{3.696000in}}%
\pgfusepath{clip}%
\pgfsetbuttcap%
\pgfsetroundjoin%
\definecolor{currentfill}{rgb}{0.121569,0.466667,0.705882}%
\pgfsetfillcolor{currentfill}%
\pgfsetfillopacity{0.844939}%
\pgfsetlinewidth{1.003750pt}%
\definecolor{currentstroke}{rgb}{0.121569,0.466667,0.705882}%
\pgfsetstrokecolor{currentstroke}%
\pgfsetstrokeopacity{0.844939}%
\pgfsetdash{}{0pt}%
\pgfpathmoveto{\pgfqpoint{1.322032in}{1.340133in}}%
\pgfpathcurveto{\pgfqpoint{1.330269in}{1.340133in}}{\pgfqpoint{1.338169in}{1.343405in}}{\pgfqpoint{1.343993in}{1.349229in}}%
\pgfpathcurveto{\pgfqpoint{1.349816in}{1.355053in}}{\pgfqpoint{1.353089in}{1.362953in}}{\pgfqpoint{1.353089in}{1.371190in}}%
\pgfpathcurveto{\pgfqpoint{1.353089in}{1.379426in}}{\pgfqpoint{1.349816in}{1.387326in}}{\pgfqpoint{1.343993in}{1.393150in}}%
\pgfpathcurveto{\pgfqpoint{1.338169in}{1.398974in}}{\pgfqpoint{1.330269in}{1.402246in}}{\pgfqpoint{1.322032in}{1.402246in}}%
\pgfpathcurveto{\pgfqpoint{1.313796in}{1.402246in}}{\pgfqpoint{1.305896in}{1.398974in}}{\pgfqpoint{1.300072in}{1.393150in}}%
\pgfpathcurveto{\pgfqpoint{1.294248in}{1.387326in}}{\pgfqpoint{1.290976in}{1.379426in}}{\pgfqpoint{1.290976in}{1.371190in}}%
\pgfpathcurveto{\pgfqpoint{1.290976in}{1.362953in}}{\pgfqpoint{1.294248in}{1.355053in}}{\pgfqpoint{1.300072in}{1.349229in}}%
\pgfpathcurveto{\pgfqpoint{1.305896in}{1.343405in}}{\pgfqpoint{1.313796in}{1.340133in}}{\pgfqpoint{1.322032in}{1.340133in}}%
\pgfpathclose%
\pgfusepath{stroke,fill}%
\end{pgfscope}%
\begin{pgfscope}%
\pgfpathrectangle{\pgfqpoint{0.100000in}{0.212622in}}{\pgfqpoint{3.696000in}{3.696000in}}%
\pgfusepath{clip}%
\pgfsetbuttcap%
\pgfsetroundjoin%
\definecolor{currentfill}{rgb}{0.121569,0.466667,0.705882}%
\pgfsetfillcolor{currentfill}%
\pgfsetfillopacity{0.845231}%
\pgfsetlinewidth{1.003750pt}%
\definecolor{currentstroke}{rgb}{0.121569,0.466667,0.705882}%
\pgfsetstrokecolor{currentstroke}%
\pgfsetstrokeopacity{0.845231}%
\pgfsetdash{}{0pt}%
\pgfpathmoveto{\pgfqpoint{1.835606in}{2.269824in}}%
\pgfpathcurveto{\pgfqpoint{1.843843in}{2.269824in}}{\pgfqpoint{1.851743in}{2.273096in}}{\pgfqpoint{1.857567in}{2.278920in}}%
\pgfpathcurveto{\pgfqpoint{1.863391in}{2.284744in}}{\pgfqpoint{1.866663in}{2.292644in}}{\pgfqpoint{1.866663in}{2.300880in}}%
\pgfpathcurveto{\pgfqpoint{1.866663in}{2.309117in}}{\pgfqpoint{1.863391in}{2.317017in}}{\pgfqpoint{1.857567in}{2.322841in}}%
\pgfpathcurveto{\pgfqpoint{1.851743in}{2.328665in}}{\pgfqpoint{1.843843in}{2.331937in}}{\pgfqpoint{1.835606in}{2.331937in}}%
\pgfpathcurveto{\pgfqpoint{1.827370in}{2.331937in}}{\pgfqpoint{1.819470in}{2.328665in}}{\pgfqpoint{1.813646in}{2.322841in}}%
\pgfpathcurveto{\pgfqpoint{1.807822in}{2.317017in}}{\pgfqpoint{1.804550in}{2.309117in}}{\pgfqpoint{1.804550in}{2.300880in}}%
\pgfpathcurveto{\pgfqpoint{1.804550in}{2.292644in}}{\pgfqpoint{1.807822in}{2.284744in}}{\pgfqpoint{1.813646in}{2.278920in}}%
\pgfpathcurveto{\pgfqpoint{1.819470in}{2.273096in}}{\pgfqpoint{1.827370in}{2.269824in}}{\pgfqpoint{1.835606in}{2.269824in}}%
\pgfpathclose%
\pgfusepath{stroke,fill}%
\end{pgfscope}%
\begin{pgfscope}%
\pgfpathrectangle{\pgfqpoint{0.100000in}{0.212622in}}{\pgfqpoint{3.696000in}{3.696000in}}%
\pgfusepath{clip}%
\pgfsetbuttcap%
\pgfsetroundjoin%
\definecolor{currentfill}{rgb}{0.121569,0.466667,0.705882}%
\pgfsetfillcolor{currentfill}%
\pgfsetfillopacity{0.845464}%
\pgfsetlinewidth{1.003750pt}%
\definecolor{currentstroke}{rgb}{0.121569,0.466667,0.705882}%
\pgfsetstrokecolor{currentstroke}%
\pgfsetstrokeopacity{0.845464}%
\pgfsetdash{}{0pt}%
\pgfpathmoveto{\pgfqpoint{2.357460in}{2.459340in}}%
\pgfpathcurveto{\pgfqpoint{2.365696in}{2.459340in}}{\pgfqpoint{2.373596in}{2.462612in}}{\pgfqpoint{2.379420in}{2.468436in}}%
\pgfpathcurveto{\pgfqpoint{2.385244in}{2.474260in}}{\pgfqpoint{2.388516in}{2.482160in}}{\pgfqpoint{2.388516in}{2.490397in}}%
\pgfpathcurveto{\pgfqpoint{2.388516in}{2.498633in}}{\pgfqpoint{2.385244in}{2.506533in}}{\pgfqpoint{2.379420in}{2.512357in}}%
\pgfpathcurveto{\pgfqpoint{2.373596in}{2.518181in}}{\pgfqpoint{2.365696in}{2.521453in}}{\pgfqpoint{2.357460in}{2.521453in}}%
\pgfpathcurveto{\pgfqpoint{2.349223in}{2.521453in}}{\pgfqpoint{2.341323in}{2.518181in}}{\pgfqpoint{2.335499in}{2.512357in}}%
\pgfpathcurveto{\pgfqpoint{2.329675in}{2.506533in}}{\pgfqpoint{2.326403in}{2.498633in}}{\pgfqpoint{2.326403in}{2.490397in}}%
\pgfpathcurveto{\pgfqpoint{2.326403in}{2.482160in}}{\pgfqpoint{2.329675in}{2.474260in}}{\pgfqpoint{2.335499in}{2.468436in}}%
\pgfpathcurveto{\pgfqpoint{2.341323in}{2.462612in}}{\pgfqpoint{2.349223in}{2.459340in}}{\pgfqpoint{2.357460in}{2.459340in}}%
\pgfpathclose%
\pgfusepath{stroke,fill}%
\end{pgfscope}%
\begin{pgfscope}%
\pgfpathrectangle{\pgfqpoint{0.100000in}{0.212622in}}{\pgfqpoint{3.696000in}{3.696000in}}%
\pgfusepath{clip}%
\pgfsetbuttcap%
\pgfsetroundjoin%
\definecolor{currentfill}{rgb}{0.121569,0.466667,0.705882}%
\pgfsetfillcolor{currentfill}%
\pgfsetfillopacity{0.846167}%
\pgfsetlinewidth{1.003750pt}%
\definecolor{currentstroke}{rgb}{0.121569,0.466667,0.705882}%
\pgfsetstrokecolor{currentstroke}%
\pgfsetstrokeopacity{0.846167}%
\pgfsetdash{}{0pt}%
\pgfpathmoveto{\pgfqpoint{1.833803in}{2.265156in}}%
\pgfpathcurveto{\pgfqpoint{1.842039in}{2.265156in}}{\pgfqpoint{1.849940in}{2.268429in}}{\pgfqpoint{1.855763in}{2.274253in}}%
\pgfpathcurveto{\pgfqpoint{1.861587in}{2.280076in}}{\pgfqpoint{1.864860in}{2.287976in}}{\pgfqpoint{1.864860in}{2.296213in}}%
\pgfpathcurveto{\pgfqpoint{1.864860in}{2.304449in}}{\pgfqpoint{1.861587in}{2.312349in}}{\pgfqpoint{1.855763in}{2.318173in}}%
\pgfpathcurveto{\pgfqpoint{1.849940in}{2.323997in}}{\pgfqpoint{1.842039in}{2.327269in}}{\pgfqpoint{1.833803in}{2.327269in}}%
\pgfpathcurveto{\pgfqpoint{1.825567in}{2.327269in}}{\pgfqpoint{1.817667in}{2.323997in}}{\pgfqpoint{1.811843in}{2.318173in}}%
\pgfpathcurveto{\pgfqpoint{1.806019in}{2.312349in}}{\pgfqpoint{1.802747in}{2.304449in}}{\pgfqpoint{1.802747in}{2.296213in}}%
\pgfpathcurveto{\pgfqpoint{1.802747in}{2.287976in}}{\pgfqpoint{1.806019in}{2.280076in}}{\pgfqpoint{1.811843in}{2.274253in}}%
\pgfpathcurveto{\pgfqpoint{1.817667in}{2.268429in}}{\pgfqpoint{1.825567in}{2.265156in}}{\pgfqpoint{1.833803in}{2.265156in}}%
\pgfpathclose%
\pgfusepath{stroke,fill}%
\end{pgfscope}%
\begin{pgfscope}%
\pgfpathrectangle{\pgfqpoint{0.100000in}{0.212622in}}{\pgfqpoint{3.696000in}{3.696000in}}%
\pgfusepath{clip}%
\pgfsetbuttcap%
\pgfsetroundjoin%
\definecolor{currentfill}{rgb}{0.121569,0.466667,0.705882}%
\pgfsetfillcolor{currentfill}%
\pgfsetfillopacity{0.846228}%
\pgfsetlinewidth{1.003750pt}%
\definecolor{currentstroke}{rgb}{0.121569,0.466667,0.705882}%
\pgfsetstrokecolor{currentstroke}%
\pgfsetstrokeopacity{0.846228}%
\pgfsetdash{}{0pt}%
\pgfpathmoveto{\pgfqpoint{2.356363in}{2.453625in}}%
\pgfpathcurveto{\pgfqpoint{2.364599in}{2.453625in}}{\pgfqpoint{2.372499in}{2.456897in}}{\pgfqpoint{2.378323in}{2.462721in}}%
\pgfpathcurveto{\pgfqpoint{2.384147in}{2.468545in}}{\pgfqpoint{2.387420in}{2.476445in}}{\pgfqpoint{2.387420in}{2.484681in}}%
\pgfpathcurveto{\pgfqpoint{2.387420in}{2.492918in}}{\pgfqpoint{2.384147in}{2.500818in}}{\pgfqpoint{2.378323in}{2.506642in}}%
\pgfpathcurveto{\pgfqpoint{2.372499in}{2.512466in}}{\pgfqpoint{2.364599in}{2.515738in}}{\pgfqpoint{2.356363in}{2.515738in}}%
\pgfpathcurveto{\pgfqpoint{2.348127in}{2.515738in}}{\pgfqpoint{2.340227in}{2.512466in}}{\pgfqpoint{2.334403in}{2.506642in}}%
\pgfpathcurveto{\pgfqpoint{2.328579in}{2.500818in}}{\pgfqpoint{2.325307in}{2.492918in}}{\pgfqpoint{2.325307in}{2.484681in}}%
\pgfpathcurveto{\pgfqpoint{2.325307in}{2.476445in}}{\pgfqpoint{2.328579in}{2.468545in}}{\pgfqpoint{2.334403in}{2.462721in}}%
\pgfpathcurveto{\pgfqpoint{2.340227in}{2.456897in}}{\pgfqpoint{2.348127in}{2.453625in}}{\pgfqpoint{2.356363in}{2.453625in}}%
\pgfpathclose%
\pgfusepath{stroke,fill}%
\end{pgfscope}%
\begin{pgfscope}%
\pgfpathrectangle{\pgfqpoint{0.100000in}{0.212622in}}{\pgfqpoint{3.696000in}{3.696000in}}%
\pgfusepath{clip}%
\pgfsetbuttcap%
\pgfsetroundjoin%
\definecolor{currentfill}{rgb}{0.121569,0.466667,0.705882}%
\pgfsetfillcolor{currentfill}%
\pgfsetfillopacity{0.846416}%
\pgfsetlinewidth{1.003750pt}%
\definecolor{currentstroke}{rgb}{0.121569,0.466667,0.705882}%
\pgfsetstrokecolor{currentstroke}%
\pgfsetstrokeopacity{0.846416}%
\pgfsetdash{}{0pt}%
\pgfpathmoveto{\pgfqpoint{1.329636in}{1.338862in}}%
\pgfpathcurveto{\pgfqpoint{1.337872in}{1.338862in}}{\pgfqpoint{1.345772in}{1.342135in}}{\pgfqpoint{1.351596in}{1.347959in}}%
\pgfpathcurveto{\pgfqpoint{1.357420in}{1.353782in}}{\pgfqpoint{1.360692in}{1.361683in}}{\pgfqpoint{1.360692in}{1.369919in}}%
\pgfpathcurveto{\pgfqpoint{1.360692in}{1.378155in}}{\pgfqpoint{1.357420in}{1.386055in}}{\pgfqpoint{1.351596in}{1.391879in}}%
\pgfpathcurveto{\pgfqpoint{1.345772in}{1.397703in}}{\pgfqpoint{1.337872in}{1.400975in}}{\pgfqpoint{1.329636in}{1.400975in}}%
\pgfpathcurveto{\pgfqpoint{1.321399in}{1.400975in}}{\pgfqpoint{1.313499in}{1.397703in}}{\pgfqpoint{1.307675in}{1.391879in}}%
\pgfpathcurveto{\pgfqpoint{1.301851in}{1.386055in}}{\pgfqpoint{1.298579in}{1.378155in}}{\pgfqpoint{1.298579in}{1.369919in}}%
\pgfpathcurveto{\pgfqpoint{1.298579in}{1.361683in}}{\pgfqpoint{1.301851in}{1.353782in}}{\pgfqpoint{1.307675in}{1.347959in}}%
\pgfpathcurveto{\pgfqpoint{1.313499in}{1.342135in}}{\pgfqpoint{1.321399in}{1.338862in}}{\pgfqpoint{1.329636in}{1.338862in}}%
\pgfpathclose%
\pgfusepath{stroke,fill}%
\end{pgfscope}%
\begin{pgfscope}%
\pgfpathrectangle{\pgfqpoint{0.100000in}{0.212622in}}{\pgfqpoint{3.696000in}{3.696000in}}%
\pgfusepath{clip}%
\pgfsetbuttcap%
\pgfsetroundjoin%
\definecolor{currentfill}{rgb}{0.121569,0.466667,0.705882}%
\pgfsetfillcolor{currentfill}%
\pgfsetfillopacity{0.846605}%
\pgfsetlinewidth{1.003750pt}%
\definecolor{currentstroke}{rgb}{0.121569,0.466667,0.705882}%
\pgfsetstrokecolor{currentstroke}%
\pgfsetstrokeopacity{0.846605}%
\pgfsetdash{}{0pt}%
\pgfpathmoveto{\pgfqpoint{1.832277in}{2.262793in}}%
\pgfpathcurveto{\pgfqpoint{1.840513in}{2.262793in}}{\pgfqpoint{1.848413in}{2.266065in}}{\pgfqpoint{1.854237in}{2.271889in}}%
\pgfpathcurveto{\pgfqpoint{1.860061in}{2.277713in}}{\pgfqpoint{1.863333in}{2.285613in}}{\pgfqpoint{1.863333in}{2.293849in}}%
\pgfpathcurveto{\pgfqpoint{1.863333in}{2.302085in}}{\pgfqpoint{1.860061in}{2.309985in}}{\pgfqpoint{1.854237in}{2.315809in}}%
\pgfpathcurveto{\pgfqpoint{1.848413in}{2.321633in}}{\pgfqpoint{1.840513in}{2.324906in}}{\pgfqpoint{1.832277in}{2.324906in}}%
\pgfpathcurveto{\pgfqpoint{1.824041in}{2.324906in}}{\pgfqpoint{1.816141in}{2.321633in}}{\pgfqpoint{1.810317in}{2.315809in}}%
\pgfpathcurveto{\pgfqpoint{1.804493in}{2.309985in}}{\pgfqpoint{1.801220in}{2.302085in}}{\pgfqpoint{1.801220in}{2.293849in}}%
\pgfpathcurveto{\pgfqpoint{1.801220in}{2.285613in}}{\pgfqpoint{1.804493in}{2.277713in}}{\pgfqpoint{1.810317in}{2.271889in}}%
\pgfpathcurveto{\pgfqpoint{1.816141in}{2.266065in}}{\pgfqpoint{1.824041in}{2.262793in}}{\pgfqpoint{1.832277in}{2.262793in}}%
\pgfpathclose%
\pgfusepath{stroke,fill}%
\end{pgfscope}%
\begin{pgfscope}%
\pgfpathrectangle{\pgfqpoint{0.100000in}{0.212622in}}{\pgfqpoint{3.696000in}{3.696000in}}%
\pgfusepath{clip}%
\pgfsetbuttcap%
\pgfsetroundjoin%
\definecolor{currentfill}{rgb}{0.121569,0.466667,0.705882}%
\pgfsetfillcolor{currentfill}%
\pgfsetfillopacity{0.846672}%
\pgfsetlinewidth{1.003750pt}%
\definecolor{currentstroke}{rgb}{0.121569,0.466667,0.705882}%
\pgfsetstrokecolor{currentstroke}%
\pgfsetstrokeopacity{0.846672}%
\pgfsetdash{}{0pt}%
\pgfpathmoveto{\pgfqpoint{2.354620in}{2.450766in}}%
\pgfpathcurveto{\pgfqpoint{2.362857in}{2.450766in}}{\pgfqpoint{2.370757in}{2.454038in}}{\pgfqpoint{2.376581in}{2.459862in}}%
\pgfpathcurveto{\pgfqpoint{2.382405in}{2.465686in}}{\pgfqpoint{2.385677in}{2.473586in}}{\pgfqpoint{2.385677in}{2.481822in}}%
\pgfpathcurveto{\pgfqpoint{2.385677in}{2.490058in}}{\pgfqpoint{2.382405in}{2.497959in}}{\pgfqpoint{2.376581in}{2.503782in}}%
\pgfpathcurveto{\pgfqpoint{2.370757in}{2.509606in}}{\pgfqpoint{2.362857in}{2.512879in}}{\pgfqpoint{2.354620in}{2.512879in}}%
\pgfpathcurveto{\pgfqpoint{2.346384in}{2.512879in}}{\pgfqpoint{2.338484in}{2.509606in}}{\pgfqpoint{2.332660in}{2.503782in}}%
\pgfpathcurveto{\pgfqpoint{2.326836in}{2.497959in}}{\pgfqpoint{2.323564in}{2.490058in}}{\pgfqpoint{2.323564in}{2.481822in}}%
\pgfpathcurveto{\pgfqpoint{2.323564in}{2.473586in}}{\pgfqpoint{2.326836in}{2.465686in}}{\pgfqpoint{2.332660in}{2.459862in}}%
\pgfpathcurveto{\pgfqpoint{2.338484in}{2.454038in}}{\pgfqpoint{2.346384in}{2.450766in}}{\pgfqpoint{2.354620in}{2.450766in}}%
\pgfpathclose%
\pgfusepath{stroke,fill}%
\end{pgfscope}%
\begin{pgfscope}%
\pgfpathrectangle{\pgfqpoint{0.100000in}{0.212622in}}{\pgfqpoint{3.696000in}{3.696000in}}%
\pgfusepath{clip}%
\pgfsetbuttcap%
\pgfsetroundjoin%
\definecolor{currentfill}{rgb}{0.121569,0.466667,0.705882}%
\pgfsetfillcolor{currentfill}%
\pgfsetfillopacity{0.847186}%
\pgfsetlinewidth{1.003750pt}%
\definecolor{currentstroke}{rgb}{0.121569,0.466667,0.705882}%
\pgfsetstrokecolor{currentstroke}%
\pgfsetstrokeopacity{0.847186}%
\pgfsetdash{}{0pt}%
\pgfpathmoveto{\pgfqpoint{1.830809in}{2.259807in}}%
\pgfpathcurveto{\pgfqpoint{1.839046in}{2.259807in}}{\pgfqpoint{1.846946in}{2.263079in}}{\pgfqpoint{1.852770in}{2.268903in}}%
\pgfpathcurveto{\pgfqpoint{1.858594in}{2.274727in}}{\pgfqpoint{1.861866in}{2.282627in}}{\pgfqpoint{1.861866in}{2.290863in}}%
\pgfpathcurveto{\pgfqpoint{1.861866in}{2.299100in}}{\pgfqpoint{1.858594in}{2.307000in}}{\pgfqpoint{1.852770in}{2.312824in}}%
\pgfpathcurveto{\pgfqpoint{1.846946in}{2.318648in}}{\pgfqpoint{1.839046in}{2.321920in}}{\pgfqpoint{1.830809in}{2.321920in}}%
\pgfpathcurveto{\pgfqpoint{1.822573in}{2.321920in}}{\pgfqpoint{1.814673in}{2.318648in}}{\pgfqpoint{1.808849in}{2.312824in}}%
\pgfpathcurveto{\pgfqpoint{1.803025in}{2.307000in}}{\pgfqpoint{1.799753in}{2.299100in}}{\pgfqpoint{1.799753in}{2.290863in}}%
\pgfpathcurveto{\pgfqpoint{1.799753in}{2.282627in}}{\pgfqpoint{1.803025in}{2.274727in}}{\pgfqpoint{1.808849in}{2.268903in}}%
\pgfpathcurveto{\pgfqpoint{1.814673in}{2.263079in}}{\pgfqpoint{1.822573in}{2.259807in}}{\pgfqpoint{1.830809in}{2.259807in}}%
\pgfpathclose%
\pgfusepath{stroke,fill}%
\end{pgfscope}%
\begin{pgfscope}%
\pgfpathrectangle{\pgfqpoint{0.100000in}{0.212622in}}{\pgfqpoint{3.696000in}{3.696000in}}%
\pgfusepath{clip}%
\pgfsetbuttcap%
\pgfsetroundjoin%
\definecolor{currentfill}{rgb}{0.121569,0.466667,0.705882}%
\pgfsetfillcolor{currentfill}%
\pgfsetfillopacity{0.847363}%
\pgfsetlinewidth{1.003750pt}%
\definecolor{currentstroke}{rgb}{0.121569,0.466667,0.705882}%
\pgfsetstrokecolor{currentstroke}%
\pgfsetstrokeopacity{0.847363}%
\pgfsetdash{}{0pt}%
\pgfpathmoveto{\pgfqpoint{2.351080in}{2.445813in}}%
\pgfpathcurveto{\pgfqpoint{2.359317in}{2.445813in}}{\pgfqpoint{2.367217in}{2.449085in}}{\pgfqpoint{2.373041in}{2.454909in}}%
\pgfpathcurveto{\pgfqpoint{2.378865in}{2.460733in}}{\pgfqpoint{2.382137in}{2.468633in}}{\pgfqpoint{2.382137in}{2.476869in}}%
\pgfpathcurveto{\pgfqpoint{2.382137in}{2.485106in}}{\pgfqpoint{2.378865in}{2.493006in}}{\pgfqpoint{2.373041in}{2.498830in}}%
\pgfpathcurveto{\pgfqpoint{2.367217in}{2.504653in}}{\pgfqpoint{2.359317in}{2.507926in}}{\pgfqpoint{2.351080in}{2.507926in}}%
\pgfpathcurveto{\pgfqpoint{2.342844in}{2.507926in}}{\pgfqpoint{2.334944in}{2.504653in}}{\pgfqpoint{2.329120in}{2.498830in}}%
\pgfpathcurveto{\pgfqpoint{2.323296in}{2.493006in}}{\pgfqpoint{2.320024in}{2.485106in}}{\pgfqpoint{2.320024in}{2.476869in}}%
\pgfpathcurveto{\pgfqpoint{2.320024in}{2.468633in}}{\pgfqpoint{2.323296in}{2.460733in}}{\pgfqpoint{2.329120in}{2.454909in}}%
\pgfpathcurveto{\pgfqpoint{2.334944in}{2.449085in}}{\pgfqpoint{2.342844in}{2.445813in}}{\pgfqpoint{2.351080in}{2.445813in}}%
\pgfpathclose%
\pgfusepath{stroke,fill}%
\end{pgfscope}%
\begin{pgfscope}%
\pgfpathrectangle{\pgfqpoint{0.100000in}{0.212622in}}{\pgfqpoint{3.696000in}{3.696000in}}%
\pgfusepath{clip}%
\pgfsetbuttcap%
\pgfsetroundjoin%
\definecolor{currentfill}{rgb}{0.121569,0.466667,0.705882}%
\pgfsetfillcolor{currentfill}%
\pgfsetfillopacity{0.847840}%
\pgfsetlinewidth{1.003750pt}%
\definecolor{currentstroke}{rgb}{0.121569,0.466667,0.705882}%
\pgfsetstrokecolor{currentstroke}%
\pgfsetstrokeopacity{0.847840}%
\pgfsetdash{}{0pt}%
\pgfpathmoveto{\pgfqpoint{1.829266in}{2.256358in}}%
\pgfpathcurveto{\pgfqpoint{1.837502in}{2.256358in}}{\pgfqpoint{1.845402in}{2.259631in}}{\pgfqpoint{1.851226in}{2.265455in}}%
\pgfpathcurveto{\pgfqpoint{1.857050in}{2.271279in}}{\pgfqpoint{1.860322in}{2.279179in}}{\pgfqpoint{1.860322in}{2.287415in}}%
\pgfpathcurveto{\pgfqpoint{1.860322in}{2.295651in}}{\pgfqpoint{1.857050in}{2.303551in}}{\pgfqpoint{1.851226in}{2.309375in}}%
\pgfpathcurveto{\pgfqpoint{1.845402in}{2.315199in}}{\pgfqpoint{1.837502in}{2.318471in}}{\pgfqpoint{1.829266in}{2.318471in}}%
\pgfpathcurveto{\pgfqpoint{1.821029in}{2.318471in}}{\pgfqpoint{1.813129in}{2.315199in}}{\pgfqpoint{1.807306in}{2.309375in}}%
\pgfpathcurveto{\pgfqpoint{1.801482in}{2.303551in}}{\pgfqpoint{1.798209in}{2.295651in}}{\pgfqpoint{1.798209in}{2.287415in}}%
\pgfpathcurveto{\pgfqpoint{1.798209in}{2.279179in}}{\pgfqpoint{1.801482in}{2.271279in}}{\pgfqpoint{1.807306in}{2.265455in}}%
\pgfpathcurveto{\pgfqpoint{1.813129in}{2.259631in}}{\pgfqpoint{1.821029in}{2.256358in}}{\pgfqpoint{1.829266in}{2.256358in}}%
\pgfpathclose%
\pgfusepath{stroke,fill}%
\end{pgfscope}%
\begin{pgfscope}%
\pgfpathrectangle{\pgfqpoint{0.100000in}{0.212622in}}{\pgfqpoint{3.696000in}{3.696000in}}%
\pgfusepath{clip}%
\pgfsetbuttcap%
\pgfsetroundjoin%
\definecolor{currentfill}{rgb}{0.121569,0.466667,0.705882}%
\pgfsetfillcolor{currentfill}%
\pgfsetfillopacity{0.848104}%
\pgfsetlinewidth{1.003750pt}%
\definecolor{currentstroke}{rgb}{0.121569,0.466667,0.705882}%
\pgfsetstrokecolor{currentstroke}%
\pgfsetstrokeopacity{0.848104}%
\pgfsetdash{}{0pt}%
\pgfpathmoveto{\pgfqpoint{2.348861in}{2.441195in}}%
\pgfpathcurveto{\pgfqpoint{2.357098in}{2.441195in}}{\pgfqpoint{2.364998in}{2.444468in}}{\pgfqpoint{2.370822in}{2.450291in}}%
\pgfpathcurveto{\pgfqpoint{2.376645in}{2.456115in}}{\pgfqpoint{2.379918in}{2.464015in}}{\pgfqpoint{2.379918in}{2.472252in}}%
\pgfpathcurveto{\pgfqpoint{2.379918in}{2.480488in}}{\pgfqpoint{2.376645in}{2.488388in}}{\pgfqpoint{2.370822in}{2.494212in}}%
\pgfpathcurveto{\pgfqpoint{2.364998in}{2.500036in}}{\pgfqpoint{2.357098in}{2.503308in}}{\pgfqpoint{2.348861in}{2.503308in}}%
\pgfpathcurveto{\pgfqpoint{2.340625in}{2.503308in}}{\pgfqpoint{2.332725in}{2.500036in}}{\pgfqpoint{2.326901in}{2.494212in}}%
\pgfpathcurveto{\pgfqpoint{2.321077in}{2.488388in}}{\pgfqpoint{2.317805in}{2.480488in}}{\pgfqpoint{2.317805in}{2.472252in}}%
\pgfpathcurveto{\pgfqpoint{2.317805in}{2.464015in}}{\pgfqpoint{2.321077in}{2.456115in}}{\pgfqpoint{2.326901in}{2.450291in}}%
\pgfpathcurveto{\pgfqpoint{2.332725in}{2.444468in}}{\pgfqpoint{2.340625in}{2.441195in}}{\pgfqpoint{2.348861in}{2.441195in}}%
\pgfpathclose%
\pgfusepath{stroke,fill}%
\end{pgfscope}%
\begin{pgfscope}%
\pgfpathrectangle{\pgfqpoint{0.100000in}{0.212622in}}{\pgfqpoint{3.696000in}{3.696000in}}%
\pgfusepath{clip}%
\pgfsetbuttcap%
\pgfsetroundjoin%
\definecolor{currentfill}{rgb}{0.121569,0.466667,0.705882}%
\pgfsetfillcolor{currentfill}%
\pgfsetfillopacity{0.848279}%
\pgfsetlinewidth{1.003750pt}%
\definecolor{currentstroke}{rgb}{0.121569,0.466667,0.705882}%
\pgfsetstrokecolor{currentstroke}%
\pgfsetstrokeopacity{0.848279}%
\pgfsetdash{}{0pt}%
\pgfpathmoveto{\pgfqpoint{1.338734in}{1.335267in}}%
\pgfpathcurveto{\pgfqpoint{1.346970in}{1.335267in}}{\pgfqpoint{1.354870in}{1.338540in}}{\pgfqpoint{1.360694in}{1.344363in}}%
\pgfpathcurveto{\pgfqpoint{1.366518in}{1.350187in}}{\pgfqpoint{1.369790in}{1.358087in}}{\pgfqpoint{1.369790in}{1.366324in}}%
\pgfpathcurveto{\pgfqpoint{1.369790in}{1.374560in}}{\pgfqpoint{1.366518in}{1.382460in}}{\pgfqpoint{1.360694in}{1.388284in}}%
\pgfpathcurveto{\pgfqpoint{1.354870in}{1.394108in}}{\pgfqpoint{1.346970in}{1.397380in}}{\pgfqpoint{1.338734in}{1.397380in}}%
\pgfpathcurveto{\pgfqpoint{1.330497in}{1.397380in}}{\pgfqpoint{1.322597in}{1.394108in}}{\pgfqpoint{1.316773in}{1.388284in}}%
\pgfpathcurveto{\pgfqpoint{1.310949in}{1.382460in}}{\pgfqpoint{1.307677in}{1.374560in}}{\pgfqpoint{1.307677in}{1.366324in}}%
\pgfpathcurveto{\pgfqpoint{1.307677in}{1.358087in}}{\pgfqpoint{1.310949in}{1.350187in}}{\pgfqpoint{1.316773in}{1.344363in}}%
\pgfpathcurveto{\pgfqpoint{1.322597in}{1.338540in}}{\pgfqpoint{1.330497in}{1.335267in}}{\pgfqpoint{1.338734in}{1.335267in}}%
\pgfpathclose%
\pgfusepath{stroke,fill}%
\end{pgfscope}%
\begin{pgfscope}%
\pgfpathrectangle{\pgfqpoint{0.100000in}{0.212622in}}{\pgfqpoint{3.696000in}{3.696000in}}%
\pgfusepath{clip}%
\pgfsetbuttcap%
\pgfsetroundjoin%
\definecolor{currentfill}{rgb}{0.121569,0.466667,0.705882}%
\pgfsetfillcolor{currentfill}%
\pgfsetfillopacity{0.848397}%
\pgfsetlinewidth{1.003750pt}%
\definecolor{currentstroke}{rgb}{0.121569,0.466667,0.705882}%
\pgfsetstrokecolor{currentstroke}%
\pgfsetstrokeopacity{0.848397}%
\pgfsetdash{}{0pt}%
\pgfpathmoveto{\pgfqpoint{1.826790in}{2.252796in}}%
\pgfpathcurveto{\pgfqpoint{1.835027in}{2.252796in}}{\pgfqpoint{1.842927in}{2.256068in}}{\pgfqpoint{1.848751in}{2.261892in}}%
\pgfpathcurveto{\pgfqpoint{1.854575in}{2.267716in}}{\pgfqpoint{1.857847in}{2.275616in}}{\pgfqpoint{1.857847in}{2.283853in}}%
\pgfpathcurveto{\pgfqpoint{1.857847in}{2.292089in}}{\pgfqpoint{1.854575in}{2.299989in}}{\pgfqpoint{1.848751in}{2.305813in}}%
\pgfpathcurveto{\pgfqpoint{1.842927in}{2.311637in}}{\pgfqpoint{1.835027in}{2.314909in}}{\pgfqpoint{1.826790in}{2.314909in}}%
\pgfpathcurveto{\pgfqpoint{1.818554in}{2.314909in}}{\pgfqpoint{1.810654in}{2.311637in}}{\pgfqpoint{1.804830in}{2.305813in}}%
\pgfpathcurveto{\pgfqpoint{1.799006in}{2.299989in}}{\pgfqpoint{1.795734in}{2.292089in}}{\pgfqpoint{1.795734in}{2.283853in}}%
\pgfpathcurveto{\pgfqpoint{1.795734in}{2.275616in}}{\pgfqpoint{1.799006in}{2.267716in}}{\pgfqpoint{1.804830in}{2.261892in}}%
\pgfpathcurveto{\pgfqpoint{1.810654in}{2.256068in}}{\pgfqpoint{1.818554in}{2.252796in}}{\pgfqpoint{1.826790in}{2.252796in}}%
\pgfpathclose%
\pgfusepath{stroke,fill}%
\end{pgfscope}%
\begin{pgfscope}%
\pgfpathrectangle{\pgfqpoint{0.100000in}{0.212622in}}{\pgfqpoint{3.696000in}{3.696000in}}%
\pgfusepath{clip}%
\pgfsetbuttcap%
\pgfsetroundjoin%
\definecolor{currentfill}{rgb}{0.121569,0.466667,0.705882}%
\pgfsetfillcolor{currentfill}%
\pgfsetfillopacity{0.849154}%
\pgfsetlinewidth{1.003750pt}%
\definecolor{currentstroke}{rgb}{0.121569,0.466667,0.705882}%
\pgfsetstrokecolor{currentstroke}%
\pgfsetstrokeopacity{0.849154}%
\pgfsetdash{}{0pt}%
\pgfpathmoveto{\pgfqpoint{1.824517in}{2.248205in}}%
\pgfpathcurveto{\pgfqpoint{1.832754in}{2.248205in}}{\pgfqpoint{1.840654in}{2.251477in}}{\pgfqpoint{1.846478in}{2.257301in}}%
\pgfpathcurveto{\pgfqpoint{1.852302in}{2.263125in}}{\pgfqpoint{1.855574in}{2.271025in}}{\pgfqpoint{1.855574in}{2.279261in}}%
\pgfpathcurveto{\pgfqpoint{1.855574in}{2.287498in}}{\pgfqpoint{1.852302in}{2.295398in}}{\pgfqpoint{1.846478in}{2.301222in}}%
\pgfpathcurveto{\pgfqpoint{1.840654in}{2.307046in}}{\pgfqpoint{1.832754in}{2.310318in}}{\pgfqpoint{1.824517in}{2.310318in}}%
\pgfpathcurveto{\pgfqpoint{1.816281in}{2.310318in}}{\pgfqpoint{1.808381in}{2.307046in}}{\pgfqpoint{1.802557in}{2.301222in}}%
\pgfpathcurveto{\pgfqpoint{1.796733in}{2.295398in}}{\pgfqpoint{1.793461in}{2.287498in}}{\pgfqpoint{1.793461in}{2.279261in}}%
\pgfpathcurveto{\pgfqpoint{1.793461in}{2.271025in}}{\pgfqpoint{1.796733in}{2.263125in}}{\pgfqpoint{1.802557in}{2.257301in}}%
\pgfpathcurveto{\pgfqpoint{1.808381in}{2.251477in}}{\pgfqpoint{1.816281in}{2.248205in}}{\pgfqpoint{1.824517in}{2.248205in}}%
\pgfpathclose%
\pgfusepath{stroke,fill}%
\end{pgfscope}%
\begin{pgfscope}%
\pgfpathrectangle{\pgfqpoint{0.100000in}{0.212622in}}{\pgfqpoint{3.696000in}{3.696000in}}%
\pgfusepath{clip}%
\pgfsetbuttcap%
\pgfsetroundjoin%
\definecolor{currentfill}{rgb}{0.121569,0.466667,0.705882}%
\pgfsetfillcolor{currentfill}%
\pgfsetfillopacity{0.849357}%
\pgfsetlinewidth{1.003750pt}%
\definecolor{currentstroke}{rgb}{0.121569,0.466667,0.705882}%
\pgfsetstrokecolor{currentstroke}%
\pgfsetstrokeopacity{0.849357}%
\pgfsetdash{}{0pt}%
\pgfpathmoveto{\pgfqpoint{2.347098in}{2.431238in}}%
\pgfpathcurveto{\pgfqpoint{2.355334in}{2.431238in}}{\pgfqpoint{2.363234in}{2.434510in}}{\pgfqpoint{2.369058in}{2.440334in}}%
\pgfpathcurveto{\pgfqpoint{2.374882in}{2.446158in}}{\pgfqpoint{2.378154in}{2.454058in}}{\pgfqpoint{2.378154in}{2.462294in}}%
\pgfpathcurveto{\pgfqpoint{2.378154in}{2.470531in}}{\pgfqpoint{2.374882in}{2.478431in}}{\pgfqpoint{2.369058in}{2.484255in}}%
\pgfpathcurveto{\pgfqpoint{2.363234in}{2.490079in}}{\pgfqpoint{2.355334in}{2.493351in}}{\pgfqpoint{2.347098in}{2.493351in}}%
\pgfpathcurveto{\pgfqpoint{2.338861in}{2.493351in}}{\pgfqpoint{2.330961in}{2.490079in}}{\pgfqpoint{2.325137in}{2.484255in}}%
\pgfpathcurveto{\pgfqpoint{2.319314in}{2.478431in}}{\pgfqpoint{2.316041in}{2.470531in}}{\pgfqpoint{2.316041in}{2.462294in}}%
\pgfpathcurveto{\pgfqpoint{2.316041in}{2.454058in}}{\pgfqpoint{2.319314in}{2.446158in}}{\pgfqpoint{2.325137in}{2.440334in}}%
\pgfpathcurveto{\pgfqpoint{2.330961in}{2.434510in}}{\pgfqpoint{2.338861in}{2.431238in}}{\pgfqpoint{2.347098in}{2.431238in}}%
\pgfpathclose%
\pgfusepath{stroke,fill}%
\end{pgfscope}%
\begin{pgfscope}%
\pgfpathrectangle{\pgfqpoint{0.100000in}{0.212622in}}{\pgfqpoint{3.696000in}{3.696000in}}%
\pgfusepath{clip}%
\pgfsetbuttcap%
\pgfsetroundjoin%
\definecolor{currentfill}{rgb}{0.121569,0.466667,0.705882}%
\pgfsetfillcolor{currentfill}%
\pgfsetfillopacity{0.849548}%
\pgfsetlinewidth{1.003750pt}%
\definecolor{currentstroke}{rgb}{0.121569,0.466667,0.705882}%
\pgfsetstrokecolor{currentstroke}%
\pgfsetstrokeopacity{0.849548}%
\pgfsetdash{}{0pt}%
\pgfpathmoveto{\pgfqpoint{1.823262in}{2.245585in}}%
\pgfpathcurveto{\pgfqpoint{1.831498in}{2.245585in}}{\pgfqpoint{1.839398in}{2.248857in}}{\pgfqpoint{1.845222in}{2.254681in}}%
\pgfpathcurveto{\pgfqpoint{1.851046in}{2.260505in}}{\pgfqpoint{1.854318in}{2.268405in}}{\pgfqpoint{1.854318in}{2.276641in}}%
\pgfpathcurveto{\pgfqpoint{1.854318in}{2.284877in}}{\pgfqpoint{1.851046in}{2.292777in}}{\pgfqpoint{1.845222in}{2.298601in}}%
\pgfpathcurveto{\pgfqpoint{1.839398in}{2.304425in}}{\pgfqpoint{1.831498in}{2.307698in}}{\pgfqpoint{1.823262in}{2.307698in}}%
\pgfpathcurveto{\pgfqpoint{1.815025in}{2.307698in}}{\pgfqpoint{1.807125in}{2.304425in}}{\pgfqpoint{1.801301in}{2.298601in}}%
\pgfpathcurveto{\pgfqpoint{1.795477in}{2.292777in}}{\pgfqpoint{1.792205in}{2.284877in}}{\pgfqpoint{1.792205in}{2.276641in}}%
\pgfpathcurveto{\pgfqpoint{1.792205in}{2.268405in}}{\pgfqpoint{1.795477in}{2.260505in}}{\pgfqpoint{1.801301in}{2.254681in}}%
\pgfpathcurveto{\pgfqpoint{1.807125in}{2.248857in}}{\pgfqpoint{1.815025in}{2.245585in}}{\pgfqpoint{1.823262in}{2.245585in}}%
\pgfpathclose%
\pgfusepath{stroke,fill}%
\end{pgfscope}%
\begin{pgfscope}%
\pgfpathrectangle{\pgfqpoint{0.100000in}{0.212622in}}{\pgfqpoint{3.696000in}{3.696000in}}%
\pgfusepath{clip}%
\pgfsetbuttcap%
\pgfsetroundjoin%
\definecolor{currentfill}{rgb}{0.121569,0.466667,0.705882}%
\pgfsetfillcolor{currentfill}%
\pgfsetfillopacity{0.849701}%
\pgfsetlinewidth{1.003750pt}%
\definecolor{currentstroke}{rgb}{0.121569,0.466667,0.705882}%
\pgfsetstrokecolor{currentstroke}%
\pgfsetstrokeopacity{0.849701}%
\pgfsetdash{}{0pt}%
\pgfpathmoveto{\pgfqpoint{1.822385in}{2.244162in}}%
\pgfpathcurveto{\pgfqpoint{1.830621in}{2.244162in}}{\pgfqpoint{1.838521in}{2.247434in}}{\pgfqpoint{1.844345in}{2.253258in}}%
\pgfpathcurveto{\pgfqpoint{1.850169in}{2.259082in}}{\pgfqpoint{1.853441in}{2.266982in}}{\pgfqpoint{1.853441in}{2.275218in}}%
\pgfpathcurveto{\pgfqpoint{1.853441in}{2.283454in}}{\pgfqpoint{1.850169in}{2.291355in}}{\pgfqpoint{1.844345in}{2.297178in}}%
\pgfpathcurveto{\pgfqpoint{1.838521in}{2.303002in}}{\pgfqpoint{1.830621in}{2.306275in}}{\pgfqpoint{1.822385in}{2.306275in}}%
\pgfpathcurveto{\pgfqpoint{1.814149in}{2.306275in}}{\pgfqpoint{1.806248in}{2.303002in}}{\pgfqpoint{1.800425in}{2.297178in}}%
\pgfpathcurveto{\pgfqpoint{1.794601in}{2.291355in}}{\pgfqpoint{1.791328in}{2.283454in}}{\pgfqpoint{1.791328in}{2.275218in}}%
\pgfpathcurveto{\pgfqpoint{1.791328in}{2.266982in}}{\pgfqpoint{1.794601in}{2.259082in}}{\pgfqpoint{1.800425in}{2.253258in}}%
\pgfpathcurveto{\pgfqpoint{1.806248in}{2.247434in}}{\pgfqpoint{1.814149in}{2.244162in}}{\pgfqpoint{1.822385in}{2.244162in}}%
\pgfpathclose%
\pgfusepath{stroke,fill}%
\end{pgfscope}%
\begin{pgfscope}%
\pgfpathrectangle{\pgfqpoint{0.100000in}{0.212622in}}{\pgfqpoint{3.696000in}{3.696000in}}%
\pgfusepath{clip}%
\pgfsetbuttcap%
\pgfsetroundjoin%
\definecolor{currentfill}{rgb}{0.121569,0.466667,0.705882}%
\pgfsetfillcolor{currentfill}%
\pgfsetfillopacity{0.849924}%
\pgfsetlinewidth{1.003750pt}%
\definecolor{currentstroke}{rgb}{0.121569,0.466667,0.705882}%
\pgfsetstrokecolor{currentstroke}%
\pgfsetstrokeopacity{0.849924}%
\pgfsetdash{}{0pt}%
\pgfpathmoveto{\pgfqpoint{1.347926in}{1.326640in}}%
\pgfpathcurveto{\pgfqpoint{1.356162in}{1.326640in}}{\pgfqpoint{1.364062in}{1.329912in}}{\pgfqpoint{1.369886in}{1.335736in}}%
\pgfpathcurveto{\pgfqpoint{1.375710in}{1.341560in}}{\pgfqpoint{1.378982in}{1.349460in}}{\pgfqpoint{1.378982in}{1.357696in}}%
\pgfpathcurveto{\pgfqpoint{1.378982in}{1.365932in}}{\pgfqpoint{1.375710in}{1.373833in}}{\pgfqpoint{1.369886in}{1.379656in}}%
\pgfpathcurveto{\pgfqpoint{1.364062in}{1.385480in}}{\pgfqpoint{1.356162in}{1.388753in}}{\pgfqpoint{1.347926in}{1.388753in}}%
\pgfpathcurveto{\pgfqpoint{1.339690in}{1.388753in}}{\pgfqpoint{1.331789in}{1.385480in}}{\pgfqpoint{1.325966in}{1.379656in}}%
\pgfpathcurveto{\pgfqpoint{1.320142in}{1.373833in}}{\pgfqpoint{1.316869in}{1.365932in}}{\pgfqpoint{1.316869in}{1.357696in}}%
\pgfpathcurveto{\pgfqpoint{1.316869in}{1.349460in}}{\pgfqpoint{1.320142in}{1.341560in}}{\pgfqpoint{1.325966in}{1.335736in}}%
\pgfpathcurveto{\pgfqpoint{1.331789in}{1.329912in}}{\pgfqpoint{1.339690in}{1.326640in}}{\pgfqpoint{1.347926in}{1.326640in}}%
\pgfpathclose%
\pgfusepath{stroke,fill}%
\end{pgfscope}%
\begin{pgfscope}%
\pgfpathrectangle{\pgfqpoint{0.100000in}{0.212622in}}{\pgfqpoint{3.696000in}{3.696000in}}%
\pgfusepath{clip}%
\pgfsetbuttcap%
\pgfsetroundjoin%
\definecolor{currentfill}{rgb}{0.121569,0.466667,0.705882}%
\pgfsetfillcolor{currentfill}%
\pgfsetfillopacity{0.850208}%
\pgfsetlinewidth{1.003750pt}%
\definecolor{currentstroke}{rgb}{0.121569,0.466667,0.705882}%
\pgfsetstrokecolor{currentstroke}%
\pgfsetstrokeopacity{0.850208}%
\pgfsetdash{}{0pt}%
\pgfpathmoveto{\pgfqpoint{1.821068in}{2.241286in}}%
\pgfpathcurveto{\pgfqpoint{1.829304in}{2.241286in}}{\pgfqpoint{1.837204in}{2.244558in}}{\pgfqpoint{1.843028in}{2.250382in}}%
\pgfpathcurveto{\pgfqpoint{1.848852in}{2.256206in}}{\pgfqpoint{1.852124in}{2.264106in}}{\pgfqpoint{1.852124in}{2.272342in}}%
\pgfpathcurveto{\pgfqpoint{1.852124in}{2.280578in}}{\pgfqpoint{1.848852in}{2.288478in}}{\pgfqpoint{1.843028in}{2.294302in}}%
\pgfpathcurveto{\pgfqpoint{1.837204in}{2.300126in}}{\pgfqpoint{1.829304in}{2.303399in}}{\pgfqpoint{1.821068in}{2.303399in}}%
\pgfpathcurveto{\pgfqpoint{1.812832in}{2.303399in}}{\pgfqpoint{1.804931in}{2.300126in}}{\pgfqpoint{1.799108in}{2.294302in}}%
\pgfpathcurveto{\pgfqpoint{1.793284in}{2.288478in}}{\pgfqpoint{1.790011in}{2.280578in}}{\pgfqpoint{1.790011in}{2.272342in}}%
\pgfpathcurveto{\pgfqpoint{1.790011in}{2.264106in}}{\pgfqpoint{1.793284in}{2.256206in}}{\pgfqpoint{1.799108in}{2.250382in}}%
\pgfpathcurveto{\pgfqpoint{1.804931in}{2.244558in}}{\pgfqpoint{1.812832in}{2.241286in}}{\pgfqpoint{1.821068in}{2.241286in}}%
\pgfpathclose%
\pgfusepath{stroke,fill}%
\end{pgfscope}%
\begin{pgfscope}%
\pgfpathrectangle{\pgfqpoint{0.100000in}{0.212622in}}{\pgfqpoint{3.696000in}{3.696000in}}%
\pgfusepath{clip}%
\pgfsetbuttcap%
\pgfsetroundjoin%
\definecolor{currentfill}{rgb}{0.121569,0.466667,0.705882}%
\pgfsetfillcolor{currentfill}%
\pgfsetfillopacity{0.850459}%
\pgfsetlinewidth{1.003750pt}%
\definecolor{currentstroke}{rgb}{0.121569,0.466667,0.705882}%
\pgfsetstrokecolor{currentstroke}%
\pgfsetstrokeopacity{0.850459}%
\pgfsetdash{}{0pt}%
\pgfpathmoveto{\pgfqpoint{1.820384in}{2.239546in}}%
\pgfpathcurveto{\pgfqpoint{1.828620in}{2.239546in}}{\pgfqpoint{1.836520in}{2.242818in}}{\pgfqpoint{1.842344in}{2.248642in}}%
\pgfpathcurveto{\pgfqpoint{1.848168in}{2.254466in}}{\pgfqpoint{1.851440in}{2.262366in}}{\pgfqpoint{1.851440in}{2.270602in}}%
\pgfpathcurveto{\pgfqpoint{1.851440in}{2.278839in}}{\pgfqpoint{1.848168in}{2.286739in}}{\pgfqpoint{1.842344in}{2.292563in}}%
\pgfpathcurveto{\pgfqpoint{1.836520in}{2.298387in}}{\pgfqpoint{1.828620in}{2.301659in}}{\pgfqpoint{1.820384in}{2.301659in}}%
\pgfpathcurveto{\pgfqpoint{1.812147in}{2.301659in}}{\pgfqpoint{1.804247in}{2.298387in}}{\pgfqpoint{1.798423in}{2.292563in}}%
\pgfpathcurveto{\pgfqpoint{1.792600in}{2.286739in}}{\pgfqpoint{1.789327in}{2.278839in}}{\pgfqpoint{1.789327in}{2.270602in}}%
\pgfpathcurveto{\pgfqpoint{1.789327in}{2.262366in}}{\pgfqpoint{1.792600in}{2.254466in}}{\pgfqpoint{1.798423in}{2.248642in}}%
\pgfpathcurveto{\pgfqpoint{1.804247in}{2.242818in}}{\pgfqpoint{1.812147in}{2.239546in}}{\pgfqpoint{1.820384in}{2.239546in}}%
\pgfpathclose%
\pgfusepath{stroke,fill}%
\end{pgfscope}%
\begin{pgfscope}%
\pgfpathrectangle{\pgfqpoint{0.100000in}{0.212622in}}{\pgfqpoint{3.696000in}{3.696000in}}%
\pgfusepath{clip}%
\pgfsetbuttcap%
\pgfsetroundjoin%
\definecolor{currentfill}{rgb}{0.121569,0.466667,0.705882}%
\pgfsetfillcolor{currentfill}%
\pgfsetfillopacity{0.850517}%
\pgfsetlinewidth{1.003750pt}%
\definecolor{currentstroke}{rgb}{0.121569,0.466667,0.705882}%
\pgfsetstrokecolor{currentstroke}%
\pgfsetstrokeopacity{0.850517}%
\pgfsetdash{}{0pt}%
\pgfpathmoveto{\pgfqpoint{2.343344in}{2.425182in}}%
\pgfpathcurveto{\pgfqpoint{2.351581in}{2.425182in}}{\pgfqpoint{2.359481in}{2.428454in}}{\pgfqpoint{2.365305in}{2.434278in}}%
\pgfpathcurveto{\pgfqpoint{2.371128in}{2.440102in}}{\pgfqpoint{2.374401in}{2.448002in}}{\pgfqpoint{2.374401in}{2.456238in}}%
\pgfpathcurveto{\pgfqpoint{2.374401in}{2.464475in}}{\pgfqpoint{2.371128in}{2.472375in}}{\pgfqpoint{2.365305in}{2.478199in}}%
\pgfpathcurveto{\pgfqpoint{2.359481in}{2.484023in}}{\pgfqpoint{2.351581in}{2.487295in}}{\pgfqpoint{2.343344in}{2.487295in}}%
\pgfpathcurveto{\pgfqpoint{2.335108in}{2.487295in}}{\pgfqpoint{2.327208in}{2.484023in}}{\pgfqpoint{2.321384in}{2.478199in}}%
\pgfpathcurveto{\pgfqpoint{2.315560in}{2.472375in}}{\pgfqpoint{2.312288in}{2.464475in}}{\pgfqpoint{2.312288in}{2.456238in}}%
\pgfpathcurveto{\pgfqpoint{2.312288in}{2.448002in}}{\pgfqpoint{2.315560in}{2.440102in}}{\pgfqpoint{2.321384in}{2.434278in}}%
\pgfpathcurveto{\pgfqpoint{2.327208in}{2.428454in}}{\pgfqpoint{2.335108in}{2.425182in}}{\pgfqpoint{2.343344in}{2.425182in}}%
\pgfpathclose%
\pgfusepath{stroke,fill}%
\end{pgfscope}%
\begin{pgfscope}%
\pgfpathrectangle{\pgfqpoint{0.100000in}{0.212622in}}{\pgfqpoint{3.696000in}{3.696000in}}%
\pgfusepath{clip}%
\pgfsetbuttcap%
\pgfsetroundjoin%
\definecolor{currentfill}{rgb}{0.121569,0.466667,0.705882}%
\pgfsetfillcolor{currentfill}%
\pgfsetfillopacity{0.850574}%
\pgfsetlinewidth{1.003750pt}%
\definecolor{currentstroke}{rgb}{0.121569,0.466667,0.705882}%
\pgfsetstrokecolor{currentstroke}%
\pgfsetstrokeopacity{0.850574}%
\pgfsetdash{}{0pt}%
\pgfpathmoveto{\pgfqpoint{1.819848in}{2.238695in}}%
\pgfpathcurveto{\pgfqpoint{1.828084in}{2.238695in}}{\pgfqpoint{1.835984in}{2.241967in}}{\pgfqpoint{1.841808in}{2.247791in}}%
\pgfpathcurveto{\pgfqpoint{1.847632in}{2.253615in}}{\pgfqpoint{1.850904in}{2.261515in}}{\pgfqpoint{1.850904in}{2.269751in}}%
\pgfpathcurveto{\pgfqpoint{1.850904in}{2.277988in}}{\pgfqpoint{1.847632in}{2.285888in}}{\pgfqpoint{1.841808in}{2.291712in}}%
\pgfpathcurveto{\pgfqpoint{1.835984in}{2.297536in}}{\pgfqpoint{1.828084in}{2.300808in}}{\pgfqpoint{1.819848in}{2.300808in}}%
\pgfpathcurveto{\pgfqpoint{1.811612in}{2.300808in}}{\pgfqpoint{1.803712in}{2.297536in}}{\pgfqpoint{1.797888in}{2.291712in}}%
\pgfpathcurveto{\pgfqpoint{1.792064in}{2.285888in}}{\pgfqpoint{1.788791in}{2.277988in}}{\pgfqpoint{1.788791in}{2.269751in}}%
\pgfpathcurveto{\pgfqpoint{1.788791in}{2.261515in}}{\pgfqpoint{1.792064in}{2.253615in}}{\pgfqpoint{1.797888in}{2.247791in}}%
\pgfpathcurveto{\pgfqpoint{1.803712in}{2.241967in}}{\pgfqpoint{1.811612in}{2.238695in}}{\pgfqpoint{1.819848in}{2.238695in}}%
\pgfpathclose%
\pgfusepath{stroke,fill}%
\end{pgfscope}%
\begin{pgfscope}%
\pgfpathrectangle{\pgfqpoint{0.100000in}{0.212622in}}{\pgfqpoint{3.696000in}{3.696000in}}%
\pgfusepath{clip}%
\pgfsetbuttcap%
\pgfsetroundjoin%
\definecolor{currentfill}{rgb}{0.121569,0.466667,0.705882}%
\pgfsetfillcolor{currentfill}%
\pgfsetfillopacity{0.850978}%
\pgfsetlinewidth{1.003750pt}%
\definecolor{currentstroke}{rgb}{0.121569,0.466667,0.705882}%
\pgfsetstrokecolor{currentstroke}%
\pgfsetstrokeopacity{0.850978}%
\pgfsetdash{}{0pt}%
\pgfpathmoveto{\pgfqpoint{1.818889in}{2.236587in}}%
\pgfpathcurveto{\pgfqpoint{1.827126in}{2.236587in}}{\pgfqpoint{1.835026in}{2.239860in}}{\pgfqpoint{1.840850in}{2.245683in}}%
\pgfpathcurveto{\pgfqpoint{1.846673in}{2.251507in}}{\pgfqpoint{1.849946in}{2.259407in}}{\pgfqpoint{1.849946in}{2.267644in}}%
\pgfpathcurveto{\pgfqpoint{1.849946in}{2.275880in}}{\pgfqpoint{1.846673in}{2.283780in}}{\pgfqpoint{1.840850in}{2.289604in}}%
\pgfpathcurveto{\pgfqpoint{1.835026in}{2.295428in}}{\pgfqpoint{1.827126in}{2.298700in}}{\pgfqpoint{1.818889in}{2.298700in}}%
\pgfpathcurveto{\pgfqpoint{1.810653in}{2.298700in}}{\pgfqpoint{1.802753in}{2.295428in}}{\pgfqpoint{1.796929in}{2.289604in}}%
\pgfpathcurveto{\pgfqpoint{1.791105in}{2.283780in}}{\pgfqpoint{1.787833in}{2.275880in}}{\pgfqpoint{1.787833in}{2.267644in}}%
\pgfpathcurveto{\pgfqpoint{1.787833in}{2.259407in}}{\pgfqpoint{1.791105in}{2.251507in}}{\pgfqpoint{1.796929in}{2.245683in}}%
\pgfpathcurveto{\pgfqpoint{1.802753in}{2.239860in}}{\pgfqpoint{1.810653in}{2.236587in}}{\pgfqpoint{1.818889in}{2.236587in}}%
\pgfpathclose%
\pgfusepath{stroke,fill}%
\end{pgfscope}%
\begin{pgfscope}%
\pgfpathrectangle{\pgfqpoint{0.100000in}{0.212622in}}{\pgfqpoint{3.696000in}{3.696000in}}%
\pgfusepath{clip}%
\pgfsetbuttcap%
\pgfsetroundjoin%
\definecolor{currentfill}{rgb}{0.121569,0.466667,0.705882}%
\pgfsetfillcolor{currentfill}%
\pgfsetfillopacity{0.851178}%
\pgfsetlinewidth{1.003750pt}%
\definecolor{currentstroke}{rgb}{0.121569,0.466667,0.705882}%
\pgfsetstrokecolor{currentstroke}%
\pgfsetstrokeopacity{0.851178}%
\pgfsetdash{}{0pt}%
\pgfpathmoveto{\pgfqpoint{1.818484in}{2.235255in}}%
\pgfpathcurveto{\pgfqpoint{1.826721in}{2.235255in}}{\pgfqpoint{1.834621in}{2.238528in}}{\pgfqpoint{1.840445in}{2.244352in}}%
\pgfpathcurveto{\pgfqpoint{1.846268in}{2.250175in}}{\pgfqpoint{1.849541in}{2.258076in}}{\pgfqpoint{1.849541in}{2.266312in}}%
\pgfpathcurveto{\pgfqpoint{1.849541in}{2.274548in}}{\pgfqpoint{1.846268in}{2.282448in}}{\pgfqpoint{1.840445in}{2.288272in}}%
\pgfpathcurveto{\pgfqpoint{1.834621in}{2.294096in}}{\pgfqpoint{1.826721in}{2.297368in}}{\pgfqpoint{1.818484in}{2.297368in}}%
\pgfpathcurveto{\pgfqpoint{1.810248in}{2.297368in}}{\pgfqpoint{1.802348in}{2.294096in}}{\pgfqpoint{1.796524in}{2.288272in}}%
\pgfpathcurveto{\pgfqpoint{1.790700in}{2.282448in}}{\pgfqpoint{1.787428in}{2.274548in}}{\pgfqpoint{1.787428in}{2.266312in}}%
\pgfpathcurveto{\pgfqpoint{1.787428in}{2.258076in}}{\pgfqpoint{1.790700in}{2.250175in}}{\pgfqpoint{1.796524in}{2.244352in}}%
\pgfpathcurveto{\pgfqpoint{1.802348in}{2.238528in}}{\pgfqpoint{1.810248in}{2.235255in}}{\pgfqpoint{1.818484in}{2.235255in}}%
\pgfpathclose%
\pgfusepath{stroke,fill}%
\end{pgfscope}%
\begin{pgfscope}%
\pgfpathrectangle{\pgfqpoint{0.100000in}{0.212622in}}{\pgfqpoint{3.696000in}{3.696000in}}%
\pgfusepath{clip}%
\pgfsetbuttcap%
\pgfsetroundjoin%
\definecolor{currentfill}{rgb}{0.121569,0.466667,0.705882}%
\pgfsetfillcolor{currentfill}%
\pgfsetfillopacity{0.851276}%
\pgfsetlinewidth{1.003750pt}%
\definecolor{currentstroke}{rgb}{0.121569,0.466667,0.705882}%
\pgfsetstrokecolor{currentstroke}%
\pgfsetstrokeopacity{0.851276}%
\pgfsetdash{}{0pt}%
\pgfpathmoveto{\pgfqpoint{2.339132in}{2.419706in}}%
\pgfpathcurveto{\pgfqpoint{2.347368in}{2.419706in}}{\pgfqpoint{2.355268in}{2.422978in}}{\pgfqpoint{2.361092in}{2.428802in}}%
\pgfpathcurveto{\pgfqpoint{2.366916in}{2.434626in}}{\pgfqpoint{2.370189in}{2.442526in}}{\pgfqpoint{2.370189in}{2.450762in}}%
\pgfpathcurveto{\pgfqpoint{2.370189in}{2.458999in}}{\pgfqpoint{2.366916in}{2.466899in}}{\pgfqpoint{2.361092in}{2.472723in}}%
\pgfpathcurveto{\pgfqpoint{2.355268in}{2.478546in}}{\pgfqpoint{2.347368in}{2.481819in}}{\pgfqpoint{2.339132in}{2.481819in}}%
\pgfpathcurveto{\pgfqpoint{2.330896in}{2.481819in}}{\pgfqpoint{2.322996in}{2.478546in}}{\pgfqpoint{2.317172in}{2.472723in}}%
\pgfpathcurveto{\pgfqpoint{2.311348in}{2.466899in}}{\pgfqpoint{2.308076in}{2.458999in}}{\pgfqpoint{2.308076in}{2.450762in}}%
\pgfpathcurveto{\pgfqpoint{2.308076in}{2.442526in}}{\pgfqpoint{2.311348in}{2.434626in}}{\pgfqpoint{2.317172in}{2.428802in}}%
\pgfpathcurveto{\pgfqpoint{2.322996in}{2.422978in}}{\pgfqpoint{2.330896in}{2.419706in}}{\pgfqpoint{2.339132in}{2.419706in}}%
\pgfpathclose%
\pgfusepath{stroke,fill}%
\end{pgfscope}%
\begin{pgfscope}%
\pgfpathrectangle{\pgfqpoint{0.100000in}{0.212622in}}{\pgfqpoint{3.696000in}{3.696000in}}%
\pgfusepath{clip}%
\pgfsetbuttcap%
\pgfsetroundjoin%
\definecolor{currentfill}{rgb}{0.121569,0.466667,0.705882}%
\pgfsetfillcolor{currentfill}%
\pgfsetfillopacity{0.851489}%
\pgfsetlinewidth{1.003750pt}%
\definecolor{currentstroke}{rgb}{0.121569,0.466667,0.705882}%
\pgfsetstrokecolor{currentstroke}%
\pgfsetstrokeopacity{0.851489}%
\pgfsetdash{}{0pt}%
\pgfpathmoveto{\pgfqpoint{1.817248in}{2.233312in}}%
\pgfpathcurveto{\pgfqpoint{1.825484in}{2.233312in}}{\pgfqpoint{1.833384in}{2.236584in}}{\pgfqpoint{1.839208in}{2.242408in}}%
\pgfpathcurveto{\pgfqpoint{1.845032in}{2.248232in}}{\pgfqpoint{1.848304in}{2.256132in}}{\pgfqpoint{1.848304in}{2.264369in}}%
\pgfpathcurveto{\pgfqpoint{1.848304in}{2.272605in}}{\pgfqpoint{1.845032in}{2.280505in}}{\pgfqpoint{1.839208in}{2.286329in}}%
\pgfpathcurveto{\pgfqpoint{1.833384in}{2.292153in}}{\pgfqpoint{1.825484in}{2.295425in}}{\pgfqpoint{1.817248in}{2.295425in}}%
\pgfpathcurveto{\pgfqpoint{1.809012in}{2.295425in}}{\pgfqpoint{1.801112in}{2.292153in}}{\pgfqpoint{1.795288in}{2.286329in}}%
\pgfpathcurveto{\pgfqpoint{1.789464in}{2.280505in}}{\pgfqpoint{1.786191in}{2.272605in}}{\pgfqpoint{1.786191in}{2.264369in}}%
\pgfpathcurveto{\pgfqpoint{1.786191in}{2.256132in}}{\pgfqpoint{1.789464in}{2.248232in}}{\pgfqpoint{1.795288in}{2.242408in}}%
\pgfpathcurveto{\pgfqpoint{1.801112in}{2.236584in}}{\pgfqpoint{1.809012in}{2.233312in}}{\pgfqpoint{1.817248in}{2.233312in}}%
\pgfpathclose%
\pgfusepath{stroke,fill}%
\end{pgfscope}%
\begin{pgfscope}%
\pgfpathrectangle{\pgfqpoint{0.100000in}{0.212622in}}{\pgfqpoint{3.696000in}{3.696000in}}%
\pgfusepath{clip}%
\pgfsetbuttcap%
\pgfsetroundjoin%
\definecolor{currentfill}{rgb}{0.121569,0.466667,0.705882}%
\pgfsetfillcolor{currentfill}%
\pgfsetfillopacity{0.851604}%
\pgfsetlinewidth{1.003750pt}%
\definecolor{currentstroke}{rgb}{0.121569,0.466667,0.705882}%
\pgfsetstrokecolor{currentstroke}%
\pgfsetstrokeopacity{0.851604}%
\pgfsetdash{}{0pt}%
\pgfpathmoveto{\pgfqpoint{2.932927in}{1.645686in}}%
\pgfpathcurveto{\pgfqpoint{2.941164in}{1.645686in}}{\pgfqpoint{2.949064in}{1.648958in}}{\pgfqpoint{2.954888in}{1.654782in}}%
\pgfpathcurveto{\pgfqpoint{2.960712in}{1.660606in}}{\pgfqpoint{2.963984in}{1.668506in}}{\pgfqpoint{2.963984in}{1.676742in}}%
\pgfpathcurveto{\pgfqpoint{2.963984in}{1.684978in}}{\pgfqpoint{2.960712in}{1.692878in}}{\pgfqpoint{2.954888in}{1.698702in}}%
\pgfpathcurveto{\pgfqpoint{2.949064in}{1.704526in}}{\pgfqpoint{2.941164in}{1.707799in}}{\pgfqpoint{2.932927in}{1.707799in}}%
\pgfpathcurveto{\pgfqpoint{2.924691in}{1.707799in}}{\pgfqpoint{2.916791in}{1.704526in}}{\pgfqpoint{2.910967in}{1.698702in}}%
\pgfpathcurveto{\pgfqpoint{2.905143in}{1.692878in}}{\pgfqpoint{2.901871in}{1.684978in}}{\pgfqpoint{2.901871in}{1.676742in}}%
\pgfpathcurveto{\pgfqpoint{2.901871in}{1.668506in}}{\pgfqpoint{2.905143in}{1.660606in}}{\pgfqpoint{2.910967in}{1.654782in}}%
\pgfpathcurveto{\pgfqpoint{2.916791in}{1.648958in}}{\pgfqpoint{2.924691in}{1.645686in}}{\pgfqpoint{2.932927in}{1.645686in}}%
\pgfpathclose%
\pgfusepath{stroke,fill}%
\end{pgfscope}%
\begin{pgfscope}%
\pgfpathrectangle{\pgfqpoint{0.100000in}{0.212622in}}{\pgfqpoint{3.696000in}{3.696000in}}%
\pgfusepath{clip}%
\pgfsetbuttcap%
\pgfsetroundjoin%
\definecolor{currentfill}{rgb}{0.121569,0.466667,0.705882}%
\pgfsetfillcolor{currentfill}%
\pgfsetfillopacity{0.851618}%
\pgfsetlinewidth{1.003750pt}%
\definecolor{currentstroke}{rgb}{0.121569,0.466667,0.705882}%
\pgfsetstrokecolor{currentstroke}%
\pgfsetstrokeopacity{0.851618}%
\pgfsetdash{}{0pt}%
\pgfpathmoveto{\pgfqpoint{1.358508in}{1.320362in}}%
\pgfpathcurveto{\pgfqpoint{1.366744in}{1.320362in}}{\pgfqpoint{1.374644in}{1.323634in}}{\pgfqpoint{1.380468in}{1.329458in}}%
\pgfpathcurveto{\pgfqpoint{1.386292in}{1.335282in}}{\pgfqpoint{1.389564in}{1.343182in}}{\pgfqpoint{1.389564in}{1.351418in}}%
\pgfpathcurveto{\pgfqpoint{1.389564in}{1.359655in}}{\pgfqpoint{1.386292in}{1.367555in}}{\pgfqpoint{1.380468in}{1.373379in}}%
\pgfpathcurveto{\pgfqpoint{1.374644in}{1.379203in}}{\pgfqpoint{1.366744in}{1.382475in}}{\pgfqpoint{1.358508in}{1.382475in}}%
\pgfpathcurveto{\pgfqpoint{1.350272in}{1.382475in}}{\pgfqpoint{1.342372in}{1.379203in}}{\pgfqpoint{1.336548in}{1.373379in}}%
\pgfpathcurveto{\pgfqpoint{1.330724in}{1.367555in}}{\pgfqpoint{1.327451in}{1.359655in}}{\pgfqpoint{1.327451in}{1.351418in}}%
\pgfpathcurveto{\pgfqpoint{1.327451in}{1.343182in}}{\pgfqpoint{1.330724in}{1.335282in}}{\pgfqpoint{1.336548in}{1.329458in}}%
\pgfpathcurveto{\pgfqpoint{1.342372in}{1.323634in}}{\pgfqpoint{1.350272in}{1.320362in}}{\pgfqpoint{1.358508in}{1.320362in}}%
\pgfpathclose%
\pgfusepath{stroke,fill}%
\end{pgfscope}%
\begin{pgfscope}%
\pgfpathrectangle{\pgfqpoint{0.100000in}{0.212622in}}{\pgfqpoint{3.696000in}{3.696000in}}%
\pgfusepath{clip}%
\pgfsetbuttcap%
\pgfsetroundjoin%
\definecolor{currentfill}{rgb}{0.121569,0.466667,0.705882}%
\pgfsetfillcolor{currentfill}%
\pgfsetfillopacity{0.851997}%
\pgfsetlinewidth{1.003750pt}%
\definecolor{currentstroke}{rgb}{0.121569,0.466667,0.705882}%
\pgfsetstrokecolor{currentstroke}%
\pgfsetstrokeopacity{0.851997}%
\pgfsetdash{}{0pt}%
\pgfpathmoveto{\pgfqpoint{1.815948in}{2.230310in}}%
\pgfpathcurveto{\pgfqpoint{1.824184in}{2.230310in}}{\pgfqpoint{1.832084in}{2.233582in}}{\pgfqpoint{1.837908in}{2.239406in}}%
\pgfpathcurveto{\pgfqpoint{1.843732in}{2.245230in}}{\pgfqpoint{1.847004in}{2.253130in}}{\pgfqpoint{1.847004in}{2.261366in}}%
\pgfpathcurveto{\pgfqpoint{1.847004in}{2.269602in}}{\pgfqpoint{1.843732in}{2.277502in}}{\pgfqpoint{1.837908in}{2.283326in}}%
\pgfpathcurveto{\pgfqpoint{1.832084in}{2.289150in}}{\pgfqpoint{1.824184in}{2.292423in}}{\pgfqpoint{1.815948in}{2.292423in}}%
\pgfpathcurveto{\pgfqpoint{1.807711in}{2.292423in}}{\pgfqpoint{1.799811in}{2.289150in}}{\pgfqpoint{1.793988in}{2.283326in}}%
\pgfpathcurveto{\pgfqpoint{1.788164in}{2.277502in}}{\pgfqpoint{1.784891in}{2.269602in}}{\pgfqpoint{1.784891in}{2.261366in}}%
\pgfpathcurveto{\pgfqpoint{1.784891in}{2.253130in}}{\pgfqpoint{1.788164in}{2.245230in}}{\pgfqpoint{1.793988in}{2.239406in}}%
\pgfpathcurveto{\pgfqpoint{1.799811in}{2.233582in}}{\pgfqpoint{1.807711in}{2.230310in}}{\pgfqpoint{1.815948in}{2.230310in}}%
\pgfpathclose%
\pgfusepath{stroke,fill}%
\end{pgfscope}%
\begin{pgfscope}%
\pgfpathrectangle{\pgfqpoint{0.100000in}{0.212622in}}{\pgfqpoint{3.696000in}{3.696000in}}%
\pgfusepath{clip}%
\pgfsetbuttcap%
\pgfsetroundjoin%
\definecolor{currentfill}{rgb}{0.121569,0.466667,0.705882}%
\pgfsetfillcolor{currentfill}%
\pgfsetfillopacity{0.852108}%
\pgfsetlinewidth{1.003750pt}%
\definecolor{currentstroke}{rgb}{0.121569,0.466667,0.705882}%
\pgfsetstrokecolor{currentstroke}%
\pgfsetstrokeopacity{0.852108}%
\pgfsetdash{}{0pt}%
\pgfpathmoveto{\pgfqpoint{2.336690in}{2.414100in}}%
\pgfpathcurveto{\pgfqpoint{2.344927in}{2.414100in}}{\pgfqpoint{2.352827in}{2.417372in}}{\pgfqpoint{2.358651in}{2.423196in}}%
\pgfpathcurveto{\pgfqpoint{2.364475in}{2.429020in}}{\pgfqpoint{2.367747in}{2.436920in}}{\pgfqpoint{2.367747in}{2.445157in}}%
\pgfpathcurveto{\pgfqpoint{2.367747in}{2.453393in}}{\pgfqpoint{2.364475in}{2.461293in}}{\pgfqpoint{2.358651in}{2.467117in}}%
\pgfpathcurveto{\pgfqpoint{2.352827in}{2.472941in}}{\pgfqpoint{2.344927in}{2.476213in}}{\pgfqpoint{2.336690in}{2.476213in}}%
\pgfpathcurveto{\pgfqpoint{2.328454in}{2.476213in}}{\pgfqpoint{2.320554in}{2.472941in}}{\pgfqpoint{2.314730in}{2.467117in}}%
\pgfpathcurveto{\pgfqpoint{2.308906in}{2.461293in}}{\pgfqpoint{2.305634in}{2.453393in}}{\pgfqpoint{2.305634in}{2.445157in}}%
\pgfpathcurveto{\pgfqpoint{2.305634in}{2.436920in}}{\pgfqpoint{2.308906in}{2.429020in}}{\pgfqpoint{2.314730in}{2.423196in}}%
\pgfpathcurveto{\pgfqpoint{2.320554in}{2.417372in}}{\pgfqpoint{2.328454in}{2.414100in}}{\pgfqpoint{2.336690in}{2.414100in}}%
\pgfpathclose%
\pgfusepath{stroke,fill}%
\end{pgfscope}%
\begin{pgfscope}%
\pgfpathrectangle{\pgfqpoint{0.100000in}{0.212622in}}{\pgfqpoint{3.696000in}{3.696000in}}%
\pgfusepath{clip}%
\pgfsetbuttcap%
\pgfsetroundjoin%
\definecolor{currentfill}{rgb}{0.121569,0.466667,0.705882}%
\pgfsetfillcolor{currentfill}%
\pgfsetfillopacity{0.852532}%
\pgfsetlinewidth{1.003750pt}%
\definecolor{currentstroke}{rgb}{0.121569,0.466667,0.705882}%
\pgfsetstrokecolor{currentstroke}%
\pgfsetstrokeopacity{0.852532}%
\pgfsetdash{}{0pt}%
\pgfpathmoveto{\pgfqpoint{1.815114in}{2.226447in}}%
\pgfpathcurveto{\pgfqpoint{1.823350in}{2.226447in}}{\pgfqpoint{1.831251in}{2.229719in}}{\pgfqpoint{1.837074in}{2.235543in}}%
\pgfpathcurveto{\pgfqpoint{1.842898in}{2.241367in}}{\pgfqpoint{1.846171in}{2.249267in}}{\pgfqpoint{1.846171in}{2.257503in}}%
\pgfpathcurveto{\pgfqpoint{1.846171in}{2.265740in}}{\pgfqpoint{1.842898in}{2.273640in}}{\pgfqpoint{1.837074in}{2.279464in}}%
\pgfpathcurveto{\pgfqpoint{1.831251in}{2.285287in}}{\pgfqpoint{1.823350in}{2.288560in}}{\pgfqpoint{1.815114in}{2.288560in}}%
\pgfpathcurveto{\pgfqpoint{1.806878in}{2.288560in}}{\pgfqpoint{1.798978in}{2.285287in}}{\pgfqpoint{1.793154in}{2.279464in}}%
\pgfpathcurveto{\pgfqpoint{1.787330in}{2.273640in}}{\pgfqpoint{1.784058in}{2.265740in}}{\pgfqpoint{1.784058in}{2.257503in}}%
\pgfpathcurveto{\pgfqpoint{1.784058in}{2.249267in}}{\pgfqpoint{1.787330in}{2.241367in}}{\pgfqpoint{1.793154in}{2.235543in}}%
\pgfpathcurveto{\pgfqpoint{1.798978in}{2.229719in}}{\pgfqpoint{1.806878in}{2.226447in}}{\pgfqpoint{1.815114in}{2.226447in}}%
\pgfpathclose%
\pgfusepath{stroke,fill}%
\end{pgfscope}%
\begin{pgfscope}%
\pgfpathrectangle{\pgfqpoint{0.100000in}{0.212622in}}{\pgfqpoint{3.696000in}{3.696000in}}%
\pgfusepath{clip}%
\pgfsetbuttcap%
\pgfsetroundjoin%
\definecolor{currentfill}{rgb}{0.121569,0.466667,0.705882}%
\pgfsetfillcolor{currentfill}%
\pgfsetfillopacity{0.852694}%
\pgfsetlinewidth{1.003750pt}%
\definecolor{currentstroke}{rgb}{0.121569,0.466667,0.705882}%
\pgfsetstrokecolor{currentstroke}%
\pgfsetstrokeopacity{0.852694}%
\pgfsetdash{}{0pt}%
\pgfpathmoveto{\pgfqpoint{1.363901in}{1.316251in}}%
\pgfpathcurveto{\pgfqpoint{1.372137in}{1.316251in}}{\pgfqpoint{1.380037in}{1.319523in}}{\pgfqpoint{1.385861in}{1.325347in}}%
\pgfpathcurveto{\pgfqpoint{1.391685in}{1.331171in}}{\pgfqpoint{1.394957in}{1.339071in}}{\pgfqpoint{1.394957in}{1.347307in}}%
\pgfpathcurveto{\pgfqpoint{1.394957in}{1.355543in}}{\pgfqpoint{1.391685in}{1.363443in}}{\pgfqpoint{1.385861in}{1.369267in}}%
\pgfpathcurveto{\pgfqpoint{1.380037in}{1.375091in}}{\pgfqpoint{1.372137in}{1.378364in}}{\pgfqpoint{1.363901in}{1.378364in}}%
\pgfpathcurveto{\pgfqpoint{1.355664in}{1.378364in}}{\pgfqpoint{1.347764in}{1.375091in}}{\pgfqpoint{1.341940in}{1.369267in}}%
\pgfpathcurveto{\pgfqpoint{1.336116in}{1.363443in}}{\pgfqpoint{1.332844in}{1.355543in}}{\pgfqpoint{1.332844in}{1.347307in}}%
\pgfpathcurveto{\pgfqpoint{1.332844in}{1.339071in}}{\pgfqpoint{1.336116in}{1.331171in}}{\pgfqpoint{1.341940in}{1.325347in}}%
\pgfpathcurveto{\pgfqpoint{1.347764in}{1.319523in}}{\pgfqpoint{1.355664in}{1.316251in}}{\pgfqpoint{1.363901in}{1.316251in}}%
\pgfpathclose%
\pgfusepath{stroke,fill}%
\end{pgfscope}%
\begin{pgfscope}%
\pgfpathrectangle{\pgfqpoint{0.100000in}{0.212622in}}{\pgfqpoint{3.696000in}{3.696000in}}%
\pgfusepath{clip}%
\pgfsetbuttcap%
\pgfsetroundjoin%
\definecolor{currentfill}{rgb}{0.121569,0.466667,0.705882}%
\pgfsetfillcolor{currentfill}%
\pgfsetfillopacity{0.852834}%
\pgfsetlinewidth{1.003750pt}%
\definecolor{currentstroke}{rgb}{0.121569,0.466667,0.705882}%
\pgfsetstrokecolor{currentstroke}%
\pgfsetstrokeopacity{0.852834}%
\pgfsetdash{}{0pt}%
\pgfpathmoveto{\pgfqpoint{2.335851in}{2.408515in}}%
\pgfpathcurveto{\pgfqpoint{2.344087in}{2.408515in}}{\pgfqpoint{2.351987in}{2.411787in}}{\pgfqpoint{2.357811in}{2.417611in}}%
\pgfpathcurveto{\pgfqpoint{2.363635in}{2.423435in}}{\pgfqpoint{2.366907in}{2.431335in}}{\pgfqpoint{2.366907in}{2.439571in}}%
\pgfpathcurveto{\pgfqpoint{2.366907in}{2.447808in}}{\pgfqpoint{2.363635in}{2.455708in}}{\pgfqpoint{2.357811in}{2.461532in}}%
\pgfpathcurveto{\pgfqpoint{2.351987in}{2.467356in}}{\pgfqpoint{2.344087in}{2.470628in}}{\pgfqpoint{2.335851in}{2.470628in}}%
\pgfpathcurveto{\pgfqpoint{2.327615in}{2.470628in}}{\pgfqpoint{2.319715in}{2.467356in}}{\pgfqpoint{2.313891in}{2.461532in}}%
\pgfpathcurveto{\pgfqpoint{2.308067in}{2.455708in}}{\pgfqpoint{2.304794in}{2.447808in}}{\pgfqpoint{2.304794in}{2.439571in}}%
\pgfpathcurveto{\pgfqpoint{2.304794in}{2.431335in}}{\pgfqpoint{2.308067in}{2.423435in}}{\pgfqpoint{2.313891in}{2.417611in}}%
\pgfpathcurveto{\pgfqpoint{2.319715in}{2.411787in}}{\pgfqpoint{2.327615in}{2.408515in}}{\pgfqpoint{2.335851in}{2.408515in}}%
\pgfpathclose%
\pgfusepath{stroke,fill}%
\end{pgfscope}%
\begin{pgfscope}%
\pgfpathrectangle{\pgfqpoint{0.100000in}{0.212622in}}{\pgfqpoint{3.696000in}{3.696000in}}%
\pgfusepath{clip}%
\pgfsetbuttcap%
\pgfsetroundjoin%
\definecolor{currentfill}{rgb}{0.121569,0.466667,0.705882}%
\pgfsetfillcolor{currentfill}%
\pgfsetfillopacity{0.853285}%
\pgfsetlinewidth{1.003750pt}%
\definecolor{currentstroke}{rgb}{0.121569,0.466667,0.705882}%
\pgfsetstrokecolor{currentstroke}%
\pgfsetstrokeopacity{0.853285}%
\pgfsetdash{}{0pt}%
\pgfpathmoveto{\pgfqpoint{2.334193in}{2.406081in}}%
\pgfpathcurveto{\pgfqpoint{2.342429in}{2.406081in}}{\pgfqpoint{2.350329in}{2.409354in}}{\pgfqpoint{2.356153in}{2.415178in}}%
\pgfpathcurveto{\pgfqpoint{2.361977in}{2.421001in}}{\pgfqpoint{2.365249in}{2.428902in}}{\pgfqpoint{2.365249in}{2.437138in}}%
\pgfpathcurveto{\pgfqpoint{2.365249in}{2.445374in}}{\pgfqpoint{2.361977in}{2.453274in}}{\pgfqpoint{2.356153in}{2.459098in}}%
\pgfpathcurveto{\pgfqpoint{2.350329in}{2.464922in}}{\pgfqpoint{2.342429in}{2.468194in}}{\pgfqpoint{2.334193in}{2.468194in}}%
\pgfpathcurveto{\pgfqpoint{2.325957in}{2.468194in}}{\pgfqpoint{2.318057in}{2.464922in}}{\pgfqpoint{2.312233in}{2.459098in}}%
\pgfpathcurveto{\pgfqpoint{2.306409in}{2.453274in}}{\pgfqpoint{2.303136in}{2.445374in}}{\pgfqpoint{2.303136in}{2.437138in}}%
\pgfpathcurveto{\pgfqpoint{2.303136in}{2.428902in}}{\pgfqpoint{2.306409in}{2.421001in}}{\pgfqpoint{2.312233in}{2.415178in}}%
\pgfpathcurveto{\pgfqpoint{2.318057in}{2.409354in}}{\pgfqpoint{2.325957in}{2.406081in}}{\pgfqpoint{2.334193in}{2.406081in}}%
\pgfpathclose%
\pgfusepath{stroke,fill}%
\end{pgfscope}%
\begin{pgfscope}%
\pgfpathrectangle{\pgfqpoint{0.100000in}{0.212622in}}{\pgfqpoint{3.696000in}{3.696000in}}%
\pgfusepath{clip}%
\pgfsetbuttcap%
\pgfsetroundjoin%
\definecolor{currentfill}{rgb}{0.121569,0.466667,0.705882}%
\pgfsetfillcolor{currentfill}%
\pgfsetfillopacity{0.853346}%
\pgfsetlinewidth{1.003750pt}%
\definecolor{currentstroke}{rgb}{0.121569,0.466667,0.705882}%
\pgfsetstrokecolor{currentstroke}%
\pgfsetstrokeopacity{0.853346}%
\pgfsetdash{}{0pt}%
\pgfpathmoveto{\pgfqpoint{1.812306in}{2.222024in}}%
\pgfpathcurveto{\pgfqpoint{1.820542in}{2.222024in}}{\pgfqpoint{1.828442in}{2.225297in}}{\pgfqpoint{1.834266in}{2.231121in}}%
\pgfpathcurveto{\pgfqpoint{1.840090in}{2.236944in}}{\pgfqpoint{1.843363in}{2.244845in}}{\pgfqpoint{1.843363in}{2.253081in}}%
\pgfpathcurveto{\pgfqpoint{1.843363in}{2.261317in}}{\pgfqpoint{1.840090in}{2.269217in}}{\pgfqpoint{1.834266in}{2.275041in}}%
\pgfpathcurveto{\pgfqpoint{1.828442in}{2.280865in}}{\pgfqpoint{1.820542in}{2.284137in}}{\pgfqpoint{1.812306in}{2.284137in}}%
\pgfpathcurveto{\pgfqpoint{1.804070in}{2.284137in}}{\pgfqpoint{1.796170in}{2.280865in}}{\pgfqpoint{1.790346in}{2.275041in}}%
\pgfpathcurveto{\pgfqpoint{1.784522in}{2.269217in}}{\pgfqpoint{1.781250in}{2.261317in}}{\pgfqpoint{1.781250in}{2.253081in}}%
\pgfpathcurveto{\pgfqpoint{1.781250in}{2.244845in}}{\pgfqpoint{1.784522in}{2.236944in}}{\pgfqpoint{1.790346in}{2.231121in}}%
\pgfpathcurveto{\pgfqpoint{1.796170in}{2.225297in}}{\pgfqpoint{1.804070in}{2.222024in}}{\pgfqpoint{1.812306in}{2.222024in}}%
\pgfpathclose%
\pgfusepath{stroke,fill}%
\end{pgfscope}%
\begin{pgfscope}%
\pgfpathrectangle{\pgfqpoint{0.100000in}{0.212622in}}{\pgfqpoint{3.696000in}{3.696000in}}%
\pgfusepath{clip}%
\pgfsetbuttcap%
\pgfsetroundjoin%
\definecolor{currentfill}{rgb}{0.121569,0.466667,0.705882}%
\pgfsetfillcolor{currentfill}%
\pgfsetfillopacity{0.853813}%
\pgfsetlinewidth{1.003750pt}%
\definecolor{currentstroke}{rgb}{0.121569,0.466667,0.705882}%
\pgfsetstrokecolor{currentstroke}%
\pgfsetstrokeopacity{0.853813}%
\pgfsetdash{}{0pt}%
\pgfpathmoveto{\pgfqpoint{1.810826in}{2.219579in}}%
\pgfpathcurveto{\pgfqpoint{1.819062in}{2.219579in}}{\pgfqpoint{1.826962in}{2.222852in}}{\pgfqpoint{1.832786in}{2.228675in}}%
\pgfpathcurveto{\pgfqpoint{1.838610in}{2.234499in}}{\pgfqpoint{1.841883in}{2.242399in}}{\pgfqpoint{1.841883in}{2.250636in}}%
\pgfpathcurveto{\pgfqpoint{1.841883in}{2.258872in}}{\pgfqpoint{1.838610in}{2.266772in}}{\pgfqpoint{1.832786in}{2.272596in}}%
\pgfpathcurveto{\pgfqpoint{1.826962in}{2.278420in}}{\pgfqpoint{1.819062in}{2.281692in}}{\pgfqpoint{1.810826in}{2.281692in}}%
\pgfpathcurveto{\pgfqpoint{1.802590in}{2.281692in}}{\pgfqpoint{1.794690in}{2.278420in}}{\pgfqpoint{1.788866in}{2.272596in}}%
\pgfpathcurveto{\pgfqpoint{1.783042in}{2.266772in}}{\pgfqpoint{1.779770in}{2.258872in}}{\pgfqpoint{1.779770in}{2.250636in}}%
\pgfpathcurveto{\pgfqpoint{1.779770in}{2.242399in}}{\pgfqpoint{1.783042in}{2.234499in}}{\pgfqpoint{1.788866in}{2.228675in}}%
\pgfpathcurveto{\pgfqpoint{1.794690in}{2.222852in}}{\pgfqpoint{1.802590in}{2.219579in}}{\pgfqpoint{1.810826in}{2.219579in}}%
\pgfpathclose%
\pgfusepath{stroke,fill}%
\end{pgfscope}%
\begin{pgfscope}%
\pgfpathrectangle{\pgfqpoint{0.100000in}{0.212622in}}{\pgfqpoint{3.696000in}{3.696000in}}%
\pgfusepath{clip}%
\pgfsetbuttcap%
\pgfsetroundjoin%
\definecolor{currentfill}{rgb}{0.121569,0.466667,0.705882}%
\pgfsetfillcolor{currentfill}%
\pgfsetfillopacity{0.853895}%
\pgfsetlinewidth{1.003750pt}%
\definecolor{currentstroke}{rgb}{0.121569,0.466667,0.705882}%
\pgfsetstrokecolor{currentstroke}%
\pgfsetstrokeopacity{0.853895}%
\pgfsetdash{}{0pt}%
\pgfpathmoveto{\pgfqpoint{2.330863in}{2.401393in}}%
\pgfpathcurveto{\pgfqpoint{2.339100in}{2.401393in}}{\pgfqpoint{2.347000in}{2.404665in}}{\pgfqpoint{2.352824in}{2.410489in}}%
\pgfpathcurveto{\pgfqpoint{2.358648in}{2.416313in}}{\pgfqpoint{2.361920in}{2.424213in}}{\pgfqpoint{2.361920in}{2.432449in}}%
\pgfpathcurveto{\pgfqpoint{2.361920in}{2.440685in}}{\pgfqpoint{2.358648in}{2.448585in}}{\pgfqpoint{2.352824in}{2.454409in}}%
\pgfpathcurveto{\pgfqpoint{2.347000in}{2.460233in}}{\pgfqpoint{2.339100in}{2.463506in}}{\pgfqpoint{2.330863in}{2.463506in}}%
\pgfpathcurveto{\pgfqpoint{2.322627in}{2.463506in}}{\pgfqpoint{2.314727in}{2.460233in}}{\pgfqpoint{2.308903in}{2.454409in}}%
\pgfpathcurveto{\pgfqpoint{2.303079in}{2.448585in}}{\pgfqpoint{2.299807in}{2.440685in}}{\pgfqpoint{2.299807in}{2.432449in}}%
\pgfpathcurveto{\pgfqpoint{2.299807in}{2.424213in}}{\pgfqpoint{2.303079in}{2.416313in}}{\pgfqpoint{2.308903in}{2.410489in}}%
\pgfpathcurveto{\pgfqpoint{2.314727in}{2.404665in}}{\pgfqpoint{2.322627in}{2.401393in}}{\pgfqpoint{2.330863in}{2.401393in}}%
\pgfpathclose%
\pgfusepath{stroke,fill}%
\end{pgfscope}%
\begin{pgfscope}%
\pgfpathrectangle{\pgfqpoint{0.100000in}{0.212622in}}{\pgfqpoint{3.696000in}{3.696000in}}%
\pgfusepath{clip}%
\pgfsetbuttcap%
\pgfsetroundjoin%
\definecolor{currentfill}{rgb}{0.121569,0.466667,0.705882}%
\pgfsetfillcolor{currentfill}%
\pgfsetfillopacity{0.854173}%
\pgfsetlinewidth{1.003750pt}%
\definecolor{currentstroke}{rgb}{0.121569,0.466667,0.705882}%
\pgfsetstrokecolor{currentstroke}%
\pgfsetstrokeopacity{0.854173}%
\pgfsetdash{}{0pt}%
\pgfpathmoveto{\pgfqpoint{1.371418in}{1.316144in}}%
\pgfpathcurveto{\pgfqpoint{1.379654in}{1.316144in}}{\pgfqpoint{1.387554in}{1.319417in}}{\pgfqpoint{1.393378in}{1.325241in}}%
\pgfpathcurveto{\pgfqpoint{1.399202in}{1.331064in}}{\pgfqpoint{1.402474in}{1.338965in}}{\pgfqpoint{1.402474in}{1.347201in}}%
\pgfpathcurveto{\pgfqpoint{1.402474in}{1.355437in}}{\pgfqpoint{1.399202in}{1.363337in}}{\pgfqpoint{1.393378in}{1.369161in}}%
\pgfpathcurveto{\pgfqpoint{1.387554in}{1.374985in}}{\pgfqpoint{1.379654in}{1.378257in}}{\pgfqpoint{1.371418in}{1.378257in}}%
\pgfpathcurveto{\pgfqpoint{1.363182in}{1.378257in}}{\pgfqpoint{1.355282in}{1.374985in}}{\pgfqpoint{1.349458in}{1.369161in}}%
\pgfpathcurveto{\pgfqpoint{1.343634in}{1.363337in}}{\pgfqpoint{1.340361in}{1.355437in}}{\pgfqpoint{1.340361in}{1.347201in}}%
\pgfpathcurveto{\pgfqpoint{1.340361in}{1.338965in}}{\pgfqpoint{1.343634in}{1.331064in}}{\pgfqpoint{1.349458in}{1.325241in}}%
\pgfpathcurveto{\pgfqpoint{1.355282in}{1.319417in}}{\pgfqpoint{1.363182in}{1.316144in}}{\pgfqpoint{1.371418in}{1.316144in}}%
\pgfpathclose%
\pgfusepath{stroke,fill}%
\end{pgfscope}%
\begin{pgfscope}%
\pgfpathrectangle{\pgfqpoint{0.100000in}{0.212622in}}{\pgfqpoint{3.696000in}{3.696000in}}%
\pgfusepath{clip}%
\pgfsetbuttcap%
\pgfsetroundjoin%
\definecolor{currentfill}{rgb}{0.121569,0.466667,0.705882}%
\pgfsetfillcolor{currentfill}%
\pgfsetfillopacity{0.854497}%
\pgfsetlinewidth{1.003750pt}%
\definecolor{currentstroke}{rgb}{0.121569,0.466667,0.705882}%
\pgfsetstrokecolor{currentstroke}%
\pgfsetstrokeopacity{0.854497}%
\pgfsetdash{}{0pt}%
\pgfpathmoveto{\pgfqpoint{1.810043in}{2.215185in}}%
\pgfpathcurveto{\pgfqpoint{1.818279in}{2.215185in}}{\pgfqpoint{1.826179in}{2.218457in}}{\pgfqpoint{1.832003in}{2.224281in}}%
\pgfpathcurveto{\pgfqpoint{1.837827in}{2.230105in}}{\pgfqpoint{1.841099in}{2.238005in}}{\pgfqpoint{1.841099in}{2.246241in}}%
\pgfpathcurveto{\pgfqpoint{1.841099in}{2.254477in}}{\pgfqpoint{1.837827in}{2.262377in}}{\pgfqpoint{1.832003in}{2.268201in}}%
\pgfpathcurveto{\pgfqpoint{1.826179in}{2.274025in}}{\pgfqpoint{1.818279in}{2.277298in}}{\pgfqpoint{1.810043in}{2.277298in}}%
\pgfpathcurveto{\pgfqpoint{1.801806in}{2.277298in}}{\pgfqpoint{1.793906in}{2.274025in}}{\pgfqpoint{1.788082in}{2.268201in}}%
\pgfpathcurveto{\pgfqpoint{1.782258in}{2.262377in}}{\pgfqpoint{1.778986in}{2.254477in}}{\pgfqpoint{1.778986in}{2.246241in}}%
\pgfpathcurveto{\pgfqpoint{1.778986in}{2.238005in}}{\pgfqpoint{1.782258in}{2.230105in}}{\pgfqpoint{1.788082in}{2.224281in}}%
\pgfpathcurveto{\pgfqpoint{1.793906in}{2.218457in}}{\pgfqpoint{1.801806in}{2.215185in}}{\pgfqpoint{1.810043in}{2.215185in}}%
\pgfpathclose%
\pgfusepath{stroke,fill}%
\end{pgfscope}%
\begin{pgfscope}%
\pgfpathrectangle{\pgfqpoint{0.100000in}{0.212622in}}{\pgfqpoint{3.696000in}{3.696000in}}%
\pgfusepath{clip}%
\pgfsetbuttcap%
\pgfsetroundjoin%
\definecolor{currentfill}{rgb}{0.121569,0.466667,0.705882}%
\pgfsetfillcolor{currentfill}%
\pgfsetfillopacity{0.854627}%
\pgfsetlinewidth{1.003750pt}%
\definecolor{currentstroke}{rgb}{0.121569,0.466667,0.705882}%
\pgfsetstrokecolor{currentstroke}%
\pgfsetstrokeopacity{0.854627}%
\pgfsetdash{}{0pt}%
\pgfpathmoveto{\pgfqpoint{2.329163in}{2.397242in}}%
\pgfpathcurveto{\pgfqpoint{2.337399in}{2.397242in}}{\pgfqpoint{2.345299in}{2.400514in}}{\pgfqpoint{2.351123in}{2.406338in}}%
\pgfpathcurveto{\pgfqpoint{2.356947in}{2.412162in}}{\pgfqpoint{2.360219in}{2.420062in}}{\pgfqpoint{2.360219in}{2.428298in}}%
\pgfpathcurveto{\pgfqpoint{2.360219in}{2.436534in}}{\pgfqpoint{2.356947in}{2.444435in}}{\pgfqpoint{2.351123in}{2.450258in}}%
\pgfpathcurveto{\pgfqpoint{2.345299in}{2.456082in}}{\pgfqpoint{2.337399in}{2.459355in}}{\pgfqpoint{2.329163in}{2.459355in}}%
\pgfpathcurveto{\pgfqpoint{2.320926in}{2.459355in}}{\pgfqpoint{2.313026in}{2.456082in}}{\pgfqpoint{2.307202in}{2.450258in}}%
\pgfpathcurveto{\pgfqpoint{2.301379in}{2.444435in}}{\pgfqpoint{2.298106in}{2.436534in}}{\pgfqpoint{2.298106in}{2.428298in}}%
\pgfpathcurveto{\pgfqpoint{2.298106in}{2.420062in}}{\pgfqpoint{2.301379in}{2.412162in}}{\pgfqpoint{2.307202in}{2.406338in}}%
\pgfpathcurveto{\pgfqpoint{2.313026in}{2.400514in}}{\pgfqpoint{2.320926in}{2.397242in}}{\pgfqpoint{2.329163in}{2.397242in}}%
\pgfpathclose%
\pgfusepath{stroke,fill}%
\end{pgfscope}%
\begin{pgfscope}%
\pgfpathrectangle{\pgfqpoint{0.100000in}{0.212622in}}{\pgfqpoint{3.696000in}{3.696000in}}%
\pgfusepath{clip}%
\pgfsetbuttcap%
\pgfsetroundjoin%
\definecolor{currentfill}{rgb}{0.121569,0.466667,0.705882}%
\pgfsetfillcolor{currentfill}%
\pgfsetfillopacity{0.855320}%
\pgfsetlinewidth{1.003750pt}%
\definecolor{currentstroke}{rgb}{0.121569,0.466667,0.705882}%
\pgfsetstrokecolor{currentstroke}%
\pgfsetstrokeopacity{0.855320}%
\pgfsetdash{}{0pt}%
\pgfpathmoveto{\pgfqpoint{1.381113in}{1.309550in}}%
\pgfpathcurveto{\pgfqpoint{1.389349in}{1.309550in}}{\pgfqpoint{1.397249in}{1.312822in}}{\pgfqpoint{1.403073in}{1.318646in}}%
\pgfpathcurveto{\pgfqpoint{1.408897in}{1.324470in}}{\pgfqpoint{1.412170in}{1.332370in}}{\pgfqpoint{1.412170in}{1.340607in}}%
\pgfpathcurveto{\pgfqpoint{1.412170in}{1.348843in}}{\pgfqpoint{1.408897in}{1.356743in}}{\pgfqpoint{1.403073in}{1.362567in}}%
\pgfpathcurveto{\pgfqpoint{1.397249in}{1.368391in}}{\pgfqpoint{1.389349in}{1.371663in}}{\pgfqpoint{1.381113in}{1.371663in}}%
\pgfpathcurveto{\pgfqpoint{1.372877in}{1.371663in}}{\pgfqpoint{1.364977in}{1.368391in}}{\pgfqpoint{1.359153in}{1.362567in}}%
\pgfpathcurveto{\pgfqpoint{1.353329in}{1.356743in}}{\pgfqpoint{1.350057in}{1.348843in}}{\pgfqpoint{1.350057in}{1.340607in}}%
\pgfpathcurveto{\pgfqpoint{1.350057in}{1.332370in}}{\pgfqpoint{1.353329in}{1.324470in}}{\pgfqpoint{1.359153in}{1.318646in}}%
\pgfpathcurveto{\pgfqpoint{1.364977in}{1.312822in}}{\pgfqpoint{1.372877in}{1.309550in}}{\pgfqpoint{1.381113in}{1.309550in}}%
\pgfpathclose%
\pgfusepath{stroke,fill}%
\end{pgfscope}%
\begin{pgfscope}%
\pgfpathrectangle{\pgfqpoint{0.100000in}{0.212622in}}{\pgfqpoint{3.696000in}{3.696000in}}%
\pgfusepath{clip}%
\pgfsetbuttcap%
\pgfsetroundjoin%
\definecolor{currentfill}{rgb}{0.121569,0.466667,0.705882}%
\pgfsetfillcolor{currentfill}%
\pgfsetfillopacity{0.855386}%
\pgfsetlinewidth{1.003750pt}%
\definecolor{currentstroke}{rgb}{0.121569,0.466667,0.705882}%
\pgfsetstrokecolor{currentstroke}%
\pgfsetstrokeopacity{0.855386}%
\pgfsetdash{}{0pt}%
\pgfpathmoveto{\pgfqpoint{1.808112in}{2.210973in}}%
\pgfpathcurveto{\pgfqpoint{1.816348in}{2.210973in}}{\pgfqpoint{1.824248in}{2.214245in}}{\pgfqpoint{1.830072in}{2.220069in}}%
\pgfpathcurveto{\pgfqpoint{1.835896in}{2.225893in}}{\pgfqpoint{1.839168in}{2.233793in}}{\pgfqpoint{1.839168in}{2.242029in}}%
\pgfpathcurveto{\pgfqpoint{1.839168in}{2.250265in}}{\pgfqpoint{1.835896in}{2.258166in}}{\pgfqpoint{1.830072in}{2.263989in}}%
\pgfpathcurveto{\pgfqpoint{1.824248in}{2.269813in}}{\pgfqpoint{1.816348in}{2.273086in}}{\pgfqpoint{1.808112in}{2.273086in}}%
\pgfpathcurveto{\pgfqpoint{1.799876in}{2.273086in}}{\pgfqpoint{1.791976in}{2.269813in}}{\pgfqpoint{1.786152in}{2.263989in}}%
\pgfpathcurveto{\pgfqpoint{1.780328in}{2.258166in}}{\pgfqpoint{1.777055in}{2.250265in}}{\pgfqpoint{1.777055in}{2.242029in}}%
\pgfpathcurveto{\pgfqpoint{1.777055in}{2.233793in}}{\pgfqpoint{1.780328in}{2.225893in}}{\pgfqpoint{1.786152in}{2.220069in}}%
\pgfpathcurveto{\pgfqpoint{1.791976in}{2.214245in}}{\pgfqpoint{1.799876in}{2.210973in}}{\pgfqpoint{1.808112in}{2.210973in}}%
\pgfpathclose%
\pgfusepath{stroke,fill}%
\end{pgfscope}%
\begin{pgfscope}%
\pgfpathrectangle{\pgfqpoint{0.100000in}{0.212622in}}{\pgfqpoint{3.696000in}{3.696000in}}%
\pgfusepath{clip}%
\pgfsetbuttcap%
\pgfsetroundjoin%
\definecolor{currentfill}{rgb}{0.121569,0.466667,0.705882}%
\pgfsetfillcolor{currentfill}%
\pgfsetfillopacity{0.855736}%
\pgfsetlinewidth{1.003750pt}%
\definecolor{currentstroke}{rgb}{0.121569,0.466667,0.705882}%
\pgfsetstrokecolor{currentstroke}%
\pgfsetstrokeopacity{0.855736}%
\pgfsetdash{}{0pt}%
\pgfpathmoveto{\pgfqpoint{2.327274in}{2.388196in}}%
\pgfpathcurveto{\pgfqpoint{2.335511in}{2.388196in}}{\pgfqpoint{2.343411in}{2.391468in}}{\pgfqpoint{2.349235in}{2.397292in}}%
\pgfpathcurveto{\pgfqpoint{2.355059in}{2.403116in}}{\pgfqpoint{2.358331in}{2.411016in}}{\pgfqpoint{2.358331in}{2.419252in}}%
\pgfpathcurveto{\pgfqpoint{2.358331in}{2.427488in}}{\pgfqpoint{2.355059in}{2.435388in}}{\pgfqpoint{2.349235in}{2.441212in}}%
\pgfpathcurveto{\pgfqpoint{2.343411in}{2.447036in}}{\pgfqpoint{2.335511in}{2.450309in}}{\pgfqpoint{2.327274in}{2.450309in}}%
\pgfpathcurveto{\pgfqpoint{2.319038in}{2.450309in}}{\pgfqpoint{2.311138in}{2.447036in}}{\pgfqpoint{2.305314in}{2.441212in}}%
\pgfpathcurveto{\pgfqpoint{2.299490in}{2.435388in}}{\pgfqpoint{2.296218in}{2.427488in}}{\pgfqpoint{2.296218in}{2.419252in}}%
\pgfpathcurveto{\pgfqpoint{2.296218in}{2.411016in}}{\pgfqpoint{2.299490in}{2.403116in}}{\pgfqpoint{2.305314in}{2.397292in}}%
\pgfpathcurveto{\pgfqpoint{2.311138in}{2.391468in}}{\pgfqpoint{2.319038in}{2.388196in}}{\pgfqpoint{2.327274in}{2.388196in}}%
\pgfpathclose%
\pgfusepath{stroke,fill}%
\end{pgfscope}%
\begin{pgfscope}%
\pgfpathrectangle{\pgfqpoint{0.100000in}{0.212622in}}{\pgfqpoint{3.696000in}{3.696000in}}%
\pgfusepath{clip}%
\pgfsetbuttcap%
\pgfsetroundjoin%
\definecolor{currentfill}{rgb}{0.121569,0.466667,0.705882}%
\pgfsetfillcolor{currentfill}%
\pgfsetfillopacity{0.856212}%
\pgfsetlinewidth{1.003750pt}%
\definecolor{currentstroke}{rgb}{0.121569,0.466667,0.705882}%
\pgfsetstrokecolor{currentstroke}%
\pgfsetstrokeopacity{0.856212}%
\pgfsetdash{}{0pt}%
\pgfpathmoveto{\pgfqpoint{1.805276in}{2.206468in}}%
\pgfpathcurveto{\pgfqpoint{1.813512in}{2.206468in}}{\pgfqpoint{1.821412in}{2.209740in}}{\pgfqpoint{1.827236in}{2.215564in}}%
\pgfpathcurveto{\pgfqpoint{1.833060in}{2.221388in}}{\pgfqpoint{1.836332in}{2.229288in}}{\pgfqpoint{1.836332in}{2.237524in}}%
\pgfpathcurveto{\pgfqpoint{1.836332in}{2.245761in}}{\pgfqpoint{1.833060in}{2.253661in}}{\pgfqpoint{1.827236in}{2.259485in}}%
\pgfpathcurveto{\pgfqpoint{1.821412in}{2.265309in}}{\pgfqpoint{1.813512in}{2.268581in}}{\pgfqpoint{1.805276in}{2.268581in}}%
\pgfpathcurveto{\pgfqpoint{1.797039in}{2.268581in}}{\pgfqpoint{1.789139in}{2.265309in}}{\pgfqpoint{1.783315in}{2.259485in}}%
\pgfpathcurveto{\pgfqpoint{1.777491in}{2.253661in}}{\pgfqpoint{1.774219in}{2.245761in}}{\pgfqpoint{1.774219in}{2.237524in}}%
\pgfpathcurveto{\pgfqpoint{1.774219in}{2.229288in}}{\pgfqpoint{1.777491in}{2.221388in}}{\pgfqpoint{1.783315in}{2.215564in}}%
\pgfpathcurveto{\pgfqpoint{1.789139in}{2.209740in}}{\pgfqpoint{1.797039in}{2.206468in}}{\pgfqpoint{1.805276in}{2.206468in}}%
\pgfpathclose%
\pgfusepath{stroke,fill}%
\end{pgfscope}%
\begin{pgfscope}%
\pgfpathrectangle{\pgfqpoint{0.100000in}{0.212622in}}{\pgfqpoint{3.696000in}{3.696000in}}%
\pgfusepath{clip}%
\pgfsetbuttcap%
\pgfsetroundjoin%
\definecolor{currentfill}{rgb}{0.121569,0.466667,0.705882}%
\pgfsetfillcolor{currentfill}%
\pgfsetfillopacity{0.856524}%
\pgfsetlinewidth{1.003750pt}%
\definecolor{currentstroke}{rgb}{0.121569,0.466667,0.705882}%
\pgfsetstrokecolor{currentstroke}%
\pgfsetstrokeopacity{0.856524}%
\pgfsetdash{}{0pt}%
\pgfpathmoveto{\pgfqpoint{2.323855in}{2.383800in}}%
\pgfpathcurveto{\pgfqpoint{2.332091in}{2.383800in}}{\pgfqpoint{2.339991in}{2.387072in}}{\pgfqpoint{2.345815in}{2.392896in}}%
\pgfpathcurveto{\pgfqpoint{2.351639in}{2.398720in}}{\pgfqpoint{2.354912in}{2.406620in}}{\pgfqpoint{2.354912in}{2.414856in}}%
\pgfpathcurveto{\pgfqpoint{2.354912in}{2.423093in}}{\pgfqpoint{2.351639in}{2.430993in}}{\pgfqpoint{2.345815in}{2.436817in}}%
\pgfpathcurveto{\pgfqpoint{2.339991in}{2.442641in}}{\pgfqpoint{2.332091in}{2.445913in}}{\pgfqpoint{2.323855in}{2.445913in}}%
\pgfpathcurveto{\pgfqpoint{2.315619in}{2.445913in}}{\pgfqpoint{2.307719in}{2.442641in}}{\pgfqpoint{2.301895in}{2.436817in}}%
\pgfpathcurveto{\pgfqpoint{2.296071in}{2.430993in}}{\pgfqpoint{2.292799in}{2.423093in}}{\pgfqpoint{2.292799in}{2.414856in}}%
\pgfpathcurveto{\pgfqpoint{2.292799in}{2.406620in}}{\pgfqpoint{2.296071in}{2.398720in}}{\pgfqpoint{2.301895in}{2.392896in}}%
\pgfpathcurveto{\pgfqpoint{2.307719in}{2.387072in}}{\pgfqpoint{2.315619in}{2.383800in}}{\pgfqpoint{2.323855in}{2.383800in}}%
\pgfpathclose%
\pgfusepath{stroke,fill}%
\end{pgfscope}%
\begin{pgfscope}%
\pgfpathrectangle{\pgfqpoint{0.100000in}{0.212622in}}{\pgfqpoint{3.696000in}{3.696000in}}%
\pgfusepath{clip}%
\pgfsetbuttcap%
\pgfsetroundjoin%
\definecolor{currentfill}{rgb}{0.121569,0.466667,0.705882}%
\pgfsetfillcolor{currentfill}%
\pgfsetfillopacity{0.857153}%
\pgfsetlinewidth{1.003750pt}%
\definecolor{currentstroke}{rgb}{0.121569,0.466667,0.705882}%
\pgfsetstrokecolor{currentstroke}%
\pgfsetstrokeopacity{0.857153}%
\pgfsetdash{}{0pt}%
\pgfpathmoveto{\pgfqpoint{2.320784in}{2.379032in}}%
\pgfpathcurveto{\pgfqpoint{2.329020in}{2.379032in}}{\pgfqpoint{2.336920in}{2.382304in}}{\pgfqpoint{2.342744in}{2.388128in}}%
\pgfpathcurveto{\pgfqpoint{2.348568in}{2.393952in}}{\pgfqpoint{2.351840in}{2.401852in}}{\pgfqpoint{2.351840in}{2.410089in}}%
\pgfpathcurveto{\pgfqpoint{2.351840in}{2.418325in}}{\pgfqpoint{2.348568in}{2.426225in}}{\pgfqpoint{2.342744in}{2.432049in}}%
\pgfpathcurveto{\pgfqpoint{2.336920in}{2.437873in}}{\pgfqpoint{2.329020in}{2.441145in}}{\pgfqpoint{2.320784in}{2.441145in}}%
\pgfpathcurveto{\pgfqpoint{2.312547in}{2.441145in}}{\pgfqpoint{2.304647in}{2.437873in}}{\pgfqpoint{2.298823in}{2.432049in}}%
\pgfpathcurveto{\pgfqpoint{2.292999in}{2.426225in}}{\pgfqpoint{2.289727in}{2.418325in}}{\pgfqpoint{2.289727in}{2.410089in}}%
\pgfpathcurveto{\pgfqpoint{2.289727in}{2.401852in}}{\pgfqpoint{2.292999in}{2.393952in}}{\pgfqpoint{2.298823in}{2.388128in}}%
\pgfpathcurveto{\pgfqpoint{2.304647in}{2.382304in}}{\pgfqpoint{2.312547in}{2.379032in}}{\pgfqpoint{2.320784in}{2.379032in}}%
\pgfpathclose%
\pgfusepath{stroke,fill}%
\end{pgfscope}%
\begin{pgfscope}%
\pgfpathrectangle{\pgfqpoint{0.100000in}{0.212622in}}{\pgfqpoint{3.696000in}{3.696000in}}%
\pgfusepath{clip}%
\pgfsetbuttcap%
\pgfsetroundjoin%
\definecolor{currentfill}{rgb}{0.121569,0.466667,0.705882}%
\pgfsetfillcolor{currentfill}%
\pgfsetfillopacity{0.857523}%
\pgfsetlinewidth{1.003750pt}%
\definecolor{currentstroke}{rgb}{0.121569,0.466667,0.705882}%
\pgfsetstrokecolor{currentstroke}%
\pgfsetstrokeopacity{0.857523}%
\pgfsetdash{}{0pt}%
\pgfpathmoveto{\pgfqpoint{1.803157in}{2.199220in}}%
\pgfpathcurveto{\pgfqpoint{1.811393in}{2.199220in}}{\pgfqpoint{1.819293in}{2.202493in}}{\pgfqpoint{1.825117in}{2.208316in}}%
\pgfpathcurveto{\pgfqpoint{1.830941in}{2.214140in}}{\pgfqpoint{1.834213in}{2.222040in}}{\pgfqpoint{1.834213in}{2.230277in}}%
\pgfpathcurveto{\pgfqpoint{1.834213in}{2.238513in}}{\pgfqpoint{1.830941in}{2.246413in}}{\pgfqpoint{1.825117in}{2.252237in}}%
\pgfpathcurveto{\pgfqpoint{1.819293in}{2.258061in}}{\pgfqpoint{1.811393in}{2.261333in}}{\pgfqpoint{1.803157in}{2.261333in}}%
\pgfpathcurveto{\pgfqpoint{1.794920in}{2.261333in}}{\pgfqpoint{1.787020in}{2.258061in}}{\pgfqpoint{1.781196in}{2.252237in}}%
\pgfpathcurveto{\pgfqpoint{1.775372in}{2.246413in}}{\pgfqpoint{1.772100in}{2.238513in}}{\pgfqpoint{1.772100in}{2.230277in}}%
\pgfpathcurveto{\pgfqpoint{1.772100in}{2.222040in}}{\pgfqpoint{1.775372in}{2.214140in}}{\pgfqpoint{1.781196in}{2.208316in}}%
\pgfpathcurveto{\pgfqpoint{1.787020in}{2.202493in}}{\pgfqpoint{1.794920in}{2.199220in}}{\pgfqpoint{1.803157in}{2.199220in}}%
\pgfpathclose%
\pgfusepath{stroke,fill}%
\end{pgfscope}%
\begin{pgfscope}%
\pgfpathrectangle{\pgfqpoint{0.100000in}{0.212622in}}{\pgfqpoint{3.696000in}{3.696000in}}%
\pgfusepath{clip}%
\pgfsetbuttcap%
\pgfsetroundjoin%
\definecolor{currentfill}{rgb}{0.121569,0.466667,0.705882}%
\pgfsetfillcolor{currentfill}%
\pgfsetfillopacity{0.857562}%
\pgfsetlinewidth{1.003750pt}%
\definecolor{currentstroke}{rgb}{0.121569,0.466667,0.705882}%
\pgfsetstrokecolor{currentstroke}%
\pgfsetstrokeopacity{0.857562}%
\pgfsetdash{}{0pt}%
\pgfpathmoveto{\pgfqpoint{1.390825in}{1.301776in}}%
\pgfpathcurveto{\pgfqpoint{1.399061in}{1.301776in}}{\pgfqpoint{1.406961in}{1.305048in}}{\pgfqpoint{1.412785in}{1.310872in}}%
\pgfpathcurveto{\pgfqpoint{1.418609in}{1.316696in}}{\pgfqpoint{1.421881in}{1.324596in}}{\pgfqpoint{1.421881in}{1.332832in}}%
\pgfpathcurveto{\pgfqpoint{1.421881in}{1.341069in}}{\pgfqpoint{1.418609in}{1.348969in}}{\pgfqpoint{1.412785in}{1.354792in}}%
\pgfpathcurveto{\pgfqpoint{1.406961in}{1.360616in}}{\pgfqpoint{1.399061in}{1.363889in}}{\pgfqpoint{1.390825in}{1.363889in}}%
\pgfpathcurveto{\pgfqpoint{1.382589in}{1.363889in}}{\pgfqpoint{1.374689in}{1.360616in}}{\pgfqpoint{1.368865in}{1.354792in}}%
\pgfpathcurveto{\pgfqpoint{1.363041in}{1.348969in}}{\pgfqpoint{1.359768in}{1.341069in}}{\pgfqpoint{1.359768in}{1.332832in}}%
\pgfpathcurveto{\pgfqpoint{1.359768in}{1.324596in}}{\pgfqpoint{1.363041in}{1.316696in}}{\pgfqpoint{1.368865in}{1.310872in}}%
\pgfpathcurveto{\pgfqpoint{1.374689in}{1.305048in}}{\pgfqpoint{1.382589in}{1.301776in}}{\pgfqpoint{1.390825in}{1.301776in}}%
\pgfpathclose%
\pgfusepath{stroke,fill}%
\end{pgfscope}%
\begin{pgfscope}%
\pgfpathrectangle{\pgfqpoint{0.100000in}{0.212622in}}{\pgfqpoint{3.696000in}{3.696000in}}%
\pgfusepath{clip}%
\pgfsetbuttcap%
\pgfsetroundjoin%
\definecolor{currentfill}{rgb}{0.121569,0.466667,0.705882}%
\pgfsetfillcolor{currentfill}%
\pgfsetfillopacity{0.857702}%
\pgfsetlinewidth{1.003750pt}%
\definecolor{currentstroke}{rgb}{0.121569,0.466667,0.705882}%
\pgfsetstrokecolor{currentstroke}%
\pgfsetstrokeopacity{0.857702}%
\pgfsetdash{}{0pt}%
\pgfpathmoveto{\pgfqpoint{2.319617in}{2.375098in}}%
\pgfpathcurveto{\pgfqpoint{2.327853in}{2.375098in}}{\pgfqpoint{2.335753in}{2.378371in}}{\pgfqpoint{2.341577in}{2.384195in}}%
\pgfpathcurveto{\pgfqpoint{2.347401in}{2.390019in}}{\pgfqpoint{2.350674in}{2.397919in}}{\pgfqpoint{2.350674in}{2.406155in}}%
\pgfpathcurveto{\pgfqpoint{2.350674in}{2.414391in}}{\pgfqpoint{2.347401in}{2.422291in}}{\pgfqpoint{2.341577in}{2.428115in}}%
\pgfpathcurveto{\pgfqpoint{2.335753in}{2.433939in}}{\pgfqpoint{2.327853in}{2.437211in}}{\pgfqpoint{2.319617in}{2.437211in}}%
\pgfpathcurveto{\pgfqpoint{2.311381in}{2.437211in}}{\pgfqpoint{2.303481in}{2.433939in}}{\pgfqpoint{2.297657in}{2.428115in}}%
\pgfpathcurveto{\pgfqpoint{2.291833in}{2.422291in}}{\pgfqpoint{2.288561in}{2.414391in}}{\pgfqpoint{2.288561in}{2.406155in}}%
\pgfpathcurveto{\pgfqpoint{2.288561in}{2.397919in}}{\pgfqpoint{2.291833in}{2.390019in}}{\pgfqpoint{2.297657in}{2.384195in}}%
\pgfpathcurveto{\pgfqpoint{2.303481in}{2.378371in}}{\pgfqpoint{2.311381in}{2.375098in}}{\pgfqpoint{2.319617in}{2.375098in}}%
\pgfpathclose%
\pgfusepath{stroke,fill}%
\end{pgfscope}%
\begin{pgfscope}%
\pgfpathrectangle{\pgfqpoint{0.100000in}{0.212622in}}{\pgfqpoint{3.696000in}{3.696000in}}%
\pgfusepath{clip}%
\pgfsetbuttcap%
\pgfsetroundjoin%
\definecolor{currentfill}{rgb}{0.121569,0.466667,0.705882}%
\pgfsetfillcolor{currentfill}%
\pgfsetfillopacity{0.858500}%
\pgfsetlinewidth{1.003750pt}%
\definecolor{currentstroke}{rgb}{0.121569,0.466667,0.705882}%
\pgfsetstrokecolor{currentstroke}%
\pgfsetstrokeopacity{0.858500}%
\pgfsetdash{}{0pt}%
\pgfpathmoveto{\pgfqpoint{1.396302in}{1.296519in}}%
\pgfpathcurveto{\pgfqpoint{1.404538in}{1.296519in}}{\pgfqpoint{1.412438in}{1.299791in}}{\pgfqpoint{1.418262in}{1.305615in}}%
\pgfpathcurveto{\pgfqpoint{1.424086in}{1.311439in}}{\pgfqpoint{1.427358in}{1.319339in}}{\pgfqpoint{1.427358in}{1.327575in}}%
\pgfpathcurveto{\pgfqpoint{1.427358in}{1.335812in}}{\pgfqpoint{1.424086in}{1.343712in}}{\pgfqpoint{1.418262in}{1.349536in}}%
\pgfpathcurveto{\pgfqpoint{1.412438in}{1.355360in}}{\pgfqpoint{1.404538in}{1.358632in}}{\pgfqpoint{1.396302in}{1.358632in}}%
\pgfpathcurveto{\pgfqpoint{1.388066in}{1.358632in}}{\pgfqpoint{1.380166in}{1.355360in}}{\pgfqpoint{1.374342in}{1.349536in}}%
\pgfpathcurveto{\pgfqpoint{1.368518in}{1.343712in}}{\pgfqpoint{1.365245in}{1.335812in}}{\pgfqpoint{1.365245in}{1.327575in}}%
\pgfpathcurveto{\pgfqpoint{1.365245in}{1.319339in}}{\pgfqpoint{1.368518in}{1.311439in}}{\pgfqpoint{1.374342in}{1.305615in}}%
\pgfpathcurveto{\pgfqpoint{1.380166in}{1.299791in}}{\pgfqpoint{1.388066in}{1.296519in}}{\pgfqpoint{1.396302in}{1.296519in}}%
\pgfpathclose%
\pgfusepath{stroke,fill}%
\end{pgfscope}%
\begin{pgfscope}%
\pgfpathrectangle{\pgfqpoint{0.100000in}{0.212622in}}{\pgfqpoint{3.696000in}{3.696000in}}%
\pgfusepath{clip}%
\pgfsetbuttcap%
\pgfsetroundjoin%
\definecolor{currentfill}{rgb}{0.121569,0.466667,0.705882}%
\pgfsetfillcolor{currentfill}%
\pgfsetfillopacity{0.858823}%
\pgfsetlinewidth{1.003750pt}%
\definecolor{currentstroke}{rgb}{0.121569,0.466667,0.705882}%
\pgfsetstrokecolor{currentstroke}%
\pgfsetstrokeopacity{0.858823}%
\pgfsetdash{}{0pt}%
\pgfpathmoveto{\pgfqpoint{2.318208in}{2.368333in}}%
\pgfpathcurveto{\pgfqpoint{2.326445in}{2.368333in}}{\pgfqpoint{2.334345in}{2.371605in}}{\pgfqpoint{2.340169in}{2.377429in}}%
\pgfpathcurveto{\pgfqpoint{2.345992in}{2.383253in}}{\pgfqpoint{2.349265in}{2.391153in}}{\pgfqpoint{2.349265in}{2.399389in}}%
\pgfpathcurveto{\pgfqpoint{2.349265in}{2.407626in}}{\pgfqpoint{2.345992in}{2.415526in}}{\pgfqpoint{2.340169in}{2.421350in}}%
\pgfpathcurveto{\pgfqpoint{2.334345in}{2.427174in}}{\pgfqpoint{2.326445in}{2.430446in}}{\pgfqpoint{2.318208in}{2.430446in}}%
\pgfpathcurveto{\pgfqpoint{2.309972in}{2.430446in}}{\pgfqpoint{2.302072in}{2.427174in}}{\pgfqpoint{2.296248in}{2.421350in}}%
\pgfpathcurveto{\pgfqpoint{2.290424in}{2.415526in}}{\pgfqpoint{2.287152in}{2.407626in}}{\pgfqpoint{2.287152in}{2.399389in}}%
\pgfpathcurveto{\pgfqpoint{2.287152in}{2.391153in}}{\pgfqpoint{2.290424in}{2.383253in}}{\pgfqpoint{2.296248in}{2.377429in}}%
\pgfpathcurveto{\pgfqpoint{2.302072in}{2.371605in}}{\pgfqpoint{2.309972in}{2.368333in}}{\pgfqpoint{2.318208in}{2.368333in}}%
\pgfpathclose%
\pgfusepath{stroke,fill}%
\end{pgfscope}%
\begin{pgfscope}%
\pgfpathrectangle{\pgfqpoint{0.100000in}{0.212622in}}{\pgfqpoint{3.696000in}{3.696000in}}%
\pgfusepath{clip}%
\pgfsetbuttcap%
\pgfsetroundjoin%
\definecolor{currentfill}{rgb}{0.121569,0.466667,0.705882}%
\pgfsetfillcolor{currentfill}%
\pgfsetfillopacity{0.858940}%
\pgfsetlinewidth{1.003750pt}%
\definecolor{currentstroke}{rgb}{0.121569,0.466667,0.705882}%
\pgfsetstrokecolor{currentstroke}%
\pgfsetstrokeopacity{0.858940}%
\pgfsetdash{}{0pt}%
\pgfpathmoveto{\pgfqpoint{1.800771in}{2.191621in}}%
\pgfpathcurveto{\pgfqpoint{1.809007in}{2.191621in}}{\pgfqpoint{1.816907in}{2.194894in}}{\pgfqpoint{1.822731in}{2.200718in}}%
\pgfpathcurveto{\pgfqpoint{1.828555in}{2.206542in}}{\pgfqpoint{1.831827in}{2.214442in}}{\pgfqpoint{1.831827in}{2.222678in}}%
\pgfpathcurveto{\pgfqpoint{1.831827in}{2.230914in}}{\pgfqpoint{1.828555in}{2.238814in}}{\pgfqpoint{1.822731in}{2.244638in}}%
\pgfpathcurveto{\pgfqpoint{1.816907in}{2.250462in}}{\pgfqpoint{1.809007in}{2.253734in}}{\pgfqpoint{1.800771in}{2.253734in}}%
\pgfpathcurveto{\pgfqpoint{1.792534in}{2.253734in}}{\pgfqpoint{1.784634in}{2.250462in}}{\pgfqpoint{1.778810in}{2.244638in}}%
\pgfpathcurveto{\pgfqpoint{1.772986in}{2.238814in}}{\pgfqpoint{1.769714in}{2.230914in}}{\pgfqpoint{1.769714in}{2.222678in}}%
\pgfpathcurveto{\pgfqpoint{1.769714in}{2.214442in}}{\pgfqpoint{1.772986in}{2.206542in}}{\pgfqpoint{1.778810in}{2.200718in}}%
\pgfpathcurveto{\pgfqpoint{1.784634in}{2.194894in}}{\pgfqpoint{1.792534in}{2.191621in}}{\pgfqpoint{1.800771in}{2.191621in}}%
\pgfpathclose%
\pgfusepath{stroke,fill}%
\end{pgfscope}%
\begin{pgfscope}%
\pgfpathrectangle{\pgfqpoint{0.100000in}{0.212622in}}{\pgfqpoint{3.696000in}{3.696000in}}%
\pgfusepath{clip}%
\pgfsetbuttcap%
\pgfsetroundjoin%
\definecolor{currentfill}{rgb}{0.121569,0.466667,0.705882}%
\pgfsetfillcolor{currentfill}%
\pgfsetfillopacity{0.858998}%
\pgfsetlinewidth{1.003750pt}%
\definecolor{currentstroke}{rgb}{0.121569,0.466667,0.705882}%
\pgfsetstrokecolor{currentstroke}%
\pgfsetstrokeopacity{0.858998}%
\pgfsetdash{}{0pt}%
\pgfpathmoveto{\pgfqpoint{2.908129in}{1.616314in}}%
\pgfpathcurveto{\pgfqpoint{2.916365in}{1.616314in}}{\pgfqpoint{2.924265in}{1.619586in}}{\pgfqpoint{2.930089in}{1.625410in}}%
\pgfpathcurveto{\pgfqpoint{2.935913in}{1.631234in}}{\pgfqpoint{2.939185in}{1.639134in}}{\pgfqpoint{2.939185in}{1.647370in}}%
\pgfpathcurveto{\pgfqpoint{2.939185in}{1.655607in}}{\pgfqpoint{2.935913in}{1.663507in}}{\pgfqpoint{2.930089in}{1.669331in}}%
\pgfpathcurveto{\pgfqpoint{2.924265in}{1.675155in}}{\pgfqpoint{2.916365in}{1.678427in}}{\pgfqpoint{2.908129in}{1.678427in}}%
\pgfpathcurveto{\pgfqpoint{2.899892in}{1.678427in}}{\pgfqpoint{2.891992in}{1.675155in}}{\pgfqpoint{2.886168in}{1.669331in}}%
\pgfpathcurveto{\pgfqpoint{2.880345in}{1.663507in}}{\pgfqpoint{2.877072in}{1.655607in}}{\pgfqpoint{2.877072in}{1.647370in}}%
\pgfpathcurveto{\pgfqpoint{2.877072in}{1.639134in}}{\pgfqpoint{2.880345in}{1.631234in}}{\pgfqpoint{2.886168in}{1.625410in}}%
\pgfpathcurveto{\pgfqpoint{2.891992in}{1.619586in}}{\pgfqpoint{2.899892in}{1.616314in}}{\pgfqpoint{2.908129in}{1.616314in}}%
\pgfpathclose%
\pgfusepath{stroke,fill}%
\end{pgfscope}%
\begin{pgfscope}%
\pgfpathrectangle{\pgfqpoint{0.100000in}{0.212622in}}{\pgfqpoint{3.696000in}{3.696000in}}%
\pgfusepath{clip}%
\pgfsetbuttcap%
\pgfsetroundjoin%
\definecolor{currentfill}{rgb}{0.121569,0.466667,0.705882}%
\pgfsetfillcolor{currentfill}%
\pgfsetfillopacity{0.859270}%
\pgfsetlinewidth{1.003750pt}%
\definecolor{currentstroke}{rgb}{0.121569,0.466667,0.705882}%
\pgfsetstrokecolor{currentstroke}%
\pgfsetstrokeopacity{0.859270}%
\pgfsetdash{}{0pt}%
\pgfpathmoveto{\pgfqpoint{2.316305in}{2.365794in}}%
\pgfpathcurveto{\pgfqpoint{2.324541in}{2.365794in}}{\pgfqpoint{2.332442in}{2.369066in}}{\pgfqpoint{2.338265in}{2.374890in}}%
\pgfpathcurveto{\pgfqpoint{2.344089in}{2.380714in}}{\pgfqpoint{2.347362in}{2.388614in}}{\pgfqpoint{2.347362in}{2.396850in}}%
\pgfpathcurveto{\pgfqpoint{2.347362in}{2.405087in}}{\pgfqpoint{2.344089in}{2.412987in}}{\pgfqpoint{2.338265in}{2.418810in}}%
\pgfpathcurveto{\pgfqpoint{2.332442in}{2.424634in}}{\pgfqpoint{2.324541in}{2.427907in}}{\pgfqpoint{2.316305in}{2.427907in}}%
\pgfpathcurveto{\pgfqpoint{2.308069in}{2.427907in}}{\pgfqpoint{2.300169in}{2.424634in}}{\pgfqpoint{2.294345in}{2.418810in}}%
\pgfpathcurveto{\pgfqpoint{2.288521in}{2.412987in}}{\pgfqpoint{2.285249in}{2.405087in}}{\pgfqpoint{2.285249in}{2.396850in}}%
\pgfpathcurveto{\pgfqpoint{2.285249in}{2.388614in}}{\pgfqpoint{2.288521in}{2.380714in}}{\pgfqpoint{2.294345in}{2.374890in}}%
\pgfpathcurveto{\pgfqpoint{2.300169in}{2.369066in}}{\pgfqpoint{2.308069in}{2.365794in}}{\pgfqpoint{2.316305in}{2.365794in}}%
\pgfpathclose%
\pgfusepath{stroke,fill}%
\end{pgfscope}%
\begin{pgfscope}%
\pgfpathrectangle{\pgfqpoint{0.100000in}{0.212622in}}{\pgfqpoint{3.696000in}{3.696000in}}%
\pgfusepath{clip}%
\pgfsetbuttcap%
\pgfsetroundjoin%
\definecolor{currentfill}{rgb}{0.121569,0.466667,0.705882}%
\pgfsetfillcolor{currentfill}%
\pgfsetfillopacity{0.859516}%
\pgfsetlinewidth{1.003750pt}%
\definecolor{currentstroke}{rgb}{0.121569,0.466667,0.705882}%
\pgfsetstrokecolor{currentstroke}%
\pgfsetstrokeopacity{0.859516}%
\pgfsetdash{}{0pt}%
\pgfpathmoveto{\pgfqpoint{1.402858in}{1.292320in}}%
\pgfpathcurveto{\pgfqpoint{1.411094in}{1.292320in}}{\pgfqpoint{1.418994in}{1.295592in}}{\pgfqpoint{1.424818in}{1.301416in}}%
\pgfpathcurveto{\pgfqpoint{1.430642in}{1.307240in}}{\pgfqpoint{1.433914in}{1.315140in}}{\pgfqpoint{1.433914in}{1.323376in}}%
\pgfpathcurveto{\pgfqpoint{1.433914in}{1.331612in}}{\pgfqpoint{1.430642in}{1.339512in}}{\pgfqpoint{1.424818in}{1.345336in}}%
\pgfpathcurveto{\pgfqpoint{1.418994in}{1.351160in}}{\pgfqpoint{1.411094in}{1.354433in}}{\pgfqpoint{1.402858in}{1.354433in}}%
\pgfpathcurveto{\pgfqpoint{1.394621in}{1.354433in}}{\pgfqpoint{1.386721in}{1.351160in}}{\pgfqpoint{1.380897in}{1.345336in}}%
\pgfpathcurveto{\pgfqpoint{1.375074in}{1.339512in}}{\pgfqpoint{1.371801in}{1.331612in}}{\pgfqpoint{1.371801in}{1.323376in}}%
\pgfpathcurveto{\pgfqpoint{1.371801in}{1.315140in}}{\pgfqpoint{1.375074in}{1.307240in}}{\pgfqpoint{1.380897in}{1.301416in}}%
\pgfpathcurveto{\pgfqpoint{1.386721in}{1.295592in}}{\pgfqpoint{1.394621in}{1.292320in}}{\pgfqpoint{1.402858in}{1.292320in}}%
\pgfpathclose%
\pgfusepath{stroke,fill}%
\end{pgfscope}%
\begin{pgfscope}%
\pgfpathrectangle{\pgfqpoint{0.100000in}{0.212622in}}{\pgfqpoint{3.696000in}{3.696000in}}%
\pgfusepath{clip}%
\pgfsetbuttcap%
\pgfsetroundjoin%
\definecolor{currentfill}{rgb}{0.121569,0.466667,0.705882}%
\pgfsetfillcolor{currentfill}%
\pgfsetfillopacity{0.859567}%
\pgfsetlinewidth{1.003750pt}%
\definecolor{currentstroke}{rgb}{0.121569,0.466667,0.705882}%
\pgfsetstrokecolor{currentstroke}%
\pgfsetstrokeopacity{0.859567}%
\pgfsetdash{}{0pt}%
\pgfpathmoveto{\pgfqpoint{1.798473in}{2.187594in}}%
\pgfpathcurveto{\pgfqpoint{1.806709in}{2.187594in}}{\pgfqpoint{1.814610in}{2.190866in}}{\pgfqpoint{1.820433in}{2.196690in}}%
\pgfpathcurveto{\pgfqpoint{1.826257in}{2.202514in}}{\pgfqpoint{1.829530in}{2.210414in}}{\pgfqpoint{1.829530in}{2.218650in}}%
\pgfpathcurveto{\pgfqpoint{1.829530in}{2.226886in}}{\pgfqpoint{1.826257in}{2.234786in}}{\pgfqpoint{1.820433in}{2.240610in}}%
\pgfpathcurveto{\pgfqpoint{1.814610in}{2.246434in}}{\pgfqpoint{1.806709in}{2.249707in}}{\pgfqpoint{1.798473in}{2.249707in}}%
\pgfpathcurveto{\pgfqpoint{1.790237in}{2.249707in}}{\pgfqpoint{1.782337in}{2.246434in}}{\pgfqpoint{1.776513in}{2.240610in}}%
\pgfpathcurveto{\pgfqpoint{1.770689in}{2.234786in}}{\pgfqpoint{1.767417in}{2.226886in}}{\pgfqpoint{1.767417in}{2.218650in}}%
\pgfpathcurveto{\pgfqpoint{1.767417in}{2.210414in}}{\pgfqpoint{1.770689in}{2.202514in}}{\pgfqpoint{1.776513in}{2.196690in}}%
\pgfpathcurveto{\pgfqpoint{1.782337in}{2.190866in}}{\pgfqpoint{1.790237in}{2.187594in}}{\pgfqpoint{1.798473in}{2.187594in}}%
\pgfpathclose%
\pgfusepath{stroke,fill}%
\end{pgfscope}%
\begin{pgfscope}%
\pgfpathrectangle{\pgfqpoint{0.100000in}{0.212622in}}{\pgfqpoint{3.696000in}{3.696000in}}%
\pgfusepath{clip}%
\pgfsetbuttcap%
\pgfsetroundjoin%
\definecolor{currentfill}{rgb}{0.121569,0.466667,0.705882}%
\pgfsetfillcolor{currentfill}%
\pgfsetfillopacity{0.860134}%
\pgfsetlinewidth{1.003750pt}%
\definecolor{currentstroke}{rgb}{0.121569,0.466667,0.705882}%
\pgfsetstrokecolor{currentstroke}%
\pgfsetstrokeopacity{0.860134}%
\pgfsetdash{}{0pt}%
\pgfpathmoveto{\pgfqpoint{2.313345in}{2.360488in}}%
\pgfpathcurveto{\pgfqpoint{2.321581in}{2.360488in}}{\pgfqpoint{2.329482in}{2.363760in}}{\pgfqpoint{2.335305in}{2.369584in}}%
\pgfpathcurveto{\pgfqpoint{2.341129in}{2.375408in}}{\pgfqpoint{2.344402in}{2.383308in}}{\pgfqpoint{2.344402in}{2.391544in}}%
\pgfpathcurveto{\pgfqpoint{2.344402in}{2.399781in}}{\pgfqpoint{2.341129in}{2.407681in}}{\pgfqpoint{2.335305in}{2.413505in}}%
\pgfpathcurveto{\pgfqpoint{2.329482in}{2.419329in}}{\pgfqpoint{2.321581in}{2.422601in}}{\pgfqpoint{2.313345in}{2.422601in}}%
\pgfpathcurveto{\pgfqpoint{2.305109in}{2.422601in}}{\pgfqpoint{2.297209in}{2.419329in}}{\pgfqpoint{2.291385in}{2.413505in}}%
\pgfpathcurveto{\pgfqpoint{2.285561in}{2.407681in}}{\pgfqpoint{2.282289in}{2.399781in}}{\pgfqpoint{2.282289in}{2.391544in}}%
\pgfpathcurveto{\pgfqpoint{2.282289in}{2.383308in}}{\pgfqpoint{2.285561in}{2.375408in}}{\pgfqpoint{2.291385in}{2.369584in}}%
\pgfpathcurveto{\pgfqpoint{2.297209in}{2.363760in}}{\pgfqpoint{2.305109in}{2.360488in}}{\pgfqpoint{2.313345in}{2.360488in}}%
\pgfpathclose%
\pgfusepath{stroke,fill}%
\end{pgfscope}%
\begin{pgfscope}%
\pgfpathrectangle{\pgfqpoint{0.100000in}{0.212622in}}{\pgfqpoint{3.696000in}{3.696000in}}%
\pgfusepath{clip}%
\pgfsetbuttcap%
\pgfsetroundjoin%
\definecolor{currentfill}{rgb}{0.121569,0.466667,0.705882}%
\pgfsetfillcolor{currentfill}%
\pgfsetfillopacity{0.860613}%
\pgfsetlinewidth{1.003750pt}%
\definecolor{currentstroke}{rgb}{0.121569,0.466667,0.705882}%
\pgfsetstrokecolor{currentstroke}%
\pgfsetstrokeopacity{0.860613}%
\pgfsetdash{}{0pt}%
\pgfpathmoveto{\pgfqpoint{1.796496in}{2.182148in}}%
\pgfpathcurveto{\pgfqpoint{1.804732in}{2.182148in}}{\pgfqpoint{1.812632in}{2.185420in}}{\pgfqpoint{1.818456in}{2.191244in}}%
\pgfpathcurveto{\pgfqpoint{1.824280in}{2.197068in}}{\pgfqpoint{1.827552in}{2.204968in}}{\pgfqpoint{1.827552in}{2.213205in}}%
\pgfpathcurveto{\pgfqpoint{1.827552in}{2.221441in}}{\pgfqpoint{1.824280in}{2.229341in}}{\pgfqpoint{1.818456in}{2.235165in}}%
\pgfpathcurveto{\pgfqpoint{1.812632in}{2.240989in}}{\pgfqpoint{1.804732in}{2.244261in}}{\pgfqpoint{1.796496in}{2.244261in}}%
\pgfpathcurveto{\pgfqpoint{1.788259in}{2.244261in}}{\pgfqpoint{1.780359in}{2.240989in}}{\pgfqpoint{1.774536in}{2.235165in}}%
\pgfpathcurveto{\pgfqpoint{1.768712in}{2.229341in}}{\pgfqpoint{1.765439in}{2.221441in}}{\pgfqpoint{1.765439in}{2.213205in}}%
\pgfpathcurveto{\pgfqpoint{1.765439in}{2.204968in}}{\pgfqpoint{1.768712in}{2.197068in}}{\pgfqpoint{1.774536in}{2.191244in}}%
\pgfpathcurveto{\pgfqpoint{1.780359in}{2.185420in}}{\pgfqpoint{1.788259in}{2.182148in}}{\pgfqpoint{1.796496in}{2.182148in}}%
\pgfpathclose%
\pgfusepath{stroke,fill}%
\end{pgfscope}%
\begin{pgfscope}%
\pgfpathrectangle{\pgfqpoint{0.100000in}{0.212622in}}{\pgfqpoint{3.696000in}{3.696000in}}%
\pgfusepath{clip}%
\pgfsetbuttcap%
\pgfsetroundjoin%
\definecolor{currentfill}{rgb}{0.121569,0.466667,0.705882}%
\pgfsetfillcolor{currentfill}%
\pgfsetfillopacity{0.860753}%
\pgfsetlinewidth{1.003750pt}%
\definecolor{currentstroke}{rgb}{0.121569,0.466667,0.705882}%
\pgfsetstrokecolor{currentstroke}%
\pgfsetstrokeopacity{0.860753}%
\pgfsetdash{}{0pt}%
\pgfpathmoveto{\pgfqpoint{2.312395in}{2.356661in}}%
\pgfpathcurveto{\pgfqpoint{2.320631in}{2.356661in}}{\pgfqpoint{2.328531in}{2.359933in}}{\pgfqpoint{2.334355in}{2.365757in}}%
\pgfpathcurveto{\pgfqpoint{2.340179in}{2.371581in}}{\pgfqpoint{2.343452in}{2.379481in}}{\pgfqpoint{2.343452in}{2.387717in}}%
\pgfpathcurveto{\pgfqpoint{2.343452in}{2.395953in}}{\pgfqpoint{2.340179in}{2.403853in}}{\pgfqpoint{2.334355in}{2.409677in}}%
\pgfpathcurveto{\pgfqpoint{2.328531in}{2.415501in}}{\pgfqpoint{2.320631in}{2.418774in}}{\pgfqpoint{2.312395in}{2.418774in}}%
\pgfpathcurveto{\pgfqpoint{2.304159in}{2.418774in}}{\pgfqpoint{2.296259in}{2.415501in}}{\pgfqpoint{2.290435in}{2.409677in}}%
\pgfpathcurveto{\pgfqpoint{2.284611in}{2.403853in}}{\pgfqpoint{2.281339in}{2.395953in}}{\pgfqpoint{2.281339in}{2.387717in}}%
\pgfpathcurveto{\pgfqpoint{2.281339in}{2.379481in}}{\pgfqpoint{2.284611in}{2.371581in}}{\pgfqpoint{2.290435in}{2.365757in}}%
\pgfpathcurveto{\pgfqpoint{2.296259in}{2.359933in}}{\pgfqpoint{2.304159in}{2.356661in}}{\pgfqpoint{2.312395in}{2.356661in}}%
\pgfpathclose%
\pgfusepath{stroke,fill}%
\end{pgfscope}%
\begin{pgfscope}%
\pgfpathrectangle{\pgfqpoint{0.100000in}{0.212622in}}{\pgfqpoint{3.696000in}{3.696000in}}%
\pgfusepath{clip}%
\pgfsetbuttcap%
\pgfsetroundjoin%
\definecolor{currentfill}{rgb}{0.121569,0.466667,0.705882}%
\pgfsetfillcolor{currentfill}%
\pgfsetfillopacity{0.860928}%
\pgfsetlinewidth{1.003750pt}%
\definecolor{currentstroke}{rgb}{0.121569,0.466667,0.705882}%
\pgfsetstrokecolor{currentstroke}%
\pgfsetstrokeopacity{0.860928}%
\pgfsetdash{}{0pt}%
\pgfpathmoveto{\pgfqpoint{1.410529in}{1.287052in}}%
\pgfpathcurveto{\pgfqpoint{1.418766in}{1.287052in}}{\pgfqpoint{1.426666in}{1.290325in}}{\pgfqpoint{1.432490in}{1.296148in}}%
\pgfpathcurveto{\pgfqpoint{1.438313in}{1.301972in}}{\pgfqpoint{1.441586in}{1.309872in}}{\pgfqpoint{1.441586in}{1.318109in}}%
\pgfpathcurveto{\pgfqpoint{1.441586in}{1.326345in}}{\pgfqpoint{1.438313in}{1.334245in}}{\pgfqpoint{1.432490in}{1.340069in}}%
\pgfpathcurveto{\pgfqpoint{1.426666in}{1.345893in}}{\pgfqpoint{1.418766in}{1.349165in}}{\pgfqpoint{1.410529in}{1.349165in}}%
\pgfpathcurveto{\pgfqpoint{1.402293in}{1.349165in}}{\pgfqpoint{1.394393in}{1.345893in}}{\pgfqpoint{1.388569in}{1.340069in}}%
\pgfpathcurveto{\pgfqpoint{1.382745in}{1.334245in}}{\pgfqpoint{1.379473in}{1.326345in}}{\pgfqpoint{1.379473in}{1.318109in}}%
\pgfpathcurveto{\pgfqpoint{1.379473in}{1.309872in}}{\pgfqpoint{1.382745in}{1.301972in}}{\pgfqpoint{1.388569in}{1.296148in}}%
\pgfpathcurveto{\pgfqpoint{1.394393in}{1.290325in}}{\pgfqpoint{1.402293in}{1.287052in}}{\pgfqpoint{1.410529in}{1.287052in}}%
\pgfpathclose%
\pgfusepath{stroke,fill}%
\end{pgfscope}%
\begin{pgfscope}%
\pgfpathrectangle{\pgfqpoint{0.100000in}{0.212622in}}{\pgfqpoint{3.696000in}{3.696000in}}%
\pgfusepath{clip}%
\pgfsetbuttcap%
\pgfsetroundjoin%
\definecolor{currentfill}{rgb}{0.121569,0.466667,0.705882}%
\pgfsetfillcolor{currentfill}%
\pgfsetfillopacity{0.861592}%
\pgfsetlinewidth{1.003750pt}%
\definecolor{currentstroke}{rgb}{0.121569,0.466667,0.705882}%
\pgfsetstrokecolor{currentstroke}%
\pgfsetstrokeopacity{0.861592}%
\pgfsetdash{}{0pt}%
\pgfpathmoveto{\pgfqpoint{1.795499in}{2.175282in}}%
\pgfpathcurveto{\pgfqpoint{1.803735in}{2.175282in}}{\pgfqpoint{1.811635in}{2.178554in}}{\pgfqpoint{1.817459in}{2.184378in}}%
\pgfpathcurveto{\pgfqpoint{1.823283in}{2.190202in}}{\pgfqpoint{1.826555in}{2.198102in}}{\pgfqpoint{1.826555in}{2.206338in}}%
\pgfpathcurveto{\pgfqpoint{1.826555in}{2.214574in}}{\pgfqpoint{1.823283in}{2.222474in}}{\pgfqpoint{1.817459in}{2.228298in}}%
\pgfpathcurveto{\pgfqpoint{1.811635in}{2.234122in}}{\pgfqpoint{1.803735in}{2.237395in}}{\pgfqpoint{1.795499in}{2.237395in}}%
\pgfpathcurveto{\pgfqpoint{1.787262in}{2.237395in}}{\pgfqpoint{1.779362in}{2.234122in}}{\pgfqpoint{1.773538in}{2.228298in}}%
\pgfpathcurveto{\pgfqpoint{1.767715in}{2.222474in}}{\pgfqpoint{1.764442in}{2.214574in}}{\pgfqpoint{1.764442in}{2.206338in}}%
\pgfpathcurveto{\pgfqpoint{1.764442in}{2.198102in}}{\pgfqpoint{1.767715in}{2.190202in}}{\pgfqpoint{1.773538in}{2.184378in}}%
\pgfpathcurveto{\pgfqpoint{1.779362in}{2.178554in}}{\pgfqpoint{1.787262in}{2.175282in}}{\pgfqpoint{1.795499in}{2.175282in}}%
\pgfpathclose%
\pgfusepath{stroke,fill}%
\end{pgfscope}%
\begin{pgfscope}%
\pgfpathrectangle{\pgfqpoint{0.100000in}{0.212622in}}{\pgfqpoint{3.696000in}{3.696000in}}%
\pgfusepath{clip}%
\pgfsetbuttcap%
\pgfsetroundjoin%
\definecolor{currentfill}{rgb}{0.121569,0.466667,0.705882}%
\pgfsetfillcolor{currentfill}%
\pgfsetfillopacity{0.861980}%
\pgfsetlinewidth{1.003750pt}%
\definecolor{currentstroke}{rgb}{0.121569,0.466667,0.705882}%
\pgfsetstrokecolor{currentstroke}%
\pgfsetstrokeopacity{0.861980}%
\pgfsetdash{}{0pt}%
\pgfpathmoveto{\pgfqpoint{2.310292in}{2.350259in}}%
\pgfpathcurveto{\pgfqpoint{2.318528in}{2.350259in}}{\pgfqpoint{2.326428in}{2.353531in}}{\pgfqpoint{2.332252in}{2.359355in}}%
\pgfpathcurveto{\pgfqpoint{2.338076in}{2.365179in}}{\pgfqpoint{2.341348in}{2.373079in}}{\pgfqpoint{2.341348in}{2.381316in}}%
\pgfpathcurveto{\pgfqpoint{2.341348in}{2.389552in}}{\pgfqpoint{2.338076in}{2.397452in}}{\pgfqpoint{2.332252in}{2.403276in}}%
\pgfpathcurveto{\pgfqpoint{2.326428in}{2.409100in}}{\pgfqpoint{2.318528in}{2.412372in}}{\pgfqpoint{2.310292in}{2.412372in}}%
\pgfpathcurveto{\pgfqpoint{2.302056in}{2.412372in}}{\pgfqpoint{2.294156in}{2.409100in}}{\pgfqpoint{2.288332in}{2.403276in}}%
\pgfpathcurveto{\pgfqpoint{2.282508in}{2.397452in}}{\pgfqpoint{2.279235in}{2.389552in}}{\pgfqpoint{2.279235in}{2.381316in}}%
\pgfpathcurveto{\pgfqpoint{2.279235in}{2.373079in}}{\pgfqpoint{2.282508in}{2.365179in}}{\pgfqpoint{2.288332in}{2.359355in}}%
\pgfpathcurveto{\pgfqpoint{2.294156in}{2.353531in}}{\pgfqpoint{2.302056in}{2.350259in}}{\pgfqpoint{2.310292in}{2.350259in}}%
\pgfpathclose%
\pgfusepath{stroke,fill}%
\end{pgfscope}%
\begin{pgfscope}%
\pgfpathrectangle{\pgfqpoint{0.100000in}{0.212622in}}{\pgfqpoint{3.696000in}{3.696000in}}%
\pgfusepath{clip}%
\pgfsetbuttcap%
\pgfsetroundjoin%
\definecolor{currentfill}{rgb}{0.121569,0.466667,0.705882}%
\pgfsetfillcolor{currentfill}%
\pgfsetfillopacity{0.862535}%
\pgfsetlinewidth{1.003750pt}%
\definecolor{currentstroke}{rgb}{0.121569,0.466667,0.705882}%
\pgfsetstrokecolor{currentstroke}%
\pgfsetstrokeopacity{0.862535}%
\pgfsetdash{}{0pt}%
\pgfpathmoveto{\pgfqpoint{2.308197in}{2.347630in}}%
\pgfpathcurveto{\pgfqpoint{2.316433in}{2.347630in}}{\pgfqpoint{2.324333in}{2.350902in}}{\pgfqpoint{2.330157in}{2.356726in}}%
\pgfpathcurveto{\pgfqpoint{2.335981in}{2.362550in}}{\pgfqpoint{2.339253in}{2.370450in}}{\pgfqpoint{2.339253in}{2.378686in}}%
\pgfpathcurveto{\pgfqpoint{2.339253in}{2.386923in}}{\pgfqpoint{2.335981in}{2.394823in}}{\pgfqpoint{2.330157in}{2.400647in}}%
\pgfpathcurveto{\pgfqpoint{2.324333in}{2.406471in}}{\pgfqpoint{2.316433in}{2.409743in}}{\pgfqpoint{2.308197in}{2.409743in}}%
\pgfpathcurveto{\pgfqpoint{2.299961in}{2.409743in}}{\pgfqpoint{2.292061in}{2.406471in}}{\pgfqpoint{2.286237in}{2.400647in}}%
\pgfpathcurveto{\pgfqpoint{2.280413in}{2.394823in}}{\pgfqpoint{2.277140in}{2.386923in}}{\pgfqpoint{2.277140in}{2.378686in}}%
\pgfpathcurveto{\pgfqpoint{2.277140in}{2.370450in}}{\pgfqpoint{2.280413in}{2.362550in}}{\pgfqpoint{2.286237in}{2.356726in}}%
\pgfpathcurveto{\pgfqpoint{2.292061in}{2.350902in}}{\pgfqpoint{2.299961in}{2.347630in}}{\pgfqpoint{2.308197in}{2.347630in}}%
\pgfpathclose%
\pgfusepath{stroke,fill}%
\end{pgfscope}%
\begin{pgfscope}%
\pgfpathrectangle{\pgfqpoint{0.100000in}{0.212622in}}{\pgfqpoint{3.696000in}{3.696000in}}%
\pgfusepath{clip}%
\pgfsetbuttcap%
\pgfsetroundjoin%
\definecolor{currentfill}{rgb}{0.121569,0.466667,0.705882}%
\pgfsetfillcolor{currentfill}%
\pgfsetfillopacity{0.862786}%
\pgfsetlinewidth{1.003750pt}%
\definecolor{currentstroke}{rgb}{0.121569,0.466667,0.705882}%
\pgfsetstrokecolor{currentstroke}%
\pgfsetstrokeopacity{0.862786}%
\pgfsetdash{}{0pt}%
\pgfpathmoveto{\pgfqpoint{1.791817in}{2.168520in}}%
\pgfpathcurveto{\pgfqpoint{1.800053in}{2.168520in}}{\pgfqpoint{1.807953in}{2.171792in}}{\pgfqpoint{1.813777in}{2.177616in}}%
\pgfpathcurveto{\pgfqpoint{1.819601in}{2.183440in}}{\pgfqpoint{1.822873in}{2.191340in}}{\pgfqpoint{1.822873in}{2.199577in}}%
\pgfpathcurveto{\pgfqpoint{1.822873in}{2.207813in}}{\pgfqpoint{1.819601in}{2.215713in}}{\pgfqpoint{1.813777in}{2.221537in}}%
\pgfpathcurveto{\pgfqpoint{1.807953in}{2.227361in}}{\pgfqpoint{1.800053in}{2.230633in}}{\pgfqpoint{1.791817in}{2.230633in}}%
\pgfpathcurveto{\pgfqpoint{1.783581in}{2.230633in}}{\pgfqpoint{1.775681in}{2.227361in}}{\pgfqpoint{1.769857in}{2.221537in}}%
\pgfpathcurveto{\pgfqpoint{1.764033in}{2.215713in}}{\pgfqpoint{1.760760in}{2.207813in}}{\pgfqpoint{1.760760in}{2.199577in}}%
\pgfpathcurveto{\pgfqpoint{1.760760in}{2.191340in}}{\pgfqpoint{1.764033in}{2.183440in}}{\pgfqpoint{1.769857in}{2.177616in}}%
\pgfpathcurveto{\pgfqpoint{1.775681in}{2.171792in}}{\pgfqpoint{1.783581in}{2.168520in}}{\pgfqpoint{1.791817in}{2.168520in}}%
\pgfpathclose%
\pgfusepath{stroke,fill}%
\end{pgfscope}%
\begin{pgfscope}%
\pgfpathrectangle{\pgfqpoint{0.100000in}{0.212622in}}{\pgfqpoint{3.696000in}{3.696000in}}%
\pgfusepath{clip}%
\pgfsetbuttcap%
\pgfsetroundjoin%
\definecolor{currentfill}{rgb}{0.121569,0.466667,0.705882}%
\pgfsetfillcolor{currentfill}%
\pgfsetfillopacity{0.862959}%
\pgfsetlinewidth{1.003750pt}%
\definecolor{currentstroke}{rgb}{0.121569,0.466667,0.705882}%
\pgfsetstrokecolor{currentstroke}%
\pgfsetstrokeopacity{0.862959}%
\pgfsetdash{}{0pt}%
\pgfpathmoveto{\pgfqpoint{1.419227in}{1.284430in}}%
\pgfpathcurveto{\pgfqpoint{1.427463in}{1.284430in}}{\pgfqpoint{1.435363in}{1.287702in}}{\pgfqpoint{1.441187in}{1.293526in}}%
\pgfpathcurveto{\pgfqpoint{1.447011in}{1.299350in}}{\pgfqpoint{1.450284in}{1.307250in}}{\pgfqpoint{1.450284in}{1.315486in}}%
\pgfpathcurveto{\pgfqpoint{1.450284in}{1.323722in}}{\pgfqpoint{1.447011in}{1.331623in}}{\pgfqpoint{1.441187in}{1.337446in}}%
\pgfpathcurveto{\pgfqpoint{1.435363in}{1.343270in}}{\pgfqpoint{1.427463in}{1.346543in}}{\pgfqpoint{1.419227in}{1.346543in}}%
\pgfpathcurveto{\pgfqpoint{1.410991in}{1.346543in}}{\pgfqpoint{1.403091in}{1.343270in}}{\pgfqpoint{1.397267in}{1.337446in}}%
\pgfpathcurveto{\pgfqpoint{1.391443in}{1.331623in}}{\pgfqpoint{1.388171in}{1.323722in}}{\pgfqpoint{1.388171in}{1.315486in}}%
\pgfpathcurveto{\pgfqpoint{1.388171in}{1.307250in}}{\pgfqpoint{1.391443in}{1.299350in}}{\pgfqpoint{1.397267in}{1.293526in}}%
\pgfpathcurveto{\pgfqpoint{1.403091in}{1.287702in}}{\pgfqpoint{1.410991in}{1.284430in}}{\pgfqpoint{1.419227in}{1.284430in}}%
\pgfpathclose%
\pgfusepath{stroke,fill}%
\end{pgfscope}%
\begin{pgfscope}%
\pgfpathrectangle{\pgfqpoint{0.100000in}{0.212622in}}{\pgfqpoint{3.696000in}{3.696000in}}%
\pgfusepath{clip}%
\pgfsetbuttcap%
\pgfsetroundjoin%
\definecolor{currentfill}{rgb}{0.121569,0.466667,0.705882}%
\pgfsetfillcolor{currentfill}%
\pgfsetfillopacity{0.863426}%
\pgfsetlinewidth{1.003750pt}%
\definecolor{currentstroke}{rgb}{0.121569,0.466667,0.705882}%
\pgfsetstrokecolor{currentstroke}%
\pgfsetstrokeopacity{0.863426}%
\pgfsetdash{}{0pt}%
\pgfpathmoveto{\pgfqpoint{2.304575in}{2.341929in}}%
\pgfpathcurveto{\pgfqpoint{2.312811in}{2.341929in}}{\pgfqpoint{2.320711in}{2.345201in}}{\pgfqpoint{2.326535in}{2.351025in}}%
\pgfpathcurveto{\pgfqpoint{2.332359in}{2.356849in}}{\pgfqpoint{2.335631in}{2.364749in}}{\pgfqpoint{2.335631in}{2.372985in}}%
\pgfpathcurveto{\pgfqpoint{2.335631in}{2.381221in}}{\pgfqpoint{2.332359in}{2.389121in}}{\pgfqpoint{2.326535in}{2.394945in}}%
\pgfpathcurveto{\pgfqpoint{2.320711in}{2.400769in}}{\pgfqpoint{2.312811in}{2.404042in}}{\pgfqpoint{2.304575in}{2.404042in}}%
\pgfpathcurveto{\pgfqpoint{2.296339in}{2.404042in}}{\pgfqpoint{2.288438in}{2.400769in}}{\pgfqpoint{2.282615in}{2.394945in}}%
\pgfpathcurveto{\pgfqpoint{2.276791in}{2.389121in}}{\pgfqpoint{2.273518in}{2.381221in}}{\pgfqpoint{2.273518in}{2.372985in}}%
\pgfpathcurveto{\pgfqpoint{2.273518in}{2.364749in}}{\pgfqpoint{2.276791in}{2.356849in}}{\pgfqpoint{2.282615in}{2.351025in}}%
\pgfpathcurveto{\pgfqpoint{2.288438in}{2.345201in}}{\pgfqpoint{2.296339in}{2.341929in}}{\pgfqpoint{2.304575in}{2.341929in}}%
\pgfpathclose%
\pgfusepath{stroke,fill}%
\end{pgfscope}%
\begin{pgfscope}%
\pgfpathrectangle{\pgfqpoint{0.100000in}{0.212622in}}{\pgfqpoint{3.696000in}{3.696000in}}%
\pgfusepath{clip}%
\pgfsetbuttcap%
\pgfsetroundjoin%
\definecolor{currentfill}{rgb}{0.121569,0.466667,0.705882}%
\pgfsetfillcolor{currentfill}%
\pgfsetfillopacity{0.863532}%
\pgfsetlinewidth{1.003750pt}%
\definecolor{currentstroke}{rgb}{0.121569,0.466667,0.705882}%
\pgfsetstrokecolor{currentstroke}%
\pgfsetstrokeopacity{0.863532}%
\pgfsetdash{}{0pt}%
\pgfpathmoveto{\pgfqpoint{1.424048in}{1.280606in}}%
\pgfpathcurveto{\pgfqpoint{1.432285in}{1.280606in}}{\pgfqpoint{1.440185in}{1.283878in}}{\pgfqpoint{1.446009in}{1.289702in}}%
\pgfpathcurveto{\pgfqpoint{1.451833in}{1.295526in}}{\pgfqpoint{1.455105in}{1.303426in}}{\pgfqpoint{1.455105in}{1.311662in}}%
\pgfpathcurveto{\pgfqpoint{1.455105in}{1.319898in}}{\pgfqpoint{1.451833in}{1.327798in}}{\pgfqpoint{1.446009in}{1.333622in}}%
\pgfpathcurveto{\pgfqpoint{1.440185in}{1.339446in}}{\pgfqpoint{1.432285in}{1.342719in}}{\pgfqpoint{1.424048in}{1.342719in}}%
\pgfpathcurveto{\pgfqpoint{1.415812in}{1.342719in}}{\pgfqpoint{1.407912in}{1.339446in}}{\pgfqpoint{1.402088in}{1.333622in}}%
\pgfpathcurveto{\pgfqpoint{1.396264in}{1.327798in}}{\pgfqpoint{1.392992in}{1.319898in}}{\pgfqpoint{1.392992in}{1.311662in}}%
\pgfpathcurveto{\pgfqpoint{1.392992in}{1.303426in}}{\pgfqpoint{1.396264in}{1.295526in}}{\pgfqpoint{1.402088in}{1.289702in}}%
\pgfpathcurveto{\pgfqpoint{1.407912in}{1.283878in}}{\pgfqpoint{1.415812in}{1.280606in}}{\pgfqpoint{1.424048in}{1.280606in}}%
\pgfpathclose%
\pgfusepath{stroke,fill}%
\end{pgfscope}%
\begin{pgfscope}%
\pgfpathrectangle{\pgfqpoint{0.100000in}{0.212622in}}{\pgfqpoint{3.696000in}{3.696000in}}%
\pgfusepath{clip}%
\pgfsetbuttcap%
\pgfsetroundjoin%
\definecolor{currentfill}{rgb}{0.121569,0.466667,0.705882}%
\pgfsetfillcolor{currentfill}%
\pgfsetfillopacity{0.864060}%
\pgfsetlinewidth{1.003750pt}%
\definecolor{currentstroke}{rgb}{0.121569,0.466667,0.705882}%
\pgfsetstrokecolor{currentstroke}%
\pgfsetstrokeopacity{0.864060}%
\pgfsetdash{}{0pt}%
\pgfpathmoveto{\pgfqpoint{2.304034in}{2.337318in}}%
\pgfpathcurveto{\pgfqpoint{2.312270in}{2.337318in}}{\pgfqpoint{2.320170in}{2.340590in}}{\pgfqpoint{2.325994in}{2.346414in}}%
\pgfpathcurveto{\pgfqpoint{2.331818in}{2.352238in}}{\pgfqpoint{2.335090in}{2.360138in}}{\pgfqpoint{2.335090in}{2.368374in}}%
\pgfpathcurveto{\pgfqpoint{2.335090in}{2.376611in}}{\pgfqpoint{2.331818in}{2.384511in}}{\pgfqpoint{2.325994in}{2.390335in}}%
\pgfpathcurveto{\pgfqpoint{2.320170in}{2.396159in}}{\pgfqpoint{2.312270in}{2.399431in}}{\pgfqpoint{2.304034in}{2.399431in}}%
\pgfpathcurveto{\pgfqpoint{2.295797in}{2.399431in}}{\pgfqpoint{2.287897in}{2.396159in}}{\pgfqpoint{2.282073in}{2.390335in}}%
\pgfpathcurveto{\pgfqpoint{2.276250in}{2.384511in}}{\pgfqpoint{2.272977in}{2.376611in}}{\pgfqpoint{2.272977in}{2.368374in}}%
\pgfpathcurveto{\pgfqpoint{2.272977in}{2.360138in}}{\pgfqpoint{2.276250in}{2.352238in}}{\pgfqpoint{2.282073in}{2.346414in}}%
\pgfpathcurveto{\pgfqpoint{2.287897in}{2.340590in}}{\pgfqpoint{2.295797in}{2.337318in}}{\pgfqpoint{2.304034in}{2.337318in}}%
\pgfpathclose%
\pgfusepath{stroke,fill}%
\end{pgfscope}%
\begin{pgfscope}%
\pgfpathrectangle{\pgfqpoint{0.100000in}{0.212622in}}{\pgfqpoint{3.696000in}{3.696000in}}%
\pgfusepath{clip}%
\pgfsetbuttcap%
\pgfsetroundjoin%
\definecolor{currentfill}{rgb}{0.121569,0.466667,0.705882}%
\pgfsetfillcolor{currentfill}%
\pgfsetfillopacity{0.864189}%
\pgfsetlinewidth{1.003750pt}%
\definecolor{currentstroke}{rgb}{0.121569,0.466667,0.705882}%
\pgfsetstrokecolor{currentstroke}%
\pgfsetstrokeopacity{0.864189}%
\pgfsetdash{}{0pt}%
\pgfpathmoveto{\pgfqpoint{1.787990in}{2.161712in}}%
\pgfpathcurveto{\pgfqpoint{1.796226in}{2.161712in}}{\pgfqpoint{1.804126in}{2.164984in}}{\pgfqpoint{1.809950in}{2.170808in}}%
\pgfpathcurveto{\pgfqpoint{1.815774in}{2.176632in}}{\pgfqpoint{1.819047in}{2.184532in}}{\pgfqpoint{1.819047in}{2.192768in}}%
\pgfpathcurveto{\pgfqpoint{1.819047in}{2.201004in}}{\pgfqpoint{1.815774in}{2.208904in}}{\pgfqpoint{1.809950in}{2.214728in}}%
\pgfpathcurveto{\pgfqpoint{1.804126in}{2.220552in}}{\pgfqpoint{1.796226in}{2.223825in}}{\pgfqpoint{1.787990in}{2.223825in}}%
\pgfpathcurveto{\pgfqpoint{1.779754in}{2.223825in}}{\pgfqpoint{1.771854in}{2.220552in}}{\pgfqpoint{1.766030in}{2.214728in}}%
\pgfpathcurveto{\pgfqpoint{1.760206in}{2.208904in}}{\pgfqpoint{1.756934in}{2.201004in}}{\pgfqpoint{1.756934in}{2.192768in}}%
\pgfpathcurveto{\pgfqpoint{1.756934in}{2.184532in}}{\pgfqpoint{1.760206in}{2.176632in}}{\pgfqpoint{1.766030in}{2.170808in}}%
\pgfpathcurveto{\pgfqpoint{1.771854in}{2.164984in}}{\pgfqpoint{1.779754in}{2.161712in}}{\pgfqpoint{1.787990in}{2.161712in}}%
\pgfpathclose%
\pgfusepath{stroke,fill}%
\end{pgfscope}%
\begin{pgfscope}%
\pgfpathrectangle{\pgfqpoint{0.100000in}{0.212622in}}{\pgfqpoint{3.696000in}{3.696000in}}%
\pgfusepath{clip}%
\pgfsetbuttcap%
\pgfsetroundjoin%
\definecolor{currentfill}{rgb}{0.121569,0.466667,0.705882}%
\pgfsetfillcolor{currentfill}%
\pgfsetfillopacity{0.864308}%
\pgfsetlinewidth{1.003750pt}%
\definecolor{currentstroke}{rgb}{0.121569,0.466667,0.705882}%
\pgfsetstrokecolor{currentstroke}%
\pgfsetstrokeopacity{0.864308}%
\pgfsetdash{}{0pt}%
\pgfpathmoveto{\pgfqpoint{1.429508in}{1.274816in}}%
\pgfpathcurveto{\pgfqpoint{1.437744in}{1.274816in}}{\pgfqpoint{1.445644in}{1.278088in}}{\pgfqpoint{1.451468in}{1.283912in}}%
\pgfpathcurveto{\pgfqpoint{1.457292in}{1.289736in}}{\pgfqpoint{1.460564in}{1.297636in}}{\pgfqpoint{1.460564in}{1.305872in}}%
\pgfpathcurveto{\pgfqpoint{1.460564in}{1.314109in}}{\pgfqpoint{1.457292in}{1.322009in}}{\pgfqpoint{1.451468in}{1.327833in}}%
\pgfpathcurveto{\pgfqpoint{1.445644in}{1.333656in}}{\pgfqpoint{1.437744in}{1.336929in}}{\pgfqpoint{1.429508in}{1.336929in}}%
\pgfpathcurveto{\pgfqpoint{1.421272in}{1.336929in}}{\pgfqpoint{1.413372in}{1.333656in}}{\pgfqpoint{1.407548in}{1.327833in}}%
\pgfpathcurveto{\pgfqpoint{1.401724in}{1.322009in}}{\pgfqpoint{1.398451in}{1.314109in}}{\pgfqpoint{1.398451in}{1.305872in}}%
\pgfpathcurveto{\pgfqpoint{1.398451in}{1.297636in}}{\pgfqpoint{1.401724in}{1.289736in}}{\pgfqpoint{1.407548in}{1.283912in}}%
\pgfpathcurveto{\pgfqpoint{1.413372in}{1.278088in}}{\pgfqpoint{1.421272in}{1.274816in}}{\pgfqpoint{1.429508in}{1.274816in}}%
\pgfpathclose%
\pgfusepath{stroke,fill}%
\end{pgfscope}%
\begin{pgfscope}%
\pgfpathrectangle{\pgfqpoint{0.100000in}{0.212622in}}{\pgfqpoint{3.696000in}{3.696000in}}%
\pgfusepath{clip}%
\pgfsetbuttcap%
\pgfsetroundjoin%
\definecolor{currentfill}{rgb}{0.121569,0.466667,0.705882}%
\pgfsetfillcolor{currentfill}%
\pgfsetfillopacity{0.865230}%
\pgfsetlinewidth{1.003750pt}%
\definecolor{currentstroke}{rgb}{0.121569,0.466667,0.705882}%
\pgfsetstrokecolor{currentstroke}%
\pgfsetstrokeopacity{0.865230}%
\pgfsetdash{}{0pt}%
\pgfpathmoveto{\pgfqpoint{2.884458in}{1.582072in}}%
\pgfpathcurveto{\pgfqpoint{2.892695in}{1.582072in}}{\pgfqpoint{2.900595in}{1.585344in}}{\pgfqpoint{2.906419in}{1.591168in}}%
\pgfpathcurveto{\pgfqpoint{2.912243in}{1.596992in}}{\pgfqpoint{2.915515in}{1.604892in}}{\pgfqpoint{2.915515in}{1.613128in}}%
\pgfpathcurveto{\pgfqpoint{2.915515in}{1.621364in}}{\pgfqpoint{2.912243in}{1.629264in}}{\pgfqpoint{2.906419in}{1.635088in}}%
\pgfpathcurveto{\pgfqpoint{2.900595in}{1.640912in}}{\pgfqpoint{2.892695in}{1.644185in}}{\pgfqpoint{2.884458in}{1.644185in}}%
\pgfpathcurveto{\pgfqpoint{2.876222in}{1.644185in}}{\pgfqpoint{2.868322in}{1.640912in}}{\pgfqpoint{2.862498in}{1.635088in}}%
\pgfpathcurveto{\pgfqpoint{2.856674in}{1.629264in}}{\pgfqpoint{2.853402in}{1.621364in}}{\pgfqpoint{2.853402in}{1.613128in}}%
\pgfpathcurveto{\pgfqpoint{2.853402in}{1.604892in}}{\pgfqpoint{2.856674in}{1.596992in}}{\pgfqpoint{2.862498in}{1.591168in}}%
\pgfpathcurveto{\pgfqpoint{2.868322in}{1.585344in}}{\pgfqpoint{2.876222in}{1.582072in}}{\pgfqpoint{2.884458in}{1.582072in}}%
\pgfpathclose%
\pgfusepath{stroke,fill}%
\end{pgfscope}%
\begin{pgfscope}%
\pgfpathrectangle{\pgfqpoint{0.100000in}{0.212622in}}{\pgfqpoint{3.696000in}{3.696000in}}%
\pgfusepath{clip}%
\pgfsetbuttcap%
\pgfsetroundjoin%
\definecolor{currentfill}{rgb}{0.121569,0.466667,0.705882}%
\pgfsetfillcolor{currentfill}%
\pgfsetfillopacity{0.865443}%
\pgfsetlinewidth{1.003750pt}%
\definecolor{currentstroke}{rgb}{0.121569,0.466667,0.705882}%
\pgfsetstrokecolor{currentstroke}%
\pgfsetstrokeopacity{0.865443}%
\pgfsetdash{}{0pt}%
\pgfpathmoveto{\pgfqpoint{2.300962in}{2.330449in}}%
\pgfpathcurveto{\pgfqpoint{2.309199in}{2.330449in}}{\pgfqpoint{2.317099in}{2.333722in}}{\pgfqpoint{2.322923in}{2.339546in}}%
\pgfpathcurveto{\pgfqpoint{2.328747in}{2.345369in}}{\pgfqpoint{2.332019in}{2.353269in}}{\pgfqpoint{2.332019in}{2.361506in}}%
\pgfpathcurveto{\pgfqpoint{2.332019in}{2.369742in}}{\pgfqpoint{2.328747in}{2.377642in}}{\pgfqpoint{2.322923in}{2.383466in}}%
\pgfpathcurveto{\pgfqpoint{2.317099in}{2.389290in}}{\pgfqpoint{2.309199in}{2.392562in}}{\pgfqpoint{2.300962in}{2.392562in}}%
\pgfpathcurveto{\pgfqpoint{2.292726in}{2.392562in}}{\pgfqpoint{2.284826in}{2.389290in}}{\pgfqpoint{2.279002in}{2.383466in}}%
\pgfpathcurveto{\pgfqpoint{2.273178in}{2.377642in}}{\pgfqpoint{2.269906in}{2.369742in}}{\pgfqpoint{2.269906in}{2.361506in}}%
\pgfpathcurveto{\pgfqpoint{2.269906in}{2.353269in}}{\pgfqpoint{2.273178in}{2.345369in}}{\pgfqpoint{2.279002in}{2.339546in}}%
\pgfpathcurveto{\pgfqpoint{2.284826in}{2.333722in}}{\pgfqpoint{2.292726in}{2.330449in}}{\pgfqpoint{2.300962in}{2.330449in}}%
\pgfpathclose%
\pgfusepath{stroke,fill}%
\end{pgfscope}%
\begin{pgfscope}%
\pgfpathrectangle{\pgfqpoint{0.100000in}{0.212622in}}{\pgfqpoint{3.696000in}{3.696000in}}%
\pgfusepath{clip}%
\pgfsetbuttcap%
\pgfsetroundjoin%
\definecolor{currentfill}{rgb}{0.121569,0.466667,0.705882}%
\pgfsetfillcolor{currentfill}%
\pgfsetfillopacity{0.865569}%
\pgfsetlinewidth{1.003750pt}%
\definecolor{currentstroke}{rgb}{0.121569,0.466667,0.705882}%
\pgfsetstrokecolor{currentstroke}%
\pgfsetstrokeopacity{0.865569}%
\pgfsetdash{}{0pt}%
\pgfpathmoveto{\pgfqpoint{1.437260in}{1.269016in}}%
\pgfpathcurveto{\pgfqpoint{1.445496in}{1.269016in}}{\pgfqpoint{1.453397in}{1.272288in}}{\pgfqpoint{1.459220in}{1.278112in}}%
\pgfpathcurveto{\pgfqpoint{1.465044in}{1.283936in}}{\pgfqpoint{1.468317in}{1.291836in}}{\pgfqpoint{1.468317in}{1.300072in}}%
\pgfpathcurveto{\pgfqpoint{1.468317in}{1.308308in}}{\pgfqpoint{1.465044in}{1.316209in}}{\pgfqpoint{1.459220in}{1.322032in}}%
\pgfpathcurveto{\pgfqpoint{1.453397in}{1.327856in}}{\pgfqpoint{1.445496in}{1.331129in}}{\pgfqpoint{1.437260in}{1.331129in}}%
\pgfpathcurveto{\pgfqpoint{1.429024in}{1.331129in}}{\pgfqpoint{1.421124in}{1.327856in}}{\pgfqpoint{1.415300in}{1.322032in}}%
\pgfpathcurveto{\pgfqpoint{1.409476in}{1.316209in}}{\pgfqpoint{1.406204in}{1.308308in}}{\pgfqpoint{1.406204in}{1.300072in}}%
\pgfpathcurveto{\pgfqpoint{1.406204in}{1.291836in}}{\pgfqpoint{1.409476in}{1.283936in}}{\pgfqpoint{1.415300in}{1.278112in}}%
\pgfpathcurveto{\pgfqpoint{1.421124in}{1.272288in}}{\pgfqpoint{1.429024in}{1.269016in}}{\pgfqpoint{1.437260in}{1.269016in}}%
\pgfpathclose%
\pgfusepath{stroke,fill}%
\end{pgfscope}%
\begin{pgfscope}%
\pgfpathrectangle{\pgfqpoint{0.100000in}{0.212622in}}{\pgfqpoint{3.696000in}{3.696000in}}%
\pgfusepath{clip}%
\pgfsetbuttcap%
\pgfsetroundjoin%
\definecolor{currentfill}{rgb}{0.121569,0.466667,0.705882}%
\pgfsetfillcolor{currentfill}%
\pgfsetfillopacity{0.865602}%
\pgfsetlinewidth{1.003750pt}%
\definecolor{currentstroke}{rgb}{0.121569,0.466667,0.705882}%
\pgfsetstrokecolor{currentstroke}%
\pgfsetstrokeopacity{0.865602}%
\pgfsetdash{}{0pt}%
\pgfpathmoveto{\pgfqpoint{1.786242in}{2.152604in}}%
\pgfpathcurveto{\pgfqpoint{1.794479in}{2.152604in}}{\pgfqpoint{1.802379in}{2.155876in}}{\pgfqpoint{1.808203in}{2.161700in}}%
\pgfpathcurveto{\pgfqpoint{1.814027in}{2.167524in}}{\pgfqpoint{1.817299in}{2.175424in}}{\pgfqpoint{1.817299in}{2.183661in}}%
\pgfpathcurveto{\pgfqpoint{1.817299in}{2.191897in}}{\pgfqpoint{1.814027in}{2.199797in}}{\pgfqpoint{1.808203in}{2.205621in}}%
\pgfpathcurveto{\pgfqpoint{1.802379in}{2.211445in}}{\pgfqpoint{1.794479in}{2.214717in}}{\pgfqpoint{1.786242in}{2.214717in}}%
\pgfpathcurveto{\pgfqpoint{1.778006in}{2.214717in}}{\pgfqpoint{1.770106in}{2.211445in}}{\pgfqpoint{1.764282in}{2.205621in}}%
\pgfpathcurveto{\pgfqpoint{1.758458in}{2.199797in}}{\pgfqpoint{1.755186in}{2.191897in}}{\pgfqpoint{1.755186in}{2.183661in}}%
\pgfpathcurveto{\pgfqpoint{1.755186in}{2.175424in}}{\pgfqpoint{1.758458in}{2.167524in}}{\pgfqpoint{1.764282in}{2.161700in}}%
\pgfpathcurveto{\pgfqpoint{1.770106in}{2.155876in}}{\pgfqpoint{1.778006in}{2.152604in}}{\pgfqpoint{1.786242in}{2.152604in}}%
\pgfpathclose%
\pgfusepath{stroke,fill}%
\end{pgfscope}%
\begin{pgfscope}%
\pgfpathrectangle{\pgfqpoint{0.100000in}{0.212622in}}{\pgfqpoint{3.696000in}{3.696000in}}%
\pgfusepath{clip}%
\pgfsetbuttcap%
\pgfsetroundjoin%
\definecolor{currentfill}{rgb}{0.121569,0.466667,0.705882}%
\pgfsetfillcolor{currentfill}%
\pgfsetfillopacity{0.866227}%
\pgfsetlinewidth{1.003750pt}%
\definecolor{currentstroke}{rgb}{0.121569,0.466667,0.705882}%
\pgfsetstrokecolor{currentstroke}%
\pgfsetstrokeopacity{0.866227}%
\pgfsetdash{}{0pt}%
\pgfpathmoveto{\pgfqpoint{2.297739in}{2.326167in}}%
\pgfpathcurveto{\pgfqpoint{2.305975in}{2.326167in}}{\pgfqpoint{2.313875in}{2.329440in}}{\pgfqpoint{2.319699in}{2.335264in}}%
\pgfpathcurveto{\pgfqpoint{2.325523in}{2.341087in}}{\pgfqpoint{2.328795in}{2.348988in}}{\pgfqpoint{2.328795in}{2.357224in}}%
\pgfpathcurveto{\pgfqpoint{2.328795in}{2.365460in}}{\pgfqpoint{2.325523in}{2.373360in}}{\pgfqpoint{2.319699in}{2.379184in}}%
\pgfpathcurveto{\pgfqpoint{2.313875in}{2.385008in}}{\pgfqpoint{2.305975in}{2.388280in}}{\pgfqpoint{2.297739in}{2.388280in}}%
\pgfpathcurveto{\pgfqpoint{2.289502in}{2.388280in}}{\pgfqpoint{2.281602in}{2.385008in}}{\pgfqpoint{2.275778in}{2.379184in}}%
\pgfpathcurveto{\pgfqpoint{2.269955in}{2.373360in}}{\pgfqpoint{2.266682in}{2.365460in}}{\pgfqpoint{2.266682in}{2.357224in}}%
\pgfpathcurveto{\pgfqpoint{2.266682in}{2.348988in}}{\pgfqpoint{2.269955in}{2.341087in}}{\pgfqpoint{2.275778in}{2.335264in}}%
\pgfpathcurveto{\pgfqpoint{2.281602in}{2.329440in}}{\pgfqpoint{2.289502in}{2.326167in}}{\pgfqpoint{2.297739in}{2.326167in}}%
\pgfpathclose%
\pgfusepath{stroke,fill}%
\end{pgfscope}%
\begin{pgfscope}%
\pgfpathrectangle{\pgfqpoint{0.100000in}{0.212622in}}{\pgfqpoint{3.696000in}{3.696000in}}%
\pgfusepath{clip}%
\pgfsetbuttcap%
\pgfsetroundjoin%
\definecolor{currentfill}{rgb}{0.121569,0.466667,0.705882}%
\pgfsetfillcolor{currentfill}%
\pgfsetfillopacity{0.866971}%
\pgfsetlinewidth{1.003750pt}%
\definecolor{currentstroke}{rgb}{0.121569,0.466667,0.705882}%
\pgfsetstrokecolor{currentstroke}%
\pgfsetstrokeopacity{0.866971}%
\pgfsetdash{}{0pt}%
\pgfpathmoveto{\pgfqpoint{2.294882in}{2.322075in}}%
\pgfpathcurveto{\pgfqpoint{2.303119in}{2.322075in}}{\pgfqpoint{2.311019in}{2.325347in}}{\pgfqpoint{2.316843in}{2.331171in}}%
\pgfpathcurveto{\pgfqpoint{2.322667in}{2.336995in}}{\pgfqpoint{2.325939in}{2.344895in}}{\pgfqpoint{2.325939in}{2.353132in}}%
\pgfpathcurveto{\pgfqpoint{2.325939in}{2.361368in}}{\pgfqpoint{2.322667in}{2.369268in}}{\pgfqpoint{2.316843in}{2.375092in}}%
\pgfpathcurveto{\pgfqpoint{2.311019in}{2.380916in}}{\pgfqpoint{2.303119in}{2.384188in}}{\pgfqpoint{2.294882in}{2.384188in}}%
\pgfpathcurveto{\pgfqpoint{2.286646in}{2.384188in}}{\pgfqpoint{2.278746in}{2.380916in}}{\pgfqpoint{2.272922in}{2.375092in}}%
\pgfpathcurveto{\pgfqpoint{2.267098in}{2.369268in}}{\pgfqpoint{2.263826in}{2.361368in}}{\pgfqpoint{2.263826in}{2.353132in}}%
\pgfpathcurveto{\pgfqpoint{2.263826in}{2.344895in}}{\pgfqpoint{2.267098in}{2.336995in}}{\pgfqpoint{2.272922in}{2.331171in}}%
\pgfpathcurveto{\pgfqpoint{2.278746in}{2.325347in}}{\pgfqpoint{2.286646in}{2.322075in}}{\pgfqpoint{2.294882in}{2.322075in}}%
\pgfpathclose%
\pgfusepath{stroke,fill}%
\end{pgfscope}%
\begin{pgfscope}%
\pgfpathrectangle{\pgfqpoint{0.100000in}{0.212622in}}{\pgfqpoint{3.696000in}{3.696000in}}%
\pgfusepath{clip}%
\pgfsetbuttcap%
\pgfsetroundjoin%
\definecolor{currentfill}{rgb}{0.121569,0.466667,0.705882}%
\pgfsetfillcolor{currentfill}%
\pgfsetfillopacity{0.867278}%
\pgfsetlinewidth{1.003750pt}%
\definecolor{currentstroke}{rgb}{0.121569,0.466667,0.705882}%
\pgfsetstrokecolor{currentstroke}%
\pgfsetstrokeopacity{0.867278}%
\pgfsetdash{}{0pt}%
\pgfpathmoveto{\pgfqpoint{1.448662in}{1.264327in}}%
\pgfpathcurveto{\pgfqpoint{1.456898in}{1.264327in}}{\pgfqpoint{1.464798in}{1.267600in}}{\pgfqpoint{1.470622in}{1.273424in}}%
\pgfpathcurveto{\pgfqpoint{1.476446in}{1.279248in}}{\pgfqpoint{1.479718in}{1.287148in}}{\pgfqpoint{1.479718in}{1.295384in}}%
\pgfpathcurveto{\pgfqpoint{1.479718in}{1.303620in}}{\pgfqpoint{1.476446in}{1.311520in}}{\pgfqpoint{1.470622in}{1.317344in}}%
\pgfpathcurveto{\pgfqpoint{1.464798in}{1.323168in}}{\pgfqpoint{1.456898in}{1.326440in}}{\pgfqpoint{1.448662in}{1.326440in}}%
\pgfpathcurveto{\pgfqpoint{1.440426in}{1.326440in}}{\pgfqpoint{1.432526in}{1.323168in}}{\pgfqpoint{1.426702in}{1.317344in}}%
\pgfpathcurveto{\pgfqpoint{1.420878in}{1.311520in}}{\pgfqpoint{1.417605in}{1.303620in}}{\pgfqpoint{1.417605in}{1.295384in}}%
\pgfpathcurveto{\pgfqpoint{1.417605in}{1.287148in}}{\pgfqpoint{1.420878in}{1.279248in}}{\pgfqpoint{1.426702in}{1.273424in}}%
\pgfpathcurveto{\pgfqpoint{1.432526in}{1.267600in}}{\pgfqpoint{1.440426in}{1.264327in}}{\pgfqpoint{1.448662in}{1.264327in}}%
\pgfpathclose%
\pgfusepath{stroke,fill}%
\end{pgfscope}%
\begin{pgfscope}%
\pgfpathrectangle{\pgfqpoint{0.100000in}{0.212622in}}{\pgfqpoint{3.696000in}{3.696000in}}%
\pgfusepath{clip}%
\pgfsetbuttcap%
\pgfsetroundjoin%
\definecolor{currentfill}{rgb}{0.121569,0.466667,0.705882}%
\pgfsetfillcolor{currentfill}%
\pgfsetfillopacity{0.867280}%
\pgfsetlinewidth{1.003750pt}%
\definecolor{currentstroke}{rgb}{0.121569,0.466667,0.705882}%
\pgfsetstrokecolor{currentstroke}%
\pgfsetstrokeopacity{0.867280}%
\pgfsetdash{}{0pt}%
\pgfpathmoveto{\pgfqpoint{1.782335in}{2.143949in}}%
\pgfpathcurveto{\pgfqpoint{1.790572in}{2.143949in}}{\pgfqpoint{1.798472in}{2.147221in}}{\pgfqpoint{1.804296in}{2.153045in}}%
\pgfpathcurveto{\pgfqpoint{1.810120in}{2.158869in}}{\pgfqpoint{1.813392in}{2.166769in}}{\pgfqpoint{1.813392in}{2.175006in}}%
\pgfpathcurveto{\pgfqpoint{1.813392in}{2.183242in}}{\pgfqpoint{1.810120in}{2.191142in}}{\pgfqpoint{1.804296in}{2.196966in}}%
\pgfpathcurveto{\pgfqpoint{1.798472in}{2.202790in}}{\pgfqpoint{1.790572in}{2.206062in}}{\pgfqpoint{1.782335in}{2.206062in}}%
\pgfpathcurveto{\pgfqpoint{1.774099in}{2.206062in}}{\pgfqpoint{1.766199in}{2.202790in}}{\pgfqpoint{1.760375in}{2.196966in}}%
\pgfpathcurveto{\pgfqpoint{1.754551in}{2.191142in}}{\pgfqpoint{1.751279in}{2.183242in}}{\pgfqpoint{1.751279in}{2.175006in}}%
\pgfpathcurveto{\pgfqpoint{1.751279in}{2.166769in}}{\pgfqpoint{1.754551in}{2.158869in}}{\pgfqpoint{1.760375in}{2.153045in}}%
\pgfpathcurveto{\pgfqpoint{1.766199in}{2.147221in}}{\pgfqpoint{1.774099in}{2.143949in}}{\pgfqpoint{1.782335in}{2.143949in}}%
\pgfpathclose%
\pgfusepath{stroke,fill}%
\end{pgfscope}%
\begin{pgfscope}%
\pgfpathrectangle{\pgfqpoint{0.100000in}{0.212622in}}{\pgfqpoint{3.696000in}{3.696000in}}%
\pgfusepath{clip}%
\pgfsetbuttcap%
\pgfsetroundjoin%
\definecolor{currentfill}{rgb}{0.121569,0.466667,0.705882}%
\pgfsetfillcolor{currentfill}%
\pgfsetfillopacity{0.867315}%
\pgfsetlinewidth{1.003750pt}%
\definecolor{currentstroke}{rgb}{0.121569,0.466667,0.705882}%
\pgfsetstrokecolor{currentstroke}%
\pgfsetstrokeopacity{0.867315}%
\pgfsetdash{}{0pt}%
\pgfpathmoveto{\pgfqpoint{2.294530in}{2.319182in}}%
\pgfpathcurveto{\pgfqpoint{2.302767in}{2.319182in}}{\pgfqpoint{2.310667in}{2.322455in}}{\pgfqpoint{2.316491in}{2.328279in}}%
\pgfpathcurveto{\pgfqpoint{2.322315in}{2.334103in}}{\pgfqpoint{2.325587in}{2.342003in}}{\pgfqpoint{2.325587in}{2.350239in}}%
\pgfpathcurveto{\pgfqpoint{2.325587in}{2.358475in}}{\pgfqpoint{2.322315in}{2.366375in}}{\pgfqpoint{2.316491in}{2.372199in}}%
\pgfpathcurveto{\pgfqpoint{2.310667in}{2.378023in}}{\pgfqpoint{2.302767in}{2.381295in}}{\pgfqpoint{2.294530in}{2.381295in}}%
\pgfpathcurveto{\pgfqpoint{2.286294in}{2.381295in}}{\pgfqpoint{2.278394in}{2.378023in}}{\pgfqpoint{2.272570in}{2.372199in}}%
\pgfpathcurveto{\pgfqpoint{2.266746in}{2.366375in}}{\pgfqpoint{2.263474in}{2.358475in}}{\pgfqpoint{2.263474in}{2.350239in}}%
\pgfpathcurveto{\pgfqpoint{2.263474in}{2.342003in}}{\pgfqpoint{2.266746in}{2.334103in}}{\pgfqpoint{2.272570in}{2.328279in}}%
\pgfpathcurveto{\pgfqpoint{2.278394in}{2.322455in}}{\pgfqpoint{2.286294in}{2.319182in}}{\pgfqpoint{2.294530in}{2.319182in}}%
\pgfpathclose%
\pgfusepath{stroke,fill}%
\end{pgfscope}%
\begin{pgfscope}%
\pgfpathrectangle{\pgfqpoint{0.100000in}{0.212622in}}{\pgfqpoint{3.696000in}{3.696000in}}%
\pgfusepath{clip}%
\pgfsetbuttcap%
\pgfsetroundjoin%
\definecolor{currentfill}{rgb}{0.121569,0.466667,0.705882}%
\pgfsetfillcolor{currentfill}%
\pgfsetfillopacity{0.867983}%
\pgfsetlinewidth{1.003750pt}%
\definecolor{currentstroke}{rgb}{0.121569,0.466667,0.705882}%
\pgfsetstrokecolor{currentstroke}%
\pgfsetstrokeopacity{0.867983}%
\pgfsetdash{}{0pt}%
\pgfpathmoveto{\pgfqpoint{1.779514in}{2.139001in}}%
\pgfpathcurveto{\pgfqpoint{1.787750in}{2.139001in}}{\pgfqpoint{1.795650in}{2.142273in}}{\pgfqpoint{1.801474in}{2.148097in}}%
\pgfpathcurveto{\pgfqpoint{1.807298in}{2.153921in}}{\pgfqpoint{1.810570in}{2.161821in}}{\pgfqpoint{1.810570in}{2.170057in}}%
\pgfpathcurveto{\pgfqpoint{1.810570in}{2.178294in}}{\pgfqpoint{1.807298in}{2.186194in}}{\pgfqpoint{1.801474in}{2.192017in}}%
\pgfpathcurveto{\pgfqpoint{1.795650in}{2.197841in}}{\pgfqpoint{1.787750in}{2.201114in}}{\pgfqpoint{1.779514in}{2.201114in}}%
\pgfpathcurveto{\pgfqpoint{1.771277in}{2.201114in}}{\pgfqpoint{1.763377in}{2.197841in}}{\pgfqpoint{1.757553in}{2.192017in}}%
\pgfpathcurveto{\pgfqpoint{1.751729in}{2.186194in}}{\pgfqpoint{1.748457in}{2.178294in}}{\pgfqpoint{1.748457in}{2.170057in}}%
\pgfpathcurveto{\pgfqpoint{1.748457in}{2.161821in}}{\pgfqpoint{1.751729in}{2.153921in}}{\pgfqpoint{1.757553in}{2.148097in}}%
\pgfpathcurveto{\pgfqpoint{1.763377in}{2.142273in}}{\pgfqpoint{1.771277in}{2.139001in}}{\pgfqpoint{1.779514in}{2.139001in}}%
\pgfpathclose%
\pgfusepath{stroke,fill}%
\end{pgfscope}%
\begin{pgfscope}%
\pgfpathrectangle{\pgfqpoint{0.100000in}{0.212622in}}{\pgfqpoint{3.696000in}{3.696000in}}%
\pgfusepath{clip}%
\pgfsetbuttcap%
\pgfsetroundjoin%
\definecolor{currentfill}{rgb}{0.121569,0.466667,0.705882}%
\pgfsetfillcolor{currentfill}%
\pgfsetfillopacity{0.868050}%
\pgfsetlinewidth{1.003750pt}%
\definecolor{currentstroke}{rgb}{0.121569,0.466667,0.705882}%
\pgfsetstrokecolor{currentstroke}%
\pgfsetstrokeopacity{0.868050}%
\pgfsetdash{}{0pt}%
\pgfpathmoveto{\pgfqpoint{2.292994in}{2.314520in}}%
\pgfpathcurveto{\pgfqpoint{2.301230in}{2.314520in}}{\pgfqpoint{2.309130in}{2.317793in}}{\pgfqpoint{2.314954in}{2.323616in}}%
\pgfpathcurveto{\pgfqpoint{2.320778in}{2.329440in}}{\pgfqpoint{2.324050in}{2.337340in}}{\pgfqpoint{2.324050in}{2.345577in}}%
\pgfpathcurveto{\pgfqpoint{2.324050in}{2.353813in}}{\pgfqpoint{2.320778in}{2.361713in}}{\pgfqpoint{2.314954in}{2.367537in}}%
\pgfpathcurveto{\pgfqpoint{2.309130in}{2.373361in}}{\pgfqpoint{2.301230in}{2.376633in}}{\pgfqpoint{2.292994in}{2.376633in}}%
\pgfpathcurveto{\pgfqpoint{2.284758in}{2.376633in}}{\pgfqpoint{2.276858in}{2.373361in}}{\pgfqpoint{2.271034in}{2.367537in}}%
\pgfpathcurveto{\pgfqpoint{2.265210in}{2.361713in}}{\pgfqpoint{2.261937in}{2.353813in}}{\pgfqpoint{2.261937in}{2.345577in}}%
\pgfpathcurveto{\pgfqpoint{2.261937in}{2.337340in}}{\pgfqpoint{2.265210in}{2.329440in}}{\pgfqpoint{2.271034in}{2.323616in}}%
\pgfpathcurveto{\pgfqpoint{2.276858in}{2.317793in}}{\pgfqpoint{2.284758in}{2.314520in}}{\pgfqpoint{2.292994in}{2.314520in}}%
\pgfpathclose%
\pgfusepath{stroke,fill}%
\end{pgfscope}%
\begin{pgfscope}%
\pgfpathrectangle{\pgfqpoint{0.100000in}{0.212622in}}{\pgfqpoint{3.696000in}{3.696000in}}%
\pgfusepath{clip}%
\pgfsetbuttcap%
\pgfsetroundjoin%
\definecolor{currentfill}{rgb}{0.121569,0.466667,0.705882}%
\pgfsetfillcolor{currentfill}%
\pgfsetfillopacity{0.868551}%
\pgfsetlinewidth{1.003750pt}%
\definecolor{currentstroke}{rgb}{0.121569,0.466667,0.705882}%
\pgfsetstrokecolor{currentstroke}%
\pgfsetstrokeopacity{0.868551}%
\pgfsetdash{}{0pt}%
\pgfpathmoveto{\pgfqpoint{2.291032in}{2.312088in}}%
\pgfpathcurveto{\pgfqpoint{2.299269in}{2.312088in}}{\pgfqpoint{2.307169in}{2.315361in}}{\pgfqpoint{2.312993in}{2.321185in}}%
\pgfpathcurveto{\pgfqpoint{2.318816in}{2.327009in}}{\pgfqpoint{2.322089in}{2.334909in}}{\pgfqpoint{2.322089in}{2.343145in}}%
\pgfpathcurveto{\pgfqpoint{2.322089in}{2.351381in}}{\pgfqpoint{2.318816in}{2.359281in}}{\pgfqpoint{2.312993in}{2.365105in}}%
\pgfpathcurveto{\pgfqpoint{2.307169in}{2.370929in}}{\pgfqpoint{2.299269in}{2.374201in}}{\pgfqpoint{2.291032in}{2.374201in}}%
\pgfpathcurveto{\pgfqpoint{2.282796in}{2.374201in}}{\pgfqpoint{2.274896in}{2.370929in}}{\pgfqpoint{2.269072in}{2.365105in}}%
\pgfpathcurveto{\pgfqpoint{2.263248in}{2.359281in}}{\pgfqpoint{2.259976in}{2.351381in}}{\pgfqpoint{2.259976in}{2.343145in}}%
\pgfpathcurveto{\pgfqpoint{2.259976in}{2.334909in}}{\pgfqpoint{2.263248in}{2.327009in}}{\pgfqpoint{2.269072in}{2.321185in}}%
\pgfpathcurveto{\pgfqpoint{2.274896in}{2.315361in}}{\pgfqpoint{2.282796in}{2.312088in}}{\pgfqpoint{2.291032in}{2.312088in}}%
\pgfpathclose%
\pgfusepath{stroke,fill}%
\end{pgfscope}%
\begin{pgfscope}%
\pgfpathrectangle{\pgfqpoint{0.100000in}{0.212622in}}{\pgfqpoint{3.696000in}{3.696000in}}%
\pgfusepath{clip}%
\pgfsetbuttcap%
\pgfsetroundjoin%
\definecolor{currentfill}{rgb}{0.121569,0.466667,0.705882}%
\pgfsetfillcolor{currentfill}%
\pgfsetfillopacity{0.869025}%
\pgfsetlinewidth{1.003750pt}%
\definecolor{currentstroke}{rgb}{0.121569,0.466667,0.705882}%
\pgfsetstrokecolor{currentstroke}%
\pgfsetstrokeopacity{0.869025}%
\pgfsetdash{}{0pt}%
\pgfpathmoveto{\pgfqpoint{1.776965in}{2.132557in}}%
\pgfpathcurveto{\pgfqpoint{1.785202in}{2.132557in}}{\pgfqpoint{1.793102in}{2.135829in}}{\pgfqpoint{1.798926in}{2.141653in}}%
\pgfpathcurveto{\pgfqpoint{1.804750in}{2.147477in}}{\pgfqpoint{1.808022in}{2.155377in}}{\pgfqpoint{1.808022in}{2.163614in}}%
\pgfpathcurveto{\pgfqpoint{1.808022in}{2.171850in}}{\pgfqpoint{1.804750in}{2.179750in}}{\pgfqpoint{1.798926in}{2.185574in}}%
\pgfpathcurveto{\pgfqpoint{1.793102in}{2.191398in}}{\pgfqpoint{1.785202in}{2.194670in}}{\pgfqpoint{1.776965in}{2.194670in}}%
\pgfpathcurveto{\pgfqpoint{1.768729in}{2.194670in}}{\pgfqpoint{1.760829in}{2.191398in}}{\pgfqpoint{1.755005in}{2.185574in}}%
\pgfpathcurveto{\pgfqpoint{1.749181in}{2.179750in}}{\pgfqpoint{1.745909in}{2.171850in}}{\pgfqpoint{1.745909in}{2.163614in}}%
\pgfpathcurveto{\pgfqpoint{1.745909in}{2.155377in}}{\pgfqpoint{1.749181in}{2.147477in}}{\pgfqpoint{1.755005in}{2.141653in}}%
\pgfpathcurveto{\pgfqpoint{1.760829in}{2.135829in}}{\pgfqpoint{1.768729in}{2.132557in}}{\pgfqpoint{1.776965in}{2.132557in}}%
\pgfpathclose%
\pgfusepath{stroke,fill}%
\end{pgfscope}%
\begin{pgfscope}%
\pgfpathrectangle{\pgfqpoint{0.100000in}{0.212622in}}{\pgfqpoint{3.696000in}{3.696000in}}%
\pgfusepath{clip}%
\pgfsetbuttcap%
\pgfsetroundjoin%
\definecolor{currentfill}{rgb}{0.121569,0.466667,0.705882}%
\pgfsetfillcolor{currentfill}%
\pgfsetfillopacity{0.869399}%
\pgfsetlinewidth{1.003750pt}%
\definecolor{currentstroke}{rgb}{0.121569,0.466667,0.705882}%
\pgfsetstrokecolor{currentstroke}%
\pgfsetstrokeopacity{0.869399}%
\pgfsetdash{}{0pt}%
\pgfpathmoveto{\pgfqpoint{2.287625in}{2.307029in}}%
\pgfpathcurveto{\pgfqpoint{2.295862in}{2.307029in}}{\pgfqpoint{2.303762in}{2.310302in}}{\pgfqpoint{2.309586in}{2.316126in}}%
\pgfpathcurveto{\pgfqpoint{2.315410in}{2.321950in}}{\pgfqpoint{2.318682in}{2.329850in}}{\pgfqpoint{2.318682in}{2.338086in}}%
\pgfpathcurveto{\pgfqpoint{2.318682in}{2.346322in}}{\pgfqpoint{2.315410in}{2.354222in}}{\pgfqpoint{2.309586in}{2.360046in}}%
\pgfpathcurveto{\pgfqpoint{2.303762in}{2.365870in}}{\pgfqpoint{2.295862in}{2.369142in}}{\pgfqpoint{2.287625in}{2.369142in}}%
\pgfpathcurveto{\pgfqpoint{2.279389in}{2.369142in}}{\pgfqpoint{2.271489in}{2.365870in}}{\pgfqpoint{2.265665in}{2.360046in}}%
\pgfpathcurveto{\pgfqpoint{2.259841in}{2.354222in}}{\pgfqpoint{2.256569in}{2.346322in}}{\pgfqpoint{2.256569in}{2.338086in}}%
\pgfpathcurveto{\pgfqpoint{2.256569in}{2.329850in}}{\pgfqpoint{2.259841in}{2.321950in}}{\pgfqpoint{2.265665in}{2.316126in}}%
\pgfpathcurveto{\pgfqpoint{2.271489in}{2.310302in}}{\pgfqpoint{2.279389in}{2.307029in}}{\pgfqpoint{2.287625in}{2.307029in}}%
\pgfpathclose%
\pgfusepath{stroke,fill}%
\end{pgfscope}%
\begin{pgfscope}%
\pgfpathrectangle{\pgfqpoint{0.100000in}{0.212622in}}{\pgfqpoint{3.696000in}{3.696000in}}%
\pgfusepath{clip}%
\pgfsetbuttcap%
\pgfsetroundjoin%
\definecolor{currentfill}{rgb}{0.121569,0.466667,0.705882}%
\pgfsetfillcolor{currentfill}%
\pgfsetfillopacity{0.869676}%
\pgfsetlinewidth{1.003750pt}%
\definecolor{currentstroke}{rgb}{0.121569,0.466667,0.705882}%
\pgfsetstrokecolor{currentstroke}%
\pgfsetstrokeopacity{0.869676}%
\pgfsetdash{}{0pt}%
\pgfpathmoveto{\pgfqpoint{1.461059in}{1.262223in}}%
\pgfpathcurveto{\pgfqpoint{1.469295in}{1.262223in}}{\pgfqpoint{1.477195in}{1.265495in}}{\pgfqpoint{1.483019in}{1.271319in}}%
\pgfpathcurveto{\pgfqpoint{1.488843in}{1.277143in}}{\pgfqpoint{1.492115in}{1.285043in}}{\pgfqpoint{1.492115in}{1.293280in}}%
\pgfpathcurveto{\pgfqpoint{1.492115in}{1.301516in}}{\pgfqpoint{1.488843in}{1.309416in}}{\pgfqpoint{1.483019in}{1.315240in}}%
\pgfpathcurveto{\pgfqpoint{1.477195in}{1.321064in}}{\pgfqpoint{1.469295in}{1.324336in}}{\pgfqpoint{1.461059in}{1.324336in}}%
\pgfpathcurveto{\pgfqpoint{1.452822in}{1.324336in}}{\pgfqpoint{1.444922in}{1.321064in}}{\pgfqpoint{1.439098in}{1.315240in}}%
\pgfpathcurveto{\pgfqpoint{1.433274in}{1.309416in}}{\pgfqpoint{1.430002in}{1.301516in}}{\pgfqpoint{1.430002in}{1.293280in}}%
\pgfpathcurveto{\pgfqpoint{1.430002in}{1.285043in}}{\pgfqpoint{1.433274in}{1.277143in}}{\pgfqpoint{1.439098in}{1.271319in}}%
\pgfpathcurveto{\pgfqpoint{1.444922in}{1.265495in}}{\pgfqpoint{1.452822in}{1.262223in}}{\pgfqpoint{1.461059in}{1.262223in}}%
\pgfpathclose%
\pgfusepath{stroke,fill}%
\end{pgfscope}%
\begin{pgfscope}%
\pgfpathrectangle{\pgfqpoint{0.100000in}{0.212622in}}{\pgfqpoint{3.696000in}{3.696000in}}%
\pgfusepath{clip}%
\pgfsetbuttcap%
\pgfsetroundjoin%
\definecolor{currentfill}{rgb}{0.121569,0.466667,0.705882}%
\pgfsetfillcolor{currentfill}%
\pgfsetfillopacity{0.869988}%
\pgfsetlinewidth{1.003750pt}%
\definecolor{currentstroke}{rgb}{0.121569,0.466667,0.705882}%
\pgfsetstrokecolor{currentstroke}%
\pgfsetstrokeopacity{0.869988}%
\pgfsetdash{}{0pt}%
\pgfpathmoveto{\pgfqpoint{2.286622in}{2.303028in}}%
\pgfpathcurveto{\pgfqpoint{2.294859in}{2.303028in}}{\pgfqpoint{2.302759in}{2.306301in}}{\pgfqpoint{2.308583in}{2.312124in}}%
\pgfpathcurveto{\pgfqpoint{2.314407in}{2.317948in}}{\pgfqpoint{2.317679in}{2.325848in}}{\pgfqpoint{2.317679in}{2.334085in}}%
\pgfpathcurveto{\pgfqpoint{2.317679in}{2.342321in}}{\pgfqpoint{2.314407in}{2.350221in}}{\pgfqpoint{2.308583in}{2.356045in}}%
\pgfpathcurveto{\pgfqpoint{2.302759in}{2.361869in}}{\pgfqpoint{2.294859in}{2.365141in}}{\pgfqpoint{2.286622in}{2.365141in}}%
\pgfpathcurveto{\pgfqpoint{2.278386in}{2.365141in}}{\pgfqpoint{2.270486in}{2.361869in}}{\pgfqpoint{2.264662in}{2.356045in}}%
\pgfpathcurveto{\pgfqpoint{2.258838in}{2.350221in}}{\pgfqpoint{2.255566in}{2.342321in}}{\pgfqpoint{2.255566in}{2.334085in}}%
\pgfpathcurveto{\pgfqpoint{2.255566in}{2.325848in}}{\pgfqpoint{2.258838in}{2.317948in}}{\pgfqpoint{2.264662in}{2.312124in}}%
\pgfpathcurveto{\pgfqpoint{2.270486in}{2.306301in}}{\pgfqpoint{2.278386in}{2.303028in}}{\pgfqpoint{2.286622in}{2.303028in}}%
\pgfpathclose%
\pgfusepath{stroke,fill}%
\end{pgfscope}%
\begin{pgfscope}%
\pgfpathrectangle{\pgfqpoint{0.100000in}{0.212622in}}{\pgfqpoint{3.696000in}{3.696000in}}%
\pgfusepath{clip}%
\pgfsetbuttcap%
\pgfsetroundjoin%
\definecolor{currentfill}{rgb}{0.121569,0.466667,0.705882}%
\pgfsetfillcolor{currentfill}%
\pgfsetfillopacity{0.870116}%
\pgfsetlinewidth{1.003750pt}%
\definecolor{currentstroke}{rgb}{0.121569,0.466667,0.705882}%
\pgfsetstrokecolor{currentstroke}%
\pgfsetstrokeopacity{0.870116}%
\pgfsetdash{}{0pt}%
\pgfpathmoveto{\pgfqpoint{1.774312in}{2.125295in}}%
\pgfpathcurveto{\pgfqpoint{1.782548in}{2.125295in}}{\pgfqpoint{1.790448in}{2.128567in}}{\pgfqpoint{1.796272in}{2.134391in}}%
\pgfpathcurveto{\pgfqpoint{1.802096in}{2.140215in}}{\pgfqpoint{1.805368in}{2.148115in}}{\pgfqpoint{1.805368in}{2.156352in}}%
\pgfpathcurveto{\pgfqpoint{1.805368in}{2.164588in}}{\pgfqpoint{1.802096in}{2.172488in}}{\pgfqpoint{1.796272in}{2.178312in}}%
\pgfpathcurveto{\pgfqpoint{1.790448in}{2.184136in}}{\pgfqpoint{1.782548in}{2.187408in}}{\pgfqpoint{1.774312in}{2.187408in}}%
\pgfpathcurveto{\pgfqpoint{1.766075in}{2.187408in}}{\pgfqpoint{1.758175in}{2.184136in}}{\pgfqpoint{1.752351in}{2.178312in}}%
\pgfpathcurveto{\pgfqpoint{1.746527in}{2.172488in}}{\pgfqpoint{1.743255in}{2.164588in}}{\pgfqpoint{1.743255in}{2.156352in}}%
\pgfpathcurveto{\pgfqpoint{1.743255in}{2.148115in}}{\pgfqpoint{1.746527in}{2.140215in}}{\pgfqpoint{1.752351in}{2.134391in}}%
\pgfpathcurveto{\pgfqpoint{1.758175in}{2.128567in}}{\pgfqpoint{1.766075in}{2.125295in}}{\pgfqpoint{1.774312in}{2.125295in}}%
\pgfpathclose%
\pgfusepath{stroke,fill}%
\end{pgfscope}%
\begin{pgfscope}%
\pgfpathrectangle{\pgfqpoint{0.100000in}{0.212622in}}{\pgfqpoint{3.696000in}{3.696000in}}%
\pgfusepath{clip}%
\pgfsetbuttcap%
\pgfsetroundjoin%
\definecolor{currentfill}{rgb}{0.121569,0.466667,0.705882}%
\pgfsetfillcolor{currentfill}%
\pgfsetfillopacity{0.870363}%
\pgfsetlinewidth{1.003750pt}%
\definecolor{currentstroke}{rgb}{0.121569,0.466667,0.705882}%
\pgfsetstrokecolor{currentstroke}%
\pgfsetstrokeopacity{0.870363}%
\pgfsetdash{}{0pt}%
\pgfpathmoveto{\pgfqpoint{1.467271in}{1.256119in}}%
\pgfpathcurveto{\pgfqpoint{1.475508in}{1.256119in}}{\pgfqpoint{1.483408in}{1.259391in}}{\pgfqpoint{1.489232in}{1.265215in}}%
\pgfpathcurveto{\pgfqpoint{1.495056in}{1.271039in}}{\pgfqpoint{1.498328in}{1.278939in}}{\pgfqpoint{1.498328in}{1.287176in}}%
\pgfpathcurveto{\pgfqpoint{1.498328in}{1.295412in}}{\pgfqpoint{1.495056in}{1.303312in}}{\pgfqpoint{1.489232in}{1.309136in}}%
\pgfpathcurveto{\pgfqpoint{1.483408in}{1.314960in}}{\pgfqpoint{1.475508in}{1.318232in}}{\pgfqpoint{1.467271in}{1.318232in}}%
\pgfpathcurveto{\pgfqpoint{1.459035in}{1.318232in}}{\pgfqpoint{1.451135in}{1.314960in}}{\pgfqpoint{1.445311in}{1.309136in}}%
\pgfpathcurveto{\pgfqpoint{1.439487in}{1.303312in}}{\pgfqpoint{1.436215in}{1.295412in}}{\pgfqpoint{1.436215in}{1.287176in}}%
\pgfpathcurveto{\pgfqpoint{1.436215in}{1.278939in}}{\pgfqpoint{1.439487in}{1.271039in}}{\pgfqpoint{1.445311in}{1.265215in}}%
\pgfpathcurveto{\pgfqpoint{1.451135in}{1.259391in}}{\pgfqpoint{1.459035in}{1.256119in}}{\pgfqpoint{1.467271in}{1.256119in}}%
\pgfpathclose%
\pgfusepath{stroke,fill}%
\end{pgfscope}%
\begin{pgfscope}%
\pgfpathrectangle{\pgfqpoint{0.100000in}{0.212622in}}{\pgfqpoint{3.696000in}{3.696000in}}%
\pgfusepath{clip}%
\pgfsetbuttcap%
\pgfsetroundjoin%
\definecolor{currentfill}{rgb}{0.121569,0.466667,0.705882}%
\pgfsetfillcolor{currentfill}%
\pgfsetfillopacity{0.871077}%
\pgfsetlinewidth{1.003750pt}%
\definecolor{currentstroke}{rgb}{0.121569,0.466667,0.705882}%
\pgfsetstrokecolor{currentstroke}%
\pgfsetstrokeopacity{0.871077}%
\pgfsetdash{}{0pt}%
\pgfpathmoveto{\pgfqpoint{2.283865in}{2.296298in}}%
\pgfpathcurveto{\pgfqpoint{2.292101in}{2.296298in}}{\pgfqpoint{2.300001in}{2.299570in}}{\pgfqpoint{2.305825in}{2.305394in}}%
\pgfpathcurveto{\pgfqpoint{2.311649in}{2.311218in}}{\pgfqpoint{2.314921in}{2.319118in}}{\pgfqpoint{2.314921in}{2.327354in}}%
\pgfpathcurveto{\pgfqpoint{2.314921in}{2.335591in}}{\pgfqpoint{2.311649in}{2.343491in}}{\pgfqpoint{2.305825in}{2.349315in}}%
\pgfpathcurveto{\pgfqpoint{2.300001in}{2.355138in}}{\pgfqpoint{2.292101in}{2.358411in}}{\pgfqpoint{2.283865in}{2.358411in}}%
\pgfpathcurveto{\pgfqpoint{2.275629in}{2.358411in}}{\pgfqpoint{2.267728in}{2.355138in}}{\pgfqpoint{2.261905in}{2.349315in}}%
\pgfpathcurveto{\pgfqpoint{2.256081in}{2.343491in}}{\pgfqpoint{2.252808in}{2.335591in}}{\pgfqpoint{2.252808in}{2.327354in}}%
\pgfpathcurveto{\pgfqpoint{2.252808in}{2.319118in}}{\pgfqpoint{2.256081in}{2.311218in}}{\pgfqpoint{2.261905in}{2.305394in}}%
\pgfpathcurveto{\pgfqpoint{2.267728in}{2.299570in}}{\pgfqpoint{2.275629in}{2.296298in}}{\pgfqpoint{2.283865in}{2.296298in}}%
\pgfpathclose%
\pgfusepath{stroke,fill}%
\end{pgfscope}%
\begin{pgfscope}%
\pgfpathrectangle{\pgfqpoint{0.100000in}{0.212622in}}{\pgfqpoint{3.696000in}{3.696000in}}%
\pgfusepath{clip}%
\pgfsetbuttcap%
\pgfsetroundjoin%
\definecolor{currentfill}{rgb}{0.121569,0.466667,0.705882}%
\pgfsetfillcolor{currentfill}%
\pgfsetfillopacity{0.871151}%
\pgfsetlinewidth{1.003750pt}%
\definecolor{currentstroke}{rgb}{0.121569,0.466667,0.705882}%
\pgfsetstrokecolor{currentstroke}%
\pgfsetstrokeopacity{0.871151}%
\pgfsetdash{}{0pt}%
\pgfpathmoveto{\pgfqpoint{1.769968in}{2.118133in}}%
\pgfpathcurveto{\pgfqpoint{1.778205in}{2.118133in}}{\pgfqpoint{1.786105in}{2.121405in}}{\pgfqpoint{1.791929in}{2.127229in}}%
\pgfpathcurveto{\pgfqpoint{1.797753in}{2.133053in}}{\pgfqpoint{1.801025in}{2.140953in}}{\pgfqpoint{1.801025in}{2.149189in}}%
\pgfpathcurveto{\pgfqpoint{1.801025in}{2.157426in}}{\pgfqpoint{1.797753in}{2.165326in}}{\pgfqpoint{1.791929in}{2.171150in}}%
\pgfpathcurveto{\pgfqpoint{1.786105in}{2.176973in}}{\pgfqpoint{1.778205in}{2.180246in}}{\pgfqpoint{1.769968in}{2.180246in}}%
\pgfpathcurveto{\pgfqpoint{1.761732in}{2.180246in}}{\pgfqpoint{1.753832in}{2.176973in}}{\pgfqpoint{1.748008in}{2.171150in}}%
\pgfpathcurveto{\pgfqpoint{1.742184in}{2.165326in}}{\pgfqpoint{1.738912in}{2.157426in}}{\pgfqpoint{1.738912in}{2.149189in}}%
\pgfpathcurveto{\pgfqpoint{1.738912in}{2.140953in}}{\pgfqpoint{1.742184in}{2.133053in}}{\pgfqpoint{1.748008in}{2.127229in}}%
\pgfpathcurveto{\pgfqpoint{1.753832in}{2.121405in}}{\pgfqpoint{1.761732in}{2.118133in}}{\pgfqpoint{1.769968in}{2.118133in}}%
\pgfpathclose%
\pgfusepath{stroke,fill}%
\end{pgfscope}%
\begin{pgfscope}%
\pgfpathrectangle{\pgfqpoint{0.100000in}{0.212622in}}{\pgfqpoint{3.696000in}{3.696000in}}%
\pgfusepath{clip}%
\pgfsetbuttcap%
\pgfsetroundjoin%
\definecolor{currentfill}{rgb}{0.121569,0.466667,0.705882}%
\pgfsetfillcolor{currentfill}%
\pgfsetfillopacity{0.871331}%
\pgfsetlinewidth{1.003750pt}%
\definecolor{currentstroke}{rgb}{0.121569,0.466667,0.705882}%
\pgfsetstrokecolor{currentstroke}%
\pgfsetstrokeopacity{0.871331}%
\pgfsetdash{}{0pt}%
\pgfpathmoveto{\pgfqpoint{1.473796in}{1.248579in}}%
\pgfpathcurveto{\pgfqpoint{1.482032in}{1.248579in}}{\pgfqpoint{1.489932in}{1.251852in}}{\pgfqpoint{1.495756in}{1.257676in}}%
\pgfpathcurveto{\pgfqpoint{1.501580in}{1.263499in}}{\pgfqpoint{1.504852in}{1.271400in}}{\pgfqpoint{1.504852in}{1.279636in}}%
\pgfpathcurveto{\pgfqpoint{1.504852in}{1.287872in}}{\pgfqpoint{1.501580in}{1.295772in}}{\pgfqpoint{1.495756in}{1.301596in}}%
\pgfpathcurveto{\pgfqpoint{1.489932in}{1.307420in}}{\pgfqpoint{1.482032in}{1.310692in}}{\pgfqpoint{1.473796in}{1.310692in}}%
\pgfpathcurveto{\pgfqpoint{1.465559in}{1.310692in}}{\pgfqpoint{1.457659in}{1.307420in}}{\pgfqpoint{1.451835in}{1.301596in}}%
\pgfpathcurveto{\pgfqpoint{1.446012in}{1.295772in}}{\pgfqpoint{1.442739in}{1.287872in}}{\pgfqpoint{1.442739in}{1.279636in}}%
\pgfpathcurveto{\pgfqpoint{1.442739in}{1.271400in}}{\pgfqpoint{1.446012in}{1.263499in}}{\pgfqpoint{1.451835in}{1.257676in}}%
\pgfpathcurveto{\pgfqpoint{1.457659in}{1.251852in}}{\pgfqpoint{1.465559in}{1.248579in}}{\pgfqpoint{1.473796in}{1.248579in}}%
\pgfpathclose%
\pgfusepath{stroke,fill}%
\end{pgfscope}%
\begin{pgfscope}%
\pgfpathrectangle{\pgfqpoint{0.100000in}{0.212622in}}{\pgfqpoint{3.696000in}{3.696000in}}%
\pgfusepath{clip}%
\pgfsetbuttcap%
\pgfsetroundjoin%
\definecolor{currentfill}{rgb}{0.121569,0.466667,0.705882}%
\pgfsetfillcolor{currentfill}%
\pgfsetfillopacity{0.871692}%
\pgfsetlinewidth{1.003750pt}%
\definecolor{currentstroke}{rgb}{0.121569,0.466667,0.705882}%
\pgfsetstrokecolor{currentstroke}%
\pgfsetstrokeopacity{0.871692}%
\pgfsetdash{}{0pt}%
\pgfpathmoveto{\pgfqpoint{2.862691in}{1.547779in}}%
\pgfpathcurveto{\pgfqpoint{2.870927in}{1.547779in}}{\pgfqpoint{2.878827in}{1.551051in}}{\pgfqpoint{2.884651in}{1.556875in}}%
\pgfpathcurveto{\pgfqpoint{2.890475in}{1.562699in}}{\pgfqpoint{2.893747in}{1.570599in}}{\pgfqpoint{2.893747in}{1.578836in}}%
\pgfpathcurveto{\pgfqpoint{2.893747in}{1.587072in}}{\pgfqpoint{2.890475in}{1.594972in}}{\pgfqpoint{2.884651in}{1.600796in}}%
\pgfpathcurveto{\pgfqpoint{2.878827in}{1.606620in}}{\pgfqpoint{2.870927in}{1.609892in}}{\pgfqpoint{2.862691in}{1.609892in}}%
\pgfpathcurveto{\pgfqpoint{2.854455in}{1.609892in}}{\pgfqpoint{2.846555in}{1.606620in}}{\pgfqpoint{2.840731in}{1.600796in}}%
\pgfpathcurveto{\pgfqpoint{2.834907in}{1.594972in}}{\pgfqpoint{2.831634in}{1.587072in}}{\pgfqpoint{2.831634in}{1.578836in}}%
\pgfpathcurveto{\pgfqpoint{2.831634in}{1.570599in}}{\pgfqpoint{2.834907in}{1.562699in}}{\pgfqpoint{2.840731in}{1.556875in}}%
\pgfpathcurveto{\pgfqpoint{2.846555in}{1.551051in}}{\pgfqpoint{2.854455in}{1.547779in}}{\pgfqpoint{2.862691in}{1.547779in}}%
\pgfpathclose%
\pgfusepath{stroke,fill}%
\end{pgfscope}%
\begin{pgfscope}%
\pgfpathrectangle{\pgfqpoint{0.100000in}{0.212622in}}{\pgfqpoint{3.696000in}{3.696000in}}%
\pgfusepath{clip}%
\pgfsetbuttcap%
\pgfsetroundjoin%
\definecolor{currentfill}{rgb}{0.121569,0.466667,0.705882}%
\pgfsetfillcolor{currentfill}%
\pgfsetfillopacity{0.871743}%
\pgfsetlinewidth{1.003750pt}%
\definecolor{currentstroke}{rgb}{0.121569,0.466667,0.705882}%
\pgfsetstrokecolor{currentstroke}%
\pgfsetstrokeopacity{0.871743}%
\pgfsetdash{}{0pt}%
\pgfpathmoveto{\pgfqpoint{2.281212in}{2.292575in}}%
\pgfpathcurveto{\pgfqpoint{2.289448in}{2.292575in}}{\pgfqpoint{2.297348in}{2.295847in}}{\pgfqpoint{2.303172in}{2.301671in}}%
\pgfpathcurveto{\pgfqpoint{2.308996in}{2.307495in}}{\pgfqpoint{2.312268in}{2.315395in}}{\pgfqpoint{2.312268in}{2.323632in}}%
\pgfpathcurveto{\pgfqpoint{2.312268in}{2.331868in}}{\pgfqpoint{2.308996in}{2.339768in}}{\pgfqpoint{2.303172in}{2.345592in}}%
\pgfpathcurveto{\pgfqpoint{2.297348in}{2.351416in}}{\pgfqpoint{2.289448in}{2.354688in}}{\pgfqpoint{2.281212in}{2.354688in}}%
\pgfpathcurveto{\pgfqpoint{2.272975in}{2.354688in}}{\pgfqpoint{2.265075in}{2.351416in}}{\pgfqpoint{2.259251in}{2.345592in}}%
\pgfpathcurveto{\pgfqpoint{2.253428in}{2.339768in}}{\pgfqpoint{2.250155in}{2.331868in}}{\pgfqpoint{2.250155in}{2.323632in}}%
\pgfpathcurveto{\pgfqpoint{2.250155in}{2.315395in}}{\pgfqpoint{2.253428in}{2.307495in}}{\pgfqpoint{2.259251in}{2.301671in}}%
\pgfpathcurveto{\pgfqpoint{2.265075in}{2.295847in}}{\pgfqpoint{2.272975in}{2.292575in}}{\pgfqpoint{2.281212in}{2.292575in}}%
\pgfpathclose%
\pgfusepath{stroke,fill}%
\end{pgfscope}%
\begin{pgfscope}%
\pgfpathrectangle{\pgfqpoint{0.100000in}{0.212622in}}{\pgfqpoint{3.696000in}{3.696000in}}%
\pgfusepath{clip}%
\pgfsetbuttcap%
\pgfsetroundjoin%
\definecolor{currentfill}{rgb}{0.121569,0.466667,0.705882}%
\pgfsetfillcolor{currentfill}%
\pgfsetfillopacity{0.871848}%
\pgfsetlinewidth{1.003750pt}%
\definecolor{currentstroke}{rgb}{0.121569,0.466667,0.705882}%
\pgfsetstrokecolor{currentstroke}%
\pgfsetstrokeopacity{0.871848}%
\pgfsetdash{}{0pt}%
\pgfpathmoveto{\pgfqpoint{1.767973in}{2.114200in}}%
\pgfpathcurveto{\pgfqpoint{1.776210in}{2.114200in}}{\pgfqpoint{1.784110in}{2.117473in}}{\pgfqpoint{1.789934in}{2.123297in}}%
\pgfpathcurveto{\pgfqpoint{1.795757in}{2.129120in}}{\pgfqpoint{1.799030in}{2.137021in}}{\pgfqpoint{1.799030in}{2.145257in}}%
\pgfpathcurveto{\pgfqpoint{1.799030in}{2.153493in}}{\pgfqpoint{1.795757in}{2.161393in}}{\pgfqpoint{1.789934in}{2.167217in}}%
\pgfpathcurveto{\pgfqpoint{1.784110in}{2.173041in}}{\pgfqpoint{1.776210in}{2.176313in}}{\pgfqpoint{1.767973in}{2.176313in}}%
\pgfpathcurveto{\pgfqpoint{1.759737in}{2.176313in}}{\pgfqpoint{1.751837in}{2.173041in}}{\pgfqpoint{1.746013in}{2.167217in}}%
\pgfpathcurveto{\pgfqpoint{1.740189in}{2.161393in}}{\pgfqpoint{1.736917in}{2.153493in}}{\pgfqpoint{1.736917in}{2.145257in}}%
\pgfpathcurveto{\pgfqpoint{1.736917in}{2.137021in}}{\pgfqpoint{1.740189in}{2.129120in}}{\pgfqpoint{1.746013in}{2.123297in}}%
\pgfpathcurveto{\pgfqpoint{1.751837in}{2.117473in}}{\pgfqpoint{1.759737in}{2.114200in}}{\pgfqpoint{1.767973in}{2.114200in}}%
\pgfpathclose%
\pgfusepath{stroke,fill}%
\end{pgfscope}%
\begin{pgfscope}%
\pgfpathrectangle{\pgfqpoint{0.100000in}{0.212622in}}{\pgfqpoint{3.696000in}{3.696000in}}%
\pgfusepath{clip}%
\pgfsetbuttcap%
\pgfsetroundjoin%
\definecolor{currentfill}{rgb}{0.121569,0.466667,0.705882}%
\pgfsetfillcolor{currentfill}%
\pgfsetfillopacity{0.872333}%
\pgfsetlinewidth{1.003750pt}%
\definecolor{currentstroke}{rgb}{0.121569,0.466667,0.705882}%
\pgfsetstrokecolor{currentstroke}%
\pgfsetstrokeopacity{0.872333}%
\pgfsetdash{}{0pt}%
\pgfpathmoveto{\pgfqpoint{2.278875in}{2.288937in}}%
\pgfpathcurveto{\pgfqpoint{2.287112in}{2.288937in}}{\pgfqpoint{2.295012in}{2.292209in}}{\pgfqpoint{2.300836in}{2.298033in}}%
\pgfpathcurveto{\pgfqpoint{2.306660in}{2.303857in}}{\pgfqpoint{2.309932in}{2.311757in}}{\pgfqpoint{2.309932in}{2.319994in}}%
\pgfpathcurveto{\pgfqpoint{2.309932in}{2.328230in}}{\pgfqpoint{2.306660in}{2.336130in}}{\pgfqpoint{2.300836in}{2.341954in}}%
\pgfpathcurveto{\pgfqpoint{2.295012in}{2.347778in}}{\pgfqpoint{2.287112in}{2.351050in}}{\pgfqpoint{2.278875in}{2.351050in}}%
\pgfpathcurveto{\pgfqpoint{2.270639in}{2.351050in}}{\pgfqpoint{2.262739in}{2.347778in}}{\pgfqpoint{2.256915in}{2.341954in}}%
\pgfpathcurveto{\pgfqpoint{2.251091in}{2.336130in}}{\pgfqpoint{2.247819in}{2.328230in}}{\pgfqpoint{2.247819in}{2.319994in}}%
\pgfpathcurveto{\pgfqpoint{2.247819in}{2.311757in}}{\pgfqpoint{2.251091in}{2.303857in}}{\pgfqpoint{2.256915in}{2.298033in}}%
\pgfpathcurveto{\pgfqpoint{2.262739in}{2.292209in}}{\pgfqpoint{2.270639in}{2.288937in}}{\pgfqpoint{2.278875in}{2.288937in}}%
\pgfpathclose%
\pgfusepath{stroke,fill}%
\end{pgfscope}%
\begin{pgfscope}%
\pgfpathrectangle{\pgfqpoint{0.100000in}{0.212622in}}{\pgfqpoint{3.696000in}{3.696000in}}%
\pgfusepath{clip}%
\pgfsetbuttcap%
\pgfsetroundjoin%
\definecolor{currentfill}{rgb}{0.121569,0.466667,0.705882}%
\pgfsetfillcolor{currentfill}%
\pgfsetfillopacity{0.872357}%
\pgfsetlinewidth{1.003750pt}%
\definecolor{currentstroke}{rgb}{0.121569,0.466667,0.705882}%
\pgfsetstrokecolor{currentstroke}%
\pgfsetstrokeopacity{0.872357}%
\pgfsetdash{}{0pt}%
\pgfpathmoveto{\pgfqpoint{1.483739in}{1.241998in}}%
\pgfpathcurveto{\pgfqpoint{1.491975in}{1.241998in}}{\pgfqpoint{1.499875in}{1.245270in}}{\pgfqpoint{1.505699in}{1.251094in}}%
\pgfpathcurveto{\pgfqpoint{1.511523in}{1.256918in}}{\pgfqpoint{1.514795in}{1.264818in}}{\pgfqpoint{1.514795in}{1.273054in}}%
\pgfpathcurveto{\pgfqpoint{1.514795in}{1.281291in}}{\pgfqpoint{1.511523in}{1.289191in}}{\pgfqpoint{1.505699in}{1.295015in}}%
\pgfpathcurveto{\pgfqpoint{1.499875in}{1.300839in}}{\pgfqpoint{1.491975in}{1.304111in}}{\pgfqpoint{1.483739in}{1.304111in}}%
\pgfpathcurveto{\pgfqpoint{1.475502in}{1.304111in}}{\pgfqpoint{1.467602in}{1.300839in}}{\pgfqpoint{1.461778in}{1.295015in}}%
\pgfpathcurveto{\pgfqpoint{1.455954in}{1.289191in}}{\pgfqpoint{1.452682in}{1.281291in}}{\pgfqpoint{1.452682in}{1.273054in}}%
\pgfpathcurveto{\pgfqpoint{1.452682in}{1.264818in}}{\pgfqpoint{1.455954in}{1.256918in}}{\pgfqpoint{1.461778in}{1.251094in}}%
\pgfpathcurveto{\pgfqpoint{1.467602in}{1.245270in}}{\pgfqpoint{1.475502in}{1.241998in}}{\pgfqpoint{1.483739in}{1.241998in}}%
\pgfpathclose%
\pgfusepath{stroke,fill}%
\end{pgfscope}%
\begin{pgfscope}%
\pgfpathrectangle{\pgfqpoint{0.100000in}{0.212622in}}{\pgfqpoint{3.696000in}{3.696000in}}%
\pgfusepath{clip}%
\pgfsetbuttcap%
\pgfsetroundjoin%
\definecolor{currentfill}{rgb}{0.121569,0.466667,0.705882}%
\pgfsetfillcolor{currentfill}%
\pgfsetfillopacity{0.872581}%
\pgfsetlinewidth{1.003750pt}%
\definecolor{currentstroke}{rgb}{0.121569,0.466667,0.705882}%
\pgfsetstrokecolor{currentstroke}%
\pgfsetstrokeopacity{0.872581}%
\pgfsetdash{}{0pt}%
\pgfpathmoveto{\pgfqpoint{1.766491in}{2.108861in}}%
\pgfpathcurveto{\pgfqpoint{1.774728in}{2.108861in}}{\pgfqpoint{1.782628in}{2.112133in}}{\pgfqpoint{1.788452in}{2.117957in}}%
\pgfpathcurveto{\pgfqpoint{1.794275in}{2.123781in}}{\pgfqpoint{1.797548in}{2.131681in}}{\pgfqpoint{1.797548in}{2.139917in}}%
\pgfpathcurveto{\pgfqpoint{1.797548in}{2.148154in}}{\pgfqpoint{1.794275in}{2.156054in}}{\pgfqpoint{1.788452in}{2.161878in}}%
\pgfpathcurveto{\pgfqpoint{1.782628in}{2.167702in}}{\pgfqpoint{1.774728in}{2.170974in}}{\pgfqpoint{1.766491in}{2.170974in}}%
\pgfpathcurveto{\pgfqpoint{1.758255in}{2.170974in}}{\pgfqpoint{1.750355in}{2.167702in}}{\pgfqpoint{1.744531in}{2.161878in}}%
\pgfpathcurveto{\pgfqpoint{1.738707in}{2.156054in}}{\pgfqpoint{1.735435in}{2.148154in}}{\pgfqpoint{1.735435in}{2.139917in}}%
\pgfpathcurveto{\pgfqpoint{1.735435in}{2.131681in}}{\pgfqpoint{1.738707in}{2.123781in}}{\pgfqpoint{1.744531in}{2.117957in}}%
\pgfpathcurveto{\pgfqpoint{1.750355in}{2.112133in}}{\pgfqpoint{1.758255in}{2.108861in}}{\pgfqpoint{1.766491in}{2.108861in}}%
\pgfpathclose%
\pgfusepath{stroke,fill}%
\end{pgfscope}%
\begin{pgfscope}%
\pgfpathrectangle{\pgfqpoint{0.100000in}{0.212622in}}{\pgfqpoint{3.696000in}{3.696000in}}%
\pgfusepath{clip}%
\pgfsetbuttcap%
\pgfsetroundjoin%
\definecolor{currentfill}{rgb}{0.121569,0.466667,0.705882}%
\pgfsetfillcolor{currentfill}%
\pgfsetfillopacity{0.872722}%
\pgfsetlinewidth{1.003750pt}%
\definecolor{currentstroke}{rgb}{0.121569,0.466667,0.705882}%
\pgfsetstrokecolor{currentstroke}%
\pgfsetstrokeopacity{0.872722}%
\pgfsetdash{}{0pt}%
\pgfpathmoveto{\pgfqpoint{2.278285in}{2.286109in}}%
\pgfpathcurveto{\pgfqpoint{2.286521in}{2.286109in}}{\pgfqpoint{2.294421in}{2.289381in}}{\pgfqpoint{2.300245in}{2.295205in}}%
\pgfpathcurveto{\pgfqpoint{2.306069in}{2.301029in}}{\pgfqpoint{2.309341in}{2.308929in}}{\pgfqpoint{2.309341in}{2.317165in}}%
\pgfpathcurveto{\pgfqpoint{2.309341in}{2.325402in}}{\pgfqpoint{2.306069in}{2.333302in}}{\pgfqpoint{2.300245in}{2.339126in}}%
\pgfpathcurveto{\pgfqpoint{2.294421in}{2.344949in}}{\pgfqpoint{2.286521in}{2.348222in}}{\pgfqpoint{2.278285in}{2.348222in}}%
\pgfpathcurveto{\pgfqpoint{2.270048in}{2.348222in}}{\pgfqpoint{2.262148in}{2.344949in}}{\pgfqpoint{2.256324in}{2.339126in}}%
\pgfpathcurveto{\pgfqpoint{2.250500in}{2.333302in}}{\pgfqpoint{2.247228in}{2.325402in}}{\pgfqpoint{2.247228in}{2.317165in}}%
\pgfpathcurveto{\pgfqpoint{2.247228in}{2.308929in}}{\pgfqpoint{2.250500in}{2.301029in}}{\pgfqpoint{2.256324in}{2.295205in}}%
\pgfpathcurveto{\pgfqpoint{2.262148in}{2.289381in}}{\pgfqpoint{2.270048in}{2.286109in}}{\pgfqpoint{2.278285in}{2.286109in}}%
\pgfpathclose%
\pgfusepath{stroke,fill}%
\end{pgfscope}%
\begin{pgfscope}%
\pgfpathrectangle{\pgfqpoint{0.100000in}{0.212622in}}{\pgfqpoint{3.696000in}{3.696000in}}%
\pgfusepath{clip}%
\pgfsetbuttcap%
\pgfsetroundjoin%
\definecolor{currentfill}{rgb}{0.121569,0.466667,0.705882}%
\pgfsetfillcolor{currentfill}%
\pgfsetfillopacity{0.873525}%
\pgfsetlinewidth{1.003750pt}%
\definecolor{currentstroke}{rgb}{0.121569,0.466667,0.705882}%
\pgfsetstrokecolor{currentstroke}%
\pgfsetstrokeopacity{0.873525}%
\pgfsetdash{}{0pt}%
\pgfpathmoveto{\pgfqpoint{1.763134in}{2.103350in}}%
\pgfpathcurveto{\pgfqpoint{1.771370in}{2.103350in}}{\pgfqpoint{1.779270in}{2.106623in}}{\pgfqpoint{1.785094in}{2.112447in}}%
\pgfpathcurveto{\pgfqpoint{1.790918in}{2.118271in}}{\pgfqpoint{1.794190in}{2.126171in}}{\pgfqpoint{1.794190in}{2.134407in}}%
\pgfpathcurveto{\pgfqpoint{1.794190in}{2.142643in}}{\pgfqpoint{1.790918in}{2.150543in}}{\pgfqpoint{1.785094in}{2.156367in}}%
\pgfpathcurveto{\pgfqpoint{1.779270in}{2.162191in}}{\pgfqpoint{1.771370in}{2.165463in}}{\pgfqpoint{1.763134in}{2.165463in}}%
\pgfpathcurveto{\pgfqpoint{1.754897in}{2.165463in}}{\pgfqpoint{1.746997in}{2.162191in}}{\pgfqpoint{1.741173in}{2.156367in}}%
\pgfpathcurveto{\pgfqpoint{1.735350in}{2.150543in}}{\pgfqpoint{1.732077in}{2.142643in}}{\pgfqpoint{1.732077in}{2.134407in}}%
\pgfpathcurveto{\pgfqpoint{1.732077in}{2.126171in}}{\pgfqpoint{1.735350in}{2.118271in}}{\pgfqpoint{1.741173in}{2.112447in}}%
\pgfpathcurveto{\pgfqpoint{1.746997in}{2.106623in}}{\pgfqpoint{1.754897in}{2.103350in}}{\pgfqpoint{1.763134in}{2.103350in}}%
\pgfpathclose%
\pgfusepath{stroke,fill}%
\end{pgfscope}%
\begin{pgfscope}%
\pgfpathrectangle{\pgfqpoint{0.100000in}{0.212622in}}{\pgfqpoint{3.696000in}{3.696000in}}%
\pgfusepath{clip}%
\pgfsetbuttcap%
\pgfsetroundjoin%
\definecolor{currentfill}{rgb}{0.121569,0.466667,0.705882}%
\pgfsetfillcolor{currentfill}%
\pgfsetfillopacity{0.873564}%
\pgfsetlinewidth{1.003750pt}%
\definecolor{currentstroke}{rgb}{0.121569,0.466667,0.705882}%
\pgfsetstrokecolor{currentstroke}%
\pgfsetstrokeopacity{0.873564}%
\pgfsetdash{}{0pt}%
\pgfpathmoveto{\pgfqpoint{2.276530in}{2.281812in}}%
\pgfpathcurveto{\pgfqpoint{2.284767in}{2.281812in}}{\pgfqpoint{2.292667in}{2.285084in}}{\pgfqpoint{2.298491in}{2.290908in}}%
\pgfpathcurveto{\pgfqpoint{2.304314in}{2.296732in}}{\pgfqpoint{2.307587in}{2.304632in}}{\pgfqpoint{2.307587in}{2.312868in}}%
\pgfpathcurveto{\pgfqpoint{2.307587in}{2.321104in}}{\pgfqpoint{2.304314in}{2.329005in}}{\pgfqpoint{2.298491in}{2.334828in}}%
\pgfpathcurveto{\pgfqpoint{2.292667in}{2.340652in}}{\pgfqpoint{2.284767in}{2.343925in}}{\pgfqpoint{2.276530in}{2.343925in}}%
\pgfpathcurveto{\pgfqpoint{2.268294in}{2.343925in}}{\pgfqpoint{2.260394in}{2.340652in}}{\pgfqpoint{2.254570in}{2.334828in}}%
\pgfpathcurveto{\pgfqpoint{2.248746in}{2.329005in}}{\pgfqpoint{2.245474in}{2.321104in}}{\pgfqpoint{2.245474in}{2.312868in}}%
\pgfpathcurveto{\pgfqpoint{2.245474in}{2.304632in}}{\pgfqpoint{2.248746in}{2.296732in}}{\pgfqpoint{2.254570in}{2.290908in}}%
\pgfpathcurveto{\pgfqpoint{2.260394in}{2.285084in}}{\pgfqpoint{2.268294in}{2.281812in}}{\pgfqpoint{2.276530in}{2.281812in}}%
\pgfpathclose%
\pgfusepath{stroke,fill}%
\end{pgfscope}%
\begin{pgfscope}%
\pgfpathrectangle{\pgfqpoint{0.100000in}{0.212622in}}{\pgfqpoint{3.696000in}{3.696000in}}%
\pgfusepath{clip}%
\pgfsetbuttcap%
\pgfsetroundjoin%
\definecolor{currentfill}{rgb}{0.121569,0.466667,0.705882}%
\pgfsetfillcolor{currentfill}%
\pgfsetfillopacity{0.873859}%
\pgfsetlinewidth{1.003750pt}%
\definecolor{currentstroke}{rgb}{0.121569,0.466667,0.705882}%
\pgfsetstrokecolor{currentstroke}%
\pgfsetstrokeopacity{0.873859}%
\pgfsetdash{}{0pt}%
\pgfpathmoveto{\pgfqpoint{2.275038in}{2.279859in}}%
\pgfpathcurveto{\pgfqpoint{2.283274in}{2.279859in}}{\pgfqpoint{2.291175in}{2.283132in}}{\pgfqpoint{2.296998in}{2.288956in}}%
\pgfpathcurveto{\pgfqpoint{2.302822in}{2.294779in}}{\pgfqpoint{2.306095in}{2.302680in}}{\pgfqpoint{2.306095in}{2.310916in}}%
\pgfpathcurveto{\pgfqpoint{2.306095in}{2.319152in}}{\pgfqpoint{2.302822in}{2.327052in}}{\pgfqpoint{2.296998in}{2.332876in}}%
\pgfpathcurveto{\pgfqpoint{2.291175in}{2.338700in}}{\pgfqpoint{2.283274in}{2.341972in}}{\pgfqpoint{2.275038in}{2.341972in}}%
\pgfpathcurveto{\pgfqpoint{2.266802in}{2.341972in}}{\pgfqpoint{2.258902in}{2.338700in}}{\pgfqpoint{2.253078in}{2.332876in}}%
\pgfpathcurveto{\pgfqpoint{2.247254in}{2.327052in}}{\pgfqpoint{2.243982in}{2.319152in}}{\pgfqpoint{2.243982in}{2.310916in}}%
\pgfpathcurveto{\pgfqpoint{2.243982in}{2.302680in}}{\pgfqpoint{2.247254in}{2.294779in}}{\pgfqpoint{2.253078in}{2.288956in}}%
\pgfpathcurveto{\pgfqpoint{2.258902in}{2.283132in}}{\pgfqpoint{2.266802in}{2.279859in}}{\pgfqpoint{2.275038in}{2.279859in}}%
\pgfpathclose%
\pgfusepath{stroke,fill}%
\end{pgfscope}%
\begin{pgfscope}%
\pgfpathrectangle{\pgfqpoint{0.100000in}{0.212622in}}{\pgfqpoint{3.696000in}{3.696000in}}%
\pgfusepath{clip}%
\pgfsetbuttcap%
\pgfsetroundjoin%
\definecolor{currentfill}{rgb}{0.121569,0.466667,0.705882}%
\pgfsetfillcolor{currentfill}%
\pgfsetfillopacity{0.874091}%
\pgfsetlinewidth{1.003750pt}%
\definecolor{currentstroke}{rgb}{0.121569,0.466667,0.705882}%
\pgfsetstrokecolor{currentstroke}%
\pgfsetstrokeopacity{0.874091}%
\pgfsetdash{}{0pt}%
\pgfpathmoveto{\pgfqpoint{1.761280in}{2.100535in}}%
\pgfpathcurveto{\pgfqpoint{1.769517in}{2.100535in}}{\pgfqpoint{1.777417in}{2.103807in}}{\pgfqpoint{1.783241in}{2.109631in}}%
\pgfpathcurveto{\pgfqpoint{1.789065in}{2.115455in}}{\pgfqpoint{1.792337in}{2.123355in}}{\pgfqpoint{1.792337in}{2.131591in}}%
\pgfpathcurveto{\pgfqpoint{1.792337in}{2.139827in}}{\pgfqpoint{1.789065in}{2.147728in}}{\pgfqpoint{1.783241in}{2.153551in}}%
\pgfpathcurveto{\pgfqpoint{1.777417in}{2.159375in}}{\pgfqpoint{1.769517in}{2.162648in}}{\pgfqpoint{1.761280in}{2.162648in}}%
\pgfpathcurveto{\pgfqpoint{1.753044in}{2.162648in}}{\pgfqpoint{1.745144in}{2.159375in}}{\pgfqpoint{1.739320in}{2.153551in}}%
\pgfpathcurveto{\pgfqpoint{1.733496in}{2.147728in}}{\pgfqpoint{1.730224in}{2.139827in}}{\pgfqpoint{1.730224in}{2.131591in}}%
\pgfpathcurveto{\pgfqpoint{1.730224in}{2.123355in}}{\pgfqpoint{1.733496in}{2.115455in}}{\pgfqpoint{1.739320in}{2.109631in}}%
\pgfpathcurveto{\pgfqpoint{1.745144in}{2.103807in}}{\pgfqpoint{1.753044in}{2.100535in}}{\pgfqpoint{1.761280in}{2.100535in}}%
\pgfpathclose%
\pgfusepath{stroke,fill}%
\end{pgfscope}%
\begin{pgfscope}%
\pgfpathrectangle{\pgfqpoint{0.100000in}{0.212622in}}{\pgfqpoint{3.696000in}{3.696000in}}%
\pgfusepath{clip}%
\pgfsetbuttcap%
\pgfsetroundjoin%
\definecolor{currentfill}{rgb}{0.121569,0.466667,0.705882}%
\pgfsetfillcolor{currentfill}%
\pgfsetfillopacity{0.874404}%
\pgfsetlinewidth{1.003750pt}%
\definecolor{currentstroke}{rgb}{0.121569,0.466667,0.705882}%
\pgfsetstrokecolor{currentstroke}%
\pgfsetstrokeopacity{0.874404}%
\pgfsetdash{}{0pt}%
\pgfpathmoveto{\pgfqpoint{1.496776in}{1.239826in}}%
\pgfpathcurveto{\pgfqpoint{1.505012in}{1.239826in}}{\pgfqpoint{1.512912in}{1.243099in}}{\pgfqpoint{1.518736in}{1.248923in}}%
\pgfpathcurveto{\pgfqpoint{1.524560in}{1.254746in}}{\pgfqpoint{1.527832in}{1.262646in}}{\pgfqpoint{1.527832in}{1.270883in}}%
\pgfpathcurveto{\pgfqpoint{1.527832in}{1.279119in}}{\pgfqpoint{1.524560in}{1.287019in}}{\pgfqpoint{1.518736in}{1.292843in}}%
\pgfpathcurveto{\pgfqpoint{1.512912in}{1.298667in}}{\pgfqpoint{1.505012in}{1.301939in}}{\pgfqpoint{1.496776in}{1.301939in}}%
\pgfpathcurveto{\pgfqpoint{1.488539in}{1.301939in}}{\pgfqpoint{1.480639in}{1.298667in}}{\pgfqpoint{1.474816in}{1.292843in}}%
\pgfpathcurveto{\pgfqpoint{1.468992in}{1.287019in}}{\pgfqpoint{1.465719in}{1.279119in}}{\pgfqpoint{1.465719in}{1.270883in}}%
\pgfpathcurveto{\pgfqpoint{1.465719in}{1.262646in}}{\pgfqpoint{1.468992in}{1.254746in}}{\pgfqpoint{1.474816in}{1.248923in}}%
\pgfpathcurveto{\pgfqpoint{1.480639in}{1.243099in}}{\pgfqpoint{1.488539in}{1.239826in}}{\pgfqpoint{1.496776in}{1.239826in}}%
\pgfpathclose%
\pgfusepath{stroke,fill}%
\end{pgfscope}%
\begin{pgfscope}%
\pgfpathrectangle{\pgfqpoint{0.100000in}{0.212622in}}{\pgfqpoint{3.696000in}{3.696000in}}%
\pgfusepath{clip}%
\pgfsetbuttcap%
\pgfsetroundjoin%
\definecolor{currentfill}{rgb}{0.121569,0.466667,0.705882}%
\pgfsetfillcolor{currentfill}%
\pgfsetfillopacity{0.874445}%
\pgfsetlinewidth{1.003750pt}%
\definecolor{currentstroke}{rgb}{0.121569,0.466667,0.705882}%
\pgfsetstrokecolor{currentstroke}%
\pgfsetstrokeopacity{0.874445}%
\pgfsetdash{}{0pt}%
\pgfpathmoveto{\pgfqpoint{2.272528in}{2.276054in}}%
\pgfpathcurveto{\pgfqpoint{2.280764in}{2.276054in}}{\pgfqpoint{2.288664in}{2.279327in}}{\pgfqpoint{2.294488in}{2.285151in}}%
\pgfpathcurveto{\pgfqpoint{2.300312in}{2.290975in}}{\pgfqpoint{2.303584in}{2.298875in}}{\pgfqpoint{2.303584in}{2.307111in}}%
\pgfpathcurveto{\pgfqpoint{2.303584in}{2.315347in}}{\pgfqpoint{2.300312in}{2.323247in}}{\pgfqpoint{2.294488in}{2.329071in}}%
\pgfpathcurveto{\pgfqpoint{2.288664in}{2.334895in}}{\pgfqpoint{2.280764in}{2.338167in}}{\pgfqpoint{2.272528in}{2.338167in}}%
\pgfpathcurveto{\pgfqpoint{2.264291in}{2.338167in}}{\pgfqpoint{2.256391in}{2.334895in}}{\pgfqpoint{2.250567in}{2.329071in}}%
\pgfpathcurveto{\pgfqpoint{2.244744in}{2.323247in}}{\pgfqpoint{2.241471in}{2.315347in}}{\pgfqpoint{2.241471in}{2.307111in}}%
\pgfpathcurveto{\pgfqpoint{2.241471in}{2.298875in}}{\pgfqpoint{2.244744in}{2.290975in}}{\pgfqpoint{2.250567in}{2.285151in}}%
\pgfpathcurveto{\pgfqpoint{2.256391in}{2.279327in}}{\pgfqpoint{2.264291in}{2.276054in}}{\pgfqpoint{2.272528in}{2.276054in}}%
\pgfpathclose%
\pgfusepath{stroke,fill}%
\end{pgfscope}%
\begin{pgfscope}%
\pgfpathrectangle{\pgfqpoint{0.100000in}{0.212622in}}{\pgfqpoint{3.696000in}{3.696000in}}%
\pgfusepath{clip}%
\pgfsetbuttcap%
\pgfsetroundjoin%
\definecolor{currentfill}{rgb}{0.121569,0.466667,0.705882}%
\pgfsetfillcolor{currentfill}%
\pgfsetfillopacity{0.874854}%
\pgfsetlinewidth{1.003750pt}%
\definecolor{currentstroke}{rgb}{0.121569,0.466667,0.705882}%
\pgfsetstrokecolor{currentstroke}%
\pgfsetstrokeopacity{0.874854}%
\pgfsetdash{}{0pt}%
\pgfpathmoveto{\pgfqpoint{2.272119in}{2.273235in}}%
\pgfpathcurveto{\pgfqpoint{2.280355in}{2.273235in}}{\pgfqpoint{2.288255in}{2.276507in}}{\pgfqpoint{2.294079in}{2.282331in}}%
\pgfpathcurveto{\pgfqpoint{2.299903in}{2.288155in}}{\pgfqpoint{2.303175in}{2.296055in}}{\pgfqpoint{2.303175in}{2.304291in}}%
\pgfpathcurveto{\pgfqpoint{2.303175in}{2.312528in}}{\pgfqpoint{2.299903in}{2.320428in}}{\pgfqpoint{2.294079in}{2.326252in}}%
\pgfpathcurveto{\pgfqpoint{2.288255in}{2.332076in}}{\pgfqpoint{2.280355in}{2.335348in}}{\pgfqpoint{2.272119in}{2.335348in}}%
\pgfpathcurveto{\pgfqpoint{2.263883in}{2.335348in}}{\pgfqpoint{2.255982in}{2.332076in}}{\pgfqpoint{2.250159in}{2.326252in}}%
\pgfpathcurveto{\pgfqpoint{2.244335in}{2.320428in}}{\pgfqpoint{2.241062in}{2.312528in}}{\pgfqpoint{2.241062in}{2.304291in}}%
\pgfpathcurveto{\pgfqpoint{2.241062in}{2.296055in}}{\pgfqpoint{2.244335in}{2.288155in}}{\pgfqpoint{2.250159in}{2.282331in}}%
\pgfpathcurveto{\pgfqpoint{2.255982in}{2.276507in}}{\pgfqpoint{2.263883in}{2.273235in}}{\pgfqpoint{2.272119in}{2.273235in}}%
\pgfpathclose%
\pgfusepath{stroke,fill}%
\end{pgfscope}%
\begin{pgfscope}%
\pgfpathrectangle{\pgfqpoint{0.100000in}{0.212622in}}{\pgfqpoint{3.696000in}{3.696000in}}%
\pgfusepath{clip}%
\pgfsetbuttcap%
\pgfsetroundjoin%
\definecolor{currentfill}{rgb}{0.121569,0.466667,0.705882}%
\pgfsetfillcolor{currentfill}%
\pgfsetfillopacity{0.874887}%
\pgfsetlinewidth{1.003750pt}%
\definecolor{currentstroke}{rgb}{0.121569,0.466667,0.705882}%
\pgfsetstrokecolor{currentstroke}%
\pgfsetstrokeopacity{0.874887}%
\pgfsetdash{}{0pt}%
\pgfpathmoveto{\pgfqpoint{1.759848in}{2.095560in}}%
\pgfpathcurveto{\pgfqpoint{1.768084in}{2.095560in}}{\pgfqpoint{1.775984in}{2.098833in}}{\pgfqpoint{1.781808in}{2.104657in}}%
\pgfpathcurveto{\pgfqpoint{1.787632in}{2.110481in}}{\pgfqpoint{1.790905in}{2.118381in}}{\pgfqpoint{1.790905in}{2.126617in}}%
\pgfpathcurveto{\pgfqpoint{1.790905in}{2.134853in}}{\pgfqpoint{1.787632in}{2.142753in}}{\pgfqpoint{1.781808in}{2.148577in}}%
\pgfpathcurveto{\pgfqpoint{1.775984in}{2.154401in}}{\pgfqpoint{1.768084in}{2.157673in}}{\pgfqpoint{1.759848in}{2.157673in}}%
\pgfpathcurveto{\pgfqpoint{1.751612in}{2.157673in}}{\pgfqpoint{1.743712in}{2.154401in}}{\pgfqpoint{1.737888in}{2.148577in}}%
\pgfpathcurveto{\pgfqpoint{1.732064in}{2.142753in}}{\pgfqpoint{1.728792in}{2.134853in}}{\pgfqpoint{1.728792in}{2.126617in}}%
\pgfpathcurveto{\pgfqpoint{1.728792in}{2.118381in}}{\pgfqpoint{1.732064in}{2.110481in}}{\pgfqpoint{1.737888in}{2.104657in}}%
\pgfpathcurveto{\pgfqpoint{1.743712in}{2.098833in}}{\pgfqpoint{1.751612in}{2.095560in}}{\pgfqpoint{1.759848in}{2.095560in}}%
\pgfpathclose%
\pgfusepath{stroke,fill}%
\end{pgfscope}%
\begin{pgfscope}%
\pgfpathrectangle{\pgfqpoint{0.100000in}{0.212622in}}{\pgfqpoint{3.696000in}{3.696000in}}%
\pgfusepath{clip}%
\pgfsetbuttcap%
\pgfsetroundjoin%
\definecolor{currentfill}{rgb}{0.121569,0.466667,0.705882}%
\pgfsetfillcolor{currentfill}%
\pgfsetfillopacity{0.875314}%
\pgfsetlinewidth{1.003750pt}%
\definecolor{currentstroke}{rgb}{0.121569,0.466667,0.705882}%
\pgfsetstrokecolor{currentstroke}%
\pgfsetstrokeopacity{0.875314}%
\pgfsetdash{}{0pt}%
\pgfpathmoveto{\pgfqpoint{1.758530in}{2.093164in}}%
\pgfpathcurveto{\pgfqpoint{1.766767in}{2.093164in}}{\pgfqpoint{1.774667in}{2.096436in}}{\pgfqpoint{1.780491in}{2.102260in}}%
\pgfpathcurveto{\pgfqpoint{1.786315in}{2.108084in}}{\pgfqpoint{1.789587in}{2.115984in}}{\pgfqpoint{1.789587in}{2.124220in}}%
\pgfpathcurveto{\pgfqpoint{1.789587in}{2.132457in}}{\pgfqpoint{1.786315in}{2.140357in}}{\pgfqpoint{1.780491in}{2.146181in}}%
\pgfpathcurveto{\pgfqpoint{1.774667in}{2.152005in}}{\pgfqpoint{1.766767in}{2.155277in}}{\pgfqpoint{1.758530in}{2.155277in}}%
\pgfpathcurveto{\pgfqpoint{1.750294in}{2.155277in}}{\pgfqpoint{1.742394in}{2.152005in}}{\pgfqpoint{1.736570in}{2.146181in}}%
\pgfpathcurveto{\pgfqpoint{1.730746in}{2.140357in}}{\pgfqpoint{1.727474in}{2.132457in}}{\pgfqpoint{1.727474in}{2.124220in}}%
\pgfpathcurveto{\pgfqpoint{1.727474in}{2.115984in}}{\pgfqpoint{1.730746in}{2.108084in}}{\pgfqpoint{1.736570in}{2.102260in}}%
\pgfpathcurveto{\pgfqpoint{1.742394in}{2.096436in}}{\pgfqpoint{1.750294in}{2.093164in}}{\pgfqpoint{1.758530in}{2.093164in}}%
\pgfpathclose%
\pgfusepath{stroke,fill}%
\end{pgfscope}%
\begin{pgfscope}%
\pgfpathrectangle{\pgfqpoint{0.100000in}{0.212622in}}{\pgfqpoint{3.696000in}{3.696000in}}%
\pgfusepath{clip}%
\pgfsetbuttcap%
\pgfsetroundjoin%
\definecolor{currentfill}{rgb}{0.121569,0.466667,0.705882}%
\pgfsetfillcolor{currentfill}%
\pgfsetfillopacity{0.875546}%
\pgfsetlinewidth{1.003750pt}%
\definecolor{currentstroke}{rgb}{0.121569,0.466667,0.705882}%
\pgfsetstrokecolor{currentstroke}%
\pgfsetstrokeopacity{0.875546}%
\pgfsetdash{}{0pt}%
\pgfpathmoveto{\pgfqpoint{1.757723in}{2.091957in}}%
\pgfpathcurveto{\pgfqpoint{1.765959in}{2.091957in}}{\pgfqpoint{1.773859in}{2.095229in}}{\pgfqpoint{1.779683in}{2.101053in}}%
\pgfpathcurveto{\pgfqpoint{1.785507in}{2.106877in}}{\pgfqpoint{1.788779in}{2.114777in}}{\pgfqpoint{1.788779in}{2.123013in}}%
\pgfpathcurveto{\pgfqpoint{1.788779in}{2.131249in}}{\pgfqpoint{1.785507in}{2.139149in}}{\pgfqpoint{1.779683in}{2.144973in}}%
\pgfpathcurveto{\pgfqpoint{1.773859in}{2.150797in}}{\pgfqpoint{1.765959in}{2.154070in}}{\pgfqpoint{1.757723in}{2.154070in}}%
\pgfpathcurveto{\pgfqpoint{1.749487in}{2.154070in}}{\pgfqpoint{1.741587in}{2.150797in}}{\pgfqpoint{1.735763in}{2.144973in}}%
\pgfpathcurveto{\pgfqpoint{1.729939in}{2.139149in}}{\pgfqpoint{1.726666in}{2.131249in}}{\pgfqpoint{1.726666in}{2.123013in}}%
\pgfpathcurveto{\pgfqpoint{1.726666in}{2.114777in}}{\pgfqpoint{1.729939in}{2.106877in}}{\pgfqpoint{1.735763in}{2.101053in}}%
\pgfpathcurveto{\pgfqpoint{1.741587in}{2.095229in}}{\pgfqpoint{1.749487in}{2.091957in}}{\pgfqpoint{1.757723in}{2.091957in}}%
\pgfpathclose%
\pgfusepath{stroke,fill}%
\end{pgfscope}%
\begin{pgfscope}%
\pgfpathrectangle{\pgfqpoint{0.100000in}{0.212622in}}{\pgfqpoint{3.696000in}{3.696000in}}%
\pgfusepath{clip}%
\pgfsetbuttcap%
\pgfsetroundjoin%
\definecolor{currentfill}{rgb}{0.121569,0.466667,0.705882}%
\pgfsetfillcolor{currentfill}%
\pgfsetfillopacity{0.875700}%
\pgfsetlinewidth{1.003750pt}%
\definecolor{currentstroke}{rgb}{0.121569,0.466667,0.705882}%
\pgfsetstrokecolor{currentstroke}%
\pgfsetstrokeopacity{0.875700}%
\pgfsetdash{}{0pt}%
\pgfpathmoveto{\pgfqpoint{2.270257in}{2.268853in}}%
\pgfpathcurveto{\pgfqpoint{2.278493in}{2.268853in}}{\pgfqpoint{2.286393in}{2.272126in}}{\pgfqpoint{2.292217in}{2.277950in}}%
\pgfpathcurveto{\pgfqpoint{2.298041in}{2.283774in}}{\pgfqpoint{2.301313in}{2.291674in}}{\pgfqpoint{2.301313in}{2.299910in}}%
\pgfpathcurveto{\pgfqpoint{2.301313in}{2.308146in}}{\pgfqpoint{2.298041in}{2.316046in}}{\pgfqpoint{2.292217in}{2.321870in}}%
\pgfpathcurveto{\pgfqpoint{2.286393in}{2.327694in}}{\pgfqpoint{2.278493in}{2.330966in}}{\pgfqpoint{2.270257in}{2.330966in}}%
\pgfpathcurveto{\pgfqpoint{2.262020in}{2.330966in}}{\pgfqpoint{2.254120in}{2.327694in}}{\pgfqpoint{2.248296in}{2.321870in}}%
\pgfpathcurveto{\pgfqpoint{2.242472in}{2.316046in}}{\pgfqpoint{2.239200in}{2.308146in}}{\pgfqpoint{2.239200in}{2.299910in}}%
\pgfpathcurveto{\pgfqpoint{2.239200in}{2.291674in}}{\pgfqpoint{2.242472in}{2.283774in}}{\pgfqpoint{2.248296in}{2.277950in}}%
\pgfpathcurveto{\pgfqpoint{2.254120in}{2.272126in}}{\pgfqpoint{2.262020in}{2.268853in}}{\pgfqpoint{2.270257in}{2.268853in}}%
\pgfpathclose%
\pgfusepath{stroke,fill}%
\end{pgfscope}%
\begin{pgfscope}%
\pgfpathrectangle{\pgfqpoint{0.100000in}{0.212622in}}{\pgfqpoint{3.696000in}{3.696000in}}%
\pgfusepath{clip}%
\pgfsetbuttcap%
\pgfsetroundjoin%
\definecolor{currentfill}{rgb}{0.121569,0.466667,0.705882}%
\pgfsetfillcolor{currentfill}%
\pgfsetfillopacity{0.876029}%
\pgfsetlinewidth{1.003750pt}%
\definecolor{currentstroke}{rgb}{0.121569,0.466667,0.705882}%
\pgfsetstrokecolor{currentstroke}%
\pgfsetstrokeopacity{0.876029}%
\pgfsetdash{}{0pt}%
\pgfpathmoveto{\pgfqpoint{1.756676in}{2.089210in}}%
\pgfpathcurveto{\pgfqpoint{1.764912in}{2.089210in}}{\pgfqpoint{1.772812in}{2.092482in}}{\pgfqpoint{1.778636in}{2.098306in}}%
\pgfpathcurveto{\pgfqpoint{1.784460in}{2.104130in}}{\pgfqpoint{1.787732in}{2.112030in}}{\pgfqpoint{1.787732in}{2.120266in}}%
\pgfpathcurveto{\pgfqpoint{1.787732in}{2.128502in}}{\pgfqpoint{1.784460in}{2.136402in}}{\pgfqpoint{1.778636in}{2.142226in}}%
\pgfpathcurveto{\pgfqpoint{1.772812in}{2.148050in}}{\pgfqpoint{1.764912in}{2.151323in}}{\pgfqpoint{1.756676in}{2.151323in}}%
\pgfpathcurveto{\pgfqpoint{1.748440in}{2.151323in}}{\pgfqpoint{1.740540in}{2.148050in}}{\pgfqpoint{1.734716in}{2.142226in}}%
\pgfpathcurveto{\pgfqpoint{1.728892in}{2.136402in}}{\pgfqpoint{1.725619in}{2.128502in}}{\pgfqpoint{1.725619in}{2.120266in}}%
\pgfpathcurveto{\pgfqpoint{1.725619in}{2.112030in}}{\pgfqpoint{1.728892in}{2.104130in}}{\pgfqpoint{1.734716in}{2.098306in}}%
\pgfpathcurveto{\pgfqpoint{1.740540in}{2.092482in}}{\pgfqpoint{1.748440in}{2.089210in}}{\pgfqpoint{1.756676in}{2.089210in}}%
\pgfpathclose%
\pgfusepath{stroke,fill}%
\end{pgfscope}%
\begin{pgfscope}%
\pgfpathrectangle{\pgfqpoint{0.100000in}{0.212622in}}{\pgfqpoint{3.696000in}{3.696000in}}%
\pgfusepath{clip}%
\pgfsetbuttcap%
\pgfsetroundjoin%
\definecolor{currentfill}{rgb}{0.121569,0.466667,0.705882}%
\pgfsetfillcolor{currentfill}%
\pgfsetfillopacity{0.876077}%
\pgfsetlinewidth{1.003750pt}%
\definecolor{currentstroke}{rgb}{0.121569,0.466667,0.705882}%
\pgfsetstrokecolor{currentstroke}%
\pgfsetstrokeopacity{0.876077}%
\pgfsetdash{}{0pt}%
\pgfpathmoveto{\pgfqpoint{2.268459in}{2.266468in}}%
\pgfpathcurveto{\pgfqpoint{2.276695in}{2.266468in}}{\pgfqpoint{2.284595in}{2.269740in}}{\pgfqpoint{2.290419in}{2.275564in}}%
\pgfpathcurveto{\pgfqpoint{2.296243in}{2.281388in}}{\pgfqpoint{2.299515in}{2.289288in}}{\pgfqpoint{2.299515in}{2.297524in}}%
\pgfpathcurveto{\pgfqpoint{2.299515in}{2.305760in}}{\pgfqpoint{2.296243in}{2.313661in}}{\pgfqpoint{2.290419in}{2.319484in}}%
\pgfpathcurveto{\pgfqpoint{2.284595in}{2.325308in}}{\pgfqpoint{2.276695in}{2.328581in}}{\pgfqpoint{2.268459in}{2.328581in}}%
\pgfpathcurveto{\pgfqpoint{2.260222in}{2.328581in}}{\pgfqpoint{2.252322in}{2.325308in}}{\pgfqpoint{2.246498in}{2.319484in}}%
\pgfpathcurveto{\pgfqpoint{2.240674in}{2.313661in}}{\pgfqpoint{2.237402in}{2.305760in}}{\pgfqpoint{2.237402in}{2.297524in}}%
\pgfpathcurveto{\pgfqpoint{2.237402in}{2.289288in}}{\pgfqpoint{2.240674in}{2.281388in}}{\pgfqpoint{2.246498in}{2.275564in}}%
\pgfpathcurveto{\pgfqpoint{2.252322in}{2.269740in}}{\pgfqpoint{2.260222in}{2.266468in}}{\pgfqpoint{2.268459in}{2.266468in}}%
\pgfpathclose%
\pgfusepath{stroke,fill}%
\end{pgfscope}%
\begin{pgfscope}%
\pgfpathrectangle{\pgfqpoint{0.100000in}{0.212622in}}{\pgfqpoint{3.696000in}{3.696000in}}%
\pgfusepath{clip}%
\pgfsetbuttcap%
\pgfsetroundjoin%
\definecolor{currentfill}{rgb}{0.121569,0.466667,0.705882}%
\pgfsetfillcolor{currentfill}%
\pgfsetfillopacity{0.876585}%
\pgfsetlinewidth{1.003750pt}%
\definecolor{currentstroke}{rgb}{0.121569,0.466667,0.705882}%
\pgfsetstrokecolor{currentstroke}%
\pgfsetstrokeopacity{0.876585}%
\pgfsetdash{}{0pt}%
\pgfpathmoveto{\pgfqpoint{1.755596in}{2.085941in}}%
\pgfpathcurveto{\pgfqpoint{1.763832in}{2.085941in}}{\pgfqpoint{1.771732in}{2.089213in}}{\pgfqpoint{1.777556in}{2.095037in}}%
\pgfpathcurveto{\pgfqpoint{1.783380in}{2.100861in}}{\pgfqpoint{1.786652in}{2.108761in}}{\pgfqpoint{1.786652in}{2.116998in}}%
\pgfpathcurveto{\pgfqpoint{1.786652in}{2.125234in}}{\pgfqpoint{1.783380in}{2.133134in}}{\pgfqpoint{1.777556in}{2.138958in}}%
\pgfpathcurveto{\pgfqpoint{1.771732in}{2.144782in}}{\pgfqpoint{1.763832in}{2.148054in}}{\pgfqpoint{1.755596in}{2.148054in}}%
\pgfpathcurveto{\pgfqpoint{1.747360in}{2.148054in}}{\pgfqpoint{1.739459in}{2.144782in}}{\pgfqpoint{1.733636in}{2.138958in}}%
\pgfpathcurveto{\pgfqpoint{1.727812in}{2.133134in}}{\pgfqpoint{1.724539in}{2.125234in}}{\pgfqpoint{1.724539in}{2.116998in}}%
\pgfpathcurveto{\pgfqpoint{1.724539in}{2.108761in}}{\pgfqpoint{1.727812in}{2.100861in}}{\pgfqpoint{1.733636in}{2.095037in}}%
\pgfpathcurveto{\pgfqpoint{1.739459in}{2.089213in}}{\pgfqpoint{1.747360in}{2.085941in}}{\pgfqpoint{1.755596in}{2.085941in}}%
\pgfpathclose%
\pgfusepath{stroke,fill}%
\end{pgfscope}%
\begin{pgfscope}%
\pgfpathrectangle{\pgfqpoint{0.100000in}{0.212622in}}{\pgfqpoint{3.696000in}{3.696000in}}%
\pgfusepath{clip}%
\pgfsetbuttcap%
\pgfsetroundjoin%
\definecolor{currentfill}{rgb}{0.121569,0.466667,0.705882}%
\pgfsetfillcolor{currentfill}%
\pgfsetfillopacity{0.876769}%
\pgfsetlinewidth{1.003750pt}%
\definecolor{currentstroke}{rgb}{0.121569,0.466667,0.705882}%
\pgfsetstrokecolor{currentstroke}%
\pgfsetstrokeopacity{0.876769}%
\pgfsetdash{}{0pt}%
\pgfpathmoveto{\pgfqpoint{1.510722in}{1.238080in}}%
\pgfpathcurveto{\pgfqpoint{1.518958in}{1.238080in}}{\pgfqpoint{1.526858in}{1.241352in}}{\pgfqpoint{1.532682in}{1.247176in}}%
\pgfpathcurveto{\pgfqpoint{1.538506in}{1.253000in}}{\pgfqpoint{1.541778in}{1.260900in}}{\pgfqpoint{1.541778in}{1.269137in}}%
\pgfpathcurveto{\pgfqpoint{1.541778in}{1.277373in}}{\pgfqpoint{1.538506in}{1.285273in}}{\pgfqpoint{1.532682in}{1.291097in}}%
\pgfpathcurveto{\pgfqpoint{1.526858in}{1.296921in}}{\pgfqpoint{1.518958in}{1.300193in}}{\pgfqpoint{1.510722in}{1.300193in}}%
\pgfpathcurveto{\pgfqpoint{1.502486in}{1.300193in}}{\pgfqpoint{1.494585in}{1.296921in}}{\pgfqpoint{1.488762in}{1.291097in}}%
\pgfpathcurveto{\pgfqpoint{1.482938in}{1.285273in}}{\pgfqpoint{1.479665in}{1.277373in}}{\pgfqpoint{1.479665in}{1.269137in}}%
\pgfpathcurveto{\pgfqpoint{1.479665in}{1.260900in}}{\pgfqpoint{1.482938in}{1.253000in}}{\pgfqpoint{1.488762in}{1.247176in}}%
\pgfpathcurveto{\pgfqpoint{1.494585in}{1.241352in}}{\pgfqpoint{1.502486in}{1.238080in}}{\pgfqpoint{1.510722in}{1.238080in}}%
\pgfpathclose%
\pgfusepath{stroke,fill}%
\end{pgfscope}%
\begin{pgfscope}%
\pgfpathrectangle{\pgfqpoint{0.100000in}{0.212622in}}{\pgfqpoint{3.696000in}{3.696000in}}%
\pgfusepath{clip}%
\pgfsetbuttcap%
\pgfsetroundjoin%
\definecolor{currentfill}{rgb}{0.121569,0.466667,0.705882}%
\pgfsetfillcolor{currentfill}%
\pgfsetfillopacity{0.876836}%
\pgfsetlinewidth{1.003750pt}%
\definecolor{currentstroke}{rgb}{0.121569,0.466667,0.705882}%
\pgfsetstrokecolor{currentstroke}%
\pgfsetstrokeopacity{0.876836}%
\pgfsetdash{}{0pt}%
\pgfpathmoveto{\pgfqpoint{2.265614in}{2.261596in}}%
\pgfpathcurveto{\pgfqpoint{2.273851in}{2.261596in}}{\pgfqpoint{2.281751in}{2.264868in}}{\pgfqpoint{2.287575in}{2.270692in}}%
\pgfpathcurveto{\pgfqpoint{2.293399in}{2.276516in}}{\pgfqpoint{2.296671in}{2.284416in}}{\pgfqpoint{2.296671in}{2.292652in}}%
\pgfpathcurveto{\pgfqpoint{2.296671in}{2.300889in}}{\pgfqpoint{2.293399in}{2.308789in}}{\pgfqpoint{2.287575in}{2.314613in}}%
\pgfpathcurveto{\pgfqpoint{2.281751in}{2.320437in}}{\pgfqpoint{2.273851in}{2.323709in}}{\pgfqpoint{2.265614in}{2.323709in}}%
\pgfpathcurveto{\pgfqpoint{2.257378in}{2.323709in}}{\pgfqpoint{2.249478in}{2.320437in}}{\pgfqpoint{2.243654in}{2.314613in}}%
\pgfpathcurveto{\pgfqpoint{2.237830in}{2.308789in}}{\pgfqpoint{2.234558in}{2.300889in}}{\pgfqpoint{2.234558in}{2.292652in}}%
\pgfpathcurveto{\pgfqpoint{2.234558in}{2.284416in}}{\pgfqpoint{2.237830in}{2.276516in}}{\pgfqpoint{2.243654in}{2.270692in}}%
\pgfpathcurveto{\pgfqpoint{2.249478in}{2.264868in}}{\pgfqpoint{2.257378in}{2.261596in}}{\pgfqpoint{2.265614in}{2.261596in}}%
\pgfpathclose%
\pgfusepath{stroke,fill}%
\end{pgfscope}%
\begin{pgfscope}%
\pgfpathrectangle{\pgfqpoint{0.100000in}{0.212622in}}{\pgfqpoint{3.696000in}{3.696000in}}%
\pgfusepath{clip}%
\pgfsetbuttcap%
\pgfsetroundjoin%
\definecolor{currentfill}{rgb}{0.121569,0.466667,0.705882}%
\pgfsetfillcolor{currentfill}%
\pgfsetfillopacity{0.877184}%
\pgfsetlinewidth{1.003750pt}%
\definecolor{currentstroke}{rgb}{0.121569,0.466667,0.705882}%
\pgfsetstrokecolor{currentstroke}%
\pgfsetstrokeopacity{0.877184}%
\pgfsetdash{}{0pt}%
\pgfpathmoveto{\pgfqpoint{1.753442in}{2.082175in}}%
\pgfpathcurveto{\pgfqpoint{1.761678in}{2.082175in}}{\pgfqpoint{1.769578in}{2.085448in}}{\pgfqpoint{1.775402in}{2.091272in}}%
\pgfpathcurveto{\pgfqpoint{1.781226in}{2.097096in}}{\pgfqpoint{1.784498in}{2.104996in}}{\pgfqpoint{1.784498in}{2.113232in}}%
\pgfpathcurveto{\pgfqpoint{1.784498in}{2.121468in}}{\pgfqpoint{1.781226in}{2.129368in}}{\pgfqpoint{1.775402in}{2.135192in}}%
\pgfpathcurveto{\pgfqpoint{1.769578in}{2.141016in}}{\pgfqpoint{1.761678in}{2.144288in}}{\pgfqpoint{1.753442in}{2.144288in}}%
\pgfpathcurveto{\pgfqpoint{1.745206in}{2.144288in}}{\pgfqpoint{1.737306in}{2.141016in}}{\pgfqpoint{1.731482in}{2.135192in}}%
\pgfpathcurveto{\pgfqpoint{1.725658in}{2.129368in}}{\pgfqpoint{1.722385in}{2.121468in}}{\pgfqpoint{1.722385in}{2.113232in}}%
\pgfpathcurveto{\pgfqpoint{1.722385in}{2.104996in}}{\pgfqpoint{1.725658in}{2.097096in}}{\pgfqpoint{1.731482in}{2.091272in}}%
\pgfpathcurveto{\pgfqpoint{1.737306in}{2.085448in}}{\pgfqpoint{1.745206in}{2.082175in}}{\pgfqpoint{1.753442in}{2.082175in}}%
\pgfpathclose%
\pgfusepath{stroke,fill}%
\end{pgfscope}%
\begin{pgfscope}%
\pgfpathrectangle{\pgfqpoint{0.100000in}{0.212622in}}{\pgfqpoint{3.696000in}{3.696000in}}%
\pgfusepath{clip}%
\pgfsetbuttcap%
\pgfsetroundjoin%
\definecolor{currentfill}{rgb}{0.121569,0.466667,0.705882}%
\pgfsetfillcolor{currentfill}%
\pgfsetfillopacity{0.877412}%
\pgfsetlinewidth{1.003750pt}%
\definecolor{currentstroke}{rgb}{0.121569,0.466667,0.705882}%
\pgfsetstrokecolor{currentstroke}%
\pgfsetstrokeopacity{0.877412}%
\pgfsetdash{}{0pt}%
\pgfpathmoveto{\pgfqpoint{2.265051in}{2.257579in}}%
\pgfpathcurveto{\pgfqpoint{2.273288in}{2.257579in}}{\pgfqpoint{2.281188in}{2.260852in}}{\pgfqpoint{2.287012in}{2.266676in}}%
\pgfpathcurveto{\pgfqpoint{2.292836in}{2.272500in}}{\pgfqpoint{2.296108in}{2.280400in}}{\pgfqpoint{2.296108in}{2.288636in}}%
\pgfpathcurveto{\pgfqpoint{2.296108in}{2.296872in}}{\pgfqpoint{2.292836in}{2.304772in}}{\pgfqpoint{2.287012in}{2.310596in}}%
\pgfpathcurveto{\pgfqpoint{2.281188in}{2.316420in}}{\pgfqpoint{2.273288in}{2.319692in}}{\pgfqpoint{2.265051in}{2.319692in}}%
\pgfpathcurveto{\pgfqpoint{2.256815in}{2.319692in}}{\pgfqpoint{2.248915in}{2.316420in}}{\pgfqpoint{2.243091in}{2.310596in}}%
\pgfpathcurveto{\pgfqpoint{2.237267in}{2.304772in}}{\pgfqpoint{2.233995in}{2.296872in}}{\pgfqpoint{2.233995in}{2.288636in}}%
\pgfpathcurveto{\pgfqpoint{2.233995in}{2.280400in}}{\pgfqpoint{2.237267in}{2.272500in}}{\pgfqpoint{2.243091in}{2.266676in}}%
\pgfpathcurveto{\pgfqpoint{2.248915in}{2.260852in}}{\pgfqpoint{2.256815in}{2.257579in}}{\pgfqpoint{2.265051in}{2.257579in}}%
\pgfpathclose%
\pgfusepath{stroke,fill}%
\end{pgfscope}%
\begin{pgfscope}%
\pgfpathrectangle{\pgfqpoint{0.100000in}{0.212622in}}{\pgfqpoint{3.696000in}{3.696000in}}%
\pgfusepath{clip}%
\pgfsetbuttcap%
\pgfsetroundjoin%
\definecolor{currentfill}{rgb}{0.121569,0.466667,0.705882}%
\pgfsetfillcolor{currentfill}%
\pgfsetfillopacity{0.877593}%
\pgfsetlinewidth{1.003750pt}%
\definecolor{currentstroke}{rgb}{0.121569,0.466667,0.705882}%
\pgfsetstrokecolor{currentstroke}%
\pgfsetstrokeopacity{0.877593}%
\pgfsetdash{}{0pt}%
\pgfpathmoveto{\pgfqpoint{1.752381in}{2.080293in}}%
\pgfpathcurveto{\pgfqpoint{1.760617in}{2.080293in}}{\pgfqpoint{1.768517in}{2.083565in}}{\pgfqpoint{1.774341in}{2.089389in}}%
\pgfpathcurveto{\pgfqpoint{1.780165in}{2.095213in}}{\pgfqpoint{1.783437in}{2.103113in}}{\pgfqpoint{1.783437in}{2.111349in}}%
\pgfpathcurveto{\pgfqpoint{1.783437in}{2.119585in}}{\pgfqpoint{1.780165in}{2.127485in}}{\pgfqpoint{1.774341in}{2.133309in}}%
\pgfpathcurveto{\pgfqpoint{1.768517in}{2.139133in}}{\pgfqpoint{1.760617in}{2.142406in}}{\pgfqpoint{1.752381in}{2.142406in}}%
\pgfpathcurveto{\pgfqpoint{1.744145in}{2.142406in}}{\pgfqpoint{1.736245in}{2.139133in}}{\pgfqpoint{1.730421in}{2.133309in}}%
\pgfpathcurveto{\pgfqpoint{1.724597in}{2.127485in}}{\pgfqpoint{1.721324in}{2.119585in}}{\pgfqpoint{1.721324in}{2.111349in}}%
\pgfpathcurveto{\pgfqpoint{1.721324in}{2.103113in}}{\pgfqpoint{1.724597in}{2.095213in}}{\pgfqpoint{1.730421in}{2.089389in}}%
\pgfpathcurveto{\pgfqpoint{1.736245in}{2.083565in}}{\pgfqpoint{1.744145in}{2.080293in}}{\pgfqpoint{1.752381in}{2.080293in}}%
\pgfpathclose%
\pgfusepath{stroke,fill}%
\end{pgfscope}%
\begin{pgfscope}%
\pgfpathrectangle{\pgfqpoint{0.100000in}{0.212622in}}{\pgfqpoint{3.696000in}{3.696000in}}%
\pgfusepath{clip}%
\pgfsetbuttcap%
\pgfsetroundjoin%
\definecolor{currentfill}{rgb}{0.121569,0.466667,0.705882}%
\pgfsetfillcolor{currentfill}%
\pgfsetfillopacity{0.877790}%
\pgfsetlinewidth{1.003750pt}%
\definecolor{currentstroke}{rgb}{0.121569,0.466667,0.705882}%
\pgfsetstrokecolor{currentstroke}%
\pgfsetstrokeopacity{0.877790}%
\pgfsetdash{}{0pt}%
\pgfpathmoveto{\pgfqpoint{1.752114in}{2.078976in}}%
\pgfpathcurveto{\pgfqpoint{1.760351in}{2.078976in}}{\pgfqpoint{1.768251in}{2.082248in}}{\pgfqpoint{1.774075in}{2.088072in}}%
\pgfpathcurveto{\pgfqpoint{1.779898in}{2.093896in}}{\pgfqpoint{1.783171in}{2.101796in}}{\pgfqpoint{1.783171in}{2.110032in}}%
\pgfpathcurveto{\pgfqpoint{1.783171in}{2.118268in}}{\pgfqpoint{1.779898in}{2.126168in}}{\pgfqpoint{1.774075in}{2.131992in}}%
\pgfpathcurveto{\pgfqpoint{1.768251in}{2.137816in}}{\pgfqpoint{1.760351in}{2.141089in}}{\pgfqpoint{1.752114in}{2.141089in}}%
\pgfpathcurveto{\pgfqpoint{1.743878in}{2.141089in}}{\pgfqpoint{1.735978in}{2.137816in}}{\pgfqpoint{1.730154in}{2.131992in}}%
\pgfpathcurveto{\pgfqpoint{1.724330in}{2.126168in}}{\pgfqpoint{1.721058in}{2.118268in}}{\pgfqpoint{1.721058in}{2.110032in}}%
\pgfpathcurveto{\pgfqpoint{1.721058in}{2.101796in}}{\pgfqpoint{1.724330in}{2.093896in}}{\pgfqpoint{1.730154in}{2.088072in}}%
\pgfpathcurveto{\pgfqpoint{1.735978in}{2.082248in}}{\pgfqpoint{1.743878in}{2.078976in}}{\pgfqpoint{1.752114in}{2.078976in}}%
\pgfpathclose%
\pgfusepath{stroke,fill}%
\end{pgfscope}%
\begin{pgfscope}%
\pgfpathrectangle{\pgfqpoint{0.100000in}{0.212622in}}{\pgfqpoint{3.696000in}{3.696000in}}%
\pgfusepath{clip}%
\pgfsetbuttcap%
\pgfsetroundjoin%
\definecolor{currentfill}{rgb}{0.121569,0.466667,0.705882}%
\pgfsetfillcolor{currentfill}%
\pgfsetfillopacity{0.878216}%
\pgfsetlinewidth{1.003750pt}%
\definecolor{currentstroke}{rgb}{0.121569,0.466667,0.705882}%
\pgfsetstrokecolor{currentstroke}%
\pgfsetstrokeopacity{0.878216}%
\pgfsetdash{}{0pt}%
\pgfpathmoveto{\pgfqpoint{1.750886in}{2.076851in}}%
\pgfpathcurveto{\pgfqpoint{1.759122in}{2.076851in}}{\pgfqpoint{1.767022in}{2.080123in}}{\pgfqpoint{1.772846in}{2.085947in}}%
\pgfpathcurveto{\pgfqpoint{1.778670in}{2.091771in}}{\pgfqpoint{1.781942in}{2.099671in}}{\pgfqpoint{1.781942in}{2.107907in}}%
\pgfpathcurveto{\pgfqpoint{1.781942in}{2.116143in}}{\pgfqpoint{1.778670in}{2.124043in}}{\pgfqpoint{1.772846in}{2.129867in}}%
\pgfpathcurveto{\pgfqpoint{1.767022in}{2.135691in}}{\pgfqpoint{1.759122in}{2.138964in}}{\pgfqpoint{1.750886in}{2.138964in}}%
\pgfpathcurveto{\pgfqpoint{1.742649in}{2.138964in}}{\pgfqpoint{1.734749in}{2.135691in}}{\pgfqpoint{1.728925in}{2.129867in}}%
\pgfpathcurveto{\pgfqpoint{1.723101in}{2.124043in}}{\pgfqpoint{1.719829in}{2.116143in}}{\pgfqpoint{1.719829in}{2.107907in}}%
\pgfpathcurveto{\pgfqpoint{1.719829in}{2.099671in}}{\pgfqpoint{1.723101in}{2.091771in}}{\pgfqpoint{1.728925in}{2.085947in}}%
\pgfpathcurveto{\pgfqpoint{1.734749in}{2.080123in}}{\pgfqpoint{1.742649in}{2.076851in}}{\pgfqpoint{1.750886in}{2.076851in}}%
\pgfpathclose%
\pgfusepath{stroke,fill}%
\end{pgfscope}%
\begin{pgfscope}%
\pgfpathrectangle{\pgfqpoint{0.100000in}{0.212622in}}{\pgfqpoint{3.696000in}{3.696000in}}%
\pgfusepath{clip}%
\pgfsetbuttcap%
\pgfsetroundjoin%
\definecolor{currentfill}{rgb}{0.121569,0.466667,0.705882}%
\pgfsetfillcolor{currentfill}%
\pgfsetfillopacity{0.878438}%
\pgfsetlinewidth{1.003750pt}%
\definecolor{currentstroke}{rgb}{0.121569,0.466667,0.705882}%
\pgfsetstrokecolor{currentstroke}%
\pgfsetstrokeopacity{0.878438}%
\pgfsetdash{}{0pt}%
\pgfpathmoveto{\pgfqpoint{1.750168in}{2.075688in}}%
\pgfpathcurveto{\pgfqpoint{1.758404in}{2.075688in}}{\pgfqpoint{1.766304in}{2.078961in}}{\pgfqpoint{1.772128in}{2.084784in}}%
\pgfpathcurveto{\pgfqpoint{1.777952in}{2.090608in}}{\pgfqpoint{1.781224in}{2.098508in}}{\pgfqpoint{1.781224in}{2.106745in}}%
\pgfpathcurveto{\pgfqpoint{1.781224in}{2.114981in}}{\pgfqpoint{1.777952in}{2.122881in}}{\pgfqpoint{1.772128in}{2.128705in}}%
\pgfpathcurveto{\pgfqpoint{1.766304in}{2.134529in}}{\pgfqpoint{1.758404in}{2.137801in}}{\pgfqpoint{1.750168in}{2.137801in}}%
\pgfpathcurveto{\pgfqpoint{1.741931in}{2.137801in}}{\pgfqpoint{1.734031in}{2.134529in}}{\pgfqpoint{1.728207in}{2.128705in}}%
\pgfpathcurveto{\pgfqpoint{1.722383in}{2.122881in}}{\pgfqpoint{1.719111in}{2.114981in}}{\pgfqpoint{1.719111in}{2.106745in}}%
\pgfpathcurveto{\pgfqpoint{1.719111in}{2.098508in}}{\pgfqpoint{1.722383in}{2.090608in}}{\pgfqpoint{1.728207in}{2.084784in}}%
\pgfpathcurveto{\pgfqpoint{1.734031in}{2.078961in}}{\pgfqpoint{1.741931in}{2.075688in}}{\pgfqpoint{1.750168in}{2.075688in}}%
\pgfpathclose%
\pgfusepath{stroke,fill}%
\end{pgfscope}%
\begin{pgfscope}%
\pgfpathrectangle{\pgfqpoint{0.100000in}{0.212622in}}{\pgfqpoint{3.696000in}{3.696000in}}%
\pgfusepath{clip}%
\pgfsetbuttcap%
\pgfsetroundjoin%
\definecolor{currentfill}{rgb}{0.121569,0.466667,0.705882}%
\pgfsetfillcolor{currentfill}%
\pgfsetfillopacity{0.878593}%
\pgfsetlinewidth{1.003750pt}%
\definecolor{currentstroke}{rgb}{0.121569,0.466667,0.705882}%
\pgfsetstrokecolor{currentstroke}%
\pgfsetstrokeopacity{0.878593}%
\pgfsetdash{}{0pt}%
\pgfpathmoveto{\pgfqpoint{2.262582in}{2.251153in}}%
\pgfpathcurveto{\pgfqpoint{2.270819in}{2.251153in}}{\pgfqpoint{2.278719in}{2.254425in}}{\pgfqpoint{2.284543in}{2.260249in}}%
\pgfpathcurveto{\pgfqpoint{2.290367in}{2.266073in}}{\pgfqpoint{2.293639in}{2.273973in}}{\pgfqpoint{2.293639in}{2.282210in}}%
\pgfpathcurveto{\pgfqpoint{2.293639in}{2.290446in}}{\pgfqpoint{2.290367in}{2.298346in}}{\pgfqpoint{2.284543in}{2.304170in}}%
\pgfpathcurveto{\pgfqpoint{2.278719in}{2.309994in}}{\pgfqpoint{2.270819in}{2.313266in}}{\pgfqpoint{2.262582in}{2.313266in}}%
\pgfpathcurveto{\pgfqpoint{2.254346in}{2.313266in}}{\pgfqpoint{2.246446in}{2.309994in}}{\pgfqpoint{2.240622in}{2.304170in}}%
\pgfpathcurveto{\pgfqpoint{2.234798in}{2.298346in}}{\pgfqpoint{2.231526in}{2.290446in}}{\pgfqpoint{2.231526in}{2.282210in}}%
\pgfpathcurveto{\pgfqpoint{2.231526in}{2.273973in}}{\pgfqpoint{2.234798in}{2.266073in}}{\pgfqpoint{2.240622in}{2.260249in}}%
\pgfpathcurveto{\pgfqpoint{2.246446in}{2.254425in}}{\pgfqpoint{2.254346in}{2.251153in}}{\pgfqpoint{2.262582in}{2.251153in}}%
\pgfpathclose%
\pgfusepath{stroke,fill}%
\end{pgfscope}%
\begin{pgfscope}%
\pgfpathrectangle{\pgfqpoint{0.100000in}{0.212622in}}{\pgfqpoint{3.696000in}{3.696000in}}%
\pgfusepath{clip}%
\pgfsetbuttcap%
\pgfsetroundjoin%
\definecolor{currentfill}{rgb}{0.121569,0.466667,0.705882}%
\pgfsetfillcolor{currentfill}%
\pgfsetfillopacity{0.878598}%
\pgfsetlinewidth{1.003750pt}%
\definecolor{currentstroke}{rgb}{0.121569,0.466667,0.705882}%
\pgfsetstrokecolor{currentstroke}%
\pgfsetstrokeopacity{0.878598}%
\pgfsetdash{}{0pt}%
\pgfpathmoveto{\pgfqpoint{2.855531in}{1.509030in}}%
\pgfpathcurveto{\pgfqpoint{2.863767in}{1.509030in}}{\pgfqpoint{2.871667in}{1.512302in}}{\pgfqpoint{2.877491in}{1.518126in}}%
\pgfpathcurveto{\pgfqpoint{2.883315in}{1.523950in}}{\pgfqpoint{2.886587in}{1.531850in}}{\pgfqpoint{2.886587in}{1.540086in}}%
\pgfpathcurveto{\pgfqpoint{2.886587in}{1.548322in}}{\pgfqpoint{2.883315in}{1.556222in}}{\pgfqpoint{2.877491in}{1.562046in}}%
\pgfpathcurveto{\pgfqpoint{2.871667in}{1.567870in}}{\pgfqpoint{2.863767in}{1.571143in}}{\pgfqpoint{2.855531in}{1.571143in}}%
\pgfpathcurveto{\pgfqpoint{2.847294in}{1.571143in}}{\pgfqpoint{2.839394in}{1.567870in}}{\pgfqpoint{2.833570in}{1.562046in}}%
\pgfpathcurveto{\pgfqpoint{2.827746in}{1.556222in}}{\pgfqpoint{2.824474in}{1.548322in}}{\pgfqpoint{2.824474in}{1.540086in}}%
\pgfpathcurveto{\pgfqpoint{2.824474in}{1.531850in}}{\pgfqpoint{2.827746in}{1.523950in}}{\pgfqpoint{2.833570in}{1.518126in}}%
\pgfpathcurveto{\pgfqpoint{2.839394in}{1.512302in}}{\pgfqpoint{2.847294in}{1.509030in}}{\pgfqpoint{2.855531in}{1.509030in}}%
\pgfpathclose%
\pgfusepath{stroke,fill}%
\end{pgfscope}%
\begin{pgfscope}%
\pgfpathrectangle{\pgfqpoint{0.100000in}{0.212622in}}{\pgfqpoint{3.696000in}{3.696000in}}%
\pgfusepath{clip}%
\pgfsetbuttcap%
\pgfsetroundjoin%
\definecolor{currentfill}{rgb}{0.121569,0.466667,0.705882}%
\pgfsetfillcolor{currentfill}%
\pgfsetfillopacity{0.878729}%
\pgfsetlinewidth{1.003750pt}%
\definecolor{currentstroke}{rgb}{0.121569,0.466667,0.705882}%
\pgfsetstrokecolor{currentstroke}%
\pgfsetstrokeopacity{0.878729}%
\pgfsetdash{}{0pt}%
\pgfpathmoveto{\pgfqpoint{1.525428in}{1.229162in}}%
\pgfpathcurveto{\pgfqpoint{1.533665in}{1.229162in}}{\pgfqpoint{1.541565in}{1.232435in}}{\pgfqpoint{1.547389in}{1.238259in}}%
\pgfpathcurveto{\pgfqpoint{1.553213in}{1.244083in}}{\pgfqpoint{1.556485in}{1.251983in}}{\pgfqpoint{1.556485in}{1.260219in}}%
\pgfpathcurveto{\pgfqpoint{1.556485in}{1.268455in}}{\pgfqpoint{1.553213in}{1.276355in}}{\pgfqpoint{1.547389in}{1.282179in}}%
\pgfpathcurveto{\pgfqpoint{1.541565in}{1.288003in}}{\pgfqpoint{1.533665in}{1.291275in}}{\pgfqpoint{1.525428in}{1.291275in}}%
\pgfpathcurveto{\pgfqpoint{1.517192in}{1.291275in}}{\pgfqpoint{1.509292in}{1.288003in}}{\pgfqpoint{1.503468in}{1.282179in}}%
\pgfpathcurveto{\pgfqpoint{1.497644in}{1.276355in}}{\pgfqpoint{1.494372in}{1.268455in}}{\pgfqpoint{1.494372in}{1.260219in}}%
\pgfpathcurveto{\pgfqpoint{1.494372in}{1.251983in}}{\pgfqpoint{1.497644in}{1.244083in}}{\pgfqpoint{1.503468in}{1.238259in}}%
\pgfpathcurveto{\pgfqpoint{1.509292in}{1.232435in}}{\pgfqpoint{1.517192in}{1.229162in}}{\pgfqpoint{1.525428in}{1.229162in}}%
\pgfpathclose%
\pgfusepath{stroke,fill}%
\end{pgfscope}%
\begin{pgfscope}%
\pgfpathrectangle{\pgfqpoint{0.100000in}{0.212622in}}{\pgfqpoint{3.696000in}{3.696000in}}%
\pgfusepath{clip}%
\pgfsetbuttcap%
\pgfsetroundjoin%
\definecolor{currentfill}{rgb}{0.121569,0.466667,0.705882}%
\pgfsetfillcolor{currentfill}%
\pgfsetfillopacity{0.878942}%
\pgfsetlinewidth{1.003750pt}%
\definecolor{currentstroke}{rgb}{0.121569,0.466667,0.705882}%
\pgfsetstrokecolor{currentstroke}%
\pgfsetstrokeopacity{0.878942}%
\pgfsetdash{}{0pt}%
\pgfpathmoveto{\pgfqpoint{1.749210in}{2.072729in}}%
\pgfpathcurveto{\pgfqpoint{1.757446in}{2.072729in}}{\pgfqpoint{1.765346in}{2.076001in}}{\pgfqpoint{1.771170in}{2.081825in}}%
\pgfpathcurveto{\pgfqpoint{1.776994in}{2.087649in}}{\pgfqpoint{1.780267in}{2.095549in}}{\pgfqpoint{1.780267in}{2.103785in}}%
\pgfpathcurveto{\pgfqpoint{1.780267in}{2.112021in}}{\pgfqpoint{1.776994in}{2.119921in}}{\pgfqpoint{1.771170in}{2.125745in}}%
\pgfpathcurveto{\pgfqpoint{1.765346in}{2.131569in}}{\pgfqpoint{1.757446in}{2.134842in}}{\pgfqpoint{1.749210in}{2.134842in}}%
\pgfpathcurveto{\pgfqpoint{1.740974in}{2.134842in}}{\pgfqpoint{1.733074in}{2.131569in}}{\pgfqpoint{1.727250in}{2.125745in}}%
\pgfpathcurveto{\pgfqpoint{1.721426in}{2.119921in}}{\pgfqpoint{1.718154in}{2.112021in}}{\pgfqpoint{1.718154in}{2.103785in}}%
\pgfpathcurveto{\pgfqpoint{1.718154in}{2.095549in}}{\pgfqpoint{1.721426in}{2.087649in}}{\pgfqpoint{1.727250in}{2.081825in}}%
\pgfpathcurveto{\pgfqpoint{1.733074in}{2.076001in}}{\pgfqpoint{1.740974in}{2.072729in}}{\pgfqpoint{1.749210in}{2.072729in}}%
\pgfpathclose%
\pgfusepath{stroke,fill}%
\end{pgfscope}%
\begin{pgfscope}%
\pgfpathrectangle{\pgfqpoint{0.100000in}{0.212622in}}{\pgfqpoint{3.696000in}{3.696000in}}%
\pgfusepath{clip}%
\pgfsetbuttcap%
\pgfsetroundjoin%
\definecolor{currentfill}{rgb}{0.121569,0.466667,0.705882}%
\pgfsetfillcolor{currentfill}%
\pgfsetfillopacity{0.879230}%
\pgfsetlinewidth{1.003750pt}%
\definecolor{currentstroke}{rgb}{0.121569,0.466667,0.705882}%
\pgfsetstrokecolor{currentstroke}%
\pgfsetstrokeopacity{0.879230}%
\pgfsetdash{}{0pt}%
\pgfpathmoveto{\pgfqpoint{1.748530in}{2.071243in}}%
\pgfpathcurveto{\pgfqpoint{1.756767in}{2.071243in}}{\pgfqpoint{1.764667in}{2.074515in}}{\pgfqpoint{1.770491in}{2.080339in}}%
\pgfpathcurveto{\pgfqpoint{1.776315in}{2.086163in}}{\pgfqpoint{1.779587in}{2.094063in}}{\pgfqpoint{1.779587in}{2.102299in}}%
\pgfpathcurveto{\pgfqpoint{1.779587in}{2.110535in}}{\pgfqpoint{1.776315in}{2.118435in}}{\pgfqpoint{1.770491in}{2.124259in}}%
\pgfpathcurveto{\pgfqpoint{1.764667in}{2.130083in}}{\pgfqpoint{1.756767in}{2.133356in}}{\pgfqpoint{1.748530in}{2.133356in}}%
\pgfpathcurveto{\pgfqpoint{1.740294in}{2.133356in}}{\pgfqpoint{1.732394in}{2.130083in}}{\pgfqpoint{1.726570in}{2.124259in}}%
\pgfpathcurveto{\pgfqpoint{1.720746in}{2.118435in}}{\pgfqpoint{1.717474in}{2.110535in}}{\pgfqpoint{1.717474in}{2.102299in}}%
\pgfpathcurveto{\pgfqpoint{1.717474in}{2.094063in}}{\pgfqpoint{1.720746in}{2.086163in}}{\pgfqpoint{1.726570in}{2.080339in}}%
\pgfpathcurveto{\pgfqpoint{1.732394in}{2.074515in}}{\pgfqpoint{1.740294in}{2.071243in}}{\pgfqpoint{1.748530in}{2.071243in}}%
\pgfpathclose%
\pgfusepath{stroke,fill}%
\end{pgfscope}%
\begin{pgfscope}%
\pgfpathrectangle{\pgfqpoint{0.100000in}{0.212622in}}{\pgfqpoint{3.696000in}{3.696000in}}%
\pgfusepath{clip}%
\pgfsetbuttcap%
\pgfsetroundjoin%
\definecolor{currentfill}{rgb}{0.121569,0.466667,0.705882}%
\pgfsetfillcolor{currentfill}%
\pgfsetfillopacity{0.879240}%
\pgfsetlinewidth{1.003750pt}%
\definecolor{currentstroke}{rgb}{0.121569,0.466667,0.705882}%
\pgfsetstrokecolor{currentstroke}%
\pgfsetstrokeopacity{0.879240}%
\pgfsetdash{}{0pt}%
\pgfpathmoveto{\pgfqpoint{2.259876in}{2.247586in}}%
\pgfpathcurveto{\pgfqpoint{2.268112in}{2.247586in}}{\pgfqpoint{2.276012in}{2.250858in}}{\pgfqpoint{2.281836in}{2.256682in}}%
\pgfpathcurveto{\pgfqpoint{2.287660in}{2.262506in}}{\pgfqpoint{2.290933in}{2.270406in}}{\pgfqpoint{2.290933in}{2.278642in}}%
\pgfpathcurveto{\pgfqpoint{2.290933in}{2.286879in}}{\pgfqpoint{2.287660in}{2.294779in}}{\pgfqpoint{2.281836in}{2.300603in}}%
\pgfpathcurveto{\pgfqpoint{2.276012in}{2.306427in}}{\pgfqpoint{2.268112in}{2.309699in}}{\pgfqpoint{2.259876in}{2.309699in}}%
\pgfpathcurveto{\pgfqpoint{2.251640in}{2.309699in}}{\pgfqpoint{2.243740in}{2.306427in}}{\pgfqpoint{2.237916in}{2.300603in}}%
\pgfpathcurveto{\pgfqpoint{2.232092in}{2.294779in}}{\pgfqpoint{2.228820in}{2.286879in}}{\pgfqpoint{2.228820in}{2.278642in}}%
\pgfpathcurveto{\pgfqpoint{2.228820in}{2.270406in}}{\pgfqpoint{2.232092in}{2.262506in}}{\pgfqpoint{2.237916in}{2.256682in}}%
\pgfpathcurveto{\pgfqpoint{2.243740in}{2.250858in}}{\pgfqpoint{2.251640in}{2.247586in}}{\pgfqpoint{2.259876in}{2.247586in}}%
\pgfpathclose%
\pgfusepath{stroke,fill}%
\end{pgfscope}%
\begin{pgfscope}%
\pgfpathrectangle{\pgfqpoint{0.100000in}{0.212622in}}{\pgfqpoint{3.696000in}{3.696000in}}%
\pgfusepath{clip}%
\pgfsetbuttcap%
\pgfsetroundjoin%
\definecolor{currentfill}{rgb}{0.121569,0.466667,0.705882}%
\pgfsetfillcolor{currentfill}%
\pgfsetfillopacity{0.879366}%
\pgfsetlinewidth{1.003750pt}%
\definecolor{currentstroke}{rgb}{0.121569,0.466667,0.705882}%
\pgfsetstrokecolor{currentstroke}%
\pgfsetstrokeopacity{0.879366}%
\pgfsetdash{}{0pt}%
\pgfpathmoveto{\pgfqpoint{1.748045in}{2.070456in}}%
\pgfpathcurveto{\pgfqpoint{1.756282in}{2.070456in}}{\pgfqpoint{1.764182in}{2.073728in}}{\pgfqpoint{1.770006in}{2.079552in}}%
\pgfpathcurveto{\pgfqpoint{1.775830in}{2.085376in}}{\pgfqpoint{1.779102in}{2.093276in}}{\pgfqpoint{1.779102in}{2.101512in}}%
\pgfpathcurveto{\pgfqpoint{1.779102in}{2.109748in}}{\pgfqpoint{1.775830in}{2.117648in}}{\pgfqpoint{1.770006in}{2.123472in}}%
\pgfpathcurveto{\pgfqpoint{1.764182in}{2.129296in}}{\pgfqpoint{1.756282in}{2.132569in}}{\pgfqpoint{1.748045in}{2.132569in}}%
\pgfpathcurveto{\pgfqpoint{1.739809in}{2.132569in}}{\pgfqpoint{1.731909in}{2.129296in}}{\pgfqpoint{1.726085in}{2.123472in}}%
\pgfpathcurveto{\pgfqpoint{1.720261in}{2.117648in}}{\pgfqpoint{1.716989in}{2.109748in}}{\pgfqpoint{1.716989in}{2.101512in}}%
\pgfpathcurveto{\pgfqpoint{1.716989in}{2.093276in}}{\pgfqpoint{1.720261in}{2.085376in}}{\pgfqpoint{1.726085in}{2.079552in}}%
\pgfpathcurveto{\pgfqpoint{1.731909in}{2.073728in}}{\pgfqpoint{1.739809in}{2.070456in}}{\pgfqpoint{1.748045in}{2.070456in}}%
\pgfpathclose%
\pgfusepath{stroke,fill}%
\end{pgfscope}%
\begin{pgfscope}%
\pgfpathrectangle{\pgfqpoint{0.100000in}{0.212622in}}{\pgfqpoint{3.696000in}{3.696000in}}%
\pgfusepath{clip}%
\pgfsetbuttcap%
\pgfsetroundjoin%
\definecolor{currentfill}{rgb}{0.121569,0.466667,0.705882}%
\pgfsetfillcolor{currentfill}%
\pgfsetfillopacity{0.879699}%
\pgfsetlinewidth{1.003750pt}%
\definecolor{currentstroke}{rgb}{0.121569,0.466667,0.705882}%
\pgfsetstrokecolor{currentstroke}%
\pgfsetstrokeopacity{0.879699}%
\pgfsetdash{}{0pt}%
\pgfpathmoveto{\pgfqpoint{1.747253in}{2.068359in}}%
\pgfpathcurveto{\pgfqpoint{1.755490in}{2.068359in}}{\pgfqpoint{1.763390in}{2.071631in}}{\pgfqpoint{1.769214in}{2.077455in}}%
\pgfpathcurveto{\pgfqpoint{1.775038in}{2.083279in}}{\pgfqpoint{1.778310in}{2.091179in}}{\pgfqpoint{1.778310in}{2.099415in}}%
\pgfpathcurveto{\pgfqpoint{1.778310in}{2.107652in}}{\pgfqpoint{1.775038in}{2.115552in}}{\pgfqpoint{1.769214in}{2.121376in}}%
\pgfpathcurveto{\pgfqpoint{1.763390in}{2.127199in}}{\pgfqpoint{1.755490in}{2.130472in}}{\pgfqpoint{1.747253in}{2.130472in}}%
\pgfpathcurveto{\pgfqpoint{1.739017in}{2.130472in}}{\pgfqpoint{1.731117in}{2.127199in}}{\pgfqpoint{1.725293in}{2.121376in}}%
\pgfpathcurveto{\pgfqpoint{1.719469in}{2.115552in}}{\pgfqpoint{1.716197in}{2.107652in}}{\pgfqpoint{1.716197in}{2.099415in}}%
\pgfpathcurveto{\pgfqpoint{1.716197in}{2.091179in}}{\pgfqpoint{1.719469in}{2.083279in}}{\pgfqpoint{1.725293in}{2.077455in}}%
\pgfpathcurveto{\pgfqpoint{1.731117in}{2.071631in}}{\pgfqpoint{1.739017in}{2.068359in}}{\pgfqpoint{1.747253in}{2.068359in}}%
\pgfpathclose%
\pgfusepath{stroke,fill}%
\end{pgfscope}%
\begin{pgfscope}%
\pgfpathrectangle{\pgfqpoint{0.100000in}{0.212622in}}{\pgfqpoint{3.696000in}{3.696000in}}%
\pgfusepath{clip}%
\pgfsetbuttcap%
\pgfsetroundjoin%
\definecolor{currentfill}{rgb}{0.121569,0.466667,0.705882}%
\pgfsetfillcolor{currentfill}%
\pgfsetfillopacity{0.879785}%
\pgfsetlinewidth{1.003750pt}%
\definecolor{currentstroke}{rgb}{0.121569,0.466667,0.705882}%
\pgfsetstrokecolor{currentstroke}%
\pgfsetstrokeopacity{0.879785}%
\pgfsetdash{}{0pt}%
\pgfpathmoveto{\pgfqpoint{2.257542in}{2.244125in}}%
\pgfpathcurveto{\pgfqpoint{2.265778in}{2.244125in}}{\pgfqpoint{2.273678in}{2.247398in}}{\pgfqpoint{2.279502in}{2.253222in}}%
\pgfpathcurveto{\pgfqpoint{2.285326in}{2.259045in}}{\pgfqpoint{2.288599in}{2.266946in}}{\pgfqpoint{2.288599in}{2.275182in}}%
\pgfpathcurveto{\pgfqpoint{2.288599in}{2.283418in}}{\pgfqpoint{2.285326in}{2.291318in}}{\pgfqpoint{2.279502in}{2.297142in}}%
\pgfpathcurveto{\pgfqpoint{2.273678in}{2.302966in}}{\pgfqpoint{2.265778in}{2.306238in}}{\pgfqpoint{2.257542in}{2.306238in}}%
\pgfpathcurveto{\pgfqpoint{2.249306in}{2.306238in}}{\pgfqpoint{2.241406in}{2.302966in}}{\pgfqpoint{2.235582in}{2.297142in}}%
\pgfpathcurveto{\pgfqpoint{2.229758in}{2.291318in}}{\pgfqpoint{2.226486in}{2.283418in}}{\pgfqpoint{2.226486in}{2.275182in}}%
\pgfpathcurveto{\pgfqpoint{2.226486in}{2.266946in}}{\pgfqpoint{2.229758in}{2.259045in}}{\pgfqpoint{2.235582in}{2.253222in}}%
\pgfpathcurveto{\pgfqpoint{2.241406in}{2.247398in}}{\pgfqpoint{2.249306in}{2.244125in}}{\pgfqpoint{2.257542in}{2.244125in}}%
\pgfpathclose%
\pgfusepath{stroke,fill}%
\end{pgfscope}%
\begin{pgfscope}%
\pgfpathrectangle{\pgfqpoint{0.100000in}{0.212622in}}{\pgfqpoint{3.696000in}{3.696000in}}%
\pgfusepath{clip}%
\pgfsetbuttcap%
\pgfsetroundjoin%
\definecolor{currentfill}{rgb}{0.121569,0.466667,0.705882}%
\pgfsetfillcolor{currentfill}%
\pgfsetfillopacity{0.880107}%
\pgfsetlinewidth{1.003750pt}%
\definecolor{currentstroke}{rgb}{0.121569,0.466667,0.705882}%
\pgfsetstrokecolor{currentstroke}%
\pgfsetstrokeopacity{0.880107}%
\pgfsetdash{}{0pt}%
\pgfpathmoveto{\pgfqpoint{1.746302in}{2.065827in}}%
\pgfpathcurveto{\pgfqpoint{1.754538in}{2.065827in}}{\pgfqpoint{1.762438in}{2.069100in}}{\pgfqpoint{1.768262in}{2.074924in}}%
\pgfpathcurveto{\pgfqpoint{1.774086in}{2.080748in}}{\pgfqpoint{1.777358in}{2.088648in}}{\pgfqpoint{1.777358in}{2.096884in}}%
\pgfpathcurveto{\pgfqpoint{1.777358in}{2.105120in}}{\pgfqpoint{1.774086in}{2.113020in}}{\pgfqpoint{1.768262in}{2.118844in}}%
\pgfpathcurveto{\pgfqpoint{1.762438in}{2.124668in}}{\pgfqpoint{1.754538in}{2.127940in}}{\pgfqpoint{1.746302in}{2.127940in}}%
\pgfpathcurveto{\pgfqpoint{1.738066in}{2.127940in}}{\pgfqpoint{1.730166in}{2.124668in}}{\pgfqpoint{1.724342in}{2.118844in}}%
\pgfpathcurveto{\pgfqpoint{1.718518in}{2.113020in}}{\pgfqpoint{1.715245in}{2.105120in}}{\pgfqpoint{1.715245in}{2.096884in}}%
\pgfpathcurveto{\pgfqpoint{1.715245in}{2.088648in}}{\pgfqpoint{1.718518in}{2.080748in}}{\pgfqpoint{1.724342in}{2.074924in}}%
\pgfpathcurveto{\pgfqpoint{1.730166in}{2.069100in}}{\pgfqpoint{1.738066in}{2.065827in}}{\pgfqpoint{1.746302in}{2.065827in}}%
\pgfpathclose%
\pgfusepath{stroke,fill}%
\end{pgfscope}%
\begin{pgfscope}%
\pgfpathrectangle{\pgfqpoint{0.100000in}{0.212622in}}{\pgfqpoint{3.696000in}{3.696000in}}%
\pgfusepath{clip}%
\pgfsetbuttcap%
\pgfsetroundjoin%
\definecolor{currentfill}{rgb}{0.121569,0.466667,0.705882}%
\pgfsetfillcolor{currentfill}%
\pgfsetfillopacity{0.880173}%
\pgfsetlinewidth{1.003750pt}%
\definecolor{currentstroke}{rgb}{0.121569,0.466667,0.705882}%
\pgfsetstrokecolor{currentstroke}%
\pgfsetstrokeopacity{0.880173}%
\pgfsetdash{}{0pt}%
\pgfpathmoveto{\pgfqpoint{1.540774in}{1.217750in}}%
\pgfpathcurveto{\pgfqpoint{1.549010in}{1.217750in}}{\pgfqpoint{1.556910in}{1.221023in}}{\pgfqpoint{1.562734in}{1.226847in}}%
\pgfpathcurveto{\pgfqpoint{1.568558in}{1.232671in}}{\pgfqpoint{1.571830in}{1.240571in}}{\pgfqpoint{1.571830in}{1.248807in}}%
\pgfpathcurveto{\pgfqpoint{1.571830in}{1.257043in}}{\pgfqpoint{1.568558in}{1.264943in}}{\pgfqpoint{1.562734in}{1.270767in}}%
\pgfpathcurveto{\pgfqpoint{1.556910in}{1.276591in}}{\pgfqpoint{1.549010in}{1.279863in}}{\pgfqpoint{1.540774in}{1.279863in}}%
\pgfpathcurveto{\pgfqpoint{1.532538in}{1.279863in}}{\pgfqpoint{1.524638in}{1.276591in}}{\pgfqpoint{1.518814in}{1.270767in}}%
\pgfpathcurveto{\pgfqpoint{1.512990in}{1.264943in}}{\pgfqpoint{1.509717in}{1.257043in}}{\pgfqpoint{1.509717in}{1.248807in}}%
\pgfpathcurveto{\pgfqpoint{1.509717in}{1.240571in}}{\pgfqpoint{1.512990in}{1.232671in}}{\pgfqpoint{1.518814in}{1.226847in}}%
\pgfpathcurveto{\pgfqpoint{1.524638in}{1.221023in}}{\pgfqpoint{1.532538in}{1.217750in}}{\pgfqpoint{1.540774in}{1.217750in}}%
\pgfpathclose%
\pgfusepath{stroke,fill}%
\end{pgfscope}%
\begin{pgfscope}%
\pgfpathrectangle{\pgfqpoint{0.100000in}{0.212622in}}{\pgfqpoint{3.696000in}{3.696000in}}%
\pgfusepath{clip}%
\pgfsetbuttcap%
\pgfsetroundjoin%
\definecolor{currentfill}{rgb}{0.121569,0.466667,0.705882}%
\pgfsetfillcolor{currentfill}%
\pgfsetfillopacity{0.880199}%
\pgfsetlinewidth{1.003750pt}%
\definecolor{currentstroke}{rgb}{0.121569,0.466667,0.705882}%
\pgfsetstrokecolor{currentstroke}%
\pgfsetstrokeopacity{0.880199}%
\pgfsetdash{}{0pt}%
\pgfpathmoveto{\pgfqpoint{2.256913in}{2.241624in}}%
\pgfpathcurveto{\pgfqpoint{2.265149in}{2.241624in}}{\pgfqpoint{2.273049in}{2.244897in}}{\pgfqpoint{2.278873in}{2.250721in}}%
\pgfpathcurveto{\pgfqpoint{2.284697in}{2.256545in}}{\pgfqpoint{2.287969in}{2.264445in}}{\pgfqpoint{2.287969in}{2.272681in}}%
\pgfpathcurveto{\pgfqpoint{2.287969in}{2.280917in}}{\pgfqpoint{2.284697in}{2.288817in}}{\pgfqpoint{2.278873in}{2.294641in}}%
\pgfpathcurveto{\pgfqpoint{2.273049in}{2.300465in}}{\pgfqpoint{2.265149in}{2.303737in}}{\pgfqpoint{2.256913in}{2.303737in}}%
\pgfpathcurveto{\pgfqpoint{2.248677in}{2.303737in}}{\pgfqpoint{2.240777in}{2.300465in}}{\pgfqpoint{2.234953in}{2.294641in}}%
\pgfpathcurveto{\pgfqpoint{2.229129in}{2.288817in}}{\pgfqpoint{2.225856in}{2.280917in}}{\pgfqpoint{2.225856in}{2.272681in}}%
\pgfpathcurveto{\pgfqpoint{2.225856in}{2.264445in}}{\pgfqpoint{2.229129in}{2.256545in}}{\pgfqpoint{2.234953in}{2.250721in}}%
\pgfpathcurveto{\pgfqpoint{2.240777in}{2.244897in}}{\pgfqpoint{2.248677in}{2.241624in}}{\pgfqpoint{2.256913in}{2.241624in}}%
\pgfpathclose%
\pgfusepath{stroke,fill}%
\end{pgfscope}%
\begin{pgfscope}%
\pgfpathrectangle{\pgfqpoint{0.100000in}{0.212622in}}{\pgfqpoint{3.696000in}{3.696000in}}%
\pgfusepath{clip}%
\pgfsetbuttcap%
\pgfsetroundjoin%
\definecolor{currentfill}{rgb}{0.121569,0.466667,0.705882}%
\pgfsetfillcolor{currentfill}%
\pgfsetfillopacity{0.880308}%
\pgfsetlinewidth{1.003750pt}%
\definecolor{currentstroke}{rgb}{0.121569,0.466667,0.705882}%
\pgfsetstrokecolor{currentstroke}%
\pgfsetstrokeopacity{0.880308}%
\pgfsetdash{}{0pt}%
\pgfpathmoveto{\pgfqpoint{1.745541in}{2.064591in}}%
\pgfpathcurveto{\pgfqpoint{1.753777in}{2.064591in}}{\pgfqpoint{1.761677in}{2.067863in}}{\pgfqpoint{1.767501in}{2.073687in}}%
\pgfpathcurveto{\pgfqpoint{1.773325in}{2.079511in}}{\pgfqpoint{1.776598in}{2.087411in}}{\pgfqpoint{1.776598in}{2.095648in}}%
\pgfpathcurveto{\pgfqpoint{1.776598in}{2.103884in}}{\pgfqpoint{1.773325in}{2.111784in}}{\pgfqpoint{1.767501in}{2.117608in}}%
\pgfpathcurveto{\pgfqpoint{1.761677in}{2.123432in}}{\pgfqpoint{1.753777in}{2.126704in}}{\pgfqpoint{1.745541in}{2.126704in}}%
\pgfpathcurveto{\pgfqpoint{1.737305in}{2.126704in}}{\pgfqpoint{1.729405in}{2.123432in}}{\pgfqpoint{1.723581in}{2.117608in}}%
\pgfpathcurveto{\pgfqpoint{1.717757in}{2.111784in}}{\pgfqpoint{1.714485in}{2.103884in}}{\pgfqpoint{1.714485in}{2.095648in}}%
\pgfpathcurveto{\pgfqpoint{1.714485in}{2.087411in}}{\pgfqpoint{1.717757in}{2.079511in}}{\pgfqpoint{1.723581in}{2.073687in}}%
\pgfpathcurveto{\pgfqpoint{1.729405in}{2.067863in}}{\pgfqpoint{1.737305in}{2.064591in}}{\pgfqpoint{1.745541in}{2.064591in}}%
\pgfpathclose%
\pgfusepath{stroke,fill}%
\end{pgfscope}%
\begin{pgfscope}%
\pgfpathrectangle{\pgfqpoint{0.100000in}{0.212622in}}{\pgfqpoint{3.696000in}{3.696000in}}%
\pgfusepath{clip}%
\pgfsetbuttcap%
\pgfsetroundjoin%
\definecolor{currentfill}{rgb}{0.121569,0.466667,0.705882}%
\pgfsetfillcolor{currentfill}%
\pgfsetfillopacity{0.880428}%
\pgfsetlinewidth{1.003750pt}%
\definecolor{currentstroke}{rgb}{0.121569,0.466667,0.705882}%
\pgfsetstrokecolor{currentstroke}%
\pgfsetstrokeopacity{0.880428}%
\pgfsetdash{}{0pt}%
\pgfpathmoveto{\pgfqpoint{1.745139in}{2.063927in}}%
\pgfpathcurveto{\pgfqpoint{1.753375in}{2.063927in}}{\pgfqpoint{1.761275in}{2.067200in}}{\pgfqpoint{1.767099in}{2.073024in}}%
\pgfpathcurveto{\pgfqpoint{1.772923in}{2.078847in}}{\pgfqpoint{1.776196in}{2.086747in}}{\pgfqpoint{1.776196in}{2.094984in}}%
\pgfpathcurveto{\pgfqpoint{1.776196in}{2.103220in}}{\pgfqpoint{1.772923in}{2.111120in}}{\pgfqpoint{1.767099in}{2.116944in}}%
\pgfpathcurveto{\pgfqpoint{1.761275in}{2.122768in}}{\pgfqpoint{1.753375in}{2.126040in}}{\pgfqpoint{1.745139in}{2.126040in}}%
\pgfpathcurveto{\pgfqpoint{1.736903in}{2.126040in}}{\pgfqpoint{1.729003in}{2.122768in}}{\pgfqpoint{1.723179in}{2.116944in}}%
\pgfpathcurveto{\pgfqpoint{1.717355in}{2.111120in}}{\pgfqpoint{1.714083in}{2.103220in}}{\pgfqpoint{1.714083in}{2.094984in}}%
\pgfpathcurveto{\pgfqpoint{1.714083in}{2.086747in}}{\pgfqpoint{1.717355in}{2.078847in}}{\pgfqpoint{1.723179in}{2.073024in}}%
\pgfpathcurveto{\pgfqpoint{1.729003in}{2.067200in}}{\pgfqpoint{1.736903in}{2.063927in}}{\pgfqpoint{1.745139in}{2.063927in}}%
\pgfpathclose%
\pgfusepath{stroke,fill}%
\end{pgfscope}%
\begin{pgfscope}%
\pgfpathrectangle{\pgfqpoint{0.100000in}{0.212622in}}{\pgfqpoint{3.696000in}{3.696000in}}%
\pgfusepath{clip}%
\pgfsetbuttcap%
\pgfsetroundjoin%
\definecolor{currentfill}{rgb}{0.121569,0.466667,0.705882}%
\pgfsetfillcolor{currentfill}%
\pgfsetfillopacity{0.880492}%
\pgfsetlinewidth{1.003750pt}%
\definecolor{currentstroke}{rgb}{0.121569,0.466667,0.705882}%
\pgfsetstrokecolor{currentstroke}%
\pgfsetstrokeopacity{0.880492}%
\pgfsetdash{}{0pt}%
\pgfpathmoveto{\pgfqpoint{1.745020in}{2.063474in}}%
\pgfpathcurveto{\pgfqpoint{1.753257in}{2.063474in}}{\pgfqpoint{1.761157in}{2.066746in}}{\pgfqpoint{1.766981in}{2.072570in}}%
\pgfpathcurveto{\pgfqpoint{1.772804in}{2.078394in}}{\pgfqpoint{1.776077in}{2.086294in}}{\pgfqpoint{1.776077in}{2.094530in}}%
\pgfpathcurveto{\pgfqpoint{1.776077in}{2.102766in}}{\pgfqpoint{1.772804in}{2.110666in}}{\pgfqpoint{1.766981in}{2.116490in}}%
\pgfpathcurveto{\pgfqpoint{1.761157in}{2.122314in}}{\pgfqpoint{1.753257in}{2.125587in}}{\pgfqpoint{1.745020in}{2.125587in}}%
\pgfpathcurveto{\pgfqpoint{1.736784in}{2.125587in}}{\pgfqpoint{1.728884in}{2.122314in}}{\pgfqpoint{1.723060in}{2.116490in}}%
\pgfpathcurveto{\pgfqpoint{1.717236in}{2.110666in}}{\pgfqpoint{1.713964in}{2.102766in}}{\pgfqpoint{1.713964in}{2.094530in}}%
\pgfpathcurveto{\pgfqpoint{1.713964in}{2.086294in}}{\pgfqpoint{1.717236in}{2.078394in}}{\pgfqpoint{1.723060in}{2.072570in}}%
\pgfpathcurveto{\pgfqpoint{1.728884in}{2.066746in}}{\pgfqpoint{1.736784in}{2.063474in}}{\pgfqpoint{1.745020in}{2.063474in}}%
\pgfpathclose%
\pgfusepath{stroke,fill}%
\end{pgfscope}%
\begin{pgfscope}%
\pgfpathrectangle{\pgfqpoint{0.100000in}{0.212622in}}{\pgfqpoint{3.696000in}{3.696000in}}%
\pgfusepath{clip}%
\pgfsetbuttcap%
\pgfsetroundjoin%
\definecolor{currentfill}{rgb}{0.121569,0.466667,0.705882}%
\pgfsetfillcolor{currentfill}%
\pgfsetfillopacity{0.880669}%
\pgfsetlinewidth{1.003750pt}%
\definecolor{currentstroke}{rgb}{0.121569,0.466667,0.705882}%
\pgfsetstrokecolor{currentstroke}%
\pgfsetstrokeopacity{0.880669}%
\pgfsetdash{}{0pt}%
\pgfpathmoveto{\pgfqpoint{1.744426in}{2.062449in}}%
\pgfpathcurveto{\pgfqpoint{1.752663in}{2.062449in}}{\pgfqpoint{1.760563in}{2.065722in}}{\pgfqpoint{1.766387in}{2.071546in}}%
\pgfpathcurveto{\pgfqpoint{1.772210in}{2.077370in}}{\pgfqpoint{1.775483in}{2.085270in}}{\pgfqpoint{1.775483in}{2.093506in}}%
\pgfpathcurveto{\pgfqpoint{1.775483in}{2.101742in}}{\pgfqpoint{1.772210in}{2.109642in}}{\pgfqpoint{1.766387in}{2.115466in}}%
\pgfpathcurveto{\pgfqpoint{1.760563in}{2.121290in}}{\pgfqpoint{1.752663in}{2.124562in}}{\pgfqpoint{1.744426in}{2.124562in}}%
\pgfpathcurveto{\pgfqpoint{1.736190in}{2.124562in}}{\pgfqpoint{1.728290in}{2.121290in}}{\pgfqpoint{1.722466in}{2.115466in}}%
\pgfpathcurveto{\pgfqpoint{1.716642in}{2.109642in}}{\pgfqpoint{1.713370in}{2.101742in}}{\pgfqpoint{1.713370in}{2.093506in}}%
\pgfpathcurveto{\pgfqpoint{1.713370in}{2.085270in}}{\pgfqpoint{1.716642in}{2.077370in}}{\pgfqpoint{1.722466in}{2.071546in}}%
\pgfpathcurveto{\pgfqpoint{1.728290in}{2.065722in}}{\pgfqpoint{1.736190in}{2.062449in}}{\pgfqpoint{1.744426in}{2.062449in}}%
\pgfpathclose%
\pgfusepath{stroke,fill}%
\end{pgfscope}%
\begin{pgfscope}%
\pgfpathrectangle{\pgfqpoint{0.100000in}{0.212622in}}{\pgfqpoint{3.696000in}{3.696000in}}%
\pgfusepath{clip}%
\pgfsetbuttcap%
\pgfsetroundjoin%
\definecolor{currentfill}{rgb}{0.121569,0.466667,0.705882}%
\pgfsetfillcolor{currentfill}%
\pgfsetfillopacity{0.880754}%
\pgfsetlinewidth{1.003750pt}%
\definecolor{currentstroke}{rgb}{0.121569,0.466667,0.705882}%
\pgfsetstrokecolor{currentstroke}%
\pgfsetstrokeopacity{0.880754}%
\pgfsetdash{}{0pt}%
\pgfpathmoveto{\pgfqpoint{1.744061in}{2.061896in}}%
\pgfpathcurveto{\pgfqpoint{1.752298in}{2.061896in}}{\pgfqpoint{1.760198in}{2.065169in}}{\pgfqpoint{1.766022in}{2.070993in}}%
\pgfpathcurveto{\pgfqpoint{1.771845in}{2.076817in}}{\pgfqpoint{1.775118in}{2.084717in}}{\pgfqpoint{1.775118in}{2.092953in}}%
\pgfpathcurveto{\pgfqpoint{1.775118in}{2.101189in}}{\pgfqpoint{1.771845in}{2.109089in}}{\pgfqpoint{1.766022in}{2.114913in}}%
\pgfpathcurveto{\pgfqpoint{1.760198in}{2.120737in}}{\pgfqpoint{1.752298in}{2.124009in}}{\pgfqpoint{1.744061in}{2.124009in}}%
\pgfpathcurveto{\pgfqpoint{1.735825in}{2.124009in}}{\pgfqpoint{1.727925in}{2.120737in}}{\pgfqpoint{1.722101in}{2.114913in}}%
\pgfpathcurveto{\pgfqpoint{1.716277in}{2.109089in}}{\pgfqpoint{1.713005in}{2.101189in}}{\pgfqpoint{1.713005in}{2.092953in}}%
\pgfpathcurveto{\pgfqpoint{1.713005in}{2.084717in}}{\pgfqpoint{1.716277in}{2.076817in}}{\pgfqpoint{1.722101in}{2.070993in}}%
\pgfpathcurveto{\pgfqpoint{1.727925in}{2.065169in}}{\pgfqpoint{1.735825in}{2.061896in}}{\pgfqpoint{1.744061in}{2.061896in}}%
\pgfpathclose%
\pgfusepath{stroke,fill}%
\end{pgfscope}%
\begin{pgfscope}%
\pgfpathrectangle{\pgfqpoint{0.100000in}{0.212622in}}{\pgfqpoint{3.696000in}{3.696000in}}%
\pgfusepath{clip}%
\pgfsetbuttcap%
\pgfsetroundjoin%
\definecolor{currentfill}{rgb}{0.121569,0.466667,0.705882}%
\pgfsetfillcolor{currentfill}%
\pgfsetfillopacity{0.880947}%
\pgfsetlinewidth{1.003750pt}%
\definecolor{currentstroke}{rgb}{0.121569,0.466667,0.705882}%
\pgfsetstrokecolor{currentstroke}%
\pgfsetstrokeopacity{0.880947}%
\pgfsetdash{}{0pt}%
\pgfpathmoveto{\pgfqpoint{1.557556in}{1.206242in}}%
\pgfpathcurveto{\pgfqpoint{1.565792in}{1.206242in}}{\pgfqpoint{1.573692in}{1.209515in}}{\pgfqpoint{1.579516in}{1.215339in}}%
\pgfpathcurveto{\pgfqpoint{1.585340in}{1.221163in}}{\pgfqpoint{1.588612in}{1.229063in}}{\pgfqpoint{1.588612in}{1.237299in}}%
\pgfpathcurveto{\pgfqpoint{1.588612in}{1.245535in}}{\pgfqpoint{1.585340in}{1.253435in}}{\pgfqpoint{1.579516in}{1.259259in}}%
\pgfpathcurveto{\pgfqpoint{1.573692in}{1.265083in}}{\pgfqpoint{1.565792in}{1.268355in}}{\pgfqpoint{1.557556in}{1.268355in}}%
\pgfpathcurveto{\pgfqpoint{1.549319in}{1.268355in}}{\pgfqpoint{1.541419in}{1.265083in}}{\pgfqpoint{1.535595in}{1.259259in}}%
\pgfpathcurveto{\pgfqpoint{1.529771in}{1.253435in}}{\pgfqpoint{1.526499in}{1.245535in}}{\pgfqpoint{1.526499in}{1.237299in}}%
\pgfpathcurveto{\pgfqpoint{1.526499in}{1.229063in}}{\pgfqpoint{1.529771in}{1.221163in}}{\pgfqpoint{1.535595in}{1.215339in}}%
\pgfpathcurveto{\pgfqpoint{1.541419in}{1.209515in}}{\pgfqpoint{1.549319in}{1.206242in}}{\pgfqpoint{1.557556in}{1.206242in}}%
\pgfpathclose%
\pgfusepath{stroke,fill}%
\end{pgfscope}%
\begin{pgfscope}%
\pgfpathrectangle{\pgfqpoint{0.100000in}{0.212622in}}{\pgfqpoint{3.696000in}{3.696000in}}%
\pgfusepath{clip}%
\pgfsetbuttcap%
\pgfsetroundjoin%
\definecolor{currentfill}{rgb}{0.121569,0.466667,0.705882}%
\pgfsetfillcolor{currentfill}%
\pgfsetfillopacity{0.881009}%
\pgfsetlinewidth{1.003750pt}%
\definecolor{currentstroke}{rgb}{0.121569,0.466667,0.705882}%
\pgfsetstrokecolor{currentstroke}%
\pgfsetstrokeopacity{0.881009}%
\pgfsetdash{}{0pt}%
\pgfpathmoveto{\pgfqpoint{2.255525in}{2.237399in}}%
\pgfpathcurveto{\pgfqpoint{2.263761in}{2.237399in}}{\pgfqpoint{2.271661in}{2.240671in}}{\pgfqpoint{2.277485in}{2.246495in}}%
\pgfpathcurveto{\pgfqpoint{2.283309in}{2.252319in}}{\pgfqpoint{2.286582in}{2.260219in}}{\pgfqpoint{2.286582in}{2.268455in}}%
\pgfpathcurveto{\pgfqpoint{2.286582in}{2.276692in}}{\pgfqpoint{2.283309in}{2.284592in}}{\pgfqpoint{2.277485in}{2.290416in}}%
\pgfpathcurveto{\pgfqpoint{2.271661in}{2.296240in}}{\pgfqpoint{2.263761in}{2.299512in}}{\pgfqpoint{2.255525in}{2.299512in}}%
\pgfpathcurveto{\pgfqpoint{2.247289in}{2.299512in}}{\pgfqpoint{2.239389in}{2.296240in}}{\pgfqpoint{2.233565in}{2.290416in}}%
\pgfpathcurveto{\pgfqpoint{2.227741in}{2.284592in}}{\pgfqpoint{2.224469in}{2.276692in}}{\pgfqpoint{2.224469in}{2.268455in}}%
\pgfpathcurveto{\pgfqpoint{2.224469in}{2.260219in}}{\pgfqpoint{2.227741in}{2.252319in}}{\pgfqpoint{2.233565in}{2.246495in}}%
\pgfpathcurveto{\pgfqpoint{2.239389in}{2.240671in}}{\pgfqpoint{2.247289in}{2.237399in}}{\pgfqpoint{2.255525in}{2.237399in}}%
\pgfpathclose%
\pgfusepath{stroke,fill}%
\end{pgfscope}%
\begin{pgfscope}%
\pgfpathrectangle{\pgfqpoint{0.100000in}{0.212622in}}{\pgfqpoint{3.696000in}{3.696000in}}%
\pgfusepath{clip}%
\pgfsetbuttcap%
\pgfsetroundjoin%
\definecolor{currentfill}{rgb}{0.121569,0.466667,0.705882}%
\pgfsetfillcolor{currentfill}%
\pgfsetfillopacity{0.881058}%
\pgfsetlinewidth{1.003750pt}%
\definecolor{currentstroke}{rgb}{0.121569,0.466667,0.705882}%
\pgfsetstrokecolor{currentstroke}%
\pgfsetstrokeopacity{0.881058}%
\pgfsetdash{}{0pt}%
\pgfpathmoveto{\pgfqpoint{1.743316in}{2.060206in}}%
\pgfpathcurveto{\pgfqpoint{1.751552in}{2.060206in}}{\pgfqpoint{1.759452in}{2.063478in}}{\pgfqpoint{1.765276in}{2.069302in}}%
\pgfpathcurveto{\pgfqpoint{1.771100in}{2.075126in}}{\pgfqpoint{1.774372in}{2.083026in}}{\pgfqpoint{1.774372in}{2.091262in}}%
\pgfpathcurveto{\pgfqpoint{1.774372in}{2.099499in}}{\pgfqpoint{1.771100in}{2.107399in}}{\pgfqpoint{1.765276in}{2.113223in}}%
\pgfpathcurveto{\pgfqpoint{1.759452in}{2.119047in}}{\pgfqpoint{1.751552in}{2.122319in}}{\pgfqpoint{1.743316in}{2.122319in}}%
\pgfpathcurveto{\pgfqpoint{1.735079in}{2.122319in}}{\pgfqpoint{1.727179in}{2.119047in}}{\pgfqpoint{1.721355in}{2.113223in}}%
\pgfpathcurveto{\pgfqpoint{1.715531in}{2.107399in}}{\pgfqpoint{1.712259in}{2.099499in}}{\pgfqpoint{1.712259in}{2.091262in}}%
\pgfpathcurveto{\pgfqpoint{1.712259in}{2.083026in}}{\pgfqpoint{1.715531in}{2.075126in}}{\pgfqpoint{1.721355in}{2.069302in}}%
\pgfpathcurveto{\pgfqpoint{1.727179in}{2.063478in}}{\pgfqpoint{1.735079in}{2.060206in}}{\pgfqpoint{1.743316in}{2.060206in}}%
\pgfpathclose%
\pgfusepath{stroke,fill}%
\end{pgfscope}%
\begin{pgfscope}%
\pgfpathrectangle{\pgfqpoint{0.100000in}{0.212622in}}{\pgfqpoint{3.696000in}{3.696000in}}%
\pgfusepath{clip}%
\pgfsetbuttcap%
\pgfsetroundjoin%
\definecolor{currentfill}{rgb}{0.121569,0.466667,0.705882}%
\pgfsetfillcolor{currentfill}%
\pgfsetfillopacity{0.881219}%
\pgfsetlinewidth{1.003750pt}%
\definecolor{currentstroke}{rgb}{0.121569,0.466667,0.705882}%
\pgfsetstrokecolor{currentstroke}%
\pgfsetstrokeopacity{0.881219}%
\pgfsetdash{}{0pt}%
\pgfpathmoveto{\pgfqpoint{1.742872in}{2.059275in}}%
\pgfpathcurveto{\pgfqpoint{1.751108in}{2.059275in}}{\pgfqpoint{1.759008in}{2.062548in}}{\pgfqpoint{1.764832in}{2.068372in}}%
\pgfpathcurveto{\pgfqpoint{1.770656in}{2.074195in}}{\pgfqpoint{1.773928in}{2.082096in}}{\pgfqpoint{1.773928in}{2.090332in}}%
\pgfpathcurveto{\pgfqpoint{1.773928in}{2.098568in}}{\pgfqpoint{1.770656in}{2.106468in}}{\pgfqpoint{1.764832in}{2.112292in}}%
\pgfpathcurveto{\pgfqpoint{1.759008in}{2.118116in}}{\pgfqpoint{1.751108in}{2.121388in}}{\pgfqpoint{1.742872in}{2.121388in}}%
\pgfpathcurveto{\pgfqpoint{1.734636in}{2.121388in}}{\pgfqpoint{1.726736in}{2.118116in}}{\pgfqpoint{1.720912in}{2.112292in}}%
\pgfpathcurveto{\pgfqpoint{1.715088in}{2.106468in}}{\pgfqpoint{1.711815in}{2.098568in}}{\pgfqpoint{1.711815in}{2.090332in}}%
\pgfpathcurveto{\pgfqpoint{1.711815in}{2.082096in}}{\pgfqpoint{1.715088in}{2.074195in}}{\pgfqpoint{1.720912in}{2.068372in}}%
\pgfpathcurveto{\pgfqpoint{1.726736in}{2.062548in}}{\pgfqpoint{1.734636in}{2.059275in}}{\pgfqpoint{1.742872in}{2.059275in}}%
\pgfpathclose%
\pgfusepath{stroke,fill}%
\end{pgfscope}%
\begin{pgfscope}%
\pgfpathrectangle{\pgfqpoint{0.100000in}{0.212622in}}{\pgfqpoint{3.696000in}{3.696000in}}%
\pgfusepath{clip}%
\pgfsetbuttcap%
\pgfsetroundjoin%
\definecolor{currentfill}{rgb}{0.121569,0.466667,0.705882}%
\pgfsetfillcolor{currentfill}%
\pgfsetfillopacity{0.881295}%
\pgfsetlinewidth{1.003750pt}%
\definecolor{currentstroke}{rgb}{0.121569,0.466667,0.705882}%
\pgfsetstrokecolor{currentstroke}%
\pgfsetstrokeopacity{0.881295}%
\pgfsetdash{}{0pt}%
\pgfpathmoveto{\pgfqpoint{1.742556in}{2.058814in}}%
\pgfpathcurveto{\pgfqpoint{1.750792in}{2.058814in}}{\pgfqpoint{1.758692in}{2.062086in}}{\pgfqpoint{1.764516in}{2.067910in}}%
\pgfpathcurveto{\pgfqpoint{1.770340in}{2.073734in}}{\pgfqpoint{1.773612in}{2.081634in}}{\pgfqpoint{1.773612in}{2.089870in}}%
\pgfpathcurveto{\pgfqpoint{1.773612in}{2.098107in}}{\pgfqpoint{1.770340in}{2.106007in}}{\pgfqpoint{1.764516in}{2.111831in}}%
\pgfpathcurveto{\pgfqpoint{1.758692in}{2.117654in}}{\pgfqpoint{1.750792in}{2.120927in}}{\pgfqpoint{1.742556in}{2.120927in}}%
\pgfpathcurveto{\pgfqpoint{1.734320in}{2.120927in}}{\pgfqpoint{1.726419in}{2.117654in}}{\pgfqpoint{1.720596in}{2.111831in}}%
\pgfpathcurveto{\pgfqpoint{1.714772in}{2.106007in}}{\pgfqpoint{1.711499in}{2.098107in}}{\pgfqpoint{1.711499in}{2.089870in}}%
\pgfpathcurveto{\pgfqpoint{1.711499in}{2.081634in}}{\pgfqpoint{1.714772in}{2.073734in}}{\pgfqpoint{1.720596in}{2.067910in}}%
\pgfpathcurveto{\pgfqpoint{1.726419in}{2.062086in}}{\pgfqpoint{1.734320in}{2.058814in}}{\pgfqpoint{1.742556in}{2.058814in}}%
\pgfpathclose%
\pgfusepath{stroke,fill}%
\end{pgfscope}%
\begin{pgfscope}%
\pgfpathrectangle{\pgfqpoint{0.100000in}{0.212622in}}{\pgfqpoint{3.696000in}{3.696000in}}%
\pgfusepath{clip}%
\pgfsetbuttcap%
\pgfsetroundjoin%
\definecolor{currentfill}{rgb}{0.121569,0.466667,0.705882}%
\pgfsetfillcolor{currentfill}%
\pgfsetfillopacity{0.881335}%
\pgfsetlinewidth{1.003750pt}%
\definecolor{currentstroke}{rgb}{0.121569,0.466667,0.705882}%
\pgfsetstrokecolor{currentstroke}%
\pgfsetstrokeopacity{0.881335}%
\pgfsetdash{}{0pt}%
\pgfpathmoveto{\pgfqpoint{2.254255in}{2.235763in}}%
\pgfpathcurveto{\pgfqpoint{2.262491in}{2.235763in}}{\pgfqpoint{2.270391in}{2.239035in}}{\pgfqpoint{2.276215in}{2.244859in}}%
\pgfpathcurveto{\pgfqpoint{2.282039in}{2.250683in}}{\pgfqpoint{2.285311in}{2.258583in}}{\pgfqpoint{2.285311in}{2.266819in}}%
\pgfpathcurveto{\pgfqpoint{2.285311in}{2.275055in}}{\pgfqpoint{2.282039in}{2.282955in}}{\pgfqpoint{2.276215in}{2.288779in}}%
\pgfpathcurveto{\pgfqpoint{2.270391in}{2.294603in}}{\pgfqpoint{2.262491in}{2.297876in}}{\pgfqpoint{2.254255in}{2.297876in}}%
\pgfpathcurveto{\pgfqpoint{2.246018in}{2.297876in}}{\pgfqpoint{2.238118in}{2.294603in}}{\pgfqpoint{2.232294in}{2.288779in}}%
\pgfpathcurveto{\pgfqpoint{2.226470in}{2.282955in}}{\pgfqpoint{2.223198in}{2.275055in}}{\pgfqpoint{2.223198in}{2.266819in}}%
\pgfpathcurveto{\pgfqpoint{2.223198in}{2.258583in}}{\pgfqpoint{2.226470in}{2.250683in}}{\pgfqpoint{2.232294in}{2.244859in}}%
\pgfpathcurveto{\pgfqpoint{2.238118in}{2.239035in}}{\pgfqpoint{2.246018in}{2.235763in}}{\pgfqpoint{2.254255in}{2.235763in}}%
\pgfpathclose%
\pgfusepath{stroke,fill}%
\end{pgfscope}%
\begin{pgfscope}%
\pgfpathrectangle{\pgfqpoint{0.100000in}{0.212622in}}{\pgfqpoint{3.696000in}{3.696000in}}%
\pgfusepath{clip}%
\pgfsetbuttcap%
\pgfsetroundjoin%
\definecolor{currentfill}{rgb}{0.121569,0.466667,0.705882}%
\pgfsetfillcolor{currentfill}%
\pgfsetfillopacity{0.881340}%
\pgfsetlinewidth{1.003750pt}%
\definecolor{currentstroke}{rgb}{0.121569,0.466667,0.705882}%
\pgfsetstrokecolor{currentstroke}%
\pgfsetstrokeopacity{0.881340}%
\pgfsetdash{}{0pt}%
\pgfpathmoveto{\pgfqpoint{1.742401in}{2.058545in}}%
\pgfpathcurveto{\pgfqpoint{1.750637in}{2.058545in}}{\pgfqpoint{1.758537in}{2.061818in}}{\pgfqpoint{1.764361in}{2.067642in}}%
\pgfpathcurveto{\pgfqpoint{1.770185in}{2.073466in}}{\pgfqpoint{1.773457in}{2.081366in}}{\pgfqpoint{1.773457in}{2.089602in}}%
\pgfpathcurveto{\pgfqpoint{1.773457in}{2.097838in}}{\pgfqpoint{1.770185in}{2.105738in}}{\pgfqpoint{1.764361in}{2.111562in}}%
\pgfpathcurveto{\pgfqpoint{1.758537in}{2.117386in}}{\pgfqpoint{1.750637in}{2.120658in}}{\pgfqpoint{1.742401in}{2.120658in}}%
\pgfpathcurveto{\pgfqpoint{1.734164in}{2.120658in}}{\pgfqpoint{1.726264in}{2.117386in}}{\pgfqpoint{1.720440in}{2.111562in}}%
\pgfpathcurveto{\pgfqpoint{1.714616in}{2.105738in}}{\pgfqpoint{1.711344in}{2.097838in}}{\pgfqpoint{1.711344in}{2.089602in}}%
\pgfpathcurveto{\pgfqpoint{1.711344in}{2.081366in}}{\pgfqpoint{1.714616in}{2.073466in}}{\pgfqpoint{1.720440in}{2.067642in}}%
\pgfpathcurveto{\pgfqpoint{1.726264in}{2.061818in}}{\pgfqpoint{1.734164in}{2.058545in}}{\pgfqpoint{1.742401in}{2.058545in}}%
\pgfpathclose%
\pgfusepath{stroke,fill}%
\end{pgfscope}%
\begin{pgfscope}%
\pgfpathrectangle{\pgfqpoint{0.100000in}{0.212622in}}{\pgfqpoint{3.696000in}{3.696000in}}%
\pgfusepath{clip}%
\pgfsetbuttcap%
\pgfsetroundjoin%
\definecolor{currentfill}{rgb}{0.121569,0.466667,0.705882}%
\pgfsetfillcolor{currentfill}%
\pgfsetfillopacity{0.881366}%
\pgfsetlinewidth{1.003750pt}%
\definecolor{currentstroke}{rgb}{0.121569,0.466667,0.705882}%
\pgfsetstrokecolor{currentstroke}%
\pgfsetstrokeopacity{0.881366}%
\pgfsetdash{}{0pt}%
\pgfpathmoveto{\pgfqpoint{1.742339in}{2.058374in}}%
\pgfpathcurveto{\pgfqpoint{1.750575in}{2.058374in}}{\pgfqpoint{1.758475in}{2.061647in}}{\pgfqpoint{1.764299in}{2.067471in}}%
\pgfpathcurveto{\pgfqpoint{1.770123in}{2.073295in}}{\pgfqpoint{1.773395in}{2.081195in}}{\pgfqpoint{1.773395in}{2.089431in}}%
\pgfpathcurveto{\pgfqpoint{1.773395in}{2.097667in}}{\pgfqpoint{1.770123in}{2.105567in}}{\pgfqpoint{1.764299in}{2.111391in}}%
\pgfpathcurveto{\pgfqpoint{1.758475in}{2.117215in}}{\pgfqpoint{1.750575in}{2.120487in}}{\pgfqpoint{1.742339in}{2.120487in}}%
\pgfpathcurveto{\pgfqpoint{1.734103in}{2.120487in}}{\pgfqpoint{1.726202in}{2.117215in}}{\pgfqpoint{1.720379in}{2.111391in}}%
\pgfpathcurveto{\pgfqpoint{1.714555in}{2.105567in}}{\pgfqpoint{1.711282in}{2.097667in}}{\pgfqpoint{1.711282in}{2.089431in}}%
\pgfpathcurveto{\pgfqpoint{1.711282in}{2.081195in}}{\pgfqpoint{1.714555in}{2.073295in}}{\pgfqpoint{1.720379in}{2.067471in}}%
\pgfpathcurveto{\pgfqpoint{1.726202in}{2.061647in}}{\pgfqpoint{1.734103in}{2.058374in}}{\pgfqpoint{1.742339in}{2.058374in}}%
\pgfpathclose%
\pgfusepath{stroke,fill}%
\end{pgfscope}%
\begin{pgfscope}%
\pgfpathrectangle{\pgfqpoint{0.100000in}{0.212622in}}{\pgfqpoint{3.696000in}{3.696000in}}%
\pgfusepath{clip}%
\pgfsetbuttcap%
\pgfsetroundjoin%
\definecolor{currentfill}{rgb}{0.121569,0.466667,0.705882}%
\pgfsetfillcolor{currentfill}%
\pgfsetfillopacity{0.881467}%
\pgfsetlinewidth{1.003750pt}%
\definecolor{currentstroke}{rgb}{0.121569,0.466667,0.705882}%
\pgfsetstrokecolor{currentstroke}%
\pgfsetstrokeopacity{0.881467}%
\pgfsetdash{}{0pt}%
\pgfpathmoveto{\pgfqpoint{1.741963in}{2.057700in}}%
\pgfpathcurveto{\pgfqpoint{1.750199in}{2.057700in}}{\pgfqpoint{1.758099in}{2.060972in}}{\pgfqpoint{1.763923in}{2.066796in}}%
\pgfpathcurveto{\pgfqpoint{1.769747in}{2.072620in}}{\pgfqpoint{1.773019in}{2.080520in}}{\pgfqpoint{1.773019in}{2.088757in}}%
\pgfpathcurveto{\pgfqpoint{1.773019in}{2.096993in}}{\pgfqpoint{1.769747in}{2.104893in}}{\pgfqpoint{1.763923in}{2.110717in}}%
\pgfpathcurveto{\pgfqpoint{1.758099in}{2.116541in}}{\pgfqpoint{1.750199in}{2.119813in}}{\pgfqpoint{1.741963in}{2.119813in}}%
\pgfpathcurveto{\pgfqpoint{1.733726in}{2.119813in}}{\pgfqpoint{1.725826in}{2.116541in}}{\pgfqpoint{1.720002in}{2.110717in}}%
\pgfpathcurveto{\pgfqpoint{1.714179in}{2.104893in}}{\pgfqpoint{1.710906in}{2.096993in}}{\pgfqpoint{1.710906in}{2.088757in}}%
\pgfpathcurveto{\pgfqpoint{1.710906in}{2.080520in}}{\pgfqpoint{1.714179in}{2.072620in}}{\pgfqpoint{1.720002in}{2.066796in}}%
\pgfpathcurveto{\pgfqpoint{1.725826in}{2.060972in}}{\pgfqpoint{1.733726in}{2.057700in}}{\pgfqpoint{1.741963in}{2.057700in}}%
\pgfpathclose%
\pgfusepath{stroke,fill}%
\end{pgfscope}%
\begin{pgfscope}%
\pgfpathrectangle{\pgfqpoint{0.100000in}{0.212622in}}{\pgfqpoint{3.696000in}{3.696000in}}%
\pgfusepath{clip}%
\pgfsetbuttcap%
\pgfsetroundjoin%
\definecolor{currentfill}{rgb}{0.121569,0.466667,0.705882}%
\pgfsetfillcolor{currentfill}%
\pgfsetfillopacity{0.881521}%
\pgfsetlinewidth{1.003750pt}%
\definecolor{currentstroke}{rgb}{0.121569,0.466667,0.705882}%
\pgfsetstrokecolor{currentstroke}%
\pgfsetstrokeopacity{0.881521}%
\pgfsetdash{}{0pt}%
\pgfpathmoveto{\pgfqpoint{1.741737in}{2.057353in}}%
\pgfpathcurveto{\pgfqpoint{1.749974in}{2.057353in}}{\pgfqpoint{1.757874in}{2.060626in}}{\pgfqpoint{1.763698in}{2.066450in}}%
\pgfpathcurveto{\pgfqpoint{1.769522in}{2.072274in}}{\pgfqpoint{1.772794in}{2.080174in}}{\pgfqpoint{1.772794in}{2.088410in}}%
\pgfpathcurveto{\pgfqpoint{1.772794in}{2.096646in}}{\pgfqpoint{1.769522in}{2.104546in}}{\pgfqpoint{1.763698in}{2.110370in}}%
\pgfpathcurveto{\pgfqpoint{1.757874in}{2.116194in}}{\pgfqpoint{1.749974in}{2.119466in}}{\pgfqpoint{1.741737in}{2.119466in}}%
\pgfpathcurveto{\pgfqpoint{1.733501in}{2.119466in}}{\pgfqpoint{1.725601in}{2.116194in}}{\pgfqpoint{1.719777in}{2.110370in}}%
\pgfpathcurveto{\pgfqpoint{1.713953in}{2.104546in}}{\pgfqpoint{1.710681in}{2.096646in}}{\pgfqpoint{1.710681in}{2.088410in}}%
\pgfpathcurveto{\pgfqpoint{1.710681in}{2.080174in}}{\pgfqpoint{1.713953in}{2.072274in}}{\pgfqpoint{1.719777in}{2.066450in}}%
\pgfpathcurveto{\pgfqpoint{1.725601in}{2.060626in}}{\pgfqpoint{1.733501in}{2.057353in}}{\pgfqpoint{1.741737in}{2.057353in}}%
\pgfpathclose%
\pgfusepath{stroke,fill}%
\end{pgfscope}%
\begin{pgfscope}%
\pgfpathrectangle{\pgfqpoint{0.100000in}{0.212622in}}{\pgfqpoint{3.696000in}{3.696000in}}%
\pgfusepath{clip}%
\pgfsetbuttcap%
\pgfsetroundjoin%
\definecolor{currentfill}{rgb}{0.121569,0.466667,0.705882}%
\pgfsetfillcolor{currentfill}%
\pgfsetfillopacity{0.881692}%
\pgfsetlinewidth{1.003750pt}%
\definecolor{currentstroke}{rgb}{0.121569,0.466667,0.705882}%
\pgfsetstrokecolor{currentstroke}%
\pgfsetstrokeopacity{0.881692}%
\pgfsetdash{}{0pt}%
\pgfpathmoveto{\pgfqpoint{1.741179in}{2.056342in}}%
\pgfpathcurveto{\pgfqpoint{1.749416in}{2.056342in}}{\pgfqpoint{1.757316in}{2.059615in}}{\pgfqpoint{1.763140in}{2.065439in}}%
\pgfpathcurveto{\pgfqpoint{1.768964in}{2.071262in}}{\pgfqpoint{1.772236in}{2.079163in}}{\pgfqpoint{1.772236in}{2.087399in}}%
\pgfpathcurveto{\pgfqpoint{1.772236in}{2.095635in}}{\pgfqpoint{1.768964in}{2.103535in}}{\pgfqpoint{1.763140in}{2.109359in}}%
\pgfpathcurveto{\pgfqpoint{1.757316in}{2.115183in}}{\pgfqpoint{1.749416in}{2.118455in}}{\pgfqpoint{1.741179in}{2.118455in}}%
\pgfpathcurveto{\pgfqpoint{1.732943in}{2.118455in}}{\pgfqpoint{1.725043in}{2.115183in}}{\pgfqpoint{1.719219in}{2.109359in}}%
\pgfpathcurveto{\pgfqpoint{1.713395in}{2.103535in}}{\pgfqpoint{1.710123in}{2.095635in}}{\pgfqpoint{1.710123in}{2.087399in}}%
\pgfpathcurveto{\pgfqpoint{1.710123in}{2.079163in}}{\pgfqpoint{1.713395in}{2.071262in}}{\pgfqpoint{1.719219in}{2.065439in}}%
\pgfpathcurveto{\pgfqpoint{1.725043in}{2.059615in}}{\pgfqpoint{1.732943in}{2.056342in}}{\pgfqpoint{1.741179in}{2.056342in}}%
\pgfpathclose%
\pgfusepath{stroke,fill}%
\end{pgfscope}%
\begin{pgfscope}%
\pgfpathrectangle{\pgfqpoint{0.100000in}{0.212622in}}{\pgfqpoint{3.696000in}{3.696000in}}%
\pgfusepath{clip}%
\pgfsetbuttcap%
\pgfsetroundjoin%
\definecolor{currentfill}{rgb}{0.121569,0.466667,0.705882}%
\pgfsetfillcolor{currentfill}%
\pgfsetfillopacity{0.881883}%
\pgfsetlinewidth{1.003750pt}%
\definecolor{currentstroke}{rgb}{0.121569,0.466667,0.705882}%
\pgfsetstrokecolor{currentstroke}%
\pgfsetstrokeopacity{0.881883}%
\pgfsetdash{}{0pt}%
\pgfpathmoveto{\pgfqpoint{2.251985in}{2.232502in}}%
\pgfpathcurveto{\pgfqpoint{2.260221in}{2.232502in}}{\pgfqpoint{2.268121in}{2.235775in}}{\pgfqpoint{2.273945in}{2.241599in}}%
\pgfpathcurveto{\pgfqpoint{2.279769in}{2.247423in}}{\pgfqpoint{2.283041in}{2.255323in}}{\pgfqpoint{2.283041in}{2.263559in}}%
\pgfpathcurveto{\pgfqpoint{2.283041in}{2.271795in}}{\pgfqpoint{2.279769in}{2.279695in}}{\pgfqpoint{2.273945in}{2.285519in}}%
\pgfpathcurveto{\pgfqpoint{2.268121in}{2.291343in}}{\pgfqpoint{2.260221in}{2.294615in}}{\pgfqpoint{2.251985in}{2.294615in}}%
\pgfpathcurveto{\pgfqpoint{2.243748in}{2.294615in}}{\pgfqpoint{2.235848in}{2.291343in}}{\pgfqpoint{2.230024in}{2.285519in}}%
\pgfpathcurveto{\pgfqpoint{2.224201in}{2.279695in}}{\pgfqpoint{2.220928in}{2.271795in}}{\pgfqpoint{2.220928in}{2.263559in}}%
\pgfpathcurveto{\pgfqpoint{2.220928in}{2.255323in}}{\pgfqpoint{2.224201in}{2.247423in}}{\pgfqpoint{2.230024in}{2.241599in}}%
\pgfpathcurveto{\pgfqpoint{2.235848in}{2.235775in}}{\pgfqpoint{2.243748in}{2.232502in}}{\pgfqpoint{2.251985in}{2.232502in}}%
\pgfpathclose%
\pgfusepath{stroke,fill}%
\end{pgfscope}%
\begin{pgfscope}%
\pgfpathrectangle{\pgfqpoint{0.100000in}{0.212622in}}{\pgfqpoint{3.696000in}{3.696000in}}%
\pgfusepath{clip}%
\pgfsetbuttcap%
\pgfsetroundjoin%
\definecolor{currentfill}{rgb}{0.121569,0.466667,0.705882}%
\pgfsetfillcolor{currentfill}%
\pgfsetfillopacity{0.881925}%
\pgfsetlinewidth{1.003750pt}%
\definecolor{currentstroke}{rgb}{0.121569,0.466667,0.705882}%
\pgfsetstrokecolor{currentstroke}%
\pgfsetstrokeopacity{0.881925}%
\pgfsetdash{}{0pt}%
\pgfpathmoveto{\pgfqpoint{1.740556in}{2.054797in}}%
\pgfpathcurveto{\pgfqpoint{1.748793in}{2.054797in}}{\pgfqpoint{1.756693in}{2.058070in}}{\pgfqpoint{1.762517in}{2.063893in}}%
\pgfpathcurveto{\pgfqpoint{1.768341in}{2.069717in}}{\pgfqpoint{1.771613in}{2.077617in}}{\pgfqpoint{1.771613in}{2.085854in}}%
\pgfpathcurveto{\pgfqpoint{1.771613in}{2.094090in}}{\pgfqpoint{1.768341in}{2.101990in}}{\pgfqpoint{1.762517in}{2.107814in}}%
\pgfpathcurveto{\pgfqpoint{1.756693in}{2.113638in}}{\pgfqpoint{1.748793in}{2.116910in}}{\pgfqpoint{1.740556in}{2.116910in}}%
\pgfpathcurveto{\pgfqpoint{1.732320in}{2.116910in}}{\pgfqpoint{1.724420in}{2.113638in}}{\pgfqpoint{1.718596in}{2.107814in}}%
\pgfpathcurveto{\pgfqpoint{1.712772in}{2.101990in}}{\pgfqpoint{1.709500in}{2.094090in}}{\pgfqpoint{1.709500in}{2.085854in}}%
\pgfpathcurveto{\pgfqpoint{1.709500in}{2.077617in}}{\pgfqpoint{1.712772in}{2.069717in}}{\pgfqpoint{1.718596in}{2.063893in}}%
\pgfpathcurveto{\pgfqpoint{1.724420in}{2.058070in}}{\pgfqpoint{1.732320in}{2.054797in}}{\pgfqpoint{1.740556in}{2.054797in}}%
\pgfpathclose%
\pgfusepath{stroke,fill}%
\end{pgfscope}%
\begin{pgfscope}%
\pgfpathrectangle{\pgfqpoint{0.100000in}{0.212622in}}{\pgfqpoint{3.696000in}{3.696000in}}%
\pgfusepath{clip}%
\pgfsetbuttcap%
\pgfsetroundjoin%
\definecolor{currentfill}{rgb}{0.121569,0.466667,0.705882}%
\pgfsetfillcolor{currentfill}%
\pgfsetfillopacity{0.882053}%
\pgfsetlinewidth{1.003750pt}%
\definecolor{currentstroke}{rgb}{0.121569,0.466667,0.705882}%
\pgfsetstrokecolor{currentstroke}%
\pgfsetstrokeopacity{0.882053}%
\pgfsetdash{}{0pt}%
\pgfpathmoveto{\pgfqpoint{1.740127in}{2.054039in}}%
\pgfpathcurveto{\pgfqpoint{1.748363in}{2.054039in}}{\pgfqpoint{1.756263in}{2.057311in}}{\pgfqpoint{1.762087in}{2.063135in}}%
\pgfpathcurveto{\pgfqpoint{1.767911in}{2.068959in}}{\pgfqpoint{1.771184in}{2.076859in}}{\pgfqpoint{1.771184in}{2.085096in}}%
\pgfpathcurveto{\pgfqpoint{1.771184in}{2.093332in}}{\pgfqpoint{1.767911in}{2.101232in}}{\pgfqpoint{1.762087in}{2.107056in}}%
\pgfpathcurveto{\pgfqpoint{1.756263in}{2.112880in}}{\pgfqpoint{1.748363in}{2.116152in}}{\pgfqpoint{1.740127in}{2.116152in}}%
\pgfpathcurveto{\pgfqpoint{1.731891in}{2.116152in}}{\pgfqpoint{1.723991in}{2.112880in}}{\pgfqpoint{1.718167in}{2.107056in}}%
\pgfpathcurveto{\pgfqpoint{1.712343in}{2.101232in}}{\pgfqpoint{1.709071in}{2.093332in}}{\pgfqpoint{1.709071in}{2.085096in}}%
\pgfpathcurveto{\pgfqpoint{1.709071in}{2.076859in}}{\pgfqpoint{1.712343in}{2.068959in}}{\pgfqpoint{1.718167in}{2.063135in}}%
\pgfpathcurveto{\pgfqpoint{1.723991in}{2.057311in}}{\pgfqpoint{1.731891in}{2.054039in}}{\pgfqpoint{1.740127in}{2.054039in}}%
\pgfpathclose%
\pgfusepath{stroke,fill}%
\end{pgfscope}%
\begin{pgfscope}%
\pgfpathrectangle{\pgfqpoint{0.100000in}{0.212622in}}{\pgfqpoint{3.696000in}{3.696000in}}%
\pgfusepath{clip}%
\pgfsetbuttcap%
\pgfsetroundjoin%
\definecolor{currentfill}{rgb}{0.121569,0.466667,0.705882}%
\pgfsetfillcolor{currentfill}%
\pgfsetfillopacity{0.882115}%
\pgfsetlinewidth{1.003750pt}%
\definecolor{currentstroke}{rgb}{0.121569,0.466667,0.705882}%
\pgfsetstrokecolor{currentstroke}%
\pgfsetstrokeopacity{0.882115}%
\pgfsetdash{}{0pt}%
\pgfpathmoveto{\pgfqpoint{1.739855in}{2.053648in}}%
\pgfpathcurveto{\pgfqpoint{1.748091in}{2.053648in}}{\pgfqpoint{1.755991in}{2.056920in}}{\pgfqpoint{1.761815in}{2.062744in}}%
\pgfpathcurveto{\pgfqpoint{1.767639in}{2.068568in}}{\pgfqpoint{1.770911in}{2.076468in}}{\pgfqpoint{1.770911in}{2.084704in}}%
\pgfpathcurveto{\pgfqpoint{1.770911in}{2.092941in}}{\pgfqpoint{1.767639in}{2.100841in}}{\pgfqpoint{1.761815in}{2.106665in}}%
\pgfpathcurveto{\pgfqpoint{1.755991in}{2.112489in}}{\pgfqpoint{1.748091in}{2.115761in}}{\pgfqpoint{1.739855in}{2.115761in}}%
\pgfpathcurveto{\pgfqpoint{1.731619in}{2.115761in}}{\pgfqpoint{1.723719in}{2.112489in}}{\pgfqpoint{1.717895in}{2.106665in}}%
\pgfpathcurveto{\pgfqpoint{1.712071in}{2.100841in}}{\pgfqpoint{1.708798in}{2.092941in}}{\pgfqpoint{1.708798in}{2.084704in}}%
\pgfpathcurveto{\pgfqpoint{1.708798in}{2.076468in}}{\pgfqpoint{1.712071in}{2.068568in}}{\pgfqpoint{1.717895in}{2.062744in}}%
\pgfpathcurveto{\pgfqpoint{1.723719in}{2.056920in}}{\pgfqpoint{1.731619in}{2.053648in}}{\pgfqpoint{1.739855in}{2.053648in}}%
\pgfpathclose%
\pgfusepath{stroke,fill}%
\end{pgfscope}%
\begin{pgfscope}%
\pgfpathrectangle{\pgfqpoint{0.100000in}{0.212622in}}{\pgfqpoint{3.696000in}{3.696000in}}%
\pgfusepath{clip}%
\pgfsetbuttcap%
\pgfsetroundjoin%
\definecolor{currentfill}{rgb}{0.121569,0.466667,0.705882}%
\pgfsetfillcolor{currentfill}%
\pgfsetfillopacity{0.882303}%
\pgfsetlinewidth{1.003750pt}%
\definecolor{currentstroke}{rgb}{0.121569,0.466667,0.705882}%
\pgfsetstrokecolor{currentstroke}%
\pgfsetstrokeopacity{0.882303}%
\pgfsetdash{}{0pt}%
\pgfpathmoveto{\pgfqpoint{1.739267in}{2.052517in}}%
\pgfpathcurveto{\pgfqpoint{1.747503in}{2.052517in}}{\pgfqpoint{1.755403in}{2.055789in}}{\pgfqpoint{1.761227in}{2.061613in}}%
\pgfpathcurveto{\pgfqpoint{1.767051in}{2.067437in}}{\pgfqpoint{1.770324in}{2.075337in}}{\pgfqpoint{1.770324in}{2.083573in}}%
\pgfpathcurveto{\pgfqpoint{1.770324in}{2.091809in}}{\pgfqpoint{1.767051in}{2.099709in}}{\pgfqpoint{1.761227in}{2.105533in}}%
\pgfpathcurveto{\pgfqpoint{1.755403in}{2.111357in}}{\pgfqpoint{1.747503in}{2.114630in}}{\pgfqpoint{1.739267in}{2.114630in}}%
\pgfpathcurveto{\pgfqpoint{1.731031in}{2.114630in}}{\pgfqpoint{1.723131in}{2.111357in}}{\pgfqpoint{1.717307in}{2.105533in}}%
\pgfpathcurveto{\pgfqpoint{1.711483in}{2.099709in}}{\pgfqpoint{1.708211in}{2.091809in}}{\pgfqpoint{1.708211in}{2.083573in}}%
\pgfpathcurveto{\pgfqpoint{1.708211in}{2.075337in}}{\pgfqpoint{1.711483in}{2.067437in}}{\pgfqpoint{1.717307in}{2.061613in}}%
\pgfpathcurveto{\pgfqpoint{1.723131in}{2.055789in}}{\pgfqpoint{1.731031in}{2.052517in}}{\pgfqpoint{1.739267in}{2.052517in}}%
\pgfpathclose%
\pgfusepath{stroke,fill}%
\end{pgfscope}%
\begin{pgfscope}%
\pgfpathrectangle{\pgfqpoint{0.100000in}{0.212622in}}{\pgfqpoint{3.696000in}{3.696000in}}%
\pgfusepath{clip}%
\pgfsetbuttcap%
\pgfsetroundjoin%
\definecolor{currentfill}{rgb}{0.121569,0.466667,0.705882}%
\pgfsetfillcolor{currentfill}%
\pgfsetfillopacity{0.882409}%
\pgfsetlinewidth{1.003750pt}%
\definecolor{currentstroke}{rgb}{0.121569,0.466667,0.705882}%
\pgfsetstrokecolor{currentstroke}%
\pgfsetstrokeopacity{0.882409}%
\pgfsetdash{}{0pt}%
\pgfpathmoveto{\pgfqpoint{2.251004in}{2.229557in}}%
\pgfpathcurveto{\pgfqpoint{2.259240in}{2.229557in}}{\pgfqpoint{2.267140in}{2.232830in}}{\pgfqpoint{2.272964in}{2.238654in}}%
\pgfpathcurveto{\pgfqpoint{2.278788in}{2.244478in}}{\pgfqpoint{2.282061in}{2.252378in}}{\pgfqpoint{2.282061in}{2.260614in}}%
\pgfpathcurveto{\pgfqpoint{2.282061in}{2.268850in}}{\pgfqpoint{2.278788in}{2.276750in}}{\pgfqpoint{2.272964in}{2.282574in}}%
\pgfpathcurveto{\pgfqpoint{2.267140in}{2.288398in}}{\pgfqpoint{2.259240in}{2.291670in}}{\pgfqpoint{2.251004in}{2.291670in}}%
\pgfpathcurveto{\pgfqpoint{2.242768in}{2.291670in}}{\pgfqpoint{2.234868in}{2.288398in}}{\pgfqpoint{2.229044in}{2.282574in}}%
\pgfpathcurveto{\pgfqpoint{2.223220in}{2.276750in}}{\pgfqpoint{2.219948in}{2.268850in}}{\pgfqpoint{2.219948in}{2.260614in}}%
\pgfpathcurveto{\pgfqpoint{2.219948in}{2.252378in}}{\pgfqpoint{2.223220in}{2.244478in}}{\pgfqpoint{2.229044in}{2.238654in}}%
\pgfpathcurveto{\pgfqpoint{2.234868in}{2.232830in}}{\pgfqpoint{2.242768in}{2.229557in}}{\pgfqpoint{2.251004in}{2.229557in}}%
\pgfpathclose%
\pgfusepath{stroke,fill}%
\end{pgfscope}%
\begin{pgfscope}%
\pgfpathrectangle{\pgfqpoint{0.100000in}{0.212622in}}{\pgfqpoint{3.696000in}{3.696000in}}%
\pgfusepath{clip}%
\pgfsetbuttcap%
\pgfsetroundjoin%
\definecolor{currentfill}{rgb}{0.121569,0.466667,0.705882}%
\pgfsetfillcolor{currentfill}%
\pgfsetfillopacity{0.882589}%
\pgfsetlinewidth{1.003750pt}%
\definecolor{currentstroke}{rgb}{0.121569,0.466667,0.705882}%
\pgfsetstrokecolor{currentstroke}%
\pgfsetstrokeopacity{0.882589}%
\pgfsetdash{}{0pt}%
\pgfpathmoveto{\pgfqpoint{1.738654in}{2.050812in}}%
\pgfpathcurveto{\pgfqpoint{1.746890in}{2.050812in}}{\pgfqpoint{1.754790in}{2.054084in}}{\pgfqpoint{1.760614in}{2.059908in}}%
\pgfpathcurveto{\pgfqpoint{1.766438in}{2.065732in}}{\pgfqpoint{1.769710in}{2.073632in}}{\pgfqpoint{1.769710in}{2.081868in}}%
\pgfpathcurveto{\pgfqpoint{1.769710in}{2.090105in}}{\pgfqpoint{1.766438in}{2.098005in}}{\pgfqpoint{1.760614in}{2.103829in}}%
\pgfpathcurveto{\pgfqpoint{1.754790in}{2.109652in}}{\pgfqpoint{1.746890in}{2.112925in}}{\pgfqpoint{1.738654in}{2.112925in}}%
\pgfpathcurveto{\pgfqpoint{1.730417in}{2.112925in}}{\pgfqpoint{1.722517in}{2.109652in}}{\pgfqpoint{1.716693in}{2.103829in}}%
\pgfpathcurveto{\pgfqpoint{1.710869in}{2.098005in}}{\pgfqpoint{1.707597in}{2.090105in}}{\pgfqpoint{1.707597in}{2.081868in}}%
\pgfpathcurveto{\pgfqpoint{1.707597in}{2.073632in}}{\pgfqpoint{1.710869in}{2.065732in}}{\pgfqpoint{1.716693in}{2.059908in}}%
\pgfpathcurveto{\pgfqpoint{1.722517in}{2.054084in}}{\pgfqpoint{1.730417in}{2.050812in}}{\pgfqpoint{1.738654in}{2.050812in}}%
\pgfpathclose%
\pgfusepath{stroke,fill}%
\end{pgfscope}%
\begin{pgfscope}%
\pgfpathrectangle{\pgfqpoint{0.100000in}{0.212622in}}{\pgfqpoint{3.696000in}{3.696000in}}%
\pgfusepath{clip}%
\pgfsetbuttcap%
\pgfsetroundjoin%
\definecolor{currentfill}{rgb}{0.121569,0.466667,0.705882}%
\pgfsetfillcolor{currentfill}%
\pgfsetfillopacity{0.882937}%
\pgfsetlinewidth{1.003750pt}%
\definecolor{currentstroke}{rgb}{0.121569,0.466667,0.705882}%
\pgfsetstrokecolor{currentstroke}%
\pgfsetstrokeopacity{0.882937}%
\pgfsetdash{}{0pt}%
\pgfpathmoveto{\pgfqpoint{1.737515in}{2.048984in}}%
\pgfpathcurveto{\pgfqpoint{1.745751in}{2.048984in}}{\pgfqpoint{1.753651in}{2.052256in}}{\pgfqpoint{1.759475in}{2.058080in}}%
\pgfpathcurveto{\pgfqpoint{1.765299in}{2.063904in}}{\pgfqpoint{1.768571in}{2.071804in}}{\pgfqpoint{1.768571in}{2.080040in}}%
\pgfpathcurveto{\pgfqpoint{1.768571in}{2.088276in}}{\pgfqpoint{1.765299in}{2.096176in}}{\pgfqpoint{1.759475in}{2.102000in}}%
\pgfpathcurveto{\pgfqpoint{1.753651in}{2.107824in}}{\pgfqpoint{1.745751in}{2.111097in}}{\pgfqpoint{1.737515in}{2.111097in}}%
\pgfpathcurveto{\pgfqpoint{1.729278in}{2.111097in}}{\pgfqpoint{1.721378in}{2.107824in}}{\pgfqpoint{1.715554in}{2.102000in}}%
\pgfpathcurveto{\pgfqpoint{1.709730in}{2.096176in}}{\pgfqpoint{1.706458in}{2.088276in}}{\pgfqpoint{1.706458in}{2.080040in}}%
\pgfpathcurveto{\pgfqpoint{1.706458in}{2.071804in}}{\pgfqpoint{1.709730in}{2.063904in}}{\pgfqpoint{1.715554in}{2.058080in}}%
\pgfpathcurveto{\pgfqpoint{1.721378in}{2.052256in}}{\pgfqpoint{1.729278in}{2.048984in}}{\pgfqpoint{1.737515in}{2.048984in}}%
\pgfpathclose%
\pgfusepath{stroke,fill}%
\end{pgfscope}%
\begin{pgfscope}%
\pgfpathrectangle{\pgfqpoint{0.100000in}{0.212622in}}{\pgfqpoint{3.696000in}{3.696000in}}%
\pgfusepath{clip}%
\pgfsetbuttcap%
\pgfsetroundjoin%
\definecolor{currentfill}{rgb}{0.121569,0.466667,0.705882}%
\pgfsetfillcolor{currentfill}%
\pgfsetfillopacity{0.883111}%
\pgfsetlinewidth{1.003750pt}%
\definecolor{currentstroke}{rgb}{0.121569,0.466667,0.705882}%
\pgfsetstrokecolor{currentstroke}%
\pgfsetstrokeopacity{0.883111}%
\pgfsetdash{}{0pt}%
\pgfpathmoveto{\pgfqpoint{1.736833in}{2.047997in}}%
\pgfpathcurveto{\pgfqpoint{1.745070in}{2.047997in}}{\pgfqpoint{1.752970in}{2.051269in}}{\pgfqpoint{1.758794in}{2.057093in}}%
\pgfpathcurveto{\pgfqpoint{1.764617in}{2.062917in}}{\pgfqpoint{1.767890in}{2.070817in}}{\pgfqpoint{1.767890in}{2.079054in}}%
\pgfpathcurveto{\pgfqpoint{1.767890in}{2.087290in}}{\pgfqpoint{1.764617in}{2.095190in}}{\pgfqpoint{1.758794in}{2.101014in}}%
\pgfpathcurveto{\pgfqpoint{1.752970in}{2.106838in}}{\pgfqpoint{1.745070in}{2.110110in}}{\pgfqpoint{1.736833in}{2.110110in}}%
\pgfpathcurveto{\pgfqpoint{1.728597in}{2.110110in}}{\pgfqpoint{1.720697in}{2.106838in}}{\pgfqpoint{1.714873in}{2.101014in}}%
\pgfpathcurveto{\pgfqpoint{1.709049in}{2.095190in}}{\pgfqpoint{1.705777in}{2.087290in}}{\pgfqpoint{1.705777in}{2.079054in}}%
\pgfpathcurveto{\pgfqpoint{1.705777in}{2.070817in}}{\pgfqpoint{1.709049in}{2.062917in}}{\pgfqpoint{1.714873in}{2.057093in}}%
\pgfpathcurveto{\pgfqpoint{1.720697in}{2.051269in}}{\pgfqpoint{1.728597in}{2.047997in}}{\pgfqpoint{1.736833in}{2.047997in}}%
\pgfpathclose%
\pgfusepath{stroke,fill}%
\end{pgfscope}%
\begin{pgfscope}%
\pgfpathrectangle{\pgfqpoint{0.100000in}{0.212622in}}{\pgfqpoint{3.696000in}{3.696000in}}%
\pgfusepath{clip}%
\pgfsetbuttcap%
\pgfsetroundjoin%
\definecolor{currentfill}{rgb}{0.121569,0.466667,0.705882}%
\pgfsetfillcolor{currentfill}%
\pgfsetfillopacity{0.883319}%
\pgfsetlinewidth{1.003750pt}%
\definecolor{currentstroke}{rgb}{0.121569,0.466667,0.705882}%
\pgfsetstrokecolor{currentstroke}%
\pgfsetstrokeopacity{0.883319}%
\pgfsetdash{}{0pt}%
\pgfpathmoveto{\pgfqpoint{2.250117in}{2.223877in}}%
\pgfpathcurveto{\pgfqpoint{2.258354in}{2.223877in}}{\pgfqpoint{2.266254in}{2.227150in}}{\pgfqpoint{2.272078in}{2.232974in}}%
\pgfpathcurveto{\pgfqpoint{2.277902in}{2.238798in}}{\pgfqpoint{2.281174in}{2.246698in}}{\pgfqpoint{2.281174in}{2.254934in}}%
\pgfpathcurveto{\pgfqpoint{2.281174in}{2.263170in}}{\pgfqpoint{2.277902in}{2.271070in}}{\pgfqpoint{2.272078in}{2.276894in}}%
\pgfpathcurveto{\pgfqpoint{2.266254in}{2.282718in}}{\pgfqpoint{2.258354in}{2.285990in}}{\pgfqpoint{2.250117in}{2.285990in}}%
\pgfpathcurveto{\pgfqpoint{2.241881in}{2.285990in}}{\pgfqpoint{2.233981in}{2.282718in}}{\pgfqpoint{2.228157in}{2.276894in}}%
\pgfpathcurveto{\pgfqpoint{2.222333in}{2.271070in}}{\pgfqpoint{2.219061in}{2.263170in}}{\pgfqpoint{2.219061in}{2.254934in}}%
\pgfpathcurveto{\pgfqpoint{2.219061in}{2.246698in}}{\pgfqpoint{2.222333in}{2.238798in}}{\pgfqpoint{2.228157in}{2.232974in}}%
\pgfpathcurveto{\pgfqpoint{2.233981in}{2.227150in}}{\pgfqpoint{2.241881in}{2.223877in}}{\pgfqpoint{2.250117in}{2.223877in}}%
\pgfpathclose%
\pgfusepath{stroke,fill}%
\end{pgfscope}%
\begin{pgfscope}%
\pgfpathrectangle{\pgfqpoint{0.100000in}{0.212622in}}{\pgfqpoint{3.696000in}{3.696000in}}%
\pgfusepath{clip}%
\pgfsetbuttcap%
\pgfsetroundjoin%
\definecolor{currentfill}{rgb}{0.121569,0.466667,0.705882}%
\pgfsetfillcolor{currentfill}%
\pgfsetfillopacity{0.883450}%
\pgfsetlinewidth{1.003750pt}%
\definecolor{currentstroke}{rgb}{0.121569,0.466667,0.705882}%
\pgfsetstrokecolor{currentstroke}%
\pgfsetstrokeopacity{0.883450}%
\pgfsetdash{}{0pt}%
\pgfpathmoveto{\pgfqpoint{1.735922in}{2.045675in}}%
\pgfpathcurveto{\pgfqpoint{1.744158in}{2.045675in}}{\pgfqpoint{1.752058in}{2.048947in}}{\pgfqpoint{1.757882in}{2.054771in}}%
\pgfpathcurveto{\pgfqpoint{1.763706in}{2.060595in}}{\pgfqpoint{1.766978in}{2.068495in}}{\pgfqpoint{1.766978in}{2.076731in}}%
\pgfpathcurveto{\pgfqpoint{1.766978in}{2.084967in}}{\pgfqpoint{1.763706in}{2.092867in}}{\pgfqpoint{1.757882in}{2.098691in}}%
\pgfpathcurveto{\pgfqpoint{1.752058in}{2.104515in}}{\pgfqpoint{1.744158in}{2.107788in}}{\pgfqpoint{1.735922in}{2.107788in}}%
\pgfpathcurveto{\pgfqpoint{1.727686in}{2.107788in}}{\pgfqpoint{1.719786in}{2.104515in}}{\pgfqpoint{1.713962in}{2.098691in}}%
\pgfpathcurveto{\pgfqpoint{1.708138in}{2.092867in}}{\pgfqpoint{1.704865in}{2.084967in}}{\pgfqpoint{1.704865in}{2.076731in}}%
\pgfpathcurveto{\pgfqpoint{1.704865in}{2.068495in}}{\pgfqpoint{1.708138in}{2.060595in}}{\pgfqpoint{1.713962in}{2.054771in}}%
\pgfpathcurveto{\pgfqpoint{1.719786in}{2.048947in}}{\pgfqpoint{1.727686in}{2.045675in}}{\pgfqpoint{1.735922in}{2.045675in}}%
\pgfpathclose%
\pgfusepath{stroke,fill}%
\end{pgfscope}%
\begin{pgfscope}%
\pgfpathrectangle{\pgfqpoint{0.100000in}{0.212622in}}{\pgfqpoint{3.696000in}{3.696000in}}%
\pgfusepath{clip}%
\pgfsetbuttcap%
\pgfsetroundjoin%
\definecolor{currentfill}{rgb}{0.121569,0.466667,0.705882}%
\pgfsetfillcolor{currentfill}%
\pgfsetfillopacity{0.883671}%
\pgfsetlinewidth{1.003750pt}%
\definecolor{currentstroke}{rgb}{0.121569,0.466667,0.705882}%
\pgfsetstrokecolor{currentstroke}%
\pgfsetstrokeopacity{0.883671}%
\pgfsetdash{}{0pt}%
\pgfpathmoveto{\pgfqpoint{1.573677in}{1.199854in}}%
\pgfpathcurveto{\pgfqpoint{1.581913in}{1.199854in}}{\pgfqpoint{1.589813in}{1.203127in}}{\pgfqpoint{1.595637in}{1.208951in}}%
\pgfpathcurveto{\pgfqpoint{1.601461in}{1.214775in}}{\pgfqpoint{1.604734in}{1.222675in}}{\pgfqpoint{1.604734in}{1.230911in}}%
\pgfpathcurveto{\pgfqpoint{1.604734in}{1.239147in}}{\pgfqpoint{1.601461in}{1.247047in}}{\pgfqpoint{1.595637in}{1.252871in}}%
\pgfpathcurveto{\pgfqpoint{1.589813in}{1.258695in}}{\pgfqpoint{1.581913in}{1.261967in}}{\pgfqpoint{1.573677in}{1.261967in}}%
\pgfpathcurveto{\pgfqpoint{1.565441in}{1.261967in}}{\pgfqpoint{1.557541in}{1.258695in}}{\pgfqpoint{1.551717in}{1.252871in}}%
\pgfpathcurveto{\pgfqpoint{1.545893in}{1.247047in}}{\pgfqpoint{1.542621in}{1.239147in}}{\pgfqpoint{1.542621in}{1.230911in}}%
\pgfpathcurveto{\pgfqpoint{1.542621in}{1.222675in}}{\pgfqpoint{1.545893in}{1.214775in}}{\pgfqpoint{1.551717in}{1.208951in}}%
\pgfpathcurveto{\pgfqpoint{1.557541in}{1.203127in}}{\pgfqpoint{1.565441in}{1.199854in}}{\pgfqpoint{1.573677in}{1.199854in}}%
\pgfpathclose%
\pgfusepath{stroke,fill}%
\end{pgfscope}%
\begin{pgfscope}%
\pgfpathrectangle{\pgfqpoint{0.100000in}{0.212622in}}{\pgfqpoint{3.696000in}{3.696000in}}%
\pgfusepath{clip}%
\pgfsetbuttcap%
\pgfsetroundjoin%
\definecolor{currentfill}{rgb}{0.121569,0.466667,0.705882}%
\pgfsetfillcolor{currentfill}%
\pgfsetfillopacity{0.883901}%
\pgfsetlinewidth{1.003750pt}%
\definecolor{currentstroke}{rgb}{0.121569,0.466667,0.705882}%
\pgfsetstrokecolor{currentstroke}%
\pgfsetstrokeopacity{0.883901}%
\pgfsetdash{}{0pt}%
\pgfpathmoveto{\pgfqpoint{1.734710in}{2.043266in}}%
\pgfpathcurveto{\pgfqpoint{1.742947in}{2.043266in}}{\pgfqpoint{1.750847in}{2.046538in}}{\pgfqpoint{1.756671in}{2.052362in}}%
\pgfpathcurveto{\pgfqpoint{1.762494in}{2.058186in}}{\pgfqpoint{1.765767in}{2.066086in}}{\pgfqpoint{1.765767in}{2.074322in}}%
\pgfpathcurveto{\pgfqpoint{1.765767in}{2.082558in}}{\pgfqpoint{1.762494in}{2.090458in}}{\pgfqpoint{1.756671in}{2.096282in}}%
\pgfpathcurveto{\pgfqpoint{1.750847in}{2.102106in}}{\pgfqpoint{1.742947in}{2.105379in}}{\pgfqpoint{1.734710in}{2.105379in}}%
\pgfpathcurveto{\pgfqpoint{1.726474in}{2.105379in}}{\pgfqpoint{1.718574in}{2.102106in}}{\pgfqpoint{1.712750in}{2.096282in}}%
\pgfpathcurveto{\pgfqpoint{1.706926in}{2.090458in}}{\pgfqpoint{1.703654in}{2.082558in}}{\pgfqpoint{1.703654in}{2.074322in}}%
\pgfpathcurveto{\pgfqpoint{1.703654in}{2.066086in}}{\pgfqpoint{1.706926in}{2.058186in}}{\pgfqpoint{1.712750in}{2.052362in}}%
\pgfpathcurveto{\pgfqpoint{1.718574in}{2.046538in}}{\pgfqpoint{1.726474in}{2.043266in}}{\pgfqpoint{1.734710in}{2.043266in}}%
\pgfpathclose%
\pgfusepath{stroke,fill}%
\end{pgfscope}%
\begin{pgfscope}%
\pgfpathrectangle{\pgfqpoint{0.100000in}{0.212622in}}{\pgfqpoint{3.696000in}{3.696000in}}%
\pgfusepath{clip}%
\pgfsetbuttcap%
\pgfsetroundjoin%
\definecolor{currentfill}{rgb}{0.121569,0.466667,0.705882}%
\pgfsetfillcolor{currentfill}%
\pgfsetfillopacity{0.883926}%
\pgfsetlinewidth{1.003750pt}%
\definecolor{currentstroke}{rgb}{0.121569,0.466667,0.705882}%
\pgfsetstrokecolor{currentstroke}%
\pgfsetstrokeopacity{0.883926}%
\pgfsetdash{}{0pt}%
\pgfpathmoveto{\pgfqpoint{2.248037in}{2.220545in}}%
\pgfpathcurveto{\pgfqpoint{2.256273in}{2.220545in}}{\pgfqpoint{2.264173in}{2.223817in}}{\pgfqpoint{2.269997in}{2.229641in}}%
\pgfpathcurveto{\pgfqpoint{2.275821in}{2.235465in}}{\pgfqpoint{2.279094in}{2.243365in}}{\pgfqpoint{2.279094in}{2.251602in}}%
\pgfpathcurveto{\pgfqpoint{2.279094in}{2.259838in}}{\pgfqpoint{2.275821in}{2.267738in}}{\pgfqpoint{2.269997in}{2.273562in}}%
\pgfpathcurveto{\pgfqpoint{2.264173in}{2.279386in}}{\pgfqpoint{2.256273in}{2.282658in}}{\pgfqpoint{2.248037in}{2.282658in}}%
\pgfpathcurveto{\pgfqpoint{2.239801in}{2.282658in}}{\pgfqpoint{2.231901in}{2.279386in}}{\pgfqpoint{2.226077in}{2.273562in}}%
\pgfpathcurveto{\pgfqpoint{2.220253in}{2.267738in}}{\pgfqpoint{2.216981in}{2.259838in}}{\pgfqpoint{2.216981in}{2.251602in}}%
\pgfpathcurveto{\pgfqpoint{2.216981in}{2.243365in}}{\pgfqpoint{2.220253in}{2.235465in}}{\pgfqpoint{2.226077in}{2.229641in}}%
\pgfpathcurveto{\pgfqpoint{2.231901in}{2.223817in}}{\pgfqpoint{2.239801in}{2.220545in}}{\pgfqpoint{2.248037in}{2.220545in}}%
\pgfpathclose%
\pgfusepath{stroke,fill}%
\end{pgfscope}%
\begin{pgfscope}%
\pgfpathrectangle{\pgfqpoint{0.100000in}{0.212622in}}{\pgfqpoint{3.696000in}{3.696000in}}%
\pgfusepath{clip}%
\pgfsetbuttcap%
\pgfsetroundjoin%
\definecolor{currentfill}{rgb}{0.121569,0.466667,0.705882}%
\pgfsetfillcolor{currentfill}%
\pgfsetfillopacity{0.884127}%
\pgfsetlinewidth{1.003750pt}%
\definecolor{currentstroke}{rgb}{0.121569,0.466667,0.705882}%
\pgfsetstrokecolor{currentstroke}%
\pgfsetstrokeopacity{0.884127}%
\pgfsetdash{}{0pt}%
\pgfpathmoveto{\pgfqpoint{1.733901in}{2.042038in}}%
\pgfpathcurveto{\pgfqpoint{1.742138in}{2.042038in}}{\pgfqpoint{1.750038in}{2.045310in}}{\pgfqpoint{1.755862in}{2.051134in}}%
\pgfpathcurveto{\pgfqpoint{1.761685in}{2.056958in}}{\pgfqpoint{1.764958in}{2.064858in}}{\pgfqpoint{1.764958in}{2.073094in}}%
\pgfpathcurveto{\pgfqpoint{1.764958in}{2.081330in}}{\pgfqpoint{1.761685in}{2.089230in}}{\pgfqpoint{1.755862in}{2.095054in}}%
\pgfpathcurveto{\pgfqpoint{1.750038in}{2.100878in}}{\pgfqpoint{1.742138in}{2.104151in}}{\pgfqpoint{1.733901in}{2.104151in}}%
\pgfpathcurveto{\pgfqpoint{1.725665in}{2.104151in}}{\pgfqpoint{1.717765in}{2.100878in}}{\pgfqpoint{1.711941in}{2.095054in}}%
\pgfpathcurveto{\pgfqpoint{1.706117in}{2.089230in}}{\pgfqpoint{1.702845in}{2.081330in}}{\pgfqpoint{1.702845in}{2.073094in}}%
\pgfpathcurveto{\pgfqpoint{1.702845in}{2.064858in}}{\pgfqpoint{1.706117in}{2.056958in}}{\pgfqpoint{1.711941in}{2.051134in}}%
\pgfpathcurveto{\pgfqpoint{1.717765in}{2.045310in}}{\pgfqpoint{1.725665in}{2.042038in}}{\pgfqpoint{1.733901in}{2.042038in}}%
\pgfpathclose%
\pgfusepath{stroke,fill}%
\end{pgfscope}%
\begin{pgfscope}%
\pgfpathrectangle{\pgfqpoint{0.100000in}{0.212622in}}{\pgfqpoint{3.696000in}{3.696000in}}%
\pgfusepath{clip}%
\pgfsetbuttcap%
\pgfsetroundjoin%
\definecolor{currentfill}{rgb}{0.121569,0.466667,0.705882}%
\pgfsetfillcolor{currentfill}%
\pgfsetfillopacity{0.884317}%
\pgfsetlinewidth{1.003750pt}%
\definecolor{currentstroke}{rgb}{0.121569,0.466667,0.705882}%
\pgfsetstrokecolor{currentstroke}%
\pgfsetstrokeopacity{0.884317}%
\pgfsetdash{}{0pt}%
\pgfpathmoveto{\pgfqpoint{2.246328in}{2.218418in}}%
\pgfpathcurveto{\pgfqpoint{2.254564in}{2.218418in}}{\pgfqpoint{2.262464in}{2.221690in}}{\pgfqpoint{2.268288in}{2.227514in}}%
\pgfpathcurveto{\pgfqpoint{2.274112in}{2.233338in}}{\pgfqpoint{2.277384in}{2.241238in}}{\pgfqpoint{2.277384in}{2.249474in}}%
\pgfpathcurveto{\pgfqpoint{2.277384in}{2.257710in}}{\pgfqpoint{2.274112in}{2.265610in}}{\pgfqpoint{2.268288in}{2.271434in}}%
\pgfpathcurveto{\pgfqpoint{2.262464in}{2.277258in}}{\pgfqpoint{2.254564in}{2.280531in}}{\pgfqpoint{2.246328in}{2.280531in}}%
\pgfpathcurveto{\pgfqpoint{2.238091in}{2.280531in}}{\pgfqpoint{2.230191in}{2.277258in}}{\pgfqpoint{2.224367in}{2.271434in}}%
\pgfpathcurveto{\pgfqpoint{2.218543in}{2.265610in}}{\pgfqpoint{2.215271in}{2.257710in}}{\pgfqpoint{2.215271in}{2.249474in}}%
\pgfpathcurveto{\pgfqpoint{2.215271in}{2.241238in}}{\pgfqpoint{2.218543in}{2.233338in}}{\pgfqpoint{2.224367in}{2.227514in}}%
\pgfpathcurveto{\pgfqpoint{2.230191in}{2.221690in}}{\pgfqpoint{2.238091in}{2.218418in}}{\pgfqpoint{2.246328in}{2.218418in}}%
\pgfpathclose%
\pgfusepath{stroke,fill}%
\end{pgfscope}%
\begin{pgfscope}%
\pgfpathrectangle{\pgfqpoint{0.100000in}{0.212622in}}{\pgfqpoint{3.696000in}{3.696000in}}%
\pgfusepath{clip}%
\pgfsetbuttcap%
\pgfsetroundjoin%
\definecolor{currentfill}{rgb}{0.121569,0.466667,0.705882}%
\pgfsetfillcolor{currentfill}%
\pgfsetfillopacity{0.884459}%
\pgfsetlinewidth{1.003750pt}%
\definecolor{currentstroke}{rgb}{0.121569,0.466667,0.705882}%
\pgfsetstrokecolor{currentstroke}%
\pgfsetstrokeopacity{0.884459}%
\pgfsetdash{}{0pt}%
\pgfpathmoveto{\pgfqpoint{2.849065in}{1.466576in}}%
\pgfpathcurveto{\pgfqpoint{2.857301in}{1.466576in}}{\pgfqpoint{2.865201in}{1.469848in}}{\pgfqpoint{2.871025in}{1.475672in}}%
\pgfpathcurveto{\pgfqpoint{2.876849in}{1.481496in}}{\pgfqpoint{2.880121in}{1.489396in}}{\pgfqpoint{2.880121in}{1.497632in}}%
\pgfpathcurveto{\pgfqpoint{2.880121in}{1.505869in}}{\pgfqpoint{2.876849in}{1.513769in}}{\pgfqpoint{2.871025in}{1.519593in}}%
\pgfpathcurveto{\pgfqpoint{2.865201in}{1.525417in}}{\pgfqpoint{2.857301in}{1.528689in}}{\pgfqpoint{2.849065in}{1.528689in}}%
\pgfpathcurveto{\pgfqpoint{2.840829in}{1.528689in}}{\pgfqpoint{2.832929in}{1.525417in}}{\pgfqpoint{2.827105in}{1.519593in}}%
\pgfpathcurveto{\pgfqpoint{2.821281in}{1.513769in}}{\pgfqpoint{2.818008in}{1.505869in}}{\pgfqpoint{2.818008in}{1.497632in}}%
\pgfpathcurveto{\pgfqpoint{2.818008in}{1.489396in}}{\pgfqpoint{2.821281in}{1.481496in}}{\pgfqpoint{2.827105in}{1.475672in}}%
\pgfpathcurveto{\pgfqpoint{2.832929in}{1.469848in}}{\pgfqpoint{2.840829in}{1.466576in}}{\pgfqpoint{2.849065in}{1.466576in}}%
\pgfpathclose%
\pgfusepath{stroke,fill}%
\end{pgfscope}%
\begin{pgfscope}%
\pgfpathrectangle{\pgfqpoint{0.100000in}{0.212622in}}{\pgfqpoint{3.696000in}{3.696000in}}%
\pgfusepath{clip}%
\pgfsetbuttcap%
\pgfsetroundjoin%
\definecolor{currentfill}{rgb}{0.121569,0.466667,0.705882}%
\pgfsetfillcolor{currentfill}%
\pgfsetfillopacity{0.884524}%
\pgfsetlinewidth{1.003750pt}%
\definecolor{currentstroke}{rgb}{0.121569,0.466667,0.705882}%
\pgfsetstrokecolor{currentstroke}%
\pgfsetstrokeopacity{0.884524}%
\pgfsetdash{}{0pt}%
\pgfpathmoveto{\pgfqpoint{1.732686in}{2.040189in}}%
\pgfpathcurveto{\pgfqpoint{1.740923in}{2.040189in}}{\pgfqpoint{1.748823in}{2.043461in}}{\pgfqpoint{1.754647in}{2.049285in}}%
\pgfpathcurveto{\pgfqpoint{1.760471in}{2.055109in}}{\pgfqpoint{1.763743in}{2.063009in}}{\pgfqpoint{1.763743in}{2.071245in}}%
\pgfpathcurveto{\pgfqpoint{1.763743in}{2.079481in}}{\pgfqpoint{1.760471in}{2.087382in}}{\pgfqpoint{1.754647in}{2.093205in}}%
\pgfpathcurveto{\pgfqpoint{1.748823in}{2.099029in}}{\pgfqpoint{1.740923in}{2.102302in}}{\pgfqpoint{1.732686in}{2.102302in}}%
\pgfpathcurveto{\pgfqpoint{1.724450in}{2.102302in}}{\pgfqpoint{1.716550in}{2.099029in}}{\pgfqpoint{1.710726in}{2.093205in}}%
\pgfpathcurveto{\pgfqpoint{1.704902in}{2.087382in}}{\pgfqpoint{1.701630in}{2.079481in}}{\pgfqpoint{1.701630in}{2.071245in}}%
\pgfpathcurveto{\pgfqpoint{1.701630in}{2.063009in}}{\pgfqpoint{1.704902in}{2.055109in}}{\pgfqpoint{1.710726in}{2.049285in}}%
\pgfpathcurveto{\pgfqpoint{1.716550in}{2.043461in}}{\pgfqpoint{1.724450in}{2.040189in}}{\pgfqpoint{1.732686in}{2.040189in}}%
\pgfpathclose%
\pgfusepath{stroke,fill}%
\end{pgfscope}%
\begin{pgfscope}%
\pgfpathrectangle{\pgfqpoint{0.100000in}{0.212622in}}{\pgfqpoint{3.696000in}{3.696000in}}%
\pgfusepath{clip}%
\pgfsetbuttcap%
\pgfsetroundjoin%
\definecolor{currentfill}{rgb}{0.121569,0.466667,0.705882}%
\pgfsetfillcolor{currentfill}%
\pgfsetfillopacity{0.884716}%
\pgfsetlinewidth{1.003750pt}%
\definecolor{currentstroke}{rgb}{0.121569,0.466667,0.705882}%
\pgfsetstrokecolor{currentstroke}%
\pgfsetstrokeopacity{0.884716}%
\pgfsetdash{}{0pt}%
\pgfpathmoveto{\pgfqpoint{1.732193in}{2.038872in}}%
\pgfpathcurveto{\pgfqpoint{1.740429in}{2.038872in}}{\pgfqpoint{1.748329in}{2.042144in}}{\pgfqpoint{1.754153in}{2.047968in}}%
\pgfpathcurveto{\pgfqpoint{1.759977in}{2.053792in}}{\pgfqpoint{1.763249in}{2.061692in}}{\pgfqpoint{1.763249in}{2.069929in}}%
\pgfpathcurveto{\pgfqpoint{1.763249in}{2.078165in}}{\pgfqpoint{1.759977in}{2.086065in}}{\pgfqpoint{1.754153in}{2.091889in}}%
\pgfpathcurveto{\pgfqpoint{1.748329in}{2.097713in}}{\pgfqpoint{1.740429in}{2.100985in}}{\pgfqpoint{1.732193in}{2.100985in}}%
\pgfpathcurveto{\pgfqpoint{1.723956in}{2.100985in}}{\pgfqpoint{1.716056in}{2.097713in}}{\pgfqpoint{1.710232in}{2.091889in}}%
\pgfpathcurveto{\pgfqpoint{1.704408in}{2.086065in}}{\pgfqpoint{1.701136in}{2.078165in}}{\pgfqpoint{1.701136in}{2.069929in}}%
\pgfpathcurveto{\pgfqpoint{1.701136in}{2.061692in}}{\pgfqpoint{1.704408in}{2.053792in}}{\pgfqpoint{1.710232in}{2.047968in}}%
\pgfpathcurveto{\pgfqpoint{1.716056in}{2.042144in}}{\pgfqpoint{1.723956in}{2.038872in}}{\pgfqpoint{1.732193in}{2.038872in}}%
\pgfpathclose%
\pgfusepath{stroke,fill}%
\end{pgfscope}%
\begin{pgfscope}%
\pgfpathrectangle{\pgfqpoint{0.100000in}{0.212622in}}{\pgfqpoint{3.696000in}{3.696000in}}%
\pgfusepath{clip}%
\pgfsetbuttcap%
\pgfsetroundjoin%
\definecolor{currentfill}{rgb}{0.121569,0.466667,0.705882}%
\pgfsetfillcolor{currentfill}%
\pgfsetfillopacity{0.884831}%
\pgfsetlinewidth{1.003750pt}%
\definecolor{currentstroke}{rgb}{0.121569,0.466667,0.705882}%
\pgfsetstrokecolor{currentstroke}%
\pgfsetstrokeopacity{0.884831}%
\pgfsetdash{}{0pt}%
\pgfpathmoveto{\pgfqpoint{1.731843in}{2.038264in}}%
\pgfpathcurveto{\pgfqpoint{1.740079in}{2.038264in}}{\pgfqpoint{1.747979in}{2.041536in}}{\pgfqpoint{1.753803in}{2.047360in}}%
\pgfpathcurveto{\pgfqpoint{1.759627in}{2.053184in}}{\pgfqpoint{1.762899in}{2.061084in}}{\pgfqpoint{1.762899in}{2.069320in}}%
\pgfpathcurveto{\pgfqpoint{1.762899in}{2.077557in}}{\pgfqpoint{1.759627in}{2.085457in}}{\pgfqpoint{1.753803in}{2.091281in}}%
\pgfpathcurveto{\pgfqpoint{1.747979in}{2.097104in}}{\pgfqpoint{1.740079in}{2.100377in}}{\pgfqpoint{1.731843in}{2.100377in}}%
\pgfpathcurveto{\pgfqpoint{1.723607in}{2.100377in}}{\pgfqpoint{1.715706in}{2.097104in}}{\pgfqpoint{1.709883in}{2.091281in}}%
\pgfpathcurveto{\pgfqpoint{1.704059in}{2.085457in}}{\pgfqpoint{1.700786in}{2.077557in}}{\pgfqpoint{1.700786in}{2.069320in}}%
\pgfpathcurveto{\pgfqpoint{1.700786in}{2.061084in}}{\pgfqpoint{1.704059in}{2.053184in}}{\pgfqpoint{1.709883in}{2.047360in}}%
\pgfpathcurveto{\pgfqpoint{1.715706in}{2.041536in}}{\pgfqpoint{1.723607in}{2.038264in}}{\pgfqpoint{1.731843in}{2.038264in}}%
\pgfpathclose%
\pgfusepath{stroke,fill}%
\end{pgfscope}%
\begin{pgfscope}%
\pgfpathrectangle{\pgfqpoint{0.100000in}{0.212622in}}{\pgfqpoint{3.696000in}{3.696000in}}%
\pgfusepath{clip}%
\pgfsetbuttcap%
\pgfsetroundjoin%
\definecolor{currentfill}{rgb}{0.121569,0.466667,0.705882}%
\pgfsetfillcolor{currentfill}%
\pgfsetfillopacity{0.884889}%
\pgfsetlinewidth{1.003750pt}%
\definecolor{currentstroke}{rgb}{0.121569,0.466667,0.705882}%
\pgfsetstrokecolor{currentstroke}%
\pgfsetstrokeopacity{0.884889}%
\pgfsetdash{}{0pt}%
\pgfpathmoveto{\pgfqpoint{1.731629in}{2.037941in}}%
\pgfpathcurveto{\pgfqpoint{1.739865in}{2.037941in}}{\pgfqpoint{1.747765in}{2.041214in}}{\pgfqpoint{1.753589in}{2.047038in}}%
\pgfpathcurveto{\pgfqpoint{1.759413in}{2.052862in}}{\pgfqpoint{1.762685in}{2.060762in}}{\pgfqpoint{1.762685in}{2.068998in}}%
\pgfpathcurveto{\pgfqpoint{1.762685in}{2.077234in}}{\pgfqpoint{1.759413in}{2.085134in}}{\pgfqpoint{1.753589in}{2.090958in}}%
\pgfpathcurveto{\pgfqpoint{1.747765in}{2.096782in}}{\pgfqpoint{1.739865in}{2.100054in}}{\pgfqpoint{1.731629in}{2.100054in}}%
\pgfpathcurveto{\pgfqpoint{1.723392in}{2.100054in}}{\pgfqpoint{1.715492in}{2.096782in}}{\pgfqpoint{1.709668in}{2.090958in}}%
\pgfpathcurveto{\pgfqpoint{1.703844in}{2.085134in}}{\pgfqpoint{1.700572in}{2.077234in}}{\pgfqpoint{1.700572in}{2.068998in}}%
\pgfpathcurveto{\pgfqpoint{1.700572in}{2.060762in}}{\pgfqpoint{1.703844in}{2.052862in}}{\pgfqpoint{1.709668in}{2.047038in}}%
\pgfpathcurveto{\pgfqpoint{1.715492in}{2.041214in}}{\pgfqpoint{1.723392in}{2.037941in}}{\pgfqpoint{1.731629in}{2.037941in}}%
\pgfpathclose%
\pgfusepath{stroke,fill}%
\end{pgfscope}%
\begin{pgfscope}%
\pgfpathrectangle{\pgfqpoint{0.100000in}{0.212622in}}{\pgfqpoint{3.696000in}{3.696000in}}%
\pgfusepath{clip}%
\pgfsetbuttcap%
\pgfsetroundjoin%
\definecolor{currentfill}{rgb}{0.121569,0.466667,0.705882}%
\pgfsetfillcolor{currentfill}%
\pgfsetfillopacity{0.885118}%
\pgfsetlinewidth{1.003750pt}%
\definecolor{currentstroke}{rgb}{0.121569,0.466667,0.705882}%
\pgfsetstrokecolor{currentstroke}%
\pgfsetstrokeopacity{0.885118}%
\pgfsetdash{}{0pt}%
\pgfpathmoveto{\pgfqpoint{1.582836in}{1.197150in}}%
\pgfpathcurveto{\pgfqpoint{1.591072in}{1.197150in}}{\pgfqpoint{1.598972in}{1.200422in}}{\pgfqpoint{1.604796in}{1.206246in}}%
\pgfpathcurveto{\pgfqpoint{1.610620in}{1.212070in}}{\pgfqpoint{1.613892in}{1.219970in}}{\pgfqpoint{1.613892in}{1.228206in}}%
\pgfpathcurveto{\pgfqpoint{1.613892in}{1.236443in}}{\pgfqpoint{1.610620in}{1.244343in}}{\pgfqpoint{1.604796in}{1.250167in}}%
\pgfpathcurveto{\pgfqpoint{1.598972in}{1.255991in}}{\pgfqpoint{1.591072in}{1.259263in}}{\pgfqpoint{1.582836in}{1.259263in}}%
\pgfpathcurveto{\pgfqpoint{1.574600in}{1.259263in}}{\pgfqpoint{1.566699in}{1.255991in}}{\pgfqpoint{1.560876in}{1.250167in}}%
\pgfpathcurveto{\pgfqpoint{1.555052in}{1.244343in}}{\pgfqpoint{1.551779in}{1.236443in}}{\pgfqpoint{1.551779in}{1.228206in}}%
\pgfpathcurveto{\pgfqpoint{1.551779in}{1.219970in}}{\pgfqpoint{1.555052in}{1.212070in}}{\pgfqpoint{1.560876in}{1.206246in}}%
\pgfpathcurveto{\pgfqpoint{1.566699in}{1.200422in}}{\pgfqpoint{1.574600in}{1.197150in}}{\pgfqpoint{1.582836in}{1.197150in}}%
\pgfpathclose%
\pgfusepath{stroke,fill}%
\end{pgfscope}%
\begin{pgfscope}%
\pgfpathrectangle{\pgfqpoint{0.100000in}{0.212622in}}{\pgfqpoint{3.696000in}{3.696000in}}%
\pgfusepath{clip}%
\pgfsetbuttcap%
\pgfsetroundjoin%
\definecolor{currentfill}{rgb}{0.121569,0.466667,0.705882}%
\pgfsetfillcolor{currentfill}%
\pgfsetfillopacity{0.885118}%
\pgfsetlinewidth{1.003750pt}%
\definecolor{currentstroke}{rgb}{0.121569,0.466667,0.705882}%
\pgfsetstrokecolor{currentstroke}%
\pgfsetstrokeopacity{0.885118}%
\pgfsetdash{}{0pt}%
\pgfpathmoveto{\pgfqpoint{1.731051in}{2.036701in}}%
\pgfpathcurveto{\pgfqpoint{1.739287in}{2.036701in}}{\pgfqpoint{1.747187in}{2.039973in}}{\pgfqpoint{1.753011in}{2.045797in}}%
\pgfpathcurveto{\pgfqpoint{1.758835in}{2.051621in}}{\pgfqpoint{1.762107in}{2.059521in}}{\pgfqpoint{1.762107in}{2.067757in}}%
\pgfpathcurveto{\pgfqpoint{1.762107in}{2.075994in}}{\pgfqpoint{1.758835in}{2.083894in}}{\pgfqpoint{1.753011in}{2.089718in}}%
\pgfpathcurveto{\pgfqpoint{1.747187in}{2.095542in}}{\pgfqpoint{1.739287in}{2.098814in}}{\pgfqpoint{1.731051in}{2.098814in}}%
\pgfpathcurveto{\pgfqpoint{1.722815in}{2.098814in}}{\pgfqpoint{1.714915in}{2.095542in}}{\pgfqpoint{1.709091in}{2.089718in}}%
\pgfpathcurveto{\pgfqpoint{1.703267in}{2.083894in}}{\pgfqpoint{1.699994in}{2.075994in}}{\pgfqpoint{1.699994in}{2.067757in}}%
\pgfpathcurveto{\pgfqpoint{1.699994in}{2.059521in}}{\pgfqpoint{1.703267in}{2.051621in}}{\pgfqpoint{1.709091in}{2.045797in}}%
\pgfpathcurveto{\pgfqpoint{1.714915in}{2.039973in}}{\pgfqpoint{1.722815in}{2.036701in}}{\pgfqpoint{1.731051in}{2.036701in}}%
\pgfpathclose%
\pgfusepath{stroke,fill}%
\end{pgfscope}%
\begin{pgfscope}%
\pgfpathrectangle{\pgfqpoint{0.100000in}{0.212622in}}{\pgfqpoint{3.696000in}{3.696000in}}%
\pgfusepath{clip}%
\pgfsetbuttcap%
\pgfsetroundjoin%
\definecolor{currentfill}{rgb}{0.121569,0.466667,0.705882}%
\pgfsetfillcolor{currentfill}%
\pgfsetfillopacity{0.885145}%
\pgfsetlinewidth{1.003750pt}%
\definecolor{currentstroke}{rgb}{0.121569,0.466667,0.705882}%
\pgfsetstrokecolor{currentstroke}%
\pgfsetstrokeopacity{0.885145}%
\pgfsetdash{}{0pt}%
\pgfpathmoveto{\pgfqpoint{2.243962in}{2.213789in}}%
\pgfpathcurveto{\pgfqpoint{2.252198in}{2.213789in}}{\pgfqpoint{2.260098in}{2.217062in}}{\pgfqpoint{2.265922in}{2.222886in}}%
\pgfpathcurveto{\pgfqpoint{2.271746in}{2.228710in}}{\pgfqpoint{2.275018in}{2.236610in}}{\pgfqpoint{2.275018in}{2.244846in}}%
\pgfpathcurveto{\pgfqpoint{2.275018in}{2.253082in}}{\pgfqpoint{2.271746in}{2.260982in}}{\pgfqpoint{2.265922in}{2.266806in}}%
\pgfpathcurveto{\pgfqpoint{2.260098in}{2.272630in}}{\pgfqpoint{2.252198in}{2.275902in}}{\pgfqpoint{2.243962in}{2.275902in}}%
\pgfpathcurveto{\pgfqpoint{2.235725in}{2.275902in}}{\pgfqpoint{2.227825in}{2.272630in}}{\pgfqpoint{2.222001in}{2.266806in}}%
\pgfpathcurveto{\pgfqpoint{2.216177in}{2.260982in}}{\pgfqpoint{2.212905in}{2.253082in}}{\pgfqpoint{2.212905in}{2.244846in}}%
\pgfpathcurveto{\pgfqpoint{2.212905in}{2.236610in}}{\pgfqpoint{2.216177in}{2.228710in}}{\pgfqpoint{2.222001in}{2.222886in}}%
\pgfpathcurveto{\pgfqpoint{2.227825in}{2.217062in}}{\pgfqpoint{2.235725in}{2.213789in}}{\pgfqpoint{2.243962in}{2.213789in}}%
\pgfpathclose%
\pgfusepath{stroke,fill}%
\end{pgfscope}%
\begin{pgfscope}%
\pgfpathrectangle{\pgfqpoint{0.100000in}{0.212622in}}{\pgfqpoint{3.696000in}{3.696000in}}%
\pgfusepath{clip}%
\pgfsetbuttcap%
\pgfsetroundjoin%
\definecolor{currentfill}{rgb}{0.121569,0.466667,0.705882}%
\pgfsetfillcolor{currentfill}%
\pgfsetfillopacity{0.885233}%
\pgfsetlinewidth{1.003750pt}%
\definecolor{currentstroke}{rgb}{0.121569,0.466667,0.705882}%
\pgfsetstrokecolor{currentstroke}%
\pgfsetstrokeopacity{0.885233}%
\pgfsetdash{}{0pt}%
\pgfpathmoveto{\pgfqpoint{1.730800in}{2.035924in}}%
\pgfpathcurveto{\pgfqpoint{1.739037in}{2.035924in}}{\pgfqpoint{1.746937in}{2.039197in}}{\pgfqpoint{1.752761in}{2.045021in}}%
\pgfpathcurveto{\pgfqpoint{1.758585in}{2.050845in}}{\pgfqpoint{1.761857in}{2.058745in}}{\pgfqpoint{1.761857in}{2.066981in}}%
\pgfpathcurveto{\pgfqpoint{1.761857in}{2.075217in}}{\pgfqpoint{1.758585in}{2.083117in}}{\pgfqpoint{1.752761in}{2.088941in}}%
\pgfpathcurveto{\pgfqpoint{1.746937in}{2.094765in}}{\pgfqpoint{1.739037in}{2.098037in}}{\pgfqpoint{1.730800in}{2.098037in}}%
\pgfpathcurveto{\pgfqpoint{1.722564in}{2.098037in}}{\pgfqpoint{1.714664in}{2.094765in}}{\pgfqpoint{1.708840in}{2.088941in}}%
\pgfpathcurveto{\pgfqpoint{1.703016in}{2.083117in}}{\pgfqpoint{1.699744in}{2.075217in}}{\pgfqpoint{1.699744in}{2.066981in}}%
\pgfpathcurveto{\pgfqpoint{1.699744in}{2.058745in}}{\pgfqpoint{1.703016in}{2.050845in}}{\pgfqpoint{1.708840in}{2.045021in}}%
\pgfpathcurveto{\pgfqpoint{1.714664in}{2.039197in}}{\pgfqpoint{1.722564in}{2.035924in}}{\pgfqpoint{1.730800in}{2.035924in}}%
\pgfpathclose%
\pgfusepath{stroke,fill}%
\end{pgfscope}%
\begin{pgfscope}%
\pgfpathrectangle{\pgfqpoint{0.100000in}{0.212622in}}{\pgfqpoint{3.696000in}{3.696000in}}%
\pgfusepath{clip}%
\pgfsetbuttcap%
\pgfsetroundjoin%
\definecolor{currentfill}{rgb}{0.121569,0.466667,0.705882}%
\pgfsetfillcolor{currentfill}%
\pgfsetfillopacity{0.885297}%
\pgfsetlinewidth{1.003750pt}%
\definecolor{currentstroke}{rgb}{0.121569,0.466667,0.705882}%
\pgfsetstrokecolor{currentstroke}%
\pgfsetstrokeopacity{0.885297}%
\pgfsetdash{}{0pt}%
\pgfpathmoveto{\pgfqpoint{1.730588in}{2.035569in}}%
\pgfpathcurveto{\pgfqpoint{1.738824in}{2.035569in}}{\pgfqpoint{1.746724in}{2.038842in}}{\pgfqpoint{1.752548in}{2.044666in}}%
\pgfpathcurveto{\pgfqpoint{1.758372in}{2.050490in}}{\pgfqpoint{1.761644in}{2.058390in}}{\pgfqpoint{1.761644in}{2.066626in}}%
\pgfpathcurveto{\pgfqpoint{1.761644in}{2.074862in}}{\pgfqpoint{1.758372in}{2.082762in}}{\pgfqpoint{1.752548in}{2.088586in}}%
\pgfpathcurveto{\pgfqpoint{1.746724in}{2.094410in}}{\pgfqpoint{1.738824in}{2.097682in}}{\pgfqpoint{1.730588in}{2.097682in}}%
\pgfpathcurveto{\pgfqpoint{1.722351in}{2.097682in}}{\pgfqpoint{1.714451in}{2.094410in}}{\pgfqpoint{1.708627in}{2.088586in}}%
\pgfpathcurveto{\pgfqpoint{1.702803in}{2.082762in}}{\pgfqpoint{1.699531in}{2.074862in}}{\pgfqpoint{1.699531in}{2.066626in}}%
\pgfpathcurveto{\pgfqpoint{1.699531in}{2.058390in}}{\pgfqpoint{1.702803in}{2.050490in}}{\pgfqpoint{1.708627in}{2.044666in}}%
\pgfpathcurveto{\pgfqpoint{1.714451in}{2.038842in}}{\pgfqpoint{1.722351in}{2.035569in}}{\pgfqpoint{1.730588in}{2.035569in}}%
\pgfpathclose%
\pgfusepath{stroke,fill}%
\end{pgfscope}%
\begin{pgfscope}%
\pgfpathrectangle{\pgfqpoint{0.100000in}{0.212622in}}{\pgfqpoint{3.696000in}{3.696000in}}%
\pgfusepath{clip}%
\pgfsetbuttcap%
\pgfsetroundjoin%
\definecolor{currentfill}{rgb}{0.121569,0.466667,0.705882}%
\pgfsetfillcolor{currentfill}%
\pgfsetfillopacity{0.885330}%
\pgfsetlinewidth{1.003750pt}%
\definecolor{currentstroke}{rgb}{0.121569,0.466667,0.705882}%
\pgfsetstrokecolor{currentstroke}%
\pgfsetstrokeopacity{0.885330}%
\pgfsetdash{}{0pt}%
\pgfpathmoveto{\pgfqpoint{1.730465in}{2.035377in}}%
\pgfpathcurveto{\pgfqpoint{1.738701in}{2.035377in}}{\pgfqpoint{1.746601in}{2.038649in}}{\pgfqpoint{1.752425in}{2.044473in}}%
\pgfpathcurveto{\pgfqpoint{1.758249in}{2.050297in}}{\pgfqpoint{1.761521in}{2.058197in}}{\pgfqpoint{1.761521in}{2.066434in}}%
\pgfpathcurveto{\pgfqpoint{1.761521in}{2.074670in}}{\pgfqpoint{1.758249in}{2.082570in}}{\pgfqpoint{1.752425in}{2.088394in}}%
\pgfpathcurveto{\pgfqpoint{1.746601in}{2.094218in}}{\pgfqpoint{1.738701in}{2.097490in}}{\pgfqpoint{1.730465in}{2.097490in}}%
\pgfpathcurveto{\pgfqpoint{1.722228in}{2.097490in}}{\pgfqpoint{1.714328in}{2.094218in}}{\pgfqpoint{1.708504in}{2.088394in}}%
\pgfpathcurveto{\pgfqpoint{1.702680in}{2.082570in}}{\pgfqpoint{1.699408in}{2.074670in}}{\pgfqpoint{1.699408in}{2.066434in}}%
\pgfpathcurveto{\pgfqpoint{1.699408in}{2.058197in}}{\pgfqpoint{1.702680in}{2.050297in}}{\pgfqpoint{1.708504in}{2.044473in}}%
\pgfpathcurveto{\pgfqpoint{1.714328in}{2.038649in}}{\pgfqpoint{1.722228in}{2.035377in}}{\pgfqpoint{1.730465in}{2.035377in}}%
\pgfpathclose%
\pgfusepath{stroke,fill}%
\end{pgfscope}%
\begin{pgfscope}%
\pgfpathrectangle{\pgfqpoint{0.100000in}{0.212622in}}{\pgfqpoint{3.696000in}{3.696000in}}%
\pgfusepath{clip}%
\pgfsetbuttcap%
\pgfsetroundjoin%
\definecolor{currentfill}{rgb}{0.121569,0.466667,0.705882}%
\pgfsetfillcolor{currentfill}%
\pgfsetfillopacity{0.885350}%
\pgfsetlinewidth{1.003750pt}%
\definecolor{currentstroke}{rgb}{0.121569,0.466667,0.705882}%
\pgfsetstrokecolor{currentstroke}%
\pgfsetstrokeopacity{0.885350}%
\pgfsetdash{}{0pt}%
\pgfpathmoveto{\pgfqpoint{1.730405in}{2.035266in}}%
\pgfpathcurveto{\pgfqpoint{1.738641in}{2.035266in}}{\pgfqpoint{1.746541in}{2.038538in}}{\pgfqpoint{1.752365in}{2.044362in}}%
\pgfpathcurveto{\pgfqpoint{1.758189in}{2.050186in}}{\pgfqpoint{1.761461in}{2.058086in}}{\pgfqpoint{1.761461in}{2.066323in}}%
\pgfpathcurveto{\pgfqpoint{1.761461in}{2.074559in}}{\pgfqpoint{1.758189in}{2.082459in}}{\pgfqpoint{1.752365in}{2.088283in}}%
\pgfpathcurveto{\pgfqpoint{1.746541in}{2.094107in}}{\pgfqpoint{1.738641in}{2.097379in}}{\pgfqpoint{1.730405in}{2.097379in}}%
\pgfpathcurveto{\pgfqpoint{1.722168in}{2.097379in}}{\pgfqpoint{1.714268in}{2.094107in}}{\pgfqpoint{1.708444in}{2.088283in}}%
\pgfpathcurveto{\pgfqpoint{1.702620in}{2.082459in}}{\pgfqpoint{1.699348in}{2.074559in}}{\pgfqpoint{1.699348in}{2.066323in}}%
\pgfpathcurveto{\pgfqpoint{1.699348in}{2.058086in}}{\pgfqpoint{1.702620in}{2.050186in}}{\pgfqpoint{1.708444in}{2.044362in}}%
\pgfpathcurveto{\pgfqpoint{1.714268in}{2.038538in}}{\pgfqpoint{1.722168in}{2.035266in}}{\pgfqpoint{1.730405in}{2.035266in}}%
\pgfpathclose%
\pgfusepath{stroke,fill}%
\end{pgfscope}%
\begin{pgfscope}%
\pgfpathrectangle{\pgfqpoint{0.100000in}{0.212622in}}{\pgfqpoint{3.696000in}{3.696000in}}%
\pgfusepath{clip}%
\pgfsetbuttcap%
\pgfsetroundjoin%
\definecolor{currentfill}{rgb}{0.121569,0.466667,0.705882}%
\pgfsetfillcolor{currentfill}%
\pgfsetfillopacity{0.885362}%
\pgfsetlinewidth{1.003750pt}%
\definecolor{currentstroke}{rgb}{0.121569,0.466667,0.705882}%
\pgfsetstrokecolor{currentstroke}%
\pgfsetstrokeopacity{0.885362}%
\pgfsetdash{}{0pt}%
\pgfpathmoveto{\pgfqpoint{1.730374in}{2.035206in}}%
\pgfpathcurveto{\pgfqpoint{1.738611in}{2.035206in}}{\pgfqpoint{1.746511in}{2.038479in}}{\pgfqpoint{1.752335in}{2.044302in}}%
\pgfpathcurveto{\pgfqpoint{1.758159in}{2.050126in}}{\pgfqpoint{1.761431in}{2.058026in}}{\pgfqpoint{1.761431in}{2.066263in}}%
\pgfpathcurveto{\pgfqpoint{1.761431in}{2.074499in}}{\pgfqpoint{1.758159in}{2.082399in}}{\pgfqpoint{1.752335in}{2.088223in}}%
\pgfpathcurveto{\pgfqpoint{1.746511in}{2.094047in}}{\pgfqpoint{1.738611in}{2.097319in}}{\pgfqpoint{1.730374in}{2.097319in}}%
\pgfpathcurveto{\pgfqpoint{1.722138in}{2.097319in}}{\pgfqpoint{1.714238in}{2.094047in}}{\pgfqpoint{1.708414in}{2.088223in}}%
\pgfpathcurveto{\pgfqpoint{1.702590in}{2.082399in}}{\pgfqpoint{1.699318in}{2.074499in}}{\pgfqpoint{1.699318in}{2.066263in}}%
\pgfpathcurveto{\pgfqpoint{1.699318in}{2.058026in}}{\pgfqpoint{1.702590in}{2.050126in}}{\pgfqpoint{1.708414in}{2.044302in}}%
\pgfpathcurveto{\pgfqpoint{1.714238in}{2.038479in}}{\pgfqpoint{1.722138in}{2.035206in}}{\pgfqpoint{1.730374in}{2.035206in}}%
\pgfpathclose%
\pgfusepath{stroke,fill}%
\end{pgfscope}%
\begin{pgfscope}%
\pgfpathrectangle{\pgfqpoint{0.100000in}{0.212622in}}{\pgfqpoint{3.696000in}{3.696000in}}%
\pgfusepath{clip}%
\pgfsetbuttcap%
\pgfsetroundjoin%
\definecolor{currentfill}{rgb}{0.121569,0.466667,0.705882}%
\pgfsetfillcolor{currentfill}%
\pgfsetfillopacity{0.885368}%
\pgfsetlinewidth{1.003750pt}%
\definecolor{currentstroke}{rgb}{0.121569,0.466667,0.705882}%
\pgfsetstrokecolor{currentstroke}%
\pgfsetstrokeopacity{0.885368}%
\pgfsetdash{}{0pt}%
\pgfpathmoveto{\pgfqpoint{1.730357in}{2.035171in}}%
\pgfpathcurveto{\pgfqpoint{1.738593in}{2.035171in}}{\pgfqpoint{1.746493in}{2.038444in}}{\pgfqpoint{1.752317in}{2.044268in}}%
\pgfpathcurveto{\pgfqpoint{1.758141in}{2.050091in}}{\pgfqpoint{1.761413in}{2.057992in}}{\pgfqpoint{1.761413in}{2.066228in}}%
\pgfpathcurveto{\pgfqpoint{1.761413in}{2.074464in}}{\pgfqpoint{1.758141in}{2.082364in}}{\pgfqpoint{1.752317in}{2.088188in}}%
\pgfpathcurveto{\pgfqpoint{1.746493in}{2.094012in}}{\pgfqpoint{1.738593in}{2.097284in}}{\pgfqpoint{1.730357in}{2.097284in}}%
\pgfpathcurveto{\pgfqpoint{1.722121in}{2.097284in}}{\pgfqpoint{1.714221in}{2.094012in}}{\pgfqpoint{1.708397in}{2.088188in}}%
\pgfpathcurveto{\pgfqpoint{1.702573in}{2.082364in}}{\pgfqpoint{1.699300in}{2.074464in}}{\pgfqpoint{1.699300in}{2.066228in}}%
\pgfpathcurveto{\pgfqpoint{1.699300in}{2.057992in}}{\pgfqpoint{1.702573in}{2.050091in}}{\pgfqpoint{1.708397in}{2.044268in}}%
\pgfpathcurveto{\pgfqpoint{1.714221in}{2.038444in}}{\pgfqpoint{1.722121in}{2.035171in}}{\pgfqpoint{1.730357in}{2.035171in}}%
\pgfpathclose%
\pgfusepath{stroke,fill}%
\end{pgfscope}%
\begin{pgfscope}%
\pgfpathrectangle{\pgfqpoint{0.100000in}{0.212622in}}{\pgfqpoint{3.696000in}{3.696000in}}%
\pgfusepath{clip}%
\pgfsetbuttcap%
\pgfsetroundjoin%
\definecolor{currentfill}{rgb}{0.121569,0.466667,0.705882}%
\pgfsetfillcolor{currentfill}%
\pgfsetfillopacity{0.885372}%
\pgfsetlinewidth{1.003750pt}%
\definecolor{currentstroke}{rgb}{0.121569,0.466667,0.705882}%
\pgfsetstrokecolor{currentstroke}%
\pgfsetstrokeopacity{0.885372}%
\pgfsetdash{}{0pt}%
\pgfpathmoveto{\pgfqpoint{1.730348in}{2.035154in}}%
\pgfpathcurveto{\pgfqpoint{1.738584in}{2.035154in}}{\pgfqpoint{1.746484in}{2.038426in}}{\pgfqpoint{1.752308in}{2.044250in}}%
\pgfpathcurveto{\pgfqpoint{1.758132in}{2.050074in}}{\pgfqpoint{1.761404in}{2.057974in}}{\pgfqpoint{1.761404in}{2.066210in}}%
\pgfpathcurveto{\pgfqpoint{1.761404in}{2.074446in}}{\pgfqpoint{1.758132in}{2.082346in}}{\pgfqpoint{1.752308in}{2.088170in}}%
\pgfpathcurveto{\pgfqpoint{1.746484in}{2.093994in}}{\pgfqpoint{1.738584in}{2.097267in}}{\pgfqpoint{1.730348in}{2.097267in}}%
\pgfpathcurveto{\pgfqpoint{1.722111in}{2.097267in}}{\pgfqpoint{1.714211in}{2.093994in}}{\pgfqpoint{1.708387in}{2.088170in}}%
\pgfpathcurveto{\pgfqpoint{1.702563in}{2.082346in}}{\pgfqpoint{1.699291in}{2.074446in}}{\pgfqpoint{1.699291in}{2.066210in}}%
\pgfpathcurveto{\pgfqpoint{1.699291in}{2.057974in}}{\pgfqpoint{1.702563in}{2.050074in}}{\pgfqpoint{1.708387in}{2.044250in}}%
\pgfpathcurveto{\pgfqpoint{1.714211in}{2.038426in}}{\pgfqpoint{1.722111in}{2.035154in}}{\pgfqpoint{1.730348in}{2.035154in}}%
\pgfpathclose%
\pgfusepath{stroke,fill}%
\end{pgfscope}%
\begin{pgfscope}%
\pgfpathrectangle{\pgfqpoint{0.100000in}{0.212622in}}{\pgfqpoint{3.696000in}{3.696000in}}%
\pgfusepath{clip}%
\pgfsetbuttcap%
\pgfsetroundjoin%
\definecolor{currentfill}{rgb}{0.121569,0.466667,0.705882}%
\pgfsetfillcolor{currentfill}%
\pgfsetfillopacity{0.885374}%
\pgfsetlinewidth{1.003750pt}%
\definecolor{currentstroke}{rgb}{0.121569,0.466667,0.705882}%
\pgfsetstrokecolor{currentstroke}%
\pgfsetstrokeopacity{0.885374}%
\pgfsetdash{}{0pt}%
\pgfpathmoveto{\pgfqpoint{1.730342in}{2.035144in}}%
\pgfpathcurveto{\pgfqpoint{1.738579in}{2.035144in}}{\pgfqpoint{1.746479in}{2.038416in}}{\pgfqpoint{1.752303in}{2.044240in}}%
\pgfpathcurveto{\pgfqpoint{1.758127in}{2.050064in}}{\pgfqpoint{1.761399in}{2.057964in}}{\pgfqpoint{1.761399in}{2.066200in}}%
\pgfpathcurveto{\pgfqpoint{1.761399in}{2.074436in}}{\pgfqpoint{1.758127in}{2.082336in}}{\pgfqpoint{1.752303in}{2.088160in}}%
\pgfpathcurveto{\pgfqpoint{1.746479in}{2.093984in}}{\pgfqpoint{1.738579in}{2.097257in}}{\pgfqpoint{1.730342in}{2.097257in}}%
\pgfpathcurveto{\pgfqpoint{1.722106in}{2.097257in}}{\pgfqpoint{1.714206in}{2.093984in}}{\pgfqpoint{1.708382in}{2.088160in}}%
\pgfpathcurveto{\pgfqpoint{1.702558in}{2.082336in}}{\pgfqpoint{1.699286in}{2.074436in}}{\pgfqpoint{1.699286in}{2.066200in}}%
\pgfpathcurveto{\pgfqpoint{1.699286in}{2.057964in}}{\pgfqpoint{1.702558in}{2.050064in}}{\pgfqpoint{1.708382in}{2.044240in}}%
\pgfpathcurveto{\pgfqpoint{1.714206in}{2.038416in}}{\pgfqpoint{1.722106in}{2.035144in}}{\pgfqpoint{1.730342in}{2.035144in}}%
\pgfpathclose%
\pgfusepath{stroke,fill}%
\end{pgfscope}%
\begin{pgfscope}%
\pgfpathrectangle{\pgfqpoint{0.100000in}{0.212622in}}{\pgfqpoint{3.696000in}{3.696000in}}%
\pgfusepath{clip}%
\pgfsetbuttcap%
\pgfsetroundjoin%
\definecolor{currentfill}{rgb}{0.121569,0.466667,0.705882}%
\pgfsetfillcolor{currentfill}%
\pgfsetfillopacity{0.885375}%
\pgfsetlinewidth{1.003750pt}%
\definecolor{currentstroke}{rgb}{0.121569,0.466667,0.705882}%
\pgfsetstrokecolor{currentstroke}%
\pgfsetstrokeopacity{0.885375}%
\pgfsetdash{}{0pt}%
\pgfpathmoveto{\pgfqpoint{1.730340in}{2.035138in}}%
\pgfpathcurveto{\pgfqpoint{1.738576in}{2.035138in}}{\pgfqpoint{1.746476in}{2.038411in}}{\pgfqpoint{1.752300in}{2.044234in}}%
\pgfpathcurveto{\pgfqpoint{1.758124in}{2.050058in}}{\pgfqpoint{1.761396in}{2.057958in}}{\pgfqpoint{1.761396in}{2.066195in}}%
\pgfpathcurveto{\pgfqpoint{1.761396in}{2.074431in}}{\pgfqpoint{1.758124in}{2.082331in}}{\pgfqpoint{1.752300in}{2.088155in}}%
\pgfpathcurveto{\pgfqpoint{1.746476in}{2.093979in}}{\pgfqpoint{1.738576in}{2.097251in}}{\pgfqpoint{1.730340in}{2.097251in}}%
\pgfpathcurveto{\pgfqpoint{1.722103in}{2.097251in}}{\pgfqpoint{1.714203in}{2.093979in}}{\pgfqpoint{1.708380in}{2.088155in}}%
\pgfpathcurveto{\pgfqpoint{1.702556in}{2.082331in}}{\pgfqpoint{1.699283in}{2.074431in}}{\pgfqpoint{1.699283in}{2.066195in}}%
\pgfpathcurveto{\pgfqpoint{1.699283in}{2.057958in}}{\pgfqpoint{1.702556in}{2.050058in}}{\pgfqpoint{1.708380in}{2.044234in}}%
\pgfpathcurveto{\pgfqpoint{1.714203in}{2.038411in}}{\pgfqpoint{1.722103in}{2.035138in}}{\pgfqpoint{1.730340in}{2.035138in}}%
\pgfpathclose%
\pgfusepath{stroke,fill}%
\end{pgfscope}%
\begin{pgfscope}%
\pgfpathrectangle{\pgfqpoint{0.100000in}{0.212622in}}{\pgfqpoint{3.696000in}{3.696000in}}%
\pgfusepath{clip}%
\pgfsetbuttcap%
\pgfsetroundjoin%
\definecolor{currentfill}{rgb}{0.121569,0.466667,0.705882}%
\pgfsetfillcolor{currentfill}%
\pgfsetfillopacity{0.885375}%
\pgfsetlinewidth{1.003750pt}%
\definecolor{currentstroke}{rgb}{0.121569,0.466667,0.705882}%
\pgfsetstrokecolor{currentstroke}%
\pgfsetstrokeopacity{0.885375}%
\pgfsetdash{}{0pt}%
\pgfpathmoveto{\pgfqpoint{1.730338in}{2.035135in}}%
\pgfpathcurveto{\pgfqpoint{1.738575in}{2.035135in}}{\pgfqpoint{1.746475in}{2.038407in}}{\pgfqpoint{1.752299in}{2.044231in}}%
\pgfpathcurveto{\pgfqpoint{1.758123in}{2.050055in}}{\pgfqpoint{1.761395in}{2.057955in}}{\pgfqpoint{1.761395in}{2.066192in}}%
\pgfpathcurveto{\pgfqpoint{1.761395in}{2.074428in}}{\pgfqpoint{1.758123in}{2.082328in}}{\pgfqpoint{1.752299in}{2.088152in}}%
\pgfpathcurveto{\pgfqpoint{1.746475in}{2.093976in}}{\pgfqpoint{1.738575in}{2.097248in}}{\pgfqpoint{1.730338in}{2.097248in}}%
\pgfpathcurveto{\pgfqpoint{1.722102in}{2.097248in}}{\pgfqpoint{1.714202in}{2.093976in}}{\pgfqpoint{1.708378in}{2.088152in}}%
\pgfpathcurveto{\pgfqpoint{1.702554in}{2.082328in}}{\pgfqpoint{1.699282in}{2.074428in}}{\pgfqpoint{1.699282in}{2.066192in}}%
\pgfpathcurveto{\pgfqpoint{1.699282in}{2.057955in}}{\pgfqpoint{1.702554in}{2.050055in}}{\pgfqpoint{1.708378in}{2.044231in}}%
\pgfpathcurveto{\pgfqpoint{1.714202in}{2.038407in}}{\pgfqpoint{1.722102in}{2.035135in}}{\pgfqpoint{1.730338in}{2.035135in}}%
\pgfpathclose%
\pgfusepath{stroke,fill}%
\end{pgfscope}%
\begin{pgfscope}%
\pgfpathrectangle{\pgfqpoint{0.100000in}{0.212622in}}{\pgfqpoint{3.696000in}{3.696000in}}%
\pgfusepath{clip}%
\pgfsetbuttcap%
\pgfsetroundjoin%
\definecolor{currentfill}{rgb}{0.121569,0.466667,0.705882}%
\pgfsetfillcolor{currentfill}%
\pgfsetfillopacity{0.885376}%
\pgfsetlinewidth{1.003750pt}%
\definecolor{currentstroke}{rgb}{0.121569,0.466667,0.705882}%
\pgfsetstrokecolor{currentstroke}%
\pgfsetstrokeopacity{0.885376}%
\pgfsetdash{}{0pt}%
\pgfpathmoveto{\pgfqpoint{1.730338in}{2.035133in}}%
\pgfpathcurveto{\pgfqpoint{1.738574in}{2.035133in}}{\pgfqpoint{1.746474in}{2.038406in}}{\pgfqpoint{1.752298in}{2.044230in}}%
\pgfpathcurveto{\pgfqpoint{1.758122in}{2.050054in}}{\pgfqpoint{1.761394in}{2.057954in}}{\pgfqpoint{1.761394in}{2.066190in}}%
\pgfpathcurveto{\pgfqpoint{1.761394in}{2.074426in}}{\pgfqpoint{1.758122in}{2.082326in}}{\pgfqpoint{1.752298in}{2.088150in}}%
\pgfpathcurveto{\pgfqpoint{1.746474in}{2.093974in}}{\pgfqpoint{1.738574in}{2.097246in}}{\pgfqpoint{1.730338in}{2.097246in}}%
\pgfpathcurveto{\pgfqpoint{1.722101in}{2.097246in}}{\pgfqpoint{1.714201in}{2.093974in}}{\pgfqpoint{1.708377in}{2.088150in}}%
\pgfpathcurveto{\pgfqpoint{1.702553in}{2.082326in}}{\pgfqpoint{1.699281in}{2.074426in}}{\pgfqpoint{1.699281in}{2.066190in}}%
\pgfpathcurveto{\pgfqpoint{1.699281in}{2.057954in}}{\pgfqpoint{1.702553in}{2.050054in}}{\pgfqpoint{1.708377in}{2.044230in}}%
\pgfpathcurveto{\pgfqpoint{1.714201in}{2.038406in}}{\pgfqpoint{1.722101in}{2.035133in}}{\pgfqpoint{1.730338in}{2.035133in}}%
\pgfpathclose%
\pgfusepath{stroke,fill}%
\end{pgfscope}%
\begin{pgfscope}%
\pgfpathrectangle{\pgfqpoint{0.100000in}{0.212622in}}{\pgfqpoint{3.696000in}{3.696000in}}%
\pgfusepath{clip}%
\pgfsetbuttcap%
\pgfsetroundjoin%
\definecolor{currentfill}{rgb}{0.121569,0.466667,0.705882}%
\pgfsetfillcolor{currentfill}%
\pgfsetfillopacity{0.885376}%
\pgfsetlinewidth{1.003750pt}%
\definecolor{currentstroke}{rgb}{0.121569,0.466667,0.705882}%
\pgfsetstrokecolor{currentstroke}%
\pgfsetstrokeopacity{0.885376}%
\pgfsetdash{}{0pt}%
\pgfpathmoveto{\pgfqpoint{1.730337in}{2.035133in}}%
\pgfpathcurveto{\pgfqpoint{1.738573in}{2.035133in}}{\pgfqpoint{1.746474in}{2.038405in}}{\pgfqpoint{1.752297in}{2.044229in}}%
\pgfpathcurveto{\pgfqpoint{1.758121in}{2.050053in}}{\pgfqpoint{1.761394in}{2.057953in}}{\pgfqpoint{1.761394in}{2.066189in}}%
\pgfpathcurveto{\pgfqpoint{1.761394in}{2.074425in}}{\pgfqpoint{1.758121in}{2.082325in}}{\pgfqpoint{1.752297in}{2.088149in}}%
\pgfpathcurveto{\pgfqpoint{1.746474in}{2.093973in}}{\pgfqpoint{1.738573in}{2.097246in}}{\pgfqpoint{1.730337in}{2.097246in}}%
\pgfpathcurveto{\pgfqpoint{1.722101in}{2.097246in}}{\pgfqpoint{1.714201in}{2.093973in}}{\pgfqpoint{1.708377in}{2.088149in}}%
\pgfpathcurveto{\pgfqpoint{1.702553in}{2.082325in}}{\pgfqpoint{1.699281in}{2.074425in}}{\pgfqpoint{1.699281in}{2.066189in}}%
\pgfpathcurveto{\pgfqpoint{1.699281in}{2.057953in}}{\pgfqpoint{1.702553in}{2.050053in}}{\pgfqpoint{1.708377in}{2.044229in}}%
\pgfpathcurveto{\pgfqpoint{1.714201in}{2.038405in}}{\pgfqpoint{1.722101in}{2.035133in}}{\pgfqpoint{1.730337in}{2.035133in}}%
\pgfpathclose%
\pgfusepath{stroke,fill}%
\end{pgfscope}%
\begin{pgfscope}%
\pgfpathrectangle{\pgfqpoint{0.100000in}{0.212622in}}{\pgfqpoint{3.696000in}{3.696000in}}%
\pgfusepath{clip}%
\pgfsetbuttcap%
\pgfsetroundjoin%
\definecolor{currentfill}{rgb}{0.121569,0.466667,0.705882}%
\pgfsetfillcolor{currentfill}%
\pgfsetfillopacity{0.885470}%
\pgfsetlinewidth{1.003750pt}%
\definecolor{currentstroke}{rgb}{0.121569,0.466667,0.705882}%
\pgfsetstrokecolor{currentstroke}%
\pgfsetstrokeopacity{0.885470}%
\pgfsetdash{}{0pt}%
\pgfpathmoveto{\pgfqpoint{1.730130in}{2.034645in}}%
\pgfpathcurveto{\pgfqpoint{1.738366in}{2.034645in}}{\pgfqpoint{1.746266in}{2.037917in}}{\pgfqpoint{1.752090in}{2.043741in}}%
\pgfpathcurveto{\pgfqpoint{1.757914in}{2.049565in}}{\pgfqpoint{1.761186in}{2.057465in}}{\pgfqpoint{1.761186in}{2.065701in}}%
\pgfpathcurveto{\pgfqpoint{1.761186in}{2.073937in}}{\pgfqpoint{1.757914in}{2.081838in}}{\pgfqpoint{1.752090in}{2.087661in}}%
\pgfpathcurveto{\pgfqpoint{1.746266in}{2.093485in}}{\pgfqpoint{1.738366in}{2.096758in}}{\pgfqpoint{1.730130in}{2.096758in}}%
\pgfpathcurveto{\pgfqpoint{1.721893in}{2.096758in}}{\pgfqpoint{1.713993in}{2.093485in}}{\pgfqpoint{1.708169in}{2.087661in}}%
\pgfpathcurveto{\pgfqpoint{1.702346in}{2.081838in}}{\pgfqpoint{1.699073in}{2.073937in}}{\pgfqpoint{1.699073in}{2.065701in}}%
\pgfpathcurveto{\pgfqpoint{1.699073in}{2.057465in}}{\pgfqpoint{1.702346in}{2.049565in}}{\pgfqpoint{1.708169in}{2.043741in}}%
\pgfpathcurveto{\pgfqpoint{1.713993in}{2.037917in}}{\pgfqpoint{1.721893in}{2.034645in}}{\pgfqpoint{1.730130in}{2.034645in}}%
\pgfpathclose%
\pgfusepath{stroke,fill}%
\end{pgfscope}%
\begin{pgfscope}%
\pgfpathrectangle{\pgfqpoint{0.100000in}{0.212622in}}{\pgfqpoint{3.696000in}{3.696000in}}%
\pgfusepath{clip}%
\pgfsetbuttcap%
\pgfsetroundjoin%
\definecolor{currentfill}{rgb}{0.121569,0.466667,0.705882}%
\pgfsetfillcolor{currentfill}%
\pgfsetfillopacity{0.885651}%
\pgfsetlinewidth{1.003750pt}%
\definecolor{currentstroke}{rgb}{0.121569,0.466667,0.705882}%
\pgfsetstrokecolor{currentstroke}%
\pgfsetstrokeopacity{0.885651}%
\pgfsetdash{}{0pt}%
\pgfpathmoveto{\pgfqpoint{1.729800in}{2.033796in}}%
\pgfpathcurveto{\pgfqpoint{1.738036in}{2.033796in}}{\pgfqpoint{1.745936in}{2.037069in}}{\pgfqpoint{1.751760in}{2.042893in}}%
\pgfpathcurveto{\pgfqpoint{1.757584in}{2.048716in}}{\pgfqpoint{1.760856in}{2.056617in}}{\pgfqpoint{1.760856in}{2.064853in}}%
\pgfpathcurveto{\pgfqpoint{1.760856in}{2.073089in}}{\pgfqpoint{1.757584in}{2.080989in}}{\pgfqpoint{1.751760in}{2.086813in}}%
\pgfpathcurveto{\pgfqpoint{1.745936in}{2.092637in}}{\pgfqpoint{1.738036in}{2.095909in}}{\pgfqpoint{1.729800in}{2.095909in}}%
\pgfpathcurveto{\pgfqpoint{1.721563in}{2.095909in}}{\pgfqpoint{1.713663in}{2.092637in}}{\pgfqpoint{1.707840in}{2.086813in}}%
\pgfpathcurveto{\pgfqpoint{1.702016in}{2.080989in}}{\pgfqpoint{1.698743in}{2.073089in}}{\pgfqpoint{1.698743in}{2.064853in}}%
\pgfpathcurveto{\pgfqpoint{1.698743in}{2.056617in}}{\pgfqpoint{1.702016in}{2.048716in}}{\pgfqpoint{1.707840in}{2.042893in}}%
\pgfpathcurveto{\pgfqpoint{1.713663in}{2.037069in}}{\pgfqpoint{1.721563in}{2.033796in}}{\pgfqpoint{1.729800in}{2.033796in}}%
\pgfpathclose%
\pgfusepath{stroke,fill}%
\end{pgfscope}%
\begin{pgfscope}%
\pgfpathrectangle{\pgfqpoint{0.100000in}{0.212622in}}{\pgfqpoint{3.696000in}{3.696000in}}%
\pgfusepath{clip}%
\pgfsetbuttcap%
\pgfsetroundjoin%
\definecolor{currentfill}{rgb}{0.121569,0.466667,0.705882}%
\pgfsetfillcolor{currentfill}%
\pgfsetfillopacity{0.885792}%
\pgfsetlinewidth{1.003750pt}%
\definecolor{currentstroke}{rgb}{0.121569,0.466667,0.705882}%
\pgfsetstrokecolor{currentstroke}%
\pgfsetstrokeopacity{0.885792}%
\pgfsetdash{}{0pt}%
\pgfpathmoveto{\pgfqpoint{2.243511in}{2.209509in}}%
\pgfpathcurveto{\pgfqpoint{2.251748in}{2.209509in}}{\pgfqpoint{2.259648in}{2.212782in}}{\pgfqpoint{2.265472in}{2.218606in}}%
\pgfpathcurveto{\pgfqpoint{2.271295in}{2.224430in}}{\pgfqpoint{2.274568in}{2.232330in}}{\pgfqpoint{2.274568in}{2.240566in}}%
\pgfpathcurveto{\pgfqpoint{2.274568in}{2.248802in}}{\pgfqpoint{2.271295in}{2.256702in}}{\pgfqpoint{2.265472in}{2.262526in}}%
\pgfpathcurveto{\pgfqpoint{2.259648in}{2.268350in}}{\pgfqpoint{2.251748in}{2.271622in}}{\pgfqpoint{2.243511in}{2.271622in}}%
\pgfpathcurveto{\pgfqpoint{2.235275in}{2.271622in}}{\pgfqpoint{2.227375in}{2.268350in}}{\pgfqpoint{2.221551in}{2.262526in}}%
\pgfpathcurveto{\pgfqpoint{2.215727in}{2.256702in}}{\pgfqpoint{2.212455in}{2.248802in}}{\pgfqpoint{2.212455in}{2.240566in}}%
\pgfpathcurveto{\pgfqpoint{2.212455in}{2.232330in}}{\pgfqpoint{2.215727in}{2.224430in}}{\pgfqpoint{2.221551in}{2.218606in}}%
\pgfpathcurveto{\pgfqpoint{2.227375in}{2.212782in}}{\pgfqpoint{2.235275in}{2.209509in}}{\pgfqpoint{2.243511in}{2.209509in}}%
\pgfpathclose%
\pgfusepath{stroke,fill}%
\end{pgfscope}%
\begin{pgfscope}%
\pgfpathrectangle{\pgfqpoint{0.100000in}{0.212622in}}{\pgfqpoint{3.696000in}{3.696000in}}%
\pgfusepath{clip}%
\pgfsetbuttcap%
\pgfsetroundjoin%
\definecolor{currentfill}{rgb}{0.121569,0.466667,0.705882}%
\pgfsetfillcolor{currentfill}%
\pgfsetfillopacity{0.885935}%
\pgfsetlinewidth{1.003750pt}%
\definecolor{currentstroke}{rgb}{0.121569,0.466667,0.705882}%
\pgfsetstrokecolor{currentstroke}%
\pgfsetstrokeopacity{0.885935}%
\pgfsetdash{}{0pt}%
\pgfpathmoveto{\pgfqpoint{1.729265in}{2.032324in}}%
\pgfpathcurveto{\pgfqpoint{1.737501in}{2.032324in}}{\pgfqpoint{1.745401in}{2.035596in}}{\pgfqpoint{1.751225in}{2.041420in}}%
\pgfpathcurveto{\pgfqpoint{1.757049in}{2.047244in}}{\pgfqpoint{1.760322in}{2.055144in}}{\pgfqpoint{1.760322in}{2.063380in}}%
\pgfpathcurveto{\pgfqpoint{1.760322in}{2.071617in}}{\pgfqpoint{1.757049in}{2.079517in}}{\pgfqpoint{1.751225in}{2.085341in}}%
\pgfpathcurveto{\pgfqpoint{1.745401in}{2.091165in}}{\pgfqpoint{1.737501in}{2.094437in}}{\pgfqpoint{1.729265in}{2.094437in}}%
\pgfpathcurveto{\pgfqpoint{1.721029in}{2.094437in}}{\pgfqpoint{1.713129in}{2.091165in}}{\pgfqpoint{1.707305in}{2.085341in}}%
\pgfpathcurveto{\pgfqpoint{1.701481in}{2.079517in}}{\pgfqpoint{1.698209in}{2.071617in}}{\pgfqpoint{1.698209in}{2.063380in}}%
\pgfpathcurveto{\pgfqpoint{1.698209in}{2.055144in}}{\pgfqpoint{1.701481in}{2.047244in}}{\pgfqpoint{1.707305in}{2.041420in}}%
\pgfpathcurveto{\pgfqpoint{1.713129in}{2.035596in}}{\pgfqpoint{1.721029in}{2.032324in}}{\pgfqpoint{1.729265in}{2.032324in}}%
\pgfpathclose%
\pgfusepath{stroke,fill}%
\end{pgfscope}%
\begin{pgfscope}%
\pgfpathrectangle{\pgfqpoint{0.100000in}{0.212622in}}{\pgfqpoint{3.696000in}{3.696000in}}%
\pgfusepath{clip}%
\pgfsetbuttcap%
\pgfsetroundjoin%
\definecolor{currentfill}{rgb}{0.121569,0.466667,0.705882}%
\pgfsetfillcolor{currentfill}%
\pgfsetfillopacity{0.886340}%
\pgfsetlinewidth{1.003750pt}%
\definecolor{currentstroke}{rgb}{0.121569,0.466667,0.705882}%
\pgfsetstrokecolor{currentstroke}%
\pgfsetstrokeopacity{0.886340}%
\pgfsetdash{}{0pt}%
\pgfpathmoveto{\pgfqpoint{1.728534in}{2.030168in}}%
\pgfpathcurveto{\pgfqpoint{1.736771in}{2.030168in}}{\pgfqpoint{1.744671in}{2.033440in}}{\pgfqpoint{1.750495in}{2.039264in}}%
\pgfpathcurveto{\pgfqpoint{1.756319in}{2.045088in}}{\pgfqpoint{1.759591in}{2.052988in}}{\pgfqpoint{1.759591in}{2.061224in}}%
\pgfpathcurveto{\pgfqpoint{1.759591in}{2.069460in}}{\pgfqpoint{1.756319in}{2.077360in}}{\pgfqpoint{1.750495in}{2.083184in}}%
\pgfpathcurveto{\pgfqpoint{1.744671in}{2.089008in}}{\pgfqpoint{1.736771in}{2.092281in}}{\pgfqpoint{1.728534in}{2.092281in}}%
\pgfpathcurveto{\pgfqpoint{1.720298in}{2.092281in}}{\pgfqpoint{1.712398in}{2.089008in}}{\pgfqpoint{1.706574in}{2.083184in}}%
\pgfpathcurveto{\pgfqpoint{1.700750in}{2.077360in}}{\pgfqpoint{1.697478in}{2.069460in}}{\pgfqpoint{1.697478in}{2.061224in}}%
\pgfpathcurveto{\pgfqpoint{1.697478in}{2.052988in}}{\pgfqpoint{1.700750in}{2.045088in}}{\pgfqpoint{1.706574in}{2.039264in}}%
\pgfpathcurveto{\pgfqpoint{1.712398in}{2.033440in}}{\pgfqpoint{1.720298in}{2.030168in}}{\pgfqpoint{1.728534in}{2.030168in}}%
\pgfpathclose%
\pgfusepath{stroke,fill}%
\end{pgfscope}%
\begin{pgfscope}%
\pgfpathrectangle{\pgfqpoint{0.100000in}{0.212622in}}{\pgfqpoint{3.696000in}{3.696000in}}%
\pgfusepath{clip}%
\pgfsetbuttcap%
\pgfsetroundjoin%
\definecolor{currentfill}{rgb}{0.121569,0.466667,0.705882}%
\pgfsetfillcolor{currentfill}%
\pgfsetfillopacity{0.886369}%
\pgfsetlinewidth{1.003750pt}%
\definecolor{currentstroke}{rgb}{0.121569,0.466667,0.705882}%
\pgfsetstrokecolor{currentstroke}%
\pgfsetstrokeopacity{0.886369}%
\pgfsetdash{}{0pt}%
\pgfpathmoveto{\pgfqpoint{2.241942in}{2.206415in}}%
\pgfpathcurveto{\pgfqpoint{2.250178in}{2.206415in}}{\pgfqpoint{2.258078in}{2.209688in}}{\pgfqpoint{2.263902in}{2.215512in}}%
\pgfpathcurveto{\pgfqpoint{2.269726in}{2.221336in}}{\pgfqpoint{2.272999in}{2.229236in}}{\pgfqpoint{2.272999in}{2.237472in}}%
\pgfpathcurveto{\pgfqpoint{2.272999in}{2.245708in}}{\pgfqpoint{2.269726in}{2.253608in}}{\pgfqpoint{2.263902in}{2.259432in}}%
\pgfpathcurveto{\pgfqpoint{2.258078in}{2.265256in}}{\pgfqpoint{2.250178in}{2.268528in}}{\pgfqpoint{2.241942in}{2.268528in}}%
\pgfpathcurveto{\pgfqpoint{2.233706in}{2.268528in}}{\pgfqpoint{2.225806in}{2.265256in}}{\pgfqpoint{2.219982in}{2.259432in}}%
\pgfpathcurveto{\pgfqpoint{2.214158in}{2.253608in}}{\pgfqpoint{2.210886in}{2.245708in}}{\pgfqpoint{2.210886in}{2.237472in}}%
\pgfpathcurveto{\pgfqpoint{2.210886in}{2.229236in}}{\pgfqpoint{2.214158in}{2.221336in}}{\pgfqpoint{2.219982in}{2.215512in}}%
\pgfpathcurveto{\pgfqpoint{2.225806in}{2.209688in}}{\pgfqpoint{2.233706in}{2.206415in}}{\pgfqpoint{2.241942in}{2.206415in}}%
\pgfpathclose%
\pgfusepath{stroke,fill}%
\end{pgfscope}%
\begin{pgfscope}%
\pgfpathrectangle{\pgfqpoint{0.100000in}{0.212622in}}{\pgfqpoint{3.696000in}{3.696000in}}%
\pgfusepath{clip}%
\pgfsetbuttcap%
\pgfsetroundjoin%
\definecolor{currentfill}{rgb}{0.121569,0.466667,0.705882}%
\pgfsetfillcolor{currentfill}%
\pgfsetfillopacity{0.886540}%
\pgfsetlinewidth{1.003750pt}%
\definecolor{currentstroke}{rgb}{0.121569,0.466667,0.705882}%
\pgfsetstrokecolor{currentstroke}%
\pgfsetstrokeopacity{0.886540}%
\pgfsetdash{}{0pt}%
\pgfpathmoveto{\pgfqpoint{2.241089in}{2.205343in}}%
\pgfpathcurveto{\pgfqpoint{2.249326in}{2.205343in}}{\pgfqpoint{2.257226in}{2.208616in}}{\pgfqpoint{2.263050in}{2.214440in}}%
\pgfpathcurveto{\pgfqpoint{2.268874in}{2.220263in}}{\pgfqpoint{2.272146in}{2.228164in}}{\pgfqpoint{2.272146in}{2.236400in}}%
\pgfpathcurveto{\pgfqpoint{2.272146in}{2.244636in}}{\pgfqpoint{2.268874in}{2.252536in}}{\pgfqpoint{2.263050in}{2.258360in}}%
\pgfpathcurveto{\pgfqpoint{2.257226in}{2.264184in}}{\pgfqpoint{2.249326in}{2.267456in}}{\pgfqpoint{2.241089in}{2.267456in}}%
\pgfpathcurveto{\pgfqpoint{2.232853in}{2.267456in}}{\pgfqpoint{2.224953in}{2.264184in}}{\pgfqpoint{2.219129in}{2.258360in}}%
\pgfpathcurveto{\pgfqpoint{2.213305in}{2.252536in}}{\pgfqpoint{2.210033in}{2.244636in}}{\pgfqpoint{2.210033in}{2.236400in}}%
\pgfpathcurveto{\pgfqpoint{2.210033in}{2.228164in}}{\pgfqpoint{2.213305in}{2.220263in}}{\pgfqpoint{2.219129in}{2.214440in}}%
\pgfpathcurveto{\pgfqpoint{2.224953in}{2.208616in}}{\pgfqpoint{2.232853in}{2.205343in}}{\pgfqpoint{2.241089in}{2.205343in}}%
\pgfpathclose%
\pgfusepath{stroke,fill}%
\end{pgfscope}%
\begin{pgfscope}%
\pgfpathrectangle{\pgfqpoint{0.100000in}{0.212622in}}{\pgfqpoint{3.696000in}{3.696000in}}%
\pgfusepath{clip}%
\pgfsetbuttcap%
\pgfsetroundjoin%
\definecolor{currentfill}{rgb}{0.121569,0.466667,0.705882}%
\pgfsetfillcolor{currentfill}%
\pgfsetfillopacity{0.886676}%
\pgfsetlinewidth{1.003750pt}%
\definecolor{currentstroke}{rgb}{0.121569,0.466667,0.705882}%
\pgfsetstrokecolor{currentstroke}%
\pgfsetstrokeopacity{0.886676}%
\pgfsetdash{}{0pt}%
\pgfpathmoveto{\pgfqpoint{1.594548in}{1.194551in}}%
\pgfpathcurveto{\pgfqpoint{1.602784in}{1.194551in}}{\pgfqpoint{1.610684in}{1.197823in}}{\pgfqpoint{1.616508in}{1.203647in}}%
\pgfpathcurveto{\pgfqpoint{1.622332in}{1.209471in}}{\pgfqpoint{1.625604in}{1.217371in}}{\pgfqpoint{1.625604in}{1.225607in}}%
\pgfpathcurveto{\pgfqpoint{1.625604in}{1.233843in}}{\pgfqpoint{1.622332in}{1.241743in}}{\pgfqpoint{1.616508in}{1.247567in}}%
\pgfpathcurveto{\pgfqpoint{1.610684in}{1.253391in}}{\pgfqpoint{1.602784in}{1.256664in}}{\pgfqpoint{1.594548in}{1.256664in}}%
\pgfpathcurveto{\pgfqpoint{1.586311in}{1.256664in}}{\pgfqpoint{1.578411in}{1.253391in}}{\pgfqpoint{1.572587in}{1.247567in}}%
\pgfpathcurveto{\pgfqpoint{1.566763in}{1.241743in}}{\pgfqpoint{1.563491in}{1.233843in}}{\pgfqpoint{1.563491in}{1.225607in}}%
\pgfpathcurveto{\pgfqpoint{1.563491in}{1.217371in}}{\pgfqpoint{1.566763in}{1.209471in}}{\pgfqpoint{1.572587in}{1.203647in}}%
\pgfpathcurveto{\pgfqpoint{1.578411in}{1.197823in}}{\pgfqpoint{1.586311in}{1.194551in}}{\pgfqpoint{1.594548in}{1.194551in}}%
\pgfpathclose%
\pgfusepath{stroke,fill}%
\end{pgfscope}%
\begin{pgfscope}%
\pgfpathrectangle{\pgfqpoint{0.100000in}{0.212622in}}{\pgfqpoint{3.696000in}{3.696000in}}%
\pgfusepath{clip}%
\pgfsetbuttcap%
\pgfsetroundjoin%
\definecolor{currentfill}{rgb}{0.121569,0.466667,0.705882}%
\pgfsetfillcolor{currentfill}%
\pgfsetfillopacity{0.886914}%
\pgfsetlinewidth{1.003750pt}%
\definecolor{currentstroke}{rgb}{0.121569,0.466667,0.705882}%
\pgfsetstrokecolor{currentstroke}%
\pgfsetstrokeopacity{0.886914}%
\pgfsetdash{}{0pt}%
\pgfpathmoveto{\pgfqpoint{2.239841in}{2.203071in}}%
\pgfpathcurveto{\pgfqpoint{2.248078in}{2.203071in}}{\pgfqpoint{2.255978in}{2.206343in}}{\pgfqpoint{2.261802in}{2.212167in}}%
\pgfpathcurveto{\pgfqpoint{2.267626in}{2.217991in}}{\pgfqpoint{2.270898in}{2.225891in}}{\pgfqpoint{2.270898in}{2.234128in}}%
\pgfpathcurveto{\pgfqpoint{2.270898in}{2.242364in}}{\pgfqpoint{2.267626in}{2.250264in}}{\pgfqpoint{2.261802in}{2.256088in}}%
\pgfpathcurveto{\pgfqpoint{2.255978in}{2.261912in}}{\pgfqpoint{2.248078in}{2.265184in}}{\pgfqpoint{2.239841in}{2.265184in}}%
\pgfpathcurveto{\pgfqpoint{2.231605in}{2.265184in}}{\pgfqpoint{2.223705in}{2.261912in}}{\pgfqpoint{2.217881in}{2.256088in}}%
\pgfpathcurveto{\pgfqpoint{2.212057in}{2.250264in}}{\pgfqpoint{2.208785in}{2.242364in}}{\pgfqpoint{2.208785in}{2.234128in}}%
\pgfpathcurveto{\pgfqpoint{2.208785in}{2.225891in}}{\pgfqpoint{2.212057in}{2.217991in}}{\pgfqpoint{2.217881in}{2.212167in}}%
\pgfpathcurveto{\pgfqpoint{2.223705in}{2.206343in}}{\pgfqpoint{2.231605in}{2.203071in}}{\pgfqpoint{2.239841in}{2.203071in}}%
\pgfpathclose%
\pgfusepath{stroke,fill}%
\end{pgfscope}%
\begin{pgfscope}%
\pgfpathrectangle{\pgfqpoint{0.100000in}{0.212622in}}{\pgfqpoint{3.696000in}{3.696000in}}%
\pgfusepath{clip}%
\pgfsetbuttcap%
\pgfsetroundjoin%
\definecolor{currentfill}{rgb}{0.121569,0.466667,0.705882}%
\pgfsetfillcolor{currentfill}%
\pgfsetfillopacity{0.886939}%
\pgfsetlinewidth{1.003750pt}%
\definecolor{currentstroke}{rgb}{0.121569,0.466667,0.705882}%
\pgfsetstrokecolor{currentstroke}%
\pgfsetstrokeopacity{0.886939}%
\pgfsetdash{}{0pt}%
\pgfpathmoveto{\pgfqpoint{1.727651in}{2.027517in}}%
\pgfpathcurveto{\pgfqpoint{1.735887in}{2.027517in}}{\pgfqpoint{1.743787in}{2.030789in}}{\pgfqpoint{1.749611in}{2.036613in}}%
\pgfpathcurveto{\pgfqpoint{1.755435in}{2.042437in}}{\pgfqpoint{1.758707in}{2.050337in}}{\pgfqpoint{1.758707in}{2.058573in}}%
\pgfpathcurveto{\pgfqpoint{1.758707in}{2.066809in}}{\pgfqpoint{1.755435in}{2.074710in}}{\pgfqpoint{1.749611in}{2.080533in}}%
\pgfpathcurveto{\pgfqpoint{1.743787in}{2.086357in}}{\pgfqpoint{1.735887in}{2.089630in}}{\pgfqpoint{1.727651in}{2.089630in}}%
\pgfpathcurveto{\pgfqpoint{1.719415in}{2.089630in}}{\pgfqpoint{1.711514in}{2.086357in}}{\pgfqpoint{1.705691in}{2.080533in}}%
\pgfpathcurveto{\pgfqpoint{1.699867in}{2.074710in}}{\pgfqpoint{1.696594in}{2.066809in}}{\pgfqpoint{1.696594in}{2.058573in}}%
\pgfpathcurveto{\pgfqpoint{1.696594in}{2.050337in}}{\pgfqpoint{1.699867in}{2.042437in}}{\pgfqpoint{1.705691in}{2.036613in}}%
\pgfpathcurveto{\pgfqpoint{1.711514in}{2.030789in}}{\pgfqpoint{1.719415in}{2.027517in}}{\pgfqpoint{1.727651in}{2.027517in}}%
\pgfpathclose%
\pgfusepath{stroke,fill}%
\end{pgfscope}%
\begin{pgfscope}%
\pgfpathrectangle{\pgfqpoint{0.100000in}{0.212622in}}{\pgfqpoint{3.696000in}{3.696000in}}%
\pgfusepath{clip}%
\pgfsetbuttcap%
\pgfsetroundjoin%
\definecolor{currentfill}{rgb}{0.121569,0.466667,0.705882}%
\pgfsetfillcolor{currentfill}%
\pgfsetfillopacity{0.887101}%
\pgfsetlinewidth{1.003750pt}%
\definecolor{currentstroke}{rgb}{0.121569,0.466667,0.705882}%
\pgfsetstrokecolor{currentstroke}%
\pgfsetstrokeopacity{0.887101}%
\pgfsetdash{}{0pt}%
\pgfpathmoveto{\pgfqpoint{2.239639in}{2.201827in}}%
\pgfpathcurveto{\pgfqpoint{2.247875in}{2.201827in}}{\pgfqpoint{2.255775in}{2.205099in}}{\pgfqpoint{2.261599in}{2.210923in}}%
\pgfpathcurveto{\pgfqpoint{2.267423in}{2.216747in}}{\pgfqpoint{2.270695in}{2.224647in}}{\pgfqpoint{2.270695in}{2.232883in}}%
\pgfpathcurveto{\pgfqpoint{2.270695in}{2.241120in}}{\pgfqpoint{2.267423in}{2.249020in}}{\pgfqpoint{2.261599in}{2.254844in}}%
\pgfpathcurveto{\pgfqpoint{2.255775in}{2.260668in}}{\pgfqpoint{2.247875in}{2.263940in}}{\pgfqpoint{2.239639in}{2.263940in}}%
\pgfpathcurveto{\pgfqpoint{2.231402in}{2.263940in}}{\pgfqpoint{2.223502in}{2.260668in}}{\pgfqpoint{2.217678in}{2.254844in}}%
\pgfpathcurveto{\pgfqpoint{2.211854in}{2.249020in}}{\pgfqpoint{2.208582in}{2.241120in}}{\pgfqpoint{2.208582in}{2.232883in}}%
\pgfpathcurveto{\pgfqpoint{2.208582in}{2.224647in}}{\pgfqpoint{2.211854in}{2.216747in}}{\pgfqpoint{2.217678in}{2.210923in}}%
\pgfpathcurveto{\pgfqpoint{2.223502in}{2.205099in}}{\pgfqpoint{2.231402in}{2.201827in}}{\pgfqpoint{2.239639in}{2.201827in}}%
\pgfpathclose%
\pgfusepath{stroke,fill}%
\end{pgfscope}%
\begin{pgfscope}%
\pgfpathrectangle{\pgfqpoint{0.100000in}{0.212622in}}{\pgfqpoint{3.696000in}{3.696000in}}%
\pgfusepath{clip}%
\pgfsetbuttcap%
\pgfsetroundjoin%
\definecolor{currentfill}{rgb}{0.121569,0.466667,0.705882}%
\pgfsetfillcolor{currentfill}%
\pgfsetfillopacity{0.887257}%
\pgfsetlinewidth{1.003750pt}%
\definecolor{currentstroke}{rgb}{0.121569,0.466667,0.705882}%
\pgfsetstrokecolor{currentstroke}%
\pgfsetstrokeopacity{0.887257}%
\pgfsetdash{}{0pt}%
\pgfpathmoveto{\pgfqpoint{1.727265in}{2.025993in}}%
\pgfpathcurveto{\pgfqpoint{1.735501in}{2.025993in}}{\pgfqpoint{1.743401in}{2.029265in}}{\pgfqpoint{1.749225in}{2.035089in}}%
\pgfpathcurveto{\pgfqpoint{1.755049in}{2.040913in}}{\pgfqpoint{1.758321in}{2.048813in}}{\pgfqpoint{1.758321in}{2.057049in}}%
\pgfpathcurveto{\pgfqpoint{1.758321in}{2.065285in}}{\pgfqpoint{1.755049in}{2.073185in}}{\pgfqpoint{1.749225in}{2.079009in}}%
\pgfpathcurveto{\pgfqpoint{1.743401in}{2.084833in}}{\pgfqpoint{1.735501in}{2.088106in}}{\pgfqpoint{1.727265in}{2.088106in}}%
\pgfpathcurveto{\pgfqpoint{1.719028in}{2.088106in}}{\pgfqpoint{1.711128in}{2.084833in}}{\pgfqpoint{1.705304in}{2.079009in}}%
\pgfpathcurveto{\pgfqpoint{1.699480in}{2.073185in}}{\pgfqpoint{1.696208in}{2.065285in}}{\pgfqpoint{1.696208in}{2.057049in}}%
\pgfpathcurveto{\pgfqpoint{1.696208in}{2.048813in}}{\pgfqpoint{1.699480in}{2.040913in}}{\pgfqpoint{1.705304in}{2.035089in}}%
\pgfpathcurveto{\pgfqpoint{1.711128in}{2.029265in}}{\pgfqpoint{1.719028in}{2.025993in}}{\pgfqpoint{1.727265in}{2.025993in}}%
\pgfpathclose%
\pgfusepath{stroke,fill}%
\end{pgfscope}%
\begin{pgfscope}%
\pgfpathrectangle{\pgfqpoint{0.100000in}{0.212622in}}{\pgfqpoint{3.696000in}{3.696000in}}%
\pgfusepath{clip}%
\pgfsetbuttcap%
\pgfsetroundjoin%
\definecolor{currentfill}{rgb}{0.121569,0.466667,0.705882}%
\pgfsetfillcolor{currentfill}%
\pgfsetfillopacity{0.887471}%
\pgfsetlinewidth{1.003750pt}%
\definecolor{currentstroke}{rgb}{0.121569,0.466667,0.705882}%
\pgfsetstrokecolor{currentstroke}%
\pgfsetstrokeopacity{0.887471}%
\pgfsetdash{}{0pt}%
\pgfpathmoveto{\pgfqpoint{2.238761in}{2.199866in}}%
\pgfpathcurveto{\pgfqpoint{2.246997in}{2.199866in}}{\pgfqpoint{2.254897in}{2.203139in}}{\pgfqpoint{2.260721in}{2.208963in}}%
\pgfpathcurveto{\pgfqpoint{2.266545in}{2.214787in}}{\pgfqpoint{2.269817in}{2.222687in}}{\pgfqpoint{2.269817in}{2.230923in}}%
\pgfpathcurveto{\pgfqpoint{2.269817in}{2.239159in}}{\pgfqpoint{2.266545in}{2.247059in}}{\pgfqpoint{2.260721in}{2.252883in}}%
\pgfpathcurveto{\pgfqpoint{2.254897in}{2.258707in}}{\pgfqpoint{2.246997in}{2.261979in}}{\pgfqpoint{2.238761in}{2.261979in}}%
\pgfpathcurveto{\pgfqpoint{2.230524in}{2.261979in}}{\pgfqpoint{2.222624in}{2.258707in}}{\pgfqpoint{2.216800in}{2.252883in}}%
\pgfpathcurveto{\pgfqpoint{2.210976in}{2.247059in}}{\pgfqpoint{2.207704in}{2.239159in}}{\pgfqpoint{2.207704in}{2.230923in}}%
\pgfpathcurveto{\pgfqpoint{2.207704in}{2.222687in}}{\pgfqpoint{2.210976in}{2.214787in}}{\pgfqpoint{2.216800in}{2.208963in}}%
\pgfpathcurveto{\pgfqpoint{2.222624in}{2.203139in}}{\pgfqpoint{2.230524in}{2.199866in}}{\pgfqpoint{2.238761in}{2.199866in}}%
\pgfpathclose%
\pgfusepath{stroke,fill}%
\end{pgfscope}%
\begin{pgfscope}%
\pgfpathrectangle{\pgfqpoint{0.100000in}{0.212622in}}{\pgfqpoint{3.696000in}{3.696000in}}%
\pgfusepath{clip}%
\pgfsetbuttcap%
\pgfsetroundjoin%
\definecolor{currentfill}{rgb}{0.121569,0.466667,0.705882}%
\pgfsetfillcolor{currentfill}%
\pgfsetfillopacity{0.887525}%
\pgfsetlinewidth{1.003750pt}%
\definecolor{currentstroke}{rgb}{0.121569,0.466667,0.705882}%
\pgfsetstrokecolor{currentstroke}%
\pgfsetstrokeopacity{0.887525}%
\pgfsetdash{}{0pt}%
\pgfpathmoveto{\pgfqpoint{2.238526in}{2.199547in}}%
\pgfpathcurveto{\pgfqpoint{2.246762in}{2.199547in}}{\pgfqpoint{2.254662in}{2.202819in}}{\pgfqpoint{2.260486in}{2.208643in}}%
\pgfpathcurveto{\pgfqpoint{2.266310in}{2.214467in}}{\pgfqpoint{2.269583in}{2.222367in}}{\pgfqpoint{2.269583in}{2.230603in}}%
\pgfpathcurveto{\pgfqpoint{2.269583in}{2.238840in}}{\pgfqpoint{2.266310in}{2.246740in}}{\pgfqpoint{2.260486in}{2.252564in}}%
\pgfpathcurveto{\pgfqpoint{2.254662in}{2.258387in}}{\pgfqpoint{2.246762in}{2.261660in}}{\pgfqpoint{2.238526in}{2.261660in}}%
\pgfpathcurveto{\pgfqpoint{2.230290in}{2.261660in}}{\pgfqpoint{2.222390in}{2.258387in}}{\pgfqpoint{2.216566in}{2.252564in}}%
\pgfpathcurveto{\pgfqpoint{2.210742in}{2.246740in}}{\pgfqpoint{2.207470in}{2.238840in}}{\pgfqpoint{2.207470in}{2.230603in}}%
\pgfpathcurveto{\pgfqpoint{2.207470in}{2.222367in}}{\pgfqpoint{2.210742in}{2.214467in}}{\pgfqpoint{2.216566in}{2.208643in}}%
\pgfpathcurveto{\pgfqpoint{2.222390in}{2.202819in}}{\pgfqpoint{2.230290in}{2.199547in}}{\pgfqpoint{2.238526in}{2.199547in}}%
\pgfpathclose%
\pgfusepath{stroke,fill}%
\end{pgfscope}%
\begin{pgfscope}%
\pgfpathrectangle{\pgfqpoint{0.100000in}{0.212622in}}{\pgfqpoint{3.696000in}{3.696000in}}%
\pgfusepath{clip}%
\pgfsetbuttcap%
\pgfsetroundjoin%
\definecolor{currentfill}{rgb}{0.121569,0.466667,0.705882}%
\pgfsetfillcolor{currentfill}%
\pgfsetfillopacity{0.887633}%
\pgfsetlinewidth{1.003750pt}%
\definecolor{currentstroke}{rgb}{0.121569,0.466667,0.705882}%
\pgfsetstrokecolor{currentstroke}%
\pgfsetstrokeopacity{0.887633}%
\pgfsetdash{}{0pt}%
\pgfpathmoveto{\pgfqpoint{2.238120in}{2.198972in}}%
\pgfpathcurveto{\pgfqpoint{2.246357in}{2.198972in}}{\pgfqpoint{2.254257in}{2.202244in}}{\pgfqpoint{2.260081in}{2.208068in}}%
\pgfpathcurveto{\pgfqpoint{2.265905in}{2.213892in}}{\pgfqpoint{2.269177in}{2.221792in}}{\pgfqpoint{2.269177in}{2.230028in}}%
\pgfpathcurveto{\pgfqpoint{2.269177in}{2.238264in}}{\pgfqpoint{2.265905in}{2.246164in}}{\pgfqpoint{2.260081in}{2.251988in}}%
\pgfpathcurveto{\pgfqpoint{2.254257in}{2.257812in}}{\pgfqpoint{2.246357in}{2.261085in}}{\pgfqpoint{2.238120in}{2.261085in}}%
\pgfpathcurveto{\pgfqpoint{2.229884in}{2.261085in}}{\pgfqpoint{2.221984in}{2.257812in}}{\pgfqpoint{2.216160in}{2.251988in}}%
\pgfpathcurveto{\pgfqpoint{2.210336in}{2.246164in}}{\pgfqpoint{2.207064in}{2.238264in}}{\pgfqpoint{2.207064in}{2.230028in}}%
\pgfpathcurveto{\pgfqpoint{2.207064in}{2.221792in}}{\pgfqpoint{2.210336in}{2.213892in}}{\pgfqpoint{2.216160in}{2.208068in}}%
\pgfpathcurveto{\pgfqpoint{2.221984in}{2.202244in}}{\pgfqpoint{2.229884in}{2.198972in}}{\pgfqpoint{2.238120in}{2.198972in}}%
\pgfpathclose%
\pgfusepath{stroke,fill}%
\end{pgfscope}%
\begin{pgfscope}%
\pgfpathrectangle{\pgfqpoint{0.100000in}{0.212622in}}{\pgfqpoint{3.696000in}{3.696000in}}%
\pgfusepath{clip}%
\pgfsetbuttcap%
\pgfsetroundjoin%
\definecolor{currentfill}{rgb}{0.121569,0.466667,0.705882}%
\pgfsetfillcolor{currentfill}%
\pgfsetfillopacity{0.887673}%
\pgfsetlinewidth{1.003750pt}%
\definecolor{currentstroke}{rgb}{0.121569,0.466667,0.705882}%
\pgfsetstrokecolor{currentstroke}%
\pgfsetstrokeopacity{0.887673}%
\pgfsetdash{}{0pt}%
\pgfpathmoveto{\pgfqpoint{1.726808in}{2.023902in}}%
\pgfpathcurveto{\pgfqpoint{1.735045in}{2.023902in}}{\pgfqpoint{1.742945in}{2.027174in}}{\pgfqpoint{1.748769in}{2.032998in}}%
\pgfpathcurveto{\pgfqpoint{1.754593in}{2.038822in}}{\pgfqpoint{1.757865in}{2.046722in}}{\pgfqpoint{1.757865in}{2.054958in}}%
\pgfpathcurveto{\pgfqpoint{1.757865in}{2.063195in}}{\pgfqpoint{1.754593in}{2.071095in}}{\pgfqpoint{1.748769in}{2.076919in}}%
\pgfpathcurveto{\pgfqpoint{1.742945in}{2.082743in}}{\pgfqpoint{1.735045in}{2.086015in}}{\pgfqpoint{1.726808in}{2.086015in}}%
\pgfpathcurveto{\pgfqpoint{1.718572in}{2.086015in}}{\pgfqpoint{1.710672in}{2.082743in}}{\pgfqpoint{1.704848in}{2.076919in}}%
\pgfpathcurveto{\pgfqpoint{1.699024in}{2.071095in}}{\pgfqpoint{1.695752in}{2.063195in}}{\pgfqpoint{1.695752in}{2.054958in}}%
\pgfpathcurveto{\pgfqpoint{1.695752in}{2.046722in}}{\pgfqpoint{1.699024in}{2.038822in}}{\pgfqpoint{1.704848in}{2.032998in}}%
\pgfpathcurveto{\pgfqpoint{1.710672in}{2.027174in}}{\pgfqpoint{1.718572in}{2.023902in}}{\pgfqpoint{1.726808in}{2.023902in}}%
\pgfpathclose%
\pgfusepath{stroke,fill}%
\end{pgfscope}%
\begin{pgfscope}%
\pgfpathrectangle{\pgfqpoint{0.100000in}{0.212622in}}{\pgfqpoint{3.696000in}{3.696000in}}%
\pgfusepath{clip}%
\pgfsetbuttcap%
\pgfsetroundjoin%
\definecolor{currentfill}{rgb}{0.121569,0.466667,0.705882}%
\pgfsetfillcolor{currentfill}%
\pgfsetfillopacity{0.887869}%
\pgfsetlinewidth{1.003750pt}%
\definecolor{currentstroke}{rgb}{0.121569,0.466667,0.705882}%
\pgfsetstrokecolor{currentstroke}%
\pgfsetstrokeopacity{0.887869}%
\pgfsetdash{}{0pt}%
\pgfpathmoveto{\pgfqpoint{2.237743in}{2.197733in}}%
\pgfpathcurveto{\pgfqpoint{2.245980in}{2.197733in}}{\pgfqpoint{2.253880in}{2.201005in}}{\pgfqpoint{2.259704in}{2.206829in}}%
\pgfpathcurveto{\pgfqpoint{2.265527in}{2.212653in}}{\pgfqpoint{2.268800in}{2.220553in}}{\pgfqpoint{2.268800in}{2.228790in}}%
\pgfpathcurveto{\pgfqpoint{2.268800in}{2.237026in}}{\pgfqpoint{2.265527in}{2.244926in}}{\pgfqpoint{2.259704in}{2.250750in}}%
\pgfpathcurveto{\pgfqpoint{2.253880in}{2.256574in}}{\pgfqpoint{2.245980in}{2.259846in}}{\pgfqpoint{2.237743in}{2.259846in}}%
\pgfpathcurveto{\pgfqpoint{2.229507in}{2.259846in}}{\pgfqpoint{2.221607in}{2.256574in}}{\pgfqpoint{2.215783in}{2.250750in}}%
\pgfpathcurveto{\pgfqpoint{2.209959in}{2.244926in}}{\pgfqpoint{2.206687in}{2.237026in}}{\pgfqpoint{2.206687in}{2.228790in}}%
\pgfpathcurveto{\pgfqpoint{2.206687in}{2.220553in}}{\pgfqpoint{2.209959in}{2.212653in}}{\pgfqpoint{2.215783in}{2.206829in}}%
\pgfpathcurveto{\pgfqpoint{2.221607in}{2.201005in}}{\pgfqpoint{2.229507in}{2.197733in}}{\pgfqpoint{2.237743in}{2.197733in}}%
\pgfpathclose%
\pgfusepath{stroke,fill}%
\end{pgfscope}%
\begin{pgfscope}%
\pgfpathrectangle{\pgfqpoint{0.100000in}{0.212622in}}{\pgfqpoint{3.696000in}{3.696000in}}%
\pgfusepath{clip}%
\pgfsetbuttcap%
\pgfsetroundjoin%
\definecolor{currentfill}{rgb}{0.121569,0.466667,0.705882}%
\pgfsetfillcolor{currentfill}%
\pgfsetfillopacity{0.888015}%
\pgfsetlinewidth{1.003750pt}%
\definecolor{currentstroke}{rgb}{0.121569,0.466667,0.705882}%
\pgfsetstrokecolor{currentstroke}%
\pgfsetstrokeopacity{0.888015}%
\pgfsetdash{}{0pt}%
\pgfpathmoveto{\pgfqpoint{1.607183in}{1.186472in}}%
\pgfpathcurveto{\pgfqpoint{1.615420in}{1.186472in}}{\pgfqpoint{1.623320in}{1.189745in}}{\pgfqpoint{1.629144in}{1.195569in}}%
\pgfpathcurveto{\pgfqpoint{1.634968in}{1.201393in}}{\pgfqpoint{1.638240in}{1.209293in}}{\pgfqpoint{1.638240in}{1.217529in}}%
\pgfpathcurveto{\pgfqpoint{1.638240in}{1.225765in}}{\pgfqpoint{1.634968in}{1.233665in}}{\pgfqpoint{1.629144in}{1.239489in}}%
\pgfpathcurveto{\pgfqpoint{1.623320in}{1.245313in}}{\pgfqpoint{1.615420in}{1.248585in}}{\pgfqpoint{1.607183in}{1.248585in}}%
\pgfpathcurveto{\pgfqpoint{1.598947in}{1.248585in}}{\pgfqpoint{1.591047in}{1.245313in}}{\pgfqpoint{1.585223in}{1.239489in}}%
\pgfpathcurveto{\pgfqpoint{1.579399in}{1.233665in}}{\pgfqpoint{1.576127in}{1.225765in}}{\pgfqpoint{1.576127in}{1.217529in}}%
\pgfpathcurveto{\pgfqpoint{1.576127in}{1.209293in}}{\pgfqpoint{1.579399in}{1.201393in}}{\pgfqpoint{1.585223in}{1.195569in}}%
\pgfpathcurveto{\pgfqpoint{1.591047in}{1.189745in}}{\pgfqpoint{1.598947in}{1.186472in}}{\pgfqpoint{1.607183in}{1.186472in}}%
\pgfpathclose%
\pgfusepath{stroke,fill}%
\end{pgfscope}%
\begin{pgfscope}%
\pgfpathrectangle{\pgfqpoint{0.100000in}{0.212622in}}{\pgfqpoint{3.696000in}{3.696000in}}%
\pgfusepath{clip}%
\pgfsetbuttcap%
\pgfsetroundjoin%
\definecolor{currentfill}{rgb}{0.121569,0.466667,0.705882}%
\pgfsetfillcolor{currentfill}%
\pgfsetfillopacity{0.888238}%
\pgfsetlinewidth{1.003750pt}%
\definecolor{currentstroke}{rgb}{0.121569,0.466667,0.705882}%
\pgfsetstrokecolor{currentstroke}%
\pgfsetstrokeopacity{0.888238}%
\pgfsetdash{}{0pt}%
\pgfpathmoveto{\pgfqpoint{1.726278in}{2.021059in}}%
\pgfpathcurveto{\pgfqpoint{1.734514in}{2.021059in}}{\pgfqpoint{1.742414in}{2.024332in}}{\pgfqpoint{1.748238in}{2.030156in}}%
\pgfpathcurveto{\pgfqpoint{1.754062in}{2.035979in}}{\pgfqpoint{1.757334in}{2.043880in}}{\pgfqpoint{1.757334in}{2.052116in}}%
\pgfpathcurveto{\pgfqpoint{1.757334in}{2.060352in}}{\pgfqpoint{1.754062in}{2.068252in}}{\pgfqpoint{1.748238in}{2.074076in}}%
\pgfpathcurveto{\pgfqpoint{1.742414in}{2.079900in}}{\pgfqpoint{1.734514in}{2.083172in}}{\pgfqpoint{1.726278in}{2.083172in}}%
\pgfpathcurveto{\pgfqpoint{1.718041in}{2.083172in}}{\pgfqpoint{1.710141in}{2.079900in}}{\pgfqpoint{1.704317in}{2.074076in}}%
\pgfpathcurveto{\pgfqpoint{1.698493in}{2.068252in}}{\pgfqpoint{1.695221in}{2.060352in}}{\pgfqpoint{1.695221in}{2.052116in}}%
\pgfpathcurveto{\pgfqpoint{1.695221in}{2.043880in}}{\pgfqpoint{1.698493in}{2.035979in}}{\pgfqpoint{1.704317in}{2.030156in}}%
\pgfpathcurveto{\pgfqpoint{1.710141in}{2.024332in}}{\pgfqpoint{1.718041in}{2.021059in}}{\pgfqpoint{1.726278in}{2.021059in}}%
\pgfpathclose%
\pgfusepath{stroke,fill}%
\end{pgfscope}%
\begin{pgfscope}%
\pgfpathrectangle{\pgfqpoint{0.100000in}{0.212622in}}{\pgfqpoint{3.696000in}{3.696000in}}%
\pgfusepath{clip}%
\pgfsetbuttcap%
\pgfsetroundjoin%
\definecolor{currentfill}{rgb}{0.121569,0.466667,0.705882}%
\pgfsetfillcolor{currentfill}%
\pgfsetfillopacity{0.888259}%
\pgfsetlinewidth{1.003750pt}%
\definecolor{currentstroke}{rgb}{0.121569,0.466667,0.705882}%
\pgfsetstrokecolor{currentstroke}%
\pgfsetstrokeopacity{0.888259}%
\pgfsetdash{}{0pt}%
\pgfpathmoveto{\pgfqpoint{2.237060in}{2.195313in}}%
\pgfpathcurveto{\pgfqpoint{2.245296in}{2.195313in}}{\pgfqpoint{2.253196in}{2.198585in}}{\pgfqpoint{2.259020in}{2.204409in}}%
\pgfpathcurveto{\pgfqpoint{2.264844in}{2.210233in}}{\pgfqpoint{2.268116in}{2.218133in}}{\pgfqpoint{2.268116in}{2.226369in}}%
\pgfpathcurveto{\pgfqpoint{2.268116in}{2.234606in}}{\pgfqpoint{2.264844in}{2.242506in}}{\pgfqpoint{2.259020in}{2.248330in}}%
\pgfpathcurveto{\pgfqpoint{2.253196in}{2.254154in}}{\pgfqpoint{2.245296in}{2.257426in}}{\pgfqpoint{2.237060in}{2.257426in}}%
\pgfpathcurveto{\pgfqpoint{2.228823in}{2.257426in}}{\pgfqpoint{2.220923in}{2.254154in}}{\pgfqpoint{2.215099in}{2.248330in}}%
\pgfpathcurveto{\pgfqpoint{2.209275in}{2.242506in}}{\pgfqpoint{2.206003in}{2.234606in}}{\pgfqpoint{2.206003in}{2.226369in}}%
\pgfpathcurveto{\pgfqpoint{2.206003in}{2.218133in}}{\pgfqpoint{2.209275in}{2.210233in}}{\pgfqpoint{2.215099in}{2.204409in}}%
\pgfpathcurveto{\pgfqpoint{2.220923in}{2.198585in}}{\pgfqpoint{2.228823in}{2.195313in}}{\pgfqpoint{2.237060in}{2.195313in}}%
\pgfpathclose%
\pgfusepath{stroke,fill}%
\end{pgfscope}%
\begin{pgfscope}%
\pgfpathrectangle{\pgfqpoint{0.100000in}{0.212622in}}{\pgfqpoint{3.696000in}{3.696000in}}%
\pgfusepath{clip}%
\pgfsetbuttcap%
\pgfsetroundjoin%
\definecolor{currentfill}{rgb}{0.121569,0.466667,0.705882}%
\pgfsetfillcolor{currentfill}%
\pgfsetfillopacity{0.888488}%
\pgfsetlinewidth{1.003750pt}%
\definecolor{currentstroke}{rgb}{0.121569,0.466667,0.705882}%
\pgfsetstrokecolor{currentstroke}%
\pgfsetstrokeopacity{0.888488}%
\pgfsetdash{}{0pt}%
\pgfpathmoveto{\pgfqpoint{2.236267in}{2.194265in}}%
\pgfpathcurveto{\pgfqpoint{2.244504in}{2.194265in}}{\pgfqpoint{2.252404in}{2.197538in}}{\pgfqpoint{2.258228in}{2.203362in}}%
\pgfpathcurveto{\pgfqpoint{2.264052in}{2.209186in}}{\pgfqpoint{2.267324in}{2.217086in}}{\pgfqpoint{2.267324in}{2.225322in}}%
\pgfpathcurveto{\pgfqpoint{2.267324in}{2.233558in}}{\pgfqpoint{2.264052in}{2.241458in}}{\pgfqpoint{2.258228in}{2.247282in}}%
\pgfpathcurveto{\pgfqpoint{2.252404in}{2.253106in}}{\pgfqpoint{2.244504in}{2.256378in}}{\pgfqpoint{2.236267in}{2.256378in}}%
\pgfpathcurveto{\pgfqpoint{2.228031in}{2.256378in}}{\pgfqpoint{2.220131in}{2.253106in}}{\pgfqpoint{2.214307in}{2.247282in}}%
\pgfpathcurveto{\pgfqpoint{2.208483in}{2.241458in}}{\pgfqpoint{2.205211in}{2.233558in}}{\pgfqpoint{2.205211in}{2.225322in}}%
\pgfpathcurveto{\pgfqpoint{2.205211in}{2.217086in}}{\pgfqpoint{2.208483in}{2.209186in}}{\pgfqpoint{2.214307in}{2.203362in}}%
\pgfpathcurveto{\pgfqpoint{2.220131in}{2.197538in}}{\pgfqpoint{2.228031in}{2.194265in}}{\pgfqpoint{2.236267in}{2.194265in}}%
\pgfpathclose%
\pgfusepath{stroke,fill}%
\end{pgfscope}%
\begin{pgfscope}%
\pgfpathrectangle{\pgfqpoint{0.100000in}{0.212622in}}{\pgfqpoint{3.696000in}{3.696000in}}%
\pgfusepath{clip}%
\pgfsetbuttcap%
\pgfsetroundjoin%
\definecolor{currentfill}{rgb}{0.121569,0.466667,0.705882}%
\pgfsetfillcolor{currentfill}%
\pgfsetfillopacity{0.888876}%
\pgfsetlinewidth{1.003750pt}%
\definecolor{currentstroke}{rgb}{0.121569,0.466667,0.705882}%
\pgfsetstrokecolor{currentstroke}%
\pgfsetstrokeopacity{0.888876}%
\pgfsetdash{}{0pt}%
\pgfpathmoveto{\pgfqpoint{2.234804in}{2.192275in}}%
\pgfpathcurveto{\pgfqpoint{2.243040in}{2.192275in}}{\pgfqpoint{2.250940in}{2.195547in}}{\pgfqpoint{2.256764in}{2.201371in}}%
\pgfpathcurveto{\pgfqpoint{2.262588in}{2.207195in}}{\pgfqpoint{2.265860in}{2.215095in}}{\pgfqpoint{2.265860in}{2.223331in}}%
\pgfpathcurveto{\pgfqpoint{2.265860in}{2.231567in}}{\pgfqpoint{2.262588in}{2.239467in}}{\pgfqpoint{2.256764in}{2.245291in}}%
\pgfpathcurveto{\pgfqpoint{2.250940in}{2.251115in}}{\pgfqpoint{2.243040in}{2.254388in}}{\pgfqpoint{2.234804in}{2.254388in}}%
\pgfpathcurveto{\pgfqpoint{2.226567in}{2.254388in}}{\pgfqpoint{2.218667in}{2.251115in}}{\pgfqpoint{2.212843in}{2.245291in}}%
\pgfpathcurveto{\pgfqpoint{2.207020in}{2.239467in}}{\pgfqpoint{2.203747in}{2.231567in}}{\pgfqpoint{2.203747in}{2.223331in}}%
\pgfpathcurveto{\pgfqpoint{2.203747in}{2.215095in}}{\pgfqpoint{2.207020in}{2.207195in}}{\pgfqpoint{2.212843in}{2.201371in}}%
\pgfpathcurveto{\pgfqpoint{2.218667in}{2.195547in}}{\pgfqpoint{2.226567in}{2.192275in}}{\pgfqpoint{2.234804in}{2.192275in}}%
\pgfpathclose%
\pgfusepath{stroke,fill}%
\end{pgfscope}%
\begin{pgfscope}%
\pgfpathrectangle{\pgfqpoint{0.100000in}{0.212622in}}{\pgfqpoint{3.696000in}{3.696000in}}%
\pgfusepath{clip}%
\pgfsetbuttcap%
\pgfsetroundjoin%
\definecolor{currentfill}{rgb}{0.121569,0.466667,0.705882}%
\pgfsetfillcolor{currentfill}%
\pgfsetfillopacity{0.888998}%
\pgfsetlinewidth{1.003750pt}%
\definecolor{currentstroke}{rgb}{0.121569,0.466667,0.705882}%
\pgfsetstrokecolor{currentstroke}%
\pgfsetstrokeopacity{0.888998}%
\pgfsetdash{}{0pt}%
\pgfpathmoveto{\pgfqpoint{1.726263in}{2.017840in}}%
\pgfpathcurveto{\pgfqpoint{1.734499in}{2.017840in}}{\pgfqpoint{1.742399in}{2.021113in}}{\pgfqpoint{1.748223in}{2.026937in}}%
\pgfpathcurveto{\pgfqpoint{1.754047in}{2.032761in}}{\pgfqpoint{1.757319in}{2.040661in}}{\pgfqpoint{1.757319in}{2.048897in}}%
\pgfpathcurveto{\pgfqpoint{1.757319in}{2.057133in}}{\pgfqpoint{1.754047in}{2.065033in}}{\pgfqpoint{1.748223in}{2.070857in}}%
\pgfpathcurveto{\pgfqpoint{1.742399in}{2.076681in}}{\pgfqpoint{1.734499in}{2.079953in}}{\pgfqpoint{1.726263in}{2.079953in}}%
\pgfpathcurveto{\pgfqpoint{1.718026in}{2.079953in}}{\pgfqpoint{1.710126in}{2.076681in}}{\pgfqpoint{1.704302in}{2.070857in}}%
\pgfpathcurveto{\pgfqpoint{1.698479in}{2.065033in}}{\pgfqpoint{1.695206in}{2.057133in}}{\pgfqpoint{1.695206in}{2.048897in}}%
\pgfpathcurveto{\pgfqpoint{1.695206in}{2.040661in}}{\pgfqpoint{1.698479in}{2.032761in}}{\pgfqpoint{1.704302in}{2.026937in}}%
\pgfpathcurveto{\pgfqpoint{1.710126in}{2.021113in}}{\pgfqpoint{1.718026in}{2.017840in}}{\pgfqpoint{1.726263in}{2.017840in}}%
\pgfpathclose%
\pgfusepath{stroke,fill}%
\end{pgfscope}%
\begin{pgfscope}%
\pgfpathrectangle{\pgfqpoint{0.100000in}{0.212622in}}{\pgfqpoint{3.696000in}{3.696000in}}%
\pgfusepath{clip}%
\pgfsetbuttcap%
\pgfsetroundjoin%
\definecolor{currentfill}{rgb}{0.121569,0.466667,0.705882}%
\pgfsetfillcolor{currentfill}%
\pgfsetfillopacity{0.889430}%
\pgfsetlinewidth{1.003750pt}%
\definecolor{currentstroke}{rgb}{0.121569,0.466667,0.705882}%
\pgfsetstrokecolor{currentstroke}%
\pgfsetstrokeopacity{0.889430}%
\pgfsetdash{}{0pt}%
\pgfpathmoveto{\pgfqpoint{1.618869in}{1.174481in}}%
\pgfpathcurveto{\pgfqpoint{1.627105in}{1.174481in}}{\pgfqpoint{1.635005in}{1.177754in}}{\pgfqpoint{1.640829in}{1.183578in}}%
\pgfpathcurveto{\pgfqpoint{1.646653in}{1.189402in}}{\pgfqpoint{1.649925in}{1.197302in}}{\pgfqpoint{1.649925in}{1.205538in}}%
\pgfpathcurveto{\pgfqpoint{1.649925in}{1.213774in}}{\pgfqpoint{1.646653in}{1.221674in}}{\pgfqpoint{1.640829in}{1.227498in}}%
\pgfpathcurveto{\pgfqpoint{1.635005in}{1.233322in}}{\pgfqpoint{1.627105in}{1.236594in}}{\pgfqpoint{1.618869in}{1.236594in}}%
\pgfpathcurveto{\pgfqpoint{1.610632in}{1.236594in}}{\pgfqpoint{1.602732in}{1.233322in}}{\pgfqpoint{1.596908in}{1.227498in}}%
\pgfpathcurveto{\pgfqpoint{1.591085in}{1.221674in}}{\pgfqpoint{1.587812in}{1.213774in}}{\pgfqpoint{1.587812in}{1.205538in}}%
\pgfpathcurveto{\pgfqpoint{1.587812in}{1.197302in}}{\pgfqpoint{1.591085in}{1.189402in}}{\pgfqpoint{1.596908in}{1.183578in}}%
\pgfpathcurveto{\pgfqpoint{1.602732in}{1.177754in}}{\pgfqpoint{1.610632in}{1.174481in}}{\pgfqpoint{1.618869in}{1.174481in}}%
\pgfpathclose%
\pgfusepath{stroke,fill}%
\end{pgfscope}%
\begin{pgfscope}%
\pgfpathrectangle{\pgfqpoint{0.100000in}{0.212622in}}{\pgfqpoint{3.696000in}{3.696000in}}%
\pgfusepath{clip}%
\pgfsetbuttcap%
\pgfsetroundjoin%
\definecolor{currentfill}{rgb}{0.121569,0.466667,0.705882}%
\pgfsetfillcolor{currentfill}%
\pgfsetfillopacity{0.889713}%
\pgfsetlinewidth{1.003750pt}%
\definecolor{currentstroke}{rgb}{0.121569,0.466667,0.705882}%
\pgfsetstrokecolor{currentstroke}%
\pgfsetstrokeopacity{0.889713}%
\pgfsetdash{}{0pt}%
\pgfpathmoveto{\pgfqpoint{2.233161in}{2.188015in}}%
\pgfpathcurveto{\pgfqpoint{2.241398in}{2.188015in}}{\pgfqpoint{2.249298in}{2.191287in}}{\pgfqpoint{2.255122in}{2.197111in}}%
\pgfpathcurveto{\pgfqpoint{2.260946in}{2.202935in}}{\pgfqpoint{2.264218in}{2.210835in}}{\pgfqpoint{2.264218in}{2.219072in}}%
\pgfpathcurveto{\pgfqpoint{2.264218in}{2.227308in}}{\pgfqpoint{2.260946in}{2.235208in}}{\pgfqpoint{2.255122in}{2.241032in}}%
\pgfpathcurveto{\pgfqpoint{2.249298in}{2.246856in}}{\pgfqpoint{2.241398in}{2.250128in}}{\pgfqpoint{2.233161in}{2.250128in}}%
\pgfpathcurveto{\pgfqpoint{2.224925in}{2.250128in}}{\pgfqpoint{2.217025in}{2.246856in}}{\pgfqpoint{2.211201in}{2.241032in}}%
\pgfpathcurveto{\pgfqpoint{2.205377in}{2.235208in}}{\pgfqpoint{2.202105in}{2.227308in}}{\pgfqpoint{2.202105in}{2.219072in}}%
\pgfpathcurveto{\pgfqpoint{2.202105in}{2.210835in}}{\pgfqpoint{2.205377in}{2.202935in}}{\pgfqpoint{2.211201in}{2.197111in}}%
\pgfpathcurveto{\pgfqpoint{2.217025in}{2.191287in}}{\pgfqpoint{2.224925in}{2.188015in}}{\pgfqpoint{2.233161in}{2.188015in}}%
\pgfpathclose%
\pgfusepath{stroke,fill}%
\end{pgfscope}%
\begin{pgfscope}%
\pgfpathrectangle{\pgfqpoint{0.100000in}{0.212622in}}{\pgfqpoint{3.696000in}{3.696000in}}%
\pgfusepath{clip}%
\pgfsetbuttcap%
\pgfsetroundjoin%
\definecolor{currentfill}{rgb}{0.121569,0.466667,0.705882}%
\pgfsetfillcolor{currentfill}%
\pgfsetfillopacity{0.889744}%
\pgfsetlinewidth{1.003750pt}%
\definecolor{currentstroke}{rgb}{0.121569,0.466667,0.705882}%
\pgfsetstrokecolor{currentstroke}%
\pgfsetstrokeopacity{0.889744}%
\pgfsetdash{}{0pt}%
\pgfpathmoveto{\pgfqpoint{1.726868in}{2.014343in}}%
\pgfpathcurveto{\pgfqpoint{1.735104in}{2.014343in}}{\pgfqpoint{1.743004in}{2.017615in}}{\pgfqpoint{1.748828in}{2.023439in}}%
\pgfpathcurveto{\pgfqpoint{1.754652in}{2.029263in}}{\pgfqpoint{1.757924in}{2.037163in}}{\pgfqpoint{1.757924in}{2.045400in}}%
\pgfpathcurveto{\pgfqpoint{1.757924in}{2.053636in}}{\pgfqpoint{1.754652in}{2.061536in}}{\pgfqpoint{1.748828in}{2.067360in}}%
\pgfpathcurveto{\pgfqpoint{1.743004in}{2.073184in}}{\pgfqpoint{1.735104in}{2.076456in}}{\pgfqpoint{1.726868in}{2.076456in}}%
\pgfpathcurveto{\pgfqpoint{1.718631in}{2.076456in}}{\pgfqpoint{1.710731in}{2.073184in}}{\pgfqpoint{1.704907in}{2.067360in}}%
\pgfpathcurveto{\pgfqpoint{1.699084in}{2.061536in}}{\pgfqpoint{1.695811in}{2.053636in}}{\pgfqpoint{1.695811in}{2.045400in}}%
\pgfpathcurveto{\pgfqpoint{1.695811in}{2.037163in}}{\pgfqpoint{1.699084in}{2.029263in}}{\pgfqpoint{1.704907in}{2.023439in}}%
\pgfpathcurveto{\pgfqpoint{1.710731in}{2.017615in}}{\pgfqpoint{1.718631in}{2.014343in}}{\pgfqpoint{1.726868in}{2.014343in}}%
\pgfpathclose%
\pgfusepath{stroke,fill}%
\end{pgfscope}%
\begin{pgfscope}%
\pgfpathrectangle{\pgfqpoint{0.100000in}{0.212622in}}{\pgfqpoint{3.696000in}{3.696000in}}%
\pgfusepath{clip}%
\pgfsetbuttcap%
\pgfsetroundjoin%
\definecolor{currentfill}{rgb}{0.121569,0.466667,0.705882}%
\pgfsetfillcolor{currentfill}%
\pgfsetfillopacity{0.890231}%
\pgfsetlinewidth{1.003750pt}%
\definecolor{currentstroke}{rgb}{0.121569,0.466667,0.705882}%
\pgfsetstrokecolor{currentstroke}%
\pgfsetstrokeopacity{0.890231}%
\pgfsetdash{}{0pt}%
\pgfpathmoveto{\pgfqpoint{2.831378in}{1.433775in}}%
\pgfpathcurveto{\pgfqpoint{2.839614in}{1.433775in}}{\pgfqpoint{2.847514in}{1.437048in}}{\pgfqpoint{2.853338in}{1.442872in}}%
\pgfpathcurveto{\pgfqpoint{2.859162in}{1.448696in}}{\pgfqpoint{2.862435in}{1.456596in}}{\pgfqpoint{2.862435in}{1.464832in}}%
\pgfpathcurveto{\pgfqpoint{2.862435in}{1.473068in}}{\pgfqpoint{2.859162in}{1.480968in}}{\pgfqpoint{2.853338in}{1.486792in}}%
\pgfpathcurveto{\pgfqpoint{2.847514in}{1.492616in}}{\pgfqpoint{2.839614in}{1.495888in}}{\pgfqpoint{2.831378in}{1.495888in}}%
\pgfpathcurveto{\pgfqpoint{2.823142in}{1.495888in}}{\pgfqpoint{2.815242in}{1.492616in}}{\pgfqpoint{2.809418in}{1.486792in}}%
\pgfpathcurveto{\pgfqpoint{2.803594in}{1.480968in}}{\pgfqpoint{2.800322in}{1.473068in}}{\pgfqpoint{2.800322in}{1.464832in}}%
\pgfpathcurveto{\pgfqpoint{2.800322in}{1.456596in}}{\pgfqpoint{2.803594in}{1.448696in}}{\pgfqpoint{2.809418in}{1.442872in}}%
\pgfpathcurveto{\pgfqpoint{2.815242in}{1.437048in}}{\pgfqpoint{2.823142in}{1.433775in}}{\pgfqpoint{2.831378in}{1.433775in}}%
\pgfpathclose%
\pgfusepath{stroke,fill}%
\end{pgfscope}%
\begin{pgfscope}%
\pgfpathrectangle{\pgfqpoint{0.100000in}{0.212622in}}{\pgfqpoint{3.696000in}{3.696000in}}%
\pgfusepath{clip}%
\pgfsetbuttcap%
\pgfsetroundjoin%
\definecolor{currentfill}{rgb}{0.121569,0.466667,0.705882}%
\pgfsetfillcolor{currentfill}%
\pgfsetfillopacity{0.890559}%
\pgfsetlinewidth{1.003750pt}%
\definecolor{currentstroke}{rgb}{0.121569,0.466667,0.705882}%
\pgfsetstrokecolor{currentstroke}%
\pgfsetstrokeopacity{0.890559}%
\pgfsetdash{}{0pt}%
\pgfpathmoveto{\pgfqpoint{1.631725in}{1.162098in}}%
\pgfpathcurveto{\pgfqpoint{1.639961in}{1.162098in}}{\pgfqpoint{1.647861in}{1.165371in}}{\pgfqpoint{1.653685in}{1.171194in}}%
\pgfpathcurveto{\pgfqpoint{1.659509in}{1.177018in}}{\pgfqpoint{1.662781in}{1.184918in}}{\pgfqpoint{1.662781in}{1.193155in}}%
\pgfpathcurveto{\pgfqpoint{1.662781in}{1.201391in}}{\pgfqpoint{1.659509in}{1.209291in}}{\pgfqpoint{1.653685in}{1.215115in}}%
\pgfpathcurveto{\pgfqpoint{1.647861in}{1.220939in}}{\pgfqpoint{1.639961in}{1.224211in}}{\pgfqpoint{1.631725in}{1.224211in}}%
\pgfpathcurveto{\pgfqpoint{1.623489in}{1.224211in}}{\pgfqpoint{1.615589in}{1.220939in}}{\pgfqpoint{1.609765in}{1.215115in}}%
\pgfpathcurveto{\pgfqpoint{1.603941in}{1.209291in}}{\pgfqpoint{1.600668in}{1.201391in}}{\pgfqpoint{1.600668in}{1.193155in}}%
\pgfpathcurveto{\pgfqpoint{1.600668in}{1.184918in}}{\pgfqpoint{1.603941in}{1.177018in}}{\pgfqpoint{1.609765in}{1.171194in}}%
\pgfpathcurveto{\pgfqpoint{1.615589in}{1.165371in}}{\pgfqpoint{1.623489in}{1.162098in}}{\pgfqpoint{1.631725in}{1.162098in}}%
\pgfpathclose%
\pgfusepath{stroke,fill}%
\end{pgfscope}%
\begin{pgfscope}%
\pgfpathrectangle{\pgfqpoint{0.100000in}{0.212622in}}{\pgfqpoint{3.696000in}{3.696000in}}%
\pgfusepath{clip}%
\pgfsetbuttcap%
\pgfsetroundjoin%
\definecolor{currentfill}{rgb}{0.121569,0.466667,0.705882}%
\pgfsetfillcolor{currentfill}%
\pgfsetfillopacity{0.890626}%
\pgfsetlinewidth{1.003750pt}%
\definecolor{currentstroke}{rgb}{0.121569,0.466667,0.705882}%
\pgfsetstrokecolor{currentstroke}%
\pgfsetstrokeopacity{0.890626}%
\pgfsetdash{}{0pt}%
\pgfpathmoveto{\pgfqpoint{1.727864in}{2.010654in}}%
\pgfpathcurveto{\pgfqpoint{1.736100in}{2.010654in}}{\pgfqpoint{1.744000in}{2.013926in}}{\pgfqpoint{1.749824in}{2.019750in}}%
\pgfpathcurveto{\pgfqpoint{1.755648in}{2.025574in}}{\pgfqpoint{1.758920in}{2.033474in}}{\pgfqpoint{1.758920in}{2.041710in}}%
\pgfpathcurveto{\pgfqpoint{1.758920in}{2.049947in}}{\pgfqpoint{1.755648in}{2.057847in}}{\pgfqpoint{1.749824in}{2.063671in}}%
\pgfpathcurveto{\pgfqpoint{1.744000in}{2.069494in}}{\pgfqpoint{1.736100in}{2.072767in}}{\pgfqpoint{1.727864in}{2.072767in}}%
\pgfpathcurveto{\pgfqpoint{1.719627in}{2.072767in}}{\pgfqpoint{1.711727in}{2.069494in}}{\pgfqpoint{1.705903in}{2.063671in}}%
\pgfpathcurveto{\pgfqpoint{1.700079in}{2.057847in}}{\pgfqpoint{1.696807in}{2.049947in}}{\pgfqpoint{1.696807in}{2.041710in}}%
\pgfpathcurveto{\pgfqpoint{1.696807in}{2.033474in}}{\pgfqpoint{1.700079in}{2.025574in}}{\pgfqpoint{1.705903in}{2.019750in}}%
\pgfpathcurveto{\pgfqpoint{1.711727in}{2.013926in}}{\pgfqpoint{1.719627in}{2.010654in}}{\pgfqpoint{1.727864in}{2.010654in}}%
\pgfpathclose%
\pgfusepath{stroke,fill}%
\end{pgfscope}%
\begin{pgfscope}%
\pgfpathrectangle{\pgfqpoint{0.100000in}{0.212622in}}{\pgfqpoint{3.696000in}{3.696000in}}%
\pgfusepath{clip}%
\pgfsetbuttcap%
\pgfsetroundjoin%
\definecolor{currentfill}{rgb}{0.121569,0.466667,0.705882}%
\pgfsetfillcolor{currentfill}%
\pgfsetfillopacity{0.891205}%
\pgfsetlinewidth{1.003750pt}%
\definecolor{currentstroke}{rgb}{0.121569,0.466667,0.705882}%
\pgfsetstrokecolor{currentstroke}%
\pgfsetstrokeopacity{0.891205}%
\pgfsetdash{}{0pt}%
\pgfpathmoveto{\pgfqpoint{2.230837in}{2.179859in}}%
\pgfpathcurveto{\pgfqpoint{2.239073in}{2.179859in}}{\pgfqpoint{2.246973in}{2.183132in}}{\pgfqpoint{2.252797in}{2.188956in}}%
\pgfpathcurveto{\pgfqpoint{2.258621in}{2.194780in}}{\pgfqpoint{2.261894in}{2.202680in}}{\pgfqpoint{2.261894in}{2.210916in}}%
\pgfpathcurveto{\pgfqpoint{2.261894in}{2.219152in}}{\pgfqpoint{2.258621in}{2.227052in}}{\pgfqpoint{2.252797in}{2.232876in}}%
\pgfpathcurveto{\pgfqpoint{2.246973in}{2.238700in}}{\pgfqpoint{2.239073in}{2.241972in}}{\pgfqpoint{2.230837in}{2.241972in}}%
\pgfpathcurveto{\pgfqpoint{2.222601in}{2.241972in}}{\pgfqpoint{2.214701in}{2.238700in}}{\pgfqpoint{2.208877in}{2.232876in}}%
\pgfpathcurveto{\pgfqpoint{2.203053in}{2.227052in}}{\pgfqpoint{2.199781in}{2.219152in}}{\pgfqpoint{2.199781in}{2.210916in}}%
\pgfpathcurveto{\pgfqpoint{2.199781in}{2.202680in}}{\pgfqpoint{2.203053in}{2.194780in}}{\pgfqpoint{2.208877in}{2.188956in}}%
\pgfpathcurveto{\pgfqpoint{2.214701in}{2.183132in}}{\pgfqpoint{2.222601in}{2.179859in}}{\pgfqpoint{2.230837in}{2.179859in}}%
\pgfpathclose%
\pgfusepath{stroke,fill}%
\end{pgfscope}%
\begin{pgfscope}%
\pgfpathrectangle{\pgfqpoint{0.100000in}{0.212622in}}{\pgfqpoint{3.696000in}{3.696000in}}%
\pgfusepath{clip}%
\pgfsetbuttcap%
\pgfsetroundjoin%
\definecolor{currentfill}{rgb}{0.121569,0.466667,0.705882}%
\pgfsetfillcolor{currentfill}%
\pgfsetfillopacity{0.891552}%
\pgfsetlinewidth{1.003750pt}%
\definecolor{currentstroke}{rgb}{0.121569,0.466667,0.705882}%
\pgfsetstrokecolor{currentstroke}%
\pgfsetstrokeopacity{0.891552}%
\pgfsetdash{}{0pt}%
\pgfpathmoveto{\pgfqpoint{1.729645in}{2.007447in}}%
\pgfpathcurveto{\pgfqpoint{1.737882in}{2.007447in}}{\pgfqpoint{1.745782in}{2.010720in}}{\pgfqpoint{1.751606in}{2.016544in}}%
\pgfpathcurveto{\pgfqpoint{1.757430in}{2.022368in}}{\pgfqpoint{1.760702in}{2.030268in}}{\pgfqpoint{1.760702in}{2.038504in}}%
\pgfpathcurveto{\pgfqpoint{1.760702in}{2.046740in}}{\pgfqpoint{1.757430in}{2.054640in}}{\pgfqpoint{1.751606in}{2.060464in}}%
\pgfpathcurveto{\pgfqpoint{1.745782in}{2.066288in}}{\pgfqpoint{1.737882in}{2.069560in}}{\pgfqpoint{1.729645in}{2.069560in}}%
\pgfpathcurveto{\pgfqpoint{1.721409in}{2.069560in}}{\pgfqpoint{1.713509in}{2.066288in}}{\pgfqpoint{1.707685in}{2.060464in}}%
\pgfpathcurveto{\pgfqpoint{1.701861in}{2.054640in}}{\pgfqpoint{1.698589in}{2.046740in}}{\pgfqpoint{1.698589in}{2.038504in}}%
\pgfpathcurveto{\pgfqpoint{1.698589in}{2.030268in}}{\pgfqpoint{1.701861in}{2.022368in}}{\pgfqpoint{1.707685in}{2.016544in}}%
\pgfpathcurveto{\pgfqpoint{1.713509in}{2.010720in}}{\pgfqpoint{1.721409in}{2.007447in}}{\pgfqpoint{1.729645in}{2.007447in}}%
\pgfpathclose%
\pgfusepath{stroke,fill}%
\end{pgfscope}%
\begin{pgfscope}%
\pgfpathrectangle{\pgfqpoint{0.100000in}{0.212622in}}{\pgfqpoint{3.696000in}{3.696000in}}%
\pgfusepath{clip}%
\pgfsetbuttcap%
\pgfsetroundjoin%
\definecolor{currentfill}{rgb}{0.121569,0.466667,0.705882}%
\pgfsetfillcolor{currentfill}%
\pgfsetfillopacity{0.891581}%
\pgfsetlinewidth{1.003750pt}%
\definecolor{currentstroke}{rgb}{0.121569,0.466667,0.705882}%
\pgfsetstrokecolor{currentstroke}%
\pgfsetstrokeopacity{0.891581}%
\pgfsetdash{}{0pt}%
\pgfpathmoveto{\pgfqpoint{1.639078in}{1.158005in}}%
\pgfpathcurveto{\pgfqpoint{1.647314in}{1.158005in}}{\pgfqpoint{1.655214in}{1.161277in}}{\pgfqpoint{1.661038in}{1.167101in}}%
\pgfpathcurveto{\pgfqpoint{1.666862in}{1.172925in}}{\pgfqpoint{1.670134in}{1.180825in}}{\pgfqpoint{1.670134in}{1.189062in}}%
\pgfpathcurveto{\pgfqpoint{1.670134in}{1.197298in}}{\pgfqpoint{1.666862in}{1.205198in}}{\pgfqpoint{1.661038in}{1.211022in}}%
\pgfpathcurveto{\pgfqpoint{1.655214in}{1.216846in}}{\pgfqpoint{1.647314in}{1.220118in}}{\pgfqpoint{1.639078in}{1.220118in}}%
\pgfpathcurveto{\pgfqpoint{1.630841in}{1.220118in}}{\pgfqpoint{1.622941in}{1.216846in}}{\pgfqpoint{1.617117in}{1.211022in}}%
\pgfpathcurveto{\pgfqpoint{1.611293in}{1.205198in}}{\pgfqpoint{1.608021in}{1.197298in}}{\pgfqpoint{1.608021in}{1.189062in}}%
\pgfpathcurveto{\pgfqpoint{1.608021in}{1.180825in}}{\pgfqpoint{1.611293in}{1.172925in}}{\pgfqpoint{1.617117in}{1.167101in}}%
\pgfpathcurveto{\pgfqpoint{1.622941in}{1.161277in}}{\pgfqpoint{1.630841in}{1.158005in}}{\pgfqpoint{1.639078in}{1.158005in}}%
\pgfpathclose%
\pgfusepath{stroke,fill}%
\end{pgfscope}%
\begin{pgfscope}%
\pgfpathrectangle{\pgfqpoint{0.100000in}{0.212622in}}{\pgfqpoint{3.696000in}{3.696000in}}%
\pgfusepath{clip}%
\pgfsetbuttcap%
\pgfsetroundjoin%
\definecolor{currentfill}{rgb}{0.121569,0.466667,0.705882}%
\pgfsetfillcolor{currentfill}%
\pgfsetfillopacity{0.892439}%
\pgfsetlinewidth{1.003750pt}%
\definecolor{currentstroke}{rgb}{0.121569,0.466667,0.705882}%
\pgfsetstrokecolor{currentstroke}%
\pgfsetstrokeopacity{0.892439}%
\pgfsetdash{}{0pt}%
\pgfpathmoveto{\pgfqpoint{1.732112in}{2.004217in}}%
\pgfpathcurveto{\pgfqpoint{1.740348in}{2.004217in}}{\pgfqpoint{1.748248in}{2.007490in}}{\pgfqpoint{1.754072in}{2.013314in}}%
\pgfpathcurveto{\pgfqpoint{1.759896in}{2.019137in}}{\pgfqpoint{1.763168in}{2.027037in}}{\pgfqpoint{1.763168in}{2.035274in}}%
\pgfpathcurveto{\pgfqpoint{1.763168in}{2.043510in}}{\pgfqpoint{1.759896in}{2.051410in}}{\pgfqpoint{1.754072in}{2.057234in}}%
\pgfpathcurveto{\pgfqpoint{1.748248in}{2.063058in}}{\pgfqpoint{1.740348in}{2.066330in}}{\pgfqpoint{1.732112in}{2.066330in}}%
\pgfpathcurveto{\pgfqpoint{1.723876in}{2.066330in}}{\pgfqpoint{1.715976in}{2.063058in}}{\pgfqpoint{1.710152in}{2.057234in}}%
\pgfpathcurveto{\pgfqpoint{1.704328in}{2.051410in}}{\pgfqpoint{1.701055in}{2.043510in}}{\pgfqpoint{1.701055in}{2.035274in}}%
\pgfpathcurveto{\pgfqpoint{1.701055in}{2.027037in}}{\pgfqpoint{1.704328in}{2.019137in}}{\pgfqpoint{1.710152in}{2.013314in}}%
\pgfpathcurveto{\pgfqpoint{1.715976in}{2.007490in}}{\pgfqpoint{1.723876in}{2.004217in}}{\pgfqpoint{1.732112in}{2.004217in}}%
\pgfpathclose%
\pgfusepath{stroke,fill}%
\end{pgfscope}%
\begin{pgfscope}%
\pgfpathrectangle{\pgfqpoint{0.100000in}{0.212622in}}{\pgfqpoint{3.696000in}{3.696000in}}%
\pgfusepath{clip}%
\pgfsetbuttcap%
\pgfsetroundjoin%
\definecolor{currentfill}{rgb}{0.121569,0.466667,0.705882}%
\pgfsetfillcolor{currentfill}%
\pgfsetfillopacity{0.892528}%
\pgfsetlinewidth{1.003750pt}%
\definecolor{currentstroke}{rgb}{0.121569,0.466667,0.705882}%
\pgfsetstrokecolor{currentstroke}%
\pgfsetstrokeopacity{0.892528}%
\pgfsetdash{}{0pt}%
\pgfpathmoveto{\pgfqpoint{2.227076in}{2.173943in}}%
\pgfpathcurveto{\pgfqpoint{2.235313in}{2.173943in}}{\pgfqpoint{2.243213in}{2.177216in}}{\pgfqpoint{2.249037in}{2.183040in}}%
\pgfpathcurveto{\pgfqpoint{2.254861in}{2.188863in}}{\pgfqpoint{2.258133in}{2.196764in}}{\pgfqpoint{2.258133in}{2.205000in}}%
\pgfpathcurveto{\pgfqpoint{2.258133in}{2.213236in}}{\pgfqpoint{2.254861in}{2.221136in}}{\pgfqpoint{2.249037in}{2.226960in}}%
\pgfpathcurveto{\pgfqpoint{2.243213in}{2.232784in}}{\pgfqpoint{2.235313in}{2.236056in}}{\pgfqpoint{2.227076in}{2.236056in}}%
\pgfpathcurveto{\pgfqpoint{2.218840in}{2.236056in}}{\pgfqpoint{2.210940in}{2.232784in}}{\pgfqpoint{2.205116in}{2.226960in}}%
\pgfpathcurveto{\pgfqpoint{2.199292in}{2.221136in}}{\pgfqpoint{2.196020in}{2.213236in}}{\pgfqpoint{2.196020in}{2.205000in}}%
\pgfpathcurveto{\pgfqpoint{2.196020in}{2.196764in}}{\pgfqpoint{2.199292in}{2.188863in}}{\pgfqpoint{2.205116in}{2.183040in}}%
\pgfpathcurveto{\pgfqpoint{2.210940in}{2.177216in}}{\pgfqpoint{2.218840in}{2.173943in}}{\pgfqpoint{2.227076in}{2.173943in}}%
\pgfpathclose%
\pgfusepath{stroke,fill}%
\end{pgfscope}%
\begin{pgfscope}%
\pgfpathrectangle{\pgfqpoint{0.100000in}{0.212622in}}{\pgfqpoint{3.696000in}{3.696000in}}%
\pgfusepath{clip}%
\pgfsetbuttcap%
\pgfsetroundjoin%
\definecolor{currentfill}{rgb}{0.121569,0.466667,0.705882}%
\pgfsetfillcolor{currentfill}%
\pgfsetfillopacity{0.892637}%
\pgfsetlinewidth{1.003750pt}%
\definecolor{currentstroke}{rgb}{0.121569,0.466667,0.705882}%
\pgfsetstrokecolor{currentstroke}%
\pgfsetstrokeopacity{0.892637}%
\pgfsetdash{}{0pt}%
\pgfpathmoveto{\pgfqpoint{1.647483in}{1.156320in}}%
\pgfpathcurveto{\pgfqpoint{1.655719in}{1.156320in}}{\pgfqpoint{1.663619in}{1.159592in}}{\pgfqpoint{1.669443in}{1.165416in}}%
\pgfpathcurveto{\pgfqpoint{1.675267in}{1.171240in}}{\pgfqpoint{1.678539in}{1.179140in}}{\pgfqpoint{1.678539in}{1.187376in}}%
\pgfpathcurveto{\pgfqpoint{1.678539in}{1.195613in}}{\pgfqpoint{1.675267in}{1.203513in}}{\pgfqpoint{1.669443in}{1.209337in}}%
\pgfpathcurveto{\pgfqpoint{1.663619in}{1.215160in}}{\pgfqpoint{1.655719in}{1.218433in}}{\pgfqpoint{1.647483in}{1.218433in}}%
\pgfpathcurveto{\pgfqpoint{1.639247in}{1.218433in}}{\pgfqpoint{1.631347in}{1.215160in}}{\pgfqpoint{1.625523in}{1.209337in}}%
\pgfpathcurveto{\pgfqpoint{1.619699in}{1.203513in}}{\pgfqpoint{1.616426in}{1.195613in}}{\pgfqpoint{1.616426in}{1.187376in}}%
\pgfpathcurveto{\pgfqpoint{1.616426in}{1.179140in}}{\pgfqpoint{1.619699in}{1.171240in}}{\pgfqpoint{1.625523in}{1.165416in}}%
\pgfpathcurveto{\pgfqpoint{1.631347in}{1.159592in}}{\pgfqpoint{1.639247in}{1.156320in}}{\pgfqpoint{1.647483in}{1.156320in}}%
\pgfpathclose%
\pgfusepath{stroke,fill}%
\end{pgfscope}%
\begin{pgfscope}%
\pgfpathrectangle{\pgfqpoint{0.100000in}{0.212622in}}{\pgfqpoint{3.696000in}{3.696000in}}%
\pgfusepath{clip}%
\pgfsetbuttcap%
\pgfsetroundjoin%
\definecolor{currentfill}{rgb}{0.121569,0.466667,0.705882}%
\pgfsetfillcolor{currentfill}%
\pgfsetfillopacity{0.893303}%
\pgfsetlinewidth{1.003750pt}%
\definecolor{currentstroke}{rgb}{0.121569,0.466667,0.705882}%
\pgfsetstrokecolor{currentstroke}%
\pgfsetstrokeopacity{0.893303}%
\pgfsetdash{}{0pt}%
\pgfpathmoveto{\pgfqpoint{1.735367in}{2.001659in}}%
\pgfpathcurveto{\pgfqpoint{1.743603in}{2.001659in}}{\pgfqpoint{1.751503in}{2.004931in}}{\pgfqpoint{1.757327in}{2.010755in}}%
\pgfpathcurveto{\pgfqpoint{1.763151in}{2.016579in}}{\pgfqpoint{1.766423in}{2.024479in}}{\pgfqpoint{1.766423in}{2.032715in}}%
\pgfpathcurveto{\pgfqpoint{1.766423in}{2.040952in}}{\pgfqpoint{1.763151in}{2.048852in}}{\pgfqpoint{1.757327in}{2.054676in}}%
\pgfpathcurveto{\pgfqpoint{1.751503in}{2.060499in}}{\pgfqpoint{1.743603in}{2.063772in}}{\pgfqpoint{1.735367in}{2.063772in}}%
\pgfpathcurveto{\pgfqpoint{1.727131in}{2.063772in}}{\pgfqpoint{1.719231in}{2.060499in}}{\pgfqpoint{1.713407in}{2.054676in}}%
\pgfpathcurveto{\pgfqpoint{1.707583in}{2.048852in}}{\pgfqpoint{1.704310in}{2.040952in}}{\pgfqpoint{1.704310in}{2.032715in}}%
\pgfpathcurveto{\pgfqpoint{1.704310in}{2.024479in}}{\pgfqpoint{1.707583in}{2.016579in}}{\pgfqpoint{1.713407in}{2.010755in}}%
\pgfpathcurveto{\pgfqpoint{1.719231in}{2.004931in}}{\pgfqpoint{1.727131in}{2.001659in}}{\pgfqpoint{1.735367in}{2.001659in}}%
\pgfpathclose%
\pgfusepath{stroke,fill}%
\end{pgfscope}%
\begin{pgfscope}%
\pgfpathrectangle{\pgfqpoint{0.100000in}{0.212622in}}{\pgfqpoint{3.696000in}{3.696000in}}%
\pgfusepath{clip}%
\pgfsetbuttcap%
\pgfsetroundjoin%
\definecolor{currentfill}{rgb}{0.121569,0.466667,0.705882}%
\pgfsetfillcolor{currentfill}%
\pgfsetfillopacity{0.893518}%
\pgfsetlinewidth{1.003750pt}%
\definecolor{currentstroke}{rgb}{0.121569,0.466667,0.705882}%
\pgfsetstrokecolor{currentstroke}%
\pgfsetstrokeopacity{0.893518}%
\pgfsetdash{}{0pt}%
\pgfpathmoveto{\pgfqpoint{2.223019in}{2.168959in}}%
\pgfpathcurveto{\pgfqpoint{2.231255in}{2.168959in}}{\pgfqpoint{2.239155in}{2.172231in}}{\pgfqpoint{2.244979in}{2.178055in}}%
\pgfpathcurveto{\pgfqpoint{2.250803in}{2.183879in}}{\pgfqpoint{2.254076in}{2.191779in}}{\pgfqpoint{2.254076in}{2.200016in}}%
\pgfpathcurveto{\pgfqpoint{2.254076in}{2.208252in}}{\pgfqpoint{2.250803in}{2.216152in}}{\pgfqpoint{2.244979in}{2.221976in}}%
\pgfpathcurveto{\pgfqpoint{2.239155in}{2.227800in}}{\pgfqpoint{2.231255in}{2.231072in}}{\pgfqpoint{2.223019in}{2.231072in}}%
\pgfpathcurveto{\pgfqpoint{2.214783in}{2.231072in}}{\pgfqpoint{2.206883in}{2.227800in}}{\pgfqpoint{2.201059in}{2.221976in}}%
\pgfpathcurveto{\pgfqpoint{2.195235in}{2.216152in}}{\pgfqpoint{2.191963in}{2.208252in}}{\pgfqpoint{2.191963in}{2.200016in}}%
\pgfpathcurveto{\pgfqpoint{2.191963in}{2.191779in}}{\pgfqpoint{2.195235in}{2.183879in}}{\pgfqpoint{2.201059in}{2.178055in}}%
\pgfpathcurveto{\pgfqpoint{2.206883in}{2.172231in}}{\pgfqpoint{2.214783in}{2.168959in}}{\pgfqpoint{2.223019in}{2.168959in}}%
\pgfpathclose%
\pgfusepath{stroke,fill}%
\end{pgfscope}%
\begin{pgfscope}%
\pgfpathrectangle{\pgfqpoint{0.100000in}{0.212622in}}{\pgfqpoint{3.696000in}{3.696000in}}%
\pgfusepath{clip}%
\pgfsetbuttcap%
\pgfsetroundjoin%
\definecolor{currentfill}{rgb}{0.121569,0.466667,0.705882}%
\pgfsetfillcolor{currentfill}%
\pgfsetfillopacity{0.894005}%
\pgfsetlinewidth{1.003750pt}%
\definecolor{currentstroke}{rgb}{0.121569,0.466667,0.705882}%
\pgfsetstrokecolor{currentstroke}%
\pgfsetstrokeopacity{0.894005}%
\pgfsetdash{}{0pt}%
\pgfpathmoveto{\pgfqpoint{1.657897in}{1.152805in}}%
\pgfpathcurveto{\pgfqpoint{1.666134in}{1.152805in}}{\pgfqpoint{1.674034in}{1.156077in}}{\pgfqpoint{1.679858in}{1.161901in}}%
\pgfpathcurveto{\pgfqpoint{1.685682in}{1.167725in}}{\pgfqpoint{1.688954in}{1.175625in}}{\pgfqpoint{1.688954in}{1.183862in}}%
\pgfpathcurveto{\pgfqpoint{1.688954in}{1.192098in}}{\pgfqpoint{1.685682in}{1.199998in}}{\pgfqpoint{1.679858in}{1.205822in}}%
\pgfpathcurveto{\pgfqpoint{1.674034in}{1.211646in}}{\pgfqpoint{1.666134in}{1.214918in}}{\pgfqpoint{1.657897in}{1.214918in}}%
\pgfpathcurveto{\pgfqpoint{1.649661in}{1.214918in}}{\pgfqpoint{1.641761in}{1.211646in}}{\pgfqpoint{1.635937in}{1.205822in}}%
\pgfpathcurveto{\pgfqpoint{1.630113in}{1.199998in}}{\pgfqpoint{1.626841in}{1.192098in}}{\pgfqpoint{1.626841in}{1.183862in}}%
\pgfpathcurveto{\pgfqpoint{1.626841in}{1.175625in}}{\pgfqpoint{1.630113in}{1.167725in}}{\pgfqpoint{1.635937in}{1.161901in}}%
\pgfpathcurveto{\pgfqpoint{1.641761in}{1.156077in}}{\pgfqpoint{1.649661in}{1.152805in}}{\pgfqpoint{1.657897in}{1.152805in}}%
\pgfpathclose%
\pgfusepath{stroke,fill}%
\end{pgfscope}%
\begin{pgfscope}%
\pgfpathrectangle{\pgfqpoint{0.100000in}{0.212622in}}{\pgfqpoint{3.696000in}{3.696000in}}%
\pgfusepath{clip}%
\pgfsetbuttcap%
\pgfsetroundjoin%
\definecolor{currentfill}{rgb}{0.121569,0.466667,0.705882}%
\pgfsetfillcolor{currentfill}%
\pgfsetfillopacity{0.894206}%
\pgfsetlinewidth{1.003750pt}%
\definecolor{currentstroke}{rgb}{0.121569,0.466667,0.705882}%
\pgfsetstrokecolor{currentstroke}%
\pgfsetstrokeopacity{0.894206}%
\pgfsetdash{}{0pt}%
\pgfpathmoveto{\pgfqpoint{1.739070in}{1.999117in}}%
\pgfpathcurveto{\pgfqpoint{1.747306in}{1.999117in}}{\pgfqpoint{1.755206in}{2.002389in}}{\pgfqpoint{1.761030in}{2.008213in}}%
\pgfpathcurveto{\pgfqpoint{1.766854in}{2.014037in}}{\pgfqpoint{1.770127in}{2.021937in}}{\pgfqpoint{1.770127in}{2.030174in}}%
\pgfpathcurveto{\pgfqpoint{1.770127in}{2.038410in}}{\pgfqpoint{1.766854in}{2.046310in}}{\pgfqpoint{1.761030in}{2.052134in}}%
\pgfpathcurveto{\pgfqpoint{1.755206in}{2.057958in}}{\pgfqpoint{1.747306in}{2.061230in}}{\pgfqpoint{1.739070in}{2.061230in}}%
\pgfpathcurveto{\pgfqpoint{1.730834in}{2.061230in}}{\pgfqpoint{1.722934in}{2.057958in}}{\pgfqpoint{1.717110in}{2.052134in}}%
\pgfpathcurveto{\pgfqpoint{1.711286in}{2.046310in}}{\pgfqpoint{1.708014in}{2.038410in}}{\pgfqpoint{1.708014in}{2.030174in}}%
\pgfpathcurveto{\pgfqpoint{1.708014in}{2.021937in}}{\pgfqpoint{1.711286in}{2.014037in}}{\pgfqpoint{1.717110in}{2.008213in}}%
\pgfpathcurveto{\pgfqpoint{1.722934in}{2.002389in}}{\pgfqpoint{1.730834in}{1.999117in}}{\pgfqpoint{1.739070in}{1.999117in}}%
\pgfpathclose%
\pgfusepath{stroke,fill}%
\end{pgfscope}%
\begin{pgfscope}%
\pgfpathrectangle{\pgfqpoint{0.100000in}{0.212622in}}{\pgfqpoint{3.696000in}{3.696000in}}%
\pgfusepath{clip}%
\pgfsetbuttcap%
\pgfsetroundjoin%
\definecolor{currentfill}{rgb}{0.121569,0.466667,0.705882}%
\pgfsetfillcolor{currentfill}%
\pgfsetfillopacity{0.894539}%
\pgfsetlinewidth{1.003750pt}%
\definecolor{currentstroke}{rgb}{0.121569,0.466667,0.705882}%
\pgfsetstrokecolor{currentstroke}%
\pgfsetstrokeopacity{0.894539}%
\pgfsetdash{}{0pt}%
\pgfpathmoveto{\pgfqpoint{2.219943in}{2.163634in}}%
\pgfpathcurveto{\pgfqpoint{2.228180in}{2.163634in}}{\pgfqpoint{2.236080in}{2.166906in}}{\pgfqpoint{2.241904in}{2.172730in}}%
\pgfpathcurveto{\pgfqpoint{2.247728in}{2.178554in}}{\pgfqpoint{2.251000in}{2.186454in}}{\pgfqpoint{2.251000in}{2.194690in}}%
\pgfpathcurveto{\pgfqpoint{2.251000in}{2.202926in}}{\pgfqpoint{2.247728in}{2.210826in}}{\pgfqpoint{2.241904in}{2.216650in}}%
\pgfpathcurveto{\pgfqpoint{2.236080in}{2.222474in}}{\pgfqpoint{2.228180in}{2.225747in}}{\pgfqpoint{2.219943in}{2.225747in}}%
\pgfpathcurveto{\pgfqpoint{2.211707in}{2.225747in}}{\pgfqpoint{2.203807in}{2.222474in}}{\pgfqpoint{2.197983in}{2.216650in}}%
\pgfpathcurveto{\pgfqpoint{2.192159in}{2.210826in}}{\pgfqpoint{2.188887in}{2.202926in}}{\pgfqpoint{2.188887in}{2.194690in}}%
\pgfpathcurveto{\pgfqpoint{2.188887in}{2.186454in}}{\pgfqpoint{2.192159in}{2.178554in}}{\pgfqpoint{2.197983in}{2.172730in}}%
\pgfpathcurveto{\pgfqpoint{2.203807in}{2.166906in}}{\pgfqpoint{2.211707in}{2.163634in}}{\pgfqpoint{2.219943in}{2.163634in}}%
\pgfpathclose%
\pgfusepath{stroke,fill}%
\end{pgfscope}%
\begin{pgfscope}%
\pgfpathrectangle{\pgfqpoint{0.100000in}{0.212622in}}{\pgfqpoint{3.696000in}{3.696000in}}%
\pgfusepath{clip}%
\pgfsetbuttcap%
\pgfsetroundjoin%
\definecolor{currentfill}{rgb}{0.121569,0.466667,0.705882}%
\pgfsetfillcolor{currentfill}%
\pgfsetfillopacity{0.894877}%
\pgfsetlinewidth{1.003750pt}%
\definecolor{currentstroke}{rgb}{0.121569,0.466667,0.705882}%
\pgfsetstrokecolor{currentstroke}%
\pgfsetstrokeopacity{0.894877}%
\pgfsetdash{}{0pt}%
\pgfpathmoveto{\pgfqpoint{2.811173in}{1.406144in}}%
\pgfpathcurveto{\pgfqpoint{2.819409in}{1.406144in}}{\pgfqpoint{2.827309in}{1.409416in}}{\pgfqpoint{2.833133in}{1.415240in}}%
\pgfpathcurveto{\pgfqpoint{2.838957in}{1.421064in}}{\pgfqpoint{2.842229in}{1.428964in}}{\pgfqpoint{2.842229in}{1.437200in}}%
\pgfpathcurveto{\pgfqpoint{2.842229in}{1.445437in}}{\pgfqpoint{2.838957in}{1.453337in}}{\pgfqpoint{2.833133in}{1.459161in}}%
\pgfpathcurveto{\pgfqpoint{2.827309in}{1.464985in}}{\pgfqpoint{2.819409in}{1.468257in}}{\pgfqpoint{2.811173in}{1.468257in}}%
\pgfpathcurveto{\pgfqpoint{2.802937in}{1.468257in}}{\pgfqpoint{2.795036in}{1.464985in}}{\pgfqpoint{2.789213in}{1.459161in}}%
\pgfpathcurveto{\pgfqpoint{2.783389in}{1.453337in}}{\pgfqpoint{2.780116in}{1.445437in}}{\pgfqpoint{2.780116in}{1.437200in}}%
\pgfpathcurveto{\pgfqpoint{2.780116in}{1.428964in}}{\pgfqpoint{2.783389in}{1.421064in}}{\pgfqpoint{2.789213in}{1.415240in}}%
\pgfpathcurveto{\pgfqpoint{2.795036in}{1.409416in}}{\pgfqpoint{2.802937in}{1.406144in}}{\pgfqpoint{2.811173in}{1.406144in}}%
\pgfpathclose%
\pgfusepath{stroke,fill}%
\end{pgfscope}%
\begin{pgfscope}%
\pgfpathrectangle{\pgfqpoint{0.100000in}{0.212622in}}{\pgfqpoint{3.696000in}{3.696000in}}%
\pgfusepath{clip}%
\pgfsetbuttcap%
\pgfsetroundjoin%
\definecolor{currentfill}{rgb}{0.121569,0.466667,0.705882}%
\pgfsetfillcolor{currentfill}%
\pgfsetfillopacity{0.895165}%
\pgfsetlinewidth{1.003750pt}%
\definecolor{currentstroke}{rgb}{0.121569,0.466667,0.705882}%
\pgfsetstrokecolor{currentstroke}%
\pgfsetstrokeopacity{0.895165}%
\pgfsetdash{}{0pt}%
\pgfpathmoveto{\pgfqpoint{1.743511in}{1.997308in}}%
\pgfpathcurveto{\pgfqpoint{1.751747in}{1.997308in}}{\pgfqpoint{1.759647in}{2.000581in}}{\pgfqpoint{1.765471in}{2.006404in}}%
\pgfpathcurveto{\pgfqpoint{1.771295in}{2.012228in}}{\pgfqpoint{1.774567in}{2.020128in}}{\pgfqpoint{1.774567in}{2.028365in}}%
\pgfpathcurveto{\pgfqpoint{1.774567in}{2.036601in}}{\pgfqpoint{1.771295in}{2.044501in}}{\pgfqpoint{1.765471in}{2.050325in}}%
\pgfpathcurveto{\pgfqpoint{1.759647in}{2.056149in}}{\pgfqpoint{1.751747in}{2.059421in}}{\pgfqpoint{1.743511in}{2.059421in}}%
\pgfpathcurveto{\pgfqpoint{1.735275in}{2.059421in}}{\pgfqpoint{1.727375in}{2.056149in}}{\pgfqpoint{1.721551in}{2.050325in}}%
\pgfpathcurveto{\pgfqpoint{1.715727in}{2.044501in}}{\pgfqpoint{1.712454in}{2.036601in}}{\pgfqpoint{1.712454in}{2.028365in}}%
\pgfpathcurveto{\pgfqpoint{1.712454in}{2.020128in}}{\pgfqpoint{1.715727in}{2.012228in}}{\pgfqpoint{1.721551in}{2.006404in}}%
\pgfpathcurveto{\pgfqpoint{1.727375in}{2.000581in}}{\pgfqpoint{1.735275in}{1.997308in}}{\pgfqpoint{1.743511in}{1.997308in}}%
\pgfpathclose%
\pgfusepath{stroke,fill}%
\end{pgfscope}%
\begin{pgfscope}%
\pgfpathrectangle{\pgfqpoint{0.100000in}{0.212622in}}{\pgfqpoint{3.696000in}{3.696000in}}%
\pgfusepath{clip}%
\pgfsetbuttcap%
\pgfsetroundjoin%
\definecolor{currentfill}{rgb}{0.121569,0.466667,0.705882}%
\pgfsetfillcolor{currentfill}%
\pgfsetfillopacity{0.895426}%
\pgfsetlinewidth{1.003750pt}%
\definecolor{currentstroke}{rgb}{0.121569,0.466667,0.705882}%
\pgfsetstrokecolor{currentstroke}%
\pgfsetstrokeopacity{0.895426}%
\pgfsetdash{}{0pt}%
\pgfpathmoveto{\pgfqpoint{1.668592in}{1.143083in}}%
\pgfpathcurveto{\pgfqpoint{1.676828in}{1.143083in}}{\pgfqpoint{1.684728in}{1.146355in}}{\pgfqpoint{1.690552in}{1.152179in}}%
\pgfpathcurveto{\pgfqpoint{1.696376in}{1.158003in}}{\pgfqpoint{1.699648in}{1.165903in}}{\pgfqpoint{1.699648in}{1.174140in}}%
\pgfpathcurveto{\pgfqpoint{1.699648in}{1.182376in}}{\pgfqpoint{1.696376in}{1.190276in}}{\pgfqpoint{1.690552in}{1.196100in}}%
\pgfpathcurveto{\pgfqpoint{1.684728in}{1.201924in}}{\pgfqpoint{1.676828in}{1.205196in}}{\pgfqpoint{1.668592in}{1.205196in}}%
\pgfpathcurveto{\pgfqpoint{1.660355in}{1.205196in}}{\pgfqpoint{1.652455in}{1.201924in}}{\pgfqpoint{1.646631in}{1.196100in}}%
\pgfpathcurveto{\pgfqpoint{1.640808in}{1.190276in}}{\pgfqpoint{1.637535in}{1.182376in}}{\pgfqpoint{1.637535in}{1.174140in}}%
\pgfpathcurveto{\pgfqpoint{1.637535in}{1.165903in}}{\pgfqpoint{1.640808in}{1.158003in}}{\pgfqpoint{1.646631in}{1.152179in}}%
\pgfpathcurveto{\pgfqpoint{1.652455in}{1.146355in}}{\pgfqpoint{1.660355in}{1.143083in}}{\pgfqpoint{1.668592in}{1.143083in}}%
\pgfpathclose%
\pgfusepath{stroke,fill}%
\end{pgfscope}%
\begin{pgfscope}%
\pgfpathrectangle{\pgfqpoint{0.100000in}{0.212622in}}{\pgfqpoint{3.696000in}{3.696000in}}%
\pgfusepath{clip}%
\pgfsetbuttcap%
\pgfsetroundjoin%
\definecolor{currentfill}{rgb}{0.121569,0.466667,0.705882}%
\pgfsetfillcolor{currentfill}%
\pgfsetfillopacity{0.895501}%
\pgfsetlinewidth{1.003750pt}%
\definecolor{currentstroke}{rgb}{0.121569,0.466667,0.705882}%
\pgfsetstrokecolor{currentstroke}%
\pgfsetstrokeopacity{0.895501}%
\pgfsetdash{}{0pt}%
\pgfpathmoveto{\pgfqpoint{2.218835in}{2.158119in}}%
\pgfpathcurveto{\pgfqpoint{2.227072in}{2.158119in}}{\pgfqpoint{2.234972in}{2.161391in}}{\pgfqpoint{2.240796in}{2.167215in}}%
\pgfpathcurveto{\pgfqpoint{2.246620in}{2.173039in}}{\pgfqpoint{2.249892in}{2.180939in}}{\pgfqpoint{2.249892in}{2.189175in}}%
\pgfpathcurveto{\pgfqpoint{2.249892in}{2.197411in}}{\pgfqpoint{2.246620in}{2.205312in}}{\pgfqpoint{2.240796in}{2.211135in}}%
\pgfpathcurveto{\pgfqpoint{2.234972in}{2.216959in}}{\pgfqpoint{2.227072in}{2.220232in}}{\pgfqpoint{2.218835in}{2.220232in}}%
\pgfpathcurveto{\pgfqpoint{2.210599in}{2.220232in}}{\pgfqpoint{2.202699in}{2.216959in}}{\pgfqpoint{2.196875in}{2.211135in}}%
\pgfpathcurveto{\pgfqpoint{2.191051in}{2.205312in}}{\pgfqpoint{2.187779in}{2.197411in}}{\pgfqpoint{2.187779in}{2.189175in}}%
\pgfpathcurveto{\pgfqpoint{2.187779in}{2.180939in}}{\pgfqpoint{2.191051in}{2.173039in}}{\pgfqpoint{2.196875in}{2.167215in}}%
\pgfpathcurveto{\pgfqpoint{2.202699in}{2.161391in}}{\pgfqpoint{2.210599in}{2.158119in}}{\pgfqpoint{2.218835in}{2.158119in}}%
\pgfpathclose%
\pgfusepath{stroke,fill}%
\end{pgfscope}%
\begin{pgfscope}%
\pgfpathrectangle{\pgfqpoint{0.100000in}{0.212622in}}{\pgfqpoint{3.696000in}{3.696000in}}%
\pgfusepath{clip}%
\pgfsetbuttcap%
\pgfsetroundjoin%
\definecolor{currentfill}{rgb}{0.121569,0.466667,0.705882}%
\pgfsetfillcolor{currentfill}%
\pgfsetfillopacity{0.896157}%
\pgfsetlinewidth{1.003750pt}%
\definecolor{currentstroke}{rgb}{0.121569,0.466667,0.705882}%
\pgfsetstrokecolor{currentstroke}%
\pgfsetstrokeopacity{0.896157}%
\pgfsetdash{}{0pt}%
\pgfpathmoveto{\pgfqpoint{1.748944in}{1.996326in}}%
\pgfpathcurveto{\pgfqpoint{1.757180in}{1.996326in}}{\pgfqpoint{1.765080in}{1.999598in}}{\pgfqpoint{1.770904in}{2.005422in}}%
\pgfpathcurveto{\pgfqpoint{1.776728in}{2.011246in}}{\pgfqpoint{1.780000in}{2.019146in}}{\pgfqpoint{1.780000in}{2.027382in}}%
\pgfpathcurveto{\pgfqpoint{1.780000in}{2.035619in}}{\pgfqpoint{1.776728in}{2.043519in}}{\pgfqpoint{1.770904in}{2.049343in}}%
\pgfpathcurveto{\pgfqpoint{1.765080in}{2.055167in}}{\pgfqpoint{1.757180in}{2.058439in}}{\pgfqpoint{1.748944in}{2.058439in}}%
\pgfpathcurveto{\pgfqpoint{1.740708in}{2.058439in}}{\pgfqpoint{1.732808in}{2.055167in}}{\pgfqpoint{1.726984in}{2.049343in}}%
\pgfpathcurveto{\pgfqpoint{1.721160in}{2.043519in}}{\pgfqpoint{1.717887in}{2.035619in}}{\pgfqpoint{1.717887in}{2.027382in}}%
\pgfpathcurveto{\pgfqpoint{1.717887in}{2.019146in}}{\pgfqpoint{1.721160in}{2.011246in}}{\pgfqpoint{1.726984in}{2.005422in}}%
\pgfpathcurveto{\pgfqpoint{1.732808in}{1.999598in}}{\pgfqpoint{1.740708in}{1.996326in}}{\pgfqpoint{1.748944in}{1.996326in}}%
\pgfpathclose%
\pgfusepath{stroke,fill}%
\end{pgfscope}%
\begin{pgfscope}%
\pgfpathrectangle{\pgfqpoint{0.100000in}{0.212622in}}{\pgfqpoint{3.696000in}{3.696000in}}%
\pgfusepath{clip}%
\pgfsetbuttcap%
\pgfsetroundjoin%
\definecolor{currentfill}{rgb}{0.121569,0.466667,0.705882}%
\pgfsetfillcolor{currentfill}%
\pgfsetfillopacity{0.896339}%
\pgfsetlinewidth{1.003750pt}%
\definecolor{currentstroke}{rgb}{0.121569,0.466667,0.705882}%
\pgfsetstrokecolor{currentstroke}%
\pgfsetstrokeopacity{0.896339}%
\pgfsetdash{}{0pt}%
\pgfpathmoveto{\pgfqpoint{2.216699in}{2.154212in}}%
\pgfpathcurveto{\pgfqpoint{2.224936in}{2.154212in}}{\pgfqpoint{2.232836in}{2.157484in}}{\pgfqpoint{2.238660in}{2.163308in}}%
\pgfpathcurveto{\pgfqpoint{2.244484in}{2.169132in}}{\pgfqpoint{2.247756in}{2.177032in}}{\pgfqpoint{2.247756in}{2.185268in}}%
\pgfpathcurveto{\pgfqpoint{2.247756in}{2.193504in}}{\pgfqpoint{2.244484in}{2.201405in}}{\pgfqpoint{2.238660in}{2.207228in}}%
\pgfpathcurveto{\pgfqpoint{2.232836in}{2.213052in}}{\pgfqpoint{2.224936in}{2.216325in}}{\pgfqpoint{2.216699in}{2.216325in}}%
\pgfpathcurveto{\pgfqpoint{2.208463in}{2.216325in}}{\pgfqpoint{2.200563in}{2.213052in}}{\pgfqpoint{2.194739in}{2.207228in}}%
\pgfpathcurveto{\pgfqpoint{2.188915in}{2.201405in}}{\pgfqpoint{2.185643in}{2.193504in}}{\pgfqpoint{2.185643in}{2.185268in}}%
\pgfpathcurveto{\pgfqpoint{2.185643in}{2.177032in}}{\pgfqpoint{2.188915in}{2.169132in}}{\pgfqpoint{2.194739in}{2.163308in}}%
\pgfpathcurveto{\pgfqpoint{2.200563in}{2.157484in}}{\pgfqpoint{2.208463in}{2.154212in}}{\pgfqpoint{2.216699in}{2.154212in}}%
\pgfpathclose%
\pgfusepath{stroke,fill}%
\end{pgfscope}%
\begin{pgfscope}%
\pgfpathrectangle{\pgfqpoint{0.100000in}{0.212622in}}{\pgfqpoint{3.696000in}{3.696000in}}%
\pgfusepath{clip}%
\pgfsetbuttcap%
\pgfsetroundjoin%
\definecolor{currentfill}{rgb}{0.121569,0.466667,0.705882}%
\pgfsetfillcolor{currentfill}%
\pgfsetfillopacity{0.896550}%
\pgfsetlinewidth{1.003750pt}%
\definecolor{currentstroke}{rgb}{0.121569,0.466667,0.705882}%
\pgfsetstrokecolor{currentstroke}%
\pgfsetstrokeopacity{0.896550}%
\pgfsetdash{}{0pt}%
\pgfpathmoveto{\pgfqpoint{1.680427in}{1.133250in}}%
\pgfpathcurveto{\pgfqpoint{1.688663in}{1.133250in}}{\pgfqpoint{1.696563in}{1.136522in}}{\pgfqpoint{1.702387in}{1.142346in}}%
\pgfpathcurveto{\pgfqpoint{1.708211in}{1.148170in}}{\pgfqpoint{1.711484in}{1.156070in}}{\pgfqpoint{1.711484in}{1.164307in}}%
\pgfpathcurveto{\pgfqpoint{1.711484in}{1.172543in}}{\pgfqpoint{1.708211in}{1.180443in}}{\pgfqpoint{1.702387in}{1.186267in}}%
\pgfpathcurveto{\pgfqpoint{1.696563in}{1.192091in}}{\pgfqpoint{1.688663in}{1.195363in}}{\pgfqpoint{1.680427in}{1.195363in}}%
\pgfpathcurveto{\pgfqpoint{1.672191in}{1.195363in}}{\pgfqpoint{1.664291in}{1.192091in}}{\pgfqpoint{1.658467in}{1.186267in}}%
\pgfpathcurveto{\pgfqpoint{1.652643in}{1.180443in}}{\pgfqpoint{1.649371in}{1.172543in}}{\pgfqpoint{1.649371in}{1.164307in}}%
\pgfpathcurveto{\pgfqpoint{1.649371in}{1.156070in}}{\pgfqpoint{1.652643in}{1.148170in}}{\pgfqpoint{1.658467in}{1.142346in}}%
\pgfpathcurveto{\pgfqpoint{1.664291in}{1.136522in}}{\pgfqpoint{1.672191in}{1.133250in}}{\pgfqpoint{1.680427in}{1.133250in}}%
\pgfpathclose%
\pgfusepath{stroke,fill}%
\end{pgfscope}%
\begin{pgfscope}%
\pgfpathrectangle{\pgfqpoint{0.100000in}{0.212622in}}{\pgfqpoint{3.696000in}{3.696000in}}%
\pgfusepath{clip}%
\pgfsetbuttcap%
\pgfsetroundjoin%
\definecolor{currentfill}{rgb}{0.121569,0.466667,0.705882}%
\pgfsetfillcolor{currentfill}%
\pgfsetfillopacity{0.896755}%
\pgfsetlinewidth{1.003750pt}%
\definecolor{currentstroke}{rgb}{0.121569,0.466667,0.705882}%
\pgfsetstrokecolor{currentstroke}%
\pgfsetstrokeopacity{0.896755}%
\pgfsetdash{}{0pt}%
\pgfpathmoveto{\pgfqpoint{2.214841in}{2.152081in}}%
\pgfpathcurveto{\pgfqpoint{2.223077in}{2.152081in}}{\pgfqpoint{2.230977in}{2.155353in}}{\pgfqpoint{2.236801in}{2.161177in}}%
\pgfpathcurveto{\pgfqpoint{2.242625in}{2.167001in}}{\pgfqpoint{2.245898in}{2.174901in}}{\pgfqpoint{2.245898in}{2.183137in}}%
\pgfpathcurveto{\pgfqpoint{2.245898in}{2.191374in}}{\pgfqpoint{2.242625in}{2.199274in}}{\pgfqpoint{2.236801in}{2.205098in}}%
\pgfpathcurveto{\pgfqpoint{2.230977in}{2.210921in}}{\pgfqpoint{2.223077in}{2.214194in}}{\pgfqpoint{2.214841in}{2.214194in}}%
\pgfpathcurveto{\pgfqpoint{2.206605in}{2.214194in}}{\pgfqpoint{2.198705in}{2.210921in}}{\pgfqpoint{2.192881in}{2.205098in}}%
\pgfpathcurveto{\pgfqpoint{2.187057in}{2.199274in}}{\pgfqpoint{2.183785in}{2.191374in}}{\pgfqpoint{2.183785in}{2.183137in}}%
\pgfpathcurveto{\pgfqpoint{2.183785in}{2.174901in}}{\pgfqpoint{2.187057in}{2.167001in}}{\pgfqpoint{2.192881in}{2.161177in}}%
\pgfpathcurveto{\pgfqpoint{2.198705in}{2.155353in}}{\pgfqpoint{2.206605in}{2.152081in}}{\pgfqpoint{2.214841in}{2.152081in}}%
\pgfpathclose%
\pgfusepath{stroke,fill}%
\end{pgfscope}%
\begin{pgfscope}%
\pgfpathrectangle{\pgfqpoint{0.100000in}{0.212622in}}{\pgfqpoint{3.696000in}{3.696000in}}%
\pgfusepath{clip}%
\pgfsetbuttcap%
\pgfsetroundjoin%
\definecolor{currentfill}{rgb}{0.121569,0.466667,0.705882}%
\pgfsetfillcolor{currentfill}%
\pgfsetfillopacity{0.897071}%
\pgfsetlinewidth{1.003750pt}%
\definecolor{currentstroke}{rgb}{0.121569,0.466667,0.705882}%
\pgfsetstrokecolor{currentstroke}%
\pgfsetstrokeopacity{0.897071}%
\pgfsetdash{}{0pt}%
\pgfpathmoveto{\pgfqpoint{1.755744in}{1.997473in}}%
\pgfpathcurveto{\pgfqpoint{1.763980in}{1.997473in}}{\pgfqpoint{1.771880in}{2.000746in}}{\pgfqpoint{1.777704in}{2.006570in}}%
\pgfpathcurveto{\pgfqpoint{1.783528in}{2.012394in}}{\pgfqpoint{1.786800in}{2.020294in}}{\pgfqpoint{1.786800in}{2.028530in}}%
\pgfpathcurveto{\pgfqpoint{1.786800in}{2.036766in}}{\pgfqpoint{1.783528in}{2.044666in}}{\pgfqpoint{1.777704in}{2.050490in}}%
\pgfpathcurveto{\pgfqpoint{1.771880in}{2.056314in}}{\pgfqpoint{1.763980in}{2.059586in}}{\pgfqpoint{1.755744in}{2.059586in}}%
\pgfpathcurveto{\pgfqpoint{1.747508in}{2.059586in}}{\pgfqpoint{1.739608in}{2.056314in}}{\pgfqpoint{1.733784in}{2.050490in}}%
\pgfpathcurveto{\pgfqpoint{1.727960in}{2.044666in}}{\pgfqpoint{1.724687in}{2.036766in}}{\pgfqpoint{1.724687in}{2.028530in}}%
\pgfpathcurveto{\pgfqpoint{1.724687in}{2.020294in}}{\pgfqpoint{1.727960in}{2.012394in}}{\pgfqpoint{1.733784in}{2.006570in}}%
\pgfpathcurveto{\pgfqpoint{1.739608in}{2.000746in}}{\pgfqpoint{1.747508in}{1.997473in}}{\pgfqpoint{1.755744in}{1.997473in}}%
\pgfpathclose%
\pgfusepath{stroke,fill}%
\end{pgfscope}%
\begin{pgfscope}%
\pgfpathrectangle{\pgfqpoint{0.100000in}{0.212622in}}{\pgfqpoint{3.696000in}{3.696000in}}%
\pgfusepath{clip}%
\pgfsetbuttcap%
\pgfsetroundjoin%
\definecolor{currentfill}{rgb}{0.121569,0.466667,0.705882}%
\pgfsetfillcolor{currentfill}%
\pgfsetfillopacity{0.897630}%
\pgfsetlinewidth{1.003750pt}%
\definecolor{currentstroke}{rgb}{0.121569,0.466667,0.705882}%
\pgfsetstrokecolor{currentstroke}%
\pgfsetstrokeopacity{0.897630}%
\pgfsetdash{}{0pt}%
\pgfpathmoveto{\pgfqpoint{2.211948in}{2.147678in}}%
\pgfpathcurveto{\pgfqpoint{2.220184in}{2.147678in}}{\pgfqpoint{2.228084in}{2.150951in}}{\pgfqpoint{2.233908in}{2.156775in}}%
\pgfpathcurveto{\pgfqpoint{2.239732in}{2.162599in}}{\pgfqpoint{2.243004in}{2.170499in}}{\pgfqpoint{2.243004in}{2.178735in}}%
\pgfpathcurveto{\pgfqpoint{2.243004in}{2.186971in}}{\pgfqpoint{2.239732in}{2.194871in}}{\pgfqpoint{2.233908in}{2.200695in}}%
\pgfpathcurveto{\pgfqpoint{2.228084in}{2.206519in}}{\pgfqpoint{2.220184in}{2.209791in}}{\pgfqpoint{2.211948in}{2.209791in}}%
\pgfpathcurveto{\pgfqpoint{2.203712in}{2.209791in}}{\pgfqpoint{2.195812in}{2.206519in}}{\pgfqpoint{2.189988in}{2.200695in}}%
\pgfpathcurveto{\pgfqpoint{2.184164in}{2.194871in}}{\pgfqpoint{2.180891in}{2.186971in}}{\pgfqpoint{2.180891in}{2.178735in}}%
\pgfpathcurveto{\pgfqpoint{2.180891in}{2.170499in}}{\pgfqpoint{2.184164in}{2.162599in}}{\pgfqpoint{2.189988in}{2.156775in}}%
\pgfpathcurveto{\pgfqpoint{2.195812in}{2.150951in}}{\pgfqpoint{2.203712in}{2.147678in}}{\pgfqpoint{2.211948in}{2.147678in}}%
\pgfpathclose%
\pgfusepath{stroke,fill}%
\end{pgfscope}%
\begin{pgfscope}%
\pgfpathrectangle{\pgfqpoint{0.100000in}{0.212622in}}{\pgfqpoint{3.696000in}{3.696000in}}%
\pgfusepath{clip}%
\pgfsetbuttcap%
\pgfsetroundjoin%
\definecolor{currentfill}{rgb}{0.121569,0.466667,0.705882}%
\pgfsetfillcolor{currentfill}%
\pgfsetfillopacity{0.897862}%
\pgfsetlinewidth{1.003750pt}%
\definecolor{currentstroke}{rgb}{0.121569,0.466667,0.705882}%
\pgfsetstrokecolor{currentstroke}%
\pgfsetstrokeopacity{0.897862}%
\pgfsetdash{}{0pt}%
\pgfpathmoveto{\pgfqpoint{1.693426in}{1.123361in}}%
\pgfpathcurveto{\pgfqpoint{1.701662in}{1.123361in}}{\pgfqpoint{1.709562in}{1.126633in}}{\pgfqpoint{1.715386in}{1.132457in}}%
\pgfpathcurveto{\pgfqpoint{1.721210in}{1.138281in}}{\pgfqpoint{1.724482in}{1.146181in}}{\pgfqpoint{1.724482in}{1.154417in}}%
\pgfpathcurveto{\pgfqpoint{1.724482in}{1.162654in}}{\pgfqpoint{1.721210in}{1.170554in}}{\pgfqpoint{1.715386in}{1.176378in}}%
\pgfpathcurveto{\pgfqpoint{1.709562in}{1.182202in}}{\pgfqpoint{1.701662in}{1.185474in}}{\pgfqpoint{1.693426in}{1.185474in}}%
\pgfpathcurveto{\pgfqpoint{1.685189in}{1.185474in}}{\pgfqpoint{1.677289in}{1.182202in}}{\pgfqpoint{1.671465in}{1.176378in}}%
\pgfpathcurveto{\pgfqpoint{1.665641in}{1.170554in}}{\pgfqpoint{1.662369in}{1.162654in}}{\pgfqpoint{1.662369in}{1.154417in}}%
\pgfpathcurveto{\pgfqpoint{1.662369in}{1.146181in}}{\pgfqpoint{1.665641in}{1.138281in}}{\pgfqpoint{1.671465in}{1.132457in}}%
\pgfpathcurveto{\pgfqpoint{1.677289in}{1.126633in}}{\pgfqpoint{1.685189in}{1.123361in}}{\pgfqpoint{1.693426in}{1.123361in}}%
\pgfpathclose%
\pgfusepath{stroke,fill}%
\end{pgfscope}%
\begin{pgfscope}%
\pgfpathrectangle{\pgfqpoint{0.100000in}{0.212622in}}{\pgfqpoint{3.696000in}{3.696000in}}%
\pgfusepath{clip}%
\pgfsetbuttcap%
\pgfsetroundjoin%
\definecolor{currentfill}{rgb}{0.121569,0.466667,0.705882}%
\pgfsetfillcolor{currentfill}%
\pgfsetfillopacity{0.897872}%
\pgfsetlinewidth{1.003750pt}%
\definecolor{currentstroke}{rgb}{0.121569,0.466667,0.705882}%
\pgfsetstrokecolor{currentstroke}%
\pgfsetstrokeopacity{0.897872}%
\pgfsetdash{}{0pt}%
\pgfpathmoveto{\pgfqpoint{1.763196in}{1.999344in}}%
\pgfpathcurveto{\pgfqpoint{1.771432in}{1.999344in}}{\pgfqpoint{1.779332in}{2.002616in}}{\pgfqpoint{1.785156in}{2.008440in}}%
\pgfpathcurveto{\pgfqpoint{1.790980in}{2.014264in}}{\pgfqpoint{1.794253in}{2.022164in}}{\pgfqpoint{1.794253in}{2.030400in}}%
\pgfpathcurveto{\pgfqpoint{1.794253in}{2.038637in}}{\pgfqpoint{1.790980in}{2.046537in}}{\pgfqpoint{1.785156in}{2.052361in}}%
\pgfpathcurveto{\pgfqpoint{1.779332in}{2.058185in}}{\pgfqpoint{1.771432in}{2.061457in}}{\pgfqpoint{1.763196in}{2.061457in}}%
\pgfpathcurveto{\pgfqpoint{1.754960in}{2.061457in}}{\pgfqpoint{1.747060in}{2.058185in}}{\pgfqpoint{1.741236in}{2.052361in}}%
\pgfpathcurveto{\pgfqpoint{1.735412in}{2.046537in}}{\pgfqpoint{1.732140in}{2.038637in}}{\pgfqpoint{1.732140in}{2.030400in}}%
\pgfpathcurveto{\pgfqpoint{1.732140in}{2.022164in}}{\pgfqpoint{1.735412in}{2.014264in}}{\pgfqpoint{1.741236in}{2.008440in}}%
\pgfpathcurveto{\pgfqpoint{1.747060in}{2.002616in}}{\pgfqpoint{1.754960in}{1.999344in}}{\pgfqpoint{1.763196in}{1.999344in}}%
\pgfpathclose%
\pgfusepath{stroke,fill}%
\end{pgfscope}%
\begin{pgfscope}%
\pgfpathrectangle{\pgfqpoint{0.100000in}{0.212622in}}{\pgfqpoint{3.696000in}{3.696000in}}%
\pgfusepath{clip}%
\pgfsetbuttcap%
\pgfsetroundjoin%
\definecolor{currentfill}{rgb}{0.121569,0.466667,0.705882}%
\pgfsetfillcolor{currentfill}%
\pgfsetfillopacity{0.898331}%
\pgfsetlinewidth{1.003750pt}%
\definecolor{currentstroke}{rgb}{0.121569,0.466667,0.705882}%
\pgfsetstrokecolor{currentstroke}%
\pgfsetstrokeopacity{0.898331}%
\pgfsetdash{}{0pt}%
\pgfpathmoveto{\pgfqpoint{2.211266in}{2.143319in}}%
\pgfpathcurveto{\pgfqpoint{2.219503in}{2.143319in}}{\pgfqpoint{2.227403in}{2.146591in}}{\pgfqpoint{2.233227in}{2.152415in}}%
\pgfpathcurveto{\pgfqpoint{2.239050in}{2.158239in}}{\pgfqpoint{2.242323in}{2.166139in}}{\pgfqpoint{2.242323in}{2.174375in}}%
\pgfpathcurveto{\pgfqpoint{2.242323in}{2.182612in}}{\pgfqpoint{2.239050in}{2.190512in}}{\pgfqpoint{2.233227in}{2.196336in}}%
\pgfpathcurveto{\pgfqpoint{2.227403in}{2.202160in}}{\pgfqpoint{2.219503in}{2.205432in}}{\pgfqpoint{2.211266in}{2.205432in}}%
\pgfpathcurveto{\pgfqpoint{2.203030in}{2.205432in}}{\pgfqpoint{2.195130in}{2.202160in}}{\pgfqpoint{2.189306in}{2.196336in}}%
\pgfpathcurveto{\pgfqpoint{2.183482in}{2.190512in}}{\pgfqpoint{2.180210in}{2.182612in}}{\pgfqpoint{2.180210in}{2.174375in}}%
\pgfpathcurveto{\pgfqpoint{2.180210in}{2.166139in}}{\pgfqpoint{2.183482in}{2.158239in}}{\pgfqpoint{2.189306in}{2.152415in}}%
\pgfpathcurveto{\pgfqpoint{2.195130in}{2.146591in}}{\pgfqpoint{2.203030in}{2.143319in}}{\pgfqpoint{2.211266in}{2.143319in}}%
\pgfpathclose%
\pgfusepath{stroke,fill}%
\end{pgfscope}%
\begin{pgfscope}%
\pgfpathrectangle{\pgfqpoint{0.100000in}{0.212622in}}{\pgfqpoint{3.696000in}{3.696000in}}%
\pgfusepath{clip}%
\pgfsetbuttcap%
\pgfsetroundjoin%
\definecolor{currentfill}{rgb}{0.121569,0.466667,0.705882}%
\pgfsetfillcolor{currentfill}%
\pgfsetfillopacity{0.898469}%
\pgfsetlinewidth{1.003750pt}%
\definecolor{currentstroke}{rgb}{0.121569,0.466667,0.705882}%
\pgfsetstrokecolor{currentstroke}%
\pgfsetstrokeopacity{0.898469}%
\pgfsetdash{}{0pt}%
\pgfpathmoveto{\pgfqpoint{1.767171in}{2.000349in}}%
\pgfpathcurveto{\pgfqpoint{1.775407in}{2.000349in}}{\pgfqpoint{1.783307in}{2.003621in}}{\pgfqpoint{1.789131in}{2.009445in}}%
\pgfpathcurveto{\pgfqpoint{1.794955in}{2.015269in}}{\pgfqpoint{1.798227in}{2.023169in}}{\pgfqpoint{1.798227in}{2.031405in}}%
\pgfpathcurveto{\pgfqpoint{1.798227in}{2.039642in}}{\pgfqpoint{1.794955in}{2.047542in}}{\pgfqpoint{1.789131in}{2.053366in}}%
\pgfpathcurveto{\pgfqpoint{1.783307in}{2.059190in}}{\pgfqpoint{1.775407in}{2.062462in}}{\pgfqpoint{1.767171in}{2.062462in}}%
\pgfpathcurveto{\pgfqpoint{1.758934in}{2.062462in}}{\pgfqpoint{1.751034in}{2.059190in}}{\pgfqpoint{1.745210in}{2.053366in}}%
\pgfpathcurveto{\pgfqpoint{1.739386in}{2.047542in}}{\pgfqpoint{1.736114in}{2.039642in}}{\pgfqpoint{1.736114in}{2.031405in}}%
\pgfpathcurveto{\pgfqpoint{1.736114in}{2.023169in}}{\pgfqpoint{1.739386in}{2.015269in}}{\pgfqpoint{1.745210in}{2.009445in}}%
\pgfpathcurveto{\pgfqpoint{1.751034in}{2.003621in}}{\pgfqpoint{1.758934in}{2.000349in}}{\pgfqpoint{1.767171in}{2.000349in}}%
\pgfpathclose%
\pgfusepath{stroke,fill}%
\end{pgfscope}%
\begin{pgfscope}%
\pgfpathrectangle{\pgfqpoint{0.100000in}{0.212622in}}{\pgfqpoint{3.696000in}{3.696000in}}%
\pgfusepath{clip}%
\pgfsetbuttcap%
\pgfsetroundjoin%
\definecolor{currentfill}{rgb}{0.121569,0.466667,0.705882}%
\pgfsetfillcolor{currentfill}%
\pgfsetfillopacity{0.898880}%
\pgfsetlinewidth{1.003750pt}%
\definecolor{currentstroke}{rgb}{0.121569,0.466667,0.705882}%
\pgfsetstrokecolor{currentstroke}%
\pgfsetstrokeopacity{0.898880}%
\pgfsetdash{}{0pt}%
\pgfpathmoveto{\pgfqpoint{1.769034in}{1.999917in}}%
\pgfpathcurveto{\pgfqpoint{1.777271in}{1.999917in}}{\pgfqpoint{1.785171in}{2.003189in}}{\pgfqpoint{1.790995in}{2.009013in}}%
\pgfpathcurveto{\pgfqpoint{1.796818in}{2.014837in}}{\pgfqpoint{1.800091in}{2.022737in}}{\pgfqpoint{1.800091in}{2.030973in}}%
\pgfpathcurveto{\pgfqpoint{1.800091in}{2.039209in}}{\pgfqpoint{1.796818in}{2.047109in}}{\pgfqpoint{1.790995in}{2.052933in}}%
\pgfpathcurveto{\pgfqpoint{1.785171in}{2.058757in}}{\pgfqpoint{1.777271in}{2.062030in}}{\pgfqpoint{1.769034in}{2.062030in}}%
\pgfpathcurveto{\pgfqpoint{1.760798in}{2.062030in}}{\pgfqpoint{1.752898in}{2.058757in}}{\pgfqpoint{1.747074in}{2.052933in}}%
\pgfpathcurveto{\pgfqpoint{1.741250in}{2.047109in}}{\pgfqpoint{1.737978in}{2.039209in}}{\pgfqpoint{1.737978in}{2.030973in}}%
\pgfpathcurveto{\pgfqpoint{1.737978in}{2.022737in}}{\pgfqpoint{1.741250in}{2.014837in}}{\pgfqpoint{1.747074in}{2.009013in}}%
\pgfpathcurveto{\pgfqpoint{1.752898in}{2.003189in}}{\pgfqpoint{1.760798in}{1.999917in}}{\pgfqpoint{1.769034in}{1.999917in}}%
\pgfpathclose%
\pgfusepath{stroke,fill}%
\end{pgfscope}%
\begin{pgfscope}%
\pgfpathrectangle{\pgfqpoint{0.100000in}{0.212622in}}{\pgfqpoint{3.696000in}{3.696000in}}%
\pgfusepath{clip}%
\pgfsetbuttcap%
\pgfsetroundjoin%
\definecolor{currentfill}{rgb}{0.121569,0.466667,0.705882}%
\pgfsetfillcolor{currentfill}%
\pgfsetfillopacity{0.899027}%
\pgfsetlinewidth{1.003750pt}%
\definecolor{currentstroke}{rgb}{0.121569,0.466667,0.705882}%
\pgfsetstrokecolor{currentstroke}%
\pgfsetstrokeopacity{0.899027}%
\pgfsetdash{}{0pt}%
\pgfpathmoveto{\pgfqpoint{2.209859in}{2.140381in}}%
\pgfpathcurveto{\pgfqpoint{2.218096in}{2.140381in}}{\pgfqpoint{2.225996in}{2.143653in}}{\pgfqpoint{2.231820in}{2.149477in}}%
\pgfpathcurveto{\pgfqpoint{2.237644in}{2.155301in}}{\pgfqpoint{2.240916in}{2.163201in}}{\pgfqpoint{2.240916in}{2.171437in}}%
\pgfpathcurveto{\pgfqpoint{2.240916in}{2.179673in}}{\pgfqpoint{2.237644in}{2.187573in}}{\pgfqpoint{2.231820in}{2.193397in}}%
\pgfpathcurveto{\pgfqpoint{2.225996in}{2.199221in}}{\pgfqpoint{2.218096in}{2.202494in}}{\pgfqpoint{2.209859in}{2.202494in}}%
\pgfpathcurveto{\pgfqpoint{2.201623in}{2.202494in}}{\pgfqpoint{2.193723in}{2.199221in}}{\pgfqpoint{2.187899in}{2.193397in}}%
\pgfpathcurveto{\pgfqpoint{2.182075in}{2.187573in}}{\pgfqpoint{2.178803in}{2.179673in}}{\pgfqpoint{2.178803in}{2.171437in}}%
\pgfpathcurveto{\pgfqpoint{2.178803in}{2.163201in}}{\pgfqpoint{2.182075in}{2.155301in}}{\pgfqpoint{2.187899in}{2.149477in}}%
\pgfpathcurveto{\pgfqpoint{2.193723in}{2.143653in}}{\pgfqpoint{2.201623in}{2.140381in}}{\pgfqpoint{2.209859in}{2.140381in}}%
\pgfpathclose%
\pgfusepath{stroke,fill}%
\end{pgfscope}%
\begin{pgfscope}%
\pgfpathrectangle{\pgfqpoint{0.100000in}{0.212622in}}{\pgfqpoint{3.696000in}{3.696000in}}%
\pgfusepath{clip}%
\pgfsetbuttcap%
\pgfsetroundjoin%
\definecolor{currentfill}{rgb}{0.121569,0.466667,0.705882}%
\pgfsetfillcolor{currentfill}%
\pgfsetfillopacity{0.899069}%
\pgfsetlinewidth{1.003750pt}%
\definecolor{currentstroke}{rgb}{0.121569,0.466667,0.705882}%
\pgfsetstrokecolor{currentstroke}%
\pgfsetstrokeopacity{0.899069}%
\pgfsetdash{}{0pt}%
\pgfpathmoveto{\pgfqpoint{1.770030in}{1.999415in}}%
\pgfpathcurveto{\pgfqpoint{1.778267in}{1.999415in}}{\pgfqpoint{1.786167in}{2.002688in}}{\pgfqpoint{1.791991in}{2.008512in}}%
\pgfpathcurveto{\pgfqpoint{1.797814in}{2.014336in}}{\pgfqpoint{1.801087in}{2.022236in}}{\pgfqpoint{1.801087in}{2.030472in}}%
\pgfpathcurveto{\pgfqpoint{1.801087in}{2.038708in}}{\pgfqpoint{1.797814in}{2.046608in}}{\pgfqpoint{1.791991in}{2.052432in}}%
\pgfpathcurveto{\pgfqpoint{1.786167in}{2.058256in}}{\pgfqpoint{1.778267in}{2.061528in}}{\pgfqpoint{1.770030in}{2.061528in}}%
\pgfpathcurveto{\pgfqpoint{1.761794in}{2.061528in}}{\pgfqpoint{1.753894in}{2.058256in}}{\pgfqpoint{1.748070in}{2.052432in}}%
\pgfpathcurveto{\pgfqpoint{1.742246in}{2.046608in}}{\pgfqpoint{1.738974in}{2.038708in}}{\pgfqpoint{1.738974in}{2.030472in}}%
\pgfpathcurveto{\pgfqpoint{1.738974in}{2.022236in}}{\pgfqpoint{1.742246in}{2.014336in}}{\pgfqpoint{1.748070in}{2.008512in}}%
\pgfpathcurveto{\pgfqpoint{1.753894in}{2.002688in}}{\pgfqpoint{1.761794in}{1.999415in}}{\pgfqpoint{1.770030in}{1.999415in}}%
\pgfpathclose%
\pgfusepath{stroke,fill}%
\end{pgfscope}%
\begin{pgfscope}%
\pgfpathrectangle{\pgfqpoint{0.100000in}{0.212622in}}{\pgfqpoint{3.696000in}{3.696000in}}%
\pgfusepath{clip}%
\pgfsetbuttcap%
\pgfsetroundjoin%
\definecolor{currentfill}{rgb}{0.121569,0.466667,0.705882}%
\pgfsetfillcolor{currentfill}%
\pgfsetfillopacity{0.899202}%
\pgfsetlinewidth{1.003750pt}%
\definecolor{currentstroke}{rgb}{0.121569,0.466667,0.705882}%
\pgfsetstrokecolor{currentstroke}%
\pgfsetstrokeopacity{0.899202}%
\pgfsetdash{}{0pt}%
\pgfpathmoveto{\pgfqpoint{1.770475in}{1.998985in}}%
\pgfpathcurveto{\pgfqpoint{1.778711in}{1.998985in}}{\pgfqpoint{1.786611in}{2.002257in}}{\pgfqpoint{1.792435in}{2.008081in}}%
\pgfpathcurveto{\pgfqpoint{1.798259in}{2.013905in}}{\pgfqpoint{1.801532in}{2.021805in}}{\pgfqpoint{1.801532in}{2.030041in}}%
\pgfpathcurveto{\pgfqpoint{1.801532in}{2.038277in}}{\pgfqpoint{1.798259in}{2.046177in}}{\pgfqpoint{1.792435in}{2.052001in}}%
\pgfpathcurveto{\pgfqpoint{1.786611in}{2.057825in}}{\pgfqpoint{1.778711in}{2.061098in}}{\pgfqpoint{1.770475in}{2.061098in}}%
\pgfpathcurveto{\pgfqpoint{1.762239in}{2.061098in}}{\pgfqpoint{1.754339in}{2.057825in}}{\pgfqpoint{1.748515in}{2.052001in}}%
\pgfpathcurveto{\pgfqpoint{1.742691in}{2.046177in}}{\pgfqpoint{1.739419in}{2.038277in}}{\pgfqpoint{1.739419in}{2.030041in}}%
\pgfpathcurveto{\pgfqpoint{1.739419in}{2.021805in}}{\pgfqpoint{1.742691in}{2.013905in}}{\pgfqpoint{1.748515in}{2.008081in}}%
\pgfpathcurveto{\pgfqpoint{1.754339in}{2.002257in}}{\pgfqpoint{1.762239in}{1.998985in}}{\pgfqpoint{1.770475in}{1.998985in}}%
\pgfpathclose%
\pgfusepath{stroke,fill}%
\end{pgfscope}%
\begin{pgfscope}%
\pgfpathrectangle{\pgfqpoint{0.100000in}{0.212622in}}{\pgfqpoint{3.696000in}{3.696000in}}%
\pgfusepath{clip}%
\pgfsetbuttcap%
\pgfsetroundjoin%
\definecolor{currentfill}{rgb}{0.121569,0.466667,0.705882}%
\pgfsetfillcolor{currentfill}%
\pgfsetfillopacity{0.899221}%
\pgfsetlinewidth{1.003750pt}%
\definecolor{currentstroke}{rgb}{0.121569,0.466667,0.705882}%
\pgfsetstrokecolor{currentstroke}%
\pgfsetstrokeopacity{0.899221}%
\pgfsetdash{}{0pt}%
\pgfpathmoveto{\pgfqpoint{2.209002in}{2.139419in}}%
\pgfpathcurveto{\pgfqpoint{2.217238in}{2.139419in}}{\pgfqpoint{2.225138in}{2.142691in}}{\pgfqpoint{2.230962in}{2.148515in}}%
\pgfpathcurveto{\pgfqpoint{2.236786in}{2.154339in}}{\pgfqpoint{2.240058in}{2.162239in}}{\pgfqpoint{2.240058in}{2.170476in}}%
\pgfpathcurveto{\pgfqpoint{2.240058in}{2.178712in}}{\pgfqpoint{2.236786in}{2.186612in}}{\pgfqpoint{2.230962in}{2.192436in}}%
\pgfpathcurveto{\pgfqpoint{2.225138in}{2.198260in}}{\pgfqpoint{2.217238in}{2.201532in}}{\pgfqpoint{2.209002in}{2.201532in}}%
\pgfpathcurveto{\pgfqpoint{2.200765in}{2.201532in}}{\pgfqpoint{2.192865in}{2.198260in}}{\pgfqpoint{2.187041in}{2.192436in}}%
\pgfpathcurveto{\pgfqpoint{2.181217in}{2.186612in}}{\pgfqpoint{2.177945in}{2.178712in}}{\pgfqpoint{2.177945in}{2.170476in}}%
\pgfpathcurveto{\pgfqpoint{2.177945in}{2.162239in}}{\pgfqpoint{2.181217in}{2.154339in}}{\pgfqpoint{2.187041in}{2.148515in}}%
\pgfpathcurveto{\pgfqpoint{2.192865in}{2.142691in}}{\pgfqpoint{2.200765in}{2.139419in}}{\pgfqpoint{2.209002in}{2.139419in}}%
\pgfpathclose%
\pgfusepath{stroke,fill}%
\end{pgfscope}%
\begin{pgfscope}%
\pgfpathrectangle{\pgfqpoint{0.100000in}{0.212622in}}{\pgfqpoint{3.696000in}{3.696000in}}%
\pgfusepath{clip}%
\pgfsetbuttcap%
\pgfsetroundjoin%
\definecolor{currentfill}{rgb}{0.121569,0.466667,0.705882}%
\pgfsetfillcolor{currentfill}%
\pgfsetfillopacity{0.899451}%
\pgfsetlinewidth{1.003750pt}%
\definecolor{currentstroke}{rgb}{0.121569,0.466667,0.705882}%
\pgfsetstrokecolor{currentstroke}%
\pgfsetstrokeopacity{0.899451}%
\pgfsetdash{}{0pt}%
\pgfpathmoveto{\pgfqpoint{1.771411in}{1.998028in}}%
\pgfpathcurveto{\pgfqpoint{1.779647in}{1.998028in}}{\pgfqpoint{1.787547in}{2.001300in}}{\pgfqpoint{1.793371in}{2.007124in}}%
\pgfpathcurveto{\pgfqpoint{1.799195in}{2.012948in}}{\pgfqpoint{1.802467in}{2.020848in}}{\pgfqpoint{1.802467in}{2.029085in}}%
\pgfpathcurveto{\pgfqpoint{1.802467in}{2.037321in}}{\pgfqpoint{1.799195in}{2.045221in}}{\pgfqpoint{1.793371in}{2.051045in}}%
\pgfpathcurveto{\pgfqpoint{1.787547in}{2.056869in}}{\pgfqpoint{1.779647in}{2.060141in}}{\pgfqpoint{1.771411in}{2.060141in}}%
\pgfpathcurveto{\pgfqpoint{1.763175in}{2.060141in}}{\pgfqpoint{1.755274in}{2.056869in}}{\pgfqpoint{1.749451in}{2.051045in}}%
\pgfpathcurveto{\pgfqpoint{1.743627in}{2.045221in}}{\pgfqpoint{1.740354in}{2.037321in}}{\pgfqpoint{1.740354in}{2.029085in}}%
\pgfpathcurveto{\pgfqpoint{1.740354in}{2.020848in}}{\pgfqpoint{1.743627in}{2.012948in}}{\pgfqpoint{1.749451in}{2.007124in}}%
\pgfpathcurveto{\pgfqpoint{1.755274in}{2.001300in}}{\pgfqpoint{1.763175in}{1.998028in}}{\pgfqpoint{1.771411in}{1.998028in}}%
\pgfpathclose%
\pgfusepath{stroke,fill}%
\end{pgfscope}%
\begin{pgfscope}%
\pgfpathrectangle{\pgfqpoint{0.100000in}{0.212622in}}{\pgfqpoint{3.696000in}{3.696000in}}%
\pgfusepath{clip}%
\pgfsetbuttcap%
\pgfsetroundjoin%
\definecolor{currentfill}{rgb}{0.121569,0.466667,0.705882}%
\pgfsetfillcolor{currentfill}%
\pgfsetfillopacity{0.899596}%
\pgfsetlinewidth{1.003750pt}%
\definecolor{currentstroke}{rgb}{0.121569,0.466667,0.705882}%
\pgfsetstrokecolor{currentstroke}%
\pgfsetstrokeopacity{0.899596}%
\pgfsetdash{}{0pt}%
\pgfpathmoveto{\pgfqpoint{2.792666in}{1.376850in}}%
\pgfpathcurveto{\pgfqpoint{2.800902in}{1.376850in}}{\pgfqpoint{2.808802in}{1.380123in}}{\pgfqpoint{2.814626in}{1.385947in}}%
\pgfpathcurveto{\pgfqpoint{2.820450in}{1.391770in}}{\pgfqpoint{2.823722in}{1.399671in}}{\pgfqpoint{2.823722in}{1.407907in}}%
\pgfpathcurveto{\pgfqpoint{2.823722in}{1.416143in}}{\pgfqpoint{2.820450in}{1.424043in}}{\pgfqpoint{2.814626in}{1.429867in}}%
\pgfpathcurveto{\pgfqpoint{2.808802in}{1.435691in}}{\pgfqpoint{2.800902in}{1.438963in}}{\pgfqpoint{2.792666in}{1.438963in}}%
\pgfpathcurveto{\pgfqpoint{2.784429in}{1.438963in}}{\pgfqpoint{2.776529in}{1.435691in}}{\pgfqpoint{2.770705in}{1.429867in}}%
\pgfpathcurveto{\pgfqpoint{2.764882in}{1.424043in}}{\pgfqpoint{2.761609in}{1.416143in}}{\pgfqpoint{2.761609in}{1.407907in}}%
\pgfpathcurveto{\pgfqpoint{2.761609in}{1.399671in}}{\pgfqpoint{2.764882in}{1.391770in}}{\pgfqpoint{2.770705in}{1.385947in}}%
\pgfpathcurveto{\pgfqpoint{2.776529in}{1.380123in}}{\pgfqpoint{2.784429in}{1.376850in}}{\pgfqpoint{2.792666in}{1.376850in}}%
\pgfpathclose%
\pgfusepath{stroke,fill}%
\end{pgfscope}%
\begin{pgfscope}%
\pgfpathrectangle{\pgfqpoint{0.100000in}{0.212622in}}{\pgfqpoint{3.696000in}{3.696000in}}%
\pgfusepath{clip}%
\pgfsetbuttcap%
\pgfsetroundjoin%
\definecolor{currentfill}{rgb}{0.121569,0.466667,0.705882}%
\pgfsetfillcolor{currentfill}%
\pgfsetfillopacity{0.899596}%
\pgfsetlinewidth{1.003750pt}%
\definecolor{currentstroke}{rgb}{0.121569,0.466667,0.705882}%
\pgfsetstrokecolor{currentstroke}%
\pgfsetstrokeopacity{0.899596}%
\pgfsetdash{}{0pt}%
\pgfpathmoveto{\pgfqpoint{2.207557in}{2.137490in}}%
\pgfpathcurveto{\pgfqpoint{2.215793in}{2.137490in}}{\pgfqpoint{2.223693in}{2.140763in}}{\pgfqpoint{2.229517in}{2.146586in}}%
\pgfpathcurveto{\pgfqpoint{2.235341in}{2.152410in}}{\pgfqpoint{2.238614in}{2.160310in}}{\pgfqpoint{2.238614in}{2.168547in}}%
\pgfpathcurveto{\pgfqpoint{2.238614in}{2.176783in}}{\pgfqpoint{2.235341in}{2.184683in}}{\pgfqpoint{2.229517in}{2.190507in}}%
\pgfpathcurveto{\pgfqpoint{2.223693in}{2.196331in}}{\pgfqpoint{2.215793in}{2.199603in}}{\pgfqpoint{2.207557in}{2.199603in}}%
\pgfpathcurveto{\pgfqpoint{2.199321in}{2.199603in}}{\pgfqpoint{2.191421in}{2.196331in}}{\pgfqpoint{2.185597in}{2.190507in}}%
\pgfpathcurveto{\pgfqpoint{2.179773in}{2.184683in}}{\pgfqpoint{2.176501in}{2.176783in}}{\pgfqpoint{2.176501in}{2.168547in}}%
\pgfpathcurveto{\pgfqpoint{2.176501in}{2.160310in}}{\pgfqpoint{2.179773in}{2.152410in}}{\pgfqpoint{2.185597in}{2.146586in}}%
\pgfpathcurveto{\pgfqpoint{2.191421in}{2.140763in}}{\pgfqpoint{2.199321in}{2.137490in}}{\pgfqpoint{2.207557in}{2.137490in}}%
\pgfpathclose%
\pgfusepath{stroke,fill}%
\end{pgfscope}%
\begin{pgfscope}%
\pgfpathrectangle{\pgfqpoint{0.100000in}{0.212622in}}{\pgfqpoint{3.696000in}{3.696000in}}%
\pgfusepath{clip}%
\pgfsetbuttcap%
\pgfsetroundjoin%
\definecolor{currentfill}{rgb}{0.121569,0.466667,0.705882}%
\pgfsetfillcolor{currentfill}%
\pgfsetfillopacity{0.899755}%
\pgfsetlinewidth{1.003750pt}%
\definecolor{currentstroke}{rgb}{0.121569,0.466667,0.705882}%
\pgfsetstrokecolor{currentstroke}%
\pgfsetstrokeopacity{0.899755}%
\pgfsetdash{}{0pt}%
\pgfpathmoveto{\pgfqpoint{1.772805in}{1.996993in}}%
\pgfpathcurveto{\pgfqpoint{1.781042in}{1.996993in}}{\pgfqpoint{1.788942in}{2.000266in}}{\pgfqpoint{1.794766in}{2.006090in}}%
\pgfpathcurveto{\pgfqpoint{1.800590in}{2.011914in}}{\pgfqpoint{1.803862in}{2.019814in}}{\pgfqpoint{1.803862in}{2.028050in}}%
\pgfpathcurveto{\pgfqpoint{1.803862in}{2.036286in}}{\pgfqpoint{1.800590in}{2.044186in}}{\pgfqpoint{1.794766in}{2.050010in}}%
\pgfpathcurveto{\pgfqpoint{1.788942in}{2.055834in}}{\pgfqpoint{1.781042in}{2.059106in}}{\pgfqpoint{1.772805in}{2.059106in}}%
\pgfpathcurveto{\pgfqpoint{1.764569in}{2.059106in}}{\pgfqpoint{1.756669in}{2.055834in}}{\pgfqpoint{1.750845in}{2.050010in}}%
\pgfpathcurveto{\pgfqpoint{1.745021in}{2.044186in}}{\pgfqpoint{1.741749in}{2.036286in}}{\pgfqpoint{1.741749in}{2.028050in}}%
\pgfpathcurveto{\pgfqpoint{1.741749in}{2.019814in}}{\pgfqpoint{1.745021in}{2.011914in}}{\pgfqpoint{1.750845in}{2.006090in}}%
\pgfpathcurveto{\pgfqpoint{1.756669in}{2.000266in}}{\pgfqpoint{1.764569in}{1.996993in}}{\pgfqpoint{1.772805in}{1.996993in}}%
\pgfpathclose%
\pgfusepath{stroke,fill}%
\end{pgfscope}%
\begin{pgfscope}%
\pgfpathrectangle{\pgfqpoint{0.100000in}{0.212622in}}{\pgfqpoint{3.696000in}{3.696000in}}%
\pgfusepath{clip}%
\pgfsetbuttcap%
\pgfsetroundjoin%
\definecolor{currentfill}{rgb}{0.121569,0.466667,0.705882}%
\pgfsetfillcolor{currentfill}%
\pgfsetfillopacity{0.899854}%
\pgfsetlinewidth{1.003750pt}%
\definecolor{currentstroke}{rgb}{0.121569,0.466667,0.705882}%
\pgfsetstrokecolor{currentstroke}%
\pgfsetstrokeopacity{0.899854}%
\pgfsetdash{}{0pt}%
\pgfpathmoveto{\pgfqpoint{2.207107in}{2.136239in}}%
\pgfpathcurveto{\pgfqpoint{2.215343in}{2.136239in}}{\pgfqpoint{2.223243in}{2.139512in}}{\pgfqpoint{2.229067in}{2.145336in}}%
\pgfpathcurveto{\pgfqpoint{2.234891in}{2.151160in}}{\pgfqpoint{2.238164in}{2.159060in}}{\pgfqpoint{2.238164in}{2.167296in}}%
\pgfpathcurveto{\pgfqpoint{2.238164in}{2.175532in}}{\pgfqpoint{2.234891in}{2.183432in}}{\pgfqpoint{2.229067in}{2.189256in}}%
\pgfpathcurveto{\pgfqpoint{2.223243in}{2.195080in}}{\pgfqpoint{2.215343in}{2.198352in}}{\pgfqpoint{2.207107in}{2.198352in}}%
\pgfpathcurveto{\pgfqpoint{2.198871in}{2.198352in}}{\pgfqpoint{2.190971in}{2.195080in}}{\pgfqpoint{2.185147in}{2.189256in}}%
\pgfpathcurveto{\pgfqpoint{2.179323in}{2.183432in}}{\pgfqpoint{2.176051in}{2.175532in}}{\pgfqpoint{2.176051in}{2.167296in}}%
\pgfpathcurveto{\pgfqpoint{2.176051in}{2.159060in}}{\pgfqpoint{2.179323in}{2.151160in}}{\pgfqpoint{2.185147in}{2.145336in}}%
\pgfpathcurveto{\pgfqpoint{2.190971in}{2.139512in}}{\pgfqpoint{2.198871in}{2.136239in}}{\pgfqpoint{2.207107in}{2.136239in}}%
\pgfpathclose%
\pgfusepath{stroke,fill}%
\end{pgfscope}%
\begin{pgfscope}%
\pgfpathrectangle{\pgfqpoint{0.100000in}{0.212622in}}{\pgfqpoint{3.696000in}{3.696000in}}%
\pgfusepath{clip}%
\pgfsetbuttcap%
\pgfsetroundjoin%
\definecolor{currentfill}{rgb}{0.121569,0.466667,0.705882}%
\pgfsetfillcolor{currentfill}%
\pgfsetfillopacity{0.900091}%
\pgfsetlinewidth{1.003750pt}%
\definecolor{currentstroke}{rgb}{0.121569,0.466667,0.705882}%
\pgfsetstrokecolor{currentstroke}%
\pgfsetstrokeopacity{0.900091}%
\pgfsetdash{}{0pt}%
\pgfpathmoveto{\pgfqpoint{1.774881in}{1.996304in}}%
\pgfpathcurveto{\pgfqpoint{1.783118in}{1.996304in}}{\pgfqpoint{1.791018in}{1.999577in}}{\pgfqpoint{1.796842in}{2.005401in}}%
\pgfpathcurveto{\pgfqpoint{1.802666in}{2.011225in}}{\pgfqpoint{1.805938in}{2.019125in}}{\pgfqpoint{1.805938in}{2.027361in}}%
\pgfpathcurveto{\pgfqpoint{1.805938in}{2.035597in}}{\pgfqpoint{1.802666in}{2.043497in}}{\pgfqpoint{1.796842in}{2.049321in}}%
\pgfpathcurveto{\pgfqpoint{1.791018in}{2.055145in}}{\pgfqpoint{1.783118in}{2.058417in}}{\pgfqpoint{1.774881in}{2.058417in}}%
\pgfpathcurveto{\pgfqpoint{1.766645in}{2.058417in}}{\pgfqpoint{1.758745in}{2.055145in}}{\pgfqpoint{1.752921in}{2.049321in}}%
\pgfpathcurveto{\pgfqpoint{1.747097in}{2.043497in}}{\pgfqpoint{1.743825in}{2.035597in}}{\pgfqpoint{1.743825in}{2.027361in}}%
\pgfpathcurveto{\pgfqpoint{1.743825in}{2.019125in}}{\pgfqpoint{1.747097in}{2.011225in}}{\pgfqpoint{1.752921in}{2.005401in}}%
\pgfpathcurveto{\pgfqpoint{1.758745in}{1.999577in}}{\pgfqpoint{1.766645in}{1.996304in}}{\pgfqpoint{1.774881in}{1.996304in}}%
\pgfpathclose%
\pgfusepath{stroke,fill}%
\end{pgfscope}%
\begin{pgfscope}%
\pgfpathrectangle{\pgfqpoint{0.100000in}{0.212622in}}{\pgfqpoint{3.696000in}{3.696000in}}%
\pgfusepath{clip}%
\pgfsetbuttcap%
\pgfsetroundjoin%
\definecolor{currentfill}{rgb}{0.121569,0.466667,0.705882}%
\pgfsetfillcolor{currentfill}%
\pgfsetfillopacity{0.900313}%
\pgfsetlinewidth{1.003750pt}%
\definecolor{currentstroke}{rgb}{0.121569,0.466667,0.705882}%
\pgfsetstrokecolor{currentstroke}%
\pgfsetstrokeopacity{0.900313}%
\pgfsetdash{}{0pt}%
\pgfpathmoveto{\pgfqpoint{2.206253in}{2.133943in}}%
\pgfpathcurveto{\pgfqpoint{2.214489in}{2.133943in}}{\pgfqpoint{2.222389in}{2.137215in}}{\pgfqpoint{2.228213in}{2.143039in}}%
\pgfpathcurveto{\pgfqpoint{2.234037in}{2.148863in}}{\pgfqpoint{2.237309in}{2.156763in}}{\pgfqpoint{2.237309in}{2.164999in}}%
\pgfpathcurveto{\pgfqpoint{2.237309in}{2.173235in}}{\pgfqpoint{2.234037in}{2.181135in}}{\pgfqpoint{2.228213in}{2.186959in}}%
\pgfpathcurveto{\pgfqpoint{2.222389in}{2.192783in}}{\pgfqpoint{2.214489in}{2.196056in}}{\pgfqpoint{2.206253in}{2.196056in}}%
\pgfpathcurveto{\pgfqpoint{2.198017in}{2.196056in}}{\pgfqpoint{2.190117in}{2.192783in}}{\pgfqpoint{2.184293in}{2.186959in}}%
\pgfpathcurveto{\pgfqpoint{2.178469in}{2.181135in}}{\pgfqpoint{2.175196in}{2.173235in}}{\pgfqpoint{2.175196in}{2.164999in}}%
\pgfpathcurveto{\pgfqpoint{2.175196in}{2.156763in}}{\pgfqpoint{2.178469in}{2.148863in}}{\pgfqpoint{2.184293in}{2.143039in}}%
\pgfpathcurveto{\pgfqpoint{2.190117in}{2.137215in}}{\pgfqpoint{2.198017in}{2.133943in}}{\pgfqpoint{2.206253in}{2.133943in}}%
\pgfpathclose%
\pgfusepath{stroke,fill}%
\end{pgfscope}%
\begin{pgfscope}%
\pgfpathrectangle{\pgfqpoint{0.100000in}{0.212622in}}{\pgfqpoint{3.696000in}{3.696000in}}%
\pgfusepath{clip}%
\pgfsetbuttcap%
\pgfsetroundjoin%
\definecolor{currentfill}{rgb}{0.121569,0.466667,0.705882}%
\pgfsetfillcolor{currentfill}%
\pgfsetfillopacity{0.900538}%
\pgfsetlinewidth{1.003750pt}%
\definecolor{currentstroke}{rgb}{0.121569,0.466667,0.705882}%
\pgfsetstrokecolor{currentstroke}%
\pgfsetstrokeopacity{0.900538}%
\pgfsetdash{}{0pt}%
\pgfpathmoveto{\pgfqpoint{2.205459in}{2.132938in}}%
\pgfpathcurveto{\pgfqpoint{2.213695in}{2.132938in}}{\pgfqpoint{2.221595in}{2.136211in}}{\pgfqpoint{2.227419in}{2.142035in}}%
\pgfpathcurveto{\pgfqpoint{2.233243in}{2.147858in}}{\pgfqpoint{2.236515in}{2.155759in}}{\pgfqpoint{2.236515in}{2.163995in}}%
\pgfpathcurveto{\pgfqpoint{2.236515in}{2.172231in}}{\pgfqpoint{2.233243in}{2.180131in}}{\pgfqpoint{2.227419in}{2.185955in}}%
\pgfpathcurveto{\pgfqpoint{2.221595in}{2.191779in}}{\pgfqpoint{2.213695in}{2.195051in}}{\pgfqpoint{2.205459in}{2.195051in}}%
\pgfpathcurveto{\pgfqpoint{2.197223in}{2.195051in}}{\pgfqpoint{2.189323in}{2.191779in}}{\pgfqpoint{2.183499in}{2.185955in}}%
\pgfpathcurveto{\pgfqpoint{2.177675in}{2.180131in}}{\pgfqpoint{2.174402in}{2.172231in}}{\pgfqpoint{2.174402in}{2.163995in}}%
\pgfpathcurveto{\pgfqpoint{2.174402in}{2.155759in}}{\pgfqpoint{2.177675in}{2.147858in}}{\pgfqpoint{2.183499in}{2.142035in}}%
\pgfpathcurveto{\pgfqpoint{2.189323in}{2.136211in}}{\pgfqpoint{2.197223in}{2.132938in}}{\pgfqpoint{2.205459in}{2.132938in}}%
\pgfpathclose%
\pgfusepath{stroke,fill}%
\end{pgfscope}%
\begin{pgfscope}%
\pgfpathrectangle{\pgfqpoint{0.100000in}{0.212622in}}{\pgfqpoint{3.696000in}{3.696000in}}%
\pgfusepath{clip}%
\pgfsetbuttcap%
\pgfsetroundjoin%
\definecolor{currentfill}{rgb}{0.121569,0.466667,0.705882}%
\pgfsetfillcolor{currentfill}%
\pgfsetfillopacity{0.900547}%
\pgfsetlinewidth{1.003750pt}%
\definecolor{currentstroke}{rgb}{0.121569,0.466667,0.705882}%
\pgfsetstrokecolor{currentstroke}%
\pgfsetstrokeopacity{0.900547}%
\pgfsetdash{}{0pt}%
\pgfpathmoveto{\pgfqpoint{1.777673in}{1.995639in}}%
\pgfpathcurveto{\pgfqpoint{1.785909in}{1.995639in}}{\pgfqpoint{1.793809in}{1.998911in}}{\pgfqpoint{1.799633in}{2.004735in}}%
\pgfpathcurveto{\pgfqpoint{1.805457in}{2.010559in}}{\pgfqpoint{1.808729in}{2.018459in}}{\pgfqpoint{1.808729in}{2.026695in}}%
\pgfpathcurveto{\pgfqpoint{1.808729in}{2.034932in}}{\pgfqpoint{1.805457in}{2.042832in}}{\pgfqpoint{1.799633in}{2.048656in}}%
\pgfpathcurveto{\pgfqpoint{1.793809in}{2.054480in}}{\pgfqpoint{1.785909in}{2.057752in}}{\pgfqpoint{1.777673in}{2.057752in}}%
\pgfpathcurveto{\pgfqpoint{1.769436in}{2.057752in}}{\pgfqpoint{1.761536in}{2.054480in}}{\pgfqpoint{1.755712in}{2.048656in}}%
\pgfpathcurveto{\pgfqpoint{1.749889in}{2.042832in}}{\pgfqpoint{1.746616in}{2.034932in}}{\pgfqpoint{1.746616in}{2.026695in}}%
\pgfpathcurveto{\pgfqpoint{1.746616in}{2.018459in}}{\pgfqpoint{1.749889in}{2.010559in}}{\pgfqpoint{1.755712in}{2.004735in}}%
\pgfpathcurveto{\pgfqpoint{1.761536in}{1.998911in}}{\pgfqpoint{1.769436in}{1.995639in}}{\pgfqpoint{1.777673in}{1.995639in}}%
\pgfpathclose%
\pgfusepath{stroke,fill}%
\end{pgfscope}%
\begin{pgfscope}%
\pgfpathrectangle{\pgfqpoint{0.100000in}{0.212622in}}{\pgfqpoint{3.696000in}{3.696000in}}%
\pgfusepath{clip}%
\pgfsetbuttcap%
\pgfsetroundjoin%
\definecolor{currentfill}{rgb}{0.121569,0.466667,0.705882}%
\pgfsetfillcolor{currentfill}%
\pgfsetfillopacity{0.900696}%
\pgfsetlinewidth{1.003750pt}%
\definecolor{currentstroke}{rgb}{0.121569,0.466667,0.705882}%
\pgfsetstrokecolor{currentstroke}%
\pgfsetstrokeopacity{0.900696}%
\pgfsetdash{}{0pt}%
\pgfpathmoveto{\pgfqpoint{1.710332in}{1.125787in}}%
\pgfpathcurveto{\pgfqpoint{1.718568in}{1.125787in}}{\pgfqpoint{1.726469in}{1.129059in}}{\pgfqpoint{1.732292in}{1.134883in}}%
\pgfpathcurveto{\pgfqpoint{1.738116in}{1.140707in}}{\pgfqpoint{1.741389in}{1.148607in}}{\pgfqpoint{1.741389in}{1.156843in}}%
\pgfpathcurveto{\pgfqpoint{1.741389in}{1.165080in}}{\pgfqpoint{1.738116in}{1.172980in}}{\pgfqpoint{1.732292in}{1.178804in}}%
\pgfpathcurveto{\pgfqpoint{1.726469in}{1.184628in}}{\pgfqpoint{1.718568in}{1.187900in}}{\pgfqpoint{1.710332in}{1.187900in}}%
\pgfpathcurveto{\pgfqpoint{1.702096in}{1.187900in}}{\pgfqpoint{1.694196in}{1.184628in}}{\pgfqpoint{1.688372in}{1.178804in}}%
\pgfpathcurveto{\pgfqpoint{1.682548in}{1.172980in}}{\pgfqpoint{1.679276in}{1.165080in}}{\pgfqpoint{1.679276in}{1.156843in}}%
\pgfpathcurveto{\pgfqpoint{1.679276in}{1.148607in}}{\pgfqpoint{1.682548in}{1.140707in}}{\pgfqpoint{1.688372in}{1.134883in}}%
\pgfpathcurveto{\pgfqpoint{1.694196in}{1.129059in}}{\pgfqpoint{1.702096in}{1.125787in}}{\pgfqpoint{1.710332in}{1.125787in}}%
\pgfpathclose%
\pgfusepath{stroke,fill}%
\end{pgfscope}%
\begin{pgfscope}%
\pgfpathrectangle{\pgfqpoint{0.100000in}{0.212622in}}{\pgfqpoint{3.696000in}{3.696000in}}%
\pgfusepath{clip}%
\pgfsetbuttcap%
\pgfsetroundjoin%
\definecolor{currentfill}{rgb}{0.121569,0.466667,0.705882}%
\pgfsetfillcolor{currentfill}%
\pgfsetfillopacity{0.900891}%
\pgfsetlinewidth{1.003750pt}%
\definecolor{currentstroke}{rgb}{0.121569,0.466667,0.705882}%
\pgfsetstrokecolor{currentstroke}%
\pgfsetstrokeopacity{0.900891}%
\pgfsetdash{}{0pt}%
\pgfpathmoveto{\pgfqpoint{2.203935in}{2.131039in}}%
\pgfpathcurveto{\pgfqpoint{2.212171in}{2.131039in}}{\pgfqpoint{2.220071in}{2.134312in}}{\pgfqpoint{2.225895in}{2.140136in}}%
\pgfpathcurveto{\pgfqpoint{2.231719in}{2.145959in}}{\pgfqpoint{2.234991in}{2.153859in}}{\pgfqpoint{2.234991in}{2.162096in}}%
\pgfpathcurveto{\pgfqpoint{2.234991in}{2.170332in}}{\pgfqpoint{2.231719in}{2.178232in}}{\pgfqpoint{2.225895in}{2.184056in}}%
\pgfpathcurveto{\pgfqpoint{2.220071in}{2.189880in}}{\pgfqpoint{2.212171in}{2.193152in}}{\pgfqpoint{2.203935in}{2.193152in}}%
\pgfpathcurveto{\pgfqpoint{2.195698in}{2.193152in}}{\pgfqpoint{2.187798in}{2.189880in}}{\pgfqpoint{2.181974in}{2.184056in}}%
\pgfpathcurveto{\pgfqpoint{2.176150in}{2.178232in}}{\pgfqpoint{2.172878in}{2.170332in}}{\pgfqpoint{2.172878in}{2.162096in}}%
\pgfpathcurveto{\pgfqpoint{2.172878in}{2.153859in}}{\pgfqpoint{2.176150in}{2.145959in}}{\pgfqpoint{2.181974in}{2.140136in}}%
\pgfpathcurveto{\pgfqpoint{2.187798in}{2.134312in}}{\pgfqpoint{2.195698in}{2.131039in}}{\pgfqpoint{2.203935in}{2.131039in}}%
\pgfpathclose%
\pgfusepath{stroke,fill}%
\end{pgfscope}%
\begin{pgfscope}%
\pgfpathrectangle{\pgfqpoint{0.100000in}{0.212622in}}{\pgfqpoint{3.696000in}{3.696000in}}%
\pgfusepath{clip}%
\pgfsetbuttcap%
\pgfsetroundjoin%
\definecolor{currentfill}{rgb}{0.121569,0.466667,0.705882}%
\pgfsetfillcolor{currentfill}%
\pgfsetfillopacity{0.901168}%
\pgfsetlinewidth{1.003750pt}%
\definecolor{currentstroke}{rgb}{0.121569,0.466667,0.705882}%
\pgfsetstrokecolor{currentstroke}%
\pgfsetstrokeopacity{0.901168}%
\pgfsetdash{}{0pt}%
\pgfpathmoveto{\pgfqpoint{1.780954in}{1.994852in}}%
\pgfpathcurveto{\pgfqpoint{1.789190in}{1.994852in}}{\pgfqpoint{1.797090in}{1.998125in}}{\pgfqpoint{1.802914in}{2.003949in}}%
\pgfpathcurveto{\pgfqpoint{1.808738in}{2.009773in}}{\pgfqpoint{1.812010in}{2.017673in}}{\pgfqpoint{1.812010in}{2.025909in}}%
\pgfpathcurveto{\pgfqpoint{1.812010in}{2.034145in}}{\pgfqpoint{1.808738in}{2.042045in}}{\pgfqpoint{1.802914in}{2.047869in}}%
\pgfpathcurveto{\pgfqpoint{1.797090in}{2.053693in}}{\pgfqpoint{1.789190in}{2.056965in}}{\pgfqpoint{1.780954in}{2.056965in}}%
\pgfpathcurveto{\pgfqpoint{1.772717in}{2.056965in}}{\pgfqpoint{1.764817in}{2.053693in}}{\pgfqpoint{1.758993in}{2.047869in}}%
\pgfpathcurveto{\pgfqpoint{1.753169in}{2.042045in}}{\pgfqpoint{1.749897in}{2.034145in}}{\pgfqpoint{1.749897in}{2.025909in}}%
\pgfpathcurveto{\pgfqpoint{1.749897in}{2.017673in}}{\pgfqpoint{1.753169in}{2.009773in}}{\pgfqpoint{1.758993in}{2.003949in}}%
\pgfpathcurveto{\pgfqpoint{1.764817in}{1.998125in}}{\pgfqpoint{1.772717in}{1.994852in}}{\pgfqpoint{1.780954in}{1.994852in}}%
\pgfpathclose%
\pgfusepath{stroke,fill}%
\end{pgfscope}%
\begin{pgfscope}%
\pgfpathrectangle{\pgfqpoint{0.100000in}{0.212622in}}{\pgfqpoint{3.696000in}{3.696000in}}%
\pgfusepath{clip}%
\pgfsetbuttcap%
\pgfsetroundjoin%
\definecolor{currentfill}{rgb}{0.121569,0.466667,0.705882}%
\pgfsetfillcolor{currentfill}%
\pgfsetfillopacity{0.901680}%
\pgfsetlinewidth{1.003750pt}%
\definecolor{currentstroke}{rgb}{0.121569,0.466667,0.705882}%
\pgfsetstrokecolor{currentstroke}%
\pgfsetstrokeopacity{0.901680}%
\pgfsetdash{}{0pt}%
\pgfpathmoveto{\pgfqpoint{2.201712in}{2.127244in}}%
\pgfpathcurveto{\pgfqpoint{2.209948in}{2.127244in}}{\pgfqpoint{2.217848in}{2.130517in}}{\pgfqpoint{2.223672in}{2.136341in}}%
\pgfpathcurveto{\pgfqpoint{2.229496in}{2.142165in}}{\pgfqpoint{2.232768in}{2.150065in}}{\pgfqpoint{2.232768in}{2.158301in}}%
\pgfpathcurveto{\pgfqpoint{2.232768in}{2.166537in}}{\pgfqpoint{2.229496in}{2.174437in}}{\pgfqpoint{2.223672in}{2.180261in}}%
\pgfpathcurveto{\pgfqpoint{2.217848in}{2.186085in}}{\pgfqpoint{2.209948in}{2.189357in}}{\pgfqpoint{2.201712in}{2.189357in}}%
\pgfpathcurveto{\pgfqpoint{2.193476in}{2.189357in}}{\pgfqpoint{2.185576in}{2.186085in}}{\pgfqpoint{2.179752in}{2.180261in}}%
\pgfpathcurveto{\pgfqpoint{2.173928in}{2.174437in}}{\pgfqpoint{2.170655in}{2.166537in}}{\pgfqpoint{2.170655in}{2.158301in}}%
\pgfpathcurveto{\pgfqpoint{2.170655in}{2.150065in}}{\pgfqpoint{2.173928in}{2.142165in}}{\pgfqpoint{2.179752in}{2.136341in}}%
\pgfpathcurveto{\pgfqpoint{2.185576in}{2.130517in}}{\pgfqpoint{2.193476in}{2.127244in}}{\pgfqpoint{2.201712in}{2.127244in}}%
\pgfpathclose%
\pgfusepath{stroke,fill}%
\end{pgfscope}%
\begin{pgfscope}%
\pgfpathrectangle{\pgfqpoint{0.100000in}{0.212622in}}{\pgfqpoint{3.696000in}{3.696000in}}%
\pgfusepath{clip}%
\pgfsetbuttcap%
\pgfsetroundjoin%
\definecolor{currentfill}{rgb}{0.121569,0.466667,0.705882}%
\pgfsetfillcolor{currentfill}%
\pgfsetfillopacity{0.901769}%
\pgfsetlinewidth{1.003750pt}%
\definecolor{currentstroke}{rgb}{0.121569,0.466667,0.705882}%
\pgfsetstrokecolor{currentstroke}%
\pgfsetstrokeopacity{0.901769}%
\pgfsetdash{}{0pt}%
\pgfpathmoveto{\pgfqpoint{1.784935in}{1.993063in}}%
\pgfpathcurveto{\pgfqpoint{1.793171in}{1.993063in}}{\pgfqpoint{1.801071in}{1.996336in}}{\pgfqpoint{1.806895in}{2.002159in}}%
\pgfpathcurveto{\pgfqpoint{1.812719in}{2.007983in}}{\pgfqpoint{1.815991in}{2.015883in}}{\pgfqpoint{1.815991in}{2.024120in}}%
\pgfpathcurveto{\pgfqpoint{1.815991in}{2.032356in}}{\pgfqpoint{1.812719in}{2.040256in}}{\pgfqpoint{1.806895in}{2.046080in}}%
\pgfpathcurveto{\pgfqpoint{1.801071in}{2.051904in}}{\pgfqpoint{1.793171in}{2.055176in}}{\pgfqpoint{1.784935in}{2.055176in}}%
\pgfpathcurveto{\pgfqpoint{1.776698in}{2.055176in}}{\pgfqpoint{1.768798in}{2.051904in}}{\pgfqpoint{1.762974in}{2.046080in}}%
\pgfpathcurveto{\pgfqpoint{1.757150in}{2.040256in}}{\pgfqpoint{1.753878in}{2.032356in}}{\pgfqpoint{1.753878in}{2.024120in}}%
\pgfpathcurveto{\pgfqpoint{1.753878in}{2.015883in}}{\pgfqpoint{1.757150in}{2.007983in}}{\pgfqpoint{1.762974in}{2.002159in}}%
\pgfpathcurveto{\pgfqpoint{1.768798in}{1.996336in}}{\pgfqpoint{1.776698in}{1.993063in}}{\pgfqpoint{1.784935in}{1.993063in}}%
\pgfpathclose%
\pgfusepath{stroke,fill}%
\end{pgfscope}%
\begin{pgfscope}%
\pgfpathrectangle{\pgfqpoint{0.100000in}{0.212622in}}{\pgfqpoint{3.696000in}{3.696000in}}%
\pgfusepath{clip}%
\pgfsetbuttcap%
\pgfsetroundjoin%
\definecolor{currentfill}{rgb}{0.121569,0.466667,0.705882}%
\pgfsetfillcolor{currentfill}%
\pgfsetfillopacity{0.902376}%
\pgfsetlinewidth{1.003750pt}%
\definecolor{currentstroke}{rgb}{0.121569,0.466667,0.705882}%
\pgfsetstrokecolor{currentstroke}%
\pgfsetstrokeopacity{0.902376}%
\pgfsetdash{}{0pt}%
\pgfpathmoveto{\pgfqpoint{2.200680in}{2.123194in}}%
\pgfpathcurveto{\pgfqpoint{2.208916in}{2.123194in}}{\pgfqpoint{2.216816in}{2.126466in}}{\pgfqpoint{2.222640in}{2.132290in}}%
\pgfpathcurveto{\pgfqpoint{2.228464in}{2.138114in}}{\pgfqpoint{2.231736in}{2.146014in}}{\pgfqpoint{2.231736in}{2.154251in}}%
\pgfpathcurveto{\pgfqpoint{2.231736in}{2.162487in}}{\pgfqpoint{2.228464in}{2.170387in}}{\pgfqpoint{2.222640in}{2.176211in}}%
\pgfpathcurveto{\pgfqpoint{2.216816in}{2.182035in}}{\pgfqpoint{2.208916in}{2.185307in}}{\pgfqpoint{2.200680in}{2.185307in}}%
\pgfpathcurveto{\pgfqpoint{2.192444in}{2.185307in}}{\pgfqpoint{2.184544in}{2.182035in}}{\pgfqpoint{2.178720in}{2.176211in}}%
\pgfpathcurveto{\pgfqpoint{2.172896in}{2.170387in}}{\pgfqpoint{2.169623in}{2.162487in}}{\pgfqpoint{2.169623in}{2.154251in}}%
\pgfpathcurveto{\pgfqpoint{2.169623in}{2.146014in}}{\pgfqpoint{2.172896in}{2.138114in}}{\pgfqpoint{2.178720in}{2.132290in}}%
\pgfpathcurveto{\pgfqpoint{2.184544in}{2.126466in}}{\pgfqpoint{2.192444in}{2.123194in}}{\pgfqpoint{2.200680in}{2.123194in}}%
\pgfpathclose%
\pgfusepath{stroke,fill}%
\end{pgfscope}%
\begin{pgfscope}%
\pgfpathrectangle{\pgfqpoint{0.100000in}{0.212622in}}{\pgfqpoint{3.696000in}{3.696000in}}%
\pgfusepath{clip}%
\pgfsetbuttcap%
\pgfsetroundjoin%
\definecolor{currentfill}{rgb}{0.121569,0.466667,0.705882}%
\pgfsetfillcolor{currentfill}%
\pgfsetfillopacity{0.902450}%
\pgfsetlinewidth{1.003750pt}%
\definecolor{currentstroke}{rgb}{0.121569,0.466667,0.705882}%
\pgfsetstrokecolor{currentstroke}%
\pgfsetstrokeopacity{0.902450}%
\pgfsetdash{}{0pt}%
\pgfpathmoveto{\pgfqpoint{1.789182in}{1.990912in}}%
\pgfpathcurveto{\pgfqpoint{1.797418in}{1.990912in}}{\pgfqpoint{1.805318in}{1.994185in}}{\pgfqpoint{1.811142in}{2.000009in}}%
\pgfpathcurveto{\pgfqpoint{1.816966in}{2.005832in}}{\pgfqpoint{1.820238in}{2.013733in}}{\pgfqpoint{1.820238in}{2.021969in}}%
\pgfpathcurveto{\pgfqpoint{1.820238in}{2.030205in}}{\pgfqpoint{1.816966in}{2.038105in}}{\pgfqpoint{1.811142in}{2.043929in}}%
\pgfpathcurveto{\pgfqpoint{1.805318in}{2.049753in}}{\pgfqpoint{1.797418in}{2.053025in}}{\pgfqpoint{1.789182in}{2.053025in}}%
\pgfpathcurveto{\pgfqpoint{1.780945in}{2.053025in}}{\pgfqpoint{1.773045in}{2.049753in}}{\pgfqpoint{1.767221in}{2.043929in}}%
\pgfpathcurveto{\pgfqpoint{1.761397in}{2.038105in}}{\pgfqpoint{1.758125in}{2.030205in}}{\pgfqpoint{1.758125in}{2.021969in}}%
\pgfpathcurveto{\pgfqpoint{1.758125in}{2.013733in}}{\pgfqpoint{1.761397in}{2.005832in}}{\pgfqpoint{1.767221in}{2.000009in}}%
\pgfpathcurveto{\pgfqpoint{1.773045in}{1.994185in}}{\pgfqpoint{1.780945in}{1.990912in}}{\pgfqpoint{1.789182in}{1.990912in}}%
\pgfpathclose%
\pgfusepath{stroke,fill}%
\end{pgfscope}%
\begin{pgfscope}%
\pgfpathrectangle{\pgfqpoint{0.100000in}{0.212622in}}{\pgfqpoint{3.696000in}{3.696000in}}%
\pgfusepath{clip}%
\pgfsetbuttcap%
\pgfsetroundjoin%
\definecolor{currentfill}{rgb}{0.121569,0.466667,0.705882}%
\pgfsetfillcolor{currentfill}%
\pgfsetfillopacity{0.902961}%
\pgfsetlinewidth{1.003750pt}%
\definecolor{currentstroke}{rgb}{0.121569,0.466667,0.705882}%
\pgfsetstrokecolor{currentstroke}%
\pgfsetstrokeopacity{0.902961}%
\pgfsetdash{}{0pt}%
\pgfpathmoveto{\pgfqpoint{2.199195in}{2.120534in}}%
\pgfpathcurveto{\pgfqpoint{2.207431in}{2.120534in}}{\pgfqpoint{2.215331in}{2.123807in}}{\pgfqpoint{2.221155in}{2.129630in}}%
\pgfpathcurveto{\pgfqpoint{2.226979in}{2.135454in}}{\pgfqpoint{2.230251in}{2.143354in}}{\pgfqpoint{2.230251in}{2.151591in}}%
\pgfpathcurveto{\pgfqpoint{2.230251in}{2.159827in}}{\pgfqpoint{2.226979in}{2.167727in}}{\pgfqpoint{2.221155in}{2.173551in}}%
\pgfpathcurveto{\pgfqpoint{2.215331in}{2.179375in}}{\pgfqpoint{2.207431in}{2.182647in}}{\pgfqpoint{2.199195in}{2.182647in}}%
\pgfpathcurveto{\pgfqpoint{2.190959in}{2.182647in}}{\pgfqpoint{2.183059in}{2.179375in}}{\pgfqpoint{2.177235in}{2.173551in}}%
\pgfpathcurveto{\pgfqpoint{2.171411in}{2.167727in}}{\pgfqpoint{2.168138in}{2.159827in}}{\pgfqpoint{2.168138in}{2.151591in}}%
\pgfpathcurveto{\pgfqpoint{2.168138in}{2.143354in}}{\pgfqpoint{2.171411in}{2.135454in}}{\pgfqpoint{2.177235in}{2.129630in}}%
\pgfpathcurveto{\pgfqpoint{2.183059in}{2.123807in}}{\pgfqpoint{2.190959in}{2.120534in}}{\pgfqpoint{2.199195in}{2.120534in}}%
\pgfpathclose%
\pgfusepath{stroke,fill}%
\end{pgfscope}%
\begin{pgfscope}%
\pgfpathrectangle{\pgfqpoint{0.100000in}{0.212622in}}{\pgfqpoint{3.696000in}{3.696000in}}%
\pgfusepath{clip}%
\pgfsetbuttcap%
\pgfsetroundjoin%
\definecolor{currentfill}{rgb}{0.121569,0.466667,0.705882}%
\pgfsetfillcolor{currentfill}%
\pgfsetfillopacity{0.903048}%
\pgfsetlinewidth{1.003750pt}%
\definecolor{currentstroke}{rgb}{0.121569,0.466667,0.705882}%
\pgfsetstrokecolor{currentstroke}%
\pgfsetstrokeopacity{0.903048}%
\pgfsetdash{}{0pt}%
\pgfpathmoveto{\pgfqpoint{2.198786in}{2.120050in}}%
\pgfpathcurveto{\pgfqpoint{2.207022in}{2.120050in}}{\pgfqpoint{2.214923in}{2.123322in}}{\pgfqpoint{2.220746in}{2.129146in}}%
\pgfpathcurveto{\pgfqpoint{2.226570in}{2.134970in}}{\pgfqpoint{2.229843in}{2.142870in}}{\pgfqpoint{2.229843in}{2.151106in}}%
\pgfpathcurveto{\pgfqpoint{2.229843in}{2.159343in}}{\pgfqpoint{2.226570in}{2.167243in}}{\pgfqpoint{2.220746in}{2.173067in}}%
\pgfpathcurveto{\pgfqpoint{2.214923in}{2.178890in}}{\pgfqpoint{2.207022in}{2.182163in}}{\pgfqpoint{2.198786in}{2.182163in}}%
\pgfpathcurveto{\pgfqpoint{2.190550in}{2.182163in}}{\pgfqpoint{2.182650in}{2.178890in}}{\pgfqpoint{2.176826in}{2.173067in}}%
\pgfpathcurveto{\pgfqpoint{2.171002in}{2.167243in}}{\pgfqpoint{2.167730in}{2.159343in}}{\pgfqpoint{2.167730in}{2.151106in}}%
\pgfpathcurveto{\pgfqpoint{2.167730in}{2.142870in}}{\pgfqpoint{2.171002in}{2.134970in}}{\pgfqpoint{2.176826in}{2.129146in}}%
\pgfpathcurveto{\pgfqpoint{2.182650in}{2.123322in}}{\pgfqpoint{2.190550in}{2.120050in}}{\pgfqpoint{2.198786in}{2.120050in}}%
\pgfpathclose%
\pgfusepath{stroke,fill}%
\end{pgfscope}%
\begin{pgfscope}%
\pgfpathrectangle{\pgfqpoint{0.100000in}{0.212622in}}{\pgfqpoint{3.696000in}{3.696000in}}%
\pgfusepath{clip}%
\pgfsetbuttcap%
\pgfsetroundjoin%
\definecolor{currentfill}{rgb}{0.121569,0.466667,0.705882}%
\pgfsetfillcolor{currentfill}%
\pgfsetfillopacity{0.903194}%
\pgfsetlinewidth{1.003750pt}%
\definecolor{currentstroke}{rgb}{0.121569,0.466667,0.705882}%
\pgfsetstrokecolor{currentstroke}%
\pgfsetstrokeopacity{0.903194}%
\pgfsetdash{}{0pt}%
\pgfpathmoveto{\pgfqpoint{2.198074in}{2.119025in}}%
\pgfpathcurveto{\pgfqpoint{2.206310in}{2.119025in}}{\pgfqpoint{2.214210in}{2.122297in}}{\pgfqpoint{2.220034in}{2.128121in}}%
\pgfpathcurveto{\pgfqpoint{2.225858in}{2.133945in}}{\pgfqpoint{2.229130in}{2.141845in}}{\pgfqpoint{2.229130in}{2.150082in}}%
\pgfpathcurveto{\pgfqpoint{2.229130in}{2.158318in}}{\pgfqpoint{2.225858in}{2.166218in}}{\pgfqpoint{2.220034in}{2.172042in}}%
\pgfpathcurveto{\pgfqpoint{2.214210in}{2.177866in}}{\pgfqpoint{2.206310in}{2.181138in}}{\pgfqpoint{2.198074in}{2.181138in}}%
\pgfpathcurveto{\pgfqpoint{2.189837in}{2.181138in}}{\pgfqpoint{2.181937in}{2.177866in}}{\pgfqpoint{2.176113in}{2.172042in}}%
\pgfpathcurveto{\pgfqpoint{2.170289in}{2.166218in}}{\pgfqpoint{2.167017in}{2.158318in}}{\pgfqpoint{2.167017in}{2.150082in}}%
\pgfpathcurveto{\pgfqpoint{2.167017in}{2.141845in}}{\pgfqpoint{2.170289in}{2.133945in}}{\pgfqpoint{2.176113in}{2.128121in}}%
\pgfpathcurveto{\pgfqpoint{2.181937in}{2.122297in}}{\pgfqpoint{2.189837in}{2.119025in}}{\pgfqpoint{2.198074in}{2.119025in}}%
\pgfpathclose%
\pgfusepath{stroke,fill}%
\end{pgfscope}%
\begin{pgfscope}%
\pgfpathrectangle{\pgfqpoint{0.100000in}{0.212622in}}{\pgfqpoint{3.696000in}{3.696000in}}%
\pgfusepath{clip}%
\pgfsetbuttcap%
\pgfsetroundjoin%
\definecolor{currentfill}{rgb}{0.121569,0.466667,0.705882}%
\pgfsetfillcolor{currentfill}%
\pgfsetfillopacity{0.903293}%
\pgfsetlinewidth{1.003750pt}%
\definecolor{currentstroke}{rgb}{0.121569,0.466667,0.705882}%
\pgfsetstrokecolor{currentstroke}%
\pgfsetstrokeopacity{0.903293}%
\pgfsetdash{}{0pt}%
\pgfpathmoveto{\pgfqpoint{2.197829in}{2.118560in}}%
\pgfpathcurveto{\pgfqpoint{2.206065in}{2.118560in}}{\pgfqpoint{2.213965in}{2.121832in}}{\pgfqpoint{2.219789in}{2.127656in}}%
\pgfpathcurveto{\pgfqpoint{2.225613in}{2.133480in}}{\pgfqpoint{2.228886in}{2.141380in}}{\pgfqpoint{2.228886in}{2.149616in}}%
\pgfpathcurveto{\pgfqpoint{2.228886in}{2.157853in}}{\pgfqpoint{2.225613in}{2.165753in}}{\pgfqpoint{2.219789in}{2.171577in}}%
\pgfpathcurveto{\pgfqpoint{2.213965in}{2.177401in}}{\pgfqpoint{2.206065in}{2.180673in}}{\pgfqpoint{2.197829in}{2.180673in}}%
\pgfpathcurveto{\pgfqpoint{2.189593in}{2.180673in}}{\pgfqpoint{2.181693in}{2.177401in}}{\pgfqpoint{2.175869in}{2.171577in}}%
\pgfpathcurveto{\pgfqpoint{2.170045in}{2.165753in}}{\pgfqpoint{2.166773in}{2.157853in}}{\pgfqpoint{2.166773in}{2.149616in}}%
\pgfpathcurveto{\pgfqpoint{2.166773in}{2.141380in}}{\pgfqpoint{2.170045in}{2.133480in}}{\pgfqpoint{2.175869in}{2.127656in}}%
\pgfpathcurveto{\pgfqpoint{2.181693in}{2.121832in}}{\pgfqpoint{2.189593in}{2.118560in}}{\pgfqpoint{2.197829in}{2.118560in}}%
\pgfpathclose%
\pgfusepath{stroke,fill}%
\end{pgfscope}%
\begin{pgfscope}%
\pgfpathrectangle{\pgfqpoint{0.100000in}{0.212622in}}{\pgfqpoint{3.696000in}{3.696000in}}%
\pgfusepath{clip}%
\pgfsetbuttcap%
\pgfsetroundjoin%
\definecolor{currentfill}{rgb}{0.121569,0.466667,0.705882}%
\pgfsetfillcolor{currentfill}%
\pgfsetfillopacity{0.903389}%
\pgfsetlinewidth{1.003750pt}%
\definecolor{currentstroke}{rgb}{0.121569,0.466667,0.705882}%
\pgfsetstrokecolor{currentstroke}%
\pgfsetstrokeopacity{0.903389}%
\pgfsetdash{}{0pt}%
\pgfpathmoveto{\pgfqpoint{1.793513in}{1.988277in}}%
\pgfpathcurveto{\pgfqpoint{1.801749in}{1.988277in}}{\pgfqpoint{1.809649in}{1.991549in}}{\pgfqpoint{1.815473in}{1.997373in}}%
\pgfpathcurveto{\pgfqpoint{1.821297in}{2.003197in}}{\pgfqpoint{1.824569in}{2.011097in}}{\pgfqpoint{1.824569in}{2.019333in}}%
\pgfpathcurveto{\pgfqpoint{1.824569in}{2.027569in}}{\pgfqpoint{1.821297in}{2.035469in}}{\pgfqpoint{1.815473in}{2.041293in}}%
\pgfpathcurveto{\pgfqpoint{1.809649in}{2.047117in}}{\pgfqpoint{1.801749in}{2.050390in}}{\pgfqpoint{1.793513in}{2.050390in}}%
\pgfpathcurveto{\pgfqpoint{1.785277in}{2.050390in}}{\pgfqpoint{1.777377in}{2.047117in}}{\pgfqpoint{1.771553in}{2.041293in}}%
\pgfpathcurveto{\pgfqpoint{1.765729in}{2.035469in}}{\pgfqpoint{1.762456in}{2.027569in}}{\pgfqpoint{1.762456in}{2.019333in}}%
\pgfpathcurveto{\pgfqpoint{1.762456in}{2.011097in}}{\pgfqpoint{1.765729in}{2.003197in}}{\pgfqpoint{1.771553in}{1.997373in}}%
\pgfpathcurveto{\pgfqpoint{1.777377in}{1.991549in}}{\pgfqpoint{1.785277in}{1.988277in}}{\pgfqpoint{1.793513in}{1.988277in}}%
\pgfpathclose%
\pgfusepath{stroke,fill}%
\end{pgfscope}%
\begin{pgfscope}%
\pgfpathrectangle{\pgfqpoint{0.100000in}{0.212622in}}{\pgfqpoint{3.696000in}{3.696000in}}%
\pgfusepath{clip}%
\pgfsetbuttcap%
\pgfsetroundjoin%
\definecolor{currentfill}{rgb}{0.121569,0.466667,0.705882}%
\pgfsetfillcolor{currentfill}%
\pgfsetfillopacity{0.903470}%
\pgfsetlinewidth{1.003750pt}%
\definecolor{currentstroke}{rgb}{0.121569,0.466667,0.705882}%
\pgfsetstrokecolor{currentstroke}%
\pgfsetstrokeopacity{0.903470}%
\pgfsetdash{}{0pt}%
\pgfpathmoveto{\pgfqpoint{2.197405in}{2.117672in}}%
\pgfpathcurveto{\pgfqpoint{2.205641in}{2.117672in}}{\pgfqpoint{2.213541in}{2.120944in}}{\pgfqpoint{2.219365in}{2.126768in}}%
\pgfpathcurveto{\pgfqpoint{2.225189in}{2.132592in}}{\pgfqpoint{2.228461in}{2.140492in}}{\pgfqpoint{2.228461in}{2.148729in}}%
\pgfpathcurveto{\pgfqpoint{2.228461in}{2.156965in}}{\pgfqpoint{2.225189in}{2.164865in}}{\pgfqpoint{2.219365in}{2.170689in}}%
\pgfpathcurveto{\pgfqpoint{2.213541in}{2.176513in}}{\pgfqpoint{2.205641in}{2.179785in}}{\pgfqpoint{2.197405in}{2.179785in}}%
\pgfpathcurveto{\pgfqpoint{2.189169in}{2.179785in}}{\pgfqpoint{2.181268in}{2.176513in}}{\pgfqpoint{2.175445in}{2.170689in}}%
\pgfpathcurveto{\pgfqpoint{2.169621in}{2.164865in}}{\pgfqpoint{2.166348in}{2.156965in}}{\pgfqpoint{2.166348in}{2.148729in}}%
\pgfpathcurveto{\pgfqpoint{2.166348in}{2.140492in}}{\pgfqpoint{2.169621in}{2.132592in}}{\pgfqpoint{2.175445in}{2.126768in}}%
\pgfpathcurveto{\pgfqpoint{2.181268in}{2.120944in}}{\pgfqpoint{2.189169in}{2.117672in}}{\pgfqpoint{2.197405in}{2.117672in}}%
\pgfpathclose%
\pgfusepath{stroke,fill}%
\end{pgfscope}%
\begin{pgfscope}%
\pgfpathrectangle{\pgfqpoint{0.100000in}{0.212622in}}{\pgfqpoint{3.696000in}{3.696000in}}%
\pgfusepath{clip}%
\pgfsetbuttcap%
\pgfsetroundjoin%
\definecolor{currentfill}{rgb}{0.121569,0.466667,0.705882}%
\pgfsetfillcolor{currentfill}%
\pgfsetfillopacity{0.903535}%
\pgfsetlinewidth{1.003750pt}%
\definecolor{currentstroke}{rgb}{0.121569,0.466667,0.705882}%
\pgfsetstrokecolor{currentstroke}%
\pgfsetstrokeopacity{0.903535}%
\pgfsetdash{}{0pt}%
\pgfpathmoveto{\pgfqpoint{2.197150in}{2.117335in}}%
\pgfpathcurveto{\pgfqpoint{2.205386in}{2.117335in}}{\pgfqpoint{2.213286in}{2.120607in}}{\pgfqpoint{2.219110in}{2.126431in}}%
\pgfpathcurveto{\pgfqpoint{2.224934in}{2.132255in}}{\pgfqpoint{2.228206in}{2.140155in}}{\pgfqpoint{2.228206in}{2.148392in}}%
\pgfpathcurveto{\pgfqpoint{2.228206in}{2.156628in}}{\pgfqpoint{2.224934in}{2.164528in}}{\pgfqpoint{2.219110in}{2.170352in}}%
\pgfpathcurveto{\pgfqpoint{2.213286in}{2.176176in}}{\pgfqpoint{2.205386in}{2.179448in}}{\pgfqpoint{2.197150in}{2.179448in}}%
\pgfpathcurveto{\pgfqpoint{2.188914in}{2.179448in}}{\pgfqpoint{2.181013in}{2.176176in}}{\pgfqpoint{2.175190in}{2.170352in}}%
\pgfpathcurveto{\pgfqpoint{2.169366in}{2.164528in}}{\pgfqpoint{2.166093in}{2.156628in}}{\pgfqpoint{2.166093in}{2.148392in}}%
\pgfpathcurveto{\pgfqpoint{2.166093in}{2.140155in}}{\pgfqpoint{2.169366in}{2.132255in}}{\pgfqpoint{2.175190in}{2.126431in}}%
\pgfpathcurveto{\pgfqpoint{2.181013in}{2.120607in}}{\pgfqpoint{2.188914in}{2.117335in}}{\pgfqpoint{2.197150in}{2.117335in}}%
\pgfpathclose%
\pgfusepath{stroke,fill}%
\end{pgfscope}%
\begin{pgfscope}%
\pgfpathrectangle{\pgfqpoint{0.100000in}{0.212622in}}{\pgfqpoint{3.696000in}{3.696000in}}%
\pgfusepath{clip}%
\pgfsetbuttcap%
\pgfsetroundjoin%
\definecolor{currentfill}{rgb}{0.121569,0.466667,0.705882}%
\pgfsetfillcolor{currentfill}%
\pgfsetfillopacity{0.903540}%
\pgfsetlinewidth{1.003750pt}%
\definecolor{currentstroke}{rgb}{0.121569,0.466667,0.705882}%
\pgfsetstrokecolor{currentstroke}%
\pgfsetstrokeopacity{0.903540}%
\pgfsetdash{}{0pt}%
\pgfpathmoveto{\pgfqpoint{1.726822in}{1.122564in}}%
\pgfpathcurveto{\pgfqpoint{1.735059in}{1.122564in}}{\pgfqpoint{1.742959in}{1.125837in}}{\pgfqpoint{1.748783in}{1.131661in}}%
\pgfpathcurveto{\pgfqpoint{1.754607in}{1.137485in}}{\pgfqpoint{1.757879in}{1.145385in}}{\pgfqpoint{1.757879in}{1.153621in}}%
\pgfpathcurveto{\pgfqpoint{1.757879in}{1.161857in}}{\pgfqpoint{1.754607in}{1.169757in}}{\pgfqpoint{1.748783in}{1.175581in}}%
\pgfpathcurveto{\pgfqpoint{1.742959in}{1.181405in}}{\pgfqpoint{1.735059in}{1.184677in}}{\pgfqpoint{1.726822in}{1.184677in}}%
\pgfpathcurveto{\pgfqpoint{1.718586in}{1.184677in}}{\pgfqpoint{1.710686in}{1.181405in}}{\pgfqpoint{1.704862in}{1.175581in}}%
\pgfpathcurveto{\pgfqpoint{1.699038in}{1.169757in}}{\pgfqpoint{1.695766in}{1.161857in}}{\pgfqpoint{1.695766in}{1.153621in}}%
\pgfpathcurveto{\pgfqpoint{1.695766in}{1.145385in}}{\pgfqpoint{1.699038in}{1.137485in}}{\pgfqpoint{1.704862in}{1.131661in}}%
\pgfpathcurveto{\pgfqpoint{1.710686in}{1.125837in}}{\pgfqpoint{1.718586in}{1.122564in}}{\pgfqpoint{1.726822in}{1.122564in}}%
\pgfpathclose%
\pgfusepath{stroke,fill}%
\end{pgfscope}%
\begin{pgfscope}%
\pgfpathrectangle{\pgfqpoint{0.100000in}{0.212622in}}{\pgfqpoint{3.696000in}{3.696000in}}%
\pgfusepath{clip}%
\pgfsetbuttcap%
\pgfsetroundjoin%
\definecolor{currentfill}{rgb}{0.121569,0.466667,0.705882}%
\pgfsetfillcolor{currentfill}%
\pgfsetfillopacity{0.903638}%
\pgfsetlinewidth{1.003750pt}%
\definecolor{currentstroke}{rgb}{0.121569,0.466667,0.705882}%
\pgfsetstrokecolor{currentstroke}%
\pgfsetstrokeopacity{0.903638}%
\pgfsetdash{}{0pt}%
\pgfpathmoveto{\pgfqpoint{2.196668in}{2.116699in}}%
\pgfpathcurveto{\pgfqpoint{2.204905in}{2.116699in}}{\pgfqpoint{2.212805in}{2.119971in}}{\pgfqpoint{2.218629in}{2.125795in}}%
\pgfpathcurveto{\pgfqpoint{2.224453in}{2.131619in}}{\pgfqpoint{2.227725in}{2.139519in}}{\pgfqpoint{2.227725in}{2.147755in}}%
\pgfpathcurveto{\pgfqpoint{2.227725in}{2.155992in}}{\pgfqpoint{2.224453in}{2.163892in}}{\pgfqpoint{2.218629in}{2.169716in}}%
\pgfpathcurveto{\pgfqpoint{2.212805in}{2.175540in}}{\pgfqpoint{2.204905in}{2.178812in}}{\pgfqpoint{2.196668in}{2.178812in}}%
\pgfpathcurveto{\pgfqpoint{2.188432in}{2.178812in}}{\pgfqpoint{2.180532in}{2.175540in}}{\pgfqpoint{2.174708in}{2.169716in}}%
\pgfpathcurveto{\pgfqpoint{2.168884in}{2.163892in}}{\pgfqpoint{2.165612in}{2.155992in}}{\pgfqpoint{2.165612in}{2.147755in}}%
\pgfpathcurveto{\pgfqpoint{2.165612in}{2.139519in}}{\pgfqpoint{2.168884in}{2.131619in}}{\pgfqpoint{2.174708in}{2.125795in}}%
\pgfpathcurveto{\pgfqpoint{2.180532in}{2.119971in}}{\pgfqpoint{2.188432in}{2.116699in}}{\pgfqpoint{2.196668in}{2.116699in}}%
\pgfpathclose%
\pgfusepath{stroke,fill}%
\end{pgfscope}%
\begin{pgfscope}%
\pgfpathrectangle{\pgfqpoint{0.100000in}{0.212622in}}{\pgfqpoint{3.696000in}{3.696000in}}%
\pgfusepath{clip}%
\pgfsetbuttcap%
\pgfsetroundjoin%
\definecolor{currentfill}{rgb}{0.121569,0.466667,0.705882}%
\pgfsetfillcolor{currentfill}%
\pgfsetfillopacity{0.903871}%
\pgfsetlinewidth{1.003750pt}%
\definecolor{currentstroke}{rgb}{0.121569,0.466667,0.705882}%
\pgfsetstrokecolor{currentstroke}%
\pgfsetstrokeopacity{0.903871}%
\pgfsetdash{}{0pt}%
\pgfpathmoveto{\pgfqpoint{2.195918in}{2.115499in}}%
\pgfpathcurveto{\pgfqpoint{2.204154in}{2.115499in}}{\pgfqpoint{2.212054in}{2.118771in}}{\pgfqpoint{2.217878in}{2.124595in}}%
\pgfpathcurveto{\pgfqpoint{2.223702in}{2.130419in}}{\pgfqpoint{2.226974in}{2.138319in}}{\pgfqpoint{2.226974in}{2.146556in}}%
\pgfpathcurveto{\pgfqpoint{2.226974in}{2.154792in}}{\pgfqpoint{2.223702in}{2.162692in}}{\pgfqpoint{2.217878in}{2.168516in}}%
\pgfpathcurveto{\pgfqpoint{2.212054in}{2.174340in}}{\pgfqpoint{2.204154in}{2.177612in}}{\pgfqpoint{2.195918in}{2.177612in}}%
\pgfpathcurveto{\pgfqpoint{2.187681in}{2.177612in}}{\pgfqpoint{2.179781in}{2.174340in}}{\pgfqpoint{2.173957in}{2.168516in}}%
\pgfpathcurveto{\pgfqpoint{2.168133in}{2.162692in}}{\pgfqpoint{2.164861in}{2.154792in}}{\pgfqpoint{2.164861in}{2.146556in}}%
\pgfpathcurveto{\pgfqpoint{2.164861in}{2.138319in}}{\pgfqpoint{2.168133in}{2.130419in}}{\pgfqpoint{2.173957in}{2.124595in}}%
\pgfpathcurveto{\pgfqpoint{2.179781in}{2.118771in}}{\pgfqpoint{2.187681in}{2.115499in}}{\pgfqpoint{2.195918in}{2.115499in}}%
\pgfpathclose%
\pgfusepath{stroke,fill}%
\end{pgfscope}%
\begin{pgfscope}%
\pgfpathrectangle{\pgfqpoint{0.100000in}{0.212622in}}{\pgfqpoint{3.696000in}{3.696000in}}%
\pgfusepath{clip}%
\pgfsetbuttcap%
\pgfsetroundjoin%
\definecolor{currentfill}{rgb}{0.121569,0.466667,0.705882}%
\pgfsetfillcolor{currentfill}%
\pgfsetfillopacity{0.904032}%
\pgfsetlinewidth{1.003750pt}%
\definecolor{currentstroke}{rgb}{0.121569,0.466667,0.705882}%
\pgfsetstrokecolor{currentstroke}%
\pgfsetstrokeopacity{0.904032}%
\pgfsetdash{}{0pt}%
\pgfpathmoveto{\pgfqpoint{2.195486in}{2.114580in}}%
\pgfpathcurveto{\pgfqpoint{2.203723in}{2.114580in}}{\pgfqpoint{2.211623in}{2.117853in}}{\pgfqpoint{2.217447in}{2.123676in}}%
\pgfpathcurveto{\pgfqpoint{2.223270in}{2.129500in}}{\pgfqpoint{2.226543in}{2.137400in}}{\pgfqpoint{2.226543in}{2.145637in}}%
\pgfpathcurveto{\pgfqpoint{2.226543in}{2.153873in}}{\pgfqpoint{2.223270in}{2.161773in}}{\pgfqpoint{2.217447in}{2.167597in}}%
\pgfpathcurveto{\pgfqpoint{2.211623in}{2.173421in}}{\pgfqpoint{2.203723in}{2.176693in}}{\pgfqpoint{2.195486in}{2.176693in}}%
\pgfpathcurveto{\pgfqpoint{2.187250in}{2.176693in}}{\pgfqpoint{2.179350in}{2.173421in}}{\pgfqpoint{2.173526in}{2.167597in}}%
\pgfpathcurveto{\pgfqpoint{2.167702in}{2.161773in}}{\pgfqpoint{2.164430in}{2.153873in}}{\pgfqpoint{2.164430in}{2.145637in}}%
\pgfpathcurveto{\pgfqpoint{2.164430in}{2.137400in}}{\pgfqpoint{2.167702in}{2.129500in}}{\pgfqpoint{2.173526in}{2.123676in}}%
\pgfpathcurveto{\pgfqpoint{2.179350in}{2.117853in}}{\pgfqpoint{2.187250in}{2.114580in}}{\pgfqpoint{2.195486in}{2.114580in}}%
\pgfpathclose%
\pgfusepath{stroke,fill}%
\end{pgfscope}%
\begin{pgfscope}%
\pgfpathrectangle{\pgfqpoint{0.100000in}{0.212622in}}{\pgfqpoint{3.696000in}{3.696000in}}%
\pgfusepath{clip}%
\pgfsetbuttcap%
\pgfsetroundjoin%
\definecolor{currentfill}{rgb}{0.121569,0.466667,0.705882}%
\pgfsetfillcolor{currentfill}%
\pgfsetfillopacity{0.904079}%
\pgfsetlinewidth{1.003750pt}%
\definecolor{currentstroke}{rgb}{0.121569,0.466667,0.705882}%
\pgfsetstrokecolor{currentstroke}%
\pgfsetstrokeopacity{0.904079}%
\pgfsetdash{}{0pt}%
\pgfpathmoveto{\pgfqpoint{2.195331in}{2.114320in}}%
\pgfpathcurveto{\pgfqpoint{2.203567in}{2.114320in}}{\pgfqpoint{2.211467in}{2.117593in}}{\pgfqpoint{2.217291in}{2.123417in}}%
\pgfpathcurveto{\pgfqpoint{2.223115in}{2.129241in}}{\pgfqpoint{2.226388in}{2.137141in}}{\pgfqpoint{2.226388in}{2.145377in}}%
\pgfpathcurveto{\pgfqpoint{2.226388in}{2.153613in}}{\pgfqpoint{2.223115in}{2.161513in}}{\pgfqpoint{2.217291in}{2.167337in}}%
\pgfpathcurveto{\pgfqpoint{2.211467in}{2.173161in}}{\pgfqpoint{2.203567in}{2.176433in}}{\pgfqpoint{2.195331in}{2.176433in}}%
\pgfpathcurveto{\pgfqpoint{2.187095in}{2.176433in}}{\pgfqpoint{2.179195in}{2.173161in}}{\pgfqpoint{2.173371in}{2.167337in}}%
\pgfpathcurveto{\pgfqpoint{2.167547in}{2.161513in}}{\pgfqpoint{2.164275in}{2.153613in}}{\pgfqpoint{2.164275in}{2.145377in}}%
\pgfpathcurveto{\pgfqpoint{2.164275in}{2.137141in}}{\pgfqpoint{2.167547in}{2.129241in}}{\pgfqpoint{2.173371in}{2.123417in}}%
\pgfpathcurveto{\pgfqpoint{2.179195in}{2.117593in}}{\pgfqpoint{2.187095in}{2.114320in}}{\pgfqpoint{2.195331in}{2.114320in}}%
\pgfpathclose%
\pgfusepath{stroke,fill}%
\end{pgfscope}%
\begin{pgfscope}%
\pgfpathrectangle{\pgfqpoint{0.100000in}{0.212622in}}{\pgfqpoint{3.696000in}{3.696000in}}%
\pgfusepath{clip}%
\pgfsetbuttcap%
\pgfsetroundjoin%
\definecolor{currentfill}{rgb}{0.121569,0.466667,0.705882}%
\pgfsetfillcolor{currentfill}%
\pgfsetfillopacity{0.904161}%
\pgfsetlinewidth{1.003750pt}%
\definecolor{currentstroke}{rgb}{0.121569,0.466667,0.705882}%
\pgfsetstrokecolor{currentstroke}%
\pgfsetstrokeopacity{0.904161}%
\pgfsetdash{}{0pt}%
\pgfpathmoveto{\pgfqpoint{2.195020in}{2.113879in}}%
\pgfpathcurveto{\pgfqpoint{2.203256in}{2.113879in}}{\pgfqpoint{2.211156in}{2.117151in}}{\pgfqpoint{2.216980in}{2.122975in}}%
\pgfpathcurveto{\pgfqpoint{2.222804in}{2.128799in}}{\pgfqpoint{2.226077in}{2.136699in}}{\pgfqpoint{2.226077in}{2.144935in}}%
\pgfpathcurveto{\pgfqpoint{2.226077in}{2.153172in}}{\pgfqpoint{2.222804in}{2.161072in}}{\pgfqpoint{2.216980in}{2.166896in}}%
\pgfpathcurveto{\pgfqpoint{2.211156in}{2.172720in}}{\pgfqpoint{2.203256in}{2.175992in}}{\pgfqpoint{2.195020in}{2.175992in}}%
\pgfpathcurveto{\pgfqpoint{2.186784in}{2.175992in}}{\pgfqpoint{2.178884in}{2.172720in}}{\pgfqpoint{2.173060in}{2.166896in}}%
\pgfpathcurveto{\pgfqpoint{2.167236in}{2.161072in}}{\pgfqpoint{2.163964in}{2.153172in}}{\pgfqpoint{2.163964in}{2.144935in}}%
\pgfpathcurveto{\pgfqpoint{2.163964in}{2.136699in}}{\pgfqpoint{2.167236in}{2.128799in}}{\pgfqpoint{2.173060in}{2.122975in}}%
\pgfpathcurveto{\pgfqpoint{2.178884in}{2.117151in}}{\pgfqpoint{2.186784in}{2.113879in}}{\pgfqpoint{2.195020in}{2.113879in}}%
\pgfpathclose%
\pgfusepath{stroke,fill}%
\end{pgfscope}%
\begin{pgfscope}%
\pgfpathrectangle{\pgfqpoint{0.100000in}{0.212622in}}{\pgfqpoint{3.696000in}{3.696000in}}%
\pgfusepath{clip}%
\pgfsetbuttcap%
\pgfsetroundjoin%
\definecolor{currentfill}{rgb}{0.121569,0.466667,0.705882}%
\pgfsetfillcolor{currentfill}%
\pgfsetfillopacity{0.904261}%
\pgfsetlinewidth{1.003750pt}%
\definecolor{currentstroke}{rgb}{0.121569,0.466667,0.705882}%
\pgfsetstrokecolor{currentstroke}%
\pgfsetstrokeopacity{0.904261}%
\pgfsetdash{}{0pt}%
\pgfpathmoveto{\pgfqpoint{1.798214in}{1.983997in}}%
\pgfpathcurveto{\pgfqpoint{1.806450in}{1.983997in}}{\pgfqpoint{1.814350in}{1.987270in}}{\pgfqpoint{1.820174in}{1.993093in}}%
\pgfpathcurveto{\pgfqpoint{1.825998in}{1.998917in}}{\pgfqpoint{1.829270in}{2.006817in}}{\pgfqpoint{1.829270in}{2.015054in}}%
\pgfpathcurveto{\pgfqpoint{1.829270in}{2.023290in}}{\pgfqpoint{1.825998in}{2.031190in}}{\pgfqpoint{1.820174in}{2.037014in}}%
\pgfpathcurveto{\pgfqpoint{1.814350in}{2.042838in}}{\pgfqpoint{1.806450in}{2.046110in}}{\pgfqpoint{1.798214in}{2.046110in}}%
\pgfpathcurveto{\pgfqpoint{1.789978in}{2.046110in}}{\pgfqpoint{1.782078in}{2.042838in}}{\pgfqpoint{1.776254in}{2.037014in}}%
\pgfpathcurveto{\pgfqpoint{1.770430in}{2.031190in}}{\pgfqpoint{1.767157in}{2.023290in}}{\pgfqpoint{1.767157in}{2.015054in}}%
\pgfpathcurveto{\pgfqpoint{1.767157in}{2.006817in}}{\pgfqpoint{1.770430in}{1.998917in}}{\pgfqpoint{1.776254in}{1.993093in}}%
\pgfpathcurveto{\pgfqpoint{1.782078in}{1.987270in}}{\pgfqpoint{1.789978in}{1.983997in}}{\pgfqpoint{1.798214in}{1.983997in}}%
\pgfpathclose%
\pgfusepath{stroke,fill}%
\end{pgfscope}%
\begin{pgfscope}%
\pgfpathrectangle{\pgfqpoint{0.100000in}{0.212622in}}{\pgfqpoint{3.696000in}{3.696000in}}%
\pgfusepath{clip}%
\pgfsetbuttcap%
\pgfsetroundjoin%
\definecolor{currentfill}{rgb}{0.121569,0.466667,0.705882}%
\pgfsetfillcolor{currentfill}%
\pgfsetfillopacity{0.904300}%
\pgfsetlinewidth{1.003750pt}%
\definecolor{currentstroke}{rgb}{0.121569,0.466667,0.705882}%
\pgfsetstrokecolor{currentstroke}%
\pgfsetstrokeopacity{0.904300}%
\pgfsetdash{}{0pt}%
\pgfpathmoveto{\pgfqpoint{2.194419in}{2.113116in}}%
\pgfpathcurveto{\pgfqpoint{2.202655in}{2.113116in}}{\pgfqpoint{2.210555in}{2.116389in}}{\pgfqpoint{2.216379in}{2.122213in}}%
\pgfpathcurveto{\pgfqpoint{2.222203in}{2.128037in}}{\pgfqpoint{2.225475in}{2.135937in}}{\pgfqpoint{2.225475in}{2.144173in}}%
\pgfpathcurveto{\pgfqpoint{2.225475in}{2.152409in}}{\pgfqpoint{2.222203in}{2.160309in}}{\pgfqpoint{2.216379in}{2.166133in}}%
\pgfpathcurveto{\pgfqpoint{2.210555in}{2.171957in}}{\pgfqpoint{2.202655in}{2.175229in}}{\pgfqpoint{2.194419in}{2.175229in}}%
\pgfpathcurveto{\pgfqpoint{2.186182in}{2.175229in}}{\pgfqpoint{2.178282in}{2.171957in}}{\pgfqpoint{2.172458in}{2.166133in}}%
\pgfpathcurveto{\pgfqpoint{2.166634in}{2.160309in}}{\pgfqpoint{2.163362in}{2.152409in}}{\pgfqpoint{2.163362in}{2.144173in}}%
\pgfpathcurveto{\pgfqpoint{2.163362in}{2.135937in}}{\pgfqpoint{2.166634in}{2.128037in}}{\pgfqpoint{2.172458in}{2.122213in}}%
\pgfpathcurveto{\pgfqpoint{2.178282in}{2.116389in}}{\pgfqpoint{2.186182in}{2.113116in}}{\pgfqpoint{2.194419in}{2.113116in}}%
\pgfpathclose%
\pgfusepath{stroke,fill}%
\end{pgfscope}%
\begin{pgfscope}%
\pgfpathrectangle{\pgfqpoint{0.100000in}{0.212622in}}{\pgfqpoint{3.696000in}{3.696000in}}%
\pgfusepath{clip}%
\pgfsetbuttcap%
\pgfsetroundjoin%
\definecolor{currentfill}{rgb}{0.121569,0.466667,0.705882}%
\pgfsetfillcolor{currentfill}%
\pgfsetfillopacity{0.904552}%
\pgfsetlinewidth{1.003750pt}%
\definecolor{currentstroke}{rgb}{0.121569,0.466667,0.705882}%
\pgfsetstrokecolor{currentstroke}%
\pgfsetstrokeopacity{0.904552}%
\pgfsetdash{}{0pt}%
\pgfpathmoveto{\pgfqpoint{2.193353in}{2.111659in}}%
\pgfpathcurveto{\pgfqpoint{2.201589in}{2.111659in}}{\pgfqpoint{2.209489in}{2.114931in}}{\pgfqpoint{2.215313in}{2.120755in}}%
\pgfpathcurveto{\pgfqpoint{2.221137in}{2.126579in}}{\pgfqpoint{2.224409in}{2.134479in}}{\pgfqpoint{2.224409in}{2.142715in}}%
\pgfpathcurveto{\pgfqpoint{2.224409in}{2.150952in}}{\pgfqpoint{2.221137in}{2.158852in}}{\pgfqpoint{2.215313in}{2.164676in}}%
\pgfpathcurveto{\pgfqpoint{2.209489in}{2.170500in}}{\pgfqpoint{2.201589in}{2.173772in}}{\pgfqpoint{2.193353in}{2.173772in}}%
\pgfpathcurveto{\pgfqpoint{2.185116in}{2.173772in}}{\pgfqpoint{2.177216in}{2.170500in}}{\pgfqpoint{2.171392in}{2.164676in}}%
\pgfpathcurveto{\pgfqpoint{2.165568in}{2.158852in}}{\pgfqpoint{2.162296in}{2.150952in}}{\pgfqpoint{2.162296in}{2.142715in}}%
\pgfpathcurveto{\pgfqpoint{2.162296in}{2.134479in}}{\pgfqpoint{2.165568in}{2.126579in}}{\pgfqpoint{2.171392in}{2.120755in}}%
\pgfpathcurveto{\pgfqpoint{2.177216in}{2.114931in}}{\pgfqpoint{2.185116in}{2.111659in}}{\pgfqpoint{2.193353in}{2.111659in}}%
\pgfpathclose%
\pgfusepath{stroke,fill}%
\end{pgfscope}%
\begin{pgfscope}%
\pgfpathrectangle{\pgfqpoint{0.100000in}{0.212622in}}{\pgfqpoint{3.696000in}{3.696000in}}%
\pgfusepath{clip}%
\pgfsetbuttcap%
\pgfsetroundjoin%
\definecolor{currentfill}{rgb}{0.121569,0.466667,0.705882}%
\pgfsetfillcolor{currentfill}%
\pgfsetfillopacity{0.904694}%
\pgfsetlinewidth{1.003750pt}%
\definecolor{currentstroke}{rgb}{0.121569,0.466667,0.705882}%
\pgfsetstrokecolor{currentstroke}%
\pgfsetstrokeopacity{0.904694}%
\pgfsetdash{}{0pt}%
\pgfpathmoveto{\pgfqpoint{2.780078in}{1.345323in}}%
\pgfpathcurveto{\pgfqpoint{2.788314in}{1.345323in}}{\pgfqpoint{2.796214in}{1.348596in}}{\pgfqpoint{2.802038in}{1.354420in}}%
\pgfpathcurveto{\pgfqpoint{2.807862in}{1.360243in}}{\pgfqpoint{2.811135in}{1.368144in}}{\pgfqpoint{2.811135in}{1.376380in}}%
\pgfpathcurveto{\pgfqpoint{2.811135in}{1.384616in}}{\pgfqpoint{2.807862in}{1.392516in}}{\pgfqpoint{2.802038in}{1.398340in}}%
\pgfpathcurveto{\pgfqpoint{2.796214in}{1.404164in}}{\pgfqpoint{2.788314in}{1.407436in}}{\pgfqpoint{2.780078in}{1.407436in}}%
\pgfpathcurveto{\pgfqpoint{2.771842in}{1.407436in}}{\pgfqpoint{2.763942in}{1.404164in}}{\pgfqpoint{2.758118in}{1.398340in}}%
\pgfpathcurveto{\pgfqpoint{2.752294in}{1.392516in}}{\pgfqpoint{2.749022in}{1.384616in}}{\pgfqpoint{2.749022in}{1.376380in}}%
\pgfpathcurveto{\pgfqpoint{2.749022in}{1.368144in}}{\pgfqpoint{2.752294in}{1.360243in}}{\pgfqpoint{2.758118in}{1.354420in}}%
\pgfpathcurveto{\pgfqpoint{2.763942in}{1.348596in}}{\pgfqpoint{2.771842in}{1.345323in}}{\pgfqpoint{2.780078in}{1.345323in}}%
\pgfpathclose%
\pgfusepath{stroke,fill}%
\end{pgfscope}%
\begin{pgfscope}%
\pgfpathrectangle{\pgfqpoint{0.100000in}{0.212622in}}{\pgfqpoint{3.696000in}{3.696000in}}%
\pgfusepath{clip}%
\pgfsetbuttcap%
\pgfsetroundjoin%
\definecolor{currentfill}{rgb}{0.121569,0.466667,0.705882}%
\pgfsetfillcolor{currentfill}%
\pgfsetfillopacity{0.905108}%
\pgfsetlinewidth{1.003750pt}%
\definecolor{currentstroke}{rgb}{0.121569,0.466667,0.705882}%
\pgfsetstrokecolor{currentstroke}%
\pgfsetstrokeopacity{0.905108}%
\pgfsetdash{}{0pt}%
\pgfpathmoveto{\pgfqpoint{2.191681in}{2.108954in}}%
\pgfpathcurveto{\pgfqpoint{2.199917in}{2.108954in}}{\pgfqpoint{2.207817in}{2.112226in}}{\pgfqpoint{2.213641in}{2.118050in}}%
\pgfpathcurveto{\pgfqpoint{2.219465in}{2.123874in}}{\pgfqpoint{2.222738in}{2.131774in}}{\pgfqpoint{2.222738in}{2.140010in}}%
\pgfpathcurveto{\pgfqpoint{2.222738in}{2.148247in}}{\pgfqpoint{2.219465in}{2.156147in}}{\pgfqpoint{2.213641in}{2.161971in}}%
\pgfpathcurveto{\pgfqpoint{2.207817in}{2.167795in}}{\pgfqpoint{2.199917in}{2.171067in}}{\pgfqpoint{2.191681in}{2.171067in}}%
\pgfpathcurveto{\pgfqpoint{2.183445in}{2.171067in}}{\pgfqpoint{2.175545in}{2.167795in}}{\pgfqpoint{2.169721in}{2.161971in}}%
\pgfpathcurveto{\pgfqpoint{2.163897in}{2.156147in}}{\pgfqpoint{2.160625in}{2.148247in}}{\pgfqpoint{2.160625in}{2.140010in}}%
\pgfpathcurveto{\pgfqpoint{2.160625in}{2.131774in}}{\pgfqpoint{2.163897in}{2.123874in}}{\pgfqpoint{2.169721in}{2.118050in}}%
\pgfpathcurveto{\pgfqpoint{2.175545in}{2.112226in}}{\pgfqpoint{2.183445in}{2.108954in}}{\pgfqpoint{2.191681in}{2.108954in}}%
\pgfpathclose%
\pgfusepath{stroke,fill}%
\end{pgfscope}%
\begin{pgfscope}%
\pgfpathrectangle{\pgfqpoint{0.100000in}{0.212622in}}{\pgfqpoint{3.696000in}{3.696000in}}%
\pgfusepath{clip}%
\pgfsetbuttcap%
\pgfsetroundjoin%
\definecolor{currentfill}{rgb}{0.121569,0.466667,0.705882}%
\pgfsetfillcolor{currentfill}%
\pgfsetfillopacity{0.905385}%
\pgfsetlinewidth{1.003750pt}%
\definecolor{currentstroke}{rgb}{0.121569,0.466667,0.705882}%
\pgfsetstrokecolor{currentstroke}%
\pgfsetstrokeopacity{0.905385}%
\pgfsetdash{}{0pt}%
\pgfpathmoveto{\pgfqpoint{1.803675in}{1.980613in}}%
\pgfpathcurveto{\pgfqpoint{1.811911in}{1.980613in}}{\pgfqpoint{1.819812in}{1.983886in}}{\pgfqpoint{1.825635in}{1.989710in}}%
\pgfpathcurveto{\pgfqpoint{1.831459in}{1.995534in}}{\pgfqpoint{1.834732in}{2.003434in}}{\pgfqpoint{1.834732in}{2.011670in}}%
\pgfpathcurveto{\pgfqpoint{1.834732in}{2.019906in}}{\pgfqpoint{1.831459in}{2.027806in}}{\pgfqpoint{1.825635in}{2.033630in}}%
\pgfpathcurveto{\pgfqpoint{1.819812in}{2.039454in}}{\pgfqpoint{1.811911in}{2.042726in}}{\pgfqpoint{1.803675in}{2.042726in}}%
\pgfpathcurveto{\pgfqpoint{1.795439in}{2.042726in}}{\pgfqpoint{1.787539in}{2.039454in}}{\pgfqpoint{1.781715in}{2.033630in}}%
\pgfpathcurveto{\pgfqpoint{1.775891in}{2.027806in}}{\pgfqpoint{1.772619in}{2.019906in}}{\pgfqpoint{1.772619in}{2.011670in}}%
\pgfpathcurveto{\pgfqpoint{1.772619in}{2.003434in}}{\pgfqpoint{1.775891in}{1.995534in}}{\pgfqpoint{1.781715in}{1.989710in}}%
\pgfpathcurveto{\pgfqpoint{1.787539in}{1.983886in}}{\pgfqpoint{1.795439in}{1.980613in}}{\pgfqpoint{1.803675in}{1.980613in}}%
\pgfpathclose%
\pgfusepath{stroke,fill}%
\end{pgfscope}%
\begin{pgfscope}%
\pgfpathrectangle{\pgfqpoint{0.100000in}{0.212622in}}{\pgfqpoint{3.696000in}{3.696000in}}%
\pgfusepath{clip}%
\pgfsetbuttcap%
\pgfsetroundjoin%
\definecolor{currentfill}{rgb}{0.121569,0.466667,0.705882}%
\pgfsetfillcolor{currentfill}%
\pgfsetfillopacity{0.905520}%
\pgfsetlinewidth{1.003750pt}%
\definecolor{currentstroke}{rgb}{0.121569,0.466667,0.705882}%
\pgfsetstrokecolor{currentstroke}%
\pgfsetstrokeopacity{0.905520}%
\pgfsetdash{}{0pt}%
\pgfpathmoveto{\pgfqpoint{1.744824in}{1.109572in}}%
\pgfpathcurveto{\pgfqpoint{1.753060in}{1.109572in}}{\pgfqpoint{1.760960in}{1.112844in}}{\pgfqpoint{1.766784in}{1.118668in}}%
\pgfpathcurveto{\pgfqpoint{1.772608in}{1.124492in}}{\pgfqpoint{1.775880in}{1.132392in}}{\pgfqpoint{1.775880in}{1.140628in}}%
\pgfpathcurveto{\pgfqpoint{1.775880in}{1.148864in}}{\pgfqpoint{1.772608in}{1.156764in}}{\pgfqpoint{1.766784in}{1.162588in}}%
\pgfpathcurveto{\pgfqpoint{1.760960in}{1.168412in}}{\pgfqpoint{1.753060in}{1.171685in}}{\pgfqpoint{1.744824in}{1.171685in}}%
\pgfpathcurveto{\pgfqpoint{1.736587in}{1.171685in}}{\pgfqpoint{1.728687in}{1.168412in}}{\pgfqpoint{1.722863in}{1.162588in}}%
\pgfpathcurveto{\pgfqpoint{1.717040in}{1.156764in}}{\pgfqpoint{1.713767in}{1.148864in}}{\pgfqpoint{1.713767in}{1.140628in}}%
\pgfpathcurveto{\pgfqpoint{1.713767in}{1.132392in}}{\pgfqpoint{1.717040in}{1.124492in}}{\pgfqpoint{1.722863in}{1.118668in}}%
\pgfpathcurveto{\pgfqpoint{1.728687in}{1.112844in}}{\pgfqpoint{1.736587in}{1.109572in}}{\pgfqpoint{1.744824in}{1.109572in}}%
\pgfpathclose%
\pgfusepath{stroke,fill}%
\end{pgfscope}%
\begin{pgfscope}%
\pgfpathrectangle{\pgfqpoint{0.100000in}{0.212622in}}{\pgfqpoint{3.696000in}{3.696000in}}%
\pgfusepath{clip}%
\pgfsetbuttcap%
\pgfsetroundjoin%
\definecolor{currentfill}{rgb}{0.121569,0.466667,0.705882}%
\pgfsetfillcolor{currentfill}%
\pgfsetfillopacity{0.905586}%
\pgfsetlinewidth{1.003750pt}%
\definecolor{currentstroke}{rgb}{0.121569,0.466667,0.705882}%
\pgfsetstrokecolor{currentstroke}%
\pgfsetstrokeopacity{0.905586}%
\pgfsetdash{}{0pt}%
\pgfpathmoveto{\pgfqpoint{2.190553in}{2.106214in}}%
\pgfpathcurveto{\pgfqpoint{2.198789in}{2.106214in}}{\pgfqpoint{2.206690in}{2.109487in}}{\pgfqpoint{2.212513in}{2.115311in}}%
\pgfpathcurveto{\pgfqpoint{2.218337in}{2.121134in}}{\pgfqpoint{2.221610in}{2.129034in}}{\pgfqpoint{2.221610in}{2.137271in}}%
\pgfpathcurveto{\pgfqpoint{2.221610in}{2.145507in}}{\pgfqpoint{2.218337in}{2.153407in}}{\pgfqpoint{2.212513in}{2.159231in}}%
\pgfpathcurveto{\pgfqpoint{2.206690in}{2.165055in}}{\pgfqpoint{2.198789in}{2.168327in}}{\pgfqpoint{2.190553in}{2.168327in}}%
\pgfpathcurveto{\pgfqpoint{2.182317in}{2.168327in}}{\pgfqpoint{2.174417in}{2.165055in}}{\pgfqpoint{2.168593in}{2.159231in}}%
\pgfpathcurveto{\pgfqpoint{2.162769in}{2.153407in}}{\pgfqpoint{2.159497in}{2.145507in}}{\pgfqpoint{2.159497in}{2.137271in}}%
\pgfpathcurveto{\pgfqpoint{2.159497in}{2.129034in}}{\pgfqpoint{2.162769in}{2.121134in}}{\pgfqpoint{2.168593in}{2.115311in}}%
\pgfpathcurveto{\pgfqpoint{2.174417in}{2.109487in}}{\pgfqpoint{2.182317in}{2.106214in}}{\pgfqpoint{2.190553in}{2.106214in}}%
\pgfpathclose%
\pgfusepath{stroke,fill}%
\end{pgfscope}%
\begin{pgfscope}%
\pgfpathrectangle{\pgfqpoint{0.100000in}{0.212622in}}{\pgfqpoint{3.696000in}{3.696000in}}%
\pgfusepath{clip}%
\pgfsetbuttcap%
\pgfsetroundjoin%
\definecolor{currentfill}{rgb}{0.121569,0.466667,0.705882}%
\pgfsetfillcolor{currentfill}%
\pgfsetfillopacity{0.906013}%
\pgfsetlinewidth{1.003750pt}%
\definecolor{currentstroke}{rgb}{0.121569,0.466667,0.705882}%
\pgfsetstrokecolor{currentstroke}%
\pgfsetstrokeopacity{0.906013}%
\pgfsetdash{}{0pt}%
\pgfpathmoveto{\pgfqpoint{2.189376in}{2.104178in}}%
\pgfpathcurveto{\pgfqpoint{2.197612in}{2.104178in}}{\pgfqpoint{2.205512in}{2.107450in}}{\pgfqpoint{2.211336in}{2.113274in}}%
\pgfpathcurveto{\pgfqpoint{2.217160in}{2.119098in}}{\pgfqpoint{2.220432in}{2.126998in}}{\pgfqpoint{2.220432in}{2.135234in}}%
\pgfpathcurveto{\pgfqpoint{2.220432in}{2.143471in}}{\pgfqpoint{2.217160in}{2.151371in}}{\pgfqpoint{2.211336in}{2.157195in}}%
\pgfpathcurveto{\pgfqpoint{2.205512in}{2.163019in}}{\pgfqpoint{2.197612in}{2.166291in}}{\pgfqpoint{2.189376in}{2.166291in}}%
\pgfpathcurveto{\pgfqpoint{2.181139in}{2.166291in}}{\pgfqpoint{2.173239in}{2.163019in}}{\pgfqpoint{2.167415in}{2.157195in}}%
\pgfpathcurveto{\pgfqpoint{2.161591in}{2.151371in}}{\pgfqpoint{2.158319in}{2.143471in}}{\pgfqpoint{2.158319in}{2.135234in}}%
\pgfpathcurveto{\pgfqpoint{2.158319in}{2.126998in}}{\pgfqpoint{2.161591in}{2.119098in}}{\pgfqpoint{2.167415in}{2.113274in}}%
\pgfpathcurveto{\pgfqpoint{2.173239in}{2.107450in}}{\pgfqpoint{2.181139in}{2.104178in}}{\pgfqpoint{2.189376in}{2.104178in}}%
\pgfpathclose%
\pgfusepath{stroke,fill}%
\end{pgfscope}%
\begin{pgfscope}%
\pgfpathrectangle{\pgfqpoint{0.100000in}{0.212622in}}{\pgfqpoint{3.696000in}{3.696000in}}%
\pgfusepath{clip}%
\pgfsetbuttcap%
\pgfsetroundjoin%
\definecolor{currentfill}{rgb}{0.121569,0.466667,0.705882}%
\pgfsetfillcolor{currentfill}%
\pgfsetfillopacity{0.906200}%
\pgfsetlinewidth{1.003750pt}%
\definecolor{currentstroke}{rgb}{0.121569,0.466667,0.705882}%
\pgfsetstrokecolor{currentstroke}%
\pgfsetstrokeopacity{0.906200}%
\pgfsetdash{}{0pt}%
\pgfpathmoveto{\pgfqpoint{2.188701in}{2.103222in}}%
\pgfpathcurveto{\pgfqpoint{2.196937in}{2.103222in}}{\pgfqpoint{2.204837in}{2.106494in}}{\pgfqpoint{2.210661in}{2.112318in}}%
\pgfpathcurveto{\pgfqpoint{2.216485in}{2.118142in}}{\pgfqpoint{2.219758in}{2.126042in}}{\pgfqpoint{2.219758in}{2.134278in}}%
\pgfpathcurveto{\pgfqpoint{2.219758in}{2.142514in}}{\pgfqpoint{2.216485in}{2.150415in}}{\pgfqpoint{2.210661in}{2.156238in}}%
\pgfpathcurveto{\pgfqpoint{2.204837in}{2.162062in}}{\pgfqpoint{2.196937in}{2.165335in}}{\pgfqpoint{2.188701in}{2.165335in}}%
\pgfpathcurveto{\pgfqpoint{2.180465in}{2.165335in}}{\pgfqpoint{2.172565in}{2.162062in}}{\pgfqpoint{2.166741in}{2.156238in}}%
\pgfpathcurveto{\pgfqpoint{2.160917in}{2.150415in}}{\pgfqpoint{2.157645in}{2.142514in}}{\pgfqpoint{2.157645in}{2.134278in}}%
\pgfpathcurveto{\pgfqpoint{2.157645in}{2.126042in}}{\pgfqpoint{2.160917in}{2.118142in}}{\pgfqpoint{2.166741in}{2.112318in}}%
\pgfpathcurveto{\pgfqpoint{2.172565in}{2.106494in}}{\pgfqpoint{2.180465in}{2.103222in}}{\pgfqpoint{2.188701in}{2.103222in}}%
\pgfpathclose%
\pgfusepath{stroke,fill}%
\end{pgfscope}%
\begin{pgfscope}%
\pgfpathrectangle{\pgfqpoint{0.100000in}{0.212622in}}{\pgfqpoint{3.696000in}{3.696000in}}%
\pgfusepath{clip}%
\pgfsetbuttcap%
\pgfsetroundjoin%
\definecolor{currentfill}{rgb}{0.121569,0.466667,0.705882}%
\pgfsetfillcolor{currentfill}%
\pgfsetfillopacity{0.906537}%
\pgfsetlinewidth{1.003750pt}%
\definecolor{currentstroke}{rgb}{0.121569,0.466667,0.705882}%
\pgfsetstrokecolor{currentstroke}%
\pgfsetstrokeopacity{0.906537}%
\pgfsetdash{}{0pt}%
\pgfpathmoveto{\pgfqpoint{2.187467in}{2.101489in}}%
\pgfpathcurveto{\pgfqpoint{2.195703in}{2.101489in}}{\pgfqpoint{2.203603in}{2.104761in}}{\pgfqpoint{2.209427in}{2.110585in}}%
\pgfpathcurveto{\pgfqpoint{2.215251in}{2.116409in}}{\pgfqpoint{2.218523in}{2.124309in}}{\pgfqpoint{2.218523in}{2.132546in}}%
\pgfpathcurveto{\pgfqpoint{2.218523in}{2.140782in}}{\pgfqpoint{2.215251in}{2.148682in}}{\pgfqpoint{2.209427in}{2.154506in}}%
\pgfpathcurveto{\pgfqpoint{2.203603in}{2.160330in}}{\pgfqpoint{2.195703in}{2.163602in}}{\pgfqpoint{2.187467in}{2.163602in}}%
\pgfpathcurveto{\pgfqpoint{2.179231in}{2.163602in}}{\pgfqpoint{2.171331in}{2.160330in}}{\pgfqpoint{2.165507in}{2.154506in}}%
\pgfpathcurveto{\pgfqpoint{2.159683in}{2.148682in}}{\pgfqpoint{2.156410in}{2.140782in}}{\pgfqpoint{2.156410in}{2.132546in}}%
\pgfpathcurveto{\pgfqpoint{2.156410in}{2.124309in}}{\pgfqpoint{2.159683in}{2.116409in}}{\pgfqpoint{2.165507in}{2.110585in}}%
\pgfpathcurveto{\pgfqpoint{2.171331in}{2.104761in}}{\pgfqpoint{2.179231in}{2.101489in}}{\pgfqpoint{2.187467in}{2.101489in}}%
\pgfpathclose%
\pgfusepath{stroke,fill}%
\end{pgfscope}%
\begin{pgfscope}%
\pgfpathrectangle{\pgfqpoint{0.100000in}{0.212622in}}{\pgfqpoint{3.696000in}{3.696000in}}%
\pgfusepath{clip}%
\pgfsetbuttcap%
\pgfsetroundjoin%
\definecolor{currentfill}{rgb}{0.121569,0.466667,0.705882}%
\pgfsetfillcolor{currentfill}%
\pgfsetfillopacity{0.906744}%
\pgfsetlinewidth{1.003750pt}%
\definecolor{currentstroke}{rgb}{0.121569,0.466667,0.705882}%
\pgfsetstrokecolor{currentstroke}%
\pgfsetstrokeopacity{0.906744}%
\pgfsetdash{}{0pt}%
\pgfpathmoveto{\pgfqpoint{1.809527in}{1.976972in}}%
\pgfpathcurveto{\pgfqpoint{1.817764in}{1.976972in}}{\pgfqpoint{1.825664in}{1.980245in}}{\pgfqpoint{1.831488in}{1.986069in}}%
\pgfpathcurveto{\pgfqpoint{1.837312in}{1.991892in}}{\pgfqpoint{1.840584in}{1.999793in}}{\pgfqpoint{1.840584in}{2.008029in}}%
\pgfpathcurveto{\pgfqpoint{1.840584in}{2.016265in}}{\pgfqpoint{1.837312in}{2.024165in}}{\pgfqpoint{1.831488in}{2.029989in}}%
\pgfpathcurveto{\pgfqpoint{1.825664in}{2.035813in}}{\pgfqpoint{1.817764in}{2.039085in}}{\pgfqpoint{1.809527in}{2.039085in}}%
\pgfpathcurveto{\pgfqpoint{1.801291in}{2.039085in}}{\pgfqpoint{1.793391in}{2.035813in}}{\pgfqpoint{1.787567in}{2.029989in}}%
\pgfpathcurveto{\pgfqpoint{1.781743in}{2.024165in}}{\pgfqpoint{1.778471in}{2.016265in}}{\pgfqpoint{1.778471in}{2.008029in}}%
\pgfpathcurveto{\pgfqpoint{1.778471in}{1.999793in}}{\pgfqpoint{1.781743in}{1.991892in}}{\pgfqpoint{1.787567in}{1.986069in}}%
\pgfpathcurveto{\pgfqpoint{1.793391in}{1.980245in}}{\pgfqpoint{1.801291in}{1.976972in}}{\pgfqpoint{1.809527in}{1.976972in}}%
\pgfpathclose%
\pgfusepath{stroke,fill}%
\end{pgfscope}%
\begin{pgfscope}%
\pgfpathrectangle{\pgfqpoint{0.100000in}{0.212622in}}{\pgfqpoint{3.696000in}{3.696000in}}%
\pgfusepath{clip}%
\pgfsetbuttcap%
\pgfsetroundjoin%
\definecolor{currentfill}{rgb}{0.121569,0.466667,0.705882}%
\pgfsetfillcolor{currentfill}%
\pgfsetfillopacity{0.906794}%
\pgfsetlinewidth{1.003750pt}%
\definecolor{currentstroke}{rgb}{0.121569,0.466667,0.705882}%
\pgfsetstrokecolor{currentstroke}%
\pgfsetstrokeopacity{0.906794}%
\pgfsetdash{}{0pt}%
\pgfpathmoveto{\pgfqpoint{2.186894in}{2.100140in}}%
\pgfpathcurveto{\pgfqpoint{2.195130in}{2.100140in}}{\pgfqpoint{2.203030in}{2.103413in}}{\pgfqpoint{2.208854in}{2.109237in}}%
\pgfpathcurveto{\pgfqpoint{2.214678in}{2.115061in}}{\pgfqpoint{2.217951in}{2.122961in}}{\pgfqpoint{2.217951in}{2.131197in}}%
\pgfpathcurveto{\pgfqpoint{2.217951in}{2.139433in}}{\pgfqpoint{2.214678in}{2.147333in}}{\pgfqpoint{2.208854in}{2.153157in}}%
\pgfpathcurveto{\pgfqpoint{2.203030in}{2.158981in}}{\pgfqpoint{2.195130in}{2.162253in}}{\pgfqpoint{2.186894in}{2.162253in}}%
\pgfpathcurveto{\pgfqpoint{2.178658in}{2.162253in}}{\pgfqpoint{2.170758in}{2.158981in}}{\pgfqpoint{2.164934in}{2.153157in}}%
\pgfpathcurveto{\pgfqpoint{2.159110in}{2.147333in}}{\pgfqpoint{2.155838in}{2.139433in}}{\pgfqpoint{2.155838in}{2.131197in}}%
\pgfpathcurveto{\pgfqpoint{2.155838in}{2.122961in}}{\pgfqpoint{2.159110in}{2.115061in}}{\pgfqpoint{2.164934in}{2.109237in}}%
\pgfpathcurveto{\pgfqpoint{2.170758in}{2.103413in}}{\pgfqpoint{2.178658in}{2.100140in}}{\pgfqpoint{2.186894in}{2.100140in}}%
\pgfpathclose%
\pgfusepath{stroke,fill}%
\end{pgfscope}%
\begin{pgfscope}%
\pgfpathrectangle{\pgfqpoint{0.100000in}{0.212622in}}{\pgfqpoint{3.696000in}{3.696000in}}%
\pgfusepath{clip}%
\pgfsetbuttcap%
\pgfsetroundjoin%
\definecolor{currentfill}{rgb}{0.121569,0.466667,0.705882}%
\pgfsetfillcolor{currentfill}%
\pgfsetfillopacity{0.906917}%
\pgfsetlinewidth{1.003750pt}%
\definecolor{currentstroke}{rgb}{0.121569,0.466667,0.705882}%
\pgfsetstrokecolor{currentstroke}%
\pgfsetstrokeopacity{0.906917}%
\pgfsetdash{}{0pt}%
\pgfpathmoveto{\pgfqpoint{1.765631in}{1.096253in}}%
\pgfpathcurveto{\pgfqpoint{1.773867in}{1.096253in}}{\pgfqpoint{1.781767in}{1.099526in}}{\pgfqpoint{1.787591in}{1.105350in}}%
\pgfpathcurveto{\pgfqpoint{1.793415in}{1.111174in}}{\pgfqpoint{1.796687in}{1.119074in}}{\pgfqpoint{1.796687in}{1.127310in}}%
\pgfpathcurveto{\pgfqpoint{1.796687in}{1.135546in}}{\pgfqpoint{1.793415in}{1.143446in}}{\pgfqpoint{1.787591in}{1.149270in}}%
\pgfpathcurveto{\pgfqpoint{1.781767in}{1.155094in}}{\pgfqpoint{1.773867in}{1.158366in}}{\pgfqpoint{1.765631in}{1.158366in}}%
\pgfpathcurveto{\pgfqpoint{1.757395in}{1.158366in}}{\pgfqpoint{1.749494in}{1.155094in}}{\pgfqpoint{1.743671in}{1.149270in}}%
\pgfpathcurveto{\pgfqpoint{1.737847in}{1.143446in}}{\pgfqpoint{1.734574in}{1.135546in}}{\pgfqpoint{1.734574in}{1.127310in}}%
\pgfpathcurveto{\pgfqpoint{1.734574in}{1.119074in}}{\pgfqpoint{1.737847in}{1.111174in}}{\pgfqpoint{1.743671in}{1.105350in}}%
\pgfpathcurveto{\pgfqpoint{1.749494in}{1.099526in}}{\pgfqpoint{1.757395in}{1.096253in}}{\pgfqpoint{1.765631in}{1.096253in}}%
\pgfpathclose%
\pgfusepath{stroke,fill}%
\end{pgfscope}%
\begin{pgfscope}%
\pgfpathrectangle{\pgfqpoint{0.100000in}{0.212622in}}{\pgfqpoint{3.696000in}{3.696000in}}%
\pgfusepath{clip}%
\pgfsetbuttcap%
\pgfsetroundjoin%
\definecolor{currentfill}{rgb}{0.121569,0.466667,0.705882}%
\pgfsetfillcolor{currentfill}%
\pgfsetfillopacity{0.906965}%
\pgfsetlinewidth{1.003750pt}%
\definecolor{currentstroke}{rgb}{0.121569,0.466667,0.705882}%
\pgfsetstrokecolor{currentstroke}%
\pgfsetstrokeopacity{0.906965}%
\pgfsetdash{}{0pt}%
\pgfpathmoveto{\pgfqpoint{2.186474in}{2.099299in}}%
\pgfpathcurveto{\pgfqpoint{2.194710in}{2.099299in}}{\pgfqpoint{2.202610in}{2.102571in}}{\pgfqpoint{2.208434in}{2.108395in}}%
\pgfpathcurveto{\pgfqpoint{2.214258in}{2.114219in}}{\pgfqpoint{2.217530in}{2.122119in}}{\pgfqpoint{2.217530in}{2.130355in}}%
\pgfpathcurveto{\pgfqpoint{2.217530in}{2.138592in}}{\pgfqpoint{2.214258in}{2.146492in}}{\pgfqpoint{2.208434in}{2.152316in}}%
\pgfpathcurveto{\pgfqpoint{2.202610in}{2.158140in}}{\pgfqpoint{2.194710in}{2.161412in}}{\pgfqpoint{2.186474in}{2.161412in}}%
\pgfpathcurveto{\pgfqpoint{2.178238in}{2.161412in}}{\pgfqpoint{2.170338in}{2.158140in}}{\pgfqpoint{2.164514in}{2.152316in}}%
\pgfpathcurveto{\pgfqpoint{2.158690in}{2.146492in}}{\pgfqpoint{2.155417in}{2.138592in}}{\pgfqpoint{2.155417in}{2.130355in}}%
\pgfpathcurveto{\pgfqpoint{2.155417in}{2.122119in}}{\pgfqpoint{2.158690in}{2.114219in}}{\pgfqpoint{2.164514in}{2.108395in}}%
\pgfpathcurveto{\pgfqpoint{2.170338in}{2.102571in}}{\pgfqpoint{2.178238in}{2.099299in}}{\pgfqpoint{2.186474in}{2.099299in}}%
\pgfpathclose%
\pgfusepath{stroke,fill}%
\end{pgfscope}%
\begin{pgfscope}%
\pgfpathrectangle{\pgfqpoint{0.100000in}{0.212622in}}{\pgfqpoint{3.696000in}{3.696000in}}%
\pgfusepath{clip}%
\pgfsetbuttcap%
\pgfsetroundjoin%
\definecolor{currentfill}{rgb}{0.121569,0.466667,0.705882}%
\pgfsetfillcolor{currentfill}%
\pgfsetfillopacity{0.907037}%
\pgfsetlinewidth{1.003750pt}%
\definecolor{currentstroke}{rgb}{0.121569,0.466667,0.705882}%
\pgfsetstrokecolor{currentstroke}%
\pgfsetstrokeopacity{0.907037}%
\pgfsetdash{}{0pt}%
\pgfpathmoveto{\pgfqpoint{2.186204in}{2.098922in}}%
\pgfpathcurveto{\pgfqpoint{2.194441in}{2.098922in}}{\pgfqpoint{2.202341in}{2.102194in}}{\pgfqpoint{2.208165in}{2.108018in}}%
\pgfpathcurveto{\pgfqpoint{2.213989in}{2.113842in}}{\pgfqpoint{2.217261in}{2.121742in}}{\pgfqpoint{2.217261in}{2.129978in}}%
\pgfpathcurveto{\pgfqpoint{2.217261in}{2.138214in}}{\pgfqpoint{2.213989in}{2.146115in}}{\pgfqpoint{2.208165in}{2.151938in}}%
\pgfpathcurveto{\pgfqpoint{2.202341in}{2.157762in}}{\pgfqpoint{2.194441in}{2.161035in}}{\pgfqpoint{2.186204in}{2.161035in}}%
\pgfpathcurveto{\pgfqpoint{2.177968in}{2.161035in}}{\pgfqpoint{2.170068in}{2.157762in}}{\pgfqpoint{2.164244in}{2.151938in}}%
\pgfpathcurveto{\pgfqpoint{2.158420in}{2.146115in}}{\pgfqpoint{2.155148in}{2.138214in}}{\pgfqpoint{2.155148in}{2.129978in}}%
\pgfpathcurveto{\pgfqpoint{2.155148in}{2.121742in}}{\pgfqpoint{2.158420in}{2.113842in}}{\pgfqpoint{2.164244in}{2.108018in}}%
\pgfpathcurveto{\pgfqpoint{2.170068in}{2.102194in}}{\pgfqpoint{2.177968in}{2.098922in}}{\pgfqpoint{2.186204in}{2.098922in}}%
\pgfpathclose%
\pgfusepath{stroke,fill}%
\end{pgfscope}%
\begin{pgfscope}%
\pgfpathrectangle{\pgfqpoint{0.100000in}{0.212622in}}{\pgfqpoint{3.696000in}{3.696000in}}%
\pgfusepath{clip}%
\pgfsetbuttcap%
\pgfsetroundjoin%
\definecolor{currentfill}{rgb}{0.121569,0.466667,0.705882}%
\pgfsetfillcolor{currentfill}%
\pgfsetfillopacity{0.907162}%
\pgfsetlinewidth{1.003750pt}%
\definecolor{currentstroke}{rgb}{0.121569,0.466667,0.705882}%
\pgfsetstrokecolor{currentstroke}%
\pgfsetstrokeopacity{0.907162}%
\pgfsetdash{}{0pt}%
\pgfpathmoveto{\pgfqpoint{2.185712in}{2.098218in}}%
\pgfpathcurveto{\pgfqpoint{2.193948in}{2.098218in}}{\pgfqpoint{2.201849in}{2.101490in}}{\pgfqpoint{2.207672in}{2.107314in}}%
\pgfpathcurveto{\pgfqpoint{2.213496in}{2.113138in}}{\pgfqpoint{2.216769in}{2.121038in}}{\pgfqpoint{2.216769in}{2.129274in}}%
\pgfpathcurveto{\pgfqpoint{2.216769in}{2.137510in}}{\pgfqpoint{2.213496in}{2.145410in}}{\pgfqpoint{2.207672in}{2.151234in}}%
\pgfpathcurveto{\pgfqpoint{2.201849in}{2.157058in}}{\pgfqpoint{2.193948in}{2.160331in}}{\pgfqpoint{2.185712in}{2.160331in}}%
\pgfpathcurveto{\pgfqpoint{2.177476in}{2.160331in}}{\pgfqpoint{2.169576in}{2.157058in}}{\pgfqpoint{2.163752in}{2.151234in}}%
\pgfpathcurveto{\pgfqpoint{2.157928in}{2.145410in}}{\pgfqpoint{2.154656in}{2.137510in}}{\pgfqpoint{2.154656in}{2.129274in}}%
\pgfpathcurveto{\pgfqpoint{2.154656in}{2.121038in}}{\pgfqpoint{2.157928in}{2.113138in}}{\pgfqpoint{2.163752in}{2.107314in}}%
\pgfpathcurveto{\pgfqpoint{2.169576in}{2.101490in}}{\pgfqpoint{2.177476in}{2.098218in}}{\pgfqpoint{2.185712in}{2.098218in}}%
\pgfpathclose%
\pgfusepath{stroke,fill}%
\end{pgfscope}%
\begin{pgfscope}%
\pgfpathrectangle{\pgfqpoint{0.100000in}{0.212622in}}{\pgfqpoint{3.696000in}{3.696000in}}%
\pgfusepath{clip}%
\pgfsetbuttcap%
\pgfsetroundjoin%
\definecolor{currentfill}{rgb}{0.121569,0.466667,0.705882}%
\pgfsetfillcolor{currentfill}%
\pgfsetfillopacity{0.907417}%
\pgfsetlinewidth{1.003750pt}%
\definecolor{currentstroke}{rgb}{0.121569,0.466667,0.705882}%
\pgfsetstrokecolor{currentstroke}%
\pgfsetstrokeopacity{0.907417}%
\pgfsetdash{}{0pt}%
\pgfpathmoveto{\pgfqpoint{2.184961in}{2.096827in}}%
\pgfpathcurveto{\pgfqpoint{2.193197in}{2.096827in}}{\pgfqpoint{2.201097in}{2.100100in}}{\pgfqpoint{2.206921in}{2.105924in}}%
\pgfpathcurveto{\pgfqpoint{2.212745in}{2.111748in}}{\pgfqpoint{2.216017in}{2.119648in}}{\pgfqpoint{2.216017in}{2.127884in}}%
\pgfpathcurveto{\pgfqpoint{2.216017in}{2.136120in}}{\pgfqpoint{2.212745in}{2.144020in}}{\pgfqpoint{2.206921in}{2.149844in}}%
\pgfpathcurveto{\pgfqpoint{2.201097in}{2.155668in}}{\pgfqpoint{2.193197in}{2.158940in}}{\pgfqpoint{2.184961in}{2.158940in}}%
\pgfpathcurveto{\pgfqpoint{2.176725in}{2.158940in}}{\pgfqpoint{2.168825in}{2.155668in}}{\pgfqpoint{2.163001in}{2.149844in}}%
\pgfpathcurveto{\pgfqpoint{2.157177in}{2.144020in}}{\pgfqpoint{2.153904in}{2.136120in}}{\pgfqpoint{2.153904in}{2.127884in}}%
\pgfpathcurveto{\pgfqpoint{2.153904in}{2.119648in}}{\pgfqpoint{2.157177in}{2.111748in}}{\pgfqpoint{2.163001in}{2.105924in}}%
\pgfpathcurveto{\pgfqpoint{2.168825in}{2.100100in}}{\pgfqpoint{2.176725in}{2.096827in}}{\pgfqpoint{2.184961in}{2.096827in}}%
\pgfpathclose%
\pgfusepath{stroke,fill}%
\end{pgfscope}%
\begin{pgfscope}%
\pgfpathrectangle{\pgfqpoint{0.100000in}{0.212622in}}{\pgfqpoint{3.696000in}{3.696000in}}%
\pgfusepath{clip}%
\pgfsetbuttcap%
\pgfsetroundjoin%
\definecolor{currentfill}{rgb}{0.121569,0.466667,0.705882}%
\pgfsetfillcolor{currentfill}%
\pgfsetfillopacity{0.907612}%
\pgfsetlinewidth{1.003750pt}%
\definecolor{currentstroke}{rgb}{0.121569,0.466667,0.705882}%
\pgfsetstrokecolor{currentstroke}%
\pgfsetstrokeopacity{0.907612}%
\pgfsetdash{}{0pt}%
\pgfpathmoveto{\pgfqpoint{2.184570in}{2.095766in}}%
\pgfpathcurveto{\pgfqpoint{2.192806in}{2.095766in}}{\pgfqpoint{2.200706in}{2.099039in}}{\pgfqpoint{2.206530in}{2.104863in}}%
\pgfpathcurveto{\pgfqpoint{2.212354in}{2.110686in}}{\pgfqpoint{2.215627in}{2.118587in}}{\pgfqpoint{2.215627in}{2.126823in}}%
\pgfpathcurveto{\pgfqpoint{2.215627in}{2.135059in}}{\pgfqpoint{2.212354in}{2.142959in}}{\pgfqpoint{2.206530in}{2.148783in}}%
\pgfpathcurveto{\pgfqpoint{2.200706in}{2.154607in}}{\pgfqpoint{2.192806in}{2.157879in}}{\pgfqpoint{2.184570in}{2.157879in}}%
\pgfpathcurveto{\pgfqpoint{2.176334in}{2.157879in}}{\pgfqpoint{2.168434in}{2.154607in}}{\pgfqpoint{2.162610in}{2.148783in}}%
\pgfpathcurveto{\pgfqpoint{2.156786in}{2.142959in}}{\pgfqpoint{2.153514in}{2.135059in}}{\pgfqpoint{2.153514in}{2.126823in}}%
\pgfpathcurveto{\pgfqpoint{2.153514in}{2.118587in}}{\pgfqpoint{2.156786in}{2.110686in}}{\pgfqpoint{2.162610in}{2.104863in}}%
\pgfpathcurveto{\pgfqpoint{2.168434in}{2.099039in}}{\pgfqpoint{2.176334in}{2.095766in}}{\pgfqpoint{2.184570in}{2.095766in}}%
\pgfpathclose%
\pgfusepath{stroke,fill}%
\end{pgfscope}%
\begin{pgfscope}%
\pgfpathrectangle{\pgfqpoint{0.100000in}{0.212622in}}{\pgfqpoint{3.696000in}{3.696000in}}%
\pgfusepath{clip}%
\pgfsetbuttcap%
\pgfsetroundjoin%
\definecolor{currentfill}{rgb}{0.121569,0.466667,0.705882}%
\pgfsetfillcolor{currentfill}%
\pgfsetfillopacity{0.907683}%
\pgfsetlinewidth{1.003750pt}%
\definecolor{currentstroke}{rgb}{0.121569,0.466667,0.705882}%
\pgfsetstrokecolor{currentstroke}%
\pgfsetstrokeopacity{0.907683}%
\pgfsetdash{}{0pt}%
\pgfpathmoveto{\pgfqpoint{2.184368in}{2.095481in}}%
\pgfpathcurveto{\pgfqpoint{2.192604in}{2.095481in}}{\pgfqpoint{2.200504in}{2.098754in}}{\pgfqpoint{2.206328in}{2.104578in}}%
\pgfpathcurveto{\pgfqpoint{2.212152in}{2.110402in}}{\pgfqpoint{2.215424in}{2.118302in}}{\pgfqpoint{2.215424in}{2.126538in}}%
\pgfpathcurveto{\pgfqpoint{2.215424in}{2.134774in}}{\pgfqpoint{2.212152in}{2.142674in}}{\pgfqpoint{2.206328in}{2.148498in}}%
\pgfpathcurveto{\pgfqpoint{2.200504in}{2.154322in}}{\pgfqpoint{2.192604in}{2.157594in}}{\pgfqpoint{2.184368in}{2.157594in}}%
\pgfpathcurveto{\pgfqpoint{2.176132in}{2.157594in}}{\pgfqpoint{2.168232in}{2.154322in}}{\pgfqpoint{2.162408in}{2.148498in}}%
\pgfpathcurveto{\pgfqpoint{2.156584in}{2.142674in}}{\pgfqpoint{2.153311in}{2.134774in}}{\pgfqpoint{2.153311in}{2.126538in}}%
\pgfpathcurveto{\pgfqpoint{2.153311in}{2.118302in}}{\pgfqpoint{2.156584in}{2.110402in}}{\pgfqpoint{2.162408in}{2.104578in}}%
\pgfpathcurveto{\pgfqpoint{2.168232in}{2.098754in}}{\pgfqpoint{2.176132in}{2.095481in}}{\pgfqpoint{2.184368in}{2.095481in}}%
\pgfpathclose%
\pgfusepath{stroke,fill}%
\end{pgfscope}%
\begin{pgfscope}%
\pgfpathrectangle{\pgfqpoint{0.100000in}{0.212622in}}{\pgfqpoint{3.696000in}{3.696000in}}%
\pgfusepath{clip}%
\pgfsetbuttcap%
\pgfsetroundjoin%
\definecolor{currentfill}{rgb}{0.121569,0.466667,0.705882}%
\pgfsetfillcolor{currentfill}%
\pgfsetfillopacity{0.907783}%
\pgfsetlinewidth{1.003750pt}%
\definecolor{currentstroke}{rgb}{0.121569,0.466667,0.705882}%
\pgfsetstrokecolor{currentstroke}%
\pgfsetstrokeopacity{0.907783}%
\pgfsetdash{}{0pt}%
\pgfpathmoveto{\pgfqpoint{2.183960in}{2.094910in}}%
\pgfpathcurveto{\pgfqpoint{2.192196in}{2.094910in}}{\pgfqpoint{2.200096in}{2.098182in}}{\pgfqpoint{2.205920in}{2.104006in}}%
\pgfpathcurveto{\pgfqpoint{2.211744in}{2.109830in}}{\pgfqpoint{2.215016in}{2.117730in}}{\pgfqpoint{2.215016in}{2.125966in}}%
\pgfpathcurveto{\pgfqpoint{2.215016in}{2.134203in}}{\pgfqpoint{2.211744in}{2.142103in}}{\pgfqpoint{2.205920in}{2.147927in}}%
\pgfpathcurveto{\pgfqpoint{2.200096in}{2.153751in}}{\pgfqpoint{2.192196in}{2.157023in}}{\pgfqpoint{2.183960in}{2.157023in}}%
\pgfpathcurveto{\pgfqpoint{2.175724in}{2.157023in}}{\pgfqpoint{2.167824in}{2.153751in}}{\pgfqpoint{2.162000in}{2.147927in}}%
\pgfpathcurveto{\pgfqpoint{2.156176in}{2.142103in}}{\pgfqpoint{2.152903in}{2.134203in}}{\pgfqpoint{2.152903in}{2.125966in}}%
\pgfpathcurveto{\pgfqpoint{2.152903in}{2.117730in}}{\pgfqpoint{2.156176in}{2.109830in}}{\pgfqpoint{2.162000in}{2.104006in}}%
\pgfpathcurveto{\pgfqpoint{2.167824in}{2.098182in}}{\pgfqpoint{2.175724in}{2.094910in}}{\pgfqpoint{2.183960in}{2.094910in}}%
\pgfpathclose%
\pgfusepath{stroke,fill}%
\end{pgfscope}%
\begin{pgfscope}%
\pgfpathrectangle{\pgfqpoint{0.100000in}{0.212622in}}{\pgfqpoint{3.696000in}{3.696000in}}%
\pgfusepath{clip}%
\pgfsetbuttcap%
\pgfsetroundjoin%
\definecolor{currentfill}{rgb}{0.121569,0.466667,0.705882}%
\pgfsetfillcolor{currentfill}%
\pgfsetfillopacity{0.907880}%
\pgfsetlinewidth{1.003750pt}%
\definecolor{currentstroke}{rgb}{0.121569,0.466667,0.705882}%
\pgfsetstrokecolor{currentstroke}%
\pgfsetstrokeopacity{0.907880}%
\pgfsetdash{}{0pt}%
\pgfpathmoveto{\pgfqpoint{1.816134in}{1.971762in}}%
\pgfpathcurveto{\pgfqpoint{1.824370in}{1.971762in}}{\pgfqpoint{1.832270in}{1.975034in}}{\pgfqpoint{1.838094in}{1.980858in}}%
\pgfpathcurveto{\pgfqpoint{1.843918in}{1.986682in}}{\pgfqpoint{1.847190in}{1.994582in}}{\pgfqpoint{1.847190in}{2.002818in}}%
\pgfpathcurveto{\pgfqpoint{1.847190in}{2.011055in}}{\pgfqpoint{1.843918in}{2.018955in}}{\pgfqpoint{1.838094in}{2.024779in}}%
\pgfpathcurveto{\pgfqpoint{1.832270in}{2.030603in}}{\pgfqpoint{1.824370in}{2.033875in}}{\pgfqpoint{1.816134in}{2.033875in}}%
\pgfpathcurveto{\pgfqpoint{1.807898in}{2.033875in}}{\pgfqpoint{1.799998in}{2.030603in}}{\pgfqpoint{1.794174in}{2.024779in}}%
\pgfpathcurveto{\pgfqpoint{1.788350in}{2.018955in}}{\pgfqpoint{1.785077in}{2.011055in}}{\pgfqpoint{1.785077in}{2.002818in}}%
\pgfpathcurveto{\pgfqpoint{1.785077in}{1.994582in}}{\pgfqpoint{1.788350in}{1.986682in}}{\pgfqpoint{1.794174in}{1.980858in}}%
\pgfpathcurveto{\pgfqpoint{1.799998in}{1.975034in}}{\pgfqpoint{1.807898in}{1.971762in}}{\pgfqpoint{1.816134in}{1.971762in}}%
\pgfpathclose%
\pgfusepath{stroke,fill}%
\end{pgfscope}%
\begin{pgfscope}%
\pgfpathrectangle{\pgfqpoint{0.100000in}{0.212622in}}{\pgfqpoint{3.696000in}{3.696000in}}%
\pgfusepath{clip}%
\pgfsetbuttcap%
\pgfsetroundjoin%
\definecolor{currentfill}{rgb}{0.121569,0.466667,0.705882}%
\pgfsetfillcolor{currentfill}%
\pgfsetfillopacity{0.908002}%
\pgfsetlinewidth{1.003750pt}%
\definecolor{currentstroke}{rgb}{0.121569,0.466667,0.705882}%
\pgfsetstrokecolor{currentstroke}%
\pgfsetstrokeopacity{0.908002}%
\pgfsetdash{}{0pt}%
\pgfpathmoveto{\pgfqpoint{2.183395in}{2.093772in}}%
\pgfpathcurveto{\pgfqpoint{2.191631in}{2.093772in}}{\pgfqpoint{2.199531in}{2.097045in}}{\pgfqpoint{2.205355in}{2.102869in}}%
\pgfpathcurveto{\pgfqpoint{2.211179in}{2.108692in}}{\pgfqpoint{2.214451in}{2.116592in}}{\pgfqpoint{2.214451in}{2.124829in}}%
\pgfpathcurveto{\pgfqpoint{2.214451in}{2.133065in}}{\pgfqpoint{2.211179in}{2.140965in}}{\pgfqpoint{2.205355in}{2.146789in}}%
\pgfpathcurveto{\pgfqpoint{2.199531in}{2.152613in}}{\pgfqpoint{2.191631in}{2.155885in}}{\pgfqpoint{2.183395in}{2.155885in}}%
\pgfpathcurveto{\pgfqpoint{2.175158in}{2.155885in}}{\pgfqpoint{2.167258in}{2.152613in}}{\pgfqpoint{2.161434in}{2.146789in}}%
\pgfpathcurveto{\pgfqpoint{2.155610in}{2.140965in}}{\pgfqpoint{2.152338in}{2.133065in}}{\pgfqpoint{2.152338in}{2.124829in}}%
\pgfpathcurveto{\pgfqpoint{2.152338in}{2.116592in}}{\pgfqpoint{2.155610in}{2.108692in}}{\pgfqpoint{2.161434in}{2.102869in}}%
\pgfpathcurveto{\pgfqpoint{2.167258in}{2.097045in}}{\pgfqpoint{2.175158in}{2.093772in}}{\pgfqpoint{2.183395in}{2.093772in}}%
\pgfpathclose%
\pgfusepath{stroke,fill}%
\end{pgfscope}%
\begin{pgfscope}%
\pgfpathrectangle{\pgfqpoint{0.100000in}{0.212622in}}{\pgfqpoint{3.696000in}{3.696000in}}%
\pgfusepath{clip}%
\pgfsetbuttcap%
\pgfsetroundjoin%
\definecolor{currentfill}{rgb}{0.121569,0.466667,0.705882}%
\pgfsetfillcolor{currentfill}%
\pgfsetfillopacity{0.908414}%
\pgfsetlinewidth{1.003750pt}%
\definecolor{currentstroke}{rgb}{0.121569,0.466667,0.705882}%
\pgfsetstrokecolor{currentstroke}%
\pgfsetstrokeopacity{0.908414}%
\pgfsetdash{}{0pt}%
\pgfpathmoveto{\pgfqpoint{2.182410in}{2.091725in}}%
\pgfpathcurveto{\pgfqpoint{2.190646in}{2.091725in}}{\pgfqpoint{2.198546in}{2.094997in}}{\pgfqpoint{2.204370in}{2.100821in}}%
\pgfpathcurveto{\pgfqpoint{2.210194in}{2.106645in}}{\pgfqpoint{2.213466in}{2.114545in}}{\pgfqpoint{2.213466in}{2.122781in}}%
\pgfpathcurveto{\pgfqpoint{2.213466in}{2.131018in}}{\pgfqpoint{2.210194in}{2.138918in}}{\pgfqpoint{2.204370in}{2.144742in}}%
\pgfpathcurveto{\pgfqpoint{2.198546in}{2.150566in}}{\pgfqpoint{2.190646in}{2.153838in}}{\pgfqpoint{2.182410in}{2.153838in}}%
\pgfpathcurveto{\pgfqpoint{2.174173in}{2.153838in}}{\pgfqpoint{2.166273in}{2.150566in}}{\pgfqpoint{2.160449in}{2.144742in}}%
\pgfpathcurveto{\pgfqpoint{2.154625in}{2.138918in}}{\pgfqpoint{2.151353in}{2.131018in}}{\pgfqpoint{2.151353in}{2.122781in}}%
\pgfpathcurveto{\pgfqpoint{2.151353in}{2.114545in}}{\pgfqpoint{2.154625in}{2.106645in}}{\pgfqpoint{2.160449in}{2.100821in}}%
\pgfpathcurveto{\pgfqpoint{2.166273in}{2.094997in}}{\pgfqpoint{2.174173in}{2.091725in}}{\pgfqpoint{2.182410in}{2.091725in}}%
\pgfpathclose%
\pgfusepath{stroke,fill}%
\end{pgfscope}%
\begin{pgfscope}%
\pgfpathrectangle{\pgfqpoint{0.100000in}{0.212622in}}{\pgfqpoint{3.696000in}{3.696000in}}%
\pgfusepath{clip}%
\pgfsetbuttcap%
\pgfsetroundjoin%
\definecolor{currentfill}{rgb}{0.121569,0.466667,0.705882}%
\pgfsetfillcolor{currentfill}%
\pgfsetfillopacity{0.908543}%
\pgfsetlinewidth{1.003750pt}%
\definecolor{currentstroke}{rgb}{0.121569,0.466667,0.705882}%
\pgfsetstrokecolor{currentstroke}%
\pgfsetstrokeopacity{0.908543}%
\pgfsetdash{}{0pt}%
\pgfpathmoveto{\pgfqpoint{1.786427in}{1.080637in}}%
\pgfpathcurveto{\pgfqpoint{1.794663in}{1.080637in}}{\pgfqpoint{1.802563in}{1.083909in}}{\pgfqpoint{1.808387in}{1.089733in}}%
\pgfpathcurveto{\pgfqpoint{1.814211in}{1.095557in}}{\pgfqpoint{1.817484in}{1.103457in}}{\pgfqpoint{1.817484in}{1.111694in}}%
\pgfpathcurveto{\pgfqpoint{1.817484in}{1.119930in}}{\pgfqpoint{1.814211in}{1.127830in}}{\pgfqpoint{1.808387in}{1.133654in}}%
\pgfpathcurveto{\pgfqpoint{1.802563in}{1.139478in}}{\pgfqpoint{1.794663in}{1.142750in}}{\pgfqpoint{1.786427in}{1.142750in}}%
\pgfpathcurveto{\pgfqpoint{1.778191in}{1.142750in}}{\pgfqpoint{1.770291in}{1.139478in}}{\pgfqpoint{1.764467in}{1.133654in}}%
\pgfpathcurveto{\pgfqpoint{1.758643in}{1.127830in}}{\pgfqpoint{1.755371in}{1.119930in}}{\pgfqpoint{1.755371in}{1.111694in}}%
\pgfpathcurveto{\pgfqpoint{1.755371in}{1.103457in}}{\pgfqpoint{1.758643in}{1.095557in}}{\pgfqpoint{1.764467in}{1.089733in}}%
\pgfpathcurveto{\pgfqpoint{1.770291in}{1.083909in}}{\pgfqpoint{1.778191in}{1.080637in}}{\pgfqpoint{1.786427in}{1.080637in}}%
\pgfpathclose%
\pgfusepath{stroke,fill}%
\end{pgfscope}%
\begin{pgfscope}%
\pgfpathrectangle{\pgfqpoint{0.100000in}{0.212622in}}{\pgfqpoint{3.696000in}{3.696000in}}%
\pgfusepath{clip}%
\pgfsetbuttcap%
\pgfsetroundjoin%
\definecolor{currentfill}{rgb}{0.121569,0.466667,0.705882}%
\pgfsetfillcolor{currentfill}%
\pgfsetfillopacity{0.908555}%
\pgfsetlinewidth{1.003750pt}%
\definecolor{currentstroke}{rgb}{0.121569,0.466667,0.705882}%
\pgfsetstrokecolor{currentstroke}%
\pgfsetstrokeopacity{0.908555}%
\pgfsetdash{}{0pt}%
\pgfpathmoveto{\pgfqpoint{2.181845in}{2.090962in}}%
\pgfpathcurveto{\pgfqpoint{2.190081in}{2.090962in}}{\pgfqpoint{2.197981in}{2.094234in}}{\pgfqpoint{2.203805in}{2.100058in}}%
\pgfpathcurveto{\pgfqpoint{2.209629in}{2.105882in}}{\pgfqpoint{2.212901in}{2.113782in}}{\pgfqpoint{2.212901in}{2.122018in}}%
\pgfpathcurveto{\pgfqpoint{2.212901in}{2.130255in}}{\pgfqpoint{2.209629in}{2.138155in}}{\pgfqpoint{2.203805in}{2.143979in}}%
\pgfpathcurveto{\pgfqpoint{2.197981in}{2.149803in}}{\pgfqpoint{2.190081in}{2.153075in}}{\pgfqpoint{2.181845in}{2.153075in}}%
\pgfpathcurveto{\pgfqpoint{2.173608in}{2.153075in}}{\pgfqpoint{2.165708in}{2.149803in}}{\pgfqpoint{2.159884in}{2.143979in}}%
\pgfpathcurveto{\pgfqpoint{2.154061in}{2.138155in}}{\pgfqpoint{2.150788in}{2.130255in}}{\pgfqpoint{2.150788in}{2.122018in}}%
\pgfpathcurveto{\pgfqpoint{2.150788in}{2.113782in}}{\pgfqpoint{2.154061in}{2.105882in}}{\pgfqpoint{2.159884in}{2.100058in}}%
\pgfpathcurveto{\pgfqpoint{2.165708in}{2.094234in}}{\pgfqpoint{2.173608in}{2.090962in}}{\pgfqpoint{2.181845in}{2.090962in}}%
\pgfpathclose%
\pgfusepath{stroke,fill}%
\end{pgfscope}%
\begin{pgfscope}%
\pgfpathrectangle{\pgfqpoint{0.100000in}{0.212622in}}{\pgfqpoint{3.696000in}{3.696000in}}%
\pgfusepath{clip}%
\pgfsetbuttcap%
\pgfsetroundjoin%
\definecolor{currentfill}{rgb}{0.121569,0.466667,0.705882}%
\pgfsetfillcolor{currentfill}%
\pgfsetfillopacity{0.908809}%
\pgfsetlinewidth{1.003750pt}%
\definecolor{currentstroke}{rgb}{0.121569,0.466667,0.705882}%
\pgfsetstrokecolor{currentstroke}%
\pgfsetstrokeopacity{0.908809}%
\pgfsetdash{}{0pt}%
\pgfpathmoveto{\pgfqpoint{2.180845in}{2.089506in}}%
\pgfpathcurveto{\pgfqpoint{2.189081in}{2.089506in}}{\pgfqpoint{2.196981in}{2.092778in}}{\pgfqpoint{2.202805in}{2.098602in}}%
\pgfpathcurveto{\pgfqpoint{2.208629in}{2.104426in}}{\pgfqpoint{2.211902in}{2.112326in}}{\pgfqpoint{2.211902in}{2.120562in}}%
\pgfpathcurveto{\pgfqpoint{2.211902in}{2.128798in}}{\pgfqpoint{2.208629in}{2.136699in}}{\pgfqpoint{2.202805in}{2.142522in}}%
\pgfpathcurveto{\pgfqpoint{2.196981in}{2.148346in}}{\pgfqpoint{2.189081in}{2.151619in}}{\pgfqpoint{2.180845in}{2.151619in}}%
\pgfpathcurveto{\pgfqpoint{2.172609in}{2.151619in}}{\pgfqpoint{2.164709in}{2.148346in}}{\pgfqpoint{2.158885in}{2.142522in}}%
\pgfpathcurveto{\pgfqpoint{2.153061in}{2.136699in}}{\pgfqpoint{2.149789in}{2.128798in}}{\pgfqpoint{2.149789in}{2.120562in}}%
\pgfpathcurveto{\pgfqpoint{2.149789in}{2.112326in}}{\pgfqpoint{2.153061in}{2.104426in}}{\pgfqpoint{2.158885in}{2.098602in}}%
\pgfpathcurveto{\pgfqpoint{2.164709in}{2.092778in}}{\pgfqpoint{2.172609in}{2.089506in}}{\pgfqpoint{2.180845in}{2.089506in}}%
\pgfpathclose%
\pgfusepath{stroke,fill}%
\end{pgfscope}%
\begin{pgfscope}%
\pgfpathrectangle{\pgfqpoint{0.100000in}{0.212622in}}{\pgfqpoint{3.696000in}{3.696000in}}%
\pgfusepath{clip}%
\pgfsetbuttcap%
\pgfsetroundjoin%
\definecolor{currentfill}{rgb}{0.121569,0.466667,0.705882}%
\pgfsetfillcolor{currentfill}%
\pgfsetfillopacity{0.908978}%
\pgfsetlinewidth{1.003750pt}%
\definecolor{currentstroke}{rgb}{0.121569,0.466667,0.705882}%
\pgfsetstrokecolor{currentstroke}%
\pgfsetstrokeopacity{0.908978}%
\pgfsetdash{}{0pt}%
\pgfpathmoveto{\pgfqpoint{2.180426in}{2.088560in}}%
\pgfpathcurveto{\pgfqpoint{2.188663in}{2.088560in}}{\pgfqpoint{2.196563in}{2.091833in}}{\pgfqpoint{2.202387in}{2.097656in}}%
\pgfpathcurveto{\pgfqpoint{2.208211in}{2.103480in}}{\pgfqpoint{2.211483in}{2.111380in}}{\pgfqpoint{2.211483in}{2.119617in}}%
\pgfpathcurveto{\pgfqpoint{2.211483in}{2.127853in}}{\pgfqpoint{2.208211in}{2.135753in}}{\pgfqpoint{2.202387in}{2.141577in}}%
\pgfpathcurveto{\pgfqpoint{2.196563in}{2.147401in}}{\pgfqpoint{2.188663in}{2.150673in}}{\pgfqpoint{2.180426in}{2.150673in}}%
\pgfpathcurveto{\pgfqpoint{2.172190in}{2.150673in}}{\pgfqpoint{2.164290in}{2.147401in}}{\pgfqpoint{2.158466in}{2.141577in}}%
\pgfpathcurveto{\pgfqpoint{2.152642in}{2.135753in}}{\pgfqpoint{2.149370in}{2.127853in}}{\pgfqpoint{2.149370in}{2.119617in}}%
\pgfpathcurveto{\pgfqpoint{2.149370in}{2.111380in}}{\pgfqpoint{2.152642in}{2.103480in}}{\pgfqpoint{2.158466in}{2.097656in}}%
\pgfpathcurveto{\pgfqpoint{2.164290in}{2.091833in}}{\pgfqpoint{2.172190in}{2.088560in}}{\pgfqpoint{2.180426in}{2.088560in}}%
\pgfpathclose%
\pgfusepath{stroke,fill}%
\end{pgfscope}%
\begin{pgfscope}%
\pgfpathrectangle{\pgfqpoint{0.100000in}{0.212622in}}{\pgfqpoint{3.696000in}{3.696000in}}%
\pgfusepath{clip}%
\pgfsetbuttcap%
\pgfsetroundjoin%
\definecolor{currentfill}{rgb}{0.121569,0.466667,0.705882}%
\pgfsetfillcolor{currentfill}%
\pgfsetfillopacity{0.909286}%
\pgfsetlinewidth{1.003750pt}%
\definecolor{currentstroke}{rgb}{0.121569,0.466667,0.705882}%
\pgfsetstrokecolor{currentstroke}%
\pgfsetstrokeopacity{0.909286}%
\pgfsetdash{}{0pt}%
\pgfpathmoveto{\pgfqpoint{2.179638in}{2.086869in}}%
\pgfpathcurveto{\pgfqpoint{2.187874in}{2.086869in}}{\pgfqpoint{2.195774in}{2.090141in}}{\pgfqpoint{2.201598in}{2.095965in}}%
\pgfpathcurveto{\pgfqpoint{2.207422in}{2.101789in}}{\pgfqpoint{2.210694in}{2.109689in}}{\pgfqpoint{2.210694in}{2.117925in}}%
\pgfpathcurveto{\pgfqpoint{2.210694in}{2.126162in}}{\pgfqpoint{2.207422in}{2.134062in}}{\pgfqpoint{2.201598in}{2.139886in}}%
\pgfpathcurveto{\pgfqpoint{2.195774in}{2.145710in}}{\pgfqpoint{2.187874in}{2.148982in}}{\pgfqpoint{2.179638in}{2.148982in}}%
\pgfpathcurveto{\pgfqpoint{2.171401in}{2.148982in}}{\pgfqpoint{2.163501in}{2.145710in}}{\pgfqpoint{2.157677in}{2.139886in}}%
\pgfpathcurveto{\pgfqpoint{2.151854in}{2.134062in}}{\pgfqpoint{2.148581in}{2.126162in}}{\pgfqpoint{2.148581in}{2.117925in}}%
\pgfpathcurveto{\pgfqpoint{2.148581in}{2.109689in}}{\pgfqpoint{2.151854in}{2.101789in}}{\pgfqpoint{2.157677in}{2.095965in}}%
\pgfpathcurveto{\pgfqpoint{2.163501in}{2.090141in}}{\pgfqpoint{2.171401in}{2.086869in}}{\pgfqpoint{2.179638in}{2.086869in}}%
\pgfpathclose%
\pgfusepath{stroke,fill}%
\end{pgfscope}%
\begin{pgfscope}%
\pgfpathrectangle{\pgfqpoint{0.100000in}{0.212622in}}{\pgfqpoint{3.696000in}{3.696000in}}%
\pgfusepath{clip}%
\pgfsetbuttcap%
\pgfsetroundjoin%
\definecolor{currentfill}{rgb}{0.121569,0.466667,0.705882}%
\pgfsetfillcolor{currentfill}%
\pgfsetfillopacity{0.909404}%
\pgfsetlinewidth{1.003750pt}%
\definecolor{currentstroke}{rgb}{0.121569,0.466667,0.705882}%
\pgfsetstrokecolor{currentstroke}%
\pgfsetstrokeopacity{0.909404}%
\pgfsetdash{}{0pt}%
\pgfpathmoveto{\pgfqpoint{1.823006in}{1.967189in}}%
\pgfpathcurveto{\pgfqpoint{1.831242in}{1.967189in}}{\pgfqpoint{1.839142in}{1.970461in}}{\pgfqpoint{1.844966in}{1.976285in}}%
\pgfpathcurveto{\pgfqpoint{1.850790in}{1.982109in}}{\pgfqpoint{1.854062in}{1.990009in}}{\pgfqpoint{1.854062in}{1.998245in}}%
\pgfpathcurveto{\pgfqpoint{1.854062in}{2.006482in}}{\pgfqpoint{1.850790in}{2.014382in}}{\pgfqpoint{1.844966in}{2.020206in}}%
\pgfpathcurveto{\pgfqpoint{1.839142in}{2.026030in}}{\pgfqpoint{1.831242in}{2.029302in}}{\pgfqpoint{1.823006in}{2.029302in}}%
\pgfpathcurveto{\pgfqpoint{1.814770in}{2.029302in}}{\pgfqpoint{1.806869in}{2.026030in}}{\pgfqpoint{1.801046in}{2.020206in}}%
\pgfpathcurveto{\pgfqpoint{1.795222in}{2.014382in}}{\pgfqpoint{1.791949in}{2.006482in}}{\pgfqpoint{1.791949in}{1.998245in}}%
\pgfpathcurveto{\pgfqpoint{1.791949in}{1.990009in}}{\pgfqpoint{1.795222in}{1.982109in}}{\pgfqpoint{1.801046in}{1.976285in}}%
\pgfpathcurveto{\pgfqpoint{1.806869in}{1.970461in}}{\pgfqpoint{1.814770in}{1.967189in}}{\pgfqpoint{1.823006in}{1.967189in}}%
\pgfpathclose%
\pgfusepath{stroke,fill}%
\end{pgfscope}%
\begin{pgfscope}%
\pgfpathrectangle{\pgfqpoint{0.100000in}{0.212622in}}{\pgfqpoint{3.696000in}{3.696000in}}%
\pgfusepath{clip}%
\pgfsetbuttcap%
\pgfsetroundjoin%
\definecolor{currentfill}{rgb}{0.121569,0.466667,0.705882}%
\pgfsetfillcolor{currentfill}%
\pgfsetfillopacity{0.909409}%
\pgfsetlinewidth{1.003750pt}%
\definecolor{currentstroke}{rgb}{0.121569,0.466667,0.705882}%
\pgfsetstrokecolor{currentstroke}%
\pgfsetstrokeopacity{0.909409}%
\pgfsetdash{}{0pt}%
\pgfpathmoveto{\pgfqpoint{2.775511in}{1.314213in}}%
\pgfpathcurveto{\pgfqpoint{2.783747in}{1.314213in}}{\pgfqpoint{2.791647in}{1.317486in}}{\pgfqpoint{2.797471in}{1.323310in}}%
\pgfpathcurveto{\pgfqpoint{2.803295in}{1.329134in}}{\pgfqpoint{2.806567in}{1.337034in}}{\pgfqpoint{2.806567in}{1.345270in}}%
\pgfpathcurveto{\pgfqpoint{2.806567in}{1.353506in}}{\pgfqpoint{2.803295in}{1.361406in}}{\pgfqpoint{2.797471in}{1.367230in}}%
\pgfpathcurveto{\pgfqpoint{2.791647in}{1.373054in}}{\pgfqpoint{2.783747in}{1.376326in}}{\pgfqpoint{2.775511in}{1.376326in}}%
\pgfpathcurveto{\pgfqpoint{2.767275in}{1.376326in}}{\pgfqpoint{2.759375in}{1.373054in}}{\pgfqpoint{2.753551in}{1.367230in}}%
\pgfpathcurveto{\pgfqpoint{2.747727in}{1.361406in}}{\pgfqpoint{2.744454in}{1.353506in}}{\pgfqpoint{2.744454in}{1.345270in}}%
\pgfpathcurveto{\pgfqpoint{2.744454in}{1.337034in}}{\pgfqpoint{2.747727in}{1.329134in}}{\pgfqpoint{2.753551in}{1.323310in}}%
\pgfpathcurveto{\pgfqpoint{2.759375in}{1.317486in}}{\pgfqpoint{2.767275in}{1.314213in}}{\pgfqpoint{2.775511in}{1.314213in}}%
\pgfpathclose%
\pgfusepath{stroke,fill}%
\end{pgfscope}%
\begin{pgfscope}%
\pgfpathrectangle{\pgfqpoint{0.100000in}{0.212622in}}{\pgfqpoint{3.696000in}{3.696000in}}%
\pgfusepath{clip}%
\pgfsetbuttcap%
\pgfsetroundjoin%
\definecolor{currentfill}{rgb}{0.121569,0.466667,0.705882}%
\pgfsetfillcolor{currentfill}%
\pgfsetfillopacity{0.909437}%
\pgfsetlinewidth{1.003750pt}%
\definecolor{currentstroke}{rgb}{0.121569,0.466667,0.705882}%
\pgfsetstrokecolor{currentstroke}%
\pgfsetstrokeopacity{0.909437}%
\pgfsetdash{}{0pt}%
\pgfpathmoveto{\pgfqpoint{2.178988in}{2.086011in}}%
\pgfpathcurveto{\pgfqpoint{2.187225in}{2.086011in}}{\pgfqpoint{2.195125in}{2.089283in}}{\pgfqpoint{2.200949in}{2.095107in}}%
\pgfpathcurveto{\pgfqpoint{2.206773in}{2.100931in}}{\pgfqpoint{2.210045in}{2.108831in}}{\pgfqpoint{2.210045in}{2.117068in}}%
\pgfpathcurveto{\pgfqpoint{2.210045in}{2.125304in}}{\pgfqpoint{2.206773in}{2.133204in}}{\pgfqpoint{2.200949in}{2.139028in}}%
\pgfpathcurveto{\pgfqpoint{2.195125in}{2.144852in}}{\pgfqpoint{2.187225in}{2.148124in}}{\pgfqpoint{2.178988in}{2.148124in}}%
\pgfpathcurveto{\pgfqpoint{2.170752in}{2.148124in}}{\pgfqpoint{2.162852in}{2.144852in}}{\pgfqpoint{2.157028in}{2.139028in}}%
\pgfpathcurveto{\pgfqpoint{2.151204in}{2.133204in}}{\pgfqpoint{2.147932in}{2.125304in}}{\pgfqpoint{2.147932in}{2.117068in}}%
\pgfpathcurveto{\pgfqpoint{2.147932in}{2.108831in}}{\pgfqpoint{2.151204in}{2.100931in}}{\pgfqpoint{2.157028in}{2.095107in}}%
\pgfpathcurveto{\pgfqpoint{2.162852in}{2.089283in}}{\pgfqpoint{2.170752in}{2.086011in}}{\pgfqpoint{2.178988in}{2.086011in}}%
\pgfpathclose%
\pgfusepath{stroke,fill}%
\end{pgfscope}%
\begin{pgfscope}%
\pgfpathrectangle{\pgfqpoint{0.100000in}{0.212622in}}{\pgfqpoint{3.696000in}{3.696000in}}%
\pgfusepath{clip}%
\pgfsetbuttcap%
\pgfsetroundjoin%
\definecolor{currentfill}{rgb}{0.121569,0.466667,0.705882}%
\pgfsetfillcolor{currentfill}%
\pgfsetfillopacity{0.909683}%
\pgfsetlinewidth{1.003750pt}%
\definecolor{currentstroke}{rgb}{0.121569,0.466667,0.705882}%
\pgfsetstrokecolor{currentstroke}%
\pgfsetstrokeopacity{0.909683}%
\pgfsetdash{}{0pt}%
\pgfpathmoveto{\pgfqpoint{2.177842in}{2.084251in}}%
\pgfpathcurveto{\pgfqpoint{2.186078in}{2.084251in}}{\pgfqpoint{2.193978in}{2.087523in}}{\pgfqpoint{2.199802in}{2.093347in}}%
\pgfpathcurveto{\pgfqpoint{2.205626in}{2.099171in}}{\pgfqpoint{2.208898in}{2.107071in}}{\pgfqpoint{2.208898in}{2.115308in}}%
\pgfpathcurveto{\pgfqpoint{2.208898in}{2.123544in}}{\pgfqpoint{2.205626in}{2.131444in}}{\pgfqpoint{2.199802in}{2.137268in}}%
\pgfpathcurveto{\pgfqpoint{2.193978in}{2.143092in}}{\pgfqpoint{2.186078in}{2.146364in}}{\pgfqpoint{2.177842in}{2.146364in}}%
\pgfpathcurveto{\pgfqpoint{2.169605in}{2.146364in}}{\pgfqpoint{2.161705in}{2.143092in}}{\pgfqpoint{2.155881in}{2.137268in}}%
\pgfpathcurveto{\pgfqpoint{2.150057in}{2.131444in}}{\pgfqpoint{2.146785in}{2.123544in}}{\pgfqpoint{2.146785in}{2.115308in}}%
\pgfpathcurveto{\pgfqpoint{2.146785in}{2.107071in}}{\pgfqpoint{2.150057in}{2.099171in}}{\pgfqpoint{2.155881in}{2.093347in}}%
\pgfpathcurveto{\pgfqpoint{2.161705in}{2.087523in}}{\pgfqpoint{2.169605in}{2.084251in}}{\pgfqpoint{2.177842in}{2.084251in}}%
\pgfpathclose%
\pgfusepath{stroke,fill}%
\end{pgfscope}%
\begin{pgfscope}%
\pgfpathrectangle{\pgfqpoint{0.100000in}{0.212622in}}{\pgfqpoint{3.696000in}{3.696000in}}%
\pgfusepath{clip}%
\pgfsetbuttcap%
\pgfsetroundjoin%
\definecolor{currentfill}{rgb}{0.121569,0.466667,0.705882}%
\pgfsetfillcolor{currentfill}%
\pgfsetfillopacity{0.910291}%
\pgfsetlinewidth{1.003750pt}%
\definecolor{currentstroke}{rgb}{0.121569,0.466667,0.705882}%
\pgfsetstrokecolor{currentstroke}%
\pgfsetstrokeopacity{0.910291}%
\pgfsetdash{}{0pt}%
\pgfpathmoveto{\pgfqpoint{2.176070in}{2.081229in}}%
\pgfpathcurveto{\pgfqpoint{2.184306in}{2.081229in}}{\pgfqpoint{2.192206in}{2.084501in}}{\pgfqpoint{2.198030in}{2.090325in}}%
\pgfpathcurveto{\pgfqpoint{2.203854in}{2.096149in}}{\pgfqpoint{2.207127in}{2.104049in}}{\pgfqpoint{2.207127in}{2.112285in}}%
\pgfpathcurveto{\pgfqpoint{2.207127in}{2.120522in}}{\pgfqpoint{2.203854in}{2.128422in}}{\pgfqpoint{2.198030in}{2.134246in}}%
\pgfpathcurveto{\pgfqpoint{2.192206in}{2.140070in}}{\pgfqpoint{2.184306in}{2.143342in}}{\pgfqpoint{2.176070in}{2.143342in}}%
\pgfpathcurveto{\pgfqpoint{2.167834in}{2.143342in}}{\pgfqpoint{2.159934in}{2.140070in}}{\pgfqpoint{2.154110in}{2.134246in}}%
\pgfpathcurveto{\pgfqpoint{2.148286in}{2.128422in}}{\pgfqpoint{2.145014in}{2.120522in}}{\pgfqpoint{2.145014in}{2.112285in}}%
\pgfpathcurveto{\pgfqpoint{2.145014in}{2.104049in}}{\pgfqpoint{2.148286in}{2.096149in}}{\pgfqpoint{2.154110in}{2.090325in}}%
\pgfpathcurveto{\pgfqpoint{2.159934in}{2.084501in}}{\pgfqpoint{2.167834in}{2.081229in}}{\pgfqpoint{2.176070in}{2.081229in}}%
\pgfpathclose%
\pgfusepath{stroke,fill}%
\end{pgfscope}%
\begin{pgfscope}%
\pgfpathrectangle{\pgfqpoint{0.100000in}{0.212622in}}{\pgfqpoint{3.696000in}{3.696000in}}%
\pgfusepath{clip}%
\pgfsetbuttcap%
\pgfsetroundjoin%
\definecolor{currentfill}{rgb}{0.121569,0.466667,0.705882}%
\pgfsetfillcolor{currentfill}%
\pgfsetfillopacity{0.910332}%
\pgfsetlinewidth{1.003750pt}%
\definecolor{currentstroke}{rgb}{0.121569,0.466667,0.705882}%
\pgfsetstrokecolor{currentstroke}%
\pgfsetstrokeopacity{0.910332}%
\pgfsetdash{}{0pt}%
\pgfpathmoveto{\pgfqpoint{1.798295in}{1.077561in}}%
\pgfpathcurveto{\pgfqpoint{1.806531in}{1.077561in}}{\pgfqpoint{1.814431in}{1.080834in}}{\pgfqpoint{1.820255in}{1.086658in}}%
\pgfpathcurveto{\pgfqpoint{1.826079in}{1.092482in}}{\pgfqpoint{1.829351in}{1.100382in}}{\pgfqpoint{1.829351in}{1.108618in}}%
\pgfpathcurveto{\pgfqpoint{1.829351in}{1.116854in}}{\pgfqpoint{1.826079in}{1.124754in}}{\pgfqpoint{1.820255in}{1.130578in}}%
\pgfpathcurveto{\pgfqpoint{1.814431in}{1.136402in}}{\pgfqpoint{1.806531in}{1.139674in}}{\pgfqpoint{1.798295in}{1.139674in}}%
\pgfpathcurveto{\pgfqpoint{1.790058in}{1.139674in}}{\pgfqpoint{1.782158in}{1.136402in}}{\pgfqpoint{1.776334in}{1.130578in}}%
\pgfpathcurveto{\pgfqpoint{1.770510in}{1.124754in}}{\pgfqpoint{1.767238in}{1.116854in}}{\pgfqpoint{1.767238in}{1.108618in}}%
\pgfpathcurveto{\pgfqpoint{1.767238in}{1.100382in}}{\pgfqpoint{1.770510in}{1.092482in}}{\pgfqpoint{1.776334in}{1.086658in}}%
\pgfpathcurveto{\pgfqpoint{1.782158in}{1.080834in}}{\pgfqpoint{1.790058in}{1.077561in}}{\pgfqpoint{1.798295in}{1.077561in}}%
\pgfpathclose%
\pgfusepath{stroke,fill}%
\end{pgfscope}%
\begin{pgfscope}%
\pgfpathrectangle{\pgfqpoint{0.100000in}{0.212622in}}{\pgfqpoint{3.696000in}{3.696000in}}%
\pgfusepath{clip}%
\pgfsetbuttcap%
\pgfsetroundjoin%
\definecolor{currentfill}{rgb}{0.121569,0.466667,0.705882}%
\pgfsetfillcolor{currentfill}%
\pgfsetfillopacity{0.910744}%
\pgfsetlinewidth{1.003750pt}%
\definecolor{currentstroke}{rgb}{0.121569,0.466667,0.705882}%
\pgfsetstrokecolor{currentstroke}%
\pgfsetstrokeopacity{0.910744}%
\pgfsetdash{}{0pt}%
\pgfpathmoveto{\pgfqpoint{2.175029in}{2.078732in}}%
\pgfpathcurveto{\pgfqpoint{2.183265in}{2.078732in}}{\pgfqpoint{2.191165in}{2.082004in}}{\pgfqpoint{2.196989in}{2.087828in}}%
\pgfpathcurveto{\pgfqpoint{2.202813in}{2.093652in}}{\pgfqpoint{2.206086in}{2.101552in}}{\pgfqpoint{2.206086in}{2.109788in}}%
\pgfpathcurveto{\pgfqpoint{2.206086in}{2.118025in}}{\pgfqpoint{2.202813in}{2.125925in}}{\pgfqpoint{2.196989in}{2.131749in}}%
\pgfpathcurveto{\pgfqpoint{2.191165in}{2.137572in}}{\pgfqpoint{2.183265in}{2.140845in}}{\pgfqpoint{2.175029in}{2.140845in}}%
\pgfpathcurveto{\pgfqpoint{2.166793in}{2.140845in}}{\pgfqpoint{2.158893in}{2.137572in}}{\pgfqpoint{2.153069in}{2.131749in}}%
\pgfpathcurveto{\pgfqpoint{2.147245in}{2.125925in}}{\pgfqpoint{2.143973in}{2.118025in}}{\pgfqpoint{2.143973in}{2.109788in}}%
\pgfpathcurveto{\pgfqpoint{2.143973in}{2.101552in}}{\pgfqpoint{2.147245in}{2.093652in}}{\pgfqpoint{2.153069in}{2.087828in}}%
\pgfpathcurveto{\pgfqpoint{2.158893in}{2.082004in}}{\pgfqpoint{2.166793in}{2.078732in}}{\pgfqpoint{2.175029in}{2.078732in}}%
\pgfpathclose%
\pgfusepath{stroke,fill}%
\end{pgfscope}%
\begin{pgfscope}%
\pgfpathrectangle{\pgfqpoint{0.100000in}{0.212622in}}{\pgfqpoint{3.696000in}{3.696000in}}%
\pgfusepath{clip}%
\pgfsetbuttcap%
\pgfsetroundjoin%
\definecolor{currentfill}{rgb}{0.121569,0.466667,0.705882}%
\pgfsetfillcolor{currentfill}%
\pgfsetfillopacity{0.910893}%
\pgfsetlinewidth{1.003750pt}%
\definecolor{currentstroke}{rgb}{0.121569,0.466667,0.705882}%
\pgfsetstrokecolor{currentstroke}%
\pgfsetstrokeopacity{0.910893}%
\pgfsetdash{}{0pt}%
\pgfpathmoveto{\pgfqpoint{1.831290in}{1.963367in}}%
\pgfpathcurveto{\pgfqpoint{1.839526in}{1.963367in}}{\pgfqpoint{1.847426in}{1.966639in}}{\pgfqpoint{1.853250in}{1.972463in}}%
\pgfpathcurveto{\pgfqpoint{1.859074in}{1.978287in}}{\pgfqpoint{1.862347in}{1.986187in}}{\pgfqpoint{1.862347in}{1.994424in}}%
\pgfpathcurveto{\pgfqpoint{1.862347in}{2.002660in}}{\pgfqpoint{1.859074in}{2.010560in}}{\pgfqpoint{1.853250in}{2.016384in}}%
\pgfpathcurveto{\pgfqpoint{1.847426in}{2.022208in}}{\pgfqpoint{1.839526in}{2.025480in}}{\pgfqpoint{1.831290in}{2.025480in}}%
\pgfpathcurveto{\pgfqpoint{1.823054in}{2.025480in}}{\pgfqpoint{1.815154in}{2.022208in}}{\pgfqpoint{1.809330in}{2.016384in}}%
\pgfpathcurveto{\pgfqpoint{1.803506in}{2.010560in}}{\pgfqpoint{1.800234in}{2.002660in}}{\pgfqpoint{1.800234in}{1.994424in}}%
\pgfpathcurveto{\pgfqpoint{1.800234in}{1.986187in}}{\pgfqpoint{1.803506in}{1.978287in}}{\pgfqpoint{1.809330in}{1.972463in}}%
\pgfpathcurveto{\pgfqpoint{1.815154in}{1.966639in}}{\pgfqpoint{1.823054in}{1.963367in}}{\pgfqpoint{1.831290in}{1.963367in}}%
\pgfpathclose%
\pgfusepath{stroke,fill}%
\end{pgfscope}%
\begin{pgfscope}%
\pgfpathrectangle{\pgfqpoint{0.100000in}{0.212622in}}{\pgfqpoint{3.696000in}{3.696000in}}%
\pgfusepath{clip}%
\pgfsetbuttcap%
\pgfsetroundjoin%
\definecolor{currentfill}{rgb}{0.121569,0.466667,0.705882}%
\pgfsetfillcolor{currentfill}%
\pgfsetfillopacity{0.911548}%
\pgfsetlinewidth{1.003750pt}%
\definecolor{currentstroke}{rgb}{0.121569,0.466667,0.705882}%
\pgfsetstrokecolor{currentstroke}%
\pgfsetstrokeopacity{0.911548}%
\pgfsetdash{}{0pt}%
\pgfpathmoveto{\pgfqpoint{1.805102in}{1.078038in}}%
\pgfpathcurveto{\pgfqpoint{1.813338in}{1.078038in}}{\pgfqpoint{1.821238in}{1.081310in}}{\pgfqpoint{1.827062in}{1.087134in}}%
\pgfpathcurveto{\pgfqpoint{1.832886in}{1.092958in}}{\pgfqpoint{1.836158in}{1.100858in}}{\pgfqpoint{1.836158in}{1.109094in}}%
\pgfpathcurveto{\pgfqpoint{1.836158in}{1.117331in}}{\pgfqpoint{1.832886in}{1.125231in}}{\pgfqpoint{1.827062in}{1.131055in}}%
\pgfpathcurveto{\pgfqpoint{1.821238in}{1.136879in}}{\pgfqpoint{1.813338in}{1.140151in}}{\pgfqpoint{1.805102in}{1.140151in}}%
\pgfpathcurveto{\pgfqpoint{1.796865in}{1.140151in}}{\pgfqpoint{1.788965in}{1.136879in}}{\pgfqpoint{1.783141in}{1.131055in}}%
\pgfpathcurveto{\pgfqpoint{1.777317in}{1.125231in}}{\pgfqpoint{1.774045in}{1.117331in}}{\pgfqpoint{1.774045in}{1.109094in}}%
\pgfpathcurveto{\pgfqpoint{1.774045in}{1.100858in}}{\pgfqpoint{1.777317in}{1.092958in}}{\pgfqpoint{1.783141in}{1.087134in}}%
\pgfpathcurveto{\pgfqpoint{1.788965in}{1.081310in}}{\pgfqpoint{1.796865in}{1.078038in}}{\pgfqpoint{1.805102in}{1.078038in}}%
\pgfpathclose%
\pgfusepath{stroke,fill}%
\end{pgfscope}%
\begin{pgfscope}%
\pgfpathrectangle{\pgfqpoint{0.100000in}{0.212622in}}{\pgfqpoint{3.696000in}{3.696000in}}%
\pgfusepath{clip}%
\pgfsetbuttcap%
\pgfsetroundjoin%
\definecolor{currentfill}{rgb}{0.121569,0.466667,0.705882}%
\pgfsetfillcolor{currentfill}%
\pgfsetfillopacity{0.911603}%
\pgfsetlinewidth{1.003750pt}%
\definecolor{currentstroke}{rgb}{0.121569,0.466667,0.705882}%
\pgfsetstrokecolor{currentstroke}%
\pgfsetstrokeopacity{0.911603}%
\pgfsetdash{}{0pt}%
\pgfpathmoveto{\pgfqpoint{2.172783in}{2.074686in}}%
\pgfpathcurveto{\pgfqpoint{2.181020in}{2.074686in}}{\pgfqpoint{2.188920in}{2.077958in}}{\pgfqpoint{2.194744in}{2.083782in}}%
\pgfpathcurveto{\pgfqpoint{2.200567in}{2.089606in}}{\pgfqpoint{2.203840in}{2.097506in}}{\pgfqpoint{2.203840in}{2.105742in}}%
\pgfpathcurveto{\pgfqpoint{2.203840in}{2.113979in}}{\pgfqpoint{2.200567in}{2.121879in}}{\pgfqpoint{2.194744in}{2.127703in}}%
\pgfpathcurveto{\pgfqpoint{2.188920in}{2.133526in}}{\pgfqpoint{2.181020in}{2.136799in}}{\pgfqpoint{2.172783in}{2.136799in}}%
\pgfpathcurveto{\pgfqpoint{2.164547in}{2.136799in}}{\pgfqpoint{2.156647in}{2.133526in}}{\pgfqpoint{2.150823in}{2.127703in}}%
\pgfpathcurveto{\pgfqpoint{2.144999in}{2.121879in}}{\pgfqpoint{2.141727in}{2.113979in}}{\pgfqpoint{2.141727in}{2.105742in}}%
\pgfpathcurveto{\pgfqpoint{2.141727in}{2.097506in}}{\pgfqpoint{2.144999in}{2.089606in}}{\pgfqpoint{2.150823in}{2.083782in}}%
\pgfpathcurveto{\pgfqpoint{2.156647in}{2.077958in}}{\pgfqpoint{2.164547in}{2.074686in}}{\pgfqpoint{2.172783in}{2.074686in}}%
\pgfpathclose%
\pgfusepath{stroke,fill}%
\end{pgfscope}%
\begin{pgfscope}%
\pgfpathrectangle{\pgfqpoint{0.100000in}{0.212622in}}{\pgfqpoint{3.696000in}{3.696000in}}%
\pgfusepath{clip}%
\pgfsetbuttcap%
\pgfsetroundjoin%
\definecolor{currentfill}{rgb}{0.121569,0.466667,0.705882}%
\pgfsetfillcolor{currentfill}%
\pgfsetfillopacity{0.912191}%
\pgfsetlinewidth{1.003750pt}%
\definecolor{currentstroke}{rgb}{0.121569,0.466667,0.705882}%
\pgfsetstrokecolor{currentstroke}%
\pgfsetstrokeopacity{0.912191}%
\pgfsetdash{}{0pt}%
\pgfpathmoveto{\pgfqpoint{2.170505in}{2.071169in}}%
\pgfpathcurveto{\pgfqpoint{2.178742in}{2.071169in}}{\pgfqpoint{2.186642in}{2.074441in}}{\pgfqpoint{2.192466in}{2.080265in}}%
\pgfpathcurveto{\pgfqpoint{2.198290in}{2.086089in}}{\pgfqpoint{2.201562in}{2.093989in}}{\pgfqpoint{2.201562in}{2.102225in}}%
\pgfpathcurveto{\pgfqpoint{2.201562in}{2.110462in}}{\pgfqpoint{2.198290in}{2.118362in}}{\pgfqpoint{2.192466in}{2.124186in}}%
\pgfpathcurveto{\pgfqpoint{2.186642in}{2.130010in}}{\pgfqpoint{2.178742in}{2.133282in}}{\pgfqpoint{2.170505in}{2.133282in}}%
\pgfpathcurveto{\pgfqpoint{2.162269in}{2.133282in}}{\pgfqpoint{2.154369in}{2.130010in}}{\pgfqpoint{2.148545in}{2.124186in}}%
\pgfpathcurveto{\pgfqpoint{2.142721in}{2.118362in}}{\pgfqpoint{2.139449in}{2.110462in}}{\pgfqpoint{2.139449in}{2.102225in}}%
\pgfpathcurveto{\pgfqpoint{2.139449in}{2.093989in}}{\pgfqpoint{2.142721in}{2.086089in}}{\pgfqpoint{2.148545in}{2.080265in}}%
\pgfpathcurveto{\pgfqpoint{2.154369in}{2.074441in}}{\pgfqpoint{2.162269in}{2.071169in}}{\pgfqpoint{2.170505in}{2.071169in}}%
\pgfpathclose%
\pgfusepath{stroke,fill}%
\end{pgfscope}%
\begin{pgfscope}%
\pgfpathrectangle{\pgfqpoint{0.100000in}{0.212622in}}{\pgfqpoint{3.696000in}{3.696000in}}%
\pgfusepath{clip}%
\pgfsetbuttcap%
\pgfsetroundjoin%
\definecolor{currentfill}{rgb}{0.121569,0.466667,0.705882}%
\pgfsetfillcolor{currentfill}%
\pgfsetfillopacity{0.912446}%
\pgfsetlinewidth{1.003750pt}%
\definecolor{currentstroke}{rgb}{0.121569,0.466667,0.705882}%
\pgfsetstrokecolor{currentstroke}%
\pgfsetstrokeopacity{0.912446}%
\pgfsetdash{}{0pt}%
\pgfpathmoveto{\pgfqpoint{1.815126in}{1.074068in}}%
\pgfpathcurveto{\pgfqpoint{1.823362in}{1.074068in}}{\pgfqpoint{1.831262in}{1.077341in}}{\pgfqpoint{1.837086in}{1.083165in}}%
\pgfpathcurveto{\pgfqpoint{1.842910in}{1.088989in}}{\pgfqpoint{1.846182in}{1.096889in}}{\pgfqpoint{1.846182in}{1.105125in}}%
\pgfpathcurveto{\pgfqpoint{1.846182in}{1.113361in}}{\pgfqpoint{1.842910in}{1.121261in}}{\pgfqpoint{1.837086in}{1.127085in}}%
\pgfpathcurveto{\pgfqpoint{1.831262in}{1.132909in}}{\pgfqpoint{1.823362in}{1.136181in}}{\pgfqpoint{1.815126in}{1.136181in}}%
\pgfpathcurveto{\pgfqpoint{1.806889in}{1.136181in}}{\pgfqpoint{1.798989in}{1.132909in}}{\pgfqpoint{1.793165in}{1.127085in}}%
\pgfpathcurveto{\pgfqpoint{1.787342in}{1.121261in}}{\pgfqpoint{1.784069in}{1.113361in}}{\pgfqpoint{1.784069in}{1.105125in}}%
\pgfpathcurveto{\pgfqpoint{1.784069in}{1.096889in}}{\pgfqpoint{1.787342in}{1.088989in}}{\pgfqpoint{1.793165in}{1.083165in}}%
\pgfpathcurveto{\pgfqpoint{1.798989in}{1.077341in}}{\pgfqpoint{1.806889in}{1.074068in}}{\pgfqpoint{1.815126in}{1.074068in}}%
\pgfpathclose%
\pgfusepath{stroke,fill}%
\end{pgfscope}%
\begin{pgfscope}%
\pgfpathrectangle{\pgfqpoint{0.100000in}{0.212622in}}{\pgfqpoint{3.696000in}{3.696000in}}%
\pgfusepath{clip}%
\pgfsetbuttcap%
\pgfsetroundjoin%
\definecolor{currentfill}{rgb}{0.121569,0.466667,0.705882}%
\pgfsetfillcolor{currentfill}%
\pgfsetfillopacity{0.912506}%
\pgfsetlinewidth{1.003750pt}%
\definecolor{currentstroke}{rgb}{0.121569,0.466667,0.705882}%
\pgfsetstrokecolor{currentstroke}%
\pgfsetstrokeopacity{0.912506}%
\pgfsetdash{}{0pt}%
\pgfpathmoveto{\pgfqpoint{1.839316in}{1.957310in}}%
\pgfpathcurveto{\pgfqpoint{1.847553in}{1.957310in}}{\pgfqpoint{1.855453in}{1.960582in}}{\pgfqpoint{1.861277in}{1.966406in}}%
\pgfpathcurveto{\pgfqpoint{1.867101in}{1.972230in}}{\pgfqpoint{1.870373in}{1.980130in}}{\pgfqpoint{1.870373in}{1.988367in}}%
\pgfpathcurveto{\pgfqpoint{1.870373in}{1.996603in}}{\pgfqpoint{1.867101in}{2.004503in}}{\pgfqpoint{1.861277in}{2.010327in}}%
\pgfpathcurveto{\pgfqpoint{1.855453in}{2.016151in}}{\pgfqpoint{1.847553in}{2.019423in}}{\pgfqpoint{1.839316in}{2.019423in}}%
\pgfpathcurveto{\pgfqpoint{1.831080in}{2.019423in}}{\pgfqpoint{1.823180in}{2.016151in}}{\pgfqpoint{1.817356in}{2.010327in}}%
\pgfpathcurveto{\pgfqpoint{1.811532in}{2.004503in}}{\pgfqpoint{1.808260in}{1.996603in}}{\pgfqpoint{1.808260in}{1.988367in}}%
\pgfpathcurveto{\pgfqpoint{1.808260in}{1.980130in}}{\pgfqpoint{1.811532in}{1.972230in}}{\pgfqpoint{1.817356in}{1.966406in}}%
\pgfpathcurveto{\pgfqpoint{1.823180in}{1.960582in}}{\pgfqpoint{1.831080in}{1.957310in}}{\pgfqpoint{1.839316in}{1.957310in}}%
\pgfpathclose%
\pgfusepath{stroke,fill}%
\end{pgfscope}%
\begin{pgfscope}%
\pgfpathrectangle{\pgfqpoint{0.100000in}{0.212622in}}{\pgfqpoint{3.696000in}{3.696000in}}%
\pgfusepath{clip}%
\pgfsetbuttcap%
\pgfsetroundjoin%
\definecolor{currentfill}{rgb}{0.121569,0.466667,0.705882}%
\pgfsetfillcolor{currentfill}%
\pgfsetfillopacity{0.912793}%
\pgfsetlinewidth{1.003750pt}%
\definecolor{currentstroke}{rgb}{0.121569,0.466667,0.705882}%
\pgfsetstrokecolor{currentstroke}%
\pgfsetstrokeopacity{0.912793}%
\pgfsetdash{}{0pt}%
\pgfpathmoveto{\pgfqpoint{2.168466in}{2.068332in}}%
\pgfpathcurveto{\pgfqpoint{2.176703in}{2.068332in}}{\pgfqpoint{2.184603in}{2.071605in}}{\pgfqpoint{2.190427in}{2.077429in}}%
\pgfpathcurveto{\pgfqpoint{2.196251in}{2.083253in}}{\pgfqpoint{2.199523in}{2.091153in}}{\pgfqpoint{2.199523in}{2.099389in}}%
\pgfpathcurveto{\pgfqpoint{2.199523in}{2.107625in}}{\pgfqpoint{2.196251in}{2.115525in}}{\pgfqpoint{2.190427in}{2.121349in}}%
\pgfpathcurveto{\pgfqpoint{2.184603in}{2.127173in}}{\pgfqpoint{2.176703in}{2.130445in}}{\pgfqpoint{2.168466in}{2.130445in}}%
\pgfpathcurveto{\pgfqpoint{2.160230in}{2.130445in}}{\pgfqpoint{2.152330in}{2.127173in}}{\pgfqpoint{2.146506in}{2.121349in}}%
\pgfpathcurveto{\pgfqpoint{2.140682in}{2.115525in}}{\pgfqpoint{2.137410in}{2.107625in}}{\pgfqpoint{2.137410in}{2.099389in}}%
\pgfpathcurveto{\pgfqpoint{2.137410in}{2.091153in}}{\pgfqpoint{2.140682in}{2.083253in}}{\pgfqpoint{2.146506in}{2.077429in}}%
\pgfpathcurveto{\pgfqpoint{2.152330in}{2.071605in}}{\pgfqpoint{2.160230in}{2.068332in}}{\pgfqpoint{2.168466in}{2.068332in}}%
\pgfpathclose%
\pgfusepath{stroke,fill}%
\end{pgfscope}%
\begin{pgfscope}%
\pgfpathrectangle{\pgfqpoint{0.100000in}{0.212622in}}{\pgfqpoint{3.696000in}{3.696000in}}%
\pgfusepath{clip}%
\pgfsetbuttcap%
\pgfsetroundjoin%
\definecolor{currentfill}{rgb}{0.121569,0.466667,0.705882}%
\pgfsetfillcolor{currentfill}%
\pgfsetfillopacity{0.913166}%
\pgfsetlinewidth{1.003750pt}%
\definecolor{currentstroke}{rgb}{0.121569,0.466667,0.705882}%
\pgfsetstrokecolor{currentstroke}%
\pgfsetstrokeopacity{0.913166}%
\pgfsetdash{}{0pt}%
\pgfpathmoveto{\pgfqpoint{1.827315in}{1.065750in}}%
\pgfpathcurveto{\pgfqpoint{1.835551in}{1.065750in}}{\pgfqpoint{1.843451in}{1.069022in}}{\pgfqpoint{1.849275in}{1.074846in}}%
\pgfpathcurveto{\pgfqpoint{1.855099in}{1.080670in}}{\pgfqpoint{1.858371in}{1.088570in}}{\pgfqpoint{1.858371in}{1.096806in}}%
\pgfpathcurveto{\pgfqpoint{1.858371in}{1.105043in}}{\pgfqpoint{1.855099in}{1.112943in}}{\pgfqpoint{1.849275in}{1.118767in}}%
\pgfpathcurveto{\pgfqpoint{1.843451in}{1.124591in}}{\pgfqpoint{1.835551in}{1.127863in}}{\pgfqpoint{1.827315in}{1.127863in}}%
\pgfpathcurveto{\pgfqpoint{1.819078in}{1.127863in}}{\pgfqpoint{1.811178in}{1.124591in}}{\pgfqpoint{1.805354in}{1.118767in}}%
\pgfpathcurveto{\pgfqpoint{1.799530in}{1.112943in}}{\pgfqpoint{1.796258in}{1.105043in}}{\pgfqpoint{1.796258in}{1.096806in}}%
\pgfpathcurveto{\pgfqpoint{1.796258in}{1.088570in}}{\pgfqpoint{1.799530in}{1.080670in}}{\pgfqpoint{1.805354in}{1.074846in}}%
\pgfpathcurveto{\pgfqpoint{1.811178in}{1.069022in}}{\pgfqpoint{1.819078in}{1.065750in}}{\pgfqpoint{1.827315in}{1.065750in}}%
\pgfpathclose%
\pgfusepath{stroke,fill}%
\end{pgfscope}%
\begin{pgfscope}%
\pgfpathrectangle{\pgfqpoint{0.100000in}{0.212622in}}{\pgfqpoint{3.696000in}{3.696000in}}%
\pgfusepath{clip}%
\pgfsetbuttcap%
\pgfsetroundjoin%
\definecolor{currentfill}{rgb}{0.121569,0.466667,0.705882}%
\pgfsetfillcolor{currentfill}%
\pgfsetfillopacity{0.913390}%
\pgfsetlinewidth{1.003750pt}%
\definecolor{currentstroke}{rgb}{0.121569,0.466667,0.705882}%
\pgfsetstrokecolor{currentstroke}%
\pgfsetstrokeopacity{0.913390}%
\pgfsetdash{}{0pt}%
\pgfpathmoveto{\pgfqpoint{2.167317in}{2.065368in}}%
\pgfpathcurveto{\pgfqpoint{2.175553in}{2.065368in}}{\pgfqpoint{2.183453in}{2.068640in}}{\pgfqpoint{2.189277in}{2.074464in}}%
\pgfpathcurveto{\pgfqpoint{2.195101in}{2.080288in}}{\pgfqpoint{2.198373in}{2.088188in}}{\pgfqpoint{2.198373in}{2.096424in}}%
\pgfpathcurveto{\pgfqpoint{2.198373in}{2.104661in}}{\pgfqpoint{2.195101in}{2.112561in}}{\pgfqpoint{2.189277in}{2.118385in}}%
\pgfpathcurveto{\pgfqpoint{2.183453in}{2.124209in}}{\pgfqpoint{2.175553in}{2.127481in}}{\pgfqpoint{2.167317in}{2.127481in}}%
\pgfpathcurveto{\pgfqpoint{2.159080in}{2.127481in}}{\pgfqpoint{2.151180in}{2.124209in}}{\pgfqpoint{2.145356in}{2.118385in}}%
\pgfpathcurveto{\pgfqpoint{2.139532in}{2.112561in}}{\pgfqpoint{2.136260in}{2.104661in}}{\pgfqpoint{2.136260in}{2.096424in}}%
\pgfpathcurveto{\pgfqpoint{2.136260in}{2.088188in}}{\pgfqpoint{2.139532in}{2.080288in}}{\pgfqpoint{2.145356in}{2.074464in}}%
\pgfpathcurveto{\pgfqpoint{2.151180in}{2.068640in}}{\pgfqpoint{2.159080in}{2.065368in}}{\pgfqpoint{2.167317in}{2.065368in}}%
\pgfpathclose%
\pgfusepath{stroke,fill}%
\end{pgfscope}%
\begin{pgfscope}%
\pgfpathrectangle{\pgfqpoint{0.100000in}{0.212622in}}{\pgfqpoint{3.696000in}{3.696000in}}%
\pgfusepath{clip}%
\pgfsetbuttcap%
\pgfsetroundjoin%
\definecolor{currentfill}{rgb}{0.121569,0.466667,0.705882}%
\pgfsetfillcolor{currentfill}%
\pgfsetfillopacity{0.913893}%
\pgfsetlinewidth{1.003750pt}%
\definecolor{currentstroke}{rgb}{0.121569,0.466667,0.705882}%
\pgfsetstrokecolor{currentstroke}%
\pgfsetstrokeopacity{0.913893}%
\pgfsetdash{}{0pt}%
\pgfpathmoveto{\pgfqpoint{2.767873in}{1.283386in}}%
\pgfpathcurveto{\pgfqpoint{2.776109in}{1.283386in}}{\pgfqpoint{2.784009in}{1.286658in}}{\pgfqpoint{2.789833in}{1.292482in}}%
\pgfpathcurveto{\pgfqpoint{2.795657in}{1.298306in}}{\pgfqpoint{2.798929in}{1.306206in}}{\pgfqpoint{2.798929in}{1.314442in}}%
\pgfpathcurveto{\pgfqpoint{2.798929in}{1.322678in}}{\pgfqpoint{2.795657in}{1.330578in}}{\pgfqpoint{2.789833in}{1.336402in}}%
\pgfpathcurveto{\pgfqpoint{2.784009in}{1.342226in}}{\pgfqpoint{2.776109in}{1.345499in}}{\pgfqpoint{2.767873in}{1.345499in}}%
\pgfpathcurveto{\pgfqpoint{2.759637in}{1.345499in}}{\pgfqpoint{2.751737in}{1.342226in}}{\pgfqpoint{2.745913in}{1.336402in}}%
\pgfpathcurveto{\pgfqpoint{2.740089in}{1.330578in}}{\pgfqpoint{2.736816in}{1.322678in}}{\pgfqpoint{2.736816in}{1.314442in}}%
\pgfpathcurveto{\pgfqpoint{2.736816in}{1.306206in}}{\pgfqpoint{2.740089in}{1.298306in}}{\pgfqpoint{2.745913in}{1.292482in}}%
\pgfpathcurveto{\pgfqpoint{2.751737in}{1.286658in}}{\pgfqpoint{2.759637in}{1.283386in}}{\pgfqpoint{2.767873in}{1.283386in}}%
\pgfpathclose%
\pgfusepath{stroke,fill}%
\end{pgfscope}%
\begin{pgfscope}%
\pgfpathrectangle{\pgfqpoint{0.100000in}{0.212622in}}{\pgfqpoint{3.696000in}{3.696000in}}%
\pgfusepath{clip}%
\pgfsetbuttcap%
\pgfsetroundjoin%
\definecolor{currentfill}{rgb}{0.121569,0.466667,0.705882}%
\pgfsetfillcolor{currentfill}%
\pgfsetfillopacity{0.914046}%
\pgfsetlinewidth{1.003750pt}%
\definecolor{currentstroke}{rgb}{0.121569,0.466667,0.705882}%
\pgfsetstrokecolor{currentstroke}%
\pgfsetstrokeopacity{0.914046}%
\pgfsetdash{}{0pt}%
\pgfpathmoveto{\pgfqpoint{1.839541in}{1.054895in}}%
\pgfpathcurveto{\pgfqpoint{1.847778in}{1.054895in}}{\pgfqpoint{1.855678in}{1.058168in}}{\pgfqpoint{1.861502in}{1.063992in}}%
\pgfpathcurveto{\pgfqpoint{1.867326in}{1.069815in}}{\pgfqpoint{1.870598in}{1.077716in}}{\pgfqpoint{1.870598in}{1.085952in}}%
\pgfpathcurveto{\pgfqpoint{1.870598in}{1.094188in}}{\pgfqpoint{1.867326in}{1.102088in}}{\pgfqpoint{1.861502in}{1.107912in}}%
\pgfpathcurveto{\pgfqpoint{1.855678in}{1.113736in}}{\pgfqpoint{1.847778in}{1.117008in}}{\pgfqpoint{1.839541in}{1.117008in}}%
\pgfpathcurveto{\pgfqpoint{1.831305in}{1.117008in}}{\pgfqpoint{1.823405in}{1.113736in}}{\pgfqpoint{1.817581in}{1.107912in}}%
\pgfpathcurveto{\pgfqpoint{1.811757in}{1.102088in}}{\pgfqpoint{1.808485in}{1.094188in}}{\pgfqpoint{1.808485in}{1.085952in}}%
\pgfpathcurveto{\pgfqpoint{1.808485in}{1.077716in}}{\pgfqpoint{1.811757in}{1.069815in}}{\pgfqpoint{1.817581in}{1.063992in}}%
\pgfpathcurveto{\pgfqpoint{1.823405in}{1.058168in}}{\pgfqpoint{1.831305in}{1.054895in}}{\pgfqpoint{1.839541in}{1.054895in}}%
\pgfpathclose%
\pgfusepath{stroke,fill}%
\end{pgfscope}%
\begin{pgfscope}%
\pgfpathrectangle{\pgfqpoint{0.100000in}{0.212622in}}{\pgfqpoint{3.696000in}{3.696000in}}%
\pgfusepath{clip}%
\pgfsetbuttcap%
\pgfsetroundjoin%
\definecolor{currentfill}{rgb}{0.121569,0.466667,0.705882}%
\pgfsetfillcolor{currentfill}%
\pgfsetfillopacity{0.914436}%
\pgfsetlinewidth{1.003750pt}%
\definecolor{currentstroke}{rgb}{0.121569,0.466667,0.705882}%
\pgfsetstrokecolor{currentstroke}%
\pgfsetstrokeopacity{0.914436}%
\pgfsetdash{}{0pt}%
\pgfpathmoveto{\pgfqpoint{2.165293in}{2.059765in}}%
\pgfpathcurveto{\pgfqpoint{2.173529in}{2.059765in}}{\pgfqpoint{2.181429in}{2.063037in}}{\pgfqpoint{2.187253in}{2.068861in}}%
\pgfpathcurveto{\pgfqpoint{2.193077in}{2.074685in}}{\pgfqpoint{2.196349in}{2.082585in}}{\pgfqpoint{2.196349in}{2.090821in}}%
\pgfpathcurveto{\pgfqpoint{2.196349in}{2.099057in}}{\pgfqpoint{2.193077in}{2.106957in}}{\pgfqpoint{2.187253in}{2.112781in}}%
\pgfpathcurveto{\pgfqpoint{2.181429in}{2.118605in}}{\pgfqpoint{2.173529in}{2.121878in}}{\pgfqpoint{2.165293in}{2.121878in}}%
\pgfpathcurveto{\pgfqpoint{2.157056in}{2.121878in}}{\pgfqpoint{2.149156in}{2.118605in}}{\pgfqpoint{2.143332in}{2.112781in}}%
\pgfpathcurveto{\pgfqpoint{2.137509in}{2.106957in}}{\pgfqpoint{2.134236in}{2.099057in}}{\pgfqpoint{2.134236in}{2.090821in}}%
\pgfpathcurveto{\pgfqpoint{2.134236in}{2.082585in}}{\pgfqpoint{2.137509in}{2.074685in}}{\pgfqpoint{2.143332in}{2.068861in}}%
\pgfpathcurveto{\pgfqpoint{2.149156in}{2.063037in}}{\pgfqpoint{2.157056in}{2.059765in}}{\pgfqpoint{2.165293in}{2.059765in}}%
\pgfpathclose%
\pgfusepath{stroke,fill}%
\end{pgfscope}%
\begin{pgfscope}%
\pgfpathrectangle{\pgfqpoint{0.100000in}{0.212622in}}{\pgfqpoint{3.696000in}{3.696000in}}%
\pgfusepath{clip}%
\pgfsetbuttcap%
\pgfsetroundjoin%
\definecolor{currentfill}{rgb}{0.121569,0.466667,0.705882}%
\pgfsetfillcolor{currentfill}%
\pgfsetfillopacity{0.914493}%
\pgfsetlinewidth{1.003750pt}%
\definecolor{currentstroke}{rgb}{0.121569,0.466667,0.705882}%
\pgfsetstrokecolor{currentstroke}%
\pgfsetstrokeopacity{0.914493}%
\pgfsetdash{}{0pt}%
\pgfpathmoveto{\pgfqpoint{1.848621in}{1.953811in}}%
\pgfpathcurveto{\pgfqpoint{1.856857in}{1.953811in}}{\pgfqpoint{1.864757in}{1.957084in}}{\pgfqpoint{1.870581in}{1.962908in}}%
\pgfpathcurveto{\pgfqpoint{1.876405in}{1.968732in}}{\pgfqpoint{1.879677in}{1.976632in}}{\pgfqpoint{1.879677in}{1.984868in}}%
\pgfpathcurveto{\pgfqpoint{1.879677in}{1.993104in}}{\pgfqpoint{1.876405in}{2.001004in}}{\pgfqpoint{1.870581in}{2.006828in}}%
\pgfpathcurveto{\pgfqpoint{1.864757in}{2.012652in}}{\pgfqpoint{1.856857in}{2.015924in}}{\pgfqpoint{1.848621in}{2.015924in}}%
\pgfpathcurveto{\pgfqpoint{1.840384in}{2.015924in}}{\pgfqpoint{1.832484in}{2.012652in}}{\pgfqpoint{1.826660in}{2.006828in}}%
\pgfpathcurveto{\pgfqpoint{1.820837in}{2.001004in}}{\pgfqpoint{1.817564in}{1.993104in}}{\pgfqpoint{1.817564in}{1.984868in}}%
\pgfpathcurveto{\pgfqpoint{1.817564in}{1.976632in}}{\pgfqpoint{1.820837in}{1.968732in}}{\pgfqpoint{1.826660in}{1.962908in}}%
\pgfpathcurveto{\pgfqpoint{1.832484in}{1.957084in}}{\pgfqpoint{1.840384in}{1.953811in}}{\pgfqpoint{1.848621in}{1.953811in}}%
\pgfpathclose%
\pgfusepath{stroke,fill}%
\end{pgfscope}%
\begin{pgfscope}%
\pgfpathrectangle{\pgfqpoint{0.100000in}{0.212622in}}{\pgfqpoint{3.696000in}{3.696000in}}%
\pgfusepath{clip}%
\pgfsetbuttcap%
\pgfsetroundjoin%
\definecolor{currentfill}{rgb}{0.121569,0.466667,0.705882}%
\pgfsetfillcolor{currentfill}%
\pgfsetfillopacity{0.914965}%
\pgfsetlinewidth{1.003750pt}%
\definecolor{currentstroke}{rgb}{0.121569,0.466667,0.705882}%
\pgfsetstrokecolor{currentstroke}%
\pgfsetstrokeopacity{0.914965}%
\pgfsetdash{}{0pt}%
\pgfpathmoveto{\pgfqpoint{2.163469in}{2.057319in}}%
\pgfpathcurveto{\pgfqpoint{2.171705in}{2.057319in}}{\pgfqpoint{2.179605in}{2.060592in}}{\pgfqpoint{2.185429in}{2.066416in}}%
\pgfpathcurveto{\pgfqpoint{2.191253in}{2.072239in}}{\pgfqpoint{2.194525in}{2.080139in}}{\pgfqpoint{2.194525in}{2.088376in}}%
\pgfpathcurveto{\pgfqpoint{2.194525in}{2.096612in}}{\pgfqpoint{2.191253in}{2.104512in}}{\pgfqpoint{2.185429in}{2.110336in}}%
\pgfpathcurveto{\pgfqpoint{2.179605in}{2.116160in}}{\pgfqpoint{2.171705in}{2.119432in}}{\pgfqpoint{2.163469in}{2.119432in}}%
\pgfpathcurveto{\pgfqpoint{2.155232in}{2.119432in}}{\pgfqpoint{2.147332in}{2.116160in}}{\pgfqpoint{2.141508in}{2.110336in}}%
\pgfpathcurveto{\pgfqpoint{2.135685in}{2.104512in}}{\pgfqpoint{2.132412in}{2.096612in}}{\pgfqpoint{2.132412in}{2.088376in}}%
\pgfpathcurveto{\pgfqpoint{2.132412in}{2.080139in}}{\pgfqpoint{2.135685in}{2.072239in}}{\pgfqpoint{2.141508in}{2.066416in}}%
\pgfpathcurveto{\pgfqpoint{2.147332in}{2.060592in}}{\pgfqpoint{2.155232in}{2.057319in}}{\pgfqpoint{2.163469in}{2.057319in}}%
\pgfpathclose%
\pgfusepath{stroke,fill}%
\end{pgfscope}%
\begin{pgfscope}%
\pgfpathrectangle{\pgfqpoint{0.100000in}{0.212622in}}{\pgfqpoint{3.696000in}{3.696000in}}%
\pgfusepath{clip}%
\pgfsetbuttcap%
\pgfsetroundjoin%
\definecolor{currentfill}{rgb}{0.121569,0.466667,0.705882}%
\pgfsetfillcolor{currentfill}%
\pgfsetfillopacity{0.915814}%
\pgfsetlinewidth{1.003750pt}%
\definecolor{currentstroke}{rgb}{0.121569,0.466667,0.705882}%
\pgfsetstrokecolor{currentstroke}%
\pgfsetstrokeopacity{0.915814}%
\pgfsetdash{}{0pt}%
\pgfpathmoveto{\pgfqpoint{2.160220in}{2.052243in}}%
\pgfpathcurveto{\pgfqpoint{2.168456in}{2.052243in}}{\pgfqpoint{2.176356in}{2.055516in}}{\pgfqpoint{2.182180in}{2.061340in}}%
\pgfpathcurveto{\pgfqpoint{2.188004in}{2.067164in}}{\pgfqpoint{2.191277in}{2.075064in}}{\pgfqpoint{2.191277in}{2.083300in}}%
\pgfpathcurveto{\pgfqpoint{2.191277in}{2.091536in}}{\pgfqpoint{2.188004in}{2.099436in}}{\pgfqpoint{2.182180in}{2.105260in}}%
\pgfpathcurveto{\pgfqpoint{2.176356in}{2.111084in}}{\pgfqpoint{2.168456in}{2.114356in}}{\pgfqpoint{2.160220in}{2.114356in}}%
\pgfpathcurveto{\pgfqpoint{2.151984in}{2.114356in}}{\pgfqpoint{2.144084in}{2.111084in}}{\pgfqpoint{2.138260in}{2.105260in}}%
\pgfpathcurveto{\pgfqpoint{2.132436in}{2.099436in}}{\pgfqpoint{2.129164in}{2.091536in}}{\pgfqpoint{2.129164in}{2.083300in}}%
\pgfpathcurveto{\pgfqpoint{2.129164in}{2.075064in}}{\pgfqpoint{2.132436in}{2.067164in}}{\pgfqpoint{2.138260in}{2.061340in}}%
\pgfpathcurveto{\pgfqpoint{2.144084in}{2.055516in}}{\pgfqpoint{2.151984in}{2.052243in}}{\pgfqpoint{2.160220in}{2.052243in}}%
\pgfpathclose%
\pgfusepath{stroke,fill}%
\end{pgfscope}%
\begin{pgfscope}%
\pgfpathrectangle{\pgfqpoint{0.100000in}{0.212622in}}{\pgfqpoint{3.696000in}{3.696000in}}%
\pgfusepath{clip}%
\pgfsetbuttcap%
\pgfsetroundjoin%
\definecolor{currentfill}{rgb}{0.121569,0.466667,0.705882}%
\pgfsetfillcolor{currentfill}%
\pgfsetfillopacity{0.916169}%
\pgfsetlinewidth{1.003750pt}%
\definecolor{currentstroke}{rgb}{0.121569,0.466667,0.705882}%
\pgfsetstrokecolor{currentstroke}%
\pgfsetstrokeopacity{0.916169}%
\pgfsetdash{}{0pt}%
\pgfpathmoveto{\pgfqpoint{1.853529in}{1.049298in}}%
\pgfpathcurveto{\pgfqpoint{1.861765in}{1.049298in}}{\pgfqpoint{1.869665in}{1.052570in}}{\pgfqpoint{1.875489in}{1.058394in}}%
\pgfpathcurveto{\pgfqpoint{1.881313in}{1.064218in}}{\pgfqpoint{1.884585in}{1.072118in}}{\pgfqpoint{1.884585in}{1.080354in}}%
\pgfpathcurveto{\pgfqpoint{1.884585in}{1.088591in}}{\pgfqpoint{1.881313in}{1.096491in}}{\pgfqpoint{1.875489in}{1.102315in}}%
\pgfpathcurveto{\pgfqpoint{1.869665in}{1.108139in}}{\pgfqpoint{1.861765in}{1.111411in}}{\pgfqpoint{1.853529in}{1.111411in}}%
\pgfpathcurveto{\pgfqpoint{1.845293in}{1.111411in}}{\pgfqpoint{1.837392in}{1.108139in}}{\pgfqpoint{1.831569in}{1.102315in}}%
\pgfpathcurveto{\pgfqpoint{1.825745in}{1.096491in}}{\pgfqpoint{1.822472in}{1.088591in}}{\pgfqpoint{1.822472in}{1.080354in}}%
\pgfpathcurveto{\pgfqpoint{1.822472in}{1.072118in}}{\pgfqpoint{1.825745in}{1.064218in}}{\pgfqpoint{1.831569in}{1.058394in}}%
\pgfpathcurveto{\pgfqpoint{1.837392in}{1.052570in}}{\pgfqpoint{1.845293in}{1.049298in}}{\pgfqpoint{1.853529in}{1.049298in}}%
\pgfpathclose%
\pgfusepath{stroke,fill}%
\end{pgfscope}%
\begin{pgfscope}%
\pgfpathrectangle{\pgfqpoint{0.100000in}{0.212622in}}{\pgfqpoint{3.696000in}{3.696000in}}%
\pgfusepath{clip}%
\pgfsetbuttcap%
\pgfsetroundjoin%
\definecolor{currentfill}{rgb}{0.121569,0.466667,0.705882}%
\pgfsetfillcolor{currentfill}%
\pgfsetfillopacity{0.916358}%
\pgfsetlinewidth{1.003750pt}%
\definecolor{currentstroke}{rgb}{0.121569,0.466667,0.705882}%
\pgfsetstrokecolor{currentstroke}%
\pgfsetstrokeopacity{0.916358}%
\pgfsetdash{}{0pt}%
\pgfpathmoveto{\pgfqpoint{2.159550in}{2.048925in}}%
\pgfpathcurveto{\pgfqpoint{2.167787in}{2.048925in}}{\pgfqpoint{2.175687in}{2.052197in}}{\pgfqpoint{2.181511in}{2.058021in}}%
\pgfpathcurveto{\pgfqpoint{2.187335in}{2.063845in}}{\pgfqpoint{2.190607in}{2.071745in}}{\pgfqpoint{2.190607in}{2.079981in}}%
\pgfpathcurveto{\pgfqpoint{2.190607in}{2.088218in}}{\pgfqpoint{2.187335in}{2.096118in}}{\pgfqpoint{2.181511in}{2.101942in}}%
\pgfpathcurveto{\pgfqpoint{2.175687in}{2.107766in}}{\pgfqpoint{2.167787in}{2.111038in}}{\pgfqpoint{2.159550in}{2.111038in}}%
\pgfpathcurveto{\pgfqpoint{2.151314in}{2.111038in}}{\pgfqpoint{2.143414in}{2.107766in}}{\pgfqpoint{2.137590in}{2.101942in}}%
\pgfpathcurveto{\pgfqpoint{2.131766in}{2.096118in}}{\pgfqpoint{2.128494in}{2.088218in}}{\pgfqpoint{2.128494in}{2.079981in}}%
\pgfpathcurveto{\pgfqpoint{2.128494in}{2.071745in}}{\pgfqpoint{2.131766in}{2.063845in}}{\pgfqpoint{2.137590in}{2.058021in}}%
\pgfpathcurveto{\pgfqpoint{2.143414in}{2.052197in}}{\pgfqpoint{2.151314in}{2.048925in}}{\pgfqpoint{2.159550in}{2.048925in}}%
\pgfpathclose%
\pgfusepath{stroke,fill}%
\end{pgfscope}%
\begin{pgfscope}%
\pgfpathrectangle{\pgfqpoint{0.100000in}{0.212622in}}{\pgfqpoint{3.696000in}{3.696000in}}%
\pgfusepath{clip}%
\pgfsetbuttcap%
\pgfsetroundjoin%
\definecolor{currentfill}{rgb}{0.121569,0.466667,0.705882}%
\pgfsetfillcolor{currentfill}%
\pgfsetfillopacity{0.916380}%
\pgfsetlinewidth{1.003750pt}%
\definecolor{currentstroke}{rgb}{0.121569,0.466667,0.705882}%
\pgfsetstrokecolor{currentstroke}%
\pgfsetstrokeopacity{0.916380}%
\pgfsetdash{}{0pt}%
\pgfpathmoveto{\pgfqpoint{1.858852in}{1.946689in}}%
\pgfpathcurveto{\pgfqpoint{1.867088in}{1.946689in}}{\pgfqpoint{1.874988in}{1.949961in}}{\pgfqpoint{1.880812in}{1.955785in}}%
\pgfpathcurveto{\pgfqpoint{1.886636in}{1.961609in}}{\pgfqpoint{1.889908in}{1.969509in}}{\pgfqpoint{1.889908in}{1.977746in}}%
\pgfpathcurveto{\pgfqpoint{1.889908in}{1.985982in}}{\pgfqpoint{1.886636in}{1.993882in}}{\pgfqpoint{1.880812in}{1.999706in}}%
\pgfpathcurveto{\pgfqpoint{1.874988in}{2.005530in}}{\pgfqpoint{1.867088in}{2.008802in}}{\pgfqpoint{1.858852in}{2.008802in}}%
\pgfpathcurveto{\pgfqpoint{1.850616in}{2.008802in}}{\pgfqpoint{1.842715in}{2.005530in}}{\pgfqpoint{1.836892in}{1.999706in}}%
\pgfpathcurveto{\pgfqpoint{1.831068in}{1.993882in}}{\pgfqpoint{1.827795in}{1.985982in}}{\pgfqpoint{1.827795in}{1.977746in}}%
\pgfpathcurveto{\pgfqpoint{1.827795in}{1.969509in}}{\pgfqpoint{1.831068in}{1.961609in}}{\pgfqpoint{1.836892in}{1.955785in}}%
\pgfpathcurveto{\pgfqpoint{1.842715in}{1.949961in}}{\pgfqpoint{1.850616in}{1.946689in}}{\pgfqpoint{1.858852in}{1.946689in}}%
\pgfpathclose%
\pgfusepath{stroke,fill}%
\end{pgfscope}%
\begin{pgfscope}%
\pgfpathrectangle{\pgfqpoint{0.100000in}{0.212622in}}{\pgfqpoint{3.696000in}{3.696000in}}%
\pgfusepath{clip}%
\pgfsetbuttcap%
\pgfsetroundjoin%
\definecolor{currentfill}{rgb}{0.121569,0.466667,0.705882}%
\pgfsetfillcolor{currentfill}%
\pgfsetfillopacity{0.917431}%
\pgfsetlinewidth{1.003750pt}%
\definecolor{currentstroke}{rgb}{0.121569,0.466667,0.705882}%
\pgfsetstrokecolor{currentstroke}%
\pgfsetstrokeopacity{0.917431}%
\pgfsetdash{}{0pt}%
\pgfpathmoveto{\pgfqpoint{2.157477in}{2.043491in}}%
\pgfpathcurveto{\pgfqpoint{2.165713in}{2.043491in}}{\pgfqpoint{2.173613in}{2.046763in}}{\pgfqpoint{2.179437in}{2.052587in}}%
\pgfpathcurveto{\pgfqpoint{2.185261in}{2.058411in}}{\pgfqpoint{2.188534in}{2.066311in}}{\pgfqpoint{2.188534in}{2.074547in}}%
\pgfpathcurveto{\pgfqpoint{2.188534in}{2.082783in}}{\pgfqpoint{2.185261in}{2.090683in}}{\pgfqpoint{2.179437in}{2.096507in}}%
\pgfpathcurveto{\pgfqpoint{2.173613in}{2.102331in}}{\pgfqpoint{2.165713in}{2.105604in}}{\pgfqpoint{2.157477in}{2.105604in}}%
\pgfpathcurveto{\pgfqpoint{2.149241in}{2.105604in}}{\pgfqpoint{2.141341in}{2.102331in}}{\pgfqpoint{2.135517in}{2.096507in}}%
\pgfpathcurveto{\pgfqpoint{2.129693in}{2.090683in}}{\pgfqpoint{2.126421in}{2.082783in}}{\pgfqpoint{2.126421in}{2.074547in}}%
\pgfpathcurveto{\pgfqpoint{2.126421in}{2.066311in}}{\pgfqpoint{2.129693in}{2.058411in}}{\pgfqpoint{2.135517in}{2.052587in}}%
\pgfpathcurveto{\pgfqpoint{2.141341in}{2.046763in}}{\pgfqpoint{2.149241in}{2.043491in}}{\pgfqpoint{2.157477in}{2.043491in}}%
\pgfpathclose%
\pgfusepath{stroke,fill}%
\end{pgfscope}%
\begin{pgfscope}%
\pgfpathrectangle{\pgfqpoint{0.100000in}{0.212622in}}{\pgfqpoint{3.696000in}{3.696000in}}%
\pgfusepath{clip}%
\pgfsetbuttcap%
\pgfsetroundjoin%
\definecolor{currentfill}{rgb}{0.121569,0.466667,0.705882}%
\pgfsetfillcolor{currentfill}%
\pgfsetfillopacity{0.918020}%
\pgfsetlinewidth{1.003750pt}%
\definecolor{currentstroke}{rgb}{0.121569,0.466667,0.705882}%
\pgfsetstrokecolor{currentstroke}%
\pgfsetstrokeopacity{0.918020}%
\pgfsetdash{}{0pt}%
\pgfpathmoveto{\pgfqpoint{2.155304in}{2.040247in}}%
\pgfpathcurveto{\pgfqpoint{2.163540in}{2.040247in}}{\pgfqpoint{2.171440in}{2.043519in}}{\pgfqpoint{2.177264in}{2.049343in}}%
\pgfpathcurveto{\pgfqpoint{2.183088in}{2.055167in}}{\pgfqpoint{2.186360in}{2.063067in}}{\pgfqpoint{2.186360in}{2.071304in}}%
\pgfpathcurveto{\pgfqpoint{2.186360in}{2.079540in}}{\pgfqpoint{2.183088in}{2.087440in}}{\pgfqpoint{2.177264in}{2.093264in}}%
\pgfpathcurveto{\pgfqpoint{2.171440in}{2.099088in}}{\pgfqpoint{2.163540in}{2.102360in}}{\pgfqpoint{2.155304in}{2.102360in}}%
\pgfpathcurveto{\pgfqpoint{2.147067in}{2.102360in}}{\pgfqpoint{2.139167in}{2.099088in}}{\pgfqpoint{2.133343in}{2.093264in}}%
\pgfpathcurveto{\pgfqpoint{2.127520in}{2.087440in}}{\pgfqpoint{2.124247in}{2.079540in}}{\pgfqpoint{2.124247in}{2.071304in}}%
\pgfpathcurveto{\pgfqpoint{2.124247in}{2.063067in}}{\pgfqpoint{2.127520in}{2.055167in}}{\pgfqpoint{2.133343in}{2.049343in}}%
\pgfpathcurveto{\pgfqpoint{2.139167in}{2.043519in}}{\pgfqpoint{2.147067in}{2.040247in}}{\pgfqpoint{2.155304in}{2.040247in}}%
\pgfpathclose%
\pgfusepath{stroke,fill}%
\end{pgfscope}%
\begin{pgfscope}%
\pgfpathrectangle{\pgfqpoint{0.100000in}{0.212622in}}{\pgfqpoint{3.696000in}{3.696000in}}%
\pgfusepath{clip}%
\pgfsetbuttcap%
\pgfsetroundjoin%
\definecolor{currentfill}{rgb}{0.121569,0.466667,0.705882}%
\pgfsetfillcolor{currentfill}%
\pgfsetfillopacity{0.918231}%
\pgfsetlinewidth{1.003750pt}%
\definecolor{currentstroke}{rgb}{0.121569,0.466667,0.705882}%
\pgfsetstrokecolor{currentstroke}%
\pgfsetstrokeopacity{0.918231}%
\pgfsetdash{}{0pt}%
\pgfpathmoveto{\pgfqpoint{2.756031in}{1.257817in}}%
\pgfpathcurveto{\pgfqpoint{2.764267in}{1.257817in}}{\pgfqpoint{2.772167in}{1.261089in}}{\pgfqpoint{2.777991in}{1.266913in}}%
\pgfpathcurveto{\pgfqpoint{2.783815in}{1.272737in}}{\pgfqpoint{2.787088in}{1.280637in}}{\pgfqpoint{2.787088in}{1.288873in}}%
\pgfpathcurveto{\pgfqpoint{2.787088in}{1.297110in}}{\pgfqpoint{2.783815in}{1.305010in}}{\pgfqpoint{2.777991in}{1.310834in}}%
\pgfpathcurveto{\pgfqpoint{2.772167in}{1.316658in}}{\pgfqpoint{2.764267in}{1.319930in}}{\pgfqpoint{2.756031in}{1.319930in}}%
\pgfpathcurveto{\pgfqpoint{2.747795in}{1.319930in}}{\pgfqpoint{2.739895in}{1.316658in}}{\pgfqpoint{2.734071in}{1.310834in}}%
\pgfpathcurveto{\pgfqpoint{2.728247in}{1.305010in}}{\pgfqpoint{2.724975in}{1.297110in}}{\pgfqpoint{2.724975in}{1.288873in}}%
\pgfpathcurveto{\pgfqpoint{2.724975in}{1.280637in}}{\pgfqpoint{2.728247in}{1.272737in}}{\pgfqpoint{2.734071in}{1.266913in}}%
\pgfpathcurveto{\pgfqpoint{2.739895in}{1.261089in}}{\pgfqpoint{2.747795in}{1.257817in}}{\pgfqpoint{2.756031in}{1.257817in}}%
\pgfpathclose%
\pgfusepath{stroke,fill}%
\end{pgfscope}%
\begin{pgfscope}%
\pgfpathrectangle{\pgfqpoint{0.100000in}{0.212622in}}{\pgfqpoint{3.696000in}{3.696000in}}%
\pgfusepath{clip}%
\pgfsetbuttcap%
\pgfsetroundjoin%
\definecolor{currentfill}{rgb}{0.121569,0.466667,0.705882}%
\pgfsetfillcolor{currentfill}%
\pgfsetfillopacity{0.918306}%
\pgfsetlinewidth{1.003750pt}%
\definecolor{currentstroke}{rgb}{0.121569,0.466667,0.705882}%
\pgfsetstrokecolor{currentstroke}%
\pgfsetstrokeopacity{0.918306}%
\pgfsetdash{}{0pt}%
\pgfpathmoveto{\pgfqpoint{1.869817in}{1.940188in}}%
\pgfpathcurveto{\pgfqpoint{1.878053in}{1.940188in}}{\pgfqpoint{1.885953in}{1.943460in}}{\pgfqpoint{1.891777in}{1.949284in}}%
\pgfpathcurveto{\pgfqpoint{1.897601in}{1.955108in}}{\pgfqpoint{1.900873in}{1.963008in}}{\pgfqpoint{1.900873in}{1.971244in}}%
\pgfpathcurveto{\pgfqpoint{1.900873in}{1.979480in}}{\pgfqpoint{1.897601in}{1.987380in}}{\pgfqpoint{1.891777in}{1.993204in}}%
\pgfpathcurveto{\pgfqpoint{1.885953in}{1.999028in}}{\pgfqpoint{1.878053in}{2.002301in}}{\pgfqpoint{1.869817in}{2.002301in}}%
\pgfpathcurveto{\pgfqpoint{1.861580in}{2.002301in}}{\pgfqpoint{1.853680in}{1.999028in}}{\pgfqpoint{1.847856in}{1.993204in}}%
\pgfpathcurveto{\pgfqpoint{1.842033in}{1.987380in}}{\pgfqpoint{1.838760in}{1.979480in}}{\pgfqpoint{1.838760in}{1.971244in}}%
\pgfpathcurveto{\pgfqpoint{1.838760in}{1.963008in}}{\pgfqpoint{1.842033in}{1.955108in}}{\pgfqpoint{1.847856in}{1.949284in}}%
\pgfpathcurveto{\pgfqpoint{1.853680in}{1.943460in}}{\pgfqpoint{1.861580in}{1.940188in}}{\pgfqpoint{1.869817in}{1.940188in}}%
\pgfpathclose%
\pgfusepath{stroke,fill}%
\end{pgfscope}%
\begin{pgfscope}%
\pgfpathrectangle{\pgfqpoint{0.100000in}{0.212622in}}{\pgfqpoint{3.696000in}{3.696000in}}%
\pgfusepath{clip}%
\pgfsetbuttcap%
\pgfsetroundjoin%
\definecolor{currentfill}{rgb}{0.121569,0.466667,0.705882}%
\pgfsetfillcolor{currentfill}%
\pgfsetfillopacity{0.918742}%
\pgfsetlinewidth{1.003750pt}%
\definecolor{currentstroke}{rgb}{0.121569,0.466667,0.705882}%
\pgfsetstrokecolor{currentstroke}%
\pgfsetstrokeopacity{0.918742}%
\pgfsetdash{}{0pt}%
\pgfpathmoveto{\pgfqpoint{1.870324in}{1.050115in}}%
\pgfpathcurveto{\pgfqpoint{1.878560in}{1.050115in}}{\pgfqpoint{1.886460in}{1.053387in}}{\pgfqpoint{1.892284in}{1.059211in}}%
\pgfpathcurveto{\pgfqpoint{1.898108in}{1.065035in}}{\pgfqpoint{1.901380in}{1.072935in}}{\pgfqpoint{1.901380in}{1.081172in}}%
\pgfpathcurveto{\pgfqpoint{1.901380in}{1.089408in}}{\pgfqpoint{1.898108in}{1.097308in}}{\pgfqpoint{1.892284in}{1.103132in}}%
\pgfpathcurveto{\pgfqpoint{1.886460in}{1.108956in}}{\pgfqpoint{1.878560in}{1.112228in}}{\pgfqpoint{1.870324in}{1.112228in}}%
\pgfpathcurveto{\pgfqpoint{1.862087in}{1.112228in}}{\pgfqpoint{1.854187in}{1.108956in}}{\pgfqpoint{1.848363in}{1.103132in}}%
\pgfpathcurveto{\pgfqpoint{1.842539in}{1.097308in}}{\pgfqpoint{1.839267in}{1.089408in}}{\pgfqpoint{1.839267in}{1.081172in}}%
\pgfpathcurveto{\pgfqpoint{1.839267in}{1.072935in}}{\pgfqpoint{1.842539in}{1.065035in}}{\pgfqpoint{1.848363in}{1.059211in}}%
\pgfpathcurveto{\pgfqpoint{1.854187in}{1.053387in}}{\pgfqpoint{1.862087in}{1.050115in}}{\pgfqpoint{1.870324in}{1.050115in}}%
\pgfpathclose%
\pgfusepath{stroke,fill}%
\end{pgfscope}%
\begin{pgfscope}%
\pgfpathrectangle{\pgfqpoint{0.100000in}{0.212622in}}{\pgfqpoint{3.696000in}{3.696000in}}%
\pgfusepath{clip}%
\pgfsetbuttcap%
\pgfsetroundjoin%
\definecolor{currentfill}{rgb}{0.121569,0.466667,0.705882}%
\pgfsetfillcolor{currentfill}%
\pgfsetfillopacity{0.919253}%
\pgfsetlinewidth{1.003750pt}%
\definecolor{currentstroke}{rgb}{0.121569,0.466667,0.705882}%
\pgfsetstrokecolor{currentstroke}%
\pgfsetstrokeopacity{0.919253}%
\pgfsetdash{}{0pt}%
\pgfpathmoveto{\pgfqpoint{2.152074in}{2.034022in}}%
\pgfpathcurveto{\pgfqpoint{2.160311in}{2.034022in}}{\pgfqpoint{2.168211in}{2.037294in}}{\pgfqpoint{2.174035in}{2.043118in}}%
\pgfpathcurveto{\pgfqpoint{2.179859in}{2.048942in}}{\pgfqpoint{2.183131in}{2.056842in}}{\pgfqpoint{2.183131in}{2.065078in}}%
\pgfpathcurveto{\pgfqpoint{2.183131in}{2.073315in}}{\pgfqpoint{2.179859in}{2.081215in}}{\pgfqpoint{2.174035in}{2.087039in}}%
\pgfpathcurveto{\pgfqpoint{2.168211in}{2.092863in}}{\pgfqpoint{2.160311in}{2.096135in}}{\pgfqpoint{2.152074in}{2.096135in}}%
\pgfpathcurveto{\pgfqpoint{2.143838in}{2.096135in}}{\pgfqpoint{2.135938in}{2.092863in}}{\pgfqpoint{2.130114in}{2.087039in}}%
\pgfpathcurveto{\pgfqpoint{2.124290in}{2.081215in}}{\pgfqpoint{2.121018in}{2.073315in}}{\pgfqpoint{2.121018in}{2.065078in}}%
\pgfpathcurveto{\pgfqpoint{2.121018in}{2.056842in}}{\pgfqpoint{2.124290in}{2.048942in}}{\pgfqpoint{2.130114in}{2.043118in}}%
\pgfpathcurveto{\pgfqpoint{2.135938in}{2.037294in}}{\pgfqpoint{2.143838in}{2.034022in}}{\pgfqpoint{2.152074in}{2.034022in}}%
\pgfpathclose%
\pgfusepath{stroke,fill}%
\end{pgfscope}%
\begin{pgfscope}%
\pgfpathrectangle{\pgfqpoint{0.100000in}{0.212622in}}{\pgfqpoint{3.696000in}{3.696000in}}%
\pgfusepath{clip}%
\pgfsetbuttcap%
\pgfsetroundjoin%
\definecolor{currentfill}{rgb}{0.121569,0.466667,0.705882}%
\pgfsetfillcolor{currentfill}%
\pgfsetfillopacity{0.919520}%
\pgfsetlinewidth{1.003750pt}%
\definecolor{currentstroke}{rgb}{0.121569,0.466667,0.705882}%
\pgfsetstrokecolor{currentstroke}%
\pgfsetstrokeopacity{0.919520}%
\pgfsetdash{}{0pt}%
\pgfpathmoveto{\pgfqpoint{1.875784in}{1.937093in}}%
\pgfpathcurveto{\pgfqpoint{1.884021in}{1.937093in}}{\pgfqpoint{1.891921in}{1.940365in}}{\pgfqpoint{1.897745in}{1.946189in}}%
\pgfpathcurveto{\pgfqpoint{1.903568in}{1.952013in}}{\pgfqpoint{1.906841in}{1.959913in}}{\pgfqpoint{1.906841in}{1.968150in}}%
\pgfpathcurveto{\pgfqpoint{1.906841in}{1.976386in}}{\pgfqpoint{1.903568in}{1.984286in}}{\pgfqpoint{1.897745in}{1.990110in}}%
\pgfpathcurveto{\pgfqpoint{1.891921in}{1.995934in}}{\pgfqpoint{1.884021in}{1.999206in}}{\pgfqpoint{1.875784in}{1.999206in}}%
\pgfpathcurveto{\pgfqpoint{1.867548in}{1.999206in}}{\pgfqpoint{1.859648in}{1.995934in}}{\pgfqpoint{1.853824in}{1.990110in}}%
\pgfpathcurveto{\pgfqpoint{1.848000in}{1.984286in}}{\pgfqpoint{1.844728in}{1.976386in}}{\pgfqpoint{1.844728in}{1.968150in}}%
\pgfpathcurveto{\pgfqpoint{1.844728in}{1.959913in}}{\pgfqpoint{1.848000in}{1.952013in}}{\pgfqpoint{1.853824in}{1.946189in}}%
\pgfpathcurveto{\pgfqpoint{1.859648in}{1.940365in}}{\pgfqpoint{1.867548in}{1.937093in}}{\pgfqpoint{1.875784in}{1.937093in}}%
\pgfpathclose%
\pgfusepath{stroke,fill}%
\end{pgfscope}%
\begin{pgfscope}%
\pgfpathrectangle{\pgfqpoint{0.100000in}{0.212622in}}{\pgfqpoint{3.696000in}{3.696000in}}%
\pgfusepath{clip}%
\pgfsetbuttcap%
\pgfsetroundjoin%
\definecolor{currentfill}{rgb}{0.121569,0.466667,0.705882}%
\pgfsetfillcolor{currentfill}%
\pgfsetfillopacity{0.920069}%
\pgfsetlinewidth{1.003750pt}%
\definecolor{currentstroke}{rgb}{0.121569,0.466667,0.705882}%
\pgfsetstrokecolor{currentstroke}%
\pgfsetstrokeopacity{0.920069}%
\pgfsetdash{}{0pt}%
\pgfpathmoveto{\pgfqpoint{1.878925in}{1.934432in}}%
\pgfpathcurveto{\pgfqpoint{1.887161in}{1.934432in}}{\pgfqpoint{1.895061in}{1.937705in}}{\pgfqpoint{1.900885in}{1.943529in}}%
\pgfpathcurveto{\pgfqpoint{1.906709in}{1.949353in}}{\pgfqpoint{1.909982in}{1.957253in}}{\pgfqpoint{1.909982in}{1.965489in}}%
\pgfpathcurveto{\pgfqpoint{1.909982in}{1.973725in}}{\pgfqpoint{1.906709in}{1.981625in}}{\pgfqpoint{1.900885in}{1.987449in}}%
\pgfpathcurveto{\pgfqpoint{1.895061in}{1.993273in}}{\pgfqpoint{1.887161in}{1.996545in}}{\pgfqpoint{1.878925in}{1.996545in}}%
\pgfpathcurveto{\pgfqpoint{1.870689in}{1.996545in}}{\pgfqpoint{1.862789in}{1.993273in}}{\pgfqpoint{1.856965in}{1.987449in}}%
\pgfpathcurveto{\pgfqpoint{1.851141in}{1.981625in}}{\pgfqpoint{1.847869in}{1.973725in}}{\pgfqpoint{1.847869in}{1.965489in}}%
\pgfpathcurveto{\pgfqpoint{1.847869in}{1.957253in}}{\pgfqpoint{1.851141in}{1.949353in}}{\pgfqpoint{1.856965in}{1.943529in}}%
\pgfpathcurveto{\pgfqpoint{1.862789in}{1.937705in}}{\pgfqpoint{1.870689in}{1.934432in}}{\pgfqpoint{1.878925in}{1.934432in}}%
\pgfpathclose%
\pgfusepath{stroke,fill}%
\end{pgfscope}%
\begin{pgfscope}%
\pgfpathrectangle{\pgfqpoint{0.100000in}{0.212622in}}{\pgfqpoint{3.696000in}{3.696000in}}%
\pgfusepath{clip}%
\pgfsetbuttcap%
\pgfsetroundjoin%
\definecolor{currentfill}{rgb}{0.121569,0.466667,0.705882}%
\pgfsetfillcolor{currentfill}%
\pgfsetfillopacity{0.920115}%
\pgfsetlinewidth{1.003750pt}%
\definecolor{currentstroke}{rgb}{0.121569,0.466667,0.705882}%
\pgfsetstrokecolor{currentstroke}%
\pgfsetstrokeopacity{0.920115}%
\pgfsetdash{}{0pt}%
\pgfpathmoveto{\pgfqpoint{2.151542in}{2.028610in}}%
\pgfpathcurveto{\pgfqpoint{2.159778in}{2.028610in}}{\pgfqpoint{2.167678in}{2.031882in}}{\pgfqpoint{2.173502in}{2.037706in}}%
\pgfpathcurveto{\pgfqpoint{2.179326in}{2.043530in}}{\pgfqpoint{2.182598in}{2.051430in}}{\pgfqpoint{2.182598in}{2.059666in}}%
\pgfpathcurveto{\pgfqpoint{2.182598in}{2.067902in}}{\pgfqpoint{2.179326in}{2.075803in}}{\pgfqpoint{2.173502in}{2.081626in}}%
\pgfpathcurveto{\pgfqpoint{2.167678in}{2.087450in}}{\pgfqpoint{2.159778in}{2.090723in}}{\pgfqpoint{2.151542in}{2.090723in}}%
\pgfpathcurveto{\pgfqpoint{2.143306in}{2.090723in}}{\pgfqpoint{2.135406in}{2.087450in}}{\pgfqpoint{2.129582in}{2.081626in}}%
\pgfpathcurveto{\pgfqpoint{2.123758in}{2.075803in}}{\pgfqpoint{2.120485in}{2.067902in}}{\pgfqpoint{2.120485in}{2.059666in}}%
\pgfpathcurveto{\pgfqpoint{2.120485in}{2.051430in}}{\pgfqpoint{2.123758in}{2.043530in}}{\pgfqpoint{2.129582in}{2.037706in}}%
\pgfpathcurveto{\pgfqpoint{2.135406in}{2.031882in}}{\pgfqpoint{2.143306in}{2.028610in}}{\pgfqpoint{2.151542in}{2.028610in}}%
\pgfpathclose%
\pgfusepath{stroke,fill}%
\end{pgfscope}%
\begin{pgfscope}%
\pgfpathrectangle{\pgfqpoint{0.100000in}{0.212622in}}{\pgfqpoint{3.696000in}{3.696000in}}%
\pgfusepath{clip}%
\pgfsetbuttcap%
\pgfsetroundjoin%
\definecolor{currentfill}{rgb}{0.121569,0.466667,0.705882}%
\pgfsetfillcolor{currentfill}%
\pgfsetfillopacity{0.920398}%
\pgfsetlinewidth{1.003750pt}%
\definecolor{currentstroke}{rgb}{0.121569,0.466667,0.705882}%
\pgfsetstrokecolor{currentstroke}%
\pgfsetstrokeopacity{0.920398}%
\pgfsetdash{}{0pt}%
\pgfpathmoveto{\pgfqpoint{1.880678in}{1.933164in}}%
\pgfpathcurveto{\pgfqpoint{1.888914in}{1.933164in}}{\pgfqpoint{1.896814in}{1.936436in}}{\pgfqpoint{1.902638in}{1.942260in}}%
\pgfpathcurveto{\pgfqpoint{1.908462in}{1.948084in}}{\pgfqpoint{1.911734in}{1.955984in}}{\pgfqpoint{1.911734in}{1.964220in}}%
\pgfpathcurveto{\pgfqpoint{1.911734in}{1.972457in}}{\pgfqpoint{1.908462in}{1.980357in}}{\pgfqpoint{1.902638in}{1.986180in}}%
\pgfpathcurveto{\pgfqpoint{1.896814in}{1.992004in}}{\pgfqpoint{1.888914in}{1.995277in}}{\pgfqpoint{1.880678in}{1.995277in}}%
\pgfpathcurveto{\pgfqpoint{1.872441in}{1.995277in}}{\pgfqpoint{1.864541in}{1.992004in}}{\pgfqpoint{1.858717in}{1.986180in}}%
\pgfpathcurveto{\pgfqpoint{1.852893in}{1.980357in}}{\pgfqpoint{1.849621in}{1.972457in}}{\pgfqpoint{1.849621in}{1.964220in}}%
\pgfpathcurveto{\pgfqpoint{1.849621in}{1.955984in}}{\pgfqpoint{1.852893in}{1.948084in}}{\pgfqpoint{1.858717in}{1.942260in}}%
\pgfpathcurveto{\pgfqpoint{1.864541in}{1.936436in}}{\pgfqpoint{1.872441in}{1.933164in}}{\pgfqpoint{1.880678in}{1.933164in}}%
\pgfpathclose%
\pgfusepath{stroke,fill}%
\end{pgfscope}%
\begin{pgfscope}%
\pgfpathrectangle{\pgfqpoint{0.100000in}{0.212622in}}{\pgfqpoint{3.696000in}{3.696000in}}%
\pgfusepath{clip}%
\pgfsetbuttcap%
\pgfsetroundjoin%
\definecolor{currentfill}{rgb}{0.121569,0.466667,0.705882}%
\pgfsetfillcolor{currentfill}%
\pgfsetfillopacity{0.920853}%
\pgfsetlinewidth{1.003750pt}%
\definecolor{currentstroke}{rgb}{0.121569,0.466667,0.705882}%
\pgfsetstrokecolor{currentstroke}%
\pgfsetstrokeopacity{0.920853}%
\pgfsetdash{}{0pt}%
\pgfpathmoveto{\pgfqpoint{1.882651in}{1.931896in}}%
\pgfpathcurveto{\pgfqpoint{1.890887in}{1.931896in}}{\pgfqpoint{1.898787in}{1.935168in}}{\pgfqpoint{1.904611in}{1.940992in}}%
\pgfpathcurveto{\pgfqpoint{1.910435in}{1.946816in}}{\pgfqpoint{1.913707in}{1.954716in}}{\pgfqpoint{1.913707in}{1.962952in}}%
\pgfpathcurveto{\pgfqpoint{1.913707in}{1.971188in}}{\pgfqpoint{1.910435in}{1.979088in}}{\pgfqpoint{1.904611in}{1.984912in}}%
\pgfpathcurveto{\pgfqpoint{1.898787in}{1.990736in}}{\pgfqpoint{1.890887in}{1.994009in}}{\pgfqpoint{1.882651in}{1.994009in}}%
\pgfpathcurveto{\pgfqpoint{1.874415in}{1.994009in}}{\pgfqpoint{1.866515in}{1.990736in}}{\pgfqpoint{1.860691in}{1.984912in}}%
\pgfpathcurveto{\pgfqpoint{1.854867in}{1.979088in}}{\pgfqpoint{1.851594in}{1.971188in}}{\pgfqpoint{1.851594in}{1.962952in}}%
\pgfpathcurveto{\pgfqpoint{1.851594in}{1.954716in}}{\pgfqpoint{1.854867in}{1.946816in}}{\pgfqpoint{1.860691in}{1.940992in}}%
\pgfpathcurveto{\pgfqpoint{1.866515in}{1.935168in}}{\pgfqpoint{1.874415in}{1.931896in}}{\pgfqpoint{1.882651in}{1.931896in}}%
\pgfpathclose%
\pgfusepath{stroke,fill}%
\end{pgfscope}%
\begin{pgfscope}%
\pgfpathrectangle{\pgfqpoint{0.100000in}{0.212622in}}{\pgfqpoint{3.696000in}{3.696000in}}%
\pgfusepath{clip}%
\pgfsetbuttcap%
\pgfsetroundjoin%
\definecolor{currentfill}{rgb}{0.121569,0.466667,0.705882}%
\pgfsetfillcolor{currentfill}%
\pgfsetfillopacity{0.921039}%
\pgfsetlinewidth{1.003750pt}%
\definecolor{currentstroke}{rgb}{0.121569,0.466667,0.705882}%
\pgfsetstrokecolor{currentstroke}%
\pgfsetstrokeopacity{0.921039}%
\pgfsetdash{}{0pt}%
\pgfpathmoveto{\pgfqpoint{2.149745in}{2.024608in}}%
\pgfpathcurveto{\pgfqpoint{2.157981in}{2.024608in}}{\pgfqpoint{2.165881in}{2.027880in}}{\pgfqpoint{2.171705in}{2.033704in}}%
\pgfpathcurveto{\pgfqpoint{2.177529in}{2.039528in}}{\pgfqpoint{2.180802in}{2.047428in}}{\pgfqpoint{2.180802in}{2.055664in}}%
\pgfpathcurveto{\pgfqpoint{2.180802in}{2.063901in}}{\pgfqpoint{2.177529in}{2.071801in}}{\pgfqpoint{2.171705in}{2.077625in}}%
\pgfpathcurveto{\pgfqpoint{2.165881in}{2.083448in}}{\pgfqpoint{2.157981in}{2.086721in}}{\pgfqpoint{2.149745in}{2.086721in}}%
\pgfpathcurveto{\pgfqpoint{2.141509in}{2.086721in}}{\pgfqpoint{2.133609in}{2.083448in}}{\pgfqpoint{2.127785in}{2.077625in}}%
\pgfpathcurveto{\pgfqpoint{2.121961in}{2.071801in}}{\pgfqpoint{2.118689in}{2.063901in}}{\pgfqpoint{2.118689in}{2.055664in}}%
\pgfpathcurveto{\pgfqpoint{2.118689in}{2.047428in}}{\pgfqpoint{2.121961in}{2.039528in}}{\pgfqpoint{2.127785in}{2.033704in}}%
\pgfpathcurveto{\pgfqpoint{2.133609in}{2.027880in}}{\pgfqpoint{2.141509in}{2.024608in}}{\pgfqpoint{2.149745in}{2.024608in}}%
\pgfpathclose%
\pgfusepath{stroke,fill}%
\end{pgfscope}%
\begin{pgfscope}%
\pgfpathrectangle{\pgfqpoint{0.100000in}{0.212622in}}{\pgfqpoint{3.696000in}{3.696000in}}%
\pgfusepath{clip}%
\pgfsetbuttcap%
\pgfsetroundjoin%
\definecolor{currentfill}{rgb}{0.121569,0.466667,0.705882}%
\pgfsetfillcolor{currentfill}%
\pgfsetfillopacity{0.921260}%
\pgfsetlinewidth{1.003750pt}%
\definecolor{currentstroke}{rgb}{0.121569,0.466667,0.705882}%
\pgfsetstrokecolor{currentstroke}%
\pgfsetstrokeopacity{0.921260}%
\pgfsetdash{}{0pt}%
\pgfpathmoveto{\pgfqpoint{1.884949in}{1.930005in}}%
\pgfpathcurveto{\pgfqpoint{1.893185in}{1.930005in}}{\pgfqpoint{1.901085in}{1.933278in}}{\pgfqpoint{1.906909in}{1.939102in}}%
\pgfpathcurveto{\pgfqpoint{1.912733in}{1.944926in}}{\pgfqpoint{1.916005in}{1.952826in}}{\pgfqpoint{1.916005in}{1.961062in}}%
\pgfpathcurveto{\pgfqpoint{1.916005in}{1.969298in}}{\pgfqpoint{1.912733in}{1.977198in}}{\pgfqpoint{1.906909in}{1.983022in}}%
\pgfpathcurveto{\pgfqpoint{1.901085in}{1.988846in}}{\pgfqpoint{1.893185in}{1.992118in}}{\pgfqpoint{1.884949in}{1.992118in}}%
\pgfpathcurveto{\pgfqpoint{1.876713in}{1.992118in}}{\pgfqpoint{1.868813in}{1.988846in}}{\pgfqpoint{1.862989in}{1.983022in}}%
\pgfpathcurveto{\pgfqpoint{1.857165in}{1.977198in}}{\pgfqpoint{1.853892in}{1.969298in}}{\pgfqpoint{1.853892in}{1.961062in}}%
\pgfpathcurveto{\pgfqpoint{1.853892in}{1.952826in}}{\pgfqpoint{1.857165in}{1.944926in}}{\pgfqpoint{1.862989in}{1.939102in}}%
\pgfpathcurveto{\pgfqpoint{1.868813in}{1.933278in}}{\pgfqpoint{1.876713in}{1.930005in}}{\pgfqpoint{1.884949in}{1.930005in}}%
\pgfpathclose%
\pgfusepath{stroke,fill}%
\end{pgfscope}%
\begin{pgfscope}%
\pgfpathrectangle{\pgfqpoint{0.100000in}{0.212622in}}{\pgfqpoint{3.696000in}{3.696000in}}%
\pgfusepath{clip}%
\pgfsetbuttcap%
\pgfsetroundjoin%
\definecolor{currentfill}{rgb}{0.121569,0.466667,0.705882}%
\pgfsetfillcolor{currentfill}%
\pgfsetfillopacity{0.921432}%
\pgfsetlinewidth{1.003750pt}%
\definecolor{currentstroke}{rgb}{0.121569,0.466667,0.705882}%
\pgfsetstrokecolor{currentstroke}%
\pgfsetstrokeopacity{0.921432}%
\pgfsetdash{}{0pt}%
\pgfpathmoveto{\pgfqpoint{2.740603in}{1.238915in}}%
\pgfpathcurveto{\pgfqpoint{2.748840in}{1.238915in}}{\pgfqpoint{2.756740in}{1.242187in}}{\pgfqpoint{2.762564in}{1.248011in}}%
\pgfpathcurveto{\pgfqpoint{2.768388in}{1.253835in}}{\pgfqpoint{2.771660in}{1.261735in}}{\pgfqpoint{2.771660in}{1.269971in}}%
\pgfpathcurveto{\pgfqpoint{2.771660in}{1.278208in}}{\pgfqpoint{2.768388in}{1.286108in}}{\pgfqpoint{2.762564in}{1.291932in}}%
\pgfpathcurveto{\pgfqpoint{2.756740in}{1.297756in}}{\pgfqpoint{2.748840in}{1.301028in}}{\pgfqpoint{2.740603in}{1.301028in}}%
\pgfpathcurveto{\pgfqpoint{2.732367in}{1.301028in}}{\pgfqpoint{2.724467in}{1.297756in}}{\pgfqpoint{2.718643in}{1.291932in}}%
\pgfpathcurveto{\pgfqpoint{2.712819in}{1.286108in}}{\pgfqpoint{2.709547in}{1.278208in}}{\pgfqpoint{2.709547in}{1.269971in}}%
\pgfpathcurveto{\pgfqpoint{2.709547in}{1.261735in}}{\pgfqpoint{2.712819in}{1.253835in}}{\pgfqpoint{2.718643in}{1.248011in}}%
\pgfpathcurveto{\pgfqpoint{2.724467in}{1.242187in}}{\pgfqpoint{2.732367in}{1.238915in}}{\pgfqpoint{2.740603in}{1.238915in}}%
\pgfpathclose%
\pgfusepath{stroke,fill}%
\end{pgfscope}%
\begin{pgfscope}%
\pgfpathrectangle{\pgfqpoint{0.100000in}{0.212622in}}{\pgfqpoint{3.696000in}{3.696000in}}%
\pgfusepath{clip}%
\pgfsetbuttcap%
\pgfsetroundjoin%
\definecolor{currentfill}{rgb}{0.121569,0.466667,0.705882}%
\pgfsetfillcolor{currentfill}%
\pgfsetfillopacity{0.921501}%
\pgfsetlinewidth{1.003750pt}%
\definecolor{currentstroke}{rgb}{0.121569,0.466667,0.705882}%
\pgfsetstrokecolor{currentstroke}%
\pgfsetstrokeopacity{0.921501}%
\pgfsetdash{}{0pt}%
\pgfpathmoveto{\pgfqpoint{1.886212in}{1.929035in}}%
\pgfpathcurveto{\pgfqpoint{1.894448in}{1.929035in}}{\pgfqpoint{1.902348in}{1.932307in}}{\pgfqpoint{1.908172in}{1.938131in}}%
\pgfpathcurveto{\pgfqpoint{1.913996in}{1.943955in}}{\pgfqpoint{1.917268in}{1.951855in}}{\pgfqpoint{1.917268in}{1.960091in}}%
\pgfpathcurveto{\pgfqpoint{1.917268in}{1.968328in}}{\pgfqpoint{1.913996in}{1.976228in}}{\pgfqpoint{1.908172in}{1.982052in}}%
\pgfpathcurveto{\pgfqpoint{1.902348in}{1.987875in}}{\pgfqpoint{1.894448in}{1.991148in}}{\pgfqpoint{1.886212in}{1.991148in}}%
\pgfpathcurveto{\pgfqpoint{1.877976in}{1.991148in}}{\pgfqpoint{1.870075in}{1.987875in}}{\pgfqpoint{1.864252in}{1.982052in}}%
\pgfpathcurveto{\pgfqpoint{1.858428in}{1.976228in}}{\pgfqpoint{1.855155in}{1.968328in}}{\pgfqpoint{1.855155in}{1.960091in}}%
\pgfpathcurveto{\pgfqpoint{1.855155in}{1.951855in}}{\pgfqpoint{1.858428in}{1.943955in}}{\pgfqpoint{1.864252in}{1.938131in}}%
\pgfpathcurveto{\pgfqpoint{1.870075in}{1.932307in}}{\pgfqpoint{1.877976in}{1.929035in}}{\pgfqpoint{1.886212in}{1.929035in}}%
\pgfpathclose%
\pgfusepath{stroke,fill}%
\end{pgfscope}%
\begin{pgfscope}%
\pgfpathrectangle{\pgfqpoint{0.100000in}{0.212622in}}{\pgfqpoint{3.696000in}{3.696000in}}%
\pgfusepath{clip}%
\pgfsetbuttcap%
\pgfsetroundjoin%
\definecolor{currentfill}{rgb}{0.121569,0.466667,0.705882}%
\pgfsetfillcolor{currentfill}%
\pgfsetfillopacity{0.921571}%
\pgfsetlinewidth{1.003750pt}%
\definecolor{currentstroke}{rgb}{0.121569,0.466667,0.705882}%
\pgfsetstrokecolor{currentstroke}%
\pgfsetstrokeopacity{0.921571}%
\pgfsetdash{}{0pt}%
\pgfpathmoveto{\pgfqpoint{2.147986in}{2.021974in}}%
\pgfpathcurveto{\pgfqpoint{2.156223in}{2.021974in}}{\pgfqpoint{2.164123in}{2.025246in}}{\pgfqpoint{2.169947in}{2.031070in}}%
\pgfpathcurveto{\pgfqpoint{2.175771in}{2.036894in}}{\pgfqpoint{2.179043in}{2.044794in}}{\pgfqpoint{2.179043in}{2.053030in}}%
\pgfpathcurveto{\pgfqpoint{2.179043in}{2.061267in}}{\pgfqpoint{2.175771in}{2.069167in}}{\pgfqpoint{2.169947in}{2.074991in}}%
\pgfpathcurveto{\pgfqpoint{2.164123in}{2.080815in}}{\pgfqpoint{2.156223in}{2.084087in}}{\pgfqpoint{2.147986in}{2.084087in}}%
\pgfpathcurveto{\pgfqpoint{2.139750in}{2.084087in}}{\pgfqpoint{2.131850in}{2.080815in}}{\pgfqpoint{2.126026in}{2.074991in}}%
\pgfpathcurveto{\pgfqpoint{2.120202in}{2.069167in}}{\pgfqpoint{2.116930in}{2.061267in}}{\pgfqpoint{2.116930in}{2.053030in}}%
\pgfpathcurveto{\pgfqpoint{2.116930in}{2.044794in}}{\pgfqpoint{2.120202in}{2.036894in}}{\pgfqpoint{2.126026in}{2.031070in}}%
\pgfpathcurveto{\pgfqpoint{2.131850in}{2.025246in}}{\pgfqpoint{2.139750in}{2.021974in}}{\pgfqpoint{2.147986in}{2.021974in}}%
\pgfpathclose%
\pgfusepath{stroke,fill}%
\end{pgfscope}%
\begin{pgfscope}%
\pgfpathrectangle{\pgfqpoint{0.100000in}{0.212622in}}{\pgfqpoint{3.696000in}{3.696000in}}%
\pgfusepath{clip}%
\pgfsetbuttcap%
\pgfsetroundjoin%
\definecolor{currentfill}{rgb}{0.121569,0.466667,0.705882}%
\pgfsetfillcolor{currentfill}%
\pgfsetfillopacity{0.921883}%
\pgfsetlinewidth{1.003750pt}%
\definecolor{currentstroke}{rgb}{0.121569,0.466667,0.705882}%
\pgfsetstrokecolor{currentstroke}%
\pgfsetstrokeopacity{0.921883}%
\pgfsetdash{}{0pt}%
\pgfpathmoveto{\pgfqpoint{1.888112in}{1.927767in}}%
\pgfpathcurveto{\pgfqpoint{1.896348in}{1.927767in}}{\pgfqpoint{1.904248in}{1.931039in}}{\pgfqpoint{1.910072in}{1.936863in}}%
\pgfpathcurveto{\pgfqpoint{1.915896in}{1.942687in}}{\pgfqpoint{1.919168in}{1.950587in}}{\pgfqpoint{1.919168in}{1.958823in}}%
\pgfpathcurveto{\pgfqpoint{1.919168in}{1.967059in}}{\pgfqpoint{1.915896in}{1.974959in}}{\pgfqpoint{1.910072in}{1.980783in}}%
\pgfpathcurveto{\pgfqpoint{1.904248in}{1.986607in}}{\pgfqpoint{1.896348in}{1.989880in}}{\pgfqpoint{1.888112in}{1.989880in}}%
\pgfpathcurveto{\pgfqpoint{1.879876in}{1.989880in}}{\pgfqpoint{1.871976in}{1.986607in}}{\pgfqpoint{1.866152in}{1.980783in}}%
\pgfpathcurveto{\pgfqpoint{1.860328in}{1.974959in}}{\pgfqpoint{1.857055in}{1.967059in}}{\pgfqpoint{1.857055in}{1.958823in}}%
\pgfpathcurveto{\pgfqpoint{1.857055in}{1.950587in}}{\pgfqpoint{1.860328in}{1.942687in}}{\pgfqpoint{1.866152in}{1.936863in}}%
\pgfpathcurveto{\pgfqpoint{1.871976in}{1.931039in}}{\pgfqpoint{1.879876in}{1.927767in}}{\pgfqpoint{1.888112in}{1.927767in}}%
\pgfpathclose%
\pgfusepath{stroke,fill}%
\end{pgfscope}%
\begin{pgfscope}%
\pgfpathrectangle{\pgfqpoint{0.100000in}{0.212622in}}{\pgfqpoint{3.696000in}{3.696000in}}%
\pgfusepath{clip}%
\pgfsetbuttcap%
\pgfsetroundjoin%
\definecolor{currentfill}{rgb}{0.121569,0.466667,0.705882}%
\pgfsetfillcolor{currentfill}%
\pgfsetfillopacity{0.922187}%
\pgfsetlinewidth{1.003750pt}%
\definecolor{currentstroke}{rgb}{0.121569,0.466667,0.705882}%
\pgfsetstrokecolor{currentstroke}%
\pgfsetstrokeopacity{0.922187}%
\pgfsetdash{}{0pt}%
\pgfpathmoveto{\pgfqpoint{1.888420in}{1.057632in}}%
\pgfpathcurveto{\pgfqpoint{1.896656in}{1.057632in}}{\pgfqpoint{1.904556in}{1.060904in}}{\pgfqpoint{1.910380in}{1.066728in}}%
\pgfpathcurveto{\pgfqpoint{1.916204in}{1.072552in}}{\pgfqpoint{1.919477in}{1.080452in}}{\pgfqpoint{1.919477in}{1.088688in}}%
\pgfpathcurveto{\pgfqpoint{1.919477in}{1.096925in}}{\pgfqpoint{1.916204in}{1.104825in}}{\pgfqpoint{1.910380in}{1.110649in}}%
\pgfpathcurveto{\pgfqpoint{1.904556in}{1.116473in}}{\pgfqpoint{1.896656in}{1.119745in}}{\pgfqpoint{1.888420in}{1.119745in}}%
\pgfpathcurveto{\pgfqpoint{1.880184in}{1.119745in}}{\pgfqpoint{1.872284in}{1.116473in}}{\pgfqpoint{1.866460in}{1.110649in}}%
\pgfpathcurveto{\pgfqpoint{1.860636in}{1.104825in}}{\pgfqpoint{1.857364in}{1.096925in}}{\pgfqpoint{1.857364in}{1.088688in}}%
\pgfpathcurveto{\pgfqpoint{1.857364in}{1.080452in}}{\pgfqpoint{1.860636in}{1.072552in}}{\pgfqpoint{1.866460in}{1.066728in}}%
\pgfpathcurveto{\pgfqpoint{1.872284in}{1.060904in}}{\pgfqpoint{1.880184in}{1.057632in}}{\pgfqpoint{1.888420in}{1.057632in}}%
\pgfpathclose%
\pgfusepath{stroke,fill}%
\end{pgfscope}%
\begin{pgfscope}%
\pgfpathrectangle{\pgfqpoint{0.100000in}{0.212622in}}{\pgfqpoint{3.696000in}{3.696000in}}%
\pgfusepath{clip}%
\pgfsetbuttcap%
\pgfsetroundjoin%
\definecolor{currentfill}{rgb}{0.121569,0.466667,0.705882}%
\pgfsetfillcolor{currentfill}%
\pgfsetfillopacity{0.922592}%
\pgfsetlinewidth{1.003750pt}%
\definecolor{currentstroke}{rgb}{0.121569,0.466667,0.705882}%
\pgfsetstrokecolor{currentstroke}%
\pgfsetstrokeopacity{0.922592}%
\pgfsetdash{}{0pt}%
\pgfpathmoveto{\pgfqpoint{1.891882in}{1.925398in}}%
\pgfpathcurveto{\pgfqpoint{1.900118in}{1.925398in}}{\pgfqpoint{1.908018in}{1.928670in}}{\pgfqpoint{1.913842in}{1.934494in}}%
\pgfpathcurveto{\pgfqpoint{1.919666in}{1.940318in}}{\pgfqpoint{1.922939in}{1.948218in}}{\pgfqpoint{1.922939in}{1.956454in}}%
\pgfpathcurveto{\pgfqpoint{1.922939in}{1.964690in}}{\pgfqpoint{1.919666in}{1.972591in}}{\pgfqpoint{1.913842in}{1.978414in}}%
\pgfpathcurveto{\pgfqpoint{1.908018in}{1.984238in}}{\pgfqpoint{1.900118in}{1.987511in}}{\pgfqpoint{1.891882in}{1.987511in}}%
\pgfpathcurveto{\pgfqpoint{1.883646in}{1.987511in}}{\pgfqpoint{1.875746in}{1.984238in}}{\pgfqpoint{1.869922in}{1.978414in}}%
\pgfpathcurveto{\pgfqpoint{1.864098in}{1.972591in}}{\pgfqpoint{1.860826in}{1.964690in}}{\pgfqpoint{1.860826in}{1.956454in}}%
\pgfpathcurveto{\pgfqpoint{1.860826in}{1.948218in}}{\pgfqpoint{1.864098in}{1.940318in}}{\pgfqpoint{1.869922in}{1.934494in}}%
\pgfpathcurveto{\pgfqpoint{1.875746in}{1.928670in}}{\pgfqpoint{1.883646in}{1.925398in}}{\pgfqpoint{1.891882in}{1.925398in}}%
\pgfpathclose%
\pgfusepath{stroke,fill}%
\end{pgfscope}%
\begin{pgfscope}%
\pgfpathrectangle{\pgfqpoint{0.100000in}{0.212622in}}{\pgfqpoint{3.696000in}{3.696000in}}%
\pgfusepath{clip}%
\pgfsetbuttcap%
\pgfsetroundjoin%
\definecolor{currentfill}{rgb}{0.121569,0.466667,0.705882}%
\pgfsetfillcolor{currentfill}%
\pgfsetfillopacity{0.922777}%
\pgfsetlinewidth{1.003750pt}%
\definecolor{currentstroke}{rgb}{0.121569,0.466667,0.705882}%
\pgfsetstrokecolor{currentstroke}%
\pgfsetstrokeopacity{0.922777}%
\pgfsetdash{}{0pt}%
\pgfpathmoveto{\pgfqpoint{2.145567in}{2.017282in}}%
\pgfpathcurveto{\pgfqpoint{2.153803in}{2.017282in}}{\pgfqpoint{2.161703in}{2.020554in}}{\pgfqpoint{2.167527in}{2.026378in}}%
\pgfpathcurveto{\pgfqpoint{2.173351in}{2.032202in}}{\pgfqpoint{2.176624in}{2.040102in}}{\pgfqpoint{2.176624in}{2.048338in}}%
\pgfpathcurveto{\pgfqpoint{2.176624in}{2.056575in}}{\pgfqpoint{2.173351in}{2.064475in}}{\pgfqpoint{2.167527in}{2.070299in}}%
\pgfpathcurveto{\pgfqpoint{2.161703in}{2.076123in}}{\pgfqpoint{2.153803in}{2.079395in}}{\pgfqpoint{2.145567in}{2.079395in}}%
\pgfpathcurveto{\pgfqpoint{2.137331in}{2.079395in}}{\pgfqpoint{2.129431in}{2.076123in}}{\pgfqpoint{2.123607in}{2.070299in}}%
\pgfpathcurveto{\pgfqpoint{2.117783in}{2.064475in}}{\pgfqpoint{2.114511in}{2.056575in}}{\pgfqpoint{2.114511in}{2.048338in}}%
\pgfpathcurveto{\pgfqpoint{2.114511in}{2.040102in}}{\pgfqpoint{2.117783in}{2.032202in}}{\pgfqpoint{2.123607in}{2.026378in}}%
\pgfpathcurveto{\pgfqpoint{2.129431in}{2.020554in}}{\pgfqpoint{2.137331in}{2.017282in}}{\pgfqpoint{2.145567in}{2.017282in}}%
\pgfpathclose%
\pgfusepath{stroke,fill}%
\end{pgfscope}%
\begin{pgfscope}%
\pgfpathrectangle{\pgfqpoint{0.100000in}{0.212622in}}{\pgfqpoint{3.696000in}{3.696000in}}%
\pgfusepath{clip}%
\pgfsetbuttcap%
\pgfsetroundjoin%
\definecolor{currentfill}{rgb}{0.121569,0.466667,0.705882}%
\pgfsetfillcolor{currentfill}%
\pgfsetfillopacity{0.923455}%
\pgfsetlinewidth{1.003750pt}%
\definecolor{currentstroke}{rgb}{0.121569,0.466667,0.705882}%
\pgfsetstrokecolor{currentstroke}%
\pgfsetstrokeopacity{0.923455}%
\pgfsetdash{}{0pt}%
\pgfpathmoveto{\pgfqpoint{1.896513in}{1.922807in}}%
\pgfpathcurveto{\pgfqpoint{1.904749in}{1.922807in}}{\pgfqpoint{1.912650in}{1.926080in}}{\pgfqpoint{1.918473in}{1.931904in}}%
\pgfpathcurveto{\pgfqpoint{1.924297in}{1.937728in}}{\pgfqpoint{1.927570in}{1.945628in}}{\pgfqpoint{1.927570in}{1.953864in}}%
\pgfpathcurveto{\pgfqpoint{1.927570in}{1.962100in}}{\pgfqpoint{1.924297in}{1.970000in}}{\pgfqpoint{1.918473in}{1.975824in}}%
\pgfpathcurveto{\pgfqpoint{1.912650in}{1.981648in}}{\pgfqpoint{1.904749in}{1.984920in}}{\pgfqpoint{1.896513in}{1.984920in}}%
\pgfpathcurveto{\pgfqpoint{1.888277in}{1.984920in}}{\pgfqpoint{1.880377in}{1.981648in}}{\pgfqpoint{1.874553in}{1.975824in}}%
\pgfpathcurveto{\pgfqpoint{1.868729in}{1.970000in}}{\pgfqpoint{1.865457in}{1.962100in}}{\pgfqpoint{1.865457in}{1.953864in}}%
\pgfpathcurveto{\pgfqpoint{1.865457in}{1.945628in}}{\pgfqpoint{1.868729in}{1.937728in}}{\pgfqpoint{1.874553in}{1.931904in}}%
\pgfpathcurveto{\pgfqpoint{1.880377in}{1.926080in}}{\pgfqpoint{1.888277in}{1.922807in}}{\pgfqpoint{1.896513in}{1.922807in}}%
\pgfpathclose%
\pgfusepath{stroke,fill}%
\end{pgfscope}%
\begin{pgfscope}%
\pgfpathrectangle{\pgfqpoint{0.100000in}{0.212622in}}{\pgfqpoint{3.696000in}{3.696000in}}%
\pgfusepath{clip}%
\pgfsetbuttcap%
\pgfsetroundjoin%
\definecolor{currentfill}{rgb}{0.121569,0.466667,0.705882}%
\pgfsetfillcolor{currentfill}%
\pgfsetfillopacity{0.923758}%
\pgfsetlinewidth{1.003750pt}%
\definecolor{currentstroke}{rgb}{0.121569,0.466667,0.705882}%
\pgfsetstrokecolor{currentstroke}%
\pgfsetstrokeopacity{0.923758}%
\pgfsetdash{}{0pt}%
\pgfpathmoveto{\pgfqpoint{2.144408in}{2.012010in}}%
\pgfpathcurveto{\pgfqpoint{2.152644in}{2.012010in}}{\pgfqpoint{2.160544in}{2.015282in}}{\pgfqpoint{2.166368in}{2.021106in}}%
\pgfpathcurveto{\pgfqpoint{2.172192in}{2.026930in}}{\pgfqpoint{2.175465in}{2.034830in}}{\pgfqpoint{2.175465in}{2.043067in}}%
\pgfpathcurveto{\pgfqpoint{2.175465in}{2.051303in}}{\pgfqpoint{2.172192in}{2.059203in}}{\pgfqpoint{2.166368in}{2.065027in}}%
\pgfpathcurveto{\pgfqpoint{2.160544in}{2.070851in}}{\pgfqpoint{2.152644in}{2.074123in}}{\pgfqpoint{2.144408in}{2.074123in}}%
\pgfpathcurveto{\pgfqpoint{2.136172in}{2.074123in}}{\pgfqpoint{2.128272in}{2.070851in}}{\pgfqpoint{2.122448in}{2.065027in}}%
\pgfpathcurveto{\pgfqpoint{2.116624in}{2.059203in}}{\pgfqpoint{2.113352in}{2.051303in}}{\pgfqpoint{2.113352in}{2.043067in}}%
\pgfpathcurveto{\pgfqpoint{2.113352in}{2.034830in}}{\pgfqpoint{2.116624in}{2.026930in}}{\pgfqpoint{2.122448in}{2.021106in}}%
\pgfpathcurveto{\pgfqpoint{2.128272in}{2.015282in}}{\pgfqpoint{2.136172in}{2.012010in}}{\pgfqpoint{2.144408in}{2.012010in}}%
\pgfpathclose%
\pgfusepath{stroke,fill}%
\end{pgfscope}%
\begin{pgfscope}%
\pgfpathrectangle{\pgfqpoint{0.100000in}{0.212622in}}{\pgfqpoint{3.696000in}{3.696000in}}%
\pgfusepath{clip}%
\pgfsetbuttcap%
\pgfsetroundjoin%
\definecolor{currentfill}{rgb}{0.121569,0.466667,0.705882}%
\pgfsetfillcolor{currentfill}%
\pgfsetfillopacity{0.923988}%
\pgfsetlinewidth{1.003750pt}%
\definecolor{currentstroke}{rgb}{0.121569,0.466667,0.705882}%
\pgfsetstrokecolor{currentstroke}%
\pgfsetstrokeopacity{0.923988}%
\pgfsetdash{}{0pt}%
\pgfpathmoveto{\pgfqpoint{1.899108in}{1.921794in}}%
\pgfpathcurveto{\pgfqpoint{1.907345in}{1.921794in}}{\pgfqpoint{1.915245in}{1.925066in}}{\pgfqpoint{1.921069in}{1.930890in}}%
\pgfpathcurveto{\pgfqpoint{1.926893in}{1.936714in}}{\pgfqpoint{1.930165in}{1.944614in}}{\pgfqpoint{1.930165in}{1.952850in}}%
\pgfpathcurveto{\pgfqpoint{1.930165in}{1.961087in}}{\pgfqpoint{1.926893in}{1.968987in}}{\pgfqpoint{1.921069in}{1.974811in}}%
\pgfpathcurveto{\pgfqpoint{1.915245in}{1.980635in}}{\pgfqpoint{1.907345in}{1.983907in}}{\pgfqpoint{1.899108in}{1.983907in}}%
\pgfpathcurveto{\pgfqpoint{1.890872in}{1.983907in}}{\pgfqpoint{1.882972in}{1.980635in}}{\pgfqpoint{1.877148in}{1.974811in}}%
\pgfpathcurveto{\pgfqpoint{1.871324in}{1.968987in}}{\pgfqpoint{1.868052in}{1.961087in}}{\pgfqpoint{1.868052in}{1.952850in}}%
\pgfpathcurveto{\pgfqpoint{1.868052in}{1.944614in}}{\pgfqpoint{1.871324in}{1.936714in}}{\pgfqpoint{1.877148in}{1.930890in}}%
\pgfpathcurveto{\pgfqpoint{1.882972in}{1.925066in}}{\pgfqpoint{1.890872in}{1.921794in}}{\pgfqpoint{1.899108in}{1.921794in}}%
\pgfpathclose%
\pgfusepath{stroke,fill}%
\end{pgfscope}%
\begin{pgfscope}%
\pgfpathrectangle{\pgfqpoint{0.100000in}{0.212622in}}{\pgfqpoint{3.696000in}{3.696000in}}%
\pgfusepath{clip}%
\pgfsetbuttcap%
\pgfsetroundjoin%
\definecolor{currentfill}{rgb}{0.121569,0.466667,0.705882}%
\pgfsetfillcolor{currentfill}%
\pgfsetfillopacity{0.924239}%
\pgfsetlinewidth{1.003750pt}%
\definecolor{currentstroke}{rgb}{0.121569,0.466667,0.705882}%
\pgfsetstrokecolor{currentstroke}%
\pgfsetstrokeopacity{0.924239}%
\pgfsetdash{}{0pt}%
\pgfpathmoveto{\pgfqpoint{1.900452in}{1.920788in}}%
\pgfpathcurveto{\pgfqpoint{1.908688in}{1.920788in}}{\pgfqpoint{1.916588in}{1.924060in}}{\pgfqpoint{1.922412in}{1.929884in}}%
\pgfpathcurveto{\pgfqpoint{1.928236in}{1.935708in}}{\pgfqpoint{1.931508in}{1.943608in}}{\pgfqpoint{1.931508in}{1.951844in}}%
\pgfpathcurveto{\pgfqpoint{1.931508in}{1.960080in}}{\pgfqpoint{1.928236in}{1.967980in}}{\pgfqpoint{1.922412in}{1.973804in}}%
\pgfpathcurveto{\pgfqpoint{1.916588in}{1.979628in}}{\pgfqpoint{1.908688in}{1.982901in}}{\pgfqpoint{1.900452in}{1.982901in}}%
\pgfpathcurveto{\pgfqpoint{1.892215in}{1.982901in}}{\pgfqpoint{1.884315in}{1.979628in}}{\pgfqpoint{1.878491in}{1.973804in}}%
\pgfpathcurveto{\pgfqpoint{1.872667in}{1.967980in}}{\pgfqpoint{1.869395in}{1.960080in}}{\pgfqpoint{1.869395in}{1.951844in}}%
\pgfpathcurveto{\pgfqpoint{1.869395in}{1.943608in}}{\pgfqpoint{1.872667in}{1.935708in}}{\pgfqpoint{1.878491in}{1.929884in}}%
\pgfpathcurveto{\pgfqpoint{1.884315in}{1.924060in}}{\pgfqpoint{1.892215in}{1.920788in}}{\pgfqpoint{1.900452in}{1.920788in}}%
\pgfpathclose%
\pgfusepath{stroke,fill}%
\end{pgfscope}%
\begin{pgfscope}%
\pgfpathrectangle{\pgfqpoint{0.100000in}{0.212622in}}{\pgfqpoint{3.696000in}{3.696000in}}%
\pgfusepath{clip}%
\pgfsetbuttcap%
\pgfsetroundjoin%
\definecolor{currentfill}{rgb}{0.121569,0.466667,0.705882}%
\pgfsetfillcolor{currentfill}%
\pgfsetfillopacity{0.924255}%
\pgfsetlinewidth{1.003750pt}%
\definecolor{currentstroke}{rgb}{0.121569,0.466667,0.705882}%
\pgfsetstrokecolor{currentstroke}%
\pgfsetstrokeopacity{0.924255}%
\pgfsetdash{}{0pt}%
\pgfpathmoveto{\pgfqpoint{2.142929in}{2.009861in}}%
\pgfpathcurveto{\pgfqpoint{2.151165in}{2.009861in}}{\pgfqpoint{2.159065in}{2.013134in}}{\pgfqpoint{2.164889in}{2.018958in}}%
\pgfpathcurveto{\pgfqpoint{2.170713in}{2.024781in}}{\pgfqpoint{2.173985in}{2.032682in}}{\pgfqpoint{2.173985in}{2.040918in}}%
\pgfpathcurveto{\pgfqpoint{2.173985in}{2.049154in}}{\pgfqpoint{2.170713in}{2.057054in}}{\pgfqpoint{2.164889in}{2.062878in}}%
\pgfpathcurveto{\pgfqpoint{2.159065in}{2.068702in}}{\pgfqpoint{2.151165in}{2.071974in}}{\pgfqpoint{2.142929in}{2.071974in}}%
\pgfpathcurveto{\pgfqpoint{2.134692in}{2.071974in}}{\pgfqpoint{2.126792in}{2.068702in}}{\pgfqpoint{2.120968in}{2.062878in}}%
\pgfpathcurveto{\pgfqpoint{2.115144in}{2.057054in}}{\pgfqpoint{2.111872in}{2.049154in}}{\pgfqpoint{2.111872in}{2.040918in}}%
\pgfpathcurveto{\pgfqpoint{2.111872in}{2.032682in}}{\pgfqpoint{2.115144in}{2.024781in}}{\pgfqpoint{2.120968in}{2.018958in}}%
\pgfpathcurveto{\pgfqpoint{2.126792in}{2.013134in}}{\pgfqpoint{2.134692in}{2.009861in}}{\pgfqpoint{2.142929in}{2.009861in}}%
\pgfpathclose%
\pgfusepath{stroke,fill}%
\end{pgfscope}%
\begin{pgfscope}%
\pgfpathrectangle{\pgfqpoint{0.100000in}{0.212622in}}{\pgfqpoint{3.696000in}{3.696000in}}%
\pgfusepath{clip}%
\pgfsetbuttcap%
\pgfsetroundjoin%
\definecolor{currentfill}{rgb}{0.121569,0.466667,0.705882}%
\pgfsetfillcolor{currentfill}%
\pgfsetfillopacity{0.924534}%
\pgfsetlinewidth{1.003750pt}%
\definecolor{currentstroke}{rgb}{0.121569,0.466667,0.705882}%
\pgfsetstrokecolor{currentstroke}%
\pgfsetstrokeopacity{0.924534}%
\pgfsetdash{}{0pt}%
\pgfpathmoveto{\pgfqpoint{1.902250in}{1.919585in}}%
\pgfpathcurveto{\pgfqpoint{1.910486in}{1.919585in}}{\pgfqpoint{1.918386in}{1.922858in}}{\pgfqpoint{1.924210in}{1.928682in}}%
\pgfpathcurveto{\pgfqpoint{1.930034in}{1.934505in}}{\pgfqpoint{1.933307in}{1.942406in}}{\pgfqpoint{1.933307in}{1.950642in}}%
\pgfpathcurveto{\pgfqpoint{1.933307in}{1.958878in}}{\pgfqpoint{1.930034in}{1.966778in}}{\pgfqpoint{1.924210in}{1.972602in}}%
\pgfpathcurveto{\pgfqpoint{1.918386in}{1.978426in}}{\pgfqpoint{1.910486in}{1.981698in}}{\pgfqpoint{1.902250in}{1.981698in}}%
\pgfpathcurveto{\pgfqpoint{1.894014in}{1.981698in}}{\pgfqpoint{1.886114in}{1.978426in}}{\pgfqpoint{1.880290in}{1.972602in}}%
\pgfpathcurveto{\pgfqpoint{1.874466in}{1.966778in}}{\pgfqpoint{1.871194in}{1.958878in}}{\pgfqpoint{1.871194in}{1.950642in}}%
\pgfpathcurveto{\pgfqpoint{1.871194in}{1.942406in}}{\pgfqpoint{1.874466in}{1.934505in}}{\pgfqpoint{1.880290in}{1.928682in}}%
\pgfpathcurveto{\pgfqpoint{1.886114in}{1.922858in}}{\pgfqpoint{1.894014in}{1.919585in}}{\pgfqpoint{1.902250in}{1.919585in}}%
\pgfpathclose%
\pgfusepath{stroke,fill}%
\end{pgfscope}%
\begin{pgfscope}%
\pgfpathrectangle{\pgfqpoint{0.100000in}{0.212622in}}{\pgfqpoint{3.696000in}{3.696000in}}%
\pgfusepath{clip}%
\pgfsetbuttcap%
\pgfsetroundjoin%
\definecolor{currentfill}{rgb}{0.121569,0.466667,0.705882}%
\pgfsetfillcolor{currentfill}%
\pgfsetfillopacity{0.924600}%
\pgfsetlinewidth{1.003750pt}%
\definecolor{currentstroke}{rgb}{0.121569,0.466667,0.705882}%
\pgfsetstrokecolor{currentstroke}%
\pgfsetstrokeopacity{0.924600}%
\pgfsetdash{}{0pt}%
\pgfpathmoveto{\pgfqpoint{2.726236in}{1.218664in}}%
\pgfpathcurveto{\pgfqpoint{2.734473in}{1.218664in}}{\pgfqpoint{2.742373in}{1.221936in}}{\pgfqpoint{2.748197in}{1.227760in}}%
\pgfpathcurveto{\pgfqpoint{2.754021in}{1.233584in}}{\pgfqpoint{2.757293in}{1.241484in}}{\pgfqpoint{2.757293in}{1.249720in}}%
\pgfpathcurveto{\pgfqpoint{2.757293in}{1.257957in}}{\pgfqpoint{2.754021in}{1.265857in}}{\pgfqpoint{2.748197in}{1.271681in}}%
\pgfpathcurveto{\pgfqpoint{2.742373in}{1.277505in}}{\pgfqpoint{2.734473in}{1.280777in}}{\pgfqpoint{2.726236in}{1.280777in}}%
\pgfpathcurveto{\pgfqpoint{2.718000in}{1.280777in}}{\pgfqpoint{2.710100in}{1.277505in}}{\pgfqpoint{2.704276in}{1.271681in}}%
\pgfpathcurveto{\pgfqpoint{2.698452in}{1.265857in}}{\pgfqpoint{2.695180in}{1.257957in}}{\pgfqpoint{2.695180in}{1.249720in}}%
\pgfpathcurveto{\pgfqpoint{2.695180in}{1.241484in}}{\pgfqpoint{2.698452in}{1.233584in}}{\pgfqpoint{2.704276in}{1.227760in}}%
\pgfpathcurveto{\pgfqpoint{2.710100in}{1.221936in}}{\pgfqpoint{2.718000in}{1.218664in}}{\pgfqpoint{2.726236in}{1.218664in}}%
\pgfpathclose%
\pgfusepath{stroke,fill}%
\end{pgfscope}%
\begin{pgfscope}%
\pgfpathrectangle{\pgfqpoint{0.100000in}{0.212622in}}{\pgfqpoint{3.696000in}{3.696000in}}%
\pgfusepath{clip}%
\pgfsetbuttcap%
\pgfsetroundjoin%
\definecolor{currentfill}{rgb}{0.121569,0.466667,0.705882}%
\pgfsetfillcolor{currentfill}%
\pgfsetfillopacity{0.925017}%
\pgfsetlinewidth{1.003750pt}%
\definecolor{currentstroke}{rgb}{0.121569,0.466667,0.705882}%
\pgfsetstrokecolor{currentstroke}%
\pgfsetstrokeopacity{0.925017}%
\pgfsetdash{}{0pt}%
\pgfpathmoveto{\pgfqpoint{2.140329in}{2.005189in}}%
\pgfpathcurveto{\pgfqpoint{2.148565in}{2.005189in}}{\pgfqpoint{2.156465in}{2.008461in}}{\pgfqpoint{2.162289in}{2.014285in}}%
\pgfpathcurveto{\pgfqpoint{2.168113in}{2.020109in}}{\pgfqpoint{2.171386in}{2.028009in}}{\pgfqpoint{2.171386in}{2.036245in}}%
\pgfpathcurveto{\pgfqpoint{2.171386in}{2.044481in}}{\pgfqpoint{2.168113in}{2.052381in}}{\pgfqpoint{2.162289in}{2.058205in}}%
\pgfpathcurveto{\pgfqpoint{2.156465in}{2.064029in}}{\pgfqpoint{2.148565in}{2.067302in}}{\pgfqpoint{2.140329in}{2.067302in}}%
\pgfpathcurveto{\pgfqpoint{2.132093in}{2.067302in}}{\pgfqpoint{2.124193in}{2.064029in}}{\pgfqpoint{2.118369in}{2.058205in}}%
\pgfpathcurveto{\pgfqpoint{2.112545in}{2.052381in}}{\pgfqpoint{2.109273in}{2.044481in}}{\pgfqpoint{2.109273in}{2.036245in}}%
\pgfpathcurveto{\pgfqpoint{2.109273in}{2.028009in}}{\pgfqpoint{2.112545in}{2.020109in}}{\pgfqpoint{2.118369in}{2.014285in}}%
\pgfpathcurveto{\pgfqpoint{2.124193in}{2.008461in}}{\pgfqpoint{2.132093in}{2.005189in}}{\pgfqpoint{2.140329in}{2.005189in}}%
\pgfpathclose%
\pgfusepath{stroke,fill}%
\end{pgfscope}%
\begin{pgfscope}%
\pgfpathrectangle{\pgfqpoint{0.100000in}{0.212622in}}{\pgfqpoint{3.696000in}{3.696000in}}%
\pgfusepath{clip}%
\pgfsetbuttcap%
\pgfsetroundjoin%
\definecolor{currentfill}{rgb}{0.121569,0.466667,0.705882}%
\pgfsetfillcolor{currentfill}%
\pgfsetfillopacity{0.925200}%
\pgfsetlinewidth{1.003750pt}%
\definecolor{currentstroke}{rgb}{0.121569,0.466667,0.705882}%
\pgfsetstrokecolor{currentstroke}%
\pgfsetstrokeopacity{0.925200}%
\pgfsetdash{}{0pt}%
\pgfpathmoveto{\pgfqpoint{1.904953in}{1.918231in}}%
\pgfpathcurveto{\pgfqpoint{1.913189in}{1.918231in}}{\pgfqpoint{1.921089in}{1.921503in}}{\pgfqpoint{1.926913in}{1.927327in}}%
\pgfpathcurveto{\pgfqpoint{1.932737in}{1.933151in}}{\pgfqpoint{1.936009in}{1.941051in}}{\pgfqpoint{1.936009in}{1.949288in}}%
\pgfpathcurveto{\pgfqpoint{1.936009in}{1.957524in}}{\pgfqpoint{1.932737in}{1.965424in}}{\pgfqpoint{1.926913in}{1.971248in}}%
\pgfpathcurveto{\pgfqpoint{1.921089in}{1.977072in}}{\pgfqpoint{1.913189in}{1.980344in}}{\pgfqpoint{1.904953in}{1.980344in}}%
\pgfpathcurveto{\pgfqpoint{1.896717in}{1.980344in}}{\pgfqpoint{1.888816in}{1.977072in}}{\pgfqpoint{1.882993in}{1.971248in}}%
\pgfpathcurveto{\pgfqpoint{1.877169in}{1.965424in}}{\pgfqpoint{1.873896in}{1.957524in}}{\pgfqpoint{1.873896in}{1.949288in}}%
\pgfpathcurveto{\pgfqpoint{1.873896in}{1.941051in}}{\pgfqpoint{1.877169in}{1.933151in}}{\pgfqpoint{1.882993in}{1.927327in}}%
\pgfpathcurveto{\pgfqpoint{1.888816in}{1.921503in}}{\pgfqpoint{1.896717in}{1.918231in}}{\pgfqpoint{1.904953in}{1.918231in}}%
\pgfpathclose%
\pgfusepath{stroke,fill}%
\end{pgfscope}%
\begin{pgfscope}%
\pgfpathrectangle{\pgfqpoint{0.100000in}{0.212622in}}{\pgfqpoint{3.696000in}{3.696000in}}%
\pgfusepath{clip}%
\pgfsetbuttcap%
\pgfsetroundjoin%
\definecolor{currentfill}{rgb}{0.121569,0.466667,0.705882}%
\pgfsetfillcolor{currentfill}%
\pgfsetfillopacity{0.925389}%
\pgfsetlinewidth{1.003750pt}%
\definecolor{currentstroke}{rgb}{0.121569,0.466667,0.705882}%
\pgfsetstrokecolor{currentstroke}%
\pgfsetstrokeopacity{0.925389}%
\pgfsetdash{}{0pt}%
\pgfpathmoveto{\pgfqpoint{1.908227in}{1.052880in}}%
\pgfpathcurveto{\pgfqpoint{1.916463in}{1.052880in}}{\pgfqpoint{1.924363in}{1.056152in}}{\pgfqpoint{1.930187in}{1.061976in}}%
\pgfpathcurveto{\pgfqpoint{1.936011in}{1.067800in}}{\pgfqpoint{1.939284in}{1.075700in}}{\pgfqpoint{1.939284in}{1.083936in}}%
\pgfpathcurveto{\pgfqpoint{1.939284in}{1.092172in}}{\pgfqpoint{1.936011in}{1.100073in}}{\pgfqpoint{1.930187in}{1.105896in}}%
\pgfpathcurveto{\pgfqpoint{1.924363in}{1.111720in}}{\pgfqpoint{1.916463in}{1.114993in}}{\pgfqpoint{1.908227in}{1.114993in}}%
\pgfpathcurveto{\pgfqpoint{1.899991in}{1.114993in}}{\pgfqpoint{1.892091in}{1.111720in}}{\pgfqpoint{1.886267in}{1.105896in}}%
\pgfpathcurveto{\pgfqpoint{1.880443in}{1.100073in}}{\pgfqpoint{1.877171in}{1.092172in}}{\pgfqpoint{1.877171in}{1.083936in}}%
\pgfpathcurveto{\pgfqpoint{1.877171in}{1.075700in}}{\pgfqpoint{1.880443in}{1.067800in}}{\pgfqpoint{1.886267in}{1.061976in}}%
\pgfpathcurveto{\pgfqpoint{1.892091in}{1.056152in}}{\pgfqpoint{1.899991in}{1.052880in}}{\pgfqpoint{1.908227in}{1.052880in}}%
\pgfpathclose%
\pgfusepath{stroke,fill}%
\end{pgfscope}%
\begin{pgfscope}%
\pgfpathrectangle{\pgfqpoint{0.100000in}{0.212622in}}{\pgfqpoint{3.696000in}{3.696000in}}%
\pgfusepath{clip}%
\pgfsetbuttcap%
\pgfsetroundjoin%
\definecolor{currentfill}{rgb}{0.121569,0.466667,0.705882}%
\pgfsetfillcolor{currentfill}%
\pgfsetfillopacity{0.925588}%
\pgfsetlinewidth{1.003750pt}%
\definecolor{currentstroke}{rgb}{0.121569,0.466667,0.705882}%
\pgfsetstrokecolor{currentstroke}%
\pgfsetstrokeopacity{0.925588}%
\pgfsetdash{}{0pt}%
\pgfpathmoveto{\pgfqpoint{2.139590in}{2.001686in}}%
\pgfpathcurveto{\pgfqpoint{2.147826in}{2.001686in}}{\pgfqpoint{2.155726in}{2.004958in}}{\pgfqpoint{2.161550in}{2.010782in}}%
\pgfpathcurveto{\pgfqpoint{2.167374in}{2.016606in}}{\pgfqpoint{2.170647in}{2.024506in}}{\pgfqpoint{2.170647in}{2.032742in}}%
\pgfpathcurveto{\pgfqpoint{2.170647in}{2.040979in}}{\pgfqpoint{2.167374in}{2.048879in}}{\pgfqpoint{2.161550in}{2.054703in}}%
\pgfpathcurveto{\pgfqpoint{2.155726in}{2.060526in}}{\pgfqpoint{2.147826in}{2.063799in}}{\pgfqpoint{2.139590in}{2.063799in}}%
\pgfpathcurveto{\pgfqpoint{2.131354in}{2.063799in}}{\pgfqpoint{2.123454in}{2.060526in}}{\pgfqpoint{2.117630in}{2.054703in}}%
\pgfpathcurveto{\pgfqpoint{2.111806in}{2.048879in}}{\pgfqpoint{2.108534in}{2.040979in}}{\pgfqpoint{2.108534in}{2.032742in}}%
\pgfpathcurveto{\pgfqpoint{2.108534in}{2.024506in}}{\pgfqpoint{2.111806in}{2.016606in}}{\pgfqpoint{2.117630in}{2.010782in}}%
\pgfpathcurveto{\pgfqpoint{2.123454in}{2.004958in}}{\pgfqpoint{2.131354in}{2.001686in}}{\pgfqpoint{2.139590in}{2.001686in}}%
\pgfpathclose%
\pgfusepath{stroke,fill}%
\end{pgfscope}%
\begin{pgfscope}%
\pgfpathrectangle{\pgfqpoint{0.100000in}{0.212622in}}{\pgfqpoint{3.696000in}{3.696000in}}%
\pgfusepath{clip}%
\pgfsetbuttcap%
\pgfsetroundjoin%
\definecolor{currentfill}{rgb}{0.121569,0.466667,0.705882}%
\pgfsetfillcolor{currentfill}%
\pgfsetfillopacity{0.925914}%
\pgfsetlinewidth{1.003750pt}%
\definecolor{currentstroke}{rgb}{0.121569,0.466667,0.705882}%
\pgfsetstrokecolor{currentstroke}%
\pgfsetstrokeopacity{0.925914}%
\pgfsetdash{}{0pt}%
\pgfpathmoveto{\pgfqpoint{1.908472in}{1.915323in}}%
\pgfpathcurveto{\pgfqpoint{1.916708in}{1.915323in}}{\pgfqpoint{1.924608in}{1.918595in}}{\pgfqpoint{1.930432in}{1.924419in}}%
\pgfpathcurveto{\pgfqpoint{1.936256in}{1.930243in}}{\pgfqpoint{1.939528in}{1.938143in}}{\pgfqpoint{1.939528in}{1.946379in}}%
\pgfpathcurveto{\pgfqpoint{1.939528in}{1.954615in}}{\pgfqpoint{1.936256in}{1.962515in}}{\pgfqpoint{1.930432in}{1.968339in}}%
\pgfpathcurveto{\pgfqpoint{1.924608in}{1.974163in}}{\pgfqpoint{1.916708in}{1.977436in}}{\pgfqpoint{1.908472in}{1.977436in}}%
\pgfpathcurveto{\pgfqpoint{1.900236in}{1.977436in}}{\pgfqpoint{1.892336in}{1.974163in}}{\pgfqpoint{1.886512in}{1.968339in}}%
\pgfpathcurveto{\pgfqpoint{1.880688in}{1.962515in}}{\pgfqpoint{1.877415in}{1.954615in}}{\pgfqpoint{1.877415in}{1.946379in}}%
\pgfpathcurveto{\pgfqpoint{1.877415in}{1.938143in}}{\pgfqpoint{1.880688in}{1.930243in}}{\pgfqpoint{1.886512in}{1.924419in}}%
\pgfpathcurveto{\pgfqpoint{1.892336in}{1.918595in}}{\pgfqpoint{1.900236in}{1.915323in}}{\pgfqpoint{1.908472in}{1.915323in}}%
\pgfpathclose%
\pgfusepath{stroke,fill}%
\end{pgfscope}%
\begin{pgfscope}%
\pgfpathrectangle{\pgfqpoint{0.100000in}{0.212622in}}{\pgfqpoint{3.696000in}{3.696000in}}%
\pgfusepath{clip}%
\pgfsetbuttcap%
\pgfsetroundjoin%
\definecolor{currentfill}{rgb}{0.121569,0.466667,0.705882}%
\pgfsetfillcolor{currentfill}%
\pgfsetfillopacity{0.926015}%
\pgfsetlinewidth{1.003750pt}%
\definecolor{currentstroke}{rgb}{0.121569,0.466667,0.705882}%
\pgfsetstrokecolor{currentstroke}%
\pgfsetstrokeopacity{0.926015}%
\pgfsetdash{}{0pt}%
\pgfpathmoveto{\pgfqpoint{2.138249in}{1.999602in}}%
\pgfpathcurveto{\pgfqpoint{2.146486in}{1.999602in}}{\pgfqpoint{2.154386in}{2.002875in}}{\pgfqpoint{2.160210in}{2.008699in}}%
\pgfpathcurveto{\pgfqpoint{2.166034in}{2.014523in}}{\pgfqpoint{2.169306in}{2.022423in}}{\pgfqpoint{2.169306in}{2.030659in}}%
\pgfpathcurveto{\pgfqpoint{2.169306in}{2.038895in}}{\pgfqpoint{2.166034in}{2.046795in}}{\pgfqpoint{2.160210in}{2.052619in}}%
\pgfpathcurveto{\pgfqpoint{2.154386in}{2.058443in}}{\pgfqpoint{2.146486in}{2.061715in}}{\pgfqpoint{2.138249in}{2.061715in}}%
\pgfpathcurveto{\pgfqpoint{2.130013in}{2.061715in}}{\pgfqpoint{2.122113in}{2.058443in}}{\pgfqpoint{2.116289in}{2.052619in}}%
\pgfpathcurveto{\pgfqpoint{2.110465in}{2.046795in}}{\pgfqpoint{2.107193in}{2.038895in}}{\pgfqpoint{2.107193in}{2.030659in}}%
\pgfpathcurveto{\pgfqpoint{2.107193in}{2.022423in}}{\pgfqpoint{2.110465in}{2.014523in}}{\pgfqpoint{2.116289in}{2.008699in}}%
\pgfpathcurveto{\pgfqpoint{2.122113in}{2.002875in}}{\pgfqpoint{2.130013in}{1.999602in}}{\pgfqpoint{2.138249in}{1.999602in}}%
\pgfpathclose%
\pgfusepath{stroke,fill}%
\end{pgfscope}%
\begin{pgfscope}%
\pgfpathrectangle{\pgfqpoint{0.100000in}{0.212622in}}{\pgfqpoint{3.696000in}{3.696000in}}%
\pgfusepath{clip}%
\pgfsetbuttcap%
\pgfsetroundjoin%
\definecolor{currentfill}{rgb}{0.121569,0.466667,0.705882}%
\pgfsetfillcolor{currentfill}%
\pgfsetfillopacity{0.926725}%
\pgfsetlinewidth{1.003750pt}%
\definecolor{currentstroke}{rgb}{0.121569,0.466667,0.705882}%
\pgfsetstrokecolor{currentstroke}%
\pgfsetstrokeopacity{0.926725}%
\pgfsetdash{}{0pt}%
\pgfpathmoveto{\pgfqpoint{1.913093in}{1.912830in}}%
\pgfpathcurveto{\pgfqpoint{1.921329in}{1.912830in}}{\pgfqpoint{1.929229in}{1.916102in}}{\pgfqpoint{1.935053in}{1.921926in}}%
\pgfpathcurveto{\pgfqpoint{1.940877in}{1.927750in}}{\pgfqpoint{1.944149in}{1.935650in}}{\pgfqpoint{1.944149in}{1.943886in}}%
\pgfpathcurveto{\pgfqpoint{1.944149in}{1.952123in}}{\pgfqpoint{1.940877in}{1.960023in}}{\pgfqpoint{1.935053in}{1.965847in}}%
\pgfpathcurveto{\pgfqpoint{1.929229in}{1.971671in}}{\pgfqpoint{1.921329in}{1.974943in}}{\pgfqpoint{1.913093in}{1.974943in}}%
\pgfpathcurveto{\pgfqpoint{1.904856in}{1.974943in}}{\pgfqpoint{1.896956in}{1.971671in}}{\pgfqpoint{1.891133in}{1.965847in}}%
\pgfpathcurveto{\pgfqpoint{1.885309in}{1.960023in}}{\pgfqpoint{1.882036in}{1.952123in}}{\pgfqpoint{1.882036in}{1.943886in}}%
\pgfpathcurveto{\pgfqpoint{1.882036in}{1.935650in}}{\pgfqpoint{1.885309in}{1.927750in}}{\pgfqpoint{1.891133in}{1.921926in}}%
\pgfpathcurveto{\pgfqpoint{1.896956in}{1.916102in}}{\pgfqpoint{1.904856in}{1.912830in}}{\pgfqpoint{1.913093in}{1.912830in}}%
\pgfpathclose%
\pgfusepath{stroke,fill}%
\end{pgfscope}%
\begin{pgfscope}%
\pgfpathrectangle{\pgfqpoint{0.100000in}{0.212622in}}{\pgfqpoint{3.696000in}{3.696000in}}%
\pgfusepath{clip}%
\pgfsetbuttcap%
\pgfsetroundjoin%
\definecolor{currentfill}{rgb}{0.121569,0.466667,0.705882}%
\pgfsetfillcolor{currentfill}%
\pgfsetfillopacity{0.926726}%
\pgfsetlinewidth{1.003750pt}%
\definecolor{currentstroke}{rgb}{0.121569,0.466667,0.705882}%
\pgfsetstrokecolor{currentstroke}%
\pgfsetstrokeopacity{0.926726}%
\pgfsetdash{}{0pt}%
\pgfpathmoveto{\pgfqpoint{2.135741in}{1.995634in}}%
\pgfpathcurveto{\pgfqpoint{2.143977in}{1.995634in}}{\pgfqpoint{2.151877in}{1.998907in}}{\pgfqpoint{2.157701in}{2.004731in}}%
\pgfpathcurveto{\pgfqpoint{2.163525in}{2.010555in}}{\pgfqpoint{2.166797in}{2.018455in}}{\pgfqpoint{2.166797in}{2.026691in}}%
\pgfpathcurveto{\pgfqpoint{2.166797in}{2.034927in}}{\pgfqpoint{2.163525in}{2.042827in}}{\pgfqpoint{2.157701in}{2.048651in}}%
\pgfpathcurveto{\pgfqpoint{2.151877in}{2.054475in}}{\pgfqpoint{2.143977in}{2.057747in}}{\pgfqpoint{2.135741in}{2.057747in}}%
\pgfpathcurveto{\pgfqpoint{2.127504in}{2.057747in}}{\pgfqpoint{2.119604in}{2.054475in}}{\pgfqpoint{2.113781in}{2.048651in}}%
\pgfpathcurveto{\pgfqpoint{2.107957in}{2.042827in}}{\pgfqpoint{2.104684in}{2.034927in}}{\pgfqpoint{2.104684in}{2.026691in}}%
\pgfpathcurveto{\pgfqpoint{2.104684in}{2.018455in}}{\pgfqpoint{2.107957in}{2.010555in}}{\pgfqpoint{2.113781in}{2.004731in}}%
\pgfpathcurveto{\pgfqpoint{2.119604in}{1.998907in}}{\pgfqpoint{2.127504in}{1.995634in}}{\pgfqpoint{2.135741in}{1.995634in}}%
\pgfpathclose%
\pgfusepath{stroke,fill}%
\end{pgfscope}%
\begin{pgfscope}%
\pgfpathrectangle{\pgfqpoint{0.100000in}{0.212622in}}{\pgfqpoint{3.696000in}{3.696000in}}%
\pgfusepath{clip}%
\pgfsetbuttcap%
\pgfsetroundjoin%
\definecolor{currentfill}{rgb}{0.121569,0.466667,0.705882}%
\pgfsetfillcolor{currentfill}%
\pgfsetfillopacity{0.926938}%
\pgfsetlinewidth{1.003750pt}%
\definecolor{currentstroke}{rgb}{0.121569,0.466667,0.705882}%
\pgfsetstrokecolor{currentstroke}%
\pgfsetstrokeopacity{0.926938}%
\pgfsetdash{}{0pt}%
\pgfpathmoveto{\pgfqpoint{1.930927in}{1.040350in}}%
\pgfpathcurveto{\pgfqpoint{1.939163in}{1.040350in}}{\pgfqpoint{1.947063in}{1.043623in}}{\pgfqpoint{1.952887in}{1.049447in}}%
\pgfpathcurveto{\pgfqpoint{1.958711in}{1.055270in}}{\pgfqpoint{1.961984in}{1.063170in}}{\pgfqpoint{1.961984in}{1.071407in}}%
\pgfpathcurveto{\pgfqpoint{1.961984in}{1.079643in}}{\pgfqpoint{1.958711in}{1.087543in}}{\pgfqpoint{1.952887in}{1.093367in}}%
\pgfpathcurveto{\pgfqpoint{1.947063in}{1.099191in}}{\pgfqpoint{1.939163in}{1.102463in}}{\pgfqpoint{1.930927in}{1.102463in}}%
\pgfpathcurveto{\pgfqpoint{1.922691in}{1.102463in}}{\pgfqpoint{1.914791in}{1.099191in}}{\pgfqpoint{1.908967in}{1.093367in}}%
\pgfpathcurveto{\pgfqpoint{1.903143in}{1.087543in}}{\pgfqpoint{1.899871in}{1.079643in}}{\pgfqpoint{1.899871in}{1.071407in}}%
\pgfpathcurveto{\pgfqpoint{1.899871in}{1.063170in}}{\pgfqpoint{1.903143in}{1.055270in}}{\pgfqpoint{1.908967in}{1.049447in}}%
\pgfpathcurveto{\pgfqpoint{1.914791in}{1.043623in}}{\pgfqpoint{1.922691in}{1.040350in}}{\pgfqpoint{1.930927in}{1.040350in}}%
\pgfpathclose%
\pgfusepath{stroke,fill}%
\end{pgfscope}%
\begin{pgfscope}%
\pgfpathrectangle{\pgfqpoint{0.100000in}{0.212622in}}{\pgfqpoint{3.696000in}{3.696000in}}%
\pgfusepath{clip}%
\pgfsetbuttcap%
\pgfsetroundjoin%
\definecolor{currentfill}{rgb}{0.121569,0.466667,0.705882}%
\pgfsetfillcolor{currentfill}%
\pgfsetfillopacity{0.927045}%
\pgfsetlinewidth{1.003750pt}%
\definecolor{currentstroke}{rgb}{0.121569,0.466667,0.705882}%
\pgfsetstrokecolor{currentstroke}%
\pgfsetstrokeopacity{0.927045}%
\pgfsetdash{}{0pt}%
\pgfpathmoveto{\pgfqpoint{2.135304in}{1.993565in}}%
\pgfpathcurveto{\pgfqpoint{2.143540in}{1.993565in}}{\pgfqpoint{2.151440in}{1.996838in}}{\pgfqpoint{2.157264in}{2.002662in}}%
\pgfpathcurveto{\pgfqpoint{2.163088in}{2.008486in}}{\pgfqpoint{2.166360in}{2.016386in}}{\pgfqpoint{2.166360in}{2.024622in}}%
\pgfpathcurveto{\pgfqpoint{2.166360in}{2.032858in}}{\pgfqpoint{2.163088in}{2.040758in}}{\pgfqpoint{2.157264in}{2.046582in}}%
\pgfpathcurveto{\pgfqpoint{2.151440in}{2.052406in}}{\pgfqpoint{2.143540in}{2.055678in}}{\pgfqpoint{2.135304in}{2.055678in}}%
\pgfpathcurveto{\pgfqpoint{2.127068in}{2.055678in}}{\pgfqpoint{2.119167in}{2.052406in}}{\pgfqpoint{2.113344in}{2.046582in}}%
\pgfpathcurveto{\pgfqpoint{2.107520in}{2.040758in}}{\pgfqpoint{2.104247in}{2.032858in}}{\pgfqpoint{2.104247in}{2.024622in}}%
\pgfpathcurveto{\pgfqpoint{2.104247in}{2.016386in}}{\pgfqpoint{2.107520in}{2.008486in}}{\pgfqpoint{2.113344in}{2.002662in}}%
\pgfpathcurveto{\pgfqpoint{2.119167in}{1.996838in}}{\pgfqpoint{2.127068in}{1.993565in}}{\pgfqpoint{2.135304in}{1.993565in}}%
\pgfpathclose%
\pgfusepath{stroke,fill}%
\end{pgfscope}%
\begin{pgfscope}%
\pgfpathrectangle{\pgfqpoint{0.100000in}{0.212622in}}{\pgfqpoint{3.696000in}{3.696000in}}%
\pgfusepath{clip}%
\pgfsetbuttcap%
\pgfsetroundjoin%
\definecolor{currentfill}{rgb}{0.121569,0.466667,0.705882}%
\pgfsetfillcolor{currentfill}%
\pgfsetfillopacity{0.927646}%
\pgfsetlinewidth{1.003750pt}%
\definecolor{currentstroke}{rgb}{0.121569,0.466667,0.705882}%
\pgfsetstrokecolor{currentstroke}%
\pgfsetstrokeopacity{0.927646}%
\pgfsetdash{}{0pt}%
\pgfpathmoveto{\pgfqpoint{2.133988in}{1.990058in}}%
\pgfpathcurveto{\pgfqpoint{2.142224in}{1.990058in}}{\pgfqpoint{2.150124in}{1.993331in}}{\pgfqpoint{2.155948in}{1.999155in}}%
\pgfpathcurveto{\pgfqpoint{2.161772in}{2.004979in}}{\pgfqpoint{2.165044in}{2.012879in}}{\pgfqpoint{2.165044in}{2.021115in}}%
\pgfpathcurveto{\pgfqpoint{2.165044in}{2.029351in}}{\pgfqpoint{2.161772in}{2.037251in}}{\pgfqpoint{2.155948in}{2.043075in}}%
\pgfpathcurveto{\pgfqpoint{2.150124in}{2.048899in}}{\pgfqpoint{2.142224in}{2.052171in}}{\pgfqpoint{2.133988in}{2.052171in}}%
\pgfpathcurveto{\pgfqpoint{2.125751in}{2.052171in}}{\pgfqpoint{2.117851in}{2.048899in}}{\pgfqpoint{2.112027in}{2.043075in}}%
\pgfpathcurveto{\pgfqpoint{2.106204in}{2.037251in}}{\pgfqpoint{2.102931in}{2.029351in}}{\pgfqpoint{2.102931in}{2.021115in}}%
\pgfpathcurveto{\pgfqpoint{2.102931in}{2.012879in}}{\pgfqpoint{2.106204in}{2.004979in}}{\pgfqpoint{2.112027in}{1.999155in}}%
\pgfpathcurveto{\pgfqpoint{2.117851in}{1.993331in}}{\pgfqpoint{2.125751in}{1.990058in}}{\pgfqpoint{2.133988in}{1.990058in}}%
\pgfpathclose%
\pgfusepath{stroke,fill}%
\end{pgfscope}%
\begin{pgfscope}%
\pgfpathrectangle{\pgfqpoint{0.100000in}{0.212622in}}{\pgfqpoint{3.696000in}{3.696000in}}%
\pgfusepath{clip}%
\pgfsetbuttcap%
\pgfsetroundjoin%
\definecolor{currentfill}{rgb}{0.121569,0.466667,0.705882}%
\pgfsetfillcolor{currentfill}%
\pgfsetfillopacity{0.927760}%
\pgfsetlinewidth{1.003750pt}%
\definecolor{currentstroke}{rgb}{0.121569,0.466667,0.705882}%
\pgfsetstrokecolor{currentstroke}%
\pgfsetstrokeopacity{0.927760}%
\pgfsetdash{}{0pt}%
\pgfpathmoveto{\pgfqpoint{1.943288in}{1.032810in}}%
\pgfpathcurveto{\pgfqpoint{1.951524in}{1.032810in}}{\pgfqpoint{1.959424in}{1.036083in}}{\pgfqpoint{1.965248in}{1.041907in}}%
\pgfpathcurveto{\pgfqpoint{1.971072in}{1.047731in}}{\pgfqpoint{1.974344in}{1.055631in}}{\pgfqpoint{1.974344in}{1.063867in}}%
\pgfpathcurveto{\pgfqpoint{1.974344in}{1.072103in}}{\pgfqpoint{1.971072in}{1.080003in}}{\pgfqpoint{1.965248in}{1.085827in}}%
\pgfpathcurveto{\pgfqpoint{1.959424in}{1.091651in}}{\pgfqpoint{1.951524in}{1.094923in}}{\pgfqpoint{1.943288in}{1.094923in}}%
\pgfpathcurveto{\pgfqpoint{1.935052in}{1.094923in}}{\pgfqpoint{1.927152in}{1.091651in}}{\pgfqpoint{1.921328in}{1.085827in}}%
\pgfpathcurveto{\pgfqpoint{1.915504in}{1.080003in}}{\pgfqpoint{1.912231in}{1.072103in}}{\pgfqpoint{1.912231in}{1.063867in}}%
\pgfpathcurveto{\pgfqpoint{1.912231in}{1.055631in}}{\pgfqpoint{1.915504in}{1.047731in}}{\pgfqpoint{1.921328in}{1.041907in}}%
\pgfpathcurveto{\pgfqpoint{1.927152in}{1.036083in}}{\pgfqpoint{1.935052in}{1.032810in}}{\pgfqpoint{1.943288in}{1.032810in}}%
\pgfpathclose%
\pgfusepath{stroke,fill}%
\end{pgfscope}%
\begin{pgfscope}%
\pgfpathrectangle{\pgfqpoint{0.100000in}{0.212622in}}{\pgfqpoint{3.696000in}{3.696000in}}%
\pgfusepath{clip}%
\pgfsetbuttcap%
\pgfsetroundjoin%
\definecolor{currentfill}{rgb}{0.121569,0.466667,0.705882}%
\pgfsetfillcolor{currentfill}%
\pgfsetfillopacity{0.927820}%
\pgfsetlinewidth{1.003750pt}%
\definecolor{currentstroke}{rgb}{0.121569,0.466667,0.705882}%
\pgfsetstrokecolor{currentstroke}%
\pgfsetstrokeopacity{0.927820}%
\pgfsetdash{}{0pt}%
\pgfpathmoveto{\pgfqpoint{2.133115in}{1.988785in}}%
\pgfpathcurveto{\pgfqpoint{2.141351in}{1.988785in}}{\pgfqpoint{2.149251in}{1.992057in}}{\pgfqpoint{2.155075in}{1.997881in}}%
\pgfpathcurveto{\pgfqpoint{2.160899in}{2.003705in}}{\pgfqpoint{2.164171in}{2.011605in}}{\pgfqpoint{2.164171in}{2.019841in}}%
\pgfpathcurveto{\pgfqpoint{2.164171in}{2.028077in}}{\pgfqpoint{2.160899in}{2.035978in}}{\pgfqpoint{2.155075in}{2.041801in}}%
\pgfpathcurveto{\pgfqpoint{2.149251in}{2.047625in}}{\pgfqpoint{2.141351in}{2.050898in}}{\pgfqpoint{2.133115in}{2.050898in}}%
\pgfpathcurveto{\pgfqpoint{2.124879in}{2.050898in}}{\pgfqpoint{2.116979in}{2.047625in}}{\pgfqpoint{2.111155in}{2.041801in}}%
\pgfpathcurveto{\pgfqpoint{2.105331in}{2.035978in}}{\pgfqpoint{2.102058in}{2.028077in}}{\pgfqpoint{2.102058in}{2.019841in}}%
\pgfpathcurveto{\pgfqpoint{2.102058in}{2.011605in}}{\pgfqpoint{2.105331in}{2.003705in}}{\pgfqpoint{2.111155in}{1.997881in}}%
\pgfpathcurveto{\pgfqpoint{2.116979in}{1.992057in}}{\pgfqpoint{2.124879in}{1.988785in}}{\pgfqpoint{2.133115in}{1.988785in}}%
\pgfpathclose%
\pgfusepath{stroke,fill}%
\end{pgfscope}%
\begin{pgfscope}%
\pgfpathrectangle{\pgfqpoint{0.100000in}{0.212622in}}{\pgfqpoint{3.696000in}{3.696000in}}%
\pgfusepath{clip}%
\pgfsetbuttcap%
\pgfsetroundjoin%
\definecolor{currentfill}{rgb}{0.121569,0.466667,0.705882}%
\pgfsetfillcolor{currentfill}%
\pgfsetfillopacity{0.927966}%
\pgfsetlinewidth{1.003750pt}%
\definecolor{currentstroke}{rgb}{0.121569,0.466667,0.705882}%
\pgfsetstrokecolor{currentstroke}%
\pgfsetstrokeopacity{0.927966}%
\pgfsetdash{}{0pt}%
\pgfpathmoveto{\pgfqpoint{1.917755in}{1.910264in}}%
\pgfpathcurveto{\pgfqpoint{1.925991in}{1.910264in}}{\pgfqpoint{1.933891in}{1.913536in}}{\pgfqpoint{1.939715in}{1.919360in}}%
\pgfpathcurveto{\pgfqpoint{1.945539in}{1.925184in}}{\pgfqpoint{1.948811in}{1.933084in}}{\pgfqpoint{1.948811in}{1.941321in}}%
\pgfpathcurveto{\pgfqpoint{1.948811in}{1.949557in}}{\pgfqpoint{1.945539in}{1.957457in}}{\pgfqpoint{1.939715in}{1.963281in}}%
\pgfpathcurveto{\pgfqpoint{1.933891in}{1.969105in}}{\pgfqpoint{1.925991in}{1.972377in}}{\pgfqpoint{1.917755in}{1.972377in}}%
\pgfpathcurveto{\pgfqpoint{1.909519in}{1.972377in}}{\pgfqpoint{1.901619in}{1.969105in}}{\pgfqpoint{1.895795in}{1.963281in}}%
\pgfpathcurveto{\pgfqpoint{1.889971in}{1.957457in}}{\pgfqpoint{1.886698in}{1.949557in}}{\pgfqpoint{1.886698in}{1.941321in}}%
\pgfpathcurveto{\pgfqpoint{1.886698in}{1.933084in}}{\pgfqpoint{1.889971in}{1.925184in}}{\pgfqpoint{1.895795in}{1.919360in}}%
\pgfpathcurveto{\pgfqpoint{1.901619in}{1.913536in}}{\pgfqpoint{1.909519in}{1.910264in}}{\pgfqpoint{1.917755in}{1.910264in}}%
\pgfpathclose%
\pgfusepath{stroke,fill}%
\end{pgfscope}%
\begin{pgfscope}%
\pgfpathrectangle{\pgfqpoint{0.100000in}{0.212622in}}{\pgfqpoint{3.696000in}{3.696000in}}%
\pgfusepath{clip}%
\pgfsetbuttcap%
\pgfsetroundjoin%
\definecolor{currentfill}{rgb}{0.121569,0.466667,0.705882}%
\pgfsetfillcolor{currentfill}%
\pgfsetfillopacity{0.928270}%
\pgfsetlinewidth{1.003750pt}%
\definecolor{currentstroke}{rgb}{0.121569,0.466667,0.705882}%
\pgfsetstrokecolor{currentstroke}%
\pgfsetstrokeopacity{0.928270}%
\pgfsetdash{}{0pt}%
\pgfpathmoveto{\pgfqpoint{2.132081in}{1.986272in}}%
\pgfpathcurveto{\pgfqpoint{2.140317in}{1.986272in}}{\pgfqpoint{2.148217in}{1.989544in}}{\pgfqpoint{2.154041in}{1.995368in}}%
\pgfpathcurveto{\pgfqpoint{2.159865in}{2.001192in}}{\pgfqpoint{2.163137in}{2.009092in}}{\pgfqpoint{2.163137in}{2.017328in}}%
\pgfpathcurveto{\pgfqpoint{2.163137in}{2.025565in}}{\pgfqpoint{2.159865in}{2.033465in}}{\pgfqpoint{2.154041in}{2.039289in}}%
\pgfpathcurveto{\pgfqpoint{2.148217in}{2.045112in}}{\pgfqpoint{2.140317in}{2.048385in}}{\pgfqpoint{2.132081in}{2.048385in}}%
\pgfpathcurveto{\pgfqpoint{2.123844in}{2.048385in}}{\pgfqpoint{2.115944in}{2.045112in}}{\pgfqpoint{2.110120in}{2.039289in}}%
\pgfpathcurveto{\pgfqpoint{2.104296in}{2.033465in}}{\pgfqpoint{2.101024in}{2.025565in}}{\pgfqpoint{2.101024in}{2.017328in}}%
\pgfpathcurveto{\pgfqpoint{2.101024in}{2.009092in}}{\pgfqpoint{2.104296in}{2.001192in}}{\pgfqpoint{2.110120in}{1.995368in}}%
\pgfpathcurveto{\pgfqpoint{2.115944in}{1.989544in}}{\pgfqpoint{2.123844in}{1.986272in}}{\pgfqpoint{2.132081in}{1.986272in}}%
\pgfpathclose%
\pgfusepath{stroke,fill}%
\end{pgfscope}%
\begin{pgfscope}%
\pgfpathrectangle{\pgfqpoint{0.100000in}{0.212622in}}{\pgfqpoint{3.696000in}{3.696000in}}%
\pgfusepath{clip}%
\pgfsetbuttcap%
\pgfsetroundjoin%
\definecolor{currentfill}{rgb}{0.121569,0.466667,0.705882}%
\pgfsetfillcolor{currentfill}%
\pgfsetfillopacity{0.928286}%
\pgfsetlinewidth{1.003750pt}%
\definecolor{currentstroke}{rgb}{0.121569,0.466667,0.705882}%
\pgfsetstrokecolor{currentstroke}%
\pgfsetstrokeopacity{0.928286}%
\pgfsetdash{}{0pt}%
\pgfpathmoveto{\pgfqpoint{1.956467in}{1.023414in}}%
\pgfpathcurveto{\pgfqpoint{1.964704in}{1.023414in}}{\pgfqpoint{1.972604in}{1.026686in}}{\pgfqpoint{1.978428in}{1.032510in}}%
\pgfpathcurveto{\pgfqpoint{1.984252in}{1.038334in}}{\pgfqpoint{1.987524in}{1.046234in}}{\pgfqpoint{1.987524in}{1.054471in}}%
\pgfpathcurveto{\pgfqpoint{1.987524in}{1.062707in}}{\pgfqpoint{1.984252in}{1.070607in}}{\pgfqpoint{1.978428in}{1.076431in}}%
\pgfpathcurveto{\pgfqpoint{1.972604in}{1.082255in}}{\pgfqpoint{1.964704in}{1.085527in}}{\pgfqpoint{1.956467in}{1.085527in}}%
\pgfpathcurveto{\pgfqpoint{1.948231in}{1.085527in}}{\pgfqpoint{1.940331in}{1.082255in}}{\pgfqpoint{1.934507in}{1.076431in}}%
\pgfpathcurveto{\pgfqpoint{1.928683in}{1.070607in}}{\pgfqpoint{1.925411in}{1.062707in}}{\pgfqpoint{1.925411in}{1.054471in}}%
\pgfpathcurveto{\pgfqpoint{1.925411in}{1.046234in}}{\pgfqpoint{1.928683in}{1.038334in}}{\pgfqpoint{1.934507in}{1.032510in}}%
\pgfpathcurveto{\pgfqpoint{1.940331in}{1.026686in}}{\pgfqpoint{1.948231in}{1.023414in}}{\pgfqpoint{1.956467in}{1.023414in}}%
\pgfpathclose%
\pgfusepath{stroke,fill}%
\end{pgfscope}%
\begin{pgfscope}%
\pgfpathrectangle{\pgfqpoint{0.100000in}{0.212622in}}{\pgfqpoint{3.696000in}{3.696000in}}%
\pgfusepath{clip}%
\pgfsetbuttcap%
\pgfsetroundjoin%
\definecolor{currentfill}{rgb}{0.121569,0.466667,0.705882}%
\pgfsetfillcolor{currentfill}%
\pgfsetfillopacity{0.928370}%
\pgfsetlinewidth{1.003750pt}%
\definecolor{currentstroke}{rgb}{0.121569,0.466667,0.705882}%
\pgfsetstrokecolor{currentstroke}%
\pgfsetstrokeopacity{0.928370}%
\pgfsetdash{}{0pt}%
\pgfpathmoveto{\pgfqpoint{2.717221in}{1.197769in}}%
\pgfpathcurveto{\pgfqpoint{2.725457in}{1.197769in}}{\pgfqpoint{2.733357in}{1.201042in}}{\pgfqpoint{2.739181in}{1.206866in}}%
\pgfpathcurveto{\pgfqpoint{2.745005in}{1.212689in}}{\pgfqpoint{2.748277in}{1.220590in}}{\pgfqpoint{2.748277in}{1.228826in}}%
\pgfpathcurveto{\pgfqpoint{2.748277in}{1.237062in}}{\pgfqpoint{2.745005in}{1.244962in}}{\pgfqpoint{2.739181in}{1.250786in}}%
\pgfpathcurveto{\pgfqpoint{2.733357in}{1.256610in}}{\pgfqpoint{2.725457in}{1.259882in}}{\pgfqpoint{2.717221in}{1.259882in}}%
\pgfpathcurveto{\pgfqpoint{2.708984in}{1.259882in}}{\pgfqpoint{2.701084in}{1.256610in}}{\pgfqpoint{2.695260in}{1.250786in}}%
\pgfpathcurveto{\pgfqpoint{2.689437in}{1.244962in}}{\pgfqpoint{2.686164in}{1.237062in}}{\pgfqpoint{2.686164in}{1.228826in}}%
\pgfpathcurveto{\pgfqpoint{2.686164in}{1.220590in}}{\pgfqpoint{2.689437in}{1.212689in}}{\pgfqpoint{2.695260in}{1.206866in}}%
\pgfpathcurveto{\pgfqpoint{2.701084in}{1.201042in}}{\pgfqpoint{2.708984in}{1.197769in}}{\pgfqpoint{2.717221in}{1.197769in}}%
\pgfpathclose%
\pgfusepath{stroke,fill}%
\end{pgfscope}%
\begin{pgfscope}%
\pgfpathrectangle{\pgfqpoint{0.100000in}{0.212622in}}{\pgfqpoint{3.696000in}{3.696000in}}%
\pgfusepath{clip}%
\pgfsetbuttcap%
\pgfsetroundjoin%
\definecolor{currentfill}{rgb}{0.121569,0.466667,0.705882}%
\pgfsetfillcolor{currentfill}%
\pgfsetfillopacity{0.928839}%
\pgfsetlinewidth{1.003750pt}%
\definecolor{currentstroke}{rgb}{0.121569,0.466667,0.705882}%
\pgfsetstrokecolor{currentstroke}%
\pgfsetstrokeopacity{0.928839}%
\pgfsetdash{}{0pt}%
\pgfpathmoveto{\pgfqpoint{1.923489in}{1.907177in}}%
\pgfpathcurveto{\pgfqpoint{1.931725in}{1.907177in}}{\pgfqpoint{1.939626in}{1.910449in}}{\pgfqpoint{1.945449in}{1.916273in}}%
\pgfpathcurveto{\pgfqpoint{1.951273in}{1.922097in}}{\pgfqpoint{1.954546in}{1.929997in}}{\pgfqpoint{1.954546in}{1.938234in}}%
\pgfpathcurveto{\pgfqpoint{1.954546in}{1.946470in}}{\pgfqpoint{1.951273in}{1.954370in}}{\pgfqpoint{1.945449in}{1.960194in}}%
\pgfpathcurveto{\pgfqpoint{1.939626in}{1.966018in}}{\pgfqpoint{1.931725in}{1.969290in}}{\pgfqpoint{1.923489in}{1.969290in}}%
\pgfpathcurveto{\pgfqpoint{1.915253in}{1.969290in}}{\pgfqpoint{1.907353in}{1.966018in}}{\pgfqpoint{1.901529in}{1.960194in}}%
\pgfpathcurveto{\pgfqpoint{1.895705in}{1.954370in}}{\pgfqpoint{1.892433in}{1.946470in}}{\pgfqpoint{1.892433in}{1.938234in}}%
\pgfpathcurveto{\pgfqpoint{1.892433in}{1.929997in}}{\pgfqpoint{1.895705in}{1.922097in}}{\pgfqpoint{1.901529in}{1.916273in}}%
\pgfpathcurveto{\pgfqpoint{1.907353in}{1.910449in}}{\pgfqpoint{1.915253in}{1.907177in}}{\pgfqpoint{1.923489in}{1.907177in}}%
\pgfpathclose%
\pgfusepath{stroke,fill}%
\end{pgfscope}%
\begin{pgfscope}%
\pgfpathrectangle{\pgfqpoint{0.100000in}{0.212622in}}{\pgfqpoint{3.696000in}{3.696000in}}%
\pgfusepath{clip}%
\pgfsetbuttcap%
\pgfsetroundjoin%
\definecolor{currentfill}{rgb}{0.121569,0.466667,0.705882}%
\pgfsetfillcolor{currentfill}%
\pgfsetfillopacity{0.929065}%
\pgfsetlinewidth{1.003750pt}%
\definecolor{currentstroke}{rgb}{0.121569,0.466667,0.705882}%
\pgfsetstrokecolor{currentstroke}%
\pgfsetstrokeopacity{0.929065}%
\pgfsetdash{}{0pt}%
\pgfpathmoveto{\pgfqpoint{2.130440in}{1.981450in}}%
\pgfpathcurveto{\pgfqpoint{2.138676in}{1.981450in}}{\pgfqpoint{2.146576in}{1.984722in}}{\pgfqpoint{2.152400in}{1.990546in}}%
\pgfpathcurveto{\pgfqpoint{2.158224in}{1.996370in}}{\pgfqpoint{2.161497in}{2.004270in}}{\pgfqpoint{2.161497in}{2.012507in}}%
\pgfpathcurveto{\pgfqpoint{2.161497in}{2.020743in}}{\pgfqpoint{2.158224in}{2.028643in}}{\pgfqpoint{2.152400in}{2.034467in}}%
\pgfpathcurveto{\pgfqpoint{2.146576in}{2.040291in}}{\pgfqpoint{2.138676in}{2.043563in}}{\pgfqpoint{2.130440in}{2.043563in}}%
\pgfpathcurveto{\pgfqpoint{2.122204in}{2.043563in}}{\pgfqpoint{2.114304in}{2.040291in}}{\pgfqpoint{2.108480in}{2.034467in}}%
\pgfpathcurveto{\pgfqpoint{2.102656in}{2.028643in}}{\pgfqpoint{2.099384in}{2.020743in}}{\pgfqpoint{2.099384in}{2.012507in}}%
\pgfpathcurveto{\pgfqpoint{2.099384in}{2.004270in}}{\pgfqpoint{2.102656in}{1.996370in}}{\pgfqpoint{2.108480in}{1.990546in}}%
\pgfpathcurveto{\pgfqpoint{2.114304in}{1.984722in}}{\pgfqpoint{2.122204in}{1.981450in}}{\pgfqpoint{2.130440in}{1.981450in}}%
\pgfpathclose%
\pgfusepath{stroke,fill}%
\end{pgfscope}%
\begin{pgfscope}%
\pgfpathrectangle{\pgfqpoint{0.100000in}{0.212622in}}{\pgfqpoint{3.696000in}{3.696000in}}%
\pgfusepath{clip}%
\pgfsetbuttcap%
\pgfsetroundjoin%
\definecolor{currentfill}{rgb}{0.121569,0.466667,0.705882}%
\pgfsetfillcolor{currentfill}%
\pgfsetfillopacity{0.929379}%
\pgfsetlinewidth{1.003750pt}%
\definecolor{currentstroke}{rgb}{0.121569,0.466667,0.705882}%
\pgfsetstrokecolor{currentstroke}%
\pgfsetstrokeopacity{0.929379}%
\pgfsetdash{}{0pt}%
\pgfpathmoveto{\pgfqpoint{2.128999in}{1.979529in}}%
\pgfpathcurveto{\pgfqpoint{2.137235in}{1.979529in}}{\pgfqpoint{2.145136in}{1.982802in}}{\pgfqpoint{2.150959in}{1.988626in}}%
\pgfpathcurveto{\pgfqpoint{2.156783in}{1.994450in}}{\pgfqpoint{2.160056in}{2.002350in}}{\pgfqpoint{2.160056in}{2.010586in}}%
\pgfpathcurveto{\pgfqpoint{2.160056in}{2.018822in}}{\pgfqpoint{2.156783in}{2.026722in}}{\pgfqpoint{2.150959in}{2.032546in}}%
\pgfpathcurveto{\pgfqpoint{2.145136in}{2.038370in}}{\pgfqpoint{2.137235in}{2.041642in}}{\pgfqpoint{2.128999in}{2.041642in}}%
\pgfpathcurveto{\pgfqpoint{2.120763in}{2.041642in}}{\pgfqpoint{2.112863in}{2.038370in}}{\pgfqpoint{2.107039in}{2.032546in}}%
\pgfpathcurveto{\pgfqpoint{2.101215in}{2.026722in}}{\pgfqpoint{2.097943in}{2.018822in}}{\pgfqpoint{2.097943in}{2.010586in}}%
\pgfpathcurveto{\pgfqpoint{2.097943in}{2.002350in}}{\pgfqpoint{2.101215in}{1.994450in}}{\pgfqpoint{2.107039in}{1.988626in}}%
\pgfpathcurveto{\pgfqpoint{2.112863in}{1.982802in}}{\pgfqpoint{2.120763in}{1.979529in}}{\pgfqpoint{2.128999in}{1.979529in}}%
\pgfpathclose%
\pgfusepath{stroke,fill}%
\end{pgfscope}%
\begin{pgfscope}%
\pgfpathrectangle{\pgfqpoint{0.100000in}{0.212622in}}{\pgfqpoint{3.696000in}{3.696000in}}%
\pgfusepath{clip}%
\pgfsetbuttcap%
\pgfsetroundjoin%
\definecolor{currentfill}{rgb}{0.121569,0.466667,0.705882}%
\pgfsetfillcolor{currentfill}%
\pgfsetfillopacity{0.929767}%
\pgfsetlinewidth{1.003750pt}%
\definecolor{currentstroke}{rgb}{0.121569,0.466667,0.705882}%
\pgfsetstrokecolor{currentstroke}%
\pgfsetstrokeopacity{0.929767}%
\pgfsetdash{}{0pt}%
\pgfpathmoveto{\pgfqpoint{1.929610in}{1.902465in}}%
\pgfpathcurveto{\pgfqpoint{1.937847in}{1.902465in}}{\pgfqpoint{1.945747in}{1.905737in}}{\pgfqpoint{1.951571in}{1.911561in}}%
\pgfpathcurveto{\pgfqpoint{1.957395in}{1.917385in}}{\pgfqpoint{1.960667in}{1.925285in}}{\pgfqpoint{1.960667in}{1.933522in}}%
\pgfpathcurveto{\pgfqpoint{1.960667in}{1.941758in}}{\pgfqpoint{1.957395in}{1.949658in}}{\pgfqpoint{1.951571in}{1.955482in}}%
\pgfpathcurveto{\pgfqpoint{1.945747in}{1.961306in}}{\pgfqpoint{1.937847in}{1.964578in}}{\pgfqpoint{1.929610in}{1.964578in}}%
\pgfpathcurveto{\pgfqpoint{1.921374in}{1.964578in}}{\pgfqpoint{1.913474in}{1.961306in}}{\pgfqpoint{1.907650in}{1.955482in}}%
\pgfpathcurveto{\pgfqpoint{1.901826in}{1.949658in}}{\pgfqpoint{1.898554in}{1.941758in}}{\pgfqpoint{1.898554in}{1.933522in}}%
\pgfpathcurveto{\pgfqpoint{1.898554in}{1.925285in}}{\pgfqpoint{1.901826in}{1.917385in}}{\pgfqpoint{1.907650in}{1.911561in}}%
\pgfpathcurveto{\pgfqpoint{1.913474in}{1.905737in}}{\pgfqpoint{1.921374in}{1.902465in}}{\pgfqpoint{1.929610in}{1.902465in}}%
\pgfpathclose%
\pgfusepath{stroke,fill}%
\end{pgfscope}%
\begin{pgfscope}%
\pgfpathrectangle{\pgfqpoint{0.100000in}{0.212622in}}{\pgfqpoint{3.696000in}{3.696000in}}%
\pgfusepath{clip}%
\pgfsetbuttcap%
\pgfsetroundjoin%
\definecolor{currentfill}{rgb}{0.121569,0.466667,0.705882}%
\pgfsetfillcolor{currentfill}%
\pgfsetfillopacity{0.930052}%
\pgfsetlinewidth{1.003750pt}%
\definecolor{currentstroke}{rgb}{0.121569,0.466667,0.705882}%
\pgfsetstrokecolor{currentstroke}%
\pgfsetstrokeopacity{0.930052}%
\pgfsetdash{}{0pt}%
\pgfpathmoveto{\pgfqpoint{2.126918in}{1.975531in}}%
\pgfpathcurveto{\pgfqpoint{2.135154in}{1.975531in}}{\pgfqpoint{2.143054in}{1.978803in}}{\pgfqpoint{2.148878in}{1.984627in}}%
\pgfpathcurveto{\pgfqpoint{2.154702in}{1.990451in}}{\pgfqpoint{2.157975in}{1.998351in}}{\pgfqpoint{2.157975in}{2.006587in}}%
\pgfpathcurveto{\pgfqpoint{2.157975in}{2.014824in}}{\pgfqpoint{2.154702in}{2.022724in}}{\pgfqpoint{2.148878in}{2.028548in}}%
\pgfpathcurveto{\pgfqpoint{2.143054in}{2.034372in}}{\pgfqpoint{2.135154in}{2.037644in}}{\pgfqpoint{2.126918in}{2.037644in}}%
\pgfpathcurveto{\pgfqpoint{2.118682in}{2.037644in}}{\pgfqpoint{2.110782in}{2.034372in}}{\pgfqpoint{2.104958in}{2.028548in}}%
\pgfpathcurveto{\pgfqpoint{2.099134in}{2.022724in}}{\pgfqpoint{2.095862in}{2.014824in}}{\pgfqpoint{2.095862in}{2.006587in}}%
\pgfpathcurveto{\pgfqpoint{2.095862in}{1.998351in}}{\pgfqpoint{2.099134in}{1.990451in}}{\pgfqpoint{2.104958in}{1.984627in}}%
\pgfpathcurveto{\pgfqpoint{2.110782in}{1.978803in}}{\pgfqpoint{2.118682in}{1.975531in}}{\pgfqpoint{2.126918in}{1.975531in}}%
\pgfpathclose%
\pgfusepath{stroke,fill}%
\end{pgfscope}%
\begin{pgfscope}%
\pgfpathrectangle{\pgfqpoint{0.100000in}{0.212622in}}{\pgfqpoint{3.696000in}{3.696000in}}%
\pgfusepath{clip}%
\pgfsetbuttcap%
\pgfsetroundjoin%
\definecolor{currentfill}{rgb}{0.121569,0.466667,0.705882}%
\pgfsetfillcolor{currentfill}%
\pgfsetfillopacity{0.930278}%
\pgfsetlinewidth{1.003750pt}%
\definecolor{currentstroke}{rgb}{0.121569,0.466667,0.705882}%
\pgfsetstrokecolor{currentstroke}%
\pgfsetstrokeopacity{0.930278}%
\pgfsetdash{}{0pt}%
\pgfpathmoveto{\pgfqpoint{1.971095in}{1.022892in}}%
\pgfpathcurveto{\pgfqpoint{1.979331in}{1.022892in}}{\pgfqpoint{1.987231in}{1.026164in}}{\pgfqpoint{1.993055in}{1.031988in}}%
\pgfpathcurveto{\pgfqpoint{1.998879in}{1.037812in}}{\pgfqpoint{2.002151in}{1.045712in}}{\pgfqpoint{2.002151in}{1.053949in}}%
\pgfpathcurveto{\pgfqpoint{2.002151in}{1.062185in}}{\pgfqpoint{1.998879in}{1.070085in}}{\pgfqpoint{1.993055in}{1.075909in}}%
\pgfpathcurveto{\pgfqpoint{1.987231in}{1.081733in}}{\pgfqpoint{1.979331in}{1.085005in}}{\pgfqpoint{1.971095in}{1.085005in}}%
\pgfpathcurveto{\pgfqpoint{1.962858in}{1.085005in}}{\pgfqpoint{1.954958in}{1.081733in}}{\pgfqpoint{1.949135in}{1.075909in}}%
\pgfpathcurveto{\pgfqpoint{1.943311in}{1.070085in}}{\pgfqpoint{1.940038in}{1.062185in}}{\pgfqpoint{1.940038in}{1.053949in}}%
\pgfpathcurveto{\pgfqpoint{1.940038in}{1.045712in}}{\pgfqpoint{1.943311in}{1.037812in}}{\pgfqpoint{1.949135in}{1.031988in}}%
\pgfpathcurveto{\pgfqpoint{1.954958in}{1.026164in}}{\pgfqpoint{1.962858in}{1.022892in}}{\pgfqpoint{1.971095in}{1.022892in}}%
\pgfpathclose%
\pgfusepath{stroke,fill}%
\end{pgfscope}%
\begin{pgfscope}%
\pgfpathrectangle{\pgfqpoint{0.100000in}{0.212622in}}{\pgfqpoint{3.696000in}{3.696000in}}%
\pgfusepath{clip}%
\pgfsetbuttcap%
\pgfsetroundjoin%
\definecolor{currentfill}{rgb}{0.121569,0.466667,0.705882}%
\pgfsetfillcolor{currentfill}%
\pgfsetfillopacity{0.930534}%
\pgfsetlinewidth{1.003750pt}%
\definecolor{currentstroke}{rgb}{0.121569,0.466667,0.705882}%
\pgfsetstrokecolor{currentstroke}%
\pgfsetstrokeopacity{0.930534}%
\pgfsetdash{}{0pt}%
\pgfpathmoveto{\pgfqpoint{2.126059in}{1.972875in}}%
\pgfpathcurveto{\pgfqpoint{2.134296in}{1.972875in}}{\pgfqpoint{2.142196in}{1.976147in}}{\pgfqpoint{2.148020in}{1.981971in}}%
\pgfpathcurveto{\pgfqpoint{2.153844in}{1.987795in}}{\pgfqpoint{2.157116in}{1.995695in}}{\pgfqpoint{2.157116in}{2.003931in}}%
\pgfpathcurveto{\pgfqpoint{2.157116in}{2.012167in}}{\pgfqpoint{2.153844in}{2.020067in}}{\pgfqpoint{2.148020in}{2.025891in}}%
\pgfpathcurveto{\pgfqpoint{2.142196in}{2.031715in}}{\pgfqpoint{2.134296in}{2.034988in}}{\pgfqpoint{2.126059in}{2.034988in}}%
\pgfpathcurveto{\pgfqpoint{2.117823in}{2.034988in}}{\pgfqpoint{2.109923in}{2.031715in}}{\pgfqpoint{2.104099in}{2.025891in}}%
\pgfpathcurveto{\pgfqpoint{2.098275in}{2.020067in}}{\pgfqpoint{2.095003in}{2.012167in}}{\pgfqpoint{2.095003in}{2.003931in}}%
\pgfpathcurveto{\pgfqpoint{2.095003in}{1.995695in}}{\pgfqpoint{2.098275in}{1.987795in}}{\pgfqpoint{2.104099in}{1.981971in}}%
\pgfpathcurveto{\pgfqpoint{2.109923in}{1.976147in}}{\pgfqpoint{2.117823in}{1.972875in}}{\pgfqpoint{2.126059in}{1.972875in}}%
\pgfpathclose%
\pgfusepath{stroke,fill}%
\end{pgfscope}%
\begin{pgfscope}%
\pgfpathrectangle{\pgfqpoint{0.100000in}{0.212622in}}{\pgfqpoint{3.696000in}{3.696000in}}%
\pgfusepath{clip}%
\pgfsetbuttcap%
\pgfsetroundjoin%
\definecolor{currentfill}{rgb}{0.121569,0.466667,0.705882}%
\pgfsetfillcolor{currentfill}%
\pgfsetfillopacity{0.930901}%
\pgfsetlinewidth{1.003750pt}%
\definecolor{currentstroke}{rgb}{0.121569,0.466667,0.705882}%
\pgfsetstrokecolor{currentstroke}%
\pgfsetstrokeopacity{0.930901}%
\pgfsetdash{}{0pt}%
\pgfpathmoveto{\pgfqpoint{2.125049in}{1.970826in}}%
\pgfpathcurveto{\pgfqpoint{2.133285in}{1.970826in}}{\pgfqpoint{2.141185in}{1.974098in}}{\pgfqpoint{2.147009in}{1.979922in}}%
\pgfpathcurveto{\pgfqpoint{2.152833in}{1.985746in}}{\pgfqpoint{2.156105in}{1.993646in}}{\pgfqpoint{2.156105in}{2.001882in}}%
\pgfpathcurveto{\pgfqpoint{2.156105in}{2.010118in}}{\pgfqpoint{2.152833in}{2.018018in}}{\pgfqpoint{2.147009in}{2.023842in}}%
\pgfpathcurveto{\pgfqpoint{2.141185in}{2.029666in}}{\pgfqpoint{2.133285in}{2.032939in}}{\pgfqpoint{2.125049in}{2.032939in}}%
\pgfpathcurveto{\pgfqpoint{2.116812in}{2.032939in}}{\pgfqpoint{2.108912in}{2.029666in}}{\pgfqpoint{2.103088in}{2.023842in}}%
\pgfpathcurveto{\pgfqpoint{2.097264in}{2.018018in}}{\pgfqpoint{2.093992in}{2.010118in}}{\pgfqpoint{2.093992in}{2.001882in}}%
\pgfpathcurveto{\pgfqpoint{2.093992in}{1.993646in}}{\pgfqpoint{2.097264in}{1.985746in}}{\pgfqpoint{2.103088in}{1.979922in}}%
\pgfpathcurveto{\pgfqpoint{2.108912in}{1.974098in}}{\pgfqpoint{2.116812in}{1.970826in}}{\pgfqpoint{2.125049in}{1.970826in}}%
\pgfpathclose%
\pgfusepath{stroke,fill}%
\end{pgfscope}%
\begin{pgfscope}%
\pgfpathrectangle{\pgfqpoint{0.100000in}{0.212622in}}{\pgfqpoint{3.696000in}{3.696000in}}%
\pgfusepath{clip}%
\pgfsetbuttcap%
\pgfsetroundjoin%
\definecolor{currentfill}{rgb}{0.121569,0.466667,0.705882}%
\pgfsetfillcolor{currentfill}%
\pgfsetfillopacity{0.931256}%
\pgfsetlinewidth{1.003750pt}%
\definecolor{currentstroke}{rgb}{0.121569,0.466667,0.705882}%
\pgfsetstrokecolor{currentstroke}%
\pgfsetstrokeopacity{0.931256}%
\pgfsetdash{}{0pt}%
\pgfpathmoveto{\pgfqpoint{1.936867in}{1.898024in}}%
\pgfpathcurveto{\pgfqpoint{1.945103in}{1.898024in}}{\pgfqpoint{1.953003in}{1.901296in}}{\pgfqpoint{1.958827in}{1.907120in}}%
\pgfpathcurveto{\pgfqpoint{1.964651in}{1.912944in}}{\pgfqpoint{1.967923in}{1.920844in}}{\pgfqpoint{1.967923in}{1.929080in}}%
\pgfpathcurveto{\pgfqpoint{1.967923in}{1.937317in}}{\pgfqpoint{1.964651in}{1.945217in}}{\pgfqpoint{1.958827in}{1.951041in}}%
\pgfpathcurveto{\pgfqpoint{1.953003in}{1.956865in}}{\pgfqpoint{1.945103in}{1.960137in}}{\pgfqpoint{1.936867in}{1.960137in}}%
\pgfpathcurveto{\pgfqpoint{1.928631in}{1.960137in}}{\pgfqpoint{1.920731in}{1.956865in}}{\pgfqpoint{1.914907in}{1.951041in}}%
\pgfpathcurveto{\pgfqpoint{1.909083in}{1.945217in}}{\pgfqpoint{1.905810in}{1.937317in}}{\pgfqpoint{1.905810in}{1.929080in}}%
\pgfpathcurveto{\pgfqpoint{1.905810in}{1.920844in}}{\pgfqpoint{1.909083in}{1.912944in}}{\pgfqpoint{1.914907in}{1.907120in}}%
\pgfpathcurveto{\pgfqpoint{1.920731in}{1.901296in}}{\pgfqpoint{1.928631in}{1.898024in}}{\pgfqpoint{1.936867in}{1.898024in}}%
\pgfpathclose%
\pgfusepath{stroke,fill}%
\end{pgfscope}%
\begin{pgfscope}%
\pgfpathrectangle{\pgfqpoint{0.100000in}{0.212622in}}{\pgfqpoint{3.696000in}{3.696000in}}%
\pgfusepath{clip}%
\pgfsetbuttcap%
\pgfsetroundjoin%
\definecolor{currentfill}{rgb}{0.121569,0.466667,0.705882}%
\pgfsetfillcolor{currentfill}%
\pgfsetfillopacity{0.931351}%
\pgfsetlinewidth{1.003750pt}%
\definecolor{currentstroke}{rgb}{0.121569,0.466667,0.705882}%
\pgfsetstrokecolor{currentstroke}%
\pgfsetstrokeopacity{0.931351}%
\pgfsetdash{}{0pt}%
\pgfpathmoveto{\pgfqpoint{2.122636in}{1.967066in}}%
\pgfpathcurveto{\pgfqpoint{2.130872in}{1.967066in}}{\pgfqpoint{2.138772in}{1.970338in}}{\pgfqpoint{2.144596in}{1.976162in}}%
\pgfpathcurveto{\pgfqpoint{2.150420in}{1.981986in}}{\pgfqpoint{2.153692in}{1.989886in}}{\pgfqpoint{2.153692in}{1.998123in}}%
\pgfpathcurveto{\pgfqpoint{2.153692in}{2.006359in}}{\pgfqpoint{2.150420in}{2.014259in}}{\pgfqpoint{2.144596in}{2.020083in}}%
\pgfpathcurveto{\pgfqpoint{2.138772in}{2.025907in}}{\pgfqpoint{2.130872in}{2.029179in}}{\pgfqpoint{2.122636in}{2.029179in}}%
\pgfpathcurveto{\pgfqpoint{2.114400in}{2.029179in}}{\pgfqpoint{2.106499in}{2.025907in}}{\pgfqpoint{2.100676in}{2.020083in}}%
\pgfpathcurveto{\pgfqpoint{2.094852in}{2.014259in}}{\pgfqpoint{2.091579in}{2.006359in}}{\pgfqpoint{2.091579in}{1.998123in}}%
\pgfpathcurveto{\pgfqpoint{2.091579in}{1.989886in}}{\pgfqpoint{2.094852in}{1.981986in}}{\pgfqpoint{2.100676in}{1.976162in}}%
\pgfpathcurveto{\pgfqpoint{2.106499in}{1.970338in}}{\pgfqpoint{2.114400in}{1.967066in}}{\pgfqpoint{2.122636in}{1.967066in}}%
\pgfpathclose%
\pgfusepath{stroke,fill}%
\end{pgfscope}%
\begin{pgfscope}%
\pgfpathrectangle{\pgfqpoint{0.100000in}{0.212622in}}{\pgfqpoint{3.696000in}{3.696000in}}%
\pgfusepath{clip}%
\pgfsetbuttcap%
\pgfsetroundjoin%
\definecolor{currentfill}{rgb}{0.121569,0.466667,0.705882}%
\pgfsetfillcolor{currentfill}%
\pgfsetfillopacity{0.931797}%
\pgfsetlinewidth{1.003750pt}%
\definecolor{currentstroke}{rgb}{0.121569,0.466667,0.705882}%
\pgfsetstrokecolor{currentstroke}%
\pgfsetstrokeopacity{0.931797}%
\pgfsetdash{}{0pt}%
\pgfpathmoveto{\pgfqpoint{2.121851in}{1.964934in}}%
\pgfpathcurveto{\pgfqpoint{2.130088in}{1.964934in}}{\pgfqpoint{2.137988in}{1.968206in}}{\pgfqpoint{2.143812in}{1.974030in}}%
\pgfpathcurveto{\pgfqpoint{2.149635in}{1.979854in}}{\pgfqpoint{2.152908in}{1.987754in}}{\pgfqpoint{2.152908in}{1.995990in}}%
\pgfpathcurveto{\pgfqpoint{2.152908in}{2.004227in}}{\pgfqpoint{2.149635in}{2.012127in}}{\pgfqpoint{2.143812in}{2.017951in}}%
\pgfpathcurveto{\pgfqpoint{2.137988in}{2.023774in}}{\pgfqpoint{2.130088in}{2.027047in}}{\pgfqpoint{2.121851in}{2.027047in}}%
\pgfpathcurveto{\pgfqpoint{2.113615in}{2.027047in}}{\pgfqpoint{2.105715in}{2.023774in}}{\pgfqpoint{2.099891in}{2.017951in}}%
\pgfpathcurveto{\pgfqpoint{2.094067in}{2.012127in}}{\pgfqpoint{2.090795in}{2.004227in}}{\pgfqpoint{2.090795in}{1.995990in}}%
\pgfpathcurveto{\pgfqpoint{2.090795in}{1.987754in}}{\pgfqpoint{2.094067in}{1.979854in}}{\pgfqpoint{2.099891in}{1.974030in}}%
\pgfpathcurveto{\pgfqpoint{2.105715in}{1.968206in}}{\pgfqpoint{2.113615in}{1.964934in}}{\pgfqpoint{2.121851in}{1.964934in}}%
\pgfpathclose%
\pgfusepath{stroke,fill}%
\end{pgfscope}%
\begin{pgfscope}%
\pgfpathrectangle{\pgfqpoint{0.100000in}{0.212622in}}{\pgfqpoint{3.696000in}{3.696000in}}%
\pgfusepath{clip}%
\pgfsetbuttcap%
\pgfsetroundjoin%
\definecolor{currentfill}{rgb}{0.121569,0.466667,0.705882}%
\pgfsetfillcolor{currentfill}%
\pgfsetfillopacity{0.931809}%
\pgfsetlinewidth{1.003750pt}%
\definecolor{currentstroke}{rgb}{0.121569,0.466667,0.705882}%
\pgfsetstrokecolor{currentstroke}%
\pgfsetstrokeopacity{0.931809}%
\pgfsetdash{}{0pt}%
\pgfpathmoveto{\pgfqpoint{1.978879in}{1.023299in}}%
\pgfpathcurveto{\pgfqpoint{1.987115in}{1.023299in}}{\pgfqpoint{1.995015in}{1.026572in}}{\pgfqpoint{2.000839in}{1.032396in}}%
\pgfpathcurveto{\pgfqpoint{2.006663in}{1.038220in}}{\pgfqpoint{2.009935in}{1.046120in}}{\pgfqpoint{2.009935in}{1.054356in}}%
\pgfpathcurveto{\pgfqpoint{2.009935in}{1.062592in}}{\pgfqpoint{2.006663in}{1.070492in}}{\pgfqpoint{2.000839in}{1.076316in}}%
\pgfpathcurveto{\pgfqpoint{1.995015in}{1.082140in}}{\pgfqpoint{1.987115in}{1.085412in}}{\pgfqpoint{1.978879in}{1.085412in}}%
\pgfpathcurveto{\pgfqpoint{1.970642in}{1.085412in}}{\pgfqpoint{1.962742in}{1.082140in}}{\pgfqpoint{1.956918in}{1.076316in}}%
\pgfpathcurveto{\pgfqpoint{1.951094in}{1.070492in}}{\pgfqpoint{1.947822in}{1.062592in}}{\pgfqpoint{1.947822in}{1.054356in}}%
\pgfpathcurveto{\pgfqpoint{1.947822in}{1.046120in}}{\pgfqpoint{1.951094in}{1.038220in}}{\pgfqpoint{1.956918in}{1.032396in}}%
\pgfpathcurveto{\pgfqpoint{1.962742in}{1.026572in}}{\pgfqpoint{1.970642in}{1.023299in}}{\pgfqpoint{1.978879in}{1.023299in}}%
\pgfpathclose%
\pgfusepath{stroke,fill}%
\end{pgfscope}%
\begin{pgfscope}%
\pgfpathrectangle{\pgfqpoint{0.100000in}{0.212622in}}{\pgfqpoint{3.696000in}{3.696000in}}%
\pgfusepath{clip}%
\pgfsetbuttcap%
\pgfsetroundjoin%
\definecolor{currentfill}{rgb}{0.121569,0.466667,0.705882}%
\pgfsetfillcolor{currentfill}%
\pgfsetfillopacity{0.931855}%
\pgfsetlinewidth{1.003750pt}%
\definecolor{currentstroke}{rgb}{0.121569,0.466667,0.705882}%
\pgfsetstrokecolor{currentstroke}%
\pgfsetstrokeopacity{0.931855}%
\pgfsetdash{}{0pt}%
\pgfpathmoveto{\pgfqpoint{2.714357in}{1.178296in}}%
\pgfpathcurveto{\pgfqpoint{2.722593in}{1.178296in}}{\pgfqpoint{2.730493in}{1.181569in}}{\pgfqpoint{2.736317in}{1.187393in}}%
\pgfpathcurveto{\pgfqpoint{2.742141in}{1.193216in}}{\pgfqpoint{2.745413in}{1.201117in}}{\pgfqpoint{2.745413in}{1.209353in}}%
\pgfpathcurveto{\pgfqpoint{2.745413in}{1.217589in}}{\pgfqpoint{2.742141in}{1.225489in}}{\pgfqpoint{2.736317in}{1.231313in}}%
\pgfpathcurveto{\pgfqpoint{2.730493in}{1.237137in}}{\pgfqpoint{2.722593in}{1.240409in}}{\pgfqpoint{2.714357in}{1.240409in}}%
\pgfpathcurveto{\pgfqpoint{2.706120in}{1.240409in}}{\pgfqpoint{2.698220in}{1.237137in}}{\pgfqpoint{2.692396in}{1.231313in}}%
\pgfpathcurveto{\pgfqpoint{2.686572in}{1.225489in}}{\pgfqpoint{2.683300in}{1.217589in}}{\pgfqpoint{2.683300in}{1.209353in}}%
\pgfpathcurveto{\pgfqpoint{2.683300in}{1.201117in}}{\pgfqpoint{2.686572in}{1.193216in}}{\pgfqpoint{2.692396in}{1.187393in}}%
\pgfpathcurveto{\pgfqpoint{2.698220in}{1.181569in}}{\pgfqpoint{2.706120in}{1.178296in}}{\pgfqpoint{2.714357in}{1.178296in}}%
\pgfpathclose%
\pgfusepath{stroke,fill}%
\end{pgfscope}%
\begin{pgfscope}%
\pgfpathrectangle{\pgfqpoint{0.100000in}{0.212622in}}{\pgfqpoint{3.696000in}{3.696000in}}%
\pgfusepath{clip}%
\pgfsetbuttcap%
\pgfsetroundjoin%
\definecolor{currentfill}{rgb}{0.121569,0.466667,0.705882}%
\pgfsetfillcolor{currentfill}%
\pgfsetfillopacity{0.932591}%
\pgfsetlinewidth{1.003750pt}%
\definecolor{currentstroke}{rgb}{0.121569,0.466667,0.705882}%
\pgfsetstrokecolor{currentstroke}%
\pgfsetstrokeopacity{0.932591}%
\pgfsetdash{}{0pt}%
\pgfpathmoveto{\pgfqpoint{2.120144in}{1.961156in}}%
\pgfpathcurveto{\pgfqpoint{2.128380in}{1.961156in}}{\pgfqpoint{2.136280in}{1.964429in}}{\pgfqpoint{2.142104in}{1.970253in}}%
\pgfpathcurveto{\pgfqpoint{2.147928in}{1.976077in}}{\pgfqpoint{2.151201in}{1.983977in}}{\pgfqpoint{2.151201in}{1.992213in}}%
\pgfpathcurveto{\pgfqpoint{2.151201in}{2.000449in}}{\pgfqpoint{2.147928in}{2.008349in}}{\pgfqpoint{2.142104in}{2.014173in}}%
\pgfpathcurveto{\pgfqpoint{2.136280in}{2.019997in}}{\pgfqpoint{2.128380in}{2.023269in}}{\pgfqpoint{2.120144in}{2.023269in}}%
\pgfpathcurveto{\pgfqpoint{2.111908in}{2.023269in}}{\pgfqpoint{2.104008in}{2.019997in}}{\pgfqpoint{2.098184in}{2.014173in}}%
\pgfpathcurveto{\pgfqpoint{2.092360in}{2.008349in}}{\pgfqpoint{2.089088in}{2.000449in}}{\pgfqpoint{2.089088in}{1.992213in}}%
\pgfpathcurveto{\pgfqpoint{2.089088in}{1.983977in}}{\pgfqpoint{2.092360in}{1.976077in}}{\pgfqpoint{2.098184in}{1.970253in}}%
\pgfpathcurveto{\pgfqpoint{2.104008in}{1.964429in}}{\pgfqpoint{2.111908in}{1.961156in}}{\pgfqpoint{2.120144in}{1.961156in}}%
\pgfpathclose%
\pgfusepath{stroke,fill}%
\end{pgfscope}%
\begin{pgfscope}%
\pgfpathrectangle{\pgfqpoint{0.100000in}{0.212622in}}{\pgfqpoint{3.696000in}{3.696000in}}%
\pgfusepath{clip}%
\pgfsetbuttcap%
\pgfsetroundjoin%
\definecolor{currentfill}{rgb}{0.121569,0.466667,0.705882}%
\pgfsetfillcolor{currentfill}%
\pgfsetfillopacity{0.932791}%
\pgfsetlinewidth{1.003750pt}%
\definecolor{currentstroke}{rgb}{0.121569,0.466667,0.705882}%
\pgfsetstrokecolor{currentstroke}%
\pgfsetstrokeopacity{0.932791}%
\pgfsetdash{}{0pt}%
\pgfpathmoveto{\pgfqpoint{2.119237in}{1.959824in}}%
\pgfpathcurveto{\pgfqpoint{2.127473in}{1.959824in}}{\pgfqpoint{2.135373in}{1.963096in}}{\pgfqpoint{2.141197in}{1.968920in}}%
\pgfpathcurveto{\pgfqpoint{2.147021in}{1.974744in}}{\pgfqpoint{2.150294in}{1.982644in}}{\pgfqpoint{2.150294in}{1.990880in}}%
\pgfpathcurveto{\pgfqpoint{2.150294in}{1.999116in}}{\pgfqpoint{2.147021in}{2.007016in}}{\pgfqpoint{2.141197in}{2.012840in}}%
\pgfpathcurveto{\pgfqpoint{2.135373in}{2.018664in}}{\pgfqpoint{2.127473in}{2.021937in}}{\pgfqpoint{2.119237in}{2.021937in}}%
\pgfpathcurveto{\pgfqpoint{2.111001in}{2.021937in}}{\pgfqpoint{2.103101in}{2.018664in}}{\pgfqpoint{2.097277in}{2.012840in}}%
\pgfpathcurveto{\pgfqpoint{2.091453in}{2.007016in}}{\pgfqpoint{2.088181in}{1.999116in}}{\pgfqpoint{2.088181in}{1.990880in}}%
\pgfpathcurveto{\pgfqpoint{2.088181in}{1.982644in}}{\pgfqpoint{2.091453in}{1.974744in}}{\pgfqpoint{2.097277in}{1.968920in}}%
\pgfpathcurveto{\pgfqpoint{2.103101in}{1.963096in}}{\pgfqpoint{2.111001in}{1.959824in}}{\pgfqpoint{2.119237in}{1.959824in}}%
\pgfpathclose%
\pgfusepath{stroke,fill}%
\end{pgfscope}%
\begin{pgfscope}%
\pgfpathrectangle{\pgfqpoint{0.100000in}{0.212622in}}{\pgfqpoint{3.696000in}{3.696000in}}%
\pgfusepath{clip}%
\pgfsetbuttcap%
\pgfsetroundjoin%
\definecolor{currentfill}{rgb}{0.121569,0.466667,0.705882}%
\pgfsetfillcolor{currentfill}%
\pgfsetfillopacity{0.932849}%
\pgfsetlinewidth{1.003750pt}%
\definecolor{currentstroke}{rgb}{0.121569,0.466667,0.705882}%
\pgfsetstrokecolor{currentstroke}%
\pgfsetstrokeopacity{0.932849}%
\pgfsetdash{}{0pt}%
\pgfpathmoveto{\pgfqpoint{1.945382in}{1.894777in}}%
\pgfpathcurveto{\pgfqpoint{1.953619in}{1.894777in}}{\pgfqpoint{1.961519in}{1.898049in}}{\pgfqpoint{1.967343in}{1.903873in}}%
\pgfpathcurveto{\pgfqpoint{1.973167in}{1.909697in}}{\pgfqpoint{1.976439in}{1.917597in}}{\pgfqpoint{1.976439in}{1.925834in}}%
\pgfpathcurveto{\pgfqpoint{1.976439in}{1.934070in}}{\pgfqpoint{1.973167in}{1.941970in}}{\pgfqpoint{1.967343in}{1.947794in}}%
\pgfpathcurveto{\pgfqpoint{1.961519in}{1.953618in}}{\pgfqpoint{1.953619in}{1.956890in}}{\pgfqpoint{1.945382in}{1.956890in}}%
\pgfpathcurveto{\pgfqpoint{1.937146in}{1.956890in}}{\pgfqpoint{1.929246in}{1.953618in}}{\pgfqpoint{1.923422in}{1.947794in}}%
\pgfpathcurveto{\pgfqpoint{1.917598in}{1.941970in}}{\pgfqpoint{1.914326in}{1.934070in}}{\pgfqpoint{1.914326in}{1.925834in}}%
\pgfpathcurveto{\pgfqpoint{1.914326in}{1.917597in}}{\pgfqpoint{1.917598in}{1.909697in}}{\pgfqpoint{1.923422in}{1.903873in}}%
\pgfpathcurveto{\pgfqpoint{1.929246in}{1.898049in}}{\pgfqpoint{1.937146in}{1.894777in}}{\pgfqpoint{1.945382in}{1.894777in}}%
\pgfpathclose%
\pgfusepath{stroke,fill}%
\end{pgfscope}%
\begin{pgfscope}%
\pgfpathrectangle{\pgfqpoint{0.100000in}{0.212622in}}{\pgfqpoint{3.696000in}{3.696000in}}%
\pgfusepath{clip}%
\pgfsetbuttcap%
\pgfsetroundjoin%
\definecolor{currentfill}{rgb}{0.121569,0.466667,0.705882}%
\pgfsetfillcolor{currentfill}%
\pgfsetfillopacity{0.932966}%
\pgfsetlinewidth{1.003750pt}%
\definecolor{currentstroke}{rgb}{0.121569,0.466667,0.705882}%
\pgfsetstrokecolor{currentstroke}%
\pgfsetstrokeopacity{0.932966}%
\pgfsetdash{}{0pt}%
\pgfpathmoveto{\pgfqpoint{2.118812in}{1.959104in}}%
\pgfpathcurveto{\pgfqpoint{2.127049in}{1.959104in}}{\pgfqpoint{2.134949in}{1.962377in}}{\pgfqpoint{2.140773in}{1.968200in}}%
\pgfpathcurveto{\pgfqpoint{2.146597in}{1.974024in}}{\pgfqpoint{2.149869in}{1.981924in}}{\pgfqpoint{2.149869in}{1.990161in}}%
\pgfpathcurveto{\pgfqpoint{2.149869in}{1.998397in}}{\pgfqpoint{2.146597in}{2.006297in}}{\pgfqpoint{2.140773in}{2.012121in}}%
\pgfpathcurveto{\pgfqpoint{2.134949in}{2.017945in}}{\pgfqpoint{2.127049in}{2.021217in}}{\pgfqpoint{2.118812in}{2.021217in}}%
\pgfpathcurveto{\pgfqpoint{2.110576in}{2.021217in}}{\pgfqpoint{2.102676in}{2.017945in}}{\pgfqpoint{2.096852in}{2.012121in}}%
\pgfpathcurveto{\pgfqpoint{2.091028in}{2.006297in}}{\pgfqpoint{2.087756in}{1.998397in}}{\pgfqpoint{2.087756in}{1.990161in}}%
\pgfpathcurveto{\pgfqpoint{2.087756in}{1.981924in}}{\pgfqpoint{2.091028in}{1.974024in}}{\pgfqpoint{2.096852in}{1.968200in}}%
\pgfpathcurveto{\pgfqpoint{2.102676in}{1.962377in}}{\pgfqpoint{2.110576in}{1.959104in}}{\pgfqpoint{2.118812in}{1.959104in}}%
\pgfpathclose%
\pgfusepath{stroke,fill}%
\end{pgfscope}%
\begin{pgfscope}%
\pgfpathrectangle{\pgfqpoint{0.100000in}{0.212622in}}{\pgfqpoint{3.696000in}{3.696000in}}%
\pgfusepath{clip}%
\pgfsetbuttcap%
\pgfsetroundjoin%
\definecolor{currentfill}{rgb}{0.121569,0.466667,0.705882}%
\pgfsetfillcolor{currentfill}%
\pgfsetfillopacity{0.933004}%
\pgfsetlinewidth{1.003750pt}%
\definecolor{currentstroke}{rgb}{0.121569,0.466667,0.705882}%
\pgfsetstrokecolor{currentstroke}%
\pgfsetstrokeopacity{0.933004}%
\pgfsetdash{}{0pt}%
\pgfpathmoveto{\pgfqpoint{2.118727in}{1.958929in}}%
\pgfpathcurveto{\pgfqpoint{2.126964in}{1.958929in}}{\pgfqpoint{2.134864in}{1.962202in}}{\pgfqpoint{2.140688in}{1.968026in}}%
\pgfpathcurveto{\pgfqpoint{2.146512in}{1.973850in}}{\pgfqpoint{2.149784in}{1.981750in}}{\pgfqpoint{2.149784in}{1.989986in}}%
\pgfpathcurveto{\pgfqpoint{2.149784in}{1.998222in}}{\pgfqpoint{2.146512in}{2.006122in}}{\pgfqpoint{2.140688in}{2.011946in}}%
\pgfpathcurveto{\pgfqpoint{2.134864in}{2.017770in}}{\pgfqpoint{2.126964in}{2.021042in}}{\pgfqpoint{2.118727in}{2.021042in}}%
\pgfpathcurveto{\pgfqpoint{2.110491in}{2.021042in}}{\pgfqpoint{2.102591in}{2.017770in}}{\pgfqpoint{2.096767in}{2.011946in}}%
\pgfpathcurveto{\pgfqpoint{2.090943in}{2.006122in}}{\pgfqpoint{2.087671in}{1.998222in}}{\pgfqpoint{2.087671in}{1.989986in}}%
\pgfpathcurveto{\pgfqpoint{2.087671in}{1.981750in}}{\pgfqpoint{2.090943in}{1.973850in}}{\pgfqpoint{2.096767in}{1.968026in}}%
\pgfpathcurveto{\pgfqpoint{2.102591in}{1.962202in}}{\pgfqpoint{2.110491in}{1.958929in}}{\pgfqpoint{2.118727in}{1.958929in}}%
\pgfpathclose%
\pgfusepath{stroke,fill}%
\end{pgfscope}%
\begin{pgfscope}%
\pgfpathrectangle{\pgfqpoint{0.100000in}{0.212622in}}{\pgfqpoint{3.696000in}{3.696000in}}%
\pgfusepath{clip}%
\pgfsetbuttcap%
\pgfsetroundjoin%
\definecolor{currentfill}{rgb}{0.121569,0.466667,0.705882}%
\pgfsetfillcolor{currentfill}%
\pgfsetfillopacity{0.933058}%
\pgfsetlinewidth{1.003750pt}%
\definecolor{currentstroke}{rgb}{0.121569,0.466667,0.705882}%
\pgfsetstrokecolor{currentstroke}%
\pgfsetstrokeopacity{0.933058}%
\pgfsetdash{}{0pt}%
\pgfpathmoveto{\pgfqpoint{2.118522in}{1.958608in}}%
\pgfpathcurveto{\pgfqpoint{2.126758in}{1.958608in}}{\pgfqpoint{2.134658in}{1.961880in}}{\pgfqpoint{2.140482in}{1.967704in}}%
\pgfpathcurveto{\pgfqpoint{2.146306in}{1.973528in}}{\pgfqpoint{2.149579in}{1.981428in}}{\pgfqpoint{2.149579in}{1.989664in}}%
\pgfpathcurveto{\pgfqpoint{2.149579in}{1.997901in}}{\pgfqpoint{2.146306in}{2.005801in}}{\pgfqpoint{2.140482in}{2.011625in}}%
\pgfpathcurveto{\pgfqpoint{2.134658in}{2.017449in}}{\pgfqpoint{2.126758in}{2.020721in}}{\pgfqpoint{2.118522in}{2.020721in}}%
\pgfpathcurveto{\pgfqpoint{2.110286in}{2.020721in}}{\pgfqpoint{2.102386in}{2.017449in}}{\pgfqpoint{2.096562in}{2.011625in}}%
\pgfpathcurveto{\pgfqpoint{2.090738in}{2.005801in}}{\pgfqpoint{2.087466in}{1.997901in}}{\pgfqpoint{2.087466in}{1.989664in}}%
\pgfpathcurveto{\pgfqpoint{2.087466in}{1.981428in}}{\pgfqpoint{2.090738in}{1.973528in}}{\pgfqpoint{2.096562in}{1.967704in}}%
\pgfpathcurveto{\pgfqpoint{2.102386in}{1.961880in}}{\pgfqpoint{2.110286in}{1.958608in}}{\pgfqpoint{2.118522in}{1.958608in}}%
\pgfpathclose%
\pgfusepath{stroke,fill}%
\end{pgfscope}%
\begin{pgfscope}%
\pgfpathrectangle{\pgfqpoint{0.100000in}{0.212622in}}{\pgfqpoint{3.696000in}{3.696000in}}%
\pgfusepath{clip}%
\pgfsetbuttcap%
\pgfsetroundjoin%
\definecolor{currentfill}{rgb}{0.121569,0.466667,0.705882}%
\pgfsetfillcolor{currentfill}%
\pgfsetfillopacity{0.933190}%
\pgfsetlinewidth{1.003750pt}%
\definecolor{currentstroke}{rgb}{0.121569,0.466667,0.705882}%
\pgfsetstrokecolor{currentstroke}%
\pgfsetstrokeopacity{0.933190}%
\pgfsetdash{}{0pt}%
\pgfpathmoveto{\pgfqpoint{2.118232in}{1.958063in}}%
\pgfpathcurveto{\pgfqpoint{2.126469in}{1.958063in}}{\pgfqpoint{2.134369in}{1.961336in}}{\pgfqpoint{2.140193in}{1.967159in}}%
\pgfpathcurveto{\pgfqpoint{2.146017in}{1.972983in}}{\pgfqpoint{2.149289in}{1.980883in}}{\pgfqpoint{2.149289in}{1.989120in}}%
\pgfpathcurveto{\pgfqpoint{2.149289in}{1.997356in}}{\pgfqpoint{2.146017in}{2.005256in}}{\pgfqpoint{2.140193in}{2.011080in}}%
\pgfpathcurveto{\pgfqpoint{2.134369in}{2.016904in}}{\pgfqpoint{2.126469in}{2.020176in}}{\pgfqpoint{2.118232in}{2.020176in}}%
\pgfpathcurveto{\pgfqpoint{2.109996in}{2.020176in}}{\pgfqpoint{2.102096in}{2.016904in}}{\pgfqpoint{2.096272in}{2.011080in}}%
\pgfpathcurveto{\pgfqpoint{2.090448in}{2.005256in}}{\pgfqpoint{2.087176in}{1.997356in}}{\pgfqpoint{2.087176in}{1.989120in}}%
\pgfpathcurveto{\pgfqpoint{2.087176in}{1.980883in}}{\pgfqpoint{2.090448in}{1.972983in}}{\pgfqpoint{2.096272in}{1.967159in}}%
\pgfpathcurveto{\pgfqpoint{2.102096in}{1.961336in}}{\pgfqpoint{2.109996in}{1.958063in}}{\pgfqpoint{2.118232in}{1.958063in}}%
\pgfpathclose%
\pgfusepath{stroke,fill}%
\end{pgfscope}%
\begin{pgfscope}%
\pgfpathrectangle{\pgfqpoint{0.100000in}{0.212622in}}{\pgfqpoint{3.696000in}{3.696000in}}%
\pgfusepath{clip}%
\pgfsetbuttcap%
\pgfsetroundjoin%
\definecolor{currentfill}{rgb}{0.121569,0.466667,0.705882}%
\pgfsetfillcolor{currentfill}%
\pgfsetfillopacity{0.933382}%
\pgfsetlinewidth{1.003750pt}%
\definecolor{currentstroke}{rgb}{0.121569,0.466667,0.705882}%
\pgfsetstrokecolor{currentstroke}%
\pgfsetstrokeopacity{0.933382}%
\pgfsetdash{}{0pt}%
\pgfpathmoveto{\pgfqpoint{2.117666in}{1.956900in}}%
\pgfpathcurveto{\pgfqpoint{2.125903in}{1.956900in}}{\pgfqpoint{2.133803in}{1.960173in}}{\pgfqpoint{2.139627in}{1.965997in}}%
\pgfpathcurveto{\pgfqpoint{2.145451in}{1.971821in}}{\pgfqpoint{2.148723in}{1.979721in}}{\pgfqpoint{2.148723in}{1.987957in}}%
\pgfpathcurveto{\pgfqpoint{2.148723in}{1.996193in}}{\pgfqpoint{2.145451in}{2.004093in}}{\pgfqpoint{2.139627in}{2.009917in}}%
\pgfpathcurveto{\pgfqpoint{2.133803in}{2.015741in}}{\pgfqpoint{2.125903in}{2.019013in}}{\pgfqpoint{2.117666in}{2.019013in}}%
\pgfpathcurveto{\pgfqpoint{2.109430in}{2.019013in}}{\pgfqpoint{2.101530in}{2.015741in}}{\pgfqpoint{2.095706in}{2.009917in}}%
\pgfpathcurveto{\pgfqpoint{2.089882in}{2.004093in}}{\pgfqpoint{2.086610in}{1.996193in}}{\pgfqpoint{2.086610in}{1.987957in}}%
\pgfpathcurveto{\pgfqpoint{2.086610in}{1.979721in}}{\pgfqpoint{2.089882in}{1.971821in}}{\pgfqpoint{2.095706in}{1.965997in}}%
\pgfpathcurveto{\pgfqpoint{2.101530in}{1.960173in}}{\pgfqpoint{2.109430in}{1.956900in}}{\pgfqpoint{2.117666in}{1.956900in}}%
\pgfpathclose%
\pgfusepath{stroke,fill}%
\end{pgfscope}%
\begin{pgfscope}%
\pgfpathrectangle{\pgfqpoint{0.100000in}{0.212622in}}{\pgfqpoint{3.696000in}{3.696000in}}%
\pgfusepath{clip}%
\pgfsetbuttcap%
\pgfsetroundjoin%
\definecolor{currentfill}{rgb}{0.121569,0.466667,0.705882}%
\pgfsetfillcolor{currentfill}%
\pgfsetfillopacity{0.933634}%
\pgfsetlinewidth{1.003750pt}%
\definecolor{currentstroke}{rgb}{0.121569,0.466667,0.705882}%
\pgfsetstrokecolor{currentstroke}%
\pgfsetstrokeopacity{0.933634}%
\pgfsetdash{}{0pt}%
\pgfpathmoveto{\pgfqpoint{1.949891in}{1.892012in}}%
\pgfpathcurveto{\pgfqpoint{1.958127in}{1.892012in}}{\pgfqpoint{1.966027in}{1.895284in}}{\pgfqpoint{1.971851in}{1.901108in}}%
\pgfpathcurveto{\pgfqpoint{1.977675in}{1.906932in}}{\pgfqpoint{1.980948in}{1.914832in}}{\pgfqpoint{1.980948in}{1.923069in}}%
\pgfpathcurveto{\pgfqpoint{1.980948in}{1.931305in}}{\pgfqpoint{1.977675in}{1.939205in}}{\pgfqpoint{1.971851in}{1.945029in}}%
\pgfpathcurveto{\pgfqpoint{1.966027in}{1.950853in}}{\pgfqpoint{1.958127in}{1.954125in}}{\pgfqpoint{1.949891in}{1.954125in}}%
\pgfpathcurveto{\pgfqpoint{1.941655in}{1.954125in}}{\pgfqpoint{1.933755in}{1.950853in}}{\pgfqpoint{1.927931in}{1.945029in}}%
\pgfpathcurveto{\pgfqpoint{1.922107in}{1.939205in}}{\pgfqpoint{1.918835in}{1.931305in}}{\pgfqpoint{1.918835in}{1.923069in}}%
\pgfpathcurveto{\pgfqpoint{1.918835in}{1.914832in}}{\pgfqpoint{1.922107in}{1.906932in}}{\pgfqpoint{1.927931in}{1.901108in}}%
\pgfpathcurveto{\pgfqpoint{1.933755in}{1.895284in}}{\pgfqpoint{1.941655in}{1.892012in}}{\pgfqpoint{1.949891in}{1.892012in}}%
\pgfpathclose%
\pgfusepath{stroke,fill}%
\end{pgfscope}%
\begin{pgfscope}%
\pgfpathrectangle{\pgfqpoint{0.100000in}{0.212622in}}{\pgfqpoint{3.696000in}{3.696000in}}%
\pgfusepath{clip}%
\pgfsetbuttcap%
\pgfsetroundjoin%
\definecolor{currentfill}{rgb}{0.121569,0.466667,0.705882}%
\pgfsetfillcolor{currentfill}%
\pgfsetfillopacity{0.933651}%
\pgfsetlinewidth{1.003750pt}%
\definecolor{currentstroke}{rgb}{0.121569,0.466667,0.705882}%
\pgfsetstrokecolor{currentstroke}%
\pgfsetstrokeopacity{0.933651}%
\pgfsetdash{}{0pt}%
\pgfpathmoveto{\pgfqpoint{1.988565in}{1.025372in}}%
\pgfpathcurveto{\pgfqpoint{1.996801in}{1.025372in}}{\pgfqpoint{2.004701in}{1.028645in}}{\pgfqpoint{2.010525in}{1.034469in}}%
\pgfpathcurveto{\pgfqpoint{2.016349in}{1.040292in}}{\pgfqpoint{2.019621in}{1.048193in}}{\pgfqpoint{2.019621in}{1.056429in}}%
\pgfpathcurveto{\pgfqpoint{2.019621in}{1.064665in}}{\pgfqpoint{2.016349in}{1.072565in}}{\pgfqpoint{2.010525in}{1.078389in}}%
\pgfpathcurveto{\pgfqpoint{2.004701in}{1.084213in}}{\pgfqpoint{1.996801in}{1.087485in}}{\pgfqpoint{1.988565in}{1.087485in}}%
\pgfpathcurveto{\pgfqpoint{1.980328in}{1.087485in}}{\pgfqpoint{1.972428in}{1.084213in}}{\pgfqpoint{1.966604in}{1.078389in}}%
\pgfpathcurveto{\pgfqpoint{1.960780in}{1.072565in}}{\pgfqpoint{1.957508in}{1.064665in}}{\pgfqpoint{1.957508in}{1.056429in}}%
\pgfpathcurveto{\pgfqpoint{1.957508in}{1.048193in}}{\pgfqpoint{1.960780in}{1.040292in}}{\pgfqpoint{1.966604in}{1.034469in}}%
\pgfpathcurveto{\pgfqpoint{1.972428in}{1.028645in}}{\pgfqpoint{1.980328in}{1.025372in}}{\pgfqpoint{1.988565in}{1.025372in}}%
\pgfpathclose%
\pgfusepath{stroke,fill}%
\end{pgfscope}%
\begin{pgfscope}%
\pgfpathrectangle{\pgfqpoint{0.100000in}{0.212622in}}{\pgfqpoint{3.696000in}{3.696000in}}%
\pgfusepath{clip}%
\pgfsetbuttcap%
\pgfsetroundjoin%
\definecolor{currentfill}{rgb}{0.121569,0.466667,0.705882}%
\pgfsetfillcolor{currentfill}%
\pgfsetfillopacity{0.933739}%
\pgfsetlinewidth{1.003750pt}%
\definecolor{currentstroke}{rgb}{0.121569,0.466667,0.705882}%
\pgfsetstrokecolor{currentstroke}%
\pgfsetstrokeopacity{0.933739}%
\pgfsetdash{}{0pt}%
\pgfpathmoveto{\pgfqpoint{2.116459in}{1.955068in}}%
\pgfpathcurveto{\pgfqpoint{2.124695in}{1.955068in}}{\pgfqpoint{2.132595in}{1.958340in}}{\pgfqpoint{2.138419in}{1.964164in}}%
\pgfpathcurveto{\pgfqpoint{2.144243in}{1.969988in}}{\pgfqpoint{2.147515in}{1.977888in}}{\pgfqpoint{2.147515in}{1.986125in}}%
\pgfpathcurveto{\pgfqpoint{2.147515in}{1.994361in}}{\pgfqpoint{2.144243in}{2.002261in}}{\pgfqpoint{2.138419in}{2.008085in}}%
\pgfpathcurveto{\pgfqpoint{2.132595in}{2.013909in}}{\pgfqpoint{2.124695in}{2.017181in}}{\pgfqpoint{2.116459in}{2.017181in}}%
\pgfpathcurveto{\pgfqpoint{2.108223in}{2.017181in}}{\pgfqpoint{2.100323in}{2.013909in}}{\pgfqpoint{2.094499in}{2.008085in}}%
\pgfpathcurveto{\pgfqpoint{2.088675in}{2.002261in}}{\pgfqpoint{2.085402in}{1.994361in}}{\pgfqpoint{2.085402in}{1.986125in}}%
\pgfpathcurveto{\pgfqpoint{2.085402in}{1.977888in}}{\pgfqpoint{2.088675in}{1.969988in}}{\pgfqpoint{2.094499in}{1.964164in}}%
\pgfpathcurveto{\pgfqpoint{2.100323in}{1.958340in}}{\pgfqpoint{2.108223in}{1.955068in}}{\pgfqpoint{2.116459in}{1.955068in}}%
\pgfpathclose%
\pgfusepath{stroke,fill}%
\end{pgfscope}%
\begin{pgfscope}%
\pgfpathrectangle{\pgfqpoint{0.100000in}{0.212622in}}{\pgfqpoint{3.696000in}{3.696000in}}%
\pgfusepath{clip}%
\pgfsetbuttcap%
\pgfsetroundjoin%
\definecolor{currentfill}{rgb}{0.121569,0.466667,0.705882}%
\pgfsetfillcolor{currentfill}%
\pgfsetfillopacity{0.934056}%
\pgfsetlinewidth{1.003750pt}%
\definecolor{currentstroke}{rgb}{0.121569,0.466667,0.705882}%
\pgfsetstrokecolor{currentstroke}%
\pgfsetstrokeopacity{0.934056}%
\pgfsetdash{}{0pt}%
\pgfpathmoveto{\pgfqpoint{2.115881in}{1.953811in}}%
\pgfpathcurveto{\pgfqpoint{2.124117in}{1.953811in}}{\pgfqpoint{2.132017in}{1.957083in}}{\pgfqpoint{2.137841in}{1.962907in}}%
\pgfpathcurveto{\pgfqpoint{2.143665in}{1.968731in}}{\pgfqpoint{2.146938in}{1.976631in}}{\pgfqpoint{2.146938in}{1.984867in}}%
\pgfpathcurveto{\pgfqpoint{2.146938in}{1.993103in}}{\pgfqpoint{2.143665in}{2.001003in}}{\pgfqpoint{2.137841in}{2.006827in}}%
\pgfpathcurveto{\pgfqpoint{2.132017in}{2.012651in}}{\pgfqpoint{2.124117in}{2.015924in}}{\pgfqpoint{2.115881in}{2.015924in}}%
\pgfpathcurveto{\pgfqpoint{2.107645in}{2.015924in}}{\pgfqpoint{2.099745in}{2.012651in}}{\pgfqpoint{2.093921in}{2.006827in}}%
\pgfpathcurveto{\pgfqpoint{2.088097in}{2.001003in}}{\pgfqpoint{2.084825in}{1.993103in}}{\pgfqpoint{2.084825in}{1.984867in}}%
\pgfpathcurveto{\pgfqpoint{2.084825in}{1.976631in}}{\pgfqpoint{2.088097in}{1.968731in}}{\pgfqpoint{2.093921in}{1.962907in}}%
\pgfpathcurveto{\pgfqpoint{2.099745in}{1.957083in}}{\pgfqpoint{2.107645in}{1.953811in}}{\pgfqpoint{2.115881in}{1.953811in}}%
\pgfpathclose%
\pgfusepath{stroke,fill}%
\end{pgfscope}%
\begin{pgfscope}%
\pgfpathrectangle{\pgfqpoint{0.100000in}{0.212622in}}{\pgfqpoint{3.696000in}{3.696000in}}%
\pgfusepath{clip}%
\pgfsetbuttcap%
\pgfsetroundjoin%
\definecolor{currentfill}{rgb}{0.121569,0.466667,0.705882}%
\pgfsetfillcolor{currentfill}%
\pgfsetfillopacity{0.934183}%
\pgfsetlinewidth{1.003750pt}%
\definecolor{currentstroke}{rgb}{0.121569,0.466667,0.705882}%
\pgfsetstrokecolor{currentstroke}%
\pgfsetstrokeopacity{0.934183}%
\pgfsetdash{}{0pt}%
\pgfpathmoveto{\pgfqpoint{2.115502in}{1.953207in}}%
\pgfpathcurveto{\pgfqpoint{2.123739in}{1.953207in}}{\pgfqpoint{2.131639in}{1.956479in}}{\pgfqpoint{2.137463in}{1.962303in}}%
\pgfpathcurveto{\pgfqpoint{2.143286in}{1.968127in}}{\pgfqpoint{2.146559in}{1.976027in}}{\pgfqpoint{2.146559in}{1.984263in}}%
\pgfpathcurveto{\pgfqpoint{2.146559in}{1.992500in}}{\pgfqpoint{2.143286in}{2.000400in}}{\pgfqpoint{2.137463in}{2.006224in}}%
\pgfpathcurveto{\pgfqpoint{2.131639in}{2.012048in}}{\pgfqpoint{2.123739in}{2.015320in}}{\pgfqpoint{2.115502in}{2.015320in}}%
\pgfpathcurveto{\pgfqpoint{2.107266in}{2.015320in}}{\pgfqpoint{2.099366in}{2.012048in}}{\pgfqpoint{2.093542in}{2.006224in}}%
\pgfpathcurveto{\pgfqpoint{2.087718in}{2.000400in}}{\pgfqpoint{2.084446in}{1.992500in}}{\pgfqpoint{2.084446in}{1.984263in}}%
\pgfpathcurveto{\pgfqpoint{2.084446in}{1.976027in}}{\pgfqpoint{2.087718in}{1.968127in}}{\pgfqpoint{2.093542in}{1.962303in}}%
\pgfpathcurveto{\pgfqpoint{2.099366in}{1.956479in}}{\pgfqpoint{2.107266in}{1.953207in}}{\pgfqpoint{2.115502in}{1.953207in}}%
\pgfpathclose%
\pgfusepath{stroke,fill}%
\end{pgfscope}%
\begin{pgfscope}%
\pgfpathrectangle{\pgfqpoint{0.100000in}{0.212622in}}{\pgfqpoint{3.696000in}{3.696000in}}%
\pgfusepath{clip}%
\pgfsetbuttcap%
\pgfsetroundjoin%
\definecolor{currentfill}{rgb}{0.121569,0.466667,0.705882}%
\pgfsetfillcolor{currentfill}%
\pgfsetfillopacity{0.934466}%
\pgfsetlinewidth{1.003750pt}%
\definecolor{currentstroke}{rgb}{0.121569,0.466667,0.705882}%
\pgfsetstrokecolor{currentstroke}%
\pgfsetstrokeopacity{0.934466}%
\pgfsetdash{}{0pt}%
\pgfpathmoveto{\pgfqpoint{2.114884in}{1.952251in}}%
\pgfpathcurveto{\pgfqpoint{2.123120in}{1.952251in}}{\pgfqpoint{2.131020in}{1.955523in}}{\pgfqpoint{2.136844in}{1.961347in}}%
\pgfpathcurveto{\pgfqpoint{2.142668in}{1.967171in}}{\pgfqpoint{2.145940in}{1.975071in}}{\pgfqpoint{2.145940in}{1.983307in}}%
\pgfpathcurveto{\pgfqpoint{2.145940in}{1.991543in}}{\pgfqpoint{2.142668in}{1.999444in}}{\pgfqpoint{2.136844in}{2.005267in}}%
\pgfpathcurveto{\pgfqpoint{2.131020in}{2.011091in}}{\pgfqpoint{2.123120in}{2.014364in}}{\pgfqpoint{2.114884in}{2.014364in}}%
\pgfpathcurveto{\pgfqpoint{2.106648in}{2.014364in}}{\pgfqpoint{2.098747in}{2.011091in}}{\pgfqpoint{2.092924in}{2.005267in}}%
\pgfpathcurveto{\pgfqpoint{2.087100in}{1.999444in}}{\pgfqpoint{2.083827in}{1.991543in}}{\pgfqpoint{2.083827in}{1.983307in}}%
\pgfpathcurveto{\pgfqpoint{2.083827in}{1.975071in}}{\pgfqpoint{2.087100in}{1.967171in}}{\pgfqpoint{2.092924in}{1.961347in}}%
\pgfpathcurveto{\pgfqpoint{2.098747in}{1.955523in}}{\pgfqpoint{2.106648in}{1.952251in}}{\pgfqpoint{2.114884in}{1.952251in}}%
\pgfpathclose%
\pgfusepath{stroke,fill}%
\end{pgfscope}%
\begin{pgfscope}%
\pgfpathrectangle{\pgfqpoint{0.100000in}{0.212622in}}{\pgfqpoint{3.696000in}{3.696000in}}%
\pgfusepath{clip}%
\pgfsetbuttcap%
\pgfsetroundjoin%
\definecolor{currentfill}{rgb}{0.121569,0.466667,0.705882}%
\pgfsetfillcolor{currentfill}%
\pgfsetfillopacity{0.934509}%
\pgfsetlinewidth{1.003750pt}%
\definecolor{currentstroke}{rgb}{0.121569,0.466667,0.705882}%
\pgfsetstrokecolor{currentstroke}%
\pgfsetstrokeopacity{0.934509}%
\pgfsetdash{}{0pt}%
\pgfpathmoveto{\pgfqpoint{1.954948in}{1.889880in}}%
\pgfpathcurveto{\pgfqpoint{1.963185in}{1.889880in}}{\pgfqpoint{1.971085in}{1.893152in}}{\pgfqpoint{1.976909in}{1.898976in}}%
\pgfpathcurveto{\pgfqpoint{1.982732in}{1.904800in}}{\pgfqpoint{1.986005in}{1.912700in}}{\pgfqpoint{1.986005in}{1.920937in}}%
\pgfpathcurveto{\pgfqpoint{1.986005in}{1.929173in}}{\pgfqpoint{1.982732in}{1.937073in}}{\pgfqpoint{1.976909in}{1.942897in}}%
\pgfpathcurveto{\pgfqpoint{1.971085in}{1.948721in}}{\pgfqpoint{1.963185in}{1.951993in}}{\pgfqpoint{1.954948in}{1.951993in}}%
\pgfpathcurveto{\pgfqpoint{1.946712in}{1.951993in}}{\pgfqpoint{1.938812in}{1.948721in}}{\pgfqpoint{1.932988in}{1.942897in}}%
\pgfpathcurveto{\pgfqpoint{1.927164in}{1.937073in}}{\pgfqpoint{1.923892in}{1.929173in}}{\pgfqpoint{1.923892in}{1.920937in}}%
\pgfpathcurveto{\pgfqpoint{1.923892in}{1.912700in}}{\pgfqpoint{1.927164in}{1.904800in}}{\pgfqpoint{1.932988in}{1.898976in}}%
\pgfpathcurveto{\pgfqpoint{1.938812in}{1.893152in}}{\pgfqpoint{1.946712in}{1.889880in}}{\pgfqpoint{1.954948in}{1.889880in}}%
\pgfpathclose%
\pgfusepath{stroke,fill}%
\end{pgfscope}%
\begin{pgfscope}%
\pgfpathrectangle{\pgfqpoint{0.100000in}{0.212622in}}{\pgfqpoint{3.696000in}{3.696000in}}%
\pgfusepath{clip}%
\pgfsetbuttcap%
\pgfsetroundjoin%
\definecolor{currentfill}{rgb}{0.121569,0.466667,0.705882}%
\pgfsetfillcolor{currentfill}%
\pgfsetfillopacity{0.934537}%
\pgfsetlinewidth{1.003750pt}%
\definecolor{currentstroke}{rgb}{0.121569,0.466667,0.705882}%
\pgfsetstrokecolor{currentstroke}%
\pgfsetstrokeopacity{0.934537}%
\pgfsetdash{}{0pt}%
\pgfpathmoveto{\pgfqpoint{1.999741in}{1.019657in}}%
\pgfpathcurveto{\pgfqpoint{2.007977in}{1.019657in}}{\pgfqpoint{2.015877in}{1.022929in}}{\pgfqpoint{2.021701in}{1.028753in}}%
\pgfpathcurveto{\pgfqpoint{2.027525in}{1.034577in}}{\pgfqpoint{2.030797in}{1.042477in}}{\pgfqpoint{2.030797in}{1.050713in}}%
\pgfpathcurveto{\pgfqpoint{2.030797in}{1.058949in}}{\pgfqpoint{2.027525in}{1.066849in}}{\pgfqpoint{2.021701in}{1.072673in}}%
\pgfpathcurveto{\pgfqpoint{2.015877in}{1.078497in}}{\pgfqpoint{2.007977in}{1.081770in}}{\pgfqpoint{1.999741in}{1.081770in}}%
\pgfpathcurveto{\pgfqpoint{1.991504in}{1.081770in}}{\pgfqpoint{1.983604in}{1.078497in}}{\pgfqpoint{1.977781in}{1.072673in}}%
\pgfpathcurveto{\pgfqpoint{1.971957in}{1.066849in}}{\pgfqpoint{1.968684in}{1.058949in}}{\pgfqpoint{1.968684in}{1.050713in}}%
\pgfpathcurveto{\pgfqpoint{1.968684in}{1.042477in}}{\pgfqpoint{1.971957in}{1.034577in}}{\pgfqpoint{1.977781in}{1.028753in}}%
\pgfpathcurveto{\pgfqpoint{1.983604in}{1.022929in}}{\pgfqpoint{1.991504in}{1.019657in}}{\pgfqpoint{1.999741in}{1.019657in}}%
\pgfpathclose%
\pgfusepath{stroke,fill}%
\end{pgfscope}%
\begin{pgfscope}%
\pgfpathrectangle{\pgfqpoint{0.100000in}{0.212622in}}{\pgfqpoint{3.696000in}{3.696000in}}%
\pgfusepath{clip}%
\pgfsetbuttcap%
\pgfsetroundjoin%
\definecolor{currentfill}{rgb}{0.121569,0.466667,0.705882}%
\pgfsetfillcolor{currentfill}%
\pgfsetfillopacity{0.934867}%
\pgfsetlinewidth{1.003750pt}%
\definecolor{currentstroke}{rgb}{0.121569,0.466667,0.705882}%
\pgfsetstrokecolor{currentstroke}%
\pgfsetstrokeopacity{0.934867}%
\pgfsetdash{}{0pt}%
\pgfpathmoveto{\pgfqpoint{2.113526in}{1.950350in}}%
\pgfpathcurveto{\pgfqpoint{2.121762in}{1.950350in}}{\pgfqpoint{2.129662in}{1.953622in}}{\pgfqpoint{2.135486in}{1.959446in}}%
\pgfpathcurveto{\pgfqpoint{2.141310in}{1.965270in}}{\pgfqpoint{2.144583in}{1.973170in}}{\pgfqpoint{2.144583in}{1.981406in}}%
\pgfpathcurveto{\pgfqpoint{2.144583in}{1.989642in}}{\pgfqpoint{2.141310in}{1.997542in}}{\pgfqpoint{2.135486in}{2.003366in}}%
\pgfpathcurveto{\pgfqpoint{2.129662in}{2.009190in}}{\pgfqpoint{2.121762in}{2.012463in}}{\pgfqpoint{2.113526in}{2.012463in}}%
\pgfpathcurveto{\pgfqpoint{2.105290in}{2.012463in}}{\pgfqpoint{2.097390in}{2.009190in}}{\pgfqpoint{2.091566in}{2.003366in}}%
\pgfpathcurveto{\pgfqpoint{2.085742in}{1.997542in}}{\pgfqpoint{2.082470in}{1.989642in}}{\pgfqpoint{2.082470in}{1.981406in}}%
\pgfpathcurveto{\pgfqpoint{2.082470in}{1.973170in}}{\pgfqpoint{2.085742in}{1.965270in}}{\pgfqpoint{2.091566in}{1.959446in}}%
\pgfpathcurveto{\pgfqpoint{2.097390in}{1.953622in}}{\pgfqpoint{2.105290in}{1.950350in}}{\pgfqpoint{2.113526in}{1.950350in}}%
\pgfpathclose%
\pgfusepath{stroke,fill}%
\end{pgfscope}%
\begin{pgfscope}%
\pgfpathrectangle{\pgfqpoint{0.100000in}{0.212622in}}{\pgfqpoint{3.696000in}{3.696000in}}%
\pgfusepath{clip}%
\pgfsetbuttcap%
\pgfsetroundjoin%
\definecolor{currentfill}{rgb}{0.121569,0.466667,0.705882}%
\pgfsetfillcolor{currentfill}%
\pgfsetfillopacity{0.935082}%
\pgfsetlinewidth{1.003750pt}%
\definecolor{currentstroke}{rgb}{0.121569,0.466667,0.705882}%
\pgfsetstrokecolor{currentstroke}%
\pgfsetstrokeopacity{0.935082}%
\pgfsetdash{}{0pt}%
\pgfpathmoveto{\pgfqpoint{2.112501in}{1.948803in}}%
\pgfpathcurveto{\pgfqpoint{2.120738in}{1.948803in}}{\pgfqpoint{2.128638in}{1.952076in}}{\pgfqpoint{2.134462in}{1.957899in}}%
\pgfpathcurveto{\pgfqpoint{2.140286in}{1.963723in}}{\pgfqpoint{2.143558in}{1.971623in}}{\pgfqpoint{2.143558in}{1.979860in}}%
\pgfpathcurveto{\pgfqpoint{2.143558in}{1.988096in}}{\pgfqpoint{2.140286in}{1.995996in}}{\pgfqpoint{2.134462in}{2.001820in}}%
\pgfpathcurveto{\pgfqpoint{2.128638in}{2.007644in}}{\pgfqpoint{2.120738in}{2.010916in}}{\pgfqpoint{2.112501in}{2.010916in}}%
\pgfpathcurveto{\pgfqpoint{2.104265in}{2.010916in}}{\pgfqpoint{2.096365in}{2.007644in}}{\pgfqpoint{2.090541in}{2.001820in}}%
\pgfpathcurveto{\pgfqpoint{2.084717in}{1.995996in}}{\pgfqpoint{2.081445in}{1.988096in}}{\pgfqpoint{2.081445in}{1.979860in}}%
\pgfpathcurveto{\pgfqpoint{2.081445in}{1.971623in}}{\pgfqpoint{2.084717in}{1.963723in}}{\pgfqpoint{2.090541in}{1.957899in}}%
\pgfpathcurveto{\pgfqpoint{2.096365in}{1.952076in}}{\pgfqpoint{2.104265in}{1.948803in}}{\pgfqpoint{2.112501in}{1.948803in}}%
\pgfpathclose%
\pgfusepath{stroke,fill}%
\end{pgfscope}%
\begin{pgfscope}%
\pgfpathrectangle{\pgfqpoint{0.100000in}{0.212622in}}{\pgfqpoint{3.696000in}{3.696000in}}%
\pgfusepath{clip}%
\pgfsetbuttcap%
\pgfsetroundjoin%
\definecolor{currentfill}{rgb}{0.121569,0.466667,0.705882}%
\pgfsetfillcolor{currentfill}%
\pgfsetfillopacity{0.935189}%
\pgfsetlinewidth{1.003750pt}%
\definecolor{currentstroke}{rgb}{0.121569,0.466667,0.705882}%
\pgfsetstrokecolor{currentstroke}%
\pgfsetstrokeopacity{0.935189}%
\pgfsetdash{}{0pt}%
\pgfpathmoveto{\pgfqpoint{2.112237in}{1.948201in}}%
\pgfpathcurveto{\pgfqpoint{2.120473in}{1.948201in}}{\pgfqpoint{2.128373in}{1.951474in}}{\pgfqpoint{2.134197in}{1.957298in}}%
\pgfpathcurveto{\pgfqpoint{2.140021in}{1.963121in}}{\pgfqpoint{2.143293in}{1.971022in}}{\pgfqpoint{2.143293in}{1.979258in}}%
\pgfpathcurveto{\pgfqpoint{2.143293in}{1.987494in}}{\pgfqpoint{2.140021in}{1.995394in}}{\pgfqpoint{2.134197in}{2.001218in}}%
\pgfpathcurveto{\pgfqpoint{2.128373in}{2.007042in}}{\pgfqpoint{2.120473in}{2.010314in}}{\pgfqpoint{2.112237in}{2.010314in}}%
\pgfpathcurveto{\pgfqpoint{2.104001in}{2.010314in}}{\pgfqpoint{2.096101in}{2.007042in}}{\pgfqpoint{2.090277in}{2.001218in}}%
\pgfpathcurveto{\pgfqpoint{2.084453in}{1.995394in}}{\pgfqpoint{2.081180in}{1.987494in}}{\pgfqpoint{2.081180in}{1.979258in}}%
\pgfpathcurveto{\pgfqpoint{2.081180in}{1.971022in}}{\pgfqpoint{2.084453in}{1.963121in}}{\pgfqpoint{2.090277in}{1.957298in}}%
\pgfpathcurveto{\pgfqpoint{2.096101in}{1.951474in}}{\pgfqpoint{2.104001in}{1.948201in}}{\pgfqpoint{2.112237in}{1.948201in}}%
\pgfpathclose%
\pgfusepath{stroke,fill}%
\end{pgfscope}%
\begin{pgfscope}%
\pgfpathrectangle{\pgfqpoint{0.100000in}{0.212622in}}{\pgfqpoint{3.696000in}{3.696000in}}%
\pgfusepath{clip}%
\pgfsetbuttcap%
\pgfsetroundjoin%
\definecolor{currentfill}{rgb}{0.121569,0.466667,0.705882}%
\pgfsetfillcolor{currentfill}%
\pgfsetfillopacity{0.935264}%
\pgfsetlinewidth{1.003750pt}%
\definecolor{currentstroke}{rgb}{0.121569,0.466667,0.705882}%
\pgfsetstrokecolor{currentstroke}%
\pgfsetstrokeopacity{0.935264}%
\pgfsetdash{}{0pt}%
\pgfpathmoveto{\pgfqpoint{1.961094in}{1.888044in}}%
\pgfpathcurveto{\pgfqpoint{1.969330in}{1.888044in}}{\pgfqpoint{1.977230in}{1.891317in}}{\pgfqpoint{1.983054in}{1.897140in}}%
\pgfpathcurveto{\pgfqpoint{1.988878in}{1.902964in}}{\pgfqpoint{1.992150in}{1.910864in}}{\pgfqpoint{1.992150in}{1.919101in}}%
\pgfpathcurveto{\pgfqpoint{1.992150in}{1.927337in}}{\pgfqpoint{1.988878in}{1.935237in}}{\pgfqpoint{1.983054in}{1.941061in}}%
\pgfpathcurveto{\pgfqpoint{1.977230in}{1.946885in}}{\pgfqpoint{1.969330in}{1.950157in}}{\pgfqpoint{1.961094in}{1.950157in}}%
\pgfpathcurveto{\pgfqpoint{1.952858in}{1.950157in}}{\pgfqpoint{1.944958in}{1.946885in}}{\pgfqpoint{1.939134in}{1.941061in}}%
\pgfpathcurveto{\pgfqpoint{1.933310in}{1.935237in}}{\pgfqpoint{1.930037in}{1.927337in}}{\pgfqpoint{1.930037in}{1.919101in}}%
\pgfpathcurveto{\pgfqpoint{1.930037in}{1.910864in}}{\pgfqpoint{1.933310in}{1.902964in}}{\pgfqpoint{1.939134in}{1.897140in}}%
\pgfpathcurveto{\pgfqpoint{1.944958in}{1.891317in}}{\pgfqpoint{1.952858in}{1.888044in}}{\pgfqpoint{1.961094in}{1.888044in}}%
\pgfpathclose%
\pgfusepath{stroke,fill}%
\end{pgfscope}%
\begin{pgfscope}%
\pgfpathrectangle{\pgfqpoint{0.100000in}{0.212622in}}{\pgfqpoint{3.696000in}{3.696000in}}%
\pgfusepath{clip}%
\pgfsetbuttcap%
\pgfsetroundjoin%
\definecolor{currentfill}{rgb}{0.121569,0.466667,0.705882}%
\pgfsetfillcolor{currentfill}%
\pgfsetfillopacity{0.935370}%
\pgfsetlinewidth{1.003750pt}%
\definecolor{currentstroke}{rgb}{0.121569,0.466667,0.705882}%
\pgfsetstrokecolor{currentstroke}%
\pgfsetstrokeopacity{0.935370}%
\pgfsetdash{}{0pt}%
\pgfpathmoveto{\pgfqpoint{2.111624in}{1.947209in}}%
\pgfpathcurveto{\pgfqpoint{2.119860in}{1.947209in}}{\pgfqpoint{2.127760in}{1.950481in}}{\pgfqpoint{2.133584in}{1.956305in}}%
\pgfpathcurveto{\pgfqpoint{2.139408in}{1.962129in}}{\pgfqpoint{2.142680in}{1.970029in}}{\pgfqpoint{2.142680in}{1.978265in}}%
\pgfpathcurveto{\pgfqpoint{2.142680in}{1.986501in}}{\pgfqpoint{2.139408in}{1.994401in}}{\pgfqpoint{2.133584in}{2.000225in}}%
\pgfpathcurveto{\pgfqpoint{2.127760in}{2.006049in}}{\pgfqpoint{2.119860in}{2.009322in}}{\pgfqpoint{2.111624in}{2.009322in}}%
\pgfpathcurveto{\pgfqpoint{2.103387in}{2.009322in}}{\pgfqpoint{2.095487in}{2.006049in}}{\pgfqpoint{2.089663in}{2.000225in}}%
\pgfpathcurveto{\pgfqpoint{2.083840in}{1.994401in}}{\pgfqpoint{2.080567in}{1.986501in}}{\pgfqpoint{2.080567in}{1.978265in}}%
\pgfpathcurveto{\pgfqpoint{2.080567in}{1.970029in}}{\pgfqpoint{2.083840in}{1.962129in}}{\pgfqpoint{2.089663in}{1.956305in}}%
\pgfpathcurveto{\pgfqpoint{2.095487in}{1.950481in}}{\pgfqpoint{2.103387in}{1.947209in}}{\pgfqpoint{2.111624in}{1.947209in}}%
\pgfpathclose%
\pgfusepath{stroke,fill}%
\end{pgfscope}%
\begin{pgfscope}%
\pgfpathrectangle{\pgfqpoint{0.100000in}{0.212622in}}{\pgfqpoint{3.696000in}{3.696000in}}%
\pgfusepath{clip}%
\pgfsetbuttcap%
\pgfsetroundjoin%
\definecolor{currentfill}{rgb}{0.121569,0.466667,0.705882}%
\pgfsetfillcolor{currentfill}%
\pgfsetfillopacity{0.935450}%
\pgfsetlinewidth{1.003750pt}%
\definecolor{currentstroke}{rgb}{0.121569,0.466667,0.705882}%
\pgfsetstrokecolor{currentstroke}%
\pgfsetstrokeopacity{0.935450}%
\pgfsetdash{}{0pt}%
\pgfpathmoveto{\pgfqpoint{2.709312in}{1.160481in}}%
\pgfpathcurveto{\pgfqpoint{2.717548in}{1.160481in}}{\pgfqpoint{2.725448in}{1.163753in}}{\pgfqpoint{2.731272in}{1.169577in}}%
\pgfpathcurveto{\pgfqpoint{2.737096in}{1.175401in}}{\pgfqpoint{2.740368in}{1.183301in}}{\pgfqpoint{2.740368in}{1.191537in}}%
\pgfpathcurveto{\pgfqpoint{2.740368in}{1.199773in}}{\pgfqpoint{2.737096in}{1.207673in}}{\pgfqpoint{2.731272in}{1.213497in}}%
\pgfpathcurveto{\pgfqpoint{2.725448in}{1.219321in}}{\pgfqpoint{2.717548in}{1.222594in}}{\pgfqpoint{2.709312in}{1.222594in}}%
\pgfpathcurveto{\pgfqpoint{2.701075in}{1.222594in}}{\pgfqpoint{2.693175in}{1.219321in}}{\pgfqpoint{2.687351in}{1.213497in}}%
\pgfpathcurveto{\pgfqpoint{2.681527in}{1.207673in}}{\pgfqpoint{2.678255in}{1.199773in}}{\pgfqpoint{2.678255in}{1.191537in}}%
\pgfpathcurveto{\pgfqpoint{2.678255in}{1.183301in}}{\pgfqpoint{2.681527in}{1.175401in}}{\pgfqpoint{2.687351in}{1.169577in}}%
\pgfpathcurveto{\pgfqpoint{2.693175in}{1.163753in}}{\pgfqpoint{2.701075in}{1.160481in}}{\pgfqpoint{2.709312in}{1.160481in}}%
\pgfpathclose%
\pgfusepath{stroke,fill}%
\end{pgfscope}%
\begin{pgfscope}%
\pgfpathrectangle{\pgfqpoint{0.100000in}{0.212622in}}{\pgfqpoint{3.696000in}{3.696000in}}%
\pgfusepath{clip}%
\pgfsetbuttcap%
\pgfsetroundjoin%
\definecolor{currentfill}{rgb}{0.121569,0.466667,0.705882}%
\pgfsetfillcolor{currentfill}%
\pgfsetfillopacity{0.935716}%
\pgfsetlinewidth{1.003750pt}%
\definecolor{currentstroke}{rgb}{0.121569,0.466667,0.705882}%
\pgfsetstrokecolor{currentstroke}%
\pgfsetstrokeopacity{0.935716}%
\pgfsetdash{}{0pt}%
\pgfpathmoveto{\pgfqpoint{2.110547in}{1.945419in}}%
\pgfpathcurveto{\pgfqpoint{2.118783in}{1.945419in}}{\pgfqpoint{2.126683in}{1.948692in}}{\pgfqpoint{2.132507in}{1.954516in}}%
\pgfpathcurveto{\pgfqpoint{2.138331in}{1.960340in}}{\pgfqpoint{2.141604in}{1.968240in}}{\pgfqpoint{2.141604in}{1.976476in}}%
\pgfpathcurveto{\pgfqpoint{2.141604in}{1.984712in}}{\pgfqpoint{2.138331in}{1.992612in}}{\pgfqpoint{2.132507in}{1.998436in}}%
\pgfpathcurveto{\pgfqpoint{2.126683in}{2.004260in}}{\pgfqpoint{2.118783in}{2.007532in}}{\pgfqpoint{2.110547in}{2.007532in}}%
\pgfpathcurveto{\pgfqpoint{2.102311in}{2.007532in}}{\pgfqpoint{2.094411in}{2.004260in}}{\pgfqpoint{2.088587in}{1.998436in}}%
\pgfpathcurveto{\pgfqpoint{2.082763in}{1.992612in}}{\pgfqpoint{2.079491in}{1.984712in}}{\pgfqpoint{2.079491in}{1.976476in}}%
\pgfpathcurveto{\pgfqpoint{2.079491in}{1.968240in}}{\pgfqpoint{2.082763in}{1.960340in}}{\pgfqpoint{2.088587in}{1.954516in}}%
\pgfpathcurveto{\pgfqpoint{2.094411in}{1.948692in}}{\pgfqpoint{2.102311in}{1.945419in}}{\pgfqpoint{2.110547in}{1.945419in}}%
\pgfpathclose%
\pgfusepath{stroke,fill}%
\end{pgfscope}%
\begin{pgfscope}%
\pgfpathrectangle{\pgfqpoint{0.100000in}{0.212622in}}{\pgfqpoint{3.696000in}{3.696000in}}%
\pgfusepath{clip}%
\pgfsetbuttcap%
\pgfsetroundjoin%
\definecolor{currentfill}{rgb}{0.121569,0.466667,0.705882}%
\pgfsetfillcolor{currentfill}%
\pgfsetfillopacity{0.935793}%
\pgfsetlinewidth{1.003750pt}%
\definecolor{currentstroke}{rgb}{0.121569,0.466667,0.705882}%
\pgfsetstrokecolor{currentstroke}%
\pgfsetstrokeopacity{0.935793}%
\pgfsetdash{}{0pt}%
\pgfpathmoveto{\pgfqpoint{2.010006in}{1.010541in}}%
\pgfpathcurveto{\pgfqpoint{2.018243in}{1.010541in}}{\pgfqpoint{2.026143in}{1.013813in}}{\pgfqpoint{2.031967in}{1.019637in}}%
\pgfpathcurveto{\pgfqpoint{2.037791in}{1.025461in}}{\pgfqpoint{2.041063in}{1.033361in}}{\pgfqpoint{2.041063in}{1.041597in}}%
\pgfpathcurveto{\pgfqpoint{2.041063in}{1.049833in}}{\pgfqpoint{2.037791in}{1.057733in}}{\pgfqpoint{2.031967in}{1.063557in}}%
\pgfpathcurveto{\pgfqpoint{2.026143in}{1.069381in}}{\pgfqpoint{2.018243in}{1.072654in}}{\pgfqpoint{2.010006in}{1.072654in}}%
\pgfpathcurveto{\pgfqpoint{2.001770in}{1.072654in}}{\pgfqpoint{1.993870in}{1.069381in}}{\pgfqpoint{1.988046in}{1.063557in}}%
\pgfpathcurveto{\pgfqpoint{1.982222in}{1.057733in}}{\pgfqpoint{1.978950in}{1.049833in}}{\pgfqpoint{1.978950in}{1.041597in}}%
\pgfpathcurveto{\pgfqpoint{1.978950in}{1.033361in}}{\pgfqpoint{1.982222in}{1.025461in}}{\pgfqpoint{1.988046in}{1.019637in}}%
\pgfpathcurveto{\pgfqpoint{1.993870in}{1.013813in}}{\pgfqpoint{2.001770in}{1.010541in}}{\pgfqpoint{2.010006in}{1.010541in}}%
\pgfpathclose%
\pgfusepath{stroke,fill}%
\end{pgfscope}%
\begin{pgfscope}%
\pgfpathrectangle{\pgfqpoint{0.100000in}{0.212622in}}{\pgfqpoint{3.696000in}{3.696000in}}%
\pgfusepath{clip}%
\pgfsetbuttcap%
\pgfsetroundjoin%
\definecolor{currentfill}{rgb}{0.121569,0.466667,0.705882}%
\pgfsetfillcolor{currentfill}%
\pgfsetfillopacity{0.935995}%
\pgfsetlinewidth{1.003750pt}%
\definecolor{currentstroke}{rgb}{0.121569,0.466667,0.705882}%
\pgfsetstrokecolor{currentstroke}%
\pgfsetstrokeopacity{0.935995}%
\pgfsetdash{}{0pt}%
\pgfpathmoveto{\pgfqpoint{2.109862in}{1.943998in}}%
\pgfpathcurveto{\pgfqpoint{2.118098in}{1.943998in}}{\pgfqpoint{2.125999in}{1.947270in}}{\pgfqpoint{2.131822in}{1.953094in}}%
\pgfpathcurveto{\pgfqpoint{2.137646in}{1.958918in}}{\pgfqpoint{2.140919in}{1.966818in}}{\pgfqpoint{2.140919in}{1.975054in}}%
\pgfpathcurveto{\pgfqpoint{2.140919in}{1.983290in}}{\pgfqpoint{2.137646in}{1.991191in}}{\pgfqpoint{2.131822in}{1.997014in}}%
\pgfpathcurveto{\pgfqpoint{2.125999in}{2.002838in}}{\pgfqpoint{2.118098in}{2.006111in}}{\pgfqpoint{2.109862in}{2.006111in}}%
\pgfpathcurveto{\pgfqpoint{2.101626in}{2.006111in}}{\pgfqpoint{2.093726in}{2.002838in}}{\pgfqpoint{2.087902in}{1.997014in}}%
\pgfpathcurveto{\pgfqpoint{2.082078in}{1.991191in}}{\pgfqpoint{2.078806in}{1.983290in}}{\pgfqpoint{2.078806in}{1.975054in}}%
\pgfpathcurveto{\pgfqpoint{2.078806in}{1.966818in}}{\pgfqpoint{2.082078in}{1.958918in}}{\pgfqpoint{2.087902in}{1.953094in}}%
\pgfpathcurveto{\pgfqpoint{2.093726in}{1.947270in}}{\pgfqpoint{2.101626in}{1.943998in}}{\pgfqpoint{2.109862in}{1.943998in}}%
\pgfpathclose%
\pgfusepath{stroke,fill}%
\end{pgfscope}%
\begin{pgfscope}%
\pgfpathrectangle{\pgfqpoint{0.100000in}{0.212622in}}{\pgfqpoint{3.696000in}{3.696000in}}%
\pgfusepath{clip}%
\pgfsetbuttcap%
\pgfsetroundjoin%
\definecolor{currentfill}{rgb}{0.121569,0.466667,0.705882}%
\pgfsetfillcolor{currentfill}%
\pgfsetfillopacity{0.936379}%
\pgfsetlinewidth{1.003750pt}%
\definecolor{currentstroke}{rgb}{0.121569,0.466667,0.705882}%
\pgfsetstrokecolor{currentstroke}%
\pgfsetstrokeopacity{0.936379}%
\pgfsetdash{}{0pt}%
\pgfpathmoveto{\pgfqpoint{2.015681in}{1.005165in}}%
\pgfpathcurveto{\pgfqpoint{2.023917in}{1.005165in}}{\pgfqpoint{2.031817in}{1.008437in}}{\pgfqpoint{2.037641in}{1.014261in}}%
\pgfpathcurveto{\pgfqpoint{2.043465in}{1.020085in}}{\pgfqpoint{2.046738in}{1.027985in}}{\pgfqpoint{2.046738in}{1.036222in}}%
\pgfpathcurveto{\pgfqpoint{2.046738in}{1.044458in}}{\pgfqpoint{2.043465in}{1.052358in}}{\pgfqpoint{2.037641in}{1.058182in}}%
\pgfpathcurveto{\pgfqpoint{2.031817in}{1.064006in}}{\pgfqpoint{2.023917in}{1.067278in}}{\pgfqpoint{2.015681in}{1.067278in}}%
\pgfpathcurveto{\pgfqpoint{2.007445in}{1.067278in}}{\pgfqpoint{1.999545in}{1.064006in}}{\pgfqpoint{1.993721in}{1.058182in}}%
\pgfpathcurveto{\pgfqpoint{1.987897in}{1.052358in}}{\pgfqpoint{1.984625in}{1.044458in}}{\pgfqpoint{1.984625in}{1.036222in}}%
\pgfpathcurveto{\pgfqpoint{1.984625in}{1.027985in}}{\pgfqpoint{1.987897in}{1.020085in}}{\pgfqpoint{1.993721in}{1.014261in}}%
\pgfpathcurveto{\pgfqpoint{1.999545in}{1.008437in}}{\pgfqpoint{2.007445in}{1.005165in}}{\pgfqpoint{2.015681in}{1.005165in}}%
\pgfpathclose%
\pgfusepath{stroke,fill}%
\end{pgfscope}%
\begin{pgfscope}%
\pgfpathrectangle{\pgfqpoint{0.100000in}{0.212622in}}{\pgfqpoint{3.696000in}{3.696000in}}%
\pgfusepath{clip}%
\pgfsetbuttcap%
\pgfsetroundjoin%
\definecolor{currentfill}{rgb}{0.121569,0.466667,0.705882}%
\pgfsetfillcolor{currentfill}%
\pgfsetfillopacity{0.936381}%
\pgfsetlinewidth{1.003750pt}%
\definecolor{currentstroke}{rgb}{0.121569,0.466667,0.705882}%
\pgfsetstrokecolor{currentstroke}%
\pgfsetstrokeopacity{0.936381}%
\pgfsetdash{}{0pt}%
\pgfpathmoveto{\pgfqpoint{1.967535in}{1.886402in}}%
\pgfpathcurveto{\pgfqpoint{1.975771in}{1.886402in}}{\pgfqpoint{1.983671in}{1.889674in}}{\pgfqpoint{1.989495in}{1.895498in}}%
\pgfpathcurveto{\pgfqpoint{1.995319in}{1.901322in}}{\pgfqpoint{1.998591in}{1.909222in}}{\pgfqpoint{1.998591in}{1.917458in}}%
\pgfpathcurveto{\pgfqpoint{1.998591in}{1.925695in}}{\pgfqpoint{1.995319in}{1.933595in}}{\pgfqpoint{1.989495in}{1.939419in}}%
\pgfpathcurveto{\pgfqpoint{1.983671in}{1.945243in}}{\pgfqpoint{1.975771in}{1.948515in}}{\pgfqpoint{1.967535in}{1.948515in}}%
\pgfpathcurveto{\pgfqpoint{1.959299in}{1.948515in}}{\pgfqpoint{1.951398in}{1.945243in}}{\pgfqpoint{1.945575in}{1.939419in}}%
\pgfpathcurveto{\pgfqpoint{1.939751in}{1.933595in}}{\pgfqpoint{1.936478in}{1.925695in}}{\pgfqpoint{1.936478in}{1.917458in}}%
\pgfpathcurveto{\pgfqpoint{1.936478in}{1.909222in}}{\pgfqpoint{1.939751in}{1.901322in}}{\pgfqpoint{1.945575in}{1.895498in}}%
\pgfpathcurveto{\pgfqpoint{1.951398in}{1.889674in}}{\pgfqpoint{1.959299in}{1.886402in}}{\pgfqpoint{1.967535in}{1.886402in}}%
\pgfpathclose%
\pgfusepath{stroke,fill}%
\end{pgfscope}%
\begin{pgfscope}%
\pgfpathrectangle{\pgfqpoint{0.100000in}{0.212622in}}{\pgfqpoint{3.696000in}{3.696000in}}%
\pgfusepath{clip}%
\pgfsetbuttcap%
\pgfsetroundjoin%
\definecolor{currentfill}{rgb}{0.121569,0.466667,0.705882}%
\pgfsetfillcolor{currentfill}%
\pgfsetfillopacity{0.936496}%
\pgfsetlinewidth{1.003750pt}%
\definecolor{currentstroke}{rgb}{0.121569,0.466667,0.705882}%
\pgfsetstrokecolor{currentstroke}%
\pgfsetstrokeopacity{0.936496}%
\pgfsetdash{}{0pt}%
\pgfpathmoveto{\pgfqpoint{2.108438in}{1.941593in}}%
\pgfpathcurveto{\pgfqpoint{2.116675in}{1.941593in}}{\pgfqpoint{2.124575in}{1.944865in}}{\pgfqpoint{2.130399in}{1.950689in}}%
\pgfpathcurveto{\pgfqpoint{2.136223in}{1.956513in}}{\pgfqpoint{2.139495in}{1.964413in}}{\pgfqpoint{2.139495in}{1.972649in}}%
\pgfpathcurveto{\pgfqpoint{2.139495in}{1.980885in}}{\pgfqpoint{2.136223in}{1.988785in}}{\pgfqpoint{2.130399in}{1.994609in}}%
\pgfpathcurveto{\pgfqpoint{2.124575in}{2.000433in}}{\pgfqpoint{2.116675in}{2.003706in}}{\pgfqpoint{2.108438in}{2.003706in}}%
\pgfpathcurveto{\pgfqpoint{2.100202in}{2.003706in}}{\pgfqpoint{2.092302in}{2.000433in}}{\pgfqpoint{2.086478in}{1.994609in}}%
\pgfpathcurveto{\pgfqpoint{2.080654in}{1.988785in}}{\pgfqpoint{2.077382in}{1.980885in}}{\pgfqpoint{2.077382in}{1.972649in}}%
\pgfpathcurveto{\pgfqpoint{2.077382in}{1.964413in}}{\pgfqpoint{2.080654in}{1.956513in}}{\pgfqpoint{2.086478in}{1.950689in}}%
\pgfpathcurveto{\pgfqpoint{2.092302in}{1.944865in}}{\pgfqpoint{2.100202in}{1.941593in}}{\pgfqpoint{2.108438in}{1.941593in}}%
\pgfpathclose%
\pgfusepath{stroke,fill}%
\end{pgfscope}%
\begin{pgfscope}%
\pgfpathrectangle{\pgfqpoint{0.100000in}{0.212622in}}{\pgfqpoint{3.696000in}{3.696000in}}%
\pgfusepath{clip}%
\pgfsetbuttcap%
\pgfsetroundjoin%
\definecolor{currentfill}{rgb}{0.121569,0.466667,0.705882}%
\pgfsetfillcolor{currentfill}%
\pgfsetfillopacity{0.936952}%
\pgfsetlinewidth{1.003750pt}%
\definecolor{currentstroke}{rgb}{0.121569,0.466667,0.705882}%
\pgfsetstrokecolor{currentstroke}%
\pgfsetstrokeopacity{0.936952}%
\pgfsetdash{}{0pt}%
\pgfpathmoveto{\pgfqpoint{1.971010in}{1.885060in}}%
\pgfpathcurveto{\pgfqpoint{1.979246in}{1.885060in}}{\pgfqpoint{1.987147in}{1.888333in}}{\pgfqpoint{1.992970in}{1.894156in}}%
\pgfpathcurveto{\pgfqpoint{1.998794in}{1.899980in}}{\pgfqpoint{2.002067in}{1.907880in}}{\pgfqpoint{2.002067in}{1.916117in}}%
\pgfpathcurveto{\pgfqpoint{2.002067in}{1.924353in}}{\pgfqpoint{1.998794in}{1.932253in}}{\pgfqpoint{1.992970in}{1.938077in}}%
\pgfpathcurveto{\pgfqpoint{1.987147in}{1.943901in}}{\pgfqpoint{1.979246in}{1.947173in}}{\pgfqpoint{1.971010in}{1.947173in}}%
\pgfpathcurveto{\pgfqpoint{1.962774in}{1.947173in}}{\pgfqpoint{1.954874in}{1.943901in}}{\pgfqpoint{1.949050in}{1.938077in}}%
\pgfpathcurveto{\pgfqpoint{1.943226in}{1.932253in}}{\pgfqpoint{1.939954in}{1.924353in}}{\pgfqpoint{1.939954in}{1.916117in}}%
\pgfpathcurveto{\pgfqpoint{1.939954in}{1.907880in}}{\pgfqpoint{1.943226in}{1.899980in}}{\pgfqpoint{1.949050in}{1.894156in}}%
\pgfpathcurveto{\pgfqpoint{1.954874in}{1.888333in}}{\pgfqpoint{1.962774in}{1.885060in}}{\pgfqpoint{1.971010in}{1.885060in}}%
\pgfpathclose%
\pgfusepath{stroke,fill}%
\end{pgfscope}%
\begin{pgfscope}%
\pgfpathrectangle{\pgfqpoint{0.100000in}{0.212622in}}{\pgfqpoint{3.696000in}{3.696000in}}%
\pgfusepath{clip}%
\pgfsetbuttcap%
\pgfsetroundjoin%
\definecolor{currentfill}{rgb}{0.121569,0.466667,0.705882}%
\pgfsetfillcolor{currentfill}%
\pgfsetfillopacity{0.937489}%
\pgfsetlinewidth{1.003750pt}%
\definecolor{currentstroke}{rgb}{0.121569,0.466667,0.705882}%
\pgfsetstrokecolor{currentstroke}%
\pgfsetstrokeopacity{0.937489}%
\pgfsetdash{}{0pt}%
\pgfpathmoveto{\pgfqpoint{2.106484in}{1.936963in}}%
\pgfpathcurveto{\pgfqpoint{2.114721in}{1.936963in}}{\pgfqpoint{2.122621in}{1.940236in}}{\pgfqpoint{2.128445in}{1.946060in}}%
\pgfpathcurveto{\pgfqpoint{2.134269in}{1.951883in}}{\pgfqpoint{2.137541in}{1.959784in}}{\pgfqpoint{2.137541in}{1.968020in}}%
\pgfpathcurveto{\pgfqpoint{2.137541in}{1.976256in}}{\pgfqpoint{2.134269in}{1.984156in}}{\pgfqpoint{2.128445in}{1.989980in}}%
\pgfpathcurveto{\pgfqpoint{2.122621in}{1.995804in}}{\pgfqpoint{2.114721in}{1.999076in}}{\pgfqpoint{2.106484in}{1.999076in}}%
\pgfpathcurveto{\pgfqpoint{2.098248in}{1.999076in}}{\pgfqpoint{2.090348in}{1.995804in}}{\pgfqpoint{2.084524in}{1.989980in}}%
\pgfpathcurveto{\pgfqpoint{2.078700in}{1.984156in}}{\pgfqpoint{2.075428in}{1.976256in}}{\pgfqpoint{2.075428in}{1.968020in}}%
\pgfpathcurveto{\pgfqpoint{2.075428in}{1.959784in}}{\pgfqpoint{2.078700in}{1.951883in}}{\pgfqpoint{2.084524in}{1.946060in}}%
\pgfpathcurveto{\pgfqpoint{2.090348in}{1.940236in}}{\pgfqpoint{2.098248in}{1.936963in}}{\pgfqpoint{2.106484in}{1.936963in}}%
\pgfpathclose%
\pgfusepath{stroke,fill}%
\end{pgfscope}%
\begin{pgfscope}%
\pgfpathrectangle{\pgfqpoint{0.100000in}{0.212622in}}{\pgfqpoint{3.696000in}{3.696000in}}%
\pgfusepath{clip}%
\pgfsetbuttcap%
\pgfsetroundjoin%
\definecolor{currentfill}{rgb}{0.121569,0.466667,0.705882}%
\pgfsetfillcolor{currentfill}%
\pgfsetfillopacity{0.937558}%
\pgfsetlinewidth{1.003750pt}%
\definecolor{currentstroke}{rgb}{0.121569,0.466667,0.705882}%
\pgfsetstrokecolor{currentstroke}%
\pgfsetstrokeopacity{0.937558}%
\pgfsetdash{}{0pt}%
\pgfpathmoveto{\pgfqpoint{1.975482in}{1.883389in}}%
\pgfpathcurveto{\pgfqpoint{1.983718in}{1.883389in}}{\pgfqpoint{1.991618in}{1.886661in}}{\pgfqpoint{1.997442in}{1.892485in}}%
\pgfpathcurveto{\pgfqpoint{2.003266in}{1.898309in}}{\pgfqpoint{2.006539in}{1.906209in}}{\pgfqpoint{2.006539in}{1.914446in}}%
\pgfpathcurveto{\pgfqpoint{2.006539in}{1.922682in}}{\pgfqpoint{2.003266in}{1.930582in}}{\pgfqpoint{1.997442in}{1.936406in}}%
\pgfpathcurveto{\pgfqpoint{1.991618in}{1.942230in}}{\pgfqpoint{1.983718in}{1.945502in}}{\pgfqpoint{1.975482in}{1.945502in}}%
\pgfpathcurveto{\pgfqpoint{1.967246in}{1.945502in}}{\pgfqpoint{1.959346in}{1.942230in}}{\pgfqpoint{1.953522in}{1.936406in}}%
\pgfpathcurveto{\pgfqpoint{1.947698in}{1.930582in}}{\pgfqpoint{1.944426in}{1.922682in}}{\pgfqpoint{1.944426in}{1.914446in}}%
\pgfpathcurveto{\pgfqpoint{1.944426in}{1.906209in}}{\pgfqpoint{1.947698in}{1.898309in}}{\pgfqpoint{1.953522in}{1.892485in}}%
\pgfpathcurveto{\pgfqpoint{1.959346in}{1.886661in}}{\pgfqpoint{1.967246in}{1.883389in}}{\pgfqpoint{1.975482in}{1.883389in}}%
\pgfpathclose%
\pgfusepath{stroke,fill}%
\end{pgfscope}%
\begin{pgfscope}%
\pgfpathrectangle{\pgfqpoint{0.100000in}{0.212622in}}{\pgfqpoint{3.696000in}{3.696000in}}%
\pgfusepath{clip}%
\pgfsetbuttcap%
\pgfsetroundjoin%
\definecolor{currentfill}{rgb}{0.121569,0.466667,0.705882}%
\pgfsetfillcolor{currentfill}%
\pgfsetfillopacity{0.937606}%
\pgfsetlinewidth{1.003750pt}%
\definecolor{currentstroke}{rgb}{0.121569,0.466667,0.705882}%
\pgfsetstrokecolor{currentstroke}%
\pgfsetstrokeopacity{0.937606}%
\pgfsetdash{}{0pt}%
\pgfpathmoveto{\pgfqpoint{2.023671in}{1.006143in}}%
\pgfpathcurveto{\pgfqpoint{2.031907in}{1.006143in}}{\pgfqpoint{2.039807in}{1.009415in}}{\pgfqpoint{2.045631in}{1.015239in}}%
\pgfpathcurveto{\pgfqpoint{2.051455in}{1.021063in}}{\pgfqpoint{2.054727in}{1.028963in}}{\pgfqpoint{2.054727in}{1.037199in}}%
\pgfpathcurveto{\pgfqpoint{2.054727in}{1.045436in}}{\pgfqpoint{2.051455in}{1.053336in}}{\pgfqpoint{2.045631in}{1.059160in}}%
\pgfpathcurveto{\pgfqpoint{2.039807in}{1.064984in}}{\pgfqpoint{2.031907in}{1.068256in}}{\pgfqpoint{2.023671in}{1.068256in}}%
\pgfpathcurveto{\pgfqpoint{2.015434in}{1.068256in}}{\pgfqpoint{2.007534in}{1.064984in}}{\pgfqpoint{2.001710in}{1.059160in}}%
\pgfpathcurveto{\pgfqpoint{1.995886in}{1.053336in}}{\pgfqpoint{1.992614in}{1.045436in}}{\pgfqpoint{1.992614in}{1.037199in}}%
\pgfpathcurveto{\pgfqpoint{1.992614in}{1.028963in}}{\pgfqpoint{1.995886in}{1.021063in}}{\pgfqpoint{2.001710in}{1.015239in}}%
\pgfpathcurveto{\pgfqpoint{2.007534in}{1.009415in}}{\pgfqpoint{2.015434in}{1.006143in}}{\pgfqpoint{2.023671in}{1.006143in}}%
\pgfpathclose%
\pgfusepath{stroke,fill}%
\end{pgfscope}%
\begin{pgfscope}%
\pgfpathrectangle{\pgfqpoint{0.100000in}{0.212622in}}{\pgfqpoint{3.696000in}{3.696000in}}%
\pgfusepath{clip}%
\pgfsetbuttcap%
\pgfsetroundjoin%
\definecolor{currentfill}{rgb}{0.121569,0.466667,0.705882}%
\pgfsetfillcolor{currentfill}%
\pgfsetfillopacity{0.938140}%
\pgfsetlinewidth{1.003750pt}%
\definecolor{currentstroke}{rgb}{0.121569,0.466667,0.705882}%
\pgfsetstrokecolor{currentstroke}%
\pgfsetstrokeopacity{0.938140}%
\pgfsetdash{}{0pt}%
\pgfpathmoveto{\pgfqpoint{2.103982in}{1.933293in}}%
\pgfpathcurveto{\pgfqpoint{2.112218in}{1.933293in}}{\pgfqpoint{2.120118in}{1.936565in}}{\pgfqpoint{2.125942in}{1.942389in}}%
\pgfpathcurveto{\pgfqpoint{2.131766in}{1.948213in}}{\pgfqpoint{2.135038in}{1.956113in}}{\pgfqpoint{2.135038in}{1.964350in}}%
\pgfpathcurveto{\pgfqpoint{2.135038in}{1.972586in}}{\pgfqpoint{2.131766in}{1.980486in}}{\pgfqpoint{2.125942in}{1.986310in}}%
\pgfpathcurveto{\pgfqpoint{2.120118in}{1.992134in}}{\pgfqpoint{2.112218in}{1.995406in}}{\pgfqpoint{2.103982in}{1.995406in}}%
\pgfpathcurveto{\pgfqpoint{2.095746in}{1.995406in}}{\pgfqpoint{2.087846in}{1.992134in}}{\pgfqpoint{2.082022in}{1.986310in}}%
\pgfpathcurveto{\pgfqpoint{2.076198in}{1.980486in}}{\pgfqpoint{2.072925in}{1.972586in}}{\pgfqpoint{2.072925in}{1.964350in}}%
\pgfpathcurveto{\pgfqpoint{2.072925in}{1.956113in}}{\pgfqpoint{2.076198in}{1.948213in}}{\pgfqpoint{2.082022in}{1.942389in}}%
\pgfpathcurveto{\pgfqpoint{2.087846in}{1.936565in}}{\pgfqpoint{2.095746in}{1.933293in}}{\pgfqpoint{2.103982in}{1.933293in}}%
\pgfpathclose%
\pgfusepath{stroke,fill}%
\end{pgfscope}%
\begin{pgfscope}%
\pgfpathrectangle{\pgfqpoint{0.100000in}{0.212622in}}{\pgfqpoint{3.696000in}{3.696000in}}%
\pgfusepath{clip}%
\pgfsetbuttcap%
\pgfsetroundjoin%
\definecolor{currentfill}{rgb}{0.121569,0.466667,0.705882}%
\pgfsetfillcolor{currentfill}%
\pgfsetfillopacity{0.938310}%
\pgfsetlinewidth{1.003750pt}%
\definecolor{currentstroke}{rgb}{0.121569,0.466667,0.705882}%
\pgfsetstrokecolor{currentstroke}%
\pgfsetstrokeopacity{0.938310}%
\pgfsetdash{}{0pt}%
\pgfpathmoveto{\pgfqpoint{1.980296in}{1.882218in}}%
\pgfpathcurveto{\pgfqpoint{1.988532in}{1.882218in}}{\pgfqpoint{1.996432in}{1.885491in}}{\pgfqpoint{2.002256in}{1.891315in}}%
\pgfpathcurveto{\pgfqpoint{2.008080in}{1.897139in}}{\pgfqpoint{2.011353in}{1.905039in}}{\pgfqpoint{2.011353in}{1.913275in}}%
\pgfpathcurveto{\pgfqpoint{2.011353in}{1.921511in}}{\pgfqpoint{2.008080in}{1.929411in}}{\pgfqpoint{2.002256in}{1.935235in}}%
\pgfpathcurveto{\pgfqpoint{1.996432in}{1.941059in}}{\pgfqpoint{1.988532in}{1.944331in}}{\pgfqpoint{1.980296in}{1.944331in}}%
\pgfpathcurveto{\pgfqpoint{1.972060in}{1.944331in}}{\pgfqpoint{1.964160in}{1.941059in}}{\pgfqpoint{1.958336in}{1.935235in}}%
\pgfpathcurveto{\pgfqpoint{1.952512in}{1.929411in}}{\pgfqpoint{1.949240in}{1.921511in}}{\pgfqpoint{1.949240in}{1.913275in}}%
\pgfpathcurveto{\pgfqpoint{1.949240in}{1.905039in}}{\pgfqpoint{1.952512in}{1.897139in}}{\pgfqpoint{1.958336in}{1.891315in}}%
\pgfpathcurveto{\pgfqpoint{1.964160in}{1.885491in}}{\pgfqpoint{1.972060in}{1.882218in}}{\pgfqpoint{1.980296in}{1.882218in}}%
\pgfpathclose%
\pgfusepath{stroke,fill}%
\end{pgfscope}%
\begin{pgfscope}%
\pgfpathrectangle{\pgfqpoint{0.100000in}{0.212622in}}{\pgfqpoint{3.696000in}{3.696000in}}%
\pgfusepath{clip}%
\pgfsetbuttcap%
\pgfsetroundjoin%
\definecolor{currentfill}{rgb}{0.121569,0.466667,0.705882}%
\pgfsetfillcolor{currentfill}%
\pgfsetfillopacity{0.938337}%
\pgfsetlinewidth{1.003750pt}%
\definecolor{currentstroke}{rgb}{0.121569,0.466667,0.705882}%
\pgfsetstrokecolor{currentstroke}%
\pgfsetstrokeopacity{0.938337}%
\pgfsetdash{}{0pt}%
\pgfpathmoveto{\pgfqpoint{2.028096in}{1.007068in}}%
\pgfpathcurveto{\pgfqpoint{2.036332in}{1.007068in}}{\pgfqpoint{2.044232in}{1.010340in}}{\pgfqpoint{2.050056in}{1.016164in}}%
\pgfpathcurveto{\pgfqpoint{2.055880in}{1.021988in}}{\pgfqpoint{2.059152in}{1.029888in}}{\pgfqpoint{2.059152in}{1.038124in}}%
\pgfpathcurveto{\pgfqpoint{2.059152in}{1.046361in}}{\pgfqpoint{2.055880in}{1.054261in}}{\pgfqpoint{2.050056in}{1.060085in}}%
\pgfpathcurveto{\pgfqpoint{2.044232in}{1.065909in}}{\pgfqpoint{2.036332in}{1.069181in}}{\pgfqpoint{2.028096in}{1.069181in}}%
\pgfpathcurveto{\pgfqpoint{2.019859in}{1.069181in}}{\pgfqpoint{2.011959in}{1.065909in}}{\pgfqpoint{2.006135in}{1.060085in}}%
\pgfpathcurveto{\pgfqpoint{2.000311in}{1.054261in}}{\pgfqpoint{1.997039in}{1.046361in}}{\pgfqpoint{1.997039in}{1.038124in}}%
\pgfpathcurveto{\pgfqpoint{1.997039in}{1.029888in}}{\pgfqpoint{2.000311in}{1.021988in}}{\pgfqpoint{2.006135in}{1.016164in}}%
\pgfpathcurveto{\pgfqpoint{2.011959in}{1.010340in}}{\pgfqpoint{2.019859in}{1.007068in}}{\pgfqpoint{2.028096in}{1.007068in}}%
\pgfpathclose%
\pgfusepath{stroke,fill}%
\end{pgfscope}%
\begin{pgfscope}%
\pgfpathrectangle{\pgfqpoint{0.100000in}{0.212622in}}{\pgfqpoint{3.696000in}{3.696000in}}%
\pgfusepath{clip}%
\pgfsetbuttcap%
\pgfsetroundjoin%
\definecolor{currentfill}{rgb}{0.121569,0.466667,0.705882}%
\pgfsetfillcolor{currentfill}%
\pgfsetfillopacity{0.938395}%
\pgfsetlinewidth{1.003750pt}%
\definecolor{currentstroke}{rgb}{0.121569,0.466667,0.705882}%
\pgfsetstrokecolor{currentstroke}%
\pgfsetstrokeopacity{0.938395}%
\pgfsetdash{}{0pt}%
\pgfpathmoveto{\pgfqpoint{2.699771in}{1.146760in}}%
\pgfpathcurveto{\pgfqpoint{2.708007in}{1.146760in}}{\pgfqpoint{2.715907in}{1.150033in}}{\pgfqpoint{2.721731in}{1.155856in}}%
\pgfpathcurveto{\pgfqpoint{2.727555in}{1.161680in}}{\pgfqpoint{2.730827in}{1.169580in}}{\pgfqpoint{2.730827in}{1.177817in}}%
\pgfpathcurveto{\pgfqpoint{2.730827in}{1.186053in}}{\pgfqpoint{2.727555in}{1.193953in}}{\pgfqpoint{2.721731in}{1.199777in}}%
\pgfpathcurveto{\pgfqpoint{2.715907in}{1.205601in}}{\pgfqpoint{2.708007in}{1.208873in}}{\pgfqpoint{2.699771in}{1.208873in}}%
\pgfpathcurveto{\pgfqpoint{2.691535in}{1.208873in}}{\pgfqpoint{2.683635in}{1.205601in}}{\pgfqpoint{2.677811in}{1.199777in}}%
\pgfpathcurveto{\pgfqpoint{2.671987in}{1.193953in}}{\pgfqpoint{2.668714in}{1.186053in}}{\pgfqpoint{2.668714in}{1.177817in}}%
\pgfpathcurveto{\pgfqpoint{2.668714in}{1.169580in}}{\pgfqpoint{2.671987in}{1.161680in}}{\pgfqpoint{2.677811in}{1.155856in}}%
\pgfpathcurveto{\pgfqpoint{2.683635in}{1.150033in}}{\pgfqpoint{2.691535in}{1.146760in}}{\pgfqpoint{2.699771in}{1.146760in}}%
\pgfpathclose%
\pgfusepath{stroke,fill}%
\end{pgfscope}%
\begin{pgfscope}%
\pgfpathrectangle{\pgfqpoint{0.100000in}{0.212622in}}{\pgfqpoint{3.696000in}{3.696000in}}%
\pgfusepath{clip}%
\pgfsetbuttcap%
\pgfsetroundjoin%
\definecolor{currentfill}{rgb}{0.121569,0.466667,0.705882}%
\pgfsetfillcolor{currentfill}%
\pgfsetfillopacity{0.938698}%
\pgfsetlinewidth{1.003750pt}%
\definecolor{currentstroke}{rgb}{0.121569,0.466667,0.705882}%
\pgfsetstrokecolor{currentstroke}%
\pgfsetstrokeopacity{0.938698}%
\pgfsetdash{}{0pt}%
\pgfpathmoveto{\pgfqpoint{1.983067in}{1.881968in}}%
\pgfpathcurveto{\pgfqpoint{1.991304in}{1.881968in}}{\pgfqpoint{1.999204in}{1.885240in}}{\pgfqpoint{2.005028in}{1.891064in}}%
\pgfpathcurveto{\pgfqpoint{2.010852in}{1.896888in}}{\pgfqpoint{2.014124in}{1.904788in}}{\pgfqpoint{2.014124in}{1.913024in}}%
\pgfpathcurveto{\pgfqpoint{2.014124in}{1.921261in}}{\pgfqpoint{2.010852in}{1.929161in}}{\pgfqpoint{2.005028in}{1.934985in}}%
\pgfpathcurveto{\pgfqpoint{1.999204in}{1.940809in}}{\pgfqpoint{1.991304in}{1.944081in}}{\pgfqpoint{1.983067in}{1.944081in}}%
\pgfpathcurveto{\pgfqpoint{1.974831in}{1.944081in}}{\pgfqpoint{1.966931in}{1.940809in}}{\pgfqpoint{1.961107in}{1.934985in}}%
\pgfpathcurveto{\pgfqpoint{1.955283in}{1.929161in}}{\pgfqpoint{1.952011in}{1.921261in}}{\pgfqpoint{1.952011in}{1.913024in}}%
\pgfpathcurveto{\pgfqpoint{1.952011in}{1.904788in}}{\pgfqpoint{1.955283in}{1.896888in}}{\pgfqpoint{1.961107in}{1.891064in}}%
\pgfpathcurveto{\pgfqpoint{1.966931in}{1.885240in}}{\pgfqpoint{1.974831in}{1.881968in}}{\pgfqpoint{1.983067in}{1.881968in}}%
\pgfpathclose%
\pgfusepath{stroke,fill}%
\end{pgfscope}%
\begin{pgfscope}%
\pgfpathrectangle{\pgfqpoint{0.100000in}{0.212622in}}{\pgfqpoint{3.696000in}{3.696000in}}%
\pgfusepath{clip}%
\pgfsetbuttcap%
\pgfsetroundjoin%
\definecolor{currentfill}{rgb}{0.121569,0.466667,0.705882}%
\pgfsetfillcolor{currentfill}%
\pgfsetfillopacity{0.938725}%
\pgfsetlinewidth{1.003750pt}%
\definecolor{currentstroke}{rgb}{0.121569,0.466667,0.705882}%
\pgfsetstrokecolor{currentstroke}%
\pgfsetstrokeopacity{0.938725}%
\pgfsetdash{}{0pt}%
\pgfpathmoveto{\pgfqpoint{2.102829in}{1.930361in}}%
\pgfpathcurveto{\pgfqpoint{2.111065in}{1.930361in}}{\pgfqpoint{2.118965in}{1.933633in}}{\pgfqpoint{2.124789in}{1.939457in}}%
\pgfpathcurveto{\pgfqpoint{2.130613in}{1.945281in}}{\pgfqpoint{2.133886in}{1.953181in}}{\pgfqpoint{2.133886in}{1.961418in}}%
\pgfpathcurveto{\pgfqpoint{2.133886in}{1.969654in}}{\pgfqpoint{2.130613in}{1.977554in}}{\pgfqpoint{2.124789in}{1.983378in}}%
\pgfpathcurveto{\pgfqpoint{2.118965in}{1.989202in}}{\pgfqpoint{2.111065in}{1.992474in}}{\pgfqpoint{2.102829in}{1.992474in}}%
\pgfpathcurveto{\pgfqpoint{2.094593in}{1.992474in}}{\pgfqpoint{2.086693in}{1.989202in}}{\pgfqpoint{2.080869in}{1.983378in}}%
\pgfpathcurveto{\pgfqpoint{2.075045in}{1.977554in}}{\pgfqpoint{2.071773in}{1.969654in}}{\pgfqpoint{2.071773in}{1.961418in}}%
\pgfpathcurveto{\pgfqpoint{2.071773in}{1.953181in}}{\pgfqpoint{2.075045in}{1.945281in}}{\pgfqpoint{2.080869in}{1.939457in}}%
\pgfpathcurveto{\pgfqpoint{2.086693in}{1.933633in}}{\pgfqpoint{2.094593in}{1.930361in}}{\pgfqpoint{2.102829in}{1.930361in}}%
\pgfpathclose%
\pgfusepath{stroke,fill}%
\end{pgfscope}%
\begin{pgfscope}%
\pgfpathrectangle{\pgfqpoint{0.100000in}{0.212622in}}{\pgfqpoint{3.696000in}{3.696000in}}%
\pgfusepath{clip}%
\pgfsetbuttcap%
\pgfsetroundjoin%
\definecolor{currentfill}{rgb}{0.121569,0.466667,0.705882}%
\pgfsetfillcolor{currentfill}%
\pgfsetfillopacity{0.939155}%
\pgfsetlinewidth{1.003750pt}%
\definecolor{currentstroke}{rgb}{0.121569,0.466667,0.705882}%
\pgfsetstrokecolor{currentstroke}%
\pgfsetstrokeopacity{0.939155}%
\pgfsetdash{}{0pt}%
\pgfpathmoveto{\pgfqpoint{2.035672in}{1.005557in}}%
\pgfpathcurveto{\pgfqpoint{2.043908in}{1.005557in}}{\pgfqpoint{2.051808in}{1.008830in}}{\pgfqpoint{2.057632in}{1.014653in}}%
\pgfpathcurveto{\pgfqpoint{2.063456in}{1.020477in}}{\pgfqpoint{2.066729in}{1.028377in}}{\pgfqpoint{2.066729in}{1.036614in}}%
\pgfpathcurveto{\pgfqpoint{2.066729in}{1.044850in}}{\pgfqpoint{2.063456in}{1.052750in}}{\pgfqpoint{2.057632in}{1.058574in}}%
\pgfpathcurveto{\pgfqpoint{2.051808in}{1.064398in}}{\pgfqpoint{2.043908in}{1.067670in}}{\pgfqpoint{2.035672in}{1.067670in}}%
\pgfpathcurveto{\pgfqpoint{2.027436in}{1.067670in}}{\pgfqpoint{2.019536in}{1.064398in}}{\pgfqpoint{2.013712in}{1.058574in}}%
\pgfpathcurveto{\pgfqpoint{2.007888in}{1.052750in}}{\pgfqpoint{2.004616in}{1.044850in}}{\pgfqpoint{2.004616in}{1.036614in}}%
\pgfpathcurveto{\pgfqpoint{2.004616in}{1.028377in}}{\pgfqpoint{2.007888in}{1.020477in}}{\pgfqpoint{2.013712in}{1.014653in}}%
\pgfpathcurveto{\pgfqpoint{2.019536in}{1.008830in}}{\pgfqpoint{2.027436in}{1.005557in}}{\pgfqpoint{2.035672in}{1.005557in}}%
\pgfpathclose%
\pgfusepath{stroke,fill}%
\end{pgfscope}%
\begin{pgfscope}%
\pgfpathrectangle{\pgfqpoint{0.100000in}{0.212622in}}{\pgfqpoint{3.696000in}{3.696000in}}%
\pgfusepath{clip}%
\pgfsetbuttcap%
\pgfsetroundjoin%
\definecolor{currentfill}{rgb}{0.121569,0.466667,0.705882}%
\pgfsetfillcolor{currentfill}%
\pgfsetfillopacity{0.939179}%
\pgfsetlinewidth{1.003750pt}%
\definecolor{currentstroke}{rgb}{0.121569,0.466667,0.705882}%
\pgfsetstrokecolor{currentstroke}%
\pgfsetstrokeopacity{0.939179}%
\pgfsetdash{}{0pt}%
\pgfpathmoveto{\pgfqpoint{2.101418in}{1.928257in}}%
\pgfpathcurveto{\pgfqpoint{2.109654in}{1.928257in}}{\pgfqpoint{2.117554in}{1.931529in}}{\pgfqpoint{2.123378in}{1.937353in}}%
\pgfpathcurveto{\pgfqpoint{2.129202in}{1.943177in}}{\pgfqpoint{2.132475in}{1.951077in}}{\pgfqpoint{2.132475in}{1.959313in}}%
\pgfpathcurveto{\pgfqpoint{2.132475in}{1.967549in}}{\pgfqpoint{2.129202in}{1.975450in}}{\pgfqpoint{2.123378in}{1.981273in}}%
\pgfpathcurveto{\pgfqpoint{2.117554in}{1.987097in}}{\pgfqpoint{2.109654in}{1.990370in}}{\pgfqpoint{2.101418in}{1.990370in}}%
\pgfpathcurveto{\pgfqpoint{2.093182in}{1.990370in}}{\pgfqpoint{2.085282in}{1.987097in}}{\pgfqpoint{2.079458in}{1.981273in}}%
\pgfpathcurveto{\pgfqpoint{2.073634in}{1.975450in}}{\pgfqpoint{2.070362in}{1.967549in}}{\pgfqpoint{2.070362in}{1.959313in}}%
\pgfpathcurveto{\pgfqpoint{2.070362in}{1.951077in}}{\pgfqpoint{2.073634in}{1.943177in}}{\pgfqpoint{2.079458in}{1.937353in}}%
\pgfpathcurveto{\pgfqpoint{2.085282in}{1.931529in}}{\pgfqpoint{2.093182in}{1.928257in}}{\pgfqpoint{2.101418in}{1.928257in}}%
\pgfpathclose%
\pgfusepath{stroke,fill}%
\end{pgfscope}%
\begin{pgfscope}%
\pgfpathrectangle{\pgfqpoint{0.100000in}{0.212622in}}{\pgfqpoint{3.696000in}{3.696000in}}%
\pgfusepath{clip}%
\pgfsetbuttcap%
\pgfsetroundjoin%
\definecolor{currentfill}{rgb}{0.121569,0.466667,0.705882}%
\pgfsetfillcolor{currentfill}%
\pgfsetfillopacity{0.939184}%
\pgfsetlinewidth{1.003750pt}%
\definecolor{currentstroke}{rgb}{0.121569,0.466667,0.705882}%
\pgfsetstrokecolor{currentstroke}%
\pgfsetstrokeopacity{0.939184}%
\pgfsetdash{}{0pt}%
\pgfpathmoveto{\pgfqpoint{1.986646in}{1.881008in}}%
\pgfpathcurveto{\pgfqpoint{1.994883in}{1.881008in}}{\pgfqpoint{2.002783in}{1.884280in}}{\pgfqpoint{2.008607in}{1.890104in}}%
\pgfpathcurveto{\pgfqpoint{2.014430in}{1.895928in}}{\pgfqpoint{2.017703in}{1.903828in}}{\pgfqpoint{2.017703in}{1.912065in}}%
\pgfpathcurveto{\pgfqpoint{2.017703in}{1.920301in}}{\pgfqpoint{2.014430in}{1.928201in}}{\pgfqpoint{2.008607in}{1.934025in}}%
\pgfpathcurveto{\pgfqpoint{2.002783in}{1.939849in}}{\pgfqpoint{1.994883in}{1.943121in}}{\pgfqpoint{1.986646in}{1.943121in}}%
\pgfpathcurveto{\pgfqpoint{1.978410in}{1.943121in}}{\pgfqpoint{1.970510in}{1.939849in}}{\pgfqpoint{1.964686in}{1.934025in}}%
\pgfpathcurveto{\pgfqpoint{1.958862in}{1.928201in}}{\pgfqpoint{1.955590in}{1.920301in}}{\pgfqpoint{1.955590in}{1.912065in}}%
\pgfpathcurveto{\pgfqpoint{1.955590in}{1.903828in}}{\pgfqpoint{1.958862in}{1.895928in}}{\pgfqpoint{1.964686in}{1.890104in}}%
\pgfpathcurveto{\pgfqpoint{1.970510in}{1.884280in}}{\pgfqpoint{1.978410in}{1.881008in}}{\pgfqpoint{1.986646in}{1.881008in}}%
\pgfpathclose%
\pgfusepath{stroke,fill}%
\end{pgfscope}%
\begin{pgfscope}%
\pgfpathrectangle{\pgfqpoint{0.100000in}{0.212622in}}{\pgfqpoint{3.696000in}{3.696000in}}%
\pgfusepath{clip}%
\pgfsetbuttcap%
\pgfsetroundjoin%
\definecolor{currentfill}{rgb}{0.121569,0.466667,0.705882}%
\pgfsetfillcolor{currentfill}%
\pgfsetfillopacity{0.939583}%
\pgfsetlinewidth{1.003750pt}%
\definecolor{currentstroke}{rgb}{0.121569,0.466667,0.705882}%
\pgfsetstrokecolor{currentstroke}%
\pgfsetstrokeopacity{0.939583}%
\pgfsetdash{}{0pt}%
\pgfpathmoveto{\pgfqpoint{2.100705in}{1.926391in}}%
\pgfpathcurveto{\pgfqpoint{2.108941in}{1.926391in}}{\pgfqpoint{2.116841in}{1.929664in}}{\pgfqpoint{2.122665in}{1.935488in}}%
\pgfpathcurveto{\pgfqpoint{2.128489in}{1.941312in}}{\pgfqpoint{2.131761in}{1.949212in}}{\pgfqpoint{2.131761in}{1.957448in}}%
\pgfpathcurveto{\pgfqpoint{2.131761in}{1.965684in}}{\pgfqpoint{2.128489in}{1.973584in}}{\pgfqpoint{2.122665in}{1.979408in}}%
\pgfpathcurveto{\pgfqpoint{2.116841in}{1.985232in}}{\pgfqpoint{2.108941in}{1.988504in}}{\pgfqpoint{2.100705in}{1.988504in}}%
\pgfpathcurveto{\pgfqpoint{2.092468in}{1.988504in}}{\pgfqpoint{2.084568in}{1.985232in}}{\pgfqpoint{2.078744in}{1.979408in}}%
\pgfpathcurveto{\pgfqpoint{2.072920in}{1.973584in}}{\pgfqpoint{2.069648in}{1.965684in}}{\pgfqpoint{2.069648in}{1.957448in}}%
\pgfpathcurveto{\pgfqpoint{2.069648in}{1.949212in}}{\pgfqpoint{2.072920in}{1.941312in}}{\pgfqpoint{2.078744in}{1.935488in}}%
\pgfpathcurveto{\pgfqpoint{2.084568in}{1.929664in}}{\pgfqpoint{2.092468in}{1.926391in}}{\pgfqpoint{2.100705in}{1.926391in}}%
\pgfpathclose%
\pgfusepath{stroke,fill}%
\end{pgfscope}%
\begin{pgfscope}%
\pgfpathrectangle{\pgfqpoint{0.100000in}{0.212622in}}{\pgfqpoint{3.696000in}{3.696000in}}%
\pgfusepath{clip}%
\pgfsetbuttcap%
\pgfsetroundjoin%
\definecolor{currentfill}{rgb}{0.121569,0.466667,0.705882}%
\pgfsetfillcolor{currentfill}%
\pgfsetfillopacity{0.939834}%
\pgfsetlinewidth{1.003750pt}%
\definecolor{currentstroke}{rgb}{0.121569,0.466667,0.705882}%
\pgfsetstrokecolor{currentstroke}%
\pgfsetstrokeopacity{0.939834}%
\pgfsetdash{}{0pt}%
\pgfpathmoveto{\pgfqpoint{2.099832in}{1.924928in}}%
\pgfpathcurveto{\pgfqpoint{2.108068in}{1.924928in}}{\pgfqpoint{2.115968in}{1.928201in}}{\pgfqpoint{2.121792in}{1.934025in}}%
\pgfpathcurveto{\pgfqpoint{2.127616in}{1.939849in}}{\pgfqpoint{2.130888in}{1.947749in}}{\pgfqpoint{2.130888in}{1.955985in}}%
\pgfpathcurveto{\pgfqpoint{2.130888in}{1.964221in}}{\pgfqpoint{2.127616in}{1.972121in}}{\pgfqpoint{2.121792in}{1.977945in}}%
\pgfpathcurveto{\pgfqpoint{2.115968in}{1.983769in}}{\pgfqpoint{2.108068in}{1.987041in}}{\pgfqpoint{2.099832in}{1.987041in}}%
\pgfpathcurveto{\pgfqpoint{2.091595in}{1.987041in}}{\pgfqpoint{2.083695in}{1.983769in}}{\pgfqpoint{2.077871in}{1.977945in}}%
\pgfpathcurveto{\pgfqpoint{2.072047in}{1.972121in}}{\pgfqpoint{2.068775in}{1.964221in}}{\pgfqpoint{2.068775in}{1.955985in}}%
\pgfpathcurveto{\pgfqpoint{2.068775in}{1.947749in}}{\pgfqpoint{2.072047in}{1.939849in}}{\pgfqpoint{2.077871in}{1.934025in}}%
\pgfpathcurveto{\pgfqpoint{2.083695in}{1.928201in}}{\pgfqpoint{2.091595in}{1.924928in}}{\pgfqpoint{2.099832in}{1.924928in}}%
\pgfpathclose%
\pgfusepath{stroke,fill}%
\end{pgfscope}%
\begin{pgfscope}%
\pgfpathrectangle{\pgfqpoint{0.100000in}{0.212622in}}{\pgfqpoint{3.696000in}{3.696000in}}%
\pgfusepath{clip}%
\pgfsetbuttcap%
\pgfsetroundjoin%
\definecolor{currentfill}{rgb}{0.121569,0.466667,0.705882}%
\pgfsetfillcolor{currentfill}%
\pgfsetfillopacity{0.939873}%
\pgfsetlinewidth{1.003750pt}%
\definecolor{currentstroke}{rgb}{0.121569,0.466667,0.705882}%
\pgfsetstrokecolor{currentstroke}%
\pgfsetstrokeopacity{0.939873}%
\pgfsetdash{}{0pt}%
\pgfpathmoveto{\pgfqpoint{1.991438in}{1.880475in}}%
\pgfpathcurveto{\pgfqpoint{1.999675in}{1.880475in}}{\pgfqpoint{2.007575in}{1.883747in}}{\pgfqpoint{2.013399in}{1.889571in}}%
\pgfpathcurveto{\pgfqpoint{2.019223in}{1.895395in}}{\pgfqpoint{2.022495in}{1.903295in}}{\pgfqpoint{2.022495in}{1.911531in}}%
\pgfpathcurveto{\pgfqpoint{2.022495in}{1.919768in}}{\pgfqpoint{2.019223in}{1.927668in}}{\pgfqpoint{2.013399in}{1.933492in}}%
\pgfpathcurveto{\pgfqpoint{2.007575in}{1.939316in}}{\pgfqpoint{1.999675in}{1.942588in}}{\pgfqpoint{1.991438in}{1.942588in}}%
\pgfpathcurveto{\pgfqpoint{1.983202in}{1.942588in}}{\pgfqpoint{1.975302in}{1.939316in}}{\pgfqpoint{1.969478in}{1.933492in}}%
\pgfpathcurveto{\pgfqpoint{1.963654in}{1.927668in}}{\pgfqpoint{1.960382in}{1.919768in}}{\pgfqpoint{1.960382in}{1.911531in}}%
\pgfpathcurveto{\pgfqpoint{1.960382in}{1.903295in}}{\pgfqpoint{1.963654in}{1.895395in}}{\pgfqpoint{1.969478in}{1.889571in}}%
\pgfpathcurveto{\pgfqpoint{1.975302in}{1.883747in}}{\pgfqpoint{1.983202in}{1.880475in}}{\pgfqpoint{1.991438in}{1.880475in}}%
\pgfpathclose%
\pgfusepath{stroke,fill}%
\end{pgfscope}%
\begin{pgfscope}%
\pgfpathrectangle{\pgfqpoint{0.100000in}{0.212622in}}{\pgfqpoint{3.696000in}{3.696000in}}%
\pgfusepath{clip}%
\pgfsetbuttcap%
\pgfsetroundjoin%
\definecolor{currentfill}{rgb}{0.121569,0.466667,0.705882}%
\pgfsetfillcolor{currentfill}%
\pgfsetfillopacity{0.939908}%
\pgfsetlinewidth{1.003750pt}%
\definecolor{currentstroke}{rgb}{0.121569,0.466667,0.705882}%
\pgfsetstrokecolor{currentstroke}%
\pgfsetstrokeopacity{0.939908}%
\pgfsetdash{}{0pt}%
\pgfpathmoveto{\pgfqpoint{2.045013in}{0.999171in}}%
\pgfpathcurveto{\pgfqpoint{2.053249in}{0.999171in}}{\pgfqpoint{2.061150in}{1.002444in}}{\pgfqpoint{2.066973in}{1.008268in}}%
\pgfpathcurveto{\pgfqpoint{2.072797in}{1.014092in}}{\pgfqpoint{2.076070in}{1.021992in}}{\pgfqpoint{2.076070in}{1.030228in}}%
\pgfpathcurveto{\pgfqpoint{2.076070in}{1.038464in}}{\pgfqpoint{2.072797in}{1.046364in}}{\pgfqpoint{2.066973in}{1.052188in}}%
\pgfpathcurveto{\pgfqpoint{2.061150in}{1.058012in}}{\pgfqpoint{2.053249in}{1.061284in}}{\pgfqpoint{2.045013in}{1.061284in}}%
\pgfpathcurveto{\pgfqpoint{2.036777in}{1.061284in}}{\pgfqpoint{2.028877in}{1.058012in}}{\pgfqpoint{2.023053in}{1.052188in}}%
\pgfpathcurveto{\pgfqpoint{2.017229in}{1.046364in}}{\pgfqpoint{2.013957in}{1.038464in}}{\pgfqpoint{2.013957in}{1.030228in}}%
\pgfpathcurveto{\pgfqpoint{2.013957in}{1.021992in}}{\pgfqpoint{2.017229in}{1.014092in}}{\pgfqpoint{2.023053in}{1.008268in}}%
\pgfpathcurveto{\pgfqpoint{2.028877in}{1.002444in}}{\pgfqpoint{2.036777in}{0.999171in}}{\pgfqpoint{2.045013in}{0.999171in}}%
\pgfpathclose%
\pgfusepath{stroke,fill}%
\end{pgfscope}%
\begin{pgfscope}%
\pgfpathrectangle{\pgfqpoint{0.100000in}{0.212622in}}{\pgfqpoint{3.696000in}{3.696000in}}%
\pgfusepath{clip}%
\pgfsetbuttcap%
\pgfsetroundjoin%
\definecolor{currentfill}{rgb}{0.121569,0.466667,0.705882}%
\pgfsetfillcolor{currentfill}%
\pgfsetfillopacity{0.940383}%
\pgfsetlinewidth{1.003750pt}%
\definecolor{currentstroke}{rgb}{0.121569,0.466667,0.705882}%
\pgfsetstrokecolor{currentstroke}%
\pgfsetstrokeopacity{0.940383}%
\pgfsetdash{}{0pt}%
\pgfpathmoveto{\pgfqpoint{2.098715in}{1.922172in}}%
\pgfpathcurveto{\pgfqpoint{2.106951in}{1.922172in}}{\pgfqpoint{2.114851in}{1.925445in}}{\pgfqpoint{2.120675in}{1.931269in}}%
\pgfpathcurveto{\pgfqpoint{2.126499in}{1.937093in}}{\pgfqpoint{2.129771in}{1.944993in}}{\pgfqpoint{2.129771in}{1.953229in}}%
\pgfpathcurveto{\pgfqpoint{2.129771in}{1.961465in}}{\pgfqpoint{2.126499in}{1.969365in}}{\pgfqpoint{2.120675in}{1.975189in}}%
\pgfpathcurveto{\pgfqpoint{2.114851in}{1.981013in}}{\pgfqpoint{2.106951in}{1.984285in}}{\pgfqpoint{2.098715in}{1.984285in}}%
\pgfpathcurveto{\pgfqpoint{2.090479in}{1.984285in}}{\pgfqpoint{2.082579in}{1.981013in}}{\pgfqpoint{2.076755in}{1.975189in}}%
\pgfpathcurveto{\pgfqpoint{2.070931in}{1.969365in}}{\pgfqpoint{2.067658in}{1.961465in}}{\pgfqpoint{2.067658in}{1.953229in}}%
\pgfpathcurveto{\pgfqpoint{2.067658in}{1.944993in}}{\pgfqpoint{2.070931in}{1.937093in}}{\pgfqpoint{2.076755in}{1.931269in}}%
\pgfpathcurveto{\pgfqpoint{2.082579in}{1.925445in}}{\pgfqpoint{2.090479in}{1.922172in}}{\pgfqpoint{2.098715in}{1.922172in}}%
\pgfpathclose%
\pgfusepath{stroke,fill}%
\end{pgfscope}%
\begin{pgfscope}%
\pgfpathrectangle{\pgfqpoint{0.100000in}{0.212622in}}{\pgfqpoint{3.696000in}{3.696000in}}%
\pgfusepath{clip}%
\pgfsetbuttcap%
\pgfsetroundjoin%
\definecolor{currentfill}{rgb}{0.121569,0.466667,0.705882}%
\pgfsetfillcolor{currentfill}%
\pgfsetfillopacity{0.940523}%
\pgfsetlinewidth{1.003750pt}%
\definecolor{currentstroke}{rgb}{0.121569,0.466667,0.705882}%
\pgfsetstrokecolor{currentstroke}%
\pgfsetstrokeopacity{0.940523}%
\pgfsetdash{}{0pt}%
\pgfpathmoveto{\pgfqpoint{2.054733in}{0.990463in}}%
\pgfpathcurveto{\pgfqpoint{2.062970in}{0.990463in}}{\pgfqpoint{2.070870in}{0.993735in}}{\pgfqpoint{2.076693in}{0.999559in}}%
\pgfpathcurveto{\pgfqpoint{2.082517in}{1.005383in}}{\pgfqpoint{2.085790in}{1.013283in}}{\pgfqpoint{2.085790in}{1.021519in}}%
\pgfpathcurveto{\pgfqpoint{2.085790in}{1.029756in}}{\pgfqpoint{2.082517in}{1.037656in}}{\pgfqpoint{2.076693in}{1.043480in}}%
\pgfpathcurveto{\pgfqpoint{2.070870in}{1.049304in}}{\pgfqpoint{2.062970in}{1.052576in}}{\pgfqpoint{2.054733in}{1.052576in}}%
\pgfpathcurveto{\pgfqpoint{2.046497in}{1.052576in}}{\pgfqpoint{2.038597in}{1.049304in}}{\pgfqpoint{2.032773in}{1.043480in}}%
\pgfpathcurveto{\pgfqpoint{2.026949in}{1.037656in}}{\pgfqpoint{2.023677in}{1.029756in}}{\pgfqpoint{2.023677in}{1.021519in}}%
\pgfpathcurveto{\pgfqpoint{2.023677in}{1.013283in}}{\pgfqpoint{2.026949in}{1.005383in}}{\pgfqpoint{2.032773in}{0.999559in}}%
\pgfpathcurveto{\pgfqpoint{2.038597in}{0.993735in}}{\pgfqpoint{2.046497in}{0.990463in}}{\pgfqpoint{2.054733in}{0.990463in}}%
\pgfpathclose%
\pgfusepath{stroke,fill}%
\end{pgfscope}%
\begin{pgfscope}%
\pgfpathrectangle{\pgfqpoint{0.100000in}{0.212622in}}{\pgfqpoint{3.696000in}{3.696000in}}%
\pgfusepath{clip}%
\pgfsetbuttcap%
\pgfsetroundjoin%
\definecolor{currentfill}{rgb}{0.121569,0.466667,0.705882}%
\pgfsetfillcolor{currentfill}%
\pgfsetfillopacity{0.940734}%
\pgfsetlinewidth{1.003750pt}%
\definecolor{currentstroke}{rgb}{0.121569,0.466667,0.705882}%
\pgfsetstrokecolor{currentstroke}%
\pgfsetstrokeopacity{0.940734}%
\pgfsetdash{}{0pt}%
\pgfpathmoveto{\pgfqpoint{2.065499in}{0.979575in}}%
\pgfpathcurveto{\pgfqpoint{2.073735in}{0.979575in}}{\pgfqpoint{2.081635in}{0.982847in}}{\pgfqpoint{2.087459in}{0.988671in}}%
\pgfpathcurveto{\pgfqpoint{2.093283in}{0.994495in}}{\pgfqpoint{2.096556in}{1.002395in}}{\pgfqpoint{2.096556in}{1.010631in}}%
\pgfpathcurveto{\pgfqpoint{2.096556in}{1.018868in}}{\pgfqpoint{2.093283in}{1.026768in}}{\pgfqpoint{2.087459in}{1.032592in}}%
\pgfpathcurveto{\pgfqpoint{2.081635in}{1.038415in}}{\pgfqpoint{2.073735in}{1.041688in}}{\pgfqpoint{2.065499in}{1.041688in}}%
\pgfpathcurveto{\pgfqpoint{2.057263in}{1.041688in}}{\pgfqpoint{2.049363in}{1.038415in}}{\pgfqpoint{2.043539in}{1.032592in}}%
\pgfpathcurveto{\pgfqpoint{2.037715in}{1.026768in}}{\pgfqpoint{2.034443in}{1.018868in}}{\pgfqpoint{2.034443in}{1.010631in}}%
\pgfpathcurveto{\pgfqpoint{2.034443in}{1.002395in}}{\pgfqpoint{2.037715in}{0.994495in}}{\pgfqpoint{2.043539in}{0.988671in}}%
\pgfpathcurveto{\pgfqpoint{2.049363in}{0.982847in}}{\pgfqpoint{2.057263in}{0.979575in}}{\pgfqpoint{2.065499in}{0.979575in}}%
\pgfpathclose%
\pgfusepath{stroke,fill}%
\end{pgfscope}%
\begin{pgfscope}%
\pgfpathrectangle{\pgfqpoint{0.100000in}{0.212622in}}{\pgfqpoint{3.696000in}{3.696000in}}%
\pgfusepath{clip}%
\pgfsetbuttcap%
\pgfsetroundjoin%
\definecolor{currentfill}{rgb}{0.121569,0.466667,0.705882}%
\pgfsetfillcolor{currentfill}%
\pgfsetfillopacity{0.940745}%
\pgfsetlinewidth{1.003750pt}%
\definecolor{currentstroke}{rgb}{0.121569,0.466667,0.705882}%
\pgfsetstrokecolor{currentstroke}%
\pgfsetstrokeopacity{0.940745}%
\pgfsetdash{}{0pt}%
\pgfpathmoveto{\pgfqpoint{2.690410in}{1.134918in}}%
\pgfpathcurveto{\pgfqpoint{2.698646in}{1.134918in}}{\pgfqpoint{2.706546in}{1.138191in}}{\pgfqpoint{2.712370in}{1.144015in}}%
\pgfpathcurveto{\pgfqpoint{2.718194in}{1.149839in}}{\pgfqpoint{2.721467in}{1.157739in}}{\pgfqpoint{2.721467in}{1.165975in}}%
\pgfpathcurveto{\pgfqpoint{2.721467in}{1.174211in}}{\pgfqpoint{2.718194in}{1.182111in}}{\pgfqpoint{2.712370in}{1.187935in}}%
\pgfpathcurveto{\pgfqpoint{2.706546in}{1.193759in}}{\pgfqpoint{2.698646in}{1.197031in}}{\pgfqpoint{2.690410in}{1.197031in}}%
\pgfpathcurveto{\pgfqpoint{2.682174in}{1.197031in}}{\pgfqpoint{2.674274in}{1.193759in}}{\pgfqpoint{2.668450in}{1.187935in}}%
\pgfpathcurveto{\pgfqpoint{2.662626in}{1.182111in}}{\pgfqpoint{2.659354in}{1.174211in}}{\pgfqpoint{2.659354in}{1.165975in}}%
\pgfpathcurveto{\pgfqpoint{2.659354in}{1.157739in}}{\pgfqpoint{2.662626in}{1.149839in}}{\pgfqpoint{2.668450in}{1.144015in}}%
\pgfpathcurveto{\pgfqpoint{2.674274in}{1.138191in}}{\pgfqpoint{2.682174in}{1.134918in}}{\pgfqpoint{2.690410in}{1.134918in}}%
\pgfpathclose%
\pgfusepath{stroke,fill}%
\end{pgfscope}%
\begin{pgfscope}%
\pgfpathrectangle{\pgfqpoint{0.100000in}{0.212622in}}{\pgfqpoint{3.696000in}{3.696000in}}%
\pgfusepath{clip}%
\pgfsetbuttcap%
\pgfsetroundjoin%
\definecolor{currentfill}{rgb}{0.121569,0.466667,0.705882}%
\pgfsetfillcolor{currentfill}%
\pgfsetfillopacity{0.940769}%
\pgfsetlinewidth{1.003750pt}%
\definecolor{currentstroke}{rgb}{0.121569,0.466667,0.705882}%
\pgfsetstrokecolor{currentstroke}%
\pgfsetstrokeopacity{0.940769}%
\pgfsetdash{}{0pt}%
\pgfpathmoveto{\pgfqpoint{1.997238in}{1.881613in}}%
\pgfpathcurveto{\pgfqpoint{2.005475in}{1.881613in}}{\pgfqpoint{2.013375in}{1.884886in}}{\pgfqpoint{2.019199in}{1.890710in}}%
\pgfpathcurveto{\pgfqpoint{2.025023in}{1.896533in}}{\pgfqpoint{2.028295in}{1.904434in}}{\pgfqpoint{2.028295in}{1.912670in}}%
\pgfpathcurveto{\pgfqpoint{2.028295in}{1.920906in}}{\pgfqpoint{2.025023in}{1.928806in}}{\pgfqpoint{2.019199in}{1.934630in}}%
\pgfpathcurveto{\pgfqpoint{2.013375in}{1.940454in}}{\pgfqpoint{2.005475in}{1.943726in}}{\pgfqpoint{1.997238in}{1.943726in}}%
\pgfpathcurveto{\pgfqpoint{1.989002in}{1.943726in}}{\pgfqpoint{1.981102in}{1.940454in}}{\pgfqpoint{1.975278in}{1.934630in}}%
\pgfpathcurveto{\pgfqpoint{1.969454in}{1.928806in}}{\pgfqpoint{1.966182in}{1.920906in}}{\pgfqpoint{1.966182in}{1.912670in}}%
\pgfpathcurveto{\pgfqpoint{1.966182in}{1.904434in}}{\pgfqpoint{1.969454in}{1.896533in}}{\pgfqpoint{1.975278in}{1.890710in}}%
\pgfpathcurveto{\pgfqpoint{1.981102in}{1.884886in}}{\pgfqpoint{1.989002in}{1.881613in}}{\pgfqpoint{1.997238in}{1.881613in}}%
\pgfpathclose%
\pgfusepath{stroke,fill}%
\end{pgfscope}%
\begin{pgfscope}%
\pgfpathrectangle{\pgfqpoint{0.100000in}{0.212622in}}{\pgfqpoint{3.696000in}{3.696000in}}%
\pgfusepath{clip}%
\pgfsetbuttcap%
\pgfsetroundjoin%
\definecolor{currentfill}{rgb}{0.121569,0.466667,0.705882}%
\pgfsetfillcolor{currentfill}%
\pgfsetfillopacity{0.941277}%
\pgfsetlinewidth{1.003750pt}%
\definecolor{currentstroke}{rgb}{0.121569,0.466667,0.705882}%
\pgfsetstrokecolor{currentstroke}%
\pgfsetstrokeopacity{0.941277}%
\pgfsetdash{}{0pt}%
\pgfpathmoveto{\pgfqpoint{2.096060in}{1.917286in}}%
\pgfpathcurveto{\pgfqpoint{2.104296in}{1.917286in}}{\pgfqpoint{2.112196in}{1.920558in}}{\pgfqpoint{2.118020in}{1.926382in}}%
\pgfpathcurveto{\pgfqpoint{2.123844in}{1.932206in}}{\pgfqpoint{2.127116in}{1.940106in}}{\pgfqpoint{2.127116in}{1.948343in}}%
\pgfpathcurveto{\pgfqpoint{2.127116in}{1.956579in}}{\pgfqpoint{2.123844in}{1.964479in}}{\pgfqpoint{2.118020in}{1.970303in}}%
\pgfpathcurveto{\pgfqpoint{2.112196in}{1.976127in}}{\pgfqpoint{2.104296in}{1.979399in}}{\pgfqpoint{2.096060in}{1.979399in}}%
\pgfpathcurveto{\pgfqpoint{2.087824in}{1.979399in}}{\pgfqpoint{2.079924in}{1.976127in}}{\pgfqpoint{2.074100in}{1.970303in}}%
\pgfpathcurveto{\pgfqpoint{2.068276in}{1.964479in}}{\pgfqpoint{2.065003in}{1.956579in}}{\pgfqpoint{2.065003in}{1.948343in}}%
\pgfpathcurveto{\pgfqpoint{2.065003in}{1.940106in}}{\pgfqpoint{2.068276in}{1.932206in}}{\pgfqpoint{2.074100in}{1.926382in}}%
\pgfpathcurveto{\pgfqpoint{2.079924in}{1.920558in}}{\pgfqpoint{2.087824in}{1.917286in}}{\pgfqpoint{2.096060in}{1.917286in}}%
\pgfpathclose%
\pgfusepath{stroke,fill}%
\end{pgfscope}%
\begin{pgfscope}%
\pgfpathrectangle{\pgfqpoint{0.100000in}{0.212622in}}{\pgfqpoint{3.696000in}{3.696000in}}%
\pgfusepath{clip}%
\pgfsetbuttcap%
\pgfsetroundjoin%
\definecolor{currentfill}{rgb}{0.121569,0.466667,0.705882}%
\pgfsetfillcolor{currentfill}%
\pgfsetfillopacity{0.941958}%
\pgfsetlinewidth{1.003750pt}%
\definecolor{currentstroke}{rgb}{0.121569,0.466667,0.705882}%
\pgfsetstrokecolor{currentstroke}%
\pgfsetstrokeopacity{0.941958}%
\pgfsetdash{}{0pt}%
\pgfpathmoveto{\pgfqpoint{2.003249in}{1.880235in}}%
\pgfpathcurveto{\pgfqpoint{2.011486in}{1.880235in}}{\pgfqpoint{2.019386in}{1.883507in}}{\pgfqpoint{2.025210in}{1.889331in}}%
\pgfpathcurveto{\pgfqpoint{2.031034in}{1.895155in}}{\pgfqpoint{2.034306in}{1.903055in}}{\pgfqpoint{2.034306in}{1.911292in}}%
\pgfpathcurveto{\pgfqpoint{2.034306in}{1.919528in}}{\pgfqpoint{2.031034in}{1.927428in}}{\pgfqpoint{2.025210in}{1.933252in}}%
\pgfpathcurveto{\pgfqpoint{2.019386in}{1.939076in}}{\pgfqpoint{2.011486in}{1.942348in}}{\pgfqpoint{2.003249in}{1.942348in}}%
\pgfpathcurveto{\pgfqpoint{1.995013in}{1.942348in}}{\pgfqpoint{1.987113in}{1.939076in}}{\pgfqpoint{1.981289in}{1.933252in}}%
\pgfpathcurveto{\pgfqpoint{1.975465in}{1.927428in}}{\pgfqpoint{1.972193in}{1.919528in}}{\pgfqpoint{1.972193in}{1.911292in}}%
\pgfpathcurveto{\pgfqpoint{1.972193in}{1.903055in}}{\pgfqpoint{1.975465in}{1.895155in}}{\pgfqpoint{1.981289in}{1.889331in}}%
\pgfpathcurveto{\pgfqpoint{1.987113in}{1.883507in}}{\pgfqpoint{1.995013in}{1.880235in}}{\pgfqpoint{2.003249in}{1.880235in}}%
\pgfpathclose%
\pgfusepath{stroke,fill}%
\end{pgfscope}%
\begin{pgfscope}%
\pgfpathrectangle{\pgfqpoint{0.100000in}{0.212622in}}{\pgfqpoint{3.696000in}{3.696000in}}%
\pgfusepath{clip}%
\pgfsetbuttcap%
\pgfsetroundjoin%
\definecolor{currentfill}{rgb}{0.121569,0.466667,0.705882}%
\pgfsetfillcolor{currentfill}%
\pgfsetfillopacity{0.942123}%
\pgfsetlinewidth{1.003750pt}%
\definecolor{currentstroke}{rgb}{0.121569,0.466667,0.705882}%
\pgfsetstrokecolor{currentstroke}%
\pgfsetstrokeopacity{0.942123}%
\pgfsetdash{}{0pt}%
\pgfpathmoveto{\pgfqpoint{2.093580in}{1.912955in}}%
\pgfpathcurveto{\pgfqpoint{2.101816in}{1.912955in}}{\pgfqpoint{2.109716in}{1.916228in}}{\pgfqpoint{2.115540in}{1.922052in}}%
\pgfpathcurveto{\pgfqpoint{2.121364in}{1.927875in}}{\pgfqpoint{2.124636in}{1.935776in}}{\pgfqpoint{2.124636in}{1.944012in}}%
\pgfpathcurveto{\pgfqpoint{2.124636in}{1.952248in}}{\pgfqpoint{2.121364in}{1.960148in}}{\pgfqpoint{2.115540in}{1.965972in}}%
\pgfpathcurveto{\pgfqpoint{2.109716in}{1.971796in}}{\pgfqpoint{2.101816in}{1.975068in}}{\pgfqpoint{2.093580in}{1.975068in}}%
\pgfpathcurveto{\pgfqpoint{2.085344in}{1.975068in}}{\pgfqpoint{2.077444in}{1.971796in}}{\pgfqpoint{2.071620in}{1.965972in}}%
\pgfpathcurveto{\pgfqpoint{2.065796in}{1.960148in}}{\pgfqpoint{2.062523in}{1.952248in}}{\pgfqpoint{2.062523in}{1.944012in}}%
\pgfpathcurveto{\pgfqpoint{2.062523in}{1.935776in}}{\pgfqpoint{2.065796in}{1.927875in}}{\pgfqpoint{2.071620in}{1.922052in}}%
\pgfpathcurveto{\pgfqpoint{2.077444in}{1.916228in}}{\pgfqpoint{2.085344in}{1.912955in}}{\pgfqpoint{2.093580in}{1.912955in}}%
\pgfpathclose%
\pgfusepath{stroke,fill}%
\end{pgfscope}%
\begin{pgfscope}%
\pgfpathrectangle{\pgfqpoint{0.100000in}{0.212622in}}{\pgfqpoint{3.696000in}{3.696000in}}%
\pgfusepath{clip}%
\pgfsetbuttcap%
\pgfsetroundjoin%
\definecolor{currentfill}{rgb}{0.121569,0.466667,0.705882}%
\pgfsetfillcolor{currentfill}%
\pgfsetfillopacity{0.942501}%
\pgfsetlinewidth{1.003750pt}%
\definecolor{currentstroke}{rgb}{0.121569,0.466667,0.705882}%
\pgfsetstrokecolor{currentstroke}%
\pgfsetstrokeopacity{0.942501}%
\pgfsetdash{}{0pt}%
\pgfpathmoveto{\pgfqpoint{2.006451in}{1.878632in}}%
\pgfpathcurveto{\pgfqpoint{2.014687in}{1.878632in}}{\pgfqpoint{2.022587in}{1.881904in}}{\pgfqpoint{2.028411in}{1.887728in}}%
\pgfpathcurveto{\pgfqpoint{2.034235in}{1.893552in}}{\pgfqpoint{2.037508in}{1.901452in}}{\pgfqpoint{2.037508in}{1.909689in}}%
\pgfpathcurveto{\pgfqpoint{2.037508in}{1.917925in}}{\pgfqpoint{2.034235in}{1.925825in}}{\pgfqpoint{2.028411in}{1.931649in}}%
\pgfpathcurveto{\pgfqpoint{2.022587in}{1.937473in}}{\pgfqpoint{2.014687in}{1.940745in}}{\pgfqpoint{2.006451in}{1.940745in}}%
\pgfpathcurveto{\pgfqpoint{1.998215in}{1.940745in}}{\pgfqpoint{1.990315in}{1.937473in}}{\pgfqpoint{1.984491in}{1.931649in}}%
\pgfpathcurveto{\pgfqpoint{1.978667in}{1.925825in}}{\pgfqpoint{1.975395in}{1.917925in}}{\pgfqpoint{1.975395in}{1.909689in}}%
\pgfpathcurveto{\pgfqpoint{1.975395in}{1.901452in}}{\pgfqpoint{1.978667in}{1.893552in}}{\pgfqpoint{1.984491in}{1.887728in}}%
\pgfpathcurveto{\pgfqpoint{1.990315in}{1.881904in}}{\pgfqpoint{1.998215in}{1.878632in}}{\pgfqpoint{2.006451in}{1.878632in}}%
\pgfpathclose%
\pgfusepath{stroke,fill}%
\end{pgfscope}%
\begin{pgfscope}%
\pgfpathrectangle{\pgfqpoint{0.100000in}{0.212622in}}{\pgfqpoint{3.696000in}{3.696000in}}%
\pgfusepath{clip}%
\pgfsetbuttcap%
\pgfsetroundjoin%
\definecolor{currentfill}{rgb}{0.121569,0.466667,0.705882}%
\pgfsetfillcolor{currentfill}%
\pgfsetfillopacity{0.942751}%
\pgfsetlinewidth{1.003750pt}%
\definecolor{currentstroke}{rgb}{0.121569,0.466667,0.705882}%
\pgfsetstrokecolor{currentstroke}%
\pgfsetstrokeopacity{0.942751}%
\pgfsetdash{}{0pt}%
\pgfpathmoveto{\pgfqpoint{2.078487in}{0.979151in}}%
\pgfpathcurveto{\pgfqpoint{2.086723in}{0.979151in}}{\pgfqpoint{2.094623in}{0.982423in}}{\pgfqpoint{2.100447in}{0.988247in}}%
\pgfpathcurveto{\pgfqpoint{2.106271in}{0.994071in}}{\pgfqpoint{2.109544in}{1.001971in}}{\pgfqpoint{2.109544in}{1.010207in}}%
\pgfpathcurveto{\pgfqpoint{2.109544in}{1.018443in}}{\pgfqpoint{2.106271in}{1.026343in}}{\pgfqpoint{2.100447in}{1.032167in}}%
\pgfpathcurveto{\pgfqpoint{2.094623in}{1.037991in}}{\pgfqpoint{2.086723in}{1.041264in}}{\pgfqpoint{2.078487in}{1.041264in}}%
\pgfpathcurveto{\pgfqpoint{2.070251in}{1.041264in}}{\pgfqpoint{2.062351in}{1.037991in}}{\pgfqpoint{2.056527in}{1.032167in}}%
\pgfpathcurveto{\pgfqpoint{2.050703in}{1.026343in}}{\pgfqpoint{2.047431in}{1.018443in}}{\pgfqpoint{2.047431in}{1.010207in}}%
\pgfpathcurveto{\pgfqpoint{2.047431in}{1.001971in}}{\pgfqpoint{2.050703in}{0.994071in}}{\pgfqpoint{2.056527in}{0.988247in}}%
\pgfpathcurveto{\pgfqpoint{2.062351in}{0.982423in}}{\pgfqpoint{2.070251in}{0.979151in}}{\pgfqpoint{2.078487in}{0.979151in}}%
\pgfpathclose%
\pgfusepath{stroke,fill}%
\end{pgfscope}%
\begin{pgfscope}%
\pgfpathrectangle{\pgfqpoint{0.100000in}{0.212622in}}{\pgfqpoint{3.696000in}{3.696000in}}%
\pgfusepath{clip}%
\pgfsetbuttcap%
\pgfsetroundjoin%
\definecolor{currentfill}{rgb}{0.121569,0.466667,0.705882}%
\pgfsetfillcolor{currentfill}%
\pgfsetfillopacity{0.942840}%
\pgfsetlinewidth{1.003750pt}%
\definecolor{currentstroke}{rgb}{0.121569,0.466667,0.705882}%
\pgfsetstrokecolor{currentstroke}%
\pgfsetstrokeopacity{0.942840}%
\pgfsetdash{}{0pt}%
\pgfpathmoveto{\pgfqpoint{2.008157in}{1.877747in}}%
\pgfpathcurveto{\pgfqpoint{2.016393in}{1.877747in}}{\pgfqpoint{2.024293in}{1.881020in}}{\pgfqpoint{2.030117in}{1.886843in}}%
\pgfpathcurveto{\pgfqpoint{2.035941in}{1.892667in}}{\pgfqpoint{2.039213in}{1.900567in}}{\pgfqpoint{2.039213in}{1.908804in}}%
\pgfpathcurveto{\pgfqpoint{2.039213in}{1.917040in}}{\pgfqpoint{2.035941in}{1.924940in}}{\pgfqpoint{2.030117in}{1.930764in}}%
\pgfpathcurveto{\pgfqpoint{2.024293in}{1.936588in}}{\pgfqpoint{2.016393in}{1.939860in}}{\pgfqpoint{2.008157in}{1.939860in}}%
\pgfpathcurveto{\pgfqpoint{1.999921in}{1.939860in}}{\pgfqpoint{1.992020in}{1.936588in}}{\pgfqpoint{1.986197in}{1.930764in}}%
\pgfpathcurveto{\pgfqpoint{1.980373in}{1.924940in}}{\pgfqpoint{1.977100in}{1.917040in}}{\pgfqpoint{1.977100in}{1.908804in}}%
\pgfpathcurveto{\pgfqpoint{1.977100in}{1.900567in}}{\pgfqpoint{1.980373in}{1.892667in}}{\pgfqpoint{1.986197in}{1.886843in}}%
\pgfpathcurveto{\pgfqpoint{1.992020in}{1.881020in}}{\pgfqpoint{1.999921in}{1.877747in}}{\pgfqpoint{2.008157in}{1.877747in}}%
\pgfpathclose%
\pgfusepath{stroke,fill}%
\end{pgfscope}%
\begin{pgfscope}%
\pgfpathrectangle{\pgfqpoint{0.100000in}{0.212622in}}{\pgfqpoint{3.696000in}{3.696000in}}%
\pgfusepath{clip}%
\pgfsetbuttcap%
\pgfsetroundjoin%
\definecolor{currentfill}{rgb}{0.121569,0.466667,0.705882}%
\pgfsetfillcolor{currentfill}%
\pgfsetfillopacity{0.943012}%
\pgfsetlinewidth{1.003750pt}%
\definecolor{currentstroke}{rgb}{0.121569,0.466667,0.705882}%
\pgfsetstrokecolor{currentstroke}%
\pgfsetstrokeopacity{0.943012}%
\pgfsetdash{}{0pt}%
\pgfpathmoveto{\pgfqpoint{2.091846in}{1.909033in}}%
\pgfpathcurveto{\pgfqpoint{2.100083in}{1.909033in}}{\pgfqpoint{2.107983in}{1.912305in}}{\pgfqpoint{2.113807in}{1.918129in}}%
\pgfpathcurveto{\pgfqpoint{2.119631in}{1.923953in}}{\pgfqpoint{2.122903in}{1.931853in}}{\pgfqpoint{2.122903in}{1.940089in}}%
\pgfpathcurveto{\pgfqpoint{2.122903in}{1.948326in}}{\pgfqpoint{2.119631in}{1.956226in}}{\pgfqpoint{2.113807in}{1.962050in}}%
\pgfpathcurveto{\pgfqpoint{2.107983in}{1.967873in}}{\pgfqpoint{2.100083in}{1.971146in}}{\pgfqpoint{2.091846in}{1.971146in}}%
\pgfpathcurveto{\pgfqpoint{2.083610in}{1.971146in}}{\pgfqpoint{2.075710in}{1.967873in}}{\pgfqpoint{2.069886in}{1.962050in}}%
\pgfpathcurveto{\pgfqpoint{2.064062in}{1.956226in}}{\pgfqpoint{2.060790in}{1.948326in}}{\pgfqpoint{2.060790in}{1.940089in}}%
\pgfpathcurveto{\pgfqpoint{2.060790in}{1.931853in}}{\pgfqpoint{2.064062in}{1.923953in}}{\pgfqpoint{2.069886in}{1.918129in}}%
\pgfpathcurveto{\pgfqpoint{2.075710in}{1.912305in}}{\pgfqpoint{2.083610in}{1.909033in}}{\pgfqpoint{2.091846in}{1.909033in}}%
\pgfpathclose%
\pgfusepath{stroke,fill}%
\end{pgfscope}%
\begin{pgfscope}%
\pgfpathrectangle{\pgfqpoint{0.100000in}{0.212622in}}{\pgfqpoint{3.696000in}{3.696000in}}%
\pgfusepath{clip}%
\pgfsetbuttcap%
\pgfsetroundjoin%
\definecolor{currentfill}{rgb}{0.121569,0.466667,0.705882}%
\pgfsetfillcolor{currentfill}%
\pgfsetfillopacity{0.943323}%
\pgfsetlinewidth{1.003750pt}%
\definecolor{currentstroke}{rgb}{0.121569,0.466667,0.705882}%
\pgfsetstrokecolor{currentstroke}%
\pgfsetstrokeopacity{0.943323}%
\pgfsetdash{}{0pt}%
\pgfpathmoveto{\pgfqpoint{2.682501in}{1.123062in}}%
\pgfpathcurveto{\pgfqpoint{2.690737in}{1.123062in}}{\pgfqpoint{2.698637in}{1.126334in}}{\pgfqpoint{2.704461in}{1.132158in}}%
\pgfpathcurveto{\pgfqpoint{2.710285in}{1.137982in}}{\pgfqpoint{2.713557in}{1.145882in}}{\pgfqpoint{2.713557in}{1.154119in}}%
\pgfpathcurveto{\pgfqpoint{2.713557in}{1.162355in}}{\pgfqpoint{2.710285in}{1.170255in}}{\pgfqpoint{2.704461in}{1.176079in}}%
\pgfpathcurveto{\pgfqpoint{2.698637in}{1.181903in}}{\pgfqpoint{2.690737in}{1.185175in}}{\pgfqpoint{2.682501in}{1.185175in}}%
\pgfpathcurveto{\pgfqpoint{2.674264in}{1.185175in}}{\pgfqpoint{2.666364in}{1.181903in}}{\pgfqpoint{2.660540in}{1.176079in}}%
\pgfpathcurveto{\pgfqpoint{2.654717in}{1.170255in}}{\pgfqpoint{2.651444in}{1.162355in}}{\pgfqpoint{2.651444in}{1.154119in}}%
\pgfpathcurveto{\pgfqpoint{2.651444in}{1.145882in}}{\pgfqpoint{2.654717in}{1.137982in}}{\pgfqpoint{2.660540in}{1.132158in}}%
\pgfpathcurveto{\pgfqpoint{2.666364in}{1.126334in}}{\pgfqpoint{2.674264in}{1.123062in}}{\pgfqpoint{2.682501in}{1.123062in}}%
\pgfpathclose%
\pgfusepath{stroke,fill}%
\end{pgfscope}%
\begin{pgfscope}%
\pgfpathrectangle{\pgfqpoint{0.100000in}{0.212622in}}{\pgfqpoint{3.696000in}{3.696000in}}%
\pgfusepath{clip}%
\pgfsetbuttcap%
\pgfsetroundjoin%
\definecolor{currentfill}{rgb}{0.121569,0.466667,0.705882}%
\pgfsetfillcolor{currentfill}%
\pgfsetfillopacity{0.943377}%
\pgfsetlinewidth{1.003750pt}%
\definecolor{currentstroke}{rgb}{0.121569,0.466667,0.705882}%
\pgfsetstrokecolor{currentstroke}%
\pgfsetstrokeopacity{0.943377}%
\pgfsetdash{}{0pt}%
\pgfpathmoveto{\pgfqpoint{2.010950in}{1.876748in}}%
\pgfpathcurveto{\pgfqpoint{2.019186in}{1.876748in}}{\pgfqpoint{2.027086in}{1.880020in}}{\pgfqpoint{2.032910in}{1.885844in}}%
\pgfpathcurveto{\pgfqpoint{2.038734in}{1.891668in}}{\pgfqpoint{2.042006in}{1.899568in}}{\pgfqpoint{2.042006in}{1.907805in}}%
\pgfpathcurveto{\pgfqpoint{2.042006in}{1.916041in}}{\pgfqpoint{2.038734in}{1.923941in}}{\pgfqpoint{2.032910in}{1.929765in}}%
\pgfpathcurveto{\pgfqpoint{2.027086in}{1.935589in}}{\pgfqpoint{2.019186in}{1.938861in}}{\pgfqpoint{2.010950in}{1.938861in}}%
\pgfpathcurveto{\pgfqpoint{2.002714in}{1.938861in}}{\pgfqpoint{1.994814in}{1.935589in}}{\pgfqpoint{1.988990in}{1.929765in}}%
\pgfpathcurveto{\pgfqpoint{1.983166in}{1.923941in}}{\pgfqpoint{1.979893in}{1.916041in}}{\pgfqpoint{1.979893in}{1.907805in}}%
\pgfpathcurveto{\pgfqpoint{1.979893in}{1.899568in}}{\pgfqpoint{1.983166in}{1.891668in}}{\pgfqpoint{1.988990in}{1.885844in}}%
\pgfpathcurveto{\pgfqpoint{1.994814in}{1.880020in}}{\pgfqpoint{2.002714in}{1.876748in}}{\pgfqpoint{2.010950in}{1.876748in}}%
\pgfpathclose%
\pgfusepath{stroke,fill}%
\end{pgfscope}%
\begin{pgfscope}%
\pgfpathrectangle{\pgfqpoint{0.100000in}{0.212622in}}{\pgfqpoint{3.696000in}{3.696000in}}%
\pgfusepath{clip}%
\pgfsetbuttcap%
\pgfsetroundjoin%
\definecolor{currentfill}{rgb}{0.121569,0.466667,0.705882}%
\pgfsetfillcolor{currentfill}%
\pgfsetfillopacity{0.943428}%
\pgfsetlinewidth{1.003750pt}%
\definecolor{currentstroke}{rgb}{0.121569,0.466667,0.705882}%
\pgfsetstrokecolor{currentstroke}%
\pgfsetstrokeopacity{0.943428}%
\pgfsetdash{}{0pt}%
\pgfpathmoveto{\pgfqpoint{2.089863in}{1.906067in}}%
\pgfpathcurveto{\pgfqpoint{2.098100in}{1.906067in}}{\pgfqpoint{2.106000in}{1.909339in}}{\pgfqpoint{2.111823in}{1.915163in}}%
\pgfpathcurveto{\pgfqpoint{2.117647in}{1.920987in}}{\pgfqpoint{2.120920in}{1.928887in}}{\pgfqpoint{2.120920in}{1.937123in}}%
\pgfpathcurveto{\pgfqpoint{2.120920in}{1.945359in}}{\pgfqpoint{2.117647in}{1.953259in}}{\pgfqpoint{2.111823in}{1.959083in}}%
\pgfpathcurveto{\pgfqpoint{2.106000in}{1.964907in}}{\pgfqpoint{2.098100in}{1.968180in}}{\pgfqpoint{2.089863in}{1.968180in}}%
\pgfpathcurveto{\pgfqpoint{2.081627in}{1.968180in}}{\pgfqpoint{2.073727in}{1.964907in}}{\pgfqpoint{2.067903in}{1.959083in}}%
\pgfpathcurveto{\pgfqpoint{2.062079in}{1.953259in}}{\pgfqpoint{2.058807in}{1.945359in}}{\pgfqpoint{2.058807in}{1.937123in}}%
\pgfpathcurveto{\pgfqpoint{2.058807in}{1.928887in}}{\pgfqpoint{2.062079in}{1.920987in}}{\pgfqpoint{2.067903in}{1.915163in}}%
\pgfpathcurveto{\pgfqpoint{2.073727in}{1.909339in}}{\pgfqpoint{2.081627in}{1.906067in}}{\pgfqpoint{2.089863in}{1.906067in}}%
\pgfpathclose%
\pgfusepath{stroke,fill}%
\end{pgfscope}%
\begin{pgfscope}%
\pgfpathrectangle{\pgfqpoint{0.100000in}{0.212622in}}{\pgfqpoint{3.696000in}{3.696000in}}%
\pgfusepath{clip}%
\pgfsetbuttcap%
\pgfsetroundjoin%
\definecolor{currentfill}{rgb}{0.121569,0.466667,0.705882}%
\pgfsetfillcolor{currentfill}%
\pgfsetfillopacity{0.943605}%
\pgfsetlinewidth{1.003750pt}%
\definecolor{currentstroke}{rgb}{0.121569,0.466667,0.705882}%
\pgfsetstrokecolor{currentstroke}%
\pgfsetstrokeopacity{0.943605}%
\pgfsetdash{}{0pt}%
\pgfpathmoveto{\pgfqpoint{2.089573in}{1.905065in}}%
\pgfpathcurveto{\pgfqpoint{2.097809in}{1.905065in}}{\pgfqpoint{2.105709in}{1.908338in}}{\pgfqpoint{2.111533in}{1.914161in}}%
\pgfpathcurveto{\pgfqpoint{2.117357in}{1.919985in}}{\pgfqpoint{2.120629in}{1.927885in}}{\pgfqpoint{2.120629in}{1.936122in}}%
\pgfpathcurveto{\pgfqpoint{2.120629in}{1.944358in}}{\pgfqpoint{2.117357in}{1.952258in}}{\pgfqpoint{2.111533in}{1.958082in}}%
\pgfpathcurveto{\pgfqpoint{2.105709in}{1.963906in}}{\pgfqpoint{2.097809in}{1.967178in}}{\pgfqpoint{2.089573in}{1.967178in}}%
\pgfpathcurveto{\pgfqpoint{2.081337in}{1.967178in}}{\pgfqpoint{2.073437in}{1.963906in}}{\pgfqpoint{2.067613in}{1.958082in}}%
\pgfpathcurveto{\pgfqpoint{2.061789in}{1.952258in}}{\pgfqpoint{2.058516in}{1.944358in}}{\pgfqpoint{2.058516in}{1.936122in}}%
\pgfpathcurveto{\pgfqpoint{2.058516in}{1.927885in}}{\pgfqpoint{2.061789in}{1.919985in}}{\pgfqpoint{2.067613in}{1.914161in}}%
\pgfpathcurveto{\pgfqpoint{2.073437in}{1.908338in}}{\pgfqpoint{2.081337in}{1.905065in}}{\pgfqpoint{2.089573in}{1.905065in}}%
\pgfpathclose%
\pgfusepath{stroke,fill}%
\end{pgfscope}%
\begin{pgfscope}%
\pgfpathrectangle{\pgfqpoint{0.100000in}{0.212622in}}{\pgfqpoint{3.696000in}{3.696000in}}%
\pgfusepath{clip}%
\pgfsetbuttcap%
\pgfsetroundjoin%
\definecolor{currentfill}{rgb}{0.121569,0.466667,0.705882}%
\pgfsetfillcolor{currentfill}%
\pgfsetfillopacity{0.943629}%
\pgfsetlinewidth{1.003750pt}%
\definecolor{currentstroke}{rgb}{0.121569,0.466667,0.705882}%
\pgfsetstrokecolor{currentstroke}%
\pgfsetstrokeopacity{0.943629}%
\pgfsetdash{}{0pt}%
\pgfpathmoveto{\pgfqpoint{2.089487in}{1.904917in}}%
\pgfpathcurveto{\pgfqpoint{2.097724in}{1.904917in}}{\pgfqpoint{2.105624in}{1.908189in}}{\pgfqpoint{2.111448in}{1.914013in}}%
\pgfpathcurveto{\pgfqpoint{2.117272in}{1.919837in}}{\pgfqpoint{2.120544in}{1.927737in}}{\pgfqpoint{2.120544in}{1.935973in}}%
\pgfpathcurveto{\pgfqpoint{2.120544in}{1.944210in}}{\pgfqpoint{2.117272in}{1.952110in}}{\pgfqpoint{2.111448in}{1.957934in}}%
\pgfpathcurveto{\pgfqpoint{2.105624in}{1.963758in}}{\pgfqpoint{2.097724in}{1.967030in}}{\pgfqpoint{2.089487in}{1.967030in}}%
\pgfpathcurveto{\pgfqpoint{2.081251in}{1.967030in}}{\pgfqpoint{2.073351in}{1.963758in}}{\pgfqpoint{2.067527in}{1.957934in}}%
\pgfpathcurveto{\pgfqpoint{2.061703in}{1.952110in}}{\pgfqpoint{2.058431in}{1.944210in}}{\pgfqpoint{2.058431in}{1.935973in}}%
\pgfpathcurveto{\pgfqpoint{2.058431in}{1.927737in}}{\pgfqpoint{2.061703in}{1.919837in}}{\pgfqpoint{2.067527in}{1.914013in}}%
\pgfpathcurveto{\pgfqpoint{2.073351in}{1.908189in}}{\pgfqpoint{2.081251in}{1.904917in}}{\pgfqpoint{2.089487in}{1.904917in}}%
\pgfpathclose%
\pgfusepath{stroke,fill}%
\end{pgfscope}%
\begin{pgfscope}%
\pgfpathrectangle{\pgfqpoint{0.100000in}{0.212622in}}{\pgfqpoint{3.696000in}{3.696000in}}%
\pgfusepath{clip}%
\pgfsetbuttcap%
\pgfsetroundjoin%
\definecolor{currentfill}{rgb}{0.121569,0.466667,0.705882}%
\pgfsetfillcolor{currentfill}%
\pgfsetfillopacity{0.943635}%
\pgfsetlinewidth{1.003750pt}%
\definecolor{currentstroke}{rgb}{0.121569,0.466667,0.705882}%
\pgfsetstrokecolor{currentstroke}%
\pgfsetstrokeopacity{0.943635}%
\pgfsetdash{}{0pt}%
\pgfpathmoveto{\pgfqpoint{2.012412in}{1.875793in}}%
\pgfpathcurveto{\pgfqpoint{2.020648in}{1.875793in}}{\pgfqpoint{2.028548in}{1.879065in}}{\pgfqpoint{2.034372in}{1.884889in}}%
\pgfpathcurveto{\pgfqpoint{2.040196in}{1.890713in}}{\pgfqpoint{2.043469in}{1.898613in}}{\pgfqpoint{2.043469in}{1.906850in}}%
\pgfpathcurveto{\pgfqpoint{2.043469in}{1.915086in}}{\pgfqpoint{2.040196in}{1.922986in}}{\pgfqpoint{2.034372in}{1.928810in}}%
\pgfpathcurveto{\pgfqpoint{2.028548in}{1.934634in}}{\pgfqpoint{2.020648in}{1.937906in}}{\pgfqpoint{2.012412in}{1.937906in}}%
\pgfpathcurveto{\pgfqpoint{2.004176in}{1.937906in}}{\pgfqpoint{1.996276in}{1.934634in}}{\pgfqpoint{1.990452in}{1.928810in}}%
\pgfpathcurveto{\pgfqpoint{1.984628in}{1.922986in}}{\pgfqpoint{1.981356in}{1.915086in}}{\pgfqpoint{1.981356in}{1.906850in}}%
\pgfpathcurveto{\pgfqpoint{1.981356in}{1.898613in}}{\pgfqpoint{1.984628in}{1.890713in}}{\pgfqpoint{1.990452in}{1.884889in}}%
\pgfpathcurveto{\pgfqpoint{1.996276in}{1.879065in}}{\pgfqpoint{2.004176in}{1.875793in}}{\pgfqpoint{2.012412in}{1.875793in}}%
\pgfpathclose%
\pgfusepath{stroke,fill}%
\end{pgfscope}%
\begin{pgfscope}%
\pgfpathrectangle{\pgfqpoint{0.100000in}{0.212622in}}{\pgfqpoint{3.696000in}{3.696000in}}%
\pgfusepath{clip}%
\pgfsetbuttcap%
\pgfsetroundjoin%
\definecolor{currentfill}{rgb}{0.121569,0.466667,0.705882}%
\pgfsetfillcolor{currentfill}%
\pgfsetfillopacity{0.943683}%
\pgfsetlinewidth{1.003750pt}%
\definecolor{currentstroke}{rgb}{0.121569,0.466667,0.705882}%
\pgfsetstrokecolor{currentstroke}%
\pgfsetstrokeopacity{0.943683}%
\pgfsetdash{}{0pt}%
\pgfpathmoveto{\pgfqpoint{2.089389in}{1.904646in}}%
\pgfpathcurveto{\pgfqpoint{2.097625in}{1.904646in}}{\pgfqpoint{2.105526in}{1.907918in}}{\pgfqpoint{2.111349in}{1.913742in}}%
\pgfpathcurveto{\pgfqpoint{2.117173in}{1.919566in}}{\pgfqpoint{2.120446in}{1.927466in}}{\pgfqpoint{2.120446in}{1.935702in}}%
\pgfpathcurveto{\pgfqpoint{2.120446in}{1.943938in}}{\pgfqpoint{2.117173in}{1.951838in}}{\pgfqpoint{2.111349in}{1.957662in}}%
\pgfpathcurveto{\pgfqpoint{2.105526in}{1.963486in}}{\pgfqpoint{2.097625in}{1.966759in}}{\pgfqpoint{2.089389in}{1.966759in}}%
\pgfpathcurveto{\pgfqpoint{2.081153in}{1.966759in}}{\pgfqpoint{2.073253in}{1.963486in}}{\pgfqpoint{2.067429in}{1.957662in}}%
\pgfpathcurveto{\pgfqpoint{2.061605in}{1.951838in}}{\pgfqpoint{2.058333in}{1.943938in}}{\pgfqpoint{2.058333in}{1.935702in}}%
\pgfpathcurveto{\pgfqpoint{2.058333in}{1.927466in}}{\pgfqpoint{2.061605in}{1.919566in}}{\pgfqpoint{2.067429in}{1.913742in}}%
\pgfpathcurveto{\pgfqpoint{2.073253in}{1.907918in}}{\pgfqpoint{2.081153in}{1.904646in}}{\pgfqpoint{2.089389in}{1.904646in}}%
\pgfpathclose%
\pgfusepath{stroke,fill}%
\end{pgfscope}%
\begin{pgfscope}%
\pgfpathrectangle{\pgfqpoint{0.100000in}{0.212622in}}{\pgfqpoint{3.696000in}{3.696000in}}%
\pgfusepath{clip}%
\pgfsetbuttcap%
\pgfsetroundjoin%
\definecolor{currentfill}{rgb}{0.121569,0.466667,0.705882}%
\pgfsetfillcolor{currentfill}%
\pgfsetfillopacity{0.943766}%
\pgfsetlinewidth{1.003750pt}%
\definecolor{currentstroke}{rgb}{0.121569,0.466667,0.705882}%
\pgfsetstrokecolor{currentstroke}%
\pgfsetstrokeopacity{0.943766}%
\pgfsetdash{}{0pt}%
\pgfpathmoveto{\pgfqpoint{2.089111in}{1.904172in}}%
\pgfpathcurveto{\pgfqpoint{2.097348in}{1.904172in}}{\pgfqpoint{2.105248in}{1.907445in}}{\pgfqpoint{2.111072in}{1.913268in}}%
\pgfpathcurveto{\pgfqpoint{2.116896in}{1.919092in}}{\pgfqpoint{2.120168in}{1.926992in}}{\pgfqpoint{2.120168in}{1.935229in}}%
\pgfpathcurveto{\pgfqpoint{2.120168in}{1.943465in}}{\pgfqpoint{2.116896in}{1.951365in}}{\pgfqpoint{2.111072in}{1.957189in}}%
\pgfpathcurveto{\pgfqpoint{2.105248in}{1.963013in}}{\pgfqpoint{2.097348in}{1.966285in}}{\pgfqpoint{2.089111in}{1.966285in}}%
\pgfpathcurveto{\pgfqpoint{2.080875in}{1.966285in}}{\pgfqpoint{2.072975in}{1.963013in}}{\pgfqpoint{2.067151in}{1.957189in}}%
\pgfpathcurveto{\pgfqpoint{2.061327in}{1.951365in}}{\pgfqpoint{2.058055in}{1.943465in}}{\pgfqpoint{2.058055in}{1.935229in}}%
\pgfpathcurveto{\pgfqpoint{2.058055in}{1.926992in}}{\pgfqpoint{2.061327in}{1.919092in}}{\pgfqpoint{2.067151in}{1.913268in}}%
\pgfpathcurveto{\pgfqpoint{2.072975in}{1.907445in}}{\pgfqpoint{2.080875in}{1.904172in}}{\pgfqpoint{2.089111in}{1.904172in}}%
\pgfpathclose%
\pgfusepath{stroke,fill}%
\end{pgfscope}%
\begin{pgfscope}%
\pgfpathrectangle{\pgfqpoint{0.100000in}{0.212622in}}{\pgfqpoint{3.696000in}{3.696000in}}%
\pgfusepath{clip}%
\pgfsetbuttcap%
\pgfsetroundjoin%
\definecolor{currentfill}{rgb}{0.121569,0.466667,0.705882}%
\pgfsetfillcolor{currentfill}%
\pgfsetfillopacity{0.943778}%
\pgfsetlinewidth{1.003750pt}%
\definecolor{currentstroke}{rgb}{0.121569,0.466667,0.705882}%
\pgfsetstrokecolor{currentstroke}%
\pgfsetstrokeopacity{0.943778}%
\pgfsetdash{}{0pt}%
\pgfpathmoveto{\pgfqpoint{2.013258in}{1.875407in}}%
\pgfpathcurveto{\pgfqpoint{2.021494in}{1.875407in}}{\pgfqpoint{2.029394in}{1.878679in}}{\pgfqpoint{2.035218in}{1.884503in}}%
\pgfpathcurveto{\pgfqpoint{2.041042in}{1.890327in}}{\pgfqpoint{2.044314in}{1.898227in}}{\pgfqpoint{2.044314in}{1.906464in}}%
\pgfpathcurveto{\pgfqpoint{2.044314in}{1.914700in}}{\pgfqpoint{2.041042in}{1.922600in}}{\pgfqpoint{2.035218in}{1.928424in}}%
\pgfpathcurveto{\pgfqpoint{2.029394in}{1.934248in}}{\pgfqpoint{2.021494in}{1.937520in}}{\pgfqpoint{2.013258in}{1.937520in}}%
\pgfpathcurveto{\pgfqpoint{2.005022in}{1.937520in}}{\pgfqpoint{1.997121in}{1.934248in}}{\pgfqpoint{1.991298in}{1.928424in}}%
\pgfpathcurveto{\pgfqpoint{1.985474in}{1.922600in}}{\pgfqpoint{1.982201in}{1.914700in}}{\pgfqpoint{1.982201in}{1.906464in}}%
\pgfpathcurveto{\pgfqpoint{1.982201in}{1.898227in}}{\pgfqpoint{1.985474in}{1.890327in}}{\pgfqpoint{1.991298in}{1.884503in}}%
\pgfpathcurveto{\pgfqpoint{1.997121in}{1.878679in}}{\pgfqpoint{2.005022in}{1.875407in}}{\pgfqpoint{2.013258in}{1.875407in}}%
\pgfpathclose%
\pgfusepath{stroke,fill}%
\end{pgfscope}%
\begin{pgfscope}%
\pgfpathrectangle{\pgfqpoint{0.100000in}{0.212622in}}{\pgfqpoint{3.696000in}{3.696000in}}%
\pgfusepath{clip}%
\pgfsetbuttcap%
\pgfsetroundjoin%
\definecolor{currentfill}{rgb}{0.121569,0.466667,0.705882}%
\pgfsetfillcolor{currentfill}%
\pgfsetfillopacity{0.943862}%
\pgfsetlinewidth{1.003750pt}%
\definecolor{currentstroke}{rgb}{0.121569,0.466667,0.705882}%
\pgfsetstrokecolor{currentstroke}%
\pgfsetstrokeopacity{0.943862}%
\pgfsetdash{}{0pt}%
\pgfpathmoveto{\pgfqpoint{2.013713in}{1.875185in}}%
\pgfpathcurveto{\pgfqpoint{2.021949in}{1.875185in}}{\pgfqpoint{2.029849in}{1.878457in}}{\pgfqpoint{2.035673in}{1.884281in}}%
\pgfpathcurveto{\pgfqpoint{2.041497in}{1.890105in}}{\pgfqpoint{2.044769in}{1.898005in}}{\pgfqpoint{2.044769in}{1.906241in}}%
\pgfpathcurveto{\pgfqpoint{2.044769in}{1.914477in}}{\pgfqpoint{2.041497in}{1.922377in}}{\pgfqpoint{2.035673in}{1.928201in}}%
\pgfpathcurveto{\pgfqpoint{2.029849in}{1.934025in}}{\pgfqpoint{2.021949in}{1.937298in}}{\pgfqpoint{2.013713in}{1.937298in}}%
\pgfpathcurveto{\pgfqpoint{2.005476in}{1.937298in}}{\pgfqpoint{1.997576in}{1.934025in}}{\pgfqpoint{1.991752in}{1.928201in}}%
\pgfpathcurveto{\pgfqpoint{1.985928in}{1.922377in}}{\pgfqpoint{1.982656in}{1.914477in}}{\pgfqpoint{1.982656in}{1.906241in}}%
\pgfpathcurveto{\pgfqpoint{1.982656in}{1.898005in}}{\pgfqpoint{1.985928in}{1.890105in}}{\pgfqpoint{1.991752in}{1.884281in}}%
\pgfpathcurveto{\pgfqpoint{1.997576in}{1.878457in}}{\pgfqpoint{2.005476in}{1.875185in}}{\pgfqpoint{2.013713in}{1.875185in}}%
\pgfpathclose%
\pgfusepath{stroke,fill}%
\end{pgfscope}%
\begin{pgfscope}%
\pgfpathrectangle{\pgfqpoint{0.100000in}{0.212622in}}{\pgfqpoint{3.696000in}{3.696000in}}%
\pgfusepath{clip}%
\pgfsetbuttcap%
\pgfsetroundjoin%
\definecolor{currentfill}{rgb}{0.121569,0.466667,0.705882}%
\pgfsetfillcolor{currentfill}%
\pgfsetfillopacity{0.943934}%
\pgfsetlinewidth{1.003750pt}%
\definecolor{currentstroke}{rgb}{0.121569,0.466667,0.705882}%
\pgfsetstrokecolor{currentstroke}%
\pgfsetstrokeopacity{0.943934}%
\pgfsetdash{}{0pt}%
\pgfpathmoveto{\pgfqpoint{2.088634in}{1.903349in}}%
\pgfpathcurveto{\pgfqpoint{2.096870in}{1.903349in}}{\pgfqpoint{2.104770in}{1.906622in}}{\pgfqpoint{2.110594in}{1.912446in}}%
\pgfpathcurveto{\pgfqpoint{2.116418in}{1.918269in}}{\pgfqpoint{2.119690in}{1.926170in}}{\pgfqpoint{2.119690in}{1.934406in}}%
\pgfpathcurveto{\pgfqpoint{2.119690in}{1.942642in}}{\pgfqpoint{2.116418in}{1.950542in}}{\pgfqpoint{2.110594in}{1.956366in}}%
\pgfpathcurveto{\pgfqpoint{2.104770in}{1.962190in}}{\pgfqpoint{2.096870in}{1.965462in}}{\pgfqpoint{2.088634in}{1.965462in}}%
\pgfpathcurveto{\pgfqpoint{2.080398in}{1.965462in}}{\pgfqpoint{2.072498in}{1.962190in}}{\pgfqpoint{2.066674in}{1.956366in}}%
\pgfpathcurveto{\pgfqpoint{2.060850in}{1.950542in}}{\pgfqpoint{2.057577in}{1.942642in}}{\pgfqpoint{2.057577in}{1.934406in}}%
\pgfpathcurveto{\pgfqpoint{2.057577in}{1.926170in}}{\pgfqpoint{2.060850in}{1.918269in}}{\pgfqpoint{2.066674in}{1.912446in}}%
\pgfpathcurveto{\pgfqpoint{2.072498in}{1.906622in}}{\pgfqpoint{2.080398in}{1.903349in}}{\pgfqpoint{2.088634in}{1.903349in}}%
\pgfpathclose%
\pgfusepath{stroke,fill}%
\end{pgfscope}%
\begin{pgfscope}%
\pgfpathrectangle{\pgfqpoint{0.100000in}{0.212622in}}{\pgfqpoint{3.696000in}{3.696000in}}%
\pgfusepath{clip}%
\pgfsetbuttcap%
\pgfsetroundjoin%
\definecolor{currentfill}{rgb}{0.121569,0.466667,0.705882}%
\pgfsetfillcolor{currentfill}%
\pgfsetfillopacity{0.944055}%
\pgfsetlinewidth{1.003750pt}%
\definecolor{currentstroke}{rgb}{0.121569,0.466667,0.705882}%
\pgfsetstrokecolor{currentstroke}%
\pgfsetstrokeopacity{0.944055}%
\pgfsetdash{}{0pt}%
\pgfpathmoveto{\pgfqpoint{2.014709in}{1.874583in}}%
\pgfpathcurveto{\pgfqpoint{2.022945in}{1.874583in}}{\pgfqpoint{2.030845in}{1.877855in}}{\pgfqpoint{2.036669in}{1.883679in}}%
\pgfpathcurveto{\pgfqpoint{2.042493in}{1.889503in}}{\pgfqpoint{2.045766in}{1.897403in}}{\pgfqpoint{2.045766in}{1.905639in}}%
\pgfpathcurveto{\pgfqpoint{2.045766in}{1.913875in}}{\pgfqpoint{2.042493in}{1.921776in}}{\pgfqpoint{2.036669in}{1.927599in}}%
\pgfpathcurveto{\pgfqpoint{2.030845in}{1.933423in}}{\pgfqpoint{2.022945in}{1.936696in}}{\pgfqpoint{2.014709in}{1.936696in}}%
\pgfpathcurveto{\pgfqpoint{2.006473in}{1.936696in}}{\pgfqpoint{1.998573in}{1.933423in}}{\pgfqpoint{1.992749in}{1.927599in}}%
\pgfpathcurveto{\pgfqpoint{1.986925in}{1.921776in}}{\pgfqpoint{1.983653in}{1.913875in}}{\pgfqpoint{1.983653in}{1.905639in}}%
\pgfpathcurveto{\pgfqpoint{1.983653in}{1.897403in}}{\pgfqpoint{1.986925in}{1.889503in}}{\pgfqpoint{1.992749in}{1.883679in}}%
\pgfpathcurveto{\pgfqpoint{1.998573in}{1.877855in}}{\pgfqpoint{2.006473in}{1.874583in}}{\pgfqpoint{2.014709in}{1.874583in}}%
\pgfpathclose%
\pgfusepath{stroke,fill}%
\end{pgfscope}%
\begin{pgfscope}%
\pgfpathrectangle{\pgfqpoint{0.100000in}{0.212622in}}{\pgfqpoint{3.696000in}{3.696000in}}%
\pgfusepath{clip}%
\pgfsetbuttcap%
\pgfsetroundjoin%
\definecolor{currentfill}{rgb}{0.121569,0.466667,0.705882}%
\pgfsetfillcolor{currentfill}%
\pgfsetfillopacity{0.944241}%
\pgfsetlinewidth{1.003750pt}%
\definecolor{currentstroke}{rgb}{0.121569,0.466667,0.705882}%
\pgfsetstrokecolor{currentstroke}%
\pgfsetstrokeopacity{0.944241}%
\pgfsetdash{}{0pt}%
\pgfpathmoveto{\pgfqpoint{2.087932in}{1.901691in}}%
\pgfpathcurveto{\pgfqpoint{2.096168in}{1.901691in}}{\pgfqpoint{2.104068in}{1.904964in}}{\pgfqpoint{2.109892in}{1.910788in}}%
\pgfpathcurveto{\pgfqpoint{2.115716in}{1.916612in}}{\pgfqpoint{2.118988in}{1.924512in}}{\pgfqpoint{2.118988in}{1.932748in}}%
\pgfpathcurveto{\pgfqpoint{2.118988in}{1.940984in}}{\pgfqpoint{2.115716in}{1.948884in}}{\pgfqpoint{2.109892in}{1.954708in}}%
\pgfpathcurveto{\pgfqpoint{2.104068in}{1.960532in}}{\pgfqpoint{2.096168in}{1.963804in}}{\pgfqpoint{2.087932in}{1.963804in}}%
\pgfpathcurveto{\pgfqpoint{2.079696in}{1.963804in}}{\pgfqpoint{2.071796in}{1.960532in}}{\pgfqpoint{2.065972in}{1.954708in}}%
\pgfpathcurveto{\pgfqpoint{2.060148in}{1.948884in}}{\pgfqpoint{2.056875in}{1.940984in}}{\pgfqpoint{2.056875in}{1.932748in}}%
\pgfpathcurveto{\pgfqpoint{2.056875in}{1.924512in}}{\pgfqpoint{2.060148in}{1.916612in}}{\pgfqpoint{2.065972in}{1.910788in}}%
\pgfpathcurveto{\pgfqpoint{2.071796in}{1.904964in}}{\pgfqpoint{2.079696in}{1.901691in}}{\pgfqpoint{2.087932in}{1.901691in}}%
\pgfpathclose%
\pgfusepath{stroke,fill}%
\end{pgfscope}%
\begin{pgfscope}%
\pgfpathrectangle{\pgfqpoint{0.100000in}{0.212622in}}{\pgfqpoint{3.696000in}{3.696000in}}%
\pgfusepath{clip}%
\pgfsetbuttcap%
\pgfsetroundjoin%
\definecolor{currentfill}{rgb}{0.121569,0.466667,0.705882}%
\pgfsetfillcolor{currentfill}%
\pgfsetfillopacity{0.944268}%
\pgfsetlinewidth{1.003750pt}%
\definecolor{currentstroke}{rgb}{0.121569,0.466667,0.705882}%
\pgfsetstrokecolor{currentstroke}%
\pgfsetstrokeopacity{0.944268}%
\pgfsetdash{}{0pt}%
\pgfpathmoveto{\pgfqpoint{2.016124in}{1.873869in}}%
\pgfpathcurveto{\pgfqpoint{2.024361in}{1.873869in}}{\pgfqpoint{2.032261in}{1.877141in}}{\pgfqpoint{2.038085in}{1.882965in}}%
\pgfpathcurveto{\pgfqpoint{2.043909in}{1.888789in}}{\pgfqpoint{2.047181in}{1.896689in}}{\pgfqpoint{2.047181in}{1.904926in}}%
\pgfpathcurveto{\pgfqpoint{2.047181in}{1.913162in}}{\pgfqpoint{2.043909in}{1.921062in}}{\pgfqpoint{2.038085in}{1.926886in}}%
\pgfpathcurveto{\pgfqpoint{2.032261in}{1.932710in}}{\pgfqpoint{2.024361in}{1.935982in}}{\pgfqpoint{2.016124in}{1.935982in}}%
\pgfpathcurveto{\pgfqpoint{2.007888in}{1.935982in}}{\pgfqpoint{1.999988in}{1.932710in}}{\pgfqpoint{1.994164in}{1.926886in}}%
\pgfpathcurveto{\pgfqpoint{1.988340in}{1.921062in}}{\pgfqpoint{1.985068in}{1.913162in}}{\pgfqpoint{1.985068in}{1.904926in}}%
\pgfpathcurveto{\pgfqpoint{1.985068in}{1.896689in}}{\pgfqpoint{1.988340in}{1.888789in}}{\pgfqpoint{1.994164in}{1.882965in}}%
\pgfpathcurveto{\pgfqpoint{1.999988in}{1.877141in}}{\pgfqpoint{2.007888in}{1.873869in}}{\pgfqpoint{2.016124in}{1.873869in}}%
\pgfpathclose%
\pgfusepath{stroke,fill}%
\end{pgfscope}%
\begin{pgfscope}%
\pgfpathrectangle{\pgfqpoint{0.100000in}{0.212622in}}{\pgfqpoint{3.696000in}{3.696000in}}%
\pgfusepath{clip}%
\pgfsetbuttcap%
\pgfsetroundjoin%
\definecolor{currentfill}{rgb}{0.121569,0.466667,0.705882}%
\pgfsetfillcolor{currentfill}%
\pgfsetfillopacity{0.944391}%
\pgfsetlinewidth{1.003750pt}%
\definecolor{currentstroke}{rgb}{0.121569,0.466667,0.705882}%
\pgfsetstrokecolor{currentstroke}%
\pgfsetstrokeopacity{0.944391}%
\pgfsetdash{}{0pt}%
\pgfpathmoveto{\pgfqpoint{2.016879in}{1.873421in}}%
\pgfpathcurveto{\pgfqpoint{2.025115in}{1.873421in}}{\pgfqpoint{2.033015in}{1.876693in}}{\pgfqpoint{2.038839in}{1.882517in}}%
\pgfpathcurveto{\pgfqpoint{2.044663in}{1.888341in}}{\pgfqpoint{2.047936in}{1.896241in}}{\pgfqpoint{2.047936in}{1.904478in}}%
\pgfpathcurveto{\pgfqpoint{2.047936in}{1.912714in}}{\pgfqpoint{2.044663in}{1.920614in}}{\pgfqpoint{2.038839in}{1.926438in}}%
\pgfpathcurveto{\pgfqpoint{2.033015in}{1.932262in}}{\pgfqpoint{2.025115in}{1.935534in}}{\pgfqpoint{2.016879in}{1.935534in}}%
\pgfpathcurveto{\pgfqpoint{2.008643in}{1.935534in}}{\pgfqpoint{2.000743in}{1.932262in}}{\pgfqpoint{1.994919in}{1.926438in}}%
\pgfpathcurveto{\pgfqpoint{1.989095in}{1.920614in}}{\pgfqpoint{1.985823in}{1.912714in}}{\pgfqpoint{1.985823in}{1.904478in}}%
\pgfpathcurveto{\pgfqpoint{1.985823in}{1.896241in}}{\pgfqpoint{1.989095in}{1.888341in}}{\pgfqpoint{1.994919in}{1.882517in}}%
\pgfpathcurveto{\pgfqpoint{2.000743in}{1.876693in}}{\pgfqpoint{2.008643in}{1.873421in}}{\pgfqpoint{2.016879in}{1.873421in}}%
\pgfpathclose%
\pgfusepath{stroke,fill}%
\end{pgfscope}%
\begin{pgfscope}%
\pgfpathrectangle{\pgfqpoint{0.100000in}{0.212622in}}{\pgfqpoint{3.696000in}{3.696000in}}%
\pgfusepath{clip}%
\pgfsetbuttcap%
\pgfsetroundjoin%
\definecolor{currentfill}{rgb}{0.121569,0.466667,0.705882}%
\pgfsetfillcolor{currentfill}%
\pgfsetfillopacity{0.944422}%
\pgfsetlinewidth{1.003750pt}%
\definecolor{currentstroke}{rgb}{0.121569,0.466667,0.705882}%
\pgfsetstrokecolor{currentstroke}%
\pgfsetstrokeopacity{0.944422}%
\pgfsetdash{}{0pt}%
\pgfpathmoveto{\pgfqpoint{2.087260in}{1.900545in}}%
\pgfpathcurveto{\pgfqpoint{2.095497in}{1.900545in}}{\pgfqpoint{2.103397in}{1.903817in}}{\pgfqpoint{2.109221in}{1.909641in}}%
\pgfpathcurveto{\pgfqpoint{2.115045in}{1.915465in}}{\pgfqpoint{2.118317in}{1.923365in}}{\pgfqpoint{2.118317in}{1.931601in}}%
\pgfpathcurveto{\pgfqpoint{2.118317in}{1.939838in}}{\pgfqpoint{2.115045in}{1.947738in}}{\pgfqpoint{2.109221in}{1.953562in}}%
\pgfpathcurveto{\pgfqpoint{2.103397in}{1.959386in}}{\pgfqpoint{2.095497in}{1.962658in}}{\pgfqpoint{2.087260in}{1.962658in}}%
\pgfpathcurveto{\pgfqpoint{2.079024in}{1.962658in}}{\pgfqpoint{2.071124in}{1.959386in}}{\pgfqpoint{2.065300in}{1.953562in}}%
\pgfpathcurveto{\pgfqpoint{2.059476in}{1.947738in}}{\pgfqpoint{2.056204in}{1.939838in}}{\pgfqpoint{2.056204in}{1.931601in}}%
\pgfpathcurveto{\pgfqpoint{2.056204in}{1.923365in}}{\pgfqpoint{2.059476in}{1.915465in}}{\pgfqpoint{2.065300in}{1.909641in}}%
\pgfpathcurveto{\pgfqpoint{2.071124in}{1.903817in}}{\pgfqpoint{2.079024in}{1.900545in}}{\pgfqpoint{2.087260in}{1.900545in}}%
\pgfpathclose%
\pgfusepath{stroke,fill}%
\end{pgfscope}%
\begin{pgfscope}%
\pgfpathrectangle{\pgfqpoint{0.100000in}{0.212622in}}{\pgfqpoint{3.696000in}{3.696000in}}%
\pgfusepath{clip}%
\pgfsetbuttcap%
\pgfsetroundjoin%
\definecolor{currentfill}{rgb}{0.121569,0.466667,0.705882}%
\pgfsetfillcolor{currentfill}%
\pgfsetfillopacity{0.944509}%
\pgfsetlinewidth{1.003750pt}%
\definecolor{currentstroke}{rgb}{0.121569,0.466667,0.705882}%
\pgfsetstrokecolor{currentstroke}%
\pgfsetstrokeopacity{0.944509}%
\pgfsetdash{}{0pt}%
\pgfpathmoveto{\pgfqpoint{2.087112in}{1.900130in}}%
\pgfpathcurveto{\pgfqpoint{2.095348in}{1.900130in}}{\pgfqpoint{2.103249in}{1.903402in}}{\pgfqpoint{2.109072in}{1.909226in}}%
\pgfpathcurveto{\pgfqpoint{2.114896in}{1.915050in}}{\pgfqpoint{2.118169in}{1.922950in}}{\pgfqpoint{2.118169in}{1.931187in}}%
\pgfpathcurveto{\pgfqpoint{2.118169in}{1.939423in}}{\pgfqpoint{2.114896in}{1.947323in}}{\pgfqpoint{2.109072in}{1.953147in}}%
\pgfpathcurveto{\pgfqpoint{2.103249in}{1.958971in}}{\pgfqpoint{2.095348in}{1.962243in}}{\pgfqpoint{2.087112in}{1.962243in}}%
\pgfpathcurveto{\pgfqpoint{2.078876in}{1.962243in}}{\pgfqpoint{2.070976in}{1.958971in}}{\pgfqpoint{2.065152in}{1.953147in}}%
\pgfpathcurveto{\pgfqpoint{2.059328in}{1.947323in}}{\pgfqpoint{2.056056in}{1.939423in}}{\pgfqpoint{2.056056in}{1.931187in}}%
\pgfpathcurveto{\pgfqpoint{2.056056in}{1.922950in}}{\pgfqpoint{2.059328in}{1.915050in}}{\pgfqpoint{2.065152in}{1.909226in}}%
\pgfpathcurveto{\pgfqpoint{2.070976in}{1.903402in}}{\pgfqpoint{2.078876in}{1.900130in}}{\pgfqpoint{2.087112in}{1.900130in}}%
\pgfpathclose%
\pgfusepath{stroke,fill}%
\end{pgfscope}%
\begin{pgfscope}%
\pgfpathrectangle{\pgfqpoint{0.100000in}{0.212622in}}{\pgfqpoint{3.696000in}{3.696000in}}%
\pgfusepath{clip}%
\pgfsetbuttcap%
\pgfsetroundjoin%
\definecolor{currentfill}{rgb}{0.121569,0.466667,0.705882}%
\pgfsetfillcolor{currentfill}%
\pgfsetfillopacity{0.944600}%
\pgfsetlinewidth{1.003750pt}%
\definecolor{currentstroke}{rgb}{0.121569,0.466667,0.705882}%
\pgfsetstrokecolor{currentstroke}%
\pgfsetstrokeopacity{0.944600}%
\pgfsetdash{}{0pt}%
\pgfpathmoveto{\pgfqpoint{2.018200in}{1.872982in}}%
\pgfpathcurveto{\pgfqpoint{2.026436in}{1.872982in}}{\pgfqpoint{2.034336in}{1.876254in}}{\pgfqpoint{2.040160in}{1.882078in}}%
\pgfpathcurveto{\pgfqpoint{2.045984in}{1.887902in}}{\pgfqpoint{2.049256in}{1.895802in}}{\pgfqpoint{2.049256in}{1.904038in}}%
\pgfpathcurveto{\pgfqpoint{2.049256in}{1.912275in}}{\pgfqpoint{2.045984in}{1.920175in}}{\pgfqpoint{2.040160in}{1.925999in}}%
\pgfpathcurveto{\pgfqpoint{2.034336in}{1.931823in}}{\pgfqpoint{2.026436in}{1.935095in}}{\pgfqpoint{2.018200in}{1.935095in}}%
\pgfpathcurveto{\pgfqpoint{2.009964in}{1.935095in}}{\pgfqpoint{2.002064in}{1.931823in}}{\pgfqpoint{1.996240in}{1.925999in}}%
\pgfpathcurveto{\pgfqpoint{1.990416in}{1.920175in}}{\pgfqpoint{1.987143in}{1.912275in}}{\pgfqpoint{1.987143in}{1.904038in}}%
\pgfpathcurveto{\pgfqpoint{1.987143in}{1.895802in}}{\pgfqpoint{1.990416in}{1.887902in}}{\pgfqpoint{1.996240in}{1.882078in}}%
\pgfpathcurveto{\pgfqpoint{2.002064in}{1.876254in}}{\pgfqpoint{2.009964in}{1.872982in}}{\pgfqpoint{2.018200in}{1.872982in}}%
\pgfpathclose%
\pgfusepath{stroke,fill}%
\end{pgfscope}%
\begin{pgfscope}%
\pgfpathrectangle{\pgfqpoint{0.100000in}{0.212622in}}{\pgfqpoint{3.696000in}{3.696000in}}%
\pgfusepath{clip}%
\pgfsetbuttcap%
\pgfsetroundjoin%
\definecolor{currentfill}{rgb}{0.121569,0.466667,0.705882}%
\pgfsetfillcolor{currentfill}%
\pgfsetfillopacity{0.944653}%
\pgfsetlinewidth{1.003750pt}%
\definecolor{currentstroke}{rgb}{0.121569,0.466667,0.705882}%
\pgfsetstrokecolor{currentstroke}%
\pgfsetstrokeopacity{0.944653}%
\pgfsetdash{}{0pt}%
\pgfpathmoveto{\pgfqpoint{2.086719in}{1.899414in}}%
\pgfpathcurveto{\pgfqpoint{2.094955in}{1.899414in}}{\pgfqpoint{2.102855in}{1.902686in}}{\pgfqpoint{2.108679in}{1.908510in}}%
\pgfpathcurveto{\pgfqpoint{2.114503in}{1.914334in}}{\pgfqpoint{2.117776in}{1.922234in}}{\pgfqpoint{2.117776in}{1.930471in}}%
\pgfpathcurveto{\pgfqpoint{2.117776in}{1.938707in}}{\pgfqpoint{2.114503in}{1.946607in}}{\pgfqpoint{2.108679in}{1.952431in}}%
\pgfpathcurveto{\pgfqpoint{2.102855in}{1.958255in}}{\pgfqpoint{2.094955in}{1.961527in}}{\pgfqpoint{2.086719in}{1.961527in}}%
\pgfpathcurveto{\pgfqpoint{2.078483in}{1.961527in}}{\pgfqpoint{2.070583in}{1.958255in}}{\pgfqpoint{2.064759in}{1.952431in}}%
\pgfpathcurveto{\pgfqpoint{2.058935in}{1.946607in}}{\pgfqpoint{2.055663in}{1.938707in}}{\pgfqpoint{2.055663in}{1.930471in}}%
\pgfpathcurveto{\pgfqpoint{2.055663in}{1.922234in}}{\pgfqpoint{2.058935in}{1.914334in}}{\pgfqpoint{2.064759in}{1.908510in}}%
\pgfpathcurveto{\pgfqpoint{2.070583in}{1.902686in}}{\pgfqpoint{2.078483in}{1.899414in}}{\pgfqpoint{2.086719in}{1.899414in}}%
\pgfpathclose%
\pgfusepath{stroke,fill}%
\end{pgfscope}%
\begin{pgfscope}%
\pgfpathrectangle{\pgfqpoint{0.100000in}{0.212622in}}{\pgfqpoint{3.696000in}{3.696000in}}%
\pgfusepath{clip}%
\pgfsetbuttcap%
\pgfsetroundjoin%
\definecolor{currentfill}{rgb}{0.121569,0.466667,0.705882}%
\pgfsetfillcolor{currentfill}%
\pgfsetfillopacity{0.944800}%
\pgfsetlinewidth{1.003750pt}%
\definecolor{currentstroke}{rgb}{0.121569,0.466667,0.705882}%
\pgfsetstrokecolor{currentstroke}%
\pgfsetstrokeopacity{0.944800}%
\pgfsetdash{}{0pt}%
\pgfpathmoveto{\pgfqpoint{2.091851in}{0.978615in}}%
\pgfpathcurveto{\pgfqpoint{2.100088in}{0.978615in}}{\pgfqpoint{2.107988in}{0.981888in}}{\pgfqpoint{2.113811in}{0.987711in}}%
\pgfpathcurveto{\pgfqpoint{2.119635in}{0.993535in}}{\pgfqpoint{2.122908in}{1.001435in}}{\pgfqpoint{2.122908in}{1.009672in}}%
\pgfpathcurveto{\pgfqpoint{2.122908in}{1.017908in}}{\pgfqpoint{2.119635in}{1.025808in}}{\pgfqpoint{2.113811in}{1.031632in}}%
\pgfpathcurveto{\pgfqpoint{2.107988in}{1.037456in}}{\pgfqpoint{2.100088in}{1.040728in}}{\pgfqpoint{2.091851in}{1.040728in}}%
\pgfpathcurveto{\pgfqpoint{2.083615in}{1.040728in}}{\pgfqpoint{2.075715in}{1.037456in}}{\pgfqpoint{2.069891in}{1.031632in}}%
\pgfpathcurveto{\pgfqpoint{2.064067in}{1.025808in}}{\pgfqpoint{2.060795in}{1.017908in}}{\pgfqpoint{2.060795in}{1.009672in}}%
\pgfpathcurveto{\pgfqpoint{2.060795in}{1.001435in}}{\pgfqpoint{2.064067in}{0.993535in}}{\pgfqpoint{2.069891in}{0.987711in}}%
\pgfpathcurveto{\pgfqpoint{2.075715in}{0.981888in}}{\pgfqpoint{2.083615in}{0.978615in}}{\pgfqpoint{2.091851in}{0.978615in}}%
\pgfpathclose%
\pgfusepath{stroke,fill}%
\end{pgfscope}%
\begin{pgfscope}%
\pgfpathrectangle{\pgfqpoint{0.100000in}{0.212622in}}{\pgfqpoint{3.696000in}{3.696000in}}%
\pgfusepath{clip}%
\pgfsetbuttcap%
\pgfsetroundjoin%
\definecolor{currentfill}{rgb}{0.121569,0.466667,0.705882}%
\pgfsetfillcolor{currentfill}%
\pgfsetfillopacity{0.944858}%
\pgfsetlinewidth{1.003750pt}%
\definecolor{currentstroke}{rgb}{0.121569,0.466667,0.705882}%
\pgfsetstrokecolor{currentstroke}%
\pgfsetstrokeopacity{0.944858}%
\pgfsetdash{}{0pt}%
\pgfpathmoveto{\pgfqpoint{2.019764in}{1.872175in}}%
\pgfpathcurveto{\pgfqpoint{2.028001in}{1.872175in}}{\pgfqpoint{2.035901in}{1.875447in}}{\pgfqpoint{2.041725in}{1.881271in}}%
\pgfpathcurveto{\pgfqpoint{2.047548in}{1.887095in}}{\pgfqpoint{2.050821in}{1.894995in}}{\pgfqpoint{2.050821in}{1.903231in}}%
\pgfpathcurveto{\pgfqpoint{2.050821in}{1.911467in}}{\pgfqpoint{2.047548in}{1.919367in}}{\pgfqpoint{2.041725in}{1.925191in}}%
\pgfpathcurveto{\pgfqpoint{2.035901in}{1.931015in}}{\pgfqpoint{2.028001in}{1.934288in}}{\pgfqpoint{2.019764in}{1.934288in}}%
\pgfpathcurveto{\pgfqpoint{2.011528in}{1.934288in}}{\pgfqpoint{2.003628in}{1.931015in}}{\pgfqpoint{1.997804in}{1.925191in}}%
\pgfpathcurveto{\pgfqpoint{1.991980in}{1.919367in}}{\pgfqpoint{1.988708in}{1.911467in}}{\pgfqpoint{1.988708in}{1.903231in}}%
\pgfpathcurveto{\pgfqpoint{1.988708in}{1.894995in}}{\pgfqpoint{1.991980in}{1.887095in}}{\pgfqpoint{1.997804in}{1.881271in}}%
\pgfpathcurveto{\pgfqpoint{2.003628in}{1.875447in}}{\pgfqpoint{2.011528in}{1.872175in}}{\pgfqpoint{2.019764in}{1.872175in}}%
\pgfpathclose%
\pgfusepath{stroke,fill}%
\end{pgfscope}%
\begin{pgfscope}%
\pgfpathrectangle{\pgfqpoint{0.100000in}{0.212622in}}{\pgfqpoint{3.696000in}{3.696000in}}%
\pgfusepath{clip}%
\pgfsetbuttcap%
\pgfsetroundjoin%
\definecolor{currentfill}{rgb}{0.121569,0.466667,0.705882}%
\pgfsetfillcolor{currentfill}%
\pgfsetfillopacity{0.944904}%
\pgfsetlinewidth{1.003750pt}%
\definecolor{currentstroke}{rgb}{0.121569,0.466667,0.705882}%
\pgfsetstrokecolor{currentstroke}%
\pgfsetstrokeopacity{0.944904}%
\pgfsetdash{}{0pt}%
\pgfpathmoveto{\pgfqpoint{2.085984in}{1.898089in}}%
\pgfpathcurveto{\pgfqpoint{2.094221in}{1.898089in}}{\pgfqpoint{2.102121in}{1.901361in}}{\pgfqpoint{2.107945in}{1.907185in}}%
\pgfpathcurveto{\pgfqpoint{2.113769in}{1.913009in}}{\pgfqpoint{2.117041in}{1.920909in}}{\pgfqpoint{2.117041in}{1.929145in}}%
\pgfpathcurveto{\pgfqpoint{2.117041in}{1.937382in}}{\pgfqpoint{2.113769in}{1.945282in}}{\pgfqpoint{2.107945in}{1.951106in}}%
\pgfpathcurveto{\pgfqpoint{2.102121in}{1.956930in}}{\pgfqpoint{2.094221in}{1.960202in}}{\pgfqpoint{2.085984in}{1.960202in}}%
\pgfpathcurveto{\pgfqpoint{2.077748in}{1.960202in}}{\pgfqpoint{2.069848in}{1.956930in}}{\pgfqpoint{2.064024in}{1.951106in}}%
\pgfpathcurveto{\pgfqpoint{2.058200in}{1.945282in}}{\pgfqpoint{2.054928in}{1.937382in}}{\pgfqpoint{2.054928in}{1.929145in}}%
\pgfpathcurveto{\pgfqpoint{2.054928in}{1.920909in}}{\pgfqpoint{2.058200in}{1.913009in}}{\pgfqpoint{2.064024in}{1.907185in}}%
\pgfpathcurveto{\pgfqpoint{2.069848in}{1.901361in}}{\pgfqpoint{2.077748in}{1.898089in}}{\pgfqpoint{2.085984in}{1.898089in}}%
\pgfpathclose%
\pgfusepath{stroke,fill}%
\end{pgfscope}%
\begin{pgfscope}%
\pgfpathrectangle{\pgfqpoint{0.100000in}{0.212622in}}{\pgfqpoint{3.696000in}{3.696000in}}%
\pgfusepath{clip}%
\pgfsetbuttcap%
\pgfsetroundjoin%
\definecolor{currentfill}{rgb}{0.121569,0.466667,0.705882}%
\pgfsetfillcolor{currentfill}%
\pgfsetfillopacity{0.945149}%
\pgfsetlinewidth{1.003750pt}%
\definecolor{currentstroke}{rgb}{0.121569,0.466667,0.705882}%
\pgfsetstrokecolor{currentstroke}%
\pgfsetstrokeopacity{0.945149}%
\pgfsetdash{}{0pt}%
\pgfpathmoveto{\pgfqpoint{2.021894in}{1.870773in}}%
\pgfpathcurveto{\pgfqpoint{2.030130in}{1.870773in}}{\pgfqpoint{2.038030in}{1.874045in}}{\pgfqpoint{2.043854in}{1.879869in}}%
\pgfpathcurveto{\pgfqpoint{2.049678in}{1.885693in}}{\pgfqpoint{2.052950in}{1.893593in}}{\pgfqpoint{2.052950in}{1.901830in}}%
\pgfpathcurveto{\pgfqpoint{2.052950in}{1.910066in}}{\pgfqpoint{2.049678in}{1.917966in}}{\pgfqpoint{2.043854in}{1.923790in}}%
\pgfpathcurveto{\pgfqpoint{2.038030in}{1.929614in}}{\pgfqpoint{2.030130in}{1.932886in}}{\pgfqpoint{2.021894in}{1.932886in}}%
\pgfpathcurveto{\pgfqpoint{2.013657in}{1.932886in}}{\pgfqpoint{2.005757in}{1.929614in}}{\pgfqpoint{1.999934in}{1.923790in}}%
\pgfpathcurveto{\pgfqpoint{1.994110in}{1.917966in}}{\pgfqpoint{1.990837in}{1.910066in}}{\pgfqpoint{1.990837in}{1.901830in}}%
\pgfpathcurveto{\pgfqpoint{1.990837in}{1.893593in}}{\pgfqpoint{1.994110in}{1.885693in}}{\pgfqpoint{1.999934in}{1.879869in}}%
\pgfpathcurveto{\pgfqpoint{2.005757in}{1.874045in}}{\pgfqpoint{2.013657in}{1.870773in}}{\pgfqpoint{2.021894in}{1.870773in}}%
\pgfpathclose%
\pgfusepath{stroke,fill}%
\end{pgfscope}%
\begin{pgfscope}%
\pgfpathrectangle{\pgfqpoint{0.100000in}{0.212622in}}{\pgfqpoint{3.696000in}{3.696000in}}%
\pgfusepath{clip}%
\pgfsetbuttcap%
\pgfsetroundjoin%
\definecolor{currentfill}{rgb}{0.121569,0.466667,0.705882}%
\pgfsetfillcolor{currentfill}%
\pgfsetfillopacity{0.945324}%
\pgfsetlinewidth{1.003750pt}%
\definecolor{currentstroke}{rgb}{0.121569,0.466667,0.705882}%
\pgfsetstrokecolor{currentstroke}%
\pgfsetstrokeopacity{0.945324}%
\pgfsetdash{}{0pt}%
\pgfpathmoveto{\pgfqpoint{2.023069in}{1.870079in}}%
\pgfpathcurveto{\pgfqpoint{2.031305in}{1.870079in}}{\pgfqpoint{2.039205in}{1.873351in}}{\pgfqpoint{2.045029in}{1.879175in}}%
\pgfpathcurveto{\pgfqpoint{2.050853in}{1.884999in}}{\pgfqpoint{2.054125in}{1.892899in}}{\pgfqpoint{2.054125in}{1.901136in}}%
\pgfpathcurveto{\pgfqpoint{2.054125in}{1.909372in}}{\pgfqpoint{2.050853in}{1.917272in}}{\pgfqpoint{2.045029in}{1.923096in}}%
\pgfpathcurveto{\pgfqpoint{2.039205in}{1.928920in}}{\pgfqpoint{2.031305in}{1.932192in}}{\pgfqpoint{2.023069in}{1.932192in}}%
\pgfpathcurveto{\pgfqpoint{2.014832in}{1.932192in}}{\pgfqpoint{2.006932in}{1.928920in}}{\pgfqpoint{2.001109in}{1.923096in}}%
\pgfpathcurveto{\pgfqpoint{1.995285in}{1.917272in}}{\pgfqpoint{1.992012in}{1.909372in}}{\pgfqpoint{1.992012in}{1.901136in}}%
\pgfpathcurveto{\pgfqpoint{1.992012in}{1.892899in}}{\pgfqpoint{1.995285in}{1.884999in}}{\pgfqpoint{2.001109in}{1.879175in}}%
\pgfpathcurveto{\pgfqpoint{2.006932in}{1.873351in}}{\pgfqpoint{2.014832in}{1.870079in}}{\pgfqpoint{2.023069in}{1.870079in}}%
\pgfpathclose%
\pgfusepath{stroke,fill}%
\end{pgfscope}%
\begin{pgfscope}%
\pgfpathrectangle{\pgfqpoint{0.100000in}{0.212622in}}{\pgfqpoint{3.696000in}{3.696000in}}%
\pgfusepath{clip}%
\pgfsetbuttcap%
\pgfsetroundjoin%
\definecolor{currentfill}{rgb}{0.121569,0.466667,0.705882}%
\pgfsetfillcolor{currentfill}%
\pgfsetfillopacity{0.945400}%
\pgfsetlinewidth{1.003750pt}%
\definecolor{currentstroke}{rgb}{0.121569,0.466667,0.705882}%
\pgfsetstrokecolor{currentstroke}%
\pgfsetstrokeopacity{0.945400}%
\pgfsetdash{}{0pt}%
\pgfpathmoveto{\pgfqpoint{2.084906in}{1.895592in}}%
\pgfpathcurveto{\pgfqpoint{2.093143in}{1.895592in}}{\pgfqpoint{2.101043in}{1.898864in}}{\pgfqpoint{2.106867in}{1.904688in}}%
\pgfpathcurveto{\pgfqpoint{2.112691in}{1.910512in}}{\pgfqpoint{2.115963in}{1.918412in}}{\pgfqpoint{2.115963in}{1.926649in}}%
\pgfpathcurveto{\pgfqpoint{2.115963in}{1.934885in}}{\pgfqpoint{2.112691in}{1.942785in}}{\pgfqpoint{2.106867in}{1.948609in}}%
\pgfpathcurveto{\pgfqpoint{2.101043in}{1.954433in}}{\pgfqpoint{2.093143in}{1.957705in}}{\pgfqpoint{2.084906in}{1.957705in}}%
\pgfpathcurveto{\pgfqpoint{2.076670in}{1.957705in}}{\pgfqpoint{2.068770in}{1.954433in}}{\pgfqpoint{2.062946in}{1.948609in}}%
\pgfpathcurveto{\pgfqpoint{2.057122in}{1.942785in}}{\pgfqpoint{2.053850in}{1.934885in}}{\pgfqpoint{2.053850in}{1.926649in}}%
\pgfpathcurveto{\pgfqpoint{2.053850in}{1.918412in}}{\pgfqpoint{2.057122in}{1.910512in}}{\pgfqpoint{2.062946in}{1.904688in}}%
\pgfpathcurveto{\pgfqpoint{2.068770in}{1.898864in}}{\pgfqpoint{2.076670in}{1.895592in}}{\pgfqpoint{2.084906in}{1.895592in}}%
\pgfpathclose%
\pgfusepath{stroke,fill}%
\end{pgfscope}%
\begin{pgfscope}%
\pgfpathrectangle{\pgfqpoint{0.100000in}{0.212622in}}{\pgfqpoint{3.696000in}{3.696000in}}%
\pgfusepath{clip}%
\pgfsetbuttcap%
\pgfsetroundjoin%
\definecolor{currentfill}{rgb}{0.121569,0.466667,0.705882}%
\pgfsetfillcolor{currentfill}%
\pgfsetfillopacity{0.945622}%
\pgfsetlinewidth{1.003750pt}%
\definecolor{currentstroke}{rgb}{0.121569,0.466667,0.705882}%
\pgfsetstrokecolor{currentstroke}%
\pgfsetstrokeopacity{0.945622}%
\pgfsetdash{}{0pt}%
\pgfpathmoveto{\pgfqpoint{2.025016in}{1.869322in}}%
\pgfpathcurveto{\pgfqpoint{2.033252in}{1.869322in}}{\pgfqpoint{2.041152in}{1.872594in}}{\pgfqpoint{2.046976in}{1.878418in}}%
\pgfpathcurveto{\pgfqpoint{2.052800in}{1.884242in}}{\pgfqpoint{2.056072in}{1.892142in}}{\pgfqpoint{2.056072in}{1.900378in}}%
\pgfpathcurveto{\pgfqpoint{2.056072in}{1.908615in}}{\pgfqpoint{2.052800in}{1.916515in}}{\pgfqpoint{2.046976in}{1.922339in}}%
\pgfpathcurveto{\pgfqpoint{2.041152in}{1.928162in}}{\pgfqpoint{2.033252in}{1.931435in}}{\pgfqpoint{2.025016in}{1.931435in}}%
\pgfpathcurveto{\pgfqpoint{2.016779in}{1.931435in}}{\pgfqpoint{2.008879in}{1.928162in}}{\pgfqpoint{2.003055in}{1.922339in}}%
\pgfpathcurveto{\pgfqpoint{1.997231in}{1.916515in}}{\pgfqpoint{1.993959in}{1.908615in}}{\pgfqpoint{1.993959in}{1.900378in}}%
\pgfpathcurveto{\pgfqpoint{1.993959in}{1.892142in}}{\pgfqpoint{1.997231in}{1.884242in}}{\pgfqpoint{2.003055in}{1.878418in}}%
\pgfpathcurveto{\pgfqpoint{2.008879in}{1.872594in}}{\pgfqpoint{2.016779in}{1.869322in}}{\pgfqpoint{2.025016in}{1.869322in}}%
\pgfpathclose%
\pgfusepath{stroke,fill}%
\end{pgfscope}%
\begin{pgfscope}%
\pgfpathrectangle{\pgfqpoint{0.100000in}{0.212622in}}{\pgfqpoint{3.696000in}{3.696000in}}%
\pgfusepath{clip}%
\pgfsetbuttcap%
\pgfsetroundjoin%
\definecolor{currentfill}{rgb}{0.121569,0.466667,0.705882}%
\pgfsetfillcolor{currentfill}%
\pgfsetfillopacity{0.945723}%
\pgfsetlinewidth{1.003750pt}%
\definecolor{currentstroke}{rgb}{0.121569,0.466667,0.705882}%
\pgfsetstrokecolor{currentstroke}%
\pgfsetstrokeopacity{0.945723}%
\pgfsetdash{}{0pt}%
\pgfpathmoveto{\pgfqpoint{2.679935in}{1.111283in}}%
\pgfpathcurveto{\pgfqpoint{2.688171in}{1.111283in}}{\pgfqpoint{2.696071in}{1.114555in}}{\pgfqpoint{2.701895in}{1.120379in}}%
\pgfpathcurveto{\pgfqpoint{2.707719in}{1.126203in}}{\pgfqpoint{2.710991in}{1.134103in}}{\pgfqpoint{2.710991in}{1.142339in}}%
\pgfpathcurveto{\pgfqpoint{2.710991in}{1.150575in}}{\pgfqpoint{2.707719in}{1.158476in}}{\pgfqpoint{2.701895in}{1.164299in}}%
\pgfpathcurveto{\pgfqpoint{2.696071in}{1.170123in}}{\pgfqpoint{2.688171in}{1.173396in}}{\pgfqpoint{2.679935in}{1.173396in}}%
\pgfpathcurveto{\pgfqpoint{2.671698in}{1.173396in}}{\pgfqpoint{2.663798in}{1.170123in}}{\pgfqpoint{2.657974in}{1.164299in}}%
\pgfpathcurveto{\pgfqpoint{2.652150in}{1.158476in}}{\pgfqpoint{2.648878in}{1.150575in}}{\pgfqpoint{2.648878in}{1.142339in}}%
\pgfpathcurveto{\pgfqpoint{2.648878in}{1.134103in}}{\pgfqpoint{2.652150in}{1.126203in}}{\pgfqpoint{2.657974in}{1.120379in}}%
\pgfpathcurveto{\pgfqpoint{2.663798in}{1.114555in}}{\pgfqpoint{2.671698in}{1.111283in}}{\pgfqpoint{2.679935in}{1.111283in}}%
\pgfpathclose%
\pgfusepath{stroke,fill}%
\end{pgfscope}%
\begin{pgfscope}%
\pgfpathrectangle{\pgfqpoint{0.100000in}{0.212622in}}{\pgfqpoint{3.696000in}{3.696000in}}%
\pgfusepath{clip}%
\pgfsetbuttcap%
\pgfsetroundjoin%
\definecolor{currentfill}{rgb}{0.121569,0.466667,0.705882}%
\pgfsetfillcolor{currentfill}%
\pgfsetfillopacity{0.945765}%
\pgfsetlinewidth{1.003750pt}%
\definecolor{currentstroke}{rgb}{0.121569,0.466667,0.705882}%
\pgfsetstrokecolor{currentstroke}%
\pgfsetstrokeopacity{0.945765}%
\pgfsetdash{}{0pt}%
\pgfpathmoveto{\pgfqpoint{2.083715in}{1.893600in}}%
\pgfpathcurveto{\pgfqpoint{2.091951in}{1.893600in}}{\pgfqpoint{2.099851in}{1.896872in}}{\pgfqpoint{2.105675in}{1.902696in}}%
\pgfpathcurveto{\pgfqpoint{2.111499in}{1.908520in}}{\pgfqpoint{2.114771in}{1.916420in}}{\pgfqpoint{2.114771in}{1.924656in}}%
\pgfpathcurveto{\pgfqpoint{2.114771in}{1.932892in}}{\pgfqpoint{2.111499in}{1.940792in}}{\pgfqpoint{2.105675in}{1.946616in}}%
\pgfpathcurveto{\pgfqpoint{2.099851in}{1.952440in}}{\pgfqpoint{2.091951in}{1.955713in}}{\pgfqpoint{2.083715in}{1.955713in}}%
\pgfpathcurveto{\pgfqpoint{2.075479in}{1.955713in}}{\pgfqpoint{2.067579in}{1.952440in}}{\pgfqpoint{2.061755in}{1.946616in}}%
\pgfpathcurveto{\pgfqpoint{2.055931in}{1.940792in}}{\pgfqpoint{2.052658in}{1.932892in}}{\pgfqpoint{2.052658in}{1.924656in}}%
\pgfpathcurveto{\pgfqpoint{2.052658in}{1.916420in}}{\pgfqpoint{2.055931in}{1.908520in}}{\pgfqpoint{2.061755in}{1.902696in}}%
\pgfpathcurveto{\pgfqpoint{2.067579in}{1.896872in}}{\pgfqpoint{2.075479in}{1.893600in}}{\pgfqpoint{2.083715in}{1.893600in}}%
\pgfpathclose%
\pgfusepath{stroke,fill}%
\end{pgfscope}%
\begin{pgfscope}%
\pgfpathrectangle{\pgfqpoint{0.100000in}{0.212622in}}{\pgfqpoint{3.696000in}{3.696000in}}%
\pgfusepath{clip}%
\pgfsetbuttcap%
\pgfsetroundjoin%
\definecolor{currentfill}{rgb}{0.121569,0.466667,0.705882}%
\pgfsetfillcolor{currentfill}%
\pgfsetfillopacity{0.945980}%
\pgfsetlinewidth{1.003750pt}%
\definecolor{currentstroke}{rgb}{0.121569,0.466667,0.705882}%
\pgfsetstrokecolor{currentstroke}%
\pgfsetstrokeopacity{0.945980}%
\pgfsetdash{}{0pt}%
\pgfpathmoveto{\pgfqpoint{2.027369in}{1.868761in}}%
\pgfpathcurveto{\pgfqpoint{2.035606in}{1.868761in}}{\pgfqpoint{2.043506in}{1.872033in}}{\pgfqpoint{2.049330in}{1.877857in}}%
\pgfpathcurveto{\pgfqpoint{2.055153in}{1.883681in}}{\pgfqpoint{2.058426in}{1.891581in}}{\pgfqpoint{2.058426in}{1.899817in}}%
\pgfpathcurveto{\pgfqpoint{2.058426in}{1.908054in}}{\pgfqpoint{2.055153in}{1.915954in}}{\pgfqpoint{2.049330in}{1.921778in}}%
\pgfpathcurveto{\pgfqpoint{2.043506in}{1.927602in}}{\pgfqpoint{2.035606in}{1.930874in}}{\pgfqpoint{2.027369in}{1.930874in}}%
\pgfpathcurveto{\pgfqpoint{2.019133in}{1.930874in}}{\pgfqpoint{2.011233in}{1.927602in}}{\pgfqpoint{2.005409in}{1.921778in}}%
\pgfpathcurveto{\pgfqpoint{1.999585in}{1.915954in}}{\pgfqpoint{1.996313in}{1.908054in}}{\pgfqpoint{1.996313in}{1.899817in}}%
\pgfpathcurveto{\pgfqpoint{1.996313in}{1.891581in}}{\pgfqpoint{1.999585in}{1.883681in}}{\pgfqpoint{2.005409in}{1.877857in}}%
\pgfpathcurveto{\pgfqpoint{2.011233in}{1.872033in}}{\pgfqpoint{2.019133in}{1.868761in}}{\pgfqpoint{2.027369in}{1.868761in}}%
\pgfpathclose%
\pgfusepath{stroke,fill}%
\end{pgfscope}%
\begin{pgfscope}%
\pgfpathrectangle{\pgfqpoint{0.100000in}{0.212622in}}{\pgfqpoint{3.696000in}{3.696000in}}%
\pgfusepath{clip}%
\pgfsetbuttcap%
\pgfsetroundjoin%
\definecolor{currentfill}{rgb}{0.121569,0.466667,0.705882}%
\pgfsetfillcolor{currentfill}%
\pgfsetfillopacity{0.946009}%
\pgfsetlinewidth{1.003750pt}%
\definecolor{currentstroke}{rgb}{0.121569,0.466667,0.705882}%
\pgfsetstrokecolor{currentstroke}%
\pgfsetstrokeopacity{0.946009}%
\pgfsetdash{}{0pt}%
\pgfpathmoveto{\pgfqpoint{2.083165in}{1.892240in}}%
\pgfpathcurveto{\pgfqpoint{2.091401in}{1.892240in}}{\pgfqpoint{2.099301in}{1.895513in}}{\pgfqpoint{2.105125in}{1.901337in}}%
\pgfpathcurveto{\pgfqpoint{2.110949in}{1.907161in}}{\pgfqpoint{2.114221in}{1.915061in}}{\pgfqpoint{2.114221in}{1.923297in}}%
\pgfpathcurveto{\pgfqpoint{2.114221in}{1.931533in}}{\pgfqpoint{2.110949in}{1.939433in}}{\pgfqpoint{2.105125in}{1.945257in}}%
\pgfpathcurveto{\pgfqpoint{2.099301in}{1.951081in}}{\pgfqpoint{2.091401in}{1.954353in}}{\pgfqpoint{2.083165in}{1.954353in}}%
\pgfpathcurveto{\pgfqpoint{2.074929in}{1.954353in}}{\pgfqpoint{2.067029in}{1.951081in}}{\pgfqpoint{2.061205in}{1.945257in}}%
\pgfpathcurveto{\pgfqpoint{2.055381in}{1.939433in}}{\pgfqpoint{2.052108in}{1.931533in}}{\pgfqpoint{2.052108in}{1.923297in}}%
\pgfpathcurveto{\pgfqpoint{2.052108in}{1.915061in}}{\pgfqpoint{2.055381in}{1.907161in}}{\pgfqpoint{2.061205in}{1.901337in}}%
\pgfpathcurveto{\pgfqpoint{2.067029in}{1.895513in}}{\pgfqpoint{2.074929in}{1.892240in}}{\pgfqpoint{2.083165in}{1.892240in}}%
\pgfpathclose%
\pgfusepath{stroke,fill}%
\end{pgfscope}%
\begin{pgfscope}%
\pgfpathrectangle{\pgfqpoint{0.100000in}{0.212622in}}{\pgfqpoint{3.696000in}{3.696000in}}%
\pgfusepath{clip}%
\pgfsetbuttcap%
\pgfsetroundjoin%
\definecolor{currentfill}{rgb}{0.121569,0.466667,0.705882}%
\pgfsetfillcolor{currentfill}%
\pgfsetfillopacity{0.946119}%
\pgfsetlinewidth{1.003750pt}%
\definecolor{currentstroke}{rgb}{0.121569,0.466667,0.705882}%
\pgfsetstrokecolor{currentstroke}%
\pgfsetstrokeopacity{0.946119}%
\pgfsetdash{}{0pt}%
\pgfpathmoveto{\pgfqpoint{2.082731in}{1.891507in}}%
\pgfpathcurveto{\pgfqpoint{2.090967in}{1.891507in}}{\pgfqpoint{2.098867in}{1.894780in}}{\pgfqpoint{2.104691in}{1.900603in}}%
\pgfpathcurveto{\pgfqpoint{2.110515in}{1.906427in}}{\pgfqpoint{2.113787in}{1.914327in}}{\pgfqpoint{2.113787in}{1.922564in}}%
\pgfpathcurveto{\pgfqpoint{2.113787in}{1.930800in}}{\pgfqpoint{2.110515in}{1.938700in}}{\pgfqpoint{2.104691in}{1.944524in}}%
\pgfpathcurveto{\pgfqpoint{2.098867in}{1.950348in}}{\pgfqpoint{2.090967in}{1.953620in}}{\pgfqpoint{2.082731in}{1.953620in}}%
\pgfpathcurveto{\pgfqpoint{2.074494in}{1.953620in}}{\pgfqpoint{2.066594in}{1.950348in}}{\pgfqpoint{2.060770in}{1.944524in}}%
\pgfpathcurveto{\pgfqpoint{2.054946in}{1.938700in}}{\pgfqpoint{2.051674in}{1.930800in}}{\pgfqpoint{2.051674in}{1.922564in}}%
\pgfpathcurveto{\pgfqpoint{2.051674in}{1.914327in}}{\pgfqpoint{2.054946in}{1.906427in}}{\pgfqpoint{2.060770in}{1.900603in}}%
\pgfpathcurveto{\pgfqpoint{2.066594in}{1.894780in}}{\pgfqpoint{2.074494in}{1.891507in}}{\pgfqpoint{2.082731in}{1.891507in}}%
\pgfpathclose%
\pgfusepath{stroke,fill}%
\end{pgfscope}%
\begin{pgfscope}%
\pgfpathrectangle{\pgfqpoint{0.100000in}{0.212622in}}{\pgfqpoint{3.696000in}{3.696000in}}%
\pgfusepath{clip}%
\pgfsetbuttcap%
\pgfsetroundjoin%
\definecolor{currentfill}{rgb}{0.121569,0.466667,0.705882}%
\pgfsetfillcolor{currentfill}%
\pgfsetfillopacity{0.946168}%
\pgfsetlinewidth{1.003750pt}%
\definecolor{currentstroke}{rgb}{0.121569,0.466667,0.705882}%
\pgfsetstrokecolor{currentstroke}%
\pgfsetstrokeopacity{0.946168}%
\pgfsetdash{}{0pt}%
\pgfpathmoveto{\pgfqpoint{2.082615in}{1.891239in}}%
\pgfpathcurveto{\pgfqpoint{2.090851in}{1.891239in}}{\pgfqpoint{2.098752in}{1.894511in}}{\pgfqpoint{2.104575in}{1.900335in}}%
\pgfpathcurveto{\pgfqpoint{2.110399in}{1.906159in}}{\pgfqpoint{2.113672in}{1.914059in}}{\pgfqpoint{2.113672in}{1.922295in}}%
\pgfpathcurveto{\pgfqpoint{2.113672in}{1.930532in}}{\pgfqpoint{2.110399in}{1.938432in}}{\pgfqpoint{2.104575in}{1.944256in}}%
\pgfpathcurveto{\pgfqpoint{2.098752in}{1.950080in}}{\pgfqpoint{2.090851in}{1.953352in}}{\pgfqpoint{2.082615in}{1.953352in}}%
\pgfpathcurveto{\pgfqpoint{2.074379in}{1.953352in}}{\pgfqpoint{2.066479in}{1.950080in}}{\pgfqpoint{2.060655in}{1.944256in}}%
\pgfpathcurveto{\pgfqpoint{2.054831in}{1.938432in}}{\pgfqpoint{2.051559in}{1.930532in}}{\pgfqpoint{2.051559in}{1.922295in}}%
\pgfpathcurveto{\pgfqpoint{2.051559in}{1.914059in}}{\pgfqpoint{2.054831in}{1.906159in}}{\pgfqpoint{2.060655in}{1.900335in}}%
\pgfpathcurveto{\pgfqpoint{2.066479in}{1.894511in}}{\pgfqpoint{2.074379in}{1.891239in}}{\pgfqpoint{2.082615in}{1.891239in}}%
\pgfpathclose%
\pgfusepath{stroke,fill}%
\end{pgfscope}%
\begin{pgfscope}%
\pgfpathrectangle{\pgfqpoint{0.100000in}{0.212622in}}{\pgfqpoint{3.696000in}{3.696000in}}%
\pgfusepath{clip}%
\pgfsetbuttcap%
\pgfsetroundjoin%
\definecolor{currentfill}{rgb}{0.121569,0.466667,0.705882}%
\pgfsetfillcolor{currentfill}%
\pgfsetfillopacity{0.946245}%
\pgfsetlinewidth{1.003750pt}%
\definecolor{currentstroke}{rgb}{0.121569,0.466667,0.705882}%
\pgfsetstrokecolor{currentstroke}%
\pgfsetstrokeopacity{0.946245}%
\pgfsetdash{}{0pt}%
\pgfpathmoveto{\pgfqpoint{2.082341in}{1.890771in}}%
\pgfpathcurveto{\pgfqpoint{2.090577in}{1.890771in}}{\pgfqpoint{2.098477in}{1.894044in}}{\pgfqpoint{2.104301in}{1.899867in}}%
\pgfpathcurveto{\pgfqpoint{2.110125in}{1.905691in}}{\pgfqpoint{2.113397in}{1.913591in}}{\pgfqpoint{2.113397in}{1.921828in}}%
\pgfpathcurveto{\pgfqpoint{2.113397in}{1.930064in}}{\pgfqpoint{2.110125in}{1.937964in}}{\pgfqpoint{2.104301in}{1.943788in}}%
\pgfpathcurveto{\pgfqpoint{2.098477in}{1.949612in}}{\pgfqpoint{2.090577in}{1.952884in}}{\pgfqpoint{2.082341in}{1.952884in}}%
\pgfpathcurveto{\pgfqpoint{2.074104in}{1.952884in}}{\pgfqpoint{2.066204in}{1.949612in}}{\pgfqpoint{2.060380in}{1.943788in}}%
\pgfpathcurveto{\pgfqpoint{2.054556in}{1.937964in}}{\pgfqpoint{2.051284in}{1.930064in}}{\pgfqpoint{2.051284in}{1.921828in}}%
\pgfpathcurveto{\pgfqpoint{2.051284in}{1.913591in}}{\pgfqpoint{2.054556in}{1.905691in}}{\pgfqpoint{2.060380in}{1.899867in}}%
\pgfpathcurveto{\pgfqpoint{2.066204in}{1.894044in}}{\pgfqpoint{2.074104in}{1.890771in}}{\pgfqpoint{2.082341in}{1.890771in}}%
\pgfpathclose%
\pgfusepath{stroke,fill}%
\end{pgfscope}%
\begin{pgfscope}%
\pgfpathrectangle{\pgfqpoint{0.100000in}{0.212622in}}{\pgfqpoint{3.696000in}{3.696000in}}%
\pgfusepath{clip}%
\pgfsetbuttcap%
\pgfsetroundjoin%
\definecolor{currentfill}{rgb}{0.121569,0.466667,0.705882}%
\pgfsetfillcolor{currentfill}%
\pgfsetfillopacity{0.946383}%
\pgfsetlinewidth{1.003750pt}%
\definecolor{currentstroke}{rgb}{0.121569,0.466667,0.705882}%
\pgfsetstrokecolor{currentstroke}%
\pgfsetstrokeopacity{0.946383}%
\pgfsetdash{}{0pt}%
\pgfpathmoveto{\pgfqpoint{2.081870in}{1.889871in}}%
\pgfpathcurveto{\pgfqpoint{2.090107in}{1.889871in}}{\pgfqpoint{2.098007in}{1.893144in}}{\pgfqpoint{2.103830in}{1.898968in}}%
\pgfpathcurveto{\pgfqpoint{2.109654in}{1.904792in}}{\pgfqpoint{2.112927in}{1.912692in}}{\pgfqpoint{2.112927in}{1.920928in}}%
\pgfpathcurveto{\pgfqpoint{2.112927in}{1.929164in}}{\pgfqpoint{2.109654in}{1.937064in}}{\pgfqpoint{2.103830in}{1.942888in}}%
\pgfpathcurveto{\pgfqpoint{2.098007in}{1.948712in}}{\pgfqpoint{2.090107in}{1.951984in}}{\pgfqpoint{2.081870in}{1.951984in}}%
\pgfpathcurveto{\pgfqpoint{2.073634in}{1.951984in}}{\pgfqpoint{2.065734in}{1.948712in}}{\pgfqpoint{2.059910in}{1.942888in}}%
\pgfpathcurveto{\pgfqpoint{2.054086in}{1.937064in}}{\pgfqpoint{2.050814in}{1.929164in}}{\pgfqpoint{2.050814in}{1.920928in}}%
\pgfpathcurveto{\pgfqpoint{2.050814in}{1.912692in}}{\pgfqpoint{2.054086in}{1.904792in}}{\pgfqpoint{2.059910in}{1.898968in}}%
\pgfpathcurveto{\pgfqpoint{2.065734in}{1.893144in}}{\pgfqpoint{2.073634in}{1.889871in}}{\pgfqpoint{2.081870in}{1.889871in}}%
\pgfpathclose%
\pgfusepath{stroke,fill}%
\end{pgfscope}%
\begin{pgfscope}%
\pgfpathrectangle{\pgfqpoint{0.100000in}{0.212622in}}{\pgfqpoint{3.696000in}{3.696000in}}%
\pgfusepath{clip}%
\pgfsetbuttcap%
\pgfsetroundjoin%
\definecolor{currentfill}{rgb}{0.121569,0.466667,0.705882}%
\pgfsetfillcolor{currentfill}%
\pgfsetfillopacity{0.946400}%
\pgfsetlinewidth{1.003750pt}%
\definecolor{currentstroke}{rgb}{0.121569,0.466667,0.705882}%
\pgfsetstrokecolor{currentstroke}%
\pgfsetstrokeopacity{0.946400}%
\pgfsetdash{}{0pt}%
\pgfpathmoveto{\pgfqpoint{2.029962in}{1.867990in}}%
\pgfpathcurveto{\pgfqpoint{2.038198in}{1.867990in}}{\pgfqpoint{2.046098in}{1.871262in}}{\pgfqpoint{2.051922in}{1.877086in}}%
\pgfpathcurveto{\pgfqpoint{2.057746in}{1.882910in}}{\pgfqpoint{2.061018in}{1.890810in}}{\pgfqpoint{2.061018in}{1.899046in}}%
\pgfpathcurveto{\pgfqpoint{2.061018in}{1.907283in}}{\pgfqpoint{2.057746in}{1.915183in}}{\pgfqpoint{2.051922in}{1.921007in}}%
\pgfpathcurveto{\pgfqpoint{2.046098in}{1.926830in}}{\pgfqpoint{2.038198in}{1.930103in}}{\pgfqpoint{2.029962in}{1.930103in}}%
\pgfpathcurveto{\pgfqpoint{2.021726in}{1.930103in}}{\pgfqpoint{2.013825in}{1.926830in}}{\pgfqpoint{2.008002in}{1.921007in}}%
\pgfpathcurveto{\pgfqpoint{2.002178in}{1.915183in}}{\pgfqpoint{1.998905in}{1.907283in}}{\pgfqpoint{1.998905in}{1.899046in}}%
\pgfpathcurveto{\pgfqpoint{1.998905in}{1.890810in}}{\pgfqpoint{2.002178in}{1.882910in}}{\pgfqpoint{2.008002in}{1.877086in}}%
\pgfpathcurveto{\pgfqpoint{2.013825in}{1.871262in}}{\pgfqpoint{2.021726in}{1.867990in}}{\pgfqpoint{2.029962in}{1.867990in}}%
\pgfpathclose%
\pgfusepath{stroke,fill}%
\end{pgfscope}%
\begin{pgfscope}%
\pgfpathrectangle{\pgfqpoint{0.100000in}{0.212622in}}{\pgfqpoint{3.696000in}{3.696000in}}%
\pgfusepath{clip}%
\pgfsetbuttcap%
\pgfsetroundjoin%
\definecolor{currentfill}{rgb}{0.121569,0.466667,0.705882}%
\pgfsetfillcolor{currentfill}%
\pgfsetfillopacity{0.946616}%
\pgfsetlinewidth{1.003750pt}%
\definecolor{currentstroke}{rgb}{0.121569,0.466667,0.705882}%
\pgfsetstrokecolor{currentstroke}%
\pgfsetstrokeopacity{0.946616}%
\pgfsetdash{}{0pt}%
\pgfpathmoveto{\pgfqpoint{2.031384in}{1.867487in}}%
\pgfpathcurveto{\pgfqpoint{2.039620in}{1.867487in}}{\pgfqpoint{2.047520in}{1.870759in}}{\pgfqpoint{2.053344in}{1.876583in}}%
\pgfpathcurveto{\pgfqpoint{2.059168in}{1.882407in}}{\pgfqpoint{2.062440in}{1.890307in}}{\pgfqpoint{2.062440in}{1.898543in}}%
\pgfpathcurveto{\pgfqpoint{2.062440in}{1.906780in}}{\pgfqpoint{2.059168in}{1.914680in}}{\pgfqpoint{2.053344in}{1.920504in}}%
\pgfpathcurveto{\pgfqpoint{2.047520in}{1.926328in}}{\pgfqpoint{2.039620in}{1.929600in}}{\pgfqpoint{2.031384in}{1.929600in}}%
\pgfpathcurveto{\pgfqpoint{2.023148in}{1.929600in}}{\pgfqpoint{2.015248in}{1.926328in}}{\pgfqpoint{2.009424in}{1.920504in}}%
\pgfpathcurveto{\pgfqpoint{2.003600in}{1.914680in}}{\pgfqpoint{2.000327in}{1.906780in}}{\pgfqpoint{2.000327in}{1.898543in}}%
\pgfpathcurveto{\pgfqpoint{2.000327in}{1.890307in}}{\pgfqpoint{2.003600in}{1.882407in}}{\pgfqpoint{2.009424in}{1.876583in}}%
\pgfpathcurveto{\pgfqpoint{2.015248in}{1.870759in}}{\pgfqpoint{2.023148in}{1.867487in}}{\pgfqpoint{2.031384in}{1.867487in}}%
\pgfpathclose%
\pgfusepath{stroke,fill}%
\end{pgfscope}%
\begin{pgfscope}%
\pgfpathrectangle{\pgfqpoint{0.100000in}{0.212622in}}{\pgfqpoint{3.696000in}{3.696000in}}%
\pgfusepath{clip}%
\pgfsetbuttcap%
\pgfsetroundjoin%
\definecolor{currentfill}{rgb}{0.121569,0.466667,0.705882}%
\pgfsetfillcolor{currentfill}%
\pgfsetfillopacity{0.946685}%
\pgfsetlinewidth{1.003750pt}%
\definecolor{currentstroke}{rgb}{0.121569,0.466667,0.705882}%
\pgfsetstrokecolor{currentstroke}%
\pgfsetstrokeopacity{0.946685}%
\pgfsetdash{}{0pt}%
\pgfpathmoveto{\pgfqpoint{2.081146in}{1.888317in}}%
\pgfpathcurveto{\pgfqpoint{2.089383in}{1.888317in}}{\pgfqpoint{2.097283in}{1.891590in}}{\pgfqpoint{2.103107in}{1.897414in}}%
\pgfpathcurveto{\pgfqpoint{2.108931in}{1.903238in}}{\pgfqpoint{2.112203in}{1.911138in}}{\pgfqpoint{2.112203in}{1.919374in}}%
\pgfpathcurveto{\pgfqpoint{2.112203in}{1.927610in}}{\pgfqpoint{2.108931in}{1.935510in}}{\pgfqpoint{2.103107in}{1.941334in}}%
\pgfpathcurveto{\pgfqpoint{2.097283in}{1.947158in}}{\pgfqpoint{2.089383in}{1.950430in}}{\pgfqpoint{2.081146in}{1.950430in}}%
\pgfpathcurveto{\pgfqpoint{2.072910in}{1.950430in}}{\pgfqpoint{2.065010in}{1.947158in}}{\pgfqpoint{2.059186in}{1.941334in}}%
\pgfpathcurveto{\pgfqpoint{2.053362in}{1.935510in}}{\pgfqpoint{2.050090in}{1.927610in}}{\pgfqpoint{2.050090in}{1.919374in}}%
\pgfpathcurveto{\pgfqpoint{2.050090in}{1.911138in}}{\pgfqpoint{2.053362in}{1.903238in}}{\pgfqpoint{2.059186in}{1.897414in}}%
\pgfpathcurveto{\pgfqpoint{2.065010in}{1.891590in}}{\pgfqpoint{2.072910in}{1.888317in}}{\pgfqpoint{2.081146in}{1.888317in}}%
\pgfpathclose%
\pgfusepath{stroke,fill}%
\end{pgfscope}%
\begin{pgfscope}%
\pgfpathrectangle{\pgfqpoint{0.100000in}{0.212622in}}{\pgfqpoint{3.696000in}{3.696000in}}%
\pgfusepath{clip}%
\pgfsetbuttcap%
\pgfsetroundjoin%
\definecolor{currentfill}{rgb}{0.121569,0.466667,0.705882}%
\pgfsetfillcolor{currentfill}%
\pgfsetfillopacity{0.946840}%
\pgfsetlinewidth{1.003750pt}%
\definecolor{currentstroke}{rgb}{0.121569,0.466667,0.705882}%
\pgfsetstrokecolor{currentstroke}%
\pgfsetstrokeopacity{0.946840}%
\pgfsetdash{}{0pt}%
\pgfpathmoveto{\pgfqpoint{2.080463in}{1.887237in}}%
\pgfpathcurveto{\pgfqpoint{2.088699in}{1.887237in}}{\pgfqpoint{2.096599in}{1.890509in}}{\pgfqpoint{2.102423in}{1.896333in}}%
\pgfpathcurveto{\pgfqpoint{2.108247in}{1.902157in}}{\pgfqpoint{2.111519in}{1.910057in}}{\pgfqpoint{2.111519in}{1.918293in}}%
\pgfpathcurveto{\pgfqpoint{2.111519in}{1.926529in}}{\pgfqpoint{2.108247in}{1.934429in}}{\pgfqpoint{2.102423in}{1.940253in}}%
\pgfpathcurveto{\pgfqpoint{2.096599in}{1.946077in}}{\pgfqpoint{2.088699in}{1.949350in}}{\pgfqpoint{2.080463in}{1.949350in}}%
\pgfpathcurveto{\pgfqpoint{2.072227in}{1.949350in}}{\pgfqpoint{2.064327in}{1.946077in}}{\pgfqpoint{2.058503in}{1.940253in}}%
\pgfpathcurveto{\pgfqpoint{2.052679in}{1.934429in}}{\pgfqpoint{2.049406in}{1.926529in}}{\pgfqpoint{2.049406in}{1.918293in}}%
\pgfpathcurveto{\pgfqpoint{2.049406in}{1.910057in}}{\pgfqpoint{2.052679in}{1.902157in}}{\pgfqpoint{2.058503in}{1.896333in}}%
\pgfpathcurveto{\pgfqpoint{2.064327in}{1.890509in}}{\pgfqpoint{2.072227in}{1.887237in}}{\pgfqpoint{2.080463in}{1.887237in}}%
\pgfpathclose%
\pgfusepath{stroke,fill}%
\end{pgfscope}%
\begin{pgfscope}%
\pgfpathrectangle{\pgfqpoint{0.100000in}{0.212622in}}{\pgfqpoint{3.696000in}{3.696000in}}%
\pgfusepath{clip}%
\pgfsetbuttcap%
\pgfsetroundjoin%
\definecolor{currentfill}{rgb}{0.121569,0.466667,0.705882}%
\pgfsetfillcolor{currentfill}%
\pgfsetfillopacity{0.946927}%
\pgfsetlinewidth{1.003750pt}%
\definecolor{currentstroke}{rgb}{0.121569,0.466667,0.705882}%
\pgfsetstrokecolor{currentstroke}%
\pgfsetstrokeopacity{0.946927}%
\pgfsetdash{}{0pt}%
\pgfpathmoveto{\pgfqpoint{2.080258in}{1.886844in}}%
\pgfpathcurveto{\pgfqpoint{2.088494in}{1.886844in}}{\pgfqpoint{2.096394in}{1.890116in}}{\pgfqpoint{2.102218in}{1.895940in}}%
\pgfpathcurveto{\pgfqpoint{2.108042in}{1.901764in}}{\pgfqpoint{2.111314in}{1.909664in}}{\pgfqpoint{2.111314in}{1.917900in}}%
\pgfpathcurveto{\pgfqpoint{2.111314in}{1.926137in}}{\pgfqpoint{2.108042in}{1.934037in}}{\pgfqpoint{2.102218in}{1.939860in}}%
\pgfpathcurveto{\pgfqpoint{2.096394in}{1.945684in}}{\pgfqpoint{2.088494in}{1.948957in}}{\pgfqpoint{2.080258in}{1.948957in}}%
\pgfpathcurveto{\pgfqpoint{2.072022in}{1.948957in}}{\pgfqpoint{2.064122in}{1.945684in}}{\pgfqpoint{2.058298in}{1.939860in}}%
\pgfpathcurveto{\pgfqpoint{2.052474in}{1.934037in}}{\pgfqpoint{2.049201in}{1.926137in}}{\pgfqpoint{2.049201in}{1.917900in}}%
\pgfpathcurveto{\pgfqpoint{2.049201in}{1.909664in}}{\pgfqpoint{2.052474in}{1.901764in}}{\pgfqpoint{2.058298in}{1.895940in}}%
\pgfpathcurveto{\pgfqpoint{2.064122in}{1.890116in}}{\pgfqpoint{2.072022in}{1.886844in}}{\pgfqpoint{2.080258in}{1.886844in}}%
\pgfpathclose%
\pgfusepath{stroke,fill}%
\end{pgfscope}%
\begin{pgfscope}%
\pgfpathrectangle{\pgfqpoint{0.100000in}{0.212622in}}{\pgfqpoint{3.696000in}{3.696000in}}%
\pgfusepath{clip}%
\pgfsetbuttcap%
\pgfsetroundjoin%
\definecolor{currentfill}{rgb}{0.121569,0.466667,0.705882}%
\pgfsetfillcolor{currentfill}%
\pgfsetfillopacity{0.946930}%
\pgfsetlinewidth{1.003750pt}%
\definecolor{currentstroke}{rgb}{0.121569,0.466667,0.705882}%
\pgfsetstrokecolor{currentstroke}%
\pgfsetstrokeopacity{0.946930}%
\pgfsetdash{}{0pt}%
\pgfpathmoveto{\pgfqpoint{2.080244in}{1.886821in}}%
\pgfpathcurveto{\pgfqpoint{2.088480in}{1.886821in}}{\pgfqpoint{2.096380in}{1.890093in}}{\pgfqpoint{2.102204in}{1.895917in}}%
\pgfpathcurveto{\pgfqpoint{2.108028in}{1.901741in}}{\pgfqpoint{2.111300in}{1.909641in}}{\pgfqpoint{2.111300in}{1.917877in}}%
\pgfpathcurveto{\pgfqpoint{2.111300in}{1.926114in}}{\pgfqpoint{2.108028in}{1.934014in}}{\pgfqpoint{2.102204in}{1.939838in}}%
\pgfpathcurveto{\pgfqpoint{2.096380in}{1.945662in}}{\pgfqpoint{2.088480in}{1.948934in}}{\pgfqpoint{2.080244in}{1.948934in}}%
\pgfpathcurveto{\pgfqpoint{2.072008in}{1.948934in}}{\pgfqpoint{2.064108in}{1.945662in}}{\pgfqpoint{2.058284in}{1.939838in}}%
\pgfpathcurveto{\pgfqpoint{2.052460in}{1.934014in}}{\pgfqpoint{2.049187in}{1.926114in}}{\pgfqpoint{2.049187in}{1.917877in}}%
\pgfpathcurveto{\pgfqpoint{2.049187in}{1.909641in}}{\pgfqpoint{2.052460in}{1.901741in}}{\pgfqpoint{2.058284in}{1.895917in}}%
\pgfpathcurveto{\pgfqpoint{2.064108in}{1.890093in}}{\pgfqpoint{2.072008in}{1.886821in}}{\pgfqpoint{2.080244in}{1.886821in}}%
\pgfpathclose%
\pgfusepath{stroke,fill}%
\end{pgfscope}%
\begin{pgfscope}%
\pgfpathrectangle{\pgfqpoint{0.100000in}{0.212622in}}{\pgfqpoint{3.696000in}{3.696000in}}%
\pgfusepath{clip}%
\pgfsetbuttcap%
\pgfsetroundjoin%
\definecolor{currentfill}{rgb}{0.121569,0.466667,0.705882}%
\pgfsetfillcolor{currentfill}%
\pgfsetfillopacity{0.946939}%
\pgfsetlinewidth{1.003750pt}%
\definecolor{currentstroke}{rgb}{0.121569,0.466667,0.705882}%
\pgfsetstrokecolor{currentstroke}%
\pgfsetstrokeopacity{0.946939}%
\pgfsetdash{}{0pt}%
\pgfpathmoveto{\pgfqpoint{2.080224in}{1.886781in}}%
\pgfpathcurveto{\pgfqpoint{2.088461in}{1.886781in}}{\pgfqpoint{2.096361in}{1.890053in}}{\pgfqpoint{2.102185in}{1.895877in}}%
\pgfpathcurveto{\pgfqpoint{2.108008in}{1.901701in}}{\pgfqpoint{2.111281in}{1.909601in}}{\pgfqpoint{2.111281in}{1.917837in}}%
\pgfpathcurveto{\pgfqpoint{2.111281in}{1.926074in}}{\pgfqpoint{2.108008in}{1.933974in}}{\pgfqpoint{2.102185in}{1.939798in}}%
\pgfpathcurveto{\pgfqpoint{2.096361in}{1.945621in}}{\pgfqpoint{2.088461in}{1.948894in}}{\pgfqpoint{2.080224in}{1.948894in}}%
\pgfpathcurveto{\pgfqpoint{2.071988in}{1.948894in}}{\pgfqpoint{2.064088in}{1.945621in}}{\pgfqpoint{2.058264in}{1.939798in}}%
\pgfpathcurveto{\pgfqpoint{2.052440in}{1.933974in}}{\pgfqpoint{2.049168in}{1.926074in}}{\pgfqpoint{2.049168in}{1.917837in}}%
\pgfpathcurveto{\pgfqpoint{2.049168in}{1.909601in}}{\pgfqpoint{2.052440in}{1.901701in}}{\pgfqpoint{2.058264in}{1.895877in}}%
\pgfpathcurveto{\pgfqpoint{2.064088in}{1.890053in}}{\pgfqpoint{2.071988in}{1.886781in}}{\pgfqpoint{2.080224in}{1.886781in}}%
\pgfpathclose%
\pgfusepath{stroke,fill}%
\end{pgfscope}%
\begin{pgfscope}%
\pgfpathrectangle{\pgfqpoint{0.100000in}{0.212622in}}{\pgfqpoint{3.696000in}{3.696000in}}%
\pgfusepath{clip}%
\pgfsetbuttcap%
\pgfsetroundjoin%
\definecolor{currentfill}{rgb}{0.121569,0.466667,0.705882}%
\pgfsetfillcolor{currentfill}%
\pgfsetfillopacity{0.946943}%
\pgfsetlinewidth{1.003750pt}%
\definecolor{currentstroke}{rgb}{0.121569,0.466667,0.705882}%
\pgfsetstrokecolor{currentstroke}%
\pgfsetstrokeopacity{0.946943}%
\pgfsetdash{}{0pt}%
\pgfpathmoveto{\pgfqpoint{2.033593in}{1.866446in}}%
\pgfpathcurveto{\pgfqpoint{2.041829in}{1.866446in}}{\pgfqpoint{2.049730in}{1.869719in}}{\pgfqpoint{2.055553in}{1.875543in}}%
\pgfpathcurveto{\pgfqpoint{2.061377in}{1.881367in}}{\pgfqpoint{2.064650in}{1.889267in}}{\pgfqpoint{2.064650in}{1.897503in}}%
\pgfpathcurveto{\pgfqpoint{2.064650in}{1.905739in}}{\pgfqpoint{2.061377in}{1.913639in}}{\pgfqpoint{2.055553in}{1.919463in}}%
\pgfpathcurveto{\pgfqpoint{2.049730in}{1.925287in}}{\pgfqpoint{2.041829in}{1.928559in}}{\pgfqpoint{2.033593in}{1.928559in}}%
\pgfpathcurveto{\pgfqpoint{2.025357in}{1.928559in}}{\pgfqpoint{2.017457in}{1.925287in}}{\pgfqpoint{2.011633in}{1.919463in}}%
\pgfpathcurveto{\pgfqpoint{2.005809in}{1.913639in}}{\pgfqpoint{2.002537in}{1.905739in}}{\pgfqpoint{2.002537in}{1.897503in}}%
\pgfpathcurveto{\pgfqpoint{2.002537in}{1.889267in}}{\pgfqpoint{2.005809in}{1.881367in}}{\pgfqpoint{2.011633in}{1.875543in}}%
\pgfpathcurveto{\pgfqpoint{2.017457in}{1.869719in}}{\pgfqpoint{2.025357in}{1.866446in}}{\pgfqpoint{2.033593in}{1.866446in}}%
\pgfpathclose%
\pgfusepath{stroke,fill}%
\end{pgfscope}%
\begin{pgfscope}%
\pgfpathrectangle{\pgfqpoint{0.100000in}{0.212622in}}{\pgfqpoint{3.696000in}{3.696000in}}%
\pgfusepath{clip}%
\pgfsetbuttcap%
\pgfsetroundjoin%
\definecolor{currentfill}{rgb}{0.121569,0.466667,0.705882}%
\pgfsetfillcolor{currentfill}%
\pgfsetfillopacity{0.946950}%
\pgfsetlinewidth{1.003750pt}%
\definecolor{currentstroke}{rgb}{0.121569,0.466667,0.705882}%
\pgfsetstrokecolor{currentstroke}%
\pgfsetstrokeopacity{0.946950}%
\pgfsetdash{}{0pt}%
\pgfpathmoveto{\pgfqpoint{2.080178in}{1.886708in}}%
\pgfpathcurveto{\pgfqpoint{2.088415in}{1.886708in}}{\pgfqpoint{2.096315in}{1.889980in}}{\pgfqpoint{2.102139in}{1.895804in}}%
\pgfpathcurveto{\pgfqpoint{2.107962in}{1.901628in}}{\pgfqpoint{2.111235in}{1.909528in}}{\pgfqpoint{2.111235in}{1.917764in}}%
\pgfpathcurveto{\pgfqpoint{2.111235in}{1.926001in}}{\pgfqpoint{2.107962in}{1.933901in}}{\pgfqpoint{2.102139in}{1.939725in}}%
\pgfpathcurveto{\pgfqpoint{2.096315in}{1.945549in}}{\pgfqpoint{2.088415in}{1.948821in}}{\pgfqpoint{2.080178in}{1.948821in}}%
\pgfpathcurveto{\pgfqpoint{2.071942in}{1.948821in}}{\pgfqpoint{2.064042in}{1.945549in}}{\pgfqpoint{2.058218in}{1.939725in}}%
\pgfpathcurveto{\pgfqpoint{2.052394in}{1.933901in}}{\pgfqpoint{2.049122in}{1.926001in}}{\pgfqpoint{2.049122in}{1.917764in}}%
\pgfpathcurveto{\pgfqpoint{2.049122in}{1.909528in}}{\pgfqpoint{2.052394in}{1.901628in}}{\pgfqpoint{2.058218in}{1.895804in}}%
\pgfpathcurveto{\pgfqpoint{2.064042in}{1.889980in}}{\pgfqpoint{2.071942in}{1.886708in}}{\pgfqpoint{2.080178in}{1.886708in}}%
\pgfpathclose%
\pgfusepath{stroke,fill}%
\end{pgfscope}%
\begin{pgfscope}%
\pgfpathrectangle{\pgfqpoint{0.100000in}{0.212622in}}{\pgfqpoint{3.696000in}{3.696000in}}%
\pgfusepath{clip}%
\pgfsetbuttcap%
\pgfsetroundjoin%
\definecolor{currentfill}{rgb}{0.121569,0.466667,0.705882}%
\pgfsetfillcolor{currentfill}%
\pgfsetfillopacity{0.946975}%
\pgfsetlinewidth{1.003750pt}%
\definecolor{currentstroke}{rgb}{0.121569,0.466667,0.705882}%
\pgfsetstrokecolor{currentstroke}%
\pgfsetstrokeopacity{0.946975}%
\pgfsetdash{}{0pt}%
\pgfpathmoveto{\pgfqpoint{2.080110in}{1.886569in}}%
\pgfpathcurveto{\pgfqpoint{2.088346in}{1.886569in}}{\pgfqpoint{2.096246in}{1.889842in}}{\pgfqpoint{2.102070in}{1.895665in}}%
\pgfpathcurveto{\pgfqpoint{2.107894in}{1.901489in}}{\pgfqpoint{2.111167in}{1.909389in}}{\pgfqpoint{2.111167in}{1.917626in}}%
\pgfpathcurveto{\pgfqpoint{2.111167in}{1.925862in}}{\pgfqpoint{2.107894in}{1.933762in}}{\pgfqpoint{2.102070in}{1.939586in}}%
\pgfpathcurveto{\pgfqpoint{2.096246in}{1.945410in}}{\pgfqpoint{2.088346in}{1.948682in}}{\pgfqpoint{2.080110in}{1.948682in}}%
\pgfpathcurveto{\pgfqpoint{2.071874in}{1.948682in}}{\pgfqpoint{2.063974in}{1.945410in}}{\pgfqpoint{2.058150in}{1.939586in}}%
\pgfpathcurveto{\pgfqpoint{2.052326in}{1.933762in}}{\pgfqpoint{2.049054in}{1.925862in}}{\pgfqpoint{2.049054in}{1.917626in}}%
\pgfpathcurveto{\pgfqpoint{2.049054in}{1.909389in}}{\pgfqpoint{2.052326in}{1.901489in}}{\pgfqpoint{2.058150in}{1.895665in}}%
\pgfpathcurveto{\pgfqpoint{2.063974in}{1.889842in}}{\pgfqpoint{2.071874in}{1.886569in}}{\pgfqpoint{2.080110in}{1.886569in}}%
\pgfpathclose%
\pgfusepath{stroke,fill}%
\end{pgfscope}%
\begin{pgfscope}%
\pgfpathrectangle{\pgfqpoint{0.100000in}{0.212622in}}{\pgfqpoint{3.696000in}{3.696000in}}%
\pgfusepath{clip}%
\pgfsetbuttcap%
\pgfsetroundjoin%
\definecolor{currentfill}{rgb}{0.121569,0.466667,0.705882}%
\pgfsetfillcolor{currentfill}%
\pgfsetfillopacity{0.947019}%
\pgfsetlinewidth{1.003750pt}%
\definecolor{currentstroke}{rgb}{0.121569,0.466667,0.705882}%
\pgfsetstrokecolor{currentstroke}%
\pgfsetstrokeopacity{0.947019}%
\pgfsetdash{}{0pt}%
\pgfpathmoveto{\pgfqpoint{2.079979in}{1.886318in}}%
\pgfpathcurveto{\pgfqpoint{2.088215in}{1.886318in}}{\pgfqpoint{2.096115in}{1.889590in}}{\pgfqpoint{2.101939in}{1.895414in}}%
\pgfpathcurveto{\pgfqpoint{2.107763in}{1.901238in}}{\pgfqpoint{2.111035in}{1.909138in}}{\pgfqpoint{2.111035in}{1.917374in}}%
\pgfpathcurveto{\pgfqpoint{2.111035in}{1.925611in}}{\pgfqpoint{2.107763in}{1.933511in}}{\pgfqpoint{2.101939in}{1.939335in}}%
\pgfpathcurveto{\pgfqpoint{2.096115in}{1.945159in}}{\pgfqpoint{2.088215in}{1.948431in}}{\pgfqpoint{2.079979in}{1.948431in}}%
\pgfpathcurveto{\pgfqpoint{2.071743in}{1.948431in}}{\pgfqpoint{2.063843in}{1.945159in}}{\pgfqpoint{2.058019in}{1.939335in}}%
\pgfpathcurveto{\pgfqpoint{2.052195in}{1.933511in}}{\pgfqpoint{2.048923in}{1.925611in}}{\pgfqpoint{2.048923in}{1.917374in}}%
\pgfpathcurveto{\pgfqpoint{2.048923in}{1.909138in}}{\pgfqpoint{2.052195in}{1.901238in}}{\pgfqpoint{2.058019in}{1.895414in}}%
\pgfpathcurveto{\pgfqpoint{2.063843in}{1.889590in}}{\pgfqpoint{2.071743in}{1.886318in}}{\pgfqpoint{2.079979in}{1.886318in}}%
\pgfpathclose%
\pgfusepath{stroke,fill}%
\end{pgfscope}%
\begin{pgfscope}%
\pgfpathrectangle{\pgfqpoint{0.100000in}{0.212622in}}{\pgfqpoint{3.696000in}{3.696000in}}%
\pgfusepath{clip}%
\pgfsetbuttcap%
\pgfsetroundjoin%
\definecolor{currentfill}{rgb}{0.121569,0.466667,0.705882}%
\pgfsetfillcolor{currentfill}%
\pgfsetfillopacity{0.947093}%
\pgfsetlinewidth{1.003750pt}%
\definecolor{currentstroke}{rgb}{0.121569,0.466667,0.705882}%
\pgfsetstrokecolor{currentstroke}%
\pgfsetstrokeopacity{0.947093}%
\pgfsetdash{}{0pt}%
\pgfpathmoveto{\pgfqpoint{2.034831in}{1.865827in}}%
\pgfpathcurveto{\pgfqpoint{2.043067in}{1.865827in}}{\pgfqpoint{2.050967in}{1.869100in}}{\pgfqpoint{2.056791in}{1.874924in}}%
\pgfpathcurveto{\pgfqpoint{2.062615in}{1.880748in}}{\pgfqpoint{2.065888in}{1.888648in}}{\pgfqpoint{2.065888in}{1.896884in}}%
\pgfpathcurveto{\pgfqpoint{2.065888in}{1.905120in}}{\pgfqpoint{2.062615in}{1.913020in}}{\pgfqpoint{2.056791in}{1.918844in}}%
\pgfpathcurveto{\pgfqpoint{2.050967in}{1.924668in}}{\pgfqpoint{2.043067in}{1.927940in}}{\pgfqpoint{2.034831in}{1.927940in}}%
\pgfpathcurveto{\pgfqpoint{2.026595in}{1.927940in}}{\pgfqpoint{2.018695in}{1.924668in}}{\pgfqpoint{2.012871in}{1.918844in}}%
\pgfpathcurveto{\pgfqpoint{2.007047in}{1.913020in}}{\pgfqpoint{2.003775in}{1.905120in}}{\pgfqpoint{2.003775in}{1.896884in}}%
\pgfpathcurveto{\pgfqpoint{2.003775in}{1.888648in}}{\pgfqpoint{2.007047in}{1.880748in}}{\pgfqpoint{2.012871in}{1.874924in}}%
\pgfpathcurveto{\pgfqpoint{2.018695in}{1.869100in}}{\pgfqpoint{2.026595in}{1.865827in}}{\pgfqpoint{2.034831in}{1.865827in}}%
\pgfpathclose%
\pgfusepath{stroke,fill}%
\end{pgfscope}%
\begin{pgfscope}%
\pgfpathrectangle{\pgfqpoint{0.100000in}{0.212622in}}{\pgfqpoint{3.696000in}{3.696000in}}%
\pgfusepath{clip}%
\pgfsetbuttcap%
\pgfsetroundjoin%
\definecolor{currentfill}{rgb}{0.121569,0.466667,0.705882}%
\pgfsetfillcolor{currentfill}%
\pgfsetfillopacity{0.947097}%
\pgfsetlinewidth{1.003750pt}%
\definecolor{currentstroke}{rgb}{0.121569,0.466667,0.705882}%
\pgfsetstrokecolor{currentstroke}%
\pgfsetstrokeopacity{0.947097}%
\pgfsetdash{}{0pt}%
\pgfpathmoveto{\pgfqpoint{2.079716in}{1.885886in}}%
\pgfpathcurveto{\pgfqpoint{2.087952in}{1.885886in}}{\pgfqpoint{2.095853in}{1.889159in}}{\pgfqpoint{2.101676in}{1.894983in}}%
\pgfpathcurveto{\pgfqpoint{2.107500in}{1.900806in}}{\pgfqpoint{2.110773in}{1.908707in}}{\pgfqpoint{2.110773in}{1.916943in}}%
\pgfpathcurveto{\pgfqpoint{2.110773in}{1.925179in}}{\pgfqpoint{2.107500in}{1.933079in}}{\pgfqpoint{2.101676in}{1.938903in}}%
\pgfpathcurveto{\pgfqpoint{2.095853in}{1.944727in}}{\pgfqpoint{2.087952in}{1.947999in}}{\pgfqpoint{2.079716in}{1.947999in}}%
\pgfpathcurveto{\pgfqpoint{2.071480in}{1.947999in}}{\pgfqpoint{2.063580in}{1.944727in}}{\pgfqpoint{2.057756in}{1.938903in}}%
\pgfpathcurveto{\pgfqpoint{2.051932in}{1.933079in}}{\pgfqpoint{2.048660in}{1.925179in}}{\pgfqpoint{2.048660in}{1.916943in}}%
\pgfpathcurveto{\pgfqpoint{2.048660in}{1.908707in}}{\pgfqpoint{2.051932in}{1.900806in}}{\pgfqpoint{2.057756in}{1.894983in}}%
\pgfpathcurveto{\pgfqpoint{2.063580in}{1.889159in}}{\pgfqpoint{2.071480in}{1.885886in}}{\pgfqpoint{2.079716in}{1.885886in}}%
\pgfpathclose%
\pgfusepath{stroke,fill}%
\end{pgfscope}%
\begin{pgfscope}%
\pgfpathrectangle{\pgfqpoint{0.100000in}{0.212622in}}{\pgfqpoint{3.696000in}{3.696000in}}%
\pgfusepath{clip}%
\pgfsetbuttcap%
\pgfsetroundjoin%
\definecolor{currentfill}{rgb}{0.121569,0.466667,0.705882}%
\pgfsetfillcolor{currentfill}%
\pgfsetfillopacity{0.947253}%
\pgfsetlinewidth{1.003750pt}%
\definecolor{currentstroke}{rgb}{0.121569,0.466667,0.705882}%
\pgfsetstrokecolor{currentstroke}%
\pgfsetstrokeopacity{0.947253}%
\pgfsetdash{}{0pt}%
\pgfpathmoveto{\pgfqpoint{2.079332in}{1.885055in}}%
\pgfpathcurveto{\pgfqpoint{2.087568in}{1.885055in}}{\pgfqpoint{2.095468in}{1.888327in}}{\pgfqpoint{2.101292in}{1.894151in}}%
\pgfpathcurveto{\pgfqpoint{2.107116in}{1.899975in}}{\pgfqpoint{2.110388in}{1.907875in}}{\pgfqpoint{2.110388in}{1.916111in}}%
\pgfpathcurveto{\pgfqpoint{2.110388in}{1.924348in}}{\pgfqpoint{2.107116in}{1.932248in}}{\pgfqpoint{2.101292in}{1.938072in}}%
\pgfpathcurveto{\pgfqpoint{2.095468in}{1.943895in}}{\pgfqpoint{2.087568in}{1.947168in}}{\pgfqpoint{2.079332in}{1.947168in}}%
\pgfpathcurveto{\pgfqpoint{2.071095in}{1.947168in}}{\pgfqpoint{2.063195in}{1.943895in}}{\pgfqpoint{2.057371in}{1.938072in}}%
\pgfpathcurveto{\pgfqpoint{2.051547in}{1.932248in}}{\pgfqpoint{2.048275in}{1.924348in}}{\pgfqpoint{2.048275in}{1.916111in}}%
\pgfpathcurveto{\pgfqpoint{2.048275in}{1.907875in}}{\pgfqpoint{2.051547in}{1.899975in}}{\pgfqpoint{2.057371in}{1.894151in}}%
\pgfpathcurveto{\pgfqpoint{2.063195in}{1.888327in}}{\pgfqpoint{2.071095in}{1.885055in}}{\pgfqpoint{2.079332in}{1.885055in}}%
\pgfpathclose%
\pgfusepath{stroke,fill}%
\end{pgfscope}%
\begin{pgfscope}%
\pgfpathrectangle{\pgfqpoint{0.100000in}{0.212622in}}{\pgfqpoint{3.696000in}{3.696000in}}%
\pgfusepath{clip}%
\pgfsetbuttcap%
\pgfsetroundjoin%
\definecolor{currentfill}{rgb}{0.121569,0.466667,0.705882}%
\pgfsetfillcolor{currentfill}%
\pgfsetfillopacity{0.947357}%
\pgfsetlinewidth{1.003750pt}%
\definecolor{currentstroke}{rgb}{0.121569,0.466667,0.705882}%
\pgfsetstrokecolor{currentstroke}%
\pgfsetstrokeopacity{0.947357}%
\pgfsetdash{}{0pt}%
\pgfpathmoveto{\pgfqpoint{2.036403in}{1.865399in}}%
\pgfpathcurveto{\pgfqpoint{2.044639in}{1.865399in}}{\pgfqpoint{2.052539in}{1.868671in}}{\pgfqpoint{2.058363in}{1.874495in}}%
\pgfpathcurveto{\pgfqpoint{2.064187in}{1.880319in}}{\pgfqpoint{2.067459in}{1.888219in}}{\pgfqpoint{2.067459in}{1.896455in}}%
\pgfpathcurveto{\pgfqpoint{2.067459in}{1.904691in}}{\pgfqpoint{2.064187in}{1.912591in}}{\pgfqpoint{2.058363in}{1.918415in}}%
\pgfpathcurveto{\pgfqpoint{2.052539in}{1.924239in}}{\pgfqpoint{2.044639in}{1.927512in}}{\pgfqpoint{2.036403in}{1.927512in}}%
\pgfpathcurveto{\pgfqpoint{2.028166in}{1.927512in}}{\pgfqpoint{2.020266in}{1.924239in}}{\pgfqpoint{2.014442in}{1.918415in}}%
\pgfpathcurveto{\pgfqpoint{2.008618in}{1.912591in}}{\pgfqpoint{2.005346in}{1.904691in}}{\pgfqpoint{2.005346in}{1.896455in}}%
\pgfpathcurveto{\pgfqpoint{2.005346in}{1.888219in}}{\pgfqpoint{2.008618in}{1.880319in}}{\pgfqpoint{2.014442in}{1.874495in}}%
\pgfpathcurveto{\pgfqpoint{2.020266in}{1.868671in}}{\pgfqpoint{2.028166in}{1.865399in}}{\pgfqpoint{2.036403in}{1.865399in}}%
\pgfpathclose%
\pgfusepath{stroke,fill}%
\end{pgfscope}%
\begin{pgfscope}%
\pgfpathrectangle{\pgfqpoint{0.100000in}{0.212622in}}{\pgfqpoint{3.696000in}{3.696000in}}%
\pgfusepath{clip}%
\pgfsetbuttcap%
\pgfsetroundjoin%
\definecolor{currentfill}{rgb}{0.121569,0.466667,0.705882}%
\pgfsetfillcolor{currentfill}%
\pgfsetfillopacity{0.947487}%
\pgfsetlinewidth{1.003750pt}%
\definecolor{currentstroke}{rgb}{0.121569,0.466667,0.705882}%
\pgfsetstrokecolor{currentstroke}%
\pgfsetstrokeopacity{0.947487}%
\pgfsetdash{}{0pt}%
\pgfpathmoveto{\pgfqpoint{2.037247in}{1.865024in}}%
\pgfpathcurveto{\pgfqpoint{2.045484in}{1.865024in}}{\pgfqpoint{2.053384in}{1.868297in}}{\pgfqpoint{2.059208in}{1.874121in}}%
\pgfpathcurveto{\pgfqpoint{2.065032in}{1.879945in}}{\pgfqpoint{2.068304in}{1.887845in}}{\pgfqpoint{2.068304in}{1.896081in}}%
\pgfpathcurveto{\pgfqpoint{2.068304in}{1.904317in}}{\pgfqpoint{2.065032in}{1.912217in}}{\pgfqpoint{2.059208in}{1.918041in}}%
\pgfpathcurveto{\pgfqpoint{2.053384in}{1.923865in}}{\pgfqpoint{2.045484in}{1.927137in}}{\pgfqpoint{2.037247in}{1.927137in}}%
\pgfpathcurveto{\pgfqpoint{2.029011in}{1.927137in}}{\pgfqpoint{2.021111in}{1.923865in}}{\pgfqpoint{2.015287in}{1.918041in}}%
\pgfpathcurveto{\pgfqpoint{2.009463in}{1.912217in}}{\pgfqpoint{2.006191in}{1.904317in}}{\pgfqpoint{2.006191in}{1.896081in}}%
\pgfpathcurveto{\pgfqpoint{2.006191in}{1.887845in}}{\pgfqpoint{2.009463in}{1.879945in}}{\pgfqpoint{2.015287in}{1.874121in}}%
\pgfpathcurveto{\pgfqpoint{2.021111in}{1.868297in}}{\pgfqpoint{2.029011in}{1.865024in}}{\pgfqpoint{2.037247in}{1.865024in}}%
\pgfpathclose%
\pgfusepath{stroke,fill}%
\end{pgfscope}%
\begin{pgfscope}%
\pgfpathrectangle{\pgfqpoint{0.100000in}{0.212622in}}{\pgfqpoint{3.696000in}{3.696000in}}%
\pgfusepath{clip}%
\pgfsetbuttcap%
\pgfsetroundjoin%
\definecolor{currentfill}{rgb}{0.121569,0.466667,0.705882}%
\pgfsetfillcolor{currentfill}%
\pgfsetfillopacity{0.947496}%
\pgfsetlinewidth{1.003750pt}%
\definecolor{currentstroke}{rgb}{0.121569,0.466667,0.705882}%
\pgfsetstrokecolor{currentstroke}%
\pgfsetstrokeopacity{0.947496}%
\pgfsetdash{}{0pt}%
\pgfpathmoveto{\pgfqpoint{2.078420in}{1.883644in}}%
\pgfpathcurveto{\pgfqpoint{2.086656in}{1.883644in}}{\pgfqpoint{2.094556in}{1.886916in}}{\pgfqpoint{2.100380in}{1.892740in}}%
\pgfpathcurveto{\pgfqpoint{2.106204in}{1.898564in}}{\pgfqpoint{2.109477in}{1.906464in}}{\pgfqpoint{2.109477in}{1.914700in}}%
\pgfpathcurveto{\pgfqpoint{2.109477in}{1.922937in}}{\pgfqpoint{2.106204in}{1.930837in}}{\pgfqpoint{2.100380in}{1.936661in}}%
\pgfpathcurveto{\pgfqpoint{2.094556in}{1.942485in}}{\pgfqpoint{2.086656in}{1.945757in}}{\pgfqpoint{2.078420in}{1.945757in}}%
\pgfpathcurveto{\pgfqpoint{2.070184in}{1.945757in}}{\pgfqpoint{2.062284in}{1.942485in}}{\pgfqpoint{2.056460in}{1.936661in}}%
\pgfpathcurveto{\pgfqpoint{2.050636in}{1.930837in}}{\pgfqpoint{2.047364in}{1.922937in}}{\pgfqpoint{2.047364in}{1.914700in}}%
\pgfpathcurveto{\pgfqpoint{2.047364in}{1.906464in}}{\pgfqpoint{2.050636in}{1.898564in}}{\pgfqpoint{2.056460in}{1.892740in}}%
\pgfpathcurveto{\pgfqpoint{2.062284in}{1.886916in}}{\pgfqpoint{2.070184in}{1.883644in}}{\pgfqpoint{2.078420in}{1.883644in}}%
\pgfpathclose%
\pgfusepath{stroke,fill}%
\end{pgfscope}%
\begin{pgfscope}%
\pgfpathrectangle{\pgfqpoint{0.100000in}{0.212622in}}{\pgfqpoint{3.696000in}{3.696000in}}%
\pgfusepath{clip}%
\pgfsetbuttcap%
\pgfsetroundjoin%
\definecolor{currentfill}{rgb}{0.121569,0.466667,0.705882}%
\pgfsetfillcolor{currentfill}%
\pgfsetfillopacity{0.947496}%
\pgfsetlinewidth{1.003750pt}%
\definecolor{currentstroke}{rgb}{0.121569,0.466667,0.705882}%
\pgfsetstrokecolor{currentstroke}%
\pgfsetstrokeopacity{0.947496}%
\pgfsetdash{}{0pt}%
\pgfpathmoveto{\pgfqpoint{2.078420in}{1.883644in}}%
\pgfpathcurveto{\pgfqpoint{2.086656in}{1.883644in}}{\pgfqpoint{2.094556in}{1.886916in}}{\pgfqpoint{2.100380in}{1.892740in}}%
\pgfpathcurveto{\pgfqpoint{2.106204in}{1.898564in}}{\pgfqpoint{2.109477in}{1.906464in}}{\pgfqpoint{2.109477in}{1.914700in}}%
\pgfpathcurveto{\pgfqpoint{2.109477in}{1.922937in}}{\pgfqpoint{2.106204in}{1.930837in}}{\pgfqpoint{2.100380in}{1.936661in}}%
\pgfpathcurveto{\pgfqpoint{2.094556in}{1.942485in}}{\pgfqpoint{2.086656in}{1.945757in}}{\pgfqpoint{2.078420in}{1.945757in}}%
\pgfpathcurveto{\pgfqpoint{2.070184in}{1.945757in}}{\pgfqpoint{2.062284in}{1.942485in}}{\pgfqpoint{2.056460in}{1.936661in}}%
\pgfpathcurveto{\pgfqpoint{2.050636in}{1.930837in}}{\pgfqpoint{2.047364in}{1.922937in}}{\pgfqpoint{2.047364in}{1.914700in}}%
\pgfpathcurveto{\pgfqpoint{2.047364in}{1.906464in}}{\pgfqpoint{2.050636in}{1.898564in}}{\pgfqpoint{2.056460in}{1.892740in}}%
\pgfpathcurveto{\pgfqpoint{2.062284in}{1.886916in}}{\pgfqpoint{2.070184in}{1.883644in}}{\pgfqpoint{2.078420in}{1.883644in}}%
\pgfpathclose%
\pgfusepath{stroke,fill}%
\end{pgfscope}%
\begin{pgfscope}%
\pgfpathrectangle{\pgfqpoint{0.100000in}{0.212622in}}{\pgfqpoint{3.696000in}{3.696000in}}%
\pgfusepath{clip}%
\pgfsetbuttcap%
\pgfsetroundjoin%
\definecolor{currentfill}{rgb}{0.121569,0.466667,0.705882}%
\pgfsetfillcolor{currentfill}%
\pgfsetfillopacity{0.947496}%
\pgfsetlinewidth{1.003750pt}%
\definecolor{currentstroke}{rgb}{0.121569,0.466667,0.705882}%
\pgfsetstrokecolor{currentstroke}%
\pgfsetstrokeopacity{0.947496}%
\pgfsetdash{}{0pt}%
\pgfpathmoveto{\pgfqpoint{2.078420in}{1.883644in}}%
\pgfpathcurveto{\pgfqpoint{2.086656in}{1.883644in}}{\pgfqpoint{2.094556in}{1.886916in}}{\pgfqpoint{2.100380in}{1.892740in}}%
\pgfpathcurveto{\pgfqpoint{2.106204in}{1.898564in}}{\pgfqpoint{2.109477in}{1.906464in}}{\pgfqpoint{2.109477in}{1.914700in}}%
\pgfpathcurveto{\pgfqpoint{2.109477in}{1.922937in}}{\pgfqpoint{2.106204in}{1.930837in}}{\pgfqpoint{2.100380in}{1.936661in}}%
\pgfpathcurveto{\pgfqpoint{2.094556in}{1.942485in}}{\pgfqpoint{2.086656in}{1.945757in}}{\pgfqpoint{2.078420in}{1.945757in}}%
\pgfpathcurveto{\pgfqpoint{2.070184in}{1.945757in}}{\pgfqpoint{2.062284in}{1.942485in}}{\pgfqpoint{2.056460in}{1.936661in}}%
\pgfpathcurveto{\pgfqpoint{2.050636in}{1.930837in}}{\pgfqpoint{2.047364in}{1.922937in}}{\pgfqpoint{2.047364in}{1.914700in}}%
\pgfpathcurveto{\pgfqpoint{2.047364in}{1.906464in}}{\pgfqpoint{2.050636in}{1.898564in}}{\pgfqpoint{2.056460in}{1.892740in}}%
\pgfpathcurveto{\pgfqpoint{2.062284in}{1.886916in}}{\pgfqpoint{2.070184in}{1.883644in}}{\pgfqpoint{2.078420in}{1.883644in}}%
\pgfpathclose%
\pgfusepath{stroke,fill}%
\end{pgfscope}%
\begin{pgfscope}%
\pgfpathrectangle{\pgfqpoint{0.100000in}{0.212622in}}{\pgfqpoint{3.696000in}{3.696000in}}%
\pgfusepath{clip}%
\pgfsetbuttcap%
\pgfsetroundjoin%
\definecolor{currentfill}{rgb}{0.121569,0.466667,0.705882}%
\pgfsetfillcolor{currentfill}%
\pgfsetfillopacity{0.947496}%
\pgfsetlinewidth{1.003750pt}%
\definecolor{currentstroke}{rgb}{0.121569,0.466667,0.705882}%
\pgfsetstrokecolor{currentstroke}%
\pgfsetstrokeopacity{0.947496}%
\pgfsetdash{}{0pt}%
\pgfpathmoveto{\pgfqpoint{2.078420in}{1.883644in}}%
\pgfpathcurveto{\pgfqpoint{2.086656in}{1.883644in}}{\pgfqpoint{2.094556in}{1.886916in}}{\pgfqpoint{2.100380in}{1.892740in}}%
\pgfpathcurveto{\pgfqpoint{2.106204in}{1.898564in}}{\pgfqpoint{2.109477in}{1.906464in}}{\pgfqpoint{2.109477in}{1.914700in}}%
\pgfpathcurveto{\pgfqpoint{2.109477in}{1.922937in}}{\pgfqpoint{2.106204in}{1.930837in}}{\pgfqpoint{2.100380in}{1.936661in}}%
\pgfpathcurveto{\pgfqpoint{2.094556in}{1.942485in}}{\pgfqpoint{2.086656in}{1.945757in}}{\pgfqpoint{2.078420in}{1.945757in}}%
\pgfpathcurveto{\pgfqpoint{2.070184in}{1.945757in}}{\pgfqpoint{2.062284in}{1.942485in}}{\pgfqpoint{2.056460in}{1.936661in}}%
\pgfpathcurveto{\pgfqpoint{2.050636in}{1.930837in}}{\pgfqpoint{2.047364in}{1.922937in}}{\pgfqpoint{2.047364in}{1.914700in}}%
\pgfpathcurveto{\pgfqpoint{2.047364in}{1.906464in}}{\pgfqpoint{2.050636in}{1.898564in}}{\pgfqpoint{2.056460in}{1.892740in}}%
\pgfpathcurveto{\pgfqpoint{2.062284in}{1.886916in}}{\pgfqpoint{2.070184in}{1.883644in}}{\pgfqpoint{2.078420in}{1.883644in}}%
\pgfpathclose%
\pgfusepath{stroke,fill}%
\end{pgfscope}%
\begin{pgfscope}%
\pgfpathrectangle{\pgfqpoint{0.100000in}{0.212622in}}{\pgfqpoint{3.696000in}{3.696000in}}%
\pgfusepath{clip}%
\pgfsetbuttcap%
\pgfsetroundjoin%
\definecolor{currentfill}{rgb}{0.121569,0.466667,0.705882}%
\pgfsetfillcolor{currentfill}%
\pgfsetfillopacity{0.947496}%
\pgfsetlinewidth{1.003750pt}%
\definecolor{currentstroke}{rgb}{0.121569,0.466667,0.705882}%
\pgfsetstrokecolor{currentstroke}%
\pgfsetstrokeopacity{0.947496}%
\pgfsetdash{}{0pt}%
\pgfpathmoveto{\pgfqpoint{2.078420in}{1.883644in}}%
\pgfpathcurveto{\pgfqpoint{2.086656in}{1.883644in}}{\pgfqpoint{2.094556in}{1.886916in}}{\pgfqpoint{2.100380in}{1.892740in}}%
\pgfpathcurveto{\pgfqpoint{2.106204in}{1.898564in}}{\pgfqpoint{2.109477in}{1.906464in}}{\pgfqpoint{2.109477in}{1.914700in}}%
\pgfpathcurveto{\pgfqpoint{2.109477in}{1.922937in}}{\pgfqpoint{2.106204in}{1.930837in}}{\pgfqpoint{2.100380in}{1.936661in}}%
\pgfpathcurveto{\pgfqpoint{2.094556in}{1.942485in}}{\pgfqpoint{2.086656in}{1.945757in}}{\pgfqpoint{2.078420in}{1.945757in}}%
\pgfpathcurveto{\pgfqpoint{2.070184in}{1.945757in}}{\pgfqpoint{2.062284in}{1.942485in}}{\pgfqpoint{2.056460in}{1.936661in}}%
\pgfpathcurveto{\pgfqpoint{2.050636in}{1.930837in}}{\pgfqpoint{2.047364in}{1.922937in}}{\pgfqpoint{2.047364in}{1.914700in}}%
\pgfpathcurveto{\pgfqpoint{2.047364in}{1.906464in}}{\pgfqpoint{2.050636in}{1.898564in}}{\pgfqpoint{2.056460in}{1.892740in}}%
\pgfpathcurveto{\pgfqpoint{2.062284in}{1.886916in}}{\pgfqpoint{2.070184in}{1.883644in}}{\pgfqpoint{2.078420in}{1.883644in}}%
\pgfpathclose%
\pgfusepath{stroke,fill}%
\end{pgfscope}%
\begin{pgfscope}%
\pgfpathrectangle{\pgfqpoint{0.100000in}{0.212622in}}{\pgfqpoint{3.696000in}{3.696000in}}%
\pgfusepath{clip}%
\pgfsetbuttcap%
\pgfsetroundjoin%
\definecolor{currentfill}{rgb}{0.121569,0.466667,0.705882}%
\pgfsetfillcolor{currentfill}%
\pgfsetfillopacity{0.947496}%
\pgfsetlinewidth{1.003750pt}%
\definecolor{currentstroke}{rgb}{0.121569,0.466667,0.705882}%
\pgfsetstrokecolor{currentstroke}%
\pgfsetstrokeopacity{0.947496}%
\pgfsetdash{}{0pt}%
\pgfpathmoveto{\pgfqpoint{2.078420in}{1.883644in}}%
\pgfpathcurveto{\pgfqpoint{2.086656in}{1.883644in}}{\pgfqpoint{2.094556in}{1.886916in}}{\pgfqpoint{2.100380in}{1.892740in}}%
\pgfpathcurveto{\pgfqpoint{2.106204in}{1.898564in}}{\pgfqpoint{2.109477in}{1.906464in}}{\pgfqpoint{2.109477in}{1.914700in}}%
\pgfpathcurveto{\pgfqpoint{2.109477in}{1.922937in}}{\pgfqpoint{2.106204in}{1.930837in}}{\pgfqpoint{2.100380in}{1.936661in}}%
\pgfpathcurveto{\pgfqpoint{2.094556in}{1.942485in}}{\pgfqpoint{2.086656in}{1.945757in}}{\pgfqpoint{2.078420in}{1.945757in}}%
\pgfpathcurveto{\pgfqpoint{2.070184in}{1.945757in}}{\pgfqpoint{2.062284in}{1.942485in}}{\pgfqpoint{2.056460in}{1.936661in}}%
\pgfpathcurveto{\pgfqpoint{2.050636in}{1.930837in}}{\pgfqpoint{2.047364in}{1.922937in}}{\pgfqpoint{2.047364in}{1.914700in}}%
\pgfpathcurveto{\pgfqpoint{2.047364in}{1.906464in}}{\pgfqpoint{2.050636in}{1.898564in}}{\pgfqpoint{2.056460in}{1.892740in}}%
\pgfpathcurveto{\pgfqpoint{2.062284in}{1.886916in}}{\pgfqpoint{2.070184in}{1.883644in}}{\pgfqpoint{2.078420in}{1.883644in}}%
\pgfpathclose%
\pgfusepath{stroke,fill}%
\end{pgfscope}%
\begin{pgfscope}%
\pgfpathrectangle{\pgfqpoint{0.100000in}{0.212622in}}{\pgfqpoint{3.696000in}{3.696000in}}%
\pgfusepath{clip}%
\pgfsetbuttcap%
\pgfsetroundjoin%
\definecolor{currentfill}{rgb}{0.121569,0.466667,0.705882}%
\pgfsetfillcolor{currentfill}%
\pgfsetfillopacity{0.947496}%
\pgfsetlinewidth{1.003750pt}%
\definecolor{currentstroke}{rgb}{0.121569,0.466667,0.705882}%
\pgfsetstrokecolor{currentstroke}%
\pgfsetstrokeopacity{0.947496}%
\pgfsetdash{}{0pt}%
\pgfpathmoveto{\pgfqpoint{2.078420in}{1.883644in}}%
\pgfpathcurveto{\pgfqpoint{2.086656in}{1.883644in}}{\pgfqpoint{2.094556in}{1.886916in}}{\pgfqpoint{2.100380in}{1.892740in}}%
\pgfpathcurveto{\pgfqpoint{2.106204in}{1.898564in}}{\pgfqpoint{2.109477in}{1.906464in}}{\pgfqpoint{2.109477in}{1.914700in}}%
\pgfpathcurveto{\pgfqpoint{2.109477in}{1.922937in}}{\pgfqpoint{2.106204in}{1.930837in}}{\pgfqpoint{2.100380in}{1.936661in}}%
\pgfpathcurveto{\pgfqpoint{2.094556in}{1.942485in}}{\pgfqpoint{2.086656in}{1.945757in}}{\pgfqpoint{2.078420in}{1.945757in}}%
\pgfpathcurveto{\pgfqpoint{2.070184in}{1.945757in}}{\pgfqpoint{2.062284in}{1.942485in}}{\pgfqpoint{2.056460in}{1.936661in}}%
\pgfpathcurveto{\pgfqpoint{2.050636in}{1.930837in}}{\pgfqpoint{2.047364in}{1.922937in}}{\pgfqpoint{2.047364in}{1.914700in}}%
\pgfpathcurveto{\pgfqpoint{2.047364in}{1.906464in}}{\pgfqpoint{2.050636in}{1.898564in}}{\pgfqpoint{2.056460in}{1.892740in}}%
\pgfpathcurveto{\pgfqpoint{2.062284in}{1.886916in}}{\pgfqpoint{2.070184in}{1.883644in}}{\pgfqpoint{2.078420in}{1.883644in}}%
\pgfpathclose%
\pgfusepath{stroke,fill}%
\end{pgfscope}%
\begin{pgfscope}%
\pgfpathrectangle{\pgfqpoint{0.100000in}{0.212622in}}{\pgfqpoint{3.696000in}{3.696000in}}%
\pgfusepath{clip}%
\pgfsetbuttcap%
\pgfsetroundjoin%
\definecolor{currentfill}{rgb}{0.121569,0.466667,0.705882}%
\pgfsetfillcolor{currentfill}%
\pgfsetfillopacity{0.947496}%
\pgfsetlinewidth{1.003750pt}%
\definecolor{currentstroke}{rgb}{0.121569,0.466667,0.705882}%
\pgfsetstrokecolor{currentstroke}%
\pgfsetstrokeopacity{0.947496}%
\pgfsetdash{}{0pt}%
\pgfpathmoveto{\pgfqpoint{2.078420in}{1.883644in}}%
\pgfpathcurveto{\pgfqpoint{2.086656in}{1.883644in}}{\pgfqpoint{2.094556in}{1.886916in}}{\pgfqpoint{2.100380in}{1.892740in}}%
\pgfpathcurveto{\pgfqpoint{2.106204in}{1.898564in}}{\pgfqpoint{2.109477in}{1.906464in}}{\pgfqpoint{2.109477in}{1.914700in}}%
\pgfpathcurveto{\pgfqpoint{2.109477in}{1.922937in}}{\pgfqpoint{2.106204in}{1.930837in}}{\pgfqpoint{2.100380in}{1.936661in}}%
\pgfpathcurveto{\pgfqpoint{2.094556in}{1.942485in}}{\pgfqpoint{2.086656in}{1.945757in}}{\pgfqpoint{2.078420in}{1.945757in}}%
\pgfpathcurveto{\pgfqpoint{2.070184in}{1.945757in}}{\pgfqpoint{2.062284in}{1.942485in}}{\pgfqpoint{2.056460in}{1.936661in}}%
\pgfpathcurveto{\pgfqpoint{2.050636in}{1.930837in}}{\pgfqpoint{2.047364in}{1.922937in}}{\pgfqpoint{2.047364in}{1.914700in}}%
\pgfpathcurveto{\pgfqpoint{2.047364in}{1.906464in}}{\pgfqpoint{2.050636in}{1.898564in}}{\pgfqpoint{2.056460in}{1.892740in}}%
\pgfpathcurveto{\pgfqpoint{2.062284in}{1.886916in}}{\pgfqpoint{2.070184in}{1.883644in}}{\pgfqpoint{2.078420in}{1.883644in}}%
\pgfpathclose%
\pgfusepath{stroke,fill}%
\end{pgfscope}%
\begin{pgfscope}%
\pgfpathrectangle{\pgfqpoint{0.100000in}{0.212622in}}{\pgfqpoint{3.696000in}{3.696000in}}%
\pgfusepath{clip}%
\pgfsetbuttcap%
\pgfsetroundjoin%
\definecolor{currentfill}{rgb}{0.121569,0.466667,0.705882}%
\pgfsetfillcolor{currentfill}%
\pgfsetfillopacity{0.947496}%
\pgfsetlinewidth{1.003750pt}%
\definecolor{currentstroke}{rgb}{0.121569,0.466667,0.705882}%
\pgfsetstrokecolor{currentstroke}%
\pgfsetstrokeopacity{0.947496}%
\pgfsetdash{}{0pt}%
\pgfpathmoveto{\pgfqpoint{2.078420in}{1.883644in}}%
\pgfpathcurveto{\pgfqpoint{2.086656in}{1.883644in}}{\pgfqpoint{2.094556in}{1.886916in}}{\pgfqpoint{2.100380in}{1.892740in}}%
\pgfpathcurveto{\pgfqpoint{2.106204in}{1.898564in}}{\pgfqpoint{2.109477in}{1.906464in}}{\pgfqpoint{2.109477in}{1.914700in}}%
\pgfpathcurveto{\pgfqpoint{2.109477in}{1.922937in}}{\pgfqpoint{2.106204in}{1.930837in}}{\pgfqpoint{2.100380in}{1.936661in}}%
\pgfpathcurveto{\pgfqpoint{2.094556in}{1.942485in}}{\pgfqpoint{2.086656in}{1.945757in}}{\pgfqpoint{2.078420in}{1.945757in}}%
\pgfpathcurveto{\pgfqpoint{2.070184in}{1.945757in}}{\pgfqpoint{2.062284in}{1.942485in}}{\pgfqpoint{2.056460in}{1.936661in}}%
\pgfpathcurveto{\pgfqpoint{2.050636in}{1.930837in}}{\pgfqpoint{2.047364in}{1.922937in}}{\pgfqpoint{2.047364in}{1.914700in}}%
\pgfpathcurveto{\pgfqpoint{2.047364in}{1.906464in}}{\pgfqpoint{2.050636in}{1.898564in}}{\pgfqpoint{2.056460in}{1.892740in}}%
\pgfpathcurveto{\pgfqpoint{2.062284in}{1.886916in}}{\pgfqpoint{2.070184in}{1.883644in}}{\pgfqpoint{2.078420in}{1.883644in}}%
\pgfpathclose%
\pgfusepath{stroke,fill}%
\end{pgfscope}%
\begin{pgfscope}%
\pgfpathrectangle{\pgfqpoint{0.100000in}{0.212622in}}{\pgfqpoint{3.696000in}{3.696000in}}%
\pgfusepath{clip}%
\pgfsetbuttcap%
\pgfsetroundjoin%
\definecolor{currentfill}{rgb}{0.121569,0.466667,0.705882}%
\pgfsetfillcolor{currentfill}%
\pgfsetfillopacity{0.947496}%
\pgfsetlinewidth{1.003750pt}%
\definecolor{currentstroke}{rgb}{0.121569,0.466667,0.705882}%
\pgfsetstrokecolor{currentstroke}%
\pgfsetstrokeopacity{0.947496}%
\pgfsetdash{}{0pt}%
\pgfpathmoveto{\pgfqpoint{2.078420in}{1.883644in}}%
\pgfpathcurveto{\pgfqpoint{2.086656in}{1.883644in}}{\pgfqpoint{2.094556in}{1.886916in}}{\pgfqpoint{2.100380in}{1.892740in}}%
\pgfpathcurveto{\pgfqpoint{2.106204in}{1.898564in}}{\pgfqpoint{2.109477in}{1.906464in}}{\pgfqpoint{2.109477in}{1.914700in}}%
\pgfpathcurveto{\pgfqpoint{2.109477in}{1.922937in}}{\pgfqpoint{2.106204in}{1.930837in}}{\pgfqpoint{2.100380in}{1.936661in}}%
\pgfpathcurveto{\pgfqpoint{2.094556in}{1.942485in}}{\pgfqpoint{2.086656in}{1.945757in}}{\pgfqpoint{2.078420in}{1.945757in}}%
\pgfpathcurveto{\pgfqpoint{2.070184in}{1.945757in}}{\pgfqpoint{2.062284in}{1.942485in}}{\pgfqpoint{2.056460in}{1.936661in}}%
\pgfpathcurveto{\pgfqpoint{2.050636in}{1.930837in}}{\pgfqpoint{2.047364in}{1.922937in}}{\pgfqpoint{2.047364in}{1.914700in}}%
\pgfpathcurveto{\pgfqpoint{2.047364in}{1.906464in}}{\pgfqpoint{2.050636in}{1.898564in}}{\pgfqpoint{2.056460in}{1.892740in}}%
\pgfpathcurveto{\pgfqpoint{2.062284in}{1.886916in}}{\pgfqpoint{2.070184in}{1.883644in}}{\pgfqpoint{2.078420in}{1.883644in}}%
\pgfpathclose%
\pgfusepath{stroke,fill}%
\end{pgfscope}%
\begin{pgfscope}%
\pgfpathrectangle{\pgfqpoint{0.100000in}{0.212622in}}{\pgfqpoint{3.696000in}{3.696000in}}%
\pgfusepath{clip}%
\pgfsetbuttcap%
\pgfsetroundjoin%
\definecolor{currentfill}{rgb}{0.121569,0.466667,0.705882}%
\pgfsetfillcolor{currentfill}%
\pgfsetfillopacity{0.947496}%
\pgfsetlinewidth{1.003750pt}%
\definecolor{currentstroke}{rgb}{0.121569,0.466667,0.705882}%
\pgfsetstrokecolor{currentstroke}%
\pgfsetstrokeopacity{0.947496}%
\pgfsetdash{}{0pt}%
\pgfpathmoveto{\pgfqpoint{2.078420in}{1.883644in}}%
\pgfpathcurveto{\pgfqpoint{2.086656in}{1.883644in}}{\pgfqpoint{2.094556in}{1.886916in}}{\pgfqpoint{2.100380in}{1.892740in}}%
\pgfpathcurveto{\pgfqpoint{2.106204in}{1.898564in}}{\pgfqpoint{2.109477in}{1.906464in}}{\pgfqpoint{2.109477in}{1.914700in}}%
\pgfpathcurveto{\pgfqpoint{2.109477in}{1.922937in}}{\pgfqpoint{2.106204in}{1.930837in}}{\pgfqpoint{2.100380in}{1.936661in}}%
\pgfpathcurveto{\pgfqpoint{2.094556in}{1.942485in}}{\pgfqpoint{2.086656in}{1.945757in}}{\pgfqpoint{2.078420in}{1.945757in}}%
\pgfpathcurveto{\pgfqpoint{2.070184in}{1.945757in}}{\pgfqpoint{2.062284in}{1.942485in}}{\pgfqpoint{2.056460in}{1.936661in}}%
\pgfpathcurveto{\pgfqpoint{2.050636in}{1.930837in}}{\pgfqpoint{2.047364in}{1.922937in}}{\pgfqpoint{2.047364in}{1.914700in}}%
\pgfpathcurveto{\pgfqpoint{2.047364in}{1.906464in}}{\pgfqpoint{2.050636in}{1.898564in}}{\pgfqpoint{2.056460in}{1.892740in}}%
\pgfpathcurveto{\pgfqpoint{2.062284in}{1.886916in}}{\pgfqpoint{2.070184in}{1.883644in}}{\pgfqpoint{2.078420in}{1.883644in}}%
\pgfpathclose%
\pgfusepath{stroke,fill}%
\end{pgfscope}%
\begin{pgfscope}%
\pgfpathrectangle{\pgfqpoint{0.100000in}{0.212622in}}{\pgfqpoint{3.696000in}{3.696000in}}%
\pgfusepath{clip}%
\pgfsetbuttcap%
\pgfsetroundjoin%
\definecolor{currentfill}{rgb}{0.121569,0.466667,0.705882}%
\pgfsetfillcolor{currentfill}%
\pgfsetfillopacity{0.947496}%
\pgfsetlinewidth{1.003750pt}%
\definecolor{currentstroke}{rgb}{0.121569,0.466667,0.705882}%
\pgfsetstrokecolor{currentstroke}%
\pgfsetstrokeopacity{0.947496}%
\pgfsetdash{}{0pt}%
\pgfpathmoveto{\pgfqpoint{2.078420in}{1.883644in}}%
\pgfpathcurveto{\pgfqpoint{2.086656in}{1.883644in}}{\pgfqpoint{2.094556in}{1.886916in}}{\pgfqpoint{2.100380in}{1.892740in}}%
\pgfpathcurveto{\pgfqpoint{2.106204in}{1.898564in}}{\pgfqpoint{2.109477in}{1.906464in}}{\pgfqpoint{2.109477in}{1.914700in}}%
\pgfpathcurveto{\pgfqpoint{2.109477in}{1.922937in}}{\pgfqpoint{2.106204in}{1.930837in}}{\pgfqpoint{2.100380in}{1.936661in}}%
\pgfpathcurveto{\pgfqpoint{2.094556in}{1.942485in}}{\pgfqpoint{2.086656in}{1.945757in}}{\pgfqpoint{2.078420in}{1.945757in}}%
\pgfpathcurveto{\pgfqpoint{2.070184in}{1.945757in}}{\pgfqpoint{2.062284in}{1.942485in}}{\pgfqpoint{2.056460in}{1.936661in}}%
\pgfpathcurveto{\pgfqpoint{2.050636in}{1.930837in}}{\pgfqpoint{2.047364in}{1.922937in}}{\pgfqpoint{2.047364in}{1.914700in}}%
\pgfpathcurveto{\pgfqpoint{2.047364in}{1.906464in}}{\pgfqpoint{2.050636in}{1.898564in}}{\pgfqpoint{2.056460in}{1.892740in}}%
\pgfpathcurveto{\pgfqpoint{2.062284in}{1.886916in}}{\pgfqpoint{2.070184in}{1.883644in}}{\pgfqpoint{2.078420in}{1.883644in}}%
\pgfpathclose%
\pgfusepath{stroke,fill}%
\end{pgfscope}%
\begin{pgfscope}%
\pgfpathrectangle{\pgfqpoint{0.100000in}{0.212622in}}{\pgfqpoint{3.696000in}{3.696000in}}%
\pgfusepath{clip}%
\pgfsetbuttcap%
\pgfsetroundjoin%
\definecolor{currentfill}{rgb}{0.121569,0.466667,0.705882}%
\pgfsetfillcolor{currentfill}%
\pgfsetfillopacity{0.947496}%
\pgfsetlinewidth{1.003750pt}%
\definecolor{currentstroke}{rgb}{0.121569,0.466667,0.705882}%
\pgfsetstrokecolor{currentstroke}%
\pgfsetstrokeopacity{0.947496}%
\pgfsetdash{}{0pt}%
\pgfpathmoveto{\pgfqpoint{2.078420in}{1.883644in}}%
\pgfpathcurveto{\pgfqpoint{2.086656in}{1.883644in}}{\pgfqpoint{2.094556in}{1.886916in}}{\pgfqpoint{2.100380in}{1.892740in}}%
\pgfpathcurveto{\pgfqpoint{2.106204in}{1.898564in}}{\pgfqpoint{2.109477in}{1.906464in}}{\pgfqpoint{2.109477in}{1.914700in}}%
\pgfpathcurveto{\pgfqpoint{2.109477in}{1.922937in}}{\pgfqpoint{2.106204in}{1.930837in}}{\pgfqpoint{2.100380in}{1.936661in}}%
\pgfpathcurveto{\pgfqpoint{2.094556in}{1.942485in}}{\pgfqpoint{2.086656in}{1.945757in}}{\pgfqpoint{2.078420in}{1.945757in}}%
\pgfpathcurveto{\pgfqpoint{2.070184in}{1.945757in}}{\pgfqpoint{2.062284in}{1.942485in}}{\pgfqpoint{2.056460in}{1.936661in}}%
\pgfpathcurveto{\pgfqpoint{2.050636in}{1.930837in}}{\pgfqpoint{2.047364in}{1.922937in}}{\pgfqpoint{2.047364in}{1.914700in}}%
\pgfpathcurveto{\pgfqpoint{2.047364in}{1.906464in}}{\pgfqpoint{2.050636in}{1.898564in}}{\pgfqpoint{2.056460in}{1.892740in}}%
\pgfpathcurveto{\pgfqpoint{2.062284in}{1.886916in}}{\pgfqpoint{2.070184in}{1.883644in}}{\pgfqpoint{2.078420in}{1.883644in}}%
\pgfpathclose%
\pgfusepath{stroke,fill}%
\end{pgfscope}%
\begin{pgfscope}%
\pgfpathrectangle{\pgfqpoint{0.100000in}{0.212622in}}{\pgfqpoint{3.696000in}{3.696000in}}%
\pgfusepath{clip}%
\pgfsetbuttcap%
\pgfsetroundjoin%
\definecolor{currentfill}{rgb}{0.121569,0.466667,0.705882}%
\pgfsetfillcolor{currentfill}%
\pgfsetfillopacity{0.947496}%
\pgfsetlinewidth{1.003750pt}%
\definecolor{currentstroke}{rgb}{0.121569,0.466667,0.705882}%
\pgfsetstrokecolor{currentstroke}%
\pgfsetstrokeopacity{0.947496}%
\pgfsetdash{}{0pt}%
\pgfpathmoveto{\pgfqpoint{2.078420in}{1.883644in}}%
\pgfpathcurveto{\pgfqpoint{2.086656in}{1.883644in}}{\pgfqpoint{2.094556in}{1.886916in}}{\pgfqpoint{2.100380in}{1.892740in}}%
\pgfpathcurveto{\pgfqpoint{2.106204in}{1.898564in}}{\pgfqpoint{2.109477in}{1.906464in}}{\pgfqpoint{2.109477in}{1.914700in}}%
\pgfpathcurveto{\pgfqpoint{2.109477in}{1.922937in}}{\pgfqpoint{2.106204in}{1.930837in}}{\pgfqpoint{2.100380in}{1.936661in}}%
\pgfpathcurveto{\pgfqpoint{2.094556in}{1.942485in}}{\pgfqpoint{2.086656in}{1.945757in}}{\pgfqpoint{2.078420in}{1.945757in}}%
\pgfpathcurveto{\pgfqpoint{2.070184in}{1.945757in}}{\pgfqpoint{2.062284in}{1.942485in}}{\pgfqpoint{2.056460in}{1.936661in}}%
\pgfpathcurveto{\pgfqpoint{2.050636in}{1.930837in}}{\pgfqpoint{2.047364in}{1.922937in}}{\pgfqpoint{2.047364in}{1.914700in}}%
\pgfpathcurveto{\pgfqpoint{2.047364in}{1.906464in}}{\pgfqpoint{2.050636in}{1.898564in}}{\pgfqpoint{2.056460in}{1.892740in}}%
\pgfpathcurveto{\pgfqpoint{2.062284in}{1.886916in}}{\pgfqpoint{2.070184in}{1.883644in}}{\pgfqpoint{2.078420in}{1.883644in}}%
\pgfpathclose%
\pgfusepath{stroke,fill}%
\end{pgfscope}%
\begin{pgfscope}%
\pgfpathrectangle{\pgfqpoint{0.100000in}{0.212622in}}{\pgfqpoint{3.696000in}{3.696000in}}%
\pgfusepath{clip}%
\pgfsetbuttcap%
\pgfsetroundjoin%
\definecolor{currentfill}{rgb}{0.121569,0.466667,0.705882}%
\pgfsetfillcolor{currentfill}%
\pgfsetfillopacity{0.947496}%
\pgfsetlinewidth{1.003750pt}%
\definecolor{currentstroke}{rgb}{0.121569,0.466667,0.705882}%
\pgfsetstrokecolor{currentstroke}%
\pgfsetstrokeopacity{0.947496}%
\pgfsetdash{}{0pt}%
\pgfpathmoveto{\pgfqpoint{2.078420in}{1.883644in}}%
\pgfpathcurveto{\pgfqpoint{2.086656in}{1.883644in}}{\pgfqpoint{2.094556in}{1.886916in}}{\pgfqpoint{2.100380in}{1.892740in}}%
\pgfpathcurveto{\pgfqpoint{2.106204in}{1.898564in}}{\pgfqpoint{2.109477in}{1.906464in}}{\pgfqpoint{2.109477in}{1.914700in}}%
\pgfpathcurveto{\pgfqpoint{2.109477in}{1.922937in}}{\pgfqpoint{2.106204in}{1.930837in}}{\pgfqpoint{2.100380in}{1.936661in}}%
\pgfpathcurveto{\pgfqpoint{2.094556in}{1.942485in}}{\pgfqpoint{2.086656in}{1.945757in}}{\pgfqpoint{2.078420in}{1.945757in}}%
\pgfpathcurveto{\pgfqpoint{2.070184in}{1.945757in}}{\pgfqpoint{2.062284in}{1.942485in}}{\pgfqpoint{2.056460in}{1.936661in}}%
\pgfpathcurveto{\pgfqpoint{2.050636in}{1.930837in}}{\pgfqpoint{2.047364in}{1.922937in}}{\pgfqpoint{2.047364in}{1.914700in}}%
\pgfpathcurveto{\pgfqpoint{2.047364in}{1.906464in}}{\pgfqpoint{2.050636in}{1.898564in}}{\pgfqpoint{2.056460in}{1.892740in}}%
\pgfpathcurveto{\pgfqpoint{2.062284in}{1.886916in}}{\pgfqpoint{2.070184in}{1.883644in}}{\pgfqpoint{2.078420in}{1.883644in}}%
\pgfpathclose%
\pgfusepath{stroke,fill}%
\end{pgfscope}%
\begin{pgfscope}%
\pgfpathrectangle{\pgfqpoint{0.100000in}{0.212622in}}{\pgfqpoint{3.696000in}{3.696000in}}%
\pgfusepath{clip}%
\pgfsetbuttcap%
\pgfsetroundjoin%
\definecolor{currentfill}{rgb}{0.121569,0.466667,0.705882}%
\pgfsetfillcolor{currentfill}%
\pgfsetfillopacity{0.947496}%
\pgfsetlinewidth{1.003750pt}%
\definecolor{currentstroke}{rgb}{0.121569,0.466667,0.705882}%
\pgfsetstrokecolor{currentstroke}%
\pgfsetstrokeopacity{0.947496}%
\pgfsetdash{}{0pt}%
\pgfpathmoveto{\pgfqpoint{2.078420in}{1.883644in}}%
\pgfpathcurveto{\pgfqpoint{2.086656in}{1.883644in}}{\pgfqpoint{2.094556in}{1.886916in}}{\pgfqpoint{2.100380in}{1.892740in}}%
\pgfpathcurveto{\pgfqpoint{2.106204in}{1.898564in}}{\pgfqpoint{2.109477in}{1.906464in}}{\pgfqpoint{2.109477in}{1.914700in}}%
\pgfpathcurveto{\pgfqpoint{2.109477in}{1.922937in}}{\pgfqpoint{2.106204in}{1.930837in}}{\pgfqpoint{2.100380in}{1.936661in}}%
\pgfpathcurveto{\pgfqpoint{2.094556in}{1.942485in}}{\pgfqpoint{2.086656in}{1.945757in}}{\pgfqpoint{2.078420in}{1.945757in}}%
\pgfpathcurveto{\pgfqpoint{2.070184in}{1.945757in}}{\pgfqpoint{2.062284in}{1.942485in}}{\pgfqpoint{2.056460in}{1.936661in}}%
\pgfpathcurveto{\pgfqpoint{2.050636in}{1.930837in}}{\pgfqpoint{2.047364in}{1.922937in}}{\pgfqpoint{2.047364in}{1.914700in}}%
\pgfpathcurveto{\pgfqpoint{2.047364in}{1.906464in}}{\pgfqpoint{2.050636in}{1.898564in}}{\pgfqpoint{2.056460in}{1.892740in}}%
\pgfpathcurveto{\pgfqpoint{2.062284in}{1.886916in}}{\pgfqpoint{2.070184in}{1.883644in}}{\pgfqpoint{2.078420in}{1.883644in}}%
\pgfpathclose%
\pgfusepath{stroke,fill}%
\end{pgfscope}%
\begin{pgfscope}%
\pgfpathrectangle{\pgfqpoint{0.100000in}{0.212622in}}{\pgfqpoint{3.696000in}{3.696000in}}%
\pgfusepath{clip}%
\pgfsetbuttcap%
\pgfsetroundjoin%
\definecolor{currentfill}{rgb}{0.121569,0.466667,0.705882}%
\pgfsetfillcolor{currentfill}%
\pgfsetfillopacity{0.947496}%
\pgfsetlinewidth{1.003750pt}%
\definecolor{currentstroke}{rgb}{0.121569,0.466667,0.705882}%
\pgfsetstrokecolor{currentstroke}%
\pgfsetstrokeopacity{0.947496}%
\pgfsetdash{}{0pt}%
\pgfpathmoveto{\pgfqpoint{2.078420in}{1.883644in}}%
\pgfpathcurveto{\pgfqpoint{2.086656in}{1.883644in}}{\pgfqpoint{2.094556in}{1.886916in}}{\pgfqpoint{2.100380in}{1.892740in}}%
\pgfpathcurveto{\pgfqpoint{2.106204in}{1.898564in}}{\pgfqpoint{2.109477in}{1.906464in}}{\pgfqpoint{2.109477in}{1.914700in}}%
\pgfpathcurveto{\pgfqpoint{2.109477in}{1.922937in}}{\pgfqpoint{2.106204in}{1.930837in}}{\pgfqpoint{2.100380in}{1.936661in}}%
\pgfpathcurveto{\pgfqpoint{2.094556in}{1.942485in}}{\pgfqpoint{2.086656in}{1.945757in}}{\pgfqpoint{2.078420in}{1.945757in}}%
\pgfpathcurveto{\pgfqpoint{2.070184in}{1.945757in}}{\pgfqpoint{2.062284in}{1.942485in}}{\pgfqpoint{2.056460in}{1.936661in}}%
\pgfpathcurveto{\pgfqpoint{2.050636in}{1.930837in}}{\pgfqpoint{2.047364in}{1.922937in}}{\pgfqpoint{2.047364in}{1.914700in}}%
\pgfpathcurveto{\pgfqpoint{2.047364in}{1.906464in}}{\pgfqpoint{2.050636in}{1.898564in}}{\pgfqpoint{2.056460in}{1.892740in}}%
\pgfpathcurveto{\pgfqpoint{2.062284in}{1.886916in}}{\pgfqpoint{2.070184in}{1.883644in}}{\pgfqpoint{2.078420in}{1.883644in}}%
\pgfpathclose%
\pgfusepath{stroke,fill}%
\end{pgfscope}%
\begin{pgfscope}%
\pgfpathrectangle{\pgfqpoint{0.100000in}{0.212622in}}{\pgfqpoint{3.696000in}{3.696000in}}%
\pgfusepath{clip}%
\pgfsetbuttcap%
\pgfsetroundjoin%
\definecolor{currentfill}{rgb}{0.121569,0.466667,0.705882}%
\pgfsetfillcolor{currentfill}%
\pgfsetfillopacity{0.947496}%
\pgfsetlinewidth{1.003750pt}%
\definecolor{currentstroke}{rgb}{0.121569,0.466667,0.705882}%
\pgfsetstrokecolor{currentstroke}%
\pgfsetstrokeopacity{0.947496}%
\pgfsetdash{}{0pt}%
\pgfpathmoveto{\pgfqpoint{2.078420in}{1.883644in}}%
\pgfpathcurveto{\pgfqpoint{2.086656in}{1.883644in}}{\pgfqpoint{2.094556in}{1.886916in}}{\pgfqpoint{2.100380in}{1.892740in}}%
\pgfpathcurveto{\pgfqpoint{2.106204in}{1.898564in}}{\pgfqpoint{2.109477in}{1.906464in}}{\pgfqpoint{2.109477in}{1.914700in}}%
\pgfpathcurveto{\pgfqpoint{2.109477in}{1.922937in}}{\pgfqpoint{2.106204in}{1.930837in}}{\pgfqpoint{2.100380in}{1.936661in}}%
\pgfpathcurveto{\pgfqpoint{2.094556in}{1.942485in}}{\pgfqpoint{2.086656in}{1.945757in}}{\pgfqpoint{2.078420in}{1.945757in}}%
\pgfpathcurveto{\pgfqpoint{2.070184in}{1.945757in}}{\pgfqpoint{2.062284in}{1.942485in}}{\pgfqpoint{2.056460in}{1.936661in}}%
\pgfpathcurveto{\pgfqpoint{2.050636in}{1.930837in}}{\pgfqpoint{2.047364in}{1.922937in}}{\pgfqpoint{2.047364in}{1.914700in}}%
\pgfpathcurveto{\pgfqpoint{2.047364in}{1.906464in}}{\pgfqpoint{2.050636in}{1.898564in}}{\pgfqpoint{2.056460in}{1.892740in}}%
\pgfpathcurveto{\pgfqpoint{2.062284in}{1.886916in}}{\pgfqpoint{2.070184in}{1.883644in}}{\pgfqpoint{2.078420in}{1.883644in}}%
\pgfpathclose%
\pgfusepath{stroke,fill}%
\end{pgfscope}%
\begin{pgfscope}%
\pgfpathrectangle{\pgfqpoint{0.100000in}{0.212622in}}{\pgfqpoint{3.696000in}{3.696000in}}%
\pgfusepath{clip}%
\pgfsetbuttcap%
\pgfsetroundjoin%
\definecolor{currentfill}{rgb}{0.121569,0.466667,0.705882}%
\pgfsetfillcolor{currentfill}%
\pgfsetfillopacity{0.947496}%
\pgfsetlinewidth{1.003750pt}%
\definecolor{currentstroke}{rgb}{0.121569,0.466667,0.705882}%
\pgfsetstrokecolor{currentstroke}%
\pgfsetstrokeopacity{0.947496}%
\pgfsetdash{}{0pt}%
\pgfpathmoveto{\pgfqpoint{2.078420in}{1.883644in}}%
\pgfpathcurveto{\pgfqpoint{2.086656in}{1.883644in}}{\pgfqpoint{2.094556in}{1.886916in}}{\pgfqpoint{2.100380in}{1.892740in}}%
\pgfpathcurveto{\pgfqpoint{2.106204in}{1.898564in}}{\pgfqpoint{2.109477in}{1.906464in}}{\pgfqpoint{2.109477in}{1.914700in}}%
\pgfpathcurveto{\pgfqpoint{2.109477in}{1.922937in}}{\pgfqpoint{2.106204in}{1.930837in}}{\pgfqpoint{2.100380in}{1.936661in}}%
\pgfpathcurveto{\pgfqpoint{2.094556in}{1.942485in}}{\pgfqpoint{2.086656in}{1.945757in}}{\pgfqpoint{2.078420in}{1.945757in}}%
\pgfpathcurveto{\pgfqpoint{2.070184in}{1.945757in}}{\pgfqpoint{2.062284in}{1.942485in}}{\pgfqpoint{2.056460in}{1.936661in}}%
\pgfpathcurveto{\pgfqpoint{2.050636in}{1.930837in}}{\pgfqpoint{2.047364in}{1.922937in}}{\pgfqpoint{2.047364in}{1.914700in}}%
\pgfpathcurveto{\pgfqpoint{2.047364in}{1.906464in}}{\pgfqpoint{2.050636in}{1.898564in}}{\pgfqpoint{2.056460in}{1.892740in}}%
\pgfpathcurveto{\pgfqpoint{2.062284in}{1.886916in}}{\pgfqpoint{2.070184in}{1.883644in}}{\pgfqpoint{2.078420in}{1.883644in}}%
\pgfpathclose%
\pgfusepath{stroke,fill}%
\end{pgfscope}%
\begin{pgfscope}%
\pgfpathrectangle{\pgfqpoint{0.100000in}{0.212622in}}{\pgfqpoint{3.696000in}{3.696000in}}%
\pgfusepath{clip}%
\pgfsetbuttcap%
\pgfsetroundjoin%
\definecolor{currentfill}{rgb}{0.121569,0.466667,0.705882}%
\pgfsetfillcolor{currentfill}%
\pgfsetfillopacity{0.947496}%
\pgfsetlinewidth{1.003750pt}%
\definecolor{currentstroke}{rgb}{0.121569,0.466667,0.705882}%
\pgfsetstrokecolor{currentstroke}%
\pgfsetstrokeopacity{0.947496}%
\pgfsetdash{}{0pt}%
\pgfpathmoveto{\pgfqpoint{2.078420in}{1.883644in}}%
\pgfpathcurveto{\pgfqpoint{2.086656in}{1.883644in}}{\pgfqpoint{2.094556in}{1.886916in}}{\pgfqpoint{2.100380in}{1.892740in}}%
\pgfpathcurveto{\pgfqpoint{2.106204in}{1.898564in}}{\pgfqpoint{2.109477in}{1.906464in}}{\pgfqpoint{2.109477in}{1.914700in}}%
\pgfpathcurveto{\pgfqpoint{2.109477in}{1.922937in}}{\pgfqpoint{2.106204in}{1.930837in}}{\pgfqpoint{2.100380in}{1.936661in}}%
\pgfpathcurveto{\pgfqpoint{2.094556in}{1.942485in}}{\pgfqpoint{2.086656in}{1.945757in}}{\pgfqpoint{2.078420in}{1.945757in}}%
\pgfpathcurveto{\pgfqpoint{2.070184in}{1.945757in}}{\pgfqpoint{2.062284in}{1.942485in}}{\pgfqpoint{2.056460in}{1.936661in}}%
\pgfpathcurveto{\pgfqpoint{2.050636in}{1.930837in}}{\pgfqpoint{2.047364in}{1.922937in}}{\pgfqpoint{2.047364in}{1.914700in}}%
\pgfpathcurveto{\pgfqpoint{2.047364in}{1.906464in}}{\pgfqpoint{2.050636in}{1.898564in}}{\pgfqpoint{2.056460in}{1.892740in}}%
\pgfpathcurveto{\pgfqpoint{2.062284in}{1.886916in}}{\pgfqpoint{2.070184in}{1.883644in}}{\pgfqpoint{2.078420in}{1.883644in}}%
\pgfpathclose%
\pgfusepath{stroke,fill}%
\end{pgfscope}%
\begin{pgfscope}%
\pgfpathrectangle{\pgfqpoint{0.100000in}{0.212622in}}{\pgfqpoint{3.696000in}{3.696000in}}%
\pgfusepath{clip}%
\pgfsetbuttcap%
\pgfsetroundjoin%
\definecolor{currentfill}{rgb}{0.121569,0.466667,0.705882}%
\pgfsetfillcolor{currentfill}%
\pgfsetfillopacity{0.947496}%
\pgfsetlinewidth{1.003750pt}%
\definecolor{currentstroke}{rgb}{0.121569,0.466667,0.705882}%
\pgfsetstrokecolor{currentstroke}%
\pgfsetstrokeopacity{0.947496}%
\pgfsetdash{}{0pt}%
\pgfpathmoveto{\pgfqpoint{2.078420in}{1.883644in}}%
\pgfpathcurveto{\pgfqpoint{2.086656in}{1.883644in}}{\pgfqpoint{2.094556in}{1.886916in}}{\pgfqpoint{2.100380in}{1.892740in}}%
\pgfpathcurveto{\pgfqpoint{2.106204in}{1.898564in}}{\pgfqpoint{2.109477in}{1.906464in}}{\pgfqpoint{2.109477in}{1.914700in}}%
\pgfpathcurveto{\pgfqpoint{2.109477in}{1.922937in}}{\pgfqpoint{2.106204in}{1.930837in}}{\pgfqpoint{2.100380in}{1.936661in}}%
\pgfpathcurveto{\pgfqpoint{2.094556in}{1.942485in}}{\pgfqpoint{2.086656in}{1.945757in}}{\pgfqpoint{2.078420in}{1.945757in}}%
\pgfpathcurveto{\pgfqpoint{2.070184in}{1.945757in}}{\pgfqpoint{2.062284in}{1.942485in}}{\pgfqpoint{2.056460in}{1.936661in}}%
\pgfpathcurveto{\pgfqpoint{2.050636in}{1.930837in}}{\pgfqpoint{2.047364in}{1.922937in}}{\pgfqpoint{2.047364in}{1.914700in}}%
\pgfpathcurveto{\pgfqpoint{2.047364in}{1.906464in}}{\pgfqpoint{2.050636in}{1.898564in}}{\pgfqpoint{2.056460in}{1.892740in}}%
\pgfpathcurveto{\pgfqpoint{2.062284in}{1.886916in}}{\pgfqpoint{2.070184in}{1.883644in}}{\pgfqpoint{2.078420in}{1.883644in}}%
\pgfpathclose%
\pgfusepath{stroke,fill}%
\end{pgfscope}%
\begin{pgfscope}%
\pgfpathrectangle{\pgfqpoint{0.100000in}{0.212622in}}{\pgfqpoint{3.696000in}{3.696000in}}%
\pgfusepath{clip}%
\pgfsetbuttcap%
\pgfsetroundjoin%
\definecolor{currentfill}{rgb}{0.121569,0.466667,0.705882}%
\pgfsetfillcolor{currentfill}%
\pgfsetfillopacity{0.947496}%
\pgfsetlinewidth{1.003750pt}%
\definecolor{currentstroke}{rgb}{0.121569,0.466667,0.705882}%
\pgfsetstrokecolor{currentstroke}%
\pgfsetstrokeopacity{0.947496}%
\pgfsetdash{}{0pt}%
\pgfpathmoveto{\pgfqpoint{2.078420in}{1.883644in}}%
\pgfpathcurveto{\pgfqpoint{2.086656in}{1.883644in}}{\pgfqpoint{2.094556in}{1.886916in}}{\pgfqpoint{2.100380in}{1.892740in}}%
\pgfpathcurveto{\pgfqpoint{2.106204in}{1.898564in}}{\pgfqpoint{2.109477in}{1.906464in}}{\pgfqpoint{2.109477in}{1.914700in}}%
\pgfpathcurveto{\pgfqpoint{2.109477in}{1.922937in}}{\pgfqpoint{2.106204in}{1.930837in}}{\pgfqpoint{2.100380in}{1.936661in}}%
\pgfpathcurveto{\pgfqpoint{2.094556in}{1.942485in}}{\pgfqpoint{2.086656in}{1.945757in}}{\pgfqpoint{2.078420in}{1.945757in}}%
\pgfpathcurveto{\pgfqpoint{2.070184in}{1.945757in}}{\pgfqpoint{2.062284in}{1.942485in}}{\pgfqpoint{2.056460in}{1.936661in}}%
\pgfpathcurveto{\pgfqpoint{2.050636in}{1.930837in}}{\pgfqpoint{2.047364in}{1.922937in}}{\pgfqpoint{2.047364in}{1.914700in}}%
\pgfpathcurveto{\pgfqpoint{2.047364in}{1.906464in}}{\pgfqpoint{2.050636in}{1.898564in}}{\pgfqpoint{2.056460in}{1.892740in}}%
\pgfpathcurveto{\pgfqpoint{2.062284in}{1.886916in}}{\pgfqpoint{2.070184in}{1.883644in}}{\pgfqpoint{2.078420in}{1.883644in}}%
\pgfpathclose%
\pgfusepath{stroke,fill}%
\end{pgfscope}%
\begin{pgfscope}%
\pgfpathrectangle{\pgfqpoint{0.100000in}{0.212622in}}{\pgfqpoint{3.696000in}{3.696000in}}%
\pgfusepath{clip}%
\pgfsetbuttcap%
\pgfsetroundjoin%
\definecolor{currentfill}{rgb}{0.121569,0.466667,0.705882}%
\pgfsetfillcolor{currentfill}%
\pgfsetfillopacity{0.947496}%
\pgfsetlinewidth{1.003750pt}%
\definecolor{currentstroke}{rgb}{0.121569,0.466667,0.705882}%
\pgfsetstrokecolor{currentstroke}%
\pgfsetstrokeopacity{0.947496}%
\pgfsetdash{}{0pt}%
\pgfpathmoveto{\pgfqpoint{2.078420in}{1.883644in}}%
\pgfpathcurveto{\pgfqpoint{2.086656in}{1.883644in}}{\pgfqpoint{2.094556in}{1.886916in}}{\pgfqpoint{2.100380in}{1.892740in}}%
\pgfpathcurveto{\pgfqpoint{2.106204in}{1.898564in}}{\pgfqpoint{2.109477in}{1.906464in}}{\pgfqpoint{2.109477in}{1.914700in}}%
\pgfpathcurveto{\pgfqpoint{2.109477in}{1.922937in}}{\pgfqpoint{2.106204in}{1.930837in}}{\pgfqpoint{2.100380in}{1.936661in}}%
\pgfpathcurveto{\pgfqpoint{2.094556in}{1.942485in}}{\pgfqpoint{2.086656in}{1.945757in}}{\pgfqpoint{2.078420in}{1.945757in}}%
\pgfpathcurveto{\pgfqpoint{2.070184in}{1.945757in}}{\pgfqpoint{2.062284in}{1.942485in}}{\pgfqpoint{2.056460in}{1.936661in}}%
\pgfpathcurveto{\pgfqpoint{2.050636in}{1.930837in}}{\pgfqpoint{2.047364in}{1.922937in}}{\pgfqpoint{2.047364in}{1.914700in}}%
\pgfpathcurveto{\pgfqpoint{2.047364in}{1.906464in}}{\pgfqpoint{2.050636in}{1.898564in}}{\pgfqpoint{2.056460in}{1.892740in}}%
\pgfpathcurveto{\pgfqpoint{2.062284in}{1.886916in}}{\pgfqpoint{2.070184in}{1.883644in}}{\pgfqpoint{2.078420in}{1.883644in}}%
\pgfpathclose%
\pgfusepath{stroke,fill}%
\end{pgfscope}%
\begin{pgfscope}%
\pgfpathrectangle{\pgfqpoint{0.100000in}{0.212622in}}{\pgfqpoint{3.696000in}{3.696000in}}%
\pgfusepath{clip}%
\pgfsetbuttcap%
\pgfsetroundjoin%
\definecolor{currentfill}{rgb}{0.121569,0.466667,0.705882}%
\pgfsetfillcolor{currentfill}%
\pgfsetfillopacity{0.947496}%
\pgfsetlinewidth{1.003750pt}%
\definecolor{currentstroke}{rgb}{0.121569,0.466667,0.705882}%
\pgfsetstrokecolor{currentstroke}%
\pgfsetstrokeopacity{0.947496}%
\pgfsetdash{}{0pt}%
\pgfpathmoveto{\pgfqpoint{2.078420in}{1.883644in}}%
\pgfpathcurveto{\pgfqpoint{2.086656in}{1.883644in}}{\pgfqpoint{2.094556in}{1.886916in}}{\pgfqpoint{2.100380in}{1.892740in}}%
\pgfpathcurveto{\pgfqpoint{2.106204in}{1.898564in}}{\pgfqpoint{2.109477in}{1.906464in}}{\pgfqpoint{2.109477in}{1.914700in}}%
\pgfpathcurveto{\pgfqpoint{2.109477in}{1.922937in}}{\pgfqpoint{2.106204in}{1.930837in}}{\pgfqpoint{2.100380in}{1.936661in}}%
\pgfpathcurveto{\pgfqpoint{2.094556in}{1.942485in}}{\pgfqpoint{2.086656in}{1.945757in}}{\pgfqpoint{2.078420in}{1.945757in}}%
\pgfpathcurveto{\pgfqpoint{2.070184in}{1.945757in}}{\pgfqpoint{2.062284in}{1.942485in}}{\pgfqpoint{2.056460in}{1.936661in}}%
\pgfpathcurveto{\pgfqpoint{2.050636in}{1.930837in}}{\pgfqpoint{2.047364in}{1.922937in}}{\pgfqpoint{2.047364in}{1.914700in}}%
\pgfpathcurveto{\pgfqpoint{2.047364in}{1.906464in}}{\pgfqpoint{2.050636in}{1.898564in}}{\pgfqpoint{2.056460in}{1.892740in}}%
\pgfpathcurveto{\pgfqpoint{2.062284in}{1.886916in}}{\pgfqpoint{2.070184in}{1.883644in}}{\pgfqpoint{2.078420in}{1.883644in}}%
\pgfpathclose%
\pgfusepath{stroke,fill}%
\end{pgfscope}%
\begin{pgfscope}%
\pgfpathrectangle{\pgfqpoint{0.100000in}{0.212622in}}{\pgfqpoint{3.696000in}{3.696000in}}%
\pgfusepath{clip}%
\pgfsetbuttcap%
\pgfsetroundjoin%
\definecolor{currentfill}{rgb}{0.121569,0.466667,0.705882}%
\pgfsetfillcolor{currentfill}%
\pgfsetfillopacity{0.947497}%
\pgfsetlinewidth{1.003750pt}%
\definecolor{currentstroke}{rgb}{0.121569,0.466667,0.705882}%
\pgfsetstrokecolor{currentstroke}%
\pgfsetstrokeopacity{0.947497}%
\pgfsetdash{}{0pt}%
\pgfpathmoveto{\pgfqpoint{2.078420in}{1.883644in}}%
\pgfpathcurveto{\pgfqpoint{2.086656in}{1.883644in}}{\pgfqpoint{2.094556in}{1.886916in}}{\pgfqpoint{2.100380in}{1.892740in}}%
\pgfpathcurveto{\pgfqpoint{2.106204in}{1.898564in}}{\pgfqpoint{2.109476in}{1.906464in}}{\pgfqpoint{2.109476in}{1.914700in}}%
\pgfpathcurveto{\pgfqpoint{2.109476in}{1.922936in}}{\pgfqpoint{2.106204in}{1.930837in}}{\pgfqpoint{2.100380in}{1.936660in}}%
\pgfpathcurveto{\pgfqpoint{2.094556in}{1.942484in}}{\pgfqpoint{2.086656in}{1.945757in}}{\pgfqpoint{2.078420in}{1.945757in}}%
\pgfpathcurveto{\pgfqpoint{2.070184in}{1.945757in}}{\pgfqpoint{2.062284in}{1.942484in}}{\pgfqpoint{2.056460in}{1.936660in}}%
\pgfpathcurveto{\pgfqpoint{2.050636in}{1.930837in}}{\pgfqpoint{2.047363in}{1.922936in}}{\pgfqpoint{2.047363in}{1.914700in}}%
\pgfpathcurveto{\pgfqpoint{2.047363in}{1.906464in}}{\pgfqpoint{2.050636in}{1.898564in}}{\pgfqpoint{2.056460in}{1.892740in}}%
\pgfpathcurveto{\pgfqpoint{2.062284in}{1.886916in}}{\pgfqpoint{2.070184in}{1.883644in}}{\pgfqpoint{2.078420in}{1.883644in}}%
\pgfpathclose%
\pgfusepath{stroke,fill}%
\end{pgfscope}%
\begin{pgfscope}%
\pgfpathrectangle{\pgfqpoint{0.100000in}{0.212622in}}{\pgfqpoint{3.696000in}{3.696000in}}%
\pgfusepath{clip}%
\pgfsetbuttcap%
\pgfsetroundjoin%
\definecolor{currentfill}{rgb}{0.121569,0.466667,0.705882}%
\pgfsetfillcolor{currentfill}%
\pgfsetfillopacity{0.947497}%
\pgfsetlinewidth{1.003750pt}%
\definecolor{currentstroke}{rgb}{0.121569,0.466667,0.705882}%
\pgfsetstrokecolor{currentstroke}%
\pgfsetstrokeopacity{0.947497}%
\pgfsetdash{}{0pt}%
\pgfpathmoveto{\pgfqpoint{2.078420in}{1.883643in}}%
\pgfpathcurveto{\pgfqpoint{2.086656in}{1.883643in}}{\pgfqpoint{2.094556in}{1.886916in}}{\pgfqpoint{2.100380in}{1.892740in}}%
\pgfpathcurveto{\pgfqpoint{2.106204in}{1.898564in}}{\pgfqpoint{2.109476in}{1.906464in}}{\pgfqpoint{2.109476in}{1.914700in}}%
\pgfpathcurveto{\pgfqpoint{2.109476in}{1.922936in}}{\pgfqpoint{2.106204in}{1.930836in}}{\pgfqpoint{2.100380in}{1.936660in}}%
\pgfpathcurveto{\pgfqpoint{2.094556in}{1.942484in}}{\pgfqpoint{2.086656in}{1.945756in}}{\pgfqpoint{2.078420in}{1.945756in}}%
\pgfpathcurveto{\pgfqpoint{2.070184in}{1.945756in}}{\pgfqpoint{2.062283in}{1.942484in}}{\pgfqpoint{2.056460in}{1.936660in}}%
\pgfpathcurveto{\pgfqpoint{2.050636in}{1.930836in}}{\pgfqpoint{2.047363in}{1.922936in}}{\pgfqpoint{2.047363in}{1.914700in}}%
\pgfpathcurveto{\pgfqpoint{2.047363in}{1.906464in}}{\pgfqpoint{2.050636in}{1.898564in}}{\pgfqpoint{2.056460in}{1.892740in}}%
\pgfpathcurveto{\pgfqpoint{2.062283in}{1.886916in}}{\pgfqpoint{2.070184in}{1.883643in}}{\pgfqpoint{2.078420in}{1.883643in}}%
\pgfpathclose%
\pgfusepath{stroke,fill}%
\end{pgfscope}%
\begin{pgfscope}%
\pgfpathrectangle{\pgfqpoint{0.100000in}{0.212622in}}{\pgfqpoint{3.696000in}{3.696000in}}%
\pgfusepath{clip}%
\pgfsetbuttcap%
\pgfsetroundjoin%
\definecolor{currentfill}{rgb}{0.121569,0.466667,0.705882}%
\pgfsetfillcolor{currentfill}%
\pgfsetfillopacity{0.947497}%
\pgfsetlinewidth{1.003750pt}%
\definecolor{currentstroke}{rgb}{0.121569,0.466667,0.705882}%
\pgfsetstrokecolor{currentstroke}%
\pgfsetstrokeopacity{0.947497}%
\pgfsetdash{}{0pt}%
\pgfpathmoveto{\pgfqpoint{2.078420in}{1.883643in}}%
\pgfpathcurveto{\pgfqpoint{2.086656in}{1.883643in}}{\pgfqpoint{2.094556in}{1.886915in}}{\pgfqpoint{2.100380in}{1.892739in}}%
\pgfpathcurveto{\pgfqpoint{2.106204in}{1.898563in}}{\pgfqpoint{2.109476in}{1.906463in}}{\pgfqpoint{2.109476in}{1.914699in}}%
\pgfpathcurveto{\pgfqpoint{2.109476in}{1.922936in}}{\pgfqpoint{2.106204in}{1.930836in}}{\pgfqpoint{2.100380in}{1.936660in}}%
\pgfpathcurveto{\pgfqpoint{2.094556in}{1.942484in}}{\pgfqpoint{2.086656in}{1.945756in}}{\pgfqpoint{2.078420in}{1.945756in}}%
\pgfpathcurveto{\pgfqpoint{2.070183in}{1.945756in}}{\pgfqpoint{2.062283in}{1.942484in}}{\pgfqpoint{2.056459in}{1.936660in}}%
\pgfpathcurveto{\pgfqpoint{2.050635in}{1.930836in}}{\pgfqpoint{2.047363in}{1.922936in}}{\pgfqpoint{2.047363in}{1.914699in}}%
\pgfpathcurveto{\pgfqpoint{2.047363in}{1.906463in}}{\pgfqpoint{2.050635in}{1.898563in}}{\pgfqpoint{2.056459in}{1.892739in}}%
\pgfpathcurveto{\pgfqpoint{2.062283in}{1.886915in}}{\pgfqpoint{2.070183in}{1.883643in}}{\pgfqpoint{2.078420in}{1.883643in}}%
\pgfpathclose%
\pgfusepath{stroke,fill}%
\end{pgfscope}%
\begin{pgfscope}%
\pgfpathrectangle{\pgfqpoint{0.100000in}{0.212622in}}{\pgfqpoint{3.696000in}{3.696000in}}%
\pgfusepath{clip}%
\pgfsetbuttcap%
\pgfsetroundjoin%
\definecolor{currentfill}{rgb}{0.121569,0.466667,0.705882}%
\pgfsetfillcolor{currentfill}%
\pgfsetfillopacity{0.947497}%
\pgfsetlinewidth{1.003750pt}%
\definecolor{currentstroke}{rgb}{0.121569,0.466667,0.705882}%
\pgfsetstrokecolor{currentstroke}%
\pgfsetstrokeopacity{0.947497}%
\pgfsetdash{}{0pt}%
\pgfpathmoveto{\pgfqpoint{2.078419in}{1.883642in}}%
\pgfpathcurveto{\pgfqpoint{2.086655in}{1.883642in}}{\pgfqpoint{2.094555in}{1.886914in}}{\pgfqpoint{2.100379in}{1.892738in}}%
\pgfpathcurveto{\pgfqpoint{2.106203in}{1.898562in}}{\pgfqpoint{2.109476in}{1.906462in}}{\pgfqpoint{2.109476in}{1.914699in}}%
\pgfpathcurveto{\pgfqpoint{2.109476in}{1.922935in}}{\pgfqpoint{2.106203in}{1.930835in}}{\pgfqpoint{2.100379in}{1.936659in}}%
\pgfpathcurveto{\pgfqpoint{2.094555in}{1.942483in}}{\pgfqpoint{2.086655in}{1.945755in}}{\pgfqpoint{2.078419in}{1.945755in}}%
\pgfpathcurveto{\pgfqpoint{2.070183in}{1.945755in}}{\pgfqpoint{2.062283in}{1.942483in}}{\pgfqpoint{2.056459in}{1.936659in}}%
\pgfpathcurveto{\pgfqpoint{2.050635in}{1.930835in}}{\pgfqpoint{2.047363in}{1.922935in}}{\pgfqpoint{2.047363in}{1.914699in}}%
\pgfpathcurveto{\pgfqpoint{2.047363in}{1.906462in}}{\pgfqpoint{2.050635in}{1.898562in}}{\pgfqpoint{2.056459in}{1.892738in}}%
\pgfpathcurveto{\pgfqpoint{2.062283in}{1.886914in}}{\pgfqpoint{2.070183in}{1.883642in}}{\pgfqpoint{2.078419in}{1.883642in}}%
\pgfpathclose%
\pgfusepath{stroke,fill}%
\end{pgfscope}%
\begin{pgfscope}%
\pgfpathrectangle{\pgfqpoint{0.100000in}{0.212622in}}{\pgfqpoint{3.696000in}{3.696000in}}%
\pgfusepath{clip}%
\pgfsetbuttcap%
\pgfsetroundjoin%
\definecolor{currentfill}{rgb}{0.121569,0.466667,0.705882}%
\pgfsetfillcolor{currentfill}%
\pgfsetfillopacity{0.947497}%
\pgfsetlinewidth{1.003750pt}%
\definecolor{currentstroke}{rgb}{0.121569,0.466667,0.705882}%
\pgfsetstrokecolor{currentstroke}%
\pgfsetstrokeopacity{0.947497}%
\pgfsetdash{}{0pt}%
\pgfpathmoveto{\pgfqpoint{2.078418in}{1.883641in}}%
\pgfpathcurveto{\pgfqpoint{2.086655in}{1.883641in}}{\pgfqpoint{2.094555in}{1.886913in}}{\pgfqpoint{2.100379in}{1.892737in}}%
\pgfpathcurveto{\pgfqpoint{2.106203in}{1.898561in}}{\pgfqpoint{2.109475in}{1.906461in}}{\pgfqpoint{2.109475in}{1.914697in}}%
\pgfpathcurveto{\pgfqpoint{2.109475in}{1.922933in}}{\pgfqpoint{2.106203in}{1.930833in}}{\pgfqpoint{2.100379in}{1.936657in}}%
\pgfpathcurveto{\pgfqpoint{2.094555in}{1.942481in}}{\pgfqpoint{2.086655in}{1.945754in}}{\pgfqpoint{2.078418in}{1.945754in}}%
\pgfpathcurveto{\pgfqpoint{2.070182in}{1.945754in}}{\pgfqpoint{2.062282in}{1.942481in}}{\pgfqpoint{2.056458in}{1.936657in}}%
\pgfpathcurveto{\pgfqpoint{2.050634in}{1.930833in}}{\pgfqpoint{2.047362in}{1.922933in}}{\pgfqpoint{2.047362in}{1.914697in}}%
\pgfpathcurveto{\pgfqpoint{2.047362in}{1.906461in}}{\pgfqpoint{2.050634in}{1.898561in}}{\pgfqpoint{2.056458in}{1.892737in}}%
\pgfpathcurveto{\pgfqpoint{2.062282in}{1.886913in}}{\pgfqpoint{2.070182in}{1.883641in}}{\pgfqpoint{2.078418in}{1.883641in}}%
\pgfpathclose%
\pgfusepath{stroke,fill}%
\end{pgfscope}%
\begin{pgfscope}%
\pgfpathrectangle{\pgfqpoint{0.100000in}{0.212622in}}{\pgfqpoint{3.696000in}{3.696000in}}%
\pgfusepath{clip}%
\pgfsetbuttcap%
\pgfsetroundjoin%
\definecolor{currentfill}{rgb}{0.121569,0.466667,0.705882}%
\pgfsetfillcolor{currentfill}%
\pgfsetfillopacity{0.947498}%
\pgfsetlinewidth{1.003750pt}%
\definecolor{currentstroke}{rgb}{0.121569,0.466667,0.705882}%
\pgfsetstrokecolor{currentstroke}%
\pgfsetstrokeopacity{0.947498}%
\pgfsetdash{}{0pt}%
\pgfpathmoveto{\pgfqpoint{2.078417in}{1.883638in}}%
\pgfpathcurveto{\pgfqpoint{2.086653in}{1.883638in}}{\pgfqpoint{2.094553in}{1.886910in}}{\pgfqpoint{2.100377in}{1.892734in}}%
\pgfpathcurveto{\pgfqpoint{2.106201in}{1.898558in}}{\pgfqpoint{2.109473in}{1.906458in}}{\pgfqpoint{2.109473in}{1.914694in}}%
\pgfpathcurveto{\pgfqpoint{2.109473in}{1.922931in}}{\pgfqpoint{2.106201in}{1.930831in}}{\pgfqpoint{2.100377in}{1.936655in}}%
\pgfpathcurveto{\pgfqpoint{2.094553in}{1.942478in}}{\pgfqpoint{2.086653in}{1.945751in}}{\pgfqpoint{2.078417in}{1.945751in}}%
\pgfpathcurveto{\pgfqpoint{2.070180in}{1.945751in}}{\pgfqpoint{2.062280in}{1.942478in}}{\pgfqpoint{2.056457in}{1.936655in}}%
\pgfpathcurveto{\pgfqpoint{2.050633in}{1.930831in}}{\pgfqpoint{2.047360in}{1.922931in}}{\pgfqpoint{2.047360in}{1.914694in}}%
\pgfpathcurveto{\pgfqpoint{2.047360in}{1.906458in}}{\pgfqpoint{2.050633in}{1.898558in}}{\pgfqpoint{2.056457in}{1.892734in}}%
\pgfpathcurveto{\pgfqpoint{2.062280in}{1.886910in}}{\pgfqpoint{2.070180in}{1.883638in}}{\pgfqpoint{2.078417in}{1.883638in}}%
\pgfpathclose%
\pgfusepath{stroke,fill}%
\end{pgfscope}%
\begin{pgfscope}%
\pgfpathrectangle{\pgfqpoint{0.100000in}{0.212622in}}{\pgfqpoint{3.696000in}{3.696000in}}%
\pgfusepath{clip}%
\pgfsetbuttcap%
\pgfsetroundjoin%
\definecolor{currentfill}{rgb}{0.121569,0.466667,0.705882}%
\pgfsetfillcolor{currentfill}%
\pgfsetfillopacity{0.947499}%
\pgfsetlinewidth{1.003750pt}%
\definecolor{currentstroke}{rgb}{0.121569,0.466667,0.705882}%
\pgfsetstrokecolor{currentstroke}%
\pgfsetstrokeopacity{0.947499}%
\pgfsetdash{}{0pt}%
\pgfpathmoveto{\pgfqpoint{2.078414in}{1.883633in}}%
\pgfpathcurveto{\pgfqpoint{2.086650in}{1.883633in}}{\pgfqpoint{2.094550in}{1.886905in}}{\pgfqpoint{2.100374in}{1.892729in}}%
\pgfpathcurveto{\pgfqpoint{2.106198in}{1.898553in}}{\pgfqpoint{2.109471in}{1.906453in}}{\pgfqpoint{2.109471in}{1.914689in}}%
\pgfpathcurveto{\pgfqpoint{2.109471in}{1.922925in}}{\pgfqpoint{2.106198in}{1.930826in}}{\pgfqpoint{2.100374in}{1.936649in}}%
\pgfpathcurveto{\pgfqpoint{2.094550in}{1.942473in}}{\pgfqpoint{2.086650in}{1.945746in}}{\pgfqpoint{2.078414in}{1.945746in}}%
\pgfpathcurveto{\pgfqpoint{2.070178in}{1.945746in}}{\pgfqpoint{2.062278in}{1.942473in}}{\pgfqpoint{2.056454in}{1.936649in}}%
\pgfpathcurveto{\pgfqpoint{2.050630in}{1.930826in}}{\pgfqpoint{2.047358in}{1.922925in}}{\pgfqpoint{2.047358in}{1.914689in}}%
\pgfpathcurveto{\pgfqpoint{2.047358in}{1.906453in}}{\pgfqpoint{2.050630in}{1.898553in}}{\pgfqpoint{2.056454in}{1.892729in}}%
\pgfpathcurveto{\pgfqpoint{2.062278in}{1.886905in}}{\pgfqpoint{2.070178in}{1.883633in}}{\pgfqpoint{2.078414in}{1.883633in}}%
\pgfpathclose%
\pgfusepath{stroke,fill}%
\end{pgfscope}%
\begin{pgfscope}%
\pgfpathrectangle{\pgfqpoint{0.100000in}{0.212622in}}{\pgfqpoint{3.696000in}{3.696000in}}%
\pgfusepath{clip}%
\pgfsetbuttcap%
\pgfsetroundjoin%
\definecolor{currentfill}{rgb}{0.121569,0.466667,0.705882}%
\pgfsetfillcolor{currentfill}%
\pgfsetfillopacity{0.947501}%
\pgfsetlinewidth{1.003750pt}%
\definecolor{currentstroke}{rgb}{0.121569,0.466667,0.705882}%
\pgfsetstrokecolor{currentstroke}%
\pgfsetstrokeopacity{0.947501}%
\pgfsetdash{}{0pt}%
\pgfpathmoveto{\pgfqpoint{2.078409in}{1.883625in}}%
\pgfpathcurveto{\pgfqpoint{2.086645in}{1.883625in}}{\pgfqpoint{2.094545in}{1.886897in}}{\pgfqpoint{2.100369in}{1.892721in}}%
\pgfpathcurveto{\pgfqpoint{2.106193in}{1.898545in}}{\pgfqpoint{2.109466in}{1.906445in}}{\pgfqpoint{2.109466in}{1.914681in}}%
\pgfpathcurveto{\pgfqpoint{2.109466in}{1.922917in}}{\pgfqpoint{2.106193in}{1.930817in}}{\pgfqpoint{2.100369in}{1.936641in}}%
\pgfpathcurveto{\pgfqpoint{2.094545in}{1.942465in}}{\pgfqpoint{2.086645in}{1.945738in}}{\pgfqpoint{2.078409in}{1.945738in}}%
\pgfpathcurveto{\pgfqpoint{2.070173in}{1.945738in}}{\pgfqpoint{2.062273in}{1.942465in}}{\pgfqpoint{2.056449in}{1.936641in}}%
\pgfpathcurveto{\pgfqpoint{2.050625in}{1.930817in}}{\pgfqpoint{2.047353in}{1.922917in}}{\pgfqpoint{2.047353in}{1.914681in}}%
\pgfpathcurveto{\pgfqpoint{2.047353in}{1.906445in}}{\pgfqpoint{2.050625in}{1.898545in}}{\pgfqpoint{2.056449in}{1.892721in}}%
\pgfpathcurveto{\pgfqpoint{2.062273in}{1.886897in}}{\pgfqpoint{2.070173in}{1.883625in}}{\pgfqpoint{2.078409in}{1.883625in}}%
\pgfpathclose%
\pgfusepath{stroke,fill}%
\end{pgfscope}%
\begin{pgfscope}%
\pgfpathrectangle{\pgfqpoint{0.100000in}{0.212622in}}{\pgfqpoint{3.696000in}{3.696000in}}%
\pgfusepath{clip}%
\pgfsetbuttcap%
\pgfsetroundjoin%
\definecolor{currentfill}{rgb}{0.121569,0.466667,0.705882}%
\pgfsetfillcolor{currentfill}%
\pgfsetfillopacity{0.947505}%
\pgfsetlinewidth{1.003750pt}%
\definecolor{currentstroke}{rgb}{0.121569,0.466667,0.705882}%
\pgfsetstrokecolor{currentstroke}%
\pgfsetstrokeopacity{0.947505}%
\pgfsetdash{}{0pt}%
\pgfpathmoveto{\pgfqpoint{2.078400in}{1.883609in}}%
\pgfpathcurveto{\pgfqpoint{2.086636in}{1.883609in}}{\pgfqpoint{2.094536in}{1.886881in}}{\pgfqpoint{2.100360in}{1.892705in}}%
\pgfpathcurveto{\pgfqpoint{2.106184in}{1.898529in}}{\pgfqpoint{2.109456in}{1.906429in}}{\pgfqpoint{2.109456in}{1.914666in}}%
\pgfpathcurveto{\pgfqpoint{2.109456in}{1.922902in}}{\pgfqpoint{2.106184in}{1.930802in}}{\pgfqpoint{2.100360in}{1.936626in}}%
\pgfpathcurveto{\pgfqpoint{2.094536in}{1.942450in}}{\pgfqpoint{2.086636in}{1.945722in}}{\pgfqpoint{2.078400in}{1.945722in}}%
\pgfpathcurveto{\pgfqpoint{2.070163in}{1.945722in}}{\pgfqpoint{2.062263in}{1.942450in}}{\pgfqpoint{2.056439in}{1.936626in}}%
\pgfpathcurveto{\pgfqpoint{2.050615in}{1.930802in}}{\pgfqpoint{2.047343in}{1.922902in}}{\pgfqpoint{2.047343in}{1.914666in}}%
\pgfpathcurveto{\pgfqpoint{2.047343in}{1.906429in}}{\pgfqpoint{2.050615in}{1.898529in}}{\pgfqpoint{2.056439in}{1.892705in}}%
\pgfpathcurveto{\pgfqpoint{2.062263in}{1.886881in}}{\pgfqpoint{2.070163in}{1.883609in}}{\pgfqpoint{2.078400in}{1.883609in}}%
\pgfpathclose%
\pgfusepath{stroke,fill}%
\end{pgfscope}%
\begin{pgfscope}%
\pgfpathrectangle{\pgfqpoint{0.100000in}{0.212622in}}{\pgfqpoint{3.696000in}{3.696000in}}%
\pgfusepath{clip}%
\pgfsetbuttcap%
\pgfsetroundjoin%
\definecolor{currentfill}{rgb}{0.121569,0.466667,0.705882}%
\pgfsetfillcolor{currentfill}%
\pgfsetfillopacity{0.947512}%
\pgfsetlinewidth{1.003750pt}%
\definecolor{currentstroke}{rgb}{0.121569,0.466667,0.705882}%
\pgfsetstrokecolor{currentstroke}%
\pgfsetstrokeopacity{0.947512}%
\pgfsetdash{}{0pt}%
\pgfpathmoveto{\pgfqpoint{2.078383in}{1.883579in}}%
\pgfpathcurveto{\pgfqpoint{2.086619in}{1.883579in}}{\pgfqpoint{2.094520in}{1.886851in}}{\pgfqpoint{2.100343in}{1.892675in}}%
\pgfpathcurveto{\pgfqpoint{2.106167in}{1.898499in}}{\pgfqpoint{2.109440in}{1.906399in}}{\pgfqpoint{2.109440in}{1.914635in}}%
\pgfpathcurveto{\pgfqpoint{2.109440in}{1.922871in}}{\pgfqpoint{2.106167in}{1.930771in}}{\pgfqpoint{2.100343in}{1.936595in}}%
\pgfpathcurveto{\pgfqpoint{2.094520in}{1.942419in}}{\pgfqpoint{2.086619in}{1.945692in}}{\pgfqpoint{2.078383in}{1.945692in}}%
\pgfpathcurveto{\pgfqpoint{2.070147in}{1.945692in}}{\pgfqpoint{2.062247in}{1.942419in}}{\pgfqpoint{2.056423in}{1.936595in}}%
\pgfpathcurveto{\pgfqpoint{2.050599in}{1.930771in}}{\pgfqpoint{2.047327in}{1.922871in}}{\pgfqpoint{2.047327in}{1.914635in}}%
\pgfpathcurveto{\pgfqpoint{2.047327in}{1.906399in}}{\pgfqpoint{2.050599in}{1.898499in}}{\pgfqpoint{2.056423in}{1.892675in}}%
\pgfpathcurveto{\pgfqpoint{2.062247in}{1.886851in}}{\pgfqpoint{2.070147in}{1.883579in}}{\pgfqpoint{2.078383in}{1.883579in}}%
\pgfpathclose%
\pgfusepath{stroke,fill}%
\end{pgfscope}%
\begin{pgfscope}%
\pgfpathrectangle{\pgfqpoint{0.100000in}{0.212622in}}{\pgfqpoint{3.696000in}{3.696000in}}%
\pgfusepath{clip}%
\pgfsetbuttcap%
\pgfsetroundjoin%
\definecolor{currentfill}{rgb}{0.121569,0.466667,0.705882}%
\pgfsetfillcolor{currentfill}%
\pgfsetfillopacity{0.947526}%
\pgfsetlinewidth{1.003750pt}%
\definecolor{currentstroke}{rgb}{0.121569,0.466667,0.705882}%
\pgfsetstrokecolor{currentstroke}%
\pgfsetstrokeopacity{0.947526}%
\pgfsetdash{}{0pt}%
\pgfpathmoveto{\pgfqpoint{2.078353in}{1.883526in}}%
\pgfpathcurveto{\pgfqpoint{2.086590in}{1.883526in}}{\pgfqpoint{2.094490in}{1.886799in}}{\pgfqpoint{2.100314in}{1.892622in}}%
\pgfpathcurveto{\pgfqpoint{2.106138in}{1.898446in}}{\pgfqpoint{2.109410in}{1.906346in}}{\pgfqpoint{2.109410in}{1.914583in}}%
\pgfpathcurveto{\pgfqpoint{2.109410in}{1.922819in}}{\pgfqpoint{2.106138in}{1.930719in}}{\pgfqpoint{2.100314in}{1.936543in}}%
\pgfpathcurveto{\pgfqpoint{2.094490in}{1.942367in}}{\pgfqpoint{2.086590in}{1.945639in}}{\pgfqpoint{2.078353in}{1.945639in}}%
\pgfpathcurveto{\pgfqpoint{2.070117in}{1.945639in}}{\pgfqpoint{2.062217in}{1.942367in}}{\pgfqpoint{2.056393in}{1.936543in}}%
\pgfpathcurveto{\pgfqpoint{2.050569in}{1.930719in}}{\pgfqpoint{2.047297in}{1.922819in}}{\pgfqpoint{2.047297in}{1.914583in}}%
\pgfpathcurveto{\pgfqpoint{2.047297in}{1.906346in}}{\pgfqpoint{2.050569in}{1.898446in}}{\pgfqpoint{2.056393in}{1.892622in}}%
\pgfpathcurveto{\pgfqpoint{2.062217in}{1.886799in}}{\pgfqpoint{2.070117in}{1.883526in}}{\pgfqpoint{2.078353in}{1.883526in}}%
\pgfpathclose%
\pgfusepath{stroke,fill}%
\end{pgfscope}%
\begin{pgfscope}%
\pgfpathrectangle{\pgfqpoint{0.100000in}{0.212622in}}{\pgfqpoint{3.696000in}{3.696000in}}%
\pgfusepath{clip}%
\pgfsetbuttcap%
\pgfsetroundjoin%
\definecolor{currentfill}{rgb}{0.121569,0.466667,0.705882}%
\pgfsetfillcolor{currentfill}%
\pgfsetfillopacity{0.947549}%
\pgfsetlinewidth{1.003750pt}%
\definecolor{currentstroke}{rgb}{0.121569,0.466667,0.705882}%
\pgfsetstrokecolor{currentstroke}%
\pgfsetstrokeopacity{0.947549}%
\pgfsetdash{}{0pt}%
\pgfpathmoveto{\pgfqpoint{2.078300in}{1.883426in}}%
\pgfpathcurveto{\pgfqpoint{2.086537in}{1.883426in}}{\pgfqpoint{2.094437in}{1.886698in}}{\pgfqpoint{2.100261in}{1.892522in}}%
\pgfpathcurveto{\pgfqpoint{2.106085in}{1.898346in}}{\pgfqpoint{2.109357in}{1.906246in}}{\pgfqpoint{2.109357in}{1.914483in}}%
\pgfpathcurveto{\pgfqpoint{2.109357in}{1.922719in}}{\pgfqpoint{2.106085in}{1.930619in}}{\pgfqpoint{2.100261in}{1.936443in}}%
\pgfpathcurveto{\pgfqpoint{2.094437in}{1.942267in}}{\pgfqpoint{2.086537in}{1.945539in}}{\pgfqpoint{2.078300in}{1.945539in}}%
\pgfpathcurveto{\pgfqpoint{2.070064in}{1.945539in}}{\pgfqpoint{2.062164in}{1.942267in}}{\pgfqpoint{2.056340in}{1.936443in}}%
\pgfpathcurveto{\pgfqpoint{2.050516in}{1.930619in}}{\pgfqpoint{2.047244in}{1.922719in}}{\pgfqpoint{2.047244in}{1.914483in}}%
\pgfpathcurveto{\pgfqpoint{2.047244in}{1.906246in}}{\pgfqpoint{2.050516in}{1.898346in}}{\pgfqpoint{2.056340in}{1.892522in}}%
\pgfpathcurveto{\pgfqpoint{2.062164in}{1.886698in}}{\pgfqpoint{2.070064in}{1.883426in}}{\pgfqpoint{2.078300in}{1.883426in}}%
\pgfpathclose%
\pgfusepath{stroke,fill}%
\end{pgfscope}%
\begin{pgfscope}%
\pgfpathrectangle{\pgfqpoint{0.100000in}{0.212622in}}{\pgfqpoint{3.696000in}{3.696000in}}%
\pgfusepath{clip}%
\pgfsetbuttcap%
\pgfsetroundjoin%
\definecolor{currentfill}{rgb}{0.121569,0.466667,0.705882}%
\pgfsetfillcolor{currentfill}%
\pgfsetfillopacity{0.947593}%
\pgfsetlinewidth{1.003750pt}%
\definecolor{currentstroke}{rgb}{0.121569,0.466667,0.705882}%
\pgfsetstrokecolor{currentstroke}%
\pgfsetstrokeopacity{0.947593}%
\pgfsetdash{}{0pt}%
\pgfpathmoveto{\pgfqpoint{2.078204in}{1.883248in}}%
\pgfpathcurveto{\pgfqpoint{2.086440in}{1.883248in}}{\pgfqpoint{2.094340in}{1.886520in}}{\pgfqpoint{2.100164in}{1.892344in}}%
\pgfpathcurveto{\pgfqpoint{2.105988in}{1.898168in}}{\pgfqpoint{2.109260in}{1.906068in}}{\pgfqpoint{2.109260in}{1.914304in}}%
\pgfpathcurveto{\pgfqpoint{2.109260in}{1.922541in}}{\pgfqpoint{2.105988in}{1.930441in}}{\pgfqpoint{2.100164in}{1.936265in}}%
\pgfpathcurveto{\pgfqpoint{2.094340in}{1.942089in}}{\pgfqpoint{2.086440in}{1.945361in}}{\pgfqpoint{2.078204in}{1.945361in}}%
\pgfpathcurveto{\pgfqpoint{2.069968in}{1.945361in}}{\pgfqpoint{2.062068in}{1.942089in}}{\pgfqpoint{2.056244in}{1.936265in}}%
\pgfpathcurveto{\pgfqpoint{2.050420in}{1.930441in}}{\pgfqpoint{2.047147in}{1.922541in}}{\pgfqpoint{2.047147in}{1.914304in}}%
\pgfpathcurveto{\pgfqpoint{2.047147in}{1.906068in}}{\pgfqpoint{2.050420in}{1.898168in}}{\pgfqpoint{2.056244in}{1.892344in}}%
\pgfpathcurveto{\pgfqpoint{2.062068in}{1.886520in}}{\pgfqpoint{2.069968in}{1.883248in}}{\pgfqpoint{2.078204in}{1.883248in}}%
\pgfpathclose%
\pgfusepath{stroke,fill}%
\end{pgfscope}%
\begin{pgfscope}%
\pgfpathrectangle{\pgfqpoint{0.100000in}{0.212622in}}{\pgfqpoint{3.696000in}{3.696000in}}%
\pgfusepath{clip}%
\pgfsetbuttcap%
\pgfsetroundjoin%
\definecolor{currentfill}{rgb}{0.121569,0.466667,0.705882}%
\pgfsetfillcolor{currentfill}%
\pgfsetfillopacity{0.947622}%
\pgfsetlinewidth{1.003750pt}%
\definecolor{currentstroke}{rgb}{0.121569,0.466667,0.705882}%
\pgfsetstrokecolor{currentstroke}%
\pgfsetstrokeopacity{0.947622}%
\pgfsetdash{}{0pt}%
\pgfpathmoveto{\pgfqpoint{2.679708in}{1.098718in}}%
\pgfpathcurveto{\pgfqpoint{2.687944in}{1.098718in}}{\pgfqpoint{2.695844in}{1.101990in}}{\pgfqpoint{2.701668in}{1.107814in}}%
\pgfpathcurveto{\pgfqpoint{2.707492in}{1.113638in}}{\pgfqpoint{2.710765in}{1.121538in}}{\pgfqpoint{2.710765in}{1.129775in}}%
\pgfpathcurveto{\pgfqpoint{2.710765in}{1.138011in}}{\pgfqpoint{2.707492in}{1.145911in}}{\pgfqpoint{2.701668in}{1.151735in}}%
\pgfpathcurveto{\pgfqpoint{2.695844in}{1.157559in}}{\pgfqpoint{2.687944in}{1.160831in}}{\pgfqpoint{2.679708in}{1.160831in}}%
\pgfpathcurveto{\pgfqpoint{2.671472in}{1.160831in}}{\pgfqpoint{2.663572in}{1.157559in}}{\pgfqpoint{2.657748in}{1.151735in}}%
\pgfpathcurveto{\pgfqpoint{2.651924in}{1.145911in}}{\pgfqpoint{2.648652in}{1.138011in}}{\pgfqpoint{2.648652in}{1.129775in}}%
\pgfpathcurveto{\pgfqpoint{2.648652in}{1.121538in}}{\pgfqpoint{2.651924in}{1.113638in}}{\pgfqpoint{2.657748in}{1.107814in}}%
\pgfpathcurveto{\pgfqpoint{2.663572in}{1.101990in}}{\pgfqpoint{2.671472in}{1.098718in}}{\pgfqpoint{2.679708in}{1.098718in}}%
\pgfpathclose%
\pgfusepath{stroke,fill}%
\end{pgfscope}%
\begin{pgfscope}%
\pgfpathrectangle{\pgfqpoint{0.100000in}{0.212622in}}{\pgfqpoint{3.696000in}{3.696000in}}%
\pgfusepath{clip}%
\pgfsetbuttcap%
\pgfsetroundjoin%
\definecolor{currentfill}{rgb}{0.121569,0.466667,0.705882}%
\pgfsetfillcolor{currentfill}%
\pgfsetfillopacity{0.947639}%
\pgfsetlinewidth{1.003750pt}%
\definecolor{currentstroke}{rgb}{0.121569,0.466667,0.705882}%
\pgfsetstrokecolor{currentstroke}%
\pgfsetstrokeopacity{0.947639}%
\pgfsetdash{}{0pt}%
\pgfpathmoveto{\pgfqpoint{2.038605in}{1.864206in}}%
\pgfpathcurveto{\pgfqpoint{2.046841in}{1.864206in}}{\pgfqpoint{2.054741in}{1.867478in}}{\pgfqpoint{2.060565in}{1.873302in}}%
\pgfpathcurveto{\pgfqpoint{2.066389in}{1.879126in}}{\pgfqpoint{2.069662in}{1.887026in}}{\pgfqpoint{2.069662in}{1.895263in}}%
\pgfpathcurveto{\pgfqpoint{2.069662in}{1.903499in}}{\pgfqpoint{2.066389in}{1.911399in}}{\pgfqpoint{2.060565in}{1.917223in}}%
\pgfpathcurveto{\pgfqpoint{2.054741in}{1.923047in}}{\pgfqpoint{2.046841in}{1.926319in}}{\pgfqpoint{2.038605in}{1.926319in}}%
\pgfpathcurveto{\pgfqpoint{2.030369in}{1.926319in}}{\pgfqpoint{2.022469in}{1.923047in}}{\pgfqpoint{2.016645in}{1.917223in}}%
\pgfpathcurveto{\pgfqpoint{2.010821in}{1.911399in}}{\pgfqpoint{2.007549in}{1.903499in}}{\pgfqpoint{2.007549in}{1.895263in}}%
\pgfpathcurveto{\pgfqpoint{2.007549in}{1.887026in}}{\pgfqpoint{2.010821in}{1.879126in}}{\pgfqpoint{2.016645in}{1.873302in}}%
\pgfpathcurveto{\pgfqpoint{2.022469in}{1.867478in}}{\pgfqpoint{2.030369in}{1.864206in}}{\pgfqpoint{2.038605in}{1.864206in}}%
\pgfpathclose%
\pgfusepath{stroke,fill}%
\end{pgfscope}%
\begin{pgfscope}%
\pgfpathrectangle{\pgfqpoint{0.100000in}{0.212622in}}{\pgfqpoint{3.696000in}{3.696000in}}%
\pgfusepath{clip}%
\pgfsetbuttcap%
\pgfsetroundjoin%
\definecolor{currentfill}{rgb}{0.121569,0.466667,0.705882}%
\pgfsetfillcolor{currentfill}%
\pgfsetfillopacity{0.947647}%
\pgfsetlinewidth{1.003750pt}%
\definecolor{currentstroke}{rgb}{0.121569,0.466667,0.705882}%
\pgfsetstrokecolor{currentstroke}%
\pgfsetstrokeopacity{0.947647}%
\pgfsetdash{}{0pt}%
\pgfpathmoveto{\pgfqpoint{2.106047in}{0.976765in}}%
\pgfpathcurveto{\pgfqpoint{2.114283in}{0.976765in}}{\pgfqpoint{2.122183in}{0.980037in}}{\pgfqpoint{2.128007in}{0.985861in}}%
\pgfpathcurveto{\pgfqpoint{2.133831in}{0.991685in}}{\pgfqpoint{2.137103in}{0.999585in}}{\pgfqpoint{2.137103in}{1.007821in}}%
\pgfpathcurveto{\pgfqpoint{2.137103in}{1.016057in}}{\pgfqpoint{2.133831in}{1.023957in}}{\pgfqpoint{2.128007in}{1.029781in}}%
\pgfpathcurveto{\pgfqpoint{2.122183in}{1.035605in}}{\pgfqpoint{2.114283in}{1.038878in}}{\pgfqpoint{2.106047in}{1.038878in}}%
\pgfpathcurveto{\pgfqpoint{2.097811in}{1.038878in}}{\pgfqpoint{2.089911in}{1.035605in}}{\pgfqpoint{2.084087in}{1.029781in}}%
\pgfpathcurveto{\pgfqpoint{2.078263in}{1.023957in}}{\pgfqpoint{2.074990in}{1.016057in}}{\pgfqpoint{2.074990in}{1.007821in}}%
\pgfpathcurveto{\pgfqpoint{2.074990in}{0.999585in}}{\pgfqpoint{2.078263in}{0.991685in}}{\pgfqpoint{2.084087in}{0.985861in}}%
\pgfpathcurveto{\pgfqpoint{2.089911in}{0.980037in}}{\pgfqpoint{2.097811in}{0.976765in}}{\pgfqpoint{2.106047in}{0.976765in}}%
\pgfpathclose%
\pgfusepath{stroke,fill}%
\end{pgfscope}%
\begin{pgfscope}%
\pgfpathrectangle{\pgfqpoint{0.100000in}{0.212622in}}{\pgfqpoint{3.696000in}{3.696000in}}%
\pgfusepath{clip}%
\pgfsetbuttcap%
\pgfsetroundjoin%
\definecolor{currentfill}{rgb}{0.121569,0.466667,0.705882}%
\pgfsetfillcolor{currentfill}%
\pgfsetfillopacity{0.947672}%
\pgfsetlinewidth{1.003750pt}%
\definecolor{currentstroke}{rgb}{0.121569,0.466667,0.705882}%
\pgfsetstrokecolor{currentstroke}%
\pgfsetstrokeopacity{0.947672}%
\pgfsetdash{}{0pt}%
\pgfpathmoveto{\pgfqpoint{2.078028in}{1.882920in}}%
\pgfpathcurveto{\pgfqpoint{2.086265in}{1.882920in}}{\pgfqpoint{2.094165in}{1.886192in}}{\pgfqpoint{2.099989in}{1.892016in}}%
\pgfpathcurveto{\pgfqpoint{2.105813in}{1.897840in}}{\pgfqpoint{2.109085in}{1.905740in}}{\pgfqpoint{2.109085in}{1.913976in}}%
\pgfpathcurveto{\pgfqpoint{2.109085in}{1.922213in}}{\pgfqpoint{2.105813in}{1.930113in}}{\pgfqpoint{2.099989in}{1.935937in}}%
\pgfpathcurveto{\pgfqpoint{2.094165in}{1.941761in}}{\pgfqpoint{2.086265in}{1.945033in}}{\pgfqpoint{2.078028in}{1.945033in}}%
\pgfpathcurveto{\pgfqpoint{2.069792in}{1.945033in}}{\pgfqpoint{2.061892in}{1.941761in}}{\pgfqpoint{2.056068in}{1.935937in}}%
\pgfpathcurveto{\pgfqpoint{2.050244in}{1.930113in}}{\pgfqpoint{2.046972in}{1.922213in}}{\pgfqpoint{2.046972in}{1.913976in}}%
\pgfpathcurveto{\pgfqpoint{2.046972in}{1.905740in}}{\pgfqpoint{2.050244in}{1.897840in}}{\pgfqpoint{2.056068in}{1.892016in}}%
\pgfpathcurveto{\pgfqpoint{2.061892in}{1.886192in}}{\pgfqpoint{2.069792in}{1.882920in}}{\pgfqpoint{2.078028in}{1.882920in}}%
\pgfpathclose%
\pgfusepath{stroke,fill}%
\end{pgfscope}%
\begin{pgfscope}%
\pgfpathrectangle{\pgfqpoint{0.100000in}{0.212622in}}{\pgfqpoint{3.696000in}{3.696000in}}%
\pgfusepath{clip}%
\pgfsetbuttcap%
\pgfsetroundjoin%
\definecolor{currentfill}{rgb}{0.121569,0.466667,0.705882}%
\pgfsetfillcolor{currentfill}%
\pgfsetfillopacity{0.947808}%
\pgfsetlinewidth{1.003750pt}%
\definecolor{currentstroke}{rgb}{0.121569,0.466667,0.705882}%
\pgfsetstrokecolor{currentstroke}%
\pgfsetstrokeopacity{0.947808}%
\pgfsetdash{}{0pt}%
\pgfpathmoveto{\pgfqpoint{2.077704in}{1.882302in}}%
\pgfpathcurveto{\pgfqpoint{2.085940in}{1.882302in}}{\pgfqpoint{2.093840in}{1.885575in}}{\pgfqpoint{2.099664in}{1.891398in}}%
\pgfpathcurveto{\pgfqpoint{2.105488in}{1.897222in}}{\pgfqpoint{2.108760in}{1.905122in}}{\pgfqpoint{2.108760in}{1.913359in}}%
\pgfpathcurveto{\pgfqpoint{2.108760in}{1.921595in}}{\pgfqpoint{2.105488in}{1.929495in}}{\pgfqpoint{2.099664in}{1.935319in}}%
\pgfpathcurveto{\pgfqpoint{2.093840in}{1.941143in}}{\pgfqpoint{2.085940in}{1.944415in}}{\pgfqpoint{2.077704in}{1.944415in}}%
\pgfpathcurveto{\pgfqpoint{2.069467in}{1.944415in}}{\pgfqpoint{2.061567in}{1.941143in}}{\pgfqpoint{2.055743in}{1.935319in}}%
\pgfpathcurveto{\pgfqpoint{2.049920in}{1.929495in}}{\pgfqpoint{2.046647in}{1.921595in}}{\pgfqpoint{2.046647in}{1.913359in}}%
\pgfpathcurveto{\pgfqpoint{2.046647in}{1.905122in}}{\pgfqpoint{2.049920in}{1.897222in}}{\pgfqpoint{2.055743in}{1.891398in}}%
\pgfpathcurveto{\pgfqpoint{2.061567in}{1.885575in}}{\pgfqpoint{2.069467in}{1.882302in}}{\pgfqpoint{2.077704in}{1.882302in}}%
\pgfpathclose%
\pgfusepath{stroke,fill}%
\end{pgfscope}%
\begin{pgfscope}%
\pgfpathrectangle{\pgfqpoint{0.100000in}{0.212622in}}{\pgfqpoint{3.696000in}{3.696000in}}%
\pgfusepath{clip}%
\pgfsetbuttcap%
\pgfsetroundjoin%
\definecolor{currentfill}{rgb}{0.121569,0.466667,0.705882}%
\pgfsetfillcolor{currentfill}%
\pgfsetfillopacity{0.947888}%
\pgfsetlinewidth{1.003750pt}%
\definecolor{currentstroke}{rgb}{0.121569,0.466667,0.705882}%
\pgfsetstrokecolor{currentstroke}%
\pgfsetstrokeopacity{0.947888}%
\pgfsetdash{}{0pt}%
\pgfpathmoveto{\pgfqpoint{2.040231in}{1.863333in}}%
\pgfpathcurveto{\pgfqpoint{2.048467in}{1.863333in}}{\pgfqpoint{2.056367in}{1.866606in}}{\pgfqpoint{2.062191in}{1.872430in}}%
\pgfpathcurveto{\pgfqpoint{2.068015in}{1.878254in}}{\pgfqpoint{2.071287in}{1.886154in}}{\pgfqpoint{2.071287in}{1.894390in}}%
\pgfpathcurveto{\pgfqpoint{2.071287in}{1.902626in}}{\pgfqpoint{2.068015in}{1.910526in}}{\pgfqpoint{2.062191in}{1.916350in}}%
\pgfpathcurveto{\pgfqpoint{2.056367in}{1.922174in}}{\pgfqpoint{2.048467in}{1.925446in}}{\pgfqpoint{2.040231in}{1.925446in}}%
\pgfpathcurveto{\pgfqpoint{2.031995in}{1.925446in}}{\pgfqpoint{2.024095in}{1.922174in}}{\pgfqpoint{2.018271in}{1.916350in}}%
\pgfpathcurveto{\pgfqpoint{2.012447in}{1.910526in}}{\pgfqpoint{2.009174in}{1.902626in}}{\pgfqpoint{2.009174in}{1.894390in}}%
\pgfpathcurveto{\pgfqpoint{2.009174in}{1.886154in}}{\pgfqpoint{2.012447in}{1.878254in}}{\pgfqpoint{2.018271in}{1.872430in}}%
\pgfpathcurveto{\pgfqpoint{2.024095in}{1.866606in}}{\pgfqpoint{2.031995in}{1.863333in}}{\pgfqpoint{2.040231in}{1.863333in}}%
\pgfpathclose%
\pgfusepath{stroke,fill}%
\end{pgfscope}%
\begin{pgfscope}%
\pgfpathrectangle{\pgfqpoint{0.100000in}{0.212622in}}{\pgfqpoint{3.696000in}{3.696000in}}%
\pgfusepath{clip}%
\pgfsetbuttcap%
\pgfsetroundjoin%
\definecolor{currentfill}{rgb}{0.121569,0.466667,0.705882}%
\pgfsetfillcolor{currentfill}%
\pgfsetfillopacity{0.948061}%
\pgfsetlinewidth{1.003750pt}%
\definecolor{currentstroke}{rgb}{0.121569,0.466667,0.705882}%
\pgfsetstrokecolor{currentstroke}%
\pgfsetstrokeopacity{0.948061}%
\pgfsetdash{}{0pt}%
\pgfpathmoveto{\pgfqpoint{2.077125in}{1.881186in}}%
\pgfpathcurveto{\pgfqpoint{2.085362in}{1.881186in}}{\pgfqpoint{2.093262in}{1.884458in}}{\pgfqpoint{2.099086in}{1.890282in}}%
\pgfpathcurveto{\pgfqpoint{2.104910in}{1.896106in}}{\pgfqpoint{2.108182in}{1.904006in}}{\pgfqpoint{2.108182in}{1.912242in}}%
\pgfpathcurveto{\pgfqpoint{2.108182in}{1.920479in}}{\pgfqpoint{2.104910in}{1.928379in}}{\pgfqpoint{2.099086in}{1.934203in}}%
\pgfpathcurveto{\pgfqpoint{2.093262in}{1.940027in}}{\pgfqpoint{2.085362in}{1.943299in}}{\pgfqpoint{2.077125in}{1.943299in}}%
\pgfpathcurveto{\pgfqpoint{2.068889in}{1.943299in}}{\pgfqpoint{2.060989in}{1.940027in}}{\pgfqpoint{2.055165in}{1.934203in}}%
\pgfpathcurveto{\pgfqpoint{2.049341in}{1.928379in}}{\pgfqpoint{2.046069in}{1.920479in}}{\pgfqpoint{2.046069in}{1.912242in}}%
\pgfpathcurveto{\pgfqpoint{2.046069in}{1.904006in}}{\pgfqpoint{2.049341in}{1.896106in}}{\pgfqpoint{2.055165in}{1.890282in}}%
\pgfpathcurveto{\pgfqpoint{2.060989in}{1.884458in}}{\pgfqpoint{2.068889in}{1.881186in}}{\pgfqpoint{2.077125in}{1.881186in}}%
\pgfpathclose%
\pgfusepath{stroke,fill}%
\end{pgfscope}%
\begin{pgfscope}%
\pgfpathrectangle{\pgfqpoint{0.100000in}{0.212622in}}{\pgfqpoint{3.696000in}{3.696000in}}%
\pgfusepath{clip}%
\pgfsetbuttcap%
\pgfsetroundjoin%
\definecolor{currentfill}{rgb}{0.121569,0.466667,0.705882}%
\pgfsetfillcolor{currentfill}%
\pgfsetfillopacity{0.948094}%
\pgfsetlinewidth{1.003750pt}%
\definecolor{currentstroke}{rgb}{0.121569,0.466667,0.705882}%
\pgfsetstrokecolor{currentstroke}%
\pgfsetstrokeopacity{0.948094}%
\pgfsetdash{}{0pt}%
\pgfpathmoveto{\pgfqpoint{2.042374in}{1.862650in}}%
\pgfpathcurveto{\pgfqpoint{2.050610in}{1.862650in}}{\pgfqpoint{2.058510in}{1.865922in}}{\pgfqpoint{2.064334in}{1.871746in}}%
\pgfpathcurveto{\pgfqpoint{2.070158in}{1.877570in}}{\pgfqpoint{2.073431in}{1.885470in}}{\pgfqpoint{2.073431in}{1.893706in}}%
\pgfpathcurveto{\pgfqpoint{2.073431in}{1.901942in}}{\pgfqpoint{2.070158in}{1.909843in}}{\pgfqpoint{2.064334in}{1.915666in}}%
\pgfpathcurveto{\pgfqpoint{2.058510in}{1.921490in}}{\pgfqpoint{2.050610in}{1.924763in}}{\pgfqpoint{2.042374in}{1.924763in}}%
\pgfpathcurveto{\pgfqpoint{2.034138in}{1.924763in}}{\pgfqpoint{2.026238in}{1.921490in}}{\pgfqpoint{2.020414in}{1.915666in}}%
\pgfpathcurveto{\pgfqpoint{2.014590in}{1.909843in}}{\pgfqpoint{2.011318in}{1.901942in}}{\pgfqpoint{2.011318in}{1.893706in}}%
\pgfpathcurveto{\pgfqpoint{2.011318in}{1.885470in}}{\pgfqpoint{2.014590in}{1.877570in}}{\pgfqpoint{2.020414in}{1.871746in}}%
\pgfpathcurveto{\pgfqpoint{2.026238in}{1.865922in}}{\pgfqpoint{2.034138in}{1.862650in}}{\pgfqpoint{2.042374in}{1.862650in}}%
\pgfpathclose%
\pgfusepath{stroke,fill}%
\end{pgfscope}%
\begin{pgfscope}%
\pgfpathrectangle{\pgfqpoint{0.100000in}{0.212622in}}{\pgfqpoint{3.696000in}{3.696000in}}%
\pgfusepath{clip}%
\pgfsetbuttcap%
\pgfsetroundjoin%
\definecolor{currentfill}{rgb}{0.121569,0.466667,0.705882}%
\pgfsetfillcolor{currentfill}%
\pgfsetfillopacity{0.948253}%
\pgfsetlinewidth{1.003750pt}%
\definecolor{currentstroke}{rgb}{0.121569,0.466667,0.705882}%
\pgfsetstrokecolor{currentstroke}%
\pgfsetstrokeopacity{0.948253}%
\pgfsetdash{}{0pt}%
\pgfpathmoveto{\pgfqpoint{2.043455in}{1.862093in}}%
\pgfpathcurveto{\pgfqpoint{2.051691in}{1.862093in}}{\pgfqpoint{2.059591in}{1.865365in}}{\pgfqpoint{2.065415in}{1.871189in}}%
\pgfpathcurveto{\pgfqpoint{2.071239in}{1.877013in}}{\pgfqpoint{2.074512in}{1.884913in}}{\pgfqpoint{2.074512in}{1.893149in}}%
\pgfpathcurveto{\pgfqpoint{2.074512in}{1.901386in}}{\pgfqpoint{2.071239in}{1.909286in}}{\pgfqpoint{2.065415in}{1.915110in}}%
\pgfpathcurveto{\pgfqpoint{2.059591in}{1.920934in}}{\pgfqpoint{2.051691in}{1.924206in}}{\pgfqpoint{2.043455in}{1.924206in}}%
\pgfpathcurveto{\pgfqpoint{2.035219in}{1.924206in}}{\pgfqpoint{2.027319in}{1.920934in}}{\pgfqpoint{2.021495in}{1.915110in}}%
\pgfpathcurveto{\pgfqpoint{2.015671in}{1.909286in}}{\pgfqpoint{2.012399in}{1.901386in}}{\pgfqpoint{2.012399in}{1.893149in}}%
\pgfpathcurveto{\pgfqpoint{2.012399in}{1.884913in}}{\pgfqpoint{2.015671in}{1.877013in}}{\pgfqpoint{2.021495in}{1.871189in}}%
\pgfpathcurveto{\pgfqpoint{2.027319in}{1.865365in}}{\pgfqpoint{2.035219in}{1.862093in}}{\pgfqpoint{2.043455in}{1.862093in}}%
\pgfpathclose%
\pgfusepath{stroke,fill}%
\end{pgfscope}%
\begin{pgfscope}%
\pgfpathrectangle{\pgfqpoint{0.100000in}{0.212622in}}{\pgfqpoint{3.696000in}{3.696000in}}%
\pgfusepath{clip}%
\pgfsetbuttcap%
\pgfsetroundjoin%
\definecolor{currentfill}{rgb}{0.121569,0.466667,0.705882}%
\pgfsetfillcolor{currentfill}%
\pgfsetfillopacity{0.948516}%
\pgfsetlinewidth{1.003750pt}%
\definecolor{currentstroke}{rgb}{0.121569,0.466667,0.705882}%
\pgfsetstrokecolor{currentstroke}%
\pgfsetstrokeopacity{0.948516}%
\pgfsetdash{}{0pt}%
\pgfpathmoveto{\pgfqpoint{2.045081in}{1.861416in}}%
\pgfpathcurveto{\pgfqpoint{2.053317in}{1.861416in}}{\pgfqpoint{2.061217in}{1.864688in}}{\pgfqpoint{2.067041in}{1.870512in}}%
\pgfpathcurveto{\pgfqpoint{2.072865in}{1.876336in}}{\pgfqpoint{2.076137in}{1.884236in}}{\pgfqpoint{2.076137in}{1.892472in}}%
\pgfpathcurveto{\pgfqpoint{2.076137in}{1.900709in}}{\pgfqpoint{2.072865in}{1.908609in}}{\pgfqpoint{2.067041in}{1.914433in}}%
\pgfpathcurveto{\pgfqpoint{2.061217in}{1.920257in}}{\pgfqpoint{2.053317in}{1.923529in}}{\pgfqpoint{2.045081in}{1.923529in}}%
\pgfpathcurveto{\pgfqpoint{2.036844in}{1.923529in}}{\pgfqpoint{2.028944in}{1.920257in}}{\pgfqpoint{2.023120in}{1.914433in}}%
\pgfpathcurveto{\pgfqpoint{2.017296in}{1.908609in}}{\pgfqpoint{2.014024in}{1.900709in}}{\pgfqpoint{2.014024in}{1.892472in}}%
\pgfpathcurveto{\pgfqpoint{2.014024in}{1.884236in}}{\pgfqpoint{2.017296in}{1.876336in}}{\pgfqpoint{2.023120in}{1.870512in}}%
\pgfpathcurveto{\pgfqpoint{2.028944in}{1.864688in}}{\pgfqpoint{2.036844in}{1.861416in}}{\pgfqpoint{2.045081in}{1.861416in}}%
\pgfpathclose%
\pgfusepath{stroke,fill}%
\end{pgfscope}%
\begin{pgfscope}%
\pgfpathrectangle{\pgfqpoint{0.100000in}{0.212622in}}{\pgfqpoint{3.696000in}{3.696000in}}%
\pgfusepath{clip}%
\pgfsetbuttcap%
\pgfsetroundjoin%
\definecolor{currentfill}{rgb}{0.121569,0.466667,0.705882}%
\pgfsetfillcolor{currentfill}%
\pgfsetfillopacity{0.948544}%
\pgfsetlinewidth{1.003750pt}%
\definecolor{currentstroke}{rgb}{0.121569,0.466667,0.705882}%
\pgfsetstrokecolor{currentstroke}%
\pgfsetstrokeopacity{0.948544}%
\pgfsetdash{}{0pt}%
\pgfpathmoveto{\pgfqpoint{2.076075in}{1.879254in}}%
\pgfpathcurveto{\pgfqpoint{2.084311in}{1.879254in}}{\pgfqpoint{2.092211in}{1.882526in}}{\pgfqpoint{2.098035in}{1.888350in}}%
\pgfpathcurveto{\pgfqpoint{2.103859in}{1.894174in}}{\pgfqpoint{2.107131in}{1.902074in}}{\pgfqpoint{2.107131in}{1.910311in}}%
\pgfpathcurveto{\pgfqpoint{2.107131in}{1.918547in}}{\pgfqpoint{2.103859in}{1.926447in}}{\pgfqpoint{2.098035in}{1.932271in}}%
\pgfpathcurveto{\pgfqpoint{2.092211in}{1.938095in}}{\pgfqpoint{2.084311in}{1.941367in}}{\pgfqpoint{2.076075in}{1.941367in}}%
\pgfpathcurveto{\pgfqpoint{2.067839in}{1.941367in}}{\pgfqpoint{2.059939in}{1.938095in}}{\pgfqpoint{2.054115in}{1.932271in}}%
\pgfpathcurveto{\pgfqpoint{2.048291in}{1.926447in}}{\pgfqpoint{2.045018in}{1.918547in}}{\pgfqpoint{2.045018in}{1.910311in}}%
\pgfpathcurveto{\pgfqpoint{2.045018in}{1.902074in}}{\pgfqpoint{2.048291in}{1.894174in}}{\pgfqpoint{2.054115in}{1.888350in}}%
\pgfpathcurveto{\pgfqpoint{2.059939in}{1.882526in}}{\pgfqpoint{2.067839in}{1.879254in}}{\pgfqpoint{2.076075in}{1.879254in}}%
\pgfpathclose%
\pgfusepath{stroke,fill}%
\end{pgfscope}%
\begin{pgfscope}%
\pgfpathrectangle{\pgfqpoint{0.100000in}{0.212622in}}{\pgfqpoint{3.696000in}{3.696000in}}%
\pgfusepath{clip}%
\pgfsetbuttcap%
\pgfsetroundjoin%
\definecolor{currentfill}{rgb}{0.121569,0.466667,0.705882}%
\pgfsetfillcolor{currentfill}%
\pgfsetfillopacity{0.948826}%
\pgfsetlinewidth{1.003750pt}%
\definecolor{currentstroke}{rgb}{0.121569,0.466667,0.705882}%
\pgfsetstrokecolor{currentstroke}%
\pgfsetstrokeopacity{0.948826}%
\pgfsetdash{}{0pt}%
\pgfpathmoveto{\pgfqpoint{2.047559in}{1.860361in}}%
\pgfpathcurveto{\pgfqpoint{2.055795in}{1.860361in}}{\pgfqpoint{2.063695in}{1.863633in}}{\pgfqpoint{2.069519in}{1.869457in}}%
\pgfpathcurveto{\pgfqpoint{2.075343in}{1.875281in}}{\pgfqpoint{2.078615in}{1.883181in}}{\pgfqpoint{2.078615in}{1.891417in}}%
\pgfpathcurveto{\pgfqpoint{2.078615in}{1.899653in}}{\pgfqpoint{2.075343in}{1.907553in}}{\pgfqpoint{2.069519in}{1.913377in}}%
\pgfpathcurveto{\pgfqpoint{2.063695in}{1.919201in}}{\pgfqpoint{2.055795in}{1.922474in}}{\pgfqpoint{2.047559in}{1.922474in}}%
\pgfpathcurveto{\pgfqpoint{2.039322in}{1.922474in}}{\pgfqpoint{2.031422in}{1.919201in}}{\pgfqpoint{2.025599in}{1.913377in}}%
\pgfpathcurveto{\pgfqpoint{2.019775in}{1.907553in}}{\pgfqpoint{2.016502in}{1.899653in}}{\pgfqpoint{2.016502in}{1.891417in}}%
\pgfpathcurveto{\pgfqpoint{2.016502in}{1.883181in}}{\pgfqpoint{2.019775in}{1.875281in}}{\pgfqpoint{2.025599in}{1.869457in}}%
\pgfpathcurveto{\pgfqpoint{2.031422in}{1.863633in}}{\pgfqpoint{2.039322in}{1.860361in}}{\pgfqpoint{2.047559in}{1.860361in}}%
\pgfpathclose%
\pgfusepath{stroke,fill}%
\end{pgfscope}%
\begin{pgfscope}%
\pgfpathrectangle{\pgfqpoint{0.100000in}{0.212622in}}{\pgfqpoint{3.696000in}{3.696000in}}%
\pgfusepath{clip}%
\pgfsetbuttcap%
\pgfsetroundjoin%
\definecolor{currentfill}{rgb}{0.121569,0.466667,0.705882}%
\pgfsetfillcolor{currentfill}%
\pgfsetfillopacity{0.949025}%
\pgfsetlinewidth{1.003750pt}%
\definecolor{currentstroke}{rgb}{0.121569,0.466667,0.705882}%
\pgfsetstrokecolor{currentstroke}%
\pgfsetstrokeopacity{0.949025}%
\pgfsetdash{}{0pt}%
\pgfpathmoveto{\pgfqpoint{2.122119in}{0.968161in}}%
\pgfpathcurveto{\pgfqpoint{2.130355in}{0.968161in}}{\pgfqpoint{2.138255in}{0.971433in}}{\pgfqpoint{2.144079in}{0.977257in}}%
\pgfpathcurveto{\pgfqpoint{2.149903in}{0.983081in}}{\pgfqpoint{2.153175in}{0.990981in}}{\pgfqpoint{2.153175in}{0.999218in}}%
\pgfpathcurveto{\pgfqpoint{2.153175in}{1.007454in}}{\pgfqpoint{2.149903in}{1.015354in}}{\pgfqpoint{2.144079in}{1.021178in}}%
\pgfpathcurveto{\pgfqpoint{2.138255in}{1.027002in}}{\pgfqpoint{2.130355in}{1.030274in}}{\pgfqpoint{2.122119in}{1.030274in}}%
\pgfpathcurveto{\pgfqpoint{2.113882in}{1.030274in}}{\pgfqpoint{2.105982in}{1.027002in}}{\pgfqpoint{2.100158in}{1.021178in}}%
\pgfpathcurveto{\pgfqpoint{2.094334in}{1.015354in}}{\pgfqpoint{2.091062in}{1.007454in}}{\pgfqpoint{2.091062in}{0.999218in}}%
\pgfpathcurveto{\pgfqpoint{2.091062in}{0.990981in}}{\pgfqpoint{2.094334in}{0.983081in}}{\pgfqpoint{2.100158in}{0.977257in}}%
\pgfpathcurveto{\pgfqpoint{2.105982in}{0.971433in}}{\pgfqpoint{2.113882in}{0.968161in}}{\pgfqpoint{2.122119in}{0.968161in}}%
\pgfpathclose%
\pgfusepath{stroke,fill}%
\end{pgfscope}%
\begin{pgfscope}%
\pgfpathrectangle{\pgfqpoint{0.100000in}{0.212622in}}{\pgfqpoint{3.696000in}{3.696000in}}%
\pgfusepath{clip}%
\pgfsetbuttcap%
\pgfsetroundjoin%
\definecolor{currentfill}{rgb}{0.121569,0.466667,0.705882}%
\pgfsetfillcolor{currentfill}%
\pgfsetfillopacity{0.949230}%
\pgfsetlinewidth{1.003750pt}%
\definecolor{currentstroke}{rgb}{0.121569,0.466667,0.705882}%
\pgfsetstrokecolor{currentstroke}%
\pgfsetstrokeopacity{0.949230}%
\pgfsetdash{}{0pt}%
\pgfpathmoveto{\pgfqpoint{2.050357in}{1.859394in}}%
\pgfpathcurveto{\pgfqpoint{2.058593in}{1.859394in}}{\pgfqpoint{2.066493in}{1.862666in}}{\pgfqpoint{2.072317in}{1.868490in}}%
\pgfpathcurveto{\pgfqpoint{2.078141in}{1.874314in}}{\pgfqpoint{2.081414in}{1.882214in}}{\pgfqpoint{2.081414in}{1.890451in}}%
\pgfpathcurveto{\pgfqpoint{2.081414in}{1.898687in}}{\pgfqpoint{2.078141in}{1.906587in}}{\pgfqpoint{2.072317in}{1.912411in}}%
\pgfpathcurveto{\pgfqpoint{2.066493in}{1.918235in}}{\pgfqpoint{2.058593in}{1.921507in}}{\pgfqpoint{2.050357in}{1.921507in}}%
\pgfpathcurveto{\pgfqpoint{2.042121in}{1.921507in}}{\pgfqpoint{2.034221in}{1.918235in}}{\pgfqpoint{2.028397in}{1.912411in}}%
\pgfpathcurveto{\pgfqpoint{2.022573in}{1.906587in}}{\pgfqpoint{2.019301in}{1.898687in}}{\pgfqpoint{2.019301in}{1.890451in}}%
\pgfpathcurveto{\pgfqpoint{2.019301in}{1.882214in}}{\pgfqpoint{2.022573in}{1.874314in}}{\pgfqpoint{2.028397in}{1.868490in}}%
\pgfpathcurveto{\pgfqpoint{2.034221in}{1.862666in}}{\pgfqpoint{2.042121in}{1.859394in}}{\pgfqpoint{2.050357in}{1.859394in}}%
\pgfpathclose%
\pgfusepath{stroke,fill}%
\end{pgfscope}%
\begin{pgfscope}%
\pgfpathrectangle{\pgfqpoint{0.100000in}{0.212622in}}{\pgfqpoint{3.696000in}{3.696000in}}%
\pgfusepath{clip}%
\pgfsetbuttcap%
\pgfsetroundjoin%
\definecolor{currentfill}{rgb}{0.121569,0.466667,0.705882}%
\pgfsetfillcolor{currentfill}%
\pgfsetfillopacity{0.949253}%
\pgfsetlinewidth{1.003750pt}%
\definecolor{currentstroke}{rgb}{0.121569,0.466667,0.705882}%
\pgfsetstrokecolor{currentstroke}%
\pgfsetstrokeopacity{0.949253}%
\pgfsetdash{}{0pt}%
\pgfpathmoveto{\pgfqpoint{2.073822in}{1.875411in}}%
\pgfpathcurveto{\pgfqpoint{2.082058in}{1.875411in}}{\pgfqpoint{2.089958in}{1.878683in}}{\pgfqpoint{2.095782in}{1.884507in}}%
\pgfpathcurveto{\pgfqpoint{2.101606in}{1.890331in}}{\pgfqpoint{2.104878in}{1.898231in}}{\pgfqpoint{2.104878in}{1.906468in}}%
\pgfpathcurveto{\pgfqpoint{2.104878in}{1.914704in}}{\pgfqpoint{2.101606in}{1.922604in}}{\pgfqpoint{2.095782in}{1.928428in}}%
\pgfpathcurveto{\pgfqpoint{2.089958in}{1.934252in}}{\pgfqpoint{2.082058in}{1.937524in}}{\pgfqpoint{2.073822in}{1.937524in}}%
\pgfpathcurveto{\pgfqpoint{2.065585in}{1.937524in}}{\pgfqpoint{2.057685in}{1.934252in}}{\pgfqpoint{2.051861in}{1.928428in}}%
\pgfpathcurveto{\pgfqpoint{2.046037in}{1.922604in}}{\pgfqpoint{2.042765in}{1.914704in}}{\pgfqpoint{2.042765in}{1.906468in}}%
\pgfpathcurveto{\pgfqpoint{2.042765in}{1.898231in}}{\pgfqpoint{2.046037in}{1.890331in}}{\pgfqpoint{2.051861in}{1.884507in}}%
\pgfpathcurveto{\pgfqpoint{2.057685in}{1.878683in}}{\pgfqpoint{2.065585in}{1.875411in}}{\pgfqpoint{2.073822in}{1.875411in}}%
\pgfpathclose%
\pgfusepath{stroke,fill}%
\end{pgfscope}%
\begin{pgfscope}%
\pgfpathrectangle{\pgfqpoint{0.100000in}{0.212622in}}{\pgfqpoint{3.696000in}{3.696000in}}%
\pgfusepath{clip}%
\pgfsetbuttcap%
\pgfsetroundjoin%
\definecolor{currentfill}{rgb}{0.121569,0.466667,0.705882}%
\pgfsetfillcolor{currentfill}%
\pgfsetfillopacity{0.949465}%
\pgfsetlinewidth{1.003750pt}%
\definecolor{currentstroke}{rgb}{0.121569,0.466667,0.705882}%
\pgfsetstrokecolor{currentstroke}%
\pgfsetstrokeopacity{0.949465}%
\pgfsetdash{}{0pt}%
\pgfpathmoveto{\pgfqpoint{2.051919in}{1.859004in}}%
\pgfpathcurveto{\pgfqpoint{2.060155in}{1.859004in}}{\pgfqpoint{2.068055in}{1.862277in}}{\pgfqpoint{2.073879in}{1.868101in}}%
\pgfpathcurveto{\pgfqpoint{2.079703in}{1.873924in}}{\pgfqpoint{2.082976in}{1.881825in}}{\pgfqpoint{2.082976in}{1.890061in}}%
\pgfpathcurveto{\pgfqpoint{2.082976in}{1.898297in}}{\pgfqpoint{2.079703in}{1.906197in}}{\pgfqpoint{2.073879in}{1.912021in}}%
\pgfpathcurveto{\pgfqpoint{2.068055in}{1.917845in}}{\pgfqpoint{2.060155in}{1.921117in}}{\pgfqpoint{2.051919in}{1.921117in}}%
\pgfpathcurveto{\pgfqpoint{2.043683in}{1.921117in}}{\pgfqpoint{2.035783in}{1.917845in}}{\pgfqpoint{2.029959in}{1.912021in}}%
\pgfpathcurveto{\pgfqpoint{2.024135in}{1.906197in}}{\pgfqpoint{2.020863in}{1.898297in}}{\pgfqpoint{2.020863in}{1.890061in}}%
\pgfpathcurveto{\pgfqpoint{2.020863in}{1.881825in}}{\pgfqpoint{2.024135in}{1.873924in}}{\pgfqpoint{2.029959in}{1.868101in}}%
\pgfpathcurveto{\pgfqpoint{2.035783in}{1.862277in}}{\pgfqpoint{2.043683in}{1.859004in}}{\pgfqpoint{2.051919in}{1.859004in}}%
\pgfpathclose%
\pgfusepath{stroke,fill}%
\end{pgfscope}%
\begin{pgfscope}%
\pgfpathrectangle{\pgfqpoint{0.100000in}{0.212622in}}{\pgfqpoint{3.696000in}{3.696000in}}%
\pgfusepath{clip}%
\pgfsetbuttcap%
\pgfsetroundjoin%
\definecolor{currentfill}{rgb}{0.121569,0.466667,0.705882}%
\pgfsetfillcolor{currentfill}%
\pgfsetfillopacity{0.949705}%
\pgfsetlinewidth{1.003750pt}%
\definecolor{currentstroke}{rgb}{0.121569,0.466667,0.705882}%
\pgfsetstrokecolor{currentstroke}%
\pgfsetstrokeopacity{0.949705}%
\pgfsetdash{}{0pt}%
\pgfpathmoveto{\pgfqpoint{2.130719in}{0.962189in}}%
\pgfpathcurveto{\pgfqpoint{2.138955in}{0.962189in}}{\pgfqpoint{2.146855in}{0.965461in}}{\pgfqpoint{2.152679in}{0.971285in}}%
\pgfpathcurveto{\pgfqpoint{2.158503in}{0.977109in}}{\pgfqpoint{2.161775in}{0.985009in}}{\pgfqpoint{2.161775in}{0.993245in}}%
\pgfpathcurveto{\pgfqpoint{2.161775in}{1.001481in}}{\pgfqpoint{2.158503in}{1.009381in}}{\pgfqpoint{2.152679in}{1.015205in}}%
\pgfpathcurveto{\pgfqpoint{2.146855in}{1.021029in}}{\pgfqpoint{2.138955in}{1.024302in}}{\pgfqpoint{2.130719in}{1.024302in}}%
\pgfpathcurveto{\pgfqpoint{2.122482in}{1.024302in}}{\pgfqpoint{2.114582in}{1.021029in}}{\pgfqpoint{2.108758in}{1.015205in}}%
\pgfpathcurveto{\pgfqpoint{2.102935in}{1.009381in}}{\pgfqpoint{2.099662in}{1.001481in}}{\pgfqpoint{2.099662in}{0.993245in}}%
\pgfpathcurveto{\pgfqpoint{2.099662in}{0.985009in}}{\pgfqpoint{2.102935in}{0.977109in}}{\pgfqpoint{2.108758in}{0.971285in}}%
\pgfpathcurveto{\pgfqpoint{2.114582in}{0.965461in}}{\pgfqpoint{2.122482in}{0.962189in}}{\pgfqpoint{2.130719in}{0.962189in}}%
\pgfpathclose%
\pgfusepath{stroke,fill}%
\end{pgfscope}%
\begin{pgfscope}%
\pgfpathrectangle{\pgfqpoint{0.100000in}{0.212622in}}{\pgfqpoint{3.696000in}{3.696000in}}%
\pgfusepath{clip}%
\pgfsetbuttcap%
\pgfsetroundjoin%
\definecolor{currentfill}{rgb}{0.121569,0.466667,0.705882}%
\pgfsetfillcolor{currentfill}%
\pgfsetfillopacity{0.949733}%
\pgfsetlinewidth{1.003750pt}%
\definecolor{currentstroke}{rgb}{0.121569,0.466667,0.705882}%
\pgfsetstrokecolor{currentstroke}%
\pgfsetstrokeopacity{0.949733}%
\pgfsetdash{}{0pt}%
\pgfpathmoveto{\pgfqpoint{2.676063in}{1.089863in}}%
\pgfpathcurveto{\pgfqpoint{2.684300in}{1.089863in}}{\pgfqpoint{2.692200in}{1.093135in}}{\pgfqpoint{2.698024in}{1.098959in}}%
\pgfpathcurveto{\pgfqpoint{2.703848in}{1.104783in}}{\pgfqpoint{2.707120in}{1.112683in}}{\pgfqpoint{2.707120in}{1.120919in}}%
\pgfpathcurveto{\pgfqpoint{2.707120in}{1.129155in}}{\pgfqpoint{2.703848in}{1.137055in}}{\pgfqpoint{2.698024in}{1.142879in}}%
\pgfpathcurveto{\pgfqpoint{2.692200in}{1.148703in}}{\pgfqpoint{2.684300in}{1.151976in}}{\pgfqpoint{2.676063in}{1.151976in}}%
\pgfpathcurveto{\pgfqpoint{2.667827in}{1.151976in}}{\pgfqpoint{2.659927in}{1.148703in}}{\pgfqpoint{2.654103in}{1.142879in}}%
\pgfpathcurveto{\pgfqpoint{2.648279in}{1.137055in}}{\pgfqpoint{2.645007in}{1.129155in}}{\pgfqpoint{2.645007in}{1.120919in}}%
\pgfpathcurveto{\pgfqpoint{2.645007in}{1.112683in}}{\pgfqpoint{2.648279in}{1.104783in}}{\pgfqpoint{2.654103in}{1.098959in}}%
\pgfpathcurveto{\pgfqpoint{2.659927in}{1.093135in}}{\pgfqpoint{2.667827in}{1.089863in}}{\pgfqpoint{2.676063in}{1.089863in}}%
\pgfpathclose%
\pgfusepath{stroke,fill}%
\end{pgfscope}%
\begin{pgfscope}%
\pgfpathrectangle{\pgfqpoint{0.100000in}{0.212622in}}{\pgfqpoint{3.696000in}{3.696000in}}%
\pgfusepath{clip}%
\pgfsetbuttcap%
\pgfsetroundjoin%
\definecolor{currentfill}{rgb}{0.121569,0.466667,0.705882}%
\pgfsetfillcolor{currentfill}%
\pgfsetfillopacity{0.949839}%
\pgfsetlinewidth{1.003750pt}%
\definecolor{currentstroke}{rgb}{0.121569,0.466667,0.705882}%
\pgfsetstrokecolor{currentstroke}%
\pgfsetstrokeopacity{0.949839}%
\pgfsetdash{}{0pt}%
\pgfpathmoveto{\pgfqpoint{2.054506in}{1.858430in}}%
\pgfpathcurveto{\pgfqpoint{2.062742in}{1.858430in}}{\pgfqpoint{2.070642in}{1.861702in}}{\pgfqpoint{2.076466in}{1.867526in}}%
\pgfpathcurveto{\pgfqpoint{2.082290in}{1.873350in}}{\pgfqpoint{2.085562in}{1.881250in}}{\pgfqpoint{2.085562in}{1.889487in}}%
\pgfpathcurveto{\pgfqpoint{2.085562in}{1.897723in}}{\pgfqpoint{2.082290in}{1.905623in}}{\pgfqpoint{2.076466in}{1.911447in}}%
\pgfpathcurveto{\pgfqpoint{2.070642in}{1.917271in}}{\pgfqpoint{2.062742in}{1.920543in}}{\pgfqpoint{2.054506in}{1.920543in}}%
\pgfpathcurveto{\pgfqpoint{2.046270in}{1.920543in}}{\pgfqpoint{2.038370in}{1.917271in}}{\pgfqpoint{2.032546in}{1.911447in}}%
\pgfpathcurveto{\pgfqpoint{2.026722in}{1.905623in}}{\pgfqpoint{2.023449in}{1.897723in}}{\pgfqpoint{2.023449in}{1.889487in}}%
\pgfpathcurveto{\pgfqpoint{2.023449in}{1.881250in}}{\pgfqpoint{2.026722in}{1.873350in}}{\pgfqpoint{2.032546in}{1.867526in}}%
\pgfpathcurveto{\pgfqpoint{2.038370in}{1.861702in}}{\pgfqpoint{2.046270in}{1.858430in}}{\pgfqpoint{2.054506in}{1.858430in}}%
\pgfpathclose%
\pgfusepath{stroke,fill}%
\end{pgfscope}%
\begin{pgfscope}%
\pgfpathrectangle{\pgfqpoint{0.100000in}{0.212622in}}{\pgfqpoint{3.696000in}{3.696000in}}%
\pgfusepath{clip}%
\pgfsetbuttcap%
\pgfsetroundjoin%
\definecolor{currentfill}{rgb}{0.121569,0.466667,0.705882}%
\pgfsetfillcolor{currentfill}%
\pgfsetfillopacity{0.950010}%
\pgfsetlinewidth{1.003750pt}%
\definecolor{currentstroke}{rgb}{0.121569,0.466667,0.705882}%
\pgfsetstrokecolor{currentstroke}%
\pgfsetstrokeopacity{0.950010}%
\pgfsetdash{}{0pt}%
\pgfpathmoveto{\pgfqpoint{2.072264in}{1.872165in}}%
\pgfpathcurveto{\pgfqpoint{2.080500in}{1.872165in}}{\pgfqpoint{2.088400in}{1.875437in}}{\pgfqpoint{2.094224in}{1.881261in}}%
\pgfpathcurveto{\pgfqpoint{2.100048in}{1.887085in}}{\pgfqpoint{2.103320in}{1.894985in}}{\pgfqpoint{2.103320in}{1.903222in}}%
\pgfpathcurveto{\pgfqpoint{2.103320in}{1.911458in}}{\pgfqpoint{2.100048in}{1.919358in}}{\pgfqpoint{2.094224in}{1.925182in}}%
\pgfpathcurveto{\pgfqpoint{2.088400in}{1.931006in}}{\pgfqpoint{2.080500in}{1.934278in}}{\pgfqpoint{2.072264in}{1.934278in}}%
\pgfpathcurveto{\pgfqpoint{2.064027in}{1.934278in}}{\pgfqpoint{2.056127in}{1.931006in}}{\pgfqpoint{2.050303in}{1.925182in}}%
\pgfpathcurveto{\pgfqpoint{2.044480in}{1.919358in}}{\pgfqpoint{2.041207in}{1.911458in}}{\pgfqpoint{2.041207in}{1.903222in}}%
\pgfpathcurveto{\pgfqpoint{2.041207in}{1.894985in}}{\pgfqpoint{2.044480in}{1.887085in}}{\pgfqpoint{2.050303in}{1.881261in}}%
\pgfpathcurveto{\pgfqpoint{2.056127in}{1.875437in}}{\pgfqpoint{2.064027in}{1.872165in}}{\pgfqpoint{2.072264in}{1.872165in}}%
\pgfpathclose%
\pgfusepath{stroke,fill}%
\end{pgfscope}%
\begin{pgfscope}%
\pgfpathrectangle{\pgfqpoint{0.100000in}{0.212622in}}{\pgfqpoint{3.696000in}{3.696000in}}%
\pgfusepath{clip}%
\pgfsetbuttcap%
\pgfsetroundjoin%
\definecolor{currentfill}{rgb}{0.121569,0.466667,0.705882}%
\pgfsetfillcolor{currentfill}%
\pgfsetfillopacity{0.950044}%
\pgfsetlinewidth{1.003750pt}%
\definecolor{currentstroke}{rgb}{0.121569,0.466667,0.705882}%
\pgfsetstrokecolor{currentstroke}%
\pgfsetstrokeopacity{0.950044}%
\pgfsetdash{}{0pt}%
\pgfpathmoveto{\pgfqpoint{2.056011in}{1.858471in}}%
\pgfpathcurveto{\pgfqpoint{2.064247in}{1.858471in}}{\pgfqpoint{2.072147in}{1.861743in}}{\pgfqpoint{2.077971in}{1.867567in}}%
\pgfpathcurveto{\pgfqpoint{2.083795in}{1.873391in}}{\pgfqpoint{2.087067in}{1.881291in}}{\pgfqpoint{2.087067in}{1.889528in}}%
\pgfpathcurveto{\pgfqpoint{2.087067in}{1.897764in}}{\pgfqpoint{2.083795in}{1.905664in}}{\pgfqpoint{2.077971in}{1.911488in}}%
\pgfpathcurveto{\pgfqpoint{2.072147in}{1.917312in}}{\pgfqpoint{2.064247in}{1.920584in}}{\pgfqpoint{2.056011in}{1.920584in}}%
\pgfpathcurveto{\pgfqpoint{2.047774in}{1.920584in}}{\pgfqpoint{2.039874in}{1.917312in}}{\pgfqpoint{2.034050in}{1.911488in}}%
\pgfpathcurveto{\pgfqpoint{2.028227in}{1.905664in}}{\pgfqpoint{2.024954in}{1.897764in}}{\pgfqpoint{2.024954in}{1.889528in}}%
\pgfpathcurveto{\pgfqpoint{2.024954in}{1.881291in}}{\pgfqpoint{2.028227in}{1.873391in}}{\pgfqpoint{2.034050in}{1.867567in}}%
\pgfpathcurveto{\pgfqpoint{2.039874in}{1.861743in}}{\pgfqpoint{2.047774in}{1.858471in}}{\pgfqpoint{2.056011in}{1.858471in}}%
\pgfpathclose%
\pgfusepath{stroke,fill}%
\end{pgfscope}%
\begin{pgfscope}%
\pgfpathrectangle{\pgfqpoint{0.100000in}{0.212622in}}{\pgfqpoint{3.696000in}{3.696000in}}%
\pgfusepath{clip}%
\pgfsetbuttcap%
\pgfsetroundjoin%
\definecolor{currentfill}{rgb}{0.121569,0.466667,0.705882}%
\pgfsetfillcolor{currentfill}%
\pgfsetfillopacity{0.950127}%
\pgfsetlinewidth{1.003750pt}%
\definecolor{currentstroke}{rgb}{0.121569,0.466667,0.705882}%
\pgfsetstrokecolor{currentstroke}%
\pgfsetstrokeopacity{0.950127}%
\pgfsetdash{}{0pt}%
\pgfpathmoveto{\pgfqpoint{2.134962in}{0.957496in}}%
\pgfpathcurveto{\pgfqpoint{2.143198in}{0.957496in}}{\pgfqpoint{2.151098in}{0.960768in}}{\pgfqpoint{2.156922in}{0.966592in}}%
\pgfpathcurveto{\pgfqpoint{2.162746in}{0.972416in}}{\pgfqpoint{2.166018in}{0.980316in}}{\pgfqpoint{2.166018in}{0.988552in}}%
\pgfpathcurveto{\pgfqpoint{2.166018in}{0.996789in}}{\pgfqpoint{2.162746in}{1.004689in}}{\pgfqpoint{2.156922in}{1.010513in}}%
\pgfpathcurveto{\pgfqpoint{2.151098in}{1.016336in}}{\pgfqpoint{2.143198in}{1.019609in}}{\pgfqpoint{2.134962in}{1.019609in}}%
\pgfpathcurveto{\pgfqpoint{2.126725in}{1.019609in}}{\pgfqpoint{2.118825in}{1.016336in}}{\pgfqpoint{2.113001in}{1.010513in}}%
\pgfpathcurveto{\pgfqpoint{2.107178in}{1.004689in}}{\pgfqpoint{2.103905in}{0.996789in}}{\pgfqpoint{2.103905in}{0.988552in}}%
\pgfpathcurveto{\pgfqpoint{2.103905in}{0.980316in}}{\pgfqpoint{2.107178in}{0.972416in}}{\pgfqpoint{2.113001in}{0.966592in}}%
\pgfpathcurveto{\pgfqpoint{2.118825in}{0.960768in}}{\pgfqpoint{2.126725in}{0.957496in}}{\pgfqpoint{2.134962in}{0.957496in}}%
\pgfpathclose%
\pgfusepath{stroke,fill}%
\end{pgfscope}%
\begin{pgfscope}%
\pgfpathrectangle{\pgfqpoint{0.100000in}{0.212622in}}{\pgfqpoint{3.696000in}{3.696000in}}%
\pgfusepath{clip}%
\pgfsetbuttcap%
\pgfsetroundjoin%
\definecolor{currentfill}{rgb}{0.121569,0.466667,0.705882}%
\pgfsetfillcolor{currentfill}%
\pgfsetfillopacity{0.950248}%
\pgfsetlinewidth{1.003750pt}%
\definecolor{currentstroke}{rgb}{0.121569,0.466667,0.705882}%
\pgfsetstrokecolor{currentstroke}%
\pgfsetstrokeopacity{0.950248}%
\pgfsetdash{}{0pt}%
\pgfpathmoveto{\pgfqpoint{2.057917in}{1.858786in}}%
\pgfpathcurveto{\pgfqpoint{2.066153in}{1.858786in}}{\pgfqpoint{2.074053in}{1.862059in}}{\pgfqpoint{2.079877in}{1.867883in}}%
\pgfpathcurveto{\pgfqpoint{2.085701in}{1.873706in}}{\pgfqpoint{2.088973in}{1.881607in}}{\pgfqpoint{2.088973in}{1.889843in}}%
\pgfpathcurveto{\pgfqpoint{2.088973in}{1.898079in}}{\pgfqpoint{2.085701in}{1.905979in}}{\pgfqpoint{2.079877in}{1.911803in}}%
\pgfpathcurveto{\pgfqpoint{2.074053in}{1.917627in}}{\pgfqpoint{2.066153in}{1.920899in}}{\pgfqpoint{2.057917in}{1.920899in}}%
\pgfpathcurveto{\pgfqpoint{2.049681in}{1.920899in}}{\pgfqpoint{2.041781in}{1.917627in}}{\pgfqpoint{2.035957in}{1.911803in}}%
\pgfpathcurveto{\pgfqpoint{2.030133in}{1.905979in}}{\pgfqpoint{2.026860in}{1.898079in}}{\pgfqpoint{2.026860in}{1.889843in}}%
\pgfpathcurveto{\pgfqpoint{2.026860in}{1.881607in}}{\pgfqpoint{2.030133in}{1.873706in}}{\pgfqpoint{2.035957in}{1.867883in}}%
\pgfpathcurveto{\pgfqpoint{2.041781in}{1.862059in}}{\pgfqpoint{2.049681in}{1.858786in}}{\pgfqpoint{2.057917in}{1.858786in}}%
\pgfpathclose%
\pgfusepath{stroke,fill}%
\end{pgfscope}%
\begin{pgfscope}%
\pgfpathrectangle{\pgfqpoint{0.100000in}{0.212622in}}{\pgfqpoint{3.696000in}{3.696000in}}%
\pgfusepath{clip}%
\pgfsetbuttcap%
\pgfsetroundjoin%
\definecolor{currentfill}{rgb}{0.121569,0.466667,0.705882}%
\pgfsetfillcolor{currentfill}%
\pgfsetfillopacity{0.950528}%
\pgfsetlinewidth{1.003750pt}%
\definecolor{currentstroke}{rgb}{0.121569,0.466667,0.705882}%
\pgfsetstrokecolor{currentstroke}%
\pgfsetstrokeopacity{0.950528}%
\pgfsetdash{}{0pt}%
\pgfpathmoveto{\pgfqpoint{2.060533in}{1.860069in}}%
\pgfpathcurveto{\pgfqpoint{2.068769in}{1.860069in}}{\pgfqpoint{2.076669in}{1.863341in}}{\pgfqpoint{2.082493in}{1.869165in}}%
\pgfpathcurveto{\pgfqpoint{2.088317in}{1.874989in}}{\pgfqpoint{2.091590in}{1.882889in}}{\pgfqpoint{2.091590in}{1.891125in}}%
\pgfpathcurveto{\pgfqpoint{2.091590in}{1.899361in}}{\pgfqpoint{2.088317in}{1.907261in}}{\pgfqpoint{2.082493in}{1.913085in}}%
\pgfpathcurveto{\pgfqpoint{2.076669in}{1.918909in}}{\pgfqpoint{2.068769in}{1.922182in}}{\pgfqpoint{2.060533in}{1.922182in}}%
\pgfpathcurveto{\pgfqpoint{2.052297in}{1.922182in}}{\pgfqpoint{2.044397in}{1.918909in}}{\pgfqpoint{2.038573in}{1.913085in}}%
\pgfpathcurveto{\pgfqpoint{2.032749in}{1.907261in}}{\pgfqpoint{2.029477in}{1.899361in}}{\pgfqpoint{2.029477in}{1.891125in}}%
\pgfpathcurveto{\pgfqpoint{2.029477in}{1.882889in}}{\pgfqpoint{2.032749in}{1.874989in}}{\pgfqpoint{2.038573in}{1.869165in}}%
\pgfpathcurveto{\pgfqpoint{2.044397in}{1.863341in}}{\pgfqpoint{2.052297in}{1.860069in}}{\pgfqpoint{2.060533in}{1.860069in}}%
\pgfpathclose%
\pgfusepath{stroke,fill}%
\end{pgfscope}%
\begin{pgfscope}%
\pgfpathrectangle{\pgfqpoint{0.100000in}{0.212622in}}{\pgfqpoint{3.696000in}{3.696000in}}%
\pgfusepath{clip}%
\pgfsetbuttcap%
\pgfsetroundjoin%
\definecolor{currentfill}{rgb}{0.121569,0.466667,0.705882}%
\pgfsetfillcolor{currentfill}%
\pgfsetfillopacity{0.950568}%
\pgfsetlinewidth{1.003750pt}%
\definecolor{currentstroke}{rgb}{0.121569,0.466667,0.705882}%
\pgfsetstrokecolor{currentstroke}%
\pgfsetstrokeopacity{0.950568}%
\pgfsetdash{}{0pt}%
\pgfpathmoveto{\pgfqpoint{2.070626in}{1.870090in}}%
\pgfpathcurveto{\pgfqpoint{2.078862in}{1.870090in}}{\pgfqpoint{2.086762in}{1.873362in}}{\pgfqpoint{2.092586in}{1.879186in}}%
\pgfpathcurveto{\pgfqpoint{2.098410in}{1.885010in}}{\pgfqpoint{2.101682in}{1.892910in}}{\pgfqpoint{2.101682in}{1.901146in}}%
\pgfpathcurveto{\pgfqpoint{2.101682in}{1.909382in}}{\pgfqpoint{2.098410in}{1.917282in}}{\pgfqpoint{2.092586in}{1.923106in}}%
\pgfpathcurveto{\pgfqpoint{2.086762in}{1.928930in}}{\pgfqpoint{2.078862in}{1.932203in}}{\pgfqpoint{2.070626in}{1.932203in}}%
\pgfpathcurveto{\pgfqpoint{2.062389in}{1.932203in}}{\pgfqpoint{2.054489in}{1.928930in}}{\pgfqpoint{2.048665in}{1.923106in}}%
\pgfpathcurveto{\pgfqpoint{2.042841in}{1.917282in}}{\pgfqpoint{2.039569in}{1.909382in}}{\pgfqpoint{2.039569in}{1.901146in}}%
\pgfpathcurveto{\pgfqpoint{2.039569in}{1.892910in}}{\pgfqpoint{2.042841in}{1.885010in}}{\pgfqpoint{2.048665in}{1.879186in}}%
\pgfpathcurveto{\pgfqpoint{2.054489in}{1.873362in}}{\pgfqpoint{2.062389in}{1.870090in}}{\pgfqpoint{2.070626in}{1.870090in}}%
\pgfpathclose%
\pgfusepath{stroke,fill}%
\end{pgfscope}%
\begin{pgfscope}%
\pgfpathrectangle{\pgfqpoint{0.100000in}{0.212622in}}{\pgfqpoint{3.696000in}{3.696000in}}%
\pgfusepath{clip}%
\pgfsetbuttcap%
\pgfsetroundjoin%
\definecolor{currentfill}{rgb}{0.121569,0.466667,0.705882}%
\pgfsetfillcolor{currentfill}%
\pgfsetfillopacity{0.950684}%
\pgfsetlinewidth{1.003750pt}%
\definecolor{currentstroke}{rgb}{0.121569,0.466667,0.705882}%
\pgfsetstrokecolor{currentstroke}%
\pgfsetstrokeopacity{0.950684}%
\pgfsetdash{}{0pt}%
\pgfpathmoveto{\pgfqpoint{2.061996in}{1.861018in}}%
\pgfpathcurveto{\pgfqpoint{2.070232in}{1.861018in}}{\pgfqpoint{2.078132in}{1.864290in}}{\pgfqpoint{2.083956in}{1.870114in}}%
\pgfpathcurveto{\pgfqpoint{2.089780in}{1.875938in}}{\pgfqpoint{2.093052in}{1.883838in}}{\pgfqpoint{2.093052in}{1.892075in}}%
\pgfpathcurveto{\pgfqpoint{2.093052in}{1.900311in}}{\pgfqpoint{2.089780in}{1.908211in}}{\pgfqpoint{2.083956in}{1.914035in}}%
\pgfpathcurveto{\pgfqpoint{2.078132in}{1.919859in}}{\pgfqpoint{2.070232in}{1.923131in}}{\pgfqpoint{2.061996in}{1.923131in}}%
\pgfpathcurveto{\pgfqpoint{2.053759in}{1.923131in}}{\pgfqpoint{2.045859in}{1.919859in}}{\pgfqpoint{2.040036in}{1.914035in}}%
\pgfpathcurveto{\pgfqpoint{2.034212in}{1.908211in}}{\pgfqpoint{2.030939in}{1.900311in}}{\pgfqpoint{2.030939in}{1.892075in}}%
\pgfpathcurveto{\pgfqpoint{2.030939in}{1.883838in}}{\pgfqpoint{2.034212in}{1.875938in}}{\pgfqpoint{2.040036in}{1.870114in}}%
\pgfpathcurveto{\pgfqpoint{2.045859in}{1.864290in}}{\pgfqpoint{2.053759in}{1.861018in}}{\pgfqpoint{2.061996in}{1.861018in}}%
\pgfpathclose%
\pgfusepath{stroke,fill}%
\end{pgfscope}%
\begin{pgfscope}%
\pgfpathrectangle{\pgfqpoint{0.100000in}{0.212622in}}{\pgfqpoint{3.696000in}{3.696000in}}%
\pgfusepath{clip}%
\pgfsetbuttcap%
\pgfsetroundjoin%
\definecolor{currentfill}{rgb}{0.121569,0.466667,0.705882}%
\pgfsetfillcolor{currentfill}%
\pgfsetfillopacity{0.950855}%
\pgfsetlinewidth{1.003750pt}%
\definecolor{currentstroke}{rgb}{0.121569,0.466667,0.705882}%
\pgfsetstrokecolor{currentstroke}%
\pgfsetstrokeopacity{0.950855}%
\pgfsetdash{}{0pt}%
\pgfpathmoveto{\pgfqpoint{2.671716in}{1.084428in}}%
\pgfpathcurveto{\pgfqpoint{2.679952in}{1.084428in}}{\pgfqpoint{2.687852in}{1.087700in}}{\pgfqpoint{2.693676in}{1.093524in}}%
\pgfpathcurveto{\pgfqpoint{2.699500in}{1.099348in}}{\pgfqpoint{2.702772in}{1.107248in}}{\pgfqpoint{2.702772in}{1.115485in}}%
\pgfpathcurveto{\pgfqpoint{2.702772in}{1.123721in}}{\pgfqpoint{2.699500in}{1.131621in}}{\pgfqpoint{2.693676in}{1.137445in}}%
\pgfpathcurveto{\pgfqpoint{2.687852in}{1.143269in}}{\pgfqpoint{2.679952in}{1.146541in}}{\pgfqpoint{2.671716in}{1.146541in}}%
\pgfpathcurveto{\pgfqpoint{2.663480in}{1.146541in}}{\pgfqpoint{2.655580in}{1.143269in}}{\pgfqpoint{2.649756in}{1.137445in}}%
\pgfpathcurveto{\pgfqpoint{2.643932in}{1.131621in}}{\pgfqpoint{2.640659in}{1.123721in}}{\pgfqpoint{2.640659in}{1.115485in}}%
\pgfpathcurveto{\pgfqpoint{2.640659in}{1.107248in}}{\pgfqpoint{2.643932in}{1.099348in}}{\pgfqpoint{2.649756in}{1.093524in}}%
\pgfpathcurveto{\pgfqpoint{2.655580in}{1.087700in}}{\pgfqpoint{2.663480in}{1.084428in}}{\pgfqpoint{2.671716in}{1.084428in}}%
\pgfpathclose%
\pgfusepath{stroke,fill}%
\end{pgfscope}%
\begin{pgfscope}%
\pgfpathrectangle{\pgfqpoint{0.100000in}{0.212622in}}{\pgfqpoint{3.696000in}{3.696000in}}%
\pgfusepath{clip}%
\pgfsetbuttcap%
\pgfsetroundjoin%
\definecolor{currentfill}{rgb}{0.121569,0.466667,0.705882}%
\pgfsetfillcolor{currentfill}%
\pgfsetfillopacity{0.950858}%
\pgfsetlinewidth{1.003750pt}%
\definecolor{currentstroke}{rgb}{0.121569,0.466667,0.705882}%
\pgfsetstrokecolor{currentstroke}%
\pgfsetstrokeopacity{0.950858}%
\pgfsetdash{}{0pt}%
\pgfpathmoveto{\pgfqpoint{2.069330in}{1.868121in}}%
\pgfpathcurveto{\pgfqpoint{2.077567in}{1.868121in}}{\pgfqpoint{2.085467in}{1.871393in}}{\pgfqpoint{2.091291in}{1.877217in}}%
\pgfpathcurveto{\pgfqpoint{2.097115in}{1.883041in}}{\pgfqpoint{2.100387in}{1.890941in}}{\pgfqpoint{2.100387in}{1.899177in}}%
\pgfpathcurveto{\pgfqpoint{2.100387in}{1.907413in}}{\pgfqpoint{2.097115in}{1.915313in}}{\pgfqpoint{2.091291in}{1.921137in}}%
\pgfpathcurveto{\pgfqpoint{2.085467in}{1.926961in}}{\pgfqpoint{2.077567in}{1.930234in}}{\pgfqpoint{2.069330in}{1.930234in}}%
\pgfpathcurveto{\pgfqpoint{2.061094in}{1.930234in}}{\pgfqpoint{2.053194in}{1.926961in}}{\pgfqpoint{2.047370in}{1.921137in}}%
\pgfpathcurveto{\pgfqpoint{2.041546in}{1.915313in}}{\pgfqpoint{2.038274in}{1.907413in}}{\pgfqpoint{2.038274in}{1.899177in}}%
\pgfpathcurveto{\pgfqpoint{2.038274in}{1.890941in}}{\pgfqpoint{2.041546in}{1.883041in}}{\pgfqpoint{2.047370in}{1.877217in}}%
\pgfpathcurveto{\pgfqpoint{2.053194in}{1.871393in}}{\pgfqpoint{2.061094in}{1.868121in}}{\pgfqpoint{2.069330in}{1.868121in}}%
\pgfpathclose%
\pgfusepath{stroke,fill}%
\end{pgfscope}%
\begin{pgfscope}%
\pgfpathrectangle{\pgfqpoint{0.100000in}{0.212622in}}{\pgfqpoint{3.696000in}{3.696000in}}%
\pgfusepath{clip}%
\pgfsetbuttcap%
\pgfsetroundjoin%
\definecolor{currentfill}{rgb}{0.121569,0.466667,0.705882}%
\pgfsetfillcolor{currentfill}%
\pgfsetfillopacity{0.951005}%
\pgfsetlinewidth{1.003750pt}%
\definecolor{currentstroke}{rgb}{0.121569,0.466667,0.705882}%
\pgfsetstrokecolor{currentstroke}%
\pgfsetstrokeopacity{0.951005}%
\pgfsetdash{}{0pt}%
\pgfpathmoveto{\pgfqpoint{2.063896in}{1.862242in}}%
\pgfpathcurveto{\pgfqpoint{2.072133in}{1.862242in}}{\pgfqpoint{2.080033in}{1.865514in}}{\pgfqpoint{2.085857in}{1.871338in}}%
\pgfpathcurveto{\pgfqpoint{2.091681in}{1.877162in}}{\pgfqpoint{2.094953in}{1.885062in}}{\pgfqpoint{2.094953in}{1.893298in}}%
\pgfpathcurveto{\pgfqpoint{2.094953in}{1.901535in}}{\pgfqpoint{2.091681in}{1.909435in}}{\pgfqpoint{2.085857in}{1.915259in}}%
\pgfpathcurveto{\pgfqpoint{2.080033in}{1.921083in}}{\pgfqpoint{2.072133in}{1.924355in}}{\pgfqpoint{2.063896in}{1.924355in}}%
\pgfpathcurveto{\pgfqpoint{2.055660in}{1.924355in}}{\pgfqpoint{2.047760in}{1.921083in}}{\pgfqpoint{2.041936in}{1.915259in}}%
\pgfpathcurveto{\pgfqpoint{2.036112in}{1.909435in}}{\pgfqpoint{2.032840in}{1.901535in}}{\pgfqpoint{2.032840in}{1.893298in}}%
\pgfpathcurveto{\pgfqpoint{2.032840in}{1.885062in}}{\pgfqpoint{2.036112in}{1.877162in}}{\pgfqpoint{2.041936in}{1.871338in}}%
\pgfpathcurveto{\pgfqpoint{2.047760in}{1.865514in}}{\pgfqpoint{2.055660in}{1.862242in}}{\pgfqpoint{2.063896in}{1.862242in}}%
\pgfpathclose%
\pgfusepath{stroke,fill}%
\end{pgfscope}%
\begin{pgfscope}%
\pgfpathrectangle{\pgfqpoint{0.100000in}{0.212622in}}{\pgfqpoint{3.696000in}{3.696000in}}%
\pgfusepath{clip}%
\pgfsetbuttcap%
\pgfsetroundjoin%
\definecolor{currentfill}{rgb}{0.121569,0.466667,0.705882}%
\pgfsetfillcolor{currentfill}%
\pgfsetfillopacity{0.951009}%
\pgfsetlinewidth{1.003750pt}%
\definecolor{currentstroke}{rgb}{0.121569,0.466667,0.705882}%
\pgfsetstrokecolor{currentstroke}%
\pgfsetstrokeopacity{0.951009}%
\pgfsetdash{}{0pt}%
\pgfpathmoveto{\pgfqpoint{2.139600in}{0.953736in}}%
\pgfpathcurveto{\pgfqpoint{2.147837in}{0.953736in}}{\pgfqpoint{2.155737in}{0.957009in}}{\pgfqpoint{2.161561in}{0.962833in}}%
\pgfpathcurveto{\pgfqpoint{2.167385in}{0.968657in}}{\pgfqpoint{2.170657in}{0.976557in}}{\pgfqpoint{2.170657in}{0.984793in}}%
\pgfpathcurveto{\pgfqpoint{2.170657in}{0.993029in}}{\pgfqpoint{2.167385in}{1.000929in}}{\pgfqpoint{2.161561in}{1.006753in}}%
\pgfpathcurveto{\pgfqpoint{2.155737in}{1.012577in}}{\pgfqpoint{2.147837in}{1.015849in}}{\pgfqpoint{2.139600in}{1.015849in}}%
\pgfpathcurveto{\pgfqpoint{2.131364in}{1.015849in}}{\pgfqpoint{2.123464in}{1.012577in}}{\pgfqpoint{2.117640in}{1.006753in}}%
\pgfpathcurveto{\pgfqpoint{2.111816in}{1.000929in}}{\pgfqpoint{2.108544in}{0.993029in}}{\pgfqpoint{2.108544in}{0.984793in}}%
\pgfpathcurveto{\pgfqpoint{2.108544in}{0.976557in}}{\pgfqpoint{2.111816in}{0.968657in}}{\pgfqpoint{2.117640in}{0.962833in}}%
\pgfpathcurveto{\pgfqpoint{2.123464in}{0.957009in}}{\pgfqpoint{2.131364in}{0.953736in}}{\pgfqpoint{2.139600in}{0.953736in}}%
\pgfpathclose%
\pgfusepath{stroke,fill}%
\end{pgfscope}%
\begin{pgfscope}%
\pgfpathrectangle{\pgfqpoint{0.100000in}{0.212622in}}{\pgfqpoint{3.696000in}{3.696000in}}%
\pgfusepath{clip}%
\pgfsetbuttcap%
\pgfsetroundjoin%
\definecolor{currentfill}{rgb}{0.121569,0.466667,0.705882}%
\pgfsetfillcolor{currentfill}%
\pgfsetfillopacity{0.951127}%
\pgfsetlinewidth{1.003750pt}%
\definecolor{currentstroke}{rgb}{0.121569,0.466667,0.705882}%
\pgfsetstrokecolor{currentstroke}%
\pgfsetstrokeopacity{0.951127}%
\pgfsetdash{}{0pt}%
\pgfpathmoveto{\pgfqpoint{2.066559in}{1.864649in}}%
\pgfpathcurveto{\pgfqpoint{2.074795in}{1.864649in}}{\pgfqpoint{2.082695in}{1.867921in}}{\pgfqpoint{2.088519in}{1.873745in}}%
\pgfpathcurveto{\pgfqpoint{2.094343in}{1.879569in}}{\pgfqpoint{2.097615in}{1.887469in}}{\pgfqpoint{2.097615in}{1.895706in}}%
\pgfpathcurveto{\pgfqpoint{2.097615in}{1.903942in}}{\pgfqpoint{2.094343in}{1.911842in}}{\pgfqpoint{2.088519in}{1.917666in}}%
\pgfpathcurveto{\pgfqpoint{2.082695in}{1.923490in}}{\pgfqpoint{2.074795in}{1.926762in}}{\pgfqpoint{2.066559in}{1.926762in}}%
\pgfpathcurveto{\pgfqpoint{2.058323in}{1.926762in}}{\pgfqpoint{2.050423in}{1.923490in}}{\pgfqpoint{2.044599in}{1.917666in}}%
\pgfpathcurveto{\pgfqpoint{2.038775in}{1.911842in}}{\pgfqpoint{2.035502in}{1.903942in}}{\pgfqpoint{2.035502in}{1.895706in}}%
\pgfpathcurveto{\pgfqpoint{2.035502in}{1.887469in}}{\pgfqpoint{2.038775in}{1.879569in}}{\pgfqpoint{2.044599in}{1.873745in}}%
\pgfpathcurveto{\pgfqpoint{2.050423in}{1.867921in}}{\pgfqpoint{2.058323in}{1.864649in}}{\pgfqpoint{2.066559in}{1.864649in}}%
\pgfpathclose%
\pgfusepath{stroke,fill}%
\end{pgfscope}%
\begin{pgfscope}%
\pgfpathrectangle{\pgfqpoint{0.100000in}{0.212622in}}{\pgfqpoint{3.696000in}{3.696000in}}%
\pgfusepath{clip}%
\pgfsetbuttcap%
\pgfsetroundjoin%
\definecolor{currentfill}{rgb}{0.121569,0.466667,0.705882}%
\pgfsetfillcolor{currentfill}%
\pgfsetfillopacity{0.951843}%
\pgfsetlinewidth{1.003750pt}%
\definecolor{currentstroke}{rgb}{0.121569,0.466667,0.705882}%
\pgfsetstrokecolor{currentstroke}%
\pgfsetstrokeopacity{0.951843}%
\pgfsetdash{}{0pt}%
\pgfpathmoveto{\pgfqpoint{2.667920in}{1.078667in}}%
\pgfpathcurveto{\pgfqpoint{2.676156in}{1.078667in}}{\pgfqpoint{2.684056in}{1.081940in}}{\pgfqpoint{2.689880in}{1.087763in}}%
\pgfpathcurveto{\pgfqpoint{2.695704in}{1.093587in}}{\pgfqpoint{2.698977in}{1.101487in}}{\pgfqpoint{2.698977in}{1.109724in}}%
\pgfpathcurveto{\pgfqpoint{2.698977in}{1.117960in}}{\pgfqpoint{2.695704in}{1.125860in}}{\pgfqpoint{2.689880in}{1.131684in}}%
\pgfpathcurveto{\pgfqpoint{2.684056in}{1.137508in}}{\pgfqpoint{2.676156in}{1.140780in}}{\pgfqpoint{2.667920in}{1.140780in}}%
\pgfpathcurveto{\pgfqpoint{2.659684in}{1.140780in}}{\pgfqpoint{2.651784in}{1.137508in}}{\pgfqpoint{2.645960in}{1.131684in}}%
\pgfpathcurveto{\pgfqpoint{2.640136in}{1.125860in}}{\pgfqpoint{2.636864in}{1.117960in}}{\pgfqpoint{2.636864in}{1.109724in}}%
\pgfpathcurveto{\pgfqpoint{2.636864in}{1.101487in}}{\pgfqpoint{2.640136in}{1.093587in}}{\pgfqpoint{2.645960in}{1.087763in}}%
\pgfpathcurveto{\pgfqpoint{2.651784in}{1.081940in}}{\pgfqpoint{2.659684in}{1.078667in}}{\pgfqpoint{2.667920in}{1.078667in}}%
\pgfpathclose%
\pgfusepath{stroke,fill}%
\end{pgfscope}%
\begin{pgfscope}%
\pgfpathrectangle{\pgfqpoint{0.100000in}{0.212622in}}{\pgfqpoint{3.696000in}{3.696000in}}%
\pgfusepath{clip}%
\pgfsetbuttcap%
\pgfsetroundjoin%
\definecolor{currentfill}{rgb}{0.121569,0.466667,0.705882}%
\pgfsetfillcolor{currentfill}%
\pgfsetfillopacity{0.951940}%
\pgfsetlinewidth{1.003750pt}%
\definecolor{currentstroke}{rgb}{0.121569,0.466667,0.705882}%
\pgfsetstrokecolor{currentstroke}%
\pgfsetstrokeopacity{0.951940}%
\pgfsetdash{}{0pt}%
\pgfpathmoveto{\pgfqpoint{2.145460in}{0.952657in}}%
\pgfpathcurveto{\pgfqpoint{2.153696in}{0.952657in}}{\pgfqpoint{2.161596in}{0.955929in}}{\pgfqpoint{2.167420in}{0.961753in}}%
\pgfpathcurveto{\pgfqpoint{2.173244in}{0.967577in}}{\pgfqpoint{2.176516in}{0.975477in}}{\pgfqpoint{2.176516in}{0.983713in}}%
\pgfpathcurveto{\pgfqpoint{2.176516in}{0.991950in}}{\pgfqpoint{2.173244in}{0.999850in}}{\pgfqpoint{2.167420in}{1.005674in}}%
\pgfpathcurveto{\pgfqpoint{2.161596in}{1.011498in}}{\pgfqpoint{2.153696in}{1.014770in}}{\pgfqpoint{2.145460in}{1.014770in}}%
\pgfpathcurveto{\pgfqpoint{2.137224in}{1.014770in}}{\pgfqpoint{2.129324in}{1.011498in}}{\pgfqpoint{2.123500in}{1.005674in}}%
\pgfpathcurveto{\pgfqpoint{2.117676in}{0.999850in}}{\pgfqpoint{2.114403in}{0.991950in}}{\pgfqpoint{2.114403in}{0.983713in}}%
\pgfpathcurveto{\pgfqpoint{2.114403in}{0.975477in}}{\pgfqpoint{2.117676in}{0.967577in}}{\pgfqpoint{2.123500in}{0.961753in}}%
\pgfpathcurveto{\pgfqpoint{2.129324in}{0.955929in}}{\pgfqpoint{2.137224in}{0.952657in}}{\pgfqpoint{2.145460in}{0.952657in}}%
\pgfpathclose%
\pgfusepath{stroke,fill}%
\end{pgfscope}%
\begin{pgfscope}%
\pgfpathrectangle{\pgfqpoint{0.100000in}{0.212622in}}{\pgfqpoint{3.696000in}{3.696000in}}%
\pgfusepath{clip}%
\pgfsetbuttcap%
\pgfsetroundjoin%
\definecolor{currentfill}{rgb}{0.121569,0.466667,0.705882}%
\pgfsetfillcolor{currentfill}%
\pgfsetfillopacity{0.952740}%
\pgfsetlinewidth{1.003750pt}%
\definecolor{currentstroke}{rgb}{0.121569,0.466667,0.705882}%
\pgfsetstrokecolor{currentstroke}%
\pgfsetstrokeopacity{0.952740}%
\pgfsetdash{}{0pt}%
\pgfpathmoveto{\pgfqpoint{2.666387in}{1.074022in}}%
\pgfpathcurveto{\pgfqpoint{2.674624in}{1.074022in}}{\pgfqpoint{2.682524in}{1.077295in}}{\pgfqpoint{2.688348in}{1.083119in}}%
\pgfpathcurveto{\pgfqpoint{2.694172in}{1.088942in}}{\pgfqpoint{2.697444in}{1.096843in}}{\pgfqpoint{2.697444in}{1.105079in}}%
\pgfpathcurveto{\pgfqpoint{2.697444in}{1.113315in}}{\pgfqpoint{2.694172in}{1.121215in}}{\pgfqpoint{2.688348in}{1.127039in}}%
\pgfpathcurveto{\pgfqpoint{2.682524in}{1.132863in}}{\pgfqpoint{2.674624in}{1.136135in}}{\pgfqpoint{2.666387in}{1.136135in}}%
\pgfpathcurveto{\pgfqpoint{2.658151in}{1.136135in}}{\pgfqpoint{2.650251in}{1.132863in}}{\pgfqpoint{2.644427in}{1.127039in}}%
\pgfpathcurveto{\pgfqpoint{2.638603in}{1.121215in}}{\pgfqpoint{2.635331in}{1.113315in}}{\pgfqpoint{2.635331in}{1.105079in}}%
\pgfpathcurveto{\pgfqpoint{2.635331in}{1.096843in}}{\pgfqpoint{2.638603in}{1.088942in}}{\pgfqpoint{2.644427in}{1.083119in}}%
\pgfpathcurveto{\pgfqpoint{2.650251in}{1.077295in}}{\pgfqpoint{2.658151in}{1.074022in}}{\pgfqpoint{2.666387in}{1.074022in}}%
\pgfpathclose%
\pgfusepath{stroke,fill}%
\end{pgfscope}%
\begin{pgfscope}%
\pgfpathrectangle{\pgfqpoint{0.100000in}{0.212622in}}{\pgfqpoint{3.696000in}{3.696000in}}%
\pgfusepath{clip}%
\pgfsetbuttcap%
\pgfsetroundjoin%
\definecolor{currentfill}{rgb}{0.121569,0.466667,0.705882}%
\pgfsetfillcolor{currentfill}%
\pgfsetfillopacity{0.953104}%
\pgfsetlinewidth{1.003750pt}%
\definecolor{currentstroke}{rgb}{0.121569,0.466667,0.705882}%
\pgfsetstrokecolor{currentstroke}%
\pgfsetstrokeopacity{0.953104}%
\pgfsetdash{}{0pt}%
\pgfpathmoveto{\pgfqpoint{2.153350in}{0.951018in}}%
\pgfpathcurveto{\pgfqpoint{2.161587in}{0.951018in}}{\pgfqpoint{2.169487in}{0.954290in}}{\pgfqpoint{2.175311in}{0.960114in}}%
\pgfpathcurveto{\pgfqpoint{2.181134in}{0.965938in}}{\pgfqpoint{2.184407in}{0.973838in}}{\pgfqpoint{2.184407in}{0.982074in}}%
\pgfpathcurveto{\pgfqpoint{2.184407in}{0.990310in}}{\pgfqpoint{2.181134in}{0.998210in}}{\pgfqpoint{2.175311in}{1.004034in}}%
\pgfpathcurveto{\pgfqpoint{2.169487in}{1.009858in}}{\pgfqpoint{2.161587in}{1.013131in}}{\pgfqpoint{2.153350in}{1.013131in}}%
\pgfpathcurveto{\pgfqpoint{2.145114in}{1.013131in}}{\pgfqpoint{2.137214in}{1.009858in}}{\pgfqpoint{2.131390in}{1.004034in}}%
\pgfpathcurveto{\pgfqpoint{2.125566in}{0.998210in}}{\pgfqpoint{2.122294in}{0.990310in}}{\pgfqpoint{2.122294in}{0.982074in}}%
\pgfpathcurveto{\pgfqpoint{2.122294in}{0.973838in}}{\pgfqpoint{2.125566in}{0.965938in}}{\pgfqpoint{2.131390in}{0.960114in}}%
\pgfpathcurveto{\pgfqpoint{2.137214in}{0.954290in}}{\pgfqpoint{2.145114in}{0.951018in}}{\pgfqpoint{2.153350in}{0.951018in}}%
\pgfpathclose%
\pgfusepath{stroke,fill}%
\end{pgfscope}%
\begin{pgfscope}%
\pgfpathrectangle{\pgfqpoint{0.100000in}{0.212622in}}{\pgfqpoint{3.696000in}{3.696000in}}%
\pgfusepath{clip}%
\pgfsetbuttcap%
\pgfsetroundjoin%
\definecolor{currentfill}{rgb}{0.121569,0.466667,0.705882}%
\pgfsetfillcolor{currentfill}%
\pgfsetfillopacity{0.953500}%
\pgfsetlinewidth{1.003750pt}%
\definecolor{currentstroke}{rgb}{0.121569,0.466667,0.705882}%
\pgfsetstrokecolor{currentstroke}%
\pgfsetstrokeopacity{0.953500}%
\pgfsetdash{}{0pt}%
\pgfpathmoveto{\pgfqpoint{2.665387in}{1.069848in}}%
\pgfpathcurveto{\pgfqpoint{2.673623in}{1.069848in}}{\pgfqpoint{2.681523in}{1.073120in}}{\pgfqpoint{2.687347in}{1.078944in}}%
\pgfpathcurveto{\pgfqpoint{2.693171in}{1.084768in}}{\pgfqpoint{2.696443in}{1.092668in}}{\pgfqpoint{2.696443in}{1.100905in}}%
\pgfpathcurveto{\pgfqpoint{2.696443in}{1.109141in}}{\pgfqpoint{2.693171in}{1.117041in}}{\pgfqpoint{2.687347in}{1.122865in}}%
\pgfpathcurveto{\pgfqpoint{2.681523in}{1.128689in}}{\pgfqpoint{2.673623in}{1.131961in}}{\pgfqpoint{2.665387in}{1.131961in}}%
\pgfpathcurveto{\pgfqpoint{2.657150in}{1.131961in}}{\pgfqpoint{2.649250in}{1.128689in}}{\pgfqpoint{2.643426in}{1.122865in}}%
\pgfpathcurveto{\pgfqpoint{2.637603in}{1.117041in}}{\pgfqpoint{2.634330in}{1.109141in}}{\pgfqpoint{2.634330in}{1.100905in}}%
\pgfpathcurveto{\pgfqpoint{2.634330in}{1.092668in}}{\pgfqpoint{2.637603in}{1.084768in}}{\pgfqpoint{2.643426in}{1.078944in}}%
\pgfpathcurveto{\pgfqpoint{2.649250in}{1.073120in}}{\pgfqpoint{2.657150in}{1.069848in}}{\pgfqpoint{2.665387in}{1.069848in}}%
\pgfpathclose%
\pgfusepath{stroke,fill}%
\end{pgfscope}%
\begin{pgfscope}%
\pgfpathrectangle{\pgfqpoint{0.100000in}{0.212622in}}{\pgfqpoint{3.696000in}{3.696000in}}%
\pgfusepath{clip}%
\pgfsetbuttcap%
\pgfsetroundjoin%
\definecolor{currentfill}{rgb}{0.121569,0.466667,0.705882}%
\pgfsetfillcolor{currentfill}%
\pgfsetfillopacity{0.953988}%
\pgfsetlinewidth{1.003750pt}%
\definecolor{currentstroke}{rgb}{0.121569,0.466667,0.705882}%
\pgfsetstrokecolor{currentstroke}%
\pgfsetstrokeopacity{0.953988}%
\pgfsetdash{}{0pt}%
\pgfpathmoveto{\pgfqpoint{2.162399in}{0.943863in}}%
\pgfpathcurveto{\pgfqpoint{2.170635in}{0.943863in}}{\pgfqpoint{2.178535in}{0.947135in}}{\pgfqpoint{2.184359in}{0.952959in}}%
\pgfpathcurveto{\pgfqpoint{2.190183in}{0.958783in}}{\pgfqpoint{2.193455in}{0.966683in}}{\pgfqpoint{2.193455in}{0.974919in}}%
\pgfpathcurveto{\pgfqpoint{2.193455in}{0.983156in}}{\pgfqpoint{2.190183in}{0.991056in}}{\pgfqpoint{2.184359in}{0.996880in}}%
\pgfpathcurveto{\pgfqpoint{2.178535in}{1.002703in}}{\pgfqpoint{2.170635in}{1.005976in}}{\pgfqpoint{2.162399in}{1.005976in}}%
\pgfpathcurveto{\pgfqpoint{2.154162in}{1.005976in}}{\pgfqpoint{2.146262in}{1.002703in}}{\pgfqpoint{2.140438in}{0.996880in}}%
\pgfpathcurveto{\pgfqpoint{2.134614in}{0.991056in}}{\pgfqpoint{2.131342in}{0.983156in}}{\pgfqpoint{2.131342in}{0.974919in}}%
\pgfpathcurveto{\pgfqpoint{2.131342in}{0.966683in}}{\pgfqpoint{2.134614in}{0.958783in}}{\pgfqpoint{2.140438in}{0.952959in}}%
\pgfpathcurveto{\pgfqpoint{2.146262in}{0.947135in}}{\pgfqpoint{2.154162in}{0.943863in}}{\pgfqpoint{2.162399in}{0.943863in}}%
\pgfpathclose%
\pgfusepath{stroke,fill}%
\end{pgfscope}%
\begin{pgfscope}%
\pgfpathrectangle{\pgfqpoint{0.100000in}{0.212622in}}{\pgfqpoint{3.696000in}{3.696000in}}%
\pgfusepath{clip}%
\pgfsetbuttcap%
\pgfsetroundjoin%
\definecolor{currentfill}{rgb}{0.121569,0.466667,0.705882}%
\pgfsetfillcolor{currentfill}%
\pgfsetfillopacity{0.954490}%
\pgfsetlinewidth{1.003750pt}%
\definecolor{currentstroke}{rgb}{0.121569,0.466667,0.705882}%
\pgfsetstrokecolor{currentstroke}%
\pgfsetstrokeopacity{0.954490}%
\pgfsetdash{}{0pt}%
\pgfpathmoveto{\pgfqpoint{2.167136in}{0.939215in}}%
\pgfpathcurveto{\pgfqpoint{2.175372in}{0.939215in}}{\pgfqpoint{2.183272in}{0.942488in}}{\pgfqpoint{2.189096in}{0.948312in}}%
\pgfpathcurveto{\pgfqpoint{2.194920in}{0.954135in}}{\pgfqpoint{2.198192in}{0.962036in}}{\pgfqpoint{2.198192in}{0.970272in}}%
\pgfpathcurveto{\pgfqpoint{2.198192in}{0.978508in}}{\pgfqpoint{2.194920in}{0.986408in}}{\pgfqpoint{2.189096in}{0.992232in}}%
\pgfpathcurveto{\pgfqpoint{2.183272in}{0.998056in}}{\pgfqpoint{2.175372in}{1.001328in}}{\pgfqpoint{2.167136in}{1.001328in}}%
\pgfpathcurveto{\pgfqpoint{2.158900in}{1.001328in}}{\pgfqpoint{2.151000in}{0.998056in}}{\pgfqpoint{2.145176in}{0.992232in}}%
\pgfpathcurveto{\pgfqpoint{2.139352in}{0.986408in}}{\pgfqpoint{2.136079in}{0.978508in}}{\pgfqpoint{2.136079in}{0.970272in}}%
\pgfpathcurveto{\pgfqpoint{2.136079in}{0.962036in}}{\pgfqpoint{2.139352in}{0.954135in}}{\pgfqpoint{2.145176in}{0.948312in}}%
\pgfpathcurveto{\pgfqpoint{2.151000in}{0.942488in}}{\pgfqpoint{2.158900in}{0.939215in}}{\pgfqpoint{2.167136in}{0.939215in}}%
\pgfpathclose%
\pgfusepath{stroke,fill}%
\end{pgfscope}%
\begin{pgfscope}%
\pgfpathrectangle{\pgfqpoint{0.100000in}{0.212622in}}{\pgfqpoint{3.696000in}{3.696000in}}%
\pgfusepath{clip}%
\pgfsetbuttcap%
\pgfsetroundjoin%
\definecolor{currentfill}{rgb}{0.121569,0.466667,0.705882}%
\pgfsetfillcolor{currentfill}%
\pgfsetfillopacity{0.954832}%
\pgfsetlinewidth{1.003750pt}%
\definecolor{currentstroke}{rgb}{0.121569,0.466667,0.705882}%
\pgfsetstrokecolor{currentstroke}%
\pgfsetstrokeopacity{0.954832}%
\pgfsetdash{}{0pt}%
\pgfpathmoveto{\pgfqpoint{2.661888in}{1.063071in}}%
\pgfpathcurveto{\pgfqpoint{2.670125in}{1.063071in}}{\pgfqpoint{2.678025in}{1.066343in}}{\pgfqpoint{2.683849in}{1.072167in}}%
\pgfpathcurveto{\pgfqpoint{2.689672in}{1.077991in}}{\pgfqpoint{2.692945in}{1.085891in}}{\pgfqpoint{2.692945in}{1.094127in}}%
\pgfpathcurveto{\pgfqpoint{2.692945in}{1.102364in}}{\pgfqpoint{2.689672in}{1.110264in}}{\pgfqpoint{2.683849in}{1.116088in}}%
\pgfpathcurveto{\pgfqpoint{2.678025in}{1.121912in}}{\pgfqpoint{2.670125in}{1.125184in}}{\pgfqpoint{2.661888in}{1.125184in}}%
\pgfpathcurveto{\pgfqpoint{2.653652in}{1.125184in}}{\pgfqpoint{2.645752in}{1.121912in}}{\pgfqpoint{2.639928in}{1.116088in}}%
\pgfpathcurveto{\pgfqpoint{2.634104in}{1.110264in}}{\pgfqpoint{2.630832in}{1.102364in}}{\pgfqpoint{2.630832in}{1.094127in}}%
\pgfpathcurveto{\pgfqpoint{2.630832in}{1.085891in}}{\pgfqpoint{2.634104in}{1.077991in}}{\pgfqpoint{2.639928in}{1.072167in}}%
\pgfpathcurveto{\pgfqpoint{2.645752in}{1.066343in}}{\pgfqpoint{2.653652in}{1.063071in}}{\pgfqpoint{2.661888in}{1.063071in}}%
\pgfpathclose%
\pgfusepath{stroke,fill}%
\end{pgfscope}%
\begin{pgfscope}%
\pgfpathrectangle{\pgfqpoint{0.100000in}{0.212622in}}{\pgfqpoint{3.696000in}{3.696000in}}%
\pgfusepath{clip}%
\pgfsetbuttcap%
\pgfsetroundjoin%
\definecolor{currentfill}{rgb}{0.121569,0.466667,0.705882}%
\pgfsetfillcolor{currentfill}%
\pgfsetfillopacity{0.955010}%
\pgfsetlinewidth{1.003750pt}%
\definecolor{currentstroke}{rgb}{0.121569,0.466667,0.705882}%
\pgfsetstrokecolor{currentstroke}%
\pgfsetstrokeopacity{0.955010}%
\pgfsetdash{}{0pt}%
\pgfpathmoveto{\pgfqpoint{2.172520in}{0.935095in}}%
\pgfpathcurveto{\pgfqpoint{2.180757in}{0.935095in}}{\pgfqpoint{2.188657in}{0.938368in}}{\pgfqpoint{2.194481in}{0.944192in}}%
\pgfpathcurveto{\pgfqpoint{2.200305in}{0.950015in}}{\pgfqpoint{2.203577in}{0.957916in}}{\pgfqpoint{2.203577in}{0.966152in}}%
\pgfpathcurveto{\pgfqpoint{2.203577in}{0.974388in}}{\pgfqpoint{2.200305in}{0.982288in}}{\pgfqpoint{2.194481in}{0.988112in}}%
\pgfpathcurveto{\pgfqpoint{2.188657in}{0.993936in}}{\pgfqpoint{2.180757in}{0.997208in}}{\pgfqpoint{2.172520in}{0.997208in}}%
\pgfpathcurveto{\pgfqpoint{2.164284in}{0.997208in}}{\pgfqpoint{2.156384in}{0.993936in}}{\pgfqpoint{2.150560in}{0.988112in}}%
\pgfpathcurveto{\pgfqpoint{2.144736in}{0.982288in}}{\pgfqpoint{2.141464in}{0.974388in}}{\pgfqpoint{2.141464in}{0.966152in}}%
\pgfpathcurveto{\pgfqpoint{2.141464in}{0.957916in}}{\pgfqpoint{2.144736in}{0.950015in}}{\pgfqpoint{2.150560in}{0.944192in}}%
\pgfpathcurveto{\pgfqpoint{2.156384in}{0.938368in}}{\pgfqpoint{2.164284in}{0.935095in}}{\pgfqpoint{2.172520in}{0.935095in}}%
\pgfpathclose%
\pgfusepath{stroke,fill}%
\end{pgfscope}%
\begin{pgfscope}%
\pgfpathrectangle{\pgfqpoint{0.100000in}{0.212622in}}{\pgfqpoint{3.696000in}{3.696000in}}%
\pgfusepath{clip}%
\pgfsetbuttcap%
\pgfsetroundjoin%
\definecolor{currentfill}{rgb}{0.121569,0.466667,0.705882}%
\pgfsetfillcolor{currentfill}%
\pgfsetfillopacity{0.955749}%
\pgfsetlinewidth{1.003750pt}%
\definecolor{currentstroke}{rgb}{0.121569,0.466667,0.705882}%
\pgfsetstrokecolor{currentstroke}%
\pgfsetstrokeopacity{0.955749}%
\pgfsetdash{}{0pt}%
\pgfpathmoveto{\pgfqpoint{2.658094in}{1.058687in}}%
\pgfpathcurveto{\pgfqpoint{2.666330in}{1.058687in}}{\pgfqpoint{2.674230in}{1.061959in}}{\pgfqpoint{2.680054in}{1.067783in}}%
\pgfpathcurveto{\pgfqpoint{2.685878in}{1.073607in}}{\pgfqpoint{2.689151in}{1.081507in}}{\pgfqpoint{2.689151in}{1.089743in}}%
\pgfpathcurveto{\pgfqpoint{2.689151in}{1.097980in}}{\pgfqpoint{2.685878in}{1.105880in}}{\pgfqpoint{2.680054in}{1.111704in}}%
\pgfpathcurveto{\pgfqpoint{2.674230in}{1.117528in}}{\pgfqpoint{2.666330in}{1.120800in}}{\pgfqpoint{2.658094in}{1.120800in}}%
\pgfpathcurveto{\pgfqpoint{2.649858in}{1.120800in}}{\pgfqpoint{2.641958in}{1.117528in}}{\pgfqpoint{2.636134in}{1.111704in}}%
\pgfpathcurveto{\pgfqpoint{2.630310in}{1.105880in}}{\pgfqpoint{2.627038in}{1.097980in}}{\pgfqpoint{2.627038in}{1.089743in}}%
\pgfpathcurveto{\pgfqpoint{2.627038in}{1.081507in}}{\pgfqpoint{2.630310in}{1.073607in}}{\pgfqpoint{2.636134in}{1.067783in}}%
\pgfpathcurveto{\pgfqpoint{2.641958in}{1.061959in}}{\pgfqpoint{2.649858in}{1.058687in}}{\pgfqpoint{2.658094in}{1.058687in}}%
\pgfpathclose%
\pgfusepath{stroke,fill}%
\end{pgfscope}%
\begin{pgfscope}%
\pgfpathrectangle{\pgfqpoint{0.100000in}{0.212622in}}{\pgfqpoint{3.696000in}{3.696000in}}%
\pgfusepath{clip}%
\pgfsetbuttcap%
\pgfsetroundjoin%
\definecolor{currentfill}{rgb}{0.121569,0.466667,0.705882}%
\pgfsetfillcolor{currentfill}%
\pgfsetfillopacity{0.955837}%
\pgfsetlinewidth{1.003750pt}%
\definecolor{currentstroke}{rgb}{0.121569,0.466667,0.705882}%
\pgfsetstrokecolor{currentstroke}%
\pgfsetstrokeopacity{0.955837}%
\pgfsetdash{}{0pt}%
\pgfpathmoveto{\pgfqpoint{2.180083in}{0.932410in}}%
\pgfpathcurveto{\pgfqpoint{2.188319in}{0.932410in}}{\pgfqpoint{2.196219in}{0.935682in}}{\pgfqpoint{2.202043in}{0.941506in}}%
\pgfpathcurveto{\pgfqpoint{2.207867in}{0.947330in}}{\pgfqpoint{2.211139in}{0.955230in}}{\pgfqpoint{2.211139in}{0.963466in}}%
\pgfpathcurveto{\pgfqpoint{2.211139in}{0.971703in}}{\pgfqpoint{2.207867in}{0.979603in}}{\pgfqpoint{2.202043in}{0.985427in}}%
\pgfpathcurveto{\pgfqpoint{2.196219in}{0.991250in}}{\pgfqpoint{2.188319in}{0.994523in}}{\pgfqpoint{2.180083in}{0.994523in}}%
\pgfpathcurveto{\pgfqpoint{2.171846in}{0.994523in}}{\pgfqpoint{2.163946in}{0.991250in}}{\pgfqpoint{2.158122in}{0.985427in}}%
\pgfpathcurveto{\pgfqpoint{2.152298in}{0.979603in}}{\pgfqpoint{2.149026in}{0.971703in}}{\pgfqpoint{2.149026in}{0.963466in}}%
\pgfpathcurveto{\pgfqpoint{2.149026in}{0.955230in}}{\pgfqpoint{2.152298in}{0.947330in}}{\pgfqpoint{2.158122in}{0.941506in}}%
\pgfpathcurveto{\pgfqpoint{2.163946in}{0.935682in}}{\pgfqpoint{2.171846in}{0.932410in}}{\pgfqpoint{2.180083in}{0.932410in}}%
\pgfpathclose%
\pgfusepath{stroke,fill}%
\end{pgfscope}%
\begin{pgfscope}%
\pgfpathrectangle{\pgfqpoint{0.100000in}{0.212622in}}{\pgfqpoint{3.696000in}{3.696000in}}%
\pgfusepath{clip}%
\pgfsetbuttcap%
\pgfsetroundjoin%
\definecolor{currentfill}{rgb}{0.121569,0.466667,0.705882}%
\pgfsetfillcolor{currentfill}%
\pgfsetfillopacity{0.957095}%
\pgfsetlinewidth{1.003750pt}%
\definecolor{currentstroke}{rgb}{0.121569,0.466667,0.705882}%
\pgfsetstrokecolor{currentstroke}%
\pgfsetstrokeopacity{0.957095}%
\pgfsetdash{}{0pt}%
\pgfpathmoveto{\pgfqpoint{2.651080in}{1.049584in}}%
\pgfpathcurveto{\pgfqpoint{2.659316in}{1.049584in}}{\pgfqpoint{2.667216in}{1.052857in}}{\pgfqpoint{2.673040in}{1.058681in}}%
\pgfpathcurveto{\pgfqpoint{2.678864in}{1.064505in}}{\pgfqpoint{2.682136in}{1.072405in}}{\pgfqpoint{2.682136in}{1.080641in}}%
\pgfpathcurveto{\pgfqpoint{2.682136in}{1.088877in}}{\pgfqpoint{2.678864in}{1.096777in}}{\pgfqpoint{2.673040in}{1.102601in}}%
\pgfpathcurveto{\pgfqpoint{2.667216in}{1.108425in}}{\pgfqpoint{2.659316in}{1.111697in}}{\pgfqpoint{2.651080in}{1.111697in}}%
\pgfpathcurveto{\pgfqpoint{2.642843in}{1.111697in}}{\pgfqpoint{2.634943in}{1.108425in}}{\pgfqpoint{2.629119in}{1.102601in}}%
\pgfpathcurveto{\pgfqpoint{2.623296in}{1.096777in}}{\pgfqpoint{2.620023in}{1.088877in}}{\pgfqpoint{2.620023in}{1.080641in}}%
\pgfpathcurveto{\pgfqpoint{2.620023in}{1.072405in}}{\pgfqpoint{2.623296in}{1.064505in}}{\pgfqpoint{2.629119in}{1.058681in}}%
\pgfpathcurveto{\pgfqpoint{2.634943in}{1.052857in}}{\pgfqpoint{2.642843in}{1.049584in}}{\pgfqpoint{2.651080in}{1.049584in}}%
\pgfpathclose%
\pgfusepath{stroke,fill}%
\end{pgfscope}%
\begin{pgfscope}%
\pgfpathrectangle{\pgfqpoint{0.100000in}{0.212622in}}{\pgfqpoint{3.696000in}{3.696000in}}%
\pgfusepath{clip}%
\pgfsetbuttcap%
\pgfsetroundjoin%
\definecolor{currentfill}{rgb}{0.121569,0.466667,0.705882}%
\pgfsetfillcolor{currentfill}%
\pgfsetfillopacity{0.957118}%
\pgfsetlinewidth{1.003750pt}%
\definecolor{currentstroke}{rgb}{0.121569,0.466667,0.705882}%
\pgfsetstrokecolor{currentstroke}%
\pgfsetstrokeopacity{0.957118}%
\pgfsetdash{}{0pt}%
\pgfpathmoveto{\pgfqpoint{2.188584in}{0.932570in}}%
\pgfpathcurveto{\pgfqpoint{2.196820in}{0.932570in}}{\pgfqpoint{2.204720in}{0.935843in}}{\pgfqpoint{2.210544in}{0.941667in}}%
\pgfpathcurveto{\pgfqpoint{2.216368in}{0.947491in}}{\pgfqpoint{2.219641in}{0.955391in}}{\pgfqpoint{2.219641in}{0.963627in}}%
\pgfpathcurveto{\pgfqpoint{2.219641in}{0.971863in}}{\pgfqpoint{2.216368in}{0.979763in}}{\pgfqpoint{2.210544in}{0.985587in}}%
\pgfpathcurveto{\pgfqpoint{2.204720in}{0.991411in}}{\pgfqpoint{2.196820in}{0.994683in}}{\pgfqpoint{2.188584in}{0.994683in}}%
\pgfpathcurveto{\pgfqpoint{2.180348in}{0.994683in}}{\pgfqpoint{2.172448in}{0.991411in}}{\pgfqpoint{2.166624in}{0.985587in}}%
\pgfpathcurveto{\pgfqpoint{2.160800in}{0.979763in}}{\pgfqpoint{2.157528in}{0.971863in}}{\pgfqpoint{2.157528in}{0.963627in}}%
\pgfpathcurveto{\pgfqpoint{2.157528in}{0.955391in}}{\pgfqpoint{2.160800in}{0.947491in}}{\pgfqpoint{2.166624in}{0.941667in}}%
\pgfpathcurveto{\pgfqpoint{2.172448in}{0.935843in}}{\pgfqpoint{2.180348in}{0.932570in}}{\pgfqpoint{2.188584in}{0.932570in}}%
\pgfpathclose%
\pgfusepath{stroke,fill}%
\end{pgfscope}%
\begin{pgfscope}%
\pgfpathrectangle{\pgfqpoint{0.100000in}{0.212622in}}{\pgfqpoint{3.696000in}{3.696000in}}%
\pgfusepath{clip}%
\pgfsetbuttcap%
\pgfsetroundjoin%
\definecolor{currentfill}{rgb}{0.121569,0.466667,0.705882}%
\pgfsetfillcolor{currentfill}%
\pgfsetfillopacity{0.958087}%
\pgfsetlinewidth{1.003750pt}%
\definecolor{currentstroke}{rgb}{0.121569,0.466667,0.705882}%
\pgfsetstrokecolor{currentstroke}%
\pgfsetstrokeopacity{0.958087}%
\pgfsetdash{}{0pt}%
\pgfpathmoveto{\pgfqpoint{2.192786in}{0.931861in}}%
\pgfpathcurveto{\pgfqpoint{2.201022in}{0.931861in}}{\pgfqpoint{2.208922in}{0.935133in}}{\pgfqpoint{2.214746in}{0.940957in}}%
\pgfpathcurveto{\pgfqpoint{2.220570in}{0.946781in}}{\pgfqpoint{2.223842in}{0.954681in}}{\pgfqpoint{2.223842in}{0.962917in}}%
\pgfpathcurveto{\pgfqpoint{2.223842in}{0.971153in}}{\pgfqpoint{2.220570in}{0.979053in}}{\pgfqpoint{2.214746in}{0.984877in}}%
\pgfpathcurveto{\pgfqpoint{2.208922in}{0.990701in}}{\pgfqpoint{2.201022in}{0.993974in}}{\pgfqpoint{2.192786in}{0.993974in}}%
\pgfpathcurveto{\pgfqpoint{2.184550in}{0.993974in}}{\pgfqpoint{2.176650in}{0.990701in}}{\pgfqpoint{2.170826in}{0.984877in}}%
\pgfpathcurveto{\pgfqpoint{2.165002in}{0.979053in}}{\pgfqpoint{2.161729in}{0.971153in}}{\pgfqpoint{2.161729in}{0.962917in}}%
\pgfpathcurveto{\pgfqpoint{2.161729in}{0.954681in}}{\pgfqpoint{2.165002in}{0.946781in}}{\pgfqpoint{2.170826in}{0.940957in}}%
\pgfpathcurveto{\pgfqpoint{2.176650in}{0.935133in}}{\pgfqpoint{2.184550in}{0.931861in}}{\pgfqpoint{2.192786in}{0.931861in}}%
\pgfpathclose%
\pgfusepath{stroke,fill}%
\end{pgfscope}%
\begin{pgfscope}%
\pgfpathrectangle{\pgfqpoint{0.100000in}{0.212622in}}{\pgfqpoint{3.696000in}{3.696000in}}%
\pgfusepath{clip}%
\pgfsetbuttcap%
\pgfsetroundjoin%
\definecolor{currentfill}{rgb}{0.121569,0.466667,0.705882}%
\pgfsetfillcolor{currentfill}%
\pgfsetfillopacity{0.958575}%
\pgfsetlinewidth{1.003750pt}%
\definecolor{currentstroke}{rgb}{0.121569,0.466667,0.705882}%
\pgfsetstrokecolor{currentstroke}%
\pgfsetstrokeopacity{0.958575}%
\pgfsetdash{}{0pt}%
\pgfpathmoveto{\pgfqpoint{2.646579in}{1.040333in}}%
\pgfpathcurveto{\pgfqpoint{2.654815in}{1.040333in}}{\pgfqpoint{2.662715in}{1.043605in}}{\pgfqpoint{2.668539in}{1.049429in}}%
\pgfpathcurveto{\pgfqpoint{2.674363in}{1.055253in}}{\pgfqpoint{2.677636in}{1.063153in}}{\pgfqpoint{2.677636in}{1.071390in}}%
\pgfpathcurveto{\pgfqpoint{2.677636in}{1.079626in}}{\pgfqpoint{2.674363in}{1.087526in}}{\pgfqpoint{2.668539in}{1.093350in}}%
\pgfpathcurveto{\pgfqpoint{2.662715in}{1.099174in}}{\pgfqpoint{2.654815in}{1.102446in}}{\pgfqpoint{2.646579in}{1.102446in}}%
\pgfpathcurveto{\pgfqpoint{2.638343in}{1.102446in}}{\pgfqpoint{2.630443in}{1.099174in}}{\pgfqpoint{2.624619in}{1.093350in}}%
\pgfpathcurveto{\pgfqpoint{2.618795in}{1.087526in}}{\pgfqpoint{2.615523in}{1.079626in}}{\pgfqpoint{2.615523in}{1.071390in}}%
\pgfpathcurveto{\pgfqpoint{2.615523in}{1.063153in}}{\pgfqpoint{2.618795in}{1.055253in}}{\pgfqpoint{2.624619in}{1.049429in}}%
\pgfpathcurveto{\pgfqpoint{2.630443in}{1.043605in}}{\pgfqpoint{2.638343in}{1.040333in}}{\pgfqpoint{2.646579in}{1.040333in}}%
\pgfpathclose%
\pgfusepath{stroke,fill}%
\end{pgfscope}%
\begin{pgfscope}%
\pgfpathrectangle{\pgfqpoint{0.100000in}{0.212622in}}{\pgfqpoint{3.696000in}{3.696000in}}%
\pgfusepath{clip}%
\pgfsetbuttcap%
\pgfsetroundjoin%
\definecolor{currentfill}{rgb}{0.121569,0.466667,0.705882}%
\pgfsetfillcolor{currentfill}%
\pgfsetfillopacity{0.958867}%
\pgfsetlinewidth{1.003750pt}%
\definecolor{currentstroke}{rgb}{0.121569,0.466667,0.705882}%
\pgfsetstrokecolor{currentstroke}%
\pgfsetstrokeopacity{0.958867}%
\pgfsetdash{}{0pt}%
\pgfpathmoveto{\pgfqpoint{2.198830in}{0.927558in}}%
\pgfpathcurveto{\pgfqpoint{2.207066in}{0.927558in}}{\pgfqpoint{2.214966in}{0.930830in}}{\pgfqpoint{2.220790in}{0.936654in}}%
\pgfpathcurveto{\pgfqpoint{2.226614in}{0.942478in}}{\pgfqpoint{2.229886in}{0.950378in}}{\pgfqpoint{2.229886in}{0.958614in}}%
\pgfpathcurveto{\pgfqpoint{2.229886in}{0.966851in}}{\pgfqpoint{2.226614in}{0.974751in}}{\pgfqpoint{2.220790in}{0.980575in}}%
\pgfpathcurveto{\pgfqpoint{2.214966in}{0.986399in}}{\pgfqpoint{2.207066in}{0.989671in}}{\pgfqpoint{2.198830in}{0.989671in}}%
\pgfpathcurveto{\pgfqpoint{2.190594in}{0.989671in}}{\pgfqpoint{2.182694in}{0.986399in}}{\pgfqpoint{2.176870in}{0.980575in}}%
\pgfpathcurveto{\pgfqpoint{2.171046in}{0.974751in}}{\pgfqpoint{2.167773in}{0.966851in}}{\pgfqpoint{2.167773in}{0.958614in}}%
\pgfpathcurveto{\pgfqpoint{2.167773in}{0.950378in}}{\pgfqpoint{2.171046in}{0.942478in}}{\pgfqpoint{2.176870in}{0.936654in}}%
\pgfpathcurveto{\pgfqpoint{2.182694in}{0.930830in}}{\pgfqpoint{2.190594in}{0.927558in}}{\pgfqpoint{2.198830in}{0.927558in}}%
\pgfpathclose%
\pgfusepath{stroke,fill}%
\end{pgfscope}%
\begin{pgfscope}%
\pgfpathrectangle{\pgfqpoint{0.100000in}{0.212622in}}{\pgfqpoint{3.696000in}{3.696000in}}%
\pgfusepath{clip}%
\pgfsetbuttcap%
\pgfsetroundjoin%
\definecolor{currentfill}{rgb}{0.121569,0.466667,0.705882}%
\pgfsetfillcolor{currentfill}%
\pgfsetfillopacity{0.959634}%
\pgfsetlinewidth{1.003750pt}%
\definecolor{currentstroke}{rgb}{0.121569,0.466667,0.705882}%
\pgfsetstrokecolor{currentstroke}%
\pgfsetstrokeopacity{0.959634}%
\pgfsetdash{}{0pt}%
\pgfpathmoveto{\pgfqpoint{2.206029in}{0.921335in}}%
\pgfpathcurveto{\pgfqpoint{2.214265in}{0.921335in}}{\pgfqpoint{2.222165in}{0.924607in}}{\pgfqpoint{2.227989in}{0.930431in}}%
\pgfpathcurveto{\pgfqpoint{2.233813in}{0.936255in}}{\pgfqpoint{2.237085in}{0.944155in}}{\pgfqpoint{2.237085in}{0.952392in}}%
\pgfpathcurveto{\pgfqpoint{2.237085in}{0.960628in}}{\pgfqpoint{2.233813in}{0.968528in}}{\pgfqpoint{2.227989in}{0.974352in}}%
\pgfpathcurveto{\pgfqpoint{2.222165in}{0.980176in}}{\pgfqpoint{2.214265in}{0.983448in}}{\pgfqpoint{2.206029in}{0.983448in}}%
\pgfpathcurveto{\pgfqpoint{2.197792in}{0.983448in}}{\pgfqpoint{2.189892in}{0.980176in}}{\pgfqpoint{2.184069in}{0.974352in}}%
\pgfpathcurveto{\pgfqpoint{2.178245in}{0.968528in}}{\pgfqpoint{2.174972in}{0.960628in}}{\pgfqpoint{2.174972in}{0.952392in}}%
\pgfpathcurveto{\pgfqpoint{2.174972in}{0.944155in}}{\pgfqpoint{2.178245in}{0.936255in}}{\pgfqpoint{2.184069in}{0.930431in}}%
\pgfpathcurveto{\pgfqpoint{2.189892in}{0.924607in}}{\pgfqpoint{2.197792in}{0.921335in}}{\pgfqpoint{2.206029in}{0.921335in}}%
\pgfpathclose%
\pgfusepath{stroke,fill}%
\end{pgfscope}%
\begin{pgfscope}%
\pgfpathrectangle{\pgfqpoint{0.100000in}{0.212622in}}{\pgfqpoint{3.696000in}{3.696000in}}%
\pgfusepath{clip}%
\pgfsetbuttcap%
\pgfsetroundjoin%
\definecolor{currentfill}{rgb}{0.121569,0.466667,0.705882}%
\pgfsetfillcolor{currentfill}%
\pgfsetfillopacity{0.959800}%
\pgfsetlinewidth{1.003750pt}%
\definecolor{currentstroke}{rgb}{0.121569,0.466667,0.705882}%
\pgfsetstrokecolor{currentstroke}%
\pgfsetstrokeopacity{0.959800}%
\pgfsetdash{}{0pt}%
\pgfpathmoveto{\pgfqpoint{2.645129in}{1.032816in}}%
\pgfpathcurveto{\pgfqpoint{2.653365in}{1.032816in}}{\pgfqpoint{2.661265in}{1.036088in}}{\pgfqpoint{2.667089in}{1.041912in}}%
\pgfpathcurveto{\pgfqpoint{2.672913in}{1.047736in}}{\pgfqpoint{2.676186in}{1.055636in}}{\pgfqpoint{2.676186in}{1.063872in}}%
\pgfpathcurveto{\pgfqpoint{2.676186in}{1.072109in}}{\pgfqpoint{2.672913in}{1.080009in}}{\pgfqpoint{2.667089in}{1.085833in}}%
\pgfpathcurveto{\pgfqpoint{2.661265in}{1.091656in}}{\pgfqpoint{2.653365in}{1.094929in}}{\pgfqpoint{2.645129in}{1.094929in}}%
\pgfpathcurveto{\pgfqpoint{2.636893in}{1.094929in}}{\pgfqpoint{2.628993in}{1.091656in}}{\pgfqpoint{2.623169in}{1.085833in}}%
\pgfpathcurveto{\pgfqpoint{2.617345in}{1.080009in}}{\pgfqpoint{2.614073in}{1.072109in}}{\pgfqpoint{2.614073in}{1.063872in}}%
\pgfpathcurveto{\pgfqpoint{2.614073in}{1.055636in}}{\pgfqpoint{2.617345in}{1.047736in}}{\pgfqpoint{2.623169in}{1.041912in}}%
\pgfpathcurveto{\pgfqpoint{2.628993in}{1.036088in}}{\pgfqpoint{2.636893in}{1.032816in}}{\pgfqpoint{2.645129in}{1.032816in}}%
\pgfpathclose%
\pgfusepath{stroke,fill}%
\end{pgfscope}%
\begin{pgfscope}%
\pgfpathrectangle{\pgfqpoint{0.100000in}{0.212622in}}{\pgfqpoint{3.696000in}{3.696000in}}%
\pgfusepath{clip}%
\pgfsetbuttcap%
\pgfsetroundjoin%
\definecolor{currentfill}{rgb}{0.121569,0.466667,0.705882}%
\pgfsetfillcolor{currentfill}%
\pgfsetfillopacity{0.960237}%
\pgfsetlinewidth{1.003750pt}%
\definecolor{currentstroke}{rgb}{0.121569,0.466667,0.705882}%
\pgfsetstrokecolor{currentstroke}%
\pgfsetstrokeopacity{0.960237}%
\pgfsetdash{}{0pt}%
\pgfpathmoveto{\pgfqpoint{2.214533in}{0.913316in}}%
\pgfpathcurveto{\pgfqpoint{2.222770in}{0.913316in}}{\pgfqpoint{2.230670in}{0.916589in}}{\pgfqpoint{2.236494in}{0.922413in}}%
\pgfpathcurveto{\pgfqpoint{2.242318in}{0.928237in}}{\pgfqpoint{2.245590in}{0.936137in}}{\pgfqpoint{2.245590in}{0.944373in}}%
\pgfpathcurveto{\pgfqpoint{2.245590in}{0.952609in}}{\pgfqpoint{2.242318in}{0.960509in}}{\pgfqpoint{2.236494in}{0.966333in}}%
\pgfpathcurveto{\pgfqpoint{2.230670in}{0.972157in}}{\pgfqpoint{2.222770in}{0.975429in}}{\pgfqpoint{2.214533in}{0.975429in}}%
\pgfpathcurveto{\pgfqpoint{2.206297in}{0.975429in}}{\pgfqpoint{2.198397in}{0.972157in}}{\pgfqpoint{2.192573in}{0.966333in}}%
\pgfpathcurveto{\pgfqpoint{2.186749in}{0.960509in}}{\pgfqpoint{2.183477in}{0.952609in}}{\pgfqpoint{2.183477in}{0.944373in}}%
\pgfpathcurveto{\pgfqpoint{2.183477in}{0.936137in}}{\pgfqpoint{2.186749in}{0.928237in}}{\pgfqpoint{2.192573in}{0.922413in}}%
\pgfpathcurveto{\pgfqpoint{2.198397in}{0.916589in}}{\pgfqpoint{2.206297in}{0.913316in}}{\pgfqpoint{2.214533in}{0.913316in}}%
\pgfpathclose%
\pgfusepath{stroke,fill}%
\end{pgfscope}%
\begin{pgfscope}%
\pgfpathrectangle{\pgfqpoint{0.100000in}{0.212622in}}{\pgfqpoint{3.696000in}{3.696000in}}%
\pgfusepath{clip}%
\pgfsetbuttcap%
\pgfsetroundjoin%
\definecolor{currentfill}{rgb}{0.121569,0.466667,0.705882}%
\pgfsetfillcolor{currentfill}%
\pgfsetfillopacity{0.961964}%
\pgfsetlinewidth{1.003750pt}%
\definecolor{currentstroke}{rgb}{0.121569,0.466667,0.705882}%
\pgfsetstrokecolor{currentstroke}%
\pgfsetstrokeopacity{0.961964}%
\pgfsetdash{}{0pt}%
\pgfpathmoveto{\pgfqpoint{2.642664in}{1.018872in}}%
\pgfpathcurveto{\pgfqpoint{2.650900in}{1.018872in}}{\pgfqpoint{2.658800in}{1.022145in}}{\pgfqpoint{2.664624in}{1.027969in}}%
\pgfpathcurveto{\pgfqpoint{2.670448in}{1.033793in}}{\pgfqpoint{2.673720in}{1.041693in}}{\pgfqpoint{2.673720in}{1.049929in}}%
\pgfpathcurveto{\pgfqpoint{2.673720in}{1.058165in}}{\pgfqpoint{2.670448in}{1.066065in}}{\pgfqpoint{2.664624in}{1.071889in}}%
\pgfpathcurveto{\pgfqpoint{2.658800in}{1.077713in}}{\pgfqpoint{2.650900in}{1.080985in}}{\pgfqpoint{2.642664in}{1.080985in}}%
\pgfpathcurveto{\pgfqpoint{2.634427in}{1.080985in}}{\pgfqpoint{2.626527in}{1.077713in}}{\pgfqpoint{2.620703in}{1.071889in}}%
\pgfpathcurveto{\pgfqpoint{2.614880in}{1.066065in}}{\pgfqpoint{2.611607in}{1.058165in}}{\pgfqpoint{2.611607in}{1.049929in}}%
\pgfpathcurveto{\pgfqpoint{2.611607in}{1.041693in}}{\pgfqpoint{2.614880in}{1.033793in}}{\pgfqpoint{2.620703in}{1.027969in}}%
\pgfpathcurveto{\pgfqpoint{2.626527in}{1.022145in}}{\pgfqpoint{2.634427in}{1.018872in}}{\pgfqpoint{2.642664in}{1.018872in}}%
\pgfpathclose%
\pgfusepath{stroke,fill}%
\end{pgfscope}%
\begin{pgfscope}%
\pgfpathrectangle{\pgfqpoint{0.100000in}{0.212622in}}{\pgfqpoint{3.696000in}{3.696000in}}%
\pgfusepath{clip}%
\pgfsetbuttcap%
\pgfsetroundjoin%
\definecolor{currentfill}{rgb}{0.121569,0.466667,0.705882}%
\pgfsetfillcolor{currentfill}%
\pgfsetfillopacity{0.962014}%
\pgfsetlinewidth{1.003750pt}%
\definecolor{currentstroke}{rgb}{0.121569,0.466667,0.705882}%
\pgfsetstrokecolor{currentstroke}%
\pgfsetstrokeopacity{0.962014}%
\pgfsetdash{}{0pt}%
\pgfpathmoveto{\pgfqpoint{2.225383in}{0.907226in}}%
\pgfpathcurveto{\pgfqpoint{2.233619in}{0.907226in}}{\pgfqpoint{2.241519in}{0.910498in}}{\pgfqpoint{2.247343in}{0.916322in}}%
\pgfpathcurveto{\pgfqpoint{2.253167in}{0.922146in}}{\pgfqpoint{2.256439in}{0.930046in}}{\pgfqpoint{2.256439in}{0.938282in}}%
\pgfpathcurveto{\pgfqpoint{2.256439in}{0.946518in}}{\pgfqpoint{2.253167in}{0.954419in}}{\pgfqpoint{2.247343in}{0.960242in}}%
\pgfpathcurveto{\pgfqpoint{2.241519in}{0.966066in}}{\pgfqpoint{2.233619in}{0.969339in}}{\pgfqpoint{2.225383in}{0.969339in}}%
\pgfpathcurveto{\pgfqpoint{2.217146in}{0.969339in}}{\pgfqpoint{2.209246in}{0.966066in}}{\pgfqpoint{2.203422in}{0.960242in}}%
\pgfpathcurveto{\pgfqpoint{2.197599in}{0.954419in}}{\pgfqpoint{2.194326in}{0.946518in}}{\pgfqpoint{2.194326in}{0.938282in}}%
\pgfpathcurveto{\pgfqpoint{2.194326in}{0.930046in}}{\pgfqpoint{2.197599in}{0.922146in}}{\pgfqpoint{2.203422in}{0.916322in}}%
\pgfpathcurveto{\pgfqpoint{2.209246in}{0.910498in}}{\pgfqpoint{2.217146in}{0.907226in}}{\pgfqpoint{2.225383in}{0.907226in}}%
\pgfpathclose%
\pgfusepath{stroke,fill}%
\end{pgfscope}%
\begin{pgfscope}%
\pgfpathrectangle{\pgfqpoint{0.100000in}{0.212622in}}{\pgfqpoint{3.696000in}{3.696000in}}%
\pgfusepath{clip}%
\pgfsetbuttcap%
\pgfsetroundjoin%
\definecolor{currentfill}{rgb}{0.121569,0.466667,0.705882}%
\pgfsetfillcolor{currentfill}%
\pgfsetfillopacity{0.963837}%
\pgfsetlinewidth{1.003750pt}%
\definecolor{currentstroke}{rgb}{0.121569,0.466667,0.705882}%
\pgfsetstrokecolor{currentstroke}%
\pgfsetstrokeopacity{0.963837}%
\pgfsetdash{}{0pt}%
\pgfpathmoveto{\pgfqpoint{2.638075in}{1.010100in}}%
\pgfpathcurveto{\pgfqpoint{2.646312in}{1.010100in}}{\pgfqpoint{2.654212in}{1.013372in}}{\pgfqpoint{2.660036in}{1.019196in}}%
\pgfpathcurveto{\pgfqpoint{2.665860in}{1.025020in}}{\pgfqpoint{2.669132in}{1.032920in}}{\pgfqpoint{2.669132in}{1.041156in}}%
\pgfpathcurveto{\pgfqpoint{2.669132in}{1.049393in}}{\pgfqpoint{2.665860in}{1.057293in}}{\pgfqpoint{2.660036in}{1.063116in}}%
\pgfpathcurveto{\pgfqpoint{2.654212in}{1.068940in}}{\pgfqpoint{2.646312in}{1.072213in}}{\pgfqpoint{2.638075in}{1.072213in}}%
\pgfpathcurveto{\pgfqpoint{2.629839in}{1.072213in}}{\pgfqpoint{2.621939in}{1.068940in}}{\pgfqpoint{2.616115in}{1.063116in}}%
\pgfpathcurveto{\pgfqpoint{2.610291in}{1.057293in}}{\pgfqpoint{2.607019in}{1.049393in}}{\pgfqpoint{2.607019in}{1.041156in}}%
\pgfpathcurveto{\pgfqpoint{2.607019in}{1.032920in}}{\pgfqpoint{2.610291in}{1.025020in}}{\pgfqpoint{2.616115in}{1.019196in}}%
\pgfpathcurveto{\pgfqpoint{2.621939in}{1.013372in}}{\pgfqpoint{2.629839in}{1.010100in}}{\pgfqpoint{2.638075in}{1.010100in}}%
\pgfpathclose%
\pgfusepath{stroke,fill}%
\end{pgfscope}%
\begin{pgfscope}%
\pgfpathrectangle{\pgfqpoint{0.100000in}{0.212622in}}{\pgfqpoint{3.696000in}{3.696000in}}%
\pgfusepath{clip}%
\pgfsetbuttcap%
\pgfsetroundjoin%
\definecolor{currentfill}{rgb}{0.121569,0.466667,0.705882}%
\pgfsetfillcolor{currentfill}%
\pgfsetfillopacity{0.964159}%
\pgfsetlinewidth{1.003750pt}%
\definecolor{currentstroke}{rgb}{0.121569,0.466667,0.705882}%
\pgfsetstrokecolor{currentstroke}%
\pgfsetstrokeopacity{0.964159}%
\pgfsetdash{}{0pt}%
\pgfpathmoveto{\pgfqpoint{2.238864in}{0.901756in}}%
\pgfpathcurveto{\pgfqpoint{2.247100in}{0.901756in}}{\pgfqpoint{2.255001in}{0.905028in}}{\pgfqpoint{2.260824in}{0.910852in}}%
\pgfpathcurveto{\pgfqpoint{2.266648in}{0.916676in}}{\pgfqpoint{2.269921in}{0.924576in}}{\pgfqpoint{2.269921in}{0.932812in}}%
\pgfpathcurveto{\pgfqpoint{2.269921in}{0.941049in}}{\pgfqpoint{2.266648in}{0.948949in}}{\pgfqpoint{2.260824in}{0.954773in}}%
\pgfpathcurveto{\pgfqpoint{2.255001in}{0.960597in}}{\pgfqpoint{2.247100in}{0.963869in}}{\pgfqpoint{2.238864in}{0.963869in}}%
\pgfpathcurveto{\pgfqpoint{2.230628in}{0.963869in}}{\pgfqpoint{2.222728in}{0.960597in}}{\pgfqpoint{2.216904in}{0.954773in}}%
\pgfpathcurveto{\pgfqpoint{2.211080in}{0.948949in}}{\pgfqpoint{2.207808in}{0.941049in}}{\pgfqpoint{2.207808in}{0.932812in}}%
\pgfpathcurveto{\pgfqpoint{2.207808in}{0.924576in}}{\pgfqpoint{2.211080in}{0.916676in}}{\pgfqpoint{2.216904in}{0.910852in}}%
\pgfpathcurveto{\pgfqpoint{2.222728in}{0.905028in}}{\pgfqpoint{2.230628in}{0.901756in}}{\pgfqpoint{2.238864in}{0.901756in}}%
\pgfpathclose%
\pgfusepath{stroke,fill}%
\end{pgfscope}%
\begin{pgfscope}%
\pgfpathrectangle{\pgfqpoint{0.100000in}{0.212622in}}{\pgfqpoint{3.696000in}{3.696000in}}%
\pgfusepath{clip}%
\pgfsetbuttcap%
\pgfsetroundjoin%
\definecolor{currentfill}{rgb}{0.121569,0.466667,0.705882}%
\pgfsetfillcolor{currentfill}%
\pgfsetfillopacity{0.964929}%
\pgfsetlinewidth{1.003750pt}%
\definecolor{currentstroke}{rgb}{0.121569,0.466667,0.705882}%
\pgfsetstrokecolor{currentstroke}%
\pgfsetstrokeopacity{0.964929}%
\pgfsetdash{}{0pt}%
\pgfpathmoveto{\pgfqpoint{2.633773in}{1.004539in}}%
\pgfpathcurveto{\pgfqpoint{2.642010in}{1.004539in}}{\pgfqpoint{2.649910in}{1.007811in}}{\pgfqpoint{2.655734in}{1.013635in}}%
\pgfpathcurveto{\pgfqpoint{2.661558in}{1.019459in}}{\pgfqpoint{2.664830in}{1.027359in}}{\pgfqpoint{2.664830in}{1.035595in}}%
\pgfpathcurveto{\pgfqpoint{2.664830in}{1.043832in}}{\pgfqpoint{2.661558in}{1.051732in}}{\pgfqpoint{2.655734in}{1.057556in}}%
\pgfpathcurveto{\pgfqpoint{2.649910in}{1.063379in}}{\pgfqpoint{2.642010in}{1.066652in}}{\pgfqpoint{2.633773in}{1.066652in}}%
\pgfpathcurveto{\pgfqpoint{2.625537in}{1.066652in}}{\pgfqpoint{2.617637in}{1.063379in}}{\pgfqpoint{2.611813in}{1.057556in}}%
\pgfpathcurveto{\pgfqpoint{2.605989in}{1.051732in}}{\pgfqpoint{2.602717in}{1.043832in}}{\pgfqpoint{2.602717in}{1.035595in}}%
\pgfpathcurveto{\pgfqpoint{2.602717in}{1.027359in}}{\pgfqpoint{2.605989in}{1.019459in}}{\pgfqpoint{2.611813in}{1.013635in}}%
\pgfpathcurveto{\pgfqpoint{2.617637in}{1.007811in}}{\pgfqpoint{2.625537in}{1.004539in}}{\pgfqpoint{2.633773in}{1.004539in}}%
\pgfpathclose%
\pgfusepath{stroke,fill}%
\end{pgfscope}%
\begin{pgfscope}%
\pgfpathrectangle{\pgfqpoint{0.100000in}{0.212622in}}{\pgfqpoint{3.696000in}{3.696000in}}%
\pgfusepath{clip}%
\pgfsetbuttcap%
\pgfsetroundjoin%
\definecolor{currentfill}{rgb}{0.121569,0.466667,0.705882}%
\pgfsetfillcolor{currentfill}%
\pgfsetfillopacity{0.965504}%
\pgfsetlinewidth{1.003750pt}%
\definecolor{currentstroke}{rgb}{0.121569,0.466667,0.705882}%
\pgfsetstrokecolor{currentstroke}%
\pgfsetstrokeopacity{0.965504}%
\pgfsetdash{}{0pt}%
\pgfpathmoveto{\pgfqpoint{2.246523in}{0.900322in}}%
\pgfpathcurveto{\pgfqpoint{2.254759in}{0.900322in}}{\pgfqpoint{2.262659in}{0.903594in}}{\pgfqpoint{2.268483in}{0.909418in}}%
\pgfpathcurveto{\pgfqpoint{2.274307in}{0.915242in}}{\pgfqpoint{2.277580in}{0.923142in}}{\pgfqpoint{2.277580in}{0.931378in}}%
\pgfpathcurveto{\pgfqpoint{2.277580in}{0.939615in}}{\pgfqpoint{2.274307in}{0.947515in}}{\pgfqpoint{2.268483in}{0.953339in}}%
\pgfpathcurveto{\pgfqpoint{2.262659in}{0.959163in}}{\pgfqpoint{2.254759in}{0.962435in}}{\pgfqpoint{2.246523in}{0.962435in}}%
\pgfpathcurveto{\pgfqpoint{2.238287in}{0.962435in}}{\pgfqpoint{2.230387in}{0.959163in}}{\pgfqpoint{2.224563in}{0.953339in}}%
\pgfpathcurveto{\pgfqpoint{2.218739in}{0.947515in}}{\pgfqpoint{2.215467in}{0.939615in}}{\pgfqpoint{2.215467in}{0.931378in}}%
\pgfpathcurveto{\pgfqpoint{2.215467in}{0.923142in}}{\pgfqpoint{2.218739in}{0.915242in}}{\pgfqpoint{2.224563in}{0.909418in}}%
\pgfpathcurveto{\pgfqpoint{2.230387in}{0.903594in}}{\pgfqpoint{2.238287in}{0.900322in}}{\pgfqpoint{2.246523in}{0.900322in}}%
\pgfpathclose%
\pgfusepath{stroke,fill}%
\end{pgfscope}%
\begin{pgfscope}%
\pgfpathrectangle{\pgfqpoint{0.100000in}{0.212622in}}{\pgfqpoint{3.696000in}{3.696000in}}%
\pgfusepath{clip}%
\pgfsetbuttcap%
\pgfsetroundjoin%
\definecolor{currentfill}{rgb}{0.121569,0.466667,0.705882}%
\pgfsetfillcolor{currentfill}%
\pgfsetfillopacity{0.965875}%
\pgfsetlinewidth{1.003750pt}%
\definecolor{currentstroke}{rgb}{0.121569,0.466667,0.705882}%
\pgfsetstrokecolor{currentstroke}%
\pgfsetstrokeopacity{0.965875}%
\pgfsetdash{}{0pt}%
\pgfpathmoveto{\pgfqpoint{2.630126in}{0.998777in}}%
\pgfpathcurveto{\pgfqpoint{2.638362in}{0.998777in}}{\pgfqpoint{2.646262in}{1.002049in}}{\pgfqpoint{2.652086in}{1.007873in}}%
\pgfpathcurveto{\pgfqpoint{2.657910in}{1.013697in}}{\pgfqpoint{2.661182in}{1.021597in}}{\pgfqpoint{2.661182in}{1.029833in}}%
\pgfpathcurveto{\pgfqpoint{2.661182in}{1.038070in}}{\pgfqpoint{2.657910in}{1.045970in}}{\pgfqpoint{2.652086in}{1.051794in}}%
\pgfpathcurveto{\pgfqpoint{2.646262in}{1.057618in}}{\pgfqpoint{2.638362in}{1.060890in}}{\pgfqpoint{2.630126in}{1.060890in}}%
\pgfpathcurveto{\pgfqpoint{2.621889in}{1.060890in}}{\pgfqpoint{2.613989in}{1.057618in}}{\pgfqpoint{2.608165in}{1.051794in}}%
\pgfpathcurveto{\pgfqpoint{2.602341in}{1.045970in}}{\pgfqpoint{2.599069in}{1.038070in}}{\pgfqpoint{2.599069in}{1.029833in}}%
\pgfpathcurveto{\pgfqpoint{2.599069in}{1.021597in}}{\pgfqpoint{2.602341in}{1.013697in}}{\pgfqpoint{2.608165in}{1.007873in}}%
\pgfpathcurveto{\pgfqpoint{2.613989in}{1.002049in}}{\pgfqpoint{2.621889in}{0.998777in}}{\pgfqpoint{2.630126in}{0.998777in}}%
\pgfpathclose%
\pgfusepath{stroke,fill}%
\end{pgfscope}%
\begin{pgfscope}%
\pgfpathrectangle{\pgfqpoint{0.100000in}{0.212622in}}{\pgfqpoint{3.696000in}{3.696000in}}%
\pgfusepath{clip}%
\pgfsetbuttcap%
\pgfsetroundjoin%
\definecolor{currentfill}{rgb}{0.121569,0.466667,0.705882}%
\pgfsetfillcolor{currentfill}%
\pgfsetfillopacity{0.966072}%
\pgfsetlinewidth{1.003750pt}%
\definecolor{currentstroke}{rgb}{0.121569,0.466667,0.705882}%
\pgfsetstrokecolor{currentstroke}%
\pgfsetstrokeopacity{0.966072}%
\pgfsetdash{}{0pt}%
\pgfpathmoveto{\pgfqpoint{2.254805in}{0.893856in}}%
\pgfpathcurveto{\pgfqpoint{2.263041in}{0.893856in}}{\pgfqpoint{2.270941in}{0.897128in}}{\pgfqpoint{2.276765in}{0.902952in}}%
\pgfpathcurveto{\pgfqpoint{2.282589in}{0.908776in}}{\pgfqpoint{2.285861in}{0.916676in}}{\pgfqpoint{2.285861in}{0.924912in}}%
\pgfpathcurveto{\pgfqpoint{2.285861in}{0.933148in}}{\pgfqpoint{2.282589in}{0.941048in}}{\pgfqpoint{2.276765in}{0.946872in}}%
\pgfpathcurveto{\pgfqpoint{2.270941in}{0.952696in}}{\pgfqpoint{2.263041in}{0.955969in}}{\pgfqpoint{2.254805in}{0.955969in}}%
\pgfpathcurveto{\pgfqpoint{2.246569in}{0.955969in}}{\pgfqpoint{2.238669in}{0.952696in}}{\pgfqpoint{2.232845in}{0.946872in}}%
\pgfpathcurveto{\pgfqpoint{2.227021in}{0.941048in}}{\pgfqpoint{2.223748in}{0.933148in}}{\pgfqpoint{2.223748in}{0.924912in}}%
\pgfpathcurveto{\pgfqpoint{2.223748in}{0.916676in}}{\pgfqpoint{2.227021in}{0.908776in}}{\pgfqpoint{2.232845in}{0.902952in}}%
\pgfpathcurveto{\pgfqpoint{2.238669in}{0.897128in}}{\pgfqpoint{2.246569in}{0.893856in}}{\pgfqpoint{2.254805in}{0.893856in}}%
\pgfpathclose%
\pgfusepath{stroke,fill}%
\end{pgfscope}%
\begin{pgfscope}%
\pgfpathrectangle{\pgfqpoint{0.100000in}{0.212622in}}{\pgfqpoint{3.696000in}{3.696000in}}%
\pgfusepath{clip}%
\pgfsetbuttcap%
\pgfsetroundjoin%
\definecolor{currentfill}{rgb}{0.121569,0.466667,0.705882}%
\pgfsetfillcolor{currentfill}%
\pgfsetfillopacity{0.966742}%
\pgfsetlinewidth{1.003750pt}%
\definecolor{currentstroke}{rgb}{0.121569,0.466667,0.705882}%
\pgfsetstrokecolor{currentstroke}%
\pgfsetstrokeopacity{0.966742}%
\pgfsetdash{}{0pt}%
\pgfpathmoveto{\pgfqpoint{2.627651in}{0.993877in}}%
\pgfpathcurveto{\pgfqpoint{2.635888in}{0.993877in}}{\pgfqpoint{2.643788in}{0.997150in}}{\pgfqpoint{2.649612in}{1.002974in}}%
\pgfpathcurveto{\pgfqpoint{2.655436in}{1.008798in}}{\pgfqpoint{2.658708in}{1.016698in}}{\pgfqpoint{2.658708in}{1.024934in}}%
\pgfpathcurveto{\pgfqpoint{2.658708in}{1.033170in}}{\pgfqpoint{2.655436in}{1.041070in}}{\pgfqpoint{2.649612in}{1.046894in}}%
\pgfpathcurveto{\pgfqpoint{2.643788in}{1.052718in}}{\pgfqpoint{2.635888in}{1.055990in}}{\pgfqpoint{2.627651in}{1.055990in}}%
\pgfpathcurveto{\pgfqpoint{2.619415in}{1.055990in}}{\pgfqpoint{2.611515in}{1.052718in}}{\pgfqpoint{2.605691in}{1.046894in}}%
\pgfpathcurveto{\pgfqpoint{2.599867in}{1.041070in}}{\pgfqpoint{2.596595in}{1.033170in}}{\pgfqpoint{2.596595in}{1.024934in}}%
\pgfpathcurveto{\pgfqpoint{2.596595in}{1.016698in}}{\pgfqpoint{2.599867in}{1.008798in}}{\pgfqpoint{2.605691in}{1.002974in}}%
\pgfpathcurveto{\pgfqpoint{2.611515in}{0.997150in}}{\pgfqpoint{2.619415in}{0.993877in}}{\pgfqpoint{2.627651in}{0.993877in}}%
\pgfpathclose%
\pgfusepath{stroke,fill}%
\end{pgfscope}%
\begin{pgfscope}%
\pgfpathrectangle{\pgfqpoint{0.100000in}{0.212622in}}{\pgfqpoint{3.696000in}{3.696000in}}%
\pgfusepath{clip}%
\pgfsetbuttcap%
\pgfsetroundjoin%
\definecolor{currentfill}{rgb}{0.121569,0.466667,0.705882}%
\pgfsetfillcolor{currentfill}%
\pgfsetfillopacity{0.966962}%
\pgfsetlinewidth{1.003750pt}%
\definecolor{currentstroke}{rgb}{0.121569,0.466667,0.705882}%
\pgfsetstrokecolor{currentstroke}%
\pgfsetstrokeopacity{0.966962}%
\pgfsetdash{}{0pt}%
\pgfpathmoveto{\pgfqpoint{2.264197in}{0.889705in}}%
\pgfpathcurveto{\pgfqpoint{2.272433in}{0.889705in}}{\pgfqpoint{2.280333in}{0.892977in}}{\pgfqpoint{2.286157in}{0.898801in}}%
\pgfpathcurveto{\pgfqpoint{2.291981in}{0.904625in}}{\pgfqpoint{2.295254in}{0.912525in}}{\pgfqpoint{2.295254in}{0.920762in}}%
\pgfpathcurveto{\pgfqpoint{2.295254in}{0.928998in}}{\pgfqpoint{2.291981in}{0.936898in}}{\pgfqpoint{2.286157in}{0.942722in}}%
\pgfpathcurveto{\pgfqpoint{2.280333in}{0.948546in}}{\pgfqpoint{2.272433in}{0.951818in}}{\pgfqpoint{2.264197in}{0.951818in}}%
\pgfpathcurveto{\pgfqpoint{2.255961in}{0.951818in}}{\pgfqpoint{2.248061in}{0.948546in}}{\pgfqpoint{2.242237in}{0.942722in}}%
\pgfpathcurveto{\pgfqpoint{2.236413in}{0.936898in}}{\pgfqpoint{2.233141in}{0.928998in}}{\pgfqpoint{2.233141in}{0.920762in}}%
\pgfpathcurveto{\pgfqpoint{2.233141in}{0.912525in}}{\pgfqpoint{2.236413in}{0.904625in}}{\pgfqpoint{2.242237in}{0.898801in}}%
\pgfpathcurveto{\pgfqpoint{2.248061in}{0.892977in}}{\pgfqpoint{2.255961in}{0.889705in}}{\pgfqpoint{2.264197in}{0.889705in}}%
\pgfpathclose%
\pgfusepath{stroke,fill}%
\end{pgfscope}%
\begin{pgfscope}%
\pgfpathrectangle{\pgfqpoint{0.100000in}{0.212622in}}{\pgfqpoint{3.696000in}{3.696000in}}%
\pgfusepath{clip}%
\pgfsetbuttcap%
\pgfsetroundjoin%
\definecolor{currentfill}{rgb}{0.121569,0.466667,0.705882}%
\pgfsetfillcolor{currentfill}%
\pgfsetfillopacity{0.967412}%
\pgfsetlinewidth{1.003750pt}%
\definecolor{currentstroke}{rgb}{0.121569,0.466667,0.705882}%
\pgfsetstrokecolor{currentstroke}%
\pgfsetstrokeopacity{0.967412}%
\pgfsetdash{}{0pt}%
\pgfpathmoveto{\pgfqpoint{2.626893in}{0.989752in}}%
\pgfpathcurveto{\pgfqpoint{2.635129in}{0.989752in}}{\pgfqpoint{2.643029in}{0.993024in}}{\pgfqpoint{2.648853in}{0.998848in}}%
\pgfpathcurveto{\pgfqpoint{2.654677in}{1.004672in}}{\pgfqpoint{2.657950in}{1.012572in}}{\pgfqpoint{2.657950in}{1.020808in}}%
\pgfpathcurveto{\pgfqpoint{2.657950in}{1.029044in}}{\pgfqpoint{2.654677in}{1.036945in}}{\pgfqpoint{2.648853in}{1.042768in}}%
\pgfpathcurveto{\pgfqpoint{2.643029in}{1.048592in}}{\pgfqpoint{2.635129in}{1.051865in}}{\pgfqpoint{2.626893in}{1.051865in}}%
\pgfpathcurveto{\pgfqpoint{2.618657in}{1.051865in}}{\pgfqpoint{2.610757in}{1.048592in}}{\pgfqpoint{2.604933in}{1.042768in}}%
\pgfpathcurveto{\pgfqpoint{2.599109in}{1.036945in}}{\pgfqpoint{2.595837in}{1.029044in}}{\pgfqpoint{2.595837in}{1.020808in}}%
\pgfpathcurveto{\pgfqpoint{2.595837in}{1.012572in}}{\pgfqpoint{2.599109in}{1.004672in}}{\pgfqpoint{2.604933in}{0.998848in}}%
\pgfpathcurveto{\pgfqpoint{2.610757in}{0.993024in}}{\pgfqpoint{2.618657in}{0.989752in}}{\pgfqpoint{2.626893in}{0.989752in}}%
\pgfpathclose%
\pgfusepath{stroke,fill}%
\end{pgfscope}%
\begin{pgfscope}%
\pgfpathrectangle{\pgfqpoint{0.100000in}{0.212622in}}{\pgfqpoint{3.696000in}{3.696000in}}%
\pgfusepath{clip}%
\pgfsetbuttcap%
\pgfsetroundjoin%
\definecolor{currentfill}{rgb}{0.121569,0.466667,0.705882}%
\pgfsetfillcolor{currentfill}%
\pgfsetfillopacity{0.967461}%
\pgfsetlinewidth{1.003750pt}%
\definecolor{currentstroke}{rgb}{0.121569,0.466667,0.705882}%
\pgfsetstrokecolor{currentstroke}%
\pgfsetstrokeopacity{0.967461}%
\pgfsetdash{}{0pt}%
\pgfpathmoveto{\pgfqpoint{2.268975in}{0.886029in}}%
\pgfpathcurveto{\pgfqpoint{2.277211in}{0.886029in}}{\pgfqpoint{2.285111in}{0.889301in}}{\pgfqpoint{2.290935in}{0.895125in}}%
\pgfpathcurveto{\pgfqpoint{2.296759in}{0.900949in}}{\pgfqpoint{2.300031in}{0.908849in}}{\pgfqpoint{2.300031in}{0.917085in}}%
\pgfpathcurveto{\pgfqpoint{2.300031in}{0.925321in}}{\pgfqpoint{2.296759in}{0.933221in}}{\pgfqpoint{2.290935in}{0.939045in}}%
\pgfpathcurveto{\pgfqpoint{2.285111in}{0.944869in}}{\pgfqpoint{2.277211in}{0.948142in}}{\pgfqpoint{2.268975in}{0.948142in}}%
\pgfpathcurveto{\pgfqpoint{2.260738in}{0.948142in}}{\pgfqpoint{2.252838in}{0.944869in}}{\pgfqpoint{2.247014in}{0.939045in}}%
\pgfpathcurveto{\pgfqpoint{2.241190in}{0.933221in}}{\pgfqpoint{2.237918in}{0.925321in}}{\pgfqpoint{2.237918in}{0.917085in}}%
\pgfpathcurveto{\pgfqpoint{2.237918in}{0.908849in}}{\pgfqpoint{2.241190in}{0.900949in}}{\pgfqpoint{2.247014in}{0.895125in}}%
\pgfpathcurveto{\pgfqpoint{2.252838in}{0.889301in}}{\pgfqpoint{2.260738in}{0.886029in}}{\pgfqpoint{2.268975in}{0.886029in}}%
\pgfpathclose%
\pgfusepath{stroke,fill}%
\end{pgfscope}%
\begin{pgfscope}%
\pgfpathrectangle{\pgfqpoint{0.100000in}{0.212622in}}{\pgfqpoint{3.696000in}{3.696000in}}%
\pgfusepath{clip}%
\pgfsetbuttcap%
\pgfsetroundjoin%
\definecolor{currentfill}{rgb}{0.121569,0.466667,0.705882}%
\pgfsetfillcolor{currentfill}%
\pgfsetfillopacity{0.968203}%
\pgfsetlinewidth{1.003750pt}%
\definecolor{currentstroke}{rgb}{0.121569,0.466667,0.705882}%
\pgfsetstrokecolor{currentstroke}%
\pgfsetstrokeopacity{0.968203}%
\pgfsetdash{}{0pt}%
\pgfpathmoveto{\pgfqpoint{2.275281in}{0.884482in}}%
\pgfpathcurveto{\pgfqpoint{2.283518in}{0.884482in}}{\pgfqpoint{2.291418in}{0.887754in}}{\pgfqpoint{2.297242in}{0.893578in}}%
\pgfpathcurveto{\pgfqpoint{2.303066in}{0.899402in}}{\pgfqpoint{2.306338in}{0.907302in}}{\pgfqpoint{2.306338in}{0.915538in}}%
\pgfpathcurveto{\pgfqpoint{2.306338in}{0.923774in}}{\pgfqpoint{2.303066in}{0.931675in}}{\pgfqpoint{2.297242in}{0.937498in}}%
\pgfpathcurveto{\pgfqpoint{2.291418in}{0.943322in}}{\pgfqpoint{2.283518in}{0.946595in}}{\pgfqpoint{2.275281in}{0.946595in}}%
\pgfpathcurveto{\pgfqpoint{2.267045in}{0.946595in}}{\pgfqpoint{2.259145in}{0.943322in}}{\pgfqpoint{2.253321in}{0.937498in}}%
\pgfpathcurveto{\pgfqpoint{2.247497in}{0.931675in}}{\pgfqpoint{2.244225in}{0.923774in}}{\pgfqpoint{2.244225in}{0.915538in}}%
\pgfpathcurveto{\pgfqpoint{2.244225in}{0.907302in}}{\pgfqpoint{2.247497in}{0.899402in}}{\pgfqpoint{2.253321in}{0.893578in}}%
\pgfpathcurveto{\pgfqpoint{2.259145in}{0.887754in}}{\pgfqpoint{2.267045in}{0.884482in}}{\pgfqpoint{2.275281in}{0.884482in}}%
\pgfpathclose%
\pgfusepath{stroke,fill}%
\end{pgfscope}%
\begin{pgfscope}%
\pgfpathrectangle{\pgfqpoint{0.100000in}{0.212622in}}{\pgfqpoint{3.696000in}{3.696000in}}%
\pgfusepath{clip}%
\pgfsetbuttcap%
\pgfsetroundjoin%
\definecolor{currentfill}{rgb}{0.121569,0.466667,0.705882}%
\pgfsetfillcolor{currentfill}%
\pgfsetfillopacity{0.968574}%
\pgfsetlinewidth{1.003750pt}%
\definecolor{currentstroke}{rgb}{0.121569,0.466667,0.705882}%
\pgfsetstrokecolor{currentstroke}%
\pgfsetstrokeopacity{0.968574}%
\pgfsetdash{}{0pt}%
\pgfpathmoveto{\pgfqpoint{2.625708in}{0.982015in}}%
\pgfpathcurveto{\pgfqpoint{2.633944in}{0.982015in}}{\pgfqpoint{2.641845in}{0.985288in}}{\pgfqpoint{2.647668in}{0.991112in}}%
\pgfpathcurveto{\pgfqpoint{2.653492in}{0.996936in}}{\pgfqpoint{2.656765in}{1.004836in}}{\pgfqpoint{2.656765in}{1.013072in}}%
\pgfpathcurveto{\pgfqpoint{2.656765in}{1.021308in}}{\pgfqpoint{2.653492in}{1.029208in}}{\pgfqpoint{2.647668in}{1.035032in}}%
\pgfpathcurveto{\pgfqpoint{2.641845in}{1.040856in}}{\pgfqpoint{2.633944in}{1.044128in}}{\pgfqpoint{2.625708in}{1.044128in}}%
\pgfpathcurveto{\pgfqpoint{2.617472in}{1.044128in}}{\pgfqpoint{2.609572in}{1.040856in}}{\pgfqpoint{2.603748in}{1.035032in}}%
\pgfpathcurveto{\pgfqpoint{2.597924in}{1.029208in}}{\pgfqpoint{2.594652in}{1.021308in}}{\pgfqpoint{2.594652in}{1.013072in}}%
\pgfpathcurveto{\pgfqpoint{2.594652in}{1.004836in}}{\pgfqpoint{2.597924in}{0.996936in}}{\pgfqpoint{2.603748in}{0.991112in}}%
\pgfpathcurveto{\pgfqpoint{2.609572in}{0.985288in}}{\pgfqpoint{2.617472in}{0.982015in}}{\pgfqpoint{2.625708in}{0.982015in}}%
\pgfpathclose%
\pgfusepath{stroke,fill}%
\end{pgfscope}%
\begin{pgfscope}%
\pgfpathrectangle{\pgfqpoint{0.100000in}{0.212622in}}{\pgfqpoint{3.696000in}{3.696000in}}%
\pgfusepath{clip}%
\pgfsetbuttcap%
\pgfsetroundjoin%
\definecolor{currentfill}{rgb}{0.121569,0.466667,0.705882}%
\pgfsetfillcolor{currentfill}%
\pgfsetfillopacity{0.968826}%
\pgfsetlinewidth{1.003750pt}%
\definecolor{currentstroke}{rgb}{0.121569,0.466667,0.705882}%
\pgfsetstrokecolor{currentstroke}%
\pgfsetstrokeopacity{0.968826}%
\pgfsetdash{}{0pt}%
\pgfpathmoveto{\pgfqpoint{2.278743in}{0.884478in}}%
\pgfpathcurveto{\pgfqpoint{2.286979in}{0.884478in}}{\pgfqpoint{2.294879in}{0.887751in}}{\pgfqpoint{2.300703in}{0.893575in}}%
\pgfpathcurveto{\pgfqpoint{2.306527in}{0.899399in}}{\pgfqpoint{2.309799in}{0.907299in}}{\pgfqpoint{2.309799in}{0.915535in}}%
\pgfpathcurveto{\pgfqpoint{2.309799in}{0.923771in}}{\pgfqpoint{2.306527in}{0.931671in}}{\pgfqpoint{2.300703in}{0.937495in}}%
\pgfpathcurveto{\pgfqpoint{2.294879in}{0.943319in}}{\pgfqpoint{2.286979in}{0.946591in}}{\pgfqpoint{2.278743in}{0.946591in}}%
\pgfpathcurveto{\pgfqpoint{2.270507in}{0.946591in}}{\pgfqpoint{2.262607in}{0.943319in}}{\pgfqpoint{2.256783in}{0.937495in}}%
\pgfpathcurveto{\pgfqpoint{2.250959in}{0.931671in}}{\pgfqpoint{2.247686in}{0.923771in}}{\pgfqpoint{2.247686in}{0.915535in}}%
\pgfpathcurveto{\pgfqpoint{2.247686in}{0.907299in}}{\pgfqpoint{2.250959in}{0.899399in}}{\pgfqpoint{2.256783in}{0.893575in}}%
\pgfpathcurveto{\pgfqpoint{2.262607in}{0.887751in}}{\pgfqpoint{2.270507in}{0.884478in}}{\pgfqpoint{2.278743in}{0.884478in}}%
\pgfpathclose%
\pgfusepath{stroke,fill}%
\end{pgfscope}%
\begin{pgfscope}%
\pgfpathrectangle{\pgfqpoint{0.100000in}{0.212622in}}{\pgfqpoint{3.696000in}{3.696000in}}%
\pgfusepath{clip}%
\pgfsetbuttcap%
\pgfsetroundjoin%
\definecolor{currentfill}{rgb}{0.121569,0.466667,0.705882}%
\pgfsetfillcolor{currentfill}%
\pgfsetfillopacity{0.969570}%
\pgfsetlinewidth{1.003750pt}%
\definecolor{currentstroke}{rgb}{0.121569,0.466667,0.705882}%
\pgfsetstrokecolor{currentstroke}%
\pgfsetstrokeopacity{0.969570}%
\pgfsetdash{}{0pt}%
\pgfpathmoveto{\pgfqpoint{2.282980in}{0.884885in}}%
\pgfpathcurveto{\pgfqpoint{2.291216in}{0.884885in}}{\pgfqpoint{2.299116in}{0.888157in}}{\pgfqpoint{2.304940in}{0.893981in}}%
\pgfpathcurveto{\pgfqpoint{2.310764in}{0.899805in}}{\pgfqpoint{2.314037in}{0.907705in}}{\pgfqpoint{2.314037in}{0.915941in}}%
\pgfpathcurveto{\pgfqpoint{2.314037in}{0.924177in}}{\pgfqpoint{2.310764in}{0.932077in}}{\pgfqpoint{2.304940in}{0.937901in}}%
\pgfpathcurveto{\pgfqpoint{2.299116in}{0.943725in}}{\pgfqpoint{2.291216in}{0.946998in}}{\pgfqpoint{2.282980in}{0.946998in}}%
\pgfpathcurveto{\pgfqpoint{2.274744in}{0.946998in}}{\pgfqpoint{2.266844in}{0.943725in}}{\pgfqpoint{2.261020in}{0.937901in}}%
\pgfpathcurveto{\pgfqpoint{2.255196in}{0.932077in}}{\pgfqpoint{2.251924in}{0.924177in}}{\pgfqpoint{2.251924in}{0.915941in}}%
\pgfpathcurveto{\pgfqpoint{2.251924in}{0.907705in}}{\pgfqpoint{2.255196in}{0.899805in}}{\pgfqpoint{2.261020in}{0.893981in}}%
\pgfpathcurveto{\pgfqpoint{2.266844in}{0.888157in}}{\pgfqpoint{2.274744in}{0.884885in}}{\pgfqpoint{2.282980in}{0.884885in}}%
\pgfpathclose%
\pgfusepath{stroke,fill}%
\end{pgfscope}%
\begin{pgfscope}%
\pgfpathrectangle{\pgfqpoint{0.100000in}{0.212622in}}{\pgfqpoint{3.696000in}{3.696000in}}%
\pgfusepath{clip}%
\pgfsetbuttcap%
\pgfsetroundjoin%
\definecolor{currentfill}{rgb}{0.121569,0.466667,0.705882}%
\pgfsetfillcolor{currentfill}%
\pgfsetfillopacity{0.969683}%
\pgfsetlinewidth{1.003750pt}%
\definecolor{currentstroke}{rgb}{0.121569,0.466667,0.705882}%
\pgfsetstrokecolor{currentstroke}%
\pgfsetstrokeopacity{0.969683}%
\pgfsetdash{}{0pt}%
\pgfpathmoveto{\pgfqpoint{2.623030in}{0.976079in}}%
\pgfpathcurveto{\pgfqpoint{2.631267in}{0.976079in}}{\pgfqpoint{2.639167in}{0.979352in}}{\pgfqpoint{2.644991in}{0.985176in}}%
\pgfpathcurveto{\pgfqpoint{2.650815in}{0.991000in}}{\pgfqpoint{2.654087in}{0.998900in}}{\pgfqpoint{2.654087in}{1.007136in}}%
\pgfpathcurveto{\pgfqpoint{2.654087in}{1.015372in}}{\pgfqpoint{2.650815in}{1.023272in}}{\pgfqpoint{2.644991in}{1.029096in}}%
\pgfpathcurveto{\pgfqpoint{2.639167in}{1.034920in}}{\pgfqpoint{2.631267in}{1.038192in}}{\pgfqpoint{2.623030in}{1.038192in}}%
\pgfpathcurveto{\pgfqpoint{2.614794in}{1.038192in}}{\pgfqpoint{2.606894in}{1.034920in}}{\pgfqpoint{2.601070in}{1.029096in}}%
\pgfpathcurveto{\pgfqpoint{2.595246in}{1.023272in}}{\pgfqpoint{2.591974in}{1.015372in}}{\pgfqpoint{2.591974in}{1.007136in}}%
\pgfpathcurveto{\pgfqpoint{2.591974in}{0.998900in}}{\pgfqpoint{2.595246in}{0.991000in}}{\pgfqpoint{2.601070in}{0.985176in}}%
\pgfpathcurveto{\pgfqpoint{2.606894in}{0.979352in}}{\pgfqpoint{2.614794in}{0.976079in}}{\pgfqpoint{2.623030in}{0.976079in}}%
\pgfpathclose%
\pgfusepath{stroke,fill}%
\end{pgfscope}%
\begin{pgfscope}%
\pgfpathrectangle{\pgfqpoint{0.100000in}{0.212622in}}{\pgfqpoint{3.696000in}{3.696000in}}%
\pgfusepath{clip}%
\pgfsetbuttcap%
\pgfsetroundjoin%
\definecolor{currentfill}{rgb}{0.121569,0.466667,0.705882}%
\pgfsetfillcolor{currentfill}%
\pgfsetfillopacity{0.970246}%
\pgfsetlinewidth{1.003750pt}%
\definecolor{currentstroke}{rgb}{0.121569,0.466667,0.705882}%
\pgfsetstrokecolor{currentstroke}%
\pgfsetstrokeopacity{0.970246}%
\pgfsetdash{}{0pt}%
\pgfpathmoveto{\pgfqpoint{2.288217in}{0.883233in}}%
\pgfpathcurveto{\pgfqpoint{2.296453in}{0.883233in}}{\pgfqpoint{2.304353in}{0.886505in}}{\pgfqpoint{2.310177in}{0.892329in}}%
\pgfpathcurveto{\pgfqpoint{2.316001in}{0.898153in}}{\pgfqpoint{2.319273in}{0.906053in}}{\pgfqpoint{2.319273in}{0.914289in}}%
\pgfpathcurveto{\pgfqpoint{2.319273in}{0.922526in}}{\pgfqpoint{2.316001in}{0.930426in}}{\pgfqpoint{2.310177in}{0.936250in}}%
\pgfpathcurveto{\pgfqpoint{2.304353in}{0.942074in}}{\pgfqpoint{2.296453in}{0.945346in}}{\pgfqpoint{2.288217in}{0.945346in}}%
\pgfpathcurveto{\pgfqpoint{2.279981in}{0.945346in}}{\pgfqpoint{2.272081in}{0.942074in}}{\pgfqpoint{2.266257in}{0.936250in}}%
\pgfpathcurveto{\pgfqpoint{2.260433in}{0.930426in}}{\pgfqpoint{2.257160in}{0.922526in}}{\pgfqpoint{2.257160in}{0.914289in}}%
\pgfpathcurveto{\pgfqpoint{2.257160in}{0.906053in}}{\pgfqpoint{2.260433in}{0.898153in}}{\pgfqpoint{2.266257in}{0.892329in}}%
\pgfpathcurveto{\pgfqpoint{2.272081in}{0.886505in}}{\pgfqpoint{2.279981in}{0.883233in}}{\pgfqpoint{2.288217in}{0.883233in}}%
\pgfpathclose%
\pgfusepath{stroke,fill}%
\end{pgfscope}%
\begin{pgfscope}%
\pgfpathrectangle{\pgfqpoint{0.100000in}{0.212622in}}{\pgfqpoint{3.696000in}{3.696000in}}%
\pgfusepath{clip}%
\pgfsetbuttcap%
\pgfsetroundjoin%
\definecolor{currentfill}{rgb}{0.121569,0.466667,0.705882}%
\pgfsetfillcolor{currentfill}%
\pgfsetfillopacity{0.970436}%
\pgfsetlinewidth{1.003750pt}%
\definecolor{currentstroke}{rgb}{0.121569,0.466667,0.705882}%
\pgfsetstrokecolor{currentstroke}%
\pgfsetstrokeopacity{0.970436}%
\pgfsetdash{}{0pt}%
\pgfpathmoveto{\pgfqpoint{2.620165in}{0.972499in}}%
\pgfpathcurveto{\pgfqpoint{2.628401in}{0.972499in}}{\pgfqpoint{2.636301in}{0.975772in}}{\pgfqpoint{2.642125in}{0.981596in}}%
\pgfpathcurveto{\pgfqpoint{2.647949in}{0.987419in}}{\pgfqpoint{2.651221in}{0.995320in}}{\pgfqpoint{2.651221in}{1.003556in}}%
\pgfpathcurveto{\pgfqpoint{2.651221in}{1.011792in}}{\pgfqpoint{2.647949in}{1.019692in}}{\pgfqpoint{2.642125in}{1.025516in}}%
\pgfpathcurveto{\pgfqpoint{2.636301in}{1.031340in}}{\pgfqpoint{2.628401in}{1.034612in}}{\pgfqpoint{2.620165in}{1.034612in}}%
\pgfpathcurveto{\pgfqpoint{2.611929in}{1.034612in}}{\pgfqpoint{2.604029in}{1.031340in}}{\pgfqpoint{2.598205in}{1.025516in}}%
\pgfpathcurveto{\pgfqpoint{2.592381in}{1.019692in}}{\pgfqpoint{2.589108in}{1.011792in}}{\pgfqpoint{2.589108in}{1.003556in}}%
\pgfpathcurveto{\pgfqpoint{2.589108in}{0.995320in}}{\pgfqpoint{2.592381in}{0.987419in}}{\pgfqpoint{2.598205in}{0.981596in}}%
\pgfpathcurveto{\pgfqpoint{2.604029in}{0.975772in}}{\pgfqpoint{2.611929in}{0.972499in}}{\pgfqpoint{2.620165in}{0.972499in}}%
\pgfpathclose%
\pgfusepath{stroke,fill}%
\end{pgfscope}%
\begin{pgfscope}%
\pgfpathrectangle{\pgfqpoint{0.100000in}{0.212622in}}{\pgfqpoint{3.696000in}{3.696000in}}%
\pgfusepath{clip}%
\pgfsetbuttcap%
\pgfsetroundjoin%
\definecolor{currentfill}{rgb}{0.121569,0.466667,0.705882}%
\pgfsetfillcolor{currentfill}%
\pgfsetfillopacity{0.970522}%
\pgfsetlinewidth{1.003750pt}%
\definecolor{currentstroke}{rgb}{0.121569,0.466667,0.705882}%
\pgfsetstrokecolor{currentstroke}%
\pgfsetstrokeopacity{0.970522}%
\pgfsetdash{}{0pt}%
\pgfpathmoveto{\pgfqpoint{2.290747in}{0.880710in}}%
\pgfpathcurveto{\pgfqpoint{2.298984in}{0.880710in}}{\pgfqpoint{2.306884in}{0.883983in}}{\pgfqpoint{2.312708in}{0.889807in}}%
\pgfpathcurveto{\pgfqpoint{2.318532in}{0.895631in}}{\pgfqpoint{2.321804in}{0.903531in}}{\pgfqpoint{2.321804in}{0.911767in}}%
\pgfpathcurveto{\pgfqpoint{2.321804in}{0.920003in}}{\pgfqpoint{2.318532in}{0.927903in}}{\pgfqpoint{2.312708in}{0.933727in}}%
\pgfpathcurveto{\pgfqpoint{2.306884in}{0.939551in}}{\pgfqpoint{2.298984in}{0.942823in}}{\pgfqpoint{2.290747in}{0.942823in}}%
\pgfpathcurveto{\pgfqpoint{2.282511in}{0.942823in}}{\pgfqpoint{2.274611in}{0.939551in}}{\pgfqpoint{2.268787in}{0.933727in}}%
\pgfpathcurveto{\pgfqpoint{2.262963in}{0.927903in}}{\pgfqpoint{2.259691in}{0.920003in}}{\pgfqpoint{2.259691in}{0.911767in}}%
\pgfpathcurveto{\pgfqpoint{2.259691in}{0.903531in}}{\pgfqpoint{2.262963in}{0.895631in}}{\pgfqpoint{2.268787in}{0.889807in}}%
\pgfpathcurveto{\pgfqpoint{2.274611in}{0.883983in}}{\pgfqpoint{2.282511in}{0.880710in}}{\pgfqpoint{2.290747in}{0.880710in}}%
\pgfpathclose%
\pgfusepath{stroke,fill}%
\end{pgfscope}%
\begin{pgfscope}%
\pgfpathrectangle{\pgfqpoint{0.100000in}{0.212622in}}{\pgfqpoint{3.696000in}{3.696000in}}%
\pgfusepath{clip}%
\pgfsetbuttcap%
\pgfsetroundjoin%
\definecolor{currentfill}{rgb}{0.121569,0.466667,0.705882}%
\pgfsetfillcolor{currentfill}%
\pgfsetfillopacity{0.970885}%
\pgfsetlinewidth{1.003750pt}%
\definecolor{currentstroke}{rgb}{0.121569,0.466667,0.705882}%
\pgfsetstrokecolor{currentstroke}%
\pgfsetstrokeopacity{0.970885}%
\pgfsetdash{}{0pt}%
\pgfpathmoveto{\pgfqpoint{2.294328in}{0.877343in}}%
\pgfpathcurveto{\pgfqpoint{2.302564in}{0.877343in}}{\pgfqpoint{2.310464in}{0.880615in}}{\pgfqpoint{2.316288in}{0.886439in}}%
\pgfpathcurveto{\pgfqpoint{2.322112in}{0.892263in}}{\pgfqpoint{2.325384in}{0.900163in}}{\pgfqpoint{2.325384in}{0.908399in}}%
\pgfpathcurveto{\pgfqpoint{2.325384in}{0.916636in}}{\pgfqpoint{2.322112in}{0.924536in}}{\pgfqpoint{2.316288in}{0.930360in}}%
\pgfpathcurveto{\pgfqpoint{2.310464in}{0.936184in}}{\pgfqpoint{2.302564in}{0.939456in}}{\pgfqpoint{2.294328in}{0.939456in}}%
\pgfpathcurveto{\pgfqpoint{2.286091in}{0.939456in}}{\pgfqpoint{2.278191in}{0.936184in}}{\pgfqpoint{2.272368in}{0.930360in}}%
\pgfpathcurveto{\pgfqpoint{2.266544in}{0.924536in}}{\pgfqpoint{2.263271in}{0.916636in}}{\pgfqpoint{2.263271in}{0.908399in}}%
\pgfpathcurveto{\pgfqpoint{2.263271in}{0.900163in}}{\pgfqpoint{2.266544in}{0.892263in}}{\pgfqpoint{2.272368in}{0.886439in}}%
\pgfpathcurveto{\pgfqpoint{2.278191in}{0.880615in}}{\pgfqpoint{2.286091in}{0.877343in}}{\pgfqpoint{2.294328in}{0.877343in}}%
\pgfpathclose%
\pgfusepath{stroke,fill}%
\end{pgfscope}%
\begin{pgfscope}%
\pgfpathrectangle{\pgfqpoint{0.100000in}{0.212622in}}{\pgfqpoint{3.696000in}{3.696000in}}%
\pgfusepath{clip}%
\pgfsetbuttcap%
\pgfsetroundjoin%
\definecolor{currentfill}{rgb}{0.121569,0.466667,0.705882}%
\pgfsetfillcolor{currentfill}%
\pgfsetfillopacity{0.970982}%
\pgfsetlinewidth{1.003750pt}%
\definecolor{currentstroke}{rgb}{0.121569,0.466667,0.705882}%
\pgfsetstrokecolor{currentstroke}%
\pgfsetstrokeopacity{0.970982}%
\pgfsetdash{}{0pt}%
\pgfpathmoveto{\pgfqpoint{2.617680in}{0.969413in}}%
\pgfpathcurveto{\pgfqpoint{2.625916in}{0.969413in}}{\pgfqpoint{2.633816in}{0.972685in}}{\pgfqpoint{2.639640in}{0.978509in}}%
\pgfpathcurveto{\pgfqpoint{2.645464in}{0.984333in}}{\pgfqpoint{2.648736in}{0.992233in}}{\pgfqpoint{2.648736in}{1.000469in}}%
\pgfpathcurveto{\pgfqpoint{2.648736in}{1.008705in}}{\pgfqpoint{2.645464in}{1.016605in}}{\pgfqpoint{2.639640in}{1.022429in}}%
\pgfpathcurveto{\pgfqpoint{2.633816in}{1.028253in}}{\pgfqpoint{2.625916in}{1.031526in}}{\pgfqpoint{2.617680in}{1.031526in}}%
\pgfpathcurveto{\pgfqpoint{2.609443in}{1.031526in}}{\pgfqpoint{2.601543in}{1.028253in}}{\pgfqpoint{2.595719in}{1.022429in}}%
\pgfpathcurveto{\pgfqpoint{2.589895in}{1.016605in}}{\pgfqpoint{2.586623in}{1.008705in}}{\pgfqpoint{2.586623in}{1.000469in}}%
\pgfpathcurveto{\pgfqpoint{2.586623in}{0.992233in}}{\pgfqpoint{2.589895in}{0.984333in}}{\pgfqpoint{2.595719in}{0.978509in}}%
\pgfpathcurveto{\pgfqpoint{2.601543in}{0.972685in}}{\pgfqpoint{2.609443in}{0.969413in}}{\pgfqpoint{2.617680in}{0.969413in}}%
\pgfpathclose%
\pgfusepath{stroke,fill}%
\end{pgfscope}%
\begin{pgfscope}%
\pgfpathrectangle{\pgfqpoint{0.100000in}{0.212622in}}{\pgfqpoint{3.696000in}{3.696000in}}%
\pgfusepath{clip}%
\pgfsetbuttcap%
\pgfsetroundjoin%
\definecolor{currentfill}{rgb}{0.121569,0.466667,0.705882}%
\pgfsetfillcolor{currentfill}%
\pgfsetfillopacity{0.971107}%
\pgfsetlinewidth{1.003750pt}%
\definecolor{currentstroke}{rgb}{0.121569,0.466667,0.705882}%
\pgfsetstrokecolor{currentstroke}%
\pgfsetstrokeopacity{0.971107}%
\pgfsetdash{}{0pt}%
\pgfpathmoveto{\pgfqpoint{2.296441in}{0.876071in}}%
\pgfpathcurveto{\pgfqpoint{2.304678in}{0.876071in}}{\pgfqpoint{2.312578in}{0.879343in}}{\pgfqpoint{2.318402in}{0.885167in}}%
\pgfpathcurveto{\pgfqpoint{2.324226in}{0.890991in}}{\pgfqpoint{2.327498in}{0.898891in}}{\pgfqpoint{2.327498in}{0.907127in}}%
\pgfpathcurveto{\pgfqpoint{2.327498in}{0.915363in}}{\pgfqpoint{2.324226in}{0.923263in}}{\pgfqpoint{2.318402in}{0.929087in}}%
\pgfpathcurveto{\pgfqpoint{2.312578in}{0.934911in}}{\pgfqpoint{2.304678in}{0.938184in}}{\pgfqpoint{2.296441in}{0.938184in}}%
\pgfpathcurveto{\pgfqpoint{2.288205in}{0.938184in}}{\pgfqpoint{2.280305in}{0.934911in}}{\pgfqpoint{2.274481in}{0.929087in}}%
\pgfpathcurveto{\pgfqpoint{2.268657in}{0.923263in}}{\pgfqpoint{2.265385in}{0.915363in}}{\pgfqpoint{2.265385in}{0.907127in}}%
\pgfpathcurveto{\pgfqpoint{2.265385in}{0.898891in}}{\pgfqpoint{2.268657in}{0.890991in}}{\pgfqpoint{2.274481in}{0.885167in}}%
\pgfpathcurveto{\pgfqpoint{2.280305in}{0.879343in}}{\pgfqpoint{2.288205in}{0.876071in}}{\pgfqpoint{2.296441in}{0.876071in}}%
\pgfpathclose%
\pgfusepath{stroke,fill}%
\end{pgfscope}%
\begin{pgfscope}%
\pgfpathrectangle{\pgfqpoint{0.100000in}{0.212622in}}{\pgfqpoint{3.696000in}{3.696000in}}%
\pgfusepath{clip}%
\pgfsetbuttcap%
\pgfsetroundjoin%
\definecolor{currentfill}{rgb}{0.121569,0.466667,0.705882}%
\pgfsetfillcolor{currentfill}%
\pgfsetfillopacity{0.971477}%
\pgfsetlinewidth{1.003750pt}%
\definecolor{currentstroke}{rgb}{0.121569,0.466667,0.705882}%
\pgfsetstrokecolor{currentstroke}%
\pgfsetstrokeopacity{0.971477}%
\pgfsetdash{}{0pt}%
\pgfpathmoveto{\pgfqpoint{2.299082in}{0.875516in}}%
\pgfpathcurveto{\pgfqpoint{2.307318in}{0.875516in}}{\pgfqpoint{2.315218in}{0.878789in}}{\pgfqpoint{2.321042in}{0.884613in}}%
\pgfpathcurveto{\pgfqpoint{2.326866in}{0.890437in}}{\pgfqpoint{2.330138in}{0.898337in}}{\pgfqpoint{2.330138in}{0.906573in}}%
\pgfpathcurveto{\pgfqpoint{2.330138in}{0.914809in}}{\pgfqpoint{2.326866in}{0.922709in}}{\pgfqpoint{2.321042in}{0.928533in}}%
\pgfpathcurveto{\pgfqpoint{2.315218in}{0.934357in}}{\pgfqpoint{2.307318in}{0.937629in}}{\pgfqpoint{2.299082in}{0.937629in}}%
\pgfpathcurveto{\pgfqpoint{2.290846in}{0.937629in}}{\pgfqpoint{2.282946in}{0.934357in}}{\pgfqpoint{2.277122in}{0.928533in}}%
\pgfpathcurveto{\pgfqpoint{2.271298in}{0.922709in}}{\pgfqpoint{2.268025in}{0.914809in}}{\pgfqpoint{2.268025in}{0.906573in}}%
\pgfpathcurveto{\pgfqpoint{2.268025in}{0.898337in}}{\pgfqpoint{2.271298in}{0.890437in}}{\pgfqpoint{2.277122in}{0.884613in}}%
\pgfpathcurveto{\pgfqpoint{2.282946in}{0.878789in}}{\pgfqpoint{2.290846in}{0.875516in}}{\pgfqpoint{2.299082in}{0.875516in}}%
\pgfpathclose%
\pgfusepath{stroke,fill}%
\end{pgfscope}%
\begin{pgfscope}%
\pgfpathrectangle{\pgfqpoint{0.100000in}{0.212622in}}{\pgfqpoint{3.696000in}{3.696000in}}%
\pgfusepath{clip}%
\pgfsetbuttcap%
\pgfsetroundjoin%
\definecolor{currentfill}{rgb}{0.121569,0.466667,0.705882}%
\pgfsetfillcolor{currentfill}%
\pgfsetfillopacity{0.971957}%
\pgfsetlinewidth{1.003750pt}%
\definecolor{currentstroke}{rgb}{0.121569,0.466667,0.705882}%
\pgfsetstrokecolor{currentstroke}%
\pgfsetstrokeopacity{0.971957}%
\pgfsetdash{}{0pt}%
\pgfpathmoveto{\pgfqpoint{2.613384in}{0.963159in}}%
\pgfpathcurveto{\pgfqpoint{2.621620in}{0.963159in}}{\pgfqpoint{2.629520in}{0.966431in}}{\pgfqpoint{2.635344in}{0.972255in}}%
\pgfpathcurveto{\pgfqpoint{2.641168in}{0.978079in}}{\pgfqpoint{2.644441in}{0.985979in}}{\pgfqpoint{2.644441in}{0.994215in}}%
\pgfpathcurveto{\pgfqpoint{2.644441in}{1.002451in}}{\pgfqpoint{2.641168in}{1.010351in}}{\pgfqpoint{2.635344in}{1.016175in}}%
\pgfpathcurveto{\pgfqpoint{2.629520in}{1.021999in}}{\pgfqpoint{2.621620in}{1.025272in}}{\pgfqpoint{2.613384in}{1.025272in}}%
\pgfpathcurveto{\pgfqpoint{2.605148in}{1.025272in}}{\pgfqpoint{2.597248in}{1.021999in}}{\pgfqpoint{2.591424in}{1.016175in}}%
\pgfpathcurveto{\pgfqpoint{2.585600in}{1.010351in}}{\pgfqpoint{2.582328in}{1.002451in}}{\pgfqpoint{2.582328in}{0.994215in}}%
\pgfpathcurveto{\pgfqpoint{2.582328in}{0.985979in}}{\pgfqpoint{2.585600in}{0.978079in}}{\pgfqpoint{2.591424in}{0.972255in}}%
\pgfpathcurveto{\pgfqpoint{2.597248in}{0.966431in}}{\pgfqpoint{2.605148in}{0.963159in}}{\pgfqpoint{2.613384in}{0.963159in}}%
\pgfpathclose%
\pgfusepath{stroke,fill}%
\end{pgfscope}%
\begin{pgfscope}%
\pgfpathrectangle{\pgfqpoint{0.100000in}{0.212622in}}{\pgfqpoint{3.696000in}{3.696000in}}%
\pgfusepath{clip}%
\pgfsetbuttcap%
\pgfsetroundjoin%
\definecolor{currentfill}{rgb}{0.121569,0.466667,0.705882}%
\pgfsetfillcolor{currentfill}%
\pgfsetfillopacity{0.971991}%
\pgfsetlinewidth{1.003750pt}%
\definecolor{currentstroke}{rgb}{0.121569,0.466667,0.705882}%
\pgfsetstrokecolor{currentstroke}%
\pgfsetstrokeopacity{0.971991}%
\pgfsetdash{}{0pt}%
\pgfpathmoveto{\pgfqpoint{2.302244in}{0.874878in}}%
\pgfpathcurveto{\pgfqpoint{2.310480in}{0.874878in}}{\pgfqpoint{2.318380in}{0.878150in}}{\pgfqpoint{2.324204in}{0.883974in}}%
\pgfpathcurveto{\pgfqpoint{2.330028in}{0.889798in}}{\pgfqpoint{2.333301in}{0.897698in}}{\pgfqpoint{2.333301in}{0.905934in}}%
\pgfpathcurveto{\pgfqpoint{2.333301in}{0.914171in}}{\pgfqpoint{2.330028in}{0.922071in}}{\pgfqpoint{2.324204in}{0.927895in}}%
\pgfpathcurveto{\pgfqpoint{2.318380in}{0.933719in}}{\pgfqpoint{2.310480in}{0.936991in}}{\pgfqpoint{2.302244in}{0.936991in}}%
\pgfpathcurveto{\pgfqpoint{2.294008in}{0.936991in}}{\pgfqpoint{2.286108in}{0.933719in}}{\pgfqpoint{2.280284in}{0.927895in}}%
\pgfpathcurveto{\pgfqpoint{2.274460in}{0.922071in}}{\pgfqpoint{2.271188in}{0.914171in}}{\pgfqpoint{2.271188in}{0.905934in}}%
\pgfpathcurveto{\pgfqpoint{2.271188in}{0.897698in}}{\pgfqpoint{2.274460in}{0.889798in}}{\pgfqpoint{2.280284in}{0.883974in}}%
\pgfpathcurveto{\pgfqpoint{2.286108in}{0.878150in}}{\pgfqpoint{2.294008in}{0.874878in}}{\pgfqpoint{2.302244in}{0.874878in}}%
\pgfpathclose%
\pgfusepath{stroke,fill}%
\end{pgfscope}%
\begin{pgfscope}%
\pgfpathrectangle{\pgfqpoint{0.100000in}{0.212622in}}{\pgfqpoint{3.696000in}{3.696000in}}%
\pgfusepath{clip}%
\pgfsetbuttcap%
\pgfsetroundjoin%
\definecolor{currentfill}{rgb}{0.121569,0.466667,0.705882}%
\pgfsetfillcolor{currentfill}%
\pgfsetfillopacity{0.973031}%
\pgfsetlinewidth{1.003750pt}%
\definecolor{currentstroke}{rgb}{0.121569,0.466667,0.705882}%
\pgfsetstrokecolor{currentstroke}%
\pgfsetstrokeopacity{0.973031}%
\pgfsetdash{}{0pt}%
\pgfpathmoveto{\pgfqpoint{2.308160in}{0.873762in}}%
\pgfpathcurveto{\pgfqpoint{2.316397in}{0.873762in}}{\pgfqpoint{2.324297in}{0.877034in}}{\pgfqpoint{2.330121in}{0.882858in}}%
\pgfpathcurveto{\pgfqpoint{2.335945in}{0.888682in}}{\pgfqpoint{2.339217in}{0.896582in}}{\pgfqpoint{2.339217in}{0.904818in}}%
\pgfpathcurveto{\pgfqpoint{2.339217in}{0.913055in}}{\pgfqpoint{2.335945in}{0.920955in}}{\pgfqpoint{2.330121in}{0.926779in}}%
\pgfpathcurveto{\pgfqpoint{2.324297in}{0.932603in}}{\pgfqpoint{2.316397in}{0.935875in}}{\pgfqpoint{2.308160in}{0.935875in}}%
\pgfpathcurveto{\pgfqpoint{2.299924in}{0.935875in}}{\pgfqpoint{2.292024in}{0.932603in}}{\pgfqpoint{2.286200in}{0.926779in}}%
\pgfpathcurveto{\pgfqpoint{2.280376in}{0.920955in}}{\pgfqpoint{2.277104in}{0.913055in}}{\pgfqpoint{2.277104in}{0.904818in}}%
\pgfpathcurveto{\pgfqpoint{2.277104in}{0.896582in}}{\pgfqpoint{2.280376in}{0.888682in}}{\pgfqpoint{2.286200in}{0.882858in}}%
\pgfpathcurveto{\pgfqpoint{2.292024in}{0.877034in}}{\pgfqpoint{2.299924in}{0.873762in}}{\pgfqpoint{2.308160in}{0.873762in}}%
\pgfpathclose%
\pgfusepath{stroke,fill}%
\end{pgfscope}%
\begin{pgfscope}%
\pgfpathrectangle{\pgfqpoint{0.100000in}{0.212622in}}{\pgfqpoint{3.696000in}{3.696000in}}%
\pgfusepath{clip}%
\pgfsetbuttcap%
\pgfsetroundjoin%
\definecolor{currentfill}{rgb}{0.121569,0.466667,0.705882}%
\pgfsetfillcolor{currentfill}%
\pgfsetfillopacity{0.973057}%
\pgfsetlinewidth{1.003750pt}%
\definecolor{currentstroke}{rgb}{0.121569,0.466667,0.705882}%
\pgfsetstrokecolor{currentstroke}%
\pgfsetstrokeopacity{0.973057}%
\pgfsetdash{}{0pt}%
\pgfpathmoveto{\pgfqpoint{2.611895in}{0.957659in}}%
\pgfpathcurveto{\pgfqpoint{2.620131in}{0.957659in}}{\pgfqpoint{2.628031in}{0.960931in}}{\pgfqpoint{2.633855in}{0.966755in}}%
\pgfpathcurveto{\pgfqpoint{2.639679in}{0.972579in}}{\pgfqpoint{2.642951in}{0.980479in}}{\pgfqpoint{2.642951in}{0.988715in}}%
\pgfpathcurveto{\pgfqpoint{2.642951in}{0.996951in}}{\pgfqpoint{2.639679in}{1.004851in}}{\pgfqpoint{2.633855in}{1.010675in}}%
\pgfpathcurveto{\pgfqpoint{2.628031in}{1.016499in}}{\pgfqpoint{2.620131in}{1.019772in}}{\pgfqpoint{2.611895in}{1.019772in}}%
\pgfpathcurveto{\pgfqpoint{2.603659in}{1.019772in}}{\pgfqpoint{2.595759in}{1.016499in}}{\pgfqpoint{2.589935in}{1.010675in}}%
\pgfpathcurveto{\pgfqpoint{2.584111in}{1.004851in}}{\pgfqpoint{2.580838in}{0.996951in}}{\pgfqpoint{2.580838in}{0.988715in}}%
\pgfpathcurveto{\pgfqpoint{2.580838in}{0.980479in}}{\pgfqpoint{2.584111in}{0.972579in}}{\pgfqpoint{2.589935in}{0.966755in}}%
\pgfpathcurveto{\pgfqpoint{2.595759in}{0.960931in}}{\pgfqpoint{2.603659in}{0.957659in}}{\pgfqpoint{2.611895in}{0.957659in}}%
\pgfpathclose%
\pgfusepath{stroke,fill}%
\end{pgfscope}%
\begin{pgfscope}%
\pgfpathrectangle{\pgfqpoint{0.100000in}{0.212622in}}{\pgfqpoint{3.696000in}{3.696000in}}%
\pgfusepath{clip}%
\pgfsetbuttcap%
\pgfsetroundjoin%
\definecolor{currentfill}{rgb}{0.121569,0.466667,0.705882}%
\pgfsetfillcolor{currentfill}%
\pgfsetfillopacity{0.973662}%
\pgfsetlinewidth{1.003750pt}%
\definecolor{currentstroke}{rgb}{0.121569,0.466667,0.705882}%
\pgfsetstrokecolor{currentstroke}%
\pgfsetstrokeopacity{0.973662}%
\pgfsetdash{}{0pt}%
\pgfpathmoveto{\pgfqpoint{2.315622in}{0.867170in}}%
\pgfpathcurveto{\pgfqpoint{2.323858in}{0.867170in}}{\pgfqpoint{2.331758in}{0.870442in}}{\pgfqpoint{2.337582in}{0.876266in}}%
\pgfpathcurveto{\pgfqpoint{2.343406in}{0.882090in}}{\pgfqpoint{2.346678in}{0.889990in}}{\pgfqpoint{2.346678in}{0.898226in}}%
\pgfpathcurveto{\pgfqpoint{2.346678in}{0.906462in}}{\pgfqpoint{2.343406in}{0.914362in}}{\pgfqpoint{2.337582in}{0.920186in}}%
\pgfpathcurveto{\pgfqpoint{2.331758in}{0.926010in}}{\pgfqpoint{2.323858in}{0.929283in}}{\pgfqpoint{2.315622in}{0.929283in}}%
\pgfpathcurveto{\pgfqpoint{2.307386in}{0.929283in}}{\pgfqpoint{2.299486in}{0.926010in}}{\pgfqpoint{2.293662in}{0.920186in}}%
\pgfpathcurveto{\pgfqpoint{2.287838in}{0.914362in}}{\pgfqpoint{2.284565in}{0.906462in}}{\pgfqpoint{2.284565in}{0.898226in}}%
\pgfpathcurveto{\pgfqpoint{2.284565in}{0.889990in}}{\pgfqpoint{2.287838in}{0.882090in}}{\pgfqpoint{2.293662in}{0.876266in}}%
\pgfpathcurveto{\pgfqpoint{2.299486in}{0.870442in}}{\pgfqpoint{2.307386in}{0.867170in}}{\pgfqpoint{2.315622in}{0.867170in}}%
\pgfpathclose%
\pgfusepath{stroke,fill}%
\end{pgfscope}%
\begin{pgfscope}%
\pgfpathrectangle{\pgfqpoint{0.100000in}{0.212622in}}{\pgfqpoint{3.696000in}{3.696000in}}%
\pgfusepath{clip}%
\pgfsetbuttcap%
\pgfsetroundjoin%
\definecolor{currentfill}{rgb}{0.121569,0.466667,0.705882}%
\pgfsetfillcolor{currentfill}%
\pgfsetfillopacity{0.973827}%
\pgfsetlinewidth{1.003750pt}%
\definecolor{currentstroke}{rgb}{0.121569,0.466667,0.705882}%
\pgfsetstrokecolor{currentstroke}%
\pgfsetstrokeopacity{0.973827}%
\pgfsetdash{}{0pt}%
\pgfpathmoveto{\pgfqpoint{2.611269in}{0.952780in}}%
\pgfpathcurveto{\pgfqpoint{2.619505in}{0.952780in}}{\pgfqpoint{2.627405in}{0.956052in}}{\pgfqpoint{2.633229in}{0.961876in}}%
\pgfpathcurveto{\pgfqpoint{2.639053in}{0.967700in}}{\pgfqpoint{2.642325in}{0.975600in}}{\pgfqpoint{2.642325in}{0.983837in}}%
\pgfpathcurveto{\pgfqpoint{2.642325in}{0.992073in}}{\pgfqpoint{2.639053in}{0.999973in}}{\pgfqpoint{2.633229in}{1.005797in}}%
\pgfpathcurveto{\pgfqpoint{2.627405in}{1.011621in}}{\pgfqpoint{2.619505in}{1.014893in}}{\pgfqpoint{2.611269in}{1.014893in}}%
\pgfpathcurveto{\pgfqpoint{2.603032in}{1.014893in}}{\pgfqpoint{2.595132in}{1.011621in}}{\pgfqpoint{2.589308in}{1.005797in}}%
\pgfpathcurveto{\pgfqpoint{2.583484in}{0.999973in}}{\pgfqpoint{2.580212in}{0.992073in}}{\pgfqpoint{2.580212in}{0.983837in}}%
\pgfpathcurveto{\pgfqpoint{2.580212in}{0.975600in}}{\pgfqpoint{2.583484in}{0.967700in}}{\pgfqpoint{2.589308in}{0.961876in}}%
\pgfpathcurveto{\pgfqpoint{2.595132in}{0.956052in}}{\pgfqpoint{2.603032in}{0.952780in}}{\pgfqpoint{2.611269in}{0.952780in}}%
\pgfpathclose%
\pgfusepath{stroke,fill}%
\end{pgfscope}%
\begin{pgfscope}%
\pgfpathrectangle{\pgfqpoint{0.100000in}{0.212622in}}{\pgfqpoint{3.696000in}{3.696000in}}%
\pgfusepath{clip}%
\pgfsetbuttcap%
\pgfsetroundjoin%
\definecolor{currentfill}{rgb}{0.121569,0.466667,0.705882}%
\pgfsetfillcolor{currentfill}%
\pgfsetfillopacity{0.974789}%
\pgfsetlinewidth{1.003750pt}%
\definecolor{currentstroke}{rgb}{0.121569,0.466667,0.705882}%
\pgfsetstrokecolor{currentstroke}%
\pgfsetstrokeopacity{0.974789}%
\pgfsetdash{}{0pt}%
\pgfpathmoveto{\pgfqpoint{2.324217in}{0.861273in}}%
\pgfpathcurveto{\pgfqpoint{2.332454in}{0.861273in}}{\pgfqpoint{2.340354in}{0.864545in}}{\pgfqpoint{2.346178in}{0.870369in}}%
\pgfpathcurveto{\pgfqpoint{2.352001in}{0.876193in}}{\pgfqpoint{2.355274in}{0.884093in}}{\pgfqpoint{2.355274in}{0.892329in}}%
\pgfpathcurveto{\pgfqpoint{2.355274in}{0.900565in}}{\pgfqpoint{2.352001in}{0.908466in}}{\pgfqpoint{2.346178in}{0.914289in}}%
\pgfpathcurveto{\pgfqpoint{2.340354in}{0.920113in}}{\pgfqpoint{2.332454in}{0.923386in}}{\pgfqpoint{2.324217in}{0.923386in}}%
\pgfpathcurveto{\pgfqpoint{2.315981in}{0.923386in}}{\pgfqpoint{2.308081in}{0.920113in}}{\pgfqpoint{2.302257in}{0.914289in}}%
\pgfpathcurveto{\pgfqpoint{2.296433in}{0.908466in}}{\pgfqpoint{2.293161in}{0.900565in}}{\pgfqpoint{2.293161in}{0.892329in}}%
\pgfpathcurveto{\pgfqpoint{2.293161in}{0.884093in}}{\pgfqpoint{2.296433in}{0.876193in}}{\pgfqpoint{2.302257in}{0.870369in}}%
\pgfpathcurveto{\pgfqpoint{2.308081in}{0.864545in}}{\pgfqpoint{2.315981in}{0.861273in}}{\pgfqpoint{2.324217in}{0.861273in}}%
\pgfpathclose%
\pgfusepath{stroke,fill}%
\end{pgfscope}%
\begin{pgfscope}%
\pgfpathrectangle{\pgfqpoint{0.100000in}{0.212622in}}{\pgfqpoint{3.696000in}{3.696000in}}%
\pgfusepath{clip}%
\pgfsetbuttcap%
\pgfsetroundjoin%
\definecolor{currentfill}{rgb}{0.121569,0.466667,0.705882}%
\pgfsetfillcolor{currentfill}%
\pgfsetfillopacity{0.975217}%
\pgfsetlinewidth{1.003750pt}%
\definecolor{currentstroke}{rgb}{0.121569,0.466667,0.705882}%
\pgfsetstrokecolor{currentstroke}%
\pgfsetstrokeopacity{0.975217}%
\pgfsetdash{}{0pt}%
\pgfpathmoveto{\pgfqpoint{2.609890in}{0.943834in}}%
\pgfpathcurveto{\pgfqpoint{2.618127in}{0.943834in}}{\pgfqpoint{2.626027in}{0.947106in}}{\pgfqpoint{2.631851in}{0.952930in}}%
\pgfpathcurveto{\pgfqpoint{2.637675in}{0.958754in}}{\pgfqpoint{2.640947in}{0.966654in}}{\pgfqpoint{2.640947in}{0.974891in}}%
\pgfpathcurveto{\pgfqpoint{2.640947in}{0.983127in}}{\pgfqpoint{2.637675in}{0.991027in}}{\pgfqpoint{2.631851in}{0.996851in}}%
\pgfpathcurveto{\pgfqpoint{2.626027in}{1.002675in}}{\pgfqpoint{2.618127in}{1.005947in}}{\pgfqpoint{2.609890in}{1.005947in}}%
\pgfpathcurveto{\pgfqpoint{2.601654in}{1.005947in}}{\pgfqpoint{2.593754in}{1.002675in}}{\pgfqpoint{2.587930in}{0.996851in}}%
\pgfpathcurveto{\pgfqpoint{2.582106in}{0.991027in}}{\pgfqpoint{2.578834in}{0.983127in}}{\pgfqpoint{2.578834in}{0.974891in}}%
\pgfpathcurveto{\pgfqpoint{2.578834in}{0.966654in}}{\pgfqpoint{2.582106in}{0.958754in}}{\pgfqpoint{2.587930in}{0.952930in}}%
\pgfpathcurveto{\pgfqpoint{2.593754in}{0.947106in}}{\pgfqpoint{2.601654in}{0.943834in}}{\pgfqpoint{2.609890in}{0.943834in}}%
\pgfpathclose%
\pgfusepath{stroke,fill}%
\end{pgfscope}%
\begin{pgfscope}%
\pgfpathrectangle{\pgfqpoint{0.100000in}{0.212622in}}{\pgfqpoint{3.696000in}{3.696000in}}%
\pgfusepath{clip}%
\pgfsetbuttcap%
\pgfsetroundjoin%
\definecolor{currentfill}{rgb}{0.121569,0.466667,0.705882}%
\pgfsetfillcolor{currentfill}%
\pgfsetfillopacity{0.975651}%
\pgfsetlinewidth{1.003750pt}%
\definecolor{currentstroke}{rgb}{0.121569,0.466667,0.705882}%
\pgfsetstrokecolor{currentstroke}%
\pgfsetstrokeopacity{0.975651}%
\pgfsetdash{}{0pt}%
\pgfpathmoveto{\pgfqpoint{2.333455in}{0.854366in}}%
\pgfpathcurveto{\pgfqpoint{2.341691in}{0.854366in}}{\pgfqpoint{2.349591in}{0.857638in}}{\pgfqpoint{2.355415in}{0.863462in}}%
\pgfpathcurveto{\pgfqpoint{2.361239in}{0.869286in}}{\pgfqpoint{2.364511in}{0.877186in}}{\pgfqpoint{2.364511in}{0.885423in}}%
\pgfpathcurveto{\pgfqpoint{2.364511in}{0.893659in}}{\pgfqpoint{2.361239in}{0.901559in}}{\pgfqpoint{2.355415in}{0.907383in}}%
\pgfpathcurveto{\pgfqpoint{2.349591in}{0.913207in}}{\pgfqpoint{2.341691in}{0.916479in}}{\pgfqpoint{2.333455in}{0.916479in}}%
\pgfpathcurveto{\pgfqpoint{2.325218in}{0.916479in}}{\pgfqpoint{2.317318in}{0.913207in}}{\pgfqpoint{2.311494in}{0.907383in}}%
\pgfpathcurveto{\pgfqpoint{2.305671in}{0.901559in}}{\pgfqpoint{2.302398in}{0.893659in}}{\pgfqpoint{2.302398in}{0.885423in}}%
\pgfpathcurveto{\pgfqpoint{2.302398in}{0.877186in}}{\pgfqpoint{2.305671in}{0.869286in}}{\pgfqpoint{2.311494in}{0.863462in}}%
\pgfpathcurveto{\pgfqpoint{2.317318in}{0.857638in}}{\pgfqpoint{2.325218in}{0.854366in}}{\pgfqpoint{2.333455in}{0.854366in}}%
\pgfpathclose%
\pgfusepath{stroke,fill}%
\end{pgfscope}%
\begin{pgfscope}%
\pgfpathrectangle{\pgfqpoint{0.100000in}{0.212622in}}{\pgfqpoint{3.696000in}{3.696000in}}%
\pgfusepath{clip}%
\pgfsetbuttcap%
\pgfsetroundjoin%
\definecolor{currentfill}{rgb}{0.121569,0.466667,0.705882}%
\pgfsetfillcolor{currentfill}%
\pgfsetfillopacity{0.976266}%
\pgfsetlinewidth{1.003750pt}%
\definecolor{currentstroke}{rgb}{0.121569,0.466667,0.705882}%
\pgfsetstrokecolor{currentstroke}%
\pgfsetstrokeopacity{0.976266}%
\pgfsetdash{}{0pt}%
\pgfpathmoveto{\pgfqpoint{2.606754in}{0.939196in}}%
\pgfpathcurveto{\pgfqpoint{2.614991in}{0.939196in}}{\pgfqpoint{2.622891in}{0.942469in}}{\pgfqpoint{2.628715in}{0.948293in}}%
\pgfpathcurveto{\pgfqpoint{2.634539in}{0.954117in}}{\pgfqpoint{2.637811in}{0.962017in}}{\pgfqpoint{2.637811in}{0.970253in}}%
\pgfpathcurveto{\pgfqpoint{2.637811in}{0.978489in}}{\pgfqpoint{2.634539in}{0.986389in}}{\pgfqpoint{2.628715in}{0.992213in}}%
\pgfpathcurveto{\pgfqpoint{2.622891in}{0.998037in}}{\pgfqpoint{2.614991in}{1.001309in}}{\pgfqpoint{2.606754in}{1.001309in}}%
\pgfpathcurveto{\pgfqpoint{2.598518in}{1.001309in}}{\pgfqpoint{2.590618in}{0.998037in}}{\pgfqpoint{2.584794in}{0.992213in}}%
\pgfpathcurveto{\pgfqpoint{2.578970in}{0.986389in}}{\pgfqpoint{2.575698in}{0.978489in}}{\pgfqpoint{2.575698in}{0.970253in}}%
\pgfpathcurveto{\pgfqpoint{2.575698in}{0.962017in}}{\pgfqpoint{2.578970in}{0.954117in}}{\pgfqpoint{2.584794in}{0.948293in}}%
\pgfpathcurveto{\pgfqpoint{2.590618in}{0.942469in}}{\pgfqpoint{2.598518in}{0.939196in}}{\pgfqpoint{2.606754in}{0.939196in}}%
\pgfpathclose%
\pgfusepath{stroke,fill}%
\end{pgfscope}%
\begin{pgfscope}%
\pgfpathrectangle{\pgfqpoint{0.100000in}{0.212622in}}{\pgfqpoint{3.696000in}{3.696000in}}%
\pgfusepath{clip}%
\pgfsetbuttcap%
\pgfsetroundjoin%
\definecolor{currentfill}{rgb}{0.121569,0.466667,0.705882}%
\pgfsetfillcolor{currentfill}%
\pgfsetfillopacity{0.976726}%
\pgfsetlinewidth{1.003750pt}%
\definecolor{currentstroke}{rgb}{0.121569,0.466667,0.705882}%
\pgfsetstrokecolor{currentstroke}%
\pgfsetstrokeopacity{0.976726}%
\pgfsetdash{}{0pt}%
\pgfpathmoveto{\pgfqpoint{2.605085in}{0.937152in}}%
\pgfpathcurveto{\pgfqpoint{2.613321in}{0.937152in}}{\pgfqpoint{2.621221in}{0.940424in}}{\pgfqpoint{2.627045in}{0.946248in}}%
\pgfpathcurveto{\pgfqpoint{2.632869in}{0.952072in}}{\pgfqpoint{2.636141in}{0.959972in}}{\pgfqpoint{2.636141in}{0.968208in}}%
\pgfpathcurveto{\pgfqpoint{2.636141in}{0.976444in}}{\pgfqpoint{2.632869in}{0.984344in}}{\pgfqpoint{2.627045in}{0.990168in}}%
\pgfpathcurveto{\pgfqpoint{2.621221in}{0.995992in}}{\pgfqpoint{2.613321in}{0.999265in}}{\pgfqpoint{2.605085in}{0.999265in}}%
\pgfpathcurveto{\pgfqpoint{2.596849in}{0.999265in}}{\pgfqpoint{2.588949in}{0.995992in}}{\pgfqpoint{2.583125in}{0.990168in}}%
\pgfpathcurveto{\pgfqpoint{2.577301in}{0.984344in}}{\pgfqpoint{2.574028in}{0.976444in}}{\pgfqpoint{2.574028in}{0.968208in}}%
\pgfpathcurveto{\pgfqpoint{2.574028in}{0.959972in}}{\pgfqpoint{2.577301in}{0.952072in}}{\pgfqpoint{2.583125in}{0.946248in}}%
\pgfpathcurveto{\pgfqpoint{2.588949in}{0.940424in}}{\pgfqpoint{2.596849in}{0.937152in}}{\pgfqpoint{2.605085in}{0.937152in}}%
\pgfpathclose%
\pgfusepath{stroke,fill}%
\end{pgfscope}%
\begin{pgfscope}%
\pgfpathrectangle{\pgfqpoint{0.100000in}{0.212622in}}{\pgfqpoint{3.696000in}{3.696000in}}%
\pgfusepath{clip}%
\pgfsetbuttcap%
\pgfsetroundjoin%
\definecolor{currentfill}{rgb}{0.121569,0.466667,0.705882}%
\pgfsetfillcolor{currentfill}%
\pgfsetfillopacity{0.977311}%
\pgfsetlinewidth{1.003750pt}%
\definecolor{currentstroke}{rgb}{0.121569,0.466667,0.705882}%
\pgfsetstrokecolor{currentstroke}%
\pgfsetstrokeopacity{0.977311}%
\pgfsetdash{}{0pt}%
\pgfpathmoveto{\pgfqpoint{2.343090in}{0.850427in}}%
\pgfpathcurveto{\pgfqpoint{2.351327in}{0.850427in}}{\pgfqpoint{2.359227in}{0.853699in}}{\pgfqpoint{2.365051in}{0.859523in}}%
\pgfpathcurveto{\pgfqpoint{2.370875in}{0.865347in}}{\pgfqpoint{2.374147in}{0.873247in}}{\pgfqpoint{2.374147in}{0.881483in}}%
\pgfpathcurveto{\pgfqpoint{2.374147in}{0.889720in}}{\pgfqpoint{2.370875in}{0.897620in}}{\pgfqpoint{2.365051in}{0.903444in}}%
\pgfpathcurveto{\pgfqpoint{2.359227in}{0.909268in}}{\pgfqpoint{2.351327in}{0.912540in}}{\pgfqpoint{2.343090in}{0.912540in}}%
\pgfpathcurveto{\pgfqpoint{2.334854in}{0.912540in}}{\pgfqpoint{2.326954in}{0.909268in}}{\pgfqpoint{2.321130in}{0.903444in}}%
\pgfpathcurveto{\pgfqpoint{2.315306in}{0.897620in}}{\pgfqpoint{2.312034in}{0.889720in}}{\pgfqpoint{2.312034in}{0.881483in}}%
\pgfpathcurveto{\pgfqpoint{2.312034in}{0.873247in}}{\pgfqpoint{2.315306in}{0.865347in}}{\pgfqpoint{2.321130in}{0.859523in}}%
\pgfpathcurveto{\pgfqpoint{2.326954in}{0.853699in}}{\pgfqpoint{2.334854in}{0.850427in}}{\pgfqpoint{2.343090in}{0.850427in}}%
\pgfpathclose%
\pgfusepath{stroke,fill}%
\end{pgfscope}%
\begin{pgfscope}%
\pgfpathrectangle{\pgfqpoint{0.100000in}{0.212622in}}{\pgfqpoint{3.696000in}{3.696000in}}%
\pgfusepath{clip}%
\pgfsetbuttcap%
\pgfsetroundjoin%
\definecolor{currentfill}{rgb}{0.121569,0.466667,0.705882}%
\pgfsetfillcolor{currentfill}%
\pgfsetfillopacity{0.977443}%
\pgfsetlinewidth{1.003750pt}%
\definecolor{currentstroke}{rgb}{0.121569,0.466667,0.705882}%
\pgfsetstrokecolor{currentstroke}%
\pgfsetstrokeopacity{0.977443}%
\pgfsetdash{}{0pt}%
\pgfpathmoveto{\pgfqpoint{2.602215in}{0.932555in}}%
\pgfpathcurveto{\pgfqpoint{2.610451in}{0.932555in}}{\pgfqpoint{2.618351in}{0.935828in}}{\pgfqpoint{2.624175in}{0.941651in}}%
\pgfpathcurveto{\pgfqpoint{2.629999in}{0.947475in}}{\pgfqpoint{2.633271in}{0.955375in}}{\pgfqpoint{2.633271in}{0.963612in}}%
\pgfpathcurveto{\pgfqpoint{2.633271in}{0.971848in}}{\pgfqpoint{2.629999in}{0.979748in}}{\pgfqpoint{2.624175in}{0.985572in}}%
\pgfpathcurveto{\pgfqpoint{2.618351in}{0.991396in}}{\pgfqpoint{2.610451in}{0.994668in}}{\pgfqpoint{2.602215in}{0.994668in}}%
\pgfpathcurveto{\pgfqpoint{2.593978in}{0.994668in}}{\pgfqpoint{2.586078in}{0.991396in}}{\pgfqpoint{2.580254in}{0.985572in}}%
\pgfpathcurveto{\pgfqpoint{2.574430in}{0.979748in}}{\pgfqpoint{2.571158in}{0.971848in}}{\pgfqpoint{2.571158in}{0.963612in}}%
\pgfpathcurveto{\pgfqpoint{2.571158in}{0.955375in}}{\pgfqpoint{2.574430in}{0.947475in}}{\pgfqpoint{2.580254in}{0.941651in}}%
\pgfpathcurveto{\pgfqpoint{2.586078in}{0.935828in}}{\pgfqpoint{2.593978in}{0.932555in}}{\pgfqpoint{2.602215in}{0.932555in}}%
\pgfpathclose%
\pgfusepath{stroke,fill}%
\end{pgfscope}%
\begin{pgfscope}%
\pgfpathrectangle{\pgfqpoint{0.100000in}{0.212622in}}{\pgfqpoint{3.696000in}{3.696000in}}%
\pgfusepath{clip}%
\pgfsetbuttcap%
\pgfsetroundjoin%
\definecolor{currentfill}{rgb}{0.121569,0.466667,0.705882}%
\pgfsetfillcolor{currentfill}%
\pgfsetfillopacity{0.978129}%
\pgfsetlinewidth{1.003750pt}%
\definecolor{currentstroke}{rgb}{0.121569,0.466667,0.705882}%
\pgfsetstrokecolor{currentstroke}%
\pgfsetstrokeopacity{0.978129}%
\pgfsetdash{}{0pt}%
\pgfpathmoveto{\pgfqpoint{2.600715in}{0.929539in}}%
\pgfpathcurveto{\pgfqpoint{2.608951in}{0.929539in}}{\pgfqpoint{2.616851in}{0.932811in}}{\pgfqpoint{2.622675in}{0.938635in}}%
\pgfpathcurveto{\pgfqpoint{2.628499in}{0.944459in}}{\pgfqpoint{2.631772in}{0.952359in}}{\pgfqpoint{2.631772in}{0.960595in}}%
\pgfpathcurveto{\pgfqpoint{2.631772in}{0.968831in}}{\pgfqpoint{2.628499in}{0.976731in}}{\pgfqpoint{2.622675in}{0.982555in}}%
\pgfpathcurveto{\pgfqpoint{2.616851in}{0.988379in}}{\pgfqpoint{2.608951in}{0.991652in}}{\pgfqpoint{2.600715in}{0.991652in}}%
\pgfpathcurveto{\pgfqpoint{2.592479in}{0.991652in}}{\pgfqpoint{2.584579in}{0.988379in}}{\pgfqpoint{2.578755in}{0.982555in}}%
\pgfpathcurveto{\pgfqpoint{2.572931in}{0.976731in}}{\pgfqpoint{2.569659in}{0.968831in}}{\pgfqpoint{2.569659in}{0.960595in}}%
\pgfpathcurveto{\pgfqpoint{2.569659in}{0.952359in}}{\pgfqpoint{2.572931in}{0.944459in}}{\pgfqpoint{2.578755in}{0.938635in}}%
\pgfpathcurveto{\pgfqpoint{2.584579in}{0.932811in}}{\pgfqpoint{2.592479in}{0.929539in}}{\pgfqpoint{2.600715in}{0.929539in}}%
\pgfpathclose%
\pgfusepath{stroke,fill}%
\end{pgfscope}%
\begin{pgfscope}%
\pgfpathrectangle{\pgfqpoint{0.100000in}{0.212622in}}{\pgfqpoint{3.696000in}{3.696000in}}%
\pgfusepath{clip}%
\pgfsetbuttcap%
\pgfsetroundjoin%
\definecolor{currentfill}{rgb}{0.121569,0.466667,0.705882}%
\pgfsetfillcolor{currentfill}%
\pgfsetfillopacity{0.978321}%
\pgfsetlinewidth{1.003750pt}%
\definecolor{currentstroke}{rgb}{0.121569,0.466667,0.705882}%
\pgfsetstrokecolor{currentstroke}%
\pgfsetstrokeopacity{0.978321}%
\pgfsetdash{}{0pt}%
\pgfpathmoveto{\pgfqpoint{2.348845in}{0.850387in}}%
\pgfpathcurveto{\pgfqpoint{2.357081in}{0.850387in}}{\pgfqpoint{2.364981in}{0.853659in}}{\pgfqpoint{2.370805in}{0.859483in}}%
\pgfpathcurveto{\pgfqpoint{2.376629in}{0.865307in}}{\pgfqpoint{2.379901in}{0.873207in}}{\pgfqpoint{2.379901in}{0.881444in}}%
\pgfpathcurveto{\pgfqpoint{2.379901in}{0.889680in}}{\pgfqpoint{2.376629in}{0.897580in}}{\pgfqpoint{2.370805in}{0.903404in}}%
\pgfpathcurveto{\pgfqpoint{2.364981in}{0.909228in}}{\pgfqpoint{2.357081in}{0.912500in}}{\pgfqpoint{2.348845in}{0.912500in}}%
\pgfpathcurveto{\pgfqpoint{2.340608in}{0.912500in}}{\pgfqpoint{2.332708in}{0.909228in}}{\pgfqpoint{2.326884in}{0.903404in}}%
\pgfpathcurveto{\pgfqpoint{2.321060in}{0.897580in}}{\pgfqpoint{2.317788in}{0.889680in}}{\pgfqpoint{2.317788in}{0.881444in}}%
\pgfpathcurveto{\pgfqpoint{2.317788in}{0.873207in}}{\pgfqpoint{2.321060in}{0.865307in}}{\pgfqpoint{2.326884in}{0.859483in}}%
\pgfpathcurveto{\pgfqpoint{2.332708in}{0.853659in}}{\pgfqpoint{2.340608in}{0.850387in}}{\pgfqpoint{2.348845in}{0.850387in}}%
\pgfpathclose%
\pgfusepath{stroke,fill}%
\end{pgfscope}%
\begin{pgfscope}%
\pgfpathrectangle{\pgfqpoint{0.100000in}{0.212622in}}{\pgfqpoint{3.696000in}{3.696000in}}%
\pgfusepath{clip}%
\pgfsetbuttcap%
\pgfsetroundjoin%
\definecolor{currentfill}{rgb}{0.121569,0.466667,0.705882}%
\pgfsetfillcolor{currentfill}%
\pgfsetfillopacity{0.978398}%
\pgfsetlinewidth{1.003750pt}%
\definecolor{currentstroke}{rgb}{0.121569,0.466667,0.705882}%
\pgfsetstrokecolor{currentstroke}%
\pgfsetstrokeopacity{0.978398}%
\pgfsetdash{}{0pt}%
\pgfpathmoveto{\pgfqpoint{2.600558in}{0.928211in}}%
\pgfpathcurveto{\pgfqpoint{2.608794in}{0.928211in}}{\pgfqpoint{2.616694in}{0.931483in}}{\pgfqpoint{2.622518in}{0.937307in}}%
\pgfpathcurveto{\pgfqpoint{2.628342in}{0.943131in}}{\pgfqpoint{2.631614in}{0.951031in}}{\pgfqpoint{2.631614in}{0.959267in}}%
\pgfpathcurveto{\pgfqpoint{2.631614in}{0.967503in}}{\pgfqpoint{2.628342in}{0.975403in}}{\pgfqpoint{2.622518in}{0.981227in}}%
\pgfpathcurveto{\pgfqpoint{2.616694in}{0.987051in}}{\pgfqpoint{2.608794in}{0.990324in}}{\pgfqpoint{2.600558in}{0.990324in}}%
\pgfpathcurveto{\pgfqpoint{2.592321in}{0.990324in}}{\pgfqpoint{2.584421in}{0.987051in}}{\pgfqpoint{2.578597in}{0.981227in}}%
\pgfpathcurveto{\pgfqpoint{2.572773in}{0.975403in}}{\pgfqpoint{2.569501in}{0.967503in}}{\pgfqpoint{2.569501in}{0.959267in}}%
\pgfpathcurveto{\pgfqpoint{2.569501in}{0.951031in}}{\pgfqpoint{2.572773in}{0.943131in}}{\pgfqpoint{2.578597in}{0.937307in}}%
\pgfpathcurveto{\pgfqpoint{2.584421in}{0.931483in}}{\pgfqpoint{2.592321in}{0.928211in}}{\pgfqpoint{2.600558in}{0.928211in}}%
\pgfpathclose%
\pgfusepath{stroke,fill}%
\end{pgfscope}%
\begin{pgfscope}%
\pgfpathrectangle{\pgfqpoint{0.100000in}{0.212622in}}{\pgfqpoint{3.696000in}{3.696000in}}%
\pgfusepath{clip}%
\pgfsetbuttcap%
\pgfsetroundjoin%
\definecolor{currentfill}{rgb}{0.121569,0.466667,0.705882}%
\pgfsetfillcolor{currentfill}%
\pgfsetfillopacity{0.978793}%
\pgfsetlinewidth{1.003750pt}%
\definecolor{currentstroke}{rgb}{0.121569,0.466667,0.705882}%
\pgfsetstrokecolor{currentstroke}%
\pgfsetstrokeopacity{0.978793}%
\pgfsetdash{}{0pt}%
\pgfpathmoveto{\pgfqpoint{2.600407in}{0.925447in}}%
\pgfpathcurveto{\pgfqpoint{2.608644in}{0.925447in}}{\pgfqpoint{2.616544in}{0.928719in}}{\pgfqpoint{2.622368in}{0.934543in}}%
\pgfpathcurveto{\pgfqpoint{2.628192in}{0.940367in}}{\pgfqpoint{2.631464in}{0.948267in}}{\pgfqpoint{2.631464in}{0.956503in}}%
\pgfpathcurveto{\pgfqpoint{2.631464in}{0.964740in}}{\pgfqpoint{2.628192in}{0.972640in}}{\pgfqpoint{2.622368in}{0.978464in}}%
\pgfpathcurveto{\pgfqpoint{2.616544in}{0.984287in}}{\pgfqpoint{2.608644in}{0.987560in}}{\pgfqpoint{2.600407in}{0.987560in}}%
\pgfpathcurveto{\pgfqpoint{2.592171in}{0.987560in}}{\pgfqpoint{2.584271in}{0.984287in}}{\pgfqpoint{2.578447in}{0.978464in}}%
\pgfpathcurveto{\pgfqpoint{2.572623in}{0.972640in}}{\pgfqpoint{2.569351in}{0.964740in}}{\pgfqpoint{2.569351in}{0.956503in}}%
\pgfpathcurveto{\pgfqpoint{2.569351in}{0.948267in}}{\pgfqpoint{2.572623in}{0.940367in}}{\pgfqpoint{2.578447in}{0.934543in}}%
\pgfpathcurveto{\pgfqpoint{2.584271in}{0.928719in}}{\pgfqpoint{2.592171in}{0.925447in}}{\pgfqpoint{2.600407in}{0.925447in}}%
\pgfpathclose%
\pgfusepath{stroke,fill}%
\end{pgfscope}%
\begin{pgfscope}%
\pgfpathrectangle{\pgfqpoint{0.100000in}{0.212622in}}{\pgfqpoint{3.696000in}{3.696000in}}%
\pgfusepath{clip}%
\pgfsetbuttcap%
\pgfsetroundjoin%
\definecolor{currentfill}{rgb}{0.121569,0.466667,0.705882}%
\pgfsetfillcolor{currentfill}%
\pgfsetfillopacity{0.979171}%
\pgfsetlinewidth{1.003750pt}%
\definecolor{currentstroke}{rgb}{0.121569,0.466667,0.705882}%
\pgfsetstrokecolor{currentstroke}%
\pgfsetstrokeopacity{0.979171}%
\pgfsetdash{}{0pt}%
\pgfpathmoveto{\pgfqpoint{2.599984in}{0.923446in}}%
\pgfpathcurveto{\pgfqpoint{2.608220in}{0.923446in}}{\pgfqpoint{2.616120in}{0.926718in}}{\pgfqpoint{2.621944in}{0.932542in}}%
\pgfpathcurveto{\pgfqpoint{2.627768in}{0.938366in}}{\pgfqpoint{2.631041in}{0.946266in}}{\pgfqpoint{2.631041in}{0.954503in}}%
\pgfpathcurveto{\pgfqpoint{2.631041in}{0.962739in}}{\pgfqpoint{2.627768in}{0.970639in}}{\pgfqpoint{2.621944in}{0.976463in}}%
\pgfpathcurveto{\pgfqpoint{2.616120in}{0.982287in}}{\pgfqpoint{2.608220in}{0.985559in}}{\pgfqpoint{2.599984in}{0.985559in}}%
\pgfpathcurveto{\pgfqpoint{2.591748in}{0.985559in}}{\pgfqpoint{2.583848in}{0.982287in}}{\pgfqpoint{2.578024in}{0.976463in}}%
\pgfpathcurveto{\pgfqpoint{2.572200in}{0.970639in}}{\pgfqpoint{2.568928in}{0.962739in}}{\pgfqpoint{2.568928in}{0.954503in}}%
\pgfpathcurveto{\pgfqpoint{2.568928in}{0.946266in}}{\pgfqpoint{2.572200in}{0.938366in}}{\pgfqpoint{2.578024in}{0.932542in}}%
\pgfpathcurveto{\pgfqpoint{2.583848in}{0.926718in}}{\pgfqpoint{2.591748in}{0.923446in}}{\pgfqpoint{2.599984in}{0.923446in}}%
\pgfpathclose%
\pgfusepath{stroke,fill}%
\end{pgfscope}%
\begin{pgfscope}%
\pgfpathrectangle{\pgfqpoint{0.100000in}{0.212622in}}{\pgfqpoint{3.696000in}{3.696000in}}%
\pgfusepath{clip}%
\pgfsetbuttcap%
\pgfsetroundjoin%
\definecolor{currentfill}{rgb}{0.121569,0.466667,0.705882}%
\pgfsetfillcolor{currentfill}%
\pgfsetfillopacity{0.979676}%
\pgfsetlinewidth{1.003750pt}%
\definecolor{currentstroke}{rgb}{0.121569,0.466667,0.705882}%
\pgfsetstrokecolor{currentstroke}%
\pgfsetstrokeopacity{0.979676}%
\pgfsetdash{}{0pt}%
\pgfpathmoveto{\pgfqpoint{2.357829in}{0.850437in}}%
\pgfpathcurveto{\pgfqpoint{2.366065in}{0.850437in}}{\pgfqpoint{2.373966in}{0.853710in}}{\pgfqpoint{2.379789in}{0.859534in}}%
\pgfpathcurveto{\pgfqpoint{2.385613in}{0.865358in}}{\pgfqpoint{2.388886in}{0.873258in}}{\pgfqpoint{2.388886in}{0.881494in}}%
\pgfpathcurveto{\pgfqpoint{2.388886in}{0.889730in}}{\pgfqpoint{2.385613in}{0.897630in}}{\pgfqpoint{2.379789in}{0.903454in}}%
\pgfpathcurveto{\pgfqpoint{2.373966in}{0.909278in}}{\pgfqpoint{2.366065in}{0.912550in}}{\pgfqpoint{2.357829in}{0.912550in}}%
\pgfpathcurveto{\pgfqpoint{2.349593in}{0.912550in}}{\pgfqpoint{2.341693in}{0.909278in}}{\pgfqpoint{2.335869in}{0.903454in}}%
\pgfpathcurveto{\pgfqpoint{2.330045in}{0.897630in}}{\pgfqpoint{2.326773in}{0.889730in}}{\pgfqpoint{2.326773in}{0.881494in}}%
\pgfpathcurveto{\pgfqpoint{2.326773in}{0.873258in}}{\pgfqpoint{2.330045in}{0.865358in}}{\pgfqpoint{2.335869in}{0.859534in}}%
\pgfpathcurveto{\pgfqpoint{2.341693in}{0.853710in}}{\pgfqpoint{2.349593in}{0.850437in}}{\pgfqpoint{2.357829in}{0.850437in}}%
\pgfpathclose%
\pgfusepath{stroke,fill}%
\end{pgfscope}%
\begin{pgfscope}%
\pgfpathrectangle{\pgfqpoint{0.100000in}{0.212622in}}{\pgfqpoint{3.696000in}{3.696000in}}%
\pgfusepath{clip}%
\pgfsetbuttcap%
\pgfsetroundjoin%
\definecolor{currentfill}{rgb}{0.121569,0.466667,0.705882}%
\pgfsetfillcolor{currentfill}%
\pgfsetfillopacity{0.979983}%
\pgfsetlinewidth{1.003750pt}%
\definecolor{currentstroke}{rgb}{0.121569,0.466667,0.705882}%
\pgfsetstrokecolor{currentstroke}%
\pgfsetstrokeopacity{0.979983}%
\pgfsetdash{}{0pt}%
\pgfpathmoveto{\pgfqpoint{2.598457in}{0.920680in}}%
\pgfpathcurveto{\pgfqpoint{2.606693in}{0.920680in}}{\pgfqpoint{2.614593in}{0.923953in}}{\pgfqpoint{2.620417in}{0.929776in}}%
\pgfpathcurveto{\pgfqpoint{2.626241in}{0.935600in}}{\pgfqpoint{2.629513in}{0.943500in}}{\pgfqpoint{2.629513in}{0.951737in}}%
\pgfpathcurveto{\pgfqpoint{2.629513in}{0.959973in}}{\pgfqpoint{2.626241in}{0.967873in}}{\pgfqpoint{2.620417in}{0.973697in}}%
\pgfpathcurveto{\pgfqpoint{2.614593in}{0.979521in}}{\pgfqpoint{2.606693in}{0.982793in}}{\pgfqpoint{2.598457in}{0.982793in}}%
\pgfpathcurveto{\pgfqpoint{2.590220in}{0.982793in}}{\pgfqpoint{2.582320in}{0.979521in}}{\pgfqpoint{2.576496in}{0.973697in}}%
\pgfpathcurveto{\pgfqpoint{2.570673in}{0.967873in}}{\pgfqpoint{2.567400in}{0.959973in}}{\pgfqpoint{2.567400in}{0.951737in}}%
\pgfpathcurveto{\pgfqpoint{2.567400in}{0.943500in}}{\pgfqpoint{2.570673in}{0.935600in}}{\pgfqpoint{2.576496in}{0.929776in}}%
\pgfpathcurveto{\pgfqpoint{2.582320in}{0.923953in}}{\pgfqpoint{2.590220in}{0.920680in}}{\pgfqpoint{2.598457in}{0.920680in}}%
\pgfpathclose%
\pgfusepath{stroke,fill}%
\end{pgfscope}%
\begin{pgfscope}%
\pgfpathrectangle{\pgfqpoint{0.100000in}{0.212622in}}{\pgfqpoint{3.696000in}{3.696000in}}%
\pgfusepath{clip}%
\pgfsetbuttcap%
\pgfsetroundjoin%
\definecolor{currentfill}{rgb}{0.121569,0.466667,0.705882}%
\pgfsetfillcolor{currentfill}%
\pgfsetfillopacity{0.980440}%
\pgfsetlinewidth{1.003750pt}%
\definecolor{currentstroke}{rgb}{0.121569,0.466667,0.705882}%
\pgfsetstrokecolor{currentstroke}%
\pgfsetstrokeopacity{0.980440}%
\pgfsetdash{}{0pt}%
\pgfpathmoveto{\pgfqpoint{2.597109in}{0.919128in}}%
\pgfpathcurveto{\pgfqpoint{2.605345in}{0.919128in}}{\pgfqpoint{2.613245in}{0.922400in}}{\pgfqpoint{2.619069in}{0.928224in}}%
\pgfpathcurveto{\pgfqpoint{2.624893in}{0.934048in}}{\pgfqpoint{2.628166in}{0.941948in}}{\pgfqpoint{2.628166in}{0.950184in}}%
\pgfpathcurveto{\pgfqpoint{2.628166in}{0.958421in}}{\pgfqpoint{2.624893in}{0.966321in}}{\pgfqpoint{2.619069in}{0.972145in}}%
\pgfpathcurveto{\pgfqpoint{2.613245in}{0.977969in}}{\pgfqpoint{2.605345in}{0.981241in}}{\pgfqpoint{2.597109in}{0.981241in}}%
\pgfpathcurveto{\pgfqpoint{2.588873in}{0.981241in}}{\pgfqpoint{2.580973in}{0.977969in}}{\pgfqpoint{2.575149in}{0.972145in}}%
\pgfpathcurveto{\pgfqpoint{2.569325in}{0.966321in}}{\pgfqpoint{2.566053in}{0.958421in}}{\pgfqpoint{2.566053in}{0.950184in}}%
\pgfpathcurveto{\pgfqpoint{2.566053in}{0.941948in}}{\pgfqpoint{2.569325in}{0.934048in}}{\pgfqpoint{2.575149in}{0.928224in}}%
\pgfpathcurveto{\pgfqpoint{2.580973in}{0.922400in}}{\pgfqpoint{2.588873in}{0.919128in}}{\pgfqpoint{2.597109in}{0.919128in}}%
\pgfpathclose%
\pgfusepath{stroke,fill}%
\end{pgfscope}%
\begin{pgfscope}%
\pgfpathrectangle{\pgfqpoint{0.100000in}{0.212622in}}{\pgfqpoint{3.696000in}{3.696000in}}%
\pgfusepath{clip}%
\pgfsetbuttcap%
\pgfsetroundjoin%
\definecolor{currentfill}{rgb}{0.121569,0.466667,0.705882}%
\pgfsetfillcolor{currentfill}%
\pgfsetfillopacity{0.980633}%
\pgfsetlinewidth{1.003750pt}%
\definecolor{currentstroke}{rgb}{0.121569,0.466667,0.705882}%
\pgfsetstrokecolor{currentstroke}%
\pgfsetstrokeopacity{0.980633}%
\pgfsetdash{}{0pt}%
\pgfpathmoveto{\pgfqpoint{2.367709in}{0.841965in}}%
\pgfpathcurveto{\pgfqpoint{2.375945in}{0.841965in}}{\pgfqpoint{2.383845in}{0.845237in}}{\pgfqpoint{2.389669in}{0.851061in}}%
\pgfpathcurveto{\pgfqpoint{2.395493in}{0.856885in}}{\pgfqpoint{2.398765in}{0.864785in}}{\pgfqpoint{2.398765in}{0.873022in}}%
\pgfpathcurveto{\pgfqpoint{2.398765in}{0.881258in}}{\pgfqpoint{2.395493in}{0.889158in}}{\pgfqpoint{2.389669in}{0.894982in}}%
\pgfpathcurveto{\pgfqpoint{2.383845in}{0.900806in}}{\pgfqpoint{2.375945in}{0.904078in}}{\pgfqpoint{2.367709in}{0.904078in}}%
\pgfpathcurveto{\pgfqpoint{2.359473in}{0.904078in}}{\pgfqpoint{2.351572in}{0.900806in}}{\pgfqpoint{2.345749in}{0.894982in}}%
\pgfpathcurveto{\pgfqpoint{2.339925in}{0.889158in}}{\pgfqpoint{2.336652in}{0.881258in}}{\pgfqpoint{2.336652in}{0.873022in}}%
\pgfpathcurveto{\pgfqpoint{2.336652in}{0.864785in}}{\pgfqpoint{2.339925in}{0.856885in}}{\pgfqpoint{2.345749in}{0.851061in}}%
\pgfpathcurveto{\pgfqpoint{2.351572in}{0.845237in}}{\pgfqpoint{2.359473in}{0.841965in}}{\pgfqpoint{2.367709in}{0.841965in}}%
\pgfpathclose%
\pgfusepath{stroke,fill}%
\end{pgfscope}%
\begin{pgfscope}%
\pgfpathrectangle{\pgfqpoint{0.100000in}{0.212622in}}{\pgfqpoint{3.696000in}{3.696000in}}%
\pgfusepath{clip}%
\pgfsetbuttcap%
\pgfsetroundjoin%
\definecolor{currentfill}{rgb}{0.121569,0.466667,0.705882}%
\pgfsetfillcolor{currentfill}%
\pgfsetfillopacity{0.981027}%
\pgfsetlinewidth{1.003750pt}%
\definecolor{currentstroke}{rgb}{0.121569,0.466667,0.705882}%
\pgfsetstrokecolor{currentstroke}%
\pgfsetstrokeopacity{0.981027}%
\pgfsetdash{}{0pt}%
\pgfpathmoveto{\pgfqpoint{2.373158in}{0.836809in}}%
\pgfpathcurveto{\pgfqpoint{2.381394in}{0.836809in}}{\pgfqpoint{2.389294in}{0.840081in}}{\pgfqpoint{2.395118in}{0.845905in}}%
\pgfpathcurveto{\pgfqpoint{2.400942in}{0.851729in}}{\pgfqpoint{2.404214in}{0.859629in}}{\pgfqpoint{2.404214in}{0.867865in}}%
\pgfpathcurveto{\pgfqpoint{2.404214in}{0.876101in}}{\pgfqpoint{2.400942in}{0.884001in}}{\pgfqpoint{2.395118in}{0.889825in}}%
\pgfpathcurveto{\pgfqpoint{2.389294in}{0.895649in}}{\pgfqpoint{2.381394in}{0.898922in}}{\pgfqpoint{2.373158in}{0.898922in}}%
\pgfpathcurveto{\pgfqpoint{2.364921in}{0.898922in}}{\pgfqpoint{2.357021in}{0.895649in}}{\pgfqpoint{2.351197in}{0.889825in}}%
\pgfpathcurveto{\pgfqpoint{2.345374in}{0.884001in}}{\pgfqpoint{2.342101in}{0.876101in}}{\pgfqpoint{2.342101in}{0.867865in}}%
\pgfpathcurveto{\pgfqpoint{2.342101in}{0.859629in}}{\pgfqpoint{2.345374in}{0.851729in}}{\pgfqpoint{2.351197in}{0.845905in}}%
\pgfpathcurveto{\pgfqpoint{2.357021in}{0.840081in}}{\pgfqpoint{2.364921in}{0.836809in}}{\pgfqpoint{2.373158in}{0.836809in}}%
\pgfpathclose%
\pgfusepath{stroke,fill}%
\end{pgfscope}%
\begin{pgfscope}%
\pgfpathrectangle{\pgfqpoint{0.100000in}{0.212622in}}{\pgfqpoint{3.696000in}{3.696000in}}%
\pgfusepath{clip}%
\pgfsetbuttcap%
\pgfsetroundjoin%
\definecolor{currentfill}{rgb}{0.121569,0.466667,0.705882}%
\pgfsetfillcolor{currentfill}%
\pgfsetfillopacity{0.981176}%
\pgfsetlinewidth{1.003750pt}%
\definecolor{currentstroke}{rgb}{0.121569,0.466667,0.705882}%
\pgfsetstrokecolor{currentstroke}%
\pgfsetstrokeopacity{0.981176}%
\pgfsetdash{}{0pt}%
\pgfpathmoveto{\pgfqpoint{2.594563in}{0.916091in}}%
\pgfpathcurveto{\pgfqpoint{2.602800in}{0.916091in}}{\pgfqpoint{2.610700in}{0.919363in}}{\pgfqpoint{2.616524in}{0.925187in}}%
\pgfpathcurveto{\pgfqpoint{2.622347in}{0.931011in}}{\pgfqpoint{2.625620in}{0.938911in}}{\pgfqpoint{2.625620in}{0.947147in}}%
\pgfpathcurveto{\pgfqpoint{2.625620in}{0.955384in}}{\pgfqpoint{2.622347in}{0.963284in}}{\pgfqpoint{2.616524in}{0.969108in}}%
\pgfpathcurveto{\pgfqpoint{2.610700in}{0.974932in}}{\pgfqpoint{2.602800in}{0.978204in}}{\pgfqpoint{2.594563in}{0.978204in}}%
\pgfpathcurveto{\pgfqpoint{2.586327in}{0.978204in}}{\pgfqpoint{2.578427in}{0.974932in}}{\pgfqpoint{2.572603in}{0.969108in}}%
\pgfpathcurveto{\pgfqpoint{2.566779in}{0.963284in}}{\pgfqpoint{2.563507in}{0.955384in}}{\pgfqpoint{2.563507in}{0.947147in}}%
\pgfpathcurveto{\pgfqpoint{2.563507in}{0.938911in}}{\pgfqpoint{2.566779in}{0.931011in}}{\pgfqpoint{2.572603in}{0.925187in}}%
\pgfpathcurveto{\pgfqpoint{2.578427in}{0.919363in}}{\pgfqpoint{2.586327in}{0.916091in}}{\pgfqpoint{2.594563in}{0.916091in}}%
\pgfpathclose%
\pgfusepath{stroke,fill}%
\end{pgfscope}%
\begin{pgfscope}%
\pgfpathrectangle{\pgfqpoint{0.100000in}{0.212622in}}{\pgfqpoint{3.696000in}{3.696000in}}%
\pgfusepath{clip}%
\pgfsetbuttcap%
\pgfsetroundjoin%
\definecolor{currentfill}{rgb}{0.121569,0.466667,0.705882}%
\pgfsetfillcolor{currentfill}%
\pgfsetfillopacity{0.981700}%
\pgfsetlinewidth{1.003750pt}%
\definecolor{currentstroke}{rgb}{0.121569,0.466667,0.705882}%
\pgfsetstrokecolor{currentstroke}%
\pgfsetstrokeopacity{0.981700}%
\pgfsetdash{}{0pt}%
\pgfpathmoveto{\pgfqpoint{2.379410in}{0.833594in}}%
\pgfpathcurveto{\pgfqpoint{2.387647in}{0.833594in}}{\pgfqpoint{2.395547in}{0.836867in}}{\pgfqpoint{2.401371in}{0.842691in}}%
\pgfpathcurveto{\pgfqpoint{2.407194in}{0.848515in}}{\pgfqpoint{2.410467in}{0.856415in}}{\pgfqpoint{2.410467in}{0.864651in}}%
\pgfpathcurveto{\pgfqpoint{2.410467in}{0.872887in}}{\pgfqpoint{2.407194in}{0.880787in}}{\pgfqpoint{2.401371in}{0.886611in}}%
\pgfpathcurveto{\pgfqpoint{2.395547in}{0.892435in}}{\pgfqpoint{2.387647in}{0.895707in}}{\pgfqpoint{2.379410in}{0.895707in}}%
\pgfpathcurveto{\pgfqpoint{2.371174in}{0.895707in}}{\pgfqpoint{2.363274in}{0.892435in}}{\pgfqpoint{2.357450in}{0.886611in}}%
\pgfpathcurveto{\pgfqpoint{2.351626in}{0.880787in}}{\pgfqpoint{2.348354in}{0.872887in}}{\pgfqpoint{2.348354in}{0.864651in}}%
\pgfpathcurveto{\pgfqpoint{2.348354in}{0.856415in}}{\pgfqpoint{2.351626in}{0.848515in}}{\pgfqpoint{2.357450in}{0.842691in}}%
\pgfpathcurveto{\pgfqpoint{2.363274in}{0.836867in}}{\pgfqpoint{2.371174in}{0.833594in}}{\pgfqpoint{2.379410in}{0.833594in}}%
\pgfpathclose%
\pgfusepath{stroke,fill}%
\end{pgfscope}%
\begin{pgfscope}%
\pgfpathrectangle{\pgfqpoint{0.100000in}{0.212622in}}{\pgfqpoint{3.696000in}{3.696000in}}%
\pgfusepath{clip}%
\pgfsetbuttcap%
\pgfsetroundjoin%
\definecolor{currentfill}{rgb}{0.121569,0.466667,0.705882}%
\pgfsetfillcolor{currentfill}%
\pgfsetfillopacity{0.981994}%
\pgfsetlinewidth{1.003750pt}%
\definecolor{currentstroke}{rgb}{0.121569,0.466667,0.705882}%
\pgfsetstrokecolor{currentstroke}%
\pgfsetstrokeopacity{0.981994}%
\pgfsetdash{}{0pt}%
\pgfpathmoveto{\pgfqpoint{2.593354in}{0.912726in}}%
\pgfpathcurveto{\pgfqpoint{2.601590in}{0.912726in}}{\pgfqpoint{2.609490in}{0.915998in}}{\pgfqpoint{2.615314in}{0.921822in}}%
\pgfpathcurveto{\pgfqpoint{2.621138in}{0.927646in}}{\pgfqpoint{2.624410in}{0.935546in}}{\pgfqpoint{2.624410in}{0.943782in}}%
\pgfpathcurveto{\pgfqpoint{2.624410in}{0.952018in}}{\pgfqpoint{2.621138in}{0.959918in}}{\pgfqpoint{2.615314in}{0.965742in}}%
\pgfpathcurveto{\pgfqpoint{2.609490in}{0.971566in}}{\pgfqpoint{2.601590in}{0.974839in}}{\pgfqpoint{2.593354in}{0.974839in}}%
\pgfpathcurveto{\pgfqpoint{2.585118in}{0.974839in}}{\pgfqpoint{2.577218in}{0.971566in}}{\pgfqpoint{2.571394in}{0.965742in}}%
\pgfpathcurveto{\pgfqpoint{2.565570in}{0.959918in}}{\pgfqpoint{2.562297in}{0.952018in}}{\pgfqpoint{2.562297in}{0.943782in}}%
\pgfpathcurveto{\pgfqpoint{2.562297in}{0.935546in}}{\pgfqpoint{2.565570in}{0.927646in}}{\pgfqpoint{2.571394in}{0.921822in}}%
\pgfpathcurveto{\pgfqpoint{2.577218in}{0.915998in}}{\pgfqpoint{2.585118in}{0.912726in}}{\pgfqpoint{2.593354in}{0.912726in}}%
\pgfpathclose%
\pgfusepath{stroke,fill}%
\end{pgfscope}%
\begin{pgfscope}%
\pgfpathrectangle{\pgfqpoint{0.100000in}{0.212622in}}{\pgfqpoint{3.696000in}{3.696000in}}%
\pgfusepath{clip}%
\pgfsetbuttcap%
\pgfsetroundjoin%
\definecolor{currentfill}{rgb}{0.121569,0.466667,0.705882}%
\pgfsetfillcolor{currentfill}%
\pgfsetfillopacity{0.982488}%
\pgfsetlinewidth{1.003750pt}%
\definecolor{currentstroke}{rgb}{0.121569,0.466667,0.705882}%
\pgfsetstrokecolor{currentstroke}%
\pgfsetstrokeopacity{0.982488}%
\pgfsetdash{}{0pt}%
\pgfpathmoveto{\pgfqpoint{2.593124in}{0.910357in}}%
\pgfpathcurveto{\pgfqpoint{2.601360in}{0.910357in}}{\pgfqpoint{2.609260in}{0.913629in}}{\pgfqpoint{2.615084in}{0.919453in}}%
\pgfpathcurveto{\pgfqpoint{2.620908in}{0.925277in}}{\pgfqpoint{2.624181in}{0.933177in}}{\pgfqpoint{2.624181in}{0.941414in}}%
\pgfpathcurveto{\pgfqpoint{2.624181in}{0.949650in}}{\pgfqpoint{2.620908in}{0.957550in}}{\pgfqpoint{2.615084in}{0.963374in}}%
\pgfpathcurveto{\pgfqpoint{2.609260in}{0.969198in}}{\pgfqpoint{2.601360in}{0.972470in}}{\pgfqpoint{2.593124in}{0.972470in}}%
\pgfpathcurveto{\pgfqpoint{2.584888in}{0.972470in}}{\pgfqpoint{2.576988in}{0.969198in}}{\pgfqpoint{2.571164in}{0.963374in}}%
\pgfpathcurveto{\pgfqpoint{2.565340in}{0.957550in}}{\pgfqpoint{2.562068in}{0.949650in}}{\pgfqpoint{2.562068in}{0.941414in}}%
\pgfpathcurveto{\pgfqpoint{2.562068in}{0.933177in}}{\pgfqpoint{2.565340in}{0.925277in}}{\pgfqpoint{2.571164in}{0.919453in}}%
\pgfpathcurveto{\pgfqpoint{2.576988in}{0.913629in}}{\pgfqpoint{2.584888in}{0.910357in}}{\pgfqpoint{2.593124in}{0.910357in}}%
\pgfpathclose%
\pgfusepath{stroke,fill}%
\end{pgfscope}%
\begin{pgfscope}%
\pgfpathrectangle{\pgfqpoint{0.100000in}{0.212622in}}{\pgfqpoint{3.696000in}{3.696000in}}%
\pgfusepath{clip}%
\pgfsetbuttcap%
\pgfsetroundjoin%
\definecolor{currentfill}{rgb}{0.121569,0.466667,0.705882}%
\pgfsetfillcolor{currentfill}%
\pgfsetfillopacity{0.982588}%
\pgfsetlinewidth{1.003750pt}%
\definecolor{currentstroke}{rgb}{0.121569,0.466667,0.705882}%
\pgfsetstrokecolor{currentstroke}%
\pgfsetstrokeopacity{0.982588}%
\pgfsetdash{}{0pt}%
\pgfpathmoveto{\pgfqpoint{2.386321in}{0.830884in}}%
\pgfpathcurveto{\pgfqpoint{2.394558in}{0.830884in}}{\pgfqpoint{2.402458in}{0.834156in}}{\pgfqpoint{2.408282in}{0.839980in}}%
\pgfpathcurveto{\pgfqpoint{2.414106in}{0.845804in}}{\pgfqpoint{2.417378in}{0.853704in}}{\pgfqpoint{2.417378in}{0.861941in}}%
\pgfpathcurveto{\pgfqpoint{2.417378in}{0.870177in}}{\pgfqpoint{2.414106in}{0.878077in}}{\pgfqpoint{2.408282in}{0.883901in}}%
\pgfpathcurveto{\pgfqpoint{2.402458in}{0.889725in}}{\pgfqpoint{2.394558in}{0.892997in}}{\pgfqpoint{2.386321in}{0.892997in}}%
\pgfpathcurveto{\pgfqpoint{2.378085in}{0.892997in}}{\pgfqpoint{2.370185in}{0.889725in}}{\pgfqpoint{2.364361in}{0.883901in}}%
\pgfpathcurveto{\pgfqpoint{2.358537in}{0.878077in}}{\pgfqpoint{2.355265in}{0.870177in}}{\pgfqpoint{2.355265in}{0.861941in}}%
\pgfpathcurveto{\pgfqpoint{2.355265in}{0.853704in}}{\pgfqpoint{2.358537in}{0.845804in}}{\pgfqpoint{2.364361in}{0.839980in}}%
\pgfpathcurveto{\pgfqpoint{2.370185in}{0.834156in}}{\pgfqpoint{2.378085in}{0.830884in}}{\pgfqpoint{2.386321in}{0.830884in}}%
\pgfpathclose%
\pgfusepath{stroke,fill}%
\end{pgfscope}%
\begin{pgfscope}%
\pgfpathrectangle{\pgfqpoint{0.100000in}{0.212622in}}{\pgfqpoint{3.696000in}{3.696000in}}%
\pgfusepath{clip}%
\pgfsetbuttcap%
\pgfsetroundjoin%
\definecolor{currentfill}{rgb}{0.121569,0.466667,0.705882}%
\pgfsetfillcolor{currentfill}%
\pgfsetfillopacity{0.983293}%
\pgfsetlinewidth{1.003750pt}%
\definecolor{currentstroke}{rgb}{0.121569,0.466667,0.705882}%
\pgfsetstrokecolor{currentstroke}%
\pgfsetstrokeopacity{0.983293}%
\pgfsetdash{}{0pt}%
\pgfpathmoveto{\pgfqpoint{2.593262in}{0.905952in}}%
\pgfpathcurveto{\pgfqpoint{2.601498in}{0.905952in}}{\pgfqpoint{2.609398in}{0.909225in}}{\pgfqpoint{2.615222in}{0.915049in}}%
\pgfpathcurveto{\pgfqpoint{2.621046in}{0.920872in}}{\pgfqpoint{2.624318in}{0.928772in}}{\pgfqpoint{2.624318in}{0.937009in}}%
\pgfpathcurveto{\pgfqpoint{2.624318in}{0.945245in}}{\pgfqpoint{2.621046in}{0.953145in}}{\pgfqpoint{2.615222in}{0.958969in}}%
\pgfpathcurveto{\pgfqpoint{2.609398in}{0.964793in}}{\pgfqpoint{2.601498in}{0.968065in}}{\pgfqpoint{2.593262in}{0.968065in}}%
\pgfpathcurveto{\pgfqpoint{2.585025in}{0.968065in}}{\pgfqpoint{2.577125in}{0.964793in}}{\pgfqpoint{2.571301in}{0.958969in}}%
\pgfpathcurveto{\pgfqpoint{2.565477in}{0.953145in}}{\pgfqpoint{2.562205in}{0.945245in}}{\pgfqpoint{2.562205in}{0.937009in}}%
\pgfpathcurveto{\pgfqpoint{2.562205in}{0.928772in}}{\pgfqpoint{2.565477in}{0.920872in}}{\pgfqpoint{2.571301in}{0.915049in}}%
\pgfpathcurveto{\pgfqpoint{2.577125in}{0.909225in}}{\pgfqpoint{2.585025in}{0.905952in}}{\pgfqpoint{2.593262in}{0.905952in}}%
\pgfpathclose%
\pgfusepath{stroke,fill}%
\end{pgfscope}%
\begin{pgfscope}%
\pgfpathrectangle{\pgfqpoint{0.100000in}{0.212622in}}{\pgfqpoint{3.696000in}{3.696000in}}%
\pgfusepath{clip}%
\pgfsetbuttcap%
\pgfsetroundjoin%
\definecolor{currentfill}{rgb}{0.121569,0.466667,0.705882}%
\pgfsetfillcolor{currentfill}%
\pgfsetfillopacity{0.983421}%
\pgfsetlinewidth{1.003750pt}%
\definecolor{currentstroke}{rgb}{0.121569,0.466667,0.705882}%
\pgfsetstrokecolor{currentstroke}%
\pgfsetstrokeopacity{0.983421}%
\pgfsetdash{}{0pt}%
\pgfpathmoveto{\pgfqpoint{2.390273in}{0.831377in}}%
\pgfpathcurveto{\pgfqpoint{2.398509in}{0.831377in}}{\pgfqpoint{2.406409in}{0.834649in}}{\pgfqpoint{2.412233in}{0.840473in}}%
\pgfpathcurveto{\pgfqpoint{2.418057in}{0.846297in}}{\pgfqpoint{2.421329in}{0.854197in}}{\pgfqpoint{2.421329in}{0.862434in}}%
\pgfpathcurveto{\pgfqpoint{2.421329in}{0.870670in}}{\pgfqpoint{2.418057in}{0.878570in}}{\pgfqpoint{2.412233in}{0.884394in}}%
\pgfpathcurveto{\pgfqpoint{2.406409in}{0.890218in}}{\pgfqpoint{2.398509in}{0.893490in}}{\pgfqpoint{2.390273in}{0.893490in}}%
\pgfpathcurveto{\pgfqpoint{2.382036in}{0.893490in}}{\pgfqpoint{2.374136in}{0.890218in}}{\pgfqpoint{2.368313in}{0.884394in}}%
\pgfpathcurveto{\pgfqpoint{2.362489in}{0.878570in}}{\pgfqpoint{2.359216in}{0.870670in}}{\pgfqpoint{2.359216in}{0.862434in}}%
\pgfpathcurveto{\pgfqpoint{2.359216in}{0.854197in}}{\pgfqpoint{2.362489in}{0.846297in}}{\pgfqpoint{2.368313in}{0.840473in}}%
\pgfpathcurveto{\pgfqpoint{2.374136in}{0.834649in}}{\pgfqpoint{2.382036in}{0.831377in}}{\pgfqpoint{2.390273in}{0.831377in}}%
\pgfpathclose%
\pgfusepath{stroke,fill}%
\end{pgfscope}%
\begin{pgfscope}%
\pgfpathrectangle{\pgfqpoint{0.100000in}{0.212622in}}{\pgfqpoint{3.696000in}{3.696000in}}%
\pgfusepath{clip}%
\pgfsetbuttcap%
\pgfsetroundjoin%
\definecolor{currentfill}{rgb}{0.121569,0.466667,0.705882}%
\pgfsetfillcolor{currentfill}%
\pgfsetfillopacity{0.984072}%
\pgfsetlinewidth{1.003750pt}%
\definecolor{currentstroke}{rgb}{0.121569,0.466667,0.705882}%
\pgfsetstrokecolor{currentstroke}%
\pgfsetstrokeopacity{0.984072}%
\pgfsetdash{}{0pt}%
\pgfpathmoveto{\pgfqpoint{2.592864in}{0.902177in}}%
\pgfpathcurveto{\pgfqpoint{2.601100in}{0.902177in}}{\pgfqpoint{2.609000in}{0.905449in}}{\pgfqpoint{2.614824in}{0.911273in}}%
\pgfpathcurveto{\pgfqpoint{2.620648in}{0.917097in}}{\pgfqpoint{2.623920in}{0.924997in}}{\pgfqpoint{2.623920in}{0.933233in}}%
\pgfpathcurveto{\pgfqpoint{2.623920in}{0.941470in}}{\pgfqpoint{2.620648in}{0.949370in}}{\pgfqpoint{2.614824in}{0.955194in}}%
\pgfpathcurveto{\pgfqpoint{2.609000in}{0.961018in}}{\pgfqpoint{2.601100in}{0.964290in}}{\pgfqpoint{2.592864in}{0.964290in}}%
\pgfpathcurveto{\pgfqpoint{2.584628in}{0.964290in}}{\pgfqpoint{2.576728in}{0.961018in}}{\pgfqpoint{2.570904in}{0.955194in}}%
\pgfpathcurveto{\pgfqpoint{2.565080in}{0.949370in}}{\pgfqpoint{2.561807in}{0.941470in}}{\pgfqpoint{2.561807in}{0.933233in}}%
\pgfpathcurveto{\pgfqpoint{2.561807in}{0.924997in}}{\pgfqpoint{2.565080in}{0.917097in}}{\pgfqpoint{2.570904in}{0.911273in}}%
\pgfpathcurveto{\pgfqpoint{2.576728in}{0.905449in}}{\pgfqpoint{2.584628in}{0.902177in}}{\pgfqpoint{2.592864in}{0.902177in}}%
\pgfpathclose%
\pgfusepath{stroke,fill}%
\end{pgfscope}%
\begin{pgfscope}%
\pgfpathrectangle{\pgfqpoint{0.100000in}{0.212622in}}{\pgfqpoint{3.696000in}{3.696000in}}%
\pgfusepath{clip}%
\pgfsetbuttcap%
\pgfsetroundjoin%
\definecolor{currentfill}{rgb}{0.121569,0.466667,0.705882}%
\pgfsetfillcolor{currentfill}%
\pgfsetfillopacity{0.984655}%
\pgfsetlinewidth{1.003750pt}%
\definecolor{currentstroke}{rgb}{0.121569,0.466667,0.705882}%
\pgfsetstrokecolor{currentstroke}%
\pgfsetstrokeopacity{0.984655}%
\pgfsetdash{}{0pt}%
\pgfpathmoveto{\pgfqpoint{2.395199in}{0.831552in}}%
\pgfpathcurveto{\pgfqpoint{2.403435in}{0.831552in}}{\pgfqpoint{2.411335in}{0.834824in}}{\pgfqpoint{2.417159in}{0.840648in}}%
\pgfpathcurveto{\pgfqpoint{2.422983in}{0.846472in}}{\pgfqpoint{2.426256in}{0.854372in}}{\pgfqpoint{2.426256in}{0.862608in}}%
\pgfpathcurveto{\pgfqpoint{2.426256in}{0.870844in}}{\pgfqpoint{2.422983in}{0.878745in}}{\pgfqpoint{2.417159in}{0.884568in}}%
\pgfpathcurveto{\pgfqpoint{2.411335in}{0.890392in}}{\pgfqpoint{2.403435in}{0.893665in}}{\pgfqpoint{2.395199in}{0.893665in}}%
\pgfpathcurveto{\pgfqpoint{2.386963in}{0.893665in}}{\pgfqpoint{2.379063in}{0.890392in}}{\pgfqpoint{2.373239in}{0.884568in}}%
\pgfpathcurveto{\pgfqpoint{2.367415in}{0.878745in}}{\pgfqpoint{2.364143in}{0.870844in}}{\pgfqpoint{2.364143in}{0.862608in}}%
\pgfpathcurveto{\pgfqpoint{2.364143in}{0.854372in}}{\pgfqpoint{2.367415in}{0.846472in}}{\pgfqpoint{2.373239in}{0.840648in}}%
\pgfpathcurveto{\pgfqpoint{2.379063in}{0.834824in}}{\pgfqpoint{2.386963in}{0.831552in}}{\pgfqpoint{2.395199in}{0.831552in}}%
\pgfpathclose%
\pgfusepath{stroke,fill}%
\end{pgfscope}%
\begin{pgfscope}%
\pgfpathrectangle{\pgfqpoint{0.100000in}{0.212622in}}{\pgfqpoint{3.696000in}{3.696000in}}%
\pgfusepath{clip}%
\pgfsetbuttcap%
\pgfsetroundjoin%
\definecolor{currentfill}{rgb}{0.121569,0.466667,0.705882}%
\pgfsetfillcolor{currentfill}%
\pgfsetfillopacity{0.985443}%
\pgfsetlinewidth{1.003750pt}%
\definecolor{currentstroke}{rgb}{0.121569,0.466667,0.705882}%
\pgfsetstrokecolor{currentstroke}%
\pgfsetstrokeopacity{0.985443}%
\pgfsetdash{}{0pt}%
\pgfpathmoveto{\pgfqpoint{2.400462in}{0.829714in}}%
\pgfpathcurveto{\pgfqpoint{2.408698in}{0.829714in}}{\pgfqpoint{2.416598in}{0.832987in}}{\pgfqpoint{2.422422in}{0.838811in}}%
\pgfpathcurveto{\pgfqpoint{2.428246in}{0.844635in}}{\pgfqpoint{2.431518in}{0.852535in}}{\pgfqpoint{2.431518in}{0.860771in}}%
\pgfpathcurveto{\pgfqpoint{2.431518in}{0.869007in}}{\pgfqpoint{2.428246in}{0.876907in}}{\pgfqpoint{2.422422in}{0.882731in}}%
\pgfpathcurveto{\pgfqpoint{2.416598in}{0.888555in}}{\pgfqpoint{2.408698in}{0.891827in}}{\pgfqpoint{2.400462in}{0.891827in}}%
\pgfpathcurveto{\pgfqpoint{2.392225in}{0.891827in}}{\pgfqpoint{2.384325in}{0.888555in}}{\pgfqpoint{2.378501in}{0.882731in}}%
\pgfpathcurveto{\pgfqpoint{2.372677in}{0.876907in}}{\pgfqpoint{2.369405in}{0.869007in}}{\pgfqpoint{2.369405in}{0.860771in}}%
\pgfpathcurveto{\pgfqpoint{2.369405in}{0.852535in}}{\pgfqpoint{2.372677in}{0.844635in}}{\pgfqpoint{2.378501in}{0.838811in}}%
\pgfpathcurveto{\pgfqpoint{2.384325in}{0.832987in}}{\pgfqpoint{2.392225in}{0.829714in}}{\pgfqpoint{2.400462in}{0.829714in}}%
\pgfpathclose%
\pgfusepath{stroke,fill}%
\end{pgfscope}%
\begin{pgfscope}%
\pgfpathrectangle{\pgfqpoint{0.100000in}{0.212622in}}{\pgfqpoint{3.696000in}{3.696000in}}%
\pgfusepath{clip}%
\pgfsetbuttcap%
\pgfsetroundjoin%
\definecolor{currentfill}{rgb}{0.121569,0.466667,0.705882}%
\pgfsetfillcolor{currentfill}%
\pgfsetfillopacity{0.985637}%
\pgfsetlinewidth{1.003750pt}%
\definecolor{currentstroke}{rgb}{0.121569,0.466667,0.705882}%
\pgfsetstrokecolor{currentstroke}%
\pgfsetstrokeopacity{0.985637}%
\pgfsetdash{}{0pt}%
\pgfpathmoveto{\pgfqpoint{2.590688in}{0.895965in}}%
\pgfpathcurveto{\pgfqpoint{2.598924in}{0.895965in}}{\pgfqpoint{2.606825in}{0.899237in}}{\pgfqpoint{2.612648in}{0.905061in}}%
\pgfpathcurveto{\pgfqpoint{2.618472in}{0.910885in}}{\pgfqpoint{2.621745in}{0.918785in}}{\pgfqpoint{2.621745in}{0.927021in}}%
\pgfpathcurveto{\pgfqpoint{2.621745in}{0.935257in}}{\pgfqpoint{2.618472in}{0.943157in}}{\pgfqpoint{2.612648in}{0.948981in}}%
\pgfpathcurveto{\pgfqpoint{2.606825in}{0.954805in}}{\pgfqpoint{2.598924in}{0.958078in}}{\pgfqpoint{2.590688in}{0.958078in}}%
\pgfpathcurveto{\pgfqpoint{2.582452in}{0.958078in}}{\pgfqpoint{2.574552in}{0.954805in}}{\pgfqpoint{2.568728in}{0.948981in}}%
\pgfpathcurveto{\pgfqpoint{2.562904in}{0.943157in}}{\pgfqpoint{2.559632in}{0.935257in}}{\pgfqpoint{2.559632in}{0.927021in}}%
\pgfpathcurveto{\pgfqpoint{2.559632in}{0.918785in}}{\pgfqpoint{2.562904in}{0.910885in}}{\pgfqpoint{2.568728in}{0.905061in}}%
\pgfpathcurveto{\pgfqpoint{2.574552in}{0.899237in}}{\pgfqpoint{2.582452in}{0.895965in}}{\pgfqpoint{2.590688in}{0.895965in}}%
\pgfpathclose%
\pgfusepath{stroke,fill}%
\end{pgfscope}%
\begin{pgfscope}%
\pgfpathrectangle{\pgfqpoint{0.100000in}{0.212622in}}{\pgfqpoint{3.696000in}{3.696000in}}%
\pgfusepath{clip}%
\pgfsetbuttcap%
\pgfsetroundjoin%
\definecolor{currentfill}{rgb}{0.121569,0.466667,0.705882}%
\pgfsetfillcolor{currentfill}%
\pgfsetfillopacity{0.986410}%
\pgfsetlinewidth{1.003750pt}%
\definecolor{currentstroke}{rgb}{0.121569,0.466667,0.705882}%
\pgfsetstrokecolor{currentstroke}%
\pgfsetstrokeopacity{0.986410}%
\pgfsetdash{}{0pt}%
\pgfpathmoveto{\pgfqpoint{2.405992in}{0.825055in}}%
\pgfpathcurveto{\pgfqpoint{2.414228in}{0.825055in}}{\pgfqpoint{2.422128in}{0.828327in}}{\pgfqpoint{2.427952in}{0.834151in}}%
\pgfpathcurveto{\pgfqpoint{2.433776in}{0.839975in}}{\pgfqpoint{2.437048in}{0.847875in}}{\pgfqpoint{2.437048in}{0.856112in}}%
\pgfpathcurveto{\pgfqpoint{2.437048in}{0.864348in}}{\pgfqpoint{2.433776in}{0.872248in}}{\pgfqpoint{2.427952in}{0.878072in}}%
\pgfpathcurveto{\pgfqpoint{2.422128in}{0.883896in}}{\pgfqpoint{2.414228in}{0.887168in}}{\pgfqpoint{2.405992in}{0.887168in}}%
\pgfpathcurveto{\pgfqpoint{2.397755in}{0.887168in}}{\pgfqpoint{2.389855in}{0.883896in}}{\pgfqpoint{2.384031in}{0.878072in}}%
\pgfpathcurveto{\pgfqpoint{2.378207in}{0.872248in}}{\pgfqpoint{2.374935in}{0.864348in}}{\pgfqpoint{2.374935in}{0.856112in}}%
\pgfpathcurveto{\pgfqpoint{2.374935in}{0.847875in}}{\pgfqpoint{2.378207in}{0.839975in}}{\pgfqpoint{2.384031in}{0.834151in}}%
\pgfpathcurveto{\pgfqpoint{2.389855in}{0.828327in}}{\pgfqpoint{2.397755in}{0.825055in}}{\pgfqpoint{2.405992in}{0.825055in}}%
\pgfpathclose%
\pgfusepath{stroke,fill}%
\end{pgfscope}%
\begin{pgfscope}%
\pgfpathrectangle{\pgfqpoint{0.100000in}{0.212622in}}{\pgfqpoint{3.696000in}{3.696000in}}%
\pgfusepath{clip}%
\pgfsetbuttcap%
\pgfsetroundjoin%
\definecolor{currentfill}{rgb}{0.121569,0.466667,0.705882}%
\pgfsetfillcolor{currentfill}%
\pgfsetfillopacity{0.986799}%
\pgfsetlinewidth{1.003750pt}%
\definecolor{currentstroke}{rgb}{0.121569,0.466667,0.705882}%
\pgfsetstrokecolor{currentstroke}%
\pgfsetstrokeopacity{0.986799}%
\pgfsetdash{}{0pt}%
\pgfpathmoveto{\pgfqpoint{2.587552in}{0.891513in}}%
\pgfpathcurveto{\pgfqpoint{2.595789in}{0.891513in}}{\pgfqpoint{2.603689in}{0.894785in}}{\pgfqpoint{2.609513in}{0.900609in}}%
\pgfpathcurveto{\pgfqpoint{2.615337in}{0.906433in}}{\pgfqpoint{2.618609in}{0.914333in}}{\pgfqpoint{2.618609in}{0.922569in}}%
\pgfpathcurveto{\pgfqpoint{2.618609in}{0.930806in}}{\pgfqpoint{2.615337in}{0.938706in}}{\pgfqpoint{2.609513in}{0.944529in}}%
\pgfpathcurveto{\pgfqpoint{2.603689in}{0.950353in}}{\pgfqpoint{2.595789in}{0.953626in}}{\pgfqpoint{2.587552in}{0.953626in}}%
\pgfpathcurveto{\pgfqpoint{2.579316in}{0.953626in}}{\pgfqpoint{2.571416in}{0.950353in}}{\pgfqpoint{2.565592in}{0.944529in}}%
\pgfpathcurveto{\pgfqpoint{2.559768in}{0.938706in}}{\pgfqpoint{2.556496in}{0.930806in}}{\pgfqpoint{2.556496in}{0.922569in}}%
\pgfpathcurveto{\pgfqpoint{2.556496in}{0.914333in}}{\pgfqpoint{2.559768in}{0.906433in}}{\pgfqpoint{2.565592in}{0.900609in}}%
\pgfpathcurveto{\pgfqpoint{2.571416in}{0.894785in}}{\pgfqpoint{2.579316in}{0.891513in}}{\pgfqpoint{2.587552in}{0.891513in}}%
\pgfpathclose%
\pgfusepath{stroke,fill}%
\end{pgfscope}%
\begin{pgfscope}%
\pgfpathrectangle{\pgfqpoint{0.100000in}{0.212622in}}{\pgfqpoint{3.696000in}{3.696000in}}%
\pgfusepath{clip}%
\pgfsetbuttcap%
\pgfsetroundjoin%
\definecolor{currentfill}{rgb}{0.121569,0.466667,0.705882}%
\pgfsetfillcolor{currentfill}%
\pgfsetfillopacity{0.987245}%
\pgfsetlinewidth{1.003750pt}%
\definecolor{currentstroke}{rgb}{0.121569,0.466667,0.705882}%
\pgfsetstrokecolor{currentstroke}%
\pgfsetstrokeopacity{0.987245}%
\pgfsetdash{}{0pt}%
\pgfpathmoveto{\pgfqpoint{2.412195in}{0.820734in}}%
\pgfpathcurveto{\pgfqpoint{2.420432in}{0.820734in}}{\pgfqpoint{2.428332in}{0.824007in}}{\pgfqpoint{2.434156in}{0.829831in}}%
\pgfpathcurveto{\pgfqpoint{2.439980in}{0.835654in}}{\pgfqpoint{2.443252in}{0.843554in}}{\pgfqpoint{2.443252in}{0.851791in}}%
\pgfpathcurveto{\pgfqpoint{2.443252in}{0.860027in}}{\pgfqpoint{2.439980in}{0.867927in}}{\pgfqpoint{2.434156in}{0.873751in}}%
\pgfpathcurveto{\pgfqpoint{2.428332in}{0.879575in}}{\pgfqpoint{2.420432in}{0.882847in}}{\pgfqpoint{2.412195in}{0.882847in}}%
\pgfpathcurveto{\pgfqpoint{2.403959in}{0.882847in}}{\pgfqpoint{2.396059in}{0.879575in}}{\pgfqpoint{2.390235in}{0.873751in}}%
\pgfpathcurveto{\pgfqpoint{2.384411in}{0.867927in}}{\pgfqpoint{2.381139in}{0.860027in}}{\pgfqpoint{2.381139in}{0.851791in}}%
\pgfpathcurveto{\pgfqpoint{2.381139in}{0.843554in}}{\pgfqpoint{2.384411in}{0.835654in}}{\pgfqpoint{2.390235in}{0.829831in}}%
\pgfpathcurveto{\pgfqpoint{2.396059in}{0.824007in}}{\pgfqpoint{2.403959in}{0.820734in}}{\pgfqpoint{2.412195in}{0.820734in}}%
\pgfpathclose%
\pgfusepath{stroke,fill}%
\end{pgfscope}%
\begin{pgfscope}%
\pgfpathrectangle{\pgfqpoint{0.100000in}{0.212622in}}{\pgfqpoint{3.696000in}{3.696000in}}%
\pgfusepath{clip}%
\pgfsetbuttcap%
\pgfsetroundjoin%
\definecolor{currentfill}{rgb}{0.121569,0.466667,0.705882}%
\pgfsetfillcolor{currentfill}%
\pgfsetfillopacity{0.987491}%
\pgfsetlinewidth{1.003750pt}%
\definecolor{currentstroke}{rgb}{0.121569,0.466667,0.705882}%
\pgfsetstrokecolor{currentstroke}%
\pgfsetstrokeopacity{0.987491}%
\pgfsetdash{}{0pt}%
\pgfpathmoveto{\pgfqpoint{2.584954in}{0.888345in}}%
\pgfpathcurveto{\pgfqpoint{2.593190in}{0.888345in}}{\pgfqpoint{2.601090in}{0.891617in}}{\pgfqpoint{2.606914in}{0.897441in}}%
\pgfpathcurveto{\pgfqpoint{2.612738in}{0.903265in}}{\pgfqpoint{2.616011in}{0.911165in}}{\pgfqpoint{2.616011in}{0.919401in}}%
\pgfpathcurveto{\pgfqpoint{2.616011in}{0.927637in}}{\pgfqpoint{2.612738in}{0.935538in}}{\pgfqpoint{2.606914in}{0.941361in}}%
\pgfpathcurveto{\pgfqpoint{2.601090in}{0.947185in}}{\pgfqpoint{2.593190in}{0.950458in}}{\pgfqpoint{2.584954in}{0.950458in}}%
\pgfpathcurveto{\pgfqpoint{2.576718in}{0.950458in}}{\pgfqpoint{2.568818in}{0.947185in}}{\pgfqpoint{2.562994in}{0.941361in}}%
\pgfpathcurveto{\pgfqpoint{2.557170in}{0.935538in}}{\pgfqpoint{2.553898in}{0.927637in}}{\pgfqpoint{2.553898in}{0.919401in}}%
\pgfpathcurveto{\pgfqpoint{2.553898in}{0.911165in}}{\pgfqpoint{2.557170in}{0.903265in}}{\pgfqpoint{2.562994in}{0.897441in}}%
\pgfpathcurveto{\pgfqpoint{2.568818in}{0.891617in}}{\pgfqpoint{2.576718in}{0.888345in}}{\pgfqpoint{2.584954in}{0.888345in}}%
\pgfpathclose%
\pgfusepath{stroke,fill}%
\end{pgfscope}%
\begin{pgfscope}%
\pgfpathrectangle{\pgfqpoint{0.100000in}{0.212622in}}{\pgfqpoint{3.696000in}{3.696000in}}%
\pgfusepath{clip}%
\pgfsetbuttcap%
\pgfsetroundjoin%
\definecolor{currentfill}{rgb}{0.121569,0.466667,0.705882}%
\pgfsetfillcolor{currentfill}%
\pgfsetfillopacity{0.988208}%
\pgfsetlinewidth{1.003750pt}%
\definecolor{currentstroke}{rgb}{0.121569,0.466667,0.705882}%
\pgfsetstrokecolor{currentstroke}%
\pgfsetstrokeopacity{0.988208}%
\pgfsetdash{}{0pt}%
\pgfpathmoveto{\pgfqpoint{2.419804in}{0.816776in}}%
\pgfpathcurveto{\pgfqpoint{2.428040in}{0.816776in}}{\pgfqpoint{2.435940in}{0.820049in}}{\pgfqpoint{2.441764in}{0.825873in}}%
\pgfpathcurveto{\pgfqpoint{2.447588in}{0.831696in}}{\pgfqpoint{2.450860in}{0.839596in}}{\pgfqpoint{2.450860in}{0.847833in}}%
\pgfpathcurveto{\pgfqpoint{2.450860in}{0.856069in}}{\pgfqpoint{2.447588in}{0.863969in}}{\pgfqpoint{2.441764in}{0.869793in}}%
\pgfpathcurveto{\pgfqpoint{2.435940in}{0.875617in}}{\pgfqpoint{2.428040in}{0.878889in}}{\pgfqpoint{2.419804in}{0.878889in}}%
\pgfpathcurveto{\pgfqpoint{2.411567in}{0.878889in}}{\pgfqpoint{2.403667in}{0.875617in}}{\pgfqpoint{2.397843in}{0.869793in}}%
\pgfpathcurveto{\pgfqpoint{2.392019in}{0.863969in}}{\pgfqpoint{2.388747in}{0.856069in}}{\pgfqpoint{2.388747in}{0.847833in}}%
\pgfpathcurveto{\pgfqpoint{2.388747in}{0.839596in}}{\pgfqpoint{2.392019in}{0.831696in}}{\pgfqpoint{2.397843in}{0.825873in}}%
\pgfpathcurveto{\pgfqpoint{2.403667in}{0.820049in}}{\pgfqpoint{2.411567in}{0.816776in}}{\pgfqpoint{2.419804in}{0.816776in}}%
\pgfpathclose%
\pgfusepath{stroke,fill}%
\end{pgfscope}%
\begin{pgfscope}%
\pgfpathrectangle{\pgfqpoint{0.100000in}{0.212622in}}{\pgfqpoint{3.696000in}{3.696000in}}%
\pgfusepath{clip}%
\pgfsetbuttcap%
\pgfsetroundjoin%
\definecolor{currentfill}{rgb}{0.121569,0.466667,0.705882}%
\pgfsetfillcolor{currentfill}%
\pgfsetfillopacity{0.988749}%
\pgfsetlinewidth{1.003750pt}%
\definecolor{currentstroke}{rgb}{0.121569,0.466667,0.705882}%
\pgfsetstrokecolor{currentstroke}%
\pgfsetstrokeopacity{0.988749}%
\pgfsetdash{}{0pt}%
\pgfpathmoveto{\pgfqpoint{2.580660in}{0.881681in}}%
\pgfpathcurveto{\pgfqpoint{2.588896in}{0.881681in}}{\pgfqpoint{2.596796in}{0.884953in}}{\pgfqpoint{2.602620in}{0.890777in}}%
\pgfpathcurveto{\pgfqpoint{2.608444in}{0.896601in}}{\pgfqpoint{2.611716in}{0.904501in}}{\pgfqpoint{2.611716in}{0.912737in}}%
\pgfpathcurveto{\pgfqpoint{2.611716in}{0.920974in}}{\pgfqpoint{2.608444in}{0.928874in}}{\pgfqpoint{2.602620in}{0.934698in}}%
\pgfpathcurveto{\pgfqpoint{2.596796in}{0.940522in}}{\pgfqpoint{2.588896in}{0.943794in}}{\pgfqpoint{2.580660in}{0.943794in}}%
\pgfpathcurveto{\pgfqpoint{2.572423in}{0.943794in}}{\pgfqpoint{2.564523in}{0.940522in}}{\pgfqpoint{2.558699in}{0.934698in}}%
\pgfpathcurveto{\pgfqpoint{2.552875in}{0.928874in}}{\pgfqpoint{2.549603in}{0.920974in}}{\pgfqpoint{2.549603in}{0.912737in}}%
\pgfpathcurveto{\pgfqpoint{2.549603in}{0.904501in}}{\pgfqpoint{2.552875in}{0.896601in}}{\pgfqpoint{2.558699in}{0.890777in}}%
\pgfpathcurveto{\pgfqpoint{2.564523in}{0.884953in}}{\pgfqpoint{2.572423in}{0.881681in}}{\pgfqpoint{2.580660in}{0.881681in}}%
\pgfpathclose%
\pgfusepath{stroke,fill}%
\end{pgfscope}%
\begin{pgfscope}%
\pgfpathrectangle{\pgfqpoint{0.100000in}{0.212622in}}{\pgfqpoint{3.696000in}{3.696000in}}%
\pgfusepath{clip}%
\pgfsetbuttcap%
\pgfsetroundjoin%
\definecolor{currentfill}{rgb}{0.121569,0.466667,0.705882}%
\pgfsetfillcolor{currentfill}%
\pgfsetfillopacity{0.989053}%
\pgfsetlinewidth{1.003750pt}%
\definecolor{currentstroke}{rgb}{0.121569,0.466667,0.705882}%
\pgfsetstrokecolor{currentstroke}%
\pgfsetstrokeopacity{0.989053}%
\pgfsetdash{}{0pt}%
\pgfpathmoveto{\pgfqpoint{2.427958in}{0.813140in}}%
\pgfpathcurveto{\pgfqpoint{2.436194in}{0.813140in}}{\pgfqpoint{2.444094in}{0.816413in}}{\pgfqpoint{2.449918in}{0.822237in}}%
\pgfpathcurveto{\pgfqpoint{2.455742in}{0.828061in}}{\pgfqpoint{2.459014in}{0.835961in}}{\pgfqpoint{2.459014in}{0.844197in}}%
\pgfpathcurveto{\pgfqpoint{2.459014in}{0.852433in}}{\pgfqpoint{2.455742in}{0.860333in}}{\pgfqpoint{2.449918in}{0.866157in}}%
\pgfpathcurveto{\pgfqpoint{2.444094in}{0.871981in}}{\pgfqpoint{2.436194in}{0.875253in}}{\pgfqpoint{2.427958in}{0.875253in}}%
\pgfpathcurveto{\pgfqpoint{2.419722in}{0.875253in}}{\pgfqpoint{2.411822in}{0.871981in}}{\pgfqpoint{2.405998in}{0.866157in}}%
\pgfpathcurveto{\pgfqpoint{2.400174in}{0.860333in}}{\pgfqpoint{2.396901in}{0.852433in}}{\pgfqpoint{2.396901in}{0.844197in}}%
\pgfpathcurveto{\pgfqpoint{2.396901in}{0.835961in}}{\pgfqpoint{2.400174in}{0.828061in}}{\pgfqpoint{2.405998in}{0.822237in}}%
\pgfpathcurveto{\pgfqpoint{2.411822in}{0.816413in}}{\pgfqpoint{2.419722in}{0.813140in}}{\pgfqpoint{2.427958in}{0.813140in}}%
\pgfpathclose%
\pgfusepath{stroke,fill}%
\end{pgfscope}%
\begin{pgfscope}%
\pgfpathrectangle{\pgfqpoint{0.100000in}{0.212622in}}{\pgfqpoint{3.696000in}{3.696000in}}%
\pgfusepath{clip}%
\pgfsetbuttcap%
\pgfsetroundjoin%
\definecolor{currentfill}{rgb}{0.121569,0.466667,0.705882}%
\pgfsetfillcolor{currentfill}%
\pgfsetfillopacity{0.989759}%
\pgfsetlinewidth{1.003750pt}%
\definecolor{currentstroke}{rgb}{0.121569,0.466667,0.705882}%
\pgfsetstrokecolor{currentstroke}%
\pgfsetstrokeopacity{0.989759}%
\pgfsetdash{}{0pt}%
\pgfpathmoveto{\pgfqpoint{2.576624in}{0.876956in}}%
\pgfpathcurveto{\pgfqpoint{2.584860in}{0.876956in}}{\pgfqpoint{2.592760in}{0.880228in}}{\pgfqpoint{2.598584in}{0.886052in}}%
\pgfpathcurveto{\pgfqpoint{2.604408in}{0.891876in}}{\pgfqpoint{2.607680in}{0.899776in}}{\pgfqpoint{2.607680in}{0.908012in}}%
\pgfpathcurveto{\pgfqpoint{2.607680in}{0.916248in}}{\pgfqpoint{2.604408in}{0.924148in}}{\pgfqpoint{2.598584in}{0.929972in}}%
\pgfpathcurveto{\pgfqpoint{2.592760in}{0.935796in}}{\pgfqpoint{2.584860in}{0.939069in}}{\pgfqpoint{2.576624in}{0.939069in}}%
\pgfpathcurveto{\pgfqpoint{2.568388in}{0.939069in}}{\pgfqpoint{2.560487in}{0.935796in}}{\pgfqpoint{2.554664in}{0.929972in}}%
\pgfpathcurveto{\pgfqpoint{2.548840in}{0.924148in}}{\pgfqpoint{2.545567in}{0.916248in}}{\pgfqpoint{2.545567in}{0.908012in}}%
\pgfpathcurveto{\pgfqpoint{2.545567in}{0.899776in}}{\pgfqpoint{2.548840in}{0.891876in}}{\pgfqpoint{2.554664in}{0.886052in}}%
\pgfpathcurveto{\pgfqpoint{2.560487in}{0.880228in}}{\pgfqpoint{2.568388in}{0.876956in}}{\pgfqpoint{2.576624in}{0.876956in}}%
\pgfpathclose%
\pgfusepath{stroke,fill}%
\end{pgfscope}%
\begin{pgfscope}%
\pgfpathrectangle{\pgfqpoint{0.100000in}{0.212622in}}{\pgfqpoint{3.696000in}{3.696000in}}%
\pgfusepath{clip}%
\pgfsetbuttcap%
\pgfsetroundjoin%
\definecolor{currentfill}{rgb}{0.121569,0.466667,0.705882}%
\pgfsetfillcolor{currentfill}%
\pgfsetfillopacity{0.990311}%
\pgfsetlinewidth{1.003750pt}%
\definecolor{currentstroke}{rgb}{0.121569,0.466667,0.705882}%
\pgfsetstrokecolor{currentstroke}%
\pgfsetstrokeopacity{0.990311}%
\pgfsetdash{}{0pt}%
\pgfpathmoveto{\pgfqpoint{2.436580in}{0.807414in}}%
\pgfpathcurveto{\pgfqpoint{2.444816in}{0.807414in}}{\pgfqpoint{2.452716in}{0.810687in}}{\pgfqpoint{2.458540in}{0.816511in}}%
\pgfpathcurveto{\pgfqpoint{2.464364in}{0.822334in}}{\pgfqpoint{2.467636in}{0.830235in}}{\pgfqpoint{2.467636in}{0.838471in}}%
\pgfpathcurveto{\pgfqpoint{2.467636in}{0.846707in}}{\pgfqpoint{2.464364in}{0.854607in}}{\pgfqpoint{2.458540in}{0.860431in}}%
\pgfpathcurveto{\pgfqpoint{2.452716in}{0.866255in}}{\pgfqpoint{2.444816in}{0.869527in}}{\pgfqpoint{2.436580in}{0.869527in}}%
\pgfpathcurveto{\pgfqpoint{2.428344in}{0.869527in}}{\pgfqpoint{2.420444in}{0.866255in}}{\pgfqpoint{2.414620in}{0.860431in}}%
\pgfpathcurveto{\pgfqpoint{2.408796in}{0.854607in}}{\pgfqpoint{2.405523in}{0.846707in}}{\pgfqpoint{2.405523in}{0.838471in}}%
\pgfpathcurveto{\pgfqpoint{2.405523in}{0.830235in}}{\pgfqpoint{2.408796in}{0.822334in}}{\pgfqpoint{2.414620in}{0.816511in}}%
\pgfpathcurveto{\pgfqpoint{2.420444in}{0.810687in}}{\pgfqpoint{2.428344in}{0.807414in}}{\pgfqpoint{2.436580in}{0.807414in}}%
\pgfpathclose%
\pgfusepath{stroke,fill}%
\end{pgfscope}%
\begin{pgfscope}%
\pgfpathrectangle{\pgfqpoint{0.100000in}{0.212622in}}{\pgfqpoint{3.696000in}{3.696000in}}%
\pgfusepath{clip}%
\pgfsetbuttcap%
\pgfsetroundjoin%
\definecolor{currentfill}{rgb}{0.121569,0.466667,0.705882}%
\pgfsetfillcolor{currentfill}%
\pgfsetfillopacity{0.990616}%
\pgfsetlinewidth{1.003750pt}%
\definecolor{currentstroke}{rgb}{0.121569,0.466667,0.705882}%
\pgfsetstrokecolor{currentstroke}%
\pgfsetstrokeopacity{0.990616}%
\pgfsetdash{}{0pt}%
\pgfpathmoveto{\pgfqpoint{2.574406in}{0.871453in}}%
\pgfpathcurveto{\pgfqpoint{2.582643in}{0.871453in}}{\pgfqpoint{2.590543in}{0.874725in}}{\pgfqpoint{2.596367in}{0.880549in}}%
\pgfpathcurveto{\pgfqpoint{2.602191in}{0.886373in}}{\pgfqpoint{2.605463in}{0.894273in}}{\pgfqpoint{2.605463in}{0.902509in}}%
\pgfpathcurveto{\pgfqpoint{2.605463in}{0.910745in}}{\pgfqpoint{2.602191in}{0.918645in}}{\pgfqpoint{2.596367in}{0.924469in}}%
\pgfpathcurveto{\pgfqpoint{2.590543in}{0.930293in}}{\pgfqpoint{2.582643in}{0.933566in}}{\pgfqpoint{2.574406in}{0.933566in}}%
\pgfpathcurveto{\pgfqpoint{2.566170in}{0.933566in}}{\pgfqpoint{2.558270in}{0.930293in}}{\pgfqpoint{2.552446in}{0.924469in}}%
\pgfpathcurveto{\pgfqpoint{2.546622in}{0.918645in}}{\pgfqpoint{2.543350in}{0.910745in}}{\pgfqpoint{2.543350in}{0.902509in}}%
\pgfpathcurveto{\pgfqpoint{2.543350in}{0.894273in}}{\pgfqpoint{2.546622in}{0.886373in}}{\pgfqpoint{2.552446in}{0.880549in}}%
\pgfpathcurveto{\pgfqpoint{2.558270in}{0.874725in}}{\pgfqpoint{2.566170in}{0.871453in}}{\pgfqpoint{2.574406in}{0.871453in}}%
\pgfpathclose%
\pgfusepath{stroke,fill}%
\end{pgfscope}%
\begin{pgfscope}%
\pgfpathrectangle{\pgfqpoint{0.100000in}{0.212622in}}{\pgfqpoint{3.696000in}{3.696000in}}%
\pgfusepath{clip}%
\pgfsetbuttcap%
\pgfsetroundjoin%
\definecolor{currentfill}{rgb}{0.121569,0.466667,0.705882}%
\pgfsetfillcolor{currentfill}%
\pgfsetfillopacity{0.991258}%
\pgfsetlinewidth{1.003750pt}%
\definecolor{currentstroke}{rgb}{0.121569,0.466667,0.705882}%
\pgfsetstrokecolor{currentstroke}%
\pgfsetstrokeopacity{0.991258}%
\pgfsetdash{}{0pt}%
\pgfpathmoveto{\pgfqpoint{2.573036in}{0.867234in}}%
\pgfpathcurveto{\pgfqpoint{2.581272in}{0.867234in}}{\pgfqpoint{2.589173in}{0.870506in}}{\pgfqpoint{2.594996in}{0.876330in}}%
\pgfpathcurveto{\pgfqpoint{2.600820in}{0.882154in}}{\pgfqpoint{2.604093in}{0.890054in}}{\pgfqpoint{2.604093in}{0.898290in}}%
\pgfpathcurveto{\pgfqpoint{2.604093in}{0.906526in}}{\pgfqpoint{2.600820in}{0.914426in}}{\pgfqpoint{2.594996in}{0.920250in}}%
\pgfpathcurveto{\pgfqpoint{2.589173in}{0.926074in}}{\pgfqpoint{2.581272in}{0.929347in}}{\pgfqpoint{2.573036in}{0.929347in}}%
\pgfpathcurveto{\pgfqpoint{2.564800in}{0.929347in}}{\pgfqpoint{2.556900in}{0.926074in}}{\pgfqpoint{2.551076in}{0.920250in}}%
\pgfpathcurveto{\pgfqpoint{2.545252in}{0.914426in}}{\pgfqpoint{2.541980in}{0.906526in}}{\pgfqpoint{2.541980in}{0.898290in}}%
\pgfpathcurveto{\pgfqpoint{2.541980in}{0.890054in}}{\pgfqpoint{2.545252in}{0.882154in}}{\pgfqpoint{2.551076in}{0.876330in}}%
\pgfpathcurveto{\pgfqpoint{2.556900in}{0.870506in}}{\pgfqpoint{2.564800in}{0.867234in}}{\pgfqpoint{2.573036in}{0.867234in}}%
\pgfpathclose%
\pgfusepath{stroke,fill}%
\end{pgfscope}%
\begin{pgfscope}%
\pgfpathrectangle{\pgfqpoint{0.100000in}{0.212622in}}{\pgfqpoint{3.696000in}{3.696000in}}%
\pgfusepath{clip}%
\pgfsetbuttcap%
\pgfsetroundjoin%
\definecolor{currentfill}{rgb}{0.121569,0.466667,0.705882}%
\pgfsetfillcolor{currentfill}%
\pgfsetfillopacity{0.991480}%
\pgfsetlinewidth{1.003750pt}%
\definecolor{currentstroke}{rgb}{0.121569,0.466667,0.705882}%
\pgfsetstrokecolor{currentstroke}%
\pgfsetstrokeopacity{0.991480}%
\pgfsetdash{}{0pt}%
\pgfpathmoveto{\pgfqpoint{2.446251in}{0.803359in}}%
\pgfpathcurveto{\pgfqpoint{2.454487in}{0.803359in}}{\pgfqpoint{2.462387in}{0.806631in}}{\pgfqpoint{2.468211in}{0.812455in}}%
\pgfpathcurveto{\pgfqpoint{2.474035in}{0.818279in}}{\pgfqpoint{2.477307in}{0.826179in}}{\pgfqpoint{2.477307in}{0.834415in}}%
\pgfpathcurveto{\pgfqpoint{2.477307in}{0.842652in}}{\pgfqpoint{2.474035in}{0.850552in}}{\pgfqpoint{2.468211in}{0.856376in}}%
\pgfpathcurveto{\pgfqpoint{2.462387in}{0.862200in}}{\pgfqpoint{2.454487in}{0.865472in}}{\pgfqpoint{2.446251in}{0.865472in}}%
\pgfpathcurveto{\pgfqpoint{2.438015in}{0.865472in}}{\pgfqpoint{2.430114in}{0.862200in}}{\pgfqpoint{2.424291in}{0.856376in}}%
\pgfpathcurveto{\pgfqpoint{2.418467in}{0.850552in}}{\pgfqpoint{2.415194in}{0.842652in}}{\pgfqpoint{2.415194in}{0.834415in}}%
\pgfpathcurveto{\pgfqpoint{2.415194in}{0.826179in}}{\pgfqpoint{2.418467in}{0.818279in}}{\pgfqpoint{2.424291in}{0.812455in}}%
\pgfpathcurveto{\pgfqpoint{2.430114in}{0.806631in}}{\pgfqpoint{2.438015in}{0.803359in}}{\pgfqpoint{2.446251in}{0.803359in}}%
\pgfpathclose%
\pgfusepath{stroke,fill}%
\end{pgfscope}%
\begin{pgfscope}%
\pgfpathrectangle{\pgfqpoint{0.100000in}{0.212622in}}{\pgfqpoint{3.696000in}{3.696000in}}%
\pgfusepath{clip}%
\pgfsetbuttcap%
\pgfsetroundjoin%
\definecolor{currentfill}{rgb}{0.121569,0.466667,0.705882}%
\pgfsetfillcolor{currentfill}%
\pgfsetfillopacity{0.992420}%
\pgfsetlinewidth{1.003750pt}%
\definecolor{currentstroke}{rgb}{0.121569,0.466667,0.705882}%
\pgfsetstrokecolor{currentstroke}%
\pgfsetstrokeopacity{0.992420}%
\pgfsetdash{}{0pt}%
\pgfpathmoveto{\pgfqpoint{2.570005in}{0.859926in}}%
\pgfpathcurveto{\pgfqpoint{2.578241in}{0.859926in}}{\pgfqpoint{2.586142in}{0.863198in}}{\pgfqpoint{2.591965in}{0.869022in}}%
\pgfpathcurveto{\pgfqpoint{2.597789in}{0.874846in}}{\pgfqpoint{2.601062in}{0.882746in}}{\pgfqpoint{2.601062in}{0.890982in}}%
\pgfpathcurveto{\pgfqpoint{2.601062in}{0.899218in}}{\pgfqpoint{2.597789in}{0.907118in}}{\pgfqpoint{2.591965in}{0.912942in}}%
\pgfpathcurveto{\pgfqpoint{2.586142in}{0.918766in}}{\pgfqpoint{2.578241in}{0.922039in}}{\pgfqpoint{2.570005in}{0.922039in}}%
\pgfpathcurveto{\pgfqpoint{2.561769in}{0.922039in}}{\pgfqpoint{2.553869in}{0.918766in}}{\pgfqpoint{2.548045in}{0.912942in}}%
\pgfpathcurveto{\pgfqpoint{2.542221in}{0.907118in}}{\pgfqpoint{2.538949in}{0.899218in}}{\pgfqpoint{2.538949in}{0.890982in}}%
\pgfpathcurveto{\pgfqpoint{2.538949in}{0.882746in}}{\pgfqpoint{2.542221in}{0.874846in}}{\pgfqpoint{2.548045in}{0.869022in}}%
\pgfpathcurveto{\pgfqpoint{2.553869in}{0.863198in}}{\pgfqpoint{2.561769in}{0.859926in}}{\pgfqpoint{2.570005in}{0.859926in}}%
\pgfpathclose%
\pgfusepath{stroke,fill}%
\end{pgfscope}%
\begin{pgfscope}%
\pgfpathrectangle{\pgfqpoint{0.100000in}{0.212622in}}{\pgfqpoint{3.696000in}{3.696000in}}%
\pgfusepath{clip}%
\pgfsetbuttcap%
\pgfsetroundjoin%
\definecolor{currentfill}{rgb}{0.121569,0.466667,0.705882}%
\pgfsetfillcolor{currentfill}%
\pgfsetfillopacity{0.992827}%
\pgfsetlinewidth{1.003750pt}%
\definecolor{currentstroke}{rgb}{0.121569,0.466667,0.705882}%
\pgfsetstrokecolor{currentstroke}%
\pgfsetstrokeopacity{0.992827}%
\pgfsetdash{}{0pt}%
\pgfpathmoveto{\pgfqpoint{2.457113in}{0.798903in}}%
\pgfpathcurveto{\pgfqpoint{2.465349in}{0.798903in}}{\pgfqpoint{2.473249in}{0.802175in}}{\pgfqpoint{2.479073in}{0.807999in}}%
\pgfpathcurveto{\pgfqpoint{2.484897in}{0.813823in}}{\pgfqpoint{2.488169in}{0.821723in}}{\pgfqpoint{2.488169in}{0.829959in}}%
\pgfpathcurveto{\pgfqpoint{2.488169in}{0.838196in}}{\pgfqpoint{2.484897in}{0.846096in}}{\pgfqpoint{2.479073in}{0.851920in}}%
\pgfpathcurveto{\pgfqpoint{2.473249in}{0.857744in}}{\pgfqpoint{2.465349in}{0.861016in}}{\pgfqpoint{2.457113in}{0.861016in}}%
\pgfpathcurveto{\pgfqpoint{2.448877in}{0.861016in}}{\pgfqpoint{2.440977in}{0.857744in}}{\pgfqpoint{2.435153in}{0.851920in}}%
\pgfpathcurveto{\pgfqpoint{2.429329in}{0.846096in}}{\pgfqpoint{2.426056in}{0.838196in}}{\pgfqpoint{2.426056in}{0.829959in}}%
\pgfpathcurveto{\pgfqpoint{2.426056in}{0.821723in}}{\pgfqpoint{2.429329in}{0.813823in}}{\pgfqpoint{2.435153in}{0.807999in}}%
\pgfpathcurveto{\pgfqpoint{2.440977in}{0.802175in}}{\pgfqpoint{2.448877in}{0.798903in}}{\pgfqpoint{2.457113in}{0.798903in}}%
\pgfpathclose%
\pgfusepath{stroke,fill}%
\end{pgfscope}%
\begin{pgfscope}%
\pgfpathrectangle{\pgfqpoint{0.100000in}{0.212622in}}{\pgfqpoint{3.696000in}{3.696000in}}%
\pgfusepath{clip}%
\pgfsetbuttcap%
\pgfsetroundjoin%
\definecolor{currentfill}{rgb}{0.121569,0.466667,0.705882}%
\pgfsetfillcolor{currentfill}%
\pgfsetfillopacity{0.993023}%
\pgfsetlinewidth{1.003750pt}%
\definecolor{currentstroke}{rgb}{0.121569,0.466667,0.705882}%
\pgfsetstrokecolor{currentstroke}%
\pgfsetstrokeopacity{0.993023}%
\pgfsetdash{}{0pt}%
\pgfpathmoveto{\pgfqpoint{2.567082in}{0.856113in}}%
\pgfpathcurveto{\pgfqpoint{2.575318in}{0.856113in}}{\pgfqpoint{2.583218in}{0.859385in}}{\pgfqpoint{2.589042in}{0.865209in}}%
\pgfpathcurveto{\pgfqpoint{2.594866in}{0.871033in}}{\pgfqpoint{2.598138in}{0.878933in}}{\pgfqpoint{2.598138in}{0.887170in}}%
\pgfpathcurveto{\pgfqpoint{2.598138in}{0.895406in}}{\pgfqpoint{2.594866in}{0.903306in}}{\pgfqpoint{2.589042in}{0.909130in}}%
\pgfpathcurveto{\pgfqpoint{2.583218in}{0.914954in}}{\pgfqpoint{2.575318in}{0.918226in}}{\pgfqpoint{2.567082in}{0.918226in}}%
\pgfpathcurveto{\pgfqpoint{2.558846in}{0.918226in}}{\pgfqpoint{2.550946in}{0.914954in}}{\pgfqpoint{2.545122in}{0.909130in}}%
\pgfpathcurveto{\pgfqpoint{2.539298in}{0.903306in}}{\pgfqpoint{2.536025in}{0.895406in}}{\pgfqpoint{2.536025in}{0.887170in}}%
\pgfpathcurveto{\pgfqpoint{2.536025in}{0.878933in}}{\pgfqpoint{2.539298in}{0.871033in}}{\pgfqpoint{2.545122in}{0.865209in}}%
\pgfpathcurveto{\pgfqpoint{2.550946in}{0.859385in}}{\pgfqpoint{2.558846in}{0.856113in}}{\pgfqpoint{2.567082in}{0.856113in}}%
\pgfpathclose%
\pgfusepath{stroke,fill}%
\end{pgfscope}%
\begin{pgfscope}%
\pgfpathrectangle{\pgfqpoint{0.100000in}{0.212622in}}{\pgfqpoint{3.696000in}{3.696000in}}%
\pgfusepath{clip}%
\pgfsetbuttcap%
\pgfsetroundjoin%
\definecolor{currentfill}{rgb}{0.121569,0.466667,0.705882}%
\pgfsetfillcolor{currentfill}%
\pgfsetfillopacity{0.994077}%
\pgfsetlinewidth{1.003750pt}%
\definecolor{currentstroke}{rgb}{0.121569,0.466667,0.705882}%
\pgfsetstrokecolor{currentstroke}%
\pgfsetstrokeopacity{0.994077}%
\pgfsetdash{}{0pt}%
\pgfpathmoveto{\pgfqpoint{2.562082in}{0.848210in}}%
\pgfpathcurveto{\pgfqpoint{2.570319in}{0.848210in}}{\pgfqpoint{2.578219in}{0.851482in}}{\pgfqpoint{2.584043in}{0.857306in}}%
\pgfpathcurveto{\pgfqpoint{2.589866in}{0.863130in}}{\pgfqpoint{2.593139in}{0.871030in}}{\pgfqpoint{2.593139in}{0.879266in}}%
\pgfpathcurveto{\pgfqpoint{2.593139in}{0.887503in}}{\pgfqpoint{2.589866in}{0.895403in}}{\pgfqpoint{2.584043in}{0.901227in}}%
\pgfpathcurveto{\pgfqpoint{2.578219in}{0.907051in}}{\pgfqpoint{2.570319in}{0.910323in}}{\pgfqpoint{2.562082in}{0.910323in}}%
\pgfpathcurveto{\pgfqpoint{2.553846in}{0.910323in}}{\pgfqpoint{2.545946in}{0.907051in}}{\pgfqpoint{2.540122in}{0.901227in}}%
\pgfpathcurveto{\pgfqpoint{2.534298in}{0.895403in}}{\pgfqpoint{2.531026in}{0.887503in}}{\pgfqpoint{2.531026in}{0.879266in}}%
\pgfpathcurveto{\pgfqpoint{2.531026in}{0.871030in}}{\pgfqpoint{2.534298in}{0.863130in}}{\pgfqpoint{2.540122in}{0.857306in}}%
\pgfpathcurveto{\pgfqpoint{2.545946in}{0.851482in}}{\pgfqpoint{2.553846in}{0.848210in}}{\pgfqpoint{2.562082in}{0.848210in}}%
\pgfpathclose%
\pgfusepath{stroke,fill}%
\end{pgfscope}%
\begin{pgfscope}%
\pgfpathrectangle{\pgfqpoint{0.100000in}{0.212622in}}{\pgfqpoint{3.696000in}{3.696000in}}%
\pgfusepath{clip}%
\pgfsetbuttcap%
\pgfsetroundjoin%
\definecolor{currentfill}{rgb}{0.121569,0.466667,0.705882}%
\pgfsetfillcolor{currentfill}%
\pgfsetfillopacity{0.994325}%
\pgfsetlinewidth{1.003750pt}%
\definecolor{currentstroke}{rgb}{0.121569,0.466667,0.705882}%
\pgfsetstrokecolor{currentstroke}%
\pgfsetstrokeopacity{0.994325}%
\pgfsetdash{}{0pt}%
\pgfpathmoveto{\pgfqpoint{2.468141in}{0.794057in}}%
\pgfpathcurveto{\pgfqpoint{2.476377in}{0.794057in}}{\pgfqpoint{2.484277in}{0.797330in}}{\pgfqpoint{2.490101in}{0.803154in}}%
\pgfpathcurveto{\pgfqpoint{2.495925in}{0.808978in}}{\pgfqpoint{2.499197in}{0.816878in}}{\pgfqpoint{2.499197in}{0.825114in}}%
\pgfpathcurveto{\pgfqpoint{2.499197in}{0.833350in}}{\pgfqpoint{2.495925in}{0.841250in}}{\pgfqpoint{2.490101in}{0.847074in}}%
\pgfpathcurveto{\pgfqpoint{2.484277in}{0.852898in}}{\pgfqpoint{2.476377in}{0.856170in}}{\pgfqpoint{2.468141in}{0.856170in}}%
\pgfpathcurveto{\pgfqpoint{2.459904in}{0.856170in}}{\pgfqpoint{2.452004in}{0.852898in}}{\pgfqpoint{2.446180in}{0.847074in}}%
\pgfpathcurveto{\pgfqpoint{2.440357in}{0.841250in}}{\pgfqpoint{2.437084in}{0.833350in}}{\pgfqpoint{2.437084in}{0.825114in}}%
\pgfpathcurveto{\pgfqpoint{2.437084in}{0.816878in}}{\pgfqpoint{2.440357in}{0.808978in}}{\pgfqpoint{2.446180in}{0.803154in}}%
\pgfpathcurveto{\pgfqpoint{2.452004in}{0.797330in}}{\pgfqpoint{2.459904in}{0.794057in}}{\pgfqpoint{2.468141in}{0.794057in}}%
\pgfpathclose%
\pgfusepath{stroke,fill}%
\end{pgfscope}%
\begin{pgfscope}%
\pgfpathrectangle{\pgfqpoint{0.100000in}{0.212622in}}{\pgfqpoint{3.696000in}{3.696000in}}%
\pgfusepath{clip}%
\pgfsetbuttcap%
\pgfsetroundjoin%
\definecolor{currentfill}{rgb}{0.121569,0.466667,0.705882}%
\pgfsetfillcolor{currentfill}%
\pgfsetfillopacity{0.995382}%
\pgfsetlinewidth{1.003750pt}%
\definecolor{currentstroke}{rgb}{0.121569,0.466667,0.705882}%
\pgfsetstrokecolor{currentstroke}%
\pgfsetstrokeopacity{0.995382}%
\pgfsetdash{}{0pt}%
\pgfpathmoveto{\pgfqpoint{2.559590in}{0.841771in}}%
\pgfpathcurveto{\pgfqpoint{2.567826in}{0.841771in}}{\pgfqpoint{2.575726in}{0.845044in}}{\pgfqpoint{2.581550in}{0.850868in}}%
\pgfpathcurveto{\pgfqpoint{2.587374in}{0.856692in}}{\pgfqpoint{2.590646in}{0.864592in}}{\pgfqpoint{2.590646in}{0.872828in}}%
\pgfpathcurveto{\pgfqpoint{2.590646in}{0.881064in}}{\pgfqpoint{2.587374in}{0.888964in}}{\pgfqpoint{2.581550in}{0.894788in}}%
\pgfpathcurveto{\pgfqpoint{2.575726in}{0.900612in}}{\pgfqpoint{2.567826in}{0.903884in}}{\pgfqpoint{2.559590in}{0.903884in}}%
\pgfpathcurveto{\pgfqpoint{2.551354in}{0.903884in}}{\pgfqpoint{2.543454in}{0.900612in}}{\pgfqpoint{2.537630in}{0.894788in}}%
\pgfpathcurveto{\pgfqpoint{2.531806in}{0.888964in}}{\pgfqpoint{2.528533in}{0.881064in}}{\pgfqpoint{2.528533in}{0.872828in}}%
\pgfpathcurveto{\pgfqpoint{2.528533in}{0.864592in}}{\pgfqpoint{2.531806in}{0.856692in}}{\pgfqpoint{2.537630in}{0.850868in}}%
\pgfpathcurveto{\pgfqpoint{2.543454in}{0.845044in}}{\pgfqpoint{2.551354in}{0.841771in}}{\pgfqpoint{2.559590in}{0.841771in}}%
\pgfpathclose%
\pgfusepath{stroke,fill}%
\end{pgfscope}%
\begin{pgfscope}%
\pgfpathrectangle{\pgfqpoint{0.100000in}{0.212622in}}{\pgfqpoint{3.696000in}{3.696000in}}%
\pgfusepath{clip}%
\pgfsetbuttcap%
\pgfsetroundjoin%
\definecolor{currentfill}{rgb}{0.121569,0.466667,0.705882}%
\pgfsetfillcolor{currentfill}%
\pgfsetfillopacity{0.995898}%
\pgfsetlinewidth{1.003750pt}%
\definecolor{currentstroke}{rgb}{0.121569,0.466667,0.705882}%
\pgfsetstrokecolor{currentstroke}%
\pgfsetstrokeopacity{0.995898}%
\pgfsetdash{}{0pt}%
\pgfpathmoveto{\pgfqpoint{2.480863in}{0.790997in}}%
\pgfpathcurveto{\pgfqpoint{2.489099in}{0.790997in}}{\pgfqpoint{2.496999in}{0.794269in}}{\pgfqpoint{2.502823in}{0.800093in}}%
\pgfpathcurveto{\pgfqpoint{2.508647in}{0.805917in}}{\pgfqpoint{2.511919in}{0.813817in}}{\pgfqpoint{2.511919in}{0.822053in}}%
\pgfpathcurveto{\pgfqpoint{2.511919in}{0.830290in}}{\pgfqpoint{2.508647in}{0.838190in}}{\pgfqpoint{2.502823in}{0.844014in}}%
\pgfpathcurveto{\pgfqpoint{2.496999in}{0.849838in}}{\pgfqpoint{2.489099in}{0.853110in}}{\pgfqpoint{2.480863in}{0.853110in}}%
\pgfpathcurveto{\pgfqpoint{2.472626in}{0.853110in}}{\pgfqpoint{2.464726in}{0.849838in}}{\pgfqpoint{2.458902in}{0.844014in}}%
\pgfpathcurveto{\pgfqpoint{2.453079in}{0.838190in}}{\pgfqpoint{2.449806in}{0.830290in}}{\pgfqpoint{2.449806in}{0.822053in}}%
\pgfpathcurveto{\pgfqpoint{2.449806in}{0.813817in}}{\pgfqpoint{2.453079in}{0.805917in}}{\pgfqpoint{2.458902in}{0.800093in}}%
\pgfpathcurveto{\pgfqpoint{2.464726in}{0.794269in}}{\pgfqpoint{2.472626in}{0.790997in}}{\pgfqpoint{2.480863in}{0.790997in}}%
\pgfpathclose%
\pgfusepath{stroke,fill}%
\end{pgfscope}%
\begin{pgfscope}%
\pgfpathrectangle{\pgfqpoint{0.100000in}{0.212622in}}{\pgfqpoint{3.696000in}{3.696000in}}%
\pgfusepath{clip}%
\pgfsetbuttcap%
\pgfsetroundjoin%
\definecolor{currentfill}{rgb}{0.121569,0.466667,0.705882}%
\pgfsetfillcolor{currentfill}%
\pgfsetfillopacity{0.996395}%
\pgfsetlinewidth{1.003750pt}%
\definecolor{currentstroke}{rgb}{0.121569,0.466667,0.705882}%
\pgfsetstrokecolor{currentstroke}%
\pgfsetstrokeopacity{0.996395}%
\pgfsetdash{}{0pt}%
\pgfpathmoveto{\pgfqpoint{2.558007in}{0.834608in}}%
\pgfpathcurveto{\pgfqpoint{2.566243in}{0.834608in}}{\pgfqpoint{2.574143in}{0.837880in}}{\pgfqpoint{2.579967in}{0.843704in}}%
\pgfpathcurveto{\pgfqpoint{2.585791in}{0.849528in}}{\pgfqpoint{2.589063in}{0.857428in}}{\pgfqpoint{2.589063in}{0.865664in}}%
\pgfpathcurveto{\pgfqpoint{2.589063in}{0.873901in}}{\pgfqpoint{2.585791in}{0.881801in}}{\pgfqpoint{2.579967in}{0.887624in}}%
\pgfpathcurveto{\pgfqpoint{2.574143in}{0.893448in}}{\pgfqpoint{2.566243in}{0.896721in}}{\pgfqpoint{2.558007in}{0.896721in}}%
\pgfpathcurveto{\pgfqpoint{2.549770in}{0.896721in}}{\pgfqpoint{2.541870in}{0.893448in}}{\pgfqpoint{2.536046in}{0.887624in}}%
\pgfpathcurveto{\pgfqpoint{2.530223in}{0.881801in}}{\pgfqpoint{2.526950in}{0.873901in}}{\pgfqpoint{2.526950in}{0.865664in}}%
\pgfpathcurveto{\pgfqpoint{2.526950in}{0.857428in}}{\pgfqpoint{2.530223in}{0.849528in}}{\pgfqpoint{2.536046in}{0.843704in}}%
\pgfpathcurveto{\pgfqpoint{2.541870in}{0.837880in}}{\pgfqpoint{2.549770in}{0.834608in}}{\pgfqpoint{2.558007in}{0.834608in}}%
\pgfpathclose%
\pgfusepath{stroke,fill}%
\end{pgfscope}%
\begin{pgfscope}%
\pgfpathrectangle{\pgfqpoint{0.100000in}{0.212622in}}{\pgfqpoint{3.696000in}{3.696000in}}%
\pgfusepath{clip}%
\pgfsetbuttcap%
\pgfsetroundjoin%
\definecolor{currentfill}{rgb}{0.121569,0.466667,0.705882}%
\pgfsetfillcolor{currentfill}%
\pgfsetfillopacity{0.997316}%
\pgfsetlinewidth{1.003750pt}%
\definecolor{currentstroke}{rgb}{0.121569,0.466667,0.705882}%
\pgfsetstrokecolor{currentstroke}%
\pgfsetstrokeopacity{0.997316}%
\pgfsetdash{}{0pt}%
\pgfpathmoveto{\pgfqpoint{2.555401in}{0.829162in}}%
\pgfpathcurveto{\pgfqpoint{2.563638in}{0.829162in}}{\pgfqpoint{2.571538in}{0.832434in}}{\pgfqpoint{2.577362in}{0.838258in}}%
\pgfpathcurveto{\pgfqpoint{2.583185in}{0.844082in}}{\pgfqpoint{2.586458in}{0.851982in}}{\pgfqpoint{2.586458in}{0.860219in}}%
\pgfpathcurveto{\pgfqpoint{2.586458in}{0.868455in}}{\pgfqpoint{2.583185in}{0.876355in}}{\pgfqpoint{2.577362in}{0.882179in}}%
\pgfpathcurveto{\pgfqpoint{2.571538in}{0.888003in}}{\pgfqpoint{2.563638in}{0.891275in}}{\pgfqpoint{2.555401in}{0.891275in}}%
\pgfpathcurveto{\pgfqpoint{2.547165in}{0.891275in}}{\pgfqpoint{2.539265in}{0.888003in}}{\pgfqpoint{2.533441in}{0.882179in}}%
\pgfpathcurveto{\pgfqpoint{2.527617in}{0.876355in}}{\pgfqpoint{2.524345in}{0.868455in}}{\pgfqpoint{2.524345in}{0.860219in}}%
\pgfpathcurveto{\pgfqpoint{2.524345in}{0.851982in}}{\pgfqpoint{2.527617in}{0.844082in}}{\pgfqpoint{2.533441in}{0.838258in}}%
\pgfpathcurveto{\pgfqpoint{2.539265in}{0.832434in}}{\pgfqpoint{2.547165in}{0.829162in}}{\pgfqpoint{2.555401in}{0.829162in}}%
\pgfpathclose%
\pgfusepath{stroke,fill}%
\end{pgfscope}%
\begin{pgfscope}%
\pgfpathrectangle{\pgfqpoint{0.100000in}{0.212622in}}{\pgfqpoint{3.696000in}{3.696000in}}%
\pgfusepath{clip}%
\pgfsetbuttcap%
\pgfsetroundjoin%
\definecolor{currentfill}{rgb}{0.121569,0.466667,0.705882}%
\pgfsetfillcolor{currentfill}%
\pgfsetfillopacity{0.997511}%
\pgfsetlinewidth{1.003750pt}%
\definecolor{currentstroke}{rgb}{0.121569,0.466667,0.705882}%
\pgfsetstrokecolor{currentstroke}%
\pgfsetstrokeopacity{0.997511}%
\pgfsetdash{}{0pt}%
\pgfpathmoveto{\pgfqpoint{2.493979in}{0.788124in}}%
\pgfpathcurveto{\pgfqpoint{2.502215in}{0.788124in}}{\pgfqpoint{2.510116in}{0.791396in}}{\pgfqpoint{2.515939in}{0.797220in}}%
\pgfpathcurveto{\pgfqpoint{2.521763in}{0.803044in}}{\pgfqpoint{2.525036in}{0.810944in}}{\pgfqpoint{2.525036in}{0.819180in}}%
\pgfpathcurveto{\pgfqpoint{2.525036in}{0.827417in}}{\pgfqpoint{2.521763in}{0.835317in}}{\pgfqpoint{2.515939in}{0.841141in}}%
\pgfpathcurveto{\pgfqpoint{2.510116in}{0.846965in}}{\pgfqpoint{2.502215in}{0.850237in}}{\pgfqpoint{2.493979in}{0.850237in}}%
\pgfpathcurveto{\pgfqpoint{2.485743in}{0.850237in}}{\pgfqpoint{2.477843in}{0.846965in}}{\pgfqpoint{2.472019in}{0.841141in}}%
\pgfpathcurveto{\pgfqpoint{2.466195in}{0.835317in}}{\pgfqpoint{2.462923in}{0.827417in}}{\pgfqpoint{2.462923in}{0.819180in}}%
\pgfpathcurveto{\pgfqpoint{2.462923in}{0.810944in}}{\pgfqpoint{2.466195in}{0.803044in}}{\pgfqpoint{2.472019in}{0.797220in}}%
\pgfpathcurveto{\pgfqpoint{2.477843in}{0.791396in}}{\pgfqpoint{2.485743in}{0.788124in}}{\pgfqpoint{2.493979in}{0.788124in}}%
\pgfpathclose%
\pgfusepath{stroke,fill}%
\end{pgfscope}%
\begin{pgfscope}%
\pgfpathrectangle{\pgfqpoint{0.100000in}{0.212622in}}{\pgfqpoint{3.696000in}{3.696000in}}%
\pgfusepath{clip}%
\pgfsetbuttcap%
\pgfsetroundjoin%
\definecolor{currentfill}{rgb}{0.121569,0.466667,0.705882}%
\pgfsetfillcolor{currentfill}%
\pgfsetfillopacity{0.997793}%
\pgfsetlinewidth{1.003750pt}%
\definecolor{currentstroke}{rgb}{0.121569,0.466667,0.705882}%
\pgfsetstrokecolor{currentstroke}%
\pgfsetstrokeopacity{0.997793}%
\pgfsetdash{}{0pt}%
\pgfpathmoveto{\pgfqpoint{2.552917in}{0.825882in}}%
\pgfpathcurveto{\pgfqpoint{2.561154in}{0.825882in}}{\pgfqpoint{2.569054in}{0.829155in}}{\pgfqpoint{2.574878in}{0.834979in}}%
\pgfpathcurveto{\pgfqpoint{2.580702in}{0.840803in}}{\pgfqpoint{2.583974in}{0.848703in}}{\pgfqpoint{2.583974in}{0.856939in}}%
\pgfpathcurveto{\pgfqpoint{2.583974in}{0.865175in}}{\pgfqpoint{2.580702in}{0.873075in}}{\pgfqpoint{2.574878in}{0.878899in}}%
\pgfpathcurveto{\pgfqpoint{2.569054in}{0.884723in}}{\pgfqpoint{2.561154in}{0.887995in}}{\pgfqpoint{2.552917in}{0.887995in}}%
\pgfpathcurveto{\pgfqpoint{2.544681in}{0.887995in}}{\pgfqpoint{2.536781in}{0.884723in}}{\pgfqpoint{2.530957in}{0.878899in}}%
\pgfpathcurveto{\pgfqpoint{2.525133in}{0.873075in}}{\pgfqpoint{2.521861in}{0.865175in}}{\pgfqpoint{2.521861in}{0.856939in}}%
\pgfpathcurveto{\pgfqpoint{2.521861in}{0.848703in}}{\pgfqpoint{2.525133in}{0.840803in}}{\pgfqpoint{2.530957in}{0.834979in}}%
\pgfpathcurveto{\pgfqpoint{2.536781in}{0.829155in}}{\pgfqpoint{2.544681in}{0.825882in}}{\pgfqpoint{2.552917in}{0.825882in}}%
\pgfpathclose%
\pgfusepath{stroke,fill}%
\end{pgfscope}%
\begin{pgfscope}%
\pgfpathrectangle{\pgfqpoint{0.100000in}{0.212622in}}{\pgfqpoint{3.696000in}{3.696000in}}%
\pgfusepath{clip}%
\pgfsetbuttcap%
\pgfsetroundjoin%
\definecolor{currentfill}{rgb}{0.121569,0.466667,0.705882}%
\pgfsetfillcolor{currentfill}%
\pgfsetfillopacity{0.998196}%
\pgfsetlinewidth{1.003750pt}%
\definecolor{currentstroke}{rgb}{0.121569,0.466667,0.705882}%
\pgfsetstrokecolor{currentstroke}%
\pgfsetstrokeopacity{0.998196}%
\pgfsetdash{}{0pt}%
\pgfpathmoveto{\pgfqpoint{2.551731in}{0.823757in}}%
\pgfpathcurveto{\pgfqpoint{2.559967in}{0.823757in}}{\pgfqpoint{2.567867in}{0.827030in}}{\pgfqpoint{2.573691in}{0.832854in}}%
\pgfpathcurveto{\pgfqpoint{2.579515in}{0.838678in}}{\pgfqpoint{2.582788in}{0.846578in}}{\pgfqpoint{2.582788in}{0.854814in}}%
\pgfpathcurveto{\pgfqpoint{2.582788in}{0.863050in}}{\pgfqpoint{2.579515in}{0.870950in}}{\pgfqpoint{2.573691in}{0.876774in}}%
\pgfpathcurveto{\pgfqpoint{2.567867in}{0.882598in}}{\pgfqpoint{2.559967in}{0.885870in}}{\pgfqpoint{2.551731in}{0.885870in}}%
\pgfpathcurveto{\pgfqpoint{2.543495in}{0.885870in}}{\pgfqpoint{2.535595in}{0.882598in}}{\pgfqpoint{2.529771in}{0.876774in}}%
\pgfpathcurveto{\pgfqpoint{2.523947in}{0.870950in}}{\pgfqpoint{2.520675in}{0.863050in}}{\pgfqpoint{2.520675in}{0.854814in}}%
\pgfpathcurveto{\pgfqpoint{2.520675in}{0.846578in}}{\pgfqpoint{2.523947in}{0.838678in}}{\pgfqpoint{2.529771in}{0.832854in}}%
\pgfpathcurveto{\pgfqpoint{2.535595in}{0.827030in}}{\pgfqpoint{2.543495in}{0.823757in}}{\pgfqpoint{2.551731in}{0.823757in}}%
\pgfpathclose%
\pgfusepath{stroke,fill}%
\end{pgfscope}%
\begin{pgfscope}%
\pgfpathrectangle{\pgfqpoint{0.100000in}{0.212622in}}{\pgfqpoint{3.696000in}{3.696000in}}%
\pgfusepath{clip}%
\pgfsetbuttcap%
\pgfsetroundjoin%
\definecolor{currentfill}{rgb}{0.121569,0.466667,0.705882}%
\pgfsetfillcolor{currentfill}%
\pgfsetfillopacity{0.998269}%
\pgfsetlinewidth{1.003750pt}%
\definecolor{currentstroke}{rgb}{0.121569,0.466667,0.705882}%
\pgfsetstrokecolor{currentstroke}%
\pgfsetstrokeopacity{0.998269}%
\pgfsetdash{}{0pt}%
\pgfpathmoveto{\pgfqpoint{2.551642in}{0.823322in}}%
\pgfpathcurveto{\pgfqpoint{2.559879in}{0.823322in}}{\pgfqpoint{2.567779in}{0.826595in}}{\pgfqpoint{2.573603in}{0.832418in}}%
\pgfpathcurveto{\pgfqpoint{2.579427in}{0.838242in}}{\pgfqpoint{2.582699in}{0.846142in}}{\pgfqpoint{2.582699in}{0.854379in}}%
\pgfpathcurveto{\pgfqpoint{2.582699in}{0.862615in}}{\pgfqpoint{2.579427in}{0.870515in}}{\pgfqpoint{2.573603in}{0.876339in}}%
\pgfpathcurveto{\pgfqpoint{2.567779in}{0.882163in}}{\pgfqpoint{2.559879in}{0.885435in}}{\pgfqpoint{2.551642in}{0.885435in}}%
\pgfpathcurveto{\pgfqpoint{2.543406in}{0.885435in}}{\pgfqpoint{2.535506in}{0.882163in}}{\pgfqpoint{2.529682in}{0.876339in}}%
\pgfpathcurveto{\pgfqpoint{2.523858in}{0.870515in}}{\pgfqpoint{2.520586in}{0.862615in}}{\pgfqpoint{2.520586in}{0.854379in}}%
\pgfpathcurveto{\pgfqpoint{2.520586in}{0.846142in}}{\pgfqpoint{2.523858in}{0.838242in}}{\pgfqpoint{2.529682in}{0.832418in}}%
\pgfpathcurveto{\pgfqpoint{2.535506in}{0.826595in}}{\pgfqpoint{2.543406in}{0.823322in}}{\pgfqpoint{2.551642in}{0.823322in}}%
\pgfpathclose%
\pgfusepath{stroke,fill}%
\end{pgfscope}%
\begin{pgfscope}%
\pgfpathrectangle{\pgfqpoint{0.100000in}{0.212622in}}{\pgfqpoint{3.696000in}{3.696000in}}%
\pgfusepath{clip}%
\pgfsetbuttcap%
\pgfsetroundjoin%
\definecolor{currentfill}{rgb}{0.121569,0.466667,0.705882}%
\pgfsetfillcolor{currentfill}%
\pgfsetfillopacity{0.998404}%
\pgfsetlinewidth{1.003750pt}%
\definecolor{currentstroke}{rgb}{0.121569,0.466667,0.705882}%
\pgfsetstrokecolor{currentstroke}%
\pgfsetstrokeopacity{0.998404}%
\pgfsetdash{}{0pt}%
\pgfpathmoveto{\pgfqpoint{2.551359in}{0.822584in}}%
\pgfpathcurveto{\pgfqpoint{2.559596in}{0.822584in}}{\pgfqpoint{2.567496in}{0.825856in}}{\pgfqpoint{2.573320in}{0.831680in}}%
\pgfpathcurveto{\pgfqpoint{2.579144in}{0.837504in}}{\pgfqpoint{2.582416in}{0.845404in}}{\pgfqpoint{2.582416in}{0.853641in}}%
\pgfpathcurveto{\pgfqpoint{2.582416in}{0.861877in}}{\pgfqpoint{2.579144in}{0.869777in}}{\pgfqpoint{2.573320in}{0.875601in}}%
\pgfpathcurveto{\pgfqpoint{2.567496in}{0.881425in}}{\pgfqpoint{2.559596in}{0.884697in}}{\pgfqpoint{2.551359in}{0.884697in}}%
\pgfpathcurveto{\pgfqpoint{2.543123in}{0.884697in}}{\pgfqpoint{2.535223in}{0.881425in}}{\pgfqpoint{2.529399in}{0.875601in}}%
\pgfpathcurveto{\pgfqpoint{2.523575in}{0.869777in}}{\pgfqpoint{2.520303in}{0.861877in}}{\pgfqpoint{2.520303in}{0.853641in}}%
\pgfpathcurveto{\pgfqpoint{2.520303in}{0.845404in}}{\pgfqpoint{2.523575in}{0.837504in}}{\pgfqpoint{2.529399in}{0.831680in}}%
\pgfpathcurveto{\pgfqpoint{2.535223in}{0.825856in}}{\pgfqpoint{2.543123in}{0.822584in}}{\pgfqpoint{2.551359in}{0.822584in}}%
\pgfpathclose%
\pgfusepath{stroke,fill}%
\end{pgfscope}%
\begin{pgfscope}%
\pgfpathrectangle{\pgfqpoint{0.100000in}{0.212622in}}{\pgfqpoint{3.696000in}{3.696000in}}%
\pgfusepath{clip}%
\pgfsetbuttcap%
\pgfsetroundjoin%
\definecolor{currentfill}{rgb}{0.121569,0.466667,0.705882}%
\pgfsetfillcolor{currentfill}%
\pgfsetfillopacity{0.998404}%
\pgfsetlinewidth{1.003750pt}%
\definecolor{currentstroke}{rgb}{0.121569,0.466667,0.705882}%
\pgfsetstrokecolor{currentstroke}%
\pgfsetstrokeopacity{0.998404}%
\pgfsetdash{}{0pt}%
\pgfpathmoveto{\pgfqpoint{2.551359in}{0.822584in}}%
\pgfpathcurveto{\pgfqpoint{2.559596in}{0.822584in}}{\pgfqpoint{2.567496in}{0.825856in}}{\pgfqpoint{2.573320in}{0.831680in}}%
\pgfpathcurveto{\pgfqpoint{2.579144in}{0.837504in}}{\pgfqpoint{2.582416in}{0.845404in}}{\pgfqpoint{2.582416in}{0.853641in}}%
\pgfpathcurveto{\pgfqpoint{2.582416in}{0.861877in}}{\pgfqpoint{2.579144in}{0.869777in}}{\pgfqpoint{2.573320in}{0.875601in}}%
\pgfpathcurveto{\pgfqpoint{2.567496in}{0.881425in}}{\pgfqpoint{2.559596in}{0.884697in}}{\pgfqpoint{2.551359in}{0.884697in}}%
\pgfpathcurveto{\pgfqpoint{2.543123in}{0.884697in}}{\pgfqpoint{2.535223in}{0.881425in}}{\pgfqpoint{2.529399in}{0.875601in}}%
\pgfpathcurveto{\pgfqpoint{2.523575in}{0.869777in}}{\pgfqpoint{2.520303in}{0.861877in}}{\pgfqpoint{2.520303in}{0.853641in}}%
\pgfpathcurveto{\pgfqpoint{2.520303in}{0.845404in}}{\pgfqpoint{2.523575in}{0.837504in}}{\pgfqpoint{2.529399in}{0.831680in}}%
\pgfpathcurveto{\pgfqpoint{2.535223in}{0.825856in}}{\pgfqpoint{2.543123in}{0.822584in}}{\pgfqpoint{2.551359in}{0.822584in}}%
\pgfpathclose%
\pgfusepath{stroke,fill}%
\end{pgfscope}%
\begin{pgfscope}%
\pgfpathrectangle{\pgfqpoint{0.100000in}{0.212622in}}{\pgfqpoint{3.696000in}{3.696000in}}%
\pgfusepath{clip}%
\pgfsetbuttcap%
\pgfsetroundjoin%
\definecolor{currentfill}{rgb}{0.121569,0.466667,0.705882}%
\pgfsetfillcolor{currentfill}%
\pgfsetfillopacity{0.998404}%
\pgfsetlinewidth{1.003750pt}%
\definecolor{currentstroke}{rgb}{0.121569,0.466667,0.705882}%
\pgfsetstrokecolor{currentstroke}%
\pgfsetstrokeopacity{0.998404}%
\pgfsetdash{}{0pt}%
\pgfpathmoveto{\pgfqpoint{2.551359in}{0.822584in}}%
\pgfpathcurveto{\pgfqpoint{2.559596in}{0.822584in}}{\pgfqpoint{2.567496in}{0.825856in}}{\pgfqpoint{2.573320in}{0.831680in}}%
\pgfpathcurveto{\pgfqpoint{2.579144in}{0.837504in}}{\pgfqpoint{2.582416in}{0.845404in}}{\pgfqpoint{2.582416in}{0.853641in}}%
\pgfpathcurveto{\pgfqpoint{2.582416in}{0.861877in}}{\pgfqpoint{2.579144in}{0.869777in}}{\pgfqpoint{2.573320in}{0.875601in}}%
\pgfpathcurveto{\pgfqpoint{2.567496in}{0.881425in}}{\pgfqpoint{2.559596in}{0.884697in}}{\pgfqpoint{2.551359in}{0.884697in}}%
\pgfpathcurveto{\pgfqpoint{2.543123in}{0.884697in}}{\pgfqpoint{2.535223in}{0.881425in}}{\pgfqpoint{2.529399in}{0.875601in}}%
\pgfpathcurveto{\pgfqpoint{2.523575in}{0.869777in}}{\pgfqpoint{2.520303in}{0.861877in}}{\pgfqpoint{2.520303in}{0.853641in}}%
\pgfpathcurveto{\pgfqpoint{2.520303in}{0.845404in}}{\pgfqpoint{2.523575in}{0.837504in}}{\pgfqpoint{2.529399in}{0.831680in}}%
\pgfpathcurveto{\pgfqpoint{2.535223in}{0.825856in}}{\pgfqpoint{2.543123in}{0.822584in}}{\pgfqpoint{2.551359in}{0.822584in}}%
\pgfpathclose%
\pgfusepath{stroke,fill}%
\end{pgfscope}%
\begin{pgfscope}%
\pgfpathrectangle{\pgfqpoint{0.100000in}{0.212622in}}{\pgfqpoint{3.696000in}{3.696000in}}%
\pgfusepath{clip}%
\pgfsetbuttcap%
\pgfsetroundjoin%
\definecolor{currentfill}{rgb}{0.121569,0.466667,0.705882}%
\pgfsetfillcolor{currentfill}%
\pgfsetfillopacity{0.998404}%
\pgfsetlinewidth{1.003750pt}%
\definecolor{currentstroke}{rgb}{0.121569,0.466667,0.705882}%
\pgfsetstrokecolor{currentstroke}%
\pgfsetstrokeopacity{0.998404}%
\pgfsetdash{}{0pt}%
\pgfpathmoveto{\pgfqpoint{2.551359in}{0.822584in}}%
\pgfpathcurveto{\pgfqpoint{2.559596in}{0.822584in}}{\pgfqpoint{2.567496in}{0.825856in}}{\pgfqpoint{2.573320in}{0.831680in}}%
\pgfpathcurveto{\pgfqpoint{2.579144in}{0.837504in}}{\pgfqpoint{2.582416in}{0.845404in}}{\pgfqpoint{2.582416in}{0.853641in}}%
\pgfpathcurveto{\pgfqpoint{2.582416in}{0.861877in}}{\pgfqpoint{2.579144in}{0.869777in}}{\pgfqpoint{2.573320in}{0.875601in}}%
\pgfpathcurveto{\pgfqpoint{2.567496in}{0.881425in}}{\pgfqpoint{2.559596in}{0.884697in}}{\pgfqpoint{2.551359in}{0.884697in}}%
\pgfpathcurveto{\pgfqpoint{2.543123in}{0.884697in}}{\pgfqpoint{2.535223in}{0.881425in}}{\pgfqpoint{2.529399in}{0.875601in}}%
\pgfpathcurveto{\pgfqpoint{2.523575in}{0.869777in}}{\pgfqpoint{2.520303in}{0.861877in}}{\pgfqpoint{2.520303in}{0.853641in}}%
\pgfpathcurveto{\pgfqpoint{2.520303in}{0.845404in}}{\pgfqpoint{2.523575in}{0.837504in}}{\pgfqpoint{2.529399in}{0.831680in}}%
\pgfpathcurveto{\pgfqpoint{2.535223in}{0.825856in}}{\pgfqpoint{2.543123in}{0.822584in}}{\pgfqpoint{2.551359in}{0.822584in}}%
\pgfpathclose%
\pgfusepath{stroke,fill}%
\end{pgfscope}%
\begin{pgfscope}%
\pgfpathrectangle{\pgfqpoint{0.100000in}{0.212622in}}{\pgfqpoint{3.696000in}{3.696000in}}%
\pgfusepath{clip}%
\pgfsetbuttcap%
\pgfsetroundjoin%
\definecolor{currentfill}{rgb}{0.121569,0.466667,0.705882}%
\pgfsetfillcolor{currentfill}%
\pgfsetfillopacity{0.998404}%
\pgfsetlinewidth{1.003750pt}%
\definecolor{currentstroke}{rgb}{0.121569,0.466667,0.705882}%
\pgfsetstrokecolor{currentstroke}%
\pgfsetstrokeopacity{0.998404}%
\pgfsetdash{}{0pt}%
\pgfpathmoveto{\pgfqpoint{2.551359in}{0.822584in}}%
\pgfpathcurveto{\pgfqpoint{2.559596in}{0.822584in}}{\pgfqpoint{2.567496in}{0.825856in}}{\pgfqpoint{2.573320in}{0.831680in}}%
\pgfpathcurveto{\pgfqpoint{2.579144in}{0.837504in}}{\pgfqpoint{2.582416in}{0.845404in}}{\pgfqpoint{2.582416in}{0.853641in}}%
\pgfpathcurveto{\pgfqpoint{2.582416in}{0.861877in}}{\pgfqpoint{2.579144in}{0.869777in}}{\pgfqpoint{2.573320in}{0.875601in}}%
\pgfpathcurveto{\pgfqpoint{2.567496in}{0.881425in}}{\pgfqpoint{2.559596in}{0.884697in}}{\pgfqpoint{2.551359in}{0.884697in}}%
\pgfpathcurveto{\pgfqpoint{2.543123in}{0.884697in}}{\pgfqpoint{2.535223in}{0.881425in}}{\pgfqpoint{2.529399in}{0.875601in}}%
\pgfpathcurveto{\pgfqpoint{2.523575in}{0.869777in}}{\pgfqpoint{2.520303in}{0.861877in}}{\pgfqpoint{2.520303in}{0.853641in}}%
\pgfpathcurveto{\pgfqpoint{2.520303in}{0.845404in}}{\pgfqpoint{2.523575in}{0.837504in}}{\pgfqpoint{2.529399in}{0.831680in}}%
\pgfpathcurveto{\pgfqpoint{2.535223in}{0.825856in}}{\pgfqpoint{2.543123in}{0.822584in}}{\pgfqpoint{2.551359in}{0.822584in}}%
\pgfpathclose%
\pgfusepath{stroke,fill}%
\end{pgfscope}%
\begin{pgfscope}%
\pgfpathrectangle{\pgfqpoint{0.100000in}{0.212622in}}{\pgfqpoint{3.696000in}{3.696000in}}%
\pgfusepath{clip}%
\pgfsetbuttcap%
\pgfsetroundjoin%
\definecolor{currentfill}{rgb}{0.121569,0.466667,0.705882}%
\pgfsetfillcolor{currentfill}%
\pgfsetfillopacity{0.998404}%
\pgfsetlinewidth{1.003750pt}%
\definecolor{currentstroke}{rgb}{0.121569,0.466667,0.705882}%
\pgfsetstrokecolor{currentstroke}%
\pgfsetstrokeopacity{0.998404}%
\pgfsetdash{}{0pt}%
\pgfpathmoveto{\pgfqpoint{2.551359in}{0.822584in}}%
\pgfpathcurveto{\pgfqpoint{2.559596in}{0.822584in}}{\pgfqpoint{2.567496in}{0.825856in}}{\pgfqpoint{2.573320in}{0.831680in}}%
\pgfpathcurveto{\pgfqpoint{2.579144in}{0.837504in}}{\pgfqpoint{2.582416in}{0.845404in}}{\pgfqpoint{2.582416in}{0.853641in}}%
\pgfpathcurveto{\pgfqpoint{2.582416in}{0.861877in}}{\pgfqpoint{2.579144in}{0.869777in}}{\pgfqpoint{2.573320in}{0.875601in}}%
\pgfpathcurveto{\pgfqpoint{2.567496in}{0.881425in}}{\pgfqpoint{2.559596in}{0.884697in}}{\pgfqpoint{2.551359in}{0.884697in}}%
\pgfpathcurveto{\pgfqpoint{2.543123in}{0.884697in}}{\pgfqpoint{2.535223in}{0.881425in}}{\pgfqpoint{2.529399in}{0.875601in}}%
\pgfpathcurveto{\pgfqpoint{2.523575in}{0.869777in}}{\pgfqpoint{2.520303in}{0.861877in}}{\pgfqpoint{2.520303in}{0.853641in}}%
\pgfpathcurveto{\pgfqpoint{2.520303in}{0.845404in}}{\pgfqpoint{2.523575in}{0.837504in}}{\pgfqpoint{2.529399in}{0.831680in}}%
\pgfpathcurveto{\pgfqpoint{2.535223in}{0.825856in}}{\pgfqpoint{2.543123in}{0.822584in}}{\pgfqpoint{2.551359in}{0.822584in}}%
\pgfpathclose%
\pgfusepath{stroke,fill}%
\end{pgfscope}%
\begin{pgfscope}%
\pgfpathrectangle{\pgfqpoint{0.100000in}{0.212622in}}{\pgfqpoint{3.696000in}{3.696000in}}%
\pgfusepath{clip}%
\pgfsetbuttcap%
\pgfsetroundjoin%
\definecolor{currentfill}{rgb}{0.121569,0.466667,0.705882}%
\pgfsetfillcolor{currentfill}%
\pgfsetfillopacity{0.998404}%
\pgfsetlinewidth{1.003750pt}%
\definecolor{currentstroke}{rgb}{0.121569,0.466667,0.705882}%
\pgfsetstrokecolor{currentstroke}%
\pgfsetstrokeopacity{0.998404}%
\pgfsetdash{}{0pt}%
\pgfpathmoveto{\pgfqpoint{2.551359in}{0.822584in}}%
\pgfpathcurveto{\pgfqpoint{2.559596in}{0.822584in}}{\pgfqpoint{2.567496in}{0.825856in}}{\pgfqpoint{2.573320in}{0.831680in}}%
\pgfpathcurveto{\pgfqpoint{2.579144in}{0.837504in}}{\pgfqpoint{2.582416in}{0.845404in}}{\pgfqpoint{2.582416in}{0.853641in}}%
\pgfpathcurveto{\pgfqpoint{2.582416in}{0.861877in}}{\pgfqpoint{2.579144in}{0.869777in}}{\pgfqpoint{2.573320in}{0.875601in}}%
\pgfpathcurveto{\pgfqpoint{2.567496in}{0.881425in}}{\pgfqpoint{2.559596in}{0.884697in}}{\pgfqpoint{2.551359in}{0.884697in}}%
\pgfpathcurveto{\pgfqpoint{2.543123in}{0.884697in}}{\pgfqpoint{2.535223in}{0.881425in}}{\pgfqpoint{2.529399in}{0.875601in}}%
\pgfpathcurveto{\pgfqpoint{2.523575in}{0.869777in}}{\pgfqpoint{2.520303in}{0.861877in}}{\pgfqpoint{2.520303in}{0.853641in}}%
\pgfpathcurveto{\pgfqpoint{2.520303in}{0.845404in}}{\pgfqpoint{2.523575in}{0.837504in}}{\pgfqpoint{2.529399in}{0.831680in}}%
\pgfpathcurveto{\pgfqpoint{2.535223in}{0.825856in}}{\pgfqpoint{2.543123in}{0.822584in}}{\pgfqpoint{2.551359in}{0.822584in}}%
\pgfpathclose%
\pgfusepath{stroke,fill}%
\end{pgfscope}%
\begin{pgfscope}%
\pgfpathrectangle{\pgfqpoint{0.100000in}{0.212622in}}{\pgfqpoint{3.696000in}{3.696000in}}%
\pgfusepath{clip}%
\pgfsetbuttcap%
\pgfsetroundjoin%
\definecolor{currentfill}{rgb}{0.121569,0.466667,0.705882}%
\pgfsetfillcolor{currentfill}%
\pgfsetfillopacity{0.998404}%
\pgfsetlinewidth{1.003750pt}%
\definecolor{currentstroke}{rgb}{0.121569,0.466667,0.705882}%
\pgfsetstrokecolor{currentstroke}%
\pgfsetstrokeopacity{0.998404}%
\pgfsetdash{}{0pt}%
\pgfpathmoveto{\pgfqpoint{2.551359in}{0.822584in}}%
\pgfpathcurveto{\pgfqpoint{2.559596in}{0.822584in}}{\pgfqpoint{2.567496in}{0.825856in}}{\pgfqpoint{2.573320in}{0.831680in}}%
\pgfpathcurveto{\pgfqpoint{2.579144in}{0.837504in}}{\pgfqpoint{2.582416in}{0.845404in}}{\pgfqpoint{2.582416in}{0.853641in}}%
\pgfpathcurveto{\pgfqpoint{2.582416in}{0.861877in}}{\pgfqpoint{2.579144in}{0.869777in}}{\pgfqpoint{2.573320in}{0.875601in}}%
\pgfpathcurveto{\pgfqpoint{2.567496in}{0.881425in}}{\pgfqpoint{2.559596in}{0.884697in}}{\pgfqpoint{2.551359in}{0.884697in}}%
\pgfpathcurveto{\pgfqpoint{2.543123in}{0.884697in}}{\pgfqpoint{2.535223in}{0.881425in}}{\pgfqpoint{2.529399in}{0.875601in}}%
\pgfpathcurveto{\pgfqpoint{2.523575in}{0.869777in}}{\pgfqpoint{2.520303in}{0.861877in}}{\pgfqpoint{2.520303in}{0.853641in}}%
\pgfpathcurveto{\pgfqpoint{2.520303in}{0.845404in}}{\pgfqpoint{2.523575in}{0.837504in}}{\pgfqpoint{2.529399in}{0.831680in}}%
\pgfpathcurveto{\pgfqpoint{2.535223in}{0.825856in}}{\pgfqpoint{2.543123in}{0.822584in}}{\pgfqpoint{2.551359in}{0.822584in}}%
\pgfpathclose%
\pgfusepath{stroke,fill}%
\end{pgfscope}%
\begin{pgfscope}%
\pgfpathrectangle{\pgfqpoint{0.100000in}{0.212622in}}{\pgfqpoint{3.696000in}{3.696000in}}%
\pgfusepath{clip}%
\pgfsetbuttcap%
\pgfsetroundjoin%
\definecolor{currentfill}{rgb}{0.121569,0.466667,0.705882}%
\pgfsetfillcolor{currentfill}%
\pgfsetfillopacity{0.998404}%
\pgfsetlinewidth{1.003750pt}%
\definecolor{currentstroke}{rgb}{0.121569,0.466667,0.705882}%
\pgfsetstrokecolor{currentstroke}%
\pgfsetstrokeopacity{0.998404}%
\pgfsetdash{}{0pt}%
\pgfpathmoveto{\pgfqpoint{2.551359in}{0.822584in}}%
\pgfpathcurveto{\pgfqpoint{2.559596in}{0.822584in}}{\pgfqpoint{2.567496in}{0.825856in}}{\pgfqpoint{2.573320in}{0.831680in}}%
\pgfpathcurveto{\pgfqpoint{2.579144in}{0.837504in}}{\pgfqpoint{2.582416in}{0.845404in}}{\pgfqpoint{2.582416in}{0.853641in}}%
\pgfpathcurveto{\pgfqpoint{2.582416in}{0.861877in}}{\pgfqpoint{2.579144in}{0.869777in}}{\pgfqpoint{2.573320in}{0.875601in}}%
\pgfpathcurveto{\pgfqpoint{2.567496in}{0.881425in}}{\pgfqpoint{2.559596in}{0.884697in}}{\pgfqpoint{2.551359in}{0.884697in}}%
\pgfpathcurveto{\pgfqpoint{2.543123in}{0.884697in}}{\pgfqpoint{2.535223in}{0.881425in}}{\pgfqpoint{2.529399in}{0.875601in}}%
\pgfpathcurveto{\pgfqpoint{2.523575in}{0.869777in}}{\pgfqpoint{2.520303in}{0.861877in}}{\pgfqpoint{2.520303in}{0.853641in}}%
\pgfpathcurveto{\pgfqpoint{2.520303in}{0.845404in}}{\pgfqpoint{2.523575in}{0.837504in}}{\pgfqpoint{2.529399in}{0.831680in}}%
\pgfpathcurveto{\pgfqpoint{2.535223in}{0.825856in}}{\pgfqpoint{2.543123in}{0.822584in}}{\pgfqpoint{2.551359in}{0.822584in}}%
\pgfpathclose%
\pgfusepath{stroke,fill}%
\end{pgfscope}%
\begin{pgfscope}%
\pgfpathrectangle{\pgfqpoint{0.100000in}{0.212622in}}{\pgfqpoint{3.696000in}{3.696000in}}%
\pgfusepath{clip}%
\pgfsetbuttcap%
\pgfsetroundjoin%
\definecolor{currentfill}{rgb}{0.121569,0.466667,0.705882}%
\pgfsetfillcolor{currentfill}%
\pgfsetfillopacity{0.998404}%
\pgfsetlinewidth{1.003750pt}%
\definecolor{currentstroke}{rgb}{0.121569,0.466667,0.705882}%
\pgfsetstrokecolor{currentstroke}%
\pgfsetstrokeopacity{0.998404}%
\pgfsetdash{}{0pt}%
\pgfpathmoveto{\pgfqpoint{2.551359in}{0.822584in}}%
\pgfpathcurveto{\pgfqpoint{2.559596in}{0.822584in}}{\pgfqpoint{2.567496in}{0.825856in}}{\pgfqpoint{2.573320in}{0.831680in}}%
\pgfpathcurveto{\pgfqpoint{2.579144in}{0.837504in}}{\pgfqpoint{2.582416in}{0.845404in}}{\pgfqpoint{2.582416in}{0.853641in}}%
\pgfpathcurveto{\pgfqpoint{2.582416in}{0.861877in}}{\pgfqpoint{2.579144in}{0.869777in}}{\pgfqpoint{2.573320in}{0.875601in}}%
\pgfpathcurveto{\pgfqpoint{2.567496in}{0.881425in}}{\pgfqpoint{2.559596in}{0.884697in}}{\pgfqpoint{2.551359in}{0.884697in}}%
\pgfpathcurveto{\pgfqpoint{2.543123in}{0.884697in}}{\pgfqpoint{2.535223in}{0.881425in}}{\pgfqpoint{2.529399in}{0.875601in}}%
\pgfpathcurveto{\pgfqpoint{2.523575in}{0.869777in}}{\pgfqpoint{2.520303in}{0.861877in}}{\pgfqpoint{2.520303in}{0.853641in}}%
\pgfpathcurveto{\pgfqpoint{2.520303in}{0.845404in}}{\pgfqpoint{2.523575in}{0.837504in}}{\pgfqpoint{2.529399in}{0.831680in}}%
\pgfpathcurveto{\pgfqpoint{2.535223in}{0.825856in}}{\pgfqpoint{2.543123in}{0.822584in}}{\pgfqpoint{2.551359in}{0.822584in}}%
\pgfpathclose%
\pgfusepath{stroke,fill}%
\end{pgfscope}%
\begin{pgfscope}%
\pgfpathrectangle{\pgfqpoint{0.100000in}{0.212622in}}{\pgfqpoint{3.696000in}{3.696000in}}%
\pgfusepath{clip}%
\pgfsetbuttcap%
\pgfsetroundjoin%
\definecolor{currentfill}{rgb}{0.121569,0.466667,0.705882}%
\pgfsetfillcolor{currentfill}%
\pgfsetfillopacity{0.998404}%
\pgfsetlinewidth{1.003750pt}%
\definecolor{currentstroke}{rgb}{0.121569,0.466667,0.705882}%
\pgfsetstrokecolor{currentstroke}%
\pgfsetstrokeopacity{0.998404}%
\pgfsetdash{}{0pt}%
\pgfpathmoveto{\pgfqpoint{2.551359in}{0.822584in}}%
\pgfpathcurveto{\pgfqpoint{2.559596in}{0.822584in}}{\pgfqpoint{2.567496in}{0.825856in}}{\pgfqpoint{2.573320in}{0.831680in}}%
\pgfpathcurveto{\pgfqpoint{2.579144in}{0.837504in}}{\pgfqpoint{2.582416in}{0.845404in}}{\pgfqpoint{2.582416in}{0.853641in}}%
\pgfpathcurveto{\pgfqpoint{2.582416in}{0.861877in}}{\pgfqpoint{2.579144in}{0.869777in}}{\pgfqpoint{2.573320in}{0.875601in}}%
\pgfpathcurveto{\pgfqpoint{2.567496in}{0.881425in}}{\pgfqpoint{2.559596in}{0.884697in}}{\pgfqpoint{2.551359in}{0.884697in}}%
\pgfpathcurveto{\pgfqpoint{2.543123in}{0.884697in}}{\pgfqpoint{2.535223in}{0.881425in}}{\pgfqpoint{2.529399in}{0.875601in}}%
\pgfpathcurveto{\pgfqpoint{2.523575in}{0.869777in}}{\pgfqpoint{2.520303in}{0.861877in}}{\pgfqpoint{2.520303in}{0.853641in}}%
\pgfpathcurveto{\pgfqpoint{2.520303in}{0.845404in}}{\pgfqpoint{2.523575in}{0.837504in}}{\pgfqpoint{2.529399in}{0.831680in}}%
\pgfpathcurveto{\pgfqpoint{2.535223in}{0.825856in}}{\pgfqpoint{2.543123in}{0.822584in}}{\pgfqpoint{2.551359in}{0.822584in}}%
\pgfpathclose%
\pgfusepath{stroke,fill}%
\end{pgfscope}%
\begin{pgfscope}%
\pgfpathrectangle{\pgfqpoint{0.100000in}{0.212622in}}{\pgfqpoint{3.696000in}{3.696000in}}%
\pgfusepath{clip}%
\pgfsetbuttcap%
\pgfsetroundjoin%
\definecolor{currentfill}{rgb}{0.121569,0.466667,0.705882}%
\pgfsetfillcolor{currentfill}%
\pgfsetfillopacity{0.998404}%
\pgfsetlinewidth{1.003750pt}%
\definecolor{currentstroke}{rgb}{0.121569,0.466667,0.705882}%
\pgfsetstrokecolor{currentstroke}%
\pgfsetstrokeopacity{0.998404}%
\pgfsetdash{}{0pt}%
\pgfpathmoveto{\pgfqpoint{2.551359in}{0.822584in}}%
\pgfpathcurveto{\pgfqpoint{2.559596in}{0.822584in}}{\pgfqpoint{2.567496in}{0.825856in}}{\pgfqpoint{2.573320in}{0.831680in}}%
\pgfpathcurveto{\pgfqpoint{2.579144in}{0.837504in}}{\pgfqpoint{2.582416in}{0.845404in}}{\pgfqpoint{2.582416in}{0.853641in}}%
\pgfpathcurveto{\pgfqpoint{2.582416in}{0.861877in}}{\pgfqpoint{2.579144in}{0.869777in}}{\pgfqpoint{2.573320in}{0.875601in}}%
\pgfpathcurveto{\pgfqpoint{2.567496in}{0.881425in}}{\pgfqpoint{2.559596in}{0.884697in}}{\pgfqpoint{2.551359in}{0.884697in}}%
\pgfpathcurveto{\pgfqpoint{2.543123in}{0.884697in}}{\pgfqpoint{2.535223in}{0.881425in}}{\pgfqpoint{2.529399in}{0.875601in}}%
\pgfpathcurveto{\pgfqpoint{2.523575in}{0.869777in}}{\pgfqpoint{2.520303in}{0.861877in}}{\pgfqpoint{2.520303in}{0.853641in}}%
\pgfpathcurveto{\pgfqpoint{2.520303in}{0.845404in}}{\pgfqpoint{2.523575in}{0.837504in}}{\pgfqpoint{2.529399in}{0.831680in}}%
\pgfpathcurveto{\pgfqpoint{2.535223in}{0.825856in}}{\pgfqpoint{2.543123in}{0.822584in}}{\pgfqpoint{2.551359in}{0.822584in}}%
\pgfpathclose%
\pgfusepath{stroke,fill}%
\end{pgfscope}%
\begin{pgfscope}%
\pgfpathrectangle{\pgfqpoint{0.100000in}{0.212622in}}{\pgfqpoint{3.696000in}{3.696000in}}%
\pgfusepath{clip}%
\pgfsetbuttcap%
\pgfsetroundjoin%
\definecolor{currentfill}{rgb}{0.121569,0.466667,0.705882}%
\pgfsetfillcolor{currentfill}%
\pgfsetfillopacity{0.998404}%
\pgfsetlinewidth{1.003750pt}%
\definecolor{currentstroke}{rgb}{0.121569,0.466667,0.705882}%
\pgfsetstrokecolor{currentstroke}%
\pgfsetstrokeopacity{0.998404}%
\pgfsetdash{}{0pt}%
\pgfpathmoveto{\pgfqpoint{2.551359in}{0.822584in}}%
\pgfpathcurveto{\pgfqpoint{2.559596in}{0.822584in}}{\pgfqpoint{2.567496in}{0.825856in}}{\pgfqpoint{2.573320in}{0.831680in}}%
\pgfpathcurveto{\pgfqpoint{2.579144in}{0.837504in}}{\pgfqpoint{2.582416in}{0.845404in}}{\pgfqpoint{2.582416in}{0.853641in}}%
\pgfpathcurveto{\pgfqpoint{2.582416in}{0.861877in}}{\pgfqpoint{2.579144in}{0.869777in}}{\pgfqpoint{2.573320in}{0.875601in}}%
\pgfpathcurveto{\pgfqpoint{2.567496in}{0.881425in}}{\pgfqpoint{2.559596in}{0.884697in}}{\pgfqpoint{2.551359in}{0.884697in}}%
\pgfpathcurveto{\pgfqpoint{2.543123in}{0.884697in}}{\pgfqpoint{2.535223in}{0.881425in}}{\pgfqpoint{2.529399in}{0.875601in}}%
\pgfpathcurveto{\pgfqpoint{2.523575in}{0.869777in}}{\pgfqpoint{2.520303in}{0.861877in}}{\pgfqpoint{2.520303in}{0.853641in}}%
\pgfpathcurveto{\pgfqpoint{2.520303in}{0.845404in}}{\pgfqpoint{2.523575in}{0.837504in}}{\pgfqpoint{2.529399in}{0.831680in}}%
\pgfpathcurveto{\pgfqpoint{2.535223in}{0.825856in}}{\pgfqpoint{2.543123in}{0.822584in}}{\pgfqpoint{2.551359in}{0.822584in}}%
\pgfpathclose%
\pgfusepath{stroke,fill}%
\end{pgfscope}%
\begin{pgfscope}%
\pgfpathrectangle{\pgfqpoint{0.100000in}{0.212622in}}{\pgfqpoint{3.696000in}{3.696000in}}%
\pgfusepath{clip}%
\pgfsetbuttcap%
\pgfsetroundjoin%
\definecolor{currentfill}{rgb}{0.121569,0.466667,0.705882}%
\pgfsetfillcolor{currentfill}%
\pgfsetfillopacity{0.998404}%
\pgfsetlinewidth{1.003750pt}%
\definecolor{currentstroke}{rgb}{0.121569,0.466667,0.705882}%
\pgfsetstrokecolor{currentstroke}%
\pgfsetstrokeopacity{0.998404}%
\pgfsetdash{}{0pt}%
\pgfpathmoveto{\pgfqpoint{2.551359in}{0.822584in}}%
\pgfpathcurveto{\pgfqpoint{2.559596in}{0.822584in}}{\pgfqpoint{2.567496in}{0.825856in}}{\pgfqpoint{2.573320in}{0.831680in}}%
\pgfpathcurveto{\pgfqpoint{2.579144in}{0.837504in}}{\pgfqpoint{2.582416in}{0.845404in}}{\pgfqpoint{2.582416in}{0.853641in}}%
\pgfpathcurveto{\pgfqpoint{2.582416in}{0.861877in}}{\pgfqpoint{2.579144in}{0.869777in}}{\pgfqpoint{2.573320in}{0.875601in}}%
\pgfpathcurveto{\pgfqpoint{2.567496in}{0.881425in}}{\pgfqpoint{2.559596in}{0.884697in}}{\pgfqpoint{2.551359in}{0.884697in}}%
\pgfpathcurveto{\pgfqpoint{2.543123in}{0.884697in}}{\pgfqpoint{2.535223in}{0.881425in}}{\pgfqpoint{2.529399in}{0.875601in}}%
\pgfpathcurveto{\pgfqpoint{2.523575in}{0.869777in}}{\pgfqpoint{2.520303in}{0.861877in}}{\pgfqpoint{2.520303in}{0.853641in}}%
\pgfpathcurveto{\pgfqpoint{2.520303in}{0.845404in}}{\pgfqpoint{2.523575in}{0.837504in}}{\pgfqpoint{2.529399in}{0.831680in}}%
\pgfpathcurveto{\pgfqpoint{2.535223in}{0.825856in}}{\pgfqpoint{2.543123in}{0.822584in}}{\pgfqpoint{2.551359in}{0.822584in}}%
\pgfpathclose%
\pgfusepath{stroke,fill}%
\end{pgfscope}%
\begin{pgfscope}%
\pgfpathrectangle{\pgfqpoint{0.100000in}{0.212622in}}{\pgfqpoint{3.696000in}{3.696000in}}%
\pgfusepath{clip}%
\pgfsetbuttcap%
\pgfsetroundjoin%
\definecolor{currentfill}{rgb}{0.121569,0.466667,0.705882}%
\pgfsetfillcolor{currentfill}%
\pgfsetfillopacity{0.998404}%
\pgfsetlinewidth{1.003750pt}%
\definecolor{currentstroke}{rgb}{0.121569,0.466667,0.705882}%
\pgfsetstrokecolor{currentstroke}%
\pgfsetstrokeopacity{0.998404}%
\pgfsetdash{}{0pt}%
\pgfpathmoveto{\pgfqpoint{2.551359in}{0.822584in}}%
\pgfpathcurveto{\pgfqpoint{2.559596in}{0.822584in}}{\pgfqpoint{2.567496in}{0.825856in}}{\pgfqpoint{2.573320in}{0.831680in}}%
\pgfpathcurveto{\pgfqpoint{2.579144in}{0.837504in}}{\pgfqpoint{2.582416in}{0.845404in}}{\pgfqpoint{2.582416in}{0.853641in}}%
\pgfpathcurveto{\pgfqpoint{2.582416in}{0.861877in}}{\pgfqpoint{2.579144in}{0.869777in}}{\pgfqpoint{2.573320in}{0.875601in}}%
\pgfpathcurveto{\pgfqpoint{2.567496in}{0.881425in}}{\pgfqpoint{2.559596in}{0.884697in}}{\pgfqpoint{2.551359in}{0.884697in}}%
\pgfpathcurveto{\pgfqpoint{2.543123in}{0.884697in}}{\pgfqpoint{2.535223in}{0.881425in}}{\pgfqpoint{2.529399in}{0.875601in}}%
\pgfpathcurveto{\pgfqpoint{2.523575in}{0.869777in}}{\pgfqpoint{2.520303in}{0.861877in}}{\pgfqpoint{2.520303in}{0.853641in}}%
\pgfpathcurveto{\pgfqpoint{2.520303in}{0.845404in}}{\pgfqpoint{2.523575in}{0.837504in}}{\pgfqpoint{2.529399in}{0.831680in}}%
\pgfpathcurveto{\pgfqpoint{2.535223in}{0.825856in}}{\pgfqpoint{2.543123in}{0.822584in}}{\pgfqpoint{2.551359in}{0.822584in}}%
\pgfpathclose%
\pgfusepath{stroke,fill}%
\end{pgfscope}%
\begin{pgfscope}%
\pgfpathrectangle{\pgfqpoint{0.100000in}{0.212622in}}{\pgfqpoint{3.696000in}{3.696000in}}%
\pgfusepath{clip}%
\pgfsetbuttcap%
\pgfsetroundjoin%
\definecolor{currentfill}{rgb}{0.121569,0.466667,0.705882}%
\pgfsetfillcolor{currentfill}%
\pgfsetfillopacity{0.998404}%
\pgfsetlinewidth{1.003750pt}%
\definecolor{currentstroke}{rgb}{0.121569,0.466667,0.705882}%
\pgfsetstrokecolor{currentstroke}%
\pgfsetstrokeopacity{0.998404}%
\pgfsetdash{}{0pt}%
\pgfpathmoveto{\pgfqpoint{2.551359in}{0.822584in}}%
\pgfpathcurveto{\pgfqpoint{2.559596in}{0.822584in}}{\pgfqpoint{2.567496in}{0.825856in}}{\pgfqpoint{2.573320in}{0.831680in}}%
\pgfpathcurveto{\pgfqpoint{2.579144in}{0.837504in}}{\pgfqpoint{2.582416in}{0.845404in}}{\pgfqpoint{2.582416in}{0.853641in}}%
\pgfpathcurveto{\pgfqpoint{2.582416in}{0.861877in}}{\pgfqpoint{2.579144in}{0.869777in}}{\pgfqpoint{2.573320in}{0.875601in}}%
\pgfpathcurveto{\pgfqpoint{2.567496in}{0.881425in}}{\pgfqpoint{2.559596in}{0.884697in}}{\pgfqpoint{2.551359in}{0.884697in}}%
\pgfpathcurveto{\pgfqpoint{2.543123in}{0.884697in}}{\pgfqpoint{2.535223in}{0.881425in}}{\pgfqpoint{2.529399in}{0.875601in}}%
\pgfpathcurveto{\pgfqpoint{2.523575in}{0.869777in}}{\pgfqpoint{2.520303in}{0.861877in}}{\pgfqpoint{2.520303in}{0.853641in}}%
\pgfpathcurveto{\pgfqpoint{2.520303in}{0.845404in}}{\pgfqpoint{2.523575in}{0.837504in}}{\pgfqpoint{2.529399in}{0.831680in}}%
\pgfpathcurveto{\pgfqpoint{2.535223in}{0.825856in}}{\pgfqpoint{2.543123in}{0.822584in}}{\pgfqpoint{2.551359in}{0.822584in}}%
\pgfpathclose%
\pgfusepath{stroke,fill}%
\end{pgfscope}%
\begin{pgfscope}%
\pgfpathrectangle{\pgfqpoint{0.100000in}{0.212622in}}{\pgfqpoint{3.696000in}{3.696000in}}%
\pgfusepath{clip}%
\pgfsetbuttcap%
\pgfsetroundjoin%
\definecolor{currentfill}{rgb}{0.121569,0.466667,0.705882}%
\pgfsetfillcolor{currentfill}%
\pgfsetfillopacity{0.998404}%
\pgfsetlinewidth{1.003750pt}%
\definecolor{currentstroke}{rgb}{0.121569,0.466667,0.705882}%
\pgfsetstrokecolor{currentstroke}%
\pgfsetstrokeopacity{0.998404}%
\pgfsetdash{}{0pt}%
\pgfpathmoveto{\pgfqpoint{2.551359in}{0.822584in}}%
\pgfpathcurveto{\pgfqpoint{2.559596in}{0.822584in}}{\pgfqpoint{2.567496in}{0.825856in}}{\pgfqpoint{2.573320in}{0.831680in}}%
\pgfpathcurveto{\pgfqpoint{2.579144in}{0.837504in}}{\pgfqpoint{2.582416in}{0.845404in}}{\pgfqpoint{2.582416in}{0.853641in}}%
\pgfpathcurveto{\pgfqpoint{2.582416in}{0.861877in}}{\pgfqpoint{2.579144in}{0.869777in}}{\pgfqpoint{2.573320in}{0.875601in}}%
\pgfpathcurveto{\pgfqpoint{2.567496in}{0.881425in}}{\pgfqpoint{2.559596in}{0.884697in}}{\pgfqpoint{2.551359in}{0.884697in}}%
\pgfpathcurveto{\pgfqpoint{2.543123in}{0.884697in}}{\pgfqpoint{2.535223in}{0.881425in}}{\pgfqpoint{2.529399in}{0.875601in}}%
\pgfpathcurveto{\pgfqpoint{2.523575in}{0.869777in}}{\pgfqpoint{2.520303in}{0.861877in}}{\pgfqpoint{2.520303in}{0.853641in}}%
\pgfpathcurveto{\pgfqpoint{2.520303in}{0.845404in}}{\pgfqpoint{2.523575in}{0.837504in}}{\pgfqpoint{2.529399in}{0.831680in}}%
\pgfpathcurveto{\pgfqpoint{2.535223in}{0.825856in}}{\pgfqpoint{2.543123in}{0.822584in}}{\pgfqpoint{2.551359in}{0.822584in}}%
\pgfpathclose%
\pgfusepath{stroke,fill}%
\end{pgfscope}%
\begin{pgfscope}%
\pgfpathrectangle{\pgfqpoint{0.100000in}{0.212622in}}{\pgfqpoint{3.696000in}{3.696000in}}%
\pgfusepath{clip}%
\pgfsetbuttcap%
\pgfsetroundjoin%
\definecolor{currentfill}{rgb}{0.121569,0.466667,0.705882}%
\pgfsetfillcolor{currentfill}%
\pgfsetfillopacity{0.998404}%
\pgfsetlinewidth{1.003750pt}%
\definecolor{currentstroke}{rgb}{0.121569,0.466667,0.705882}%
\pgfsetstrokecolor{currentstroke}%
\pgfsetstrokeopacity{0.998404}%
\pgfsetdash{}{0pt}%
\pgfpathmoveto{\pgfqpoint{2.551359in}{0.822584in}}%
\pgfpathcurveto{\pgfqpoint{2.559596in}{0.822584in}}{\pgfqpoint{2.567496in}{0.825856in}}{\pgfqpoint{2.573320in}{0.831680in}}%
\pgfpathcurveto{\pgfqpoint{2.579144in}{0.837504in}}{\pgfqpoint{2.582416in}{0.845404in}}{\pgfqpoint{2.582416in}{0.853641in}}%
\pgfpathcurveto{\pgfqpoint{2.582416in}{0.861877in}}{\pgfqpoint{2.579144in}{0.869777in}}{\pgfqpoint{2.573320in}{0.875601in}}%
\pgfpathcurveto{\pgfqpoint{2.567496in}{0.881425in}}{\pgfqpoint{2.559596in}{0.884697in}}{\pgfqpoint{2.551359in}{0.884697in}}%
\pgfpathcurveto{\pgfqpoint{2.543123in}{0.884697in}}{\pgfqpoint{2.535223in}{0.881425in}}{\pgfqpoint{2.529399in}{0.875601in}}%
\pgfpathcurveto{\pgfqpoint{2.523575in}{0.869777in}}{\pgfqpoint{2.520303in}{0.861877in}}{\pgfqpoint{2.520303in}{0.853641in}}%
\pgfpathcurveto{\pgfqpoint{2.520303in}{0.845404in}}{\pgfqpoint{2.523575in}{0.837504in}}{\pgfqpoint{2.529399in}{0.831680in}}%
\pgfpathcurveto{\pgfqpoint{2.535223in}{0.825856in}}{\pgfqpoint{2.543123in}{0.822584in}}{\pgfqpoint{2.551359in}{0.822584in}}%
\pgfpathclose%
\pgfusepath{stroke,fill}%
\end{pgfscope}%
\begin{pgfscope}%
\pgfpathrectangle{\pgfqpoint{0.100000in}{0.212622in}}{\pgfqpoint{3.696000in}{3.696000in}}%
\pgfusepath{clip}%
\pgfsetbuttcap%
\pgfsetroundjoin%
\definecolor{currentfill}{rgb}{0.121569,0.466667,0.705882}%
\pgfsetfillcolor{currentfill}%
\pgfsetfillopacity{0.998404}%
\pgfsetlinewidth{1.003750pt}%
\definecolor{currentstroke}{rgb}{0.121569,0.466667,0.705882}%
\pgfsetstrokecolor{currentstroke}%
\pgfsetstrokeopacity{0.998404}%
\pgfsetdash{}{0pt}%
\pgfpathmoveto{\pgfqpoint{2.551359in}{0.822584in}}%
\pgfpathcurveto{\pgfqpoint{2.559596in}{0.822584in}}{\pgfqpoint{2.567496in}{0.825856in}}{\pgfqpoint{2.573320in}{0.831680in}}%
\pgfpathcurveto{\pgfqpoint{2.579144in}{0.837504in}}{\pgfqpoint{2.582416in}{0.845404in}}{\pgfqpoint{2.582416in}{0.853641in}}%
\pgfpathcurveto{\pgfqpoint{2.582416in}{0.861877in}}{\pgfqpoint{2.579144in}{0.869777in}}{\pgfqpoint{2.573320in}{0.875601in}}%
\pgfpathcurveto{\pgfqpoint{2.567496in}{0.881425in}}{\pgfqpoint{2.559596in}{0.884697in}}{\pgfqpoint{2.551359in}{0.884697in}}%
\pgfpathcurveto{\pgfqpoint{2.543123in}{0.884697in}}{\pgfqpoint{2.535223in}{0.881425in}}{\pgfqpoint{2.529399in}{0.875601in}}%
\pgfpathcurveto{\pgfqpoint{2.523575in}{0.869777in}}{\pgfqpoint{2.520303in}{0.861877in}}{\pgfqpoint{2.520303in}{0.853641in}}%
\pgfpathcurveto{\pgfqpoint{2.520303in}{0.845404in}}{\pgfqpoint{2.523575in}{0.837504in}}{\pgfqpoint{2.529399in}{0.831680in}}%
\pgfpathcurveto{\pgfqpoint{2.535223in}{0.825856in}}{\pgfqpoint{2.543123in}{0.822584in}}{\pgfqpoint{2.551359in}{0.822584in}}%
\pgfpathclose%
\pgfusepath{stroke,fill}%
\end{pgfscope}%
\begin{pgfscope}%
\pgfpathrectangle{\pgfqpoint{0.100000in}{0.212622in}}{\pgfqpoint{3.696000in}{3.696000in}}%
\pgfusepath{clip}%
\pgfsetbuttcap%
\pgfsetroundjoin%
\definecolor{currentfill}{rgb}{0.121569,0.466667,0.705882}%
\pgfsetfillcolor{currentfill}%
\pgfsetfillopacity{0.998404}%
\pgfsetlinewidth{1.003750pt}%
\definecolor{currentstroke}{rgb}{0.121569,0.466667,0.705882}%
\pgfsetstrokecolor{currentstroke}%
\pgfsetstrokeopacity{0.998404}%
\pgfsetdash{}{0pt}%
\pgfpathmoveto{\pgfqpoint{2.551359in}{0.822584in}}%
\pgfpathcurveto{\pgfqpoint{2.559596in}{0.822584in}}{\pgfqpoint{2.567496in}{0.825856in}}{\pgfqpoint{2.573320in}{0.831680in}}%
\pgfpathcurveto{\pgfqpoint{2.579144in}{0.837504in}}{\pgfqpoint{2.582416in}{0.845404in}}{\pgfqpoint{2.582416in}{0.853641in}}%
\pgfpathcurveto{\pgfqpoint{2.582416in}{0.861877in}}{\pgfqpoint{2.579144in}{0.869777in}}{\pgfqpoint{2.573320in}{0.875601in}}%
\pgfpathcurveto{\pgfqpoint{2.567496in}{0.881425in}}{\pgfqpoint{2.559596in}{0.884697in}}{\pgfqpoint{2.551359in}{0.884697in}}%
\pgfpathcurveto{\pgfqpoint{2.543123in}{0.884697in}}{\pgfqpoint{2.535223in}{0.881425in}}{\pgfqpoint{2.529399in}{0.875601in}}%
\pgfpathcurveto{\pgfqpoint{2.523575in}{0.869777in}}{\pgfqpoint{2.520303in}{0.861877in}}{\pgfqpoint{2.520303in}{0.853641in}}%
\pgfpathcurveto{\pgfqpoint{2.520303in}{0.845404in}}{\pgfqpoint{2.523575in}{0.837504in}}{\pgfqpoint{2.529399in}{0.831680in}}%
\pgfpathcurveto{\pgfqpoint{2.535223in}{0.825856in}}{\pgfqpoint{2.543123in}{0.822584in}}{\pgfqpoint{2.551359in}{0.822584in}}%
\pgfpathclose%
\pgfusepath{stroke,fill}%
\end{pgfscope}%
\begin{pgfscope}%
\pgfpathrectangle{\pgfqpoint{0.100000in}{0.212622in}}{\pgfqpoint{3.696000in}{3.696000in}}%
\pgfusepath{clip}%
\pgfsetbuttcap%
\pgfsetroundjoin%
\definecolor{currentfill}{rgb}{0.121569,0.466667,0.705882}%
\pgfsetfillcolor{currentfill}%
\pgfsetfillopacity{0.998404}%
\pgfsetlinewidth{1.003750pt}%
\definecolor{currentstroke}{rgb}{0.121569,0.466667,0.705882}%
\pgfsetstrokecolor{currentstroke}%
\pgfsetstrokeopacity{0.998404}%
\pgfsetdash{}{0pt}%
\pgfpathmoveto{\pgfqpoint{2.551359in}{0.822584in}}%
\pgfpathcurveto{\pgfqpoint{2.559596in}{0.822584in}}{\pgfqpoint{2.567496in}{0.825856in}}{\pgfqpoint{2.573320in}{0.831680in}}%
\pgfpathcurveto{\pgfqpoint{2.579144in}{0.837504in}}{\pgfqpoint{2.582416in}{0.845404in}}{\pgfqpoint{2.582416in}{0.853641in}}%
\pgfpathcurveto{\pgfqpoint{2.582416in}{0.861877in}}{\pgfqpoint{2.579144in}{0.869777in}}{\pgfqpoint{2.573320in}{0.875601in}}%
\pgfpathcurveto{\pgfqpoint{2.567496in}{0.881425in}}{\pgfqpoint{2.559596in}{0.884697in}}{\pgfqpoint{2.551359in}{0.884697in}}%
\pgfpathcurveto{\pgfqpoint{2.543123in}{0.884697in}}{\pgfqpoint{2.535223in}{0.881425in}}{\pgfqpoint{2.529399in}{0.875601in}}%
\pgfpathcurveto{\pgfqpoint{2.523575in}{0.869777in}}{\pgfqpoint{2.520303in}{0.861877in}}{\pgfqpoint{2.520303in}{0.853641in}}%
\pgfpathcurveto{\pgfqpoint{2.520303in}{0.845404in}}{\pgfqpoint{2.523575in}{0.837504in}}{\pgfqpoint{2.529399in}{0.831680in}}%
\pgfpathcurveto{\pgfqpoint{2.535223in}{0.825856in}}{\pgfqpoint{2.543123in}{0.822584in}}{\pgfqpoint{2.551359in}{0.822584in}}%
\pgfpathclose%
\pgfusepath{stroke,fill}%
\end{pgfscope}%
\begin{pgfscope}%
\pgfpathrectangle{\pgfqpoint{0.100000in}{0.212622in}}{\pgfqpoint{3.696000in}{3.696000in}}%
\pgfusepath{clip}%
\pgfsetbuttcap%
\pgfsetroundjoin%
\definecolor{currentfill}{rgb}{0.121569,0.466667,0.705882}%
\pgfsetfillcolor{currentfill}%
\pgfsetfillopacity{0.998404}%
\pgfsetlinewidth{1.003750pt}%
\definecolor{currentstroke}{rgb}{0.121569,0.466667,0.705882}%
\pgfsetstrokecolor{currentstroke}%
\pgfsetstrokeopacity{0.998404}%
\pgfsetdash{}{0pt}%
\pgfpathmoveto{\pgfqpoint{2.551359in}{0.822584in}}%
\pgfpathcurveto{\pgfqpoint{2.559596in}{0.822584in}}{\pgfqpoint{2.567496in}{0.825856in}}{\pgfqpoint{2.573320in}{0.831680in}}%
\pgfpathcurveto{\pgfqpoint{2.579144in}{0.837504in}}{\pgfqpoint{2.582416in}{0.845404in}}{\pgfqpoint{2.582416in}{0.853641in}}%
\pgfpathcurveto{\pgfqpoint{2.582416in}{0.861877in}}{\pgfqpoint{2.579144in}{0.869777in}}{\pgfqpoint{2.573320in}{0.875601in}}%
\pgfpathcurveto{\pgfqpoint{2.567496in}{0.881425in}}{\pgfqpoint{2.559596in}{0.884697in}}{\pgfqpoint{2.551359in}{0.884697in}}%
\pgfpathcurveto{\pgfqpoint{2.543123in}{0.884697in}}{\pgfqpoint{2.535223in}{0.881425in}}{\pgfqpoint{2.529399in}{0.875601in}}%
\pgfpathcurveto{\pgfqpoint{2.523575in}{0.869777in}}{\pgfqpoint{2.520303in}{0.861877in}}{\pgfqpoint{2.520303in}{0.853641in}}%
\pgfpathcurveto{\pgfqpoint{2.520303in}{0.845404in}}{\pgfqpoint{2.523575in}{0.837504in}}{\pgfqpoint{2.529399in}{0.831680in}}%
\pgfpathcurveto{\pgfqpoint{2.535223in}{0.825856in}}{\pgfqpoint{2.543123in}{0.822584in}}{\pgfqpoint{2.551359in}{0.822584in}}%
\pgfpathclose%
\pgfusepath{stroke,fill}%
\end{pgfscope}%
\begin{pgfscope}%
\pgfpathrectangle{\pgfqpoint{0.100000in}{0.212622in}}{\pgfqpoint{3.696000in}{3.696000in}}%
\pgfusepath{clip}%
\pgfsetbuttcap%
\pgfsetroundjoin%
\definecolor{currentfill}{rgb}{0.121569,0.466667,0.705882}%
\pgfsetfillcolor{currentfill}%
\pgfsetfillopacity{0.998404}%
\pgfsetlinewidth{1.003750pt}%
\definecolor{currentstroke}{rgb}{0.121569,0.466667,0.705882}%
\pgfsetstrokecolor{currentstroke}%
\pgfsetstrokeopacity{0.998404}%
\pgfsetdash{}{0pt}%
\pgfpathmoveto{\pgfqpoint{2.551359in}{0.822584in}}%
\pgfpathcurveto{\pgfqpoint{2.559596in}{0.822584in}}{\pgfqpoint{2.567496in}{0.825856in}}{\pgfqpoint{2.573320in}{0.831680in}}%
\pgfpathcurveto{\pgfqpoint{2.579144in}{0.837504in}}{\pgfqpoint{2.582416in}{0.845404in}}{\pgfqpoint{2.582416in}{0.853641in}}%
\pgfpathcurveto{\pgfqpoint{2.582416in}{0.861877in}}{\pgfqpoint{2.579144in}{0.869777in}}{\pgfqpoint{2.573320in}{0.875601in}}%
\pgfpathcurveto{\pgfqpoint{2.567496in}{0.881425in}}{\pgfqpoint{2.559596in}{0.884697in}}{\pgfqpoint{2.551359in}{0.884697in}}%
\pgfpathcurveto{\pgfqpoint{2.543123in}{0.884697in}}{\pgfqpoint{2.535223in}{0.881425in}}{\pgfqpoint{2.529399in}{0.875601in}}%
\pgfpathcurveto{\pgfqpoint{2.523575in}{0.869777in}}{\pgfqpoint{2.520303in}{0.861877in}}{\pgfqpoint{2.520303in}{0.853641in}}%
\pgfpathcurveto{\pgfqpoint{2.520303in}{0.845404in}}{\pgfqpoint{2.523575in}{0.837504in}}{\pgfqpoint{2.529399in}{0.831680in}}%
\pgfpathcurveto{\pgfqpoint{2.535223in}{0.825856in}}{\pgfqpoint{2.543123in}{0.822584in}}{\pgfqpoint{2.551359in}{0.822584in}}%
\pgfpathclose%
\pgfusepath{stroke,fill}%
\end{pgfscope}%
\begin{pgfscope}%
\pgfpathrectangle{\pgfqpoint{0.100000in}{0.212622in}}{\pgfqpoint{3.696000in}{3.696000in}}%
\pgfusepath{clip}%
\pgfsetbuttcap%
\pgfsetroundjoin%
\definecolor{currentfill}{rgb}{0.121569,0.466667,0.705882}%
\pgfsetfillcolor{currentfill}%
\pgfsetfillopacity{0.998404}%
\pgfsetlinewidth{1.003750pt}%
\definecolor{currentstroke}{rgb}{0.121569,0.466667,0.705882}%
\pgfsetstrokecolor{currentstroke}%
\pgfsetstrokeopacity{0.998404}%
\pgfsetdash{}{0pt}%
\pgfpathmoveto{\pgfqpoint{2.551359in}{0.822584in}}%
\pgfpathcurveto{\pgfqpoint{2.559596in}{0.822584in}}{\pgfqpoint{2.567496in}{0.825856in}}{\pgfqpoint{2.573320in}{0.831680in}}%
\pgfpathcurveto{\pgfqpoint{2.579144in}{0.837504in}}{\pgfqpoint{2.582416in}{0.845404in}}{\pgfqpoint{2.582416in}{0.853641in}}%
\pgfpathcurveto{\pgfqpoint{2.582416in}{0.861877in}}{\pgfqpoint{2.579144in}{0.869777in}}{\pgfqpoint{2.573320in}{0.875601in}}%
\pgfpathcurveto{\pgfqpoint{2.567496in}{0.881425in}}{\pgfqpoint{2.559596in}{0.884697in}}{\pgfqpoint{2.551359in}{0.884697in}}%
\pgfpathcurveto{\pgfqpoint{2.543123in}{0.884697in}}{\pgfqpoint{2.535223in}{0.881425in}}{\pgfqpoint{2.529399in}{0.875601in}}%
\pgfpathcurveto{\pgfqpoint{2.523575in}{0.869777in}}{\pgfqpoint{2.520303in}{0.861877in}}{\pgfqpoint{2.520303in}{0.853641in}}%
\pgfpathcurveto{\pgfqpoint{2.520303in}{0.845404in}}{\pgfqpoint{2.523575in}{0.837504in}}{\pgfqpoint{2.529399in}{0.831680in}}%
\pgfpathcurveto{\pgfqpoint{2.535223in}{0.825856in}}{\pgfqpoint{2.543123in}{0.822584in}}{\pgfqpoint{2.551359in}{0.822584in}}%
\pgfpathclose%
\pgfusepath{stroke,fill}%
\end{pgfscope}%
\begin{pgfscope}%
\pgfpathrectangle{\pgfqpoint{0.100000in}{0.212622in}}{\pgfqpoint{3.696000in}{3.696000in}}%
\pgfusepath{clip}%
\pgfsetbuttcap%
\pgfsetroundjoin%
\definecolor{currentfill}{rgb}{0.121569,0.466667,0.705882}%
\pgfsetfillcolor{currentfill}%
\pgfsetfillopacity{0.998404}%
\pgfsetlinewidth{1.003750pt}%
\definecolor{currentstroke}{rgb}{0.121569,0.466667,0.705882}%
\pgfsetstrokecolor{currentstroke}%
\pgfsetstrokeopacity{0.998404}%
\pgfsetdash{}{0pt}%
\pgfpathmoveto{\pgfqpoint{2.551359in}{0.822584in}}%
\pgfpathcurveto{\pgfqpoint{2.559596in}{0.822584in}}{\pgfqpoint{2.567496in}{0.825856in}}{\pgfqpoint{2.573320in}{0.831680in}}%
\pgfpathcurveto{\pgfqpoint{2.579144in}{0.837504in}}{\pgfqpoint{2.582416in}{0.845404in}}{\pgfqpoint{2.582416in}{0.853640in}}%
\pgfpathcurveto{\pgfqpoint{2.582416in}{0.861877in}}{\pgfqpoint{2.579144in}{0.869777in}}{\pgfqpoint{2.573320in}{0.875601in}}%
\pgfpathcurveto{\pgfqpoint{2.567496in}{0.881425in}}{\pgfqpoint{2.559596in}{0.884697in}}{\pgfqpoint{2.551359in}{0.884697in}}%
\pgfpathcurveto{\pgfqpoint{2.543123in}{0.884697in}}{\pgfqpoint{2.535223in}{0.881425in}}{\pgfqpoint{2.529399in}{0.875601in}}%
\pgfpathcurveto{\pgfqpoint{2.523575in}{0.869777in}}{\pgfqpoint{2.520303in}{0.861877in}}{\pgfqpoint{2.520303in}{0.853640in}}%
\pgfpathcurveto{\pgfqpoint{2.520303in}{0.845404in}}{\pgfqpoint{2.523575in}{0.837504in}}{\pgfqpoint{2.529399in}{0.831680in}}%
\pgfpathcurveto{\pgfqpoint{2.535223in}{0.825856in}}{\pgfqpoint{2.543123in}{0.822584in}}{\pgfqpoint{2.551359in}{0.822584in}}%
\pgfpathclose%
\pgfusepath{stroke,fill}%
\end{pgfscope}%
\begin{pgfscope}%
\pgfpathrectangle{\pgfqpoint{0.100000in}{0.212622in}}{\pgfqpoint{3.696000in}{3.696000in}}%
\pgfusepath{clip}%
\pgfsetbuttcap%
\pgfsetroundjoin%
\definecolor{currentfill}{rgb}{0.121569,0.466667,0.705882}%
\pgfsetfillcolor{currentfill}%
\pgfsetfillopacity{0.998404}%
\pgfsetlinewidth{1.003750pt}%
\definecolor{currentstroke}{rgb}{0.121569,0.466667,0.705882}%
\pgfsetstrokecolor{currentstroke}%
\pgfsetstrokeopacity{0.998404}%
\pgfsetdash{}{0pt}%
\pgfpathmoveto{\pgfqpoint{2.551359in}{0.822584in}}%
\pgfpathcurveto{\pgfqpoint{2.559596in}{0.822584in}}{\pgfqpoint{2.567496in}{0.825856in}}{\pgfqpoint{2.573320in}{0.831680in}}%
\pgfpathcurveto{\pgfqpoint{2.579144in}{0.837504in}}{\pgfqpoint{2.582416in}{0.845404in}}{\pgfqpoint{2.582416in}{0.853640in}}%
\pgfpathcurveto{\pgfqpoint{2.582416in}{0.861877in}}{\pgfqpoint{2.579144in}{0.869777in}}{\pgfqpoint{2.573320in}{0.875601in}}%
\pgfpathcurveto{\pgfqpoint{2.567496in}{0.881425in}}{\pgfqpoint{2.559596in}{0.884697in}}{\pgfqpoint{2.551359in}{0.884697in}}%
\pgfpathcurveto{\pgfqpoint{2.543123in}{0.884697in}}{\pgfqpoint{2.535223in}{0.881425in}}{\pgfqpoint{2.529399in}{0.875601in}}%
\pgfpathcurveto{\pgfqpoint{2.523575in}{0.869777in}}{\pgfqpoint{2.520303in}{0.861877in}}{\pgfqpoint{2.520303in}{0.853640in}}%
\pgfpathcurveto{\pgfqpoint{2.520303in}{0.845404in}}{\pgfqpoint{2.523575in}{0.837504in}}{\pgfqpoint{2.529399in}{0.831680in}}%
\pgfpathcurveto{\pgfqpoint{2.535223in}{0.825856in}}{\pgfqpoint{2.543123in}{0.822584in}}{\pgfqpoint{2.551359in}{0.822584in}}%
\pgfpathclose%
\pgfusepath{stroke,fill}%
\end{pgfscope}%
\begin{pgfscope}%
\pgfpathrectangle{\pgfqpoint{0.100000in}{0.212622in}}{\pgfqpoint{3.696000in}{3.696000in}}%
\pgfusepath{clip}%
\pgfsetbuttcap%
\pgfsetroundjoin%
\definecolor{currentfill}{rgb}{0.121569,0.466667,0.705882}%
\pgfsetfillcolor{currentfill}%
\pgfsetfillopacity{0.998404}%
\pgfsetlinewidth{1.003750pt}%
\definecolor{currentstroke}{rgb}{0.121569,0.466667,0.705882}%
\pgfsetstrokecolor{currentstroke}%
\pgfsetstrokeopacity{0.998404}%
\pgfsetdash{}{0pt}%
\pgfpathmoveto{\pgfqpoint{2.551359in}{0.822584in}}%
\pgfpathcurveto{\pgfqpoint{2.559596in}{0.822584in}}{\pgfqpoint{2.567496in}{0.825856in}}{\pgfqpoint{2.573320in}{0.831680in}}%
\pgfpathcurveto{\pgfqpoint{2.579144in}{0.837504in}}{\pgfqpoint{2.582416in}{0.845404in}}{\pgfqpoint{2.582416in}{0.853640in}}%
\pgfpathcurveto{\pgfqpoint{2.582416in}{0.861877in}}{\pgfqpoint{2.579144in}{0.869777in}}{\pgfqpoint{2.573320in}{0.875601in}}%
\pgfpathcurveto{\pgfqpoint{2.567496in}{0.881424in}}{\pgfqpoint{2.559596in}{0.884697in}}{\pgfqpoint{2.551359in}{0.884697in}}%
\pgfpathcurveto{\pgfqpoint{2.543123in}{0.884697in}}{\pgfqpoint{2.535223in}{0.881424in}}{\pgfqpoint{2.529399in}{0.875601in}}%
\pgfpathcurveto{\pgfqpoint{2.523575in}{0.869777in}}{\pgfqpoint{2.520303in}{0.861877in}}{\pgfqpoint{2.520303in}{0.853640in}}%
\pgfpathcurveto{\pgfqpoint{2.520303in}{0.845404in}}{\pgfqpoint{2.523575in}{0.837504in}}{\pgfqpoint{2.529399in}{0.831680in}}%
\pgfpathcurveto{\pgfqpoint{2.535223in}{0.825856in}}{\pgfqpoint{2.543123in}{0.822584in}}{\pgfqpoint{2.551359in}{0.822584in}}%
\pgfpathclose%
\pgfusepath{stroke,fill}%
\end{pgfscope}%
\begin{pgfscope}%
\pgfpathrectangle{\pgfqpoint{0.100000in}{0.212622in}}{\pgfqpoint{3.696000in}{3.696000in}}%
\pgfusepath{clip}%
\pgfsetbuttcap%
\pgfsetroundjoin%
\definecolor{currentfill}{rgb}{0.121569,0.466667,0.705882}%
\pgfsetfillcolor{currentfill}%
\pgfsetfillopacity{0.998404}%
\pgfsetlinewidth{1.003750pt}%
\definecolor{currentstroke}{rgb}{0.121569,0.466667,0.705882}%
\pgfsetstrokecolor{currentstroke}%
\pgfsetstrokeopacity{0.998404}%
\pgfsetdash{}{0pt}%
\pgfpathmoveto{\pgfqpoint{2.551359in}{0.822583in}}%
\pgfpathcurveto{\pgfqpoint{2.559595in}{0.822583in}}{\pgfqpoint{2.567496in}{0.825856in}}{\pgfqpoint{2.573319in}{0.831680in}}%
\pgfpathcurveto{\pgfqpoint{2.579143in}{0.837504in}}{\pgfqpoint{2.582416in}{0.845404in}}{\pgfqpoint{2.582416in}{0.853640in}}%
\pgfpathcurveto{\pgfqpoint{2.582416in}{0.861876in}}{\pgfqpoint{2.579143in}{0.869776in}}{\pgfqpoint{2.573319in}{0.875600in}}%
\pgfpathcurveto{\pgfqpoint{2.567496in}{0.881424in}}{\pgfqpoint{2.559595in}{0.884696in}}{\pgfqpoint{2.551359in}{0.884696in}}%
\pgfpathcurveto{\pgfqpoint{2.543123in}{0.884696in}}{\pgfqpoint{2.535223in}{0.881424in}}{\pgfqpoint{2.529399in}{0.875600in}}%
\pgfpathcurveto{\pgfqpoint{2.523575in}{0.869776in}}{\pgfqpoint{2.520303in}{0.861876in}}{\pgfqpoint{2.520303in}{0.853640in}}%
\pgfpathcurveto{\pgfqpoint{2.520303in}{0.845404in}}{\pgfqpoint{2.523575in}{0.837504in}}{\pgfqpoint{2.529399in}{0.831680in}}%
\pgfpathcurveto{\pgfqpoint{2.535223in}{0.825856in}}{\pgfqpoint{2.543123in}{0.822583in}}{\pgfqpoint{2.551359in}{0.822583in}}%
\pgfpathclose%
\pgfusepath{stroke,fill}%
\end{pgfscope}%
\begin{pgfscope}%
\pgfpathrectangle{\pgfqpoint{0.100000in}{0.212622in}}{\pgfqpoint{3.696000in}{3.696000in}}%
\pgfusepath{clip}%
\pgfsetbuttcap%
\pgfsetroundjoin%
\definecolor{currentfill}{rgb}{0.121569,0.466667,0.705882}%
\pgfsetfillcolor{currentfill}%
\pgfsetfillopacity{0.998404}%
\pgfsetlinewidth{1.003750pt}%
\definecolor{currentstroke}{rgb}{0.121569,0.466667,0.705882}%
\pgfsetstrokecolor{currentstroke}%
\pgfsetstrokeopacity{0.998404}%
\pgfsetdash{}{0pt}%
\pgfpathmoveto{\pgfqpoint{2.551359in}{0.822583in}}%
\pgfpathcurveto{\pgfqpoint{2.559595in}{0.822583in}}{\pgfqpoint{2.567495in}{0.825855in}}{\pgfqpoint{2.573319in}{0.831679in}}%
\pgfpathcurveto{\pgfqpoint{2.579143in}{0.837503in}}{\pgfqpoint{2.582415in}{0.845403in}}{\pgfqpoint{2.582415in}{0.853640in}}%
\pgfpathcurveto{\pgfqpoint{2.582415in}{0.861876in}}{\pgfqpoint{2.579143in}{0.869776in}}{\pgfqpoint{2.573319in}{0.875600in}}%
\pgfpathcurveto{\pgfqpoint{2.567495in}{0.881424in}}{\pgfqpoint{2.559595in}{0.884696in}}{\pgfqpoint{2.551359in}{0.884696in}}%
\pgfpathcurveto{\pgfqpoint{2.543123in}{0.884696in}}{\pgfqpoint{2.535223in}{0.881424in}}{\pgfqpoint{2.529399in}{0.875600in}}%
\pgfpathcurveto{\pgfqpoint{2.523575in}{0.869776in}}{\pgfqpoint{2.520302in}{0.861876in}}{\pgfqpoint{2.520302in}{0.853640in}}%
\pgfpathcurveto{\pgfqpoint{2.520302in}{0.845403in}}{\pgfqpoint{2.523575in}{0.837503in}}{\pgfqpoint{2.529399in}{0.831679in}}%
\pgfpathcurveto{\pgfqpoint{2.535223in}{0.825855in}}{\pgfqpoint{2.543123in}{0.822583in}}{\pgfqpoint{2.551359in}{0.822583in}}%
\pgfpathclose%
\pgfusepath{stroke,fill}%
\end{pgfscope}%
\begin{pgfscope}%
\pgfpathrectangle{\pgfqpoint{0.100000in}{0.212622in}}{\pgfqpoint{3.696000in}{3.696000in}}%
\pgfusepath{clip}%
\pgfsetbuttcap%
\pgfsetroundjoin%
\definecolor{currentfill}{rgb}{0.121569,0.466667,0.705882}%
\pgfsetfillcolor{currentfill}%
\pgfsetfillopacity{0.998404}%
\pgfsetlinewidth{1.003750pt}%
\definecolor{currentstroke}{rgb}{0.121569,0.466667,0.705882}%
\pgfsetstrokecolor{currentstroke}%
\pgfsetstrokeopacity{0.998404}%
\pgfsetdash{}{0pt}%
\pgfpathmoveto{\pgfqpoint{2.551359in}{0.822582in}}%
\pgfpathcurveto{\pgfqpoint{2.559595in}{0.822582in}}{\pgfqpoint{2.567495in}{0.825855in}}{\pgfqpoint{2.573319in}{0.831678in}}%
\pgfpathcurveto{\pgfqpoint{2.579143in}{0.837502in}}{\pgfqpoint{2.582415in}{0.845402in}}{\pgfqpoint{2.582415in}{0.853639in}}%
\pgfpathcurveto{\pgfqpoint{2.582415in}{0.861875in}}{\pgfqpoint{2.579143in}{0.869775in}}{\pgfqpoint{2.573319in}{0.875599in}}%
\pgfpathcurveto{\pgfqpoint{2.567495in}{0.881423in}}{\pgfqpoint{2.559595in}{0.884695in}}{\pgfqpoint{2.551359in}{0.884695in}}%
\pgfpathcurveto{\pgfqpoint{2.543122in}{0.884695in}}{\pgfqpoint{2.535222in}{0.881423in}}{\pgfqpoint{2.529398in}{0.875599in}}%
\pgfpathcurveto{\pgfqpoint{2.523574in}{0.869775in}}{\pgfqpoint{2.520302in}{0.861875in}}{\pgfqpoint{2.520302in}{0.853639in}}%
\pgfpathcurveto{\pgfqpoint{2.520302in}{0.845402in}}{\pgfqpoint{2.523574in}{0.837502in}}{\pgfqpoint{2.529398in}{0.831678in}}%
\pgfpathcurveto{\pgfqpoint{2.535222in}{0.825855in}}{\pgfqpoint{2.543122in}{0.822582in}}{\pgfqpoint{2.551359in}{0.822582in}}%
\pgfpathclose%
\pgfusepath{stroke,fill}%
\end{pgfscope}%
\begin{pgfscope}%
\pgfpathrectangle{\pgfqpoint{0.100000in}{0.212622in}}{\pgfqpoint{3.696000in}{3.696000in}}%
\pgfusepath{clip}%
\pgfsetbuttcap%
\pgfsetroundjoin%
\definecolor{currentfill}{rgb}{0.121569,0.466667,0.705882}%
\pgfsetfillcolor{currentfill}%
\pgfsetfillopacity{0.998405}%
\pgfsetlinewidth{1.003750pt}%
\definecolor{currentstroke}{rgb}{0.121569,0.466667,0.705882}%
\pgfsetstrokecolor{currentstroke}%
\pgfsetstrokeopacity{0.998405}%
\pgfsetdash{}{0pt}%
\pgfpathmoveto{\pgfqpoint{2.551358in}{0.822581in}}%
\pgfpathcurveto{\pgfqpoint{2.559594in}{0.822581in}}{\pgfqpoint{2.567494in}{0.825853in}}{\pgfqpoint{2.573318in}{0.831677in}}%
\pgfpathcurveto{\pgfqpoint{2.579142in}{0.837501in}}{\pgfqpoint{2.582414in}{0.845401in}}{\pgfqpoint{2.582414in}{0.853637in}}%
\pgfpathcurveto{\pgfqpoint{2.582414in}{0.861873in}}{\pgfqpoint{2.579142in}{0.869773in}}{\pgfqpoint{2.573318in}{0.875597in}}%
\pgfpathcurveto{\pgfqpoint{2.567494in}{0.881421in}}{\pgfqpoint{2.559594in}{0.884694in}}{\pgfqpoint{2.551358in}{0.884694in}}%
\pgfpathcurveto{\pgfqpoint{2.543122in}{0.884694in}}{\pgfqpoint{2.535221in}{0.881421in}}{\pgfqpoint{2.529398in}{0.875597in}}%
\pgfpathcurveto{\pgfqpoint{2.523574in}{0.869773in}}{\pgfqpoint{2.520301in}{0.861873in}}{\pgfqpoint{2.520301in}{0.853637in}}%
\pgfpathcurveto{\pgfqpoint{2.520301in}{0.845401in}}{\pgfqpoint{2.523574in}{0.837501in}}{\pgfqpoint{2.529398in}{0.831677in}}%
\pgfpathcurveto{\pgfqpoint{2.535221in}{0.825853in}}{\pgfqpoint{2.543122in}{0.822581in}}{\pgfqpoint{2.551358in}{0.822581in}}%
\pgfpathclose%
\pgfusepath{stroke,fill}%
\end{pgfscope}%
\begin{pgfscope}%
\pgfpathrectangle{\pgfqpoint{0.100000in}{0.212622in}}{\pgfqpoint{3.696000in}{3.696000in}}%
\pgfusepath{clip}%
\pgfsetbuttcap%
\pgfsetroundjoin%
\definecolor{currentfill}{rgb}{0.121569,0.466667,0.705882}%
\pgfsetfillcolor{currentfill}%
\pgfsetfillopacity{0.998405}%
\pgfsetlinewidth{1.003750pt}%
\definecolor{currentstroke}{rgb}{0.121569,0.466667,0.705882}%
\pgfsetstrokecolor{currentstroke}%
\pgfsetstrokeopacity{0.998405}%
\pgfsetdash{}{0pt}%
\pgfpathmoveto{\pgfqpoint{2.551356in}{0.822578in}}%
\pgfpathcurveto{\pgfqpoint{2.559593in}{0.822578in}}{\pgfqpoint{2.567493in}{0.825850in}}{\pgfqpoint{2.573317in}{0.831674in}}%
\pgfpathcurveto{\pgfqpoint{2.579141in}{0.837498in}}{\pgfqpoint{2.582413in}{0.845398in}}{\pgfqpoint{2.582413in}{0.853634in}}%
\pgfpathcurveto{\pgfqpoint{2.582413in}{0.861871in}}{\pgfqpoint{2.579141in}{0.869771in}}{\pgfqpoint{2.573317in}{0.875595in}}%
\pgfpathcurveto{\pgfqpoint{2.567493in}{0.881419in}}{\pgfqpoint{2.559593in}{0.884691in}}{\pgfqpoint{2.551356in}{0.884691in}}%
\pgfpathcurveto{\pgfqpoint{2.543120in}{0.884691in}}{\pgfqpoint{2.535220in}{0.881419in}}{\pgfqpoint{2.529396in}{0.875595in}}%
\pgfpathcurveto{\pgfqpoint{2.523572in}{0.869771in}}{\pgfqpoint{2.520300in}{0.861871in}}{\pgfqpoint{2.520300in}{0.853634in}}%
\pgfpathcurveto{\pgfqpoint{2.520300in}{0.845398in}}{\pgfqpoint{2.523572in}{0.837498in}}{\pgfqpoint{2.529396in}{0.831674in}}%
\pgfpathcurveto{\pgfqpoint{2.535220in}{0.825850in}}{\pgfqpoint{2.543120in}{0.822578in}}{\pgfqpoint{2.551356in}{0.822578in}}%
\pgfpathclose%
\pgfusepath{stroke,fill}%
\end{pgfscope}%
\begin{pgfscope}%
\pgfpathrectangle{\pgfqpoint{0.100000in}{0.212622in}}{\pgfqpoint{3.696000in}{3.696000in}}%
\pgfusepath{clip}%
\pgfsetbuttcap%
\pgfsetroundjoin%
\definecolor{currentfill}{rgb}{0.121569,0.466667,0.705882}%
\pgfsetfillcolor{currentfill}%
\pgfsetfillopacity{0.998406}%
\pgfsetlinewidth{1.003750pt}%
\definecolor{currentstroke}{rgb}{0.121569,0.466667,0.705882}%
\pgfsetstrokecolor{currentstroke}%
\pgfsetstrokeopacity{0.998406}%
\pgfsetdash{}{0pt}%
\pgfpathmoveto{\pgfqpoint{2.551354in}{0.822573in}}%
\pgfpathcurveto{\pgfqpoint{2.559590in}{0.822573in}}{\pgfqpoint{2.567490in}{0.825845in}}{\pgfqpoint{2.573314in}{0.831669in}}%
\pgfpathcurveto{\pgfqpoint{2.579138in}{0.837493in}}{\pgfqpoint{2.582410in}{0.845393in}}{\pgfqpoint{2.582410in}{0.853629in}}%
\pgfpathcurveto{\pgfqpoint{2.582410in}{0.861866in}}{\pgfqpoint{2.579138in}{0.869766in}}{\pgfqpoint{2.573314in}{0.875590in}}%
\pgfpathcurveto{\pgfqpoint{2.567490in}{0.881413in}}{\pgfqpoint{2.559590in}{0.884686in}}{\pgfqpoint{2.551354in}{0.884686in}}%
\pgfpathcurveto{\pgfqpoint{2.543118in}{0.884686in}}{\pgfqpoint{2.535218in}{0.881413in}}{\pgfqpoint{2.529394in}{0.875590in}}%
\pgfpathcurveto{\pgfqpoint{2.523570in}{0.869766in}}{\pgfqpoint{2.520297in}{0.861866in}}{\pgfqpoint{2.520297in}{0.853629in}}%
\pgfpathcurveto{\pgfqpoint{2.520297in}{0.845393in}}{\pgfqpoint{2.523570in}{0.837493in}}{\pgfqpoint{2.529394in}{0.831669in}}%
\pgfpathcurveto{\pgfqpoint{2.535218in}{0.825845in}}{\pgfqpoint{2.543118in}{0.822573in}}{\pgfqpoint{2.551354in}{0.822573in}}%
\pgfpathclose%
\pgfusepath{stroke,fill}%
\end{pgfscope}%
\begin{pgfscope}%
\pgfpathrectangle{\pgfqpoint{0.100000in}{0.212622in}}{\pgfqpoint{3.696000in}{3.696000in}}%
\pgfusepath{clip}%
\pgfsetbuttcap%
\pgfsetroundjoin%
\definecolor{currentfill}{rgb}{0.121569,0.466667,0.705882}%
\pgfsetfillcolor{currentfill}%
\pgfsetfillopacity{0.998408}%
\pgfsetlinewidth{1.003750pt}%
\definecolor{currentstroke}{rgb}{0.121569,0.466667,0.705882}%
\pgfsetstrokecolor{currentstroke}%
\pgfsetstrokeopacity{0.998408}%
\pgfsetdash{}{0pt}%
\pgfpathmoveto{\pgfqpoint{2.551349in}{0.822564in}}%
\pgfpathcurveto{\pgfqpoint{2.559586in}{0.822564in}}{\pgfqpoint{2.567486in}{0.825836in}}{\pgfqpoint{2.573310in}{0.831660in}}%
\pgfpathcurveto{\pgfqpoint{2.579134in}{0.837484in}}{\pgfqpoint{2.582406in}{0.845384in}}{\pgfqpoint{2.582406in}{0.853620in}}%
\pgfpathcurveto{\pgfqpoint{2.582406in}{0.861856in}}{\pgfqpoint{2.579134in}{0.869757in}}{\pgfqpoint{2.573310in}{0.875580in}}%
\pgfpathcurveto{\pgfqpoint{2.567486in}{0.881404in}}{\pgfqpoint{2.559586in}{0.884677in}}{\pgfqpoint{2.551349in}{0.884677in}}%
\pgfpathcurveto{\pgfqpoint{2.543113in}{0.884677in}}{\pgfqpoint{2.535213in}{0.881404in}}{\pgfqpoint{2.529389in}{0.875580in}}%
\pgfpathcurveto{\pgfqpoint{2.523565in}{0.869757in}}{\pgfqpoint{2.520293in}{0.861856in}}{\pgfqpoint{2.520293in}{0.853620in}}%
\pgfpathcurveto{\pgfqpoint{2.520293in}{0.845384in}}{\pgfqpoint{2.523565in}{0.837484in}}{\pgfqpoint{2.529389in}{0.831660in}}%
\pgfpathcurveto{\pgfqpoint{2.535213in}{0.825836in}}{\pgfqpoint{2.543113in}{0.822564in}}{\pgfqpoint{2.551349in}{0.822564in}}%
\pgfpathclose%
\pgfusepath{stroke,fill}%
\end{pgfscope}%
\begin{pgfscope}%
\pgfpathrectangle{\pgfqpoint{0.100000in}{0.212622in}}{\pgfqpoint{3.696000in}{3.696000in}}%
\pgfusepath{clip}%
\pgfsetbuttcap%
\pgfsetroundjoin%
\definecolor{currentfill}{rgb}{0.121569,0.466667,0.705882}%
\pgfsetfillcolor{currentfill}%
\pgfsetfillopacity{0.998411}%
\pgfsetlinewidth{1.003750pt}%
\definecolor{currentstroke}{rgb}{0.121569,0.466667,0.705882}%
\pgfsetstrokecolor{currentstroke}%
\pgfsetstrokeopacity{0.998411}%
\pgfsetdash{}{0pt}%
\pgfpathmoveto{\pgfqpoint{2.551341in}{0.822547in}}%
\pgfpathcurveto{\pgfqpoint{2.559578in}{0.822547in}}{\pgfqpoint{2.567478in}{0.825820in}}{\pgfqpoint{2.573302in}{0.831644in}}%
\pgfpathcurveto{\pgfqpoint{2.579126in}{0.837468in}}{\pgfqpoint{2.582398in}{0.845368in}}{\pgfqpoint{2.582398in}{0.853604in}}%
\pgfpathcurveto{\pgfqpoint{2.582398in}{0.861840in}}{\pgfqpoint{2.579126in}{0.869740in}}{\pgfqpoint{2.573302in}{0.875564in}}%
\pgfpathcurveto{\pgfqpoint{2.567478in}{0.881388in}}{\pgfqpoint{2.559578in}{0.884660in}}{\pgfqpoint{2.551341in}{0.884660in}}%
\pgfpathcurveto{\pgfqpoint{2.543105in}{0.884660in}}{\pgfqpoint{2.535205in}{0.881388in}}{\pgfqpoint{2.529381in}{0.875564in}}%
\pgfpathcurveto{\pgfqpoint{2.523557in}{0.869740in}}{\pgfqpoint{2.520285in}{0.861840in}}{\pgfqpoint{2.520285in}{0.853604in}}%
\pgfpathcurveto{\pgfqpoint{2.520285in}{0.845368in}}{\pgfqpoint{2.523557in}{0.837468in}}{\pgfqpoint{2.529381in}{0.831644in}}%
\pgfpathcurveto{\pgfqpoint{2.535205in}{0.825820in}}{\pgfqpoint{2.543105in}{0.822547in}}{\pgfqpoint{2.551341in}{0.822547in}}%
\pgfpathclose%
\pgfusepath{stroke,fill}%
\end{pgfscope}%
\begin{pgfscope}%
\pgfpathrectangle{\pgfqpoint{0.100000in}{0.212622in}}{\pgfqpoint{3.696000in}{3.696000in}}%
\pgfusepath{clip}%
\pgfsetbuttcap%
\pgfsetroundjoin%
\definecolor{currentfill}{rgb}{0.121569,0.466667,0.705882}%
\pgfsetfillcolor{currentfill}%
\pgfsetfillopacity{0.998416}%
\pgfsetlinewidth{1.003750pt}%
\definecolor{currentstroke}{rgb}{0.121569,0.466667,0.705882}%
\pgfsetstrokecolor{currentstroke}%
\pgfsetstrokeopacity{0.998416}%
\pgfsetdash{}{0pt}%
\pgfpathmoveto{\pgfqpoint{2.551327in}{0.822516in}}%
\pgfpathcurveto{\pgfqpoint{2.559563in}{0.822516in}}{\pgfqpoint{2.567463in}{0.825788in}}{\pgfqpoint{2.573287in}{0.831612in}}%
\pgfpathcurveto{\pgfqpoint{2.579111in}{0.837436in}}{\pgfqpoint{2.582383in}{0.845336in}}{\pgfqpoint{2.582383in}{0.853572in}}%
\pgfpathcurveto{\pgfqpoint{2.582383in}{0.861809in}}{\pgfqpoint{2.579111in}{0.869709in}}{\pgfqpoint{2.573287in}{0.875533in}}%
\pgfpathcurveto{\pgfqpoint{2.567463in}{0.881357in}}{\pgfqpoint{2.559563in}{0.884629in}}{\pgfqpoint{2.551327in}{0.884629in}}%
\pgfpathcurveto{\pgfqpoint{2.543090in}{0.884629in}}{\pgfqpoint{2.535190in}{0.881357in}}{\pgfqpoint{2.529366in}{0.875533in}}%
\pgfpathcurveto{\pgfqpoint{2.523543in}{0.869709in}}{\pgfqpoint{2.520270in}{0.861809in}}{\pgfqpoint{2.520270in}{0.853572in}}%
\pgfpathcurveto{\pgfqpoint{2.520270in}{0.845336in}}{\pgfqpoint{2.523543in}{0.837436in}}{\pgfqpoint{2.529366in}{0.831612in}}%
\pgfpathcurveto{\pgfqpoint{2.535190in}{0.825788in}}{\pgfqpoint{2.543090in}{0.822516in}}{\pgfqpoint{2.551327in}{0.822516in}}%
\pgfpathclose%
\pgfusepath{stroke,fill}%
\end{pgfscope}%
\begin{pgfscope}%
\pgfpathrectangle{\pgfqpoint{0.100000in}{0.212622in}}{\pgfqpoint{3.696000in}{3.696000in}}%
\pgfusepath{clip}%
\pgfsetbuttcap%
\pgfsetroundjoin%
\definecolor{currentfill}{rgb}{0.121569,0.466667,0.705882}%
\pgfsetfillcolor{currentfill}%
\pgfsetfillopacity{0.998425}%
\pgfsetlinewidth{1.003750pt}%
\definecolor{currentstroke}{rgb}{0.121569,0.466667,0.705882}%
\pgfsetstrokecolor{currentstroke}%
\pgfsetstrokeopacity{0.998425}%
\pgfsetdash{}{0pt}%
\pgfpathmoveto{\pgfqpoint{2.551302in}{0.822456in}}%
\pgfpathcurveto{\pgfqpoint{2.559538in}{0.822456in}}{\pgfqpoint{2.567438in}{0.825728in}}{\pgfqpoint{2.573262in}{0.831552in}}%
\pgfpathcurveto{\pgfqpoint{2.579086in}{0.837376in}}{\pgfqpoint{2.582359in}{0.845276in}}{\pgfqpoint{2.582359in}{0.853513in}}%
\pgfpathcurveto{\pgfqpoint{2.582359in}{0.861749in}}{\pgfqpoint{2.579086in}{0.869649in}}{\pgfqpoint{2.573262in}{0.875473in}}%
\pgfpathcurveto{\pgfqpoint{2.567438in}{0.881297in}}{\pgfqpoint{2.559538in}{0.884569in}}{\pgfqpoint{2.551302in}{0.884569in}}%
\pgfpathcurveto{\pgfqpoint{2.543066in}{0.884569in}}{\pgfqpoint{2.535166in}{0.881297in}}{\pgfqpoint{2.529342in}{0.875473in}}%
\pgfpathcurveto{\pgfqpoint{2.523518in}{0.869649in}}{\pgfqpoint{2.520246in}{0.861749in}}{\pgfqpoint{2.520246in}{0.853513in}}%
\pgfpathcurveto{\pgfqpoint{2.520246in}{0.845276in}}{\pgfqpoint{2.523518in}{0.837376in}}{\pgfqpoint{2.529342in}{0.831552in}}%
\pgfpathcurveto{\pgfqpoint{2.535166in}{0.825728in}}{\pgfqpoint{2.543066in}{0.822456in}}{\pgfqpoint{2.551302in}{0.822456in}}%
\pgfpathclose%
\pgfusepath{stroke,fill}%
\end{pgfscope}%
\begin{pgfscope}%
\pgfpathrectangle{\pgfqpoint{0.100000in}{0.212622in}}{\pgfqpoint{3.696000in}{3.696000in}}%
\pgfusepath{clip}%
\pgfsetbuttcap%
\pgfsetroundjoin%
\definecolor{currentfill}{rgb}{0.121569,0.466667,0.705882}%
\pgfsetfillcolor{currentfill}%
\pgfsetfillopacity{0.998440}%
\pgfsetlinewidth{1.003750pt}%
\definecolor{currentstroke}{rgb}{0.121569,0.466667,0.705882}%
\pgfsetstrokecolor{currentstroke}%
\pgfsetstrokeopacity{0.998440}%
\pgfsetdash{}{0pt}%
\pgfpathmoveto{\pgfqpoint{2.551255in}{0.822344in}}%
\pgfpathcurveto{\pgfqpoint{2.559491in}{0.822344in}}{\pgfqpoint{2.567391in}{0.825616in}}{\pgfqpoint{2.573215in}{0.831440in}}%
\pgfpathcurveto{\pgfqpoint{2.579039in}{0.837264in}}{\pgfqpoint{2.582311in}{0.845164in}}{\pgfqpoint{2.582311in}{0.853400in}}%
\pgfpathcurveto{\pgfqpoint{2.582311in}{0.861636in}}{\pgfqpoint{2.579039in}{0.869536in}}{\pgfqpoint{2.573215in}{0.875360in}}%
\pgfpathcurveto{\pgfqpoint{2.567391in}{0.881184in}}{\pgfqpoint{2.559491in}{0.884457in}}{\pgfqpoint{2.551255in}{0.884457in}}%
\pgfpathcurveto{\pgfqpoint{2.543019in}{0.884457in}}{\pgfqpoint{2.535119in}{0.881184in}}{\pgfqpoint{2.529295in}{0.875360in}}%
\pgfpathcurveto{\pgfqpoint{2.523471in}{0.869536in}}{\pgfqpoint{2.520198in}{0.861636in}}{\pgfqpoint{2.520198in}{0.853400in}}%
\pgfpathcurveto{\pgfqpoint{2.520198in}{0.845164in}}{\pgfqpoint{2.523471in}{0.837264in}}{\pgfqpoint{2.529295in}{0.831440in}}%
\pgfpathcurveto{\pgfqpoint{2.535119in}{0.825616in}}{\pgfqpoint{2.543019in}{0.822344in}}{\pgfqpoint{2.551255in}{0.822344in}}%
\pgfpathclose%
\pgfusepath{stroke,fill}%
\end{pgfscope}%
\begin{pgfscope}%
\pgfpathrectangle{\pgfqpoint{0.100000in}{0.212622in}}{\pgfqpoint{3.696000in}{3.696000in}}%
\pgfusepath{clip}%
\pgfsetbuttcap%
\pgfsetroundjoin%
\definecolor{currentfill}{rgb}{0.121569,0.466667,0.705882}%
\pgfsetfillcolor{currentfill}%
\pgfsetfillopacity{0.998467}%
\pgfsetlinewidth{1.003750pt}%
\definecolor{currentstroke}{rgb}{0.121569,0.466667,0.705882}%
\pgfsetstrokecolor{currentstroke}%
\pgfsetstrokeopacity{0.998467}%
\pgfsetdash{}{0pt}%
\pgfpathmoveto{\pgfqpoint{2.551158in}{0.822150in}}%
\pgfpathcurveto{\pgfqpoint{2.559395in}{0.822150in}}{\pgfqpoint{2.567295in}{0.825423in}}{\pgfqpoint{2.573118in}{0.831246in}}%
\pgfpathcurveto{\pgfqpoint{2.578942in}{0.837070in}}{\pgfqpoint{2.582215in}{0.844970in}}{\pgfqpoint{2.582215in}{0.853207in}}%
\pgfpathcurveto{\pgfqpoint{2.582215in}{0.861443in}}{\pgfqpoint{2.578942in}{0.869343in}}{\pgfqpoint{2.573118in}{0.875167in}}%
\pgfpathcurveto{\pgfqpoint{2.567295in}{0.880991in}}{\pgfqpoint{2.559395in}{0.884263in}}{\pgfqpoint{2.551158in}{0.884263in}}%
\pgfpathcurveto{\pgfqpoint{2.542922in}{0.884263in}}{\pgfqpoint{2.535022in}{0.880991in}}{\pgfqpoint{2.529198in}{0.875167in}}%
\pgfpathcurveto{\pgfqpoint{2.523374in}{0.869343in}}{\pgfqpoint{2.520102in}{0.861443in}}{\pgfqpoint{2.520102in}{0.853207in}}%
\pgfpathcurveto{\pgfqpoint{2.520102in}{0.844970in}}{\pgfqpoint{2.523374in}{0.837070in}}{\pgfqpoint{2.529198in}{0.831246in}}%
\pgfpathcurveto{\pgfqpoint{2.535022in}{0.825423in}}{\pgfqpoint{2.542922in}{0.822150in}}{\pgfqpoint{2.551158in}{0.822150in}}%
\pgfpathclose%
\pgfusepath{stroke,fill}%
\end{pgfscope}%
\begin{pgfscope}%
\pgfpathrectangle{\pgfqpoint{0.100000in}{0.212622in}}{\pgfqpoint{3.696000in}{3.696000in}}%
\pgfusepath{clip}%
\pgfsetbuttcap%
\pgfsetroundjoin%
\definecolor{currentfill}{rgb}{0.121569,0.466667,0.705882}%
\pgfsetfillcolor{currentfill}%
\pgfsetfillopacity{0.998513}%
\pgfsetlinewidth{1.003750pt}%
\definecolor{currentstroke}{rgb}{0.121569,0.466667,0.705882}%
\pgfsetstrokecolor{currentstroke}%
\pgfsetstrokeopacity{0.998513}%
\pgfsetdash{}{0pt}%
\pgfpathmoveto{\pgfqpoint{2.550988in}{0.821778in}}%
\pgfpathcurveto{\pgfqpoint{2.559224in}{0.821778in}}{\pgfqpoint{2.567124in}{0.825050in}}{\pgfqpoint{2.572948in}{0.830874in}}%
\pgfpathcurveto{\pgfqpoint{2.578772in}{0.836698in}}{\pgfqpoint{2.582045in}{0.844598in}}{\pgfqpoint{2.582045in}{0.852834in}}%
\pgfpathcurveto{\pgfqpoint{2.582045in}{0.861070in}}{\pgfqpoint{2.578772in}{0.868971in}}{\pgfqpoint{2.572948in}{0.874794in}}%
\pgfpathcurveto{\pgfqpoint{2.567124in}{0.880618in}}{\pgfqpoint{2.559224in}{0.883891in}}{\pgfqpoint{2.550988in}{0.883891in}}%
\pgfpathcurveto{\pgfqpoint{2.542752in}{0.883891in}}{\pgfqpoint{2.534852in}{0.880618in}}{\pgfqpoint{2.529028in}{0.874794in}}%
\pgfpathcurveto{\pgfqpoint{2.523204in}{0.868971in}}{\pgfqpoint{2.519932in}{0.861070in}}{\pgfqpoint{2.519932in}{0.852834in}}%
\pgfpathcurveto{\pgfqpoint{2.519932in}{0.844598in}}{\pgfqpoint{2.523204in}{0.836698in}}{\pgfqpoint{2.529028in}{0.830874in}}%
\pgfpathcurveto{\pgfqpoint{2.534852in}{0.825050in}}{\pgfqpoint{2.542752in}{0.821778in}}{\pgfqpoint{2.550988in}{0.821778in}}%
\pgfpathclose%
\pgfusepath{stroke,fill}%
\end{pgfscope}%
\begin{pgfscope}%
\pgfpathrectangle{\pgfqpoint{0.100000in}{0.212622in}}{\pgfqpoint{3.696000in}{3.696000in}}%
\pgfusepath{clip}%
\pgfsetbuttcap%
\pgfsetroundjoin%
\definecolor{currentfill}{rgb}{0.121569,0.466667,0.705882}%
\pgfsetfillcolor{currentfill}%
\pgfsetfillopacity{0.998598}%
\pgfsetlinewidth{1.003750pt}%
\definecolor{currentstroke}{rgb}{0.121569,0.466667,0.705882}%
\pgfsetstrokecolor{currentstroke}%
\pgfsetstrokeopacity{0.998598}%
\pgfsetdash{}{0pt}%
\pgfpathmoveto{\pgfqpoint{2.550643in}{0.821162in}}%
\pgfpathcurveto{\pgfqpoint{2.558880in}{0.821162in}}{\pgfqpoint{2.566780in}{0.824434in}}{\pgfqpoint{2.572604in}{0.830258in}}%
\pgfpathcurveto{\pgfqpoint{2.578428in}{0.836082in}}{\pgfqpoint{2.581700in}{0.843982in}}{\pgfqpoint{2.581700in}{0.852219in}}%
\pgfpathcurveto{\pgfqpoint{2.581700in}{0.860455in}}{\pgfqpoint{2.578428in}{0.868355in}}{\pgfqpoint{2.572604in}{0.874179in}}%
\pgfpathcurveto{\pgfqpoint{2.566780in}{0.880003in}}{\pgfqpoint{2.558880in}{0.883275in}}{\pgfqpoint{2.550643in}{0.883275in}}%
\pgfpathcurveto{\pgfqpoint{2.542407in}{0.883275in}}{\pgfqpoint{2.534507in}{0.880003in}}{\pgfqpoint{2.528683in}{0.874179in}}%
\pgfpathcurveto{\pgfqpoint{2.522859in}{0.868355in}}{\pgfqpoint{2.519587in}{0.860455in}}{\pgfqpoint{2.519587in}{0.852219in}}%
\pgfpathcurveto{\pgfqpoint{2.519587in}{0.843982in}}{\pgfqpoint{2.522859in}{0.836082in}}{\pgfqpoint{2.528683in}{0.830258in}}%
\pgfpathcurveto{\pgfqpoint{2.534507in}{0.824434in}}{\pgfqpoint{2.542407in}{0.821162in}}{\pgfqpoint{2.550643in}{0.821162in}}%
\pgfpathclose%
\pgfusepath{stroke,fill}%
\end{pgfscope}%
\begin{pgfscope}%
\pgfpathrectangle{\pgfqpoint{0.100000in}{0.212622in}}{\pgfqpoint{3.696000in}{3.696000in}}%
\pgfusepath{clip}%
\pgfsetbuttcap%
\pgfsetroundjoin%
\definecolor{currentfill}{rgb}{0.121569,0.466667,0.705882}%
\pgfsetfillcolor{currentfill}%
\pgfsetfillopacity{0.998728}%
\pgfsetlinewidth{1.003750pt}%
\definecolor{currentstroke}{rgb}{0.121569,0.466667,0.705882}%
\pgfsetstrokecolor{currentstroke}%
\pgfsetstrokeopacity{0.998728}%
\pgfsetdash{}{0pt}%
\pgfpathmoveto{\pgfqpoint{2.549957in}{0.820052in}}%
\pgfpathcurveto{\pgfqpoint{2.558193in}{0.820052in}}{\pgfqpoint{2.566093in}{0.823324in}}{\pgfqpoint{2.571917in}{0.829148in}}%
\pgfpathcurveto{\pgfqpoint{2.577741in}{0.834972in}}{\pgfqpoint{2.581013in}{0.842872in}}{\pgfqpoint{2.581013in}{0.851109in}}%
\pgfpathcurveto{\pgfqpoint{2.581013in}{0.859345in}}{\pgfqpoint{2.577741in}{0.867245in}}{\pgfqpoint{2.571917in}{0.873069in}}%
\pgfpathcurveto{\pgfqpoint{2.566093in}{0.878893in}}{\pgfqpoint{2.558193in}{0.882165in}}{\pgfqpoint{2.549957in}{0.882165in}}%
\pgfpathcurveto{\pgfqpoint{2.541720in}{0.882165in}}{\pgfqpoint{2.533820in}{0.878893in}}{\pgfqpoint{2.527996in}{0.873069in}}%
\pgfpathcurveto{\pgfqpoint{2.522172in}{0.867245in}}{\pgfqpoint{2.518900in}{0.859345in}}{\pgfqpoint{2.518900in}{0.851109in}}%
\pgfpathcurveto{\pgfqpoint{2.518900in}{0.842872in}}{\pgfqpoint{2.522172in}{0.834972in}}{\pgfqpoint{2.527996in}{0.829148in}}%
\pgfpathcurveto{\pgfqpoint{2.533820in}{0.823324in}}{\pgfqpoint{2.541720in}{0.820052in}}{\pgfqpoint{2.549957in}{0.820052in}}%
\pgfpathclose%
\pgfusepath{stroke,fill}%
\end{pgfscope}%
\begin{pgfscope}%
\pgfpathrectangle{\pgfqpoint{0.100000in}{0.212622in}}{\pgfqpoint{3.696000in}{3.696000in}}%
\pgfusepath{clip}%
\pgfsetbuttcap%
\pgfsetroundjoin%
\definecolor{currentfill}{rgb}{0.121569,0.466667,0.705882}%
\pgfsetfillcolor{currentfill}%
\pgfsetfillopacity{0.998884}%
\pgfsetlinewidth{1.003750pt}%
\definecolor{currentstroke}{rgb}{0.121569,0.466667,0.705882}%
\pgfsetstrokecolor{currentstroke}%
\pgfsetstrokeopacity{0.998884}%
\pgfsetdash{}{0pt}%
\pgfpathmoveto{\pgfqpoint{2.548638in}{0.817856in}}%
\pgfpathcurveto{\pgfqpoint{2.556874in}{0.817856in}}{\pgfqpoint{2.564774in}{0.821128in}}{\pgfqpoint{2.570598in}{0.826952in}}%
\pgfpathcurveto{\pgfqpoint{2.576422in}{0.832776in}}{\pgfqpoint{2.579694in}{0.840676in}}{\pgfqpoint{2.579694in}{0.848912in}}%
\pgfpathcurveto{\pgfqpoint{2.579694in}{0.857149in}}{\pgfqpoint{2.576422in}{0.865049in}}{\pgfqpoint{2.570598in}{0.870873in}}%
\pgfpathcurveto{\pgfqpoint{2.564774in}{0.876697in}}{\pgfqpoint{2.556874in}{0.879969in}}{\pgfqpoint{2.548638in}{0.879969in}}%
\pgfpathcurveto{\pgfqpoint{2.540402in}{0.879969in}}{\pgfqpoint{2.532502in}{0.876697in}}{\pgfqpoint{2.526678in}{0.870873in}}%
\pgfpathcurveto{\pgfqpoint{2.520854in}{0.865049in}}{\pgfqpoint{2.517581in}{0.857149in}}{\pgfqpoint{2.517581in}{0.848912in}}%
\pgfpathcurveto{\pgfqpoint{2.517581in}{0.840676in}}{\pgfqpoint{2.520854in}{0.832776in}}{\pgfqpoint{2.526678in}{0.826952in}}%
\pgfpathcurveto{\pgfqpoint{2.532502in}{0.821128in}}{\pgfqpoint{2.540402in}{0.817856in}}{\pgfqpoint{2.548638in}{0.817856in}}%
\pgfpathclose%
\pgfusepath{stroke,fill}%
\end{pgfscope}%
\begin{pgfscope}%
\pgfpathrectangle{\pgfqpoint{0.100000in}{0.212622in}}{\pgfqpoint{3.696000in}{3.696000in}}%
\pgfusepath{clip}%
\pgfsetbuttcap%
\pgfsetroundjoin%
\definecolor{currentfill}{rgb}{0.121569,0.466667,0.705882}%
\pgfsetfillcolor{currentfill}%
\pgfsetfillopacity{0.999127}%
\pgfsetlinewidth{1.003750pt}%
\definecolor{currentstroke}{rgb}{0.121569,0.466667,0.705882}%
\pgfsetstrokecolor{currentstroke}%
\pgfsetstrokeopacity{0.999127}%
\pgfsetdash{}{0pt}%
\pgfpathmoveto{\pgfqpoint{2.508662in}{0.787885in}}%
\pgfpathcurveto{\pgfqpoint{2.516898in}{0.787885in}}{\pgfqpoint{2.524798in}{0.791157in}}{\pgfqpoint{2.530622in}{0.796981in}}%
\pgfpathcurveto{\pgfqpoint{2.536446in}{0.802805in}}{\pgfqpoint{2.539719in}{0.810705in}}{\pgfqpoint{2.539719in}{0.818941in}}%
\pgfpathcurveto{\pgfqpoint{2.539719in}{0.827177in}}{\pgfqpoint{2.536446in}{0.835078in}}{\pgfqpoint{2.530622in}{0.840901in}}%
\pgfpathcurveto{\pgfqpoint{2.524798in}{0.846725in}}{\pgfqpoint{2.516898in}{0.849998in}}{\pgfqpoint{2.508662in}{0.849998in}}%
\pgfpathcurveto{\pgfqpoint{2.500426in}{0.849998in}}{\pgfqpoint{2.492526in}{0.846725in}}{\pgfqpoint{2.486702in}{0.840901in}}%
\pgfpathcurveto{\pgfqpoint{2.480878in}{0.835078in}}{\pgfqpoint{2.477606in}{0.827177in}}{\pgfqpoint{2.477606in}{0.818941in}}%
\pgfpathcurveto{\pgfqpoint{2.477606in}{0.810705in}}{\pgfqpoint{2.480878in}{0.802805in}}{\pgfqpoint{2.486702in}{0.796981in}}%
\pgfpathcurveto{\pgfqpoint{2.492526in}{0.791157in}}{\pgfqpoint{2.500426in}{0.787885in}}{\pgfqpoint{2.508662in}{0.787885in}}%
\pgfpathclose%
\pgfusepath{stroke,fill}%
\end{pgfscope}%
\begin{pgfscope}%
\pgfpathrectangle{\pgfqpoint{0.100000in}{0.212622in}}{\pgfqpoint{3.696000in}{3.696000in}}%
\pgfusepath{clip}%
\pgfsetbuttcap%
\pgfsetroundjoin%
\definecolor{currentfill}{rgb}{0.121569,0.466667,0.705882}%
\pgfsetfillcolor{currentfill}%
\pgfsetfillopacity{0.999226}%
\pgfsetlinewidth{1.003750pt}%
\definecolor{currentstroke}{rgb}{0.121569,0.466667,0.705882}%
\pgfsetstrokecolor{currentstroke}%
\pgfsetstrokeopacity{0.999226}%
\pgfsetdash{}{0pt}%
\pgfpathmoveto{\pgfqpoint{2.546316in}{0.813924in}}%
\pgfpathcurveto{\pgfqpoint{2.554553in}{0.813924in}}{\pgfqpoint{2.562453in}{0.817196in}}{\pgfqpoint{2.568277in}{0.823020in}}%
\pgfpathcurveto{\pgfqpoint{2.574100in}{0.828844in}}{\pgfqpoint{2.577373in}{0.836744in}}{\pgfqpoint{2.577373in}{0.844980in}}%
\pgfpathcurveto{\pgfqpoint{2.577373in}{0.853216in}}{\pgfqpoint{2.574100in}{0.861116in}}{\pgfqpoint{2.568277in}{0.866940in}}%
\pgfpathcurveto{\pgfqpoint{2.562453in}{0.872764in}}{\pgfqpoint{2.554553in}{0.876037in}}{\pgfqpoint{2.546316in}{0.876037in}}%
\pgfpathcurveto{\pgfqpoint{2.538080in}{0.876037in}}{\pgfqpoint{2.530180in}{0.872764in}}{\pgfqpoint{2.524356in}{0.866940in}}%
\pgfpathcurveto{\pgfqpoint{2.518532in}{0.861116in}}{\pgfqpoint{2.515260in}{0.853216in}}{\pgfqpoint{2.515260in}{0.844980in}}%
\pgfpathcurveto{\pgfqpoint{2.515260in}{0.836744in}}{\pgfqpoint{2.518532in}{0.828844in}}{\pgfqpoint{2.524356in}{0.823020in}}%
\pgfpathcurveto{\pgfqpoint{2.530180in}{0.817196in}}{\pgfqpoint{2.538080in}{0.813924in}}{\pgfqpoint{2.546316in}{0.813924in}}%
\pgfpathclose%
\pgfusepath{stroke,fill}%
\end{pgfscope}%
\begin{pgfscope}%
\pgfpathrectangle{\pgfqpoint{0.100000in}{0.212622in}}{\pgfqpoint{3.696000in}{3.696000in}}%
\pgfusepath{clip}%
\pgfsetbuttcap%
\pgfsetroundjoin%
\definecolor{currentfill}{rgb}{0.121569,0.466667,0.705882}%
\pgfsetfillcolor{currentfill}%
\pgfsetfillopacity{0.999842}%
\pgfsetlinewidth{1.003750pt}%
\definecolor{currentstroke}{rgb}{0.121569,0.466667,0.705882}%
\pgfsetstrokecolor{currentstroke}%
\pgfsetstrokeopacity{0.999842}%
\pgfsetdash{}{0pt}%
\pgfpathmoveto{\pgfqpoint{2.524291in}{0.787573in}}%
\pgfpathcurveto{\pgfqpoint{2.532527in}{0.787573in}}{\pgfqpoint{2.540427in}{0.790846in}}{\pgfqpoint{2.546251in}{0.796669in}}%
\pgfpathcurveto{\pgfqpoint{2.552075in}{0.802493in}}{\pgfqpoint{2.555347in}{0.810393in}}{\pgfqpoint{2.555347in}{0.818630in}}%
\pgfpathcurveto{\pgfqpoint{2.555347in}{0.826866in}}{\pgfqpoint{2.552075in}{0.834766in}}{\pgfqpoint{2.546251in}{0.840590in}}%
\pgfpathcurveto{\pgfqpoint{2.540427in}{0.846414in}}{\pgfqpoint{2.532527in}{0.849686in}}{\pgfqpoint{2.524291in}{0.849686in}}%
\pgfpathcurveto{\pgfqpoint{2.516055in}{0.849686in}}{\pgfqpoint{2.508155in}{0.846414in}}{\pgfqpoint{2.502331in}{0.840590in}}%
\pgfpathcurveto{\pgfqpoint{2.496507in}{0.834766in}}{\pgfqpoint{2.493234in}{0.826866in}}{\pgfqpoint{2.493234in}{0.818630in}}%
\pgfpathcurveto{\pgfqpoint{2.493234in}{0.810393in}}{\pgfqpoint{2.496507in}{0.802493in}}{\pgfqpoint{2.502331in}{0.796669in}}%
\pgfpathcurveto{\pgfqpoint{2.508155in}{0.790846in}}{\pgfqpoint{2.516055in}{0.787573in}}{\pgfqpoint{2.524291in}{0.787573in}}%
\pgfpathclose%
\pgfusepath{stroke,fill}%
\end{pgfscope}%
\begin{pgfscope}%
\pgfpathrectangle{\pgfqpoint{0.100000in}{0.212622in}}{\pgfqpoint{3.696000in}{3.696000in}}%
\pgfusepath{clip}%
\pgfsetbuttcap%
\pgfsetroundjoin%
\definecolor{currentfill}{rgb}{0.121569,0.466667,0.705882}%
\pgfsetfillcolor{currentfill}%
\pgfsetfillopacity{0.999925}%
\pgfsetlinewidth{1.003750pt}%
\definecolor{currentstroke}{rgb}{0.121569,0.466667,0.705882}%
\pgfsetstrokecolor{currentstroke}%
\pgfsetstrokeopacity{0.999925}%
\pgfsetdash{}{0pt}%
\pgfpathmoveto{\pgfqpoint{2.533474in}{0.793551in}}%
\pgfpathcurveto{\pgfqpoint{2.541710in}{0.793551in}}{\pgfqpoint{2.549610in}{0.796823in}}{\pgfqpoint{2.555434in}{0.802647in}}%
\pgfpathcurveto{\pgfqpoint{2.561258in}{0.808471in}}{\pgfqpoint{2.564530in}{0.816371in}}{\pgfqpoint{2.564530in}{0.824607in}}%
\pgfpathcurveto{\pgfqpoint{2.564530in}{0.832843in}}{\pgfqpoint{2.561258in}{0.840743in}}{\pgfqpoint{2.555434in}{0.846567in}}%
\pgfpathcurveto{\pgfqpoint{2.549610in}{0.852391in}}{\pgfqpoint{2.541710in}{0.855664in}}{\pgfqpoint{2.533474in}{0.855664in}}%
\pgfpathcurveto{\pgfqpoint{2.525238in}{0.855664in}}{\pgfqpoint{2.517338in}{0.852391in}}{\pgfqpoint{2.511514in}{0.846567in}}%
\pgfpathcurveto{\pgfqpoint{2.505690in}{0.840743in}}{\pgfqpoint{2.502417in}{0.832843in}}{\pgfqpoint{2.502417in}{0.824607in}}%
\pgfpathcurveto{\pgfqpoint{2.502417in}{0.816371in}}{\pgfqpoint{2.505690in}{0.808471in}}{\pgfqpoint{2.511514in}{0.802647in}}%
\pgfpathcurveto{\pgfqpoint{2.517338in}{0.796823in}}{\pgfqpoint{2.525238in}{0.793551in}}{\pgfqpoint{2.533474in}{0.793551in}}%
\pgfpathclose%
\pgfusepath{stroke,fill}%
\end{pgfscope}%
\begin{pgfscope}%
\pgfpathrectangle{\pgfqpoint{0.100000in}{0.212622in}}{\pgfqpoint{3.696000in}{3.696000in}}%
\pgfusepath{clip}%
\pgfsetbuttcap%
\pgfsetroundjoin%
\definecolor{currentfill}{rgb}{0.121569,0.466667,0.705882}%
\pgfsetfillcolor{currentfill}%
\pgfsetlinewidth{1.003750pt}%
\definecolor{currentstroke}{rgb}{0.121569,0.466667,0.705882}%
\pgfsetstrokecolor{currentstroke}%
\pgfsetdash{}{0pt}%
\pgfpathmoveto{\pgfqpoint{2.542610in}{0.806397in}}%
\pgfpathcurveto{\pgfqpoint{2.550847in}{0.806397in}}{\pgfqpoint{2.558747in}{0.809669in}}{\pgfqpoint{2.564571in}{0.815493in}}%
\pgfpathcurveto{\pgfqpoint{2.570395in}{0.821317in}}{\pgfqpoint{2.573667in}{0.829217in}}{\pgfqpoint{2.573667in}{0.837454in}}%
\pgfpathcurveto{\pgfqpoint{2.573667in}{0.845690in}}{\pgfqpoint{2.570395in}{0.853590in}}{\pgfqpoint{2.564571in}{0.859414in}}%
\pgfpathcurveto{\pgfqpoint{2.558747in}{0.865238in}}{\pgfqpoint{2.550847in}{0.868510in}}{\pgfqpoint{2.542610in}{0.868510in}}%
\pgfpathcurveto{\pgfqpoint{2.534374in}{0.868510in}}{\pgfqpoint{2.526474in}{0.865238in}}{\pgfqpoint{2.520650in}{0.859414in}}%
\pgfpathcurveto{\pgfqpoint{2.514826in}{0.853590in}}{\pgfqpoint{2.511554in}{0.845690in}}{\pgfqpoint{2.511554in}{0.837454in}}%
\pgfpathcurveto{\pgfqpoint{2.511554in}{0.829217in}}{\pgfqpoint{2.514826in}{0.821317in}}{\pgfqpoint{2.520650in}{0.815493in}}%
\pgfpathcurveto{\pgfqpoint{2.526474in}{0.809669in}}{\pgfqpoint{2.534374in}{0.806397in}}{\pgfqpoint{2.542610in}{0.806397in}}%
\pgfpathclose%
\pgfusepath{stroke,fill}%
\end{pgfscope}%
\begin{pgfscope}%
\definecolor{textcolor}{rgb}{0.000000,0.000000,0.000000}%
\pgfsetstrokecolor{textcolor}%
\pgfsetfillcolor{textcolor}%
\pgftext[x=1.948000in,y=3.991956in,,base]{\color{textcolor}\rmfamily\fontsize{12.000000}{14.400000}\selectfont Davenport}%
\end{pgfscope}%
\begin{pgfscope}%
\pgfsetbuttcap%
\pgfsetmiterjoin%
\definecolor{currentfill}{rgb}{1.000000,1.000000,1.000000}%
\pgfsetfillcolor{currentfill}%
\pgfsetfillopacity{0.800000}%
\pgfsetlinewidth{1.003750pt}%
\definecolor{currentstroke}{rgb}{0.800000,0.800000,0.800000}%
\pgfsetstrokecolor{currentstroke}%
\pgfsetstrokeopacity{0.800000}%
\pgfsetdash{}{0pt}%
\pgfpathmoveto{\pgfqpoint{2.104889in}{3.410289in}}%
\pgfpathlineto{\pgfqpoint{3.698778in}{3.410289in}}%
\pgfpathquadraticcurveto{\pgfqpoint{3.726556in}{3.410289in}}{\pgfqpoint{3.726556in}{3.438067in}}%
\pgfpathlineto{\pgfqpoint{3.726556in}{3.811400in}}%
\pgfpathquadraticcurveto{\pgfqpoint{3.726556in}{3.839178in}}{\pgfqpoint{3.698778in}{3.839178in}}%
\pgfpathlineto{\pgfqpoint{2.104889in}{3.839178in}}%
\pgfpathquadraticcurveto{\pgfqpoint{2.077111in}{3.839178in}}{\pgfqpoint{2.077111in}{3.811400in}}%
\pgfpathlineto{\pgfqpoint{2.077111in}{3.438067in}}%
\pgfpathquadraticcurveto{\pgfqpoint{2.077111in}{3.410289in}}{\pgfqpoint{2.104889in}{3.410289in}}%
\pgfpathclose%
\pgfusepath{stroke,fill}%
\end{pgfscope}%
\begin{pgfscope}%
\pgfsetrectcap%
\pgfsetroundjoin%
\pgfsetlinewidth{1.505625pt}%
\definecolor{currentstroke}{rgb}{0.121569,0.466667,0.705882}%
\pgfsetstrokecolor{currentstroke}%
\pgfsetdash{}{0pt}%
\pgfpathmoveto{\pgfqpoint{2.132667in}{3.735011in}}%
\pgfpathlineto{\pgfqpoint{2.410444in}{3.735011in}}%
\pgfusepath{stroke}%
\end{pgfscope}%
\begin{pgfscope}%
\definecolor{textcolor}{rgb}{0.000000,0.000000,0.000000}%
\pgfsetstrokecolor{textcolor}%
\pgfsetfillcolor{textcolor}%
\pgftext[x=2.521555in,y=3.686400in,left,base]{\color{textcolor}\rmfamily\fontsize{10.000000}{12.000000}\selectfont Ground truth}%
\end{pgfscope}%
\begin{pgfscope}%
\pgfsetbuttcap%
\pgfsetroundjoin%
\definecolor{currentfill}{rgb}{0.121569,0.466667,0.705882}%
\pgfsetfillcolor{currentfill}%
\pgfsetlinewidth{1.003750pt}%
\definecolor{currentstroke}{rgb}{0.121569,0.466667,0.705882}%
\pgfsetstrokecolor{currentstroke}%
\pgfsetdash{}{0pt}%
\pgfsys@defobject{currentmarker}{\pgfqpoint{-0.031056in}{-0.031056in}}{\pgfqpoint{0.031056in}{0.031056in}}{%
\pgfpathmoveto{\pgfqpoint{0.000000in}{-0.031056in}}%
\pgfpathcurveto{\pgfqpoint{0.008236in}{-0.031056in}}{\pgfqpoint{0.016136in}{-0.027784in}}{\pgfqpoint{0.021960in}{-0.021960in}}%
\pgfpathcurveto{\pgfqpoint{0.027784in}{-0.016136in}}{\pgfqpoint{0.031056in}{-0.008236in}}{\pgfqpoint{0.031056in}{0.000000in}}%
\pgfpathcurveto{\pgfqpoint{0.031056in}{0.008236in}}{\pgfqpoint{0.027784in}{0.016136in}}{\pgfqpoint{0.021960in}{0.021960in}}%
\pgfpathcurveto{\pgfqpoint{0.016136in}{0.027784in}}{\pgfqpoint{0.008236in}{0.031056in}}{\pgfqpoint{0.000000in}{0.031056in}}%
\pgfpathcurveto{\pgfqpoint{-0.008236in}{0.031056in}}{\pgfqpoint{-0.016136in}{0.027784in}}{\pgfqpoint{-0.021960in}{0.021960in}}%
\pgfpathcurveto{\pgfqpoint{-0.027784in}{0.016136in}}{\pgfqpoint{-0.031056in}{0.008236in}}{\pgfqpoint{-0.031056in}{0.000000in}}%
\pgfpathcurveto{\pgfqpoint{-0.031056in}{-0.008236in}}{\pgfqpoint{-0.027784in}{-0.016136in}}{\pgfqpoint{-0.021960in}{-0.021960in}}%
\pgfpathcurveto{\pgfqpoint{-0.016136in}{-0.027784in}}{\pgfqpoint{-0.008236in}{-0.031056in}}{\pgfqpoint{0.000000in}{-0.031056in}}%
\pgfpathclose%
\pgfusepath{stroke,fill}%
}%
\begin{pgfscope}%
\pgfsys@transformshift{2.271555in}{3.529248in}%
\pgfsys@useobject{currentmarker}{}%
\end{pgfscope}%
\end{pgfscope}%
\begin{pgfscope}%
\definecolor{textcolor}{rgb}{0.000000,0.000000,0.000000}%
\pgfsetstrokecolor{textcolor}%
\pgfsetfillcolor{textcolor}%
\pgftext[x=2.521555in,y=3.492789in,left,base]{\color{textcolor}\rmfamily\fontsize{10.000000}{12.000000}\selectfont Estimated position}%
\end{pgfscope}%
\end{pgfpicture}%
\makeatother%
\endgroup%
}
%         \caption{Davenport's 3D position estimation had the lowest displacement error for the spiral experiment.}
%         \label{fig:spiral2D}
%     \end{subfigure}
%     \begin{subfigure}{0.49\textwidth}
%         \centering
%         \resizebox{1\linewidth}{!}{%% Creator: Matplotlib, PGF backend
%%
%% To include the figure in your LaTeX document, write
%%   \input{<filename>.pgf}
%%
%% Make sure the required packages are loaded in your preamble
%%   \usepackage{pgf}
%%
%% and, on pdftex
%%   \usepackage[utf8]{inputenc}\DeclareUnicodeCharacter{2212}{-}
%%
%% or, on luatex and xetex
%%   \usepackage{unicode-math}
%%
%% Figures using additional raster images can only be included by \input if
%% they are in the same directory as the main LaTeX file. For loading figures
%% from other directories you can use the `import` package
%%   \usepackage{import}
%%
%% and then include the figures with
%%   \import{<path to file>}{<filename>.pgf}
%%
%% Matplotlib used the following preamble
%%   \usepackage{fontspec}
%%
\begingroup%
\makeatletter%
\begin{pgfpicture}%
\pgfpathrectangle{\pgfpointorigin}{\pgfqpoint{4.342355in}{4.207622in}}%
\pgfusepath{use as bounding box, clip}%
\begin{pgfscope}%
\pgfsetbuttcap%
\pgfsetmiterjoin%
\definecolor{currentfill}{rgb}{1.000000,1.000000,1.000000}%
\pgfsetfillcolor{currentfill}%
\pgfsetlinewidth{0.000000pt}%
\definecolor{currentstroke}{rgb}{1.000000,1.000000,1.000000}%
\pgfsetstrokecolor{currentstroke}%
\pgfsetdash{}{0pt}%
\pgfpathmoveto{\pgfqpoint{0.000000in}{0.000000in}}%
\pgfpathlineto{\pgfqpoint{4.342355in}{0.000000in}}%
\pgfpathlineto{\pgfqpoint{4.342355in}{4.207622in}}%
\pgfpathlineto{\pgfqpoint{0.000000in}{4.207622in}}%
\pgfpathclose%
\pgfusepath{fill}%
\end{pgfscope}%
\begin{pgfscope}%
\pgfsetbuttcap%
\pgfsetmiterjoin%
\definecolor{currentfill}{rgb}{1.000000,1.000000,1.000000}%
\pgfsetfillcolor{currentfill}%
\pgfsetlinewidth{0.000000pt}%
\definecolor{currentstroke}{rgb}{0.000000,0.000000,0.000000}%
\pgfsetstrokecolor{currentstroke}%
\pgfsetstrokeopacity{0.000000}%
\pgfsetdash{}{0pt}%
\pgfpathmoveto{\pgfqpoint{0.100000in}{0.212622in}}%
\pgfpathlineto{\pgfqpoint{3.796000in}{0.212622in}}%
\pgfpathlineto{\pgfqpoint{3.796000in}{3.908622in}}%
\pgfpathlineto{\pgfqpoint{0.100000in}{3.908622in}}%
\pgfpathclose%
\pgfusepath{fill}%
\end{pgfscope}%
\begin{pgfscope}%
\pgfsetbuttcap%
\pgfsetmiterjoin%
\definecolor{currentfill}{rgb}{0.950000,0.950000,0.950000}%
\pgfsetfillcolor{currentfill}%
\pgfsetfillopacity{0.500000}%
\pgfsetlinewidth{1.003750pt}%
\definecolor{currentstroke}{rgb}{0.950000,0.950000,0.950000}%
\pgfsetstrokecolor{currentstroke}%
\pgfsetstrokeopacity{0.500000}%
\pgfsetdash{}{0pt}%
\pgfpathmoveto{\pgfqpoint{0.379073in}{1.123938in}}%
\pgfpathlineto{\pgfqpoint{1.599613in}{2.147018in}}%
\pgfpathlineto{\pgfqpoint{1.582647in}{3.622484in}}%
\pgfpathlineto{\pgfqpoint{0.303698in}{2.689165in}}%
\pgfusepath{stroke,fill}%
\end{pgfscope}%
\begin{pgfscope}%
\pgfsetbuttcap%
\pgfsetmiterjoin%
\definecolor{currentfill}{rgb}{0.900000,0.900000,0.900000}%
\pgfsetfillcolor{currentfill}%
\pgfsetfillopacity{0.500000}%
\pgfsetlinewidth{1.003750pt}%
\definecolor{currentstroke}{rgb}{0.900000,0.900000,0.900000}%
\pgfsetstrokecolor{currentstroke}%
\pgfsetstrokeopacity{0.500000}%
\pgfsetdash{}{0pt}%
\pgfpathmoveto{\pgfqpoint{1.599613in}{2.147018in}}%
\pgfpathlineto{\pgfqpoint{3.558144in}{1.577751in}}%
\pgfpathlineto{\pgfqpoint{3.628038in}{3.104037in}}%
\pgfpathlineto{\pgfqpoint{1.582647in}{3.622484in}}%
\pgfusepath{stroke,fill}%
\end{pgfscope}%
\begin{pgfscope}%
\pgfsetbuttcap%
\pgfsetmiterjoin%
\definecolor{currentfill}{rgb}{0.925000,0.925000,0.925000}%
\pgfsetfillcolor{currentfill}%
\pgfsetfillopacity{0.500000}%
\pgfsetlinewidth{1.003750pt}%
\definecolor{currentstroke}{rgb}{0.925000,0.925000,0.925000}%
\pgfsetstrokecolor{currentstroke}%
\pgfsetstrokeopacity{0.500000}%
\pgfsetdash{}{0pt}%
\pgfpathmoveto{\pgfqpoint{0.379073in}{1.123938in}}%
\pgfpathlineto{\pgfqpoint{2.455212in}{0.445871in}}%
\pgfpathlineto{\pgfqpoint{3.558144in}{1.577751in}}%
\pgfpathlineto{\pgfqpoint{1.599613in}{2.147018in}}%
\pgfusepath{stroke,fill}%
\end{pgfscope}%
\begin{pgfscope}%
\pgfsetrectcap%
\pgfsetroundjoin%
\pgfsetlinewidth{0.803000pt}%
\definecolor{currentstroke}{rgb}{0.000000,0.000000,0.000000}%
\pgfsetstrokecolor{currentstroke}%
\pgfsetdash{}{0pt}%
\pgfpathmoveto{\pgfqpoint{0.379073in}{1.123938in}}%
\pgfpathlineto{\pgfqpoint{2.455212in}{0.445871in}}%
\pgfusepath{stroke}%
\end{pgfscope}%
\begin{pgfscope}%
\definecolor{textcolor}{rgb}{0.000000,0.000000,0.000000}%
\pgfsetstrokecolor{textcolor}%
\pgfsetfillcolor{textcolor}%
\pgftext[x=0.730374in, y=0.408886in, left, base,rotate=341.912962]{\color{textcolor}\rmfamily\fontsize{10.000000}{12.000000}\selectfont Position X [\(\displaystyle m\)]}%
\end{pgfscope}%
\begin{pgfscope}%
\pgfsetbuttcap%
\pgfsetroundjoin%
\pgfsetlinewidth{0.803000pt}%
\definecolor{currentstroke}{rgb}{0.690196,0.690196,0.690196}%
\pgfsetstrokecolor{currentstroke}%
\pgfsetdash{}{0pt}%
\pgfpathmoveto{\pgfqpoint{0.845214in}{0.971696in}}%
\pgfpathlineto{\pgfqpoint{2.040856in}{2.018767in}}%
\pgfpathlineto{\pgfqpoint{2.042705in}{3.505872in}}%
\pgfusepath{stroke}%
\end{pgfscope}%
\begin{pgfscope}%
\pgfsetbuttcap%
\pgfsetroundjoin%
\pgfsetlinewidth{0.803000pt}%
\definecolor{currentstroke}{rgb}{0.690196,0.690196,0.690196}%
\pgfsetstrokecolor{currentstroke}%
\pgfsetdash{}{0pt}%
\pgfpathmoveto{\pgfqpoint{1.302144in}{0.822463in}}%
\pgfpathlineto{\pgfqpoint{2.472530in}{1.893296in}}%
\pgfpathlineto{\pgfqpoint{2.493210in}{3.391682in}}%
\pgfusepath{stroke}%
\end{pgfscope}%
\begin{pgfscope}%
\pgfsetbuttcap%
\pgfsetroundjoin%
\pgfsetlinewidth{0.803000pt}%
\definecolor{currentstroke}{rgb}{0.690196,0.690196,0.690196}%
\pgfsetstrokecolor{currentstroke}%
\pgfsetdash{}{0pt}%
\pgfpathmoveto{\pgfqpoint{1.769962in}{0.669673in}}%
\pgfpathlineto{\pgfqpoint{2.913623in}{1.765088in}}%
\pgfpathlineto{\pgfqpoint{2.953978in}{3.274891in}}%
\pgfusepath{stroke}%
\end{pgfscope}%
\begin{pgfscope}%
\pgfsetbuttcap%
\pgfsetroundjoin%
\pgfsetlinewidth{0.803000pt}%
\definecolor{currentstroke}{rgb}{0.690196,0.690196,0.690196}%
\pgfsetstrokecolor{currentstroke}%
\pgfsetdash{}{0pt}%
\pgfpathmoveto{\pgfqpoint{2.249064in}{0.513199in}}%
\pgfpathlineto{\pgfqpoint{3.364446in}{1.634051in}}%
\pgfpathlineto{\pgfqpoint{3.425364in}{3.155409in}}%
\pgfusepath{stroke}%
\end{pgfscope}%
\begin{pgfscope}%
\pgfsetrectcap%
\pgfsetroundjoin%
\pgfsetlinewidth{0.803000pt}%
\definecolor{currentstroke}{rgb}{0.000000,0.000000,0.000000}%
\pgfsetstrokecolor{currentstroke}%
\pgfsetdash{}{0pt}%
\pgfpathmoveto{\pgfqpoint{0.855633in}{0.980820in}}%
\pgfpathlineto{\pgfqpoint{0.824331in}{0.953408in}}%
\pgfusepath{stroke}%
\end{pgfscope}%
\begin{pgfscope}%
\definecolor{textcolor}{rgb}{0.000000,0.000000,0.000000}%
\pgfsetstrokecolor{textcolor}%
\pgfsetfillcolor{textcolor}%
\pgftext[x=0.740992in,y=0.751136in,,top]{\color{textcolor}\rmfamily\fontsize{10.000000}{12.000000}\selectfont \(\displaystyle {−10}\)}%
\end{pgfscope}%
\begin{pgfscope}%
\pgfsetrectcap%
\pgfsetroundjoin%
\pgfsetlinewidth{0.803000pt}%
\definecolor{currentstroke}{rgb}{0.000000,0.000000,0.000000}%
\pgfsetstrokecolor{currentstroke}%
\pgfsetdash{}{0pt}%
\pgfpathmoveto{\pgfqpoint{1.312352in}{0.831803in}}%
\pgfpathlineto{\pgfqpoint{1.281681in}{0.803741in}}%
\pgfusepath{stroke}%
\end{pgfscope}%
\begin{pgfscope}%
\definecolor{textcolor}{rgb}{0.000000,0.000000,0.000000}%
\pgfsetstrokecolor{textcolor}%
\pgfsetfillcolor{textcolor}%
\pgftext[x=1.198423in,y=0.598730in,,top]{\color{textcolor}\rmfamily\fontsize{10.000000}{12.000000}\selectfont \(\displaystyle {0}\)}%
\end{pgfscope}%
\begin{pgfscope}%
\pgfsetrectcap%
\pgfsetroundjoin%
\pgfsetlinewidth{0.803000pt}%
\definecolor{currentstroke}{rgb}{0.000000,0.000000,0.000000}%
\pgfsetstrokecolor{currentstroke}%
\pgfsetdash{}{0pt}%
\pgfpathmoveto{\pgfqpoint{1.779948in}{0.679238in}}%
\pgfpathlineto{\pgfqpoint{1.749947in}{0.650502in}}%
\pgfusepath{stroke}%
\end{pgfscope}%
\begin{pgfscope}%
\definecolor{textcolor}{rgb}{0.000000,0.000000,0.000000}%
\pgfsetstrokecolor{textcolor}%
\pgfsetfillcolor{textcolor}%
\pgftext[x=1.666798in,y=0.442678in,,top]{\color{textcolor}\rmfamily\fontsize{10.000000}{12.000000}\selectfont \(\displaystyle {10}\)}%
\end{pgfscope}%
\begin{pgfscope}%
\pgfsetrectcap%
\pgfsetroundjoin%
\pgfsetlinewidth{0.803000pt}%
\definecolor{currentstroke}{rgb}{0.000000,0.000000,0.000000}%
\pgfsetstrokecolor{currentstroke}%
\pgfsetdash{}{0pt}%
\pgfpathmoveto{\pgfqpoint{2.258812in}{0.522995in}}%
\pgfpathlineto{\pgfqpoint{2.229523in}{0.493562in}}%
\pgfusepath{stroke}%
\end{pgfscope}%
\begin{pgfscope}%
\definecolor{textcolor}{rgb}{0.000000,0.000000,0.000000}%
\pgfsetstrokecolor{textcolor}%
\pgfsetfillcolor{textcolor}%
\pgftext[x=2.146512in,y=0.282847in,,top]{\color{textcolor}\rmfamily\fontsize{10.000000}{12.000000}\selectfont \(\displaystyle {20}\)}%
\end{pgfscope}%
\begin{pgfscope}%
\pgfsetrectcap%
\pgfsetroundjoin%
\pgfsetlinewidth{0.803000pt}%
\definecolor{currentstroke}{rgb}{0.000000,0.000000,0.000000}%
\pgfsetstrokecolor{currentstroke}%
\pgfsetdash{}{0pt}%
\pgfpathmoveto{\pgfqpoint{3.558144in}{1.577751in}}%
\pgfpathlineto{\pgfqpoint{2.455212in}{0.445871in}}%
\pgfusepath{stroke}%
\end{pgfscope}%
\begin{pgfscope}%
\definecolor{textcolor}{rgb}{0.000000,0.000000,0.000000}%
\pgfsetstrokecolor{textcolor}%
\pgfsetfillcolor{textcolor}%
\pgftext[x=3.120747in, y=0.305657in, left, base,rotate=45.742112]{\color{textcolor}\rmfamily\fontsize{10.000000}{12.000000}\selectfont Position Y [\(\displaystyle m\)]}%
\end{pgfscope}%
\begin{pgfscope}%
\pgfsetbuttcap%
\pgfsetroundjoin%
\pgfsetlinewidth{0.803000pt}%
\definecolor{currentstroke}{rgb}{0.690196,0.690196,0.690196}%
\pgfsetstrokecolor{currentstroke}%
\pgfsetdash{}{0pt}%
\pgfpathmoveto{\pgfqpoint{0.481640in}{2.819019in}}%
\pgfpathlineto{\pgfqpoint{0.548335in}{1.265816in}}%
\pgfpathlineto{\pgfqpoint{2.608746in}{0.603435in}}%
\pgfusepath{stroke}%
\end{pgfscope}%
\begin{pgfscope}%
\pgfsetbuttcap%
\pgfsetroundjoin%
\pgfsetlinewidth{0.803000pt}%
\definecolor{currentstroke}{rgb}{0.690196,0.690196,0.690196}%
\pgfsetstrokecolor{currentstroke}%
\pgfsetdash{}{0pt}%
\pgfpathmoveto{\pgfqpoint{0.700762in}{2.978925in}}%
\pgfpathlineto{\pgfqpoint{0.757013in}{1.440734in}}%
\pgfpathlineto{\pgfqpoint{2.797776in}{0.797426in}}%
\pgfusepath{stroke}%
\end{pgfscope}%
\begin{pgfscope}%
\pgfsetbuttcap%
\pgfsetroundjoin%
\pgfsetlinewidth{0.803000pt}%
\definecolor{currentstroke}{rgb}{0.690196,0.690196,0.690196}%
\pgfsetstrokecolor{currentstroke}%
\pgfsetdash{}{0pt}%
\pgfpathmoveto{\pgfqpoint{0.913506in}{3.134175in}}%
\pgfpathlineto{\pgfqpoint{0.959876in}{1.610778in}}%
\pgfpathlineto{\pgfqpoint{2.981265in}{0.985730in}}%
\pgfusepath{stroke}%
\end{pgfscope}%
\begin{pgfscope}%
\pgfsetbuttcap%
\pgfsetroundjoin%
\pgfsetlinewidth{0.803000pt}%
\definecolor{currentstroke}{rgb}{0.690196,0.690196,0.690196}%
\pgfsetstrokecolor{currentstroke}%
\pgfsetdash{}{0pt}%
\pgfpathmoveto{\pgfqpoint{1.120145in}{3.284971in}}%
\pgfpathlineto{\pgfqpoint{1.157163in}{1.776148in}}%
\pgfpathlineto{\pgfqpoint{3.159452in}{1.168594in}}%
\pgfusepath{stroke}%
\end{pgfscope}%
\begin{pgfscope}%
\pgfsetbuttcap%
\pgfsetroundjoin%
\pgfsetlinewidth{0.803000pt}%
\definecolor{currentstroke}{rgb}{0.690196,0.690196,0.690196}%
\pgfsetstrokecolor{currentstroke}%
\pgfsetdash{}{0pt}%
\pgfpathmoveto{\pgfqpoint{1.320939in}{3.431501in}}%
\pgfpathlineto{\pgfqpoint{1.349103in}{1.937035in}}%
\pgfpathlineto{\pgfqpoint{3.332565in}{1.346251in}}%
\pgfusepath{stroke}%
\end{pgfscope}%
\begin{pgfscope}%
\pgfsetbuttcap%
\pgfsetroundjoin%
\pgfsetlinewidth{0.803000pt}%
\definecolor{currentstroke}{rgb}{0.690196,0.690196,0.690196}%
\pgfsetstrokecolor{currentstroke}%
\pgfsetdash{}{0pt}%
\pgfpathmoveto{\pgfqpoint{1.516132in}{3.573944in}}%
\pgfpathlineto{\pgfqpoint{1.535908in}{2.093619in}}%
\pgfpathlineto{\pgfqpoint{3.500818in}{1.518919in}}%
\pgfusepath{stroke}%
\end{pgfscope}%
\begin{pgfscope}%
\pgfsetrectcap%
\pgfsetroundjoin%
\pgfsetlinewidth{0.803000pt}%
\definecolor{currentstroke}{rgb}{0.000000,0.000000,0.000000}%
\pgfsetstrokecolor{currentstroke}%
\pgfsetdash{}{0pt}%
\pgfpathmoveto{\pgfqpoint{2.591389in}{0.609015in}}%
\pgfpathlineto{\pgfqpoint{2.643507in}{0.592260in}}%
\pgfusepath{stroke}%
\end{pgfscope}%
\begin{pgfscope}%
\definecolor{textcolor}{rgb}{0.000000,0.000000,0.000000}%
\pgfsetstrokecolor{textcolor}%
\pgfsetfillcolor{textcolor}%
\pgftext[x=2.786609in,y=0.418174in,,top]{\color{textcolor}\rmfamily\fontsize{10.000000}{12.000000}\selectfont \(\displaystyle {−10}\)}%
\end{pgfscope}%
\begin{pgfscope}%
\pgfsetrectcap%
\pgfsetroundjoin%
\pgfsetlinewidth{0.803000pt}%
\definecolor{currentstroke}{rgb}{0.000000,0.000000,0.000000}%
\pgfsetstrokecolor{currentstroke}%
\pgfsetdash{}{0pt}%
\pgfpathmoveto{\pgfqpoint{2.780597in}{0.802841in}}%
\pgfpathlineto{\pgfqpoint{2.832178in}{0.786581in}}%
\pgfusepath{stroke}%
\end{pgfscope}%
\begin{pgfscope}%
\definecolor{textcolor}{rgb}{0.000000,0.000000,0.000000}%
\pgfsetstrokecolor{textcolor}%
\pgfsetfillcolor{textcolor}%
\pgftext[x=2.973102in,y=0.615036in,,top]{\color{textcolor}\rmfamily\fontsize{10.000000}{12.000000}\selectfont \(\displaystyle {−5}\)}%
\end{pgfscope}%
\begin{pgfscope}%
\pgfsetrectcap%
\pgfsetroundjoin%
\pgfsetlinewidth{0.803000pt}%
\definecolor{currentstroke}{rgb}{0.000000,0.000000,0.000000}%
\pgfsetstrokecolor{currentstroke}%
\pgfsetdash{}{0pt}%
\pgfpathmoveto{\pgfqpoint{2.964261in}{0.990988in}}%
\pgfpathlineto{\pgfqpoint{3.015315in}{0.975201in}}%
\pgfusepath{stroke}%
\end{pgfscope}%
\begin{pgfscope}%
\definecolor{textcolor}{rgb}{0.000000,0.000000,0.000000}%
\pgfsetstrokecolor{textcolor}%
\pgfsetfillcolor{textcolor}%
\pgftext[x=3.154126in,y=0.806124in,,top]{\color{textcolor}\rmfamily\fontsize{10.000000}{12.000000}\selectfont \(\displaystyle {0}\)}%
\end{pgfscope}%
\begin{pgfscope}%
\pgfsetrectcap%
\pgfsetroundjoin%
\pgfsetlinewidth{0.803000pt}%
\definecolor{currentstroke}{rgb}{0.000000,0.000000,0.000000}%
\pgfsetstrokecolor{currentstroke}%
\pgfsetdash{}{0pt}%
\pgfpathmoveto{\pgfqpoint{3.142621in}{1.173702in}}%
\pgfpathlineto{\pgfqpoint{3.193156in}{1.158368in}}%
\pgfusepath{stroke}%
\end{pgfscope}%
\begin{pgfscope}%
\definecolor{textcolor}{rgb}{0.000000,0.000000,0.000000}%
\pgfsetstrokecolor{textcolor}%
\pgfsetfillcolor{textcolor}%
\pgftext[x=3.329916in,y=0.991688in,,top]{\color{textcolor}\rmfamily\fontsize{10.000000}{12.000000}\selectfont \(\displaystyle {5}\)}%
\end{pgfscope}%
\begin{pgfscope}%
\pgfsetrectcap%
\pgfsetroundjoin%
\pgfsetlinewidth{0.803000pt}%
\definecolor{currentstroke}{rgb}{0.000000,0.000000,0.000000}%
\pgfsetstrokecolor{currentstroke}%
\pgfsetdash{}{0pt}%
\pgfpathmoveto{\pgfqpoint{3.315904in}{1.351214in}}%
\pgfpathlineto{\pgfqpoint{3.365929in}{1.336314in}}%
\pgfusepath{stroke}%
\end{pgfscope}%
\begin{pgfscope}%
\definecolor{textcolor}{rgb}{0.000000,0.000000,0.000000}%
\pgfsetstrokecolor{textcolor}%
\pgfsetfillcolor{textcolor}%
\pgftext[x=3.500698in,y=1.171965in,,top]{\color{textcolor}\rmfamily\fontsize{10.000000}{12.000000}\selectfont \(\displaystyle {10}\)}%
\end{pgfscope}%
\begin{pgfscope}%
\pgfsetrectcap%
\pgfsetroundjoin%
\pgfsetlinewidth{0.803000pt}%
\definecolor{currentstroke}{rgb}{0.000000,0.000000,0.000000}%
\pgfsetstrokecolor{currentstroke}%
\pgfsetdash{}{0pt}%
\pgfpathmoveto{\pgfqpoint{3.484323in}{1.523744in}}%
\pgfpathlineto{\pgfqpoint{3.533847in}{1.509259in}}%
\pgfusepath{stroke}%
\end{pgfscope}%
\begin{pgfscope}%
\definecolor{textcolor}{rgb}{0.000000,0.000000,0.000000}%
\pgfsetstrokecolor{textcolor}%
\pgfsetfillcolor{textcolor}%
\pgftext[x=3.666681in,y=1.347177in,,top]{\color{textcolor}\rmfamily\fontsize{10.000000}{12.000000}\selectfont \(\displaystyle {15}\)}%
\end{pgfscope}%
\begin{pgfscope}%
\pgfsetrectcap%
\pgfsetroundjoin%
\pgfsetlinewidth{0.803000pt}%
\definecolor{currentstroke}{rgb}{0.000000,0.000000,0.000000}%
\pgfsetstrokecolor{currentstroke}%
\pgfsetdash{}{0pt}%
\pgfpathmoveto{\pgfqpoint{3.558144in}{1.577751in}}%
\pgfpathlineto{\pgfqpoint{3.628038in}{3.104037in}}%
\pgfusepath{stroke}%
\end{pgfscope}%
\begin{pgfscope}%
\definecolor{textcolor}{rgb}{0.000000,0.000000,0.000000}%
\pgfsetstrokecolor{textcolor}%
\pgfsetfillcolor{textcolor}%
\pgftext[x=4.167903in, y=1.963517in, left, base,rotate=87.378092]{\color{textcolor}\rmfamily\fontsize{10.000000}{12.000000}\selectfont Position Z [\(\displaystyle m\)]}%
\end{pgfscope}%
\begin{pgfscope}%
\pgfsetbuttcap%
\pgfsetroundjoin%
\pgfsetlinewidth{0.803000pt}%
\definecolor{currentstroke}{rgb}{0.690196,0.690196,0.690196}%
\pgfsetstrokecolor{currentstroke}%
\pgfsetdash{}{0pt}%
\pgfpathmoveto{\pgfqpoint{3.567143in}{1.774259in}}%
\pgfpathlineto{\pgfqpoint{1.597425in}{2.337333in}}%
\pgfpathlineto{\pgfqpoint{0.369383in}{1.325163in}}%
\pgfusepath{stroke}%
\end{pgfscope}%
\begin{pgfscope}%
\pgfsetbuttcap%
\pgfsetroundjoin%
\pgfsetlinewidth{0.803000pt}%
\definecolor{currentstroke}{rgb}{0.690196,0.690196,0.690196}%
\pgfsetstrokecolor{currentstroke}%
\pgfsetdash{}{0pt}%
\pgfpathmoveto{\pgfqpoint{3.576239in}{1.972888in}}%
\pgfpathlineto{\pgfqpoint{1.595214in}{2.529596in}}%
\pgfpathlineto{\pgfqpoint{0.359584in}{1.528650in}}%
\pgfusepath{stroke}%
\end{pgfscope}%
\begin{pgfscope}%
\pgfsetbuttcap%
\pgfsetroundjoin%
\pgfsetlinewidth{0.803000pt}%
\definecolor{currentstroke}{rgb}{0.690196,0.690196,0.690196}%
\pgfsetstrokecolor{currentstroke}%
\pgfsetdash{}{0pt}%
\pgfpathmoveto{\pgfqpoint{3.585441in}{2.173833in}}%
\pgfpathlineto{\pgfqpoint{1.592979in}{2.723994in}}%
\pgfpathlineto{\pgfqpoint{0.349666in}{1.734600in}}%
\pgfusepath{stroke}%
\end{pgfscope}%
\begin{pgfscope}%
\pgfsetbuttcap%
\pgfsetroundjoin%
\pgfsetlinewidth{0.803000pt}%
\definecolor{currentstroke}{rgb}{0.690196,0.690196,0.690196}%
\pgfsetstrokecolor{currentstroke}%
\pgfsetdash{}{0pt}%
\pgfpathmoveto{\pgfqpoint{3.594751in}{2.377135in}}%
\pgfpathlineto{\pgfqpoint{1.590718in}{2.920562in}}%
\pgfpathlineto{\pgfqpoint{0.339628in}{1.943059in}}%
\pgfusepath{stroke}%
\end{pgfscope}%
\begin{pgfscope}%
\pgfsetbuttcap%
\pgfsetroundjoin%
\pgfsetlinewidth{0.803000pt}%
\definecolor{currentstroke}{rgb}{0.690196,0.690196,0.690196}%
\pgfsetstrokecolor{currentstroke}%
\pgfsetdash{}{0pt}%
\pgfpathmoveto{\pgfqpoint{3.604170in}{2.582835in}}%
\pgfpathlineto{\pgfqpoint{1.588433in}{3.119337in}}%
\pgfpathlineto{\pgfqpoint{0.329466in}{2.154072in}}%
\pgfusepath{stroke}%
\end{pgfscope}%
\begin{pgfscope}%
\pgfsetbuttcap%
\pgfsetroundjoin%
\pgfsetlinewidth{0.803000pt}%
\definecolor{currentstroke}{rgb}{0.690196,0.690196,0.690196}%
\pgfsetstrokecolor{currentstroke}%
\pgfsetdash{}{0pt}%
\pgfpathmoveto{\pgfqpoint{3.613702in}{2.790977in}}%
\pgfpathlineto{\pgfqpoint{1.586121in}{3.320356in}}%
\pgfpathlineto{\pgfqpoint{0.319179in}{2.367688in}}%
\pgfusepath{stroke}%
\end{pgfscope}%
\begin{pgfscope}%
\pgfsetbuttcap%
\pgfsetroundjoin%
\pgfsetlinewidth{0.803000pt}%
\definecolor{currentstroke}{rgb}{0.690196,0.690196,0.690196}%
\pgfsetstrokecolor{currentstroke}%
\pgfsetdash{}{0pt}%
\pgfpathmoveto{\pgfqpoint{3.623347in}{3.001604in}}%
\pgfpathlineto{\pgfqpoint{1.583783in}{3.523657in}}%
\pgfpathlineto{\pgfqpoint{0.308764in}{2.583954in}}%
\pgfusepath{stroke}%
\end{pgfscope}%
\begin{pgfscope}%
\pgfsetrectcap%
\pgfsetroundjoin%
\pgfsetlinewidth{0.803000pt}%
\definecolor{currentstroke}{rgb}{0.000000,0.000000,0.000000}%
\pgfsetstrokecolor{currentstroke}%
\pgfsetdash{}{0pt}%
\pgfpathmoveto{\pgfqpoint{3.550607in}{1.778986in}}%
\pgfpathlineto{\pgfqpoint{3.600254in}{1.764793in}}%
\pgfusepath{stroke}%
\end{pgfscope}%
\begin{pgfscope}%
\definecolor{textcolor}{rgb}{0.000000,0.000000,0.000000}%
\pgfsetstrokecolor{textcolor}%
\pgfsetfillcolor{textcolor}%
\pgftext[x=3.822027in,y=1.810055in,,top]{\color{textcolor}\rmfamily\fontsize{10.000000}{12.000000}\selectfont \(\displaystyle {−1.2}\)}%
\end{pgfscope}%
\begin{pgfscope}%
\pgfsetrectcap%
\pgfsetroundjoin%
\pgfsetlinewidth{0.803000pt}%
\definecolor{currentstroke}{rgb}{0.000000,0.000000,0.000000}%
\pgfsetstrokecolor{currentstroke}%
\pgfsetdash{}{0pt}%
\pgfpathmoveto{\pgfqpoint{3.559604in}{1.977562in}}%
\pgfpathlineto{\pgfqpoint{3.609549in}{1.963527in}}%
\pgfusepath{stroke}%
\end{pgfscope}%
\begin{pgfscope}%
\definecolor{textcolor}{rgb}{0.000000,0.000000,0.000000}%
\pgfsetstrokecolor{textcolor}%
\pgfsetfillcolor{textcolor}%
\pgftext[x=3.832568in,y=2.008288in,,top]{\color{textcolor}\rmfamily\fontsize{10.000000}{12.000000}\selectfont \(\displaystyle {−1.0}\)}%
\end{pgfscope}%
\begin{pgfscope}%
\pgfsetrectcap%
\pgfsetroundjoin%
\pgfsetlinewidth{0.803000pt}%
\definecolor{currentstroke}{rgb}{0.000000,0.000000,0.000000}%
\pgfsetstrokecolor{currentstroke}%
\pgfsetdash{}{0pt}%
\pgfpathmoveto{\pgfqpoint{3.568705in}{2.178454in}}%
\pgfpathlineto{\pgfqpoint{3.618953in}{2.164579in}}%
\pgfusepath{stroke}%
\end{pgfscope}%
\begin{pgfscope}%
\definecolor{textcolor}{rgb}{0.000000,0.000000,0.000000}%
\pgfsetstrokecolor{textcolor}%
\pgfsetfillcolor{textcolor}%
\pgftext[x=3.843232in,y=2.208826in,,top]{\color{textcolor}\rmfamily\fontsize{10.000000}{12.000000}\selectfont \(\displaystyle {−0.8}\)}%
\end{pgfscope}%
\begin{pgfscope}%
\pgfsetrectcap%
\pgfsetroundjoin%
\pgfsetlinewidth{0.803000pt}%
\definecolor{currentstroke}{rgb}{0.000000,0.000000,0.000000}%
\pgfsetstrokecolor{currentstroke}%
\pgfsetdash{}{0pt}%
\pgfpathmoveto{\pgfqpoint{3.577913in}{2.381701in}}%
\pgfpathlineto{\pgfqpoint{3.628467in}{2.367992in}}%
\pgfusepath{stroke}%
\end{pgfscope}%
\begin{pgfscope}%
\definecolor{textcolor}{rgb}{0.000000,0.000000,0.000000}%
\pgfsetstrokecolor{textcolor}%
\pgfsetfillcolor{textcolor}%
\pgftext[x=3.854020in,y=2.411709in,,top]{\color{textcolor}\rmfamily\fontsize{10.000000}{12.000000}\selectfont \(\displaystyle {−0.6}\)}%
\end{pgfscope}%
\begin{pgfscope}%
\pgfsetrectcap%
\pgfsetroundjoin%
\pgfsetlinewidth{0.803000pt}%
\definecolor{currentstroke}{rgb}{0.000000,0.000000,0.000000}%
\pgfsetstrokecolor{currentstroke}%
\pgfsetdash{}{0pt}%
\pgfpathmoveto{\pgfqpoint{3.587229in}{2.587344in}}%
\pgfpathlineto{\pgfqpoint{3.638094in}{2.573807in}}%
\pgfusepath{stroke}%
\end{pgfscope}%
\begin{pgfscope}%
\definecolor{textcolor}{rgb}{0.000000,0.000000,0.000000}%
\pgfsetstrokecolor{textcolor}%
\pgfsetfillcolor{textcolor}%
\pgftext[x=3.864935in,y=2.616978in,,top]{\color{textcolor}\rmfamily\fontsize{10.000000}{12.000000}\selectfont \(\displaystyle {−0.4}\)}%
\end{pgfscope}%
\begin{pgfscope}%
\pgfsetrectcap%
\pgfsetroundjoin%
\pgfsetlinewidth{0.803000pt}%
\definecolor{currentstroke}{rgb}{0.000000,0.000000,0.000000}%
\pgfsetstrokecolor{currentstroke}%
\pgfsetdash{}{0pt}%
\pgfpathmoveto{\pgfqpoint{3.596656in}{2.795428in}}%
\pgfpathlineto{\pgfqpoint{3.647834in}{2.782066in}}%
\pgfusepath{stroke}%
\end{pgfscope}%
\begin{pgfscope}%
\definecolor{textcolor}{rgb}{0.000000,0.000000,0.000000}%
\pgfsetstrokecolor{textcolor}%
\pgfsetfillcolor{textcolor}%
\pgftext[x=3.875980in,y=2.824676in,,top]{\color{textcolor}\rmfamily\fontsize{10.000000}{12.000000}\selectfont \(\displaystyle {−0.2}\)}%
\end{pgfscope}%
\begin{pgfscope}%
\pgfsetrectcap%
\pgfsetroundjoin%
\pgfsetlinewidth{0.803000pt}%
\definecolor{currentstroke}{rgb}{0.000000,0.000000,0.000000}%
\pgfsetstrokecolor{currentstroke}%
\pgfsetdash{}{0pt}%
\pgfpathmoveto{\pgfqpoint{3.606196in}{3.005994in}}%
\pgfpathlineto{\pgfqpoint{3.657691in}{2.992813in}}%
\pgfusepath{stroke}%
\end{pgfscope}%
\begin{pgfscope}%
\definecolor{textcolor}{rgb}{0.000000,0.000000,0.000000}%
\pgfsetstrokecolor{textcolor}%
\pgfsetfillcolor{textcolor}%
\pgftext[x=3.887156in,y=3.034845in,,top]{\color{textcolor}\rmfamily\fontsize{10.000000}{12.000000}\selectfont \(\displaystyle {0.0}\)}%
\end{pgfscope}%
\begin{pgfscope}%
\pgfpathrectangle{\pgfqpoint{0.100000in}{0.212622in}}{\pgfqpoint{3.696000in}{3.696000in}}%
\pgfusepath{clip}%
\pgfsetrectcap%
\pgfsetroundjoin%
\pgfsetlinewidth{1.505625pt}%
\definecolor{currentstroke}{rgb}{0.121569,0.466667,0.705882}%
\pgfsetstrokecolor{currentstroke}%
\pgfsetdash{}{0pt}%
\pgfpathmoveto{\pgfqpoint{1.853752in}{2.776107in}}%
\pgfpathlineto{\pgfqpoint{1.691738in}{2.645310in}}%
\pgfpathlineto{\pgfqpoint{2.073851in}{2.539305in}}%
\pgfpathlineto{\pgfqpoint{2.539579in}{2.929871in}}%
\pgfpathlineto{\pgfqpoint{1.800211in}{3.125496in}}%
\pgfpathlineto{\pgfqpoint{0.974935in}{2.483209in}}%
\pgfpathlineto{\pgfqpoint{2.140093in}{2.149171in}}%
\pgfpathlineto{\pgfqpoint{3.064647in}{2.955787in}}%
\pgfpathlineto{\pgfqpoint{1.597027in}{3.338199in}}%
\pgfusepath{stroke}%
\end{pgfscope}%
\begin{pgfscope}%
\pgfpathrectangle{\pgfqpoint{0.100000in}{0.212622in}}{\pgfqpoint{3.696000in}{3.696000in}}%
\pgfusepath{clip}%
\pgfsetrectcap%
\pgfsetroundjoin%
\pgfsetlinewidth{1.505625pt}%
\definecolor{currentstroke}{rgb}{1.000000,0.000000,0.000000}%
\pgfsetstrokecolor{currentstroke}%
\pgfsetdash{}{0pt}%
\pgfpathmoveto{\pgfqpoint{1.853752in}{2.776107in}}%
\pgfpathlineto{\pgfqpoint{1.853752in}{2.776107in}}%
\pgfusepath{stroke}%
\end{pgfscope}%
\begin{pgfscope}%
\pgfpathrectangle{\pgfqpoint{0.100000in}{0.212622in}}{\pgfqpoint{3.696000in}{3.696000in}}%
\pgfusepath{clip}%
\pgfsetrectcap%
\pgfsetroundjoin%
\pgfsetlinewidth{1.505625pt}%
\definecolor{currentstroke}{rgb}{1.000000,0.000000,0.000000}%
\pgfsetstrokecolor{currentstroke}%
\pgfsetdash{}{0pt}%
\pgfpathmoveto{\pgfqpoint{1.734829in}{2.623552in}}%
\pgfpathlineto{\pgfqpoint{1.691738in}{2.645310in}}%
\pgfusepath{stroke}%
\end{pgfscope}%
\begin{pgfscope}%
\pgfpathrectangle{\pgfqpoint{0.100000in}{0.212622in}}{\pgfqpoint{3.696000in}{3.696000in}}%
\pgfusepath{clip}%
\pgfsetrectcap%
\pgfsetroundjoin%
\pgfsetlinewidth{1.505625pt}%
\definecolor{currentstroke}{rgb}{1.000000,0.000000,0.000000}%
\pgfsetstrokecolor{currentstroke}%
\pgfsetdash{}{0pt}%
\pgfpathmoveto{\pgfqpoint{1.737134in}{2.584070in}}%
\pgfpathlineto{\pgfqpoint{1.691738in}{2.645310in}}%
\pgfusepath{stroke}%
\end{pgfscope}%
\begin{pgfscope}%
\pgfpathrectangle{\pgfqpoint{0.100000in}{0.212622in}}{\pgfqpoint{3.696000in}{3.696000in}}%
\pgfusepath{clip}%
\pgfsetrectcap%
\pgfsetroundjoin%
\pgfsetlinewidth{1.505625pt}%
\definecolor{currentstroke}{rgb}{1.000000,0.000000,0.000000}%
\pgfsetstrokecolor{currentstroke}%
\pgfsetdash{}{0pt}%
\pgfpathmoveto{\pgfqpoint{2.073173in}{2.427883in}}%
\pgfpathlineto{\pgfqpoint{2.073851in}{2.539305in}}%
\pgfusepath{stroke}%
\end{pgfscope}%
\begin{pgfscope}%
\pgfpathrectangle{\pgfqpoint{0.100000in}{0.212622in}}{\pgfqpoint{3.696000in}{3.696000in}}%
\pgfusepath{clip}%
\pgfsetrectcap%
\pgfsetroundjoin%
\pgfsetlinewidth{1.505625pt}%
\definecolor{currentstroke}{rgb}{1.000000,0.000000,0.000000}%
\pgfsetstrokecolor{currentstroke}%
\pgfsetdash{}{0pt}%
\pgfpathmoveto{\pgfqpoint{2.602262in}{3.009986in}}%
\pgfpathlineto{\pgfqpoint{2.539579in}{2.929871in}}%
\pgfusepath{stroke}%
\end{pgfscope}%
\begin{pgfscope}%
\pgfpathrectangle{\pgfqpoint{0.100000in}{0.212622in}}{\pgfqpoint{3.696000in}{3.696000in}}%
\pgfusepath{clip}%
\pgfsetrectcap%
\pgfsetroundjoin%
\pgfsetlinewidth{1.505625pt}%
\definecolor{currentstroke}{rgb}{1.000000,0.000000,0.000000}%
\pgfsetstrokecolor{currentstroke}%
\pgfsetdash{}{0pt}%
\pgfpathmoveto{\pgfqpoint{1.720537in}{3.128019in}}%
\pgfpathlineto{\pgfqpoint{1.800211in}{3.125496in}}%
\pgfusepath{stroke}%
\end{pgfscope}%
\begin{pgfscope}%
\pgfpathrectangle{\pgfqpoint{0.100000in}{0.212622in}}{\pgfqpoint{3.696000in}{3.696000in}}%
\pgfusepath{clip}%
\pgfsetrectcap%
\pgfsetroundjoin%
\pgfsetlinewidth{1.505625pt}%
\definecolor{currentstroke}{rgb}{1.000000,0.000000,0.000000}%
\pgfsetstrokecolor{currentstroke}%
\pgfsetdash{}{0pt}%
\pgfpathmoveto{\pgfqpoint{1.600405in}{2.955903in}}%
\pgfpathlineto{\pgfqpoint{1.800211in}{3.125496in}}%
\pgfusepath{stroke}%
\end{pgfscope}%
\begin{pgfscope}%
\pgfpathrectangle{\pgfqpoint{0.100000in}{0.212622in}}{\pgfqpoint{3.696000in}{3.696000in}}%
\pgfusepath{clip}%
\pgfsetrectcap%
\pgfsetroundjoin%
\pgfsetlinewidth{1.505625pt}%
\definecolor{currentstroke}{rgb}{1.000000,0.000000,0.000000}%
\pgfsetstrokecolor{currentstroke}%
\pgfsetdash{}{0pt}%
\pgfpathmoveto{\pgfqpoint{0.940882in}{2.158024in}}%
\pgfpathlineto{\pgfqpoint{0.974935in}{2.483209in}}%
\pgfusepath{stroke}%
\end{pgfscope}%
\begin{pgfscope}%
\pgfpathrectangle{\pgfqpoint{0.100000in}{0.212622in}}{\pgfqpoint{3.696000in}{3.696000in}}%
\pgfusepath{clip}%
\pgfsetrectcap%
\pgfsetroundjoin%
\pgfsetlinewidth{1.505625pt}%
\definecolor{currentstroke}{rgb}{1.000000,0.000000,0.000000}%
\pgfsetstrokecolor{currentstroke}%
\pgfsetdash{}{0pt}%
\pgfpathmoveto{\pgfqpoint{0.945106in}{2.108376in}}%
\pgfpathlineto{\pgfqpoint{0.974935in}{2.483209in}}%
\pgfusepath{stroke}%
\end{pgfscope}%
\begin{pgfscope}%
\pgfpathrectangle{\pgfqpoint{0.100000in}{0.212622in}}{\pgfqpoint{3.696000in}{3.696000in}}%
\pgfusepath{clip}%
\pgfsetrectcap%
\pgfsetroundjoin%
\pgfsetlinewidth{1.505625pt}%
\definecolor{currentstroke}{rgb}{1.000000,0.000000,0.000000}%
\pgfsetstrokecolor{currentstroke}%
\pgfsetdash{}{0pt}%
\pgfpathmoveto{\pgfqpoint{1.432934in}{1.799862in}}%
\pgfpathlineto{\pgfqpoint{1.691738in}{2.645310in}}%
\pgfusepath{stroke}%
\end{pgfscope}%
\begin{pgfscope}%
\pgfpathrectangle{\pgfqpoint{0.100000in}{0.212622in}}{\pgfqpoint{3.696000in}{3.696000in}}%
\pgfusepath{clip}%
\pgfsetrectcap%
\pgfsetroundjoin%
\pgfsetlinewidth{1.505625pt}%
\definecolor{currentstroke}{rgb}{1.000000,0.000000,0.000000}%
\pgfsetstrokecolor{currentstroke}%
\pgfsetdash{}{0pt}%
\pgfpathmoveto{\pgfqpoint{2.491724in}{1.121199in}}%
\pgfpathlineto{\pgfqpoint{2.140093in}{2.149171in}}%
\pgfusepath{stroke}%
\end{pgfscope}%
\begin{pgfscope}%
\pgfpathrectangle{\pgfqpoint{0.100000in}{0.212622in}}{\pgfqpoint{3.696000in}{3.696000in}}%
\pgfusepath{clip}%
\pgfsetrectcap%
\pgfsetroundjoin%
\pgfsetlinewidth{1.505625pt}%
\definecolor{currentstroke}{rgb}{1.000000,0.000000,0.000000}%
\pgfsetstrokecolor{currentstroke}%
\pgfsetdash{}{0pt}%
\pgfpathmoveto{\pgfqpoint{3.305108in}{2.115337in}}%
\pgfpathlineto{\pgfqpoint{3.064647in}{2.955787in}}%
\pgfusepath{stroke}%
\end{pgfscope}%
\begin{pgfscope}%
\pgfpathrectangle{\pgfqpoint{0.100000in}{0.212622in}}{\pgfqpoint{3.696000in}{3.696000in}}%
\pgfusepath{clip}%
\pgfsetrectcap%
\pgfsetroundjoin%
\pgfsetlinewidth{1.505625pt}%
\definecolor{currentstroke}{rgb}{1.000000,0.000000,0.000000}%
\pgfsetstrokecolor{currentstroke}%
\pgfsetdash{}{0pt}%
\pgfpathmoveto{\pgfqpoint{1.614699in}{2.115957in}}%
\pgfpathlineto{\pgfqpoint{1.597027in}{3.338199in}}%
\pgfusepath{stroke}%
\end{pgfscope}%
\begin{pgfscope}%
\pgfpathrectangle{\pgfqpoint{0.100000in}{0.212622in}}{\pgfqpoint{3.696000in}{3.696000in}}%
\pgfusepath{clip}%
\pgfsetbuttcap%
\pgfsetroundjoin%
\definecolor{currentfill}{rgb}{1.000000,0.498039,0.054902}%
\pgfsetfillcolor{currentfill}%
\pgfsetfillopacity{0.300000}%
\pgfsetlinewidth{1.003750pt}%
\definecolor{currentstroke}{rgb}{1.000000,0.498039,0.054902}%
\pgfsetstrokecolor{currentstroke}%
\pgfsetstrokeopacity{0.300000}%
\pgfsetdash{}{0pt}%
\pgfpathmoveto{\pgfqpoint{1.614699in}{2.084900in}}%
\pgfpathcurveto{\pgfqpoint{1.622935in}{2.084900in}}{\pgfqpoint{1.630835in}{2.088173in}}{\pgfqpoint{1.636659in}{2.093997in}}%
\pgfpathcurveto{\pgfqpoint{1.642483in}{2.099821in}}{\pgfqpoint{1.645755in}{2.107721in}}{\pgfqpoint{1.645755in}{2.115957in}}%
\pgfpathcurveto{\pgfqpoint{1.645755in}{2.124193in}}{\pgfqpoint{1.642483in}{2.132093in}}{\pgfqpoint{1.636659in}{2.137917in}}%
\pgfpathcurveto{\pgfqpoint{1.630835in}{2.143741in}}{\pgfqpoint{1.622935in}{2.147013in}}{\pgfqpoint{1.614699in}{2.147013in}}%
\pgfpathcurveto{\pgfqpoint{1.606463in}{2.147013in}}{\pgfqpoint{1.598563in}{2.143741in}}{\pgfqpoint{1.592739in}{2.137917in}}%
\pgfpathcurveto{\pgfqpoint{1.586915in}{2.132093in}}{\pgfqpoint{1.583642in}{2.124193in}}{\pgfqpoint{1.583642in}{2.115957in}}%
\pgfpathcurveto{\pgfqpoint{1.583642in}{2.107721in}}{\pgfqpoint{1.586915in}{2.099821in}}{\pgfqpoint{1.592739in}{2.093997in}}%
\pgfpathcurveto{\pgfqpoint{1.598563in}{2.088173in}}{\pgfqpoint{1.606463in}{2.084900in}}{\pgfqpoint{1.614699in}{2.084900in}}%
\pgfpathclose%
\pgfusepath{stroke,fill}%
\end{pgfscope}%
\begin{pgfscope}%
\pgfpathrectangle{\pgfqpoint{0.100000in}{0.212622in}}{\pgfqpoint{3.696000in}{3.696000in}}%
\pgfusepath{clip}%
\pgfsetbuttcap%
\pgfsetroundjoin%
\definecolor{currentfill}{rgb}{1.000000,0.498039,0.054902}%
\pgfsetfillcolor{currentfill}%
\pgfsetfillopacity{0.585044}%
\pgfsetlinewidth{1.003750pt}%
\definecolor{currentstroke}{rgb}{1.000000,0.498039,0.054902}%
\pgfsetstrokecolor{currentstroke}%
\pgfsetstrokeopacity{0.585044}%
\pgfsetdash{}{0pt}%
\pgfpathmoveto{\pgfqpoint{3.305108in}{2.084281in}}%
\pgfpathcurveto{\pgfqpoint{3.313345in}{2.084281in}}{\pgfqpoint{3.321245in}{2.087553in}}{\pgfqpoint{3.327069in}{2.093377in}}%
\pgfpathcurveto{\pgfqpoint{3.332893in}{2.099201in}}{\pgfqpoint{3.336165in}{2.107101in}}{\pgfqpoint{3.336165in}{2.115337in}}%
\pgfpathcurveto{\pgfqpoint{3.336165in}{2.123573in}}{\pgfqpoint{3.332893in}{2.131473in}}{\pgfqpoint{3.327069in}{2.137297in}}%
\pgfpathcurveto{\pgfqpoint{3.321245in}{2.143121in}}{\pgfqpoint{3.313345in}{2.146394in}}{\pgfqpoint{3.305108in}{2.146394in}}%
\pgfpathcurveto{\pgfqpoint{3.296872in}{2.146394in}}{\pgfqpoint{3.288972in}{2.143121in}}{\pgfqpoint{3.283148in}{2.137297in}}%
\pgfpathcurveto{\pgfqpoint{3.277324in}{2.131473in}}{\pgfqpoint{3.274052in}{2.123573in}}{\pgfqpoint{3.274052in}{2.115337in}}%
\pgfpathcurveto{\pgfqpoint{3.274052in}{2.107101in}}{\pgfqpoint{3.277324in}{2.099201in}}{\pgfqpoint{3.283148in}{2.093377in}}%
\pgfpathcurveto{\pgfqpoint{3.288972in}{2.087553in}}{\pgfqpoint{3.296872in}{2.084281in}}{\pgfqpoint{3.305108in}{2.084281in}}%
\pgfpathclose%
\pgfusepath{stroke,fill}%
\end{pgfscope}%
\begin{pgfscope}%
\pgfpathrectangle{\pgfqpoint{0.100000in}{0.212622in}}{\pgfqpoint{3.696000in}{3.696000in}}%
\pgfusepath{clip}%
\pgfsetbuttcap%
\pgfsetroundjoin%
\definecolor{currentfill}{rgb}{1.000000,0.498039,0.054902}%
\pgfsetfillcolor{currentfill}%
\pgfsetfillopacity{0.604907}%
\pgfsetlinewidth{1.003750pt}%
\definecolor{currentstroke}{rgb}{1.000000,0.498039,0.054902}%
\pgfsetstrokecolor{currentstroke}%
\pgfsetstrokeopacity{0.604907}%
\pgfsetdash{}{0pt}%
\pgfpathmoveto{\pgfqpoint{1.720537in}{3.096963in}}%
\pgfpathcurveto{\pgfqpoint{1.728774in}{3.096963in}}{\pgfqpoint{1.736674in}{3.100235in}}{\pgfqpoint{1.742498in}{3.106059in}}%
\pgfpathcurveto{\pgfqpoint{1.748322in}{3.111883in}}{\pgfqpoint{1.751594in}{3.119783in}}{\pgfqpoint{1.751594in}{3.128019in}}%
\pgfpathcurveto{\pgfqpoint{1.751594in}{3.136255in}}{\pgfqpoint{1.748322in}{3.144155in}}{\pgfqpoint{1.742498in}{3.149979in}}%
\pgfpathcurveto{\pgfqpoint{1.736674in}{3.155803in}}{\pgfqpoint{1.728774in}{3.159076in}}{\pgfqpoint{1.720537in}{3.159076in}}%
\pgfpathcurveto{\pgfqpoint{1.712301in}{3.159076in}}{\pgfqpoint{1.704401in}{3.155803in}}{\pgfqpoint{1.698577in}{3.149979in}}%
\pgfpathcurveto{\pgfqpoint{1.692753in}{3.144155in}}{\pgfqpoint{1.689481in}{3.136255in}}{\pgfqpoint{1.689481in}{3.128019in}}%
\pgfpathcurveto{\pgfqpoint{1.689481in}{3.119783in}}{\pgfqpoint{1.692753in}{3.111883in}}{\pgfqpoint{1.698577in}{3.106059in}}%
\pgfpathcurveto{\pgfqpoint{1.704401in}{3.100235in}}{\pgfqpoint{1.712301in}{3.096963in}}{\pgfqpoint{1.720537in}{3.096963in}}%
\pgfpathclose%
\pgfusepath{stroke,fill}%
\end{pgfscope}%
\begin{pgfscope}%
\pgfpathrectangle{\pgfqpoint{0.100000in}{0.212622in}}{\pgfqpoint{3.696000in}{3.696000in}}%
\pgfusepath{clip}%
\pgfsetbuttcap%
\pgfsetroundjoin%
\definecolor{currentfill}{rgb}{1.000000,0.498039,0.054902}%
\pgfsetfillcolor{currentfill}%
\pgfsetfillopacity{0.665234}%
\pgfsetlinewidth{1.003750pt}%
\definecolor{currentstroke}{rgb}{1.000000,0.498039,0.054902}%
\pgfsetstrokecolor{currentstroke}%
\pgfsetstrokeopacity{0.665234}%
\pgfsetdash{}{0pt}%
\pgfpathmoveto{\pgfqpoint{1.600405in}{2.924846in}}%
\pgfpathcurveto{\pgfqpoint{1.608641in}{2.924846in}}{\pgfqpoint{1.616541in}{2.928119in}}{\pgfqpoint{1.622365in}{2.933943in}}%
\pgfpathcurveto{\pgfqpoint{1.628189in}{2.939766in}}{\pgfqpoint{1.631462in}{2.947667in}}{\pgfqpoint{1.631462in}{2.955903in}}%
\pgfpathcurveto{\pgfqpoint{1.631462in}{2.964139in}}{\pgfqpoint{1.628189in}{2.972039in}}{\pgfqpoint{1.622365in}{2.977863in}}%
\pgfpathcurveto{\pgfqpoint{1.616541in}{2.983687in}}{\pgfqpoint{1.608641in}{2.986959in}}{\pgfqpoint{1.600405in}{2.986959in}}%
\pgfpathcurveto{\pgfqpoint{1.592169in}{2.986959in}}{\pgfqpoint{1.584269in}{2.983687in}}{\pgfqpoint{1.578445in}{2.977863in}}%
\pgfpathcurveto{\pgfqpoint{1.572621in}{2.972039in}}{\pgfqpoint{1.569349in}{2.964139in}}{\pgfqpoint{1.569349in}{2.955903in}}%
\pgfpathcurveto{\pgfqpoint{1.569349in}{2.947667in}}{\pgfqpoint{1.572621in}{2.939766in}}{\pgfqpoint{1.578445in}{2.933943in}}%
\pgfpathcurveto{\pgfqpoint{1.584269in}{2.928119in}}{\pgfqpoint{1.592169in}{2.924846in}}{\pgfqpoint{1.600405in}{2.924846in}}%
\pgfpathclose%
\pgfusepath{stroke,fill}%
\end{pgfscope}%
\begin{pgfscope}%
\pgfpathrectangle{\pgfqpoint{0.100000in}{0.212622in}}{\pgfqpoint{3.696000in}{3.696000in}}%
\pgfusepath{clip}%
\pgfsetbuttcap%
\pgfsetroundjoin%
\definecolor{currentfill}{rgb}{1.000000,0.498039,0.054902}%
\pgfsetfillcolor{currentfill}%
\pgfsetfillopacity{0.722442}%
\pgfsetlinewidth{1.003750pt}%
\definecolor{currentstroke}{rgb}{1.000000,0.498039,0.054902}%
\pgfsetstrokecolor{currentstroke}%
\pgfsetstrokeopacity{0.722442}%
\pgfsetdash{}{0pt}%
\pgfpathmoveto{\pgfqpoint{2.602262in}{2.978930in}}%
\pgfpathcurveto{\pgfqpoint{2.610498in}{2.978930in}}{\pgfqpoint{2.618398in}{2.982202in}}{\pgfqpoint{2.624222in}{2.988026in}}%
\pgfpathcurveto{\pgfqpoint{2.630046in}{2.993850in}}{\pgfqpoint{2.633318in}{3.001750in}}{\pgfqpoint{2.633318in}{3.009986in}}%
\pgfpathcurveto{\pgfqpoint{2.633318in}{3.018222in}}{\pgfqpoint{2.630046in}{3.026122in}}{\pgfqpoint{2.624222in}{3.031946in}}%
\pgfpathcurveto{\pgfqpoint{2.618398in}{3.037770in}}{\pgfqpoint{2.610498in}{3.041043in}}{\pgfqpoint{2.602262in}{3.041043in}}%
\pgfpathcurveto{\pgfqpoint{2.594026in}{3.041043in}}{\pgfqpoint{2.586126in}{3.037770in}}{\pgfqpoint{2.580302in}{3.031946in}}%
\pgfpathcurveto{\pgfqpoint{2.574478in}{3.026122in}}{\pgfqpoint{2.571205in}{3.018222in}}{\pgfqpoint{2.571205in}{3.009986in}}%
\pgfpathcurveto{\pgfqpoint{2.571205in}{3.001750in}}{\pgfqpoint{2.574478in}{2.993850in}}{\pgfqpoint{2.580302in}{2.988026in}}%
\pgfpathcurveto{\pgfqpoint{2.586126in}{2.982202in}}{\pgfqpoint{2.594026in}{2.978930in}}{\pgfqpoint{2.602262in}{2.978930in}}%
\pgfpathclose%
\pgfusepath{stroke,fill}%
\end{pgfscope}%
\begin{pgfscope}%
\pgfpathrectangle{\pgfqpoint{0.100000in}{0.212622in}}{\pgfqpoint{3.696000in}{3.696000in}}%
\pgfusepath{clip}%
\pgfsetbuttcap%
\pgfsetroundjoin%
\definecolor{currentfill}{rgb}{1.000000,0.498039,0.054902}%
\pgfsetfillcolor{currentfill}%
\pgfsetfillopacity{0.850249}%
\pgfsetlinewidth{1.003750pt}%
\definecolor{currentstroke}{rgb}{1.000000,0.498039,0.054902}%
\pgfsetstrokecolor{currentstroke}%
\pgfsetstrokeopacity{0.850249}%
\pgfsetdash{}{0pt}%
\pgfpathmoveto{\pgfqpoint{1.853752in}{2.745050in}}%
\pgfpathcurveto{\pgfqpoint{1.861988in}{2.745050in}}{\pgfqpoint{1.869888in}{2.748323in}}{\pgfqpoint{1.875712in}{2.754147in}}%
\pgfpathcurveto{\pgfqpoint{1.881536in}{2.759970in}}{\pgfqpoint{1.884808in}{2.767870in}}{\pgfqpoint{1.884808in}{2.776107in}}%
\pgfpathcurveto{\pgfqpoint{1.884808in}{2.784343in}}{\pgfqpoint{1.881536in}{2.792243in}}{\pgfqpoint{1.875712in}{2.798067in}}%
\pgfpathcurveto{\pgfqpoint{1.869888in}{2.803891in}}{\pgfqpoint{1.861988in}{2.807163in}}{\pgfqpoint{1.853752in}{2.807163in}}%
\pgfpathcurveto{\pgfqpoint{1.845516in}{2.807163in}}{\pgfqpoint{1.837616in}{2.803891in}}{\pgfqpoint{1.831792in}{2.798067in}}%
\pgfpathcurveto{\pgfqpoint{1.825968in}{2.792243in}}{\pgfqpoint{1.822695in}{2.784343in}}{\pgfqpoint{1.822695in}{2.776107in}}%
\pgfpathcurveto{\pgfqpoint{1.822695in}{2.767870in}}{\pgfqpoint{1.825968in}{2.759970in}}{\pgfqpoint{1.831792in}{2.754147in}}%
\pgfpathcurveto{\pgfqpoint{1.837616in}{2.748323in}}{\pgfqpoint{1.845516in}{2.745050in}}{\pgfqpoint{1.853752in}{2.745050in}}%
\pgfpathclose%
\pgfusepath{stroke,fill}%
\end{pgfscope}%
\begin{pgfscope}%
\pgfpathrectangle{\pgfqpoint{0.100000in}{0.212622in}}{\pgfqpoint{3.696000in}{3.696000in}}%
\pgfusepath{clip}%
\pgfsetbuttcap%
\pgfsetroundjoin%
\definecolor{currentfill}{rgb}{1.000000,0.498039,0.054902}%
\pgfsetfillcolor{currentfill}%
\pgfsetfillopacity{0.876672}%
\pgfsetlinewidth{1.003750pt}%
\definecolor{currentstroke}{rgb}{1.000000,0.498039,0.054902}%
\pgfsetstrokecolor{currentstroke}%
\pgfsetstrokeopacity{0.876672}%
\pgfsetdash{}{0pt}%
\pgfpathmoveto{\pgfqpoint{0.940882in}{2.126968in}}%
\pgfpathcurveto{\pgfqpoint{0.949118in}{2.126968in}}{\pgfqpoint{0.957018in}{2.130240in}}{\pgfqpoint{0.962842in}{2.136064in}}%
\pgfpathcurveto{\pgfqpoint{0.968666in}{2.141888in}}{\pgfqpoint{0.971938in}{2.149788in}}{\pgfqpoint{0.971938in}{2.158024in}}%
\pgfpathcurveto{\pgfqpoint{0.971938in}{2.166260in}}{\pgfqpoint{0.968666in}{2.174160in}}{\pgfqpoint{0.962842in}{2.179984in}}%
\pgfpathcurveto{\pgfqpoint{0.957018in}{2.185808in}}{\pgfqpoint{0.949118in}{2.189081in}}{\pgfqpoint{0.940882in}{2.189081in}}%
\pgfpathcurveto{\pgfqpoint{0.932645in}{2.189081in}}{\pgfqpoint{0.924745in}{2.185808in}}{\pgfqpoint{0.918921in}{2.179984in}}%
\pgfpathcurveto{\pgfqpoint{0.913098in}{2.174160in}}{\pgfqpoint{0.909825in}{2.166260in}}{\pgfqpoint{0.909825in}{2.158024in}}%
\pgfpathcurveto{\pgfqpoint{0.909825in}{2.149788in}}{\pgfqpoint{0.913098in}{2.141888in}}{\pgfqpoint{0.918921in}{2.136064in}}%
\pgfpathcurveto{\pgfqpoint{0.924745in}{2.130240in}}{\pgfqpoint{0.932645in}{2.126968in}}{\pgfqpoint{0.940882in}{2.126968in}}%
\pgfpathclose%
\pgfusepath{stroke,fill}%
\end{pgfscope}%
\begin{pgfscope}%
\pgfpathrectangle{\pgfqpoint{0.100000in}{0.212622in}}{\pgfqpoint{3.696000in}{3.696000in}}%
\pgfusepath{clip}%
\pgfsetbuttcap%
\pgfsetroundjoin%
\definecolor{currentfill}{rgb}{1.000000,0.498039,0.054902}%
\pgfsetfillcolor{currentfill}%
\pgfsetfillopacity{0.900627}%
\pgfsetlinewidth{1.003750pt}%
\definecolor{currentstroke}{rgb}{1.000000,0.498039,0.054902}%
\pgfsetstrokecolor{currentstroke}%
\pgfsetstrokeopacity{0.900627}%
\pgfsetdash{}{0pt}%
\pgfpathmoveto{\pgfqpoint{0.945106in}{2.077319in}}%
\pgfpathcurveto{\pgfqpoint{0.953342in}{2.077319in}}{\pgfqpoint{0.961242in}{2.080592in}}{\pgfqpoint{0.967066in}{2.086416in}}%
\pgfpathcurveto{\pgfqpoint{0.972890in}{2.092240in}}{\pgfqpoint{0.976162in}{2.100140in}}{\pgfqpoint{0.976162in}{2.108376in}}%
\pgfpathcurveto{\pgfqpoint{0.976162in}{2.116612in}}{\pgfqpoint{0.972890in}{2.124512in}}{\pgfqpoint{0.967066in}{2.130336in}}%
\pgfpathcurveto{\pgfqpoint{0.961242in}{2.136160in}}{\pgfqpoint{0.953342in}{2.139432in}}{\pgfqpoint{0.945106in}{2.139432in}}%
\pgfpathcurveto{\pgfqpoint{0.936869in}{2.139432in}}{\pgfqpoint{0.928969in}{2.136160in}}{\pgfqpoint{0.923145in}{2.130336in}}%
\pgfpathcurveto{\pgfqpoint{0.917322in}{2.124512in}}{\pgfqpoint{0.914049in}{2.116612in}}{\pgfqpoint{0.914049in}{2.108376in}}%
\pgfpathcurveto{\pgfqpoint{0.914049in}{2.100140in}}{\pgfqpoint{0.917322in}{2.092240in}}{\pgfqpoint{0.923145in}{2.086416in}}%
\pgfpathcurveto{\pgfqpoint{0.928969in}{2.080592in}}{\pgfqpoint{0.936869in}{2.077319in}}{\pgfqpoint{0.945106in}{2.077319in}}%
\pgfpathclose%
\pgfusepath{stroke,fill}%
\end{pgfscope}%
\begin{pgfscope}%
\pgfpathrectangle{\pgfqpoint{0.100000in}{0.212622in}}{\pgfqpoint{3.696000in}{3.696000in}}%
\pgfusepath{clip}%
\pgfsetbuttcap%
\pgfsetroundjoin%
\definecolor{currentfill}{rgb}{1.000000,0.498039,0.054902}%
\pgfsetfillcolor{currentfill}%
\pgfsetfillopacity{0.909460}%
\pgfsetlinewidth{1.003750pt}%
\definecolor{currentstroke}{rgb}{1.000000,0.498039,0.054902}%
\pgfsetstrokecolor{currentstroke}%
\pgfsetstrokeopacity{0.909460}%
\pgfsetdash{}{0pt}%
\pgfpathmoveto{\pgfqpoint{1.734829in}{2.592495in}}%
\pgfpathcurveto{\pgfqpoint{1.743065in}{2.592495in}}{\pgfqpoint{1.750965in}{2.595767in}}{\pgfqpoint{1.756789in}{2.601591in}}%
\pgfpathcurveto{\pgfqpoint{1.762613in}{2.607415in}}{\pgfqpoint{1.765886in}{2.615315in}}{\pgfqpoint{1.765886in}{2.623552in}}%
\pgfpathcurveto{\pgfqpoint{1.765886in}{2.631788in}}{\pgfqpoint{1.762613in}{2.639688in}}{\pgfqpoint{1.756789in}{2.645512in}}%
\pgfpathcurveto{\pgfqpoint{1.750965in}{2.651336in}}{\pgfqpoint{1.743065in}{2.654608in}}{\pgfqpoint{1.734829in}{2.654608in}}%
\pgfpathcurveto{\pgfqpoint{1.726593in}{2.654608in}}{\pgfqpoint{1.718693in}{2.651336in}}{\pgfqpoint{1.712869in}{2.645512in}}%
\pgfpathcurveto{\pgfqpoint{1.707045in}{2.639688in}}{\pgfqpoint{1.703773in}{2.631788in}}{\pgfqpoint{1.703773in}{2.623552in}}%
\pgfpathcurveto{\pgfqpoint{1.703773in}{2.615315in}}{\pgfqpoint{1.707045in}{2.607415in}}{\pgfqpoint{1.712869in}{2.601591in}}%
\pgfpathcurveto{\pgfqpoint{1.718693in}{2.595767in}}{\pgfqpoint{1.726593in}{2.592495in}}{\pgfqpoint{1.734829in}{2.592495in}}%
\pgfpathclose%
\pgfusepath{stroke,fill}%
\end{pgfscope}%
\begin{pgfscope}%
\pgfpathrectangle{\pgfqpoint{0.100000in}{0.212622in}}{\pgfqpoint{3.696000in}{3.696000in}}%
\pgfusepath{clip}%
\pgfsetbuttcap%
\pgfsetroundjoin%
\definecolor{currentfill}{rgb}{1.000000,0.498039,0.054902}%
\pgfsetfillcolor{currentfill}%
\pgfsetfillopacity{0.925759}%
\pgfsetlinewidth{1.003750pt}%
\definecolor{currentstroke}{rgb}{1.000000,0.498039,0.054902}%
\pgfsetstrokecolor{currentstroke}%
\pgfsetstrokeopacity{0.925759}%
\pgfsetdash{}{0pt}%
\pgfpathmoveto{\pgfqpoint{1.737134in}{2.553013in}}%
\pgfpathcurveto{\pgfqpoint{1.745370in}{2.553013in}}{\pgfqpoint{1.753270in}{2.556286in}}{\pgfqpoint{1.759094in}{2.562110in}}%
\pgfpathcurveto{\pgfqpoint{1.764918in}{2.567934in}}{\pgfqpoint{1.768190in}{2.575834in}}{\pgfqpoint{1.768190in}{2.584070in}}%
\pgfpathcurveto{\pgfqpoint{1.768190in}{2.592306in}}{\pgfqpoint{1.764918in}{2.600206in}}{\pgfqpoint{1.759094in}{2.606030in}}%
\pgfpathcurveto{\pgfqpoint{1.753270in}{2.611854in}}{\pgfqpoint{1.745370in}{2.615126in}}{\pgfqpoint{1.737134in}{2.615126in}}%
\pgfpathcurveto{\pgfqpoint{1.728897in}{2.615126in}}{\pgfqpoint{1.720997in}{2.611854in}}{\pgfqpoint{1.715173in}{2.606030in}}%
\pgfpathcurveto{\pgfqpoint{1.709349in}{2.600206in}}{\pgfqpoint{1.706077in}{2.592306in}}{\pgfqpoint{1.706077in}{2.584070in}}%
\pgfpathcurveto{\pgfqpoint{1.706077in}{2.575834in}}{\pgfqpoint{1.709349in}{2.567934in}}{\pgfqpoint{1.715173in}{2.562110in}}%
\pgfpathcurveto{\pgfqpoint{1.720997in}{2.556286in}}{\pgfqpoint{1.728897in}{2.553013in}}{\pgfqpoint{1.737134in}{2.553013in}}%
\pgfpathclose%
\pgfusepath{stroke,fill}%
\end{pgfscope}%
\begin{pgfscope}%
\pgfpathrectangle{\pgfqpoint{0.100000in}{0.212622in}}{\pgfqpoint{3.696000in}{3.696000in}}%
\pgfusepath{clip}%
\pgfsetbuttcap%
\pgfsetroundjoin%
\definecolor{currentfill}{rgb}{1.000000,0.498039,0.054902}%
\pgfsetfillcolor{currentfill}%
\pgfsetfillopacity{0.956813}%
\pgfsetlinewidth{1.003750pt}%
\definecolor{currentstroke}{rgb}{1.000000,0.498039,0.054902}%
\pgfsetstrokecolor{currentstroke}%
\pgfsetstrokeopacity{0.956813}%
\pgfsetdash{}{0pt}%
\pgfpathmoveto{\pgfqpoint{1.432934in}{1.768805in}}%
\pgfpathcurveto{\pgfqpoint{1.441171in}{1.768805in}}{\pgfqpoint{1.449071in}{1.772077in}}{\pgfqpoint{1.454895in}{1.777901in}}%
\pgfpathcurveto{\pgfqpoint{1.460719in}{1.783725in}}{\pgfqpoint{1.463991in}{1.791625in}}{\pgfqpoint{1.463991in}{1.799862in}}%
\pgfpathcurveto{\pgfqpoint{1.463991in}{1.808098in}}{\pgfqpoint{1.460719in}{1.815998in}}{\pgfqpoint{1.454895in}{1.821822in}}%
\pgfpathcurveto{\pgfqpoint{1.449071in}{1.827646in}}{\pgfqpoint{1.441171in}{1.830918in}}{\pgfqpoint{1.432934in}{1.830918in}}%
\pgfpathcurveto{\pgfqpoint{1.424698in}{1.830918in}}{\pgfqpoint{1.416798in}{1.827646in}}{\pgfqpoint{1.410974in}{1.821822in}}%
\pgfpathcurveto{\pgfqpoint{1.405150in}{1.815998in}}{\pgfqpoint{1.401878in}{1.808098in}}{\pgfqpoint{1.401878in}{1.799862in}}%
\pgfpathcurveto{\pgfqpoint{1.401878in}{1.791625in}}{\pgfqpoint{1.405150in}{1.783725in}}{\pgfqpoint{1.410974in}{1.777901in}}%
\pgfpathcurveto{\pgfqpoint{1.416798in}{1.772077in}}{\pgfqpoint{1.424698in}{1.768805in}}{\pgfqpoint{1.432934in}{1.768805in}}%
\pgfpathclose%
\pgfusepath{stroke,fill}%
\end{pgfscope}%
\begin{pgfscope}%
\pgfpathrectangle{\pgfqpoint{0.100000in}{0.212622in}}{\pgfqpoint{3.696000in}{3.696000in}}%
\pgfusepath{clip}%
\pgfsetbuttcap%
\pgfsetroundjoin%
\definecolor{currentfill}{rgb}{1.000000,0.498039,0.054902}%
\pgfsetfillcolor{currentfill}%
\pgfsetfillopacity{0.957407}%
\pgfsetlinewidth{1.003750pt}%
\definecolor{currentstroke}{rgb}{1.000000,0.498039,0.054902}%
\pgfsetstrokecolor{currentstroke}%
\pgfsetstrokeopacity{0.957407}%
\pgfsetdash{}{0pt}%
\pgfpathmoveto{\pgfqpoint{2.073173in}{2.396826in}}%
\pgfpathcurveto{\pgfqpoint{2.081410in}{2.396826in}}{\pgfqpoint{2.089310in}{2.400099in}}{\pgfqpoint{2.095134in}{2.405923in}}%
\pgfpathcurveto{\pgfqpoint{2.100958in}{2.411747in}}{\pgfqpoint{2.104230in}{2.419647in}}{\pgfqpoint{2.104230in}{2.427883in}}%
\pgfpathcurveto{\pgfqpoint{2.104230in}{2.436119in}}{\pgfqpoint{2.100958in}{2.444019in}}{\pgfqpoint{2.095134in}{2.449843in}}%
\pgfpathcurveto{\pgfqpoint{2.089310in}{2.455667in}}{\pgfqpoint{2.081410in}{2.458939in}}{\pgfqpoint{2.073173in}{2.458939in}}%
\pgfpathcurveto{\pgfqpoint{2.064937in}{2.458939in}}{\pgfqpoint{2.057037in}{2.455667in}}{\pgfqpoint{2.051213in}{2.449843in}}%
\pgfpathcurveto{\pgfqpoint{2.045389in}{2.444019in}}{\pgfqpoint{2.042117in}{2.436119in}}{\pgfqpoint{2.042117in}{2.427883in}}%
\pgfpathcurveto{\pgfqpoint{2.042117in}{2.419647in}}{\pgfqpoint{2.045389in}{2.411747in}}{\pgfqpoint{2.051213in}{2.405923in}}%
\pgfpathcurveto{\pgfqpoint{2.057037in}{2.400099in}}{\pgfqpoint{2.064937in}{2.396826in}}{\pgfqpoint{2.073173in}{2.396826in}}%
\pgfpathclose%
\pgfusepath{stroke,fill}%
\end{pgfscope}%
\begin{pgfscope}%
\pgfpathrectangle{\pgfqpoint{0.100000in}{0.212622in}}{\pgfqpoint{3.696000in}{3.696000in}}%
\pgfusepath{clip}%
\pgfsetbuttcap%
\pgfsetroundjoin%
\definecolor{currentfill}{rgb}{1.000000,0.498039,0.054902}%
\pgfsetfillcolor{currentfill}%
\pgfsetlinewidth{1.003750pt}%
\definecolor{currentstroke}{rgb}{1.000000,0.498039,0.054902}%
\pgfsetstrokecolor{currentstroke}%
\pgfsetdash{}{0pt}%
\pgfpathmoveto{\pgfqpoint{2.491724in}{1.090142in}}%
\pgfpathcurveto{\pgfqpoint{2.499961in}{1.090142in}}{\pgfqpoint{2.507861in}{1.093415in}}{\pgfqpoint{2.513685in}{1.099238in}}%
\pgfpathcurveto{\pgfqpoint{2.519509in}{1.105062in}}{\pgfqpoint{2.522781in}{1.112962in}}{\pgfqpoint{2.522781in}{1.121199in}}%
\pgfpathcurveto{\pgfqpoint{2.522781in}{1.129435in}}{\pgfqpoint{2.519509in}{1.137335in}}{\pgfqpoint{2.513685in}{1.143159in}}%
\pgfpathcurveto{\pgfqpoint{2.507861in}{1.148983in}}{\pgfqpoint{2.499961in}{1.152255in}}{\pgfqpoint{2.491724in}{1.152255in}}%
\pgfpathcurveto{\pgfqpoint{2.483488in}{1.152255in}}{\pgfqpoint{2.475588in}{1.148983in}}{\pgfqpoint{2.469764in}{1.143159in}}%
\pgfpathcurveto{\pgfqpoint{2.463940in}{1.137335in}}{\pgfqpoint{2.460668in}{1.129435in}}{\pgfqpoint{2.460668in}{1.121199in}}%
\pgfpathcurveto{\pgfqpoint{2.460668in}{1.112962in}}{\pgfqpoint{2.463940in}{1.105062in}}{\pgfqpoint{2.469764in}{1.099238in}}%
\pgfpathcurveto{\pgfqpoint{2.475588in}{1.093415in}}{\pgfqpoint{2.483488in}{1.090142in}}{\pgfqpoint{2.491724in}{1.090142in}}%
\pgfpathclose%
\pgfusepath{stroke,fill}%
\end{pgfscope}%
\begin{pgfscope}%
\definecolor{textcolor}{rgb}{0.000000,0.000000,0.000000}%
\pgfsetstrokecolor{textcolor}%
\pgfsetfillcolor{textcolor}%
\pgftext[x=1.948000in,y=3.991956in,,base]{\color{textcolor}\rmfamily\fontsize{12.000000}{14.400000}\selectfont Madgwick}%
\end{pgfscope}%
\begin{pgfscope}%
\pgfpathrectangle{\pgfqpoint{0.100000in}{0.212622in}}{\pgfqpoint{3.696000in}{3.696000in}}%
\pgfusepath{clip}%
\pgfsetbuttcap%
\pgfsetroundjoin%
\definecolor{currentfill}{rgb}{0.121569,0.466667,0.705882}%
\pgfsetfillcolor{currentfill}%
\pgfsetfillopacity{0.300000}%
\pgfsetlinewidth{1.003750pt}%
\definecolor{currentstroke}{rgb}{0.121569,0.466667,0.705882}%
\pgfsetstrokecolor{currentstroke}%
\pgfsetstrokeopacity{0.300000}%
\pgfsetdash{}{0pt}%
\pgfpathmoveto{\pgfqpoint{1.614699in}{2.084900in}}%
\pgfpathcurveto{\pgfqpoint{1.622935in}{2.084900in}}{\pgfqpoint{1.630835in}{2.088173in}}{\pgfqpoint{1.636659in}{2.093997in}}%
\pgfpathcurveto{\pgfqpoint{1.642483in}{2.099821in}}{\pgfqpoint{1.645755in}{2.107721in}}{\pgfqpoint{1.645755in}{2.115957in}}%
\pgfpathcurveto{\pgfqpoint{1.645755in}{2.124193in}}{\pgfqpoint{1.642483in}{2.132093in}}{\pgfqpoint{1.636659in}{2.137917in}}%
\pgfpathcurveto{\pgfqpoint{1.630835in}{2.143741in}}{\pgfqpoint{1.622935in}{2.147013in}}{\pgfqpoint{1.614699in}{2.147013in}}%
\pgfpathcurveto{\pgfqpoint{1.606463in}{2.147013in}}{\pgfqpoint{1.598563in}{2.143741in}}{\pgfqpoint{1.592739in}{2.137917in}}%
\pgfpathcurveto{\pgfqpoint{1.586915in}{2.132093in}}{\pgfqpoint{1.583642in}{2.124193in}}{\pgfqpoint{1.583642in}{2.115957in}}%
\pgfpathcurveto{\pgfqpoint{1.583642in}{2.107721in}}{\pgfqpoint{1.586915in}{2.099821in}}{\pgfqpoint{1.592739in}{2.093997in}}%
\pgfpathcurveto{\pgfqpoint{1.598563in}{2.088173in}}{\pgfqpoint{1.606463in}{2.084900in}}{\pgfqpoint{1.614699in}{2.084900in}}%
\pgfpathclose%
\pgfusepath{stroke,fill}%
\end{pgfscope}%
\begin{pgfscope}%
\pgfpathrectangle{\pgfqpoint{0.100000in}{0.212622in}}{\pgfqpoint{3.696000in}{3.696000in}}%
\pgfusepath{clip}%
\pgfsetbuttcap%
\pgfsetroundjoin%
\definecolor{currentfill}{rgb}{0.121569,0.466667,0.705882}%
\pgfsetfillcolor{currentfill}%
\pgfsetfillopacity{0.300062}%
\pgfsetlinewidth{1.003750pt}%
\definecolor{currentstroke}{rgb}{0.121569,0.466667,0.705882}%
\pgfsetstrokecolor{currentstroke}%
\pgfsetstrokeopacity{0.300062}%
\pgfsetdash{}{0pt}%
\pgfpathmoveto{\pgfqpoint{1.614875in}{2.084935in}}%
\pgfpathcurveto{\pgfqpoint{1.623111in}{2.084935in}}{\pgfqpoint{1.631011in}{2.088207in}}{\pgfqpoint{1.636835in}{2.094031in}}%
\pgfpathcurveto{\pgfqpoint{1.642659in}{2.099855in}}{\pgfqpoint{1.645932in}{2.107755in}}{\pgfqpoint{1.645932in}{2.115991in}}%
\pgfpathcurveto{\pgfqpoint{1.645932in}{2.124228in}}{\pgfqpoint{1.642659in}{2.132128in}}{\pgfqpoint{1.636835in}{2.137952in}}%
\pgfpathcurveto{\pgfqpoint{1.631011in}{2.143776in}}{\pgfqpoint{1.623111in}{2.147048in}}{\pgfqpoint{1.614875in}{2.147048in}}%
\pgfpathcurveto{\pgfqpoint{1.606639in}{2.147048in}}{\pgfqpoint{1.598739in}{2.143776in}}{\pgfqpoint{1.592915in}{2.137952in}}%
\pgfpathcurveto{\pgfqpoint{1.587091in}{2.132128in}}{\pgfqpoint{1.583819in}{2.124228in}}{\pgfqpoint{1.583819in}{2.115991in}}%
\pgfpathcurveto{\pgfqpoint{1.583819in}{2.107755in}}{\pgfqpoint{1.587091in}{2.099855in}}{\pgfqpoint{1.592915in}{2.094031in}}%
\pgfpathcurveto{\pgfqpoint{1.598739in}{2.088207in}}{\pgfqpoint{1.606639in}{2.084935in}}{\pgfqpoint{1.614875in}{2.084935in}}%
\pgfpathclose%
\pgfusepath{stroke,fill}%
\end{pgfscope}%
\begin{pgfscope}%
\pgfpathrectangle{\pgfqpoint{0.100000in}{0.212622in}}{\pgfqpoint{3.696000in}{3.696000in}}%
\pgfusepath{clip}%
\pgfsetbuttcap%
\pgfsetroundjoin%
\definecolor{currentfill}{rgb}{0.121569,0.466667,0.705882}%
\pgfsetfillcolor{currentfill}%
\pgfsetfillopacity{0.300062}%
\pgfsetlinewidth{1.003750pt}%
\definecolor{currentstroke}{rgb}{0.121569,0.466667,0.705882}%
\pgfsetstrokecolor{currentstroke}%
\pgfsetstrokeopacity{0.300062}%
\pgfsetdash{}{0pt}%
\pgfpathmoveto{\pgfqpoint{1.614875in}{2.084935in}}%
\pgfpathcurveto{\pgfqpoint{1.623111in}{2.084935in}}{\pgfqpoint{1.631011in}{2.088207in}}{\pgfqpoint{1.636835in}{2.094031in}}%
\pgfpathcurveto{\pgfqpoint{1.642659in}{2.099855in}}{\pgfqpoint{1.645932in}{2.107755in}}{\pgfqpoint{1.645932in}{2.115991in}}%
\pgfpathcurveto{\pgfqpoint{1.645932in}{2.124228in}}{\pgfqpoint{1.642659in}{2.132128in}}{\pgfqpoint{1.636835in}{2.137952in}}%
\pgfpathcurveto{\pgfqpoint{1.631011in}{2.143776in}}{\pgfqpoint{1.623111in}{2.147048in}}{\pgfqpoint{1.614875in}{2.147048in}}%
\pgfpathcurveto{\pgfqpoint{1.606639in}{2.147048in}}{\pgfqpoint{1.598739in}{2.143776in}}{\pgfqpoint{1.592915in}{2.137952in}}%
\pgfpathcurveto{\pgfqpoint{1.587091in}{2.132128in}}{\pgfqpoint{1.583819in}{2.124228in}}{\pgfqpoint{1.583819in}{2.115991in}}%
\pgfpathcurveto{\pgfqpoint{1.583819in}{2.107755in}}{\pgfqpoint{1.587091in}{2.099855in}}{\pgfqpoint{1.592915in}{2.094031in}}%
\pgfpathcurveto{\pgfqpoint{1.598739in}{2.088207in}}{\pgfqpoint{1.606639in}{2.084935in}}{\pgfqpoint{1.614875in}{2.084935in}}%
\pgfpathclose%
\pgfusepath{stroke,fill}%
\end{pgfscope}%
\begin{pgfscope}%
\pgfpathrectangle{\pgfqpoint{0.100000in}{0.212622in}}{\pgfqpoint{3.696000in}{3.696000in}}%
\pgfusepath{clip}%
\pgfsetbuttcap%
\pgfsetroundjoin%
\definecolor{currentfill}{rgb}{0.121569,0.466667,0.705882}%
\pgfsetfillcolor{currentfill}%
\pgfsetfillopacity{0.300062}%
\pgfsetlinewidth{1.003750pt}%
\definecolor{currentstroke}{rgb}{0.121569,0.466667,0.705882}%
\pgfsetstrokecolor{currentstroke}%
\pgfsetstrokeopacity{0.300062}%
\pgfsetdash{}{0pt}%
\pgfpathmoveto{\pgfqpoint{1.614875in}{2.084935in}}%
\pgfpathcurveto{\pgfqpoint{1.623112in}{2.084935in}}{\pgfqpoint{1.631012in}{2.088207in}}{\pgfqpoint{1.636836in}{2.094031in}}%
\pgfpathcurveto{\pgfqpoint{1.642659in}{2.099855in}}{\pgfqpoint{1.645932in}{2.107755in}}{\pgfqpoint{1.645932in}{2.115991in}}%
\pgfpathcurveto{\pgfqpoint{1.645932in}{2.124228in}}{\pgfqpoint{1.642659in}{2.132128in}}{\pgfqpoint{1.636836in}{2.137952in}}%
\pgfpathcurveto{\pgfqpoint{1.631012in}{2.143776in}}{\pgfqpoint{1.623112in}{2.147048in}}{\pgfqpoint{1.614875in}{2.147048in}}%
\pgfpathcurveto{\pgfqpoint{1.606639in}{2.147048in}}{\pgfqpoint{1.598739in}{2.143776in}}{\pgfqpoint{1.592915in}{2.137952in}}%
\pgfpathcurveto{\pgfqpoint{1.587091in}{2.132128in}}{\pgfqpoint{1.583819in}{2.124228in}}{\pgfqpoint{1.583819in}{2.115991in}}%
\pgfpathcurveto{\pgfqpoint{1.583819in}{2.107755in}}{\pgfqpoint{1.587091in}{2.099855in}}{\pgfqpoint{1.592915in}{2.094031in}}%
\pgfpathcurveto{\pgfqpoint{1.598739in}{2.088207in}}{\pgfqpoint{1.606639in}{2.084935in}}{\pgfqpoint{1.614875in}{2.084935in}}%
\pgfpathclose%
\pgfusepath{stroke,fill}%
\end{pgfscope}%
\begin{pgfscope}%
\pgfpathrectangle{\pgfqpoint{0.100000in}{0.212622in}}{\pgfqpoint{3.696000in}{3.696000in}}%
\pgfusepath{clip}%
\pgfsetbuttcap%
\pgfsetroundjoin%
\definecolor{currentfill}{rgb}{0.121569,0.466667,0.705882}%
\pgfsetfillcolor{currentfill}%
\pgfsetfillopacity{0.300062}%
\pgfsetlinewidth{1.003750pt}%
\definecolor{currentstroke}{rgb}{0.121569,0.466667,0.705882}%
\pgfsetstrokecolor{currentstroke}%
\pgfsetstrokeopacity{0.300062}%
\pgfsetdash{}{0pt}%
\pgfpathmoveto{\pgfqpoint{1.614875in}{2.084935in}}%
\pgfpathcurveto{\pgfqpoint{1.623112in}{2.084935in}}{\pgfqpoint{1.631012in}{2.088207in}}{\pgfqpoint{1.636836in}{2.094031in}}%
\pgfpathcurveto{\pgfqpoint{1.642660in}{2.099855in}}{\pgfqpoint{1.645932in}{2.107755in}}{\pgfqpoint{1.645932in}{2.115992in}}%
\pgfpathcurveto{\pgfqpoint{1.645932in}{2.124228in}}{\pgfqpoint{1.642660in}{2.132128in}}{\pgfqpoint{1.636836in}{2.137952in}}%
\pgfpathcurveto{\pgfqpoint{1.631012in}{2.143776in}}{\pgfqpoint{1.623112in}{2.147048in}}{\pgfqpoint{1.614875in}{2.147048in}}%
\pgfpathcurveto{\pgfqpoint{1.606639in}{2.147048in}}{\pgfqpoint{1.598739in}{2.143776in}}{\pgfqpoint{1.592915in}{2.137952in}}%
\pgfpathcurveto{\pgfqpoint{1.587091in}{2.132128in}}{\pgfqpoint{1.583819in}{2.124228in}}{\pgfqpoint{1.583819in}{2.115992in}}%
\pgfpathcurveto{\pgfqpoint{1.583819in}{2.107755in}}{\pgfqpoint{1.587091in}{2.099855in}}{\pgfqpoint{1.592915in}{2.094031in}}%
\pgfpathcurveto{\pgfqpoint{1.598739in}{2.088207in}}{\pgfqpoint{1.606639in}{2.084935in}}{\pgfqpoint{1.614875in}{2.084935in}}%
\pgfpathclose%
\pgfusepath{stroke,fill}%
\end{pgfscope}%
\begin{pgfscope}%
\pgfpathrectangle{\pgfqpoint{0.100000in}{0.212622in}}{\pgfqpoint{3.696000in}{3.696000in}}%
\pgfusepath{clip}%
\pgfsetbuttcap%
\pgfsetroundjoin%
\definecolor{currentfill}{rgb}{0.121569,0.466667,0.705882}%
\pgfsetfillcolor{currentfill}%
\pgfsetfillopacity{0.300063}%
\pgfsetlinewidth{1.003750pt}%
\definecolor{currentstroke}{rgb}{0.121569,0.466667,0.705882}%
\pgfsetstrokecolor{currentstroke}%
\pgfsetstrokeopacity{0.300063}%
\pgfsetdash{}{0pt}%
\pgfpathmoveto{\pgfqpoint{1.614876in}{2.084935in}}%
\pgfpathcurveto{\pgfqpoint{1.623112in}{2.084935in}}{\pgfqpoint{1.631012in}{2.088207in}}{\pgfqpoint{1.636836in}{2.094031in}}%
\pgfpathcurveto{\pgfqpoint{1.642660in}{2.099855in}}{\pgfqpoint{1.645932in}{2.107755in}}{\pgfqpoint{1.645932in}{2.115992in}}%
\pgfpathcurveto{\pgfqpoint{1.645932in}{2.124228in}}{\pgfqpoint{1.642660in}{2.132128in}}{\pgfqpoint{1.636836in}{2.137952in}}%
\pgfpathcurveto{\pgfqpoint{1.631012in}{2.143776in}}{\pgfqpoint{1.623112in}{2.147048in}}{\pgfqpoint{1.614876in}{2.147048in}}%
\pgfpathcurveto{\pgfqpoint{1.606639in}{2.147048in}}{\pgfqpoint{1.598739in}{2.143776in}}{\pgfqpoint{1.592915in}{2.137952in}}%
\pgfpathcurveto{\pgfqpoint{1.587092in}{2.132128in}}{\pgfqpoint{1.583819in}{2.124228in}}{\pgfqpoint{1.583819in}{2.115992in}}%
\pgfpathcurveto{\pgfqpoint{1.583819in}{2.107755in}}{\pgfqpoint{1.587092in}{2.099855in}}{\pgfqpoint{1.592915in}{2.094031in}}%
\pgfpathcurveto{\pgfqpoint{1.598739in}{2.088207in}}{\pgfqpoint{1.606639in}{2.084935in}}{\pgfqpoint{1.614876in}{2.084935in}}%
\pgfpathclose%
\pgfusepath{stroke,fill}%
\end{pgfscope}%
\begin{pgfscope}%
\pgfpathrectangle{\pgfqpoint{0.100000in}{0.212622in}}{\pgfqpoint{3.696000in}{3.696000in}}%
\pgfusepath{clip}%
\pgfsetbuttcap%
\pgfsetroundjoin%
\definecolor{currentfill}{rgb}{0.121569,0.466667,0.705882}%
\pgfsetfillcolor{currentfill}%
\pgfsetfillopacity{0.300063}%
\pgfsetlinewidth{1.003750pt}%
\definecolor{currentstroke}{rgb}{0.121569,0.466667,0.705882}%
\pgfsetstrokecolor{currentstroke}%
\pgfsetstrokeopacity{0.300063}%
\pgfsetdash{}{0pt}%
\pgfpathmoveto{\pgfqpoint{1.614876in}{2.084935in}}%
\pgfpathcurveto{\pgfqpoint{1.623113in}{2.084935in}}{\pgfqpoint{1.631013in}{2.088207in}}{\pgfqpoint{1.636837in}{2.094031in}}%
\pgfpathcurveto{\pgfqpoint{1.642661in}{2.099855in}}{\pgfqpoint{1.645933in}{2.107755in}}{\pgfqpoint{1.645933in}{2.115991in}}%
\pgfpathcurveto{\pgfqpoint{1.645933in}{2.124228in}}{\pgfqpoint{1.642661in}{2.132128in}}{\pgfqpoint{1.636837in}{2.137952in}}%
\pgfpathcurveto{\pgfqpoint{1.631013in}{2.143776in}}{\pgfqpoint{1.623113in}{2.147048in}}{\pgfqpoint{1.614876in}{2.147048in}}%
\pgfpathcurveto{\pgfqpoint{1.606640in}{2.147048in}}{\pgfqpoint{1.598740in}{2.143776in}}{\pgfqpoint{1.592916in}{2.137952in}}%
\pgfpathcurveto{\pgfqpoint{1.587092in}{2.132128in}}{\pgfqpoint{1.583820in}{2.124228in}}{\pgfqpoint{1.583820in}{2.115991in}}%
\pgfpathcurveto{\pgfqpoint{1.583820in}{2.107755in}}{\pgfqpoint{1.587092in}{2.099855in}}{\pgfqpoint{1.592916in}{2.094031in}}%
\pgfpathcurveto{\pgfqpoint{1.598740in}{2.088207in}}{\pgfqpoint{1.606640in}{2.084935in}}{\pgfqpoint{1.614876in}{2.084935in}}%
\pgfpathclose%
\pgfusepath{stroke,fill}%
\end{pgfscope}%
\begin{pgfscope}%
\pgfpathrectangle{\pgfqpoint{0.100000in}{0.212622in}}{\pgfqpoint{3.696000in}{3.696000in}}%
\pgfusepath{clip}%
\pgfsetbuttcap%
\pgfsetroundjoin%
\definecolor{currentfill}{rgb}{0.121569,0.466667,0.705882}%
\pgfsetfillcolor{currentfill}%
\pgfsetfillopacity{0.300063}%
\pgfsetlinewidth{1.003750pt}%
\definecolor{currentstroke}{rgb}{0.121569,0.466667,0.705882}%
\pgfsetstrokecolor{currentstroke}%
\pgfsetstrokeopacity{0.300063}%
\pgfsetdash{}{0pt}%
\pgfpathmoveto{\pgfqpoint{1.614878in}{2.084935in}}%
\pgfpathcurveto{\pgfqpoint{1.623114in}{2.084935in}}{\pgfqpoint{1.631014in}{2.088207in}}{\pgfqpoint{1.636838in}{2.094031in}}%
\pgfpathcurveto{\pgfqpoint{1.642662in}{2.099855in}}{\pgfqpoint{1.645934in}{2.107755in}}{\pgfqpoint{1.645934in}{2.115992in}}%
\pgfpathcurveto{\pgfqpoint{1.645934in}{2.124228in}}{\pgfqpoint{1.642662in}{2.132128in}}{\pgfqpoint{1.636838in}{2.137952in}}%
\pgfpathcurveto{\pgfqpoint{1.631014in}{2.143776in}}{\pgfqpoint{1.623114in}{2.147048in}}{\pgfqpoint{1.614878in}{2.147048in}}%
\pgfpathcurveto{\pgfqpoint{1.606641in}{2.147048in}}{\pgfqpoint{1.598741in}{2.143776in}}{\pgfqpoint{1.592917in}{2.137952in}}%
\pgfpathcurveto{\pgfqpoint{1.587093in}{2.132128in}}{\pgfqpoint{1.583821in}{2.124228in}}{\pgfqpoint{1.583821in}{2.115992in}}%
\pgfpathcurveto{\pgfqpoint{1.583821in}{2.107755in}}{\pgfqpoint{1.587093in}{2.099855in}}{\pgfqpoint{1.592917in}{2.094031in}}%
\pgfpathcurveto{\pgfqpoint{1.598741in}{2.088207in}}{\pgfqpoint{1.606641in}{2.084935in}}{\pgfqpoint{1.614878in}{2.084935in}}%
\pgfpathclose%
\pgfusepath{stroke,fill}%
\end{pgfscope}%
\begin{pgfscope}%
\pgfpathrectangle{\pgfqpoint{0.100000in}{0.212622in}}{\pgfqpoint{3.696000in}{3.696000in}}%
\pgfusepath{clip}%
\pgfsetbuttcap%
\pgfsetroundjoin%
\definecolor{currentfill}{rgb}{0.121569,0.466667,0.705882}%
\pgfsetfillcolor{currentfill}%
\pgfsetfillopacity{0.300064}%
\pgfsetlinewidth{1.003750pt}%
\definecolor{currentstroke}{rgb}{0.121569,0.466667,0.705882}%
\pgfsetstrokecolor{currentstroke}%
\pgfsetstrokeopacity{0.300064}%
\pgfsetdash{}{0pt}%
\pgfpathmoveto{\pgfqpoint{1.614880in}{2.084935in}}%
\pgfpathcurveto{\pgfqpoint{1.623116in}{2.084935in}}{\pgfqpoint{1.631016in}{2.088207in}}{\pgfqpoint{1.636840in}{2.094031in}}%
\pgfpathcurveto{\pgfqpoint{1.642664in}{2.099855in}}{\pgfqpoint{1.645936in}{2.107755in}}{\pgfqpoint{1.645936in}{2.115992in}}%
\pgfpathcurveto{\pgfqpoint{1.645936in}{2.124228in}}{\pgfqpoint{1.642664in}{2.132128in}}{\pgfqpoint{1.636840in}{2.137952in}}%
\pgfpathcurveto{\pgfqpoint{1.631016in}{2.143776in}}{\pgfqpoint{1.623116in}{2.147048in}}{\pgfqpoint{1.614880in}{2.147048in}}%
\pgfpathcurveto{\pgfqpoint{1.606643in}{2.147048in}}{\pgfqpoint{1.598743in}{2.143776in}}{\pgfqpoint{1.592919in}{2.137952in}}%
\pgfpathcurveto{\pgfqpoint{1.587095in}{2.132128in}}{\pgfqpoint{1.583823in}{2.124228in}}{\pgfqpoint{1.583823in}{2.115992in}}%
\pgfpathcurveto{\pgfqpoint{1.583823in}{2.107755in}}{\pgfqpoint{1.587095in}{2.099855in}}{\pgfqpoint{1.592919in}{2.094031in}}%
\pgfpathcurveto{\pgfqpoint{1.598743in}{2.088207in}}{\pgfqpoint{1.606643in}{2.084935in}}{\pgfqpoint{1.614880in}{2.084935in}}%
\pgfpathclose%
\pgfusepath{stroke,fill}%
\end{pgfscope}%
\begin{pgfscope}%
\pgfpathrectangle{\pgfqpoint{0.100000in}{0.212622in}}{\pgfqpoint{3.696000in}{3.696000in}}%
\pgfusepath{clip}%
\pgfsetbuttcap%
\pgfsetroundjoin%
\definecolor{currentfill}{rgb}{0.121569,0.466667,0.705882}%
\pgfsetfillcolor{currentfill}%
\pgfsetfillopacity{0.300065}%
\pgfsetlinewidth{1.003750pt}%
\definecolor{currentstroke}{rgb}{0.121569,0.466667,0.705882}%
\pgfsetstrokecolor{currentstroke}%
\pgfsetstrokeopacity{0.300065}%
\pgfsetdash{}{0pt}%
\pgfpathmoveto{\pgfqpoint{1.614884in}{2.084936in}}%
\pgfpathcurveto{\pgfqpoint{1.623120in}{2.084936in}}{\pgfqpoint{1.631020in}{2.088208in}}{\pgfqpoint{1.636844in}{2.094032in}}%
\pgfpathcurveto{\pgfqpoint{1.642668in}{2.099856in}}{\pgfqpoint{1.645940in}{2.107756in}}{\pgfqpoint{1.645940in}{2.115992in}}%
\pgfpathcurveto{\pgfqpoint{1.645940in}{2.124229in}}{\pgfqpoint{1.642668in}{2.132129in}}{\pgfqpoint{1.636844in}{2.137952in}}%
\pgfpathcurveto{\pgfqpoint{1.631020in}{2.143776in}}{\pgfqpoint{1.623120in}{2.147049in}}{\pgfqpoint{1.614884in}{2.147049in}}%
\pgfpathcurveto{\pgfqpoint{1.606647in}{2.147049in}}{\pgfqpoint{1.598747in}{2.143776in}}{\pgfqpoint{1.592923in}{2.137952in}}%
\pgfpathcurveto{\pgfqpoint{1.587099in}{2.132129in}}{\pgfqpoint{1.583827in}{2.124229in}}{\pgfqpoint{1.583827in}{2.115992in}}%
\pgfpathcurveto{\pgfqpoint{1.583827in}{2.107756in}}{\pgfqpoint{1.587099in}{2.099856in}}{\pgfqpoint{1.592923in}{2.094032in}}%
\pgfpathcurveto{\pgfqpoint{1.598747in}{2.088208in}}{\pgfqpoint{1.606647in}{2.084936in}}{\pgfqpoint{1.614884in}{2.084936in}}%
\pgfpathclose%
\pgfusepath{stroke,fill}%
\end{pgfscope}%
\begin{pgfscope}%
\pgfpathrectangle{\pgfqpoint{0.100000in}{0.212622in}}{\pgfqpoint{3.696000in}{3.696000in}}%
\pgfusepath{clip}%
\pgfsetbuttcap%
\pgfsetroundjoin%
\definecolor{currentfill}{rgb}{0.121569,0.466667,0.705882}%
\pgfsetfillcolor{currentfill}%
\pgfsetfillopacity{0.300067}%
\pgfsetlinewidth{1.003750pt}%
\definecolor{currentstroke}{rgb}{0.121569,0.466667,0.705882}%
\pgfsetstrokecolor{currentstroke}%
\pgfsetstrokeopacity{0.300067}%
\pgfsetdash{}{0pt}%
\pgfpathmoveto{\pgfqpoint{1.614890in}{2.084935in}}%
\pgfpathcurveto{\pgfqpoint{1.623127in}{2.084935in}}{\pgfqpoint{1.631027in}{2.088208in}}{\pgfqpoint{1.636851in}{2.094032in}}%
\pgfpathcurveto{\pgfqpoint{1.642675in}{2.099856in}}{\pgfqpoint{1.645947in}{2.107756in}}{\pgfqpoint{1.645947in}{2.115992in}}%
\pgfpathcurveto{\pgfqpoint{1.645947in}{2.124228in}}{\pgfqpoint{1.642675in}{2.132128in}}{\pgfqpoint{1.636851in}{2.137952in}}%
\pgfpathcurveto{\pgfqpoint{1.631027in}{2.143776in}}{\pgfqpoint{1.623127in}{2.147048in}}{\pgfqpoint{1.614890in}{2.147048in}}%
\pgfpathcurveto{\pgfqpoint{1.606654in}{2.147048in}}{\pgfqpoint{1.598754in}{2.143776in}}{\pgfqpoint{1.592930in}{2.137952in}}%
\pgfpathcurveto{\pgfqpoint{1.587106in}{2.132128in}}{\pgfqpoint{1.583834in}{2.124228in}}{\pgfqpoint{1.583834in}{2.115992in}}%
\pgfpathcurveto{\pgfqpoint{1.583834in}{2.107756in}}{\pgfqpoint{1.587106in}{2.099856in}}{\pgfqpoint{1.592930in}{2.094032in}}%
\pgfpathcurveto{\pgfqpoint{1.598754in}{2.088208in}}{\pgfqpoint{1.606654in}{2.084935in}}{\pgfqpoint{1.614890in}{2.084935in}}%
\pgfpathclose%
\pgfusepath{stroke,fill}%
\end{pgfscope}%
\begin{pgfscope}%
\pgfpathrectangle{\pgfqpoint{0.100000in}{0.212622in}}{\pgfqpoint{3.696000in}{3.696000in}}%
\pgfusepath{clip}%
\pgfsetbuttcap%
\pgfsetroundjoin%
\definecolor{currentfill}{rgb}{0.121569,0.466667,0.705882}%
\pgfsetfillcolor{currentfill}%
\pgfsetfillopacity{0.300071}%
\pgfsetlinewidth{1.003750pt}%
\definecolor{currentstroke}{rgb}{0.121569,0.466667,0.705882}%
\pgfsetstrokecolor{currentstroke}%
\pgfsetstrokeopacity{0.300071}%
\pgfsetdash{}{0pt}%
\pgfpathmoveto{\pgfqpoint{1.614904in}{2.084937in}}%
\pgfpathcurveto{\pgfqpoint{1.623140in}{2.084937in}}{\pgfqpoint{1.631040in}{2.088210in}}{\pgfqpoint{1.636864in}{2.094034in}}%
\pgfpathcurveto{\pgfqpoint{1.642688in}{2.099858in}}{\pgfqpoint{1.645960in}{2.107758in}}{\pgfqpoint{1.645960in}{2.115994in}}%
\pgfpathcurveto{\pgfqpoint{1.645960in}{2.124230in}}{\pgfqpoint{1.642688in}{2.132130in}}{\pgfqpoint{1.636864in}{2.137954in}}%
\pgfpathcurveto{\pgfqpoint{1.631040in}{2.143778in}}{\pgfqpoint{1.623140in}{2.147050in}}{\pgfqpoint{1.614904in}{2.147050in}}%
\pgfpathcurveto{\pgfqpoint{1.606668in}{2.147050in}}{\pgfqpoint{1.598768in}{2.143778in}}{\pgfqpoint{1.592944in}{2.137954in}}%
\pgfpathcurveto{\pgfqpoint{1.587120in}{2.132130in}}{\pgfqpoint{1.583847in}{2.124230in}}{\pgfqpoint{1.583847in}{2.115994in}}%
\pgfpathcurveto{\pgfqpoint{1.583847in}{2.107758in}}{\pgfqpoint{1.587120in}{2.099858in}}{\pgfqpoint{1.592944in}{2.094034in}}%
\pgfpathcurveto{\pgfqpoint{1.598768in}{2.088210in}}{\pgfqpoint{1.606668in}{2.084937in}}{\pgfqpoint{1.614904in}{2.084937in}}%
\pgfpathclose%
\pgfusepath{stroke,fill}%
\end{pgfscope}%
\begin{pgfscope}%
\pgfpathrectangle{\pgfqpoint{0.100000in}{0.212622in}}{\pgfqpoint{3.696000in}{3.696000in}}%
\pgfusepath{clip}%
\pgfsetbuttcap%
\pgfsetroundjoin%
\definecolor{currentfill}{rgb}{0.121569,0.466667,0.705882}%
\pgfsetfillcolor{currentfill}%
\pgfsetfillopacity{0.300079}%
\pgfsetlinewidth{1.003750pt}%
\definecolor{currentstroke}{rgb}{0.121569,0.466667,0.705882}%
\pgfsetstrokecolor{currentstroke}%
\pgfsetstrokeopacity{0.300079}%
\pgfsetdash{}{0pt}%
\pgfpathmoveto{\pgfqpoint{1.614927in}{2.084942in}}%
\pgfpathcurveto{\pgfqpoint{1.623164in}{2.084942in}}{\pgfqpoint{1.631064in}{2.088214in}}{\pgfqpoint{1.636888in}{2.094038in}}%
\pgfpathcurveto{\pgfqpoint{1.642712in}{2.099862in}}{\pgfqpoint{1.645984in}{2.107762in}}{\pgfqpoint{1.645984in}{2.115998in}}%
\pgfpathcurveto{\pgfqpoint{1.645984in}{2.124235in}}{\pgfqpoint{1.642712in}{2.132135in}}{\pgfqpoint{1.636888in}{2.137959in}}%
\pgfpathcurveto{\pgfqpoint{1.631064in}{2.143783in}}{\pgfqpoint{1.623164in}{2.147055in}}{\pgfqpoint{1.614927in}{2.147055in}}%
\pgfpathcurveto{\pgfqpoint{1.606691in}{2.147055in}}{\pgfqpoint{1.598791in}{2.143783in}}{\pgfqpoint{1.592967in}{2.137959in}}%
\pgfpathcurveto{\pgfqpoint{1.587143in}{2.132135in}}{\pgfqpoint{1.583871in}{2.124235in}}{\pgfqpoint{1.583871in}{2.115998in}}%
\pgfpathcurveto{\pgfqpoint{1.583871in}{2.107762in}}{\pgfqpoint{1.587143in}{2.099862in}}{\pgfqpoint{1.592967in}{2.094038in}}%
\pgfpathcurveto{\pgfqpoint{1.598791in}{2.088214in}}{\pgfqpoint{1.606691in}{2.084942in}}{\pgfqpoint{1.614927in}{2.084942in}}%
\pgfpathclose%
\pgfusepath{stroke,fill}%
\end{pgfscope}%
\begin{pgfscope}%
\pgfpathrectangle{\pgfqpoint{0.100000in}{0.212622in}}{\pgfqpoint{3.696000in}{3.696000in}}%
\pgfusepath{clip}%
\pgfsetbuttcap%
\pgfsetroundjoin%
\definecolor{currentfill}{rgb}{0.121569,0.466667,0.705882}%
\pgfsetfillcolor{currentfill}%
\pgfsetfillopacity{0.300092}%
\pgfsetlinewidth{1.003750pt}%
\definecolor{currentstroke}{rgb}{0.121569,0.466667,0.705882}%
\pgfsetstrokecolor{currentstroke}%
\pgfsetstrokeopacity{0.300092}%
\pgfsetdash{}{0pt}%
\pgfpathmoveto{\pgfqpoint{1.614969in}{2.084942in}}%
\pgfpathcurveto{\pgfqpoint{1.623205in}{2.084942in}}{\pgfqpoint{1.631105in}{2.088215in}}{\pgfqpoint{1.636929in}{2.094039in}}%
\pgfpathcurveto{\pgfqpoint{1.642753in}{2.099863in}}{\pgfqpoint{1.646026in}{2.107763in}}{\pgfqpoint{1.646026in}{2.115999in}}%
\pgfpathcurveto{\pgfqpoint{1.646026in}{2.124235in}}{\pgfqpoint{1.642753in}{2.132135in}}{\pgfqpoint{1.636929in}{2.137959in}}%
\pgfpathcurveto{\pgfqpoint{1.631105in}{2.143783in}}{\pgfqpoint{1.623205in}{2.147055in}}{\pgfqpoint{1.614969in}{2.147055in}}%
\pgfpathcurveto{\pgfqpoint{1.606733in}{2.147055in}}{\pgfqpoint{1.598833in}{2.143783in}}{\pgfqpoint{1.593009in}{2.137959in}}%
\pgfpathcurveto{\pgfqpoint{1.587185in}{2.132135in}}{\pgfqpoint{1.583913in}{2.124235in}}{\pgfqpoint{1.583913in}{2.115999in}}%
\pgfpathcurveto{\pgfqpoint{1.583913in}{2.107763in}}{\pgfqpoint{1.587185in}{2.099863in}}{\pgfqpoint{1.593009in}{2.094039in}}%
\pgfpathcurveto{\pgfqpoint{1.598833in}{2.088215in}}{\pgfqpoint{1.606733in}{2.084942in}}{\pgfqpoint{1.614969in}{2.084942in}}%
\pgfpathclose%
\pgfusepath{stroke,fill}%
\end{pgfscope}%
\begin{pgfscope}%
\pgfpathrectangle{\pgfqpoint{0.100000in}{0.212622in}}{\pgfqpoint{3.696000in}{3.696000in}}%
\pgfusepath{clip}%
\pgfsetbuttcap%
\pgfsetroundjoin%
\definecolor{currentfill}{rgb}{0.121569,0.466667,0.705882}%
\pgfsetfillcolor{currentfill}%
\pgfsetfillopacity{0.300118}%
\pgfsetlinewidth{1.003750pt}%
\definecolor{currentstroke}{rgb}{0.121569,0.466667,0.705882}%
\pgfsetstrokecolor{currentstroke}%
\pgfsetstrokeopacity{0.300118}%
\pgfsetdash{}{0pt}%
\pgfpathmoveto{\pgfqpoint{1.615048in}{2.084961in}}%
\pgfpathcurveto{\pgfqpoint{1.623284in}{2.084961in}}{\pgfqpoint{1.631184in}{2.088233in}}{\pgfqpoint{1.637008in}{2.094057in}}%
\pgfpathcurveto{\pgfqpoint{1.642832in}{2.099881in}}{\pgfqpoint{1.646104in}{2.107781in}}{\pgfqpoint{1.646104in}{2.116018in}}%
\pgfpathcurveto{\pgfqpoint{1.646104in}{2.124254in}}{\pgfqpoint{1.642832in}{2.132154in}}{\pgfqpoint{1.637008in}{2.137978in}}%
\pgfpathcurveto{\pgfqpoint{1.631184in}{2.143802in}}{\pgfqpoint{1.623284in}{2.147074in}}{\pgfqpoint{1.615048in}{2.147074in}}%
\pgfpathcurveto{\pgfqpoint{1.606811in}{2.147074in}}{\pgfqpoint{1.598911in}{2.143802in}}{\pgfqpoint{1.593087in}{2.137978in}}%
\pgfpathcurveto{\pgfqpoint{1.587263in}{2.132154in}}{\pgfqpoint{1.583991in}{2.124254in}}{\pgfqpoint{1.583991in}{2.116018in}}%
\pgfpathcurveto{\pgfqpoint{1.583991in}{2.107781in}}{\pgfqpoint{1.587263in}{2.099881in}}{\pgfqpoint{1.593087in}{2.094057in}}%
\pgfpathcurveto{\pgfqpoint{1.598911in}{2.088233in}}{\pgfqpoint{1.606811in}{2.084961in}}{\pgfqpoint{1.615048in}{2.084961in}}%
\pgfpathclose%
\pgfusepath{stroke,fill}%
\end{pgfscope}%
\begin{pgfscope}%
\pgfpathrectangle{\pgfqpoint{0.100000in}{0.212622in}}{\pgfqpoint{3.696000in}{3.696000in}}%
\pgfusepath{clip}%
\pgfsetbuttcap%
\pgfsetroundjoin%
\definecolor{currentfill}{rgb}{0.121569,0.466667,0.705882}%
\pgfsetfillcolor{currentfill}%
\pgfsetfillopacity{0.300161}%
\pgfsetlinewidth{1.003750pt}%
\definecolor{currentstroke}{rgb}{0.121569,0.466667,0.705882}%
\pgfsetstrokecolor{currentstroke}%
\pgfsetstrokeopacity{0.300161}%
\pgfsetdash{}{0pt}%
\pgfpathmoveto{\pgfqpoint{1.615189in}{2.084964in}}%
\pgfpathcurveto{\pgfqpoint{1.623425in}{2.084964in}}{\pgfqpoint{1.631325in}{2.088236in}}{\pgfqpoint{1.637149in}{2.094060in}}%
\pgfpathcurveto{\pgfqpoint{1.642973in}{2.099884in}}{\pgfqpoint{1.646246in}{2.107784in}}{\pgfqpoint{1.646246in}{2.116021in}}%
\pgfpathcurveto{\pgfqpoint{1.646246in}{2.124257in}}{\pgfqpoint{1.642973in}{2.132157in}}{\pgfqpoint{1.637149in}{2.137981in}}%
\pgfpathcurveto{\pgfqpoint{1.631325in}{2.143805in}}{\pgfqpoint{1.623425in}{2.147077in}}{\pgfqpoint{1.615189in}{2.147077in}}%
\pgfpathcurveto{\pgfqpoint{1.606953in}{2.147077in}}{\pgfqpoint{1.599053in}{2.143805in}}{\pgfqpoint{1.593229in}{2.137981in}}%
\pgfpathcurveto{\pgfqpoint{1.587405in}{2.132157in}}{\pgfqpoint{1.584133in}{2.124257in}}{\pgfqpoint{1.584133in}{2.116021in}}%
\pgfpathcurveto{\pgfqpoint{1.584133in}{2.107784in}}{\pgfqpoint{1.587405in}{2.099884in}}{\pgfqpoint{1.593229in}{2.094060in}}%
\pgfpathcurveto{\pgfqpoint{1.599053in}{2.088236in}}{\pgfqpoint{1.606953in}{2.084964in}}{\pgfqpoint{1.615189in}{2.084964in}}%
\pgfpathclose%
\pgfusepath{stroke,fill}%
\end{pgfscope}%
\begin{pgfscope}%
\pgfpathrectangle{\pgfqpoint{0.100000in}{0.212622in}}{\pgfqpoint{3.696000in}{3.696000in}}%
\pgfusepath{clip}%
\pgfsetbuttcap%
\pgfsetroundjoin%
\definecolor{currentfill}{rgb}{0.121569,0.466667,0.705882}%
\pgfsetfillcolor{currentfill}%
\pgfsetfillopacity{0.300243}%
\pgfsetlinewidth{1.003750pt}%
\definecolor{currentstroke}{rgb}{0.121569,0.466667,0.705882}%
\pgfsetstrokecolor{currentstroke}%
\pgfsetstrokeopacity{0.300243}%
\pgfsetdash{}{0pt}%
\pgfpathmoveto{\pgfqpoint{1.615456in}{2.085028in}}%
\pgfpathcurveto{\pgfqpoint{1.623693in}{2.085028in}}{\pgfqpoint{1.631593in}{2.088301in}}{\pgfqpoint{1.637417in}{2.094125in}}%
\pgfpathcurveto{\pgfqpoint{1.643241in}{2.099949in}}{\pgfqpoint{1.646513in}{2.107849in}}{\pgfqpoint{1.646513in}{2.116085in}}%
\pgfpathcurveto{\pgfqpoint{1.646513in}{2.124321in}}{\pgfqpoint{1.643241in}{2.132221in}}{\pgfqpoint{1.637417in}{2.138045in}}%
\pgfpathcurveto{\pgfqpoint{1.631593in}{2.143869in}}{\pgfqpoint{1.623693in}{2.147141in}}{\pgfqpoint{1.615456in}{2.147141in}}%
\pgfpathcurveto{\pgfqpoint{1.607220in}{2.147141in}}{\pgfqpoint{1.599320in}{2.143869in}}{\pgfqpoint{1.593496in}{2.138045in}}%
\pgfpathcurveto{\pgfqpoint{1.587672in}{2.132221in}}{\pgfqpoint{1.584400in}{2.124321in}}{\pgfqpoint{1.584400in}{2.116085in}}%
\pgfpathcurveto{\pgfqpoint{1.584400in}{2.107849in}}{\pgfqpoint{1.587672in}{2.099949in}}{\pgfqpoint{1.593496in}{2.094125in}}%
\pgfpathcurveto{\pgfqpoint{1.599320in}{2.088301in}}{\pgfqpoint{1.607220in}{2.085028in}}{\pgfqpoint{1.615456in}{2.085028in}}%
\pgfpathclose%
\pgfusepath{stroke,fill}%
\end{pgfscope}%
\begin{pgfscope}%
\pgfpathrectangle{\pgfqpoint{0.100000in}{0.212622in}}{\pgfqpoint{3.696000in}{3.696000in}}%
\pgfusepath{clip}%
\pgfsetbuttcap%
\pgfsetroundjoin%
\definecolor{currentfill}{rgb}{0.121569,0.466667,0.705882}%
\pgfsetfillcolor{currentfill}%
\pgfsetfillopacity{0.300377}%
\pgfsetlinewidth{1.003750pt}%
\definecolor{currentstroke}{rgb}{0.121569,0.466667,0.705882}%
\pgfsetstrokecolor{currentstroke}%
\pgfsetstrokeopacity{0.300377}%
\pgfsetdash{}{0pt}%
\pgfpathmoveto{\pgfqpoint{1.615943in}{2.085047in}}%
\pgfpathcurveto{\pgfqpoint{1.624179in}{2.085047in}}{\pgfqpoint{1.632079in}{2.088319in}}{\pgfqpoint{1.637903in}{2.094143in}}%
\pgfpathcurveto{\pgfqpoint{1.643727in}{2.099967in}}{\pgfqpoint{1.646999in}{2.107867in}}{\pgfqpoint{1.646999in}{2.116103in}}%
\pgfpathcurveto{\pgfqpoint{1.646999in}{2.124340in}}{\pgfqpoint{1.643727in}{2.132240in}}{\pgfqpoint{1.637903in}{2.138064in}}%
\pgfpathcurveto{\pgfqpoint{1.632079in}{2.143888in}}{\pgfqpoint{1.624179in}{2.147160in}}{\pgfqpoint{1.615943in}{2.147160in}}%
\pgfpathcurveto{\pgfqpoint{1.607707in}{2.147160in}}{\pgfqpoint{1.599807in}{2.143888in}}{\pgfqpoint{1.593983in}{2.138064in}}%
\pgfpathcurveto{\pgfqpoint{1.588159in}{2.132240in}}{\pgfqpoint{1.584886in}{2.124340in}}{\pgfqpoint{1.584886in}{2.116103in}}%
\pgfpathcurveto{\pgfqpoint{1.584886in}{2.107867in}}{\pgfqpoint{1.588159in}{2.099967in}}{\pgfqpoint{1.593983in}{2.094143in}}%
\pgfpathcurveto{\pgfqpoint{1.599807in}{2.088319in}}{\pgfqpoint{1.607707in}{2.085047in}}{\pgfqpoint{1.615943in}{2.085047in}}%
\pgfpathclose%
\pgfusepath{stroke,fill}%
\end{pgfscope}%
\begin{pgfscope}%
\pgfpathrectangle{\pgfqpoint{0.100000in}{0.212622in}}{\pgfqpoint{3.696000in}{3.696000in}}%
\pgfusepath{clip}%
\pgfsetbuttcap%
\pgfsetroundjoin%
\definecolor{currentfill}{rgb}{0.121569,0.466667,0.705882}%
\pgfsetfillcolor{currentfill}%
\pgfsetfillopacity{0.300456}%
\pgfsetlinewidth{1.003750pt}%
\definecolor{currentstroke}{rgb}{0.121569,0.466667,0.705882}%
\pgfsetstrokecolor{currentstroke}%
\pgfsetstrokeopacity{0.300456}%
\pgfsetdash{}{0pt}%
\pgfpathmoveto{\pgfqpoint{1.616244in}{2.085112in}}%
\pgfpathcurveto{\pgfqpoint{1.624480in}{2.085112in}}{\pgfqpoint{1.632381in}{2.088384in}}{\pgfqpoint{1.638204in}{2.094208in}}%
\pgfpathcurveto{\pgfqpoint{1.644028in}{2.100032in}}{\pgfqpoint{1.647301in}{2.107932in}}{\pgfqpoint{1.647301in}{2.116169in}}%
\pgfpathcurveto{\pgfqpoint{1.647301in}{2.124405in}}{\pgfqpoint{1.644028in}{2.132305in}}{\pgfqpoint{1.638204in}{2.138129in}}%
\pgfpathcurveto{\pgfqpoint{1.632381in}{2.143953in}}{\pgfqpoint{1.624480in}{2.147225in}}{\pgfqpoint{1.616244in}{2.147225in}}%
\pgfpathcurveto{\pgfqpoint{1.608008in}{2.147225in}}{\pgfqpoint{1.600108in}{2.143953in}}{\pgfqpoint{1.594284in}{2.138129in}}%
\pgfpathcurveto{\pgfqpoint{1.588460in}{2.132305in}}{\pgfqpoint{1.585188in}{2.124405in}}{\pgfqpoint{1.585188in}{2.116169in}}%
\pgfpathcurveto{\pgfqpoint{1.585188in}{2.107932in}}{\pgfqpoint{1.588460in}{2.100032in}}{\pgfqpoint{1.594284in}{2.094208in}}%
\pgfpathcurveto{\pgfqpoint{1.600108in}{2.088384in}}{\pgfqpoint{1.608008in}{2.085112in}}{\pgfqpoint{1.616244in}{2.085112in}}%
\pgfpathclose%
\pgfusepath{stroke,fill}%
\end{pgfscope}%
\begin{pgfscope}%
\pgfpathrectangle{\pgfqpoint{0.100000in}{0.212622in}}{\pgfqpoint{3.696000in}{3.696000in}}%
\pgfusepath{clip}%
\pgfsetbuttcap%
\pgfsetroundjoin%
\definecolor{currentfill}{rgb}{0.121569,0.466667,0.705882}%
\pgfsetfillcolor{currentfill}%
\pgfsetfillopacity{0.300599}%
\pgfsetlinewidth{1.003750pt}%
\definecolor{currentstroke}{rgb}{0.121569,0.466667,0.705882}%
\pgfsetstrokecolor{currentstroke}%
\pgfsetstrokeopacity{0.300599}%
\pgfsetdash{}{0pt}%
\pgfpathmoveto{\pgfqpoint{1.616760in}{2.085127in}}%
\pgfpathcurveto{\pgfqpoint{1.624997in}{2.085127in}}{\pgfqpoint{1.632897in}{2.088399in}}{\pgfqpoint{1.638721in}{2.094223in}}%
\pgfpathcurveto{\pgfqpoint{1.644544in}{2.100047in}}{\pgfqpoint{1.647817in}{2.107947in}}{\pgfqpoint{1.647817in}{2.116183in}}%
\pgfpathcurveto{\pgfqpoint{1.647817in}{2.124419in}}{\pgfqpoint{1.644544in}{2.132319in}}{\pgfqpoint{1.638721in}{2.138143in}}%
\pgfpathcurveto{\pgfqpoint{1.632897in}{2.143967in}}{\pgfqpoint{1.624997in}{2.147240in}}{\pgfqpoint{1.616760in}{2.147240in}}%
\pgfpathcurveto{\pgfqpoint{1.608524in}{2.147240in}}{\pgfqpoint{1.600624in}{2.143967in}}{\pgfqpoint{1.594800in}{2.138143in}}%
\pgfpathcurveto{\pgfqpoint{1.588976in}{2.132319in}}{\pgfqpoint{1.585704in}{2.124419in}}{\pgfqpoint{1.585704in}{2.116183in}}%
\pgfpathcurveto{\pgfqpoint{1.585704in}{2.107947in}}{\pgfqpoint{1.588976in}{2.100047in}}{\pgfqpoint{1.594800in}{2.094223in}}%
\pgfpathcurveto{\pgfqpoint{1.600624in}{2.088399in}}{\pgfqpoint{1.608524in}{2.085127in}}{\pgfqpoint{1.616760in}{2.085127in}}%
\pgfpathclose%
\pgfusepath{stroke,fill}%
\end{pgfscope}%
\begin{pgfscope}%
\pgfpathrectangle{\pgfqpoint{0.100000in}{0.212622in}}{\pgfqpoint{3.696000in}{3.696000in}}%
\pgfusepath{clip}%
\pgfsetbuttcap%
\pgfsetroundjoin%
\definecolor{currentfill}{rgb}{0.121569,0.466667,0.705882}%
\pgfsetfillcolor{currentfill}%
\pgfsetfillopacity{0.300862}%
\pgfsetlinewidth{1.003750pt}%
\definecolor{currentstroke}{rgb}{0.121569,0.466667,0.705882}%
\pgfsetstrokecolor{currentstroke}%
\pgfsetstrokeopacity{0.300862}%
\pgfsetdash{}{0pt}%
\pgfpathmoveto{\pgfqpoint{1.617750in}{2.085333in}}%
\pgfpathcurveto{\pgfqpoint{1.625986in}{2.085333in}}{\pgfqpoint{1.633886in}{2.088605in}}{\pgfqpoint{1.639710in}{2.094429in}}%
\pgfpathcurveto{\pgfqpoint{1.645534in}{2.100253in}}{\pgfqpoint{1.648806in}{2.108153in}}{\pgfqpoint{1.648806in}{2.116390in}}%
\pgfpathcurveto{\pgfqpoint{1.648806in}{2.124626in}}{\pgfqpoint{1.645534in}{2.132526in}}{\pgfqpoint{1.639710in}{2.138350in}}%
\pgfpathcurveto{\pgfqpoint{1.633886in}{2.144174in}}{\pgfqpoint{1.625986in}{2.147446in}}{\pgfqpoint{1.617750in}{2.147446in}}%
\pgfpathcurveto{\pgfqpoint{1.609513in}{2.147446in}}{\pgfqpoint{1.601613in}{2.144174in}}{\pgfqpoint{1.595789in}{2.138350in}}%
\pgfpathcurveto{\pgfqpoint{1.589966in}{2.132526in}}{\pgfqpoint{1.586693in}{2.124626in}}{\pgfqpoint{1.586693in}{2.116390in}}%
\pgfpathcurveto{\pgfqpoint{1.586693in}{2.108153in}}{\pgfqpoint{1.589966in}{2.100253in}}{\pgfqpoint{1.595789in}{2.094429in}}%
\pgfpathcurveto{\pgfqpoint{1.601613in}{2.088605in}}{\pgfqpoint{1.609513in}{2.085333in}}{\pgfqpoint{1.617750in}{2.085333in}}%
\pgfpathclose%
\pgfusepath{stroke,fill}%
\end{pgfscope}%
\begin{pgfscope}%
\pgfpathrectangle{\pgfqpoint{0.100000in}{0.212622in}}{\pgfqpoint{3.696000in}{3.696000in}}%
\pgfusepath{clip}%
\pgfsetbuttcap%
\pgfsetroundjoin%
\definecolor{currentfill}{rgb}{0.121569,0.466667,0.705882}%
\pgfsetfillcolor{currentfill}%
\pgfsetfillopacity{0.301334}%
\pgfsetlinewidth{1.003750pt}%
\definecolor{currentstroke}{rgb}{0.121569,0.466667,0.705882}%
\pgfsetstrokecolor{currentstroke}%
\pgfsetstrokeopacity{0.301334}%
\pgfsetdash{}{0pt}%
\pgfpathmoveto{\pgfqpoint{1.619520in}{2.085571in}}%
\pgfpathcurveto{\pgfqpoint{1.627756in}{2.085571in}}{\pgfqpoint{1.635656in}{2.088843in}}{\pgfqpoint{1.641480in}{2.094667in}}%
\pgfpathcurveto{\pgfqpoint{1.647304in}{2.100491in}}{\pgfqpoint{1.650576in}{2.108391in}}{\pgfqpoint{1.650576in}{2.116627in}}%
\pgfpathcurveto{\pgfqpoint{1.650576in}{2.124864in}}{\pgfqpoint{1.647304in}{2.132764in}}{\pgfqpoint{1.641480in}{2.138588in}}%
\pgfpathcurveto{\pgfqpoint{1.635656in}{2.144412in}}{\pgfqpoint{1.627756in}{2.147684in}}{\pgfqpoint{1.619520in}{2.147684in}}%
\pgfpathcurveto{\pgfqpoint{1.611284in}{2.147684in}}{\pgfqpoint{1.603384in}{2.144412in}}{\pgfqpoint{1.597560in}{2.138588in}}%
\pgfpathcurveto{\pgfqpoint{1.591736in}{2.132764in}}{\pgfqpoint{1.588463in}{2.124864in}}{\pgfqpoint{1.588463in}{2.116627in}}%
\pgfpathcurveto{\pgfqpoint{1.588463in}{2.108391in}}{\pgfqpoint{1.591736in}{2.100491in}}{\pgfqpoint{1.597560in}{2.094667in}}%
\pgfpathcurveto{\pgfqpoint{1.603384in}{2.088843in}}{\pgfqpoint{1.611284in}{2.085571in}}{\pgfqpoint{1.619520in}{2.085571in}}%
\pgfpathclose%
\pgfusepath{stroke,fill}%
\end{pgfscope}%
\begin{pgfscope}%
\pgfpathrectangle{\pgfqpoint{0.100000in}{0.212622in}}{\pgfqpoint{3.696000in}{3.696000in}}%
\pgfusepath{clip}%
\pgfsetbuttcap%
\pgfsetroundjoin%
\definecolor{currentfill}{rgb}{0.121569,0.466667,0.705882}%
\pgfsetfillcolor{currentfill}%
\pgfsetfillopacity{0.302207}%
\pgfsetlinewidth{1.003750pt}%
\definecolor{currentstroke}{rgb}{0.121569,0.466667,0.705882}%
\pgfsetstrokecolor{currentstroke}%
\pgfsetstrokeopacity{0.302207}%
\pgfsetdash{}{0pt}%
\pgfpathmoveto{\pgfqpoint{1.622905in}{2.086628in}}%
\pgfpathcurveto{\pgfqpoint{1.631141in}{2.086628in}}{\pgfqpoint{1.639041in}{2.089900in}}{\pgfqpoint{1.644865in}{2.095724in}}%
\pgfpathcurveto{\pgfqpoint{1.650689in}{2.101548in}}{\pgfqpoint{1.653962in}{2.109448in}}{\pgfqpoint{1.653962in}{2.117684in}}%
\pgfpathcurveto{\pgfqpoint{1.653962in}{2.125921in}}{\pgfqpoint{1.650689in}{2.133821in}}{\pgfqpoint{1.644865in}{2.139645in}}%
\pgfpathcurveto{\pgfqpoint{1.639041in}{2.145469in}}{\pgfqpoint{1.631141in}{2.148741in}}{\pgfqpoint{1.622905in}{2.148741in}}%
\pgfpathcurveto{\pgfqpoint{1.614669in}{2.148741in}}{\pgfqpoint{1.606769in}{2.145469in}}{\pgfqpoint{1.600945in}{2.139645in}}%
\pgfpathcurveto{\pgfqpoint{1.595121in}{2.133821in}}{\pgfqpoint{1.591849in}{2.125921in}}{\pgfqpoint{1.591849in}{2.117684in}}%
\pgfpathcurveto{\pgfqpoint{1.591849in}{2.109448in}}{\pgfqpoint{1.595121in}{2.101548in}}{\pgfqpoint{1.600945in}{2.095724in}}%
\pgfpathcurveto{\pgfqpoint{1.606769in}{2.089900in}}{\pgfqpoint{1.614669in}{2.086628in}}{\pgfqpoint{1.622905in}{2.086628in}}%
\pgfpathclose%
\pgfusepath{stroke,fill}%
\end{pgfscope}%
\begin{pgfscope}%
\pgfpathrectangle{\pgfqpoint{0.100000in}{0.212622in}}{\pgfqpoint{3.696000in}{3.696000in}}%
\pgfusepath{clip}%
\pgfsetbuttcap%
\pgfsetroundjoin%
\definecolor{currentfill}{rgb}{0.121569,0.466667,0.705882}%
\pgfsetfillcolor{currentfill}%
\pgfsetfillopacity{0.303696}%
\pgfsetlinewidth{1.003750pt}%
\definecolor{currentstroke}{rgb}{0.121569,0.466667,0.705882}%
\pgfsetstrokecolor{currentstroke}%
\pgfsetstrokeopacity{0.303696}%
\pgfsetdash{}{0pt}%
\pgfpathmoveto{\pgfqpoint{1.628886in}{2.087329in}}%
\pgfpathcurveto{\pgfqpoint{1.637123in}{2.087329in}}{\pgfqpoint{1.645023in}{2.090601in}}{\pgfqpoint{1.650846in}{2.096425in}}%
\pgfpathcurveto{\pgfqpoint{1.656670in}{2.102249in}}{\pgfqpoint{1.659943in}{2.110149in}}{\pgfqpoint{1.659943in}{2.118385in}}%
\pgfpathcurveto{\pgfqpoint{1.659943in}{2.126621in}}{\pgfqpoint{1.656670in}{2.134521in}}{\pgfqpoint{1.650846in}{2.140345in}}%
\pgfpathcurveto{\pgfqpoint{1.645023in}{2.146169in}}{\pgfqpoint{1.637123in}{2.149442in}}{\pgfqpoint{1.628886in}{2.149442in}}%
\pgfpathcurveto{\pgfqpoint{1.620650in}{2.149442in}}{\pgfqpoint{1.612750in}{2.146169in}}{\pgfqpoint{1.606926in}{2.140345in}}%
\pgfpathcurveto{\pgfqpoint{1.601102in}{2.134521in}}{\pgfqpoint{1.597830in}{2.126621in}}{\pgfqpoint{1.597830in}{2.118385in}}%
\pgfpathcurveto{\pgfqpoint{1.597830in}{2.110149in}}{\pgfqpoint{1.601102in}{2.102249in}}{\pgfqpoint{1.606926in}{2.096425in}}%
\pgfpathcurveto{\pgfqpoint{1.612750in}{2.090601in}}{\pgfqpoint{1.620650in}{2.087329in}}{\pgfqpoint{1.628886in}{2.087329in}}%
\pgfpathclose%
\pgfusepath{stroke,fill}%
\end{pgfscope}%
\begin{pgfscope}%
\pgfpathrectangle{\pgfqpoint{0.100000in}{0.212622in}}{\pgfqpoint{3.696000in}{3.696000in}}%
\pgfusepath{clip}%
\pgfsetbuttcap%
\pgfsetroundjoin%
\definecolor{currentfill}{rgb}{0.121569,0.466667,0.705882}%
\pgfsetfillcolor{currentfill}%
\pgfsetfillopacity{0.305974}%
\pgfsetlinewidth{1.003750pt}%
\definecolor{currentstroke}{rgb}{0.121569,0.466667,0.705882}%
\pgfsetstrokecolor{currentstroke}%
\pgfsetstrokeopacity{0.305974}%
\pgfsetdash{}{0pt}%
\pgfpathmoveto{\pgfqpoint{1.640448in}{2.087943in}}%
\pgfpathcurveto{\pgfqpoint{1.648684in}{2.087943in}}{\pgfqpoint{1.656584in}{2.091216in}}{\pgfqpoint{1.662408in}{2.097039in}}%
\pgfpathcurveto{\pgfqpoint{1.668232in}{2.102863in}}{\pgfqpoint{1.671504in}{2.110763in}}{\pgfqpoint{1.671504in}{2.119000in}}%
\pgfpathcurveto{\pgfqpoint{1.671504in}{2.127236in}}{\pgfqpoint{1.668232in}{2.135136in}}{\pgfqpoint{1.662408in}{2.140960in}}%
\pgfpathcurveto{\pgfqpoint{1.656584in}{2.146784in}}{\pgfqpoint{1.648684in}{2.150056in}}{\pgfqpoint{1.640448in}{2.150056in}}%
\pgfpathcurveto{\pgfqpoint{1.632211in}{2.150056in}}{\pgfqpoint{1.624311in}{2.146784in}}{\pgfqpoint{1.618487in}{2.140960in}}%
\pgfpathcurveto{\pgfqpoint{1.612664in}{2.135136in}}{\pgfqpoint{1.609391in}{2.127236in}}{\pgfqpoint{1.609391in}{2.119000in}}%
\pgfpathcurveto{\pgfqpoint{1.609391in}{2.110763in}}{\pgfqpoint{1.612664in}{2.102863in}}{\pgfqpoint{1.618487in}{2.097039in}}%
\pgfpathcurveto{\pgfqpoint{1.624311in}{2.091216in}}{\pgfqpoint{1.632211in}{2.087943in}}{\pgfqpoint{1.640448in}{2.087943in}}%
\pgfpathclose%
\pgfusepath{stroke,fill}%
\end{pgfscope}%
\begin{pgfscope}%
\pgfpathrectangle{\pgfqpoint{0.100000in}{0.212622in}}{\pgfqpoint{3.696000in}{3.696000in}}%
\pgfusepath{clip}%
\pgfsetbuttcap%
\pgfsetroundjoin%
\definecolor{currentfill}{rgb}{0.121569,0.466667,0.705882}%
\pgfsetfillcolor{currentfill}%
\pgfsetfillopacity{0.308086}%
\pgfsetlinewidth{1.003750pt}%
\definecolor{currentstroke}{rgb}{0.121569,0.466667,0.705882}%
\pgfsetstrokecolor{currentstroke}%
\pgfsetstrokeopacity{0.308086}%
\pgfsetdash{}{0pt}%
\pgfpathmoveto{\pgfqpoint{1.651702in}{2.087353in}}%
\pgfpathcurveto{\pgfqpoint{1.659938in}{2.087353in}}{\pgfqpoint{1.667838in}{2.090625in}}{\pgfqpoint{1.673662in}{2.096449in}}%
\pgfpathcurveto{\pgfqpoint{1.679486in}{2.102273in}}{\pgfqpoint{1.682758in}{2.110173in}}{\pgfqpoint{1.682758in}{2.118409in}}%
\pgfpathcurveto{\pgfqpoint{1.682758in}{2.126645in}}{\pgfqpoint{1.679486in}{2.134545in}}{\pgfqpoint{1.673662in}{2.140369in}}%
\pgfpathcurveto{\pgfqpoint{1.667838in}{2.146193in}}{\pgfqpoint{1.659938in}{2.149466in}}{\pgfqpoint{1.651702in}{2.149466in}}%
\pgfpathcurveto{\pgfqpoint{1.643465in}{2.149466in}}{\pgfqpoint{1.635565in}{2.146193in}}{\pgfqpoint{1.629741in}{2.140369in}}%
\pgfpathcurveto{\pgfqpoint{1.623918in}{2.134545in}}{\pgfqpoint{1.620645in}{2.126645in}}{\pgfqpoint{1.620645in}{2.118409in}}%
\pgfpathcurveto{\pgfqpoint{1.620645in}{2.110173in}}{\pgfqpoint{1.623918in}{2.102273in}}{\pgfqpoint{1.629741in}{2.096449in}}%
\pgfpathcurveto{\pgfqpoint{1.635565in}{2.090625in}}{\pgfqpoint{1.643465in}{2.087353in}}{\pgfqpoint{1.651702in}{2.087353in}}%
\pgfpathclose%
\pgfusepath{stroke,fill}%
\end{pgfscope}%
\begin{pgfscope}%
\pgfpathrectangle{\pgfqpoint{0.100000in}{0.212622in}}{\pgfqpoint{3.696000in}{3.696000in}}%
\pgfusepath{clip}%
\pgfsetbuttcap%
\pgfsetroundjoin%
\definecolor{currentfill}{rgb}{0.121569,0.466667,0.705882}%
\pgfsetfillcolor{currentfill}%
\pgfsetfillopacity{0.310341}%
\pgfsetlinewidth{1.003750pt}%
\definecolor{currentstroke}{rgb}{0.121569,0.466667,0.705882}%
\pgfsetstrokecolor{currentstroke}%
\pgfsetstrokeopacity{0.310341}%
\pgfsetdash{}{0pt}%
\pgfpathmoveto{\pgfqpoint{1.663001in}{2.088804in}}%
\pgfpathcurveto{\pgfqpoint{1.671237in}{2.088804in}}{\pgfqpoint{1.679137in}{2.092076in}}{\pgfqpoint{1.684961in}{2.097900in}}%
\pgfpathcurveto{\pgfqpoint{1.690785in}{2.103724in}}{\pgfqpoint{1.694057in}{2.111624in}}{\pgfqpoint{1.694057in}{2.119861in}}%
\pgfpathcurveto{\pgfqpoint{1.694057in}{2.128097in}}{\pgfqpoint{1.690785in}{2.135997in}}{\pgfqpoint{1.684961in}{2.141821in}}%
\pgfpathcurveto{\pgfqpoint{1.679137in}{2.147645in}}{\pgfqpoint{1.671237in}{2.150917in}}{\pgfqpoint{1.663001in}{2.150917in}}%
\pgfpathcurveto{\pgfqpoint{1.654764in}{2.150917in}}{\pgfqpoint{1.646864in}{2.147645in}}{\pgfqpoint{1.641040in}{2.141821in}}%
\pgfpathcurveto{\pgfqpoint{1.635216in}{2.135997in}}{\pgfqpoint{1.631944in}{2.128097in}}{\pgfqpoint{1.631944in}{2.119861in}}%
\pgfpathcurveto{\pgfqpoint{1.631944in}{2.111624in}}{\pgfqpoint{1.635216in}{2.103724in}}{\pgfqpoint{1.641040in}{2.097900in}}%
\pgfpathcurveto{\pgfqpoint{1.646864in}{2.092076in}}{\pgfqpoint{1.654764in}{2.088804in}}{\pgfqpoint{1.663001in}{2.088804in}}%
\pgfpathclose%
\pgfusepath{stroke,fill}%
\end{pgfscope}%
\begin{pgfscope}%
\pgfpathrectangle{\pgfqpoint{0.100000in}{0.212622in}}{\pgfqpoint{3.696000in}{3.696000in}}%
\pgfusepath{clip}%
\pgfsetbuttcap%
\pgfsetroundjoin%
\definecolor{currentfill}{rgb}{0.121569,0.466667,0.705882}%
\pgfsetfillcolor{currentfill}%
\pgfsetfillopacity{0.314269}%
\pgfsetlinewidth{1.003750pt}%
\definecolor{currentstroke}{rgb}{0.121569,0.466667,0.705882}%
\pgfsetstrokecolor{currentstroke}%
\pgfsetstrokeopacity{0.314269}%
\pgfsetdash{}{0pt}%
\pgfpathmoveto{\pgfqpoint{1.683036in}{2.088475in}}%
\pgfpathcurveto{\pgfqpoint{1.691272in}{2.088475in}}{\pgfqpoint{1.699172in}{2.091747in}}{\pgfqpoint{1.704996in}{2.097571in}}%
\pgfpathcurveto{\pgfqpoint{1.710820in}{2.103395in}}{\pgfqpoint{1.714092in}{2.111295in}}{\pgfqpoint{1.714092in}{2.119531in}}%
\pgfpathcurveto{\pgfqpoint{1.714092in}{2.127767in}}{\pgfqpoint{1.710820in}{2.135667in}}{\pgfqpoint{1.704996in}{2.141491in}}%
\pgfpathcurveto{\pgfqpoint{1.699172in}{2.147315in}}{\pgfqpoint{1.691272in}{2.150588in}}{\pgfqpoint{1.683036in}{2.150588in}}%
\pgfpathcurveto{\pgfqpoint{1.674800in}{2.150588in}}{\pgfqpoint{1.666900in}{2.147315in}}{\pgfqpoint{1.661076in}{2.141491in}}%
\pgfpathcurveto{\pgfqpoint{1.655252in}{2.135667in}}{\pgfqpoint{1.651979in}{2.127767in}}{\pgfqpoint{1.651979in}{2.119531in}}%
\pgfpathcurveto{\pgfqpoint{1.651979in}{2.111295in}}{\pgfqpoint{1.655252in}{2.103395in}}{\pgfqpoint{1.661076in}{2.097571in}}%
\pgfpathcurveto{\pgfqpoint{1.666900in}{2.091747in}}{\pgfqpoint{1.674800in}{2.088475in}}{\pgfqpoint{1.683036in}{2.088475in}}%
\pgfpathclose%
\pgfusepath{stroke,fill}%
\end{pgfscope}%
\begin{pgfscope}%
\pgfpathrectangle{\pgfqpoint{0.100000in}{0.212622in}}{\pgfqpoint{3.696000in}{3.696000in}}%
\pgfusepath{clip}%
\pgfsetbuttcap%
\pgfsetroundjoin%
\definecolor{currentfill}{rgb}{0.121569,0.466667,0.705882}%
\pgfsetfillcolor{currentfill}%
\pgfsetfillopacity{0.318634}%
\pgfsetlinewidth{1.003750pt}%
\definecolor{currentstroke}{rgb}{0.121569,0.466667,0.705882}%
\pgfsetstrokecolor{currentstroke}%
\pgfsetstrokeopacity{0.318634}%
\pgfsetdash{}{0pt}%
\pgfpathmoveto{\pgfqpoint{1.702624in}{2.090390in}}%
\pgfpathcurveto{\pgfqpoint{1.710861in}{2.090390in}}{\pgfqpoint{1.718761in}{2.093663in}}{\pgfqpoint{1.724585in}{2.099487in}}%
\pgfpathcurveto{\pgfqpoint{1.730409in}{2.105310in}}{\pgfqpoint{1.733681in}{2.113211in}}{\pgfqpoint{1.733681in}{2.121447in}}%
\pgfpathcurveto{\pgfqpoint{1.733681in}{2.129683in}}{\pgfqpoint{1.730409in}{2.137583in}}{\pgfqpoint{1.724585in}{2.143407in}}%
\pgfpathcurveto{\pgfqpoint{1.718761in}{2.149231in}}{\pgfqpoint{1.710861in}{2.152503in}}{\pgfqpoint{1.702624in}{2.152503in}}%
\pgfpathcurveto{\pgfqpoint{1.694388in}{2.152503in}}{\pgfqpoint{1.686488in}{2.149231in}}{\pgfqpoint{1.680664in}{2.143407in}}%
\pgfpathcurveto{\pgfqpoint{1.674840in}{2.137583in}}{\pgfqpoint{1.671568in}{2.129683in}}{\pgfqpoint{1.671568in}{2.121447in}}%
\pgfpathcurveto{\pgfqpoint{1.671568in}{2.113211in}}{\pgfqpoint{1.674840in}{2.105310in}}{\pgfqpoint{1.680664in}{2.099487in}}%
\pgfpathcurveto{\pgfqpoint{1.686488in}{2.093663in}}{\pgfqpoint{1.694388in}{2.090390in}}{\pgfqpoint{1.702624in}{2.090390in}}%
\pgfpathclose%
\pgfusepath{stroke,fill}%
\end{pgfscope}%
\begin{pgfscope}%
\pgfpathrectangle{\pgfqpoint{0.100000in}{0.212622in}}{\pgfqpoint{3.696000in}{3.696000in}}%
\pgfusepath{clip}%
\pgfsetbuttcap%
\pgfsetroundjoin%
\definecolor{currentfill}{rgb}{0.121569,0.466667,0.705882}%
\pgfsetfillcolor{currentfill}%
\pgfsetfillopacity{0.322825}%
\pgfsetlinewidth{1.003750pt}%
\definecolor{currentstroke}{rgb}{0.121569,0.466667,0.705882}%
\pgfsetstrokecolor{currentstroke}%
\pgfsetstrokeopacity{0.322825}%
\pgfsetdash{}{0pt}%
\pgfpathmoveto{\pgfqpoint{1.721093in}{2.088460in}}%
\pgfpathcurveto{\pgfqpoint{1.729329in}{2.088460in}}{\pgfqpoint{1.737229in}{2.091732in}}{\pgfqpoint{1.743053in}{2.097556in}}%
\pgfpathcurveto{\pgfqpoint{1.748877in}{2.103380in}}{\pgfqpoint{1.752149in}{2.111280in}}{\pgfqpoint{1.752149in}{2.119517in}}%
\pgfpathcurveto{\pgfqpoint{1.752149in}{2.127753in}}{\pgfqpoint{1.748877in}{2.135653in}}{\pgfqpoint{1.743053in}{2.141477in}}%
\pgfpathcurveto{\pgfqpoint{1.737229in}{2.147301in}}{\pgfqpoint{1.729329in}{2.150573in}}{\pgfqpoint{1.721093in}{2.150573in}}%
\pgfpathcurveto{\pgfqpoint{1.712856in}{2.150573in}}{\pgfqpoint{1.704956in}{2.147301in}}{\pgfqpoint{1.699132in}{2.141477in}}%
\pgfpathcurveto{\pgfqpoint{1.693308in}{2.135653in}}{\pgfqpoint{1.690036in}{2.127753in}}{\pgfqpoint{1.690036in}{2.119517in}}%
\pgfpathcurveto{\pgfqpoint{1.690036in}{2.111280in}}{\pgfqpoint{1.693308in}{2.103380in}}{\pgfqpoint{1.699132in}{2.097556in}}%
\pgfpathcurveto{\pgfqpoint{1.704956in}{2.091732in}}{\pgfqpoint{1.712856in}{2.088460in}}{\pgfqpoint{1.721093in}{2.088460in}}%
\pgfpathclose%
\pgfusepath{stroke,fill}%
\end{pgfscope}%
\begin{pgfscope}%
\pgfpathrectangle{\pgfqpoint{0.100000in}{0.212622in}}{\pgfqpoint{3.696000in}{3.696000in}}%
\pgfusepath{clip}%
\pgfsetbuttcap%
\pgfsetroundjoin%
\definecolor{currentfill}{rgb}{0.121569,0.466667,0.705882}%
\pgfsetfillcolor{currentfill}%
\pgfsetfillopacity{0.326943}%
\pgfsetlinewidth{1.003750pt}%
\definecolor{currentstroke}{rgb}{0.121569,0.466667,0.705882}%
\pgfsetstrokecolor{currentstroke}%
\pgfsetstrokeopacity{0.326943}%
\pgfsetdash{}{0pt}%
\pgfpathmoveto{\pgfqpoint{1.740537in}{2.090053in}}%
\pgfpathcurveto{\pgfqpoint{1.748774in}{2.090053in}}{\pgfqpoint{1.756674in}{2.093325in}}{\pgfqpoint{1.762498in}{2.099149in}}%
\pgfpathcurveto{\pgfqpoint{1.768322in}{2.104973in}}{\pgfqpoint{1.771594in}{2.112873in}}{\pgfqpoint{1.771594in}{2.121110in}}%
\pgfpathcurveto{\pgfqpoint{1.771594in}{2.129346in}}{\pgfqpoint{1.768322in}{2.137246in}}{\pgfqpoint{1.762498in}{2.143070in}}%
\pgfpathcurveto{\pgfqpoint{1.756674in}{2.148894in}}{\pgfqpoint{1.748774in}{2.152166in}}{\pgfqpoint{1.740537in}{2.152166in}}%
\pgfpathcurveto{\pgfqpoint{1.732301in}{2.152166in}}{\pgfqpoint{1.724401in}{2.148894in}}{\pgfqpoint{1.718577in}{2.143070in}}%
\pgfpathcurveto{\pgfqpoint{1.712753in}{2.137246in}}{\pgfqpoint{1.709481in}{2.129346in}}{\pgfqpoint{1.709481in}{2.121110in}}%
\pgfpathcurveto{\pgfqpoint{1.709481in}{2.112873in}}{\pgfqpoint{1.712753in}{2.104973in}}{\pgfqpoint{1.718577in}{2.099149in}}%
\pgfpathcurveto{\pgfqpoint{1.724401in}{2.093325in}}{\pgfqpoint{1.732301in}{2.090053in}}{\pgfqpoint{1.740537in}{2.090053in}}%
\pgfpathclose%
\pgfusepath{stroke,fill}%
\end{pgfscope}%
\begin{pgfscope}%
\pgfpathrectangle{\pgfqpoint{0.100000in}{0.212622in}}{\pgfqpoint{3.696000in}{3.696000in}}%
\pgfusepath{clip}%
\pgfsetbuttcap%
\pgfsetroundjoin%
\definecolor{currentfill}{rgb}{0.121569,0.466667,0.705882}%
\pgfsetfillcolor{currentfill}%
\pgfsetfillopacity{0.331197}%
\pgfsetlinewidth{1.003750pt}%
\definecolor{currentstroke}{rgb}{0.121569,0.466667,0.705882}%
\pgfsetstrokecolor{currentstroke}%
\pgfsetstrokeopacity{0.331197}%
\pgfsetdash{}{0pt}%
\pgfpathmoveto{\pgfqpoint{1.758600in}{2.089091in}}%
\pgfpathcurveto{\pgfqpoint{1.766836in}{2.089091in}}{\pgfqpoint{1.774736in}{2.092363in}}{\pgfqpoint{1.780560in}{2.098187in}}%
\pgfpathcurveto{\pgfqpoint{1.786384in}{2.104011in}}{\pgfqpoint{1.789656in}{2.111911in}}{\pgfqpoint{1.789656in}{2.120147in}}%
\pgfpathcurveto{\pgfqpoint{1.789656in}{2.128384in}}{\pgfqpoint{1.786384in}{2.136284in}}{\pgfqpoint{1.780560in}{2.142108in}}%
\pgfpathcurveto{\pgfqpoint{1.774736in}{2.147931in}}{\pgfqpoint{1.766836in}{2.151204in}}{\pgfqpoint{1.758600in}{2.151204in}}%
\pgfpathcurveto{\pgfqpoint{1.750364in}{2.151204in}}{\pgfqpoint{1.742463in}{2.147931in}}{\pgfqpoint{1.736640in}{2.142108in}}%
\pgfpathcurveto{\pgfqpoint{1.730816in}{2.136284in}}{\pgfqpoint{1.727543in}{2.128384in}}{\pgfqpoint{1.727543in}{2.120147in}}%
\pgfpathcurveto{\pgfqpoint{1.727543in}{2.111911in}}{\pgfqpoint{1.730816in}{2.104011in}}{\pgfqpoint{1.736640in}{2.098187in}}%
\pgfpathcurveto{\pgfqpoint{1.742463in}{2.092363in}}{\pgfqpoint{1.750364in}{2.089091in}}{\pgfqpoint{1.758600in}{2.089091in}}%
\pgfpathclose%
\pgfusepath{stroke,fill}%
\end{pgfscope}%
\begin{pgfscope}%
\pgfpathrectangle{\pgfqpoint{0.100000in}{0.212622in}}{\pgfqpoint{3.696000in}{3.696000in}}%
\pgfusepath{clip}%
\pgfsetbuttcap%
\pgfsetroundjoin%
\definecolor{currentfill}{rgb}{0.121569,0.466667,0.705882}%
\pgfsetfillcolor{currentfill}%
\pgfsetfillopacity{0.334925}%
\pgfsetlinewidth{1.003750pt}%
\definecolor{currentstroke}{rgb}{0.121569,0.466667,0.705882}%
\pgfsetstrokecolor{currentstroke}%
\pgfsetstrokeopacity{0.334925}%
\pgfsetdash{}{0pt}%
\pgfpathmoveto{\pgfqpoint{1.777734in}{2.089025in}}%
\pgfpathcurveto{\pgfqpoint{1.785970in}{2.089025in}}{\pgfqpoint{1.793870in}{2.092297in}}{\pgfqpoint{1.799694in}{2.098121in}}%
\pgfpathcurveto{\pgfqpoint{1.805518in}{2.103945in}}{\pgfqpoint{1.808790in}{2.111845in}}{\pgfqpoint{1.808790in}{2.120081in}}%
\pgfpathcurveto{\pgfqpoint{1.808790in}{2.128317in}}{\pgfqpoint{1.805518in}{2.136217in}}{\pgfqpoint{1.799694in}{2.142041in}}%
\pgfpathcurveto{\pgfqpoint{1.793870in}{2.147865in}}{\pgfqpoint{1.785970in}{2.151138in}}{\pgfqpoint{1.777734in}{2.151138in}}%
\pgfpathcurveto{\pgfqpoint{1.769498in}{2.151138in}}{\pgfqpoint{1.761598in}{2.147865in}}{\pgfqpoint{1.755774in}{2.142041in}}%
\pgfpathcurveto{\pgfqpoint{1.749950in}{2.136217in}}{\pgfqpoint{1.746677in}{2.128317in}}{\pgfqpoint{1.746677in}{2.120081in}}%
\pgfpathcurveto{\pgfqpoint{1.746677in}{2.111845in}}{\pgfqpoint{1.749950in}{2.103945in}}{\pgfqpoint{1.755774in}{2.098121in}}%
\pgfpathcurveto{\pgfqpoint{1.761598in}{2.092297in}}{\pgfqpoint{1.769498in}{2.089025in}}{\pgfqpoint{1.777734in}{2.089025in}}%
\pgfpathclose%
\pgfusepath{stroke,fill}%
\end{pgfscope}%
\begin{pgfscope}%
\pgfpathrectangle{\pgfqpoint{0.100000in}{0.212622in}}{\pgfqpoint{3.696000in}{3.696000in}}%
\pgfusepath{clip}%
\pgfsetbuttcap%
\pgfsetroundjoin%
\definecolor{currentfill}{rgb}{0.121569,0.466667,0.705882}%
\pgfsetfillcolor{currentfill}%
\pgfsetfillopacity{0.338688}%
\pgfsetlinewidth{1.003750pt}%
\definecolor{currentstroke}{rgb}{0.121569,0.466667,0.705882}%
\pgfsetstrokecolor{currentstroke}%
\pgfsetstrokeopacity{0.338688}%
\pgfsetdash{}{0pt}%
\pgfpathmoveto{\pgfqpoint{1.797538in}{2.092567in}}%
\pgfpathcurveto{\pgfqpoint{1.805774in}{2.092567in}}{\pgfqpoint{1.813674in}{2.095840in}}{\pgfqpoint{1.819498in}{2.101664in}}%
\pgfpathcurveto{\pgfqpoint{1.825322in}{2.107487in}}{\pgfqpoint{1.828594in}{2.115388in}}{\pgfqpoint{1.828594in}{2.123624in}}%
\pgfpathcurveto{\pgfqpoint{1.828594in}{2.131860in}}{\pgfqpoint{1.825322in}{2.139760in}}{\pgfqpoint{1.819498in}{2.145584in}}%
\pgfpathcurveto{\pgfqpoint{1.813674in}{2.151408in}}{\pgfqpoint{1.805774in}{2.154680in}}{\pgfqpoint{1.797538in}{2.154680in}}%
\pgfpathcurveto{\pgfqpoint{1.789301in}{2.154680in}}{\pgfqpoint{1.781401in}{2.151408in}}{\pgfqpoint{1.775577in}{2.145584in}}%
\pgfpathcurveto{\pgfqpoint{1.769754in}{2.139760in}}{\pgfqpoint{1.766481in}{2.131860in}}{\pgfqpoint{1.766481in}{2.123624in}}%
\pgfpathcurveto{\pgfqpoint{1.766481in}{2.115388in}}{\pgfqpoint{1.769754in}{2.107487in}}{\pgfqpoint{1.775577in}{2.101664in}}%
\pgfpathcurveto{\pgfqpoint{1.781401in}{2.095840in}}{\pgfqpoint{1.789301in}{2.092567in}}{\pgfqpoint{1.797538in}{2.092567in}}%
\pgfpathclose%
\pgfusepath{stroke,fill}%
\end{pgfscope}%
\begin{pgfscope}%
\pgfpathrectangle{\pgfqpoint{0.100000in}{0.212622in}}{\pgfqpoint{3.696000in}{3.696000in}}%
\pgfusepath{clip}%
\pgfsetbuttcap%
\pgfsetroundjoin%
\definecolor{currentfill}{rgb}{0.121569,0.466667,0.705882}%
\pgfsetfillcolor{currentfill}%
\pgfsetfillopacity{0.342386}%
\pgfsetlinewidth{1.003750pt}%
\definecolor{currentstroke}{rgb}{0.121569,0.466667,0.705882}%
\pgfsetstrokecolor{currentstroke}%
\pgfsetstrokeopacity{0.342386}%
\pgfsetdash{}{0pt}%
\pgfpathmoveto{\pgfqpoint{1.816177in}{2.092718in}}%
\pgfpathcurveto{\pgfqpoint{1.824414in}{2.092718in}}{\pgfqpoint{1.832314in}{2.095991in}}{\pgfqpoint{1.838138in}{2.101815in}}%
\pgfpathcurveto{\pgfqpoint{1.843961in}{2.107639in}}{\pgfqpoint{1.847234in}{2.115539in}}{\pgfqpoint{1.847234in}{2.123775in}}%
\pgfpathcurveto{\pgfqpoint{1.847234in}{2.132011in}}{\pgfqpoint{1.843961in}{2.139911in}}{\pgfqpoint{1.838138in}{2.145735in}}%
\pgfpathcurveto{\pgfqpoint{1.832314in}{2.151559in}}{\pgfqpoint{1.824414in}{2.154831in}}{\pgfqpoint{1.816177in}{2.154831in}}%
\pgfpathcurveto{\pgfqpoint{1.807941in}{2.154831in}}{\pgfqpoint{1.800041in}{2.151559in}}{\pgfqpoint{1.794217in}{2.145735in}}%
\pgfpathcurveto{\pgfqpoint{1.788393in}{2.139911in}}{\pgfqpoint{1.785121in}{2.132011in}}{\pgfqpoint{1.785121in}{2.123775in}}%
\pgfpathcurveto{\pgfqpoint{1.785121in}{2.115539in}}{\pgfqpoint{1.788393in}{2.107639in}}{\pgfqpoint{1.794217in}{2.101815in}}%
\pgfpathcurveto{\pgfqpoint{1.800041in}{2.095991in}}{\pgfqpoint{1.807941in}{2.092718in}}{\pgfqpoint{1.816177in}{2.092718in}}%
\pgfpathclose%
\pgfusepath{stroke,fill}%
\end{pgfscope}%
\begin{pgfscope}%
\pgfpathrectangle{\pgfqpoint{0.100000in}{0.212622in}}{\pgfqpoint{3.696000in}{3.696000in}}%
\pgfusepath{clip}%
\pgfsetbuttcap%
\pgfsetroundjoin%
\definecolor{currentfill}{rgb}{0.121569,0.466667,0.705882}%
\pgfsetfillcolor{currentfill}%
\pgfsetfillopacity{0.345547}%
\pgfsetlinewidth{1.003750pt}%
\definecolor{currentstroke}{rgb}{0.121569,0.466667,0.705882}%
\pgfsetstrokecolor{currentstroke}%
\pgfsetstrokeopacity{0.345547}%
\pgfsetdash{}{0pt}%
\pgfpathmoveto{\pgfqpoint{1.834478in}{2.089975in}}%
\pgfpathcurveto{\pgfqpoint{1.842714in}{2.089975in}}{\pgfqpoint{1.850614in}{2.093248in}}{\pgfqpoint{1.856438in}{2.099072in}}%
\pgfpathcurveto{\pgfqpoint{1.862262in}{2.104896in}}{\pgfqpoint{1.865534in}{2.112796in}}{\pgfqpoint{1.865534in}{2.121032in}}%
\pgfpathcurveto{\pgfqpoint{1.865534in}{2.129268in}}{\pgfqpoint{1.862262in}{2.137168in}}{\pgfqpoint{1.856438in}{2.142992in}}%
\pgfpathcurveto{\pgfqpoint{1.850614in}{2.148816in}}{\pgfqpoint{1.842714in}{2.152088in}}{\pgfqpoint{1.834478in}{2.152088in}}%
\pgfpathcurveto{\pgfqpoint{1.826242in}{2.152088in}}{\pgfqpoint{1.818342in}{2.148816in}}{\pgfqpoint{1.812518in}{2.142992in}}%
\pgfpathcurveto{\pgfqpoint{1.806694in}{2.137168in}}{\pgfqpoint{1.803421in}{2.129268in}}{\pgfqpoint{1.803421in}{2.121032in}}%
\pgfpathcurveto{\pgfqpoint{1.803421in}{2.112796in}}{\pgfqpoint{1.806694in}{2.104896in}}{\pgfqpoint{1.812518in}{2.099072in}}%
\pgfpathcurveto{\pgfqpoint{1.818342in}{2.093248in}}{\pgfqpoint{1.826242in}{2.089975in}}{\pgfqpoint{1.834478in}{2.089975in}}%
\pgfpathclose%
\pgfusepath{stroke,fill}%
\end{pgfscope}%
\begin{pgfscope}%
\pgfpathrectangle{\pgfqpoint{0.100000in}{0.212622in}}{\pgfqpoint{3.696000in}{3.696000in}}%
\pgfusepath{clip}%
\pgfsetbuttcap%
\pgfsetroundjoin%
\definecolor{currentfill}{rgb}{0.121569,0.466667,0.705882}%
\pgfsetfillcolor{currentfill}%
\pgfsetfillopacity{0.348630}%
\pgfsetlinewidth{1.003750pt}%
\definecolor{currentstroke}{rgb}{0.121569,0.466667,0.705882}%
\pgfsetstrokecolor{currentstroke}%
\pgfsetstrokeopacity{0.348630}%
\pgfsetdash{}{0pt}%
\pgfpathmoveto{\pgfqpoint{1.852766in}{2.089206in}}%
\pgfpathcurveto{\pgfqpoint{1.861002in}{2.089206in}}{\pgfqpoint{1.868902in}{2.092478in}}{\pgfqpoint{1.874726in}{2.098302in}}%
\pgfpathcurveto{\pgfqpoint{1.880550in}{2.104126in}}{\pgfqpoint{1.883822in}{2.112026in}}{\pgfqpoint{1.883822in}{2.120262in}}%
\pgfpathcurveto{\pgfqpoint{1.883822in}{2.128499in}}{\pgfqpoint{1.880550in}{2.136399in}}{\pgfqpoint{1.874726in}{2.142223in}}%
\pgfpathcurveto{\pgfqpoint{1.868902in}{2.148047in}}{\pgfqpoint{1.861002in}{2.151319in}}{\pgfqpoint{1.852766in}{2.151319in}}%
\pgfpathcurveto{\pgfqpoint{1.844529in}{2.151319in}}{\pgfqpoint{1.836629in}{2.148047in}}{\pgfqpoint{1.830805in}{2.142223in}}%
\pgfpathcurveto{\pgfqpoint{1.824982in}{2.136399in}}{\pgfqpoint{1.821709in}{2.128499in}}{\pgfqpoint{1.821709in}{2.120262in}}%
\pgfpathcurveto{\pgfqpoint{1.821709in}{2.112026in}}{\pgfqpoint{1.824982in}{2.104126in}}{\pgfqpoint{1.830805in}{2.098302in}}%
\pgfpathcurveto{\pgfqpoint{1.836629in}{2.092478in}}{\pgfqpoint{1.844529in}{2.089206in}}{\pgfqpoint{1.852766in}{2.089206in}}%
\pgfpathclose%
\pgfusepath{stroke,fill}%
\end{pgfscope}%
\begin{pgfscope}%
\pgfpathrectangle{\pgfqpoint{0.100000in}{0.212622in}}{\pgfqpoint{3.696000in}{3.696000in}}%
\pgfusepath{clip}%
\pgfsetbuttcap%
\pgfsetroundjoin%
\definecolor{currentfill}{rgb}{0.121569,0.466667,0.705882}%
\pgfsetfillcolor{currentfill}%
\pgfsetfillopacity{0.354159}%
\pgfsetlinewidth{1.003750pt}%
\definecolor{currentstroke}{rgb}{0.121569,0.466667,0.705882}%
\pgfsetstrokecolor{currentstroke}%
\pgfsetstrokeopacity{0.354159}%
\pgfsetdash{}{0pt}%
\pgfpathmoveto{\pgfqpoint{1.886634in}{2.089288in}}%
\pgfpathcurveto{\pgfqpoint{1.894870in}{2.089288in}}{\pgfqpoint{1.902770in}{2.092560in}}{\pgfqpoint{1.908594in}{2.098384in}}%
\pgfpathcurveto{\pgfqpoint{1.914418in}{2.104208in}}{\pgfqpoint{1.917691in}{2.112108in}}{\pgfqpoint{1.917691in}{2.120344in}}%
\pgfpathcurveto{\pgfqpoint{1.917691in}{2.128580in}}{\pgfqpoint{1.914418in}{2.136481in}}{\pgfqpoint{1.908594in}{2.142304in}}%
\pgfpathcurveto{\pgfqpoint{1.902770in}{2.148128in}}{\pgfqpoint{1.894870in}{2.151401in}}{\pgfqpoint{1.886634in}{2.151401in}}%
\pgfpathcurveto{\pgfqpoint{1.878398in}{2.151401in}}{\pgfqpoint{1.870498in}{2.148128in}}{\pgfqpoint{1.864674in}{2.142304in}}%
\pgfpathcurveto{\pgfqpoint{1.858850in}{2.136481in}}{\pgfqpoint{1.855578in}{2.128580in}}{\pgfqpoint{1.855578in}{2.120344in}}%
\pgfpathcurveto{\pgfqpoint{1.855578in}{2.112108in}}{\pgfqpoint{1.858850in}{2.104208in}}{\pgfqpoint{1.864674in}{2.098384in}}%
\pgfpathcurveto{\pgfqpoint{1.870498in}{2.092560in}}{\pgfqpoint{1.878398in}{2.089288in}}{\pgfqpoint{1.886634in}{2.089288in}}%
\pgfpathclose%
\pgfusepath{stroke,fill}%
\end{pgfscope}%
\begin{pgfscope}%
\pgfpathrectangle{\pgfqpoint{0.100000in}{0.212622in}}{\pgfqpoint{3.696000in}{3.696000in}}%
\pgfusepath{clip}%
\pgfsetbuttcap%
\pgfsetroundjoin%
\definecolor{currentfill}{rgb}{0.121569,0.466667,0.705882}%
\pgfsetfillcolor{currentfill}%
\pgfsetfillopacity{0.359483}%
\pgfsetlinewidth{1.003750pt}%
\definecolor{currentstroke}{rgb}{0.121569,0.466667,0.705882}%
\pgfsetstrokecolor{currentstroke}%
\pgfsetstrokeopacity{0.359483}%
\pgfsetdash{}{0pt}%
\pgfpathmoveto{\pgfqpoint{1.919999in}{2.091617in}}%
\pgfpathcurveto{\pgfqpoint{1.928235in}{2.091617in}}{\pgfqpoint{1.936135in}{2.094889in}}{\pgfqpoint{1.941959in}{2.100713in}}%
\pgfpathcurveto{\pgfqpoint{1.947783in}{2.106537in}}{\pgfqpoint{1.951055in}{2.114437in}}{\pgfqpoint{1.951055in}{2.122673in}}%
\pgfpathcurveto{\pgfqpoint{1.951055in}{2.130909in}}{\pgfqpoint{1.947783in}{2.138809in}}{\pgfqpoint{1.941959in}{2.144633in}}%
\pgfpathcurveto{\pgfqpoint{1.936135in}{2.150457in}}{\pgfqpoint{1.928235in}{2.153730in}}{\pgfqpoint{1.919999in}{2.153730in}}%
\pgfpathcurveto{\pgfqpoint{1.911763in}{2.153730in}}{\pgfqpoint{1.903863in}{2.150457in}}{\pgfqpoint{1.898039in}{2.144633in}}%
\pgfpathcurveto{\pgfqpoint{1.892215in}{2.138809in}}{\pgfqpoint{1.888942in}{2.130909in}}{\pgfqpoint{1.888942in}{2.122673in}}%
\pgfpathcurveto{\pgfqpoint{1.888942in}{2.114437in}}{\pgfqpoint{1.892215in}{2.106537in}}{\pgfqpoint{1.898039in}{2.100713in}}%
\pgfpathcurveto{\pgfqpoint{1.903863in}{2.094889in}}{\pgfqpoint{1.911763in}{2.091617in}}{\pgfqpoint{1.919999in}{2.091617in}}%
\pgfpathclose%
\pgfusepath{stroke,fill}%
\end{pgfscope}%
\begin{pgfscope}%
\pgfpathrectangle{\pgfqpoint{0.100000in}{0.212622in}}{\pgfqpoint{3.696000in}{3.696000in}}%
\pgfusepath{clip}%
\pgfsetbuttcap%
\pgfsetroundjoin%
\definecolor{currentfill}{rgb}{0.121569,0.466667,0.705882}%
\pgfsetfillcolor{currentfill}%
\pgfsetfillopacity{0.364416}%
\pgfsetlinewidth{1.003750pt}%
\definecolor{currentstroke}{rgb}{0.121569,0.466667,0.705882}%
\pgfsetstrokecolor{currentstroke}%
\pgfsetstrokeopacity{0.364416}%
\pgfsetdash{}{0pt}%
\pgfpathmoveto{\pgfqpoint{1.951877in}{2.095662in}}%
\pgfpathcurveto{\pgfqpoint{1.960114in}{2.095662in}}{\pgfqpoint{1.968014in}{2.098935in}}{\pgfqpoint{1.973838in}{2.104759in}}%
\pgfpathcurveto{\pgfqpoint{1.979661in}{2.110583in}}{\pgfqpoint{1.982934in}{2.118483in}}{\pgfqpoint{1.982934in}{2.126719in}}%
\pgfpathcurveto{\pgfqpoint{1.982934in}{2.134955in}}{\pgfqpoint{1.979661in}{2.142855in}}{\pgfqpoint{1.973838in}{2.148679in}}%
\pgfpathcurveto{\pgfqpoint{1.968014in}{2.154503in}}{\pgfqpoint{1.960114in}{2.157775in}}{\pgfqpoint{1.951877in}{2.157775in}}%
\pgfpathcurveto{\pgfqpoint{1.943641in}{2.157775in}}{\pgfqpoint{1.935741in}{2.154503in}}{\pgfqpoint{1.929917in}{2.148679in}}%
\pgfpathcurveto{\pgfqpoint{1.924093in}{2.142855in}}{\pgfqpoint{1.920821in}{2.134955in}}{\pgfqpoint{1.920821in}{2.126719in}}%
\pgfpathcurveto{\pgfqpoint{1.920821in}{2.118483in}}{\pgfqpoint{1.924093in}{2.110583in}}{\pgfqpoint{1.929917in}{2.104759in}}%
\pgfpathcurveto{\pgfqpoint{1.935741in}{2.098935in}}{\pgfqpoint{1.943641in}{2.095662in}}{\pgfqpoint{1.951877in}{2.095662in}}%
\pgfpathclose%
\pgfusepath{stroke,fill}%
\end{pgfscope}%
\begin{pgfscope}%
\pgfpathrectangle{\pgfqpoint{0.100000in}{0.212622in}}{\pgfqpoint{3.696000in}{3.696000in}}%
\pgfusepath{clip}%
\pgfsetbuttcap%
\pgfsetroundjoin%
\definecolor{currentfill}{rgb}{0.121569,0.466667,0.705882}%
\pgfsetfillcolor{currentfill}%
\pgfsetfillopacity{0.368968}%
\pgfsetlinewidth{1.003750pt}%
\definecolor{currentstroke}{rgb}{0.121569,0.466667,0.705882}%
\pgfsetstrokecolor{currentstroke}%
\pgfsetstrokeopacity{0.368968}%
\pgfsetdash{}{0pt}%
\pgfpathmoveto{\pgfqpoint{1.982962in}{2.096584in}}%
\pgfpathcurveto{\pgfqpoint{1.991199in}{2.096584in}}{\pgfqpoint{1.999099in}{2.099856in}}{\pgfqpoint{2.004923in}{2.105680in}}%
\pgfpathcurveto{\pgfqpoint{2.010747in}{2.111504in}}{\pgfqpoint{2.014019in}{2.119404in}}{\pgfqpoint{2.014019in}{2.127640in}}%
\pgfpathcurveto{\pgfqpoint{2.014019in}{2.135876in}}{\pgfqpoint{2.010747in}{2.143776in}}{\pgfqpoint{2.004923in}{2.149600in}}%
\pgfpathcurveto{\pgfqpoint{1.999099in}{2.155424in}}{\pgfqpoint{1.991199in}{2.158697in}}{\pgfqpoint{1.982962in}{2.158697in}}%
\pgfpathcurveto{\pgfqpoint{1.974726in}{2.158697in}}{\pgfqpoint{1.966826in}{2.155424in}}{\pgfqpoint{1.961002in}{2.149600in}}%
\pgfpathcurveto{\pgfqpoint{1.955178in}{2.143776in}}{\pgfqpoint{1.951906in}{2.135876in}}{\pgfqpoint{1.951906in}{2.127640in}}%
\pgfpathcurveto{\pgfqpoint{1.951906in}{2.119404in}}{\pgfqpoint{1.955178in}{2.111504in}}{\pgfqpoint{1.961002in}{2.105680in}}%
\pgfpathcurveto{\pgfqpoint{1.966826in}{2.099856in}}{\pgfqpoint{1.974726in}{2.096584in}}{\pgfqpoint{1.982962in}{2.096584in}}%
\pgfpathclose%
\pgfusepath{stroke,fill}%
\end{pgfscope}%
\begin{pgfscope}%
\pgfpathrectangle{\pgfqpoint{0.100000in}{0.212622in}}{\pgfqpoint{3.696000in}{3.696000in}}%
\pgfusepath{clip}%
\pgfsetbuttcap%
\pgfsetroundjoin%
\definecolor{currentfill}{rgb}{0.121569,0.466667,0.705882}%
\pgfsetfillcolor{currentfill}%
\pgfsetfillopacity{0.373283}%
\pgfsetlinewidth{1.003750pt}%
\definecolor{currentstroke}{rgb}{0.121569,0.466667,0.705882}%
\pgfsetstrokecolor{currentstroke}%
\pgfsetstrokeopacity{0.373283}%
\pgfsetdash{}{0pt}%
\pgfpathmoveto{\pgfqpoint{2.012155in}{2.094883in}}%
\pgfpathcurveto{\pgfqpoint{2.020392in}{2.094883in}}{\pgfqpoint{2.028292in}{2.098155in}}{\pgfqpoint{2.034116in}{2.103979in}}%
\pgfpathcurveto{\pgfqpoint{2.039940in}{2.109803in}}{\pgfqpoint{2.043212in}{2.117703in}}{\pgfqpoint{2.043212in}{2.125939in}}%
\pgfpathcurveto{\pgfqpoint{2.043212in}{2.134176in}}{\pgfqpoint{2.039940in}{2.142076in}}{\pgfqpoint{2.034116in}{2.147900in}}%
\pgfpathcurveto{\pgfqpoint{2.028292in}{2.153724in}}{\pgfqpoint{2.020392in}{2.156996in}}{\pgfqpoint{2.012155in}{2.156996in}}%
\pgfpathcurveto{\pgfqpoint{2.003919in}{2.156996in}}{\pgfqpoint{1.996019in}{2.153724in}}{\pgfqpoint{1.990195in}{2.147900in}}%
\pgfpathcurveto{\pgfqpoint{1.984371in}{2.142076in}}{\pgfqpoint{1.981099in}{2.134176in}}{\pgfqpoint{1.981099in}{2.125939in}}%
\pgfpathcurveto{\pgfqpoint{1.981099in}{2.117703in}}{\pgfqpoint{1.984371in}{2.109803in}}{\pgfqpoint{1.990195in}{2.103979in}}%
\pgfpathcurveto{\pgfqpoint{1.996019in}{2.098155in}}{\pgfqpoint{2.003919in}{2.094883in}}{\pgfqpoint{2.012155in}{2.094883in}}%
\pgfpathclose%
\pgfusepath{stroke,fill}%
\end{pgfscope}%
\begin{pgfscope}%
\pgfpathrectangle{\pgfqpoint{0.100000in}{0.212622in}}{\pgfqpoint{3.696000in}{3.696000in}}%
\pgfusepath{clip}%
\pgfsetbuttcap%
\pgfsetroundjoin%
\definecolor{currentfill}{rgb}{0.121569,0.466667,0.705882}%
\pgfsetfillcolor{currentfill}%
\pgfsetfillopacity{0.377448}%
\pgfsetlinewidth{1.003750pt}%
\definecolor{currentstroke}{rgb}{0.121569,0.466667,0.705882}%
\pgfsetstrokecolor{currentstroke}%
\pgfsetstrokeopacity{0.377448}%
\pgfsetdash{}{0pt}%
\pgfpathmoveto{\pgfqpoint{2.039496in}{2.091640in}}%
\pgfpathcurveto{\pgfqpoint{2.047733in}{2.091640in}}{\pgfqpoint{2.055633in}{2.094912in}}{\pgfqpoint{2.061457in}{2.100736in}}%
\pgfpathcurveto{\pgfqpoint{2.067281in}{2.106560in}}{\pgfqpoint{2.070553in}{2.114460in}}{\pgfqpoint{2.070553in}{2.122696in}}%
\pgfpathcurveto{\pgfqpoint{2.070553in}{2.130933in}}{\pgfqpoint{2.067281in}{2.138833in}}{\pgfqpoint{2.061457in}{2.144657in}}%
\pgfpathcurveto{\pgfqpoint{2.055633in}{2.150481in}}{\pgfqpoint{2.047733in}{2.153753in}}{\pgfqpoint{2.039496in}{2.153753in}}%
\pgfpathcurveto{\pgfqpoint{2.031260in}{2.153753in}}{\pgfqpoint{2.023360in}{2.150481in}}{\pgfqpoint{2.017536in}{2.144657in}}%
\pgfpathcurveto{\pgfqpoint{2.011712in}{2.138833in}}{\pgfqpoint{2.008440in}{2.130933in}}{\pgfqpoint{2.008440in}{2.122696in}}%
\pgfpathcurveto{\pgfqpoint{2.008440in}{2.114460in}}{\pgfqpoint{2.011712in}{2.106560in}}{\pgfqpoint{2.017536in}{2.100736in}}%
\pgfpathcurveto{\pgfqpoint{2.023360in}{2.094912in}}{\pgfqpoint{2.031260in}{2.091640in}}{\pgfqpoint{2.039496in}{2.091640in}}%
\pgfpathclose%
\pgfusepath{stroke,fill}%
\end{pgfscope}%
\begin{pgfscope}%
\pgfpathrectangle{\pgfqpoint{0.100000in}{0.212622in}}{\pgfqpoint{3.696000in}{3.696000in}}%
\pgfusepath{clip}%
\pgfsetbuttcap%
\pgfsetroundjoin%
\definecolor{currentfill}{rgb}{0.121569,0.466667,0.705882}%
\pgfsetfillcolor{currentfill}%
\pgfsetfillopacity{0.381558}%
\pgfsetlinewidth{1.003750pt}%
\definecolor{currentstroke}{rgb}{0.121569,0.466667,0.705882}%
\pgfsetstrokecolor{currentstroke}%
\pgfsetstrokeopacity{0.381558}%
\pgfsetdash{}{0pt}%
\pgfpathmoveto{\pgfqpoint{2.065705in}{2.087471in}}%
\pgfpathcurveto{\pgfqpoint{2.073941in}{2.087471in}}{\pgfqpoint{2.081842in}{2.090743in}}{\pgfqpoint{2.087665in}{2.096567in}}%
\pgfpathcurveto{\pgfqpoint{2.093489in}{2.102391in}}{\pgfqpoint{2.096762in}{2.110291in}}{\pgfqpoint{2.096762in}{2.118527in}}%
\pgfpathcurveto{\pgfqpoint{2.096762in}{2.126763in}}{\pgfqpoint{2.093489in}{2.134663in}}{\pgfqpoint{2.087665in}{2.140487in}}%
\pgfpathcurveto{\pgfqpoint{2.081842in}{2.146311in}}{\pgfqpoint{2.073941in}{2.149584in}}{\pgfqpoint{2.065705in}{2.149584in}}%
\pgfpathcurveto{\pgfqpoint{2.057469in}{2.149584in}}{\pgfqpoint{2.049569in}{2.146311in}}{\pgfqpoint{2.043745in}{2.140487in}}%
\pgfpathcurveto{\pgfqpoint{2.037921in}{2.134663in}}{\pgfqpoint{2.034649in}{2.126763in}}{\pgfqpoint{2.034649in}{2.118527in}}%
\pgfpathcurveto{\pgfqpoint{2.034649in}{2.110291in}}{\pgfqpoint{2.037921in}{2.102391in}}{\pgfqpoint{2.043745in}{2.096567in}}%
\pgfpathcurveto{\pgfqpoint{2.049569in}{2.090743in}}{\pgfqpoint{2.057469in}{2.087471in}}{\pgfqpoint{2.065705in}{2.087471in}}%
\pgfpathclose%
\pgfusepath{stroke,fill}%
\end{pgfscope}%
\begin{pgfscope}%
\pgfpathrectangle{\pgfqpoint{0.100000in}{0.212622in}}{\pgfqpoint{3.696000in}{3.696000in}}%
\pgfusepath{clip}%
\pgfsetbuttcap%
\pgfsetroundjoin%
\definecolor{currentfill}{rgb}{0.121569,0.466667,0.705882}%
\pgfsetfillcolor{currentfill}%
\pgfsetfillopacity{0.385739}%
\pgfsetlinewidth{1.003750pt}%
\definecolor{currentstroke}{rgb}{0.121569,0.466667,0.705882}%
\pgfsetstrokecolor{currentstroke}%
\pgfsetstrokeopacity{0.385739}%
\pgfsetdash{}{0pt}%
\pgfpathmoveto{\pgfqpoint{2.091436in}{2.084418in}}%
\pgfpathcurveto{\pgfqpoint{2.099672in}{2.084418in}}{\pgfqpoint{2.107572in}{2.087690in}}{\pgfqpoint{2.113396in}{2.093514in}}%
\pgfpathcurveto{\pgfqpoint{2.119220in}{2.099338in}}{\pgfqpoint{2.122493in}{2.107238in}}{\pgfqpoint{2.122493in}{2.115475in}}%
\pgfpathcurveto{\pgfqpoint{2.122493in}{2.123711in}}{\pgfqpoint{2.119220in}{2.131611in}}{\pgfqpoint{2.113396in}{2.137435in}}%
\pgfpathcurveto{\pgfqpoint{2.107572in}{2.143259in}}{\pgfqpoint{2.099672in}{2.146531in}}{\pgfqpoint{2.091436in}{2.146531in}}%
\pgfpathcurveto{\pgfqpoint{2.083200in}{2.146531in}}{\pgfqpoint{2.075300in}{2.143259in}}{\pgfqpoint{2.069476in}{2.137435in}}%
\pgfpathcurveto{\pgfqpoint{2.063652in}{2.131611in}}{\pgfqpoint{2.060380in}{2.123711in}}{\pgfqpoint{2.060380in}{2.115475in}}%
\pgfpathcurveto{\pgfqpoint{2.060380in}{2.107238in}}{\pgfqpoint{2.063652in}{2.099338in}}{\pgfqpoint{2.069476in}{2.093514in}}%
\pgfpathcurveto{\pgfqpoint{2.075300in}{2.087690in}}{\pgfqpoint{2.083200in}{2.084418in}}{\pgfqpoint{2.091436in}{2.084418in}}%
\pgfpathclose%
\pgfusepath{stroke,fill}%
\end{pgfscope}%
\begin{pgfscope}%
\pgfpathrectangle{\pgfqpoint{0.100000in}{0.212622in}}{\pgfqpoint{3.696000in}{3.696000in}}%
\pgfusepath{clip}%
\pgfsetbuttcap%
\pgfsetroundjoin%
\definecolor{currentfill}{rgb}{0.121569,0.466667,0.705882}%
\pgfsetfillcolor{currentfill}%
\pgfsetfillopacity{0.389832}%
\pgfsetlinewidth{1.003750pt}%
\definecolor{currentstroke}{rgb}{0.121569,0.466667,0.705882}%
\pgfsetstrokecolor{currentstroke}%
\pgfsetstrokeopacity{0.389832}%
\pgfsetdash{}{0pt}%
\pgfpathmoveto{\pgfqpoint{2.116355in}{2.081868in}}%
\pgfpathcurveto{\pgfqpoint{2.124591in}{2.081868in}}{\pgfqpoint{2.132491in}{2.085141in}}{\pgfqpoint{2.138315in}{2.090965in}}%
\pgfpathcurveto{\pgfqpoint{2.144139in}{2.096789in}}{\pgfqpoint{2.147412in}{2.104689in}}{\pgfqpoint{2.147412in}{2.112925in}}%
\pgfpathcurveto{\pgfqpoint{2.147412in}{2.121161in}}{\pgfqpoint{2.144139in}{2.129061in}}{\pgfqpoint{2.138315in}{2.134885in}}%
\pgfpathcurveto{\pgfqpoint{2.132491in}{2.140709in}}{\pgfqpoint{2.124591in}{2.143981in}}{\pgfqpoint{2.116355in}{2.143981in}}%
\pgfpathcurveto{\pgfqpoint{2.108119in}{2.143981in}}{\pgfqpoint{2.100219in}{2.140709in}}{\pgfqpoint{2.094395in}{2.134885in}}%
\pgfpathcurveto{\pgfqpoint{2.088571in}{2.129061in}}{\pgfqpoint{2.085299in}{2.121161in}}{\pgfqpoint{2.085299in}{2.112925in}}%
\pgfpathcurveto{\pgfqpoint{2.085299in}{2.104689in}}{\pgfqpoint{2.088571in}{2.096789in}}{\pgfqpoint{2.094395in}{2.090965in}}%
\pgfpathcurveto{\pgfqpoint{2.100219in}{2.085141in}}{\pgfqpoint{2.108119in}{2.081868in}}{\pgfqpoint{2.116355in}{2.081868in}}%
\pgfpathclose%
\pgfusepath{stroke,fill}%
\end{pgfscope}%
\begin{pgfscope}%
\pgfpathrectangle{\pgfqpoint{0.100000in}{0.212622in}}{\pgfqpoint{3.696000in}{3.696000in}}%
\pgfusepath{clip}%
\pgfsetbuttcap%
\pgfsetroundjoin%
\definecolor{currentfill}{rgb}{0.121569,0.466667,0.705882}%
\pgfsetfillcolor{currentfill}%
\pgfsetfillopacity{0.393576}%
\pgfsetlinewidth{1.003750pt}%
\definecolor{currentstroke}{rgb}{0.121569,0.466667,0.705882}%
\pgfsetstrokecolor{currentstroke}%
\pgfsetstrokeopacity{0.393576}%
\pgfsetdash{}{0pt}%
\pgfpathmoveto{\pgfqpoint{2.139670in}{2.081249in}}%
\pgfpathcurveto{\pgfqpoint{2.147907in}{2.081249in}}{\pgfqpoint{2.155807in}{2.084522in}}{\pgfqpoint{2.161631in}{2.090346in}}%
\pgfpathcurveto{\pgfqpoint{2.167455in}{2.096169in}}{\pgfqpoint{2.170727in}{2.104069in}}{\pgfqpoint{2.170727in}{2.112306in}}%
\pgfpathcurveto{\pgfqpoint{2.170727in}{2.120542in}}{\pgfqpoint{2.167455in}{2.128442in}}{\pgfqpoint{2.161631in}{2.134266in}}%
\pgfpathcurveto{\pgfqpoint{2.155807in}{2.140090in}}{\pgfqpoint{2.147907in}{2.143362in}}{\pgfqpoint{2.139670in}{2.143362in}}%
\pgfpathcurveto{\pgfqpoint{2.131434in}{2.143362in}}{\pgfqpoint{2.123534in}{2.140090in}}{\pgfqpoint{2.117710in}{2.134266in}}%
\pgfpathcurveto{\pgfqpoint{2.111886in}{2.128442in}}{\pgfqpoint{2.108614in}{2.120542in}}{\pgfqpoint{2.108614in}{2.112306in}}%
\pgfpathcurveto{\pgfqpoint{2.108614in}{2.104069in}}{\pgfqpoint{2.111886in}{2.096169in}}{\pgfqpoint{2.117710in}{2.090346in}}%
\pgfpathcurveto{\pgfqpoint{2.123534in}{2.084522in}}{\pgfqpoint{2.131434in}{2.081249in}}{\pgfqpoint{2.139670in}{2.081249in}}%
\pgfpathclose%
\pgfusepath{stroke,fill}%
\end{pgfscope}%
\begin{pgfscope}%
\pgfpathrectangle{\pgfqpoint{0.100000in}{0.212622in}}{\pgfqpoint{3.696000in}{3.696000in}}%
\pgfusepath{clip}%
\pgfsetbuttcap%
\pgfsetroundjoin%
\definecolor{currentfill}{rgb}{0.121569,0.466667,0.705882}%
\pgfsetfillcolor{currentfill}%
\pgfsetfillopacity{0.396853}%
\pgfsetlinewidth{1.003750pt}%
\definecolor{currentstroke}{rgb}{0.121569,0.466667,0.705882}%
\pgfsetstrokecolor{currentstroke}%
\pgfsetstrokeopacity{0.396853}%
\pgfsetdash{}{0pt}%
\pgfpathmoveto{\pgfqpoint{2.160860in}{2.081162in}}%
\pgfpathcurveto{\pgfqpoint{2.169096in}{2.081162in}}{\pgfqpoint{2.176996in}{2.084435in}}{\pgfqpoint{2.182820in}{2.090259in}}%
\pgfpathcurveto{\pgfqpoint{2.188644in}{2.096083in}}{\pgfqpoint{2.191916in}{2.103983in}}{\pgfqpoint{2.191916in}{2.112219in}}%
\pgfpathcurveto{\pgfqpoint{2.191916in}{2.120455in}}{\pgfqpoint{2.188644in}{2.128355in}}{\pgfqpoint{2.182820in}{2.134179in}}%
\pgfpathcurveto{\pgfqpoint{2.176996in}{2.140003in}}{\pgfqpoint{2.169096in}{2.143275in}}{\pgfqpoint{2.160860in}{2.143275in}}%
\pgfpathcurveto{\pgfqpoint{2.152623in}{2.143275in}}{\pgfqpoint{2.144723in}{2.140003in}}{\pgfqpoint{2.138899in}{2.134179in}}%
\pgfpathcurveto{\pgfqpoint{2.133075in}{2.128355in}}{\pgfqpoint{2.129803in}{2.120455in}}{\pgfqpoint{2.129803in}{2.112219in}}%
\pgfpathcurveto{\pgfqpoint{2.129803in}{2.103983in}}{\pgfqpoint{2.133075in}{2.096083in}}{\pgfqpoint{2.138899in}{2.090259in}}%
\pgfpathcurveto{\pgfqpoint{2.144723in}{2.084435in}}{\pgfqpoint{2.152623in}{2.081162in}}{\pgfqpoint{2.160860in}{2.081162in}}%
\pgfpathclose%
\pgfusepath{stroke,fill}%
\end{pgfscope}%
\begin{pgfscope}%
\pgfpathrectangle{\pgfqpoint{0.100000in}{0.212622in}}{\pgfqpoint{3.696000in}{3.696000in}}%
\pgfusepath{clip}%
\pgfsetbuttcap%
\pgfsetroundjoin%
\definecolor{currentfill}{rgb}{0.121569,0.466667,0.705882}%
\pgfsetfillcolor{currentfill}%
\pgfsetfillopacity{0.402769}%
\pgfsetlinewidth{1.003750pt}%
\definecolor{currentstroke}{rgb}{0.121569,0.466667,0.705882}%
\pgfsetstrokecolor{currentstroke}%
\pgfsetstrokeopacity{0.402769}%
\pgfsetdash{}{0pt}%
\pgfpathmoveto{\pgfqpoint{2.200924in}{2.086416in}}%
\pgfpathcurveto{\pgfqpoint{2.209160in}{2.086416in}}{\pgfqpoint{2.217060in}{2.089688in}}{\pgfqpoint{2.222884in}{2.095512in}}%
\pgfpathcurveto{\pgfqpoint{2.228708in}{2.101336in}}{\pgfqpoint{2.231981in}{2.109236in}}{\pgfqpoint{2.231981in}{2.117472in}}%
\pgfpathcurveto{\pgfqpoint{2.231981in}{2.125709in}}{\pgfqpoint{2.228708in}{2.133609in}}{\pgfqpoint{2.222884in}{2.139433in}}%
\pgfpathcurveto{\pgfqpoint{2.217060in}{2.145257in}}{\pgfqpoint{2.209160in}{2.148529in}}{\pgfqpoint{2.200924in}{2.148529in}}%
\pgfpathcurveto{\pgfqpoint{2.192688in}{2.148529in}}{\pgfqpoint{2.184788in}{2.145257in}}{\pgfqpoint{2.178964in}{2.139433in}}%
\pgfpathcurveto{\pgfqpoint{2.173140in}{2.133609in}}{\pgfqpoint{2.169868in}{2.125709in}}{\pgfqpoint{2.169868in}{2.117472in}}%
\pgfpathcurveto{\pgfqpoint{2.169868in}{2.109236in}}{\pgfqpoint{2.173140in}{2.101336in}}{\pgfqpoint{2.178964in}{2.095512in}}%
\pgfpathcurveto{\pgfqpoint{2.184788in}{2.089688in}}{\pgfqpoint{2.192688in}{2.086416in}}{\pgfqpoint{2.200924in}{2.086416in}}%
\pgfpathclose%
\pgfusepath{stroke,fill}%
\end{pgfscope}%
\begin{pgfscope}%
\pgfpathrectangle{\pgfqpoint{0.100000in}{0.212622in}}{\pgfqpoint{3.696000in}{3.696000in}}%
\pgfusepath{clip}%
\pgfsetbuttcap%
\pgfsetroundjoin%
\definecolor{currentfill}{rgb}{0.121569,0.466667,0.705882}%
\pgfsetfillcolor{currentfill}%
\pgfsetfillopacity{0.408511}%
\pgfsetlinewidth{1.003750pt}%
\definecolor{currentstroke}{rgb}{0.121569,0.466667,0.705882}%
\pgfsetstrokecolor{currentstroke}%
\pgfsetstrokeopacity{0.408511}%
\pgfsetdash{}{0pt}%
\pgfpathmoveto{\pgfqpoint{2.239552in}{2.089078in}}%
\pgfpathcurveto{\pgfqpoint{2.247789in}{2.089078in}}{\pgfqpoint{2.255689in}{2.092350in}}{\pgfqpoint{2.261513in}{2.098174in}}%
\pgfpathcurveto{\pgfqpoint{2.267337in}{2.103998in}}{\pgfqpoint{2.270609in}{2.111898in}}{\pgfqpoint{2.270609in}{2.120134in}}%
\pgfpathcurveto{\pgfqpoint{2.270609in}{2.128370in}}{\pgfqpoint{2.267337in}{2.136271in}}{\pgfqpoint{2.261513in}{2.142094in}}%
\pgfpathcurveto{\pgfqpoint{2.255689in}{2.147918in}}{\pgfqpoint{2.247789in}{2.151191in}}{\pgfqpoint{2.239552in}{2.151191in}}%
\pgfpathcurveto{\pgfqpoint{2.231316in}{2.151191in}}{\pgfqpoint{2.223416in}{2.147918in}}{\pgfqpoint{2.217592in}{2.142094in}}%
\pgfpathcurveto{\pgfqpoint{2.211768in}{2.136271in}}{\pgfqpoint{2.208496in}{2.128370in}}{\pgfqpoint{2.208496in}{2.120134in}}%
\pgfpathcurveto{\pgfqpoint{2.208496in}{2.111898in}}{\pgfqpoint{2.211768in}{2.103998in}}{\pgfqpoint{2.217592in}{2.098174in}}%
\pgfpathcurveto{\pgfqpoint{2.223416in}{2.092350in}}{\pgfqpoint{2.231316in}{2.089078in}}{\pgfqpoint{2.239552in}{2.089078in}}%
\pgfpathclose%
\pgfusepath{stroke,fill}%
\end{pgfscope}%
\begin{pgfscope}%
\pgfpathrectangle{\pgfqpoint{0.100000in}{0.212622in}}{\pgfqpoint{3.696000in}{3.696000in}}%
\pgfusepath{clip}%
\pgfsetbuttcap%
\pgfsetroundjoin%
\definecolor{currentfill}{rgb}{0.121569,0.466667,0.705882}%
\pgfsetfillcolor{currentfill}%
\pgfsetfillopacity{0.414175}%
\pgfsetlinewidth{1.003750pt}%
\definecolor{currentstroke}{rgb}{0.121569,0.466667,0.705882}%
\pgfsetstrokecolor{currentstroke}%
\pgfsetstrokeopacity{0.414175}%
\pgfsetdash{}{0pt}%
\pgfpathmoveto{\pgfqpoint{2.276116in}{2.089818in}}%
\pgfpathcurveto{\pgfqpoint{2.284352in}{2.089818in}}{\pgfqpoint{2.292252in}{2.093090in}}{\pgfqpoint{2.298076in}{2.098914in}}%
\pgfpathcurveto{\pgfqpoint{2.303900in}{2.104738in}}{\pgfqpoint{2.307172in}{2.112638in}}{\pgfqpoint{2.307172in}{2.120874in}}%
\pgfpathcurveto{\pgfqpoint{2.307172in}{2.129111in}}{\pgfqpoint{2.303900in}{2.137011in}}{\pgfqpoint{2.298076in}{2.142835in}}%
\pgfpathcurveto{\pgfqpoint{2.292252in}{2.148659in}}{\pgfqpoint{2.284352in}{2.151931in}}{\pgfqpoint{2.276116in}{2.151931in}}%
\pgfpathcurveto{\pgfqpoint{2.267879in}{2.151931in}}{\pgfqpoint{2.259979in}{2.148659in}}{\pgfqpoint{2.254155in}{2.142835in}}%
\pgfpathcurveto{\pgfqpoint{2.248331in}{2.137011in}}{\pgfqpoint{2.245059in}{2.129111in}}{\pgfqpoint{2.245059in}{2.120874in}}%
\pgfpathcurveto{\pgfqpoint{2.245059in}{2.112638in}}{\pgfqpoint{2.248331in}{2.104738in}}{\pgfqpoint{2.254155in}{2.098914in}}%
\pgfpathcurveto{\pgfqpoint{2.259979in}{2.093090in}}{\pgfqpoint{2.267879in}{2.089818in}}{\pgfqpoint{2.276116in}{2.089818in}}%
\pgfpathclose%
\pgfusepath{stroke,fill}%
\end{pgfscope}%
\begin{pgfscope}%
\pgfpathrectangle{\pgfqpoint{0.100000in}{0.212622in}}{\pgfqpoint{3.696000in}{3.696000in}}%
\pgfusepath{clip}%
\pgfsetbuttcap%
\pgfsetroundjoin%
\definecolor{currentfill}{rgb}{0.121569,0.466667,0.705882}%
\pgfsetfillcolor{currentfill}%
\pgfsetfillopacity{0.419886}%
\pgfsetlinewidth{1.003750pt}%
\definecolor{currentstroke}{rgb}{0.121569,0.466667,0.705882}%
\pgfsetstrokecolor{currentstroke}%
\pgfsetstrokeopacity{0.419886}%
\pgfsetdash{}{0pt}%
\pgfpathmoveto{\pgfqpoint{2.311148in}{2.089247in}}%
\pgfpathcurveto{\pgfqpoint{2.319384in}{2.089247in}}{\pgfqpoint{2.327284in}{2.092519in}}{\pgfqpoint{2.333108in}{2.098343in}}%
\pgfpathcurveto{\pgfqpoint{2.338932in}{2.104167in}}{\pgfqpoint{2.342205in}{2.112067in}}{\pgfqpoint{2.342205in}{2.120303in}}%
\pgfpathcurveto{\pgfqpoint{2.342205in}{2.128539in}}{\pgfqpoint{2.338932in}{2.136440in}}{\pgfqpoint{2.333108in}{2.142263in}}%
\pgfpathcurveto{\pgfqpoint{2.327284in}{2.148087in}}{\pgfqpoint{2.319384in}{2.151360in}}{\pgfqpoint{2.311148in}{2.151360in}}%
\pgfpathcurveto{\pgfqpoint{2.302912in}{2.151360in}}{\pgfqpoint{2.295012in}{2.148087in}}{\pgfqpoint{2.289188in}{2.142263in}}%
\pgfpathcurveto{\pgfqpoint{2.283364in}{2.136440in}}{\pgfqpoint{2.280092in}{2.128539in}}{\pgfqpoint{2.280092in}{2.120303in}}%
\pgfpathcurveto{\pgfqpoint{2.280092in}{2.112067in}}{\pgfqpoint{2.283364in}{2.104167in}}{\pgfqpoint{2.289188in}{2.098343in}}%
\pgfpathcurveto{\pgfqpoint{2.295012in}{2.092519in}}{\pgfqpoint{2.302912in}{2.089247in}}{\pgfqpoint{2.311148in}{2.089247in}}%
\pgfpathclose%
\pgfusepath{stroke,fill}%
\end{pgfscope}%
\begin{pgfscope}%
\pgfpathrectangle{\pgfqpoint{0.100000in}{0.212622in}}{\pgfqpoint{3.696000in}{3.696000in}}%
\pgfusepath{clip}%
\pgfsetbuttcap%
\pgfsetroundjoin%
\definecolor{currentfill}{rgb}{0.121569,0.466667,0.705882}%
\pgfsetfillcolor{currentfill}%
\pgfsetfillopacity{0.425727}%
\pgfsetlinewidth{1.003750pt}%
\definecolor{currentstroke}{rgb}{0.121569,0.466667,0.705882}%
\pgfsetstrokecolor{currentstroke}%
\pgfsetstrokeopacity{0.425727}%
\pgfsetdash{}{0pt}%
\pgfpathmoveto{\pgfqpoint{2.345609in}{2.089237in}}%
\pgfpathcurveto{\pgfqpoint{2.353845in}{2.089237in}}{\pgfqpoint{2.361745in}{2.092509in}}{\pgfqpoint{2.367569in}{2.098333in}}%
\pgfpathcurveto{\pgfqpoint{2.373393in}{2.104157in}}{\pgfqpoint{2.376665in}{2.112057in}}{\pgfqpoint{2.376665in}{2.120293in}}%
\pgfpathcurveto{\pgfqpoint{2.376665in}{2.128530in}}{\pgfqpoint{2.373393in}{2.136430in}}{\pgfqpoint{2.367569in}{2.142254in}}%
\pgfpathcurveto{\pgfqpoint{2.361745in}{2.148078in}}{\pgfqpoint{2.353845in}{2.151350in}}{\pgfqpoint{2.345609in}{2.151350in}}%
\pgfpathcurveto{\pgfqpoint{2.337372in}{2.151350in}}{\pgfqpoint{2.329472in}{2.148078in}}{\pgfqpoint{2.323648in}{2.142254in}}%
\pgfpathcurveto{\pgfqpoint{2.317824in}{2.136430in}}{\pgfqpoint{2.314552in}{2.128530in}}{\pgfqpoint{2.314552in}{2.120293in}}%
\pgfpathcurveto{\pgfqpoint{2.314552in}{2.112057in}}{\pgfqpoint{2.317824in}{2.104157in}}{\pgfqpoint{2.323648in}{2.098333in}}%
\pgfpathcurveto{\pgfqpoint{2.329472in}{2.092509in}}{\pgfqpoint{2.337372in}{2.089237in}}{\pgfqpoint{2.345609in}{2.089237in}}%
\pgfpathclose%
\pgfusepath{stroke,fill}%
\end{pgfscope}%
\begin{pgfscope}%
\pgfpathrectangle{\pgfqpoint{0.100000in}{0.212622in}}{\pgfqpoint{3.696000in}{3.696000in}}%
\pgfusepath{clip}%
\pgfsetbuttcap%
\pgfsetroundjoin%
\definecolor{currentfill}{rgb}{0.121569,0.466667,0.705882}%
\pgfsetfillcolor{currentfill}%
\pgfsetfillopacity{0.431406}%
\pgfsetlinewidth{1.003750pt}%
\definecolor{currentstroke}{rgb}{0.121569,0.466667,0.705882}%
\pgfsetstrokecolor{currentstroke}%
\pgfsetstrokeopacity{0.431406}%
\pgfsetdash{}{0pt}%
\pgfpathmoveto{\pgfqpoint{2.379844in}{2.091923in}}%
\pgfpathcurveto{\pgfqpoint{2.388081in}{2.091923in}}{\pgfqpoint{2.395981in}{2.095196in}}{\pgfqpoint{2.401805in}{2.101020in}}%
\pgfpathcurveto{\pgfqpoint{2.407629in}{2.106844in}}{\pgfqpoint{2.410901in}{2.114744in}}{\pgfqpoint{2.410901in}{2.122980in}}%
\pgfpathcurveto{\pgfqpoint{2.410901in}{2.131216in}}{\pgfqpoint{2.407629in}{2.139116in}}{\pgfqpoint{2.401805in}{2.144940in}}%
\pgfpathcurveto{\pgfqpoint{2.395981in}{2.150764in}}{\pgfqpoint{2.388081in}{2.154036in}}{\pgfqpoint{2.379844in}{2.154036in}}%
\pgfpathcurveto{\pgfqpoint{2.371608in}{2.154036in}}{\pgfqpoint{2.363708in}{2.150764in}}{\pgfqpoint{2.357884in}{2.144940in}}%
\pgfpathcurveto{\pgfqpoint{2.352060in}{2.139116in}}{\pgfqpoint{2.348788in}{2.131216in}}{\pgfqpoint{2.348788in}{2.122980in}}%
\pgfpathcurveto{\pgfqpoint{2.348788in}{2.114744in}}{\pgfqpoint{2.352060in}{2.106844in}}{\pgfqpoint{2.357884in}{2.101020in}}%
\pgfpathcurveto{\pgfqpoint{2.363708in}{2.095196in}}{\pgfqpoint{2.371608in}{2.091923in}}{\pgfqpoint{2.379844in}{2.091923in}}%
\pgfpathclose%
\pgfusepath{stroke,fill}%
\end{pgfscope}%
\begin{pgfscope}%
\pgfpathrectangle{\pgfqpoint{0.100000in}{0.212622in}}{\pgfqpoint{3.696000in}{3.696000in}}%
\pgfusepath{clip}%
\pgfsetbuttcap%
\pgfsetroundjoin%
\definecolor{currentfill}{rgb}{0.121569,0.466667,0.705882}%
\pgfsetfillcolor{currentfill}%
\pgfsetfillopacity{0.436728}%
\pgfsetlinewidth{1.003750pt}%
\definecolor{currentstroke}{rgb}{0.121569,0.466667,0.705882}%
\pgfsetstrokecolor{currentstroke}%
\pgfsetstrokeopacity{0.436728}%
\pgfsetdash{}{0pt}%
\pgfpathmoveto{\pgfqpoint{2.412000in}{2.095896in}}%
\pgfpathcurveto{\pgfqpoint{2.420237in}{2.095896in}}{\pgfqpoint{2.428137in}{2.099169in}}{\pgfqpoint{2.433961in}{2.104993in}}%
\pgfpathcurveto{\pgfqpoint{2.439784in}{2.110817in}}{\pgfqpoint{2.443057in}{2.118717in}}{\pgfqpoint{2.443057in}{2.126953in}}%
\pgfpathcurveto{\pgfqpoint{2.443057in}{2.135189in}}{\pgfqpoint{2.439784in}{2.143089in}}{\pgfqpoint{2.433961in}{2.148913in}}%
\pgfpathcurveto{\pgfqpoint{2.428137in}{2.154737in}}{\pgfqpoint{2.420237in}{2.158009in}}{\pgfqpoint{2.412000in}{2.158009in}}%
\pgfpathcurveto{\pgfqpoint{2.403764in}{2.158009in}}{\pgfqpoint{2.395864in}{2.154737in}}{\pgfqpoint{2.390040in}{2.148913in}}%
\pgfpathcurveto{\pgfqpoint{2.384216in}{2.143089in}}{\pgfqpoint{2.380944in}{2.135189in}}{\pgfqpoint{2.380944in}{2.126953in}}%
\pgfpathcurveto{\pgfqpoint{2.380944in}{2.118717in}}{\pgfqpoint{2.384216in}{2.110817in}}{\pgfqpoint{2.390040in}{2.104993in}}%
\pgfpathcurveto{\pgfqpoint{2.395864in}{2.099169in}}{\pgfqpoint{2.403764in}{2.095896in}}{\pgfqpoint{2.412000in}{2.095896in}}%
\pgfpathclose%
\pgfusepath{stroke,fill}%
\end{pgfscope}%
\begin{pgfscope}%
\pgfpathrectangle{\pgfqpoint{0.100000in}{0.212622in}}{\pgfqpoint{3.696000in}{3.696000in}}%
\pgfusepath{clip}%
\pgfsetbuttcap%
\pgfsetroundjoin%
\definecolor{currentfill}{rgb}{0.121569,0.466667,0.705882}%
\pgfsetfillcolor{currentfill}%
\pgfsetfillopacity{0.441713}%
\pgfsetlinewidth{1.003750pt}%
\definecolor{currentstroke}{rgb}{0.121569,0.466667,0.705882}%
\pgfsetstrokecolor{currentstroke}%
\pgfsetstrokeopacity{0.441713}%
\pgfsetdash{}{0pt}%
\pgfpathmoveto{\pgfqpoint{2.442696in}{2.100652in}}%
\pgfpathcurveto{\pgfqpoint{2.450932in}{2.100652in}}{\pgfqpoint{2.458832in}{2.103924in}}{\pgfqpoint{2.464656in}{2.109748in}}%
\pgfpathcurveto{\pgfqpoint{2.470480in}{2.115572in}}{\pgfqpoint{2.473753in}{2.123472in}}{\pgfqpoint{2.473753in}{2.131709in}}%
\pgfpathcurveto{\pgfqpoint{2.473753in}{2.139945in}}{\pgfqpoint{2.470480in}{2.147845in}}{\pgfqpoint{2.464656in}{2.153669in}}%
\pgfpathcurveto{\pgfqpoint{2.458832in}{2.159493in}}{\pgfqpoint{2.450932in}{2.162765in}}{\pgfqpoint{2.442696in}{2.162765in}}%
\pgfpathcurveto{\pgfqpoint{2.434460in}{2.162765in}}{\pgfqpoint{2.426560in}{2.159493in}}{\pgfqpoint{2.420736in}{2.153669in}}%
\pgfpathcurveto{\pgfqpoint{2.414912in}{2.147845in}}{\pgfqpoint{2.411640in}{2.139945in}}{\pgfqpoint{2.411640in}{2.131709in}}%
\pgfpathcurveto{\pgfqpoint{2.411640in}{2.123472in}}{\pgfqpoint{2.414912in}{2.115572in}}{\pgfqpoint{2.420736in}{2.109748in}}%
\pgfpathcurveto{\pgfqpoint{2.426560in}{2.103924in}}{\pgfqpoint{2.434460in}{2.100652in}}{\pgfqpoint{2.442696in}{2.100652in}}%
\pgfpathclose%
\pgfusepath{stroke,fill}%
\end{pgfscope}%
\begin{pgfscope}%
\pgfpathrectangle{\pgfqpoint{0.100000in}{0.212622in}}{\pgfqpoint{3.696000in}{3.696000in}}%
\pgfusepath{clip}%
\pgfsetbuttcap%
\pgfsetroundjoin%
\definecolor{currentfill}{rgb}{0.121569,0.466667,0.705882}%
\pgfsetfillcolor{currentfill}%
\pgfsetfillopacity{0.446489}%
\pgfsetlinewidth{1.003750pt}%
\definecolor{currentstroke}{rgb}{0.121569,0.466667,0.705882}%
\pgfsetstrokecolor{currentstroke}%
\pgfsetstrokeopacity{0.446489}%
\pgfsetdash{}{0pt}%
\pgfpathmoveto{\pgfqpoint{2.472785in}{2.103478in}}%
\pgfpathcurveto{\pgfqpoint{2.481021in}{2.103478in}}{\pgfqpoint{2.488921in}{2.106750in}}{\pgfqpoint{2.494745in}{2.112574in}}%
\pgfpathcurveto{\pgfqpoint{2.500569in}{2.118398in}}{\pgfqpoint{2.503841in}{2.126298in}}{\pgfqpoint{2.503841in}{2.134534in}}%
\pgfpathcurveto{\pgfqpoint{2.503841in}{2.142771in}}{\pgfqpoint{2.500569in}{2.150671in}}{\pgfqpoint{2.494745in}{2.156495in}}%
\pgfpathcurveto{\pgfqpoint{2.488921in}{2.162318in}}{\pgfqpoint{2.481021in}{2.165591in}}{\pgfqpoint{2.472785in}{2.165591in}}%
\pgfpathcurveto{\pgfqpoint{2.464549in}{2.165591in}}{\pgfqpoint{2.456649in}{2.162318in}}{\pgfqpoint{2.450825in}{2.156495in}}%
\pgfpathcurveto{\pgfqpoint{2.445001in}{2.150671in}}{\pgfqpoint{2.441728in}{2.142771in}}{\pgfqpoint{2.441728in}{2.134534in}}%
\pgfpathcurveto{\pgfqpoint{2.441728in}{2.126298in}}{\pgfqpoint{2.445001in}{2.118398in}}{\pgfqpoint{2.450825in}{2.112574in}}%
\pgfpathcurveto{\pgfqpoint{2.456649in}{2.106750in}}{\pgfqpoint{2.464549in}{2.103478in}}{\pgfqpoint{2.472785in}{2.103478in}}%
\pgfpathclose%
\pgfusepath{stroke,fill}%
\end{pgfscope}%
\begin{pgfscope}%
\pgfpathrectangle{\pgfqpoint{0.100000in}{0.212622in}}{\pgfqpoint{3.696000in}{3.696000in}}%
\pgfusepath{clip}%
\pgfsetbuttcap%
\pgfsetroundjoin%
\definecolor{currentfill}{rgb}{0.121569,0.466667,0.705882}%
\pgfsetfillcolor{currentfill}%
\pgfsetfillopacity{0.451111}%
\pgfsetlinewidth{1.003750pt}%
\definecolor{currentstroke}{rgb}{0.121569,0.466667,0.705882}%
\pgfsetstrokecolor{currentstroke}%
\pgfsetstrokeopacity{0.451111}%
\pgfsetdash{}{0pt}%
\pgfpathmoveto{\pgfqpoint{2.501275in}{2.104587in}}%
\pgfpathcurveto{\pgfqpoint{2.509511in}{2.104587in}}{\pgfqpoint{2.517411in}{2.107859in}}{\pgfqpoint{2.523235in}{2.113683in}}%
\pgfpathcurveto{\pgfqpoint{2.529059in}{2.119507in}}{\pgfqpoint{2.532331in}{2.127407in}}{\pgfqpoint{2.532331in}{2.135643in}}%
\pgfpathcurveto{\pgfqpoint{2.532331in}{2.143879in}}{\pgfqpoint{2.529059in}{2.151779in}}{\pgfqpoint{2.523235in}{2.157603in}}%
\pgfpathcurveto{\pgfqpoint{2.517411in}{2.163427in}}{\pgfqpoint{2.509511in}{2.166700in}}{\pgfqpoint{2.501275in}{2.166700in}}%
\pgfpathcurveto{\pgfqpoint{2.493038in}{2.166700in}}{\pgfqpoint{2.485138in}{2.163427in}}{\pgfqpoint{2.479314in}{2.157603in}}%
\pgfpathcurveto{\pgfqpoint{2.473490in}{2.151779in}}{\pgfqpoint{2.470218in}{2.143879in}}{\pgfqpoint{2.470218in}{2.135643in}}%
\pgfpathcurveto{\pgfqpoint{2.470218in}{2.127407in}}{\pgfqpoint{2.473490in}{2.119507in}}{\pgfqpoint{2.479314in}{2.113683in}}%
\pgfpathcurveto{\pgfqpoint{2.485138in}{2.107859in}}{\pgfqpoint{2.493038in}{2.104587in}}{\pgfqpoint{2.501275in}{2.104587in}}%
\pgfpathclose%
\pgfusepath{stroke,fill}%
\end{pgfscope}%
\begin{pgfscope}%
\pgfpathrectangle{\pgfqpoint{0.100000in}{0.212622in}}{\pgfqpoint{3.696000in}{3.696000in}}%
\pgfusepath{clip}%
\pgfsetbuttcap%
\pgfsetroundjoin%
\definecolor{currentfill}{rgb}{0.121569,0.466667,0.705882}%
\pgfsetfillcolor{currentfill}%
\pgfsetfillopacity{0.455566}%
\pgfsetlinewidth{1.003750pt}%
\definecolor{currentstroke}{rgb}{0.121569,0.466667,0.705882}%
\pgfsetstrokecolor{currentstroke}%
\pgfsetstrokeopacity{0.455566}%
\pgfsetdash{}{0pt}%
\pgfpathmoveto{\pgfqpoint{2.528194in}{2.105057in}}%
\pgfpathcurveto{\pgfqpoint{2.536431in}{2.105057in}}{\pgfqpoint{2.544331in}{2.108330in}}{\pgfqpoint{2.550155in}{2.114154in}}%
\pgfpathcurveto{\pgfqpoint{2.555979in}{2.119977in}}{\pgfqpoint{2.559251in}{2.127878in}}{\pgfqpoint{2.559251in}{2.136114in}}%
\pgfpathcurveto{\pgfqpoint{2.559251in}{2.144350in}}{\pgfqpoint{2.555979in}{2.152250in}}{\pgfqpoint{2.550155in}{2.158074in}}%
\pgfpathcurveto{\pgfqpoint{2.544331in}{2.163898in}}{\pgfqpoint{2.536431in}{2.167170in}}{\pgfqpoint{2.528194in}{2.167170in}}%
\pgfpathcurveto{\pgfqpoint{2.519958in}{2.167170in}}{\pgfqpoint{2.512058in}{2.163898in}}{\pgfqpoint{2.506234in}{2.158074in}}%
\pgfpathcurveto{\pgfqpoint{2.500410in}{2.152250in}}{\pgfqpoint{2.497138in}{2.144350in}}{\pgfqpoint{2.497138in}{2.136114in}}%
\pgfpathcurveto{\pgfqpoint{2.497138in}{2.127878in}}{\pgfqpoint{2.500410in}{2.119977in}}{\pgfqpoint{2.506234in}{2.114154in}}%
\pgfpathcurveto{\pgfqpoint{2.512058in}{2.108330in}}{\pgfqpoint{2.519958in}{2.105057in}}{\pgfqpoint{2.528194in}{2.105057in}}%
\pgfpathclose%
\pgfusepath{stroke,fill}%
\end{pgfscope}%
\begin{pgfscope}%
\pgfpathrectangle{\pgfqpoint{0.100000in}{0.212622in}}{\pgfqpoint{3.696000in}{3.696000in}}%
\pgfusepath{clip}%
\pgfsetbuttcap%
\pgfsetroundjoin%
\definecolor{currentfill}{rgb}{0.121569,0.466667,0.705882}%
\pgfsetfillcolor{currentfill}%
\pgfsetfillopacity{0.460382}%
\pgfsetlinewidth{1.003750pt}%
\definecolor{currentstroke}{rgb}{0.121569,0.466667,0.705882}%
\pgfsetstrokecolor{currentstroke}%
\pgfsetstrokeopacity{0.460382}%
\pgfsetdash{}{0pt}%
\pgfpathmoveto{\pgfqpoint{2.553514in}{2.103761in}}%
\pgfpathcurveto{\pgfqpoint{2.561751in}{2.103761in}}{\pgfqpoint{2.569651in}{2.107033in}}{\pgfqpoint{2.575475in}{2.112857in}}%
\pgfpathcurveto{\pgfqpoint{2.581298in}{2.118681in}}{\pgfqpoint{2.584571in}{2.126581in}}{\pgfqpoint{2.584571in}{2.134817in}}%
\pgfpathcurveto{\pgfqpoint{2.584571in}{2.143054in}}{\pgfqpoint{2.581298in}{2.150954in}}{\pgfqpoint{2.575475in}{2.156778in}}%
\pgfpathcurveto{\pgfqpoint{2.569651in}{2.162602in}}{\pgfqpoint{2.561751in}{2.165874in}}{\pgfqpoint{2.553514in}{2.165874in}}%
\pgfpathcurveto{\pgfqpoint{2.545278in}{2.165874in}}{\pgfqpoint{2.537378in}{2.162602in}}{\pgfqpoint{2.531554in}{2.156778in}}%
\pgfpathcurveto{\pgfqpoint{2.525730in}{2.150954in}}{\pgfqpoint{2.522458in}{2.143054in}}{\pgfqpoint{2.522458in}{2.134817in}}%
\pgfpathcurveto{\pgfqpoint{2.522458in}{2.126581in}}{\pgfqpoint{2.525730in}{2.118681in}}{\pgfqpoint{2.531554in}{2.112857in}}%
\pgfpathcurveto{\pgfqpoint{2.537378in}{2.107033in}}{\pgfqpoint{2.545278in}{2.103761in}}{\pgfqpoint{2.553514in}{2.103761in}}%
\pgfpathclose%
\pgfusepath{stroke,fill}%
\end{pgfscope}%
\begin{pgfscope}%
\pgfpathrectangle{\pgfqpoint{0.100000in}{0.212622in}}{\pgfqpoint{3.696000in}{3.696000in}}%
\pgfusepath{clip}%
\pgfsetbuttcap%
\pgfsetroundjoin%
\definecolor{currentfill}{rgb}{0.121569,0.466667,0.705882}%
\pgfsetfillcolor{currentfill}%
\pgfsetfillopacity{0.465347}%
\pgfsetlinewidth{1.003750pt}%
\definecolor{currentstroke}{rgb}{0.121569,0.466667,0.705882}%
\pgfsetstrokecolor{currentstroke}%
\pgfsetstrokeopacity{0.465347}%
\pgfsetdash{}{0pt}%
\pgfpathmoveto{\pgfqpoint{2.578707in}{2.104770in}}%
\pgfpathcurveto{\pgfqpoint{2.586943in}{2.104770in}}{\pgfqpoint{2.594843in}{2.108043in}}{\pgfqpoint{2.600667in}{2.113867in}}%
\pgfpathcurveto{\pgfqpoint{2.606491in}{2.119690in}}{\pgfqpoint{2.609764in}{2.127590in}}{\pgfqpoint{2.609764in}{2.135827in}}%
\pgfpathcurveto{\pgfqpoint{2.609764in}{2.144063in}}{\pgfqpoint{2.606491in}{2.151963in}}{\pgfqpoint{2.600667in}{2.157787in}}%
\pgfpathcurveto{\pgfqpoint{2.594843in}{2.163611in}}{\pgfqpoint{2.586943in}{2.166883in}}{\pgfqpoint{2.578707in}{2.166883in}}%
\pgfpathcurveto{\pgfqpoint{2.570471in}{2.166883in}}{\pgfqpoint{2.562571in}{2.163611in}}{\pgfqpoint{2.556747in}{2.157787in}}%
\pgfpathcurveto{\pgfqpoint{2.550923in}{2.151963in}}{\pgfqpoint{2.547651in}{2.144063in}}{\pgfqpoint{2.547651in}{2.135827in}}%
\pgfpathcurveto{\pgfqpoint{2.547651in}{2.127590in}}{\pgfqpoint{2.550923in}{2.119690in}}{\pgfqpoint{2.556747in}{2.113867in}}%
\pgfpathcurveto{\pgfqpoint{2.562571in}{2.108043in}}{\pgfqpoint{2.570471in}{2.104770in}}{\pgfqpoint{2.578707in}{2.104770in}}%
\pgfpathclose%
\pgfusepath{stroke,fill}%
\end{pgfscope}%
\begin{pgfscope}%
\pgfpathrectangle{\pgfqpoint{0.100000in}{0.212622in}}{\pgfqpoint{3.696000in}{3.696000in}}%
\pgfusepath{clip}%
\pgfsetbuttcap%
\pgfsetroundjoin%
\definecolor{currentfill}{rgb}{0.121569,0.466667,0.705882}%
\pgfsetfillcolor{currentfill}%
\pgfsetfillopacity{0.470238}%
\pgfsetlinewidth{1.003750pt}%
\definecolor{currentstroke}{rgb}{0.121569,0.466667,0.705882}%
\pgfsetstrokecolor{currentstroke}%
\pgfsetstrokeopacity{0.470238}%
\pgfsetdash{}{0pt}%
\pgfpathmoveto{\pgfqpoint{2.603324in}{2.107119in}}%
\pgfpathcurveto{\pgfqpoint{2.611561in}{2.107119in}}{\pgfqpoint{2.619461in}{2.110392in}}{\pgfqpoint{2.625285in}{2.116216in}}%
\pgfpathcurveto{\pgfqpoint{2.631108in}{2.122039in}}{\pgfqpoint{2.634381in}{2.129940in}}{\pgfqpoint{2.634381in}{2.138176in}}%
\pgfpathcurveto{\pgfqpoint{2.634381in}{2.146412in}}{\pgfqpoint{2.631108in}{2.154312in}}{\pgfqpoint{2.625285in}{2.160136in}}%
\pgfpathcurveto{\pgfqpoint{2.619461in}{2.165960in}}{\pgfqpoint{2.611561in}{2.169232in}}{\pgfqpoint{2.603324in}{2.169232in}}%
\pgfpathcurveto{\pgfqpoint{2.595088in}{2.169232in}}{\pgfqpoint{2.587188in}{2.165960in}}{\pgfqpoint{2.581364in}{2.160136in}}%
\pgfpathcurveto{\pgfqpoint{2.575540in}{2.154312in}}{\pgfqpoint{2.572268in}{2.146412in}}{\pgfqpoint{2.572268in}{2.138176in}}%
\pgfpathcurveto{\pgfqpoint{2.572268in}{2.129940in}}{\pgfqpoint{2.575540in}{2.122039in}}{\pgfqpoint{2.581364in}{2.116216in}}%
\pgfpathcurveto{\pgfqpoint{2.587188in}{2.110392in}}{\pgfqpoint{2.595088in}{2.107119in}}{\pgfqpoint{2.603324in}{2.107119in}}%
\pgfpathclose%
\pgfusepath{stroke,fill}%
\end{pgfscope}%
\begin{pgfscope}%
\pgfpathrectangle{\pgfqpoint{0.100000in}{0.212622in}}{\pgfqpoint{3.696000in}{3.696000in}}%
\pgfusepath{clip}%
\pgfsetbuttcap%
\pgfsetroundjoin%
\definecolor{currentfill}{rgb}{0.121569,0.466667,0.705882}%
\pgfsetfillcolor{currentfill}%
\pgfsetfillopacity{0.474650}%
\pgfsetlinewidth{1.003750pt}%
\definecolor{currentstroke}{rgb}{0.121569,0.466667,0.705882}%
\pgfsetstrokecolor{currentstroke}%
\pgfsetstrokeopacity{0.474650}%
\pgfsetdash{}{0pt}%
\pgfpathmoveto{\pgfqpoint{2.625889in}{2.110398in}}%
\pgfpathcurveto{\pgfqpoint{2.634126in}{2.110398in}}{\pgfqpoint{2.642026in}{2.113670in}}{\pgfqpoint{2.647850in}{2.119494in}}%
\pgfpathcurveto{\pgfqpoint{2.653673in}{2.125318in}}{\pgfqpoint{2.656946in}{2.133218in}}{\pgfqpoint{2.656946in}{2.141454in}}%
\pgfpathcurveto{\pgfqpoint{2.656946in}{2.149691in}}{\pgfqpoint{2.653673in}{2.157591in}}{\pgfqpoint{2.647850in}{2.163415in}}%
\pgfpathcurveto{\pgfqpoint{2.642026in}{2.169239in}}{\pgfqpoint{2.634126in}{2.172511in}}{\pgfqpoint{2.625889in}{2.172511in}}%
\pgfpathcurveto{\pgfqpoint{2.617653in}{2.172511in}}{\pgfqpoint{2.609753in}{2.169239in}}{\pgfqpoint{2.603929in}{2.163415in}}%
\pgfpathcurveto{\pgfqpoint{2.598105in}{2.157591in}}{\pgfqpoint{2.594833in}{2.149691in}}{\pgfqpoint{2.594833in}{2.141454in}}%
\pgfpathcurveto{\pgfqpoint{2.594833in}{2.133218in}}{\pgfqpoint{2.598105in}{2.125318in}}{\pgfqpoint{2.603929in}{2.119494in}}%
\pgfpathcurveto{\pgfqpoint{2.609753in}{2.113670in}}{\pgfqpoint{2.617653in}{2.110398in}}{\pgfqpoint{2.625889in}{2.110398in}}%
\pgfpathclose%
\pgfusepath{stroke,fill}%
\end{pgfscope}%
\begin{pgfscope}%
\pgfpathrectangle{\pgfqpoint{0.100000in}{0.212622in}}{\pgfqpoint{3.696000in}{3.696000in}}%
\pgfusepath{clip}%
\pgfsetbuttcap%
\pgfsetroundjoin%
\definecolor{currentfill}{rgb}{0.121569,0.466667,0.705882}%
\pgfsetfillcolor{currentfill}%
\pgfsetfillopacity{0.478417}%
\pgfsetlinewidth{1.003750pt}%
\definecolor{currentstroke}{rgb}{0.121569,0.466667,0.705882}%
\pgfsetstrokecolor{currentstroke}%
\pgfsetstrokeopacity{0.478417}%
\pgfsetdash{}{0pt}%
\pgfpathmoveto{\pgfqpoint{2.646292in}{2.114643in}}%
\pgfpathcurveto{\pgfqpoint{2.654528in}{2.114643in}}{\pgfqpoint{2.662428in}{2.117916in}}{\pgfqpoint{2.668252in}{2.123740in}}%
\pgfpathcurveto{\pgfqpoint{2.674076in}{2.129564in}}{\pgfqpoint{2.677348in}{2.137464in}}{\pgfqpoint{2.677348in}{2.145700in}}%
\pgfpathcurveto{\pgfqpoint{2.677348in}{2.153936in}}{\pgfqpoint{2.674076in}{2.161836in}}{\pgfqpoint{2.668252in}{2.167660in}}%
\pgfpathcurveto{\pgfqpoint{2.662428in}{2.173484in}}{\pgfqpoint{2.654528in}{2.176756in}}{\pgfqpoint{2.646292in}{2.176756in}}%
\pgfpathcurveto{\pgfqpoint{2.638056in}{2.176756in}}{\pgfqpoint{2.630156in}{2.173484in}}{\pgfqpoint{2.624332in}{2.167660in}}%
\pgfpathcurveto{\pgfqpoint{2.618508in}{2.161836in}}{\pgfqpoint{2.615235in}{2.153936in}}{\pgfqpoint{2.615235in}{2.145700in}}%
\pgfpathcurveto{\pgfqpoint{2.615235in}{2.137464in}}{\pgfqpoint{2.618508in}{2.129564in}}{\pgfqpoint{2.624332in}{2.123740in}}%
\pgfpathcurveto{\pgfqpoint{2.630156in}{2.117916in}}{\pgfqpoint{2.638056in}{2.114643in}}{\pgfqpoint{2.646292in}{2.114643in}}%
\pgfpathclose%
\pgfusepath{stroke,fill}%
\end{pgfscope}%
\begin{pgfscope}%
\pgfpathrectangle{\pgfqpoint{0.100000in}{0.212622in}}{\pgfqpoint{3.696000in}{3.696000in}}%
\pgfusepath{clip}%
\pgfsetbuttcap%
\pgfsetroundjoin%
\definecolor{currentfill}{rgb}{0.121569,0.466667,0.705882}%
\pgfsetfillcolor{currentfill}%
\pgfsetfillopacity{0.481650}%
\pgfsetlinewidth{1.003750pt}%
\definecolor{currentstroke}{rgb}{0.121569,0.466667,0.705882}%
\pgfsetstrokecolor{currentstroke}%
\pgfsetstrokeopacity{0.481650}%
\pgfsetdash{}{0pt}%
\pgfpathmoveto{\pgfqpoint{2.666259in}{2.117059in}}%
\pgfpathcurveto{\pgfqpoint{2.674495in}{2.117059in}}{\pgfqpoint{2.682395in}{2.120332in}}{\pgfqpoint{2.688219in}{2.126156in}}%
\pgfpathcurveto{\pgfqpoint{2.694043in}{2.131979in}}{\pgfqpoint{2.697315in}{2.139880in}}{\pgfqpoint{2.697315in}{2.148116in}}%
\pgfpathcurveto{\pgfqpoint{2.697315in}{2.156352in}}{\pgfqpoint{2.694043in}{2.164252in}}{\pgfqpoint{2.688219in}{2.170076in}}%
\pgfpathcurveto{\pgfqpoint{2.682395in}{2.175900in}}{\pgfqpoint{2.674495in}{2.179172in}}{\pgfqpoint{2.666259in}{2.179172in}}%
\pgfpathcurveto{\pgfqpoint{2.658022in}{2.179172in}}{\pgfqpoint{2.650122in}{2.175900in}}{\pgfqpoint{2.644298in}{2.170076in}}%
\pgfpathcurveto{\pgfqpoint{2.638474in}{2.164252in}}{\pgfqpoint{2.635202in}{2.156352in}}{\pgfqpoint{2.635202in}{2.148116in}}%
\pgfpathcurveto{\pgfqpoint{2.635202in}{2.139880in}}{\pgfqpoint{2.638474in}{2.131979in}}{\pgfqpoint{2.644298in}{2.126156in}}%
\pgfpathcurveto{\pgfqpoint{2.650122in}{2.120332in}}{\pgfqpoint{2.658022in}{2.117059in}}{\pgfqpoint{2.666259in}{2.117059in}}%
\pgfpathclose%
\pgfusepath{stroke,fill}%
\end{pgfscope}%
\begin{pgfscope}%
\pgfpathrectangle{\pgfqpoint{0.100000in}{0.212622in}}{\pgfqpoint{3.696000in}{3.696000in}}%
\pgfusepath{clip}%
\pgfsetbuttcap%
\pgfsetroundjoin%
\definecolor{currentfill}{rgb}{0.121569,0.466667,0.705882}%
\pgfsetfillcolor{currentfill}%
\pgfsetfillopacity{0.484789}%
\pgfsetlinewidth{1.003750pt}%
\definecolor{currentstroke}{rgb}{0.121569,0.466667,0.705882}%
\pgfsetstrokecolor{currentstroke}%
\pgfsetstrokeopacity{0.484789}%
\pgfsetdash{}{0pt}%
\pgfpathmoveto{\pgfqpoint{2.685641in}{2.118043in}}%
\pgfpathcurveto{\pgfqpoint{2.693878in}{2.118043in}}{\pgfqpoint{2.701778in}{2.121315in}}{\pgfqpoint{2.707602in}{2.127139in}}%
\pgfpathcurveto{\pgfqpoint{2.713426in}{2.132963in}}{\pgfqpoint{2.716698in}{2.140863in}}{\pgfqpoint{2.716698in}{2.149099in}}%
\pgfpathcurveto{\pgfqpoint{2.716698in}{2.157335in}}{\pgfqpoint{2.713426in}{2.165235in}}{\pgfqpoint{2.707602in}{2.171059in}}%
\pgfpathcurveto{\pgfqpoint{2.701778in}{2.176883in}}{\pgfqpoint{2.693878in}{2.180156in}}{\pgfqpoint{2.685641in}{2.180156in}}%
\pgfpathcurveto{\pgfqpoint{2.677405in}{2.180156in}}{\pgfqpoint{2.669505in}{2.176883in}}{\pgfqpoint{2.663681in}{2.171059in}}%
\pgfpathcurveto{\pgfqpoint{2.657857in}{2.165235in}}{\pgfqpoint{2.654585in}{2.157335in}}{\pgfqpoint{2.654585in}{2.149099in}}%
\pgfpathcurveto{\pgfqpoint{2.654585in}{2.140863in}}{\pgfqpoint{2.657857in}{2.132963in}}{\pgfqpoint{2.663681in}{2.127139in}}%
\pgfpathcurveto{\pgfqpoint{2.669505in}{2.121315in}}{\pgfqpoint{2.677405in}{2.118043in}}{\pgfqpoint{2.685641in}{2.118043in}}%
\pgfpathclose%
\pgfusepath{stroke,fill}%
\end{pgfscope}%
\begin{pgfscope}%
\pgfpathrectangle{\pgfqpoint{0.100000in}{0.212622in}}{\pgfqpoint{3.696000in}{3.696000in}}%
\pgfusepath{clip}%
\pgfsetbuttcap%
\pgfsetroundjoin%
\definecolor{currentfill}{rgb}{0.121569,0.466667,0.705882}%
\pgfsetfillcolor{currentfill}%
\pgfsetfillopacity{0.487741}%
\pgfsetlinewidth{1.003750pt}%
\definecolor{currentstroke}{rgb}{0.121569,0.466667,0.705882}%
\pgfsetstrokecolor{currentstroke}%
\pgfsetstrokeopacity{0.487741}%
\pgfsetdash{}{0pt}%
\pgfpathmoveto{\pgfqpoint{2.703517in}{2.117590in}}%
\pgfpathcurveto{\pgfqpoint{2.711754in}{2.117590in}}{\pgfqpoint{2.719654in}{2.120863in}}{\pgfqpoint{2.725478in}{2.126687in}}%
\pgfpathcurveto{\pgfqpoint{2.731302in}{2.132510in}}{\pgfqpoint{2.734574in}{2.140411in}}{\pgfqpoint{2.734574in}{2.148647in}}%
\pgfpathcurveto{\pgfqpoint{2.734574in}{2.156883in}}{\pgfqpoint{2.731302in}{2.164783in}}{\pgfqpoint{2.725478in}{2.170607in}}%
\pgfpathcurveto{\pgfqpoint{2.719654in}{2.176431in}}{\pgfqpoint{2.711754in}{2.179703in}}{\pgfqpoint{2.703517in}{2.179703in}}%
\pgfpathcurveto{\pgfqpoint{2.695281in}{2.179703in}}{\pgfqpoint{2.687381in}{2.176431in}}{\pgfqpoint{2.681557in}{2.170607in}}%
\pgfpathcurveto{\pgfqpoint{2.675733in}{2.164783in}}{\pgfqpoint{2.672461in}{2.156883in}}{\pgfqpoint{2.672461in}{2.148647in}}%
\pgfpathcurveto{\pgfqpoint{2.672461in}{2.140411in}}{\pgfqpoint{2.675733in}{2.132510in}}{\pgfqpoint{2.681557in}{2.126687in}}%
\pgfpathcurveto{\pgfqpoint{2.687381in}{2.120863in}}{\pgfqpoint{2.695281in}{2.117590in}}{\pgfqpoint{2.703517in}{2.117590in}}%
\pgfpathclose%
\pgfusepath{stroke,fill}%
\end{pgfscope}%
\begin{pgfscope}%
\pgfpathrectangle{\pgfqpoint{0.100000in}{0.212622in}}{\pgfqpoint{3.696000in}{3.696000in}}%
\pgfusepath{clip}%
\pgfsetbuttcap%
\pgfsetroundjoin%
\definecolor{currentfill}{rgb}{0.121569,0.466667,0.705882}%
\pgfsetfillcolor{currentfill}%
\pgfsetfillopacity{0.490426}%
\pgfsetlinewidth{1.003750pt}%
\definecolor{currentstroke}{rgb}{0.121569,0.466667,0.705882}%
\pgfsetstrokecolor{currentstroke}%
\pgfsetstrokeopacity{0.490426}%
\pgfsetdash{}{0pt}%
\pgfpathmoveto{\pgfqpoint{2.719676in}{2.116312in}}%
\pgfpathcurveto{\pgfqpoint{2.727912in}{2.116312in}}{\pgfqpoint{2.735812in}{2.119584in}}{\pgfqpoint{2.741636in}{2.125408in}}%
\pgfpathcurveto{\pgfqpoint{2.747460in}{2.131232in}}{\pgfqpoint{2.750733in}{2.139132in}}{\pgfqpoint{2.750733in}{2.147368in}}%
\pgfpathcurveto{\pgfqpoint{2.750733in}{2.155605in}}{\pgfqpoint{2.747460in}{2.163505in}}{\pgfqpoint{2.741636in}{2.169329in}}%
\pgfpathcurveto{\pgfqpoint{2.735812in}{2.175152in}}{\pgfqpoint{2.727912in}{2.178425in}}{\pgfqpoint{2.719676in}{2.178425in}}%
\pgfpathcurveto{\pgfqpoint{2.711440in}{2.178425in}}{\pgfqpoint{2.703540in}{2.175152in}}{\pgfqpoint{2.697716in}{2.169329in}}%
\pgfpathcurveto{\pgfqpoint{2.691892in}{2.163505in}}{\pgfqpoint{2.688620in}{2.155605in}}{\pgfqpoint{2.688620in}{2.147368in}}%
\pgfpathcurveto{\pgfqpoint{2.688620in}{2.139132in}}{\pgfqpoint{2.691892in}{2.131232in}}{\pgfqpoint{2.697716in}{2.125408in}}%
\pgfpathcurveto{\pgfqpoint{2.703540in}{2.119584in}}{\pgfqpoint{2.711440in}{2.116312in}}{\pgfqpoint{2.719676in}{2.116312in}}%
\pgfpathclose%
\pgfusepath{stroke,fill}%
\end{pgfscope}%
\begin{pgfscope}%
\pgfpathrectangle{\pgfqpoint{0.100000in}{0.212622in}}{\pgfqpoint{3.696000in}{3.696000in}}%
\pgfusepath{clip}%
\pgfsetbuttcap%
\pgfsetroundjoin%
\definecolor{currentfill}{rgb}{0.121569,0.466667,0.705882}%
\pgfsetfillcolor{currentfill}%
\pgfsetfillopacity{0.493035}%
\pgfsetlinewidth{1.003750pt}%
\definecolor{currentstroke}{rgb}{0.121569,0.466667,0.705882}%
\pgfsetstrokecolor{currentstroke}%
\pgfsetstrokeopacity{0.493035}%
\pgfsetdash{}{0pt}%
\pgfpathmoveto{\pgfqpoint{2.735305in}{2.114632in}}%
\pgfpathcurveto{\pgfqpoint{2.743541in}{2.114632in}}{\pgfqpoint{2.751441in}{2.117904in}}{\pgfqpoint{2.757265in}{2.123728in}}%
\pgfpathcurveto{\pgfqpoint{2.763089in}{2.129552in}}{\pgfqpoint{2.766361in}{2.137452in}}{\pgfqpoint{2.766361in}{2.145689in}}%
\pgfpathcurveto{\pgfqpoint{2.766361in}{2.153925in}}{\pgfqpoint{2.763089in}{2.161825in}}{\pgfqpoint{2.757265in}{2.167649in}}%
\pgfpathcurveto{\pgfqpoint{2.751441in}{2.173473in}}{\pgfqpoint{2.743541in}{2.176745in}}{\pgfqpoint{2.735305in}{2.176745in}}%
\pgfpathcurveto{\pgfqpoint{2.727068in}{2.176745in}}{\pgfqpoint{2.719168in}{2.173473in}}{\pgfqpoint{2.713344in}{2.167649in}}%
\pgfpathcurveto{\pgfqpoint{2.707520in}{2.161825in}}{\pgfqpoint{2.704248in}{2.153925in}}{\pgfqpoint{2.704248in}{2.145689in}}%
\pgfpathcurveto{\pgfqpoint{2.704248in}{2.137452in}}{\pgfqpoint{2.707520in}{2.129552in}}{\pgfqpoint{2.713344in}{2.123728in}}%
\pgfpathcurveto{\pgfqpoint{2.719168in}{2.117904in}}{\pgfqpoint{2.727068in}{2.114632in}}{\pgfqpoint{2.735305in}{2.114632in}}%
\pgfpathclose%
\pgfusepath{stroke,fill}%
\end{pgfscope}%
\begin{pgfscope}%
\pgfpathrectangle{\pgfqpoint{0.100000in}{0.212622in}}{\pgfqpoint{3.696000in}{3.696000in}}%
\pgfusepath{clip}%
\pgfsetbuttcap%
\pgfsetroundjoin%
\definecolor{currentfill}{rgb}{0.121569,0.466667,0.705882}%
\pgfsetfillcolor{currentfill}%
\pgfsetfillopacity{0.495629}%
\pgfsetlinewidth{1.003750pt}%
\definecolor{currentstroke}{rgb}{0.121569,0.466667,0.705882}%
\pgfsetstrokecolor{currentstroke}%
\pgfsetstrokeopacity{0.495629}%
\pgfsetdash{}{0pt}%
\pgfpathmoveto{\pgfqpoint{2.750651in}{2.113781in}}%
\pgfpathcurveto{\pgfqpoint{2.758887in}{2.113781in}}{\pgfqpoint{2.766787in}{2.117054in}}{\pgfqpoint{2.772611in}{2.122878in}}%
\pgfpathcurveto{\pgfqpoint{2.778435in}{2.128702in}}{\pgfqpoint{2.781707in}{2.136602in}}{\pgfqpoint{2.781707in}{2.144838in}}%
\pgfpathcurveto{\pgfqpoint{2.781707in}{2.153074in}}{\pgfqpoint{2.778435in}{2.160974in}}{\pgfqpoint{2.772611in}{2.166798in}}%
\pgfpathcurveto{\pgfqpoint{2.766787in}{2.172622in}}{\pgfqpoint{2.758887in}{2.175894in}}{\pgfqpoint{2.750651in}{2.175894in}}%
\pgfpathcurveto{\pgfqpoint{2.742415in}{2.175894in}}{\pgfqpoint{2.734514in}{2.172622in}}{\pgfqpoint{2.728691in}{2.166798in}}%
\pgfpathcurveto{\pgfqpoint{2.722867in}{2.160974in}}{\pgfqpoint{2.719594in}{2.153074in}}{\pgfqpoint{2.719594in}{2.144838in}}%
\pgfpathcurveto{\pgfqpoint{2.719594in}{2.136602in}}{\pgfqpoint{2.722867in}{2.128702in}}{\pgfqpoint{2.728691in}{2.122878in}}%
\pgfpathcurveto{\pgfqpoint{2.734514in}{2.117054in}}{\pgfqpoint{2.742415in}{2.113781in}}{\pgfqpoint{2.750651in}{2.113781in}}%
\pgfpathclose%
\pgfusepath{stroke,fill}%
\end{pgfscope}%
\begin{pgfscope}%
\pgfpathrectangle{\pgfqpoint{0.100000in}{0.212622in}}{\pgfqpoint{3.696000in}{3.696000in}}%
\pgfusepath{clip}%
\pgfsetbuttcap%
\pgfsetroundjoin%
\definecolor{currentfill}{rgb}{0.121569,0.466667,0.705882}%
\pgfsetfillcolor{currentfill}%
\pgfsetfillopacity{0.498129}%
\pgfsetlinewidth{1.003750pt}%
\definecolor{currentstroke}{rgb}{0.121569,0.466667,0.705882}%
\pgfsetstrokecolor{currentstroke}%
\pgfsetstrokeopacity{0.498129}%
\pgfsetdash{}{0pt}%
\pgfpathmoveto{\pgfqpoint{2.765503in}{2.113890in}}%
\pgfpathcurveto{\pgfqpoint{2.773739in}{2.113890in}}{\pgfqpoint{2.781639in}{2.117163in}}{\pgfqpoint{2.787463in}{2.122987in}}%
\pgfpathcurveto{\pgfqpoint{2.793287in}{2.128810in}}{\pgfqpoint{2.796559in}{2.136710in}}{\pgfqpoint{2.796559in}{2.144947in}}%
\pgfpathcurveto{\pgfqpoint{2.796559in}{2.153183in}}{\pgfqpoint{2.793287in}{2.161083in}}{\pgfqpoint{2.787463in}{2.166907in}}%
\pgfpathcurveto{\pgfqpoint{2.781639in}{2.172731in}}{\pgfqpoint{2.773739in}{2.176003in}}{\pgfqpoint{2.765503in}{2.176003in}}%
\pgfpathcurveto{\pgfqpoint{2.757266in}{2.176003in}}{\pgfqpoint{2.749366in}{2.172731in}}{\pgfqpoint{2.743542in}{2.166907in}}%
\pgfpathcurveto{\pgfqpoint{2.737718in}{2.161083in}}{\pgfqpoint{2.734446in}{2.153183in}}{\pgfqpoint{2.734446in}{2.144947in}}%
\pgfpathcurveto{\pgfqpoint{2.734446in}{2.136710in}}{\pgfqpoint{2.737718in}{2.128810in}}{\pgfqpoint{2.743542in}{2.122987in}}%
\pgfpathcurveto{\pgfqpoint{2.749366in}{2.117163in}}{\pgfqpoint{2.757266in}{2.113890in}}{\pgfqpoint{2.765503in}{2.113890in}}%
\pgfpathclose%
\pgfusepath{stroke,fill}%
\end{pgfscope}%
\begin{pgfscope}%
\pgfpathrectangle{\pgfqpoint{0.100000in}{0.212622in}}{\pgfqpoint{3.696000in}{3.696000in}}%
\pgfusepath{clip}%
\pgfsetbuttcap%
\pgfsetroundjoin%
\definecolor{currentfill}{rgb}{0.121569,0.466667,0.705882}%
\pgfsetfillcolor{currentfill}%
\pgfsetfillopacity{0.500195}%
\pgfsetlinewidth{1.003750pt}%
\definecolor{currentstroke}{rgb}{0.121569,0.466667,0.705882}%
\pgfsetstrokecolor{currentstroke}%
\pgfsetstrokeopacity{0.500195}%
\pgfsetdash{}{0pt}%
\pgfpathmoveto{\pgfqpoint{2.778348in}{2.114942in}}%
\pgfpathcurveto{\pgfqpoint{2.786584in}{2.114942in}}{\pgfqpoint{2.794484in}{2.118214in}}{\pgfqpoint{2.800308in}{2.124038in}}%
\pgfpathcurveto{\pgfqpoint{2.806132in}{2.129862in}}{\pgfqpoint{2.809404in}{2.137762in}}{\pgfqpoint{2.809404in}{2.145998in}}%
\pgfpathcurveto{\pgfqpoint{2.809404in}{2.154234in}}{\pgfqpoint{2.806132in}{2.162134in}}{\pgfqpoint{2.800308in}{2.167958in}}%
\pgfpathcurveto{\pgfqpoint{2.794484in}{2.173782in}}{\pgfqpoint{2.786584in}{2.177055in}}{\pgfqpoint{2.778348in}{2.177055in}}%
\pgfpathcurveto{\pgfqpoint{2.770112in}{2.177055in}}{\pgfqpoint{2.762212in}{2.173782in}}{\pgfqpoint{2.756388in}{2.167958in}}%
\pgfpathcurveto{\pgfqpoint{2.750564in}{2.162134in}}{\pgfqpoint{2.747291in}{2.154234in}}{\pgfqpoint{2.747291in}{2.145998in}}%
\pgfpathcurveto{\pgfqpoint{2.747291in}{2.137762in}}{\pgfqpoint{2.750564in}{2.129862in}}{\pgfqpoint{2.756388in}{2.124038in}}%
\pgfpathcurveto{\pgfqpoint{2.762212in}{2.118214in}}{\pgfqpoint{2.770112in}{2.114942in}}{\pgfqpoint{2.778348in}{2.114942in}}%
\pgfpathclose%
\pgfusepath{stroke,fill}%
\end{pgfscope}%
\begin{pgfscope}%
\pgfpathrectangle{\pgfqpoint{0.100000in}{0.212622in}}{\pgfqpoint{3.696000in}{3.696000in}}%
\pgfusepath{clip}%
\pgfsetbuttcap%
\pgfsetroundjoin%
\definecolor{currentfill}{rgb}{0.121569,0.466667,0.705882}%
\pgfsetfillcolor{currentfill}%
\pgfsetfillopacity{0.501805}%
\pgfsetlinewidth{1.003750pt}%
\definecolor{currentstroke}{rgb}{0.121569,0.466667,0.705882}%
\pgfsetstrokecolor{currentstroke}%
\pgfsetstrokeopacity{0.501805}%
\pgfsetdash{}{0pt}%
\pgfpathmoveto{\pgfqpoint{2.788865in}{2.115654in}}%
\pgfpathcurveto{\pgfqpoint{2.797101in}{2.115654in}}{\pgfqpoint{2.805001in}{2.118927in}}{\pgfqpoint{2.810825in}{2.124751in}}%
\pgfpathcurveto{\pgfqpoint{2.816649in}{2.130575in}}{\pgfqpoint{2.819921in}{2.138475in}}{\pgfqpoint{2.819921in}{2.146711in}}%
\pgfpathcurveto{\pgfqpoint{2.819921in}{2.154947in}}{\pgfqpoint{2.816649in}{2.162847in}}{\pgfqpoint{2.810825in}{2.168671in}}%
\pgfpathcurveto{\pgfqpoint{2.805001in}{2.174495in}}{\pgfqpoint{2.797101in}{2.177767in}}{\pgfqpoint{2.788865in}{2.177767in}}%
\pgfpathcurveto{\pgfqpoint{2.780629in}{2.177767in}}{\pgfqpoint{2.772728in}{2.174495in}}{\pgfqpoint{2.766905in}{2.168671in}}%
\pgfpathcurveto{\pgfqpoint{2.761081in}{2.162847in}}{\pgfqpoint{2.757808in}{2.154947in}}{\pgfqpoint{2.757808in}{2.146711in}}%
\pgfpathcurveto{\pgfqpoint{2.757808in}{2.138475in}}{\pgfqpoint{2.761081in}{2.130575in}}{\pgfqpoint{2.766905in}{2.124751in}}%
\pgfpathcurveto{\pgfqpoint{2.772728in}{2.118927in}}{\pgfqpoint{2.780629in}{2.115654in}}{\pgfqpoint{2.788865in}{2.115654in}}%
\pgfpathclose%
\pgfusepath{stroke,fill}%
\end{pgfscope}%
\begin{pgfscope}%
\pgfpathrectangle{\pgfqpoint{0.100000in}{0.212622in}}{\pgfqpoint{3.696000in}{3.696000in}}%
\pgfusepath{clip}%
\pgfsetbuttcap%
\pgfsetroundjoin%
\definecolor{currentfill}{rgb}{0.121569,0.466667,0.705882}%
\pgfsetfillcolor{currentfill}%
\pgfsetfillopacity{0.503152}%
\pgfsetlinewidth{1.003750pt}%
\definecolor{currentstroke}{rgb}{0.121569,0.466667,0.705882}%
\pgfsetstrokecolor{currentstroke}%
\pgfsetstrokeopacity{0.503152}%
\pgfsetdash{}{0pt}%
\pgfpathmoveto{\pgfqpoint{2.798776in}{2.115789in}}%
\pgfpathcurveto{\pgfqpoint{2.807012in}{2.115789in}}{\pgfqpoint{2.814912in}{2.119062in}}{\pgfqpoint{2.820736in}{2.124886in}}%
\pgfpathcurveto{\pgfqpoint{2.826560in}{2.130710in}}{\pgfqpoint{2.829832in}{2.138610in}}{\pgfqpoint{2.829832in}{2.146846in}}%
\pgfpathcurveto{\pgfqpoint{2.829832in}{2.155082in}}{\pgfqpoint{2.826560in}{2.162982in}}{\pgfqpoint{2.820736in}{2.168806in}}%
\pgfpathcurveto{\pgfqpoint{2.814912in}{2.174630in}}{\pgfqpoint{2.807012in}{2.177902in}}{\pgfqpoint{2.798776in}{2.177902in}}%
\pgfpathcurveto{\pgfqpoint{2.790540in}{2.177902in}}{\pgfqpoint{2.782639in}{2.174630in}}{\pgfqpoint{2.776816in}{2.168806in}}%
\pgfpathcurveto{\pgfqpoint{2.770992in}{2.162982in}}{\pgfqpoint{2.767719in}{2.155082in}}{\pgfqpoint{2.767719in}{2.146846in}}%
\pgfpathcurveto{\pgfqpoint{2.767719in}{2.138610in}}{\pgfqpoint{2.770992in}{2.130710in}}{\pgfqpoint{2.776816in}{2.124886in}}%
\pgfpathcurveto{\pgfqpoint{2.782639in}{2.119062in}}{\pgfqpoint{2.790540in}{2.115789in}}{\pgfqpoint{2.798776in}{2.115789in}}%
\pgfpathclose%
\pgfusepath{stroke,fill}%
\end{pgfscope}%
\begin{pgfscope}%
\pgfpathrectangle{\pgfqpoint{0.100000in}{0.212622in}}{\pgfqpoint{3.696000in}{3.696000in}}%
\pgfusepath{clip}%
\pgfsetbuttcap%
\pgfsetroundjoin%
\definecolor{currentfill}{rgb}{0.121569,0.466667,0.705882}%
\pgfsetfillcolor{currentfill}%
\pgfsetfillopacity{0.504436}%
\pgfsetlinewidth{1.003750pt}%
\definecolor{currentstroke}{rgb}{0.121569,0.466667,0.705882}%
\pgfsetstrokecolor{currentstroke}%
\pgfsetstrokeopacity{0.504436}%
\pgfsetdash{}{0pt}%
\pgfpathmoveto{\pgfqpoint{2.807849in}{2.115064in}}%
\pgfpathcurveto{\pgfqpoint{2.816085in}{2.115064in}}{\pgfqpoint{2.823985in}{2.118337in}}{\pgfqpoint{2.829809in}{2.124161in}}%
\pgfpathcurveto{\pgfqpoint{2.835633in}{2.129985in}}{\pgfqpoint{2.838905in}{2.137885in}}{\pgfqpoint{2.838905in}{2.146121in}}%
\pgfpathcurveto{\pgfqpoint{2.838905in}{2.154357in}}{\pgfqpoint{2.835633in}{2.162257in}}{\pgfqpoint{2.829809in}{2.168081in}}%
\pgfpathcurveto{\pgfqpoint{2.823985in}{2.173905in}}{\pgfqpoint{2.816085in}{2.177177in}}{\pgfqpoint{2.807849in}{2.177177in}}%
\pgfpathcurveto{\pgfqpoint{2.799613in}{2.177177in}}{\pgfqpoint{2.791713in}{2.173905in}}{\pgfqpoint{2.785889in}{2.168081in}}%
\pgfpathcurveto{\pgfqpoint{2.780065in}{2.162257in}}{\pgfqpoint{2.776792in}{2.154357in}}{\pgfqpoint{2.776792in}{2.146121in}}%
\pgfpathcurveto{\pgfqpoint{2.776792in}{2.137885in}}{\pgfqpoint{2.780065in}{2.129985in}}{\pgfqpoint{2.785889in}{2.124161in}}%
\pgfpathcurveto{\pgfqpoint{2.791713in}{2.118337in}}{\pgfqpoint{2.799613in}{2.115064in}}{\pgfqpoint{2.807849in}{2.115064in}}%
\pgfpathclose%
\pgfusepath{stroke,fill}%
\end{pgfscope}%
\begin{pgfscope}%
\pgfpathrectangle{\pgfqpoint{0.100000in}{0.212622in}}{\pgfqpoint{3.696000in}{3.696000in}}%
\pgfusepath{clip}%
\pgfsetbuttcap%
\pgfsetroundjoin%
\definecolor{currentfill}{rgb}{0.121569,0.466667,0.705882}%
\pgfsetfillcolor{currentfill}%
\pgfsetfillopacity{0.505530}%
\pgfsetlinewidth{1.003750pt}%
\definecolor{currentstroke}{rgb}{0.121569,0.466667,0.705882}%
\pgfsetstrokecolor{currentstroke}%
\pgfsetstrokeopacity{0.505530}%
\pgfsetdash{}{0pt}%
\pgfpathmoveto{\pgfqpoint{2.815396in}{2.113908in}}%
\pgfpathcurveto{\pgfqpoint{2.823632in}{2.113908in}}{\pgfqpoint{2.831532in}{2.117180in}}{\pgfqpoint{2.837356in}{2.123004in}}%
\pgfpathcurveto{\pgfqpoint{2.843180in}{2.128828in}}{\pgfqpoint{2.846453in}{2.136728in}}{\pgfqpoint{2.846453in}{2.144965in}}%
\pgfpathcurveto{\pgfqpoint{2.846453in}{2.153201in}}{\pgfqpoint{2.843180in}{2.161101in}}{\pgfqpoint{2.837356in}{2.166925in}}%
\pgfpathcurveto{\pgfqpoint{2.831532in}{2.172749in}}{\pgfqpoint{2.823632in}{2.176021in}}{\pgfqpoint{2.815396in}{2.176021in}}%
\pgfpathcurveto{\pgfqpoint{2.807160in}{2.176021in}}{\pgfqpoint{2.799260in}{2.172749in}}{\pgfqpoint{2.793436in}{2.166925in}}%
\pgfpathcurveto{\pgfqpoint{2.787612in}{2.161101in}}{\pgfqpoint{2.784340in}{2.153201in}}{\pgfqpoint{2.784340in}{2.144965in}}%
\pgfpathcurveto{\pgfqpoint{2.784340in}{2.136728in}}{\pgfqpoint{2.787612in}{2.128828in}}{\pgfqpoint{2.793436in}{2.123004in}}%
\pgfpathcurveto{\pgfqpoint{2.799260in}{2.117180in}}{\pgfqpoint{2.807160in}{2.113908in}}{\pgfqpoint{2.815396in}{2.113908in}}%
\pgfpathclose%
\pgfusepath{stroke,fill}%
\end{pgfscope}%
\begin{pgfscope}%
\pgfpathrectangle{\pgfqpoint{0.100000in}{0.212622in}}{\pgfqpoint{3.696000in}{3.696000in}}%
\pgfusepath{clip}%
\pgfsetbuttcap%
\pgfsetroundjoin%
\definecolor{currentfill}{rgb}{0.121569,0.466667,0.705882}%
\pgfsetfillcolor{currentfill}%
\pgfsetfillopacity{0.506425}%
\pgfsetlinewidth{1.003750pt}%
\definecolor{currentstroke}{rgb}{0.121569,0.466667,0.705882}%
\pgfsetstrokecolor{currentstroke}%
\pgfsetstrokeopacity{0.506425}%
\pgfsetdash{}{0pt}%
\pgfpathmoveto{\pgfqpoint{2.821392in}{2.112855in}}%
\pgfpathcurveto{\pgfqpoint{2.829629in}{2.112855in}}{\pgfqpoint{2.837529in}{2.116127in}}{\pgfqpoint{2.843353in}{2.121951in}}%
\pgfpathcurveto{\pgfqpoint{2.849176in}{2.127775in}}{\pgfqpoint{2.852449in}{2.135675in}}{\pgfqpoint{2.852449in}{2.143912in}}%
\pgfpathcurveto{\pgfqpoint{2.852449in}{2.152148in}}{\pgfqpoint{2.849176in}{2.160048in}}{\pgfqpoint{2.843353in}{2.165872in}}%
\pgfpathcurveto{\pgfqpoint{2.837529in}{2.171696in}}{\pgfqpoint{2.829629in}{2.174968in}}{\pgfqpoint{2.821392in}{2.174968in}}%
\pgfpathcurveto{\pgfqpoint{2.813156in}{2.174968in}}{\pgfqpoint{2.805256in}{2.171696in}}{\pgfqpoint{2.799432in}{2.165872in}}%
\pgfpathcurveto{\pgfqpoint{2.793608in}{2.160048in}}{\pgfqpoint{2.790336in}{2.152148in}}{\pgfqpoint{2.790336in}{2.143912in}}%
\pgfpathcurveto{\pgfqpoint{2.790336in}{2.135675in}}{\pgfqpoint{2.793608in}{2.127775in}}{\pgfqpoint{2.799432in}{2.121951in}}%
\pgfpathcurveto{\pgfqpoint{2.805256in}{2.116127in}}{\pgfqpoint{2.813156in}{2.112855in}}{\pgfqpoint{2.821392in}{2.112855in}}%
\pgfpathclose%
\pgfusepath{stroke,fill}%
\end{pgfscope}%
\begin{pgfscope}%
\pgfpathrectangle{\pgfqpoint{0.100000in}{0.212622in}}{\pgfqpoint{3.696000in}{3.696000in}}%
\pgfusepath{clip}%
\pgfsetbuttcap%
\pgfsetroundjoin%
\definecolor{currentfill}{rgb}{0.121569,0.466667,0.705882}%
\pgfsetfillcolor{currentfill}%
\pgfsetfillopacity{0.508186}%
\pgfsetlinewidth{1.003750pt}%
\definecolor{currentstroke}{rgb}{0.121569,0.466667,0.705882}%
\pgfsetstrokecolor{currentstroke}%
\pgfsetstrokeopacity{0.508186}%
\pgfsetdash{}{0pt}%
\pgfpathmoveto{\pgfqpoint{2.832122in}{2.111083in}}%
\pgfpathcurveto{\pgfqpoint{2.840358in}{2.111083in}}{\pgfqpoint{2.848258in}{2.114355in}}{\pgfqpoint{2.854082in}{2.120179in}}%
\pgfpathcurveto{\pgfqpoint{2.859906in}{2.126003in}}{\pgfqpoint{2.863178in}{2.133903in}}{\pgfqpoint{2.863178in}{2.142140in}}%
\pgfpathcurveto{\pgfqpoint{2.863178in}{2.150376in}}{\pgfqpoint{2.859906in}{2.158276in}}{\pgfqpoint{2.854082in}{2.164100in}}%
\pgfpathcurveto{\pgfqpoint{2.848258in}{2.169924in}}{\pgfqpoint{2.840358in}{2.173196in}}{\pgfqpoint{2.832122in}{2.173196in}}%
\pgfpathcurveto{\pgfqpoint{2.823885in}{2.173196in}}{\pgfqpoint{2.815985in}{2.169924in}}{\pgfqpoint{2.810161in}{2.164100in}}%
\pgfpathcurveto{\pgfqpoint{2.804337in}{2.158276in}}{\pgfqpoint{2.801065in}{2.150376in}}{\pgfqpoint{2.801065in}{2.142140in}}%
\pgfpathcurveto{\pgfqpoint{2.801065in}{2.133903in}}{\pgfqpoint{2.804337in}{2.126003in}}{\pgfqpoint{2.810161in}{2.120179in}}%
\pgfpathcurveto{\pgfqpoint{2.815985in}{2.114355in}}{\pgfqpoint{2.823885in}{2.111083in}}{\pgfqpoint{2.832122in}{2.111083in}}%
\pgfpathclose%
\pgfusepath{stroke,fill}%
\end{pgfscope}%
\begin{pgfscope}%
\pgfpathrectangle{\pgfqpoint{0.100000in}{0.212622in}}{\pgfqpoint{3.696000in}{3.696000in}}%
\pgfusepath{clip}%
\pgfsetbuttcap%
\pgfsetroundjoin%
\definecolor{currentfill}{rgb}{0.121569,0.466667,0.705882}%
\pgfsetfillcolor{currentfill}%
\pgfsetfillopacity{0.509874}%
\pgfsetlinewidth{1.003750pt}%
\definecolor{currentstroke}{rgb}{0.121569,0.466667,0.705882}%
\pgfsetstrokecolor{currentstroke}%
\pgfsetstrokeopacity{0.509874}%
\pgfsetdash{}{0pt}%
\pgfpathmoveto{\pgfqpoint{2.842164in}{2.109908in}}%
\pgfpathcurveto{\pgfqpoint{2.850400in}{2.109908in}}{\pgfqpoint{2.858300in}{2.113180in}}{\pgfqpoint{2.864124in}{2.119004in}}%
\pgfpathcurveto{\pgfqpoint{2.869948in}{2.124828in}}{\pgfqpoint{2.873220in}{2.132728in}}{\pgfqpoint{2.873220in}{2.140964in}}%
\pgfpathcurveto{\pgfqpoint{2.873220in}{2.149200in}}{\pgfqpoint{2.869948in}{2.157100in}}{\pgfqpoint{2.864124in}{2.162924in}}%
\pgfpathcurveto{\pgfqpoint{2.858300in}{2.168748in}}{\pgfqpoint{2.850400in}{2.172021in}}{\pgfqpoint{2.842164in}{2.172021in}}%
\pgfpathcurveto{\pgfqpoint{2.833927in}{2.172021in}}{\pgfqpoint{2.826027in}{2.168748in}}{\pgfqpoint{2.820203in}{2.162924in}}%
\pgfpathcurveto{\pgfqpoint{2.814380in}{2.157100in}}{\pgfqpoint{2.811107in}{2.149200in}}{\pgfqpoint{2.811107in}{2.140964in}}%
\pgfpathcurveto{\pgfqpoint{2.811107in}{2.132728in}}{\pgfqpoint{2.814380in}{2.124828in}}{\pgfqpoint{2.820203in}{2.119004in}}%
\pgfpathcurveto{\pgfqpoint{2.826027in}{2.113180in}}{\pgfqpoint{2.833927in}{2.109908in}}{\pgfqpoint{2.842164in}{2.109908in}}%
\pgfpathclose%
\pgfusepath{stroke,fill}%
\end{pgfscope}%
\begin{pgfscope}%
\pgfpathrectangle{\pgfqpoint{0.100000in}{0.212622in}}{\pgfqpoint{3.696000in}{3.696000in}}%
\pgfusepath{clip}%
\pgfsetbuttcap%
\pgfsetroundjoin%
\definecolor{currentfill}{rgb}{0.121569,0.466667,0.705882}%
\pgfsetfillcolor{currentfill}%
\pgfsetfillopacity{0.512990}%
\pgfsetlinewidth{1.003750pt}%
\definecolor{currentstroke}{rgb}{0.121569,0.466667,0.705882}%
\pgfsetstrokecolor{currentstroke}%
\pgfsetstrokeopacity{0.512990}%
\pgfsetdash{}{0pt}%
\pgfpathmoveto{\pgfqpoint{2.860655in}{2.108774in}}%
\pgfpathcurveto{\pgfqpoint{2.868891in}{2.108774in}}{\pgfqpoint{2.876791in}{2.112046in}}{\pgfqpoint{2.882615in}{2.117870in}}%
\pgfpathcurveto{\pgfqpoint{2.888439in}{2.123694in}}{\pgfqpoint{2.891711in}{2.131594in}}{\pgfqpoint{2.891711in}{2.139831in}}%
\pgfpathcurveto{\pgfqpoint{2.891711in}{2.148067in}}{\pgfqpoint{2.888439in}{2.155967in}}{\pgfqpoint{2.882615in}{2.161791in}}%
\pgfpathcurveto{\pgfqpoint{2.876791in}{2.167615in}}{\pgfqpoint{2.868891in}{2.170887in}}{\pgfqpoint{2.860655in}{2.170887in}}%
\pgfpathcurveto{\pgfqpoint{2.852418in}{2.170887in}}{\pgfqpoint{2.844518in}{2.167615in}}{\pgfqpoint{2.838694in}{2.161791in}}%
\pgfpathcurveto{\pgfqpoint{2.832871in}{2.155967in}}{\pgfqpoint{2.829598in}{2.148067in}}{\pgfqpoint{2.829598in}{2.139831in}}%
\pgfpathcurveto{\pgfqpoint{2.829598in}{2.131594in}}{\pgfqpoint{2.832871in}{2.123694in}}{\pgfqpoint{2.838694in}{2.117870in}}%
\pgfpathcurveto{\pgfqpoint{2.844518in}{2.112046in}}{\pgfqpoint{2.852418in}{2.108774in}}{\pgfqpoint{2.860655in}{2.108774in}}%
\pgfpathclose%
\pgfusepath{stroke,fill}%
\end{pgfscope}%
\begin{pgfscope}%
\pgfpathrectangle{\pgfqpoint{0.100000in}{0.212622in}}{\pgfqpoint{3.696000in}{3.696000in}}%
\pgfusepath{clip}%
\pgfsetbuttcap%
\pgfsetroundjoin%
\definecolor{currentfill}{rgb}{0.121569,0.466667,0.705882}%
\pgfsetfillcolor{currentfill}%
\pgfsetfillopacity{0.515835}%
\pgfsetlinewidth{1.003750pt}%
\definecolor{currentstroke}{rgb}{0.121569,0.466667,0.705882}%
\pgfsetstrokecolor{currentstroke}%
\pgfsetstrokeopacity{0.515835}%
\pgfsetdash{}{0pt}%
\pgfpathmoveto{\pgfqpoint{2.878093in}{2.108345in}}%
\pgfpathcurveto{\pgfqpoint{2.886329in}{2.108345in}}{\pgfqpoint{2.894229in}{2.111617in}}{\pgfqpoint{2.900053in}{2.117441in}}%
\pgfpathcurveto{\pgfqpoint{2.905877in}{2.123265in}}{\pgfqpoint{2.909150in}{2.131165in}}{\pgfqpoint{2.909150in}{2.139401in}}%
\pgfpathcurveto{\pgfqpoint{2.909150in}{2.147638in}}{\pgfqpoint{2.905877in}{2.155538in}}{\pgfqpoint{2.900053in}{2.161362in}}%
\pgfpathcurveto{\pgfqpoint{2.894229in}{2.167186in}}{\pgfqpoint{2.886329in}{2.170458in}}{\pgfqpoint{2.878093in}{2.170458in}}%
\pgfpathcurveto{\pgfqpoint{2.869857in}{2.170458in}}{\pgfqpoint{2.861957in}{2.167186in}}{\pgfqpoint{2.856133in}{2.161362in}}%
\pgfpathcurveto{\pgfqpoint{2.850309in}{2.155538in}}{\pgfqpoint{2.847037in}{2.147638in}}{\pgfqpoint{2.847037in}{2.139401in}}%
\pgfpathcurveto{\pgfqpoint{2.847037in}{2.131165in}}{\pgfqpoint{2.850309in}{2.123265in}}{\pgfqpoint{2.856133in}{2.117441in}}%
\pgfpathcurveto{\pgfqpoint{2.861957in}{2.111617in}}{\pgfqpoint{2.869857in}{2.108345in}}{\pgfqpoint{2.878093in}{2.108345in}}%
\pgfpathclose%
\pgfusepath{stroke,fill}%
\end{pgfscope}%
\begin{pgfscope}%
\pgfpathrectangle{\pgfqpoint{0.100000in}{0.212622in}}{\pgfqpoint{3.696000in}{3.696000in}}%
\pgfusepath{clip}%
\pgfsetbuttcap%
\pgfsetroundjoin%
\definecolor{currentfill}{rgb}{0.121569,0.466667,0.705882}%
\pgfsetfillcolor{currentfill}%
\pgfsetfillopacity{0.518222}%
\pgfsetlinewidth{1.003750pt}%
\definecolor{currentstroke}{rgb}{0.121569,0.466667,0.705882}%
\pgfsetstrokecolor{currentstroke}%
\pgfsetstrokeopacity{0.518222}%
\pgfsetdash{}{0pt}%
\pgfpathmoveto{\pgfqpoint{2.893751in}{2.109261in}}%
\pgfpathcurveto{\pgfqpoint{2.901987in}{2.109261in}}{\pgfqpoint{2.909887in}{2.112533in}}{\pgfqpoint{2.915711in}{2.118357in}}%
\pgfpathcurveto{\pgfqpoint{2.921535in}{2.124181in}}{\pgfqpoint{2.924808in}{2.132081in}}{\pgfqpoint{2.924808in}{2.140317in}}%
\pgfpathcurveto{\pgfqpoint{2.924808in}{2.148554in}}{\pgfqpoint{2.921535in}{2.156454in}}{\pgfqpoint{2.915711in}{2.162278in}}%
\pgfpathcurveto{\pgfqpoint{2.909887in}{2.168101in}}{\pgfqpoint{2.901987in}{2.171374in}}{\pgfqpoint{2.893751in}{2.171374in}}%
\pgfpathcurveto{\pgfqpoint{2.885515in}{2.171374in}}{\pgfqpoint{2.877615in}{2.168101in}}{\pgfqpoint{2.871791in}{2.162278in}}%
\pgfpathcurveto{\pgfqpoint{2.865967in}{2.156454in}}{\pgfqpoint{2.862695in}{2.148554in}}{\pgfqpoint{2.862695in}{2.140317in}}%
\pgfpathcurveto{\pgfqpoint{2.862695in}{2.132081in}}{\pgfqpoint{2.865967in}{2.124181in}}{\pgfqpoint{2.871791in}{2.118357in}}%
\pgfpathcurveto{\pgfqpoint{2.877615in}{2.112533in}}{\pgfqpoint{2.885515in}{2.109261in}}{\pgfqpoint{2.893751in}{2.109261in}}%
\pgfpathclose%
\pgfusepath{stroke,fill}%
\end{pgfscope}%
\begin{pgfscope}%
\pgfpathrectangle{\pgfqpoint{0.100000in}{0.212622in}}{\pgfqpoint{3.696000in}{3.696000in}}%
\pgfusepath{clip}%
\pgfsetbuttcap%
\pgfsetroundjoin%
\definecolor{currentfill}{rgb}{0.121569,0.466667,0.705882}%
\pgfsetfillcolor{currentfill}%
\pgfsetfillopacity{0.520633}%
\pgfsetlinewidth{1.003750pt}%
\definecolor{currentstroke}{rgb}{0.121569,0.466667,0.705882}%
\pgfsetstrokecolor{currentstroke}%
\pgfsetstrokeopacity{0.520633}%
\pgfsetdash{}{0pt}%
\pgfpathmoveto{\pgfqpoint{2.907952in}{2.110066in}}%
\pgfpathcurveto{\pgfqpoint{2.916188in}{2.110066in}}{\pgfqpoint{2.924088in}{2.113338in}}{\pgfqpoint{2.929912in}{2.119162in}}%
\pgfpathcurveto{\pgfqpoint{2.935736in}{2.124986in}}{\pgfqpoint{2.939008in}{2.132886in}}{\pgfqpoint{2.939008in}{2.141123in}}%
\pgfpathcurveto{\pgfqpoint{2.939008in}{2.149359in}}{\pgfqpoint{2.935736in}{2.157259in}}{\pgfqpoint{2.929912in}{2.163083in}}%
\pgfpathcurveto{\pgfqpoint{2.924088in}{2.168907in}}{\pgfqpoint{2.916188in}{2.172179in}}{\pgfqpoint{2.907952in}{2.172179in}}%
\pgfpathcurveto{\pgfqpoint{2.899715in}{2.172179in}}{\pgfqpoint{2.891815in}{2.168907in}}{\pgfqpoint{2.885991in}{2.163083in}}%
\pgfpathcurveto{\pgfqpoint{2.880167in}{2.157259in}}{\pgfqpoint{2.876895in}{2.149359in}}{\pgfqpoint{2.876895in}{2.141123in}}%
\pgfpathcurveto{\pgfqpoint{2.876895in}{2.132886in}}{\pgfqpoint{2.880167in}{2.124986in}}{\pgfqpoint{2.885991in}{2.119162in}}%
\pgfpathcurveto{\pgfqpoint{2.891815in}{2.113338in}}{\pgfqpoint{2.899715in}{2.110066in}}{\pgfqpoint{2.907952in}{2.110066in}}%
\pgfpathclose%
\pgfusepath{stroke,fill}%
\end{pgfscope}%
\begin{pgfscope}%
\pgfpathrectangle{\pgfqpoint{0.100000in}{0.212622in}}{\pgfqpoint{3.696000in}{3.696000in}}%
\pgfusepath{clip}%
\pgfsetbuttcap%
\pgfsetroundjoin%
\definecolor{currentfill}{rgb}{0.121569,0.466667,0.705882}%
\pgfsetfillcolor{currentfill}%
\pgfsetfillopacity{0.522916}%
\pgfsetlinewidth{1.003750pt}%
\definecolor{currentstroke}{rgb}{0.121569,0.466667,0.705882}%
\pgfsetstrokecolor{currentstroke}%
\pgfsetstrokeopacity{0.522916}%
\pgfsetdash{}{0pt}%
\pgfpathmoveto{\pgfqpoint{2.921502in}{2.109466in}}%
\pgfpathcurveto{\pgfqpoint{2.929739in}{2.109466in}}{\pgfqpoint{2.937639in}{2.112738in}}{\pgfqpoint{2.943463in}{2.118562in}}%
\pgfpathcurveto{\pgfqpoint{2.949287in}{2.124386in}}{\pgfqpoint{2.952559in}{2.132286in}}{\pgfqpoint{2.952559in}{2.140522in}}%
\pgfpathcurveto{\pgfqpoint{2.952559in}{2.148758in}}{\pgfqpoint{2.949287in}{2.156658in}}{\pgfqpoint{2.943463in}{2.162482in}}%
\pgfpathcurveto{\pgfqpoint{2.937639in}{2.168306in}}{\pgfqpoint{2.929739in}{2.171579in}}{\pgfqpoint{2.921502in}{2.171579in}}%
\pgfpathcurveto{\pgfqpoint{2.913266in}{2.171579in}}{\pgfqpoint{2.905366in}{2.168306in}}{\pgfqpoint{2.899542in}{2.162482in}}%
\pgfpathcurveto{\pgfqpoint{2.893718in}{2.156658in}}{\pgfqpoint{2.890446in}{2.148758in}}{\pgfqpoint{2.890446in}{2.140522in}}%
\pgfpathcurveto{\pgfqpoint{2.890446in}{2.132286in}}{\pgfqpoint{2.893718in}{2.124386in}}{\pgfqpoint{2.899542in}{2.118562in}}%
\pgfpathcurveto{\pgfqpoint{2.905366in}{2.112738in}}{\pgfqpoint{2.913266in}{2.109466in}}{\pgfqpoint{2.921502in}{2.109466in}}%
\pgfpathclose%
\pgfusepath{stroke,fill}%
\end{pgfscope}%
\begin{pgfscope}%
\pgfpathrectangle{\pgfqpoint{0.100000in}{0.212622in}}{\pgfqpoint{3.696000in}{3.696000in}}%
\pgfusepath{clip}%
\pgfsetbuttcap%
\pgfsetroundjoin%
\definecolor{currentfill}{rgb}{0.121569,0.466667,0.705882}%
\pgfsetfillcolor{currentfill}%
\pgfsetfillopacity{0.525071}%
\pgfsetlinewidth{1.003750pt}%
\definecolor{currentstroke}{rgb}{0.121569,0.466667,0.705882}%
\pgfsetstrokecolor{currentstroke}%
\pgfsetstrokeopacity{0.525071}%
\pgfsetdash{}{0pt}%
\pgfpathmoveto{\pgfqpoint{2.933881in}{2.107537in}}%
\pgfpathcurveto{\pgfqpoint{2.942118in}{2.107537in}}{\pgfqpoint{2.950018in}{2.110809in}}{\pgfqpoint{2.955842in}{2.116633in}}%
\pgfpathcurveto{\pgfqpoint{2.961666in}{2.122457in}}{\pgfqpoint{2.964938in}{2.130357in}}{\pgfqpoint{2.964938in}{2.138594in}}%
\pgfpathcurveto{\pgfqpoint{2.964938in}{2.146830in}}{\pgfqpoint{2.961666in}{2.154730in}}{\pgfqpoint{2.955842in}{2.160554in}}%
\pgfpathcurveto{\pgfqpoint{2.950018in}{2.166378in}}{\pgfqpoint{2.942118in}{2.169650in}}{\pgfqpoint{2.933881in}{2.169650in}}%
\pgfpathcurveto{\pgfqpoint{2.925645in}{2.169650in}}{\pgfqpoint{2.917745in}{2.166378in}}{\pgfqpoint{2.911921in}{2.160554in}}%
\pgfpathcurveto{\pgfqpoint{2.906097in}{2.154730in}}{\pgfqpoint{2.902825in}{2.146830in}}{\pgfqpoint{2.902825in}{2.138594in}}%
\pgfpathcurveto{\pgfqpoint{2.902825in}{2.130357in}}{\pgfqpoint{2.906097in}{2.122457in}}{\pgfqpoint{2.911921in}{2.116633in}}%
\pgfpathcurveto{\pgfqpoint{2.917745in}{2.110809in}}{\pgfqpoint{2.925645in}{2.107537in}}{\pgfqpoint{2.933881in}{2.107537in}}%
\pgfpathclose%
\pgfusepath{stroke,fill}%
\end{pgfscope}%
\begin{pgfscope}%
\pgfpathrectangle{\pgfqpoint{0.100000in}{0.212622in}}{\pgfqpoint{3.696000in}{3.696000in}}%
\pgfusepath{clip}%
\pgfsetbuttcap%
\pgfsetroundjoin%
\definecolor{currentfill}{rgb}{0.121569,0.466667,0.705882}%
\pgfsetfillcolor{currentfill}%
\pgfsetfillopacity{0.526986}%
\pgfsetlinewidth{1.003750pt}%
\definecolor{currentstroke}{rgb}{0.121569,0.466667,0.705882}%
\pgfsetstrokecolor{currentstroke}%
\pgfsetstrokeopacity{0.526986}%
\pgfsetdash{}{0pt}%
\pgfpathmoveto{\pgfqpoint{2.944794in}{2.105147in}}%
\pgfpathcurveto{\pgfqpoint{2.953031in}{2.105147in}}{\pgfqpoint{2.960931in}{2.108419in}}{\pgfqpoint{2.966755in}{2.114243in}}%
\pgfpathcurveto{\pgfqpoint{2.972579in}{2.120067in}}{\pgfqpoint{2.975851in}{2.127967in}}{\pgfqpoint{2.975851in}{2.136204in}}%
\pgfpathcurveto{\pgfqpoint{2.975851in}{2.144440in}}{\pgfqpoint{2.972579in}{2.152340in}}{\pgfqpoint{2.966755in}{2.158164in}}%
\pgfpathcurveto{\pgfqpoint{2.960931in}{2.163988in}}{\pgfqpoint{2.953031in}{2.167260in}}{\pgfqpoint{2.944794in}{2.167260in}}%
\pgfpathcurveto{\pgfqpoint{2.936558in}{2.167260in}}{\pgfqpoint{2.928658in}{2.163988in}}{\pgfqpoint{2.922834in}{2.158164in}}%
\pgfpathcurveto{\pgfqpoint{2.917010in}{2.152340in}}{\pgfqpoint{2.913738in}{2.144440in}}{\pgfqpoint{2.913738in}{2.136204in}}%
\pgfpathcurveto{\pgfqpoint{2.913738in}{2.127967in}}{\pgfqpoint{2.917010in}{2.120067in}}{\pgfqpoint{2.922834in}{2.114243in}}%
\pgfpathcurveto{\pgfqpoint{2.928658in}{2.108419in}}{\pgfqpoint{2.936558in}{2.105147in}}{\pgfqpoint{2.944794in}{2.105147in}}%
\pgfpathclose%
\pgfusepath{stroke,fill}%
\end{pgfscope}%
\begin{pgfscope}%
\pgfpathrectangle{\pgfqpoint{0.100000in}{0.212622in}}{\pgfqpoint{3.696000in}{3.696000in}}%
\pgfusepath{clip}%
\pgfsetbuttcap%
\pgfsetroundjoin%
\definecolor{currentfill}{rgb}{0.121569,0.466667,0.705882}%
\pgfsetfillcolor{currentfill}%
\pgfsetfillopacity{0.528696}%
\pgfsetlinewidth{1.003750pt}%
\definecolor{currentstroke}{rgb}{0.121569,0.466667,0.705882}%
\pgfsetstrokecolor{currentstroke}%
\pgfsetstrokeopacity{0.528696}%
\pgfsetdash{}{0pt}%
\pgfpathmoveto{\pgfqpoint{2.954471in}{2.103041in}}%
\pgfpathcurveto{\pgfqpoint{2.962707in}{2.103041in}}{\pgfqpoint{2.970607in}{2.106313in}}{\pgfqpoint{2.976431in}{2.112137in}}%
\pgfpathcurveto{\pgfqpoint{2.982255in}{2.117961in}}{\pgfqpoint{2.985528in}{2.125861in}}{\pgfqpoint{2.985528in}{2.134098in}}%
\pgfpathcurveto{\pgfqpoint{2.985528in}{2.142334in}}{\pgfqpoint{2.982255in}{2.150234in}}{\pgfqpoint{2.976431in}{2.156058in}}%
\pgfpathcurveto{\pgfqpoint{2.970607in}{2.161882in}}{\pgfqpoint{2.962707in}{2.165154in}}{\pgfqpoint{2.954471in}{2.165154in}}%
\pgfpathcurveto{\pgfqpoint{2.946235in}{2.165154in}}{\pgfqpoint{2.938335in}{2.161882in}}{\pgfqpoint{2.932511in}{2.156058in}}%
\pgfpathcurveto{\pgfqpoint{2.926687in}{2.150234in}}{\pgfqpoint{2.923415in}{2.142334in}}{\pgfqpoint{2.923415in}{2.134098in}}%
\pgfpathcurveto{\pgfqpoint{2.923415in}{2.125861in}}{\pgfqpoint{2.926687in}{2.117961in}}{\pgfqpoint{2.932511in}{2.112137in}}%
\pgfpathcurveto{\pgfqpoint{2.938335in}{2.106313in}}{\pgfqpoint{2.946235in}{2.103041in}}{\pgfqpoint{2.954471in}{2.103041in}}%
\pgfpathclose%
\pgfusepath{stroke,fill}%
\end{pgfscope}%
\begin{pgfscope}%
\pgfpathrectangle{\pgfqpoint{0.100000in}{0.212622in}}{\pgfqpoint{3.696000in}{3.696000in}}%
\pgfusepath{clip}%
\pgfsetbuttcap%
\pgfsetroundjoin%
\definecolor{currentfill}{rgb}{0.121569,0.466667,0.705882}%
\pgfsetfillcolor{currentfill}%
\pgfsetfillopacity{0.530399}%
\pgfsetlinewidth{1.003750pt}%
\definecolor{currentstroke}{rgb}{0.121569,0.466667,0.705882}%
\pgfsetstrokecolor{currentstroke}%
\pgfsetstrokeopacity{0.530399}%
\pgfsetdash{}{0pt}%
\pgfpathmoveto{\pgfqpoint{2.963980in}{2.102051in}}%
\pgfpathcurveto{\pgfqpoint{2.972217in}{2.102051in}}{\pgfqpoint{2.980117in}{2.105324in}}{\pgfqpoint{2.985941in}{2.111148in}}%
\pgfpathcurveto{\pgfqpoint{2.991765in}{2.116972in}}{\pgfqpoint{2.995037in}{2.124872in}}{\pgfqpoint{2.995037in}{2.133108in}}%
\pgfpathcurveto{\pgfqpoint{2.995037in}{2.141344in}}{\pgfqpoint{2.991765in}{2.149244in}}{\pgfqpoint{2.985941in}{2.155068in}}%
\pgfpathcurveto{\pgfqpoint{2.980117in}{2.160892in}}{\pgfqpoint{2.972217in}{2.164164in}}{\pgfqpoint{2.963980in}{2.164164in}}%
\pgfpathcurveto{\pgfqpoint{2.955744in}{2.164164in}}{\pgfqpoint{2.947844in}{2.160892in}}{\pgfqpoint{2.942020in}{2.155068in}}%
\pgfpathcurveto{\pgfqpoint{2.936196in}{2.149244in}}{\pgfqpoint{2.932924in}{2.141344in}}{\pgfqpoint{2.932924in}{2.133108in}}%
\pgfpathcurveto{\pgfqpoint{2.932924in}{2.124872in}}{\pgfqpoint{2.936196in}{2.116972in}}{\pgfqpoint{2.942020in}{2.111148in}}%
\pgfpathcurveto{\pgfqpoint{2.947844in}{2.105324in}}{\pgfqpoint{2.955744in}{2.102051in}}{\pgfqpoint{2.963980in}{2.102051in}}%
\pgfpathclose%
\pgfusepath{stroke,fill}%
\end{pgfscope}%
\begin{pgfscope}%
\pgfpathrectangle{\pgfqpoint{0.100000in}{0.212622in}}{\pgfqpoint{3.696000in}{3.696000in}}%
\pgfusepath{clip}%
\pgfsetbuttcap%
\pgfsetroundjoin%
\definecolor{currentfill}{rgb}{0.121569,0.466667,0.705882}%
\pgfsetfillcolor{currentfill}%
\pgfsetfillopacity{0.532033}%
\pgfsetlinewidth{1.003750pt}%
\definecolor{currentstroke}{rgb}{0.121569,0.466667,0.705882}%
\pgfsetstrokecolor{currentstroke}%
\pgfsetstrokeopacity{0.532033}%
\pgfsetdash{}{0pt}%
\pgfpathmoveto{\pgfqpoint{2.972873in}{2.101283in}}%
\pgfpathcurveto{\pgfqpoint{2.981109in}{2.101283in}}{\pgfqpoint{2.989009in}{2.104556in}}{\pgfqpoint{2.994833in}{2.110379in}}%
\pgfpathcurveto{\pgfqpoint{3.000657in}{2.116203in}}{\pgfqpoint{3.003929in}{2.124103in}}{\pgfqpoint{3.003929in}{2.132340in}}%
\pgfpathcurveto{\pgfqpoint{3.003929in}{2.140576in}}{\pgfqpoint{3.000657in}{2.148476in}}{\pgfqpoint{2.994833in}{2.154300in}}%
\pgfpathcurveto{\pgfqpoint{2.989009in}{2.160124in}}{\pgfqpoint{2.981109in}{2.163396in}}{\pgfqpoint{2.972873in}{2.163396in}}%
\pgfpathcurveto{\pgfqpoint{2.964636in}{2.163396in}}{\pgfqpoint{2.956736in}{2.160124in}}{\pgfqpoint{2.950912in}{2.154300in}}%
\pgfpathcurveto{\pgfqpoint{2.945088in}{2.148476in}}{\pgfqpoint{2.941816in}{2.140576in}}{\pgfqpoint{2.941816in}{2.132340in}}%
\pgfpathcurveto{\pgfqpoint{2.941816in}{2.124103in}}{\pgfqpoint{2.945088in}{2.116203in}}{\pgfqpoint{2.950912in}{2.110379in}}%
\pgfpathcurveto{\pgfqpoint{2.956736in}{2.104556in}}{\pgfqpoint{2.964636in}{2.101283in}}{\pgfqpoint{2.972873in}{2.101283in}}%
\pgfpathclose%
\pgfusepath{stroke,fill}%
\end{pgfscope}%
\begin{pgfscope}%
\pgfpathrectangle{\pgfqpoint{0.100000in}{0.212622in}}{\pgfqpoint{3.696000in}{3.696000in}}%
\pgfusepath{clip}%
\pgfsetbuttcap%
\pgfsetroundjoin%
\definecolor{currentfill}{rgb}{0.121569,0.466667,0.705882}%
\pgfsetfillcolor{currentfill}%
\pgfsetfillopacity{0.535051}%
\pgfsetlinewidth{1.003750pt}%
\definecolor{currentstroke}{rgb}{0.121569,0.466667,0.705882}%
\pgfsetstrokecolor{currentstroke}%
\pgfsetstrokeopacity{0.535051}%
\pgfsetdash{}{0pt}%
\pgfpathmoveto{\pgfqpoint{2.989032in}{2.100067in}}%
\pgfpathcurveto{\pgfqpoint{2.997268in}{2.100067in}}{\pgfqpoint{3.005168in}{2.103340in}}{\pgfqpoint{3.010992in}{2.109164in}}%
\pgfpathcurveto{\pgfqpoint{3.016816in}{2.114988in}}{\pgfqpoint{3.020088in}{2.122888in}}{\pgfqpoint{3.020088in}{2.131124in}}%
\pgfpathcurveto{\pgfqpoint{3.020088in}{2.139360in}}{\pgfqpoint{3.016816in}{2.147260in}}{\pgfqpoint{3.010992in}{2.153084in}}%
\pgfpathcurveto{\pgfqpoint{3.005168in}{2.158908in}}{\pgfqpoint{2.997268in}{2.162180in}}{\pgfqpoint{2.989032in}{2.162180in}}%
\pgfpathcurveto{\pgfqpoint{2.980795in}{2.162180in}}{\pgfqpoint{2.972895in}{2.158908in}}{\pgfqpoint{2.967072in}{2.153084in}}%
\pgfpathcurveto{\pgfqpoint{2.961248in}{2.147260in}}{\pgfqpoint{2.957975in}{2.139360in}}{\pgfqpoint{2.957975in}{2.131124in}}%
\pgfpathcurveto{\pgfqpoint{2.957975in}{2.122888in}}{\pgfqpoint{2.961248in}{2.114988in}}{\pgfqpoint{2.967072in}{2.109164in}}%
\pgfpathcurveto{\pgfqpoint{2.972895in}{2.103340in}}{\pgfqpoint{2.980795in}{2.100067in}}{\pgfqpoint{2.989032in}{2.100067in}}%
\pgfpathclose%
\pgfusepath{stroke,fill}%
\end{pgfscope}%
\begin{pgfscope}%
\pgfpathrectangle{\pgfqpoint{0.100000in}{0.212622in}}{\pgfqpoint{3.696000in}{3.696000in}}%
\pgfusepath{clip}%
\pgfsetbuttcap%
\pgfsetroundjoin%
\definecolor{currentfill}{rgb}{0.121569,0.466667,0.705882}%
\pgfsetfillcolor{currentfill}%
\pgfsetfillopacity{0.537749}%
\pgfsetlinewidth{1.003750pt}%
\definecolor{currentstroke}{rgb}{0.121569,0.466667,0.705882}%
\pgfsetstrokecolor{currentstroke}%
\pgfsetstrokeopacity{0.537749}%
\pgfsetdash{}{0pt}%
\pgfpathmoveto{\pgfqpoint{3.004267in}{2.099877in}}%
\pgfpathcurveto{\pgfqpoint{3.012504in}{2.099877in}}{\pgfqpoint{3.020404in}{2.103150in}}{\pgfqpoint{3.026228in}{2.108974in}}%
\pgfpathcurveto{\pgfqpoint{3.032052in}{2.114798in}}{\pgfqpoint{3.035324in}{2.122698in}}{\pgfqpoint{3.035324in}{2.130934in}}%
\pgfpathcurveto{\pgfqpoint{3.035324in}{2.139170in}}{\pgfqpoint{3.032052in}{2.147070in}}{\pgfqpoint{3.026228in}{2.152894in}}%
\pgfpathcurveto{\pgfqpoint{3.020404in}{2.158718in}}{\pgfqpoint{3.012504in}{2.161990in}}{\pgfqpoint{3.004267in}{2.161990in}}%
\pgfpathcurveto{\pgfqpoint{2.996031in}{2.161990in}}{\pgfqpoint{2.988131in}{2.158718in}}{\pgfqpoint{2.982307in}{2.152894in}}%
\pgfpathcurveto{\pgfqpoint{2.976483in}{2.147070in}}{\pgfqpoint{2.973211in}{2.139170in}}{\pgfqpoint{2.973211in}{2.130934in}}%
\pgfpathcurveto{\pgfqpoint{2.973211in}{2.122698in}}{\pgfqpoint{2.976483in}{2.114798in}}{\pgfqpoint{2.982307in}{2.108974in}}%
\pgfpathcurveto{\pgfqpoint{2.988131in}{2.103150in}}{\pgfqpoint{2.996031in}{2.099877in}}{\pgfqpoint{3.004267in}{2.099877in}}%
\pgfpathclose%
\pgfusepath{stroke,fill}%
\end{pgfscope}%
\begin{pgfscope}%
\pgfpathrectangle{\pgfqpoint{0.100000in}{0.212622in}}{\pgfqpoint{3.696000in}{3.696000in}}%
\pgfusepath{clip}%
\pgfsetbuttcap%
\pgfsetroundjoin%
\definecolor{currentfill}{rgb}{0.121569,0.466667,0.705882}%
\pgfsetfillcolor{currentfill}%
\pgfsetfillopacity{0.539986}%
\pgfsetlinewidth{1.003750pt}%
\definecolor{currentstroke}{rgb}{0.121569,0.466667,0.705882}%
\pgfsetstrokecolor{currentstroke}%
\pgfsetstrokeopacity{0.539986}%
\pgfsetdash{}{0pt}%
\pgfpathmoveto{\pgfqpoint{3.017297in}{2.099549in}}%
\pgfpathcurveto{\pgfqpoint{3.025533in}{2.099549in}}{\pgfqpoint{3.033433in}{2.102821in}}{\pgfqpoint{3.039257in}{2.108645in}}%
\pgfpathcurveto{\pgfqpoint{3.045081in}{2.114469in}}{\pgfqpoint{3.048354in}{2.122369in}}{\pgfqpoint{3.048354in}{2.130606in}}%
\pgfpathcurveto{\pgfqpoint{3.048354in}{2.138842in}}{\pgfqpoint{3.045081in}{2.146742in}}{\pgfqpoint{3.039257in}{2.152566in}}%
\pgfpathcurveto{\pgfqpoint{3.033433in}{2.158390in}}{\pgfqpoint{3.025533in}{2.161662in}}{\pgfqpoint{3.017297in}{2.161662in}}%
\pgfpathcurveto{\pgfqpoint{3.009061in}{2.161662in}}{\pgfqpoint{3.001161in}{2.158390in}}{\pgfqpoint{2.995337in}{2.152566in}}%
\pgfpathcurveto{\pgfqpoint{2.989513in}{2.146742in}}{\pgfqpoint{2.986241in}{2.138842in}}{\pgfqpoint{2.986241in}{2.130606in}}%
\pgfpathcurveto{\pgfqpoint{2.986241in}{2.122369in}}{\pgfqpoint{2.989513in}{2.114469in}}{\pgfqpoint{2.995337in}{2.108645in}}%
\pgfpathcurveto{\pgfqpoint{3.001161in}{2.102821in}}{\pgfqpoint{3.009061in}{2.099549in}}{\pgfqpoint{3.017297in}{2.099549in}}%
\pgfpathclose%
\pgfusepath{stroke,fill}%
\end{pgfscope}%
\begin{pgfscope}%
\pgfpathrectangle{\pgfqpoint{0.100000in}{0.212622in}}{\pgfqpoint{3.696000in}{3.696000in}}%
\pgfusepath{clip}%
\pgfsetbuttcap%
\pgfsetroundjoin%
\definecolor{currentfill}{rgb}{0.121569,0.466667,0.705882}%
\pgfsetfillcolor{currentfill}%
\pgfsetfillopacity{0.541880}%
\pgfsetlinewidth{1.003750pt}%
\definecolor{currentstroke}{rgb}{0.121569,0.466667,0.705882}%
\pgfsetstrokecolor{currentstroke}%
\pgfsetstrokeopacity{0.541880}%
\pgfsetdash{}{0pt}%
\pgfpathmoveto{\pgfqpoint{3.029547in}{2.099405in}}%
\pgfpathcurveto{\pgfqpoint{3.037783in}{2.099405in}}{\pgfqpoint{3.045683in}{2.102677in}}{\pgfqpoint{3.051507in}{2.108501in}}%
\pgfpathcurveto{\pgfqpoint{3.057331in}{2.114325in}}{\pgfqpoint{3.060603in}{2.122225in}}{\pgfqpoint{3.060603in}{2.130462in}}%
\pgfpathcurveto{\pgfqpoint{3.060603in}{2.138698in}}{\pgfqpoint{3.057331in}{2.146598in}}{\pgfqpoint{3.051507in}{2.152422in}}%
\pgfpathcurveto{\pgfqpoint{3.045683in}{2.158246in}}{\pgfqpoint{3.037783in}{2.161518in}}{\pgfqpoint{3.029547in}{2.161518in}}%
\pgfpathcurveto{\pgfqpoint{3.021310in}{2.161518in}}{\pgfqpoint{3.013410in}{2.158246in}}{\pgfqpoint{3.007586in}{2.152422in}}%
\pgfpathcurveto{\pgfqpoint{3.001762in}{2.146598in}}{\pgfqpoint{2.998490in}{2.138698in}}{\pgfqpoint{2.998490in}{2.130462in}}%
\pgfpathcurveto{\pgfqpoint{2.998490in}{2.122225in}}{\pgfqpoint{3.001762in}{2.114325in}}{\pgfqpoint{3.007586in}{2.108501in}}%
\pgfpathcurveto{\pgfqpoint{3.013410in}{2.102677in}}{\pgfqpoint{3.021310in}{2.099405in}}{\pgfqpoint{3.029547in}{2.099405in}}%
\pgfpathclose%
\pgfusepath{stroke,fill}%
\end{pgfscope}%
\begin{pgfscope}%
\pgfpathrectangle{\pgfqpoint{0.100000in}{0.212622in}}{\pgfqpoint{3.696000in}{3.696000in}}%
\pgfusepath{clip}%
\pgfsetbuttcap%
\pgfsetroundjoin%
\definecolor{currentfill}{rgb}{0.121569,0.466667,0.705882}%
\pgfsetfillcolor{currentfill}%
\pgfsetfillopacity{0.543638}%
\pgfsetlinewidth{1.003750pt}%
\definecolor{currentstroke}{rgb}{0.121569,0.466667,0.705882}%
\pgfsetstrokecolor{currentstroke}%
\pgfsetstrokeopacity{0.543638}%
\pgfsetdash{}{0pt}%
\pgfpathmoveto{\pgfqpoint{3.041179in}{2.098900in}}%
\pgfpathcurveto{\pgfqpoint{3.049415in}{2.098900in}}{\pgfqpoint{3.057315in}{2.102172in}}{\pgfqpoint{3.063139in}{2.107996in}}%
\pgfpathcurveto{\pgfqpoint{3.068963in}{2.113820in}}{\pgfqpoint{3.072236in}{2.121720in}}{\pgfqpoint{3.072236in}{2.129956in}}%
\pgfpathcurveto{\pgfqpoint{3.072236in}{2.138192in}}{\pgfqpoint{3.068963in}{2.146092in}}{\pgfqpoint{3.063139in}{2.151916in}}%
\pgfpathcurveto{\pgfqpoint{3.057315in}{2.157740in}}{\pgfqpoint{3.049415in}{2.161013in}}{\pgfqpoint{3.041179in}{2.161013in}}%
\pgfpathcurveto{\pgfqpoint{3.032943in}{2.161013in}}{\pgfqpoint{3.025043in}{2.157740in}}{\pgfqpoint{3.019219in}{2.151916in}}%
\pgfpathcurveto{\pgfqpoint{3.013395in}{2.146092in}}{\pgfqpoint{3.010123in}{2.138192in}}{\pgfqpoint{3.010123in}{2.129956in}}%
\pgfpathcurveto{\pgfqpoint{3.010123in}{2.121720in}}{\pgfqpoint{3.013395in}{2.113820in}}{\pgfqpoint{3.019219in}{2.107996in}}%
\pgfpathcurveto{\pgfqpoint{3.025043in}{2.102172in}}{\pgfqpoint{3.032943in}{2.098900in}}{\pgfqpoint{3.041179in}{2.098900in}}%
\pgfpathclose%
\pgfusepath{stroke,fill}%
\end{pgfscope}%
\begin{pgfscope}%
\pgfpathrectangle{\pgfqpoint{0.100000in}{0.212622in}}{\pgfqpoint{3.696000in}{3.696000in}}%
\pgfusepath{clip}%
\pgfsetbuttcap%
\pgfsetroundjoin%
\definecolor{currentfill}{rgb}{0.121569,0.466667,0.705882}%
\pgfsetfillcolor{currentfill}%
\pgfsetfillopacity{0.545330}%
\pgfsetlinewidth{1.003750pt}%
\definecolor{currentstroke}{rgb}{0.121569,0.466667,0.705882}%
\pgfsetstrokecolor{currentstroke}%
\pgfsetstrokeopacity{0.545330}%
\pgfsetdash{}{0pt}%
\pgfpathmoveto{\pgfqpoint{3.052522in}{2.098259in}}%
\pgfpathcurveto{\pgfqpoint{3.060759in}{2.098259in}}{\pgfqpoint{3.068659in}{2.101531in}}{\pgfqpoint{3.074483in}{2.107355in}}%
\pgfpathcurveto{\pgfqpoint{3.080307in}{2.113179in}}{\pgfqpoint{3.083579in}{2.121079in}}{\pgfqpoint{3.083579in}{2.129316in}}%
\pgfpathcurveto{\pgfqpoint{3.083579in}{2.137552in}}{\pgfqpoint{3.080307in}{2.145452in}}{\pgfqpoint{3.074483in}{2.151276in}}%
\pgfpathcurveto{\pgfqpoint{3.068659in}{2.157100in}}{\pgfqpoint{3.060759in}{2.160372in}}{\pgfqpoint{3.052522in}{2.160372in}}%
\pgfpathcurveto{\pgfqpoint{3.044286in}{2.160372in}}{\pgfqpoint{3.036386in}{2.157100in}}{\pgfqpoint{3.030562in}{2.151276in}}%
\pgfpathcurveto{\pgfqpoint{3.024738in}{2.145452in}}{\pgfqpoint{3.021466in}{2.137552in}}{\pgfqpoint{3.021466in}{2.129316in}}%
\pgfpathcurveto{\pgfqpoint{3.021466in}{2.121079in}}{\pgfqpoint{3.024738in}{2.113179in}}{\pgfqpoint{3.030562in}{2.107355in}}%
\pgfpathcurveto{\pgfqpoint{3.036386in}{2.101531in}}{\pgfqpoint{3.044286in}{2.098259in}}{\pgfqpoint{3.052522in}{2.098259in}}%
\pgfpathclose%
\pgfusepath{stroke,fill}%
\end{pgfscope}%
\begin{pgfscope}%
\pgfpathrectangle{\pgfqpoint{0.100000in}{0.212622in}}{\pgfqpoint{3.696000in}{3.696000in}}%
\pgfusepath{clip}%
\pgfsetbuttcap%
\pgfsetroundjoin%
\definecolor{currentfill}{rgb}{0.121569,0.466667,0.705882}%
\pgfsetfillcolor{currentfill}%
\pgfsetfillopacity{0.546868}%
\pgfsetlinewidth{1.003750pt}%
\definecolor{currentstroke}{rgb}{0.121569,0.466667,0.705882}%
\pgfsetstrokecolor{currentstroke}%
\pgfsetstrokeopacity{0.546868}%
\pgfsetdash{}{0pt}%
\pgfpathmoveto{\pgfqpoint{3.062895in}{2.097488in}}%
\pgfpathcurveto{\pgfqpoint{3.071131in}{2.097488in}}{\pgfqpoint{3.079031in}{2.100760in}}{\pgfqpoint{3.084855in}{2.106584in}}%
\pgfpathcurveto{\pgfqpoint{3.090679in}{2.112408in}}{\pgfqpoint{3.093952in}{2.120308in}}{\pgfqpoint{3.093952in}{2.128544in}}%
\pgfpathcurveto{\pgfqpoint{3.093952in}{2.136781in}}{\pgfqpoint{3.090679in}{2.144681in}}{\pgfqpoint{3.084855in}{2.150505in}}%
\pgfpathcurveto{\pgfqpoint{3.079031in}{2.156328in}}{\pgfqpoint{3.071131in}{2.159601in}}{\pgfqpoint{3.062895in}{2.159601in}}%
\pgfpathcurveto{\pgfqpoint{3.054659in}{2.159601in}}{\pgfqpoint{3.046759in}{2.156328in}}{\pgfqpoint{3.040935in}{2.150505in}}%
\pgfpathcurveto{\pgfqpoint{3.035111in}{2.144681in}}{\pgfqpoint{3.031839in}{2.136781in}}{\pgfqpoint{3.031839in}{2.128544in}}%
\pgfpathcurveto{\pgfqpoint{3.031839in}{2.120308in}}{\pgfqpoint{3.035111in}{2.112408in}}{\pgfqpoint{3.040935in}{2.106584in}}%
\pgfpathcurveto{\pgfqpoint{3.046759in}{2.100760in}}{\pgfqpoint{3.054659in}{2.097488in}}{\pgfqpoint{3.062895in}{2.097488in}}%
\pgfpathclose%
\pgfusepath{stroke,fill}%
\end{pgfscope}%
\begin{pgfscope}%
\pgfpathrectangle{\pgfqpoint{0.100000in}{0.212622in}}{\pgfqpoint{3.696000in}{3.696000in}}%
\pgfusepath{clip}%
\pgfsetbuttcap%
\pgfsetroundjoin%
\definecolor{currentfill}{rgb}{0.121569,0.466667,0.705882}%
\pgfsetfillcolor{currentfill}%
\pgfsetfillopacity{0.548210}%
\pgfsetlinewidth{1.003750pt}%
\definecolor{currentstroke}{rgb}{0.121569,0.466667,0.705882}%
\pgfsetstrokecolor{currentstroke}%
\pgfsetstrokeopacity{0.548210}%
\pgfsetdash{}{0pt}%
\pgfpathmoveto{\pgfqpoint{3.071737in}{2.096740in}}%
\pgfpathcurveto{\pgfqpoint{3.079974in}{2.096740in}}{\pgfqpoint{3.087874in}{2.100013in}}{\pgfqpoint{3.093698in}{2.105837in}}%
\pgfpathcurveto{\pgfqpoint{3.099522in}{2.111661in}}{\pgfqpoint{3.102794in}{2.119561in}}{\pgfqpoint{3.102794in}{2.127797in}}%
\pgfpathcurveto{\pgfqpoint{3.102794in}{2.136033in}}{\pgfqpoint{3.099522in}{2.143933in}}{\pgfqpoint{3.093698in}{2.149757in}}%
\pgfpathcurveto{\pgfqpoint{3.087874in}{2.155581in}}{\pgfqpoint{3.079974in}{2.158853in}}{\pgfqpoint{3.071737in}{2.158853in}}%
\pgfpathcurveto{\pgfqpoint{3.063501in}{2.158853in}}{\pgfqpoint{3.055601in}{2.155581in}}{\pgfqpoint{3.049777in}{2.149757in}}%
\pgfpathcurveto{\pgfqpoint{3.043953in}{2.143933in}}{\pgfqpoint{3.040681in}{2.136033in}}{\pgfqpoint{3.040681in}{2.127797in}}%
\pgfpathcurveto{\pgfqpoint{3.040681in}{2.119561in}}{\pgfqpoint{3.043953in}{2.111661in}}{\pgfqpoint{3.049777in}{2.105837in}}%
\pgfpathcurveto{\pgfqpoint{3.055601in}{2.100013in}}{\pgfqpoint{3.063501in}{2.096740in}}{\pgfqpoint{3.071737in}{2.096740in}}%
\pgfpathclose%
\pgfusepath{stroke,fill}%
\end{pgfscope}%
\begin{pgfscope}%
\pgfpathrectangle{\pgfqpoint{0.100000in}{0.212622in}}{\pgfqpoint{3.696000in}{3.696000in}}%
\pgfusepath{clip}%
\pgfsetbuttcap%
\pgfsetroundjoin%
\definecolor{currentfill}{rgb}{0.121569,0.466667,0.705882}%
\pgfsetfillcolor{currentfill}%
\pgfsetfillopacity{0.549430}%
\pgfsetlinewidth{1.003750pt}%
\definecolor{currentstroke}{rgb}{0.121569,0.466667,0.705882}%
\pgfsetstrokecolor{currentstroke}%
\pgfsetstrokeopacity{0.549430}%
\pgfsetdash{}{0pt}%
\pgfpathmoveto{\pgfqpoint{3.079966in}{2.095418in}}%
\pgfpathcurveto{\pgfqpoint{3.088203in}{2.095418in}}{\pgfqpoint{3.096103in}{2.098690in}}{\pgfqpoint{3.101927in}{2.104514in}}%
\pgfpathcurveto{\pgfqpoint{3.107750in}{2.110338in}}{\pgfqpoint{3.111023in}{2.118238in}}{\pgfqpoint{3.111023in}{2.126474in}}%
\pgfpathcurveto{\pgfqpoint{3.111023in}{2.134711in}}{\pgfqpoint{3.107750in}{2.142611in}}{\pgfqpoint{3.101927in}{2.148435in}}%
\pgfpathcurveto{\pgfqpoint{3.096103in}{2.154258in}}{\pgfqpoint{3.088203in}{2.157531in}}{\pgfqpoint{3.079966in}{2.157531in}}%
\pgfpathcurveto{\pgfqpoint{3.071730in}{2.157531in}}{\pgfqpoint{3.063830in}{2.154258in}}{\pgfqpoint{3.058006in}{2.148435in}}%
\pgfpathcurveto{\pgfqpoint{3.052182in}{2.142611in}}{\pgfqpoint{3.048910in}{2.134711in}}{\pgfqpoint{3.048910in}{2.126474in}}%
\pgfpathcurveto{\pgfqpoint{3.048910in}{2.118238in}}{\pgfqpoint{3.052182in}{2.110338in}}{\pgfqpoint{3.058006in}{2.104514in}}%
\pgfpathcurveto{\pgfqpoint{3.063830in}{2.098690in}}{\pgfqpoint{3.071730in}{2.095418in}}{\pgfqpoint{3.079966in}{2.095418in}}%
\pgfpathclose%
\pgfusepath{stroke,fill}%
\end{pgfscope}%
\begin{pgfscope}%
\pgfpathrectangle{\pgfqpoint{0.100000in}{0.212622in}}{\pgfqpoint{3.696000in}{3.696000in}}%
\pgfusepath{clip}%
\pgfsetbuttcap%
\pgfsetroundjoin%
\definecolor{currentfill}{rgb}{0.121569,0.466667,0.705882}%
\pgfsetfillcolor{currentfill}%
\pgfsetfillopacity{0.550529}%
\pgfsetlinewidth{1.003750pt}%
\definecolor{currentstroke}{rgb}{0.121569,0.466667,0.705882}%
\pgfsetstrokecolor{currentstroke}%
\pgfsetstrokeopacity{0.550529}%
\pgfsetdash{}{0pt}%
\pgfpathmoveto{\pgfqpoint{3.087684in}{2.094835in}}%
\pgfpathcurveto{\pgfqpoint{3.095920in}{2.094835in}}{\pgfqpoint{3.103820in}{2.098107in}}{\pgfqpoint{3.109644in}{2.103931in}}%
\pgfpathcurveto{\pgfqpoint{3.115468in}{2.109755in}}{\pgfqpoint{3.118740in}{2.117655in}}{\pgfqpoint{3.118740in}{2.125891in}}%
\pgfpathcurveto{\pgfqpoint{3.118740in}{2.134128in}}{\pgfqpoint{3.115468in}{2.142028in}}{\pgfqpoint{3.109644in}{2.147852in}}%
\pgfpathcurveto{\pgfqpoint{3.103820in}{2.153676in}}{\pgfqpoint{3.095920in}{2.156948in}}{\pgfqpoint{3.087684in}{2.156948in}}%
\pgfpathcurveto{\pgfqpoint{3.079448in}{2.156948in}}{\pgfqpoint{3.071548in}{2.153676in}}{\pgfqpoint{3.065724in}{2.147852in}}%
\pgfpathcurveto{\pgfqpoint{3.059900in}{2.142028in}}{\pgfqpoint{3.056627in}{2.134128in}}{\pgfqpoint{3.056627in}{2.125891in}}%
\pgfpathcurveto{\pgfqpoint{3.056627in}{2.117655in}}{\pgfqpoint{3.059900in}{2.109755in}}{\pgfqpoint{3.065724in}{2.103931in}}%
\pgfpathcurveto{\pgfqpoint{3.071548in}{2.098107in}}{\pgfqpoint{3.079448in}{2.094835in}}{\pgfqpoint{3.087684in}{2.094835in}}%
\pgfpathclose%
\pgfusepath{stroke,fill}%
\end{pgfscope}%
\begin{pgfscope}%
\pgfpathrectangle{\pgfqpoint{0.100000in}{0.212622in}}{\pgfqpoint{3.696000in}{3.696000in}}%
\pgfusepath{clip}%
\pgfsetbuttcap%
\pgfsetroundjoin%
\definecolor{currentfill}{rgb}{0.121569,0.466667,0.705882}%
\pgfsetfillcolor{currentfill}%
\pgfsetfillopacity{0.552565}%
\pgfsetlinewidth{1.003750pt}%
\definecolor{currentstroke}{rgb}{0.121569,0.466667,0.705882}%
\pgfsetstrokecolor{currentstroke}%
\pgfsetstrokeopacity{0.552565}%
\pgfsetdash{}{0pt}%
\pgfpathmoveto{\pgfqpoint{3.101779in}{2.094173in}}%
\pgfpathcurveto{\pgfqpoint{3.110016in}{2.094173in}}{\pgfqpoint{3.117916in}{2.097445in}}{\pgfqpoint{3.123740in}{2.103269in}}%
\pgfpathcurveto{\pgfqpoint{3.129564in}{2.109093in}}{\pgfqpoint{3.132836in}{2.116993in}}{\pgfqpoint{3.132836in}{2.125229in}}%
\pgfpathcurveto{\pgfqpoint{3.132836in}{2.133465in}}{\pgfqpoint{3.129564in}{2.141365in}}{\pgfqpoint{3.123740in}{2.147189in}}%
\pgfpathcurveto{\pgfqpoint{3.117916in}{2.153013in}}{\pgfqpoint{3.110016in}{2.156286in}}{\pgfqpoint{3.101779in}{2.156286in}}%
\pgfpathcurveto{\pgfqpoint{3.093543in}{2.156286in}}{\pgfqpoint{3.085643in}{2.153013in}}{\pgfqpoint{3.079819in}{2.147189in}}%
\pgfpathcurveto{\pgfqpoint{3.073995in}{2.141365in}}{\pgfqpoint{3.070723in}{2.133465in}}{\pgfqpoint{3.070723in}{2.125229in}}%
\pgfpathcurveto{\pgfqpoint{3.070723in}{2.116993in}}{\pgfqpoint{3.073995in}{2.109093in}}{\pgfqpoint{3.079819in}{2.103269in}}%
\pgfpathcurveto{\pgfqpoint{3.085643in}{2.097445in}}{\pgfqpoint{3.093543in}{2.094173in}}{\pgfqpoint{3.101779in}{2.094173in}}%
\pgfpathclose%
\pgfusepath{stroke,fill}%
\end{pgfscope}%
\begin{pgfscope}%
\pgfpathrectangle{\pgfqpoint{0.100000in}{0.212622in}}{\pgfqpoint{3.696000in}{3.696000in}}%
\pgfusepath{clip}%
\pgfsetbuttcap%
\pgfsetroundjoin%
\definecolor{currentfill}{rgb}{0.121569,0.466667,0.705882}%
\pgfsetfillcolor{currentfill}%
\pgfsetfillopacity{0.554386}%
\pgfsetlinewidth{1.003750pt}%
\definecolor{currentstroke}{rgb}{0.121569,0.466667,0.705882}%
\pgfsetstrokecolor{currentstroke}%
\pgfsetstrokeopacity{0.554386}%
\pgfsetdash{}{0pt}%
\pgfpathmoveto{\pgfqpoint{3.114976in}{2.094072in}}%
\pgfpathcurveto{\pgfqpoint{3.123212in}{2.094072in}}{\pgfqpoint{3.131112in}{2.097344in}}{\pgfqpoint{3.136936in}{2.103168in}}%
\pgfpathcurveto{\pgfqpoint{3.142760in}{2.108992in}}{\pgfqpoint{3.146032in}{2.116892in}}{\pgfqpoint{3.146032in}{2.125128in}}%
\pgfpathcurveto{\pgfqpoint{3.146032in}{2.133365in}}{\pgfqpoint{3.142760in}{2.141265in}}{\pgfqpoint{3.136936in}{2.147089in}}%
\pgfpathcurveto{\pgfqpoint{3.131112in}{2.152913in}}{\pgfqpoint{3.123212in}{2.156185in}}{\pgfqpoint{3.114976in}{2.156185in}}%
\pgfpathcurveto{\pgfqpoint{3.106739in}{2.156185in}}{\pgfqpoint{3.098839in}{2.152913in}}{\pgfqpoint{3.093015in}{2.147089in}}%
\pgfpathcurveto{\pgfqpoint{3.087191in}{2.141265in}}{\pgfqpoint{3.083919in}{2.133365in}}{\pgfqpoint{3.083919in}{2.125128in}}%
\pgfpathcurveto{\pgfqpoint{3.083919in}{2.116892in}}{\pgfqpoint{3.087191in}{2.108992in}}{\pgfqpoint{3.093015in}{2.103168in}}%
\pgfpathcurveto{\pgfqpoint{3.098839in}{2.097344in}}{\pgfqpoint{3.106739in}{2.094072in}}{\pgfqpoint{3.114976in}{2.094072in}}%
\pgfpathclose%
\pgfusepath{stroke,fill}%
\end{pgfscope}%
\begin{pgfscope}%
\pgfpathrectangle{\pgfqpoint{0.100000in}{0.212622in}}{\pgfqpoint{3.696000in}{3.696000in}}%
\pgfusepath{clip}%
\pgfsetbuttcap%
\pgfsetroundjoin%
\definecolor{currentfill}{rgb}{0.121569,0.466667,0.705882}%
\pgfsetfillcolor{currentfill}%
\pgfsetfillopacity{0.555951}%
\pgfsetlinewidth{1.003750pt}%
\definecolor{currentstroke}{rgb}{0.121569,0.466667,0.705882}%
\pgfsetstrokecolor{currentstroke}%
\pgfsetstrokeopacity{0.555951}%
\pgfsetdash{}{0pt}%
\pgfpathmoveto{\pgfqpoint{3.126274in}{2.094190in}}%
\pgfpathcurveto{\pgfqpoint{3.134510in}{2.094190in}}{\pgfqpoint{3.142411in}{2.097462in}}{\pgfqpoint{3.148234in}{2.103286in}}%
\pgfpathcurveto{\pgfqpoint{3.154058in}{2.109110in}}{\pgfqpoint{3.157331in}{2.117010in}}{\pgfqpoint{3.157331in}{2.125246in}}%
\pgfpathcurveto{\pgfqpoint{3.157331in}{2.133483in}}{\pgfqpoint{3.154058in}{2.141383in}}{\pgfqpoint{3.148234in}{2.147207in}}%
\pgfpathcurveto{\pgfqpoint{3.142411in}{2.153030in}}{\pgfqpoint{3.134510in}{2.156303in}}{\pgfqpoint{3.126274in}{2.156303in}}%
\pgfpathcurveto{\pgfqpoint{3.118038in}{2.156303in}}{\pgfqpoint{3.110138in}{2.153030in}}{\pgfqpoint{3.104314in}{2.147207in}}%
\pgfpathcurveto{\pgfqpoint{3.098490in}{2.141383in}}{\pgfqpoint{3.095218in}{2.133483in}}{\pgfqpoint{3.095218in}{2.125246in}}%
\pgfpathcurveto{\pgfqpoint{3.095218in}{2.117010in}}{\pgfqpoint{3.098490in}{2.109110in}}{\pgfqpoint{3.104314in}{2.103286in}}%
\pgfpathcurveto{\pgfqpoint{3.110138in}{2.097462in}}{\pgfqpoint{3.118038in}{2.094190in}}{\pgfqpoint{3.126274in}{2.094190in}}%
\pgfpathclose%
\pgfusepath{stroke,fill}%
\end{pgfscope}%
\begin{pgfscope}%
\pgfpathrectangle{\pgfqpoint{0.100000in}{0.212622in}}{\pgfqpoint{3.696000in}{3.696000in}}%
\pgfusepath{clip}%
\pgfsetbuttcap%
\pgfsetroundjoin%
\definecolor{currentfill}{rgb}{0.121569,0.466667,0.705882}%
\pgfsetfillcolor{currentfill}%
\pgfsetfillopacity{0.557311}%
\pgfsetlinewidth{1.003750pt}%
\definecolor{currentstroke}{rgb}{0.121569,0.466667,0.705882}%
\pgfsetstrokecolor{currentstroke}%
\pgfsetstrokeopacity{0.557311}%
\pgfsetdash{}{0pt}%
\pgfpathmoveto{\pgfqpoint{3.135497in}{2.094370in}}%
\pgfpathcurveto{\pgfqpoint{3.143734in}{2.094370in}}{\pgfqpoint{3.151634in}{2.097643in}}{\pgfqpoint{3.157458in}{2.103466in}}%
\pgfpathcurveto{\pgfqpoint{3.163282in}{2.109290in}}{\pgfqpoint{3.166554in}{2.117190in}}{\pgfqpoint{3.166554in}{2.125427in}}%
\pgfpathcurveto{\pgfqpoint{3.166554in}{2.133663in}}{\pgfqpoint{3.163282in}{2.141563in}}{\pgfqpoint{3.157458in}{2.147387in}}%
\pgfpathcurveto{\pgfqpoint{3.151634in}{2.153211in}}{\pgfqpoint{3.143734in}{2.156483in}}{\pgfqpoint{3.135497in}{2.156483in}}%
\pgfpathcurveto{\pgfqpoint{3.127261in}{2.156483in}}{\pgfqpoint{3.119361in}{2.153211in}}{\pgfqpoint{3.113537in}{2.147387in}}%
\pgfpathcurveto{\pgfqpoint{3.107713in}{2.141563in}}{\pgfqpoint{3.104441in}{2.133663in}}{\pgfqpoint{3.104441in}{2.125427in}}%
\pgfpathcurveto{\pgfqpoint{3.104441in}{2.117190in}}{\pgfqpoint{3.107713in}{2.109290in}}{\pgfqpoint{3.113537in}{2.103466in}}%
\pgfpathcurveto{\pgfqpoint{3.119361in}{2.097643in}}{\pgfqpoint{3.127261in}{2.094370in}}{\pgfqpoint{3.135497in}{2.094370in}}%
\pgfpathclose%
\pgfusepath{stroke,fill}%
\end{pgfscope}%
\begin{pgfscope}%
\pgfpathrectangle{\pgfqpoint{0.100000in}{0.212622in}}{\pgfqpoint{3.696000in}{3.696000in}}%
\pgfusepath{clip}%
\pgfsetbuttcap%
\pgfsetroundjoin%
\definecolor{currentfill}{rgb}{0.121569,0.466667,0.705882}%
\pgfsetfillcolor{currentfill}%
\pgfsetfillopacity{0.559791}%
\pgfsetlinewidth{1.003750pt}%
\definecolor{currentstroke}{rgb}{0.121569,0.466667,0.705882}%
\pgfsetstrokecolor{currentstroke}%
\pgfsetstrokeopacity{0.559791}%
\pgfsetdash{}{0pt}%
\pgfpathmoveto{\pgfqpoint{3.152177in}{2.094349in}}%
\pgfpathcurveto{\pgfqpoint{3.160413in}{2.094349in}}{\pgfqpoint{3.168314in}{2.097622in}}{\pgfqpoint{3.174137in}{2.103446in}}%
\pgfpathcurveto{\pgfqpoint{3.179961in}{2.109270in}}{\pgfqpoint{3.183234in}{2.117170in}}{\pgfqpoint{3.183234in}{2.125406in}}%
\pgfpathcurveto{\pgfqpoint{3.183234in}{2.133642in}}{\pgfqpoint{3.179961in}{2.141542in}}{\pgfqpoint{3.174137in}{2.147366in}}%
\pgfpathcurveto{\pgfqpoint{3.168314in}{2.153190in}}{\pgfqpoint{3.160413in}{2.156462in}}{\pgfqpoint{3.152177in}{2.156462in}}%
\pgfpathcurveto{\pgfqpoint{3.143941in}{2.156462in}}{\pgfqpoint{3.136041in}{2.153190in}}{\pgfqpoint{3.130217in}{2.147366in}}%
\pgfpathcurveto{\pgfqpoint{3.124393in}{2.141542in}}{\pgfqpoint{3.121121in}{2.133642in}}{\pgfqpoint{3.121121in}{2.125406in}}%
\pgfpathcurveto{\pgfqpoint{3.121121in}{2.117170in}}{\pgfqpoint{3.124393in}{2.109270in}}{\pgfqpoint{3.130217in}{2.103446in}}%
\pgfpathcurveto{\pgfqpoint{3.136041in}{2.097622in}}{\pgfqpoint{3.143941in}{2.094349in}}{\pgfqpoint{3.152177in}{2.094349in}}%
\pgfpathclose%
\pgfusepath{stroke,fill}%
\end{pgfscope}%
\begin{pgfscope}%
\pgfpathrectangle{\pgfqpoint{0.100000in}{0.212622in}}{\pgfqpoint{3.696000in}{3.696000in}}%
\pgfusepath{clip}%
\pgfsetbuttcap%
\pgfsetroundjoin%
\definecolor{currentfill}{rgb}{0.121569,0.466667,0.705882}%
\pgfsetfillcolor{currentfill}%
\pgfsetfillopacity{0.562126}%
\pgfsetlinewidth{1.003750pt}%
\definecolor{currentstroke}{rgb}{0.121569,0.466667,0.705882}%
\pgfsetstrokecolor{currentstroke}%
\pgfsetstrokeopacity{0.562126}%
\pgfsetdash{}{0pt}%
\pgfpathmoveto{\pgfqpoint{3.167843in}{2.094207in}}%
\pgfpathcurveto{\pgfqpoint{3.176079in}{2.094207in}}{\pgfqpoint{3.183979in}{2.097479in}}{\pgfqpoint{3.189803in}{2.103303in}}%
\pgfpathcurveto{\pgfqpoint{3.195627in}{2.109127in}}{\pgfqpoint{3.198899in}{2.117027in}}{\pgfqpoint{3.198899in}{2.125263in}}%
\pgfpathcurveto{\pgfqpoint{3.198899in}{2.133500in}}{\pgfqpoint{3.195627in}{2.141400in}}{\pgfqpoint{3.189803in}{2.147224in}}%
\pgfpathcurveto{\pgfqpoint{3.183979in}{2.153047in}}{\pgfqpoint{3.176079in}{2.156320in}}{\pgfqpoint{3.167843in}{2.156320in}}%
\pgfpathcurveto{\pgfqpoint{3.159607in}{2.156320in}}{\pgfqpoint{3.151707in}{2.153047in}}{\pgfqpoint{3.145883in}{2.147224in}}%
\pgfpathcurveto{\pgfqpoint{3.140059in}{2.141400in}}{\pgfqpoint{3.136786in}{2.133500in}}{\pgfqpoint{3.136786in}{2.125263in}}%
\pgfpathcurveto{\pgfqpoint{3.136786in}{2.117027in}}{\pgfqpoint{3.140059in}{2.109127in}}{\pgfqpoint{3.145883in}{2.103303in}}%
\pgfpathcurveto{\pgfqpoint{3.151707in}{2.097479in}}{\pgfqpoint{3.159607in}{2.094207in}}{\pgfqpoint{3.167843in}{2.094207in}}%
\pgfpathclose%
\pgfusepath{stroke,fill}%
\end{pgfscope}%
\begin{pgfscope}%
\pgfpathrectangle{\pgfqpoint{0.100000in}{0.212622in}}{\pgfqpoint{3.696000in}{3.696000in}}%
\pgfusepath{clip}%
\pgfsetbuttcap%
\pgfsetroundjoin%
\definecolor{currentfill}{rgb}{0.121569,0.466667,0.705882}%
\pgfsetfillcolor{currentfill}%
\pgfsetfillopacity{0.564254}%
\pgfsetlinewidth{1.003750pt}%
\definecolor{currentstroke}{rgb}{0.121569,0.466667,0.705882}%
\pgfsetstrokecolor{currentstroke}%
\pgfsetstrokeopacity{0.564254}%
\pgfsetdash{}{0pt}%
\pgfpathmoveto{\pgfqpoint{3.181776in}{2.093594in}}%
\pgfpathcurveto{\pgfqpoint{3.190012in}{2.093594in}}{\pgfqpoint{3.197912in}{2.096866in}}{\pgfqpoint{3.203736in}{2.102690in}}%
\pgfpathcurveto{\pgfqpoint{3.209560in}{2.108514in}}{\pgfqpoint{3.212832in}{2.116414in}}{\pgfqpoint{3.212832in}{2.124650in}}%
\pgfpathcurveto{\pgfqpoint{3.212832in}{2.132887in}}{\pgfqpoint{3.209560in}{2.140787in}}{\pgfqpoint{3.203736in}{2.146611in}}%
\pgfpathcurveto{\pgfqpoint{3.197912in}{2.152435in}}{\pgfqpoint{3.190012in}{2.155707in}}{\pgfqpoint{3.181776in}{2.155707in}}%
\pgfpathcurveto{\pgfqpoint{3.173539in}{2.155707in}}{\pgfqpoint{3.165639in}{2.152435in}}{\pgfqpoint{3.159815in}{2.146611in}}%
\pgfpathcurveto{\pgfqpoint{3.153992in}{2.140787in}}{\pgfqpoint{3.150719in}{2.132887in}}{\pgfqpoint{3.150719in}{2.124650in}}%
\pgfpathcurveto{\pgfqpoint{3.150719in}{2.116414in}}{\pgfqpoint{3.153992in}{2.108514in}}{\pgfqpoint{3.159815in}{2.102690in}}%
\pgfpathcurveto{\pgfqpoint{3.165639in}{2.096866in}}{\pgfqpoint{3.173539in}{2.093594in}}{\pgfqpoint{3.181776in}{2.093594in}}%
\pgfpathclose%
\pgfusepath{stroke,fill}%
\end{pgfscope}%
\begin{pgfscope}%
\pgfpathrectangle{\pgfqpoint{0.100000in}{0.212622in}}{\pgfqpoint{3.696000in}{3.696000in}}%
\pgfusepath{clip}%
\pgfsetbuttcap%
\pgfsetroundjoin%
\definecolor{currentfill}{rgb}{0.121569,0.466667,0.705882}%
\pgfsetfillcolor{currentfill}%
\pgfsetfillopacity{0.566338}%
\pgfsetlinewidth{1.003750pt}%
\definecolor{currentstroke}{rgb}{0.121569,0.466667,0.705882}%
\pgfsetstrokecolor{currentstroke}%
\pgfsetstrokeopacity{0.566338}%
\pgfsetdash{}{0pt}%
\pgfpathmoveto{\pgfqpoint{3.194922in}{2.092452in}}%
\pgfpathcurveto{\pgfqpoint{3.203159in}{2.092452in}}{\pgfqpoint{3.211059in}{2.095724in}}{\pgfqpoint{3.216883in}{2.101548in}}%
\pgfpathcurveto{\pgfqpoint{3.222707in}{2.107372in}}{\pgfqpoint{3.225979in}{2.115272in}}{\pgfqpoint{3.225979in}{2.123509in}}%
\pgfpathcurveto{\pgfqpoint{3.225979in}{2.131745in}}{\pgfqpoint{3.222707in}{2.139645in}}{\pgfqpoint{3.216883in}{2.145469in}}%
\pgfpathcurveto{\pgfqpoint{3.211059in}{2.151293in}}{\pgfqpoint{3.203159in}{2.154565in}}{\pgfqpoint{3.194922in}{2.154565in}}%
\pgfpathcurveto{\pgfqpoint{3.186686in}{2.154565in}}{\pgfqpoint{3.178786in}{2.151293in}}{\pgfqpoint{3.172962in}{2.145469in}}%
\pgfpathcurveto{\pgfqpoint{3.167138in}{2.139645in}}{\pgfqpoint{3.163866in}{2.131745in}}{\pgfqpoint{3.163866in}{2.123509in}}%
\pgfpathcurveto{\pgfqpoint{3.163866in}{2.115272in}}{\pgfqpoint{3.167138in}{2.107372in}}{\pgfqpoint{3.172962in}{2.101548in}}%
\pgfpathcurveto{\pgfqpoint{3.178786in}{2.095724in}}{\pgfqpoint{3.186686in}{2.092452in}}{\pgfqpoint{3.194922in}{2.092452in}}%
\pgfpathclose%
\pgfusepath{stroke,fill}%
\end{pgfscope}%
\begin{pgfscope}%
\pgfpathrectangle{\pgfqpoint{0.100000in}{0.212622in}}{\pgfqpoint{3.696000in}{3.696000in}}%
\pgfusepath{clip}%
\pgfsetbuttcap%
\pgfsetroundjoin%
\definecolor{currentfill}{rgb}{0.121569,0.466667,0.705882}%
\pgfsetfillcolor{currentfill}%
\pgfsetfillopacity{0.568456}%
\pgfsetlinewidth{1.003750pt}%
\definecolor{currentstroke}{rgb}{0.121569,0.466667,0.705882}%
\pgfsetstrokecolor{currentstroke}%
\pgfsetstrokeopacity{0.568456}%
\pgfsetdash{}{0pt}%
\pgfpathmoveto{\pgfqpoint{3.207716in}{2.091513in}}%
\pgfpathcurveto{\pgfqpoint{3.215952in}{2.091513in}}{\pgfqpoint{3.223852in}{2.094785in}}{\pgfqpoint{3.229676in}{2.100609in}}%
\pgfpathcurveto{\pgfqpoint{3.235500in}{2.106433in}}{\pgfqpoint{3.238772in}{2.114333in}}{\pgfqpoint{3.238772in}{2.122569in}}%
\pgfpathcurveto{\pgfqpoint{3.238772in}{2.130805in}}{\pgfqpoint{3.235500in}{2.138706in}}{\pgfqpoint{3.229676in}{2.144529in}}%
\pgfpathcurveto{\pgfqpoint{3.223852in}{2.150353in}}{\pgfqpoint{3.215952in}{2.153626in}}{\pgfqpoint{3.207716in}{2.153626in}}%
\pgfpathcurveto{\pgfqpoint{3.199480in}{2.153626in}}{\pgfqpoint{3.191579in}{2.150353in}}{\pgfqpoint{3.185756in}{2.144529in}}%
\pgfpathcurveto{\pgfqpoint{3.179932in}{2.138706in}}{\pgfqpoint{3.176659in}{2.130805in}}{\pgfqpoint{3.176659in}{2.122569in}}%
\pgfpathcurveto{\pgfqpoint{3.176659in}{2.114333in}}{\pgfqpoint{3.179932in}{2.106433in}}{\pgfqpoint{3.185756in}{2.100609in}}%
\pgfpathcurveto{\pgfqpoint{3.191579in}{2.094785in}}{\pgfqpoint{3.199480in}{2.091513in}}{\pgfqpoint{3.207716in}{2.091513in}}%
\pgfpathclose%
\pgfusepath{stroke,fill}%
\end{pgfscope}%
\begin{pgfscope}%
\pgfpathrectangle{\pgfqpoint{0.100000in}{0.212622in}}{\pgfqpoint{3.696000in}{3.696000in}}%
\pgfusepath{clip}%
\pgfsetbuttcap%
\pgfsetroundjoin%
\definecolor{currentfill}{rgb}{0.121569,0.466667,0.705882}%
\pgfsetfillcolor{currentfill}%
\pgfsetfillopacity{0.570414}%
\pgfsetlinewidth{1.003750pt}%
\definecolor{currentstroke}{rgb}{0.121569,0.466667,0.705882}%
\pgfsetstrokecolor{currentstroke}%
\pgfsetstrokeopacity{0.570414}%
\pgfsetdash{}{0pt}%
\pgfpathmoveto{\pgfqpoint{3.219596in}{2.091085in}}%
\pgfpathcurveto{\pgfqpoint{3.227832in}{2.091085in}}{\pgfqpoint{3.235732in}{2.094357in}}{\pgfqpoint{3.241556in}{2.100181in}}%
\pgfpathcurveto{\pgfqpoint{3.247380in}{2.106005in}}{\pgfqpoint{3.250653in}{2.113905in}}{\pgfqpoint{3.250653in}{2.122142in}}%
\pgfpathcurveto{\pgfqpoint{3.250653in}{2.130378in}}{\pgfqpoint{3.247380in}{2.138278in}}{\pgfqpoint{3.241556in}{2.144102in}}%
\pgfpathcurveto{\pgfqpoint{3.235732in}{2.149926in}}{\pgfqpoint{3.227832in}{2.153198in}}{\pgfqpoint{3.219596in}{2.153198in}}%
\pgfpathcurveto{\pgfqpoint{3.211360in}{2.153198in}}{\pgfqpoint{3.203460in}{2.149926in}}{\pgfqpoint{3.197636in}{2.144102in}}%
\pgfpathcurveto{\pgfqpoint{3.191812in}{2.138278in}}{\pgfqpoint{3.188540in}{2.130378in}}{\pgfqpoint{3.188540in}{2.122142in}}%
\pgfpathcurveto{\pgfqpoint{3.188540in}{2.113905in}}{\pgfqpoint{3.191812in}{2.106005in}}{\pgfqpoint{3.197636in}{2.100181in}}%
\pgfpathcurveto{\pgfqpoint{3.203460in}{2.094357in}}{\pgfqpoint{3.211360in}{2.091085in}}{\pgfqpoint{3.219596in}{2.091085in}}%
\pgfpathclose%
\pgfusepath{stroke,fill}%
\end{pgfscope}%
\begin{pgfscope}%
\pgfpathrectangle{\pgfqpoint{0.100000in}{0.212622in}}{\pgfqpoint{3.696000in}{3.696000in}}%
\pgfusepath{clip}%
\pgfsetbuttcap%
\pgfsetroundjoin%
\definecolor{currentfill}{rgb}{0.121569,0.466667,0.705882}%
\pgfsetfillcolor{currentfill}%
\pgfsetfillopacity{0.572025}%
\pgfsetlinewidth{1.003750pt}%
\definecolor{currentstroke}{rgb}{0.121569,0.466667,0.705882}%
\pgfsetstrokecolor{currentstroke}%
\pgfsetstrokeopacity{0.572025}%
\pgfsetdash{}{0pt}%
\pgfpathmoveto{\pgfqpoint{3.229652in}{2.091032in}}%
\pgfpathcurveto{\pgfqpoint{3.237888in}{2.091032in}}{\pgfqpoint{3.245788in}{2.094304in}}{\pgfqpoint{3.251612in}{2.100128in}}%
\pgfpathcurveto{\pgfqpoint{3.257436in}{2.105952in}}{\pgfqpoint{3.260708in}{2.113852in}}{\pgfqpoint{3.260708in}{2.122088in}}%
\pgfpathcurveto{\pgfqpoint{3.260708in}{2.130325in}}{\pgfqpoint{3.257436in}{2.138225in}}{\pgfqpoint{3.251612in}{2.144049in}}%
\pgfpathcurveto{\pgfqpoint{3.245788in}{2.149872in}}{\pgfqpoint{3.237888in}{2.153145in}}{\pgfqpoint{3.229652in}{2.153145in}}%
\pgfpathcurveto{\pgfqpoint{3.221416in}{2.153145in}}{\pgfqpoint{3.213516in}{2.149872in}}{\pgfqpoint{3.207692in}{2.144049in}}%
\pgfpathcurveto{\pgfqpoint{3.201868in}{2.138225in}}{\pgfqpoint{3.198595in}{2.130325in}}{\pgfqpoint{3.198595in}{2.122088in}}%
\pgfpathcurveto{\pgfqpoint{3.198595in}{2.113852in}}{\pgfqpoint{3.201868in}{2.105952in}}{\pgfqpoint{3.207692in}{2.100128in}}%
\pgfpathcurveto{\pgfqpoint{3.213516in}{2.094304in}}{\pgfqpoint{3.221416in}{2.091032in}}{\pgfqpoint{3.229652in}{2.091032in}}%
\pgfpathclose%
\pgfusepath{stroke,fill}%
\end{pgfscope}%
\begin{pgfscope}%
\pgfpathrectangle{\pgfqpoint{0.100000in}{0.212622in}}{\pgfqpoint{3.696000in}{3.696000in}}%
\pgfusepath{clip}%
\pgfsetbuttcap%
\pgfsetroundjoin%
\definecolor{currentfill}{rgb}{0.121569,0.466667,0.705882}%
\pgfsetfillcolor{currentfill}%
\pgfsetfillopacity{0.573504}%
\pgfsetlinewidth{1.003750pt}%
\definecolor{currentstroke}{rgb}{0.121569,0.466667,0.705882}%
\pgfsetstrokecolor{currentstroke}%
\pgfsetstrokeopacity{0.573504}%
\pgfsetdash{}{0pt}%
\pgfpathmoveto{\pgfqpoint{3.239362in}{2.090966in}}%
\pgfpathcurveto{\pgfqpoint{3.247598in}{2.090966in}}{\pgfqpoint{3.255498in}{2.094238in}}{\pgfqpoint{3.261322in}{2.100062in}}%
\pgfpathcurveto{\pgfqpoint{3.267146in}{2.105886in}}{\pgfqpoint{3.270419in}{2.113786in}}{\pgfqpoint{3.270419in}{2.122022in}}%
\pgfpathcurveto{\pgfqpoint{3.270419in}{2.130259in}}{\pgfqpoint{3.267146in}{2.138159in}}{\pgfqpoint{3.261322in}{2.143983in}}%
\pgfpathcurveto{\pgfqpoint{3.255498in}{2.149806in}}{\pgfqpoint{3.247598in}{2.153079in}}{\pgfqpoint{3.239362in}{2.153079in}}%
\pgfpathcurveto{\pgfqpoint{3.231126in}{2.153079in}}{\pgfqpoint{3.223226in}{2.149806in}}{\pgfqpoint{3.217402in}{2.143983in}}%
\pgfpathcurveto{\pgfqpoint{3.211578in}{2.138159in}}{\pgfqpoint{3.208306in}{2.130259in}}{\pgfqpoint{3.208306in}{2.122022in}}%
\pgfpathcurveto{\pgfqpoint{3.208306in}{2.113786in}}{\pgfqpoint{3.211578in}{2.105886in}}{\pgfqpoint{3.217402in}{2.100062in}}%
\pgfpathcurveto{\pgfqpoint{3.223226in}{2.094238in}}{\pgfqpoint{3.231126in}{2.090966in}}{\pgfqpoint{3.239362in}{2.090966in}}%
\pgfpathclose%
\pgfusepath{stroke,fill}%
\end{pgfscope}%
\begin{pgfscope}%
\pgfpathrectangle{\pgfqpoint{0.100000in}{0.212622in}}{\pgfqpoint{3.696000in}{3.696000in}}%
\pgfusepath{clip}%
\pgfsetbuttcap%
\pgfsetroundjoin%
\definecolor{currentfill}{rgb}{0.121569,0.466667,0.705882}%
\pgfsetfillcolor{currentfill}%
\pgfsetfillopacity{0.574725}%
\pgfsetlinewidth{1.003750pt}%
\definecolor{currentstroke}{rgb}{0.121569,0.466667,0.705882}%
\pgfsetstrokecolor{currentstroke}%
\pgfsetstrokeopacity{0.574725}%
\pgfsetdash{}{0pt}%
\pgfpathmoveto{\pgfqpoint{3.247604in}{2.090450in}}%
\pgfpathcurveto{\pgfqpoint{3.255840in}{2.090450in}}{\pgfqpoint{3.263740in}{2.093723in}}{\pgfqpoint{3.269564in}{2.099547in}}%
\pgfpathcurveto{\pgfqpoint{3.275388in}{2.105371in}}{\pgfqpoint{3.278661in}{2.113271in}}{\pgfqpoint{3.278661in}{2.121507in}}%
\pgfpathcurveto{\pgfqpoint{3.278661in}{2.129743in}}{\pgfqpoint{3.275388in}{2.137643in}}{\pgfqpoint{3.269564in}{2.143467in}}%
\pgfpathcurveto{\pgfqpoint{3.263740in}{2.149291in}}{\pgfqpoint{3.255840in}{2.152563in}}{\pgfqpoint{3.247604in}{2.152563in}}%
\pgfpathcurveto{\pgfqpoint{3.239368in}{2.152563in}}{\pgfqpoint{3.231468in}{2.149291in}}{\pgfqpoint{3.225644in}{2.143467in}}%
\pgfpathcurveto{\pgfqpoint{3.219820in}{2.137643in}}{\pgfqpoint{3.216548in}{2.129743in}}{\pgfqpoint{3.216548in}{2.121507in}}%
\pgfpathcurveto{\pgfqpoint{3.216548in}{2.113271in}}{\pgfqpoint{3.219820in}{2.105371in}}{\pgfqpoint{3.225644in}{2.099547in}}%
\pgfpathcurveto{\pgfqpoint{3.231468in}{2.093723in}}{\pgfqpoint{3.239368in}{2.090450in}}{\pgfqpoint{3.247604in}{2.090450in}}%
\pgfpathclose%
\pgfusepath{stroke,fill}%
\end{pgfscope}%
\begin{pgfscope}%
\pgfpathrectangle{\pgfqpoint{0.100000in}{0.212622in}}{\pgfqpoint{3.696000in}{3.696000in}}%
\pgfusepath{clip}%
\pgfsetbuttcap%
\pgfsetroundjoin%
\definecolor{currentfill}{rgb}{0.121569,0.466667,0.705882}%
\pgfsetfillcolor{currentfill}%
\pgfsetfillopacity{0.575722}%
\pgfsetlinewidth{1.003750pt}%
\definecolor{currentstroke}{rgb}{0.121569,0.466667,0.705882}%
\pgfsetstrokecolor{currentstroke}%
\pgfsetstrokeopacity{0.575722}%
\pgfsetdash{}{0pt}%
\pgfpathmoveto{\pgfqpoint{3.254138in}{2.089699in}}%
\pgfpathcurveto{\pgfqpoint{3.262375in}{2.089699in}}{\pgfqpoint{3.270275in}{2.092971in}}{\pgfqpoint{3.276099in}{2.098795in}}%
\pgfpathcurveto{\pgfqpoint{3.281922in}{2.104619in}}{\pgfqpoint{3.285195in}{2.112519in}}{\pgfqpoint{3.285195in}{2.120755in}}%
\pgfpathcurveto{\pgfqpoint{3.285195in}{2.128992in}}{\pgfqpoint{3.281922in}{2.136892in}}{\pgfqpoint{3.276099in}{2.142716in}}%
\pgfpathcurveto{\pgfqpoint{3.270275in}{2.148540in}}{\pgfqpoint{3.262375in}{2.151812in}}{\pgfqpoint{3.254138in}{2.151812in}}%
\pgfpathcurveto{\pgfqpoint{3.245902in}{2.151812in}}{\pgfqpoint{3.238002in}{2.148540in}}{\pgfqpoint{3.232178in}{2.142716in}}%
\pgfpathcurveto{\pgfqpoint{3.226354in}{2.136892in}}{\pgfqpoint{3.223082in}{2.128992in}}{\pgfqpoint{3.223082in}{2.120755in}}%
\pgfpathcurveto{\pgfqpoint{3.223082in}{2.112519in}}{\pgfqpoint{3.226354in}{2.104619in}}{\pgfqpoint{3.232178in}{2.098795in}}%
\pgfpathcurveto{\pgfqpoint{3.238002in}{2.092971in}}{\pgfqpoint{3.245902in}{2.089699in}}{\pgfqpoint{3.254138in}{2.089699in}}%
\pgfpathclose%
\pgfusepath{stroke,fill}%
\end{pgfscope}%
\begin{pgfscope}%
\pgfpathrectangle{\pgfqpoint{0.100000in}{0.212622in}}{\pgfqpoint{3.696000in}{3.696000in}}%
\pgfusepath{clip}%
\pgfsetbuttcap%
\pgfsetroundjoin%
\definecolor{currentfill}{rgb}{0.121569,0.466667,0.705882}%
\pgfsetfillcolor{currentfill}%
\pgfsetfillopacity{0.576703}%
\pgfsetlinewidth{1.003750pt}%
\definecolor{currentstroke}{rgb}{0.121569,0.466667,0.705882}%
\pgfsetstrokecolor{currentstroke}%
\pgfsetstrokeopacity{0.576703}%
\pgfsetdash{}{0pt}%
\pgfpathmoveto{\pgfqpoint{3.260323in}{2.088764in}}%
\pgfpathcurveto{\pgfqpoint{3.268559in}{2.088764in}}{\pgfqpoint{3.276459in}{2.092036in}}{\pgfqpoint{3.282283in}{2.097860in}}%
\pgfpathcurveto{\pgfqpoint{3.288107in}{2.103684in}}{\pgfqpoint{3.291380in}{2.111584in}}{\pgfqpoint{3.291380in}{2.119820in}}%
\pgfpathcurveto{\pgfqpoint{3.291380in}{2.128056in}}{\pgfqpoint{3.288107in}{2.135956in}}{\pgfqpoint{3.282283in}{2.141780in}}%
\pgfpathcurveto{\pgfqpoint{3.276459in}{2.147604in}}{\pgfqpoint{3.268559in}{2.150877in}}{\pgfqpoint{3.260323in}{2.150877in}}%
\pgfpathcurveto{\pgfqpoint{3.252087in}{2.150877in}}{\pgfqpoint{3.244187in}{2.147604in}}{\pgfqpoint{3.238363in}{2.141780in}}%
\pgfpathcurveto{\pgfqpoint{3.232539in}{2.135956in}}{\pgfqpoint{3.229267in}{2.128056in}}{\pgfqpoint{3.229267in}{2.119820in}}%
\pgfpathcurveto{\pgfqpoint{3.229267in}{2.111584in}}{\pgfqpoint{3.232539in}{2.103684in}}{\pgfqpoint{3.238363in}{2.097860in}}%
\pgfpathcurveto{\pgfqpoint{3.244187in}{2.092036in}}{\pgfqpoint{3.252087in}{2.088764in}}{\pgfqpoint{3.260323in}{2.088764in}}%
\pgfpathclose%
\pgfusepath{stroke,fill}%
\end{pgfscope}%
\begin{pgfscope}%
\pgfpathrectangle{\pgfqpoint{0.100000in}{0.212622in}}{\pgfqpoint{3.696000in}{3.696000in}}%
\pgfusepath{clip}%
\pgfsetbuttcap%
\pgfsetroundjoin%
\definecolor{currentfill}{rgb}{0.121569,0.466667,0.705882}%
\pgfsetfillcolor{currentfill}%
\pgfsetfillopacity{0.577528}%
\pgfsetlinewidth{1.003750pt}%
\definecolor{currentstroke}{rgb}{0.121569,0.466667,0.705882}%
\pgfsetstrokecolor{currentstroke}%
\pgfsetstrokeopacity{0.577528}%
\pgfsetdash{}{0pt}%
\pgfpathmoveto{\pgfqpoint{3.265531in}{2.088241in}}%
\pgfpathcurveto{\pgfqpoint{3.273767in}{2.088241in}}{\pgfqpoint{3.281667in}{2.091513in}}{\pgfqpoint{3.287491in}{2.097337in}}%
\pgfpathcurveto{\pgfqpoint{3.293315in}{2.103161in}}{\pgfqpoint{3.296587in}{2.111061in}}{\pgfqpoint{3.296587in}{2.119298in}}%
\pgfpathcurveto{\pgfqpoint{3.296587in}{2.127534in}}{\pgfqpoint{3.293315in}{2.135434in}}{\pgfqpoint{3.287491in}{2.141258in}}%
\pgfpathcurveto{\pgfqpoint{3.281667in}{2.147082in}}{\pgfqpoint{3.273767in}{2.150354in}}{\pgfqpoint{3.265531in}{2.150354in}}%
\pgfpathcurveto{\pgfqpoint{3.257294in}{2.150354in}}{\pgfqpoint{3.249394in}{2.147082in}}{\pgfqpoint{3.243571in}{2.141258in}}%
\pgfpathcurveto{\pgfqpoint{3.237747in}{2.135434in}}{\pgfqpoint{3.234474in}{2.127534in}}{\pgfqpoint{3.234474in}{2.119298in}}%
\pgfpathcurveto{\pgfqpoint{3.234474in}{2.111061in}}{\pgfqpoint{3.237747in}{2.103161in}}{\pgfqpoint{3.243571in}{2.097337in}}%
\pgfpathcurveto{\pgfqpoint{3.249394in}{2.091513in}}{\pgfqpoint{3.257294in}{2.088241in}}{\pgfqpoint{3.265531in}{2.088241in}}%
\pgfpathclose%
\pgfusepath{stroke,fill}%
\end{pgfscope}%
\begin{pgfscope}%
\pgfpathrectangle{\pgfqpoint{0.100000in}{0.212622in}}{\pgfqpoint{3.696000in}{3.696000in}}%
\pgfusepath{clip}%
\pgfsetbuttcap%
\pgfsetroundjoin%
\definecolor{currentfill}{rgb}{0.121569,0.466667,0.705882}%
\pgfsetfillcolor{currentfill}%
\pgfsetfillopacity{0.578032}%
\pgfsetlinewidth{1.003750pt}%
\definecolor{currentstroke}{rgb}{0.121569,0.466667,0.705882}%
\pgfsetstrokecolor{currentstroke}%
\pgfsetstrokeopacity{0.578032}%
\pgfsetdash{}{0pt}%
\pgfpathmoveto{\pgfqpoint{3.268846in}{2.088005in}}%
\pgfpathcurveto{\pgfqpoint{3.277082in}{2.088005in}}{\pgfqpoint{3.284982in}{2.091277in}}{\pgfqpoint{3.290806in}{2.097101in}}%
\pgfpathcurveto{\pgfqpoint{3.296630in}{2.102925in}}{\pgfqpoint{3.299902in}{2.110825in}}{\pgfqpoint{3.299902in}{2.119061in}}%
\pgfpathcurveto{\pgfqpoint{3.299902in}{2.127297in}}{\pgfqpoint{3.296630in}{2.135197in}}{\pgfqpoint{3.290806in}{2.141021in}}%
\pgfpathcurveto{\pgfqpoint{3.284982in}{2.146845in}}{\pgfqpoint{3.277082in}{2.150118in}}{\pgfqpoint{3.268846in}{2.150118in}}%
\pgfpathcurveto{\pgfqpoint{3.260610in}{2.150118in}}{\pgfqpoint{3.252710in}{2.146845in}}{\pgfqpoint{3.246886in}{2.141021in}}%
\pgfpathcurveto{\pgfqpoint{3.241062in}{2.135197in}}{\pgfqpoint{3.237789in}{2.127297in}}{\pgfqpoint{3.237789in}{2.119061in}}%
\pgfpathcurveto{\pgfqpoint{3.237789in}{2.110825in}}{\pgfqpoint{3.241062in}{2.102925in}}{\pgfqpoint{3.246886in}{2.097101in}}%
\pgfpathcurveto{\pgfqpoint{3.252710in}{2.091277in}}{\pgfqpoint{3.260610in}{2.088005in}}{\pgfqpoint{3.268846in}{2.088005in}}%
\pgfpathclose%
\pgfusepath{stroke,fill}%
\end{pgfscope}%
\begin{pgfscope}%
\pgfpathrectangle{\pgfqpoint{0.100000in}{0.212622in}}{\pgfqpoint{3.696000in}{3.696000in}}%
\pgfusepath{clip}%
\pgfsetbuttcap%
\pgfsetroundjoin%
\definecolor{currentfill}{rgb}{0.121569,0.466667,0.705882}%
\pgfsetfillcolor{currentfill}%
\pgfsetfillopacity{0.578936}%
\pgfsetlinewidth{1.003750pt}%
\definecolor{currentstroke}{rgb}{0.121569,0.466667,0.705882}%
\pgfsetstrokecolor{currentstroke}%
\pgfsetstrokeopacity{0.578936}%
\pgfsetdash{}{0pt}%
\pgfpathmoveto{\pgfqpoint{3.274807in}{2.087267in}}%
\pgfpathcurveto{\pgfqpoint{3.283044in}{2.087267in}}{\pgfqpoint{3.290944in}{2.090539in}}{\pgfqpoint{3.296768in}{2.096363in}}%
\pgfpathcurveto{\pgfqpoint{3.302592in}{2.102187in}}{\pgfqpoint{3.305864in}{2.110087in}}{\pgfqpoint{3.305864in}{2.118323in}}%
\pgfpathcurveto{\pgfqpoint{3.305864in}{2.126560in}}{\pgfqpoint{3.302592in}{2.134460in}}{\pgfqpoint{3.296768in}{2.140284in}}%
\pgfpathcurveto{\pgfqpoint{3.290944in}{2.146108in}}{\pgfqpoint{3.283044in}{2.149380in}}{\pgfqpoint{3.274807in}{2.149380in}}%
\pgfpathcurveto{\pgfqpoint{3.266571in}{2.149380in}}{\pgfqpoint{3.258671in}{2.146108in}}{\pgfqpoint{3.252847in}{2.140284in}}%
\pgfpathcurveto{\pgfqpoint{3.247023in}{2.134460in}}{\pgfqpoint{3.243751in}{2.126560in}}{\pgfqpoint{3.243751in}{2.118323in}}%
\pgfpathcurveto{\pgfqpoint{3.243751in}{2.110087in}}{\pgfqpoint{3.247023in}{2.102187in}}{\pgfqpoint{3.252847in}{2.096363in}}%
\pgfpathcurveto{\pgfqpoint{3.258671in}{2.090539in}}{\pgfqpoint{3.266571in}{2.087267in}}{\pgfqpoint{3.274807in}{2.087267in}}%
\pgfpathclose%
\pgfusepath{stroke,fill}%
\end{pgfscope}%
\begin{pgfscope}%
\pgfpathrectangle{\pgfqpoint{0.100000in}{0.212622in}}{\pgfqpoint{3.696000in}{3.696000in}}%
\pgfusepath{clip}%
\pgfsetbuttcap%
\pgfsetroundjoin%
\definecolor{currentfill}{rgb}{0.121569,0.466667,0.705882}%
\pgfsetfillcolor{currentfill}%
\pgfsetfillopacity{0.579610}%
\pgfsetlinewidth{1.003750pt}%
\definecolor{currentstroke}{rgb}{0.121569,0.466667,0.705882}%
\pgfsetstrokecolor{currentstroke}%
\pgfsetstrokeopacity{0.579610}%
\pgfsetdash{}{0pt}%
\pgfpathmoveto{\pgfqpoint{3.279293in}{2.086365in}}%
\pgfpathcurveto{\pgfqpoint{3.287530in}{2.086365in}}{\pgfqpoint{3.295430in}{2.089637in}}{\pgfqpoint{3.301254in}{2.095461in}}%
\pgfpathcurveto{\pgfqpoint{3.307077in}{2.101285in}}{\pgfqpoint{3.310350in}{2.109185in}}{\pgfqpoint{3.310350in}{2.117421in}}%
\pgfpathcurveto{\pgfqpoint{3.310350in}{2.125658in}}{\pgfqpoint{3.307077in}{2.133558in}}{\pgfqpoint{3.301254in}{2.139382in}}%
\pgfpathcurveto{\pgfqpoint{3.295430in}{2.145205in}}{\pgfqpoint{3.287530in}{2.148478in}}{\pgfqpoint{3.279293in}{2.148478in}}%
\pgfpathcurveto{\pgfqpoint{3.271057in}{2.148478in}}{\pgfqpoint{3.263157in}{2.145205in}}{\pgfqpoint{3.257333in}{2.139382in}}%
\pgfpathcurveto{\pgfqpoint{3.251509in}{2.133558in}}{\pgfqpoint{3.248237in}{2.125658in}}{\pgfqpoint{3.248237in}{2.117421in}}%
\pgfpathcurveto{\pgfqpoint{3.248237in}{2.109185in}}{\pgfqpoint{3.251509in}{2.101285in}}{\pgfqpoint{3.257333in}{2.095461in}}%
\pgfpathcurveto{\pgfqpoint{3.263157in}{2.089637in}}{\pgfqpoint{3.271057in}{2.086365in}}{\pgfqpoint{3.279293in}{2.086365in}}%
\pgfpathclose%
\pgfusepath{stroke,fill}%
\end{pgfscope}%
\begin{pgfscope}%
\pgfpathrectangle{\pgfqpoint{0.100000in}{0.212622in}}{\pgfqpoint{3.696000in}{3.696000in}}%
\pgfusepath{clip}%
\pgfsetbuttcap%
\pgfsetroundjoin%
\definecolor{currentfill}{rgb}{0.121569,0.466667,0.705882}%
\pgfsetfillcolor{currentfill}%
\pgfsetfillopacity{0.580109}%
\pgfsetlinewidth{1.003750pt}%
\definecolor{currentstroke}{rgb}{0.121569,0.466667,0.705882}%
\pgfsetstrokecolor{currentstroke}%
\pgfsetstrokeopacity{0.580109}%
\pgfsetdash{}{0pt}%
\pgfpathmoveto{\pgfqpoint{3.282644in}{2.085559in}}%
\pgfpathcurveto{\pgfqpoint{3.290881in}{2.085559in}}{\pgfqpoint{3.298781in}{2.088831in}}{\pgfqpoint{3.304605in}{2.094655in}}%
\pgfpathcurveto{\pgfqpoint{3.310429in}{2.100479in}}{\pgfqpoint{3.313701in}{2.108379in}}{\pgfqpoint{3.313701in}{2.116615in}}%
\pgfpathcurveto{\pgfqpoint{3.313701in}{2.124852in}}{\pgfqpoint{3.310429in}{2.132752in}}{\pgfqpoint{3.304605in}{2.138576in}}%
\pgfpathcurveto{\pgfqpoint{3.298781in}{2.144399in}}{\pgfqpoint{3.290881in}{2.147672in}}{\pgfqpoint{3.282644in}{2.147672in}}%
\pgfpathcurveto{\pgfqpoint{3.274408in}{2.147672in}}{\pgfqpoint{3.266508in}{2.144399in}}{\pgfqpoint{3.260684in}{2.138576in}}%
\pgfpathcurveto{\pgfqpoint{3.254860in}{2.132752in}}{\pgfqpoint{3.251588in}{2.124852in}}{\pgfqpoint{3.251588in}{2.116615in}}%
\pgfpathcurveto{\pgfqpoint{3.251588in}{2.108379in}}{\pgfqpoint{3.254860in}{2.100479in}}{\pgfqpoint{3.260684in}{2.094655in}}%
\pgfpathcurveto{\pgfqpoint{3.266508in}{2.088831in}}{\pgfqpoint{3.274408in}{2.085559in}}{\pgfqpoint{3.282644in}{2.085559in}}%
\pgfpathclose%
\pgfusepath{stroke,fill}%
\end{pgfscope}%
\begin{pgfscope}%
\pgfpathrectangle{\pgfqpoint{0.100000in}{0.212622in}}{\pgfqpoint{3.696000in}{3.696000in}}%
\pgfusepath{clip}%
\pgfsetbuttcap%
\pgfsetroundjoin%
\definecolor{currentfill}{rgb}{0.121569,0.466667,0.705882}%
\pgfsetfillcolor{currentfill}%
\pgfsetfillopacity{0.581241}%
\pgfsetlinewidth{1.003750pt}%
\definecolor{currentstroke}{rgb}{0.121569,0.466667,0.705882}%
\pgfsetstrokecolor{currentstroke}%
\pgfsetstrokeopacity{0.581241}%
\pgfsetdash{}{0pt}%
\pgfpathmoveto{\pgfqpoint{3.288652in}{2.085008in}}%
\pgfpathcurveto{\pgfqpoint{3.296888in}{2.085008in}}{\pgfqpoint{3.304788in}{2.088281in}}{\pgfqpoint{3.310612in}{2.094105in}}%
\pgfpathcurveto{\pgfqpoint{3.316436in}{2.099929in}}{\pgfqpoint{3.319708in}{2.107829in}}{\pgfqpoint{3.319708in}{2.116065in}}%
\pgfpathcurveto{\pgfqpoint{3.319708in}{2.124301in}}{\pgfqpoint{3.316436in}{2.132201in}}{\pgfqpoint{3.310612in}{2.138025in}}%
\pgfpathcurveto{\pgfqpoint{3.304788in}{2.143849in}}{\pgfqpoint{3.296888in}{2.147121in}}{\pgfqpoint{3.288652in}{2.147121in}}%
\pgfpathcurveto{\pgfqpoint{3.280416in}{2.147121in}}{\pgfqpoint{3.272516in}{2.143849in}}{\pgfqpoint{3.266692in}{2.138025in}}%
\pgfpathcurveto{\pgfqpoint{3.260868in}{2.132201in}}{\pgfqpoint{3.257595in}{2.124301in}}{\pgfqpoint{3.257595in}{2.116065in}}%
\pgfpathcurveto{\pgfqpoint{3.257595in}{2.107829in}}{\pgfqpoint{3.260868in}{2.099929in}}{\pgfqpoint{3.266692in}{2.094105in}}%
\pgfpathcurveto{\pgfqpoint{3.272516in}{2.088281in}}{\pgfqpoint{3.280416in}{2.085008in}}{\pgfqpoint{3.288652in}{2.085008in}}%
\pgfpathclose%
\pgfusepath{stroke,fill}%
\end{pgfscope}%
\begin{pgfscope}%
\pgfpathrectangle{\pgfqpoint{0.100000in}{0.212622in}}{\pgfqpoint{3.696000in}{3.696000in}}%
\pgfusepath{clip}%
\pgfsetbuttcap%
\pgfsetroundjoin%
\definecolor{currentfill}{rgb}{0.121569,0.466667,0.705882}%
\pgfsetfillcolor{currentfill}%
\pgfsetfillopacity{0.582169}%
\pgfsetlinewidth{1.003750pt}%
\definecolor{currentstroke}{rgb}{0.121569,0.466667,0.705882}%
\pgfsetstrokecolor{currentstroke}%
\pgfsetstrokeopacity{0.582169}%
\pgfsetdash{}{0pt}%
\pgfpathmoveto{\pgfqpoint{3.293535in}{2.084712in}}%
\pgfpathcurveto{\pgfqpoint{3.301771in}{2.084712in}}{\pgfqpoint{3.309671in}{2.087984in}}{\pgfqpoint{3.315495in}{2.093808in}}%
\pgfpathcurveto{\pgfqpoint{3.321319in}{2.099632in}}{\pgfqpoint{3.324591in}{2.107532in}}{\pgfqpoint{3.324591in}{2.115768in}}%
\pgfpathcurveto{\pgfqpoint{3.324591in}{2.124004in}}{\pgfqpoint{3.321319in}{2.131904in}}{\pgfqpoint{3.315495in}{2.137728in}}%
\pgfpathcurveto{\pgfqpoint{3.309671in}{2.143552in}}{\pgfqpoint{3.301771in}{2.146825in}}{\pgfqpoint{3.293535in}{2.146825in}}%
\pgfpathcurveto{\pgfqpoint{3.285298in}{2.146825in}}{\pgfqpoint{3.277398in}{2.143552in}}{\pgfqpoint{3.271574in}{2.137728in}}%
\pgfpathcurveto{\pgfqpoint{3.265751in}{2.131904in}}{\pgfqpoint{3.262478in}{2.124004in}}{\pgfqpoint{3.262478in}{2.115768in}}%
\pgfpathcurveto{\pgfqpoint{3.262478in}{2.107532in}}{\pgfqpoint{3.265751in}{2.099632in}}{\pgfqpoint{3.271574in}{2.093808in}}%
\pgfpathcurveto{\pgfqpoint{3.277398in}{2.087984in}}{\pgfqpoint{3.285298in}{2.084712in}}{\pgfqpoint{3.293535in}{2.084712in}}%
\pgfpathclose%
\pgfusepath{stroke,fill}%
\end{pgfscope}%
\begin{pgfscope}%
\pgfpathrectangle{\pgfqpoint{0.100000in}{0.212622in}}{\pgfqpoint{3.696000in}{3.696000in}}%
\pgfusepath{clip}%
\pgfsetbuttcap%
\pgfsetroundjoin%
\definecolor{currentfill}{rgb}{0.121569,0.466667,0.705882}%
\pgfsetfillcolor{currentfill}%
\pgfsetfillopacity{0.582975}%
\pgfsetlinewidth{1.003750pt}%
\definecolor{currentstroke}{rgb}{0.121569,0.466667,0.705882}%
\pgfsetstrokecolor{currentstroke}%
\pgfsetstrokeopacity{0.582975}%
\pgfsetdash{}{0pt}%
\pgfpathmoveto{\pgfqpoint{3.297781in}{2.084686in}}%
\pgfpathcurveto{\pgfqpoint{3.306017in}{2.084686in}}{\pgfqpoint{3.313917in}{2.087959in}}{\pgfqpoint{3.319741in}{2.093782in}}%
\pgfpathcurveto{\pgfqpoint{3.325565in}{2.099606in}}{\pgfqpoint{3.328837in}{2.107506in}}{\pgfqpoint{3.328837in}{2.115743in}}%
\pgfpathcurveto{\pgfqpoint{3.328837in}{2.123979in}}{\pgfqpoint{3.325565in}{2.131879in}}{\pgfqpoint{3.319741in}{2.137703in}}%
\pgfpathcurveto{\pgfqpoint{3.313917in}{2.143527in}}{\pgfqpoint{3.306017in}{2.146799in}}{\pgfqpoint{3.297781in}{2.146799in}}%
\pgfpathcurveto{\pgfqpoint{3.289545in}{2.146799in}}{\pgfqpoint{3.281645in}{2.143527in}}{\pgfqpoint{3.275821in}{2.137703in}}%
\pgfpathcurveto{\pgfqpoint{3.269997in}{2.131879in}}{\pgfqpoint{3.266724in}{2.123979in}}{\pgfqpoint{3.266724in}{2.115743in}}%
\pgfpathcurveto{\pgfqpoint{3.266724in}{2.107506in}}{\pgfqpoint{3.269997in}{2.099606in}}{\pgfqpoint{3.275821in}{2.093782in}}%
\pgfpathcurveto{\pgfqpoint{3.281645in}{2.087959in}}{\pgfqpoint{3.289545in}{2.084686in}}{\pgfqpoint{3.297781in}{2.084686in}}%
\pgfpathclose%
\pgfusepath{stroke,fill}%
\end{pgfscope}%
\begin{pgfscope}%
\pgfpathrectangle{\pgfqpoint{0.100000in}{0.212622in}}{\pgfqpoint{3.696000in}{3.696000in}}%
\pgfusepath{clip}%
\pgfsetbuttcap%
\pgfsetroundjoin%
\definecolor{currentfill}{rgb}{0.121569,0.466667,0.705882}%
\pgfsetfillcolor{currentfill}%
\pgfsetfillopacity{0.583562}%
\pgfsetlinewidth{1.003750pt}%
\definecolor{currentstroke}{rgb}{0.121569,0.466667,0.705882}%
\pgfsetstrokecolor{currentstroke}%
\pgfsetstrokeopacity{0.583562}%
\pgfsetdash{}{0pt}%
\pgfpathmoveto{\pgfqpoint{3.300855in}{2.084689in}}%
\pgfpathcurveto{\pgfqpoint{3.309091in}{2.084689in}}{\pgfqpoint{3.316991in}{2.087961in}}{\pgfqpoint{3.322815in}{2.093785in}}%
\pgfpathcurveto{\pgfqpoint{3.328639in}{2.099609in}}{\pgfqpoint{3.331911in}{2.107509in}}{\pgfqpoint{3.331911in}{2.115745in}}%
\pgfpathcurveto{\pgfqpoint{3.331911in}{2.123982in}}{\pgfqpoint{3.328639in}{2.131882in}}{\pgfqpoint{3.322815in}{2.137706in}}%
\pgfpathcurveto{\pgfqpoint{3.316991in}{2.143530in}}{\pgfqpoint{3.309091in}{2.146802in}}{\pgfqpoint{3.300855in}{2.146802in}}%
\pgfpathcurveto{\pgfqpoint{3.292618in}{2.146802in}}{\pgfqpoint{3.284718in}{2.143530in}}{\pgfqpoint{3.278894in}{2.137706in}}%
\pgfpathcurveto{\pgfqpoint{3.273070in}{2.131882in}}{\pgfqpoint{3.269798in}{2.123982in}}{\pgfqpoint{3.269798in}{2.115745in}}%
\pgfpathcurveto{\pgfqpoint{3.269798in}{2.107509in}}{\pgfqpoint{3.273070in}{2.099609in}}{\pgfqpoint{3.278894in}{2.093785in}}%
\pgfpathcurveto{\pgfqpoint{3.284718in}{2.087961in}}{\pgfqpoint{3.292618in}{2.084689in}}{\pgfqpoint{3.300855in}{2.084689in}}%
\pgfpathclose%
\pgfusepath{stroke,fill}%
\end{pgfscope}%
\begin{pgfscope}%
\pgfpathrectangle{\pgfqpoint{0.100000in}{0.212622in}}{\pgfqpoint{3.696000in}{3.696000in}}%
\pgfusepath{clip}%
\pgfsetbuttcap%
\pgfsetroundjoin%
\definecolor{currentfill}{rgb}{0.121569,0.466667,0.705882}%
\pgfsetfillcolor{currentfill}%
\pgfsetfillopacity{0.583973}%
\pgfsetlinewidth{1.003750pt}%
\definecolor{currentstroke}{rgb}{0.121569,0.466667,0.705882}%
\pgfsetstrokecolor{currentstroke}%
\pgfsetstrokeopacity{0.583973}%
\pgfsetdash{}{0pt}%
\pgfpathmoveto{\pgfqpoint{3.302942in}{2.084620in}}%
\pgfpathcurveto{\pgfqpoint{3.311178in}{2.084620in}}{\pgfqpoint{3.319078in}{2.087892in}}{\pgfqpoint{3.324902in}{2.093716in}}%
\pgfpathcurveto{\pgfqpoint{3.330726in}{2.099540in}}{\pgfqpoint{3.333999in}{2.107440in}}{\pgfqpoint{3.333999in}{2.115676in}}%
\pgfpathcurveto{\pgfqpoint{3.333999in}{2.123913in}}{\pgfqpoint{3.330726in}{2.131813in}}{\pgfqpoint{3.324902in}{2.137637in}}%
\pgfpathcurveto{\pgfqpoint{3.319078in}{2.143461in}}{\pgfqpoint{3.311178in}{2.146733in}}{\pgfqpoint{3.302942in}{2.146733in}}%
\pgfpathcurveto{\pgfqpoint{3.294706in}{2.146733in}}{\pgfqpoint{3.286806in}{2.143461in}}{\pgfqpoint{3.280982in}{2.137637in}}%
\pgfpathcurveto{\pgfqpoint{3.275158in}{2.131813in}}{\pgfqpoint{3.271886in}{2.123913in}}{\pgfqpoint{3.271886in}{2.115676in}}%
\pgfpathcurveto{\pgfqpoint{3.271886in}{2.107440in}}{\pgfqpoint{3.275158in}{2.099540in}}{\pgfqpoint{3.280982in}{2.093716in}}%
\pgfpathcurveto{\pgfqpoint{3.286806in}{2.087892in}}{\pgfqpoint{3.294706in}{2.084620in}}{\pgfqpoint{3.302942in}{2.084620in}}%
\pgfpathclose%
\pgfusepath{stroke,fill}%
\end{pgfscope}%
\begin{pgfscope}%
\pgfpathrectangle{\pgfqpoint{0.100000in}{0.212622in}}{\pgfqpoint{3.696000in}{3.696000in}}%
\pgfusepath{clip}%
\pgfsetbuttcap%
\pgfsetroundjoin%
\definecolor{currentfill}{rgb}{0.121569,0.466667,0.705882}%
\pgfsetfillcolor{currentfill}%
\pgfsetfillopacity{0.584296}%
\pgfsetlinewidth{1.003750pt}%
\definecolor{currentstroke}{rgb}{0.121569,0.466667,0.705882}%
\pgfsetstrokecolor{currentstroke}%
\pgfsetstrokeopacity{0.584296}%
\pgfsetdash{}{0pt}%
\pgfpathmoveto{\pgfqpoint{3.304453in}{2.084372in}}%
\pgfpathcurveto{\pgfqpoint{3.312690in}{2.084372in}}{\pgfqpoint{3.320590in}{2.087644in}}{\pgfqpoint{3.326414in}{2.093468in}}%
\pgfpathcurveto{\pgfqpoint{3.332237in}{2.099292in}}{\pgfqpoint{3.335510in}{2.107192in}}{\pgfqpoint{3.335510in}{2.115428in}}%
\pgfpathcurveto{\pgfqpoint{3.335510in}{2.123664in}}{\pgfqpoint{3.332237in}{2.131564in}}{\pgfqpoint{3.326414in}{2.137388in}}%
\pgfpathcurveto{\pgfqpoint{3.320590in}{2.143212in}}{\pgfqpoint{3.312690in}{2.146485in}}{\pgfqpoint{3.304453in}{2.146485in}}%
\pgfpathcurveto{\pgfqpoint{3.296217in}{2.146485in}}{\pgfqpoint{3.288317in}{2.143212in}}{\pgfqpoint{3.282493in}{2.137388in}}%
\pgfpathcurveto{\pgfqpoint{3.276669in}{2.131564in}}{\pgfqpoint{3.273397in}{2.123664in}}{\pgfqpoint{3.273397in}{2.115428in}}%
\pgfpathcurveto{\pgfqpoint{3.273397in}{2.107192in}}{\pgfqpoint{3.276669in}{2.099292in}}{\pgfqpoint{3.282493in}{2.093468in}}%
\pgfpathcurveto{\pgfqpoint{3.288317in}{2.087644in}}{\pgfqpoint{3.296217in}{2.084372in}}{\pgfqpoint{3.304453in}{2.084372in}}%
\pgfpathclose%
\pgfusepath{stroke,fill}%
\end{pgfscope}%
\begin{pgfscope}%
\pgfpathrectangle{\pgfqpoint{0.100000in}{0.212622in}}{\pgfqpoint{3.696000in}{3.696000in}}%
\pgfusepath{clip}%
\pgfsetbuttcap%
\pgfsetroundjoin%
\definecolor{currentfill}{rgb}{0.121569,0.466667,0.705882}%
\pgfsetfillcolor{currentfill}%
\pgfsetfillopacity{0.584418}%
\pgfsetlinewidth{1.003750pt}%
\definecolor{currentstroke}{rgb}{0.121569,0.466667,0.705882}%
\pgfsetstrokecolor{currentstroke}%
\pgfsetstrokeopacity{0.584418}%
\pgfsetdash{}{0pt}%
\pgfpathmoveto{\pgfqpoint{3.305049in}{2.084312in}}%
\pgfpathcurveto{\pgfqpoint{3.313285in}{2.084312in}}{\pgfqpoint{3.321185in}{2.087584in}}{\pgfqpoint{3.327009in}{2.093408in}}%
\pgfpathcurveto{\pgfqpoint{3.332833in}{2.099232in}}{\pgfqpoint{3.336105in}{2.107132in}}{\pgfqpoint{3.336105in}{2.115368in}}%
\pgfpathcurveto{\pgfqpoint{3.336105in}{2.123604in}}{\pgfqpoint{3.332833in}{2.131504in}}{\pgfqpoint{3.327009in}{2.137328in}}%
\pgfpathcurveto{\pgfqpoint{3.321185in}{2.143152in}}{\pgfqpoint{3.313285in}{2.146425in}}{\pgfqpoint{3.305049in}{2.146425in}}%
\pgfpathcurveto{\pgfqpoint{3.296812in}{2.146425in}}{\pgfqpoint{3.288912in}{2.143152in}}{\pgfqpoint{3.283088in}{2.137328in}}%
\pgfpathcurveto{\pgfqpoint{3.277265in}{2.131504in}}{\pgfqpoint{3.273992in}{2.123604in}}{\pgfqpoint{3.273992in}{2.115368in}}%
\pgfpathcurveto{\pgfqpoint{3.273992in}{2.107132in}}{\pgfqpoint{3.277265in}{2.099232in}}{\pgfqpoint{3.283088in}{2.093408in}}%
\pgfpathcurveto{\pgfqpoint{3.288912in}{2.087584in}}{\pgfqpoint{3.296812in}{2.084312in}}{\pgfqpoint{3.305049in}{2.084312in}}%
\pgfpathclose%
\pgfusepath{stroke,fill}%
\end{pgfscope}%
\begin{pgfscope}%
\pgfpathrectangle{\pgfqpoint{0.100000in}{0.212622in}}{\pgfqpoint{3.696000in}{3.696000in}}%
\pgfusepath{clip}%
\pgfsetbuttcap%
\pgfsetroundjoin%
\definecolor{currentfill}{rgb}{0.121569,0.466667,0.705882}%
\pgfsetfillcolor{currentfill}%
\pgfsetfillopacity{0.584418}%
\pgfsetlinewidth{1.003750pt}%
\definecolor{currentstroke}{rgb}{0.121569,0.466667,0.705882}%
\pgfsetstrokecolor{currentstroke}%
\pgfsetstrokeopacity{0.584418}%
\pgfsetdash{}{0pt}%
\pgfpathmoveto{\pgfqpoint{3.305049in}{2.084312in}}%
\pgfpathcurveto{\pgfqpoint{3.313285in}{2.084312in}}{\pgfqpoint{3.321185in}{2.087584in}}{\pgfqpoint{3.327009in}{2.093408in}}%
\pgfpathcurveto{\pgfqpoint{3.332833in}{2.099232in}}{\pgfqpoint{3.336105in}{2.107132in}}{\pgfqpoint{3.336105in}{2.115368in}}%
\pgfpathcurveto{\pgfqpoint{3.336105in}{2.123604in}}{\pgfqpoint{3.332833in}{2.131504in}}{\pgfqpoint{3.327009in}{2.137328in}}%
\pgfpathcurveto{\pgfqpoint{3.321185in}{2.143152in}}{\pgfqpoint{3.313285in}{2.146425in}}{\pgfqpoint{3.305049in}{2.146425in}}%
\pgfpathcurveto{\pgfqpoint{3.296812in}{2.146425in}}{\pgfqpoint{3.288912in}{2.143152in}}{\pgfqpoint{3.283088in}{2.137328in}}%
\pgfpathcurveto{\pgfqpoint{3.277265in}{2.131504in}}{\pgfqpoint{3.273992in}{2.123604in}}{\pgfqpoint{3.273992in}{2.115368in}}%
\pgfpathcurveto{\pgfqpoint{3.273992in}{2.107132in}}{\pgfqpoint{3.277265in}{2.099232in}}{\pgfqpoint{3.283088in}{2.093408in}}%
\pgfpathcurveto{\pgfqpoint{3.288912in}{2.087584in}}{\pgfqpoint{3.296812in}{2.084312in}}{\pgfqpoint{3.305049in}{2.084312in}}%
\pgfpathclose%
\pgfusepath{stroke,fill}%
\end{pgfscope}%
\begin{pgfscope}%
\pgfpathrectangle{\pgfqpoint{0.100000in}{0.212622in}}{\pgfqpoint{3.696000in}{3.696000in}}%
\pgfusepath{clip}%
\pgfsetbuttcap%
\pgfsetroundjoin%
\definecolor{currentfill}{rgb}{0.121569,0.466667,0.705882}%
\pgfsetfillcolor{currentfill}%
\pgfsetfillopacity{0.584418}%
\pgfsetlinewidth{1.003750pt}%
\definecolor{currentstroke}{rgb}{0.121569,0.466667,0.705882}%
\pgfsetstrokecolor{currentstroke}%
\pgfsetstrokeopacity{0.584418}%
\pgfsetdash{}{0pt}%
\pgfpathmoveto{\pgfqpoint{3.305049in}{2.084312in}}%
\pgfpathcurveto{\pgfqpoint{3.313285in}{2.084312in}}{\pgfqpoint{3.321185in}{2.087584in}}{\pgfqpoint{3.327009in}{2.093408in}}%
\pgfpathcurveto{\pgfqpoint{3.332833in}{2.099232in}}{\pgfqpoint{3.336105in}{2.107132in}}{\pgfqpoint{3.336105in}{2.115368in}}%
\pgfpathcurveto{\pgfqpoint{3.336105in}{2.123604in}}{\pgfqpoint{3.332833in}{2.131504in}}{\pgfqpoint{3.327009in}{2.137328in}}%
\pgfpathcurveto{\pgfqpoint{3.321185in}{2.143152in}}{\pgfqpoint{3.313285in}{2.146425in}}{\pgfqpoint{3.305049in}{2.146425in}}%
\pgfpathcurveto{\pgfqpoint{3.296812in}{2.146425in}}{\pgfqpoint{3.288912in}{2.143152in}}{\pgfqpoint{3.283088in}{2.137328in}}%
\pgfpathcurveto{\pgfqpoint{3.277265in}{2.131504in}}{\pgfqpoint{3.273992in}{2.123604in}}{\pgfqpoint{3.273992in}{2.115368in}}%
\pgfpathcurveto{\pgfqpoint{3.273992in}{2.107132in}}{\pgfqpoint{3.277265in}{2.099232in}}{\pgfqpoint{3.283088in}{2.093408in}}%
\pgfpathcurveto{\pgfqpoint{3.288912in}{2.087584in}}{\pgfqpoint{3.296812in}{2.084312in}}{\pgfqpoint{3.305049in}{2.084312in}}%
\pgfpathclose%
\pgfusepath{stroke,fill}%
\end{pgfscope}%
\begin{pgfscope}%
\pgfpathrectangle{\pgfqpoint{0.100000in}{0.212622in}}{\pgfqpoint{3.696000in}{3.696000in}}%
\pgfusepath{clip}%
\pgfsetbuttcap%
\pgfsetroundjoin%
\definecolor{currentfill}{rgb}{0.121569,0.466667,0.705882}%
\pgfsetfillcolor{currentfill}%
\pgfsetfillopacity{0.584418}%
\pgfsetlinewidth{1.003750pt}%
\definecolor{currentstroke}{rgb}{0.121569,0.466667,0.705882}%
\pgfsetstrokecolor{currentstroke}%
\pgfsetstrokeopacity{0.584418}%
\pgfsetdash{}{0pt}%
\pgfpathmoveto{\pgfqpoint{3.305049in}{2.084312in}}%
\pgfpathcurveto{\pgfqpoint{3.313285in}{2.084312in}}{\pgfqpoint{3.321185in}{2.087584in}}{\pgfqpoint{3.327009in}{2.093408in}}%
\pgfpathcurveto{\pgfqpoint{3.332833in}{2.099232in}}{\pgfqpoint{3.336105in}{2.107132in}}{\pgfqpoint{3.336105in}{2.115368in}}%
\pgfpathcurveto{\pgfqpoint{3.336105in}{2.123604in}}{\pgfqpoint{3.332833in}{2.131504in}}{\pgfqpoint{3.327009in}{2.137328in}}%
\pgfpathcurveto{\pgfqpoint{3.321185in}{2.143152in}}{\pgfqpoint{3.313285in}{2.146425in}}{\pgfqpoint{3.305049in}{2.146425in}}%
\pgfpathcurveto{\pgfqpoint{3.296812in}{2.146425in}}{\pgfqpoint{3.288912in}{2.143152in}}{\pgfqpoint{3.283088in}{2.137328in}}%
\pgfpathcurveto{\pgfqpoint{3.277265in}{2.131504in}}{\pgfqpoint{3.273992in}{2.123604in}}{\pgfqpoint{3.273992in}{2.115368in}}%
\pgfpathcurveto{\pgfqpoint{3.273992in}{2.107132in}}{\pgfqpoint{3.277265in}{2.099232in}}{\pgfqpoint{3.283088in}{2.093408in}}%
\pgfpathcurveto{\pgfqpoint{3.288912in}{2.087584in}}{\pgfqpoint{3.296812in}{2.084312in}}{\pgfqpoint{3.305049in}{2.084312in}}%
\pgfpathclose%
\pgfusepath{stroke,fill}%
\end{pgfscope}%
\begin{pgfscope}%
\pgfpathrectangle{\pgfqpoint{0.100000in}{0.212622in}}{\pgfqpoint{3.696000in}{3.696000in}}%
\pgfusepath{clip}%
\pgfsetbuttcap%
\pgfsetroundjoin%
\definecolor{currentfill}{rgb}{0.121569,0.466667,0.705882}%
\pgfsetfillcolor{currentfill}%
\pgfsetfillopacity{0.584418}%
\pgfsetlinewidth{1.003750pt}%
\definecolor{currentstroke}{rgb}{0.121569,0.466667,0.705882}%
\pgfsetstrokecolor{currentstroke}%
\pgfsetstrokeopacity{0.584418}%
\pgfsetdash{}{0pt}%
\pgfpathmoveto{\pgfqpoint{3.305049in}{2.084312in}}%
\pgfpathcurveto{\pgfqpoint{3.313285in}{2.084312in}}{\pgfqpoint{3.321185in}{2.087584in}}{\pgfqpoint{3.327009in}{2.093408in}}%
\pgfpathcurveto{\pgfqpoint{3.332833in}{2.099232in}}{\pgfqpoint{3.336105in}{2.107132in}}{\pgfqpoint{3.336105in}{2.115368in}}%
\pgfpathcurveto{\pgfqpoint{3.336105in}{2.123604in}}{\pgfqpoint{3.332833in}{2.131504in}}{\pgfqpoint{3.327009in}{2.137328in}}%
\pgfpathcurveto{\pgfqpoint{3.321185in}{2.143152in}}{\pgfqpoint{3.313285in}{2.146425in}}{\pgfqpoint{3.305049in}{2.146425in}}%
\pgfpathcurveto{\pgfqpoint{3.296812in}{2.146425in}}{\pgfqpoint{3.288912in}{2.143152in}}{\pgfqpoint{3.283088in}{2.137328in}}%
\pgfpathcurveto{\pgfqpoint{3.277265in}{2.131504in}}{\pgfqpoint{3.273992in}{2.123604in}}{\pgfqpoint{3.273992in}{2.115368in}}%
\pgfpathcurveto{\pgfqpoint{3.273992in}{2.107132in}}{\pgfqpoint{3.277265in}{2.099232in}}{\pgfqpoint{3.283088in}{2.093408in}}%
\pgfpathcurveto{\pgfqpoint{3.288912in}{2.087584in}}{\pgfqpoint{3.296812in}{2.084312in}}{\pgfqpoint{3.305049in}{2.084312in}}%
\pgfpathclose%
\pgfusepath{stroke,fill}%
\end{pgfscope}%
\begin{pgfscope}%
\pgfpathrectangle{\pgfqpoint{0.100000in}{0.212622in}}{\pgfqpoint{3.696000in}{3.696000in}}%
\pgfusepath{clip}%
\pgfsetbuttcap%
\pgfsetroundjoin%
\definecolor{currentfill}{rgb}{0.121569,0.466667,0.705882}%
\pgfsetfillcolor{currentfill}%
\pgfsetfillopacity{0.584418}%
\pgfsetlinewidth{1.003750pt}%
\definecolor{currentstroke}{rgb}{0.121569,0.466667,0.705882}%
\pgfsetstrokecolor{currentstroke}%
\pgfsetstrokeopacity{0.584418}%
\pgfsetdash{}{0pt}%
\pgfpathmoveto{\pgfqpoint{3.305049in}{2.084312in}}%
\pgfpathcurveto{\pgfqpoint{3.313285in}{2.084312in}}{\pgfqpoint{3.321185in}{2.087584in}}{\pgfqpoint{3.327009in}{2.093408in}}%
\pgfpathcurveto{\pgfqpoint{3.332833in}{2.099232in}}{\pgfqpoint{3.336105in}{2.107132in}}{\pgfqpoint{3.336105in}{2.115368in}}%
\pgfpathcurveto{\pgfqpoint{3.336105in}{2.123604in}}{\pgfqpoint{3.332833in}{2.131504in}}{\pgfqpoint{3.327009in}{2.137328in}}%
\pgfpathcurveto{\pgfqpoint{3.321185in}{2.143152in}}{\pgfqpoint{3.313285in}{2.146425in}}{\pgfqpoint{3.305049in}{2.146425in}}%
\pgfpathcurveto{\pgfqpoint{3.296812in}{2.146425in}}{\pgfqpoint{3.288912in}{2.143152in}}{\pgfqpoint{3.283088in}{2.137328in}}%
\pgfpathcurveto{\pgfqpoint{3.277265in}{2.131504in}}{\pgfqpoint{3.273992in}{2.123604in}}{\pgfqpoint{3.273992in}{2.115368in}}%
\pgfpathcurveto{\pgfqpoint{3.273992in}{2.107132in}}{\pgfqpoint{3.277265in}{2.099232in}}{\pgfqpoint{3.283088in}{2.093408in}}%
\pgfpathcurveto{\pgfqpoint{3.288912in}{2.087584in}}{\pgfqpoint{3.296812in}{2.084312in}}{\pgfqpoint{3.305049in}{2.084312in}}%
\pgfpathclose%
\pgfusepath{stroke,fill}%
\end{pgfscope}%
\begin{pgfscope}%
\pgfpathrectangle{\pgfqpoint{0.100000in}{0.212622in}}{\pgfqpoint{3.696000in}{3.696000in}}%
\pgfusepath{clip}%
\pgfsetbuttcap%
\pgfsetroundjoin%
\definecolor{currentfill}{rgb}{0.121569,0.466667,0.705882}%
\pgfsetfillcolor{currentfill}%
\pgfsetfillopacity{0.584418}%
\pgfsetlinewidth{1.003750pt}%
\definecolor{currentstroke}{rgb}{0.121569,0.466667,0.705882}%
\pgfsetstrokecolor{currentstroke}%
\pgfsetstrokeopacity{0.584418}%
\pgfsetdash{}{0pt}%
\pgfpathmoveto{\pgfqpoint{3.305049in}{2.084312in}}%
\pgfpathcurveto{\pgfqpoint{3.313285in}{2.084312in}}{\pgfqpoint{3.321185in}{2.087584in}}{\pgfqpoint{3.327009in}{2.093408in}}%
\pgfpathcurveto{\pgfqpoint{3.332833in}{2.099232in}}{\pgfqpoint{3.336105in}{2.107132in}}{\pgfqpoint{3.336105in}{2.115368in}}%
\pgfpathcurveto{\pgfqpoint{3.336105in}{2.123604in}}{\pgfqpoint{3.332833in}{2.131504in}}{\pgfqpoint{3.327009in}{2.137328in}}%
\pgfpathcurveto{\pgfqpoint{3.321185in}{2.143152in}}{\pgfqpoint{3.313285in}{2.146425in}}{\pgfqpoint{3.305049in}{2.146425in}}%
\pgfpathcurveto{\pgfqpoint{3.296812in}{2.146425in}}{\pgfqpoint{3.288912in}{2.143152in}}{\pgfqpoint{3.283088in}{2.137328in}}%
\pgfpathcurveto{\pgfqpoint{3.277265in}{2.131504in}}{\pgfqpoint{3.273992in}{2.123604in}}{\pgfqpoint{3.273992in}{2.115368in}}%
\pgfpathcurveto{\pgfqpoint{3.273992in}{2.107132in}}{\pgfqpoint{3.277265in}{2.099232in}}{\pgfqpoint{3.283088in}{2.093408in}}%
\pgfpathcurveto{\pgfqpoint{3.288912in}{2.087584in}}{\pgfqpoint{3.296812in}{2.084312in}}{\pgfqpoint{3.305049in}{2.084312in}}%
\pgfpathclose%
\pgfusepath{stroke,fill}%
\end{pgfscope}%
\begin{pgfscope}%
\pgfpathrectangle{\pgfqpoint{0.100000in}{0.212622in}}{\pgfqpoint{3.696000in}{3.696000in}}%
\pgfusepath{clip}%
\pgfsetbuttcap%
\pgfsetroundjoin%
\definecolor{currentfill}{rgb}{0.121569,0.466667,0.705882}%
\pgfsetfillcolor{currentfill}%
\pgfsetfillopacity{0.584418}%
\pgfsetlinewidth{1.003750pt}%
\definecolor{currentstroke}{rgb}{0.121569,0.466667,0.705882}%
\pgfsetstrokecolor{currentstroke}%
\pgfsetstrokeopacity{0.584418}%
\pgfsetdash{}{0pt}%
\pgfpathmoveto{\pgfqpoint{3.305049in}{2.084312in}}%
\pgfpathcurveto{\pgfqpoint{3.313285in}{2.084312in}}{\pgfqpoint{3.321185in}{2.087584in}}{\pgfqpoint{3.327009in}{2.093408in}}%
\pgfpathcurveto{\pgfqpoint{3.332833in}{2.099232in}}{\pgfqpoint{3.336105in}{2.107132in}}{\pgfqpoint{3.336105in}{2.115368in}}%
\pgfpathcurveto{\pgfqpoint{3.336105in}{2.123604in}}{\pgfqpoint{3.332833in}{2.131504in}}{\pgfqpoint{3.327009in}{2.137328in}}%
\pgfpathcurveto{\pgfqpoint{3.321185in}{2.143152in}}{\pgfqpoint{3.313285in}{2.146425in}}{\pgfqpoint{3.305049in}{2.146425in}}%
\pgfpathcurveto{\pgfqpoint{3.296812in}{2.146425in}}{\pgfqpoint{3.288912in}{2.143152in}}{\pgfqpoint{3.283088in}{2.137328in}}%
\pgfpathcurveto{\pgfqpoint{3.277265in}{2.131504in}}{\pgfqpoint{3.273992in}{2.123604in}}{\pgfqpoint{3.273992in}{2.115368in}}%
\pgfpathcurveto{\pgfqpoint{3.273992in}{2.107132in}}{\pgfqpoint{3.277265in}{2.099232in}}{\pgfqpoint{3.283088in}{2.093408in}}%
\pgfpathcurveto{\pgfqpoint{3.288912in}{2.087584in}}{\pgfqpoint{3.296812in}{2.084312in}}{\pgfqpoint{3.305049in}{2.084312in}}%
\pgfpathclose%
\pgfusepath{stroke,fill}%
\end{pgfscope}%
\begin{pgfscope}%
\pgfpathrectangle{\pgfqpoint{0.100000in}{0.212622in}}{\pgfqpoint{3.696000in}{3.696000in}}%
\pgfusepath{clip}%
\pgfsetbuttcap%
\pgfsetroundjoin%
\definecolor{currentfill}{rgb}{0.121569,0.466667,0.705882}%
\pgfsetfillcolor{currentfill}%
\pgfsetfillopacity{0.584418}%
\pgfsetlinewidth{1.003750pt}%
\definecolor{currentstroke}{rgb}{0.121569,0.466667,0.705882}%
\pgfsetstrokecolor{currentstroke}%
\pgfsetstrokeopacity{0.584418}%
\pgfsetdash{}{0pt}%
\pgfpathmoveto{\pgfqpoint{3.305049in}{2.084312in}}%
\pgfpathcurveto{\pgfqpoint{3.313285in}{2.084312in}}{\pgfqpoint{3.321185in}{2.087584in}}{\pgfqpoint{3.327009in}{2.093408in}}%
\pgfpathcurveto{\pgfqpoint{3.332833in}{2.099232in}}{\pgfqpoint{3.336105in}{2.107132in}}{\pgfqpoint{3.336105in}{2.115368in}}%
\pgfpathcurveto{\pgfqpoint{3.336105in}{2.123604in}}{\pgfqpoint{3.332833in}{2.131504in}}{\pgfqpoint{3.327009in}{2.137328in}}%
\pgfpathcurveto{\pgfqpoint{3.321185in}{2.143152in}}{\pgfqpoint{3.313285in}{2.146425in}}{\pgfqpoint{3.305049in}{2.146425in}}%
\pgfpathcurveto{\pgfqpoint{3.296812in}{2.146425in}}{\pgfqpoint{3.288912in}{2.143152in}}{\pgfqpoint{3.283088in}{2.137328in}}%
\pgfpathcurveto{\pgfqpoint{3.277265in}{2.131504in}}{\pgfqpoint{3.273992in}{2.123604in}}{\pgfqpoint{3.273992in}{2.115368in}}%
\pgfpathcurveto{\pgfqpoint{3.273992in}{2.107132in}}{\pgfqpoint{3.277265in}{2.099232in}}{\pgfqpoint{3.283088in}{2.093408in}}%
\pgfpathcurveto{\pgfqpoint{3.288912in}{2.087584in}}{\pgfqpoint{3.296812in}{2.084312in}}{\pgfqpoint{3.305049in}{2.084312in}}%
\pgfpathclose%
\pgfusepath{stroke,fill}%
\end{pgfscope}%
\begin{pgfscope}%
\pgfpathrectangle{\pgfqpoint{0.100000in}{0.212622in}}{\pgfqpoint{3.696000in}{3.696000in}}%
\pgfusepath{clip}%
\pgfsetbuttcap%
\pgfsetroundjoin%
\definecolor{currentfill}{rgb}{0.121569,0.466667,0.705882}%
\pgfsetfillcolor{currentfill}%
\pgfsetfillopacity{0.584418}%
\pgfsetlinewidth{1.003750pt}%
\definecolor{currentstroke}{rgb}{0.121569,0.466667,0.705882}%
\pgfsetstrokecolor{currentstroke}%
\pgfsetstrokeopacity{0.584418}%
\pgfsetdash{}{0pt}%
\pgfpathmoveto{\pgfqpoint{3.305049in}{2.084312in}}%
\pgfpathcurveto{\pgfqpoint{3.313285in}{2.084312in}}{\pgfqpoint{3.321185in}{2.087584in}}{\pgfqpoint{3.327009in}{2.093408in}}%
\pgfpathcurveto{\pgfqpoint{3.332833in}{2.099232in}}{\pgfqpoint{3.336105in}{2.107132in}}{\pgfqpoint{3.336105in}{2.115368in}}%
\pgfpathcurveto{\pgfqpoint{3.336105in}{2.123604in}}{\pgfqpoint{3.332833in}{2.131504in}}{\pgfqpoint{3.327009in}{2.137328in}}%
\pgfpathcurveto{\pgfqpoint{3.321185in}{2.143152in}}{\pgfqpoint{3.313285in}{2.146425in}}{\pgfqpoint{3.305049in}{2.146425in}}%
\pgfpathcurveto{\pgfqpoint{3.296812in}{2.146425in}}{\pgfqpoint{3.288912in}{2.143152in}}{\pgfqpoint{3.283088in}{2.137328in}}%
\pgfpathcurveto{\pgfqpoint{3.277265in}{2.131504in}}{\pgfqpoint{3.273992in}{2.123604in}}{\pgfqpoint{3.273992in}{2.115368in}}%
\pgfpathcurveto{\pgfqpoint{3.273992in}{2.107132in}}{\pgfqpoint{3.277265in}{2.099232in}}{\pgfqpoint{3.283088in}{2.093408in}}%
\pgfpathcurveto{\pgfqpoint{3.288912in}{2.087584in}}{\pgfqpoint{3.296812in}{2.084312in}}{\pgfqpoint{3.305049in}{2.084312in}}%
\pgfpathclose%
\pgfusepath{stroke,fill}%
\end{pgfscope}%
\begin{pgfscope}%
\pgfpathrectangle{\pgfqpoint{0.100000in}{0.212622in}}{\pgfqpoint{3.696000in}{3.696000in}}%
\pgfusepath{clip}%
\pgfsetbuttcap%
\pgfsetroundjoin%
\definecolor{currentfill}{rgb}{0.121569,0.466667,0.705882}%
\pgfsetfillcolor{currentfill}%
\pgfsetfillopacity{0.584418}%
\pgfsetlinewidth{1.003750pt}%
\definecolor{currentstroke}{rgb}{0.121569,0.466667,0.705882}%
\pgfsetstrokecolor{currentstroke}%
\pgfsetstrokeopacity{0.584418}%
\pgfsetdash{}{0pt}%
\pgfpathmoveto{\pgfqpoint{3.305049in}{2.084312in}}%
\pgfpathcurveto{\pgfqpoint{3.313285in}{2.084312in}}{\pgfqpoint{3.321185in}{2.087584in}}{\pgfqpoint{3.327009in}{2.093408in}}%
\pgfpathcurveto{\pgfqpoint{3.332833in}{2.099232in}}{\pgfqpoint{3.336105in}{2.107132in}}{\pgfqpoint{3.336105in}{2.115368in}}%
\pgfpathcurveto{\pgfqpoint{3.336105in}{2.123604in}}{\pgfqpoint{3.332833in}{2.131504in}}{\pgfqpoint{3.327009in}{2.137328in}}%
\pgfpathcurveto{\pgfqpoint{3.321185in}{2.143152in}}{\pgfqpoint{3.313285in}{2.146425in}}{\pgfqpoint{3.305049in}{2.146425in}}%
\pgfpathcurveto{\pgfqpoint{3.296812in}{2.146425in}}{\pgfqpoint{3.288912in}{2.143152in}}{\pgfqpoint{3.283088in}{2.137328in}}%
\pgfpathcurveto{\pgfqpoint{3.277265in}{2.131504in}}{\pgfqpoint{3.273992in}{2.123604in}}{\pgfqpoint{3.273992in}{2.115368in}}%
\pgfpathcurveto{\pgfqpoint{3.273992in}{2.107132in}}{\pgfqpoint{3.277265in}{2.099232in}}{\pgfqpoint{3.283088in}{2.093408in}}%
\pgfpathcurveto{\pgfqpoint{3.288912in}{2.087584in}}{\pgfqpoint{3.296812in}{2.084312in}}{\pgfqpoint{3.305049in}{2.084312in}}%
\pgfpathclose%
\pgfusepath{stroke,fill}%
\end{pgfscope}%
\begin{pgfscope}%
\pgfpathrectangle{\pgfqpoint{0.100000in}{0.212622in}}{\pgfqpoint{3.696000in}{3.696000in}}%
\pgfusepath{clip}%
\pgfsetbuttcap%
\pgfsetroundjoin%
\definecolor{currentfill}{rgb}{0.121569,0.466667,0.705882}%
\pgfsetfillcolor{currentfill}%
\pgfsetfillopacity{0.584418}%
\pgfsetlinewidth{1.003750pt}%
\definecolor{currentstroke}{rgb}{0.121569,0.466667,0.705882}%
\pgfsetstrokecolor{currentstroke}%
\pgfsetstrokeopacity{0.584418}%
\pgfsetdash{}{0pt}%
\pgfpathmoveto{\pgfqpoint{3.305049in}{2.084312in}}%
\pgfpathcurveto{\pgfqpoint{3.313285in}{2.084312in}}{\pgfqpoint{3.321185in}{2.087584in}}{\pgfqpoint{3.327009in}{2.093408in}}%
\pgfpathcurveto{\pgfqpoint{3.332833in}{2.099232in}}{\pgfqpoint{3.336105in}{2.107132in}}{\pgfqpoint{3.336105in}{2.115368in}}%
\pgfpathcurveto{\pgfqpoint{3.336105in}{2.123604in}}{\pgfqpoint{3.332833in}{2.131504in}}{\pgfqpoint{3.327009in}{2.137328in}}%
\pgfpathcurveto{\pgfqpoint{3.321185in}{2.143152in}}{\pgfqpoint{3.313285in}{2.146425in}}{\pgfqpoint{3.305049in}{2.146425in}}%
\pgfpathcurveto{\pgfqpoint{3.296812in}{2.146425in}}{\pgfqpoint{3.288912in}{2.143152in}}{\pgfqpoint{3.283088in}{2.137328in}}%
\pgfpathcurveto{\pgfqpoint{3.277265in}{2.131504in}}{\pgfqpoint{3.273992in}{2.123604in}}{\pgfqpoint{3.273992in}{2.115368in}}%
\pgfpathcurveto{\pgfqpoint{3.273992in}{2.107132in}}{\pgfqpoint{3.277265in}{2.099232in}}{\pgfqpoint{3.283088in}{2.093408in}}%
\pgfpathcurveto{\pgfqpoint{3.288912in}{2.087584in}}{\pgfqpoint{3.296812in}{2.084312in}}{\pgfqpoint{3.305049in}{2.084312in}}%
\pgfpathclose%
\pgfusepath{stroke,fill}%
\end{pgfscope}%
\begin{pgfscope}%
\pgfpathrectangle{\pgfqpoint{0.100000in}{0.212622in}}{\pgfqpoint{3.696000in}{3.696000in}}%
\pgfusepath{clip}%
\pgfsetbuttcap%
\pgfsetroundjoin%
\definecolor{currentfill}{rgb}{0.121569,0.466667,0.705882}%
\pgfsetfillcolor{currentfill}%
\pgfsetfillopacity{0.584418}%
\pgfsetlinewidth{1.003750pt}%
\definecolor{currentstroke}{rgb}{0.121569,0.466667,0.705882}%
\pgfsetstrokecolor{currentstroke}%
\pgfsetstrokeopacity{0.584418}%
\pgfsetdash{}{0pt}%
\pgfpathmoveto{\pgfqpoint{3.305049in}{2.084312in}}%
\pgfpathcurveto{\pgfqpoint{3.313285in}{2.084312in}}{\pgfqpoint{3.321185in}{2.087584in}}{\pgfqpoint{3.327009in}{2.093408in}}%
\pgfpathcurveto{\pgfqpoint{3.332833in}{2.099232in}}{\pgfqpoint{3.336105in}{2.107132in}}{\pgfqpoint{3.336105in}{2.115368in}}%
\pgfpathcurveto{\pgfqpoint{3.336105in}{2.123604in}}{\pgfqpoint{3.332833in}{2.131504in}}{\pgfqpoint{3.327009in}{2.137328in}}%
\pgfpathcurveto{\pgfqpoint{3.321185in}{2.143152in}}{\pgfqpoint{3.313285in}{2.146425in}}{\pgfqpoint{3.305049in}{2.146425in}}%
\pgfpathcurveto{\pgfqpoint{3.296812in}{2.146425in}}{\pgfqpoint{3.288912in}{2.143152in}}{\pgfqpoint{3.283088in}{2.137328in}}%
\pgfpathcurveto{\pgfqpoint{3.277265in}{2.131504in}}{\pgfqpoint{3.273992in}{2.123604in}}{\pgfqpoint{3.273992in}{2.115368in}}%
\pgfpathcurveto{\pgfqpoint{3.273992in}{2.107132in}}{\pgfqpoint{3.277265in}{2.099232in}}{\pgfqpoint{3.283088in}{2.093408in}}%
\pgfpathcurveto{\pgfqpoint{3.288912in}{2.087584in}}{\pgfqpoint{3.296812in}{2.084312in}}{\pgfqpoint{3.305049in}{2.084312in}}%
\pgfpathclose%
\pgfusepath{stroke,fill}%
\end{pgfscope}%
\begin{pgfscope}%
\pgfpathrectangle{\pgfqpoint{0.100000in}{0.212622in}}{\pgfqpoint{3.696000in}{3.696000in}}%
\pgfusepath{clip}%
\pgfsetbuttcap%
\pgfsetroundjoin%
\definecolor{currentfill}{rgb}{0.121569,0.466667,0.705882}%
\pgfsetfillcolor{currentfill}%
\pgfsetfillopacity{0.584418}%
\pgfsetlinewidth{1.003750pt}%
\definecolor{currentstroke}{rgb}{0.121569,0.466667,0.705882}%
\pgfsetstrokecolor{currentstroke}%
\pgfsetstrokeopacity{0.584418}%
\pgfsetdash{}{0pt}%
\pgfpathmoveto{\pgfqpoint{3.305049in}{2.084312in}}%
\pgfpathcurveto{\pgfqpoint{3.313285in}{2.084312in}}{\pgfqpoint{3.321185in}{2.087584in}}{\pgfqpoint{3.327009in}{2.093408in}}%
\pgfpathcurveto{\pgfqpoint{3.332833in}{2.099232in}}{\pgfqpoint{3.336105in}{2.107132in}}{\pgfqpoint{3.336105in}{2.115368in}}%
\pgfpathcurveto{\pgfqpoint{3.336105in}{2.123604in}}{\pgfqpoint{3.332833in}{2.131504in}}{\pgfqpoint{3.327009in}{2.137328in}}%
\pgfpathcurveto{\pgfqpoint{3.321185in}{2.143152in}}{\pgfqpoint{3.313285in}{2.146425in}}{\pgfqpoint{3.305049in}{2.146425in}}%
\pgfpathcurveto{\pgfqpoint{3.296812in}{2.146425in}}{\pgfqpoint{3.288912in}{2.143152in}}{\pgfqpoint{3.283088in}{2.137328in}}%
\pgfpathcurveto{\pgfqpoint{3.277265in}{2.131504in}}{\pgfqpoint{3.273992in}{2.123604in}}{\pgfqpoint{3.273992in}{2.115368in}}%
\pgfpathcurveto{\pgfqpoint{3.273992in}{2.107132in}}{\pgfqpoint{3.277265in}{2.099232in}}{\pgfqpoint{3.283088in}{2.093408in}}%
\pgfpathcurveto{\pgfqpoint{3.288912in}{2.087584in}}{\pgfqpoint{3.296812in}{2.084312in}}{\pgfqpoint{3.305049in}{2.084312in}}%
\pgfpathclose%
\pgfusepath{stroke,fill}%
\end{pgfscope}%
\begin{pgfscope}%
\pgfpathrectangle{\pgfqpoint{0.100000in}{0.212622in}}{\pgfqpoint{3.696000in}{3.696000in}}%
\pgfusepath{clip}%
\pgfsetbuttcap%
\pgfsetroundjoin%
\definecolor{currentfill}{rgb}{0.121569,0.466667,0.705882}%
\pgfsetfillcolor{currentfill}%
\pgfsetfillopacity{0.584418}%
\pgfsetlinewidth{1.003750pt}%
\definecolor{currentstroke}{rgb}{0.121569,0.466667,0.705882}%
\pgfsetstrokecolor{currentstroke}%
\pgfsetstrokeopacity{0.584418}%
\pgfsetdash{}{0pt}%
\pgfpathmoveto{\pgfqpoint{3.305049in}{2.084312in}}%
\pgfpathcurveto{\pgfqpoint{3.313285in}{2.084312in}}{\pgfqpoint{3.321185in}{2.087584in}}{\pgfqpoint{3.327009in}{2.093408in}}%
\pgfpathcurveto{\pgfqpoint{3.332833in}{2.099232in}}{\pgfqpoint{3.336105in}{2.107132in}}{\pgfqpoint{3.336105in}{2.115368in}}%
\pgfpathcurveto{\pgfqpoint{3.336105in}{2.123604in}}{\pgfqpoint{3.332833in}{2.131504in}}{\pgfqpoint{3.327009in}{2.137328in}}%
\pgfpathcurveto{\pgfqpoint{3.321185in}{2.143152in}}{\pgfqpoint{3.313285in}{2.146425in}}{\pgfqpoint{3.305049in}{2.146425in}}%
\pgfpathcurveto{\pgfqpoint{3.296812in}{2.146425in}}{\pgfqpoint{3.288912in}{2.143152in}}{\pgfqpoint{3.283088in}{2.137328in}}%
\pgfpathcurveto{\pgfqpoint{3.277265in}{2.131504in}}{\pgfqpoint{3.273992in}{2.123604in}}{\pgfqpoint{3.273992in}{2.115368in}}%
\pgfpathcurveto{\pgfqpoint{3.273992in}{2.107132in}}{\pgfqpoint{3.277265in}{2.099232in}}{\pgfqpoint{3.283088in}{2.093408in}}%
\pgfpathcurveto{\pgfqpoint{3.288912in}{2.087584in}}{\pgfqpoint{3.296812in}{2.084312in}}{\pgfqpoint{3.305049in}{2.084312in}}%
\pgfpathclose%
\pgfusepath{stroke,fill}%
\end{pgfscope}%
\begin{pgfscope}%
\pgfpathrectangle{\pgfqpoint{0.100000in}{0.212622in}}{\pgfqpoint{3.696000in}{3.696000in}}%
\pgfusepath{clip}%
\pgfsetbuttcap%
\pgfsetroundjoin%
\definecolor{currentfill}{rgb}{0.121569,0.466667,0.705882}%
\pgfsetfillcolor{currentfill}%
\pgfsetfillopacity{0.584418}%
\pgfsetlinewidth{1.003750pt}%
\definecolor{currentstroke}{rgb}{0.121569,0.466667,0.705882}%
\pgfsetstrokecolor{currentstroke}%
\pgfsetstrokeopacity{0.584418}%
\pgfsetdash{}{0pt}%
\pgfpathmoveto{\pgfqpoint{3.305049in}{2.084312in}}%
\pgfpathcurveto{\pgfqpoint{3.313285in}{2.084312in}}{\pgfqpoint{3.321185in}{2.087584in}}{\pgfqpoint{3.327009in}{2.093408in}}%
\pgfpathcurveto{\pgfqpoint{3.332833in}{2.099232in}}{\pgfqpoint{3.336105in}{2.107132in}}{\pgfqpoint{3.336105in}{2.115368in}}%
\pgfpathcurveto{\pgfqpoint{3.336105in}{2.123604in}}{\pgfqpoint{3.332833in}{2.131504in}}{\pgfqpoint{3.327009in}{2.137328in}}%
\pgfpathcurveto{\pgfqpoint{3.321185in}{2.143152in}}{\pgfqpoint{3.313285in}{2.146425in}}{\pgfqpoint{3.305049in}{2.146425in}}%
\pgfpathcurveto{\pgfqpoint{3.296812in}{2.146425in}}{\pgfqpoint{3.288912in}{2.143152in}}{\pgfqpoint{3.283088in}{2.137328in}}%
\pgfpathcurveto{\pgfqpoint{3.277265in}{2.131504in}}{\pgfqpoint{3.273992in}{2.123604in}}{\pgfqpoint{3.273992in}{2.115368in}}%
\pgfpathcurveto{\pgfqpoint{3.273992in}{2.107132in}}{\pgfqpoint{3.277265in}{2.099232in}}{\pgfqpoint{3.283088in}{2.093408in}}%
\pgfpathcurveto{\pgfqpoint{3.288912in}{2.087584in}}{\pgfqpoint{3.296812in}{2.084312in}}{\pgfqpoint{3.305049in}{2.084312in}}%
\pgfpathclose%
\pgfusepath{stroke,fill}%
\end{pgfscope}%
\begin{pgfscope}%
\pgfpathrectangle{\pgfqpoint{0.100000in}{0.212622in}}{\pgfqpoint{3.696000in}{3.696000in}}%
\pgfusepath{clip}%
\pgfsetbuttcap%
\pgfsetroundjoin%
\definecolor{currentfill}{rgb}{0.121569,0.466667,0.705882}%
\pgfsetfillcolor{currentfill}%
\pgfsetfillopacity{0.584418}%
\pgfsetlinewidth{1.003750pt}%
\definecolor{currentstroke}{rgb}{0.121569,0.466667,0.705882}%
\pgfsetstrokecolor{currentstroke}%
\pgfsetstrokeopacity{0.584418}%
\pgfsetdash{}{0pt}%
\pgfpathmoveto{\pgfqpoint{3.305049in}{2.084312in}}%
\pgfpathcurveto{\pgfqpoint{3.313285in}{2.084312in}}{\pgfqpoint{3.321185in}{2.087584in}}{\pgfqpoint{3.327009in}{2.093408in}}%
\pgfpathcurveto{\pgfqpoint{3.332833in}{2.099232in}}{\pgfqpoint{3.336105in}{2.107132in}}{\pgfqpoint{3.336105in}{2.115368in}}%
\pgfpathcurveto{\pgfqpoint{3.336105in}{2.123604in}}{\pgfqpoint{3.332833in}{2.131504in}}{\pgfqpoint{3.327009in}{2.137328in}}%
\pgfpathcurveto{\pgfqpoint{3.321185in}{2.143152in}}{\pgfqpoint{3.313285in}{2.146425in}}{\pgfqpoint{3.305049in}{2.146425in}}%
\pgfpathcurveto{\pgfqpoint{3.296812in}{2.146425in}}{\pgfqpoint{3.288912in}{2.143152in}}{\pgfqpoint{3.283088in}{2.137328in}}%
\pgfpathcurveto{\pgfqpoint{3.277265in}{2.131504in}}{\pgfqpoint{3.273992in}{2.123604in}}{\pgfqpoint{3.273992in}{2.115368in}}%
\pgfpathcurveto{\pgfqpoint{3.273992in}{2.107132in}}{\pgfqpoint{3.277265in}{2.099232in}}{\pgfqpoint{3.283088in}{2.093408in}}%
\pgfpathcurveto{\pgfqpoint{3.288912in}{2.087584in}}{\pgfqpoint{3.296812in}{2.084312in}}{\pgfqpoint{3.305049in}{2.084312in}}%
\pgfpathclose%
\pgfusepath{stroke,fill}%
\end{pgfscope}%
\begin{pgfscope}%
\pgfpathrectangle{\pgfqpoint{0.100000in}{0.212622in}}{\pgfqpoint{3.696000in}{3.696000in}}%
\pgfusepath{clip}%
\pgfsetbuttcap%
\pgfsetroundjoin%
\definecolor{currentfill}{rgb}{0.121569,0.466667,0.705882}%
\pgfsetfillcolor{currentfill}%
\pgfsetfillopacity{0.584418}%
\pgfsetlinewidth{1.003750pt}%
\definecolor{currentstroke}{rgb}{0.121569,0.466667,0.705882}%
\pgfsetstrokecolor{currentstroke}%
\pgfsetstrokeopacity{0.584418}%
\pgfsetdash{}{0pt}%
\pgfpathmoveto{\pgfqpoint{3.305049in}{2.084312in}}%
\pgfpathcurveto{\pgfqpoint{3.313285in}{2.084312in}}{\pgfqpoint{3.321185in}{2.087584in}}{\pgfqpoint{3.327009in}{2.093408in}}%
\pgfpathcurveto{\pgfqpoint{3.332833in}{2.099232in}}{\pgfqpoint{3.336105in}{2.107132in}}{\pgfqpoint{3.336105in}{2.115368in}}%
\pgfpathcurveto{\pgfqpoint{3.336105in}{2.123604in}}{\pgfqpoint{3.332833in}{2.131504in}}{\pgfqpoint{3.327009in}{2.137328in}}%
\pgfpathcurveto{\pgfqpoint{3.321185in}{2.143152in}}{\pgfqpoint{3.313285in}{2.146425in}}{\pgfqpoint{3.305049in}{2.146425in}}%
\pgfpathcurveto{\pgfqpoint{3.296812in}{2.146425in}}{\pgfqpoint{3.288912in}{2.143152in}}{\pgfqpoint{3.283088in}{2.137328in}}%
\pgfpathcurveto{\pgfqpoint{3.277265in}{2.131504in}}{\pgfqpoint{3.273992in}{2.123604in}}{\pgfqpoint{3.273992in}{2.115368in}}%
\pgfpathcurveto{\pgfqpoint{3.273992in}{2.107132in}}{\pgfqpoint{3.277265in}{2.099232in}}{\pgfqpoint{3.283088in}{2.093408in}}%
\pgfpathcurveto{\pgfqpoint{3.288912in}{2.087584in}}{\pgfqpoint{3.296812in}{2.084312in}}{\pgfqpoint{3.305049in}{2.084312in}}%
\pgfpathclose%
\pgfusepath{stroke,fill}%
\end{pgfscope}%
\begin{pgfscope}%
\pgfpathrectangle{\pgfqpoint{0.100000in}{0.212622in}}{\pgfqpoint{3.696000in}{3.696000in}}%
\pgfusepath{clip}%
\pgfsetbuttcap%
\pgfsetroundjoin%
\definecolor{currentfill}{rgb}{0.121569,0.466667,0.705882}%
\pgfsetfillcolor{currentfill}%
\pgfsetfillopacity{0.584418}%
\pgfsetlinewidth{1.003750pt}%
\definecolor{currentstroke}{rgb}{0.121569,0.466667,0.705882}%
\pgfsetstrokecolor{currentstroke}%
\pgfsetstrokeopacity{0.584418}%
\pgfsetdash{}{0pt}%
\pgfpathmoveto{\pgfqpoint{3.305049in}{2.084312in}}%
\pgfpathcurveto{\pgfqpoint{3.313285in}{2.084312in}}{\pgfqpoint{3.321185in}{2.087584in}}{\pgfqpoint{3.327009in}{2.093408in}}%
\pgfpathcurveto{\pgfqpoint{3.332833in}{2.099232in}}{\pgfqpoint{3.336105in}{2.107132in}}{\pgfqpoint{3.336105in}{2.115368in}}%
\pgfpathcurveto{\pgfqpoint{3.336105in}{2.123604in}}{\pgfqpoint{3.332833in}{2.131504in}}{\pgfqpoint{3.327009in}{2.137328in}}%
\pgfpathcurveto{\pgfqpoint{3.321185in}{2.143152in}}{\pgfqpoint{3.313285in}{2.146425in}}{\pgfqpoint{3.305049in}{2.146425in}}%
\pgfpathcurveto{\pgfqpoint{3.296812in}{2.146425in}}{\pgfqpoint{3.288912in}{2.143152in}}{\pgfqpoint{3.283088in}{2.137328in}}%
\pgfpathcurveto{\pgfqpoint{3.277265in}{2.131504in}}{\pgfqpoint{3.273992in}{2.123604in}}{\pgfqpoint{3.273992in}{2.115368in}}%
\pgfpathcurveto{\pgfqpoint{3.273992in}{2.107132in}}{\pgfqpoint{3.277265in}{2.099232in}}{\pgfqpoint{3.283088in}{2.093408in}}%
\pgfpathcurveto{\pgfqpoint{3.288912in}{2.087584in}}{\pgfqpoint{3.296812in}{2.084312in}}{\pgfqpoint{3.305049in}{2.084312in}}%
\pgfpathclose%
\pgfusepath{stroke,fill}%
\end{pgfscope}%
\begin{pgfscope}%
\pgfpathrectangle{\pgfqpoint{0.100000in}{0.212622in}}{\pgfqpoint{3.696000in}{3.696000in}}%
\pgfusepath{clip}%
\pgfsetbuttcap%
\pgfsetroundjoin%
\definecolor{currentfill}{rgb}{0.121569,0.466667,0.705882}%
\pgfsetfillcolor{currentfill}%
\pgfsetfillopacity{0.584418}%
\pgfsetlinewidth{1.003750pt}%
\definecolor{currentstroke}{rgb}{0.121569,0.466667,0.705882}%
\pgfsetstrokecolor{currentstroke}%
\pgfsetstrokeopacity{0.584418}%
\pgfsetdash{}{0pt}%
\pgfpathmoveto{\pgfqpoint{3.305049in}{2.084312in}}%
\pgfpathcurveto{\pgfqpoint{3.313285in}{2.084312in}}{\pgfqpoint{3.321185in}{2.087584in}}{\pgfqpoint{3.327009in}{2.093408in}}%
\pgfpathcurveto{\pgfqpoint{3.332833in}{2.099232in}}{\pgfqpoint{3.336105in}{2.107132in}}{\pgfqpoint{3.336105in}{2.115368in}}%
\pgfpathcurveto{\pgfqpoint{3.336105in}{2.123604in}}{\pgfqpoint{3.332833in}{2.131504in}}{\pgfqpoint{3.327009in}{2.137328in}}%
\pgfpathcurveto{\pgfqpoint{3.321185in}{2.143152in}}{\pgfqpoint{3.313285in}{2.146425in}}{\pgfqpoint{3.305049in}{2.146425in}}%
\pgfpathcurveto{\pgfqpoint{3.296812in}{2.146425in}}{\pgfqpoint{3.288912in}{2.143152in}}{\pgfqpoint{3.283088in}{2.137328in}}%
\pgfpathcurveto{\pgfqpoint{3.277265in}{2.131504in}}{\pgfqpoint{3.273992in}{2.123604in}}{\pgfqpoint{3.273992in}{2.115368in}}%
\pgfpathcurveto{\pgfqpoint{3.273992in}{2.107132in}}{\pgfqpoint{3.277265in}{2.099232in}}{\pgfqpoint{3.283088in}{2.093408in}}%
\pgfpathcurveto{\pgfqpoint{3.288912in}{2.087584in}}{\pgfqpoint{3.296812in}{2.084312in}}{\pgfqpoint{3.305049in}{2.084312in}}%
\pgfpathclose%
\pgfusepath{stroke,fill}%
\end{pgfscope}%
\begin{pgfscope}%
\pgfpathrectangle{\pgfqpoint{0.100000in}{0.212622in}}{\pgfqpoint{3.696000in}{3.696000in}}%
\pgfusepath{clip}%
\pgfsetbuttcap%
\pgfsetroundjoin%
\definecolor{currentfill}{rgb}{0.121569,0.466667,0.705882}%
\pgfsetfillcolor{currentfill}%
\pgfsetfillopacity{0.584418}%
\pgfsetlinewidth{1.003750pt}%
\definecolor{currentstroke}{rgb}{0.121569,0.466667,0.705882}%
\pgfsetstrokecolor{currentstroke}%
\pgfsetstrokeopacity{0.584418}%
\pgfsetdash{}{0pt}%
\pgfpathmoveto{\pgfqpoint{3.305049in}{2.084312in}}%
\pgfpathcurveto{\pgfqpoint{3.313285in}{2.084312in}}{\pgfqpoint{3.321185in}{2.087584in}}{\pgfqpoint{3.327009in}{2.093408in}}%
\pgfpathcurveto{\pgfqpoint{3.332833in}{2.099232in}}{\pgfqpoint{3.336105in}{2.107132in}}{\pgfqpoint{3.336105in}{2.115368in}}%
\pgfpathcurveto{\pgfqpoint{3.336105in}{2.123604in}}{\pgfqpoint{3.332833in}{2.131504in}}{\pgfqpoint{3.327009in}{2.137328in}}%
\pgfpathcurveto{\pgfqpoint{3.321185in}{2.143152in}}{\pgfqpoint{3.313285in}{2.146425in}}{\pgfqpoint{3.305049in}{2.146425in}}%
\pgfpathcurveto{\pgfqpoint{3.296812in}{2.146425in}}{\pgfqpoint{3.288912in}{2.143152in}}{\pgfqpoint{3.283088in}{2.137328in}}%
\pgfpathcurveto{\pgfqpoint{3.277265in}{2.131504in}}{\pgfqpoint{3.273992in}{2.123604in}}{\pgfqpoint{3.273992in}{2.115368in}}%
\pgfpathcurveto{\pgfqpoint{3.273992in}{2.107132in}}{\pgfqpoint{3.277265in}{2.099232in}}{\pgfqpoint{3.283088in}{2.093408in}}%
\pgfpathcurveto{\pgfqpoint{3.288912in}{2.087584in}}{\pgfqpoint{3.296812in}{2.084312in}}{\pgfqpoint{3.305049in}{2.084312in}}%
\pgfpathclose%
\pgfusepath{stroke,fill}%
\end{pgfscope}%
\begin{pgfscope}%
\pgfpathrectangle{\pgfqpoint{0.100000in}{0.212622in}}{\pgfqpoint{3.696000in}{3.696000in}}%
\pgfusepath{clip}%
\pgfsetbuttcap%
\pgfsetroundjoin%
\definecolor{currentfill}{rgb}{0.121569,0.466667,0.705882}%
\pgfsetfillcolor{currentfill}%
\pgfsetfillopacity{0.584418}%
\pgfsetlinewidth{1.003750pt}%
\definecolor{currentstroke}{rgb}{0.121569,0.466667,0.705882}%
\pgfsetstrokecolor{currentstroke}%
\pgfsetstrokeopacity{0.584418}%
\pgfsetdash{}{0pt}%
\pgfpathmoveto{\pgfqpoint{3.305049in}{2.084312in}}%
\pgfpathcurveto{\pgfqpoint{3.313285in}{2.084312in}}{\pgfqpoint{3.321185in}{2.087584in}}{\pgfqpoint{3.327009in}{2.093408in}}%
\pgfpathcurveto{\pgfqpoint{3.332833in}{2.099232in}}{\pgfqpoint{3.336105in}{2.107132in}}{\pgfqpoint{3.336105in}{2.115368in}}%
\pgfpathcurveto{\pgfqpoint{3.336105in}{2.123604in}}{\pgfqpoint{3.332833in}{2.131504in}}{\pgfqpoint{3.327009in}{2.137328in}}%
\pgfpathcurveto{\pgfqpoint{3.321185in}{2.143152in}}{\pgfqpoint{3.313285in}{2.146425in}}{\pgfqpoint{3.305049in}{2.146425in}}%
\pgfpathcurveto{\pgfqpoint{3.296812in}{2.146425in}}{\pgfqpoint{3.288912in}{2.143152in}}{\pgfqpoint{3.283088in}{2.137328in}}%
\pgfpathcurveto{\pgfqpoint{3.277265in}{2.131504in}}{\pgfqpoint{3.273992in}{2.123604in}}{\pgfqpoint{3.273992in}{2.115368in}}%
\pgfpathcurveto{\pgfqpoint{3.273992in}{2.107132in}}{\pgfqpoint{3.277265in}{2.099232in}}{\pgfqpoint{3.283088in}{2.093408in}}%
\pgfpathcurveto{\pgfqpoint{3.288912in}{2.087584in}}{\pgfqpoint{3.296812in}{2.084312in}}{\pgfqpoint{3.305049in}{2.084312in}}%
\pgfpathclose%
\pgfusepath{stroke,fill}%
\end{pgfscope}%
\begin{pgfscope}%
\pgfpathrectangle{\pgfqpoint{0.100000in}{0.212622in}}{\pgfqpoint{3.696000in}{3.696000in}}%
\pgfusepath{clip}%
\pgfsetbuttcap%
\pgfsetroundjoin%
\definecolor{currentfill}{rgb}{0.121569,0.466667,0.705882}%
\pgfsetfillcolor{currentfill}%
\pgfsetfillopacity{0.584418}%
\pgfsetlinewidth{1.003750pt}%
\definecolor{currentstroke}{rgb}{0.121569,0.466667,0.705882}%
\pgfsetstrokecolor{currentstroke}%
\pgfsetstrokeopacity{0.584418}%
\pgfsetdash{}{0pt}%
\pgfpathmoveto{\pgfqpoint{3.305049in}{2.084312in}}%
\pgfpathcurveto{\pgfqpoint{3.313285in}{2.084312in}}{\pgfqpoint{3.321185in}{2.087584in}}{\pgfqpoint{3.327009in}{2.093408in}}%
\pgfpathcurveto{\pgfqpoint{3.332833in}{2.099232in}}{\pgfqpoint{3.336105in}{2.107132in}}{\pgfqpoint{3.336105in}{2.115368in}}%
\pgfpathcurveto{\pgfqpoint{3.336105in}{2.123604in}}{\pgfqpoint{3.332833in}{2.131504in}}{\pgfqpoint{3.327009in}{2.137328in}}%
\pgfpathcurveto{\pgfqpoint{3.321185in}{2.143152in}}{\pgfqpoint{3.313285in}{2.146425in}}{\pgfqpoint{3.305049in}{2.146425in}}%
\pgfpathcurveto{\pgfqpoint{3.296812in}{2.146425in}}{\pgfqpoint{3.288912in}{2.143152in}}{\pgfqpoint{3.283088in}{2.137328in}}%
\pgfpathcurveto{\pgfqpoint{3.277265in}{2.131504in}}{\pgfqpoint{3.273992in}{2.123604in}}{\pgfqpoint{3.273992in}{2.115368in}}%
\pgfpathcurveto{\pgfqpoint{3.273992in}{2.107132in}}{\pgfqpoint{3.277265in}{2.099232in}}{\pgfqpoint{3.283088in}{2.093408in}}%
\pgfpathcurveto{\pgfqpoint{3.288912in}{2.087584in}}{\pgfqpoint{3.296812in}{2.084312in}}{\pgfqpoint{3.305049in}{2.084312in}}%
\pgfpathclose%
\pgfusepath{stroke,fill}%
\end{pgfscope}%
\begin{pgfscope}%
\pgfpathrectangle{\pgfqpoint{0.100000in}{0.212622in}}{\pgfqpoint{3.696000in}{3.696000in}}%
\pgfusepath{clip}%
\pgfsetbuttcap%
\pgfsetroundjoin%
\definecolor{currentfill}{rgb}{0.121569,0.466667,0.705882}%
\pgfsetfillcolor{currentfill}%
\pgfsetfillopacity{0.584418}%
\pgfsetlinewidth{1.003750pt}%
\definecolor{currentstroke}{rgb}{0.121569,0.466667,0.705882}%
\pgfsetstrokecolor{currentstroke}%
\pgfsetstrokeopacity{0.584418}%
\pgfsetdash{}{0pt}%
\pgfpathmoveto{\pgfqpoint{3.305049in}{2.084312in}}%
\pgfpathcurveto{\pgfqpoint{3.313285in}{2.084312in}}{\pgfqpoint{3.321185in}{2.087584in}}{\pgfqpoint{3.327009in}{2.093408in}}%
\pgfpathcurveto{\pgfqpoint{3.332833in}{2.099232in}}{\pgfqpoint{3.336105in}{2.107132in}}{\pgfqpoint{3.336105in}{2.115368in}}%
\pgfpathcurveto{\pgfqpoint{3.336105in}{2.123604in}}{\pgfqpoint{3.332833in}{2.131504in}}{\pgfqpoint{3.327009in}{2.137328in}}%
\pgfpathcurveto{\pgfqpoint{3.321185in}{2.143152in}}{\pgfqpoint{3.313285in}{2.146425in}}{\pgfqpoint{3.305049in}{2.146425in}}%
\pgfpathcurveto{\pgfqpoint{3.296812in}{2.146425in}}{\pgfqpoint{3.288912in}{2.143152in}}{\pgfqpoint{3.283089in}{2.137328in}}%
\pgfpathcurveto{\pgfqpoint{3.277265in}{2.131504in}}{\pgfqpoint{3.273992in}{2.123604in}}{\pgfqpoint{3.273992in}{2.115368in}}%
\pgfpathcurveto{\pgfqpoint{3.273992in}{2.107132in}}{\pgfqpoint{3.277265in}{2.099232in}}{\pgfqpoint{3.283089in}{2.093408in}}%
\pgfpathcurveto{\pgfqpoint{3.288912in}{2.087584in}}{\pgfqpoint{3.296812in}{2.084312in}}{\pgfqpoint{3.305049in}{2.084312in}}%
\pgfpathclose%
\pgfusepath{stroke,fill}%
\end{pgfscope}%
\begin{pgfscope}%
\pgfpathrectangle{\pgfqpoint{0.100000in}{0.212622in}}{\pgfqpoint{3.696000in}{3.696000in}}%
\pgfusepath{clip}%
\pgfsetbuttcap%
\pgfsetroundjoin%
\definecolor{currentfill}{rgb}{0.121569,0.466667,0.705882}%
\pgfsetfillcolor{currentfill}%
\pgfsetfillopacity{0.584418}%
\pgfsetlinewidth{1.003750pt}%
\definecolor{currentstroke}{rgb}{0.121569,0.466667,0.705882}%
\pgfsetstrokecolor{currentstroke}%
\pgfsetstrokeopacity{0.584418}%
\pgfsetdash{}{0pt}%
\pgfpathmoveto{\pgfqpoint{3.305049in}{2.084312in}}%
\pgfpathcurveto{\pgfqpoint{3.313285in}{2.084312in}}{\pgfqpoint{3.321185in}{2.087584in}}{\pgfqpoint{3.327009in}{2.093408in}}%
\pgfpathcurveto{\pgfqpoint{3.332833in}{2.099232in}}{\pgfqpoint{3.336105in}{2.107132in}}{\pgfqpoint{3.336105in}{2.115368in}}%
\pgfpathcurveto{\pgfqpoint{3.336105in}{2.123604in}}{\pgfqpoint{3.332833in}{2.131504in}}{\pgfqpoint{3.327009in}{2.137328in}}%
\pgfpathcurveto{\pgfqpoint{3.321185in}{2.143152in}}{\pgfqpoint{3.313285in}{2.146425in}}{\pgfqpoint{3.305049in}{2.146425in}}%
\pgfpathcurveto{\pgfqpoint{3.296813in}{2.146425in}}{\pgfqpoint{3.288912in}{2.143152in}}{\pgfqpoint{3.283089in}{2.137328in}}%
\pgfpathcurveto{\pgfqpoint{3.277265in}{2.131504in}}{\pgfqpoint{3.273992in}{2.123604in}}{\pgfqpoint{3.273992in}{2.115368in}}%
\pgfpathcurveto{\pgfqpoint{3.273992in}{2.107132in}}{\pgfqpoint{3.277265in}{2.099232in}}{\pgfqpoint{3.283089in}{2.093408in}}%
\pgfpathcurveto{\pgfqpoint{3.288912in}{2.087584in}}{\pgfqpoint{3.296813in}{2.084312in}}{\pgfqpoint{3.305049in}{2.084312in}}%
\pgfpathclose%
\pgfusepath{stroke,fill}%
\end{pgfscope}%
\begin{pgfscope}%
\pgfpathrectangle{\pgfqpoint{0.100000in}{0.212622in}}{\pgfqpoint{3.696000in}{3.696000in}}%
\pgfusepath{clip}%
\pgfsetbuttcap%
\pgfsetroundjoin%
\definecolor{currentfill}{rgb}{0.121569,0.466667,0.705882}%
\pgfsetfillcolor{currentfill}%
\pgfsetfillopacity{0.584418}%
\pgfsetlinewidth{1.003750pt}%
\definecolor{currentstroke}{rgb}{0.121569,0.466667,0.705882}%
\pgfsetstrokecolor{currentstroke}%
\pgfsetstrokeopacity{0.584418}%
\pgfsetdash{}{0pt}%
\pgfpathmoveto{\pgfqpoint{3.305049in}{2.084312in}}%
\pgfpathcurveto{\pgfqpoint{3.313285in}{2.084312in}}{\pgfqpoint{3.321185in}{2.087584in}}{\pgfqpoint{3.327009in}{2.093408in}}%
\pgfpathcurveto{\pgfqpoint{3.332833in}{2.099232in}}{\pgfqpoint{3.336105in}{2.107132in}}{\pgfqpoint{3.336105in}{2.115368in}}%
\pgfpathcurveto{\pgfqpoint{3.336105in}{2.123604in}}{\pgfqpoint{3.332833in}{2.131504in}}{\pgfqpoint{3.327009in}{2.137328in}}%
\pgfpathcurveto{\pgfqpoint{3.321185in}{2.143152in}}{\pgfqpoint{3.313285in}{2.146425in}}{\pgfqpoint{3.305049in}{2.146425in}}%
\pgfpathcurveto{\pgfqpoint{3.296813in}{2.146425in}}{\pgfqpoint{3.288912in}{2.143152in}}{\pgfqpoint{3.283089in}{2.137328in}}%
\pgfpathcurveto{\pgfqpoint{3.277265in}{2.131504in}}{\pgfqpoint{3.273992in}{2.123604in}}{\pgfqpoint{3.273992in}{2.115368in}}%
\pgfpathcurveto{\pgfqpoint{3.273992in}{2.107132in}}{\pgfqpoint{3.277265in}{2.099232in}}{\pgfqpoint{3.283089in}{2.093408in}}%
\pgfpathcurveto{\pgfqpoint{3.288912in}{2.087584in}}{\pgfqpoint{3.296813in}{2.084312in}}{\pgfqpoint{3.305049in}{2.084312in}}%
\pgfpathclose%
\pgfusepath{stroke,fill}%
\end{pgfscope}%
\begin{pgfscope}%
\pgfpathrectangle{\pgfqpoint{0.100000in}{0.212622in}}{\pgfqpoint{3.696000in}{3.696000in}}%
\pgfusepath{clip}%
\pgfsetbuttcap%
\pgfsetroundjoin%
\definecolor{currentfill}{rgb}{0.121569,0.466667,0.705882}%
\pgfsetfillcolor{currentfill}%
\pgfsetfillopacity{0.584418}%
\pgfsetlinewidth{1.003750pt}%
\definecolor{currentstroke}{rgb}{0.121569,0.466667,0.705882}%
\pgfsetstrokecolor{currentstroke}%
\pgfsetstrokeopacity{0.584418}%
\pgfsetdash{}{0pt}%
\pgfpathmoveto{\pgfqpoint{3.305049in}{2.084312in}}%
\pgfpathcurveto{\pgfqpoint{3.313285in}{2.084312in}}{\pgfqpoint{3.321185in}{2.087584in}}{\pgfqpoint{3.327009in}{2.093408in}}%
\pgfpathcurveto{\pgfqpoint{3.332833in}{2.099232in}}{\pgfqpoint{3.336105in}{2.107132in}}{\pgfqpoint{3.336105in}{2.115368in}}%
\pgfpathcurveto{\pgfqpoint{3.336105in}{2.123604in}}{\pgfqpoint{3.332833in}{2.131504in}}{\pgfqpoint{3.327009in}{2.137328in}}%
\pgfpathcurveto{\pgfqpoint{3.321185in}{2.143152in}}{\pgfqpoint{3.313285in}{2.146425in}}{\pgfqpoint{3.305049in}{2.146425in}}%
\pgfpathcurveto{\pgfqpoint{3.296813in}{2.146425in}}{\pgfqpoint{3.288913in}{2.143152in}}{\pgfqpoint{3.283089in}{2.137328in}}%
\pgfpathcurveto{\pgfqpoint{3.277265in}{2.131504in}}{\pgfqpoint{3.273992in}{2.123604in}}{\pgfqpoint{3.273992in}{2.115368in}}%
\pgfpathcurveto{\pgfqpoint{3.273992in}{2.107132in}}{\pgfqpoint{3.277265in}{2.099232in}}{\pgfqpoint{3.283089in}{2.093408in}}%
\pgfpathcurveto{\pgfqpoint{3.288913in}{2.087584in}}{\pgfqpoint{3.296813in}{2.084312in}}{\pgfqpoint{3.305049in}{2.084312in}}%
\pgfpathclose%
\pgfusepath{stroke,fill}%
\end{pgfscope}%
\begin{pgfscope}%
\pgfpathrectangle{\pgfqpoint{0.100000in}{0.212622in}}{\pgfqpoint{3.696000in}{3.696000in}}%
\pgfusepath{clip}%
\pgfsetbuttcap%
\pgfsetroundjoin%
\definecolor{currentfill}{rgb}{0.121569,0.466667,0.705882}%
\pgfsetfillcolor{currentfill}%
\pgfsetfillopacity{0.584418}%
\pgfsetlinewidth{1.003750pt}%
\definecolor{currentstroke}{rgb}{0.121569,0.466667,0.705882}%
\pgfsetstrokecolor{currentstroke}%
\pgfsetstrokeopacity{0.584418}%
\pgfsetdash{}{0pt}%
\pgfpathmoveto{\pgfqpoint{3.305049in}{2.084312in}}%
\pgfpathcurveto{\pgfqpoint{3.313285in}{2.084312in}}{\pgfqpoint{3.321185in}{2.087584in}}{\pgfqpoint{3.327009in}{2.093408in}}%
\pgfpathcurveto{\pgfqpoint{3.332833in}{2.099232in}}{\pgfqpoint{3.336106in}{2.107132in}}{\pgfqpoint{3.336106in}{2.115368in}}%
\pgfpathcurveto{\pgfqpoint{3.336106in}{2.123604in}}{\pgfqpoint{3.332833in}{2.131504in}}{\pgfqpoint{3.327009in}{2.137328in}}%
\pgfpathcurveto{\pgfqpoint{3.321185in}{2.143152in}}{\pgfqpoint{3.313285in}{2.146425in}}{\pgfqpoint{3.305049in}{2.146425in}}%
\pgfpathcurveto{\pgfqpoint{3.296813in}{2.146425in}}{\pgfqpoint{3.288913in}{2.143152in}}{\pgfqpoint{3.283089in}{2.137328in}}%
\pgfpathcurveto{\pgfqpoint{3.277265in}{2.131504in}}{\pgfqpoint{3.273993in}{2.123604in}}{\pgfqpoint{3.273993in}{2.115368in}}%
\pgfpathcurveto{\pgfqpoint{3.273993in}{2.107132in}}{\pgfqpoint{3.277265in}{2.099232in}}{\pgfqpoint{3.283089in}{2.093408in}}%
\pgfpathcurveto{\pgfqpoint{3.288913in}{2.087584in}}{\pgfqpoint{3.296813in}{2.084312in}}{\pgfqpoint{3.305049in}{2.084312in}}%
\pgfpathclose%
\pgfusepath{stroke,fill}%
\end{pgfscope}%
\begin{pgfscope}%
\pgfpathrectangle{\pgfqpoint{0.100000in}{0.212622in}}{\pgfqpoint{3.696000in}{3.696000in}}%
\pgfusepath{clip}%
\pgfsetbuttcap%
\pgfsetroundjoin%
\definecolor{currentfill}{rgb}{0.121569,0.466667,0.705882}%
\pgfsetfillcolor{currentfill}%
\pgfsetfillopacity{0.584418}%
\pgfsetlinewidth{1.003750pt}%
\definecolor{currentstroke}{rgb}{0.121569,0.466667,0.705882}%
\pgfsetstrokecolor{currentstroke}%
\pgfsetstrokeopacity{0.584418}%
\pgfsetdash{}{0pt}%
\pgfpathmoveto{\pgfqpoint{3.305049in}{2.084312in}}%
\pgfpathcurveto{\pgfqpoint{3.313286in}{2.084312in}}{\pgfqpoint{3.321186in}{2.087584in}}{\pgfqpoint{3.327010in}{2.093408in}}%
\pgfpathcurveto{\pgfqpoint{3.332833in}{2.099232in}}{\pgfqpoint{3.336106in}{2.107132in}}{\pgfqpoint{3.336106in}{2.115368in}}%
\pgfpathcurveto{\pgfqpoint{3.336106in}{2.123604in}}{\pgfqpoint{3.332833in}{2.131504in}}{\pgfqpoint{3.327010in}{2.137328in}}%
\pgfpathcurveto{\pgfqpoint{3.321186in}{2.143152in}}{\pgfqpoint{3.313286in}{2.146425in}}{\pgfqpoint{3.305049in}{2.146425in}}%
\pgfpathcurveto{\pgfqpoint{3.296813in}{2.146425in}}{\pgfqpoint{3.288913in}{2.143152in}}{\pgfqpoint{3.283089in}{2.137328in}}%
\pgfpathcurveto{\pgfqpoint{3.277265in}{2.131504in}}{\pgfqpoint{3.273993in}{2.123604in}}{\pgfqpoint{3.273993in}{2.115368in}}%
\pgfpathcurveto{\pgfqpoint{3.273993in}{2.107132in}}{\pgfqpoint{3.277265in}{2.099232in}}{\pgfqpoint{3.283089in}{2.093408in}}%
\pgfpathcurveto{\pgfqpoint{3.288913in}{2.087584in}}{\pgfqpoint{3.296813in}{2.084312in}}{\pgfqpoint{3.305049in}{2.084312in}}%
\pgfpathclose%
\pgfusepath{stroke,fill}%
\end{pgfscope}%
\begin{pgfscope}%
\pgfpathrectangle{\pgfqpoint{0.100000in}{0.212622in}}{\pgfqpoint{3.696000in}{3.696000in}}%
\pgfusepath{clip}%
\pgfsetbuttcap%
\pgfsetroundjoin%
\definecolor{currentfill}{rgb}{0.121569,0.466667,0.705882}%
\pgfsetfillcolor{currentfill}%
\pgfsetfillopacity{0.584418}%
\pgfsetlinewidth{1.003750pt}%
\definecolor{currentstroke}{rgb}{0.121569,0.466667,0.705882}%
\pgfsetstrokecolor{currentstroke}%
\pgfsetstrokeopacity{0.584418}%
\pgfsetdash{}{0pt}%
\pgfpathmoveto{\pgfqpoint{3.305050in}{2.084311in}}%
\pgfpathcurveto{\pgfqpoint{3.313286in}{2.084311in}}{\pgfqpoint{3.321186in}{2.087584in}}{\pgfqpoint{3.327010in}{2.093408in}}%
\pgfpathcurveto{\pgfqpoint{3.332834in}{2.099232in}}{\pgfqpoint{3.336106in}{2.107132in}}{\pgfqpoint{3.336106in}{2.115368in}}%
\pgfpathcurveto{\pgfqpoint{3.336106in}{2.123604in}}{\pgfqpoint{3.332834in}{2.131504in}}{\pgfqpoint{3.327010in}{2.137328in}}%
\pgfpathcurveto{\pgfqpoint{3.321186in}{2.143152in}}{\pgfqpoint{3.313286in}{2.146424in}}{\pgfqpoint{3.305050in}{2.146424in}}%
\pgfpathcurveto{\pgfqpoint{3.296813in}{2.146424in}}{\pgfqpoint{3.288913in}{2.143152in}}{\pgfqpoint{3.283089in}{2.137328in}}%
\pgfpathcurveto{\pgfqpoint{3.277265in}{2.131504in}}{\pgfqpoint{3.273993in}{2.123604in}}{\pgfqpoint{3.273993in}{2.115368in}}%
\pgfpathcurveto{\pgfqpoint{3.273993in}{2.107132in}}{\pgfqpoint{3.277265in}{2.099232in}}{\pgfqpoint{3.283089in}{2.093408in}}%
\pgfpathcurveto{\pgfqpoint{3.288913in}{2.087584in}}{\pgfqpoint{3.296813in}{2.084311in}}{\pgfqpoint{3.305050in}{2.084311in}}%
\pgfpathclose%
\pgfusepath{stroke,fill}%
\end{pgfscope}%
\begin{pgfscope}%
\pgfpathrectangle{\pgfqpoint{0.100000in}{0.212622in}}{\pgfqpoint{3.696000in}{3.696000in}}%
\pgfusepath{clip}%
\pgfsetbuttcap%
\pgfsetroundjoin%
\definecolor{currentfill}{rgb}{0.121569,0.466667,0.705882}%
\pgfsetfillcolor{currentfill}%
\pgfsetfillopacity{0.584418}%
\pgfsetlinewidth{1.003750pt}%
\definecolor{currentstroke}{rgb}{0.121569,0.466667,0.705882}%
\pgfsetstrokecolor{currentstroke}%
\pgfsetstrokeopacity{0.584418}%
\pgfsetdash{}{0pt}%
\pgfpathmoveto{\pgfqpoint{3.305050in}{2.084311in}}%
\pgfpathcurveto{\pgfqpoint{3.313287in}{2.084311in}}{\pgfqpoint{3.321187in}{2.087584in}}{\pgfqpoint{3.327011in}{2.093407in}}%
\pgfpathcurveto{\pgfqpoint{3.332835in}{2.099231in}}{\pgfqpoint{3.336107in}{2.107131in}}{\pgfqpoint{3.336107in}{2.115368in}}%
\pgfpathcurveto{\pgfqpoint{3.336107in}{2.123604in}}{\pgfqpoint{3.332835in}{2.131504in}}{\pgfqpoint{3.327011in}{2.137328in}}%
\pgfpathcurveto{\pgfqpoint{3.321187in}{2.143152in}}{\pgfqpoint{3.313287in}{2.146424in}}{\pgfqpoint{3.305050in}{2.146424in}}%
\pgfpathcurveto{\pgfqpoint{3.296814in}{2.146424in}}{\pgfqpoint{3.288914in}{2.143152in}}{\pgfqpoint{3.283090in}{2.137328in}}%
\pgfpathcurveto{\pgfqpoint{3.277266in}{2.131504in}}{\pgfqpoint{3.273994in}{2.123604in}}{\pgfqpoint{3.273994in}{2.115368in}}%
\pgfpathcurveto{\pgfqpoint{3.273994in}{2.107131in}}{\pgfqpoint{3.277266in}{2.099231in}}{\pgfqpoint{3.283090in}{2.093407in}}%
\pgfpathcurveto{\pgfqpoint{3.288914in}{2.087584in}}{\pgfqpoint{3.296814in}{2.084311in}}{\pgfqpoint{3.305050in}{2.084311in}}%
\pgfpathclose%
\pgfusepath{stroke,fill}%
\end{pgfscope}%
\begin{pgfscope}%
\pgfpathrectangle{\pgfqpoint{0.100000in}{0.212622in}}{\pgfqpoint{3.696000in}{3.696000in}}%
\pgfusepath{clip}%
\pgfsetbuttcap%
\pgfsetroundjoin%
\definecolor{currentfill}{rgb}{0.121569,0.466667,0.705882}%
\pgfsetfillcolor{currentfill}%
\pgfsetfillopacity{0.584418}%
\pgfsetlinewidth{1.003750pt}%
\definecolor{currentstroke}{rgb}{0.121569,0.466667,0.705882}%
\pgfsetstrokecolor{currentstroke}%
\pgfsetstrokeopacity{0.584418}%
\pgfsetdash{}{0pt}%
\pgfpathmoveto{\pgfqpoint{3.305052in}{2.084311in}}%
\pgfpathcurveto{\pgfqpoint{3.313288in}{2.084311in}}{\pgfqpoint{3.321188in}{2.087583in}}{\pgfqpoint{3.327012in}{2.093407in}}%
\pgfpathcurveto{\pgfqpoint{3.332836in}{2.099231in}}{\pgfqpoint{3.336108in}{2.107131in}}{\pgfqpoint{3.336108in}{2.115367in}}%
\pgfpathcurveto{\pgfqpoint{3.336108in}{2.123604in}}{\pgfqpoint{3.332836in}{2.131504in}}{\pgfqpoint{3.327012in}{2.137328in}}%
\pgfpathcurveto{\pgfqpoint{3.321188in}{2.143152in}}{\pgfqpoint{3.313288in}{2.146424in}}{\pgfqpoint{3.305052in}{2.146424in}}%
\pgfpathcurveto{\pgfqpoint{3.296815in}{2.146424in}}{\pgfqpoint{3.288915in}{2.143152in}}{\pgfqpoint{3.283091in}{2.137328in}}%
\pgfpathcurveto{\pgfqpoint{3.277267in}{2.131504in}}{\pgfqpoint{3.273995in}{2.123604in}}{\pgfqpoint{3.273995in}{2.115367in}}%
\pgfpathcurveto{\pgfqpoint{3.273995in}{2.107131in}}{\pgfqpoint{3.277267in}{2.099231in}}{\pgfqpoint{3.283091in}{2.093407in}}%
\pgfpathcurveto{\pgfqpoint{3.288915in}{2.087583in}}{\pgfqpoint{3.296815in}{2.084311in}}{\pgfqpoint{3.305052in}{2.084311in}}%
\pgfpathclose%
\pgfusepath{stroke,fill}%
\end{pgfscope}%
\begin{pgfscope}%
\pgfpathrectangle{\pgfqpoint{0.100000in}{0.212622in}}{\pgfqpoint{3.696000in}{3.696000in}}%
\pgfusepath{clip}%
\pgfsetbuttcap%
\pgfsetroundjoin%
\definecolor{currentfill}{rgb}{0.121569,0.466667,0.705882}%
\pgfsetfillcolor{currentfill}%
\pgfsetfillopacity{0.584419}%
\pgfsetlinewidth{1.003750pt}%
\definecolor{currentstroke}{rgb}{0.121569,0.466667,0.705882}%
\pgfsetstrokecolor{currentstroke}%
\pgfsetstrokeopacity{0.584419}%
\pgfsetdash{}{0pt}%
\pgfpathmoveto{\pgfqpoint{3.305054in}{2.084311in}}%
\pgfpathcurveto{\pgfqpoint{3.313290in}{2.084311in}}{\pgfqpoint{3.321190in}{2.087583in}}{\pgfqpoint{3.327014in}{2.093407in}}%
\pgfpathcurveto{\pgfqpoint{3.332838in}{2.099231in}}{\pgfqpoint{3.336110in}{2.107131in}}{\pgfqpoint{3.336110in}{2.115367in}}%
\pgfpathcurveto{\pgfqpoint{3.336110in}{2.123603in}}{\pgfqpoint{3.332838in}{2.131503in}}{\pgfqpoint{3.327014in}{2.137327in}}%
\pgfpathcurveto{\pgfqpoint{3.321190in}{2.143151in}}{\pgfqpoint{3.313290in}{2.146424in}}{\pgfqpoint{3.305054in}{2.146424in}}%
\pgfpathcurveto{\pgfqpoint{3.296817in}{2.146424in}}{\pgfqpoint{3.288917in}{2.143151in}}{\pgfqpoint{3.283093in}{2.137327in}}%
\pgfpathcurveto{\pgfqpoint{3.277270in}{2.131503in}}{\pgfqpoint{3.273997in}{2.123603in}}{\pgfqpoint{3.273997in}{2.115367in}}%
\pgfpathcurveto{\pgfqpoint{3.273997in}{2.107131in}}{\pgfqpoint{3.277270in}{2.099231in}}{\pgfqpoint{3.283093in}{2.093407in}}%
\pgfpathcurveto{\pgfqpoint{3.288917in}{2.087583in}}{\pgfqpoint{3.296817in}{2.084311in}}{\pgfqpoint{3.305054in}{2.084311in}}%
\pgfpathclose%
\pgfusepath{stroke,fill}%
\end{pgfscope}%
\begin{pgfscope}%
\pgfpathrectangle{\pgfqpoint{0.100000in}{0.212622in}}{\pgfqpoint{3.696000in}{3.696000in}}%
\pgfusepath{clip}%
\pgfsetbuttcap%
\pgfsetroundjoin%
\definecolor{currentfill}{rgb}{0.121569,0.466667,0.705882}%
\pgfsetfillcolor{currentfill}%
\pgfsetfillopacity{0.584421}%
\pgfsetlinewidth{1.003750pt}%
\definecolor{currentstroke}{rgb}{0.121569,0.466667,0.705882}%
\pgfsetstrokecolor{currentstroke}%
\pgfsetstrokeopacity{0.584421}%
\pgfsetdash{}{0pt}%
\pgfpathmoveto{\pgfqpoint{3.305057in}{2.084310in}}%
\pgfpathcurveto{\pgfqpoint{3.313293in}{2.084310in}}{\pgfqpoint{3.321193in}{2.087582in}}{\pgfqpoint{3.327017in}{2.093406in}}%
\pgfpathcurveto{\pgfqpoint{3.332841in}{2.099230in}}{\pgfqpoint{3.336114in}{2.107130in}}{\pgfqpoint{3.336114in}{2.115366in}}%
\pgfpathcurveto{\pgfqpoint{3.336114in}{2.123602in}}{\pgfqpoint{3.332841in}{2.131503in}}{\pgfqpoint{3.327017in}{2.137326in}}%
\pgfpathcurveto{\pgfqpoint{3.321193in}{2.143150in}}{\pgfqpoint{3.313293in}{2.146423in}}{\pgfqpoint{3.305057in}{2.146423in}}%
\pgfpathcurveto{\pgfqpoint{3.296821in}{2.146423in}}{\pgfqpoint{3.288921in}{2.143150in}}{\pgfqpoint{3.283097in}{2.137326in}}%
\pgfpathcurveto{\pgfqpoint{3.277273in}{2.131503in}}{\pgfqpoint{3.274001in}{2.123602in}}{\pgfqpoint{3.274001in}{2.115366in}}%
\pgfpathcurveto{\pgfqpoint{3.274001in}{2.107130in}}{\pgfqpoint{3.277273in}{2.099230in}}{\pgfqpoint{3.283097in}{2.093406in}}%
\pgfpathcurveto{\pgfqpoint{3.288921in}{2.087582in}}{\pgfqpoint{3.296821in}{2.084310in}}{\pgfqpoint{3.305057in}{2.084310in}}%
\pgfpathclose%
\pgfusepath{stroke,fill}%
\end{pgfscope}%
\begin{pgfscope}%
\pgfpathrectangle{\pgfqpoint{0.100000in}{0.212622in}}{\pgfqpoint{3.696000in}{3.696000in}}%
\pgfusepath{clip}%
\pgfsetbuttcap%
\pgfsetroundjoin%
\definecolor{currentfill}{rgb}{0.121569,0.466667,0.705882}%
\pgfsetfillcolor{currentfill}%
\pgfsetfillopacity{0.584423}%
\pgfsetlinewidth{1.003750pt}%
\definecolor{currentstroke}{rgb}{0.121569,0.466667,0.705882}%
\pgfsetstrokecolor{currentstroke}%
\pgfsetstrokeopacity{0.584423}%
\pgfsetdash{}{0pt}%
\pgfpathmoveto{\pgfqpoint{3.305063in}{2.084309in}}%
\pgfpathcurveto{\pgfqpoint{3.313299in}{2.084309in}}{\pgfqpoint{3.321199in}{2.087581in}}{\pgfqpoint{3.327023in}{2.093405in}}%
\pgfpathcurveto{\pgfqpoint{3.332847in}{2.099229in}}{\pgfqpoint{3.336120in}{2.107129in}}{\pgfqpoint{3.336120in}{2.115365in}}%
\pgfpathcurveto{\pgfqpoint{3.336120in}{2.123601in}}{\pgfqpoint{3.332847in}{2.131501in}}{\pgfqpoint{3.327023in}{2.137325in}}%
\pgfpathcurveto{\pgfqpoint{3.321199in}{2.143149in}}{\pgfqpoint{3.313299in}{2.146422in}}{\pgfqpoint{3.305063in}{2.146422in}}%
\pgfpathcurveto{\pgfqpoint{3.296827in}{2.146422in}}{\pgfqpoint{3.288927in}{2.143149in}}{\pgfqpoint{3.283103in}{2.137325in}}%
\pgfpathcurveto{\pgfqpoint{3.277279in}{2.131501in}}{\pgfqpoint{3.274007in}{2.123601in}}{\pgfqpoint{3.274007in}{2.115365in}}%
\pgfpathcurveto{\pgfqpoint{3.274007in}{2.107129in}}{\pgfqpoint{3.277279in}{2.099229in}}{\pgfqpoint{3.283103in}{2.093405in}}%
\pgfpathcurveto{\pgfqpoint{3.288927in}{2.087581in}}{\pgfqpoint{3.296827in}{2.084309in}}{\pgfqpoint{3.305063in}{2.084309in}}%
\pgfpathclose%
\pgfusepath{stroke,fill}%
\end{pgfscope}%
\begin{pgfscope}%
\pgfpathrectangle{\pgfqpoint{0.100000in}{0.212622in}}{\pgfqpoint{3.696000in}{3.696000in}}%
\pgfusepath{clip}%
\pgfsetbuttcap%
\pgfsetroundjoin%
\definecolor{currentfill}{rgb}{0.121569,0.466667,0.705882}%
\pgfsetfillcolor{currentfill}%
\pgfsetfillopacity{0.584428}%
\pgfsetlinewidth{1.003750pt}%
\definecolor{currentstroke}{rgb}{0.121569,0.466667,0.705882}%
\pgfsetstrokecolor{currentstroke}%
\pgfsetstrokeopacity{0.584428}%
\pgfsetdash{}{0pt}%
\pgfpathmoveto{\pgfqpoint{3.305073in}{2.084306in}}%
\pgfpathcurveto{\pgfqpoint{3.313309in}{2.084306in}}{\pgfqpoint{3.321209in}{2.087578in}}{\pgfqpoint{3.327033in}{2.093402in}}%
\pgfpathcurveto{\pgfqpoint{3.332857in}{2.099226in}}{\pgfqpoint{3.336129in}{2.107126in}}{\pgfqpoint{3.336129in}{2.115362in}}%
\pgfpathcurveto{\pgfqpoint{3.336129in}{2.123598in}}{\pgfqpoint{3.332857in}{2.131498in}}{\pgfqpoint{3.327033in}{2.137322in}}%
\pgfpathcurveto{\pgfqpoint{3.321209in}{2.143146in}}{\pgfqpoint{3.313309in}{2.146419in}}{\pgfqpoint{3.305073in}{2.146419in}}%
\pgfpathcurveto{\pgfqpoint{3.296836in}{2.146419in}}{\pgfqpoint{3.288936in}{2.143146in}}{\pgfqpoint{3.283112in}{2.137322in}}%
\pgfpathcurveto{\pgfqpoint{3.277288in}{2.131498in}}{\pgfqpoint{3.274016in}{2.123598in}}{\pgfqpoint{3.274016in}{2.115362in}}%
\pgfpathcurveto{\pgfqpoint{3.274016in}{2.107126in}}{\pgfqpoint{3.277288in}{2.099226in}}{\pgfqpoint{3.283112in}{2.093402in}}%
\pgfpathcurveto{\pgfqpoint{3.288936in}{2.087578in}}{\pgfqpoint{3.296836in}{2.084306in}}{\pgfqpoint{3.305073in}{2.084306in}}%
\pgfpathclose%
\pgfusepath{stroke,fill}%
\end{pgfscope}%
\begin{pgfscope}%
\pgfpathrectangle{\pgfqpoint{0.100000in}{0.212622in}}{\pgfqpoint{3.696000in}{3.696000in}}%
\pgfusepath{clip}%
\pgfsetbuttcap%
\pgfsetroundjoin%
\definecolor{currentfill}{rgb}{0.121569,0.466667,0.705882}%
\pgfsetfillcolor{currentfill}%
\pgfsetfillopacity{0.584438}%
\pgfsetlinewidth{1.003750pt}%
\definecolor{currentstroke}{rgb}{0.121569,0.466667,0.705882}%
\pgfsetstrokecolor{currentstroke}%
\pgfsetstrokeopacity{0.584438}%
\pgfsetdash{}{0pt}%
\pgfpathmoveto{\pgfqpoint{3.305087in}{2.084298in}}%
\pgfpathcurveto{\pgfqpoint{3.313323in}{2.084298in}}{\pgfqpoint{3.321223in}{2.087570in}}{\pgfqpoint{3.327047in}{2.093394in}}%
\pgfpathcurveto{\pgfqpoint{3.332871in}{2.099218in}}{\pgfqpoint{3.336144in}{2.107118in}}{\pgfqpoint{3.336144in}{2.115354in}}%
\pgfpathcurveto{\pgfqpoint{3.336144in}{2.123591in}}{\pgfqpoint{3.332871in}{2.131491in}}{\pgfqpoint{3.327047in}{2.137315in}}%
\pgfpathcurveto{\pgfqpoint{3.321223in}{2.143138in}}{\pgfqpoint{3.313323in}{2.146411in}}{\pgfqpoint{3.305087in}{2.146411in}}%
\pgfpathcurveto{\pgfqpoint{3.296851in}{2.146411in}}{\pgfqpoint{3.288951in}{2.143138in}}{\pgfqpoint{3.283127in}{2.137315in}}%
\pgfpathcurveto{\pgfqpoint{3.277303in}{2.131491in}}{\pgfqpoint{3.274031in}{2.123591in}}{\pgfqpoint{3.274031in}{2.115354in}}%
\pgfpathcurveto{\pgfqpoint{3.274031in}{2.107118in}}{\pgfqpoint{3.277303in}{2.099218in}}{\pgfqpoint{3.283127in}{2.093394in}}%
\pgfpathcurveto{\pgfqpoint{3.288951in}{2.087570in}}{\pgfqpoint{3.296851in}{2.084298in}}{\pgfqpoint{3.305087in}{2.084298in}}%
\pgfpathclose%
\pgfusepath{stroke,fill}%
\end{pgfscope}%
\begin{pgfscope}%
\pgfpathrectangle{\pgfqpoint{0.100000in}{0.212622in}}{\pgfqpoint{3.696000in}{3.696000in}}%
\pgfusepath{clip}%
\pgfsetbuttcap%
\pgfsetroundjoin%
\definecolor{currentfill}{rgb}{0.121569,0.466667,0.705882}%
\pgfsetfillcolor{currentfill}%
\pgfsetfillopacity{0.584457}%
\pgfsetlinewidth{1.003750pt}%
\definecolor{currentstroke}{rgb}{0.121569,0.466667,0.705882}%
\pgfsetstrokecolor{currentstroke}%
\pgfsetstrokeopacity{0.584457}%
\pgfsetdash{}{0pt}%
\pgfpathmoveto{\pgfqpoint{3.305108in}{2.084281in}}%
\pgfpathcurveto{\pgfqpoint{3.313345in}{2.084281in}}{\pgfqpoint{3.321245in}{2.087553in}}{\pgfqpoint{3.327069in}{2.093377in}}%
\pgfpathcurveto{\pgfqpoint{3.332893in}{2.099201in}}{\pgfqpoint{3.336165in}{2.107101in}}{\pgfqpoint{3.336165in}{2.115337in}}%
\pgfpathcurveto{\pgfqpoint{3.336165in}{2.123573in}}{\pgfqpoint{3.332893in}{2.131473in}}{\pgfqpoint{3.327069in}{2.137297in}}%
\pgfpathcurveto{\pgfqpoint{3.321245in}{2.143121in}}{\pgfqpoint{3.313345in}{2.146394in}}{\pgfqpoint{3.305108in}{2.146394in}}%
\pgfpathcurveto{\pgfqpoint{3.296872in}{2.146394in}}{\pgfqpoint{3.288972in}{2.143121in}}{\pgfqpoint{3.283148in}{2.137297in}}%
\pgfpathcurveto{\pgfqpoint{3.277324in}{2.131473in}}{\pgfqpoint{3.274052in}{2.123573in}}{\pgfqpoint{3.274052in}{2.115337in}}%
\pgfpathcurveto{\pgfqpoint{3.274052in}{2.107101in}}{\pgfqpoint{3.277324in}{2.099201in}}{\pgfqpoint{3.283148in}{2.093377in}}%
\pgfpathcurveto{\pgfqpoint{3.288972in}{2.087553in}}{\pgfqpoint{3.296872in}{2.084281in}}{\pgfqpoint{3.305108in}{2.084281in}}%
\pgfpathclose%
\pgfusepath{stroke,fill}%
\end{pgfscope}%
\begin{pgfscope}%
\pgfpathrectangle{\pgfqpoint{0.100000in}{0.212622in}}{\pgfqpoint{3.696000in}{3.696000in}}%
\pgfusepath{clip}%
\pgfsetbuttcap%
\pgfsetroundjoin%
\definecolor{currentfill}{rgb}{0.121569,0.466667,0.705882}%
\pgfsetfillcolor{currentfill}%
\pgfsetfillopacity{0.584493}%
\pgfsetlinewidth{1.003750pt}%
\definecolor{currentstroke}{rgb}{0.121569,0.466667,0.705882}%
\pgfsetstrokecolor{currentstroke}%
\pgfsetstrokeopacity{0.584493}%
\pgfsetdash{}{0pt}%
\pgfpathmoveto{\pgfqpoint{3.305138in}{2.084246in}}%
\pgfpathcurveto{\pgfqpoint{3.313375in}{2.084246in}}{\pgfqpoint{3.321275in}{2.087519in}}{\pgfqpoint{3.327099in}{2.093343in}}%
\pgfpathcurveto{\pgfqpoint{3.332923in}{2.099167in}}{\pgfqpoint{3.336195in}{2.107067in}}{\pgfqpoint{3.336195in}{2.115303in}}%
\pgfpathcurveto{\pgfqpoint{3.336195in}{2.123539in}}{\pgfqpoint{3.332923in}{2.131439in}}{\pgfqpoint{3.327099in}{2.137263in}}%
\pgfpathcurveto{\pgfqpoint{3.321275in}{2.143087in}}{\pgfqpoint{3.313375in}{2.146359in}}{\pgfqpoint{3.305138in}{2.146359in}}%
\pgfpathcurveto{\pgfqpoint{3.296902in}{2.146359in}}{\pgfqpoint{3.289002in}{2.143087in}}{\pgfqpoint{3.283178in}{2.137263in}}%
\pgfpathcurveto{\pgfqpoint{3.277354in}{2.131439in}}{\pgfqpoint{3.274082in}{2.123539in}}{\pgfqpoint{3.274082in}{2.115303in}}%
\pgfpathcurveto{\pgfqpoint{3.274082in}{2.107067in}}{\pgfqpoint{3.277354in}{2.099167in}}{\pgfqpoint{3.283178in}{2.093343in}}%
\pgfpathcurveto{\pgfqpoint{3.289002in}{2.087519in}}{\pgfqpoint{3.296902in}{2.084246in}}{\pgfqpoint{3.305138in}{2.084246in}}%
\pgfpathclose%
\pgfusepath{stroke,fill}%
\end{pgfscope}%
\begin{pgfscope}%
\pgfpathrectangle{\pgfqpoint{0.100000in}{0.212622in}}{\pgfqpoint{3.696000in}{3.696000in}}%
\pgfusepath{clip}%
\pgfsetbuttcap%
\pgfsetroundjoin%
\definecolor{currentfill}{rgb}{0.121569,0.466667,0.705882}%
\pgfsetfillcolor{currentfill}%
\pgfsetfillopacity{0.584559}%
\pgfsetlinewidth{1.003750pt}%
\definecolor{currentstroke}{rgb}{0.121569,0.466667,0.705882}%
\pgfsetstrokecolor{currentstroke}%
\pgfsetstrokeopacity{0.584559}%
\pgfsetdash{}{0pt}%
\pgfpathmoveto{\pgfqpoint{3.305177in}{2.084175in}}%
\pgfpathcurveto{\pgfqpoint{3.313413in}{2.084175in}}{\pgfqpoint{3.321313in}{2.087447in}}{\pgfqpoint{3.327137in}{2.093271in}}%
\pgfpathcurveto{\pgfqpoint{3.332961in}{2.099095in}}{\pgfqpoint{3.336234in}{2.106995in}}{\pgfqpoint{3.336234in}{2.115231in}}%
\pgfpathcurveto{\pgfqpoint{3.336234in}{2.123468in}}{\pgfqpoint{3.332961in}{2.131368in}}{\pgfqpoint{3.327137in}{2.137192in}}%
\pgfpathcurveto{\pgfqpoint{3.321313in}{2.143015in}}{\pgfqpoint{3.313413in}{2.146288in}}{\pgfqpoint{3.305177in}{2.146288in}}%
\pgfpathcurveto{\pgfqpoint{3.296941in}{2.146288in}}{\pgfqpoint{3.289041in}{2.143015in}}{\pgfqpoint{3.283217in}{2.137192in}}%
\pgfpathcurveto{\pgfqpoint{3.277393in}{2.131368in}}{\pgfqpoint{3.274121in}{2.123468in}}{\pgfqpoint{3.274121in}{2.115231in}}%
\pgfpathcurveto{\pgfqpoint{3.274121in}{2.106995in}}{\pgfqpoint{3.277393in}{2.099095in}}{\pgfqpoint{3.283217in}{2.093271in}}%
\pgfpathcurveto{\pgfqpoint{3.289041in}{2.087447in}}{\pgfqpoint{3.296941in}{2.084175in}}{\pgfqpoint{3.305177in}{2.084175in}}%
\pgfpathclose%
\pgfusepath{stroke,fill}%
\end{pgfscope}%
\begin{pgfscope}%
\pgfpathrectangle{\pgfqpoint{0.100000in}{0.212622in}}{\pgfqpoint{3.696000in}{3.696000in}}%
\pgfusepath{clip}%
\pgfsetbuttcap%
\pgfsetroundjoin%
\definecolor{currentfill}{rgb}{0.121569,0.466667,0.705882}%
\pgfsetfillcolor{currentfill}%
\pgfsetfillopacity{0.584681}%
\pgfsetlinewidth{1.003750pt}%
\definecolor{currentstroke}{rgb}{0.121569,0.466667,0.705882}%
\pgfsetstrokecolor{currentstroke}%
\pgfsetstrokeopacity{0.584681}%
\pgfsetdash{}{0pt}%
\pgfpathmoveto{\pgfqpoint{3.305212in}{2.084025in}}%
\pgfpathcurveto{\pgfqpoint{3.313448in}{2.084025in}}{\pgfqpoint{3.321348in}{2.087297in}}{\pgfqpoint{3.327172in}{2.093121in}}%
\pgfpathcurveto{\pgfqpoint{3.332996in}{2.098945in}}{\pgfqpoint{3.336268in}{2.106845in}}{\pgfqpoint{3.336268in}{2.115081in}}%
\pgfpathcurveto{\pgfqpoint{3.336268in}{2.123317in}}{\pgfqpoint{3.332996in}{2.131217in}}{\pgfqpoint{3.327172in}{2.137041in}}%
\pgfpathcurveto{\pgfqpoint{3.321348in}{2.142865in}}{\pgfqpoint{3.313448in}{2.146138in}}{\pgfqpoint{3.305212in}{2.146138in}}%
\pgfpathcurveto{\pgfqpoint{3.296976in}{2.146138in}}{\pgfqpoint{3.289076in}{2.142865in}}{\pgfqpoint{3.283252in}{2.137041in}}%
\pgfpathcurveto{\pgfqpoint{3.277428in}{2.131217in}}{\pgfqpoint{3.274155in}{2.123317in}}{\pgfqpoint{3.274155in}{2.115081in}}%
\pgfpathcurveto{\pgfqpoint{3.274155in}{2.106845in}}{\pgfqpoint{3.277428in}{2.098945in}}{\pgfqpoint{3.283252in}{2.093121in}}%
\pgfpathcurveto{\pgfqpoint{3.289076in}{2.087297in}}{\pgfqpoint{3.296976in}{2.084025in}}{\pgfqpoint{3.305212in}{2.084025in}}%
\pgfpathclose%
\pgfusepath{stroke,fill}%
\end{pgfscope}%
\begin{pgfscope}%
\pgfpathrectangle{\pgfqpoint{0.100000in}{0.212622in}}{\pgfqpoint{3.696000in}{3.696000in}}%
\pgfusepath{clip}%
\pgfsetbuttcap%
\pgfsetroundjoin%
\definecolor{currentfill}{rgb}{0.121569,0.466667,0.705882}%
\pgfsetfillcolor{currentfill}%
\pgfsetfillopacity{0.584907}%
\pgfsetlinewidth{1.003750pt}%
\definecolor{currentstroke}{rgb}{0.121569,0.466667,0.705882}%
\pgfsetstrokecolor{currentstroke}%
\pgfsetstrokeopacity{0.584907}%
\pgfsetdash{}{0pt}%
\pgfpathmoveto{\pgfqpoint{3.305220in}{2.083734in}}%
\pgfpathcurveto{\pgfqpoint{3.313456in}{2.083734in}}{\pgfqpoint{3.321356in}{2.087007in}}{\pgfqpoint{3.327180in}{2.092830in}}%
\pgfpathcurveto{\pgfqpoint{3.333004in}{2.098654in}}{\pgfqpoint{3.336277in}{2.106554in}}{\pgfqpoint{3.336277in}{2.114791in}}%
\pgfpathcurveto{\pgfqpoint{3.336277in}{2.123027in}}{\pgfqpoint{3.333004in}{2.130927in}}{\pgfqpoint{3.327180in}{2.136751in}}%
\pgfpathcurveto{\pgfqpoint{3.321356in}{2.142575in}}{\pgfqpoint{3.313456in}{2.145847in}}{\pgfqpoint{3.305220in}{2.145847in}}%
\pgfpathcurveto{\pgfqpoint{3.296984in}{2.145847in}}{\pgfqpoint{3.289084in}{2.142575in}}{\pgfqpoint{3.283260in}{2.136751in}}%
\pgfpathcurveto{\pgfqpoint{3.277436in}{2.130927in}}{\pgfqpoint{3.274164in}{2.123027in}}{\pgfqpoint{3.274164in}{2.114791in}}%
\pgfpathcurveto{\pgfqpoint{3.274164in}{2.106554in}}{\pgfqpoint{3.277436in}{2.098654in}}{\pgfqpoint{3.283260in}{2.092830in}}%
\pgfpathcurveto{\pgfqpoint{3.289084in}{2.087007in}}{\pgfqpoint{3.296984in}{2.083734in}}{\pgfqpoint{3.305220in}{2.083734in}}%
\pgfpathclose%
\pgfusepath{stroke,fill}%
\end{pgfscope}%
\begin{pgfscope}%
\pgfpathrectangle{\pgfqpoint{0.100000in}{0.212622in}}{\pgfqpoint{3.696000in}{3.696000in}}%
\pgfusepath{clip}%
\pgfsetbuttcap%
\pgfsetroundjoin%
\definecolor{currentfill}{rgb}{0.121569,0.466667,0.705882}%
\pgfsetfillcolor{currentfill}%
\pgfsetfillopacity{0.585324}%
\pgfsetlinewidth{1.003750pt}%
\definecolor{currentstroke}{rgb}{0.121569,0.466667,0.705882}%
\pgfsetstrokecolor{currentstroke}%
\pgfsetstrokeopacity{0.585324}%
\pgfsetdash{}{0pt}%
\pgfpathmoveto{\pgfqpoint{3.305132in}{2.083186in}}%
\pgfpathcurveto{\pgfqpoint{3.313368in}{2.083186in}}{\pgfqpoint{3.321268in}{2.086458in}}{\pgfqpoint{3.327092in}{2.092282in}}%
\pgfpathcurveto{\pgfqpoint{3.332916in}{2.098106in}}{\pgfqpoint{3.336188in}{2.106006in}}{\pgfqpoint{3.336188in}{2.114242in}}%
\pgfpathcurveto{\pgfqpoint{3.336188in}{2.122479in}}{\pgfqpoint{3.332916in}{2.130379in}}{\pgfqpoint{3.327092in}{2.136203in}}%
\pgfpathcurveto{\pgfqpoint{3.321268in}{2.142027in}}{\pgfqpoint{3.313368in}{2.145299in}}{\pgfqpoint{3.305132in}{2.145299in}}%
\pgfpathcurveto{\pgfqpoint{3.296895in}{2.145299in}}{\pgfqpoint{3.288995in}{2.142027in}}{\pgfqpoint{3.283171in}{2.136203in}}%
\pgfpathcurveto{\pgfqpoint{3.277347in}{2.130379in}}{\pgfqpoint{3.274075in}{2.122479in}}{\pgfqpoint{3.274075in}{2.114242in}}%
\pgfpathcurveto{\pgfqpoint{3.274075in}{2.106006in}}{\pgfqpoint{3.277347in}{2.098106in}}{\pgfqpoint{3.283171in}{2.092282in}}%
\pgfpathcurveto{\pgfqpoint{3.288995in}{2.086458in}}{\pgfqpoint{3.296895in}{2.083186in}}{\pgfqpoint{3.305132in}{2.083186in}}%
\pgfpathclose%
\pgfusepath{stroke,fill}%
\end{pgfscope}%
\begin{pgfscope}%
\pgfpathrectangle{\pgfqpoint{0.100000in}{0.212622in}}{\pgfqpoint{3.696000in}{3.696000in}}%
\pgfusepath{clip}%
\pgfsetbuttcap%
\pgfsetroundjoin%
\definecolor{currentfill}{rgb}{0.121569,0.466667,0.705882}%
\pgfsetfillcolor{currentfill}%
\pgfsetfillopacity{0.586096}%
\pgfsetlinewidth{1.003750pt}%
\definecolor{currentstroke}{rgb}{0.121569,0.466667,0.705882}%
\pgfsetstrokecolor{currentstroke}%
\pgfsetstrokeopacity{0.586096}%
\pgfsetdash{}{0pt}%
\pgfpathmoveto{\pgfqpoint{3.304775in}{2.082214in}}%
\pgfpathcurveto{\pgfqpoint{3.313012in}{2.082214in}}{\pgfqpoint{3.320912in}{2.085486in}}{\pgfqpoint{3.326736in}{2.091310in}}%
\pgfpathcurveto{\pgfqpoint{3.332560in}{2.097134in}}{\pgfqpoint{3.335832in}{2.105034in}}{\pgfqpoint{3.335832in}{2.113270in}}%
\pgfpathcurveto{\pgfqpoint{3.335832in}{2.121506in}}{\pgfqpoint{3.332560in}{2.129407in}}{\pgfqpoint{3.326736in}{2.135230in}}%
\pgfpathcurveto{\pgfqpoint{3.320912in}{2.141054in}}{\pgfqpoint{3.313012in}{2.144327in}}{\pgfqpoint{3.304775in}{2.144327in}}%
\pgfpathcurveto{\pgfqpoint{3.296539in}{2.144327in}}{\pgfqpoint{3.288639in}{2.141054in}}{\pgfqpoint{3.282815in}{2.135230in}}%
\pgfpathcurveto{\pgfqpoint{3.276991in}{2.129407in}}{\pgfqpoint{3.273719in}{2.121506in}}{\pgfqpoint{3.273719in}{2.113270in}}%
\pgfpathcurveto{\pgfqpoint{3.273719in}{2.105034in}}{\pgfqpoint{3.276991in}{2.097134in}}{\pgfqpoint{3.282815in}{2.091310in}}%
\pgfpathcurveto{\pgfqpoint{3.288639in}{2.085486in}}{\pgfqpoint{3.296539in}{2.082214in}}{\pgfqpoint{3.304775in}{2.082214in}}%
\pgfpathclose%
\pgfusepath{stroke,fill}%
\end{pgfscope}%
\begin{pgfscope}%
\pgfpathrectangle{\pgfqpoint{0.100000in}{0.212622in}}{\pgfqpoint{3.696000in}{3.696000in}}%
\pgfusepath{clip}%
\pgfsetbuttcap%
\pgfsetroundjoin%
\definecolor{currentfill}{rgb}{0.121569,0.466667,0.705882}%
\pgfsetfillcolor{currentfill}%
\pgfsetfillopacity{0.586758}%
\pgfsetlinewidth{1.003750pt}%
\definecolor{currentstroke}{rgb}{0.121569,0.466667,0.705882}%
\pgfsetstrokecolor{currentstroke}%
\pgfsetstrokeopacity{0.586758}%
\pgfsetdash{}{0pt}%
\pgfpathmoveto{\pgfqpoint{3.304307in}{2.081381in}}%
\pgfpathcurveto{\pgfqpoint{3.312543in}{2.081381in}}{\pgfqpoint{3.320443in}{2.084653in}}{\pgfqpoint{3.326267in}{2.090477in}}%
\pgfpathcurveto{\pgfqpoint{3.332091in}{2.096301in}}{\pgfqpoint{3.335363in}{2.104201in}}{\pgfqpoint{3.335363in}{2.112438in}}%
\pgfpathcurveto{\pgfqpoint{3.335363in}{2.120674in}}{\pgfqpoint{3.332091in}{2.128574in}}{\pgfqpoint{3.326267in}{2.134398in}}%
\pgfpathcurveto{\pgfqpoint{3.320443in}{2.140222in}}{\pgfqpoint{3.312543in}{2.143494in}}{\pgfqpoint{3.304307in}{2.143494in}}%
\pgfpathcurveto{\pgfqpoint{3.296070in}{2.143494in}}{\pgfqpoint{3.288170in}{2.140222in}}{\pgfqpoint{3.282346in}{2.134398in}}%
\pgfpathcurveto{\pgfqpoint{3.276522in}{2.128574in}}{\pgfqpoint{3.273250in}{2.120674in}}{\pgfqpoint{3.273250in}{2.112438in}}%
\pgfpathcurveto{\pgfqpoint{3.273250in}{2.104201in}}{\pgfqpoint{3.276522in}{2.096301in}}{\pgfqpoint{3.282346in}{2.090477in}}%
\pgfpathcurveto{\pgfqpoint{3.288170in}{2.084653in}}{\pgfqpoint{3.296070in}{2.081381in}}{\pgfqpoint{3.304307in}{2.081381in}}%
\pgfpathclose%
\pgfusepath{stroke,fill}%
\end{pgfscope}%
\begin{pgfscope}%
\pgfpathrectangle{\pgfqpoint{0.100000in}{0.212622in}}{\pgfqpoint{3.696000in}{3.696000in}}%
\pgfusepath{clip}%
\pgfsetbuttcap%
\pgfsetroundjoin%
\definecolor{currentfill}{rgb}{0.121569,0.466667,0.705882}%
\pgfsetfillcolor{currentfill}%
\pgfsetfillopacity{0.587229}%
\pgfsetlinewidth{1.003750pt}%
\definecolor{currentstroke}{rgb}{0.121569,0.466667,0.705882}%
\pgfsetstrokecolor{currentstroke}%
\pgfsetstrokeopacity{0.587229}%
\pgfsetdash{}{0pt}%
\pgfpathmoveto{\pgfqpoint{3.303855in}{2.080745in}}%
\pgfpathcurveto{\pgfqpoint{3.312092in}{2.080745in}}{\pgfqpoint{3.319992in}{2.084018in}}{\pgfqpoint{3.325816in}{2.089842in}}%
\pgfpathcurveto{\pgfqpoint{3.331640in}{2.095665in}}{\pgfqpoint{3.334912in}{2.103565in}}{\pgfqpoint{3.334912in}{2.111802in}}%
\pgfpathcurveto{\pgfqpoint{3.334912in}{2.120038in}}{\pgfqpoint{3.331640in}{2.127938in}}{\pgfqpoint{3.325816in}{2.133762in}}%
\pgfpathcurveto{\pgfqpoint{3.319992in}{2.139586in}}{\pgfqpoint{3.312092in}{2.142858in}}{\pgfqpoint{3.303855in}{2.142858in}}%
\pgfpathcurveto{\pgfqpoint{3.295619in}{2.142858in}}{\pgfqpoint{3.287719in}{2.139586in}}{\pgfqpoint{3.281895in}{2.133762in}}%
\pgfpathcurveto{\pgfqpoint{3.276071in}{2.127938in}}{\pgfqpoint{3.272799in}{2.120038in}}{\pgfqpoint{3.272799in}{2.111802in}}%
\pgfpathcurveto{\pgfqpoint{3.272799in}{2.103565in}}{\pgfqpoint{3.276071in}{2.095665in}}{\pgfqpoint{3.281895in}{2.089842in}}%
\pgfpathcurveto{\pgfqpoint{3.287719in}{2.084018in}}{\pgfqpoint{3.295619in}{2.080745in}}{\pgfqpoint{3.303855in}{2.080745in}}%
\pgfpathclose%
\pgfusepath{stroke,fill}%
\end{pgfscope}%
\begin{pgfscope}%
\pgfpathrectangle{\pgfqpoint{0.100000in}{0.212622in}}{\pgfqpoint{3.696000in}{3.696000in}}%
\pgfusepath{clip}%
\pgfsetbuttcap%
\pgfsetroundjoin%
\definecolor{currentfill}{rgb}{0.121569,0.466667,0.705882}%
\pgfsetfillcolor{currentfill}%
\pgfsetfillopacity{0.587481}%
\pgfsetlinewidth{1.003750pt}%
\definecolor{currentstroke}{rgb}{0.121569,0.466667,0.705882}%
\pgfsetstrokecolor{currentstroke}%
\pgfsetstrokeopacity{0.587481}%
\pgfsetdash{}{0pt}%
\pgfpathmoveto{\pgfqpoint{3.303548in}{2.080376in}}%
\pgfpathcurveto{\pgfqpoint{3.311785in}{2.080376in}}{\pgfqpoint{3.319685in}{2.083648in}}{\pgfqpoint{3.325508in}{2.089472in}}%
\pgfpathcurveto{\pgfqpoint{3.331332in}{2.095296in}}{\pgfqpoint{3.334605in}{2.103196in}}{\pgfqpoint{3.334605in}{2.111432in}}%
\pgfpathcurveto{\pgfqpoint{3.334605in}{2.119668in}}{\pgfqpoint{3.331332in}{2.127569in}}{\pgfqpoint{3.325508in}{2.133392in}}%
\pgfpathcurveto{\pgfqpoint{3.319685in}{2.139216in}}{\pgfqpoint{3.311785in}{2.142489in}}{\pgfqpoint{3.303548in}{2.142489in}}%
\pgfpathcurveto{\pgfqpoint{3.295312in}{2.142489in}}{\pgfqpoint{3.287412in}{2.139216in}}{\pgfqpoint{3.281588in}{2.133392in}}%
\pgfpathcurveto{\pgfqpoint{3.275764in}{2.127569in}}{\pgfqpoint{3.272492in}{2.119668in}}{\pgfqpoint{3.272492in}{2.111432in}}%
\pgfpathcurveto{\pgfqpoint{3.272492in}{2.103196in}}{\pgfqpoint{3.275764in}{2.095296in}}{\pgfqpoint{3.281588in}{2.089472in}}%
\pgfpathcurveto{\pgfqpoint{3.287412in}{2.083648in}}{\pgfqpoint{3.295312in}{2.080376in}}{\pgfqpoint{3.303548in}{2.080376in}}%
\pgfpathclose%
\pgfusepath{stroke,fill}%
\end{pgfscope}%
\begin{pgfscope}%
\pgfpathrectangle{\pgfqpoint{0.100000in}{0.212622in}}{\pgfqpoint{3.696000in}{3.696000in}}%
\pgfusepath{clip}%
\pgfsetbuttcap%
\pgfsetroundjoin%
\definecolor{currentfill}{rgb}{0.121569,0.466667,0.705882}%
\pgfsetfillcolor{currentfill}%
\pgfsetfillopacity{0.587557}%
\pgfsetlinewidth{1.003750pt}%
\definecolor{currentstroke}{rgb}{0.121569,0.466667,0.705882}%
\pgfsetstrokecolor{currentstroke}%
\pgfsetstrokeopacity{0.587557}%
\pgfsetdash{}{0pt}%
\pgfpathmoveto{\pgfqpoint{3.303434in}{2.080259in}}%
\pgfpathcurveto{\pgfqpoint{3.311671in}{2.080259in}}{\pgfqpoint{3.319571in}{2.083532in}}{\pgfqpoint{3.325395in}{2.089356in}}%
\pgfpathcurveto{\pgfqpoint{3.331219in}{2.095179in}}{\pgfqpoint{3.334491in}{2.103080in}}{\pgfqpoint{3.334491in}{2.111316in}}%
\pgfpathcurveto{\pgfqpoint{3.334491in}{2.119552in}}{\pgfqpoint{3.331219in}{2.127452in}}{\pgfqpoint{3.325395in}{2.133276in}}%
\pgfpathcurveto{\pgfqpoint{3.319571in}{2.139100in}}{\pgfqpoint{3.311671in}{2.142372in}}{\pgfqpoint{3.303434in}{2.142372in}}%
\pgfpathcurveto{\pgfqpoint{3.295198in}{2.142372in}}{\pgfqpoint{3.287298in}{2.139100in}}{\pgfqpoint{3.281474in}{2.133276in}}%
\pgfpathcurveto{\pgfqpoint{3.275650in}{2.127452in}}{\pgfqpoint{3.272378in}{2.119552in}}{\pgfqpoint{3.272378in}{2.111316in}}%
\pgfpathcurveto{\pgfqpoint{3.272378in}{2.103080in}}{\pgfqpoint{3.275650in}{2.095179in}}{\pgfqpoint{3.281474in}{2.089356in}}%
\pgfpathcurveto{\pgfqpoint{3.287298in}{2.083532in}}{\pgfqpoint{3.295198in}{2.080259in}}{\pgfqpoint{3.303434in}{2.080259in}}%
\pgfpathclose%
\pgfusepath{stroke,fill}%
\end{pgfscope}%
\begin{pgfscope}%
\pgfpathrectangle{\pgfqpoint{0.100000in}{0.212622in}}{\pgfqpoint{3.696000in}{3.696000in}}%
\pgfusepath{clip}%
\pgfsetbuttcap%
\pgfsetroundjoin%
\definecolor{currentfill}{rgb}{0.121569,0.466667,0.705882}%
\pgfsetfillcolor{currentfill}%
\pgfsetfillopacity{0.587692}%
\pgfsetlinewidth{1.003750pt}%
\definecolor{currentstroke}{rgb}{0.121569,0.466667,0.705882}%
\pgfsetstrokecolor{currentstroke}%
\pgfsetstrokeopacity{0.587692}%
\pgfsetdash{}{0pt}%
\pgfpathmoveto{\pgfqpoint{3.303202in}{2.080036in}}%
\pgfpathcurveto{\pgfqpoint{3.311438in}{2.080036in}}{\pgfqpoint{3.319338in}{2.083308in}}{\pgfqpoint{3.325162in}{2.089132in}}%
\pgfpathcurveto{\pgfqpoint{3.330986in}{2.094956in}}{\pgfqpoint{3.334258in}{2.102856in}}{\pgfqpoint{3.334258in}{2.111093in}}%
\pgfpathcurveto{\pgfqpoint{3.334258in}{2.119329in}}{\pgfqpoint{3.330986in}{2.127229in}}{\pgfqpoint{3.325162in}{2.133053in}}%
\pgfpathcurveto{\pgfqpoint{3.319338in}{2.138877in}}{\pgfqpoint{3.311438in}{2.142149in}}{\pgfqpoint{3.303202in}{2.142149in}}%
\pgfpathcurveto{\pgfqpoint{3.294966in}{2.142149in}}{\pgfqpoint{3.287066in}{2.138877in}}{\pgfqpoint{3.281242in}{2.133053in}}%
\pgfpathcurveto{\pgfqpoint{3.275418in}{2.127229in}}{\pgfqpoint{3.272145in}{2.119329in}}{\pgfqpoint{3.272145in}{2.111093in}}%
\pgfpathcurveto{\pgfqpoint{3.272145in}{2.102856in}}{\pgfqpoint{3.275418in}{2.094956in}}{\pgfqpoint{3.281242in}{2.089132in}}%
\pgfpathcurveto{\pgfqpoint{3.287066in}{2.083308in}}{\pgfqpoint{3.294966in}{2.080036in}}{\pgfqpoint{3.303202in}{2.080036in}}%
\pgfpathclose%
\pgfusepath{stroke,fill}%
\end{pgfscope}%
\begin{pgfscope}%
\pgfpathrectangle{\pgfqpoint{0.100000in}{0.212622in}}{\pgfqpoint{3.696000in}{3.696000in}}%
\pgfusepath{clip}%
\pgfsetbuttcap%
\pgfsetroundjoin%
\definecolor{currentfill}{rgb}{0.121569,0.466667,0.705882}%
\pgfsetfillcolor{currentfill}%
\pgfsetfillopacity{0.587932}%
\pgfsetlinewidth{1.003750pt}%
\definecolor{currentstroke}{rgb}{0.121569,0.466667,0.705882}%
\pgfsetstrokecolor{currentstroke}%
\pgfsetstrokeopacity{0.587932}%
\pgfsetdash{}{0pt}%
\pgfpathmoveto{\pgfqpoint{3.302754in}{2.079625in}}%
\pgfpathcurveto{\pgfqpoint{3.310990in}{2.079625in}}{\pgfqpoint{3.318890in}{2.082898in}}{\pgfqpoint{3.324714in}{2.088721in}}%
\pgfpathcurveto{\pgfqpoint{3.330538in}{2.094545in}}{\pgfqpoint{3.333810in}{2.102445in}}{\pgfqpoint{3.333810in}{2.110682in}}%
\pgfpathcurveto{\pgfqpoint{3.333810in}{2.118918in}}{\pgfqpoint{3.330538in}{2.126818in}}{\pgfqpoint{3.324714in}{2.132642in}}%
\pgfpathcurveto{\pgfqpoint{3.318890in}{2.138466in}}{\pgfqpoint{3.310990in}{2.141738in}}{\pgfqpoint{3.302754in}{2.141738in}}%
\pgfpathcurveto{\pgfqpoint{3.294518in}{2.141738in}}{\pgfqpoint{3.286618in}{2.138466in}}{\pgfqpoint{3.280794in}{2.132642in}}%
\pgfpathcurveto{\pgfqpoint{3.274970in}{2.126818in}}{\pgfqpoint{3.271697in}{2.118918in}}{\pgfqpoint{3.271697in}{2.110682in}}%
\pgfpathcurveto{\pgfqpoint{3.271697in}{2.102445in}}{\pgfqpoint{3.274970in}{2.094545in}}{\pgfqpoint{3.280794in}{2.088721in}}%
\pgfpathcurveto{\pgfqpoint{3.286618in}{2.082898in}}{\pgfqpoint{3.294518in}{2.079625in}}{\pgfqpoint{3.302754in}{2.079625in}}%
\pgfpathclose%
\pgfusepath{stroke,fill}%
\end{pgfscope}%
\begin{pgfscope}%
\pgfpathrectangle{\pgfqpoint{0.100000in}{0.212622in}}{\pgfqpoint{3.696000in}{3.696000in}}%
\pgfusepath{clip}%
\pgfsetbuttcap%
\pgfsetroundjoin%
\definecolor{currentfill}{rgb}{0.121569,0.466667,0.705882}%
\pgfsetfillcolor{currentfill}%
\pgfsetfillopacity{0.588359}%
\pgfsetlinewidth{1.003750pt}%
\definecolor{currentstroke}{rgb}{0.121569,0.466667,0.705882}%
\pgfsetstrokecolor{currentstroke}%
\pgfsetstrokeopacity{0.588359}%
\pgfsetdash{}{0pt}%
\pgfpathmoveto{\pgfqpoint{3.301910in}{2.078848in}}%
\pgfpathcurveto{\pgfqpoint{3.310146in}{2.078848in}}{\pgfqpoint{3.318046in}{2.082120in}}{\pgfqpoint{3.323870in}{2.087944in}}%
\pgfpathcurveto{\pgfqpoint{3.329694in}{2.093768in}}{\pgfqpoint{3.332966in}{2.101668in}}{\pgfqpoint{3.332966in}{2.109905in}}%
\pgfpathcurveto{\pgfqpoint{3.332966in}{2.118141in}}{\pgfqpoint{3.329694in}{2.126041in}}{\pgfqpoint{3.323870in}{2.131865in}}%
\pgfpathcurveto{\pgfqpoint{3.318046in}{2.137689in}}{\pgfqpoint{3.310146in}{2.140961in}}{\pgfqpoint{3.301910in}{2.140961in}}%
\pgfpathcurveto{\pgfqpoint{3.293674in}{2.140961in}}{\pgfqpoint{3.285773in}{2.137689in}}{\pgfqpoint{3.279950in}{2.131865in}}%
\pgfpathcurveto{\pgfqpoint{3.274126in}{2.126041in}}{\pgfqpoint{3.270853in}{2.118141in}}{\pgfqpoint{3.270853in}{2.109905in}}%
\pgfpathcurveto{\pgfqpoint{3.270853in}{2.101668in}}{\pgfqpoint{3.274126in}{2.093768in}}{\pgfqpoint{3.279950in}{2.087944in}}%
\pgfpathcurveto{\pgfqpoint{3.285773in}{2.082120in}}{\pgfqpoint{3.293674in}{2.078848in}}{\pgfqpoint{3.301910in}{2.078848in}}%
\pgfpathclose%
\pgfusepath{stroke,fill}%
\end{pgfscope}%
\begin{pgfscope}%
\pgfpathrectangle{\pgfqpoint{0.100000in}{0.212622in}}{\pgfqpoint{3.696000in}{3.696000in}}%
\pgfusepath{clip}%
\pgfsetbuttcap%
\pgfsetroundjoin%
\definecolor{currentfill}{rgb}{0.121569,0.466667,0.705882}%
\pgfsetfillcolor{currentfill}%
\pgfsetfillopacity{0.589113}%
\pgfsetlinewidth{1.003750pt}%
\definecolor{currentstroke}{rgb}{0.121569,0.466667,0.705882}%
\pgfsetstrokecolor{currentstroke}%
\pgfsetstrokeopacity{0.589113}%
\pgfsetdash{}{0pt}%
\pgfpathmoveto{\pgfqpoint{3.300264in}{2.077420in}}%
\pgfpathcurveto{\pgfqpoint{3.308501in}{2.077420in}}{\pgfqpoint{3.316401in}{2.080692in}}{\pgfqpoint{3.322225in}{2.086516in}}%
\pgfpathcurveto{\pgfqpoint{3.328049in}{2.092340in}}{\pgfqpoint{3.331321in}{2.100240in}}{\pgfqpoint{3.331321in}{2.108477in}}%
\pgfpathcurveto{\pgfqpoint{3.331321in}{2.116713in}}{\pgfqpoint{3.328049in}{2.124613in}}{\pgfqpoint{3.322225in}{2.130437in}}%
\pgfpathcurveto{\pgfqpoint{3.316401in}{2.136261in}}{\pgfqpoint{3.308501in}{2.139533in}}{\pgfqpoint{3.300264in}{2.139533in}}%
\pgfpathcurveto{\pgfqpoint{3.292028in}{2.139533in}}{\pgfqpoint{3.284128in}{2.136261in}}{\pgfqpoint{3.278304in}{2.130437in}}%
\pgfpathcurveto{\pgfqpoint{3.272480in}{2.124613in}}{\pgfqpoint{3.269208in}{2.116713in}}{\pgfqpoint{3.269208in}{2.108477in}}%
\pgfpathcurveto{\pgfqpoint{3.269208in}{2.100240in}}{\pgfqpoint{3.272480in}{2.092340in}}{\pgfqpoint{3.278304in}{2.086516in}}%
\pgfpathcurveto{\pgfqpoint{3.284128in}{2.080692in}}{\pgfqpoint{3.292028in}{2.077420in}}{\pgfqpoint{3.300264in}{2.077420in}}%
\pgfpathclose%
\pgfusepath{stroke,fill}%
\end{pgfscope}%
\begin{pgfscope}%
\pgfpathrectangle{\pgfqpoint{0.100000in}{0.212622in}}{\pgfqpoint{3.696000in}{3.696000in}}%
\pgfusepath{clip}%
\pgfsetbuttcap%
\pgfsetroundjoin%
\definecolor{currentfill}{rgb}{0.121569,0.466667,0.705882}%
\pgfsetfillcolor{currentfill}%
\pgfsetfillopacity{0.589723}%
\pgfsetlinewidth{1.003750pt}%
\definecolor{currentstroke}{rgb}{0.121569,0.466667,0.705882}%
\pgfsetstrokecolor{currentstroke}%
\pgfsetstrokeopacity{0.589723}%
\pgfsetdash{}{0pt}%
\pgfpathmoveto{\pgfqpoint{3.298824in}{2.076146in}}%
\pgfpathcurveto{\pgfqpoint{3.307061in}{2.076146in}}{\pgfqpoint{3.314961in}{2.079418in}}{\pgfqpoint{3.320785in}{2.085242in}}%
\pgfpathcurveto{\pgfqpoint{3.326608in}{2.091066in}}{\pgfqpoint{3.329881in}{2.098966in}}{\pgfqpoint{3.329881in}{2.107202in}}%
\pgfpathcurveto{\pgfqpoint{3.329881in}{2.115438in}}{\pgfqpoint{3.326608in}{2.123338in}}{\pgfqpoint{3.320785in}{2.129162in}}%
\pgfpathcurveto{\pgfqpoint{3.314961in}{2.134986in}}{\pgfqpoint{3.307061in}{2.138259in}}{\pgfqpoint{3.298824in}{2.138259in}}%
\pgfpathcurveto{\pgfqpoint{3.290588in}{2.138259in}}{\pgfqpoint{3.282688in}{2.134986in}}{\pgfqpoint{3.276864in}{2.129162in}}%
\pgfpathcurveto{\pgfqpoint{3.271040in}{2.123338in}}{\pgfqpoint{3.267768in}{2.115438in}}{\pgfqpoint{3.267768in}{2.107202in}}%
\pgfpathcurveto{\pgfqpoint{3.267768in}{2.098966in}}{\pgfqpoint{3.271040in}{2.091066in}}{\pgfqpoint{3.276864in}{2.085242in}}%
\pgfpathcurveto{\pgfqpoint{3.282688in}{2.079418in}}{\pgfqpoint{3.290588in}{2.076146in}}{\pgfqpoint{3.298824in}{2.076146in}}%
\pgfpathclose%
\pgfusepath{stroke,fill}%
\end{pgfscope}%
\begin{pgfscope}%
\pgfpathrectangle{\pgfqpoint{0.100000in}{0.212622in}}{\pgfqpoint{3.696000in}{3.696000in}}%
\pgfusepath{clip}%
\pgfsetbuttcap%
\pgfsetroundjoin%
\definecolor{currentfill}{rgb}{0.121569,0.466667,0.705882}%
\pgfsetfillcolor{currentfill}%
\pgfsetfillopacity{0.590849}%
\pgfsetlinewidth{1.003750pt}%
\definecolor{currentstroke}{rgb}{0.121569,0.466667,0.705882}%
\pgfsetstrokecolor{currentstroke}%
\pgfsetstrokeopacity{0.590849}%
\pgfsetdash{}{0pt}%
\pgfpathmoveto{\pgfqpoint{3.296456in}{2.073617in}}%
\pgfpathcurveto{\pgfqpoint{3.304692in}{2.073617in}}{\pgfqpoint{3.312592in}{2.076890in}}{\pgfqpoint{3.318416in}{2.082714in}}%
\pgfpathcurveto{\pgfqpoint{3.324240in}{2.088538in}}{\pgfqpoint{3.327512in}{2.096438in}}{\pgfqpoint{3.327512in}{2.104674in}}%
\pgfpathcurveto{\pgfqpoint{3.327512in}{2.112910in}}{\pgfqpoint{3.324240in}{2.120810in}}{\pgfqpoint{3.318416in}{2.126634in}}%
\pgfpathcurveto{\pgfqpoint{3.312592in}{2.132458in}}{\pgfqpoint{3.304692in}{2.135730in}}{\pgfqpoint{3.296456in}{2.135730in}}%
\pgfpathcurveto{\pgfqpoint{3.288219in}{2.135730in}}{\pgfqpoint{3.280319in}{2.132458in}}{\pgfqpoint{3.274495in}{2.126634in}}%
\pgfpathcurveto{\pgfqpoint{3.268671in}{2.120810in}}{\pgfqpoint{3.265399in}{2.112910in}}{\pgfqpoint{3.265399in}{2.104674in}}%
\pgfpathcurveto{\pgfqpoint{3.265399in}{2.096438in}}{\pgfqpoint{3.268671in}{2.088538in}}{\pgfqpoint{3.274495in}{2.082714in}}%
\pgfpathcurveto{\pgfqpoint{3.280319in}{2.076890in}}{\pgfqpoint{3.288219in}{2.073617in}}{\pgfqpoint{3.296456in}{2.073617in}}%
\pgfpathclose%
\pgfusepath{stroke,fill}%
\end{pgfscope}%
\begin{pgfscope}%
\pgfpathrectangle{\pgfqpoint{0.100000in}{0.212622in}}{\pgfqpoint{3.696000in}{3.696000in}}%
\pgfusepath{clip}%
\pgfsetbuttcap%
\pgfsetroundjoin%
\definecolor{currentfill}{rgb}{0.121569,0.466667,0.705882}%
\pgfsetfillcolor{currentfill}%
\pgfsetfillopacity{0.591860}%
\pgfsetlinewidth{1.003750pt}%
\definecolor{currentstroke}{rgb}{0.121569,0.466667,0.705882}%
\pgfsetstrokecolor{currentstroke}%
\pgfsetstrokeopacity{0.591860}%
\pgfsetdash{}{0pt}%
\pgfpathmoveto{\pgfqpoint{3.294437in}{2.071096in}}%
\pgfpathcurveto{\pgfqpoint{3.302673in}{2.071096in}}{\pgfqpoint{3.310573in}{2.074368in}}{\pgfqpoint{3.316397in}{2.080192in}}%
\pgfpathcurveto{\pgfqpoint{3.322221in}{2.086016in}}{\pgfqpoint{3.325493in}{2.093916in}}{\pgfqpoint{3.325493in}{2.102152in}}%
\pgfpathcurveto{\pgfqpoint{3.325493in}{2.110388in}}{\pgfqpoint{3.322221in}{2.118288in}}{\pgfqpoint{3.316397in}{2.124112in}}%
\pgfpathcurveto{\pgfqpoint{3.310573in}{2.129936in}}{\pgfqpoint{3.302673in}{2.133209in}}{\pgfqpoint{3.294437in}{2.133209in}}%
\pgfpathcurveto{\pgfqpoint{3.286201in}{2.133209in}}{\pgfqpoint{3.278301in}{2.129936in}}{\pgfqpoint{3.272477in}{2.124112in}}%
\pgfpathcurveto{\pgfqpoint{3.266653in}{2.118288in}}{\pgfqpoint{3.263380in}{2.110388in}}{\pgfqpoint{3.263380in}{2.102152in}}%
\pgfpathcurveto{\pgfqpoint{3.263380in}{2.093916in}}{\pgfqpoint{3.266653in}{2.086016in}}{\pgfqpoint{3.272477in}{2.080192in}}%
\pgfpathcurveto{\pgfqpoint{3.278301in}{2.074368in}}{\pgfqpoint{3.286201in}{2.071096in}}{\pgfqpoint{3.294437in}{2.071096in}}%
\pgfpathclose%
\pgfusepath{stroke,fill}%
\end{pgfscope}%
\begin{pgfscope}%
\pgfpathrectangle{\pgfqpoint{0.100000in}{0.212622in}}{\pgfqpoint{3.696000in}{3.696000in}}%
\pgfusepath{clip}%
\pgfsetbuttcap%
\pgfsetroundjoin%
\definecolor{currentfill}{rgb}{0.121569,0.466667,0.705882}%
\pgfsetfillcolor{currentfill}%
\pgfsetfillopacity{0.593545}%
\pgfsetlinewidth{1.003750pt}%
\definecolor{currentstroke}{rgb}{0.121569,0.466667,0.705882}%
\pgfsetstrokecolor{currentstroke}%
\pgfsetstrokeopacity{0.593545}%
\pgfsetdash{}{0pt}%
\pgfpathmoveto{\pgfqpoint{3.290783in}{2.065527in}}%
\pgfpathcurveto{\pgfqpoint{3.299019in}{2.065527in}}{\pgfqpoint{3.306919in}{2.068799in}}{\pgfqpoint{3.312743in}{2.074623in}}%
\pgfpathcurveto{\pgfqpoint{3.318567in}{2.080447in}}{\pgfqpoint{3.321839in}{2.088347in}}{\pgfqpoint{3.321839in}{2.096584in}}%
\pgfpathcurveto{\pgfqpoint{3.321839in}{2.104820in}}{\pgfqpoint{3.318567in}{2.112720in}}{\pgfqpoint{3.312743in}{2.118544in}}%
\pgfpathcurveto{\pgfqpoint{3.306919in}{2.124368in}}{\pgfqpoint{3.299019in}{2.127640in}}{\pgfqpoint{3.290783in}{2.127640in}}%
\pgfpathcurveto{\pgfqpoint{3.282547in}{2.127640in}}{\pgfqpoint{3.274647in}{2.124368in}}{\pgfqpoint{3.268823in}{2.118544in}}%
\pgfpathcurveto{\pgfqpoint{3.262999in}{2.112720in}}{\pgfqpoint{3.259726in}{2.104820in}}{\pgfqpoint{3.259726in}{2.096584in}}%
\pgfpathcurveto{\pgfqpoint{3.259726in}{2.088347in}}{\pgfqpoint{3.262999in}{2.080447in}}{\pgfqpoint{3.268823in}{2.074623in}}%
\pgfpathcurveto{\pgfqpoint{3.274647in}{2.068799in}}{\pgfqpoint{3.282547in}{2.065527in}}{\pgfqpoint{3.290783in}{2.065527in}}%
\pgfpathclose%
\pgfusepath{stroke,fill}%
\end{pgfscope}%
\begin{pgfscope}%
\pgfpathrectangle{\pgfqpoint{0.100000in}{0.212622in}}{\pgfqpoint{3.696000in}{3.696000in}}%
\pgfusepath{clip}%
\pgfsetbuttcap%
\pgfsetroundjoin%
\definecolor{currentfill}{rgb}{0.121569,0.466667,0.705882}%
\pgfsetfillcolor{currentfill}%
\pgfsetfillopacity{0.596385}%
\pgfsetlinewidth{1.003750pt}%
\definecolor{currentstroke}{rgb}{0.121569,0.466667,0.705882}%
\pgfsetstrokecolor{currentstroke}%
\pgfsetstrokeopacity{0.596385}%
\pgfsetdash{}{0pt}%
\pgfpathmoveto{\pgfqpoint{3.283656in}{2.054417in}}%
\pgfpathcurveto{\pgfqpoint{3.291892in}{2.054417in}}{\pgfqpoint{3.299793in}{2.057689in}}{\pgfqpoint{3.305616in}{2.063513in}}%
\pgfpathcurveto{\pgfqpoint{3.311440in}{2.069337in}}{\pgfqpoint{3.314713in}{2.077237in}}{\pgfqpoint{3.314713in}{2.085474in}}%
\pgfpathcurveto{\pgfqpoint{3.314713in}{2.093710in}}{\pgfqpoint{3.311440in}{2.101610in}}{\pgfqpoint{3.305616in}{2.107434in}}%
\pgfpathcurveto{\pgfqpoint{3.299793in}{2.113258in}}{\pgfqpoint{3.291892in}{2.116530in}}{\pgfqpoint{3.283656in}{2.116530in}}%
\pgfpathcurveto{\pgfqpoint{3.275420in}{2.116530in}}{\pgfqpoint{3.267520in}{2.113258in}}{\pgfqpoint{3.261696in}{2.107434in}}%
\pgfpathcurveto{\pgfqpoint{3.255872in}{2.101610in}}{\pgfqpoint{3.252600in}{2.093710in}}{\pgfqpoint{3.252600in}{2.085474in}}%
\pgfpathcurveto{\pgfqpoint{3.252600in}{2.077237in}}{\pgfqpoint{3.255872in}{2.069337in}}{\pgfqpoint{3.261696in}{2.063513in}}%
\pgfpathcurveto{\pgfqpoint{3.267520in}{2.057689in}}{\pgfqpoint{3.275420in}{2.054417in}}{\pgfqpoint{3.283656in}{2.054417in}}%
\pgfpathclose%
\pgfusepath{stroke,fill}%
\end{pgfscope}%
\begin{pgfscope}%
\pgfpathrectangle{\pgfqpoint{0.100000in}{0.212622in}}{\pgfqpoint{3.696000in}{3.696000in}}%
\pgfusepath{clip}%
\pgfsetbuttcap%
\pgfsetroundjoin%
\definecolor{currentfill}{rgb}{0.121569,0.466667,0.705882}%
\pgfsetfillcolor{currentfill}%
\pgfsetfillopacity{0.599275}%
\pgfsetlinewidth{1.003750pt}%
\definecolor{currentstroke}{rgb}{0.121569,0.466667,0.705882}%
\pgfsetstrokecolor{currentstroke}%
\pgfsetstrokeopacity{0.599275}%
\pgfsetdash{}{0pt}%
\pgfpathmoveto{\pgfqpoint{3.276645in}{2.044507in}}%
\pgfpathcurveto{\pgfqpoint{3.284881in}{2.044507in}}{\pgfqpoint{3.292781in}{2.047780in}}{\pgfqpoint{3.298605in}{2.053603in}}%
\pgfpathcurveto{\pgfqpoint{3.304429in}{2.059427in}}{\pgfqpoint{3.307701in}{2.067327in}}{\pgfqpoint{3.307701in}{2.075564in}}%
\pgfpathcurveto{\pgfqpoint{3.307701in}{2.083800in}}{\pgfqpoint{3.304429in}{2.091700in}}{\pgfqpoint{3.298605in}{2.097524in}}%
\pgfpathcurveto{\pgfqpoint{3.292781in}{2.103348in}}{\pgfqpoint{3.284881in}{2.106620in}}{\pgfqpoint{3.276645in}{2.106620in}}%
\pgfpathcurveto{\pgfqpoint{3.268408in}{2.106620in}}{\pgfqpoint{3.260508in}{2.103348in}}{\pgfqpoint{3.254684in}{2.097524in}}%
\pgfpathcurveto{\pgfqpoint{3.248860in}{2.091700in}}{\pgfqpoint{3.245588in}{2.083800in}}{\pgfqpoint{3.245588in}{2.075564in}}%
\pgfpathcurveto{\pgfqpoint{3.245588in}{2.067327in}}{\pgfqpoint{3.248860in}{2.059427in}}{\pgfqpoint{3.254684in}{2.053603in}}%
\pgfpathcurveto{\pgfqpoint{3.260508in}{2.047780in}}{\pgfqpoint{3.268408in}{2.044507in}}{\pgfqpoint{3.276645in}{2.044507in}}%
\pgfpathclose%
\pgfusepath{stroke,fill}%
\end{pgfscope}%
\begin{pgfscope}%
\pgfpathrectangle{\pgfqpoint{0.100000in}{0.212622in}}{\pgfqpoint{3.696000in}{3.696000in}}%
\pgfusepath{clip}%
\pgfsetbuttcap%
\pgfsetroundjoin%
\definecolor{currentfill}{rgb}{0.121569,0.466667,0.705882}%
\pgfsetfillcolor{currentfill}%
\pgfsetfillopacity{0.602005}%
\pgfsetlinewidth{1.003750pt}%
\definecolor{currentstroke}{rgb}{0.121569,0.466667,0.705882}%
\pgfsetstrokecolor{currentstroke}%
\pgfsetstrokeopacity{0.602005}%
\pgfsetdash{}{0pt}%
\pgfpathmoveto{\pgfqpoint{3.270215in}{2.036071in}}%
\pgfpathcurveto{\pgfqpoint{3.278452in}{2.036071in}}{\pgfqpoint{3.286352in}{2.039343in}}{\pgfqpoint{3.292176in}{2.045167in}}%
\pgfpathcurveto{\pgfqpoint{3.298000in}{2.050991in}}{\pgfqpoint{3.301272in}{2.058891in}}{\pgfqpoint{3.301272in}{2.067128in}}%
\pgfpathcurveto{\pgfqpoint{3.301272in}{2.075364in}}{\pgfqpoint{3.298000in}{2.083264in}}{\pgfqpoint{3.292176in}{2.089088in}}%
\pgfpathcurveto{\pgfqpoint{3.286352in}{2.094912in}}{\pgfqpoint{3.278452in}{2.098184in}}{\pgfqpoint{3.270215in}{2.098184in}}%
\pgfpathcurveto{\pgfqpoint{3.261979in}{2.098184in}}{\pgfqpoint{3.254079in}{2.094912in}}{\pgfqpoint{3.248255in}{2.089088in}}%
\pgfpathcurveto{\pgfqpoint{3.242431in}{2.083264in}}{\pgfqpoint{3.239159in}{2.075364in}}{\pgfqpoint{3.239159in}{2.067128in}}%
\pgfpathcurveto{\pgfqpoint{3.239159in}{2.058891in}}{\pgfqpoint{3.242431in}{2.050991in}}{\pgfqpoint{3.248255in}{2.045167in}}%
\pgfpathcurveto{\pgfqpoint{3.254079in}{2.039343in}}{\pgfqpoint{3.261979in}{2.036071in}}{\pgfqpoint{3.270215in}{2.036071in}}%
\pgfpathclose%
\pgfusepath{stroke,fill}%
\end{pgfscope}%
\begin{pgfscope}%
\pgfpathrectangle{\pgfqpoint{0.100000in}{0.212622in}}{\pgfqpoint{3.696000in}{3.696000in}}%
\pgfusepath{clip}%
\pgfsetbuttcap%
\pgfsetroundjoin%
\definecolor{currentfill}{rgb}{0.121569,0.466667,0.705882}%
\pgfsetfillcolor{currentfill}%
\pgfsetfillopacity{0.604279}%
\pgfsetlinewidth{1.003750pt}%
\definecolor{currentstroke}{rgb}{0.121569,0.466667,0.705882}%
\pgfsetstrokecolor{currentstroke}%
\pgfsetstrokeopacity{0.604279}%
\pgfsetdash{}{0pt}%
\pgfpathmoveto{\pgfqpoint{1.720561in}{3.096967in}}%
\pgfpathcurveto{\pgfqpoint{1.728797in}{3.096967in}}{\pgfqpoint{1.736697in}{3.100240in}}{\pgfqpoint{1.742521in}{3.106064in}}%
\pgfpathcurveto{\pgfqpoint{1.748345in}{3.111888in}}{\pgfqpoint{1.751617in}{3.119788in}}{\pgfqpoint{1.751617in}{3.128024in}}%
\pgfpathcurveto{\pgfqpoint{1.751617in}{3.136260in}}{\pgfqpoint{1.748345in}{3.144160in}}{\pgfqpoint{1.742521in}{3.149984in}}%
\pgfpathcurveto{\pgfqpoint{1.736697in}{3.155808in}}{\pgfqpoint{1.728797in}{3.159080in}}{\pgfqpoint{1.720561in}{3.159080in}}%
\pgfpathcurveto{\pgfqpoint{1.712324in}{3.159080in}}{\pgfqpoint{1.704424in}{3.155808in}}{\pgfqpoint{1.698600in}{3.149984in}}%
\pgfpathcurveto{\pgfqpoint{1.692777in}{3.144160in}}{\pgfqpoint{1.689504in}{3.136260in}}{\pgfqpoint{1.689504in}{3.128024in}}%
\pgfpathcurveto{\pgfqpoint{1.689504in}{3.119788in}}{\pgfqpoint{1.692777in}{3.111888in}}{\pgfqpoint{1.698600in}{3.106064in}}%
\pgfpathcurveto{\pgfqpoint{1.704424in}{3.100240in}}{\pgfqpoint{1.712324in}{3.096967in}}{\pgfqpoint{1.720561in}{3.096967in}}%
\pgfpathclose%
\pgfusepath{stroke,fill}%
\end{pgfscope}%
\begin{pgfscope}%
\pgfpathrectangle{\pgfqpoint{0.100000in}{0.212622in}}{\pgfqpoint{3.696000in}{3.696000in}}%
\pgfusepath{clip}%
\pgfsetbuttcap%
\pgfsetroundjoin%
\definecolor{currentfill}{rgb}{0.121569,0.466667,0.705882}%
\pgfsetfillcolor{currentfill}%
\pgfsetfillopacity{0.604279}%
\pgfsetlinewidth{1.003750pt}%
\definecolor{currentstroke}{rgb}{0.121569,0.466667,0.705882}%
\pgfsetstrokecolor{currentstroke}%
\pgfsetstrokeopacity{0.604279}%
\pgfsetdash{}{0pt}%
\pgfpathmoveto{\pgfqpoint{1.720579in}{3.096971in}}%
\pgfpathcurveto{\pgfqpoint{1.728816in}{3.096971in}}{\pgfqpoint{1.736716in}{3.100243in}}{\pgfqpoint{1.742540in}{3.106067in}}%
\pgfpathcurveto{\pgfqpoint{1.748364in}{3.111891in}}{\pgfqpoint{1.751636in}{3.119791in}}{\pgfqpoint{1.751636in}{3.128027in}}%
\pgfpathcurveto{\pgfqpoint{1.751636in}{3.136263in}}{\pgfqpoint{1.748364in}{3.144163in}}{\pgfqpoint{1.742540in}{3.149987in}}%
\pgfpathcurveto{\pgfqpoint{1.736716in}{3.155811in}}{\pgfqpoint{1.728816in}{3.159084in}}{\pgfqpoint{1.720579in}{3.159084in}}%
\pgfpathcurveto{\pgfqpoint{1.712343in}{3.159084in}}{\pgfqpoint{1.704443in}{3.155811in}}{\pgfqpoint{1.698619in}{3.149987in}}%
\pgfpathcurveto{\pgfqpoint{1.692795in}{3.144163in}}{\pgfqpoint{1.689523in}{3.136263in}}{\pgfqpoint{1.689523in}{3.128027in}}%
\pgfpathcurveto{\pgfqpoint{1.689523in}{3.119791in}}{\pgfqpoint{1.692795in}{3.111891in}}{\pgfqpoint{1.698619in}{3.106067in}}%
\pgfpathcurveto{\pgfqpoint{1.704443in}{3.100243in}}{\pgfqpoint{1.712343in}{3.096971in}}{\pgfqpoint{1.720579in}{3.096971in}}%
\pgfpathclose%
\pgfusepath{stroke,fill}%
\end{pgfscope}%
\begin{pgfscope}%
\pgfpathrectangle{\pgfqpoint{0.100000in}{0.212622in}}{\pgfqpoint{3.696000in}{3.696000in}}%
\pgfusepath{clip}%
\pgfsetbuttcap%
\pgfsetroundjoin%
\definecolor{currentfill}{rgb}{0.121569,0.466667,0.705882}%
\pgfsetfillcolor{currentfill}%
\pgfsetfillopacity{0.604279}%
\pgfsetlinewidth{1.003750pt}%
\definecolor{currentstroke}{rgb}{0.121569,0.466667,0.705882}%
\pgfsetstrokecolor{currentstroke}%
\pgfsetstrokeopacity{0.604279}%
\pgfsetdash{}{0pt}%
\pgfpathmoveto{\pgfqpoint{1.720550in}{3.096965in}}%
\pgfpathcurveto{\pgfqpoint{1.728787in}{3.096965in}}{\pgfqpoint{1.736687in}{3.100238in}}{\pgfqpoint{1.742511in}{3.106062in}}%
\pgfpathcurveto{\pgfqpoint{1.748335in}{3.111886in}}{\pgfqpoint{1.751607in}{3.119786in}}{\pgfqpoint{1.751607in}{3.128022in}}%
\pgfpathcurveto{\pgfqpoint{1.751607in}{3.136258in}}{\pgfqpoint{1.748335in}{3.144158in}}{\pgfqpoint{1.742511in}{3.149982in}}%
\pgfpathcurveto{\pgfqpoint{1.736687in}{3.155806in}}{\pgfqpoint{1.728787in}{3.159078in}}{\pgfqpoint{1.720550in}{3.159078in}}%
\pgfpathcurveto{\pgfqpoint{1.712314in}{3.159078in}}{\pgfqpoint{1.704414in}{3.155806in}}{\pgfqpoint{1.698590in}{3.149982in}}%
\pgfpathcurveto{\pgfqpoint{1.692766in}{3.144158in}}{\pgfqpoint{1.689494in}{3.136258in}}{\pgfqpoint{1.689494in}{3.128022in}}%
\pgfpathcurveto{\pgfqpoint{1.689494in}{3.119786in}}{\pgfqpoint{1.692766in}{3.111886in}}{\pgfqpoint{1.698590in}{3.106062in}}%
\pgfpathcurveto{\pgfqpoint{1.704414in}{3.100238in}}{\pgfqpoint{1.712314in}{3.096965in}}{\pgfqpoint{1.720550in}{3.096965in}}%
\pgfpathclose%
\pgfusepath{stroke,fill}%
\end{pgfscope}%
\begin{pgfscope}%
\pgfpathrectangle{\pgfqpoint{0.100000in}{0.212622in}}{\pgfqpoint{3.696000in}{3.696000in}}%
\pgfusepath{clip}%
\pgfsetbuttcap%
\pgfsetroundjoin%
\definecolor{currentfill}{rgb}{0.121569,0.466667,0.705882}%
\pgfsetfillcolor{currentfill}%
\pgfsetfillopacity{0.604279}%
\pgfsetlinewidth{1.003750pt}%
\definecolor{currentstroke}{rgb}{0.121569,0.466667,0.705882}%
\pgfsetstrokecolor{currentstroke}%
\pgfsetstrokeopacity{0.604279}%
\pgfsetdash{}{0pt}%
\pgfpathmoveto{\pgfqpoint{1.720545in}{3.096964in}}%
\pgfpathcurveto{\pgfqpoint{1.728781in}{3.096964in}}{\pgfqpoint{1.736681in}{3.100237in}}{\pgfqpoint{1.742505in}{3.106060in}}%
\pgfpathcurveto{\pgfqpoint{1.748329in}{3.111884in}}{\pgfqpoint{1.751601in}{3.119784in}}{\pgfqpoint{1.751601in}{3.128021in}}%
\pgfpathcurveto{\pgfqpoint{1.751601in}{3.136257in}}{\pgfqpoint{1.748329in}{3.144157in}}{\pgfqpoint{1.742505in}{3.149981in}}%
\pgfpathcurveto{\pgfqpoint{1.736681in}{3.155805in}}{\pgfqpoint{1.728781in}{3.159077in}}{\pgfqpoint{1.720545in}{3.159077in}}%
\pgfpathcurveto{\pgfqpoint{1.712308in}{3.159077in}}{\pgfqpoint{1.704408in}{3.155805in}}{\pgfqpoint{1.698584in}{3.149981in}}%
\pgfpathcurveto{\pgfqpoint{1.692760in}{3.144157in}}{\pgfqpoint{1.689488in}{3.136257in}}{\pgfqpoint{1.689488in}{3.128021in}}%
\pgfpathcurveto{\pgfqpoint{1.689488in}{3.119784in}}{\pgfqpoint{1.692760in}{3.111884in}}{\pgfqpoint{1.698584in}{3.106060in}}%
\pgfpathcurveto{\pgfqpoint{1.704408in}{3.100237in}}{\pgfqpoint{1.712308in}{3.096964in}}{\pgfqpoint{1.720545in}{3.096964in}}%
\pgfpathclose%
\pgfusepath{stroke,fill}%
\end{pgfscope}%
\begin{pgfscope}%
\pgfpathrectangle{\pgfqpoint{0.100000in}{0.212622in}}{\pgfqpoint{3.696000in}{3.696000in}}%
\pgfusepath{clip}%
\pgfsetbuttcap%
\pgfsetroundjoin%
\definecolor{currentfill}{rgb}{0.121569,0.466667,0.705882}%
\pgfsetfillcolor{currentfill}%
\pgfsetfillopacity{0.604279}%
\pgfsetlinewidth{1.003750pt}%
\definecolor{currentstroke}{rgb}{0.121569,0.466667,0.705882}%
\pgfsetstrokecolor{currentstroke}%
\pgfsetstrokeopacity{0.604279}%
\pgfsetdash{}{0pt}%
\pgfpathmoveto{\pgfqpoint{1.720541in}{3.096964in}}%
\pgfpathcurveto{\pgfqpoint{1.728778in}{3.096964in}}{\pgfqpoint{1.736678in}{3.100236in}}{\pgfqpoint{1.742502in}{3.106060in}}%
\pgfpathcurveto{\pgfqpoint{1.748326in}{3.111884in}}{\pgfqpoint{1.751598in}{3.119784in}}{\pgfqpoint{1.751598in}{3.128020in}}%
\pgfpathcurveto{\pgfqpoint{1.751598in}{3.136256in}}{\pgfqpoint{1.748326in}{3.144156in}}{\pgfqpoint{1.742502in}{3.149980in}}%
\pgfpathcurveto{\pgfqpoint{1.736678in}{3.155804in}}{\pgfqpoint{1.728778in}{3.159077in}}{\pgfqpoint{1.720541in}{3.159077in}}%
\pgfpathcurveto{\pgfqpoint{1.712305in}{3.159077in}}{\pgfqpoint{1.704405in}{3.155804in}}{\pgfqpoint{1.698581in}{3.149980in}}%
\pgfpathcurveto{\pgfqpoint{1.692757in}{3.144156in}}{\pgfqpoint{1.689485in}{3.136256in}}{\pgfqpoint{1.689485in}{3.128020in}}%
\pgfpathcurveto{\pgfqpoint{1.689485in}{3.119784in}}{\pgfqpoint{1.692757in}{3.111884in}}{\pgfqpoint{1.698581in}{3.106060in}}%
\pgfpathcurveto{\pgfqpoint{1.704405in}{3.100236in}}{\pgfqpoint{1.712305in}{3.096964in}}{\pgfqpoint{1.720541in}{3.096964in}}%
\pgfpathclose%
\pgfusepath{stroke,fill}%
\end{pgfscope}%
\begin{pgfscope}%
\pgfpathrectangle{\pgfqpoint{0.100000in}{0.212622in}}{\pgfqpoint{3.696000in}{3.696000in}}%
\pgfusepath{clip}%
\pgfsetbuttcap%
\pgfsetroundjoin%
\definecolor{currentfill}{rgb}{0.121569,0.466667,0.705882}%
\pgfsetfillcolor{currentfill}%
\pgfsetfillopacity{0.604279}%
\pgfsetlinewidth{1.003750pt}%
\definecolor{currentstroke}{rgb}{0.121569,0.466667,0.705882}%
\pgfsetstrokecolor{currentstroke}%
\pgfsetstrokeopacity{0.604279}%
\pgfsetdash{}{0pt}%
\pgfpathmoveto{\pgfqpoint{1.720540in}{3.096963in}}%
\pgfpathcurveto{\pgfqpoint{1.728776in}{3.096963in}}{\pgfqpoint{1.736676in}{3.100235in}}{\pgfqpoint{1.742500in}{3.106059in}}%
\pgfpathcurveto{\pgfqpoint{1.748324in}{3.111883in}}{\pgfqpoint{1.751596in}{3.119783in}}{\pgfqpoint{1.751596in}{3.128020in}}%
\pgfpathcurveto{\pgfqpoint{1.751596in}{3.136256in}}{\pgfqpoint{1.748324in}{3.144156in}}{\pgfqpoint{1.742500in}{3.149980in}}%
\pgfpathcurveto{\pgfqpoint{1.736676in}{3.155804in}}{\pgfqpoint{1.728776in}{3.159076in}}{\pgfqpoint{1.720540in}{3.159076in}}%
\pgfpathcurveto{\pgfqpoint{1.712303in}{3.159076in}}{\pgfqpoint{1.704403in}{3.155804in}}{\pgfqpoint{1.698579in}{3.149980in}}%
\pgfpathcurveto{\pgfqpoint{1.692755in}{3.144156in}}{\pgfqpoint{1.689483in}{3.136256in}}{\pgfqpoint{1.689483in}{3.128020in}}%
\pgfpathcurveto{\pgfqpoint{1.689483in}{3.119783in}}{\pgfqpoint{1.692755in}{3.111883in}}{\pgfqpoint{1.698579in}{3.106059in}}%
\pgfpathcurveto{\pgfqpoint{1.704403in}{3.100235in}}{\pgfqpoint{1.712303in}{3.096963in}}{\pgfqpoint{1.720540in}{3.096963in}}%
\pgfpathclose%
\pgfusepath{stroke,fill}%
\end{pgfscope}%
\begin{pgfscope}%
\pgfpathrectangle{\pgfqpoint{0.100000in}{0.212622in}}{\pgfqpoint{3.696000in}{3.696000in}}%
\pgfusepath{clip}%
\pgfsetbuttcap%
\pgfsetroundjoin%
\definecolor{currentfill}{rgb}{0.121569,0.466667,0.705882}%
\pgfsetfillcolor{currentfill}%
\pgfsetfillopacity{0.604279}%
\pgfsetlinewidth{1.003750pt}%
\definecolor{currentstroke}{rgb}{0.121569,0.466667,0.705882}%
\pgfsetstrokecolor{currentstroke}%
\pgfsetstrokeopacity{0.604279}%
\pgfsetdash{}{0pt}%
\pgfpathmoveto{\pgfqpoint{1.720539in}{3.096963in}}%
\pgfpathcurveto{\pgfqpoint{1.728775in}{3.096963in}}{\pgfqpoint{1.736675in}{3.100235in}}{\pgfqpoint{1.742499in}{3.106059in}}%
\pgfpathcurveto{\pgfqpoint{1.748323in}{3.111883in}}{\pgfqpoint{1.751595in}{3.119783in}}{\pgfqpoint{1.751595in}{3.128019in}}%
\pgfpathcurveto{\pgfqpoint{1.751595in}{3.136256in}}{\pgfqpoint{1.748323in}{3.144156in}}{\pgfqpoint{1.742499in}{3.149980in}}%
\pgfpathcurveto{\pgfqpoint{1.736675in}{3.155804in}}{\pgfqpoint{1.728775in}{3.159076in}}{\pgfqpoint{1.720539in}{3.159076in}}%
\pgfpathcurveto{\pgfqpoint{1.712302in}{3.159076in}}{\pgfqpoint{1.704402in}{3.155804in}}{\pgfqpoint{1.698578in}{3.149980in}}%
\pgfpathcurveto{\pgfqpoint{1.692754in}{3.144156in}}{\pgfqpoint{1.689482in}{3.136256in}}{\pgfqpoint{1.689482in}{3.128019in}}%
\pgfpathcurveto{\pgfqpoint{1.689482in}{3.119783in}}{\pgfqpoint{1.692754in}{3.111883in}}{\pgfqpoint{1.698578in}{3.106059in}}%
\pgfpathcurveto{\pgfqpoint{1.704402in}{3.100235in}}{\pgfqpoint{1.712302in}{3.096963in}}{\pgfqpoint{1.720539in}{3.096963in}}%
\pgfpathclose%
\pgfusepath{stroke,fill}%
\end{pgfscope}%
\begin{pgfscope}%
\pgfpathrectangle{\pgfqpoint{0.100000in}{0.212622in}}{\pgfqpoint{3.696000in}{3.696000in}}%
\pgfusepath{clip}%
\pgfsetbuttcap%
\pgfsetroundjoin%
\definecolor{currentfill}{rgb}{0.121569,0.466667,0.705882}%
\pgfsetfillcolor{currentfill}%
\pgfsetfillopacity{0.604279}%
\pgfsetlinewidth{1.003750pt}%
\definecolor{currentstroke}{rgb}{0.121569,0.466667,0.705882}%
\pgfsetstrokecolor{currentstroke}%
\pgfsetstrokeopacity{0.604279}%
\pgfsetdash{}{0pt}%
\pgfpathmoveto{\pgfqpoint{1.720538in}{3.096963in}}%
\pgfpathcurveto{\pgfqpoint{1.728774in}{3.096963in}}{\pgfqpoint{1.736674in}{3.100235in}}{\pgfqpoint{1.742498in}{3.106059in}}%
\pgfpathcurveto{\pgfqpoint{1.748322in}{3.111883in}}{\pgfqpoint{1.751595in}{3.119783in}}{\pgfqpoint{1.751595in}{3.128019in}}%
\pgfpathcurveto{\pgfqpoint{1.751595in}{3.136255in}}{\pgfqpoint{1.748322in}{3.144156in}}{\pgfqpoint{1.742498in}{3.149979in}}%
\pgfpathcurveto{\pgfqpoint{1.736674in}{3.155803in}}{\pgfqpoint{1.728774in}{3.159076in}}{\pgfqpoint{1.720538in}{3.159076in}}%
\pgfpathcurveto{\pgfqpoint{1.712302in}{3.159076in}}{\pgfqpoint{1.704402in}{3.155803in}}{\pgfqpoint{1.698578in}{3.149979in}}%
\pgfpathcurveto{\pgfqpoint{1.692754in}{3.144156in}}{\pgfqpoint{1.689482in}{3.136255in}}{\pgfqpoint{1.689482in}{3.128019in}}%
\pgfpathcurveto{\pgfqpoint{1.689482in}{3.119783in}}{\pgfqpoint{1.692754in}{3.111883in}}{\pgfqpoint{1.698578in}{3.106059in}}%
\pgfpathcurveto{\pgfqpoint{1.704402in}{3.100235in}}{\pgfqpoint{1.712302in}{3.096963in}}{\pgfqpoint{1.720538in}{3.096963in}}%
\pgfpathclose%
\pgfusepath{stroke,fill}%
\end{pgfscope}%
\begin{pgfscope}%
\pgfpathrectangle{\pgfqpoint{0.100000in}{0.212622in}}{\pgfqpoint{3.696000in}{3.696000in}}%
\pgfusepath{clip}%
\pgfsetbuttcap%
\pgfsetroundjoin%
\definecolor{currentfill}{rgb}{0.121569,0.466667,0.705882}%
\pgfsetfillcolor{currentfill}%
\pgfsetfillopacity{0.604279}%
\pgfsetlinewidth{1.003750pt}%
\definecolor{currentstroke}{rgb}{0.121569,0.466667,0.705882}%
\pgfsetstrokecolor{currentstroke}%
\pgfsetstrokeopacity{0.604279}%
\pgfsetdash{}{0pt}%
\pgfpathmoveto{\pgfqpoint{1.720538in}{3.096963in}}%
\pgfpathcurveto{\pgfqpoint{1.728774in}{3.096963in}}{\pgfqpoint{1.736674in}{3.100235in}}{\pgfqpoint{1.742498in}{3.106059in}}%
\pgfpathcurveto{\pgfqpoint{1.748322in}{3.111883in}}{\pgfqpoint{1.751594in}{3.119783in}}{\pgfqpoint{1.751594in}{3.128019in}}%
\pgfpathcurveto{\pgfqpoint{1.751594in}{3.136255in}}{\pgfqpoint{1.748322in}{3.144155in}}{\pgfqpoint{1.742498in}{3.149979in}}%
\pgfpathcurveto{\pgfqpoint{1.736674in}{3.155803in}}{\pgfqpoint{1.728774in}{3.159076in}}{\pgfqpoint{1.720538in}{3.159076in}}%
\pgfpathcurveto{\pgfqpoint{1.712301in}{3.159076in}}{\pgfqpoint{1.704401in}{3.155803in}}{\pgfqpoint{1.698577in}{3.149979in}}%
\pgfpathcurveto{\pgfqpoint{1.692754in}{3.144155in}}{\pgfqpoint{1.689481in}{3.136255in}}{\pgfqpoint{1.689481in}{3.128019in}}%
\pgfpathcurveto{\pgfqpoint{1.689481in}{3.119783in}}{\pgfqpoint{1.692754in}{3.111883in}}{\pgfqpoint{1.698577in}{3.106059in}}%
\pgfpathcurveto{\pgfqpoint{1.704401in}{3.100235in}}{\pgfqpoint{1.712301in}{3.096963in}}{\pgfqpoint{1.720538in}{3.096963in}}%
\pgfpathclose%
\pgfusepath{stroke,fill}%
\end{pgfscope}%
\begin{pgfscope}%
\pgfpathrectangle{\pgfqpoint{0.100000in}{0.212622in}}{\pgfqpoint{3.696000in}{3.696000in}}%
\pgfusepath{clip}%
\pgfsetbuttcap%
\pgfsetroundjoin%
\definecolor{currentfill}{rgb}{0.121569,0.466667,0.705882}%
\pgfsetfillcolor{currentfill}%
\pgfsetfillopacity{0.604279}%
\pgfsetlinewidth{1.003750pt}%
\definecolor{currentstroke}{rgb}{0.121569,0.466667,0.705882}%
\pgfsetstrokecolor{currentstroke}%
\pgfsetstrokeopacity{0.604279}%
\pgfsetdash{}{0pt}%
\pgfpathmoveto{\pgfqpoint{1.720538in}{3.096963in}}%
\pgfpathcurveto{\pgfqpoint{1.728774in}{3.096963in}}{\pgfqpoint{1.736674in}{3.100235in}}{\pgfqpoint{1.742498in}{3.106059in}}%
\pgfpathcurveto{\pgfqpoint{1.748322in}{3.111883in}}{\pgfqpoint{1.751594in}{3.119783in}}{\pgfqpoint{1.751594in}{3.128019in}}%
\pgfpathcurveto{\pgfqpoint{1.751594in}{3.136255in}}{\pgfqpoint{1.748322in}{3.144155in}}{\pgfqpoint{1.742498in}{3.149979in}}%
\pgfpathcurveto{\pgfqpoint{1.736674in}{3.155803in}}{\pgfqpoint{1.728774in}{3.159076in}}{\pgfqpoint{1.720538in}{3.159076in}}%
\pgfpathcurveto{\pgfqpoint{1.712301in}{3.159076in}}{\pgfqpoint{1.704401in}{3.155803in}}{\pgfqpoint{1.698577in}{3.149979in}}%
\pgfpathcurveto{\pgfqpoint{1.692753in}{3.144155in}}{\pgfqpoint{1.689481in}{3.136255in}}{\pgfqpoint{1.689481in}{3.128019in}}%
\pgfpathcurveto{\pgfqpoint{1.689481in}{3.119783in}}{\pgfqpoint{1.692753in}{3.111883in}}{\pgfqpoint{1.698577in}{3.106059in}}%
\pgfpathcurveto{\pgfqpoint{1.704401in}{3.100235in}}{\pgfqpoint{1.712301in}{3.096963in}}{\pgfqpoint{1.720538in}{3.096963in}}%
\pgfpathclose%
\pgfusepath{stroke,fill}%
\end{pgfscope}%
\begin{pgfscope}%
\pgfpathrectangle{\pgfqpoint{0.100000in}{0.212622in}}{\pgfqpoint{3.696000in}{3.696000in}}%
\pgfusepath{clip}%
\pgfsetbuttcap%
\pgfsetroundjoin%
\definecolor{currentfill}{rgb}{0.121569,0.466667,0.705882}%
\pgfsetfillcolor{currentfill}%
\pgfsetfillopacity{0.604279}%
\pgfsetlinewidth{1.003750pt}%
\definecolor{currentstroke}{rgb}{0.121569,0.466667,0.705882}%
\pgfsetstrokecolor{currentstroke}%
\pgfsetstrokeopacity{0.604279}%
\pgfsetdash{}{0pt}%
\pgfpathmoveto{\pgfqpoint{1.720537in}{3.096963in}}%
\pgfpathcurveto{\pgfqpoint{1.728774in}{3.096963in}}{\pgfqpoint{1.736674in}{3.100235in}}{\pgfqpoint{1.742498in}{3.106059in}}%
\pgfpathcurveto{\pgfqpoint{1.748322in}{3.111883in}}{\pgfqpoint{1.751594in}{3.119783in}}{\pgfqpoint{1.751594in}{3.128019in}}%
\pgfpathcurveto{\pgfqpoint{1.751594in}{3.136255in}}{\pgfqpoint{1.748322in}{3.144155in}}{\pgfqpoint{1.742498in}{3.149979in}}%
\pgfpathcurveto{\pgfqpoint{1.736674in}{3.155803in}}{\pgfqpoint{1.728774in}{3.159076in}}{\pgfqpoint{1.720537in}{3.159076in}}%
\pgfpathcurveto{\pgfqpoint{1.712301in}{3.159076in}}{\pgfqpoint{1.704401in}{3.155803in}}{\pgfqpoint{1.698577in}{3.149979in}}%
\pgfpathcurveto{\pgfqpoint{1.692753in}{3.144155in}}{\pgfqpoint{1.689481in}{3.136255in}}{\pgfqpoint{1.689481in}{3.128019in}}%
\pgfpathcurveto{\pgfqpoint{1.689481in}{3.119783in}}{\pgfqpoint{1.692753in}{3.111883in}}{\pgfqpoint{1.698577in}{3.106059in}}%
\pgfpathcurveto{\pgfqpoint{1.704401in}{3.100235in}}{\pgfqpoint{1.712301in}{3.096963in}}{\pgfqpoint{1.720537in}{3.096963in}}%
\pgfpathclose%
\pgfusepath{stroke,fill}%
\end{pgfscope}%
\begin{pgfscope}%
\pgfpathrectangle{\pgfqpoint{0.100000in}{0.212622in}}{\pgfqpoint{3.696000in}{3.696000in}}%
\pgfusepath{clip}%
\pgfsetbuttcap%
\pgfsetroundjoin%
\definecolor{currentfill}{rgb}{0.121569,0.466667,0.705882}%
\pgfsetfillcolor{currentfill}%
\pgfsetfillopacity{0.604279}%
\pgfsetlinewidth{1.003750pt}%
\definecolor{currentstroke}{rgb}{0.121569,0.466667,0.705882}%
\pgfsetstrokecolor{currentstroke}%
\pgfsetstrokeopacity{0.604279}%
\pgfsetdash{}{0pt}%
\pgfpathmoveto{\pgfqpoint{1.720537in}{3.096962in}}%
\pgfpathcurveto{\pgfqpoint{1.728774in}{3.096962in}}{\pgfqpoint{1.736674in}{3.100235in}}{\pgfqpoint{1.742498in}{3.106059in}}%
\pgfpathcurveto{\pgfqpoint{1.748322in}{3.111883in}}{\pgfqpoint{1.751594in}{3.119783in}}{\pgfqpoint{1.751594in}{3.128019in}}%
\pgfpathcurveto{\pgfqpoint{1.751594in}{3.136255in}}{\pgfqpoint{1.748322in}{3.144155in}}{\pgfqpoint{1.742498in}{3.149979in}}%
\pgfpathcurveto{\pgfqpoint{1.736674in}{3.155803in}}{\pgfqpoint{1.728774in}{3.159075in}}{\pgfqpoint{1.720537in}{3.159075in}}%
\pgfpathcurveto{\pgfqpoint{1.712301in}{3.159075in}}{\pgfqpoint{1.704401in}{3.155803in}}{\pgfqpoint{1.698577in}{3.149979in}}%
\pgfpathcurveto{\pgfqpoint{1.692753in}{3.144155in}}{\pgfqpoint{1.689481in}{3.136255in}}{\pgfqpoint{1.689481in}{3.128019in}}%
\pgfpathcurveto{\pgfqpoint{1.689481in}{3.119783in}}{\pgfqpoint{1.692753in}{3.111883in}}{\pgfqpoint{1.698577in}{3.106059in}}%
\pgfpathcurveto{\pgfqpoint{1.704401in}{3.100235in}}{\pgfqpoint{1.712301in}{3.096962in}}{\pgfqpoint{1.720537in}{3.096962in}}%
\pgfpathclose%
\pgfusepath{stroke,fill}%
\end{pgfscope}%
\begin{pgfscope}%
\pgfpathrectangle{\pgfqpoint{0.100000in}{0.212622in}}{\pgfqpoint{3.696000in}{3.696000in}}%
\pgfusepath{clip}%
\pgfsetbuttcap%
\pgfsetroundjoin%
\definecolor{currentfill}{rgb}{0.121569,0.466667,0.705882}%
\pgfsetfillcolor{currentfill}%
\pgfsetfillopacity{0.604279}%
\pgfsetlinewidth{1.003750pt}%
\definecolor{currentstroke}{rgb}{0.121569,0.466667,0.705882}%
\pgfsetstrokecolor{currentstroke}%
\pgfsetstrokeopacity{0.604279}%
\pgfsetdash{}{0pt}%
\pgfpathmoveto{\pgfqpoint{1.720537in}{3.096962in}}%
\pgfpathcurveto{\pgfqpoint{1.728774in}{3.096962in}}{\pgfqpoint{1.736674in}{3.100235in}}{\pgfqpoint{1.742498in}{3.106059in}}%
\pgfpathcurveto{\pgfqpoint{1.748322in}{3.111883in}}{\pgfqpoint{1.751594in}{3.119783in}}{\pgfqpoint{1.751594in}{3.128019in}}%
\pgfpathcurveto{\pgfqpoint{1.751594in}{3.136255in}}{\pgfqpoint{1.748322in}{3.144155in}}{\pgfqpoint{1.742498in}{3.149979in}}%
\pgfpathcurveto{\pgfqpoint{1.736674in}{3.155803in}}{\pgfqpoint{1.728774in}{3.159075in}}{\pgfqpoint{1.720537in}{3.159075in}}%
\pgfpathcurveto{\pgfqpoint{1.712301in}{3.159075in}}{\pgfqpoint{1.704401in}{3.155803in}}{\pgfqpoint{1.698577in}{3.149979in}}%
\pgfpathcurveto{\pgfqpoint{1.692753in}{3.144155in}}{\pgfqpoint{1.689481in}{3.136255in}}{\pgfqpoint{1.689481in}{3.128019in}}%
\pgfpathcurveto{\pgfqpoint{1.689481in}{3.119783in}}{\pgfqpoint{1.692753in}{3.111883in}}{\pgfqpoint{1.698577in}{3.106059in}}%
\pgfpathcurveto{\pgfqpoint{1.704401in}{3.100235in}}{\pgfqpoint{1.712301in}{3.096962in}}{\pgfqpoint{1.720537in}{3.096962in}}%
\pgfpathclose%
\pgfusepath{stroke,fill}%
\end{pgfscope}%
\begin{pgfscope}%
\pgfpathrectangle{\pgfqpoint{0.100000in}{0.212622in}}{\pgfqpoint{3.696000in}{3.696000in}}%
\pgfusepath{clip}%
\pgfsetbuttcap%
\pgfsetroundjoin%
\definecolor{currentfill}{rgb}{0.121569,0.466667,0.705882}%
\pgfsetfillcolor{currentfill}%
\pgfsetfillopacity{0.604279}%
\pgfsetlinewidth{1.003750pt}%
\definecolor{currentstroke}{rgb}{0.121569,0.466667,0.705882}%
\pgfsetstrokecolor{currentstroke}%
\pgfsetstrokeopacity{0.604279}%
\pgfsetdash{}{0pt}%
\pgfpathmoveto{\pgfqpoint{1.720537in}{3.096962in}}%
\pgfpathcurveto{\pgfqpoint{1.728774in}{3.096962in}}{\pgfqpoint{1.736674in}{3.100235in}}{\pgfqpoint{1.742498in}{3.106059in}}%
\pgfpathcurveto{\pgfqpoint{1.748322in}{3.111883in}}{\pgfqpoint{1.751594in}{3.119783in}}{\pgfqpoint{1.751594in}{3.128019in}}%
\pgfpathcurveto{\pgfqpoint{1.751594in}{3.136255in}}{\pgfqpoint{1.748322in}{3.144155in}}{\pgfqpoint{1.742498in}{3.149979in}}%
\pgfpathcurveto{\pgfqpoint{1.736674in}{3.155803in}}{\pgfqpoint{1.728774in}{3.159075in}}{\pgfqpoint{1.720537in}{3.159075in}}%
\pgfpathcurveto{\pgfqpoint{1.712301in}{3.159075in}}{\pgfqpoint{1.704401in}{3.155803in}}{\pgfqpoint{1.698577in}{3.149979in}}%
\pgfpathcurveto{\pgfqpoint{1.692753in}{3.144155in}}{\pgfqpoint{1.689481in}{3.136255in}}{\pgfqpoint{1.689481in}{3.128019in}}%
\pgfpathcurveto{\pgfqpoint{1.689481in}{3.119783in}}{\pgfqpoint{1.692753in}{3.111883in}}{\pgfqpoint{1.698577in}{3.106059in}}%
\pgfpathcurveto{\pgfqpoint{1.704401in}{3.100235in}}{\pgfqpoint{1.712301in}{3.096962in}}{\pgfqpoint{1.720537in}{3.096962in}}%
\pgfpathclose%
\pgfusepath{stroke,fill}%
\end{pgfscope}%
\begin{pgfscope}%
\pgfpathrectangle{\pgfqpoint{0.100000in}{0.212622in}}{\pgfqpoint{3.696000in}{3.696000in}}%
\pgfusepath{clip}%
\pgfsetbuttcap%
\pgfsetroundjoin%
\definecolor{currentfill}{rgb}{0.121569,0.466667,0.705882}%
\pgfsetfillcolor{currentfill}%
\pgfsetfillopacity{0.604279}%
\pgfsetlinewidth{1.003750pt}%
\definecolor{currentstroke}{rgb}{0.121569,0.466667,0.705882}%
\pgfsetstrokecolor{currentstroke}%
\pgfsetstrokeopacity{0.604279}%
\pgfsetdash{}{0pt}%
\pgfpathmoveto{\pgfqpoint{1.720537in}{3.096962in}}%
\pgfpathcurveto{\pgfqpoint{1.728774in}{3.096962in}}{\pgfqpoint{1.736674in}{3.100235in}}{\pgfqpoint{1.742498in}{3.106059in}}%
\pgfpathcurveto{\pgfqpoint{1.748322in}{3.111883in}}{\pgfqpoint{1.751594in}{3.119783in}}{\pgfqpoint{1.751594in}{3.128019in}}%
\pgfpathcurveto{\pgfqpoint{1.751594in}{3.136255in}}{\pgfqpoint{1.748322in}{3.144155in}}{\pgfqpoint{1.742498in}{3.149979in}}%
\pgfpathcurveto{\pgfqpoint{1.736674in}{3.155803in}}{\pgfqpoint{1.728774in}{3.159075in}}{\pgfqpoint{1.720537in}{3.159075in}}%
\pgfpathcurveto{\pgfqpoint{1.712301in}{3.159075in}}{\pgfqpoint{1.704401in}{3.155803in}}{\pgfqpoint{1.698577in}{3.149979in}}%
\pgfpathcurveto{\pgfqpoint{1.692753in}{3.144155in}}{\pgfqpoint{1.689481in}{3.136255in}}{\pgfqpoint{1.689481in}{3.128019in}}%
\pgfpathcurveto{\pgfqpoint{1.689481in}{3.119783in}}{\pgfqpoint{1.692753in}{3.111883in}}{\pgfqpoint{1.698577in}{3.106059in}}%
\pgfpathcurveto{\pgfqpoint{1.704401in}{3.100235in}}{\pgfqpoint{1.712301in}{3.096962in}}{\pgfqpoint{1.720537in}{3.096962in}}%
\pgfpathclose%
\pgfusepath{stroke,fill}%
\end{pgfscope}%
\begin{pgfscope}%
\pgfpathrectangle{\pgfqpoint{0.100000in}{0.212622in}}{\pgfqpoint{3.696000in}{3.696000in}}%
\pgfusepath{clip}%
\pgfsetbuttcap%
\pgfsetroundjoin%
\definecolor{currentfill}{rgb}{0.121569,0.466667,0.705882}%
\pgfsetfillcolor{currentfill}%
\pgfsetfillopacity{0.604279}%
\pgfsetlinewidth{1.003750pt}%
\definecolor{currentstroke}{rgb}{0.121569,0.466667,0.705882}%
\pgfsetstrokecolor{currentstroke}%
\pgfsetstrokeopacity{0.604279}%
\pgfsetdash{}{0pt}%
\pgfpathmoveto{\pgfqpoint{1.720537in}{3.096962in}}%
\pgfpathcurveto{\pgfqpoint{1.728774in}{3.096962in}}{\pgfqpoint{1.736674in}{3.100235in}}{\pgfqpoint{1.742498in}{3.106059in}}%
\pgfpathcurveto{\pgfqpoint{1.748322in}{3.111883in}}{\pgfqpoint{1.751594in}{3.119783in}}{\pgfqpoint{1.751594in}{3.128019in}}%
\pgfpathcurveto{\pgfqpoint{1.751594in}{3.136255in}}{\pgfqpoint{1.748322in}{3.144155in}}{\pgfqpoint{1.742498in}{3.149979in}}%
\pgfpathcurveto{\pgfqpoint{1.736674in}{3.155803in}}{\pgfqpoint{1.728774in}{3.159075in}}{\pgfqpoint{1.720537in}{3.159075in}}%
\pgfpathcurveto{\pgfqpoint{1.712301in}{3.159075in}}{\pgfqpoint{1.704401in}{3.155803in}}{\pgfqpoint{1.698577in}{3.149979in}}%
\pgfpathcurveto{\pgfqpoint{1.692753in}{3.144155in}}{\pgfqpoint{1.689481in}{3.136255in}}{\pgfqpoint{1.689481in}{3.128019in}}%
\pgfpathcurveto{\pgfqpoint{1.689481in}{3.119783in}}{\pgfqpoint{1.692753in}{3.111883in}}{\pgfqpoint{1.698577in}{3.106059in}}%
\pgfpathcurveto{\pgfqpoint{1.704401in}{3.100235in}}{\pgfqpoint{1.712301in}{3.096962in}}{\pgfqpoint{1.720537in}{3.096962in}}%
\pgfpathclose%
\pgfusepath{stroke,fill}%
\end{pgfscope}%
\begin{pgfscope}%
\pgfpathrectangle{\pgfqpoint{0.100000in}{0.212622in}}{\pgfqpoint{3.696000in}{3.696000in}}%
\pgfusepath{clip}%
\pgfsetbuttcap%
\pgfsetroundjoin%
\definecolor{currentfill}{rgb}{0.121569,0.466667,0.705882}%
\pgfsetfillcolor{currentfill}%
\pgfsetfillopacity{0.604279}%
\pgfsetlinewidth{1.003750pt}%
\definecolor{currentstroke}{rgb}{0.121569,0.466667,0.705882}%
\pgfsetstrokecolor{currentstroke}%
\pgfsetstrokeopacity{0.604279}%
\pgfsetdash{}{0pt}%
\pgfpathmoveto{\pgfqpoint{1.720537in}{3.096962in}}%
\pgfpathcurveto{\pgfqpoint{1.728774in}{3.096962in}}{\pgfqpoint{1.736674in}{3.100235in}}{\pgfqpoint{1.742498in}{3.106059in}}%
\pgfpathcurveto{\pgfqpoint{1.748322in}{3.111883in}}{\pgfqpoint{1.751594in}{3.119783in}}{\pgfqpoint{1.751594in}{3.128019in}}%
\pgfpathcurveto{\pgfqpoint{1.751594in}{3.136255in}}{\pgfqpoint{1.748322in}{3.144155in}}{\pgfqpoint{1.742498in}{3.149979in}}%
\pgfpathcurveto{\pgfqpoint{1.736674in}{3.155803in}}{\pgfqpoint{1.728774in}{3.159075in}}{\pgfqpoint{1.720537in}{3.159075in}}%
\pgfpathcurveto{\pgfqpoint{1.712301in}{3.159075in}}{\pgfqpoint{1.704401in}{3.155803in}}{\pgfqpoint{1.698577in}{3.149979in}}%
\pgfpathcurveto{\pgfqpoint{1.692753in}{3.144155in}}{\pgfqpoint{1.689481in}{3.136255in}}{\pgfqpoint{1.689481in}{3.128019in}}%
\pgfpathcurveto{\pgfqpoint{1.689481in}{3.119783in}}{\pgfqpoint{1.692753in}{3.111883in}}{\pgfqpoint{1.698577in}{3.106059in}}%
\pgfpathcurveto{\pgfqpoint{1.704401in}{3.100235in}}{\pgfqpoint{1.712301in}{3.096962in}}{\pgfqpoint{1.720537in}{3.096962in}}%
\pgfpathclose%
\pgfusepath{stroke,fill}%
\end{pgfscope}%
\begin{pgfscope}%
\pgfpathrectangle{\pgfqpoint{0.100000in}{0.212622in}}{\pgfqpoint{3.696000in}{3.696000in}}%
\pgfusepath{clip}%
\pgfsetbuttcap%
\pgfsetroundjoin%
\definecolor{currentfill}{rgb}{0.121569,0.466667,0.705882}%
\pgfsetfillcolor{currentfill}%
\pgfsetfillopacity{0.604279}%
\pgfsetlinewidth{1.003750pt}%
\definecolor{currentstroke}{rgb}{0.121569,0.466667,0.705882}%
\pgfsetstrokecolor{currentstroke}%
\pgfsetstrokeopacity{0.604279}%
\pgfsetdash{}{0pt}%
\pgfpathmoveto{\pgfqpoint{1.720537in}{3.096962in}}%
\pgfpathcurveto{\pgfqpoint{1.728774in}{3.096962in}}{\pgfqpoint{1.736674in}{3.100235in}}{\pgfqpoint{1.742498in}{3.106059in}}%
\pgfpathcurveto{\pgfqpoint{1.748322in}{3.111883in}}{\pgfqpoint{1.751594in}{3.119783in}}{\pgfqpoint{1.751594in}{3.128019in}}%
\pgfpathcurveto{\pgfqpoint{1.751594in}{3.136255in}}{\pgfqpoint{1.748322in}{3.144155in}}{\pgfqpoint{1.742498in}{3.149979in}}%
\pgfpathcurveto{\pgfqpoint{1.736674in}{3.155803in}}{\pgfqpoint{1.728774in}{3.159075in}}{\pgfqpoint{1.720537in}{3.159075in}}%
\pgfpathcurveto{\pgfqpoint{1.712301in}{3.159075in}}{\pgfqpoint{1.704401in}{3.155803in}}{\pgfqpoint{1.698577in}{3.149979in}}%
\pgfpathcurveto{\pgfqpoint{1.692753in}{3.144155in}}{\pgfqpoint{1.689481in}{3.136255in}}{\pgfqpoint{1.689481in}{3.128019in}}%
\pgfpathcurveto{\pgfqpoint{1.689481in}{3.119783in}}{\pgfqpoint{1.692753in}{3.111883in}}{\pgfqpoint{1.698577in}{3.106059in}}%
\pgfpathcurveto{\pgfqpoint{1.704401in}{3.100235in}}{\pgfqpoint{1.712301in}{3.096962in}}{\pgfqpoint{1.720537in}{3.096962in}}%
\pgfpathclose%
\pgfusepath{stroke,fill}%
\end{pgfscope}%
\begin{pgfscope}%
\pgfpathrectangle{\pgfqpoint{0.100000in}{0.212622in}}{\pgfqpoint{3.696000in}{3.696000in}}%
\pgfusepath{clip}%
\pgfsetbuttcap%
\pgfsetroundjoin%
\definecolor{currentfill}{rgb}{0.121569,0.466667,0.705882}%
\pgfsetfillcolor{currentfill}%
\pgfsetfillopacity{0.604279}%
\pgfsetlinewidth{1.003750pt}%
\definecolor{currentstroke}{rgb}{0.121569,0.466667,0.705882}%
\pgfsetstrokecolor{currentstroke}%
\pgfsetstrokeopacity{0.604279}%
\pgfsetdash{}{0pt}%
\pgfpathmoveto{\pgfqpoint{1.720537in}{3.096962in}}%
\pgfpathcurveto{\pgfqpoint{1.728774in}{3.096962in}}{\pgfqpoint{1.736674in}{3.100235in}}{\pgfqpoint{1.742498in}{3.106059in}}%
\pgfpathcurveto{\pgfqpoint{1.748322in}{3.111883in}}{\pgfqpoint{1.751594in}{3.119783in}}{\pgfqpoint{1.751594in}{3.128019in}}%
\pgfpathcurveto{\pgfqpoint{1.751594in}{3.136255in}}{\pgfqpoint{1.748322in}{3.144155in}}{\pgfqpoint{1.742498in}{3.149979in}}%
\pgfpathcurveto{\pgfqpoint{1.736674in}{3.155803in}}{\pgfqpoint{1.728774in}{3.159075in}}{\pgfqpoint{1.720537in}{3.159075in}}%
\pgfpathcurveto{\pgfqpoint{1.712301in}{3.159075in}}{\pgfqpoint{1.704401in}{3.155803in}}{\pgfqpoint{1.698577in}{3.149979in}}%
\pgfpathcurveto{\pgfqpoint{1.692753in}{3.144155in}}{\pgfqpoint{1.689481in}{3.136255in}}{\pgfqpoint{1.689481in}{3.128019in}}%
\pgfpathcurveto{\pgfqpoint{1.689481in}{3.119783in}}{\pgfqpoint{1.692753in}{3.111883in}}{\pgfqpoint{1.698577in}{3.106059in}}%
\pgfpathcurveto{\pgfqpoint{1.704401in}{3.100235in}}{\pgfqpoint{1.712301in}{3.096962in}}{\pgfqpoint{1.720537in}{3.096962in}}%
\pgfpathclose%
\pgfusepath{stroke,fill}%
\end{pgfscope}%
\begin{pgfscope}%
\pgfpathrectangle{\pgfqpoint{0.100000in}{0.212622in}}{\pgfqpoint{3.696000in}{3.696000in}}%
\pgfusepath{clip}%
\pgfsetbuttcap%
\pgfsetroundjoin%
\definecolor{currentfill}{rgb}{0.121569,0.466667,0.705882}%
\pgfsetfillcolor{currentfill}%
\pgfsetfillopacity{0.604279}%
\pgfsetlinewidth{1.003750pt}%
\definecolor{currentstroke}{rgb}{0.121569,0.466667,0.705882}%
\pgfsetstrokecolor{currentstroke}%
\pgfsetstrokeopacity{0.604279}%
\pgfsetdash{}{0pt}%
\pgfpathmoveto{\pgfqpoint{1.720537in}{3.096962in}}%
\pgfpathcurveto{\pgfqpoint{1.728774in}{3.096962in}}{\pgfqpoint{1.736674in}{3.100235in}}{\pgfqpoint{1.742498in}{3.106059in}}%
\pgfpathcurveto{\pgfqpoint{1.748322in}{3.111883in}}{\pgfqpoint{1.751594in}{3.119783in}}{\pgfqpoint{1.751594in}{3.128019in}}%
\pgfpathcurveto{\pgfqpoint{1.751594in}{3.136255in}}{\pgfqpoint{1.748322in}{3.144155in}}{\pgfqpoint{1.742498in}{3.149979in}}%
\pgfpathcurveto{\pgfqpoint{1.736674in}{3.155803in}}{\pgfqpoint{1.728774in}{3.159075in}}{\pgfqpoint{1.720537in}{3.159075in}}%
\pgfpathcurveto{\pgfqpoint{1.712301in}{3.159075in}}{\pgfqpoint{1.704401in}{3.155803in}}{\pgfqpoint{1.698577in}{3.149979in}}%
\pgfpathcurveto{\pgfqpoint{1.692753in}{3.144155in}}{\pgfqpoint{1.689481in}{3.136255in}}{\pgfqpoint{1.689481in}{3.128019in}}%
\pgfpathcurveto{\pgfqpoint{1.689481in}{3.119783in}}{\pgfqpoint{1.692753in}{3.111883in}}{\pgfqpoint{1.698577in}{3.106059in}}%
\pgfpathcurveto{\pgfqpoint{1.704401in}{3.100235in}}{\pgfqpoint{1.712301in}{3.096962in}}{\pgfqpoint{1.720537in}{3.096962in}}%
\pgfpathclose%
\pgfusepath{stroke,fill}%
\end{pgfscope}%
\begin{pgfscope}%
\pgfpathrectangle{\pgfqpoint{0.100000in}{0.212622in}}{\pgfqpoint{3.696000in}{3.696000in}}%
\pgfusepath{clip}%
\pgfsetbuttcap%
\pgfsetroundjoin%
\definecolor{currentfill}{rgb}{0.121569,0.466667,0.705882}%
\pgfsetfillcolor{currentfill}%
\pgfsetfillopacity{0.604279}%
\pgfsetlinewidth{1.003750pt}%
\definecolor{currentstroke}{rgb}{0.121569,0.466667,0.705882}%
\pgfsetstrokecolor{currentstroke}%
\pgfsetstrokeopacity{0.604279}%
\pgfsetdash{}{0pt}%
\pgfpathmoveto{\pgfqpoint{1.720537in}{3.096962in}}%
\pgfpathcurveto{\pgfqpoint{1.728774in}{3.096962in}}{\pgfqpoint{1.736674in}{3.100235in}}{\pgfqpoint{1.742498in}{3.106059in}}%
\pgfpathcurveto{\pgfqpoint{1.748322in}{3.111883in}}{\pgfqpoint{1.751594in}{3.119783in}}{\pgfqpoint{1.751594in}{3.128019in}}%
\pgfpathcurveto{\pgfqpoint{1.751594in}{3.136255in}}{\pgfqpoint{1.748322in}{3.144155in}}{\pgfqpoint{1.742498in}{3.149979in}}%
\pgfpathcurveto{\pgfqpoint{1.736674in}{3.155803in}}{\pgfqpoint{1.728774in}{3.159075in}}{\pgfqpoint{1.720537in}{3.159075in}}%
\pgfpathcurveto{\pgfqpoint{1.712301in}{3.159075in}}{\pgfqpoint{1.704401in}{3.155803in}}{\pgfqpoint{1.698577in}{3.149979in}}%
\pgfpathcurveto{\pgfqpoint{1.692753in}{3.144155in}}{\pgfqpoint{1.689481in}{3.136255in}}{\pgfqpoint{1.689481in}{3.128019in}}%
\pgfpathcurveto{\pgfqpoint{1.689481in}{3.119783in}}{\pgfqpoint{1.692753in}{3.111883in}}{\pgfqpoint{1.698577in}{3.106059in}}%
\pgfpathcurveto{\pgfqpoint{1.704401in}{3.100235in}}{\pgfqpoint{1.712301in}{3.096962in}}{\pgfqpoint{1.720537in}{3.096962in}}%
\pgfpathclose%
\pgfusepath{stroke,fill}%
\end{pgfscope}%
\begin{pgfscope}%
\pgfpathrectangle{\pgfqpoint{0.100000in}{0.212622in}}{\pgfqpoint{3.696000in}{3.696000in}}%
\pgfusepath{clip}%
\pgfsetbuttcap%
\pgfsetroundjoin%
\definecolor{currentfill}{rgb}{0.121569,0.466667,0.705882}%
\pgfsetfillcolor{currentfill}%
\pgfsetfillopacity{0.604279}%
\pgfsetlinewidth{1.003750pt}%
\definecolor{currentstroke}{rgb}{0.121569,0.466667,0.705882}%
\pgfsetstrokecolor{currentstroke}%
\pgfsetstrokeopacity{0.604279}%
\pgfsetdash{}{0pt}%
\pgfpathmoveto{\pgfqpoint{1.720537in}{3.096962in}}%
\pgfpathcurveto{\pgfqpoint{1.728774in}{3.096962in}}{\pgfqpoint{1.736674in}{3.100235in}}{\pgfqpoint{1.742498in}{3.106059in}}%
\pgfpathcurveto{\pgfqpoint{1.748322in}{3.111883in}}{\pgfqpoint{1.751594in}{3.119783in}}{\pgfqpoint{1.751594in}{3.128019in}}%
\pgfpathcurveto{\pgfqpoint{1.751594in}{3.136255in}}{\pgfqpoint{1.748322in}{3.144155in}}{\pgfqpoint{1.742498in}{3.149979in}}%
\pgfpathcurveto{\pgfqpoint{1.736674in}{3.155803in}}{\pgfqpoint{1.728774in}{3.159075in}}{\pgfqpoint{1.720537in}{3.159075in}}%
\pgfpathcurveto{\pgfqpoint{1.712301in}{3.159075in}}{\pgfqpoint{1.704401in}{3.155803in}}{\pgfqpoint{1.698577in}{3.149979in}}%
\pgfpathcurveto{\pgfqpoint{1.692753in}{3.144155in}}{\pgfqpoint{1.689481in}{3.136255in}}{\pgfqpoint{1.689481in}{3.128019in}}%
\pgfpathcurveto{\pgfqpoint{1.689481in}{3.119783in}}{\pgfqpoint{1.692753in}{3.111883in}}{\pgfqpoint{1.698577in}{3.106059in}}%
\pgfpathcurveto{\pgfqpoint{1.704401in}{3.100235in}}{\pgfqpoint{1.712301in}{3.096962in}}{\pgfqpoint{1.720537in}{3.096962in}}%
\pgfpathclose%
\pgfusepath{stroke,fill}%
\end{pgfscope}%
\begin{pgfscope}%
\pgfpathrectangle{\pgfqpoint{0.100000in}{0.212622in}}{\pgfqpoint{3.696000in}{3.696000in}}%
\pgfusepath{clip}%
\pgfsetbuttcap%
\pgfsetroundjoin%
\definecolor{currentfill}{rgb}{0.121569,0.466667,0.705882}%
\pgfsetfillcolor{currentfill}%
\pgfsetfillopacity{0.604279}%
\pgfsetlinewidth{1.003750pt}%
\definecolor{currentstroke}{rgb}{0.121569,0.466667,0.705882}%
\pgfsetstrokecolor{currentstroke}%
\pgfsetstrokeopacity{0.604279}%
\pgfsetdash{}{0pt}%
\pgfpathmoveto{\pgfqpoint{1.720537in}{3.096962in}}%
\pgfpathcurveto{\pgfqpoint{1.728774in}{3.096962in}}{\pgfqpoint{1.736674in}{3.100235in}}{\pgfqpoint{1.742498in}{3.106059in}}%
\pgfpathcurveto{\pgfqpoint{1.748322in}{3.111883in}}{\pgfqpoint{1.751594in}{3.119783in}}{\pgfqpoint{1.751594in}{3.128019in}}%
\pgfpathcurveto{\pgfqpoint{1.751594in}{3.136255in}}{\pgfqpoint{1.748322in}{3.144155in}}{\pgfqpoint{1.742498in}{3.149979in}}%
\pgfpathcurveto{\pgfqpoint{1.736674in}{3.155803in}}{\pgfqpoint{1.728774in}{3.159075in}}{\pgfqpoint{1.720537in}{3.159075in}}%
\pgfpathcurveto{\pgfqpoint{1.712301in}{3.159075in}}{\pgfqpoint{1.704401in}{3.155803in}}{\pgfqpoint{1.698577in}{3.149979in}}%
\pgfpathcurveto{\pgfqpoint{1.692753in}{3.144155in}}{\pgfqpoint{1.689481in}{3.136255in}}{\pgfqpoint{1.689481in}{3.128019in}}%
\pgfpathcurveto{\pgfqpoint{1.689481in}{3.119783in}}{\pgfqpoint{1.692753in}{3.111883in}}{\pgfqpoint{1.698577in}{3.106059in}}%
\pgfpathcurveto{\pgfqpoint{1.704401in}{3.100235in}}{\pgfqpoint{1.712301in}{3.096962in}}{\pgfqpoint{1.720537in}{3.096962in}}%
\pgfpathclose%
\pgfusepath{stroke,fill}%
\end{pgfscope}%
\begin{pgfscope}%
\pgfpathrectangle{\pgfqpoint{0.100000in}{0.212622in}}{\pgfqpoint{3.696000in}{3.696000in}}%
\pgfusepath{clip}%
\pgfsetbuttcap%
\pgfsetroundjoin%
\definecolor{currentfill}{rgb}{0.121569,0.466667,0.705882}%
\pgfsetfillcolor{currentfill}%
\pgfsetfillopacity{0.604279}%
\pgfsetlinewidth{1.003750pt}%
\definecolor{currentstroke}{rgb}{0.121569,0.466667,0.705882}%
\pgfsetstrokecolor{currentstroke}%
\pgfsetstrokeopacity{0.604279}%
\pgfsetdash{}{0pt}%
\pgfpathmoveto{\pgfqpoint{1.720537in}{3.096962in}}%
\pgfpathcurveto{\pgfqpoint{1.728774in}{3.096962in}}{\pgfqpoint{1.736674in}{3.100235in}}{\pgfqpoint{1.742498in}{3.106059in}}%
\pgfpathcurveto{\pgfqpoint{1.748322in}{3.111883in}}{\pgfqpoint{1.751594in}{3.119783in}}{\pgfqpoint{1.751594in}{3.128019in}}%
\pgfpathcurveto{\pgfqpoint{1.751594in}{3.136255in}}{\pgfqpoint{1.748322in}{3.144155in}}{\pgfqpoint{1.742498in}{3.149979in}}%
\pgfpathcurveto{\pgfqpoint{1.736674in}{3.155803in}}{\pgfqpoint{1.728774in}{3.159075in}}{\pgfqpoint{1.720537in}{3.159075in}}%
\pgfpathcurveto{\pgfqpoint{1.712301in}{3.159075in}}{\pgfqpoint{1.704401in}{3.155803in}}{\pgfqpoint{1.698577in}{3.149979in}}%
\pgfpathcurveto{\pgfqpoint{1.692753in}{3.144155in}}{\pgfqpoint{1.689481in}{3.136255in}}{\pgfqpoint{1.689481in}{3.128019in}}%
\pgfpathcurveto{\pgfqpoint{1.689481in}{3.119783in}}{\pgfqpoint{1.692753in}{3.111883in}}{\pgfqpoint{1.698577in}{3.106059in}}%
\pgfpathcurveto{\pgfqpoint{1.704401in}{3.100235in}}{\pgfqpoint{1.712301in}{3.096962in}}{\pgfqpoint{1.720537in}{3.096962in}}%
\pgfpathclose%
\pgfusepath{stroke,fill}%
\end{pgfscope}%
\begin{pgfscope}%
\pgfpathrectangle{\pgfqpoint{0.100000in}{0.212622in}}{\pgfqpoint{3.696000in}{3.696000in}}%
\pgfusepath{clip}%
\pgfsetbuttcap%
\pgfsetroundjoin%
\definecolor{currentfill}{rgb}{0.121569,0.466667,0.705882}%
\pgfsetfillcolor{currentfill}%
\pgfsetfillopacity{0.604279}%
\pgfsetlinewidth{1.003750pt}%
\definecolor{currentstroke}{rgb}{0.121569,0.466667,0.705882}%
\pgfsetstrokecolor{currentstroke}%
\pgfsetstrokeopacity{0.604279}%
\pgfsetdash{}{0pt}%
\pgfpathmoveto{\pgfqpoint{1.720537in}{3.096962in}}%
\pgfpathcurveto{\pgfqpoint{1.728774in}{3.096962in}}{\pgfqpoint{1.736674in}{3.100235in}}{\pgfqpoint{1.742498in}{3.106059in}}%
\pgfpathcurveto{\pgfqpoint{1.748322in}{3.111883in}}{\pgfqpoint{1.751594in}{3.119783in}}{\pgfqpoint{1.751594in}{3.128019in}}%
\pgfpathcurveto{\pgfqpoint{1.751594in}{3.136255in}}{\pgfqpoint{1.748322in}{3.144155in}}{\pgfqpoint{1.742498in}{3.149979in}}%
\pgfpathcurveto{\pgfqpoint{1.736674in}{3.155803in}}{\pgfqpoint{1.728774in}{3.159075in}}{\pgfqpoint{1.720537in}{3.159075in}}%
\pgfpathcurveto{\pgfqpoint{1.712301in}{3.159075in}}{\pgfqpoint{1.704401in}{3.155803in}}{\pgfqpoint{1.698577in}{3.149979in}}%
\pgfpathcurveto{\pgfqpoint{1.692753in}{3.144155in}}{\pgfqpoint{1.689481in}{3.136255in}}{\pgfqpoint{1.689481in}{3.128019in}}%
\pgfpathcurveto{\pgfqpoint{1.689481in}{3.119783in}}{\pgfqpoint{1.692753in}{3.111883in}}{\pgfqpoint{1.698577in}{3.106059in}}%
\pgfpathcurveto{\pgfqpoint{1.704401in}{3.100235in}}{\pgfqpoint{1.712301in}{3.096962in}}{\pgfqpoint{1.720537in}{3.096962in}}%
\pgfpathclose%
\pgfusepath{stroke,fill}%
\end{pgfscope}%
\begin{pgfscope}%
\pgfpathrectangle{\pgfqpoint{0.100000in}{0.212622in}}{\pgfqpoint{3.696000in}{3.696000in}}%
\pgfusepath{clip}%
\pgfsetbuttcap%
\pgfsetroundjoin%
\definecolor{currentfill}{rgb}{0.121569,0.466667,0.705882}%
\pgfsetfillcolor{currentfill}%
\pgfsetfillopacity{0.604279}%
\pgfsetlinewidth{1.003750pt}%
\definecolor{currentstroke}{rgb}{0.121569,0.466667,0.705882}%
\pgfsetstrokecolor{currentstroke}%
\pgfsetstrokeopacity{0.604279}%
\pgfsetdash{}{0pt}%
\pgfpathmoveto{\pgfqpoint{1.720537in}{3.096962in}}%
\pgfpathcurveto{\pgfqpoint{1.728774in}{3.096962in}}{\pgfqpoint{1.736674in}{3.100235in}}{\pgfqpoint{1.742498in}{3.106059in}}%
\pgfpathcurveto{\pgfqpoint{1.748322in}{3.111883in}}{\pgfqpoint{1.751594in}{3.119783in}}{\pgfqpoint{1.751594in}{3.128019in}}%
\pgfpathcurveto{\pgfqpoint{1.751594in}{3.136255in}}{\pgfqpoint{1.748322in}{3.144155in}}{\pgfqpoint{1.742498in}{3.149979in}}%
\pgfpathcurveto{\pgfqpoint{1.736674in}{3.155803in}}{\pgfqpoint{1.728774in}{3.159075in}}{\pgfqpoint{1.720537in}{3.159075in}}%
\pgfpathcurveto{\pgfqpoint{1.712301in}{3.159075in}}{\pgfqpoint{1.704401in}{3.155803in}}{\pgfqpoint{1.698577in}{3.149979in}}%
\pgfpathcurveto{\pgfqpoint{1.692753in}{3.144155in}}{\pgfqpoint{1.689481in}{3.136255in}}{\pgfqpoint{1.689481in}{3.128019in}}%
\pgfpathcurveto{\pgfqpoint{1.689481in}{3.119783in}}{\pgfqpoint{1.692753in}{3.111883in}}{\pgfqpoint{1.698577in}{3.106059in}}%
\pgfpathcurveto{\pgfqpoint{1.704401in}{3.100235in}}{\pgfqpoint{1.712301in}{3.096962in}}{\pgfqpoint{1.720537in}{3.096962in}}%
\pgfpathclose%
\pgfusepath{stroke,fill}%
\end{pgfscope}%
\begin{pgfscope}%
\pgfpathrectangle{\pgfqpoint{0.100000in}{0.212622in}}{\pgfqpoint{3.696000in}{3.696000in}}%
\pgfusepath{clip}%
\pgfsetbuttcap%
\pgfsetroundjoin%
\definecolor{currentfill}{rgb}{0.121569,0.466667,0.705882}%
\pgfsetfillcolor{currentfill}%
\pgfsetfillopacity{0.604279}%
\pgfsetlinewidth{1.003750pt}%
\definecolor{currentstroke}{rgb}{0.121569,0.466667,0.705882}%
\pgfsetstrokecolor{currentstroke}%
\pgfsetstrokeopacity{0.604279}%
\pgfsetdash{}{0pt}%
\pgfpathmoveto{\pgfqpoint{1.720537in}{3.096962in}}%
\pgfpathcurveto{\pgfqpoint{1.728774in}{3.096962in}}{\pgfqpoint{1.736674in}{3.100235in}}{\pgfqpoint{1.742498in}{3.106059in}}%
\pgfpathcurveto{\pgfqpoint{1.748322in}{3.111883in}}{\pgfqpoint{1.751594in}{3.119783in}}{\pgfqpoint{1.751594in}{3.128019in}}%
\pgfpathcurveto{\pgfqpoint{1.751594in}{3.136255in}}{\pgfqpoint{1.748322in}{3.144155in}}{\pgfqpoint{1.742498in}{3.149979in}}%
\pgfpathcurveto{\pgfqpoint{1.736674in}{3.155803in}}{\pgfqpoint{1.728774in}{3.159075in}}{\pgfqpoint{1.720537in}{3.159075in}}%
\pgfpathcurveto{\pgfqpoint{1.712301in}{3.159075in}}{\pgfqpoint{1.704401in}{3.155803in}}{\pgfqpoint{1.698577in}{3.149979in}}%
\pgfpathcurveto{\pgfqpoint{1.692753in}{3.144155in}}{\pgfqpoint{1.689481in}{3.136255in}}{\pgfqpoint{1.689481in}{3.128019in}}%
\pgfpathcurveto{\pgfqpoint{1.689481in}{3.119783in}}{\pgfqpoint{1.692753in}{3.111883in}}{\pgfqpoint{1.698577in}{3.106059in}}%
\pgfpathcurveto{\pgfqpoint{1.704401in}{3.100235in}}{\pgfqpoint{1.712301in}{3.096962in}}{\pgfqpoint{1.720537in}{3.096962in}}%
\pgfpathclose%
\pgfusepath{stroke,fill}%
\end{pgfscope}%
\begin{pgfscope}%
\pgfpathrectangle{\pgfqpoint{0.100000in}{0.212622in}}{\pgfqpoint{3.696000in}{3.696000in}}%
\pgfusepath{clip}%
\pgfsetbuttcap%
\pgfsetroundjoin%
\definecolor{currentfill}{rgb}{0.121569,0.466667,0.705882}%
\pgfsetfillcolor{currentfill}%
\pgfsetfillopacity{0.604279}%
\pgfsetlinewidth{1.003750pt}%
\definecolor{currentstroke}{rgb}{0.121569,0.466667,0.705882}%
\pgfsetstrokecolor{currentstroke}%
\pgfsetstrokeopacity{0.604279}%
\pgfsetdash{}{0pt}%
\pgfpathmoveto{\pgfqpoint{1.720537in}{3.096962in}}%
\pgfpathcurveto{\pgfqpoint{1.728774in}{3.096962in}}{\pgfqpoint{1.736674in}{3.100235in}}{\pgfqpoint{1.742498in}{3.106059in}}%
\pgfpathcurveto{\pgfqpoint{1.748322in}{3.111883in}}{\pgfqpoint{1.751594in}{3.119783in}}{\pgfqpoint{1.751594in}{3.128019in}}%
\pgfpathcurveto{\pgfqpoint{1.751594in}{3.136255in}}{\pgfqpoint{1.748322in}{3.144155in}}{\pgfqpoint{1.742498in}{3.149979in}}%
\pgfpathcurveto{\pgfqpoint{1.736674in}{3.155803in}}{\pgfqpoint{1.728774in}{3.159075in}}{\pgfqpoint{1.720537in}{3.159075in}}%
\pgfpathcurveto{\pgfqpoint{1.712301in}{3.159075in}}{\pgfqpoint{1.704401in}{3.155803in}}{\pgfqpoint{1.698577in}{3.149979in}}%
\pgfpathcurveto{\pgfqpoint{1.692753in}{3.144155in}}{\pgfqpoint{1.689481in}{3.136255in}}{\pgfqpoint{1.689481in}{3.128019in}}%
\pgfpathcurveto{\pgfqpoint{1.689481in}{3.119783in}}{\pgfqpoint{1.692753in}{3.111883in}}{\pgfqpoint{1.698577in}{3.106059in}}%
\pgfpathcurveto{\pgfqpoint{1.704401in}{3.100235in}}{\pgfqpoint{1.712301in}{3.096962in}}{\pgfqpoint{1.720537in}{3.096962in}}%
\pgfpathclose%
\pgfusepath{stroke,fill}%
\end{pgfscope}%
\begin{pgfscope}%
\pgfpathrectangle{\pgfqpoint{0.100000in}{0.212622in}}{\pgfqpoint{3.696000in}{3.696000in}}%
\pgfusepath{clip}%
\pgfsetbuttcap%
\pgfsetroundjoin%
\definecolor{currentfill}{rgb}{0.121569,0.466667,0.705882}%
\pgfsetfillcolor{currentfill}%
\pgfsetfillopacity{0.604279}%
\pgfsetlinewidth{1.003750pt}%
\definecolor{currentstroke}{rgb}{0.121569,0.466667,0.705882}%
\pgfsetstrokecolor{currentstroke}%
\pgfsetstrokeopacity{0.604279}%
\pgfsetdash{}{0pt}%
\pgfpathmoveto{\pgfqpoint{1.720537in}{3.096962in}}%
\pgfpathcurveto{\pgfqpoint{1.728774in}{3.096962in}}{\pgfqpoint{1.736674in}{3.100235in}}{\pgfqpoint{1.742498in}{3.106059in}}%
\pgfpathcurveto{\pgfqpoint{1.748322in}{3.111883in}}{\pgfqpoint{1.751594in}{3.119783in}}{\pgfqpoint{1.751594in}{3.128019in}}%
\pgfpathcurveto{\pgfqpoint{1.751594in}{3.136255in}}{\pgfqpoint{1.748322in}{3.144155in}}{\pgfqpoint{1.742498in}{3.149979in}}%
\pgfpathcurveto{\pgfqpoint{1.736674in}{3.155803in}}{\pgfqpoint{1.728774in}{3.159075in}}{\pgfqpoint{1.720537in}{3.159075in}}%
\pgfpathcurveto{\pgfqpoint{1.712301in}{3.159075in}}{\pgfqpoint{1.704401in}{3.155803in}}{\pgfqpoint{1.698577in}{3.149979in}}%
\pgfpathcurveto{\pgfqpoint{1.692753in}{3.144155in}}{\pgfqpoint{1.689481in}{3.136255in}}{\pgfqpoint{1.689481in}{3.128019in}}%
\pgfpathcurveto{\pgfqpoint{1.689481in}{3.119783in}}{\pgfqpoint{1.692753in}{3.111883in}}{\pgfqpoint{1.698577in}{3.106059in}}%
\pgfpathcurveto{\pgfqpoint{1.704401in}{3.100235in}}{\pgfqpoint{1.712301in}{3.096962in}}{\pgfqpoint{1.720537in}{3.096962in}}%
\pgfpathclose%
\pgfusepath{stroke,fill}%
\end{pgfscope}%
\begin{pgfscope}%
\pgfpathrectangle{\pgfqpoint{0.100000in}{0.212622in}}{\pgfqpoint{3.696000in}{3.696000in}}%
\pgfusepath{clip}%
\pgfsetbuttcap%
\pgfsetroundjoin%
\definecolor{currentfill}{rgb}{0.121569,0.466667,0.705882}%
\pgfsetfillcolor{currentfill}%
\pgfsetfillopacity{0.604279}%
\pgfsetlinewidth{1.003750pt}%
\definecolor{currentstroke}{rgb}{0.121569,0.466667,0.705882}%
\pgfsetstrokecolor{currentstroke}%
\pgfsetstrokeopacity{0.604279}%
\pgfsetdash{}{0pt}%
\pgfpathmoveto{\pgfqpoint{1.720537in}{3.096962in}}%
\pgfpathcurveto{\pgfqpoint{1.728774in}{3.096962in}}{\pgfqpoint{1.736674in}{3.100235in}}{\pgfqpoint{1.742498in}{3.106059in}}%
\pgfpathcurveto{\pgfqpoint{1.748322in}{3.111883in}}{\pgfqpoint{1.751594in}{3.119783in}}{\pgfqpoint{1.751594in}{3.128019in}}%
\pgfpathcurveto{\pgfqpoint{1.751594in}{3.136255in}}{\pgfqpoint{1.748322in}{3.144155in}}{\pgfqpoint{1.742498in}{3.149979in}}%
\pgfpathcurveto{\pgfqpoint{1.736674in}{3.155803in}}{\pgfqpoint{1.728774in}{3.159075in}}{\pgfqpoint{1.720537in}{3.159075in}}%
\pgfpathcurveto{\pgfqpoint{1.712301in}{3.159075in}}{\pgfqpoint{1.704401in}{3.155803in}}{\pgfqpoint{1.698577in}{3.149979in}}%
\pgfpathcurveto{\pgfqpoint{1.692753in}{3.144155in}}{\pgfqpoint{1.689481in}{3.136255in}}{\pgfqpoint{1.689481in}{3.128019in}}%
\pgfpathcurveto{\pgfqpoint{1.689481in}{3.119783in}}{\pgfqpoint{1.692753in}{3.111883in}}{\pgfqpoint{1.698577in}{3.106059in}}%
\pgfpathcurveto{\pgfqpoint{1.704401in}{3.100235in}}{\pgfqpoint{1.712301in}{3.096962in}}{\pgfqpoint{1.720537in}{3.096962in}}%
\pgfpathclose%
\pgfusepath{stroke,fill}%
\end{pgfscope}%
\begin{pgfscope}%
\pgfpathrectangle{\pgfqpoint{0.100000in}{0.212622in}}{\pgfqpoint{3.696000in}{3.696000in}}%
\pgfusepath{clip}%
\pgfsetbuttcap%
\pgfsetroundjoin%
\definecolor{currentfill}{rgb}{0.121569,0.466667,0.705882}%
\pgfsetfillcolor{currentfill}%
\pgfsetfillopacity{0.604279}%
\pgfsetlinewidth{1.003750pt}%
\definecolor{currentstroke}{rgb}{0.121569,0.466667,0.705882}%
\pgfsetstrokecolor{currentstroke}%
\pgfsetstrokeopacity{0.604279}%
\pgfsetdash{}{0pt}%
\pgfpathmoveto{\pgfqpoint{1.720537in}{3.096962in}}%
\pgfpathcurveto{\pgfqpoint{1.728774in}{3.096962in}}{\pgfqpoint{1.736674in}{3.100235in}}{\pgfqpoint{1.742498in}{3.106059in}}%
\pgfpathcurveto{\pgfqpoint{1.748322in}{3.111883in}}{\pgfqpoint{1.751594in}{3.119783in}}{\pgfqpoint{1.751594in}{3.128019in}}%
\pgfpathcurveto{\pgfqpoint{1.751594in}{3.136255in}}{\pgfqpoint{1.748322in}{3.144155in}}{\pgfqpoint{1.742498in}{3.149979in}}%
\pgfpathcurveto{\pgfqpoint{1.736674in}{3.155803in}}{\pgfqpoint{1.728774in}{3.159075in}}{\pgfqpoint{1.720537in}{3.159075in}}%
\pgfpathcurveto{\pgfqpoint{1.712301in}{3.159075in}}{\pgfqpoint{1.704401in}{3.155803in}}{\pgfqpoint{1.698577in}{3.149979in}}%
\pgfpathcurveto{\pgfqpoint{1.692753in}{3.144155in}}{\pgfqpoint{1.689481in}{3.136255in}}{\pgfqpoint{1.689481in}{3.128019in}}%
\pgfpathcurveto{\pgfqpoint{1.689481in}{3.119783in}}{\pgfqpoint{1.692753in}{3.111883in}}{\pgfqpoint{1.698577in}{3.106059in}}%
\pgfpathcurveto{\pgfqpoint{1.704401in}{3.100235in}}{\pgfqpoint{1.712301in}{3.096962in}}{\pgfqpoint{1.720537in}{3.096962in}}%
\pgfpathclose%
\pgfusepath{stroke,fill}%
\end{pgfscope}%
\begin{pgfscope}%
\pgfpathrectangle{\pgfqpoint{0.100000in}{0.212622in}}{\pgfqpoint{3.696000in}{3.696000in}}%
\pgfusepath{clip}%
\pgfsetbuttcap%
\pgfsetroundjoin%
\definecolor{currentfill}{rgb}{0.121569,0.466667,0.705882}%
\pgfsetfillcolor{currentfill}%
\pgfsetfillopacity{0.604279}%
\pgfsetlinewidth{1.003750pt}%
\definecolor{currentstroke}{rgb}{0.121569,0.466667,0.705882}%
\pgfsetstrokecolor{currentstroke}%
\pgfsetstrokeopacity{0.604279}%
\pgfsetdash{}{0pt}%
\pgfpathmoveto{\pgfqpoint{1.720537in}{3.096962in}}%
\pgfpathcurveto{\pgfqpoint{1.728774in}{3.096962in}}{\pgfqpoint{1.736674in}{3.100235in}}{\pgfqpoint{1.742498in}{3.106059in}}%
\pgfpathcurveto{\pgfqpoint{1.748322in}{3.111883in}}{\pgfqpoint{1.751594in}{3.119783in}}{\pgfqpoint{1.751594in}{3.128019in}}%
\pgfpathcurveto{\pgfqpoint{1.751594in}{3.136255in}}{\pgfqpoint{1.748322in}{3.144155in}}{\pgfqpoint{1.742498in}{3.149979in}}%
\pgfpathcurveto{\pgfqpoint{1.736674in}{3.155803in}}{\pgfqpoint{1.728774in}{3.159075in}}{\pgfqpoint{1.720537in}{3.159075in}}%
\pgfpathcurveto{\pgfqpoint{1.712301in}{3.159075in}}{\pgfqpoint{1.704401in}{3.155803in}}{\pgfqpoint{1.698577in}{3.149979in}}%
\pgfpathcurveto{\pgfqpoint{1.692753in}{3.144155in}}{\pgfqpoint{1.689481in}{3.136255in}}{\pgfqpoint{1.689481in}{3.128019in}}%
\pgfpathcurveto{\pgfqpoint{1.689481in}{3.119783in}}{\pgfqpoint{1.692753in}{3.111883in}}{\pgfqpoint{1.698577in}{3.106059in}}%
\pgfpathcurveto{\pgfqpoint{1.704401in}{3.100235in}}{\pgfqpoint{1.712301in}{3.096962in}}{\pgfqpoint{1.720537in}{3.096962in}}%
\pgfpathclose%
\pgfusepath{stroke,fill}%
\end{pgfscope}%
\begin{pgfscope}%
\pgfpathrectangle{\pgfqpoint{0.100000in}{0.212622in}}{\pgfqpoint{3.696000in}{3.696000in}}%
\pgfusepath{clip}%
\pgfsetbuttcap%
\pgfsetroundjoin%
\definecolor{currentfill}{rgb}{0.121569,0.466667,0.705882}%
\pgfsetfillcolor{currentfill}%
\pgfsetfillopacity{0.604279}%
\pgfsetlinewidth{1.003750pt}%
\definecolor{currentstroke}{rgb}{0.121569,0.466667,0.705882}%
\pgfsetstrokecolor{currentstroke}%
\pgfsetstrokeopacity{0.604279}%
\pgfsetdash{}{0pt}%
\pgfpathmoveto{\pgfqpoint{1.720537in}{3.096962in}}%
\pgfpathcurveto{\pgfqpoint{1.728774in}{3.096962in}}{\pgfqpoint{1.736674in}{3.100235in}}{\pgfqpoint{1.742498in}{3.106059in}}%
\pgfpathcurveto{\pgfqpoint{1.748322in}{3.111883in}}{\pgfqpoint{1.751594in}{3.119783in}}{\pgfqpoint{1.751594in}{3.128019in}}%
\pgfpathcurveto{\pgfqpoint{1.751594in}{3.136255in}}{\pgfqpoint{1.748322in}{3.144155in}}{\pgfqpoint{1.742498in}{3.149979in}}%
\pgfpathcurveto{\pgfqpoint{1.736674in}{3.155803in}}{\pgfqpoint{1.728774in}{3.159075in}}{\pgfqpoint{1.720537in}{3.159075in}}%
\pgfpathcurveto{\pgfqpoint{1.712301in}{3.159075in}}{\pgfqpoint{1.704401in}{3.155803in}}{\pgfqpoint{1.698577in}{3.149979in}}%
\pgfpathcurveto{\pgfqpoint{1.692753in}{3.144155in}}{\pgfqpoint{1.689481in}{3.136255in}}{\pgfqpoint{1.689481in}{3.128019in}}%
\pgfpathcurveto{\pgfqpoint{1.689481in}{3.119783in}}{\pgfqpoint{1.692753in}{3.111883in}}{\pgfqpoint{1.698577in}{3.106059in}}%
\pgfpathcurveto{\pgfqpoint{1.704401in}{3.100235in}}{\pgfqpoint{1.712301in}{3.096962in}}{\pgfqpoint{1.720537in}{3.096962in}}%
\pgfpathclose%
\pgfusepath{stroke,fill}%
\end{pgfscope}%
\begin{pgfscope}%
\pgfpathrectangle{\pgfqpoint{0.100000in}{0.212622in}}{\pgfqpoint{3.696000in}{3.696000in}}%
\pgfusepath{clip}%
\pgfsetbuttcap%
\pgfsetroundjoin%
\definecolor{currentfill}{rgb}{0.121569,0.466667,0.705882}%
\pgfsetfillcolor{currentfill}%
\pgfsetfillopacity{0.604279}%
\pgfsetlinewidth{1.003750pt}%
\definecolor{currentstroke}{rgb}{0.121569,0.466667,0.705882}%
\pgfsetstrokecolor{currentstroke}%
\pgfsetstrokeopacity{0.604279}%
\pgfsetdash{}{0pt}%
\pgfpathmoveto{\pgfqpoint{1.720537in}{3.096962in}}%
\pgfpathcurveto{\pgfqpoint{1.728774in}{3.096962in}}{\pgfqpoint{1.736674in}{3.100235in}}{\pgfqpoint{1.742498in}{3.106059in}}%
\pgfpathcurveto{\pgfqpoint{1.748322in}{3.111883in}}{\pgfqpoint{1.751594in}{3.119783in}}{\pgfqpoint{1.751594in}{3.128019in}}%
\pgfpathcurveto{\pgfqpoint{1.751594in}{3.136255in}}{\pgfqpoint{1.748322in}{3.144155in}}{\pgfqpoint{1.742498in}{3.149979in}}%
\pgfpathcurveto{\pgfqpoint{1.736674in}{3.155803in}}{\pgfqpoint{1.728774in}{3.159075in}}{\pgfqpoint{1.720537in}{3.159075in}}%
\pgfpathcurveto{\pgfqpoint{1.712301in}{3.159075in}}{\pgfqpoint{1.704401in}{3.155803in}}{\pgfqpoint{1.698577in}{3.149979in}}%
\pgfpathcurveto{\pgfqpoint{1.692753in}{3.144155in}}{\pgfqpoint{1.689481in}{3.136255in}}{\pgfqpoint{1.689481in}{3.128019in}}%
\pgfpathcurveto{\pgfqpoint{1.689481in}{3.119783in}}{\pgfqpoint{1.692753in}{3.111883in}}{\pgfqpoint{1.698577in}{3.106059in}}%
\pgfpathcurveto{\pgfqpoint{1.704401in}{3.100235in}}{\pgfqpoint{1.712301in}{3.096962in}}{\pgfqpoint{1.720537in}{3.096962in}}%
\pgfpathclose%
\pgfusepath{stroke,fill}%
\end{pgfscope}%
\begin{pgfscope}%
\pgfpathrectangle{\pgfqpoint{0.100000in}{0.212622in}}{\pgfqpoint{3.696000in}{3.696000in}}%
\pgfusepath{clip}%
\pgfsetbuttcap%
\pgfsetroundjoin%
\definecolor{currentfill}{rgb}{0.121569,0.466667,0.705882}%
\pgfsetfillcolor{currentfill}%
\pgfsetfillopacity{0.604279}%
\pgfsetlinewidth{1.003750pt}%
\definecolor{currentstroke}{rgb}{0.121569,0.466667,0.705882}%
\pgfsetstrokecolor{currentstroke}%
\pgfsetstrokeopacity{0.604279}%
\pgfsetdash{}{0pt}%
\pgfpathmoveto{\pgfqpoint{1.720537in}{3.096962in}}%
\pgfpathcurveto{\pgfqpoint{1.728774in}{3.096962in}}{\pgfqpoint{1.736674in}{3.100235in}}{\pgfqpoint{1.742498in}{3.106059in}}%
\pgfpathcurveto{\pgfqpoint{1.748322in}{3.111883in}}{\pgfqpoint{1.751594in}{3.119783in}}{\pgfqpoint{1.751594in}{3.128019in}}%
\pgfpathcurveto{\pgfqpoint{1.751594in}{3.136255in}}{\pgfqpoint{1.748322in}{3.144155in}}{\pgfqpoint{1.742498in}{3.149979in}}%
\pgfpathcurveto{\pgfqpoint{1.736674in}{3.155803in}}{\pgfqpoint{1.728774in}{3.159075in}}{\pgfqpoint{1.720537in}{3.159075in}}%
\pgfpathcurveto{\pgfqpoint{1.712301in}{3.159075in}}{\pgfqpoint{1.704401in}{3.155803in}}{\pgfqpoint{1.698577in}{3.149979in}}%
\pgfpathcurveto{\pgfqpoint{1.692753in}{3.144155in}}{\pgfqpoint{1.689481in}{3.136255in}}{\pgfqpoint{1.689481in}{3.128019in}}%
\pgfpathcurveto{\pgfqpoint{1.689481in}{3.119783in}}{\pgfqpoint{1.692753in}{3.111883in}}{\pgfqpoint{1.698577in}{3.106059in}}%
\pgfpathcurveto{\pgfqpoint{1.704401in}{3.100235in}}{\pgfqpoint{1.712301in}{3.096962in}}{\pgfqpoint{1.720537in}{3.096962in}}%
\pgfpathclose%
\pgfusepath{stroke,fill}%
\end{pgfscope}%
\begin{pgfscope}%
\pgfpathrectangle{\pgfqpoint{0.100000in}{0.212622in}}{\pgfqpoint{3.696000in}{3.696000in}}%
\pgfusepath{clip}%
\pgfsetbuttcap%
\pgfsetroundjoin%
\definecolor{currentfill}{rgb}{0.121569,0.466667,0.705882}%
\pgfsetfillcolor{currentfill}%
\pgfsetfillopacity{0.604279}%
\pgfsetlinewidth{1.003750pt}%
\definecolor{currentstroke}{rgb}{0.121569,0.466667,0.705882}%
\pgfsetstrokecolor{currentstroke}%
\pgfsetstrokeopacity{0.604279}%
\pgfsetdash{}{0pt}%
\pgfpathmoveto{\pgfqpoint{1.720613in}{3.096975in}}%
\pgfpathcurveto{\pgfqpoint{1.728849in}{3.096975in}}{\pgfqpoint{1.736750in}{3.100248in}}{\pgfqpoint{1.742573in}{3.106071in}}%
\pgfpathcurveto{\pgfqpoint{1.748397in}{3.111895in}}{\pgfqpoint{1.751670in}{3.119795in}}{\pgfqpoint{1.751670in}{3.128032in}}%
\pgfpathcurveto{\pgfqpoint{1.751670in}{3.136268in}}{\pgfqpoint{1.748397in}{3.144168in}}{\pgfqpoint{1.742573in}{3.149992in}}%
\pgfpathcurveto{\pgfqpoint{1.736750in}{3.155816in}}{\pgfqpoint{1.728849in}{3.159088in}}{\pgfqpoint{1.720613in}{3.159088in}}%
\pgfpathcurveto{\pgfqpoint{1.712377in}{3.159088in}}{\pgfqpoint{1.704477in}{3.155816in}}{\pgfqpoint{1.698653in}{3.149992in}}%
\pgfpathcurveto{\pgfqpoint{1.692829in}{3.144168in}}{\pgfqpoint{1.689557in}{3.136268in}}{\pgfqpoint{1.689557in}{3.128032in}}%
\pgfpathcurveto{\pgfqpoint{1.689557in}{3.119795in}}{\pgfqpoint{1.692829in}{3.111895in}}{\pgfqpoint{1.698653in}{3.106071in}}%
\pgfpathcurveto{\pgfqpoint{1.704477in}{3.100248in}}{\pgfqpoint{1.712377in}{3.096975in}}{\pgfqpoint{1.720613in}{3.096975in}}%
\pgfpathclose%
\pgfusepath{stroke,fill}%
\end{pgfscope}%
\begin{pgfscope}%
\pgfpathrectangle{\pgfqpoint{0.100000in}{0.212622in}}{\pgfqpoint{3.696000in}{3.696000in}}%
\pgfusepath{clip}%
\pgfsetbuttcap%
\pgfsetroundjoin%
\definecolor{currentfill}{rgb}{0.121569,0.466667,0.705882}%
\pgfsetfillcolor{currentfill}%
\pgfsetfillopacity{0.604281}%
\pgfsetlinewidth{1.003750pt}%
\definecolor{currentstroke}{rgb}{0.121569,0.466667,0.705882}%
\pgfsetstrokecolor{currentstroke}%
\pgfsetstrokeopacity{0.604281}%
\pgfsetdash{}{0pt}%
\pgfpathmoveto{\pgfqpoint{1.720674in}{3.096982in}}%
\pgfpathcurveto{\pgfqpoint{1.728910in}{3.096982in}}{\pgfqpoint{1.736810in}{3.100254in}}{\pgfqpoint{1.742634in}{3.106078in}}%
\pgfpathcurveto{\pgfqpoint{1.748458in}{3.111902in}}{\pgfqpoint{1.751730in}{3.119802in}}{\pgfqpoint{1.751730in}{3.128038in}}%
\pgfpathcurveto{\pgfqpoint{1.751730in}{3.136275in}}{\pgfqpoint{1.748458in}{3.144175in}}{\pgfqpoint{1.742634in}{3.149999in}}%
\pgfpathcurveto{\pgfqpoint{1.736810in}{3.155823in}}{\pgfqpoint{1.728910in}{3.159095in}}{\pgfqpoint{1.720674in}{3.159095in}}%
\pgfpathcurveto{\pgfqpoint{1.712437in}{3.159095in}}{\pgfqpoint{1.704537in}{3.155823in}}{\pgfqpoint{1.698713in}{3.149999in}}%
\pgfpathcurveto{\pgfqpoint{1.692889in}{3.144175in}}{\pgfqpoint{1.689617in}{3.136275in}}{\pgfqpoint{1.689617in}{3.128038in}}%
\pgfpathcurveto{\pgfqpoint{1.689617in}{3.119802in}}{\pgfqpoint{1.692889in}{3.111902in}}{\pgfqpoint{1.698713in}{3.106078in}}%
\pgfpathcurveto{\pgfqpoint{1.704537in}{3.100254in}}{\pgfqpoint{1.712437in}{3.096982in}}{\pgfqpoint{1.720674in}{3.096982in}}%
\pgfpathclose%
\pgfusepath{stroke,fill}%
\end{pgfscope}%
\begin{pgfscope}%
\pgfpathrectangle{\pgfqpoint{0.100000in}{0.212622in}}{\pgfqpoint{3.696000in}{3.696000in}}%
\pgfusepath{clip}%
\pgfsetbuttcap%
\pgfsetroundjoin%
\definecolor{currentfill}{rgb}{0.121569,0.466667,0.705882}%
\pgfsetfillcolor{currentfill}%
\pgfsetfillopacity{0.604285}%
\pgfsetlinewidth{1.003750pt}%
\definecolor{currentstroke}{rgb}{0.121569,0.466667,0.705882}%
\pgfsetstrokecolor{currentstroke}%
\pgfsetstrokeopacity{0.604285}%
\pgfsetdash{}{0pt}%
\pgfpathmoveto{\pgfqpoint{1.720782in}{3.096991in}}%
\pgfpathcurveto{\pgfqpoint{1.729018in}{3.096991in}}{\pgfqpoint{1.736918in}{3.100264in}}{\pgfqpoint{1.742742in}{3.106088in}}%
\pgfpathcurveto{\pgfqpoint{1.748566in}{3.111911in}}{\pgfqpoint{1.751839in}{3.119811in}}{\pgfqpoint{1.751839in}{3.128048in}}%
\pgfpathcurveto{\pgfqpoint{1.751839in}{3.136284in}}{\pgfqpoint{1.748566in}{3.144184in}}{\pgfqpoint{1.742742in}{3.150008in}}%
\pgfpathcurveto{\pgfqpoint{1.736918in}{3.155832in}}{\pgfqpoint{1.729018in}{3.159104in}}{\pgfqpoint{1.720782in}{3.159104in}}%
\pgfpathcurveto{\pgfqpoint{1.712546in}{3.159104in}}{\pgfqpoint{1.704646in}{3.155832in}}{\pgfqpoint{1.698822in}{3.150008in}}%
\pgfpathcurveto{\pgfqpoint{1.692998in}{3.144184in}}{\pgfqpoint{1.689726in}{3.136284in}}{\pgfqpoint{1.689726in}{3.128048in}}%
\pgfpathcurveto{\pgfqpoint{1.689726in}{3.119811in}}{\pgfqpoint{1.692998in}{3.111911in}}{\pgfqpoint{1.698822in}{3.106088in}}%
\pgfpathcurveto{\pgfqpoint{1.704646in}{3.100264in}}{\pgfqpoint{1.712546in}{3.096991in}}{\pgfqpoint{1.720782in}{3.096991in}}%
\pgfpathclose%
\pgfusepath{stroke,fill}%
\end{pgfscope}%
\begin{pgfscope}%
\pgfpathrectangle{\pgfqpoint{0.100000in}{0.212622in}}{\pgfqpoint{3.696000in}{3.696000in}}%
\pgfusepath{clip}%
\pgfsetbuttcap%
\pgfsetroundjoin%
\definecolor{currentfill}{rgb}{0.121569,0.466667,0.705882}%
\pgfsetfillcolor{currentfill}%
\pgfsetfillopacity{0.604294}%
\pgfsetlinewidth{1.003750pt}%
\definecolor{currentstroke}{rgb}{0.121569,0.466667,0.705882}%
\pgfsetstrokecolor{currentstroke}%
\pgfsetstrokeopacity{0.604294}%
\pgfsetdash{}{0pt}%
\pgfpathmoveto{\pgfqpoint{1.720977in}{3.097004in}}%
\pgfpathcurveto{\pgfqpoint{1.729213in}{3.097004in}}{\pgfqpoint{1.737113in}{3.100276in}}{\pgfqpoint{1.742937in}{3.106100in}}%
\pgfpathcurveto{\pgfqpoint{1.748761in}{3.111924in}}{\pgfqpoint{1.752034in}{3.119824in}}{\pgfqpoint{1.752034in}{3.128061in}}%
\pgfpathcurveto{\pgfqpoint{1.752034in}{3.136297in}}{\pgfqpoint{1.748761in}{3.144197in}}{\pgfqpoint{1.742937in}{3.150021in}}%
\pgfpathcurveto{\pgfqpoint{1.737113in}{3.155845in}}{\pgfqpoint{1.729213in}{3.159117in}}{\pgfqpoint{1.720977in}{3.159117in}}%
\pgfpathcurveto{\pgfqpoint{1.712741in}{3.159117in}}{\pgfqpoint{1.704841in}{3.155845in}}{\pgfqpoint{1.699017in}{3.150021in}}%
\pgfpathcurveto{\pgfqpoint{1.693193in}{3.144197in}}{\pgfqpoint{1.689921in}{3.136297in}}{\pgfqpoint{1.689921in}{3.128061in}}%
\pgfpathcurveto{\pgfqpoint{1.689921in}{3.119824in}}{\pgfqpoint{1.693193in}{3.111924in}}{\pgfqpoint{1.699017in}{3.106100in}}%
\pgfpathcurveto{\pgfqpoint{1.704841in}{3.100276in}}{\pgfqpoint{1.712741in}{3.097004in}}{\pgfqpoint{1.720977in}{3.097004in}}%
\pgfpathclose%
\pgfusepath{stroke,fill}%
\end{pgfscope}%
\begin{pgfscope}%
\pgfpathrectangle{\pgfqpoint{0.100000in}{0.212622in}}{\pgfqpoint{3.696000in}{3.696000in}}%
\pgfusepath{clip}%
\pgfsetbuttcap%
\pgfsetroundjoin%
\definecolor{currentfill}{rgb}{0.121569,0.466667,0.705882}%
\pgfsetfillcolor{currentfill}%
\pgfsetfillopacity{0.604311}%
\pgfsetlinewidth{1.003750pt}%
\definecolor{currentstroke}{rgb}{0.121569,0.466667,0.705882}%
\pgfsetstrokecolor{currentstroke}%
\pgfsetstrokeopacity{0.604311}%
\pgfsetdash{}{0pt}%
\pgfpathmoveto{\pgfqpoint{1.721328in}{3.097019in}}%
\pgfpathcurveto{\pgfqpoint{1.729564in}{3.097019in}}{\pgfqpoint{1.737465in}{3.100291in}}{\pgfqpoint{1.743288in}{3.106115in}}%
\pgfpathcurveto{\pgfqpoint{1.749112in}{3.111939in}}{\pgfqpoint{1.752385in}{3.119839in}}{\pgfqpoint{1.752385in}{3.128075in}}%
\pgfpathcurveto{\pgfqpoint{1.752385in}{3.136312in}}{\pgfqpoint{1.749112in}{3.144212in}}{\pgfqpoint{1.743288in}{3.150036in}}%
\pgfpathcurveto{\pgfqpoint{1.737465in}{3.155860in}}{\pgfqpoint{1.729564in}{3.159132in}}{\pgfqpoint{1.721328in}{3.159132in}}%
\pgfpathcurveto{\pgfqpoint{1.713092in}{3.159132in}}{\pgfqpoint{1.705192in}{3.155860in}}{\pgfqpoint{1.699368in}{3.150036in}}%
\pgfpathcurveto{\pgfqpoint{1.693544in}{3.144212in}}{\pgfqpoint{1.690272in}{3.136312in}}{\pgfqpoint{1.690272in}{3.128075in}}%
\pgfpathcurveto{\pgfqpoint{1.690272in}{3.119839in}}{\pgfqpoint{1.693544in}{3.111939in}}{\pgfqpoint{1.699368in}{3.106115in}}%
\pgfpathcurveto{\pgfqpoint{1.705192in}{3.100291in}}{\pgfqpoint{1.713092in}{3.097019in}}{\pgfqpoint{1.721328in}{3.097019in}}%
\pgfpathclose%
\pgfusepath{stroke,fill}%
\end{pgfscope}%
\begin{pgfscope}%
\pgfpathrectangle{\pgfqpoint{0.100000in}{0.212622in}}{\pgfqpoint{3.696000in}{3.696000in}}%
\pgfusepath{clip}%
\pgfsetbuttcap%
\pgfsetroundjoin%
\definecolor{currentfill}{rgb}{0.121569,0.466667,0.705882}%
\pgfsetfillcolor{currentfill}%
\pgfsetfillopacity{0.604311}%
\pgfsetlinewidth{1.003750pt}%
\definecolor{currentstroke}{rgb}{0.121569,0.466667,0.705882}%
\pgfsetstrokecolor{currentstroke}%
\pgfsetstrokeopacity{0.604311}%
\pgfsetdash{}{0pt}%
\pgfpathmoveto{\pgfqpoint{1.721328in}{3.097019in}}%
\pgfpathcurveto{\pgfqpoint{1.729564in}{3.097019in}}{\pgfqpoint{1.737465in}{3.100291in}}{\pgfqpoint{1.743288in}{3.106115in}}%
\pgfpathcurveto{\pgfqpoint{1.749112in}{3.111939in}}{\pgfqpoint{1.752385in}{3.119839in}}{\pgfqpoint{1.752385in}{3.128075in}}%
\pgfpathcurveto{\pgfqpoint{1.752385in}{3.136312in}}{\pgfqpoint{1.749112in}{3.144212in}}{\pgfqpoint{1.743288in}{3.150036in}}%
\pgfpathcurveto{\pgfqpoint{1.737465in}{3.155860in}}{\pgfqpoint{1.729564in}{3.159132in}}{\pgfqpoint{1.721328in}{3.159132in}}%
\pgfpathcurveto{\pgfqpoint{1.713092in}{3.159132in}}{\pgfqpoint{1.705192in}{3.155860in}}{\pgfqpoint{1.699368in}{3.150036in}}%
\pgfpathcurveto{\pgfqpoint{1.693544in}{3.144212in}}{\pgfqpoint{1.690272in}{3.136312in}}{\pgfqpoint{1.690272in}{3.128075in}}%
\pgfpathcurveto{\pgfqpoint{1.690272in}{3.119839in}}{\pgfqpoint{1.693544in}{3.111939in}}{\pgfqpoint{1.699368in}{3.106115in}}%
\pgfpathcurveto{\pgfqpoint{1.705192in}{3.100291in}}{\pgfqpoint{1.713092in}{3.097019in}}{\pgfqpoint{1.721328in}{3.097019in}}%
\pgfpathclose%
\pgfusepath{stroke,fill}%
\end{pgfscope}%
\begin{pgfscope}%
\pgfpathrectangle{\pgfqpoint{0.100000in}{0.212622in}}{\pgfqpoint{3.696000in}{3.696000in}}%
\pgfusepath{clip}%
\pgfsetbuttcap%
\pgfsetroundjoin%
\definecolor{currentfill}{rgb}{0.121569,0.466667,0.705882}%
\pgfsetfillcolor{currentfill}%
\pgfsetfillopacity{0.604311}%
\pgfsetlinewidth{1.003750pt}%
\definecolor{currentstroke}{rgb}{0.121569,0.466667,0.705882}%
\pgfsetstrokecolor{currentstroke}%
\pgfsetstrokeopacity{0.604311}%
\pgfsetdash{}{0pt}%
\pgfpathmoveto{\pgfqpoint{1.721328in}{3.097019in}}%
\pgfpathcurveto{\pgfqpoint{1.729564in}{3.097019in}}{\pgfqpoint{1.737465in}{3.100291in}}{\pgfqpoint{1.743288in}{3.106115in}}%
\pgfpathcurveto{\pgfqpoint{1.749112in}{3.111939in}}{\pgfqpoint{1.752385in}{3.119839in}}{\pgfqpoint{1.752385in}{3.128075in}}%
\pgfpathcurveto{\pgfqpoint{1.752385in}{3.136312in}}{\pgfqpoint{1.749112in}{3.144212in}}{\pgfqpoint{1.743288in}{3.150036in}}%
\pgfpathcurveto{\pgfqpoint{1.737465in}{3.155860in}}{\pgfqpoint{1.729564in}{3.159132in}}{\pgfqpoint{1.721328in}{3.159132in}}%
\pgfpathcurveto{\pgfqpoint{1.713092in}{3.159132in}}{\pgfqpoint{1.705192in}{3.155860in}}{\pgfqpoint{1.699368in}{3.150036in}}%
\pgfpathcurveto{\pgfqpoint{1.693544in}{3.144212in}}{\pgfqpoint{1.690272in}{3.136312in}}{\pgfqpoint{1.690272in}{3.128075in}}%
\pgfpathcurveto{\pgfqpoint{1.690272in}{3.119839in}}{\pgfqpoint{1.693544in}{3.111939in}}{\pgfqpoint{1.699368in}{3.106115in}}%
\pgfpathcurveto{\pgfqpoint{1.705192in}{3.100291in}}{\pgfqpoint{1.713092in}{3.097019in}}{\pgfqpoint{1.721328in}{3.097019in}}%
\pgfpathclose%
\pgfusepath{stroke,fill}%
\end{pgfscope}%
\begin{pgfscope}%
\pgfpathrectangle{\pgfqpoint{0.100000in}{0.212622in}}{\pgfqpoint{3.696000in}{3.696000in}}%
\pgfusepath{clip}%
\pgfsetbuttcap%
\pgfsetroundjoin%
\definecolor{currentfill}{rgb}{0.121569,0.466667,0.705882}%
\pgfsetfillcolor{currentfill}%
\pgfsetfillopacity{0.604311}%
\pgfsetlinewidth{1.003750pt}%
\definecolor{currentstroke}{rgb}{0.121569,0.466667,0.705882}%
\pgfsetstrokecolor{currentstroke}%
\pgfsetstrokeopacity{0.604311}%
\pgfsetdash{}{0pt}%
\pgfpathmoveto{\pgfqpoint{1.721328in}{3.097019in}}%
\pgfpathcurveto{\pgfqpoint{1.729564in}{3.097019in}}{\pgfqpoint{1.737465in}{3.100291in}}{\pgfqpoint{1.743288in}{3.106115in}}%
\pgfpathcurveto{\pgfqpoint{1.749112in}{3.111939in}}{\pgfqpoint{1.752385in}{3.119839in}}{\pgfqpoint{1.752385in}{3.128075in}}%
\pgfpathcurveto{\pgfqpoint{1.752385in}{3.136312in}}{\pgfqpoint{1.749112in}{3.144212in}}{\pgfqpoint{1.743288in}{3.150036in}}%
\pgfpathcurveto{\pgfqpoint{1.737465in}{3.155860in}}{\pgfqpoint{1.729564in}{3.159132in}}{\pgfqpoint{1.721328in}{3.159132in}}%
\pgfpathcurveto{\pgfqpoint{1.713092in}{3.159132in}}{\pgfqpoint{1.705192in}{3.155860in}}{\pgfqpoint{1.699368in}{3.150036in}}%
\pgfpathcurveto{\pgfqpoint{1.693544in}{3.144212in}}{\pgfqpoint{1.690272in}{3.136312in}}{\pgfqpoint{1.690272in}{3.128075in}}%
\pgfpathcurveto{\pgfqpoint{1.690272in}{3.119839in}}{\pgfqpoint{1.693544in}{3.111939in}}{\pgfqpoint{1.699368in}{3.106115in}}%
\pgfpathcurveto{\pgfqpoint{1.705192in}{3.100291in}}{\pgfqpoint{1.713092in}{3.097019in}}{\pgfqpoint{1.721328in}{3.097019in}}%
\pgfpathclose%
\pgfusepath{stroke,fill}%
\end{pgfscope}%
\begin{pgfscope}%
\pgfpathrectangle{\pgfqpoint{0.100000in}{0.212622in}}{\pgfqpoint{3.696000in}{3.696000in}}%
\pgfusepath{clip}%
\pgfsetbuttcap%
\pgfsetroundjoin%
\definecolor{currentfill}{rgb}{0.121569,0.466667,0.705882}%
\pgfsetfillcolor{currentfill}%
\pgfsetfillopacity{0.604311}%
\pgfsetlinewidth{1.003750pt}%
\definecolor{currentstroke}{rgb}{0.121569,0.466667,0.705882}%
\pgfsetstrokecolor{currentstroke}%
\pgfsetstrokeopacity{0.604311}%
\pgfsetdash{}{0pt}%
\pgfpathmoveto{\pgfqpoint{1.721328in}{3.097019in}}%
\pgfpathcurveto{\pgfqpoint{1.729564in}{3.097019in}}{\pgfqpoint{1.737465in}{3.100291in}}{\pgfqpoint{1.743288in}{3.106115in}}%
\pgfpathcurveto{\pgfqpoint{1.749112in}{3.111939in}}{\pgfqpoint{1.752385in}{3.119839in}}{\pgfqpoint{1.752385in}{3.128075in}}%
\pgfpathcurveto{\pgfqpoint{1.752385in}{3.136312in}}{\pgfqpoint{1.749112in}{3.144212in}}{\pgfqpoint{1.743288in}{3.150036in}}%
\pgfpathcurveto{\pgfqpoint{1.737465in}{3.155860in}}{\pgfqpoint{1.729564in}{3.159132in}}{\pgfqpoint{1.721328in}{3.159132in}}%
\pgfpathcurveto{\pgfqpoint{1.713092in}{3.159132in}}{\pgfqpoint{1.705192in}{3.155860in}}{\pgfqpoint{1.699368in}{3.150036in}}%
\pgfpathcurveto{\pgfqpoint{1.693544in}{3.144212in}}{\pgfqpoint{1.690272in}{3.136312in}}{\pgfqpoint{1.690272in}{3.128075in}}%
\pgfpathcurveto{\pgfqpoint{1.690272in}{3.119839in}}{\pgfqpoint{1.693544in}{3.111939in}}{\pgfqpoint{1.699368in}{3.106115in}}%
\pgfpathcurveto{\pgfqpoint{1.705192in}{3.100291in}}{\pgfqpoint{1.713092in}{3.097019in}}{\pgfqpoint{1.721328in}{3.097019in}}%
\pgfpathclose%
\pgfusepath{stroke,fill}%
\end{pgfscope}%
\begin{pgfscope}%
\pgfpathrectangle{\pgfqpoint{0.100000in}{0.212622in}}{\pgfqpoint{3.696000in}{3.696000in}}%
\pgfusepath{clip}%
\pgfsetbuttcap%
\pgfsetroundjoin%
\definecolor{currentfill}{rgb}{0.121569,0.466667,0.705882}%
\pgfsetfillcolor{currentfill}%
\pgfsetfillopacity{0.604311}%
\pgfsetlinewidth{1.003750pt}%
\definecolor{currentstroke}{rgb}{0.121569,0.466667,0.705882}%
\pgfsetstrokecolor{currentstroke}%
\pgfsetstrokeopacity{0.604311}%
\pgfsetdash{}{0pt}%
\pgfpathmoveto{\pgfqpoint{1.721328in}{3.097019in}}%
\pgfpathcurveto{\pgfqpoint{1.729564in}{3.097019in}}{\pgfqpoint{1.737465in}{3.100291in}}{\pgfqpoint{1.743288in}{3.106115in}}%
\pgfpathcurveto{\pgfqpoint{1.749112in}{3.111939in}}{\pgfqpoint{1.752385in}{3.119839in}}{\pgfqpoint{1.752385in}{3.128075in}}%
\pgfpathcurveto{\pgfqpoint{1.752385in}{3.136312in}}{\pgfqpoint{1.749112in}{3.144212in}}{\pgfqpoint{1.743288in}{3.150036in}}%
\pgfpathcurveto{\pgfqpoint{1.737465in}{3.155860in}}{\pgfqpoint{1.729564in}{3.159132in}}{\pgfqpoint{1.721328in}{3.159132in}}%
\pgfpathcurveto{\pgfqpoint{1.713092in}{3.159132in}}{\pgfqpoint{1.705192in}{3.155860in}}{\pgfqpoint{1.699368in}{3.150036in}}%
\pgfpathcurveto{\pgfqpoint{1.693544in}{3.144212in}}{\pgfqpoint{1.690272in}{3.136312in}}{\pgfqpoint{1.690272in}{3.128075in}}%
\pgfpathcurveto{\pgfqpoint{1.690272in}{3.119839in}}{\pgfqpoint{1.693544in}{3.111939in}}{\pgfqpoint{1.699368in}{3.106115in}}%
\pgfpathcurveto{\pgfqpoint{1.705192in}{3.100291in}}{\pgfqpoint{1.713092in}{3.097019in}}{\pgfqpoint{1.721328in}{3.097019in}}%
\pgfpathclose%
\pgfusepath{stroke,fill}%
\end{pgfscope}%
\begin{pgfscope}%
\pgfpathrectangle{\pgfqpoint{0.100000in}{0.212622in}}{\pgfqpoint{3.696000in}{3.696000in}}%
\pgfusepath{clip}%
\pgfsetbuttcap%
\pgfsetroundjoin%
\definecolor{currentfill}{rgb}{0.121569,0.466667,0.705882}%
\pgfsetfillcolor{currentfill}%
\pgfsetfillopacity{0.604311}%
\pgfsetlinewidth{1.003750pt}%
\definecolor{currentstroke}{rgb}{0.121569,0.466667,0.705882}%
\pgfsetstrokecolor{currentstroke}%
\pgfsetstrokeopacity{0.604311}%
\pgfsetdash{}{0pt}%
\pgfpathmoveto{\pgfqpoint{1.721328in}{3.097019in}}%
\pgfpathcurveto{\pgfqpoint{1.729564in}{3.097019in}}{\pgfqpoint{1.737465in}{3.100291in}}{\pgfqpoint{1.743288in}{3.106115in}}%
\pgfpathcurveto{\pgfqpoint{1.749112in}{3.111939in}}{\pgfqpoint{1.752385in}{3.119839in}}{\pgfqpoint{1.752385in}{3.128075in}}%
\pgfpathcurveto{\pgfqpoint{1.752385in}{3.136312in}}{\pgfqpoint{1.749112in}{3.144212in}}{\pgfqpoint{1.743288in}{3.150036in}}%
\pgfpathcurveto{\pgfqpoint{1.737465in}{3.155860in}}{\pgfqpoint{1.729564in}{3.159132in}}{\pgfqpoint{1.721328in}{3.159132in}}%
\pgfpathcurveto{\pgfqpoint{1.713092in}{3.159132in}}{\pgfqpoint{1.705192in}{3.155860in}}{\pgfqpoint{1.699368in}{3.150036in}}%
\pgfpathcurveto{\pgfqpoint{1.693544in}{3.144212in}}{\pgfqpoint{1.690272in}{3.136312in}}{\pgfqpoint{1.690272in}{3.128075in}}%
\pgfpathcurveto{\pgfqpoint{1.690272in}{3.119839in}}{\pgfqpoint{1.693544in}{3.111939in}}{\pgfqpoint{1.699368in}{3.106115in}}%
\pgfpathcurveto{\pgfqpoint{1.705192in}{3.100291in}}{\pgfqpoint{1.713092in}{3.097019in}}{\pgfqpoint{1.721328in}{3.097019in}}%
\pgfpathclose%
\pgfusepath{stroke,fill}%
\end{pgfscope}%
\begin{pgfscope}%
\pgfpathrectangle{\pgfqpoint{0.100000in}{0.212622in}}{\pgfqpoint{3.696000in}{3.696000in}}%
\pgfusepath{clip}%
\pgfsetbuttcap%
\pgfsetroundjoin%
\definecolor{currentfill}{rgb}{0.121569,0.466667,0.705882}%
\pgfsetfillcolor{currentfill}%
\pgfsetfillopacity{0.604311}%
\pgfsetlinewidth{1.003750pt}%
\definecolor{currentstroke}{rgb}{0.121569,0.466667,0.705882}%
\pgfsetstrokecolor{currentstroke}%
\pgfsetstrokeopacity{0.604311}%
\pgfsetdash{}{0pt}%
\pgfpathmoveto{\pgfqpoint{1.721328in}{3.097019in}}%
\pgfpathcurveto{\pgfqpoint{1.729564in}{3.097019in}}{\pgfqpoint{1.737465in}{3.100291in}}{\pgfqpoint{1.743288in}{3.106115in}}%
\pgfpathcurveto{\pgfqpoint{1.749112in}{3.111939in}}{\pgfqpoint{1.752385in}{3.119839in}}{\pgfqpoint{1.752385in}{3.128075in}}%
\pgfpathcurveto{\pgfqpoint{1.752385in}{3.136312in}}{\pgfqpoint{1.749112in}{3.144212in}}{\pgfqpoint{1.743288in}{3.150036in}}%
\pgfpathcurveto{\pgfqpoint{1.737465in}{3.155860in}}{\pgfqpoint{1.729564in}{3.159132in}}{\pgfqpoint{1.721328in}{3.159132in}}%
\pgfpathcurveto{\pgfqpoint{1.713092in}{3.159132in}}{\pgfqpoint{1.705192in}{3.155860in}}{\pgfqpoint{1.699368in}{3.150036in}}%
\pgfpathcurveto{\pgfqpoint{1.693544in}{3.144212in}}{\pgfqpoint{1.690272in}{3.136312in}}{\pgfqpoint{1.690272in}{3.128075in}}%
\pgfpathcurveto{\pgfqpoint{1.690272in}{3.119839in}}{\pgfqpoint{1.693544in}{3.111939in}}{\pgfqpoint{1.699368in}{3.106115in}}%
\pgfpathcurveto{\pgfqpoint{1.705192in}{3.100291in}}{\pgfqpoint{1.713092in}{3.097019in}}{\pgfqpoint{1.721328in}{3.097019in}}%
\pgfpathclose%
\pgfusepath{stroke,fill}%
\end{pgfscope}%
\begin{pgfscope}%
\pgfpathrectangle{\pgfqpoint{0.100000in}{0.212622in}}{\pgfqpoint{3.696000in}{3.696000in}}%
\pgfusepath{clip}%
\pgfsetbuttcap%
\pgfsetroundjoin%
\definecolor{currentfill}{rgb}{0.121569,0.466667,0.705882}%
\pgfsetfillcolor{currentfill}%
\pgfsetfillopacity{0.604311}%
\pgfsetlinewidth{1.003750pt}%
\definecolor{currentstroke}{rgb}{0.121569,0.466667,0.705882}%
\pgfsetstrokecolor{currentstroke}%
\pgfsetstrokeopacity{0.604311}%
\pgfsetdash{}{0pt}%
\pgfpathmoveto{\pgfqpoint{1.721328in}{3.097019in}}%
\pgfpathcurveto{\pgfqpoint{1.729564in}{3.097019in}}{\pgfqpoint{1.737465in}{3.100291in}}{\pgfqpoint{1.743288in}{3.106115in}}%
\pgfpathcurveto{\pgfqpoint{1.749112in}{3.111939in}}{\pgfqpoint{1.752385in}{3.119839in}}{\pgfqpoint{1.752385in}{3.128075in}}%
\pgfpathcurveto{\pgfqpoint{1.752385in}{3.136312in}}{\pgfqpoint{1.749112in}{3.144212in}}{\pgfqpoint{1.743288in}{3.150036in}}%
\pgfpathcurveto{\pgfqpoint{1.737465in}{3.155860in}}{\pgfqpoint{1.729564in}{3.159132in}}{\pgfqpoint{1.721328in}{3.159132in}}%
\pgfpathcurveto{\pgfqpoint{1.713092in}{3.159132in}}{\pgfqpoint{1.705192in}{3.155860in}}{\pgfqpoint{1.699368in}{3.150036in}}%
\pgfpathcurveto{\pgfqpoint{1.693544in}{3.144212in}}{\pgfqpoint{1.690272in}{3.136312in}}{\pgfqpoint{1.690272in}{3.128075in}}%
\pgfpathcurveto{\pgfqpoint{1.690272in}{3.119839in}}{\pgfqpoint{1.693544in}{3.111939in}}{\pgfqpoint{1.699368in}{3.106115in}}%
\pgfpathcurveto{\pgfqpoint{1.705192in}{3.100291in}}{\pgfqpoint{1.713092in}{3.097019in}}{\pgfqpoint{1.721328in}{3.097019in}}%
\pgfpathclose%
\pgfusepath{stroke,fill}%
\end{pgfscope}%
\begin{pgfscope}%
\pgfpathrectangle{\pgfqpoint{0.100000in}{0.212622in}}{\pgfqpoint{3.696000in}{3.696000in}}%
\pgfusepath{clip}%
\pgfsetbuttcap%
\pgfsetroundjoin%
\definecolor{currentfill}{rgb}{0.121569,0.466667,0.705882}%
\pgfsetfillcolor{currentfill}%
\pgfsetfillopacity{0.604311}%
\pgfsetlinewidth{1.003750pt}%
\definecolor{currentstroke}{rgb}{0.121569,0.466667,0.705882}%
\pgfsetstrokecolor{currentstroke}%
\pgfsetstrokeopacity{0.604311}%
\pgfsetdash{}{0pt}%
\pgfpathmoveto{\pgfqpoint{1.721328in}{3.097019in}}%
\pgfpathcurveto{\pgfqpoint{1.729564in}{3.097019in}}{\pgfqpoint{1.737465in}{3.100291in}}{\pgfqpoint{1.743288in}{3.106115in}}%
\pgfpathcurveto{\pgfqpoint{1.749112in}{3.111939in}}{\pgfqpoint{1.752385in}{3.119839in}}{\pgfqpoint{1.752385in}{3.128075in}}%
\pgfpathcurveto{\pgfqpoint{1.752385in}{3.136312in}}{\pgfqpoint{1.749112in}{3.144212in}}{\pgfqpoint{1.743288in}{3.150036in}}%
\pgfpathcurveto{\pgfqpoint{1.737465in}{3.155860in}}{\pgfqpoint{1.729564in}{3.159132in}}{\pgfqpoint{1.721328in}{3.159132in}}%
\pgfpathcurveto{\pgfqpoint{1.713092in}{3.159132in}}{\pgfqpoint{1.705192in}{3.155860in}}{\pgfqpoint{1.699368in}{3.150036in}}%
\pgfpathcurveto{\pgfqpoint{1.693544in}{3.144212in}}{\pgfqpoint{1.690272in}{3.136312in}}{\pgfqpoint{1.690272in}{3.128075in}}%
\pgfpathcurveto{\pgfqpoint{1.690272in}{3.119839in}}{\pgfqpoint{1.693544in}{3.111939in}}{\pgfqpoint{1.699368in}{3.106115in}}%
\pgfpathcurveto{\pgfqpoint{1.705192in}{3.100291in}}{\pgfqpoint{1.713092in}{3.097019in}}{\pgfqpoint{1.721328in}{3.097019in}}%
\pgfpathclose%
\pgfusepath{stroke,fill}%
\end{pgfscope}%
\begin{pgfscope}%
\pgfpathrectangle{\pgfqpoint{0.100000in}{0.212622in}}{\pgfqpoint{3.696000in}{3.696000in}}%
\pgfusepath{clip}%
\pgfsetbuttcap%
\pgfsetroundjoin%
\definecolor{currentfill}{rgb}{0.121569,0.466667,0.705882}%
\pgfsetfillcolor{currentfill}%
\pgfsetfillopacity{0.604311}%
\pgfsetlinewidth{1.003750pt}%
\definecolor{currentstroke}{rgb}{0.121569,0.466667,0.705882}%
\pgfsetstrokecolor{currentstroke}%
\pgfsetstrokeopacity{0.604311}%
\pgfsetdash{}{0pt}%
\pgfpathmoveto{\pgfqpoint{1.721328in}{3.097019in}}%
\pgfpathcurveto{\pgfqpoint{1.729565in}{3.097019in}}{\pgfqpoint{1.737465in}{3.100291in}}{\pgfqpoint{1.743288in}{3.106115in}}%
\pgfpathcurveto{\pgfqpoint{1.749112in}{3.111939in}}{\pgfqpoint{1.752385in}{3.119839in}}{\pgfqpoint{1.752385in}{3.128075in}}%
\pgfpathcurveto{\pgfqpoint{1.752385in}{3.136312in}}{\pgfqpoint{1.749112in}{3.144212in}}{\pgfqpoint{1.743288in}{3.150036in}}%
\pgfpathcurveto{\pgfqpoint{1.737465in}{3.155860in}}{\pgfqpoint{1.729565in}{3.159132in}}{\pgfqpoint{1.721328in}{3.159132in}}%
\pgfpathcurveto{\pgfqpoint{1.713092in}{3.159132in}}{\pgfqpoint{1.705192in}{3.155860in}}{\pgfqpoint{1.699368in}{3.150036in}}%
\pgfpathcurveto{\pgfqpoint{1.693544in}{3.144212in}}{\pgfqpoint{1.690272in}{3.136312in}}{\pgfqpoint{1.690272in}{3.128075in}}%
\pgfpathcurveto{\pgfqpoint{1.690272in}{3.119839in}}{\pgfqpoint{1.693544in}{3.111939in}}{\pgfqpoint{1.699368in}{3.106115in}}%
\pgfpathcurveto{\pgfqpoint{1.705192in}{3.100291in}}{\pgfqpoint{1.713092in}{3.097019in}}{\pgfqpoint{1.721328in}{3.097019in}}%
\pgfpathclose%
\pgfusepath{stroke,fill}%
\end{pgfscope}%
\begin{pgfscope}%
\pgfpathrectangle{\pgfqpoint{0.100000in}{0.212622in}}{\pgfqpoint{3.696000in}{3.696000in}}%
\pgfusepath{clip}%
\pgfsetbuttcap%
\pgfsetroundjoin%
\definecolor{currentfill}{rgb}{0.121569,0.466667,0.705882}%
\pgfsetfillcolor{currentfill}%
\pgfsetfillopacity{0.604311}%
\pgfsetlinewidth{1.003750pt}%
\definecolor{currentstroke}{rgb}{0.121569,0.466667,0.705882}%
\pgfsetstrokecolor{currentstroke}%
\pgfsetstrokeopacity{0.604311}%
\pgfsetdash{}{0pt}%
\pgfpathmoveto{\pgfqpoint{1.721328in}{3.097019in}}%
\pgfpathcurveto{\pgfqpoint{1.729565in}{3.097019in}}{\pgfqpoint{1.737465in}{3.100291in}}{\pgfqpoint{1.743289in}{3.106115in}}%
\pgfpathcurveto{\pgfqpoint{1.749112in}{3.111939in}}{\pgfqpoint{1.752385in}{3.119839in}}{\pgfqpoint{1.752385in}{3.128075in}}%
\pgfpathcurveto{\pgfqpoint{1.752385in}{3.136312in}}{\pgfqpoint{1.749112in}{3.144212in}}{\pgfqpoint{1.743289in}{3.150036in}}%
\pgfpathcurveto{\pgfqpoint{1.737465in}{3.155860in}}{\pgfqpoint{1.729565in}{3.159132in}}{\pgfqpoint{1.721328in}{3.159132in}}%
\pgfpathcurveto{\pgfqpoint{1.713092in}{3.159132in}}{\pgfqpoint{1.705192in}{3.155860in}}{\pgfqpoint{1.699368in}{3.150036in}}%
\pgfpathcurveto{\pgfqpoint{1.693544in}{3.144212in}}{\pgfqpoint{1.690272in}{3.136312in}}{\pgfqpoint{1.690272in}{3.128075in}}%
\pgfpathcurveto{\pgfqpoint{1.690272in}{3.119839in}}{\pgfqpoint{1.693544in}{3.111939in}}{\pgfqpoint{1.699368in}{3.106115in}}%
\pgfpathcurveto{\pgfqpoint{1.705192in}{3.100291in}}{\pgfqpoint{1.713092in}{3.097019in}}{\pgfqpoint{1.721328in}{3.097019in}}%
\pgfpathclose%
\pgfusepath{stroke,fill}%
\end{pgfscope}%
\begin{pgfscope}%
\pgfpathrectangle{\pgfqpoint{0.100000in}{0.212622in}}{\pgfqpoint{3.696000in}{3.696000in}}%
\pgfusepath{clip}%
\pgfsetbuttcap%
\pgfsetroundjoin%
\definecolor{currentfill}{rgb}{0.121569,0.466667,0.705882}%
\pgfsetfillcolor{currentfill}%
\pgfsetfillopacity{0.604311}%
\pgfsetlinewidth{1.003750pt}%
\definecolor{currentstroke}{rgb}{0.121569,0.466667,0.705882}%
\pgfsetstrokecolor{currentstroke}%
\pgfsetstrokeopacity{0.604311}%
\pgfsetdash{}{0pt}%
\pgfpathmoveto{\pgfqpoint{1.721328in}{3.097019in}}%
\pgfpathcurveto{\pgfqpoint{1.729565in}{3.097019in}}{\pgfqpoint{1.737465in}{3.100291in}}{\pgfqpoint{1.743289in}{3.106115in}}%
\pgfpathcurveto{\pgfqpoint{1.749113in}{3.111939in}}{\pgfqpoint{1.752385in}{3.119839in}}{\pgfqpoint{1.752385in}{3.128075in}}%
\pgfpathcurveto{\pgfqpoint{1.752385in}{3.136312in}}{\pgfqpoint{1.749113in}{3.144212in}}{\pgfqpoint{1.743289in}{3.150036in}}%
\pgfpathcurveto{\pgfqpoint{1.737465in}{3.155860in}}{\pgfqpoint{1.729565in}{3.159132in}}{\pgfqpoint{1.721328in}{3.159132in}}%
\pgfpathcurveto{\pgfqpoint{1.713092in}{3.159132in}}{\pgfqpoint{1.705192in}{3.155860in}}{\pgfqpoint{1.699368in}{3.150036in}}%
\pgfpathcurveto{\pgfqpoint{1.693544in}{3.144212in}}{\pgfqpoint{1.690272in}{3.136312in}}{\pgfqpoint{1.690272in}{3.128075in}}%
\pgfpathcurveto{\pgfqpoint{1.690272in}{3.119839in}}{\pgfqpoint{1.693544in}{3.111939in}}{\pgfqpoint{1.699368in}{3.106115in}}%
\pgfpathcurveto{\pgfqpoint{1.705192in}{3.100291in}}{\pgfqpoint{1.713092in}{3.097019in}}{\pgfqpoint{1.721328in}{3.097019in}}%
\pgfpathclose%
\pgfusepath{stroke,fill}%
\end{pgfscope}%
\begin{pgfscope}%
\pgfpathrectangle{\pgfqpoint{0.100000in}{0.212622in}}{\pgfqpoint{3.696000in}{3.696000in}}%
\pgfusepath{clip}%
\pgfsetbuttcap%
\pgfsetroundjoin%
\definecolor{currentfill}{rgb}{0.121569,0.466667,0.705882}%
\pgfsetfillcolor{currentfill}%
\pgfsetfillopacity{0.604311}%
\pgfsetlinewidth{1.003750pt}%
\definecolor{currentstroke}{rgb}{0.121569,0.466667,0.705882}%
\pgfsetstrokecolor{currentstroke}%
\pgfsetstrokeopacity{0.604311}%
\pgfsetdash{}{0pt}%
\pgfpathmoveto{\pgfqpoint{1.721328in}{3.097019in}}%
\pgfpathcurveto{\pgfqpoint{1.729565in}{3.097019in}}{\pgfqpoint{1.737465in}{3.100291in}}{\pgfqpoint{1.743289in}{3.106115in}}%
\pgfpathcurveto{\pgfqpoint{1.749113in}{3.111939in}}{\pgfqpoint{1.752385in}{3.119839in}}{\pgfqpoint{1.752385in}{3.128075in}}%
\pgfpathcurveto{\pgfqpoint{1.752385in}{3.136312in}}{\pgfqpoint{1.749113in}{3.144212in}}{\pgfqpoint{1.743289in}{3.150036in}}%
\pgfpathcurveto{\pgfqpoint{1.737465in}{3.155860in}}{\pgfqpoint{1.729565in}{3.159132in}}{\pgfqpoint{1.721328in}{3.159132in}}%
\pgfpathcurveto{\pgfqpoint{1.713092in}{3.159132in}}{\pgfqpoint{1.705192in}{3.155860in}}{\pgfqpoint{1.699368in}{3.150036in}}%
\pgfpathcurveto{\pgfqpoint{1.693544in}{3.144212in}}{\pgfqpoint{1.690272in}{3.136312in}}{\pgfqpoint{1.690272in}{3.128075in}}%
\pgfpathcurveto{\pgfqpoint{1.690272in}{3.119839in}}{\pgfqpoint{1.693544in}{3.111939in}}{\pgfqpoint{1.699368in}{3.106115in}}%
\pgfpathcurveto{\pgfqpoint{1.705192in}{3.100291in}}{\pgfqpoint{1.713092in}{3.097019in}}{\pgfqpoint{1.721328in}{3.097019in}}%
\pgfpathclose%
\pgfusepath{stroke,fill}%
\end{pgfscope}%
\begin{pgfscope}%
\pgfpathrectangle{\pgfqpoint{0.100000in}{0.212622in}}{\pgfqpoint{3.696000in}{3.696000in}}%
\pgfusepath{clip}%
\pgfsetbuttcap%
\pgfsetroundjoin%
\definecolor{currentfill}{rgb}{0.121569,0.466667,0.705882}%
\pgfsetfillcolor{currentfill}%
\pgfsetfillopacity{0.604311}%
\pgfsetlinewidth{1.003750pt}%
\definecolor{currentstroke}{rgb}{0.121569,0.466667,0.705882}%
\pgfsetstrokecolor{currentstroke}%
\pgfsetstrokeopacity{0.604311}%
\pgfsetdash{}{0pt}%
\pgfpathmoveto{\pgfqpoint{1.721329in}{3.097019in}}%
\pgfpathcurveto{\pgfqpoint{1.729565in}{3.097019in}}{\pgfqpoint{1.737465in}{3.100291in}}{\pgfqpoint{1.743289in}{3.106115in}}%
\pgfpathcurveto{\pgfqpoint{1.749113in}{3.111939in}}{\pgfqpoint{1.752385in}{3.119839in}}{\pgfqpoint{1.752385in}{3.128075in}}%
\pgfpathcurveto{\pgfqpoint{1.752385in}{3.136312in}}{\pgfqpoint{1.749113in}{3.144212in}}{\pgfqpoint{1.743289in}{3.150036in}}%
\pgfpathcurveto{\pgfqpoint{1.737465in}{3.155860in}}{\pgfqpoint{1.729565in}{3.159132in}}{\pgfqpoint{1.721329in}{3.159132in}}%
\pgfpathcurveto{\pgfqpoint{1.713092in}{3.159132in}}{\pgfqpoint{1.705192in}{3.155860in}}{\pgfqpoint{1.699368in}{3.150036in}}%
\pgfpathcurveto{\pgfqpoint{1.693544in}{3.144212in}}{\pgfqpoint{1.690272in}{3.136312in}}{\pgfqpoint{1.690272in}{3.128075in}}%
\pgfpathcurveto{\pgfqpoint{1.690272in}{3.119839in}}{\pgfqpoint{1.693544in}{3.111939in}}{\pgfqpoint{1.699368in}{3.106115in}}%
\pgfpathcurveto{\pgfqpoint{1.705192in}{3.100291in}}{\pgfqpoint{1.713092in}{3.097019in}}{\pgfqpoint{1.721329in}{3.097019in}}%
\pgfpathclose%
\pgfusepath{stroke,fill}%
\end{pgfscope}%
\begin{pgfscope}%
\pgfpathrectangle{\pgfqpoint{0.100000in}{0.212622in}}{\pgfqpoint{3.696000in}{3.696000in}}%
\pgfusepath{clip}%
\pgfsetbuttcap%
\pgfsetroundjoin%
\definecolor{currentfill}{rgb}{0.121569,0.466667,0.705882}%
\pgfsetfillcolor{currentfill}%
\pgfsetfillopacity{0.604311}%
\pgfsetlinewidth{1.003750pt}%
\definecolor{currentstroke}{rgb}{0.121569,0.466667,0.705882}%
\pgfsetstrokecolor{currentstroke}%
\pgfsetstrokeopacity{0.604311}%
\pgfsetdash{}{0pt}%
\pgfpathmoveto{\pgfqpoint{1.721329in}{3.097019in}}%
\pgfpathcurveto{\pgfqpoint{1.729565in}{3.097019in}}{\pgfqpoint{1.737465in}{3.100291in}}{\pgfqpoint{1.743289in}{3.106115in}}%
\pgfpathcurveto{\pgfqpoint{1.749113in}{3.111939in}}{\pgfqpoint{1.752386in}{3.119839in}}{\pgfqpoint{1.752386in}{3.128075in}}%
\pgfpathcurveto{\pgfqpoint{1.752386in}{3.136312in}}{\pgfqpoint{1.749113in}{3.144212in}}{\pgfqpoint{1.743289in}{3.150036in}}%
\pgfpathcurveto{\pgfqpoint{1.737465in}{3.155860in}}{\pgfqpoint{1.729565in}{3.159132in}}{\pgfqpoint{1.721329in}{3.159132in}}%
\pgfpathcurveto{\pgfqpoint{1.713093in}{3.159132in}}{\pgfqpoint{1.705193in}{3.155860in}}{\pgfqpoint{1.699369in}{3.150036in}}%
\pgfpathcurveto{\pgfqpoint{1.693545in}{3.144212in}}{\pgfqpoint{1.690273in}{3.136312in}}{\pgfqpoint{1.690273in}{3.128075in}}%
\pgfpathcurveto{\pgfqpoint{1.690273in}{3.119839in}}{\pgfqpoint{1.693545in}{3.111939in}}{\pgfqpoint{1.699369in}{3.106115in}}%
\pgfpathcurveto{\pgfqpoint{1.705193in}{3.100291in}}{\pgfqpoint{1.713093in}{3.097019in}}{\pgfqpoint{1.721329in}{3.097019in}}%
\pgfpathclose%
\pgfusepath{stroke,fill}%
\end{pgfscope}%
\begin{pgfscope}%
\pgfpathrectangle{\pgfqpoint{0.100000in}{0.212622in}}{\pgfqpoint{3.696000in}{3.696000in}}%
\pgfusepath{clip}%
\pgfsetbuttcap%
\pgfsetroundjoin%
\definecolor{currentfill}{rgb}{0.121569,0.466667,0.705882}%
\pgfsetfillcolor{currentfill}%
\pgfsetfillopacity{0.604311}%
\pgfsetlinewidth{1.003750pt}%
\definecolor{currentstroke}{rgb}{0.121569,0.466667,0.705882}%
\pgfsetstrokecolor{currentstroke}%
\pgfsetstrokeopacity{0.604311}%
\pgfsetdash{}{0pt}%
\pgfpathmoveto{\pgfqpoint{1.721330in}{3.097019in}}%
\pgfpathcurveto{\pgfqpoint{1.729566in}{3.097019in}}{\pgfqpoint{1.737466in}{3.100291in}}{\pgfqpoint{1.743290in}{3.106115in}}%
\pgfpathcurveto{\pgfqpoint{1.749114in}{3.111939in}}{\pgfqpoint{1.752386in}{3.119839in}}{\pgfqpoint{1.752386in}{3.128075in}}%
\pgfpathcurveto{\pgfqpoint{1.752386in}{3.136311in}}{\pgfqpoint{1.749114in}{3.144212in}}{\pgfqpoint{1.743290in}{3.150035in}}%
\pgfpathcurveto{\pgfqpoint{1.737466in}{3.155859in}}{\pgfqpoint{1.729566in}{3.159132in}}{\pgfqpoint{1.721330in}{3.159132in}}%
\pgfpathcurveto{\pgfqpoint{1.713093in}{3.159132in}}{\pgfqpoint{1.705193in}{3.155859in}}{\pgfqpoint{1.699369in}{3.150035in}}%
\pgfpathcurveto{\pgfqpoint{1.693545in}{3.144212in}}{\pgfqpoint{1.690273in}{3.136311in}}{\pgfqpoint{1.690273in}{3.128075in}}%
\pgfpathcurveto{\pgfqpoint{1.690273in}{3.119839in}}{\pgfqpoint{1.693545in}{3.111939in}}{\pgfqpoint{1.699369in}{3.106115in}}%
\pgfpathcurveto{\pgfqpoint{1.705193in}{3.100291in}}{\pgfqpoint{1.713093in}{3.097019in}}{\pgfqpoint{1.721330in}{3.097019in}}%
\pgfpathclose%
\pgfusepath{stroke,fill}%
\end{pgfscope}%
\begin{pgfscope}%
\pgfpathrectangle{\pgfqpoint{0.100000in}{0.212622in}}{\pgfqpoint{3.696000in}{3.696000in}}%
\pgfusepath{clip}%
\pgfsetbuttcap%
\pgfsetroundjoin%
\definecolor{currentfill}{rgb}{0.121569,0.466667,0.705882}%
\pgfsetfillcolor{currentfill}%
\pgfsetfillopacity{0.604312}%
\pgfsetlinewidth{1.003750pt}%
\definecolor{currentstroke}{rgb}{0.121569,0.466667,0.705882}%
\pgfsetstrokecolor{currentstroke}%
\pgfsetstrokeopacity{0.604312}%
\pgfsetdash{}{0pt}%
\pgfpathmoveto{\pgfqpoint{1.721331in}{3.097019in}}%
\pgfpathcurveto{\pgfqpoint{1.729567in}{3.097019in}}{\pgfqpoint{1.737467in}{3.100291in}}{\pgfqpoint{1.743291in}{3.106115in}}%
\pgfpathcurveto{\pgfqpoint{1.749115in}{3.111939in}}{\pgfqpoint{1.752387in}{3.119839in}}{\pgfqpoint{1.752387in}{3.128075in}}%
\pgfpathcurveto{\pgfqpoint{1.752387in}{3.136311in}}{\pgfqpoint{1.749115in}{3.144211in}}{\pgfqpoint{1.743291in}{3.150035in}}%
\pgfpathcurveto{\pgfqpoint{1.737467in}{3.155859in}}{\pgfqpoint{1.729567in}{3.159132in}}{\pgfqpoint{1.721331in}{3.159132in}}%
\pgfpathcurveto{\pgfqpoint{1.713095in}{3.159132in}}{\pgfqpoint{1.705195in}{3.155859in}}{\pgfqpoint{1.699371in}{3.150035in}}%
\pgfpathcurveto{\pgfqpoint{1.693547in}{3.144211in}}{\pgfqpoint{1.690274in}{3.136311in}}{\pgfqpoint{1.690274in}{3.128075in}}%
\pgfpathcurveto{\pgfqpoint{1.690274in}{3.119839in}}{\pgfqpoint{1.693547in}{3.111939in}}{\pgfqpoint{1.699371in}{3.106115in}}%
\pgfpathcurveto{\pgfqpoint{1.705195in}{3.100291in}}{\pgfqpoint{1.713095in}{3.097019in}}{\pgfqpoint{1.721331in}{3.097019in}}%
\pgfpathclose%
\pgfusepath{stroke,fill}%
\end{pgfscope}%
\begin{pgfscope}%
\pgfpathrectangle{\pgfqpoint{0.100000in}{0.212622in}}{\pgfqpoint{3.696000in}{3.696000in}}%
\pgfusepath{clip}%
\pgfsetbuttcap%
\pgfsetroundjoin%
\definecolor{currentfill}{rgb}{0.121569,0.466667,0.705882}%
\pgfsetfillcolor{currentfill}%
\pgfsetfillopacity{0.604312}%
\pgfsetlinewidth{1.003750pt}%
\definecolor{currentstroke}{rgb}{0.121569,0.466667,0.705882}%
\pgfsetstrokecolor{currentstroke}%
\pgfsetstrokeopacity{0.604312}%
\pgfsetdash{}{0pt}%
\pgfpathmoveto{\pgfqpoint{1.721333in}{3.097018in}}%
\pgfpathcurveto{\pgfqpoint{1.729569in}{3.097018in}}{\pgfqpoint{1.737469in}{3.100291in}}{\pgfqpoint{1.743293in}{3.106114in}}%
\pgfpathcurveto{\pgfqpoint{1.749117in}{3.111938in}}{\pgfqpoint{1.752390in}{3.119838in}}{\pgfqpoint{1.752390in}{3.128075in}}%
\pgfpathcurveto{\pgfqpoint{1.752390in}{3.136311in}}{\pgfqpoint{1.749117in}{3.144211in}}{\pgfqpoint{1.743293in}{3.150035in}}%
\pgfpathcurveto{\pgfqpoint{1.737469in}{3.155859in}}{\pgfqpoint{1.729569in}{3.159131in}}{\pgfqpoint{1.721333in}{3.159131in}}%
\pgfpathcurveto{\pgfqpoint{1.713097in}{3.159131in}}{\pgfqpoint{1.705197in}{3.155859in}}{\pgfqpoint{1.699373in}{3.150035in}}%
\pgfpathcurveto{\pgfqpoint{1.693549in}{3.144211in}}{\pgfqpoint{1.690277in}{3.136311in}}{\pgfqpoint{1.690277in}{3.128075in}}%
\pgfpathcurveto{\pgfqpoint{1.690277in}{3.119838in}}{\pgfqpoint{1.693549in}{3.111938in}}{\pgfqpoint{1.699373in}{3.106114in}}%
\pgfpathcurveto{\pgfqpoint{1.705197in}{3.100291in}}{\pgfqpoint{1.713097in}{3.097018in}}{\pgfqpoint{1.721333in}{3.097018in}}%
\pgfpathclose%
\pgfusepath{stroke,fill}%
\end{pgfscope}%
\begin{pgfscope}%
\pgfpathrectangle{\pgfqpoint{0.100000in}{0.212622in}}{\pgfqpoint{3.696000in}{3.696000in}}%
\pgfusepath{clip}%
\pgfsetbuttcap%
\pgfsetroundjoin%
\definecolor{currentfill}{rgb}{0.121569,0.466667,0.705882}%
\pgfsetfillcolor{currentfill}%
\pgfsetfillopacity{0.604312}%
\pgfsetlinewidth{1.003750pt}%
\definecolor{currentstroke}{rgb}{0.121569,0.466667,0.705882}%
\pgfsetstrokecolor{currentstroke}%
\pgfsetstrokeopacity{0.604312}%
\pgfsetdash{}{0pt}%
\pgfpathmoveto{\pgfqpoint{1.721337in}{3.097018in}}%
\pgfpathcurveto{\pgfqpoint{1.729573in}{3.097018in}}{\pgfqpoint{1.737473in}{3.100290in}}{\pgfqpoint{1.743297in}{3.106114in}}%
\pgfpathcurveto{\pgfqpoint{1.749121in}{3.111938in}}{\pgfqpoint{1.752394in}{3.119838in}}{\pgfqpoint{1.752394in}{3.128075in}}%
\pgfpathcurveto{\pgfqpoint{1.752394in}{3.136311in}}{\pgfqpoint{1.749121in}{3.144211in}}{\pgfqpoint{1.743297in}{3.150035in}}%
\pgfpathcurveto{\pgfqpoint{1.737473in}{3.155859in}}{\pgfqpoint{1.729573in}{3.159131in}}{\pgfqpoint{1.721337in}{3.159131in}}%
\pgfpathcurveto{\pgfqpoint{1.713101in}{3.159131in}}{\pgfqpoint{1.705201in}{3.155859in}}{\pgfqpoint{1.699377in}{3.150035in}}%
\pgfpathcurveto{\pgfqpoint{1.693553in}{3.144211in}}{\pgfqpoint{1.690281in}{3.136311in}}{\pgfqpoint{1.690281in}{3.128075in}}%
\pgfpathcurveto{\pgfqpoint{1.690281in}{3.119838in}}{\pgfqpoint{1.693553in}{3.111938in}}{\pgfqpoint{1.699377in}{3.106114in}}%
\pgfpathcurveto{\pgfqpoint{1.705201in}{3.100290in}}{\pgfqpoint{1.713101in}{3.097018in}}{\pgfqpoint{1.721337in}{3.097018in}}%
\pgfpathclose%
\pgfusepath{stroke,fill}%
\end{pgfscope}%
\begin{pgfscope}%
\pgfpathrectangle{\pgfqpoint{0.100000in}{0.212622in}}{\pgfqpoint{3.696000in}{3.696000in}}%
\pgfusepath{clip}%
\pgfsetbuttcap%
\pgfsetroundjoin%
\definecolor{currentfill}{rgb}{0.121569,0.466667,0.705882}%
\pgfsetfillcolor{currentfill}%
\pgfsetfillopacity{0.604314}%
\pgfsetlinewidth{1.003750pt}%
\definecolor{currentstroke}{rgb}{0.121569,0.466667,0.705882}%
\pgfsetstrokecolor{currentstroke}%
\pgfsetstrokeopacity{0.604314}%
\pgfsetdash{}{0pt}%
\pgfpathmoveto{\pgfqpoint{1.721344in}{3.097016in}}%
\pgfpathcurveto{\pgfqpoint{1.729581in}{3.097016in}}{\pgfqpoint{1.737481in}{3.100289in}}{\pgfqpoint{1.743305in}{3.106113in}}%
\pgfpathcurveto{\pgfqpoint{1.749128in}{3.111937in}}{\pgfqpoint{1.752401in}{3.119837in}}{\pgfqpoint{1.752401in}{3.128073in}}%
\pgfpathcurveto{\pgfqpoint{1.752401in}{3.136309in}}{\pgfqpoint{1.749128in}{3.144209in}}{\pgfqpoint{1.743305in}{3.150033in}}%
\pgfpathcurveto{\pgfqpoint{1.737481in}{3.155857in}}{\pgfqpoint{1.729581in}{3.159129in}}{\pgfqpoint{1.721344in}{3.159129in}}%
\pgfpathcurveto{\pgfqpoint{1.713108in}{3.159129in}}{\pgfqpoint{1.705208in}{3.155857in}}{\pgfqpoint{1.699384in}{3.150033in}}%
\pgfpathcurveto{\pgfqpoint{1.693560in}{3.144209in}}{\pgfqpoint{1.690288in}{3.136309in}}{\pgfqpoint{1.690288in}{3.128073in}}%
\pgfpathcurveto{\pgfqpoint{1.690288in}{3.119837in}}{\pgfqpoint{1.693560in}{3.111937in}}{\pgfqpoint{1.699384in}{3.106113in}}%
\pgfpathcurveto{\pgfqpoint{1.705208in}{3.100289in}}{\pgfqpoint{1.713108in}{3.097016in}}{\pgfqpoint{1.721344in}{3.097016in}}%
\pgfpathclose%
\pgfusepath{stroke,fill}%
\end{pgfscope}%
\begin{pgfscope}%
\pgfpathrectangle{\pgfqpoint{0.100000in}{0.212622in}}{\pgfqpoint{3.696000in}{3.696000in}}%
\pgfusepath{clip}%
\pgfsetbuttcap%
\pgfsetroundjoin%
\definecolor{currentfill}{rgb}{0.121569,0.466667,0.705882}%
\pgfsetfillcolor{currentfill}%
\pgfsetfillopacity{0.604315}%
\pgfsetlinewidth{1.003750pt}%
\definecolor{currentstroke}{rgb}{0.121569,0.466667,0.705882}%
\pgfsetstrokecolor{currentstroke}%
\pgfsetstrokeopacity{0.604315}%
\pgfsetdash{}{0pt}%
\pgfpathmoveto{\pgfqpoint{1.721358in}{3.097015in}}%
\pgfpathcurveto{\pgfqpoint{1.729594in}{3.097015in}}{\pgfqpoint{1.737494in}{3.100287in}}{\pgfqpoint{1.743318in}{3.106111in}}%
\pgfpathcurveto{\pgfqpoint{1.749142in}{3.111935in}}{\pgfqpoint{1.752415in}{3.119835in}}{\pgfqpoint{1.752415in}{3.128071in}}%
\pgfpathcurveto{\pgfqpoint{1.752415in}{3.136308in}}{\pgfqpoint{1.749142in}{3.144208in}}{\pgfqpoint{1.743318in}{3.150032in}}%
\pgfpathcurveto{\pgfqpoint{1.737494in}{3.155856in}}{\pgfqpoint{1.729594in}{3.159128in}}{\pgfqpoint{1.721358in}{3.159128in}}%
\pgfpathcurveto{\pgfqpoint{1.713122in}{3.159128in}}{\pgfqpoint{1.705222in}{3.155856in}}{\pgfqpoint{1.699398in}{3.150032in}}%
\pgfpathcurveto{\pgfqpoint{1.693574in}{3.144208in}}{\pgfqpoint{1.690302in}{3.136308in}}{\pgfqpoint{1.690302in}{3.128071in}}%
\pgfpathcurveto{\pgfqpoint{1.690302in}{3.119835in}}{\pgfqpoint{1.693574in}{3.111935in}}{\pgfqpoint{1.699398in}{3.106111in}}%
\pgfpathcurveto{\pgfqpoint{1.705222in}{3.100287in}}{\pgfqpoint{1.713122in}{3.097015in}}{\pgfqpoint{1.721358in}{3.097015in}}%
\pgfpathclose%
\pgfusepath{stroke,fill}%
\end{pgfscope}%
\begin{pgfscope}%
\pgfpathrectangle{\pgfqpoint{0.100000in}{0.212622in}}{\pgfqpoint{3.696000in}{3.696000in}}%
\pgfusepath{clip}%
\pgfsetbuttcap%
\pgfsetroundjoin%
\definecolor{currentfill}{rgb}{0.121569,0.466667,0.705882}%
\pgfsetfillcolor{currentfill}%
\pgfsetfillopacity{0.604318}%
\pgfsetlinewidth{1.003750pt}%
\definecolor{currentstroke}{rgb}{0.121569,0.466667,0.705882}%
\pgfsetstrokecolor{currentstroke}%
\pgfsetstrokeopacity{0.604318}%
\pgfsetdash{}{0pt}%
\pgfpathmoveto{\pgfqpoint{1.721383in}{3.097010in}}%
\pgfpathcurveto{\pgfqpoint{1.729620in}{3.097010in}}{\pgfqpoint{1.737520in}{3.100282in}}{\pgfqpoint{1.743344in}{3.106106in}}%
\pgfpathcurveto{\pgfqpoint{1.749168in}{3.111930in}}{\pgfqpoint{1.752440in}{3.119830in}}{\pgfqpoint{1.752440in}{3.128067in}}%
\pgfpathcurveto{\pgfqpoint{1.752440in}{3.136303in}}{\pgfqpoint{1.749168in}{3.144203in}}{\pgfqpoint{1.743344in}{3.150027in}}%
\pgfpathcurveto{\pgfqpoint{1.737520in}{3.155851in}}{\pgfqpoint{1.729620in}{3.159123in}}{\pgfqpoint{1.721383in}{3.159123in}}%
\pgfpathcurveto{\pgfqpoint{1.713147in}{3.159123in}}{\pgfqpoint{1.705247in}{3.155851in}}{\pgfqpoint{1.699423in}{3.150027in}}%
\pgfpathcurveto{\pgfqpoint{1.693599in}{3.144203in}}{\pgfqpoint{1.690327in}{3.136303in}}{\pgfqpoint{1.690327in}{3.128067in}}%
\pgfpathcurveto{\pgfqpoint{1.690327in}{3.119830in}}{\pgfqpoint{1.693599in}{3.111930in}}{\pgfqpoint{1.699423in}{3.106106in}}%
\pgfpathcurveto{\pgfqpoint{1.705247in}{3.100282in}}{\pgfqpoint{1.713147in}{3.097010in}}{\pgfqpoint{1.721383in}{3.097010in}}%
\pgfpathclose%
\pgfusepath{stroke,fill}%
\end{pgfscope}%
\begin{pgfscope}%
\pgfpathrectangle{\pgfqpoint{0.100000in}{0.212622in}}{\pgfqpoint{3.696000in}{3.696000in}}%
\pgfusepath{clip}%
\pgfsetbuttcap%
\pgfsetroundjoin%
\definecolor{currentfill}{rgb}{0.121569,0.466667,0.705882}%
\pgfsetfillcolor{currentfill}%
\pgfsetfillopacity{0.604324}%
\pgfsetlinewidth{1.003750pt}%
\definecolor{currentstroke}{rgb}{0.121569,0.466667,0.705882}%
\pgfsetstrokecolor{currentstroke}%
\pgfsetstrokeopacity{0.604324}%
\pgfsetdash{}{0pt}%
\pgfpathmoveto{\pgfqpoint{1.721428in}{3.097006in}}%
\pgfpathcurveto{\pgfqpoint{1.729664in}{3.097006in}}{\pgfqpoint{1.737565in}{3.100278in}}{\pgfqpoint{1.743388in}{3.106102in}}%
\pgfpathcurveto{\pgfqpoint{1.749212in}{3.111926in}}{\pgfqpoint{1.752485in}{3.119826in}}{\pgfqpoint{1.752485in}{3.128062in}}%
\pgfpathcurveto{\pgfqpoint{1.752485in}{3.136298in}}{\pgfqpoint{1.749212in}{3.144198in}}{\pgfqpoint{1.743388in}{3.150022in}}%
\pgfpathcurveto{\pgfqpoint{1.737565in}{3.155846in}}{\pgfqpoint{1.729664in}{3.159119in}}{\pgfqpoint{1.721428in}{3.159119in}}%
\pgfpathcurveto{\pgfqpoint{1.713192in}{3.159119in}}{\pgfqpoint{1.705292in}{3.155846in}}{\pgfqpoint{1.699468in}{3.150022in}}%
\pgfpathcurveto{\pgfqpoint{1.693644in}{3.144198in}}{\pgfqpoint{1.690372in}{3.136298in}}{\pgfqpoint{1.690372in}{3.128062in}}%
\pgfpathcurveto{\pgfqpoint{1.690372in}{3.119826in}}{\pgfqpoint{1.693644in}{3.111926in}}{\pgfqpoint{1.699468in}{3.106102in}}%
\pgfpathcurveto{\pgfqpoint{1.705292in}{3.100278in}}{\pgfqpoint{1.713192in}{3.097006in}}{\pgfqpoint{1.721428in}{3.097006in}}%
\pgfpathclose%
\pgfusepath{stroke,fill}%
\end{pgfscope}%
\begin{pgfscope}%
\pgfpathrectangle{\pgfqpoint{0.100000in}{0.212622in}}{\pgfqpoint{3.696000in}{3.696000in}}%
\pgfusepath{clip}%
\pgfsetbuttcap%
\pgfsetroundjoin%
\definecolor{currentfill}{rgb}{0.121569,0.466667,0.705882}%
\pgfsetfillcolor{currentfill}%
\pgfsetfillopacity{0.604336}%
\pgfsetlinewidth{1.003750pt}%
\definecolor{currentstroke}{rgb}{0.121569,0.466667,0.705882}%
\pgfsetstrokecolor{currentstroke}%
\pgfsetstrokeopacity{0.604336}%
\pgfsetdash{}{0pt}%
\pgfpathmoveto{\pgfqpoint{1.721505in}{3.096985in}}%
\pgfpathcurveto{\pgfqpoint{1.729741in}{3.096985in}}{\pgfqpoint{1.737641in}{3.100258in}}{\pgfqpoint{1.743465in}{3.106081in}}%
\pgfpathcurveto{\pgfqpoint{1.749289in}{3.111905in}}{\pgfqpoint{1.752561in}{3.119805in}}{\pgfqpoint{1.752561in}{3.128042in}}%
\pgfpathcurveto{\pgfqpoint{1.752561in}{3.136278in}}{\pgfqpoint{1.749289in}{3.144178in}}{\pgfqpoint{1.743465in}{3.150002in}}%
\pgfpathcurveto{\pgfqpoint{1.737641in}{3.155826in}}{\pgfqpoint{1.729741in}{3.159098in}}{\pgfqpoint{1.721505in}{3.159098in}}%
\pgfpathcurveto{\pgfqpoint{1.713269in}{3.159098in}}{\pgfqpoint{1.705369in}{3.155826in}}{\pgfqpoint{1.699545in}{3.150002in}}%
\pgfpathcurveto{\pgfqpoint{1.693721in}{3.144178in}}{\pgfqpoint{1.690448in}{3.136278in}}{\pgfqpoint{1.690448in}{3.128042in}}%
\pgfpathcurveto{\pgfqpoint{1.690448in}{3.119805in}}{\pgfqpoint{1.693721in}{3.111905in}}{\pgfqpoint{1.699545in}{3.106081in}}%
\pgfpathcurveto{\pgfqpoint{1.705369in}{3.100258in}}{\pgfqpoint{1.713269in}{3.096985in}}{\pgfqpoint{1.721505in}{3.096985in}}%
\pgfpathclose%
\pgfusepath{stroke,fill}%
\end{pgfscope}%
\begin{pgfscope}%
\pgfpathrectangle{\pgfqpoint{0.100000in}{0.212622in}}{\pgfqpoint{3.696000in}{3.696000in}}%
\pgfusepath{clip}%
\pgfsetbuttcap%
\pgfsetroundjoin%
\definecolor{currentfill}{rgb}{0.121569,0.466667,0.705882}%
\pgfsetfillcolor{currentfill}%
\pgfsetfillopacity{0.604357}%
\pgfsetlinewidth{1.003750pt}%
\definecolor{currentstroke}{rgb}{0.121569,0.466667,0.705882}%
\pgfsetstrokecolor{currentstroke}%
\pgfsetstrokeopacity{0.604357}%
\pgfsetdash{}{0pt}%
\pgfpathmoveto{\pgfqpoint{1.721652in}{3.096972in}}%
\pgfpathcurveto{\pgfqpoint{1.729888in}{3.096972in}}{\pgfqpoint{1.737788in}{3.100244in}}{\pgfqpoint{1.743612in}{3.106068in}}%
\pgfpathcurveto{\pgfqpoint{1.749436in}{3.111892in}}{\pgfqpoint{1.752708in}{3.119792in}}{\pgfqpoint{1.752708in}{3.128028in}}%
\pgfpathcurveto{\pgfqpoint{1.752708in}{3.136265in}}{\pgfqpoint{1.749436in}{3.144165in}}{\pgfqpoint{1.743612in}{3.149989in}}%
\pgfpathcurveto{\pgfqpoint{1.737788in}{3.155813in}}{\pgfqpoint{1.729888in}{3.159085in}}{\pgfqpoint{1.721652in}{3.159085in}}%
\pgfpathcurveto{\pgfqpoint{1.713416in}{3.159085in}}{\pgfqpoint{1.705516in}{3.155813in}}{\pgfqpoint{1.699692in}{3.149989in}}%
\pgfpathcurveto{\pgfqpoint{1.693868in}{3.144165in}}{\pgfqpoint{1.690595in}{3.136265in}}{\pgfqpoint{1.690595in}{3.128028in}}%
\pgfpathcurveto{\pgfqpoint{1.690595in}{3.119792in}}{\pgfqpoint{1.693868in}{3.111892in}}{\pgfqpoint{1.699692in}{3.106068in}}%
\pgfpathcurveto{\pgfqpoint{1.705516in}{3.100244in}}{\pgfqpoint{1.713416in}{3.096972in}}{\pgfqpoint{1.721652in}{3.096972in}}%
\pgfpathclose%
\pgfusepath{stroke,fill}%
\end{pgfscope}%
\begin{pgfscope}%
\pgfpathrectangle{\pgfqpoint{0.100000in}{0.212622in}}{\pgfqpoint{3.696000in}{3.696000in}}%
\pgfusepath{clip}%
\pgfsetbuttcap%
\pgfsetroundjoin%
\definecolor{currentfill}{rgb}{0.121569,0.466667,0.705882}%
\pgfsetfillcolor{currentfill}%
\pgfsetfillopacity{0.604390}%
\pgfsetlinewidth{1.003750pt}%
\definecolor{currentstroke}{rgb}{0.121569,0.466667,0.705882}%
\pgfsetstrokecolor{currentstroke}%
\pgfsetstrokeopacity{0.604390}%
\pgfsetdash{}{0pt}%
\pgfpathmoveto{\pgfqpoint{1.720330in}{3.096828in}}%
\pgfpathcurveto{\pgfqpoint{1.728567in}{3.096828in}}{\pgfqpoint{1.736467in}{3.100100in}}{\pgfqpoint{1.742291in}{3.105924in}}%
\pgfpathcurveto{\pgfqpoint{1.748115in}{3.111748in}}{\pgfqpoint{1.751387in}{3.119648in}}{\pgfqpoint{1.751387in}{3.127884in}}%
\pgfpathcurveto{\pgfqpoint{1.751387in}{3.136120in}}{\pgfqpoint{1.748115in}{3.144020in}}{\pgfqpoint{1.742291in}{3.149844in}}%
\pgfpathcurveto{\pgfqpoint{1.736467in}{3.155668in}}{\pgfqpoint{1.728567in}{3.158941in}}{\pgfqpoint{1.720330in}{3.158941in}}%
\pgfpathcurveto{\pgfqpoint{1.712094in}{3.158941in}}{\pgfqpoint{1.704194in}{3.155668in}}{\pgfqpoint{1.698370in}{3.149844in}}%
\pgfpathcurveto{\pgfqpoint{1.692546in}{3.144020in}}{\pgfqpoint{1.689274in}{3.136120in}}{\pgfqpoint{1.689274in}{3.127884in}}%
\pgfpathcurveto{\pgfqpoint{1.689274in}{3.119648in}}{\pgfqpoint{1.692546in}{3.111748in}}{\pgfqpoint{1.698370in}{3.105924in}}%
\pgfpathcurveto{\pgfqpoint{1.704194in}{3.100100in}}{\pgfqpoint{1.712094in}{3.096828in}}{\pgfqpoint{1.720330in}{3.096828in}}%
\pgfpathclose%
\pgfusepath{stroke,fill}%
\end{pgfscope}%
\begin{pgfscope}%
\pgfpathrectangle{\pgfqpoint{0.100000in}{0.212622in}}{\pgfqpoint{3.696000in}{3.696000in}}%
\pgfusepath{clip}%
\pgfsetbuttcap%
\pgfsetroundjoin%
\definecolor{currentfill}{rgb}{0.121569,0.466667,0.705882}%
\pgfsetfillcolor{currentfill}%
\pgfsetfillopacity{0.604396}%
\pgfsetlinewidth{1.003750pt}%
\definecolor{currentstroke}{rgb}{0.121569,0.466667,0.705882}%
\pgfsetstrokecolor{currentstroke}%
\pgfsetstrokeopacity{0.604396}%
\pgfsetdash{}{0pt}%
\pgfpathmoveto{\pgfqpoint{1.721905in}{3.096904in}}%
\pgfpathcurveto{\pgfqpoint{1.730142in}{3.096904in}}{\pgfqpoint{1.738042in}{3.100176in}}{\pgfqpoint{1.743866in}{3.106000in}}%
\pgfpathcurveto{\pgfqpoint{1.749690in}{3.111824in}}{\pgfqpoint{1.752962in}{3.119724in}}{\pgfqpoint{1.752962in}{3.127960in}}%
\pgfpathcurveto{\pgfqpoint{1.752962in}{3.136197in}}{\pgfqpoint{1.749690in}{3.144097in}}{\pgfqpoint{1.743866in}{3.149921in}}%
\pgfpathcurveto{\pgfqpoint{1.738042in}{3.155745in}}{\pgfqpoint{1.730142in}{3.159017in}}{\pgfqpoint{1.721905in}{3.159017in}}%
\pgfpathcurveto{\pgfqpoint{1.713669in}{3.159017in}}{\pgfqpoint{1.705769in}{3.155745in}}{\pgfqpoint{1.699945in}{3.149921in}}%
\pgfpathcurveto{\pgfqpoint{1.694121in}{3.144097in}}{\pgfqpoint{1.690849in}{3.136197in}}{\pgfqpoint{1.690849in}{3.127960in}}%
\pgfpathcurveto{\pgfqpoint{1.690849in}{3.119724in}}{\pgfqpoint{1.694121in}{3.111824in}}{\pgfqpoint{1.699945in}{3.106000in}}%
\pgfpathcurveto{\pgfqpoint{1.705769in}{3.100176in}}{\pgfqpoint{1.713669in}{3.096904in}}{\pgfqpoint{1.721905in}{3.096904in}}%
\pgfpathclose%
\pgfusepath{stroke,fill}%
\end{pgfscope}%
\begin{pgfscope}%
\pgfpathrectangle{\pgfqpoint{0.100000in}{0.212622in}}{\pgfqpoint{3.696000in}{3.696000in}}%
\pgfusepath{clip}%
\pgfsetbuttcap%
\pgfsetroundjoin%
\definecolor{currentfill}{rgb}{0.121569,0.466667,0.705882}%
\pgfsetfillcolor{currentfill}%
\pgfsetfillopacity{0.604416}%
\pgfsetlinewidth{1.003750pt}%
\definecolor{currentstroke}{rgb}{0.121569,0.466667,0.705882}%
\pgfsetstrokecolor{currentstroke}%
\pgfsetstrokeopacity{0.604416}%
\pgfsetdash{}{0pt}%
\pgfpathmoveto{\pgfqpoint{3.264705in}{2.028948in}}%
\pgfpathcurveto{\pgfqpoint{3.272942in}{2.028948in}}{\pgfqpoint{3.280842in}{2.032220in}}{\pgfqpoint{3.286666in}{2.038044in}}%
\pgfpathcurveto{\pgfqpoint{3.292490in}{2.043868in}}{\pgfqpoint{3.295762in}{2.051768in}}{\pgfqpoint{3.295762in}{2.060004in}}%
\pgfpathcurveto{\pgfqpoint{3.295762in}{2.068240in}}{\pgfqpoint{3.292490in}{2.076140in}}{\pgfqpoint{3.286666in}{2.081964in}}%
\pgfpathcurveto{\pgfqpoint{3.280842in}{2.087788in}}{\pgfqpoint{3.272942in}{2.091061in}}{\pgfqpoint{3.264705in}{2.091061in}}%
\pgfpathcurveto{\pgfqpoint{3.256469in}{2.091061in}}{\pgfqpoint{3.248569in}{2.087788in}}{\pgfqpoint{3.242745in}{2.081964in}}%
\pgfpathcurveto{\pgfqpoint{3.236921in}{2.076140in}}{\pgfqpoint{3.233649in}{2.068240in}}{\pgfqpoint{3.233649in}{2.060004in}}%
\pgfpathcurveto{\pgfqpoint{3.233649in}{2.051768in}}{\pgfqpoint{3.236921in}{2.043868in}}{\pgfqpoint{3.242745in}{2.038044in}}%
\pgfpathcurveto{\pgfqpoint{3.248569in}{2.032220in}}{\pgfqpoint{3.256469in}{2.028948in}}{\pgfqpoint{3.264705in}{2.028948in}}%
\pgfpathclose%
\pgfusepath{stroke,fill}%
\end{pgfscope}%
\begin{pgfscope}%
\pgfpathrectangle{\pgfqpoint{0.100000in}{0.212622in}}{\pgfqpoint{3.696000in}{3.696000in}}%
\pgfusepath{clip}%
\pgfsetbuttcap%
\pgfsetroundjoin%
\definecolor{currentfill}{rgb}{0.121569,0.466667,0.705882}%
\pgfsetfillcolor{currentfill}%
\pgfsetfillopacity{0.604463}%
\pgfsetlinewidth{1.003750pt}%
\definecolor{currentstroke}{rgb}{0.121569,0.466667,0.705882}%
\pgfsetstrokecolor{currentstroke}%
\pgfsetstrokeopacity{0.604463}%
\pgfsetdash{}{0pt}%
\pgfpathmoveto{\pgfqpoint{1.722394in}{3.096853in}}%
\pgfpathcurveto{\pgfqpoint{1.730630in}{3.096853in}}{\pgfqpoint{1.738530in}{3.100125in}}{\pgfqpoint{1.744354in}{3.105949in}}%
\pgfpathcurveto{\pgfqpoint{1.750178in}{3.111773in}}{\pgfqpoint{1.753450in}{3.119673in}}{\pgfqpoint{1.753450in}{3.127909in}}%
\pgfpathcurveto{\pgfqpoint{1.753450in}{3.136145in}}{\pgfqpoint{1.750178in}{3.144045in}}{\pgfqpoint{1.744354in}{3.149869in}}%
\pgfpathcurveto{\pgfqpoint{1.738530in}{3.155693in}}{\pgfqpoint{1.730630in}{3.158966in}}{\pgfqpoint{1.722394in}{3.158966in}}%
\pgfpathcurveto{\pgfqpoint{1.714158in}{3.158966in}}{\pgfqpoint{1.706258in}{3.155693in}}{\pgfqpoint{1.700434in}{3.149869in}}%
\pgfpathcurveto{\pgfqpoint{1.694610in}{3.144045in}}{\pgfqpoint{1.691337in}{3.136145in}}{\pgfqpoint{1.691337in}{3.127909in}}%
\pgfpathcurveto{\pgfqpoint{1.691337in}{3.119673in}}{\pgfqpoint{1.694610in}{3.111773in}}{\pgfqpoint{1.700434in}{3.105949in}}%
\pgfpathcurveto{\pgfqpoint{1.706258in}{3.100125in}}{\pgfqpoint{1.714158in}{3.096853in}}{\pgfqpoint{1.722394in}{3.096853in}}%
\pgfpathclose%
\pgfusepath{stroke,fill}%
\end{pgfscope}%
\begin{pgfscope}%
\pgfpathrectangle{\pgfqpoint{0.100000in}{0.212622in}}{\pgfqpoint{3.696000in}{3.696000in}}%
\pgfusepath{clip}%
\pgfsetbuttcap%
\pgfsetroundjoin%
\definecolor{currentfill}{rgb}{0.121569,0.466667,0.705882}%
\pgfsetfillcolor{currentfill}%
\pgfsetfillopacity{0.604588}%
\pgfsetlinewidth{1.003750pt}%
\definecolor{currentstroke}{rgb}{0.121569,0.466667,0.705882}%
\pgfsetstrokecolor{currentstroke}%
\pgfsetstrokeopacity{0.604588}%
\pgfsetdash{}{0pt}%
\pgfpathmoveto{\pgfqpoint{1.719964in}{3.096523in}}%
\pgfpathcurveto{\pgfqpoint{1.728200in}{3.096523in}}{\pgfqpoint{1.736100in}{3.099795in}}{\pgfqpoint{1.741924in}{3.105619in}}%
\pgfpathcurveto{\pgfqpoint{1.747748in}{3.111443in}}{\pgfqpoint{1.751020in}{3.119343in}}{\pgfqpoint{1.751020in}{3.127579in}}%
\pgfpathcurveto{\pgfqpoint{1.751020in}{3.135816in}}{\pgfqpoint{1.747748in}{3.143716in}}{\pgfqpoint{1.741924in}{3.149540in}}%
\pgfpathcurveto{\pgfqpoint{1.736100in}{3.155364in}}{\pgfqpoint{1.728200in}{3.158636in}}{\pgfqpoint{1.719964in}{3.158636in}}%
\pgfpathcurveto{\pgfqpoint{1.711727in}{3.158636in}}{\pgfqpoint{1.703827in}{3.155364in}}{\pgfqpoint{1.698003in}{3.149540in}}%
\pgfpathcurveto{\pgfqpoint{1.692179in}{3.143716in}}{\pgfqpoint{1.688907in}{3.135816in}}{\pgfqpoint{1.688907in}{3.127579in}}%
\pgfpathcurveto{\pgfqpoint{1.688907in}{3.119343in}}{\pgfqpoint{1.692179in}{3.111443in}}{\pgfqpoint{1.698003in}{3.105619in}}%
\pgfpathcurveto{\pgfqpoint{1.703827in}{3.099795in}}{\pgfqpoint{1.711727in}{3.096523in}}{\pgfqpoint{1.719964in}{3.096523in}}%
\pgfpathclose%
\pgfusepath{stroke,fill}%
\end{pgfscope}%
\begin{pgfscope}%
\pgfpathrectangle{\pgfqpoint{0.100000in}{0.212622in}}{\pgfqpoint{3.696000in}{3.696000in}}%
\pgfusepath{clip}%
\pgfsetbuttcap%
\pgfsetroundjoin%
\definecolor{currentfill}{rgb}{0.121569,0.466667,0.705882}%
\pgfsetfillcolor{currentfill}%
\pgfsetfillopacity{0.604609}%
\pgfsetlinewidth{1.003750pt}%
\definecolor{currentstroke}{rgb}{0.121569,0.466667,0.705882}%
\pgfsetstrokecolor{currentstroke}%
\pgfsetstrokeopacity{0.604609}%
\pgfsetdash{}{0pt}%
\pgfpathmoveto{\pgfqpoint{1.723230in}{3.096726in}}%
\pgfpathcurveto{\pgfqpoint{1.731466in}{3.096726in}}{\pgfqpoint{1.739367in}{3.099998in}}{\pgfqpoint{1.745190in}{3.105822in}}%
\pgfpathcurveto{\pgfqpoint{1.751014in}{3.111646in}}{\pgfqpoint{1.754287in}{3.119546in}}{\pgfqpoint{1.754287in}{3.127782in}}%
\pgfpathcurveto{\pgfqpoint{1.754287in}{3.136018in}}{\pgfqpoint{1.751014in}{3.143918in}}{\pgfqpoint{1.745190in}{3.149742in}}%
\pgfpathcurveto{\pgfqpoint{1.739367in}{3.155566in}}{\pgfqpoint{1.731466in}{3.158839in}}{\pgfqpoint{1.723230in}{3.158839in}}%
\pgfpathcurveto{\pgfqpoint{1.714994in}{3.158839in}}{\pgfqpoint{1.707094in}{3.155566in}}{\pgfqpoint{1.701270in}{3.149742in}}%
\pgfpathcurveto{\pgfqpoint{1.695446in}{3.143918in}}{\pgfqpoint{1.692174in}{3.136018in}}{\pgfqpoint{1.692174in}{3.127782in}}%
\pgfpathcurveto{\pgfqpoint{1.692174in}{3.119546in}}{\pgfqpoint{1.695446in}{3.111646in}}{\pgfqpoint{1.701270in}{3.105822in}}%
\pgfpathcurveto{\pgfqpoint{1.707094in}{3.099998in}}{\pgfqpoint{1.714994in}{3.096726in}}{\pgfqpoint{1.723230in}{3.096726in}}%
\pgfpathclose%
\pgfusepath{stroke,fill}%
\end{pgfscope}%
\begin{pgfscope}%
\pgfpathrectangle{\pgfqpoint{0.100000in}{0.212622in}}{\pgfqpoint{3.696000in}{3.696000in}}%
\pgfusepath{clip}%
\pgfsetbuttcap%
\pgfsetroundjoin%
\definecolor{currentfill}{rgb}{0.121569,0.466667,0.705882}%
\pgfsetfillcolor{currentfill}%
\pgfsetfillopacity{0.604699}%
\pgfsetlinewidth{1.003750pt}%
\definecolor{currentstroke}{rgb}{0.121569,0.466667,0.705882}%
\pgfsetstrokecolor{currentstroke}%
\pgfsetstrokeopacity{0.604699}%
\pgfsetdash{}{0pt}%
\pgfpathmoveto{\pgfqpoint{1.719772in}{3.096363in}}%
\pgfpathcurveto{\pgfqpoint{1.728008in}{3.096363in}}{\pgfqpoint{1.735908in}{3.099635in}}{\pgfqpoint{1.741732in}{3.105459in}}%
\pgfpathcurveto{\pgfqpoint{1.747556in}{3.111283in}}{\pgfqpoint{1.750828in}{3.119183in}}{\pgfqpoint{1.750828in}{3.127419in}}%
\pgfpathcurveto{\pgfqpoint{1.750828in}{3.135656in}}{\pgfqpoint{1.747556in}{3.143556in}}{\pgfqpoint{1.741732in}{3.149380in}}%
\pgfpathcurveto{\pgfqpoint{1.735908in}{3.155204in}}{\pgfqpoint{1.728008in}{3.158476in}}{\pgfqpoint{1.719772in}{3.158476in}}%
\pgfpathcurveto{\pgfqpoint{1.711535in}{3.158476in}}{\pgfqpoint{1.703635in}{3.155204in}}{\pgfqpoint{1.697811in}{3.149380in}}%
\pgfpathcurveto{\pgfqpoint{1.691988in}{3.143556in}}{\pgfqpoint{1.688715in}{3.135656in}}{\pgfqpoint{1.688715in}{3.127419in}}%
\pgfpathcurveto{\pgfqpoint{1.688715in}{3.119183in}}{\pgfqpoint{1.691988in}{3.111283in}}{\pgfqpoint{1.697811in}{3.105459in}}%
\pgfpathcurveto{\pgfqpoint{1.703635in}{3.099635in}}{\pgfqpoint{1.711535in}{3.096363in}}{\pgfqpoint{1.719772in}{3.096363in}}%
\pgfpathclose%
\pgfusepath{stroke,fill}%
\end{pgfscope}%
\begin{pgfscope}%
\pgfpathrectangle{\pgfqpoint{0.100000in}{0.212622in}}{\pgfqpoint{3.696000in}{3.696000in}}%
\pgfusepath{clip}%
\pgfsetbuttcap%
\pgfsetroundjoin%
\definecolor{currentfill}{rgb}{0.121569,0.466667,0.705882}%
\pgfsetfillcolor{currentfill}%
\pgfsetfillopacity{0.604761}%
\pgfsetlinewidth{1.003750pt}%
\definecolor{currentstroke}{rgb}{0.121569,0.466667,0.705882}%
\pgfsetstrokecolor{currentstroke}%
\pgfsetstrokeopacity{0.604761}%
\pgfsetdash{}{0pt}%
\pgfpathmoveto{\pgfqpoint{1.719670in}{3.096278in}}%
\pgfpathcurveto{\pgfqpoint{1.727906in}{3.096278in}}{\pgfqpoint{1.735806in}{3.099550in}}{\pgfqpoint{1.741630in}{3.105374in}}%
\pgfpathcurveto{\pgfqpoint{1.747454in}{3.111198in}}{\pgfqpoint{1.750726in}{3.119098in}}{\pgfqpoint{1.750726in}{3.127334in}}%
\pgfpathcurveto{\pgfqpoint{1.750726in}{3.135571in}}{\pgfqpoint{1.747454in}{3.143471in}}{\pgfqpoint{1.741630in}{3.149295in}}%
\pgfpathcurveto{\pgfqpoint{1.735806in}{3.155119in}}{\pgfqpoint{1.727906in}{3.158391in}}{\pgfqpoint{1.719670in}{3.158391in}}%
\pgfpathcurveto{\pgfqpoint{1.711434in}{3.158391in}}{\pgfqpoint{1.703534in}{3.155119in}}{\pgfqpoint{1.697710in}{3.149295in}}%
\pgfpathcurveto{\pgfqpoint{1.691886in}{3.143471in}}{\pgfqpoint{1.688613in}{3.135571in}}{\pgfqpoint{1.688613in}{3.127334in}}%
\pgfpathcurveto{\pgfqpoint{1.688613in}{3.119098in}}{\pgfqpoint{1.691886in}{3.111198in}}{\pgfqpoint{1.697710in}{3.105374in}}%
\pgfpathcurveto{\pgfqpoint{1.703534in}{3.099550in}}{\pgfqpoint{1.711434in}{3.096278in}}{\pgfqpoint{1.719670in}{3.096278in}}%
\pgfpathclose%
\pgfusepath{stroke,fill}%
\end{pgfscope}%
\begin{pgfscope}%
\pgfpathrectangle{\pgfqpoint{0.100000in}{0.212622in}}{\pgfqpoint{3.696000in}{3.696000in}}%
\pgfusepath{clip}%
\pgfsetbuttcap%
\pgfsetroundjoin%
\definecolor{currentfill}{rgb}{0.121569,0.466667,0.705882}%
\pgfsetfillcolor{currentfill}%
\pgfsetfillopacity{0.604795}%
\pgfsetlinewidth{1.003750pt}%
\definecolor{currentstroke}{rgb}{0.121569,0.466667,0.705882}%
\pgfsetstrokecolor{currentstroke}%
\pgfsetstrokeopacity{0.604795}%
\pgfsetdash{}{0pt}%
\pgfpathmoveto{\pgfqpoint{1.719616in}{3.096228in}}%
\pgfpathcurveto{\pgfqpoint{1.727852in}{3.096228in}}{\pgfqpoint{1.735752in}{3.099501in}}{\pgfqpoint{1.741576in}{3.105325in}}%
\pgfpathcurveto{\pgfqpoint{1.747400in}{3.111149in}}{\pgfqpoint{1.750672in}{3.119049in}}{\pgfqpoint{1.750672in}{3.127285in}}%
\pgfpathcurveto{\pgfqpoint{1.750672in}{3.135521in}}{\pgfqpoint{1.747400in}{3.143421in}}{\pgfqpoint{1.741576in}{3.149245in}}%
\pgfpathcurveto{\pgfqpoint{1.735752in}{3.155069in}}{\pgfqpoint{1.727852in}{3.158341in}}{\pgfqpoint{1.719616in}{3.158341in}}%
\pgfpathcurveto{\pgfqpoint{1.711380in}{3.158341in}}{\pgfqpoint{1.703480in}{3.155069in}}{\pgfqpoint{1.697656in}{3.149245in}}%
\pgfpathcurveto{\pgfqpoint{1.691832in}{3.143421in}}{\pgfqpoint{1.688559in}{3.135521in}}{\pgfqpoint{1.688559in}{3.127285in}}%
\pgfpathcurveto{\pgfqpoint{1.688559in}{3.119049in}}{\pgfqpoint{1.691832in}{3.111149in}}{\pgfqpoint{1.697656in}{3.105325in}}%
\pgfpathcurveto{\pgfqpoint{1.703480in}{3.099501in}}{\pgfqpoint{1.711380in}{3.096228in}}{\pgfqpoint{1.719616in}{3.096228in}}%
\pgfpathclose%
\pgfusepath{stroke,fill}%
\end{pgfscope}%
\begin{pgfscope}%
\pgfpathrectangle{\pgfqpoint{0.100000in}{0.212622in}}{\pgfqpoint{3.696000in}{3.696000in}}%
\pgfusepath{clip}%
\pgfsetbuttcap%
\pgfsetroundjoin%
\definecolor{currentfill}{rgb}{0.121569,0.466667,0.705882}%
\pgfsetfillcolor{currentfill}%
\pgfsetfillopacity{0.604830}%
\pgfsetlinewidth{1.003750pt}%
\definecolor{currentstroke}{rgb}{0.121569,0.466667,0.705882}%
\pgfsetstrokecolor{currentstroke}%
\pgfsetstrokeopacity{0.604830}%
\pgfsetdash{}{0pt}%
\pgfpathmoveto{\pgfqpoint{1.724783in}{3.096322in}}%
\pgfpathcurveto{\pgfqpoint{1.733020in}{3.096322in}}{\pgfqpoint{1.740920in}{3.099595in}}{\pgfqpoint{1.746744in}{3.105419in}}%
\pgfpathcurveto{\pgfqpoint{1.752568in}{3.111243in}}{\pgfqpoint{1.755840in}{3.119143in}}{\pgfqpoint{1.755840in}{3.127379in}}%
\pgfpathcurveto{\pgfqpoint{1.755840in}{3.135615in}}{\pgfqpoint{1.752568in}{3.143515in}}{\pgfqpoint{1.746744in}{3.149339in}}%
\pgfpathcurveto{\pgfqpoint{1.740920in}{3.155163in}}{\pgfqpoint{1.733020in}{3.158435in}}{\pgfqpoint{1.724783in}{3.158435in}}%
\pgfpathcurveto{\pgfqpoint{1.716547in}{3.158435in}}{\pgfqpoint{1.708647in}{3.155163in}}{\pgfqpoint{1.702823in}{3.149339in}}%
\pgfpathcurveto{\pgfqpoint{1.696999in}{3.143515in}}{\pgfqpoint{1.693727in}{3.135615in}}{\pgfqpoint{1.693727in}{3.127379in}}%
\pgfpathcurveto{\pgfqpoint{1.693727in}{3.119143in}}{\pgfqpoint{1.696999in}{3.111243in}}{\pgfqpoint{1.702823in}{3.105419in}}%
\pgfpathcurveto{\pgfqpoint{1.708647in}{3.099595in}}{\pgfqpoint{1.716547in}{3.096322in}}{\pgfqpoint{1.724783in}{3.096322in}}%
\pgfpathclose%
\pgfusepath{stroke,fill}%
\end{pgfscope}%
\begin{pgfscope}%
\pgfpathrectangle{\pgfqpoint{0.100000in}{0.212622in}}{\pgfqpoint{3.696000in}{3.696000in}}%
\pgfusepath{clip}%
\pgfsetbuttcap%
\pgfsetroundjoin%
\definecolor{currentfill}{rgb}{0.121569,0.466667,0.705882}%
\pgfsetfillcolor{currentfill}%
\pgfsetfillopacity{0.605013}%
\pgfsetlinewidth{1.003750pt}%
\definecolor{currentstroke}{rgb}{0.121569,0.466667,0.705882}%
\pgfsetstrokecolor{currentstroke}%
\pgfsetstrokeopacity{0.605013}%
\pgfsetdash{}{0pt}%
\pgfpathmoveto{\pgfqpoint{1.719264in}{3.095965in}}%
\pgfpathcurveto{\pgfqpoint{1.727501in}{3.095965in}}{\pgfqpoint{1.735401in}{3.099237in}}{\pgfqpoint{1.741225in}{3.105061in}}%
\pgfpathcurveto{\pgfqpoint{1.747049in}{3.110885in}}{\pgfqpoint{1.750321in}{3.118785in}}{\pgfqpoint{1.750321in}{3.127021in}}%
\pgfpathcurveto{\pgfqpoint{1.750321in}{3.135257in}}{\pgfqpoint{1.747049in}{3.143157in}}{\pgfqpoint{1.741225in}{3.148981in}}%
\pgfpathcurveto{\pgfqpoint{1.735401in}{3.154805in}}{\pgfqpoint{1.727501in}{3.158078in}}{\pgfqpoint{1.719264in}{3.158078in}}%
\pgfpathcurveto{\pgfqpoint{1.711028in}{3.158078in}}{\pgfqpoint{1.703128in}{3.154805in}}{\pgfqpoint{1.697304in}{3.148981in}}%
\pgfpathcurveto{\pgfqpoint{1.691480in}{3.143157in}}{\pgfqpoint{1.688208in}{3.135257in}}{\pgfqpoint{1.688208in}{3.127021in}}%
\pgfpathcurveto{\pgfqpoint{1.688208in}{3.118785in}}{\pgfqpoint{1.691480in}{3.110885in}}{\pgfqpoint{1.697304in}{3.105061in}}%
\pgfpathcurveto{\pgfqpoint{1.703128in}{3.099237in}}{\pgfqpoint{1.711028in}{3.095965in}}{\pgfqpoint{1.719264in}{3.095965in}}%
\pgfpathclose%
\pgfusepath{stroke,fill}%
\end{pgfscope}%
\begin{pgfscope}%
\pgfpathrectangle{\pgfqpoint{0.100000in}{0.212622in}}{\pgfqpoint{3.696000in}{3.696000in}}%
\pgfusepath{clip}%
\pgfsetbuttcap%
\pgfsetroundjoin%
\definecolor{currentfill}{rgb}{0.121569,0.466667,0.705882}%
\pgfsetfillcolor{currentfill}%
\pgfsetfillopacity{0.605132}%
\pgfsetlinewidth{1.003750pt}%
\definecolor{currentstroke}{rgb}{0.121569,0.466667,0.705882}%
\pgfsetstrokecolor{currentstroke}%
\pgfsetstrokeopacity{0.605132}%
\pgfsetdash{}{0pt}%
\pgfpathmoveto{\pgfqpoint{1.719069in}{3.095813in}}%
\pgfpathcurveto{\pgfqpoint{1.727305in}{3.095813in}}{\pgfqpoint{1.735205in}{3.099085in}}{\pgfqpoint{1.741029in}{3.104909in}}%
\pgfpathcurveto{\pgfqpoint{1.746853in}{3.110733in}}{\pgfqpoint{1.750125in}{3.118633in}}{\pgfqpoint{1.750125in}{3.126869in}}%
\pgfpathcurveto{\pgfqpoint{1.750125in}{3.135106in}}{\pgfqpoint{1.746853in}{3.143006in}}{\pgfqpoint{1.741029in}{3.148830in}}%
\pgfpathcurveto{\pgfqpoint{1.735205in}{3.154654in}}{\pgfqpoint{1.727305in}{3.157926in}}{\pgfqpoint{1.719069in}{3.157926in}}%
\pgfpathcurveto{\pgfqpoint{1.710832in}{3.157926in}}{\pgfqpoint{1.702932in}{3.154654in}}{\pgfqpoint{1.697108in}{3.148830in}}%
\pgfpathcurveto{\pgfqpoint{1.691284in}{3.143006in}}{\pgfqpoint{1.688012in}{3.135106in}}{\pgfqpoint{1.688012in}{3.126869in}}%
\pgfpathcurveto{\pgfqpoint{1.688012in}{3.118633in}}{\pgfqpoint{1.691284in}{3.110733in}}{\pgfqpoint{1.697108in}{3.104909in}}%
\pgfpathcurveto{\pgfqpoint{1.702932in}{3.099085in}}{\pgfqpoint{1.710832in}{3.095813in}}{\pgfqpoint{1.719069in}{3.095813in}}%
\pgfpathclose%
\pgfusepath{stroke,fill}%
\end{pgfscope}%
\begin{pgfscope}%
\pgfpathrectangle{\pgfqpoint{0.100000in}{0.212622in}}{\pgfqpoint{3.696000in}{3.696000in}}%
\pgfusepath{clip}%
\pgfsetbuttcap%
\pgfsetroundjoin%
\definecolor{currentfill}{rgb}{0.121569,0.466667,0.705882}%
\pgfsetfillcolor{currentfill}%
\pgfsetfillopacity{0.605268}%
\pgfsetlinewidth{1.003750pt}%
\definecolor{currentstroke}{rgb}{0.121569,0.466667,0.705882}%
\pgfsetstrokecolor{currentstroke}%
\pgfsetstrokeopacity{0.605268}%
\pgfsetdash{}{0pt}%
\pgfpathmoveto{\pgfqpoint{1.727698in}{3.096149in}}%
\pgfpathcurveto{\pgfqpoint{1.735934in}{3.096149in}}{\pgfqpoint{1.743834in}{3.099421in}}{\pgfqpoint{1.749658in}{3.105245in}}%
\pgfpathcurveto{\pgfqpoint{1.755482in}{3.111069in}}{\pgfqpoint{1.758754in}{3.118969in}}{\pgfqpoint{1.758754in}{3.127205in}}%
\pgfpathcurveto{\pgfqpoint{1.758754in}{3.135442in}}{\pgfqpoint{1.755482in}{3.143342in}}{\pgfqpoint{1.749658in}{3.149166in}}%
\pgfpathcurveto{\pgfqpoint{1.743834in}{3.154989in}}{\pgfqpoint{1.735934in}{3.158262in}}{\pgfqpoint{1.727698in}{3.158262in}}%
\pgfpathcurveto{\pgfqpoint{1.719462in}{3.158262in}}{\pgfqpoint{1.711561in}{3.154989in}}{\pgfqpoint{1.705738in}{3.149166in}}%
\pgfpathcurveto{\pgfqpoint{1.699914in}{3.143342in}}{\pgfqpoint{1.696641in}{3.135442in}}{\pgfqpoint{1.696641in}{3.127205in}}%
\pgfpathcurveto{\pgfqpoint{1.696641in}{3.118969in}}{\pgfqpoint{1.699914in}{3.111069in}}{\pgfqpoint{1.705738in}{3.105245in}}%
\pgfpathcurveto{\pgfqpoint{1.711561in}{3.099421in}}{\pgfqpoint{1.719462in}{3.096149in}}{\pgfqpoint{1.727698in}{3.096149in}}%
\pgfpathclose%
\pgfusepath{stroke,fill}%
\end{pgfscope}%
\begin{pgfscope}%
\pgfpathrectangle{\pgfqpoint{0.100000in}{0.212622in}}{\pgfqpoint{3.696000in}{3.696000in}}%
\pgfusepath{clip}%
\pgfsetbuttcap%
\pgfsetroundjoin%
\definecolor{currentfill}{rgb}{0.121569,0.466667,0.705882}%
\pgfsetfillcolor{currentfill}%
\pgfsetfillopacity{0.605395}%
\pgfsetlinewidth{1.003750pt}%
\definecolor{currentstroke}{rgb}{0.121569,0.466667,0.705882}%
\pgfsetstrokecolor{currentstroke}%
\pgfsetstrokeopacity{0.605395}%
\pgfsetdash{}{0pt}%
\pgfpathmoveto{\pgfqpoint{1.718629in}{3.095374in}}%
\pgfpathcurveto{\pgfqpoint{1.726865in}{3.095374in}}{\pgfqpoint{1.734765in}{3.098646in}}{\pgfqpoint{1.740589in}{3.104470in}}%
\pgfpathcurveto{\pgfqpoint{1.746413in}{3.110294in}}{\pgfqpoint{1.749686in}{3.118194in}}{\pgfqpoint{1.749686in}{3.126430in}}%
\pgfpathcurveto{\pgfqpoint{1.749686in}{3.134666in}}{\pgfqpoint{1.746413in}{3.142566in}}{\pgfqpoint{1.740589in}{3.148390in}}%
\pgfpathcurveto{\pgfqpoint{1.734765in}{3.154214in}}{\pgfqpoint{1.726865in}{3.157487in}}{\pgfqpoint{1.718629in}{3.157487in}}%
\pgfpathcurveto{\pgfqpoint{1.710393in}{3.157487in}}{\pgfqpoint{1.702493in}{3.154214in}}{\pgfqpoint{1.696669in}{3.148390in}}%
\pgfpathcurveto{\pgfqpoint{1.690845in}{3.142566in}}{\pgfqpoint{1.687573in}{3.134666in}}{\pgfqpoint{1.687573in}{3.126430in}}%
\pgfpathcurveto{\pgfqpoint{1.687573in}{3.118194in}}{\pgfqpoint{1.690845in}{3.110294in}}{\pgfqpoint{1.696669in}{3.104470in}}%
\pgfpathcurveto{\pgfqpoint{1.702493in}{3.098646in}}{\pgfqpoint{1.710393in}{3.095374in}}{\pgfqpoint{1.718629in}{3.095374in}}%
\pgfpathclose%
\pgfusepath{stroke,fill}%
\end{pgfscope}%
\begin{pgfscope}%
\pgfpathrectangle{\pgfqpoint{0.100000in}{0.212622in}}{\pgfqpoint{3.696000in}{3.696000in}}%
\pgfusepath{clip}%
\pgfsetbuttcap%
\pgfsetroundjoin%
\definecolor{currentfill}{rgb}{0.121569,0.466667,0.705882}%
\pgfsetfillcolor{currentfill}%
\pgfsetfillopacity{0.605984}%
\pgfsetlinewidth{1.003750pt}%
\definecolor{currentstroke}{rgb}{0.121569,0.466667,0.705882}%
\pgfsetstrokecolor{currentstroke}%
\pgfsetstrokeopacity{0.605984}%
\pgfsetdash{}{0pt}%
\pgfpathmoveto{\pgfqpoint{1.732859in}{3.094813in}}%
\pgfpathcurveto{\pgfqpoint{1.741096in}{3.094813in}}{\pgfqpoint{1.748996in}{3.098085in}}{\pgfqpoint{1.754820in}{3.103909in}}%
\pgfpathcurveto{\pgfqpoint{1.760644in}{3.109733in}}{\pgfqpoint{1.763916in}{3.117633in}}{\pgfqpoint{1.763916in}{3.125869in}}%
\pgfpathcurveto{\pgfqpoint{1.763916in}{3.134105in}}{\pgfqpoint{1.760644in}{3.142006in}}{\pgfqpoint{1.754820in}{3.147829in}}%
\pgfpathcurveto{\pgfqpoint{1.748996in}{3.153653in}}{\pgfqpoint{1.741096in}{3.156926in}}{\pgfqpoint{1.732859in}{3.156926in}}%
\pgfpathcurveto{\pgfqpoint{1.724623in}{3.156926in}}{\pgfqpoint{1.716723in}{3.153653in}}{\pgfqpoint{1.710899in}{3.147829in}}%
\pgfpathcurveto{\pgfqpoint{1.705075in}{3.142006in}}{\pgfqpoint{1.701803in}{3.134105in}}{\pgfqpoint{1.701803in}{3.125869in}}%
\pgfpathcurveto{\pgfqpoint{1.701803in}{3.117633in}}{\pgfqpoint{1.705075in}{3.109733in}}{\pgfqpoint{1.710899in}{3.103909in}}%
\pgfpathcurveto{\pgfqpoint{1.716723in}{3.098085in}}{\pgfqpoint{1.724623in}{3.094813in}}{\pgfqpoint{1.732859in}{3.094813in}}%
\pgfpathclose%
\pgfusepath{stroke,fill}%
\end{pgfscope}%
\begin{pgfscope}%
\pgfpathrectangle{\pgfqpoint{0.100000in}{0.212622in}}{\pgfqpoint{3.696000in}{3.696000in}}%
\pgfusepath{clip}%
\pgfsetbuttcap%
\pgfsetroundjoin%
\definecolor{currentfill}{rgb}{0.121569,0.466667,0.705882}%
\pgfsetfillcolor{currentfill}%
\pgfsetfillopacity{0.606076}%
\pgfsetlinewidth{1.003750pt}%
\definecolor{currentstroke}{rgb}{0.121569,0.466667,0.705882}%
\pgfsetstrokecolor{currentstroke}%
\pgfsetstrokeopacity{0.606076}%
\pgfsetdash{}{0pt}%
\pgfpathmoveto{\pgfqpoint{1.717518in}{3.094091in}}%
\pgfpathcurveto{\pgfqpoint{1.725755in}{3.094091in}}{\pgfqpoint{1.733655in}{3.097364in}}{\pgfqpoint{1.739479in}{3.103187in}}%
\pgfpathcurveto{\pgfqpoint{1.745303in}{3.109011in}}{\pgfqpoint{1.748575in}{3.116911in}}{\pgfqpoint{1.748575in}{3.125148in}}%
\pgfpathcurveto{\pgfqpoint{1.748575in}{3.133384in}}{\pgfqpoint{1.745303in}{3.141284in}}{\pgfqpoint{1.739479in}{3.147108in}}%
\pgfpathcurveto{\pgfqpoint{1.733655in}{3.152932in}}{\pgfqpoint{1.725755in}{3.156204in}}{\pgfqpoint{1.717518in}{3.156204in}}%
\pgfpathcurveto{\pgfqpoint{1.709282in}{3.156204in}}{\pgfqpoint{1.701382in}{3.152932in}}{\pgfqpoint{1.695558in}{3.147108in}}%
\pgfpathcurveto{\pgfqpoint{1.689734in}{3.141284in}}{\pgfqpoint{1.686462in}{3.133384in}}{\pgfqpoint{1.686462in}{3.125148in}}%
\pgfpathcurveto{\pgfqpoint{1.686462in}{3.116911in}}{\pgfqpoint{1.689734in}{3.109011in}}{\pgfqpoint{1.695558in}{3.103187in}}%
\pgfpathcurveto{\pgfqpoint{1.701382in}{3.097364in}}{\pgfqpoint{1.709282in}{3.094091in}}{\pgfqpoint{1.717518in}{3.094091in}}%
\pgfpathclose%
\pgfusepath{stroke,fill}%
\end{pgfscope}%
\begin{pgfscope}%
\pgfpathrectangle{\pgfqpoint{0.100000in}{0.212622in}}{\pgfqpoint{3.696000in}{3.696000in}}%
\pgfusepath{clip}%
\pgfsetbuttcap%
\pgfsetroundjoin%
\definecolor{currentfill}{rgb}{0.121569,0.466667,0.705882}%
\pgfsetfillcolor{currentfill}%
\pgfsetfillopacity{0.606442}%
\pgfsetlinewidth{1.003750pt}%
\definecolor{currentstroke}{rgb}{0.121569,0.466667,0.705882}%
\pgfsetstrokecolor{currentstroke}%
\pgfsetstrokeopacity{0.606442}%
\pgfsetdash{}{0pt}%
\pgfpathmoveto{\pgfqpoint{1.716899in}{3.093340in}}%
\pgfpathcurveto{\pgfqpoint{1.725136in}{3.093340in}}{\pgfqpoint{1.733036in}{3.096613in}}{\pgfqpoint{1.738860in}{3.102437in}}%
\pgfpathcurveto{\pgfqpoint{1.744684in}{3.108261in}}{\pgfqpoint{1.747956in}{3.116161in}}{\pgfqpoint{1.747956in}{3.124397in}}%
\pgfpathcurveto{\pgfqpoint{1.747956in}{3.132633in}}{\pgfqpoint{1.744684in}{3.140533in}}{\pgfqpoint{1.738860in}{3.146357in}}%
\pgfpathcurveto{\pgfqpoint{1.733036in}{3.152181in}}{\pgfqpoint{1.725136in}{3.155453in}}{\pgfqpoint{1.716899in}{3.155453in}}%
\pgfpathcurveto{\pgfqpoint{1.708663in}{3.155453in}}{\pgfqpoint{1.700763in}{3.152181in}}{\pgfqpoint{1.694939in}{3.146357in}}%
\pgfpathcurveto{\pgfqpoint{1.689115in}{3.140533in}}{\pgfqpoint{1.685843in}{3.132633in}}{\pgfqpoint{1.685843in}{3.124397in}}%
\pgfpathcurveto{\pgfqpoint{1.685843in}{3.116161in}}{\pgfqpoint{1.689115in}{3.108261in}}{\pgfqpoint{1.694939in}{3.102437in}}%
\pgfpathcurveto{\pgfqpoint{1.700763in}{3.096613in}}{\pgfqpoint{1.708663in}{3.093340in}}{\pgfqpoint{1.716899in}{3.093340in}}%
\pgfpathclose%
\pgfusepath{stroke,fill}%
\end{pgfscope}%
\begin{pgfscope}%
\pgfpathrectangle{\pgfqpoint{0.100000in}{0.212622in}}{\pgfqpoint{3.696000in}{3.696000in}}%
\pgfusepath{clip}%
\pgfsetbuttcap%
\pgfsetroundjoin%
\definecolor{currentfill}{rgb}{0.121569,0.466667,0.705882}%
\pgfsetfillcolor{currentfill}%
\pgfsetfillopacity{0.606724}%
\pgfsetlinewidth{1.003750pt}%
\definecolor{currentstroke}{rgb}{0.121569,0.466667,0.705882}%
\pgfsetstrokecolor{currentstroke}%
\pgfsetstrokeopacity{0.606724}%
\pgfsetdash{}{0pt}%
\pgfpathmoveto{\pgfqpoint{3.259707in}{2.021928in}}%
\pgfpathcurveto{\pgfqpoint{3.267943in}{2.021928in}}{\pgfqpoint{3.275843in}{2.025201in}}{\pgfqpoint{3.281667in}{2.031025in}}%
\pgfpathcurveto{\pgfqpoint{3.287491in}{2.036849in}}{\pgfqpoint{3.290763in}{2.044749in}}{\pgfqpoint{3.290763in}{2.052985in}}%
\pgfpathcurveto{\pgfqpoint{3.290763in}{2.061221in}}{\pgfqpoint{3.287491in}{2.069121in}}{\pgfqpoint{3.281667in}{2.074945in}}%
\pgfpathcurveto{\pgfqpoint{3.275843in}{2.080769in}}{\pgfqpoint{3.267943in}{2.084041in}}{\pgfqpoint{3.259707in}{2.084041in}}%
\pgfpathcurveto{\pgfqpoint{3.251470in}{2.084041in}}{\pgfqpoint{3.243570in}{2.080769in}}{\pgfqpoint{3.237746in}{2.074945in}}%
\pgfpathcurveto{\pgfqpoint{3.231922in}{2.069121in}}{\pgfqpoint{3.228650in}{2.061221in}}{\pgfqpoint{3.228650in}{2.052985in}}%
\pgfpathcurveto{\pgfqpoint{3.228650in}{2.044749in}}{\pgfqpoint{3.231922in}{2.036849in}}{\pgfqpoint{3.237746in}{2.031025in}}%
\pgfpathcurveto{\pgfqpoint{3.243570in}{2.025201in}}{\pgfqpoint{3.251470in}{2.021928in}}{\pgfqpoint{3.259707in}{2.021928in}}%
\pgfpathclose%
\pgfusepath{stroke,fill}%
\end{pgfscope}%
\begin{pgfscope}%
\pgfpathrectangle{\pgfqpoint{0.100000in}{0.212622in}}{\pgfqpoint{3.696000in}{3.696000in}}%
\pgfusepath{clip}%
\pgfsetbuttcap%
\pgfsetroundjoin%
\definecolor{currentfill}{rgb}{0.121569,0.466667,0.705882}%
\pgfsetfillcolor{currentfill}%
\pgfsetfillopacity{0.606919}%
\pgfsetlinewidth{1.003750pt}%
\definecolor{currentstroke}{rgb}{0.121569,0.466667,0.705882}%
\pgfsetstrokecolor{currentstroke}%
\pgfsetstrokeopacity{0.606919}%
\pgfsetdash{}{0pt}%
\pgfpathmoveto{\pgfqpoint{1.716137in}{3.092138in}}%
\pgfpathcurveto{\pgfqpoint{1.724373in}{3.092138in}}{\pgfqpoint{1.732273in}{3.095411in}}{\pgfqpoint{1.738097in}{3.101235in}}%
\pgfpathcurveto{\pgfqpoint{1.743921in}{3.107059in}}{\pgfqpoint{1.747193in}{3.114959in}}{\pgfqpoint{1.747193in}{3.123195in}}%
\pgfpathcurveto{\pgfqpoint{1.747193in}{3.131431in}}{\pgfqpoint{1.743921in}{3.139331in}}{\pgfqpoint{1.738097in}{3.145155in}}%
\pgfpathcurveto{\pgfqpoint{1.732273in}{3.150979in}}{\pgfqpoint{1.724373in}{3.154251in}}{\pgfqpoint{1.716137in}{3.154251in}}%
\pgfpathcurveto{\pgfqpoint{1.707901in}{3.154251in}}{\pgfqpoint{1.700000in}{3.150979in}}{\pgfqpoint{1.694177in}{3.145155in}}%
\pgfpathcurveto{\pgfqpoint{1.688353in}{3.139331in}}{\pgfqpoint{1.685080in}{3.131431in}}{\pgfqpoint{1.685080in}{3.123195in}}%
\pgfpathcurveto{\pgfqpoint{1.685080in}{3.114959in}}{\pgfqpoint{1.688353in}{3.107059in}}{\pgfqpoint{1.694177in}{3.101235in}}%
\pgfpathcurveto{\pgfqpoint{1.700000in}{3.095411in}}{\pgfqpoint{1.707901in}{3.092138in}}{\pgfqpoint{1.716137in}{3.092138in}}%
\pgfpathclose%
\pgfusepath{stroke,fill}%
\end{pgfscope}%
\begin{pgfscope}%
\pgfpathrectangle{\pgfqpoint{0.100000in}{0.212622in}}{\pgfqpoint{3.696000in}{3.696000in}}%
\pgfusepath{clip}%
\pgfsetbuttcap%
\pgfsetroundjoin%
\definecolor{currentfill}{rgb}{0.121569,0.466667,0.705882}%
\pgfsetfillcolor{currentfill}%
\pgfsetfillopacity{0.607160}%
\pgfsetlinewidth{1.003750pt}%
\definecolor{currentstroke}{rgb}{0.121569,0.466667,0.705882}%
\pgfsetstrokecolor{currentstroke}%
\pgfsetstrokeopacity{0.607160}%
\pgfsetdash{}{0pt}%
\pgfpathmoveto{\pgfqpoint{1.742645in}{3.093035in}}%
\pgfpathcurveto{\pgfqpoint{1.750882in}{3.093035in}}{\pgfqpoint{1.758782in}{3.096307in}}{\pgfqpoint{1.764606in}{3.102131in}}%
\pgfpathcurveto{\pgfqpoint{1.770429in}{3.107955in}}{\pgfqpoint{1.773702in}{3.115855in}}{\pgfqpoint{1.773702in}{3.124091in}}%
\pgfpathcurveto{\pgfqpoint{1.773702in}{3.132327in}}{\pgfqpoint{1.770429in}{3.140227in}}{\pgfqpoint{1.764606in}{3.146051in}}%
\pgfpathcurveto{\pgfqpoint{1.758782in}{3.151875in}}{\pgfqpoint{1.750882in}{3.155148in}}{\pgfqpoint{1.742645in}{3.155148in}}%
\pgfpathcurveto{\pgfqpoint{1.734409in}{3.155148in}}{\pgfqpoint{1.726509in}{3.151875in}}{\pgfqpoint{1.720685in}{3.146051in}}%
\pgfpathcurveto{\pgfqpoint{1.714861in}{3.140227in}}{\pgfqpoint{1.711589in}{3.132327in}}{\pgfqpoint{1.711589in}{3.124091in}}%
\pgfpathcurveto{\pgfqpoint{1.711589in}{3.115855in}}{\pgfqpoint{1.714861in}{3.107955in}}{\pgfqpoint{1.720685in}{3.102131in}}%
\pgfpathcurveto{\pgfqpoint{1.726509in}{3.096307in}}{\pgfqpoint{1.734409in}{3.093035in}}{\pgfqpoint{1.742645in}{3.093035in}}%
\pgfpathclose%
\pgfusepath{stroke,fill}%
\end{pgfscope}%
\begin{pgfscope}%
\pgfpathrectangle{\pgfqpoint{0.100000in}{0.212622in}}{\pgfqpoint{3.696000in}{3.696000in}}%
\pgfusepath{clip}%
\pgfsetbuttcap%
\pgfsetroundjoin%
\definecolor{currentfill}{rgb}{0.121569,0.466667,0.705882}%
\pgfsetfillcolor{currentfill}%
\pgfsetfillopacity{0.607639}%
\pgfsetlinewidth{1.003750pt}%
\definecolor{currentstroke}{rgb}{0.121569,0.466667,0.705882}%
\pgfsetstrokecolor{currentstroke}%
\pgfsetstrokeopacity{0.607639}%
\pgfsetdash{}{0pt}%
\pgfpathmoveto{\pgfqpoint{1.714991in}{3.090302in}}%
\pgfpathcurveto{\pgfqpoint{1.723227in}{3.090302in}}{\pgfqpoint{1.731127in}{3.093574in}}{\pgfqpoint{1.736951in}{3.099398in}}%
\pgfpathcurveto{\pgfqpoint{1.742775in}{3.105222in}}{\pgfqpoint{1.746047in}{3.113122in}}{\pgfqpoint{1.746047in}{3.121358in}}%
\pgfpathcurveto{\pgfqpoint{1.746047in}{3.129595in}}{\pgfqpoint{1.742775in}{3.137495in}}{\pgfqpoint{1.736951in}{3.143319in}}%
\pgfpathcurveto{\pgfqpoint{1.731127in}{3.149143in}}{\pgfqpoint{1.723227in}{3.152415in}}{\pgfqpoint{1.714991in}{3.152415in}}%
\pgfpathcurveto{\pgfqpoint{1.706755in}{3.152415in}}{\pgfqpoint{1.698855in}{3.149143in}}{\pgfqpoint{1.693031in}{3.143319in}}%
\pgfpathcurveto{\pgfqpoint{1.687207in}{3.137495in}}{\pgfqpoint{1.683934in}{3.129595in}}{\pgfqpoint{1.683934in}{3.121358in}}%
\pgfpathcurveto{\pgfqpoint{1.683934in}{3.113122in}}{\pgfqpoint{1.687207in}{3.105222in}}{\pgfqpoint{1.693031in}{3.099398in}}%
\pgfpathcurveto{\pgfqpoint{1.698855in}{3.093574in}}{\pgfqpoint{1.706755in}{3.090302in}}{\pgfqpoint{1.714991in}{3.090302in}}%
\pgfpathclose%
\pgfusepath{stroke,fill}%
\end{pgfscope}%
\begin{pgfscope}%
\pgfpathrectangle{\pgfqpoint{0.100000in}{0.212622in}}{\pgfqpoint{3.696000in}{3.696000in}}%
\pgfusepath{clip}%
\pgfsetbuttcap%
\pgfsetroundjoin%
\definecolor{currentfill}{rgb}{0.121569,0.466667,0.705882}%
\pgfsetfillcolor{currentfill}%
\pgfsetfillopacity{0.608454}%
\pgfsetlinewidth{1.003750pt}%
\definecolor{currentstroke}{rgb}{0.121569,0.466667,0.705882}%
\pgfsetstrokecolor{currentstroke}%
\pgfsetstrokeopacity{0.608454}%
\pgfsetdash{}{0pt}%
\pgfpathmoveto{\pgfqpoint{1.713671in}{3.088141in}}%
\pgfpathcurveto{\pgfqpoint{1.721907in}{3.088141in}}{\pgfqpoint{1.729807in}{3.091414in}}{\pgfqpoint{1.735631in}{3.097238in}}%
\pgfpathcurveto{\pgfqpoint{1.741455in}{3.103061in}}{\pgfqpoint{1.744727in}{3.110962in}}{\pgfqpoint{1.744727in}{3.119198in}}%
\pgfpathcurveto{\pgfqpoint{1.744727in}{3.127434in}}{\pgfqpoint{1.741455in}{3.135334in}}{\pgfqpoint{1.735631in}{3.141158in}}%
\pgfpathcurveto{\pgfqpoint{1.729807in}{3.146982in}}{\pgfqpoint{1.721907in}{3.150254in}}{\pgfqpoint{1.713671in}{3.150254in}}%
\pgfpathcurveto{\pgfqpoint{1.705434in}{3.150254in}}{\pgfqpoint{1.697534in}{3.146982in}}{\pgfqpoint{1.691710in}{3.141158in}}%
\pgfpathcurveto{\pgfqpoint{1.685886in}{3.135334in}}{\pgfqpoint{1.682614in}{3.127434in}}{\pgfqpoint{1.682614in}{3.119198in}}%
\pgfpathcurveto{\pgfqpoint{1.682614in}{3.110962in}}{\pgfqpoint{1.685886in}{3.103061in}}{\pgfqpoint{1.691710in}{3.097238in}}%
\pgfpathcurveto{\pgfqpoint{1.697534in}{3.091414in}}{\pgfqpoint{1.705434in}{3.088141in}}{\pgfqpoint{1.713671in}{3.088141in}}%
\pgfpathclose%
\pgfusepath{stroke,fill}%
\end{pgfscope}%
\begin{pgfscope}%
\pgfpathrectangle{\pgfqpoint{0.100000in}{0.212622in}}{\pgfqpoint{3.696000in}{3.696000in}}%
\pgfusepath{clip}%
\pgfsetbuttcap%
\pgfsetroundjoin%
\definecolor{currentfill}{rgb}{0.121569,0.466667,0.705882}%
\pgfsetfillcolor{currentfill}%
\pgfsetfillopacity{0.608693}%
\pgfsetlinewidth{1.003750pt}%
\definecolor{currentstroke}{rgb}{0.121569,0.466667,0.705882}%
\pgfsetstrokecolor{currentstroke}%
\pgfsetstrokeopacity{0.608693}%
\pgfsetdash{}{0pt}%
\pgfpathmoveto{\pgfqpoint{3.255384in}{2.015445in}}%
\pgfpathcurveto{\pgfqpoint{3.263620in}{2.015445in}}{\pgfqpoint{3.271520in}{2.018718in}}{\pgfqpoint{3.277344in}{2.024542in}}%
\pgfpathcurveto{\pgfqpoint{3.283168in}{2.030366in}}{\pgfqpoint{3.286440in}{2.038266in}}{\pgfqpoint{3.286440in}{2.046502in}}%
\pgfpathcurveto{\pgfqpoint{3.286440in}{2.054738in}}{\pgfqpoint{3.283168in}{2.062638in}}{\pgfqpoint{3.277344in}{2.068462in}}%
\pgfpathcurveto{\pgfqpoint{3.271520in}{2.074286in}}{\pgfqpoint{3.263620in}{2.077558in}}{\pgfqpoint{3.255384in}{2.077558in}}%
\pgfpathcurveto{\pgfqpoint{3.247148in}{2.077558in}}{\pgfqpoint{3.239248in}{2.074286in}}{\pgfqpoint{3.233424in}{2.068462in}}%
\pgfpathcurveto{\pgfqpoint{3.227600in}{2.062638in}}{\pgfqpoint{3.224327in}{2.054738in}}{\pgfqpoint{3.224327in}{2.046502in}}%
\pgfpathcurveto{\pgfqpoint{3.224327in}{2.038266in}}{\pgfqpoint{3.227600in}{2.030366in}}{\pgfqpoint{3.233424in}{2.024542in}}%
\pgfpathcurveto{\pgfqpoint{3.239248in}{2.018718in}}{\pgfqpoint{3.247148in}{2.015445in}}{\pgfqpoint{3.255384in}{2.015445in}}%
\pgfpathclose%
\pgfusepath{stroke,fill}%
\end{pgfscope}%
\begin{pgfscope}%
\pgfpathrectangle{\pgfqpoint{0.100000in}{0.212622in}}{\pgfqpoint{3.696000in}{3.696000in}}%
\pgfusepath{clip}%
\pgfsetbuttcap%
\pgfsetroundjoin%
\definecolor{currentfill}{rgb}{0.121569,0.466667,0.705882}%
\pgfsetfillcolor{currentfill}%
\pgfsetfillopacity{0.609723}%
\pgfsetlinewidth{1.003750pt}%
\definecolor{currentstroke}{rgb}{0.121569,0.466667,0.705882}%
\pgfsetstrokecolor{currentstroke}%
\pgfsetstrokeopacity{0.609723}%
\pgfsetdash{}{0pt}%
\pgfpathmoveto{\pgfqpoint{1.759674in}{3.089680in}}%
\pgfpathcurveto{\pgfqpoint{1.767911in}{3.089680in}}{\pgfqpoint{1.775811in}{3.092952in}}{\pgfqpoint{1.781634in}{3.098776in}}%
\pgfpathcurveto{\pgfqpoint{1.787458in}{3.104600in}}{\pgfqpoint{1.790731in}{3.112500in}}{\pgfqpoint{1.790731in}{3.120737in}}%
\pgfpathcurveto{\pgfqpoint{1.790731in}{3.128973in}}{\pgfqpoint{1.787458in}{3.136873in}}{\pgfqpoint{1.781634in}{3.142697in}}%
\pgfpathcurveto{\pgfqpoint{1.775811in}{3.148521in}}{\pgfqpoint{1.767911in}{3.151793in}}{\pgfqpoint{1.759674in}{3.151793in}}%
\pgfpathcurveto{\pgfqpoint{1.751438in}{3.151793in}}{\pgfqpoint{1.743538in}{3.148521in}}{\pgfqpoint{1.737714in}{3.142697in}}%
\pgfpathcurveto{\pgfqpoint{1.731890in}{3.136873in}}{\pgfqpoint{1.728618in}{3.128973in}}{\pgfqpoint{1.728618in}{3.120737in}}%
\pgfpathcurveto{\pgfqpoint{1.728618in}{3.112500in}}{\pgfqpoint{1.731890in}{3.104600in}}{\pgfqpoint{1.737714in}{3.098776in}}%
\pgfpathcurveto{\pgfqpoint{1.743538in}{3.092952in}}{\pgfqpoint{1.751438in}{3.089680in}}{\pgfqpoint{1.759674in}{3.089680in}}%
\pgfpathclose%
\pgfusepath{stroke,fill}%
\end{pgfscope}%
\begin{pgfscope}%
\pgfpathrectangle{\pgfqpoint{0.100000in}{0.212622in}}{\pgfqpoint{3.696000in}{3.696000in}}%
\pgfusepath{clip}%
\pgfsetbuttcap%
\pgfsetroundjoin%
\definecolor{currentfill}{rgb}{0.121569,0.466667,0.705882}%
\pgfsetfillcolor{currentfill}%
\pgfsetfillopacity{0.609849}%
\pgfsetlinewidth{1.003750pt}%
\definecolor{currentstroke}{rgb}{0.121569,0.466667,0.705882}%
\pgfsetstrokecolor{currentstroke}%
\pgfsetstrokeopacity{0.609849}%
\pgfsetdash{}{0pt}%
\pgfpathmoveto{\pgfqpoint{1.711344in}{3.084709in}}%
\pgfpathcurveto{\pgfqpoint{1.719580in}{3.084709in}}{\pgfqpoint{1.727480in}{3.087982in}}{\pgfqpoint{1.733304in}{3.093805in}}%
\pgfpathcurveto{\pgfqpoint{1.739128in}{3.099629in}}{\pgfqpoint{1.742400in}{3.107529in}}{\pgfqpoint{1.742400in}{3.115766in}}%
\pgfpathcurveto{\pgfqpoint{1.742400in}{3.124002in}}{\pgfqpoint{1.739128in}{3.131902in}}{\pgfqpoint{1.733304in}{3.137726in}}%
\pgfpathcurveto{\pgfqpoint{1.727480in}{3.143550in}}{\pgfqpoint{1.719580in}{3.146822in}}{\pgfqpoint{1.711344in}{3.146822in}}%
\pgfpathcurveto{\pgfqpoint{1.703107in}{3.146822in}}{\pgfqpoint{1.695207in}{3.143550in}}{\pgfqpoint{1.689383in}{3.137726in}}%
\pgfpathcurveto{\pgfqpoint{1.683559in}{3.131902in}}{\pgfqpoint{1.680287in}{3.124002in}}{\pgfqpoint{1.680287in}{3.115766in}}%
\pgfpathcurveto{\pgfqpoint{1.680287in}{3.107529in}}{\pgfqpoint{1.683559in}{3.099629in}}{\pgfqpoint{1.689383in}{3.093805in}}%
\pgfpathcurveto{\pgfqpoint{1.695207in}{3.087982in}}{\pgfqpoint{1.703107in}{3.084709in}}{\pgfqpoint{1.711344in}{3.084709in}}%
\pgfpathclose%
\pgfusepath{stroke,fill}%
\end{pgfscope}%
\begin{pgfscope}%
\pgfpathrectangle{\pgfqpoint{0.100000in}{0.212622in}}{\pgfqpoint{3.696000in}{3.696000in}}%
\pgfusepath{clip}%
\pgfsetbuttcap%
\pgfsetroundjoin%
\definecolor{currentfill}{rgb}{0.121569,0.466667,0.705882}%
\pgfsetfillcolor{currentfill}%
\pgfsetfillopacity{0.610412}%
\pgfsetlinewidth{1.003750pt}%
\definecolor{currentstroke}{rgb}{0.121569,0.466667,0.705882}%
\pgfsetstrokecolor{currentstroke}%
\pgfsetstrokeopacity{0.610412}%
\pgfsetdash{}{0pt}%
\pgfpathmoveto{\pgfqpoint{3.251677in}{2.009697in}}%
\pgfpathcurveto{\pgfqpoint{3.259913in}{2.009697in}}{\pgfqpoint{3.267813in}{2.012969in}}{\pgfqpoint{3.273637in}{2.018793in}}%
\pgfpathcurveto{\pgfqpoint{3.279461in}{2.024617in}}{\pgfqpoint{3.282734in}{2.032517in}}{\pgfqpoint{3.282734in}{2.040753in}}%
\pgfpathcurveto{\pgfqpoint{3.282734in}{2.048990in}}{\pgfqpoint{3.279461in}{2.056890in}}{\pgfqpoint{3.273637in}{2.062714in}}%
\pgfpathcurveto{\pgfqpoint{3.267813in}{2.068538in}}{\pgfqpoint{3.259913in}{2.071810in}}{\pgfqpoint{3.251677in}{2.071810in}}%
\pgfpathcurveto{\pgfqpoint{3.243441in}{2.071810in}}{\pgfqpoint{3.235541in}{2.068538in}}{\pgfqpoint{3.229717in}{2.062714in}}%
\pgfpathcurveto{\pgfqpoint{3.223893in}{2.056890in}}{\pgfqpoint{3.220621in}{2.048990in}}{\pgfqpoint{3.220621in}{2.040753in}}%
\pgfpathcurveto{\pgfqpoint{3.220621in}{2.032517in}}{\pgfqpoint{3.223893in}{2.024617in}}{\pgfqpoint{3.229717in}{2.018793in}}%
\pgfpathcurveto{\pgfqpoint{3.235541in}{2.012969in}}{\pgfqpoint{3.243441in}{2.009697in}}{\pgfqpoint{3.251677in}{2.009697in}}%
\pgfpathclose%
\pgfusepath{stroke,fill}%
\end{pgfscope}%
\begin{pgfscope}%
\pgfpathrectangle{\pgfqpoint{0.100000in}{0.212622in}}{\pgfqpoint{3.696000in}{3.696000in}}%
\pgfusepath{clip}%
\pgfsetbuttcap%
\pgfsetroundjoin%
\definecolor{currentfill}{rgb}{0.121569,0.466667,0.705882}%
\pgfsetfillcolor{currentfill}%
\pgfsetfillopacity{0.611342}%
\pgfsetlinewidth{1.003750pt}%
\definecolor{currentstroke}{rgb}{0.121569,0.466667,0.705882}%
\pgfsetstrokecolor{currentstroke}%
\pgfsetstrokeopacity{0.611342}%
\pgfsetdash{}{0pt}%
\pgfpathmoveto{\pgfqpoint{1.708920in}{3.081082in}}%
\pgfpathcurveto{\pgfqpoint{1.717156in}{3.081082in}}{\pgfqpoint{1.725056in}{3.084354in}}{\pgfqpoint{1.730880in}{3.090178in}}%
\pgfpathcurveto{\pgfqpoint{1.736704in}{3.096002in}}{\pgfqpoint{1.739976in}{3.103902in}}{\pgfqpoint{1.739976in}{3.112139in}}%
\pgfpathcurveto{\pgfqpoint{1.739976in}{3.120375in}}{\pgfqpoint{1.736704in}{3.128275in}}{\pgfqpoint{1.730880in}{3.134099in}}%
\pgfpathcurveto{\pgfqpoint{1.725056in}{3.139923in}}{\pgfqpoint{1.717156in}{3.143195in}}{\pgfqpoint{1.708920in}{3.143195in}}%
\pgfpathcurveto{\pgfqpoint{1.700683in}{3.143195in}}{\pgfqpoint{1.692783in}{3.139923in}}{\pgfqpoint{1.686959in}{3.134099in}}%
\pgfpathcurveto{\pgfqpoint{1.681135in}{3.128275in}}{\pgfqpoint{1.677863in}{3.120375in}}{\pgfqpoint{1.677863in}{3.112139in}}%
\pgfpathcurveto{\pgfqpoint{1.677863in}{3.103902in}}{\pgfqpoint{1.681135in}{3.096002in}}{\pgfqpoint{1.686959in}{3.090178in}}%
\pgfpathcurveto{\pgfqpoint{1.692783in}{3.084354in}}{\pgfqpoint{1.700683in}{3.081082in}}{\pgfqpoint{1.708920in}{3.081082in}}%
\pgfpathclose%
\pgfusepath{stroke,fill}%
\end{pgfscope}%
\begin{pgfscope}%
\pgfpathrectangle{\pgfqpoint{0.100000in}{0.212622in}}{\pgfqpoint{3.696000in}{3.696000in}}%
\pgfusepath{clip}%
\pgfsetbuttcap%
\pgfsetroundjoin%
\definecolor{currentfill}{rgb}{0.121569,0.466667,0.705882}%
\pgfsetfillcolor{currentfill}%
\pgfsetfillopacity{0.612826}%
\pgfsetlinewidth{1.003750pt}%
\definecolor{currentstroke}{rgb}{0.121569,0.466667,0.705882}%
\pgfsetstrokecolor{currentstroke}%
\pgfsetstrokeopacity{0.612826}%
\pgfsetdash{}{0pt}%
\pgfpathmoveto{\pgfqpoint{1.775972in}{3.088280in}}%
\pgfpathcurveto{\pgfqpoint{1.784208in}{3.088280in}}{\pgfqpoint{1.792108in}{3.091553in}}{\pgfqpoint{1.797932in}{3.097377in}}%
\pgfpathcurveto{\pgfqpoint{1.803756in}{3.103201in}}{\pgfqpoint{1.807028in}{3.111101in}}{\pgfqpoint{1.807028in}{3.119337in}}%
\pgfpathcurveto{\pgfqpoint{1.807028in}{3.127573in}}{\pgfqpoint{1.803756in}{3.135473in}}{\pgfqpoint{1.797932in}{3.141297in}}%
\pgfpathcurveto{\pgfqpoint{1.792108in}{3.147121in}}{\pgfqpoint{1.784208in}{3.150393in}}{\pgfqpoint{1.775972in}{3.150393in}}%
\pgfpathcurveto{\pgfqpoint{1.767736in}{3.150393in}}{\pgfqpoint{1.759836in}{3.147121in}}{\pgfqpoint{1.754012in}{3.141297in}}%
\pgfpathcurveto{\pgfqpoint{1.748188in}{3.135473in}}{\pgfqpoint{1.744915in}{3.127573in}}{\pgfqpoint{1.744915in}{3.119337in}}%
\pgfpathcurveto{\pgfqpoint{1.744915in}{3.111101in}}{\pgfqpoint{1.748188in}{3.103201in}}{\pgfqpoint{1.754012in}{3.097377in}}%
\pgfpathcurveto{\pgfqpoint{1.759836in}{3.091553in}}{\pgfqpoint{1.767736in}{3.088280in}}{\pgfqpoint{1.775972in}{3.088280in}}%
\pgfpathclose%
\pgfusepath{stroke,fill}%
\end{pgfscope}%
\begin{pgfscope}%
\pgfpathrectangle{\pgfqpoint{0.100000in}{0.212622in}}{\pgfqpoint{3.696000in}{3.696000in}}%
\pgfusepath{clip}%
\pgfsetbuttcap%
\pgfsetroundjoin%
\definecolor{currentfill}{rgb}{0.121569,0.466667,0.705882}%
\pgfsetfillcolor{currentfill}%
\pgfsetfillopacity{0.612896}%
\pgfsetlinewidth{1.003750pt}%
\definecolor{currentstroke}{rgb}{0.121569,0.466667,0.705882}%
\pgfsetstrokecolor{currentstroke}%
\pgfsetstrokeopacity{0.612896}%
\pgfsetdash{}{0pt}%
\pgfpathmoveto{\pgfqpoint{1.706306in}{3.077066in}}%
\pgfpathcurveto{\pgfqpoint{1.714542in}{3.077066in}}{\pgfqpoint{1.722442in}{3.080339in}}{\pgfqpoint{1.728266in}{3.086163in}}%
\pgfpathcurveto{\pgfqpoint{1.734090in}{3.091987in}}{\pgfqpoint{1.737362in}{3.099887in}}{\pgfqpoint{1.737362in}{3.108123in}}%
\pgfpathcurveto{\pgfqpoint{1.737362in}{3.116359in}}{\pgfqpoint{1.734090in}{3.124259in}}{\pgfqpoint{1.728266in}{3.130083in}}%
\pgfpathcurveto{\pgfqpoint{1.722442in}{3.135907in}}{\pgfqpoint{1.714542in}{3.139179in}}{\pgfqpoint{1.706306in}{3.139179in}}%
\pgfpathcurveto{\pgfqpoint{1.698069in}{3.139179in}}{\pgfqpoint{1.690169in}{3.135907in}}{\pgfqpoint{1.684345in}{3.130083in}}%
\pgfpathcurveto{\pgfqpoint{1.678521in}{3.124259in}}{\pgfqpoint{1.675249in}{3.116359in}}{\pgfqpoint{1.675249in}{3.108123in}}%
\pgfpathcurveto{\pgfqpoint{1.675249in}{3.099887in}}{\pgfqpoint{1.678521in}{3.091987in}}{\pgfqpoint{1.684345in}{3.086163in}}%
\pgfpathcurveto{\pgfqpoint{1.690169in}{3.080339in}}{\pgfqpoint{1.698069in}{3.077066in}}{\pgfqpoint{1.706306in}{3.077066in}}%
\pgfpathclose%
\pgfusepath{stroke,fill}%
\end{pgfscope}%
\begin{pgfscope}%
\pgfpathrectangle{\pgfqpoint{0.100000in}{0.212622in}}{\pgfqpoint{3.696000in}{3.696000in}}%
\pgfusepath{clip}%
\pgfsetbuttcap%
\pgfsetroundjoin%
\definecolor{currentfill}{rgb}{0.121569,0.466667,0.705882}%
\pgfsetfillcolor{currentfill}%
\pgfsetfillopacity{0.613460}%
\pgfsetlinewidth{1.003750pt}%
\definecolor{currentstroke}{rgb}{0.121569,0.466667,0.705882}%
\pgfsetstrokecolor{currentstroke}%
\pgfsetstrokeopacity{0.613460}%
\pgfsetdash{}{0pt}%
\pgfpathmoveto{\pgfqpoint{3.244710in}{1.998931in}}%
\pgfpathcurveto{\pgfqpoint{3.252947in}{1.998931in}}{\pgfqpoint{3.260847in}{2.002203in}}{\pgfqpoint{3.266671in}{2.008027in}}%
\pgfpathcurveto{\pgfqpoint{3.272494in}{2.013851in}}{\pgfqpoint{3.275767in}{2.021751in}}{\pgfqpoint{3.275767in}{2.029987in}}%
\pgfpathcurveto{\pgfqpoint{3.275767in}{2.038224in}}{\pgfqpoint{3.272494in}{2.046124in}}{\pgfqpoint{3.266671in}{2.051948in}}%
\pgfpathcurveto{\pgfqpoint{3.260847in}{2.057772in}}{\pgfqpoint{3.252947in}{2.061044in}}{\pgfqpoint{3.244710in}{2.061044in}}%
\pgfpathcurveto{\pgfqpoint{3.236474in}{2.061044in}}{\pgfqpoint{3.228574in}{2.057772in}}{\pgfqpoint{3.222750in}{2.051948in}}%
\pgfpathcurveto{\pgfqpoint{3.216926in}{2.046124in}}{\pgfqpoint{3.213654in}{2.038224in}}{\pgfqpoint{3.213654in}{2.029987in}}%
\pgfpathcurveto{\pgfqpoint{3.213654in}{2.021751in}}{\pgfqpoint{3.216926in}{2.013851in}}{\pgfqpoint{3.222750in}{2.008027in}}%
\pgfpathcurveto{\pgfqpoint{3.228574in}{2.002203in}}{\pgfqpoint{3.236474in}{1.998931in}}{\pgfqpoint{3.244710in}{1.998931in}}%
\pgfpathclose%
\pgfusepath{stroke,fill}%
\end{pgfscope}%
\begin{pgfscope}%
\pgfpathrectangle{\pgfqpoint{0.100000in}{0.212622in}}{\pgfqpoint{3.696000in}{3.696000in}}%
\pgfusepath{clip}%
\pgfsetbuttcap%
\pgfsetroundjoin%
\definecolor{currentfill}{rgb}{0.121569,0.466667,0.705882}%
\pgfsetfillcolor{currentfill}%
\pgfsetfillopacity{0.614835}%
\pgfsetlinewidth{1.003750pt}%
\definecolor{currentstroke}{rgb}{0.121569,0.466667,0.705882}%
\pgfsetstrokecolor{currentstroke}%
\pgfsetstrokeopacity{0.614835}%
\pgfsetdash{}{0pt}%
\pgfpathmoveto{\pgfqpoint{1.703198in}{3.071597in}}%
\pgfpathcurveto{\pgfqpoint{1.711435in}{3.071597in}}{\pgfqpoint{1.719335in}{3.074869in}}{\pgfqpoint{1.725159in}{3.080693in}}%
\pgfpathcurveto{\pgfqpoint{1.730982in}{3.086517in}}{\pgfqpoint{1.734255in}{3.094417in}}{\pgfqpoint{1.734255in}{3.102653in}}%
\pgfpathcurveto{\pgfqpoint{1.734255in}{3.110890in}}{\pgfqpoint{1.730982in}{3.118790in}}{\pgfqpoint{1.725159in}{3.124614in}}%
\pgfpathcurveto{\pgfqpoint{1.719335in}{3.130437in}}{\pgfqpoint{1.711435in}{3.133710in}}{\pgfqpoint{1.703198in}{3.133710in}}%
\pgfpathcurveto{\pgfqpoint{1.694962in}{3.133710in}}{\pgfqpoint{1.687062in}{3.130437in}}{\pgfqpoint{1.681238in}{3.124614in}}%
\pgfpathcurveto{\pgfqpoint{1.675414in}{3.118790in}}{\pgfqpoint{1.672142in}{3.110890in}}{\pgfqpoint{1.672142in}{3.102653in}}%
\pgfpathcurveto{\pgfqpoint{1.672142in}{3.094417in}}{\pgfqpoint{1.675414in}{3.086517in}}{\pgfqpoint{1.681238in}{3.080693in}}%
\pgfpathcurveto{\pgfqpoint{1.687062in}{3.074869in}}{\pgfqpoint{1.694962in}{3.071597in}}{\pgfqpoint{1.703198in}{3.071597in}}%
\pgfpathclose%
\pgfusepath{stroke,fill}%
\end{pgfscope}%
\begin{pgfscope}%
\pgfpathrectangle{\pgfqpoint{0.100000in}{0.212622in}}{\pgfqpoint{3.696000in}{3.696000in}}%
\pgfusepath{clip}%
\pgfsetbuttcap%
\pgfsetroundjoin%
\definecolor{currentfill}{rgb}{0.121569,0.466667,0.705882}%
\pgfsetfillcolor{currentfill}%
\pgfsetfillopacity{0.615251}%
\pgfsetlinewidth{1.003750pt}%
\definecolor{currentstroke}{rgb}{0.121569,0.466667,0.705882}%
\pgfsetstrokecolor{currentstroke}%
\pgfsetstrokeopacity{0.615251}%
\pgfsetdash{}{0pt}%
\pgfpathmoveto{\pgfqpoint{1.792748in}{3.085222in}}%
\pgfpathcurveto{\pgfqpoint{1.800985in}{3.085222in}}{\pgfqpoint{1.808885in}{3.088495in}}{\pgfqpoint{1.814709in}{3.094318in}}%
\pgfpathcurveto{\pgfqpoint{1.820532in}{3.100142in}}{\pgfqpoint{1.823805in}{3.108042in}}{\pgfqpoint{1.823805in}{3.116279in}}%
\pgfpathcurveto{\pgfqpoint{1.823805in}{3.124515in}}{\pgfqpoint{1.820532in}{3.132415in}}{\pgfqpoint{1.814709in}{3.138239in}}%
\pgfpathcurveto{\pgfqpoint{1.808885in}{3.144063in}}{\pgfqpoint{1.800985in}{3.147335in}}{\pgfqpoint{1.792748in}{3.147335in}}%
\pgfpathcurveto{\pgfqpoint{1.784512in}{3.147335in}}{\pgfqpoint{1.776612in}{3.144063in}}{\pgfqpoint{1.770788in}{3.138239in}}%
\pgfpathcurveto{\pgfqpoint{1.764964in}{3.132415in}}{\pgfqpoint{1.761692in}{3.124515in}}{\pgfqpoint{1.761692in}{3.116279in}}%
\pgfpathcurveto{\pgfqpoint{1.761692in}{3.108042in}}{\pgfqpoint{1.764964in}{3.100142in}}{\pgfqpoint{1.770788in}{3.094318in}}%
\pgfpathcurveto{\pgfqpoint{1.776612in}{3.088495in}}{\pgfqpoint{1.784512in}{3.085222in}}{\pgfqpoint{1.792748in}{3.085222in}}%
\pgfpathclose%
\pgfusepath{stroke,fill}%
\end{pgfscope}%
\begin{pgfscope}%
\pgfpathrectangle{\pgfqpoint{0.100000in}{0.212622in}}{\pgfqpoint{3.696000in}{3.696000in}}%
\pgfusepath{clip}%
\pgfsetbuttcap%
\pgfsetroundjoin%
\definecolor{currentfill}{rgb}{0.121569,0.466667,0.705882}%
\pgfsetfillcolor{currentfill}%
\pgfsetfillopacity{0.616014}%
\pgfsetlinewidth{1.003750pt}%
\definecolor{currentstroke}{rgb}{0.121569,0.466667,0.705882}%
\pgfsetstrokecolor{currentstroke}%
\pgfsetstrokeopacity{0.616014}%
\pgfsetdash{}{0pt}%
\pgfpathmoveto{\pgfqpoint{3.238845in}{1.990332in}}%
\pgfpathcurveto{\pgfqpoint{3.247081in}{1.990332in}}{\pgfqpoint{3.254981in}{1.993604in}}{\pgfqpoint{3.260805in}{1.999428in}}%
\pgfpathcurveto{\pgfqpoint{3.266629in}{2.005252in}}{\pgfqpoint{3.269901in}{2.013152in}}{\pgfqpoint{3.269901in}{2.021388in}}%
\pgfpathcurveto{\pgfqpoint{3.269901in}{2.029625in}}{\pgfqpoint{3.266629in}{2.037525in}}{\pgfqpoint{3.260805in}{2.043348in}}%
\pgfpathcurveto{\pgfqpoint{3.254981in}{2.049172in}}{\pgfqpoint{3.247081in}{2.052445in}}{\pgfqpoint{3.238845in}{2.052445in}}%
\pgfpathcurveto{\pgfqpoint{3.230608in}{2.052445in}}{\pgfqpoint{3.222708in}{2.049172in}}{\pgfqpoint{3.216884in}{2.043348in}}%
\pgfpathcurveto{\pgfqpoint{3.211060in}{2.037525in}}{\pgfqpoint{3.207788in}{2.029625in}}{\pgfqpoint{3.207788in}{2.021388in}}%
\pgfpathcurveto{\pgfqpoint{3.207788in}{2.013152in}}{\pgfqpoint{3.211060in}{2.005252in}}{\pgfqpoint{3.216884in}{1.999428in}}%
\pgfpathcurveto{\pgfqpoint{3.222708in}{1.993604in}}{\pgfqpoint{3.230608in}{1.990332in}}{\pgfqpoint{3.238845in}{1.990332in}}%
\pgfpathclose%
\pgfusepath{stroke,fill}%
\end{pgfscope}%
\begin{pgfscope}%
\pgfpathrectangle{\pgfqpoint{0.100000in}{0.212622in}}{\pgfqpoint{3.696000in}{3.696000in}}%
\pgfusepath{clip}%
\pgfsetbuttcap%
\pgfsetroundjoin%
\definecolor{currentfill}{rgb}{0.121569,0.466667,0.705882}%
\pgfsetfillcolor{currentfill}%
\pgfsetfillopacity{0.616868}%
\pgfsetlinewidth{1.003750pt}%
\definecolor{currentstroke}{rgb}{0.121569,0.466667,0.705882}%
\pgfsetstrokecolor{currentstroke}%
\pgfsetstrokeopacity{0.616868}%
\pgfsetdash{}{0pt}%
\pgfpathmoveto{\pgfqpoint{1.699985in}{3.065911in}}%
\pgfpathcurveto{\pgfqpoint{1.708221in}{3.065911in}}{\pgfqpoint{1.716121in}{3.069183in}}{\pgfqpoint{1.721945in}{3.075007in}}%
\pgfpathcurveto{\pgfqpoint{1.727769in}{3.080831in}}{\pgfqpoint{1.731042in}{3.088731in}}{\pgfqpoint{1.731042in}{3.096967in}}%
\pgfpathcurveto{\pgfqpoint{1.731042in}{3.105203in}}{\pgfqpoint{1.727769in}{3.113104in}}{\pgfqpoint{1.721945in}{3.118927in}}%
\pgfpathcurveto{\pgfqpoint{1.716121in}{3.124751in}}{\pgfqpoint{1.708221in}{3.128024in}}{\pgfqpoint{1.699985in}{3.128024in}}%
\pgfpathcurveto{\pgfqpoint{1.691749in}{3.128024in}}{\pgfqpoint{1.683849in}{3.124751in}}{\pgfqpoint{1.678025in}{3.118927in}}%
\pgfpathcurveto{\pgfqpoint{1.672201in}{3.113104in}}{\pgfqpoint{1.668929in}{3.105203in}}{\pgfqpoint{1.668929in}{3.096967in}}%
\pgfpathcurveto{\pgfqpoint{1.668929in}{3.088731in}}{\pgfqpoint{1.672201in}{3.080831in}}{\pgfqpoint{1.678025in}{3.075007in}}%
\pgfpathcurveto{\pgfqpoint{1.683849in}{3.069183in}}{\pgfqpoint{1.691749in}{3.065911in}}{\pgfqpoint{1.699985in}{3.065911in}}%
\pgfpathclose%
\pgfusepath{stroke,fill}%
\end{pgfscope}%
\begin{pgfscope}%
\pgfpathrectangle{\pgfqpoint{0.100000in}{0.212622in}}{\pgfqpoint{3.696000in}{3.696000in}}%
\pgfusepath{clip}%
\pgfsetbuttcap%
\pgfsetroundjoin%
\definecolor{currentfill}{rgb}{0.121569,0.466667,0.705882}%
\pgfsetfillcolor{currentfill}%
\pgfsetfillopacity{0.617792}%
\pgfsetlinewidth{1.003750pt}%
\definecolor{currentstroke}{rgb}{0.121569,0.466667,0.705882}%
\pgfsetstrokecolor{currentstroke}%
\pgfsetstrokeopacity{0.617792}%
\pgfsetdash{}{0pt}%
\pgfpathmoveto{\pgfqpoint{1.808987in}{3.082456in}}%
\pgfpathcurveto{\pgfqpoint{1.817223in}{3.082456in}}{\pgfqpoint{1.825123in}{3.085728in}}{\pgfqpoint{1.830947in}{3.091552in}}%
\pgfpathcurveto{\pgfqpoint{1.836771in}{3.097376in}}{\pgfqpoint{1.840044in}{3.105276in}}{\pgfqpoint{1.840044in}{3.113513in}}%
\pgfpathcurveto{\pgfqpoint{1.840044in}{3.121749in}}{\pgfqpoint{1.836771in}{3.129649in}}{\pgfqpoint{1.830947in}{3.135473in}}%
\pgfpathcurveto{\pgfqpoint{1.825123in}{3.141297in}}{\pgfqpoint{1.817223in}{3.144569in}}{\pgfqpoint{1.808987in}{3.144569in}}%
\pgfpathcurveto{\pgfqpoint{1.800751in}{3.144569in}}{\pgfqpoint{1.792851in}{3.141297in}}{\pgfqpoint{1.787027in}{3.135473in}}%
\pgfpathcurveto{\pgfqpoint{1.781203in}{3.129649in}}{\pgfqpoint{1.777931in}{3.121749in}}{\pgfqpoint{1.777931in}{3.113513in}}%
\pgfpathcurveto{\pgfqpoint{1.777931in}{3.105276in}}{\pgfqpoint{1.781203in}{3.097376in}}{\pgfqpoint{1.787027in}{3.091552in}}%
\pgfpathcurveto{\pgfqpoint{1.792851in}{3.085728in}}{\pgfqpoint{1.800751in}{3.082456in}}{\pgfqpoint{1.808987in}{3.082456in}}%
\pgfpathclose%
\pgfusepath{stroke,fill}%
\end{pgfscope}%
\begin{pgfscope}%
\pgfpathrectangle{\pgfqpoint{0.100000in}{0.212622in}}{\pgfqpoint{3.696000in}{3.696000in}}%
\pgfusepath{clip}%
\pgfsetbuttcap%
\pgfsetroundjoin%
\definecolor{currentfill}{rgb}{0.121569,0.466667,0.705882}%
\pgfsetfillcolor{currentfill}%
\pgfsetfillopacity{0.618160}%
\pgfsetlinewidth{1.003750pt}%
\definecolor{currentstroke}{rgb}{0.121569,0.466667,0.705882}%
\pgfsetstrokecolor{currentstroke}%
\pgfsetstrokeopacity{0.618160}%
\pgfsetdash{}{0pt}%
\pgfpathmoveto{\pgfqpoint{3.234035in}{1.983608in}}%
\pgfpathcurveto{\pgfqpoint{3.242271in}{1.983608in}}{\pgfqpoint{3.250171in}{1.986880in}}{\pgfqpoint{3.255995in}{1.992704in}}%
\pgfpathcurveto{\pgfqpoint{3.261819in}{1.998528in}}{\pgfqpoint{3.265092in}{2.006428in}}{\pgfqpoint{3.265092in}{2.014665in}}%
\pgfpathcurveto{\pgfqpoint{3.265092in}{2.022901in}}{\pgfqpoint{3.261819in}{2.030801in}}{\pgfqpoint{3.255995in}{2.036625in}}%
\pgfpathcurveto{\pgfqpoint{3.250171in}{2.042449in}}{\pgfqpoint{3.242271in}{2.045721in}}{\pgfqpoint{3.234035in}{2.045721in}}%
\pgfpathcurveto{\pgfqpoint{3.225799in}{2.045721in}}{\pgfqpoint{3.217899in}{2.042449in}}{\pgfqpoint{3.212075in}{2.036625in}}%
\pgfpathcurveto{\pgfqpoint{3.206251in}{2.030801in}}{\pgfqpoint{3.202979in}{2.022901in}}{\pgfqpoint{3.202979in}{2.014665in}}%
\pgfpathcurveto{\pgfqpoint{3.202979in}{2.006428in}}{\pgfqpoint{3.206251in}{1.998528in}}{\pgfqpoint{3.212075in}{1.992704in}}%
\pgfpathcurveto{\pgfqpoint{3.217899in}{1.986880in}}{\pgfqpoint{3.225799in}{1.983608in}}{\pgfqpoint{3.234035in}{1.983608in}}%
\pgfpathclose%
\pgfusepath{stroke,fill}%
\end{pgfscope}%
\begin{pgfscope}%
\pgfpathrectangle{\pgfqpoint{0.100000in}{0.212622in}}{\pgfqpoint{3.696000in}{3.696000in}}%
\pgfusepath{clip}%
\pgfsetbuttcap%
\pgfsetroundjoin%
\definecolor{currentfill}{rgb}{0.121569,0.466667,0.705882}%
\pgfsetfillcolor{currentfill}%
\pgfsetfillopacity{0.619173}%
\pgfsetlinewidth{1.003750pt}%
\definecolor{currentstroke}{rgb}{0.121569,0.466667,0.705882}%
\pgfsetstrokecolor{currentstroke}%
\pgfsetstrokeopacity{0.619173}%
\pgfsetdash{}{0pt}%
\pgfpathmoveto{\pgfqpoint{1.696268in}{3.059653in}}%
\pgfpathcurveto{\pgfqpoint{1.704504in}{3.059653in}}{\pgfqpoint{1.712404in}{3.062925in}}{\pgfqpoint{1.718228in}{3.068749in}}%
\pgfpathcurveto{\pgfqpoint{1.724052in}{3.074573in}}{\pgfqpoint{1.727325in}{3.082473in}}{\pgfqpoint{1.727325in}{3.090709in}}%
\pgfpathcurveto{\pgfqpoint{1.727325in}{3.098945in}}{\pgfqpoint{1.724052in}{3.106845in}}{\pgfqpoint{1.718228in}{3.112669in}}%
\pgfpathcurveto{\pgfqpoint{1.712404in}{3.118493in}}{\pgfqpoint{1.704504in}{3.121766in}}{\pgfqpoint{1.696268in}{3.121766in}}%
\pgfpathcurveto{\pgfqpoint{1.688032in}{3.121766in}}{\pgfqpoint{1.680132in}{3.118493in}}{\pgfqpoint{1.674308in}{3.112669in}}%
\pgfpathcurveto{\pgfqpoint{1.668484in}{3.106845in}}{\pgfqpoint{1.665212in}{3.098945in}}{\pgfqpoint{1.665212in}{3.090709in}}%
\pgfpathcurveto{\pgfqpoint{1.665212in}{3.082473in}}{\pgfqpoint{1.668484in}{3.074573in}}{\pgfqpoint{1.674308in}{3.068749in}}%
\pgfpathcurveto{\pgfqpoint{1.680132in}{3.062925in}}{\pgfqpoint{1.688032in}{3.059653in}}{\pgfqpoint{1.696268in}{3.059653in}}%
\pgfpathclose%
\pgfusepath{stroke,fill}%
\end{pgfscope}%
\begin{pgfscope}%
\pgfpathrectangle{\pgfqpoint{0.100000in}{0.212622in}}{\pgfqpoint{3.696000in}{3.696000in}}%
\pgfusepath{clip}%
\pgfsetbuttcap%
\pgfsetroundjoin%
\definecolor{currentfill}{rgb}{0.121569,0.466667,0.705882}%
\pgfsetfillcolor{currentfill}%
\pgfsetfillopacity{0.620032}%
\pgfsetlinewidth{1.003750pt}%
\definecolor{currentstroke}{rgb}{0.121569,0.466667,0.705882}%
\pgfsetstrokecolor{currentstroke}%
\pgfsetstrokeopacity{0.620032}%
\pgfsetdash{}{0pt}%
\pgfpathmoveto{\pgfqpoint{3.229854in}{1.976530in}}%
\pgfpathcurveto{\pgfqpoint{3.238091in}{1.976530in}}{\pgfqpoint{3.245991in}{1.979803in}}{\pgfqpoint{3.251815in}{1.985626in}}%
\pgfpathcurveto{\pgfqpoint{3.257639in}{1.991450in}}{\pgfqpoint{3.260911in}{1.999350in}}{\pgfqpoint{3.260911in}{2.007587in}}%
\pgfpathcurveto{\pgfqpoint{3.260911in}{2.015823in}}{\pgfqpoint{3.257639in}{2.023723in}}{\pgfqpoint{3.251815in}{2.029547in}}%
\pgfpathcurveto{\pgfqpoint{3.245991in}{2.035371in}}{\pgfqpoint{3.238091in}{2.038643in}}{\pgfqpoint{3.229854in}{2.038643in}}%
\pgfpathcurveto{\pgfqpoint{3.221618in}{2.038643in}}{\pgfqpoint{3.213718in}{2.035371in}}{\pgfqpoint{3.207894in}{2.029547in}}%
\pgfpathcurveto{\pgfqpoint{3.202070in}{2.023723in}}{\pgfqpoint{3.198798in}{2.015823in}}{\pgfqpoint{3.198798in}{2.007587in}}%
\pgfpathcurveto{\pgfqpoint{3.198798in}{1.999350in}}{\pgfqpoint{3.202070in}{1.991450in}}{\pgfqpoint{3.207894in}{1.985626in}}%
\pgfpathcurveto{\pgfqpoint{3.213718in}{1.979803in}}{\pgfqpoint{3.221618in}{1.976530in}}{\pgfqpoint{3.229854in}{1.976530in}}%
\pgfpathclose%
\pgfusepath{stroke,fill}%
\end{pgfscope}%
\begin{pgfscope}%
\pgfpathrectangle{\pgfqpoint{0.100000in}{0.212622in}}{\pgfqpoint{3.696000in}{3.696000in}}%
\pgfusepath{clip}%
\pgfsetbuttcap%
\pgfsetroundjoin%
\definecolor{currentfill}{rgb}{0.121569,0.466667,0.705882}%
\pgfsetfillcolor{currentfill}%
\pgfsetfillopacity{0.620070}%
\pgfsetlinewidth{1.003750pt}%
\definecolor{currentstroke}{rgb}{0.121569,0.466667,0.705882}%
\pgfsetstrokecolor{currentstroke}%
\pgfsetstrokeopacity{0.620070}%
\pgfsetdash{}{0pt}%
\pgfpathmoveto{\pgfqpoint{1.824312in}{3.079895in}}%
\pgfpathcurveto{\pgfqpoint{1.832548in}{3.079895in}}{\pgfqpoint{1.840448in}{3.083167in}}{\pgfqpoint{1.846272in}{3.088991in}}%
\pgfpathcurveto{\pgfqpoint{1.852096in}{3.094815in}}{\pgfqpoint{1.855368in}{3.102715in}}{\pgfqpoint{1.855368in}{3.110951in}}%
\pgfpathcurveto{\pgfqpoint{1.855368in}{3.119187in}}{\pgfqpoint{1.852096in}{3.127087in}}{\pgfqpoint{1.846272in}{3.132911in}}%
\pgfpathcurveto{\pgfqpoint{1.840448in}{3.138735in}}{\pgfqpoint{1.832548in}{3.142008in}}{\pgfqpoint{1.824312in}{3.142008in}}%
\pgfpathcurveto{\pgfqpoint{1.816075in}{3.142008in}}{\pgfqpoint{1.808175in}{3.138735in}}{\pgfqpoint{1.802351in}{3.132911in}}%
\pgfpathcurveto{\pgfqpoint{1.796527in}{3.127087in}}{\pgfqpoint{1.793255in}{3.119187in}}{\pgfqpoint{1.793255in}{3.110951in}}%
\pgfpathcurveto{\pgfqpoint{1.793255in}{3.102715in}}{\pgfqpoint{1.796527in}{3.094815in}}{\pgfqpoint{1.802351in}{3.088991in}}%
\pgfpathcurveto{\pgfqpoint{1.808175in}{3.083167in}}{\pgfqpoint{1.816075in}{3.079895in}}{\pgfqpoint{1.824312in}{3.079895in}}%
\pgfpathclose%
\pgfusepath{stroke,fill}%
\end{pgfscope}%
\begin{pgfscope}%
\pgfpathrectangle{\pgfqpoint{0.100000in}{0.212622in}}{\pgfqpoint{3.696000in}{3.696000in}}%
\pgfusepath{clip}%
\pgfsetbuttcap%
\pgfsetroundjoin%
\definecolor{currentfill}{rgb}{0.121569,0.466667,0.705882}%
\pgfsetfillcolor{currentfill}%
\pgfsetfillopacity{0.621658}%
\pgfsetlinewidth{1.003750pt}%
\definecolor{currentstroke}{rgb}{0.121569,0.466667,0.705882}%
\pgfsetstrokecolor{currentstroke}%
\pgfsetstrokeopacity{0.621658}%
\pgfsetdash{}{0pt}%
\pgfpathmoveto{\pgfqpoint{3.226161in}{1.969813in}}%
\pgfpathcurveto{\pgfqpoint{3.234397in}{1.969813in}}{\pgfqpoint{3.242297in}{1.973086in}}{\pgfqpoint{3.248121in}{1.978910in}}%
\pgfpathcurveto{\pgfqpoint{3.253945in}{1.984734in}}{\pgfqpoint{3.257218in}{1.992634in}}{\pgfqpoint{3.257218in}{2.000870in}}%
\pgfpathcurveto{\pgfqpoint{3.257218in}{2.009106in}}{\pgfqpoint{3.253945in}{2.017006in}}{\pgfqpoint{3.248121in}{2.022830in}}%
\pgfpathcurveto{\pgfqpoint{3.242297in}{2.028654in}}{\pgfqpoint{3.234397in}{2.031926in}}{\pgfqpoint{3.226161in}{2.031926in}}%
\pgfpathcurveto{\pgfqpoint{3.217925in}{2.031926in}}{\pgfqpoint{3.210025in}{2.028654in}}{\pgfqpoint{3.204201in}{2.022830in}}%
\pgfpathcurveto{\pgfqpoint{3.198377in}{2.017006in}}{\pgfqpoint{3.195105in}{2.009106in}}{\pgfqpoint{3.195105in}{2.000870in}}%
\pgfpathcurveto{\pgfqpoint{3.195105in}{1.992634in}}{\pgfqpoint{3.198377in}{1.984734in}}{\pgfqpoint{3.204201in}{1.978910in}}%
\pgfpathcurveto{\pgfqpoint{3.210025in}{1.973086in}}{\pgfqpoint{3.217925in}{1.969813in}}{\pgfqpoint{3.226161in}{1.969813in}}%
\pgfpathclose%
\pgfusepath{stroke,fill}%
\end{pgfscope}%
\begin{pgfscope}%
\pgfpathrectangle{\pgfqpoint{0.100000in}{0.212622in}}{\pgfqpoint{3.696000in}{3.696000in}}%
\pgfusepath{clip}%
\pgfsetbuttcap%
\pgfsetroundjoin%
\definecolor{currentfill}{rgb}{0.121569,0.466667,0.705882}%
\pgfsetfillcolor{currentfill}%
\pgfsetfillopacity{0.621747}%
\pgfsetlinewidth{1.003750pt}%
\definecolor{currentstroke}{rgb}{0.121569,0.466667,0.705882}%
\pgfsetstrokecolor{currentstroke}%
\pgfsetstrokeopacity{0.621747}%
\pgfsetdash{}{0pt}%
\pgfpathmoveto{\pgfqpoint{1.692183in}{3.052301in}}%
\pgfpathcurveto{\pgfqpoint{1.700420in}{3.052301in}}{\pgfqpoint{1.708320in}{3.055573in}}{\pgfqpoint{1.714144in}{3.061397in}}%
\pgfpathcurveto{\pgfqpoint{1.719968in}{3.067221in}}{\pgfqpoint{1.723240in}{3.075121in}}{\pgfqpoint{1.723240in}{3.083358in}}%
\pgfpathcurveto{\pgfqpoint{1.723240in}{3.091594in}}{\pgfqpoint{1.719968in}{3.099494in}}{\pgfqpoint{1.714144in}{3.105318in}}%
\pgfpathcurveto{\pgfqpoint{1.708320in}{3.111142in}}{\pgfqpoint{1.700420in}{3.114414in}}{\pgfqpoint{1.692183in}{3.114414in}}%
\pgfpathcurveto{\pgfqpoint{1.683947in}{3.114414in}}{\pgfqpoint{1.676047in}{3.111142in}}{\pgfqpoint{1.670223in}{3.105318in}}%
\pgfpathcurveto{\pgfqpoint{1.664399in}{3.099494in}}{\pgfqpoint{1.661127in}{3.091594in}}{\pgfqpoint{1.661127in}{3.083358in}}%
\pgfpathcurveto{\pgfqpoint{1.661127in}{3.075121in}}{\pgfqpoint{1.664399in}{3.067221in}}{\pgfqpoint{1.670223in}{3.061397in}}%
\pgfpathcurveto{\pgfqpoint{1.676047in}{3.055573in}}{\pgfqpoint{1.683947in}{3.052301in}}{\pgfqpoint{1.692183in}{3.052301in}}%
\pgfpathclose%
\pgfusepath{stroke,fill}%
\end{pgfscope}%
\begin{pgfscope}%
\pgfpathrectangle{\pgfqpoint{0.100000in}{0.212622in}}{\pgfqpoint{3.696000in}{3.696000in}}%
\pgfusepath{clip}%
\pgfsetbuttcap%
\pgfsetroundjoin%
\definecolor{currentfill}{rgb}{0.121569,0.466667,0.705882}%
\pgfsetfillcolor{currentfill}%
\pgfsetfillopacity{0.622238}%
\pgfsetlinewidth{1.003750pt}%
\definecolor{currentstroke}{rgb}{0.121569,0.466667,0.705882}%
\pgfsetstrokecolor{currentstroke}%
\pgfsetstrokeopacity{0.622238}%
\pgfsetdash{}{0pt}%
\pgfpathmoveto{\pgfqpoint{1.839189in}{3.077698in}}%
\pgfpathcurveto{\pgfqpoint{1.847425in}{3.077698in}}{\pgfqpoint{1.855325in}{3.080970in}}{\pgfqpoint{1.861149in}{3.086794in}}%
\pgfpathcurveto{\pgfqpoint{1.866973in}{3.092618in}}{\pgfqpoint{1.870245in}{3.100518in}}{\pgfqpoint{1.870245in}{3.108755in}}%
\pgfpathcurveto{\pgfqpoint{1.870245in}{3.116991in}}{\pgfqpoint{1.866973in}{3.124891in}}{\pgfqpoint{1.861149in}{3.130715in}}%
\pgfpathcurveto{\pgfqpoint{1.855325in}{3.136539in}}{\pgfqpoint{1.847425in}{3.139811in}}{\pgfqpoint{1.839189in}{3.139811in}}%
\pgfpathcurveto{\pgfqpoint{1.830953in}{3.139811in}}{\pgfqpoint{1.823053in}{3.136539in}}{\pgfqpoint{1.817229in}{3.130715in}}%
\pgfpathcurveto{\pgfqpoint{1.811405in}{3.124891in}}{\pgfqpoint{1.808132in}{3.116991in}}{\pgfqpoint{1.808132in}{3.108755in}}%
\pgfpathcurveto{\pgfqpoint{1.808132in}{3.100518in}}{\pgfqpoint{1.811405in}{3.092618in}}{\pgfqpoint{1.817229in}{3.086794in}}%
\pgfpathcurveto{\pgfqpoint{1.823053in}{3.080970in}}{\pgfqpoint{1.830953in}{3.077698in}}{\pgfqpoint{1.839189in}{3.077698in}}%
\pgfpathclose%
\pgfusepath{stroke,fill}%
\end{pgfscope}%
\begin{pgfscope}%
\pgfpathrectangle{\pgfqpoint{0.100000in}{0.212622in}}{\pgfqpoint{3.696000in}{3.696000in}}%
\pgfusepath{clip}%
\pgfsetbuttcap%
\pgfsetroundjoin%
\definecolor{currentfill}{rgb}{0.121569,0.466667,0.705882}%
\pgfsetfillcolor{currentfill}%
\pgfsetfillopacity{0.623041}%
\pgfsetlinewidth{1.003750pt}%
\definecolor{currentstroke}{rgb}{0.121569,0.466667,0.705882}%
\pgfsetstrokecolor{currentstroke}%
\pgfsetstrokeopacity{0.623041}%
\pgfsetdash{}{0pt}%
\pgfpathmoveto{\pgfqpoint{3.222910in}{1.963851in}}%
\pgfpathcurveto{\pgfqpoint{3.231146in}{1.963851in}}{\pgfqpoint{3.239046in}{1.967123in}}{\pgfqpoint{3.244870in}{1.972947in}}%
\pgfpathcurveto{\pgfqpoint{3.250694in}{1.978771in}}{\pgfqpoint{3.253967in}{1.986671in}}{\pgfqpoint{3.253967in}{1.994907in}}%
\pgfpathcurveto{\pgfqpoint{3.253967in}{2.003144in}}{\pgfqpoint{3.250694in}{2.011044in}}{\pgfqpoint{3.244870in}{2.016868in}}%
\pgfpathcurveto{\pgfqpoint{3.239046in}{2.022691in}}{\pgfqpoint{3.231146in}{2.025964in}}{\pgfqpoint{3.222910in}{2.025964in}}%
\pgfpathcurveto{\pgfqpoint{3.214674in}{2.025964in}}{\pgfqpoint{3.206774in}{2.022691in}}{\pgfqpoint{3.200950in}{2.016868in}}%
\pgfpathcurveto{\pgfqpoint{3.195126in}{2.011044in}}{\pgfqpoint{3.191854in}{2.003144in}}{\pgfqpoint{3.191854in}{1.994907in}}%
\pgfpathcurveto{\pgfqpoint{3.191854in}{1.986671in}}{\pgfqpoint{3.195126in}{1.978771in}}{\pgfqpoint{3.200950in}{1.972947in}}%
\pgfpathcurveto{\pgfqpoint{3.206774in}{1.967123in}}{\pgfqpoint{3.214674in}{1.963851in}}{\pgfqpoint{3.222910in}{1.963851in}}%
\pgfpathclose%
\pgfusepath{stroke,fill}%
\end{pgfscope}%
\begin{pgfscope}%
\pgfpathrectangle{\pgfqpoint{0.100000in}{0.212622in}}{\pgfqpoint{3.696000in}{3.696000in}}%
\pgfusepath{clip}%
\pgfsetbuttcap%
\pgfsetroundjoin%
\definecolor{currentfill}{rgb}{0.121569,0.466667,0.705882}%
\pgfsetfillcolor{currentfill}%
\pgfsetfillopacity{0.624175}%
\pgfsetlinewidth{1.003750pt}%
\definecolor{currentstroke}{rgb}{0.121569,0.466667,0.705882}%
\pgfsetstrokecolor{currentstroke}%
\pgfsetstrokeopacity{0.624175}%
\pgfsetdash{}{0pt}%
\pgfpathmoveto{\pgfqpoint{1.853227in}{3.075012in}}%
\pgfpathcurveto{\pgfqpoint{1.861463in}{3.075012in}}{\pgfqpoint{1.869363in}{3.078284in}}{\pgfqpoint{1.875187in}{3.084108in}}%
\pgfpathcurveto{\pgfqpoint{1.881011in}{3.089932in}}{\pgfqpoint{1.884283in}{3.097832in}}{\pgfqpoint{1.884283in}{3.106068in}}%
\pgfpathcurveto{\pgfqpoint{1.884283in}{3.114304in}}{\pgfqpoint{1.881011in}{3.122204in}}{\pgfqpoint{1.875187in}{3.128028in}}%
\pgfpathcurveto{\pgfqpoint{1.869363in}{3.133852in}}{\pgfqpoint{1.861463in}{3.137125in}}{\pgfqpoint{1.853227in}{3.137125in}}%
\pgfpathcurveto{\pgfqpoint{1.844991in}{3.137125in}}{\pgfqpoint{1.837091in}{3.133852in}}{\pgfqpoint{1.831267in}{3.128028in}}%
\pgfpathcurveto{\pgfqpoint{1.825443in}{3.122204in}}{\pgfqpoint{1.822170in}{3.114304in}}{\pgfqpoint{1.822170in}{3.106068in}}%
\pgfpathcurveto{\pgfqpoint{1.822170in}{3.097832in}}{\pgfqpoint{1.825443in}{3.089932in}}{\pgfqpoint{1.831267in}{3.084108in}}%
\pgfpathcurveto{\pgfqpoint{1.837091in}{3.078284in}}{\pgfqpoint{1.844991in}{3.075012in}}{\pgfqpoint{1.853227in}{3.075012in}}%
\pgfpathclose%
\pgfusepath{stroke,fill}%
\end{pgfscope}%
\begin{pgfscope}%
\pgfpathrectangle{\pgfqpoint{0.100000in}{0.212622in}}{\pgfqpoint{3.696000in}{3.696000in}}%
\pgfusepath{clip}%
\pgfsetbuttcap%
\pgfsetroundjoin%
\definecolor{currentfill}{rgb}{0.121569,0.466667,0.705882}%
\pgfsetfillcolor{currentfill}%
\pgfsetfillopacity{0.624445}%
\pgfsetlinewidth{1.003750pt}%
\definecolor{currentstroke}{rgb}{0.121569,0.466667,0.705882}%
\pgfsetstrokecolor{currentstroke}%
\pgfsetstrokeopacity{0.624445}%
\pgfsetdash{}{0pt}%
\pgfpathmoveto{\pgfqpoint{3.219756in}{1.958758in}}%
\pgfpathcurveto{\pgfqpoint{3.227992in}{1.958758in}}{\pgfqpoint{3.235892in}{1.962030in}}{\pgfqpoint{3.241716in}{1.967854in}}%
\pgfpathcurveto{\pgfqpoint{3.247540in}{1.973678in}}{\pgfqpoint{3.250812in}{1.981578in}}{\pgfqpoint{3.250812in}{1.989814in}}%
\pgfpathcurveto{\pgfqpoint{3.250812in}{1.998050in}}{\pgfqpoint{3.247540in}{2.005950in}}{\pgfqpoint{3.241716in}{2.011774in}}%
\pgfpathcurveto{\pgfqpoint{3.235892in}{2.017598in}}{\pgfqpoint{3.227992in}{2.020871in}}{\pgfqpoint{3.219756in}{2.020871in}}%
\pgfpathcurveto{\pgfqpoint{3.211519in}{2.020871in}}{\pgfqpoint{3.203619in}{2.017598in}}{\pgfqpoint{3.197795in}{2.011774in}}%
\pgfpathcurveto{\pgfqpoint{3.191971in}{2.005950in}}{\pgfqpoint{3.188699in}{1.998050in}}{\pgfqpoint{3.188699in}{1.989814in}}%
\pgfpathcurveto{\pgfqpoint{3.188699in}{1.981578in}}{\pgfqpoint{3.191971in}{1.973678in}}{\pgfqpoint{3.197795in}{1.967854in}}%
\pgfpathcurveto{\pgfqpoint{3.203619in}{1.962030in}}{\pgfqpoint{3.211519in}{1.958758in}}{\pgfqpoint{3.219756in}{1.958758in}}%
\pgfpathclose%
\pgfusepath{stroke,fill}%
\end{pgfscope}%
\begin{pgfscope}%
\pgfpathrectangle{\pgfqpoint{0.100000in}{0.212622in}}{\pgfqpoint{3.696000in}{3.696000in}}%
\pgfusepath{clip}%
\pgfsetbuttcap%
\pgfsetroundjoin%
\definecolor{currentfill}{rgb}{0.121569,0.466667,0.705882}%
\pgfsetfillcolor{currentfill}%
\pgfsetfillopacity{0.624525}%
\pgfsetlinewidth{1.003750pt}%
\definecolor{currentstroke}{rgb}{0.121569,0.466667,0.705882}%
\pgfsetstrokecolor{currentstroke}%
\pgfsetstrokeopacity{0.624525}%
\pgfsetdash{}{0pt}%
\pgfpathmoveto{\pgfqpoint{1.688052in}{3.044575in}}%
\pgfpathcurveto{\pgfqpoint{1.696289in}{3.044575in}}{\pgfqpoint{1.704189in}{3.047847in}}{\pgfqpoint{1.710013in}{3.053671in}}%
\pgfpathcurveto{\pgfqpoint{1.715836in}{3.059495in}}{\pgfqpoint{1.719109in}{3.067395in}}{\pgfqpoint{1.719109in}{3.075631in}}%
\pgfpathcurveto{\pgfqpoint{1.719109in}{3.083868in}}{\pgfqpoint{1.715836in}{3.091768in}}{\pgfqpoint{1.710013in}{3.097592in}}%
\pgfpathcurveto{\pgfqpoint{1.704189in}{3.103416in}}{\pgfqpoint{1.696289in}{3.106688in}}{\pgfqpoint{1.688052in}{3.106688in}}%
\pgfpathcurveto{\pgfqpoint{1.679816in}{3.106688in}}{\pgfqpoint{1.671916in}{3.103416in}}{\pgfqpoint{1.666092in}{3.097592in}}%
\pgfpathcurveto{\pgfqpoint{1.660268in}{3.091768in}}{\pgfqpoint{1.656996in}{3.083868in}}{\pgfqpoint{1.656996in}{3.075631in}}%
\pgfpathcurveto{\pgfqpoint{1.656996in}{3.067395in}}{\pgfqpoint{1.660268in}{3.059495in}}{\pgfqpoint{1.666092in}{3.053671in}}%
\pgfpathcurveto{\pgfqpoint{1.671916in}{3.047847in}}{\pgfqpoint{1.679816in}{3.044575in}}{\pgfqpoint{1.688052in}{3.044575in}}%
\pgfpathclose%
\pgfusepath{stroke,fill}%
\end{pgfscope}%
\begin{pgfscope}%
\pgfpathrectangle{\pgfqpoint{0.100000in}{0.212622in}}{\pgfqpoint{3.696000in}{3.696000in}}%
\pgfusepath{clip}%
\pgfsetbuttcap%
\pgfsetroundjoin%
\definecolor{currentfill}{rgb}{0.121569,0.466667,0.705882}%
\pgfsetfillcolor{currentfill}%
\pgfsetfillopacity{0.625725}%
\pgfsetlinewidth{1.003750pt}%
\definecolor{currentstroke}{rgb}{0.121569,0.466667,0.705882}%
\pgfsetstrokecolor{currentstroke}%
\pgfsetstrokeopacity{0.625725}%
\pgfsetdash{}{0pt}%
\pgfpathmoveto{\pgfqpoint{3.216934in}{1.954357in}}%
\pgfpathcurveto{\pgfqpoint{3.225170in}{1.954357in}}{\pgfqpoint{3.233070in}{1.957630in}}{\pgfqpoint{3.238894in}{1.963454in}}%
\pgfpathcurveto{\pgfqpoint{3.244718in}{1.969278in}}{\pgfqpoint{3.247990in}{1.977178in}}{\pgfqpoint{3.247990in}{1.985414in}}%
\pgfpathcurveto{\pgfqpoint{3.247990in}{1.993650in}}{\pgfqpoint{3.244718in}{2.001550in}}{\pgfqpoint{3.238894in}{2.007374in}}%
\pgfpathcurveto{\pgfqpoint{3.233070in}{2.013198in}}{\pgfqpoint{3.225170in}{2.016470in}}{\pgfqpoint{3.216934in}{2.016470in}}%
\pgfpathcurveto{\pgfqpoint{3.208697in}{2.016470in}}{\pgfqpoint{3.200797in}{2.013198in}}{\pgfqpoint{3.194974in}{2.007374in}}%
\pgfpathcurveto{\pgfqpoint{3.189150in}{2.001550in}}{\pgfqpoint{3.185877in}{1.993650in}}{\pgfqpoint{3.185877in}{1.985414in}}%
\pgfpathcurveto{\pgfqpoint{3.185877in}{1.977178in}}{\pgfqpoint{3.189150in}{1.969278in}}{\pgfqpoint{3.194974in}{1.963454in}}%
\pgfpathcurveto{\pgfqpoint{3.200797in}{1.957630in}}{\pgfqpoint{3.208697in}{1.954357in}}{\pgfqpoint{3.216934in}{1.954357in}}%
\pgfpathclose%
\pgfusepath{stroke,fill}%
\end{pgfscope}%
\begin{pgfscope}%
\pgfpathrectangle{\pgfqpoint{0.100000in}{0.212622in}}{\pgfqpoint{3.696000in}{3.696000in}}%
\pgfusepath{clip}%
\pgfsetbuttcap%
\pgfsetroundjoin%
\definecolor{currentfill}{rgb}{0.121569,0.466667,0.705882}%
\pgfsetfillcolor{currentfill}%
\pgfsetfillopacity{0.626059}%
\pgfsetlinewidth{1.003750pt}%
\definecolor{currentstroke}{rgb}{0.121569,0.466667,0.705882}%
\pgfsetstrokecolor{currentstroke}%
\pgfsetstrokeopacity{0.626059}%
\pgfsetdash{}{0pt}%
\pgfpathmoveto{\pgfqpoint{1.866866in}{3.072204in}}%
\pgfpathcurveto{\pgfqpoint{1.875102in}{3.072204in}}{\pgfqpoint{1.883002in}{3.075476in}}{\pgfqpoint{1.888826in}{3.081300in}}%
\pgfpathcurveto{\pgfqpoint{1.894650in}{3.087124in}}{\pgfqpoint{1.897923in}{3.095024in}}{\pgfqpoint{1.897923in}{3.103261in}}%
\pgfpathcurveto{\pgfqpoint{1.897923in}{3.111497in}}{\pgfqpoint{1.894650in}{3.119397in}}{\pgfqpoint{1.888826in}{3.125221in}}%
\pgfpathcurveto{\pgfqpoint{1.883002in}{3.131045in}}{\pgfqpoint{1.875102in}{3.134317in}}{\pgfqpoint{1.866866in}{3.134317in}}%
\pgfpathcurveto{\pgfqpoint{1.858630in}{3.134317in}}{\pgfqpoint{1.850730in}{3.131045in}}{\pgfqpoint{1.844906in}{3.125221in}}%
\pgfpathcurveto{\pgfqpoint{1.839082in}{3.119397in}}{\pgfqpoint{1.835810in}{3.111497in}}{\pgfqpoint{1.835810in}{3.103261in}}%
\pgfpathcurveto{\pgfqpoint{1.835810in}{3.095024in}}{\pgfqpoint{1.839082in}{3.087124in}}{\pgfqpoint{1.844906in}{3.081300in}}%
\pgfpathcurveto{\pgfqpoint{1.850730in}{3.075476in}}{\pgfqpoint{1.858630in}{3.072204in}}{\pgfqpoint{1.866866in}{3.072204in}}%
\pgfpathclose%
\pgfusepath{stroke,fill}%
\end{pgfscope}%
\begin{pgfscope}%
\pgfpathrectangle{\pgfqpoint{0.100000in}{0.212622in}}{\pgfqpoint{3.696000in}{3.696000in}}%
\pgfusepath{clip}%
\pgfsetbuttcap%
\pgfsetroundjoin%
\definecolor{currentfill}{rgb}{0.121569,0.466667,0.705882}%
\pgfsetfillcolor{currentfill}%
\pgfsetfillopacity{0.626627}%
\pgfsetlinewidth{1.003750pt}%
\definecolor{currentstroke}{rgb}{0.121569,0.466667,0.705882}%
\pgfsetstrokecolor{currentstroke}%
\pgfsetstrokeopacity{0.626627}%
\pgfsetdash{}{0pt}%
\pgfpathmoveto{\pgfqpoint{3.214972in}{1.951378in}}%
\pgfpathcurveto{\pgfqpoint{3.223208in}{1.951378in}}{\pgfqpoint{3.231108in}{1.954650in}}{\pgfqpoint{3.236932in}{1.960474in}}%
\pgfpathcurveto{\pgfqpoint{3.242756in}{1.966298in}}{\pgfqpoint{3.246028in}{1.974198in}}{\pgfqpoint{3.246028in}{1.982434in}}%
\pgfpathcurveto{\pgfqpoint{3.246028in}{1.990671in}}{\pgfqpoint{3.242756in}{1.998571in}}{\pgfqpoint{3.236932in}{2.004395in}}%
\pgfpathcurveto{\pgfqpoint{3.231108in}{2.010218in}}{\pgfqpoint{3.223208in}{2.013491in}}{\pgfqpoint{3.214972in}{2.013491in}}%
\pgfpathcurveto{\pgfqpoint{3.206736in}{2.013491in}}{\pgfqpoint{3.198836in}{2.010218in}}{\pgfqpoint{3.193012in}{2.004395in}}%
\pgfpathcurveto{\pgfqpoint{3.187188in}{1.998571in}}{\pgfqpoint{3.183915in}{1.990671in}}{\pgfqpoint{3.183915in}{1.982434in}}%
\pgfpathcurveto{\pgfqpoint{3.183915in}{1.974198in}}{\pgfqpoint{3.187188in}{1.966298in}}{\pgfqpoint{3.193012in}{1.960474in}}%
\pgfpathcurveto{\pgfqpoint{3.198836in}{1.954650in}}{\pgfqpoint{3.206736in}{1.951378in}}{\pgfqpoint{3.214972in}{1.951378in}}%
\pgfpathclose%
\pgfusepath{stroke,fill}%
\end{pgfscope}%
\begin{pgfscope}%
\pgfpathrectangle{\pgfqpoint{0.100000in}{0.212622in}}{\pgfqpoint{3.696000in}{3.696000in}}%
\pgfusepath{clip}%
\pgfsetbuttcap%
\pgfsetroundjoin%
\definecolor{currentfill}{rgb}{0.121569,0.466667,0.705882}%
\pgfsetfillcolor{currentfill}%
\pgfsetfillopacity{0.627616}%
\pgfsetlinewidth{1.003750pt}%
\definecolor{currentstroke}{rgb}{0.121569,0.466667,0.705882}%
\pgfsetstrokecolor{currentstroke}%
\pgfsetstrokeopacity{0.627616}%
\pgfsetdash{}{0pt}%
\pgfpathmoveto{\pgfqpoint{1.683437in}{3.036784in}}%
\pgfpathcurveto{\pgfqpoint{1.691673in}{3.036784in}}{\pgfqpoint{1.699573in}{3.040056in}}{\pgfqpoint{1.705397in}{3.045880in}}%
\pgfpathcurveto{\pgfqpoint{1.711221in}{3.051704in}}{\pgfqpoint{1.714493in}{3.059604in}}{\pgfqpoint{1.714493in}{3.067840in}}%
\pgfpathcurveto{\pgfqpoint{1.714493in}{3.076076in}}{\pgfqpoint{1.711221in}{3.083976in}}{\pgfqpoint{1.705397in}{3.089800in}}%
\pgfpathcurveto{\pgfqpoint{1.699573in}{3.095624in}}{\pgfqpoint{1.691673in}{3.098897in}}{\pgfqpoint{1.683437in}{3.098897in}}%
\pgfpathcurveto{\pgfqpoint{1.675201in}{3.098897in}}{\pgfqpoint{1.667301in}{3.095624in}}{\pgfqpoint{1.661477in}{3.089800in}}%
\pgfpathcurveto{\pgfqpoint{1.655653in}{3.083976in}}{\pgfqpoint{1.652380in}{3.076076in}}{\pgfqpoint{1.652380in}{3.067840in}}%
\pgfpathcurveto{\pgfqpoint{1.652380in}{3.059604in}}{\pgfqpoint{1.655653in}{3.051704in}}{\pgfqpoint{1.661477in}{3.045880in}}%
\pgfpathcurveto{\pgfqpoint{1.667301in}{3.040056in}}{\pgfqpoint{1.675201in}{3.036784in}}{\pgfqpoint{1.683437in}{3.036784in}}%
\pgfpathclose%
\pgfusepath{stroke,fill}%
\end{pgfscope}%
\begin{pgfscope}%
\pgfpathrectangle{\pgfqpoint{0.100000in}{0.212622in}}{\pgfqpoint{3.696000in}{3.696000in}}%
\pgfusepath{clip}%
\pgfsetbuttcap%
\pgfsetroundjoin%
\definecolor{currentfill}{rgb}{0.121569,0.466667,0.705882}%
\pgfsetfillcolor{currentfill}%
\pgfsetfillopacity{0.627635}%
\pgfsetlinewidth{1.003750pt}%
\definecolor{currentstroke}{rgb}{0.121569,0.466667,0.705882}%
\pgfsetstrokecolor{currentstroke}%
\pgfsetstrokeopacity{0.627635}%
\pgfsetdash{}{0pt}%
\pgfpathmoveto{\pgfqpoint{1.879766in}{3.070708in}}%
\pgfpathcurveto{\pgfqpoint{1.888002in}{3.070708in}}{\pgfqpoint{1.895902in}{3.073980in}}{\pgfqpoint{1.901726in}{3.079804in}}%
\pgfpathcurveto{\pgfqpoint{1.907550in}{3.085628in}}{\pgfqpoint{1.910822in}{3.093528in}}{\pgfqpoint{1.910822in}{3.101764in}}%
\pgfpathcurveto{\pgfqpoint{1.910822in}{3.110000in}}{\pgfqpoint{1.907550in}{3.117901in}}{\pgfqpoint{1.901726in}{3.123724in}}%
\pgfpathcurveto{\pgfqpoint{1.895902in}{3.129548in}}{\pgfqpoint{1.888002in}{3.132821in}}{\pgfqpoint{1.879766in}{3.132821in}}%
\pgfpathcurveto{\pgfqpoint{1.871530in}{3.132821in}}{\pgfqpoint{1.863630in}{3.129548in}}{\pgfqpoint{1.857806in}{3.123724in}}%
\pgfpathcurveto{\pgfqpoint{1.851982in}{3.117901in}}{\pgfqpoint{1.848709in}{3.110000in}}{\pgfqpoint{1.848709in}{3.101764in}}%
\pgfpathcurveto{\pgfqpoint{1.848709in}{3.093528in}}{\pgfqpoint{1.851982in}{3.085628in}}{\pgfqpoint{1.857806in}{3.079804in}}%
\pgfpathcurveto{\pgfqpoint{1.863630in}{3.073980in}}{\pgfqpoint{1.871530in}{3.070708in}}{\pgfqpoint{1.879766in}{3.070708in}}%
\pgfpathclose%
\pgfusepath{stroke,fill}%
\end{pgfscope}%
\begin{pgfscope}%
\pgfpathrectangle{\pgfqpoint{0.100000in}{0.212622in}}{\pgfqpoint{3.696000in}{3.696000in}}%
\pgfusepath{clip}%
\pgfsetbuttcap%
\pgfsetroundjoin%
\definecolor{currentfill}{rgb}{0.121569,0.466667,0.705882}%
\pgfsetfillcolor{currentfill}%
\pgfsetfillopacity{0.628294}%
\pgfsetlinewidth{1.003750pt}%
\definecolor{currentstroke}{rgb}{0.121569,0.466667,0.705882}%
\pgfsetstrokecolor{currentstroke}%
\pgfsetstrokeopacity{0.628294}%
\pgfsetdash{}{0pt}%
\pgfpathmoveto{\pgfqpoint{3.211448in}{1.946080in}}%
\pgfpathcurveto{\pgfqpoint{3.219684in}{1.946080in}}{\pgfqpoint{3.227584in}{1.949352in}}{\pgfqpoint{3.233408in}{1.955176in}}%
\pgfpathcurveto{\pgfqpoint{3.239232in}{1.961000in}}{\pgfqpoint{3.242504in}{1.968900in}}{\pgfqpoint{3.242504in}{1.977136in}}%
\pgfpathcurveto{\pgfqpoint{3.242504in}{1.985372in}}{\pgfqpoint{3.239232in}{1.993272in}}{\pgfqpoint{3.233408in}{1.999096in}}%
\pgfpathcurveto{\pgfqpoint{3.227584in}{2.004920in}}{\pgfqpoint{3.219684in}{2.008193in}}{\pgfqpoint{3.211448in}{2.008193in}}%
\pgfpathcurveto{\pgfqpoint{3.203212in}{2.008193in}}{\pgfqpoint{3.195312in}{2.004920in}}{\pgfqpoint{3.189488in}{1.999096in}}%
\pgfpathcurveto{\pgfqpoint{3.183664in}{1.993272in}}{\pgfqpoint{3.180391in}{1.985372in}}{\pgfqpoint{3.180391in}{1.977136in}}%
\pgfpathcurveto{\pgfqpoint{3.180391in}{1.968900in}}{\pgfqpoint{3.183664in}{1.961000in}}{\pgfqpoint{3.189488in}{1.955176in}}%
\pgfpathcurveto{\pgfqpoint{3.195312in}{1.949352in}}{\pgfqpoint{3.203212in}{1.946080in}}{\pgfqpoint{3.211448in}{1.946080in}}%
\pgfpathclose%
\pgfusepath{stroke,fill}%
\end{pgfscope}%
\begin{pgfscope}%
\pgfpathrectangle{\pgfqpoint{0.100000in}{0.212622in}}{\pgfqpoint{3.696000in}{3.696000in}}%
\pgfusepath{clip}%
\pgfsetbuttcap%
\pgfsetroundjoin%
\definecolor{currentfill}{rgb}{0.121569,0.466667,0.705882}%
\pgfsetfillcolor{currentfill}%
\pgfsetfillopacity{0.629109}%
\pgfsetlinewidth{1.003750pt}%
\definecolor{currentstroke}{rgb}{0.121569,0.466667,0.705882}%
\pgfsetstrokecolor{currentstroke}%
\pgfsetstrokeopacity{0.629109}%
\pgfsetdash{}{0pt}%
\pgfpathmoveto{\pgfqpoint{1.891974in}{3.069780in}}%
\pgfpathcurveto{\pgfqpoint{1.900210in}{3.069780in}}{\pgfqpoint{1.908110in}{3.073052in}}{\pgfqpoint{1.913934in}{3.078876in}}%
\pgfpathcurveto{\pgfqpoint{1.919758in}{3.084700in}}{\pgfqpoint{1.923030in}{3.092600in}}{\pgfqpoint{1.923030in}{3.100836in}}%
\pgfpathcurveto{\pgfqpoint{1.923030in}{3.109072in}}{\pgfqpoint{1.919758in}{3.116972in}}{\pgfqpoint{1.913934in}{3.122796in}}%
\pgfpathcurveto{\pgfqpoint{1.908110in}{3.128620in}}{\pgfqpoint{1.900210in}{3.131893in}}{\pgfqpoint{1.891974in}{3.131893in}}%
\pgfpathcurveto{\pgfqpoint{1.883738in}{3.131893in}}{\pgfqpoint{1.875838in}{3.128620in}}{\pgfqpoint{1.870014in}{3.122796in}}%
\pgfpathcurveto{\pgfqpoint{1.864190in}{3.116972in}}{\pgfqpoint{1.860917in}{3.109072in}}{\pgfqpoint{1.860917in}{3.100836in}}%
\pgfpathcurveto{\pgfqpoint{1.860917in}{3.092600in}}{\pgfqpoint{1.864190in}{3.084700in}}{\pgfqpoint{1.870014in}{3.078876in}}%
\pgfpathcurveto{\pgfqpoint{1.875838in}{3.073052in}}{\pgfqpoint{1.883738in}{3.069780in}}{\pgfqpoint{1.891974in}{3.069780in}}%
\pgfpathclose%
\pgfusepath{stroke,fill}%
\end{pgfscope}%
\begin{pgfscope}%
\pgfpathrectangle{\pgfqpoint{0.100000in}{0.212622in}}{\pgfqpoint{3.696000in}{3.696000in}}%
\pgfusepath{clip}%
\pgfsetbuttcap%
\pgfsetroundjoin%
\definecolor{currentfill}{rgb}{0.121569,0.466667,0.705882}%
\pgfsetfillcolor{currentfill}%
\pgfsetfillopacity{0.629276}%
\pgfsetlinewidth{1.003750pt}%
\definecolor{currentstroke}{rgb}{0.121569,0.466667,0.705882}%
\pgfsetstrokecolor{currentstroke}%
\pgfsetstrokeopacity{0.629276}%
\pgfsetdash{}{0pt}%
\pgfpathmoveto{\pgfqpoint{1.680683in}{3.032297in}}%
\pgfpathcurveto{\pgfqpoint{1.688919in}{3.032297in}}{\pgfqpoint{1.696819in}{3.035570in}}{\pgfqpoint{1.702643in}{3.041394in}}%
\pgfpathcurveto{\pgfqpoint{1.708467in}{3.047218in}}{\pgfqpoint{1.711739in}{3.055118in}}{\pgfqpoint{1.711739in}{3.063354in}}%
\pgfpathcurveto{\pgfqpoint{1.711739in}{3.071590in}}{\pgfqpoint{1.708467in}{3.079490in}}{\pgfqpoint{1.702643in}{3.085314in}}%
\pgfpathcurveto{\pgfqpoint{1.696819in}{3.091138in}}{\pgfqpoint{1.688919in}{3.094410in}}{\pgfqpoint{1.680683in}{3.094410in}}%
\pgfpathcurveto{\pgfqpoint{1.672446in}{3.094410in}}{\pgfqpoint{1.664546in}{3.091138in}}{\pgfqpoint{1.658722in}{3.085314in}}%
\pgfpathcurveto{\pgfqpoint{1.652899in}{3.079490in}}{\pgfqpoint{1.649626in}{3.071590in}}{\pgfqpoint{1.649626in}{3.063354in}}%
\pgfpathcurveto{\pgfqpoint{1.649626in}{3.055118in}}{\pgfqpoint{1.652899in}{3.047218in}}{\pgfqpoint{1.658722in}{3.041394in}}%
\pgfpathcurveto{\pgfqpoint{1.664546in}{3.035570in}}{\pgfqpoint{1.672446in}{3.032297in}}{\pgfqpoint{1.680683in}{3.032297in}}%
\pgfpathclose%
\pgfusepath{stroke,fill}%
\end{pgfscope}%
\begin{pgfscope}%
\pgfpathrectangle{\pgfqpoint{0.100000in}{0.212622in}}{\pgfqpoint{3.696000in}{3.696000in}}%
\pgfusepath{clip}%
\pgfsetbuttcap%
\pgfsetroundjoin%
\definecolor{currentfill}{rgb}{0.121569,0.466667,0.705882}%
\pgfsetfillcolor{currentfill}%
\pgfsetfillopacity{0.629677}%
\pgfsetlinewidth{1.003750pt}%
\definecolor{currentstroke}{rgb}{0.121569,0.466667,0.705882}%
\pgfsetstrokecolor{currentstroke}%
\pgfsetstrokeopacity{0.629677}%
\pgfsetdash{}{0pt}%
\pgfpathmoveto{\pgfqpoint{3.208438in}{1.941141in}}%
\pgfpathcurveto{\pgfqpoint{3.216674in}{1.941141in}}{\pgfqpoint{3.224574in}{1.944413in}}{\pgfqpoint{3.230398in}{1.950237in}}%
\pgfpathcurveto{\pgfqpoint{3.236222in}{1.956061in}}{\pgfqpoint{3.239494in}{1.963961in}}{\pgfqpoint{3.239494in}{1.972197in}}%
\pgfpathcurveto{\pgfqpoint{3.239494in}{1.980434in}}{\pgfqpoint{3.236222in}{1.988334in}}{\pgfqpoint{3.230398in}{1.994158in}}%
\pgfpathcurveto{\pgfqpoint{3.224574in}{1.999982in}}{\pgfqpoint{3.216674in}{2.003254in}}{\pgfqpoint{3.208438in}{2.003254in}}%
\pgfpathcurveto{\pgfqpoint{3.200202in}{2.003254in}}{\pgfqpoint{3.192301in}{1.999982in}}{\pgfqpoint{3.186478in}{1.994158in}}%
\pgfpathcurveto{\pgfqpoint{3.180654in}{1.988334in}}{\pgfqpoint{3.177381in}{1.980434in}}{\pgfqpoint{3.177381in}{1.972197in}}%
\pgfpathcurveto{\pgfqpoint{3.177381in}{1.963961in}}{\pgfqpoint{3.180654in}{1.956061in}}{\pgfqpoint{3.186478in}{1.950237in}}%
\pgfpathcurveto{\pgfqpoint{3.192301in}{1.944413in}}{\pgfqpoint{3.200202in}{1.941141in}}{\pgfqpoint{3.208438in}{1.941141in}}%
\pgfpathclose%
\pgfusepath{stroke,fill}%
\end{pgfscope}%
\begin{pgfscope}%
\pgfpathrectangle{\pgfqpoint{0.100000in}{0.212622in}}{\pgfqpoint{3.696000in}{3.696000in}}%
\pgfusepath{clip}%
\pgfsetbuttcap%
\pgfsetroundjoin%
\definecolor{currentfill}{rgb}{0.121569,0.466667,0.705882}%
\pgfsetfillcolor{currentfill}%
\pgfsetfillopacity{0.630172}%
\pgfsetlinewidth{1.003750pt}%
\definecolor{currentstroke}{rgb}{0.121569,0.466667,0.705882}%
\pgfsetstrokecolor{currentstroke}%
\pgfsetstrokeopacity{0.630172}%
\pgfsetdash{}{0pt}%
\pgfpathmoveto{\pgfqpoint{1.678981in}{3.029790in}}%
\pgfpathcurveto{\pgfqpoint{1.687217in}{3.029790in}}{\pgfqpoint{1.695118in}{3.033062in}}{\pgfqpoint{1.700941in}{3.038886in}}%
\pgfpathcurveto{\pgfqpoint{1.706765in}{3.044710in}}{\pgfqpoint{1.710038in}{3.052610in}}{\pgfqpoint{1.710038in}{3.060846in}}%
\pgfpathcurveto{\pgfqpoint{1.710038in}{3.069083in}}{\pgfqpoint{1.706765in}{3.076983in}}{\pgfqpoint{1.700941in}{3.082807in}}%
\pgfpathcurveto{\pgfqpoint{1.695118in}{3.088631in}}{\pgfqpoint{1.687217in}{3.091903in}}{\pgfqpoint{1.678981in}{3.091903in}}%
\pgfpathcurveto{\pgfqpoint{1.670745in}{3.091903in}}{\pgfqpoint{1.662845in}{3.088631in}}{\pgfqpoint{1.657021in}{3.082807in}}%
\pgfpathcurveto{\pgfqpoint{1.651197in}{3.076983in}}{\pgfqpoint{1.647925in}{3.069083in}}{\pgfqpoint{1.647925in}{3.060846in}}%
\pgfpathcurveto{\pgfqpoint{1.647925in}{3.052610in}}{\pgfqpoint{1.651197in}{3.044710in}}{\pgfqpoint{1.657021in}{3.038886in}}%
\pgfpathcurveto{\pgfqpoint{1.662845in}{3.033062in}}{\pgfqpoint{1.670745in}{3.029790in}}{\pgfqpoint{1.678981in}{3.029790in}}%
\pgfpathclose%
\pgfusepath{stroke,fill}%
\end{pgfscope}%
\begin{pgfscope}%
\pgfpathrectangle{\pgfqpoint{0.100000in}{0.212622in}}{\pgfqpoint{3.696000in}{3.696000in}}%
\pgfusepath{clip}%
\pgfsetbuttcap%
\pgfsetroundjoin%
\definecolor{currentfill}{rgb}{0.121569,0.466667,0.705882}%
\pgfsetfillcolor{currentfill}%
\pgfsetfillopacity{0.630315}%
\pgfsetlinewidth{1.003750pt}%
\definecolor{currentstroke}{rgb}{0.121569,0.466667,0.705882}%
\pgfsetstrokecolor{currentstroke}%
\pgfsetstrokeopacity{0.630315}%
\pgfsetdash{}{0pt}%
\pgfpathmoveto{\pgfqpoint{1.903143in}{3.068139in}}%
\pgfpathcurveto{\pgfqpoint{1.911380in}{3.068139in}}{\pgfqpoint{1.919280in}{3.071411in}}{\pgfqpoint{1.925104in}{3.077235in}}%
\pgfpathcurveto{\pgfqpoint{1.930928in}{3.083059in}}{\pgfqpoint{1.934200in}{3.090959in}}{\pgfqpoint{1.934200in}{3.099196in}}%
\pgfpathcurveto{\pgfqpoint{1.934200in}{3.107432in}}{\pgfqpoint{1.930928in}{3.115332in}}{\pgfqpoint{1.925104in}{3.121156in}}%
\pgfpathcurveto{\pgfqpoint{1.919280in}{3.126980in}}{\pgfqpoint{1.911380in}{3.130252in}}{\pgfqpoint{1.903143in}{3.130252in}}%
\pgfpathcurveto{\pgfqpoint{1.894907in}{3.130252in}}{\pgfqpoint{1.887007in}{3.126980in}}{\pgfqpoint{1.881183in}{3.121156in}}%
\pgfpathcurveto{\pgfqpoint{1.875359in}{3.115332in}}{\pgfqpoint{1.872087in}{3.107432in}}{\pgfqpoint{1.872087in}{3.099196in}}%
\pgfpathcurveto{\pgfqpoint{1.872087in}{3.090959in}}{\pgfqpoint{1.875359in}{3.083059in}}{\pgfqpoint{1.881183in}{3.077235in}}%
\pgfpathcurveto{\pgfqpoint{1.887007in}{3.071411in}}{\pgfqpoint{1.894907in}{3.068139in}}{\pgfqpoint{1.903143in}{3.068139in}}%
\pgfpathclose%
\pgfusepath{stroke,fill}%
\end{pgfscope}%
\begin{pgfscope}%
\pgfpathrectangle{\pgfqpoint{0.100000in}{0.212622in}}{\pgfqpoint{3.696000in}{3.696000in}}%
\pgfusepath{clip}%
\pgfsetbuttcap%
\pgfsetroundjoin%
\definecolor{currentfill}{rgb}{0.121569,0.466667,0.705882}%
\pgfsetfillcolor{currentfill}%
\pgfsetfillopacity{0.630658}%
\pgfsetlinewidth{1.003750pt}%
\definecolor{currentstroke}{rgb}{0.121569,0.466667,0.705882}%
\pgfsetstrokecolor{currentstroke}%
\pgfsetstrokeopacity{0.630658}%
\pgfsetdash{}{0pt}%
\pgfpathmoveto{\pgfqpoint{1.678034in}{3.028380in}}%
\pgfpathcurveto{\pgfqpoint{1.686270in}{3.028380in}}{\pgfqpoint{1.694170in}{3.031652in}}{\pgfqpoint{1.699994in}{3.037476in}}%
\pgfpathcurveto{\pgfqpoint{1.705818in}{3.043300in}}{\pgfqpoint{1.709090in}{3.051200in}}{\pgfqpoint{1.709090in}{3.059436in}}%
\pgfpathcurveto{\pgfqpoint{1.709090in}{3.067672in}}{\pgfqpoint{1.705818in}{3.075572in}}{\pgfqpoint{1.699994in}{3.081396in}}%
\pgfpathcurveto{\pgfqpoint{1.694170in}{3.087220in}}{\pgfqpoint{1.686270in}{3.090493in}}{\pgfqpoint{1.678034in}{3.090493in}}%
\pgfpathcurveto{\pgfqpoint{1.669798in}{3.090493in}}{\pgfqpoint{1.661898in}{3.087220in}}{\pgfqpoint{1.656074in}{3.081396in}}%
\pgfpathcurveto{\pgfqpoint{1.650250in}{3.075572in}}{\pgfqpoint{1.646977in}{3.067672in}}{\pgfqpoint{1.646977in}{3.059436in}}%
\pgfpathcurveto{\pgfqpoint{1.646977in}{3.051200in}}{\pgfqpoint{1.650250in}{3.043300in}}{\pgfqpoint{1.656074in}{3.037476in}}%
\pgfpathcurveto{\pgfqpoint{1.661898in}{3.031652in}}{\pgfqpoint{1.669798in}{3.028380in}}{\pgfqpoint{1.678034in}{3.028380in}}%
\pgfpathclose%
\pgfusepath{stroke,fill}%
\end{pgfscope}%
\begin{pgfscope}%
\pgfpathrectangle{\pgfqpoint{0.100000in}{0.212622in}}{\pgfqpoint{3.696000in}{3.696000in}}%
\pgfusepath{clip}%
\pgfsetbuttcap%
\pgfsetroundjoin%
\definecolor{currentfill}{rgb}{0.121569,0.466667,0.705882}%
\pgfsetfillcolor{currentfill}%
\pgfsetfillopacity{0.630781}%
\pgfsetlinewidth{1.003750pt}%
\definecolor{currentstroke}{rgb}{0.121569,0.466667,0.705882}%
\pgfsetstrokecolor{currentstroke}%
\pgfsetstrokeopacity{0.630781}%
\pgfsetdash{}{0pt}%
\pgfpathmoveto{\pgfqpoint{3.205968in}{1.937045in}}%
\pgfpathcurveto{\pgfqpoint{3.214205in}{1.937045in}}{\pgfqpoint{3.222105in}{1.940317in}}{\pgfqpoint{3.227929in}{1.946141in}}%
\pgfpathcurveto{\pgfqpoint{3.233753in}{1.951965in}}{\pgfqpoint{3.237025in}{1.959865in}}{\pgfqpoint{3.237025in}{1.968101in}}%
\pgfpathcurveto{\pgfqpoint{3.237025in}{1.976337in}}{\pgfqpoint{3.233753in}{1.984237in}}{\pgfqpoint{3.227929in}{1.990061in}}%
\pgfpathcurveto{\pgfqpoint{3.222105in}{1.995885in}}{\pgfqpoint{3.214205in}{1.999158in}}{\pgfqpoint{3.205968in}{1.999158in}}%
\pgfpathcurveto{\pgfqpoint{3.197732in}{1.999158in}}{\pgfqpoint{3.189832in}{1.995885in}}{\pgfqpoint{3.184008in}{1.990061in}}%
\pgfpathcurveto{\pgfqpoint{3.178184in}{1.984237in}}{\pgfqpoint{3.174912in}{1.976337in}}{\pgfqpoint{3.174912in}{1.968101in}}%
\pgfpathcurveto{\pgfqpoint{3.174912in}{1.959865in}}{\pgfqpoint{3.178184in}{1.951965in}}{\pgfqpoint{3.184008in}{1.946141in}}%
\pgfpathcurveto{\pgfqpoint{3.189832in}{1.940317in}}{\pgfqpoint{3.197732in}{1.937045in}}{\pgfqpoint{3.205968in}{1.937045in}}%
\pgfpathclose%
\pgfusepath{stroke,fill}%
\end{pgfscope}%
\begin{pgfscope}%
\pgfpathrectangle{\pgfqpoint{0.100000in}{0.212622in}}{\pgfqpoint{3.696000in}{3.696000in}}%
\pgfusepath{clip}%
\pgfsetbuttcap%
\pgfsetroundjoin%
\definecolor{currentfill}{rgb}{0.121569,0.466667,0.705882}%
\pgfsetfillcolor{currentfill}%
\pgfsetfillopacity{0.631242}%
\pgfsetlinewidth{1.003750pt}%
\definecolor{currentstroke}{rgb}{0.121569,0.466667,0.705882}%
\pgfsetstrokecolor{currentstroke}%
\pgfsetstrokeopacity{0.631242}%
\pgfsetdash{}{0pt}%
\pgfpathmoveto{\pgfqpoint{1.676842in}{3.026876in}}%
\pgfpathcurveto{\pgfqpoint{1.685079in}{3.026876in}}{\pgfqpoint{1.692979in}{3.030148in}}{\pgfqpoint{1.698803in}{3.035972in}}%
\pgfpathcurveto{\pgfqpoint{1.704627in}{3.041796in}}{\pgfqpoint{1.707899in}{3.049696in}}{\pgfqpoint{1.707899in}{3.057932in}}%
\pgfpathcurveto{\pgfqpoint{1.707899in}{3.066168in}}{\pgfqpoint{1.704627in}{3.074069in}}{\pgfqpoint{1.698803in}{3.079892in}}%
\pgfpathcurveto{\pgfqpoint{1.692979in}{3.085716in}}{\pgfqpoint{1.685079in}{3.088989in}}{\pgfqpoint{1.676842in}{3.088989in}}%
\pgfpathcurveto{\pgfqpoint{1.668606in}{3.088989in}}{\pgfqpoint{1.660706in}{3.085716in}}{\pgfqpoint{1.654882in}{3.079892in}}%
\pgfpathcurveto{\pgfqpoint{1.649058in}{3.074069in}}{\pgfqpoint{1.645786in}{3.066168in}}{\pgfqpoint{1.645786in}{3.057932in}}%
\pgfpathcurveto{\pgfqpoint{1.645786in}{3.049696in}}{\pgfqpoint{1.649058in}{3.041796in}}{\pgfqpoint{1.654882in}{3.035972in}}%
\pgfpathcurveto{\pgfqpoint{1.660706in}{3.030148in}}{\pgfqpoint{1.668606in}{3.026876in}}{\pgfqpoint{1.676842in}{3.026876in}}%
\pgfpathclose%
\pgfusepath{stroke,fill}%
\end{pgfscope}%
\begin{pgfscope}%
\pgfpathrectangle{\pgfqpoint{0.100000in}{0.212622in}}{\pgfqpoint{3.696000in}{3.696000in}}%
\pgfusepath{clip}%
\pgfsetbuttcap%
\pgfsetroundjoin%
\definecolor{currentfill}{rgb}{0.121569,0.466667,0.705882}%
\pgfsetfillcolor{currentfill}%
\pgfsetfillopacity{0.631377}%
\pgfsetlinewidth{1.003750pt}%
\definecolor{currentstroke}{rgb}{0.121569,0.466667,0.705882}%
\pgfsetstrokecolor{currentstroke}%
\pgfsetstrokeopacity{0.631377}%
\pgfsetdash{}{0pt}%
\pgfpathmoveto{\pgfqpoint{1.913090in}{3.066188in}}%
\pgfpathcurveto{\pgfqpoint{1.921326in}{3.066188in}}{\pgfqpoint{1.929226in}{3.069461in}}{\pgfqpoint{1.935050in}{3.075284in}}%
\pgfpathcurveto{\pgfqpoint{1.940874in}{3.081108in}}{\pgfqpoint{1.944147in}{3.089008in}}{\pgfqpoint{1.944147in}{3.097245in}}%
\pgfpathcurveto{\pgfqpoint{1.944147in}{3.105481in}}{\pgfqpoint{1.940874in}{3.113381in}}{\pgfqpoint{1.935050in}{3.119205in}}%
\pgfpathcurveto{\pgfqpoint{1.929226in}{3.125029in}}{\pgfqpoint{1.921326in}{3.128301in}}{\pgfqpoint{1.913090in}{3.128301in}}%
\pgfpathcurveto{\pgfqpoint{1.904854in}{3.128301in}}{\pgfqpoint{1.896954in}{3.125029in}}{\pgfqpoint{1.891130in}{3.119205in}}%
\pgfpathcurveto{\pgfqpoint{1.885306in}{3.113381in}}{\pgfqpoint{1.882034in}{3.105481in}}{\pgfqpoint{1.882034in}{3.097245in}}%
\pgfpathcurveto{\pgfqpoint{1.882034in}{3.089008in}}{\pgfqpoint{1.885306in}{3.081108in}}{\pgfqpoint{1.891130in}{3.075284in}}%
\pgfpathcurveto{\pgfqpoint{1.896954in}{3.069461in}}{\pgfqpoint{1.904854in}{3.066188in}}{\pgfqpoint{1.913090in}{3.066188in}}%
\pgfpathclose%
\pgfusepath{stroke,fill}%
\end{pgfscope}%
\begin{pgfscope}%
\pgfpathrectangle{\pgfqpoint{0.100000in}{0.212622in}}{\pgfqpoint{3.696000in}{3.696000in}}%
\pgfusepath{clip}%
\pgfsetbuttcap%
\pgfsetroundjoin%
\definecolor{currentfill}{rgb}{0.121569,0.466667,0.705882}%
\pgfsetfillcolor{currentfill}%
\pgfsetfillopacity{0.631903}%
\pgfsetlinewidth{1.003750pt}%
\definecolor{currentstroke}{rgb}{0.121569,0.466667,0.705882}%
\pgfsetstrokecolor{currentstroke}%
\pgfsetstrokeopacity{0.631903}%
\pgfsetdash{}{0pt}%
\pgfpathmoveto{\pgfqpoint{1.675416in}{3.025007in}}%
\pgfpathcurveto{\pgfqpoint{1.683653in}{3.025007in}}{\pgfqpoint{1.691553in}{3.028279in}}{\pgfqpoint{1.697377in}{3.034103in}}%
\pgfpathcurveto{\pgfqpoint{1.703200in}{3.039927in}}{\pgfqpoint{1.706473in}{3.047827in}}{\pgfqpoint{1.706473in}{3.056063in}}%
\pgfpathcurveto{\pgfqpoint{1.706473in}{3.064300in}}{\pgfqpoint{1.703200in}{3.072200in}}{\pgfqpoint{1.697377in}{3.078024in}}%
\pgfpathcurveto{\pgfqpoint{1.691553in}{3.083848in}}{\pgfqpoint{1.683653in}{3.087120in}}{\pgfqpoint{1.675416in}{3.087120in}}%
\pgfpathcurveto{\pgfqpoint{1.667180in}{3.087120in}}{\pgfqpoint{1.659280in}{3.083848in}}{\pgfqpoint{1.653456in}{3.078024in}}%
\pgfpathcurveto{\pgfqpoint{1.647632in}{3.072200in}}{\pgfqpoint{1.644360in}{3.064300in}}{\pgfqpoint{1.644360in}{3.056063in}}%
\pgfpathcurveto{\pgfqpoint{1.644360in}{3.047827in}}{\pgfqpoint{1.647632in}{3.039927in}}{\pgfqpoint{1.653456in}{3.034103in}}%
\pgfpathcurveto{\pgfqpoint{1.659280in}{3.028279in}}{\pgfqpoint{1.667180in}{3.025007in}}{\pgfqpoint{1.675416in}{3.025007in}}%
\pgfpathclose%
\pgfusepath{stroke,fill}%
\end{pgfscope}%
\begin{pgfscope}%
\pgfpathrectangle{\pgfqpoint{0.100000in}{0.212622in}}{\pgfqpoint{3.696000in}{3.696000in}}%
\pgfusepath{clip}%
\pgfsetbuttcap%
\pgfsetroundjoin%
\definecolor{currentfill}{rgb}{0.121569,0.466667,0.705882}%
\pgfsetfillcolor{currentfill}%
\pgfsetfillopacity{0.632264}%
\pgfsetlinewidth{1.003750pt}%
\definecolor{currentstroke}{rgb}{0.121569,0.466667,0.705882}%
\pgfsetstrokecolor{currentstroke}%
\pgfsetstrokeopacity{0.632264}%
\pgfsetdash{}{0pt}%
\pgfpathmoveto{\pgfqpoint{1.674629in}{3.023965in}}%
\pgfpathcurveto{\pgfqpoint{1.682865in}{3.023965in}}{\pgfqpoint{1.690765in}{3.027237in}}{\pgfqpoint{1.696589in}{3.033061in}}%
\pgfpathcurveto{\pgfqpoint{1.702413in}{3.038885in}}{\pgfqpoint{1.705685in}{3.046785in}}{\pgfqpoint{1.705685in}{3.055022in}}%
\pgfpathcurveto{\pgfqpoint{1.705685in}{3.063258in}}{\pgfqpoint{1.702413in}{3.071158in}}{\pgfqpoint{1.696589in}{3.076982in}}%
\pgfpathcurveto{\pgfqpoint{1.690765in}{3.082806in}}{\pgfqpoint{1.682865in}{3.086078in}}{\pgfqpoint{1.674629in}{3.086078in}}%
\pgfpathcurveto{\pgfqpoint{1.666392in}{3.086078in}}{\pgfqpoint{1.658492in}{3.082806in}}{\pgfqpoint{1.652668in}{3.076982in}}%
\pgfpathcurveto{\pgfqpoint{1.646844in}{3.071158in}}{\pgfqpoint{1.643572in}{3.063258in}}{\pgfqpoint{1.643572in}{3.055022in}}%
\pgfpathcurveto{\pgfqpoint{1.643572in}{3.046785in}}{\pgfqpoint{1.646844in}{3.038885in}}{\pgfqpoint{1.652668in}{3.033061in}}%
\pgfpathcurveto{\pgfqpoint{1.658492in}{3.027237in}}{\pgfqpoint{1.666392in}{3.023965in}}{\pgfqpoint{1.674629in}{3.023965in}}%
\pgfpathclose%
\pgfusepath{stroke,fill}%
\end{pgfscope}%
\begin{pgfscope}%
\pgfpathrectangle{\pgfqpoint{0.100000in}{0.212622in}}{\pgfqpoint{3.696000in}{3.696000in}}%
\pgfusepath{clip}%
\pgfsetbuttcap%
\pgfsetroundjoin%
\definecolor{currentfill}{rgb}{0.121569,0.466667,0.705882}%
\pgfsetfillcolor{currentfill}%
\pgfsetfillopacity{0.632484}%
\pgfsetlinewidth{1.003750pt}%
\definecolor{currentstroke}{rgb}{0.121569,0.466667,0.705882}%
\pgfsetstrokecolor{currentstroke}%
\pgfsetstrokeopacity{0.632484}%
\pgfsetdash{}{0pt}%
\pgfpathmoveto{\pgfqpoint{1.922743in}{3.064990in}}%
\pgfpathcurveto{\pgfqpoint{1.930979in}{3.064990in}}{\pgfqpoint{1.938879in}{3.068263in}}{\pgfqpoint{1.944703in}{3.074087in}}%
\pgfpathcurveto{\pgfqpoint{1.950527in}{3.079911in}}{\pgfqpoint{1.953799in}{3.087811in}}{\pgfqpoint{1.953799in}{3.096047in}}%
\pgfpathcurveto{\pgfqpoint{1.953799in}{3.104283in}}{\pgfqpoint{1.950527in}{3.112183in}}{\pgfqpoint{1.944703in}{3.118007in}}%
\pgfpathcurveto{\pgfqpoint{1.938879in}{3.123831in}}{\pgfqpoint{1.930979in}{3.127103in}}{\pgfqpoint{1.922743in}{3.127103in}}%
\pgfpathcurveto{\pgfqpoint{1.914507in}{3.127103in}}{\pgfqpoint{1.906606in}{3.123831in}}{\pgfqpoint{1.900783in}{3.118007in}}%
\pgfpathcurveto{\pgfqpoint{1.894959in}{3.112183in}}{\pgfqpoint{1.891686in}{3.104283in}}{\pgfqpoint{1.891686in}{3.096047in}}%
\pgfpathcurveto{\pgfqpoint{1.891686in}{3.087811in}}{\pgfqpoint{1.894959in}{3.079911in}}{\pgfqpoint{1.900783in}{3.074087in}}%
\pgfpathcurveto{\pgfqpoint{1.906606in}{3.068263in}}{\pgfqpoint{1.914507in}{3.064990in}}{\pgfqpoint{1.922743in}{3.064990in}}%
\pgfpathclose%
\pgfusepath{stroke,fill}%
\end{pgfscope}%
\begin{pgfscope}%
\pgfpathrectangle{\pgfqpoint{0.100000in}{0.212622in}}{\pgfqpoint{3.696000in}{3.696000in}}%
\pgfusepath{clip}%
\pgfsetbuttcap%
\pgfsetroundjoin%
\definecolor{currentfill}{rgb}{0.121569,0.466667,0.705882}%
\pgfsetfillcolor{currentfill}%
\pgfsetfillopacity{0.632838}%
\pgfsetlinewidth{1.003750pt}%
\definecolor{currentstroke}{rgb}{0.121569,0.466667,0.705882}%
\pgfsetstrokecolor{currentstroke}%
\pgfsetstrokeopacity{0.632838}%
\pgfsetdash{}{0pt}%
\pgfpathmoveto{\pgfqpoint{3.201293in}{1.930064in}}%
\pgfpathcurveto{\pgfqpoint{3.209529in}{1.930064in}}{\pgfqpoint{3.217429in}{1.933336in}}{\pgfqpoint{3.223253in}{1.939160in}}%
\pgfpathcurveto{\pgfqpoint{3.229077in}{1.944984in}}{\pgfqpoint{3.232350in}{1.952884in}}{\pgfqpoint{3.232350in}{1.961120in}}%
\pgfpathcurveto{\pgfqpoint{3.232350in}{1.969357in}}{\pgfqpoint{3.229077in}{1.977257in}}{\pgfqpoint{3.223253in}{1.983081in}}%
\pgfpathcurveto{\pgfqpoint{3.217429in}{1.988904in}}{\pgfqpoint{3.209529in}{1.992177in}}{\pgfqpoint{3.201293in}{1.992177in}}%
\pgfpathcurveto{\pgfqpoint{3.193057in}{1.992177in}}{\pgfqpoint{3.185157in}{1.988904in}}{\pgfqpoint{3.179333in}{1.983081in}}%
\pgfpathcurveto{\pgfqpoint{3.173509in}{1.977257in}}{\pgfqpoint{3.170237in}{1.969357in}}{\pgfqpoint{3.170237in}{1.961120in}}%
\pgfpathcurveto{\pgfqpoint{3.170237in}{1.952884in}}{\pgfqpoint{3.173509in}{1.944984in}}{\pgfqpoint{3.179333in}{1.939160in}}%
\pgfpathcurveto{\pgfqpoint{3.185157in}{1.933336in}}{\pgfqpoint{3.193057in}{1.930064in}}{\pgfqpoint{3.201293in}{1.930064in}}%
\pgfpathclose%
\pgfusepath{stroke,fill}%
\end{pgfscope}%
\begin{pgfscope}%
\pgfpathrectangle{\pgfqpoint{0.100000in}{0.212622in}}{\pgfqpoint{3.696000in}{3.696000in}}%
\pgfusepath{clip}%
\pgfsetbuttcap%
\pgfsetroundjoin%
\definecolor{currentfill}{rgb}{0.121569,0.466667,0.705882}%
\pgfsetfillcolor{currentfill}%
\pgfsetfillopacity{0.633023}%
\pgfsetlinewidth{1.003750pt}%
\definecolor{currentstroke}{rgb}{0.121569,0.466667,0.705882}%
\pgfsetstrokecolor{currentstroke}%
\pgfsetstrokeopacity{0.633023}%
\pgfsetdash{}{0pt}%
\pgfpathmoveto{\pgfqpoint{1.672988in}{3.021982in}}%
\pgfpathcurveto{\pgfqpoint{1.681225in}{3.021982in}}{\pgfqpoint{1.689125in}{3.025254in}}{\pgfqpoint{1.694949in}{3.031078in}}%
\pgfpathcurveto{\pgfqpoint{1.700773in}{3.036902in}}{\pgfqpoint{1.704045in}{3.044802in}}{\pgfqpoint{1.704045in}{3.053039in}}%
\pgfpathcurveto{\pgfqpoint{1.704045in}{3.061275in}}{\pgfqpoint{1.700773in}{3.069175in}}{\pgfqpoint{1.694949in}{3.074999in}}%
\pgfpathcurveto{\pgfqpoint{1.689125in}{3.080823in}}{\pgfqpoint{1.681225in}{3.084095in}}{\pgfqpoint{1.672988in}{3.084095in}}%
\pgfpathcurveto{\pgfqpoint{1.664752in}{3.084095in}}{\pgfqpoint{1.656852in}{3.080823in}}{\pgfqpoint{1.651028in}{3.074999in}}%
\pgfpathcurveto{\pgfqpoint{1.645204in}{3.069175in}}{\pgfqpoint{1.641932in}{3.061275in}}{\pgfqpoint{1.641932in}{3.053039in}}%
\pgfpathcurveto{\pgfqpoint{1.641932in}{3.044802in}}{\pgfqpoint{1.645204in}{3.036902in}}{\pgfqpoint{1.651028in}{3.031078in}}%
\pgfpathcurveto{\pgfqpoint{1.656852in}{3.025254in}}{\pgfqpoint{1.664752in}{3.021982in}}{\pgfqpoint{1.672988in}{3.021982in}}%
\pgfpathclose%
\pgfusepath{stroke,fill}%
\end{pgfscope}%
\begin{pgfscope}%
\pgfpathrectangle{\pgfqpoint{0.100000in}{0.212622in}}{\pgfqpoint{3.696000in}{3.696000in}}%
\pgfusepath{clip}%
\pgfsetbuttcap%
\pgfsetroundjoin%
\definecolor{currentfill}{rgb}{0.121569,0.466667,0.705882}%
\pgfsetfillcolor{currentfill}%
\pgfsetfillopacity{0.633434}%
\pgfsetlinewidth{1.003750pt}%
\definecolor{currentstroke}{rgb}{0.121569,0.466667,0.705882}%
\pgfsetstrokecolor{currentstroke}%
\pgfsetstrokeopacity{0.633434}%
\pgfsetdash{}{0pt}%
\pgfpathmoveto{\pgfqpoint{1.672048in}{3.020883in}}%
\pgfpathcurveto{\pgfqpoint{1.680285in}{3.020883in}}{\pgfqpoint{1.688185in}{3.024155in}}{\pgfqpoint{1.694009in}{3.029979in}}%
\pgfpathcurveto{\pgfqpoint{1.699832in}{3.035803in}}{\pgfqpoint{1.703105in}{3.043703in}}{\pgfqpoint{1.703105in}{3.051939in}}%
\pgfpathcurveto{\pgfqpoint{1.703105in}{3.060176in}}{\pgfqpoint{1.699832in}{3.068076in}}{\pgfqpoint{1.694009in}{3.073900in}}%
\pgfpathcurveto{\pgfqpoint{1.688185in}{3.079724in}}{\pgfqpoint{1.680285in}{3.082996in}}{\pgfqpoint{1.672048in}{3.082996in}}%
\pgfpathcurveto{\pgfqpoint{1.663812in}{3.082996in}}{\pgfqpoint{1.655912in}{3.079724in}}{\pgfqpoint{1.650088in}{3.073900in}}%
\pgfpathcurveto{\pgfqpoint{1.644264in}{3.068076in}}{\pgfqpoint{1.640992in}{3.060176in}}{\pgfqpoint{1.640992in}{3.051939in}}%
\pgfpathcurveto{\pgfqpoint{1.640992in}{3.043703in}}{\pgfqpoint{1.644264in}{3.035803in}}{\pgfqpoint{1.650088in}{3.029979in}}%
\pgfpathcurveto{\pgfqpoint{1.655912in}{3.024155in}}{\pgfqpoint{1.663812in}{3.020883in}}{\pgfqpoint{1.672048in}{3.020883in}}%
\pgfpathclose%
\pgfusepath{stroke,fill}%
\end{pgfscope}%
\begin{pgfscope}%
\pgfpathrectangle{\pgfqpoint{0.100000in}{0.212622in}}{\pgfqpoint{3.696000in}{3.696000in}}%
\pgfusepath{clip}%
\pgfsetbuttcap%
\pgfsetroundjoin%
\definecolor{currentfill}{rgb}{0.121569,0.466667,0.705882}%
\pgfsetfillcolor{currentfill}%
\pgfsetfillopacity{0.633478}%
\pgfsetlinewidth{1.003750pt}%
\definecolor{currentstroke}{rgb}{0.121569,0.466667,0.705882}%
\pgfsetstrokecolor{currentstroke}%
\pgfsetstrokeopacity{0.633478}%
\pgfsetdash{}{0pt}%
\pgfpathmoveto{\pgfqpoint{1.930630in}{3.064411in}}%
\pgfpathcurveto{\pgfqpoint{1.938867in}{3.064411in}}{\pgfqpoint{1.946767in}{3.067683in}}{\pgfqpoint{1.952591in}{3.073507in}}%
\pgfpathcurveto{\pgfqpoint{1.958414in}{3.079331in}}{\pgfqpoint{1.961687in}{3.087231in}}{\pgfqpoint{1.961687in}{3.095468in}}%
\pgfpathcurveto{\pgfqpoint{1.961687in}{3.103704in}}{\pgfqpoint{1.958414in}{3.111604in}}{\pgfqpoint{1.952591in}{3.117428in}}%
\pgfpathcurveto{\pgfqpoint{1.946767in}{3.123252in}}{\pgfqpoint{1.938867in}{3.126524in}}{\pgfqpoint{1.930630in}{3.126524in}}%
\pgfpathcurveto{\pgfqpoint{1.922394in}{3.126524in}}{\pgfqpoint{1.914494in}{3.123252in}}{\pgfqpoint{1.908670in}{3.117428in}}%
\pgfpathcurveto{\pgfqpoint{1.902846in}{3.111604in}}{\pgfqpoint{1.899574in}{3.103704in}}{\pgfqpoint{1.899574in}{3.095468in}}%
\pgfpathcurveto{\pgfqpoint{1.899574in}{3.087231in}}{\pgfqpoint{1.902846in}{3.079331in}}{\pgfqpoint{1.908670in}{3.073507in}}%
\pgfpathcurveto{\pgfqpoint{1.914494in}{3.067683in}}{\pgfqpoint{1.922394in}{3.064411in}}{\pgfqpoint{1.930630in}{3.064411in}}%
\pgfpathclose%
\pgfusepath{stroke,fill}%
\end{pgfscope}%
\begin{pgfscope}%
\pgfpathrectangle{\pgfqpoint{0.100000in}{0.212622in}}{\pgfqpoint{3.696000in}{3.696000in}}%
\pgfusepath{clip}%
\pgfsetbuttcap%
\pgfsetroundjoin%
\definecolor{currentfill}{rgb}{0.121569,0.466667,0.705882}%
\pgfsetfillcolor{currentfill}%
\pgfsetfillopacity{0.634200}%
\pgfsetlinewidth{1.003750pt}%
\definecolor{currentstroke}{rgb}{0.121569,0.466667,0.705882}%
\pgfsetstrokecolor{currentstroke}%
\pgfsetstrokeopacity{0.634200}%
\pgfsetdash{}{0pt}%
\pgfpathmoveto{\pgfqpoint{1.670218in}{3.018606in}}%
\pgfpathcurveto{\pgfqpoint{1.678455in}{3.018606in}}{\pgfqpoint{1.686355in}{3.021879in}}{\pgfqpoint{1.692179in}{3.027703in}}%
\pgfpathcurveto{\pgfqpoint{1.698003in}{3.033526in}}{\pgfqpoint{1.701275in}{3.041426in}}{\pgfqpoint{1.701275in}{3.049663in}}%
\pgfpathcurveto{\pgfqpoint{1.701275in}{3.057899in}}{\pgfqpoint{1.698003in}{3.065799in}}{\pgfqpoint{1.692179in}{3.071623in}}%
\pgfpathcurveto{\pgfqpoint{1.686355in}{3.077447in}}{\pgfqpoint{1.678455in}{3.080719in}}{\pgfqpoint{1.670218in}{3.080719in}}%
\pgfpathcurveto{\pgfqpoint{1.661982in}{3.080719in}}{\pgfqpoint{1.654082in}{3.077447in}}{\pgfqpoint{1.648258in}{3.071623in}}%
\pgfpathcurveto{\pgfqpoint{1.642434in}{3.065799in}}{\pgfqpoint{1.639162in}{3.057899in}}{\pgfqpoint{1.639162in}{3.049663in}}%
\pgfpathcurveto{\pgfqpoint{1.639162in}{3.041426in}}{\pgfqpoint{1.642434in}{3.033526in}}{\pgfqpoint{1.648258in}{3.027703in}}%
\pgfpathcurveto{\pgfqpoint{1.654082in}{3.021879in}}{\pgfqpoint{1.661982in}{3.018606in}}{\pgfqpoint{1.670218in}{3.018606in}}%
\pgfpathclose%
\pgfusepath{stroke,fill}%
\end{pgfscope}%
\begin{pgfscope}%
\pgfpathrectangle{\pgfqpoint{0.100000in}{0.212622in}}{\pgfqpoint{3.696000in}{3.696000in}}%
\pgfusepath{clip}%
\pgfsetbuttcap%
\pgfsetroundjoin%
\definecolor{currentfill}{rgb}{0.121569,0.466667,0.705882}%
\pgfsetfillcolor{currentfill}%
\pgfsetfillopacity{0.634451}%
\pgfsetlinewidth{1.003750pt}%
\definecolor{currentstroke}{rgb}{0.121569,0.466667,0.705882}%
\pgfsetstrokecolor{currentstroke}%
\pgfsetstrokeopacity{0.634451}%
\pgfsetdash{}{0pt}%
\pgfpathmoveto{\pgfqpoint{1.937916in}{3.063572in}}%
\pgfpathcurveto{\pgfqpoint{1.946152in}{3.063572in}}{\pgfqpoint{1.954052in}{3.066844in}}{\pgfqpoint{1.959876in}{3.072668in}}%
\pgfpathcurveto{\pgfqpoint{1.965700in}{3.078492in}}{\pgfqpoint{1.968972in}{3.086392in}}{\pgfqpoint{1.968972in}{3.094628in}}%
\pgfpathcurveto{\pgfqpoint{1.968972in}{3.102865in}}{\pgfqpoint{1.965700in}{3.110765in}}{\pgfqpoint{1.959876in}{3.116589in}}%
\pgfpathcurveto{\pgfqpoint{1.954052in}{3.122413in}}{\pgfqpoint{1.946152in}{3.125685in}}{\pgfqpoint{1.937916in}{3.125685in}}%
\pgfpathcurveto{\pgfqpoint{1.929679in}{3.125685in}}{\pgfqpoint{1.921779in}{3.122413in}}{\pgfqpoint{1.915955in}{3.116589in}}%
\pgfpathcurveto{\pgfqpoint{1.910131in}{3.110765in}}{\pgfqpoint{1.906859in}{3.102865in}}{\pgfqpoint{1.906859in}{3.094628in}}%
\pgfpathcurveto{\pgfqpoint{1.906859in}{3.086392in}}{\pgfqpoint{1.910131in}{3.078492in}}{\pgfqpoint{1.915955in}{3.072668in}}%
\pgfpathcurveto{\pgfqpoint{1.921779in}{3.066844in}}{\pgfqpoint{1.929679in}{3.063572in}}{\pgfqpoint{1.937916in}{3.063572in}}%
\pgfpathclose%
\pgfusepath{stroke,fill}%
\end{pgfscope}%
\begin{pgfscope}%
\pgfpathrectangle{\pgfqpoint{0.100000in}{0.212622in}}{\pgfqpoint{3.696000in}{3.696000in}}%
\pgfusepath{clip}%
\pgfsetbuttcap%
\pgfsetroundjoin%
\definecolor{currentfill}{rgb}{0.121569,0.466667,0.705882}%
\pgfsetfillcolor{currentfill}%
\pgfsetfillopacity{0.634772}%
\pgfsetlinewidth{1.003750pt}%
\definecolor{currentstroke}{rgb}{0.121569,0.466667,0.705882}%
\pgfsetstrokecolor{currentstroke}%
\pgfsetstrokeopacity{0.634772}%
\pgfsetdash{}{0pt}%
\pgfpathmoveto{\pgfqpoint{3.197053in}{1.924181in}}%
\pgfpathcurveto{\pgfqpoint{3.205289in}{1.924181in}}{\pgfqpoint{3.213189in}{1.927453in}}{\pgfqpoint{3.219013in}{1.933277in}}%
\pgfpathcurveto{\pgfqpoint{3.224837in}{1.939101in}}{\pgfqpoint{3.228109in}{1.947001in}}{\pgfqpoint{3.228109in}{1.955237in}}%
\pgfpathcurveto{\pgfqpoint{3.228109in}{1.963473in}}{\pgfqpoint{3.224837in}{1.971374in}}{\pgfqpoint{3.219013in}{1.977197in}}%
\pgfpathcurveto{\pgfqpoint{3.213189in}{1.983021in}}{\pgfqpoint{3.205289in}{1.986294in}}{\pgfqpoint{3.197053in}{1.986294in}}%
\pgfpathcurveto{\pgfqpoint{3.188816in}{1.986294in}}{\pgfqpoint{3.180916in}{1.983021in}}{\pgfqpoint{3.175092in}{1.977197in}}%
\pgfpathcurveto{\pgfqpoint{3.169268in}{1.971374in}}{\pgfqpoint{3.165996in}{1.963473in}}{\pgfqpoint{3.165996in}{1.955237in}}%
\pgfpathcurveto{\pgfqpoint{3.165996in}{1.947001in}}{\pgfqpoint{3.169268in}{1.939101in}}{\pgfqpoint{3.175092in}{1.933277in}}%
\pgfpathcurveto{\pgfqpoint{3.180916in}{1.927453in}}{\pgfqpoint{3.188816in}{1.924181in}}{\pgfqpoint{3.197053in}{1.924181in}}%
\pgfpathclose%
\pgfusepath{stroke,fill}%
\end{pgfscope}%
\begin{pgfscope}%
\pgfpathrectangle{\pgfqpoint{0.100000in}{0.212622in}}{\pgfqpoint{3.696000in}{3.696000in}}%
\pgfusepath{clip}%
\pgfsetbuttcap%
\pgfsetroundjoin%
\definecolor{currentfill}{rgb}{0.121569,0.466667,0.705882}%
\pgfsetfillcolor{currentfill}%
\pgfsetfillopacity{0.635051}%
\pgfsetlinewidth{1.003750pt}%
\definecolor{currentstroke}{rgb}{0.121569,0.466667,0.705882}%
\pgfsetstrokecolor{currentstroke}%
\pgfsetstrokeopacity{0.635051}%
\pgfsetdash{}{0pt}%
\pgfpathmoveto{\pgfqpoint{1.668228in}{3.016156in}}%
\pgfpathcurveto{\pgfqpoint{1.676464in}{3.016156in}}{\pgfqpoint{1.684364in}{3.019428in}}{\pgfqpoint{1.690188in}{3.025252in}}%
\pgfpathcurveto{\pgfqpoint{1.696012in}{3.031076in}}{\pgfqpoint{1.699284in}{3.038976in}}{\pgfqpoint{1.699284in}{3.047212in}}%
\pgfpathcurveto{\pgfqpoint{1.699284in}{3.055449in}}{\pgfqpoint{1.696012in}{3.063349in}}{\pgfqpoint{1.690188in}{3.069173in}}%
\pgfpathcurveto{\pgfqpoint{1.684364in}{3.074996in}}{\pgfqpoint{1.676464in}{3.078269in}}{\pgfqpoint{1.668228in}{3.078269in}}%
\pgfpathcurveto{\pgfqpoint{1.659991in}{3.078269in}}{\pgfqpoint{1.652091in}{3.074996in}}{\pgfqpoint{1.646267in}{3.069173in}}%
\pgfpathcurveto{\pgfqpoint{1.640443in}{3.063349in}}{\pgfqpoint{1.637171in}{3.055449in}}{\pgfqpoint{1.637171in}{3.047212in}}%
\pgfpathcurveto{\pgfqpoint{1.637171in}{3.038976in}}{\pgfqpoint{1.640443in}{3.031076in}}{\pgfqpoint{1.646267in}{3.025252in}}%
\pgfpathcurveto{\pgfqpoint{1.652091in}{3.019428in}}{\pgfqpoint{1.659991in}{3.016156in}}{\pgfqpoint{1.668228in}{3.016156in}}%
\pgfpathclose%
\pgfusepath{stroke,fill}%
\end{pgfscope}%
\begin{pgfscope}%
\pgfpathrectangle{\pgfqpoint{0.100000in}{0.212622in}}{\pgfqpoint{3.696000in}{3.696000in}}%
\pgfusepath{clip}%
\pgfsetbuttcap%
\pgfsetroundjoin%
\definecolor{currentfill}{rgb}{0.121569,0.466667,0.705882}%
\pgfsetfillcolor{currentfill}%
\pgfsetfillopacity{0.635270}%
\pgfsetlinewidth{1.003750pt}%
\definecolor{currentstroke}{rgb}{0.121569,0.466667,0.705882}%
\pgfsetstrokecolor{currentstroke}%
\pgfsetstrokeopacity{0.635270}%
\pgfsetdash{}{0pt}%
\pgfpathmoveto{\pgfqpoint{1.943979in}{3.062611in}}%
\pgfpathcurveto{\pgfqpoint{1.952216in}{3.062611in}}{\pgfqpoint{1.960116in}{3.065883in}}{\pgfqpoint{1.965940in}{3.071707in}}%
\pgfpathcurveto{\pgfqpoint{1.971763in}{3.077531in}}{\pgfqpoint{1.975036in}{3.085431in}}{\pgfqpoint{1.975036in}{3.093667in}}%
\pgfpathcurveto{\pgfqpoint{1.975036in}{3.101904in}}{\pgfqpoint{1.971763in}{3.109804in}}{\pgfqpoint{1.965940in}{3.115628in}}%
\pgfpathcurveto{\pgfqpoint{1.960116in}{3.121452in}}{\pgfqpoint{1.952216in}{3.124724in}}{\pgfqpoint{1.943979in}{3.124724in}}%
\pgfpathcurveto{\pgfqpoint{1.935743in}{3.124724in}}{\pgfqpoint{1.927843in}{3.121452in}}{\pgfqpoint{1.922019in}{3.115628in}}%
\pgfpathcurveto{\pgfqpoint{1.916195in}{3.109804in}}{\pgfqpoint{1.912923in}{3.101904in}}{\pgfqpoint{1.912923in}{3.093667in}}%
\pgfpathcurveto{\pgfqpoint{1.912923in}{3.085431in}}{\pgfqpoint{1.916195in}{3.077531in}}{\pgfqpoint{1.922019in}{3.071707in}}%
\pgfpathcurveto{\pgfqpoint{1.927843in}{3.065883in}}{\pgfqpoint{1.935743in}{3.062611in}}{\pgfqpoint{1.943979in}{3.062611in}}%
\pgfpathclose%
\pgfusepath{stroke,fill}%
\end{pgfscope}%
\begin{pgfscope}%
\pgfpathrectangle{\pgfqpoint{0.100000in}{0.212622in}}{\pgfqpoint{3.696000in}{3.696000in}}%
\pgfusepath{clip}%
\pgfsetbuttcap%
\pgfsetroundjoin%
\definecolor{currentfill}{rgb}{0.121569,0.466667,0.705882}%
\pgfsetfillcolor{currentfill}%
\pgfsetfillopacity{0.636082}%
\pgfsetlinewidth{1.003750pt}%
\definecolor{currentstroke}{rgb}{0.121569,0.466667,0.705882}%
\pgfsetstrokecolor{currentstroke}%
\pgfsetstrokeopacity{0.636082}%
\pgfsetdash{}{0pt}%
\pgfpathmoveto{\pgfqpoint{1.949759in}{3.061935in}}%
\pgfpathcurveto{\pgfqpoint{1.957995in}{3.061935in}}{\pgfqpoint{1.965895in}{3.065207in}}{\pgfqpoint{1.971719in}{3.071031in}}%
\pgfpathcurveto{\pgfqpoint{1.977543in}{3.076855in}}{\pgfqpoint{1.980815in}{3.084755in}}{\pgfqpoint{1.980815in}{3.092991in}}%
\pgfpathcurveto{\pgfqpoint{1.980815in}{3.101227in}}{\pgfqpoint{1.977543in}{3.109127in}}{\pgfqpoint{1.971719in}{3.114951in}}%
\pgfpathcurveto{\pgfqpoint{1.965895in}{3.120775in}}{\pgfqpoint{1.957995in}{3.124048in}}{\pgfqpoint{1.949759in}{3.124048in}}%
\pgfpathcurveto{\pgfqpoint{1.941522in}{3.124048in}}{\pgfqpoint{1.933622in}{3.120775in}}{\pgfqpoint{1.927798in}{3.114951in}}%
\pgfpathcurveto{\pgfqpoint{1.921974in}{3.109127in}}{\pgfqpoint{1.918702in}{3.101227in}}{\pgfqpoint{1.918702in}{3.092991in}}%
\pgfpathcurveto{\pgfqpoint{1.918702in}{3.084755in}}{\pgfqpoint{1.921974in}{3.076855in}}{\pgfqpoint{1.927798in}{3.071031in}}%
\pgfpathcurveto{\pgfqpoint{1.933622in}{3.065207in}}{\pgfqpoint{1.941522in}{3.061935in}}{\pgfqpoint{1.949759in}{3.061935in}}%
\pgfpathclose%
\pgfusepath{stroke,fill}%
\end{pgfscope}%
\begin{pgfscope}%
\pgfpathrectangle{\pgfqpoint{0.100000in}{0.212622in}}{\pgfqpoint{3.696000in}{3.696000in}}%
\pgfusepath{clip}%
\pgfsetbuttcap%
\pgfsetroundjoin%
\definecolor{currentfill}{rgb}{0.121569,0.466667,0.705882}%
\pgfsetfillcolor{currentfill}%
\pgfsetfillopacity{0.636170}%
\pgfsetlinewidth{1.003750pt}%
\definecolor{currentstroke}{rgb}{0.121569,0.466667,0.705882}%
\pgfsetstrokecolor{currentstroke}%
\pgfsetstrokeopacity{0.636170}%
\pgfsetdash{}{0pt}%
\pgfpathmoveto{\pgfqpoint{1.665663in}{3.012940in}}%
\pgfpathcurveto{\pgfqpoint{1.673899in}{3.012940in}}{\pgfqpoint{1.681799in}{3.016213in}}{\pgfqpoint{1.687623in}{3.022037in}}%
\pgfpathcurveto{\pgfqpoint{1.693447in}{3.027861in}}{\pgfqpoint{1.696719in}{3.035761in}}{\pgfqpoint{1.696719in}{3.043997in}}%
\pgfpathcurveto{\pgfqpoint{1.696719in}{3.052233in}}{\pgfqpoint{1.693447in}{3.060133in}}{\pgfqpoint{1.687623in}{3.065957in}}%
\pgfpathcurveto{\pgfqpoint{1.681799in}{3.071781in}}{\pgfqpoint{1.673899in}{3.075053in}}{\pgfqpoint{1.665663in}{3.075053in}}%
\pgfpathcurveto{\pgfqpoint{1.657427in}{3.075053in}}{\pgfqpoint{1.649527in}{3.071781in}}{\pgfqpoint{1.643703in}{3.065957in}}%
\pgfpathcurveto{\pgfqpoint{1.637879in}{3.060133in}}{\pgfqpoint{1.634606in}{3.052233in}}{\pgfqpoint{1.634606in}{3.043997in}}%
\pgfpathcurveto{\pgfqpoint{1.634606in}{3.035761in}}{\pgfqpoint{1.637879in}{3.027861in}}{\pgfqpoint{1.643703in}{3.022037in}}%
\pgfpathcurveto{\pgfqpoint{1.649527in}{3.016213in}}{\pgfqpoint{1.657427in}{3.012940in}}{\pgfqpoint{1.665663in}{3.012940in}}%
\pgfpathclose%
\pgfusepath{stroke,fill}%
\end{pgfscope}%
\begin{pgfscope}%
\pgfpathrectangle{\pgfqpoint{0.100000in}{0.212622in}}{\pgfqpoint{3.696000in}{3.696000in}}%
\pgfusepath{clip}%
\pgfsetbuttcap%
\pgfsetroundjoin%
\definecolor{currentfill}{rgb}{0.121569,0.466667,0.705882}%
\pgfsetfillcolor{currentfill}%
\pgfsetfillopacity{0.636286}%
\pgfsetlinewidth{1.003750pt}%
\definecolor{currentstroke}{rgb}{0.121569,0.466667,0.705882}%
\pgfsetstrokecolor{currentstroke}%
\pgfsetstrokeopacity{0.636286}%
\pgfsetdash{}{0pt}%
\pgfpathmoveto{\pgfqpoint{3.193856in}{1.919826in}}%
\pgfpathcurveto{\pgfqpoint{3.202093in}{1.919826in}}{\pgfqpoint{3.209993in}{1.923099in}}{\pgfqpoint{3.215817in}{1.928923in}}%
\pgfpathcurveto{\pgfqpoint{3.221641in}{1.934747in}}{\pgfqpoint{3.224913in}{1.942647in}}{\pgfqpoint{3.224913in}{1.950883in}}%
\pgfpathcurveto{\pgfqpoint{3.224913in}{1.959119in}}{\pgfqpoint{3.221641in}{1.967019in}}{\pgfqpoint{3.215817in}{1.972843in}}%
\pgfpathcurveto{\pgfqpoint{3.209993in}{1.978667in}}{\pgfqpoint{3.202093in}{1.981939in}}{\pgfqpoint{3.193856in}{1.981939in}}%
\pgfpathcurveto{\pgfqpoint{3.185620in}{1.981939in}}{\pgfqpoint{3.177720in}{1.978667in}}{\pgfqpoint{3.171896in}{1.972843in}}%
\pgfpathcurveto{\pgfqpoint{3.166072in}{1.967019in}}{\pgfqpoint{3.162800in}{1.959119in}}{\pgfqpoint{3.162800in}{1.950883in}}%
\pgfpathcurveto{\pgfqpoint{3.162800in}{1.942647in}}{\pgfqpoint{3.166072in}{1.934747in}}{\pgfqpoint{3.171896in}{1.928923in}}%
\pgfpathcurveto{\pgfqpoint{3.177720in}{1.923099in}}{\pgfqpoint{3.185620in}{1.919826in}}{\pgfqpoint{3.193856in}{1.919826in}}%
\pgfpathclose%
\pgfusepath{stroke,fill}%
\end{pgfscope}%
\begin{pgfscope}%
\pgfpathrectangle{\pgfqpoint{0.100000in}{0.212622in}}{\pgfqpoint{3.696000in}{3.696000in}}%
\pgfusepath{clip}%
\pgfsetbuttcap%
\pgfsetroundjoin%
\definecolor{currentfill}{rgb}{0.121569,0.466667,0.705882}%
\pgfsetfillcolor{currentfill}%
\pgfsetfillopacity{0.636652}%
\pgfsetlinewidth{1.003750pt}%
\definecolor{currentstroke}{rgb}{0.121569,0.466667,0.705882}%
\pgfsetstrokecolor{currentstroke}%
\pgfsetstrokeopacity{0.636652}%
\pgfsetdash{}{0pt}%
\pgfpathmoveto{\pgfqpoint{1.953773in}{3.061679in}}%
\pgfpathcurveto{\pgfqpoint{1.962009in}{3.061679in}}{\pgfqpoint{1.969909in}{3.064952in}}{\pgfqpoint{1.975733in}{3.070776in}}%
\pgfpathcurveto{\pgfqpoint{1.981557in}{3.076600in}}{\pgfqpoint{1.984829in}{3.084500in}}{\pgfqpoint{1.984829in}{3.092736in}}%
\pgfpathcurveto{\pgfqpoint{1.984829in}{3.100972in}}{\pgfqpoint{1.981557in}{3.108872in}}{\pgfqpoint{1.975733in}{3.114696in}}%
\pgfpathcurveto{\pgfqpoint{1.969909in}{3.120520in}}{\pgfqpoint{1.962009in}{3.123792in}}{\pgfqpoint{1.953773in}{3.123792in}}%
\pgfpathcurveto{\pgfqpoint{1.945536in}{3.123792in}}{\pgfqpoint{1.937636in}{3.120520in}}{\pgfqpoint{1.931812in}{3.114696in}}%
\pgfpathcurveto{\pgfqpoint{1.925988in}{3.108872in}}{\pgfqpoint{1.922716in}{3.100972in}}{\pgfqpoint{1.922716in}{3.092736in}}%
\pgfpathcurveto{\pgfqpoint{1.922716in}{3.084500in}}{\pgfqpoint{1.925988in}{3.076600in}}{\pgfqpoint{1.931812in}{3.070776in}}%
\pgfpathcurveto{\pgfqpoint{1.937636in}{3.064952in}}{\pgfqpoint{1.945536in}{3.061679in}}{\pgfqpoint{1.953773in}{3.061679in}}%
\pgfpathclose%
\pgfusepath{stroke,fill}%
\end{pgfscope}%
\begin{pgfscope}%
\pgfpathrectangle{\pgfqpoint{0.100000in}{0.212622in}}{\pgfqpoint{3.696000in}{3.696000in}}%
\pgfusepath{clip}%
\pgfsetbuttcap%
\pgfsetroundjoin%
\definecolor{currentfill}{rgb}{0.121569,0.466667,0.705882}%
\pgfsetfillcolor{currentfill}%
\pgfsetfillopacity{0.637122}%
\pgfsetlinewidth{1.003750pt}%
\definecolor{currentstroke}{rgb}{0.121569,0.466667,0.705882}%
\pgfsetstrokecolor{currentstroke}%
\pgfsetstrokeopacity{0.637122}%
\pgfsetdash{}{0pt}%
\pgfpathmoveto{\pgfqpoint{1.957300in}{3.061366in}}%
\pgfpathcurveto{\pgfqpoint{1.965536in}{3.061366in}}{\pgfqpoint{1.973436in}{3.064638in}}{\pgfqpoint{1.979260in}{3.070462in}}%
\pgfpathcurveto{\pgfqpoint{1.985084in}{3.076286in}}{\pgfqpoint{1.988356in}{3.084186in}}{\pgfqpoint{1.988356in}{3.092422in}}%
\pgfpathcurveto{\pgfqpoint{1.988356in}{3.100658in}}{\pgfqpoint{1.985084in}{3.108558in}}{\pgfqpoint{1.979260in}{3.114382in}}%
\pgfpathcurveto{\pgfqpoint{1.973436in}{3.120206in}}{\pgfqpoint{1.965536in}{3.123479in}}{\pgfqpoint{1.957300in}{3.123479in}}%
\pgfpathcurveto{\pgfqpoint{1.949064in}{3.123479in}}{\pgfqpoint{1.941164in}{3.120206in}}{\pgfqpoint{1.935340in}{3.114382in}}%
\pgfpathcurveto{\pgfqpoint{1.929516in}{3.108558in}}{\pgfqpoint{1.926243in}{3.100658in}}{\pgfqpoint{1.926243in}{3.092422in}}%
\pgfpathcurveto{\pgfqpoint{1.926243in}{3.084186in}}{\pgfqpoint{1.929516in}{3.076286in}}{\pgfqpoint{1.935340in}{3.070462in}}%
\pgfpathcurveto{\pgfqpoint{1.941164in}{3.064638in}}{\pgfqpoint{1.949064in}{3.061366in}}{\pgfqpoint{1.957300in}{3.061366in}}%
\pgfpathclose%
\pgfusepath{stroke,fill}%
\end{pgfscope}%
\begin{pgfscope}%
\pgfpathrectangle{\pgfqpoint{0.100000in}{0.212622in}}{\pgfqpoint{3.696000in}{3.696000in}}%
\pgfusepath{clip}%
\pgfsetbuttcap%
\pgfsetroundjoin%
\definecolor{currentfill}{rgb}{0.121569,0.466667,0.705882}%
\pgfsetfillcolor{currentfill}%
\pgfsetfillopacity{0.637376}%
\pgfsetlinewidth{1.003750pt}%
\definecolor{currentstroke}{rgb}{0.121569,0.466667,0.705882}%
\pgfsetstrokecolor{currentstroke}%
\pgfsetstrokeopacity{0.637376}%
\pgfsetdash{}{0pt}%
\pgfpathmoveto{\pgfqpoint{1.959237in}{3.061153in}}%
\pgfpathcurveto{\pgfqpoint{1.967473in}{3.061153in}}{\pgfqpoint{1.975373in}{3.064426in}}{\pgfqpoint{1.981197in}{3.070250in}}%
\pgfpathcurveto{\pgfqpoint{1.987021in}{3.076073in}}{\pgfqpoint{1.990293in}{3.083974in}}{\pgfqpoint{1.990293in}{3.092210in}}%
\pgfpathcurveto{\pgfqpoint{1.990293in}{3.100446in}}{\pgfqpoint{1.987021in}{3.108346in}}{\pgfqpoint{1.981197in}{3.114170in}}%
\pgfpathcurveto{\pgfqpoint{1.975373in}{3.119994in}}{\pgfqpoint{1.967473in}{3.123266in}}{\pgfqpoint{1.959237in}{3.123266in}}%
\pgfpathcurveto{\pgfqpoint{1.951001in}{3.123266in}}{\pgfqpoint{1.943101in}{3.119994in}}{\pgfqpoint{1.937277in}{3.114170in}}%
\pgfpathcurveto{\pgfqpoint{1.931453in}{3.108346in}}{\pgfqpoint{1.928180in}{3.100446in}}{\pgfqpoint{1.928180in}{3.092210in}}%
\pgfpathcurveto{\pgfqpoint{1.928180in}{3.083974in}}{\pgfqpoint{1.931453in}{3.076073in}}{\pgfqpoint{1.937277in}{3.070250in}}%
\pgfpathcurveto{\pgfqpoint{1.943101in}{3.064426in}}{\pgfqpoint{1.951001in}{3.061153in}}{\pgfqpoint{1.959237in}{3.061153in}}%
\pgfpathclose%
\pgfusepath{stroke,fill}%
\end{pgfscope}%
\begin{pgfscope}%
\pgfpathrectangle{\pgfqpoint{0.100000in}{0.212622in}}{\pgfqpoint{3.696000in}{3.696000in}}%
\pgfusepath{clip}%
\pgfsetbuttcap%
\pgfsetroundjoin%
\definecolor{currentfill}{rgb}{0.121569,0.466667,0.705882}%
\pgfsetfillcolor{currentfill}%
\pgfsetfillopacity{0.637451}%
\pgfsetlinewidth{1.003750pt}%
\definecolor{currentstroke}{rgb}{0.121569,0.466667,0.705882}%
\pgfsetstrokecolor{currentstroke}%
\pgfsetstrokeopacity{0.637451}%
\pgfsetdash{}{0pt}%
\pgfpathmoveto{\pgfqpoint{1.662758in}{3.009698in}}%
\pgfpathcurveto{\pgfqpoint{1.670994in}{3.009698in}}{\pgfqpoint{1.678894in}{3.012970in}}{\pgfqpoint{1.684718in}{3.018794in}}%
\pgfpathcurveto{\pgfqpoint{1.690542in}{3.024618in}}{\pgfqpoint{1.693815in}{3.032518in}}{\pgfqpoint{1.693815in}{3.040754in}}%
\pgfpathcurveto{\pgfqpoint{1.693815in}{3.048991in}}{\pgfqpoint{1.690542in}{3.056891in}}{\pgfqpoint{1.684718in}{3.062714in}}%
\pgfpathcurveto{\pgfqpoint{1.678894in}{3.068538in}}{\pgfqpoint{1.670994in}{3.071811in}}{\pgfqpoint{1.662758in}{3.071811in}}%
\pgfpathcurveto{\pgfqpoint{1.654522in}{3.071811in}}{\pgfqpoint{1.646622in}{3.068538in}}{\pgfqpoint{1.640798in}{3.062714in}}%
\pgfpathcurveto{\pgfqpoint{1.634974in}{3.056891in}}{\pgfqpoint{1.631702in}{3.048991in}}{\pgfqpoint{1.631702in}{3.040754in}}%
\pgfpathcurveto{\pgfqpoint{1.631702in}{3.032518in}}{\pgfqpoint{1.634974in}{3.024618in}}{\pgfqpoint{1.640798in}{3.018794in}}%
\pgfpathcurveto{\pgfqpoint{1.646622in}{3.012970in}}{\pgfqpoint{1.654522in}{3.009698in}}{\pgfqpoint{1.662758in}{3.009698in}}%
\pgfpathclose%
\pgfusepath{stroke,fill}%
\end{pgfscope}%
\begin{pgfscope}%
\pgfpathrectangle{\pgfqpoint{0.100000in}{0.212622in}}{\pgfqpoint{3.696000in}{3.696000in}}%
\pgfusepath{clip}%
\pgfsetbuttcap%
\pgfsetroundjoin%
\definecolor{currentfill}{rgb}{0.121569,0.466667,0.705882}%
\pgfsetfillcolor{currentfill}%
\pgfsetfillopacity{0.637680}%
\pgfsetlinewidth{1.003750pt}%
\definecolor{currentstroke}{rgb}{0.121569,0.466667,0.705882}%
\pgfsetstrokecolor{currentstroke}%
\pgfsetstrokeopacity{0.637680}%
\pgfsetdash{}{0pt}%
\pgfpathmoveto{\pgfqpoint{3.191043in}{1.915822in}}%
\pgfpathcurveto{\pgfqpoint{3.199279in}{1.915822in}}{\pgfqpoint{3.207179in}{1.919094in}}{\pgfqpoint{3.213003in}{1.924918in}}%
\pgfpathcurveto{\pgfqpoint{3.218827in}{1.930742in}}{\pgfqpoint{3.222099in}{1.938642in}}{\pgfqpoint{3.222099in}{1.946878in}}%
\pgfpathcurveto{\pgfqpoint{3.222099in}{1.955114in}}{\pgfqpoint{3.218827in}{1.963015in}}{\pgfqpoint{3.213003in}{1.968838in}}%
\pgfpathcurveto{\pgfqpoint{3.207179in}{1.974662in}}{\pgfqpoint{3.199279in}{1.977935in}}{\pgfqpoint{3.191043in}{1.977935in}}%
\pgfpathcurveto{\pgfqpoint{3.182807in}{1.977935in}}{\pgfqpoint{3.174907in}{1.974662in}}{\pgfqpoint{3.169083in}{1.968838in}}%
\pgfpathcurveto{\pgfqpoint{3.163259in}{1.963015in}}{\pgfqpoint{3.159986in}{1.955114in}}{\pgfqpoint{3.159986in}{1.946878in}}%
\pgfpathcurveto{\pgfqpoint{3.159986in}{1.938642in}}{\pgfqpoint{3.163259in}{1.930742in}}{\pgfqpoint{3.169083in}{1.924918in}}%
\pgfpathcurveto{\pgfqpoint{3.174907in}{1.919094in}}{\pgfqpoint{3.182807in}{1.915822in}}{\pgfqpoint{3.191043in}{1.915822in}}%
\pgfpathclose%
\pgfusepath{stroke,fill}%
\end{pgfscope}%
\begin{pgfscope}%
\pgfpathrectangle{\pgfqpoint{0.100000in}{0.212622in}}{\pgfqpoint{3.696000in}{3.696000in}}%
\pgfusepath{clip}%
\pgfsetbuttcap%
\pgfsetroundjoin%
\definecolor{currentfill}{rgb}{0.121569,0.466667,0.705882}%
\pgfsetfillcolor{currentfill}%
\pgfsetfillopacity{0.637856}%
\pgfsetlinewidth{1.003750pt}%
\definecolor{currentstroke}{rgb}{0.121569,0.466667,0.705882}%
\pgfsetstrokecolor{currentstroke}%
\pgfsetstrokeopacity{0.637856}%
\pgfsetdash{}{0pt}%
\pgfpathmoveto{\pgfqpoint{1.962724in}{3.060739in}}%
\pgfpathcurveto{\pgfqpoint{1.970961in}{3.060739in}}{\pgfqpoint{1.978861in}{3.064012in}}{\pgfqpoint{1.984685in}{3.069836in}}%
\pgfpathcurveto{\pgfqpoint{1.990509in}{3.075659in}}{\pgfqpoint{1.993781in}{3.083559in}}{\pgfqpoint{1.993781in}{3.091796in}}%
\pgfpathcurveto{\pgfqpoint{1.993781in}{3.100032in}}{\pgfqpoint{1.990509in}{3.107932in}}{\pgfqpoint{1.984685in}{3.113756in}}%
\pgfpathcurveto{\pgfqpoint{1.978861in}{3.119580in}}{\pgfqpoint{1.970961in}{3.122852in}}{\pgfqpoint{1.962724in}{3.122852in}}%
\pgfpathcurveto{\pgfqpoint{1.954488in}{3.122852in}}{\pgfqpoint{1.946588in}{3.119580in}}{\pgfqpoint{1.940764in}{3.113756in}}%
\pgfpathcurveto{\pgfqpoint{1.934940in}{3.107932in}}{\pgfqpoint{1.931668in}{3.100032in}}{\pgfqpoint{1.931668in}{3.091796in}}%
\pgfpathcurveto{\pgfqpoint{1.931668in}{3.083559in}}{\pgfqpoint{1.934940in}{3.075659in}}{\pgfqpoint{1.940764in}{3.069836in}}%
\pgfpathcurveto{\pgfqpoint{1.946588in}{3.064012in}}{\pgfqpoint{1.954488in}{3.060739in}}{\pgfqpoint{1.962724in}{3.060739in}}%
\pgfpathclose%
\pgfusepath{stroke,fill}%
\end{pgfscope}%
\begin{pgfscope}%
\pgfpathrectangle{\pgfqpoint{0.100000in}{0.212622in}}{\pgfqpoint{3.696000in}{3.696000in}}%
\pgfusepath{clip}%
\pgfsetbuttcap%
\pgfsetroundjoin%
\definecolor{currentfill}{rgb}{0.121569,0.466667,0.705882}%
\pgfsetfillcolor{currentfill}%
\pgfsetfillopacity{0.638125}%
\pgfsetlinewidth{1.003750pt}%
\definecolor{currentstroke}{rgb}{0.121569,0.466667,0.705882}%
\pgfsetstrokecolor{currentstroke}%
\pgfsetstrokeopacity{0.638125}%
\pgfsetdash{}{0pt}%
\pgfpathmoveto{\pgfqpoint{1.964532in}{3.060513in}}%
\pgfpathcurveto{\pgfqpoint{1.972768in}{3.060513in}}{\pgfqpoint{1.980668in}{3.063785in}}{\pgfqpoint{1.986492in}{3.069609in}}%
\pgfpathcurveto{\pgfqpoint{1.992316in}{3.075433in}}{\pgfqpoint{1.995588in}{3.083333in}}{\pgfqpoint{1.995588in}{3.091569in}}%
\pgfpathcurveto{\pgfqpoint{1.995588in}{3.099805in}}{\pgfqpoint{1.992316in}{3.107705in}}{\pgfqpoint{1.986492in}{3.113529in}}%
\pgfpathcurveto{\pgfqpoint{1.980668in}{3.119353in}}{\pgfqpoint{1.972768in}{3.122626in}}{\pgfqpoint{1.964532in}{3.122626in}}%
\pgfpathcurveto{\pgfqpoint{1.956295in}{3.122626in}}{\pgfqpoint{1.948395in}{3.119353in}}{\pgfqpoint{1.942571in}{3.113529in}}%
\pgfpathcurveto{\pgfqpoint{1.936748in}{3.107705in}}{\pgfqpoint{1.933475in}{3.099805in}}{\pgfqpoint{1.933475in}{3.091569in}}%
\pgfpathcurveto{\pgfqpoint{1.933475in}{3.083333in}}{\pgfqpoint{1.936748in}{3.075433in}}{\pgfqpoint{1.942571in}{3.069609in}}%
\pgfpathcurveto{\pgfqpoint{1.948395in}{3.063785in}}{\pgfqpoint{1.956295in}{3.060513in}}{\pgfqpoint{1.964532in}{3.060513in}}%
\pgfpathclose%
\pgfusepath{stroke,fill}%
\end{pgfscope}%
\begin{pgfscope}%
\pgfpathrectangle{\pgfqpoint{0.100000in}{0.212622in}}{\pgfqpoint{3.696000in}{3.696000in}}%
\pgfusepath{clip}%
\pgfsetbuttcap%
\pgfsetroundjoin%
\definecolor{currentfill}{rgb}{0.121569,0.466667,0.705882}%
\pgfsetfillcolor{currentfill}%
\pgfsetfillopacity{0.638140}%
\pgfsetlinewidth{1.003750pt}%
\definecolor{currentstroke}{rgb}{0.121569,0.466667,0.705882}%
\pgfsetstrokecolor{currentstroke}%
\pgfsetstrokeopacity{0.638140}%
\pgfsetdash{}{0pt}%
\pgfpathmoveto{\pgfqpoint{1.661116in}{3.007851in}}%
\pgfpathcurveto{\pgfqpoint{1.669352in}{3.007851in}}{\pgfqpoint{1.677252in}{3.011124in}}{\pgfqpoint{1.683076in}{3.016948in}}%
\pgfpathcurveto{\pgfqpoint{1.688900in}{3.022771in}}{\pgfqpoint{1.692173in}{3.030672in}}{\pgfqpoint{1.692173in}{3.038908in}}%
\pgfpathcurveto{\pgfqpoint{1.692173in}{3.047144in}}{\pgfqpoint{1.688900in}{3.055044in}}{\pgfqpoint{1.683076in}{3.060868in}}%
\pgfpathcurveto{\pgfqpoint{1.677252in}{3.066692in}}{\pgfqpoint{1.669352in}{3.069964in}}{\pgfqpoint{1.661116in}{3.069964in}}%
\pgfpathcurveto{\pgfqpoint{1.652880in}{3.069964in}}{\pgfqpoint{1.644980in}{3.066692in}}{\pgfqpoint{1.639156in}{3.060868in}}%
\pgfpathcurveto{\pgfqpoint{1.633332in}{3.055044in}}{\pgfqpoint{1.630060in}{3.047144in}}{\pgfqpoint{1.630060in}{3.038908in}}%
\pgfpathcurveto{\pgfqpoint{1.630060in}{3.030672in}}{\pgfqpoint{1.633332in}{3.022771in}}{\pgfqpoint{1.639156in}{3.016948in}}%
\pgfpathcurveto{\pgfqpoint{1.644980in}{3.011124in}}{\pgfqpoint{1.652880in}{3.007851in}}{\pgfqpoint{1.661116in}{3.007851in}}%
\pgfpathclose%
\pgfusepath{stroke,fill}%
\end{pgfscope}%
\begin{pgfscope}%
\pgfpathrectangle{\pgfqpoint{0.100000in}{0.212622in}}{\pgfqpoint{3.696000in}{3.696000in}}%
\pgfusepath{clip}%
\pgfsetbuttcap%
\pgfsetroundjoin%
\definecolor{currentfill}{rgb}{0.121569,0.466667,0.705882}%
\pgfsetfillcolor{currentfill}%
\pgfsetfillopacity{0.638636}%
\pgfsetlinewidth{1.003750pt}%
\definecolor{currentstroke}{rgb}{0.121569,0.466667,0.705882}%
\pgfsetstrokecolor{currentstroke}%
\pgfsetstrokeopacity{0.638636}%
\pgfsetdash{}{0pt}%
\pgfpathmoveto{\pgfqpoint{1.967803in}{3.060176in}}%
\pgfpathcurveto{\pgfqpoint{1.976039in}{3.060176in}}{\pgfqpoint{1.983939in}{3.063448in}}{\pgfqpoint{1.989763in}{3.069272in}}%
\pgfpathcurveto{\pgfqpoint{1.995587in}{3.075096in}}{\pgfqpoint{1.998859in}{3.082996in}}{\pgfqpoint{1.998859in}{3.091232in}}%
\pgfpathcurveto{\pgfqpoint{1.998859in}{3.099469in}}{\pgfqpoint{1.995587in}{3.107369in}}{\pgfqpoint{1.989763in}{3.113193in}}%
\pgfpathcurveto{\pgfqpoint{1.983939in}{3.119016in}}{\pgfqpoint{1.976039in}{3.122289in}}{\pgfqpoint{1.967803in}{3.122289in}}%
\pgfpathcurveto{\pgfqpoint{1.959566in}{3.122289in}}{\pgfqpoint{1.951666in}{3.119016in}}{\pgfqpoint{1.945842in}{3.113193in}}%
\pgfpathcurveto{\pgfqpoint{1.940019in}{3.107369in}}{\pgfqpoint{1.936746in}{3.099469in}}{\pgfqpoint{1.936746in}{3.091232in}}%
\pgfpathcurveto{\pgfqpoint{1.936746in}{3.082996in}}{\pgfqpoint{1.940019in}{3.075096in}}{\pgfqpoint{1.945842in}{3.069272in}}%
\pgfpathcurveto{\pgfqpoint{1.951666in}{3.063448in}}{\pgfqpoint{1.959566in}{3.060176in}}{\pgfqpoint{1.967803in}{3.060176in}}%
\pgfpathclose%
\pgfusepath{stroke,fill}%
\end{pgfscope}%
\begin{pgfscope}%
\pgfpathrectangle{\pgfqpoint{0.100000in}{0.212622in}}{\pgfqpoint{3.696000in}{3.696000in}}%
\pgfusepath{clip}%
\pgfsetbuttcap%
\pgfsetroundjoin%
\definecolor{currentfill}{rgb}{0.121569,0.466667,0.705882}%
\pgfsetfillcolor{currentfill}%
\pgfsetfillopacity{0.638919}%
\pgfsetlinewidth{1.003750pt}%
\definecolor{currentstroke}{rgb}{0.121569,0.466667,0.705882}%
\pgfsetstrokecolor{currentstroke}%
\pgfsetstrokeopacity{0.638919}%
\pgfsetdash{}{0pt}%
\pgfpathmoveto{\pgfqpoint{1.969686in}{3.059805in}}%
\pgfpathcurveto{\pgfqpoint{1.977922in}{3.059805in}}{\pgfqpoint{1.985822in}{3.063077in}}{\pgfqpoint{1.991646in}{3.068901in}}%
\pgfpathcurveto{\pgfqpoint{1.997470in}{3.074725in}}{\pgfqpoint{2.000742in}{3.082625in}}{\pgfqpoint{2.000742in}{3.090862in}}%
\pgfpathcurveto{\pgfqpoint{2.000742in}{3.099098in}}{\pgfqpoint{1.997470in}{3.106998in}}{\pgfqpoint{1.991646in}{3.112822in}}%
\pgfpathcurveto{\pgfqpoint{1.985822in}{3.118646in}}{\pgfqpoint{1.977922in}{3.121918in}}{\pgfqpoint{1.969686in}{3.121918in}}%
\pgfpathcurveto{\pgfqpoint{1.961449in}{3.121918in}}{\pgfqpoint{1.953549in}{3.118646in}}{\pgfqpoint{1.947726in}{3.112822in}}%
\pgfpathcurveto{\pgfqpoint{1.941902in}{3.106998in}}{\pgfqpoint{1.938629in}{3.099098in}}{\pgfqpoint{1.938629in}{3.090862in}}%
\pgfpathcurveto{\pgfqpoint{1.938629in}{3.082625in}}{\pgfqpoint{1.941902in}{3.074725in}}{\pgfqpoint{1.947726in}{3.068901in}}%
\pgfpathcurveto{\pgfqpoint{1.953549in}{3.063077in}}{\pgfqpoint{1.961449in}{3.059805in}}{\pgfqpoint{1.969686in}{3.059805in}}%
\pgfpathclose%
\pgfusepath{stroke,fill}%
\end{pgfscope}%
\begin{pgfscope}%
\pgfpathrectangle{\pgfqpoint{0.100000in}{0.212622in}}{\pgfqpoint{3.696000in}{3.696000in}}%
\pgfusepath{clip}%
\pgfsetbuttcap%
\pgfsetroundjoin%
\definecolor{currentfill}{rgb}{0.121569,0.466667,0.705882}%
\pgfsetfillcolor{currentfill}%
\pgfsetfillopacity{0.639080}%
\pgfsetlinewidth{1.003750pt}%
\definecolor{currentstroke}{rgb}{0.121569,0.466667,0.705882}%
\pgfsetstrokecolor{currentstroke}%
\pgfsetstrokeopacity{0.639080}%
\pgfsetdash{}{0pt}%
\pgfpathmoveto{\pgfqpoint{1.658689in}{3.004974in}}%
\pgfpathcurveto{\pgfqpoint{1.666925in}{3.004974in}}{\pgfqpoint{1.674825in}{3.008246in}}{\pgfqpoint{1.680649in}{3.014070in}}%
\pgfpathcurveto{\pgfqpoint{1.686473in}{3.019894in}}{\pgfqpoint{1.689746in}{3.027794in}}{\pgfqpoint{1.689746in}{3.036030in}}%
\pgfpathcurveto{\pgfqpoint{1.689746in}{3.044267in}}{\pgfqpoint{1.686473in}{3.052167in}}{\pgfqpoint{1.680649in}{3.057991in}}%
\pgfpathcurveto{\pgfqpoint{1.674825in}{3.063815in}}{\pgfqpoint{1.666925in}{3.067087in}}{\pgfqpoint{1.658689in}{3.067087in}}%
\pgfpathcurveto{\pgfqpoint{1.650453in}{3.067087in}}{\pgfqpoint{1.642553in}{3.063815in}}{\pgfqpoint{1.636729in}{3.057991in}}%
\pgfpathcurveto{\pgfqpoint{1.630905in}{3.052167in}}{\pgfqpoint{1.627633in}{3.044267in}}{\pgfqpoint{1.627633in}{3.036030in}}%
\pgfpathcurveto{\pgfqpoint{1.627633in}{3.027794in}}{\pgfqpoint{1.630905in}{3.019894in}}{\pgfqpoint{1.636729in}{3.014070in}}%
\pgfpathcurveto{\pgfqpoint{1.642553in}{3.008246in}}{\pgfqpoint{1.650453in}{3.004974in}}{\pgfqpoint{1.658689in}{3.004974in}}%
\pgfpathclose%
\pgfusepath{stroke,fill}%
\end{pgfscope}%
\begin{pgfscope}%
\pgfpathrectangle{\pgfqpoint{0.100000in}{0.212622in}}{\pgfqpoint{3.696000in}{3.696000in}}%
\pgfusepath{clip}%
\pgfsetbuttcap%
\pgfsetroundjoin%
\definecolor{currentfill}{rgb}{0.121569,0.466667,0.705882}%
\pgfsetfillcolor{currentfill}%
\pgfsetfillopacity{0.639446}%
\pgfsetlinewidth{1.003750pt}%
\definecolor{currentstroke}{rgb}{0.121569,0.466667,0.705882}%
\pgfsetstrokecolor{currentstroke}%
\pgfsetstrokeopacity{0.639446}%
\pgfsetdash{}{0pt}%
\pgfpathmoveto{\pgfqpoint{1.973084in}{3.059109in}}%
\pgfpathcurveto{\pgfqpoint{1.981320in}{3.059109in}}{\pgfqpoint{1.989220in}{3.062381in}}{\pgfqpoint{1.995044in}{3.068205in}}%
\pgfpathcurveto{\pgfqpoint{2.000868in}{3.074029in}}{\pgfqpoint{2.004140in}{3.081929in}}{\pgfqpoint{2.004140in}{3.090165in}}%
\pgfpathcurveto{\pgfqpoint{2.004140in}{3.098402in}}{\pgfqpoint{2.000868in}{3.106302in}}{\pgfqpoint{1.995044in}{3.112126in}}%
\pgfpathcurveto{\pgfqpoint{1.989220in}{3.117950in}}{\pgfqpoint{1.981320in}{3.121222in}}{\pgfqpoint{1.973084in}{3.121222in}}%
\pgfpathcurveto{\pgfqpoint{1.964847in}{3.121222in}}{\pgfqpoint{1.956947in}{3.117950in}}{\pgfqpoint{1.951123in}{3.112126in}}%
\pgfpathcurveto{\pgfqpoint{1.945300in}{3.106302in}}{\pgfqpoint{1.942027in}{3.098402in}}{\pgfqpoint{1.942027in}{3.090165in}}%
\pgfpathcurveto{\pgfqpoint{1.942027in}{3.081929in}}{\pgfqpoint{1.945300in}{3.074029in}}{\pgfqpoint{1.951123in}{3.068205in}}%
\pgfpathcurveto{\pgfqpoint{1.956947in}{3.062381in}}{\pgfqpoint{1.964847in}{3.059109in}}{\pgfqpoint{1.973084in}{3.059109in}}%
\pgfpathclose%
\pgfusepath{stroke,fill}%
\end{pgfscope}%
\begin{pgfscope}%
\pgfpathrectangle{\pgfqpoint{0.100000in}{0.212622in}}{\pgfqpoint{3.696000in}{3.696000in}}%
\pgfusepath{clip}%
\pgfsetbuttcap%
\pgfsetroundjoin%
\definecolor{currentfill}{rgb}{0.121569,0.466667,0.705882}%
\pgfsetfillcolor{currentfill}%
\pgfsetfillopacity{0.639764}%
\pgfsetlinewidth{1.003750pt}%
\definecolor{currentstroke}{rgb}{0.121569,0.466667,0.705882}%
\pgfsetstrokecolor{currentstroke}%
\pgfsetstrokeopacity{0.639764}%
\pgfsetdash{}{0pt}%
\pgfpathmoveto{\pgfqpoint{1.975189in}{3.058829in}}%
\pgfpathcurveto{\pgfqpoint{1.983425in}{3.058829in}}{\pgfqpoint{1.991325in}{3.062101in}}{\pgfqpoint{1.997149in}{3.067925in}}%
\pgfpathcurveto{\pgfqpoint{2.002973in}{3.073749in}}{\pgfqpoint{2.006246in}{3.081649in}}{\pgfqpoint{2.006246in}{3.089885in}}%
\pgfpathcurveto{\pgfqpoint{2.006246in}{3.098122in}}{\pgfqpoint{2.002973in}{3.106022in}}{\pgfqpoint{1.997149in}{3.111846in}}%
\pgfpathcurveto{\pgfqpoint{1.991325in}{3.117670in}}{\pgfqpoint{1.983425in}{3.120942in}}{\pgfqpoint{1.975189in}{3.120942in}}%
\pgfpathcurveto{\pgfqpoint{1.966953in}{3.120942in}}{\pgfqpoint{1.959053in}{3.117670in}}{\pgfqpoint{1.953229in}{3.111846in}}%
\pgfpathcurveto{\pgfqpoint{1.947405in}{3.106022in}}{\pgfqpoint{1.944133in}{3.098122in}}{\pgfqpoint{1.944133in}{3.089885in}}%
\pgfpathcurveto{\pgfqpoint{1.944133in}{3.081649in}}{\pgfqpoint{1.947405in}{3.073749in}}{\pgfqpoint{1.953229in}{3.067925in}}%
\pgfpathcurveto{\pgfqpoint{1.959053in}{3.062101in}}{\pgfqpoint{1.966953in}{3.058829in}}{\pgfqpoint{1.975189in}{3.058829in}}%
\pgfpathclose%
\pgfusepath{stroke,fill}%
\end{pgfscope}%
\begin{pgfscope}%
\pgfpathrectangle{\pgfqpoint{0.100000in}{0.212622in}}{\pgfqpoint{3.696000in}{3.696000in}}%
\pgfusepath{clip}%
\pgfsetbuttcap%
\pgfsetroundjoin%
\definecolor{currentfill}{rgb}{0.121569,0.466667,0.705882}%
\pgfsetfillcolor{currentfill}%
\pgfsetfillopacity{0.640095}%
\pgfsetlinewidth{1.003750pt}%
\definecolor{currentstroke}{rgb}{0.121569,0.466667,0.705882}%
\pgfsetstrokecolor{currentstroke}%
\pgfsetstrokeopacity{0.640095}%
\pgfsetdash{}{0pt}%
\pgfpathmoveto{\pgfqpoint{3.185961in}{1.907763in}}%
\pgfpathcurveto{\pgfqpoint{3.194197in}{1.907763in}}{\pgfqpoint{3.202097in}{1.911035in}}{\pgfqpoint{3.207921in}{1.916859in}}%
\pgfpathcurveto{\pgfqpoint{3.213745in}{1.922683in}}{\pgfqpoint{3.217018in}{1.930583in}}{\pgfqpoint{3.217018in}{1.938819in}}%
\pgfpathcurveto{\pgfqpoint{3.217018in}{1.947056in}}{\pgfqpoint{3.213745in}{1.954956in}}{\pgfqpoint{3.207921in}{1.960780in}}%
\pgfpathcurveto{\pgfqpoint{3.202097in}{1.966604in}}{\pgfqpoint{3.194197in}{1.969876in}}{\pgfqpoint{3.185961in}{1.969876in}}%
\pgfpathcurveto{\pgfqpoint{3.177725in}{1.969876in}}{\pgfqpoint{3.169825in}{1.966604in}}{\pgfqpoint{3.164001in}{1.960780in}}%
\pgfpathcurveto{\pgfqpoint{3.158177in}{1.954956in}}{\pgfqpoint{3.154905in}{1.947056in}}{\pgfqpoint{3.154905in}{1.938819in}}%
\pgfpathcurveto{\pgfqpoint{3.154905in}{1.930583in}}{\pgfqpoint{3.158177in}{1.922683in}}{\pgfqpoint{3.164001in}{1.916859in}}%
\pgfpathcurveto{\pgfqpoint{3.169825in}{1.911035in}}{\pgfqpoint{3.177725in}{1.907763in}}{\pgfqpoint{3.185961in}{1.907763in}}%
\pgfpathclose%
\pgfusepath{stroke,fill}%
\end{pgfscope}%
\begin{pgfscope}%
\pgfpathrectangle{\pgfqpoint{0.100000in}{0.212622in}}{\pgfqpoint{3.696000in}{3.696000in}}%
\pgfusepath{clip}%
\pgfsetbuttcap%
\pgfsetroundjoin%
\definecolor{currentfill}{rgb}{0.121569,0.466667,0.705882}%
\pgfsetfillcolor{currentfill}%
\pgfsetfillopacity{0.640221}%
\pgfsetlinewidth{1.003750pt}%
\definecolor{currentstroke}{rgb}{0.121569,0.466667,0.705882}%
\pgfsetstrokecolor{currentstroke}%
\pgfsetstrokeopacity{0.640221}%
\pgfsetdash{}{0pt}%
\pgfpathmoveto{\pgfqpoint{1.655851in}{3.001491in}}%
\pgfpathcurveto{\pgfqpoint{1.664088in}{3.001491in}}{\pgfqpoint{1.671988in}{3.004763in}}{\pgfqpoint{1.677812in}{3.010587in}}%
\pgfpathcurveto{\pgfqpoint{1.683636in}{3.016411in}}{\pgfqpoint{1.686908in}{3.024311in}}{\pgfqpoint{1.686908in}{3.032547in}}%
\pgfpathcurveto{\pgfqpoint{1.686908in}{3.040784in}}{\pgfqpoint{1.683636in}{3.048684in}}{\pgfqpoint{1.677812in}{3.054508in}}%
\pgfpathcurveto{\pgfqpoint{1.671988in}{3.060331in}}{\pgfqpoint{1.664088in}{3.063604in}}{\pgfqpoint{1.655851in}{3.063604in}}%
\pgfpathcurveto{\pgfqpoint{1.647615in}{3.063604in}}{\pgfqpoint{1.639715in}{3.060331in}}{\pgfqpoint{1.633891in}{3.054508in}}%
\pgfpathcurveto{\pgfqpoint{1.628067in}{3.048684in}}{\pgfqpoint{1.624795in}{3.040784in}}{\pgfqpoint{1.624795in}{3.032547in}}%
\pgfpathcurveto{\pgfqpoint{1.624795in}{3.024311in}}{\pgfqpoint{1.628067in}{3.016411in}}{\pgfqpoint{1.633891in}{3.010587in}}%
\pgfpathcurveto{\pgfqpoint{1.639715in}{3.004763in}}{\pgfqpoint{1.647615in}{3.001491in}}{\pgfqpoint{1.655851in}{3.001491in}}%
\pgfpathclose%
\pgfusepath{stroke,fill}%
\end{pgfscope}%
\begin{pgfscope}%
\pgfpathrectangle{\pgfqpoint{0.100000in}{0.212622in}}{\pgfqpoint{3.696000in}{3.696000in}}%
\pgfusepath{clip}%
\pgfsetbuttcap%
\pgfsetroundjoin%
\definecolor{currentfill}{rgb}{0.121569,0.466667,0.705882}%
\pgfsetfillcolor{currentfill}%
\pgfsetfillopacity{0.640314}%
\pgfsetlinewidth{1.003750pt}%
\definecolor{currentstroke}{rgb}{0.121569,0.466667,0.705882}%
\pgfsetstrokecolor{currentstroke}%
\pgfsetstrokeopacity{0.640314}%
\pgfsetdash{}{0pt}%
\pgfpathmoveto{\pgfqpoint{1.979046in}{3.058246in}}%
\pgfpathcurveto{\pgfqpoint{1.987282in}{3.058246in}}{\pgfqpoint{1.995182in}{3.061519in}}{\pgfqpoint{2.001006in}{3.067343in}}%
\pgfpathcurveto{\pgfqpoint{2.006830in}{3.073167in}}{\pgfqpoint{2.010103in}{3.081067in}}{\pgfqpoint{2.010103in}{3.089303in}}%
\pgfpathcurveto{\pgfqpoint{2.010103in}{3.097539in}}{\pgfqpoint{2.006830in}{3.105439in}}{\pgfqpoint{2.001006in}{3.111263in}}%
\pgfpathcurveto{\pgfqpoint{1.995182in}{3.117087in}}{\pgfqpoint{1.987282in}{3.120359in}}{\pgfqpoint{1.979046in}{3.120359in}}%
\pgfpathcurveto{\pgfqpoint{1.970810in}{3.120359in}}{\pgfqpoint{1.962910in}{3.117087in}}{\pgfqpoint{1.957086in}{3.111263in}}%
\pgfpathcurveto{\pgfqpoint{1.951262in}{3.105439in}}{\pgfqpoint{1.947990in}{3.097539in}}{\pgfqpoint{1.947990in}{3.089303in}}%
\pgfpathcurveto{\pgfqpoint{1.947990in}{3.081067in}}{\pgfqpoint{1.951262in}{3.073167in}}{\pgfqpoint{1.957086in}{3.067343in}}%
\pgfpathcurveto{\pgfqpoint{1.962910in}{3.061519in}}{\pgfqpoint{1.970810in}{3.058246in}}{\pgfqpoint{1.979046in}{3.058246in}}%
\pgfpathclose%
\pgfusepath{stroke,fill}%
\end{pgfscope}%
\begin{pgfscope}%
\pgfpathrectangle{\pgfqpoint{0.100000in}{0.212622in}}{\pgfqpoint{3.696000in}{3.696000in}}%
\pgfusepath{clip}%
\pgfsetbuttcap%
\pgfsetroundjoin%
\definecolor{currentfill}{rgb}{0.121569,0.466667,0.705882}%
\pgfsetfillcolor{currentfill}%
\pgfsetfillopacity{0.640644}%
\pgfsetlinewidth{1.003750pt}%
\definecolor{currentstroke}{rgb}{0.121569,0.466667,0.705882}%
\pgfsetstrokecolor{currentstroke}%
\pgfsetstrokeopacity{0.640644}%
\pgfsetdash{}{0pt}%
\pgfpathmoveto{\pgfqpoint{1.981470in}{3.057753in}}%
\pgfpathcurveto{\pgfqpoint{1.989706in}{3.057753in}}{\pgfqpoint{1.997606in}{3.061025in}}{\pgfqpoint{2.003430in}{3.066849in}}%
\pgfpathcurveto{\pgfqpoint{2.009254in}{3.072673in}}{\pgfqpoint{2.012526in}{3.080573in}}{\pgfqpoint{2.012526in}{3.088810in}}%
\pgfpathcurveto{\pgfqpoint{2.012526in}{3.097046in}}{\pgfqpoint{2.009254in}{3.104946in}}{\pgfqpoint{2.003430in}{3.110770in}}%
\pgfpathcurveto{\pgfqpoint{1.997606in}{3.116594in}}{\pgfqpoint{1.989706in}{3.119866in}}{\pgfqpoint{1.981470in}{3.119866in}}%
\pgfpathcurveto{\pgfqpoint{1.973233in}{3.119866in}}{\pgfqpoint{1.965333in}{3.116594in}}{\pgfqpoint{1.959509in}{3.110770in}}%
\pgfpathcurveto{\pgfqpoint{1.953686in}{3.104946in}}{\pgfqpoint{1.950413in}{3.097046in}}{\pgfqpoint{1.950413in}{3.088810in}}%
\pgfpathcurveto{\pgfqpoint{1.950413in}{3.080573in}}{\pgfqpoint{1.953686in}{3.072673in}}{\pgfqpoint{1.959509in}{3.066849in}}%
\pgfpathcurveto{\pgfqpoint{1.965333in}{3.061025in}}{\pgfqpoint{1.973233in}{3.057753in}}{\pgfqpoint{1.981470in}{3.057753in}}%
\pgfpathclose%
\pgfusepath{stroke,fill}%
\end{pgfscope}%
\begin{pgfscope}%
\pgfpathrectangle{\pgfqpoint{0.100000in}{0.212622in}}{\pgfqpoint{3.696000in}{3.696000in}}%
\pgfusepath{clip}%
\pgfsetbuttcap%
\pgfsetroundjoin%
\definecolor{currentfill}{rgb}{0.121569,0.466667,0.705882}%
\pgfsetfillcolor{currentfill}%
\pgfsetfillopacity{0.640877}%
\pgfsetlinewidth{1.003750pt}%
\definecolor{currentstroke}{rgb}{0.121569,0.466667,0.705882}%
\pgfsetstrokecolor{currentstroke}%
\pgfsetstrokeopacity{0.640877}%
\pgfsetdash{}{0pt}%
\pgfpathmoveto{\pgfqpoint{1.654284in}{2.999755in}}%
\pgfpathcurveto{\pgfqpoint{1.662521in}{2.999755in}}{\pgfqpoint{1.670421in}{3.003027in}}{\pgfqpoint{1.676245in}{3.008851in}}%
\pgfpathcurveto{\pgfqpoint{1.682069in}{3.014675in}}{\pgfqpoint{1.685341in}{3.022575in}}{\pgfqpoint{1.685341in}{3.030812in}}%
\pgfpathcurveto{\pgfqpoint{1.685341in}{3.039048in}}{\pgfqpoint{1.682069in}{3.046948in}}{\pgfqpoint{1.676245in}{3.052772in}}%
\pgfpathcurveto{\pgfqpoint{1.670421in}{3.058596in}}{\pgfqpoint{1.662521in}{3.061868in}}{\pgfqpoint{1.654284in}{3.061868in}}%
\pgfpathcurveto{\pgfqpoint{1.646048in}{3.061868in}}{\pgfqpoint{1.638148in}{3.058596in}}{\pgfqpoint{1.632324in}{3.052772in}}%
\pgfpathcurveto{\pgfqpoint{1.626500in}{3.046948in}}{\pgfqpoint{1.623228in}{3.039048in}}{\pgfqpoint{1.623228in}{3.030812in}}%
\pgfpathcurveto{\pgfqpoint{1.623228in}{3.022575in}}{\pgfqpoint{1.626500in}{3.014675in}}{\pgfqpoint{1.632324in}{3.008851in}}%
\pgfpathcurveto{\pgfqpoint{1.638148in}{3.003027in}}{\pgfqpoint{1.646048in}{2.999755in}}{\pgfqpoint{1.654284in}{2.999755in}}%
\pgfpathclose%
\pgfusepath{stroke,fill}%
\end{pgfscope}%
\begin{pgfscope}%
\pgfpathrectangle{\pgfqpoint{0.100000in}{0.212622in}}{\pgfqpoint{3.696000in}{3.696000in}}%
\pgfusepath{clip}%
\pgfsetbuttcap%
\pgfsetroundjoin%
\definecolor{currentfill}{rgb}{0.121569,0.466667,0.705882}%
\pgfsetfillcolor{currentfill}%
\pgfsetfillopacity{0.641252}%
\pgfsetlinewidth{1.003750pt}%
\definecolor{currentstroke}{rgb}{0.121569,0.466667,0.705882}%
\pgfsetstrokecolor{currentstroke}%
\pgfsetstrokeopacity{0.641252}%
\pgfsetdash{}{0pt}%
\pgfpathmoveto{\pgfqpoint{1.985876in}{3.056900in}}%
\pgfpathcurveto{\pgfqpoint{1.994112in}{3.056900in}}{\pgfqpoint{2.002012in}{3.060172in}}{\pgfqpoint{2.007836in}{3.065996in}}%
\pgfpathcurveto{\pgfqpoint{2.013660in}{3.071820in}}{\pgfqpoint{2.016932in}{3.079720in}}{\pgfqpoint{2.016932in}{3.087956in}}%
\pgfpathcurveto{\pgfqpoint{2.016932in}{3.096193in}}{\pgfqpoint{2.013660in}{3.104093in}}{\pgfqpoint{2.007836in}{3.109917in}}%
\pgfpathcurveto{\pgfqpoint{2.002012in}{3.115741in}}{\pgfqpoint{1.994112in}{3.119013in}}{\pgfqpoint{1.985876in}{3.119013in}}%
\pgfpathcurveto{\pgfqpoint{1.977639in}{3.119013in}}{\pgfqpoint{1.969739in}{3.115741in}}{\pgfqpoint{1.963915in}{3.109917in}}%
\pgfpathcurveto{\pgfqpoint{1.958091in}{3.104093in}}{\pgfqpoint{1.954819in}{3.096193in}}{\pgfqpoint{1.954819in}{3.087956in}}%
\pgfpathcurveto{\pgfqpoint{1.954819in}{3.079720in}}{\pgfqpoint{1.958091in}{3.071820in}}{\pgfqpoint{1.963915in}{3.065996in}}%
\pgfpathcurveto{\pgfqpoint{1.969739in}{3.060172in}}{\pgfqpoint{1.977639in}{3.056900in}}{\pgfqpoint{1.985876in}{3.056900in}}%
\pgfpathclose%
\pgfusepath{stroke,fill}%
\end{pgfscope}%
\begin{pgfscope}%
\pgfpathrectangle{\pgfqpoint{0.100000in}{0.212622in}}{\pgfqpoint{3.696000in}{3.696000in}}%
\pgfusepath{clip}%
\pgfsetbuttcap%
\pgfsetroundjoin%
\definecolor{currentfill}{rgb}{0.121569,0.466667,0.705882}%
\pgfsetfillcolor{currentfill}%
\pgfsetfillopacity{0.641636}%
\pgfsetlinewidth{1.003750pt}%
\definecolor{currentstroke}{rgb}{0.121569,0.466667,0.705882}%
\pgfsetstrokecolor{currentstroke}%
\pgfsetstrokeopacity{0.641636}%
\pgfsetdash{}{0pt}%
\pgfpathmoveto{\pgfqpoint{1.988699in}{3.056277in}}%
\pgfpathcurveto{\pgfqpoint{1.996936in}{3.056277in}}{\pgfqpoint{2.004836in}{3.059549in}}{\pgfqpoint{2.010660in}{3.065373in}}%
\pgfpathcurveto{\pgfqpoint{2.016483in}{3.071197in}}{\pgfqpoint{2.019756in}{3.079097in}}{\pgfqpoint{2.019756in}{3.087333in}}%
\pgfpathcurveto{\pgfqpoint{2.019756in}{3.095570in}}{\pgfqpoint{2.016483in}{3.103470in}}{\pgfqpoint{2.010660in}{3.109294in}}%
\pgfpathcurveto{\pgfqpoint{2.004836in}{3.115117in}}{\pgfqpoint{1.996936in}{3.118390in}}{\pgfqpoint{1.988699in}{3.118390in}}%
\pgfpathcurveto{\pgfqpoint{1.980463in}{3.118390in}}{\pgfqpoint{1.972563in}{3.115117in}}{\pgfqpoint{1.966739in}{3.109294in}}%
\pgfpathcurveto{\pgfqpoint{1.960915in}{3.103470in}}{\pgfqpoint{1.957643in}{3.095570in}}{\pgfqpoint{1.957643in}{3.087333in}}%
\pgfpathcurveto{\pgfqpoint{1.957643in}{3.079097in}}{\pgfqpoint{1.960915in}{3.071197in}}{\pgfqpoint{1.966739in}{3.065373in}}%
\pgfpathcurveto{\pgfqpoint{1.972563in}{3.059549in}}{\pgfqpoint{1.980463in}{3.056277in}}{\pgfqpoint{1.988699in}{3.056277in}}%
\pgfpathclose%
\pgfusepath{stroke,fill}%
\end{pgfscope}%
\begin{pgfscope}%
\pgfpathrectangle{\pgfqpoint{0.100000in}{0.212622in}}{\pgfqpoint{3.696000in}{3.696000in}}%
\pgfusepath{clip}%
\pgfsetbuttcap%
\pgfsetroundjoin%
\definecolor{currentfill}{rgb}{0.121569,0.466667,0.705882}%
\pgfsetfillcolor{currentfill}%
\pgfsetfillopacity{0.641808}%
\pgfsetlinewidth{1.003750pt}%
\definecolor{currentstroke}{rgb}{0.121569,0.466667,0.705882}%
\pgfsetstrokecolor{currentstroke}%
\pgfsetstrokeopacity{0.641808}%
\pgfsetdash{}{0pt}%
\pgfpathmoveto{\pgfqpoint{1.652001in}{2.996978in}}%
\pgfpathcurveto{\pgfqpoint{1.660237in}{2.996978in}}{\pgfqpoint{1.668137in}{3.000251in}}{\pgfqpoint{1.673961in}{3.006075in}}%
\pgfpathcurveto{\pgfqpoint{1.679785in}{3.011899in}}{\pgfqpoint{1.683057in}{3.019799in}}{\pgfqpoint{1.683057in}{3.028035in}}%
\pgfpathcurveto{\pgfqpoint{1.683057in}{3.036271in}}{\pgfqpoint{1.679785in}{3.044171in}}{\pgfqpoint{1.673961in}{3.049995in}}%
\pgfpathcurveto{\pgfqpoint{1.668137in}{3.055819in}}{\pgfqpoint{1.660237in}{3.059091in}}{\pgfqpoint{1.652001in}{3.059091in}}%
\pgfpathcurveto{\pgfqpoint{1.643764in}{3.059091in}}{\pgfqpoint{1.635864in}{3.055819in}}{\pgfqpoint{1.630040in}{3.049995in}}%
\pgfpathcurveto{\pgfqpoint{1.624216in}{3.044171in}}{\pgfqpoint{1.620944in}{3.036271in}}{\pgfqpoint{1.620944in}{3.028035in}}%
\pgfpathcurveto{\pgfqpoint{1.620944in}{3.019799in}}{\pgfqpoint{1.624216in}{3.011899in}}{\pgfqpoint{1.630040in}{3.006075in}}%
\pgfpathcurveto{\pgfqpoint{1.635864in}{3.000251in}}{\pgfqpoint{1.643764in}{2.996978in}}{\pgfqpoint{1.652001in}{2.996978in}}%
\pgfpathclose%
\pgfusepath{stroke,fill}%
\end{pgfscope}%
\begin{pgfscope}%
\pgfpathrectangle{\pgfqpoint{0.100000in}{0.212622in}}{\pgfqpoint{3.696000in}{3.696000in}}%
\pgfusepath{clip}%
\pgfsetbuttcap%
\pgfsetroundjoin%
\definecolor{currentfill}{rgb}{0.121569,0.466667,0.705882}%
\pgfsetfillcolor{currentfill}%
\pgfsetfillopacity{0.642175}%
\pgfsetlinewidth{1.003750pt}%
\definecolor{currentstroke}{rgb}{0.121569,0.466667,0.705882}%
\pgfsetstrokecolor{currentstroke}%
\pgfsetstrokeopacity{0.642175}%
\pgfsetdash{}{0pt}%
\pgfpathmoveto{\pgfqpoint{3.181437in}{1.900291in}}%
\pgfpathcurveto{\pgfqpoint{3.189673in}{1.900291in}}{\pgfqpoint{3.197573in}{1.903564in}}{\pgfqpoint{3.203397in}{1.909388in}}%
\pgfpathcurveto{\pgfqpoint{3.209221in}{1.915212in}}{\pgfqpoint{3.212493in}{1.923112in}}{\pgfqpoint{3.212493in}{1.931348in}}%
\pgfpathcurveto{\pgfqpoint{3.212493in}{1.939584in}}{\pgfqpoint{3.209221in}{1.947484in}}{\pgfqpoint{3.203397in}{1.953308in}}%
\pgfpathcurveto{\pgfqpoint{3.197573in}{1.959132in}}{\pgfqpoint{3.189673in}{1.962404in}}{\pgfqpoint{3.181437in}{1.962404in}}%
\pgfpathcurveto{\pgfqpoint{3.173200in}{1.962404in}}{\pgfqpoint{3.165300in}{1.959132in}}{\pgfqpoint{3.159476in}{1.953308in}}%
\pgfpathcurveto{\pgfqpoint{3.153653in}{1.947484in}}{\pgfqpoint{3.150380in}{1.939584in}}{\pgfqpoint{3.150380in}{1.931348in}}%
\pgfpathcurveto{\pgfqpoint{3.150380in}{1.923112in}}{\pgfqpoint{3.153653in}{1.915212in}}{\pgfqpoint{3.159476in}{1.909388in}}%
\pgfpathcurveto{\pgfqpoint{3.165300in}{1.903564in}}{\pgfqpoint{3.173200in}{1.900291in}}{\pgfqpoint{3.181437in}{1.900291in}}%
\pgfpathclose%
\pgfusepath{stroke,fill}%
\end{pgfscope}%
\begin{pgfscope}%
\pgfpathrectangle{\pgfqpoint{0.100000in}{0.212622in}}{\pgfqpoint{3.696000in}{3.696000in}}%
\pgfusepath{clip}%
\pgfsetbuttcap%
\pgfsetroundjoin%
\definecolor{currentfill}{rgb}{0.121569,0.466667,0.705882}%
\pgfsetfillcolor{currentfill}%
\pgfsetfillopacity{0.642314}%
\pgfsetlinewidth{1.003750pt}%
\definecolor{currentstroke}{rgb}{0.121569,0.466667,0.705882}%
\pgfsetstrokecolor{currentstroke}%
\pgfsetstrokeopacity{0.642314}%
\pgfsetdash{}{0pt}%
\pgfpathmoveto{\pgfqpoint{1.993899in}{3.055263in}}%
\pgfpathcurveto{\pgfqpoint{2.002135in}{3.055263in}}{\pgfqpoint{2.010035in}{3.058536in}}{\pgfqpoint{2.015859in}{3.064360in}}%
\pgfpathcurveto{\pgfqpoint{2.021683in}{3.070183in}}{\pgfqpoint{2.024955in}{3.078084in}}{\pgfqpoint{2.024955in}{3.086320in}}%
\pgfpathcurveto{\pgfqpoint{2.024955in}{3.094556in}}{\pgfqpoint{2.021683in}{3.102456in}}{\pgfqpoint{2.015859in}{3.108280in}}%
\pgfpathcurveto{\pgfqpoint{2.010035in}{3.114104in}}{\pgfqpoint{2.002135in}{3.117376in}}{\pgfqpoint{1.993899in}{3.117376in}}%
\pgfpathcurveto{\pgfqpoint{1.985662in}{3.117376in}}{\pgfqpoint{1.977762in}{3.114104in}}{\pgfqpoint{1.971938in}{3.108280in}}%
\pgfpathcurveto{\pgfqpoint{1.966114in}{3.102456in}}{\pgfqpoint{1.962842in}{3.094556in}}{\pgfqpoint{1.962842in}{3.086320in}}%
\pgfpathcurveto{\pgfqpoint{1.962842in}{3.078084in}}{\pgfqpoint{1.966114in}{3.070183in}}{\pgfqpoint{1.971938in}{3.064360in}}%
\pgfpathcurveto{\pgfqpoint{1.977762in}{3.058536in}}{\pgfqpoint{1.985662in}{3.055263in}}{\pgfqpoint{1.993899in}{3.055263in}}%
\pgfpathclose%
\pgfusepath{stroke,fill}%
\end{pgfscope}%
\begin{pgfscope}%
\pgfpathrectangle{\pgfqpoint{0.100000in}{0.212622in}}{\pgfqpoint{3.696000in}{3.696000in}}%
\pgfusepath{clip}%
\pgfsetbuttcap%
\pgfsetroundjoin%
\definecolor{currentfill}{rgb}{0.121569,0.466667,0.705882}%
\pgfsetfillcolor{currentfill}%
\pgfsetfillopacity{0.642319}%
\pgfsetlinewidth{1.003750pt}%
\definecolor{currentstroke}{rgb}{0.121569,0.466667,0.705882}%
\pgfsetstrokecolor{currentstroke}%
\pgfsetstrokeopacity{0.642319}%
\pgfsetdash{}{0pt}%
\pgfpathmoveto{\pgfqpoint{1.650770in}{2.995427in}}%
\pgfpathcurveto{\pgfqpoint{1.659006in}{2.995427in}}{\pgfqpoint{1.666906in}{2.998699in}}{\pgfqpoint{1.672730in}{3.004523in}}%
\pgfpathcurveto{\pgfqpoint{1.678554in}{3.010347in}}{\pgfqpoint{1.681827in}{3.018247in}}{\pgfqpoint{1.681827in}{3.026484in}}%
\pgfpathcurveto{\pgfqpoint{1.681827in}{3.034720in}}{\pgfqpoint{1.678554in}{3.042620in}}{\pgfqpoint{1.672730in}{3.048444in}}%
\pgfpathcurveto{\pgfqpoint{1.666906in}{3.054268in}}{\pgfqpoint{1.659006in}{3.057540in}}{\pgfqpoint{1.650770in}{3.057540in}}%
\pgfpathcurveto{\pgfqpoint{1.642534in}{3.057540in}}{\pgfqpoint{1.634634in}{3.054268in}}{\pgfqpoint{1.628810in}{3.048444in}}%
\pgfpathcurveto{\pgfqpoint{1.622986in}{3.042620in}}{\pgfqpoint{1.619714in}{3.034720in}}{\pgfqpoint{1.619714in}{3.026484in}}%
\pgfpathcurveto{\pgfqpoint{1.619714in}{3.018247in}}{\pgfqpoint{1.622986in}{3.010347in}}{\pgfqpoint{1.628810in}{3.004523in}}%
\pgfpathcurveto{\pgfqpoint{1.634634in}{2.998699in}}{\pgfqpoint{1.642534in}{2.995427in}}{\pgfqpoint{1.650770in}{2.995427in}}%
\pgfpathclose%
\pgfusepath{stroke,fill}%
\end{pgfscope}%
\begin{pgfscope}%
\pgfpathrectangle{\pgfqpoint{0.100000in}{0.212622in}}{\pgfqpoint{3.696000in}{3.696000in}}%
\pgfusepath{clip}%
\pgfsetbuttcap%
\pgfsetroundjoin%
\definecolor{currentfill}{rgb}{0.121569,0.466667,0.705882}%
\pgfsetfillcolor{currentfill}%
\pgfsetfillopacity{0.642792}%
\pgfsetlinewidth{1.003750pt}%
\definecolor{currentstroke}{rgb}{0.121569,0.466667,0.705882}%
\pgfsetstrokecolor{currentstroke}%
\pgfsetstrokeopacity{0.642792}%
\pgfsetdash{}{0pt}%
\pgfpathmoveto{\pgfqpoint{1.997735in}{3.054388in}}%
\pgfpathcurveto{\pgfqpoint{2.005971in}{3.054388in}}{\pgfqpoint{2.013871in}{3.057660in}}{\pgfqpoint{2.019695in}{3.063484in}}%
\pgfpathcurveto{\pgfqpoint{2.025519in}{3.069308in}}{\pgfqpoint{2.028792in}{3.077208in}}{\pgfqpoint{2.028792in}{3.085444in}}%
\pgfpathcurveto{\pgfqpoint{2.028792in}{3.093680in}}{\pgfqpoint{2.025519in}{3.101580in}}{\pgfqpoint{2.019695in}{3.107404in}}%
\pgfpathcurveto{\pgfqpoint{2.013871in}{3.113228in}}{\pgfqpoint{2.005971in}{3.116501in}}{\pgfqpoint{1.997735in}{3.116501in}}%
\pgfpathcurveto{\pgfqpoint{1.989499in}{3.116501in}}{\pgfqpoint{1.981599in}{3.113228in}}{\pgfqpoint{1.975775in}{3.107404in}}%
\pgfpathcurveto{\pgfqpoint{1.969951in}{3.101580in}}{\pgfqpoint{1.966679in}{3.093680in}}{\pgfqpoint{1.966679in}{3.085444in}}%
\pgfpathcurveto{\pgfqpoint{1.966679in}{3.077208in}}{\pgfqpoint{1.969951in}{3.069308in}}{\pgfqpoint{1.975775in}{3.063484in}}%
\pgfpathcurveto{\pgfqpoint{1.981599in}{3.057660in}}{\pgfqpoint{1.989499in}{3.054388in}}{\pgfqpoint{1.997735in}{3.054388in}}%
\pgfpathclose%
\pgfusepath{stroke,fill}%
\end{pgfscope}%
\begin{pgfscope}%
\pgfpathrectangle{\pgfqpoint{0.100000in}{0.212622in}}{\pgfqpoint{3.696000in}{3.696000in}}%
\pgfusepath{clip}%
\pgfsetbuttcap%
\pgfsetroundjoin%
\definecolor{currentfill}{rgb}{0.121569,0.466667,0.705882}%
\pgfsetfillcolor{currentfill}%
\pgfsetfillopacity{0.643129}%
\pgfsetlinewidth{1.003750pt}%
\definecolor{currentstroke}{rgb}{0.121569,0.466667,0.705882}%
\pgfsetstrokecolor{currentstroke}%
\pgfsetstrokeopacity{0.643129}%
\pgfsetdash{}{0pt}%
\pgfpathmoveto{\pgfqpoint{1.648892in}{2.993128in}}%
\pgfpathcurveto{\pgfqpoint{1.657129in}{2.993128in}}{\pgfqpoint{1.665029in}{2.996400in}}{\pgfqpoint{1.670853in}{3.002224in}}%
\pgfpathcurveto{\pgfqpoint{1.676676in}{3.008048in}}{\pgfqpoint{1.679949in}{3.015948in}}{\pgfqpoint{1.679949in}{3.024185in}}%
\pgfpathcurveto{\pgfqpoint{1.679949in}{3.032421in}}{\pgfqpoint{1.676676in}{3.040321in}}{\pgfqpoint{1.670853in}{3.046145in}}%
\pgfpathcurveto{\pgfqpoint{1.665029in}{3.051969in}}{\pgfqpoint{1.657129in}{3.055241in}}{\pgfqpoint{1.648892in}{3.055241in}}%
\pgfpathcurveto{\pgfqpoint{1.640656in}{3.055241in}}{\pgfqpoint{1.632756in}{3.051969in}}{\pgfqpoint{1.626932in}{3.046145in}}%
\pgfpathcurveto{\pgfqpoint{1.621108in}{3.040321in}}{\pgfqpoint{1.617836in}{3.032421in}}{\pgfqpoint{1.617836in}{3.024185in}}%
\pgfpathcurveto{\pgfqpoint{1.617836in}{3.015948in}}{\pgfqpoint{1.621108in}{3.008048in}}{\pgfqpoint{1.626932in}{3.002224in}}%
\pgfpathcurveto{\pgfqpoint{1.632756in}{2.996400in}}{\pgfqpoint{1.640656in}{2.993128in}}{\pgfqpoint{1.648892in}{2.993128in}}%
\pgfpathclose%
\pgfusepath{stroke,fill}%
\end{pgfscope}%
\begin{pgfscope}%
\pgfpathrectangle{\pgfqpoint{0.100000in}{0.212622in}}{\pgfqpoint{3.696000in}{3.696000in}}%
\pgfusepath{clip}%
\pgfsetbuttcap%
\pgfsetroundjoin%
\definecolor{currentfill}{rgb}{0.121569,0.466667,0.705882}%
\pgfsetfillcolor{currentfill}%
\pgfsetfillopacity{0.643603}%
\pgfsetlinewidth{1.003750pt}%
\definecolor{currentstroke}{rgb}{0.121569,0.466667,0.705882}%
\pgfsetstrokecolor{currentstroke}%
\pgfsetstrokeopacity{0.643603}%
\pgfsetdash{}{0pt}%
\pgfpathmoveto{\pgfqpoint{1.647867in}{2.992033in}}%
\pgfpathcurveto{\pgfqpoint{1.656103in}{2.992033in}}{\pgfqpoint{1.664003in}{2.995305in}}{\pgfqpoint{1.669827in}{3.001129in}}%
\pgfpathcurveto{\pgfqpoint{1.675651in}{3.006953in}}{\pgfqpoint{1.678923in}{3.014853in}}{\pgfqpoint{1.678923in}{3.023090in}}%
\pgfpathcurveto{\pgfqpoint{1.678923in}{3.031326in}}{\pgfqpoint{1.675651in}{3.039226in}}{\pgfqpoint{1.669827in}{3.045050in}}%
\pgfpathcurveto{\pgfqpoint{1.664003in}{3.050874in}}{\pgfqpoint{1.656103in}{3.054146in}}{\pgfqpoint{1.647867in}{3.054146in}}%
\pgfpathcurveto{\pgfqpoint{1.639631in}{3.054146in}}{\pgfqpoint{1.631731in}{3.050874in}}{\pgfqpoint{1.625907in}{3.045050in}}%
\pgfpathcurveto{\pgfqpoint{1.620083in}{3.039226in}}{\pgfqpoint{1.616810in}{3.031326in}}{\pgfqpoint{1.616810in}{3.023090in}}%
\pgfpathcurveto{\pgfqpoint{1.616810in}{3.014853in}}{\pgfqpoint{1.620083in}{3.006953in}}{\pgfqpoint{1.625907in}{3.001129in}}%
\pgfpathcurveto{\pgfqpoint{1.631731in}{2.995305in}}{\pgfqpoint{1.639631in}{2.992033in}}{\pgfqpoint{1.647867in}{2.992033in}}%
\pgfpathclose%
\pgfusepath{stroke,fill}%
\end{pgfscope}%
\begin{pgfscope}%
\pgfpathrectangle{\pgfqpoint{0.100000in}{0.212622in}}{\pgfqpoint{3.696000in}{3.696000in}}%
\pgfusepath{clip}%
\pgfsetbuttcap%
\pgfsetroundjoin%
\definecolor{currentfill}{rgb}{0.121569,0.466667,0.705882}%
\pgfsetfillcolor{currentfill}%
\pgfsetfillopacity{0.643648}%
\pgfsetlinewidth{1.003750pt}%
\definecolor{currentstroke}{rgb}{0.121569,0.466667,0.705882}%
\pgfsetstrokecolor{currentstroke}%
\pgfsetstrokeopacity{0.643648}%
\pgfsetdash{}{0pt}%
\pgfpathmoveto{\pgfqpoint{2.004749in}{3.052859in}}%
\pgfpathcurveto{\pgfqpoint{2.012985in}{3.052859in}}{\pgfqpoint{2.020885in}{3.056131in}}{\pgfqpoint{2.026709in}{3.061955in}}%
\pgfpathcurveto{\pgfqpoint{2.032533in}{3.067779in}}{\pgfqpoint{2.035805in}{3.075679in}}{\pgfqpoint{2.035805in}{3.083915in}}%
\pgfpathcurveto{\pgfqpoint{2.035805in}{3.092152in}}{\pgfqpoint{2.032533in}{3.100052in}}{\pgfqpoint{2.026709in}{3.105876in}}%
\pgfpathcurveto{\pgfqpoint{2.020885in}{3.111700in}}{\pgfqpoint{2.012985in}{3.114972in}}{\pgfqpoint{2.004749in}{3.114972in}}%
\pgfpathcurveto{\pgfqpoint{1.996513in}{3.114972in}}{\pgfqpoint{1.988613in}{3.111700in}}{\pgfqpoint{1.982789in}{3.105876in}}%
\pgfpathcurveto{\pgfqpoint{1.976965in}{3.100052in}}{\pgfqpoint{1.973692in}{3.092152in}}{\pgfqpoint{1.973692in}{3.083915in}}%
\pgfpathcurveto{\pgfqpoint{1.973692in}{3.075679in}}{\pgfqpoint{1.976965in}{3.067779in}}{\pgfqpoint{1.982789in}{3.061955in}}%
\pgfpathcurveto{\pgfqpoint{1.988613in}{3.056131in}}{\pgfqpoint{1.996513in}{3.052859in}}{\pgfqpoint{2.004749in}{3.052859in}}%
\pgfpathclose%
\pgfusepath{stroke,fill}%
\end{pgfscope}%
\begin{pgfscope}%
\pgfpathrectangle{\pgfqpoint{0.100000in}{0.212622in}}{\pgfqpoint{3.696000in}{3.696000in}}%
\pgfusepath{clip}%
\pgfsetbuttcap%
\pgfsetroundjoin%
\definecolor{currentfill}{rgb}{0.121569,0.466667,0.705882}%
\pgfsetfillcolor{currentfill}%
\pgfsetfillopacity{0.644117}%
\pgfsetlinewidth{1.003750pt}%
\definecolor{currentstroke}{rgb}{0.121569,0.466667,0.705882}%
\pgfsetstrokecolor{currentstroke}%
\pgfsetstrokeopacity{0.644117}%
\pgfsetdash{}{0pt}%
\pgfpathmoveto{\pgfqpoint{3.177225in}{1.893536in}}%
\pgfpathcurveto{\pgfqpoint{3.185461in}{1.893536in}}{\pgfqpoint{3.193361in}{1.896808in}}{\pgfqpoint{3.199185in}{1.902632in}}%
\pgfpathcurveto{\pgfqpoint{3.205009in}{1.908456in}}{\pgfqpoint{3.208281in}{1.916356in}}{\pgfqpoint{3.208281in}{1.924592in}}%
\pgfpathcurveto{\pgfqpoint{3.208281in}{1.932829in}}{\pgfqpoint{3.205009in}{1.940729in}}{\pgfqpoint{3.199185in}{1.946553in}}%
\pgfpathcurveto{\pgfqpoint{3.193361in}{1.952377in}}{\pgfqpoint{3.185461in}{1.955649in}}{\pgfqpoint{3.177225in}{1.955649in}}%
\pgfpathcurveto{\pgfqpoint{3.168988in}{1.955649in}}{\pgfqpoint{3.161088in}{1.952377in}}{\pgfqpoint{3.155265in}{1.946553in}}%
\pgfpathcurveto{\pgfqpoint{3.149441in}{1.940729in}}{\pgfqpoint{3.146168in}{1.932829in}}{\pgfqpoint{3.146168in}{1.924592in}}%
\pgfpathcurveto{\pgfqpoint{3.146168in}{1.916356in}}{\pgfqpoint{3.149441in}{1.908456in}}{\pgfqpoint{3.155265in}{1.902632in}}%
\pgfpathcurveto{\pgfqpoint{3.161088in}{1.896808in}}{\pgfqpoint{3.168988in}{1.893536in}}{\pgfqpoint{3.177225in}{1.893536in}}%
\pgfpathclose%
\pgfusepath{stroke,fill}%
\end{pgfscope}%
\begin{pgfscope}%
\pgfpathrectangle{\pgfqpoint{0.100000in}{0.212622in}}{\pgfqpoint{3.696000in}{3.696000in}}%
\pgfusepath{clip}%
\pgfsetbuttcap%
\pgfsetroundjoin%
\definecolor{currentfill}{rgb}{0.121569,0.466667,0.705882}%
\pgfsetfillcolor{currentfill}%
\pgfsetfillopacity{0.644311}%
\pgfsetlinewidth{1.003750pt}%
\definecolor{currentstroke}{rgb}{0.121569,0.466667,0.705882}%
\pgfsetstrokecolor{currentstroke}%
\pgfsetstrokeopacity{0.644311}%
\pgfsetdash{}{0pt}%
\pgfpathmoveto{\pgfqpoint{2.010609in}{3.052210in}}%
\pgfpathcurveto{\pgfqpoint{2.018845in}{3.052210in}}{\pgfqpoint{2.026745in}{3.055482in}}{\pgfqpoint{2.032569in}{3.061306in}}%
\pgfpathcurveto{\pgfqpoint{2.038393in}{3.067130in}}{\pgfqpoint{2.041665in}{3.075030in}}{\pgfqpoint{2.041665in}{3.083266in}}%
\pgfpathcurveto{\pgfqpoint{2.041665in}{3.091502in}}{\pgfqpoint{2.038393in}{3.099402in}}{\pgfqpoint{2.032569in}{3.105226in}}%
\pgfpathcurveto{\pgfqpoint{2.026745in}{3.111050in}}{\pgfqpoint{2.018845in}{3.114323in}}{\pgfqpoint{2.010609in}{3.114323in}}%
\pgfpathcurveto{\pgfqpoint{2.002373in}{3.114323in}}{\pgfqpoint{1.994472in}{3.111050in}}{\pgfqpoint{1.988649in}{3.105226in}}%
\pgfpathcurveto{\pgfqpoint{1.982825in}{3.099402in}}{\pgfqpoint{1.979552in}{3.091502in}}{\pgfqpoint{1.979552in}{3.083266in}}%
\pgfpathcurveto{\pgfqpoint{1.979552in}{3.075030in}}{\pgfqpoint{1.982825in}{3.067130in}}{\pgfqpoint{1.988649in}{3.061306in}}%
\pgfpathcurveto{\pgfqpoint{1.994472in}{3.055482in}}{\pgfqpoint{2.002373in}{3.052210in}}{\pgfqpoint{2.010609in}{3.052210in}}%
\pgfpathclose%
\pgfusepath{stroke,fill}%
\end{pgfscope}%
\begin{pgfscope}%
\pgfpathrectangle{\pgfqpoint{0.100000in}{0.212622in}}{\pgfqpoint{3.696000in}{3.696000in}}%
\pgfusepath{clip}%
\pgfsetbuttcap%
\pgfsetroundjoin%
\definecolor{currentfill}{rgb}{0.121569,0.466667,0.705882}%
\pgfsetfillcolor{currentfill}%
\pgfsetfillopacity{0.644401}%
\pgfsetlinewidth{1.003750pt}%
\definecolor{currentstroke}{rgb}{0.121569,0.466667,0.705882}%
\pgfsetstrokecolor{currentstroke}%
\pgfsetstrokeopacity{0.644401}%
\pgfsetdash{}{0pt}%
\pgfpathmoveto{\pgfqpoint{1.646068in}{2.989951in}}%
\pgfpathcurveto{\pgfqpoint{1.654304in}{2.989951in}}{\pgfqpoint{1.662204in}{2.993223in}}{\pgfqpoint{1.668028in}{2.999047in}}%
\pgfpathcurveto{\pgfqpoint{1.673852in}{3.004871in}}{\pgfqpoint{1.677125in}{3.012771in}}{\pgfqpoint{1.677125in}{3.021007in}}%
\pgfpathcurveto{\pgfqpoint{1.677125in}{3.029244in}}{\pgfqpoint{1.673852in}{3.037144in}}{\pgfqpoint{1.668028in}{3.042968in}}%
\pgfpathcurveto{\pgfqpoint{1.662204in}{3.048791in}}{\pgfqpoint{1.654304in}{3.052064in}}{\pgfqpoint{1.646068in}{3.052064in}}%
\pgfpathcurveto{\pgfqpoint{1.637832in}{3.052064in}}{\pgfqpoint{1.629932in}{3.048791in}}{\pgfqpoint{1.624108in}{3.042968in}}%
\pgfpathcurveto{\pgfqpoint{1.618284in}{3.037144in}}{\pgfqpoint{1.615012in}{3.029244in}}{\pgfqpoint{1.615012in}{3.021007in}}%
\pgfpathcurveto{\pgfqpoint{1.615012in}{3.012771in}}{\pgfqpoint{1.618284in}{3.004871in}}{\pgfqpoint{1.624108in}{2.999047in}}%
\pgfpathcurveto{\pgfqpoint{1.629932in}{2.993223in}}{\pgfqpoint{1.637832in}{2.989951in}}{\pgfqpoint{1.646068in}{2.989951in}}%
\pgfpathclose%
\pgfusepath{stroke,fill}%
\end{pgfscope}%
\begin{pgfscope}%
\pgfpathrectangle{\pgfqpoint{0.100000in}{0.212622in}}{\pgfqpoint{3.696000in}{3.696000in}}%
\pgfusepath{clip}%
\pgfsetbuttcap%
\pgfsetroundjoin%
\definecolor{currentfill}{rgb}{0.121569,0.466667,0.705882}%
\pgfsetfillcolor{currentfill}%
\pgfsetfillopacity{0.645292}%
\pgfsetlinewidth{1.003750pt}%
\definecolor{currentstroke}{rgb}{0.121569,0.466667,0.705882}%
\pgfsetstrokecolor{currentstroke}%
\pgfsetstrokeopacity{0.645292}%
\pgfsetdash{}{0pt}%
\pgfpathmoveto{\pgfqpoint{1.643951in}{2.986919in}}%
\pgfpathcurveto{\pgfqpoint{1.652187in}{2.986919in}}{\pgfqpoint{1.660087in}{2.990191in}}{\pgfqpoint{1.665911in}{2.996015in}}%
\pgfpathcurveto{\pgfqpoint{1.671735in}{3.001839in}}{\pgfqpoint{1.675007in}{3.009739in}}{\pgfqpoint{1.675007in}{3.017976in}}%
\pgfpathcurveto{\pgfqpoint{1.675007in}{3.026212in}}{\pgfqpoint{1.671735in}{3.034112in}}{\pgfqpoint{1.665911in}{3.039936in}}%
\pgfpathcurveto{\pgfqpoint{1.660087in}{3.045760in}}{\pgfqpoint{1.652187in}{3.049032in}}{\pgfqpoint{1.643951in}{3.049032in}}%
\pgfpathcurveto{\pgfqpoint{1.635714in}{3.049032in}}{\pgfqpoint{1.627814in}{3.045760in}}{\pgfqpoint{1.621991in}{3.039936in}}%
\pgfpathcurveto{\pgfqpoint{1.616167in}{3.034112in}}{\pgfqpoint{1.612894in}{3.026212in}}{\pgfqpoint{1.612894in}{3.017976in}}%
\pgfpathcurveto{\pgfqpoint{1.612894in}{3.009739in}}{\pgfqpoint{1.616167in}{3.001839in}}{\pgfqpoint{1.621991in}{2.996015in}}%
\pgfpathcurveto{\pgfqpoint{1.627814in}{2.990191in}}{\pgfqpoint{1.635714in}{2.986919in}}{\pgfqpoint{1.643951in}{2.986919in}}%
\pgfpathclose%
\pgfusepath{stroke,fill}%
\end{pgfscope}%
\begin{pgfscope}%
\pgfpathrectangle{\pgfqpoint{0.100000in}{0.212622in}}{\pgfqpoint{3.696000in}{3.696000in}}%
\pgfusepath{clip}%
\pgfsetbuttcap%
\pgfsetroundjoin%
\definecolor{currentfill}{rgb}{0.121569,0.466667,0.705882}%
\pgfsetfillcolor{currentfill}%
\pgfsetfillopacity{0.645582}%
\pgfsetlinewidth{1.003750pt}%
\definecolor{currentstroke}{rgb}{0.121569,0.466667,0.705882}%
\pgfsetstrokecolor{currentstroke}%
\pgfsetstrokeopacity{0.645582}%
\pgfsetdash{}{0pt}%
\pgfpathmoveto{\pgfqpoint{2.021305in}{3.051588in}}%
\pgfpathcurveto{\pgfqpoint{2.029541in}{3.051588in}}{\pgfqpoint{2.037441in}{3.054860in}}{\pgfqpoint{2.043265in}{3.060684in}}%
\pgfpathcurveto{\pgfqpoint{2.049089in}{3.066508in}}{\pgfqpoint{2.052361in}{3.074408in}}{\pgfqpoint{2.052361in}{3.082644in}}%
\pgfpathcurveto{\pgfqpoint{2.052361in}{3.090880in}}{\pgfqpoint{2.049089in}{3.098780in}}{\pgfqpoint{2.043265in}{3.104604in}}%
\pgfpathcurveto{\pgfqpoint{2.037441in}{3.110428in}}{\pgfqpoint{2.029541in}{3.113701in}}{\pgfqpoint{2.021305in}{3.113701in}}%
\pgfpathcurveto{\pgfqpoint{2.013069in}{3.113701in}}{\pgfqpoint{2.005168in}{3.110428in}}{\pgfqpoint{1.999345in}{3.104604in}}%
\pgfpathcurveto{\pgfqpoint{1.993521in}{3.098780in}}{\pgfqpoint{1.990248in}{3.090880in}}{\pgfqpoint{1.990248in}{3.082644in}}%
\pgfpathcurveto{\pgfqpoint{1.990248in}{3.074408in}}{\pgfqpoint{1.993521in}{3.066508in}}{\pgfqpoint{1.999345in}{3.060684in}}%
\pgfpathcurveto{\pgfqpoint{2.005168in}{3.054860in}}{\pgfqpoint{2.013069in}{3.051588in}}{\pgfqpoint{2.021305in}{3.051588in}}%
\pgfpathclose%
\pgfusepath{stroke,fill}%
\end{pgfscope}%
\begin{pgfscope}%
\pgfpathrectangle{\pgfqpoint{0.100000in}{0.212622in}}{\pgfqpoint{3.696000in}{3.696000in}}%
\pgfusepath{clip}%
\pgfsetbuttcap%
\pgfsetroundjoin%
\definecolor{currentfill}{rgb}{0.121569,0.466667,0.705882}%
\pgfsetfillcolor{currentfill}%
\pgfsetfillopacity{0.646040}%
\pgfsetlinewidth{1.003750pt}%
\definecolor{currentstroke}{rgb}{0.121569,0.466667,0.705882}%
\pgfsetstrokecolor{currentstroke}%
\pgfsetstrokeopacity{0.646040}%
\pgfsetdash{}{0pt}%
\pgfpathmoveto{\pgfqpoint{3.173138in}{1.887714in}}%
\pgfpathcurveto{\pgfqpoint{3.181374in}{1.887714in}}{\pgfqpoint{3.189274in}{1.890986in}}{\pgfqpoint{3.195098in}{1.896810in}}%
\pgfpathcurveto{\pgfqpoint{3.200922in}{1.902634in}}{\pgfqpoint{3.204194in}{1.910534in}}{\pgfqpoint{3.204194in}{1.918771in}}%
\pgfpathcurveto{\pgfqpoint{3.204194in}{1.927007in}}{\pgfqpoint{3.200922in}{1.934907in}}{\pgfqpoint{3.195098in}{1.940731in}}%
\pgfpathcurveto{\pgfqpoint{3.189274in}{1.946555in}}{\pgfqpoint{3.181374in}{1.949827in}}{\pgfqpoint{3.173138in}{1.949827in}}%
\pgfpathcurveto{\pgfqpoint{3.164901in}{1.949827in}}{\pgfqpoint{3.157001in}{1.946555in}}{\pgfqpoint{3.151177in}{1.940731in}}%
\pgfpathcurveto{\pgfqpoint{3.145353in}{1.934907in}}{\pgfqpoint{3.142081in}{1.927007in}}{\pgfqpoint{3.142081in}{1.918771in}}%
\pgfpathcurveto{\pgfqpoint{3.142081in}{1.910534in}}{\pgfqpoint{3.145353in}{1.902634in}}{\pgfqpoint{3.151177in}{1.896810in}}%
\pgfpathcurveto{\pgfqpoint{3.157001in}{1.890986in}}{\pgfqpoint{3.164901in}{1.887714in}}{\pgfqpoint{3.173138in}{1.887714in}}%
\pgfpathclose%
\pgfusepath{stroke,fill}%
\end{pgfscope}%
\begin{pgfscope}%
\pgfpathrectangle{\pgfqpoint{0.100000in}{0.212622in}}{\pgfqpoint{3.696000in}{3.696000in}}%
\pgfusepath{clip}%
\pgfsetbuttcap%
\pgfsetroundjoin%
\definecolor{currentfill}{rgb}{0.121569,0.466667,0.705882}%
\pgfsetfillcolor{currentfill}%
\pgfsetfillopacity{0.646462}%
\pgfsetlinewidth{1.003750pt}%
\definecolor{currentstroke}{rgb}{0.121569,0.466667,0.705882}%
\pgfsetstrokecolor{currentstroke}%
\pgfsetstrokeopacity{0.646462}%
\pgfsetdash{}{0pt}%
\pgfpathmoveto{\pgfqpoint{1.641336in}{2.983193in}}%
\pgfpathcurveto{\pgfqpoint{1.649572in}{2.983193in}}{\pgfqpoint{1.657473in}{2.986465in}}{\pgfqpoint{1.663296in}{2.992289in}}%
\pgfpathcurveto{\pgfqpoint{1.669120in}{2.998113in}}{\pgfqpoint{1.672393in}{3.006013in}}{\pgfqpoint{1.672393in}{3.014250in}}%
\pgfpathcurveto{\pgfqpoint{1.672393in}{3.022486in}}{\pgfqpoint{1.669120in}{3.030386in}}{\pgfqpoint{1.663296in}{3.036210in}}%
\pgfpathcurveto{\pgfqpoint{1.657473in}{3.042034in}}{\pgfqpoint{1.649572in}{3.045306in}}{\pgfqpoint{1.641336in}{3.045306in}}%
\pgfpathcurveto{\pgfqpoint{1.633100in}{3.045306in}}{\pgfqpoint{1.625200in}{3.042034in}}{\pgfqpoint{1.619376in}{3.036210in}}%
\pgfpathcurveto{\pgfqpoint{1.613552in}{3.030386in}}{\pgfqpoint{1.610280in}{3.022486in}}{\pgfqpoint{1.610280in}{3.014250in}}%
\pgfpathcurveto{\pgfqpoint{1.610280in}{3.006013in}}{\pgfqpoint{1.613552in}{2.998113in}}{\pgfqpoint{1.619376in}{2.992289in}}%
\pgfpathcurveto{\pgfqpoint{1.625200in}{2.986465in}}{\pgfqpoint{1.633100in}{2.983193in}}{\pgfqpoint{1.641336in}{2.983193in}}%
\pgfpathclose%
\pgfusepath{stroke,fill}%
\end{pgfscope}%
\begin{pgfscope}%
\pgfpathrectangle{\pgfqpoint{0.100000in}{0.212622in}}{\pgfqpoint{3.696000in}{3.696000in}}%
\pgfusepath{clip}%
\pgfsetbuttcap%
\pgfsetroundjoin%
\definecolor{currentfill}{rgb}{0.121569,0.466667,0.705882}%
\pgfsetfillcolor{currentfill}%
\pgfsetfillopacity{0.646625}%
\pgfsetlinewidth{1.003750pt}%
\definecolor{currentstroke}{rgb}{0.121569,0.466667,0.705882}%
\pgfsetstrokecolor{currentstroke}%
\pgfsetstrokeopacity{0.646625}%
\pgfsetdash{}{0pt}%
\pgfpathmoveto{\pgfqpoint{2.030443in}{3.050534in}}%
\pgfpathcurveto{\pgfqpoint{2.038680in}{3.050534in}}{\pgfqpoint{2.046580in}{3.053807in}}{\pgfqpoint{2.052404in}{3.059631in}}%
\pgfpathcurveto{\pgfqpoint{2.058228in}{3.065455in}}{\pgfqpoint{2.061500in}{3.073355in}}{\pgfqpoint{2.061500in}{3.081591in}}%
\pgfpathcurveto{\pgfqpoint{2.061500in}{3.089827in}}{\pgfqpoint{2.058228in}{3.097727in}}{\pgfqpoint{2.052404in}{3.103551in}}%
\pgfpathcurveto{\pgfqpoint{2.046580in}{3.109375in}}{\pgfqpoint{2.038680in}{3.112647in}}{\pgfqpoint{2.030443in}{3.112647in}}%
\pgfpathcurveto{\pgfqpoint{2.022207in}{3.112647in}}{\pgfqpoint{2.014307in}{3.109375in}}{\pgfqpoint{2.008483in}{3.103551in}}%
\pgfpathcurveto{\pgfqpoint{2.002659in}{3.097727in}}{\pgfqpoint{1.999387in}{3.089827in}}{\pgfqpoint{1.999387in}{3.081591in}}%
\pgfpathcurveto{\pgfqpoint{1.999387in}{3.073355in}}{\pgfqpoint{2.002659in}{3.065455in}}{\pgfqpoint{2.008483in}{3.059631in}}%
\pgfpathcurveto{\pgfqpoint{2.014307in}{3.053807in}}{\pgfqpoint{2.022207in}{3.050534in}}{\pgfqpoint{2.030443in}{3.050534in}}%
\pgfpathclose%
\pgfusepath{stroke,fill}%
\end{pgfscope}%
\begin{pgfscope}%
\pgfpathrectangle{\pgfqpoint{0.100000in}{0.212622in}}{\pgfqpoint{3.696000in}{3.696000in}}%
\pgfusepath{clip}%
\pgfsetbuttcap%
\pgfsetroundjoin%
\definecolor{currentfill}{rgb}{0.121569,0.466667,0.705882}%
\pgfsetfillcolor{currentfill}%
\pgfsetfillopacity{0.647147}%
\pgfsetlinewidth{1.003750pt}%
\definecolor{currentstroke}{rgb}{0.121569,0.466667,0.705882}%
\pgfsetstrokecolor{currentstroke}%
\pgfsetstrokeopacity{0.647147}%
\pgfsetdash{}{0pt}%
\pgfpathmoveto{\pgfqpoint{1.639918in}{2.981384in}}%
\pgfpathcurveto{\pgfqpoint{1.648154in}{2.981384in}}{\pgfqpoint{1.656054in}{2.984656in}}{\pgfqpoint{1.661878in}{2.990480in}}%
\pgfpathcurveto{\pgfqpoint{1.667702in}{2.996304in}}{\pgfqpoint{1.670975in}{3.004204in}}{\pgfqpoint{1.670975in}{3.012441in}}%
\pgfpathcurveto{\pgfqpoint{1.670975in}{3.020677in}}{\pgfqpoint{1.667702in}{3.028577in}}{\pgfqpoint{1.661878in}{3.034401in}}%
\pgfpathcurveto{\pgfqpoint{1.656054in}{3.040225in}}{\pgfqpoint{1.648154in}{3.043497in}}{\pgfqpoint{1.639918in}{3.043497in}}%
\pgfpathcurveto{\pgfqpoint{1.631682in}{3.043497in}}{\pgfqpoint{1.623782in}{3.040225in}}{\pgfqpoint{1.617958in}{3.034401in}}%
\pgfpathcurveto{\pgfqpoint{1.612134in}{3.028577in}}{\pgfqpoint{1.608862in}{3.020677in}}{\pgfqpoint{1.608862in}{3.012441in}}%
\pgfpathcurveto{\pgfqpoint{1.608862in}{3.004204in}}{\pgfqpoint{1.612134in}{2.996304in}}{\pgfqpoint{1.617958in}{2.990480in}}%
\pgfpathcurveto{\pgfqpoint{1.623782in}{2.984656in}}{\pgfqpoint{1.631682in}{2.981384in}}{\pgfqpoint{1.639918in}{2.981384in}}%
\pgfpathclose%
\pgfusepath{stroke,fill}%
\end{pgfscope}%
\begin{pgfscope}%
\pgfpathrectangle{\pgfqpoint{0.100000in}{0.212622in}}{\pgfqpoint{3.696000in}{3.696000in}}%
\pgfusepath{clip}%
\pgfsetbuttcap%
\pgfsetroundjoin%
\definecolor{currentfill}{rgb}{0.121569,0.466667,0.705882}%
\pgfsetfillcolor{currentfill}%
\pgfsetfillopacity{0.647521}%
\pgfsetlinewidth{1.003750pt}%
\definecolor{currentstroke}{rgb}{0.121569,0.466667,0.705882}%
\pgfsetstrokecolor{currentstroke}%
\pgfsetstrokeopacity{0.647521}%
\pgfsetdash{}{0pt}%
\pgfpathmoveto{\pgfqpoint{1.639128in}{2.980375in}}%
\pgfpathcurveto{\pgfqpoint{1.647364in}{2.980375in}}{\pgfqpoint{1.655264in}{2.983647in}}{\pgfqpoint{1.661088in}{2.989471in}}%
\pgfpathcurveto{\pgfqpoint{1.666912in}{2.995295in}}{\pgfqpoint{1.670184in}{3.003195in}}{\pgfqpoint{1.670184in}{3.011431in}}%
\pgfpathcurveto{\pgfqpoint{1.670184in}{3.019667in}}{\pgfqpoint{1.666912in}{3.027567in}}{\pgfqpoint{1.661088in}{3.033391in}}%
\pgfpathcurveto{\pgfqpoint{1.655264in}{3.039215in}}{\pgfqpoint{1.647364in}{3.042488in}}{\pgfqpoint{1.639128in}{3.042488in}}%
\pgfpathcurveto{\pgfqpoint{1.630891in}{3.042488in}}{\pgfqpoint{1.622991in}{3.039215in}}{\pgfqpoint{1.617168in}{3.033391in}}%
\pgfpathcurveto{\pgfqpoint{1.611344in}{3.027567in}}{\pgfqpoint{1.608071in}{3.019667in}}{\pgfqpoint{1.608071in}{3.011431in}}%
\pgfpathcurveto{\pgfqpoint{1.608071in}{3.003195in}}{\pgfqpoint{1.611344in}{2.995295in}}{\pgfqpoint{1.617168in}{2.989471in}}%
\pgfpathcurveto{\pgfqpoint{1.622991in}{2.983647in}}{\pgfqpoint{1.630891in}{2.980375in}}{\pgfqpoint{1.639128in}{2.980375in}}%
\pgfpathclose%
\pgfusepath{stroke,fill}%
\end{pgfscope}%
\begin{pgfscope}%
\pgfpathrectangle{\pgfqpoint{0.100000in}{0.212622in}}{\pgfqpoint{3.696000in}{3.696000in}}%
\pgfusepath{clip}%
\pgfsetbuttcap%
\pgfsetroundjoin%
\definecolor{currentfill}{rgb}{0.121569,0.466667,0.705882}%
\pgfsetfillcolor{currentfill}%
\pgfsetfillopacity{0.647570}%
\pgfsetlinewidth{1.003750pt}%
\definecolor{currentstroke}{rgb}{0.121569,0.466667,0.705882}%
\pgfsetstrokecolor{currentstroke}%
\pgfsetstrokeopacity{0.647570}%
\pgfsetdash{}{0pt}%
\pgfpathmoveto{\pgfqpoint{3.169923in}{1.883339in}}%
\pgfpathcurveto{\pgfqpoint{3.178159in}{1.883339in}}{\pgfqpoint{3.186059in}{1.886611in}}{\pgfqpoint{3.191883in}{1.892435in}}%
\pgfpathcurveto{\pgfqpoint{3.197707in}{1.898259in}}{\pgfqpoint{3.200979in}{1.906159in}}{\pgfqpoint{3.200979in}{1.914395in}}%
\pgfpathcurveto{\pgfqpoint{3.200979in}{1.922632in}}{\pgfqpoint{3.197707in}{1.930532in}}{\pgfqpoint{3.191883in}{1.936356in}}%
\pgfpathcurveto{\pgfqpoint{3.186059in}{1.942180in}}{\pgfqpoint{3.178159in}{1.945452in}}{\pgfqpoint{3.169923in}{1.945452in}}%
\pgfpathcurveto{\pgfqpoint{3.161686in}{1.945452in}}{\pgfqpoint{3.153786in}{1.942180in}}{\pgfqpoint{3.147962in}{1.936356in}}%
\pgfpathcurveto{\pgfqpoint{3.142138in}{1.930532in}}{\pgfqpoint{3.138866in}{1.922632in}}{\pgfqpoint{3.138866in}{1.914395in}}%
\pgfpathcurveto{\pgfqpoint{3.138866in}{1.906159in}}{\pgfqpoint{3.142138in}{1.898259in}}{\pgfqpoint{3.147962in}{1.892435in}}%
\pgfpathcurveto{\pgfqpoint{3.153786in}{1.886611in}}{\pgfqpoint{3.161686in}{1.883339in}}{\pgfqpoint{3.169923in}{1.883339in}}%
\pgfpathclose%
\pgfusepath{stroke,fill}%
\end{pgfscope}%
\begin{pgfscope}%
\pgfpathrectangle{\pgfqpoint{0.100000in}{0.212622in}}{\pgfqpoint{3.696000in}{3.696000in}}%
\pgfusepath{clip}%
\pgfsetbuttcap%
\pgfsetroundjoin%
\definecolor{currentfill}{rgb}{0.121569,0.466667,0.705882}%
\pgfsetfillcolor{currentfill}%
\pgfsetfillopacity{0.647643}%
\pgfsetlinewidth{1.003750pt}%
\definecolor{currentstroke}{rgb}{0.121569,0.466667,0.705882}%
\pgfsetstrokecolor{currentstroke}%
\pgfsetstrokeopacity{0.647643}%
\pgfsetdash{}{0pt}%
\pgfpathmoveto{\pgfqpoint{2.039144in}{3.049683in}}%
\pgfpathcurveto{\pgfqpoint{2.047380in}{3.049683in}}{\pgfqpoint{2.055280in}{3.052955in}}{\pgfqpoint{2.061104in}{3.058779in}}%
\pgfpathcurveto{\pgfqpoint{2.066928in}{3.064603in}}{\pgfqpoint{2.070200in}{3.072503in}}{\pgfqpoint{2.070200in}{3.080739in}}%
\pgfpathcurveto{\pgfqpoint{2.070200in}{3.088976in}}{\pgfqpoint{2.066928in}{3.096876in}}{\pgfqpoint{2.061104in}{3.102700in}}%
\pgfpathcurveto{\pgfqpoint{2.055280in}{3.108524in}}{\pgfqpoint{2.047380in}{3.111796in}}{\pgfqpoint{2.039144in}{3.111796in}}%
\pgfpathcurveto{\pgfqpoint{2.030908in}{3.111796in}}{\pgfqpoint{2.023007in}{3.108524in}}{\pgfqpoint{2.017184in}{3.102700in}}%
\pgfpathcurveto{\pgfqpoint{2.011360in}{3.096876in}}{\pgfqpoint{2.008087in}{3.088976in}}{\pgfqpoint{2.008087in}{3.080739in}}%
\pgfpathcurveto{\pgfqpoint{2.008087in}{3.072503in}}{\pgfqpoint{2.011360in}{3.064603in}}{\pgfqpoint{2.017184in}{3.058779in}}%
\pgfpathcurveto{\pgfqpoint{2.023007in}{3.052955in}}{\pgfqpoint{2.030908in}{3.049683in}}{\pgfqpoint{2.039144in}{3.049683in}}%
\pgfpathclose%
\pgfusepath{stroke,fill}%
\end{pgfscope}%
\begin{pgfscope}%
\pgfpathrectangle{\pgfqpoint{0.100000in}{0.212622in}}{\pgfqpoint{3.696000in}{3.696000in}}%
\pgfusepath{clip}%
\pgfsetbuttcap%
\pgfsetroundjoin%
\definecolor{currentfill}{rgb}{0.121569,0.466667,0.705882}%
\pgfsetfillcolor{currentfill}%
\pgfsetfillopacity{0.648037}%
\pgfsetlinewidth{1.003750pt}%
\definecolor{currentstroke}{rgb}{0.121569,0.466667,0.705882}%
\pgfsetstrokecolor{currentstroke}%
\pgfsetstrokeopacity{0.648037}%
\pgfsetdash{}{0pt}%
\pgfpathmoveto{\pgfqpoint{1.637960in}{2.978710in}}%
\pgfpathcurveto{\pgfqpoint{1.646196in}{2.978710in}}{\pgfqpoint{1.654096in}{2.981983in}}{\pgfqpoint{1.659920in}{2.987806in}}%
\pgfpathcurveto{\pgfqpoint{1.665744in}{2.993630in}}{\pgfqpoint{1.669016in}{3.001530in}}{\pgfqpoint{1.669016in}{3.009767in}}%
\pgfpathcurveto{\pgfqpoint{1.669016in}{3.018003in}}{\pgfqpoint{1.665744in}{3.025903in}}{\pgfqpoint{1.659920in}{3.031727in}}%
\pgfpathcurveto{\pgfqpoint{1.654096in}{3.037551in}}{\pgfqpoint{1.646196in}{3.040823in}}{\pgfqpoint{1.637960in}{3.040823in}}%
\pgfpathcurveto{\pgfqpoint{1.629724in}{3.040823in}}{\pgfqpoint{1.621824in}{3.037551in}}{\pgfqpoint{1.616000in}{3.031727in}}%
\pgfpathcurveto{\pgfqpoint{1.610176in}{3.025903in}}{\pgfqpoint{1.606903in}{3.018003in}}{\pgfqpoint{1.606903in}{3.009767in}}%
\pgfpathcurveto{\pgfqpoint{1.606903in}{3.001530in}}{\pgfqpoint{1.610176in}{2.993630in}}{\pgfqpoint{1.616000in}{2.987806in}}%
\pgfpathcurveto{\pgfqpoint{1.621824in}{2.981983in}}{\pgfqpoint{1.629724in}{2.978710in}}{\pgfqpoint{1.637960in}{2.978710in}}%
\pgfpathclose%
\pgfusepath{stroke,fill}%
\end{pgfscope}%
\begin{pgfscope}%
\pgfpathrectangle{\pgfqpoint{0.100000in}{0.212622in}}{\pgfqpoint{3.696000in}{3.696000in}}%
\pgfusepath{clip}%
\pgfsetbuttcap%
\pgfsetroundjoin%
\definecolor{currentfill}{rgb}{0.121569,0.466667,0.705882}%
\pgfsetfillcolor{currentfill}%
\pgfsetfillopacity{0.648649}%
\pgfsetlinewidth{1.003750pt}%
\definecolor{currentstroke}{rgb}{0.121569,0.466667,0.705882}%
\pgfsetstrokecolor{currentstroke}%
\pgfsetstrokeopacity{0.648649}%
\pgfsetdash{}{0pt}%
\pgfpathmoveto{\pgfqpoint{3.167724in}{1.880376in}}%
\pgfpathcurveto{\pgfqpoint{3.175960in}{1.880376in}}{\pgfqpoint{3.183860in}{1.883649in}}{\pgfqpoint{3.189684in}{1.889473in}}%
\pgfpathcurveto{\pgfqpoint{3.195508in}{1.895297in}}{\pgfqpoint{3.198780in}{1.903197in}}{\pgfqpoint{3.198780in}{1.911433in}}%
\pgfpathcurveto{\pgfqpoint{3.198780in}{1.919669in}}{\pgfqpoint{3.195508in}{1.927569in}}{\pgfqpoint{3.189684in}{1.933393in}}%
\pgfpathcurveto{\pgfqpoint{3.183860in}{1.939217in}}{\pgfqpoint{3.175960in}{1.942489in}}{\pgfqpoint{3.167724in}{1.942489in}}%
\pgfpathcurveto{\pgfqpoint{3.159487in}{1.942489in}}{\pgfqpoint{3.151587in}{1.939217in}}{\pgfqpoint{3.145764in}{1.933393in}}%
\pgfpathcurveto{\pgfqpoint{3.139940in}{1.927569in}}{\pgfqpoint{3.136667in}{1.919669in}}{\pgfqpoint{3.136667in}{1.911433in}}%
\pgfpathcurveto{\pgfqpoint{3.136667in}{1.903197in}}{\pgfqpoint{3.139940in}{1.895297in}}{\pgfqpoint{3.145764in}{1.889473in}}%
\pgfpathcurveto{\pgfqpoint{3.151587in}{1.883649in}}{\pgfqpoint{3.159487in}{1.880376in}}{\pgfqpoint{3.167724in}{1.880376in}}%
\pgfpathclose%
\pgfusepath{stroke,fill}%
\end{pgfscope}%
\begin{pgfscope}%
\pgfpathrectangle{\pgfqpoint{0.100000in}{0.212622in}}{\pgfqpoint{3.696000in}{3.696000in}}%
\pgfusepath{clip}%
\pgfsetbuttcap%
\pgfsetroundjoin%
\definecolor{currentfill}{rgb}{0.121569,0.466667,0.705882}%
\pgfsetfillcolor{currentfill}%
\pgfsetfillopacity{0.648671}%
\pgfsetlinewidth{1.003750pt}%
\definecolor{currentstroke}{rgb}{0.121569,0.466667,0.705882}%
\pgfsetstrokecolor{currentstroke}%
\pgfsetstrokeopacity{0.648671}%
\pgfsetdash{}{0pt}%
\pgfpathmoveto{\pgfqpoint{2.046853in}{3.049144in}}%
\pgfpathcurveto{\pgfqpoint{2.055090in}{3.049144in}}{\pgfqpoint{2.062990in}{3.052417in}}{\pgfqpoint{2.068814in}{3.058241in}}%
\pgfpathcurveto{\pgfqpoint{2.074637in}{3.064065in}}{\pgfqpoint{2.077910in}{3.071965in}}{\pgfqpoint{2.077910in}{3.080201in}}%
\pgfpathcurveto{\pgfqpoint{2.077910in}{3.088437in}}{\pgfqpoint{2.074637in}{3.096337in}}{\pgfqpoint{2.068814in}{3.102161in}}%
\pgfpathcurveto{\pgfqpoint{2.062990in}{3.107985in}}{\pgfqpoint{2.055090in}{3.111257in}}{\pgfqpoint{2.046853in}{3.111257in}}%
\pgfpathcurveto{\pgfqpoint{2.038617in}{3.111257in}}{\pgfqpoint{2.030717in}{3.107985in}}{\pgfqpoint{2.024893in}{3.102161in}}%
\pgfpathcurveto{\pgfqpoint{2.019069in}{3.096337in}}{\pgfqpoint{2.015797in}{3.088437in}}{\pgfqpoint{2.015797in}{3.080201in}}%
\pgfpathcurveto{\pgfqpoint{2.015797in}{3.071965in}}{\pgfqpoint{2.019069in}{3.064065in}}{\pgfqpoint{2.024893in}{3.058241in}}%
\pgfpathcurveto{\pgfqpoint{2.030717in}{3.052417in}}{\pgfqpoint{2.038617in}{3.049144in}}{\pgfqpoint{2.046853in}{3.049144in}}%
\pgfpathclose%
\pgfusepath{stroke,fill}%
\end{pgfscope}%
\begin{pgfscope}%
\pgfpathrectangle{\pgfqpoint{0.100000in}{0.212622in}}{\pgfqpoint{3.696000in}{3.696000in}}%
\pgfusepath{clip}%
\pgfsetbuttcap%
\pgfsetroundjoin%
\definecolor{currentfill}{rgb}{0.121569,0.466667,0.705882}%
\pgfsetfillcolor{currentfill}%
\pgfsetfillopacity{0.648681}%
\pgfsetlinewidth{1.003750pt}%
\definecolor{currentstroke}{rgb}{0.121569,0.466667,0.705882}%
\pgfsetstrokecolor{currentstroke}%
\pgfsetstrokeopacity{0.648681}%
\pgfsetdash{}{0pt}%
\pgfpathmoveto{\pgfqpoint{1.636548in}{2.976659in}}%
\pgfpathcurveto{\pgfqpoint{1.644785in}{2.976659in}}{\pgfqpoint{1.652685in}{2.979932in}}{\pgfqpoint{1.658509in}{2.985756in}}%
\pgfpathcurveto{\pgfqpoint{1.664333in}{2.991579in}}{\pgfqpoint{1.667605in}{2.999480in}}{\pgfqpoint{1.667605in}{3.007716in}}%
\pgfpathcurveto{\pgfqpoint{1.667605in}{3.015952in}}{\pgfqpoint{1.664333in}{3.023852in}}{\pgfqpoint{1.658509in}{3.029676in}}%
\pgfpathcurveto{\pgfqpoint{1.652685in}{3.035500in}}{\pgfqpoint{1.644785in}{3.038772in}}{\pgfqpoint{1.636548in}{3.038772in}}%
\pgfpathcurveto{\pgfqpoint{1.628312in}{3.038772in}}{\pgfqpoint{1.620412in}{3.035500in}}{\pgfqpoint{1.614588in}{3.029676in}}%
\pgfpathcurveto{\pgfqpoint{1.608764in}{3.023852in}}{\pgfqpoint{1.605492in}{3.015952in}}{\pgfqpoint{1.605492in}{3.007716in}}%
\pgfpathcurveto{\pgfqpoint{1.605492in}{2.999480in}}{\pgfqpoint{1.608764in}{2.991579in}}{\pgfqpoint{1.614588in}{2.985756in}}%
\pgfpathcurveto{\pgfqpoint{1.620412in}{2.979932in}}{\pgfqpoint{1.628312in}{2.976659in}}{\pgfqpoint{1.636548in}{2.976659in}}%
\pgfpathclose%
\pgfusepath{stroke,fill}%
\end{pgfscope}%
\begin{pgfscope}%
\pgfpathrectangle{\pgfqpoint{0.100000in}{0.212622in}}{\pgfqpoint{3.696000in}{3.696000in}}%
\pgfusepath{clip}%
\pgfsetbuttcap%
\pgfsetroundjoin%
\definecolor{currentfill}{rgb}{0.121569,0.466667,0.705882}%
\pgfsetfillcolor{currentfill}%
\pgfsetfillopacity{0.649564}%
\pgfsetlinewidth{1.003750pt}%
\definecolor{currentstroke}{rgb}{0.121569,0.466667,0.705882}%
\pgfsetstrokecolor{currentstroke}%
\pgfsetstrokeopacity{0.649564}%
\pgfsetdash{}{0pt}%
\pgfpathmoveto{\pgfqpoint{2.053272in}{3.049037in}}%
\pgfpathcurveto{\pgfqpoint{2.061508in}{3.049037in}}{\pgfqpoint{2.069408in}{3.052309in}}{\pgfqpoint{2.075232in}{3.058133in}}%
\pgfpathcurveto{\pgfqpoint{2.081056in}{3.063957in}}{\pgfqpoint{2.084328in}{3.071857in}}{\pgfqpoint{2.084328in}{3.080094in}}%
\pgfpathcurveto{\pgfqpoint{2.084328in}{3.088330in}}{\pgfqpoint{2.081056in}{3.096230in}}{\pgfqpoint{2.075232in}{3.102054in}}%
\pgfpathcurveto{\pgfqpoint{2.069408in}{3.107878in}}{\pgfqpoint{2.061508in}{3.111150in}}{\pgfqpoint{2.053272in}{3.111150in}}%
\pgfpathcurveto{\pgfqpoint{2.045036in}{3.111150in}}{\pgfqpoint{2.037136in}{3.107878in}}{\pgfqpoint{2.031312in}{3.102054in}}%
\pgfpathcurveto{\pgfqpoint{2.025488in}{3.096230in}}{\pgfqpoint{2.022215in}{3.088330in}}{\pgfqpoint{2.022215in}{3.080094in}}%
\pgfpathcurveto{\pgfqpoint{2.022215in}{3.071857in}}{\pgfqpoint{2.025488in}{3.063957in}}{\pgfqpoint{2.031312in}{3.058133in}}%
\pgfpathcurveto{\pgfqpoint{2.037136in}{3.052309in}}{\pgfqpoint{2.045036in}{3.049037in}}{\pgfqpoint{2.053272in}{3.049037in}}%
\pgfpathclose%
\pgfusepath{stroke,fill}%
\end{pgfscope}%
\begin{pgfscope}%
\pgfpathrectangle{\pgfqpoint{0.100000in}{0.212622in}}{\pgfqpoint{3.696000in}{3.696000in}}%
\pgfusepath{clip}%
\pgfsetbuttcap%
\pgfsetroundjoin%
\definecolor{currentfill}{rgb}{0.121569,0.466667,0.705882}%
\pgfsetfillcolor{currentfill}%
\pgfsetfillopacity{0.649675}%
\pgfsetlinewidth{1.003750pt}%
\definecolor{currentstroke}{rgb}{0.121569,0.466667,0.705882}%
\pgfsetstrokecolor{currentstroke}%
\pgfsetstrokeopacity{0.649675}%
\pgfsetdash{}{0pt}%
\pgfpathmoveto{\pgfqpoint{1.634528in}{2.973867in}}%
\pgfpathcurveto{\pgfqpoint{1.642765in}{2.973867in}}{\pgfqpoint{1.650665in}{2.977140in}}{\pgfqpoint{1.656489in}{2.982964in}}%
\pgfpathcurveto{\pgfqpoint{1.662313in}{2.988788in}}{\pgfqpoint{1.665585in}{2.996688in}}{\pgfqpoint{1.665585in}{3.004924in}}%
\pgfpathcurveto{\pgfqpoint{1.665585in}{3.013160in}}{\pgfqpoint{1.662313in}{3.021060in}}{\pgfqpoint{1.656489in}{3.026884in}}%
\pgfpathcurveto{\pgfqpoint{1.650665in}{3.032708in}}{\pgfqpoint{1.642765in}{3.035980in}}{\pgfqpoint{1.634528in}{3.035980in}}%
\pgfpathcurveto{\pgfqpoint{1.626292in}{3.035980in}}{\pgfqpoint{1.618392in}{3.032708in}}{\pgfqpoint{1.612568in}{3.026884in}}%
\pgfpathcurveto{\pgfqpoint{1.606744in}{3.021060in}}{\pgfqpoint{1.603472in}{3.013160in}}{\pgfqpoint{1.603472in}{3.004924in}}%
\pgfpathcurveto{\pgfqpoint{1.603472in}{2.996688in}}{\pgfqpoint{1.606744in}{2.988788in}}{\pgfqpoint{1.612568in}{2.982964in}}%
\pgfpathcurveto{\pgfqpoint{1.618392in}{2.977140in}}{\pgfqpoint{1.626292in}{2.973867in}}{\pgfqpoint{1.634528in}{2.973867in}}%
\pgfpathclose%
\pgfusepath{stroke,fill}%
\end{pgfscope}%
\begin{pgfscope}%
\pgfpathrectangle{\pgfqpoint{0.100000in}{0.212622in}}{\pgfqpoint{3.696000in}{3.696000in}}%
\pgfusepath{clip}%
\pgfsetbuttcap%
\pgfsetroundjoin%
\definecolor{currentfill}{rgb}{0.121569,0.466667,0.705882}%
\pgfsetfillcolor{currentfill}%
\pgfsetfillopacity{0.650189}%
\pgfsetlinewidth{1.003750pt}%
\definecolor{currentstroke}{rgb}{0.121569,0.466667,0.705882}%
\pgfsetstrokecolor{currentstroke}%
\pgfsetstrokeopacity{0.650189}%
\pgfsetdash{}{0pt}%
\pgfpathmoveto{\pgfqpoint{2.059410in}{3.048297in}}%
\pgfpathcurveto{\pgfqpoint{2.067646in}{3.048297in}}{\pgfqpoint{2.075546in}{3.051570in}}{\pgfqpoint{2.081370in}{3.057394in}}%
\pgfpathcurveto{\pgfqpoint{2.087194in}{3.063217in}}{\pgfqpoint{2.090466in}{3.071117in}}{\pgfqpoint{2.090466in}{3.079354in}}%
\pgfpathcurveto{\pgfqpoint{2.090466in}{3.087590in}}{\pgfqpoint{2.087194in}{3.095490in}}{\pgfqpoint{2.081370in}{3.101314in}}%
\pgfpathcurveto{\pgfqpoint{2.075546in}{3.107138in}}{\pgfqpoint{2.067646in}{3.110410in}}{\pgfqpoint{2.059410in}{3.110410in}}%
\pgfpathcurveto{\pgfqpoint{2.051173in}{3.110410in}}{\pgfqpoint{2.043273in}{3.107138in}}{\pgfqpoint{2.037449in}{3.101314in}}%
\pgfpathcurveto{\pgfqpoint{2.031625in}{3.095490in}}{\pgfqpoint{2.028353in}{3.087590in}}{\pgfqpoint{2.028353in}{3.079354in}}%
\pgfpathcurveto{\pgfqpoint{2.028353in}{3.071117in}}{\pgfqpoint{2.031625in}{3.063217in}}{\pgfqpoint{2.037449in}{3.057394in}}%
\pgfpathcurveto{\pgfqpoint{2.043273in}{3.051570in}}{\pgfqpoint{2.051173in}{3.048297in}}{\pgfqpoint{2.059410in}{3.048297in}}%
\pgfpathclose%
\pgfusepath{stroke,fill}%
\end{pgfscope}%
\begin{pgfscope}%
\pgfpathrectangle{\pgfqpoint{0.100000in}{0.212622in}}{\pgfqpoint{3.696000in}{3.696000in}}%
\pgfusepath{clip}%
\pgfsetbuttcap%
\pgfsetroundjoin%
\definecolor{currentfill}{rgb}{0.121569,0.466667,0.705882}%
\pgfsetfillcolor{currentfill}%
\pgfsetfillopacity{0.650213}%
\pgfsetlinewidth{1.003750pt}%
\definecolor{currentstroke}{rgb}{0.121569,0.466667,0.705882}%
\pgfsetstrokecolor{currentstroke}%
\pgfsetstrokeopacity{0.650213}%
\pgfsetdash{}{0pt}%
\pgfpathmoveto{\pgfqpoint{1.633398in}{2.972287in}}%
\pgfpathcurveto{\pgfqpoint{1.641634in}{2.972287in}}{\pgfqpoint{1.649534in}{2.975560in}}{\pgfqpoint{1.655358in}{2.981384in}}%
\pgfpathcurveto{\pgfqpoint{1.661182in}{2.987208in}}{\pgfqpoint{1.664454in}{2.995108in}}{\pgfqpoint{1.664454in}{3.003344in}}%
\pgfpathcurveto{\pgfqpoint{1.664454in}{3.011580in}}{\pgfqpoint{1.661182in}{3.019480in}}{\pgfqpoint{1.655358in}{3.025304in}}%
\pgfpathcurveto{\pgfqpoint{1.649534in}{3.031128in}}{\pgfqpoint{1.641634in}{3.034400in}}{\pgfqpoint{1.633398in}{3.034400in}}%
\pgfpathcurveto{\pgfqpoint{1.625161in}{3.034400in}}{\pgfqpoint{1.617261in}{3.031128in}}{\pgfqpoint{1.611437in}{3.025304in}}%
\pgfpathcurveto{\pgfqpoint{1.605613in}{3.019480in}}{\pgfqpoint{1.602341in}{3.011580in}}{\pgfqpoint{1.602341in}{3.003344in}}%
\pgfpathcurveto{\pgfqpoint{1.602341in}{2.995108in}}{\pgfqpoint{1.605613in}{2.987208in}}{\pgfqpoint{1.611437in}{2.981384in}}%
\pgfpathcurveto{\pgfqpoint{1.617261in}{2.975560in}}{\pgfqpoint{1.625161in}{2.972287in}}{\pgfqpoint{1.633398in}{2.972287in}}%
\pgfpathclose%
\pgfusepath{stroke,fill}%
\end{pgfscope}%
\begin{pgfscope}%
\pgfpathrectangle{\pgfqpoint{0.100000in}{0.212622in}}{\pgfqpoint{3.696000in}{3.696000in}}%
\pgfusepath{clip}%
\pgfsetbuttcap%
\pgfsetroundjoin%
\definecolor{currentfill}{rgb}{0.121569,0.466667,0.705882}%
\pgfsetfillcolor{currentfill}%
\pgfsetfillopacity{0.650496}%
\pgfsetlinewidth{1.003750pt}%
\definecolor{currentstroke}{rgb}{0.121569,0.466667,0.705882}%
\pgfsetstrokecolor{currentstroke}%
\pgfsetstrokeopacity{0.650496}%
\pgfsetdash{}{0pt}%
\pgfpathmoveto{\pgfqpoint{3.163865in}{1.874158in}}%
\pgfpathcurveto{\pgfqpoint{3.172102in}{1.874158in}}{\pgfqpoint{3.180002in}{1.877430in}}{\pgfqpoint{3.185826in}{1.883254in}}%
\pgfpathcurveto{\pgfqpoint{3.191650in}{1.889078in}}{\pgfqpoint{3.194922in}{1.896978in}}{\pgfqpoint{3.194922in}{1.905215in}}%
\pgfpathcurveto{\pgfqpoint{3.194922in}{1.913451in}}{\pgfqpoint{3.191650in}{1.921351in}}{\pgfqpoint{3.185826in}{1.927175in}}%
\pgfpathcurveto{\pgfqpoint{3.180002in}{1.932999in}}{\pgfqpoint{3.172102in}{1.936271in}}{\pgfqpoint{3.163865in}{1.936271in}}%
\pgfpathcurveto{\pgfqpoint{3.155629in}{1.936271in}}{\pgfqpoint{3.147729in}{1.932999in}}{\pgfqpoint{3.141905in}{1.927175in}}%
\pgfpathcurveto{\pgfqpoint{3.136081in}{1.921351in}}{\pgfqpoint{3.132809in}{1.913451in}}{\pgfqpoint{3.132809in}{1.905215in}}%
\pgfpathcurveto{\pgfqpoint{3.132809in}{1.896978in}}{\pgfqpoint{3.136081in}{1.889078in}}{\pgfqpoint{3.141905in}{1.883254in}}%
\pgfpathcurveto{\pgfqpoint{3.147729in}{1.877430in}}{\pgfqpoint{3.155629in}{1.874158in}}{\pgfqpoint{3.163865in}{1.874158in}}%
\pgfpathclose%
\pgfusepath{stroke,fill}%
\end{pgfscope}%
\begin{pgfscope}%
\pgfpathrectangle{\pgfqpoint{0.100000in}{0.212622in}}{\pgfqpoint{3.696000in}{3.696000in}}%
\pgfusepath{clip}%
\pgfsetbuttcap%
\pgfsetroundjoin%
\definecolor{currentfill}{rgb}{0.121569,0.466667,0.705882}%
\pgfsetfillcolor{currentfill}%
\pgfsetfillopacity{0.650667}%
\pgfsetlinewidth{1.003750pt}%
\definecolor{currentstroke}{rgb}{0.121569,0.466667,0.705882}%
\pgfsetstrokecolor{currentstroke}%
\pgfsetstrokeopacity{0.650667}%
\pgfsetdash{}{0pt}%
\pgfpathmoveto{\pgfqpoint{2.064214in}{3.047490in}}%
\pgfpathcurveto{\pgfqpoint{2.072450in}{3.047490in}}{\pgfqpoint{2.080350in}{3.050762in}}{\pgfqpoint{2.086174in}{3.056586in}}%
\pgfpathcurveto{\pgfqpoint{2.091998in}{3.062410in}}{\pgfqpoint{2.095270in}{3.070310in}}{\pgfqpoint{2.095270in}{3.078546in}}%
\pgfpathcurveto{\pgfqpoint{2.095270in}{3.086783in}}{\pgfqpoint{2.091998in}{3.094683in}}{\pgfqpoint{2.086174in}{3.100507in}}%
\pgfpathcurveto{\pgfqpoint{2.080350in}{3.106331in}}{\pgfqpoint{2.072450in}{3.109603in}}{\pgfqpoint{2.064214in}{3.109603in}}%
\pgfpathcurveto{\pgfqpoint{2.055977in}{3.109603in}}{\pgfqpoint{2.048077in}{3.106331in}}{\pgfqpoint{2.042253in}{3.100507in}}%
\pgfpathcurveto{\pgfqpoint{2.036429in}{3.094683in}}{\pgfqpoint{2.033157in}{3.086783in}}{\pgfqpoint{2.033157in}{3.078546in}}%
\pgfpathcurveto{\pgfqpoint{2.033157in}{3.070310in}}{\pgfqpoint{2.036429in}{3.062410in}}{\pgfqpoint{2.042253in}{3.056586in}}%
\pgfpathcurveto{\pgfqpoint{2.048077in}{3.050762in}}{\pgfqpoint{2.055977in}{3.047490in}}{\pgfqpoint{2.064214in}{3.047490in}}%
\pgfpathclose%
\pgfusepath{stroke,fill}%
\end{pgfscope}%
\begin{pgfscope}%
\pgfpathrectangle{\pgfqpoint{0.100000in}{0.212622in}}{\pgfqpoint{3.696000in}{3.696000in}}%
\pgfusepath{clip}%
\pgfsetbuttcap%
\pgfsetroundjoin%
\definecolor{currentfill}{rgb}{0.121569,0.466667,0.705882}%
\pgfsetfillcolor{currentfill}%
\pgfsetfillopacity{0.651128}%
\pgfsetlinewidth{1.003750pt}%
\definecolor{currentstroke}{rgb}{0.121569,0.466667,0.705882}%
\pgfsetstrokecolor{currentstroke}%
\pgfsetstrokeopacity{0.651128}%
\pgfsetdash{}{0pt}%
\pgfpathmoveto{\pgfqpoint{1.631282in}{2.969145in}}%
\pgfpathcurveto{\pgfqpoint{1.639518in}{2.969145in}}{\pgfqpoint{1.647418in}{2.972418in}}{\pgfqpoint{1.653242in}{2.978242in}}%
\pgfpathcurveto{\pgfqpoint{1.659066in}{2.984066in}}{\pgfqpoint{1.662338in}{2.991966in}}{\pgfqpoint{1.662338in}{3.000202in}}%
\pgfpathcurveto{\pgfqpoint{1.662338in}{3.008438in}}{\pgfqpoint{1.659066in}{3.016338in}}{\pgfqpoint{1.653242in}{3.022162in}}%
\pgfpathcurveto{\pgfqpoint{1.647418in}{3.027986in}}{\pgfqpoint{1.639518in}{3.031258in}}{\pgfqpoint{1.631282in}{3.031258in}}%
\pgfpathcurveto{\pgfqpoint{1.623046in}{3.031258in}}{\pgfqpoint{1.615145in}{3.027986in}}{\pgfqpoint{1.609322in}{3.022162in}}%
\pgfpathcurveto{\pgfqpoint{1.603498in}{3.016338in}}{\pgfqpoint{1.600225in}{3.008438in}}{\pgfqpoint{1.600225in}{3.000202in}}%
\pgfpathcurveto{\pgfqpoint{1.600225in}{2.991966in}}{\pgfqpoint{1.603498in}{2.984066in}}{\pgfqpoint{1.609322in}{2.978242in}}%
\pgfpathcurveto{\pgfqpoint{1.615145in}{2.972418in}}{\pgfqpoint{1.623046in}{2.969145in}}{\pgfqpoint{1.631282in}{2.969145in}}%
\pgfpathclose%
\pgfusepath{stroke,fill}%
\end{pgfscope}%
\begin{pgfscope}%
\pgfpathrectangle{\pgfqpoint{0.100000in}{0.212622in}}{\pgfqpoint{3.696000in}{3.696000in}}%
\pgfusepath{clip}%
\pgfsetbuttcap%
\pgfsetroundjoin%
\definecolor{currentfill}{rgb}{0.121569,0.466667,0.705882}%
\pgfsetfillcolor{currentfill}%
\pgfsetfillopacity{0.651621}%
\pgfsetlinewidth{1.003750pt}%
\definecolor{currentstroke}{rgb}{0.121569,0.466667,0.705882}%
\pgfsetstrokecolor{currentstroke}%
\pgfsetstrokeopacity{0.651621}%
\pgfsetdash{}{0pt}%
\pgfpathmoveto{\pgfqpoint{1.630063in}{2.967395in}}%
\pgfpathcurveto{\pgfqpoint{1.638299in}{2.967395in}}{\pgfqpoint{1.646199in}{2.970667in}}{\pgfqpoint{1.652023in}{2.976491in}}%
\pgfpathcurveto{\pgfqpoint{1.657847in}{2.982315in}}{\pgfqpoint{1.661119in}{2.990215in}}{\pgfqpoint{1.661119in}{2.998451in}}%
\pgfpathcurveto{\pgfqpoint{1.661119in}{3.006688in}}{\pgfqpoint{1.657847in}{3.014588in}}{\pgfqpoint{1.652023in}{3.020412in}}%
\pgfpathcurveto{\pgfqpoint{1.646199in}{3.026236in}}{\pgfqpoint{1.638299in}{3.029508in}}{\pgfqpoint{1.630063in}{3.029508in}}%
\pgfpathcurveto{\pgfqpoint{1.621827in}{3.029508in}}{\pgfqpoint{1.613927in}{3.026236in}}{\pgfqpoint{1.608103in}{3.020412in}}%
\pgfpathcurveto{\pgfqpoint{1.602279in}{3.014588in}}{\pgfqpoint{1.599006in}{3.006688in}}{\pgfqpoint{1.599006in}{2.998451in}}%
\pgfpathcurveto{\pgfqpoint{1.599006in}{2.990215in}}{\pgfqpoint{1.602279in}{2.982315in}}{\pgfqpoint{1.608103in}{2.976491in}}%
\pgfpathcurveto{\pgfqpoint{1.613927in}{2.970667in}}{\pgfqpoint{1.621827in}{2.967395in}}{\pgfqpoint{1.630063in}{2.967395in}}%
\pgfpathclose%
\pgfusepath{stroke,fill}%
\end{pgfscope}%
\begin{pgfscope}%
\pgfpathrectangle{\pgfqpoint{0.100000in}{0.212622in}}{\pgfqpoint{3.696000in}{3.696000in}}%
\pgfusepath{clip}%
\pgfsetbuttcap%
\pgfsetroundjoin%
\definecolor{currentfill}{rgb}{0.121569,0.466667,0.705882}%
\pgfsetfillcolor{currentfill}%
\pgfsetfillopacity{0.651672}%
\pgfsetlinewidth{1.003750pt}%
\definecolor{currentstroke}{rgb}{0.121569,0.466667,0.705882}%
\pgfsetstrokecolor{currentstroke}%
\pgfsetstrokeopacity{0.651672}%
\pgfsetdash{}{0pt}%
\pgfpathmoveto{\pgfqpoint{2.072807in}{3.046272in}}%
\pgfpathcurveto{\pgfqpoint{2.081043in}{3.046272in}}{\pgfqpoint{2.088943in}{3.049545in}}{\pgfqpoint{2.094767in}{3.055369in}}%
\pgfpathcurveto{\pgfqpoint{2.100591in}{3.061193in}}{\pgfqpoint{2.103863in}{3.069093in}}{\pgfqpoint{2.103863in}{3.077329in}}%
\pgfpathcurveto{\pgfqpoint{2.103863in}{3.085565in}}{\pgfqpoint{2.100591in}{3.093465in}}{\pgfqpoint{2.094767in}{3.099289in}}%
\pgfpathcurveto{\pgfqpoint{2.088943in}{3.105113in}}{\pgfqpoint{2.081043in}{3.108385in}}{\pgfqpoint{2.072807in}{3.108385in}}%
\pgfpathcurveto{\pgfqpoint{2.064570in}{3.108385in}}{\pgfqpoint{2.056670in}{3.105113in}}{\pgfqpoint{2.050846in}{3.099289in}}%
\pgfpathcurveto{\pgfqpoint{2.045022in}{3.093465in}}{\pgfqpoint{2.041750in}{3.085565in}}{\pgfqpoint{2.041750in}{3.077329in}}%
\pgfpathcurveto{\pgfqpoint{2.041750in}{3.069093in}}{\pgfqpoint{2.045022in}{3.061193in}}{\pgfqpoint{2.050846in}{3.055369in}}%
\pgfpathcurveto{\pgfqpoint{2.056670in}{3.049545in}}{\pgfqpoint{2.064570in}{3.046272in}}{\pgfqpoint{2.072807in}{3.046272in}}%
\pgfpathclose%
\pgfusepath{stroke,fill}%
\end{pgfscope}%
\begin{pgfscope}%
\pgfpathrectangle{\pgfqpoint{0.100000in}{0.212622in}}{\pgfqpoint{3.696000in}{3.696000in}}%
\pgfusepath{clip}%
\pgfsetbuttcap%
\pgfsetroundjoin%
\definecolor{currentfill}{rgb}{0.121569,0.466667,0.705882}%
\pgfsetfillcolor{currentfill}%
\pgfsetfillopacity{0.652108}%
\pgfsetlinewidth{1.003750pt}%
\definecolor{currentstroke}{rgb}{0.121569,0.466667,0.705882}%
\pgfsetstrokecolor{currentstroke}%
\pgfsetstrokeopacity{0.652108}%
\pgfsetdash{}{0pt}%
\pgfpathmoveto{\pgfqpoint{3.160373in}{1.868234in}}%
\pgfpathcurveto{\pgfqpoint{3.168609in}{1.868234in}}{\pgfqpoint{3.176509in}{1.871506in}}{\pgfqpoint{3.182333in}{1.877330in}}%
\pgfpathcurveto{\pgfqpoint{3.188157in}{1.883154in}}{\pgfqpoint{3.191429in}{1.891054in}}{\pgfqpoint{3.191429in}{1.899290in}}%
\pgfpathcurveto{\pgfqpoint{3.191429in}{1.907526in}}{\pgfqpoint{3.188157in}{1.915426in}}{\pgfqpoint{3.182333in}{1.921250in}}%
\pgfpathcurveto{\pgfqpoint{3.176509in}{1.927074in}}{\pgfqpoint{3.168609in}{1.930347in}}{\pgfqpoint{3.160373in}{1.930347in}}%
\pgfpathcurveto{\pgfqpoint{3.152137in}{1.930347in}}{\pgfqpoint{3.144237in}{1.927074in}}{\pgfqpoint{3.138413in}{1.921250in}}%
\pgfpathcurveto{\pgfqpoint{3.132589in}{1.915426in}}{\pgfqpoint{3.129316in}{1.907526in}}{\pgfqpoint{3.129316in}{1.899290in}}%
\pgfpathcurveto{\pgfqpoint{3.129316in}{1.891054in}}{\pgfqpoint{3.132589in}{1.883154in}}{\pgfqpoint{3.138413in}{1.877330in}}%
\pgfpathcurveto{\pgfqpoint{3.144237in}{1.871506in}}{\pgfqpoint{3.152137in}{1.868234in}}{\pgfqpoint{3.160373in}{1.868234in}}%
\pgfpathclose%
\pgfusepath{stroke,fill}%
\end{pgfscope}%
\begin{pgfscope}%
\pgfpathrectangle{\pgfqpoint{0.100000in}{0.212622in}}{\pgfqpoint{3.696000in}{3.696000in}}%
\pgfusepath{clip}%
\pgfsetbuttcap%
\pgfsetroundjoin%
\definecolor{currentfill}{rgb}{0.121569,0.466667,0.705882}%
\pgfsetfillcolor{currentfill}%
\pgfsetfillopacity{0.652345}%
\pgfsetlinewidth{1.003750pt}%
\definecolor{currentstroke}{rgb}{0.121569,0.466667,0.705882}%
\pgfsetstrokecolor{currentstroke}%
\pgfsetstrokeopacity{0.652345}%
\pgfsetdash{}{0pt}%
\pgfpathmoveto{\pgfqpoint{1.628339in}{2.965144in}}%
\pgfpathcurveto{\pgfqpoint{1.636575in}{2.965144in}}{\pgfqpoint{1.644475in}{2.968417in}}{\pgfqpoint{1.650299in}{2.974240in}}%
\pgfpathcurveto{\pgfqpoint{1.656123in}{2.980064in}}{\pgfqpoint{1.659395in}{2.987964in}}{\pgfqpoint{1.659395in}{2.996201in}}%
\pgfpathcurveto{\pgfqpoint{1.659395in}{3.004437in}}{\pgfqpoint{1.656123in}{3.012337in}}{\pgfqpoint{1.650299in}{3.018161in}}%
\pgfpathcurveto{\pgfqpoint{1.644475in}{3.023985in}}{\pgfqpoint{1.636575in}{3.027257in}}{\pgfqpoint{1.628339in}{3.027257in}}%
\pgfpathcurveto{\pgfqpoint{1.620102in}{3.027257in}}{\pgfqpoint{1.612202in}{3.023985in}}{\pgfqpoint{1.606378in}{3.018161in}}%
\pgfpathcurveto{\pgfqpoint{1.600555in}{3.012337in}}{\pgfqpoint{1.597282in}{3.004437in}}{\pgfqpoint{1.597282in}{2.996201in}}%
\pgfpathcurveto{\pgfqpoint{1.597282in}{2.987964in}}{\pgfqpoint{1.600555in}{2.980064in}}{\pgfqpoint{1.606378in}{2.974240in}}%
\pgfpathcurveto{\pgfqpoint{1.612202in}{2.968417in}}{\pgfqpoint{1.620102in}{2.965144in}}{\pgfqpoint{1.628339in}{2.965144in}}%
\pgfpathclose%
\pgfusepath{stroke,fill}%
\end{pgfscope}%
\begin{pgfscope}%
\pgfpathrectangle{\pgfqpoint{0.100000in}{0.212622in}}{\pgfqpoint{3.696000in}{3.696000in}}%
\pgfusepath{clip}%
\pgfsetbuttcap%
\pgfsetroundjoin%
\definecolor{currentfill}{rgb}{0.121569,0.466667,0.705882}%
\pgfsetfillcolor{currentfill}%
\pgfsetfillopacity{0.652500}%
\pgfsetlinewidth{1.003750pt}%
\definecolor{currentstroke}{rgb}{0.121569,0.466667,0.705882}%
\pgfsetstrokecolor{currentstroke}%
\pgfsetstrokeopacity{0.652500}%
\pgfsetdash{}{0pt}%
\pgfpathmoveto{\pgfqpoint{2.079603in}{3.045474in}}%
\pgfpathcurveto{\pgfqpoint{2.087840in}{3.045474in}}{\pgfqpoint{2.095740in}{3.048746in}}{\pgfqpoint{2.101564in}{3.054570in}}%
\pgfpathcurveto{\pgfqpoint{2.107388in}{3.060394in}}{\pgfqpoint{2.110660in}{3.068294in}}{\pgfqpoint{2.110660in}{3.076531in}}%
\pgfpathcurveto{\pgfqpoint{2.110660in}{3.084767in}}{\pgfqpoint{2.107388in}{3.092667in}}{\pgfqpoint{2.101564in}{3.098491in}}%
\pgfpathcurveto{\pgfqpoint{2.095740in}{3.104315in}}{\pgfqpoint{2.087840in}{3.107587in}}{\pgfqpoint{2.079603in}{3.107587in}}%
\pgfpathcurveto{\pgfqpoint{2.071367in}{3.107587in}}{\pgfqpoint{2.063467in}{3.104315in}}{\pgfqpoint{2.057643in}{3.098491in}}%
\pgfpathcurveto{\pgfqpoint{2.051819in}{3.092667in}}{\pgfqpoint{2.048547in}{3.084767in}}{\pgfqpoint{2.048547in}{3.076531in}}%
\pgfpathcurveto{\pgfqpoint{2.048547in}{3.068294in}}{\pgfqpoint{2.051819in}{3.060394in}}{\pgfqpoint{2.057643in}{3.054570in}}%
\pgfpathcurveto{\pgfqpoint{2.063467in}{3.048746in}}{\pgfqpoint{2.071367in}{3.045474in}}{\pgfqpoint{2.079603in}{3.045474in}}%
\pgfpathclose%
\pgfusepath{stroke,fill}%
\end{pgfscope}%
\begin{pgfscope}%
\pgfpathrectangle{\pgfqpoint{0.100000in}{0.212622in}}{\pgfqpoint{3.696000in}{3.696000in}}%
\pgfusepath{clip}%
\pgfsetbuttcap%
\pgfsetroundjoin%
\definecolor{currentfill}{rgb}{0.121569,0.466667,0.705882}%
\pgfsetfillcolor{currentfill}%
\pgfsetfillopacity{0.653107}%
\pgfsetlinewidth{1.003750pt}%
\definecolor{currentstroke}{rgb}{0.121569,0.466667,0.705882}%
\pgfsetstrokecolor{currentstroke}%
\pgfsetstrokeopacity{0.653107}%
\pgfsetdash{}{0pt}%
\pgfpathmoveto{\pgfqpoint{1.626352in}{2.962500in}}%
\pgfpathcurveto{\pgfqpoint{1.634588in}{2.962500in}}{\pgfqpoint{1.642488in}{2.965772in}}{\pgfqpoint{1.648312in}{2.971596in}}%
\pgfpathcurveto{\pgfqpoint{1.654136in}{2.977420in}}{\pgfqpoint{1.657409in}{2.985320in}}{\pgfqpoint{1.657409in}{2.993557in}}%
\pgfpathcurveto{\pgfqpoint{1.657409in}{3.001793in}}{\pgfqpoint{1.654136in}{3.009693in}}{\pgfqpoint{1.648312in}{3.015517in}}%
\pgfpathcurveto{\pgfqpoint{1.642488in}{3.021341in}}{\pgfqpoint{1.634588in}{3.024613in}}{\pgfqpoint{1.626352in}{3.024613in}}%
\pgfpathcurveto{\pgfqpoint{1.618116in}{3.024613in}}{\pgfqpoint{1.610216in}{3.021341in}}{\pgfqpoint{1.604392in}{3.015517in}}%
\pgfpathcurveto{\pgfqpoint{1.598568in}{3.009693in}}{\pgfqpoint{1.595296in}{3.001793in}}{\pgfqpoint{1.595296in}{2.993557in}}%
\pgfpathcurveto{\pgfqpoint{1.595296in}{2.985320in}}{\pgfqpoint{1.598568in}{2.977420in}}{\pgfqpoint{1.604392in}{2.971596in}}%
\pgfpathcurveto{\pgfqpoint{1.610216in}{2.965772in}}{\pgfqpoint{1.618116in}{2.962500in}}{\pgfqpoint{1.626352in}{2.962500in}}%
\pgfpathclose%
\pgfusepath{stroke,fill}%
\end{pgfscope}%
\begin{pgfscope}%
\pgfpathrectangle{\pgfqpoint{0.100000in}{0.212622in}}{\pgfqpoint{3.696000in}{3.696000in}}%
\pgfusepath{clip}%
\pgfsetbuttcap%
\pgfsetroundjoin%
\definecolor{currentfill}{rgb}{0.121569,0.466667,0.705882}%
\pgfsetfillcolor{currentfill}%
\pgfsetfillopacity{0.653518}%
\pgfsetlinewidth{1.003750pt}%
\definecolor{currentstroke}{rgb}{0.121569,0.466667,0.705882}%
\pgfsetstrokecolor{currentstroke}%
\pgfsetstrokeopacity{0.653518}%
\pgfsetdash{}{0pt}%
\pgfpathmoveto{\pgfqpoint{3.157297in}{1.863104in}}%
\pgfpathcurveto{\pgfqpoint{3.165533in}{1.863104in}}{\pgfqpoint{3.173433in}{1.866377in}}{\pgfqpoint{3.179257in}{1.872201in}}%
\pgfpathcurveto{\pgfqpoint{3.185081in}{1.878025in}}{\pgfqpoint{3.188353in}{1.885925in}}{\pgfqpoint{3.188353in}{1.894161in}}%
\pgfpathcurveto{\pgfqpoint{3.188353in}{1.902397in}}{\pgfqpoint{3.185081in}{1.910297in}}{\pgfqpoint{3.179257in}{1.916121in}}%
\pgfpathcurveto{\pgfqpoint{3.173433in}{1.921945in}}{\pgfqpoint{3.165533in}{1.925217in}}{\pgfqpoint{3.157297in}{1.925217in}}%
\pgfpathcurveto{\pgfqpoint{3.149061in}{1.925217in}}{\pgfqpoint{3.141161in}{1.921945in}}{\pgfqpoint{3.135337in}{1.916121in}}%
\pgfpathcurveto{\pgfqpoint{3.129513in}{1.910297in}}{\pgfqpoint{3.126240in}{1.902397in}}{\pgfqpoint{3.126240in}{1.894161in}}%
\pgfpathcurveto{\pgfqpoint{3.126240in}{1.885925in}}{\pgfqpoint{3.129513in}{1.878025in}}{\pgfqpoint{3.135337in}{1.872201in}}%
\pgfpathcurveto{\pgfqpoint{3.141161in}{1.866377in}}{\pgfqpoint{3.149061in}{1.863104in}}{\pgfqpoint{3.157297in}{1.863104in}}%
\pgfpathclose%
\pgfusepath{stroke,fill}%
\end{pgfscope}%
\begin{pgfscope}%
\pgfpathrectangle{\pgfqpoint{0.100000in}{0.212622in}}{\pgfqpoint{3.696000in}{3.696000in}}%
\pgfusepath{clip}%
\pgfsetbuttcap%
\pgfsetroundjoin%
\definecolor{currentfill}{rgb}{0.121569,0.466667,0.705882}%
\pgfsetfillcolor{currentfill}%
\pgfsetfillopacity{0.654042}%
\pgfsetlinewidth{1.003750pt}%
\definecolor{currentstroke}{rgb}{0.121569,0.466667,0.705882}%
\pgfsetstrokecolor{currentstroke}%
\pgfsetstrokeopacity{0.654042}%
\pgfsetdash{}{0pt}%
\pgfpathmoveto{\pgfqpoint{2.091944in}{3.044164in}}%
\pgfpathcurveto{\pgfqpoint{2.100180in}{3.044164in}}{\pgfqpoint{2.108080in}{3.047436in}}{\pgfqpoint{2.113904in}{3.053260in}}%
\pgfpathcurveto{\pgfqpoint{2.119728in}{3.059084in}}{\pgfqpoint{2.123000in}{3.066984in}}{\pgfqpoint{2.123000in}{3.075220in}}%
\pgfpathcurveto{\pgfqpoint{2.123000in}{3.083456in}}{\pgfqpoint{2.119728in}{3.091357in}}{\pgfqpoint{2.113904in}{3.097180in}}%
\pgfpathcurveto{\pgfqpoint{2.108080in}{3.103004in}}{\pgfqpoint{2.100180in}{3.106277in}}{\pgfqpoint{2.091944in}{3.106277in}}%
\pgfpathcurveto{\pgfqpoint{2.083707in}{3.106277in}}{\pgfqpoint{2.075807in}{3.103004in}}{\pgfqpoint{2.069983in}{3.097180in}}%
\pgfpathcurveto{\pgfqpoint{2.064159in}{3.091357in}}{\pgfqpoint{2.060887in}{3.083456in}}{\pgfqpoint{2.060887in}{3.075220in}}%
\pgfpathcurveto{\pgfqpoint{2.060887in}{3.066984in}}{\pgfqpoint{2.064159in}{3.059084in}}{\pgfqpoint{2.069983in}{3.053260in}}%
\pgfpathcurveto{\pgfqpoint{2.075807in}{3.047436in}}{\pgfqpoint{2.083707in}{3.044164in}}{\pgfqpoint{2.091944in}{3.044164in}}%
\pgfpathclose%
\pgfusepath{stroke,fill}%
\end{pgfscope}%
\begin{pgfscope}%
\pgfpathrectangle{\pgfqpoint{0.100000in}{0.212622in}}{\pgfqpoint{3.696000in}{3.696000in}}%
\pgfusepath{clip}%
\pgfsetbuttcap%
\pgfsetroundjoin%
\definecolor{currentfill}{rgb}{0.121569,0.466667,0.705882}%
\pgfsetfillcolor{currentfill}%
\pgfsetfillopacity{0.654180}%
\pgfsetlinewidth{1.003750pt}%
\definecolor{currentstroke}{rgb}{0.121569,0.466667,0.705882}%
\pgfsetstrokecolor{currentstroke}%
\pgfsetstrokeopacity{0.654180}%
\pgfsetdash{}{0pt}%
\pgfpathmoveto{\pgfqpoint{1.623420in}{2.958245in}}%
\pgfpathcurveto{\pgfqpoint{1.631656in}{2.958245in}}{\pgfqpoint{1.639556in}{2.961517in}}{\pgfqpoint{1.645380in}{2.967341in}}%
\pgfpathcurveto{\pgfqpoint{1.651204in}{2.973165in}}{\pgfqpoint{1.654477in}{2.981065in}}{\pgfqpoint{1.654477in}{2.989301in}}%
\pgfpathcurveto{\pgfqpoint{1.654477in}{2.997538in}}{\pgfqpoint{1.651204in}{3.005438in}}{\pgfqpoint{1.645380in}{3.011262in}}%
\pgfpathcurveto{\pgfqpoint{1.639556in}{3.017085in}}{\pgfqpoint{1.631656in}{3.020358in}}{\pgfqpoint{1.623420in}{3.020358in}}%
\pgfpathcurveto{\pgfqpoint{1.615184in}{3.020358in}}{\pgfqpoint{1.607284in}{3.017085in}}{\pgfqpoint{1.601460in}{3.011262in}}%
\pgfpathcurveto{\pgfqpoint{1.595636in}{3.005438in}}{\pgfqpoint{1.592364in}{2.997538in}}{\pgfqpoint{1.592364in}{2.989301in}}%
\pgfpathcurveto{\pgfqpoint{1.592364in}{2.981065in}}{\pgfqpoint{1.595636in}{2.973165in}}{\pgfqpoint{1.601460in}{2.967341in}}%
\pgfpathcurveto{\pgfqpoint{1.607284in}{2.961517in}}{\pgfqpoint{1.615184in}{2.958245in}}{\pgfqpoint{1.623420in}{2.958245in}}%
\pgfpathclose%
\pgfusepath{stroke,fill}%
\end{pgfscope}%
\begin{pgfscope}%
\pgfpathrectangle{\pgfqpoint{0.100000in}{0.212622in}}{\pgfqpoint{3.696000in}{3.696000in}}%
\pgfusepath{clip}%
\pgfsetbuttcap%
\pgfsetroundjoin%
\definecolor{currentfill}{rgb}{0.121569,0.466667,0.705882}%
\pgfsetfillcolor{currentfill}%
\pgfsetfillopacity{0.654817}%
\pgfsetlinewidth{1.003750pt}%
\definecolor{currentstroke}{rgb}{0.121569,0.466667,0.705882}%
\pgfsetstrokecolor{currentstroke}%
\pgfsetstrokeopacity{0.654817}%
\pgfsetdash{}{0pt}%
\pgfpathmoveto{\pgfqpoint{1.621857in}{2.956142in}}%
\pgfpathcurveto{\pgfqpoint{1.630094in}{2.956142in}}{\pgfqpoint{1.637994in}{2.959414in}}{\pgfqpoint{1.643818in}{2.965238in}}%
\pgfpathcurveto{\pgfqpoint{1.649642in}{2.971062in}}{\pgfqpoint{1.652914in}{2.978962in}}{\pgfqpoint{1.652914in}{2.987198in}}%
\pgfpathcurveto{\pgfqpoint{1.652914in}{2.995435in}}{\pgfqpoint{1.649642in}{3.003335in}}{\pgfqpoint{1.643818in}{3.009158in}}%
\pgfpathcurveto{\pgfqpoint{1.637994in}{3.014982in}}{\pgfqpoint{1.630094in}{3.018255in}}{\pgfqpoint{1.621857in}{3.018255in}}%
\pgfpathcurveto{\pgfqpoint{1.613621in}{3.018255in}}{\pgfqpoint{1.605721in}{3.014982in}}{\pgfqpoint{1.599897in}{3.009158in}}%
\pgfpathcurveto{\pgfqpoint{1.594073in}{3.003335in}}{\pgfqpoint{1.590801in}{2.995435in}}{\pgfqpoint{1.590801in}{2.987198in}}%
\pgfpathcurveto{\pgfqpoint{1.590801in}{2.978962in}}{\pgfqpoint{1.594073in}{2.971062in}}{\pgfqpoint{1.599897in}{2.965238in}}%
\pgfpathcurveto{\pgfqpoint{1.605721in}{2.959414in}}{\pgfqpoint{1.613621in}{2.956142in}}{\pgfqpoint{1.621857in}{2.956142in}}%
\pgfpathclose%
\pgfusepath{stroke,fill}%
\end{pgfscope}%
\begin{pgfscope}%
\pgfpathrectangle{\pgfqpoint{0.100000in}{0.212622in}}{\pgfqpoint{3.696000in}{3.696000in}}%
\pgfusepath{clip}%
\pgfsetbuttcap%
\pgfsetroundjoin%
\definecolor{currentfill}{rgb}{0.121569,0.466667,0.705882}%
\pgfsetfillcolor{currentfill}%
\pgfsetfillopacity{0.655307}%
\pgfsetlinewidth{1.003750pt}%
\definecolor{currentstroke}{rgb}{0.121569,0.466667,0.705882}%
\pgfsetstrokecolor{currentstroke}%
\pgfsetstrokeopacity{0.655307}%
\pgfsetdash{}{0pt}%
\pgfpathmoveto{\pgfqpoint{2.102643in}{3.042377in}}%
\pgfpathcurveto{\pgfqpoint{2.110879in}{3.042377in}}{\pgfqpoint{2.118779in}{3.045650in}}{\pgfqpoint{2.124603in}{3.051473in}}%
\pgfpathcurveto{\pgfqpoint{2.130427in}{3.057297in}}{\pgfqpoint{2.133699in}{3.065197in}}{\pgfqpoint{2.133699in}{3.073434in}}%
\pgfpathcurveto{\pgfqpoint{2.133699in}{3.081670in}}{\pgfqpoint{2.130427in}{3.089570in}}{\pgfqpoint{2.124603in}{3.095394in}}%
\pgfpathcurveto{\pgfqpoint{2.118779in}{3.101218in}}{\pgfqpoint{2.110879in}{3.104490in}}{\pgfqpoint{2.102643in}{3.104490in}}%
\pgfpathcurveto{\pgfqpoint{2.094407in}{3.104490in}}{\pgfqpoint{2.086507in}{3.101218in}}{\pgfqpoint{2.080683in}{3.095394in}}%
\pgfpathcurveto{\pgfqpoint{2.074859in}{3.089570in}}{\pgfqpoint{2.071586in}{3.081670in}}{\pgfqpoint{2.071586in}{3.073434in}}%
\pgfpathcurveto{\pgfqpoint{2.071586in}{3.065197in}}{\pgfqpoint{2.074859in}{3.057297in}}{\pgfqpoint{2.080683in}{3.051473in}}%
\pgfpathcurveto{\pgfqpoint{2.086507in}{3.045650in}}{\pgfqpoint{2.094407in}{3.042377in}}{\pgfqpoint{2.102643in}{3.042377in}}%
\pgfpathclose%
\pgfusepath{stroke,fill}%
\end{pgfscope}%
\begin{pgfscope}%
\pgfpathrectangle{\pgfqpoint{0.100000in}{0.212622in}}{\pgfqpoint{3.696000in}{3.696000in}}%
\pgfusepath{clip}%
\pgfsetbuttcap%
\pgfsetroundjoin%
\definecolor{currentfill}{rgb}{0.121569,0.466667,0.705882}%
\pgfsetfillcolor{currentfill}%
\pgfsetfillopacity{0.655624}%
\pgfsetlinewidth{1.003750pt}%
\definecolor{currentstroke}{rgb}{0.121569,0.466667,0.705882}%
\pgfsetstrokecolor{currentstroke}%
\pgfsetstrokeopacity{0.655624}%
\pgfsetdash{}{0pt}%
\pgfpathmoveto{\pgfqpoint{1.620051in}{2.953943in}}%
\pgfpathcurveto{\pgfqpoint{1.628288in}{2.953943in}}{\pgfqpoint{1.636188in}{2.957216in}}{\pgfqpoint{1.642011in}{2.963040in}}%
\pgfpathcurveto{\pgfqpoint{1.647835in}{2.968864in}}{\pgfqpoint{1.651108in}{2.976764in}}{\pgfqpoint{1.651108in}{2.985000in}}%
\pgfpathcurveto{\pgfqpoint{1.651108in}{2.993236in}}{\pgfqpoint{1.647835in}{3.001136in}}{\pgfqpoint{1.642011in}{3.006960in}}%
\pgfpathcurveto{\pgfqpoint{1.636188in}{3.012784in}}{\pgfqpoint{1.628288in}{3.016056in}}{\pgfqpoint{1.620051in}{3.016056in}}%
\pgfpathcurveto{\pgfqpoint{1.611815in}{3.016056in}}{\pgfqpoint{1.603915in}{3.012784in}}{\pgfqpoint{1.598091in}{3.006960in}}%
\pgfpathcurveto{\pgfqpoint{1.592267in}{3.001136in}}{\pgfqpoint{1.588995in}{2.993236in}}{\pgfqpoint{1.588995in}{2.985000in}}%
\pgfpathcurveto{\pgfqpoint{1.588995in}{2.976764in}}{\pgfqpoint{1.592267in}{2.968864in}}{\pgfqpoint{1.598091in}{2.963040in}}%
\pgfpathcurveto{\pgfqpoint{1.603915in}{2.957216in}}{\pgfqpoint{1.611815in}{2.953943in}}{\pgfqpoint{1.620051in}{2.953943in}}%
\pgfpathclose%
\pgfusepath{stroke,fill}%
\end{pgfscope}%
\begin{pgfscope}%
\pgfpathrectangle{\pgfqpoint{0.100000in}{0.212622in}}{\pgfqpoint{3.696000in}{3.696000in}}%
\pgfusepath{clip}%
\pgfsetbuttcap%
\pgfsetroundjoin%
\definecolor{currentfill}{rgb}{0.121569,0.466667,0.705882}%
\pgfsetfillcolor{currentfill}%
\pgfsetfillopacity{0.656063}%
\pgfsetlinewidth{1.003750pt}%
\definecolor{currentstroke}{rgb}{0.121569,0.466667,0.705882}%
\pgfsetstrokecolor{currentstroke}%
\pgfsetstrokeopacity{0.656063}%
\pgfsetdash{}{0pt}%
\pgfpathmoveto{\pgfqpoint{1.619070in}{2.952691in}}%
\pgfpathcurveto{\pgfqpoint{1.627307in}{2.952691in}}{\pgfqpoint{1.635207in}{2.955964in}}{\pgfqpoint{1.641031in}{2.961788in}}%
\pgfpathcurveto{\pgfqpoint{1.646855in}{2.967612in}}{\pgfqpoint{1.650127in}{2.975512in}}{\pgfqpoint{1.650127in}{2.983748in}}%
\pgfpathcurveto{\pgfqpoint{1.650127in}{2.991984in}}{\pgfqpoint{1.646855in}{2.999884in}}{\pgfqpoint{1.641031in}{3.005708in}}%
\pgfpathcurveto{\pgfqpoint{1.635207in}{3.011532in}}{\pgfqpoint{1.627307in}{3.014804in}}{\pgfqpoint{1.619070in}{3.014804in}}%
\pgfpathcurveto{\pgfqpoint{1.610834in}{3.014804in}}{\pgfqpoint{1.602934in}{3.011532in}}{\pgfqpoint{1.597110in}{3.005708in}}%
\pgfpathcurveto{\pgfqpoint{1.591286in}{2.999884in}}{\pgfqpoint{1.588014in}{2.991984in}}{\pgfqpoint{1.588014in}{2.983748in}}%
\pgfpathcurveto{\pgfqpoint{1.588014in}{2.975512in}}{\pgfqpoint{1.591286in}{2.967612in}}{\pgfqpoint{1.597110in}{2.961788in}}%
\pgfpathcurveto{\pgfqpoint{1.602934in}{2.955964in}}{\pgfqpoint{1.610834in}{2.952691in}}{\pgfqpoint{1.619070in}{2.952691in}}%
\pgfpathclose%
\pgfusepath{stroke,fill}%
\end{pgfscope}%
\begin{pgfscope}%
\pgfpathrectangle{\pgfqpoint{0.100000in}{0.212622in}}{\pgfqpoint{3.696000in}{3.696000in}}%
\pgfusepath{clip}%
\pgfsetbuttcap%
\pgfsetroundjoin%
\definecolor{currentfill}{rgb}{0.121569,0.466667,0.705882}%
\pgfsetfillcolor{currentfill}%
\pgfsetfillopacity{0.656292}%
\pgfsetlinewidth{1.003750pt}%
\definecolor{currentstroke}{rgb}{0.121569,0.466667,0.705882}%
\pgfsetstrokecolor{currentstroke}%
\pgfsetstrokeopacity{0.656292}%
\pgfsetdash{}{0pt}%
\pgfpathmoveto{\pgfqpoint{3.151544in}{1.855196in}}%
\pgfpathcurveto{\pgfqpoint{3.159780in}{1.855196in}}{\pgfqpoint{3.167680in}{1.858469in}}{\pgfqpoint{3.173504in}{1.864293in}}%
\pgfpathcurveto{\pgfqpoint{3.179328in}{1.870117in}}{\pgfqpoint{3.182600in}{1.878017in}}{\pgfqpoint{3.182600in}{1.886253in}}%
\pgfpathcurveto{\pgfqpoint{3.182600in}{1.894489in}}{\pgfqpoint{3.179328in}{1.902389in}}{\pgfqpoint{3.173504in}{1.908213in}}%
\pgfpathcurveto{\pgfqpoint{3.167680in}{1.914037in}}{\pgfqpoint{3.159780in}{1.917309in}}{\pgfqpoint{3.151544in}{1.917309in}}%
\pgfpathcurveto{\pgfqpoint{3.143307in}{1.917309in}}{\pgfqpoint{3.135407in}{1.914037in}}{\pgfqpoint{3.129583in}{1.908213in}}%
\pgfpathcurveto{\pgfqpoint{3.123759in}{1.902389in}}{\pgfqpoint{3.120487in}{1.894489in}}{\pgfqpoint{3.120487in}{1.886253in}}%
\pgfpathcurveto{\pgfqpoint{3.120487in}{1.878017in}}{\pgfqpoint{3.123759in}{1.870117in}}{\pgfqpoint{3.129583in}{1.864293in}}%
\pgfpathcurveto{\pgfqpoint{3.135407in}{1.858469in}}{\pgfqpoint{3.143307in}{1.855196in}}{\pgfqpoint{3.151544in}{1.855196in}}%
\pgfpathclose%
\pgfusepath{stroke,fill}%
\end{pgfscope}%
\begin{pgfscope}%
\pgfpathrectangle{\pgfqpoint{0.100000in}{0.212622in}}{\pgfqpoint{3.696000in}{3.696000in}}%
\pgfusepath{clip}%
\pgfsetbuttcap%
\pgfsetroundjoin%
\definecolor{currentfill}{rgb}{0.121569,0.466667,0.705882}%
\pgfsetfillcolor{currentfill}%
\pgfsetfillopacity{0.656563}%
\pgfsetlinewidth{1.003750pt}%
\definecolor{currentstroke}{rgb}{0.121569,0.466667,0.705882}%
\pgfsetstrokecolor{currentstroke}%
\pgfsetstrokeopacity{0.656563}%
\pgfsetdash{}{0pt}%
\pgfpathmoveto{\pgfqpoint{2.113007in}{3.040941in}}%
\pgfpathcurveto{\pgfqpoint{2.121243in}{3.040941in}}{\pgfqpoint{2.129143in}{3.044214in}}{\pgfqpoint{2.134967in}{3.050038in}}%
\pgfpathcurveto{\pgfqpoint{2.140791in}{3.055861in}}{\pgfqpoint{2.144063in}{3.063762in}}{\pgfqpoint{2.144063in}{3.071998in}}%
\pgfpathcurveto{\pgfqpoint{2.144063in}{3.080234in}}{\pgfqpoint{2.140791in}{3.088134in}}{\pgfqpoint{2.134967in}{3.093958in}}%
\pgfpathcurveto{\pgfqpoint{2.129143in}{3.099782in}}{\pgfqpoint{2.121243in}{3.103054in}}{\pgfqpoint{2.113007in}{3.103054in}}%
\pgfpathcurveto{\pgfqpoint{2.104770in}{3.103054in}}{\pgfqpoint{2.096870in}{3.099782in}}{\pgfqpoint{2.091046in}{3.093958in}}%
\pgfpathcurveto{\pgfqpoint{2.085223in}{3.088134in}}{\pgfqpoint{2.081950in}{3.080234in}}{\pgfqpoint{2.081950in}{3.071998in}}%
\pgfpathcurveto{\pgfqpoint{2.081950in}{3.063762in}}{\pgfqpoint{2.085223in}{3.055861in}}{\pgfqpoint{2.091046in}{3.050038in}}%
\pgfpathcurveto{\pgfqpoint{2.096870in}{3.044214in}}{\pgfqpoint{2.104770in}{3.040941in}}{\pgfqpoint{2.113007in}{3.040941in}}%
\pgfpathclose%
\pgfusepath{stroke,fill}%
\end{pgfscope}%
\begin{pgfscope}%
\pgfpathrectangle{\pgfqpoint{0.100000in}{0.212622in}}{\pgfqpoint{3.696000in}{3.696000in}}%
\pgfusepath{clip}%
\pgfsetbuttcap%
\pgfsetroundjoin%
\definecolor{currentfill}{rgb}{0.121569,0.466667,0.705882}%
\pgfsetfillcolor{currentfill}%
\pgfsetfillopacity{0.656711}%
\pgfsetlinewidth{1.003750pt}%
\definecolor{currentstroke}{rgb}{0.121569,0.466667,0.705882}%
\pgfsetstrokecolor{currentstroke}%
\pgfsetstrokeopacity{0.656711}%
\pgfsetdash{}{0pt}%
\pgfpathmoveto{\pgfqpoint{1.617598in}{2.950638in}}%
\pgfpathcurveto{\pgfqpoint{1.625835in}{2.950638in}}{\pgfqpoint{1.633735in}{2.953911in}}{\pgfqpoint{1.639559in}{2.959735in}}%
\pgfpathcurveto{\pgfqpoint{1.645382in}{2.965559in}}{\pgfqpoint{1.648655in}{2.973459in}}{\pgfqpoint{1.648655in}{2.981695in}}%
\pgfpathcurveto{\pgfqpoint{1.648655in}{2.989931in}}{\pgfqpoint{1.645382in}{2.997831in}}{\pgfqpoint{1.639559in}{3.003655in}}%
\pgfpathcurveto{\pgfqpoint{1.633735in}{3.009479in}}{\pgfqpoint{1.625835in}{3.012751in}}{\pgfqpoint{1.617598in}{3.012751in}}%
\pgfpathcurveto{\pgfqpoint{1.609362in}{3.012751in}}{\pgfqpoint{1.601462in}{3.009479in}}{\pgfqpoint{1.595638in}{3.003655in}}%
\pgfpathcurveto{\pgfqpoint{1.589814in}{2.997831in}}{\pgfqpoint{1.586542in}{2.989931in}}{\pgfqpoint{1.586542in}{2.981695in}}%
\pgfpathcurveto{\pgfqpoint{1.586542in}{2.973459in}}{\pgfqpoint{1.589814in}{2.965559in}}{\pgfqpoint{1.595638in}{2.959735in}}%
\pgfpathcurveto{\pgfqpoint{1.601462in}{2.953911in}}{\pgfqpoint{1.609362in}{2.950638in}}{\pgfqpoint{1.617598in}{2.950638in}}%
\pgfpathclose%
\pgfusepath{stroke,fill}%
\end{pgfscope}%
\begin{pgfscope}%
\pgfpathrectangle{\pgfqpoint{0.100000in}{0.212622in}}{\pgfqpoint{3.696000in}{3.696000in}}%
\pgfusepath{clip}%
\pgfsetbuttcap%
\pgfsetroundjoin%
\definecolor{currentfill}{rgb}{0.121569,0.466667,0.705882}%
\pgfsetfillcolor{currentfill}%
\pgfsetfillopacity{0.657492}%
\pgfsetlinewidth{1.003750pt}%
\definecolor{currentstroke}{rgb}{0.121569,0.466667,0.705882}%
\pgfsetstrokecolor{currentstroke}%
\pgfsetstrokeopacity{0.657492}%
\pgfsetdash{}{0pt}%
\pgfpathmoveto{\pgfqpoint{1.616000in}{2.948252in}}%
\pgfpathcurveto{\pgfqpoint{1.624236in}{2.948252in}}{\pgfqpoint{1.632136in}{2.951524in}}{\pgfqpoint{1.637960in}{2.957348in}}%
\pgfpathcurveto{\pgfqpoint{1.643784in}{2.963172in}}{\pgfqpoint{1.647057in}{2.971072in}}{\pgfqpoint{1.647057in}{2.979308in}}%
\pgfpathcurveto{\pgfqpoint{1.647057in}{2.987545in}}{\pgfqpoint{1.643784in}{2.995445in}}{\pgfqpoint{1.637960in}{3.001269in}}%
\pgfpathcurveto{\pgfqpoint{1.632136in}{3.007092in}}{\pgfqpoint{1.624236in}{3.010365in}}{\pgfqpoint{1.616000in}{3.010365in}}%
\pgfpathcurveto{\pgfqpoint{1.607764in}{3.010365in}}{\pgfqpoint{1.599864in}{3.007092in}}{\pgfqpoint{1.594040in}{3.001269in}}%
\pgfpathcurveto{\pgfqpoint{1.588216in}{2.995445in}}{\pgfqpoint{1.584944in}{2.987545in}}{\pgfqpoint{1.584944in}{2.979308in}}%
\pgfpathcurveto{\pgfqpoint{1.584944in}{2.971072in}}{\pgfqpoint{1.588216in}{2.963172in}}{\pgfqpoint{1.594040in}{2.957348in}}%
\pgfpathcurveto{\pgfqpoint{1.599864in}{2.951524in}}{\pgfqpoint{1.607764in}{2.948252in}}{\pgfqpoint{1.616000in}{2.948252in}}%
\pgfpathclose%
\pgfusepath{stroke,fill}%
\end{pgfscope}%
\begin{pgfscope}%
\pgfpathrectangle{\pgfqpoint{0.100000in}{0.212622in}}{\pgfqpoint{3.696000in}{3.696000in}}%
\pgfusepath{clip}%
\pgfsetbuttcap%
\pgfsetroundjoin%
\definecolor{currentfill}{rgb}{0.121569,0.466667,0.705882}%
\pgfsetfillcolor{currentfill}%
\pgfsetfillopacity{0.657749}%
\pgfsetlinewidth{1.003750pt}%
\definecolor{currentstroke}{rgb}{0.121569,0.466667,0.705882}%
\pgfsetstrokecolor{currentstroke}%
\pgfsetstrokeopacity{0.657749}%
\pgfsetdash{}{0pt}%
\pgfpathmoveto{\pgfqpoint{2.122466in}{3.040062in}}%
\pgfpathcurveto{\pgfqpoint{2.130703in}{3.040062in}}{\pgfqpoint{2.138603in}{3.043334in}}{\pgfqpoint{2.144427in}{3.049158in}}%
\pgfpathcurveto{\pgfqpoint{2.150250in}{3.054982in}}{\pgfqpoint{2.153523in}{3.062882in}}{\pgfqpoint{2.153523in}{3.071118in}}%
\pgfpathcurveto{\pgfqpoint{2.153523in}{3.079355in}}{\pgfqpoint{2.150250in}{3.087255in}}{\pgfqpoint{2.144427in}{3.093079in}}%
\pgfpathcurveto{\pgfqpoint{2.138603in}{3.098903in}}{\pgfqpoint{2.130703in}{3.102175in}}{\pgfqpoint{2.122466in}{3.102175in}}%
\pgfpathcurveto{\pgfqpoint{2.114230in}{3.102175in}}{\pgfqpoint{2.106330in}{3.098903in}}{\pgfqpoint{2.100506in}{3.093079in}}%
\pgfpathcurveto{\pgfqpoint{2.094682in}{3.087255in}}{\pgfqpoint{2.091410in}{3.079355in}}{\pgfqpoint{2.091410in}{3.071118in}}%
\pgfpathcurveto{\pgfqpoint{2.091410in}{3.062882in}}{\pgfqpoint{2.094682in}{3.054982in}}{\pgfqpoint{2.100506in}{3.049158in}}%
\pgfpathcurveto{\pgfqpoint{2.106330in}{3.043334in}}{\pgfqpoint{2.114230in}{3.040062in}}{\pgfqpoint{2.122466in}{3.040062in}}%
\pgfpathclose%
\pgfusepath{stroke,fill}%
\end{pgfscope}%
\begin{pgfscope}%
\pgfpathrectangle{\pgfqpoint{0.100000in}{0.212622in}}{\pgfqpoint{3.696000in}{3.696000in}}%
\pgfusepath{clip}%
\pgfsetbuttcap%
\pgfsetroundjoin%
\definecolor{currentfill}{rgb}{0.121569,0.466667,0.705882}%
\pgfsetfillcolor{currentfill}%
\pgfsetfillopacity{0.658665}%
\pgfsetlinewidth{1.003750pt}%
\definecolor{currentstroke}{rgb}{0.121569,0.466667,0.705882}%
\pgfsetstrokecolor{currentstroke}%
\pgfsetstrokeopacity{0.658665}%
\pgfsetdash{}{0pt}%
\pgfpathmoveto{\pgfqpoint{1.613892in}{2.945340in}}%
\pgfpathcurveto{\pgfqpoint{1.622129in}{2.945340in}}{\pgfqpoint{1.630029in}{2.948612in}}{\pgfqpoint{1.635853in}{2.954436in}}%
\pgfpathcurveto{\pgfqpoint{1.641677in}{2.960260in}}{\pgfqpoint{1.644949in}{2.968160in}}{\pgfqpoint{1.644949in}{2.976396in}}%
\pgfpathcurveto{\pgfqpoint{1.644949in}{2.984632in}}{\pgfqpoint{1.641677in}{2.992532in}}{\pgfqpoint{1.635853in}{2.998356in}}%
\pgfpathcurveto{\pgfqpoint{1.630029in}{3.004180in}}{\pgfqpoint{1.622129in}{3.007453in}}{\pgfqpoint{1.613892in}{3.007453in}}%
\pgfpathcurveto{\pgfqpoint{1.605656in}{3.007453in}}{\pgfqpoint{1.597756in}{3.004180in}}{\pgfqpoint{1.591932in}{2.998356in}}%
\pgfpathcurveto{\pgfqpoint{1.586108in}{2.992532in}}{\pgfqpoint{1.582836in}{2.984632in}}{\pgfqpoint{1.582836in}{2.976396in}}%
\pgfpathcurveto{\pgfqpoint{1.582836in}{2.968160in}}{\pgfqpoint{1.586108in}{2.960260in}}{\pgfqpoint{1.591932in}{2.954436in}}%
\pgfpathcurveto{\pgfqpoint{1.597756in}{2.948612in}}{\pgfqpoint{1.605656in}{2.945340in}}{\pgfqpoint{1.613892in}{2.945340in}}%
\pgfpathclose%
\pgfusepath{stroke,fill}%
\end{pgfscope}%
\begin{pgfscope}%
\pgfpathrectangle{\pgfqpoint{0.100000in}{0.212622in}}{\pgfqpoint{3.696000in}{3.696000in}}%
\pgfusepath{clip}%
\pgfsetbuttcap%
\pgfsetroundjoin%
\definecolor{currentfill}{rgb}{0.121569,0.466667,0.705882}%
\pgfsetfillcolor{currentfill}%
\pgfsetfillopacity{0.658806}%
\pgfsetlinewidth{1.003750pt}%
\definecolor{currentstroke}{rgb}{0.121569,0.466667,0.705882}%
\pgfsetstrokecolor{currentstroke}%
\pgfsetstrokeopacity{0.658806}%
\pgfsetdash{}{0pt}%
\pgfpathmoveto{\pgfqpoint{3.146638in}{1.848850in}}%
\pgfpathcurveto{\pgfqpoint{3.154874in}{1.848850in}}{\pgfqpoint{3.162774in}{1.852123in}}{\pgfqpoint{3.168598in}{1.857947in}}%
\pgfpathcurveto{\pgfqpoint{3.174422in}{1.863770in}}{\pgfqpoint{3.177694in}{1.871671in}}{\pgfqpoint{3.177694in}{1.879907in}}%
\pgfpathcurveto{\pgfqpoint{3.177694in}{1.888143in}}{\pgfqpoint{3.174422in}{1.896043in}}{\pgfqpoint{3.168598in}{1.901867in}}%
\pgfpathcurveto{\pgfqpoint{3.162774in}{1.907691in}}{\pgfqpoint{3.154874in}{1.910963in}}{\pgfqpoint{3.146638in}{1.910963in}}%
\pgfpathcurveto{\pgfqpoint{3.138401in}{1.910963in}}{\pgfqpoint{3.130501in}{1.907691in}}{\pgfqpoint{3.124677in}{1.901867in}}%
\pgfpathcurveto{\pgfqpoint{3.118853in}{1.896043in}}{\pgfqpoint{3.115581in}{1.888143in}}{\pgfqpoint{3.115581in}{1.879907in}}%
\pgfpathcurveto{\pgfqpoint{3.115581in}{1.871671in}}{\pgfqpoint{3.118853in}{1.863770in}}{\pgfqpoint{3.124677in}{1.857947in}}%
\pgfpathcurveto{\pgfqpoint{3.130501in}{1.852123in}}{\pgfqpoint{3.138401in}{1.848850in}}{\pgfqpoint{3.146638in}{1.848850in}}%
\pgfpathclose%
\pgfusepath{stroke,fill}%
\end{pgfscope}%
\begin{pgfscope}%
\pgfpathrectangle{\pgfqpoint{0.100000in}{0.212622in}}{\pgfqpoint{3.696000in}{3.696000in}}%
\pgfusepath{clip}%
\pgfsetbuttcap%
\pgfsetroundjoin%
\definecolor{currentfill}{rgb}{0.121569,0.466667,0.705882}%
\pgfsetfillcolor{currentfill}%
\pgfsetfillopacity{0.658901}%
\pgfsetlinewidth{1.003750pt}%
\definecolor{currentstroke}{rgb}{0.121569,0.466667,0.705882}%
\pgfsetstrokecolor{currentstroke}%
\pgfsetstrokeopacity{0.658901}%
\pgfsetdash{}{0pt}%
\pgfpathmoveto{\pgfqpoint{2.131086in}{3.039479in}}%
\pgfpathcurveto{\pgfqpoint{2.139323in}{3.039479in}}{\pgfqpoint{2.147223in}{3.042751in}}{\pgfqpoint{2.153047in}{3.048575in}}%
\pgfpathcurveto{\pgfqpoint{2.158870in}{3.054399in}}{\pgfqpoint{2.162143in}{3.062299in}}{\pgfqpoint{2.162143in}{3.070535in}}%
\pgfpathcurveto{\pgfqpoint{2.162143in}{3.078772in}}{\pgfqpoint{2.158870in}{3.086672in}}{\pgfqpoint{2.153047in}{3.092496in}}%
\pgfpathcurveto{\pgfqpoint{2.147223in}{3.098319in}}{\pgfqpoint{2.139323in}{3.101592in}}{\pgfqpoint{2.131086in}{3.101592in}}%
\pgfpathcurveto{\pgfqpoint{2.122850in}{3.101592in}}{\pgfqpoint{2.114950in}{3.098319in}}{\pgfqpoint{2.109126in}{3.092496in}}%
\pgfpathcurveto{\pgfqpoint{2.103302in}{3.086672in}}{\pgfqpoint{2.100030in}{3.078772in}}{\pgfqpoint{2.100030in}{3.070535in}}%
\pgfpathcurveto{\pgfqpoint{2.100030in}{3.062299in}}{\pgfqpoint{2.103302in}{3.054399in}}{\pgfqpoint{2.109126in}{3.048575in}}%
\pgfpathcurveto{\pgfqpoint{2.114950in}{3.042751in}}{\pgfqpoint{2.122850in}{3.039479in}}{\pgfqpoint{2.131086in}{3.039479in}}%
\pgfpathclose%
\pgfusepath{stroke,fill}%
\end{pgfscope}%
\begin{pgfscope}%
\pgfpathrectangle{\pgfqpoint{0.100000in}{0.212622in}}{\pgfqpoint{3.696000in}{3.696000in}}%
\pgfusepath{clip}%
\pgfsetbuttcap%
\pgfsetroundjoin%
\definecolor{currentfill}{rgb}{0.121569,0.466667,0.705882}%
\pgfsetfillcolor{currentfill}%
\pgfsetfillopacity{0.659772}%
\pgfsetlinewidth{1.003750pt}%
\definecolor{currentstroke}{rgb}{0.121569,0.466667,0.705882}%
\pgfsetstrokecolor{currentstroke}%
\pgfsetstrokeopacity{0.659772}%
\pgfsetdash{}{0pt}%
\pgfpathmoveto{\pgfqpoint{2.138713in}{3.037842in}}%
\pgfpathcurveto{\pgfqpoint{2.146949in}{3.037842in}}{\pgfqpoint{2.154849in}{3.041114in}}{\pgfqpoint{2.160673in}{3.046938in}}%
\pgfpathcurveto{\pgfqpoint{2.166497in}{3.052762in}}{\pgfqpoint{2.169770in}{3.060662in}}{\pgfqpoint{2.169770in}{3.068898in}}%
\pgfpathcurveto{\pgfqpoint{2.169770in}{3.077134in}}{\pgfqpoint{2.166497in}{3.085035in}}{\pgfqpoint{2.160673in}{3.090858in}}%
\pgfpathcurveto{\pgfqpoint{2.154849in}{3.096682in}}{\pgfqpoint{2.146949in}{3.099955in}}{\pgfqpoint{2.138713in}{3.099955in}}%
\pgfpathcurveto{\pgfqpoint{2.130477in}{3.099955in}}{\pgfqpoint{2.122577in}{3.096682in}}{\pgfqpoint{2.116753in}{3.090858in}}%
\pgfpathcurveto{\pgfqpoint{2.110929in}{3.085035in}}{\pgfqpoint{2.107657in}{3.077134in}}{\pgfqpoint{2.107657in}{3.068898in}}%
\pgfpathcurveto{\pgfqpoint{2.107657in}{3.060662in}}{\pgfqpoint{2.110929in}{3.052762in}}{\pgfqpoint{2.116753in}{3.046938in}}%
\pgfpathcurveto{\pgfqpoint{2.122577in}{3.041114in}}{\pgfqpoint{2.130477in}{3.037842in}}{\pgfqpoint{2.138713in}{3.037842in}}%
\pgfpathclose%
\pgfusepath{stroke,fill}%
\end{pgfscope}%
\begin{pgfscope}%
\pgfpathrectangle{\pgfqpoint{0.100000in}{0.212622in}}{\pgfqpoint{3.696000in}{3.696000in}}%
\pgfusepath{clip}%
\pgfsetbuttcap%
\pgfsetroundjoin%
\definecolor{currentfill}{rgb}{0.121569,0.466667,0.705882}%
\pgfsetfillcolor{currentfill}%
\pgfsetfillopacity{0.659880}%
\pgfsetlinewidth{1.003750pt}%
\definecolor{currentstroke}{rgb}{0.121569,0.466667,0.705882}%
\pgfsetstrokecolor{currentstroke}%
\pgfsetstrokeopacity{0.659880}%
\pgfsetdash{}{0pt}%
\pgfpathmoveto{\pgfqpoint{1.611578in}{2.941957in}}%
\pgfpathcurveto{\pgfqpoint{1.619814in}{2.941957in}}{\pgfqpoint{1.627714in}{2.945230in}}{\pgfqpoint{1.633538in}{2.951054in}}%
\pgfpathcurveto{\pgfqpoint{1.639362in}{2.956877in}}{\pgfqpoint{1.642634in}{2.964778in}}{\pgfqpoint{1.642634in}{2.973014in}}%
\pgfpathcurveto{\pgfqpoint{1.642634in}{2.981250in}}{\pgfqpoint{1.639362in}{2.989150in}}{\pgfqpoint{1.633538in}{2.994974in}}%
\pgfpathcurveto{\pgfqpoint{1.627714in}{3.000798in}}{\pgfqpoint{1.619814in}{3.004070in}}{\pgfqpoint{1.611578in}{3.004070in}}%
\pgfpathcurveto{\pgfqpoint{1.603342in}{3.004070in}}{\pgfqpoint{1.595441in}{3.000798in}}{\pgfqpoint{1.589618in}{2.994974in}}%
\pgfpathcurveto{\pgfqpoint{1.583794in}{2.989150in}}{\pgfqpoint{1.580521in}{2.981250in}}{\pgfqpoint{1.580521in}{2.973014in}}%
\pgfpathcurveto{\pgfqpoint{1.580521in}{2.964778in}}{\pgfqpoint{1.583794in}{2.956877in}}{\pgfqpoint{1.589618in}{2.951054in}}%
\pgfpathcurveto{\pgfqpoint{1.595441in}{2.945230in}}{\pgfqpoint{1.603342in}{2.941957in}}{\pgfqpoint{1.611578in}{2.941957in}}%
\pgfpathclose%
\pgfusepath{stroke,fill}%
\end{pgfscope}%
\begin{pgfscope}%
\pgfpathrectangle{\pgfqpoint{0.100000in}{0.212622in}}{\pgfqpoint{3.696000in}{3.696000in}}%
\pgfusepath{clip}%
\pgfsetbuttcap%
\pgfsetroundjoin%
\definecolor{currentfill}{rgb}{0.121569,0.466667,0.705882}%
\pgfsetfillcolor{currentfill}%
\pgfsetfillopacity{0.660538}%
\pgfsetlinewidth{1.003750pt}%
\definecolor{currentstroke}{rgb}{0.121569,0.466667,0.705882}%
\pgfsetstrokecolor{currentstroke}%
\pgfsetstrokeopacity{0.660538}%
\pgfsetdash{}{0pt}%
\pgfpathmoveto{\pgfqpoint{2.145631in}{3.035971in}}%
\pgfpathcurveto{\pgfqpoint{2.153867in}{3.035971in}}{\pgfqpoint{2.161767in}{3.039244in}}{\pgfqpoint{2.167591in}{3.045068in}}%
\pgfpathcurveto{\pgfqpoint{2.173415in}{3.050891in}}{\pgfqpoint{2.176687in}{3.058792in}}{\pgfqpoint{2.176687in}{3.067028in}}%
\pgfpathcurveto{\pgfqpoint{2.176687in}{3.075264in}}{\pgfqpoint{2.173415in}{3.083164in}}{\pgfqpoint{2.167591in}{3.088988in}}%
\pgfpathcurveto{\pgfqpoint{2.161767in}{3.094812in}}{\pgfqpoint{2.153867in}{3.098084in}}{\pgfqpoint{2.145631in}{3.098084in}}%
\pgfpathcurveto{\pgfqpoint{2.137395in}{3.098084in}}{\pgfqpoint{2.129495in}{3.094812in}}{\pgfqpoint{2.123671in}{3.088988in}}%
\pgfpathcurveto{\pgfqpoint{2.117847in}{3.083164in}}{\pgfqpoint{2.114574in}{3.075264in}}{\pgfqpoint{2.114574in}{3.067028in}}%
\pgfpathcurveto{\pgfqpoint{2.114574in}{3.058792in}}{\pgfqpoint{2.117847in}{3.050891in}}{\pgfqpoint{2.123671in}{3.045068in}}%
\pgfpathcurveto{\pgfqpoint{2.129495in}{3.039244in}}{\pgfqpoint{2.137395in}{3.035971in}}{\pgfqpoint{2.145631in}{3.035971in}}%
\pgfpathclose%
\pgfusepath{stroke,fill}%
\end{pgfscope}%
\begin{pgfscope}%
\pgfpathrectangle{\pgfqpoint{0.100000in}{0.212622in}}{\pgfqpoint{3.696000in}{3.696000in}}%
\pgfusepath{clip}%
\pgfsetbuttcap%
\pgfsetroundjoin%
\definecolor{currentfill}{rgb}{0.121569,0.466667,0.705882}%
\pgfsetfillcolor{currentfill}%
\pgfsetfillopacity{0.661021}%
\pgfsetlinewidth{1.003750pt}%
\definecolor{currentstroke}{rgb}{0.121569,0.466667,0.705882}%
\pgfsetstrokecolor{currentstroke}%
\pgfsetstrokeopacity{0.661021}%
\pgfsetdash{}{0pt}%
\pgfpathmoveto{\pgfqpoint{3.142619in}{1.843635in}}%
\pgfpathcurveto{\pgfqpoint{3.150856in}{1.843635in}}{\pgfqpoint{3.158756in}{1.846908in}}{\pgfqpoint{3.164580in}{1.852732in}}%
\pgfpathcurveto{\pgfqpoint{3.170404in}{1.858556in}}{\pgfqpoint{3.173676in}{1.866456in}}{\pgfqpoint{3.173676in}{1.874692in}}%
\pgfpathcurveto{\pgfqpoint{3.173676in}{1.882928in}}{\pgfqpoint{3.170404in}{1.890828in}}{\pgfqpoint{3.164580in}{1.896652in}}%
\pgfpathcurveto{\pgfqpoint{3.158756in}{1.902476in}}{\pgfqpoint{3.150856in}{1.905748in}}{\pgfqpoint{3.142619in}{1.905748in}}%
\pgfpathcurveto{\pgfqpoint{3.134383in}{1.905748in}}{\pgfqpoint{3.126483in}{1.902476in}}{\pgfqpoint{3.120659in}{1.896652in}}%
\pgfpathcurveto{\pgfqpoint{3.114835in}{1.890828in}}{\pgfqpoint{3.111563in}{1.882928in}}{\pgfqpoint{3.111563in}{1.874692in}}%
\pgfpathcurveto{\pgfqpoint{3.111563in}{1.866456in}}{\pgfqpoint{3.114835in}{1.858556in}}{\pgfqpoint{3.120659in}{1.852732in}}%
\pgfpathcurveto{\pgfqpoint{3.126483in}{1.846908in}}{\pgfqpoint{3.134383in}{1.843635in}}{\pgfqpoint{3.142619in}{1.843635in}}%
\pgfpathclose%
\pgfusepath{stroke,fill}%
\end{pgfscope}%
\begin{pgfscope}%
\pgfpathrectangle{\pgfqpoint{0.100000in}{0.212622in}}{\pgfqpoint{3.696000in}{3.696000in}}%
\pgfusepath{clip}%
\pgfsetbuttcap%
\pgfsetroundjoin%
\definecolor{currentfill}{rgb}{0.121569,0.466667,0.705882}%
\pgfsetfillcolor{currentfill}%
\pgfsetfillopacity{0.661202}%
\pgfsetlinewidth{1.003750pt}%
\definecolor{currentstroke}{rgb}{0.121569,0.466667,0.705882}%
\pgfsetstrokecolor{currentstroke}%
\pgfsetstrokeopacity{0.661202}%
\pgfsetdash{}{0pt}%
\pgfpathmoveto{\pgfqpoint{1.608763in}{2.937510in}}%
\pgfpathcurveto{\pgfqpoint{1.616999in}{2.937510in}}{\pgfqpoint{1.624899in}{2.940782in}}{\pgfqpoint{1.630723in}{2.946606in}}%
\pgfpathcurveto{\pgfqpoint{1.636547in}{2.952430in}}{\pgfqpoint{1.639820in}{2.960330in}}{\pgfqpoint{1.639820in}{2.968566in}}%
\pgfpathcurveto{\pgfqpoint{1.639820in}{2.976802in}}{\pgfqpoint{1.636547in}{2.984702in}}{\pgfqpoint{1.630723in}{2.990526in}}%
\pgfpathcurveto{\pgfqpoint{1.624899in}{2.996350in}}{\pgfqpoint{1.616999in}{2.999623in}}{\pgfqpoint{1.608763in}{2.999623in}}%
\pgfpathcurveto{\pgfqpoint{1.600527in}{2.999623in}}{\pgfqpoint{1.592627in}{2.996350in}}{\pgfqpoint{1.586803in}{2.990526in}}%
\pgfpathcurveto{\pgfqpoint{1.580979in}{2.984702in}}{\pgfqpoint{1.577707in}{2.976802in}}{\pgfqpoint{1.577707in}{2.968566in}}%
\pgfpathcurveto{\pgfqpoint{1.577707in}{2.960330in}}{\pgfqpoint{1.580979in}{2.952430in}}{\pgfqpoint{1.586803in}{2.946606in}}%
\pgfpathcurveto{\pgfqpoint{1.592627in}{2.940782in}}{\pgfqpoint{1.600527in}{2.937510in}}{\pgfqpoint{1.608763in}{2.937510in}}%
\pgfpathclose%
\pgfusepath{stroke,fill}%
\end{pgfscope}%
\begin{pgfscope}%
\pgfpathrectangle{\pgfqpoint{0.100000in}{0.212622in}}{\pgfqpoint{3.696000in}{3.696000in}}%
\pgfusepath{clip}%
\pgfsetbuttcap%
\pgfsetroundjoin%
\definecolor{currentfill}{rgb}{0.121569,0.466667,0.705882}%
\pgfsetfillcolor{currentfill}%
\pgfsetfillopacity{0.661912}%
\pgfsetlinewidth{1.003750pt}%
\definecolor{currentstroke}{rgb}{0.121569,0.466667,0.705882}%
\pgfsetstrokecolor{currentstroke}%
\pgfsetstrokeopacity{0.661912}%
\pgfsetdash{}{0pt}%
\pgfpathmoveto{\pgfqpoint{1.607187in}{2.934971in}}%
\pgfpathcurveto{\pgfqpoint{1.615424in}{2.934971in}}{\pgfqpoint{1.623324in}{2.938243in}}{\pgfqpoint{1.629148in}{2.944067in}}%
\pgfpathcurveto{\pgfqpoint{1.634972in}{2.949891in}}{\pgfqpoint{1.638244in}{2.957791in}}{\pgfqpoint{1.638244in}{2.966028in}}%
\pgfpathcurveto{\pgfqpoint{1.638244in}{2.974264in}}{\pgfqpoint{1.634972in}{2.982164in}}{\pgfqpoint{1.629148in}{2.987988in}}%
\pgfpathcurveto{\pgfqpoint{1.623324in}{2.993812in}}{\pgfqpoint{1.615424in}{2.997084in}}{\pgfqpoint{1.607187in}{2.997084in}}%
\pgfpathcurveto{\pgfqpoint{1.598951in}{2.997084in}}{\pgfqpoint{1.591051in}{2.993812in}}{\pgfqpoint{1.585227in}{2.987988in}}%
\pgfpathcurveto{\pgfqpoint{1.579403in}{2.982164in}}{\pgfqpoint{1.576131in}{2.974264in}}{\pgfqpoint{1.576131in}{2.966028in}}%
\pgfpathcurveto{\pgfqpoint{1.576131in}{2.957791in}}{\pgfqpoint{1.579403in}{2.949891in}}{\pgfqpoint{1.585227in}{2.944067in}}%
\pgfpathcurveto{\pgfqpoint{1.591051in}{2.938243in}}{\pgfqpoint{1.598951in}{2.934971in}}{\pgfqpoint{1.607187in}{2.934971in}}%
\pgfpathclose%
\pgfusepath{stroke,fill}%
\end{pgfscope}%
\begin{pgfscope}%
\pgfpathrectangle{\pgfqpoint{0.100000in}{0.212622in}}{\pgfqpoint{3.696000in}{3.696000in}}%
\pgfusepath{clip}%
\pgfsetbuttcap%
\pgfsetroundjoin%
\definecolor{currentfill}{rgb}{0.121569,0.466667,0.705882}%
\pgfsetfillcolor{currentfill}%
\pgfsetfillopacity{0.662026}%
\pgfsetlinewidth{1.003750pt}%
\definecolor{currentstroke}{rgb}{0.121569,0.466667,0.705882}%
\pgfsetstrokecolor{currentstroke}%
\pgfsetstrokeopacity{0.662026}%
\pgfsetdash{}{0pt}%
\pgfpathmoveto{\pgfqpoint{2.158355in}{3.033681in}}%
\pgfpathcurveto{\pgfqpoint{2.166591in}{3.033681in}}{\pgfqpoint{2.174491in}{3.036953in}}{\pgfqpoint{2.180315in}{3.042777in}}%
\pgfpathcurveto{\pgfqpoint{2.186139in}{3.048601in}}{\pgfqpoint{2.189411in}{3.056501in}}{\pgfqpoint{2.189411in}{3.064737in}}%
\pgfpathcurveto{\pgfqpoint{2.189411in}{3.072974in}}{\pgfqpoint{2.186139in}{3.080874in}}{\pgfqpoint{2.180315in}{3.086698in}}%
\pgfpathcurveto{\pgfqpoint{2.174491in}{3.092521in}}{\pgfqpoint{2.166591in}{3.095794in}}{\pgfqpoint{2.158355in}{3.095794in}}%
\pgfpathcurveto{\pgfqpoint{2.150119in}{3.095794in}}{\pgfqpoint{2.142219in}{3.092521in}}{\pgfqpoint{2.136395in}{3.086698in}}%
\pgfpathcurveto{\pgfqpoint{2.130571in}{3.080874in}}{\pgfqpoint{2.127298in}{3.072974in}}{\pgfqpoint{2.127298in}{3.064737in}}%
\pgfpathcurveto{\pgfqpoint{2.127298in}{3.056501in}}{\pgfqpoint{2.130571in}{3.048601in}}{\pgfqpoint{2.136395in}{3.042777in}}%
\pgfpathcurveto{\pgfqpoint{2.142219in}{3.036953in}}{\pgfqpoint{2.150119in}{3.033681in}}{\pgfqpoint{2.158355in}{3.033681in}}%
\pgfpathclose%
\pgfusepath{stroke,fill}%
\end{pgfscope}%
\begin{pgfscope}%
\pgfpathrectangle{\pgfqpoint{0.100000in}{0.212622in}}{\pgfqpoint{3.696000in}{3.696000in}}%
\pgfusepath{clip}%
\pgfsetbuttcap%
\pgfsetroundjoin%
\definecolor{currentfill}{rgb}{0.121569,0.466667,0.705882}%
\pgfsetfillcolor{currentfill}%
\pgfsetfillopacity{0.662745}%
\pgfsetlinewidth{1.003750pt}%
\definecolor{currentstroke}{rgb}{0.121569,0.466667,0.705882}%
\pgfsetstrokecolor{currentstroke}%
\pgfsetstrokeopacity{0.662745}%
\pgfsetdash{}{0pt}%
\pgfpathmoveto{\pgfqpoint{1.605275in}{2.932145in}}%
\pgfpathcurveto{\pgfqpoint{1.613511in}{2.932145in}}{\pgfqpoint{1.621411in}{2.935417in}}{\pgfqpoint{1.627235in}{2.941241in}}%
\pgfpathcurveto{\pgfqpoint{1.633059in}{2.947065in}}{\pgfqpoint{1.636331in}{2.954965in}}{\pgfqpoint{1.636331in}{2.963201in}}%
\pgfpathcurveto{\pgfqpoint{1.636331in}{2.971437in}}{\pgfqpoint{1.633059in}{2.979338in}}{\pgfqpoint{1.627235in}{2.985161in}}%
\pgfpathcurveto{\pgfqpoint{1.621411in}{2.990985in}}{\pgfqpoint{1.613511in}{2.994258in}}{\pgfqpoint{1.605275in}{2.994258in}}%
\pgfpathcurveto{\pgfqpoint{1.597038in}{2.994258in}}{\pgfqpoint{1.589138in}{2.990985in}}{\pgfqpoint{1.583315in}{2.985161in}}%
\pgfpathcurveto{\pgfqpoint{1.577491in}{2.979338in}}{\pgfqpoint{1.574218in}{2.971437in}}{\pgfqpoint{1.574218in}{2.963201in}}%
\pgfpathcurveto{\pgfqpoint{1.574218in}{2.954965in}}{\pgfqpoint{1.577491in}{2.947065in}}{\pgfqpoint{1.583315in}{2.941241in}}%
\pgfpathcurveto{\pgfqpoint{1.589138in}{2.935417in}}{\pgfqpoint{1.597038in}{2.932145in}}{\pgfqpoint{1.605275in}{2.932145in}}%
\pgfpathclose%
\pgfusepath{stroke,fill}%
\end{pgfscope}%
\begin{pgfscope}%
\pgfpathrectangle{\pgfqpoint{0.100000in}{0.212622in}}{\pgfqpoint{3.696000in}{3.696000in}}%
\pgfusepath{clip}%
\pgfsetbuttcap%
\pgfsetroundjoin%
\definecolor{currentfill}{rgb}{0.121569,0.466667,0.705882}%
\pgfsetfillcolor{currentfill}%
\pgfsetfillopacity{0.663022}%
\pgfsetlinewidth{1.003750pt}%
\definecolor{currentstroke}{rgb}{0.121569,0.466667,0.705882}%
\pgfsetstrokecolor{currentstroke}%
\pgfsetstrokeopacity{0.663022}%
\pgfsetdash{}{0pt}%
\pgfpathmoveto{\pgfqpoint{3.138839in}{1.838058in}}%
\pgfpathcurveto{\pgfqpoint{3.147075in}{1.838058in}}{\pgfqpoint{3.154975in}{1.841331in}}{\pgfqpoint{3.160799in}{1.847155in}}%
\pgfpathcurveto{\pgfqpoint{3.166623in}{1.852979in}}{\pgfqpoint{3.169895in}{1.860879in}}{\pgfqpoint{3.169895in}{1.869115in}}%
\pgfpathcurveto{\pgfqpoint{3.169895in}{1.877351in}}{\pgfqpoint{3.166623in}{1.885251in}}{\pgfqpoint{3.160799in}{1.891075in}}%
\pgfpathcurveto{\pgfqpoint{3.154975in}{1.896899in}}{\pgfqpoint{3.147075in}{1.900171in}}{\pgfqpoint{3.138839in}{1.900171in}}%
\pgfpathcurveto{\pgfqpoint{3.130603in}{1.900171in}}{\pgfqpoint{3.122703in}{1.896899in}}{\pgfqpoint{3.116879in}{1.891075in}}%
\pgfpathcurveto{\pgfqpoint{3.111055in}{1.885251in}}{\pgfqpoint{3.107782in}{1.877351in}}{\pgfqpoint{3.107782in}{1.869115in}}%
\pgfpathcurveto{\pgfqpoint{3.107782in}{1.860879in}}{\pgfqpoint{3.111055in}{1.852979in}}{\pgfqpoint{3.116879in}{1.847155in}}%
\pgfpathcurveto{\pgfqpoint{3.122703in}{1.841331in}}{\pgfqpoint{3.130603in}{1.838058in}}{\pgfqpoint{3.138839in}{1.838058in}}%
\pgfpathclose%
\pgfusepath{stroke,fill}%
\end{pgfscope}%
\begin{pgfscope}%
\pgfpathrectangle{\pgfqpoint{0.100000in}{0.212622in}}{\pgfqpoint{3.696000in}{3.696000in}}%
\pgfusepath{clip}%
\pgfsetbuttcap%
\pgfsetroundjoin%
\definecolor{currentfill}{rgb}{0.121569,0.466667,0.705882}%
\pgfsetfillcolor{currentfill}%
\pgfsetfillopacity{0.663178}%
\pgfsetlinewidth{1.003750pt}%
\definecolor{currentstroke}{rgb}{0.121569,0.466667,0.705882}%
\pgfsetstrokecolor{currentstroke}%
\pgfsetstrokeopacity{0.663178}%
\pgfsetdash{}{0pt}%
\pgfpathmoveto{\pgfqpoint{1.604159in}{2.930480in}}%
\pgfpathcurveto{\pgfqpoint{1.612395in}{2.930480in}}{\pgfqpoint{1.620295in}{2.933752in}}{\pgfqpoint{1.626119in}{2.939576in}}%
\pgfpathcurveto{\pgfqpoint{1.631943in}{2.945400in}}{\pgfqpoint{1.635215in}{2.953300in}}{\pgfqpoint{1.635215in}{2.961537in}}%
\pgfpathcurveto{\pgfqpoint{1.635215in}{2.969773in}}{\pgfqpoint{1.631943in}{2.977673in}}{\pgfqpoint{1.626119in}{2.983497in}}%
\pgfpathcurveto{\pgfqpoint{1.620295in}{2.989321in}}{\pgfqpoint{1.612395in}{2.992593in}}{\pgfqpoint{1.604159in}{2.992593in}}%
\pgfpathcurveto{\pgfqpoint{1.595922in}{2.992593in}}{\pgfqpoint{1.588022in}{2.989321in}}{\pgfqpoint{1.582198in}{2.983497in}}%
\pgfpathcurveto{\pgfqpoint{1.576374in}{2.977673in}}{\pgfqpoint{1.573102in}{2.969773in}}{\pgfqpoint{1.573102in}{2.961537in}}%
\pgfpathcurveto{\pgfqpoint{1.573102in}{2.953300in}}{\pgfqpoint{1.576374in}{2.945400in}}{\pgfqpoint{1.582198in}{2.939576in}}%
\pgfpathcurveto{\pgfqpoint{1.588022in}{2.933752in}}{\pgfqpoint{1.595922in}{2.930480in}}{\pgfqpoint{1.604159in}{2.930480in}}%
\pgfpathclose%
\pgfusepath{stroke,fill}%
\end{pgfscope}%
\begin{pgfscope}%
\pgfpathrectangle{\pgfqpoint{0.100000in}{0.212622in}}{\pgfqpoint{3.696000in}{3.696000in}}%
\pgfusepath{clip}%
\pgfsetbuttcap%
\pgfsetroundjoin%
\definecolor{currentfill}{rgb}{0.121569,0.466667,0.705882}%
\pgfsetfillcolor{currentfill}%
\pgfsetfillopacity{0.663436}%
\pgfsetlinewidth{1.003750pt}%
\definecolor{currentstroke}{rgb}{0.121569,0.466667,0.705882}%
\pgfsetstrokecolor{currentstroke}%
\pgfsetstrokeopacity{0.663436}%
\pgfsetdash{}{0pt}%
\pgfpathmoveto{\pgfqpoint{2.169772in}{3.031680in}}%
\pgfpathcurveto{\pgfqpoint{2.178008in}{3.031680in}}{\pgfqpoint{2.185908in}{3.034952in}}{\pgfqpoint{2.191732in}{3.040776in}}%
\pgfpathcurveto{\pgfqpoint{2.197556in}{3.046600in}}{\pgfqpoint{2.200828in}{3.054500in}}{\pgfqpoint{2.200828in}{3.062736in}}%
\pgfpathcurveto{\pgfqpoint{2.200828in}{3.070972in}}{\pgfqpoint{2.197556in}{3.078872in}}{\pgfqpoint{2.191732in}{3.084696in}}%
\pgfpathcurveto{\pgfqpoint{2.185908in}{3.090520in}}{\pgfqpoint{2.178008in}{3.093793in}}{\pgfqpoint{2.169772in}{3.093793in}}%
\pgfpathcurveto{\pgfqpoint{2.161535in}{3.093793in}}{\pgfqpoint{2.153635in}{3.090520in}}{\pgfqpoint{2.147811in}{3.084696in}}%
\pgfpathcurveto{\pgfqpoint{2.141987in}{3.078872in}}{\pgfqpoint{2.138715in}{3.070972in}}{\pgfqpoint{2.138715in}{3.062736in}}%
\pgfpathcurveto{\pgfqpoint{2.138715in}{3.054500in}}{\pgfqpoint{2.141987in}{3.046600in}}{\pgfqpoint{2.147811in}{3.040776in}}%
\pgfpathcurveto{\pgfqpoint{2.153635in}{3.034952in}}{\pgfqpoint{2.161535in}{3.031680in}}{\pgfqpoint{2.169772in}{3.031680in}}%
\pgfpathclose%
\pgfusepath{stroke,fill}%
\end{pgfscope}%
\begin{pgfscope}%
\pgfpathrectangle{\pgfqpoint{0.100000in}{0.212622in}}{\pgfqpoint{3.696000in}{3.696000in}}%
\pgfusepath{clip}%
\pgfsetbuttcap%
\pgfsetroundjoin%
\definecolor{currentfill}{rgb}{0.121569,0.466667,0.705882}%
\pgfsetfillcolor{currentfill}%
\pgfsetfillopacity{0.663666}%
\pgfsetlinewidth{1.003750pt}%
\definecolor{currentstroke}{rgb}{0.121569,0.466667,0.705882}%
\pgfsetstrokecolor{currentstroke}%
\pgfsetstrokeopacity{0.663666}%
\pgfsetdash{}{0pt}%
\pgfpathmoveto{\pgfqpoint{1.602740in}{2.928222in}}%
\pgfpathcurveto{\pgfqpoint{1.610976in}{2.928222in}}{\pgfqpoint{1.618876in}{2.931494in}}{\pgfqpoint{1.624700in}{2.937318in}}%
\pgfpathcurveto{\pgfqpoint{1.630524in}{2.943142in}}{\pgfqpoint{1.633796in}{2.951042in}}{\pgfqpoint{1.633796in}{2.959278in}}%
\pgfpathcurveto{\pgfqpoint{1.633796in}{2.967514in}}{\pgfqpoint{1.630524in}{2.975414in}}{\pgfqpoint{1.624700in}{2.981238in}}%
\pgfpathcurveto{\pgfqpoint{1.618876in}{2.987062in}}{\pgfqpoint{1.610976in}{2.990335in}}{\pgfqpoint{1.602740in}{2.990335in}}%
\pgfpathcurveto{\pgfqpoint{1.594503in}{2.990335in}}{\pgfqpoint{1.586603in}{2.987062in}}{\pgfqpoint{1.580779in}{2.981238in}}%
\pgfpathcurveto{\pgfqpoint{1.574955in}{2.975414in}}{\pgfqpoint{1.571683in}{2.967514in}}{\pgfqpoint{1.571683in}{2.959278in}}%
\pgfpathcurveto{\pgfqpoint{1.571683in}{2.951042in}}{\pgfqpoint{1.574955in}{2.943142in}}{\pgfqpoint{1.580779in}{2.937318in}}%
\pgfpathcurveto{\pgfqpoint{1.586603in}{2.931494in}}{\pgfqpoint{1.594503in}{2.928222in}}{\pgfqpoint{1.602740in}{2.928222in}}%
\pgfpathclose%
\pgfusepath{stroke,fill}%
\end{pgfscope}%
\begin{pgfscope}%
\pgfpathrectangle{\pgfqpoint{0.100000in}{0.212622in}}{\pgfqpoint{3.696000in}{3.696000in}}%
\pgfusepath{clip}%
\pgfsetbuttcap%
\pgfsetroundjoin%
\definecolor{currentfill}{rgb}{0.121569,0.466667,0.705882}%
\pgfsetfillcolor{currentfill}%
\pgfsetfillopacity{0.663940}%
\pgfsetlinewidth{1.003750pt}%
\definecolor{currentstroke}{rgb}{0.121569,0.466667,0.705882}%
\pgfsetstrokecolor{currentstroke}%
\pgfsetstrokeopacity{0.663940}%
\pgfsetdash{}{0pt}%
\pgfpathmoveto{\pgfqpoint{1.601952in}{2.927013in}}%
\pgfpathcurveto{\pgfqpoint{1.610189in}{2.927013in}}{\pgfqpoint{1.618089in}{2.930285in}}{\pgfqpoint{1.623913in}{2.936109in}}%
\pgfpathcurveto{\pgfqpoint{1.629737in}{2.941933in}}{\pgfqpoint{1.633009in}{2.949833in}}{\pgfqpoint{1.633009in}{2.958069in}}%
\pgfpathcurveto{\pgfqpoint{1.633009in}{2.966305in}}{\pgfqpoint{1.629737in}{2.974205in}}{\pgfqpoint{1.623913in}{2.980029in}}%
\pgfpathcurveto{\pgfqpoint{1.618089in}{2.985853in}}{\pgfqpoint{1.610189in}{2.989126in}}{\pgfqpoint{1.601952in}{2.989126in}}%
\pgfpathcurveto{\pgfqpoint{1.593716in}{2.989126in}}{\pgfqpoint{1.585816in}{2.985853in}}{\pgfqpoint{1.579992in}{2.980029in}}%
\pgfpathcurveto{\pgfqpoint{1.574168in}{2.974205in}}{\pgfqpoint{1.570896in}{2.966305in}}{\pgfqpoint{1.570896in}{2.958069in}}%
\pgfpathcurveto{\pgfqpoint{1.570896in}{2.949833in}}{\pgfqpoint{1.574168in}{2.941933in}}{\pgfqpoint{1.579992in}{2.936109in}}%
\pgfpathcurveto{\pgfqpoint{1.585816in}{2.930285in}}{\pgfqpoint{1.593716in}{2.927013in}}{\pgfqpoint{1.601952in}{2.927013in}}%
\pgfpathclose%
\pgfusepath{stroke,fill}%
\end{pgfscope}%
\begin{pgfscope}%
\pgfpathrectangle{\pgfqpoint{0.100000in}{0.212622in}}{\pgfqpoint{3.696000in}{3.696000in}}%
\pgfusepath{clip}%
\pgfsetbuttcap%
\pgfsetroundjoin%
\definecolor{currentfill}{rgb}{0.121569,0.466667,0.705882}%
\pgfsetfillcolor{currentfill}%
\pgfsetfillopacity{0.664482}%
\pgfsetlinewidth{1.003750pt}%
\definecolor{currentstroke}{rgb}{0.121569,0.466667,0.705882}%
\pgfsetstrokecolor{currentstroke}%
\pgfsetstrokeopacity{0.664482}%
\pgfsetdash{}{0pt}%
\pgfpathmoveto{\pgfqpoint{1.600405in}{2.924846in}}%
\pgfpathcurveto{\pgfqpoint{1.608641in}{2.924846in}}{\pgfqpoint{1.616541in}{2.928119in}}{\pgfqpoint{1.622365in}{2.933943in}}%
\pgfpathcurveto{\pgfqpoint{1.628189in}{2.939766in}}{\pgfqpoint{1.631462in}{2.947667in}}{\pgfqpoint{1.631462in}{2.955903in}}%
\pgfpathcurveto{\pgfqpoint{1.631462in}{2.964139in}}{\pgfqpoint{1.628189in}{2.972039in}}{\pgfqpoint{1.622365in}{2.977863in}}%
\pgfpathcurveto{\pgfqpoint{1.616541in}{2.983687in}}{\pgfqpoint{1.608641in}{2.986959in}}{\pgfqpoint{1.600405in}{2.986959in}}%
\pgfpathcurveto{\pgfqpoint{1.592169in}{2.986959in}}{\pgfqpoint{1.584269in}{2.983687in}}{\pgfqpoint{1.578445in}{2.977863in}}%
\pgfpathcurveto{\pgfqpoint{1.572621in}{2.972039in}}{\pgfqpoint{1.569349in}{2.964139in}}{\pgfqpoint{1.569349in}{2.955903in}}%
\pgfpathcurveto{\pgfqpoint{1.569349in}{2.947667in}}{\pgfqpoint{1.572621in}{2.939766in}}{\pgfqpoint{1.578445in}{2.933943in}}%
\pgfpathcurveto{\pgfqpoint{1.584269in}{2.928119in}}{\pgfqpoint{1.592169in}{2.924846in}}{\pgfqpoint{1.600405in}{2.924846in}}%
\pgfpathclose%
\pgfusepath{stroke,fill}%
\end{pgfscope}%
\begin{pgfscope}%
\pgfpathrectangle{\pgfqpoint{0.100000in}{0.212622in}}{\pgfqpoint{3.696000in}{3.696000in}}%
\pgfusepath{clip}%
\pgfsetbuttcap%
\pgfsetroundjoin%
\definecolor{currentfill}{rgb}{0.121569,0.466667,0.705882}%
\pgfsetfillcolor{currentfill}%
\pgfsetfillopacity{0.664617}%
\pgfsetlinewidth{1.003750pt}%
\definecolor{currentstroke}{rgb}{0.121569,0.466667,0.705882}%
\pgfsetstrokecolor{currentstroke}%
\pgfsetstrokeopacity{0.664617}%
\pgfsetdash{}{0pt}%
\pgfpathmoveto{\pgfqpoint{3.135710in}{1.833294in}}%
\pgfpathcurveto{\pgfqpoint{3.143946in}{1.833294in}}{\pgfqpoint{3.151846in}{1.836566in}}{\pgfqpoint{3.157670in}{1.842390in}}%
\pgfpathcurveto{\pgfqpoint{3.163494in}{1.848214in}}{\pgfqpoint{3.166766in}{1.856114in}}{\pgfqpoint{3.166766in}{1.864350in}}%
\pgfpathcurveto{\pgfqpoint{3.166766in}{1.872587in}}{\pgfqpoint{3.163494in}{1.880487in}}{\pgfqpoint{3.157670in}{1.886311in}}%
\pgfpathcurveto{\pgfqpoint{3.151846in}{1.892135in}}{\pgfqpoint{3.143946in}{1.895407in}}{\pgfqpoint{3.135710in}{1.895407in}}%
\pgfpathcurveto{\pgfqpoint{3.127473in}{1.895407in}}{\pgfqpoint{3.119573in}{1.892135in}}{\pgfqpoint{3.113749in}{1.886311in}}%
\pgfpathcurveto{\pgfqpoint{3.107925in}{1.880487in}}{\pgfqpoint{3.104653in}{1.872587in}}{\pgfqpoint{3.104653in}{1.864350in}}%
\pgfpathcurveto{\pgfqpoint{3.104653in}{1.856114in}}{\pgfqpoint{3.107925in}{1.848214in}}{\pgfqpoint{3.113749in}{1.842390in}}%
\pgfpathcurveto{\pgfqpoint{3.119573in}{1.836566in}}{\pgfqpoint{3.127473in}{1.833294in}}{\pgfqpoint{3.135710in}{1.833294in}}%
\pgfpathclose%
\pgfusepath{stroke,fill}%
\end{pgfscope}%
\begin{pgfscope}%
\pgfpathrectangle{\pgfqpoint{0.100000in}{0.212622in}}{\pgfqpoint{3.696000in}{3.696000in}}%
\pgfusepath{clip}%
\pgfsetbuttcap%
\pgfsetroundjoin%
\definecolor{currentfill}{rgb}{0.121569,0.466667,0.705882}%
\pgfsetfillcolor{currentfill}%
\pgfsetfillopacity{0.664882}%
\pgfsetlinewidth{1.003750pt}%
\definecolor{currentstroke}{rgb}{0.121569,0.466667,0.705882}%
\pgfsetstrokecolor{currentstroke}%
\pgfsetstrokeopacity{0.664882}%
\pgfsetdash{}{0pt}%
\pgfpathmoveto{\pgfqpoint{2.180868in}{3.029959in}}%
\pgfpathcurveto{\pgfqpoint{2.189104in}{3.029959in}}{\pgfqpoint{2.197004in}{3.033231in}}{\pgfqpoint{2.202828in}{3.039055in}}%
\pgfpathcurveto{\pgfqpoint{2.208652in}{3.044879in}}{\pgfqpoint{2.211925in}{3.052779in}}{\pgfqpoint{2.211925in}{3.061015in}}%
\pgfpathcurveto{\pgfqpoint{2.211925in}{3.069251in}}{\pgfqpoint{2.208652in}{3.077151in}}{\pgfqpoint{2.202828in}{3.082975in}}%
\pgfpathcurveto{\pgfqpoint{2.197004in}{3.088799in}}{\pgfqpoint{2.189104in}{3.092072in}}{\pgfqpoint{2.180868in}{3.092072in}}%
\pgfpathcurveto{\pgfqpoint{2.172632in}{3.092072in}}{\pgfqpoint{2.164732in}{3.088799in}}{\pgfqpoint{2.158908in}{3.082975in}}%
\pgfpathcurveto{\pgfqpoint{2.153084in}{3.077151in}}{\pgfqpoint{2.149812in}{3.069251in}}{\pgfqpoint{2.149812in}{3.061015in}}%
\pgfpathcurveto{\pgfqpoint{2.149812in}{3.052779in}}{\pgfqpoint{2.153084in}{3.044879in}}{\pgfqpoint{2.158908in}{3.039055in}}%
\pgfpathcurveto{\pgfqpoint{2.164732in}{3.033231in}}{\pgfqpoint{2.172632in}{3.029959in}}{\pgfqpoint{2.180868in}{3.029959in}}%
\pgfpathclose%
\pgfusepath{stroke,fill}%
\end{pgfscope}%
\begin{pgfscope}%
\pgfpathrectangle{\pgfqpoint{0.100000in}{0.212622in}}{\pgfqpoint{3.696000in}{3.696000in}}%
\pgfusepath{clip}%
\pgfsetbuttcap%
\pgfsetroundjoin%
\definecolor{currentfill}{rgb}{0.121569,0.466667,0.705882}%
\pgfsetfillcolor{currentfill}%
\pgfsetfillopacity{0.665066}%
\pgfsetlinewidth{1.003750pt}%
\definecolor{currentstroke}{rgb}{0.121569,0.466667,0.705882}%
\pgfsetstrokecolor{currentstroke}%
\pgfsetstrokeopacity{0.665066}%
\pgfsetdash{}{0pt}%
\pgfpathmoveto{\pgfqpoint{1.598577in}{2.922243in}}%
\pgfpathcurveto{\pgfqpoint{1.606813in}{2.922243in}}{\pgfqpoint{1.614713in}{2.925515in}}{\pgfqpoint{1.620537in}{2.931339in}}%
\pgfpathcurveto{\pgfqpoint{1.626361in}{2.937163in}}{\pgfqpoint{1.629634in}{2.945063in}}{\pgfqpoint{1.629634in}{2.953299in}}%
\pgfpathcurveto{\pgfqpoint{1.629634in}{2.961536in}}{\pgfqpoint{1.626361in}{2.969436in}}{\pgfqpoint{1.620537in}{2.975260in}}%
\pgfpathcurveto{\pgfqpoint{1.614713in}{2.981084in}}{\pgfqpoint{1.606813in}{2.984356in}}{\pgfqpoint{1.598577in}{2.984356in}}%
\pgfpathcurveto{\pgfqpoint{1.590341in}{2.984356in}}{\pgfqpoint{1.582441in}{2.981084in}}{\pgfqpoint{1.576617in}{2.975260in}}%
\pgfpathcurveto{\pgfqpoint{1.570793in}{2.969436in}}{\pgfqpoint{1.567521in}{2.961536in}}{\pgfqpoint{1.567521in}{2.953299in}}%
\pgfpathcurveto{\pgfqpoint{1.567521in}{2.945063in}}{\pgfqpoint{1.570793in}{2.937163in}}{\pgfqpoint{1.576617in}{2.931339in}}%
\pgfpathcurveto{\pgfqpoint{1.582441in}{2.925515in}}{\pgfqpoint{1.590341in}{2.922243in}}{\pgfqpoint{1.598577in}{2.922243in}}%
\pgfpathclose%
\pgfusepath{stroke,fill}%
\end{pgfscope}%
\begin{pgfscope}%
\pgfpathrectangle{\pgfqpoint{0.100000in}{0.212622in}}{\pgfqpoint{3.696000in}{3.696000in}}%
\pgfusepath{clip}%
\pgfsetbuttcap%
\pgfsetroundjoin%
\definecolor{currentfill}{rgb}{0.121569,0.466667,0.705882}%
\pgfsetfillcolor{currentfill}%
\pgfsetfillopacity{0.665901}%
\pgfsetlinewidth{1.003750pt}%
\definecolor{currentstroke}{rgb}{0.121569,0.466667,0.705882}%
\pgfsetstrokecolor{currentstroke}%
\pgfsetstrokeopacity{0.665901}%
\pgfsetdash{}{0pt}%
\pgfpathmoveto{\pgfqpoint{1.595725in}{2.917909in}}%
\pgfpathcurveto{\pgfqpoint{1.603961in}{2.917909in}}{\pgfqpoint{1.611861in}{2.921181in}}{\pgfqpoint{1.617685in}{2.927005in}}%
\pgfpathcurveto{\pgfqpoint{1.623509in}{2.932829in}}{\pgfqpoint{1.626781in}{2.940729in}}{\pgfqpoint{1.626781in}{2.948965in}}%
\pgfpathcurveto{\pgfqpoint{1.626781in}{2.957201in}}{\pgfqpoint{1.623509in}{2.965101in}}{\pgfqpoint{1.617685in}{2.970925in}}%
\pgfpathcurveto{\pgfqpoint{1.611861in}{2.976749in}}{\pgfqpoint{1.603961in}{2.980022in}}{\pgfqpoint{1.595725in}{2.980022in}}%
\pgfpathcurveto{\pgfqpoint{1.587488in}{2.980022in}}{\pgfqpoint{1.579588in}{2.976749in}}{\pgfqpoint{1.573764in}{2.970925in}}%
\pgfpathcurveto{\pgfqpoint{1.567940in}{2.965101in}}{\pgfqpoint{1.564668in}{2.957201in}}{\pgfqpoint{1.564668in}{2.948965in}}%
\pgfpathcurveto{\pgfqpoint{1.564668in}{2.940729in}}{\pgfqpoint{1.567940in}{2.932829in}}{\pgfqpoint{1.573764in}{2.927005in}}%
\pgfpathcurveto{\pgfqpoint{1.579588in}{2.921181in}}{\pgfqpoint{1.587488in}{2.917909in}}{\pgfqpoint{1.595725in}{2.917909in}}%
\pgfpathclose%
\pgfusepath{stroke,fill}%
\end{pgfscope}%
\begin{pgfscope}%
\pgfpathrectangle{\pgfqpoint{0.100000in}{0.212622in}}{\pgfqpoint{3.696000in}{3.696000in}}%
\pgfusepath{clip}%
\pgfsetbuttcap%
\pgfsetroundjoin%
\definecolor{currentfill}{rgb}{0.121569,0.466667,0.705882}%
\pgfsetfillcolor{currentfill}%
\pgfsetfillopacity{0.666092}%
\pgfsetlinewidth{1.003750pt}%
\definecolor{currentstroke}{rgb}{0.121569,0.466667,0.705882}%
\pgfsetstrokecolor{currentstroke}%
\pgfsetstrokeopacity{0.666092}%
\pgfsetdash{}{0pt}%
\pgfpathmoveto{\pgfqpoint{3.132748in}{1.829055in}}%
\pgfpathcurveto{\pgfqpoint{3.140985in}{1.829055in}}{\pgfqpoint{3.148885in}{1.832327in}}{\pgfqpoint{3.154709in}{1.838151in}}%
\pgfpathcurveto{\pgfqpoint{3.160533in}{1.843975in}}{\pgfqpoint{3.163805in}{1.851875in}}{\pgfqpoint{3.163805in}{1.860111in}}%
\pgfpathcurveto{\pgfqpoint{3.163805in}{1.868348in}}{\pgfqpoint{3.160533in}{1.876248in}}{\pgfqpoint{3.154709in}{1.882071in}}%
\pgfpathcurveto{\pgfqpoint{3.148885in}{1.887895in}}{\pgfqpoint{3.140985in}{1.891168in}}{\pgfqpoint{3.132748in}{1.891168in}}%
\pgfpathcurveto{\pgfqpoint{3.124512in}{1.891168in}}{\pgfqpoint{3.116612in}{1.887895in}}{\pgfqpoint{3.110788in}{1.882071in}}%
\pgfpathcurveto{\pgfqpoint{3.104964in}{1.876248in}}{\pgfqpoint{3.101692in}{1.868348in}}{\pgfqpoint{3.101692in}{1.860111in}}%
\pgfpathcurveto{\pgfqpoint{3.101692in}{1.851875in}}{\pgfqpoint{3.104964in}{1.843975in}}{\pgfqpoint{3.110788in}{1.838151in}}%
\pgfpathcurveto{\pgfqpoint{3.116612in}{1.832327in}}{\pgfqpoint{3.124512in}{1.829055in}}{\pgfqpoint{3.132748in}{1.829055in}}%
\pgfpathclose%
\pgfusepath{stroke,fill}%
\end{pgfscope}%
\begin{pgfscope}%
\pgfpathrectangle{\pgfqpoint{0.100000in}{0.212622in}}{\pgfqpoint{3.696000in}{3.696000in}}%
\pgfusepath{clip}%
\pgfsetbuttcap%
\pgfsetroundjoin%
\definecolor{currentfill}{rgb}{0.121569,0.466667,0.705882}%
\pgfsetfillcolor{currentfill}%
\pgfsetfillopacity{0.666119}%
\pgfsetlinewidth{1.003750pt}%
\definecolor{currentstroke}{rgb}{0.121569,0.466667,0.705882}%
\pgfsetstrokecolor{currentstroke}%
\pgfsetstrokeopacity{0.666119}%
\pgfsetdash{}{0pt}%
\pgfpathmoveto{\pgfqpoint{2.190754in}{3.028015in}}%
\pgfpathcurveto{\pgfqpoint{2.198990in}{3.028015in}}{\pgfqpoint{2.206890in}{3.031287in}}{\pgfqpoint{2.212714in}{3.037111in}}%
\pgfpathcurveto{\pgfqpoint{2.218538in}{3.042935in}}{\pgfqpoint{2.221810in}{3.050835in}}{\pgfqpoint{2.221810in}{3.059072in}}%
\pgfpathcurveto{\pgfqpoint{2.221810in}{3.067308in}}{\pgfqpoint{2.218538in}{3.075208in}}{\pgfqpoint{2.212714in}{3.081032in}}%
\pgfpathcurveto{\pgfqpoint{2.206890in}{3.086856in}}{\pgfqpoint{2.198990in}{3.090128in}}{\pgfqpoint{2.190754in}{3.090128in}}%
\pgfpathcurveto{\pgfqpoint{2.182517in}{3.090128in}}{\pgfqpoint{2.174617in}{3.086856in}}{\pgfqpoint{2.168793in}{3.081032in}}%
\pgfpathcurveto{\pgfqpoint{2.162970in}{3.075208in}}{\pgfqpoint{2.159697in}{3.067308in}}{\pgfqpoint{2.159697in}{3.059072in}}%
\pgfpathcurveto{\pgfqpoint{2.159697in}{3.050835in}}{\pgfqpoint{2.162970in}{3.042935in}}{\pgfqpoint{2.168793in}{3.037111in}}%
\pgfpathcurveto{\pgfqpoint{2.174617in}{3.031287in}}{\pgfqpoint{2.182517in}{3.028015in}}{\pgfqpoint{2.190754in}{3.028015in}}%
\pgfpathclose%
\pgfusepath{stroke,fill}%
\end{pgfscope}%
\begin{pgfscope}%
\pgfpathrectangle{\pgfqpoint{0.100000in}{0.212622in}}{\pgfqpoint{3.696000in}{3.696000in}}%
\pgfusepath{clip}%
\pgfsetbuttcap%
\pgfsetroundjoin%
\definecolor{currentfill}{rgb}{0.121569,0.466667,0.705882}%
\pgfsetfillcolor{currentfill}%
\pgfsetfillopacity{0.666357}%
\pgfsetlinewidth{1.003750pt}%
\definecolor{currentstroke}{rgb}{0.121569,0.466667,0.705882}%
\pgfsetstrokecolor{currentstroke}%
\pgfsetstrokeopacity{0.666357}%
\pgfsetdash{}{0pt}%
\pgfpathmoveto{\pgfqpoint{1.594176in}{2.915481in}}%
\pgfpathcurveto{\pgfqpoint{1.602412in}{2.915481in}}{\pgfqpoint{1.610312in}{2.918753in}}{\pgfqpoint{1.616136in}{2.924577in}}%
\pgfpathcurveto{\pgfqpoint{1.621960in}{2.930401in}}{\pgfqpoint{1.625233in}{2.938301in}}{\pgfqpoint{1.625233in}{2.946537in}}%
\pgfpathcurveto{\pgfqpoint{1.625233in}{2.954774in}}{\pgfqpoint{1.621960in}{2.962674in}}{\pgfqpoint{1.616136in}{2.968498in}}%
\pgfpathcurveto{\pgfqpoint{1.610312in}{2.974322in}}{\pgfqpoint{1.602412in}{2.977594in}}{\pgfqpoint{1.594176in}{2.977594in}}%
\pgfpathcurveto{\pgfqpoint{1.585940in}{2.977594in}}{\pgfqpoint{1.578040in}{2.974322in}}{\pgfqpoint{1.572216in}{2.968498in}}%
\pgfpathcurveto{\pgfqpoint{1.566392in}{2.962674in}}{\pgfqpoint{1.563120in}{2.954774in}}{\pgfqpoint{1.563120in}{2.946537in}}%
\pgfpathcurveto{\pgfqpoint{1.563120in}{2.938301in}}{\pgfqpoint{1.566392in}{2.930401in}}{\pgfqpoint{1.572216in}{2.924577in}}%
\pgfpathcurveto{\pgfqpoint{1.578040in}{2.918753in}}{\pgfqpoint{1.585940in}{2.915481in}}{\pgfqpoint{1.594176in}{2.915481in}}%
\pgfpathclose%
\pgfusepath{stroke,fill}%
\end{pgfscope}%
\begin{pgfscope}%
\pgfpathrectangle{\pgfqpoint{0.100000in}{0.212622in}}{\pgfqpoint{3.696000in}{3.696000in}}%
\pgfusepath{clip}%
\pgfsetbuttcap%
\pgfsetroundjoin%
\definecolor{currentfill}{rgb}{0.121569,0.466667,0.705882}%
\pgfsetfillcolor{currentfill}%
\pgfsetfillopacity{0.667100}%
\pgfsetlinewidth{1.003750pt}%
\definecolor{currentstroke}{rgb}{0.121569,0.466667,0.705882}%
\pgfsetstrokecolor{currentstroke}%
\pgfsetstrokeopacity{0.667100}%
\pgfsetdash{}{0pt}%
\pgfpathmoveto{\pgfqpoint{1.591785in}{2.911965in}}%
\pgfpathcurveto{\pgfqpoint{1.600021in}{2.911965in}}{\pgfqpoint{1.607921in}{2.915237in}}{\pgfqpoint{1.613745in}{2.921061in}}%
\pgfpathcurveto{\pgfqpoint{1.619569in}{2.926885in}}{\pgfqpoint{1.622842in}{2.934785in}}{\pgfqpoint{1.622842in}{2.943022in}}%
\pgfpathcurveto{\pgfqpoint{1.622842in}{2.951258in}}{\pgfqpoint{1.619569in}{2.959158in}}{\pgfqpoint{1.613745in}{2.964982in}}%
\pgfpathcurveto{\pgfqpoint{1.607921in}{2.970806in}}{\pgfqpoint{1.600021in}{2.974078in}}{\pgfqpoint{1.591785in}{2.974078in}}%
\pgfpathcurveto{\pgfqpoint{1.583549in}{2.974078in}}{\pgfqpoint{1.575649in}{2.970806in}}{\pgfqpoint{1.569825in}{2.964982in}}%
\pgfpathcurveto{\pgfqpoint{1.564001in}{2.959158in}}{\pgfqpoint{1.560729in}{2.951258in}}{\pgfqpoint{1.560729in}{2.943022in}}%
\pgfpathcurveto{\pgfqpoint{1.560729in}{2.934785in}}{\pgfqpoint{1.564001in}{2.926885in}}{\pgfqpoint{1.569825in}{2.921061in}}%
\pgfpathcurveto{\pgfqpoint{1.575649in}{2.915237in}}{\pgfqpoint{1.583549in}{2.911965in}}{\pgfqpoint{1.591785in}{2.911965in}}%
\pgfpathclose%
\pgfusepath{stroke,fill}%
\end{pgfscope}%
\begin{pgfscope}%
\pgfpathrectangle{\pgfqpoint{0.100000in}{0.212622in}}{\pgfqpoint{3.696000in}{3.696000in}}%
\pgfusepath{clip}%
\pgfsetbuttcap%
\pgfsetroundjoin%
\definecolor{currentfill}{rgb}{0.121569,0.466667,0.705882}%
\pgfsetfillcolor{currentfill}%
\pgfsetfillopacity{0.667282}%
\pgfsetlinewidth{1.003750pt}%
\definecolor{currentstroke}{rgb}{0.121569,0.466667,0.705882}%
\pgfsetstrokecolor{currentstroke}%
\pgfsetstrokeopacity{0.667282}%
\pgfsetdash{}{0pt}%
\pgfpathmoveto{\pgfqpoint{2.200200in}{3.025904in}}%
\pgfpathcurveto{\pgfqpoint{2.208436in}{3.025904in}}{\pgfqpoint{2.216337in}{3.029176in}}{\pgfqpoint{2.222160in}{3.035000in}}%
\pgfpathcurveto{\pgfqpoint{2.227984in}{3.040824in}}{\pgfqpoint{2.231257in}{3.048724in}}{\pgfqpoint{2.231257in}{3.056960in}}%
\pgfpathcurveto{\pgfqpoint{2.231257in}{3.065196in}}{\pgfqpoint{2.227984in}{3.073096in}}{\pgfqpoint{2.222160in}{3.078920in}}%
\pgfpathcurveto{\pgfqpoint{2.216337in}{3.084744in}}{\pgfqpoint{2.208436in}{3.088017in}}{\pgfqpoint{2.200200in}{3.088017in}}%
\pgfpathcurveto{\pgfqpoint{2.191964in}{3.088017in}}{\pgfqpoint{2.184064in}{3.084744in}}{\pgfqpoint{2.178240in}{3.078920in}}%
\pgfpathcurveto{\pgfqpoint{2.172416in}{3.073096in}}{\pgfqpoint{2.169144in}{3.065196in}}{\pgfqpoint{2.169144in}{3.056960in}}%
\pgfpathcurveto{\pgfqpoint{2.169144in}{3.048724in}}{\pgfqpoint{2.172416in}{3.040824in}}{\pgfqpoint{2.178240in}{3.035000in}}%
\pgfpathcurveto{\pgfqpoint{2.184064in}{3.029176in}}{\pgfqpoint{2.191964in}{3.025904in}}{\pgfqpoint{2.200200in}{3.025904in}}%
\pgfpathclose%
\pgfusepath{stroke,fill}%
\end{pgfscope}%
\begin{pgfscope}%
\pgfpathrectangle{\pgfqpoint{0.100000in}{0.212622in}}{\pgfqpoint{3.696000in}{3.696000in}}%
\pgfusepath{clip}%
\pgfsetbuttcap%
\pgfsetroundjoin%
\definecolor{currentfill}{rgb}{0.121569,0.466667,0.705882}%
\pgfsetfillcolor{currentfill}%
\pgfsetfillopacity{0.667507}%
\pgfsetlinewidth{1.003750pt}%
\definecolor{currentstroke}{rgb}{0.121569,0.466667,0.705882}%
\pgfsetstrokecolor{currentstroke}%
\pgfsetstrokeopacity{0.667507}%
\pgfsetdash{}{0pt}%
\pgfpathmoveto{\pgfqpoint{3.129957in}{1.825595in}}%
\pgfpathcurveto{\pgfqpoint{3.138193in}{1.825595in}}{\pgfqpoint{3.146093in}{1.828867in}}{\pgfqpoint{3.151917in}{1.834691in}}%
\pgfpathcurveto{\pgfqpoint{3.157741in}{1.840515in}}{\pgfqpoint{3.161013in}{1.848415in}}{\pgfqpoint{3.161013in}{1.856651in}}%
\pgfpathcurveto{\pgfqpoint{3.161013in}{1.864887in}}{\pgfqpoint{3.157741in}{1.872787in}}{\pgfqpoint{3.151917in}{1.878611in}}%
\pgfpathcurveto{\pgfqpoint{3.146093in}{1.884435in}}{\pgfqpoint{3.138193in}{1.887708in}}{\pgfqpoint{3.129957in}{1.887708in}}%
\pgfpathcurveto{\pgfqpoint{3.121720in}{1.887708in}}{\pgfqpoint{3.113820in}{1.884435in}}{\pgfqpoint{3.107996in}{1.878611in}}%
\pgfpathcurveto{\pgfqpoint{3.102172in}{1.872787in}}{\pgfqpoint{3.098900in}{1.864887in}}{\pgfqpoint{3.098900in}{1.856651in}}%
\pgfpathcurveto{\pgfqpoint{3.098900in}{1.848415in}}{\pgfqpoint{3.102172in}{1.840515in}}{\pgfqpoint{3.107996in}{1.834691in}}%
\pgfpathcurveto{\pgfqpoint{3.113820in}{1.828867in}}{\pgfqpoint{3.121720in}{1.825595in}}{\pgfqpoint{3.129957in}{1.825595in}}%
\pgfpathclose%
\pgfusepath{stroke,fill}%
\end{pgfscope}%
\begin{pgfscope}%
\pgfpathrectangle{\pgfqpoint{0.100000in}{0.212622in}}{\pgfqpoint{3.696000in}{3.696000in}}%
\pgfusepath{clip}%
\pgfsetbuttcap%
\pgfsetroundjoin%
\definecolor{currentfill}{rgb}{0.121569,0.466667,0.705882}%
\pgfsetfillcolor{currentfill}%
\pgfsetfillopacity{0.667521}%
\pgfsetlinewidth{1.003750pt}%
\definecolor{currentstroke}{rgb}{0.121569,0.466667,0.705882}%
\pgfsetstrokecolor{currentstroke}%
\pgfsetstrokeopacity{0.667521}%
\pgfsetdash{}{0pt}%
\pgfpathmoveto{\pgfqpoint{1.590412in}{2.910177in}}%
\pgfpathcurveto{\pgfqpoint{1.598648in}{2.910177in}}{\pgfqpoint{1.606548in}{2.913449in}}{\pgfqpoint{1.612372in}{2.919273in}}%
\pgfpathcurveto{\pgfqpoint{1.618196in}{2.925097in}}{\pgfqpoint{1.621468in}{2.932997in}}{\pgfqpoint{1.621468in}{2.941233in}}%
\pgfpathcurveto{\pgfqpoint{1.621468in}{2.949469in}}{\pgfqpoint{1.618196in}{2.957369in}}{\pgfqpoint{1.612372in}{2.963193in}}%
\pgfpathcurveto{\pgfqpoint{1.606548in}{2.969017in}}{\pgfqpoint{1.598648in}{2.972290in}}{\pgfqpoint{1.590412in}{2.972290in}}%
\pgfpathcurveto{\pgfqpoint{1.582176in}{2.972290in}}{\pgfqpoint{1.574275in}{2.969017in}}{\pgfqpoint{1.568452in}{2.963193in}}%
\pgfpathcurveto{\pgfqpoint{1.562628in}{2.957369in}}{\pgfqpoint{1.559355in}{2.949469in}}{\pgfqpoint{1.559355in}{2.941233in}}%
\pgfpathcurveto{\pgfqpoint{1.559355in}{2.932997in}}{\pgfqpoint{1.562628in}{2.925097in}}{\pgfqpoint{1.568452in}{2.919273in}}%
\pgfpathcurveto{\pgfqpoint{1.574275in}{2.913449in}}{\pgfqpoint{1.582176in}{2.910177in}}{\pgfqpoint{1.590412in}{2.910177in}}%
\pgfpathclose%
\pgfusepath{stroke,fill}%
\end{pgfscope}%
\begin{pgfscope}%
\pgfpathrectangle{\pgfqpoint{0.100000in}{0.212622in}}{\pgfqpoint{3.696000in}{3.696000in}}%
\pgfusepath{clip}%
\pgfsetbuttcap%
\pgfsetroundjoin%
\definecolor{currentfill}{rgb}{0.121569,0.466667,0.705882}%
\pgfsetfillcolor{currentfill}%
\pgfsetfillopacity{0.668159}%
\pgfsetlinewidth{1.003750pt}%
\definecolor{currentstroke}{rgb}{0.121569,0.466667,0.705882}%
\pgfsetstrokecolor{currentstroke}%
\pgfsetstrokeopacity{0.668159}%
\pgfsetdash{}{0pt}%
\pgfpathmoveto{\pgfqpoint{1.588115in}{2.907057in}}%
\pgfpathcurveto{\pgfqpoint{1.596351in}{2.907057in}}{\pgfqpoint{1.604252in}{2.910329in}}{\pgfqpoint{1.610075in}{2.916153in}}%
\pgfpathcurveto{\pgfqpoint{1.615899in}{2.921977in}}{\pgfqpoint{1.619172in}{2.929877in}}{\pgfqpoint{1.619172in}{2.938113in}}%
\pgfpathcurveto{\pgfqpoint{1.619172in}{2.946349in}}{\pgfqpoint{1.615899in}{2.954249in}}{\pgfqpoint{1.610075in}{2.960073in}}%
\pgfpathcurveto{\pgfqpoint{1.604252in}{2.965897in}}{\pgfqpoint{1.596351in}{2.969170in}}{\pgfqpoint{1.588115in}{2.969170in}}%
\pgfpathcurveto{\pgfqpoint{1.579879in}{2.969170in}}{\pgfqpoint{1.571979in}{2.965897in}}{\pgfqpoint{1.566155in}{2.960073in}}%
\pgfpathcurveto{\pgfqpoint{1.560331in}{2.954249in}}{\pgfqpoint{1.557059in}{2.946349in}}{\pgfqpoint{1.557059in}{2.938113in}}%
\pgfpathcurveto{\pgfqpoint{1.557059in}{2.929877in}}{\pgfqpoint{1.560331in}{2.921977in}}{\pgfqpoint{1.566155in}{2.916153in}}%
\pgfpathcurveto{\pgfqpoint{1.571979in}{2.910329in}}{\pgfqpoint{1.579879in}{2.907057in}}{\pgfqpoint{1.588115in}{2.907057in}}%
\pgfpathclose%
\pgfusepath{stroke,fill}%
\end{pgfscope}%
\begin{pgfscope}%
\pgfpathrectangle{\pgfqpoint{0.100000in}{0.212622in}}{\pgfqpoint{3.696000in}{3.696000in}}%
\pgfusepath{clip}%
\pgfsetbuttcap%
\pgfsetroundjoin%
\definecolor{currentfill}{rgb}{0.121569,0.466667,0.705882}%
\pgfsetfillcolor{currentfill}%
\pgfsetfillopacity{0.668407}%
\pgfsetlinewidth{1.003750pt}%
\definecolor{currentstroke}{rgb}{0.121569,0.466667,0.705882}%
\pgfsetstrokecolor{currentstroke}%
\pgfsetstrokeopacity{0.668407}%
\pgfsetdash{}{0pt}%
\pgfpathmoveto{\pgfqpoint{3.128224in}{1.823482in}}%
\pgfpathcurveto{\pgfqpoint{3.136460in}{1.823482in}}{\pgfqpoint{3.144360in}{1.826754in}}{\pgfqpoint{3.150184in}{1.832578in}}%
\pgfpathcurveto{\pgfqpoint{3.156008in}{1.838402in}}{\pgfqpoint{3.159280in}{1.846302in}}{\pgfqpoint{3.159280in}{1.854538in}}%
\pgfpathcurveto{\pgfqpoint{3.159280in}{1.862774in}}{\pgfqpoint{3.156008in}{1.870675in}}{\pgfqpoint{3.150184in}{1.876498in}}%
\pgfpathcurveto{\pgfqpoint{3.144360in}{1.882322in}}{\pgfqpoint{3.136460in}{1.885595in}}{\pgfqpoint{3.128224in}{1.885595in}}%
\pgfpathcurveto{\pgfqpoint{3.119988in}{1.885595in}}{\pgfqpoint{3.112088in}{1.882322in}}{\pgfqpoint{3.106264in}{1.876498in}}%
\pgfpathcurveto{\pgfqpoint{3.100440in}{1.870675in}}{\pgfqpoint{3.097167in}{1.862774in}}{\pgfqpoint{3.097167in}{1.854538in}}%
\pgfpathcurveto{\pgfqpoint{3.097167in}{1.846302in}}{\pgfqpoint{3.100440in}{1.838402in}}{\pgfqpoint{3.106264in}{1.832578in}}%
\pgfpathcurveto{\pgfqpoint{3.112088in}{1.826754in}}{\pgfqpoint{3.119988in}{1.823482in}}{\pgfqpoint{3.128224in}{1.823482in}}%
\pgfpathclose%
\pgfusepath{stroke,fill}%
\end{pgfscope}%
\begin{pgfscope}%
\pgfpathrectangle{\pgfqpoint{0.100000in}{0.212622in}}{\pgfqpoint{3.696000in}{3.696000in}}%
\pgfusepath{clip}%
\pgfsetbuttcap%
\pgfsetroundjoin%
\definecolor{currentfill}{rgb}{0.121569,0.466667,0.705882}%
\pgfsetfillcolor{currentfill}%
\pgfsetfillopacity{0.668497}%
\pgfsetlinewidth{1.003750pt}%
\definecolor{currentstroke}{rgb}{0.121569,0.466667,0.705882}%
\pgfsetstrokecolor{currentstroke}%
\pgfsetstrokeopacity{0.668497}%
\pgfsetdash{}{0pt}%
\pgfpathmoveto{\pgfqpoint{2.209266in}{3.024468in}}%
\pgfpathcurveto{\pgfqpoint{2.217502in}{3.024468in}}{\pgfqpoint{2.225402in}{3.027740in}}{\pgfqpoint{2.231226in}{3.033564in}}%
\pgfpathcurveto{\pgfqpoint{2.237050in}{3.039388in}}{\pgfqpoint{2.240322in}{3.047288in}}{\pgfqpoint{2.240322in}{3.055524in}}%
\pgfpathcurveto{\pgfqpoint{2.240322in}{3.063760in}}{\pgfqpoint{2.237050in}{3.071660in}}{\pgfqpoint{2.231226in}{3.077484in}}%
\pgfpathcurveto{\pgfqpoint{2.225402in}{3.083308in}}{\pgfqpoint{2.217502in}{3.086581in}}{\pgfqpoint{2.209266in}{3.086581in}}%
\pgfpathcurveto{\pgfqpoint{2.201030in}{3.086581in}}{\pgfqpoint{2.193130in}{3.083308in}}{\pgfqpoint{2.187306in}{3.077484in}}%
\pgfpathcurveto{\pgfqpoint{2.181482in}{3.071660in}}{\pgfqpoint{2.178209in}{3.063760in}}{\pgfqpoint{2.178209in}{3.055524in}}%
\pgfpathcurveto{\pgfqpoint{2.178209in}{3.047288in}}{\pgfqpoint{2.181482in}{3.039388in}}{\pgfqpoint{2.187306in}{3.033564in}}%
\pgfpathcurveto{\pgfqpoint{2.193130in}{3.027740in}}{\pgfqpoint{2.201030in}{3.024468in}}{\pgfqpoint{2.209266in}{3.024468in}}%
\pgfpathclose%
\pgfusepath{stroke,fill}%
\end{pgfscope}%
\begin{pgfscope}%
\pgfpathrectangle{\pgfqpoint{0.100000in}{0.212622in}}{\pgfqpoint{3.696000in}{3.696000in}}%
\pgfusepath{clip}%
\pgfsetbuttcap%
\pgfsetroundjoin%
\definecolor{currentfill}{rgb}{0.121569,0.466667,0.705882}%
\pgfsetfillcolor{currentfill}%
\pgfsetfillopacity{0.668910}%
\pgfsetlinewidth{1.003750pt}%
\definecolor{currentstroke}{rgb}{0.121569,0.466667,0.705882}%
\pgfsetstrokecolor{currentstroke}%
\pgfsetstrokeopacity{0.668910}%
\pgfsetdash{}{0pt}%
\pgfpathmoveto{\pgfqpoint{1.585198in}{2.902373in}}%
\pgfpathcurveto{\pgfqpoint{1.593434in}{2.902373in}}{\pgfqpoint{1.601334in}{2.905645in}}{\pgfqpoint{1.607158in}{2.911469in}}%
\pgfpathcurveto{\pgfqpoint{1.612982in}{2.917293in}}{\pgfqpoint{1.616254in}{2.925193in}}{\pgfqpoint{1.616254in}{2.933429in}}%
\pgfpathcurveto{\pgfqpoint{1.616254in}{2.941665in}}{\pgfqpoint{1.612982in}{2.949565in}}{\pgfqpoint{1.607158in}{2.955389in}}%
\pgfpathcurveto{\pgfqpoint{1.601334in}{2.961213in}}{\pgfqpoint{1.593434in}{2.964486in}}{\pgfqpoint{1.585198in}{2.964486in}}%
\pgfpathcurveto{\pgfqpoint{1.576961in}{2.964486in}}{\pgfqpoint{1.569061in}{2.961213in}}{\pgfqpoint{1.563237in}{2.955389in}}%
\pgfpathcurveto{\pgfqpoint{1.557413in}{2.949565in}}{\pgfqpoint{1.554141in}{2.941665in}}{\pgfqpoint{1.554141in}{2.933429in}}%
\pgfpathcurveto{\pgfqpoint{1.554141in}{2.925193in}}{\pgfqpoint{1.557413in}{2.917293in}}{\pgfqpoint{1.563237in}{2.911469in}}%
\pgfpathcurveto{\pgfqpoint{1.569061in}{2.905645in}}{\pgfqpoint{1.576961in}{2.902373in}}{\pgfqpoint{1.585198in}{2.902373in}}%
\pgfpathclose%
\pgfusepath{stroke,fill}%
\end{pgfscope}%
\begin{pgfscope}%
\pgfpathrectangle{\pgfqpoint{0.100000in}{0.212622in}}{\pgfqpoint{3.696000in}{3.696000in}}%
\pgfusepath{clip}%
\pgfsetbuttcap%
\pgfsetroundjoin%
\definecolor{currentfill}{rgb}{0.121569,0.466667,0.705882}%
\pgfsetfillcolor{currentfill}%
\pgfsetfillopacity{0.669015}%
\pgfsetlinewidth{1.003750pt}%
\definecolor{currentstroke}{rgb}{0.121569,0.466667,0.705882}%
\pgfsetstrokecolor{currentstroke}%
\pgfsetstrokeopacity{0.669015}%
\pgfsetdash{}{0pt}%
\pgfpathmoveto{\pgfqpoint{3.127090in}{1.822149in}}%
\pgfpathcurveto{\pgfqpoint{3.135326in}{1.822149in}}{\pgfqpoint{3.143226in}{1.825422in}}{\pgfqpoint{3.149050in}{1.831246in}}%
\pgfpathcurveto{\pgfqpoint{3.154874in}{1.837070in}}{\pgfqpoint{3.158146in}{1.844970in}}{\pgfqpoint{3.158146in}{1.853206in}}%
\pgfpathcurveto{\pgfqpoint{3.158146in}{1.861442in}}{\pgfqpoint{3.154874in}{1.869342in}}{\pgfqpoint{3.149050in}{1.875166in}}%
\pgfpathcurveto{\pgfqpoint{3.143226in}{1.880990in}}{\pgfqpoint{3.135326in}{1.884262in}}{\pgfqpoint{3.127090in}{1.884262in}}%
\pgfpathcurveto{\pgfqpoint{3.118853in}{1.884262in}}{\pgfqpoint{3.110953in}{1.880990in}}{\pgfqpoint{3.105129in}{1.875166in}}%
\pgfpathcurveto{\pgfqpoint{3.099305in}{1.869342in}}{\pgfqpoint{3.096033in}{1.861442in}}{\pgfqpoint{3.096033in}{1.853206in}}%
\pgfpathcurveto{\pgfqpoint{3.096033in}{1.844970in}}{\pgfqpoint{3.099305in}{1.837070in}}{\pgfqpoint{3.105129in}{1.831246in}}%
\pgfpathcurveto{\pgfqpoint{3.110953in}{1.825422in}}{\pgfqpoint{3.118853in}{1.822149in}}{\pgfqpoint{3.127090in}{1.822149in}}%
\pgfpathclose%
\pgfusepath{stroke,fill}%
\end{pgfscope}%
\begin{pgfscope}%
\pgfpathrectangle{\pgfqpoint{0.100000in}{0.212622in}}{\pgfqpoint{3.696000in}{3.696000in}}%
\pgfusepath{clip}%
\pgfsetbuttcap%
\pgfsetroundjoin%
\definecolor{currentfill}{rgb}{0.121569,0.466667,0.705882}%
\pgfsetfillcolor{currentfill}%
\pgfsetfillopacity{0.669355}%
\pgfsetlinewidth{1.003750pt}%
\definecolor{currentstroke}{rgb}{0.121569,0.466667,0.705882}%
\pgfsetstrokecolor{currentstroke}%
\pgfsetstrokeopacity{0.669355}%
\pgfsetdash{}{0pt}%
\pgfpathmoveto{\pgfqpoint{1.583620in}{2.899950in}}%
\pgfpathcurveto{\pgfqpoint{1.591856in}{2.899950in}}{\pgfqpoint{1.599756in}{2.903223in}}{\pgfqpoint{1.605580in}{2.909046in}}%
\pgfpathcurveto{\pgfqpoint{1.611404in}{2.914870in}}{\pgfqpoint{1.614676in}{2.922770in}}{\pgfqpoint{1.614676in}{2.931007in}}%
\pgfpathcurveto{\pgfqpoint{1.614676in}{2.939243in}}{\pgfqpoint{1.611404in}{2.947143in}}{\pgfqpoint{1.605580in}{2.952967in}}%
\pgfpathcurveto{\pgfqpoint{1.599756in}{2.958791in}}{\pgfqpoint{1.591856in}{2.962063in}}{\pgfqpoint{1.583620in}{2.962063in}}%
\pgfpathcurveto{\pgfqpoint{1.575384in}{2.962063in}}{\pgfqpoint{1.567484in}{2.958791in}}{\pgfqpoint{1.561660in}{2.952967in}}%
\pgfpathcurveto{\pgfqpoint{1.555836in}{2.947143in}}{\pgfqpoint{1.552563in}{2.939243in}}{\pgfqpoint{1.552563in}{2.931007in}}%
\pgfpathcurveto{\pgfqpoint{1.552563in}{2.922770in}}{\pgfqpoint{1.555836in}{2.914870in}}{\pgfqpoint{1.561660in}{2.909046in}}%
\pgfpathcurveto{\pgfqpoint{1.567484in}{2.903223in}}{\pgfqpoint{1.575384in}{2.899950in}}{\pgfqpoint{1.583620in}{2.899950in}}%
\pgfpathclose%
\pgfusepath{stroke,fill}%
\end{pgfscope}%
\begin{pgfscope}%
\pgfpathrectangle{\pgfqpoint{0.100000in}{0.212622in}}{\pgfqpoint{3.696000in}{3.696000in}}%
\pgfusepath{clip}%
\pgfsetbuttcap%
\pgfsetroundjoin%
\definecolor{currentfill}{rgb}{0.121569,0.466667,0.705882}%
\pgfsetfillcolor{currentfill}%
\pgfsetfillopacity{0.669530}%
\pgfsetlinewidth{1.003750pt}%
\definecolor{currentstroke}{rgb}{0.121569,0.466667,0.705882}%
\pgfsetstrokecolor{currentstroke}%
\pgfsetstrokeopacity{0.669530}%
\pgfsetdash{}{0pt}%
\pgfpathmoveto{\pgfqpoint{2.216934in}{3.023588in}}%
\pgfpathcurveto{\pgfqpoint{2.225170in}{3.023588in}}{\pgfqpoint{2.233070in}{3.026860in}}{\pgfqpoint{2.238894in}{3.032684in}}%
\pgfpathcurveto{\pgfqpoint{2.244718in}{3.038508in}}{\pgfqpoint{2.247990in}{3.046408in}}{\pgfqpoint{2.247990in}{3.054644in}}%
\pgfpathcurveto{\pgfqpoint{2.247990in}{3.062881in}}{\pgfqpoint{2.244718in}{3.070781in}}{\pgfqpoint{2.238894in}{3.076605in}}%
\pgfpathcurveto{\pgfqpoint{2.233070in}{3.082429in}}{\pgfqpoint{2.225170in}{3.085701in}}{\pgfqpoint{2.216934in}{3.085701in}}%
\pgfpathcurveto{\pgfqpoint{2.208697in}{3.085701in}}{\pgfqpoint{2.200797in}{3.082429in}}{\pgfqpoint{2.194973in}{3.076605in}}%
\pgfpathcurveto{\pgfqpoint{2.189149in}{3.070781in}}{\pgfqpoint{2.185877in}{3.062881in}}{\pgfqpoint{2.185877in}{3.054644in}}%
\pgfpathcurveto{\pgfqpoint{2.185877in}{3.046408in}}{\pgfqpoint{2.189149in}{3.038508in}}{\pgfqpoint{2.194973in}{3.032684in}}%
\pgfpathcurveto{\pgfqpoint{2.200797in}{3.026860in}}{\pgfqpoint{2.208697in}{3.023588in}}{\pgfqpoint{2.216934in}{3.023588in}}%
\pgfpathclose%
\pgfusepath{stroke,fill}%
\end{pgfscope}%
\begin{pgfscope}%
\pgfpathrectangle{\pgfqpoint{0.100000in}{0.212622in}}{\pgfqpoint{3.696000in}{3.696000in}}%
\pgfusepath{clip}%
\pgfsetbuttcap%
\pgfsetroundjoin%
\definecolor{currentfill}{rgb}{0.121569,0.466667,0.705882}%
\pgfsetfillcolor{currentfill}%
\pgfsetfillopacity{0.669627}%
\pgfsetlinewidth{1.003750pt}%
\definecolor{currentstroke}{rgb}{0.121569,0.466667,0.705882}%
\pgfsetstrokecolor{currentstroke}%
\pgfsetstrokeopacity{0.669627}%
\pgfsetdash{}{0pt}%
\pgfpathmoveto{\pgfqpoint{1.582747in}{2.898790in}}%
\pgfpathcurveto{\pgfqpoint{1.590983in}{2.898790in}}{\pgfqpoint{1.598883in}{2.902062in}}{\pgfqpoint{1.604707in}{2.907886in}}%
\pgfpathcurveto{\pgfqpoint{1.610531in}{2.913710in}}{\pgfqpoint{1.613804in}{2.921610in}}{\pgfqpoint{1.613804in}{2.929846in}}%
\pgfpathcurveto{\pgfqpoint{1.613804in}{2.938082in}}{\pgfqpoint{1.610531in}{2.945982in}}{\pgfqpoint{1.604707in}{2.951806in}}%
\pgfpathcurveto{\pgfqpoint{1.598883in}{2.957630in}}{\pgfqpoint{1.590983in}{2.960903in}}{\pgfqpoint{1.582747in}{2.960903in}}%
\pgfpathcurveto{\pgfqpoint{1.574511in}{2.960903in}}{\pgfqpoint{1.566611in}{2.957630in}}{\pgfqpoint{1.560787in}{2.951806in}}%
\pgfpathcurveto{\pgfqpoint{1.554963in}{2.945982in}}{\pgfqpoint{1.551691in}{2.938082in}}{\pgfqpoint{1.551691in}{2.929846in}}%
\pgfpathcurveto{\pgfqpoint{1.551691in}{2.921610in}}{\pgfqpoint{1.554963in}{2.913710in}}{\pgfqpoint{1.560787in}{2.907886in}}%
\pgfpathcurveto{\pgfqpoint{1.566611in}{2.902062in}}{\pgfqpoint{1.574511in}{2.898790in}}{\pgfqpoint{1.582747in}{2.898790in}}%
\pgfpathclose%
\pgfusepath{stroke,fill}%
\end{pgfscope}%
\begin{pgfscope}%
\pgfpathrectangle{\pgfqpoint{0.100000in}{0.212622in}}{\pgfqpoint{3.696000in}{3.696000in}}%
\pgfusepath{clip}%
\pgfsetbuttcap%
\pgfsetroundjoin%
\definecolor{currentfill}{rgb}{0.121569,0.466667,0.705882}%
\pgfsetfillcolor{currentfill}%
\pgfsetfillopacity{0.669976}%
\pgfsetlinewidth{1.003750pt}%
\definecolor{currentstroke}{rgb}{0.121569,0.466667,0.705882}%
\pgfsetstrokecolor{currentstroke}%
\pgfsetstrokeopacity{0.669976}%
\pgfsetdash{}{0pt}%
\pgfpathmoveto{\pgfqpoint{1.581548in}{2.897104in}}%
\pgfpathcurveto{\pgfqpoint{1.589784in}{2.897104in}}{\pgfqpoint{1.597684in}{2.900377in}}{\pgfqpoint{1.603508in}{2.906200in}}%
\pgfpathcurveto{\pgfqpoint{1.609332in}{2.912024in}}{\pgfqpoint{1.612604in}{2.919924in}}{\pgfqpoint{1.612604in}{2.928161in}}%
\pgfpathcurveto{\pgfqpoint{1.612604in}{2.936397in}}{\pgfqpoint{1.609332in}{2.944297in}}{\pgfqpoint{1.603508in}{2.950121in}}%
\pgfpathcurveto{\pgfqpoint{1.597684in}{2.955945in}}{\pgfqpoint{1.589784in}{2.959217in}}{\pgfqpoint{1.581548in}{2.959217in}}%
\pgfpathcurveto{\pgfqpoint{1.573312in}{2.959217in}}{\pgfqpoint{1.565412in}{2.955945in}}{\pgfqpoint{1.559588in}{2.950121in}}%
\pgfpathcurveto{\pgfqpoint{1.553764in}{2.944297in}}{\pgfqpoint{1.550491in}{2.936397in}}{\pgfqpoint{1.550491in}{2.928161in}}%
\pgfpathcurveto{\pgfqpoint{1.550491in}{2.919924in}}{\pgfqpoint{1.553764in}{2.912024in}}{\pgfqpoint{1.559588in}{2.906200in}}%
\pgfpathcurveto{\pgfqpoint{1.565412in}{2.900377in}}{\pgfqpoint{1.573312in}{2.897104in}}{\pgfqpoint{1.581548in}{2.897104in}}%
\pgfpathclose%
\pgfusepath{stroke,fill}%
\end{pgfscope}%
\begin{pgfscope}%
\pgfpathrectangle{\pgfqpoint{0.100000in}{0.212622in}}{\pgfqpoint{3.696000in}{3.696000in}}%
\pgfusepath{clip}%
\pgfsetbuttcap%
\pgfsetroundjoin%
\definecolor{currentfill}{rgb}{0.121569,0.466667,0.705882}%
\pgfsetfillcolor{currentfill}%
\pgfsetfillopacity{0.670062}%
\pgfsetlinewidth{1.003750pt}%
\definecolor{currentstroke}{rgb}{0.121569,0.466667,0.705882}%
\pgfsetstrokecolor{currentstroke}%
\pgfsetstrokeopacity{0.670062}%
\pgfsetdash{}{0pt}%
\pgfpathmoveto{\pgfqpoint{3.125039in}{1.819350in}}%
\pgfpathcurveto{\pgfqpoint{3.133275in}{1.819350in}}{\pgfqpoint{3.141175in}{1.822622in}}{\pgfqpoint{3.146999in}{1.828446in}}%
\pgfpathcurveto{\pgfqpoint{3.152823in}{1.834270in}}{\pgfqpoint{3.156096in}{1.842170in}}{\pgfqpoint{3.156096in}{1.850406in}}%
\pgfpathcurveto{\pgfqpoint{3.156096in}{1.858642in}}{\pgfqpoint{3.152823in}{1.866543in}}{\pgfqpoint{3.146999in}{1.872366in}}%
\pgfpathcurveto{\pgfqpoint{3.141175in}{1.878190in}}{\pgfqpoint{3.133275in}{1.881463in}}{\pgfqpoint{3.125039in}{1.881463in}}%
\pgfpathcurveto{\pgfqpoint{3.116803in}{1.881463in}}{\pgfqpoint{3.108903in}{1.878190in}}{\pgfqpoint{3.103079in}{1.872366in}}%
\pgfpathcurveto{\pgfqpoint{3.097255in}{1.866543in}}{\pgfqpoint{3.093983in}{1.858642in}}{\pgfqpoint{3.093983in}{1.850406in}}%
\pgfpathcurveto{\pgfqpoint{3.093983in}{1.842170in}}{\pgfqpoint{3.097255in}{1.834270in}}{\pgfqpoint{3.103079in}{1.828446in}}%
\pgfpathcurveto{\pgfqpoint{3.108903in}{1.822622in}}{\pgfqpoint{3.116803in}{1.819350in}}{\pgfqpoint{3.125039in}{1.819350in}}%
\pgfpathclose%
\pgfusepath{stroke,fill}%
\end{pgfscope}%
\begin{pgfscope}%
\pgfpathrectangle{\pgfqpoint{0.100000in}{0.212622in}}{\pgfqpoint{3.696000in}{3.696000in}}%
\pgfusepath{clip}%
\pgfsetbuttcap%
\pgfsetroundjoin%
\definecolor{currentfill}{rgb}{0.121569,0.466667,0.705882}%
\pgfsetfillcolor{currentfill}%
\pgfsetfillopacity{0.670453}%
\pgfsetlinewidth{1.003750pt}%
\definecolor{currentstroke}{rgb}{0.121569,0.466667,0.705882}%
\pgfsetstrokecolor{currentstroke}%
\pgfsetstrokeopacity{0.670453}%
\pgfsetdash{}{0pt}%
\pgfpathmoveto{\pgfqpoint{2.224317in}{3.022453in}}%
\pgfpathcurveto{\pgfqpoint{2.232553in}{3.022453in}}{\pgfqpoint{2.240453in}{3.025725in}}{\pgfqpoint{2.246277in}{3.031549in}}%
\pgfpathcurveto{\pgfqpoint{2.252101in}{3.037373in}}{\pgfqpoint{2.255373in}{3.045273in}}{\pgfqpoint{2.255373in}{3.053509in}}%
\pgfpathcurveto{\pgfqpoint{2.255373in}{3.061745in}}{\pgfqpoint{2.252101in}{3.069645in}}{\pgfqpoint{2.246277in}{3.075469in}}%
\pgfpathcurveto{\pgfqpoint{2.240453in}{3.081293in}}{\pgfqpoint{2.232553in}{3.084566in}}{\pgfqpoint{2.224317in}{3.084566in}}%
\pgfpathcurveto{\pgfqpoint{2.216081in}{3.084566in}}{\pgfqpoint{2.208181in}{3.081293in}}{\pgfqpoint{2.202357in}{3.075469in}}%
\pgfpathcurveto{\pgfqpoint{2.196533in}{3.069645in}}{\pgfqpoint{2.193260in}{3.061745in}}{\pgfqpoint{2.193260in}{3.053509in}}%
\pgfpathcurveto{\pgfqpoint{2.193260in}{3.045273in}}{\pgfqpoint{2.196533in}{3.037373in}}{\pgfqpoint{2.202357in}{3.031549in}}%
\pgfpathcurveto{\pgfqpoint{2.208181in}{3.025725in}}{\pgfqpoint{2.216081in}{3.022453in}}{\pgfqpoint{2.224317in}{3.022453in}}%
\pgfpathclose%
\pgfusepath{stroke,fill}%
\end{pgfscope}%
\begin{pgfscope}%
\pgfpathrectangle{\pgfqpoint{0.100000in}{0.212622in}}{\pgfqpoint{3.696000in}{3.696000in}}%
\pgfusepath{clip}%
\pgfsetbuttcap%
\pgfsetroundjoin%
\definecolor{currentfill}{rgb}{0.121569,0.466667,0.705882}%
\pgfsetfillcolor{currentfill}%
\pgfsetfillopacity{0.670517}%
\pgfsetlinewidth{1.003750pt}%
\definecolor{currentstroke}{rgb}{0.121569,0.466667,0.705882}%
\pgfsetstrokecolor{currentstroke}%
\pgfsetstrokeopacity{0.670517}%
\pgfsetdash{}{0pt}%
\pgfpathmoveto{\pgfqpoint{1.579521in}{2.893910in}}%
\pgfpathcurveto{\pgfqpoint{1.587757in}{2.893910in}}{\pgfqpoint{1.595657in}{2.897182in}}{\pgfqpoint{1.601481in}{2.903006in}}%
\pgfpathcurveto{\pgfqpoint{1.607305in}{2.908830in}}{\pgfqpoint{1.610577in}{2.916730in}}{\pgfqpoint{1.610577in}{2.924967in}}%
\pgfpathcurveto{\pgfqpoint{1.610577in}{2.933203in}}{\pgfqpoint{1.607305in}{2.941103in}}{\pgfqpoint{1.601481in}{2.946927in}}%
\pgfpathcurveto{\pgfqpoint{1.595657in}{2.952751in}}{\pgfqpoint{1.587757in}{2.956023in}}{\pgfqpoint{1.579521in}{2.956023in}}%
\pgfpathcurveto{\pgfqpoint{1.571284in}{2.956023in}}{\pgfqpoint{1.563384in}{2.952751in}}{\pgfqpoint{1.557560in}{2.946927in}}%
\pgfpathcurveto{\pgfqpoint{1.551736in}{2.941103in}}{\pgfqpoint{1.548464in}{2.933203in}}{\pgfqpoint{1.548464in}{2.924967in}}%
\pgfpathcurveto{\pgfqpoint{1.548464in}{2.916730in}}{\pgfqpoint{1.551736in}{2.908830in}}{\pgfqpoint{1.557560in}{2.903006in}}%
\pgfpathcurveto{\pgfqpoint{1.563384in}{2.897182in}}{\pgfqpoint{1.571284in}{2.893910in}}{\pgfqpoint{1.579521in}{2.893910in}}%
\pgfpathclose%
\pgfusepath{stroke,fill}%
\end{pgfscope}%
\begin{pgfscope}%
\pgfpathrectangle{\pgfqpoint{0.100000in}{0.212622in}}{\pgfqpoint{3.696000in}{3.696000in}}%
\pgfusepath{clip}%
\pgfsetbuttcap%
\pgfsetroundjoin%
\definecolor{currentfill}{rgb}{0.121569,0.466667,0.705882}%
\pgfsetfillcolor{currentfill}%
\pgfsetfillopacity{0.670811}%
\pgfsetlinewidth{1.003750pt}%
\definecolor{currentstroke}{rgb}{0.121569,0.466667,0.705882}%
\pgfsetstrokecolor{currentstroke}%
\pgfsetstrokeopacity{0.670811}%
\pgfsetdash{}{0pt}%
\pgfpathmoveto{\pgfqpoint{3.123538in}{1.817220in}}%
\pgfpathcurveto{\pgfqpoint{3.131775in}{1.817220in}}{\pgfqpoint{3.139675in}{1.820493in}}{\pgfqpoint{3.145499in}{1.826317in}}%
\pgfpathcurveto{\pgfqpoint{3.151323in}{1.832140in}}{\pgfqpoint{3.154595in}{1.840041in}}{\pgfqpoint{3.154595in}{1.848277in}}%
\pgfpathcurveto{\pgfqpoint{3.154595in}{1.856513in}}{\pgfqpoint{3.151323in}{1.864413in}}{\pgfqpoint{3.145499in}{1.870237in}}%
\pgfpathcurveto{\pgfqpoint{3.139675in}{1.876061in}}{\pgfqpoint{3.131775in}{1.879333in}}{\pgfqpoint{3.123538in}{1.879333in}}%
\pgfpathcurveto{\pgfqpoint{3.115302in}{1.879333in}}{\pgfqpoint{3.107402in}{1.876061in}}{\pgfqpoint{3.101578in}{1.870237in}}%
\pgfpathcurveto{\pgfqpoint{3.095754in}{1.864413in}}{\pgfqpoint{3.092482in}{1.856513in}}{\pgfqpoint{3.092482in}{1.848277in}}%
\pgfpathcurveto{\pgfqpoint{3.092482in}{1.840041in}}{\pgfqpoint{3.095754in}{1.832140in}}{\pgfqpoint{3.101578in}{1.826317in}}%
\pgfpathcurveto{\pgfqpoint{3.107402in}{1.820493in}}{\pgfqpoint{3.115302in}{1.817220in}}{\pgfqpoint{3.123538in}{1.817220in}}%
\pgfpathclose%
\pgfusepath{stroke,fill}%
\end{pgfscope}%
\begin{pgfscope}%
\pgfpathrectangle{\pgfqpoint{0.100000in}{0.212622in}}{\pgfqpoint{3.696000in}{3.696000in}}%
\pgfusepath{clip}%
\pgfsetbuttcap%
\pgfsetroundjoin%
\definecolor{currentfill}{rgb}{0.121569,0.466667,0.705882}%
\pgfsetfillcolor{currentfill}%
\pgfsetfillopacity{0.670834}%
\pgfsetlinewidth{1.003750pt}%
\definecolor{currentstroke}{rgb}{0.121569,0.466667,0.705882}%
\pgfsetstrokecolor{currentstroke}%
\pgfsetstrokeopacity{0.670834}%
\pgfsetdash{}{0pt}%
\pgfpathmoveto{\pgfqpoint{1.578410in}{2.892263in}}%
\pgfpathcurveto{\pgfqpoint{1.586646in}{2.892263in}}{\pgfqpoint{1.594546in}{2.895535in}}{\pgfqpoint{1.600370in}{2.901359in}}%
\pgfpathcurveto{\pgfqpoint{1.606194in}{2.907183in}}{\pgfqpoint{1.609467in}{2.915083in}}{\pgfqpoint{1.609467in}{2.923320in}}%
\pgfpathcurveto{\pgfqpoint{1.609467in}{2.931556in}}{\pgfqpoint{1.606194in}{2.939456in}}{\pgfqpoint{1.600370in}{2.945280in}}%
\pgfpathcurveto{\pgfqpoint{1.594546in}{2.951104in}}{\pgfqpoint{1.586646in}{2.954376in}}{\pgfqpoint{1.578410in}{2.954376in}}%
\pgfpathcurveto{\pgfqpoint{1.570174in}{2.954376in}}{\pgfqpoint{1.562274in}{2.951104in}}{\pgfqpoint{1.556450in}{2.945280in}}%
\pgfpathcurveto{\pgfqpoint{1.550626in}{2.939456in}}{\pgfqpoint{1.547354in}{2.931556in}}{\pgfqpoint{1.547354in}{2.923320in}}%
\pgfpathcurveto{\pgfqpoint{1.547354in}{2.915083in}}{\pgfqpoint{1.550626in}{2.907183in}}{\pgfqpoint{1.556450in}{2.901359in}}%
\pgfpathcurveto{\pgfqpoint{1.562274in}{2.895535in}}{\pgfqpoint{1.570174in}{2.892263in}}{\pgfqpoint{1.578410in}{2.892263in}}%
\pgfpathclose%
\pgfusepath{stroke,fill}%
\end{pgfscope}%
\begin{pgfscope}%
\pgfpathrectangle{\pgfqpoint{0.100000in}{0.212622in}}{\pgfqpoint{3.696000in}{3.696000in}}%
\pgfusepath{clip}%
\pgfsetbuttcap%
\pgfsetroundjoin%
\definecolor{currentfill}{rgb}{0.121569,0.466667,0.705882}%
\pgfsetfillcolor{currentfill}%
\pgfsetfillopacity{0.671200}%
\pgfsetlinewidth{1.003750pt}%
\definecolor{currentstroke}{rgb}{0.121569,0.466667,0.705882}%
\pgfsetstrokecolor{currentstroke}%
\pgfsetstrokeopacity{0.671200}%
\pgfsetdash{}{0pt}%
\pgfpathmoveto{\pgfqpoint{2.230532in}{3.021191in}}%
\pgfpathcurveto{\pgfqpoint{2.238768in}{3.021191in}}{\pgfqpoint{2.246668in}{3.024463in}}{\pgfqpoint{2.252492in}{3.030287in}}%
\pgfpathcurveto{\pgfqpoint{2.258316in}{3.036111in}}{\pgfqpoint{2.261589in}{3.044011in}}{\pgfqpoint{2.261589in}{3.052248in}}%
\pgfpathcurveto{\pgfqpoint{2.261589in}{3.060484in}}{\pgfqpoint{2.258316in}{3.068384in}}{\pgfqpoint{2.252492in}{3.074208in}}%
\pgfpathcurveto{\pgfqpoint{2.246668in}{3.080032in}}{\pgfqpoint{2.238768in}{3.083304in}}{\pgfqpoint{2.230532in}{3.083304in}}%
\pgfpathcurveto{\pgfqpoint{2.222296in}{3.083304in}}{\pgfqpoint{2.214396in}{3.080032in}}{\pgfqpoint{2.208572in}{3.074208in}}%
\pgfpathcurveto{\pgfqpoint{2.202748in}{3.068384in}}{\pgfqpoint{2.199476in}{3.060484in}}{\pgfqpoint{2.199476in}{3.052248in}}%
\pgfpathcurveto{\pgfqpoint{2.199476in}{3.044011in}}{\pgfqpoint{2.202748in}{3.036111in}}{\pgfqpoint{2.208572in}{3.030287in}}%
\pgfpathcurveto{\pgfqpoint{2.214396in}{3.024463in}}{\pgfqpoint{2.222296in}{3.021191in}}{\pgfqpoint{2.230532in}{3.021191in}}%
\pgfpathclose%
\pgfusepath{stroke,fill}%
\end{pgfscope}%
\begin{pgfscope}%
\pgfpathrectangle{\pgfqpoint{0.100000in}{0.212622in}}{\pgfqpoint{3.696000in}{3.696000in}}%
\pgfusepath{clip}%
\pgfsetbuttcap%
\pgfsetroundjoin%
\definecolor{currentfill}{rgb}{0.121569,0.466667,0.705882}%
\pgfsetfillcolor{currentfill}%
\pgfsetfillopacity{0.671247}%
\pgfsetlinewidth{1.003750pt}%
\definecolor{currentstroke}{rgb}{0.121569,0.466667,0.705882}%
\pgfsetstrokecolor{currentstroke}%
\pgfsetstrokeopacity{0.671247}%
\pgfsetdash{}{0pt}%
\pgfpathmoveto{\pgfqpoint{1.577090in}{2.890498in}}%
\pgfpathcurveto{\pgfqpoint{1.585326in}{2.890498in}}{\pgfqpoint{1.593226in}{2.893771in}}{\pgfqpoint{1.599050in}{2.899595in}}%
\pgfpathcurveto{\pgfqpoint{1.604874in}{2.905419in}}{\pgfqpoint{1.608146in}{2.913319in}}{\pgfqpoint{1.608146in}{2.921555in}}%
\pgfpathcurveto{\pgfqpoint{1.608146in}{2.929791in}}{\pgfqpoint{1.604874in}{2.937691in}}{\pgfqpoint{1.599050in}{2.943515in}}%
\pgfpathcurveto{\pgfqpoint{1.593226in}{2.949339in}}{\pgfqpoint{1.585326in}{2.952611in}}{\pgfqpoint{1.577090in}{2.952611in}}%
\pgfpathcurveto{\pgfqpoint{1.568853in}{2.952611in}}{\pgfqpoint{1.560953in}{2.949339in}}{\pgfqpoint{1.555129in}{2.943515in}}%
\pgfpathcurveto{\pgfqpoint{1.549305in}{2.937691in}}{\pgfqpoint{1.546033in}{2.929791in}}{\pgfqpoint{1.546033in}{2.921555in}}%
\pgfpathcurveto{\pgfqpoint{1.546033in}{2.913319in}}{\pgfqpoint{1.549305in}{2.905419in}}{\pgfqpoint{1.555129in}{2.899595in}}%
\pgfpathcurveto{\pgfqpoint{1.560953in}{2.893771in}}{\pgfqpoint{1.568853in}{2.890498in}}{\pgfqpoint{1.577090in}{2.890498in}}%
\pgfpathclose%
\pgfusepath{stroke,fill}%
\end{pgfscope}%
\begin{pgfscope}%
\pgfpathrectangle{\pgfqpoint{0.100000in}{0.212622in}}{\pgfqpoint{3.696000in}{3.696000in}}%
\pgfusepath{clip}%
\pgfsetbuttcap%
\pgfsetroundjoin%
\definecolor{currentfill}{rgb}{0.121569,0.466667,0.705882}%
\pgfsetfillcolor{currentfill}%
\pgfsetfillopacity{0.671475}%
\pgfsetlinewidth{1.003750pt}%
\definecolor{currentstroke}{rgb}{0.121569,0.466667,0.705882}%
\pgfsetstrokecolor{currentstroke}%
\pgfsetstrokeopacity{0.671475}%
\pgfsetdash{}{0pt}%
\pgfpathmoveto{\pgfqpoint{3.122163in}{1.815392in}}%
\pgfpathcurveto{\pgfqpoint{3.130399in}{1.815392in}}{\pgfqpoint{3.138299in}{1.818664in}}{\pgfqpoint{3.144123in}{1.824488in}}%
\pgfpathcurveto{\pgfqpoint{3.149947in}{1.830312in}}{\pgfqpoint{3.153219in}{1.838212in}}{\pgfqpoint{3.153219in}{1.846449in}}%
\pgfpathcurveto{\pgfqpoint{3.153219in}{1.854685in}}{\pgfqpoint{3.149947in}{1.862585in}}{\pgfqpoint{3.144123in}{1.868409in}}%
\pgfpathcurveto{\pgfqpoint{3.138299in}{1.874233in}}{\pgfqpoint{3.130399in}{1.877505in}}{\pgfqpoint{3.122163in}{1.877505in}}%
\pgfpathcurveto{\pgfqpoint{3.113926in}{1.877505in}}{\pgfqpoint{3.106026in}{1.874233in}}{\pgfqpoint{3.100202in}{1.868409in}}%
\pgfpathcurveto{\pgfqpoint{3.094378in}{1.862585in}}{\pgfqpoint{3.091106in}{1.854685in}}{\pgfqpoint{3.091106in}{1.846449in}}%
\pgfpathcurveto{\pgfqpoint{3.091106in}{1.838212in}}{\pgfqpoint{3.094378in}{1.830312in}}{\pgfqpoint{3.100202in}{1.824488in}}%
\pgfpathcurveto{\pgfqpoint{3.106026in}{1.818664in}}{\pgfqpoint{3.113926in}{1.815392in}}{\pgfqpoint{3.122163in}{1.815392in}}%
\pgfpathclose%
\pgfusepath{stroke,fill}%
\end{pgfscope}%
\begin{pgfscope}%
\pgfpathrectangle{\pgfqpoint{0.100000in}{0.212622in}}{\pgfqpoint{3.696000in}{3.696000in}}%
\pgfusepath{clip}%
\pgfsetbuttcap%
\pgfsetroundjoin%
\definecolor{currentfill}{rgb}{0.121569,0.466667,0.705882}%
\pgfsetfillcolor{currentfill}%
\pgfsetfillopacity{0.671743}%
\pgfsetlinewidth{1.003750pt}%
\definecolor{currentstroke}{rgb}{0.121569,0.466667,0.705882}%
\pgfsetstrokecolor{currentstroke}%
\pgfsetstrokeopacity{0.671743}%
\pgfsetdash{}{0pt}%
\pgfpathmoveto{\pgfqpoint{1.575447in}{2.888322in}}%
\pgfpathcurveto{\pgfqpoint{1.583683in}{2.888322in}}{\pgfqpoint{1.591583in}{2.891594in}}{\pgfqpoint{1.597407in}{2.897418in}}%
\pgfpathcurveto{\pgfqpoint{1.603231in}{2.903242in}}{\pgfqpoint{1.606503in}{2.911142in}}{\pgfqpoint{1.606503in}{2.919379in}}%
\pgfpathcurveto{\pgfqpoint{1.606503in}{2.927615in}}{\pgfqpoint{1.603231in}{2.935515in}}{\pgfqpoint{1.597407in}{2.941339in}}%
\pgfpathcurveto{\pgfqpoint{1.591583in}{2.947163in}}{\pgfqpoint{1.583683in}{2.950435in}}{\pgfqpoint{1.575447in}{2.950435in}}%
\pgfpathcurveto{\pgfqpoint{1.567211in}{2.950435in}}{\pgfqpoint{1.559310in}{2.947163in}}{\pgfqpoint{1.553487in}{2.941339in}}%
\pgfpathcurveto{\pgfqpoint{1.547663in}{2.935515in}}{\pgfqpoint{1.544390in}{2.927615in}}{\pgfqpoint{1.544390in}{2.919379in}}%
\pgfpathcurveto{\pgfqpoint{1.544390in}{2.911142in}}{\pgfqpoint{1.547663in}{2.903242in}}{\pgfqpoint{1.553487in}{2.897418in}}%
\pgfpathcurveto{\pgfqpoint{1.559310in}{2.891594in}}{\pgfqpoint{1.567211in}{2.888322in}}{\pgfqpoint{1.575447in}{2.888322in}}%
\pgfpathclose%
\pgfusepath{stroke,fill}%
\end{pgfscope}%
\begin{pgfscope}%
\pgfpathrectangle{\pgfqpoint{0.100000in}{0.212622in}}{\pgfqpoint{3.696000in}{3.696000in}}%
\pgfusepath{clip}%
\pgfsetbuttcap%
\pgfsetroundjoin%
\definecolor{currentfill}{rgb}{0.121569,0.466667,0.705882}%
\pgfsetfillcolor{currentfill}%
\pgfsetfillopacity{0.672450}%
\pgfsetlinewidth{1.003750pt}%
\definecolor{currentstroke}{rgb}{0.121569,0.466667,0.705882}%
\pgfsetstrokecolor{currentstroke}%
\pgfsetstrokeopacity{0.672450}%
\pgfsetdash{}{0pt}%
\pgfpathmoveto{\pgfqpoint{1.572894in}{2.884689in}}%
\pgfpathcurveto{\pgfqpoint{1.581130in}{2.884689in}}{\pgfqpoint{1.589030in}{2.887961in}}{\pgfqpoint{1.594854in}{2.893785in}}%
\pgfpathcurveto{\pgfqpoint{1.600678in}{2.899609in}}{\pgfqpoint{1.603950in}{2.907509in}}{\pgfqpoint{1.603950in}{2.915745in}}%
\pgfpathcurveto{\pgfqpoint{1.603950in}{2.923981in}}{\pgfqpoint{1.600678in}{2.931882in}}{\pgfqpoint{1.594854in}{2.937705in}}%
\pgfpathcurveto{\pgfqpoint{1.589030in}{2.943529in}}{\pgfqpoint{1.581130in}{2.946802in}}{\pgfqpoint{1.572894in}{2.946802in}}%
\pgfpathcurveto{\pgfqpoint{1.564657in}{2.946802in}}{\pgfqpoint{1.556757in}{2.943529in}}{\pgfqpoint{1.550933in}{2.937705in}}%
\pgfpathcurveto{\pgfqpoint{1.545109in}{2.931882in}}{\pgfqpoint{1.541837in}{2.923981in}}{\pgfqpoint{1.541837in}{2.915745in}}%
\pgfpathcurveto{\pgfqpoint{1.541837in}{2.907509in}}{\pgfqpoint{1.545109in}{2.899609in}}{\pgfqpoint{1.550933in}{2.893785in}}%
\pgfpathcurveto{\pgfqpoint{1.556757in}{2.887961in}}{\pgfqpoint{1.564657in}{2.884689in}}{\pgfqpoint{1.572894in}{2.884689in}}%
\pgfpathclose%
\pgfusepath{stroke,fill}%
\end{pgfscope}%
\begin{pgfscope}%
\pgfpathrectangle{\pgfqpoint{0.100000in}{0.212622in}}{\pgfqpoint{3.696000in}{3.696000in}}%
\pgfusepath{clip}%
\pgfsetbuttcap%
\pgfsetroundjoin%
\definecolor{currentfill}{rgb}{0.121569,0.466667,0.705882}%
\pgfsetfillcolor{currentfill}%
\pgfsetfillopacity{0.672649}%
\pgfsetlinewidth{1.003750pt}%
\definecolor{currentstroke}{rgb}{0.121569,0.466667,0.705882}%
\pgfsetstrokecolor{currentstroke}%
\pgfsetstrokeopacity{0.672649}%
\pgfsetdash{}{0pt}%
\pgfpathmoveto{\pgfqpoint{2.241889in}{3.019647in}}%
\pgfpathcurveto{\pgfqpoint{2.250126in}{3.019647in}}{\pgfqpoint{2.258026in}{3.022920in}}{\pgfqpoint{2.263850in}{3.028743in}}%
\pgfpathcurveto{\pgfqpoint{2.269674in}{3.034567in}}{\pgfqpoint{2.272946in}{3.042467in}}{\pgfqpoint{2.272946in}{3.050704in}}%
\pgfpathcurveto{\pgfqpoint{2.272946in}{3.058940in}}{\pgfqpoint{2.269674in}{3.066840in}}{\pgfqpoint{2.263850in}{3.072664in}}%
\pgfpathcurveto{\pgfqpoint{2.258026in}{3.078488in}}{\pgfqpoint{2.250126in}{3.081760in}}{\pgfqpoint{2.241889in}{3.081760in}}%
\pgfpathcurveto{\pgfqpoint{2.233653in}{3.081760in}}{\pgfqpoint{2.225753in}{3.078488in}}{\pgfqpoint{2.219929in}{3.072664in}}%
\pgfpathcurveto{\pgfqpoint{2.214105in}{3.066840in}}{\pgfqpoint{2.210833in}{3.058940in}}{\pgfqpoint{2.210833in}{3.050704in}}%
\pgfpathcurveto{\pgfqpoint{2.210833in}{3.042467in}}{\pgfqpoint{2.214105in}{3.034567in}}{\pgfqpoint{2.219929in}{3.028743in}}%
\pgfpathcurveto{\pgfqpoint{2.225753in}{3.022920in}}{\pgfqpoint{2.233653in}{3.019647in}}{\pgfqpoint{2.241889in}{3.019647in}}%
\pgfpathclose%
\pgfusepath{stroke,fill}%
\end{pgfscope}%
\begin{pgfscope}%
\pgfpathrectangle{\pgfqpoint{0.100000in}{0.212622in}}{\pgfqpoint{3.696000in}{3.696000in}}%
\pgfusepath{clip}%
\pgfsetbuttcap%
\pgfsetroundjoin%
\definecolor{currentfill}{rgb}{0.121569,0.466667,0.705882}%
\pgfsetfillcolor{currentfill}%
\pgfsetfillopacity{0.672726}%
\pgfsetlinewidth{1.003750pt}%
\definecolor{currentstroke}{rgb}{0.121569,0.466667,0.705882}%
\pgfsetstrokecolor{currentstroke}%
\pgfsetstrokeopacity{0.672726}%
\pgfsetdash{}{0pt}%
\pgfpathmoveto{\pgfqpoint{3.119644in}{1.812348in}}%
\pgfpathcurveto{\pgfqpoint{3.127881in}{1.812348in}}{\pgfqpoint{3.135781in}{1.815621in}}{\pgfqpoint{3.141605in}{1.821445in}}%
\pgfpathcurveto{\pgfqpoint{3.147429in}{1.827269in}}{\pgfqpoint{3.150701in}{1.835169in}}{\pgfqpoint{3.150701in}{1.843405in}}%
\pgfpathcurveto{\pgfqpoint{3.150701in}{1.851641in}}{\pgfqpoint{3.147429in}{1.859541in}}{\pgfqpoint{3.141605in}{1.865365in}}%
\pgfpathcurveto{\pgfqpoint{3.135781in}{1.871189in}}{\pgfqpoint{3.127881in}{1.874461in}}{\pgfqpoint{3.119644in}{1.874461in}}%
\pgfpathcurveto{\pgfqpoint{3.111408in}{1.874461in}}{\pgfqpoint{3.103508in}{1.871189in}}{\pgfqpoint{3.097684in}{1.865365in}}%
\pgfpathcurveto{\pgfqpoint{3.091860in}{1.859541in}}{\pgfqpoint{3.088588in}{1.851641in}}{\pgfqpoint{3.088588in}{1.843405in}}%
\pgfpathcurveto{\pgfqpoint{3.088588in}{1.835169in}}{\pgfqpoint{3.091860in}{1.827269in}}{\pgfqpoint{3.097684in}{1.821445in}}%
\pgfpathcurveto{\pgfqpoint{3.103508in}{1.815621in}}{\pgfqpoint{3.111408in}{1.812348in}}{\pgfqpoint{3.119644in}{1.812348in}}%
\pgfpathclose%
\pgfusepath{stroke,fill}%
\end{pgfscope}%
\begin{pgfscope}%
\pgfpathrectangle{\pgfqpoint{0.100000in}{0.212622in}}{\pgfqpoint{3.696000in}{3.696000in}}%
\pgfusepath{clip}%
\pgfsetbuttcap%
\pgfsetroundjoin%
\definecolor{currentfill}{rgb}{0.121569,0.466667,0.705882}%
\pgfsetfillcolor{currentfill}%
\pgfsetfillopacity{0.672855}%
\pgfsetlinewidth{1.003750pt}%
\definecolor{currentstroke}{rgb}{0.121569,0.466667,0.705882}%
\pgfsetstrokecolor{currentstroke}%
\pgfsetstrokeopacity{0.672855}%
\pgfsetdash{}{0pt}%
\pgfpathmoveto{\pgfqpoint{1.571498in}{2.882774in}}%
\pgfpathcurveto{\pgfqpoint{1.579735in}{2.882774in}}{\pgfqpoint{1.587635in}{2.886046in}}{\pgfqpoint{1.593459in}{2.891870in}}%
\pgfpathcurveto{\pgfqpoint{1.599283in}{2.897694in}}{\pgfqpoint{1.602555in}{2.905594in}}{\pgfqpoint{1.602555in}{2.913830in}}%
\pgfpathcurveto{\pgfqpoint{1.602555in}{2.922066in}}{\pgfqpoint{1.599283in}{2.929966in}}{\pgfqpoint{1.593459in}{2.935790in}}%
\pgfpathcurveto{\pgfqpoint{1.587635in}{2.941614in}}{\pgfqpoint{1.579735in}{2.944887in}}{\pgfqpoint{1.571498in}{2.944887in}}%
\pgfpathcurveto{\pgfqpoint{1.563262in}{2.944887in}}{\pgfqpoint{1.555362in}{2.941614in}}{\pgfqpoint{1.549538in}{2.935790in}}%
\pgfpathcurveto{\pgfqpoint{1.543714in}{2.929966in}}{\pgfqpoint{1.540442in}{2.922066in}}{\pgfqpoint{1.540442in}{2.913830in}}%
\pgfpathcurveto{\pgfqpoint{1.540442in}{2.905594in}}{\pgfqpoint{1.543714in}{2.897694in}}{\pgfqpoint{1.549538in}{2.891870in}}%
\pgfpathcurveto{\pgfqpoint{1.555362in}{2.886046in}}{\pgfqpoint{1.563262in}{2.882774in}}{\pgfqpoint{1.571498in}{2.882774in}}%
\pgfpathclose%
\pgfusepath{stroke,fill}%
\end{pgfscope}%
\begin{pgfscope}%
\pgfpathrectangle{\pgfqpoint{0.100000in}{0.212622in}}{\pgfqpoint{3.696000in}{3.696000in}}%
\pgfusepath{clip}%
\pgfsetbuttcap%
\pgfsetroundjoin%
\definecolor{currentfill}{rgb}{0.121569,0.466667,0.705882}%
\pgfsetfillcolor{currentfill}%
\pgfsetfillopacity{0.673519}%
\pgfsetlinewidth{1.003750pt}%
\definecolor{currentstroke}{rgb}{0.121569,0.466667,0.705882}%
\pgfsetstrokecolor{currentstroke}%
\pgfsetstrokeopacity{0.673519}%
\pgfsetdash{}{0pt}%
\pgfpathmoveto{\pgfqpoint{1.569367in}{2.880106in}}%
\pgfpathcurveto{\pgfqpoint{1.577604in}{2.880106in}}{\pgfqpoint{1.585504in}{2.883379in}}{\pgfqpoint{1.591328in}{2.889202in}}%
\pgfpathcurveto{\pgfqpoint{1.597152in}{2.895026in}}{\pgfqpoint{1.600424in}{2.902926in}}{\pgfqpoint{1.600424in}{2.911163in}}%
\pgfpathcurveto{\pgfqpoint{1.600424in}{2.919399in}}{\pgfqpoint{1.597152in}{2.927299in}}{\pgfqpoint{1.591328in}{2.933123in}}%
\pgfpathcurveto{\pgfqpoint{1.585504in}{2.938947in}}{\pgfqpoint{1.577604in}{2.942219in}}{\pgfqpoint{1.569367in}{2.942219in}}%
\pgfpathcurveto{\pgfqpoint{1.561131in}{2.942219in}}{\pgfqpoint{1.553231in}{2.938947in}}{\pgfqpoint{1.547407in}{2.933123in}}%
\pgfpathcurveto{\pgfqpoint{1.541583in}{2.927299in}}{\pgfqpoint{1.538311in}{2.919399in}}{\pgfqpoint{1.538311in}{2.911163in}}%
\pgfpathcurveto{\pgfqpoint{1.538311in}{2.902926in}}{\pgfqpoint{1.541583in}{2.895026in}}{\pgfqpoint{1.547407in}{2.889202in}}%
\pgfpathcurveto{\pgfqpoint{1.553231in}{2.883379in}}{\pgfqpoint{1.561131in}{2.880106in}}{\pgfqpoint{1.569367in}{2.880106in}}%
\pgfpathclose%
\pgfusepath{stroke,fill}%
\end{pgfscope}%
\begin{pgfscope}%
\pgfpathrectangle{\pgfqpoint{0.100000in}{0.212622in}}{\pgfqpoint{3.696000in}{3.696000in}}%
\pgfusepath{clip}%
\pgfsetbuttcap%
\pgfsetroundjoin%
\definecolor{currentfill}{rgb}{0.121569,0.466667,0.705882}%
\pgfsetfillcolor{currentfill}%
\pgfsetfillopacity{0.673617}%
\pgfsetlinewidth{1.003750pt}%
\definecolor{currentstroke}{rgb}{0.121569,0.466667,0.705882}%
\pgfsetstrokecolor{currentstroke}%
\pgfsetstrokeopacity{0.673617}%
\pgfsetdash{}{0pt}%
\pgfpathmoveto{\pgfqpoint{3.117909in}{1.810346in}}%
\pgfpathcurveto{\pgfqpoint{3.126145in}{1.810346in}}{\pgfqpoint{3.134045in}{1.813619in}}{\pgfqpoint{3.139869in}{1.819443in}}%
\pgfpathcurveto{\pgfqpoint{3.145693in}{1.825267in}}{\pgfqpoint{3.148965in}{1.833167in}}{\pgfqpoint{3.148965in}{1.841403in}}%
\pgfpathcurveto{\pgfqpoint{3.148965in}{1.849639in}}{\pgfqpoint{3.145693in}{1.857539in}}{\pgfqpoint{3.139869in}{1.863363in}}%
\pgfpathcurveto{\pgfqpoint{3.134045in}{1.869187in}}{\pgfqpoint{3.126145in}{1.872459in}}{\pgfqpoint{3.117909in}{1.872459in}}%
\pgfpathcurveto{\pgfqpoint{3.109673in}{1.872459in}}{\pgfqpoint{3.101773in}{1.869187in}}{\pgfqpoint{3.095949in}{1.863363in}}%
\pgfpathcurveto{\pgfqpoint{3.090125in}{1.857539in}}{\pgfqpoint{3.086852in}{1.849639in}}{\pgfqpoint{3.086852in}{1.841403in}}%
\pgfpathcurveto{\pgfqpoint{3.086852in}{1.833167in}}{\pgfqpoint{3.090125in}{1.825267in}}{\pgfqpoint{3.095949in}{1.819443in}}%
\pgfpathcurveto{\pgfqpoint{3.101773in}{1.813619in}}{\pgfqpoint{3.109673in}{1.810346in}}{\pgfqpoint{3.117909in}{1.810346in}}%
\pgfpathclose%
\pgfusepath{stroke,fill}%
\end{pgfscope}%
\begin{pgfscope}%
\pgfpathrectangle{\pgfqpoint{0.100000in}{0.212622in}}{\pgfqpoint{3.696000in}{3.696000in}}%
\pgfusepath{clip}%
\pgfsetbuttcap%
\pgfsetroundjoin%
\definecolor{currentfill}{rgb}{0.121569,0.466667,0.705882}%
\pgfsetfillcolor{currentfill}%
\pgfsetfillopacity{0.673880}%
\pgfsetlinewidth{1.003750pt}%
\definecolor{currentstroke}{rgb}{0.121569,0.466667,0.705882}%
\pgfsetstrokecolor{currentstroke}%
\pgfsetstrokeopacity{0.673880}%
\pgfsetdash{}{0pt}%
\pgfpathmoveto{\pgfqpoint{1.568196in}{2.878609in}}%
\pgfpathcurveto{\pgfqpoint{1.576433in}{2.878609in}}{\pgfqpoint{1.584333in}{2.881881in}}{\pgfqpoint{1.590156in}{2.887705in}}%
\pgfpathcurveto{\pgfqpoint{1.595980in}{2.893529in}}{\pgfqpoint{1.599253in}{2.901429in}}{\pgfqpoint{1.599253in}{2.909665in}}%
\pgfpathcurveto{\pgfqpoint{1.599253in}{2.917902in}}{\pgfqpoint{1.595980in}{2.925802in}}{\pgfqpoint{1.590156in}{2.931626in}}%
\pgfpathcurveto{\pgfqpoint{1.584333in}{2.937450in}}{\pgfqpoint{1.576433in}{2.940722in}}{\pgfqpoint{1.568196in}{2.940722in}}%
\pgfpathcurveto{\pgfqpoint{1.559960in}{2.940722in}}{\pgfqpoint{1.552060in}{2.937450in}}{\pgfqpoint{1.546236in}{2.931626in}}%
\pgfpathcurveto{\pgfqpoint{1.540412in}{2.925802in}}{\pgfqpoint{1.537140in}{2.917902in}}{\pgfqpoint{1.537140in}{2.909665in}}%
\pgfpathcurveto{\pgfqpoint{1.537140in}{2.901429in}}{\pgfqpoint{1.540412in}{2.893529in}}{\pgfqpoint{1.546236in}{2.887705in}}%
\pgfpathcurveto{\pgfqpoint{1.552060in}{2.881881in}}{\pgfqpoint{1.559960in}{2.878609in}}{\pgfqpoint{1.568196in}{2.878609in}}%
\pgfpathclose%
\pgfusepath{stroke,fill}%
\end{pgfscope}%
\begin{pgfscope}%
\pgfpathrectangle{\pgfqpoint{0.100000in}{0.212622in}}{\pgfqpoint{3.696000in}{3.696000in}}%
\pgfusepath{clip}%
\pgfsetbuttcap%
\pgfsetroundjoin%
\definecolor{currentfill}{rgb}{0.121569,0.466667,0.705882}%
\pgfsetfillcolor{currentfill}%
\pgfsetfillopacity{0.673935}%
\pgfsetlinewidth{1.003750pt}%
\definecolor{currentstroke}{rgb}{0.121569,0.466667,0.705882}%
\pgfsetstrokecolor{currentstroke}%
\pgfsetstrokeopacity{0.673935}%
\pgfsetdash{}{0pt}%
\pgfpathmoveto{\pgfqpoint{2.251617in}{3.018309in}}%
\pgfpathcurveto{\pgfqpoint{2.259853in}{3.018309in}}{\pgfqpoint{2.267753in}{3.021581in}}{\pgfqpoint{2.273577in}{3.027405in}}%
\pgfpathcurveto{\pgfqpoint{2.279401in}{3.033229in}}{\pgfqpoint{2.282673in}{3.041129in}}{\pgfqpoint{2.282673in}{3.049365in}}%
\pgfpathcurveto{\pgfqpoint{2.282673in}{3.057602in}}{\pgfqpoint{2.279401in}{3.065502in}}{\pgfqpoint{2.273577in}{3.071326in}}%
\pgfpathcurveto{\pgfqpoint{2.267753in}{3.077150in}}{\pgfqpoint{2.259853in}{3.080422in}}{\pgfqpoint{2.251617in}{3.080422in}}%
\pgfpathcurveto{\pgfqpoint{2.243380in}{3.080422in}}{\pgfqpoint{2.235480in}{3.077150in}}{\pgfqpoint{2.229656in}{3.071326in}}%
\pgfpathcurveto{\pgfqpoint{2.223832in}{3.065502in}}{\pgfqpoint{2.220560in}{3.057602in}}{\pgfqpoint{2.220560in}{3.049365in}}%
\pgfpathcurveto{\pgfqpoint{2.220560in}{3.041129in}}{\pgfqpoint{2.223832in}{3.033229in}}{\pgfqpoint{2.229656in}{3.027405in}}%
\pgfpathcurveto{\pgfqpoint{2.235480in}{3.021581in}}{\pgfqpoint{2.243380in}{3.018309in}}{\pgfqpoint{2.251617in}{3.018309in}}%
\pgfpathclose%
\pgfusepath{stroke,fill}%
\end{pgfscope}%
\begin{pgfscope}%
\pgfpathrectangle{\pgfqpoint{0.100000in}{0.212622in}}{\pgfqpoint{3.696000in}{3.696000in}}%
\pgfusepath{clip}%
\pgfsetbuttcap%
\pgfsetroundjoin%
\definecolor{currentfill}{rgb}{0.121569,0.466667,0.705882}%
\pgfsetfillcolor{currentfill}%
\pgfsetfillopacity{0.674468}%
\pgfsetlinewidth{1.003750pt}%
\definecolor{currentstroke}{rgb}{0.121569,0.466667,0.705882}%
\pgfsetstrokecolor{currentstroke}%
\pgfsetstrokeopacity{0.674468}%
\pgfsetdash{}{0pt}%
\pgfpathmoveto{\pgfqpoint{1.566132in}{2.875760in}}%
\pgfpathcurveto{\pgfqpoint{1.574368in}{2.875760in}}{\pgfqpoint{1.582268in}{2.879033in}}{\pgfqpoint{1.588092in}{2.884857in}}%
\pgfpathcurveto{\pgfqpoint{1.593916in}{2.890681in}}{\pgfqpoint{1.597189in}{2.898581in}}{\pgfqpoint{1.597189in}{2.906817in}}%
\pgfpathcurveto{\pgfqpoint{1.597189in}{2.915053in}}{\pgfqpoint{1.593916in}{2.922953in}}{\pgfqpoint{1.588092in}{2.928777in}}%
\pgfpathcurveto{\pgfqpoint{1.582268in}{2.934601in}}{\pgfqpoint{1.574368in}{2.937873in}}{\pgfqpoint{1.566132in}{2.937873in}}%
\pgfpathcurveto{\pgfqpoint{1.557896in}{2.937873in}}{\pgfqpoint{1.549996in}{2.934601in}}{\pgfqpoint{1.544172in}{2.928777in}}%
\pgfpathcurveto{\pgfqpoint{1.538348in}{2.922953in}}{\pgfqpoint{1.535076in}{2.915053in}}{\pgfqpoint{1.535076in}{2.906817in}}%
\pgfpathcurveto{\pgfqpoint{1.535076in}{2.898581in}}{\pgfqpoint{1.538348in}{2.890681in}}{\pgfqpoint{1.544172in}{2.884857in}}%
\pgfpathcurveto{\pgfqpoint{1.549996in}{2.879033in}}{\pgfqpoint{1.557896in}{2.875760in}}{\pgfqpoint{1.566132in}{2.875760in}}%
\pgfpathclose%
\pgfusepath{stroke,fill}%
\end{pgfscope}%
\begin{pgfscope}%
\pgfpathrectangle{\pgfqpoint{0.100000in}{0.212622in}}{\pgfqpoint{3.696000in}{3.696000in}}%
\pgfusepath{clip}%
\pgfsetbuttcap%
\pgfsetroundjoin%
\definecolor{currentfill}{rgb}{0.121569,0.466667,0.705882}%
\pgfsetfillcolor{currentfill}%
\pgfsetfillopacity{0.675184}%
\pgfsetlinewidth{1.003750pt}%
\definecolor{currentstroke}{rgb}{0.121569,0.466667,0.705882}%
\pgfsetstrokecolor{currentstroke}%
\pgfsetstrokeopacity{0.675184}%
\pgfsetdash{}{0pt}%
\pgfpathmoveto{\pgfqpoint{1.563598in}{2.872140in}}%
\pgfpathcurveto{\pgfqpoint{1.571834in}{2.872140in}}{\pgfqpoint{1.579734in}{2.875413in}}{\pgfqpoint{1.585558in}{2.881237in}}%
\pgfpathcurveto{\pgfqpoint{1.591382in}{2.887061in}}{\pgfqpoint{1.594654in}{2.894961in}}{\pgfqpoint{1.594654in}{2.903197in}}%
\pgfpathcurveto{\pgfqpoint{1.594654in}{2.911433in}}{\pgfqpoint{1.591382in}{2.919333in}}{\pgfqpoint{1.585558in}{2.925157in}}%
\pgfpathcurveto{\pgfqpoint{1.579734in}{2.930981in}}{\pgfqpoint{1.571834in}{2.934253in}}{\pgfqpoint{1.563598in}{2.934253in}}%
\pgfpathcurveto{\pgfqpoint{1.555361in}{2.934253in}}{\pgfqpoint{1.547461in}{2.930981in}}{\pgfqpoint{1.541637in}{2.925157in}}%
\pgfpathcurveto{\pgfqpoint{1.535813in}{2.919333in}}{\pgfqpoint{1.532541in}{2.911433in}}{\pgfqpoint{1.532541in}{2.903197in}}%
\pgfpathcurveto{\pgfqpoint{1.532541in}{2.894961in}}{\pgfqpoint{1.535813in}{2.887061in}}{\pgfqpoint{1.541637in}{2.881237in}}%
\pgfpathcurveto{\pgfqpoint{1.547461in}{2.875413in}}{\pgfqpoint{1.555361in}{2.872140in}}{\pgfqpoint{1.563598in}{2.872140in}}%
\pgfpathclose%
\pgfusepath{stroke,fill}%
\end{pgfscope}%
\begin{pgfscope}%
\pgfpathrectangle{\pgfqpoint{0.100000in}{0.212622in}}{\pgfqpoint{3.696000in}{3.696000in}}%
\pgfusepath{clip}%
\pgfsetbuttcap%
\pgfsetroundjoin%
\definecolor{currentfill}{rgb}{0.121569,0.466667,0.705882}%
\pgfsetfillcolor{currentfill}%
\pgfsetfillopacity{0.675196}%
\pgfsetlinewidth{1.003750pt}%
\definecolor{currentstroke}{rgb}{0.121569,0.466667,0.705882}%
\pgfsetstrokecolor{currentstroke}%
\pgfsetstrokeopacity{0.675196}%
\pgfsetdash{}{0pt}%
\pgfpathmoveto{\pgfqpoint{2.261026in}{3.017271in}}%
\pgfpathcurveto{\pgfqpoint{2.269263in}{3.017271in}}{\pgfqpoint{2.277163in}{3.020544in}}{\pgfqpoint{2.282987in}{3.026368in}}%
\pgfpathcurveto{\pgfqpoint{2.288811in}{3.032191in}}{\pgfqpoint{2.292083in}{3.040092in}}{\pgfqpoint{2.292083in}{3.048328in}}%
\pgfpathcurveto{\pgfqpoint{2.292083in}{3.056564in}}{\pgfqpoint{2.288811in}{3.064464in}}{\pgfqpoint{2.282987in}{3.070288in}}%
\pgfpathcurveto{\pgfqpoint{2.277163in}{3.076112in}}{\pgfqpoint{2.269263in}{3.079384in}}{\pgfqpoint{2.261026in}{3.079384in}}%
\pgfpathcurveto{\pgfqpoint{2.252790in}{3.079384in}}{\pgfqpoint{2.244890in}{3.076112in}}{\pgfqpoint{2.239066in}{3.070288in}}%
\pgfpathcurveto{\pgfqpoint{2.233242in}{3.064464in}}{\pgfqpoint{2.229970in}{3.056564in}}{\pgfqpoint{2.229970in}{3.048328in}}%
\pgfpathcurveto{\pgfqpoint{2.229970in}{3.040092in}}{\pgfqpoint{2.233242in}{3.032191in}}{\pgfqpoint{2.239066in}{3.026368in}}%
\pgfpathcurveto{\pgfqpoint{2.244890in}{3.020544in}}{\pgfqpoint{2.252790in}{3.017271in}}{\pgfqpoint{2.261026in}{3.017271in}}%
\pgfpathclose%
\pgfusepath{stroke,fill}%
\end{pgfscope}%
\begin{pgfscope}%
\pgfpathrectangle{\pgfqpoint{0.100000in}{0.212622in}}{\pgfqpoint{3.696000in}{3.696000in}}%
\pgfusepath{clip}%
\pgfsetbuttcap%
\pgfsetroundjoin%
\definecolor{currentfill}{rgb}{0.121569,0.466667,0.705882}%
\pgfsetfillcolor{currentfill}%
\pgfsetfillopacity{0.675244}%
\pgfsetlinewidth{1.003750pt}%
\definecolor{currentstroke}{rgb}{0.121569,0.466667,0.705882}%
\pgfsetstrokecolor{currentstroke}%
\pgfsetstrokeopacity{0.675244}%
\pgfsetdash{}{0pt}%
\pgfpathmoveto{\pgfqpoint{3.114800in}{1.806700in}}%
\pgfpathcurveto{\pgfqpoint{3.123036in}{1.806700in}}{\pgfqpoint{3.130936in}{1.809972in}}{\pgfqpoint{3.136760in}{1.815796in}}%
\pgfpathcurveto{\pgfqpoint{3.142584in}{1.821620in}}{\pgfqpoint{3.145857in}{1.829520in}}{\pgfqpoint{3.145857in}{1.837757in}}%
\pgfpathcurveto{\pgfqpoint{3.145857in}{1.845993in}}{\pgfqpoint{3.142584in}{1.853893in}}{\pgfqpoint{3.136760in}{1.859717in}}%
\pgfpathcurveto{\pgfqpoint{3.130936in}{1.865541in}}{\pgfqpoint{3.123036in}{1.868813in}}{\pgfqpoint{3.114800in}{1.868813in}}%
\pgfpathcurveto{\pgfqpoint{3.106564in}{1.868813in}}{\pgfqpoint{3.098664in}{1.865541in}}{\pgfqpoint{3.092840in}{1.859717in}}%
\pgfpathcurveto{\pgfqpoint{3.087016in}{1.853893in}}{\pgfqpoint{3.083744in}{1.845993in}}{\pgfqpoint{3.083744in}{1.837757in}}%
\pgfpathcurveto{\pgfqpoint{3.083744in}{1.829520in}}{\pgfqpoint{3.087016in}{1.821620in}}{\pgfqpoint{3.092840in}{1.815796in}}%
\pgfpathcurveto{\pgfqpoint{3.098664in}{1.809972in}}{\pgfqpoint{3.106564in}{1.806700in}}{\pgfqpoint{3.114800in}{1.806700in}}%
\pgfpathclose%
\pgfusepath{stroke,fill}%
\end{pgfscope}%
\begin{pgfscope}%
\pgfpathrectangle{\pgfqpoint{0.100000in}{0.212622in}}{\pgfqpoint{3.696000in}{3.696000in}}%
\pgfusepath{clip}%
\pgfsetbuttcap%
\pgfsetroundjoin%
\definecolor{currentfill}{rgb}{0.121569,0.466667,0.705882}%
\pgfsetfillcolor{currentfill}%
\pgfsetfillopacity{0.676082}%
\pgfsetlinewidth{1.003750pt}%
\definecolor{currentstroke}{rgb}{0.121569,0.466667,0.705882}%
\pgfsetstrokecolor{currentstroke}%
\pgfsetstrokeopacity{0.676082}%
\pgfsetdash{}{0pt}%
\pgfpathmoveto{\pgfqpoint{1.560635in}{2.868146in}}%
\pgfpathcurveto{\pgfqpoint{1.568872in}{2.868146in}}{\pgfqpoint{1.576772in}{2.871418in}}{\pgfqpoint{1.582596in}{2.877242in}}%
\pgfpathcurveto{\pgfqpoint{1.588420in}{2.883066in}}{\pgfqpoint{1.591692in}{2.890966in}}{\pgfqpoint{1.591692in}{2.899203in}}%
\pgfpathcurveto{\pgfqpoint{1.591692in}{2.907439in}}{\pgfqpoint{1.588420in}{2.915339in}}{\pgfqpoint{1.582596in}{2.921163in}}%
\pgfpathcurveto{\pgfqpoint{1.576772in}{2.926987in}}{\pgfqpoint{1.568872in}{2.930259in}}{\pgfqpoint{1.560635in}{2.930259in}}%
\pgfpathcurveto{\pgfqpoint{1.552399in}{2.930259in}}{\pgfqpoint{1.544499in}{2.926987in}}{\pgfqpoint{1.538675in}{2.921163in}}%
\pgfpathcurveto{\pgfqpoint{1.532851in}{2.915339in}}{\pgfqpoint{1.529579in}{2.907439in}}{\pgfqpoint{1.529579in}{2.899203in}}%
\pgfpathcurveto{\pgfqpoint{1.529579in}{2.890966in}}{\pgfqpoint{1.532851in}{2.883066in}}{\pgfqpoint{1.538675in}{2.877242in}}%
\pgfpathcurveto{\pgfqpoint{1.544499in}{2.871418in}}{\pgfqpoint{1.552399in}{2.868146in}}{\pgfqpoint{1.560635in}{2.868146in}}%
\pgfpathclose%
\pgfusepath{stroke,fill}%
\end{pgfscope}%
\begin{pgfscope}%
\pgfpathrectangle{\pgfqpoint{0.100000in}{0.212622in}}{\pgfqpoint{3.696000in}{3.696000in}}%
\pgfusepath{clip}%
\pgfsetbuttcap%
\pgfsetroundjoin%
\definecolor{currentfill}{rgb}{0.121569,0.466667,0.705882}%
\pgfsetfillcolor{currentfill}%
\pgfsetfillopacity{0.676284}%
\pgfsetlinewidth{1.003750pt}%
\definecolor{currentstroke}{rgb}{0.121569,0.466667,0.705882}%
\pgfsetstrokecolor{currentstroke}%
\pgfsetstrokeopacity{0.676284}%
\pgfsetdash{}{0pt}%
\pgfpathmoveto{\pgfqpoint{2.269578in}{3.015438in}}%
\pgfpathcurveto{\pgfqpoint{2.277814in}{3.015438in}}{\pgfqpoint{2.285715in}{3.018710in}}{\pgfqpoint{2.291538in}{3.024534in}}%
\pgfpathcurveto{\pgfqpoint{2.297362in}{3.030358in}}{\pgfqpoint{2.300635in}{3.038258in}}{\pgfqpoint{2.300635in}{3.046495in}}%
\pgfpathcurveto{\pgfqpoint{2.300635in}{3.054731in}}{\pgfqpoint{2.297362in}{3.062631in}}{\pgfqpoint{2.291538in}{3.068455in}}%
\pgfpathcurveto{\pgfqpoint{2.285715in}{3.074279in}}{\pgfqpoint{2.277814in}{3.077551in}}{\pgfqpoint{2.269578in}{3.077551in}}%
\pgfpathcurveto{\pgfqpoint{2.261342in}{3.077551in}}{\pgfqpoint{2.253442in}{3.074279in}}{\pgfqpoint{2.247618in}{3.068455in}}%
\pgfpathcurveto{\pgfqpoint{2.241794in}{3.062631in}}{\pgfqpoint{2.238522in}{3.054731in}}{\pgfqpoint{2.238522in}{3.046495in}}%
\pgfpathcurveto{\pgfqpoint{2.238522in}{3.038258in}}{\pgfqpoint{2.241794in}{3.030358in}}{\pgfqpoint{2.247618in}{3.024534in}}%
\pgfpathcurveto{\pgfqpoint{2.253442in}{3.018710in}}{\pgfqpoint{2.261342in}{3.015438in}}{\pgfqpoint{2.269578in}{3.015438in}}%
\pgfpathclose%
\pgfusepath{stroke,fill}%
\end{pgfscope}%
\begin{pgfscope}%
\pgfpathrectangle{\pgfqpoint{0.100000in}{0.212622in}}{\pgfqpoint{3.696000in}{3.696000in}}%
\pgfusepath{clip}%
\pgfsetbuttcap%
\pgfsetroundjoin%
\definecolor{currentfill}{rgb}{0.121569,0.466667,0.705882}%
\pgfsetfillcolor{currentfill}%
\pgfsetfillopacity{0.676476}%
\pgfsetlinewidth{1.003750pt}%
\definecolor{currentstroke}{rgb}{0.121569,0.466667,0.705882}%
\pgfsetstrokecolor{currentstroke}%
\pgfsetstrokeopacity{0.676476}%
\pgfsetdash{}{0pt}%
\pgfpathmoveto{\pgfqpoint{3.112306in}{1.803495in}}%
\pgfpathcurveto{\pgfqpoint{3.120543in}{1.803495in}}{\pgfqpoint{3.128443in}{1.806767in}}{\pgfqpoint{3.134267in}{1.812591in}}%
\pgfpathcurveto{\pgfqpoint{3.140091in}{1.818415in}}{\pgfqpoint{3.143363in}{1.826315in}}{\pgfqpoint{3.143363in}{1.834551in}}%
\pgfpathcurveto{\pgfqpoint{3.143363in}{1.842787in}}{\pgfqpoint{3.140091in}{1.850688in}}{\pgfqpoint{3.134267in}{1.856511in}}%
\pgfpathcurveto{\pgfqpoint{3.128443in}{1.862335in}}{\pgfqpoint{3.120543in}{1.865608in}}{\pgfqpoint{3.112306in}{1.865608in}}%
\pgfpathcurveto{\pgfqpoint{3.104070in}{1.865608in}}{\pgfqpoint{3.096170in}{1.862335in}}{\pgfqpoint{3.090346in}{1.856511in}}%
\pgfpathcurveto{\pgfqpoint{3.084522in}{1.850688in}}{\pgfqpoint{3.081250in}{1.842787in}}{\pgfqpoint{3.081250in}{1.834551in}}%
\pgfpathcurveto{\pgfqpoint{3.081250in}{1.826315in}}{\pgfqpoint{3.084522in}{1.818415in}}{\pgfqpoint{3.090346in}{1.812591in}}%
\pgfpathcurveto{\pgfqpoint{3.096170in}{1.806767in}}{\pgfqpoint{3.104070in}{1.803495in}}{\pgfqpoint{3.112306in}{1.803495in}}%
\pgfpathclose%
\pgfusepath{stroke,fill}%
\end{pgfscope}%
\begin{pgfscope}%
\pgfpathrectangle{\pgfqpoint{0.100000in}{0.212622in}}{\pgfqpoint{3.696000in}{3.696000in}}%
\pgfusepath{clip}%
\pgfsetbuttcap%
\pgfsetroundjoin%
\definecolor{currentfill}{rgb}{0.121569,0.466667,0.705882}%
\pgfsetfillcolor{currentfill}%
\pgfsetfillopacity{0.676629}%
\pgfsetlinewidth{1.003750pt}%
\definecolor{currentstroke}{rgb}{0.121569,0.466667,0.705882}%
\pgfsetstrokecolor{currentstroke}%
\pgfsetstrokeopacity{0.676629}%
\pgfsetdash{}{0pt}%
\pgfpathmoveto{\pgfqpoint{1.559017in}{2.866254in}}%
\pgfpathcurveto{\pgfqpoint{1.567253in}{2.866254in}}{\pgfqpoint{1.575153in}{2.869527in}}{\pgfqpoint{1.580977in}{2.875350in}}%
\pgfpathcurveto{\pgfqpoint{1.586801in}{2.881174in}}{\pgfqpoint{1.590073in}{2.889074in}}{\pgfqpoint{1.590073in}{2.897311in}}%
\pgfpathcurveto{\pgfqpoint{1.590073in}{2.905547in}}{\pgfqpoint{1.586801in}{2.913447in}}{\pgfqpoint{1.580977in}{2.919271in}}%
\pgfpathcurveto{\pgfqpoint{1.575153in}{2.925095in}}{\pgfqpoint{1.567253in}{2.928367in}}{\pgfqpoint{1.559017in}{2.928367in}}%
\pgfpathcurveto{\pgfqpoint{1.550781in}{2.928367in}}{\pgfqpoint{1.542881in}{2.925095in}}{\pgfqpoint{1.537057in}{2.919271in}}%
\pgfpathcurveto{\pgfqpoint{1.531233in}{2.913447in}}{\pgfqpoint{1.527960in}{2.905547in}}{\pgfqpoint{1.527960in}{2.897311in}}%
\pgfpathcurveto{\pgfqpoint{1.527960in}{2.889074in}}{\pgfqpoint{1.531233in}{2.881174in}}{\pgfqpoint{1.537057in}{2.875350in}}%
\pgfpathcurveto{\pgfqpoint{1.542881in}{2.869527in}}{\pgfqpoint{1.550781in}{2.866254in}}{\pgfqpoint{1.559017in}{2.866254in}}%
\pgfpathclose%
\pgfusepath{stroke,fill}%
\end{pgfscope}%
\begin{pgfscope}%
\pgfpathrectangle{\pgfqpoint{0.100000in}{0.212622in}}{\pgfqpoint{3.696000in}{3.696000in}}%
\pgfusepath{clip}%
\pgfsetbuttcap%
\pgfsetroundjoin%
\definecolor{currentfill}{rgb}{0.121569,0.466667,0.705882}%
\pgfsetfillcolor{currentfill}%
\pgfsetfillopacity{0.677261}%
\pgfsetlinewidth{1.003750pt}%
\definecolor{currentstroke}{rgb}{0.121569,0.466667,0.705882}%
\pgfsetstrokecolor{currentstroke}%
\pgfsetstrokeopacity{0.677261}%
\pgfsetdash{}{0pt}%
\pgfpathmoveto{\pgfqpoint{2.277474in}{3.013587in}}%
\pgfpathcurveto{\pgfqpoint{2.285710in}{3.013587in}}{\pgfqpoint{2.293610in}{3.016859in}}{\pgfqpoint{2.299434in}{3.022683in}}%
\pgfpathcurveto{\pgfqpoint{2.305258in}{3.028507in}}{\pgfqpoint{2.308530in}{3.036407in}}{\pgfqpoint{2.308530in}{3.044643in}}%
\pgfpathcurveto{\pgfqpoint{2.308530in}{3.052880in}}{\pgfqpoint{2.305258in}{3.060780in}}{\pgfqpoint{2.299434in}{3.066604in}}%
\pgfpathcurveto{\pgfqpoint{2.293610in}{3.072428in}}{\pgfqpoint{2.285710in}{3.075700in}}{\pgfqpoint{2.277474in}{3.075700in}}%
\pgfpathcurveto{\pgfqpoint{2.269237in}{3.075700in}}{\pgfqpoint{2.261337in}{3.072428in}}{\pgfqpoint{2.255513in}{3.066604in}}%
\pgfpathcurveto{\pgfqpoint{2.249689in}{3.060780in}}{\pgfqpoint{2.246417in}{3.052880in}}{\pgfqpoint{2.246417in}{3.044643in}}%
\pgfpathcurveto{\pgfqpoint{2.246417in}{3.036407in}}{\pgfqpoint{2.249689in}{3.028507in}}{\pgfqpoint{2.255513in}{3.022683in}}%
\pgfpathcurveto{\pgfqpoint{2.261337in}{3.016859in}}{\pgfqpoint{2.269237in}{3.013587in}}{\pgfqpoint{2.277474in}{3.013587in}}%
\pgfpathclose%
\pgfusepath{stroke,fill}%
\end{pgfscope}%
\begin{pgfscope}%
\pgfpathrectangle{\pgfqpoint{0.100000in}{0.212622in}}{\pgfqpoint{3.696000in}{3.696000in}}%
\pgfusepath{clip}%
\pgfsetbuttcap%
\pgfsetroundjoin%
\definecolor{currentfill}{rgb}{0.121569,0.466667,0.705882}%
\pgfsetfillcolor{currentfill}%
\pgfsetfillopacity{0.677282}%
\pgfsetlinewidth{1.003750pt}%
\definecolor{currentstroke}{rgb}{0.121569,0.466667,0.705882}%
\pgfsetstrokecolor{currentstroke}%
\pgfsetstrokeopacity{0.677282}%
\pgfsetdash{}{0pt}%
\pgfpathmoveto{\pgfqpoint{1.556909in}{2.863631in}}%
\pgfpathcurveto{\pgfqpoint{1.565145in}{2.863631in}}{\pgfqpoint{1.573046in}{2.866903in}}{\pgfqpoint{1.578869in}{2.872727in}}%
\pgfpathcurveto{\pgfqpoint{1.584693in}{2.878551in}}{\pgfqpoint{1.587966in}{2.886451in}}{\pgfqpoint{1.587966in}{2.894687in}}%
\pgfpathcurveto{\pgfqpoint{1.587966in}{2.902924in}}{\pgfqpoint{1.584693in}{2.910824in}}{\pgfqpoint{1.578869in}{2.916648in}}%
\pgfpathcurveto{\pgfqpoint{1.573046in}{2.922472in}}{\pgfqpoint{1.565145in}{2.925744in}}{\pgfqpoint{1.556909in}{2.925744in}}%
\pgfpathcurveto{\pgfqpoint{1.548673in}{2.925744in}}{\pgfqpoint{1.540773in}{2.922472in}}{\pgfqpoint{1.534949in}{2.916648in}}%
\pgfpathcurveto{\pgfqpoint{1.529125in}{2.910824in}}{\pgfqpoint{1.525853in}{2.902924in}}{\pgfqpoint{1.525853in}{2.894687in}}%
\pgfpathcurveto{\pgfqpoint{1.525853in}{2.886451in}}{\pgfqpoint{1.529125in}{2.878551in}}{\pgfqpoint{1.534949in}{2.872727in}}%
\pgfpathcurveto{\pgfqpoint{1.540773in}{2.866903in}}{\pgfqpoint{1.548673in}{2.863631in}}{\pgfqpoint{1.556909in}{2.863631in}}%
\pgfpathclose%
\pgfusepath{stroke,fill}%
\end{pgfscope}%
\begin{pgfscope}%
\pgfpathrectangle{\pgfqpoint{0.100000in}{0.212622in}}{\pgfqpoint{3.696000in}{3.696000in}}%
\pgfusepath{clip}%
\pgfsetbuttcap%
\pgfsetroundjoin%
\definecolor{currentfill}{rgb}{0.121569,0.466667,0.705882}%
\pgfsetfillcolor{currentfill}%
\pgfsetfillopacity{0.677337}%
\pgfsetlinewidth{1.003750pt}%
\definecolor{currentstroke}{rgb}{0.121569,0.466667,0.705882}%
\pgfsetstrokecolor{currentstroke}%
\pgfsetstrokeopacity{0.677337}%
\pgfsetdash{}{0pt}%
\pgfpathmoveto{\pgfqpoint{3.110478in}{1.801130in}}%
\pgfpathcurveto{\pgfqpoint{3.118714in}{1.801130in}}{\pgfqpoint{3.126614in}{1.804402in}}{\pgfqpoint{3.132438in}{1.810226in}}%
\pgfpathcurveto{\pgfqpoint{3.138262in}{1.816050in}}{\pgfqpoint{3.141534in}{1.823950in}}{\pgfqpoint{3.141534in}{1.832186in}}%
\pgfpathcurveto{\pgfqpoint{3.141534in}{1.840423in}}{\pgfqpoint{3.138262in}{1.848323in}}{\pgfqpoint{3.132438in}{1.854147in}}%
\pgfpathcurveto{\pgfqpoint{3.126614in}{1.859971in}}{\pgfqpoint{3.118714in}{1.863243in}}{\pgfqpoint{3.110478in}{1.863243in}}%
\pgfpathcurveto{\pgfqpoint{3.102242in}{1.863243in}}{\pgfqpoint{3.094342in}{1.859971in}}{\pgfqpoint{3.088518in}{1.854147in}}%
\pgfpathcurveto{\pgfqpoint{3.082694in}{1.848323in}}{\pgfqpoint{3.079421in}{1.840423in}}{\pgfqpoint{3.079421in}{1.832186in}}%
\pgfpathcurveto{\pgfqpoint{3.079421in}{1.823950in}}{\pgfqpoint{3.082694in}{1.816050in}}{\pgfqpoint{3.088518in}{1.810226in}}%
\pgfpathcurveto{\pgfqpoint{3.094342in}{1.804402in}}{\pgfqpoint{3.102242in}{1.801130in}}{\pgfqpoint{3.110478in}{1.801130in}}%
\pgfpathclose%
\pgfusepath{stroke,fill}%
\end{pgfscope}%
\begin{pgfscope}%
\pgfpathrectangle{\pgfqpoint{0.100000in}{0.212622in}}{\pgfqpoint{3.696000in}{3.696000in}}%
\pgfusepath{clip}%
\pgfsetbuttcap%
\pgfsetroundjoin%
\definecolor{currentfill}{rgb}{0.121569,0.466667,0.705882}%
\pgfsetfillcolor{currentfill}%
\pgfsetfillopacity{0.678133}%
\pgfsetlinewidth{1.003750pt}%
\definecolor{currentstroke}{rgb}{0.121569,0.466667,0.705882}%
\pgfsetstrokecolor{currentstroke}%
\pgfsetstrokeopacity{0.678133}%
\pgfsetdash{}{0pt}%
\pgfpathmoveto{\pgfqpoint{3.108840in}{1.799497in}}%
\pgfpathcurveto{\pgfqpoint{3.117076in}{1.799497in}}{\pgfqpoint{3.124976in}{1.802769in}}{\pgfqpoint{3.130800in}{1.808593in}}%
\pgfpathcurveto{\pgfqpoint{3.136624in}{1.814417in}}{\pgfqpoint{3.139897in}{1.822317in}}{\pgfqpoint{3.139897in}{1.830554in}}%
\pgfpathcurveto{\pgfqpoint{3.139897in}{1.838790in}}{\pgfqpoint{3.136624in}{1.846690in}}{\pgfqpoint{3.130800in}{1.852514in}}%
\pgfpathcurveto{\pgfqpoint{3.124976in}{1.858338in}}{\pgfqpoint{3.117076in}{1.861610in}}{\pgfqpoint{3.108840in}{1.861610in}}%
\pgfpathcurveto{\pgfqpoint{3.100604in}{1.861610in}}{\pgfqpoint{3.092704in}{1.858338in}}{\pgfqpoint{3.086880in}{1.852514in}}%
\pgfpathcurveto{\pgfqpoint{3.081056in}{1.846690in}}{\pgfqpoint{3.077784in}{1.838790in}}{\pgfqpoint{3.077784in}{1.830554in}}%
\pgfpathcurveto{\pgfqpoint{3.077784in}{1.822317in}}{\pgfqpoint{3.081056in}{1.814417in}}{\pgfqpoint{3.086880in}{1.808593in}}%
\pgfpathcurveto{\pgfqpoint{3.092704in}{1.802769in}}{\pgfqpoint{3.100604in}{1.799497in}}{\pgfqpoint{3.108840in}{1.799497in}}%
\pgfpathclose%
\pgfusepath{stroke,fill}%
\end{pgfscope}%
\begin{pgfscope}%
\pgfpathrectangle{\pgfqpoint{0.100000in}{0.212622in}}{\pgfqpoint{3.696000in}{3.696000in}}%
\pgfusepath{clip}%
\pgfsetbuttcap%
\pgfsetroundjoin%
\definecolor{currentfill}{rgb}{0.121569,0.466667,0.705882}%
\pgfsetfillcolor{currentfill}%
\pgfsetfillopacity{0.678143}%
\pgfsetlinewidth{1.003750pt}%
\definecolor{currentstroke}{rgb}{0.121569,0.466667,0.705882}%
\pgfsetstrokecolor{currentstroke}%
\pgfsetstrokeopacity{0.678143}%
\pgfsetdash{}{0pt}%
\pgfpathmoveto{\pgfqpoint{1.553872in}{2.859424in}}%
\pgfpathcurveto{\pgfqpoint{1.562108in}{2.859424in}}{\pgfqpoint{1.570008in}{2.862696in}}{\pgfqpoint{1.575832in}{2.868520in}}%
\pgfpathcurveto{\pgfqpoint{1.581656in}{2.874344in}}{\pgfqpoint{1.584928in}{2.882244in}}{\pgfqpoint{1.584928in}{2.890480in}}%
\pgfpathcurveto{\pgfqpoint{1.584928in}{2.898717in}}{\pgfqpoint{1.581656in}{2.906617in}}{\pgfqpoint{1.575832in}{2.912441in}}%
\pgfpathcurveto{\pgfqpoint{1.570008in}{2.918265in}}{\pgfqpoint{1.562108in}{2.921537in}}{\pgfqpoint{1.553872in}{2.921537in}}%
\pgfpathcurveto{\pgfqpoint{1.545636in}{2.921537in}}{\pgfqpoint{1.537736in}{2.918265in}}{\pgfqpoint{1.531912in}{2.912441in}}%
\pgfpathcurveto{\pgfqpoint{1.526088in}{2.906617in}}{\pgfqpoint{1.522815in}{2.898717in}}{\pgfqpoint{1.522815in}{2.890480in}}%
\pgfpathcurveto{\pgfqpoint{1.522815in}{2.882244in}}{\pgfqpoint{1.526088in}{2.874344in}}{\pgfqpoint{1.531912in}{2.868520in}}%
\pgfpathcurveto{\pgfqpoint{1.537736in}{2.862696in}}{\pgfqpoint{1.545636in}{2.859424in}}{\pgfqpoint{1.553872in}{2.859424in}}%
\pgfpathclose%
\pgfusepath{stroke,fill}%
\end{pgfscope}%
\begin{pgfscope}%
\pgfpathrectangle{\pgfqpoint{0.100000in}{0.212622in}}{\pgfqpoint{3.696000in}{3.696000in}}%
\pgfusepath{clip}%
\pgfsetbuttcap%
\pgfsetroundjoin%
\definecolor{currentfill}{rgb}{0.121569,0.466667,0.705882}%
\pgfsetfillcolor{currentfill}%
\pgfsetfillopacity{0.678287}%
\pgfsetlinewidth{1.003750pt}%
\definecolor{currentstroke}{rgb}{0.121569,0.466667,0.705882}%
\pgfsetstrokecolor{currentstroke}%
\pgfsetstrokeopacity{0.678287}%
\pgfsetdash{}{0pt}%
\pgfpathmoveto{\pgfqpoint{2.285242in}{3.013205in}}%
\pgfpathcurveto{\pgfqpoint{2.293478in}{3.013205in}}{\pgfqpoint{2.301379in}{3.016477in}}{\pgfqpoint{2.307202in}{3.022301in}}%
\pgfpathcurveto{\pgfqpoint{2.313026in}{3.028125in}}{\pgfqpoint{2.316299in}{3.036025in}}{\pgfqpoint{2.316299in}{3.044262in}}%
\pgfpathcurveto{\pgfqpoint{2.316299in}{3.052498in}}{\pgfqpoint{2.313026in}{3.060398in}}{\pgfqpoint{2.307202in}{3.066222in}}%
\pgfpathcurveto{\pgfqpoint{2.301379in}{3.072046in}}{\pgfqpoint{2.293478in}{3.075318in}}{\pgfqpoint{2.285242in}{3.075318in}}%
\pgfpathcurveto{\pgfqpoint{2.277006in}{3.075318in}}{\pgfqpoint{2.269106in}{3.072046in}}{\pgfqpoint{2.263282in}{3.066222in}}%
\pgfpathcurveto{\pgfqpoint{2.257458in}{3.060398in}}{\pgfqpoint{2.254186in}{3.052498in}}{\pgfqpoint{2.254186in}{3.044262in}}%
\pgfpathcurveto{\pgfqpoint{2.254186in}{3.036025in}}{\pgfqpoint{2.257458in}{3.028125in}}{\pgfqpoint{2.263282in}{3.022301in}}%
\pgfpathcurveto{\pgfqpoint{2.269106in}{3.016477in}}{\pgfqpoint{2.277006in}{3.013205in}}{\pgfqpoint{2.285242in}{3.013205in}}%
\pgfpathclose%
\pgfusepath{stroke,fill}%
\end{pgfscope}%
\begin{pgfscope}%
\pgfpathrectangle{\pgfqpoint{0.100000in}{0.212622in}}{\pgfqpoint{3.696000in}{3.696000in}}%
\pgfusepath{clip}%
\pgfsetbuttcap%
\pgfsetroundjoin%
\definecolor{currentfill}{rgb}{0.121569,0.466667,0.705882}%
\pgfsetfillcolor{currentfill}%
\pgfsetfillopacity{0.678626}%
\pgfsetlinewidth{1.003750pt}%
\definecolor{currentstroke}{rgb}{0.121569,0.466667,0.705882}%
\pgfsetstrokecolor{currentstroke}%
\pgfsetstrokeopacity{0.678626}%
\pgfsetdash{}{0pt}%
\pgfpathmoveto{\pgfqpoint{1.552245in}{2.857112in}}%
\pgfpathcurveto{\pgfqpoint{1.560481in}{2.857112in}}{\pgfqpoint{1.568381in}{2.860384in}}{\pgfqpoint{1.574205in}{2.866208in}}%
\pgfpathcurveto{\pgfqpoint{1.580029in}{2.872032in}}{\pgfqpoint{1.583301in}{2.879932in}}{\pgfqpoint{1.583301in}{2.888169in}}%
\pgfpathcurveto{\pgfqpoint{1.583301in}{2.896405in}}{\pgfqpoint{1.580029in}{2.904305in}}{\pgfqpoint{1.574205in}{2.910129in}}%
\pgfpathcurveto{\pgfqpoint{1.568381in}{2.915953in}}{\pgfqpoint{1.560481in}{2.919225in}}{\pgfqpoint{1.552245in}{2.919225in}}%
\pgfpathcurveto{\pgfqpoint{1.544009in}{2.919225in}}{\pgfqpoint{1.536108in}{2.915953in}}{\pgfqpoint{1.530285in}{2.910129in}}%
\pgfpathcurveto{\pgfqpoint{1.524461in}{2.904305in}}{\pgfqpoint{1.521188in}{2.896405in}}{\pgfqpoint{1.521188in}{2.888169in}}%
\pgfpathcurveto{\pgfqpoint{1.521188in}{2.879932in}}{\pgfqpoint{1.524461in}{2.872032in}}{\pgfqpoint{1.530285in}{2.866208in}}%
\pgfpathcurveto{\pgfqpoint{1.536108in}{2.860384in}}{\pgfqpoint{1.544009in}{2.857112in}}{\pgfqpoint{1.552245in}{2.857112in}}%
\pgfpathclose%
\pgfusepath{stroke,fill}%
\end{pgfscope}%
\begin{pgfscope}%
\pgfpathrectangle{\pgfqpoint{0.100000in}{0.212622in}}{\pgfqpoint{3.696000in}{3.696000in}}%
\pgfusepath{clip}%
\pgfsetbuttcap%
\pgfsetroundjoin%
\definecolor{currentfill}{rgb}{0.121569,0.466667,0.705882}%
\pgfsetfillcolor{currentfill}%
\pgfsetfillopacity{0.678795}%
\pgfsetlinewidth{1.003750pt}%
\definecolor{currentstroke}{rgb}{0.121569,0.466667,0.705882}%
\pgfsetstrokecolor{currentstroke}%
\pgfsetstrokeopacity{0.678795}%
\pgfsetdash{}{0pt}%
\pgfpathmoveto{\pgfqpoint{3.107515in}{1.798279in}}%
\pgfpathcurveto{\pgfqpoint{3.115751in}{1.798279in}}{\pgfqpoint{3.123651in}{1.801551in}}{\pgfqpoint{3.129475in}{1.807375in}}%
\pgfpathcurveto{\pgfqpoint{3.135299in}{1.813199in}}{\pgfqpoint{3.138571in}{1.821099in}}{\pgfqpoint{3.138571in}{1.829336in}}%
\pgfpathcurveto{\pgfqpoint{3.138571in}{1.837572in}}{\pgfqpoint{3.135299in}{1.845472in}}{\pgfqpoint{3.129475in}{1.851296in}}%
\pgfpathcurveto{\pgfqpoint{3.123651in}{1.857120in}}{\pgfqpoint{3.115751in}{1.860392in}}{\pgfqpoint{3.107515in}{1.860392in}}%
\pgfpathcurveto{\pgfqpoint{3.099279in}{1.860392in}}{\pgfqpoint{3.091379in}{1.857120in}}{\pgfqpoint{3.085555in}{1.851296in}}%
\pgfpathcurveto{\pgfqpoint{3.079731in}{1.845472in}}{\pgfqpoint{3.076458in}{1.837572in}}{\pgfqpoint{3.076458in}{1.829336in}}%
\pgfpathcurveto{\pgfqpoint{3.076458in}{1.821099in}}{\pgfqpoint{3.079731in}{1.813199in}}{\pgfqpoint{3.085555in}{1.807375in}}%
\pgfpathcurveto{\pgfqpoint{3.091379in}{1.801551in}}{\pgfqpoint{3.099279in}{1.798279in}}{\pgfqpoint{3.107515in}{1.798279in}}%
\pgfpathclose%
\pgfusepath{stroke,fill}%
\end{pgfscope}%
\begin{pgfscope}%
\pgfpathrectangle{\pgfqpoint{0.100000in}{0.212622in}}{\pgfqpoint{3.696000in}{3.696000in}}%
\pgfusepath{clip}%
\pgfsetbuttcap%
\pgfsetroundjoin%
\definecolor{currentfill}{rgb}{0.121569,0.466667,0.705882}%
\pgfsetfillcolor{currentfill}%
\pgfsetfillopacity{0.679099}%
\pgfsetlinewidth{1.003750pt}%
\definecolor{currentstroke}{rgb}{0.121569,0.466667,0.705882}%
\pgfsetstrokecolor{currentstroke}%
\pgfsetstrokeopacity{0.679099}%
\pgfsetdash{}{0pt}%
\pgfpathmoveto{\pgfqpoint{3.106925in}{1.797780in}}%
\pgfpathcurveto{\pgfqpoint{3.115161in}{1.797780in}}{\pgfqpoint{3.123061in}{1.801053in}}{\pgfqpoint{3.128885in}{1.806877in}}%
\pgfpathcurveto{\pgfqpoint{3.134709in}{1.812701in}}{\pgfqpoint{3.137981in}{1.820601in}}{\pgfqpoint{3.137981in}{1.828837in}}%
\pgfpathcurveto{\pgfqpoint{3.137981in}{1.837073in}}{\pgfqpoint{3.134709in}{1.844973in}}{\pgfqpoint{3.128885in}{1.850797in}}%
\pgfpathcurveto{\pgfqpoint{3.123061in}{1.856621in}}{\pgfqpoint{3.115161in}{1.859893in}}{\pgfqpoint{3.106925in}{1.859893in}}%
\pgfpathcurveto{\pgfqpoint{3.098689in}{1.859893in}}{\pgfqpoint{3.090789in}{1.856621in}}{\pgfqpoint{3.084965in}{1.850797in}}%
\pgfpathcurveto{\pgfqpoint{3.079141in}{1.844973in}}{\pgfqpoint{3.075868in}{1.837073in}}{\pgfqpoint{3.075868in}{1.828837in}}%
\pgfpathcurveto{\pgfqpoint{3.075868in}{1.820601in}}{\pgfqpoint{3.079141in}{1.812701in}}{\pgfqpoint{3.084965in}{1.806877in}}%
\pgfpathcurveto{\pgfqpoint{3.090789in}{1.801053in}}{\pgfqpoint{3.098689in}{1.797780in}}{\pgfqpoint{3.106925in}{1.797780in}}%
\pgfpathclose%
\pgfusepath{stroke,fill}%
\end{pgfscope}%
\begin{pgfscope}%
\pgfpathrectangle{\pgfqpoint{0.100000in}{0.212622in}}{\pgfqpoint{3.696000in}{3.696000in}}%
\pgfusepath{clip}%
\pgfsetbuttcap%
\pgfsetroundjoin%
\definecolor{currentfill}{rgb}{0.121569,0.466667,0.705882}%
\pgfsetfillcolor{currentfill}%
\pgfsetfillopacity{0.679161}%
\pgfsetlinewidth{1.003750pt}%
\definecolor{currentstroke}{rgb}{0.121569,0.466667,0.705882}%
\pgfsetstrokecolor{currentstroke}%
\pgfsetstrokeopacity{0.679161}%
\pgfsetdash{}{0pt}%
\pgfpathmoveto{\pgfqpoint{2.291692in}{3.013363in}}%
\pgfpathcurveto{\pgfqpoint{2.299928in}{3.013363in}}{\pgfqpoint{2.307828in}{3.016635in}}{\pgfqpoint{2.313652in}{3.022459in}}%
\pgfpathcurveto{\pgfqpoint{2.319476in}{3.028283in}}{\pgfqpoint{2.322748in}{3.036183in}}{\pgfqpoint{2.322748in}{3.044419in}}%
\pgfpathcurveto{\pgfqpoint{2.322748in}{3.052656in}}{\pgfqpoint{2.319476in}{3.060556in}}{\pgfqpoint{2.313652in}{3.066380in}}%
\pgfpathcurveto{\pgfqpoint{2.307828in}{3.072204in}}{\pgfqpoint{2.299928in}{3.075476in}}{\pgfqpoint{2.291692in}{3.075476in}}%
\pgfpathcurveto{\pgfqpoint{2.283456in}{3.075476in}}{\pgfqpoint{2.275556in}{3.072204in}}{\pgfqpoint{2.269732in}{3.066380in}}%
\pgfpathcurveto{\pgfqpoint{2.263908in}{3.060556in}}{\pgfqpoint{2.260635in}{3.052656in}}{\pgfqpoint{2.260635in}{3.044419in}}%
\pgfpathcurveto{\pgfqpoint{2.260635in}{3.036183in}}{\pgfqpoint{2.263908in}{3.028283in}}{\pgfqpoint{2.269732in}{3.022459in}}%
\pgfpathcurveto{\pgfqpoint{2.275556in}{3.016635in}}{\pgfqpoint{2.283456in}{3.013363in}}{\pgfqpoint{2.291692in}{3.013363in}}%
\pgfpathclose%
\pgfusepath{stroke,fill}%
\end{pgfscope}%
\begin{pgfscope}%
\pgfpathrectangle{\pgfqpoint{0.100000in}{0.212622in}}{\pgfqpoint{3.696000in}{3.696000in}}%
\pgfusepath{clip}%
\pgfsetbuttcap%
\pgfsetroundjoin%
\definecolor{currentfill}{rgb}{0.121569,0.466667,0.705882}%
\pgfsetfillcolor{currentfill}%
\pgfsetfillopacity{0.679293}%
\pgfsetlinewidth{1.003750pt}%
\definecolor{currentstroke}{rgb}{0.121569,0.466667,0.705882}%
\pgfsetstrokecolor{currentstroke}%
\pgfsetstrokeopacity{0.679293}%
\pgfsetdash{}{0pt}%
\pgfpathmoveto{\pgfqpoint{1.550186in}{2.854400in}}%
\pgfpathcurveto{\pgfqpoint{1.558423in}{2.854400in}}{\pgfqpoint{1.566323in}{2.857672in}}{\pgfqpoint{1.572147in}{2.863496in}}%
\pgfpathcurveto{\pgfqpoint{1.577970in}{2.869320in}}{\pgfqpoint{1.581243in}{2.877220in}}{\pgfqpoint{1.581243in}{2.885456in}}%
\pgfpathcurveto{\pgfqpoint{1.581243in}{2.893693in}}{\pgfqpoint{1.577970in}{2.901593in}}{\pgfqpoint{1.572147in}{2.907417in}}%
\pgfpathcurveto{\pgfqpoint{1.566323in}{2.913241in}}{\pgfqpoint{1.558423in}{2.916513in}}{\pgfqpoint{1.550186in}{2.916513in}}%
\pgfpathcurveto{\pgfqpoint{1.541950in}{2.916513in}}{\pgfqpoint{1.534050in}{2.913241in}}{\pgfqpoint{1.528226in}{2.907417in}}%
\pgfpathcurveto{\pgfqpoint{1.522402in}{2.901593in}}{\pgfqpoint{1.519130in}{2.893693in}}{\pgfqpoint{1.519130in}{2.885456in}}%
\pgfpathcurveto{\pgfqpoint{1.519130in}{2.877220in}}{\pgfqpoint{1.522402in}{2.869320in}}{\pgfqpoint{1.528226in}{2.863496in}}%
\pgfpathcurveto{\pgfqpoint{1.534050in}{2.857672in}}{\pgfqpoint{1.541950in}{2.854400in}}{\pgfqpoint{1.550186in}{2.854400in}}%
\pgfpathclose%
\pgfusepath{stroke,fill}%
\end{pgfscope}%
\begin{pgfscope}%
\pgfpathrectangle{\pgfqpoint{0.100000in}{0.212622in}}{\pgfqpoint{3.696000in}{3.696000in}}%
\pgfusepath{clip}%
\pgfsetbuttcap%
\pgfsetroundjoin%
\definecolor{currentfill}{rgb}{0.121569,0.466667,0.705882}%
\pgfsetfillcolor{currentfill}%
\pgfsetfillopacity{0.679651}%
\pgfsetlinewidth{1.003750pt}%
\definecolor{currentstroke}{rgb}{0.121569,0.466667,0.705882}%
\pgfsetstrokecolor{currentstroke}%
\pgfsetstrokeopacity{0.679651}%
\pgfsetdash{}{0pt}%
\pgfpathmoveto{\pgfqpoint{3.105879in}{1.796829in}}%
\pgfpathcurveto{\pgfqpoint{3.114116in}{1.796829in}}{\pgfqpoint{3.122016in}{1.800101in}}{\pgfqpoint{3.127840in}{1.805925in}}%
\pgfpathcurveto{\pgfqpoint{3.133664in}{1.811749in}}{\pgfqpoint{3.136936in}{1.819649in}}{\pgfqpoint{3.136936in}{1.827886in}}%
\pgfpathcurveto{\pgfqpoint{3.136936in}{1.836122in}}{\pgfqpoint{3.133664in}{1.844022in}}{\pgfqpoint{3.127840in}{1.849846in}}%
\pgfpathcurveto{\pgfqpoint{3.122016in}{1.855670in}}{\pgfqpoint{3.114116in}{1.858942in}}{\pgfqpoint{3.105879in}{1.858942in}}%
\pgfpathcurveto{\pgfqpoint{3.097643in}{1.858942in}}{\pgfqpoint{3.089743in}{1.855670in}}{\pgfqpoint{3.083919in}{1.849846in}}%
\pgfpathcurveto{\pgfqpoint{3.078095in}{1.844022in}}{\pgfqpoint{3.074823in}{1.836122in}}{\pgfqpoint{3.074823in}{1.827886in}}%
\pgfpathcurveto{\pgfqpoint{3.074823in}{1.819649in}}{\pgfqpoint{3.078095in}{1.811749in}}{\pgfqpoint{3.083919in}{1.805925in}}%
\pgfpathcurveto{\pgfqpoint{3.089743in}{1.800101in}}{\pgfqpoint{3.097643in}{1.796829in}}{\pgfqpoint{3.105879in}{1.796829in}}%
\pgfpathclose%
\pgfusepath{stroke,fill}%
\end{pgfscope}%
\begin{pgfscope}%
\pgfpathrectangle{\pgfqpoint{0.100000in}{0.212622in}}{\pgfqpoint{3.696000in}{3.696000in}}%
\pgfusepath{clip}%
\pgfsetbuttcap%
\pgfsetroundjoin%
\definecolor{currentfill}{rgb}{0.121569,0.466667,0.705882}%
\pgfsetfillcolor{currentfill}%
\pgfsetfillopacity{0.679917}%
\pgfsetlinewidth{1.003750pt}%
\definecolor{currentstroke}{rgb}{0.121569,0.466667,0.705882}%
\pgfsetstrokecolor{currentstroke}%
\pgfsetstrokeopacity{0.679917}%
\pgfsetdash{}{0pt}%
\pgfpathmoveto{\pgfqpoint{2.297494in}{3.012511in}}%
\pgfpathcurveto{\pgfqpoint{2.305730in}{3.012511in}}{\pgfqpoint{2.313630in}{3.015784in}}{\pgfqpoint{2.319454in}{3.021608in}}%
\pgfpathcurveto{\pgfqpoint{2.325278in}{3.027432in}}{\pgfqpoint{2.328550in}{3.035332in}}{\pgfqpoint{2.328550in}{3.043568in}}%
\pgfpathcurveto{\pgfqpoint{2.328550in}{3.051804in}}{\pgfqpoint{2.325278in}{3.059704in}}{\pgfqpoint{2.319454in}{3.065528in}}%
\pgfpathcurveto{\pgfqpoint{2.313630in}{3.071352in}}{\pgfqpoint{2.305730in}{3.074624in}}{\pgfqpoint{2.297494in}{3.074624in}}%
\pgfpathcurveto{\pgfqpoint{2.289257in}{3.074624in}}{\pgfqpoint{2.281357in}{3.071352in}}{\pgfqpoint{2.275533in}{3.065528in}}%
\pgfpathcurveto{\pgfqpoint{2.269709in}{3.059704in}}{\pgfqpoint{2.266437in}{3.051804in}}{\pgfqpoint{2.266437in}{3.043568in}}%
\pgfpathcurveto{\pgfqpoint{2.266437in}{3.035332in}}{\pgfqpoint{2.269709in}{3.027432in}}{\pgfqpoint{2.275533in}{3.021608in}}%
\pgfpathcurveto{\pgfqpoint{2.281357in}{3.015784in}}{\pgfqpoint{2.289257in}{3.012511in}}{\pgfqpoint{2.297494in}{3.012511in}}%
\pgfpathclose%
\pgfusepath{stroke,fill}%
\end{pgfscope}%
\begin{pgfscope}%
\pgfpathrectangle{\pgfqpoint{0.100000in}{0.212622in}}{\pgfqpoint{3.696000in}{3.696000in}}%
\pgfusepath{clip}%
\pgfsetbuttcap%
\pgfsetroundjoin%
\definecolor{currentfill}{rgb}{0.121569,0.466667,0.705882}%
\pgfsetfillcolor{currentfill}%
\pgfsetfillopacity{0.680090}%
\pgfsetlinewidth{1.003750pt}%
\definecolor{currentstroke}{rgb}{0.121569,0.466667,0.705882}%
\pgfsetstrokecolor{currentstroke}%
\pgfsetstrokeopacity{0.680090}%
\pgfsetdash{}{0pt}%
\pgfpathmoveto{\pgfqpoint{3.104994in}{1.795957in}}%
\pgfpathcurveto{\pgfqpoint{3.113230in}{1.795957in}}{\pgfqpoint{3.121130in}{1.799229in}}{\pgfqpoint{3.126954in}{1.805053in}}%
\pgfpathcurveto{\pgfqpoint{3.132778in}{1.810877in}}{\pgfqpoint{3.136050in}{1.818777in}}{\pgfqpoint{3.136050in}{1.827013in}}%
\pgfpathcurveto{\pgfqpoint{3.136050in}{1.835250in}}{\pgfqpoint{3.132778in}{1.843150in}}{\pgfqpoint{3.126954in}{1.848974in}}%
\pgfpathcurveto{\pgfqpoint{3.121130in}{1.854798in}}{\pgfqpoint{3.113230in}{1.858070in}}{\pgfqpoint{3.104994in}{1.858070in}}%
\pgfpathcurveto{\pgfqpoint{3.096758in}{1.858070in}}{\pgfqpoint{3.088858in}{1.854798in}}{\pgfqpoint{3.083034in}{1.848974in}}%
\pgfpathcurveto{\pgfqpoint{3.077210in}{1.843150in}}{\pgfqpoint{3.073937in}{1.835250in}}{\pgfqpoint{3.073937in}{1.827013in}}%
\pgfpathcurveto{\pgfqpoint{3.073937in}{1.818777in}}{\pgfqpoint{3.077210in}{1.810877in}}{\pgfqpoint{3.083034in}{1.805053in}}%
\pgfpathcurveto{\pgfqpoint{3.088858in}{1.799229in}}{\pgfqpoint{3.096758in}{1.795957in}}{\pgfqpoint{3.104994in}{1.795957in}}%
\pgfpathclose%
\pgfusepath{stroke,fill}%
\end{pgfscope}%
\begin{pgfscope}%
\pgfpathrectangle{\pgfqpoint{0.100000in}{0.212622in}}{\pgfqpoint{3.696000in}{3.696000in}}%
\pgfusepath{clip}%
\pgfsetbuttcap%
\pgfsetroundjoin%
\definecolor{currentfill}{rgb}{0.121569,0.466667,0.705882}%
\pgfsetfillcolor{currentfill}%
\pgfsetfillopacity{0.680092}%
\pgfsetlinewidth{1.003750pt}%
\definecolor{currentstroke}{rgb}{0.121569,0.466667,0.705882}%
\pgfsetstrokecolor{currentstroke}%
\pgfsetstrokeopacity{0.680092}%
\pgfsetdash{}{0pt}%
\pgfpathmoveto{\pgfqpoint{1.547747in}{2.851554in}}%
\pgfpathcurveto{\pgfqpoint{1.555984in}{2.851554in}}{\pgfqpoint{1.563884in}{2.854826in}}{\pgfqpoint{1.569708in}{2.860650in}}%
\pgfpathcurveto{\pgfqpoint{1.575532in}{2.866474in}}{\pgfqpoint{1.578804in}{2.874374in}}{\pgfqpoint{1.578804in}{2.882610in}}%
\pgfpathcurveto{\pgfqpoint{1.578804in}{2.890847in}}{\pgfqpoint{1.575532in}{2.898747in}}{\pgfqpoint{1.569708in}{2.904571in}}%
\pgfpathcurveto{\pgfqpoint{1.563884in}{2.910395in}}{\pgfqpoint{1.555984in}{2.913667in}}{\pgfqpoint{1.547747in}{2.913667in}}%
\pgfpathcurveto{\pgfqpoint{1.539511in}{2.913667in}}{\pgfqpoint{1.531611in}{2.910395in}}{\pgfqpoint{1.525787in}{2.904571in}}%
\pgfpathcurveto{\pgfqpoint{1.519963in}{2.898747in}}{\pgfqpoint{1.516691in}{2.890847in}}{\pgfqpoint{1.516691in}{2.882610in}}%
\pgfpathcurveto{\pgfqpoint{1.516691in}{2.874374in}}{\pgfqpoint{1.519963in}{2.866474in}}{\pgfqpoint{1.525787in}{2.860650in}}%
\pgfpathcurveto{\pgfqpoint{1.531611in}{2.854826in}}{\pgfqpoint{1.539511in}{2.851554in}}{\pgfqpoint{1.547747in}{2.851554in}}%
\pgfpathclose%
\pgfusepath{stroke,fill}%
\end{pgfscope}%
\begin{pgfscope}%
\pgfpathrectangle{\pgfqpoint{0.100000in}{0.212622in}}{\pgfqpoint{3.696000in}{3.696000in}}%
\pgfusepath{clip}%
\pgfsetbuttcap%
\pgfsetroundjoin%
\definecolor{currentfill}{rgb}{0.121569,0.466667,0.705882}%
\pgfsetfillcolor{currentfill}%
\pgfsetfillopacity{0.680353}%
\pgfsetlinewidth{1.003750pt}%
\definecolor{currentstroke}{rgb}{0.121569,0.466667,0.705882}%
\pgfsetstrokecolor{currentstroke}%
\pgfsetstrokeopacity{0.680353}%
\pgfsetdash{}{0pt}%
\pgfpathmoveto{\pgfqpoint{3.104443in}{1.795383in}}%
\pgfpathcurveto{\pgfqpoint{3.112680in}{1.795383in}}{\pgfqpoint{3.120580in}{1.798655in}}{\pgfqpoint{3.126404in}{1.804479in}}%
\pgfpathcurveto{\pgfqpoint{3.132227in}{1.810303in}}{\pgfqpoint{3.135500in}{1.818203in}}{\pgfqpoint{3.135500in}{1.826439in}}%
\pgfpathcurveto{\pgfqpoint{3.135500in}{1.834676in}}{\pgfqpoint{3.132227in}{1.842576in}}{\pgfqpoint{3.126404in}{1.848400in}}%
\pgfpathcurveto{\pgfqpoint{3.120580in}{1.854223in}}{\pgfqpoint{3.112680in}{1.857496in}}{\pgfqpoint{3.104443in}{1.857496in}}%
\pgfpathcurveto{\pgfqpoint{3.096207in}{1.857496in}}{\pgfqpoint{3.088307in}{1.854223in}}{\pgfqpoint{3.082483in}{1.848400in}}%
\pgfpathcurveto{\pgfqpoint{3.076659in}{1.842576in}}{\pgfqpoint{3.073387in}{1.834676in}}{\pgfqpoint{3.073387in}{1.826439in}}%
\pgfpathcurveto{\pgfqpoint{3.073387in}{1.818203in}}{\pgfqpoint{3.076659in}{1.810303in}}{\pgfqpoint{3.082483in}{1.804479in}}%
\pgfpathcurveto{\pgfqpoint{3.088307in}{1.798655in}}{\pgfqpoint{3.096207in}{1.795383in}}{\pgfqpoint{3.104443in}{1.795383in}}%
\pgfpathclose%
\pgfusepath{stroke,fill}%
\end{pgfscope}%
\begin{pgfscope}%
\pgfpathrectangle{\pgfqpoint{0.100000in}{0.212622in}}{\pgfqpoint{3.696000in}{3.696000in}}%
\pgfusepath{clip}%
\pgfsetbuttcap%
\pgfsetroundjoin%
\definecolor{currentfill}{rgb}{0.121569,0.466667,0.705882}%
\pgfsetfillcolor{currentfill}%
\pgfsetfillopacity{0.680527}%
\pgfsetlinewidth{1.003750pt}%
\definecolor{currentstroke}{rgb}{0.121569,0.466667,0.705882}%
\pgfsetstrokecolor{currentstroke}%
\pgfsetstrokeopacity{0.680527}%
\pgfsetdash{}{0pt}%
\pgfpathmoveto{\pgfqpoint{2.302143in}{3.011649in}}%
\pgfpathcurveto{\pgfqpoint{2.310379in}{3.011649in}}{\pgfqpoint{2.318279in}{3.014921in}}{\pgfqpoint{2.324103in}{3.020745in}}%
\pgfpathcurveto{\pgfqpoint{2.329927in}{3.026569in}}{\pgfqpoint{2.333199in}{3.034469in}}{\pgfqpoint{2.333199in}{3.042706in}}%
\pgfpathcurveto{\pgfqpoint{2.333199in}{3.050942in}}{\pgfqpoint{2.329927in}{3.058842in}}{\pgfqpoint{2.324103in}{3.064666in}}%
\pgfpathcurveto{\pgfqpoint{2.318279in}{3.070490in}}{\pgfqpoint{2.310379in}{3.073762in}}{\pgfqpoint{2.302143in}{3.073762in}}%
\pgfpathcurveto{\pgfqpoint{2.293906in}{3.073762in}}{\pgfqpoint{2.286006in}{3.070490in}}{\pgfqpoint{2.280182in}{3.064666in}}%
\pgfpathcurveto{\pgfqpoint{2.274358in}{3.058842in}}{\pgfqpoint{2.271086in}{3.050942in}}{\pgfqpoint{2.271086in}{3.042706in}}%
\pgfpathcurveto{\pgfqpoint{2.271086in}{3.034469in}}{\pgfqpoint{2.274358in}{3.026569in}}{\pgfqpoint{2.280182in}{3.020745in}}%
\pgfpathcurveto{\pgfqpoint{2.286006in}{3.014921in}}{\pgfqpoint{2.293906in}{3.011649in}}{\pgfqpoint{2.302143in}{3.011649in}}%
\pgfpathclose%
\pgfusepath{stroke,fill}%
\end{pgfscope}%
\begin{pgfscope}%
\pgfpathrectangle{\pgfqpoint{0.100000in}{0.212622in}}{\pgfqpoint{3.696000in}{3.696000in}}%
\pgfusepath{clip}%
\pgfsetbuttcap%
\pgfsetroundjoin%
\definecolor{currentfill}{rgb}{0.121569,0.466667,0.705882}%
\pgfsetfillcolor{currentfill}%
\pgfsetfillopacity{0.680849}%
\pgfsetlinewidth{1.003750pt}%
\definecolor{currentstroke}{rgb}{0.121569,0.466667,0.705882}%
\pgfsetstrokecolor{currentstroke}%
\pgfsetstrokeopacity{0.680849}%
\pgfsetdash{}{0pt}%
\pgfpathmoveto{\pgfqpoint{3.103406in}{1.794488in}}%
\pgfpathcurveto{\pgfqpoint{3.111642in}{1.794488in}}{\pgfqpoint{3.119542in}{1.797760in}}{\pgfqpoint{3.125366in}{1.803584in}}%
\pgfpathcurveto{\pgfqpoint{3.131190in}{1.809408in}}{\pgfqpoint{3.134462in}{1.817308in}}{\pgfqpoint{3.134462in}{1.825544in}}%
\pgfpathcurveto{\pgfqpoint{3.134462in}{1.833780in}}{\pgfqpoint{3.131190in}{1.841681in}}{\pgfqpoint{3.125366in}{1.847504in}}%
\pgfpathcurveto{\pgfqpoint{3.119542in}{1.853328in}}{\pgfqpoint{3.111642in}{1.856601in}}{\pgfqpoint{3.103406in}{1.856601in}}%
\pgfpathcurveto{\pgfqpoint{3.095169in}{1.856601in}}{\pgfqpoint{3.087269in}{1.853328in}}{\pgfqpoint{3.081445in}{1.847504in}}%
\pgfpathcurveto{\pgfqpoint{3.075621in}{1.841681in}}{\pgfqpoint{3.072349in}{1.833780in}}{\pgfqpoint{3.072349in}{1.825544in}}%
\pgfpathcurveto{\pgfqpoint{3.072349in}{1.817308in}}{\pgfqpoint{3.075621in}{1.809408in}}{\pgfqpoint{3.081445in}{1.803584in}}%
\pgfpathcurveto{\pgfqpoint{3.087269in}{1.797760in}}{\pgfqpoint{3.095169in}{1.794488in}}{\pgfqpoint{3.103406in}{1.794488in}}%
\pgfpathclose%
\pgfusepath{stroke,fill}%
\end{pgfscope}%
\begin{pgfscope}%
\pgfpathrectangle{\pgfqpoint{0.100000in}{0.212622in}}{\pgfqpoint{3.696000in}{3.696000in}}%
\pgfusepath{clip}%
\pgfsetbuttcap%
\pgfsetroundjoin%
\definecolor{currentfill}{rgb}{0.121569,0.466667,0.705882}%
\pgfsetfillcolor{currentfill}%
\pgfsetfillopacity{0.681107}%
\pgfsetlinewidth{1.003750pt}%
\definecolor{currentstroke}{rgb}{0.121569,0.466667,0.705882}%
\pgfsetstrokecolor{currentstroke}%
\pgfsetstrokeopacity{0.681107}%
\pgfsetdash{}{0pt}%
\pgfpathmoveto{\pgfqpoint{2.306256in}{3.011080in}}%
\pgfpathcurveto{\pgfqpoint{2.314492in}{3.011080in}}{\pgfqpoint{2.322393in}{3.014353in}}{\pgfqpoint{2.328216in}{3.020177in}}%
\pgfpathcurveto{\pgfqpoint{2.334040in}{3.026000in}}{\pgfqpoint{2.337313in}{3.033901in}}{\pgfqpoint{2.337313in}{3.042137in}}%
\pgfpathcurveto{\pgfqpoint{2.337313in}{3.050373in}}{\pgfqpoint{2.334040in}{3.058273in}}{\pgfqpoint{2.328216in}{3.064097in}}%
\pgfpathcurveto{\pgfqpoint{2.322393in}{3.069921in}}{\pgfqpoint{2.314492in}{3.073193in}}{\pgfqpoint{2.306256in}{3.073193in}}%
\pgfpathcurveto{\pgfqpoint{2.298020in}{3.073193in}}{\pgfqpoint{2.290120in}{3.069921in}}{\pgfqpoint{2.284296in}{3.064097in}}%
\pgfpathcurveto{\pgfqpoint{2.278472in}{3.058273in}}{\pgfqpoint{2.275200in}{3.050373in}}{\pgfqpoint{2.275200in}{3.042137in}}%
\pgfpathcurveto{\pgfqpoint{2.275200in}{3.033901in}}{\pgfqpoint{2.278472in}{3.026000in}}{\pgfqpoint{2.284296in}{3.020177in}}%
\pgfpathcurveto{\pgfqpoint{2.290120in}{3.014353in}}{\pgfqpoint{2.298020in}{3.011080in}}{\pgfqpoint{2.306256in}{3.011080in}}%
\pgfpathclose%
\pgfusepath{stroke,fill}%
\end{pgfscope}%
\begin{pgfscope}%
\pgfpathrectangle{\pgfqpoint{0.100000in}{0.212622in}}{\pgfqpoint{3.696000in}{3.696000in}}%
\pgfusepath{clip}%
\pgfsetbuttcap%
\pgfsetroundjoin%
\definecolor{currentfill}{rgb}{0.121569,0.466667,0.705882}%
\pgfsetfillcolor{currentfill}%
\pgfsetfillopacity{0.681147}%
\pgfsetlinewidth{1.003750pt}%
\definecolor{currentstroke}{rgb}{0.121569,0.466667,0.705882}%
\pgfsetstrokecolor{currentstroke}%
\pgfsetstrokeopacity{0.681147}%
\pgfsetdash{}{0pt}%
\pgfpathmoveto{\pgfqpoint{1.544285in}{2.847211in}}%
\pgfpathcurveto{\pgfqpoint{1.552522in}{2.847211in}}{\pgfqpoint{1.560422in}{2.850484in}}{\pgfqpoint{1.566246in}{2.856308in}}%
\pgfpathcurveto{\pgfqpoint{1.572069in}{2.862132in}}{\pgfqpoint{1.575342in}{2.870032in}}{\pgfqpoint{1.575342in}{2.878268in}}%
\pgfpathcurveto{\pgfqpoint{1.575342in}{2.886504in}}{\pgfqpoint{1.572069in}{2.894404in}}{\pgfqpoint{1.566246in}{2.900228in}}%
\pgfpathcurveto{\pgfqpoint{1.560422in}{2.906052in}}{\pgfqpoint{1.552522in}{2.909324in}}{\pgfqpoint{1.544285in}{2.909324in}}%
\pgfpathcurveto{\pgfqpoint{1.536049in}{2.909324in}}{\pgfqpoint{1.528149in}{2.906052in}}{\pgfqpoint{1.522325in}{2.900228in}}%
\pgfpathcurveto{\pgfqpoint{1.516501in}{2.894404in}}{\pgfqpoint{1.513229in}{2.886504in}}{\pgfqpoint{1.513229in}{2.878268in}}%
\pgfpathcurveto{\pgfqpoint{1.513229in}{2.870032in}}{\pgfqpoint{1.516501in}{2.862132in}}{\pgfqpoint{1.522325in}{2.856308in}}%
\pgfpathcurveto{\pgfqpoint{1.528149in}{2.850484in}}{\pgfqpoint{1.536049in}{2.847211in}}{\pgfqpoint{1.544285in}{2.847211in}}%
\pgfpathclose%
\pgfusepath{stroke,fill}%
\end{pgfscope}%
\begin{pgfscope}%
\pgfpathrectangle{\pgfqpoint{0.100000in}{0.212622in}}{\pgfqpoint{3.696000in}{3.696000in}}%
\pgfusepath{clip}%
\pgfsetbuttcap%
\pgfsetroundjoin%
\definecolor{currentfill}{rgb}{0.121569,0.466667,0.705882}%
\pgfsetfillcolor{currentfill}%
\pgfsetfillopacity{0.681447}%
\pgfsetlinewidth{1.003750pt}%
\definecolor{currentstroke}{rgb}{0.121569,0.466667,0.705882}%
\pgfsetstrokecolor{currentstroke}%
\pgfsetstrokeopacity{0.681447}%
\pgfsetdash{}{0pt}%
\pgfpathmoveto{\pgfqpoint{2.308625in}{3.010799in}}%
\pgfpathcurveto{\pgfqpoint{2.316861in}{3.010799in}}{\pgfqpoint{2.324761in}{3.014071in}}{\pgfqpoint{2.330585in}{3.019895in}}%
\pgfpathcurveto{\pgfqpoint{2.336409in}{3.025719in}}{\pgfqpoint{2.339682in}{3.033619in}}{\pgfqpoint{2.339682in}{3.041855in}}%
\pgfpathcurveto{\pgfqpoint{2.339682in}{3.050092in}}{\pgfqpoint{2.336409in}{3.057992in}}{\pgfqpoint{2.330585in}{3.063816in}}%
\pgfpathcurveto{\pgfqpoint{2.324761in}{3.069640in}}{\pgfqpoint{2.316861in}{3.072912in}}{\pgfqpoint{2.308625in}{3.072912in}}%
\pgfpathcurveto{\pgfqpoint{2.300389in}{3.072912in}}{\pgfqpoint{2.292489in}{3.069640in}}{\pgfqpoint{2.286665in}{3.063816in}}%
\pgfpathcurveto{\pgfqpoint{2.280841in}{3.057992in}}{\pgfqpoint{2.277569in}{3.050092in}}{\pgfqpoint{2.277569in}{3.041855in}}%
\pgfpathcurveto{\pgfqpoint{2.277569in}{3.033619in}}{\pgfqpoint{2.280841in}{3.025719in}}{\pgfqpoint{2.286665in}{3.019895in}}%
\pgfpathcurveto{\pgfqpoint{2.292489in}{3.014071in}}{\pgfqpoint{2.300389in}{3.010799in}}{\pgfqpoint{2.308625in}{3.010799in}}%
\pgfpathclose%
\pgfusepath{stroke,fill}%
\end{pgfscope}%
\begin{pgfscope}%
\pgfpathrectangle{\pgfqpoint{0.100000in}{0.212622in}}{\pgfqpoint{3.696000in}{3.696000in}}%
\pgfusepath{clip}%
\pgfsetbuttcap%
\pgfsetroundjoin%
\definecolor{currentfill}{rgb}{0.121569,0.466667,0.705882}%
\pgfsetfillcolor{currentfill}%
\pgfsetfillopacity{0.681709}%
\pgfsetlinewidth{1.003750pt}%
\definecolor{currentstroke}{rgb}{0.121569,0.466667,0.705882}%
\pgfsetstrokecolor{currentstroke}%
\pgfsetstrokeopacity{0.681709}%
\pgfsetdash{}{0pt}%
\pgfpathmoveto{\pgfqpoint{2.310483in}{3.010559in}}%
\pgfpathcurveto{\pgfqpoint{2.318719in}{3.010559in}}{\pgfqpoint{2.326620in}{3.013831in}}{\pgfqpoint{2.332443in}{3.019655in}}%
\pgfpathcurveto{\pgfqpoint{2.338267in}{3.025479in}}{\pgfqpoint{2.341540in}{3.033379in}}{\pgfqpoint{2.341540in}{3.041615in}}%
\pgfpathcurveto{\pgfqpoint{2.341540in}{3.049851in}}{\pgfqpoint{2.338267in}{3.057751in}}{\pgfqpoint{2.332443in}{3.063575in}}%
\pgfpathcurveto{\pgfqpoint{2.326620in}{3.069399in}}{\pgfqpoint{2.318719in}{3.072672in}}{\pgfqpoint{2.310483in}{3.072672in}}%
\pgfpathcurveto{\pgfqpoint{2.302247in}{3.072672in}}{\pgfqpoint{2.294347in}{3.069399in}}{\pgfqpoint{2.288523in}{3.063575in}}%
\pgfpathcurveto{\pgfqpoint{2.282699in}{3.057751in}}{\pgfqpoint{2.279427in}{3.049851in}}{\pgfqpoint{2.279427in}{3.041615in}}%
\pgfpathcurveto{\pgfqpoint{2.279427in}{3.033379in}}{\pgfqpoint{2.282699in}{3.025479in}}{\pgfqpoint{2.288523in}{3.019655in}}%
\pgfpathcurveto{\pgfqpoint{2.294347in}{3.013831in}}{\pgfqpoint{2.302247in}{3.010559in}}{\pgfqpoint{2.310483in}{3.010559in}}%
\pgfpathclose%
\pgfusepath{stroke,fill}%
\end{pgfscope}%
\begin{pgfscope}%
\pgfpathrectangle{\pgfqpoint{0.100000in}{0.212622in}}{\pgfqpoint{3.696000in}{3.696000in}}%
\pgfusepath{clip}%
\pgfsetbuttcap%
\pgfsetroundjoin%
\definecolor{currentfill}{rgb}{0.121569,0.466667,0.705882}%
\pgfsetfillcolor{currentfill}%
\pgfsetfillopacity{0.681774}%
\pgfsetlinewidth{1.003750pt}%
\definecolor{currentstroke}{rgb}{0.121569,0.466667,0.705882}%
\pgfsetstrokecolor{currentstroke}%
\pgfsetstrokeopacity{0.681774}%
\pgfsetdash{}{0pt}%
\pgfpathmoveto{\pgfqpoint{3.101542in}{1.792972in}}%
\pgfpathcurveto{\pgfqpoint{3.109778in}{1.792972in}}{\pgfqpoint{3.117678in}{1.796245in}}{\pgfqpoint{3.123502in}{1.802069in}}%
\pgfpathcurveto{\pgfqpoint{3.129326in}{1.807893in}}{\pgfqpoint{3.132598in}{1.815793in}}{\pgfqpoint{3.132598in}{1.824029in}}%
\pgfpathcurveto{\pgfqpoint{3.132598in}{1.832265in}}{\pgfqpoint{3.129326in}{1.840165in}}{\pgfqpoint{3.123502in}{1.845989in}}%
\pgfpathcurveto{\pgfqpoint{3.117678in}{1.851813in}}{\pgfqpoint{3.109778in}{1.855085in}}{\pgfqpoint{3.101542in}{1.855085in}}%
\pgfpathcurveto{\pgfqpoint{3.093305in}{1.855085in}}{\pgfqpoint{3.085405in}{1.851813in}}{\pgfqpoint{3.079581in}{1.845989in}}%
\pgfpathcurveto{\pgfqpoint{3.073757in}{1.840165in}}{\pgfqpoint{3.070485in}{1.832265in}}{\pgfqpoint{3.070485in}{1.824029in}}%
\pgfpathcurveto{\pgfqpoint{3.070485in}{1.815793in}}{\pgfqpoint{3.073757in}{1.807893in}}{\pgfqpoint{3.079581in}{1.802069in}}%
\pgfpathcurveto{\pgfqpoint{3.085405in}{1.796245in}}{\pgfqpoint{3.093305in}{1.792972in}}{\pgfqpoint{3.101542in}{1.792972in}}%
\pgfpathclose%
\pgfusepath{stroke,fill}%
\end{pgfscope}%
\begin{pgfscope}%
\pgfpathrectangle{\pgfqpoint{0.100000in}{0.212622in}}{\pgfqpoint{3.696000in}{3.696000in}}%
\pgfusepath{clip}%
\pgfsetbuttcap%
\pgfsetroundjoin%
\definecolor{currentfill}{rgb}{0.121569,0.466667,0.705882}%
\pgfsetfillcolor{currentfill}%
\pgfsetfillopacity{0.682162}%
\pgfsetlinewidth{1.003750pt}%
\definecolor{currentstroke}{rgb}{0.121569,0.466667,0.705882}%
\pgfsetstrokecolor{currentstroke}%
\pgfsetstrokeopacity{0.682162}%
\pgfsetdash{}{0pt}%
\pgfpathmoveto{\pgfqpoint{2.313912in}{3.010168in}}%
\pgfpathcurveto{\pgfqpoint{2.322148in}{3.010168in}}{\pgfqpoint{2.330048in}{3.013440in}}{\pgfqpoint{2.335872in}{3.019264in}}%
\pgfpathcurveto{\pgfqpoint{2.341696in}{3.025088in}}{\pgfqpoint{2.344968in}{3.032988in}}{\pgfqpoint{2.344968in}{3.041224in}}%
\pgfpathcurveto{\pgfqpoint{2.344968in}{3.049461in}}{\pgfqpoint{2.341696in}{3.057361in}}{\pgfqpoint{2.335872in}{3.063185in}}%
\pgfpathcurveto{\pgfqpoint{2.330048in}{3.069008in}}{\pgfqpoint{2.322148in}{3.072281in}}{\pgfqpoint{2.313912in}{3.072281in}}%
\pgfpathcurveto{\pgfqpoint{2.305676in}{3.072281in}}{\pgfqpoint{2.297776in}{3.069008in}}{\pgfqpoint{2.291952in}{3.063185in}}%
\pgfpathcurveto{\pgfqpoint{2.286128in}{3.057361in}}{\pgfqpoint{2.282855in}{3.049461in}}{\pgfqpoint{2.282855in}{3.041224in}}%
\pgfpathcurveto{\pgfqpoint{2.282855in}{3.032988in}}{\pgfqpoint{2.286128in}{3.025088in}}{\pgfqpoint{2.291952in}{3.019264in}}%
\pgfpathcurveto{\pgfqpoint{2.297776in}{3.013440in}}{\pgfqpoint{2.305676in}{3.010168in}}{\pgfqpoint{2.313912in}{3.010168in}}%
\pgfpathclose%
\pgfusepath{stroke,fill}%
\end{pgfscope}%
\begin{pgfscope}%
\pgfpathrectangle{\pgfqpoint{0.100000in}{0.212622in}}{\pgfqpoint{3.696000in}{3.696000in}}%
\pgfusepath{clip}%
\pgfsetbuttcap%
\pgfsetroundjoin%
\definecolor{currentfill}{rgb}{0.121569,0.466667,0.705882}%
\pgfsetfillcolor{currentfill}%
\pgfsetfillopacity{0.682308}%
\pgfsetlinewidth{1.003750pt}%
\definecolor{currentstroke}{rgb}{0.121569,0.466667,0.705882}%
\pgfsetstrokecolor{currentstroke}%
\pgfsetstrokeopacity{0.682308}%
\pgfsetdash{}{0pt}%
\pgfpathmoveto{\pgfqpoint{1.540114in}{2.841529in}}%
\pgfpathcurveto{\pgfqpoint{1.548351in}{2.841529in}}{\pgfqpoint{1.556251in}{2.844801in}}{\pgfqpoint{1.562075in}{2.850625in}}%
\pgfpathcurveto{\pgfqpoint{1.567898in}{2.856449in}}{\pgfqpoint{1.571171in}{2.864349in}}{\pgfqpoint{1.571171in}{2.872585in}}%
\pgfpathcurveto{\pgfqpoint{1.571171in}{2.880822in}}{\pgfqpoint{1.567898in}{2.888722in}}{\pgfqpoint{1.562075in}{2.894546in}}%
\pgfpathcurveto{\pgfqpoint{1.556251in}{2.900369in}}{\pgfqpoint{1.548351in}{2.903642in}}{\pgfqpoint{1.540114in}{2.903642in}}%
\pgfpathcurveto{\pgfqpoint{1.531878in}{2.903642in}}{\pgfqpoint{1.523978in}{2.900369in}}{\pgfqpoint{1.518154in}{2.894546in}}%
\pgfpathcurveto{\pgfqpoint{1.512330in}{2.888722in}}{\pgfqpoint{1.509058in}{2.880822in}}{\pgfqpoint{1.509058in}{2.872585in}}%
\pgfpathcurveto{\pgfqpoint{1.509058in}{2.864349in}}{\pgfqpoint{1.512330in}{2.856449in}}{\pgfqpoint{1.518154in}{2.850625in}}%
\pgfpathcurveto{\pgfqpoint{1.523978in}{2.844801in}}{\pgfqpoint{1.531878in}{2.841529in}}{\pgfqpoint{1.540114in}{2.841529in}}%
\pgfpathclose%
\pgfusepath{stroke,fill}%
\end{pgfscope}%
\begin{pgfscope}%
\pgfpathrectangle{\pgfqpoint{0.100000in}{0.212622in}}{\pgfqpoint{3.696000in}{3.696000in}}%
\pgfusepath{clip}%
\pgfsetbuttcap%
\pgfsetroundjoin%
\definecolor{currentfill}{rgb}{0.121569,0.466667,0.705882}%
\pgfsetfillcolor{currentfill}%
\pgfsetfillopacity{0.682513}%
\pgfsetlinewidth{1.003750pt}%
\definecolor{currentstroke}{rgb}{0.121569,0.466667,0.705882}%
\pgfsetstrokecolor{currentstroke}%
\pgfsetstrokeopacity{0.682513}%
\pgfsetdash{}{0pt}%
\pgfpathmoveto{\pgfqpoint{2.316493in}{3.009883in}}%
\pgfpathcurveto{\pgfqpoint{2.324730in}{3.009883in}}{\pgfqpoint{2.332630in}{3.013155in}}{\pgfqpoint{2.338454in}{3.018979in}}%
\pgfpathcurveto{\pgfqpoint{2.344278in}{3.024803in}}{\pgfqpoint{2.347550in}{3.032703in}}{\pgfqpoint{2.347550in}{3.040940in}}%
\pgfpathcurveto{\pgfqpoint{2.347550in}{3.049176in}}{\pgfqpoint{2.344278in}{3.057076in}}{\pgfqpoint{2.338454in}{3.062900in}}%
\pgfpathcurveto{\pgfqpoint{2.332630in}{3.068724in}}{\pgfqpoint{2.324730in}{3.071996in}}{\pgfqpoint{2.316493in}{3.071996in}}%
\pgfpathcurveto{\pgfqpoint{2.308257in}{3.071996in}}{\pgfqpoint{2.300357in}{3.068724in}}{\pgfqpoint{2.294533in}{3.062900in}}%
\pgfpathcurveto{\pgfqpoint{2.288709in}{3.057076in}}{\pgfqpoint{2.285437in}{3.049176in}}{\pgfqpoint{2.285437in}{3.040940in}}%
\pgfpathcurveto{\pgfqpoint{2.285437in}{3.032703in}}{\pgfqpoint{2.288709in}{3.024803in}}{\pgfqpoint{2.294533in}{3.018979in}}%
\pgfpathcurveto{\pgfqpoint{2.300357in}{3.013155in}}{\pgfqpoint{2.308257in}{3.009883in}}{\pgfqpoint{2.316493in}{3.009883in}}%
\pgfpathclose%
\pgfusepath{stroke,fill}%
\end{pgfscope}%
\begin{pgfscope}%
\pgfpathrectangle{\pgfqpoint{0.100000in}{0.212622in}}{\pgfqpoint{3.696000in}{3.696000in}}%
\pgfusepath{clip}%
\pgfsetbuttcap%
\pgfsetroundjoin%
\definecolor{currentfill}{rgb}{0.121569,0.466667,0.705882}%
\pgfsetfillcolor{currentfill}%
\pgfsetfillopacity{0.682555}%
\pgfsetlinewidth{1.003750pt}%
\definecolor{currentstroke}{rgb}{0.121569,0.466667,0.705882}%
\pgfsetstrokecolor{currentstroke}%
\pgfsetstrokeopacity{0.682555}%
\pgfsetdash{}{0pt}%
\pgfpathmoveto{\pgfqpoint{3.100052in}{1.791854in}}%
\pgfpathcurveto{\pgfqpoint{3.108288in}{1.791854in}}{\pgfqpoint{3.116188in}{1.795126in}}{\pgfqpoint{3.122012in}{1.800950in}}%
\pgfpathcurveto{\pgfqpoint{3.127836in}{1.806774in}}{\pgfqpoint{3.131108in}{1.814674in}}{\pgfqpoint{3.131108in}{1.822910in}}%
\pgfpathcurveto{\pgfqpoint{3.131108in}{1.831147in}}{\pgfqpoint{3.127836in}{1.839047in}}{\pgfqpoint{3.122012in}{1.844871in}}%
\pgfpathcurveto{\pgfqpoint{3.116188in}{1.850695in}}{\pgfqpoint{3.108288in}{1.853967in}}{\pgfqpoint{3.100052in}{1.853967in}}%
\pgfpathcurveto{\pgfqpoint{3.091815in}{1.853967in}}{\pgfqpoint{3.083915in}{1.850695in}}{\pgfqpoint{3.078091in}{1.844871in}}%
\pgfpathcurveto{\pgfqpoint{3.072267in}{1.839047in}}{\pgfqpoint{3.068995in}{1.831147in}}{\pgfqpoint{3.068995in}{1.822910in}}%
\pgfpathcurveto{\pgfqpoint{3.068995in}{1.814674in}}{\pgfqpoint{3.072267in}{1.806774in}}{\pgfqpoint{3.078091in}{1.800950in}}%
\pgfpathcurveto{\pgfqpoint{3.083915in}{1.795126in}}{\pgfqpoint{3.091815in}{1.791854in}}{\pgfqpoint{3.100052in}{1.791854in}}%
\pgfpathclose%
\pgfusepath{stroke,fill}%
\end{pgfscope}%
\begin{pgfscope}%
\pgfpathrectangle{\pgfqpoint{0.100000in}{0.212622in}}{\pgfqpoint{3.696000in}{3.696000in}}%
\pgfusepath{clip}%
\pgfsetbuttcap%
\pgfsetroundjoin%
\definecolor{currentfill}{rgb}{0.121569,0.466667,0.705882}%
\pgfsetfillcolor{currentfill}%
\pgfsetfillopacity{0.682965}%
\pgfsetlinewidth{1.003750pt}%
\definecolor{currentstroke}{rgb}{0.121569,0.466667,0.705882}%
\pgfsetstrokecolor{currentstroke}%
\pgfsetstrokeopacity{0.682965}%
\pgfsetdash{}{0pt}%
\pgfpathmoveto{\pgfqpoint{1.537861in}{2.838457in}}%
\pgfpathcurveto{\pgfqpoint{1.546097in}{2.838457in}}{\pgfqpoint{1.553997in}{2.841729in}}{\pgfqpoint{1.559821in}{2.847553in}}%
\pgfpathcurveto{\pgfqpoint{1.565645in}{2.853377in}}{\pgfqpoint{1.568917in}{2.861277in}}{\pgfqpoint{1.568917in}{2.869514in}}%
\pgfpathcurveto{\pgfqpoint{1.568917in}{2.877750in}}{\pgfqpoint{1.565645in}{2.885650in}}{\pgfqpoint{1.559821in}{2.891474in}}%
\pgfpathcurveto{\pgfqpoint{1.553997in}{2.897298in}}{\pgfqpoint{1.546097in}{2.900570in}}{\pgfqpoint{1.537861in}{2.900570in}}%
\pgfpathcurveto{\pgfqpoint{1.529625in}{2.900570in}}{\pgfqpoint{1.521725in}{2.897298in}}{\pgfqpoint{1.515901in}{2.891474in}}%
\pgfpathcurveto{\pgfqpoint{1.510077in}{2.885650in}}{\pgfqpoint{1.506804in}{2.877750in}}{\pgfqpoint{1.506804in}{2.869514in}}%
\pgfpathcurveto{\pgfqpoint{1.506804in}{2.861277in}}{\pgfqpoint{1.510077in}{2.853377in}}{\pgfqpoint{1.515901in}{2.847553in}}%
\pgfpathcurveto{\pgfqpoint{1.521725in}{2.841729in}}{\pgfqpoint{1.529625in}{2.838457in}}{\pgfqpoint{1.537861in}{2.838457in}}%
\pgfpathclose%
\pgfusepath{stroke,fill}%
\end{pgfscope}%
\begin{pgfscope}%
\pgfpathrectangle{\pgfqpoint{0.100000in}{0.212622in}}{\pgfqpoint{3.696000in}{3.696000in}}%
\pgfusepath{clip}%
\pgfsetbuttcap%
\pgfsetroundjoin%
\definecolor{currentfill}{rgb}{0.121569,0.466667,0.705882}%
\pgfsetfillcolor{currentfill}%
\pgfsetfillopacity{0.683138}%
\pgfsetlinewidth{1.003750pt}%
\definecolor{currentstroke}{rgb}{0.121569,0.466667,0.705882}%
\pgfsetstrokecolor{currentstroke}%
\pgfsetstrokeopacity{0.683138}%
\pgfsetdash{}{0pt}%
\pgfpathmoveto{\pgfqpoint{2.321234in}{3.009452in}}%
\pgfpathcurveto{\pgfqpoint{2.329470in}{3.009452in}}{\pgfqpoint{2.337370in}{3.012725in}}{\pgfqpoint{2.343194in}{3.018549in}}%
\pgfpathcurveto{\pgfqpoint{2.349018in}{3.024373in}}{\pgfqpoint{2.352290in}{3.032273in}}{\pgfqpoint{2.352290in}{3.040509in}}%
\pgfpathcurveto{\pgfqpoint{2.352290in}{3.048745in}}{\pgfqpoint{2.349018in}{3.056645in}}{\pgfqpoint{2.343194in}{3.062469in}}%
\pgfpathcurveto{\pgfqpoint{2.337370in}{3.068293in}}{\pgfqpoint{2.329470in}{3.071565in}}{\pgfqpoint{2.321234in}{3.071565in}}%
\pgfpathcurveto{\pgfqpoint{2.312998in}{3.071565in}}{\pgfqpoint{2.305098in}{3.068293in}}{\pgfqpoint{2.299274in}{3.062469in}}%
\pgfpathcurveto{\pgfqpoint{2.293450in}{3.056645in}}{\pgfqpoint{2.290177in}{3.048745in}}{\pgfqpoint{2.290177in}{3.040509in}}%
\pgfpathcurveto{\pgfqpoint{2.290177in}{3.032273in}}{\pgfqpoint{2.293450in}{3.024373in}}{\pgfqpoint{2.299274in}{3.018549in}}%
\pgfpathcurveto{\pgfqpoint{2.305098in}{3.012725in}}{\pgfqpoint{2.312998in}{3.009452in}}{\pgfqpoint{2.321234in}{3.009452in}}%
\pgfpathclose%
\pgfusepath{stroke,fill}%
\end{pgfscope}%
\begin{pgfscope}%
\pgfpathrectangle{\pgfqpoint{0.100000in}{0.212622in}}{\pgfqpoint{3.696000in}{3.696000in}}%
\pgfusepath{clip}%
\pgfsetbuttcap%
\pgfsetroundjoin%
\definecolor{currentfill}{rgb}{0.121569,0.466667,0.705882}%
\pgfsetfillcolor{currentfill}%
\pgfsetfillopacity{0.683177}%
\pgfsetlinewidth{1.003750pt}%
\definecolor{currentstroke}{rgb}{0.121569,0.466667,0.705882}%
\pgfsetstrokecolor{currentstroke}%
\pgfsetstrokeopacity{0.683177}%
\pgfsetdash{}{0pt}%
\pgfpathmoveto{\pgfqpoint{3.098898in}{1.791026in}}%
\pgfpathcurveto{\pgfqpoint{3.107134in}{1.791026in}}{\pgfqpoint{3.115034in}{1.794298in}}{\pgfqpoint{3.120858in}{1.800122in}}%
\pgfpathcurveto{\pgfqpoint{3.126682in}{1.805946in}}{\pgfqpoint{3.129954in}{1.813846in}}{\pgfqpoint{3.129954in}{1.822082in}}%
\pgfpathcurveto{\pgfqpoint{3.129954in}{1.830318in}}{\pgfqpoint{3.126682in}{1.838218in}}{\pgfqpoint{3.120858in}{1.844042in}}%
\pgfpathcurveto{\pgfqpoint{3.115034in}{1.849866in}}{\pgfqpoint{3.107134in}{1.853139in}}{\pgfqpoint{3.098898in}{1.853139in}}%
\pgfpathcurveto{\pgfqpoint{3.090661in}{1.853139in}}{\pgfqpoint{3.082761in}{1.849866in}}{\pgfqpoint{3.076938in}{1.844042in}}%
\pgfpathcurveto{\pgfqpoint{3.071114in}{1.838218in}}{\pgfqpoint{3.067841in}{1.830318in}}{\pgfqpoint{3.067841in}{1.822082in}}%
\pgfpathcurveto{\pgfqpoint{3.067841in}{1.813846in}}{\pgfqpoint{3.071114in}{1.805946in}}{\pgfqpoint{3.076938in}{1.800122in}}%
\pgfpathcurveto{\pgfqpoint{3.082761in}{1.794298in}}{\pgfqpoint{3.090661in}{1.791026in}}{\pgfqpoint{3.098898in}{1.791026in}}%
\pgfpathclose%
\pgfusepath{stroke,fill}%
\end{pgfscope}%
\begin{pgfscope}%
\pgfpathrectangle{\pgfqpoint{0.100000in}{0.212622in}}{\pgfqpoint{3.696000in}{3.696000in}}%
\pgfusepath{clip}%
\pgfsetbuttcap%
\pgfsetroundjoin%
\definecolor{currentfill}{rgb}{0.121569,0.466667,0.705882}%
\pgfsetfillcolor{currentfill}%
\pgfsetfillopacity{0.683349}%
\pgfsetlinewidth{1.003750pt}%
\definecolor{currentstroke}{rgb}{0.121569,0.466667,0.705882}%
\pgfsetstrokecolor{currentstroke}%
\pgfsetstrokeopacity{0.683349}%
\pgfsetdash{}{0pt}%
\pgfpathmoveto{\pgfqpoint{1.536622in}{2.836896in}}%
\pgfpathcurveto{\pgfqpoint{1.544858in}{2.836896in}}{\pgfqpoint{1.552758in}{2.840168in}}{\pgfqpoint{1.558582in}{2.845992in}}%
\pgfpathcurveto{\pgfqpoint{1.564406in}{2.851816in}}{\pgfqpoint{1.567678in}{2.859716in}}{\pgfqpoint{1.567678in}{2.867952in}}%
\pgfpathcurveto{\pgfqpoint{1.567678in}{2.876189in}}{\pgfqpoint{1.564406in}{2.884089in}}{\pgfqpoint{1.558582in}{2.889913in}}%
\pgfpathcurveto{\pgfqpoint{1.552758in}{2.895737in}}{\pgfqpoint{1.544858in}{2.899009in}}{\pgfqpoint{1.536622in}{2.899009in}}%
\pgfpathcurveto{\pgfqpoint{1.528386in}{2.899009in}}{\pgfqpoint{1.520486in}{2.895737in}}{\pgfqpoint{1.514662in}{2.889913in}}%
\pgfpathcurveto{\pgfqpoint{1.508838in}{2.884089in}}{\pgfqpoint{1.505565in}{2.876189in}}{\pgfqpoint{1.505565in}{2.867952in}}%
\pgfpathcurveto{\pgfqpoint{1.505565in}{2.859716in}}{\pgfqpoint{1.508838in}{2.851816in}}{\pgfqpoint{1.514662in}{2.845992in}}%
\pgfpathcurveto{\pgfqpoint{1.520486in}{2.840168in}}{\pgfqpoint{1.528386in}{2.836896in}}{\pgfqpoint{1.536622in}{2.836896in}}%
\pgfpathclose%
\pgfusepath{stroke,fill}%
\end{pgfscope}%
\begin{pgfscope}%
\pgfpathrectangle{\pgfqpoint{0.100000in}{0.212622in}}{\pgfqpoint{3.696000in}{3.696000in}}%
\pgfusepath{clip}%
\pgfsetbuttcap%
\pgfsetroundjoin%
\definecolor{currentfill}{rgb}{0.121569,0.466667,0.705882}%
\pgfsetfillcolor{currentfill}%
\pgfsetfillopacity{0.683557}%
\pgfsetlinewidth{1.003750pt}%
\definecolor{currentstroke}{rgb}{0.121569,0.466667,0.705882}%
\pgfsetstrokecolor{currentstroke}%
\pgfsetstrokeopacity{0.683557}%
\pgfsetdash{}{0pt}%
\pgfpathmoveto{\pgfqpoint{2.324531in}{3.009216in}}%
\pgfpathcurveto{\pgfqpoint{2.332767in}{3.009216in}}{\pgfqpoint{2.340667in}{3.012488in}}{\pgfqpoint{2.346491in}{3.018312in}}%
\pgfpathcurveto{\pgfqpoint{2.352315in}{3.024136in}}{\pgfqpoint{2.355587in}{3.032036in}}{\pgfqpoint{2.355587in}{3.040272in}}%
\pgfpathcurveto{\pgfqpoint{2.355587in}{3.048509in}}{\pgfqpoint{2.352315in}{3.056409in}}{\pgfqpoint{2.346491in}{3.062233in}}%
\pgfpathcurveto{\pgfqpoint{2.340667in}{3.068057in}}{\pgfqpoint{2.332767in}{3.071329in}}{\pgfqpoint{2.324531in}{3.071329in}}%
\pgfpathcurveto{\pgfqpoint{2.316294in}{3.071329in}}{\pgfqpoint{2.308394in}{3.068057in}}{\pgfqpoint{2.302571in}{3.062233in}}%
\pgfpathcurveto{\pgfqpoint{2.296747in}{3.056409in}}{\pgfqpoint{2.293474in}{3.048509in}}{\pgfqpoint{2.293474in}{3.040272in}}%
\pgfpathcurveto{\pgfqpoint{2.293474in}{3.032036in}}{\pgfqpoint{2.296747in}{3.024136in}}{\pgfqpoint{2.302571in}{3.018312in}}%
\pgfpathcurveto{\pgfqpoint{2.308394in}{3.012488in}}{\pgfqpoint{2.316294in}{3.009216in}}{\pgfqpoint{2.324531in}{3.009216in}}%
\pgfpathclose%
\pgfusepath{stroke,fill}%
\end{pgfscope}%
\begin{pgfscope}%
\pgfpathrectangle{\pgfqpoint{0.100000in}{0.212622in}}{\pgfqpoint{3.696000in}{3.696000in}}%
\pgfusepath{clip}%
\pgfsetbuttcap%
\pgfsetroundjoin%
\definecolor{currentfill}{rgb}{0.121569,0.466667,0.705882}%
\pgfsetfillcolor{currentfill}%
\pgfsetfillopacity{0.683561}%
\pgfsetlinewidth{1.003750pt}%
\definecolor{currentstroke}{rgb}{0.121569,0.466667,0.705882}%
\pgfsetstrokecolor{currentstroke}%
\pgfsetstrokeopacity{0.683561}%
\pgfsetdash{}{0pt}%
\pgfpathmoveto{\pgfqpoint{3.098123in}{1.790373in}}%
\pgfpathcurveto{\pgfqpoint{3.106360in}{1.790373in}}{\pgfqpoint{3.114260in}{1.793645in}}{\pgfqpoint{3.120084in}{1.799469in}}%
\pgfpathcurveto{\pgfqpoint{3.125908in}{1.805293in}}{\pgfqpoint{3.129180in}{1.813193in}}{\pgfqpoint{3.129180in}{1.821429in}}%
\pgfpathcurveto{\pgfqpoint{3.129180in}{1.829666in}}{\pgfqpoint{3.125908in}{1.837566in}}{\pgfqpoint{3.120084in}{1.843390in}}%
\pgfpathcurveto{\pgfqpoint{3.114260in}{1.849213in}}{\pgfqpoint{3.106360in}{1.852486in}}{\pgfqpoint{3.098123in}{1.852486in}}%
\pgfpathcurveto{\pgfqpoint{3.089887in}{1.852486in}}{\pgfqpoint{3.081987in}{1.849213in}}{\pgfqpoint{3.076163in}{1.843390in}}%
\pgfpathcurveto{\pgfqpoint{3.070339in}{1.837566in}}{\pgfqpoint{3.067067in}{1.829666in}}{\pgfqpoint{3.067067in}{1.821429in}}%
\pgfpathcurveto{\pgfqpoint{3.067067in}{1.813193in}}{\pgfqpoint{3.070339in}{1.805293in}}{\pgfqpoint{3.076163in}{1.799469in}}%
\pgfpathcurveto{\pgfqpoint{3.081987in}{1.793645in}}{\pgfqpoint{3.089887in}{1.790373in}}{\pgfqpoint{3.098123in}{1.790373in}}%
\pgfpathclose%
\pgfusepath{stroke,fill}%
\end{pgfscope}%
\begin{pgfscope}%
\pgfpathrectangle{\pgfqpoint{0.100000in}{0.212622in}}{\pgfqpoint{3.696000in}{3.696000in}}%
\pgfusepath{clip}%
\pgfsetbuttcap%
\pgfsetroundjoin%
\definecolor{currentfill}{rgb}{0.121569,0.466667,0.705882}%
\pgfsetfillcolor{currentfill}%
\pgfsetfillopacity{0.683843}%
\pgfsetlinewidth{1.003750pt}%
\definecolor{currentstroke}{rgb}{0.121569,0.466667,0.705882}%
\pgfsetstrokecolor{currentstroke}%
\pgfsetstrokeopacity{0.683843}%
\pgfsetdash{}{0pt}%
\pgfpathmoveto{\pgfqpoint{1.534906in}{2.834747in}}%
\pgfpathcurveto{\pgfqpoint{1.543142in}{2.834747in}}{\pgfqpoint{1.551042in}{2.838019in}}{\pgfqpoint{1.556866in}{2.843843in}}%
\pgfpathcurveto{\pgfqpoint{1.562690in}{2.849667in}}{\pgfqpoint{1.565962in}{2.857567in}}{\pgfqpoint{1.565962in}{2.865803in}}%
\pgfpathcurveto{\pgfqpoint{1.565962in}{2.874040in}}{\pgfqpoint{1.562690in}{2.881940in}}{\pgfqpoint{1.556866in}{2.887764in}}%
\pgfpathcurveto{\pgfqpoint{1.551042in}{2.893588in}}{\pgfqpoint{1.543142in}{2.896860in}}{\pgfqpoint{1.534906in}{2.896860in}}%
\pgfpathcurveto{\pgfqpoint{1.526669in}{2.896860in}}{\pgfqpoint{1.518769in}{2.893588in}}{\pgfqpoint{1.512945in}{2.887764in}}%
\pgfpathcurveto{\pgfqpoint{1.507121in}{2.881940in}}{\pgfqpoint{1.503849in}{2.874040in}}{\pgfqpoint{1.503849in}{2.865803in}}%
\pgfpathcurveto{\pgfqpoint{1.503849in}{2.857567in}}{\pgfqpoint{1.507121in}{2.849667in}}{\pgfqpoint{1.512945in}{2.843843in}}%
\pgfpathcurveto{\pgfqpoint{1.518769in}{2.838019in}}{\pgfqpoint{1.526669in}{2.834747in}}{\pgfqpoint{1.534906in}{2.834747in}}%
\pgfpathclose%
\pgfusepath{stroke,fill}%
\end{pgfscope}%
\begin{pgfscope}%
\pgfpathrectangle{\pgfqpoint{0.100000in}{0.212622in}}{\pgfqpoint{3.696000in}{3.696000in}}%
\pgfusepath{clip}%
\pgfsetbuttcap%
\pgfsetroundjoin%
\definecolor{currentfill}{rgb}{0.121569,0.466667,0.705882}%
\pgfsetfillcolor{currentfill}%
\pgfsetfillopacity{0.684256}%
\pgfsetlinewidth{1.003750pt}%
\definecolor{currentstroke}{rgb}{0.121569,0.466667,0.705882}%
\pgfsetstrokecolor{currentstroke}%
\pgfsetstrokeopacity{0.684256}%
\pgfsetdash{}{0pt}%
\pgfpathmoveto{\pgfqpoint{3.096708in}{1.789161in}}%
\pgfpathcurveto{\pgfqpoint{3.104945in}{1.789161in}}{\pgfqpoint{3.112845in}{1.792433in}}{\pgfqpoint{3.118669in}{1.798257in}}%
\pgfpathcurveto{\pgfqpoint{3.124493in}{1.804081in}}{\pgfqpoint{3.127765in}{1.811981in}}{\pgfqpoint{3.127765in}{1.820217in}}%
\pgfpathcurveto{\pgfqpoint{3.127765in}{1.828453in}}{\pgfqpoint{3.124493in}{1.836353in}}{\pgfqpoint{3.118669in}{1.842177in}}%
\pgfpathcurveto{\pgfqpoint{3.112845in}{1.848001in}}{\pgfqpoint{3.104945in}{1.851274in}}{\pgfqpoint{3.096708in}{1.851274in}}%
\pgfpathcurveto{\pgfqpoint{3.088472in}{1.851274in}}{\pgfqpoint{3.080572in}{1.848001in}}{\pgfqpoint{3.074748in}{1.842177in}}%
\pgfpathcurveto{\pgfqpoint{3.068924in}{1.836353in}}{\pgfqpoint{3.065652in}{1.828453in}}{\pgfqpoint{3.065652in}{1.820217in}}%
\pgfpathcurveto{\pgfqpoint{3.065652in}{1.811981in}}{\pgfqpoint{3.068924in}{1.804081in}}{\pgfqpoint{3.074748in}{1.798257in}}%
\pgfpathcurveto{\pgfqpoint{3.080572in}{1.792433in}}{\pgfqpoint{3.088472in}{1.789161in}}{\pgfqpoint{3.096708in}{1.789161in}}%
\pgfpathclose%
\pgfusepath{stroke,fill}%
\end{pgfscope}%
\begin{pgfscope}%
\pgfpathrectangle{\pgfqpoint{0.100000in}{0.212622in}}{\pgfqpoint{3.696000in}{3.696000in}}%
\pgfusepath{clip}%
\pgfsetbuttcap%
\pgfsetroundjoin%
\definecolor{currentfill}{rgb}{0.121569,0.466667,0.705882}%
\pgfsetfillcolor{currentfill}%
\pgfsetfillopacity{0.684334}%
\pgfsetlinewidth{1.003750pt}%
\definecolor{currentstroke}{rgb}{0.121569,0.466667,0.705882}%
\pgfsetstrokecolor{currentstroke}%
\pgfsetstrokeopacity{0.684334}%
\pgfsetdash{}{0pt}%
\pgfpathmoveto{\pgfqpoint{2.330489in}{3.008734in}}%
\pgfpathcurveto{\pgfqpoint{2.338725in}{3.008734in}}{\pgfqpoint{2.346625in}{3.012007in}}{\pgfqpoint{2.352449in}{3.017830in}}%
\pgfpathcurveto{\pgfqpoint{2.358273in}{3.023654in}}{\pgfqpoint{2.361546in}{3.031554in}}{\pgfqpoint{2.361546in}{3.039791in}}%
\pgfpathcurveto{\pgfqpoint{2.361546in}{3.048027in}}{\pgfqpoint{2.358273in}{3.055927in}}{\pgfqpoint{2.352449in}{3.061751in}}%
\pgfpathcurveto{\pgfqpoint{2.346625in}{3.067575in}}{\pgfqpoint{2.338725in}{3.070847in}}{\pgfqpoint{2.330489in}{3.070847in}}%
\pgfpathcurveto{\pgfqpoint{2.322253in}{3.070847in}}{\pgfqpoint{2.314353in}{3.067575in}}{\pgfqpoint{2.308529in}{3.061751in}}%
\pgfpathcurveto{\pgfqpoint{2.302705in}{3.055927in}}{\pgfqpoint{2.299433in}{3.048027in}}{\pgfqpoint{2.299433in}{3.039791in}}%
\pgfpathcurveto{\pgfqpoint{2.299433in}{3.031554in}}{\pgfqpoint{2.302705in}{3.023654in}}{\pgfqpoint{2.308529in}{3.017830in}}%
\pgfpathcurveto{\pgfqpoint{2.314353in}{3.012007in}}{\pgfqpoint{2.322253in}{3.008734in}}{\pgfqpoint{2.330489in}{3.008734in}}%
\pgfpathclose%
\pgfusepath{stroke,fill}%
\end{pgfscope}%
\begin{pgfscope}%
\pgfpathrectangle{\pgfqpoint{0.100000in}{0.212622in}}{\pgfqpoint{3.696000in}{3.696000in}}%
\pgfusepath{clip}%
\pgfsetbuttcap%
\pgfsetroundjoin%
\definecolor{currentfill}{rgb}{0.121569,0.466667,0.705882}%
\pgfsetfillcolor{currentfill}%
\pgfsetfillopacity{0.684608}%
\pgfsetlinewidth{1.003750pt}%
\definecolor{currentstroke}{rgb}{0.121569,0.466667,0.705882}%
\pgfsetstrokecolor{currentstroke}%
\pgfsetstrokeopacity{0.684608}%
\pgfsetdash{}{0pt}%
\pgfpathmoveto{\pgfqpoint{1.532062in}{2.830853in}}%
\pgfpathcurveto{\pgfqpoint{1.540298in}{2.830853in}}{\pgfqpoint{1.548198in}{2.834125in}}{\pgfqpoint{1.554022in}{2.839949in}}%
\pgfpathcurveto{\pgfqpoint{1.559846in}{2.845773in}}{\pgfqpoint{1.563118in}{2.853673in}}{\pgfqpoint{1.563118in}{2.861909in}}%
\pgfpathcurveto{\pgfqpoint{1.563118in}{2.870145in}}{\pgfqpoint{1.559846in}{2.878046in}}{\pgfqpoint{1.554022in}{2.883869in}}%
\pgfpathcurveto{\pgfqpoint{1.548198in}{2.889693in}}{\pgfqpoint{1.540298in}{2.892966in}}{\pgfqpoint{1.532062in}{2.892966in}}%
\pgfpathcurveto{\pgfqpoint{1.523826in}{2.892966in}}{\pgfqpoint{1.515926in}{2.889693in}}{\pgfqpoint{1.510102in}{2.883869in}}%
\pgfpathcurveto{\pgfqpoint{1.504278in}{2.878046in}}{\pgfqpoint{1.501005in}{2.870145in}}{\pgfqpoint{1.501005in}{2.861909in}}%
\pgfpathcurveto{\pgfqpoint{1.501005in}{2.853673in}}{\pgfqpoint{1.504278in}{2.845773in}}{\pgfqpoint{1.510102in}{2.839949in}}%
\pgfpathcurveto{\pgfqpoint{1.515926in}{2.834125in}}{\pgfqpoint{1.523826in}{2.830853in}}{\pgfqpoint{1.532062in}{2.830853in}}%
\pgfpathclose%
\pgfusepath{stroke,fill}%
\end{pgfscope}%
\begin{pgfscope}%
\pgfpathrectangle{\pgfqpoint{0.100000in}{0.212622in}}{\pgfqpoint{3.696000in}{3.696000in}}%
\pgfusepath{clip}%
\pgfsetbuttcap%
\pgfsetroundjoin%
\definecolor{currentfill}{rgb}{0.121569,0.466667,0.705882}%
\pgfsetfillcolor{currentfill}%
\pgfsetfillopacity{0.684844}%
\pgfsetlinewidth{1.003750pt}%
\definecolor{currentstroke}{rgb}{0.121569,0.466667,0.705882}%
\pgfsetstrokecolor{currentstroke}%
\pgfsetstrokeopacity{0.684844}%
\pgfsetdash{}{0pt}%
\pgfpathmoveto{\pgfqpoint{3.095470in}{1.788027in}}%
\pgfpathcurveto{\pgfqpoint{3.103707in}{1.788027in}}{\pgfqpoint{3.111607in}{1.791299in}}{\pgfqpoint{3.117431in}{1.797123in}}%
\pgfpathcurveto{\pgfqpoint{3.123255in}{1.802947in}}{\pgfqpoint{3.126527in}{1.810847in}}{\pgfqpoint{3.126527in}{1.819083in}}%
\pgfpathcurveto{\pgfqpoint{3.126527in}{1.827320in}}{\pgfqpoint{3.123255in}{1.835220in}}{\pgfqpoint{3.117431in}{1.841044in}}%
\pgfpathcurveto{\pgfqpoint{3.111607in}{1.846867in}}{\pgfqpoint{3.103707in}{1.850140in}}{\pgfqpoint{3.095470in}{1.850140in}}%
\pgfpathcurveto{\pgfqpoint{3.087234in}{1.850140in}}{\pgfqpoint{3.079334in}{1.846867in}}{\pgfqpoint{3.073510in}{1.841044in}}%
\pgfpathcurveto{\pgfqpoint{3.067686in}{1.835220in}}{\pgfqpoint{3.064414in}{1.827320in}}{\pgfqpoint{3.064414in}{1.819083in}}%
\pgfpathcurveto{\pgfqpoint{3.064414in}{1.810847in}}{\pgfqpoint{3.067686in}{1.802947in}}{\pgfqpoint{3.073510in}{1.797123in}}%
\pgfpathcurveto{\pgfqpoint{3.079334in}{1.791299in}}{\pgfqpoint{3.087234in}{1.788027in}}{\pgfqpoint{3.095470in}{1.788027in}}%
\pgfpathclose%
\pgfusepath{stroke,fill}%
\end{pgfscope}%
\begin{pgfscope}%
\pgfpathrectangle{\pgfqpoint{0.100000in}{0.212622in}}{\pgfqpoint{3.696000in}{3.696000in}}%
\pgfusepath{clip}%
\pgfsetbuttcap%
\pgfsetroundjoin%
\definecolor{currentfill}{rgb}{0.121569,0.466667,0.705882}%
\pgfsetfillcolor{currentfill}%
\pgfsetfillopacity{0.684927}%
\pgfsetlinewidth{1.003750pt}%
\definecolor{currentstroke}{rgb}{0.121569,0.466667,0.705882}%
\pgfsetstrokecolor{currentstroke}%
\pgfsetstrokeopacity{0.684927}%
\pgfsetdash{}{0pt}%
\pgfpathmoveto{\pgfqpoint{2.335204in}{3.008055in}}%
\pgfpathcurveto{\pgfqpoint{2.343440in}{3.008055in}}{\pgfqpoint{2.351340in}{3.011327in}}{\pgfqpoint{2.357164in}{3.017151in}}%
\pgfpathcurveto{\pgfqpoint{2.362988in}{3.022975in}}{\pgfqpoint{2.366260in}{3.030875in}}{\pgfqpoint{2.366260in}{3.039111in}}%
\pgfpathcurveto{\pgfqpoint{2.366260in}{3.047348in}}{\pgfqpoint{2.362988in}{3.055248in}}{\pgfqpoint{2.357164in}{3.061072in}}%
\pgfpathcurveto{\pgfqpoint{2.351340in}{3.066896in}}{\pgfqpoint{2.343440in}{3.070168in}}{\pgfqpoint{2.335204in}{3.070168in}}%
\pgfpathcurveto{\pgfqpoint{2.326967in}{3.070168in}}{\pgfqpoint{2.319067in}{3.066896in}}{\pgfqpoint{2.313243in}{3.061072in}}%
\pgfpathcurveto{\pgfqpoint{2.307419in}{3.055248in}}{\pgfqpoint{2.304147in}{3.047348in}}{\pgfqpoint{2.304147in}{3.039111in}}%
\pgfpathcurveto{\pgfqpoint{2.304147in}{3.030875in}}{\pgfqpoint{2.307419in}{3.022975in}}{\pgfqpoint{2.313243in}{3.017151in}}%
\pgfpathcurveto{\pgfqpoint{2.319067in}{3.011327in}}{\pgfqpoint{2.326967in}{3.008055in}}{\pgfqpoint{2.335204in}{3.008055in}}%
\pgfpathclose%
\pgfusepath{stroke,fill}%
\end{pgfscope}%
\begin{pgfscope}%
\pgfpathrectangle{\pgfqpoint{0.100000in}{0.212622in}}{\pgfqpoint{3.696000in}{3.696000in}}%
\pgfusepath{clip}%
\pgfsetbuttcap%
\pgfsetroundjoin%
\definecolor{currentfill}{rgb}{0.121569,0.466667,0.705882}%
\pgfsetfillcolor{currentfill}%
\pgfsetfillopacity{0.685528}%
\pgfsetlinewidth{1.003750pt}%
\definecolor{currentstroke}{rgb}{0.121569,0.466667,0.705882}%
\pgfsetstrokecolor{currentstroke}%
\pgfsetstrokeopacity{0.685528}%
\pgfsetdash{}{0pt}%
\pgfpathmoveto{\pgfqpoint{1.528792in}{2.825888in}}%
\pgfpathcurveto{\pgfqpoint{1.537028in}{2.825888in}}{\pgfqpoint{1.544929in}{2.829161in}}{\pgfqpoint{1.550752in}{2.834985in}}%
\pgfpathcurveto{\pgfqpoint{1.556576in}{2.840809in}}{\pgfqpoint{1.559849in}{2.848709in}}{\pgfqpoint{1.559849in}{2.856945in}}%
\pgfpathcurveto{\pgfqpoint{1.559849in}{2.865181in}}{\pgfqpoint{1.556576in}{2.873081in}}{\pgfqpoint{1.550752in}{2.878905in}}%
\pgfpathcurveto{\pgfqpoint{1.544929in}{2.884729in}}{\pgfqpoint{1.537028in}{2.888001in}}{\pgfqpoint{1.528792in}{2.888001in}}%
\pgfpathcurveto{\pgfqpoint{1.520556in}{2.888001in}}{\pgfqpoint{1.512656in}{2.884729in}}{\pgfqpoint{1.506832in}{2.878905in}}%
\pgfpathcurveto{\pgfqpoint{1.501008in}{2.873081in}}{\pgfqpoint{1.497736in}{2.865181in}}{\pgfqpoint{1.497736in}{2.856945in}}%
\pgfpathcurveto{\pgfqpoint{1.497736in}{2.848709in}}{\pgfqpoint{1.501008in}{2.840809in}}{\pgfqpoint{1.506832in}{2.834985in}}%
\pgfpathcurveto{\pgfqpoint{1.512656in}{2.829161in}}{\pgfqpoint{1.520556in}{2.825888in}}{\pgfqpoint{1.528792in}{2.825888in}}%
\pgfpathclose%
\pgfusepath{stroke,fill}%
\end{pgfscope}%
\begin{pgfscope}%
\pgfpathrectangle{\pgfqpoint{0.100000in}{0.212622in}}{\pgfqpoint{3.696000in}{3.696000in}}%
\pgfusepath{clip}%
\pgfsetbuttcap%
\pgfsetroundjoin%
\definecolor{currentfill}{rgb}{0.121569,0.466667,0.705882}%
\pgfsetfillcolor{currentfill}%
\pgfsetfillopacity{0.685894}%
\pgfsetlinewidth{1.003750pt}%
\definecolor{currentstroke}{rgb}{0.121569,0.466667,0.705882}%
\pgfsetstrokecolor{currentstroke}%
\pgfsetstrokeopacity{0.685894}%
\pgfsetdash{}{0pt}%
\pgfpathmoveto{\pgfqpoint{3.093163in}{1.785916in}}%
\pgfpathcurveto{\pgfqpoint{3.101399in}{1.785916in}}{\pgfqpoint{3.109299in}{1.789188in}}{\pgfqpoint{3.115123in}{1.795012in}}%
\pgfpathcurveto{\pgfqpoint{3.120947in}{1.800836in}}{\pgfqpoint{3.124219in}{1.808736in}}{\pgfqpoint{3.124219in}{1.816973in}}%
\pgfpathcurveto{\pgfqpoint{3.124219in}{1.825209in}}{\pgfqpoint{3.120947in}{1.833109in}}{\pgfqpoint{3.115123in}{1.838933in}}%
\pgfpathcurveto{\pgfqpoint{3.109299in}{1.844757in}}{\pgfqpoint{3.101399in}{1.848029in}}{\pgfqpoint{3.093163in}{1.848029in}}%
\pgfpathcurveto{\pgfqpoint{3.084926in}{1.848029in}}{\pgfqpoint{3.077026in}{1.844757in}}{\pgfqpoint{3.071202in}{1.838933in}}%
\pgfpathcurveto{\pgfqpoint{3.065378in}{1.833109in}}{\pgfqpoint{3.062106in}{1.825209in}}{\pgfqpoint{3.062106in}{1.816973in}}%
\pgfpathcurveto{\pgfqpoint{3.062106in}{1.808736in}}{\pgfqpoint{3.065378in}{1.800836in}}{\pgfqpoint{3.071202in}{1.795012in}}%
\pgfpathcurveto{\pgfqpoint{3.077026in}{1.789188in}}{\pgfqpoint{3.084926in}{1.785916in}}{\pgfqpoint{3.093163in}{1.785916in}}%
\pgfpathclose%
\pgfusepath{stroke,fill}%
\end{pgfscope}%
\begin{pgfscope}%
\pgfpathrectangle{\pgfqpoint{0.100000in}{0.212622in}}{\pgfqpoint{3.696000in}{3.696000in}}%
\pgfusepath{clip}%
\pgfsetbuttcap%
\pgfsetroundjoin%
\definecolor{currentfill}{rgb}{0.121569,0.466667,0.705882}%
\pgfsetfillcolor{currentfill}%
\pgfsetfillopacity{0.686024}%
\pgfsetlinewidth{1.003750pt}%
\definecolor{currentstroke}{rgb}{0.121569,0.466667,0.705882}%
\pgfsetstrokecolor{currentstroke}%
\pgfsetstrokeopacity{0.686024}%
\pgfsetdash{}{0pt}%
\pgfpathmoveto{\pgfqpoint{2.343795in}{3.007007in}}%
\pgfpathcurveto{\pgfqpoint{2.352031in}{3.007007in}}{\pgfqpoint{2.359931in}{3.010279in}}{\pgfqpoint{2.365755in}{3.016103in}}%
\pgfpathcurveto{\pgfqpoint{2.371579in}{3.021927in}}{\pgfqpoint{2.374852in}{3.029827in}}{\pgfqpoint{2.374852in}{3.038063in}}%
\pgfpathcurveto{\pgfqpoint{2.374852in}{3.046299in}}{\pgfqpoint{2.371579in}{3.054199in}}{\pgfqpoint{2.365755in}{3.060023in}}%
\pgfpathcurveto{\pgfqpoint{2.359931in}{3.065847in}}{\pgfqpoint{2.352031in}{3.069120in}}{\pgfqpoint{2.343795in}{3.069120in}}%
\pgfpathcurveto{\pgfqpoint{2.335559in}{3.069120in}}{\pgfqpoint{2.327659in}{3.065847in}}{\pgfqpoint{2.321835in}{3.060023in}}%
\pgfpathcurveto{\pgfqpoint{2.316011in}{3.054199in}}{\pgfqpoint{2.312739in}{3.046299in}}{\pgfqpoint{2.312739in}{3.038063in}}%
\pgfpathcurveto{\pgfqpoint{2.312739in}{3.029827in}}{\pgfqpoint{2.316011in}{3.021927in}}{\pgfqpoint{2.321835in}{3.016103in}}%
\pgfpathcurveto{\pgfqpoint{2.327659in}{3.010279in}}{\pgfqpoint{2.335559in}{3.007007in}}{\pgfqpoint{2.343795in}{3.007007in}}%
\pgfpathclose%
\pgfusepath{stroke,fill}%
\end{pgfscope}%
\begin{pgfscope}%
\pgfpathrectangle{\pgfqpoint{0.100000in}{0.212622in}}{\pgfqpoint{3.696000in}{3.696000in}}%
\pgfusepath{clip}%
\pgfsetbuttcap%
\pgfsetroundjoin%
\definecolor{currentfill}{rgb}{0.121569,0.466667,0.705882}%
\pgfsetfillcolor{currentfill}%
\pgfsetfillopacity{0.686044}%
\pgfsetlinewidth{1.003750pt}%
\definecolor{currentstroke}{rgb}{0.121569,0.466667,0.705882}%
\pgfsetstrokecolor{currentstroke}%
\pgfsetstrokeopacity{0.686044}%
\pgfsetdash{}{0pt}%
\pgfpathmoveto{\pgfqpoint{1.527000in}{2.823212in}}%
\pgfpathcurveto{\pgfqpoint{1.535236in}{2.823212in}}{\pgfqpoint{1.543137in}{2.826484in}}{\pgfqpoint{1.548960in}{2.832308in}}%
\pgfpathcurveto{\pgfqpoint{1.554784in}{2.838132in}}{\pgfqpoint{1.558057in}{2.846032in}}{\pgfqpoint{1.558057in}{2.854268in}}%
\pgfpathcurveto{\pgfqpoint{1.558057in}{2.862504in}}{\pgfqpoint{1.554784in}{2.870404in}}{\pgfqpoint{1.548960in}{2.876228in}}%
\pgfpathcurveto{\pgfqpoint{1.543137in}{2.882052in}}{\pgfqpoint{1.535236in}{2.885325in}}{\pgfqpoint{1.527000in}{2.885325in}}%
\pgfpathcurveto{\pgfqpoint{1.518764in}{2.885325in}}{\pgfqpoint{1.510864in}{2.882052in}}{\pgfqpoint{1.505040in}{2.876228in}}%
\pgfpathcurveto{\pgfqpoint{1.499216in}{2.870404in}}{\pgfqpoint{1.495944in}{2.862504in}}{\pgfqpoint{1.495944in}{2.854268in}}%
\pgfpathcurveto{\pgfqpoint{1.495944in}{2.846032in}}{\pgfqpoint{1.499216in}{2.838132in}}{\pgfqpoint{1.505040in}{2.832308in}}%
\pgfpathcurveto{\pgfqpoint{1.510864in}{2.826484in}}{\pgfqpoint{1.518764in}{2.823212in}}{\pgfqpoint{1.527000in}{2.823212in}}%
\pgfpathclose%
\pgfusepath{stroke,fill}%
\end{pgfscope}%
\begin{pgfscope}%
\pgfpathrectangle{\pgfqpoint{0.100000in}{0.212622in}}{\pgfqpoint{3.696000in}{3.696000in}}%
\pgfusepath{clip}%
\pgfsetbuttcap%
\pgfsetroundjoin%
\definecolor{currentfill}{rgb}{0.121569,0.466667,0.705882}%
\pgfsetfillcolor{currentfill}%
\pgfsetfillopacity{0.686347}%
\pgfsetlinewidth{1.003750pt}%
\definecolor{currentstroke}{rgb}{0.121569,0.466667,0.705882}%
\pgfsetstrokecolor{currentstroke}%
\pgfsetstrokeopacity{0.686347}%
\pgfsetdash{}{0pt}%
\pgfpathmoveto{\pgfqpoint{1.526002in}{2.821871in}}%
\pgfpathcurveto{\pgfqpoint{1.534238in}{2.821871in}}{\pgfqpoint{1.542138in}{2.825143in}}{\pgfqpoint{1.547962in}{2.830967in}}%
\pgfpathcurveto{\pgfqpoint{1.553786in}{2.836791in}}{\pgfqpoint{1.557058in}{2.844691in}}{\pgfqpoint{1.557058in}{2.852928in}}%
\pgfpathcurveto{\pgfqpoint{1.557058in}{2.861164in}}{\pgfqpoint{1.553786in}{2.869064in}}{\pgfqpoint{1.547962in}{2.874888in}}%
\pgfpathcurveto{\pgfqpoint{1.542138in}{2.880712in}}{\pgfqpoint{1.534238in}{2.883984in}}{\pgfqpoint{1.526002in}{2.883984in}}%
\pgfpathcurveto{\pgfqpoint{1.517765in}{2.883984in}}{\pgfqpoint{1.509865in}{2.880712in}}{\pgfqpoint{1.504041in}{2.874888in}}%
\pgfpathcurveto{\pgfqpoint{1.498217in}{2.869064in}}{\pgfqpoint{1.494945in}{2.861164in}}{\pgfqpoint{1.494945in}{2.852928in}}%
\pgfpathcurveto{\pgfqpoint{1.494945in}{2.844691in}}{\pgfqpoint{1.498217in}{2.836791in}}{\pgfqpoint{1.504041in}{2.830967in}}%
\pgfpathcurveto{\pgfqpoint{1.509865in}{2.825143in}}{\pgfqpoint{1.517765in}{2.821871in}}{\pgfqpoint{1.526002in}{2.821871in}}%
\pgfpathclose%
\pgfusepath{stroke,fill}%
\end{pgfscope}%
\begin{pgfscope}%
\pgfpathrectangle{\pgfqpoint{0.100000in}{0.212622in}}{\pgfqpoint{3.696000in}{3.696000in}}%
\pgfusepath{clip}%
\pgfsetbuttcap%
\pgfsetroundjoin%
\definecolor{currentfill}{rgb}{0.121569,0.466667,0.705882}%
\pgfsetfillcolor{currentfill}%
\pgfsetfillopacity{0.686697}%
\pgfsetlinewidth{1.003750pt}%
\definecolor{currentstroke}{rgb}{0.121569,0.466667,0.705882}%
\pgfsetstrokecolor{currentstroke}%
\pgfsetstrokeopacity{0.686697}%
\pgfsetdash{}{0pt}%
\pgfpathmoveto{\pgfqpoint{1.524748in}{2.820192in}}%
\pgfpathcurveto{\pgfqpoint{1.532985in}{2.820192in}}{\pgfqpoint{1.540885in}{2.823464in}}{\pgfqpoint{1.546709in}{2.829288in}}%
\pgfpathcurveto{\pgfqpoint{1.552533in}{2.835112in}}{\pgfqpoint{1.555805in}{2.843012in}}{\pgfqpoint{1.555805in}{2.851249in}}%
\pgfpathcurveto{\pgfqpoint{1.555805in}{2.859485in}}{\pgfqpoint{1.552533in}{2.867385in}}{\pgfqpoint{1.546709in}{2.873209in}}%
\pgfpathcurveto{\pgfqpoint{1.540885in}{2.879033in}}{\pgfqpoint{1.532985in}{2.882305in}}{\pgfqpoint{1.524748in}{2.882305in}}%
\pgfpathcurveto{\pgfqpoint{1.516512in}{2.882305in}}{\pgfqpoint{1.508612in}{2.879033in}}{\pgfqpoint{1.502788in}{2.873209in}}%
\pgfpathcurveto{\pgfqpoint{1.496964in}{2.867385in}}{\pgfqpoint{1.493692in}{2.859485in}}{\pgfqpoint{1.493692in}{2.851249in}}%
\pgfpathcurveto{\pgfqpoint{1.493692in}{2.843012in}}{\pgfqpoint{1.496964in}{2.835112in}}{\pgfqpoint{1.502788in}{2.829288in}}%
\pgfpathcurveto{\pgfqpoint{1.508612in}{2.823464in}}{\pgfqpoint{1.516512in}{2.820192in}}{\pgfqpoint{1.524748in}{2.820192in}}%
\pgfpathclose%
\pgfusepath{stroke,fill}%
\end{pgfscope}%
\begin{pgfscope}%
\pgfpathrectangle{\pgfqpoint{0.100000in}{0.212622in}}{\pgfqpoint{3.696000in}{3.696000in}}%
\pgfusepath{clip}%
\pgfsetbuttcap%
\pgfsetroundjoin%
\definecolor{currentfill}{rgb}{0.121569,0.466667,0.705882}%
\pgfsetfillcolor{currentfill}%
\pgfsetfillopacity{0.686967}%
\pgfsetlinewidth{1.003750pt}%
\definecolor{currentstroke}{rgb}{0.121569,0.466667,0.705882}%
\pgfsetstrokecolor{currentstroke}%
\pgfsetstrokeopacity{0.686967}%
\pgfsetdash{}{0pt}%
\pgfpathmoveto{\pgfqpoint{2.351130in}{3.006233in}}%
\pgfpathcurveto{\pgfqpoint{2.359366in}{3.006233in}}{\pgfqpoint{2.367266in}{3.009506in}}{\pgfqpoint{2.373090in}{3.015330in}}%
\pgfpathcurveto{\pgfqpoint{2.378914in}{3.021154in}}{\pgfqpoint{2.382186in}{3.029054in}}{\pgfqpoint{2.382186in}{3.037290in}}%
\pgfpathcurveto{\pgfqpoint{2.382186in}{3.045526in}}{\pgfqpoint{2.378914in}{3.053426in}}{\pgfqpoint{2.373090in}{3.059250in}}%
\pgfpathcurveto{\pgfqpoint{2.367266in}{3.065074in}}{\pgfqpoint{2.359366in}{3.068346in}}{\pgfqpoint{2.351130in}{3.068346in}}%
\pgfpathcurveto{\pgfqpoint{2.342894in}{3.068346in}}{\pgfqpoint{2.334994in}{3.065074in}}{\pgfqpoint{2.329170in}{3.059250in}}%
\pgfpathcurveto{\pgfqpoint{2.323346in}{3.053426in}}{\pgfqpoint{2.320073in}{3.045526in}}{\pgfqpoint{2.320073in}{3.037290in}}%
\pgfpathcurveto{\pgfqpoint{2.320073in}{3.029054in}}{\pgfqpoint{2.323346in}{3.021154in}}{\pgfqpoint{2.329170in}{3.015330in}}%
\pgfpathcurveto{\pgfqpoint{2.334994in}{3.009506in}}{\pgfqpoint{2.342894in}{3.006233in}}{\pgfqpoint{2.351130in}{3.006233in}}%
\pgfpathclose%
\pgfusepath{stroke,fill}%
\end{pgfscope}%
\begin{pgfscope}%
\pgfpathrectangle{\pgfqpoint{0.100000in}{0.212622in}}{\pgfqpoint{3.696000in}{3.696000in}}%
\pgfusepath{clip}%
\pgfsetbuttcap%
\pgfsetroundjoin%
\definecolor{currentfill}{rgb}{0.121569,0.466667,0.705882}%
\pgfsetfillcolor{currentfill}%
\pgfsetfillopacity{0.687270}%
\pgfsetlinewidth{1.003750pt}%
\definecolor{currentstroke}{rgb}{0.121569,0.466667,0.705882}%
\pgfsetstrokecolor{currentstroke}%
\pgfsetstrokeopacity{0.687270}%
\pgfsetdash{}{0pt}%
\pgfpathmoveto{\pgfqpoint{1.522533in}{2.817046in}}%
\pgfpathcurveto{\pgfqpoint{1.530770in}{2.817046in}}{\pgfqpoint{1.538670in}{2.820318in}}{\pgfqpoint{1.544494in}{2.826142in}}%
\pgfpathcurveto{\pgfqpoint{1.550318in}{2.831966in}}{\pgfqpoint{1.553590in}{2.839866in}}{\pgfqpoint{1.553590in}{2.848102in}}%
\pgfpathcurveto{\pgfqpoint{1.553590in}{2.856338in}}{\pgfqpoint{1.550318in}{2.864238in}}{\pgfqpoint{1.544494in}{2.870062in}}%
\pgfpathcurveto{\pgfqpoint{1.538670in}{2.875886in}}{\pgfqpoint{1.530770in}{2.879159in}}{\pgfqpoint{1.522533in}{2.879159in}}%
\pgfpathcurveto{\pgfqpoint{1.514297in}{2.879159in}}{\pgfqpoint{1.506397in}{2.875886in}}{\pgfqpoint{1.500573in}{2.870062in}}%
\pgfpathcurveto{\pgfqpoint{1.494749in}{2.864238in}}{\pgfqpoint{1.491477in}{2.856338in}}{\pgfqpoint{1.491477in}{2.848102in}}%
\pgfpathcurveto{\pgfqpoint{1.491477in}{2.839866in}}{\pgfqpoint{1.494749in}{2.831966in}}{\pgfqpoint{1.500573in}{2.826142in}}%
\pgfpathcurveto{\pgfqpoint{1.506397in}{2.820318in}}{\pgfqpoint{1.514297in}{2.817046in}}{\pgfqpoint{1.522533in}{2.817046in}}%
\pgfpathclose%
\pgfusepath{stroke,fill}%
\end{pgfscope}%
\begin{pgfscope}%
\pgfpathrectangle{\pgfqpoint{0.100000in}{0.212622in}}{\pgfqpoint{3.696000in}{3.696000in}}%
\pgfusepath{clip}%
\pgfsetbuttcap%
\pgfsetroundjoin%
\definecolor{currentfill}{rgb}{0.121569,0.466667,0.705882}%
\pgfsetfillcolor{currentfill}%
\pgfsetfillopacity{0.687856}%
\pgfsetlinewidth{1.003750pt}%
\definecolor{currentstroke}{rgb}{0.121569,0.466667,0.705882}%
\pgfsetstrokecolor{currentstroke}%
\pgfsetstrokeopacity{0.687856}%
\pgfsetdash{}{0pt}%
\pgfpathmoveto{\pgfqpoint{2.358200in}{3.005561in}}%
\pgfpathcurveto{\pgfqpoint{2.366436in}{3.005561in}}{\pgfqpoint{2.374336in}{3.008833in}}{\pgfqpoint{2.380160in}{3.014657in}}%
\pgfpathcurveto{\pgfqpoint{2.385984in}{3.020481in}}{\pgfqpoint{2.389256in}{3.028381in}}{\pgfqpoint{2.389256in}{3.036617in}}%
\pgfpathcurveto{\pgfqpoint{2.389256in}{3.044853in}}{\pgfqpoint{2.385984in}{3.052753in}}{\pgfqpoint{2.380160in}{3.058577in}}%
\pgfpathcurveto{\pgfqpoint{2.374336in}{3.064401in}}{\pgfqpoint{2.366436in}{3.067674in}}{\pgfqpoint{2.358200in}{3.067674in}}%
\pgfpathcurveto{\pgfqpoint{2.349964in}{3.067674in}}{\pgfqpoint{2.342064in}{3.064401in}}{\pgfqpoint{2.336240in}{3.058577in}}%
\pgfpathcurveto{\pgfqpoint{2.330416in}{3.052753in}}{\pgfqpoint{2.327143in}{3.044853in}}{\pgfqpoint{2.327143in}{3.036617in}}%
\pgfpathcurveto{\pgfqpoint{2.327143in}{3.028381in}}{\pgfqpoint{2.330416in}{3.020481in}}{\pgfqpoint{2.336240in}{3.014657in}}%
\pgfpathcurveto{\pgfqpoint{2.342064in}{3.008833in}}{\pgfqpoint{2.349964in}{3.005561in}}{\pgfqpoint{2.358200in}{3.005561in}}%
\pgfpathclose%
\pgfusepath{stroke,fill}%
\end{pgfscope}%
\begin{pgfscope}%
\pgfpathrectangle{\pgfqpoint{0.100000in}{0.212622in}}{\pgfqpoint{3.696000in}{3.696000in}}%
\pgfusepath{clip}%
\pgfsetbuttcap%
\pgfsetroundjoin%
\definecolor{currentfill}{rgb}{0.121569,0.466667,0.705882}%
\pgfsetfillcolor{currentfill}%
\pgfsetfillopacity{0.687865}%
\pgfsetlinewidth{1.003750pt}%
\definecolor{currentstroke}{rgb}{0.121569,0.466667,0.705882}%
\pgfsetstrokecolor{currentstroke}%
\pgfsetstrokeopacity{0.687865}%
\pgfsetdash{}{0pt}%
\pgfpathmoveto{\pgfqpoint{1.520118in}{2.813336in}}%
\pgfpathcurveto{\pgfqpoint{1.528354in}{2.813336in}}{\pgfqpoint{1.536254in}{2.816609in}}{\pgfqpoint{1.542078in}{2.822433in}}%
\pgfpathcurveto{\pgfqpoint{1.547902in}{2.828257in}}{\pgfqpoint{1.551174in}{2.836157in}}{\pgfqpoint{1.551174in}{2.844393in}}%
\pgfpathcurveto{\pgfqpoint{1.551174in}{2.852629in}}{\pgfqpoint{1.547902in}{2.860529in}}{\pgfqpoint{1.542078in}{2.866353in}}%
\pgfpathcurveto{\pgfqpoint{1.536254in}{2.872177in}}{\pgfqpoint{1.528354in}{2.875449in}}{\pgfqpoint{1.520118in}{2.875449in}}%
\pgfpathcurveto{\pgfqpoint{1.511881in}{2.875449in}}{\pgfqpoint{1.503981in}{2.872177in}}{\pgfqpoint{1.498157in}{2.866353in}}%
\pgfpathcurveto{\pgfqpoint{1.492333in}{2.860529in}}{\pgfqpoint{1.489061in}{2.852629in}}{\pgfqpoint{1.489061in}{2.844393in}}%
\pgfpathcurveto{\pgfqpoint{1.489061in}{2.836157in}}{\pgfqpoint{1.492333in}{2.828257in}}{\pgfqpoint{1.498157in}{2.822433in}}%
\pgfpathcurveto{\pgfqpoint{1.503981in}{2.816609in}}{\pgfqpoint{1.511881in}{2.813336in}}{\pgfqpoint{1.520118in}{2.813336in}}%
\pgfpathclose%
\pgfusepath{stroke,fill}%
\end{pgfscope}%
\begin{pgfscope}%
\pgfpathrectangle{\pgfqpoint{0.100000in}{0.212622in}}{\pgfqpoint{3.696000in}{3.696000in}}%
\pgfusepath{clip}%
\pgfsetbuttcap%
\pgfsetroundjoin%
\definecolor{currentfill}{rgb}{0.121569,0.466667,0.705882}%
\pgfsetfillcolor{currentfill}%
\pgfsetfillopacity{0.687952}%
\pgfsetlinewidth{1.003750pt}%
\definecolor{currentstroke}{rgb}{0.121569,0.466667,0.705882}%
\pgfsetstrokecolor{currentstroke}%
\pgfsetstrokeopacity{0.687952}%
\pgfsetdash{}{0pt}%
\pgfpathmoveto{\pgfqpoint{3.089165in}{1.782747in}}%
\pgfpathcurveto{\pgfqpoint{3.097401in}{1.782747in}}{\pgfqpoint{3.105301in}{1.786020in}}{\pgfqpoint{3.111125in}{1.791844in}}%
\pgfpathcurveto{\pgfqpoint{3.116949in}{1.797667in}}{\pgfqpoint{3.120221in}{1.805568in}}{\pgfqpoint{3.120221in}{1.813804in}}%
\pgfpathcurveto{\pgfqpoint{3.120221in}{1.822040in}}{\pgfqpoint{3.116949in}{1.829940in}}{\pgfqpoint{3.111125in}{1.835764in}}%
\pgfpathcurveto{\pgfqpoint{3.105301in}{1.841588in}}{\pgfqpoint{3.097401in}{1.844860in}}{\pgfqpoint{3.089165in}{1.844860in}}%
\pgfpathcurveto{\pgfqpoint{3.080928in}{1.844860in}}{\pgfqpoint{3.073028in}{1.841588in}}{\pgfqpoint{3.067204in}{1.835764in}}%
\pgfpathcurveto{\pgfqpoint{3.061380in}{1.829940in}}{\pgfqpoint{3.058108in}{1.822040in}}{\pgfqpoint{3.058108in}{1.813804in}}%
\pgfpathcurveto{\pgfqpoint{3.058108in}{1.805568in}}{\pgfqpoint{3.061380in}{1.797667in}}{\pgfqpoint{3.067204in}{1.791844in}}%
\pgfpathcurveto{\pgfqpoint{3.073028in}{1.786020in}}{\pgfqpoint{3.080928in}{1.782747in}}{\pgfqpoint{3.089165in}{1.782747in}}%
\pgfpathclose%
\pgfusepath{stroke,fill}%
\end{pgfscope}%
\begin{pgfscope}%
\pgfpathrectangle{\pgfqpoint{0.100000in}{0.212622in}}{\pgfqpoint{3.696000in}{3.696000in}}%
\pgfusepath{clip}%
\pgfsetbuttcap%
\pgfsetroundjoin%
\definecolor{currentfill}{rgb}{0.121569,0.466667,0.705882}%
\pgfsetfillcolor{currentfill}%
\pgfsetfillopacity{0.688565}%
\pgfsetlinewidth{1.003750pt}%
\definecolor{currentstroke}{rgb}{0.121569,0.466667,0.705882}%
\pgfsetstrokecolor{currentstroke}%
\pgfsetstrokeopacity{0.688565}%
\pgfsetdash{}{0pt}%
\pgfpathmoveto{\pgfqpoint{2.364223in}{3.004802in}}%
\pgfpathcurveto{\pgfqpoint{2.372459in}{3.004802in}}{\pgfqpoint{2.380359in}{3.008074in}}{\pgfqpoint{2.386183in}{3.013898in}}%
\pgfpathcurveto{\pgfqpoint{2.392007in}{3.019722in}}{\pgfqpoint{2.395279in}{3.027622in}}{\pgfqpoint{2.395279in}{3.035858in}}%
\pgfpathcurveto{\pgfqpoint{2.395279in}{3.044095in}}{\pgfqpoint{2.392007in}{3.051995in}}{\pgfqpoint{2.386183in}{3.057819in}}%
\pgfpathcurveto{\pgfqpoint{2.380359in}{3.063642in}}{\pgfqpoint{2.372459in}{3.066915in}}{\pgfqpoint{2.364223in}{3.066915in}}%
\pgfpathcurveto{\pgfqpoint{2.355987in}{3.066915in}}{\pgfqpoint{2.348087in}{3.063642in}}{\pgfqpoint{2.342263in}{3.057819in}}%
\pgfpathcurveto{\pgfqpoint{2.336439in}{3.051995in}}{\pgfqpoint{2.333166in}{3.044095in}}{\pgfqpoint{2.333166in}{3.035858in}}%
\pgfpathcurveto{\pgfqpoint{2.333166in}{3.027622in}}{\pgfqpoint{2.336439in}{3.019722in}}{\pgfqpoint{2.342263in}{3.013898in}}%
\pgfpathcurveto{\pgfqpoint{2.348087in}{3.008074in}}{\pgfqpoint{2.355987in}{3.004802in}}{\pgfqpoint{2.364223in}{3.004802in}}%
\pgfpathclose%
\pgfusepath{stroke,fill}%
\end{pgfscope}%
\begin{pgfscope}%
\pgfpathrectangle{\pgfqpoint{0.100000in}{0.212622in}}{\pgfqpoint{3.696000in}{3.696000in}}%
\pgfusepath{clip}%
\pgfsetbuttcap%
\pgfsetroundjoin%
\definecolor{currentfill}{rgb}{0.121569,0.466667,0.705882}%
\pgfsetfillcolor{currentfill}%
\pgfsetfillopacity{0.688630}%
\pgfsetlinewidth{1.003750pt}%
\definecolor{currentstroke}{rgb}{0.121569,0.466667,0.705882}%
\pgfsetstrokecolor{currentstroke}%
\pgfsetstrokeopacity{0.688630}%
\pgfsetdash{}{0pt}%
\pgfpathmoveto{\pgfqpoint{1.517192in}{2.809026in}}%
\pgfpathcurveto{\pgfqpoint{1.525428in}{2.809026in}}{\pgfqpoint{1.533328in}{2.812298in}}{\pgfqpoint{1.539152in}{2.818122in}}%
\pgfpathcurveto{\pgfqpoint{1.544976in}{2.823946in}}{\pgfqpoint{1.548248in}{2.831846in}}{\pgfqpoint{1.548248in}{2.840083in}}%
\pgfpathcurveto{\pgfqpoint{1.548248in}{2.848319in}}{\pgfqpoint{1.544976in}{2.856219in}}{\pgfqpoint{1.539152in}{2.862043in}}%
\pgfpathcurveto{\pgfqpoint{1.533328in}{2.867867in}}{\pgfqpoint{1.525428in}{2.871139in}}{\pgfqpoint{1.517192in}{2.871139in}}%
\pgfpathcurveto{\pgfqpoint{1.508956in}{2.871139in}}{\pgfqpoint{1.501055in}{2.867867in}}{\pgfqpoint{1.495232in}{2.862043in}}%
\pgfpathcurveto{\pgfqpoint{1.489408in}{2.856219in}}{\pgfqpoint{1.486135in}{2.848319in}}{\pgfqpoint{1.486135in}{2.840083in}}%
\pgfpathcurveto{\pgfqpoint{1.486135in}{2.831846in}}{\pgfqpoint{1.489408in}{2.823946in}}{\pgfqpoint{1.495232in}{2.818122in}}%
\pgfpathcurveto{\pgfqpoint{1.501055in}{2.812298in}}{\pgfqpoint{1.508956in}{2.809026in}}{\pgfqpoint{1.517192in}{2.809026in}}%
\pgfpathclose%
\pgfusepath{stroke,fill}%
\end{pgfscope}%
\begin{pgfscope}%
\pgfpathrectangle{\pgfqpoint{0.100000in}{0.212622in}}{\pgfqpoint{3.696000in}{3.696000in}}%
\pgfusepath{clip}%
\pgfsetbuttcap%
\pgfsetroundjoin%
\definecolor{currentfill}{rgb}{0.121569,0.466667,0.705882}%
\pgfsetfillcolor{currentfill}%
\pgfsetfillopacity{0.689253}%
\pgfsetlinewidth{1.003750pt}%
\definecolor{currentstroke}{rgb}{0.121569,0.466667,0.705882}%
\pgfsetstrokecolor{currentstroke}%
\pgfsetstrokeopacity{0.689253}%
\pgfsetdash{}{0pt}%
\pgfpathmoveto{\pgfqpoint{2.369934in}{3.003982in}}%
\pgfpathcurveto{\pgfqpoint{2.378170in}{3.003982in}}{\pgfqpoint{2.386070in}{3.007255in}}{\pgfqpoint{2.391894in}{3.013079in}}%
\pgfpathcurveto{\pgfqpoint{2.397718in}{3.018902in}}{\pgfqpoint{2.400990in}{3.026803in}}{\pgfqpoint{2.400990in}{3.035039in}}%
\pgfpathcurveto{\pgfqpoint{2.400990in}{3.043275in}}{\pgfqpoint{2.397718in}{3.051175in}}{\pgfqpoint{2.391894in}{3.056999in}}%
\pgfpathcurveto{\pgfqpoint{2.386070in}{3.062823in}}{\pgfqpoint{2.378170in}{3.066095in}}{\pgfqpoint{2.369934in}{3.066095in}}%
\pgfpathcurveto{\pgfqpoint{2.361698in}{3.066095in}}{\pgfqpoint{2.353797in}{3.062823in}}{\pgfqpoint{2.347974in}{3.056999in}}%
\pgfpathcurveto{\pgfqpoint{2.342150in}{3.051175in}}{\pgfqpoint{2.338877in}{3.043275in}}{\pgfqpoint{2.338877in}{3.035039in}}%
\pgfpathcurveto{\pgfqpoint{2.338877in}{3.026803in}}{\pgfqpoint{2.342150in}{3.018902in}}{\pgfqpoint{2.347974in}{3.013079in}}%
\pgfpathcurveto{\pgfqpoint{2.353797in}{3.007255in}}{\pgfqpoint{2.361698in}{3.003982in}}{\pgfqpoint{2.369934in}{3.003982in}}%
\pgfpathclose%
\pgfusepath{stroke,fill}%
\end{pgfscope}%
\begin{pgfscope}%
\pgfpathrectangle{\pgfqpoint{0.100000in}{0.212622in}}{\pgfqpoint{3.696000in}{3.696000in}}%
\pgfusepath{clip}%
\pgfsetbuttcap%
\pgfsetroundjoin%
\definecolor{currentfill}{rgb}{0.121569,0.466667,0.705882}%
\pgfsetfillcolor{currentfill}%
\pgfsetfillopacity{0.689642}%
\pgfsetlinewidth{1.003750pt}%
\definecolor{currentstroke}{rgb}{0.121569,0.466667,0.705882}%
\pgfsetstrokecolor{currentstroke}%
\pgfsetstrokeopacity{0.689642}%
\pgfsetdash{}{0pt}%
\pgfpathmoveto{\pgfqpoint{1.513700in}{2.804514in}}%
\pgfpathcurveto{\pgfqpoint{1.521936in}{2.804514in}}{\pgfqpoint{1.529836in}{2.807786in}}{\pgfqpoint{1.535660in}{2.813610in}}%
\pgfpathcurveto{\pgfqpoint{1.541484in}{2.819434in}}{\pgfqpoint{1.544756in}{2.827334in}}{\pgfqpoint{1.544756in}{2.835570in}}%
\pgfpathcurveto{\pgfqpoint{1.544756in}{2.843806in}}{\pgfqpoint{1.541484in}{2.851706in}}{\pgfqpoint{1.535660in}{2.857530in}}%
\pgfpathcurveto{\pgfqpoint{1.529836in}{2.863354in}}{\pgfqpoint{1.521936in}{2.866627in}}{\pgfqpoint{1.513700in}{2.866627in}}%
\pgfpathcurveto{\pgfqpoint{1.505464in}{2.866627in}}{\pgfqpoint{1.497563in}{2.863354in}}{\pgfqpoint{1.491740in}{2.857530in}}%
\pgfpathcurveto{\pgfqpoint{1.485916in}{2.851706in}}{\pgfqpoint{1.482643in}{2.843806in}}{\pgfqpoint{1.482643in}{2.835570in}}%
\pgfpathcurveto{\pgfqpoint{1.482643in}{2.827334in}}{\pgfqpoint{1.485916in}{2.819434in}}{\pgfqpoint{1.491740in}{2.813610in}}%
\pgfpathcurveto{\pgfqpoint{1.497563in}{2.807786in}}{\pgfqpoint{1.505464in}{2.804514in}}{\pgfqpoint{1.513700in}{2.804514in}}%
\pgfpathclose%
\pgfusepath{stroke,fill}%
\end{pgfscope}%
\begin{pgfscope}%
\pgfpathrectangle{\pgfqpoint{0.100000in}{0.212622in}}{\pgfqpoint{3.696000in}{3.696000in}}%
\pgfusepath{clip}%
\pgfsetbuttcap%
\pgfsetroundjoin%
\definecolor{currentfill}{rgb}{0.121569,0.466667,0.705882}%
\pgfsetfillcolor{currentfill}%
\pgfsetfillopacity{0.689768}%
\pgfsetlinewidth{1.003750pt}%
\definecolor{currentstroke}{rgb}{0.121569,0.466667,0.705882}%
\pgfsetstrokecolor{currentstroke}%
\pgfsetstrokeopacity{0.689768}%
\pgfsetdash{}{0pt}%
\pgfpathmoveto{\pgfqpoint{2.374399in}{3.003267in}}%
\pgfpathcurveto{\pgfqpoint{2.382635in}{3.003267in}}{\pgfqpoint{2.390535in}{3.006539in}}{\pgfqpoint{2.396359in}{3.012363in}}%
\pgfpathcurveto{\pgfqpoint{2.402183in}{3.018187in}}{\pgfqpoint{2.405455in}{3.026087in}}{\pgfqpoint{2.405455in}{3.034323in}}%
\pgfpathcurveto{\pgfqpoint{2.405455in}{3.042560in}}{\pgfqpoint{2.402183in}{3.050460in}}{\pgfqpoint{2.396359in}{3.056284in}}%
\pgfpathcurveto{\pgfqpoint{2.390535in}{3.062108in}}{\pgfqpoint{2.382635in}{3.065380in}}{\pgfqpoint{2.374399in}{3.065380in}}%
\pgfpathcurveto{\pgfqpoint{2.366162in}{3.065380in}}{\pgfqpoint{2.358262in}{3.062108in}}{\pgfqpoint{2.352438in}{3.056284in}}%
\pgfpathcurveto{\pgfqpoint{2.346615in}{3.050460in}}{\pgfqpoint{2.343342in}{3.042560in}}{\pgfqpoint{2.343342in}{3.034323in}}%
\pgfpathcurveto{\pgfqpoint{2.343342in}{3.026087in}}{\pgfqpoint{2.346615in}{3.018187in}}{\pgfqpoint{2.352438in}{3.012363in}}%
\pgfpathcurveto{\pgfqpoint{2.358262in}{3.006539in}}{\pgfqpoint{2.366162in}{3.003267in}}{\pgfqpoint{2.374399in}{3.003267in}}%
\pgfpathclose%
\pgfusepath{stroke,fill}%
\end{pgfscope}%
\begin{pgfscope}%
\pgfpathrectangle{\pgfqpoint{0.100000in}{0.212622in}}{\pgfqpoint{3.696000in}{3.696000in}}%
\pgfusepath{clip}%
\pgfsetbuttcap%
\pgfsetroundjoin%
\definecolor{currentfill}{rgb}{0.121569,0.466667,0.705882}%
\pgfsetfillcolor{currentfill}%
\pgfsetfillopacity{0.690226}%
\pgfsetlinewidth{1.003750pt}%
\definecolor{currentstroke}{rgb}{0.121569,0.466667,0.705882}%
\pgfsetstrokecolor{currentstroke}%
\pgfsetstrokeopacity{0.690226}%
\pgfsetdash{}{0pt}%
\pgfpathmoveto{\pgfqpoint{2.378293in}{3.002746in}}%
\pgfpathcurveto{\pgfqpoint{2.386529in}{3.002746in}}{\pgfqpoint{2.394429in}{3.006018in}}{\pgfqpoint{2.400253in}{3.011842in}}%
\pgfpathcurveto{\pgfqpoint{2.406077in}{3.017666in}}{\pgfqpoint{2.409349in}{3.025566in}}{\pgfqpoint{2.409349in}{3.033802in}}%
\pgfpathcurveto{\pgfqpoint{2.409349in}{3.042039in}}{\pgfqpoint{2.406077in}{3.049939in}}{\pgfqpoint{2.400253in}{3.055763in}}%
\pgfpathcurveto{\pgfqpoint{2.394429in}{3.061587in}}{\pgfqpoint{2.386529in}{3.064859in}}{\pgfqpoint{2.378293in}{3.064859in}}%
\pgfpathcurveto{\pgfqpoint{2.370056in}{3.064859in}}{\pgfqpoint{2.362156in}{3.061587in}}{\pgfqpoint{2.356332in}{3.055763in}}%
\pgfpathcurveto{\pgfqpoint{2.350508in}{3.049939in}}{\pgfqpoint{2.347236in}{3.042039in}}{\pgfqpoint{2.347236in}{3.033802in}}%
\pgfpathcurveto{\pgfqpoint{2.347236in}{3.025566in}}{\pgfqpoint{2.350508in}{3.017666in}}{\pgfqpoint{2.356332in}{3.011842in}}%
\pgfpathcurveto{\pgfqpoint{2.362156in}{3.006018in}}{\pgfqpoint{2.370056in}{3.002746in}}{\pgfqpoint{2.378293in}{3.002746in}}%
\pgfpathclose%
\pgfusepath{stroke,fill}%
\end{pgfscope}%
\begin{pgfscope}%
\pgfpathrectangle{\pgfqpoint{0.100000in}{0.212622in}}{\pgfqpoint{3.696000in}{3.696000in}}%
\pgfusepath{clip}%
\pgfsetbuttcap%
\pgfsetroundjoin%
\definecolor{currentfill}{rgb}{0.121569,0.466667,0.705882}%
\pgfsetfillcolor{currentfill}%
\pgfsetfillopacity{0.690566}%
\pgfsetlinewidth{1.003750pt}%
\definecolor{currentstroke}{rgb}{0.121569,0.466667,0.705882}%
\pgfsetstrokecolor{currentstroke}%
\pgfsetstrokeopacity{0.690566}%
\pgfsetdash{}{0pt}%
\pgfpathmoveto{\pgfqpoint{2.381362in}{3.002266in}}%
\pgfpathcurveto{\pgfqpoint{2.389599in}{3.002266in}}{\pgfqpoint{2.397499in}{3.005538in}}{\pgfqpoint{2.403323in}{3.011362in}}%
\pgfpathcurveto{\pgfqpoint{2.409146in}{3.017186in}}{\pgfqpoint{2.412419in}{3.025086in}}{\pgfqpoint{2.412419in}{3.033322in}}%
\pgfpathcurveto{\pgfqpoint{2.412419in}{3.041558in}}{\pgfqpoint{2.409146in}{3.049458in}}{\pgfqpoint{2.403323in}{3.055282in}}%
\pgfpathcurveto{\pgfqpoint{2.397499in}{3.061106in}}{\pgfqpoint{2.389599in}{3.064379in}}{\pgfqpoint{2.381362in}{3.064379in}}%
\pgfpathcurveto{\pgfqpoint{2.373126in}{3.064379in}}{\pgfqpoint{2.365226in}{3.061106in}}{\pgfqpoint{2.359402in}{3.055282in}}%
\pgfpathcurveto{\pgfqpoint{2.353578in}{3.049458in}}{\pgfqpoint{2.350306in}{3.041558in}}{\pgfqpoint{2.350306in}{3.033322in}}%
\pgfpathcurveto{\pgfqpoint{2.350306in}{3.025086in}}{\pgfqpoint{2.353578in}{3.017186in}}{\pgfqpoint{2.359402in}{3.011362in}}%
\pgfpathcurveto{\pgfqpoint{2.365226in}{3.005538in}}{\pgfqpoint{2.373126in}{3.002266in}}{\pgfqpoint{2.381362in}{3.002266in}}%
\pgfpathclose%
\pgfusepath{stroke,fill}%
\end{pgfscope}%
\begin{pgfscope}%
\pgfpathrectangle{\pgfqpoint{0.100000in}{0.212622in}}{\pgfqpoint{3.696000in}{3.696000in}}%
\pgfusepath{clip}%
\pgfsetbuttcap%
\pgfsetroundjoin%
\definecolor{currentfill}{rgb}{0.121569,0.466667,0.705882}%
\pgfsetfillcolor{currentfill}%
\pgfsetfillopacity{0.690672}%
\pgfsetlinewidth{1.003750pt}%
\definecolor{currentstroke}{rgb}{0.121569,0.466667,0.705882}%
\pgfsetstrokecolor{currentstroke}%
\pgfsetstrokeopacity{0.690672}%
\pgfsetdash{}{0pt}%
\pgfpathmoveto{\pgfqpoint{1.509892in}{2.799570in}}%
\pgfpathcurveto{\pgfqpoint{1.518128in}{2.799570in}}{\pgfqpoint{1.526028in}{2.802842in}}{\pgfqpoint{1.531852in}{2.808666in}}%
\pgfpathcurveto{\pgfqpoint{1.537676in}{2.814490in}}{\pgfqpoint{1.540948in}{2.822390in}}{\pgfqpoint{1.540948in}{2.830627in}}%
\pgfpathcurveto{\pgfqpoint{1.540948in}{2.838863in}}{\pgfqpoint{1.537676in}{2.846763in}}{\pgfqpoint{1.531852in}{2.852587in}}%
\pgfpathcurveto{\pgfqpoint{1.526028in}{2.858411in}}{\pgfqpoint{1.518128in}{2.861683in}}{\pgfqpoint{1.509892in}{2.861683in}}%
\pgfpathcurveto{\pgfqpoint{1.501655in}{2.861683in}}{\pgfqpoint{1.493755in}{2.858411in}}{\pgfqpoint{1.487931in}{2.852587in}}%
\pgfpathcurveto{\pgfqpoint{1.482107in}{2.846763in}}{\pgfqpoint{1.478835in}{2.838863in}}{\pgfqpoint{1.478835in}{2.830627in}}%
\pgfpathcurveto{\pgfqpoint{1.478835in}{2.822390in}}{\pgfqpoint{1.482107in}{2.814490in}}{\pgfqpoint{1.487931in}{2.808666in}}%
\pgfpathcurveto{\pgfqpoint{1.493755in}{2.802842in}}{\pgfqpoint{1.501655in}{2.799570in}}{\pgfqpoint{1.509892in}{2.799570in}}%
\pgfpathclose%
\pgfusepath{stroke,fill}%
\end{pgfscope}%
\begin{pgfscope}%
\pgfpathrectangle{\pgfqpoint{0.100000in}{0.212622in}}{\pgfqpoint{3.696000in}{3.696000in}}%
\pgfusepath{clip}%
\pgfsetbuttcap%
\pgfsetroundjoin%
\definecolor{currentfill}{rgb}{0.121569,0.466667,0.705882}%
\pgfsetfillcolor{currentfill}%
\pgfsetfillopacity{0.691175}%
\pgfsetlinewidth{1.003750pt}%
\definecolor{currentstroke}{rgb}{0.121569,0.466667,0.705882}%
\pgfsetstrokecolor{currentstroke}%
\pgfsetstrokeopacity{0.691175}%
\pgfsetdash{}{0pt}%
\pgfpathmoveto{\pgfqpoint{2.386981in}{3.001485in}}%
\pgfpathcurveto{\pgfqpoint{2.395217in}{3.001485in}}{\pgfqpoint{2.403117in}{3.004757in}}{\pgfqpoint{2.408941in}{3.010581in}}%
\pgfpathcurveto{\pgfqpoint{2.414765in}{3.016405in}}{\pgfqpoint{2.418038in}{3.024305in}}{\pgfqpoint{2.418038in}{3.032541in}}%
\pgfpathcurveto{\pgfqpoint{2.418038in}{3.040777in}}{\pgfqpoint{2.414765in}{3.048677in}}{\pgfqpoint{2.408941in}{3.054501in}}%
\pgfpathcurveto{\pgfqpoint{2.403117in}{3.060325in}}{\pgfqpoint{2.395217in}{3.063598in}}{\pgfqpoint{2.386981in}{3.063598in}}%
\pgfpathcurveto{\pgfqpoint{2.378745in}{3.063598in}}{\pgfqpoint{2.370845in}{3.060325in}}{\pgfqpoint{2.365021in}{3.054501in}}%
\pgfpathcurveto{\pgfqpoint{2.359197in}{3.048677in}}{\pgfqpoint{2.355925in}{3.040777in}}{\pgfqpoint{2.355925in}{3.032541in}}%
\pgfpathcurveto{\pgfqpoint{2.355925in}{3.024305in}}{\pgfqpoint{2.359197in}{3.016405in}}{\pgfqpoint{2.365021in}{3.010581in}}%
\pgfpathcurveto{\pgfqpoint{2.370845in}{3.004757in}}{\pgfqpoint{2.378745in}{3.001485in}}{\pgfqpoint{2.386981in}{3.001485in}}%
\pgfpathclose%
\pgfusepath{stroke,fill}%
\end{pgfscope}%
\begin{pgfscope}%
\pgfpathrectangle{\pgfqpoint{0.100000in}{0.212622in}}{\pgfqpoint{3.696000in}{3.696000in}}%
\pgfusepath{clip}%
\pgfsetbuttcap%
\pgfsetroundjoin%
\definecolor{currentfill}{rgb}{0.121569,0.466667,0.705882}%
\pgfsetfillcolor{currentfill}%
\pgfsetfillopacity{0.691632}%
\pgfsetlinewidth{1.003750pt}%
\definecolor{currentstroke}{rgb}{0.121569,0.466667,0.705882}%
\pgfsetstrokecolor{currentstroke}%
\pgfsetstrokeopacity{0.691632}%
\pgfsetdash{}{0pt}%
\pgfpathmoveto{\pgfqpoint{3.081574in}{1.776955in}}%
\pgfpathcurveto{\pgfqpoint{3.089810in}{1.776955in}}{\pgfqpoint{3.097710in}{1.780227in}}{\pgfqpoint{3.103534in}{1.786051in}}%
\pgfpathcurveto{\pgfqpoint{3.109358in}{1.791875in}}{\pgfqpoint{3.112630in}{1.799775in}}{\pgfqpoint{3.112630in}{1.808011in}}%
\pgfpathcurveto{\pgfqpoint{3.112630in}{1.816248in}}{\pgfqpoint{3.109358in}{1.824148in}}{\pgfqpoint{3.103534in}{1.829972in}}%
\pgfpathcurveto{\pgfqpoint{3.097710in}{1.835795in}}{\pgfqpoint{3.089810in}{1.839068in}}{\pgfqpoint{3.081574in}{1.839068in}}%
\pgfpathcurveto{\pgfqpoint{3.073338in}{1.839068in}}{\pgfqpoint{3.065438in}{1.835795in}}{\pgfqpoint{3.059614in}{1.829972in}}%
\pgfpathcurveto{\pgfqpoint{3.053790in}{1.824148in}}{\pgfqpoint{3.050517in}{1.816248in}}{\pgfqpoint{3.050517in}{1.808011in}}%
\pgfpathcurveto{\pgfqpoint{3.050517in}{1.799775in}}{\pgfqpoint{3.053790in}{1.791875in}}{\pgfqpoint{3.059614in}{1.786051in}}%
\pgfpathcurveto{\pgfqpoint{3.065438in}{1.780227in}}{\pgfqpoint{3.073338in}{1.776955in}}{\pgfqpoint{3.081574in}{1.776955in}}%
\pgfpathclose%
\pgfusepath{stroke,fill}%
\end{pgfscope}%
\begin{pgfscope}%
\pgfpathrectangle{\pgfqpoint{0.100000in}{0.212622in}}{\pgfqpoint{3.696000in}{3.696000in}}%
\pgfusepath{clip}%
\pgfsetbuttcap%
\pgfsetroundjoin%
\definecolor{currentfill}{rgb}{0.121569,0.466667,0.705882}%
\pgfsetfillcolor{currentfill}%
\pgfsetfillopacity{0.691896}%
\pgfsetlinewidth{1.003750pt}%
\definecolor{currentstroke}{rgb}{0.121569,0.466667,0.705882}%
\pgfsetstrokecolor{currentstroke}%
\pgfsetstrokeopacity{0.691896}%
\pgfsetdash{}{0pt}%
\pgfpathmoveto{\pgfqpoint{1.504887in}{2.792463in}}%
\pgfpathcurveto{\pgfqpoint{1.513123in}{2.792463in}}{\pgfqpoint{1.521023in}{2.795736in}}{\pgfqpoint{1.526847in}{2.801559in}}%
\pgfpathcurveto{\pgfqpoint{1.532671in}{2.807383in}}{\pgfqpoint{1.535943in}{2.815283in}}{\pgfqpoint{1.535943in}{2.823520in}}%
\pgfpathcurveto{\pgfqpoint{1.535943in}{2.831756in}}{\pgfqpoint{1.532671in}{2.839656in}}{\pgfqpoint{1.526847in}{2.845480in}}%
\pgfpathcurveto{\pgfqpoint{1.521023in}{2.851304in}}{\pgfqpoint{1.513123in}{2.854576in}}{\pgfqpoint{1.504887in}{2.854576in}}%
\pgfpathcurveto{\pgfqpoint{1.496650in}{2.854576in}}{\pgfqpoint{1.488750in}{2.851304in}}{\pgfqpoint{1.482926in}{2.845480in}}%
\pgfpathcurveto{\pgfqpoint{1.477103in}{2.839656in}}{\pgfqpoint{1.473830in}{2.831756in}}{\pgfqpoint{1.473830in}{2.823520in}}%
\pgfpathcurveto{\pgfqpoint{1.473830in}{2.815283in}}{\pgfqpoint{1.477103in}{2.807383in}}{\pgfqpoint{1.482926in}{2.801559in}}%
\pgfpathcurveto{\pgfqpoint{1.488750in}{2.795736in}}{\pgfqpoint{1.496650in}{2.792463in}}{\pgfqpoint{1.504887in}{2.792463in}}%
\pgfpathclose%
\pgfusepath{stroke,fill}%
\end{pgfscope}%
\begin{pgfscope}%
\pgfpathrectangle{\pgfqpoint{0.100000in}{0.212622in}}{\pgfqpoint{3.696000in}{3.696000in}}%
\pgfusepath{clip}%
\pgfsetbuttcap%
\pgfsetroundjoin%
\definecolor{currentfill}{rgb}{0.121569,0.466667,0.705882}%
\pgfsetfillcolor{currentfill}%
\pgfsetfillopacity{0.691918}%
\pgfsetlinewidth{1.003750pt}%
\definecolor{currentstroke}{rgb}{0.121569,0.466667,0.705882}%
\pgfsetstrokecolor{currentstroke}%
\pgfsetstrokeopacity{0.691918}%
\pgfsetdash{}{0pt}%
\pgfpathmoveto{\pgfqpoint{2.392076in}{3.000648in}}%
\pgfpathcurveto{\pgfqpoint{2.400312in}{3.000648in}}{\pgfqpoint{2.408212in}{3.003920in}}{\pgfqpoint{2.414036in}{3.009744in}}%
\pgfpathcurveto{\pgfqpoint{2.419860in}{3.015568in}}{\pgfqpoint{2.423133in}{3.023468in}}{\pgfqpoint{2.423133in}{3.031705in}}%
\pgfpathcurveto{\pgfqpoint{2.423133in}{3.039941in}}{\pgfqpoint{2.419860in}{3.047841in}}{\pgfqpoint{2.414036in}{3.053665in}}%
\pgfpathcurveto{\pgfqpoint{2.408212in}{3.059489in}}{\pgfqpoint{2.400312in}{3.062761in}}{\pgfqpoint{2.392076in}{3.062761in}}%
\pgfpathcurveto{\pgfqpoint{2.383840in}{3.062761in}}{\pgfqpoint{2.375940in}{3.059489in}}{\pgfqpoint{2.370116in}{3.053665in}}%
\pgfpathcurveto{\pgfqpoint{2.364292in}{3.047841in}}{\pgfqpoint{2.361020in}{3.039941in}}{\pgfqpoint{2.361020in}{3.031705in}}%
\pgfpathcurveto{\pgfqpoint{2.361020in}{3.023468in}}{\pgfqpoint{2.364292in}{3.015568in}}{\pgfqpoint{2.370116in}{3.009744in}}%
\pgfpathcurveto{\pgfqpoint{2.375940in}{3.003920in}}{\pgfqpoint{2.383840in}{3.000648in}}{\pgfqpoint{2.392076in}{3.000648in}}%
\pgfpathclose%
\pgfusepath{stroke,fill}%
\end{pgfscope}%
\begin{pgfscope}%
\pgfpathrectangle{\pgfqpoint{0.100000in}{0.212622in}}{\pgfqpoint{3.696000in}{3.696000in}}%
\pgfusepath{clip}%
\pgfsetbuttcap%
\pgfsetroundjoin%
\definecolor{currentfill}{rgb}{0.121569,0.466667,0.705882}%
\pgfsetfillcolor{currentfill}%
\pgfsetfillopacity{0.692607}%
\pgfsetlinewidth{1.003750pt}%
\definecolor{currentstroke}{rgb}{0.121569,0.466667,0.705882}%
\pgfsetstrokecolor{currentstroke}%
\pgfsetstrokeopacity{0.692607}%
\pgfsetdash{}{0pt}%
\pgfpathmoveto{\pgfqpoint{1.502145in}{2.788759in}}%
\pgfpathcurveto{\pgfqpoint{1.510382in}{2.788759in}}{\pgfqpoint{1.518282in}{2.792032in}}{\pgfqpoint{1.524106in}{2.797855in}}%
\pgfpathcurveto{\pgfqpoint{1.529930in}{2.803679in}}{\pgfqpoint{1.533202in}{2.811579in}}{\pgfqpoint{1.533202in}{2.819816in}}%
\pgfpathcurveto{\pgfqpoint{1.533202in}{2.828052in}}{\pgfqpoint{1.529930in}{2.835952in}}{\pgfqpoint{1.524106in}{2.841776in}}%
\pgfpathcurveto{\pgfqpoint{1.518282in}{2.847600in}}{\pgfqpoint{1.510382in}{2.850872in}}{\pgfqpoint{1.502145in}{2.850872in}}%
\pgfpathcurveto{\pgfqpoint{1.493909in}{2.850872in}}{\pgfqpoint{1.486009in}{2.847600in}}{\pgfqpoint{1.480185in}{2.841776in}}%
\pgfpathcurveto{\pgfqpoint{1.474361in}{2.835952in}}{\pgfqpoint{1.471089in}{2.828052in}}{\pgfqpoint{1.471089in}{2.819816in}}%
\pgfpathcurveto{\pgfqpoint{1.471089in}{2.811579in}}{\pgfqpoint{1.474361in}{2.803679in}}{\pgfqpoint{1.480185in}{2.797855in}}%
\pgfpathcurveto{\pgfqpoint{1.486009in}{2.792032in}}{\pgfqpoint{1.493909in}{2.788759in}}{\pgfqpoint{1.502145in}{2.788759in}}%
\pgfpathclose%
\pgfusepath{stroke,fill}%
\end{pgfscope}%
\begin{pgfscope}%
\pgfpathrectangle{\pgfqpoint{0.100000in}{0.212622in}}{\pgfqpoint{3.696000in}{3.696000in}}%
\pgfusepath{clip}%
\pgfsetbuttcap%
\pgfsetroundjoin%
\definecolor{currentfill}{rgb}{0.121569,0.466667,0.705882}%
\pgfsetfillcolor{currentfill}%
\pgfsetfillopacity{0.692607}%
\pgfsetlinewidth{1.003750pt}%
\definecolor{currentstroke}{rgb}{0.121569,0.466667,0.705882}%
\pgfsetstrokecolor{currentstroke}%
\pgfsetstrokeopacity{0.692607}%
\pgfsetdash{}{0pt}%
\pgfpathmoveto{\pgfqpoint{2.396625in}{2.999632in}}%
\pgfpathcurveto{\pgfqpoint{2.404862in}{2.999632in}}{\pgfqpoint{2.412762in}{3.002905in}}{\pgfqpoint{2.418586in}{3.008728in}}%
\pgfpathcurveto{\pgfqpoint{2.424410in}{3.014552in}}{\pgfqpoint{2.427682in}{3.022452in}}{\pgfqpoint{2.427682in}{3.030689in}}%
\pgfpathcurveto{\pgfqpoint{2.427682in}{3.038925in}}{\pgfqpoint{2.424410in}{3.046825in}}{\pgfqpoint{2.418586in}{3.052649in}}%
\pgfpathcurveto{\pgfqpoint{2.412762in}{3.058473in}}{\pgfqpoint{2.404862in}{3.061745in}}{\pgfqpoint{2.396625in}{3.061745in}}%
\pgfpathcurveto{\pgfqpoint{2.388389in}{3.061745in}}{\pgfqpoint{2.380489in}{3.058473in}}{\pgfqpoint{2.374665in}{3.052649in}}%
\pgfpathcurveto{\pgfqpoint{2.368841in}{3.046825in}}{\pgfqpoint{2.365569in}{3.038925in}}{\pgfqpoint{2.365569in}{3.030689in}}%
\pgfpathcurveto{\pgfqpoint{2.365569in}{3.022452in}}{\pgfqpoint{2.368841in}{3.014552in}}{\pgfqpoint{2.374665in}{3.008728in}}%
\pgfpathcurveto{\pgfqpoint{2.380489in}{3.002905in}}{\pgfqpoint{2.388389in}{2.999632in}}{\pgfqpoint{2.396625in}{2.999632in}}%
\pgfpathclose%
\pgfusepath{stroke,fill}%
\end{pgfscope}%
\begin{pgfscope}%
\pgfpathrectangle{\pgfqpoint{0.100000in}{0.212622in}}{\pgfqpoint{3.696000in}{3.696000in}}%
\pgfusepath{clip}%
\pgfsetbuttcap%
\pgfsetroundjoin%
\definecolor{currentfill}{rgb}{0.121569,0.466667,0.705882}%
\pgfsetfillcolor{currentfill}%
\pgfsetfillopacity{0.693526}%
\pgfsetlinewidth{1.003750pt}%
\definecolor{currentstroke}{rgb}{0.121569,0.466667,0.705882}%
\pgfsetstrokecolor{currentstroke}%
\pgfsetstrokeopacity{0.693526}%
\pgfsetdash{}{0pt}%
\pgfpathmoveto{\pgfqpoint{1.498912in}{2.784746in}}%
\pgfpathcurveto{\pgfqpoint{1.507149in}{2.784746in}}{\pgfqpoint{1.515049in}{2.788018in}}{\pgfqpoint{1.520873in}{2.793842in}}%
\pgfpathcurveto{\pgfqpoint{1.526696in}{2.799666in}}{\pgfqpoint{1.529969in}{2.807566in}}{\pgfqpoint{1.529969in}{2.815802in}}%
\pgfpathcurveto{\pgfqpoint{1.529969in}{2.824039in}}{\pgfqpoint{1.526696in}{2.831939in}}{\pgfqpoint{1.520873in}{2.837763in}}%
\pgfpathcurveto{\pgfqpoint{1.515049in}{2.843587in}}{\pgfqpoint{1.507149in}{2.846859in}}{\pgfqpoint{1.498912in}{2.846859in}}%
\pgfpathcurveto{\pgfqpoint{1.490676in}{2.846859in}}{\pgfqpoint{1.482776in}{2.843587in}}{\pgfqpoint{1.476952in}{2.837763in}}%
\pgfpathcurveto{\pgfqpoint{1.471128in}{2.831939in}}{\pgfqpoint{1.467856in}{2.824039in}}{\pgfqpoint{1.467856in}{2.815802in}}%
\pgfpathcurveto{\pgfqpoint{1.467856in}{2.807566in}}{\pgfqpoint{1.471128in}{2.799666in}}{\pgfqpoint{1.476952in}{2.793842in}}%
\pgfpathcurveto{\pgfqpoint{1.482776in}{2.788018in}}{\pgfqpoint{1.490676in}{2.784746in}}{\pgfqpoint{1.498912in}{2.784746in}}%
\pgfpathclose%
\pgfusepath{stroke,fill}%
\end{pgfscope}%
\begin{pgfscope}%
\pgfpathrectangle{\pgfqpoint{0.100000in}{0.212622in}}{\pgfqpoint{3.696000in}{3.696000in}}%
\pgfusepath{clip}%
\pgfsetbuttcap%
\pgfsetroundjoin%
\definecolor{currentfill}{rgb}{0.121569,0.466667,0.705882}%
\pgfsetfillcolor{currentfill}%
\pgfsetfillopacity{0.693684}%
\pgfsetlinewidth{1.003750pt}%
\definecolor{currentstroke}{rgb}{0.121569,0.466667,0.705882}%
\pgfsetstrokecolor{currentstroke}%
\pgfsetstrokeopacity{0.693684}%
\pgfsetdash{}{0pt}%
\pgfpathmoveto{\pgfqpoint{2.405361in}{2.998431in}}%
\pgfpathcurveto{\pgfqpoint{2.413598in}{2.998431in}}{\pgfqpoint{2.421498in}{3.001704in}}{\pgfqpoint{2.427322in}{3.007527in}}%
\pgfpathcurveto{\pgfqpoint{2.433146in}{3.013351in}}{\pgfqpoint{2.436418in}{3.021251in}}{\pgfqpoint{2.436418in}{3.029488in}}%
\pgfpathcurveto{\pgfqpoint{2.436418in}{3.037724in}}{\pgfqpoint{2.433146in}{3.045624in}}{\pgfqpoint{2.427322in}{3.051448in}}%
\pgfpathcurveto{\pgfqpoint{2.421498in}{3.057272in}}{\pgfqpoint{2.413598in}{3.060544in}}{\pgfqpoint{2.405361in}{3.060544in}}%
\pgfpathcurveto{\pgfqpoint{2.397125in}{3.060544in}}{\pgfqpoint{2.389225in}{3.057272in}}{\pgfqpoint{2.383401in}{3.051448in}}%
\pgfpathcurveto{\pgfqpoint{2.377577in}{3.045624in}}{\pgfqpoint{2.374305in}{3.037724in}}{\pgfqpoint{2.374305in}{3.029488in}}%
\pgfpathcurveto{\pgfqpoint{2.374305in}{3.021251in}}{\pgfqpoint{2.377577in}{3.013351in}}{\pgfqpoint{2.383401in}{3.007527in}}%
\pgfpathcurveto{\pgfqpoint{2.389225in}{3.001704in}}{\pgfqpoint{2.397125in}{2.998431in}}{\pgfqpoint{2.405361in}{2.998431in}}%
\pgfpathclose%
\pgfusepath{stroke,fill}%
\end{pgfscope}%
\begin{pgfscope}%
\pgfpathrectangle{\pgfqpoint{0.100000in}{0.212622in}}{\pgfqpoint{3.696000in}{3.696000in}}%
\pgfusepath{clip}%
\pgfsetbuttcap%
\pgfsetroundjoin%
\definecolor{currentfill}{rgb}{0.121569,0.466667,0.705882}%
\pgfsetfillcolor{currentfill}%
\pgfsetfillopacity{0.694526}%
\pgfsetlinewidth{1.003750pt}%
\definecolor{currentstroke}{rgb}{0.121569,0.466667,0.705882}%
\pgfsetstrokecolor{currentstroke}%
\pgfsetstrokeopacity{0.694526}%
\pgfsetdash{}{0pt}%
\pgfpathmoveto{\pgfqpoint{1.495243in}{2.780110in}}%
\pgfpathcurveto{\pgfqpoint{1.503479in}{2.780110in}}{\pgfqpoint{1.511379in}{2.783382in}}{\pgfqpoint{1.517203in}{2.789206in}}%
\pgfpathcurveto{\pgfqpoint{1.523027in}{2.795030in}}{\pgfqpoint{1.526299in}{2.802930in}}{\pgfqpoint{1.526299in}{2.811166in}}%
\pgfpathcurveto{\pgfqpoint{1.526299in}{2.819402in}}{\pgfqpoint{1.523027in}{2.827302in}}{\pgfqpoint{1.517203in}{2.833126in}}%
\pgfpathcurveto{\pgfqpoint{1.511379in}{2.838950in}}{\pgfqpoint{1.503479in}{2.842223in}}{\pgfqpoint{1.495243in}{2.842223in}}%
\pgfpathcurveto{\pgfqpoint{1.487007in}{2.842223in}}{\pgfqpoint{1.479106in}{2.838950in}}{\pgfqpoint{1.473283in}{2.833126in}}%
\pgfpathcurveto{\pgfqpoint{1.467459in}{2.827302in}}{\pgfqpoint{1.464186in}{2.819402in}}{\pgfqpoint{1.464186in}{2.811166in}}%
\pgfpathcurveto{\pgfqpoint{1.464186in}{2.802930in}}{\pgfqpoint{1.467459in}{2.795030in}}{\pgfqpoint{1.473283in}{2.789206in}}%
\pgfpathcurveto{\pgfqpoint{1.479106in}{2.783382in}}{\pgfqpoint{1.487007in}{2.780110in}}{\pgfqpoint{1.495243in}{2.780110in}}%
\pgfpathclose%
\pgfusepath{stroke,fill}%
\end{pgfscope}%
\begin{pgfscope}%
\pgfpathrectangle{\pgfqpoint{0.100000in}{0.212622in}}{\pgfqpoint{3.696000in}{3.696000in}}%
\pgfusepath{clip}%
\pgfsetbuttcap%
\pgfsetroundjoin%
\definecolor{currentfill}{rgb}{0.121569,0.466667,0.705882}%
\pgfsetfillcolor{currentfill}%
\pgfsetfillopacity{0.694712}%
\pgfsetlinewidth{1.003750pt}%
\definecolor{currentstroke}{rgb}{0.121569,0.466667,0.705882}%
\pgfsetstrokecolor{currentstroke}%
\pgfsetstrokeopacity{0.694712}%
\pgfsetdash{}{0pt}%
\pgfpathmoveto{\pgfqpoint{2.413166in}{2.997456in}}%
\pgfpathcurveto{\pgfqpoint{2.421402in}{2.997456in}}{\pgfqpoint{2.429303in}{3.000729in}}{\pgfqpoint{2.435126in}{3.006552in}}%
\pgfpathcurveto{\pgfqpoint{2.440950in}{3.012376in}}{\pgfqpoint{2.444223in}{3.020276in}}{\pgfqpoint{2.444223in}{3.028513in}}%
\pgfpathcurveto{\pgfqpoint{2.444223in}{3.036749in}}{\pgfqpoint{2.440950in}{3.044649in}}{\pgfqpoint{2.435126in}{3.050473in}}%
\pgfpathcurveto{\pgfqpoint{2.429303in}{3.056297in}}{\pgfqpoint{2.421402in}{3.059569in}}{\pgfqpoint{2.413166in}{3.059569in}}%
\pgfpathcurveto{\pgfqpoint{2.404930in}{3.059569in}}{\pgfqpoint{2.397030in}{3.056297in}}{\pgfqpoint{2.391206in}{3.050473in}}%
\pgfpathcurveto{\pgfqpoint{2.385382in}{3.044649in}}{\pgfqpoint{2.382110in}{3.036749in}}{\pgfqpoint{2.382110in}{3.028513in}}%
\pgfpathcurveto{\pgfqpoint{2.382110in}{3.020276in}}{\pgfqpoint{2.385382in}{3.012376in}}{\pgfqpoint{2.391206in}{3.006552in}}%
\pgfpathcurveto{\pgfqpoint{2.397030in}{3.000729in}}{\pgfqpoint{2.404930in}{2.997456in}}{\pgfqpoint{2.413166in}{2.997456in}}%
\pgfpathclose%
\pgfusepath{stroke,fill}%
\end{pgfscope}%
\begin{pgfscope}%
\pgfpathrectangle{\pgfqpoint{0.100000in}{0.212622in}}{\pgfqpoint{3.696000in}{3.696000in}}%
\pgfusepath{clip}%
\pgfsetbuttcap%
\pgfsetroundjoin%
\definecolor{currentfill}{rgb}{0.121569,0.466667,0.705882}%
\pgfsetfillcolor{currentfill}%
\pgfsetfillopacity{0.695653}%
\pgfsetlinewidth{1.003750pt}%
\definecolor{currentstroke}{rgb}{0.121569,0.466667,0.705882}%
\pgfsetstrokecolor{currentstroke}%
\pgfsetstrokeopacity{0.695653}%
\pgfsetdash{}{0pt}%
\pgfpathmoveto{\pgfqpoint{2.420373in}{2.996595in}}%
\pgfpathcurveto{\pgfqpoint{2.428609in}{2.996595in}}{\pgfqpoint{2.436509in}{2.999868in}}{\pgfqpoint{2.442333in}{3.005692in}}%
\pgfpathcurveto{\pgfqpoint{2.448157in}{3.011515in}}{\pgfqpoint{2.451429in}{3.019415in}}{\pgfqpoint{2.451429in}{3.027652in}}%
\pgfpathcurveto{\pgfqpoint{2.451429in}{3.035888in}}{\pgfqpoint{2.448157in}{3.043788in}}{\pgfqpoint{2.442333in}{3.049612in}}%
\pgfpathcurveto{\pgfqpoint{2.436509in}{3.055436in}}{\pgfqpoint{2.428609in}{3.058708in}}{\pgfqpoint{2.420373in}{3.058708in}}%
\pgfpathcurveto{\pgfqpoint{2.412136in}{3.058708in}}{\pgfqpoint{2.404236in}{3.055436in}}{\pgfqpoint{2.398412in}{3.049612in}}%
\pgfpathcurveto{\pgfqpoint{2.392589in}{3.043788in}}{\pgfqpoint{2.389316in}{3.035888in}}{\pgfqpoint{2.389316in}{3.027652in}}%
\pgfpathcurveto{\pgfqpoint{2.389316in}{3.019415in}}{\pgfqpoint{2.392589in}{3.011515in}}{\pgfqpoint{2.398412in}{3.005692in}}%
\pgfpathcurveto{\pgfqpoint{2.404236in}{2.999868in}}{\pgfqpoint{2.412136in}{2.996595in}}{\pgfqpoint{2.420373in}{2.996595in}}%
\pgfpathclose%
\pgfusepath{stroke,fill}%
\end{pgfscope}%
\begin{pgfscope}%
\pgfpathrectangle{\pgfqpoint{0.100000in}{0.212622in}}{\pgfqpoint{3.696000in}{3.696000in}}%
\pgfusepath{clip}%
\pgfsetbuttcap%
\pgfsetroundjoin%
\definecolor{currentfill}{rgb}{0.121569,0.466667,0.705882}%
\pgfsetfillcolor{currentfill}%
\pgfsetfillopacity{0.695851}%
\pgfsetlinewidth{1.003750pt}%
\definecolor{currentstroke}{rgb}{0.121569,0.466667,0.705882}%
\pgfsetstrokecolor{currentstroke}%
\pgfsetstrokeopacity{0.695851}%
\pgfsetdash{}{0pt}%
\pgfpathmoveto{\pgfqpoint{1.490033in}{2.773103in}}%
\pgfpathcurveto{\pgfqpoint{1.498270in}{2.773103in}}{\pgfqpoint{1.506170in}{2.776376in}}{\pgfqpoint{1.511994in}{2.782200in}}%
\pgfpathcurveto{\pgfqpoint{1.517818in}{2.788024in}}{\pgfqpoint{1.521090in}{2.795924in}}{\pgfqpoint{1.521090in}{2.804160in}}%
\pgfpathcurveto{\pgfqpoint{1.521090in}{2.812396in}}{\pgfqpoint{1.517818in}{2.820296in}}{\pgfqpoint{1.511994in}{2.826120in}}%
\pgfpathcurveto{\pgfqpoint{1.506170in}{2.831944in}}{\pgfqpoint{1.498270in}{2.835216in}}{\pgfqpoint{1.490033in}{2.835216in}}%
\pgfpathcurveto{\pgfqpoint{1.481797in}{2.835216in}}{\pgfqpoint{1.473897in}{2.831944in}}{\pgfqpoint{1.468073in}{2.826120in}}%
\pgfpathcurveto{\pgfqpoint{1.462249in}{2.820296in}}{\pgfqpoint{1.458977in}{2.812396in}}{\pgfqpoint{1.458977in}{2.804160in}}%
\pgfpathcurveto{\pgfqpoint{1.458977in}{2.795924in}}{\pgfqpoint{1.462249in}{2.788024in}}{\pgfqpoint{1.468073in}{2.782200in}}%
\pgfpathcurveto{\pgfqpoint{1.473897in}{2.776376in}}{\pgfqpoint{1.481797in}{2.773103in}}{\pgfqpoint{1.490033in}{2.773103in}}%
\pgfpathclose%
\pgfusepath{stroke,fill}%
\end{pgfscope}%
\begin{pgfscope}%
\pgfpathrectangle{\pgfqpoint{0.100000in}{0.212622in}}{\pgfqpoint{3.696000in}{3.696000in}}%
\pgfusepath{clip}%
\pgfsetbuttcap%
\pgfsetroundjoin%
\definecolor{currentfill}{rgb}{0.121569,0.466667,0.705882}%
\pgfsetfillcolor{currentfill}%
\pgfsetfillopacity{0.696617}%
\pgfsetlinewidth{1.003750pt}%
\definecolor{currentstroke}{rgb}{0.121569,0.466667,0.705882}%
\pgfsetstrokecolor{currentstroke}%
\pgfsetstrokeopacity{0.696617}%
\pgfsetdash{}{0pt}%
\pgfpathmoveto{\pgfqpoint{2.426740in}{2.995203in}}%
\pgfpathcurveto{\pgfqpoint{2.434976in}{2.995203in}}{\pgfqpoint{2.442876in}{2.998475in}}{\pgfqpoint{2.448700in}{3.004299in}}%
\pgfpathcurveto{\pgfqpoint{2.454524in}{3.010123in}}{\pgfqpoint{2.457796in}{3.018023in}}{\pgfqpoint{2.457796in}{3.026259in}}%
\pgfpathcurveto{\pgfqpoint{2.457796in}{3.034496in}}{\pgfqpoint{2.454524in}{3.042396in}}{\pgfqpoint{2.448700in}{3.048220in}}%
\pgfpathcurveto{\pgfqpoint{2.442876in}{3.054043in}}{\pgfqpoint{2.434976in}{3.057316in}}{\pgfqpoint{2.426740in}{3.057316in}}%
\pgfpathcurveto{\pgfqpoint{2.418504in}{3.057316in}}{\pgfqpoint{2.410604in}{3.054043in}}{\pgfqpoint{2.404780in}{3.048220in}}%
\pgfpathcurveto{\pgfqpoint{2.398956in}{3.042396in}}{\pgfqpoint{2.395683in}{3.034496in}}{\pgfqpoint{2.395683in}{3.026259in}}%
\pgfpathcurveto{\pgfqpoint{2.395683in}{3.018023in}}{\pgfqpoint{2.398956in}{3.010123in}}{\pgfqpoint{2.404780in}{3.004299in}}%
\pgfpathcurveto{\pgfqpoint{2.410604in}{2.998475in}}{\pgfqpoint{2.418504in}{2.995203in}}{\pgfqpoint{2.426740in}{2.995203in}}%
\pgfpathclose%
\pgfusepath{stroke,fill}%
\end{pgfscope}%
\begin{pgfscope}%
\pgfpathrectangle{\pgfqpoint{0.100000in}{0.212622in}}{\pgfqpoint{3.696000in}{3.696000in}}%
\pgfusepath{clip}%
\pgfsetbuttcap%
\pgfsetroundjoin%
\definecolor{currentfill}{rgb}{0.121569,0.466667,0.705882}%
\pgfsetfillcolor{currentfill}%
\pgfsetfillopacity{0.697232}%
\pgfsetlinewidth{1.003750pt}%
\definecolor{currentstroke}{rgb}{0.121569,0.466667,0.705882}%
\pgfsetstrokecolor{currentstroke}%
\pgfsetstrokeopacity{0.697232}%
\pgfsetdash{}{0pt}%
\pgfpathmoveto{\pgfqpoint{1.484119in}{2.763794in}}%
\pgfpathcurveto{\pgfqpoint{1.492356in}{2.763794in}}{\pgfqpoint{1.500256in}{2.767066in}}{\pgfqpoint{1.506080in}{2.772890in}}%
\pgfpathcurveto{\pgfqpoint{1.511904in}{2.778714in}}{\pgfqpoint{1.515176in}{2.786614in}}{\pgfqpoint{1.515176in}{2.794850in}}%
\pgfpathcurveto{\pgfqpoint{1.515176in}{2.803087in}}{\pgfqpoint{1.511904in}{2.810987in}}{\pgfqpoint{1.506080in}{2.816811in}}%
\pgfpathcurveto{\pgfqpoint{1.500256in}{2.822635in}}{\pgfqpoint{1.492356in}{2.825907in}}{\pgfqpoint{1.484119in}{2.825907in}}%
\pgfpathcurveto{\pgfqpoint{1.475883in}{2.825907in}}{\pgfqpoint{1.467983in}{2.822635in}}{\pgfqpoint{1.462159in}{2.816811in}}%
\pgfpathcurveto{\pgfqpoint{1.456335in}{2.810987in}}{\pgfqpoint{1.453063in}{2.803087in}}{\pgfqpoint{1.453063in}{2.794850in}}%
\pgfpathcurveto{\pgfqpoint{1.453063in}{2.786614in}}{\pgfqpoint{1.456335in}{2.778714in}}{\pgfqpoint{1.462159in}{2.772890in}}%
\pgfpathcurveto{\pgfqpoint{1.467983in}{2.767066in}}{\pgfqpoint{1.475883in}{2.763794in}}{\pgfqpoint{1.484119in}{2.763794in}}%
\pgfpathclose%
\pgfusepath{stroke,fill}%
\end{pgfscope}%
\begin{pgfscope}%
\pgfpathrectangle{\pgfqpoint{0.100000in}{0.212622in}}{\pgfqpoint{3.696000in}{3.696000in}}%
\pgfusepath{clip}%
\pgfsetbuttcap%
\pgfsetroundjoin%
\definecolor{currentfill}{rgb}{0.121569,0.466667,0.705882}%
\pgfsetfillcolor{currentfill}%
\pgfsetfillopacity{0.697438}%
\pgfsetlinewidth{1.003750pt}%
\definecolor{currentstroke}{rgb}{0.121569,0.466667,0.705882}%
\pgfsetstrokecolor{currentstroke}%
\pgfsetstrokeopacity{0.697438}%
\pgfsetdash{}{0pt}%
\pgfpathmoveto{\pgfqpoint{2.432070in}{2.994129in}}%
\pgfpathcurveto{\pgfqpoint{2.440306in}{2.994129in}}{\pgfqpoint{2.448206in}{2.997402in}}{\pgfqpoint{2.454030in}{3.003225in}}%
\pgfpathcurveto{\pgfqpoint{2.459854in}{3.009049in}}{\pgfqpoint{2.463126in}{3.016949in}}{\pgfqpoint{2.463126in}{3.025186in}}%
\pgfpathcurveto{\pgfqpoint{2.463126in}{3.033422in}}{\pgfqpoint{2.459854in}{3.041322in}}{\pgfqpoint{2.454030in}{3.047146in}}%
\pgfpathcurveto{\pgfqpoint{2.448206in}{3.052970in}}{\pgfqpoint{2.440306in}{3.056242in}}{\pgfqpoint{2.432070in}{3.056242in}}%
\pgfpathcurveto{\pgfqpoint{2.423833in}{3.056242in}}{\pgfqpoint{2.415933in}{3.052970in}}{\pgfqpoint{2.410109in}{3.047146in}}%
\pgfpathcurveto{\pgfqpoint{2.404286in}{3.041322in}}{\pgfqpoint{2.401013in}{3.033422in}}{\pgfqpoint{2.401013in}{3.025186in}}%
\pgfpathcurveto{\pgfqpoint{2.401013in}{3.016949in}}{\pgfqpoint{2.404286in}{3.009049in}}{\pgfqpoint{2.410109in}{3.003225in}}%
\pgfpathcurveto{\pgfqpoint{2.415933in}{2.997402in}}{\pgfqpoint{2.423833in}{2.994129in}}{\pgfqpoint{2.432070in}{2.994129in}}%
\pgfpathclose%
\pgfusepath{stroke,fill}%
\end{pgfscope}%
\begin{pgfscope}%
\pgfpathrectangle{\pgfqpoint{0.100000in}{0.212622in}}{\pgfqpoint{3.696000in}{3.696000in}}%
\pgfusepath{clip}%
\pgfsetbuttcap%
\pgfsetroundjoin%
\definecolor{currentfill}{rgb}{0.121569,0.466667,0.705882}%
\pgfsetfillcolor{currentfill}%
\pgfsetfillopacity{0.698014}%
\pgfsetlinewidth{1.003750pt}%
\definecolor{currentstroke}{rgb}{0.121569,0.466667,0.705882}%
\pgfsetstrokecolor{currentstroke}%
\pgfsetstrokeopacity{0.698014}%
\pgfsetdash{}{0pt}%
\pgfpathmoveto{\pgfqpoint{3.068421in}{1.763642in}}%
\pgfpathcurveto{\pgfqpoint{3.076657in}{1.763642in}}{\pgfqpoint{3.084557in}{1.766914in}}{\pgfqpoint{3.090381in}{1.772738in}}%
\pgfpathcurveto{\pgfqpoint{3.096205in}{1.778562in}}{\pgfqpoint{3.099478in}{1.786462in}}{\pgfqpoint{3.099478in}{1.794699in}}%
\pgfpathcurveto{\pgfqpoint{3.099478in}{1.802935in}}{\pgfqpoint{3.096205in}{1.810835in}}{\pgfqpoint{3.090381in}{1.816659in}}%
\pgfpathcurveto{\pgfqpoint{3.084557in}{1.822483in}}{\pgfqpoint{3.076657in}{1.825755in}}{\pgfqpoint{3.068421in}{1.825755in}}%
\pgfpathcurveto{\pgfqpoint{3.060185in}{1.825755in}}{\pgfqpoint{3.052285in}{1.822483in}}{\pgfqpoint{3.046461in}{1.816659in}}%
\pgfpathcurveto{\pgfqpoint{3.040637in}{1.810835in}}{\pgfqpoint{3.037365in}{1.802935in}}{\pgfqpoint{3.037365in}{1.794699in}}%
\pgfpathcurveto{\pgfqpoint{3.037365in}{1.786462in}}{\pgfqpoint{3.040637in}{1.778562in}}{\pgfqpoint{3.046461in}{1.772738in}}%
\pgfpathcurveto{\pgfqpoint{3.052285in}{1.766914in}}{\pgfqpoint{3.060185in}{1.763642in}}{\pgfqpoint{3.068421in}{1.763642in}}%
\pgfpathclose%
\pgfusepath{stroke,fill}%
\end{pgfscope}%
\begin{pgfscope}%
\pgfpathrectangle{\pgfqpoint{0.100000in}{0.212622in}}{\pgfqpoint{3.696000in}{3.696000in}}%
\pgfusepath{clip}%
\pgfsetbuttcap%
\pgfsetroundjoin%
\definecolor{currentfill}{rgb}{0.121569,0.466667,0.705882}%
\pgfsetfillcolor{currentfill}%
\pgfsetfillopacity{0.698043}%
\pgfsetlinewidth{1.003750pt}%
\definecolor{currentstroke}{rgb}{0.121569,0.466667,0.705882}%
\pgfsetstrokecolor{currentstroke}%
\pgfsetstrokeopacity{0.698043}%
\pgfsetdash{}{0pt}%
\pgfpathmoveto{\pgfqpoint{1.480901in}{2.758929in}}%
\pgfpathcurveto{\pgfqpoint{1.489137in}{2.758929in}}{\pgfqpoint{1.497037in}{2.762201in}}{\pgfqpoint{1.502861in}{2.768025in}}%
\pgfpathcurveto{\pgfqpoint{1.508685in}{2.773849in}}{\pgfqpoint{1.511957in}{2.781749in}}{\pgfqpoint{1.511957in}{2.789985in}}%
\pgfpathcurveto{\pgfqpoint{1.511957in}{2.798222in}}{\pgfqpoint{1.508685in}{2.806122in}}{\pgfqpoint{1.502861in}{2.811946in}}%
\pgfpathcurveto{\pgfqpoint{1.497037in}{2.817770in}}{\pgfqpoint{1.489137in}{2.821042in}}{\pgfqpoint{1.480901in}{2.821042in}}%
\pgfpathcurveto{\pgfqpoint{1.472664in}{2.821042in}}{\pgfqpoint{1.464764in}{2.817770in}}{\pgfqpoint{1.458940in}{2.811946in}}%
\pgfpathcurveto{\pgfqpoint{1.453116in}{2.806122in}}{\pgfqpoint{1.449844in}{2.798222in}}{\pgfqpoint{1.449844in}{2.789985in}}%
\pgfpathcurveto{\pgfqpoint{1.449844in}{2.781749in}}{\pgfqpoint{1.453116in}{2.773849in}}{\pgfqpoint{1.458940in}{2.768025in}}%
\pgfpathcurveto{\pgfqpoint{1.464764in}{2.762201in}}{\pgfqpoint{1.472664in}{2.758929in}}{\pgfqpoint{1.480901in}{2.758929in}}%
\pgfpathclose%
\pgfusepath{stroke,fill}%
\end{pgfscope}%
\begin{pgfscope}%
\pgfpathrectangle{\pgfqpoint{0.100000in}{0.212622in}}{\pgfqpoint{3.696000in}{3.696000in}}%
\pgfusepath{clip}%
\pgfsetbuttcap%
\pgfsetroundjoin%
\definecolor{currentfill}{rgb}{0.121569,0.466667,0.705882}%
\pgfsetfillcolor{currentfill}%
\pgfsetfillopacity{0.698146}%
\pgfsetlinewidth{1.003750pt}%
\definecolor{currentstroke}{rgb}{0.121569,0.466667,0.705882}%
\pgfsetstrokecolor{currentstroke}%
\pgfsetstrokeopacity{0.698146}%
\pgfsetdash{}{0pt}%
\pgfpathmoveto{\pgfqpoint{2.437092in}{2.993127in}}%
\pgfpathcurveto{\pgfqpoint{2.445328in}{2.993127in}}{\pgfqpoint{2.453228in}{2.996399in}}{\pgfqpoint{2.459052in}{3.002223in}}%
\pgfpathcurveto{\pgfqpoint{2.464876in}{3.008047in}}{\pgfqpoint{2.468149in}{3.015947in}}{\pgfqpoint{2.468149in}{3.024183in}}%
\pgfpathcurveto{\pgfqpoint{2.468149in}{3.032419in}}{\pgfqpoint{2.464876in}{3.040319in}}{\pgfqpoint{2.459052in}{3.046143in}}%
\pgfpathcurveto{\pgfqpoint{2.453228in}{3.051967in}}{\pgfqpoint{2.445328in}{3.055240in}}{\pgfqpoint{2.437092in}{3.055240in}}%
\pgfpathcurveto{\pgfqpoint{2.428856in}{3.055240in}}{\pgfqpoint{2.420956in}{3.051967in}}{\pgfqpoint{2.415132in}{3.046143in}}%
\pgfpathcurveto{\pgfqpoint{2.409308in}{3.040319in}}{\pgfqpoint{2.406036in}{3.032419in}}{\pgfqpoint{2.406036in}{3.024183in}}%
\pgfpathcurveto{\pgfqpoint{2.406036in}{3.015947in}}{\pgfqpoint{2.409308in}{3.008047in}}{\pgfqpoint{2.415132in}{3.002223in}}%
\pgfpathcurveto{\pgfqpoint{2.420956in}{2.996399in}}{\pgfqpoint{2.428856in}{2.993127in}}{\pgfqpoint{2.437092in}{2.993127in}}%
\pgfpathclose%
\pgfusepath{stroke,fill}%
\end{pgfscope}%
\begin{pgfscope}%
\pgfpathrectangle{\pgfqpoint{0.100000in}{0.212622in}}{\pgfqpoint{3.696000in}{3.696000in}}%
\pgfusepath{clip}%
\pgfsetbuttcap%
\pgfsetroundjoin%
\definecolor{currentfill}{rgb}{0.121569,0.466667,0.705882}%
\pgfsetfillcolor{currentfill}%
\pgfsetfillopacity{0.698527}%
\pgfsetlinewidth{1.003750pt}%
\definecolor{currentstroke}{rgb}{0.121569,0.466667,0.705882}%
\pgfsetstrokecolor{currentstroke}%
\pgfsetstrokeopacity{0.698527}%
\pgfsetdash{}{0pt}%
\pgfpathmoveto{\pgfqpoint{1.479138in}{2.756467in}}%
\pgfpathcurveto{\pgfqpoint{1.487374in}{2.756467in}}{\pgfqpoint{1.495274in}{2.759739in}}{\pgfqpoint{1.501098in}{2.765563in}}%
\pgfpathcurveto{\pgfqpoint{1.506922in}{2.771387in}}{\pgfqpoint{1.510194in}{2.779287in}}{\pgfqpoint{1.510194in}{2.787523in}}%
\pgfpathcurveto{\pgfqpoint{1.510194in}{2.795759in}}{\pgfqpoint{1.506922in}{2.803660in}}{\pgfqpoint{1.501098in}{2.809483in}}%
\pgfpathcurveto{\pgfqpoint{1.495274in}{2.815307in}}{\pgfqpoint{1.487374in}{2.818580in}}{\pgfqpoint{1.479138in}{2.818580in}}%
\pgfpathcurveto{\pgfqpoint{1.470902in}{2.818580in}}{\pgfqpoint{1.463002in}{2.815307in}}{\pgfqpoint{1.457178in}{2.809483in}}%
\pgfpathcurveto{\pgfqpoint{1.451354in}{2.803660in}}{\pgfqpoint{1.448081in}{2.795759in}}{\pgfqpoint{1.448081in}{2.787523in}}%
\pgfpathcurveto{\pgfqpoint{1.448081in}{2.779287in}}{\pgfqpoint{1.451354in}{2.771387in}}{\pgfqpoint{1.457178in}{2.765563in}}%
\pgfpathcurveto{\pgfqpoint{1.463002in}{2.759739in}}{\pgfqpoint{1.470902in}{2.756467in}}{\pgfqpoint{1.479138in}{2.756467in}}%
\pgfpathclose%
\pgfusepath{stroke,fill}%
\end{pgfscope}%
\begin{pgfscope}%
\pgfpathrectangle{\pgfqpoint{0.100000in}{0.212622in}}{\pgfqpoint{3.696000in}{3.696000in}}%
\pgfusepath{clip}%
\pgfsetbuttcap%
\pgfsetroundjoin%
\definecolor{currentfill}{rgb}{0.121569,0.466667,0.705882}%
\pgfsetfillcolor{currentfill}%
\pgfsetfillopacity{0.698675}%
\pgfsetlinewidth{1.003750pt}%
\definecolor{currentstroke}{rgb}{0.121569,0.466667,0.705882}%
\pgfsetstrokecolor{currentstroke}%
\pgfsetstrokeopacity{0.698675}%
\pgfsetdash{}{0pt}%
\pgfpathmoveto{\pgfqpoint{2.441143in}{2.992300in}}%
\pgfpathcurveto{\pgfqpoint{2.449380in}{2.992300in}}{\pgfqpoint{2.457280in}{2.995573in}}{\pgfqpoint{2.463104in}{3.001397in}}%
\pgfpathcurveto{\pgfqpoint{2.468928in}{3.007220in}}{\pgfqpoint{2.472200in}{3.015121in}}{\pgfqpoint{2.472200in}{3.023357in}}%
\pgfpathcurveto{\pgfqpoint{2.472200in}{3.031593in}}{\pgfqpoint{2.468928in}{3.039493in}}{\pgfqpoint{2.463104in}{3.045317in}}%
\pgfpathcurveto{\pgfqpoint{2.457280in}{3.051141in}}{\pgfqpoint{2.449380in}{3.054413in}}{\pgfqpoint{2.441143in}{3.054413in}}%
\pgfpathcurveto{\pgfqpoint{2.432907in}{3.054413in}}{\pgfqpoint{2.425007in}{3.051141in}}{\pgfqpoint{2.419183in}{3.045317in}}%
\pgfpathcurveto{\pgfqpoint{2.413359in}{3.039493in}}{\pgfqpoint{2.410087in}{3.031593in}}{\pgfqpoint{2.410087in}{3.023357in}}%
\pgfpathcurveto{\pgfqpoint{2.410087in}{3.015121in}}{\pgfqpoint{2.413359in}{3.007220in}}{\pgfqpoint{2.419183in}{3.001397in}}%
\pgfpathcurveto{\pgfqpoint{2.425007in}{2.995573in}}{\pgfqpoint{2.432907in}{2.992300in}}{\pgfqpoint{2.441143in}{2.992300in}}%
\pgfpathclose%
\pgfusepath{stroke,fill}%
\end{pgfscope}%
\begin{pgfscope}%
\pgfpathrectangle{\pgfqpoint{0.100000in}{0.212622in}}{\pgfqpoint{3.696000in}{3.696000in}}%
\pgfusepath{clip}%
\pgfsetbuttcap%
\pgfsetroundjoin%
\definecolor{currentfill}{rgb}{0.121569,0.466667,0.705882}%
\pgfsetfillcolor{currentfill}%
\pgfsetfillopacity{0.699189}%
\pgfsetlinewidth{1.003750pt}%
\definecolor{currentstroke}{rgb}{0.121569,0.466667,0.705882}%
\pgfsetstrokecolor{currentstroke}%
\pgfsetstrokeopacity{0.699189}%
\pgfsetdash{}{0pt}%
\pgfpathmoveto{\pgfqpoint{1.476884in}{2.753812in}}%
\pgfpathcurveto{\pgfqpoint{1.485121in}{2.753812in}}{\pgfqpoint{1.493021in}{2.757085in}}{\pgfqpoint{1.498845in}{2.762908in}}%
\pgfpathcurveto{\pgfqpoint{1.504669in}{2.768732in}}{\pgfqpoint{1.507941in}{2.776632in}}{\pgfqpoint{1.507941in}{2.784869in}}%
\pgfpathcurveto{\pgfqpoint{1.507941in}{2.793105in}}{\pgfqpoint{1.504669in}{2.801005in}}{\pgfqpoint{1.498845in}{2.806829in}}%
\pgfpathcurveto{\pgfqpoint{1.493021in}{2.812653in}}{\pgfqpoint{1.485121in}{2.815925in}}{\pgfqpoint{1.476884in}{2.815925in}}%
\pgfpathcurveto{\pgfqpoint{1.468648in}{2.815925in}}{\pgfqpoint{1.460748in}{2.812653in}}{\pgfqpoint{1.454924in}{2.806829in}}%
\pgfpathcurveto{\pgfqpoint{1.449100in}{2.801005in}}{\pgfqpoint{1.445828in}{2.793105in}}{\pgfqpoint{1.445828in}{2.784869in}}%
\pgfpathcurveto{\pgfqpoint{1.445828in}{2.776632in}}{\pgfqpoint{1.449100in}{2.768732in}}{\pgfqpoint{1.454924in}{2.762908in}}%
\pgfpathcurveto{\pgfqpoint{1.460748in}{2.757085in}}{\pgfqpoint{1.468648in}{2.753812in}}{\pgfqpoint{1.476884in}{2.753812in}}%
\pgfpathclose%
\pgfusepath{stroke,fill}%
\end{pgfscope}%
\begin{pgfscope}%
\pgfpathrectangle{\pgfqpoint{0.100000in}{0.212622in}}{\pgfqpoint{3.696000in}{3.696000in}}%
\pgfusepath{clip}%
\pgfsetbuttcap%
\pgfsetroundjoin%
\definecolor{currentfill}{rgb}{0.121569,0.466667,0.705882}%
\pgfsetfillcolor{currentfill}%
\pgfsetfillopacity{0.699723}%
\pgfsetlinewidth{1.003750pt}%
\definecolor{currentstroke}{rgb}{0.121569,0.466667,0.705882}%
\pgfsetstrokecolor{currentstroke}%
\pgfsetstrokeopacity{0.699723}%
\pgfsetdash{}{0pt}%
\pgfpathmoveto{\pgfqpoint{2.448534in}{2.991384in}}%
\pgfpathcurveto{\pgfqpoint{2.456770in}{2.991384in}}{\pgfqpoint{2.464670in}{2.994656in}}{\pgfqpoint{2.470494in}{3.000480in}}%
\pgfpathcurveto{\pgfqpoint{2.476318in}{3.006304in}}{\pgfqpoint{2.479590in}{3.014204in}}{\pgfqpoint{2.479590in}{3.022440in}}%
\pgfpathcurveto{\pgfqpoint{2.479590in}{3.030677in}}{\pgfqpoint{2.476318in}{3.038577in}}{\pgfqpoint{2.470494in}{3.044401in}}%
\pgfpathcurveto{\pgfqpoint{2.464670in}{3.050225in}}{\pgfqpoint{2.456770in}{3.053497in}}{\pgfqpoint{2.448534in}{3.053497in}}%
\pgfpathcurveto{\pgfqpoint{2.440297in}{3.053497in}}{\pgfqpoint{2.432397in}{3.050225in}}{\pgfqpoint{2.426573in}{3.044401in}}%
\pgfpathcurveto{\pgfqpoint{2.420749in}{3.038577in}}{\pgfqpoint{2.417477in}{3.030677in}}{\pgfqpoint{2.417477in}{3.022440in}}%
\pgfpathcurveto{\pgfqpoint{2.417477in}{3.014204in}}{\pgfqpoint{2.420749in}{3.006304in}}{\pgfqpoint{2.426573in}{3.000480in}}%
\pgfpathcurveto{\pgfqpoint{2.432397in}{2.994656in}}{\pgfqpoint{2.440297in}{2.991384in}}{\pgfqpoint{2.448534in}{2.991384in}}%
\pgfpathclose%
\pgfusepath{stroke,fill}%
\end{pgfscope}%
\begin{pgfscope}%
\pgfpathrectangle{\pgfqpoint{0.100000in}{0.212622in}}{\pgfqpoint{3.696000in}{3.696000in}}%
\pgfusepath{clip}%
\pgfsetbuttcap%
\pgfsetroundjoin%
\definecolor{currentfill}{rgb}{0.121569,0.466667,0.705882}%
\pgfsetfillcolor{currentfill}%
\pgfsetfillopacity{0.700139}%
\pgfsetlinewidth{1.003750pt}%
\definecolor{currentstroke}{rgb}{0.121569,0.466667,0.705882}%
\pgfsetstrokecolor{currentstroke}%
\pgfsetstrokeopacity{0.700139}%
\pgfsetdash{}{0pt}%
\pgfpathmoveto{\pgfqpoint{1.473464in}{2.749576in}}%
\pgfpathcurveto{\pgfqpoint{1.481700in}{2.749576in}}{\pgfqpoint{1.489600in}{2.752849in}}{\pgfqpoint{1.495424in}{2.758672in}}%
\pgfpathcurveto{\pgfqpoint{1.501248in}{2.764496in}}{\pgfqpoint{1.504520in}{2.772396in}}{\pgfqpoint{1.504520in}{2.780633in}}%
\pgfpathcurveto{\pgfqpoint{1.504520in}{2.788869in}}{\pgfqpoint{1.501248in}{2.796769in}}{\pgfqpoint{1.495424in}{2.802593in}}%
\pgfpathcurveto{\pgfqpoint{1.489600in}{2.808417in}}{\pgfqpoint{1.481700in}{2.811689in}}{\pgfqpoint{1.473464in}{2.811689in}}%
\pgfpathcurveto{\pgfqpoint{1.465228in}{2.811689in}}{\pgfqpoint{1.457327in}{2.808417in}}{\pgfqpoint{1.451504in}{2.802593in}}%
\pgfpathcurveto{\pgfqpoint{1.445680in}{2.796769in}}{\pgfqpoint{1.442407in}{2.788869in}}{\pgfqpoint{1.442407in}{2.780633in}}%
\pgfpathcurveto{\pgfqpoint{1.442407in}{2.772396in}}{\pgfqpoint{1.445680in}{2.764496in}}{\pgfqpoint{1.451504in}{2.758672in}}%
\pgfpathcurveto{\pgfqpoint{1.457327in}{2.752849in}}{\pgfqpoint{1.465228in}{2.749576in}}{\pgfqpoint{1.473464in}{2.749576in}}%
\pgfpathclose%
\pgfusepath{stroke,fill}%
\end{pgfscope}%
\begin{pgfscope}%
\pgfpathrectangle{\pgfqpoint{0.100000in}{0.212622in}}{\pgfqpoint{3.696000in}{3.696000in}}%
\pgfusepath{clip}%
\pgfsetbuttcap%
\pgfsetroundjoin%
\definecolor{currentfill}{rgb}{0.121569,0.466667,0.705882}%
\pgfsetfillcolor{currentfill}%
\pgfsetfillopacity{0.700557}%
\pgfsetlinewidth{1.003750pt}%
\definecolor{currentstroke}{rgb}{0.121569,0.466667,0.705882}%
\pgfsetstrokecolor{currentstroke}%
\pgfsetstrokeopacity{0.700557}%
\pgfsetdash{}{0pt}%
\pgfpathmoveto{\pgfqpoint{2.454432in}{2.990783in}}%
\pgfpathcurveto{\pgfqpoint{2.462668in}{2.990783in}}{\pgfqpoint{2.470568in}{2.994055in}}{\pgfqpoint{2.476392in}{2.999879in}}%
\pgfpathcurveto{\pgfqpoint{2.482216in}{3.005703in}}{\pgfqpoint{2.485488in}{3.013603in}}{\pgfqpoint{2.485488in}{3.021840in}}%
\pgfpathcurveto{\pgfqpoint{2.485488in}{3.030076in}}{\pgfqpoint{2.482216in}{3.037976in}}{\pgfqpoint{2.476392in}{3.043800in}}%
\pgfpathcurveto{\pgfqpoint{2.470568in}{3.049624in}}{\pgfqpoint{2.462668in}{3.052896in}}{\pgfqpoint{2.454432in}{3.052896in}}%
\pgfpathcurveto{\pgfqpoint{2.446196in}{3.052896in}}{\pgfqpoint{2.438296in}{3.049624in}}{\pgfqpoint{2.432472in}{3.043800in}}%
\pgfpathcurveto{\pgfqpoint{2.426648in}{3.037976in}}{\pgfqpoint{2.423375in}{3.030076in}}{\pgfqpoint{2.423375in}{3.021840in}}%
\pgfpathcurveto{\pgfqpoint{2.423375in}{3.013603in}}{\pgfqpoint{2.426648in}{3.005703in}}{\pgfqpoint{2.432472in}{2.999879in}}%
\pgfpathcurveto{\pgfqpoint{2.438296in}{2.994055in}}{\pgfqpoint{2.446196in}{2.990783in}}{\pgfqpoint{2.454432in}{2.990783in}}%
\pgfpathclose%
\pgfusepath{stroke,fill}%
\end{pgfscope}%
\begin{pgfscope}%
\pgfpathrectangle{\pgfqpoint{0.100000in}{0.212622in}}{\pgfqpoint{3.696000in}{3.696000in}}%
\pgfusepath{clip}%
\pgfsetbuttcap%
\pgfsetroundjoin%
\definecolor{currentfill}{rgb}{0.121569,0.466667,0.705882}%
\pgfsetfillcolor{currentfill}%
\pgfsetfillopacity{0.701276}%
\pgfsetlinewidth{1.003750pt}%
\definecolor{currentstroke}{rgb}{0.121569,0.466667,0.705882}%
\pgfsetstrokecolor{currentstroke}%
\pgfsetstrokeopacity{0.701276}%
\pgfsetdash{}{0pt}%
\pgfpathmoveto{\pgfqpoint{1.469236in}{2.743853in}}%
\pgfpathcurveto{\pgfqpoint{1.477473in}{2.743853in}}{\pgfqpoint{1.485373in}{2.747125in}}{\pgfqpoint{1.491196in}{2.752949in}}%
\pgfpathcurveto{\pgfqpoint{1.497020in}{2.758773in}}{\pgfqpoint{1.500293in}{2.766673in}}{\pgfqpoint{1.500293in}{2.774910in}}%
\pgfpathcurveto{\pgfqpoint{1.500293in}{2.783146in}}{\pgfqpoint{1.497020in}{2.791046in}}{\pgfqpoint{1.491196in}{2.796870in}}%
\pgfpathcurveto{\pgfqpoint{1.485373in}{2.802694in}}{\pgfqpoint{1.477473in}{2.805966in}}{\pgfqpoint{1.469236in}{2.805966in}}%
\pgfpathcurveto{\pgfqpoint{1.461000in}{2.805966in}}{\pgfqpoint{1.453100in}{2.802694in}}{\pgfqpoint{1.447276in}{2.796870in}}%
\pgfpathcurveto{\pgfqpoint{1.441452in}{2.791046in}}{\pgfqpoint{1.438180in}{2.783146in}}{\pgfqpoint{1.438180in}{2.774910in}}%
\pgfpathcurveto{\pgfqpoint{1.438180in}{2.766673in}}{\pgfqpoint{1.441452in}{2.758773in}}{\pgfqpoint{1.447276in}{2.752949in}}%
\pgfpathcurveto{\pgfqpoint{1.453100in}{2.747125in}}{\pgfqpoint{1.461000in}{2.743853in}}{\pgfqpoint{1.469236in}{2.743853in}}%
\pgfpathclose%
\pgfusepath{stroke,fill}%
\end{pgfscope}%
\begin{pgfscope}%
\pgfpathrectangle{\pgfqpoint{0.100000in}{0.212622in}}{\pgfqpoint{3.696000in}{3.696000in}}%
\pgfusepath{clip}%
\pgfsetbuttcap%
\pgfsetroundjoin%
\definecolor{currentfill}{rgb}{0.121569,0.466667,0.705882}%
\pgfsetfillcolor{currentfill}%
\pgfsetfillopacity{0.701346}%
\pgfsetlinewidth{1.003750pt}%
\definecolor{currentstroke}{rgb}{0.121569,0.466667,0.705882}%
\pgfsetstrokecolor{currentstroke}%
\pgfsetstrokeopacity{0.701346}%
\pgfsetdash{}{0pt}%
\pgfpathmoveto{\pgfqpoint{2.460104in}{2.990290in}}%
\pgfpathcurveto{\pgfqpoint{2.468340in}{2.990290in}}{\pgfqpoint{2.476240in}{2.993562in}}{\pgfqpoint{2.482064in}{2.999386in}}%
\pgfpathcurveto{\pgfqpoint{2.487888in}{3.005210in}}{\pgfqpoint{2.491160in}{3.013110in}}{\pgfqpoint{2.491160in}{3.021346in}}%
\pgfpathcurveto{\pgfqpoint{2.491160in}{3.029582in}}{\pgfqpoint{2.487888in}{3.037482in}}{\pgfqpoint{2.482064in}{3.043306in}}%
\pgfpathcurveto{\pgfqpoint{2.476240in}{3.049130in}}{\pgfqpoint{2.468340in}{3.052403in}}{\pgfqpoint{2.460104in}{3.052403in}}%
\pgfpathcurveto{\pgfqpoint{2.451867in}{3.052403in}}{\pgfqpoint{2.443967in}{3.049130in}}{\pgfqpoint{2.438143in}{3.043306in}}%
\pgfpathcurveto{\pgfqpoint{2.432319in}{3.037482in}}{\pgfqpoint{2.429047in}{3.029582in}}{\pgfqpoint{2.429047in}{3.021346in}}%
\pgfpathcurveto{\pgfqpoint{2.429047in}{3.013110in}}{\pgfqpoint{2.432319in}{3.005210in}}{\pgfqpoint{2.438143in}{2.999386in}}%
\pgfpathcurveto{\pgfqpoint{2.443967in}{2.993562in}}{\pgfqpoint{2.451867in}{2.990290in}}{\pgfqpoint{2.460104in}{2.990290in}}%
\pgfpathclose%
\pgfusepath{stroke,fill}%
\end{pgfscope}%
\begin{pgfscope}%
\pgfpathrectangle{\pgfqpoint{0.100000in}{0.212622in}}{\pgfqpoint{3.696000in}{3.696000in}}%
\pgfusepath{clip}%
\pgfsetbuttcap%
\pgfsetroundjoin%
\definecolor{currentfill}{rgb}{0.121569,0.466667,0.705882}%
\pgfsetfillcolor{currentfill}%
\pgfsetfillopacity{0.701929}%
\pgfsetlinewidth{1.003750pt}%
\definecolor{currentstroke}{rgb}{0.121569,0.466667,0.705882}%
\pgfsetstrokecolor{currentstroke}%
\pgfsetstrokeopacity{0.701929}%
\pgfsetdash{}{0pt}%
\pgfpathmoveto{\pgfqpoint{1.466941in}{2.740821in}}%
\pgfpathcurveto{\pgfqpoint{1.475177in}{2.740821in}}{\pgfqpoint{1.483077in}{2.744094in}}{\pgfqpoint{1.488901in}{2.749918in}}%
\pgfpathcurveto{\pgfqpoint{1.494725in}{2.755742in}}{\pgfqpoint{1.497997in}{2.763642in}}{\pgfqpoint{1.497997in}{2.771878in}}%
\pgfpathcurveto{\pgfqpoint{1.497997in}{2.780114in}}{\pgfqpoint{1.494725in}{2.788014in}}{\pgfqpoint{1.488901in}{2.793838in}}%
\pgfpathcurveto{\pgfqpoint{1.483077in}{2.799662in}}{\pgfqpoint{1.475177in}{2.802934in}}{\pgfqpoint{1.466941in}{2.802934in}}%
\pgfpathcurveto{\pgfqpoint{1.458705in}{2.802934in}}{\pgfqpoint{1.450805in}{2.799662in}}{\pgfqpoint{1.444981in}{2.793838in}}%
\pgfpathcurveto{\pgfqpoint{1.439157in}{2.788014in}}{\pgfqpoint{1.435884in}{2.780114in}}{\pgfqpoint{1.435884in}{2.771878in}}%
\pgfpathcurveto{\pgfqpoint{1.435884in}{2.763642in}}{\pgfqpoint{1.439157in}{2.755742in}}{\pgfqpoint{1.444981in}{2.749918in}}%
\pgfpathcurveto{\pgfqpoint{1.450805in}{2.744094in}}{\pgfqpoint{1.458705in}{2.740821in}}{\pgfqpoint{1.466941in}{2.740821in}}%
\pgfpathclose%
\pgfusepath{stroke,fill}%
\end{pgfscope}%
\begin{pgfscope}%
\pgfpathrectangle{\pgfqpoint{0.100000in}{0.212622in}}{\pgfqpoint{3.696000in}{3.696000in}}%
\pgfusepath{clip}%
\pgfsetbuttcap%
\pgfsetroundjoin%
\definecolor{currentfill}{rgb}{0.121569,0.466667,0.705882}%
\pgfsetfillcolor{currentfill}%
\pgfsetfillopacity{0.701983}%
\pgfsetlinewidth{1.003750pt}%
\definecolor{currentstroke}{rgb}{0.121569,0.466667,0.705882}%
\pgfsetstrokecolor{currentstroke}%
\pgfsetstrokeopacity{0.701983}%
\pgfsetdash{}{0pt}%
\pgfpathmoveto{\pgfqpoint{2.464761in}{2.989735in}}%
\pgfpathcurveto{\pgfqpoint{2.472997in}{2.989735in}}{\pgfqpoint{2.480897in}{2.993007in}}{\pgfqpoint{2.486721in}{2.998831in}}%
\pgfpathcurveto{\pgfqpoint{2.492545in}{3.004655in}}{\pgfqpoint{2.495817in}{3.012555in}}{\pgfqpoint{2.495817in}{3.020792in}}%
\pgfpathcurveto{\pgfqpoint{2.495817in}{3.029028in}}{\pgfqpoint{2.492545in}{3.036928in}}{\pgfqpoint{2.486721in}{3.042752in}}%
\pgfpathcurveto{\pgfqpoint{2.480897in}{3.048576in}}{\pgfqpoint{2.472997in}{3.051848in}}{\pgfqpoint{2.464761in}{3.051848in}}%
\pgfpathcurveto{\pgfqpoint{2.456525in}{3.051848in}}{\pgfqpoint{2.448624in}{3.048576in}}{\pgfqpoint{2.442801in}{3.042752in}}%
\pgfpathcurveto{\pgfqpoint{2.436977in}{3.036928in}}{\pgfqpoint{2.433704in}{3.029028in}}{\pgfqpoint{2.433704in}{3.020792in}}%
\pgfpathcurveto{\pgfqpoint{2.433704in}{3.012555in}}{\pgfqpoint{2.436977in}{3.004655in}}{\pgfqpoint{2.442801in}{2.998831in}}%
\pgfpathcurveto{\pgfqpoint{2.448624in}{2.993007in}}{\pgfqpoint{2.456525in}{2.989735in}}{\pgfqpoint{2.464761in}{2.989735in}}%
\pgfpathclose%
\pgfusepath{stroke,fill}%
\end{pgfscope}%
\begin{pgfscope}%
\pgfpathrectangle{\pgfqpoint{0.100000in}{0.212622in}}{\pgfqpoint{3.696000in}{3.696000in}}%
\pgfusepath{clip}%
\pgfsetbuttcap%
\pgfsetroundjoin%
\definecolor{currentfill}{rgb}{0.121569,0.466667,0.705882}%
\pgfsetfillcolor{currentfill}%
\pgfsetfillopacity{0.702570}%
\pgfsetlinewidth{1.003750pt}%
\definecolor{currentstroke}{rgb}{0.121569,0.466667,0.705882}%
\pgfsetstrokecolor{currentstroke}%
\pgfsetstrokeopacity{0.702570}%
\pgfsetdash{}{0pt}%
\pgfpathmoveto{\pgfqpoint{2.469012in}{2.989165in}}%
\pgfpathcurveto{\pgfqpoint{2.477248in}{2.989165in}}{\pgfqpoint{2.485148in}{2.992437in}}{\pgfqpoint{2.490972in}{2.998261in}}%
\pgfpathcurveto{\pgfqpoint{2.496796in}{3.004085in}}{\pgfqpoint{2.500069in}{3.011985in}}{\pgfqpoint{2.500069in}{3.020221in}}%
\pgfpathcurveto{\pgfqpoint{2.500069in}{3.028458in}}{\pgfqpoint{2.496796in}{3.036358in}}{\pgfqpoint{2.490972in}{3.042182in}}%
\pgfpathcurveto{\pgfqpoint{2.485148in}{3.048006in}}{\pgfqpoint{2.477248in}{3.051278in}}{\pgfqpoint{2.469012in}{3.051278in}}%
\pgfpathcurveto{\pgfqpoint{2.460776in}{3.051278in}}{\pgfqpoint{2.452876in}{3.048006in}}{\pgfqpoint{2.447052in}{3.042182in}}%
\pgfpathcurveto{\pgfqpoint{2.441228in}{3.036358in}}{\pgfqpoint{2.437956in}{3.028458in}}{\pgfqpoint{2.437956in}{3.020221in}}%
\pgfpathcurveto{\pgfqpoint{2.437956in}{3.011985in}}{\pgfqpoint{2.441228in}{3.004085in}}{\pgfqpoint{2.447052in}{2.998261in}}%
\pgfpathcurveto{\pgfqpoint{2.452876in}{2.992437in}}{\pgfqpoint{2.460776in}{2.989165in}}{\pgfqpoint{2.469012in}{2.989165in}}%
\pgfpathclose%
\pgfusepath{stroke,fill}%
\end{pgfscope}%
\begin{pgfscope}%
\pgfpathrectangle{\pgfqpoint{0.100000in}{0.212622in}}{\pgfqpoint{3.696000in}{3.696000in}}%
\pgfusepath{clip}%
\pgfsetbuttcap%
\pgfsetroundjoin%
\definecolor{currentfill}{rgb}{0.121569,0.466667,0.705882}%
\pgfsetfillcolor{currentfill}%
\pgfsetfillopacity{0.702803}%
\pgfsetlinewidth{1.003750pt}%
\definecolor{currentstroke}{rgb}{0.121569,0.466667,0.705882}%
\pgfsetstrokecolor{currentstroke}%
\pgfsetstrokeopacity{0.702803}%
\pgfsetdash{}{0pt}%
\pgfpathmoveto{\pgfqpoint{1.464069in}{2.737271in}}%
\pgfpathcurveto{\pgfqpoint{1.472306in}{2.737271in}}{\pgfqpoint{1.480206in}{2.740543in}}{\pgfqpoint{1.486030in}{2.746367in}}%
\pgfpathcurveto{\pgfqpoint{1.491854in}{2.752191in}}{\pgfqpoint{1.495126in}{2.760091in}}{\pgfqpoint{1.495126in}{2.768327in}}%
\pgfpathcurveto{\pgfqpoint{1.495126in}{2.776564in}}{\pgfqpoint{1.491854in}{2.784464in}}{\pgfqpoint{1.486030in}{2.790288in}}%
\pgfpathcurveto{\pgfqpoint{1.480206in}{2.796112in}}{\pgfqpoint{1.472306in}{2.799384in}}{\pgfqpoint{1.464069in}{2.799384in}}%
\pgfpathcurveto{\pgfqpoint{1.455833in}{2.799384in}}{\pgfqpoint{1.447933in}{2.796112in}}{\pgfqpoint{1.442109in}{2.790288in}}%
\pgfpathcurveto{\pgfqpoint{1.436285in}{2.784464in}}{\pgfqpoint{1.433013in}{2.776564in}}{\pgfqpoint{1.433013in}{2.768327in}}%
\pgfpathcurveto{\pgfqpoint{1.433013in}{2.760091in}}{\pgfqpoint{1.436285in}{2.752191in}}{\pgfqpoint{1.442109in}{2.746367in}}%
\pgfpathcurveto{\pgfqpoint{1.447933in}{2.740543in}}{\pgfqpoint{1.455833in}{2.737271in}}{\pgfqpoint{1.464069in}{2.737271in}}%
\pgfpathclose%
\pgfusepath{stroke,fill}%
\end{pgfscope}%
\begin{pgfscope}%
\pgfpathrectangle{\pgfqpoint{0.100000in}{0.212622in}}{\pgfqpoint{3.696000in}{3.696000in}}%
\pgfusepath{clip}%
\pgfsetbuttcap%
\pgfsetroundjoin%
\definecolor{currentfill}{rgb}{0.121569,0.466667,0.705882}%
\pgfsetfillcolor{currentfill}%
\pgfsetfillopacity{0.703080}%
\pgfsetlinewidth{1.003750pt}%
\definecolor{currentstroke}{rgb}{0.121569,0.466667,0.705882}%
\pgfsetstrokecolor{currentstroke}%
\pgfsetstrokeopacity{0.703080}%
\pgfsetdash{}{0pt}%
\pgfpathmoveto{\pgfqpoint{2.472480in}{2.988715in}}%
\pgfpathcurveto{\pgfqpoint{2.480716in}{2.988715in}}{\pgfqpoint{2.488616in}{2.991987in}}{\pgfqpoint{2.494440in}{2.997811in}}%
\pgfpathcurveto{\pgfqpoint{2.500264in}{3.003635in}}{\pgfqpoint{2.503536in}{3.011535in}}{\pgfqpoint{2.503536in}{3.019771in}}%
\pgfpathcurveto{\pgfqpoint{2.503536in}{3.028008in}}{\pgfqpoint{2.500264in}{3.035908in}}{\pgfqpoint{2.494440in}{3.041732in}}%
\pgfpathcurveto{\pgfqpoint{2.488616in}{3.047556in}}{\pgfqpoint{2.480716in}{3.050828in}}{\pgfqpoint{2.472480in}{3.050828in}}%
\pgfpathcurveto{\pgfqpoint{2.464243in}{3.050828in}}{\pgfqpoint{2.456343in}{3.047556in}}{\pgfqpoint{2.450519in}{3.041732in}}%
\pgfpathcurveto{\pgfqpoint{2.444695in}{3.035908in}}{\pgfqpoint{2.441423in}{3.028008in}}{\pgfqpoint{2.441423in}{3.019771in}}%
\pgfpathcurveto{\pgfqpoint{2.441423in}{3.011535in}}{\pgfqpoint{2.444695in}{3.003635in}}{\pgfqpoint{2.450519in}{2.997811in}}%
\pgfpathcurveto{\pgfqpoint{2.456343in}{2.991987in}}{\pgfqpoint{2.464243in}{2.988715in}}{\pgfqpoint{2.472480in}{2.988715in}}%
\pgfpathclose%
\pgfusepath{stroke,fill}%
\end{pgfscope}%
\begin{pgfscope}%
\pgfpathrectangle{\pgfqpoint{0.100000in}{0.212622in}}{\pgfqpoint{3.696000in}{3.696000in}}%
\pgfusepath{clip}%
\pgfsetbuttcap%
\pgfsetroundjoin%
\definecolor{currentfill}{rgb}{0.121569,0.466667,0.705882}%
\pgfsetfillcolor{currentfill}%
\pgfsetfillopacity{0.703520}%
\pgfsetlinewidth{1.003750pt}%
\definecolor{currentstroke}{rgb}{0.121569,0.466667,0.705882}%
\pgfsetstrokecolor{currentstroke}%
\pgfsetstrokeopacity{0.703520}%
\pgfsetdash{}{0pt}%
\pgfpathmoveto{\pgfqpoint{2.475417in}{2.988431in}}%
\pgfpathcurveto{\pgfqpoint{2.483653in}{2.988431in}}{\pgfqpoint{2.491554in}{2.991704in}}{\pgfqpoint{2.497377in}{2.997528in}}%
\pgfpathcurveto{\pgfqpoint{2.503201in}{3.003352in}}{\pgfqpoint{2.506474in}{3.011252in}}{\pgfqpoint{2.506474in}{3.019488in}}%
\pgfpathcurveto{\pgfqpoint{2.506474in}{3.027724in}}{\pgfqpoint{2.503201in}{3.035624in}}{\pgfqpoint{2.497377in}{3.041448in}}%
\pgfpathcurveto{\pgfqpoint{2.491554in}{3.047272in}}{\pgfqpoint{2.483653in}{3.050544in}}{\pgfqpoint{2.475417in}{3.050544in}}%
\pgfpathcurveto{\pgfqpoint{2.467181in}{3.050544in}}{\pgfqpoint{2.459281in}{3.047272in}}{\pgfqpoint{2.453457in}{3.041448in}}%
\pgfpathcurveto{\pgfqpoint{2.447633in}{3.035624in}}{\pgfqpoint{2.444361in}{3.027724in}}{\pgfqpoint{2.444361in}{3.019488in}}%
\pgfpathcurveto{\pgfqpoint{2.444361in}{3.011252in}}{\pgfqpoint{2.447633in}{3.003352in}}{\pgfqpoint{2.453457in}{2.997528in}}%
\pgfpathcurveto{\pgfqpoint{2.459281in}{2.991704in}}{\pgfqpoint{2.467181in}{2.988431in}}{\pgfqpoint{2.475417in}{2.988431in}}%
\pgfpathclose%
\pgfusepath{stroke,fill}%
\end{pgfscope}%
\begin{pgfscope}%
\pgfpathrectangle{\pgfqpoint{0.100000in}{0.212622in}}{\pgfqpoint{3.696000in}{3.696000in}}%
\pgfusepath{clip}%
\pgfsetbuttcap%
\pgfsetroundjoin%
\definecolor{currentfill}{rgb}{0.121569,0.466667,0.705882}%
\pgfsetfillcolor{currentfill}%
\pgfsetfillopacity{0.703791}%
\pgfsetlinewidth{1.003750pt}%
\definecolor{currentstroke}{rgb}{0.121569,0.466667,0.705882}%
\pgfsetstrokecolor{currentstroke}%
\pgfsetstrokeopacity{0.703791}%
\pgfsetdash{}{0pt}%
\pgfpathmoveto{\pgfqpoint{2.477415in}{2.988187in}}%
\pgfpathcurveto{\pgfqpoint{2.485651in}{2.988187in}}{\pgfqpoint{2.493551in}{2.991459in}}{\pgfqpoint{2.499375in}{2.997283in}}%
\pgfpathcurveto{\pgfqpoint{2.505199in}{3.003107in}}{\pgfqpoint{2.508471in}{3.011007in}}{\pgfqpoint{2.508471in}{3.019243in}}%
\pgfpathcurveto{\pgfqpoint{2.508471in}{3.027479in}}{\pgfqpoint{2.505199in}{3.035379in}}{\pgfqpoint{2.499375in}{3.041203in}}%
\pgfpathcurveto{\pgfqpoint{2.493551in}{3.047027in}}{\pgfqpoint{2.485651in}{3.050300in}}{\pgfqpoint{2.477415in}{3.050300in}}%
\pgfpathcurveto{\pgfqpoint{2.469179in}{3.050300in}}{\pgfqpoint{2.461279in}{3.047027in}}{\pgfqpoint{2.455455in}{3.041203in}}%
\pgfpathcurveto{\pgfqpoint{2.449631in}{3.035379in}}{\pgfqpoint{2.446358in}{3.027479in}}{\pgfqpoint{2.446358in}{3.019243in}}%
\pgfpathcurveto{\pgfqpoint{2.446358in}{3.011007in}}{\pgfqpoint{2.449631in}{3.003107in}}{\pgfqpoint{2.455455in}{2.997283in}}%
\pgfpathcurveto{\pgfqpoint{2.461279in}{2.991459in}}{\pgfqpoint{2.469179in}{2.988187in}}{\pgfqpoint{2.477415in}{2.988187in}}%
\pgfpathclose%
\pgfusepath{stroke,fill}%
\end{pgfscope}%
\begin{pgfscope}%
\pgfpathrectangle{\pgfqpoint{0.100000in}{0.212622in}}{\pgfqpoint{3.696000in}{3.696000in}}%
\pgfusepath{clip}%
\pgfsetbuttcap%
\pgfsetroundjoin%
\definecolor{currentfill}{rgb}{0.121569,0.466667,0.705882}%
\pgfsetfillcolor{currentfill}%
\pgfsetfillopacity{0.703914}%
\pgfsetlinewidth{1.003750pt}%
\definecolor{currentstroke}{rgb}{0.121569,0.466667,0.705882}%
\pgfsetstrokecolor{currentstroke}%
\pgfsetstrokeopacity{0.703914}%
\pgfsetdash{}{0pt}%
\pgfpathmoveto{\pgfqpoint{1.460877in}{2.733982in}}%
\pgfpathcurveto{\pgfqpoint{1.469113in}{2.733982in}}{\pgfqpoint{1.477013in}{2.737254in}}{\pgfqpoint{1.482837in}{2.743078in}}%
\pgfpathcurveto{\pgfqpoint{1.488661in}{2.748902in}}{\pgfqpoint{1.491933in}{2.756802in}}{\pgfqpoint{1.491933in}{2.765038in}}%
\pgfpathcurveto{\pgfqpoint{1.491933in}{2.773275in}}{\pgfqpoint{1.488661in}{2.781175in}}{\pgfqpoint{1.482837in}{2.786999in}}%
\pgfpathcurveto{\pgfqpoint{1.477013in}{2.792823in}}{\pgfqpoint{1.469113in}{2.796095in}}{\pgfqpoint{1.460877in}{2.796095in}}%
\pgfpathcurveto{\pgfqpoint{1.452641in}{2.796095in}}{\pgfqpoint{1.444741in}{2.792823in}}{\pgfqpoint{1.438917in}{2.786999in}}%
\pgfpathcurveto{\pgfqpoint{1.433093in}{2.781175in}}{\pgfqpoint{1.429820in}{2.773275in}}{\pgfqpoint{1.429820in}{2.765038in}}%
\pgfpathcurveto{\pgfqpoint{1.429820in}{2.756802in}}{\pgfqpoint{1.433093in}{2.748902in}}{\pgfqpoint{1.438917in}{2.743078in}}%
\pgfpathcurveto{\pgfqpoint{1.444741in}{2.737254in}}{\pgfqpoint{1.452641in}{2.733982in}}{\pgfqpoint{1.460877in}{2.733982in}}%
\pgfpathclose%
\pgfusepath{stroke,fill}%
\end{pgfscope}%
\begin{pgfscope}%
\pgfpathrectangle{\pgfqpoint{0.100000in}{0.212622in}}{\pgfqpoint{3.696000in}{3.696000in}}%
\pgfusepath{clip}%
\pgfsetbuttcap%
\pgfsetroundjoin%
\definecolor{currentfill}{rgb}{0.121569,0.466667,0.705882}%
\pgfsetfillcolor{currentfill}%
\pgfsetfillopacity{0.704298}%
\pgfsetlinewidth{1.003750pt}%
\definecolor{currentstroke}{rgb}{0.121569,0.466667,0.705882}%
\pgfsetstrokecolor{currentstroke}%
\pgfsetstrokeopacity{0.704298}%
\pgfsetdash{}{0pt}%
\pgfpathmoveto{\pgfqpoint{2.481022in}{2.987734in}}%
\pgfpathcurveto{\pgfqpoint{2.489258in}{2.987734in}}{\pgfqpoint{2.497158in}{2.991007in}}{\pgfqpoint{2.502982in}{2.996831in}}%
\pgfpathcurveto{\pgfqpoint{2.508806in}{3.002655in}}{\pgfqpoint{2.512078in}{3.010555in}}{\pgfqpoint{2.512078in}{3.018791in}}%
\pgfpathcurveto{\pgfqpoint{2.512078in}{3.027027in}}{\pgfqpoint{2.508806in}{3.034927in}}{\pgfqpoint{2.502982in}{3.040751in}}%
\pgfpathcurveto{\pgfqpoint{2.497158in}{3.046575in}}{\pgfqpoint{2.489258in}{3.049847in}}{\pgfqpoint{2.481022in}{3.049847in}}%
\pgfpathcurveto{\pgfqpoint{2.472786in}{3.049847in}}{\pgfqpoint{2.464886in}{3.046575in}}{\pgfqpoint{2.459062in}{3.040751in}}%
\pgfpathcurveto{\pgfqpoint{2.453238in}{3.034927in}}{\pgfqpoint{2.449965in}{3.027027in}}{\pgfqpoint{2.449965in}{3.018791in}}%
\pgfpathcurveto{\pgfqpoint{2.449965in}{3.010555in}}{\pgfqpoint{2.453238in}{3.002655in}}{\pgfqpoint{2.459062in}{2.996831in}}%
\pgfpathcurveto{\pgfqpoint{2.464886in}{2.991007in}}{\pgfqpoint{2.472786in}{2.987734in}}{\pgfqpoint{2.481022in}{2.987734in}}%
\pgfpathclose%
\pgfusepath{stroke,fill}%
\end{pgfscope}%
\begin{pgfscope}%
\pgfpathrectangle{\pgfqpoint{0.100000in}{0.212622in}}{\pgfqpoint{3.696000in}{3.696000in}}%
\pgfusepath{clip}%
\pgfsetbuttcap%
\pgfsetroundjoin%
\definecolor{currentfill}{rgb}{0.121569,0.466667,0.705882}%
\pgfsetfillcolor{currentfill}%
\pgfsetfillopacity{0.704336}%
\pgfsetlinewidth{1.003750pt}%
\definecolor{currentstroke}{rgb}{0.121569,0.466667,0.705882}%
\pgfsetstrokecolor{currentstroke}%
\pgfsetstrokeopacity{0.704336}%
\pgfsetdash{}{0pt}%
\pgfpathmoveto{\pgfqpoint{3.055625in}{1.750446in}}%
\pgfpathcurveto{\pgfqpoint{3.063862in}{1.750446in}}{\pgfqpoint{3.071762in}{1.753718in}}{\pgfqpoint{3.077586in}{1.759542in}}%
\pgfpathcurveto{\pgfqpoint{3.083409in}{1.765366in}}{\pgfqpoint{3.086682in}{1.773266in}}{\pgfqpoint{3.086682in}{1.781503in}}%
\pgfpathcurveto{\pgfqpoint{3.086682in}{1.789739in}}{\pgfqpoint{3.083409in}{1.797639in}}{\pgfqpoint{3.077586in}{1.803463in}}%
\pgfpathcurveto{\pgfqpoint{3.071762in}{1.809287in}}{\pgfqpoint{3.063862in}{1.812559in}}{\pgfqpoint{3.055625in}{1.812559in}}%
\pgfpathcurveto{\pgfqpoint{3.047389in}{1.812559in}}{\pgfqpoint{3.039489in}{1.809287in}}{\pgfqpoint{3.033665in}{1.803463in}}%
\pgfpathcurveto{\pgfqpoint{3.027841in}{1.797639in}}{\pgfqpoint{3.024569in}{1.789739in}}{\pgfqpoint{3.024569in}{1.781503in}}%
\pgfpathcurveto{\pgfqpoint{3.024569in}{1.773266in}}{\pgfqpoint{3.027841in}{1.765366in}}{\pgfqpoint{3.033665in}{1.759542in}}%
\pgfpathcurveto{\pgfqpoint{3.039489in}{1.753718in}}{\pgfqpoint{3.047389in}{1.750446in}}{\pgfqpoint{3.055625in}{1.750446in}}%
\pgfpathclose%
\pgfusepath{stroke,fill}%
\end{pgfscope}%
\begin{pgfscope}%
\pgfpathrectangle{\pgfqpoint{0.100000in}{0.212622in}}{\pgfqpoint{3.696000in}{3.696000in}}%
\pgfusepath{clip}%
\pgfsetbuttcap%
\pgfsetroundjoin%
\definecolor{currentfill}{rgb}{0.121569,0.466667,0.705882}%
\pgfsetfillcolor{currentfill}%
\pgfsetfillopacity{0.704619}%
\pgfsetlinewidth{1.003750pt}%
\definecolor{currentstroke}{rgb}{0.121569,0.466667,0.705882}%
\pgfsetstrokecolor{currentstroke}%
\pgfsetstrokeopacity{0.704619}%
\pgfsetdash{}{0pt}%
\pgfpathmoveto{\pgfqpoint{2.483341in}{2.987416in}}%
\pgfpathcurveto{\pgfqpoint{2.491577in}{2.987416in}}{\pgfqpoint{2.499477in}{2.990688in}}{\pgfqpoint{2.505301in}{2.996512in}}%
\pgfpathcurveto{\pgfqpoint{2.511125in}{3.002336in}}{\pgfqpoint{2.514397in}{3.010236in}}{\pgfqpoint{2.514397in}{3.018472in}}%
\pgfpathcurveto{\pgfqpoint{2.514397in}{3.026708in}}{\pgfqpoint{2.511125in}{3.034608in}}{\pgfqpoint{2.505301in}{3.040432in}}%
\pgfpathcurveto{\pgfqpoint{2.499477in}{3.046256in}}{\pgfqpoint{2.491577in}{3.049529in}}{\pgfqpoint{2.483341in}{3.049529in}}%
\pgfpathcurveto{\pgfqpoint{2.475105in}{3.049529in}}{\pgfqpoint{2.467205in}{3.046256in}}{\pgfqpoint{2.461381in}{3.040432in}}%
\pgfpathcurveto{\pgfqpoint{2.455557in}{3.034608in}}{\pgfqpoint{2.452284in}{3.026708in}}{\pgfqpoint{2.452284in}{3.018472in}}%
\pgfpathcurveto{\pgfqpoint{2.452284in}{3.010236in}}{\pgfqpoint{2.455557in}{3.002336in}}{\pgfqpoint{2.461381in}{2.996512in}}%
\pgfpathcurveto{\pgfqpoint{2.467205in}{2.990688in}}{\pgfqpoint{2.475105in}{2.987416in}}{\pgfqpoint{2.483341in}{2.987416in}}%
\pgfpathclose%
\pgfusepath{stroke,fill}%
\end{pgfscope}%
\begin{pgfscope}%
\pgfpathrectangle{\pgfqpoint{0.100000in}{0.212622in}}{\pgfqpoint{3.696000in}{3.696000in}}%
\pgfusepath{clip}%
\pgfsetbuttcap%
\pgfsetroundjoin%
\definecolor{currentfill}{rgb}{0.121569,0.466667,0.705882}%
\pgfsetfillcolor{currentfill}%
\pgfsetfillopacity{0.705215}%
\pgfsetlinewidth{1.003750pt}%
\definecolor{currentstroke}{rgb}{0.121569,0.466667,0.705882}%
\pgfsetstrokecolor{currentstroke}%
\pgfsetstrokeopacity{0.705215}%
\pgfsetdash{}{0pt}%
\pgfpathmoveto{\pgfqpoint{2.487589in}{2.987015in}}%
\pgfpathcurveto{\pgfqpoint{2.495825in}{2.987015in}}{\pgfqpoint{2.503725in}{2.990287in}}{\pgfqpoint{2.509549in}{2.996111in}}%
\pgfpathcurveto{\pgfqpoint{2.515373in}{3.001935in}}{\pgfqpoint{2.518645in}{3.009835in}}{\pgfqpoint{2.518645in}{3.018072in}}%
\pgfpathcurveto{\pgfqpoint{2.518645in}{3.026308in}}{\pgfqpoint{2.515373in}{3.034208in}}{\pgfqpoint{2.509549in}{3.040032in}}%
\pgfpathcurveto{\pgfqpoint{2.503725in}{3.045856in}}{\pgfqpoint{2.495825in}{3.049128in}}{\pgfqpoint{2.487589in}{3.049128in}}%
\pgfpathcurveto{\pgfqpoint{2.479352in}{3.049128in}}{\pgfqpoint{2.471452in}{3.045856in}}{\pgfqpoint{2.465628in}{3.040032in}}%
\pgfpathcurveto{\pgfqpoint{2.459804in}{3.034208in}}{\pgfqpoint{2.456532in}{3.026308in}}{\pgfqpoint{2.456532in}{3.018072in}}%
\pgfpathcurveto{\pgfqpoint{2.456532in}{3.009835in}}{\pgfqpoint{2.459804in}{3.001935in}}{\pgfqpoint{2.465628in}{2.996111in}}%
\pgfpathcurveto{\pgfqpoint{2.471452in}{2.990287in}}{\pgfqpoint{2.479352in}{2.987015in}}{\pgfqpoint{2.487589in}{2.987015in}}%
\pgfpathclose%
\pgfusepath{stroke,fill}%
\end{pgfscope}%
\begin{pgfscope}%
\pgfpathrectangle{\pgfqpoint{0.100000in}{0.212622in}}{\pgfqpoint{3.696000in}{3.696000in}}%
\pgfusepath{clip}%
\pgfsetbuttcap%
\pgfsetroundjoin%
\definecolor{currentfill}{rgb}{0.121569,0.466667,0.705882}%
\pgfsetfillcolor{currentfill}%
\pgfsetfillopacity{0.705289}%
\pgfsetlinewidth{1.003750pt}%
\definecolor{currentstroke}{rgb}{0.121569,0.466667,0.705882}%
\pgfsetstrokecolor{currentstroke}%
\pgfsetstrokeopacity{0.705289}%
\pgfsetdash{}{0pt}%
\pgfpathmoveto{\pgfqpoint{1.456630in}{2.729262in}}%
\pgfpathcurveto{\pgfqpoint{1.464866in}{2.729262in}}{\pgfqpoint{1.472766in}{2.732535in}}{\pgfqpoint{1.478590in}{2.738359in}}%
\pgfpathcurveto{\pgfqpoint{1.484414in}{2.744183in}}{\pgfqpoint{1.487686in}{2.752083in}}{\pgfqpoint{1.487686in}{2.760319in}}%
\pgfpathcurveto{\pgfqpoint{1.487686in}{2.768555in}}{\pgfqpoint{1.484414in}{2.776455in}}{\pgfqpoint{1.478590in}{2.782279in}}%
\pgfpathcurveto{\pgfqpoint{1.472766in}{2.788103in}}{\pgfqpoint{1.464866in}{2.791375in}}{\pgfqpoint{1.456630in}{2.791375in}}%
\pgfpathcurveto{\pgfqpoint{1.448393in}{2.791375in}}{\pgfqpoint{1.440493in}{2.788103in}}{\pgfqpoint{1.434669in}{2.782279in}}%
\pgfpathcurveto{\pgfqpoint{1.428845in}{2.776455in}}{\pgfqpoint{1.425573in}{2.768555in}}{\pgfqpoint{1.425573in}{2.760319in}}%
\pgfpathcurveto{\pgfqpoint{1.425573in}{2.752083in}}{\pgfqpoint{1.428845in}{2.744183in}}{\pgfqpoint{1.434669in}{2.738359in}}%
\pgfpathcurveto{\pgfqpoint{1.440493in}{2.732535in}}{\pgfqpoint{1.448393in}{2.729262in}}{\pgfqpoint{1.456630in}{2.729262in}}%
\pgfpathclose%
\pgfusepath{stroke,fill}%
\end{pgfscope}%
\begin{pgfscope}%
\pgfpathrectangle{\pgfqpoint{0.100000in}{0.212622in}}{\pgfqpoint{3.696000in}{3.696000in}}%
\pgfusepath{clip}%
\pgfsetbuttcap%
\pgfsetroundjoin%
\definecolor{currentfill}{rgb}{0.121569,0.466667,0.705882}%
\pgfsetfillcolor{currentfill}%
\pgfsetfillopacity{0.705627}%
\pgfsetlinewidth{1.003750pt}%
\definecolor{currentstroke}{rgb}{0.121569,0.466667,0.705882}%
\pgfsetstrokecolor{currentstroke}%
\pgfsetstrokeopacity{0.705627}%
\pgfsetdash{}{0pt}%
\pgfpathmoveto{\pgfqpoint{2.490385in}{2.986611in}}%
\pgfpathcurveto{\pgfqpoint{2.498621in}{2.986611in}}{\pgfqpoint{2.506521in}{2.989884in}}{\pgfqpoint{2.512345in}{2.995708in}}%
\pgfpathcurveto{\pgfqpoint{2.518169in}{3.001532in}}{\pgfqpoint{2.521441in}{3.009432in}}{\pgfqpoint{2.521441in}{3.017668in}}%
\pgfpathcurveto{\pgfqpoint{2.521441in}{3.025904in}}{\pgfqpoint{2.518169in}{3.033804in}}{\pgfqpoint{2.512345in}{3.039628in}}%
\pgfpathcurveto{\pgfqpoint{2.506521in}{3.045452in}}{\pgfqpoint{2.498621in}{3.048724in}}{\pgfqpoint{2.490385in}{3.048724in}}%
\pgfpathcurveto{\pgfqpoint{2.482148in}{3.048724in}}{\pgfqpoint{2.474248in}{3.045452in}}{\pgfqpoint{2.468424in}{3.039628in}}%
\pgfpathcurveto{\pgfqpoint{2.462601in}{3.033804in}}{\pgfqpoint{2.459328in}{3.025904in}}{\pgfqpoint{2.459328in}{3.017668in}}%
\pgfpathcurveto{\pgfqpoint{2.459328in}{3.009432in}}{\pgfqpoint{2.462601in}{3.001532in}}{\pgfqpoint{2.468424in}{2.995708in}}%
\pgfpathcurveto{\pgfqpoint{2.474248in}{2.989884in}}{\pgfqpoint{2.482148in}{2.986611in}}{\pgfqpoint{2.490385in}{2.986611in}}%
\pgfpathclose%
\pgfusepath{stroke,fill}%
\end{pgfscope}%
\begin{pgfscope}%
\pgfpathrectangle{\pgfqpoint{0.100000in}{0.212622in}}{\pgfqpoint{3.696000in}{3.696000in}}%
\pgfusepath{clip}%
\pgfsetbuttcap%
\pgfsetroundjoin%
\definecolor{currentfill}{rgb}{0.121569,0.466667,0.705882}%
\pgfsetfillcolor{currentfill}%
\pgfsetfillopacity{0.706414}%
\pgfsetlinewidth{1.003750pt}%
\definecolor{currentstroke}{rgb}{0.121569,0.466667,0.705882}%
\pgfsetstrokecolor{currentstroke}%
\pgfsetstrokeopacity{0.706414}%
\pgfsetdash{}{0pt}%
\pgfpathmoveto{\pgfqpoint{2.495465in}{2.986087in}}%
\pgfpathcurveto{\pgfqpoint{2.503702in}{2.986087in}}{\pgfqpoint{2.511602in}{2.989359in}}{\pgfqpoint{2.517426in}{2.995183in}}%
\pgfpathcurveto{\pgfqpoint{2.523250in}{3.001007in}}{\pgfqpoint{2.526522in}{3.008907in}}{\pgfqpoint{2.526522in}{3.017143in}}%
\pgfpathcurveto{\pgfqpoint{2.526522in}{3.025380in}}{\pgfqpoint{2.523250in}{3.033280in}}{\pgfqpoint{2.517426in}{3.039104in}}%
\pgfpathcurveto{\pgfqpoint{2.511602in}{3.044928in}}{\pgfqpoint{2.503702in}{3.048200in}}{\pgfqpoint{2.495465in}{3.048200in}}%
\pgfpathcurveto{\pgfqpoint{2.487229in}{3.048200in}}{\pgfqpoint{2.479329in}{3.044928in}}{\pgfqpoint{2.473505in}{3.039104in}}%
\pgfpathcurveto{\pgfqpoint{2.467681in}{3.033280in}}{\pgfqpoint{2.464409in}{3.025380in}}{\pgfqpoint{2.464409in}{3.017143in}}%
\pgfpathcurveto{\pgfqpoint{2.464409in}{3.008907in}}{\pgfqpoint{2.467681in}{3.001007in}}{\pgfqpoint{2.473505in}{2.995183in}}%
\pgfpathcurveto{\pgfqpoint{2.479329in}{2.989359in}}{\pgfqpoint{2.487229in}{2.986087in}}{\pgfqpoint{2.495465in}{2.986087in}}%
\pgfpathclose%
\pgfusepath{stroke,fill}%
\end{pgfscope}%
\begin{pgfscope}%
\pgfpathrectangle{\pgfqpoint{0.100000in}{0.212622in}}{\pgfqpoint{3.696000in}{3.696000in}}%
\pgfusepath{clip}%
\pgfsetbuttcap%
\pgfsetroundjoin%
\definecolor{currentfill}{rgb}{0.121569,0.466667,0.705882}%
\pgfsetfillcolor{currentfill}%
\pgfsetfillopacity{0.706715}%
\pgfsetlinewidth{1.003750pt}%
\definecolor{currentstroke}{rgb}{0.121569,0.466667,0.705882}%
\pgfsetstrokecolor{currentstroke}%
\pgfsetstrokeopacity{0.706715}%
\pgfsetdash{}{0pt}%
\pgfpathmoveto{\pgfqpoint{1.451725in}{2.722731in}}%
\pgfpathcurveto{\pgfqpoint{1.459962in}{2.722731in}}{\pgfqpoint{1.467862in}{2.726004in}}{\pgfqpoint{1.473686in}{2.731827in}}%
\pgfpathcurveto{\pgfqpoint{1.479510in}{2.737651in}}{\pgfqpoint{1.482782in}{2.745551in}}{\pgfqpoint{1.482782in}{2.753788in}}%
\pgfpathcurveto{\pgfqpoint{1.482782in}{2.762024in}}{\pgfqpoint{1.479510in}{2.769924in}}{\pgfqpoint{1.473686in}{2.775748in}}%
\pgfpathcurveto{\pgfqpoint{1.467862in}{2.781572in}}{\pgfqpoint{1.459962in}{2.784844in}}{\pgfqpoint{1.451725in}{2.784844in}}%
\pgfpathcurveto{\pgfqpoint{1.443489in}{2.784844in}}{\pgfqpoint{1.435589in}{2.781572in}}{\pgfqpoint{1.429765in}{2.775748in}}%
\pgfpathcurveto{\pgfqpoint{1.423941in}{2.769924in}}{\pgfqpoint{1.420669in}{2.762024in}}{\pgfqpoint{1.420669in}{2.753788in}}%
\pgfpathcurveto{\pgfqpoint{1.420669in}{2.745551in}}{\pgfqpoint{1.423941in}{2.737651in}}{\pgfqpoint{1.429765in}{2.731827in}}%
\pgfpathcurveto{\pgfqpoint{1.435589in}{2.726004in}}{\pgfqpoint{1.443489in}{2.722731in}}{\pgfqpoint{1.451725in}{2.722731in}}%
\pgfpathclose%
\pgfusepath{stroke,fill}%
\end{pgfscope}%
\begin{pgfscope}%
\pgfpathrectangle{\pgfqpoint{0.100000in}{0.212622in}}{\pgfqpoint{3.696000in}{3.696000in}}%
\pgfusepath{clip}%
\pgfsetbuttcap%
\pgfsetroundjoin%
\definecolor{currentfill}{rgb}{0.121569,0.466667,0.705882}%
\pgfsetfillcolor{currentfill}%
\pgfsetfillopacity{0.707003}%
\pgfsetlinewidth{1.003750pt}%
\definecolor{currentstroke}{rgb}{0.121569,0.466667,0.705882}%
\pgfsetstrokecolor{currentstroke}%
\pgfsetstrokeopacity{0.707003}%
\pgfsetdash{}{0pt}%
\pgfpathmoveto{\pgfqpoint{2.499145in}{2.985788in}}%
\pgfpathcurveto{\pgfqpoint{2.507381in}{2.985788in}}{\pgfqpoint{2.515281in}{2.989060in}}{\pgfqpoint{2.521105in}{2.994884in}}%
\pgfpathcurveto{\pgfqpoint{2.526929in}{3.000708in}}{\pgfqpoint{2.530201in}{3.008608in}}{\pgfqpoint{2.530201in}{3.016844in}}%
\pgfpathcurveto{\pgfqpoint{2.530201in}{3.025081in}}{\pgfqpoint{2.526929in}{3.032981in}}{\pgfqpoint{2.521105in}{3.038805in}}%
\pgfpathcurveto{\pgfqpoint{2.515281in}{3.044628in}}{\pgfqpoint{2.507381in}{3.047901in}}{\pgfqpoint{2.499145in}{3.047901in}}%
\pgfpathcurveto{\pgfqpoint{2.490909in}{3.047901in}}{\pgfqpoint{2.483009in}{3.044628in}}{\pgfqpoint{2.477185in}{3.038805in}}%
\pgfpathcurveto{\pgfqpoint{2.471361in}{3.032981in}}{\pgfqpoint{2.468088in}{3.025081in}}{\pgfqpoint{2.468088in}{3.016844in}}%
\pgfpathcurveto{\pgfqpoint{2.468088in}{3.008608in}}{\pgfqpoint{2.471361in}{3.000708in}}{\pgfqpoint{2.477185in}{2.994884in}}%
\pgfpathcurveto{\pgfqpoint{2.483009in}{2.989060in}}{\pgfqpoint{2.490909in}{2.985788in}}{\pgfqpoint{2.499145in}{2.985788in}}%
\pgfpathclose%
\pgfusepath{stroke,fill}%
\end{pgfscope}%
\begin{pgfscope}%
\pgfpathrectangle{\pgfqpoint{0.100000in}{0.212622in}}{\pgfqpoint{3.696000in}{3.696000in}}%
\pgfusepath{clip}%
\pgfsetbuttcap%
\pgfsetroundjoin%
\definecolor{currentfill}{rgb}{0.121569,0.466667,0.705882}%
\pgfsetfillcolor{currentfill}%
\pgfsetfillopacity{0.707535}%
\pgfsetlinewidth{1.003750pt}%
\definecolor{currentstroke}{rgb}{0.121569,0.466667,0.705882}%
\pgfsetstrokecolor{currentstroke}%
\pgfsetstrokeopacity{0.707535}%
\pgfsetdash{}{0pt}%
\pgfpathmoveto{\pgfqpoint{1.449078in}{2.719281in}}%
\pgfpathcurveto{\pgfqpoint{1.457314in}{2.719281in}}{\pgfqpoint{1.465214in}{2.722553in}}{\pgfqpoint{1.471038in}{2.728377in}}%
\pgfpathcurveto{\pgfqpoint{1.476862in}{2.734201in}}{\pgfqpoint{1.480134in}{2.742101in}}{\pgfqpoint{1.480134in}{2.750338in}}%
\pgfpathcurveto{\pgfqpoint{1.480134in}{2.758574in}}{\pgfqpoint{1.476862in}{2.766474in}}{\pgfqpoint{1.471038in}{2.772298in}}%
\pgfpathcurveto{\pgfqpoint{1.465214in}{2.778122in}}{\pgfqpoint{1.457314in}{2.781394in}}{\pgfqpoint{1.449078in}{2.781394in}}%
\pgfpathcurveto{\pgfqpoint{1.440841in}{2.781394in}}{\pgfqpoint{1.432941in}{2.778122in}}{\pgfqpoint{1.427118in}{2.772298in}}%
\pgfpathcurveto{\pgfqpoint{1.421294in}{2.766474in}}{\pgfqpoint{1.418021in}{2.758574in}}{\pgfqpoint{1.418021in}{2.750338in}}%
\pgfpathcurveto{\pgfqpoint{1.418021in}{2.742101in}}{\pgfqpoint{1.421294in}{2.734201in}}{\pgfqpoint{1.427118in}{2.728377in}}%
\pgfpathcurveto{\pgfqpoint{1.432941in}{2.722553in}}{\pgfqpoint{1.440841in}{2.719281in}}{\pgfqpoint{1.449078in}{2.719281in}}%
\pgfpathclose%
\pgfusepath{stroke,fill}%
\end{pgfscope}%
\begin{pgfscope}%
\pgfpathrectangle{\pgfqpoint{0.100000in}{0.212622in}}{\pgfqpoint{3.696000in}{3.696000in}}%
\pgfusepath{clip}%
\pgfsetbuttcap%
\pgfsetroundjoin%
\definecolor{currentfill}{rgb}{0.121569,0.466667,0.705882}%
\pgfsetfillcolor{currentfill}%
\pgfsetfillopacity{0.707552}%
\pgfsetlinewidth{1.003750pt}%
\definecolor{currentstroke}{rgb}{0.121569,0.466667,0.705882}%
\pgfsetstrokecolor{currentstroke}%
\pgfsetstrokeopacity{0.707552}%
\pgfsetdash{}{0pt}%
\pgfpathmoveto{\pgfqpoint{2.502547in}{2.985420in}}%
\pgfpathcurveto{\pgfqpoint{2.510784in}{2.985420in}}{\pgfqpoint{2.518684in}{2.988692in}}{\pgfqpoint{2.524508in}{2.994516in}}%
\pgfpathcurveto{\pgfqpoint{2.530331in}{3.000340in}}{\pgfqpoint{2.533604in}{3.008240in}}{\pgfqpoint{2.533604in}{3.016476in}}%
\pgfpathcurveto{\pgfqpoint{2.533604in}{3.024712in}}{\pgfqpoint{2.530331in}{3.032612in}}{\pgfqpoint{2.524508in}{3.038436in}}%
\pgfpathcurveto{\pgfqpoint{2.518684in}{3.044260in}}{\pgfqpoint{2.510784in}{3.047533in}}{\pgfqpoint{2.502547in}{3.047533in}}%
\pgfpathcurveto{\pgfqpoint{2.494311in}{3.047533in}}{\pgfqpoint{2.486411in}{3.044260in}}{\pgfqpoint{2.480587in}{3.038436in}}%
\pgfpathcurveto{\pgfqpoint{2.474763in}{3.032612in}}{\pgfqpoint{2.471491in}{3.024712in}}{\pgfqpoint{2.471491in}{3.016476in}}%
\pgfpathcurveto{\pgfqpoint{2.471491in}{3.008240in}}{\pgfqpoint{2.474763in}{3.000340in}}{\pgfqpoint{2.480587in}{2.994516in}}%
\pgfpathcurveto{\pgfqpoint{2.486411in}{2.988692in}}{\pgfqpoint{2.494311in}{2.985420in}}{\pgfqpoint{2.502547in}{2.985420in}}%
\pgfpathclose%
\pgfusepath{stroke,fill}%
\end{pgfscope}%
\begin{pgfscope}%
\pgfpathrectangle{\pgfqpoint{0.100000in}{0.212622in}}{\pgfqpoint{3.696000in}{3.696000in}}%
\pgfusepath{clip}%
\pgfsetbuttcap%
\pgfsetroundjoin%
\definecolor{currentfill}{rgb}{0.121569,0.466667,0.705882}%
\pgfsetfillcolor{currentfill}%
\pgfsetfillopacity{0.707884}%
\pgfsetlinewidth{1.003750pt}%
\definecolor{currentstroke}{rgb}{0.121569,0.466667,0.705882}%
\pgfsetstrokecolor{currentstroke}%
\pgfsetstrokeopacity{0.707884}%
\pgfsetdash{}{0pt}%
\pgfpathmoveto{\pgfqpoint{2.504664in}{2.985138in}}%
\pgfpathcurveto{\pgfqpoint{2.512901in}{2.985138in}}{\pgfqpoint{2.520801in}{2.988410in}}{\pgfqpoint{2.526625in}{2.994234in}}%
\pgfpathcurveto{\pgfqpoint{2.532448in}{3.000058in}}{\pgfqpoint{2.535721in}{3.007958in}}{\pgfqpoint{2.535721in}{3.016194in}}%
\pgfpathcurveto{\pgfqpoint{2.535721in}{3.024431in}}{\pgfqpoint{2.532448in}{3.032331in}}{\pgfqpoint{2.526625in}{3.038155in}}%
\pgfpathcurveto{\pgfqpoint{2.520801in}{3.043978in}}{\pgfqpoint{2.512901in}{3.047251in}}{\pgfqpoint{2.504664in}{3.047251in}}%
\pgfpathcurveto{\pgfqpoint{2.496428in}{3.047251in}}{\pgfqpoint{2.488528in}{3.043978in}}{\pgfqpoint{2.482704in}{3.038155in}}%
\pgfpathcurveto{\pgfqpoint{2.476880in}{3.032331in}}{\pgfqpoint{2.473608in}{3.024431in}}{\pgfqpoint{2.473608in}{3.016194in}}%
\pgfpathcurveto{\pgfqpoint{2.473608in}{3.007958in}}{\pgfqpoint{2.476880in}{3.000058in}}{\pgfqpoint{2.482704in}{2.994234in}}%
\pgfpathcurveto{\pgfqpoint{2.488528in}{2.988410in}}{\pgfqpoint{2.496428in}{2.985138in}}{\pgfqpoint{2.504664in}{2.985138in}}%
\pgfpathclose%
\pgfusepath{stroke,fill}%
\end{pgfscope}%
\begin{pgfscope}%
\pgfpathrectangle{\pgfqpoint{0.100000in}{0.212622in}}{\pgfqpoint{3.696000in}{3.696000in}}%
\pgfusepath{clip}%
\pgfsetbuttcap%
\pgfsetroundjoin%
\definecolor{currentfill}{rgb}{0.121569,0.466667,0.705882}%
\pgfsetfillcolor{currentfill}%
\pgfsetfillopacity{0.708518}%
\pgfsetlinewidth{1.003750pt}%
\definecolor{currentstroke}{rgb}{0.121569,0.466667,0.705882}%
\pgfsetstrokecolor{currentstroke}%
\pgfsetstrokeopacity{0.708518}%
\pgfsetdash{}{0pt}%
\pgfpathmoveto{\pgfqpoint{2.508530in}{2.984857in}}%
\pgfpathcurveto{\pgfqpoint{2.516766in}{2.984857in}}{\pgfqpoint{2.524666in}{2.988129in}}{\pgfqpoint{2.530490in}{2.993953in}}%
\pgfpathcurveto{\pgfqpoint{2.536314in}{2.999777in}}{\pgfqpoint{2.539587in}{3.007677in}}{\pgfqpoint{2.539587in}{3.015914in}}%
\pgfpathcurveto{\pgfqpoint{2.539587in}{3.024150in}}{\pgfqpoint{2.536314in}{3.032050in}}{\pgfqpoint{2.530490in}{3.037874in}}%
\pgfpathcurveto{\pgfqpoint{2.524666in}{3.043698in}}{\pgfqpoint{2.516766in}{3.046970in}}{\pgfqpoint{2.508530in}{3.046970in}}%
\pgfpathcurveto{\pgfqpoint{2.500294in}{3.046970in}}{\pgfqpoint{2.492394in}{3.043698in}}{\pgfqpoint{2.486570in}{3.037874in}}%
\pgfpathcurveto{\pgfqpoint{2.480746in}{3.032050in}}{\pgfqpoint{2.477474in}{3.024150in}}{\pgfqpoint{2.477474in}{3.015914in}}%
\pgfpathcurveto{\pgfqpoint{2.477474in}{3.007677in}}{\pgfqpoint{2.480746in}{2.999777in}}{\pgfqpoint{2.486570in}{2.993953in}}%
\pgfpathcurveto{\pgfqpoint{2.492394in}{2.988129in}}{\pgfqpoint{2.500294in}{2.984857in}}{\pgfqpoint{2.508530in}{2.984857in}}%
\pgfpathclose%
\pgfusepath{stroke,fill}%
\end{pgfscope}%
\begin{pgfscope}%
\pgfpathrectangle{\pgfqpoint{0.100000in}{0.212622in}}{\pgfqpoint{3.696000in}{3.696000in}}%
\pgfusepath{clip}%
\pgfsetbuttcap%
\pgfsetroundjoin%
\definecolor{currentfill}{rgb}{0.121569,0.466667,0.705882}%
\pgfsetfillcolor{currentfill}%
\pgfsetfillopacity{0.708527}%
\pgfsetlinewidth{1.003750pt}%
\definecolor{currentstroke}{rgb}{0.121569,0.466667,0.705882}%
\pgfsetstrokecolor{currentstroke}%
\pgfsetstrokeopacity{0.708527}%
\pgfsetdash{}{0pt}%
\pgfpathmoveto{\pgfqpoint{1.446167in}{2.715961in}}%
\pgfpathcurveto{\pgfqpoint{1.454404in}{2.715961in}}{\pgfqpoint{1.462304in}{2.719233in}}{\pgfqpoint{1.468128in}{2.725057in}}%
\pgfpathcurveto{\pgfqpoint{1.473951in}{2.730881in}}{\pgfqpoint{1.477224in}{2.738781in}}{\pgfqpoint{1.477224in}{2.747018in}}%
\pgfpathcurveto{\pgfqpoint{1.477224in}{2.755254in}}{\pgfqpoint{1.473951in}{2.763154in}}{\pgfqpoint{1.468128in}{2.768978in}}%
\pgfpathcurveto{\pgfqpoint{1.462304in}{2.774802in}}{\pgfqpoint{1.454404in}{2.778074in}}{\pgfqpoint{1.446167in}{2.778074in}}%
\pgfpathcurveto{\pgfqpoint{1.437931in}{2.778074in}}{\pgfqpoint{1.430031in}{2.774802in}}{\pgfqpoint{1.424207in}{2.768978in}}%
\pgfpathcurveto{\pgfqpoint{1.418383in}{2.763154in}}{\pgfqpoint{1.415111in}{2.755254in}}{\pgfqpoint{1.415111in}{2.747018in}}%
\pgfpathcurveto{\pgfqpoint{1.415111in}{2.738781in}}{\pgfqpoint{1.418383in}{2.730881in}}{\pgfqpoint{1.424207in}{2.725057in}}%
\pgfpathcurveto{\pgfqpoint{1.430031in}{2.719233in}}{\pgfqpoint{1.437931in}{2.715961in}}{\pgfqpoint{1.446167in}{2.715961in}}%
\pgfpathclose%
\pgfusepath{stroke,fill}%
\end{pgfscope}%
\begin{pgfscope}%
\pgfpathrectangle{\pgfqpoint{0.100000in}{0.212622in}}{\pgfqpoint{3.696000in}{3.696000in}}%
\pgfusepath{clip}%
\pgfsetbuttcap%
\pgfsetroundjoin%
\definecolor{currentfill}{rgb}{0.121569,0.466667,0.705882}%
\pgfsetfillcolor{currentfill}%
\pgfsetfillopacity{0.708994}%
\pgfsetlinewidth{1.003750pt}%
\definecolor{currentstroke}{rgb}{0.121569,0.466667,0.705882}%
\pgfsetstrokecolor{currentstroke}%
\pgfsetstrokeopacity{0.708994}%
\pgfsetdash{}{0pt}%
\pgfpathmoveto{\pgfqpoint{2.511322in}{2.984775in}}%
\pgfpathcurveto{\pgfqpoint{2.519558in}{2.984775in}}{\pgfqpoint{2.527458in}{2.988047in}}{\pgfqpoint{2.533282in}{2.993871in}}%
\pgfpathcurveto{\pgfqpoint{2.539106in}{2.999695in}}{\pgfqpoint{2.542378in}{3.007595in}}{\pgfqpoint{2.542378in}{3.015831in}}%
\pgfpathcurveto{\pgfqpoint{2.542378in}{3.024068in}}{\pgfqpoint{2.539106in}{3.031968in}}{\pgfqpoint{2.533282in}{3.037792in}}%
\pgfpathcurveto{\pgfqpoint{2.527458in}{3.043616in}}{\pgfqpoint{2.519558in}{3.046888in}}{\pgfqpoint{2.511322in}{3.046888in}}%
\pgfpathcurveto{\pgfqpoint{2.503085in}{3.046888in}}{\pgfqpoint{2.495185in}{3.043616in}}{\pgfqpoint{2.489361in}{3.037792in}}%
\pgfpathcurveto{\pgfqpoint{2.483537in}{3.031968in}}{\pgfqpoint{2.480265in}{3.024068in}}{\pgfqpoint{2.480265in}{3.015831in}}%
\pgfpathcurveto{\pgfqpoint{2.480265in}{3.007595in}}{\pgfqpoint{2.483537in}{2.999695in}}{\pgfqpoint{2.489361in}{2.993871in}}%
\pgfpathcurveto{\pgfqpoint{2.495185in}{2.988047in}}{\pgfqpoint{2.503085in}{2.984775in}}{\pgfqpoint{2.511322in}{2.984775in}}%
\pgfpathclose%
\pgfusepath{stroke,fill}%
\end{pgfscope}%
\begin{pgfscope}%
\pgfpathrectangle{\pgfqpoint{0.100000in}{0.212622in}}{\pgfqpoint{3.696000in}{3.696000in}}%
\pgfusepath{clip}%
\pgfsetbuttcap%
\pgfsetroundjoin%
\definecolor{currentfill}{rgb}{0.121569,0.466667,0.705882}%
\pgfsetfillcolor{currentfill}%
\pgfsetfillopacity{0.709561}%
\pgfsetlinewidth{1.003750pt}%
\definecolor{currentstroke}{rgb}{0.121569,0.466667,0.705882}%
\pgfsetstrokecolor{currentstroke}%
\pgfsetstrokeopacity{0.709561}%
\pgfsetdash{}{0pt}%
\pgfpathmoveto{\pgfqpoint{1.442928in}{2.712129in}}%
\pgfpathcurveto{\pgfqpoint{1.451164in}{2.712129in}}{\pgfqpoint{1.459064in}{2.715401in}}{\pgfqpoint{1.464888in}{2.721225in}}%
\pgfpathcurveto{\pgfqpoint{1.470712in}{2.727049in}}{\pgfqpoint{1.473984in}{2.734949in}}{\pgfqpoint{1.473984in}{2.743185in}}%
\pgfpathcurveto{\pgfqpoint{1.473984in}{2.751422in}}{\pgfqpoint{1.470712in}{2.759322in}}{\pgfqpoint{1.464888in}{2.765146in}}%
\pgfpathcurveto{\pgfqpoint{1.459064in}{2.770969in}}{\pgfqpoint{1.451164in}{2.774242in}}{\pgfqpoint{1.442928in}{2.774242in}}%
\pgfpathcurveto{\pgfqpoint{1.434691in}{2.774242in}}{\pgfqpoint{1.426791in}{2.770969in}}{\pgfqpoint{1.420967in}{2.765146in}}%
\pgfpathcurveto{\pgfqpoint{1.415144in}{2.759322in}}{\pgfqpoint{1.411871in}{2.751422in}}{\pgfqpoint{1.411871in}{2.743185in}}%
\pgfpathcurveto{\pgfqpoint{1.411871in}{2.734949in}}{\pgfqpoint{1.415144in}{2.727049in}}{\pgfqpoint{1.420967in}{2.721225in}}%
\pgfpathcurveto{\pgfqpoint{1.426791in}{2.715401in}}{\pgfqpoint{1.434691in}{2.712129in}}{\pgfqpoint{1.442928in}{2.712129in}}%
\pgfpathclose%
\pgfusepath{stroke,fill}%
\end{pgfscope}%
\begin{pgfscope}%
\pgfpathrectangle{\pgfqpoint{0.100000in}{0.212622in}}{\pgfqpoint{3.696000in}{3.696000in}}%
\pgfusepath{clip}%
\pgfsetbuttcap%
\pgfsetroundjoin%
\definecolor{currentfill}{rgb}{0.121569,0.466667,0.705882}%
\pgfsetfillcolor{currentfill}%
\pgfsetfillopacity{0.709872}%
\pgfsetlinewidth{1.003750pt}%
\definecolor{currentstroke}{rgb}{0.121569,0.466667,0.705882}%
\pgfsetstrokecolor{currentstroke}%
\pgfsetstrokeopacity{0.709872}%
\pgfsetdash{}{0pt}%
\pgfpathmoveto{\pgfqpoint{2.516283in}{2.984296in}}%
\pgfpathcurveto{\pgfqpoint{2.524519in}{2.984296in}}{\pgfqpoint{2.532419in}{2.987568in}}{\pgfqpoint{2.538243in}{2.993392in}}%
\pgfpathcurveto{\pgfqpoint{2.544067in}{2.999216in}}{\pgfqpoint{2.547339in}{3.007116in}}{\pgfqpoint{2.547339in}{3.015352in}}%
\pgfpathcurveto{\pgfqpoint{2.547339in}{3.023589in}}{\pgfqpoint{2.544067in}{3.031489in}}{\pgfqpoint{2.538243in}{3.037313in}}%
\pgfpathcurveto{\pgfqpoint{2.532419in}{3.043137in}}{\pgfqpoint{2.524519in}{3.046409in}}{\pgfqpoint{2.516283in}{3.046409in}}%
\pgfpathcurveto{\pgfqpoint{2.508047in}{3.046409in}}{\pgfqpoint{2.500147in}{3.043137in}}{\pgfqpoint{2.494323in}{3.037313in}}%
\pgfpathcurveto{\pgfqpoint{2.488499in}{3.031489in}}{\pgfqpoint{2.485226in}{3.023589in}}{\pgfqpoint{2.485226in}{3.015352in}}%
\pgfpathcurveto{\pgfqpoint{2.485226in}{3.007116in}}{\pgfqpoint{2.488499in}{2.999216in}}{\pgfqpoint{2.494323in}{2.993392in}}%
\pgfpathcurveto{\pgfqpoint{2.500147in}{2.987568in}}{\pgfqpoint{2.508047in}{2.984296in}}{\pgfqpoint{2.516283in}{2.984296in}}%
\pgfpathclose%
\pgfusepath{stroke,fill}%
\end{pgfscope}%
\begin{pgfscope}%
\pgfpathrectangle{\pgfqpoint{0.100000in}{0.212622in}}{\pgfqpoint{3.696000in}{3.696000in}}%
\pgfusepath{clip}%
\pgfsetbuttcap%
\pgfsetroundjoin%
\definecolor{currentfill}{rgb}{0.121569,0.466667,0.705882}%
\pgfsetfillcolor{currentfill}%
\pgfsetfillopacity{0.710613}%
\pgfsetlinewidth{1.003750pt}%
\definecolor{currentstroke}{rgb}{0.121569,0.466667,0.705882}%
\pgfsetstrokecolor{currentstroke}%
\pgfsetstrokeopacity{0.710613}%
\pgfsetdash{}{0pt}%
\pgfpathmoveto{\pgfqpoint{2.520357in}{2.983725in}}%
\pgfpathcurveto{\pgfqpoint{2.528593in}{2.983725in}}{\pgfqpoint{2.536493in}{2.986997in}}{\pgfqpoint{2.542317in}{2.992821in}}%
\pgfpathcurveto{\pgfqpoint{2.548141in}{2.998645in}}{\pgfqpoint{2.551414in}{3.006545in}}{\pgfqpoint{2.551414in}{3.014782in}}%
\pgfpathcurveto{\pgfqpoint{2.551414in}{3.023018in}}{\pgfqpoint{2.548141in}{3.030918in}}{\pgfqpoint{2.542317in}{3.036742in}}%
\pgfpathcurveto{\pgfqpoint{2.536493in}{3.042566in}}{\pgfqpoint{2.528593in}{3.045838in}}{\pgfqpoint{2.520357in}{3.045838in}}%
\pgfpathcurveto{\pgfqpoint{2.512121in}{3.045838in}}{\pgfqpoint{2.504221in}{3.042566in}}{\pgfqpoint{2.498397in}{3.036742in}}%
\pgfpathcurveto{\pgfqpoint{2.492573in}{3.030918in}}{\pgfqpoint{2.489301in}{3.023018in}}{\pgfqpoint{2.489301in}{3.014782in}}%
\pgfpathcurveto{\pgfqpoint{2.489301in}{3.006545in}}{\pgfqpoint{2.492573in}{2.998645in}}{\pgfqpoint{2.498397in}{2.992821in}}%
\pgfpathcurveto{\pgfqpoint{2.504221in}{2.986997in}}{\pgfqpoint{2.512121in}{2.983725in}}{\pgfqpoint{2.520357in}{2.983725in}}%
\pgfpathclose%
\pgfusepath{stroke,fill}%
\end{pgfscope}%
\begin{pgfscope}%
\pgfpathrectangle{\pgfqpoint{0.100000in}{0.212622in}}{\pgfqpoint{3.696000in}{3.696000in}}%
\pgfusepath{clip}%
\pgfsetbuttcap%
\pgfsetroundjoin%
\definecolor{currentfill}{rgb}{0.121569,0.466667,0.705882}%
\pgfsetfillcolor{currentfill}%
\pgfsetfillopacity{0.710642}%
\pgfsetlinewidth{1.003750pt}%
\definecolor{currentstroke}{rgb}{0.121569,0.466667,0.705882}%
\pgfsetstrokecolor{currentstroke}%
\pgfsetstrokeopacity{0.710642}%
\pgfsetdash{}{0pt}%
\pgfpathmoveto{\pgfqpoint{3.043591in}{1.738948in}}%
\pgfpathcurveto{\pgfqpoint{3.051827in}{1.738948in}}{\pgfqpoint{3.059727in}{1.742220in}}{\pgfqpoint{3.065551in}{1.748044in}}%
\pgfpathcurveto{\pgfqpoint{3.071375in}{1.753868in}}{\pgfqpoint{3.074647in}{1.761768in}}{\pgfqpoint{3.074647in}{1.770005in}}%
\pgfpathcurveto{\pgfqpoint{3.074647in}{1.778241in}}{\pgfqpoint{3.071375in}{1.786141in}}{\pgfqpoint{3.065551in}{1.791965in}}%
\pgfpathcurveto{\pgfqpoint{3.059727in}{1.797789in}}{\pgfqpoint{3.051827in}{1.801061in}}{\pgfqpoint{3.043591in}{1.801061in}}%
\pgfpathcurveto{\pgfqpoint{3.035354in}{1.801061in}}{\pgfqpoint{3.027454in}{1.797789in}}{\pgfqpoint{3.021630in}{1.791965in}}%
\pgfpathcurveto{\pgfqpoint{3.015806in}{1.786141in}}{\pgfqpoint{3.012534in}{1.778241in}}{\pgfqpoint{3.012534in}{1.770005in}}%
\pgfpathcurveto{\pgfqpoint{3.012534in}{1.761768in}}{\pgfqpoint{3.015806in}{1.753868in}}{\pgfqpoint{3.021630in}{1.748044in}}%
\pgfpathcurveto{\pgfqpoint{3.027454in}{1.742220in}}{\pgfqpoint{3.035354in}{1.738948in}}{\pgfqpoint{3.043591in}{1.738948in}}%
\pgfpathclose%
\pgfusepath{stroke,fill}%
\end{pgfscope}%
\begin{pgfscope}%
\pgfpathrectangle{\pgfqpoint{0.100000in}{0.212622in}}{\pgfqpoint{3.696000in}{3.696000in}}%
\pgfusepath{clip}%
\pgfsetbuttcap%
\pgfsetroundjoin%
\definecolor{currentfill}{rgb}{0.121569,0.466667,0.705882}%
\pgfsetfillcolor{currentfill}%
\pgfsetfillopacity{0.710782}%
\pgfsetlinewidth{1.003750pt}%
\definecolor{currentstroke}{rgb}{0.121569,0.466667,0.705882}%
\pgfsetstrokecolor{currentstroke}%
\pgfsetstrokeopacity{0.710782}%
\pgfsetdash{}{0pt}%
\pgfpathmoveto{\pgfqpoint{1.438781in}{2.706725in}}%
\pgfpathcurveto{\pgfqpoint{1.447017in}{2.706725in}}{\pgfqpoint{1.454917in}{2.709998in}}{\pgfqpoint{1.460741in}{2.715822in}}%
\pgfpathcurveto{\pgfqpoint{1.466565in}{2.721646in}}{\pgfqpoint{1.469837in}{2.729546in}}{\pgfqpoint{1.469837in}{2.737782in}}%
\pgfpathcurveto{\pgfqpoint{1.469837in}{2.746018in}}{\pgfqpoint{1.466565in}{2.753918in}}{\pgfqpoint{1.460741in}{2.759742in}}%
\pgfpathcurveto{\pgfqpoint{1.454917in}{2.765566in}}{\pgfqpoint{1.447017in}{2.768838in}}{\pgfqpoint{1.438781in}{2.768838in}}%
\pgfpathcurveto{\pgfqpoint{1.430544in}{2.768838in}}{\pgfqpoint{1.422644in}{2.765566in}}{\pgfqpoint{1.416820in}{2.759742in}}%
\pgfpathcurveto{\pgfqpoint{1.410996in}{2.753918in}}{\pgfqpoint{1.407724in}{2.746018in}}{\pgfqpoint{1.407724in}{2.737782in}}%
\pgfpathcurveto{\pgfqpoint{1.407724in}{2.729546in}}{\pgfqpoint{1.410996in}{2.721646in}}{\pgfqpoint{1.416820in}{2.715822in}}%
\pgfpathcurveto{\pgfqpoint{1.422644in}{2.709998in}}{\pgfqpoint{1.430544in}{2.706725in}}{\pgfqpoint{1.438781in}{2.706725in}}%
\pgfpathclose%
\pgfusepath{stroke,fill}%
\end{pgfscope}%
\begin{pgfscope}%
\pgfpathrectangle{\pgfqpoint{0.100000in}{0.212622in}}{\pgfqpoint{3.696000in}{3.696000in}}%
\pgfusepath{clip}%
\pgfsetbuttcap%
\pgfsetroundjoin%
\definecolor{currentfill}{rgb}{0.121569,0.466667,0.705882}%
\pgfsetfillcolor{currentfill}%
\pgfsetfillopacity{0.711355}%
\pgfsetlinewidth{1.003750pt}%
\definecolor{currentstroke}{rgb}{0.121569,0.466667,0.705882}%
\pgfsetstrokecolor{currentstroke}%
\pgfsetstrokeopacity{0.711355}%
\pgfsetdash{}{0pt}%
\pgfpathmoveto{\pgfqpoint{2.524128in}{2.983524in}}%
\pgfpathcurveto{\pgfqpoint{2.532364in}{2.983524in}}{\pgfqpoint{2.540264in}{2.986796in}}{\pgfqpoint{2.546088in}{2.992620in}}%
\pgfpathcurveto{\pgfqpoint{2.551912in}{2.998444in}}{\pgfqpoint{2.555184in}{3.006344in}}{\pgfqpoint{2.555184in}{3.014581in}}%
\pgfpathcurveto{\pgfqpoint{2.555184in}{3.022817in}}{\pgfqpoint{2.551912in}{3.030717in}}{\pgfqpoint{2.546088in}{3.036541in}}%
\pgfpathcurveto{\pgfqpoint{2.540264in}{3.042365in}}{\pgfqpoint{2.532364in}{3.045637in}}{\pgfqpoint{2.524128in}{3.045637in}}%
\pgfpathcurveto{\pgfqpoint{2.515891in}{3.045637in}}{\pgfqpoint{2.507991in}{3.042365in}}{\pgfqpoint{2.502167in}{3.036541in}}%
\pgfpathcurveto{\pgfqpoint{2.496344in}{3.030717in}}{\pgfqpoint{2.493071in}{3.022817in}}{\pgfqpoint{2.493071in}{3.014581in}}%
\pgfpathcurveto{\pgfqpoint{2.493071in}{3.006344in}}{\pgfqpoint{2.496344in}{2.998444in}}{\pgfqpoint{2.502167in}{2.992620in}}%
\pgfpathcurveto{\pgfqpoint{2.507991in}{2.986796in}}{\pgfqpoint{2.515891in}{2.983524in}}{\pgfqpoint{2.524128in}{2.983524in}}%
\pgfpathclose%
\pgfusepath{stroke,fill}%
\end{pgfscope}%
\begin{pgfscope}%
\pgfpathrectangle{\pgfqpoint{0.100000in}{0.212622in}}{\pgfqpoint{3.696000in}{3.696000in}}%
\pgfusepath{clip}%
\pgfsetbuttcap%
\pgfsetroundjoin%
\definecolor{currentfill}{rgb}{0.121569,0.466667,0.705882}%
\pgfsetfillcolor{currentfill}%
\pgfsetfillopacity{0.711459}%
\pgfsetlinewidth{1.003750pt}%
\definecolor{currentstroke}{rgb}{0.121569,0.466667,0.705882}%
\pgfsetstrokecolor{currentstroke}%
\pgfsetstrokeopacity{0.711459}%
\pgfsetdash{}{0pt}%
\pgfpathmoveto{\pgfqpoint{1.436475in}{2.703817in}}%
\pgfpathcurveto{\pgfqpoint{1.444711in}{2.703817in}}{\pgfqpoint{1.452611in}{2.707089in}}{\pgfqpoint{1.458435in}{2.712913in}}%
\pgfpathcurveto{\pgfqpoint{1.464259in}{2.718737in}}{\pgfqpoint{1.467531in}{2.726637in}}{\pgfqpoint{1.467531in}{2.734873in}}%
\pgfpathcurveto{\pgfqpoint{1.467531in}{2.743110in}}{\pgfqpoint{1.464259in}{2.751010in}}{\pgfqpoint{1.458435in}{2.756834in}}%
\pgfpathcurveto{\pgfqpoint{1.452611in}{2.762657in}}{\pgfqpoint{1.444711in}{2.765930in}}{\pgfqpoint{1.436475in}{2.765930in}}%
\pgfpathcurveto{\pgfqpoint{1.428239in}{2.765930in}}{\pgfqpoint{1.420338in}{2.762657in}}{\pgfqpoint{1.414515in}{2.756834in}}%
\pgfpathcurveto{\pgfqpoint{1.408691in}{2.751010in}}{\pgfqpoint{1.405418in}{2.743110in}}{\pgfqpoint{1.405418in}{2.734873in}}%
\pgfpathcurveto{\pgfqpoint{1.405418in}{2.726637in}}{\pgfqpoint{1.408691in}{2.718737in}}{\pgfqpoint{1.414515in}{2.712913in}}%
\pgfpathcurveto{\pgfqpoint{1.420338in}{2.707089in}}{\pgfqpoint{1.428239in}{2.703817in}}{\pgfqpoint{1.436475in}{2.703817in}}%
\pgfpathclose%
\pgfusepath{stroke,fill}%
\end{pgfscope}%
\begin{pgfscope}%
\pgfpathrectangle{\pgfqpoint{0.100000in}{0.212622in}}{\pgfqpoint{3.696000in}{3.696000in}}%
\pgfusepath{clip}%
\pgfsetbuttcap%
\pgfsetroundjoin%
\definecolor{currentfill}{rgb}{0.121569,0.466667,0.705882}%
\pgfsetfillcolor{currentfill}%
\pgfsetfillopacity{0.712018}%
\pgfsetlinewidth{1.003750pt}%
\definecolor{currentstroke}{rgb}{0.121569,0.466667,0.705882}%
\pgfsetstrokecolor{currentstroke}%
\pgfsetstrokeopacity{0.712018}%
\pgfsetdash{}{0pt}%
\pgfpathmoveto{\pgfqpoint{2.527364in}{2.983476in}}%
\pgfpathcurveto{\pgfqpoint{2.535600in}{2.983476in}}{\pgfqpoint{2.543500in}{2.986748in}}{\pgfqpoint{2.549324in}{2.992572in}}%
\pgfpathcurveto{\pgfqpoint{2.555148in}{2.998396in}}{\pgfqpoint{2.558420in}{3.006296in}}{\pgfqpoint{2.558420in}{3.014532in}}%
\pgfpathcurveto{\pgfqpoint{2.558420in}{3.022768in}}{\pgfqpoint{2.555148in}{3.030668in}}{\pgfqpoint{2.549324in}{3.036492in}}%
\pgfpathcurveto{\pgfqpoint{2.543500in}{3.042316in}}{\pgfqpoint{2.535600in}{3.045589in}}{\pgfqpoint{2.527364in}{3.045589in}}%
\pgfpathcurveto{\pgfqpoint{2.519128in}{3.045589in}}{\pgfqpoint{2.511228in}{3.042316in}}{\pgfqpoint{2.505404in}{3.036492in}}%
\pgfpathcurveto{\pgfqpoint{2.499580in}{3.030668in}}{\pgfqpoint{2.496307in}{3.022768in}}{\pgfqpoint{2.496307in}{3.014532in}}%
\pgfpathcurveto{\pgfqpoint{2.496307in}{3.006296in}}{\pgfqpoint{2.499580in}{2.998396in}}{\pgfqpoint{2.505404in}{2.992572in}}%
\pgfpathcurveto{\pgfqpoint{2.511228in}{2.986748in}}{\pgfqpoint{2.519128in}{2.983476in}}{\pgfqpoint{2.527364in}{2.983476in}}%
\pgfpathclose%
\pgfusepath{stroke,fill}%
\end{pgfscope}%
\begin{pgfscope}%
\pgfpathrectangle{\pgfqpoint{0.100000in}{0.212622in}}{\pgfqpoint{3.696000in}{3.696000in}}%
\pgfusepath{clip}%
\pgfsetbuttcap%
\pgfsetroundjoin%
\definecolor{currentfill}{rgb}{0.121569,0.466667,0.705882}%
\pgfsetfillcolor{currentfill}%
\pgfsetfillopacity{0.712436}%
\pgfsetlinewidth{1.003750pt}%
\definecolor{currentstroke}{rgb}{0.121569,0.466667,0.705882}%
\pgfsetstrokecolor{currentstroke}%
\pgfsetstrokeopacity{0.712436}%
\pgfsetdash{}{0pt}%
\pgfpathmoveto{\pgfqpoint{2.529623in}{2.982983in}}%
\pgfpathcurveto{\pgfqpoint{2.537859in}{2.982983in}}{\pgfqpoint{2.545759in}{2.986256in}}{\pgfqpoint{2.551583in}{2.992079in}}%
\pgfpathcurveto{\pgfqpoint{2.557407in}{2.997903in}}{\pgfqpoint{2.560679in}{3.005803in}}{\pgfqpoint{2.560679in}{3.014040in}}%
\pgfpathcurveto{\pgfqpoint{2.560679in}{3.022276in}}{\pgfqpoint{2.557407in}{3.030176in}}{\pgfqpoint{2.551583in}{3.036000in}}%
\pgfpathcurveto{\pgfqpoint{2.545759in}{3.041824in}}{\pgfqpoint{2.537859in}{3.045096in}}{\pgfqpoint{2.529623in}{3.045096in}}%
\pgfpathcurveto{\pgfqpoint{2.521387in}{3.045096in}}{\pgfqpoint{2.513487in}{3.041824in}}{\pgfqpoint{2.507663in}{3.036000in}}%
\pgfpathcurveto{\pgfqpoint{2.501839in}{3.030176in}}{\pgfqpoint{2.498566in}{3.022276in}}{\pgfqpoint{2.498566in}{3.014040in}}%
\pgfpathcurveto{\pgfqpoint{2.498566in}{3.005803in}}{\pgfqpoint{2.501839in}{2.997903in}}{\pgfqpoint{2.507663in}{2.992079in}}%
\pgfpathcurveto{\pgfqpoint{2.513487in}{2.986256in}}{\pgfqpoint{2.521387in}{2.982983in}}{\pgfqpoint{2.529623in}{2.982983in}}%
\pgfpathclose%
\pgfusepath{stroke,fill}%
\end{pgfscope}%
\begin{pgfscope}%
\pgfpathrectangle{\pgfqpoint{0.100000in}{0.212622in}}{\pgfqpoint{3.696000in}{3.696000in}}%
\pgfusepath{clip}%
\pgfsetbuttcap%
\pgfsetroundjoin%
\definecolor{currentfill}{rgb}{0.121569,0.466667,0.705882}%
\pgfsetfillcolor{currentfill}%
\pgfsetfillopacity{0.712448}%
\pgfsetlinewidth{1.003750pt}%
\definecolor{currentstroke}{rgb}{0.121569,0.466667,0.705882}%
\pgfsetstrokecolor{currentstroke}%
\pgfsetstrokeopacity{0.712448}%
\pgfsetdash{}{0pt}%
\pgfpathmoveto{\pgfqpoint{1.433277in}{2.700106in}}%
\pgfpathcurveto{\pgfqpoint{1.441514in}{2.700106in}}{\pgfqpoint{1.449414in}{2.703378in}}{\pgfqpoint{1.455238in}{2.709202in}}%
\pgfpathcurveto{\pgfqpoint{1.461062in}{2.715026in}}{\pgfqpoint{1.464334in}{2.722926in}}{\pgfqpoint{1.464334in}{2.731162in}}%
\pgfpathcurveto{\pgfqpoint{1.464334in}{2.739398in}}{\pgfqpoint{1.461062in}{2.747298in}}{\pgfqpoint{1.455238in}{2.753122in}}%
\pgfpathcurveto{\pgfqpoint{1.449414in}{2.758946in}}{\pgfqpoint{1.441514in}{2.762219in}}{\pgfqpoint{1.433277in}{2.762219in}}%
\pgfpathcurveto{\pgfqpoint{1.425041in}{2.762219in}}{\pgfqpoint{1.417141in}{2.758946in}}{\pgfqpoint{1.411317in}{2.753122in}}%
\pgfpathcurveto{\pgfqpoint{1.405493in}{2.747298in}}{\pgfqpoint{1.402221in}{2.739398in}}{\pgfqpoint{1.402221in}{2.731162in}}%
\pgfpathcurveto{\pgfqpoint{1.402221in}{2.722926in}}{\pgfqpoint{1.405493in}{2.715026in}}{\pgfqpoint{1.411317in}{2.709202in}}%
\pgfpathcurveto{\pgfqpoint{1.417141in}{2.703378in}}{\pgfqpoint{1.425041in}{2.700106in}}{\pgfqpoint{1.433277in}{2.700106in}}%
\pgfpathclose%
\pgfusepath{stroke,fill}%
\end{pgfscope}%
\begin{pgfscope}%
\pgfpathrectangle{\pgfqpoint{0.100000in}{0.212622in}}{\pgfqpoint{3.696000in}{3.696000in}}%
\pgfusepath{clip}%
\pgfsetbuttcap%
\pgfsetroundjoin%
\definecolor{currentfill}{rgb}{0.121569,0.466667,0.705882}%
\pgfsetfillcolor{currentfill}%
\pgfsetfillopacity{0.713195}%
\pgfsetlinewidth{1.003750pt}%
\definecolor{currentstroke}{rgb}{0.121569,0.466667,0.705882}%
\pgfsetstrokecolor{currentstroke}%
\pgfsetstrokeopacity{0.713195}%
\pgfsetdash{}{0pt}%
\pgfpathmoveto{\pgfqpoint{2.533723in}{2.982056in}}%
\pgfpathcurveto{\pgfqpoint{2.541959in}{2.982056in}}{\pgfqpoint{2.549859in}{2.985328in}}{\pgfqpoint{2.555683in}{2.991152in}}%
\pgfpathcurveto{\pgfqpoint{2.561507in}{2.996976in}}{\pgfqpoint{2.564780in}{3.004876in}}{\pgfqpoint{2.564780in}{3.013112in}}%
\pgfpathcurveto{\pgfqpoint{2.564780in}{3.021349in}}{\pgfqpoint{2.561507in}{3.029249in}}{\pgfqpoint{2.555683in}{3.035073in}}%
\pgfpathcurveto{\pgfqpoint{2.549859in}{3.040897in}}{\pgfqpoint{2.541959in}{3.044169in}}{\pgfqpoint{2.533723in}{3.044169in}}%
\pgfpathcurveto{\pgfqpoint{2.525487in}{3.044169in}}{\pgfqpoint{2.517587in}{3.040897in}}{\pgfqpoint{2.511763in}{3.035073in}}%
\pgfpathcurveto{\pgfqpoint{2.505939in}{3.029249in}}{\pgfqpoint{2.502667in}{3.021349in}}{\pgfqpoint{2.502667in}{3.013112in}}%
\pgfpathcurveto{\pgfqpoint{2.502667in}{3.004876in}}{\pgfqpoint{2.505939in}{2.996976in}}{\pgfqpoint{2.511763in}{2.991152in}}%
\pgfpathcurveto{\pgfqpoint{2.517587in}{2.985328in}}{\pgfqpoint{2.525487in}{2.982056in}}{\pgfqpoint{2.533723in}{2.982056in}}%
\pgfpathclose%
\pgfusepath{stroke,fill}%
\end{pgfscope}%
\begin{pgfscope}%
\pgfpathrectangle{\pgfqpoint{0.100000in}{0.212622in}}{\pgfqpoint{3.696000in}{3.696000in}}%
\pgfusepath{clip}%
\pgfsetbuttcap%
\pgfsetroundjoin%
\definecolor{currentfill}{rgb}{0.121569,0.466667,0.705882}%
\pgfsetfillcolor{currentfill}%
\pgfsetfillopacity{0.713625}%
\pgfsetlinewidth{1.003750pt}%
\definecolor{currentstroke}{rgb}{0.121569,0.466667,0.705882}%
\pgfsetstrokecolor{currentstroke}%
\pgfsetstrokeopacity{0.713625}%
\pgfsetdash{}{0pt}%
\pgfpathmoveto{\pgfqpoint{1.429815in}{2.696612in}}%
\pgfpathcurveto{\pgfqpoint{1.438051in}{2.696612in}}{\pgfqpoint{1.445951in}{2.699884in}}{\pgfqpoint{1.451775in}{2.705708in}}%
\pgfpathcurveto{\pgfqpoint{1.457599in}{2.711532in}}{\pgfqpoint{1.460871in}{2.719432in}}{\pgfqpoint{1.460871in}{2.727668in}}%
\pgfpathcurveto{\pgfqpoint{1.460871in}{2.735905in}}{\pgfqpoint{1.457599in}{2.743805in}}{\pgfqpoint{1.451775in}{2.749629in}}%
\pgfpathcurveto{\pgfqpoint{1.445951in}{2.755453in}}{\pgfqpoint{1.438051in}{2.758725in}}{\pgfqpoint{1.429815in}{2.758725in}}%
\pgfpathcurveto{\pgfqpoint{1.421578in}{2.758725in}}{\pgfqpoint{1.413678in}{2.755453in}}{\pgfqpoint{1.407854in}{2.749629in}}%
\pgfpathcurveto{\pgfqpoint{1.402030in}{2.743805in}}{\pgfqpoint{1.398758in}{2.735905in}}{\pgfqpoint{1.398758in}{2.727668in}}%
\pgfpathcurveto{\pgfqpoint{1.398758in}{2.719432in}}{\pgfqpoint{1.402030in}{2.711532in}}{\pgfqpoint{1.407854in}{2.705708in}}%
\pgfpathcurveto{\pgfqpoint{1.413678in}{2.699884in}}{\pgfqpoint{1.421578in}{2.696612in}}{\pgfqpoint{1.429815in}{2.696612in}}%
\pgfpathclose%
\pgfusepath{stroke,fill}%
\end{pgfscope}%
\begin{pgfscope}%
\pgfpathrectangle{\pgfqpoint{0.100000in}{0.212622in}}{\pgfqpoint{3.696000in}{3.696000in}}%
\pgfusepath{clip}%
\pgfsetbuttcap%
\pgfsetroundjoin%
\definecolor{currentfill}{rgb}{0.121569,0.466667,0.705882}%
\pgfsetfillcolor{currentfill}%
\pgfsetfillopacity{0.713747}%
\pgfsetlinewidth{1.003750pt}%
\definecolor{currentstroke}{rgb}{0.121569,0.466667,0.705882}%
\pgfsetstrokecolor{currentstroke}%
\pgfsetstrokeopacity{0.713747}%
\pgfsetdash{}{0pt}%
\pgfpathmoveto{\pgfqpoint{2.536753in}{2.981285in}}%
\pgfpathcurveto{\pgfqpoint{2.544989in}{2.981285in}}{\pgfqpoint{2.552889in}{2.984558in}}{\pgfqpoint{2.558713in}{2.990382in}}%
\pgfpathcurveto{\pgfqpoint{2.564537in}{2.996205in}}{\pgfqpoint{2.567809in}{3.004106in}}{\pgfqpoint{2.567809in}{3.012342in}}%
\pgfpathcurveto{\pgfqpoint{2.567809in}{3.020578in}}{\pgfqpoint{2.564537in}{3.028478in}}{\pgfqpoint{2.558713in}{3.034302in}}%
\pgfpathcurveto{\pgfqpoint{2.552889in}{3.040126in}}{\pgfqpoint{2.544989in}{3.043398in}}{\pgfqpoint{2.536753in}{3.043398in}}%
\pgfpathcurveto{\pgfqpoint{2.528516in}{3.043398in}}{\pgfqpoint{2.520616in}{3.040126in}}{\pgfqpoint{2.514792in}{3.034302in}}%
\pgfpathcurveto{\pgfqpoint{2.508968in}{3.028478in}}{\pgfqpoint{2.505696in}{3.020578in}}{\pgfqpoint{2.505696in}{3.012342in}}%
\pgfpathcurveto{\pgfqpoint{2.505696in}{3.004106in}}{\pgfqpoint{2.508968in}{2.996205in}}{\pgfqpoint{2.514792in}{2.990382in}}%
\pgfpathcurveto{\pgfqpoint{2.520616in}{2.984558in}}{\pgfqpoint{2.528516in}{2.981285in}}{\pgfqpoint{2.536753in}{2.981285in}}%
\pgfpathclose%
\pgfusepath{stroke,fill}%
\end{pgfscope}%
\begin{pgfscope}%
\pgfpathrectangle{\pgfqpoint{0.100000in}{0.212622in}}{\pgfqpoint{3.696000in}{3.696000in}}%
\pgfusepath{clip}%
\pgfsetbuttcap%
\pgfsetroundjoin%
\definecolor{currentfill}{rgb}{0.121569,0.466667,0.705882}%
\pgfsetfillcolor{currentfill}%
\pgfsetfillopacity{0.714245}%
\pgfsetlinewidth{1.003750pt}%
\definecolor{currentstroke}{rgb}{0.121569,0.466667,0.705882}%
\pgfsetstrokecolor{currentstroke}%
\pgfsetstrokeopacity{0.714245}%
\pgfsetdash{}{0pt}%
\pgfpathmoveto{\pgfqpoint{1.427863in}{2.694594in}}%
\pgfpathcurveto{\pgfqpoint{1.436099in}{2.694594in}}{\pgfqpoint{1.443999in}{2.697867in}}{\pgfqpoint{1.449823in}{2.703691in}}%
\pgfpathcurveto{\pgfqpoint{1.455647in}{2.709514in}}{\pgfqpoint{1.458919in}{2.717415in}}{\pgfqpoint{1.458919in}{2.725651in}}%
\pgfpathcurveto{\pgfqpoint{1.458919in}{2.733887in}}{\pgfqpoint{1.455647in}{2.741787in}}{\pgfqpoint{1.449823in}{2.747611in}}%
\pgfpathcurveto{\pgfqpoint{1.443999in}{2.753435in}}{\pgfqpoint{1.436099in}{2.756707in}}{\pgfqpoint{1.427863in}{2.756707in}}%
\pgfpathcurveto{\pgfqpoint{1.419627in}{2.756707in}}{\pgfqpoint{1.411726in}{2.753435in}}{\pgfqpoint{1.405903in}{2.747611in}}%
\pgfpathcurveto{\pgfqpoint{1.400079in}{2.741787in}}{\pgfqpoint{1.396806in}{2.733887in}}{\pgfqpoint{1.396806in}{2.725651in}}%
\pgfpathcurveto{\pgfqpoint{1.396806in}{2.717415in}}{\pgfqpoint{1.400079in}{2.709514in}}{\pgfqpoint{1.405903in}{2.703691in}}%
\pgfpathcurveto{\pgfqpoint{1.411726in}{2.697867in}}{\pgfqpoint{1.419627in}{2.694594in}}{\pgfqpoint{1.427863in}{2.694594in}}%
\pgfpathclose%
\pgfusepath{stroke,fill}%
\end{pgfscope}%
\begin{pgfscope}%
\pgfpathrectangle{\pgfqpoint{0.100000in}{0.212622in}}{\pgfqpoint{3.696000in}{3.696000in}}%
\pgfusepath{clip}%
\pgfsetbuttcap%
\pgfsetroundjoin%
\definecolor{currentfill}{rgb}{0.121569,0.466667,0.705882}%
\pgfsetfillcolor{currentfill}%
\pgfsetfillopacity{0.714728}%
\pgfsetlinewidth{1.003750pt}%
\definecolor{currentstroke}{rgb}{0.121569,0.466667,0.705882}%
\pgfsetstrokecolor{currentstroke}%
\pgfsetstrokeopacity{0.714728}%
\pgfsetdash{}{0pt}%
\pgfpathmoveto{\pgfqpoint{2.542272in}{2.979790in}}%
\pgfpathcurveto{\pgfqpoint{2.550509in}{2.979790in}}{\pgfqpoint{2.558409in}{2.983062in}}{\pgfqpoint{2.564233in}{2.988886in}}%
\pgfpathcurveto{\pgfqpoint{2.570056in}{2.994710in}}{\pgfqpoint{2.573329in}{3.002610in}}{\pgfqpoint{2.573329in}{3.010846in}}%
\pgfpathcurveto{\pgfqpoint{2.573329in}{3.019083in}}{\pgfqpoint{2.570056in}{3.026983in}}{\pgfqpoint{2.564233in}{3.032807in}}%
\pgfpathcurveto{\pgfqpoint{2.558409in}{3.038631in}}{\pgfqpoint{2.550509in}{3.041903in}}{\pgfqpoint{2.542272in}{3.041903in}}%
\pgfpathcurveto{\pgfqpoint{2.534036in}{3.041903in}}{\pgfqpoint{2.526136in}{3.038631in}}{\pgfqpoint{2.520312in}{3.032807in}}%
\pgfpathcurveto{\pgfqpoint{2.514488in}{3.026983in}}{\pgfqpoint{2.511216in}{3.019083in}}{\pgfqpoint{2.511216in}{3.010846in}}%
\pgfpathcurveto{\pgfqpoint{2.511216in}{3.002610in}}{\pgfqpoint{2.514488in}{2.994710in}}{\pgfqpoint{2.520312in}{2.988886in}}%
\pgfpathcurveto{\pgfqpoint{2.526136in}{2.983062in}}{\pgfqpoint{2.534036in}{2.979790in}}{\pgfqpoint{2.542272in}{2.979790in}}%
\pgfpathclose%
\pgfusepath{stroke,fill}%
\end{pgfscope}%
\begin{pgfscope}%
\pgfpathrectangle{\pgfqpoint{0.100000in}{0.212622in}}{\pgfqpoint{3.696000in}{3.696000in}}%
\pgfusepath{clip}%
\pgfsetbuttcap%
\pgfsetroundjoin%
\definecolor{currentfill}{rgb}{0.121569,0.466667,0.705882}%
\pgfsetfillcolor{currentfill}%
\pgfsetfillopacity{0.715041}%
\pgfsetlinewidth{1.003750pt}%
\definecolor{currentstroke}{rgb}{0.121569,0.466667,0.705882}%
\pgfsetstrokecolor{currentstroke}%
\pgfsetstrokeopacity{0.715041}%
\pgfsetdash{}{0pt}%
\pgfpathmoveto{\pgfqpoint{1.425126in}{2.691407in}}%
\pgfpathcurveto{\pgfqpoint{1.433362in}{2.691407in}}{\pgfqpoint{1.441262in}{2.694679in}}{\pgfqpoint{1.447086in}{2.700503in}}%
\pgfpathcurveto{\pgfqpoint{1.452910in}{2.706327in}}{\pgfqpoint{1.456182in}{2.714227in}}{\pgfqpoint{1.456182in}{2.722464in}}%
\pgfpathcurveto{\pgfqpoint{1.456182in}{2.730700in}}{\pgfqpoint{1.452910in}{2.738600in}}{\pgfqpoint{1.447086in}{2.744424in}}%
\pgfpathcurveto{\pgfqpoint{1.441262in}{2.750248in}}{\pgfqpoint{1.433362in}{2.753520in}}{\pgfqpoint{1.425126in}{2.753520in}}%
\pgfpathcurveto{\pgfqpoint{1.416890in}{2.753520in}}{\pgfqpoint{1.408990in}{2.750248in}}{\pgfqpoint{1.403166in}{2.744424in}}%
\pgfpathcurveto{\pgfqpoint{1.397342in}{2.738600in}}{\pgfqpoint{1.394069in}{2.730700in}}{\pgfqpoint{1.394069in}{2.722464in}}%
\pgfpathcurveto{\pgfqpoint{1.394069in}{2.714227in}}{\pgfqpoint{1.397342in}{2.706327in}}{\pgfqpoint{1.403166in}{2.700503in}}%
\pgfpathcurveto{\pgfqpoint{1.408990in}{2.694679in}}{\pgfqpoint{1.416890in}{2.691407in}}{\pgfqpoint{1.425126in}{2.691407in}}%
\pgfpathclose%
\pgfusepath{stroke,fill}%
\end{pgfscope}%
\begin{pgfscope}%
\pgfpathrectangle{\pgfqpoint{0.100000in}{0.212622in}}{\pgfqpoint{3.696000in}{3.696000in}}%
\pgfusepath{clip}%
\pgfsetbuttcap%
\pgfsetroundjoin%
\definecolor{currentfill}{rgb}{0.121569,0.466667,0.705882}%
\pgfsetfillcolor{currentfill}%
\pgfsetfillopacity{0.715481}%
\pgfsetlinewidth{1.003750pt}%
\definecolor{currentstroke}{rgb}{0.121569,0.466667,0.705882}%
\pgfsetstrokecolor{currentstroke}%
\pgfsetstrokeopacity{0.715481}%
\pgfsetdash{}{0pt}%
\pgfpathmoveto{\pgfqpoint{1.423642in}{2.689634in}}%
\pgfpathcurveto{\pgfqpoint{1.431878in}{2.689634in}}{\pgfqpoint{1.439778in}{2.692906in}}{\pgfqpoint{1.445602in}{2.698730in}}%
\pgfpathcurveto{\pgfqpoint{1.451426in}{2.704554in}}{\pgfqpoint{1.454699in}{2.712454in}}{\pgfqpoint{1.454699in}{2.720690in}}%
\pgfpathcurveto{\pgfqpoint{1.454699in}{2.728927in}}{\pgfqpoint{1.451426in}{2.736827in}}{\pgfqpoint{1.445602in}{2.742651in}}%
\pgfpathcurveto{\pgfqpoint{1.439778in}{2.748475in}}{\pgfqpoint{1.431878in}{2.751747in}}{\pgfqpoint{1.423642in}{2.751747in}}%
\pgfpathcurveto{\pgfqpoint{1.415406in}{2.751747in}}{\pgfqpoint{1.407506in}{2.748475in}}{\pgfqpoint{1.401682in}{2.742651in}}%
\pgfpathcurveto{\pgfqpoint{1.395858in}{2.736827in}}{\pgfqpoint{1.392586in}{2.728927in}}{\pgfqpoint{1.392586in}{2.720690in}}%
\pgfpathcurveto{\pgfqpoint{1.392586in}{2.712454in}}{\pgfqpoint{1.395858in}{2.704554in}}{\pgfqpoint{1.401682in}{2.698730in}}%
\pgfpathcurveto{\pgfqpoint{1.407506in}{2.692906in}}{\pgfqpoint{1.415406in}{2.689634in}}{\pgfqpoint{1.423642in}{2.689634in}}%
\pgfpathclose%
\pgfusepath{stroke,fill}%
\end{pgfscope}%
\begin{pgfscope}%
\pgfpathrectangle{\pgfqpoint{0.100000in}{0.212622in}}{\pgfqpoint{3.696000in}{3.696000in}}%
\pgfusepath{clip}%
\pgfsetbuttcap%
\pgfsetroundjoin%
\definecolor{currentfill}{rgb}{0.121569,0.466667,0.705882}%
\pgfsetfillcolor{currentfill}%
\pgfsetfillopacity{0.715497}%
\pgfsetlinewidth{1.003750pt}%
\definecolor{currentstroke}{rgb}{0.121569,0.466667,0.705882}%
\pgfsetstrokecolor{currentstroke}%
\pgfsetstrokeopacity{0.715497}%
\pgfsetdash{}{0pt}%
\pgfpathmoveto{\pgfqpoint{2.547006in}{2.978765in}}%
\pgfpathcurveto{\pgfqpoint{2.555242in}{2.978765in}}{\pgfqpoint{2.563143in}{2.982037in}}{\pgfqpoint{2.568966in}{2.987861in}}%
\pgfpathcurveto{\pgfqpoint{2.574790in}{2.993685in}}{\pgfqpoint{2.578063in}{3.001585in}}{\pgfqpoint{2.578063in}{3.009821in}}%
\pgfpathcurveto{\pgfqpoint{2.578063in}{3.018057in}}{\pgfqpoint{2.574790in}{3.025958in}}{\pgfqpoint{2.568966in}{3.031781in}}%
\pgfpathcurveto{\pgfqpoint{2.563143in}{3.037605in}}{\pgfqpoint{2.555242in}{3.040878in}}{\pgfqpoint{2.547006in}{3.040878in}}%
\pgfpathcurveto{\pgfqpoint{2.538770in}{3.040878in}}{\pgfqpoint{2.530870in}{3.037605in}}{\pgfqpoint{2.525046in}{3.031781in}}%
\pgfpathcurveto{\pgfqpoint{2.519222in}{3.025958in}}{\pgfqpoint{2.515950in}{3.018057in}}{\pgfqpoint{2.515950in}{3.009821in}}%
\pgfpathcurveto{\pgfqpoint{2.515950in}{3.001585in}}{\pgfqpoint{2.519222in}{2.993685in}}{\pgfqpoint{2.525046in}{2.987861in}}%
\pgfpathcurveto{\pgfqpoint{2.530870in}{2.982037in}}{\pgfqpoint{2.538770in}{2.978765in}}{\pgfqpoint{2.547006in}{2.978765in}}%
\pgfpathclose%
\pgfusepath{stroke,fill}%
\end{pgfscope}%
\begin{pgfscope}%
\pgfpathrectangle{\pgfqpoint{0.100000in}{0.212622in}}{\pgfqpoint{3.696000in}{3.696000in}}%
\pgfusepath{clip}%
\pgfsetbuttcap%
\pgfsetroundjoin%
\definecolor{currentfill}{rgb}{0.121569,0.466667,0.705882}%
\pgfsetfillcolor{currentfill}%
\pgfsetfillopacity{0.716151}%
\pgfsetlinewidth{1.003750pt}%
\definecolor{currentstroke}{rgb}{0.121569,0.466667,0.705882}%
\pgfsetstrokecolor{currentstroke}%
\pgfsetstrokeopacity{0.716151}%
\pgfsetdash{}{0pt}%
\pgfpathmoveto{\pgfqpoint{1.421508in}{2.687244in}}%
\pgfpathcurveto{\pgfqpoint{1.429745in}{2.687244in}}{\pgfqpoint{1.437645in}{2.690517in}}{\pgfqpoint{1.443469in}{2.696340in}}%
\pgfpathcurveto{\pgfqpoint{1.449293in}{2.702164in}}{\pgfqpoint{1.452565in}{2.710064in}}{\pgfqpoint{1.452565in}{2.718301in}}%
\pgfpathcurveto{\pgfqpoint{1.452565in}{2.726537in}}{\pgfqpoint{1.449293in}{2.734437in}}{\pgfqpoint{1.443469in}{2.740261in}}%
\pgfpathcurveto{\pgfqpoint{1.437645in}{2.746085in}}{\pgfqpoint{1.429745in}{2.749357in}}{\pgfqpoint{1.421508in}{2.749357in}}%
\pgfpathcurveto{\pgfqpoint{1.413272in}{2.749357in}}{\pgfqpoint{1.405372in}{2.746085in}}{\pgfqpoint{1.399548in}{2.740261in}}%
\pgfpathcurveto{\pgfqpoint{1.393724in}{2.734437in}}{\pgfqpoint{1.390452in}{2.726537in}}{\pgfqpoint{1.390452in}{2.718301in}}%
\pgfpathcurveto{\pgfqpoint{1.390452in}{2.710064in}}{\pgfqpoint{1.393724in}{2.702164in}}{\pgfqpoint{1.399548in}{2.696340in}}%
\pgfpathcurveto{\pgfqpoint{1.405372in}{2.690517in}}{\pgfqpoint{1.413272in}{2.687244in}}{\pgfqpoint{1.421508in}{2.687244in}}%
\pgfpathclose%
\pgfusepath{stroke,fill}%
\end{pgfscope}%
\begin{pgfscope}%
\pgfpathrectangle{\pgfqpoint{0.100000in}{0.212622in}}{\pgfqpoint{3.696000in}{3.696000in}}%
\pgfusepath{clip}%
\pgfsetbuttcap%
\pgfsetroundjoin%
\definecolor{currentfill}{rgb}{0.121569,0.466667,0.705882}%
\pgfsetfillcolor{currentfill}%
\pgfsetfillopacity{0.716210}%
\pgfsetlinewidth{1.003750pt}%
\definecolor{currentstroke}{rgb}{0.121569,0.466667,0.705882}%
\pgfsetstrokecolor{currentstroke}%
\pgfsetstrokeopacity{0.716210}%
\pgfsetdash{}{0pt}%
\pgfpathmoveto{\pgfqpoint{2.551534in}{2.978029in}}%
\pgfpathcurveto{\pgfqpoint{2.559771in}{2.978029in}}{\pgfqpoint{2.567671in}{2.981301in}}{\pgfqpoint{2.573495in}{2.987125in}}%
\pgfpathcurveto{\pgfqpoint{2.579318in}{2.992949in}}{\pgfqpoint{2.582591in}{3.000849in}}{\pgfqpoint{2.582591in}{3.009086in}}%
\pgfpathcurveto{\pgfqpoint{2.582591in}{3.017322in}}{\pgfqpoint{2.579318in}{3.025222in}}{\pgfqpoint{2.573495in}{3.031046in}}%
\pgfpathcurveto{\pgfqpoint{2.567671in}{3.036870in}}{\pgfqpoint{2.559771in}{3.040142in}}{\pgfqpoint{2.551534in}{3.040142in}}%
\pgfpathcurveto{\pgfqpoint{2.543298in}{3.040142in}}{\pgfqpoint{2.535398in}{3.036870in}}{\pgfqpoint{2.529574in}{3.031046in}}%
\pgfpathcurveto{\pgfqpoint{2.523750in}{3.025222in}}{\pgfqpoint{2.520478in}{3.017322in}}{\pgfqpoint{2.520478in}{3.009086in}}%
\pgfpathcurveto{\pgfqpoint{2.520478in}{3.000849in}}{\pgfqpoint{2.523750in}{2.992949in}}{\pgfqpoint{2.529574in}{2.987125in}}%
\pgfpathcurveto{\pgfqpoint{2.535398in}{2.981301in}}{\pgfqpoint{2.543298in}{2.978029in}}{\pgfqpoint{2.551534in}{2.978029in}}%
\pgfpathclose%
\pgfusepath{stroke,fill}%
\end{pgfscope}%
\begin{pgfscope}%
\pgfpathrectangle{\pgfqpoint{0.100000in}{0.212622in}}{\pgfqpoint{3.696000in}{3.696000in}}%
\pgfusepath{clip}%
\pgfsetbuttcap%
\pgfsetroundjoin%
\definecolor{currentfill}{rgb}{0.121569,0.466667,0.705882}%
\pgfsetfillcolor{currentfill}%
\pgfsetfillopacity{0.716815}%
\pgfsetlinewidth{1.003750pt}%
\definecolor{currentstroke}{rgb}{0.121569,0.466667,0.705882}%
\pgfsetstrokecolor{currentstroke}%
\pgfsetstrokeopacity{0.716815}%
\pgfsetdash{}{0pt}%
\pgfpathmoveto{\pgfqpoint{2.555587in}{2.977463in}}%
\pgfpathcurveto{\pgfqpoint{2.563824in}{2.977463in}}{\pgfqpoint{2.571724in}{2.980736in}}{\pgfqpoint{2.577548in}{2.986559in}}%
\pgfpathcurveto{\pgfqpoint{2.583372in}{2.992383in}}{\pgfqpoint{2.586644in}{3.000283in}}{\pgfqpoint{2.586644in}{3.008520in}}%
\pgfpathcurveto{\pgfqpoint{2.586644in}{3.016756in}}{\pgfqpoint{2.583372in}{3.024656in}}{\pgfqpoint{2.577548in}{3.030480in}}%
\pgfpathcurveto{\pgfqpoint{2.571724in}{3.036304in}}{\pgfqpoint{2.563824in}{3.039576in}}{\pgfqpoint{2.555587in}{3.039576in}}%
\pgfpathcurveto{\pgfqpoint{2.547351in}{3.039576in}}{\pgfqpoint{2.539451in}{3.036304in}}{\pgfqpoint{2.533627in}{3.030480in}}%
\pgfpathcurveto{\pgfqpoint{2.527803in}{3.024656in}}{\pgfqpoint{2.524531in}{3.016756in}}{\pgfqpoint{2.524531in}{3.008520in}}%
\pgfpathcurveto{\pgfqpoint{2.524531in}{3.000283in}}{\pgfqpoint{2.527803in}{2.992383in}}{\pgfqpoint{2.533627in}{2.986559in}}%
\pgfpathcurveto{\pgfqpoint{2.539451in}{2.980736in}}{\pgfqpoint{2.547351in}{2.977463in}}{\pgfqpoint{2.555587in}{2.977463in}}%
\pgfpathclose%
\pgfusepath{stroke,fill}%
\end{pgfscope}%
\begin{pgfscope}%
\pgfpathrectangle{\pgfqpoint{0.100000in}{0.212622in}}{\pgfqpoint{3.696000in}{3.696000in}}%
\pgfusepath{clip}%
\pgfsetbuttcap%
\pgfsetroundjoin%
\definecolor{currentfill}{rgb}{0.121569,0.466667,0.705882}%
\pgfsetfillcolor{currentfill}%
\pgfsetfillopacity{0.716892}%
\pgfsetlinewidth{1.003750pt}%
\definecolor{currentstroke}{rgb}{0.121569,0.466667,0.705882}%
\pgfsetstrokecolor{currentstroke}%
\pgfsetstrokeopacity{0.716892}%
\pgfsetdash{}{0pt}%
\pgfpathmoveto{\pgfqpoint{3.032064in}{1.728267in}}%
\pgfpathcurveto{\pgfqpoint{3.040300in}{1.728267in}}{\pgfqpoint{3.048200in}{1.731540in}}{\pgfqpoint{3.054024in}{1.737363in}}%
\pgfpathcurveto{\pgfqpoint{3.059848in}{1.743187in}}{\pgfqpoint{3.063120in}{1.751087in}}{\pgfqpoint{3.063120in}{1.759324in}}%
\pgfpathcurveto{\pgfqpoint{3.063120in}{1.767560in}}{\pgfqpoint{3.059848in}{1.775460in}}{\pgfqpoint{3.054024in}{1.781284in}}%
\pgfpathcurveto{\pgfqpoint{3.048200in}{1.787108in}}{\pgfqpoint{3.040300in}{1.790380in}}{\pgfqpoint{3.032064in}{1.790380in}}%
\pgfpathcurveto{\pgfqpoint{3.023828in}{1.790380in}}{\pgfqpoint{3.015928in}{1.787108in}}{\pgfqpoint{3.010104in}{1.781284in}}%
\pgfpathcurveto{\pgfqpoint{3.004280in}{1.775460in}}{\pgfqpoint{3.001007in}{1.767560in}}{\pgfqpoint{3.001007in}{1.759324in}}%
\pgfpathcurveto{\pgfqpoint{3.001007in}{1.751087in}}{\pgfqpoint{3.004280in}{1.743187in}}{\pgfqpoint{3.010104in}{1.737363in}}%
\pgfpathcurveto{\pgfqpoint{3.015928in}{1.731540in}}{\pgfqpoint{3.023828in}{1.728267in}}{\pgfqpoint{3.032064in}{1.728267in}}%
\pgfpathclose%
\pgfusepath{stroke,fill}%
\end{pgfscope}%
\begin{pgfscope}%
\pgfpathrectangle{\pgfqpoint{0.100000in}{0.212622in}}{\pgfqpoint{3.696000in}{3.696000in}}%
\pgfusepath{clip}%
\pgfsetbuttcap%
\pgfsetroundjoin%
\definecolor{currentfill}{rgb}{0.121569,0.466667,0.705882}%
\pgfsetfillcolor{currentfill}%
\pgfsetfillopacity{0.716970}%
\pgfsetlinewidth{1.003750pt}%
\definecolor{currentstroke}{rgb}{0.121569,0.466667,0.705882}%
\pgfsetstrokecolor{currentstroke}%
\pgfsetstrokeopacity{0.716970}%
\pgfsetdash{}{0pt}%
\pgfpathmoveto{\pgfqpoint{1.419166in}{2.685011in}}%
\pgfpathcurveto{\pgfqpoint{1.427402in}{2.685011in}}{\pgfqpoint{1.435302in}{2.688283in}}{\pgfqpoint{1.441126in}{2.694107in}}%
\pgfpathcurveto{\pgfqpoint{1.446950in}{2.699931in}}{\pgfqpoint{1.450223in}{2.707831in}}{\pgfqpoint{1.450223in}{2.716067in}}%
\pgfpathcurveto{\pgfqpoint{1.450223in}{2.724304in}}{\pgfqpoint{1.446950in}{2.732204in}}{\pgfqpoint{1.441126in}{2.738028in}}%
\pgfpathcurveto{\pgfqpoint{1.435302in}{2.743852in}}{\pgfqpoint{1.427402in}{2.747124in}}{\pgfqpoint{1.419166in}{2.747124in}}%
\pgfpathcurveto{\pgfqpoint{1.410930in}{2.747124in}}{\pgfqpoint{1.403030in}{2.743852in}}{\pgfqpoint{1.397206in}{2.738028in}}%
\pgfpathcurveto{\pgfqpoint{1.391382in}{2.732204in}}{\pgfqpoint{1.388110in}{2.724304in}}{\pgfqpoint{1.388110in}{2.716067in}}%
\pgfpathcurveto{\pgfqpoint{1.388110in}{2.707831in}}{\pgfqpoint{1.391382in}{2.699931in}}{\pgfqpoint{1.397206in}{2.694107in}}%
\pgfpathcurveto{\pgfqpoint{1.403030in}{2.688283in}}{\pgfqpoint{1.410930in}{2.685011in}}{\pgfqpoint{1.419166in}{2.685011in}}%
\pgfpathclose%
\pgfusepath{stroke,fill}%
\end{pgfscope}%
\begin{pgfscope}%
\pgfpathrectangle{\pgfqpoint{0.100000in}{0.212622in}}{\pgfqpoint{3.696000in}{3.696000in}}%
\pgfusepath{clip}%
\pgfsetbuttcap%
\pgfsetroundjoin%
\definecolor{currentfill}{rgb}{0.121569,0.466667,0.705882}%
\pgfsetfillcolor{currentfill}%
\pgfsetfillopacity{0.717234}%
\pgfsetlinewidth{1.003750pt}%
\definecolor{currentstroke}{rgb}{0.121569,0.466667,0.705882}%
\pgfsetstrokecolor{currentstroke}%
\pgfsetstrokeopacity{0.717234}%
\pgfsetdash{}{0pt}%
\pgfpathmoveto{\pgfqpoint{2.559120in}{2.977156in}}%
\pgfpathcurveto{\pgfqpoint{2.567356in}{2.977156in}}{\pgfqpoint{2.575256in}{2.980428in}}{\pgfqpoint{2.581080in}{2.986252in}}%
\pgfpathcurveto{\pgfqpoint{2.586904in}{2.992076in}}{\pgfqpoint{2.590176in}{2.999976in}}{\pgfqpoint{2.590176in}{3.008212in}}%
\pgfpathcurveto{\pgfqpoint{2.590176in}{3.016448in}}{\pgfqpoint{2.586904in}{3.024348in}}{\pgfqpoint{2.581080in}{3.030172in}}%
\pgfpathcurveto{\pgfqpoint{2.575256in}{3.035996in}}{\pgfqpoint{2.567356in}{3.039269in}}{\pgfqpoint{2.559120in}{3.039269in}}%
\pgfpathcurveto{\pgfqpoint{2.550883in}{3.039269in}}{\pgfqpoint{2.542983in}{3.035996in}}{\pgfqpoint{2.537159in}{3.030172in}}%
\pgfpathcurveto{\pgfqpoint{2.531335in}{3.024348in}}{\pgfqpoint{2.528063in}{3.016448in}}{\pgfqpoint{2.528063in}{3.008212in}}%
\pgfpathcurveto{\pgfqpoint{2.528063in}{2.999976in}}{\pgfqpoint{2.531335in}{2.992076in}}{\pgfqpoint{2.537159in}{2.986252in}}%
\pgfpathcurveto{\pgfqpoint{2.542983in}{2.980428in}}{\pgfqpoint{2.550883in}{2.977156in}}{\pgfqpoint{2.559120in}{2.977156in}}%
\pgfpathclose%
\pgfusepath{stroke,fill}%
\end{pgfscope}%
\begin{pgfscope}%
\pgfpathrectangle{\pgfqpoint{0.100000in}{0.212622in}}{\pgfqpoint{3.696000in}{3.696000in}}%
\pgfusepath{clip}%
\pgfsetbuttcap%
\pgfsetroundjoin%
\definecolor{currentfill}{rgb}{0.121569,0.466667,0.705882}%
\pgfsetfillcolor{currentfill}%
\pgfsetfillopacity{0.717401}%
\pgfsetlinewidth{1.003750pt}%
\definecolor{currentstroke}{rgb}{0.121569,0.466667,0.705882}%
\pgfsetstrokecolor{currentstroke}%
\pgfsetstrokeopacity{0.717401}%
\pgfsetdash{}{0pt}%
\pgfpathmoveto{\pgfqpoint{1.417845in}{2.683717in}}%
\pgfpathcurveto{\pgfqpoint{1.426082in}{2.683717in}}{\pgfqpoint{1.433982in}{2.686989in}}{\pgfqpoint{1.439806in}{2.692813in}}%
\pgfpathcurveto{\pgfqpoint{1.445630in}{2.698637in}}{\pgfqpoint{1.448902in}{2.706537in}}{\pgfqpoint{1.448902in}{2.714773in}}%
\pgfpathcurveto{\pgfqpoint{1.448902in}{2.723010in}}{\pgfqpoint{1.445630in}{2.730910in}}{\pgfqpoint{1.439806in}{2.736734in}}%
\pgfpathcurveto{\pgfqpoint{1.433982in}{2.742558in}}{\pgfqpoint{1.426082in}{2.745830in}}{\pgfqpoint{1.417845in}{2.745830in}}%
\pgfpathcurveto{\pgfqpoint{1.409609in}{2.745830in}}{\pgfqpoint{1.401709in}{2.742558in}}{\pgfqpoint{1.395885in}{2.736734in}}%
\pgfpathcurveto{\pgfqpoint{1.390061in}{2.730910in}}{\pgfqpoint{1.386789in}{2.723010in}}{\pgfqpoint{1.386789in}{2.714773in}}%
\pgfpathcurveto{\pgfqpoint{1.386789in}{2.706537in}}{\pgfqpoint{1.390061in}{2.698637in}}{\pgfqpoint{1.395885in}{2.692813in}}%
\pgfpathcurveto{\pgfqpoint{1.401709in}{2.686989in}}{\pgfqpoint{1.409609in}{2.683717in}}{\pgfqpoint{1.417845in}{2.683717in}}%
\pgfpathclose%
\pgfusepath{stroke,fill}%
\end{pgfscope}%
\begin{pgfscope}%
\pgfpathrectangle{\pgfqpoint{0.100000in}{0.212622in}}{\pgfqpoint{3.696000in}{3.696000in}}%
\pgfusepath{clip}%
\pgfsetbuttcap%
\pgfsetroundjoin%
\definecolor{currentfill}{rgb}{0.121569,0.466667,0.705882}%
\pgfsetfillcolor{currentfill}%
\pgfsetfillopacity{0.717604}%
\pgfsetlinewidth{1.003750pt}%
\definecolor{currentstroke}{rgb}{0.121569,0.466667,0.705882}%
\pgfsetstrokecolor{currentstroke}%
\pgfsetstrokeopacity{0.717604}%
\pgfsetdash{}{0pt}%
\pgfpathmoveto{\pgfqpoint{2.562464in}{2.977034in}}%
\pgfpathcurveto{\pgfqpoint{2.570700in}{2.977034in}}{\pgfqpoint{2.578601in}{2.980306in}}{\pgfqpoint{2.584424in}{2.986130in}}%
\pgfpathcurveto{\pgfqpoint{2.590248in}{2.991954in}}{\pgfqpoint{2.593521in}{2.999854in}}{\pgfqpoint{2.593521in}{3.008090in}}%
\pgfpathcurveto{\pgfqpoint{2.593521in}{3.016326in}}{\pgfqpoint{2.590248in}{3.024227in}}{\pgfqpoint{2.584424in}{3.030050in}}%
\pgfpathcurveto{\pgfqpoint{2.578601in}{3.035874in}}{\pgfqpoint{2.570700in}{3.039147in}}{\pgfqpoint{2.562464in}{3.039147in}}%
\pgfpathcurveto{\pgfqpoint{2.554228in}{3.039147in}}{\pgfqpoint{2.546328in}{3.035874in}}{\pgfqpoint{2.540504in}{3.030050in}}%
\pgfpathcurveto{\pgfqpoint{2.534680in}{3.024227in}}{\pgfqpoint{2.531408in}{3.016326in}}{\pgfqpoint{2.531408in}{3.008090in}}%
\pgfpathcurveto{\pgfqpoint{2.531408in}{2.999854in}}{\pgfqpoint{2.534680in}{2.991954in}}{\pgfqpoint{2.540504in}{2.986130in}}%
\pgfpathcurveto{\pgfqpoint{2.546328in}{2.980306in}}{\pgfqpoint{2.554228in}{2.977034in}}{\pgfqpoint{2.562464in}{2.977034in}}%
\pgfpathclose%
\pgfusepath{stroke,fill}%
\end{pgfscope}%
\begin{pgfscope}%
\pgfpathrectangle{\pgfqpoint{0.100000in}{0.212622in}}{\pgfqpoint{3.696000in}{3.696000in}}%
\pgfusepath{clip}%
\pgfsetbuttcap%
\pgfsetroundjoin%
\definecolor{currentfill}{rgb}{0.121569,0.466667,0.705882}%
\pgfsetfillcolor{currentfill}%
\pgfsetfillopacity{0.717849}%
\pgfsetlinewidth{1.003750pt}%
\definecolor{currentstroke}{rgb}{0.121569,0.466667,0.705882}%
\pgfsetstrokecolor{currentstroke}%
\pgfsetstrokeopacity{0.717849}%
\pgfsetdash{}{0pt}%
\pgfpathmoveto{\pgfqpoint{2.565008in}{2.977115in}}%
\pgfpathcurveto{\pgfqpoint{2.573245in}{2.977115in}}{\pgfqpoint{2.581145in}{2.980388in}}{\pgfqpoint{2.586969in}{2.986212in}}%
\pgfpathcurveto{\pgfqpoint{2.592793in}{2.992036in}}{\pgfqpoint{2.596065in}{2.999936in}}{\pgfqpoint{2.596065in}{3.008172in}}%
\pgfpathcurveto{\pgfqpoint{2.596065in}{3.016408in}}{\pgfqpoint{2.592793in}{3.024308in}}{\pgfqpoint{2.586969in}{3.030132in}}%
\pgfpathcurveto{\pgfqpoint{2.581145in}{3.035956in}}{\pgfqpoint{2.573245in}{3.039228in}}{\pgfqpoint{2.565008in}{3.039228in}}%
\pgfpathcurveto{\pgfqpoint{2.556772in}{3.039228in}}{\pgfqpoint{2.548872in}{3.035956in}}{\pgfqpoint{2.543048in}{3.030132in}}%
\pgfpathcurveto{\pgfqpoint{2.537224in}{3.024308in}}{\pgfqpoint{2.533952in}{3.016408in}}{\pgfqpoint{2.533952in}{3.008172in}}%
\pgfpathcurveto{\pgfqpoint{2.533952in}{2.999936in}}{\pgfqpoint{2.537224in}{2.992036in}}{\pgfqpoint{2.543048in}{2.986212in}}%
\pgfpathcurveto{\pgfqpoint{2.548872in}{2.980388in}}{\pgfqpoint{2.556772in}{2.977115in}}{\pgfqpoint{2.565008in}{2.977115in}}%
\pgfpathclose%
\pgfusepath{stroke,fill}%
\end{pgfscope}%
\begin{pgfscope}%
\pgfpathrectangle{\pgfqpoint{0.100000in}{0.212622in}}{\pgfqpoint{3.696000in}{3.696000in}}%
\pgfusepath{clip}%
\pgfsetbuttcap%
\pgfsetroundjoin%
\definecolor{currentfill}{rgb}{0.121569,0.466667,0.705882}%
\pgfsetfillcolor{currentfill}%
\pgfsetfillopacity{0.718092}%
\pgfsetlinewidth{1.003750pt}%
\definecolor{currentstroke}{rgb}{0.121569,0.466667,0.705882}%
\pgfsetstrokecolor{currentstroke}%
\pgfsetstrokeopacity{0.718092}%
\pgfsetdash{}{0pt}%
\pgfpathmoveto{\pgfqpoint{1.415540in}{2.681244in}}%
\pgfpathcurveto{\pgfqpoint{1.423776in}{2.681244in}}{\pgfqpoint{1.431676in}{2.684517in}}{\pgfqpoint{1.437500in}{2.690341in}}%
\pgfpathcurveto{\pgfqpoint{1.443324in}{2.696165in}}{\pgfqpoint{1.446597in}{2.704065in}}{\pgfqpoint{1.446597in}{2.712301in}}%
\pgfpathcurveto{\pgfqpoint{1.446597in}{2.720537in}}{\pgfqpoint{1.443324in}{2.728437in}}{\pgfqpoint{1.437500in}{2.734261in}}%
\pgfpathcurveto{\pgfqpoint{1.431676in}{2.740085in}}{\pgfqpoint{1.423776in}{2.743357in}}{\pgfqpoint{1.415540in}{2.743357in}}%
\pgfpathcurveto{\pgfqpoint{1.407304in}{2.743357in}}{\pgfqpoint{1.399404in}{2.740085in}}{\pgfqpoint{1.393580in}{2.734261in}}%
\pgfpathcurveto{\pgfqpoint{1.387756in}{2.728437in}}{\pgfqpoint{1.384484in}{2.720537in}}{\pgfqpoint{1.384484in}{2.712301in}}%
\pgfpathcurveto{\pgfqpoint{1.384484in}{2.704065in}}{\pgfqpoint{1.387756in}{2.696165in}}{\pgfqpoint{1.393580in}{2.690341in}}%
\pgfpathcurveto{\pgfqpoint{1.399404in}{2.684517in}}{\pgfqpoint{1.407304in}{2.681244in}}{\pgfqpoint{1.415540in}{2.681244in}}%
\pgfpathclose%
\pgfusepath{stroke,fill}%
\end{pgfscope}%
\begin{pgfscope}%
\pgfpathrectangle{\pgfqpoint{0.100000in}{0.212622in}}{\pgfqpoint{3.696000in}{3.696000in}}%
\pgfusepath{clip}%
\pgfsetbuttcap%
\pgfsetroundjoin%
\definecolor{currentfill}{rgb}{0.121569,0.466667,0.705882}%
\pgfsetfillcolor{currentfill}%
\pgfsetfillopacity{0.718307}%
\pgfsetlinewidth{1.003750pt}%
\definecolor{currentstroke}{rgb}{0.121569,0.466667,0.705882}%
\pgfsetstrokecolor{currentstroke}%
\pgfsetstrokeopacity{0.718307}%
\pgfsetdash{}{0pt}%
\pgfpathmoveto{\pgfqpoint{2.569675in}{2.977536in}}%
\pgfpathcurveto{\pgfqpoint{2.577911in}{2.977536in}}{\pgfqpoint{2.585811in}{2.980808in}}{\pgfqpoint{2.591635in}{2.986632in}}%
\pgfpathcurveto{\pgfqpoint{2.597459in}{2.992456in}}{\pgfqpoint{2.600731in}{3.000356in}}{\pgfqpoint{2.600731in}{3.008592in}}%
\pgfpathcurveto{\pgfqpoint{2.600731in}{3.016829in}}{\pgfqpoint{2.597459in}{3.024729in}}{\pgfqpoint{2.591635in}{3.030553in}}%
\pgfpathcurveto{\pgfqpoint{2.585811in}{3.036377in}}{\pgfqpoint{2.577911in}{3.039649in}}{\pgfqpoint{2.569675in}{3.039649in}}%
\pgfpathcurveto{\pgfqpoint{2.561439in}{3.039649in}}{\pgfqpoint{2.553539in}{3.036377in}}{\pgfqpoint{2.547715in}{3.030553in}}%
\pgfpathcurveto{\pgfqpoint{2.541891in}{3.024729in}}{\pgfqpoint{2.538618in}{3.016829in}}{\pgfqpoint{2.538618in}{3.008592in}}%
\pgfpathcurveto{\pgfqpoint{2.538618in}{3.000356in}}{\pgfqpoint{2.541891in}{2.992456in}}{\pgfqpoint{2.547715in}{2.986632in}}%
\pgfpathcurveto{\pgfqpoint{2.553539in}{2.980808in}}{\pgfqpoint{2.561439in}{2.977536in}}{\pgfqpoint{2.569675in}{2.977536in}}%
\pgfpathclose%
\pgfusepath{stroke,fill}%
\end{pgfscope}%
\begin{pgfscope}%
\pgfpathrectangle{\pgfqpoint{0.100000in}{0.212622in}}{\pgfqpoint{3.696000in}{3.696000in}}%
\pgfusepath{clip}%
\pgfsetbuttcap%
\pgfsetroundjoin%
\definecolor{currentfill}{rgb}{0.121569,0.466667,0.705882}%
\pgfsetfillcolor{currentfill}%
\pgfsetfillopacity{0.718476}%
\pgfsetlinewidth{1.003750pt}%
\definecolor{currentstroke}{rgb}{0.121569,0.466667,0.705882}%
\pgfsetstrokecolor{currentstroke}%
\pgfsetstrokeopacity{0.718476}%
\pgfsetdash{}{0pt}%
\pgfpathmoveto{\pgfqpoint{1.414284in}{2.679893in}}%
\pgfpathcurveto{\pgfqpoint{1.422520in}{2.679893in}}{\pgfqpoint{1.430420in}{2.683165in}}{\pgfqpoint{1.436244in}{2.688989in}}%
\pgfpathcurveto{\pgfqpoint{1.442068in}{2.694813in}}{\pgfqpoint{1.445341in}{2.702713in}}{\pgfqpoint{1.445341in}{2.710949in}}%
\pgfpathcurveto{\pgfqpoint{1.445341in}{2.719185in}}{\pgfqpoint{1.442068in}{2.727085in}}{\pgfqpoint{1.436244in}{2.732909in}}%
\pgfpathcurveto{\pgfqpoint{1.430420in}{2.738733in}}{\pgfqpoint{1.422520in}{2.742006in}}{\pgfqpoint{1.414284in}{2.742006in}}%
\pgfpathcurveto{\pgfqpoint{1.406048in}{2.742006in}}{\pgfqpoint{1.398148in}{2.738733in}}{\pgfqpoint{1.392324in}{2.732909in}}%
\pgfpathcurveto{\pgfqpoint{1.386500in}{2.727085in}}{\pgfqpoint{1.383228in}{2.719185in}}{\pgfqpoint{1.383228in}{2.710949in}}%
\pgfpathcurveto{\pgfqpoint{1.383228in}{2.702713in}}{\pgfqpoint{1.386500in}{2.694813in}}{\pgfqpoint{1.392324in}{2.688989in}}%
\pgfpathcurveto{\pgfqpoint{1.398148in}{2.683165in}}{\pgfqpoint{1.406048in}{2.679893in}}{\pgfqpoint{1.414284in}{2.679893in}}%
\pgfpathclose%
\pgfusepath{stroke,fill}%
\end{pgfscope}%
\begin{pgfscope}%
\pgfpathrectangle{\pgfqpoint{0.100000in}{0.212622in}}{\pgfqpoint{3.696000in}{3.696000in}}%
\pgfusepath{clip}%
\pgfsetbuttcap%
\pgfsetroundjoin%
\definecolor{currentfill}{rgb}{0.121569,0.466667,0.705882}%
\pgfsetfillcolor{currentfill}%
\pgfsetfillopacity{0.718660}%
\pgfsetlinewidth{1.003750pt}%
\definecolor{currentstroke}{rgb}{0.121569,0.466667,0.705882}%
\pgfsetstrokecolor{currentstroke}%
\pgfsetstrokeopacity{0.718660}%
\pgfsetdash{}{0pt}%
\pgfpathmoveto{\pgfqpoint{2.573535in}{2.977671in}}%
\pgfpathcurveto{\pgfqpoint{2.581771in}{2.977671in}}{\pgfqpoint{2.589671in}{2.980943in}}{\pgfqpoint{2.595495in}{2.986767in}}%
\pgfpathcurveto{\pgfqpoint{2.601319in}{2.992591in}}{\pgfqpoint{2.604591in}{3.000491in}}{\pgfqpoint{2.604591in}{3.008728in}}%
\pgfpathcurveto{\pgfqpoint{2.604591in}{3.016964in}}{\pgfqpoint{2.601319in}{3.024864in}}{\pgfqpoint{2.595495in}{3.030688in}}%
\pgfpathcurveto{\pgfqpoint{2.589671in}{3.036512in}}{\pgfqpoint{2.581771in}{3.039784in}}{\pgfqpoint{2.573535in}{3.039784in}}%
\pgfpathcurveto{\pgfqpoint{2.565299in}{3.039784in}}{\pgfqpoint{2.557399in}{3.036512in}}{\pgfqpoint{2.551575in}{3.030688in}}%
\pgfpathcurveto{\pgfqpoint{2.545751in}{3.024864in}}{\pgfqpoint{2.542478in}{3.016964in}}{\pgfqpoint{2.542478in}{3.008728in}}%
\pgfpathcurveto{\pgfqpoint{2.542478in}{3.000491in}}{\pgfqpoint{2.545751in}{2.992591in}}{\pgfqpoint{2.551575in}{2.986767in}}%
\pgfpathcurveto{\pgfqpoint{2.557399in}{2.980943in}}{\pgfqpoint{2.565299in}{2.977671in}}{\pgfqpoint{2.573535in}{2.977671in}}%
\pgfpathclose%
\pgfusepath{stroke,fill}%
\end{pgfscope}%
\begin{pgfscope}%
\pgfpathrectangle{\pgfqpoint{0.100000in}{0.212622in}}{\pgfqpoint{3.696000in}{3.696000in}}%
\pgfusepath{clip}%
\pgfsetbuttcap%
\pgfsetroundjoin%
\definecolor{currentfill}{rgb}{0.121569,0.466667,0.705882}%
\pgfsetfillcolor{currentfill}%
\pgfsetfillopacity{0.718952}%
\pgfsetlinewidth{1.003750pt}%
\definecolor{currentstroke}{rgb}{0.121569,0.466667,0.705882}%
\pgfsetstrokecolor{currentstroke}%
\pgfsetstrokeopacity{0.718952}%
\pgfsetdash{}{0pt}%
\pgfpathmoveto{\pgfqpoint{2.576782in}{2.977926in}}%
\pgfpathcurveto{\pgfqpoint{2.585018in}{2.977926in}}{\pgfqpoint{2.592918in}{2.981198in}}{\pgfqpoint{2.598742in}{2.987022in}}%
\pgfpathcurveto{\pgfqpoint{2.604566in}{2.992846in}}{\pgfqpoint{2.607838in}{3.000746in}}{\pgfqpoint{2.607838in}{3.008982in}}%
\pgfpathcurveto{\pgfqpoint{2.607838in}{3.017218in}}{\pgfqpoint{2.604566in}{3.025118in}}{\pgfqpoint{2.598742in}{3.030942in}}%
\pgfpathcurveto{\pgfqpoint{2.592918in}{3.036766in}}{\pgfqpoint{2.585018in}{3.040039in}}{\pgfqpoint{2.576782in}{3.040039in}}%
\pgfpathcurveto{\pgfqpoint{2.568545in}{3.040039in}}{\pgfqpoint{2.560645in}{3.036766in}}{\pgfqpoint{2.554821in}{3.030942in}}%
\pgfpathcurveto{\pgfqpoint{2.548998in}{3.025118in}}{\pgfqpoint{2.545725in}{3.017218in}}{\pgfqpoint{2.545725in}{3.008982in}}%
\pgfpathcurveto{\pgfqpoint{2.545725in}{3.000746in}}{\pgfqpoint{2.548998in}{2.992846in}}{\pgfqpoint{2.554821in}{2.987022in}}%
\pgfpathcurveto{\pgfqpoint{2.560645in}{2.981198in}}{\pgfqpoint{2.568545in}{2.977926in}}{\pgfqpoint{2.576782in}{2.977926in}}%
\pgfpathclose%
\pgfusepath{stroke,fill}%
\end{pgfscope}%
\begin{pgfscope}%
\pgfpathrectangle{\pgfqpoint{0.100000in}{0.212622in}}{\pgfqpoint{3.696000in}{3.696000in}}%
\pgfusepath{clip}%
\pgfsetbuttcap%
\pgfsetroundjoin%
\definecolor{currentfill}{rgb}{0.121569,0.466667,0.705882}%
\pgfsetfillcolor{currentfill}%
\pgfsetfillopacity{0.719207}%
\pgfsetlinewidth{1.003750pt}%
\definecolor{currentstroke}{rgb}{0.121569,0.466667,0.705882}%
\pgfsetstrokecolor{currentstroke}%
\pgfsetstrokeopacity{0.719207}%
\pgfsetdash{}{0pt}%
\pgfpathmoveto{\pgfqpoint{2.579377in}{2.978336in}}%
\pgfpathcurveto{\pgfqpoint{2.587613in}{2.978336in}}{\pgfqpoint{2.595513in}{2.981609in}}{\pgfqpoint{2.601337in}{2.987433in}}%
\pgfpathcurveto{\pgfqpoint{2.607161in}{2.993256in}}{\pgfqpoint{2.610433in}{3.001157in}}{\pgfqpoint{2.610433in}{3.009393in}}%
\pgfpathcurveto{\pgfqpoint{2.610433in}{3.017629in}}{\pgfqpoint{2.607161in}{3.025529in}}{\pgfqpoint{2.601337in}{3.031353in}}%
\pgfpathcurveto{\pgfqpoint{2.595513in}{3.037177in}}{\pgfqpoint{2.587613in}{3.040449in}}{\pgfqpoint{2.579377in}{3.040449in}}%
\pgfpathcurveto{\pgfqpoint{2.571141in}{3.040449in}}{\pgfqpoint{2.563241in}{3.037177in}}{\pgfqpoint{2.557417in}{3.031353in}}%
\pgfpathcurveto{\pgfqpoint{2.551593in}{3.025529in}}{\pgfqpoint{2.548320in}{3.017629in}}{\pgfqpoint{2.548320in}{3.009393in}}%
\pgfpathcurveto{\pgfqpoint{2.548320in}{3.001157in}}{\pgfqpoint{2.551593in}{2.993256in}}{\pgfqpoint{2.557417in}{2.987433in}}%
\pgfpathcurveto{\pgfqpoint{2.563241in}{2.981609in}}{\pgfqpoint{2.571141in}{2.978336in}}{\pgfqpoint{2.579377in}{2.978336in}}%
\pgfpathclose%
\pgfusepath{stroke,fill}%
\end{pgfscope}%
\begin{pgfscope}%
\pgfpathrectangle{\pgfqpoint{0.100000in}{0.212622in}}{\pgfqpoint{3.696000in}{3.696000in}}%
\pgfusepath{clip}%
\pgfsetbuttcap%
\pgfsetroundjoin%
\definecolor{currentfill}{rgb}{0.121569,0.466667,0.705882}%
\pgfsetfillcolor{currentfill}%
\pgfsetfillopacity{0.719254}%
\pgfsetlinewidth{1.003750pt}%
\definecolor{currentstroke}{rgb}{0.121569,0.466667,0.705882}%
\pgfsetstrokecolor{currentstroke}%
\pgfsetstrokeopacity{0.719254}%
\pgfsetdash{}{0pt}%
\pgfpathmoveto{\pgfqpoint{1.411953in}{2.677506in}}%
\pgfpathcurveto{\pgfqpoint{1.420189in}{2.677506in}}{\pgfqpoint{1.428090in}{2.680779in}}{\pgfqpoint{1.433913in}{2.686603in}}%
\pgfpathcurveto{\pgfqpoint{1.439737in}{2.692427in}}{\pgfqpoint{1.443010in}{2.700327in}}{\pgfqpoint{1.443010in}{2.708563in}}%
\pgfpathcurveto{\pgfqpoint{1.443010in}{2.716799in}}{\pgfqpoint{1.439737in}{2.724699in}}{\pgfqpoint{1.433913in}{2.730523in}}%
\pgfpathcurveto{\pgfqpoint{1.428090in}{2.736347in}}{\pgfqpoint{1.420189in}{2.739619in}}{\pgfqpoint{1.411953in}{2.739619in}}%
\pgfpathcurveto{\pgfqpoint{1.403717in}{2.739619in}}{\pgfqpoint{1.395817in}{2.736347in}}{\pgfqpoint{1.389993in}{2.730523in}}%
\pgfpathcurveto{\pgfqpoint{1.384169in}{2.724699in}}{\pgfqpoint{1.380897in}{2.716799in}}{\pgfqpoint{1.380897in}{2.708563in}}%
\pgfpathcurveto{\pgfqpoint{1.380897in}{2.700327in}}{\pgfqpoint{1.384169in}{2.692427in}}{\pgfqpoint{1.389993in}{2.686603in}}%
\pgfpathcurveto{\pgfqpoint{1.395817in}{2.680779in}}{\pgfqpoint{1.403717in}{2.677506in}}{\pgfqpoint{1.411953in}{2.677506in}}%
\pgfpathclose%
\pgfusepath{stroke,fill}%
\end{pgfscope}%
\begin{pgfscope}%
\pgfpathrectangle{\pgfqpoint{0.100000in}{0.212622in}}{\pgfqpoint{3.696000in}{3.696000in}}%
\pgfusepath{clip}%
\pgfsetbuttcap%
\pgfsetroundjoin%
\definecolor{currentfill}{rgb}{0.121569,0.466667,0.705882}%
\pgfsetfillcolor{currentfill}%
\pgfsetfillopacity{0.719390}%
\pgfsetlinewidth{1.003750pt}%
\definecolor{currentstroke}{rgb}{0.121569,0.466667,0.705882}%
\pgfsetstrokecolor{currentstroke}%
\pgfsetstrokeopacity{0.719390}%
\pgfsetdash{}{0pt}%
\pgfpathmoveto{\pgfqpoint{2.581394in}{2.978335in}}%
\pgfpathcurveto{\pgfqpoint{2.589630in}{2.978335in}}{\pgfqpoint{2.597530in}{2.981608in}}{\pgfqpoint{2.603354in}{2.987432in}}%
\pgfpathcurveto{\pgfqpoint{2.609178in}{2.993256in}}{\pgfqpoint{2.612450in}{3.001156in}}{\pgfqpoint{2.612450in}{3.009392in}}%
\pgfpathcurveto{\pgfqpoint{2.612450in}{3.017628in}}{\pgfqpoint{2.609178in}{3.025528in}}{\pgfqpoint{2.603354in}{3.031352in}}%
\pgfpathcurveto{\pgfqpoint{2.597530in}{3.037176in}}{\pgfqpoint{2.589630in}{3.040448in}}{\pgfqpoint{2.581394in}{3.040448in}}%
\pgfpathcurveto{\pgfqpoint{2.573157in}{3.040448in}}{\pgfqpoint{2.565257in}{3.037176in}}{\pgfqpoint{2.559433in}{3.031352in}}%
\pgfpathcurveto{\pgfqpoint{2.553610in}{3.025528in}}{\pgfqpoint{2.550337in}{3.017628in}}{\pgfqpoint{2.550337in}{3.009392in}}%
\pgfpathcurveto{\pgfqpoint{2.550337in}{3.001156in}}{\pgfqpoint{2.553610in}{2.993256in}}{\pgfqpoint{2.559433in}{2.987432in}}%
\pgfpathcurveto{\pgfqpoint{2.565257in}{2.981608in}}{\pgfqpoint{2.573157in}{2.978335in}}{\pgfqpoint{2.581394in}{2.978335in}}%
\pgfpathclose%
\pgfusepath{stroke,fill}%
\end{pgfscope}%
\begin{pgfscope}%
\pgfpathrectangle{\pgfqpoint{0.100000in}{0.212622in}}{\pgfqpoint{3.696000in}{3.696000in}}%
\pgfusepath{clip}%
\pgfsetbuttcap%
\pgfsetroundjoin%
\definecolor{currentfill}{rgb}{0.121569,0.466667,0.705882}%
\pgfsetfillcolor{currentfill}%
\pgfsetfillopacity{0.719709}%
\pgfsetlinewidth{1.003750pt}%
\definecolor{currentstroke}{rgb}{0.121569,0.466667,0.705882}%
\pgfsetstrokecolor{currentstroke}%
\pgfsetstrokeopacity{0.719709}%
\pgfsetdash{}{0pt}%
\pgfpathmoveto{\pgfqpoint{2.585042in}{2.978171in}}%
\pgfpathcurveto{\pgfqpoint{2.593278in}{2.978171in}}{\pgfqpoint{2.601178in}{2.981444in}}{\pgfqpoint{2.607002in}{2.987267in}}%
\pgfpathcurveto{\pgfqpoint{2.612826in}{2.993091in}}{\pgfqpoint{2.616098in}{3.000991in}}{\pgfqpoint{2.616098in}{3.009228in}}%
\pgfpathcurveto{\pgfqpoint{2.616098in}{3.017464in}}{\pgfqpoint{2.612826in}{3.025364in}}{\pgfqpoint{2.607002in}{3.031188in}}%
\pgfpathcurveto{\pgfqpoint{2.601178in}{3.037012in}}{\pgfqpoint{2.593278in}{3.040284in}}{\pgfqpoint{2.585042in}{3.040284in}}%
\pgfpathcurveto{\pgfqpoint{2.576805in}{3.040284in}}{\pgfqpoint{2.568905in}{3.037012in}}{\pgfqpoint{2.563081in}{3.031188in}}%
\pgfpathcurveto{\pgfqpoint{2.557257in}{3.025364in}}{\pgfqpoint{2.553985in}{3.017464in}}{\pgfqpoint{2.553985in}{3.009228in}}%
\pgfpathcurveto{\pgfqpoint{2.553985in}{3.000991in}}{\pgfqpoint{2.557257in}{2.993091in}}{\pgfqpoint{2.563081in}{2.987267in}}%
\pgfpathcurveto{\pgfqpoint{2.568905in}{2.981444in}}{\pgfqpoint{2.576805in}{2.978171in}}{\pgfqpoint{2.585042in}{2.978171in}}%
\pgfpathclose%
\pgfusepath{stroke,fill}%
\end{pgfscope}%
\begin{pgfscope}%
\pgfpathrectangle{\pgfqpoint{0.100000in}{0.212622in}}{\pgfqpoint{3.696000in}{3.696000in}}%
\pgfusepath{clip}%
\pgfsetbuttcap%
\pgfsetroundjoin%
\definecolor{currentfill}{rgb}{0.121569,0.466667,0.705882}%
\pgfsetfillcolor{currentfill}%
\pgfsetfillopacity{0.719722}%
\pgfsetlinewidth{1.003750pt}%
\definecolor{currentstroke}{rgb}{0.121569,0.466667,0.705882}%
\pgfsetstrokecolor{currentstroke}%
\pgfsetstrokeopacity{0.719722}%
\pgfsetdash{}{0pt}%
\pgfpathmoveto{\pgfqpoint{1.410654in}{2.676460in}}%
\pgfpathcurveto{\pgfqpoint{1.418890in}{2.676460in}}{\pgfqpoint{1.426790in}{2.679732in}}{\pgfqpoint{1.432614in}{2.685556in}}%
\pgfpathcurveto{\pgfqpoint{1.438438in}{2.691380in}}{\pgfqpoint{1.441710in}{2.699280in}}{\pgfqpoint{1.441710in}{2.707517in}}%
\pgfpathcurveto{\pgfqpoint{1.441710in}{2.715753in}}{\pgfqpoint{1.438438in}{2.723653in}}{\pgfqpoint{1.432614in}{2.729477in}}%
\pgfpathcurveto{\pgfqpoint{1.426790in}{2.735301in}}{\pgfqpoint{1.418890in}{2.738573in}}{\pgfqpoint{1.410654in}{2.738573in}}%
\pgfpathcurveto{\pgfqpoint{1.402417in}{2.738573in}}{\pgfqpoint{1.394517in}{2.735301in}}{\pgfqpoint{1.388693in}{2.729477in}}%
\pgfpathcurveto{\pgfqpoint{1.382869in}{2.723653in}}{\pgfqpoint{1.379597in}{2.715753in}}{\pgfqpoint{1.379597in}{2.707517in}}%
\pgfpathcurveto{\pgfqpoint{1.379597in}{2.699280in}}{\pgfqpoint{1.382869in}{2.691380in}}{\pgfqpoint{1.388693in}{2.685556in}}%
\pgfpathcurveto{\pgfqpoint{1.394517in}{2.679732in}}{\pgfqpoint{1.402417in}{2.676460in}}{\pgfqpoint{1.410654in}{2.676460in}}%
\pgfpathclose%
\pgfusepath{stroke,fill}%
\end{pgfscope}%
\begin{pgfscope}%
\pgfpathrectangle{\pgfqpoint{0.100000in}{0.212622in}}{\pgfqpoint{3.696000in}{3.696000in}}%
\pgfusepath{clip}%
\pgfsetbuttcap%
\pgfsetroundjoin%
\definecolor{currentfill}{rgb}{0.121569,0.466667,0.705882}%
\pgfsetfillcolor{currentfill}%
\pgfsetfillopacity{0.719935}%
\pgfsetlinewidth{1.003750pt}%
\definecolor{currentstroke}{rgb}{0.121569,0.466667,0.705882}%
\pgfsetstrokecolor{currentstroke}%
\pgfsetstrokeopacity{0.719935}%
\pgfsetdash{}{0pt}%
\pgfpathmoveto{\pgfqpoint{2.587339in}{2.978232in}}%
\pgfpathcurveto{\pgfqpoint{2.595576in}{2.978232in}}{\pgfqpoint{2.603476in}{2.981505in}}{\pgfqpoint{2.609300in}{2.987329in}}%
\pgfpathcurveto{\pgfqpoint{2.615124in}{2.993153in}}{\pgfqpoint{2.618396in}{3.001053in}}{\pgfqpoint{2.618396in}{3.009289in}}%
\pgfpathcurveto{\pgfqpoint{2.618396in}{3.017525in}}{\pgfqpoint{2.615124in}{3.025425in}}{\pgfqpoint{2.609300in}{3.031249in}}%
\pgfpathcurveto{\pgfqpoint{2.603476in}{3.037073in}}{\pgfqpoint{2.595576in}{3.040345in}}{\pgfqpoint{2.587339in}{3.040345in}}%
\pgfpathcurveto{\pgfqpoint{2.579103in}{3.040345in}}{\pgfqpoint{2.571203in}{3.037073in}}{\pgfqpoint{2.565379in}{3.031249in}}%
\pgfpathcurveto{\pgfqpoint{2.559555in}{3.025425in}}{\pgfqpoint{2.556283in}{3.017525in}}{\pgfqpoint{2.556283in}{3.009289in}}%
\pgfpathcurveto{\pgfqpoint{2.556283in}{3.001053in}}{\pgfqpoint{2.559555in}{2.993153in}}{\pgfqpoint{2.565379in}{2.987329in}}%
\pgfpathcurveto{\pgfqpoint{2.571203in}{2.981505in}}{\pgfqpoint{2.579103in}{2.978232in}}{\pgfqpoint{2.587339in}{2.978232in}}%
\pgfpathclose%
\pgfusepath{stroke,fill}%
\end{pgfscope}%
\begin{pgfscope}%
\pgfpathrectangle{\pgfqpoint{0.100000in}{0.212622in}}{\pgfqpoint{3.696000in}{3.696000in}}%
\pgfusepath{clip}%
\pgfsetbuttcap%
\pgfsetroundjoin%
\definecolor{currentfill}{rgb}{0.121569,0.466667,0.705882}%
\pgfsetfillcolor{currentfill}%
\pgfsetfillopacity{0.720364}%
\pgfsetlinewidth{1.003750pt}%
\definecolor{currentstroke}{rgb}{0.121569,0.466667,0.705882}%
\pgfsetstrokecolor{currentstroke}%
\pgfsetstrokeopacity{0.720364}%
\pgfsetdash{}{0pt}%
\pgfpathmoveto{\pgfqpoint{2.591530in}{2.978494in}}%
\pgfpathcurveto{\pgfqpoint{2.599766in}{2.978494in}}{\pgfqpoint{2.607666in}{2.981766in}}{\pgfqpoint{2.613490in}{2.987590in}}%
\pgfpathcurveto{\pgfqpoint{2.619314in}{2.993414in}}{\pgfqpoint{2.622586in}{3.001314in}}{\pgfqpoint{2.622586in}{3.009551in}}%
\pgfpathcurveto{\pgfqpoint{2.622586in}{3.017787in}}{\pgfqpoint{2.619314in}{3.025687in}}{\pgfqpoint{2.613490in}{3.031511in}}%
\pgfpathcurveto{\pgfqpoint{2.607666in}{3.037335in}}{\pgfqpoint{2.599766in}{3.040607in}}{\pgfqpoint{2.591530in}{3.040607in}}%
\pgfpathcurveto{\pgfqpoint{2.583293in}{3.040607in}}{\pgfqpoint{2.575393in}{3.037335in}}{\pgfqpoint{2.569569in}{3.031511in}}%
\pgfpathcurveto{\pgfqpoint{2.563745in}{3.025687in}}{\pgfqpoint{2.560473in}{3.017787in}}{\pgfqpoint{2.560473in}{3.009551in}}%
\pgfpathcurveto{\pgfqpoint{2.560473in}{3.001314in}}{\pgfqpoint{2.563745in}{2.993414in}}{\pgfqpoint{2.569569in}{2.987590in}}%
\pgfpathcurveto{\pgfqpoint{2.575393in}{2.981766in}}{\pgfqpoint{2.583293in}{2.978494in}}{\pgfqpoint{2.591530in}{2.978494in}}%
\pgfpathclose%
\pgfusepath{stroke,fill}%
\end{pgfscope}%
\begin{pgfscope}%
\pgfpathrectangle{\pgfqpoint{0.100000in}{0.212622in}}{\pgfqpoint{3.696000in}{3.696000in}}%
\pgfusepath{clip}%
\pgfsetbuttcap%
\pgfsetroundjoin%
\definecolor{currentfill}{rgb}{0.121569,0.466667,0.705882}%
\pgfsetfillcolor{currentfill}%
\pgfsetfillopacity{0.720413}%
\pgfsetlinewidth{1.003750pt}%
\definecolor{currentstroke}{rgb}{0.121569,0.466667,0.705882}%
\pgfsetstrokecolor{currentstroke}%
\pgfsetstrokeopacity{0.720413}%
\pgfsetdash{}{0pt}%
\pgfpathmoveto{\pgfqpoint{1.408593in}{2.674689in}}%
\pgfpathcurveto{\pgfqpoint{1.416830in}{2.674689in}}{\pgfqpoint{1.424730in}{2.677961in}}{\pgfqpoint{1.430554in}{2.683785in}}%
\pgfpathcurveto{\pgfqpoint{1.436378in}{2.689609in}}{\pgfqpoint{1.439650in}{2.697509in}}{\pgfqpoint{1.439650in}{2.705746in}}%
\pgfpathcurveto{\pgfqpoint{1.439650in}{2.713982in}}{\pgfqpoint{1.436378in}{2.721882in}}{\pgfqpoint{1.430554in}{2.727706in}}%
\pgfpathcurveto{\pgfqpoint{1.424730in}{2.733530in}}{\pgfqpoint{1.416830in}{2.736802in}}{\pgfqpoint{1.408593in}{2.736802in}}%
\pgfpathcurveto{\pgfqpoint{1.400357in}{2.736802in}}{\pgfqpoint{1.392457in}{2.733530in}}{\pgfqpoint{1.386633in}{2.727706in}}%
\pgfpathcurveto{\pgfqpoint{1.380809in}{2.721882in}}{\pgfqpoint{1.377537in}{2.713982in}}{\pgfqpoint{1.377537in}{2.705746in}}%
\pgfpathcurveto{\pgfqpoint{1.377537in}{2.697509in}}{\pgfqpoint{1.380809in}{2.689609in}}{\pgfqpoint{1.386633in}{2.683785in}}%
\pgfpathcurveto{\pgfqpoint{1.392457in}{2.677961in}}{\pgfqpoint{1.400357in}{2.674689in}}{\pgfqpoint{1.408593in}{2.674689in}}%
\pgfpathclose%
\pgfusepath{stroke,fill}%
\end{pgfscope}%
\begin{pgfscope}%
\pgfpathrectangle{\pgfqpoint{0.100000in}{0.212622in}}{\pgfqpoint{3.696000in}{3.696000in}}%
\pgfusepath{clip}%
\pgfsetbuttcap%
\pgfsetroundjoin%
\definecolor{currentfill}{rgb}{0.121569,0.466667,0.705882}%
\pgfsetfillcolor{currentfill}%
\pgfsetfillopacity{0.720666}%
\pgfsetlinewidth{1.003750pt}%
\definecolor{currentstroke}{rgb}{0.121569,0.466667,0.705882}%
\pgfsetstrokecolor{currentstroke}%
\pgfsetstrokeopacity{0.720666}%
\pgfsetdash{}{0pt}%
\pgfpathmoveto{\pgfqpoint{2.594633in}{2.978575in}}%
\pgfpathcurveto{\pgfqpoint{2.602869in}{2.978575in}}{\pgfqpoint{2.610769in}{2.981848in}}{\pgfqpoint{2.616593in}{2.987672in}}%
\pgfpathcurveto{\pgfqpoint{2.622417in}{2.993495in}}{\pgfqpoint{2.625689in}{3.001396in}}{\pgfqpoint{2.625689in}{3.009632in}}%
\pgfpathcurveto{\pgfqpoint{2.625689in}{3.017868in}}{\pgfqpoint{2.622417in}{3.025768in}}{\pgfqpoint{2.616593in}{3.031592in}}%
\pgfpathcurveto{\pgfqpoint{2.610769in}{3.037416in}}{\pgfqpoint{2.602869in}{3.040688in}}{\pgfqpoint{2.594633in}{3.040688in}}%
\pgfpathcurveto{\pgfqpoint{2.586396in}{3.040688in}}{\pgfqpoint{2.578496in}{3.037416in}}{\pgfqpoint{2.572672in}{3.031592in}}%
\pgfpathcurveto{\pgfqpoint{2.566848in}{3.025768in}}{\pgfqpoint{2.563576in}{3.017868in}}{\pgfqpoint{2.563576in}{3.009632in}}%
\pgfpathcurveto{\pgfqpoint{2.563576in}{3.001396in}}{\pgfqpoint{2.566848in}{2.993495in}}{\pgfqpoint{2.572672in}{2.987672in}}%
\pgfpathcurveto{\pgfqpoint{2.578496in}{2.981848in}}{\pgfqpoint{2.586396in}{2.978575in}}{\pgfqpoint{2.594633in}{2.978575in}}%
\pgfpathclose%
\pgfusepath{stroke,fill}%
\end{pgfscope}%
\begin{pgfscope}%
\pgfpathrectangle{\pgfqpoint{0.100000in}{0.212622in}}{\pgfqpoint{3.696000in}{3.696000in}}%
\pgfusepath{clip}%
\pgfsetbuttcap%
\pgfsetroundjoin%
\definecolor{currentfill}{rgb}{0.121569,0.466667,0.705882}%
\pgfsetfillcolor{currentfill}%
\pgfsetfillopacity{0.720909}%
\pgfsetlinewidth{1.003750pt}%
\definecolor{currentstroke}{rgb}{0.121569,0.466667,0.705882}%
\pgfsetstrokecolor{currentstroke}%
\pgfsetstrokeopacity{0.720909}%
\pgfsetdash{}{0pt}%
\pgfpathmoveto{\pgfqpoint{2.597059in}{2.978669in}}%
\pgfpathcurveto{\pgfqpoint{2.605295in}{2.978669in}}{\pgfqpoint{2.613195in}{2.981941in}}{\pgfqpoint{2.619019in}{2.987765in}}%
\pgfpathcurveto{\pgfqpoint{2.624843in}{2.993589in}}{\pgfqpoint{2.628115in}{3.001489in}}{\pgfqpoint{2.628115in}{3.009726in}}%
\pgfpathcurveto{\pgfqpoint{2.628115in}{3.017962in}}{\pgfqpoint{2.624843in}{3.025862in}}{\pgfqpoint{2.619019in}{3.031686in}}%
\pgfpathcurveto{\pgfqpoint{2.613195in}{3.037510in}}{\pgfqpoint{2.605295in}{3.040782in}}{\pgfqpoint{2.597059in}{3.040782in}}%
\pgfpathcurveto{\pgfqpoint{2.588822in}{3.040782in}}{\pgfqpoint{2.580922in}{3.037510in}}{\pgfqpoint{2.575098in}{3.031686in}}%
\pgfpathcurveto{\pgfqpoint{2.569274in}{3.025862in}}{\pgfqpoint{2.566002in}{3.017962in}}{\pgfqpoint{2.566002in}{3.009726in}}%
\pgfpathcurveto{\pgfqpoint{2.566002in}{3.001489in}}{\pgfqpoint{2.569274in}{2.993589in}}{\pgfqpoint{2.575098in}{2.987765in}}%
\pgfpathcurveto{\pgfqpoint{2.580922in}{2.981941in}}{\pgfqpoint{2.588822in}{2.978669in}}{\pgfqpoint{2.597059in}{2.978669in}}%
\pgfpathclose%
\pgfusepath{stroke,fill}%
\end{pgfscope}%
\begin{pgfscope}%
\pgfpathrectangle{\pgfqpoint{0.100000in}{0.212622in}}{\pgfqpoint{3.696000in}{3.696000in}}%
\pgfusepath{clip}%
\pgfsetbuttcap%
\pgfsetroundjoin%
\definecolor{currentfill}{rgb}{0.121569,0.466667,0.705882}%
\pgfsetfillcolor{currentfill}%
\pgfsetfillopacity{0.721127}%
\pgfsetlinewidth{1.003750pt}%
\definecolor{currentstroke}{rgb}{0.121569,0.466667,0.705882}%
\pgfsetstrokecolor{currentstroke}%
\pgfsetstrokeopacity{0.721127}%
\pgfsetdash{}{0pt}%
\pgfpathmoveto{\pgfqpoint{2.599035in}{2.978828in}}%
\pgfpathcurveto{\pgfqpoint{2.607272in}{2.978828in}}{\pgfqpoint{2.615172in}{2.982100in}}{\pgfqpoint{2.620996in}{2.987924in}}%
\pgfpathcurveto{\pgfqpoint{2.626820in}{2.993748in}}{\pgfqpoint{2.630092in}{3.001648in}}{\pgfqpoint{2.630092in}{3.009885in}}%
\pgfpathcurveto{\pgfqpoint{2.630092in}{3.018121in}}{\pgfqpoint{2.626820in}{3.026021in}}{\pgfqpoint{2.620996in}{3.031845in}}%
\pgfpathcurveto{\pgfqpoint{2.615172in}{3.037669in}}{\pgfqpoint{2.607272in}{3.040941in}}{\pgfqpoint{2.599035in}{3.040941in}}%
\pgfpathcurveto{\pgfqpoint{2.590799in}{3.040941in}}{\pgfqpoint{2.582899in}{3.037669in}}{\pgfqpoint{2.577075in}{3.031845in}}%
\pgfpathcurveto{\pgfqpoint{2.571251in}{3.026021in}}{\pgfqpoint{2.567979in}{3.018121in}}{\pgfqpoint{2.567979in}{3.009885in}}%
\pgfpathcurveto{\pgfqpoint{2.567979in}{3.001648in}}{\pgfqpoint{2.571251in}{2.993748in}}{\pgfqpoint{2.577075in}{2.987924in}}%
\pgfpathcurveto{\pgfqpoint{2.582899in}{2.982100in}}{\pgfqpoint{2.590799in}{2.978828in}}{\pgfqpoint{2.599035in}{2.978828in}}%
\pgfpathclose%
\pgfusepath{stroke,fill}%
\end{pgfscope}%
\begin{pgfscope}%
\pgfpathrectangle{\pgfqpoint{0.100000in}{0.212622in}}{\pgfqpoint{3.696000in}{3.696000in}}%
\pgfusepath{clip}%
\pgfsetbuttcap%
\pgfsetroundjoin%
\definecolor{currentfill}{rgb}{0.121569,0.466667,0.705882}%
\pgfsetfillcolor{currentfill}%
\pgfsetfillopacity{0.721255}%
\pgfsetlinewidth{1.003750pt}%
\definecolor{currentstroke}{rgb}{0.121569,0.466667,0.705882}%
\pgfsetstrokecolor{currentstroke}%
\pgfsetstrokeopacity{0.721255}%
\pgfsetdash{}{0pt}%
\pgfpathmoveto{\pgfqpoint{2.600061in}{2.978866in}}%
\pgfpathcurveto{\pgfqpoint{2.608298in}{2.978866in}}{\pgfqpoint{2.616198in}{2.982139in}}{\pgfqpoint{2.622022in}{2.987962in}}%
\pgfpathcurveto{\pgfqpoint{2.627846in}{2.993786in}}{\pgfqpoint{2.631118in}{3.001686in}}{\pgfqpoint{2.631118in}{3.009923in}}%
\pgfpathcurveto{\pgfqpoint{2.631118in}{3.018159in}}{\pgfqpoint{2.627846in}{3.026059in}}{\pgfqpoint{2.622022in}{3.031883in}}%
\pgfpathcurveto{\pgfqpoint{2.616198in}{3.037707in}}{\pgfqpoint{2.608298in}{3.040979in}}{\pgfqpoint{2.600061in}{3.040979in}}%
\pgfpathcurveto{\pgfqpoint{2.591825in}{3.040979in}}{\pgfqpoint{2.583925in}{3.037707in}}{\pgfqpoint{2.578101in}{3.031883in}}%
\pgfpathcurveto{\pgfqpoint{2.572277in}{3.026059in}}{\pgfqpoint{2.569005in}{3.018159in}}{\pgfqpoint{2.569005in}{3.009923in}}%
\pgfpathcurveto{\pgfqpoint{2.569005in}{3.001686in}}{\pgfqpoint{2.572277in}{2.993786in}}{\pgfqpoint{2.578101in}{2.987962in}}%
\pgfpathcurveto{\pgfqpoint{2.583925in}{2.982139in}}{\pgfqpoint{2.591825in}{2.978866in}}{\pgfqpoint{2.600061in}{2.978866in}}%
\pgfpathclose%
\pgfusepath{stroke,fill}%
\end{pgfscope}%
\begin{pgfscope}%
\pgfpathrectangle{\pgfqpoint{0.100000in}{0.212622in}}{\pgfqpoint{3.696000in}{3.696000in}}%
\pgfusepath{clip}%
\pgfsetbuttcap%
\pgfsetroundjoin%
\definecolor{currentfill}{rgb}{0.121569,0.466667,0.705882}%
\pgfsetfillcolor{currentfill}%
\pgfsetfillopacity{0.721270}%
\pgfsetlinewidth{1.003750pt}%
\definecolor{currentstroke}{rgb}{0.121569,0.466667,0.705882}%
\pgfsetstrokecolor{currentstroke}%
\pgfsetstrokeopacity{0.721270}%
\pgfsetdash{}{0pt}%
\pgfpathmoveto{\pgfqpoint{1.405838in}{2.671938in}}%
\pgfpathcurveto{\pgfqpoint{1.414075in}{2.671938in}}{\pgfqpoint{1.421975in}{2.675211in}}{\pgfqpoint{1.427799in}{2.681035in}}%
\pgfpathcurveto{\pgfqpoint{1.433623in}{2.686858in}}{\pgfqpoint{1.436895in}{2.694759in}}{\pgfqpoint{1.436895in}{2.702995in}}%
\pgfpathcurveto{\pgfqpoint{1.436895in}{2.711231in}}{\pgfqpoint{1.433623in}{2.719131in}}{\pgfqpoint{1.427799in}{2.724955in}}%
\pgfpathcurveto{\pgfqpoint{1.421975in}{2.730779in}}{\pgfqpoint{1.414075in}{2.734051in}}{\pgfqpoint{1.405838in}{2.734051in}}%
\pgfpathcurveto{\pgfqpoint{1.397602in}{2.734051in}}{\pgfqpoint{1.389702in}{2.730779in}}{\pgfqpoint{1.383878in}{2.724955in}}%
\pgfpathcurveto{\pgfqpoint{1.378054in}{2.719131in}}{\pgfqpoint{1.374782in}{2.711231in}}{\pgfqpoint{1.374782in}{2.702995in}}%
\pgfpathcurveto{\pgfqpoint{1.374782in}{2.694759in}}{\pgfqpoint{1.378054in}{2.686858in}}{\pgfqpoint{1.383878in}{2.681035in}}%
\pgfpathcurveto{\pgfqpoint{1.389702in}{2.675211in}}{\pgfqpoint{1.397602in}{2.671938in}}{\pgfqpoint{1.405838in}{2.671938in}}%
\pgfpathclose%
\pgfusepath{stroke,fill}%
\end{pgfscope}%
\begin{pgfscope}%
\pgfpathrectangle{\pgfqpoint{0.100000in}{0.212622in}}{\pgfqpoint{3.696000in}{3.696000in}}%
\pgfusepath{clip}%
\pgfsetbuttcap%
\pgfsetroundjoin%
\definecolor{currentfill}{rgb}{0.121569,0.466667,0.705882}%
\pgfsetfillcolor{currentfill}%
\pgfsetfillopacity{0.721491}%
\pgfsetlinewidth{1.003750pt}%
\definecolor{currentstroke}{rgb}{0.121569,0.466667,0.705882}%
\pgfsetstrokecolor{currentstroke}%
\pgfsetstrokeopacity{0.721491}%
\pgfsetdash{}{0pt}%
\pgfpathmoveto{\pgfqpoint{2.601916in}{2.978900in}}%
\pgfpathcurveto{\pgfqpoint{2.610153in}{2.978900in}}{\pgfqpoint{2.618053in}{2.982173in}}{\pgfqpoint{2.623877in}{2.987996in}}%
\pgfpathcurveto{\pgfqpoint{2.629701in}{2.993820in}}{\pgfqpoint{2.632973in}{3.001720in}}{\pgfqpoint{2.632973in}{3.009957in}}%
\pgfpathcurveto{\pgfqpoint{2.632973in}{3.018193in}}{\pgfqpoint{2.629701in}{3.026093in}}{\pgfqpoint{2.623877in}{3.031917in}}%
\pgfpathcurveto{\pgfqpoint{2.618053in}{3.037741in}}{\pgfqpoint{2.610153in}{3.041013in}}{\pgfqpoint{2.601916in}{3.041013in}}%
\pgfpathcurveto{\pgfqpoint{2.593680in}{3.041013in}}{\pgfqpoint{2.585780in}{3.037741in}}{\pgfqpoint{2.579956in}{3.031917in}}%
\pgfpathcurveto{\pgfqpoint{2.574132in}{3.026093in}}{\pgfqpoint{2.570860in}{3.018193in}}{\pgfqpoint{2.570860in}{3.009957in}}%
\pgfpathcurveto{\pgfqpoint{2.570860in}{3.001720in}}{\pgfqpoint{2.574132in}{2.993820in}}{\pgfqpoint{2.579956in}{2.987996in}}%
\pgfpathcurveto{\pgfqpoint{2.585780in}{2.982173in}}{\pgfqpoint{2.593680in}{2.978900in}}{\pgfqpoint{2.601916in}{2.978900in}}%
\pgfpathclose%
\pgfusepath{stroke,fill}%
\end{pgfscope}%
\begin{pgfscope}%
\pgfpathrectangle{\pgfqpoint{0.100000in}{0.212622in}}{\pgfqpoint{3.696000in}{3.696000in}}%
\pgfusepath{clip}%
\pgfsetbuttcap%
\pgfsetroundjoin%
\definecolor{currentfill}{rgb}{0.121569,0.466667,0.705882}%
\pgfsetfillcolor{currentfill}%
\pgfsetfillopacity{0.721535}%
\pgfsetlinewidth{1.003750pt}%
\definecolor{currentstroke}{rgb}{0.121569,0.466667,0.705882}%
\pgfsetstrokecolor{currentstroke}%
\pgfsetstrokeopacity{0.721535}%
\pgfsetdash{}{0pt}%
\pgfpathmoveto{\pgfqpoint{2.602224in}{2.978925in}}%
\pgfpathcurveto{\pgfqpoint{2.610461in}{2.978925in}}{\pgfqpoint{2.618361in}{2.982197in}}{\pgfqpoint{2.624184in}{2.988021in}}%
\pgfpathcurveto{\pgfqpoint{2.630008in}{2.993845in}}{\pgfqpoint{2.633281in}{3.001745in}}{\pgfqpoint{2.633281in}{3.009982in}}%
\pgfpathcurveto{\pgfqpoint{2.633281in}{3.018218in}}{\pgfqpoint{2.630008in}{3.026118in}}{\pgfqpoint{2.624184in}{3.031942in}}%
\pgfpathcurveto{\pgfqpoint{2.618361in}{3.037766in}}{\pgfqpoint{2.610461in}{3.041038in}}{\pgfqpoint{2.602224in}{3.041038in}}%
\pgfpathcurveto{\pgfqpoint{2.593988in}{3.041038in}}{\pgfqpoint{2.586088in}{3.037766in}}{\pgfqpoint{2.580264in}{3.031942in}}%
\pgfpathcurveto{\pgfqpoint{2.574440in}{3.026118in}}{\pgfqpoint{2.571168in}{3.018218in}}{\pgfqpoint{2.571168in}{3.009982in}}%
\pgfpathcurveto{\pgfqpoint{2.571168in}{3.001745in}}{\pgfqpoint{2.574440in}{2.993845in}}{\pgfqpoint{2.580264in}{2.988021in}}%
\pgfpathcurveto{\pgfqpoint{2.586088in}{2.982197in}}{\pgfqpoint{2.593988in}{2.978925in}}{\pgfqpoint{2.602224in}{2.978925in}}%
\pgfpathclose%
\pgfusepath{stroke,fill}%
\end{pgfscope}%
\begin{pgfscope}%
\pgfpathrectangle{\pgfqpoint{0.100000in}{0.212622in}}{\pgfqpoint{3.696000in}{3.696000in}}%
\pgfusepath{clip}%
\pgfsetbuttcap%
\pgfsetroundjoin%
\definecolor{currentfill}{rgb}{0.121569,0.466667,0.705882}%
\pgfsetfillcolor{currentfill}%
\pgfsetfillopacity{0.721535}%
\pgfsetlinewidth{1.003750pt}%
\definecolor{currentstroke}{rgb}{0.121569,0.466667,0.705882}%
\pgfsetstrokecolor{currentstroke}%
\pgfsetstrokeopacity{0.721535}%
\pgfsetdash{}{0pt}%
\pgfpathmoveto{\pgfqpoint{2.602224in}{2.978925in}}%
\pgfpathcurveto{\pgfqpoint{2.610461in}{2.978925in}}{\pgfqpoint{2.618361in}{2.982198in}}{\pgfqpoint{2.624185in}{2.988021in}}%
\pgfpathcurveto{\pgfqpoint{2.630009in}{2.993845in}}{\pgfqpoint{2.633281in}{3.001745in}}{\pgfqpoint{2.633281in}{3.009982in}}%
\pgfpathcurveto{\pgfqpoint{2.633281in}{3.018218in}}{\pgfqpoint{2.630009in}{3.026118in}}{\pgfqpoint{2.624185in}{3.031942in}}%
\pgfpathcurveto{\pgfqpoint{2.618361in}{3.037766in}}{\pgfqpoint{2.610461in}{3.041038in}}{\pgfqpoint{2.602224in}{3.041038in}}%
\pgfpathcurveto{\pgfqpoint{2.593988in}{3.041038in}}{\pgfqpoint{2.586088in}{3.037766in}}{\pgfqpoint{2.580264in}{3.031942in}}%
\pgfpathcurveto{\pgfqpoint{2.574440in}{3.026118in}}{\pgfqpoint{2.571168in}{3.018218in}}{\pgfqpoint{2.571168in}{3.009982in}}%
\pgfpathcurveto{\pgfqpoint{2.571168in}{3.001745in}}{\pgfqpoint{2.574440in}{2.993845in}}{\pgfqpoint{2.580264in}{2.988021in}}%
\pgfpathcurveto{\pgfqpoint{2.586088in}{2.982198in}}{\pgfqpoint{2.593988in}{2.978925in}}{\pgfqpoint{2.602224in}{2.978925in}}%
\pgfpathclose%
\pgfusepath{stroke,fill}%
\end{pgfscope}%
\begin{pgfscope}%
\pgfpathrectangle{\pgfqpoint{0.100000in}{0.212622in}}{\pgfqpoint{3.696000in}{3.696000in}}%
\pgfusepath{clip}%
\pgfsetbuttcap%
\pgfsetroundjoin%
\definecolor{currentfill}{rgb}{0.121569,0.466667,0.705882}%
\pgfsetfillcolor{currentfill}%
\pgfsetfillopacity{0.721535}%
\pgfsetlinewidth{1.003750pt}%
\definecolor{currentstroke}{rgb}{0.121569,0.466667,0.705882}%
\pgfsetstrokecolor{currentstroke}%
\pgfsetstrokeopacity{0.721535}%
\pgfsetdash{}{0pt}%
\pgfpathmoveto{\pgfqpoint{2.602225in}{2.978925in}}%
\pgfpathcurveto{\pgfqpoint{2.610461in}{2.978925in}}{\pgfqpoint{2.618361in}{2.982198in}}{\pgfqpoint{2.624185in}{2.988021in}}%
\pgfpathcurveto{\pgfqpoint{2.630009in}{2.993845in}}{\pgfqpoint{2.633281in}{3.001745in}}{\pgfqpoint{2.633281in}{3.009982in}}%
\pgfpathcurveto{\pgfqpoint{2.633281in}{3.018218in}}{\pgfqpoint{2.630009in}{3.026118in}}{\pgfqpoint{2.624185in}{3.031942in}}%
\pgfpathcurveto{\pgfqpoint{2.618361in}{3.037766in}}{\pgfqpoint{2.610461in}{3.041038in}}{\pgfqpoint{2.602225in}{3.041038in}}%
\pgfpathcurveto{\pgfqpoint{2.593988in}{3.041038in}}{\pgfqpoint{2.586088in}{3.037766in}}{\pgfqpoint{2.580264in}{3.031942in}}%
\pgfpathcurveto{\pgfqpoint{2.574440in}{3.026118in}}{\pgfqpoint{2.571168in}{3.018218in}}{\pgfqpoint{2.571168in}{3.009982in}}%
\pgfpathcurveto{\pgfqpoint{2.571168in}{3.001745in}}{\pgfqpoint{2.574440in}{2.993845in}}{\pgfqpoint{2.580264in}{2.988021in}}%
\pgfpathcurveto{\pgfqpoint{2.586088in}{2.982198in}}{\pgfqpoint{2.593988in}{2.978925in}}{\pgfqpoint{2.602225in}{2.978925in}}%
\pgfpathclose%
\pgfusepath{stroke,fill}%
\end{pgfscope}%
\begin{pgfscope}%
\pgfpathrectangle{\pgfqpoint{0.100000in}{0.212622in}}{\pgfqpoint{3.696000in}{3.696000in}}%
\pgfusepath{clip}%
\pgfsetbuttcap%
\pgfsetroundjoin%
\definecolor{currentfill}{rgb}{0.121569,0.466667,0.705882}%
\pgfsetfillcolor{currentfill}%
\pgfsetfillopacity{0.721535}%
\pgfsetlinewidth{1.003750pt}%
\definecolor{currentstroke}{rgb}{0.121569,0.466667,0.705882}%
\pgfsetstrokecolor{currentstroke}%
\pgfsetstrokeopacity{0.721535}%
\pgfsetdash{}{0pt}%
\pgfpathmoveto{\pgfqpoint{2.602225in}{2.978925in}}%
\pgfpathcurveto{\pgfqpoint{2.610461in}{2.978925in}}{\pgfqpoint{2.618361in}{2.982198in}}{\pgfqpoint{2.624185in}{2.988021in}}%
\pgfpathcurveto{\pgfqpoint{2.630009in}{2.993845in}}{\pgfqpoint{2.633281in}{3.001745in}}{\pgfqpoint{2.633281in}{3.009982in}}%
\pgfpathcurveto{\pgfqpoint{2.633281in}{3.018218in}}{\pgfqpoint{2.630009in}{3.026118in}}{\pgfqpoint{2.624185in}{3.031942in}}%
\pgfpathcurveto{\pgfqpoint{2.618361in}{3.037766in}}{\pgfqpoint{2.610461in}{3.041038in}}{\pgfqpoint{2.602225in}{3.041038in}}%
\pgfpathcurveto{\pgfqpoint{2.593989in}{3.041038in}}{\pgfqpoint{2.586089in}{3.037766in}}{\pgfqpoint{2.580265in}{3.031942in}}%
\pgfpathcurveto{\pgfqpoint{2.574441in}{3.026118in}}{\pgfqpoint{2.571168in}{3.018218in}}{\pgfqpoint{2.571168in}{3.009982in}}%
\pgfpathcurveto{\pgfqpoint{2.571168in}{3.001745in}}{\pgfqpoint{2.574441in}{2.993845in}}{\pgfqpoint{2.580265in}{2.988021in}}%
\pgfpathcurveto{\pgfqpoint{2.586089in}{2.982198in}}{\pgfqpoint{2.593989in}{2.978925in}}{\pgfqpoint{2.602225in}{2.978925in}}%
\pgfpathclose%
\pgfusepath{stroke,fill}%
\end{pgfscope}%
\begin{pgfscope}%
\pgfpathrectangle{\pgfqpoint{0.100000in}{0.212622in}}{\pgfqpoint{3.696000in}{3.696000in}}%
\pgfusepath{clip}%
\pgfsetbuttcap%
\pgfsetroundjoin%
\definecolor{currentfill}{rgb}{0.121569,0.466667,0.705882}%
\pgfsetfillcolor{currentfill}%
\pgfsetfillopacity{0.721535}%
\pgfsetlinewidth{1.003750pt}%
\definecolor{currentstroke}{rgb}{0.121569,0.466667,0.705882}%
\pgfsetstrokecolor{currentstroke}%
\pgfsetstrokeopacity{0.721535}%
\pgfsetdash{}{0pt}%
\pgfpathmoveto{\pgfqpoint{2.602225in}{2.978925in}}%
\pgfpathcurveto{\pgfqpoint{2.610462in}{2.978925in}}{\pgfqpoint{2.618362in}{2.982198in}}{\pgfqpoint{2.624186in}{2.988022in}}%
\pgfpathcurveto{\pgfqpoint{2.630010in}{2.993845in}}{\pgfqpoint{2.633282in}{3.001746in}}{\pgfqpoint{2.633282in}{3.009982in}}%
\pgfpathcurveto{\pgfqpoint{2.633282in}{3.018218in}}{\pgfqpoint{2.630010in}{3.026118in}}{\pgfqpoint{2.624186in}{3.031942in}}%
\pgfpathcurveto{\pgfqpoint{2.618362in}{3.037766in}}{\pgfqpoint{2.610462in}{3.041038in}}{\pgfqpoint{2.602225in}{3.041038in}}%
\pgfpathcurveto{\pgfqpoint{2.593989in}{3.041038in}}{\pgfqpoint{2.586089in}{3.037766in}}{\pgfqpoint{2.580265in}{3.031942in}}%
\pgfpathcurveto{\pgfqpoint{2.574441in}{3.026118in}}{\pgfqpoint{2.571169in}{3.018218in}}{\pgfqpoint{2.571169in}{3.009982in}}%
\pgfpathcurveto{\pgfqpoint{2.571169in}{3.001746in}}{\pgfqpoint{2.574441in}{2.993845in}}{\pgfqpoint{2.580265in}{2.988022in}}%
\pgfpathcurveto{\pgfqpoint{2.586089in}{2.982198in}}{\pgfqpoint{2.593989in}{2.978925in}}{\pgfqpoint{2.602225in}{2.978925in}}%
\pgfpathclose%
\pgfusepath{stroke,fill}%
\end{pgfscope}%
\begin{pgfscope}%
\pgfpathrectangle{\pgfqpoint{0.100000in}{0.212622in}}{\pgfqpoint{3.696000in}{3.696000in}}%
\pgfusepath{clip}%
\pgfsetbuttcap%
\pgfsetroundjoin%
\definecolor{currentfill}{rgb}{0.121569,0.466667,0.705882}%
\pgfsetfillcolor{currentfill}%
\pgfsetfillopacity{0.721536}%
\pgfsetlinewidth{1.003750pt}%
\definecolor{currentstroke}{rgb}{0.121569,0.466667,0.705882}%
\pgfsetstrokecolor{currentstroke}%
\pgfsetstrokeopacity{0.721536}%
\pgfsetdash{}{0pt}%
\pgfpathmoveto{\pgfqpoint{2.602226in}{2.978925in}}%
\pgfpathcurveto{\pgfqpoint{2.610463in}{2.978925in}}{\pgfqpoint{2.618363in}{2.982198in}}{\pgfqpoint{2.624187in}{2.988022in}}%
\pgfpathcurveto{\pgfqpoint{2.630011in}{2.993846in}}{\pgfqpoint{2.633283in}{3.001746in}}{\pgfqpoint{2.633283in}{3.009982in}}%
\pgfpathcurveto{\pgfqpoint{2.633283in}{3.018218in}}{\pgfqpoint{2.630011in}{3.026118in}}{\pgfqpoint{2.624187in}{3.031942in}}%
\pgfpathcurveto{\pgfqpoint{2.618363in}{3.037766in}}{\pgfqpoint{2.610463in}{3.041038in}}{\pgfqpoint{2.602226in}{3.041038in}}%
\pgfpathcurveto{\pgfqpoint{2.593990in}{3.041038in}}{\pgfqpoint{2.586090in}{3.037766in}}{\pgfqpoint{2.580266in}{3.031942in}}%
\pgfpathcurveto{\pgfqpoint{2.574442in}{3.026118in}}{\pgfqpoint{2.571170in}{3.018218in}}{\pgfqpoint{2.571170in}{3.009982in}}%
\pgfpathcurveto{\pgfqpoint{2.571170in}{3.001746in}}{\pgfqpoint{2.574442in}{2.993846in}}{\pgfqpoint{2.580266in}{2.988022in}}%
\pgfpathcurveto{\pgfqpoint{2.586090in}{2.982198in}}{\pgfqpoint{2.593990in}{2.978925in}}{\pgfqpoint{2.602226in}{2.978925in}}%
\pgfpathclose%
\pgfusepath{stroke,fill}%
\end{pgfscope}%
\begin{pgfscope}%
\pgfpathrectangle{\pgfqpoint{0.100000in}{0.212622in}}{\pgfqpoint{3.696000in}{3.696000in}}%
\pgfusepath{clip}%
\pgfsetbuttcap%
\pgfsetroundjoin%
\definecolor{currentfill}{rgb}{0.121569,0.466667,0.705882}%
\pgfsetfillcolor{currentfill}%
\pgfsetfillopacity{0.721536}%
\pgfsetlinewidth{1.003750pt}%
\definecolor{currentstroke}{rgb}{0.121569,0.466667,0.705882}%
\pgfsetstrokecolor{currentstroke}%
\pgfsetstrokeopacity{0.721536}%
\pgfsetdash{}{0pt}%
\pgfpathmoveto{\pgfqpoint{2.602228in}{2.978926in}}%
\pgfpathcurveto{\pgfqpoint{2.610464in}{2.978926in}}{\pgfqpoint{2.618364in}{2.982198in}}{\pgfqpoint{2.624188in}{2.988022in}}%
\pgfpathcurveto{\pgfqpoint{2.630012in}{2.993846in}}{\pgfqpoint{2.633284in}{3.001746in}}{\pgfqpoint{2.633284in}{3.009982in}}%
\pgfpathcurveto{\pgfqpoint{2.633284in}{3.018219in}}{\pgfqpoint{2.630012in}{3.026119in}}{\pgfqpoint{2.624188in}{3.031942in}}%
\pgfpathcurveto{\pgfqpoint{2.618364in}{3.037766in}}{\pgfqpoint{2.610464in}{3.041039in}}{\pgfqpoint{2.602228in}{3.041039in}}%
\pgfpathcurveto{\pgfqpoint{2.593992in}{3.041039in}}{\pgfqpoint{2.586092in}{3.037766in}}{\pgfqpoint{2.580268in}{3.031942in}}%
\pgfpathcurveto{\pgfqpoint{2.574444in}{3.026119in}}{\pgfqpoint{2.571171in}{3.018219in}}{\pgfqpoint{2.571171in}{3.009982in}}%
\pgfpathcurveto{\pgfqpoint{2.571171in}{3.001746in}}{\pgfqpoint{2.574444in}{2.993846in}}{\pgfqpoint{2.580268in}{2.988022in}}%
\pgfpathcurveto{\pgfqpoint{2.586092in}{2.982198in}}{\pgfqpoint{2.593992in}{2.978926in}}{\pgfqpoint{2.602228in}{2.978926in}}%
\pgfpathclose%
\pgfusepath{stroke,fill}%
\end{pgfscope}%
\begin{pgfscope}%
\pgfpathrectangle{\pgfqpoint{0.100000in}{0.212622in}}{\pgfqpoint{3.696000in}{3.696000in}}%
\pgfusepath{clip}%
\pgfsetbuttcap%
\pgfsetroundjoin%
\definecolor{currentfill}{rgb}{0.121569,0.466667,0.705882}%
\pgfsetfillcolor{currentfill}%
\pgfsetfillopacity{0.721538}%
\pgfsetlinewidth{1.003750pt}%
\definecolor{currentstroke}{rgb}{0.121569,0.466667,0.705882}%
\pgfsetstrokecolor{currentstroke}%
\pgfsetstrokeopacity{0.721538}%
\pgfsetdash{}{0pt}%
\pgfpathmoveto{\pgfqpoint{2.602230in}{2.978926in}}%
\pgfpathcurveto{\pgfqpoint{2.610467in}{2.978926in}}{\pgfqpoint{2.618367in}{2.982198in}}{\pgfqpoint{2.624191in}{2.988022in}}%
\pgfpathcurveto{\pgfqpoint{2.630015in}{2.993846in}}{\pgfqpoint{2.633287in}{3.001746in}}{\pgfqpoint{2.633287in}{3.009983in}}%
\pgfpathcurveto{\pgfqpoint{2.633287in}{3.018219in}}{\pgfqpoint{2.630015in}{3.026119in}}{\pgfqpoint{2.624191in}{3.031943in}}%
\pgfpathcurveto{\pgfqpoint{2.618367in}{3.037767in}}{\pgfqpoint{2.610467in}{3.041039in}}{\pgfqpoint{2.602230in}{3.041039in}}%
\pgfpathcurveto{\pgfqpoint{2.593994in}{3.041039in}}{\pgfqpoint{2.586094in}{3.037767in}}{\pgfqpoint{2.580270in}{3.031943in}}%
\pgfpathcurveto{\pgfqpoint{2.574446in}{3.026119in}}{\pgfqpoint{2.571174in}{3.018219in}}{\pgfqpoint{2.571174in}{3.009983in}}%
\pgfpathcurveto{\pgfqpoint{2.571174in}{3.001746in}}{\pgfqpoint{2.574446in}{2.993846in}}{\pgfqpoint{2.580270in}{2.988022in}}%
\pgfpathcurveto{\pgfqpoint{2.586094in}{2.982198in}}{\pgfqpoint{2.593994in}{2.978926in}}{\pgfqpoint{2.602230in}{2.978926in}}%
\pgfpathclose%
\pgfusepath{stroke,fill}%
\end{pgfscope}%
\begin{pgfscope}%
\pgfpathrectangle{\pgfqpoint{0.100000in}{0.212622in}}{\pgfqpoint{3.696000in}{3.696000in}}%
\pgfusepath{clip}%
\pgfsetbuttcap%
\pgfsetroundjoin%
\definecolor{currentfill}{rgb}{0.121569,0.466667,0.705882}%
\pgfsetfillcolor{currentfill}%
\pgfsetfillopacity{0.721540}%
\pgfsetlinewidth{1.003750pt}%
\definecolor{currentstroke}{rgb}{0.121569,0.466667,0.705882}%
\pgfsetstrokecolor{currentstroke}%
\pgfsetstrokeopacity{0.721540}%
\pgfsetdash{}{0pt}%
\pgfpathmoveto{\pgfqpoint{2.602234in}{2.978927in}}%
\pgfpathcurveto{\pgfqpoint{2.610471in}{2.978927in}}{\pgfqpoint{2.618371in}{2.982199in}}{\pgfqpoint{2.624195in}{2.988023in}}%
\pgfpathcurveto{\pgfqpoint{2.630019in}{2.993847in}}{\pgfqpoint{2.633291in}{3.001747in}}{\pgfqpoint{2.633291in}{3.009983in}}%
\pgfpathcurveto{\pgfqpoint{2.633291in}{3.018220in}}{\pgfqpoint{2.630019in}{3.026120in}}{\pgfqpoint{2.624195in}{3.031944in}}%
\pgfpathcurveto{\pgfqpoint{2.618371in}{3.037768in}}{\pgfqpoint{2.610471in}{3.041040in}}{\pgfqpoint{2.602234in}{3.041040in}}%
\pgfpathcurveto{\pgfqpoint{2.593998in}{3.041040in}}{\pgfqpoint{2.586098in}{3.037768in}}{\pgfqpoint{2.580274in}{3.031944in}}%
\pgfpathcurveto{\pgfqpoint{2.574450in}{3.026120in}}{\pgfqpoint{2.571178in}{3.018220in}}{\pgfqpoint{2.571178in}{3.009983in}}%
\pgfpathcurveto{\pgfqpoint{2.571178in}{3.001747in}}{\pgfqpoint{2.574450in}{2.993847in}}{\pgfqpoint{2.580274in}{2.988023in}}%
\pgfpathcurveto{\pgfqpoint{2.586098in}{2.982199in}}{\pgfqpoint{2.593998in}{2.978927in}}{\pgfqpoint{2.602234in}{2.978927in}}%
\pgfpathclose%
\pgfusepath{stroke,fill}%
\end{pgfscope}%
\begin{pgfscope}%
\pgfpathrectangle{\pgfqpoint{0.100000in}{0.212622in}}{\pgfqpoint{3.696000in}{3.696000in}}%
\pgfusepath{clip}%
\pgfsetbuttcap%
\pgfsetroundjoin%
\definecolor{currentfill}{rgb}{0.121569,0.466667,0.705882}%
\pgfsetfillcolor{currentfill}%
\pgfsetfillopacity{0.721545}%
\pgfsetlinewidth{1.003750pt}%
\definecolor{currentstroke}{rgb}{0.121569,0.466667,0.705882}%
\pgfsetstrokecolor{currentstroke}%
\pgfsetstrokeopacity{0.721545}%
\pgfsetdash{}{0pt}%
\pgfpathmoveto{\pgfqpoint{2.602241in}{2.978928in}}%
\pgfpathcurveto{\pgfqpoint{2.610477in}{2.978928in}}{\pgfqpoint{2.618377in}{2.982200in}}{\pgfqpoint{2.624201in}{2.988024in}}%
\pgfpathcurveto{\pgfqpoint{2.630025in}{2.993848in}}{\pgfqpoint{2.633297in}{3.001748in}}{\pgfqpoint{2.633297in}{3.009984in}}%
\pgfpathcurveto{\pgfqpoint{2.633297in}{3.018221in}}{\pgfqpoint{2.630025in}{3.026121in}}{\pgfqpoint{2.624201in}{3.031945in}}%
\pgfpathcurveto{\pgfqpoint{2.618377in}{3.037769in}}{\pgfqpoint{2.610477in}{3.041041in}}{\pgfqpoint{2.602241in}{3.041041in}}%
\pgfpathcurveto{\pgfqpoint{2.594004in}{3.041041in}}{\pgfqpoint{2.586104in}{3.037769in}}{\pgfqpoint{2.580280in}{3.031945in}}%
\pgfpathcurveto{\pgfqpoint{2.574456in}{3.026121in}}{\pgfqpoint{2.571184in}{3.018221in}}{\pgfqpoint{2.571184in}{3.009984in}}%
\pgfpathcurveto{\pgfqpoint{2.571184in}{3.001748in}}{\pgfqpoint{2.574456in}{2.993848in}}{\pgfqpoint{2.580280in}{2.988024in}}%
\pgfpathcurveto{\pgfqpoint{2.586104in}{2.982200in}}{\pgfqpoint{2.594004in}{2.978928in}}{\pgfqpoint{2.602241in}{2.978928in}}%
\pgfpathclose%
\pgfusepath{stroke,fill}%
\end{pgfscope}%
\begin{pgfscope}%
\pgfpathrectangle{\pgfqpoint{0.100000in}{0.212622in}}{\pgfqpoint{3.696000in}{3.696000in}}%
\pgfusepath{clip}%
\pgfsetbuttcap%
\pgfsetroundjoin%
\definecolor{currentfill}{rgb}{0.121569,0.466667,0.705882}%
\pgfsetfillcolor{currentfill}%
\pgfsetfillopacity{0.721554}%
\pgfsetlinewidth{1.003750pt}%
\definecolor{currentstroke}{rgb}{0.121569,0.466667,0.705882}%
\pgfsetstrokecolor{currentstroke}%
\pgfsetstrokeopacity{0.721554}%
\pgfsetdash{}{0pt}%
\pgfpathmoveto{\pgfqpoint{2.602249in}{2.978929in}}%
\pgfpathcurveto{\pgfqpoint{2.610486in}{2.978929in}}{\pgfqpoint{2.618386in}{2.982201in}}{\pgfqpoint{2.624210in}{2.988025in}}%
\pgfpathcurveto{\pgfqpoint{2.630034in}{2.993849in}}{\pgfqpoint{2.633306in}{3.001749in}}{\pgfqpoint{2.633306in}{3.009985in}}%
\pgfpathcurveto{\pgfqpoint{2.633306in}{3.018222in}}{\pgfqpoint{2.630034in}{3.026122in}}{\pgfqpoint{2.624210in}{3.031946in}}%
\pgfpathcurveto{\pgfqpoint{2.618386in}{3.037770in}}{\pgfqpoint{2.610486in}{3.041042in}}{\pgfqpoint{2.602249in}{3.041042in}}%
\pgfpathcurveto{\pgfqpoint{2.594013in}{3.041042in}}{\pgfqpoint{2.586113in}{3.037770in}}{\pgfqpoint{2.580289in}{3.031946in}}%
\pgfpathcurveto{\pgfqpoint{2.574465in}{3.026122in}}{\pgfqpoint{2.571193in}{3.018222in}}{\pgfqpoint{2.571193in}{3.009985in}}%
\pgfpathcurveto{\pgfqpoint{2.571193in}{3.001749in}}{\pgfqpoint{2.574465in}{2.993849in}}{\pgfqpoint{2.580289in}{2.988025in}}%
\pgfpathcurveto{\pgfqpoint{2.586113in}{2.982201in}}{\pgfqpoint{2.594013in}{2.978929in}}{\pgfqpoint{2.602249in}{2.978929in}}%
\pgfpathclose%
\pgfusepath{stroke,fill}%
\end{pgfscope}%
\begin{pgfscope}%
\pgfpathrectangle{\pgfqpoint{0.100000in}{0.212622in}}{\pgfqpoint{3.696000in}{3.696000in}}%
\pgfusepath{clip}%
\pgfsetbuttcap%
\pgfsetroundjoin%
\definecolor{currentfill}{rgb}{0.121569,0.466667,0.705882}%
\pgfsetfillcolor{currentfill}%
\pgfsetfillopacity{0.721572}%
\pgfsetlinewidth{1.003750pt}%
\definecolor{currentstroke}{rgb}{0.121569,0.466667,0.705882}%
\pgfsetstrokecolor{currentstroke}%
\pgfsetstrokeopacity{0.721572}%
\pgfsetdash{}{0pt}%
\pgfpathmoveto{\pgfqpoint{2.602262in}{2.978930in}}%
\pgfpathcurveto{\pgfqpoint{2.610498in}{2.978930in}}{\pgfqpoint{2.618398in}{2.982202in}}{\pgfqpoint{2.624222in}{2.988026in}}%
\pgfpathcurveto{\pgfqpoint{2.630046in}{2.993850in}}{\pgfqpoint{2.633318in}{3.001750in}}{\pgfqpoint{2.633318in}{3.009986in}}%
\pgfpathcurveto{\pgfqpoint{2.633318in}{3.018222in}}{\pgfqpoint{2.630046in}{3.026122in}}{\pgfqpoint{2.624222in}{3.031946in}}%
\pgfpathcurveto{\pgfqpoint{2.618398in}{3.037770in}}{\pgfqpoint{2.610498in}{3.041043in}}{\pgfqpoint{2.602262in}{3.041043in}}%
\pgfpathcurveto{\pgfqpoint{2.594026in}{3.041043in}}{\pgfqpoint{2.586126in}{3.037770in}}{\pgfqpoint{2.580302in}{3.031946in}}%
\pgfpathcurveto{\pgfqpoint{2.574478in}{3.026122in}}{\pgfqpoint{2.571205in}{3.018222in}}{\pgfqpoint{2.571205in}{3.009986in}}%
\pgfpathcurveto{\pgfqpoint{2.571205in}{3.001750in}}{\pgfqpoint{2.574478in}{2.993850in}}{\pgfqpoint{2.580302in}{2.988026in}}%
\pgfpathcurveto{\pgfqpoint{2.586126in}{2.982202in}}{\pgfqpoint{2.594026in}{2.978930in}}{\pgfqpoint{2.602262in}{2.978930in}}%
\pgfpathclose%
\pgfusepath{stroke,fill}%
\end{pgfscope}%
\begin{pgfscope}%
\pgfpathrectangle{\pgfqpoint{0.100000in}{0.212622in}}{\pgfqpoint{3.696000in}{3.696000in}}%
\pgfusepath{clip}%
\pgfsetbuttcap%
\pgfsetroundjoin%
\definecolor{currentfill}{rgb}{0.121569,0.466667,0.705882}%
\pgfsetfillcolor{currentfill}%
\pgfsetfillopacity{0.721604}%
\pgfsetlinewidth{1.003750pt}%
\definecolor{currentstroke}{rgb}{0.121569,0.466667,0.705882}%
\pgfsetstrokecolor{currentstroke}%
\pgfsetstrokeopacity{0.721604}%
\pgfsetdash{}{0pt}%
\pgfpathmoveto{\pgfqpoint{2.602277in}{2.978927in}}%
\pgfpathcurveto{\pgfqpoint{2.610513in}{2.978927in}}{\pgfqpoint{2.618413in}{2.982199in}}{\pgfqpoint{2.624237in}{2.988023in}}%
\pgfpathcurveto{\pgfqpoint{2.630061in}{2.993847in}}{\pgfqpoint{2.633333in}{3.001747in}}{\pgfqpoint{2.633333in}{3.009983in}}%
\pgfpathcurveto{\pgfqpoint{2.633333in}{3.018220in}}{\pgfqpoint{2.630061in}{3.026120in}}{\pgfqpoint{2.624237in}{3.031944in}}%
\pgfpathcurveto{\pgfqpoint{2.618413in}{3.037768in}}{\pgfqpoint{2.610513in}{3.041040in}}{\pgfqpoint{2.602277in}{3.041040in}}%
\pgfpathcurveto{\pgfqpoint{2.594041in}{3.041040in}}{\pgfqpoint{2.586141in}{3.037768in}}{\pgfqpoint{2.580317in}{3.031944in}}%
\pgfpathcurveto{\pgfqpoint{2.574493in}{3.026120in}}{\pgfqpoint{2.571220in}{3.018220in}}{\pgfqpoint{2.571220in}{3.009983in}}%
\pgfpathcurveto{\pgfqpoint{2.571220in}{3.001747in}}{\pgfqpoint{2.574493in}{2.993847in}}{\pgfqpoint{2.580317in}{2.988023in}}%
\pgfpathcurveto{\pgfqpoint{2.586141in}{2.982199in}}{\pgfqpoint{2.594041in}{2.978927in}}{\pgfqpoint{2.602277in}{2.978927in}}%
\pgfpathclose%
\pgfusepath{stroke,fill}%
\end{pgfscope}%
\begin{pgfscope}%
\pgfpathrectangle{\pgfqpoint{0.100000in}{0.212622in}}{\pgfqpoint{3.696000in}{3.696000in}}%
\pgfusepath{clip}%
\pgfsetbuttcap%
\pgfsetroundjoin%
\definecolor{currentfill}{rgb}{0.121569,0.466667,0.705882}%
\pgfsetfillcolor{currentfill}%
\pgfsetfillopacity{0.721665}%
\pgfsetlinewidth{1.003750pt}%
\definecolor{currentstroke}{rgb}{0.121569,0.466667,0.705882}%
\pgfsetstrokecolor{currentstroke}%
\pgfsetstrokeopacity{0.721665}%
\pgfsetdash{}{0pt}%
\pgfpathmoveto{\pgfqpoint{2.602291in}{2.978916in}}%
\pgfpathcurveto{\pgfqpoint{2.610527in}{2.978916in}}{\pgfqpoint{2.618427in}{2.982188in}}{\pgfqpoint{2.624251in}{2.988012in}}%
\pgfpathcurveto{\pgfqpoint{2.630075in}{2.993836in}}{\pgfqpoint{2.633347in}{3.001736in}}{\pgfqpoint{2.633347in}{3.009973in}}%
\pgfpathcurveto{\pgfqpoint{2.633347in}{3.018209in}}{\pgfqpoint{2.630075in}{3.026109in}}{\pgfqpoint{2.624251in}{3.031933in}}%
\pgfpathcurveto{\pgfqpoint{2.618427in}{3.037757in}}{\pgfqpoint{2.610527in}{3.041029in}}{\pgfqpoint{2.602291in}{3.041029in}}%
\pgfpathcurveto{\pgfqpoint{2.594054in}{3.041029in}}{\pgfqpoint{2.586154in}{3.037757in}}{\pgfqpoint{2.580331in}{3.031933in}}%
\pgfpathcurveto{\pgfqpoint{2.574507in}{3.026109in}}{\pgfqpoint{2.571234in}{3.018209in}}{\pgfqpoint{2.571234in}{3.009973in}}%
\pgfpathcurveto{\pgfqpoint{2.571234in}{3.001736in}}{\pgfqpoint{2.574507in}{2.993836in}}{\pgfqpoint{2.580331in}{2.988012in}}%
\pgfpathcurveto{\pgfqpoint{2.586154in}{2.982188in}}{\pgfqpoint{2.594054in}{2.978916in}}{\pgfqpoint{2.602291in}{2.978916in}}%
\pgfpathclose%
\pgfusepath{stroke,fill}%
\end{pgfscope}%
\begin{pgfscope}%
\pgfpathrectangle{\pgfqpoint{0.100000in}{0.212622in}}{\pgfqpoint{3.696000in}{3.696000in}}%
\pgfusepath{clip}%
\pgfsetbuttcap%
\pgfsetroundjoin%
\definecolor{currentfill}{rgb}{0.121569,0.466667,0.705882}%
\pgfsetfillcolor{currentfill}%
\pgfsetfillopacity{0.721756}%
\pgfsetlinewidth{1.003750pt}%
\definecolor{currentstroke}{rgb}{0.121569,0.466667,0.705882}%
\pgfsetstrokecolor{currentstroke}%
\pgfsetstrokeopacity{0.721756}%
\pgfsetdash{}{0pt}%
\pgfpathmoveto{\pgfqpoint{1.404342in}{2.670479in}}%
\pgfpathcurveto{\pgfqpoint{1.412578in}{2.670479in}}{\pgfqpoint{1.420478in}{2.673751in}}{\pgfqpoint{1.426302in}{2.679575in}}%
\pgfpathcurveto{\pgfqpoint{1.432126in}{2.685399in}}{\pgfqpoint{1.435398in}{2.693299in}}{\pgfqpoint{1.435398in}{2.701535in}}%
\pgfpathcurveto{\pgfqpoint{1.435398in}{2.709771in}}{\pgfqpoint{1.432126in}{2.717671in}}{\pgfqpoint{1.426302in}{2.723495in}}%
\pgfpathcurveto{\pgfqpoint{1.420478in}{2.729319in}}{\pgfqpoint{1.412578in}{2.732592in}}{\pgfqpoint{1.404342in}{2.732592in}}%
\pgfpathcurveto{\pgfqpoint{1.396106in}{2.732592in}}{\pgfqpoint{1.388205in}{2.729319in}}{\pgfqpoint{1.382382in}{2.723495in}}%
\pgfpathcurveto{\pgfqpoint{1.376558in}{2.717671in}}{\pgfqpoint{1.373285in}{2.709771in}}{\pgfqpoint{1.373285in}{2.701535in}}%
\pgfpathcurveto{\pgfqpoint{1.373285in}{2.693299in}}{\pgfqpoint{1.376558in}{2.685399in}}{\pgfqpoint{1.382382in}{2.679575in}}%
\pgfpathcurveto{\pgfqpoint{1.388205in}{2.673751in}}{\pgfqpoint{1.396106in}{2.670479in}}{\pgfqpoint{1.404342in}{2.670479in}}%
\pgfpathclose%
\pgfusepath{stroke,fill}%
\end{pgfscope}%
\begin{pgfscope}%
\pgfpathrectangle{\pgfqpoint{0.100000in}{0.212622in}}{\pgfqpoint{3.696000in}{3.696000in}}%
\pgfusepath{clip}%
\pgfsetbuttcap%
\pgfsetroundjoin%
\definecolor{currentfill}{rgb}{0.121569,0.466667,0.705882}%
\pgfsetfillcolor{currentfill}%
\pgfsetfillopacity{0.721775}%
\pgfsetlinewidth{1.003750pt}%
\definecolor{currentstroke}{rgb}{0.121569,0.466667,0.705882}%
\pgfsetstrokecolor{currentstroke}%
\pgfsetstrokeopacity{0.721775}%
\pgfsetdash{}{0pt}%
\pgfpathmoveto{\pgfqpoint{2.602291in}{2.978877in}}%
\pgfpathcurveto{\pgfqpoint{2.610528in}{2.978877in}}{\pgfqpoint{2.618428in}{2.982149in}}{\pgfqpoint{2.624252in}{2.987973in}}%
\pgfpathcurveto{\pgfqpoint{2.630075in}{2.993797in}}{\pgfqpoint{2.633348in}{3.001697in}}{\pgfqpoint{2.633348in}{3.009933in}}%
\pgfpathcurveto{\pgfqpoint{2.633348in}{3.018169in}}{\pgfqpoint{2.630075in}{3.026069in}}{\pgfqpoint{2.624252in}{3.031893in}}%
\pgfpathcurveto{\pgfqpoint{2.618428in}{3.037717in}}{\pgfqpoint{2.610528in}{3.040990in}}{\pgfqpoint{2.602291in}{3.040990in}}%
\pgfpathcurveto{\pgfqpoint{2.594055in}{3.040990in}}{\pgfqpoint{2.586155in}{3.037717in}}{\pgfqpoint{2.580331in}{3.031893in}}%
\pgfpathcurveto{\pgfqpoint{2.574507in}{3.026069in}}{\pgfqpoint{2.571235in}{3.018169in}}{\pgfqpoint{2.571235in}{3.009933in}}%
\pgfpathcurveto{\pgfqpoint{2.571235in}{3.001697in}}{\pgfqpoint{2.574507in}{2.993797in}}{\pgfqpoint{2.580331in}{2.987973in}}%
\pgfpathcurveto{\pgfqpoint{2.586155in}{2.982149in}}{\pgfqpoint{2.594055in}{2.978877in}}{\pgfqpoint{2.602291in}{2.978877in}}%
\pgfpathclose%
\pgfusepath{stroke,fill}%
\end{pgfscope}%
\begin{pgfscope}%
\pgfpathrectangle{\pgfqpoint{0.100000in}{0.212622in}}{\pgfqpoint{3.696000in}{3.696000in}}%
\pgfusepath{clip}%
\pgfsetbuttcap%
\pgfsetroundjoin%
\definecolor{currentfill}{rgb}{0.121569,0.466667,0.705882}%
\pgfsetfillcolor{currentfill}%
\pgfsetfillopacity{0.721975}%
\pgfsetlinewidth{1.003750pt}%
\definecolor{currentstroke}{rgb}{0.121569,0.466667,0.705882}%
\pgfsetstrokecolor{currentstroke}%
\pgfsetstrokeopacity{0.721975}%
\pgfsetdash{}{0pt}%
\pgfpathmoveto{\pgfqpoint{2.602247in}{2.978780in}}%
\pgfpathcurveto{\pgfqpoint{2.610483in}{2.978780in}}{\pgfqpoint{2.618383in}{2.982052in}}{\pgfqpoint{2.624207in}{2.987876in}}%
\pgfpathcurveto{\pgfqpoint{2.630031in}{2.993700in}}{\pgfqpoint{2.633303in}{3.001600in}}{\pgfqpoint{2.633303in}{3.009837in}}%
\pgfpathcurveto{\pgfqpoint{2.633303in}{3.018073in}}{\pgfqpoint{2.630031in}{3.025973in}}{\pgfqpoint{2.624207in}{3.031797in}}%
\pgfpathcurveto{\pgfqpoint{2.618383in}{3.037621in}}{\pgfqpoint{2.610483in}{3.040893in}}{\pgfqpoint{2.602247in}{3.040893in}}%
\pgfpathcurveto{\pgfqpoint{2.594010in}{3.040893in}}{\pgfqpoint{2.586110in}{3.037621in}}{\pgfqpoint{2.580286in}{3.031797in}}%
\pgfpathcurveto{\pgfqpoint{2.574462in}{3.025973in}}{\pgfqpoint{2.571190in}{3.018073in}}{\pgfqpoint{2.571190in}{3.009837in}}%
\pgfpathcurveto{\pgfqpoint{2.571190in}{3.001600in}}{\pgfqpoint{2.574462in}{2.993700in}}{\pgfqpoint{2.580286in}{2.987876in}}%
\pgfpathcurveto{\pgfqpoint{2.586110in}{2.982052in}}{\pgfqpoint{2.594010in}{2.978780in}}{\pgfqpoint{2.602247in}{2.978780in}}%
\pgfpathclose%
\pgfusepath{stroke,fill}%
\end{pgfscope}%
\begin{pgfscope}%
\pgfpathrectangle{\pgfqpoint{0.100000in}{0.212622in}}{\pgfqpoint{3.696000in}{3.696000in}}%
\pgfusepath{clip}%
\pgfsetbuttcap%
\pgfsetroundjoin%
\definecolor{currentfill}{rgb}{0.121569,0.466667,0.705882}%
\pgfsetfillcolor{currentfill}%
\pgfsetfillopacity{0.722028}%
\pgfsetlinewidth{1.003750pt}%
\definecolor{currentstroke}{rgb}{0.121569,0.466667,0.705882}%
\pgfsetstrokecolor{currentstroke}%
\pgfsetstrokeopacity{0.722028}%
\pgfsetdash{}{0pt}%
\pgfpathmoveto{\pgfqpoint{2.602224in}{2.978747in}}%
\pgfpathcurveto{\pgfqpoint{2.610460in}{2.978747in}}{\pgfqpoint{2.618360in}{2.982019in}}{\pgfqpoint{2.624184in}{2.987843in}}%
\pgfpathcurveto{\pgfqpoint{2.630008in}{2.993667in}}{\pgfqpoint{2.633280in}{3.001567in}}{\pgfqpoint{2.633280in}{3.009803in}}%
\pgfpathcurveto{\pgfqpoint{2.633280in}{3.018040in}}{\pgfqpoint{2.630008in}{3.025940in}}{\pgfqpoint{2.624184in}{3.031764in}}%
\pgfpathcurveto{\pgfqpoint{2.618360in}{3.037588in}}{\pgfqpoint{2.610460in}{3.040860in}}{\pgfqpoint{2.602224in}{3.040860in}}%
\pgfpathcurveto{\pgfqpoint{2.593987in}{3.040860in}}{\pgfqpoint{2.586087in}{3.037588in}}{\pgfqpoint{2.580263in}{3.031764in}}%
\pgfpathcurveto{\pgfqpoint{2.574439in}{3.025940in}}{\pgfqpoint{2.571167in}{3.018040in}}{\pgfqpoint{2.571167in}{3.009803in}}%
\pgfpathcurveto{\pgfqpoint{2.571167in}{3.001567in}}{\pgfqpoint{2.574439in}{2.993667in}}{\pgfqpoint{2.580263in}{2.987843in}}%
\pgfpathcurveto{\pgfqpoint{2.586087in}{2.982019in}}{\pgfqpoint{2.593987in}{2.978747in}}{\pgfqpoint{2.602224in}{2.978747in}}%
\pgfpathclose%
\pgfusepath{stroke,fill}%
\end{pgfscope}%
\begin{pgfscope}%
\pgfpathrectangle{\pgfqpoint{0.100000in}{0.212622in}}{\pgfqpoint{3.696000in}{3.696000in}}%
\pgfusepath{clip}%
\pgfsetbuttcap%
\pgfsetroundjoin%
\definecolor{currentfill}{rgb}{0.121569,0.466667,0.705882}%
\pgfsetfillcolor{currentfill}%
\pgfsetfillopacity{0.722122}%
\pgfsetlinewidth{1.003750pt}%
\definecolor{currentstroke}{rgb}{0.121569,0.466667,0.705882}%
\pgfsetstrokecolor{currentstroke}%
\pgfsetstrokeopacity{0.722122}%
\pgfsetdash{}{0pt}%
\pgfpathmoveto{\pgfqpoint{2.602163in}{2.978672in}}%
\pgfpathcurveto{\pgfqpoint{2.610399in}{2.978672in}}{\pgfqpoint{2.618299in}{2.981945in}}{\pgfqpoint{2.624123in}{2.987769in}}%
\pgfpathcurveto{\pgfqpoint{2.629947in}{2.993593in}}{\pgfqpoint{2.633219in}{3.001493in}}{\pgfqpoint{2.633219in}{3.009729in}}%
\pgfpathcurveto{\pgfqpoint{2.633219in}{3.017965in}}{\pgfqpoint{2.629947in}{3.025865in}}{\pgfqpoint{2.624123in}{3.031689in}}%
\pgfpathcurveto{\pgfqpoint{2.618299in}{3.037513in}}{\pgfqpoint{2.610399in}{3.040785in}}{\pgfqpoint{2.602163in}{3.040785in}}%
\pgfpathcurveto{\pgfqpoint{2.593927in}{3.040785in}}{\pgfqpoint{2.586027in}{3.037513in}}{\pgfqpoint{2.580203in}{3.031689in}}%
\pgfpathcurveto{\pgfqpoint{2.574379in}{3.025865in}}{\pgfqpoint{2.571106in}{3.017965in}}{\pgfqpoint{2.571106in}{3.009729in}}%
\pgfpathcurveto{\pgfqpoint{2.571106in}{3.001493in}}{\pgfqpoint{2.574379in}{2.993593in}}{\pgfqpoint{2.580203in}{2.987769in}}%
\pgfpathcurveto{\pgfqpoint{2.586027in}{2.981945in}}{\pgfqpoint{2.593927in}{2.978672in}}{\pgfqpoint{2.602163in}{2.978672in}}%
\pgfpathclose%
\pgfusepath{stroke,fill}%
\end{pgfscope}%
\begin{pgfscope}%
\pgfpathrectangle{\pgfqpoint{0.100000in}{0.212622in}}{\pgfqpoint{3.696000in}{3.696000in}}%
\pgfusepath{clip}%
\pgfsetbuttcap%
\pgfsetroundjoin%
\definecolor{currentfill}{rgb}{0.121569,0.466667,0.705882}%
\pgfsetfillcolor{currentfill}%
\pgfsetfillopacity{0.722289}%
\pgfsetlinewidth{1.003750pt}%
\definecolor{currentstroke}{rgb}{0.121569,0.466667,0.705882}%
\pgfsetstrokecolor{currentstroke}%
\pgfsetstrokeopacity{0.722289}%
\pgfsetdash{}{0pt}%
\pgfpathmoveto{\pgfqpoint{2.602023in}{2.978511in}}%
\pgfpathcurveto{\pgfqpoint{2.610260in}{2.978511in}}{\pgfqpoint{2.618160in}{2.981783in}}{\pgfqpoint{2.623984in}{2.987607in}}%
\pgfpathcurveto{\pgfqpoint{2.629808in}{2.993431in}}{\pgfqpoint{2.633080in}{3.001331in}}{\pgfqpoint{2.633080in}{3.009567in}}%
\pgfpathcurveto{\pgfqpoint{2.633080in}{3.017804in}}{\pgfqpoint{2.629808in}{3.025704in}}{\pgfqpoint{2.623984in}{3.031528in}}%
\pgfpathcurveto{\pgfqpoint{2.618160in}{3.037351in}}{\pgfqpoint{2.610260in}{3.040624in}}{\pgfqpoint{2.602023in}{3.040624in}}%
\pgfpathcurveto{\pgfqpoint{2.593787in}{3.040624in}}{\pgfqpoint{2.585887in}{3.037351in}}{\pgfqpoint{2.580063in}{3.031528in}}%
\pgfpathcurveto{\pgfqpoint{2.574239in}{3.025704in}}{\pgfqpoint{2.570967in}{3.017804in}}{\pgfqpoint{2.570967in}{3.009567in}}%
\pgfpathcurveto{\pgfqpoint{2.570967in}{3.001331in}}{\pgfqpoint{2.574239in}{2.993431in}}{\pgfqpoint{2.580063in}{2.987607in}}%
\pgfpathcurveto{\pgfqpoint{2.585887in}{2.981783in}}{\pgfqpoint{2.593787in}{2.978511in}}{\pgfqpoint{2.602023in}{2.978511in}}%
\pgfpathclose%
\pgfusepath{stroke,fill}%
\end{pgfscope}%
\begin{pgfscope}%
\pgfpathrectangle{\pgfqpoint{0.100000in}{0.212622in}}{\pgfqpoint{3.696000in}{3.696000in}}%
\pgfusepath{clip}%
\pgfsetbuttcap%
\pgfsetroundjoin%
\definecolor{currentfill}{rgb}{0.121569,0.466667,0.705882}%
\pgfsetfillcolor{currentfill}%
\pgfsetfillopacity{0.722426}%
\pgfsetlinewidth{1.003750pt}%
\definecolor{currentstroke}{rgb}{0.121569,0.466667,0.705882}%
\pgfsetstrokecolor{currentstroke}%
\pgfsetstrokeopacity{0.722426}%
\pgfsetdash{}{0pt}%
\pgfpathmoveto{\pgfqpoint{1.402403in}{2.668828in}}%
\pgfpathcurveto{\pgfqpoint{1.410639in}{2.668828in}}{\pgfqpoint{1.418539in}{2.672101in}}{\pgfqpoint{1.424363in}{2.677925in}}%
\pgfpathcurveto{\pgfqpoint{1.430187in}{2.683749in}}{\pgfqpoint{1.433460in}{2.691649in}}{\pgfqpoint{1.433460in}{2.699885in}}%
\pgfpathcurveto{\pgfqpoint{1.433460in}{2.708121in}}{\pgfqpoint{1.430187in}{2.716021in}}{\pgfqpoint{1.424363in}{2.721845in}}%
\pgfpathcurveto{\pgfqpoint{1.418539in}{2.727669in}}{\pgfqpoint{1.410639in}{2.730941in}}{\pgfqpoint{1.402403in}{2.730941in}}%
\pgfpathcurveto{\pgfqpoint{1.394167in}{2.730941in}}{\pgfqpoint{1.386267in}{2.727669in}}{\pgfqpoint{1.380443in}{2.721845in}}%
\pgfpathcurveto{\pgfqpoint{1.374619in}{2.716021in}}{\pgfqpoint{1.371347in}{2.708121in}}{\pgfqpoint{1.371347in}{2.699885in}}%
\pgfpathcurveto{\pgfqpoint{1.371347in}{2.691649in}}{\pgfqpoint{1.374619in}{2.683749in}}{\pgfqpoint{1.380443in}{2.677925in}}%
\pgfpathcurveto{\pgfqpoint{1.386267in}{2.672101in}}{\pgfqpoint{1.394167in}{2.668828in}}{\pgfqpoint{1.402403in}{2.668828in}}%
\pgfpathclose%
\pgfusepath{stroke,fill}%
\end{pgfscope}%
\begin{pgfscope}%
\pgfpathrectangle{\pgfqpoint{0.100000in}{0.212622in}}{\pgfqpoint{3.696000in}{3.696000in}}%
\pgfusepath{clip}%
\pgfsetbuttcap%
\pgfsetroundjoin%
\definecolor{currentfill}{rgb}{0.121569,0.466667,0.705882}%
\pgfsetfillcolor{currentfill}%
\pgfsetfillopacity{0.722588}%
\pgfsetlinewidth{1.003750pt}%
\definecolor{currentstroke}{rgb}{0.121569,0.466667,0.705882}%
\pgfsetstrokecolor{currentstroke}%
\pgfsetstrokeopacity{0.722588}%
\pgfsetdash{}{0pt}%
\pgfpathmoveto{\pgfqpoint{2.601719in}{2.978182in}}%
\pgfpathcurveto{\pgfqpoint{2.609956in}{2.978182in}}{\pgfqpoint{2.617856in}{2.981455in}}{\pgfqpoint{2.623680in}{2.987278in}}%
\pgfpathcurveto{\pgfqpoint{2.629504in}{2.993102in}}{\pgfqpoint{2.632776in}{3.001002in}}{\pgfqpoint{2.632776in}{3.009239in}}%
\pgfpathcurveto{\pgfqpoint{2.632776in}{3.017475in}}{\pgfqpoint{2.629504in}{3.025375in}}{\pgfqpoint{2.623680in}{3.031199in}}%
\pgfpathcurveto{\pgfqpoint{2.617856in}{3.037023in}}{\pgfqpoint{2.609956in}{3.040295in}}{\pgfqpoint{2.601719in}{3.040295in}}%
\pgfpathcurveto{\pgfqpoint{2.593483in}{3.040295in}}{\pgfqpoint{2.585583in}{3.037023in}}{\pgfqpoint{2.579759in}{3.031199in}}%
\pgfpathcurveto{\pgfqpoint{2.573935in}{3.025375in}}{\pgfqpoint{2.570663in}{3.017475in}}{\pgfqpoint{2.570663in}{3.009239in}}%
\pgfpathcurveto{\pgfqpoint{2.570663in}{3.001002in}}{\pgfqpoint{2.573935in}{2.993102in}}{\pgfqpoint{2.579759in}{2.987278in}}%
\pgfpathcurveto{\pgfqpoint{2.585583in}{2.981455in}}{\pgfqpoint{2.593483in}{2.978182in}}{\pgfqpoint{2.601719in}{2.978182in}}%
\pgfpathclose%
\pgfusepath{stroke,fill}%
\end{pgfscope}%
\begin{pgfscope}%
\pgfpathrectangle{\pgfqpoint{0.100000in}{0.212622in}}{\pgfqpoint{3.696000in}{3.696000in}}%
\pgfusepath{clip}%
\pgfsetbuttcap%
\pgfsetroundjoin%
\definecolor{currentfill}{rgb}{0.121569,0.466667,0.705882}%
\pgfsetfillcolor{currentfill}%
\pgfsetfillopacity{0.722705}%
\pgfsetlinewidth{1.003750pt}%
\definecolor{currentstroke}{rgb}{0.121569,0.466667,0.705882}%
\pgfsetstrokecolor{currentstroke}%
\pgfsetstrokeopacity{0.722705}%
\pgfsetdash{}{0pt}%
\pgfpathmoveto{\pgfqpoint{3.020841in}{1.715805in}}%
\pgfpathcurveto{\pgfqpoint{3.029077in}{1.715805in}}{\pgfqpoint{3.036977in}{1.719077in}}{\pgfqpoint{3.042801in}{1.724901in}}%
\pgfpathcurveto{\pgfqpoint{3.048625in}{1.730725in}}{\pgfqpoint{3.051897in}{1.738625in}}{\pgfqpoint{3.051897in}{1.746861in}}%
\pgfpathcurveto{\pgfqpoint{3.051897in}{1.755098in}}{\pgfqpoint{3.048625in}{1.762998in}}{\pgfqpoint{3.042801in}{1.768822in}}%
\pgfpathcurveto{\pgfqpoint{3.036977in}{1.774646in}}{\pgfqpoint{3.029077in}{1.777918in}}{\pgfqpoint{3.020841in}{1.777918in}}%
\pgfpathcurveto{\pgfqpoint{3.012604in}{1.777918in}}{\pgfqpoint{3.004704in}{1.774646in}}{\pgfqpoint{2.998881in}{1.768822in}}%
\pgfpathcurveto{\pgfqpoint{2.993057in}{1.762998in}}{\pgfqpoint{2.989784in}{1.755098in}}{\pgfqpoint{2.989784in}{1.746861in}}%
\pgfpathcurveto{\pgfqpoint{2.989784in}{1.738625in}}{\pgfqpoint{2.993057in}{1.730725in}}{\pgfqpoint{2.998881in}{1.724901in}}%
\pgfpathcurveto{\pgfqpoint{3.004704in}{1.719077in}}{\pgfqpoint{3.012604in}{1.715805in}}{\pgfqpoint{3.020841in}{1.715805in}}%
\pgfpathclose%
\pgfusepath{stroke,fill}%
\end{pgfscope}%
\begin{pgfscope}%
\pgfpathrectangle{\pgfqpoint{0.100000in}{0.212622in}}{\pgfqpoint{3.696000in}{3.696000in}}%
\pgfusepath{clip}%
\pgfsetbuttcap%
\pgfsetroundjoin%
\definecolor{currentfill}{rgb}{0.121569,0.466667,0.705882}%
\pgfsetfillcolor{currentfill}%
\pgfsetfillopacity{0.722773}%
\pgfsetlinewidth{1.003750pt}%
\definecolor{currentstroke}{rgb}{0.121569,0.466667,0.705882}%
\pgfsetstrokecolor{currentstroke}%
\pgfsetstrokeopacity{0.722773}%
\pgfsetdash{}{0pt}%
\pgfpathmoveto{\pgfqpoint{2.601498in}{2.977949in}}%
\pgfpathcurveto{\pgfqpoint{2.609735in}{2.977949in}}{\pgfqpoint{2.617635in}{2.981221in}}{\pgfqpoint{2.623459in}{2.987045in}}%
\pgfpathcurveto{\pgfqpoint{2.629283in}{2.992869in}}{\pgfqpoint{2.632555in}{3.000769in}}{\pgfqpoint{2.632555in}{3.009005in}}%
\pgfpathcurveto{\pgfqpoint{2.632555in}{3.017242in}}{\pgfqpoint{2.629283in}{3.025142in}}{\pgfqpoint{2.623459in}{3.030966in}}%
\pgfpathcurveto{\pgfqpoint{2.617635in}{3.036790in}}{\pgfqpoint{2.609735in}{3.040062in}}{\pgfqpoint{2.601498in}{3.040062in}}%
\pgfpathcurveto{\pgfqpoint{2.593262in}{3.040062in}}{\pgfqpoint{2.585362in}{3.036790in}}{\pgfqpoint{2.579538in}{3.030966in}}%
\pgfpathcurveto{\pgfqpoint{2.573714in}{3.025142in}}{\pgfqpoint{2.570442in}{3.017242in}}{\pgfqpoint{2.570442in}{3.009005in}}%
\pgfpathcurveto{\pgfqpoint{2.570442in}{3.000769in}}{\pgfqpoint{2.573714in}{2.992869in}}{\pgfqpoint{2.579538in}{2.987045in}}%
\pgfpathcurveto{\pgfqpoint{2.585362in}{2.981221in}}{\pgfqpoint{2.593262in}{2.977949in}}{\pgfqpoint{2.601498in}{2.977949in}}%
\pgfpathclose%
\pgfusepath{stroke,fill}%
\end{pgfscope}%
\begin{pgfscope}%
\pgfpathrectangle{\pgfqpoint{0.100000in}{0.212622in}}{\pgfqpoint{3.696000in}{3.696000in}}%
\pgfusepath{clip}%
\pgfsetbuttcap%
\pgfsetroundjoin%
\definecolor{currentfill}{rgb}{0.121569,0.466667,0.705882}%
\pgfsetfillcolor{currentfill}%
\pgfsetfillopacity{0.722773}%
\pgfsetlinewidth{1.003750pt}%
\definecolor{currentstroke}{rgb}{0.121569,0.466667,0.705882}%
\pgfsetstrokecolor{currentstroke}%
\pgfsetstrokeopacity{0.722773}%
\pgfsetdash{}{0pt}%
\pgfpathmoveto{\pgfqpoint{2.601498in}{2.977949in}}%
\pgfpathcurveto{\pgfqpoint{2.609735in}{2.977949in}}{\pgfqpoint{2.617635in}{2.981221in}}{\pgfqpoint{2.623459in}{2.987045in}}%
\pgfpathcurveto{\pgfqpoint{2.629283in}{2.992869in}}{\pgfqpoint{2.632555in}{3.000769in}}{\pgfqpoint{2.632555in}{3.009005in}}%
\pgfpathcurveto{\pgfqpoint{2.632555in}{3.017242in}}{\pgfqpoint{2.629283in}{3.025142in}}{\pgfqpoint{2.623459in}{3.030966in}}%
\pgfpathcurveto{\pgfqpoint{2.617635in}{3.036790in}}{\pgfqpoint{2.609735in}{3.040062in}}{\pgfqpoint{2.601498in}{3.040062in}}%
\pgfpathcurveto{\pgfqpoint{2.593262in}{3.040062in}}{\pgfqpoint{2.585362in}{3.036790in}}{\pgfqpoint{2.579538in}{3.030966in}}%
\pgfpathcurveto{\pgfqpoint{2.573714in}{3.025142in}}{\pgfqpoint{2.570442in}{3.017242in}}{\pgfqpoint{2.570442in}{3.009005in}}%
\pgfpathcurveto{\pgfqpoint{2.570442in}{3.000769in}}{\pgfqpoint{2.573714in}{2.992869in}}{\pgfqpoint{2.579538in}{2.987045in}}%
\pgfpathcurveto{\pgfqpoint{2.585362in}{2.981221in}}{\pgfqpoint{2.593262in}{2.977949in}}{\pgfqpoint{2.601498in}{2.977949in}}%
\pgfpathclose%
\pgfusepath{stroke,fill}%
\end{pgfscope}%
\begin{pgfscope}%
\pgfpathrectangle{\pgfqpoint{0.100000in}{0.212622in}}{\pgfqpoint{3.696000in}{3.696000in}}%
\pgfusepath{clip}%
\pgfsetbuttcap%
\pgfsetroundjoin%
\definecolor{currentfill}{rgb}{0.121569,0.466667,0.705882}%
\pgfsetfillcolor{currentfill}%
\pgfsetfillopacity{0.722773}%
\pgfsetlinewidth{1.003750pt}%
\definecolor{currentstroke}{rgb}{0.121569,0.466667,0.705882}%
\pgfsetstrokecolor{currentstroke}%
\pgfsetstrokeopacity{0.722773}%
\pgfsetdash{}{0pt}%
\pgfpathmoveto{\pgfqpoint{2.601498in}{2.977949in}}%
\pgfpathcurveto{\pgfqpoint{2.609734in}{2.977949in}}{\pgfqpoint{2.617635in}{2.981221in}}{\pgfqpoint{2.623458in}{2.987045in}}%
\pgfpathcurveto{\pgfqpoint{2.629282in}{2.992869in}}{\pgfqpoint{2.632555in}{3.000769in}}{\pgfqpoint{2.632555in}{3.009005in}}%
\pgfpathcurveto{\pgfqpoint{2.632555in}{3.017242in}}{\pgfqpoint{2.629282in}{3.025142in}}{\pgfqpoint{2.623458in}{3.030966in}}%
\pgfpathcurveto{\pgfqpoint{2.617635in}{3.036789in}}{\pgfqpoint{2.609734in}{3.040062in}}{\pgfqpoint{2.601498in}{3.040062in}}%
\pgfpathcurveto{\pgfqpoint{2.593262in}{3.040062in}}{\pgfqpoint{2.585362in}{3.036789in}}{\pgfqpoint{2.579538in}{3.030966in}}%
\pgfpathcurveto{\pgfqpoint{2.573714in}{3.025142in}}{\pgfqpoint{2.570442in}{3.017242in}}{\pgfqpoint{2.570442in}{3.009005in}}%
\pgfpathcurveto{\pgfqpoint{2.570442in}{3.000769in}}{\pgfqpoint{2.573714in}{2.992869in}}{\pgfqpoint{2.579538in}{2.987045in}}%
\pgfpathcurveto{\pgfqpoint{2.585362in}{2.981221in}}{\pgfqpoint{2.593262in}{2.977949in}}{\pgfqpoint{2.601498in}{2.977949in}}%
\pgfpathclose%
\pgfusepath{stroke,fill}%
\end{pgfscope}%
\begin{pgfscope}%
\pgfpathrectangle{\pgfqpoint{0.100000in}{0.212622in}}{\pgfqpoint{3.696000in}{3.696000in}}%
\pgfusepath{clip}%
\pgfsetbuttcap%
\pgfsetroundjoin%
\definecolor{currentfill}{rgb}{0.121569,0.466667,0.705882}%
\pgfsetfillcolor{currentfill}%
\pgfsetfillopacity{0.722773}%
\pgfsetlinewidth{1.003750pt}%
\definecolor{currentstroke}{rgb}{0.121569,0.466667,0.705882}%
\pgfsetstrokecolor{currentstroke}%
\pgfsetstrokeopacity{0.722773}%
\pgfsetdash{}{0pt}%
\pgfpathmoveto{\pgfqpoint{2.601498in}{2.977949in}}%
\pgfpathcurveto{\pgfqpoint{2.609734in}{2.977949in}}{\pgfqpoint{2.617634in}{2.981221in}}{\pgfqpoint{2.623458in}{2.987045in}}%
\pgfpathcurveto{\pgfqpoint{2.629282in}{2.992869in}}{\pgfqpoint{2.632554in}{3.000769in}}{\pgfqpoint{2.632554in}{3.009005in}}%
\pgfpathcurveto{\pgfqpoint{2.632554in}{3.017241in}}{\pgfqpoint{2.629282in}{3.025141in}}{\pgfqpoint{2.623458in}{3.030965in}}%
\pgfpathcurveto{\pgfqpoint{2.617634in}{3.036789in}}{\pgfqpoint{2.609734in}{3.040062in}}{\pgfqpoint{2.601498in}{3.040062in}}%
\pgfpathcurveto{\pgfqpoint{2.593262in}{3.040062in}}{\pgfqpoint{2.585362in}{3.036789in}}{\pgfqpoint{2.579538in}{3.030965in}}%
\pgfpathcurveto{\pgfqpoint{2.573714in}{3.025141in}}{\pgfqpoint{2.570441in}{3.017241in}}{\pgfqpoint{2.570441in}{3.009005in}}%
\pgfpathcurveto{\pgfqpoint{2.570441in}{3.000769in}}{\pgfqpoint{2.573714in}{2.992869in}}{\pgfqpoint{2.579538in}{2.987045in}}%
\pgfpathcurveto{\pgfqpoint{2.585362in}{2.981221in}}{\pgfqpoint{2.593262in}{2.977949in}}{\pgfqpoint{2.601498in}{2.977949in}}%
\pgfpathclose%
\pgfusepath{stroke,fill}%
\end{pgfscope}%
\begin{pgfscope}%
\pgfpathrectangle{\pgfqpoint{0.100000in}{0.212622in}}{\pgfqpoint{3.696000in}{3.696000in}}%
\pgfusepath{clip}%
\pgfsetbuttcap%
\pgfsetroundjoin%
\definecolor{currentfill}{rgb}{0.121569,0.466667,0.705882}%
\pgfsetfillcolor{currentfill}%
\pgfsetfillopacity{0.722773}%
\pgfsetlinewidth{1.003750pt}%
\definecolor{currentstroke}{rgb}{0.121569,0.466667,0.705882}%
\pgfsetstrokecolor{currentstroke}%
\pgfsetstrokeopacity{0.722773}%
\pgfsetdash{}{0pt}%
\pgfpathmoveto{\pgfqpoint{2.601498in}{2.977948in}}%
\pgfpathcurveto{\pgfqpoint{2.609734in}{2.977948in}}{\pgfqpoint{2.617634in}{2.981220in}}{\pgfqpoint{2.623458in}{2.987044in}}%
\pgfpathcurveto{\pgfqpoint{2.629282in}{2.992868in}}{\pgfqpoint{2.632554in}{3.000768in}}{\pgfqpoint{2.632554in}{3.009005in}}%
\pgfpathcurveto{\pgfqpoint{2.632554in}{3.017241in}}{\pgfqpoint{2.629282in}{3.025141in}}{\pgfqpoint{2.623458in}{3.030965in}}%
\pgfpathcurveto{\pgfqpoint{2.617634in}{3.036789in}}{\pgfqpoint{2.609734in}{3.040061in}}{\pgfqpoint{2.601498in}{3.040061in}}%
\pgfpathcurveto{\pgfqpoint{2.593261in}{3.040061in}}{\pgfqpoint{2.585361in}{3.036789in}}{\pgfqpoint{2.579537in}{3.030965in}}%
\pgfpathcurveto{\pgfqpoint{2.573713in}{3.025141in}}{\pgfqpoint{2.570441in}{3.017241in}}{\pgfqpoint{2.570441in}{3.009005in}}%
\pgfpathcurveto{\pgfqpoint{2.570441in}{3.000768in}}{\pgfqpoint{2.573713in}{2.992868in}}{\pgfqpoint{2.579537in}{2.987044in}}%
\pgfpathcurveto{\pgfqpoint{2.585361in}{2.981220in}}{\pgfqpoint{2.593261in}{2.977948in}}{\pgfqpoint{2.601498in}{2.977948in}}%
\pgfpathclose%
\pgfusepath{stroke,fill}%
\end{pgfscope}%
\begin{pgfscope}%
\pgfpathrectangle{\pgfqpoint{0.100000in}{0.212622in}}{\pgfqpoint{3.696000in}{3.696000in}}%
\pgfusepath{clip}%
\pgfsetbuttcap%
\pgfsetroundjoin%
\definecolor{currentfill}{rgb}{0.121569,0.466667,0.705882}%
\pgfsetfillcolor{currentfill}%
\pgfsetfillopacity{0.722774}%
\pgfsetlinewidth{1.003750pt}%
\definecolor{currentstroke}{rgb}{0.121569,0.466667,0.705882}%
\pgfsetstrokecolor{currentstroke}%
\pgfsetstrokeopacity{0.722774}%
\pgfsetdash{}{0pt}%
\pgfpathmoveto{\pgfqpoint{2.601497in}{2.977947in}}%
\pgfpathcurveto{\pgfqpoint{2.609733in}{2.977947in}}{\pgfqpoint{2.617633in}{2.981219in}}{\pgfqpoint{2.623457in}{2.987043in}}%
\pgfpathcurveto{\pgfqpoint{2.629281in}{2.992867in}}{\pgfqpoint{2.632553in}{3.000767in}}{\pgfqpoint{2.632553in}{3.009003in}}%
\pgfpathcurveto{\pgfqpoint{2.632553in}{3.017240in}}{\pgfqpoint{2.629281in}{3.025140in}}{\pgfqpoint{2.623457in}{3.030964in}}%
\pgfpathcurveto{\pgfqpoint{2.617633in}{3.036788in}}{\pgfqpoint{2.609733in}{3.040060in}}{\pgfqpoint{2.601497in}{3.040060in}}%
\pgfpathcurveto{\pgfqpoint{2.593260in}{3.040060in}}{\pgfqpoint{2.585360in}{3.036788in}}{\pgfqpoint{2.579537in}{3.030964in}}%
\pgfpathcurveto{\pgfqpoint{2.573713in}{3.025140in}}{\pgfqpoint{2.570440in}{3.017240in}}{\pgfqpoint{2.570440in}{3.009003in}}%
\pgfpathcurveto{\pgfqpoint{2.570440in}{3.000767in}}{\pgfqpoint{2.573713in}{2.992867in}}{\pgfqpoint{2.579537in}{2.987043in}}%
\pgfpathcurveto{\pgfqpoint{2.585360in}{2.981219in}}{\pgfqpoint{2.593260in}{2.977947in}}{\pgfqpoint{2.601497in}{2.977947in}}%
\pgfpathclose%
\pgfusepath{stroke,fill}%
\end{pgfscope}%
\begin{pgfscope}%
\pgfpathrectangle{\pgfqpoint{0.100000in}{0.212622in}}{\pgfqpoint{3.696000in}{3.696000in}}%
\pgfusepath{clip}%
\pgfsetbuttcap%
\pgfsetroundjoin%
\definecolor{currentfill}{rgb}{0.121569,0.466667,0.705882}%
\pgfsetfillcolor{currentfill}%
\pgfsetfillopacity{0.722774}%
\pgfsetlinewidth{1.003750pt}%
\definecolor{currentstroke}{rgb}{0.121569,0.466667,0.705882}%
\pgfsetstrokecolor{currentstroke}%
\pgfsetstrokeopacity{0.722774}%
\pgfsetdash{}{0pt}%
\pgfpathmoveto{\pgfqpoint{2.601495in}{2.977945in}}%
\pgfpathcurveto{\pgfqpoint{2.609732in}{2.977945in}}{\pgfqpoint{2.617632in}{2.981217in}}{\pgfqpoint{2.623456in}{2.987041in}}%
\pgfpathcurveto{\pgfqpoint{2.629280in}{2.992865in}}{\pgfqpoint{2.632552in}{3.000765in}}{\pgfqpoint{2.632552in}{3.009002in}}%
\pgfpathcurveto{\pgfqpoint{2.632552in}{3.017238in}}{\pgfqpoint{2.629280in}{3.025138in}}{\pgfqpoint{2.623456in}{3.030962in}}%
\pgfpathcurveto{\pgfqpoint{2.617632in}{3.036786in}}{\pgfqpoint{2.609732in}{3.040058in}}{\pgfqpoint{2.601495in}{3.040058in}}%
\pgfpathcurveto{\pgfqpoint{2.593259in}{3.040058in}}{\pgfqpoint{2.585359in}{3.036786in}}{\pgfqpoint{2.579535in}{3.030962in}}%
\pgfpathcurveto{\pgfqpoint{2.573711in}{3.025138in}}{\pgfqpoint{2.570439in}{3.017238in}}{\pgfqpoint{2.570439in}{3.009002in}}%
\pgfpathcurveto{\pgfqpoint{2.570439in}{3.000765in}}{\pgfqpoint{2.573711in}{2.992865in}}{\pgfqpoint{2.579535in}{2.987041in}}%
\pgfpathcurveto{\pgfqpoint{2.585359in}{2.981217in}}{\pgfqpoint{2.593259in}{2.977945in}}{\pgfqpoint{2.601495in}{2.977945in}}%
\pgfpathclose%
\pgfusepath{stroke,fill}%
\end{pgfscope}%
\begin{pgfscope}%
\pgfpathrectangle{\pgfqpoint{0.100000in}{0.212622in}}{\pgfqpoint{3.696000in}{3.696000in}}%
\pgfusepath{clip}%
\pgfsetbuttcap%
\pgfsetroundjoin%
\definecolor{currentfill}{rgb}{0.121569,0.466667,0.705882}%
\pgfsetfillcolor{currentfill}%
\pgfsetfillopacity{0.722776}%
\pgfsetlinewidth{1.003750pt}%
\definecolor{currentstroke}{rgb}{0.121569,0.466667,0.705882}%
\pgfsetstrokecolor{currentstroke}%
\pgfsetstrokeopacity{0.722776}%
\pgfsetdash{}{0pt}%
\pgfpathmoveto{\pgfqpoint{2.601493in}{2.977942in}}%
\pgfpathcurveto{\pgfqpoint{2.609729in}{2.977942in}}{\pgfqpoint{2.617629in}{2.981214in}}{\pgfqpoint{2.623453in}{2.987038in}}%
\pgfpathcurveto{\pgfqpoint{2.629277in}{2.992862in}}{\pgfqpoint{2.632549in}{3.000762in}}{\pgfqpoint{2.632549in}{3.008999in}}%
\pgfpathcurveto{\pgfqpoint{2.632549in}{3.017235in}}{\pgfqpoint{2.629277in}{3.025135in}}{\pgfqpoint{2.623453in}{3.030959in}}%
\pgfpathcurveto{\pgfqpoint{2.617629in}{3.036783in}}{\pgfqpoint{2.609729in}{3.040055in}}{\pgfqpoint{2.601493in}{3.040055in}}%
\pgfpathcurveto{\pgfqpoint{2.593257in}{3.040055in}}{\pgfqpoint{2.585356in}{3.036783in}}{\pgfqpoint{2.579533in}{3.030959in}}%
\pgfpathcurveto{\pgfqpoint{2.573709in}{3.025135in}}{\pgfqpoint{2.570436in}{3.017235in}}{\pgfqpoint{2.570436in}{3.008999in}}%
\pgfpathcurveto{\pgfqpoint{2.570436in}{3.000762in}}{\pgfqpoint{2.573709in}{2.992862in}}{\pgfqpoint{2.579533in}{2.987038in}}%
\pgfpathcurveto{\pgfqpoint{2.585356in}{2.981214in}}{\pgfqpoint{2.593257in}{2.977942in}}{\pgfqpoint{2.601493in}{2.977942in}}%
\pgfpathclose%
\pgfusepath{stroke,fill}%
\end{pgfscope}%
\begin{pgfscope}%
\pgfpathrectangle{\pgfqpoint{0.100000in}{0.212622in}}{\pgfqpoint{3.696000in}{3.696000in}}%
\pgfusepath{clip}%
\pgfsetbuttcap%
\pgfsetroundjoin%
\definecolor{currentfill}{rgb}{0.121569,0.466667,0.705882}%
\pgfsetfillcolor{currentfill}%
\pgfsetfillopacity{0.722778}%
\pgfsetlinewidth{1.003750pt}%
\definecolor{currentstroke}{rgb}{0.121569,0.466667,0.705882}%
\pgfsetstrokecolor{currentstroke}%
\pgfsetstrokeopacity{0.722778}%
\pgfsetdash{}{0pt}%
\pgfpathmoveto{\pgfqpoint{2.601488in}{2.977936in}}%
\pgfpathcurveto{\pgfqpoint{2.609724in}{2.977936in}}{\pgfqpoint{2.617624in}{2.981208in}}{\pgfqpoint{2.623448in}{2.987032in}}%
\pgfpathcurveto{\pgfqpoint{2.629272in}{2.992856in}}{\pgfqpoint{2.632545in}{3.000756in}}{\pgfqpoint{2.632545in}{3.008992in}}%
\pgfpathcurveto{\pgfqpoint{2.632545in}{3.017229in}}{\pgfqpoint{2.629272in}{3.025129in}}{\pgfqpoint{2.623448in}{3.030953in}}%
\pgfpathcurveto{\pgfqpoint{2.617624in}{3.036777in}}{\pgfqpoint{2.609724in}{3.040049in}}{\pgfqpoint{2.601488in}{3.040049in}}%
\pgfpathcurveto{\pgfqpoint{2.593252in}{3.040049in}}{\pgfqpoint{2.585352in}{3.036777in}}{\pgfqpoint{2.579528in}{3.030953in}}%
\pgfpathcurveto{\pgfqpoint{2.573704in}{3.025129in}}{\pgfqpoint{2.570432in}{3.017229in}}{\pgfqpoint{2.570432in}{3.008992in}}%
\pgfpathcurveto{\pgfqpoint{2.570432in}{3.000756in}}{\pgfqpoint{2.573704in}{2.992856in}}{\pgfqpoint{2.579528in}{2.987032in}}%
\pgfpathcurveto{\pgfqpoint{2.585352in}{2.981208in}}{\pgfqpoint{2.593252in}{2.977936in}}{\pgfqpoint{2.601488in}{2.977936in}}%
\pgfpathclose%
\pgfusepath{stroke,fill}%
\end{pgfscope}%
\begin{pgfscope}%
\pgfpathrectangle{\pgfqpoint{0.100000in}{0.212622in}}{\pgfqpoint{3.696000in}{3.696000in}}%
\pgfusepath{clip}%
\pgfsetbuttcap%
\pgfsetroundjoin%
\definecolor{currentfill}{rgb}{0.121569,0.466667,0.705882}%
\pgfsetfillcolor{currentfill}%
\pgfsetfillopacity{0.722782}%
\pgfsetlinewidth{1.003750pt}%
\definecolor{currentstroke}{rgb}{0.121569,0.466667,0.705882}%
\pgfsetstrokecolor{currentstroke}%
\pgfsetstrokeopacity{0.722782}%
\pgfsetdash{}{0pt}%
\pgfpathmoveto{\pgfqpoint{2.601480in}{2.977925in}}%
\pgfpathcurveto{\pgfqpoint{2.609716in}{2.977925in}}{\pgfqpoint{2.617616in}{2.981198in}}{\pgfqpoint{2.623440in}{2.987022in}}%
\pgfpathcurveto{\pgfqpoint{2.629264in}{2.992846in}}{\pgfqpoint{2.632537in}{3.000746in}}{\pgfqpoint{2.632537in}{3.008982in}}%
\pgfpathcurveto{\pgfqpoint{2.632537in}{3.017218in}}{\pgfqpoint{2.629264in}{3.025118in}}{\pgfqpoint{2.623440in}{3.030942in}}%
\pgfpathcurveto{\pgfqpoint{2.617616in}{3.036766in}}{\pgfqpoint{2.609716in}{3.040038in}}{\pgfqpoint{2.601480in}{3.040038in}}%
\pgfpathcurveto{\pgfqpoint{2.593244in}{3.040038in}}{\pgfqpoint{2.585344in}{3.036766in}}{\pgfqpoint{2.579520in}{3.030942in}}%
\pgfpathcurveto{\pgfqpoint{2.573696in}{3.025118in}}{\pgfqpoint{2.570424in}{3.017218in}}{\pgfqpoint{2.570424in}{3.008982in}}%
\pgfpathcurveto{\pgfqpoint{2.570424in}{3.000746in}}{\pgfqpoint{2.573696in}{2.992846in}}{\pgfqpoint{2.579520in}{2.987022in}}%
\pgfpathcurveto{\pgfqpoint{2.585344in}{2.981198in}}{\pgfqpoint{2.593244in}{2.977925in}}{\pgfqpoint{2.601480in}{2.977925in}}%
\pgfpathclose%
\pgfusepath{stroke,fill}%
\end{pgfscope}%
\begin{pgfscope}%
\pgfpathrectangle{\pgfqpoint{0.100000in}{0.212622in}}{\pgfqpoint{3.696000in}{3.696000in}}%
\pgfusepath{clip}%
\pgfsetbuttcap%
\pgfsetroundjoin%
\definecolor{currentfill}{rgb}{0.121569,0.466667,0.705882}%
\pgfsetfillcolor{currentfill}%
\pgfsetfillopacity{0.722790}%
\pgfsetlinewidth{1.003750pt}%
\definecolor{currentstroke}{rgb}{0.121569,0.466667,0.705882}%
\pgfsetstrokecolor{currentstroke}%
\pgfsetstrokeopacity{0.722790}%
\pgfsetdash{}{0pt}%
\pgfpathmoveto{\pgfqpoint{2.601466in}{2.977903in}}%
\pgfpathcurveto{\pgfqpoint{2.609702in}{2.977903in}}{\pgfqpoint{2.617602in}{2.981175in}}{\pgfqpoint{2.623426in}{2.986999in}}%
\pgfpathcurveto{\pgfqpoint{2.629250in}{2.992823in}}{\pgfqpoint{2.632522in}{3.000723in}}{\pgfqpoint{2.632522in}{3.008959in}}%
\pgfpathcurveto{\pgfqpoint{2.632522in}{3.017195in}}{\pgfqpoint{2.629250in}{3.025095in}}{\pgfqpoint{2.623426in}{3.030919in}}%
\pgfpathcurveto{\pgfqpoint{2.617602in}{3.036743in}}{\pgfqpoint{2.609702in}{3.040016in}}{\pgfqpoint{2.601466in}{3.040016in}}%
\pgfpathcurveto{\pgfqpoint{2.593230in}{3.040016in}}{\pgfqpoint{2.585330in}{3.036743in}}{\pgfqpoint{2.579506in}{3.030919in}}%
\pgfpathcurveto{\pgfqpoint{2.573682in}{3.025095in}}{\pgfqpoint{2.570409in}{3.017195in}}{\pgfqpoint{2.570409in}{3.008959in}}%
\pgfpathcurveto{\pgfqpoint{2.570409in}{3.000723in}}{\pgfqpoint{2.573682in}{2.992823in}}{\pgfqpoint{2.579506in}{2.986999in}}%
\pgfpathcurveto{\pgfqpoint{2.585330in}{2.981175in}}{\pgfqpoint{2.593230in}{2.977903in}}{\pgfqpoint{2.601466in}{2.977903in}}%
\pgfpathclose%
\pgfusepath{stroke,fill}%
\end{pgfscope}%
\begin{pgfscope}%
\pgfpathrectangle{\pgfqpoint{0.100000in}{0.212622in}}{\pgfqpoint{3.696000in}{3.696000in}}%
\pgfusepath{clip}%
\pgfsetbuttcap%
\pgfsetroundjoin%
\definecolor{currentfill}{rgb}{0.121569,0.466667,0.705882}%
\pgfsetfillcolor{currentfill}%
\pgfsetfillopacity{0.722804}%
\pgfsetlinewidth{1.003750pt}%
\definecolor{currentstroke}{rgb}{0.121569,0.466667,0.705882}%
\pgfsetstrokecolor{currentstroke}%
\pgfsetstrokeopacity{0.722804}%
\pgfsetdash{}{0pt}%
\pgfpathmoveto{\pgfqpoint{2.601438in}{2.977866in}}%
\pgfpathcurveto{\pgfqpoint{2.609675in}{2.977866in}}{\pgfqpoint{2.617575in}{2.981139in}}{\pgfqpoint{2.623399in}{2.986962in}}%
\pgfpathcurveto{\pgfqpoint{2.629222in}{2.992786in}}{\pgfqpoint{2.632495in}{3.000686in}}{\pgfqpoint{2.632495in}{3.008923in}}%
\pgfpathcurveto{\pgfqpoint{2.632495in}{3.017159in}}{\pgfqpoint{2.629222in}{3.025059in}}{\pgfqpoint{2.623399in}{3.030883in}}%
\pgfpathcurveto{\pgfqpoint{2.617575in}{3.036707in}}{\pgfqpoint{2.609675in}{3.039979in}}{\pgfqpoint{2.601438in}{3.039979in}}%
\pgfpathcurveto{\pgfqpoint{2.593202in}{3.039979in}}{\pgfqpoint{2.585302in}{3.036707in}}{\pgfqpoint{2.579478in}{3.030883in}}%
\pgfpathcurveto{\pgfqpoint{2.573654in}{3.025059in}}{\pgfqpoint{2.570382in}{3.017159in}}{\pgfqpoint{2.570382in}{3.008923in}}%
\pgfpathcurveto{\pgfqpoint{2.570382in}{3.000686in}}{\pgfqpoint{2.573654in}{2.992786in}}{\pgfqpoint{2.579478in}{2.986962in}}%
\pgfpathcurveto{\pgfqpoint{2.585302in}{2.981139in}}{\pgfqpoint{2.593202in}{2.977866in}}{\pgfqpoint{2.601438in}{2.977866in}}%
\pgfpathclose%
\pgfusepath{stroke,fill}%
\end{pgfscope}%
\begin{pgfscope}%
\pgfpathrectangle{\pgfqpoint{0.100000in}{0.212622in}}{\pgfqpoint{3.696000in}{3.696000in}}%
\pgfusepath{clip}%
\pgfsetbuttcap%
\pgfsetroundjoin%
\definecolor{currentfill}{rgb}{0.121569,0.466667,0.705882}%
\pgfsetfillcolor{currentfill}%
\pgfsetfillopacity{0.722828}%
\pgfsetlinewidth{1.003750pt}%
\definecolor{currentstroke}{rgb}{0.121569,0.466667,0.705882}%
\pgfsetstrokecolor{currentstroke}%
\pgfsetstrokeopacity{0.722828}%
\pgfsetdash{}{0pt}%
\pgfpathmoveto{\pgfqpoint{2.601384in}{2.977797in}}%
\pgfpathcurveto{\pgfqpoint{2.609620in}{2.977797in}}{\pgfqpoint{2.617520in}{2.981070in}}{\pgfqpoint{2.623344in}{2.986893in}}%
\pgfpathcurveto{\pgfqpoint{2.629168in}{2.992717in}}{\pgfqpoint{2.632440in}{3.000617in}}{\pgfqpoint{2.632440in}{3.008854in}}%
\pgfpathcurveto{\pgfqpoint{2.632440in}{3.017090in}}{\pgfqpoint{2.629168in}{3.024990in}}{\pgfqpoint{2.623344in}{3.030814in}}%
\pgfpathcurveto{\pgfqpoint{2.617520in}{3.036638in}}{\pgfqpoint{2.609620in}{3.039910in}}{\pgfqpoint{2.601384in}{3.039910in}}%
\pgfpathcurveto{\pgfqpoint{2.593147in}{3.039910in}}{\pgfqpoint{2.585247in}{3.036638in}}{\pgfqpoint{2.579423in}{3.030814in}}%
\pgfpathcurveto{\pgfqpoint{2.573600in}{3.024990in}}{\pgfqpoint{2.570327in}{3.017090in}}{\pgfqpoint{2.570327in}{3.008854in}}%
\pgfpathcurveto{\pgfqpoint{2.570327in}{3.000617in}}{\pgfqpoint{2.573600in}{2.992717in}}{\pgfqpoint{2.579423in}{2.986893in}}%
\pgfpathcurveto{\pgfqpoint{2.585247in}{2.981070in}}{\pgfqpoint{2.593147in}{2.977797in}}{\pgfqpoint{2.601384in}{2.977797in}}%
\pgfpathclose%
\pgfusepath{stroke,fill}%
\end{pgfscope}%
\begin{pgfscope}%
\pgfpathrectangle{\pgfqpoint{0.100000in}{0.212622in}}{\pgfqpoint{3.696000in}{3.696000in}}%
\pgfusepath{clip}%
\pgfsetbuttcap%
\pgfsetroundjoin%
\definecolor{currentfill}{rgb}{0.121569,0.466667,0.705882}%
\pgfsetfillcolor{currentfill}%
\pgfsetfillopacity{0.722875}%
\pgfsetlinewidth{1.003750pt}%
\definecolor{currentstroke}{rgb}{0.121569,0.466667,0.705882}%
\pgfsetstrokecolor{currentstroke}%
\pgfsetstrokeopacity{0.722875}%
\pgfsetdash{}{0pt}%
\pgfpathmoveto{\pgfqpoint{2.601292in}{2.977678in}}%
\pgfpathcurveto{\pgfqpoint{2.609529in}{2.977678in}}{\pgfqpoint{2.617429in}{2.980950in}}{\pgfqpoint{2.623253in}{2.986774in}}%
\pgfpathcurveto{\pgfqpoint{2.629077in}{2.992598in}}{\pgfqpoint{2.632349in}{3.000498in}}{\pgfqpoint{2.632349in}{3.008735in}}%
\pgfpathcurveto{\pgfqpoint{2.632349in}{3.016971in}}{\pgfqpoint{2.629077in}{3.024871in}}{\pgfqpoint{2.623253in}{3.030695in}}%
\pgfpathcurveto{\pgfqpoint{2.617429in}{3.036519in}}{\pgfqpoint{2.609529in}{3.039791in}}{\pgfqpoint{2.601292in}{3.039791in}}%
\pgfpathcurveto{\pgfqpoint{2.593056in}{3.039791in}}{\pgfqpoint{2.585156in}{3.036519in}}{\pgfqpoint{2.579332in}{3.030695in}}%
\pgfpathcurveto{\pgfqpoint{2.573508in}{3.024871in}}{\pgfqpoint{2.570236in}{3.016971in}}{\pgfqpoint{2.570236in}{3.008735in}}%
\pgfpathcurveto{\pgfqpoint{2.570236in}{3.000498in}}{\pgfqpoint{2.573508in}{2.992598in}}{\pgfqpoint{2.579332in}{2.986774in}}%
\pgfpathcurveto{\pgfqpoint{2.585156in}{2.980950in}}{\pgfqpoint{2.593056in}{2.977678in}}{\pgfqpoint{2.601292in}{2.977678in}}%
\pgfpathclose%
\pgfusepath{stroke,fill}%
\end{pgfscope}%
\begin{pgfscope}%
\pgfpathrectangle{\pgfqpoint{0.100000in}{0.212622in}}{\pgfqpoint{3.696000in}{3.696000in}}%
\pgfusepath{clip}%
\pgfsetbuttcap%
\pgfsetroundjoin%
\definecolor{currentfill}{rgb}{0.121569,0.466667,0.705882}%
\pgfsetfillcolor{currentfill}%
\pgfsetfillopacity{0.722956}%
\pgfsetlinewidth{1.003750pt}%
\definecolor{currentstroke}{rgb}{0.121569,0.466667,0.705882}%
\pgfsetstrokecolor{currentstroke}%
\pgfsetstrokeopacity{0.722956}%
\pgfsetdash{}{0pt}%
\pgfpathmoveto{\pgfqpoint{2.601112in}{2.977448in}}%
\pgfpathcurveto{\pgfqpoint{2.609348in}{2.977448in}}{\pgfqpoint{2.617248in}{2.980721in}}{\pgfqpoint{2.623072in}{2.986545in}}%
\pgfpathcurveto{\pgfqpoint{2.628896in}{2.992369in}}{\pgfqpoint{2.632168in}{3.000269in}}{\pgfqpoint{2.632168in}{3.008505in}}%
\pgfpathcurveto{\pgfqpoint{2.632168in}{3.016741in}}{\pgfqpoint{2.628896in}{3.024641in}}{\pgfqpoint{2.623072in}{3.030465in}}%
\pgfpathcurveto{\pgfqpoint{2.617248in}{3.036289in}}{\pgfqpoint{2.609348in}{3.039561in}}{\pgfqpoint{2.601112in}{3.039561in}}%
\pgfpathcurveto{\pgfqpoint{2.592876in}{3.039561in}}{\pgfqpoint{2.584976in}{3.036289in}}{\pgfqpoint{2.579152in}{3.030465in}}%
\pgfpathcurveto{\pgfqpoint{2.573328in}{3.024641in}}{\pgfqpoint{2.570055in}{3.016741in}}{\pgfqpoint{2.570055in}{3.008505in}}%
\pgfpathcurveto{\pgfqpoint{2.570055in}{3.000269in}}{\pgfqpoint{2.573328in}{2.992369in}}{\pgfqpoint{2.579152in}{2.986545in}}%
\pgfpathcurveto{\pgfqpoint{2.584976in}{2.980721in}}{\pgfqpoint{2.592876in}{2.977448in}}{\pgfqpoint{2.601112in}{2.977448in}}%
\pgfpathclose%
\pgfusepath{stroke,fill}%
\end{pgfscope}%
\begin{pgfscope}%
\pgfpathrectangle{\pgfqpoint{0.100000in}{0.212622in}}{\pgfqpoint{3.696000in}{3.696000in}}%
\pgfusepath{clip}%
\pgfsetbuttcap%
\pgfsetroundjoin%
\definecolor{currentfill}{rgb}{0.121569,0.466667,0.705882}%
\pgfsetfillcolor{currentfill}%
\pgfsetfillopacity{0.723096}%
\pgfsetlinewidth{1.003750pt}%
\definecolor{currentstroke}{rgb}{0.121569,0.466667,0.705882}%
\pgfsetstrokecolor{currentstroke}%
\pgfsetstrokeopacity{0.723096}%
\pgfsetdash{}{0pt}%
\pgfpathmoveto{\pgfqpoint{2.600759in}{2.977008in}}%
\pgfpathcurveto{\pgfqpoint{2.608995in}{2.977008in}}{\pgfqpoint{2.616895in}{2.980280in}}{\pgfqpoint{2.622719in}{2.986104in}}%
\pgfpathcurveto{\pgfqpoint{2.628543in}{2.991928in}}{\pgfqpoint{2.631815in}{2.999828in}}{\pgfqpoint{2.631815in}{3.008064in}}%
\pgfpathcurveto{\pgfqpoint{2.631815in}{3.016301in}}{\pgfqpoint{2.628543in}{3.024201in}}{\pgfqpoint{2.622719in}{3.030024in}}%
\pgfpathcurveto{\pgfqpoint{2.616895in}{3.035848in}}{\pgfqpoint{2.608995in}{3.039121in}}{\pgfqpoint{2.600759in}{3.039121in}}%
\pgfpathcurveto{\pgfqpoint{2.592523in}{3.039121in}}{\pgfqpoint{2.584623in}{3.035848in}}{\pgfqpoint{2.578799in}{3.030024in}}%
\pgfpathcurveto{\pgfqpoint{2.572975in}{3.024201in}}{\pgfqpoint{2.569702in}{3.016301in}}{\pgfqpoint{2.569702in}{3.008064in}}%
\pgfpathcurveto{\pgfqpoint{2.569702in}{2.999828in}}{\pgfqpoint{2.572975in}{2.991928in}}{\pgfqpoint{2.578799in}{2.986104in}}%
\pgfpathcurveto{\pgfqpoint{2.584623in}{2.980280in}}{\pgfqpoint{2.592523in}{2.977008in}}{\pgfqpoint{2.600759in}{2.977008in}}%
\pgfpathclose%
\pgfusepath{stroke,fill}%
\end{pgfscope}%
\begin{pgfscope}%
\pgfpathrectangle{\pgfqpoint{0.100000in}{0.212622in}}{\pgfqpoint{3.696000in}{3.696000in}}%
\pgfusepath{clip}%
\pgfsetbuttcap%
\pgfsetroundjoin%
\definecolor{currentfill}{rgb}{0.121569,0.466667,0.705882}%
\pgfsetfillcolor{currentfill}%
\pgfsetfillopacity{0.723211}%
\pgfsetlinewidth{1.003750pt}%
\definecolor{currentstroke}{rgb}{0.121569,0.466667,0.705882}%
\pgfsetstrokecolor{currentstroke}%
\pgfsetstrokeopacity{0.723211}%
\pgfsetdash{}{0pt}%
\pgfpathmoveto{\pgfqpoint{1.400069in}{2.666849in}}%
\pgfpathcurveto{\pgfqpoint{1.408305in}{2.666849in}}{\pgfqpoint{1.416206in}{2.670121in}}{\pgfqpoint{1.422029in}{2.675945in}}%
\pgfpathcurveto{\pgfqpoint{1.427853in}{2.681769in}}{\pgfqpoint{1.431126in}{2.689669in}}{\pgfqpoint{1.431126in}{2.697905in}}%
\pgfpathcurveto{\pgfqpoint{1.431126in}{2.706141in}}{\pgfqpoint{1.427853in}{2.714042in}}{\pgfqpoint{1.422029in}{2.719865in}}%
\pgfpathcurveto{\pgfqpoint{1.416206in}{2.725689in}}{\pgfqpoint{1.408305in}{2.728962in}}{\pgfqpoint{1.400069in}{2.728962in}}%
\pgfpathcurveto{\pgfqpoint{1.391833in}{2.728962in}}{\pgfqpoint{1.383933in}{2.725689in}}{\pgfqpoint{1.378109in}{2.719865in}}%
\pgfpathcurveto{\pgfqpoint{1.372285in}{2.714042in}}{\pgfqpoint{1.369013in}{2.706141in}}{\pgfqpoint{1.369013in}{2.697905in}}%
\pgfpathcurveto{\pgfqpoint{1.369013in}{2.689669in}}{\pgfqpoint{1.372285in}{2.681769in}}{\pgfqpoint{1.378109in}{2.675945in}}%
\pgfpathcurveto{\pgfqpoint{1.383933in}{2.670121in}}{\pgfqpoint{1.391833in}{2.666849in}}{\pgfqpoint{1.400069in}{2.666849in}}%
\pgfpathclose%
\pgfusepath{stroke,fill}%
\end{pgfscope}%
\begin{pgfscope}%
\pgfpathrectangle{\pgfqpoint{0.100000in}{0.212622in}}{\pgfqpoint{3.696000in}{3.696000in}}%
\pgfusepath{clip}%
\pgfsetbuttcap%
\pgfsetroundjoin%
\definecolor{currentfill}{rgb}{0.121569,0.466667,0.705882}%
\pgfsetfillcolor{currentfill}%
\pgfsetfillopacity{0.723338}%
\pgfsetlinewidth{1.003750pt}%
\definecolor{currentstroke}{rgb}{0.121569,0.466667,0.705882}%
\pgfsetstrokecolor{currentstroke}%
\pgfsetstrokeopacity{0.723338}%
\pgfsetdash{}{0pt}%
\pgfpathmoveto{\pgfqpoint{2.600073in}{2.976188in}}%
\pgfpathcurveto{\pgfqpoint{2.608310in}{2.976188in}}{\pgfqpoint{2.616210in}{2.979460in}}{\pgfqpoint{2.622034in}{2.985284in}}%
\pgfpathcurveto{\pgfqpoint{2.627858in}{2.991108in}}{\pgfqpoint{2.631130in}{2.999008in}}{\pgfqpoint{2.631130in}{3.007244in}}%
\pgfpathcurveto{\pgfqpoint{2.631130in}{3.015481in}}{\pgfqpoint{2.627858in}{3.023381in}}{\pgfqpoint{2.622034in}{3.029205in}}%
\pgfpathcurveto{\pgfqpoint{2.616210in}{3.035029in}}{\pgfqpoint{2.608310in}{3.038301in}}{\pgfqpoint{2.600073in}{3.038301in}}%
\pgfpathcurveto{\pgfqpoint{2.591837in}{3.038301in}}{\pgfqpoint{2.583937in}{3.035029in}}{\pgfqpoint{2.578113in}{3.029205in}}%
\pgfpathcurveto{\pgfqpoint{2.572289in}{3.023381in}}{\pgfqpoint{2.569017in}{3.015481in}}{\pgfqpoint{2.569017in}{3.007244in}}%
\pgfpathcurveto{\pgfqpoint{2.569017in}{2.999008in}}{\pgfqpoint{2.572289in}{2.991108in}}{\pgfqpoint{2.578113in}{2.985284in}}%
\pgfpathcurveto{\pgfqpoint{2.583937in}{2.979460in}}{\pgfqpoint{2.591837in}{2.976188in}}{\pgfqpoint{2.600073in}{2.976188in}}%
\pgfpathclose%
\pgfusepath{stroke,fill}%
\end{pgfscope}%
\begin{pgfscope}%
\pgfpathrectangle{\pgfqpoint{0.100000in}{0.212622in}}{\pgfqpoint{3.696000in}{3.696000in}}%
\pgfusepath{clip}%
\pgfsetbuttcap%
\pgfsetroundjoin%
\definecolor{currentfill}{rgb}{0.121569,0.466667,0.705882}%
\pgfsetfillcolor{currentfill}%
\pgfsetfillopacity{0.723810}%
\pgfsetlinewidth{1.003750pt}%
\definecolor{currentstroke}{rgb}{0.121569,0.466667,0.705882}%
\pgfsetstrokecolor{currentstroke}%
\pgfsetstrokeopacity{0.723810}%
\pgfsetdash{}{0pt}%
\pgfpathmoveto{\pgfqpoint{2.598906in}{2.974802in}}%
\pgfpathcurveto{\pgfqpoint{2.607142in}{2.974802in}}{\pgfqpoint{2.615042in}{2.978074in}}{\pgfqpoint{2.620866in}{2.983898in}}%
\pgfpathcurveto{\pgfqpoint{2.626690in}{2.989722in}}{\pgfqpoint{2.629962in}{2.997622in}}{\pgfqpoint{2.629962in}{3.005858in}}%
\pgfpathcurveto{\pgfqpoint{2.629962in}{3.014094in}}{\pgfqpoint{2.626690in}{3.021994in}}{\pgfqpoint{2.620866in}{3.027818in}}%
\pgfpathcurveto{\pgfqpoint{2.615042in}{3.033642in}}{\pgfqpoint{2.607142in}{3.036915in}}{\pgfqpoint{2.598906in}{3.036915in}}%
\pgfpathcurveto{\pgfqpoint{2.590669in}{3.036915in}}{\pgfqpoint{2.582769in}{3.033642in}}{\pgfqpoint{2.576945in}{3.027818in}}%
\pgfpathcurveto{\pgfqpoint{2.571122in}{3.021994in}}{\pgfqpoint{2.567849in}{3.014094in}}{\pgfqpoint{2.567849in}{3.005858in}}%
\pgfpathcurveto{\pgfqpoint{2.567849in}{2.997622in}}{\pgfqpoint{2.571122in}{2.989722in}}{\pgfqpoint{2.576945in}{2.983898in}}%
\pgfpathcurveto{\pgfqpoint{2.582769in}{2.978074in}}{\pgfqpoint{2.590669in}{2.974802in}}{\pgfqpoint{2.598906in}{2.974802in}}%
\pgfpathclose%
\pgfusepath{stroke,fill}%
\end{pgfscope}%
\begin{pgfscope}%
\pgfpathrectangle{\pgfqpoint{0.100000in}{0.212622in}}{\pgfqpoint{3.696000in}{3.696000in}}%
\pgfusepath{clip}%
\pgfsetbuttcap%
\pgfsetroundjoin%
\definecolor{currentfill}{rgb}{0.121569,0.466667,0.705882}%
\pgfsetfillcolor{currentfill}%
\pgfsetfillopacity{0.724279}%
\pgfsetlinewidth{1.003750pt}%
\definecolor{currentstroke}{rgb}{0.121569,0.466667,0.705882}%
\pgfsetstrokecolor{currentstroke}%
\pgfsetstrokeopacity{0.724279}%
\pgfsetdash{}{0pt}%
\pgfpathmoveto{\pgfqpoint{1.396692in}{2.663742in}}%
\pgfpathcurveto{\pgfqpoint{1.404928in}{2.663742in}}{\pgfqpoint{1.412828in}{2.667015in}}{\pgfqpoint{1.418652in}{2.672839in}}%
\pgfpathcurveto{\pgfqpoint{1.424476in}{2.678663in}}{\pgfqpoint{1.427748in}{2.686563in}}{\pgfqpoint{1.427748in}{2.694799in}}%
\pgfpathcurveto{\pgfqpoint{1.427748in}{2.703035in}}{\pgfqpoint{1.424476in}{2.710935in}}{\pgfqpoint{1.418652in}{2.716759in}}%
\pgfpathcurveto{\pgfqpoint{1.412828in}{2.722583in}}{\pgfqpoint{1.404928in}{2.725855in}}{\pgfqpoint{1.396692in}{2.725855in}}%
\pgfpathcurveto{\pgfqpoint{1.388456in}{2.725855in}}{\pgfqpoint{1.380556in}{2.722583in}}{\pgfqpoint{1.374732in}{2.716759in}}%
\pgfpathcurveto{\pgfqpoint{1.368908in}{2.710935in}}{\pgfqpoint{1.365635in}{2.703035in}}{\pgfqpoint{1.365635in}{2.694799in}}%
\pgfpathcurveto{\pgfqpoint{1.365635in}{2.686563in}}{\pgfqpoint{1.368908in}{2.678663in}}{\pgfqpoint{1.374732in}{2.672839in}}%
\pgfpathcurveto{\pgfqpoint{1.380556in}{2.667015in}}{\pgfqpoint{1.388456in}{2.663742in}}{\pgfqpoint{1.396692in}{2.663742in}}%
\pgfpathclose%
\pgfusepath{stroke,fill}%
\end{pgfscope}%
\begin{pgfscope}%
\pgfpathrectangle{\pgfqpoint{0.100000in}{0.212622in}}{\pgfqpoint{3.696000in}{3.696000in}}%
\pgfusepath{clip}%
\pgfsetbuttcap%
\pgfsetroundjoin%
\definecolor{currentfill}{rgb}{0.121569,0.466667,0.705882}%
\pgfsetfillcolor{currentfill}%
\pgfsetfillopacity{0.724602}%
\pgfsetlinewidth{1.003750pt}%
\definecolor{currentstroke}{rgb}{0.121569,0.466667,0.705882}%
\pgfsetstrokecolor{currentstroke}%
\pgfsetstrokeopacity{0.724602}%
\pgfsetdash{}{0pt}%
\pgfpathmoveto{\pgfqpoint{2.596662in}{2.972005in}}%
\pgfpathcurveto{\pgfqpoint{2.604899in}{2.972005in}}{\pgfqpoint{2.612799in}{2.975278in}}{\pgfqpoint{2.618623in}{2.981102in}}%
\pgfpathcurveto{\pgfqpoint{2.624447in}{2.986926in}}{\pgfqpoint{2.627719in}{2.994826in}}{\pgfqpoint{2.627719in}{3.003062in}}%
\pgfpathcurveto{\pgfqpoint{2.627719in}{3.011298in}}{\pgfqpoint{2.624447in}{3.019198in}}{\pgfqpoint{2.618623in}{3.025022in}}%
\pgfpathcurveto{\pgfqpoint{2.612799in}{3.030846in}}{\pgfqpoint{2.604899in}{3.034118in}}{\pgfqpoint{2.596662in}{3.034118in}}%
\pgfpathcurveto{\pgfqpoint{2.588426in}{3.034118in}}{\pgfqpoint{2.580526in}{3.030846in}}{\pgfqpoint{2.574702in}{3.025022in}}%
\pgfpathcurveto{\pgfqpoint{2.568878in}{3.019198in}}{\pgfqpoint{2.565606in}{3.011298in}}{\pgfqpoint{2.565606in}{3.003062in}}%
\pgfpathcurveto{\pgfqpoint{2.565606in}{2.994826in}}{\pgfqpoint{2.568878in}{2.986926in}}{\pgfqpoint{2.574702in}{2.981102in}}%
\pgfpathcurveto{\pgfqpoint{2.580526in}{2.975278in}}{\pgfqpoint{2.588426in}{2.972005in}}{\pgfqpoint{2.596662in}{2.972005in}}%
\pgfpathclose%
\pgfusepath{stroke,fill}%
\end{pgfscope}%
\begin{pgfscope}%
\pgfpathrectangle{\pgfqpoint{0.100000in}{0.212622in}}{\pgfqpoint{3.696000in}{3.696000in}}%
\pgfusepath{clip}%
\pgfsetbuttcap%
\pgfsetroundjoin%
\definecolor{currentfill}{rgb}{0.121569,0.466667,0.705882}%
\pgfsetfillcolor{currentfill}%
\pgfsetfillopacity{0.725494}%
\pgfsetlinewidth{1.003750pt}%
\definecolor{currentstroke}{rgb}{0.121569,0.466667,0.705882}%
\pgfsetstrokecolor{currentstroke}%
\pgfsetstrokeopacity{0.725494}%
\pgfsetdash{}{0pt}%
\pgfpathmoveto{\pgfqpoint{1.392568in}{2.659262in}}%
\pgfpathcurveto{\pgfqpoint{1.400804in}{2.659262in}}{\pgfqpoint{1.408704in}{2.662534in}}{\pgfqpoint{1.414528in}{2.668358in}}%
\pgfpathcurveto{\pgfqpoint{1.420352in}{2.674182in}}{\pgfqpoint{1.423624in}{2.682082in}}{\pgfqpoint{1.423624in}{2.690318in}}%
\pgfpathcurveto{\pgfqpoint{1.423624in}{2.698555in}}{\pgfqpoint{1.420352in}{2.706455in}}{\pgfqpoint{1.414528in}{2.712279in}}%
\pgfpathcurveto{\pgfqpoint{1.408704in}{2.718102in}}{\pgfqpoint{1.400804in}{2.721375in}}{\pgfqpoint{1.392568in}{2.721375in}}%
\pgfpathcurveto{\pgfqpoint{1.384331in}{2.721375in}}{\pgfqpoint{1.376431in}{2.718102in}}{\pgfqpoint{1.370607in}{2.712279in}}%
\pgfpathcurveto{\pgfqpoint{1.364784in}{2.706455in}}{\pgfqpoint{1.361511in}{2.698555in}}{\pgfqpoint{1.361511in}{2.690318in}}%
\pgfpathcurveto{\pgfqpoint{1.361511in}{2.682082in}}{\pgfqpoint{1.364784in}{2.674182in}}{\pgfqpoint{1.370607in}{2.668358in}}%
\pgfpathcurveto{\pgfqpoint{1.376431in}{2.662534in}}{\pgfqpoint{1.384331in}{2.659262in}}{\pgfqpoint{1.392568in}{2.659262in}}%
\pgfpathclose%
\pgfusepath{stroke,fill}%
\end{pgfscope}%
\begin{pgfscope}%
\pgfpathrectangle{\pgfqpoint{0.100000in}{0.212622in}}{\pgfqpoint{3.696000in}{3.696000in}}%
\pgfusepath{clip}%
\pgfsetbuttcap%
\pgfsetroundjoin%
\definecolor{currentfill}{rgb}{0.121569,0.466667,0.705882}%
\pgfsetfillcolor{currentfill}%
\pgfsetfillopacity{0.726130}%
\pgfsetlinewidth{1.003750pt}%
\definecolor{currentstroke}{rgb}{0.121569,0.466667,0.705882}%
\pgfsetstrokecolor{currentstroke}%
\pgfsetstrokeopacity{0.726130}%
\pgfsetdash{}{0pt}%
\pgfpathmoveto{\pgfqpoint{2.592510in}{2.967562in}}%
\pgfpathcurveto{\pgfqpoint{2.600746in}{2.967562in}}{\pgfqpoint{2.608646in}{2.970835in}}{\pgfqpoint{2.614470in}{2.976659in}}%
\pgfpathcurveto{\pgfqpoint{2.620294in}{2.982482in}}{\pgfqpoint{2.623566in}{2.990383in}}{\pgfqpoint{2.623566in}{2.998619in}}%
\pgfpathcurveto{\pgfqpoint{2.623566in}{3.006855in}}{\pgfqpoint{2.620294in}{3.014755in}}{\pgfqpoint{2.614470in}{3.020579in}}%
\pgfpathcurveto{\pgfqpoint{2.608646in}{3.026403in}}{\pgfqpoint{2.600746in}{3.029675in}}{\pgfqpoint{2.592510in}{3.029675in}}%
\pgfpathcurveto{\pgfqpoint{2.584274in}{3.029675in}}{\pgfqpoint{2.576374in}{3.026403in}}{\pgfqpoint{2.570550in}{3.020579in}}%
\pgfpathcurveto{\pgfqpoint{2.564726in}{3.014755in}}{\pgfqpoint{2.561453in}{3.006855in}}{\pgfqpoint{2.561453in}{2.998619in}}%
\pgfpathcurveto{\pgfqpoint{2.561453in}{2.990383in}}{\pgfqpoint{2.564726in}{2.982482in}}{\pgfqpoint{2.570550in}{2.976659in}}%
\pgfpathcurveto{\pgfqpoint{2.576374in}{2.970835in}}{\pgfqpoint{2.584274in}{2.967562in}}{\pgfqpoint{2.592510in}{2.967562in}}%
\pgfpathclose%
\pgfusepath{stroke,fill}%
\end{pgfscope}%
\begin{pgfscope}%
\pgfpathrectangle{\pgfqpoint{0.100000in}{0.212622in}}{\pgfqpoint{3.696000in}{3.696000in}}%
\pgfusepath{clip}%
\pgfsetbuttcap%
\pgfsetroundjoin%
\definecolor{currentfill}{rgb}{0.121569,0.466667,0.705882}%
\pgfsetfillcolor{currentfill}%
\pgfsetfillopacity{0.726177}%
\pgfsetlinewidth{1.003750pt}%
\definecolor{currentstroke}{rgb}{0.121569,0.466667,0.705882}%
\pgfsetstrokecolor{currentstroke}%
\pgfsetstrokeopacity{0.726177}%
\pgfsetdash{}{0pt}%
\pgfpathmoveto{\pgfqpoint{1.390323in}{2.656840in}}%
\pgfpathcurveto{\pgfqpoint{1.398560in}{2.656840in}}{\pgfqpoint{1.406460in}{2.660112in}}{\pgfqpoint{1.412284in}{2.665936in}}%
\pgfpathcurveto{\pgfqpoint{1.418108in}{2.671760in}}{\pgfqpoint{1.421380in}{2.679660in}}{\pgfqpoint{1.421380in}{2.687897in}}%
\pgfpathcurveto{\pgfqpoint{1.421380in}{2.696133in}}{\pgfqpoint{1.418108in}{2.704033in}}{\pgfqpoint{1.412284in}{2.709857in}}%
\pgfpathcurveto{\pgfqpoint{1.406460in}{2.715681in}}{\pgfqpoint{1.398560in}{2.718953in}}{\pgfqpoint{1.390323in}{2.718953in}}%
\pgfpathcurveto{\pgfqpoint{1.382087in}{2.718953in}}{\pgfqpoint{1.374187in}{2.715681in}}{\pgfqpoint{1.368363in}{2.709857in}}%
\pgfpathcurveto{\pgfqpoint{1.362539in}{2.704033in}}{\pgfqpoint{1.359267in}{2.696133in}}{\pgfqpoint{1.359267in}{2.687897in}}%
\pgfpathcurveto{\pgfqpoint{1.359267in}{2.679660in}}{\pgfqpoint{1.362539in}{2.671760in}}{\pgfqpoint{1.368363in}{2.665936in}}%
\pgfpathcurveto{\pgfqpoint{1.374187in}{2.660112in}}{\pgfqpoint{1.382087in}{2.656840in}}{\pgfqpoint{1.390323in}{2.656840in}}%
\pgfpathclose%
\pgfusepath{stroke,fill}%
\end{pgfscope}%
\begin{pgfscope}%
\pgfpathrectangle{\pgfqpoint{0.100000in}{0.212622in}}{\pgfqpoint{3.696000in}{3.696000in}}%
\pgfusepath{clip}%
\pgfsetbuttcap%
\pgfsetroundjoin%
\definecolor{currentfill}{rgb}{0.121569,0.466667,0.705882}%
\pgfsetfillcolor{currentfill}%
\pgfsetfillopacity{0.726579}%
\pgfsetlinewidth{1.003750pt}%
\definecolor{currentstroke}{rgb}{0.121569,0.466667,0.705882}%
\pgfsetstrokecolor{currentstroke}%
\pgfsetstrokeopacity{0.726579}%
\pgfsetdash{}{0pt}%
\pgfpathmoveto{\pgfqpoint{1.389104in}{2.655639in}}%
\pgfpathcurveto{\pgfqpoint{1.397340in}{2.655639in}}{\pgfqpoint{1.405240in}{2.658911in}}{\pgfqpoint{1.411064in}{2.664735in}}%
\pgfpathcurveto{\pgfqpoint{1.416888in}{2.670559in}}{\pgfqpoint{1.420160in}{2.678459in}}{\pgfqpoint{1.420160in}{2.686695in}}%
\pgfpathcurveto{\pgfqpoint{1.420160in}{2.694931in}}{\pgfqpoint{1.416888in}{2.702831in}}{\pgfqpoint{1.411064in}{2.708655in}}%
\pgfpathcurveto{\pgfqpoint{1.405240in}{2.714479in}}{\pgfqpoint{1.397340in}{2.717752in}}{\pgfqpoint{1.389104in}{2.717752in}}%
\pgfpathcurveto{\pgfqpoint{1.380868in}{2.717752in}}{\pgfqpoint{1.372968in}{2.714479in}}{\pgfqpoint{1.367144in}{2.708655in}}%
\pgfpathcurveto{\pgfqpoint{1.361320in}{2.702831in}}{\pgfqpoint{1.358047in}{2.694931in}}{\pgfqpoint{1.358047in}{2.686695in}}%
\pgfpathcurveto{\pgfqpoint{1.358047in}{2.678459in}}{\pgfqpoint{1.361320in}{2.670559in}}{\pgfqpoint{1.367144in}{2.664735in}}%
\pgfpathcurveto{\pgfqpoint{1.372968in}{2.658911in}}{\pgfqpoint{1.380868in}{2.655639in}}{\pgfqpoint{1.389104in}{2.655639in}}%
\pgfpathclose%
\pgfusepath{stroke,fill}%
\end{pgfscope}%
\begin{pgfscope}%
\pgfpathrectangle{\pgfqpoint{0.100000in}{0.212622in}}{\pgfqpoint{3.696000in}{3.696000in}}%
\pgfusepath{clip}%
\pgfsetbuttcap%
\pgfsetroundjoin%
\definecolor{currentfill}{rgb}{0.121569,0.466667,0.705882}%
\pgfsetfillcolor{currentfill}%
\pgfsetfillopacity{0.727163}%
\pgfsetlinewidth{1.003750pt}%
\definecolor{currentstroke}{rgb}{0.121569,0.466667,0.705882}%
\pgfsetstrokecolor{currentstroke}%
\pgfsetstrokeopacity{0.727163}%
\pgfsetdash{}{0pt}%
\pgfpathmoveto{\pgfqpoint{1.387554in}{2.654389in}}%
\pgfpathcurveto{\pgfqpoint{1.395791in}{2.654389in}}{\pgfqpoint{1.403691in}{2.657662in}}{\pgfqpoint{1.409515in}{2.663485in}}%
\pgfpathcurveto{\pgfqpoint{1.415339in}{2.669309in}}{\pgfqpoint{1.418611in}{2.677209in}}{\pgfqpoint{1.418611in}{2.685446in}}%
\pgfpathcurveto{\pgfqpoint{1.418611in}{2.693682in}}{\pgfqpoint{1.415339in}{2.701582in}}{\pgfqpoint{1.409515in}{2.707406in}}%
\pgfpathcurveto{\pgfqpoint{1.403691in}{2.713230in}}{\pgfqpoint{1.395791in}{2.716502in}}{\pgfqpoint{1.387554in}{2.716502in}}%
\pgfpathcurveto{\pgfqpoint{1.379318in}{2.716502in}}{\pgfqpoint{1.371418in}{2.713230in}}{\pgfqpoint{1.365594in}{2.707406in}}%
\pgfpathcurveto{\pgfqpoint{1.359770in}{2.701582in}}{\pgfqpoint{1.356498in}{2.693682in}}{\pgfqpoint{1.356498in}{2.685446in}}%
\pgfpathcurveto{\pgfqpoint{1.356498in}{2.677209in}}{\pgfqpoint{1.359770in}{2.669309in}}{\pgfqpoint{1.365594in}{2.663485in}}%
\pgfpathcurveto{\pgfqpoint{1.371418in}{2.657662in}}{\pgfqpoint{1.379318in}{2.654389in}}{\pgfqpoint{1.387554in}{2.654389in}}%
\pgfpathclose%
\pgfusepath{stroke,fill}%
\end{pgfscope}%
\begin{pgfscope}%
\pgfpathrectangle{\pgfqpoint{0.100000in}{0.212622in}}{\pgfqpoint{3.696000in}{3.696000in}}%
\pgfusepath{clip}%
\pgfsetbuttcap%
\pgfsetroundjoin%
\definecolor{currentfill}{rgb}{0.121569,0.466667,0.705882}%
\pgfsetfillcolor{currentfill}%
\pgfsetfillopacity{0.727527}%
\pgfsetlinewidth{1.003750pt}%
\definecolor{currentstroke}{rgb}{0.121569,0.466667,0.705882}%
\pgfsetstrokecolor{currentstroke}%
\pgfsetstrokeopacity{0.727527}%
\pgfsetdash{}{0pt}%
\pgfpathmoveto{\pgfqpoint{2.588967in}{2.964144in}}%
\pgfpathcurveto{\pgfqpoint{2.597203in}{2.964144in}}{\pgfqpoint{2.605103in}{2.967417in}}{\pgfqpoint{2.610927in}{2.973240in}}%
\pgfpathcurveto{\pgfqpoint{2.616751in}{2.979064in}}{\pgfqpoint{2.620023in}{2.986964in}}{\pgfqpoint{2.620023in}{2.995201in}}%
\pgfpathcurveto{\pgfqpoint{2.620023in}{3.003437in}}{\pgfqpoint{2.616751in}{3.011337in}}{\pgfqpoint{2.610927in}{3.017161in}}%
\pgfpathcurveto{\pgfqpoint{2.605103in}{3.022985in}}{\pgfqpoint{2.597203in}{3.026257in}}{\pgfqpoint{2.588967in}{3.026257in}}%
\pgfpathcurveto{\pgfqpoint{2.580731in}{3.026257in}}{\pgfqpoint{2.572830in}{3.022985in}}{\pgfqpoint{2.567007in}{3.017161in}}%
\pgfpathcurveto{\pgfqpoint{2.561183in}{3.011337in}}{\pgfqpoint{2.557910in}{3.003437in}}{\pgfqpoint{2.557910in}{2.995201in}}%
\pgfpathcurveto{\pgfqpoint{2.557910in}{2.986964in}}{\pgfqpoint{2.561183in}{2.979064in}}{\pgfqpoint{2.567007in}{2.973240in}}%
\pgfpathcurveto{\pgfqpoint{2.572830in}{2.967417in}}{\pgfqpoint{2.580731in}{2.964144in}}{\pgfqpoint{2.588967in}{2.964144in}}%
\pgfpathclose%
\pgfusepath{stroke,fill}%
\end{pgfscope}%
\begin{pgfscope}%
\pgfpathrectangle{\pgfqpoint{0.100000in}{0.212622in}}{\pgfqpoint{3.696000in}{3.696000in}}%
\pgfusepath{clip}%
\pgfsetbuttcap%
\pgfsetroundjoin%
\definecolor{currentfill}{rgb}{0.121569,0.466667,0.705882}%
\pgfsetfillcolor{currentfill}%
\pgfsetfillopacity{0.727954}%
\pgfsetlinewidth{1.003750pt}%
\definecolor{currentstroke}{rgb}{0.121569,0.466667,0.705882}%
\pgfsetstrokecolor{currentstroke}%
\pgfsetstrokeopacity{0.727954}%
\pgfsetdash{}{0pt}%
\pgfpathmoveto{\pgfqpoint{1.385325in}{2.652425in}}%
\pgfpathcurveto{\pgfqpoint{1.393562in}{2.652425in}}{\pgfqpoint{1.401462in}{2.655697in}}{\pgfqpoint{1.407286in}{2.661521in}}%
\pgfpathcurveto{\pgfqpoint{1.413110in}{2.667345in}}{\pgfqpoint{1.416382in}{2.675245in}}{\pgfqpoint{1.416382in}{2.683482in}}%
\pgfpathcurveto{\pgfqpoint{1.416382in}{2.691718in}}{\pgfqpoint{1.413110in}{2.699618in}}{\pgfqpoint{1.407286in}{2.705442in}}%
\pgfpathcurveto{\pgfqpoint{1.401462in}{2.711266in}}{\pgfqpoint{1.393562in}{2.714538in}}{\pgfqpoint{1.385325in}{2.714538in}}%
\pgfpathcurveto{\pgfqpoint{1.377089in}{2.714538in}}{\pgfqpoint{1.369189in}{2.711266in}}{\pgfqpoint{1.363365in}{2.705442in}}%
\pgfpathcurveto{\pgfqpoint{1.357541in}{2.699618in}}{\pgfqpoint{1.354269in}{2.691718in}}{\pgfqpoint{1.354269in}{2.683482in}}%
\pgfpathcurveto{\pgfqpoint{1.354269in}{2.675245in}}{\pgfqpoint{1.357541in}{2.667345in}}{\pgfqpoint{1.363365in}{2.661521in}}%
\pgfpathcurveto{\pgfqpoint{1.369189in}{2.655697in}}{\pgfqpoint{1.377089in}{2.652425in}}{\pgfqpoint{1.385325in}{2.652425in}}%
\pgfpathclose%
\pgfusepath{stroke,fill}%
\end{pgfscope}%
\begin{pgfscope}%
\pgfpathrectangle{\pgfqpoint{0.100000in}{0.212622in}}{\pgfqpoint{3.696000in}{3.696000in}}%
\pgfusepath{clip}%
\pgfsetbuttcap%
\pgfsetroundjoin%
\definecolor{currentfill}{rgb}{0.121569,0.466667,0.705882}%
\pgfsetfillcolor{currentfill}%
\pgfsetfillopacity{0.728244}%
\pgfsetlinewidth{1.003750pt}%
\definecolor{currentstroke}{rgb}{0.121569,0.466667,0.705882}%
\pgfsetstrokecolor{currentstroke}%
\pgfsetstrokeopacity{0.728244}%
\pgfsetdash{}{0pt}%
\pgfpathmoveto{\pgfqpoint{3.009748in}{1.702744in}}%
\pgfpathcurveto{\pgfqpoint{3.017985in}{1.702744in}}{\pgfqpoint{3.025885in}{1.706017in}}{\pgfqpoint{3.031709in}{1.711841in}}%
\pgfpathcurveto{\pgfqpoint{3.037533in}{1.717665in}}{\pgfqpoint{3.040805in}{1.725565in}}{\pgfqpoint{3.040805in}{1.733801in}}%
\pgfpathcurveto{\pgfqpoint{3.040805in}{1.742037in}}{\pgfqpoint{3.037533in}{1.749937in}}{\pgfqpoint{3.031709in}{1.755761in}}%
\pgfpathcurveto{\pgfqpoint{3.025885in}{1.761585in}}{\pgfqpoint{3.017985in}{1.764857in}}{\pgfqpoint{3.009748in}{1.764857in}}%
\pgfpathcurveto{\pgfqpoint{3.001512in}{1.764857in}}{\pgfqpoint{2.993612in}{1.761585in}}{\pgfqpoint{2.987788in}{1.755761in}}%
\pgfpathcurveto{\pgfqpoint{2.981964in}{1.749937in}}{\pgfqpoint{2.978692in}{1.742037in}}{\pgfqpoint{2.978692in}{1.733801in}}%
\pgfpathcurveto{\pgfqpoint{2.978692in}{1.725565in}}{\pgfqpoint{2.981964in}{1.717665in}}{\pgfqpoint{2.987788in}{1.711841in}}%
\pgfpathcurveto{\pgfqpoint{2.993612in}{1.706017in}}{\pgfqpoint{3.001512in}{1.702744in}}{\pgfqpoint{3.009748in}{1.702744in}}%
\pgfpathclose%
\pgfusepath{stroke,fill}%
\end{pgfscope}%
\begin{pgfscope}%
\pgfpathrectangle{\pgfqpoint{0.100000in}{0.212622in}}{\pgfqpoint{3.696000in}{3.696000in}}%
\pgfusepath{clip}%
\pgfsetbuttcap%
\pgfsetroundjoin%
\definecolor{currentfill}{rgb}{0.121569,0.466667,0.705882}%
\pgfsetfillcolor{currentfill}%
\pgfsetfillopacity{0.728826}%
\pgfsetlinewidth{1.003750pt}%
\definecolor{currentstroke}{rgb}{0.121569,0.466667,0.705882}%
\pgfsetstrokecolor{currentstroke}%
\pgfsetstrokeopacity{0.728826}%
\pgfsetdash{}{0pt}%
\pgfpathmoveto{\pgfqpoint{2.585885in}{2.961352in}}%
\pgfpathcurveto{\pgfqpoint{2.594122in}{2.961352in}}{\pgfqpoint{2.602022in}{2.964624in}}{\pgfqpoint{2.607846in}{2.970448in}}%
\pgfpathcurveto{\pgfqpoint{2.613670in}{2.976272in}}{\pgfqpoint{2.616942in}{2.984172in}}{\pgfqpoint{2.616942in}{2.992409in}}%
\pgfpathcurveto{\pgfqpoint{2.616942in}{3.000645in}}{\pgfqpoint{2.613670in}{3.008545in}}{\pgfqpoint{2.607846in}{3.014369in}}%
\pgfpathcurveto{\pgfqpoint{2.602022in}{3.020193in}}{\pgfqpoint{2.594122in}{3.023465in}}{\pgfqpoint{2.585885in}{3.023465in}}%
\pgfpathcurveto{\pgfqpoint{2.577649in}{3.023465in}}{\pgfqpoint{2.569749in}{3.020193in}}{\pgfqpoint{2.563925in}{3.014369in}}%
\pgfpathcurveto{\pgfqpoint{2.558101in}{3.008545in}}{\pgfqpoint{2.554829in}{3.000645in}}{\pgfqpoint{2.554829in}{2.992409in}}%
\pgfpathcurveto{\pgfqpoint{2.554829in}{2.984172in}}{\pgfqpoint{2.558101in}{2.976272in}}{\pgfqpoint{2.563925in}{2.970448in}}%
\pgfpathcurveto{\pgfqpoint{2.569749in}{2.964624in}}{\pgfqpoint{2.577649in}{2.961352in}}{\pgfqpoint{2.585885in}{2.961352in}}%
\pgfpathclose%
\pgfusepath{stroke,fill}%
\end{pgfscope}%
\begin{pgfscope}%
\pgfpathrectangle{\pgfqpoint{0.100000in}{0.212622in}}{\pgfqpoint{3.696000in}{3.696000in}}%
\pgfusepath{clip}%
\pgfsetbuttcap%
\pgfsetroundjoin%
\definecolor{currentfill}{rgb}{0.121569,0.466667,0.705882}%
\pgfsetfillcolor{currentfill}%
\pgfsetfillopacity{0.728913}%
\pgfsetlinewidth{1.003750pt}%
\definecolor{currentstroke}{rgb}{0.121569,0.466667,0.705882}%
\pgfsetstrokecolor{currentstroke}%
\pgfsetstrokeopacity{0.728913}%
\pgfsetdash{}{0pt}%
\pgfpathmoveto{\pgfqpoint{1.382390in}{2.649452in}}%
\pgfpathcurveto{\pgfqpoint{1.390627in}{2.649452in}}{\pgfqpoint{1.398527in}{2.652725in}}{\pgfqpoint{1.404351in}{2.658549in}}%
\pgfpathcurveto{\pgfqpoint{1.410175in}{2.664372in}}{\pgfqpoint{1.413447in}{2.672273in}}{\pgfqpoint{1.413447in}{2.680509in}}%
\pgfpathcurveto{\pgfqpoint{1.413447in}{2.688745in}}{\pgfqpoint{1.410175in}{2.696645in}}{\pgfqpoint{1.404351in}{2.702469in}}%
\pgfpathcurveto{\pgfqpoint{1.398527in}{2.708293in}}{\pgfqpoint{1.390627in}{2.711565in}}{\pgfqpoint{1.382390in}{2.711565in}}%
\pgfpathcurveto{\pgfqpoint{1.374154in}{2.711565in}}{\pgfqpoint{1.366254in}{2.708293in}}{\pgfqpoint{1.360430in}{2.702469in}}%
\pgfpathcurveto{\pgfqpoint{1.354606in}{2.696645in}}{\pgfqpoint{1.351334in}{2.688745in}}{\pgfqpoint{1.351334in}{2.680509in}}%
\pgfpathcurveto{\pgfqpoint{1.351334in}{2.672273in}}{\pgfqpoint{1.354606in}{2.664372in}}{\pgfqpoint{1.360430in}{2.658549in}}%
\pgfpathcurveto{\pgfqpoint{1.366254in}{2.652725in}}{\pgfqpoint{1.374154in}{2.649452in}}{\pgfqpoint{1.382390in}{2.649452in}}%
\pgfpathclose%
\pgfusepath{stroke,fill}%
\end{pgfscope}%
\begin{pgfscope}%
\pgfpathrectangle{\pgfqpoint{0.100000in}{0.212622in}}{\pgfqpoint{3.696000in}{3.696000in}}%
\pgfusepath{clip}%
\pgfsetbuttcap%
\pgfsetroundjoin%
\definecolor{currentfill}{rgb}{0.121569,0.466667,0.705882}%
\pgfsetfillcolor{currentfill}%
\pgfsetfillopacity{0.729385}%
\pgfsetlinewidth{1.003750pt}%
\definecolor{currentstroke}{rgb}{0.121569,0.466667,0.705882}%
\pgfsetstrokecolor{currentstroke}%
\pgfsetstrokeopacity{0.729385}%
\pgfsetdash{}{0pt}%
\pgfpathmoveto{\pgfqpoint{1.380800in}{2.647455in}}%
\pgfpathcurveto{\pgfqpoint{1.389036in}{2.647455in}}{\pgfqpoint{1.396936in}{2.650727in}}{\pgfqpoint{1.402760in}{2.656551in}}%
\pgfpathcurveto{\pgfqpoint{1.408584in}{2.662375in}}{\pgfqpoint{1.411856in}{2.670275in}}{\pgfqpoint{1.411856in}{2.678512in}}%
\pgfpathcurveto{\pgfqpoint{1.411856in}{2.686748in}}{\pgfqpoint{1.408584in}{2.694648in}}{\pgfqpoint{1.402760in}{2.700472in}}%
\pgfpathcurveto{\pgfqpoint{1.396936in}{2.706296in}}{\pgfqpoint{1.389036in}{2.709568in}}{\pgfqpoint{1.380800in}{2.709568in}}%
\pgfpathcurveto{\pgfqpoint{1.372563in}{2.709568in}}{\pgfqpoint{1.364663in}{2.706296in}}{\pgfqpoint{1.358839in}{2.700472in}}%
\pgfpathcurveto{\pgfqpoint{1.353016in}{2.694648in}}{\pgfqpoint{1.349743in}{2.686748in}}{\pgfqpoint{1.349743in}{2.678512in}}%
\pgfpathcurveto{\pgfqpoint{1.349743in}{2.670275in}}{\pgfqpoint{1.353016in}{2.662375in}}{\pgfqpoint{1.358839in}{2.656551in}}%
\pgfpathcurveto{\pgfqpoint{1.364663in}{2.650727in}}{\pgfqpoint{1.372563in}{2.647455in}}{\pgfqpoint{1.380800in}{2.647455in}}%
\pgfpathclose%
\pgfusepath{stroke,fill}%
\end{pgfscope}%
\begin{pgfscope}%
\pgfpathrectangle{\pgfqpoint{0.100000in}{0.212622in}}{\pgfqpoint{3.696000in}{3.696000in}}%
\pgfusepath{clip}%
\pgfsetbuttcap%
\pgfsetroundjoin%
\definecolor{currentfill}{rgb}{0.121569,0.466667,0.705882}%
\pgfsetfillcolor{currentfill}%
\pgfsetfillopacity{0.729899}%
\pgfsetlinewidth{1.003750pt}%
\definecolor{currentstroke}{rgb}{0.121569,0.466667,0.705882}%
\pgfsetstrokecolor{currentstroke}%
\pgfsetstrokeopacity{0.729899}%
\pgfsetdash{}{0pt}%
\pgfpathmoveto{\pgfqpoint{2.583220in}{2.958329in}}%
\pgfpathcurveto{\pgfqpoint{2.591456in}{2.958329in}}{\pgfqpoint{2.599356in}{2.961602in}}{\pgfqpoint{2.605180in}{2.967426in}}%
\pgfpathcurveto{\pgfqpoint{2.611004in}{2.973250in}}{\pgfqpoint{2.614277in}{2.981150in}}{\pgfqpoint{2.614277in}{2.989386in}}%
\pgfpathcurveto{\pgfqpoint{2.614277in}{2.997622in}}{\pgfqpoint{2.611004in}{3.005522in}}{\pgfqpoint{2.605180in}{3.011346in}}%
\pgfpathcurveto{\pgfqpoint{2.599356in}{3.017170in}}{\pgfqpoint{2.591456in}{3.020442in}}{\pgfqpoint{2.583220in}{3.020442in}}%
\pgfpathcurveto{\pgfqpoint{2.574984in}{3.020442in}}{\pgfqpoint{2.567084in}{3.017170in}}{\pgfqpoint{2.561260in}{3.011346in}}%
\pgfpathcurveto{\pgfqpoint{2.555436in}{3.005522in}}{\pgfqpoint{2.552164in}{2.997622in}}{\pgfqpoint{2.552164in}{2.989386in}}%
\pgfpathcurveto{\pgfqpoint{2.552164in}{2.981150in}}{\pgfqpoint{2.555436in}{2.973250in}}{\pgfqpoint{2.561260in}{2.967426in}}%
\pgfpathcurveto{\pgfqpoint{2.567084in}{2.961602in}}{\pgfqpoint{2.574984in}{2.958329in}}{\pgfqpoint{2.583220in}{2.958329in}}%
\pgfpathclose%
\pgfusepath{stroke,fill}%
\end{pgfscope}%
\begin{pgfscope}%
\pgfpathrectangle{\pgfqpoint{0.100000in}{0.212622in}}{\pgfqpoint{3.696000in}{3.696000in}}%
\pgfusepath{clip}%
\pgfsetbuttcap%
\pgfsetroundjoin%
\definecolor{currentfill}{rgb}{0.121569,0.466667,0.705882}%
\pgfsetfillcolor{currentfill}%
\pgfsetfillopacity{0.729986}%
\pgfsetlinewidth{1.003750pt}%
\definecolor{currentstroke}{rgb}{0.121569,0.466667,0.705882}%
\pgfsetstrokecolor{currentstroke}%
\pgfsetstrokeopacity{0.729986}%
\pgfsetdash{}{0pt}%
\pgfpathmoveto{\pgfqpoint{1.378910in}{2.645188in}}%
\pgfpathcurveto{\pgfqpoint{1.387147in}{2.645188in}}{\pgfqpoint{1.395047in}{2.648460in}}{\pgfqpoint{1.400871in}{2.654284in}}%
\pgfpathcurveto{\pgfqpoint{1.406695in}{2.660108in}}{\pgfqpoint{1.409967in}{2.668008in}}{\pgfqpoint{1.409967in}{2.676244in}}%
\pgfpathcurveto{\pgfqpoint{1.409967in}{2.684481in}}{\pgfqpoint{1.406695in}{2.692381in}}{\pgfqpoint{1.400871in}{2.698205in}}%
\pgfpathcurveto{\pgfqpoint{1.395047in}{2.704029in}}{\pgfqpoint{1.387147in}{2.707301in}}{\pgfqpoint{1.378910in}{2.707301in}}%
\pgfpathcurveto{\pgfqpoint{1.370674in}{2.707301in}}{\pgfqpoint{1.362774in}{2.704029in}}{\pgfqpoint{1.356950in}{2.698205in}}%
\pgfpathcurveto{\pgfqpoint{1.351126in}{2.692381in}}{\pgfqpoint{1.347854in}{2.684481in}}{\pgfqpoint{1.347854in}{2.676244in}}%
\pgfpathcurveto{\pgfqpoint{1.347854in}{2.668008in}}{\pgfqpoint{1.351126in}{2.660108in}}{\pgfqpoint{1.356950in}{2.654284in}}%
\pgfpathcurveto{\pgfqpoint{1.362774in}{2.648460in}}{\pgfqpoint{1.370674in}{2.645188in}}{\pgfqpoint{1.378910in}{2.645188in}}%
\pgfpathclose%
\pgfusepath{stroke,fill}%
\end{pgfscope}%
\begin{pgfscope}%
\pgfpathrectangle{\pgfqpoint{0.100000in}{0.212622in}}{\pgfqpoint{3.696000in}{3.696000in}}%
\pgfusepath{clip}%
\pgfsetbuttcap%
\pgfsetroundjoin%
\definecolor{currentfill}{rgb}{0.121569,0.466667,0.705882}%
\pgfsetfillcolor{currentfill}%
\pgfsetfillopacity{0.730341}%
\pgfsetlinewidth{1.003750pt}%
\definecolor{currentstroke}{rgb}{0.121569,0.466667,0.705882}%
\pgfsetstrokecolor{currentstroke}%
\pgfsetstrokeopacity{0.730341}%
\pgfsetdash{}{0pt}%
\pgfpathmoveto{\pgfqpoint{1.377872in}{2.644086in}}%
\pgfpathcurveto{\pgfqpoint{1.386108in}{2.644086in}}{\pgfqpoint{1.394008in}{2.647358in}}{\pgfqpoint{1.399832in}{2.653182in}}%
\pgfpathcurveto{\pgfqpoint{1.405656in}{2.659006in}}{\pgfqpoint{1.408928in}{2.666906in}}{\pgfqpoint{1.408928in}{2.675142in}}%
\pgfpathcurveto{\pgfqpoint{1.408928in}{2.683378in}}{\pgfqpoint{1.405656in}{2.691278in}}{\pgfqpoint{1.399832in}{2.697102in}}%
\pgfpathcurveto{\pgfqpoint{1.394008in}{2.702926in}}{\pgfqpoint{1.386108in}{2.706199in}}{\pgfqpoint{1.377872in}{2.706199in}}%
\pgfpathcurveto{\pgfqpoint{1.369636in}{2.706199in}}{\pgfqpoint{1.361736in}{2.702926in}}{\pgfqpoint{1.355912in}{2.697102in}}%
\pgfpathcurveto{\pgfqpoint{1.350088in}{2.691278in}}{\pgfqpoint{1.346815in}{2.683378in}}{\pgfqpoint{1.346815in}{2.675142in}}%
\pgfpathcurveto{\pgfqpoint{1.346815in}{2.666906in}}{\pgfqpoint{1.350088in}{2.659006in}}{\pgfqpoint{1.355912in}{2.653182in}}%
\pgfpathcurveto{\pgfqpoint{1.361736in}{2.647358in}}{\pgfqpoint{1.369636in}{2.644086in}}{\pgfqpoint{1.377872in}{2.644086in}}%
\pgfpathclose%
\pgfusepath{stroke,fill}%
\end{pgfscope}%
\begin{pgfscope}%
\pgfpathrectangle{\pgfqpoint{0.100000in}{0.212622in}}{\pgfqpoint{3.696000in}{3.696000in}}%
\pgfusepath{clip}%
\pgfsetbuttcap%
\pgfsetroundjoin%
\definecolor{currentfill}{rgb}{0.121569,0.466667,0.705882}%
\pgfsetfillcolor{currentfill}%
\pgfsetfillopacity{0.730533}%
\pgfsetlinewidth{1.003750pt}%
\definecolor{currentstroke}{rgb}{0.121569,0.466667,0.705882}%
\pgfsetstrokecolor{currentstroke}%
\pgfsetstrokeopacity{0.730533}%
\pgfsetdash{}{0pt}%
\pgfpathmoveto{\pgfqpoint{1.377294in}{2.643471in}}%
\pgfpathcurveto{\pgfqpoint{1.385530in}{2.643471in}}{\pgfqpoint{1.393430in}{2.646743in}}{\pgfqpoint{1.399254in}{2.652567in}}%
\pgfpathcurveto{\pgfqpoint{1.405078in}{2.658391in}}{\pgfqpoint{1.408350in}{2.666291in}}{\pgfqpoint{1.408350in}{2.674527in}}%
\pgfpathcurveto{\pgfqpoint{1.408350in}{2.682764in}}{\pgfqpoint{1.405078in}{2.690664in}}{\pgfqpoint{1.399254in}{2.696488in}}%
\pgfpathcurveto{\pgfqpoint{1.393430in}{2.702311in}}{\pgfqpoint{1.385530in}{2.705584in}}{\pgfqpoint{1.377294in}{2.705584in}}%
\pgfpathcurveto{\pgfqpoint{1.369057in}{2.705584in}}{\pgfqpoint{1.361157in}{2.702311in}}{\pgfqpoint{1.355333in}{2.696488in}}%
\pgfpathcurveto{\pgfqpoint{1.349509in}{2.690664in}}{\pgfqpoint{1.346237in}{2.682764in}}{\pgfqpoint{1.346237in}{2.674527in}}%
\pgfpathcurveto{\pgfqpoint{1.346237in}{2.666291in}}{\pgfqpoint{1.349509in}{2.658391in}}{\pgfqpoint{1.355333in}{2.652567in}}%
\pgfpathcurveto{\pgfqpoint{1.361157in}{2.646743in}}{\pgfqpoint{1.369057in}{2.643471in}}{\pgfqpoint{1.377294in}{2.643471in}}%
\pgfpathclose%
\pgfusepath{stroke,fill}%
\end{pgfscope}%
\begin{pgfscope}%
\pgfpathrectangle{\pgfqpoint{0.100000in}{0.212622in}}{\pgfqpoint{3.696000in}{3.696000in}}%
\pgfusepath{clip}%
\pgfsetbuttcap%
\pgfsetroundjoin%
\definecolor{currentfill}{rgb}{0.121569,0.466667,0.705882}%
\pgfsetfillcolor{currentfill}%
\pgfsetfillopacity{0.730624}%
\pgfsetlinewidth{1.003750pt}%
\definecolor{currentstroke}{rgb}{0.121569,0.466667,0.705882}%
\pgfsetstrokecolor{currentstroke}%
\pgfsetstrokeopacity{0.730624}%
\pgfsetdash{}{0pt}%
\pgfpathmoveto{\pgfqpoint{2.581382in}{2.956095in}}%
\pgfpathcurveto{\pgfqpoint{2.589618in}{2.956095in}}{\pgfqpoint{2.597518in}{2.959367in}}{\pgfqpoint{2.603342in}{2.965191in}}%
\pgfpathcurveto{\pgfqpoint{2.609166in}{2.971015in}}{\pgfqpoint{2.612439in}{2.978915in}}{\pgfqpoint{2.612439in}{2.987151in}}%
\pgfpathcurveto{\pgfqpoint{2.612439in}{2.995388in}}{\pgfqpoint{2.609166in}{3.003288in}}{\pgfqpoint{2.603342in}{3.009112in}}%
\pgfpathcurveto{\pgfqpoint{2.597518in}{3.014936in}}{\pgfqpoint{2.589618in}{3.018208in}}{\pgfqpoint{2.581382in}{3.018208in}}%
\pgfpathcurveto{\pgfqpoint{2.573146in}{3.018208in}}{\pgfqpoint{2.565246in}{3.014936in}}{\pgfqpoint{2.559422in}{3.009112in}}%
\pgfpathcurveto{\pgfqpoint{2.553598in}{3.003288in}}{\pgfqpoint{2.550326in}{2.995388in}}{\pgfqpoint{2.550326in}{2.987151in}}%
\pgfpathcurveto{\pgfqpoint{2.550326in}{2.978915in}}{\pgfqpoint{2.553598in}{2.971015in}}{\pgfqpoint{2.559422in}{2.965191in}}%
\pgfpathcurveto{\pgfqpoint{2.565246in}{2.959367in}}{\pgfqpoint{2.573146in}{2.956095in}}{\pgfqpoint{2.581382in}{2.956095in}}%
\pgfpathclose%
\pgfusepath{stroke,fill}%
\end{pgfscope}%
\begin{pgfscope}%
\pgfpathrectangle{\pgfqpoint{0.100000in}{0.212622in}}{\pgfqpoint{3.696000in}{3.696000in}}%
\pgfusepath{clip}%
\pgfsetbuttcap%
\pgfsetroundjoin%
\definecolor{currentfill}{rgb}{0.121569,0.466667,0.705882}%
\pgfsetfillcolor{currentfill}%
\pgfsetfillopacity{0.730950}%
\pgfsetlinewidth{1.003750pt}%
\definecolor{currentstroke}{rgb}{0.121569,0.466667,0.705882}%
\pgfsetstrokecolor{currentstroke}%
\pgfsetstrokeopacity{0.730950}%
\pgfsetdash{}{0pt}%
\pgfpathmoveto{\pgfqpoint{1.375959in}{2.641954in}}%
\pgfpathcurveto{\pgfqpoint{1.384195in}{2.641954in}}{\pgfqpoint{1.392095in}{2.645226in}}{\pgfqpoint{1.397919in}{2.651050in}}%
\pgfpathcurveto{\pgfqpoint{1.403743in}{2.656874in}}{\pgfqpoint{1.407015in}{2.664774in}}{\pgfqpoint{1.407015in}{2.673011in}}%
\pgfpathcurveto{\pgfqpoint{1.407015in}{2.681247in}}{\pgfqpoint{1.403743in}{2.689147in}}{\pgfqpoint{1.397919in}{2.694971in}}%
\pgfpathcurveto{\pgfqpoint{1.392095in}{2.700795in}}{\pgfqpoint{1.384195in}{2.704067in}}{\pgfqpoint{1.375959in}{2.704067in}}%
\pgfpathcurveto{\pgfqpoint{1.367723in}{2.704067in}}{\pgfqpoint{1.359823in}{2.700795in}}{\pgfqpoint{1.353999in}{2.694971in}}%
\pgfpathcurveto{\pgfqpoint{1.348175in}{2.689147in}}{\pgfqpoint{1.344902in}{2.681247in}}{\pgfqpoint{1.344902in}{2.673011in}}%
\pgfpathcurveto{\pgfqpoint{1.344902in}{2.664774in}}{\pgfqpoint{1.348175in}{2.656874in}}{\pgfqpoint{1.353999in}{2.651050in}}%
\pgfpathcurveto{\pgfqpoint{1.359823in}{2.645226in}}{\pgfqpoint{1.367723in}{2.641954in}}{\pgfqpoint{1.375959in}{2.641954in}}%
\pgfpathclose%
\pgfusepath{stroke,fill}%
\end{pgfscope}%
\begin{pgfscope}%
\pgfpathrectangle{\pgfqpoint{0.100000in}{0.212622in}}{\pgfqpoint{3.696000in}{3.696000in}}%
\pgfusepath{clip}%
\pgfsetbuttcap%
\pgfsetroundjoin%
\definecolor{currentfill}{rgb}{0.121569,0.466667,0.705882}%
\pgfsetfillcolor{currentfill}%
\pgfsetfillopacity{0.731257}%
\pgfsetlinewidth{1.003750pt}%
\definecolor{currentstroke}{rgb}{0.121569,0.466667,0.705882}%
\pgfsetstrokecolor{currentstroke}%
\pgfsetstrokeopacity{0.731257}%
\pgfsetdash{}{0pt}%
\pgfpathmoveto{\pgfqpoint{2.579644in}{2.954256in}}%
\pgfpathcurveto{\pgfqpoint{2.587881in}{2.954256in}}{\pgfqpoint{2.595781in}{2.957529in}}{\pgfqpoint{2.601605in}{2.963352in}}%
\pgfpathcurveto{\pgfqpoint{2.607429in}{2.969176in}}{\pgfqpoint{2.610701in}{2.977076in}}{\pgfqpoint{2.610701in}{2.985313in}}%
\pgfpathcurveto{\pgfqpoint{2.610701in}{2.993549in}}{\pgfqpoint{2.607429in}{3.001449in}}{\pgfqpoint{2.601605in}{3.007273in}}%
\pgfpathcurveto{\pgfqpoint{2.595781in}{3.013097in}}{\pgfqpoint{2.587881in}{3.016369in}}{\pgfqpoint{2.579644in}{3.016369in}}%
\pgfpathcurveto{\pgfqpoint{2.571408in}{3.016369in}}{\pgfqpoint{2.563508in}{3.013097in}}{\pgfqpoint{2.557684in}{3.007273in}}%
\pgfpathcurveto{\pgfqpoint{2.551860in}{3.001449in}}{\pgfqpoint{2.548588in}{2.993549in}}{\pgfqpoint{2.548588in}{2.985313in}}%
\pgfpathcurveto{\pgfqpoint{2.548588in}{2.977076in}}{\pgfqpoint{2.551860in}{2.969176in}}{\pgfqpoint{2.557684in}{2.963352in}}%
\pgfpathcurveto{\pgfqpoint{2.563508in}{2.957529in}}{\pgfqpoint{2.571408in}{2.954256in}}{\pgfqpoint{2.579644in}{2.954256in}}%
\pgfpathclose%
\pgfusepath{stroke,fill}%
\end{pgfscope}%
\begin{pgfscope}%
\pgfpathrectangle{\pgfqpoint{0.100000in}{0.212622in}}{\pgfqpoint{3.696000in}{3.696000in}}%
\pgfusepath{clip}%
\pgfsetbuttcap%
\pgfsetroundjoin%
\definecolor{currentfill}{rgb}{0.121569,0.466667,0.705882}%
\pgfsetfillcolor{currentfill}%
\pgfsetfillopacity{0.731397}%
\pgfsetlinewidth{1.003750pt}%
\definecolor{currentstroke}{rgb}{0.121569,0.466667,0.705882}%
\pgfsetstrokecolor{currentstroke}%
\pgfsetstrokeopacity{0.731397}%
\pgfsetdash{}{0pt}%
\pgfpathmoveto{\pgfqpoint{1.374394in}{2.639979in}}%
\pgfpathcurveto{\pgfqpoint{1.382630in}{2.639979in}}{\pgfqpoint{1.390530in}{2.643251in}}{\pgfqpoint{1.396354in}{2.649075in}}%
\pgfpathcurveto{\pgfqpoint{1.402178in}{2.654899in}}{\pgfqpoint{1.405450in}{2.662799in}}{\pgfqpoint{1.405450in}{2.671036in}}%
\pgfpathcurveto{\pgfqpoint{1.405450in}{2.679272in}}{\pgfqpoint{1.402178in}{2.687172in}}{\pgfqpoint{1.396354in}{2.692996in}}%
\pgfpathcurveto{\pgfqpoint{1.390530in}{2.698820in}}{\pgfqpoint{1.382630in}{2.702092in}}{\pgfqpoint{1.374394in}{2.702092in}}%
\pgfpathcurveto{\pgfqpoint{1.366157in}{2.702092in}}{\pgfqpoint{1.358257in}{2.698820in}}{\pgfqpoint{1.352433in}{2.692996in}}%
\pgfpathcurveto{\pgfqpoint{1.346609in}{2.687172in}}{\pgfqpoint{1.343337in}{2.679272in}}{\pgfqpoint{1.343337in}{2.671036in}}%
\pgfpathcurveto{\pgfqpoint{1.343337in}{2.662799in}}{\pgfqpoint{1.346609in}{2.654899in}}{\pgfqpoint{1.352433in}{2.649075in}}%
\pgfpathcurveto{\pgfqpoint{1.358257in}{2.643251in}}{\pgfqpoint{1.366157in}{2.639979in}}{\pgfqpoint{1.374394in}{2.639979in}}%
\pgfpathclose%
\pgfusepath{stroke,fill}%
\end{pgfscope}%
\begin{pgfscope}%
\pgfpathrectangle{\pgfqpoint{0.100000in}{0.212622in}}{\pgfqpoint{3.696000in}{3.696000in}}%
\pgfusepath{clip}%
\pgfsetbuttcap%
\pgfsetroundjoin%
\definecolor{currentfill}{rgb}{0.121569,0.466667,0.705882}%
\pgfsetfillcolor{currentfill}%
\pgfsetfillopacity{0.731696}%
\pgfsetlinewidth{1.003750pt}%
\definecolor{currentstroke}{rgb}{0.121569,0.466667,0.705882}%
\pgfsetstrokecolor{currentstroke}%
\pgfsetstrokeopacity{0.731696}%
\pgfsetdash{}{0pt}%
\pgfpathmoveto{\pgfqpoint{2.578504in}{2.953150in}}%
\pgfpathcurveto{\pgfqpoint{2.586740in}{2.953150in}}{\pgfqpoint{2.594641in}{2.956422in}}{\pgfqpoint{2.600464in}{2.962246in}}%
\pgfpathcurveto{\pgfqpoint{2.606288in}{2.968070in}}{\pgfqpoint{2.609561in}{2.975970in}}{\pgfqpoint{2.609561in}{2.984206in}}%
\pgfpathcurveto{\pgfqpoint{2.609561in}{2.992442in}}{\pgfqpoint{2.606288in}{3.000342in}}{\pgfqpoint{2.600464in}{3.006166in}}%
\pgfpathcurveto{\pgfqpoint{2.594641in}{3.011990in}}{\pgfqpoint{2.586740in}{3.015263in}}{\pgfqpoint{2.578504in}{3.015263in}}%
\pgfpathcurveto{\pgfqpoint{2.570268in}{3.015263in}}{\pgfqpoint{2.562368in}{3.011990in}}{\pgfqpoint{2.556544in}{3.006166in}}%
\pgfpathcurveto{\pgfqpoint{2.550720in}{3.000342in}}{\pgfqpoint{2.547448in}{2.992442in}}{\pgfqpoint{2.547448in}{2.984206in}}%
\pgfpathcurveto{\pgfqpoint{2.547448in}{2.975970in}}{\pgfqpoint{2.550720in}{2.968070in}}{\pgfqpoint{2.556544in}{2.962246in}}%
\pgfpathcurveto{\pgfqpoint{2.562368in}{2.956422in}}{\pgfqpoint{2.570268in}{2.953150in}}{\pgfqpoint{2.578504in}{2.953150in}}%
\pgfpathclose%
\pgfusepath{stroke,fill}%
\end{pgfscope}%
\begin{pgfscope}%
\pgfpathrectangle{\pgfqpoint{0.100000in}{0.212622in}}{\pgfqpoint{3.696000in}{3.696000in}}%
\pgfusepath{clip}%
\pgfsetbuttcap%
\pgfsetroundjoin%
\definecolor{currentfill}{rgb}{0.121569,0.466667,0.705882}%
\pgfsetfillcolor{currentfill}%
\pgfsetfillopacity{0.732006}%
\pgfsetlinewidth{1.003750pt}%
\definecolor{currentstroke}{rgb}{0.121569,0.466667,0.705882}%
\pgfsetstrokecolor{currentstroke}%
\pgfsetstrokeopacity{0.732006}%
\pgfsetdash{}{0pt}%
\pgfpathmoveto{\pgfqpoint{1.372383in}{2.637538in}}%
\pgfpathcurveto{\pgfqpoint{1.380619in}{2.637538in}}{\pgfqpoint{1.388519in}{2.640810in}}{\pgfqpoint{1.394343in}{2.646634in}}%
\pgfpathcurveto{\pgfqpoint{1.400167in}{2.652458in}}{\pgfqpoint{1.403440in}{2.660358in}}{\pgfqpoint{1.403440in}{2.668594in}}%
\pgfpathcurveto{\pgfqpoint{1.403440in}{2.676830in}}{\pgfqpoint{1.400167in}{2.684730in}}{\pgfqpoint{1.394343in}{2.690554in}}%
\pgfpathcurveto{\pgfqpoint{1.388519in}{2.696378in}}{\pgfqpoint{1.380619in}{2.699651in}}{\pgfqpoint{1.372383in}{2.699651in}}%
\pgfpathcurveto{\pgfqpoint{1.364147in}{2.699651in}}{\pgfqpoint{1.356247in}{2.696378in}}{\pgfqpoint{1.350423in}{2.690554in}}%
\pgfpathcurveto{\pgfqpoint{1.344599in}{2.684730in}}{\pgfqpoint{1.341327in}{2.676830in}}{\pgfqpoint{1.341327in}{2.668594in}}%
\pgfpathcurveto{\pgfqpoint{1.341327in}{2.660358in}}{\pgfqpoint{1.344599in}{2.652458in}}{\pgfqpoint{1.350423in}{2.646634in}}%
\pgfpathcurveto{\pgfqpoint{1.356247in}{2.640810in}}{\pgfqpoint{1.364147in}{2.637538in}}{\pgfqpoint{1.372383in}{2.637538in}}%
\pgfpathclose%
\pgfusepath{stroke,fill}%
\end{pgfscope}%
\begin{pgfscope}%
\pgfpathrectangle{\pgfqpoint{0.100000in}{0.212622in}}{\pgfqpoint{3.696000in}{3.696000in}}%
\pgfusepath{clip}%
\pgfsetbuttcap%
\pgfsetroundjoin%
\definecolor{currentfill}{rgb}{0.121569,0.466667,0.705882}%
\pgfsetfillcolor{currentfill}%
\pgfsetfillopacity{0.732516}%
\pgfsetlinewidth{1.003750pt}%
\definecolor{currentstroke}{rgb}{0.121569,0.466667,0.705882}%
\pgfsetstrokecolor{currentstroke}%
\pgfsetstrokeopacity{0.732516}%
\pgfsetdash{}{0pt}%
\pgfpathmoveto{\pgfqpoint{2.576475in}{2.951220in}}%
\pgfpathcurveto{\pgfqpoint{2.584711in}{2.951220in}}{\pgfqpoint{2.592611in}{2.954493in}}{\pgfqpoint{2.598435in}{2.960317in}}%
\pgfpathcurveto{\pgfqpoint{2.604259in}{2.966140in}}{\pgfqpoint{2.607531in}{2.974040in}}{\pgfqpoint{2.607531in}{2.982277in}}%
\pgfpathcurveto{\pgfqpoint{2.607531in}{2.990513in}}{\pgfqpoint{2.604259in}{2.998413in}}{\pgfqpoint{2.598435in}{3.004237in}}%
\pgfpathcurveto{\pgfqpoint{2.592611in}{3.010061in}}{\pgfqpoint{2.584711in}{3.013333in}}{\pgfqpoint{2.576475in}{3.013333in}}%
\pgfpathcurveto{\pgfqpoint{2.568238in}{3.013333in}}{\pgfqpoint{2.560338in}{3.010061in}}{\pgfqpoint{2.554514in}{3.004237in}}%
\pgfpathcurveto{\pgfqpoint{2.548690in}{2.998413in}}{\pgfqpoint{2.545418in}{2.990513in}}{\pgfqpoint{2.545418in}{2.982277in}}%
\pgfpathcurveto{\pgfqpoint{2.545418in}{2.974040in}}{\pgfqpoint{2.548690in}{2.966140in}}{\pgfqpoint{2.554514in}{2.960317in}}%
\pgfpathcurveto{\pgfqpoint{2.560338in}{2.954493in}}{\pgfqpoint{2.568238in}{2.951220in}}{\pgfqpoint{2.576475in}{2.951220in}}%
\pgfpathclose%
\pgfusepath{stroke,fill}%
\end{pgfscope}%
\begin{pgfscope}%
\pgfpathrectangle{\pgfqpoint{0.100000in}{0.212622in}}{\pgfqpoint{3.696000in}{3.696000in}}%
\pgfusepath{clip}%
\pgfsetbuttcap%
\pgfsetroundjoin%
\definecolor{currentfill}{rgb}{0.121569,0.466667,0.705882}%
\pgfsetfillcolor{currentfill}%
\pgfsetfillopacity{0.732829}%
\pgfsetlinewidth{1.003750pt}%
\definecolor{currentstroke}{rgb}{0.121569,0.466667,0.705882}%
\pgfsetstrokecolor{currentstroke}%
\pgfsetstrokeopacity{0.732829}%
\pgfsetdash{}{0pt}%
\pgfpathmoveto{\pgfqpoint{1.369869in}{2.634718in}}%
\pgfpathcurveto{\pgfqpoint{1.378105in}{2.634718in}}{\pgfqpoint{1.386005in}{2.637990in}}{\pgfqpoint{1.391829in}{2.643814in}}%
\pgfpathcurveto{\pgfqpoint{1.397653in}{2.649638in}}{\pgfqpoint{1.400925in}{2.657538in}}{\pgfqpoint{1.400925in}{2.665775in}}%
\pgfpathcurveto{\pgfqpoint{1.400925in}{2.674011in}}{\pgfqpoint{1.397653in}{2.681911in}}{\pgfqpoint{1.391829in}{2.687735in}}%
\pgfpathcurveto{\pgfqpoint{1.386005in}{2.693559in}}{\pgfqpoint{1.378105in}{2.696831in}}{\pgfqpoint{1.369869in}{2.696831in}}%
\pgfpathcurveto{\pgfqpoint{1.361633in}{2.696831in}}{\pgfqpoint{1.353733in}{2.693559in}}{\pgfqpoint{1.347909in}{2.687735in}}%
\pgfpathcurveto{\pgfqpoint{1.342085in}{2.681911in}}{\pgfqpoint{1.338812in}{2.674011in}}{\pgfqpoint{1.338812in}{2.665775in}}%
\pgfpathcurveto{\pgfqpoint{1.338812in}{2.657538in}}{\pgfqpoint{1.342085in}{2.649638in}}{\pgfqpoint{1.347909in}{2.643814in}}%
\pgfpathcurveto{\pgfqpoint{1.353733in}{2.637990in}}{\pgfqpoint{1.361633in}{2.634718in}}{\pgfqpoint{1.369869in}{2.634718in}}%
\pgfpathclose%
\pgfusepath{stroke,fill}%
\end{pgfscope}%
\begin{pgfscope}%
\pgfpathrectangle{\pgfqpoint{0.100000in}{0.212622in}}{\pgfqpoint{3.696000in}{3.696000in}}%
\pgfusepath{clip}%
\pgfsetbuttcap%
\pgfsetroundjoin%
\definecolor{currentfill}{rgb}{0.121569,0.466667,0.705882}%
\pgfsetfillcolor{currentfill}%
\pgfsetfillopacity{0.733694}%
\pgfsetlinewidth{1.003750pt}%
\definecolor{currentstroke}{rgb}{0.121569,0.466667,0.705882}%
\pgfsetstrokecolor{currentstroke}%
\pgfsetstrokeopacity{0.733694}%
\pgfsetdash{}{0pt}%
\pgfpathmoveto{\pgfqpoint{2.998787in}{1.689815in}}%
\pgfpathcurveto{\pgfqpoint{3.007024in}{1.689815in}}{\pgfqpoint{3.014924in}{1.693087in}}{\pgfqpoint{3.020748in}{1.698911in}}%
\pgfpathcurveto{\pgfqpoint{3.026571in}{1.704735in}}{\pgfqpoint{3.029844in}{1.712635in}}{\pgfqpoint{3.029844in}{1.720871in}}%
\pgfpathcurveto{\pgfqpoint{3.029844in}{1.729107in}}{\pgfqpoint{3.026571in}{1.737007in}}{\pgfqpoint{3.020748in}{1.742831in}}%
\pgfpathcurveto{\pgfqpoint{3.014924in}{1.748655in}}{\pgfqpoint{3.007024in}{1.751928in}}{\pgfqpoint{2.998787in}{1.751928in}}%
\pgfpathcurveto{\pgfqpoint{2.990551in}{1.751928in}}{\pgfqpoint{2.982651in}{1.748655in}}{\pgfqpoint{2.976827in}{1.742831in}}%
\pgfpathcurveto{\pgfqpoint{2.971003in}{1.737007in}}{\pgfqpoint{2.967731in}{1.729107in}}{\pgfqpoint{2.967731in}{1.720871in}}%
\pgfpathcurveto{\pgfqpoint{2.967731in}{1.712635in}}{\pgfqpoint{2.971003in}{1.704735in}}{\pgfqpoint{2.976827in}{1.698911in}}%
\pgfpathcurveto{\pgfqpoint{2.982651in}{1.693087in}}{\pgfqpoint{2.990551in}{1.689815in}}{\pgfqpoint{2.998787in}{1.689815in}}%
\pgfpathclose%
\pgfusepath{stroke,fill}%
\end{pgfscope}%
\begin{pgfscope}%
\pgfpathrectangle{\pgfqpoint{0.100000in}{0.212622in}}{\pgfqpoint{3.696000in}{3.696000in}}%
\pgfusepath{clip}%
\pgfsetbuttcap%
\pgfsetroundjoin%
\definecolor{currentfill}{rgb}{0.121569,0.466667,0.705882}%
\pgfsetfillcolor{currentfill}%
\pgfsetfillopacity{0.733760}%
\pgfsetlinewidth{1.003750pt}%
\definecolor{currentstroke}{rgb}{0.121569,0.466667,0.705882}%
\pgfsetstrokecolor{currentstroke}%
\pgfsetstrokeopacity{0.733760}%
\pgfsetdash{}{0pt}%
\pgfpathmoveto{\pgfqpoint{1.366995in}{2.631613in}}%
\pgfpathcurveto{\pgfqpoint{1.375231in}{2.631613in}}{\pgfqpoint{1.383131in}{2.634885in}}{\pgfqpoint{1.388955in}{2.640709in}}%
\pgfpathcurveto{\pgfqpoint{1.394779in}{2.646533in}}{\pgfqpoint{1.398051in}{2.654433in}}{\pgfqpoint{1.398051in}{2.662669in}}%
\pgfpathcurveto{\pgfqpoint{1.398051in}{2.670905in}}{\pgfqpoint{1.394779in}{2.678805in}}{\pgfqpoint{1.388955in}{2.684629in}}%
\pgfpathcurveto{\pgfqpoint{1.383131in}{2.690453in}}{\pgfqpoint{1.375231in}{2.693726in}}{\pgfqpoint{1.366995in}{2.693726in}}%
\pgfpathcurveto{\pgfqpoint{1.358758in}{2.693726in}}{\pgfqpoint{1.350858in}{2.690453in}}{\pgfqpoint{1.345034in}{2.684629in}}%
\pgfpathcurveto{\pgfqpoint{1.339211in}{2.678805in}}{\pgfqpoint{1.335938in}{2.670905in}}{\pgfqpoint{1.335938in}{2.662669in}}%
\pgfpathcurveto{\pgfqpoint{1.335938in}{2.654433in}}{\pgfqpoint{1.339211in}{2.646533in}}{\pgfqpoint{1.345034in}{2.640709in}}%
\pgfpathcurveto{\pgfqpoint{1.350858in}{2.634885in}}{\pgfqpoint{1.358758in}{2.631613in}}{\pgfqpoint{1.366995in}{2.631613in}}%
\pgfpathclose%
\pgfusepath{stroke,fill}%
\end{pgfscope}%
\begin{pgfscope}%
\pgfpathrectangle{\pgfqpoint{0.100000in}{0.212622in}}{\pgfqpoint{3.696000in}{3.696000in}}%
\pgfusepath{clip}%
\pgfsetbuttcap%
\pgfsetroundjoin%
\definecolor{currentfill}{rgb}{0.121569,0.466667,0.705882}%
\pgfsetfillcolor{currentfill}%
\pgfsetfillopacity{0.733939}%
\pgfsetlinewidth{1.003750pt}%
\definecolor{currentstroke}{rgb}{0.121569,0.466667,0.705882}%
\pgfsetstrokecolor{currentstroke}%
\pgfsetstrokeopacity{0.733939}%
\pgfsetdash{}{0pt}%
\pgfpathmoveto{\pgfqpoint{2.572870in}{2.947172in}}%
\pgfpathcurveto{\pgfqpoint{2.581107in}{2.947172in}}{\pgfqpoint{2.589007in}{2.950444in}}{\pgfqpoint{2.594831in}{2.956268in}}%
\pgfpathcurveto{\pgfqpoint{2.600655in}{2.962092in}}{\pgfqpoint{2.603927in}{2.969992in}}{\pgfqpoint{2.603927in}{2.978228in}}%
\pgfpathcurveto{\pgfqpoint{2.603927in}{2.986465in}}{\pgfqpoint{2.600655in}{2.994365in}}{\pgfqpoint{2.594831in}{3.000189in}}%
\pgfpathcurveto{\pgfqpoint{2.589007in}{3.006013in}}{\pgfqpoint{2.581107in}{3.009285in}}{\pgfqpoint{2.572870in}{3.009285in}}%
\pgfpathcurveto{\pgfqpoint{2.564634in}{3.009285in}}{\pgfqpoint{2.556734in}{3.006013in}}{\pgfqpoint{2.550910in}{3.000189in}}%
\pgfpathcurveto{\pgfqpoint{2.545086in}{2.994365in}}{\pgfqpoint{2.541814in}{2.986465in}}{\pgfqpoint{2.541814in}{2.978228in}}%
\pgfpathcurveto{\pgfqpoint{2.541814in}{2.969992in}}{\pgfqpoint{2.545086in}{2.962092in}}{\pgfqpoint{2.550910in}{2.956268in}}%
\pgfpathcurveto{\pgfqpoint{2.556734in}{2.950444in}}{\pgfqpoint{2.564634in}{2.947172in}}{\pgfqpoint{2.572870in}{2.947172in}}%
\pgfpathclose%
\pgfusepath{stroke,fill}%
\end{pgfscope}%
\begin{pgfscope}%
\pgfpathrectangle{\pgfqpoint{0.100000in}{0.212622in}}{\pgfqpoint{3.696000in}{3.696000in}}%
\pgfusepath{clip}%
\pgfsetbuttcap%
\pgfsetroundjoin%
\definecolor{currentfill}{rgb}{0.121569,0.466667,0.705882}%
\pgfsetfillcolor{currentfill}%
\pgfsetfillopacity{0.734986}%
\pgfsetlinewidth{1.003750pt}%
\definecolor{currentstroke}{rgb}{0.121569,0.466667,0.705882}%
\pgfsetstrokecolor{currentstroke}%
\pgfsetstrokeopacity{0.734986}%
\pgfsetdash{}{0pt}%
\pgfpathmoveto{\pgfqpoint{1.362961in}{2.626940in}}%
\pgfpathcurveto{\pgfqpoint{1.371197in}{2.626940in}}{\pgfqpoint{1.379097in}{2.630212in}}{\pgfqpoint{1.384921in}{2.636036in}}%
\pgfpathcurveto{\pgfqpoint{1.390745in}{2.641860in}}{\pgfqpoint{1.394017in}{2.649760in}}{\pgfqpoint{1.394017in}{2.657996in}}%
\pgfpathcurveto{\pgfqpoint{1.394017in}{2.666232in}}{\pgfqpoint{1.390745in}{2.674133in}}{\pgfqpoint{1.384921in}{2.679956in}}%
\pgfpathcurveto{\pgfqpoint{1.379097in}{2.685780in}}{\pgfqpoint{1.371197in}{2.689053in}}{\pgfqpoint{1.362961in}{2.689053in}}%
\pgfpathcurveto{\pgfqpoint{1.354725in}{2.689053in}}{\pgfqpoint{1.346825in}{2.685780in}}{\pgfqpoint{1.341001in}{2.679956in}}%
\pgfpathcurveto{\pgfqpoint{1.335177in}{2.674133in}}{\pgfqpoint{1.331904in}{2.666232in}}{\pgfqpoint{1.331904in}{2.657996in}}%
\pgfpathcurveto{\pgfqpoint{1.331904in}{2.649760in}}{\pgfqpoint{1.335177in}{2.641860in}}{\pgfqpoint{1.341001in}{2.636036in}}%
\pgfpathcurveto{\pgfqpoint{1.346825in}{2.630212in}}{\pgfqpoint{1.354725in}{2.626940in}}{\pgfqpoint{1.362961in}{2.626940in}}%
\pgfpathclose%
\pgfusepath{stroke,fill}%
\end{pgfscope}%
\begin{pgfscope}%
\pgfpathrectangle{\pgfqpoint{0.100000in}{0.212622in}}{\pgfqpoint{3.696000in}{3.696000in}}%
\pgfusepath{clip}%
\pgfsetbuttcap%
\pgfsetroundjoin%
\definecolor{currentfill}{rgb}{0.121569,0.466667,0.705882}%
\pgfsetfillcolor{currentfill}%
\pgfsetfillopacity{0.735027}%
\pgfsetlinewidth{1.003750pt}%
\definecolor{currentstroke}{rgb}{0.121569,0.466667,0.705882}%
\pgfsetstrokecolor{currentstroke}%
\pgfsetstrokeopacity{0.735027}%
\pgfsetdash{}{0pt}%
\pgfpathmoveto{\pgfqpoint{2.569885in}{2.943565in}}%
\pgfpathcurveto{\pgfqpoint{2.578121in}{2.943565in}}{\pgfqpoint{2.586022in}{2.946837in}}{\pgfqpoint{2.591845in}{2.952661in}}%
\pgfpathcurveto{\pgfqpoint{2.597669in}{2.958485in}}{\pgfqpoint{2.600942in}{2.966385in}}{\pgfqpoint{2.600942in}{2.974621in}}%
\pgfpathcurveto{\pgfqpoint{2.600942in}{2.982857in}}{\pgfqpoint{2.597669in}{2.990757in}}{\pgfqpoint{2.591845in}{2.996581in}}%
\pgfpathcurveto{\pgfqpoint{2.586022in}{3.002405in}}{\pgfqpoint{2.578121in}{3.005678in}}{\pgfqpoint{2.569885in}{3.005678in}}%
\pgfpathcurveto{\pgfqpoint{2.561649in}{3.005678in}}{\pgfqpoint{2.553749in}{3.002405in}}{\pgfqpoint{2.547925in}{2.996581in}}%
\pgfpathcurveto{\pgfqpoint{2.542101in}{2.990757in}}{\pgfqpoint{2.538829in}{2.982857in}}{\pgfqpoint{2.538829in}{2.974621in}}%
\pgfpathcurveto{\pgfqpoint{2.538829in}{2.966385in}}{\pgfqpoint{2.542101in}{2.958485in}}{\pgfqpoint{2.547925in}{2.952661in}}%
\pgfpathcurveto{\pgfqpoint{2.553749in}{2.946837in}}{\pgfqpoint{2.561649in}{2.943565in}}{\pgfqpoint{2.569885in}{2.943565in}}%
\pgfpathclose%
\pgfusepath{stroke,fill}%
\end{pgfscope}%
\begin{pgfscope}%
\pgfpathrectangle{\pgfqpoint{0.100000in}{0.212622in}}{\pgfqpoint{3.696000in}{3.696000in}}%
\pgfusepath{clip}%
\pgfsetbuttcap%
\pgfsetroundjoin%
\definecolor{currentfill}{rgb}{0.121569,0.466667,0.705882}%
\pgfsetfillcolor{currentfill}%
\pgfsetfillopacity{0.736300}%
\pgfsetlinewidth{1.003750pt}%
\definecolor{currentstroke}{rgb}{0.121569,0.466667,0.705882}%
\pgfsetstrokecolor{currentstroke}%
\pgfsetstrokeopacity{0.736300}%
\pgfsetdash{}{0pt}%
\pgfpathmoveto{\pgfqpoint{1.358459in}{2.620722in}}%
\pgfpathcurveto{\pgfqpoint{1.366695in}{2.620722in}}{\pgfqpoint{1.374595in}{2.623994in}}{\pgfqpoint{1.380419in}{2.629818in}}%
\pgfpathcurveto{\pgfqpoint{1.386243in}{2.635642in}}{\pgfqpoint{1.389516in}{2.643542in}}{\pgfqpoint{1.389516in}{2.651778in}}%
\pgfpathcurveto{\pgfqpoint{1.389516in}{2.660015in}}{\pgfqpoint{1.386243in}{2.667915in}}{\pgfqpoint{1.380419in}{2.673739in}}%
\pgfpathcurveto{\pgfqpoint{1.374595in}{2.679562in}}{\pgfqpoint{1.366695in}{2.682835in}}{\pgfqpoint{1.358459in}{2.682835in}}%
\pgfpathcurveto{\pgfqpoint{1.350223in}{2.682835in}}{\pgfqpoint{1.342323in}{2.679562in}}{\pgfqpoint{1.336499in}{2.673739in}}%
\pgfpathcurveto{\pgfqpoint{1.330675in}{2.667915in}}{\pgfqpoint{1.327403in}{2.660015in}}{\pgfqpoint{1.327403in}{2.651778in}}%
\pgfpathcurveto{\pgfqpoint{1.327403in}{2.643542in}}{\pgfqpoint{1.330675in}{2.635642in}}{\pgfqpoint{1.336499in}{2.629818in}}%
\pgfpathcurveto{\pgfqpoint{1.342323in}{2.623994in}}{\pgfqpoint{1.350223in}{2.620722in}}{\pgfqpoint{1.358459in}{2.620722in}}%
\pgfpathclose%
\pgfusepath{stroke,fill}%
\end{pgfscope}%
\begin{pgfscope}%
\pgfpathrectangle{\pgfqpoint{0.100000in}{0.212622in}}{\pgfqpoint{3.696000in}{3.696000in}}%
\pgfusepath{clip}%
\pgfsetbuttcap%
\pgfsetroundjoin%
\definecolor{currentfill}{rgb}{0.121569,0.466667,0.705882}%
\pgfsetfillcolor{currentfill}%
\pgfsetfillopacity{0.736980}%
\pgfsetlinewidth{1.003750pt}%
\definecolor{currentstroke}{rgb}{0.121569,0.466667,0.705882}%
\pgfsetstrokecolor{currentstroke}%
\pgfsetstrokeopacity{0.736980}%
\pgfsetdash{}{0pt}%
\pgfpathmoveto{\pgfqpoint{2.564351in}{2.936976in}}%
\pgfpathcurveto{\pgfqpoint{2.572588in}{2.936976in}}{\pgfqpoint{2.580488in}{2.940249in}}{\pgfqpoint{2.586312in}{2.946073in}}%
\pgfpathcurveto{\pgfqpoint{2.592136in}{2.951897in}}{\pgfqpoint{2.595408in}{2.959797in}}{\pgfqpoint{2.595408in}{2.968033in}}%
\pgfpathcurveto{\pgfqpoint{2.595408in}{2.976269in}}{\pgfqpoint{2.592136in}{2.984169in}}{\pgfqpoint{2.586312in}{2.989993in}}%
\pgfpathcurveto{\pgfqpoint{2.580488in}{2.995817in}}{\pgfqpoint{2.572588in}{2.999089in}}{\pgfqpoint{2.564351in}{2.999089in}}%
\pgfpathcurveto{\pgfqpoint{2.556115in}{2.999089in}}{\pgfqpoint{2.548215in}{2.995817in}}{\pgfqpoint{2.542391in}{2.989993in}}%
\pgfpathcurveto{\pgfqpoint{2.536567in}{2.984169in}}{\pgfqpoint{2.533295in}{2.976269in}}{\pgfqpoint{2.533295in}{2.968033in}}%
\pgfpathcurveto{\pgfqpoint{2.533295in}{2.959797in}}{\pgfqpoint{2.536567in}{2.951897in}}{\pgfqpoint{2.542391in}{2.946073in}}%
\pgfpathcurveto{\pgfqpoint{2.548215in}{2.940249in}}{\pgfqpoint{2.556115in}{2.936976in}}{\pgfqpoint{2.564351in}{2.936976in}}%
\pgfpathclose%
\pgfusepath{stroke,fill}%
\end{pgfscope}%
\begin{pgfscope}%
\pgfpathrectangle{\pgfqpoint{0.100000in}{0.212622in}}{\pgfqpoint{3.696000in}{3.696000in}}%
\pgfusepath{clip}%
\pgfsetbuttcap%
\pgfsetroundjoin%
\definecolor{currentfill}{rgb}{0.121569,0.466667,0.705882}%
\pgfsetfillcolor{currentfill}%
\pgfsetfillopacity{0.737050}%
\pgfsetlinewidth{1.003750pt}%
\definecolor{currentstroke}{rgb}{0.121569,0.466667,0.705882}%
\pgfsetstrokecolor{currentstroke}%
\pgfsetstrokeopacity{0.737050}%
\pgfsetdash{}{0pt}%
\pgfpathmoveto{\pgfqpoint{1.356007in}{2.617425in}}%
\pgfpathcurveto{\pgfqpoint{1.364243in}{2.617425in}}{\pgfqpoint{1.372143in}{2.620697in}}{\pgfqpoint{1.377967in}{2.626521in}}%
\pgfpathcurveto{\pgfqpoint{1.383791in}{2.632345in}}{\pgfqpoint{1.387064in}{2.640245in}}{\pgfqpoint{1.387064in}{2.648482in}}%
\pgfpathcurveto{\pgfqpoint{1.387064in}{2.656718in}}{\pgfqpoint{1.383791in}{2.664618in}}{\pgfqpoint{1.377967in}{2.670442in}}%
\pgfpathcurveto{\pgfqpoint{1.372143in}{2.676266in}}{\pgfqpoint{1.364243in}{2.679538in}}{\pgfqpoint{1.356007in}{2.679538in}}%
\pgfpathcurveto{\pgfqpoint{1.347771in}{2.679538in}}{\pgfqpoint{1.339871in}{2.676266in}}{\pgfqpoint{1.334047in}{2.670442in}}%
\pgfpathcurveto{\pgfqpoint{1.328223in}{2.664618in}}{\pgfqpoint{1.324951in}{2.656718in}}{\pgfqpoint{1.324951in}{2.648482in}}%
\pgfpathcurveto{\pgfqpoint{1.324951in}{2.640245in}}{\pgfqpoint{1.328223in}{2.632345in}}{\pgfqpoint{1.334047in}{2.626521in}}%
\pgfpathcurveto{\pgfqpoint{1.339871in}{2.620697in}}{\pgfqpoint{1.347771in}{2.617425in}}{\pgfqpoint{1.356007in}{2.617425in}}%
\pgfpathclose%
\pgfusepath{stroke,fill}%
\end{pgfscope}%
\begin{pgfscope}%
\pgfpathrectangle{\pgfqpoint{0.100000in}{0.212622in}}{\pgfqpoint{3.696000in}{3.696000in}}%
\pgfusepath{clip}%
\pgfsetbuttcap%
\pgfsetroundjoin%
\definecolor{currentfill}{rgb}{0.121569,0.466667,0.705882}%
\pgfsetfillcolor{currentfill}%
\pgfsetfillopacity{0.737932}%
\pgfsetlinewidth{1.003750pt}%
\definecolor{currentstroke}{rgb}{0.121569,0.466667,0.705882}%
\pgfsetstrokecolor{currentstroke}%
\pgfsetstrokeopacity{0.737932}%
\pgfsetdash{}{0pt}%
\pgfpathmoveto{\pgfqpoint{1.353297in}{2.614102in}}%
\pgfpathcurveto{\pgfqpoint{1.361533in}{2.614102in}}{\pgfqpoint{1.369433in}{2.617374in}}{\pgfqpoint{1.375257in}{2.623198in}}%
\pgfpathcurveto{\pgfqpoint{1.381081in}{2.629022in}}{\pgfqpoint{1.384353in}{2.636922in}}{\pgfqpoint{1.384353in}{2.645158in}}%
\pgfpathcurveto{\pgfqpoint{1.384353in}{2.653395in}}{\pgfqpoint{1.381081in}{2.661295in}}{\pgfqpoint{1.375257in}{2.667118in}}%
\pgfpathcurveto{\pgfqpoint{1.369433in}{2.672942in}}{\pgfqpoint{1.361533in}{2.676215in}}{\pgfqpoint{1.353297in}{2.676215in}}%
\pgfpathcurveto{\pgfqpoint{1.345061in}{2.676215in}}{\pgfqpoint{1.337161in}{2.672942in}}{\pgfqpoint{1.331337in}{2.667118in}}%
\pgfpathcurveto{\pgfqpoint{1.325513in}{2.661295in}}{\pgfqpoint{1.322240in}{2.653395in}}{\pgfqpoint{1.322240in}{2.645158in}}%
\pgfpathcurveto{\pgfqpoint{1.322240in}{2.636922in}}{\pgfqpoint{1.325513in}{2.629022in}}{\pgfqpoint{1.331337in}{2.623198in}}%
\pgfpathcurveto{\pgfqpoint{1.337161in}{2.617374in}}{\pgfqpoint{1.345061in}{2.614102in}}{\pgfqpoint{1.353297in}{2.614102in}}%
\pgfpathclose%
\pgfusepath{stroke,fill}%
\end{pgfscope}%
\begin{pgfscope}%
\pgfpathrectangle{\pgfqpoint{0.100000in}{0.212622in}}{\pgfqpoint{3.696000in}{3.696000in}}%
\pgfusepath{clip}%
\pgfsetbuttcap%
\pgfsetroundjoin%
\definecolor{currentfill}{rgb}{0.121569,0.466667,0.705882}%
\pgfsetfillcolor{currentfill}%
\pgfsetfillopacity{0.738703}%
\pgfsetlinewidth{1.003750pt}%
\definecolor{currentstroke}{rgb}{0.121569,0.466667,0.705882}%
\pgfsetstrokecolor{currentstroke}%
\pgfsetstrokeopacity{0.738703}%
\pgfsetdash{}{0pt}%
\pgfpathmoveto{\pgfqpoint{2.559696in}{2.931883in}}%
\pgfpathcurveto{\pgfqpoint{2.567932in}{2.931883in}}{\pgfqpoint{2.575832in}{2.935155in}}{\pgfqpoint{2.581656in}{2.940979in}}%
\pgfpathcurveto{\pgfqpoint{2.587480in}{2.946803in}}{\pgfqpoint{2.590753in}{2.954703in}}{\pgfqpoint{2.590753in}{2.962940in}}%
\pgfpathcurveto{\pgfqpoint{2.590753in}{2.971176in}}{\pgfqpoint{2.587480in}{2.979076in}}{\pgfqpoint{2.581656in}{2.984900in}}%
\pgfpathcurveto{\pgfqpoint{2.575832in}{2.990724in}}{\pgfqpoint{2.567932in}{2.993996in}}{\pgfqpoint{2.559696in}{2.993996in}}%
\pgfpathcurveto{\pgfqpoint{2.551460in}{2.993996in}}{\pgfqpoint{2.543560in}{2.990724in}}{\pgfqpoint{2.537736in}{2.984900in}}%
\pgfpathcurveto{\pgfqpoint{2.531912in}{2.979076in}}{\pgfqpoint{2.528640in}{2.971176in}}{\pgfqpoint{2.528640in}{2.962940in}}%
\pgfpathcurveto{\pgfqpoint{2.528640in}{2.954703in}}{\pgfqpoint{2.531912in}{2.946803in}}{\pgfqpoint{2.537736in}{2.940979in}}%
\pgfpathcurveto{\pgfqpoint{2.543560in}{2.935155in}}{\pgfqpoint{2.551460in}{2.931883in}}{\pgfqpoint{2.559696in}{2.931883in}}%
\pgfpathclose%
\pgfusepath{stroke,fill}%
\end{pgfscope}%
\begin{pgfscope}%
\pgfpathrectangle{\pgfqpoint{0.100000in}{0.212622in}}{\pgfqpoint{3.696000in}{3.696000in}}%
\pgfusepath{clip}%
\pgfsetbuttcap%
\pgfsetroundjoin%
\definecolor{currentfill}{rgb}{0.121569,0.466667,0.705882}%
\pgfsetfillcolor{currentfill}%
\pgfsetfillopacity{0.738961}%
\pgfsetlinewidth{1.003750pt}%
\definecolor{currentstroke}{rgb}{0.121569,0.466667,0.705882}%
\pgfsetstrokecolor{currentstroke}%
\pgfsetstrokeopacity{0.738961}%
\pgfsetdash{}{0pt}%
\pgfpathmoveto{\pgfqpoint{1.350033in}{2.610754in}}%
\pgfpathcurveto{\pgfqpoint{1.358269in}{2.610754in}}{\pgfqpoint{1.366169in}{2.614026in}}{\pgfqpoint{1.371993in}{2.619850in}}%
\pgfpathcurveto{\pgfqpoint{1.377817in}{2.625674in}}{\pgfqpoint{1.381089in}{2.633574in}}{\pgfqpoint{1.381089in}{2.641811in}}%
\pgfpathcurveto{\pgfqpoint{1.381089in}{2.650047in}}{\pgfqpoint{1.377817in}{2.657947in}}{\pgfqpoint{1.371993in}{2.663771in}}%
\pgfpathcurveto{\pgfqpoint{1.366169in}{2.669595in}}{\pgfqpoint{1.358269in}{2.672867in}}{\pgfqpoint{1.350033in}{2.672867in}}%
\pgfpathcurveto{\pgfqpoint{1.341797in}{2.672867in}}{\pgfqpoint{1.333897in}{2.669595in}}{\pgfqpoint{1.328073in}{2.663771in}}%
\pgfpathcurveto{\pgfqpoint{1.322249in}{2.657947in}}{\pgfqpoint{1.318976in}{2.650047in}}{\pgfqpoint{1.318976in}{2.641811in}}%
\pgfpathcurveto{\pgfqpoint{1.318976in}{2.633574in}}{\pgfqpoint{1.322249in}{2.625674in}}{\pgfqpoint{1.328073in}{2.619850in}}%
\pgfpathcurveto{\pgfqpoint{1.333897in}{2.614026in}}{\pgfqpoint{1.341797in}{2.610754in}}{\pgfqpoint{1.350033in}{2.610754in}}%
\pgfpathclose%
\pgfusepath{stroke,fill}%
\end{pgfscope}%
\begin{pgfscope}%
\pgfpathrectangle{\pgfqpoint{0.100000in}{0.212622in}}{\pgfqpoint{3.696000in}{3.696000in}}%
\pgfusepath{clip}%
\pgfsetbuttcap%
\pgfsetroundjoin%
\definecolor{currentfill}{rgb}{0.121569,0.466667,0.705882}%
\pgfsetfillcolor{currentfill}%
\pgfsetfillopacity{0.739030}%
\pgfsetlinewidth{1.003750pt}%
\definecolor{currentstroke}{rgb}{0.121569,0.466667,0.705882}%
\pgfsetstrokecolor{currentstroke}%
\pgfsetstrokeopacity{0.739030}%
\pgfsetdash{}{0pt}%
\pgfpathmoveto{\pgfqpoint{2.987920in}{1.677170in}}%
\pgfpathcurveto{\pgfqpoint{2.996156in}{1.677170in}}{\pgfqpoint{3.004056in}{1.680442in}}{\pgfqpoint{3.009880in}{1.686266in}}%
\pgfpathcurveto{\pgfqpoint{3.015704in}{1.692090in}}{\pgfqpoint{3.018976in}{1.699990in}}{\pgfqpoint{3.018976in}{1.708226in}}%
\pgfpathcurveto{\pgfqpoint{3.018976in}{1.716462in}}{\pgfqpoint{3.015704in}{1.724362in}}{\pgfqpoint{3.009880in}{1.730186in}}%
\pgfpathcurveto{\pgfqpoint{3.004056in}{1.736010in}}{\pgfqpoint{2.996156in}{1.739283in}}{\pgfqpoint{2.987920in}{1.739283in}}%
\pgfpathcurveto{\pgfqpoint{2.979683in}{1.739283in}}{\pgfqpoint{2.971783in}{1.736010in}}{\pgfqpoint{2.965960in}{1.730186in}}%
\pgfpathcurveto{\pgfqpoint{2.960136in}{1.724362in}}{\pgfqpoint{2.956863in}{1.716462in}}{\pgfqpoint{2.956863in}{1.708226in}}%
\pgfpathcurveto{\pgfqpoint{2.956863in}{1.699990in}}{\pgfqpoint{2.960136in}{1.692090in}}{\pgfqpoint{2.965960in}{1.686266in}}%
\pgfpathcurveto{\pgfqpoint{2.971783in}{1.680442in}}{\pgfqpoint{2.979683in}{1.677170in}}{\pgfqpoint{2.987920in}{1.677170in}}%
\pgfpathclose%
\pgfusepath{stroke,fill}%
\end{pgfscope}%
\begin{pgfscope}%
\pgfpathrectangle{\pgfqpoint{0.100000in}{0.212622in}}{\pgfqpoint{3.696000in}{3.696000in}}%
\pgfusepath{clip}%
\pgfsetbuttcap%
\pgfsetroundjoin%
\definecolor{currentfill}{rgb}{0.121569,0.466667,0.705882}%
\pgfsetfillcolor{currentfill}%
\pgfsetfillopacity{0.740295}%
\pgfsetlinewidth{1.003750pt}%
\definecolor{currentstroke}{rgb}{0.121569,0.466667,0.705882}%
\pgfsetstrokecolor{currentstroke}%
\pgfsetstrokeopacity{0.740295}%
\pgfsetdash{}{0pt}%
\pgfpathmoveto{\pgfqpoint{1.345676in}{2.605791in}}%
\pgfpathcurveto{\pgfqpoint{1.353912in}{2.605791in}}{\pgfqpoint{1.361812in}{2.609064in}}{\pgfqpoint{1.367636in}{2.614888in}}%
\pgfpathcurveto{\pgfqpoint{1.373460in}{2.620712in}}{\pgfqpoint{1.376732in}{2.628612in}}{\pgfqpoint{1.376732in}{2.636848in}}%
\pgfpathcurveto{\pgfqpoint{1.376732in}{2.645084in}}{\pgfqpoint{1.373460in}{2.652984in}}{\pgfqpoint{1.367636in}{2.658808in}}%
\pgfpathcurveto{\pgfqpoint{1.361812in}{2.664632in}}{\pgfqpoint{1.353912in}{2.667904in}}{\pgfqpoint{1.345676in}{2.667904in}}%
\pgfpathcurveto{\pgfqpoint{1.337440in}{2.667904in}}{\pgfqpoint{1.329539in}{2.664632in}}{\pgfqpoint{1.323716in}{2.658808in}}%
\pgfpathcurveto{\pgfqpoint{1.317892in}{2.652984in}}{\pgfqpoint{1.314619in}{2.645084in}}{\pgfqpoint{1.314619in}{2.636848in}}%
\pgfpathcurveto{\pgfqpoint{1.314619in}{2.628612in}}{\pgfqpoint{1.317892in}{2.620712in}}{\pgfqpoint{1.323716in}{2.614888in}}%
\pgfpathcurveto{\pgfqpoint{1.329539in}{2.609064in}}{\pgfqpoint{1.337440in}{2.605791in}}{\pgfqpoint{1.345676in}{2.605791in}}%
\pgfpathclose%
\pgfusepath{stroke,fill}%
\end{pgfscope}%
\begin{pgfscope}%
\pgfpathrectangle{\pgfqpoint{0.100000in}{0.212622in}}{\pgfqpoint{3.696000in}{3.696000in}}%
\pgfusepath{clip}%
\pgfsetbuttcap%
\pgfsetroundjoin%
\definecolor{currentfill}{rgb}{0.121569,0.466667,0.705882}%
\pgfsetfillcolor{currentfill}%
\pgfsetfillopacity{0.740383}%
\pgfsetlinewidth{1.003750pt}%
\definecolor{currentstroke}{rgb}{0.121569,0.466667,0.705882}%
\pgfsetstrokecolor{currentstroke}%
\pgfsetstrokeopacity{0.740383}%
\pgfsetdash{}{0pt}%
\pgfpathmoveto{\pgfqpoint{2.555478in}{2.927399in}}%
\pgfpathcurveto{\pgfqpoint{2.563714in}{2.927399in}}{\pgfqpoint{2.571614in}{2.930671in}}{\pgfqpoint{2.577438in}{2.936495in}}%
\pgfpathcurveto{\pgfqpoint{2.583262in}{2.942319in}}{\pgfqpoint{2.586535in}{2.950219in}}{\pgfqpoint{2.586535in}{2.958455in}}%
\pgfpathcurveto{\pgfqpoint{2.586535in}{2.966692in}}{\pgfqpoint{2.583262in}{2.974592in}}{\pgfqpoint{2.577438in}{2.980416in}}%
\pgfpathcurveto{\pgfqpoint{2.571614in}{2.986240in}}{\pgfqpoint{2.563714in}{2.989512in}}{\pgfqpoint{2.555478in}{2.989512in}}%
\pgfpathcurveto{\pgfqpoint{2.547242in}{2.989512in}}{\pgfqpoint{2.539342in}{2.986240in}}{\pgfqpoint{2.533518in}{2.980416in}}%
\pgfpathcurveto{\pgfqpoint{2.527694in}{2.974592in}}{\pgfqpoint{2.524422in}{2.966692in}}{\pgfqpoint{2.524422in}{2.958455in}}%
\pgfpathcurveto{\pgfqpoint{2.524422in}{2.950219in}}{\pgfqpoint{2.527694in}{2.942319in}}{\pgfqpoint{2.533518in}{2.936495in}}%
\pgfpathcurveto{\pgfqpoint{2.539342in}{2.930671in}}{\pgfqpoint{2.547242in}{2.927399in}}{\pgfqpoint{2.555478in}{2.927399in}}%
\pgfpathclose%
\pgfusepath{stroke,fill}%
\end{pgfscope}%
\begin{pgfscope}%
\pgfpathrectangle{\pgfqpoint{0.100000in}{0.212622in}}{\pgfqpoint{3.696000in}{3.696000in}}%
\pgfusepath{clip}%
\pgfsetbuttcap%
\pgfsetroundjoin%
\definecolor{currentfill}{rgb}{0.121569,0.466667,0.705882}%
\pgfsetfillcolor{currentfill}%
\pgfsetfillopacity{0.741715}%
\pgfsetlinewidth{1.003750pt}%
\definecolor{currentstroke}{rgb}{0.121569,0.466667,0.705882}%
\pgfsetstrokecolor{currentstroke}%
\pgfsetstrokeopacity{0.741715}%
\pgfsetdash{}{0pt}%
\pgfpathmoveto{\pgfqpoint{1.340724in}{2.599198in}}%
\pgfpathcurveto{\pgfqpoint{1.348960in}{2.599198in}}{\pgfqpoint{1.356860in}{2.602470in}}{\pgfqpoint{1.362684in}{2.608294in}}%
\pgfpathcurveto{\pgfqpoint{1.368508in}{2.614118in}}{\pgfqpoint{1.371781in}{2.622018in}}{\pgfqpoint{1.371781in}{2.630255in}}%
\pgfpathcurveto{\pgfqpoint{1.371781in}{2.638491in}}{\pgfqpoint{1.368508in}{2.646391in}}{\pgfqpoint{1.362684in}{2.652215in}}%
\pgfpathcurveto{\pgfqpoint{1.356860in}{2.658039in}}{\pgfqpoint{1.348960in}{2.661311in}}{\pgfqpoint{1.340724in}{2.661311in}}%
\pgfpathcurveto{\pgfqpoint{1.332488in}{2.661311in}}{\pgfqpoint{1.324588in}{2.658039in}}{\pgfqpoint{1.318764in}{2.652215in}}%
\pgfpathcurveto{\pgfqpoint{1.312940in}{2.646391in}}{\pgfqpoint{1.309668in}{2.638491in}}{\pgfqpoint{1.309668in}{2.630255in}}%
\pgfpathcurveto{\pgfqpoint{1.309668in}{2.622018in}}{\pgfqpoint{1.312940in}{2.614118in}}{\pgfqpoint{1.318764in}{2.608294in}}%
\pgfpathcurveto{\pgfqpoint{1.324588in}{2.602470in}}{\pgfqpoint{1.332488in}{2.599198in}}{\pgfqpoint{1.340724in}{2.599198in}}%
\pgfpathclose%
\pgfusepath{stroke,fill}%
\end{pgfscope}%
\begin{pgfscope}%
\pgfpathrectangle{\pgfqpoint{0.100000in}{0.212622in}}{\pgfqpoint{3.696000in}{3.696000in}}%
\pgfusepath{clip}%
\pgfsetbuttcap%
\pgfsetroundjoin%
\definecolor{currentfill}{rgb}{0.121569,0.466667,0.705882}%
\pgfsetfillcolor{currentfill}%
\pgfsetfillopacity{0.741771}%
\pgfsetlinewidth{1.003750pt}%
\definecolor{currentstroke}{rgb}{0.121569,0.466667,0.705882}%
\pgfsetstrokecolor{currentstroke}%
\pgfsetstrokeopacity{0.741771}%
\pgfsetdash{}{0pt}%
\pgfpathmoveto{\pgfqpoint{2.551753in}{2.922743in}}%
\pgfpathcurveto{\pgfqpoint{2.559989in}{2.922743in}}{\pgfqpoint{2.567889in}{2.926015in}}{\pgfqpoint{2.573713in}{2.931839in}}%
\pgfpathcurveto{\pgfqpoint{2.579537in}{2.937663in}}{\pgfqpoint{2.582809in}{2.945563in}}{\pgfqpoint{2.582809in}{2.953799in}}%
\pgfpathcurveto{\pgfqpoint{2.582809in}{2.962035in}}{\pgfqpoint{2.579537in}{2.969935in}}{\pgfqpoint{2.573713in}{2.975759in}}%
\pgfpathcurveto{\pgfqpoint{2.567889in}{2.981583in}}{\pgfqpoint{2.559989in}{2.984856in}}{\pgfqpoint{2.551753in}{2.984856in}}%
\pgfpathcurveto{\pgfqpoint{2.543516in}{2.984856in}}{\pgfqpoint{2.535616in}{2.981583in}}{\pgfqpoint{2.529792in}{2.975759in}}%
\pgfpathcurveto{\pgfqpoint{2.523969in}{2.969935in}}{\pgfqpoint{2.520696in}{2.962035in}}{\pgfqpoint{2.520696in}{2.953799in}}%
\pgfpathcurveto{\pgfqpoint{2.520696in}{2.945563in}}{\pgfqpoint{2.523969in}{2.937663in}}{\pgfqpoint{2.529792in}{2.931839in}}%
\pgfpathcurveto{\pgfqpoint{2.535616in}{2.926015in}}{\pgfqpoint{2.543516in}{2.922743in}}{\pgfqpoint{2.551753in}{2.922743in}}%
\pgfpathclose%
\pgfusepath{stroke,fill}%
\end{pgfscope}%
\begin{pgfscope}%
\pgfpathrectangle{\pgfqpoint{0.100000in}{0.212622in}}{\pgfqpoint{3.696000in}{3.696000in}}%
\pgfusepath{clip}%
\pgfsetbuttcap%
\pgfsetroundjoin%
\definecolor{currentfill}{rgb}{0.121569,0.466667,0.705882}%
\pgfsetfillcolor{currentfill}%
\pgfsetfillopacity{0.742500}%
\pgfsetlinewidth{1.003750pt}%
\definecolor{currentstroke}{rgb}{0.121569,0.466667,0.705882}%
\pgfsetstrokecolor{currentstroke}%
\pgfsetstrokeopacity{0.742500}%
\pgfsetdash{}{0pt}%
\pgfpathmoveto{\pgfqpoint{1.338048in}{2.595528in}}%
\pgfpathcurveto{\pgfqpoint{1.346285in}{2.595528in}}{\pgfqpoint{1.354185in}{2.598800in}}{\pgfqpoint{1.360009in}{2.604624in}}%
\pgfpathcurveto{\pgfqpoint{1.365833in}{2.610448in}}{\pgfqpoint{1.369105in}{2.618348in}}{\pgfqpoint{1.369105in}{2.626585in}}%
\pgfpathcurveto{\pgfqpoint{1.369105in}{2.634821in}}{\pgfqpoint{1.365833in}{2.642721in}}{\pgfqpoint{1.360009in}{2.648545in}}%
\pgfpathcurveto{\pgfqpoint{1.354185in}{2.654369in}}{\pgfqpoint{1.346285in}{2.657641in}}{\pgfqpoint{1.338048in}{2.657641in}}%
\pgfpathcurveto{\pgfqpoint{1.329812in}{2.657641in}}{\pgfqpoint{1.321912in}{2.654369in}}{\pgfqpoint{1.316088in}{2.648545in}}%
\pgfpathcurveto{\pgfqpoint{1.310264in}{2.642721in}}{\pgfqpoint{1.306992in}{2.634821in}}{\pgfqpoint{1.306992in}{2.626585in}}%
\pgfpathcurveto{\pgfqpoint{1.306992in}{2.618348in}}{\pgfqpoint{1.310264in}{2.610448in}}{\pgfqpoint{1.316088in}{2.604624in}}%
\pgfpathcurveto{\pgfqpoint{1.321912in}{2.598800in}}{\pgfqpoint{1.329812in}{2.595528in}}{\pgfqpoint{1.338048in}{2.595528in}}%
\pgfpathclose%
\pgfusepath{stroke,fill}%
\end{pgfscope}%
\begin{pgfscope}%
\pgfpathrectangle{\pgfqpoint{0.100000in}{0.212622in}}{\pgfqpoint{3.696000in}{3.696000in}}%
\pgfusepath{clip}%
\pgfsetbuttcap%
\pgfsetroundjoin%
\definecolor{currentfill}{rgb}{0.121569,0.466667,0.705882}%
\pgfsetfillcolor{currentfill}%
\pgfsetfillopacity{0.742864}%
\pgfsetlinewidth{1.003750pt}%
\definecolor{currentstroke}{rgb}{0.121569,0.466667,0.705882}%
\pgfsetstrokecolor{currentstroke}%
\pgfsetstrokeopacity{0.742864}%
\pgfsetdash{}{0pt}%
\pgfpathmoveto{\pgfqpoint{2.548707in}{2.918846in}}%
\pgfpathcurveto{\pgfqpoint{2.556943in}{2.918846in}}{\pgfqpoint{2.564843in}{2.922118in}}{\pgfqpoint{2.570667in}{2.927942in}}%
\pgfpathcurveto{\pgfqpoint{2.576491in}{2.933766in}}{\pgfqpoint{2.579763in}{2.941666in}}{\pgfqpoint{2.579763in}{2.949902in}}%
\pgfpathcurveto{\pgfqpoint{2.579763in}{2.958139in}}{\pgfqpoint{2.576491in}{2.966039in}}{\pgfqpoint{2.570667in}{2.971863in}}%
\pgfpathcurveto{\pgfqpoint{2.564843in}{2.977687in}}{\pgfqpoint{2.556943in}{2.980959in}}{\pgfqpoint{2.548707in}{2.980959in}}%
\pgfpathcurveto{\pgfqpoint{2.540470in}{2.980959in}}{\pgfqpoint{2.532570in}{2.977687in}}{\pgfqpoint{2.526746in}{2.971863in}}%
\pgfpathcurveto{\pgfqpoint{2.520923in}{2.966039in}}{\pgfqpoint{2.517650in}{2.958139in}}{\pgfqpoint{2.517650in}{2.949902in}}%
\pgfpathcurveto{\pgfqpoint{2.517650in}{2.941666in}}{\pgfqpoint{2.520923in}{2.933766in}}{\pgfqpoint{2.526746in}{2.927942in}}%
\pgfpathcurveto{\pgfqpoint{2.532570in}{2.922118in}}{\pgfqpoint{2.540470in}{2.918846in}}{\pgfqpoint{2.548707in}{2.918846in}}%
\pgfpathclose%
\pgfusepath{stroke,fill}%
\end{pgfscope}%
\begin{pgfscope}%
\pgfpathrectangle{\pgfqpoint{0.100000in}{0.212622in}}{\pgfqpoint{3.696000in}{3.696000in}}%
\pgfusepath{clip}%
\pgfsetbuttcap%
\pgfsetroundjoin%
\definecolor{currentfill}{rgb}{0.121569,0.466667,0.705882}%
\pgfsetfillcolor{currentfill}%
\pgfsetfillopacity{0.743464}%
\pgfsetlinewidth{1.003750pt}%
\definecolor{currentstroke}{rgb}{0.121569,0.466667,0.705882}%
\pgfsetstrokecolor{currentstroke}%
\pgfsetstrokeopacity{0.743464}%
\pgfsetdash{}{0pt}%
\pgfpathmoveto{\pgfqpoint{1.334962in}{2.591566in}}%
\pgfpathcurveto{\pgfqpoint{1.343198in}{2.591566in}}{\pgfqpoint{1.351098in}{2.594838in}}{\pgfqpoint{1.356922in}{2.600662in}}%
\pgfpathcurveto{\pgfqpoint{1.362746in}{2.606486in}}{\pgfqpoint{1.366018in}{2.614386in}}{\pgfqpoint{1.366018in}{2.622622in}}%
\pgfpathcurveto{\pgfqpoint{1.366018in}{2.630858in}}{\pgfqpoint{1.362746in}{2.638758in}}{\pgfqpoint{1.356922in}{2.644582in}}%
\pgfpathcurveto{\pgfqpoint{1.351098in}{2.650406in}}{\pgfqpoint{1.343198in}{2.653679in}}{\pgfqpoint{1.334962in}{2.653679in}}%
\pgfpathcurveto{\pgfqpoint{1.326726in}{2.653679in}}{\pgfqpoint{1.318826in}{2.650406in}}{\pgfqpoint{1.313002in}{2.644582in}}%
\pgfpathcurveto{\pgfqpoint{1.307178in}{2.638758in}}{\pgfqpoint{1.303905in}{2.630858in}}{\pgfqpoint{1.303905in}{2.622622in}}%
\pgfpathcurveto{\pgfqpoint{1.303905in}{2.614386in}}{\pgfqpoint{1.307178in}{2.606486in}}{\pgfqpoint{1.313002in}{2.600662in}}%
\pgfpathcurveto{\pgfqpoint{1.318826in}{2.594838in}}{\pgfqpoint{1.326726in}{2.591566in}}{\pgfqpoint{1.334962in}{2.591566in}}%
\pgfpathclose%
\pgfusepath{stroke,fill}%
\end{pgfscope}%
\begin{pgfscope}%
\pgfpathrectangle{\pgfqpoint{0.100000in}{0.212622in}}{\pgfqpoint{3.696000in}{3.696000in}}%
\pgfusepath{clip}%
\pgfsetbuttcap%
\pgfsetroundjoin%
\definecolor{currentfill}{rgb}{0.121569,0.466667,0.705882}%
\pgfsetfillcolor{currentfill}%
\pgfsetfillopacity{0.743954}%
\pgfsetlinewidth{1.003750pt}%
\definecolor{currentstroke}{rgb}{0.121569,0.466667,0.705882}%
\pgfsetstrokecolor{currentstroke}%
\pgfsetstrokeopacity{0.743954}%
\pgfsetdash{}{0pt}%
\pgfpathmoveto{\pgfqpoint{2.546010in}{2.915582in}}%
\pgfpathcurveto{\pgfqpoint{2.554247in}{2.915582in}}{\pgfqpoint{2.562147in}{2.918854in}}{\pgfqpoint{2.567970in}{2.924678in}}%
\pgfpathcurveto{\pgfqpoint{2.573794in}{2.930502in}}{\pgfqpoint{2.577067in}{2.938402in}}{\pgfqpoint{2.577067in}{2.946638in}}%
\pgfpathcurveto{\pgfqpoint{2.577067in}{2.954874in}}{\pgfqpoint{2.573794in}{2.962774in}}{\pgfqpoint{2.567970in}{2.968598in}}%
\pgfpathcurveto{\pgfqpoint{2.562147in}{2.974422in}}{\pgfqpoint{2.554247in}{2.977695in}}{\pgfqpoint{2.546010in}{2.977695in}}%
\pgfpathcurveto{\pgfqpoint{2.537774in}{2.977695in}}{\pgfqpoint{2.529874in}{2.974422in}}{\pgfqpoint{2.524050in}{2.968598in}}%
\pgfpathcurveto{\pgfqpoint{2.518226in}{2.962774in}}{\pgfqpoint{2.514954in}{2.954874in}}{\pgfqpoint{2.514954in}{2.946638in}}%
\pgfpathcurveto{\pgfqpoint{2.514954in}{2.938402in}}{\pgfqpoint{2.518226in}{2.930502in}}{\pgfqpoint{2.524050in}{2.924678in}}%
\pgfpathcurveto{\pgfqpoint{2.529874in}{2.918854in}}{\pgfqpoint{2.537774in}{2.915582in}}{\pgfqpoint{2.546010in}{2.915582in}}%
\pgfpathclose%
\pgfusepath{stroke,fill}%
\end{pgfscope}%
\begin{pgfscope}%
\pgfpathrectangle{\pgfqpoint{0.100000in}{0.212622in}}{\pgfqpoint{3.696000in}{3.696000in}}%
\pgfusepath{clip}%
\pgfsetbuttcap%
\pgfsetroundjoin%
\definecolor{currentfill}{rgb}{0.121569,0.466667,0.705882}%
\pgfsetfillcolor{currentfill}%
\pgfsetfillopacity{0.743976}%
\pgfsetlinewidth{1.003750pt}%
\definecolor{currentstroke}{rgb}{0.121569,0.466667,0.705882}%
\pgfsetstrokecolor{currentstroke}%
\pgfsetstrokeopacity{0.743976}%
\pgfsetdash{}{0pt}%
\pgfpathmoveto{\pgfqpoint{2.978148in}{1.666358in}}%
\pgfpathcurveto{\pgfqpoint{2.986385in}{1.666358in}}{\pgfqpoint{2.994285in}{1.669630in}}{\pgfqpoint{3.000109in}{1.675454in}}%
\pgfpathcurveto{\pgfqpoint{3.005933in}{1.681278in}}{\pgfqpoint{3.009205in}{1.689178in}}{\pgfqpoint{3.009205in}{1.697415in}}%
\pgfpathcurveto{\pgfqpoint{3.009205in}{1.705651in}}{\pgfqpoint{3.005933in}{1.713551in}}{\pgfqpoint{3.000109in}{1.719375in}}%
\pgfpathcurveto{\pgfqpoint{2.994285in}{1.725199in}}{\pgfqpoint{2.986385in}{1.728471in}}{\pgfqpoint{2.978148in}{1.728471in}}%
\pgfpathcurveto{\pgfqpoint{2.969912in}{1.728471in}}{\pgfqpoint{2.962012in}{1.725199in}}{\pgfqpoint{2.956188in}{1.719375in}}%
\pgfpathcurveto{\pgfqpoint{2.950364in}{1.713551in}}{\pgfqpoint{2.947092in}{1.705651in}}{\pgfqpoint{2.947092in}{1.697415in}}%
\pgfpathcurveto{\pgfqpoint{2.947092in}{1.689178in}}{\pgfqpoint{2.950364in}{1.681278in}}{\pgfqpoint{2.956188in}{1.675454in}}%
\pgfpathcurveto{\pgfqpoint{2.962012in}{1.669630in}}{\pgfqpoint{2.969912in}{1.666358in}}{\pgfqpoint{2.978148in}{1.666358in}}%
\pgfpathclose%
\pgfusepath{stroke,fill}%
\end{pgfscope}%
\begin{pgfscope}%
\pgfpathrectangle{\pgfqpoint{0.100000in}{0.212622in}}{\pgfqpoint{3.696000in}{3.696000in}}%
\pgfusepath{clip}%
\pgfsetbuttcap%
\pgfsetroundjoin%
\definecolor{currentfill}{rgb}{0.121569,0.466667,0.705882}%
\pgfsetfillcolor{currentfill}%
\pgfsetfillopacity{0.744044}%
\pgfsetlinewidth{1.003750pt}%
\definecolor{currentstroke}{rgb}{0.121569,0.466667,0.705882}%
\pgfsetstrokecolor{currentstroke}%
\pgfsetstrokeopacity{0.744044}%
\pgfsetdash{}{0pt}%
\pgfpathmoveto{\pgfqpoint{1.333258in}{2.589687in}}%
\pgfpathcurveto{\pgfqpoint{1.341494in}{2.589687in}}{\pgfqpoint{1.349394in}{2.592959in}}{\pgfqpoint{1.355218in}{2.598783in}}%
\pgfpathcurveto{\pgfqpoint{1.361042in}{2.604607in}}{\pgfqpoint{1.364314in}{2.612507in}}{\pgfqpoint{1.364314in}{2.620743in}}%
\pgfpathcurveto{\pgfqpoint{1.364314in}{2.628980in}}{\pgfqpoint{1.361042in}{2.636880in}}{\pgfqpoint{1.355218in}{2.642704in}}%
\pgfpathcurveto{\pgfqpoint{1.349394in}{2.648528in}}{\pgfqpoint{1.341494in}{2.651800in}}{\pgfqpoint{1.333258in}{2.651800in}}%
\pgfpathcurveto{\pgfqpoint{1.325021in}{2.651800in}}{\pgfqpoint{1.317121in}{2.648528in}}{\pgfqpoint{1.311297in}{2.642704in}}%
\pgfpathcurveto{\pgfqpoint{1.305473in}{2.636880in}}{\pgfqpoint{1.302201in}{2.628980in}}{\pgfqpoint{1.302201in}{2.620743in}}%
\pgfpathcurveto{\pgfqpoint{1.302201in}{2.612507in}}{\pgfqpoint{1.305473in}{2.604607in}}{\pgfqpoint{1.311297in}{2.598783in}}%
\pgfpathcurveto{\pgfqpoint{1.317121in}{2.592959in}}{\pgfqpoint{1.325021in}{2.589687in}}{\pgfqpoint{1.333258in}{2.589687in}}%
\pgfpathclose%
\pgfusepath{stroke,fill}%
\end{pgfscope}%
\begin{pgfscope}%
\pgfpathrectangle{\pgfqpoint{0.100000in}{0.212622in}}{\pgfqpoint{3.696000in}{3.696000in}}%
\pgfusepath{clip}%
\pgfsetbuttcap%
\pgfsetroundjoin%
\definecolor{currentfill}{rgb}{0.121569,0.466667,0.705882}%
\pgfsetfillcolor{currentfill}%
\pgfsetfillopacity{0.744743}%
\pgfsetlinewidth{1.003750pt}%
\definecolor{currentstroke}{rgb}{0.121569,0.466667,0.705882}%
\pgfsetstrokecolor{currentstroke}%
\pgfsetstrokeopacity{0.744743}%
\pgfsetdash{}{0pt}%
\pgfpathmoveto{\pgfqpoint{1.331051in}{2.587152in}}%
\pgfpathcurveto{\pgfqpoint{1.339287in}{2.587152in}}{\pgfqpoint{1.347187in}{2.590424in}}{\pgfqpoint{1.353011in}{2.596248in}}%
\pgfpathcurveto{\pgfqpoint{1.358835in}{2.602072in}}{\pgfqpoint{1.362107in}{2.609972in}}{\pgfqpoint{1.362107in}{2.618208in}}%
\pgfpathcurveto{\pgfqpoint{1.362107in}{2.626445in}}{\pgfqpoint{1.358835in}{2.634345in}}{\pgfqpoint{1.353011in}{2.640169in}}%
\pgfpathcurveto{\pgfqpoint{1.347187in}{2.645993in}}{\pgfqpoint{1.339287in}{2.649265in}}{\pgfqpoint{1.331051in}{2.649265in}}%
\pgfpathcurveto{\pgfqpoint{1.322814in}{2.649265in}}{\pgfqpoint{1.314914in}{2.645993in}}{\pgfqpoint{1.309090in}{2.640169in}}%
\pgfpathcurveto{\pgfqpoint{1.303266in}{2.634345in}}{\pgfqpoint{1.299994in}{2.626445in}}{\pgfqpoint{1.299994in}{2.618208in}}%
\pgfpathcurveto{\pgfqpoint{1.299994in}{2.609972in}}{\pgfqpoint{1.303266in}{2.602072in}}{\pgfqpoint{1.309090in}{2.596248in}}%
\pgfpathcurveto{\pgfqpoint{1.314914in}{2.590424in}}{\pgfqpoint{1.322814in}{2.587152in}}{\pgfqpoint{1.331051in}{2.587152in}}%
\pgfpathclose%
\pgfusepath{stroke,fill}%
\end{pgfscope}%
\begin{pgfscope}%
\pgfpathrectangle{\pgfqpoint{0.100000in}{0.212622in}}{\pgfqpoint{3.696000in}{3.696000in}}%
\pgfusepath{clip}%
\pgfsetbuttcap%
\pgfsetroundjoin%
\definecolor{currentfill}{rgb}{0.121569,0.466667,0.705882}%
\pgfsetfillcolor{currentfill}%
\pgfsetfillopacity{0.744941}%
\pgfsetlinewidth{1.003750pt}%
\definecolor{currentstroke}{rgb}{0.121569,0.466667,0.705882}%
\pgfsetstrokecolor{currentstroke}%
\pgfsetstrokeopacity{0.744941}%
\pgfsetdash{}{0pt}%
\pgfpathmoveto{\pgfqpoint{2.543756in}{2.913005in}}%
\pgfpathcurveto{\pgfqpoint{2.551993in}{2.913005in}}{\pgfqpoint{2.559893in}{2.916277in}}{\pgfqpoint{2.565717in}{2.922101in}}%
\pgfpathcurveto{\pgfqpoint{2.571541in}{2.927925in}}{\pgfqpoint{2.574813in}{2.935825in}}{\pgfqpoint{2.574813in}{2.944061in}}%
\pgfpathcurveto{\pgfqpoint{2.574813in}{2.952298in}}{\pgfqpoint{2.571541in}{2.960198in}}{\pgfqpoint{2.565717in}{2.966022in}}%
\pgfpathcurveto{\pgfqpoint{2.559893in}{2.971845in}}{\pgfqpoint{2.551993in}{2.975118in}}{\pgfqpoint{2.543756in}{2.975118in}}%
\pgfpathcurveto{\pgfqpoint{2.535520in}{2.975118in}}{\pgfqpoint{2.527620in}{2.971845in}}{\pgfqpoint{2.521796in}{2.966022in}}%
\pgfpathcurveto{\pgfqpoint{2.515972in}{2.960198in}}{\pgfqpoint{2.512700in}{2.952298in}}{\pgfqpoint{2.512700in}{2.944061in}}%
\pgfpathcurveto{\pgfqpoint{2.512700in}{2.935825in}}{\pgfqpoint{2.515972in}{2.927925in}}{\pgfqpoint{2.521796in}{2.922101in}}%
\pgfpathcurveto{\pgfqpoint{2.527620in}{2.916277in}}{\pgfqpoint{2.535520in}{2.913005in}}{\pgfqpoint{2.543756in}{2.913005in}}%
\pgfpathclose%
\pgfusepath{stroke,fill}%
\end{pgfscope}%
\begin{pgfscope}%
\pgfpathrectangle{\pgfqpoint{0.100000in}{0.212622in}}{\pgfqpoint{3.696000in}{3.696000in}}%
\pgfusepath{clip}%
\pgfsetbuttcap%
\pgfsetroundjoin%
\definecolor{currentfill}{rgb}{0.121569,0.466667,0.705882}%
\pgfsetfillcolor{currentfill}%
\pgfsetfillopacity{0.745693}%
\pgfsetlinewidth{1.003750pt}%
\definecolor{currentstroke}{rgb}{0.121569,0.466667,0.705882}%
\pgfsetstrokecolor{currentstroke}%
\pgfsetstrokeopacity{0.745693}%
\pgfsetdash{}{0pt}%
\pgfpathmoveto{\pgfqpoint{2.542059in}{2.910657in}}%
\pgfpathcurveto{\pgfqpoint{2.550295in}{2.910657in}}{\pgfqpoint{2.558195in}{2.913929in}}{\pgfqpoint{2.564019in}{2.919753in}}%
\pgfpathcurveto{\pgfqpoint{2.569843in}{2.925577in}}{\pgfqpoint{2.573115in}{2.933477in}}{\pgfqpoint{2.573115in}{2.941713in}}%
\pgfpathcurveto{\pgfqpoint{2.573115in}{2.949950in}}{\pgfqpoint{2.569843in}{2.957850in}}{\pgfqpoint{2.564019in}{2.963674in}}%
\pgfpathcurveto{\pgfqpoint{2.558195in}{2.969498in}}{\pgfqpoint{2.550295in}{2.972770in}}{\pgfqpoint{2.542059in}{2.972770in}}%
\pgfpathcurveto{\pgfqpoint{2.533823in}{2.972770in}}{\pgfqpoint{2.525923in}{2.969498in}}{\pgfqpoint{2.520099in}{2.963674in}}%
\pgfpathcurveto{\pgfqpoint{2.514275in}{2.957850in}}{\pgfqpoint{2.511002in}{2.949950in}}{\pgfqpoint{2.511002in}{2.941713in}}%
\pgfpathcurveto{\pgfqpoint{2.511002in}{2.933477in}}{\pgfqpoint{2.514275in}{2.925577in}}{\pgfqpoint{2.520099in}{2.919753in}}%
\pgfpathcurveto{\pgfqpoint{2.525923in}{2.913929in}}{\pgfqpoint{2.533823in}{2.910657in}}{\pgfqpoint{2.542059in}{2.910657in}}%
\pgfpathclose%
\pgfusepath{stroke,fill}%
\end{pgfscope}%
\begin{pgfscope}%
\pgfpathrectangle{\pgfqpoint{0.100000in}{0.212622in}}{\pgfqpoint{3.696000in}{3.696000in}}%
\pgfusepath{clip}%
\pgfsetbuttcap%
\pgfsetroundjoin%
\definecolor{currentfill}{rgb}{0.121569,0.466667,0.705882}%
\pgfsetfillcolor{currentfill}%
\pgfsetfillopacity{0.745697}%
\pgfsetlinewidth{1.003750pt}%
\definecolor{currentstroke}{rgb}{0.121569,0.466667,0.705882}%
\pgfsetstrokecolor{currentstroke}%
\pgfsetstrokeopacity{0.745697}%
\pgfsetdash{}{0pt}%
\pgfpathmoveto{\pgfqpoint{1.327835in}{2.582960in}}%
\pgfpathcurveto{\pgfqpoint{1.336071in}{2.582960in}}{\pgfqpoint{1.343971in}{2.586232in}}{\pgfqpoint{1.349795in}{2.592056in}}%
\pgfpathcurveto{\pgfqpoint{1.355619in}{2.597880in}}{\pgfqpoint{1.358891in}{2.605780in}}{\pgfqpoint{1.358891in}{2.614017in}}%
\pgfpathcurveto{\pgfqpoint{1.358891in}{2.622253in}}{\pgfqpoint{1.355619in}{2.630153in}}{\pgfqpoint{1.349795in}{2.635977in}}%
\pgfpathcurveto{\pgfqpoint{1.343971in}{2.641801in}}{\pgfqpoint{1.336071in}{2.645073in}}{\pgfqpoint{1.327835in}{2.645073in}}%
\pgfpathcurveto{\pgfqpoint{1.319598in}{2.645073in}}{\pgfqpoint{1.311698in}{2.641801in}}{\pgfqpoint{1.305874in}{2.635977in}}%
\pgfpathcurveto{\pgfqpoint{1.300050in}{2.630153in}}{\pgfqpoint{1.296778in}{2.622253in}}{\pgfqpoint{1.296778in}{2.614017in}}%
\pgfpathcurveto{\pgfqpoint{1.296778in}{2.605780in}}{\pgfqpoint{1.300050in}{2.597880in}}{\pgfqpoint{1.305874in}{2.592056in}}%
\pgfpathcurveto{\pgfqpoint{1.311698in}{2.586232in}}{\pgfqpoint{1.319598in}{2.582960in}}{\pgfqpoint{1.327835in}{2.582960in}}%
\pgfpathclose%
\pgfusepath{stroke,fill}%
\end{pgfscope}%
\begin{pgfscope}%
\pgfpathrectangle{\pgfqpoint{0.100000in}{0.212622in}}{\pgfqpoint{3.696000in}{3.696000in}}%
\pgfusepath{clip}%
\pgfsetbuttcap%
\pgfsetroundjoin%
\definecolor{currentfill}{rgb}{0.121569,0.466667,0.705882}%
\pgfsetfillcolor{currentfill}%
\pgfsetfillopacity{0.746243}%
\pgfsetlinewidth{1.003750pt}%
\definecolor{currentstroke}{rgb}{0.121569,0.466667,0.705882}%
\pgfsetstrokecolor{currentstroke}%
\pgfsetstrokeopacity{0.746243}%
\pgfsetdash{}{0pt}%
\pgfpathmoveto{\pgfqpoint{1.326105in}{2.580726in}}%
\pgfpathcurveto{\pgfqpoint{1.334342in}{2.580726in}}{\pgfqpoint{1.342242in}{2.583998in}}{\pgfqpoint{1.348066in}{2.589822in}}%
\pgfpathcurveto{\pgfqpoint{1.353889in}{2.595646in}}{\pgfqpoint{1.357162in}{2.603546in}}{\pgfqpoint{1.357162in}{2.611782in}}%
\pgfpathcurveto{\pgfqpoint{1.357162in}{2.620019in}}{\pgfqpoint{1.353889in}{2.627919in}}{\pgfqpoint{1.348066in}{2.633743in}}%
\pgfpathcurveto{\pgfqpoint{1.342242in}{2.639567in}}{\pgfqpoint{1.334342in}{2.642839in}}{\pgfqpoint{1.326105in}{2.642839in}}%
\pgfpathcurveto{\pgfqpoint{1.317869in}{2.642839in}}{\pgfqpoint{1.309969in}{2.639567in}}{\pgfqpoint{1.304145in}{2.633743in}}%
\pgfpathcurveto{\pgfqpoint{1.298321in}{2.627919in}}{\pgfqpoint{1.295049in}{2.620019in}}{\pgfqpoint{1.295049in}{2.611782in}}%
\pgfpathcurveto{\pgfqpoint{1.295049in}{2.603546in}}{\pgfqpoint{1.298321in}{2.595646in}}{\pgfqpoint{1.304145in}{2.589822in}}%
\pgfpathcurveto{\pgfqpoint{1.309969in}{2.583998in}}{\pgfqpoint{1.317869in}{2.580726in}}{\pgfqpoint{1.326105in}{2.580726in}}%
\pgfpathclose%
\pgfusepath{stroke,fill}%
\end{pgfscope}%
\begin{pgfscope}%
\pgfpathrectangle{\pgfqpoint{0.100000in}{0.212622in}}{\pgfqpoint{3.696000in}{3.696000in}}%
\pgfusepath{clip}%
\pgfsetbuttcap%
\pgfsetroundjoin%
\definecolor{currentfill}{rgb}{0.121569,0.466667,0.705882}%
\pgfsetfillcolor{currentfill}%
\pgfsetfillopacity{0.746327}%
\pgfsetlinewidth{1.003750pt}%
\definecolor{currentstroke}{rgb}{0.121569,0.466667,0.705882}%
\pgfsetstrokecolor{currentstroke}%
\pgfsetstrokeopacity{0.746327}%
\pgfsetdash{}{0pt}%
\pgfpathmoveto{\pgfqpoint{2.540709in}{2.908746in}}%
\pgfpathcurveto{\pgfqpoint{2.548946in}{2.908746in}}{\pgfqpoint{2.556846in}{2.912018in}}{\pgfqpoint{2.562670in}{2.917842in}}%
\pgfpathcurveto{\pgfqpoint{2.568493in}{2.923666in}}{\pgfqpoint{2.571766in}{2.931566in}}{\pgfqpoint{2.571766in}{2.939802in}}%
\pgfpathcurveto{\pgfqpoint{2.571766in}{2.948039in}}{\pgfqpoint{2.568493in}{2.955939in}}{\pgfqpoint{2.562670in}{2.961763in}}%
\pgfpathcurveto{\pgfqpoint{2.556846in}{2.967586in}}{\pgfqpoint{2.548946in}{2.970859in}}{\pgfqpoint{2.540709in}{2.970859in}}%
\pgfpathcurveto{\pgfqpoint{2.532473in}{2.970859in}}{\pgfqpoint{2.524573in}{2.967586in}}{\pgfqpoint{2.518749in}{2.961763in}}%
\pgfpathcurveto{\pgfqpoint{2.512925in}{2.955939in}}{\pgfqpoint{2.509653in}{2.948039in}}{\pgfqpoint{2.509653in}{2.939802in}}%
\pgfpathcurveto{\pgfqpoint{2.509653in}{2.931566in}}{\pgfqpoint{2.512925in}{2.923666in}}{\pgfqpoint{2.518749in}{2.917842in}}%
\pgfpathcurveto{\pgfqpoint{2.524573in}{2.912018in}}{\pgfqpoint{2.532473in}{2.908746in}}{\pgfqpoint{2.540709in}{2.908746in}}%
\pgfpathclose%
\pgfusepath{stroke,fill}%
\end{pgfscope}%
\begin{pgfscope}%
\pgfpathrectangle{\pgfqpoint{0.100000in}{0.212622in}}{\pgfqpoint{3.696000in}{3.696000in}}%
\pgfusepath{clip}%
\pgfsetbuttcap%
\pgfsetroundjoin%
\definecolor{currentfill}{rgb}{0.121569,0.466667,0.705882}%
\pgfsetfillcolor{currentfill}%
\pgfsetfillopacity{0.746879}%
\pgfsetlinewidth{1.003750pt}%
\definecolor{currentstroke}{rgb}{0.121569,0.466667,0.705882}%
\pgfsetstrokecolor{currentstroke}%
\pgfsetstrokeopacity{0.746879}%
\pgfsetdash{}{0pt}%
\pgfpathmoveto{\pgfqpoint{2.539675in}{2.907250in}}%
\pgfpathcurveto{\pgfqpoint{2.547911in}{2.907250in}}{\pgfqpoint{2.555811in}{2.910523in}}{\pgfqpoint{2.561635in}{2.916347in}}%
\pgfpathcurveto{\pgfqpoint{2.567459in}{2.922171in}}{\pgfqpoint{2.570731in}{2.930071in}}{\pgfqpoint{2.570731in}{2.938307in}}%
\pgfpathcurveto{\pgfqpoint{2.570731in}{2.946543in}}{\pgfqpoint{2.567459in}{2.954443in}}{\pgfqpoint{2.561635in}{2.960267in}}%
\pgfpathcurveto{\pgfqpoint{2.555811in}{2.966091in}}{\pgfqpoint{2.547911in}{2.969363in}}{\pgfqpoint{2.539675in}{2.969363in}}%
\pgfpathcurveto{\pgfqpoint{2.531438in}{2.969363in}}{\pgfqpoint{2.523538in}{2.966091in}}{\pgfqpoint{2.517714in}{2.960267in}}%
\pgfpathcurveto{\pgfqpoint{2.511890in}{2.954443in}}{\pgfqpoint{2.508618in}{2.946543in}}{\pgfqpoint{2.508618in}{2.938307in}}%
\pgfpathcurveto{\pgfqpoint{2.508618in}{2.930071in}}{\pgfqpoint{2.511890in}{2.922171in}}{\pgfqpoint{2.517714in}{2.916347in}}%
\pgfpathcurveto{\pgfqpoint{2.523538in}{2.910523in}}{\pgfqpoint{2.531438in}{2.907250in}}{\pgfqpoint{2.539675in}{2.907250in}}%
\pgfpathclose%
\pgfusepath{stroke,fill}%
\end{pgfscope}%
\begin{pgfscope}%
\pgfpathrectangle{\pgfqpoint{0.100000in}{0.212622in}}{\pgfqpoint{3.696000in}{3.696000in}}%
\pgfusepath{clip}%
\pgfsetbuttcap%
\pgfsetroundjoin%
\definecolor{currentfill}{rgb}{0.121569,0.466667,0.705882}%
\pgfsetfillcolor{currentfill}%
\pgfsetfillopacity{0.747081}%
\pgfsetlinewidth{1.003750pt}%
\definecolor{currentstroke}{rgb}{0.121569,0.466667,0.705882}%
\pgfsetstrokecolor{currentstroke}%
\pgfsetstrokeopacity{0.747081}%
\pgfsetdash{}{0pt}%
\pgfpathmoveto{\pgfqpoint{1.323639in}{2.577721in}}%
\pgfpathcurveto{\pgfqpoint{1.331875in}{2.577721in}}{\pgfqpoint{1.339775in}{2.580993in}}{\pgfqpoint{1.345599in}{2.586817in}}%
\pgfpathcurveto{\pgfqpoint{1.351423in}{2.592641in}}{\pgfqpoint{1.354696in}{2.600541in}}{\pgfqpoint{1.354696in}{2.608778in}}%
\pgfpathcurveto{\pgfqpoint{1.354696in}{2.617014in}}{\pgfqpoint{1.351423in}{2.624914in}}{\pgfqpoint{1.345599in}{2.630738in}}%
\pgfpathcurveto{\pgfqpoint{1.339775in}{2.636562in}}{\pgfqpoint{1.331875in}{2.639834in}}{\pgfqpoint{1.323639in}{2.639834in}}%
\pgfpathcurveto{\pgfqpoint{1.315403in}{2.639834in}}{\pgfqpoint{1.307503in}{2.636562in}}{\pgfqpoint{1.301679in}{2.630738in}}%
\pgfpathcurveto{\pgfqpoint{1.295855in}{2.624914in}}{\pgfqpoint{1.292583in}{2.617014in}}{\pgfqpoint{1.292583in}{2.608778in}}%
\pgfpathcurveto{\pgfqpoint{1.292583in}{2.600541in}}{\pgfqpoint{1.295855in}{2.592641in}}{\pgfqpoint{1.301679in}{2.586817in}}%
\pgfpathcurveto{\pgfqpoint{1.307503in}{2.580993in}}{\pgfqpoint{1.315403in}{2.577721in}}{\pgfqpoint{1.323639in}{2.577721in}}%
\pgfpathclose%
\pgfusepath{stroke,fill}%
\end{pgfscope}%
\begin{pgfscope}%
\pgfpathrectangle{\pgfqpoint{0.100000in}{0.212622in}}{\pgfqpoint{3.696000in}{3.696000in}}%
\pgfusepath{clip}%
\pgfsetbuttcap%
\pgfsetroundjoin%
\definecolor{currentfill}{rgb}{0.121569,0.466667,0.705882}%
\pgfsetfillcolor{currentfill}%
\pgfsetfillopacity{0.747589}%
\pgfsetlinewidth{1.003750pt}%
\definecolor{currentstroke}{rgb}{0.121569,0.466667,0.705882}%
\pgfsetstrokecolor{currentstroke}%
\pgfsetstrokeopacity{0.747589}%
\pgfsetdash{}{0pt}%
\pgfpathmoveto{\pgfqpoint{1.322274in}{2.576354in}}%
\pgfpathcurveto{\pgfqpoint{1.330510in}{2.576354in}}{\pgfqpoint{1.338410in}{2.579626in}}{\pgfqpoint{1.344234in}{2.585450in}}%
\pgfpathcurveto{\pgfqpoint{1.350058in}{2.591274in}}{\pgfqpoint{1.353330in}{2.599174in}}{\pgfqpoint{1.353330in}{2.607410in}}%
\pgfpathcurveto{\pgfqpoint{1.353330in}{2.615646in}}{\pgfqpoint{1.350058in}{2.623547in}}{\pgfqpoint{1.344234in}{2.629370in}}%
\pgfpathcurveto{\pgfqpoint{1.338410in}{2.635194in}}{\pgfqpoint{1.330510in}{2.638467in}}{\pgfqpoint{1.322274in}{2.638467in}}%
\pgfpathcurveto{\pgfqpoint{1.314037in}{2.638467in}}{\pgfqpoint{1.306137in}{2.635194in}}{\pgfqpoint{1.300313in}{2.629370in}}%
\pgfpathcurveto{\pgfqpoint{1.294490in}{2.623547in}}{\pgfqpoint{1.291217in}{2.615646in}}{\pgfqpoint{1.291217in}{2.607410in}}%
\pgfpathcurveto{\pgfqpoint{1.291217in}{2.599174in}}{\pgfqpoint{1.294490in}{2.591274in}}{\pgfqpoint{1.300313in}{2.585450in}}%
\pgfpathcurveto{\pgfqpoint{1.306137in}{2.579626in}}{\pgfqpoint{1.314037in}{2.576354in}}{\pgfqpoint{1.322274in}{2.576354in}}%
\pgfpathclose%
\pgfusepath{stroke,fill}%
\end{pgfscope}%
\begin{pgfscope}%
\pgfpathrectangle{\pgfqpoint{0.100000in}{0.212622in}}{\pgfqpoint{3.696000in}{3.696000in}}%
\pgfusepath{clip}%
\pgfsetbuttcap%
\pgfsetroundjoin%
\definecolor{currentfill}{rgb}{0.121569,0.466667,0.705882}%
\pgfsetfillcolor{currentfill}%
\pgfsetfillopacity{0.747915}%
\pgfsetlinewidth{1.003750pt}%
\definecolor{currentstroke}{rgb}{0.121569,0.466667,0.705882}%
\pgfsetstrokecolor{currentstroke}%
\pgfsetstrokeopacity{0.747915}%
\pgfsetdash{}{0pt}%
\pgfpathmoveto{\pgfqpoint{2.537949in}{2.904641in}}%
\pgfpathcurveto{\pgfqpoint{2.546185in}{2.904641in}}{\pgfqpoint{2.554085in}{2.907913in}}{\pgfqpoint{2.559909in}{2.913737in}}%
\pgfpathcurveto{\pgfqpoint{2.565733in}{2.919561in}}{\pgfqpoint{2.569006in}{2.927461in}}{\pgfqpoint{2.569006in}{2.935697in}}%
\pgfpathcurveto{\pgfqpoint{2.569006in}{2.943934in}}{\pgfqpoint{2.565733in}{2.951834in}}{\pgfqpoint{2.559909in}{2.957658in}}%
\pgfpathcurveto{\pgfqpoint{2.554085in}{2.963482in}}{\pgfqpoint{2.546185in}{2.966754in}}{\pgfqpoint{2.537949in}{2.966754in}}%
\pgfpathcurveto{\pgfqpoint{2.529713in}{2.966754in}}{\pgfqpoint{2.521813in}{2.963482in}}{\pgfqpoint{2.515989in}{2.957658in}}%
\pgfpathcurveto{\pgfqpoint{2.510165in}{2.951834in}}{\pgfqpoint{2.506893in}{2.943934in}}{\pgfqpoint{2.506893in}{2.935697in}}%
\pgfpathcurveto{\pgfqpoint{2.506893in}{2.927461in}}{\pgfqpoint{2.510165in}{2.919561in}}{\pgfqpoint{2.515989in}{2.913737in}}%
\pgfpathcurveto{\pgfqpoint{2.521813in}{2.907913in}}{\pgfqpoint{2.529713in}{2.904641in}}{\pgfqpoint{2.537949in}{2.904641in}}%
\pgfpathclose%
\pgfusepath{stroke,fill}%
\end{pgfscope}%
\begin{pgfscope}%
\pgfpathrectangle{\pgfqpoint{0.100000in}{0.212622in}}{\pgfqpoint{3.696000in}{3.696000in}}%
\pgfusepath{clip}%
\pgfsetbuttcap%
\pgfsetroundjoin%
\definecolor{currentfill}{rgb}{0.121569,0.466667,0.705882}%
\pgfsetfillcolor{currentfill}%
\pgfsetfillopacity{0.748411}%
\pgfsetlinewidth{1.003750pt}%
\definecolor{currentstroke}{rgb}{0.121569,0.466667,0.705882}%
\pgfsetstrokecolor{currentstroke}%
\pgfsetstrokeopacity{0.748411}%
\pgfsetdash{}{0pt}%
\pgfpathmoveto{\pgfqpoint{1.319897in}{2.573798in}}%
\pgfpathcurveto{\pgfqpoint{1.328134in}{2.573798in}}{\pgfqpoint{1.336034in}{2.577070in}}{\pgfqpoint{1.341858in}{2.582894in}}%
\pgfpathcurveto{\pgfqpoint{1.347682in}{2.588718in}}{\pgfqpoint{1.350954in}{2.596618in}}{\pgfqpoint{1.350954in}{2.604854in}}%
\pgfpathcurveto{\pgfqpoint{1.350954in}{2.613091in}}{\pgfqpoint{1.347682in}{2.620991in}}{\pgfqpoint{1.341858in}{2.626815in}}%
\pgfpathcurveto{\pgfqpoint{1.336034in}{2.632639in}}{\pgfqpoint{1.328134in}{2.635911in}}{\pgfqpoint{1.319897in}{2.635911in}}%
\pgfpathcurveto{\pgfqpoint{1.311661in}{2.635911in}}{\pgfqpoint{1.303761in}{2.632639in}}{\pgfqpoint{1.297937in}{2.626815in}}%
\pgfpathcurveto{\pgfqpoint{1.292113in}{2.620991in}}{\pgfqpoint{1.288841in}{2.613091in}}{\pgfqpoint{1.288841in}{2.604854in}}%
\pgfpathcurveto{\pgfqpoint{1.288841in}{2.596618in}}{\pgfqpoint{1.292113in}{2.588718in}}{\pgfqpoint{1.297937in}{2.582894in}}%
\pgfpathcurveto{\pgfqpoint{1.303761in}{2.577070in}}{\pgfqpoint{1.311661in}{2.573798in}}{\pgfqpoint{1.319897in}{2.573798in}}%
\pgfpathclose%
\pgfusepath{stroke,fill}%
\end{pgfscope}%
\begin{pgfscope}%
\pgfpathrectangle{\pgfqpoint{0.100000in}{0.212622in}}{\pgfqpoint{3.696000in}{3.696000in}}%
\pgfusepath{clip}%
\pgfsetbuttcap%
\pgfsetroundjoin%
\definecolor{currentfill}{rgb}{0.121569,0.466667,0.705882}%
\pgfsetfillcolor{currentfill}%
\pgfsetfillopacity{0.748501}%
\pgfsetlinewidth{1.003750pt}%
\definecolor{currentstroke}{rgb}{0.121569,0.466667,0.705882}%
\pgfsetstrokecolor{currentstroke}%
\pgfsetstrokeopacity{0.748501}%
\pgfsetdash{}{0pt}%
\pgfpathmoveto{\pgfqpoint{2.536992in}{2.902995in}}%
\pgfpathcurveto{\pgfqpoint{2.545228in}{2.902995in}}{\pgfqpoint{2.553128in}{2.906268in}}{\pgfqpoint{2.558952in}{2.912091in}}%
\pgfpathcurveto{\pgfqpoint{2.564776in}{2.917915in}}{\pgfqpoint{2.568048in}{2.925815in}}{\pgfqpoint{2.568048in}{2.934052in}}%
\pgfpathcurveto{\pgfqpoint{2.568048in}{2.942288in}}{\pgfqpoint{2.564776in}{2.950188in}}{\pgfqpoint{2.558952in}{2.956012in}}%
\pgfpathcurveto{\pgfqpoint{2.553128in}{2.961836in}}{\pgfqpoint{2.545228in}{2.965108in}}{\pgfqpoint{2.536992in}{2.965108in}}%
\pgfpathcurveto{\pgfqpoint{2.528755in}{2.965108in}}{\pgfqpoint{2.520855in}{2.961836in}}{\pgfqpoint{2.515031in}{2.956012in}}%
\pgfpathcurveto{\pgfqpoint{2.509207in}{2.950188in}}{\pgfqpoint{2.505935in}{2.942288in}}{\pgfqpoint{2.505935in}{2.934052in}}%
\pgfpathcurveto{\pgfqpoint{2.505935in}{2.925815in}}{\pgfqpoint{2.509207in}{2.917915in}}{\pgfqpoint{2.515031in}{2.912091in}}%
\pgfpathcurveto{\pgfqpoint{2.520855in}{2.906268in}}{\pgfqpoint{2.528755in}{2.902995in}}{\pgfqpoint{2.536992in}{2.902995in}}%
\pgfpathclose%
\pgfusepath{stroke,fill}%
\end{pgfscope}%
\begin{pgfscope}%
\pgfpathrectangle{\pgfqpoint{0.100000in}{0.212622in}}{\pgfqpoint{3.696000in}{3.696000in}}%
\pgfusepath{clip}%
\pgfsetbuttcap%
\pgfsetroundjoin%
\definecolor{currentfill}{rgb}{0.121569,0.466667,0.705882}%
\pgfsetfillcolor{currentfill}%
\pgfsetfillopacity{0.748825}%
\pgfsetlinewidth{1.003750pt}%
\definecolor{currentstroke}{rgb}{0.121569,0.466667,0.705882}%
\pgfsetstrokecolor{currentstroke}%
\pgfsetstrokeopacity{0.748825}%
\pgfsetdash{}{0pt}%
\pgfpathmoveto{\pgfqpoint{2.968928in}{1.655572in}}%
\pgfpathcurveto{\pgfqpoint{2.977164in}{1.655572in}}{\pgfqpoint{2.985064in}{1.658844in}}{\pgfqpoint{2.990888in}{1.664668in}}%
\pgfpathcurveto{\pgfqpoint{2.996712in}{1.670492in}}{\pgfqpoint{2.999984in}{1.678392in}}{\pgfqpoint{2.999984in}{1.686629in}}%
\pgfpathcurveto{\pgfqpoint{2.999984in}{1.694865in}}{\pgfqpoint{2.996712in}{1.702765in}}{\pgfqpoint{2.990888in}{1.708589in}}%
\pgfpathcurveto{\pgfqpoint{2.985064in}{1.714413in}}{\pgfqpoint{2.977164in}{1.717685in}}{\pgfqpoint{2.968928in}{1.717685in}}%
\pgfpathcurveto{\pgfqpoint{2.960692in}{1.717685in}}{\pgfqpoint{2.952792in}{1.714413in}}{\pgfqpoint{2.946968in}{1.708589in}}%
\pgfpathcurveto{\pgfqpoint{2.941144in}{1.702765in}}{\pgfqpoint{2.937871in}{1.694865in}}{\pgfqpoint{2.937871in}{1.686629in}}%
\pgfpathcurveto{\pgfqpoint{2.937871in}{1.678392in}}{\pgfqpoint{2.941144in}{1.670492in}}{\pgfqpoint{2.946968in}{1.664668in}}%
\pgfpathcurveto{\pgfqpoint{2.952792in}{1.658844in}}{\pgfqpoint{2.960692in}{1.655572in}}{\pgfqpoint{2.968928in}{1.655572in}}%
\pgfpathclose%
\pgfusepath{stroke,fill}%
\end{pgfscope}%
\begin{pgfscope}%
\pgfpathrectangle{\pgfqpoint{0.100000in}{0.212622in}}{\pgfqpoint{3.696000in}{3.696000in}}%
\pgfusepath{clip}%
\pgfsetbuttcap%
\pgfsetroundjoin%
\definecolor{currentfill}{rgb}{0.121569,0.466667,0.705882}%
\pgfsetfillcolor{currentfill}%
\pgfsetfillopacity{0.749381}%
\pgfsetlinewidth{1.003750pt}%
\definecolor{currentstroke}{rgb}{0.121569,0.466667,0.705882}%
\pgfsetstrokecolor{currentstroke}%
\pgfsetstrokeopacity{0.749381}%
\pgfsetdash{}{0pt}%
\pgfpathmoveto{\pgfqpoint{1.316847in}{2.570088in}}%
\pgfpathcurveto{\pgfqpoint{1.325083in}{2.570088in}}{\pgfqpoint{1.332983in}{2.573361in}}{\pgfqpoint{1.338807in}{2.579185in}}%
\pgfpathcurveto{\pgfqpoint{1.344631in}{2.585009in}}{\pgfqpoint{1.347903in}{2.592909in}}{\pgfqpoint{1.347903in}{2.601145in}}%
\pgfpathcurveto{\pgfqpoint{1.347903in}{2.609381in}}{\pgfqpoint{1.344631in}{2.617281in}}{\pgfqpoint{1.338807in}{2.623105in}}%
\pgfpathcurveto{\pgfqpoint{1.332983in}{2.628929in}}{\pgfqpoint{1.325083in}{2.632201in}}{\pgfqpoint{1.316847in}{2.632201in}}%
\pgfpathcurveto{\pgfqpoint{1.308611in}{2.632201in}}{\pgfqpoint{1.300711in}{2.628929in}}{\pgfqpoint{1.294887in}{2.623105in}}%
\pgfpathcurveto{\pgfqpoint{1.289063in}{2.617281in}}{\pgfqpoint{1.285790in}{2.609381in}}{\pgfqpoint{1.285790in}{2.601145in}}%
\pgfpathcurveto{\pgfqpoint{1.285790in}{2.592909in}}{\pgfqpoint{1.289063in}{2.585009in}}{\pgfqpoint{1.294887in}{2.579185in}}%
\pgfpathcurveto{\pgfqpoint{1.300711in}{2.573361in}}{\pgfqpoint{1.308611in}{2.570088in}}{\pgfqpoint{1.316847in}{2.570088in}}%
\pgfpathclose%
\pgfusepath{stroke,fill}%
\end{pgfscope}%
\begin{pgfscope}%
\pgfpathrectangle{\pgfqpoint{0.100000in}{0.212622in}}{\pgfqpoint{3.696000in}{3.696000in}}%
\pgfusepath{clip}%
\pgfsetbuttcap%
\pgfsetroundjoin%
\definecolor{currentfill}{rgb}{0.121569,0.466667,0.705882}%
\pgfsetfillcolor{currentfill}%
\pgfsetfillopacity{0.749557}%
\pgfsetlinewidth{1.003750pt}%
\definecolor{currentstroke}{rgb}{0.121569,0.466667,0.705882}%
\pgfsetstrokecolor{currentstroke}%
\pgfsetstrokeopacity{0.749557}%
\pgfsetdash{}{0pt}%
\pgfpathmoveto{\pgfqpoint{2.535161in}{2.899983in}}%
\pgfpathcurveto{\pgfqpoint{2.543397in}{2.899983in}}{\pgfqpoint{2.551297in}{2.903255in}}{\pgfqpoint{2.557121in}{2.909079in}}%
\pgfpathcurveto{\pgfqpoint{2.562945in}{2.914903in}}{\pgfqpoint{2.566218in}{2.922803in}}{\pgfqpoint{2.566218in}{2.931039in}}%
\pgfpathcurveto{\pgfqpoint{2.566218in}{2.939276in}}{\pgfqpoint{2.562945in}{2.947176in}}{\pgfqpoint{2.557121in}{2.953000in}}%
\pgfpathcurveto{\pgfqpoint{2.551297in}{2.958824in}}{\pgfqpoint{2.543397in}{2.962096in}}{\pgfqpoint{2.535161in}{2.962096in}}%
\pgfpathcurveto{\pgfqpoint{2.526925in}{2.962096in}}{\pgfqpoint{2.519025in}{2.958824in}}{\pgfqpoint{2.513201in}{2.953000in}}%
\pgfpathcurveto{\pgfqpoint{2.507377in}{2.947176in}}{\pgfqpoint{2.504105in}{2.939276in}}{\pgfqpoint{2.504105in}{2.931039in}}%
\pgfpathcurveto{\pgfqpoint{2.504105in}{2.922803in}}{\pgfqpoint{2.507377in}{2.914903in}}{\pgfqpoint{2.513201in}{2.909079in}}%
\pgfpathcurveto{\pgfqpoint{2.519025in}{2.903255in}}{\pgfqpoint{2.526925in}{2.899983in}}{\pgfqpoint{2.535161in}{2.899983in}}%
\pgfpathclose%
\pgfusepath{stroke,fill}%
\end{pgfscope}%
\begin{pgfscope}%
\pgfpathrectangle{\pgfqpoint{0.100000in}{0.212622in}}{\pgfqpoint{3.696000in}{3.696000in}}%
\pgfusepath{clip}%
\pgfsetbuttcap%
\pgfsetroundjoin%
\definecolor{currentfill}{rgb}{0.121569,0.466667,0.705882}%
\pgfsetfillcolor{currentfill}%
\pgfsetfillopacity{0.749899}%
\pgfsetlinewidth{1.003750pt}%
\definecolor{currentstroke}{rgb}{0.121569,0.466667,0.705882}%
\pgfsetstrokecolor{currentstroke}%
\pgfsetstrokeopacity{0.749899}%
\pgfsetdash{}{0pt}%
\pgfpathmoveto{\pgfqpoint{1.315183in}{2.567939in}}%
\pgfpathcurveto{\pgfqpoint{1.323419in}{2.567939in}}{\pgfqpoint{1.331319in}{2.571211in}}{\pgfqpoint{1.337143in}{2.577035in}}%
\pgfpathcurveto{\pgfqpoint{1.342967in}{2.582859in}}{\pgfqpoint{1.346239in}{2.590759in}}{\pgfqpoint{1.346239in}{2.598995in}}%
\pgfpathcurveto{\pgfqpoint{1.346239in}{2.607231in}}{\pgfqpoint{1.342967in}{2.615132in}}{\pgfqpoint{1.337143in}{2.620955in}}%
\pgfpathcurveto{\pgfqpoint{1.331319in}{2.626779in}}{\pgfqpoint{1.323419in}{2.630052in}}{\pgfqpoint{1.315183in}{2.630052in}}%
\pgfpathcurveto{\pgfqpoint{1.306947in}{2.630052in}}{\pgfqpoint{1.299047in}{2.626779in}}{\pgfqpoint{1.293223in}{2.620955in}}%
\pgfpathcurveto{\pgfqpoint{1.287399in}{2.615132in}}{\pgfqpoint{1.284126in}{2.607231in}}{\pgfqpoint{1.284126in}{2.598995in}}%
\pgfpathcurveto{\pgfqpoint{1.284126in}{2.590759in}}{\pgfqpoint{1.287399in}{2.582859in}}{\pgfqpoint{1.293223in}{2.577035in}}%
\pgfpathcurveto{\pgfqpoint{1.299047in}{2.571211in}}{\pgfqpoint{1.306947in}{2.567939in}}{\pgfqpoint{1.315183in}{2.567939in}}%
\pgfpathclose%
\pgfusepath{stroke,fill}%
\end{pgfscope}%
\begin{pgfscope}%
\pgfpathrectangle{\pgfqpoint{0.100000in}{0.212622in}}{\pgfqpoint{3.696000in}{3.696000in}}%
\pgfusepath{clip}%
\pgfsetbuttcap%
\pgfsetroundjoin%
\definecolor{currentfill}{rgb}{0.121569,0.466667,0.705882}%
\pgfsetfillcolor{currentfill}%
\pgfsetfillopacity{0.750246}%
\pgfsetlinewidth{1.003750pt}%
\definecolor{currentstroke}{rgb}{0.121569,0.466667,0.705882}%
\pgfsetstrokecolor{currentstroke}%
\pgfsetstrokeopacity{0.750246}%
\pgfsetdash{}{0pt}%
\pgfpathmoveto{\pgfqpoint{2.533998in}{2.898099in}}%
\pgfpathcurveto{\pgfqpoint{2.542234in}{2.898099in}}{\pgfqpoint{2.550134in}{2.901371in}}{\pgfqpoint{2.555958in}{2.907195in}}%
\pgfpathcurveto{\pgfqpoint{2.561782in}{2.913019in}}{\pgfqpoint{2.565054in}{2.920919in}}{\pgfqpoint{2.565054in}{2.929155in}}%
\pgfpathcurveto{\pgfqpoint{2.565054in}{2.937391in}}{\pgfqpoint{2.561782in}{2.945291in}}{\pgfqpoint{2.555958in}{2.951115in}}%
\pgfpathcurveto{\pgfqpoint{2.550134in}{2.956939in}}{\pgfqpoint{2.542234in}{2.960212in}}{\pgfqpoint{2.533998in}{2.960212in}}%
\pgfpathcurveto{\pgfqpoint{2.525761in}{2.960212in}}{\pgfqpoint{2.517861in}{2.956939in}}{\pgfqpoint{2.512037in}{2.951115in}}%
\pgfpathcurveto{\pgfqpoint{2.506213in}{2.945291in}}{\pgfqpoint{2.502941in}{2.937391in}}{\pgfqpoint{2.502941in}{2.929155in}}%
\pgfpathcurveto{\pgfqpoint{2.502941in}{2.920919in}}{\pgfqpoint{2.506213in}{2.913019in}}{\pgfqpoint{2.512037in}{2.907195in}}%
\pgfpathcurveto{\pgfqpoint{2.517861in}{2.901371in}}{\pgfqpoint{2.525761in}{2.898099in}}{\pgfqpoint{2.533998in}{2.898099in}}%
\pgfpathclose%
\pgfusepath{stroke,fill}%
\end{pgfscope}%
\begin{pgfscope}%
\pgfpathrectangle{\pgfqpoint{0.100000in}{0.212622in}}{\pgfqpoint{3.696000in}{3.696000in}}%
\pgfusepath{clip}%
\pgfsetbuttcap%
\pgfsetroundjoin%
\definecolor{currentfill}{rgb}{0.121569,0.466667,0.705882}%
\pgfsetfillcolor{currentfill}%
\pgfsetfillopacity{0.750639}%
\pgfsetlinewidth{1.003750pt}%
\definecolor{currentstroke}{rgb}{0.121569,0.466667,0.705882}%
\pgfsetstrokecolor{currentstroke}%
\pgfsetstrokeopacity{0.750639}%
\pgfsetdash{}{0pt}%
\pgfpathmoveto{\pgfqpoint{1.312930in}{2.565197in}}%
\pgfpathcurveto{\pgfqpoint{1.321166in}{2.565197in}}{\pgfqpoint{1.329066in}{2.568469in}}{\pgfqpoint{1.334890in}{2.574293in}}%
\pgfpathcurveto{\pgfqpoint{1.340714in}{2.580117in}}{\pgfqpoint{1.343986in}{2.588017in}}{\pgfqpoint{1.343986in}{2.596253in}}%
\pgfpathcurveto{\pgfqpoint{1.343986in}{2.604490in}}{\pgfqpoint{1.340714in}{2.612390in}}{\pgfqpoint{1.334890in}{2.618214in}}%
\pgfpathcurveto{\pgfqpoint{1.329066in}{2.624037in}}{\pgfqpoint{1.321166in}{2.627310in}}{\pgfqpoint{1.312930in}{2.627310in}}%
\pgfpathcurveto{\pgfqpoint{1.304693in}{2.627310in}}{\pgfqpoint{1.296793in}{2.624037in}}{\pgfqpoint{1.290969in}{2.618214in}}%
\pgfpathcurveto{\pgfqpoint{1.285146in}{2.612390in}}{\pgfqpoint{1.281873in}{2.604490in}}{\pgfqpoint{1.281873in}{2.596253in}}%
\pgfpathcurveto{\pgfqpoint{1.281873in}{2.588017in}}{\pgfqpoint{1.285146in}{2.580117in}}{\pgfqpoint{1.290969in}{2.574293in}}%
\pgfpathcurveto{\pgfqpoint{1.296793in}{2.568469in}}{\pgfqpoint{1.304693in}{2.565197in}}{\pgfqpoint{1.312930in}{2.565197in}}%
\pgfpathclose%
\pgfusepath{stroke,fill}%
\end{pgfscope}%
\begin{pgfscope}%
\pgfpathrectangle{\pgfqpoint{0.100000in}{0.212622in}}{\pgfqpoint{3.696000in}{3.696000in}}%
\pgfusepath{clip}%
\pgfsetbuttcap%
\pgfsetroundjoin%
\definecolor{currentfill}{rgb}{0.121569,0.466667,0.705882}%
\pgfsetfillcolor{currentfill}%
\pgfsetfillopacity{0.751081}%
\pgfsetlinewidth{1.003750pt}%
\definecolor{currentstroke}{rgb}{0.121569,0.466667,0.705882}%
\pgfsetstrokecolor{currentstroke}%
\pgfsetstrokeopacity{0.751081}%
\pgfsetdash{}{0pt}%
\pgfpathmoveto{\pgfqpoint{1.311678in}{2.563906in}}%
\pgfpathcurveto{\pgfqpoint{1.319915in}{2.563906in}}{\pgfqpoint{1.327815in}{2.567179in}}{\pgfqpoint{1.333639in}{2.573003in}}%
\pgfpathcurveto{\pgfqpoint{1.339463in}{2.578827in}}{\pgfqpoint{1.342735in}{2.586727in}}{\pgfqpoint{1.342735in}{2.594963in}}%
\pgfpathcurveto{\pgfqpoint{1.342735in}{2.603199in}}{\pgfqpoint{1.339463in}{2.611099in}}{\pgfqpoint{1.333639in}{2.616923in}}%
\pgfpathcurveto{\pgfqpoint{1.327815in}{2.622747in}}{\pgfqpoint{1.319915in}{2.626019in}}{\pgfqpoint{1.311678in}{2.626019in}}%
\pgfpathcurveto{\pgfqpoint{1.303442in}{2.626019in}}{\pgfqpoint{1.295542in}{2.622747in}}{\pgfqpoint{1.289718in}{2.616923in}}%
\pgfpathcurveto{\pgfqpoint{1.283894in}{2.611099in}}{\pgfqpoint{1.280622in}{2.603199in}}{\pgfqpoint{1.280622in}{2.594963in}}%
\pgfpathcurveto{\pgfqpoint{1.280622in}{2.586727in}}{\pgfqpoint{1.283894in}{2.578827in}}{\pgfqpoint{1.289718in}{2.573003in}}%
\pgfpathcurveto{\pgfqpoint{1.295542in}{2.567179in}}{\pgfqpoint{1.303442in}{2.563906in}}{\pgfqpoint{1.311678in}{2.563906in}}%
\pgfpathclose%
\pgfusepath{stroke,fill}%
\end{pgfscope}%
\begin{pgfscope}%
\pgfpathrectangle{\pgfqpoint{0.100000in}{0.212622in}}{\pgfqpoint{3.696000in}{3.696000in}}%
\pgfusepath{clip}%
\pgfsetbuttcap%
\pgfsetroundjoin%
\definecolor{currentfill}{rgb}{0.121569,0.466667,0.705882}%
\pgfsetfillcolor{currentfill}%
\pgfsetfillopacity{0.751524}%
\pgfsetlinewidth{1.003750pt}%
\definecolor{currentstroke}{rgb}{0.121569,0.466667,0.705882}%
\pgfsetstrokecolor{currentstroke}%
\pgfsetstrokeopacity{0.751524}%
\pgfsetdash{}{0pt}%
\pgfpathmoveto{\pgfqpoint{2.531951in}{2.894795in}}%
\pgfpathcurveto{\pgfqpoint{2.540187in}{2.894795in}}{\pgfqpoint{2.548087in}{2.898067in}}{\pgfqpoint{2.553911in}{2.903891in}}%
\pgfpathcurveto{\pgfqpoint{2.559735in}{2.909715in}}{\pgfqpoint{2.563007in}{2.917615in}}{\pgfqpoint{2.563007in}{2.925851in}}%
\pgfpathcurveto{\pgfqpoint{2.563007in}{2.934088in}}{\pgfqpoint{2.559735in}{2.941988in}}{\pgfqpoint{2.553911in}{2.947812in}}%
\pgfpathcurveto{\pgfqpoint{2.548087in}{2.953636in}}{\pgfqpoint{2.540187in}{2.956908in}}{\pgfqpoint{2.531951in}{2.956908in}}%
\pgfpathcurveto{\pgfqpoint{2.523714in}{2.956908in}}{\pgfqpoint{2.515814in}{2.953636in}}{\pgfqpoint{2.509990in}{2.947812in}}%
\pgfpathcurveto{\pgfqpoint{2.504166in}{2.941988in}}{\pgfqpoint{2.500894in}{2.934088in}}{\pgfqpoint{2.500894in}{2.925851in}}%
\pgfpathcurveto{\pgfqpoint{2.500894in}{2.917615in}}{\pgfqpoint{2.504166in}{2.909715in}}{\pgfqpoint{2.509990in}{2.903891in}}%
\pgfpathcurveto{\pgfqpoint{2.515814in}{2.898067in}}{\pgfqpoint{2.523714in}{2.894795in}}{\pgfqpoint{2.531951in}{2.894795in}}%
\pgfpathclose%
\pgfusepath{stroke,fill}%
\end{pgfscope}%
\begin{pgfscope}%
\pgfpathrectangle{\pgfqpoint{0.100000in}{0.212622in}}{\pgfqpoint{3.696000in}{3.696000in}}%
\pgfusepath{clip}%
\pgfsetbuttcap%
\pgfsetroundjoin%
\definecolor{currentfill}{rgb}{0.121569,0.466667,0.705882}%
\pgfsetfillcolor{currentfill}%
\pgfsetfillopacity{0.751588}%
\pgfsetlinewidth{1.003750pt}%
\definecolor{currentstroke}{rgb}{0.121569,0.466667,0.705882}%
\pgfsetstrokecolor{currentstroke}%
\pgfsetstrokeopacity{0.751588}%
\pgfsetdash{}{0pt}%
\pgfpathmoveto{\pgfqpoint{1.310143in}{2.562241in}}%
\pgfpathcurveto{\pgfqpoint{1.318379in}{2.562241in}}{\pgfqpoint{1.326279in}{2.565514in}}{\pgfqpoint{1.332103in}{2.571338in}}%
\pgfpathcurveto{\pgfqpoint{1.337927in}{2.577162in}}{\pgfqpoint{1.341199in}{2.585062in}}{\pgfqpoint{1.341199in}{2.593298in}}%
\pgfpathcurveto{\pgfqpoint{1.341199in}{2.601534in}}{\pgfqpoint{1.337927in}{2.609434in}}{\pgfqpoint{1.332103in}{2.615258in}}%
\pgfpathcurveto{\pgfqpoint{1.326279in}{2.621082in}}{\pgfqpoint{1.318379in}{2.624354in}}{\pgfqpoint{1.310143in}{2.624354in}}%
\pgfpathcurveto{\pgfqpoint{1.301906in}{2.624354in}}{\pgfqpoint{1.294006in}{2.621082in}}{\pgfqpoint{1.288182in}{2.615258in}}%
\pgfpathcurveto{\pgfqpoint{1.282358in}{2.609434in}}{\pgfqpoint{1.279086in}{2.601534in}}{\pgfqpoint{1.279086in}{2.593298in}}%
\pgfpathcurveto{\pgfqpoint{1.279086in}{2.585062in}}{\pgfqpoint{1.282358in}{2.577162in}}{\pgfqpoint{1.288182in}{2.571338in}}%
\pgfpathcurveto{\pgfqpoint{1.294006in}{2.565514in}}{\pgfqpoint{1.301906in}{2.562241in}}{\pgfqpoint{1.310143in}{2.562241in}}%
\pgfpathclose%
\pgfusepath{stroke,fill}%
\end{pgfscope}%
\begin{pgfscope}%
\pgfpathrectangle{\pgfqpoint{0.100000in}{0.212622in}}{\pgfqpoint{3.696000in}{3.696000in}}%
\pgfusepath{clip}%
\pgfsetbuttcap%
\pgfsetroundjoin%
\definecolor{currentfill}{rgb}{0.121569,0.466667,0.705882}%
\pgfsetfillcolor{currentfill}%
\pgfsetfillopacity{0.752351}%
\pgfsetlinewidth{1.003750pt}%
\definecolor{currentstroke}{rgb}{0.121569,0.466667,0.705882}%
\pgfsetstrokecolor{currentstroke}%
\pgfsetstrokeopacity{0.752351}%
\pgfsetdash{}{0pt}%
\pgfpathmoveto{\pgfqpoint{1.307620in}{2.559186in}}%
\pgfpathcurveto{\pgfqpoint{1.315856in}{2.559186in}}{\pgfqpoint{1.323757in}{2.562458in}}{\pgfqpoint{1.329580in}{2.568282in}}%
\pgfpathcurveto{\pgfqpoint{1.335404in}{2.574106in}}{\pgfqpoint{1.338677in}{2.582006in}}{\pgfqpoint{1.338677in}{2.590242in}}%
\pgfpathcurveto{\pgfqpoint{1.338677in}{2.598479in}}{\pgfqpoint{1.335404in}{2.606379in}}{\pgfqpoint{1.329580in}{2.612203in}}%
\pgfpathcurveto{\pgfqpoint{1.323757in}{2.618027in}}{\pgfqpoint{1.315856in}{2.621299in}}{\pgfqpoint{1.307620in}{2.621299in}}%
\pgfpathcurveto{\pgfqpoint{1.299384in}{2.621299in}}{\pgfqpoint{1.291484in}{2.618027in}}{\pgfqpoint{1.285660in}{2.612203in}}%
\pgfpathcurveto{\pgfqpoint{1.279836in}{2.606379in}}{\pgfqpoint{1.276564in}{2.598479in}}{\pgfqpoint{1.276564in}{2.590242in}}%
\pgfpathcurveto{\pgfqpoint{1.276564in}{2.582006in}}{\pgfqpoint{1.279836in}{2.574106in}}{\pgfqpoint{1.285660in}{2.568282in}}%
\pgfpathcurveto{\pgfqpoint{1.291484in}{2.562458in}}{\pgfqpoint{1.299384in}{2.559186in}}{\pgfqpoint{1.307620in}{2.559186in}}%
\pgfpathclose%
\pgfusepath{stroke,fill}%
\end{pgfscope}%
\begin{pgfscope}%
\pgfpathrectangle{\pgfqpoint{0.100000in}{0.212622in}}{\pgfqpoint{3.696000in}{3.696000in}}%
\pgfusepath{clip}%
\pgfsetbuttcap%
\pgfsetroundjoin%
\definecolor{currentfill}{rgb}{0.121569,0.466667,0.705882}%
\pgfsetfillcolor{currentfill}%
\pgfsetfillopacity{0.752356}%
\pgfsetlinewidth{1.003750pt}%
\definecolor{currentstroke}{rgb}{0.121569,0.466667,0.705882}%
\pgfsetstrokecolor{currentstroke}%
\pgfsetstrokeopacity{0.752356}%
\pgfsetdash{}{0pt}%
\pgfpathmoveto{\pgfqpoint{2.530552in}{2.892498in}}%
\pgfpathcurveto{\pgfqpoint{2.538788in}{2.892498in}}{\pgfqpoint{2.546688in}{2.895770in}}{\pgfqpoint{2.552512in}{2.901594in}}%
\pgfpathcurveto{\pgfqpoint{2.558336in}{2.907418in}}{\pgfqpoint{2.561608in}{2.915318in}}{\pgfqpoint{2.561608in}{2.923554in}}%
\pgfpathcurveto{\pgfqpoint{2.561608in}{2.931791in}}{\pgfqpoint{2.558336in}{2.939691in}}{\pgfqpoint{2.552512in}{2.945515in}}%
\pgfpathcurveto{\pgfqpoint{2.546688in}{2.951339in}}{\pgfqpoint{2.538788in}{2.954611in}}{\pgfqpoint{2.530552in}{2.954611in}}%
\pgfpathcurveto{\pgfqpoint{2.522316in}{2.954611in}}{\pgfqpoint{2.514416in}{2.951339in}}{\pgfqpoint{2.508592in}{2.945515in}}%
\pgfpathcurveto{\pgfqpoint{2.502768in}{2.939691in}}{\pgfqpoint{2.499495in}{2.931791in}}{\pgfqpoint{2.499495in}{2.923554in}}%
\pgfpathcurveto{\pgfqpoint{2.499495in}{2.915318in}}{\pgfqpoint{2.502768in}{2.907418in}}{\pgfqpoint{2.508592in}{2.901594in}}%
\pgfpathcurveto{\pgfqpoint{2.514416in}{2.895770in}}{\pgfqpoint{2.522316in}{2.892498in}}{\pgfqpoint{2.530552in}{2.892498in}}%
\pgfpathclose%
\pgfusepath{stroke,fill}%
\end{pgfscope}%
\begin{pgfscope}%
\pgfpathrectangle{\pgfqpoint{0.100000in}{0.212622in}}{\pgfqpoint{3.696000in}{3.696000in}}%
\pgfusepath{clip}%
\pgfsetbuttcap%
\pgfsetroundjoin%
\definecolor{currentfill}{rgb}{0.121569,0.466667,0.705882}%
\pgfsetfillcolor{currentfill}%
\pgfsetfillopacity{0.752780}%
\pgfsetlinewidth{1.003750pt}%
\definecolor{currentstroke}{rgb}{0.121569,0.466667,0.705882}%
\pgfsetstrokecolor{currentstroke}%
\pgfsetstrokeopacity{0.752780}%
\pgfsetdash{}{0pt}%
\pgfpathmoveto{\pgfqpoint{1.306281in}{2.557498in}}%
\pgfpathcurveto{\pgfqpoint{1.314518in}{2.557498in}}{\pgfqpoint{1.322418in}{2.560771in}}{\pgfqpoint{1.328242in}{2.566595in}}%
\pgfpathcurveto{\pgfqpoint{1.334066in}{2.572419in}}{\pgfqpoint{1.337338in}{2.580319in}}{\pgfqpoint{1.337338in}{2.588555in}}%
\pgfpathcurveto{\pgfqpoint{1.337338in}{2.596791in}}{\pgfqpoint{1.334066in}{2.604691in}}{\pgfqpoint{1.328242in}{2.610515in}}%
\pgfpathcurveto{\pgfqpoint{1.322418in}{2.616339in}}{\pgfqpoint{1.314518in}{2.619611in}}{\pgfqpoint{1.306281in}{2.619611in}}%
\pgfpathcurveto{\pgfqpoint{1.298045in}{2.619611in}}{\pgfqpoint{1.290145in}{2.616339in}}{\pgfqpoint{1.284321in}{2.610515in}}%
\pgfpathcurveto{\pgfqpoint{1.278497in}{2.604691in}}{\pgfqpoint{1.275225in}{2.596791in}}{\pgfqpoint{1.275225in}{2.588555in}}%
\pgfpathcurveto{\pgfqpoint{1.275225in}{2.580319in}}{\pgfqpoint{1.278497in}{2.572419in}}{\pgfqpoint{1.284321in}{2.566595in}}%
\pgfpathcurveto{\pgfqpoint{1.290145in}{2.560771in}}{\pgfqpoint{1.298045in}{2.557498in}}{\pgfqpoint{1.306281in}{2.557498in}}%
\pgfpathclose%
\pgfusepath{stroke,fill}%
\end{pgfscope}%
\begin{pgfscope}%
\pgfpathrectangle{\pgfqpoint{0.100000in}{0.212622in}}{\pgfqpoint{3.696000in}{3.696000in}}%
\pgfusepath{clip}%
\pgfsetbuttcap%
\pgfsetroundjoin%
\definecolor{currentfill}{rgb}{0.121569,0.466667,0.705882}%
\pgfsetfillcolor{currentfill}%
\pgfsetfillopacity{0.753076}%
\pgfsetlinewidth{1.003750pt}%
\definecolor{currentstroke}{rgb}{0.121569,0.466667,0.705882}%
\pgfsetstrokecolor{currentstroke}%
\pgfsetstrokeopacity{0.753076}%
\pgfsetdash{}{0pt}%
\pgfpathmoveto{\pgfqpoint{2.960566in}{1.644921in}}%
\pgfpathcurveto{\pgfqpoint{2.968802in}{1.644921in}}{\pgfqpoint{2.976702in}{1.648193in}}{\pgfqpoint{2.982526in}{1.654017in}}%
\pgfpathcurveto{\pgfqpoint{2.988350in}{1.659841in}}{\pgfqpoint{2.991622in}{1.667741in}}{\pgfqpoint{2.991622in}{1.675978in}}%
\pgfpathcurveto{\pgfqpoint{2.991622in}{1.684214in}}{\pgfqpoint{2.988350in}{1.692114in}}{\pgfqpoint{2.982526in}{1.697938in}}%
\pgfpathcurveto{\pgfqpoint{2.976702in}{1.703762in}}{\pgfqpoint{2.968802in}{1.707034in}}{\pgfqpoint{2.960566in}{1.707034in}}%
\pgfpathcurveto{\pgfqpoint{2.952329in}{1.707034in}}{\pgfqpoint{2.944429in}{1.703762in}}{\pgfqpoint{2.938605in}{1.697938in}}%
\pgfpathcurveto{\pgfqpoint{2.932782in}{1.692114in}}{\pgfqpoint{2.929509in}{1.684214in}}{\pgfqpoint{2.929509in}{1.675978in}}%
\pgfpathcurveto{\pgfqpoint{2.929509in}{1.667741in}}{\pgfqpoint{2.932782in}{1.659841in}}{\pgfqpoint{2.938605in}{1.654017in}}%
\pgfpathcurveto{\pgfqpoint{2.944429in}{1.648193in}}{\pgfqpoint{2.952329in}{1.644921in}}{\pgfqpoint{2.960566in}{1.644921in}}%
\pgfpathclose%
\pgfusepath{stroke,fill}%
\end{pgfscope}%
\begin{pgfscope}%
\pgfpathrectangle{\pgfqpoint{0.100000in}{0.212622in}}{\pgfqpoint{3.696000in}{3.696000in}}%
\pgfusepath{clip}%
\pgfsetbuttcap%
\pgfsetroundjoin%
\definecolor{currentfill}{rgb}{0.121569,0.466667,0.705882}%
\pgfsetfillcolor{currentfill}%
\pgfsetfillopacity{0.753534}%
\pgfsetlinewidth{1.003750pt}%
\definecolor{currentstroke}{rgb}{0.121569,0.466667,0.705882}%
\pgfsetstrokecolor{currentstroke}%
\pgfsetstrokeopacity{0.753534}%
\pgfsetdash{}{0pt}%
\pgfpathmoveto{\pgfqpoint{1.304094in}{2.554914in}}%
\pgfpathcurveto{\pgfqpoint{1.312330in}{2.554914in}}{\pgfqpoint{1.320230in}{2.558187in}}{\pgfqpoint{1.326054in}{2.564011in}}%
\pgfpathcurveto{\pgfqpoint{1.331878in}{2.569835in}}{\pgfqpoint{1.335151in}{2.577735in}}{\pgfqpoint{1.335151in}{2.585971in}}%
\pgfpathcurveto{\pgfqpoint{1.335151in}{2.594207in}}{\pgfqpoint{1.331878in}{2.602107in}}{\pgfqpoint{1.326054in}{2.607931in}}%
\pgfpathcurveto{\pgfqpoint{1.320230in}{2.613755in}}{\pgfqpoint{1.312330in}{2.617027in}}{\pgfqpoint{1.304094in}{2.617027in}}%
\pgfpathcurveto{\pgfqpoint{1.295858in}{2.617027in}}{\pgfqpoint{1.287958in}{2.613755in}}{\pgfqpoint{1.282134in}{2.607931in}}%
\pgfpathcurveto{\pgfqpoint{1.276310in}{2.602107in}}{\pgfqpoint{1.273038in}{2.594207in}}{\pgfqpoint{1.273038in}{2.585971in}}%
\pgfpathcurveto{\pgfqpoint{1.273038in}{2.577735in}}{\pgfqpoint{1.276310in}{2.569835in}}{\pgfqpoint{1.282134in}{2.564011in}}%
\pgfpathcurveto{\pgfqpoint{1.287958in}{2.558187in}}{\pgfqpoint{1.295858in}{2.554914in}}{\pgfqpoint{1.304094in}{2.554914in}}%
\pgfpathclose%
\pgfusepath{stroke,fill}%
\end{pgfscope}%
\begin{pgfscope}%
\pgfpathrectangle{\pgfqpoint{0.100000in}{0.212622in}}{\pgfqpoint{3.696000in}{3.696000in}}%
\pgfusepath{clip}%
\pgfsetbuttcap%
\pgfsetroundjoin%
\definecolor{currentfill}{rgb}{0.121569,0.466667,0.705882}%
\pgfsetfillcolor{currentfill}%
\pgfsetfillopacity{0.753859}%
\pgfsetlinewidth{1.003750pt}%
\definecolor{currentstroke}{rgb}{0.121569,0.466667,0.705882}%
\pgfsetstrokecolor{currentstroke}%
\pgfsetstrokeopacity{0.753859}%
\pgfsetdash{}{0pt}%
\pgfpathmoveto{\pgfqpoint{2.527874in}{2.888324in}}%
\pgfpathcurveto{\pgfqpoint{2.536110in}{2.888324in}}{\pgfqpoint{2.544010in}{2.891596in}}{\pgfqpoint{2.549834in}{2.897420in}}%
\pgfpathcurveto{\pgfqpoint{2.555658in}{2.903244in}}{\pgfqpoint{2.558930in}{2.911144in}}{\pgfqpoint{2.558930in}{2.919380in}}%
\pgfpathcurveto{\pgfqpoint{2.558930in}{2.927616in}}{\pgfqpoint{2.555658in}{2.935516in}}{\pgfqpoint{2.549834in}{2.941340in}}%
\pgfpathcurveto{\pgfqpoint{2.544010in}{2.947164in}}{\pgfqpoint{2.536110in}{2.950437in}}{\pgfqpoint{2.527874in}{2.950437in}}%
\pgfpathcurveto{\pgfqpoint{2.519637in}{2.950437in}}{\pgfqpoint{2.511737in}{2.947164in}}{\pgfqpoint{2.505913in}{2.941340in}}%
\pgfpathcurveto{\pgfqpoint{2.500089in}{2.935516in}}{\pgfqpoint{2.496817in}{2.927616in}}{\pgfqpoint{2.496817in}{2.919380in}}%
\pgfpathcurveto{\pgfqpoint{2.496817in}{2.911144in}}{\pgfqpoint{2.500089in}{2.903244in}}{\pgfqpoint{2.505913in}{2.897420in}}%
\pgfpathcurveto{\pgfqpoint{2.511737in}{2.891596in}}{\pgfqpoint{2.519637in}{2.888324in}}{\pgfqpoint{2.527874in}{2.888324in}}%
\pgfpathclose%
\pgfusepath{stroke,fill}%
\end{pgfscope}%
\begin{pgfscope}%
\pgfpathrectangle{\pgfqpoint{0.100000in}{0.212622in}}{\pgfqpoint{3.696000in}{3.696000in}}%
\pgfusepath{clip}%
\pgfsetbuttcap%
\pgfsetroundjoin%
\definecolor{currentfill}{rgb}{0.121569,0.466667,0.705882}%
\pgfsetfillcolor{currentfill}%
\pgfsetfillopacity{0.753987}%
\pgfsetlinewidth{1.003750pt}%
\definecolor{currentstroke}{rgb}{0.121569,0.466667,0.705882}%
\pgfsetstrokecolor{currentstroke}%
\pgfsetstrokeopacity{0.753987}%
\pgfsetdash{}{0pt}%
\pgfpathmoveto{\pgfqpoint{1.302899in}{2.553708in}}%
\pgfpathcurveto{\pgfqpoint{1.311136in}{2.553708in}}{\pgfqpoint{1.319036in}{2.556980in}}{\pgfqpoint{1.324860in}{2.562804in}}%
\pgfpathcurveto{\pgfqpoint{1.330684in}{2.568628in}}{\pgfqpoint{1.333956in}{2.576528in}}{\pgfqpoint{1.333956in}{2.584764in}}%
\pgfpathcurveto{\pgfqpoint{1.333956in}{2.593000in}}{\pgfqpoint{1.330684in}{2.600901in}}{\pgfqpoint{1.324860in}{2.606724in}}%
\pgfpathcurveto{\pgfqpoint{1.319036in}{2.612548in}}{\pgfqpoint{1.311136in}{2.615821in}}{\pgfqpoint{1.302899in}{2.615821in}}%
\pgfpathcurveto{\pgfqpoint{1.294663in}{2.615821in}}{\pgfqpoint{1.286763in}{2.612548in}}{\pgfqpoint{1.280939in}{2.606724in}}%
\pgfpathcurveto{\pgfqpoint{1.275115in}{2.600901in}}{\pgfqpoint{1.271843in}{2.593000in}}{\pgfqpoint{1.271843in}{2.584764in}}%
\pgfpathcurveto{\pgfqpoint{1.271843in}{2.576528in}}{\pgfqpoint{1.275115in}{2.568628in}}{\pgfqpoint{1.280939in}{2.562804in}}%
\pgfpathcurveto{\pgfqpoint{1.286763in}{2.556980in}}{\pgfqpoint{1.294663in}{2.553708in}}{\pgfqpoint{1.302899in}{2.553708in}}%
\pgfpathclose%
\pgfusepath{stroke,fill}%
\end{pgfscope}%
\begin{pgfscope}%
\pgfpathrectangle{\pgfqpoint{0.100000in}{0.212622in}}{\pgfqpoint{3.696000in}{3.696000in}}%
\pgfusepath{clip}%
\pgfsetbuttcap%
\pgfsetroundjoin%
\definecolor{currentfill}{rgb}{0.121569,0.466667,0.705882}%
\pgfsetfillcolor{currentfill}%
\pgfsetfillopacity{0.754553}%
\pgfsetlinewidth{1.003750pt}%
\definecolor{currentstroke}{rgb}{0.121569,0.466667,0.705882}%
\pgfsetstrokecolor{currentstroke}%
\pgfsetstrokeopacity{0.754553}%
\pgfsetdash{}{0pt}%
\pgfpathmoveto{\pgfqpoint{1.301296in}{2.551997in}}%
\pgfpathcurveto{\pgfqpoint{1.309532in}{2.551997in}}{\pgfqpoint{1.317432in}{2.555270in}}{\pgfqpoint{1.323256in}{2.561094in}}%
\pgfpathcurveto{\pgfqpoint{1.329080in}{2.566918in}}{\pgfqpoint{1.332353in}{2.574818in}}{\pgfqpoint{1.332353in}{2.583054in}}%
\pgfpathcurveto{\pgfqpoint{1.332353in}{2.591290in}}{\pgfqpoint{1.329080in}{2.599190in}}{\pgfqpoint{1.323256in}{2.605014in}}%
\pgfpathcurveto{\pgfqpoint{1.317432in}{2.610838in}}{\pgfqpoint{1.309532in}{2.614110in}}{\pgfqpoint{1.301296in}{2.614110in}}%
\pgfpathcurveto{\pgfqpoint{1.293060in}{2.614110in}}{\pgfqpoint{1.285160in}{2.610838in}}{\pgfqpoint{1.279336in}{2.605014in}}%
\pgfpathcurveto{\pgfqpoint{1.273512in}{2.599190in}}{\pgfqpoint{1.270240in}{2.591290in}}{\pgfqpoint{1.270240in}{2.583054in}}%
\pgfpathcurveto{\pgfqpoint{1.270240in}{2.574818in}}{\pgfqpoint{1.273512in}{2.566918in}}{\pgfqpoint{1.279336in}{2.561094in}}%
\pgfpathcurveto{\pgfqpoint{1.285160in}{2.555270in}}{\pgfqpoint{1.293060in}{2.551997in}}{\pgfqpoint{1.301296in}{2.551997in}}%
\pgfpathclose%
\pgfusepath{stroke,fill}%
\end{pgfscope}%
\begin{pgfscope}%
\pgfpathrectangle{\pgfqpoint{0.100000in}{0.212622in}}{\pgfqpoint{3.696000in}{3.696000in}}%
\pgfusepath{clip}%
\pgfsetbuttcap%
\pgfsetroundjoin%
\definecolor{currentfill}{rgb}{0.121569,0.466667,0.705882}%
\pgfsetfillcolor{currentfill}%
\pgfsetfillopacity{0.754778}%
\pgfsetlinewidth{1.003750pt}%
\definecolor{currentstroke}{rgb}{0.121569,0.466667,0.705882}%
\pgfsetstrokecolor{currentstroke}%
\pgfsetstrokeopacity{0.754778}%
\pgfsetdash{}{0pt}%
\pgfpathmoveto{\pgfqpoint{2.526117in}{2.885585in}}%
\pgfpathcurveto{\pgfqpoint{2.534353in}{2.885585in}}{\pgfqpoint{2.542253in}{2.888857in}}{\pgfqpoint{2.548077in}{2.894681in}}%
\pgfpathcurveto{\pgfqpoint{2.553901in}{2.900505in}}{\pgfqpoint{2.557174in}{2.908405in}}{\pgfqpoint{2.557174in}{2.916641in}}%
\pgfpathcurveto{\pgfqpoint{2.557174in}{2.924878in}}{\pgfqpoint{2.553901in}{2.932778in}}{\pgfqpoint{2.548077in}{2.938602in}}%
\pgfpathcurveto{\pgfqpoint{2.542253in}{2.944426in}}{\pgfqpoint{2.534353in}{2.947698in}}{\pgfqpoint{2.526117in}{2.947698in}}%
\pgfpathcurveto{\pgfqpoint{2.517881in}{2.947698in}}{\pgfqpoint{2.509981in}{2.944426in}}{\pgfqpoint{2.504157in}{2.938602in}}%
\pgfpathcurveto{\pgfqpoint{2.498333in}{2.932778in}}{\pgfqpoint{2.495061in}{2.924878in}}{\pgfqpoint{2.495061in}{2.916641in}}%
\pgfpathcurveto{\pgfqpoint{2.495061in}{2.908405in}}{\pgfqpoint{2.498333in}{2.900505in}}{\pgfqpoint{2.504157in}{2.894681in}}%
\pgfpathcurveto{\pgfqpoint{2.509981in}{2.888857in}}{\pgfqpoint{2.517881in}{2.885585in}}{\pgfqpoint{2.526117in}{2.885585in}}%
\pgfpathclose%
\pgfusepath{stroke,fill}%
\end{pgfscope}%
\begin{pgfscope}%
\pgfpathrectangle{\pgfqpoint{0.100000in}{0.212622in}}{\pgfqpoint{3.696000in}{3.696000in}}%
\pgfusepath{clip}%
\pgfsetbuttcap%
\pgfsetroundjoin%
\definecolor{currentfill}{rgb}{0.121569,0.466667,0.705882}%
\pgfsetfillcolor{currentfill}%
\pgfsetfillopacity{0.755367}%
\pgfsetlinewidth{1.003750pt}%
\definecolor{currentstroke}{rgb}{0.121569,0.466667,0.705882}%
\pgfsetstrokecolor{currentstroke}%
\pgfsetstrokeopacity{0.755367}%
\pgfsetdash{}{0pt}%
\pgfpathmoveto{\pgfqpoint{1.298794in}{2.549117in}}%
\pgfpathcurveto{\pgfqpoint{1.307031in}{2.549117in}}{\pgfqpoint{1.314931in}{2.552389in}}{\pgfqpoint{1.320755in}{2.558213in}}%
\pgfpathcurveto{\pgfqpoint{1.326579in}{2.564037in}}{\pgfqpoint{1.329851in}{2.571937in}}{\pgfqpoint{1.329851in}{2.580173in}}%
\pgfpathcurveto{\pgfqpoint{1.329851in}{2.588410in}}{\pgfqpoint{1.326579in}{2.596310in}}{\pgfqpoint{1.320755in}{2.602134in}}%
\pgfpathcurveto{\pgfqpoint{1.314931in}{2.607958in}}{\pgfqpoint{1.307031in}{2.611230in}}{\pgfqpoint{1.298794in}{2.611230in}}%
\pgfpathcurveto{\pgfqpoint{1.290558in}{2.611230in}}{\pgfqpoint{1.282658in}{2.607958in}}{\pgfqpoint{1.276834in}{2.602134in}}%
\pgfpathcurveto{\pgfqpoint{1.271010in}{2.596310in}}{\pgfqpoint{1.267738in}{2.588410in}}{\pgfqpoint{1.267738in}{2.580173in}}%
\pgfpathcurveto{\pgfqpoint{1.267738in}{2.571937in}}{\pgfqpoint{1.271010in}{2.564037in}}{\pgfqpoint{1.276834in}{2.558213in}}%
\pgfpathcurveto{\pgfqpoint{1.282658in}{2.552389in}}{\pgfqpoint{1.290558in}{2.549117in}}{\pgfqpoint{1.298794in}{2.549117in}}%
\pgfpathclose%
\pgfusepath{stroke,fill}%
\end{pgfscope}%
\begin{pgfscope}%
\pgfpathrectangle{\pgfqpoint{0.100000in}{0.212622in}}{\pgfqpoint{3.696000in}{3.696000in}}%
\pgfusepath{clip}%
\pgfsetbuttcap%
\pgfsetroundjoin%
\definecolor{currentfill}{rgb}{0.121569,0.466667,0.705882}%
\pgfsetfillcolor{currentfill}%
\pgfsetfillopacity{0.755774}%
\pgfsetlinewidth{1.003750pt}%
\definecolor{currentstroke}{rgb}{0.121569,0.466667,0.705882}%
\pgfsetstrokecolor{currentstroke}%
\pgfsetstrokeopacity{0.755774}%
\pgfsetdash{}{0pt}%
\pgfpathmoveto{\pgfqpoint{1.297417in}{2.547297in}}%
\pgfpathcurveto{\pgfqpoint{1.305653in}{2.547297in}}{\pgfqpoint{1.313553in}{2.550570in}}{\pgfqpoint{1.319377in}{2.556394in}}%
\pgfpathcurveto{\pgfqpoint{1.325201in}{2.562218in}}{\pgfqpoint{1.328473in}{2.570118in}}{\pgfqpoint{1.328473in}{2.578354in}}%
\pgfpathcurveto{\pgfqpoint{1.328473in}{2.586590in}}{\pgfqpoint{1.325201in}{2.594490in}}{\pgfqpoint{1.319377in}{2.600314in}}%
\pgfpathcurveto{\pgfqpoint{1.313553in}{2.606138in}}{\pgfqpoint{1.305653in}{2.609410in}}{\pgfqpoint{1.297417in}{2.609410in}}%
\pgfpathcurveto{\pgfqpoint{1.289181in}{2.609410in}}{\pgfqpoint{1.281281in}{2.606138in}}{\pgfqpoint{1.275457in}{2.600314in}}%
\pgfpathcurveto{\pgfqpoint{1.269633in}{2.594490in}}{\pgfqpoint{1.266360in}{2.586590in}}{\pgfqpoint{1.266360in}{2.578354in}}%
\pgfpathcurveto{\pgfqpoint{1.266360in}{2.570118in}}{\pgfqpoint{1.269633in}{2.562218in}}{\pgfqpoint{1.275457in}{2.556394in}}%
\pgfpathcurveto{\pgfqpoint{1.281281in}{2.550570in}}{\pgfqpoint{1.289181in}{2.547297in}}{\pgfqpoint{1.297417in}{2.547297in}}%
\pgfpathclose%
\pgfusepath{stroke,fill}%
\end{pgfscope}%
\begin{pgfscope}%
\pgfpathrectangle{\pgfqpoint{0.100000in}{0.212622in}}{\pgfqpoint{3.696000in}{3.696000in}}%
\pgfusepath{clip}%
\pgfsetbuttcap%
\pgfsetroundjoin%
\definecolor{currentfill}{rgb}{0.121569,0.466667,0.705882}%
\pgfsetfillcolor{currentfill}%
\pgfsetfillopacity{0.756421}%
\pgfsetlinewidth{1.003750pt}%
\definecolor{currentstroke}{rgb}{0.121569,0.466667,0.705882}%
\pgfsetstrokecolor{currentstroke}%
\pgfsetstrokeopacity{0.756421}%
\pgfsetdash{}{0pt}%
\pgfpathmoveto{\pgfqpoint{1.295360in}{2.544720in}}%
\pgfpathcurveto{\pgfqpoint{1.303597in}{2.544720in}}{\pgfqpoint{1.311497in}{2.547992in}}{\pgfqpoint{1.317321in}{2.553816in}}%
\pgfpathcurveto{\pgfqpoint{1.323144in}{2.559640in}}{\pgfqpoint{1.326417in}{2.567540in}}{\pgfqpoint{1.326417in}{2.575777in}}%
\pgfpathcurveto{\pgfqpoint{1.326417in}{2.584013in}}{\pgfqpoint{1.323144in}{2.591913in}}{\pgfqpoint{1.317321in}{2.597737in}}%
\pgfpathcurveto{\pgfqpoint{1.311497in}{2.603561in}}{\pgfqpoint{1.303597in}{2.606833in}}{\pgfqpoint{1.295360in}{2.606833in}}%
\pgfpathcurveto{\pgfqpoint{1.287124in}{2.606833in}}{\pgfqpoint{1.279224in}{2.603561in}}{\pgfqpoint{1.273400in}{2.597737in}}%
\pgfpathcurveto{\pgfqpoint{1.267576in}{2.591913in}}{\pgfqpoint{1.264304in}{2.584013in}}{\pgfqpoint{1.264304in}{2.575777in}}%
\pgfpathcurveto{\pgfqpoint{1.264304in}{2.567540in}}{\pgfqpoint{1.267576in}{2.559640in}}{\pgfqpoint{1.273400in}{2.553816in}}%
\pgfpathcurveto{\pgfqpoint{1.279224in}{2.547992in}}{\pgfqpoint{1.287124in}{2.544720in}}{\pgfqpoint{1.295360in}{2.544720in}}%
\pgfpathclose%
\pgfusepath{stroke,fill}%
\end{pgfscope}%
\begin{pgfscope}%
\pgfpathrectangle{\pgfqpoint{0.100000in}{0.212622in}}{\pgfqpoint{3.696000in}{3.696000in}}%
\pgfusepath{clip}%
\pgfsetbuttcap%
\pgfsetroundjoin%
\definecolor{currentfill}{rgb}{0.121569,0.466667,0.705882}%
\pgfsetfillcolor{currentfill}%
\pgfsetfillopacity{0.756465}%
\pgfsetlinewidth{1.003750pt}%
\definecolor{currentstroke}{rgb}{0.121569,0.466667,0.705882}%
\pgfsetstrokecolor{currentstroke}%
\pgfsetstrokeopacity{0.756465}%
\pgfsetdash{}{0pt}%
\pgfpathmoveto{\pgfqpoint{2.522950in}{2.880676in}}%
\pgfpathcurveto{\pgfqpoint{2.531187in}{2.880676in}}{\pgfqpoint{2.539087in}{2.883948in}}{\pgfqpoint{2.544911in}{2.889772in}}%
\pgfpathcurveto{\pgfqpoint{2.550735in}{2.895596in}}{\pgfqpoint{2.554007in}{2.903496in}}{\pgfqpoint{2.554007in}{2.911732in}}%
\pgfpathcurveto{\pgfqpoint{2.554007in}{2.919969in}}{\pgfqpoint{2.550735in}{2.927869in}}{\pgfqpoint{2.544911in}{2.933693in}}%
\pgfpathcurveto{\pgfqpoint{2.539087in}{2.939517in}}{\pgfqpoint{2.531187in}{2.942789in}}{\pgfqpoint{2.522950in}{2.942789in}}%
\pgfpathcurveto{\pgfqpoint{2.514714in}{2.942789in}}{\pgfqpoint{2.506814in}{2.939517in}}{\pgfqpoint{2.500990in}{2.933693in}}%
\pgfpathcurveto{\pgfqpoint{2.495166in}{2.927869in}}{\pgfqpoint{2.491894in}{2.919969in}}{\pgfqpoint{2.491894in}{2.911732in}}%
\pgfpathcurveto{\pgfqpoint{2.491894in}{2.903496in}}{\pgfqpoint{2.495166in}{2.895596in}}{\pgfqpoint{2.500990in}{2.889772in}}%
\pgfpathcurveto{\pgfqpoint{2.506814in}{2.883948in}}{\pgfqpoint{2.514714in}{2.880676in}}{\pgfqpoint{2.522950in}{2.880676in}}%
\pgfpathclose%
\pgfusepath{stroke,fill}%
\end{pgfscope}%
\begin{pgfscope}%
\pgfpathrectangle{\pgfqpoint{0.100000in}{0.212622in}}{\pgfqpoint{3.696000in}{3.696000in}}%
\pgfusepath{clip}%
\pgfsetbuttcap%
\pgfsetroundjoin%
\definecolor{currentfill}{rgb}{0.121569,0.466667,0.705882}%
\pgfsetfillcolor{currentfill}%
\pgfsetfillopacity{0.756911}%
\pgfsetlinewidth{1.003750pt}%
\definecolor{currentstroke}{rgb}{0.121569,0.466667,0.705882}%
\pgfsetstrokecolor{currentstroke}%
\pgfsetstrokeopacity{0.756911}%
\pgfsetdash{}{0pt}%
\pgfpathmoveto{\pgfqpoint{2.952965in}{1.634931in}}%
\pgfpathcurveto{\pgfqpoint{2.961201in}{1.634931in}}{\pgfqpoint{2.969101in}{1.638203in}}{\pgfqpoint{2.974925in}{1.644027in}}%
\pgfpathcurveto{\pgfqpoint{2.980749in}{1.649851in}}{\pgfqpoint{2.984021in}{1.657751in}}{\pgfqpoint{2.984021in}{1.665987in}}%
\pgfpathcurveto{\pgfqpoint{2.984021in}{1.674223in}}{\pgfqpoint{2.980749in}{1.682123in}}{\pgfqpoint{2.974925in}{1.687947in}}%
\pgfpathcurveto{\pgfqpoint{2.969101in}{1.693771in}}{\pgfqpoint{2.961201in}{1.697044in}}{\pgfqpoint{2.952965in}{1.697044in}}%
\pgfpathcurveto{\pgfqpoint{2.944729in}{1.697044in}}{\pgfqpoint{2.936829in}{1.693771in}}{\pgfqpoint{2.931005in}{1.687947in}}%
\pgfpathcurveto{\pgfqpoint{2.925181in}{1.682123in}}{\pgfqpoint{2.921908in}{1.674223in}}{\pgfqpoint{2.921908in}{1.665987in}}%
\pgfpathcurveto{\pgfqpoint{2.921908in}{1.657751in}}{\pgfqpoint{2.925181in}{1.649851in}}{\pgfqpoint{2.931005in}{1.644027in}}%
\pgfpathcurveto{\pgfqpoint{2.936829in}{1.638203in}}{\pgfqpoint{2.944729in}{1.634931in}}{\pgfqpoint{2.952965in}{1.634931in}}%
\pgfpathclose%
\pgfusepath{stroke,fill}%
\end{pgfscope}%
\begin{pgfscope}%
\pgfpathrectangle{\pgfqpoint{0.100000in}{0.212622in}}{\pgfqpoint{3.696000in}{3.696000in}}%
\pgfusepath{clip}%
\pgfsetbuttcap%
\pgfsetroundjoin%
\definecolor{currentfill}{rgb}{0.121569,0.466667,0.705882}%
\pgfsetfillcolor{currentfill}%
\pgfsetfillopacity{0.757285}%
\pgfsetlinewidth{1.003750pt}%
\definecolor{currentstroke}{rgb}{0.121569,0.466667,0.705882}%
\pgfsetstrokecolor{currentstroke}%
\pgfsetstrokeopacity{0.757285}%
\pgfsetdash{}{0pt}%
\pgfpathmoveto{\pgfqpoint{1.292824in}{2.541758in}}%
\pgfpathcurveto{\pgfqpoint{1.301060in}{2.541758in}}{\pgfqpoint{1.308960in}{2.545030in}}{\pgfqpoint{1.314784in}{2.550854in}}%
\pgfpathcurveto{\pgfqpoint{1.320608in}{2.556678in}}{\pgfqpoint{1.323880in}{2.564578in}}{\pgfqpoint{1.323880in}{2.572815in}}%
\pgfpathcurveto{\pgfqpoint{1.323880in}{2.581051in}}{\pgfqpoint{1.320608in}{2.588951in}}{\pgfqpoint{1.314784in}{2.594775in}}%
\pgfpathcurveto{\pgfqpoint{1.308960in}{2.600599in}}{\pgfqpoint{1.301060in}{2.603871in}}{\pgfqpoint{1.292824in}{2.603871in}}%
\pgfpathcurveto{\pgfqpoint{1.284588in}{2.603871in}}{\pgfqpoint{1.276688in}{2.600599in}}{\pgfqpoint{1.270864in}{2.594775in}}%
\pgfpathcurveto{\pgfqpoint{1.265040in}{2.588951in}}{\pgfqpoint{1.261767in}{2.581051in}}{\pgfqpoint{1.261767in}{2.572815in}}%
\pgfpathcurveto{\pgfqpoint{1.261767in}{2.564578in}}{\pgfqpoint{1.265040in}{2.556678in}}{\pgfqpoint{1.270864in}{2.550854in}}%
\pgfpathcurveto{\pgfqpoint{1.276688in}{2.545030in}}{\pgfqpoint{1.284588in}{2.541758in}}{\pgfqpoint{1.292824in}{2.541758in}}%
\pgfpathclose%
\pgfusepath{stroke,fill}%
\end{pgfscope}%
\begin{pgfscope}%
\pgfpathrectangle{\pgfqpoint{0.100000in}{0.212622in}}{\pgfqpoint{3.696000in}{3.696000in}}%
\pgfusepath{clip}%
\pgfsetbuttcap%
\pgfsetroundjoin%
\definecolor{currentfill}{rgb}{0.121569,0.466667,0.705882}%
\pgfsetfillcolor{currentfill}%
\pgfsetfillopacity{0.757508}%
\pgfsetlinewidth{1.003750pt}%
\definecolor{currentstroke}{rgb}{0.121569,0.466667,0.705882}%
\pgfsetstrokecolor{currentstroke}%
\pgfsetstrokeopacity{0.757508}%
\pgfsetdash{}{0pt}%
\pgfpathmoveto{\pgfqpoint{2.520829in}{2.877309in}}%
\pgfpathcurveto{\pgfqpoint{2.529065in}{2.877309in}}{\pgfqpoint{2.536965in}{2.880581in}}{\pgfqpoint{2.542789in}{2.886405in}}%
\pgfpathcurveto{\pgfqpoint{2.548613in}{2.892229in}}{\pgfqpoint{2.551885in}{2.900129in}}{\pgfqpoint{2.551885in}{2.908365in}}%
\pgfpathcurveto{\pgfqpoint{2.551885in}{2.916602in}}{\pgfqpoint{2.548613in}{2.924502in}}{\pgfqpoint{2.542789in}{2.930326in}}%
\pgfpathcurveto{\pgfqpoint{2.536965in}{2.936150in}}{\pgfqpoint{2.529065in}{2.939422in}}{\pgfqpoint{2.520829in}{2.939422in}}%
\pgfpathcurveto{\pgfqpoint{2.512592in}{2.939422in}}{\pgfqpoint{2.504692in}{2.936150in}}{\pgfqpoint{2.498868in}{2.930326in}}%
\pgfpathcurveto{\pgfqpoint{2.493045in}{2.924502in}}{\pgfqpoint{2.489772in}{2.916602in}}{\pgfqpoint{2.489772in}{2.908365in}}%
\pgfpathcurveto{\pgfqpoint{2.489772in}{2.900129in}}{\pgfqpoint{2.493045in}{2.892229in}}{\pgfqpoint{2.498868in}{2.886405in}}%
\pgfpathcurveto{\pgfqpoint{2.504692in}{2.880581in}}{\pgfqpoint{2.512592in}{2.877309in}}{\pgfqpoint{2.520829in}{2.877309in}}%
\pgfpathclose%
\pgfusepath{stroke,fill}%
\end{pgfscope}%
\begin{pgfscope}%
\pgfpathrectangle{\pgfqpoint{0.100000in}{0.212622in}}{\pgfqpoint{3.696000in}{3.696000in}}%
\pgfusepath{clip}%
\pgfsetbuttcap%
\pgfsetroundjoin%
\definecolor{currentfill}{rgb}{0.121569,0.466667,0.705882}%
\pgfsetfillcolor{currentfill}%
\pgfsetfillopacity{0.758322}%
\pgfsetlinewidth{1.003750pt}%
\definecolor{currentstroke}{rgb}{0.121569,0.466667,0.705882}%
\pgfsetstrokecolor{currentstroke}%
\pgfsetstrokeopacity{0.758322}%
\pgfsetdash{}{0pt}%
\pgfpathmoveto{\pgfqpoint{1.289907in}{2.539028in}}%
\pgfpathcurveto{\pgfqpoint{1.298143in}{2.539028in}}{\pgfqpoint{1.306043in}{2.542301in}}{\pgfqpoint{1.311867in}{2.548124in}}%
\pgfpathcurveto{\pgfqpoint{1.317691in}{2.553948in}}{\pgfqpoint{1.320963in}{2.561848in}}{\pgfqpoint{1.320963in}{2.570085in}}%
\pgfpathcurveto{\pgfqpoint{1.320963in}{2.578321in}}{\pgfqpoint{1.317691in}{2.586221in}}{\pgfqpoint{1.311867in}{2.592045in}}%
\pgfpathcurveto{\pgfqpoint{1.306043in}{2.597869in}}{\pgfqpoint{1.298143in}{2.601141in}}{\pgfqpoint{1.289907in}{2.601141in}}%
\pgfpathcurveto{\pgfqpoint{1.281670in}{2.601141in}}{\pgfqpoint{1.273770in}{2.597869in}}{\pgfqpoint{1.267946in}{2.592045in}}%
\pgfpathcurveto{\pgfqpoint{1.262122in}{2.586221in}}{\pgfqpoint{1.258850in}{2.578321in}}{\pgfqpoint{1.258850in}{2.570085in}}%
\pgfpathcurveto{\pgfqpoint{1.258850in}{2.561848in}}{\pgfqpoint{1.262122in}{2.553948in}}{\pgfqpoint{1.267946in}{2.548124in}}%
\pgfpathcurveto{\pgfqpoint{1.273770in}{2.542301in}}{\pgfqpoint{1.281670in}{2.539028in}}{\pgfqpoint{1.289907in}{2.539028in}}%
\pgfpathclose%
\pgfusepath{stroke,fill}%
\end{pgfscope}%
\begin{pgfscope}%
\pgfpathrectangle{\pgfqpoint{0.100000in}{0.212622in}}{\pgfqpoint{3.696000in}{3.696000in}}%
\pgfusepath{clip}%
\pgfsetbuttcap%
\pgfsetroundjoin%
\definecolor{currentfill}{rgb}{0.121569,0.466667,0.705882}%
\pgfsetfillcolor{currentfill}%
\pgfsetfillopacity{0.759376}%
\pgfsetlinewidth{1.003750pt}%
\definecolor{currentstroke}{rgb}{0.121569,0.466667,0.705882}%
\pgfsetstrokecolor{currentstroke}%
\pgfsetstrokeopacity{0.759376}%
\pgfsetdash{}{0pt}%
\pgfpathmoveto{\pgfqpoint{2.516754in}{2.871158in}}%
\pgfpathcurveto{\pgfqpoint{2.524991in}{2.871158in}}{\pgfqpoint{2.532891in}{2.874431in}}{\pgfqpoint{2.538715in}{2.880254in}}%
\pgfpathcurveto{\pgfqpoint{2.544539in}{2.886078in}}{\pgfqpoint{2.547811in}{2.893978in}}{\pgfqpoint{2.547811in}{2.902215in}}%
\pgfpathcurveto{\pgfqpoint{2.547811in}{2.910451in}}{\pgfqpoint{2.544539in}{2.918351in}}{\pgfqpoint{2.538715in}{2.924175in}}%
\pgfpathcurveto{\pgfqpoint{2.532891in}{2.929999in}}{\pgfqpoint{2.524991in}{2.933271in}}{\pgfqpoint{2.516754in}{2.933271in}}%
\pgfpathcurveto{\pgfqpoint{2.508518in}{2.933271in}}{\pgfqpoint{2.500618in}{2.929999in}}{\pgfqpoint{2.494794in}{2.924175in}}%
\pgfpathcurveto{\pgfqpoint{2.488970in}{2.918351in}}{\pgfqpoint{2.485698in}{2.910451in}}{\pgfqpoint{2.485698in}{2.902215in}}%
\pgfpathcurveto{\pgfqpoint{2.485698in}{2.893978in}}{\pgfqpoint{2.488970in}{2.886078in}}{\pgfqpoint{2.494794in}{2.880254in}}%
\pgfpathcurveto{\pgfqpoint{2.500618in}{2.874431in}}{\pgfqpoint{2.508518in}{2.871158in}}{\pgfqpoint{2.516754in}{2.871158in}}%
\pgfpathclose%
\pgfusepath{stroke,fill}%
\end{pgfscope}%
\begin{pgfscope}%
\pgfpathrectangle{\pgfqpoint{0.100000in}{0.212622in}}{\pgfqpoint{3.696000in}{3.696000in}}%
\pgfusepath{clip}%
\pgfsetbuttcap%
\pgfsetroundjoin%
\definecolor{currentfill}{rgb}{0.121569,0.466667,0.705882}%
\pgfsetfillcolor{currentfill}%
\pgfsetfillopacity{0.759563}%
\pgfsetlinewidth{1.003750pt}%
\definecolor{currentstroke}{rgb}{0.121569,0.466667,0.705882}%
\pgfsetstrokecolor{currentstroke}%
\pgfsetstrokeopacity{0.759563}%
\pgfsetdash{}{0pt}%
\pgfpathmoveto{\pgfqpoint{1.286150in}{2.535289in}}%
\pgfpathcurveto{\pgfqpoint{1.294386in}{2.535289in}}{\pgfqpoint{1.302286in}{2.538561in}}{\pgfqpoint{1.308110in}{2.544385in}}%
\pgfpathcurveto{\pgfqpoint{1.313934in}{2.550209in}}{\pgfqpoint{1.317206in}{2.558109in}}{\pgfqpoint{1.317206in}{2.566346in}}%
\pgfpathcurveto{\pgfqpoint{1.317206in}{2.574582in}}{\pgfqpoint{1.313934in}{2.582482in}}{\pgfqpoint{1.308110in}{2.588306in}}%
\pgfpathcurveto{\pgfqpoint{1.302286in}{2.594130in}}{\pgfqpoint{1.294386in}{2.597402in}}{\pgfqpoint{1.286150in}{2.597402in}}%
\pgfpathcurveto{\pgfqpoint{1.277914in}{2.597402in}}{\pgfqpoint{1.270014in}{2.594130in}}{\pgfqpoint{1.264190in}{2.588306in}}%
\pgfpathcurveto{\pgfqpoint{1.258366in}{2.582482in}}{\pgfqpoint{1.255093in}{2.574582in}}{\pgfqpoint{1.255093in}{2.566346in}}%
\pgfpathcurveto{\pgfqpoint{1.255093in}{2.558109in}}{\pgfqpoint{1.258366in}{2.550209in}}{\pgfqpoint{1.264190in}{2.544385in}}%
\pgfpathcurveto{\pgfqpoint{1.270014in}{2.538561in}}{\pgfqpoint{1.277914in}{2.535289in}}{\pgfqpoint{1.286150in}{2.535289in}}%
\pgfpathclose%
\pgfusepath{stroke,fill}%
\end{pgfscope}%
\begin{pgfscope}%
\pgfpathrectangle{\pgfqpoint{0.100000in}{0.212622in}}{\pgfqpoint{3.696000in}{3.696000in}}%
\pgfusepath{clip}%
\pgfsetbuttcap%
\pgfsetroundjoin%
\definecolor{currentfill}{rgb}{0.121569,0.466667,0.705882}%
\pgfsetfillcolor{currentfill}%
\pgfsetfillopacity{0.760592}%
\pgfsetlinewidth{1.003750pt}%
\definecolor{currentstroke}{rgb}{0.121569,0.466667,0.705882}%
\pgfsetstrokecolor{currentstroke}%
\pgfsetstrokeopacity{0.760592}%
\pgfsetdash{}{0pt}%
\pgfpathmoveto{\pgfqpoint{2.514020in}{2.867265in}}%
\pgfpathcurveto{\pgfqpoint{2.522256in}{2.867265in}}{\pgfqpoint{2.530156in}{2.870537in}}{\pgfqpoint{2.535980in}{2.876361in}}%
\pgfpathcurveto{\pgfqpoint{2.541804in}{2.882185in}}{\pgfqpoint{2.545076in}{2.890085in}}{\pgfqpoint{2.545076in}{2.898321in}}%
\pgfpathcurveto{\pgfqpoint{2.545076in}{2.906557in}}{\pgfqpoint{2.541804in}{2.914458in}}{\pgfqpoint{2.535980in}{2.920281in}}%
\pgfpathcurveto{\pgfqpoint{2.530156in}{2.926105in}}{\pgfqpoint{2.522256in}{2.929378in}}{\pgfqpoint{2.514020in}{2.929378in}}%
\pgfpathcurveto{\pgfqpoint{2.505783in}{2.929378in}}{\pgfqpoint{2.497883in}{2.926105in}}{\pgfqpoint{2.492059in}{2.920281in}}%
\pgfpathcurveto{\pgfqpoint{2.486235in}{2.914458in}}{\pgfqpoint{2.482963in}{2.906557in}}{\pgfqpoint{2.482963in}{2.898321in}}%
\pgfpathcurveto{\pgfqpoint{2.482963in}{2.890085in}}{\pgfqpoint{2.486235in}{2.882185in}}{\pgfqpoint{2.492059in}{2.876361in}}%
\pgfpathcurveto{\pgfqpoint{2.497883in}{2.870537in}}{\pgfqpoint{2.505783in}{2.867265in}}{\pgfqpoint{2.514020in}{2.867265in}}%
\pgfpathclose%
\pgfusepath{stroke,fill}%
\end{pgfscope}%
\begin{pgfscope}%
\pgfpathrectangle{\pgfqpoint{0.100000in}{0.212622in}}{\pgfqpoint{3.696000in}{3.696000in}}%
\pgfusepath{clip}%
\pgfsetbuttcap%
\pgfsetroundjoin%
\definecolor{currentfill}{rgb}{0.121569,0.466667,0.705882}%
\pgfsetfillcolor{currentfill}%
\pgfsetfillopacity{0.760949}%
\pgfsetlinewidth{1.003750pt}%
\definecolor{currentstroke}{rgb}{0.121569,0.466667,0.705882}%
\pgfsetstrokecolor{currentstroke}%
\pgfsetstrokeopacity{0.760949}%
\pgfsetdash{}{0pt}%
\pgfpathmoveto{\pgfqpoint{2.945352in}{1.627410in}}%
\pgfpathcurveto{\pgfqpoint{2.953588in}{1.627410in}}{\pgfqpoint{2.961488in}{1.630682in}}{\pgfqpoint{2.967312in}{1.636506in}}%
\pgfpathcurveto{\pgfqpoint{2.973136in}{1.642330in}}{\pgfqpoint{2.976408in}{1.650230in}}{\pgfqpoint{2.976408in}{1.658467in}}%
\pgfpathcurveto{\pgfqpoint{2.976408in}{1.666703in}}{\pgfqpoint{2.973136in}{1.674603in}}{\pgfqpoint{2.967312in}{1.680427in}}%
\pgfpathcurveto{\pgfqpoint{2.961488in}{1.686251in}}{\pgfqpoint{2.953588in}{1.689523in}}{\pgfqpoint{2.945352in}{1.689523in}}%
\pgfpathcurveto{\pgfqpoint{2.937116in}{1.689523in}}{\pgfqpoint{2.929216in}{1.686251in}}{\pgfqpoint{2.923392in}{1.680427in}}%
\pgfpathcurveto{\pgfqpoint{2.917568in}{1.674603in}}{\pgfqpoint{2.914295in}{1.666703in}}{\pgfqpoint{2.914295in}{1.658467in}}%
\pgfpathcurveto{\pgfqpoint{2.914295in}{1.650230in}}{\pgfqpoint{2.917568in}{1.642330in}}{\pgfqpoint{2.923392in}{1.636506in}}%
\pgfpathcurveto{\pgfqpoint{2.929216in}{1.630682in}}{\pgfqpoint{2.937116in}{1.627410in}}{\pgfqpoint{2.945352in}{1.627410in}}%
\pgfpathclose%
\pgfusepath{stroke,fill}%
\end{pgfscope}%
\begin{pgfscope}%
\pgfpathrectangle{\pgfqpoint{0.100000in}{0.212622in}}{\pgfqpoint{3.696000in}{3.696000in}}%
\pgfusepath{clip}%
\pgfsetbuttcap%
\pgfsetroundjoin%
\definecolor{currentfill}{rgb}{0.121569,0.466667,0.705882}%
\pgfsetfillcolor{currentfill}%
\pgfsetfillopacity{0.761068}%
\pgfsetlinewidth{1.003750pt}%
\definecolor{currentstroke}{rgb}{0.121569,0.466667,0.705882}%
\pgfsetstrokecolor{currentstroke}%
\pgfsetstrokeopacity{0.761068}%
\pgfsetdash{}{0pt}%
\pgfpathmoveto{\pgfqpoint{1.281306in}{2.529773in}}%
\pgfpathcurveto{\pgfqpoint{1.289542in}{2.529773in}}{\pgfqpoint{1.297442in}{2.533045in}}{\pgfqpoint{1.303266in}{2.538869in}}%
\pgfpathcurveto{\pgfqpoint{1.309090in}{2.544693in}}{\pgfqpoint{1.312363in}{2.552593in}}{\pgfqpoint{1.312363in}{2.560830in}}%
\pgfpathcurveto{\pgfqpoint{1.312363in}{2.569066in}}{\pgfqpoint{1.309090in}{2.576966in}}{\pgfqpoint{1.303266in}{2.582790in}}%
\pgfpathcurveto{\pgfqpoint{1.297442in}{2.588614in}}{\pgfqpoint{1.289542in}{2.591886in}}{\pgfqpoint{1.281306in}{2.591886in}}%
\pgfpathcurveto{\pgfqpoint{1.273070in}{2.591886in}}{\pgfqpoint{1.265170in}{2.588614in}}{\pgfqpoint{1.259346in}{2.582790in}}%
\pgfpathcurveto{\pgfqpoint{1.253522in}{2.576966in}}{\pgfqpoint{1.250250in}{2.569066in}}{\pgfqpoint{1.250250in}{2.560830in}}%
\pgfpathcurveto{\pgfqpoint{1.250250in}{2.552593in}}{\pgfqpoint{1.253522in}{2.544693in}}{\pgfqpoint{1.259346in}{2.538869in}}%
\pgfpathcurveto{\pgfqpoint{1.265170in}{2.533045in}}{\pgfqpoint{1.273070in}{2.529773in}}{\pgfqpoint{1.281306in}{2.529773in}}%
\pgfpathclose%
\pgfusepath{stroke,fill}%
\end{pgfscope}%
\begin{pgfscope}%
\pgfpathrectangle{\pgfqpoint{0.100000in}{0.212622in}}{\pgfqpoint{3.696000in}{3.696000in}}%
\pgfusepath{clip}%
\pgfsetbuttcap%
\pgfsetroundjoin%
\definecolor{currentfill}{rgb}{0.121569,0.466667,0.705882}%
\pgfsetfillcolor{currentfill}%
\pgfsetfillopacity{0.761714}%
\pgfsetlinewidth{1.003750pt}%
\definecolor{currentstroke}{rgb}{0.121569,0.466667,0.705882}%
\pgfsetstrokecolor{currentstroke}%
\pgfsetstrokeopacity{0.761714}%
\pgfsetdash{}{0pt}%
\pgfpathmoveto{\pgfqpoint{2.511524in}{2.863739in}}%
\pgfpathcurveto{\pgfqpoint{2.519760in}{2.863739in}}{\pgfqpoint{2.527660in}{2.867011in}}{\pgfqpoint{2.533484in}{2.872835in}}%
\pgfpathcurveto{\pgfqpoint{2.539308in}{2.878659in}}{\pgfqpoint{2.542580in}{2.886559in}}{\pgfqpoint{2.542580in}{2.894795in}}%
\pgfpathcurveto{\pgfqpoint{2.542580in}{2.903031in}}{\pgfqpoint{2.539308in}{2.910931in}}{\pgfqpoint{2.533484in}{2.916755in}}%
\pgfpathcurveto{\pgfqpoint{2.527660in}{2.922579in}}{\pgfqpoint{2.519760in}{2.925852in}}{\pgfqpoint{2.511524in}{2.925852in}}%
\pgfpathcurveto{\pgfqpoint{2.503287in}{2.925852in}}{\pgfqpoint{2.495387in}{2.922579in}}{\pgfqpoint{2.489563in}{2.916755in}}%
\pgfpathcurveto{\pgfqpoint{2.483739in}{2.910931in}}{\pgfqpoint{2.480467in}{2.903031in}}{\pgfqpoint{2.480467in}{2.894795in}}%
\pgfpathcurveto{\pgfqpoint{2.480467in}{2.886559in}}{\pgfqpoint{2.483739in}{2.878659in}}{\pgfqpoint{2.489563in}{2.872835in}}%
\pgfpathcurveto{\pgfqpoint{2.495387in}{2.867011in}}{\pgfqpoint{2.503287in}{2.863739in}}{\pgfqpoint{2.511524in}{2.863739in}}%
\pgfpathclose%
\pgfusepath{stroke,fill}%
\end{pgfscope}%
\begin{pgfscope}%
\pgfpathrectangle{\pgfqpoint{0.100000in}{0.212622in}}{\pgfqpoint{3.696000in}{3.696000in}}%
\pgfusepath{clip}%
\pgfsetbuttcap%
\pgfsetroundjoin%
\definecolor{currentfill}{rgb}{0.121569,0.466667,0.705882}%
\pgfsetfillcolor{currentfill}%
\pgfsetfillopacity{0.762478}%
\pgfsetlinewidth{1.003750pt}%
\definecolor{currentstroke}{rgb}{0.121569,0.466667,0.705882}%
\pgfsetstrokecolor{currentstroke}%
\pgfsetstrokeopacity{0.762478}%
\pgfsetdash{}{0pt}%
\pgfpathmoveto{\pgfqpoint{2.509694in}{2.861103in}}%
\pgfpathcurveto{\pgfqpoint{2.517931in}{2.861103in}}{\pgfqpoint{2.525831in}{2.864376in}}{\pgfqpoint{2.531655in}{2.870200in}}%
\pgfpathcurveto{\pgfqpoint{2.537479in}{2.876024in}}{\pgfqpoint{2.540751in}{2.883924in}}{\pgfqpoint{2.540751in}{2.892160in}}%
\pgfpathcurveto{\pgfqpoint{2.540751in}{2.900396in}}{\pgfqpoint{2.537479in}{2.908296in}}{\pgfqpoint{2.531655in}{2.914120in}}%
\pgfpathcurveto{\pgfqpoint{2.525831in}{2.919944in}}{\pgfqpoint{2.517931in}{2.923216in}}{\pgfqpoint{2.509694in}{2.923216in}}%
\pgfpathcurveto{\pgfqpoint{2.501458in}{2.923216in}}{\pgfqpoint{2.493558in}{2.919944in}}{\pgfqpoint{2.487734in}{2.914120in}}%
\pgfpathcurveto{\pgfqpoint{2.481910in}{2.908296in}}{\pgfqpoint{2.478638in}{2.900396in}}{\pgfqpoint{2.478638in}{2.892160in}}%
\pgfpathcurveto{\pgfqpoint{2.478638in}{2.883924in}}{\pgfqpoint{2.481910in}{2.876024in}}{\pgfqpoint{2.487734in}{2.870200in}}%
\pgfpathcurveto{\pgfqpoint{2.493558in}{2.864376in}}{\pgfqpoint{2.501458in}{2.861103in}}{\pgfqpoint{2.509694in}{2.861103in}}%
\pgfpathclose%
\pgfusepath{stroke,fill}%
\end{pgfscope}%
\begin{pgfscope}%
\pgfpathrectangle{\pgfqpoint{0.100000in}{0.212622in}}{\pgfqpoint{3.696000in}{3.696000in}}%
\pgfusepath{clip}%
\pgfsetbuttcap%
\pgfsetroundjoin%
\definecolor{currentfill}{rgb}{0.121569,0.466667,0.705882}%
\pgfsetfillcolor{currentfill}%
\pgfsetfillopacity{0.762562}%
\pgfsetlinewidth{1.003750pt}%
\definecolor{currentstroke}{rgb}{0.121569,0.466667,0.705882}%
\pgfsetstrokecolor{currentstroke}%
\pgfsetstrokeopacity{0.762562}%
\pgfsetdash{}{0pt}%
\pgfpathmoveto{\pgfqpoint{1.276491in}{2.523130in}}%
\pgfpathcurveto{\pgfqpoint{1.284728in}{2.523130in}}{\pgfqpoint{1.292628in}{2.526402in}}{\pgfqpoint{1.298452in}{2.532226in}}%
\pgfpathcurveto{\pgfqpoint{1.304276in}{2.538050in}}{\pgfqpoint{1.307548in}{2.545950in}}{\pgfqpoint{1.307548in}{2.554186in}}%
\pgfpathcurveto{\pgfqpoint{1.307548in}{2.562422in}}{\pgfqpoint{1.304276in}{2.570322in}}{\pgfqpoint{1.298452in}{2.576146in}}%
\pgfpathcurveto{\pgfqpoint{1.292628in}{2.581970in}}{\pgfqpoint{1.284728in}{2.585243in}}{\pgfqpoint{1.276491in}{2.585243in}}%
\pgfpathcurveto{\pgfqpoint{1.268255in}{2.585243in}}{\pgfqpoint{1.260355in}{2.581970in}}{\pgfqpoint{1.254531in}{2.576146in}}%
\pgfpathcurveto{\pgfqpoint{1.248707in}{2.570322in}}{\pgfqpoint{1.245435in}{2.562422in}}{\pgfqpoint{1.245435in}{2.554186in}}%
\pgfpathcurveto{\pgfqpoint{1.245435in}{2.545950in}}{\pgfqpoint{1.248707in}{2.538050in}}{\pgfqpoint{1.254531in}{2.532226in}}%
\pgfpathcurveto{\pgfqpoint{1.260355in}{2.526402in}}{\pgfqpoint{1.268255in}{2.523130in}}{\pgfqpoint{1.276491in}{2.523130in}}%
\pgfpathclose%
\pgfusepath{stroke,fill}%
\end{pgfscope}%
\begin{pgfscope}%
\pgfpathrectangle{\pgfqpoint{0.100000in}{0.212622in}}{\pgfqpoint{3.696000in}{3.696000in}}%
\pgfusepath{clip}%
\pgfsetbuttcap%
\pgfsetroundjoin%
\definecolor{currentfill}{rgb}{0.121569,0.466667,0.705882}%
\pgfsetfillcolor{currentfill}%
\pgfsetfillopacity{0.763834}%
\pgfsetlinewidth{1.003750pt}%
\definecolor{currentstroke}{rgb}{0.121569,0.466667,0.705882}%
\pgfsetstrokecolor{currentstroke}%
\pgfsetstrokeopacity{0.763834}%
\pgfsetdash{}{0pt}%
\pgfpathmoveto{\pgfqpoint{2.506242in}{2.856235in}}%
\pgfpathcurveto{\pgfqpoint{2.514479in}{2.856235in}}{\pgfqpoint{2.522379in}{2.859508in}}{\pgfqpoint{2.528203in}{2.865332in}}%
\pgfpathcurveto{\pgfqpoint{2.534027in}{2.871156in}}{\pgfqpoint{2.537299in}{2.879056in}}{\pgfqpoint{2.537299in}{2.887292in}}%
\pgfpathcurveto{\pgfqpoint{2.537299in}{2.895528in}}{\pgfqpoint{2.534027in}{2.903428in}}{\pgfqpoint{2.528203in}{2.909252in}}%
\pgfpathcurveto{\pgfqpoint{2.522379in}{2.915076in}}{\pgfqpoint{2.514479in}{2.918348in}}{\pgfqpoint{2.506242in}{2.918348in}}%
\pgfpathcurveto{\pgfqpoint{2.498006in}{2.918348in}}{\pgfqpoint{2.490106in}{2.915076in}}{\pgfqpoint{2.484282in}{2.909252in}}%
\pgfpathcurveto{\pgfqpoint{2.478458in}{2.903428in}}{\pgfqpoint{2.475186in}{2.895528in}}{\pgfqpoint{2.475186in}{2.887292in}}%
\pgfpathcurveto{\pgfqpoint{2.475186in}{2.879056in}}{\pgfqpoint{2.478458in}{2.871156in}}{\pgfqpoint{2.484282in}{2.865332in}}%
\pgfpathcurveto{\pgfqpoint{2.490106in}{2.859508in}}{\pgfqpoint{2.498006in}{2.856235in}}{\pgfqpoint{2.506242in}{2.856235in}}%
\pgfpathclose%
\pgfusepath{stroke,fill}%
\end{pgfscope}%
\begin{pgfscope}%
\pgfpathrectangle{\pgfqpoint{0.100000in}{0.212622in}}{\pgfqpoint{3.696000in}{3.696000in}}%
\pgfusepath{clip}%
\pgfsetbuttcap%
\pgfsetroundjoin%
\definecolor{currentfill}{rgb}{0.121569,0.466667,0.705882}%
\pgfsetfillcolor{currentfill}%
\pgfsetfillopacity{0.764421}%
\pgfsetlinewidth{1.003750pt}%
\definecolor{currentstroke}{rgb}{0.121569,0.466667,0.705882}%
\pgfsetstrokecolor{currentstroke}%
\pgfsetstrokeopacity{0.764421}%
\pgfsetdash{}{0pt}%
\pgfpathmoveto{\pgfqpoint{1.270860in}{2.515817in}}%
\pgfpathcurveto{\pgfqpoint{1.279096in}{2.515817in}}{\pgfqpoint{1.286996in}{2.519089in}}{\pgfqpoint{1.292820in}{2.524913in}}%
\pgfpathcurveto{\pgfqpoint{1.298644in}{2.530737in}}{\pgfqpoint{1.301916in}{2.538637in}}{\pgfqpoint{1.301916in}{2.546874in}}%
\pgfpathcurveto{\pgfqpoint{1.301916in}{2.555110in}}{\pgfqpoint{1.298644in}{2.563010in}}{\pgfqpoint{1.292820in}{2.568834in}}%
\pgfpathcurveto{\pgfqpoint{1.286996in}{2.574658in}}{\pgfqpoint{1.279096in}{2.577930in}}{\pgfqpoint{1.270860in}{2.577930in}}%
\pgfpathcurveto{\pgfqpoint{1.262624in}{2.577930in}}{\pgfqpoint{1.254724in}{2.574658in}}{\pgfqpoint{1.248900in}{2.568834in}}%
\pgfpathcurveto{\pgfqpoint{1.243076in}{2.563010in}}{\pgfqpoint{1.239803in}{2.555110in}}{\pgfqpoint{1.239803in}{2.546874in}}%
\pgfpathcurveto{\pgfqpoint{1.239803in}{2.538637in}}{\pgfqpoint{1.243076in}{2.530737in}}{\pgfqpoint{1.248900in}{2.524913in}}%
\pgfpathcurveto{\pgfqpoint{1.254724in}{2.519089in}}{\pgfqpoint{1.262624in}{2.515817in}}{\pgfqpoint{1.270860in}{2.515817in}}%
\pgfpathclose%
\pgfusepath{stroke,fill}%
\end{pgfscope}%
\begin{pgfscope}%
\pgfpathrectangle{\pgfqpoint{0.100000in}{0.212622in}}{\pgfqpoint{3.696000in}{3.696000in}}%
\pgfusepath{clip}%
\pgfsetbuttcap%
\pgfsetroundjoin%
\definecolor{currentfill}{rgb}{0.121569,0.466667,0.705882}%
\pgfsetfillcolor{currentfill}%
\pgfsetfillopacity{0.764629}%
\pgfsetlinewidth{1.003750pt}%
\definecolor{currentstroke}{rgb}{0.121569,0.466667,0.705882}%
\pgfsetstrokecolor{currentstroke}%
\pgfsetstrokeopacity{0.764629}%
\pgfsetdash{}{0pt}%
\pgfpathmoveto{\pgfqpoint{2.504286in}{2.853692in}}%
\pgfpathcurveto{\pgfqpoint{2.512523in}{2.853692in}}{\pgfqpoint{2.520423in}{2.856964in}}{\pgfqpoint{2.526247in}{2.862788in}}%
\pgfpathcurveto{\pgfqpoint{2.532071in}{2.868612in}}{\pgfqpoint{2.535343in}{2.876512in}}{\pgfqpoint{2.535343in}{2.884748in}}%
\pgfpathcurveto{\pgfqpoint{2.535343in}{2.892985in}}{\pgfqpoint{2.532071in}{2.900885in}}{\pgfqpoint{2.526247in}{2.906709in}}%
\pgfpathcurveto{\pgfqpoint{2.520423in}{2.912533in}}{\pgfqpoint{2.512523in}{2.915805in}}{\pgfqpoint{2.504286in}{2.915805in}}%
\pgfpathcurveto{\pgfqpoint{2.496050in}{2.915805in}}{\pgfqpoint{2.488150in}{2.912533in}}{\pgfqpoint{2.482326in}{2.906709in}}%
\pgfpathcurveto{\pgfqpoint{2.476502in}{2.900885in}}{\pgfqpoint{2.473230in}{2.892985in}}{\pgfqpoint{2.473230in}{2.884748in}}%
\pgfpathcurveto{\pgfqpoint{2.473230in}{2.876512in}}{\pgfqpoint{2.476502in}{2.868612in}}{\pgfqpoint{2.482326in}{2.862788in}}%
\pgfpathcurveto{\pgfqpoint{2.488150in}{2.856964in}}{\pgfqpoint{2.496050in}{2.853692in}}{\pgfqpoint{2.504286in}{2.853692in}}%
\pgfpathclose%
\pgfusepath{stroke,fill}%
\end{pgfscope}%
\begin{pgfscope}%
\pgfpathrectangle{\pgfqpoint{0.100000in}{0.212622in}}{\pgfqpoint{3.696000in}{3.696000in}}%
\pgfusepath{clip}%
\pgfsetbuttcap%
\pgfsetroundjoin%
\definecolor{currentfill}{rgb}{0.121569,0.466667,0.705882}%
\pgfsetfillcolor{currentfill}%
\pgfsetfillopacity{0.764735}%
\pgfsetlinewidth{1.003750pt}%
\definecolor{currentstroke}{rgb}{0.121569,0.466667,0.705882}%
\pgfsetstrokecolor{currentstroke}%
\pgfsetstrokeopacity{0.764735}%
\pgfsetdash{}{0pt}%
\pgfpathmoveto{\pgfqpoint{2.938484in}{1.620982in}}%
\pgfpathcurveto{\pgfqpoint{2.946720in}{1.620982in}}{\pgfqpoint{2.954620in}{1.624254in}}{\pgfqpoint{2.960444in}{1.630078in}}%
\pgfpathcurveto{\pgfqpoint{2.966268in}{1.635902in}}{\pgfqpoint{2.969540in}{1.643802in}}{\pgfqpoint{2.969540in}{1.652038in}}%
\pgfpathcurveto{\pgfqpoint{2.969540in}{1.660275in}}{\pgfqpoint{2.966268in}{1.668175in}}{\pgfqpoint{2.960444in}{1.673999in}}%
\pgfpathcurveto{\pgfqpoint{2.954620in}{1.679822in}}{\pgfqpoint{2.946720in}{1.683095in}}{\pgfqpoint{2.938484in}{1.683095in}}%
\pgfpathcurveto{\pgfqpoint{2.930248in}{1.683095in}}{\pgfqpoint{2.922347in}{1.679822in}}{\pgfqpoint{2.916524in}{1.673999in}}%
\pgfpathcurveto{\pgfqpoint{2.910700in}{1.668175in}}{\pgfqpoint{2.907427in}{1.660275in}}{\pgfqpoint{2.907427in}{1.652038in}}%
\pgfpathcurveto{\pgfqpoint{2.907427in}{1.643802in}}{\pgfqpoint{2.910700in}{1.635902in}}{\pgfqpoint{2.916524in}{1.630078in}}%
\pgfpathcurveto{\pgfqpoint{2.922347in}{1.624254in}}{\pgfqpoint{2.930248in}{1.620982in}}{\pgfqpoint{2.938484in}{1.620982in}}%
\pgfpathclose%
\pgfusepath{stroke,fill}%
\end{pgfscope}%
\begin{pgfscope}%
\pgfpathrectangle{\pgfqpoint{0.100000in}{0.212622in}}{\pgfqpoint{3.696000in}{3.696000in}}%
\pgfusepath{clip}%
\pgfsetbuttcap%
\pgfsetroundjoin%
\definecolor{currentfill}{rgb}{0.121569,0.466667,0.705882}%
\pgfsetfillcolor{currentfill}%
\pgfsetfillopacity{0.766118}%
\pgfsetlinewidth{1.003750pt}%
\definecolor{currentstroke}{rgb}{0.121569,0.466667,0.705882}%
\pgfsetstrokecolor{currentstroke}%
\pgfsetstrokeopacity{0.766118}%
\pgfsetdash{}{0pt}%
\pgfpathmoveto{\pgfqpoint{2.500820in}{2.849241in}}%
\pgfpathcurveto{\pgfqpoint{2.509057in}{2.849241in}}{\pgfqpoint{2.516957in}{2.852513in}}{\pgfqpoint{2.522781in}{2.858337in}}%
\pgfpathcurveto{\pgfqpoint{2.528605in}{2.864161in}}{\pgfqpoint{2.531877in}{2.872061in}}{\pgfqpoint{2.531877in}{2.880297in}}%
\pgfpathcurveto{\pgfqpoint{2.531877in}{2.888533in}}{\pgfqpoint{2.528605in}{2.896433in}}{\pgfqpoint{2.522781in}{2.902257in}}%
\pgfpathcurveto{\pgfqpoint{2.516957in}{2.908081in}}{\pgfqpoint{2.509057in}{2.911354in}}{\pgfqpoint{2.500820in}{2.911354in}}%
\pgfpathcurveto{\pgfqpoint{2.492584in}{2.911354in}}{\pgfqpoint{2.484684in}{2.908081in}}{\pgfqpoint{2.478860in}{2.902257in}}%
\pgfpathcurveto{\pgfqpoint{2.473036in}{2.896433in}}{\pgfqpoint{2.469764in}{2.888533in}}{\pgfqpoint{2.469764in}{2.880297in}}%
\pgfpathcurveto{\pgfqpoint{2.469764in}{2.872061in}}{\pgfqpoint{2.473036in}{2.864161in}}{\pgfqpoint{2.478860in}{2.858337in}}%
\pgfpathcurveto{\pgfqpoint{2.484684in}{2.852513in}}{\pgfqpoint{2.492584in}{2.849241in}}{\pgfqpoint{2.500820in}{2.849241in}}%
\pgfpathclose%
\pgfusepath{stroke,fill}%
\end{pgfscope}%
\begin{pgfscope}%
\pgfpathrectangle{\pgfqpoint{0.100000in}{0.212622in}}{\pgfqpoint{3.696000in}{3.696000in}}%
\pgfusepath{clip}%
\pgfsetbuttcap%
\pgfsetroundjoin%
\definecolor{currentfill}{rgb}{0.121569,0.466667,0.705882}%
\pgfsetfillcolor{currentfill}%
\pgfsetfillopacity{0.766595}%
\pgfsetlinewidth{1.003750pt}%
\definecolor{currentstroke}{rgb}{0.121569,0.466667,0.705882}%
\pgfsetstrokecolor{currentstroke}%
\pgfsetstrokeopacity{0.766595}%
\pgfsetdash{}{0pt}%
\pgfpathmoveto{\pgfqpoint{1.264681in}{2.508861in}}%
\pgfpathcurveto{\pgfqpoint{1.272918in}{2.508861in}}{\pgfqpoint{1.280818in}{2.512133in}}{\pgfqpoint{1.286642in}{2.517957in}}%
\pgfpathcurveto{\pgfqpoint{1.292465in}{2.523781in}}{\pgfqpoint{1.295738in}{2.531681in}}{\pgfqpoint{1.295738in}{2.539917in}}%
\pgfpathcurveto{\pgfqpoint{1.295738in}{2.548153in}}{\pgfqpoint{1.292465in}{2.556053in}}{\pgfqpoint{1.286642in}{2.561877in}}%
\pgfpathcurveto{\pgfqpoint{1.280818in}{2.567701in}}{\pgfqpoint{1.272918in}{2.570974in}}{\pgfqpoint{1.264681in}{2.570974in}}%
\pgfpathcurveto{\pgfqpoint{1.256445in}{2.570974in}}{\pgfqpoint{1.248545in}{2.567701in}}{\pgfqpoint{1.242721in}{2.561877in}}%
\pgfpathcurveto{\pgfqpoint{1.236897in}{2.556053in}}{\pgfqpoint{1.233625in}{2.548153in}}{\pgfqpoint{1.233625in}{2.539917in}}%
\pgfpathcurveto{\pgfqpoint{1.233625in}{2.531681in}}{\pgfqpoint{1.236897in}{2.523781in}}{\pgfqpoint{1.242721in}{2.517957in}}%
\pgfpathcurveto{\pgfqpoint{1.248545in}{2.512133in}}{\pgfqpoint{1.256445in}{2.508861in}}{\pgfqpoint{1.264681in}{2.508861in}}%
\pgfpathclose%
\pgfusepath{stroke,fill}%
\end{pgfscope}%
\begin{pgfscope}%
\pgfpathrectangle{\pgfqpoint{0.100000in}{0.212622in}}{\pgfqpoint{3.696000in}{3.696000in}}%
\pgfusepath{clip}%
\pgfsetbuttcap%
\pgfsetroundjoin%
\definecolor{currentfill}{rgb}{0.121569,0.466667,0.705882}%
\pgfsetfillcolor{currentfill}%
\pgfsetfillopacity{0.767273}%
\pgfsetlinewidth{1.003750pt}%
\definecolor{currentstroke}{rgb}{0.121569,0.466667,0.705882}%
\pgfsetstrokecolor{currentstroke}%
\pgfsetstrokeopacity{0.767273}%
\pgfsetdash{}{0pt}%
\pgfpathmoveto{\pgfqpoint{2.498018in}{2.845432in}}%
\pgfpathcurveto{\pgfqpoint{2.506254in}{2.845432in}}{\pgfqpoint{2.514154in}{2.848704in}}{\pgfqpoint{2.519978in}{2.854528in}}%
\pgfpathcurveto{\pgfqpoint{2.525802in}{2.860352in}}{\pgfqpoint{2.529074in}{2.868252in}}{\pgfqpoint{2.529074in}{2.876488in}}%
\pgfpathcurveto{\pgfqpoint{2.529074in}{2.884725in}}{\pgfqpoint{2.525802in}{2.892625in}}{\pgfqpoint{2.519978in}{2.898449in}}%
\pgfpathcurveto{\pgfqpoint{2.514154in}{2.904273in}}{\pgfqpoint{2.506254in}{2.907545in}}{\pgfqpoint{2.498018in}{2.907545in}}%
\pgfpathcurveto{\pgfqpoint{2.489781in}{2.907545in}}{\pgfqpoint{2.481881in}{2.904273in}}{\pgfqpoint{2.476057in}{2.898449in}}%
\pgfpathcurveto{\pgfqpoint{2.470233in}{2.892625in}}{\pgfqpoint{2.466961in}{2.884725in}}{\pgfqpoint{2.466961in}{2.876488in}}%
\pgfpathcurveto{\pgfqpoint{2.466961in}{2.868252in}}{\pgfqpoint{2.470233in}{2.860352in}}{\pgfqpoint{2.476057in}{2.854528in}}%
\pgfpathcurveto{\pgfqpoint{2.481881in}{2.848704in}}{\pgfqpoint{2.489781in}{2.845432in}}{\pgfqpoint{2.498018in}{2.845432in}}%
\pgfpathclose%
\pgfusepath{stroke,fill}%
\end{pgfscope}%
\begin{pgfscope}%
\pgfpathrectangle{\pgfqpoint{0.100000in}{0.212622in}}{\pgfqpoint{3.696000in}{3.696000in}}%
\pgfusepath{clip}%
\pgfsetbuttcap%
\pgfsetroundjoin%
\definecolor{currentfill}{rgb}{0.121569,0.466667,0.705882}%
\pgfsetfillcolor{currentfill}%
\pgfsetfillopacity{0.768325}%
\pgfsetlinewidth{1.003750pt}%
\definecolor{currentstroke}{rgb}{0.121569,0.466667,0.705882}%
\pgfsetstrokecolor{currentstroke}%
\pgfsetstrokeopacity{0.768325}%
\pgfsetdash{}{0pt}%
\pgfpathmoveto{\pgfqpoint{2.932304in}{1.615333in}}%
\pgfpathcurveto{\pgfqpoint{2.940540in}{1.615333in}}{\pgfqpoint{2.948440in}{1.618606in}}{\pgfqpoint{2.954264in}{1.624430in}}%
\pgfpathcurveto{\pgfqpoint{2.960088in}{1.630254in}}{\pgfqpoint{2.963360in}{1.638154in}}{\pgfqpoint{2.963360in}{1.646390in}}%
\pgfpathcurveto{\pgfqpoint{2.963360in}{1.654626in}}{\pgfqpoint{2.960088in}{1.662526in}}{\pgfqpoint{2.954264in}{1.668350in}}%
\pgfpathcurveto{\pgfqpoint{2.948440in}{1.674174in}}{\pgfqpoint{2.940540in}{1.677446in}}{\pgfqpoint{2.932304in}{1.677446in}}%
\pgfpathcurveto{\pgfqpoint{2.924067in}{1.677446in}}{\pgfqpoint{2.916167in}{1.674174in}}{\pgfqpoint{2.910343in}{1.668350in}}%
\pgfpathcurveto{\pgfqpoint{2.904519in}{1.662526in}}{\pgfqpoint{2.901247in}{1.654626in}}{\pgfqpoint{2.901247in}{1.646390in}}%
\pgfpathcurveto{\pgfqpoint{2.901247in}{1.638154in}}{\pgfqpoint{2.904519in}{1.630254in}}{\pgfqpoint{2.910343in}{1.624430in}}%
\pgfpathcurveto{\pgfqpoint{2.916167in}{1.618606in}}{\pgfqpoint{2.924067in}{1.615333in}}{\pgfqpoint{2.932304in}{1.615333in}}%
\pgfpathclose%
\pgfusepath{stroke,fill}%
\end{pgfscope}%
\begin{pgfscope}%
\pgfpathrectangle{\pgfqpoint{0.100000in}{0.212622in}}{\pgfqpoint{3.696000in}{3.696000in}}%
\pgfusepath{clip}%
\pgfsetbuttcap%
\pgfsetroundjoin%
\definecolor{currentfill}{rgb}{0.121569,0.466667,0.705882}%
\pgfsetfillcolor{currentfill}%
\pgfsetfillopacity{0.768762}%
\pgfsetlinewidth{1.003750pt}%
\definecolor{currentstroke}{rgb}{0.121569,0.466667,0.705882}%
\pgfsetstrokecolor{currentstroke}%
\pgfsetstrokeopacity{0.768762}%
\pgfsetdash{}{0pt}%
\pgfpathmoveto{\pgfqpoint{1.258077in}{2.501423in}}%
\pgfpathcurveto{\pgfqpoint{1.266314in}{2.501423in}}{\pgfqpoint{1.274214in}{2.504695in}}{\pgfqpoint{1.280038in}{2.510519in}}%
\pgfpathcurveto{\pgfqpoint{1.285862in}{2.516343in}}{\pgfqpoint{1.289134in}{2.524243in}}{\pgfqpoint{1.289134in}{2.532480in}}%
\pgfpathcurveto{\pgfqpoint{1.289134in}{2.540716in}}{\pgfqpoint{1.285862in}{2.548616in}}{\pgfqpoint{1.280038in}{2.554440in}}%
\pgfpathcurveto{\pgfqpoint{1.274214in}{2.560264in}}{\pgfqpoint{1.266314in}{2.563536in}}{\pgfqpoint{1.258077in}{2.563536in}}%
\pgfpathcurveto{\pgfqpoint{1.249841in}{2.563536in}}{\pgfqpoint{1.241941in}{2.560264in}}{\pgfqpoint{1.236117in}{2.554440in}}%
\pgfpathcurveto{\pgfqpoint{1.230293in}{2.548616in}}{\pgfqpoint{1.227021in}{2.540716in}}{\pgfqpoint{1.227021in}{2.532480in}}%
\pgfpathcurveto{\pgfqpoint{1.227021in}{2.524243in}}{\pgfqpoint{1.230293in}{2.516343in}}{\pgfqpoint{1.236117in}{2.510519in}}%
\pgfpathcurveto{\pgfqpoint{1.241941in}{2.504695in}}{\pgfqpoint{1.249841in}{2.501423in}}{\pgfqpoint{1.258077in}{2.501423in}}%
\pgfpathclose%
\pgfusepath{stroke,fill}%
\end{pgfscope}%
\begin{pgfscope}%
\pgfpathrectangle{\pgfqpoint{0.100000in}{0.212622in}}{\pgfqpoint{3.696000in}{3.696000in}}%
\pgfusepath{clip}%
\pgfsetbuttcap%
\pgfsetroundjoin%
\definecolor{currentfill}{rgb}{0.121569,0.466667,0.705882}%
\pgfsetfillcolor{currentfill}%
\pgfsetfillopacity{0.769358}%
\pgfsetlinewidth{1.003750pt}%
\definecolor{currentstroke}{rgb}{0.121569,0.466667,0.705882}%
\pgfsetstrokecolor{currentstroke}%
\pgfsetstrokeopacity{0.769358}%
\pgfsetdash{}{0pt}%
\pgfpathmoveto{\pgfqpoint{2.492773in}{2.838560in}}%
\pgfpathcurveto{\pgfqpoint{2.501010in}{2.838560in}}{\pgfqpoint{2.508910in}{2.841832in}}{\pgfqpoint{2.514734in}{2.847656in}}%
\pgfpathcurveto{\pgfqpoint{2.520557in}{2.853480in}}{\pgfqpoint{2.523830in}{2.861380in}}{\pgfqpoint{2.523830in}{2.869617in}}%
\pgfpathcurveto{\pgfqpoint{2.523830in}{2.877853in}}{\pgfqpoint{2.520557in}{2.885753in}}{\pgfqpoint{2.514734in}{2.891577in}}%
\pgfpathcurveto{\pgfqpoint{2.508910in}{2.897401in}}{\pgfqpoint{2.501010in}{2.900673in}}{\pgfqpoint{2.492773in}{2.900673in}}%
\pgfpathcurveto{\pgfqpoint{2.484537in}{2.900673in}}{\pgfqpoint{2.476637in}{2.897401in}}{\pgfqpoint{2.470813in}{2.891577in}}%
\pgfpathcurveto{\pgfqpoint{2.464989in}{2.885753in}}{\pgfqpoint{2.461717in}{2.877853in}}{\pgfqpoint{2.461717in}{2.869617in}}%
\pgfpathcurveto{\pgfqpoint{2.461717in}{2.861380in}}{\pgfqpoint{2.464989in}{2.853480in}}{\pgfqpoint{2.470813in}{2.847656in}}%
\pgfpathcurveto{\pgfqpoint{2.476637in}{2.841832in}}{\pgfqpoint{2.484537in}{2.838560in}}{\pgfqpoint{2.492773in}{2.838560in}}%
\pgfpathclose%
\pgfusepath{stroke,fill}%
\end{pgfscope}%
\begin{pgfscope}%
\pgfpathrectangle{\pgfqpoint{0.100000in}{0.212622in}}{\pgfqpoint{3.696000in}{3.696000in}}%
\pgfusepath{clip}%
\pgfsetbuttcap%
\pgfsetroundjoin%
\definecolor{currentfill}{rgb}{0.121569,0.466667,0.705882}%
\pgfsetfillcolor{currentfill}%
\pgfsetfillopacity{0.771020}%
\pgfsetlinewidth{1.003750pt}%
\definecolor{currentstroke}{rgb}{0.121569,0.466667,0.705882}%
\pgfsetstrokecolor{currentstroke}%
\pgfsetstrokeopacity{0.771020}%
\pgfsetdash{}{0pt}%
\pgfpathmoveto{\pgfqpoint{2.488757in}{2.833813in}}%
\pgfpathcurveto{\pgfqpoint{2.496993in}{2.833813in}}{\pgfqpoint{2.504893in}{2.837086in}}{\pgfqpoint{2.510717in}{2.842909in}}%
\pgfpathcurveto{\pgfqpoint{2.516541in}{2.848733in}}{\pgfqpoint{2.519814in}{2.856633in}}{\pgfqpoint{2.519814in}{2.864870in}}%
\pgfpathcurveto{\pgfqpoint{2.519814in}{2.873106in}}{\pgfqpoint{2.516541in}{2.881006in}}{\pgfqpoint{2.510717in}{2.886830in}}%
\pgfpathcurveto{\pgfqpoint{2.504893in}{2.892654in}}{\pgfqpoint{2.496993in}{2.895926in}}{\pgfqpoint{2.488757in}{2.895926in}}%
\pgfpathcurveto{\pgfqpoint{2.480521in}{2.895926in}}{\pgfqpoint{2.472621in}{2.892654in}}{\pgfqpoint{2.466797in}{2.886830in}}%
\pgfpathcurveto{\pgfqpoint{2.460973in}{2.881006in}}{\pgfqpoint{2.457701in}{2.873106in}}{\pgfqpoint{2.457701in}{2.864870in}}%
\pgfpathcurveto{\pgfqpoint{2.457701in}{2.856633in}}{\pgfqpoint{2.460973in}{2.848733in}}{\pgfqpoint{2.466797in}{2.842909in}}%
\pgfpathcurveto{\pgfqpoint{2.472621in}{2.837086in}}{\pgfqpoint{2.480521in}{2.833813in}}{\pgfqpoint{2.488757in}{2.833813in}}%
\pgfpathclose%
\pgfusepath{stroke,fill}%
\end{pgfscope}%
\begin{pgfscope}%
\pgfpathrectangle{\pgfqpoint{0.100000in}{0.212622in}}{\pgfqpoint{3.696000in}{3.696000in}}%
\pgfusepath{clip}%
\pgfsetbuttcap%
\pgfsetroundjoin%
\definecolor{currentfill}{rgb}{0.121569,0.466667,0.705882}%
\pgfsetfillcolor{currentfill}%
\pgfsetfillopacity{0.771109}%
\pgfsetlinewidth{1.003750pt}%
\definecolor{currentstroke}{rgb}{0.121569,0.466667,0.705882}%
\pgfsetstrokecolor{currentstroke}%
\pgfsetstrokeopacity{0.771109}%
\pgfsetdash{}{0pt}%
\pgfpathmoveto{\pgfqpoint{1.250360in}{2.492207in}}%
\pgfpathcurveto{\pgfqpoint{1.258596in}{2.492207in}}{\pgfqpoint{1.266496in}{2.495479in}}{\pgfqpoint{1.272320in}{2.501303in}}%
\pgfpathcurveto{\pgfqpoint{1.278144in}{2.507127in}}{\pgfqpoint{1.281417in}{2.515027in}}{\pgfqpoint{1.281417in}{2.523263in}}%
\pgfpathcurveto{\pgfqpoint{1.281417in}{2.531500in}}{\pgfqpoint{1.278144in}{2.539400in}}{\pgfqpoint{1.272320in}{2.545224in}}%
\pgfpathcurveto{\pgfqpoint{1.266496in}{2.551048in}}{\pgfqpoint{1.258596in}{2.554320in}}{\pgfqpoint{1.250360in}{2.554320in}}%
\pgfpathcurveto{\pgfqpoint{1.242124in}{2.554320in}}{\pgfqpoint{1.234224in}{2.551048in}}{\pgfqpoint{1.228400in}{2.545224in}}%
\pgfpathcurveto{\pgfqpoint{1.222576in}{2.539400in}}{\pgfqpoint{1.219304in}{2.531500in}}{\pgfqpoint{1.219304in}{2.523263in}}%
\pgfpathcurveto{\pgfqpoint{1.219304in}{2.515027in}}{\pgfqpoint{1.222576in}{2.507127in}}{\pgfqpoint{1.228400in}{2.501303in}}%
\pgfpathcurveto{\pgfqpoint{1.234224in}{2.495479in}}{\pgfqpoint{1.242124in}{2.492207in}}{\pgfqpoint{1.250360in}{2.492207in}}%
\pgfpathclose%
\pgfusepath{stroke,fill}%
\end{pgfscope}%
\begin{pgfscope}%
\pgfpathrectangle{\pgfqpoint{0.100000in}{0.212622in}}{\pgfqpoint{3.696000in}{3.696000in}}%
\pgfusepath{clip}%
\pgfsetbuttcap%
\pgfsetroundjoin%
\definecolor{currentfill}{rgb}{0.121569,0.466667,0.705882}%
\pgfsetfillcolor{currentfill}%
\pgfsetfillopacity{0.772639}%
\pgfsetlinewidth{1.003750pt}%
\definecolor{currentstroke}{rgb}{0.121569,0.466667,0.705882}%
\pgfsetstrokecolor{currentstroke}%
\pgfsetstrokeopacity{0.772639}%
\pgfsetdash{}{0pt}%
\pgfpathmoveto{\pgfqpoint{2.485006in}{2.829431in}}%
\pgfpathcurveto{\pgfqpoint{2.493242in}{2.829431in}}{\pgfqpoint{2.501142in}{2.832703in}}{\pgfqpoint{2.506966in}{2.838527in}}%
\pgfpathcurveto{\pgfqpoint{2.512790in}{2.844351in}}{\pgfqpoint{2.516063in}{2.852251in}}{\pgfqpoint{2.516063in}{2.860487in}}%
\pgfpathcurveto{\pgfqpoint{2.516063in}{2.868724in}}{\pgfqpoint{2.512790in}{2.876624in}}{\pgfqpoint{2.506966in}{2.882448in}}%
\pgfpathcurveto{\pgfqpoint{2.501142in}{2.888271in}}{\pgfqpoint{2.493242in}{2.891544in}}{\pgfqpoint{2.485006in}{2.891544in}}%
\pgfpathcurveto{\pgfqpoint{2.476770in}{2.891544in}}{\pgfqpoint{2.468870in}{2.888271in}}{\pgfqpoint{2.463046in}{2.882448in}}%
\pgfpathcurveto{\pgfqpoint{2.457222in}{2.876624in}}{\pgfqpoint{2.453950in}{2.868724in}}{\pgfqpoint{2.453950in}{2.860487in}}%
\pgfpathcurveto{\pgfqpoint{2.453950in}{2.852251in}}{\pgfqpoint{2.457222in}{2.844351in}}{\pgfqpoint{2.463046in}{2.838527in}}%
\pgfpathcurveto{\pgfqpoint{2.468870in}{2.832703in}}{\pgfqpoint{2.476770in}{2.829431in}}{\pgfqpoint{2.485006in}{2.829431in}}%
\pgfpathclose%
\pgfusepath{stroke,fill}%
\end{pgfscope}%
\begin{pgfscope}%
\pgfpathrectangle{\pgfqpoint{0.100000in}{0.212622in}}{\pgfqpoint{3.696000in}{3.696000in}}%
\pgfusepath{clip}%
\pgfsetbuttcap%
\pgfsetroundjoin%
\definecolor{currentfill}{rgb}{0.121569,0.466667,0.705882}%
\pgfsetfillcolor{currentfill}%
\pgfsetfillopacity{0.773347}%
\pgfsetlinewidth{1.003750pt}%
\definecolor{currentstroke}{rgb}{0.121569,0.466667,0.705882}%
\pgfsetstrokecolor{currentstroke}%
\pgfsetstrokeopacity{0.773347}%
\pgfsetdash{}{0pt}%
\pgfpathmoveto{\pgfqpoint{1.242445in}{2.481391in}}%
\pgfpathcurveto{\pgfqpoint{1.250681in}{2.481391in}}{\pgfqpoint{1.258581in}{2.484664in}}{\pgfqpoint{1.264405in}{2.490488in}}%
\pgfpathcurveto{\pgfqpoint{1.270229in}{2.496312in}}{\pgfqpoint{1.273501in}{2.504212in}}{\pgfqpoint{1.273501in}{2.512448in}}%
\pgfpathcurveto{\pgfqpoint{1.273501in}{2.520684in}}{\pgfqpoint{1.270229in}{2.528584in}}{\pgfqpoint{1.264405in}{2.534408in}}%
\pgfpathcurveto{\pgfqpoint{1.258581in}{2.540232in}}{\pgfqpoint{1.250681in}{2.543504in}}{\pgfqpoint{1.242445in}{2.543504in}}%
\pgfpathcurveto{\pgfqpoint{1.234208in}{2.543504in}}{\pgfqpoint{1.226308in}{2.540232in}}{\pgfqpoint{1.220484in}{2.534408in}}%
\pgfpathcurveto{\pgfqpoint{1.214660in}{2.528584in}}{\pgfqpoint{1.211388in}{2.520684in}}{\pgfqpoint{1.211388in}{2.512448in}}%
\pgfpathcurveto{\pgfqpoint{1.211388in}{2.504212in}}{\pgfqpoint{1.214660in}{2.496312in}}{\pgfqpoint{1.220484in}{2.490488in}}%
\pgfpathcurveto{\pgfqpoint{1.226308in}{2.484664in}}{\pgfqpoint{1.234208in}{2.481391in}}{\pgfqpoint{1.242445in}{2.481391in}}%
\pgfpathclose%
\pgfusepath{stroke,fill}%
\end{pgfscope}%
\begin{pgfscope}%
\pgfpathrectangle{\pgfqpoint{0.100000in}{0.212622in}}{\pgfqpoint{3.696000in}{3.696000in}}%
\pgfusepath{clip}%
\pgfsetbuttcap%
\pgfsetroundjoin%
\definecolor{currentfill}{rgb}{0.121569,0.466667,0.705882}%
\pgfsetfillcolor{currentfill}%
\pgfsetfillopacity{0.773722}%
\pgfsetlinewidth{1.003750pt}%
\definecolor{currentstroke}{rgb}{0.121569,0.466667,0.705882}%
\pgfsetstrokecolor{currentstroke}%
\pgfsetstrokeopacity{0.773722}%
\pgfsetdash{}{0pt}%
\pgfpathmoveto{\pgfqpoint{2.482368in}{2.826007in}}%
\pgfpathcurveto{\pgfqpoint{2.490605in}{2.826007in}}{\pgfqpoint{2.498505in}{2.829280in}}{\pgfqpoint{2.504329in}{2.835104in}}%
\pgfpathcurveto{\pgfqpoint{2.510153in}{2.840928in}}{\pgfqpoint{2.513425in}{2.848828in}}{\pgfqpoint{2.513425in}{2.857064in}}%
\pgfpathcurveto{\pgfqpoint{2.513425in}{2.865300in}}{\pgfqpoint{2.510153in}{2.873200in}}{\pgfqpoint{2.504329in}{2.879024in}}%
\pgfpathcurveto{\pgfqpoint{2.498505in}{2.884848in}}{\pgfqpoint{2.490605in}{2.888120in}}{\pgfqpoint{2.482368in}{2.888120in}}%
\pgfpathcurveto{\pgfqpoint{2.474132in}{2.888120in}}{\pgfqpoint{2.466232in}{2.884848in}}{\pgfqpoint{2.460408in}{2.879024in}}%
\pgfpathcurveto{\pgfqpoint{2.454584in}{2.873200in}}{\pgfqpoint{2.451312in}{2.865300in}}{\pgfqpoint{2.451312in}{2.857064in}}%
\pgfpathcurveto{\pgfqpoint{2.451312in}{2.848828in}}{\pgfqpoint{2.454584in}{2.840928in}}{\pgfqpoint{2.460408in}{2.835104in}}%
\pgfpathcurveto{\pgfqpoint{2.466232in}{2.829280in}}{\pgfqpoint{2.474132in}{2.826007in}}{\pgfqpoint{2.482368in}{2.826007in}}%
\pgfpathclose%
\pgfusepath{stroke,fill}%
\end{pgfscope}%
\begin{pgfscope}%
\pgfpathrectangle{\pgfqpoint{0.100000in}{0.212622in}}{\pgfqpoint{3.696000in}{3.696000in}}%
\pgfusepath{clip}%
\pgfsetbuttcap%
\pgfsetroundjoin%
\definecolor{currentfill}{rgb}{0.121569,0.466667,0.705882}%
\pgfsetfillcolor{currentfill}%
\pgfsetfillopacity{0.774500}%
\pgfsetlinewidth{1.003750pt}%
\definecolor{currentstroke}{rgb}{0.121569,0.466667,0.705882}%
\pgfsetstrokecolor{currentstroke}%
\pgfsetstrokeopacity{0.774500}%
\pgfsetdash{}{0pt}%
\pgfpathmoveto{\pgfqpoint{2.921379in}{1.602607in}}%
\pgfpathcurveto{\pgfqpoint{2.929616in}{1.602607in}}{\pgfqpoint{2.937516in}{1.605879in}}{\pgfqpoint{2.943340in}{1.611703in}}%
\pgfpathcurveto{\pgfqpoint{2.949164in}{1.617527in}}{\pgfqpoint{2.952436in}{1.625427in}}{\pgfqpoint{2.952436in}{1.633663in}}%
\pgfpathcurveto{\pgfqpoint{2.952436in}{1.641900in}}{\pgfqpoint{2.949164in}{1.649800in}}{\pgfqpoint{2.943340in}{1.655624in}}%
\pgfpathcurveto{\pgfqpoint{2.937516in}{1.661448in}}{\pgfqpoint{2.929616in}{1.664720in}}{\pgfqpoint{2.921379in}{1.664720in}}%
\pgfpathcurveto{\pgfqpoint{2.913143in}{1.664720in}}{\pgfqpoint{2.905243in}{1.661448in}}{\pgfqpoint{2.899419in}{1.655624in}}%
\pgfpathcurveto{\pgfqpoint{2.893595in}{1.649800in}}{\pgfqpoint{2.890323in}{1.641900in}}{\pgfqpoint{2.890323in}{1.633663in}}%
\pgfpathcurveto{\pgfqpoint{2.890323in}{1.625427in}}{\pgfqpoint{2.893595in}{1.617527in}}{\pgfqpoint{2.899419in}{1.611703in}}%
\pgfpathcurveto{\pgfqpoint{2.905243in}{1.605879in}}{\pgfqpoint{2.913143in}{1.602607in}}{\pgfqpoint{2.921379in}{1.602607in}}%
\pgfpathclose%
\pgfusepath{stroke,fill}%
\end{pgfscope}%
\begin{pgfscope}%
\pgfpathrectangle{\pgfqpoint{0.100000in}{0.212622in}}{\pgfqpoint{3.696000in}{3.696000in}}%
\pgfusepath{clip}%
\pgfsetbuttcap%
\pgfsetroundjoin%
\definecolor{currentfill}{rgb}{0.121569,0.466667,0.705882}%
\pgfsetfillcolor{currentfill}%
\pgfsetfillopacity{0.775669}%
\pgfsetlinewidth{1.003750pt}%
\definecolor{currentstroke}{rgb}{0.121569,0.466667,0.705882}%
\pgfsetstrokecolor{currentstroke}%
\pgfsetstrokeopacity{0.775669}%
\pgfsetdash{}{0pt}%
\pgfpathmoveto{\pgfqpoint{2.477510in}{2.819712in}}%
\pgfpathcurveto{\pgfqpoint{2.485746in}{2.819712in}}{\pgfqpoint{2.493646in}{2.822984in}}{\pgfqpoint{2.499470in}{2.828808in}}%
\pgfpathcurveto{\pgfqpoint{2.505294in}{2.834632in}}{\pgfqpoint{2.508566in}{2.842532in}}{\pgfqpoint{2.508566in}{2.850768in}}%
\pgfpathcurveto{\pgfqpoint{2.508566in}{2.859005in}}{\pgfqpoint{2.505294in}{2.866905in}}{\pgfqpoint{2.499470in}{2.872729in}}%
\pgfpathcurveto{\pgfqpoint{2.493646in}{2.878553in}}{\pgfqpoint{2.485746in}{2.881825in}}{\pgfqpoint{2.477510in}{2.881825in}}%
\pgfpathcurveto{\pgfqpoint{2.469273in}{2.881825in}}{\pgfqpoint{2.461373in}{2.878553in}}{\pgfqpoint{2.455549in}{2.872729in}}%
\pgfpathcurveto{\pgfqpoint{2.449725in}{2.866905in}}{\pgfqpoint{2.446453in}{2.859005in}}{\pgfqpoint{2.446453in}{2.850768in}}%
\pgfpathcurveto{\pgfqpoint{2.446453in}{2.842532in}}{\pgfqpoint{2.449725in}{2.834632in}}{\pgfqpoint{2.455549in}{2.828808in}}%
\pgfpathcurveto{\pgfqpoint{2.461373in}{2.822984in}}{\pgfqpoint{2.469273in}{2.819712in}}{\pgfqpoint{2.477510in}{2.819712in}}%
\pgfpathclose%
\pgfusepath{stroke,fill}%
\end{pgfscope}%
\begin{pgfscope}%
\pgfpathrectangle{\pgfqpoint{0.100000in}{0.212622in}}{\pgfqpoint{3.696000in}{3.696000in}}%
\pgfusepath{clip}%
\pgfsetbuttcap%
\pgfsetroundjoin%
\definecolor{currentfill}{rgb}{0.121569,0.466667,0.705882}%
\pgfsetfillcolor{currentfill}%
\pgfsetfillopacity{0.775762}%
\pgfsetlinewidth{1.003750pt}%
\definecolor{currentstroke}{rgb}{0.121569,0.466667,0.705882}%
\pgfsetstrokecolor{currentstroke}%
\pgfsetstrokeopacity{0.775762}%
\pgfsetdash{}{0pt}%
\pgfpathmoveto{\pgfqpoint{1.234418in}{2.470556in}}%
\pgfpathcurveto{\pgfqpoint{1.242654in}{2.470556in}}{\pgfqpoint{1.250554in}{2.473829in}}{\pgfqpoint{1.256378in}{2.479653in}}%
\pgfpathcurveto{\pgfqpoint{1.262202in}{2.485477in}}{\pgfqpoint{1.265475in}{2.493377in}}{\pgfqpoint{1.265475in}{2.501613in}}%
\pgfpathcurveto{\pgfqpoint{1.265475in}{2.509849in}}{\pgfqpoint{1.262202in}{2.517749in}}{\pgfqpoint{1.256378in}{2.523573in}}%
\pgfpathcurveto{\pgfqpoint{1.250554in}{2.529397in}}{\pgfqpoint{1.242654in}{2.532669in}}{\pgfqpoint{1.234418in}{2.532669in}}%
\pgfpathcurveto{\pgfqpoint{1.226182in}{2.532669in}}{\pgfqpoint{1.218282in}{2.529397in}}{\pgfqpoint{1.212458in}{2.523573in}}%
\pgfpathcurveto{\pgfqpoint{1.206634in}{2.517749in}}{\pgfqpoint{1.203362in}{2.509849in}}{\pgfqpoint{1.203362in}{2.501613in}}%
\pgfpathcurveto{\pgfqpoint{1.203362in}{2.493377in}}{\pgfqpoint{1.206634in}{2.485477in}}{\pgfqpoint{1.212458in}{2.479653in}}%
\pgfpathcurveto{\pgfqpoint{1.218282in}{2.473829in}}{\pgfqpoint{1.226182in}{2.470556in}}{\pgfqpoint{1.234418in}{2.470556in}}%
\pgfpathclose%
\pgfusepath{stroke,fill}%
\end{pgfscope}%
\begin{pgfscope}%
\pgfpathrectangle{\pgfqpoint{0.100000in}{0.212622in}}{\pgfqpoint{3.696000in}{3.696000in}}%
\pgfusepath{clip}%
\pgfsetbuttcap%
\pgfsetroundjoin%
\definecolor{currentfill}{rgb}{0.121569,0.466667,0.705882}%
\pgfsetfillcolor{currentfill}%
\pgfsetfillopacity{0.777092}%
\pgfsetlinewidth{1.003750pt}%
\definecolor{currentstroke}{rgb}{0.121569,0.466667,0.705882}%
\pgfsetstrokecolor{currentstroke}%
\pgfsetstrokeopacity{0.777092}%
\pgfsetdash{}{0pt}%
\pgfpathmoveto{\pgfqpoint{2.474059in}{2.815629in}}%
\pgfpathcurveto{\pgfqpoint{2.482295in}{2.815629in}}{\pgfqpoint{2.490195in}{2.818902in}}{\pgfqpoint{2.496019in}{2.824726in}}%
\pgfpathcurveto{\pgfqpoint{2.501843in}{2.830550in}}{\pgfqpoint{2.505115in}{2.838450in}}{\pgfqpoint{2.505115in}{2.846686in}}%
\pgfpathcurveto{\pgfqpoint{2.505115in}{2.854922in}}{\pgfqpoint{2.501843in}{2.862822in}}{\pgfqpoint{2.496019in}{2.868646in}}%
\pgfpathcurveto{\pgfqpoint{2.490195in}{2.874470in}}{\pgfqpoint{2.482295in}{2.877742in}}{\pgfqpoint{2.474059in}{2.877742in}}%
\pgfpathcurveto{\pgfqpoint{2.465822in}{2.877742in}}{\pgfqpoint{2.457922in}{2.874470in}}{\pgfqpoint{2.452098in}{2.868646in}}%
\pgfpathcurveto{\pgfqpoint{2.446274in}{2.862822in}}{\pgfqpoint{2.443002in}{2.854922in}}{\pgfqpoint{2.443002in}{2.846686in}}%
\pgfpathcurveto{\pgfqpoint{2.443002in}{2.838450in}}{\pgfqpoint{2.446274in}{2.830550in}}{\pgfqpoint{2.452098in}{2.824726in}}%
\pgfpathcurveto{\pgfqpoint{2.457922in}{2.818902in}}{\pgfqpoint{2.465822in}{2.815629in}}{\pgfqpoint{2.474059in}{2.815629in}}%
\pgfpathclose%
\pgfusepath{stroke,fill}%
\end{pgfscope}%
\begin{pgfscope}%
\pgfpathrectangle{\pgfqpoint{0.100000in}{0.212622in}}{\pgfqpoint{3.696000in}{3.696000in}}%
\pgfusepath{clip}%
\pgfsetbuttcap%
\pgfsetroundjoin%
\definecolor{currentfill}{rgb}{0.121569,0.466667,0.705882}%
\pgfsetfillcolor{currentfill}%
\pgfsetfillopacity{0.778467}%
\pgfsetlinewidth{1.003750pt}%
\definecolor{currentstroke}{rgb}{0.121569,0.466667,0.705882}%
\pgfsetstrokecolor{currentstroke}%
\pgfsetstrokeopacity{0.778467}%
\pgfsetdash{}{0pt}%
\pgfpathmoveto{\pgfqpoint{2.470922in}{2.811799in}}%
\pgfpathcurveto{\pgfqpoint{2.479158in}{2.811799in}}{\pgfqpoint{2.487058in}{2.815071in}}{\pgfqpoint{2.492882in}{2.820895in}}%
\pgfpathcurveto{\pgfqpoint{2.498706in}{2.826719in}}{\pgfqpoint{2.501978in}{2.834619in}}{\pgfqpoint{2.501978in}{2.842856in}}%
\pgfpathcurveto{\pgfqpoint{2.501978in}{2.851092in}}{\pgfqpoint{2.498706in}{2.858992in}}{\pgfqpoint{2.492882in}{2.864816in}}%
\pgfpathcurveto{\pgfqpoint{2.487058in}{2.870640in}}{\pgfqpoint{2.479158in}{2.873912in}}{\pgfqpoint{2.470922in}{2.873912in}}%
\pgfpathcurveto{\pgfqpoint{2.462685in}{2.873912in}}{\pgfqpoint{2.454785in}{2.870640in}}{\pgfqpoint{2.448961in}{2.864816in}}%
\pgfpathcurveto{\pgfqpoint{2.443137in}{2.858992in}}{\pgfqpoint{2.439865in}{2.851092in}}{\pgfqpoint{2.439865in}{2.842856in}}%
\pgfpathcurveto{\pgfqpoint{2.439865in}{2.834619in}}{\pgfqpoint{2.443137in}{2.826719in}}{\pgfqpoint{2.448961in}{2.820895in}}%
\pgfpathcurveto{\pgfqpoint{2.454785in}{2.815071in}}{\pgfqpoint{2.462685in}{2.811799in}}{\pgfqpoint{2.470922in}{2.811799in}}%
\pgfpathclose%
\pgfusepath{stroke,fill}%
\end{pgfscope}%
\begin{pgfscope}%
\pgfpathrectangle{\pgfqpoint{0.100000in}{0.212622in}}{\pgfqpoint{3.696000in}{3.696000in}}%
\pgfusepath{clip}%
\pgfsetbuttcap%
\pgfsetroundjoin%
\definecolor{currentfill}{rgb}{0.121569,0.466667,0.705882}%
\pgfsetfillcolor{currentfill}%
\pgfsetfillopacity{0.778475}%
\pgfsetlinewidth{1.003750pt}%
\definecolor{currentstroke}{rgb}{0.121569,0.466667,0.705882}%
\pgfsetstrokecolor{currentstroke}%
\pgfsetstrokeopacity{0.778475}%
\pgfsetdash{}{0pt}%
\pgfpathmoveto{\pgfqpoint{1.226056in}{2.459899in}}%
\pgfpathcurveto{\pgfqpoint{1.234293in}{2.459899in}}{\pgfqpoint{1.242193in}{2.463171in}}{\pgfqpoint{1.248017in}{2.468995in}}%
\pgfpathcurveto{\pgfqpoint{1.253841in}{2.474819in}}{\pgfqpoint{1.257113in}{2.482719in}}{\pgfqpoint{1.257113in}{2.490955in}}%
\pgfpathcurveto{\pgfqpoint{1.257113in}{2.499191in}}{\pgfqpoint{1.253841in}{2.507091in}}{\pgfqpoint{1.248017in}{2.512915in}}%
\pgfpathcurveto{\pgfqpoint{1.242193in}{2.518739in}}{\pgfqpoint{1.234293in}{2.522012in}}{\pgfqpoint{1.226056in}{2.522012in}}%
\pgfpathcurveto{\pgfqpoint{1.217820in}{2.522012in}}{\pgfqpoint{1.209920in}{2.518739in}}{\pgfqpoint{1.204096in}{2.512915in}}%
\pgfpathcurveto{\pgfqpoint{1.198272in}{2.507091in}}{\pgfqpoint{1.195000in}{2.499191in}}{\pgfqpoint{1.195000in}{2.490955in}}%
\pgfpathcurveto{\pgfqpoint{1.195000in}{2.482719in}}{\pgfqpoint{1.198272in}{2.474819in}}{\pgfqpoint{1.204096in}{2.468995in}}%
\pgfpathcurveto{\pgfqpoint{1.209920in}{2.463171in}}{\pgfqpoint{1.217820in}{2.459899in}}{\pgfqpoint{1.226056in}{2.459899in}}%
\pgfpathclose%
\pgfusepath{stroke,fill}%
\end{pgfscope}%
\begin{pgfscope}%
\pgfpathrectangle{\pgfqpoint{0.100000in}{0.212622in}}{\pgfqpoint{3.696000in}{3.696000in}}%
\pgfusepath{clip}%
\pgfsetbuttcap%
\pgfsetroundjoin%
\definecolor{currentfill}{rgb}{0.121569,0.466667,0.705882}%
\pgfsetfillcolor{currentfill}%
\pgfsetfillopacity{0.779329}%
\pgfsetlinewidth{1.003750pt}%
\definecolor{currentstroke}{rgb}{0.121569,0.466667,0.705882}%
\pgfsetstrokecolor{currentstroke}%
\pgfsetstrokeopacity{0.779329}%
\pgfsetdash{}{0pt}%
\pgfpathmoveto{\pgfqpoint{2.468878in}{2.809039in}}%
\pgfpathcurveto{\pgfqpoint{2.477114in}{2.809039in}}{\pgfqpoint{2.485014in}{2.812311in}}{\pgfqpoint{2.490838in}{2.818135in}}%
\pgfpathcurveto{\pgfqpoint{2.496662in}{2.823959in}}{\pgfqpoint{2.499934in}{2.831859in}}{\pgfqpoint{2.499934in}{2.840096in}}%
\pgfpathcurveto{\pgfqpoint{2.499934in}{2.848332in}}{\pgfqpoint{2.496662in}{2.856232in}}{\pgfqpoint{2.490838in}{2.862056in}}%
\pgfpathcurveto{\pgfqpoint{2.485014in}{2.867880in}}{\pgfqpoint{2.477114in}{2.871152in}}{\pgfqpoint{2.468878in}{2.871152in}}%
\pgfpathcurveto{\pgfqpoint{2.460641in}{2.871152in}}{\pgfqpoint{2.452741in}{2.867880in}}{\pgfqpoint{2.446917in}{2.862056in}}%
\pgfpathcurveto{\pgfqpoint{2.441093in}{2.856232in}}{\pgfqpoint{2.437821in}{2.848332in}}{\pgfqpoint{2.437821in}{2.840096in}}%
\pgfpathcurveto{\pgfqpoint{2.437821in}{2.831859in}}{\pgfqpoint{2.441093in}{2.823959in}}{\pgfqpoint{2.446917in}{2.818135in}}%
\pgfpathcurveto{\pgfqpoint{2.452741in}{2.812311in}}{\pgfqpoint{2.460641in}{2.809039in}}{\pgfqpoint{2.468878in}{2.809039in}}%
\pgfpathclose%
\pgfusepath{stroke,fill}%
\end{pgfscope}%
\begin{pgfscope}%
\pgfpathrectangle{\pgfqpoint{0.100000in}{0.212622in}}{\pgfqpoint{3.696000in}{3.696000in}}%
\pgfusepath{clip}%
\pgfsetbuttcap%
\pgfsetroundjoin%
\definecolor{currentfill}{rgb}{0.121569,0.466667,0.705882}%
\pgfsetfillcolor{currentfill}%
\pgfsetfillopacity{0.780078}%
\pgfsetlinewidth{1.003750pt}%
\definecolor{currentstroke}{rgb}{0.121569,0.466667,0.705882}%
\pgfsetstrokecolor{currentstroke}%
\pgfsetstrokeopacity{0.780078}%
\pgfsetdash{}{0pt}%
\pgfpathmoveto{\pgfqpoint{2.911178in}{1.589828in}}%
\pgfpathcurveto{\pgfqpoint{2.919414in}{1.589828in}}{\pgfqpoint{2.927314in}{1.593100in}}{\pgfqpoint{2.933138in}{1.598924in}}%
\pgfpathcurveto{\pgfqpoint{2.938962in}{1.604748in}}{\pgfqpoint{2.942234in}{1.612648in}}{\pgfqpoint{2.942234in}{1.620884in}}%
\pgfpathcurveto{\pgfqpoint{2.942234in}{1.629120in}}{\pgfqpoint{2.938962in}{1.637021in}}{\pgfqpoint{2.933138in}{1.642844in}}%
\pgfpathcurveto{\pgfqpoint{2.927314in}{1.648668in}}{\pgfqpoint{2.919414in}{1.651941in}}{\pgfqpoint{2.911178in}{1.651941in}}%
\pgfpathcurveto{\pgfqpoint{2.902942in}{1.651941in}}{\pgfqpoint{2.895042in}{1.648668in}}{\pgfqpoint{2.889218in}{1.642844in}}%
\pgfpathcurveto{\pgfqpoint{2.883394in}{1.637021in}}{\pgfqpoint{2.880121in}{1.629120in}}{\pgfqpoint{2.880121in}{1.620884in}}%
\pgfpathcurveto{\pgfqpoint{2.880121in}{1.612648in}}{\pgfqpoint{2.883394in}{1.604748in}}{\pgfqpoint{2.889218in}{1.598924in}}%
\pgfpathcurveto{\pgfqpoint{2.895042in}{1.593100in}}{\pgfqpoint{2.902942in}{1.589828in}}{\pgfqpoint{2.911178in}{1.589828in}}%
\pgfpathclose%
\pgfusepath{stroke,fill}%
\end{pgfscope}%
\begin{pgfscope}%
\pgfpathrectangle{\pgfqpoint{0.100000in}{0.212622in}}{\pgfqpoint{3.696000in}{3.696000in}}%
\pgfusepath{clip}%
\pgfsetbuttcap%
\pgfsetroundjoin%
\definecolor{currentfill}{rgb}{0.121569,0.466667,0.705882}%
\pgfsetfillcolor{currentfill}%
\pgfsetfillopacity{0.780898}%
\pgfsetlinewidth{1.003750pt}%
\definecolor{currentstroke}{rgb}{0.121569,0.466667,0.705882}%
\pgfsetstrokecolor{currentstroke}%
\pgfsetstrokeopacity{0.780898}%
\pgfsetdash{}{0pt}%
\pgfpathmoveto{\pgfqpoint{2.465109in}{2.804074in}}%
\pgfpathcurveto{\pgfqpoint{2.473345in}{2.804074in}}{\pgfqpoint{2.481245in}{2.807346in}}{\pgfqpoint{2.487069in}{2.813170in}}%
\pgfpathcurveto{\pgfqpoint{2.492893in}{2.818994in}}{\pgfqpoint{2.496165in}{2.826894in}}{\pgfqpoint{2.496165in}{2.835131in}}%
\pgfpathcurveto{\pgfqpoint{2.496165in}{2.843367in}}{\pgfqpoint{2.492893in}{2.851267in}}{\pgfqpoint{2.487069in}{2.857091in}}%
\pgfpathcurveto{\pgfqpoint{2.481245in}{2.862915in}}{\pgfqpoint{2.473345in}{2.866187in}}{\pgfqpoint{2.465109in}{2.866187in}}%
\pgfpathcurveto{\pgfqpoint{2.456872in}{2.866187in}}{\pgfqpoint{2.448972in}{2.862915in}}{\pgfqpoint{2.443149in}{2.857091in}}%
\pgfpathcurveto{\pgfqpoint{2.437325in}{2.851267in}}{\pgfqpoint{2.434052in}{2.843367in}}{\pgfqpoint{2.434052in}{2.835131in}}%
\pgfpathcurveto{\pgfqpoint{2.434052in}{2.826894in}}{\pgfqpoint{2.437325in}{2.818994in}}{\pgfqpoint{2.443149in}{2.813170in}}%
\pgfpathcurveto{\pgfqpoint{2.448972in}{2.807346in}}{\pgfqpoint{2.456872in}{2.804074in}}{\pgfqpoint{2.465109in}{2.804074in}}%
\pgfpathclose%
\pgfusepath{stroke,fill}%
\end{pgfscope}%
\begin{pgfscope}%
\pgfpathrectangle{\pgfqpoint{0.100000in}{0.212622in}}{\pgfqpoint{3.696000in}{3.696000in}}%
\pgfusepath{clip}%
\pgfsetbuttcap%
\pgfsetroundjoin%
\definecolor{currentfill}{rgb}{0.121569,0.466667,0.705882}%
\pgfsetfillcolor{currentfill}%
\pgfsetfillopacity{0.781535}%
\pgfsetlinewidth{1.003750pt}%
\definecolor{currentstroke}{rgb}{0.121569,0.466667,0.705882}%
\pgfsetstrokecolor{currentstroke}%
\pgfsetstrokeopacity{0.781535}%
\pgfsetdash{}{0pt}%
\pgfpathmoveto{\pgfqpoint{1.217390in}{2.450615in}}%
\pgfpathcurveto{\pgfqpoint{1.225626in}{2.450615in}}{\pgfqpoint{1.233526in}{2.453888in}}{\pgfqpoint{1.239350in}{2.459712in}}%
\pgfpathcurveto{\pgfqpoint{1.245174in}{2.465535in}}{\pgfqpoint{1.248446in}{2.473436in}}{\pgfqpoint{1.248446in}{2.481672in}}%
\pgfpathcurveto{\pgfqpoint{1.248446in}{2.489908in}}{\pgfqpoint{1.245174in}{2.497808in}}{\pgfqpoint{1.239350in}{2.503632in}}%
\pgfpathcurveto{\pgfqpoint{1.233526in}{2.509456in}}{\pgfqpoint{1.225626in}{2.512728in}}{\pgfqpoint{1.217390in}{2.512728in}}%
\pgfpathcurveto{\pgfqpoint{1.209153in}{2.512728in}}{\pgfqpoint{1.201253in}{2.509456in}}{\pgfqpoint{1.195429in}{2.503632in}}%
\pgfpathcurveto{\pgfqpoint{1.189606in}{2.497808in}}{\pgfqpoint{1.186333in}{2.489908in}}{\pgfqpoint{1.186333in}{2.481672in}}%
\pgfpathcurveto{\pgfqpoint{1.186333in}{2.473436in}}{\pgfqpoint{1.189606in}{2.465535in}}{\pgfqpoint{1.195429in}{2.459712in}}%
\pgfpathcurveto{\pgfqpoint{1.201253in}{2.453888in}}{\pgfqpoint{1.209153in}{2.450615in}}{\pgfqpoint{1.217390in}{2.450615in}}%
\pgfpathclose%
\pgfusepath{stroke,fill}%
\end{pgfscope}%
\begin{pgfscope}%
\pgfpathrectangle{\pgfqpoint{0.100000in}{0.212622in}}{\pgfqpoint{3.696000in}{3.696000in}}%
\pgfusepath{clip}%
\pgfsetbuttcap%
\pgfsetroundjoin%
\definecolor{currentfill}{rgb}{0.121569,0.466667,0.705882}%
\pgfsetfillcolor{currentfill}%
\pgfsetfillopacity{0.781974}%
\pgfsetlinewidth{1.003750pt}%
\definecolor{currentstroke}{rgb}{0.121569,0.466667,0.705882}%
\pgfsetstrokecolor{currentstroke}%
\pgfsetstrokeopacity{0.781974}%
\pgfsetdash{}{0pt}%
\pgfpathmoveto{\pgfqpoint{2.462641in}{2.801130in}}%
\pgfpathcurveto{\pgfqpoint{2.470877in}{2.801130in}}{\pgfqpoint{2.478777in}{2.804402in}}{\pgfqpoint{2.484601in}{2.810226in}}%
\pgfpathcurveto{\pgfqpoint{2.490425in}{2.816050in}}{\pgfqpoint{2.493697in}{2.823950in}}{\pgfqpoint{2.493697in}{2.832186in}}%
\pgfpathcurveto{\pgfqpoint{2.493697in}{2.840422in}}{\pgfqpoint{2.490425in}{2.848323in}}{\pgfqpoint{2.484601in}{2.854146in}}%
\pgfpathcurveto{\pgfqpoint{2.478777in}{2.859970in}}{\pgfqpoint{2.470877in}{2.863243in}}{\pgfqpoint{2.462641in}{2.863243in}}%
\pgfpathcurveto{\pgfqpoint{2.454405in}{2.863243in}}{\pgfqpoint{2.446505in}{2.859970in}}{\pgfqpoint{2.440681in}{2.854146in}}%
\pgfpathcurveto{\pgfqpoint{2.434857in}{2.848323in}}{\pgfqpoint{2.431584in}{2.840422in}}{\pgfqpoint{2.431584in}{2.832186in}}%
\pgfpathcurveto{\pgfqpoint{2.431584in}{2.823950in}}{\pgfqpoint{2.434857in}{2.816050in}}{\pgfqpoint{2.440681in}{2.810226in}}%
\pgfpathcurveto{\pgfqpoint{2.446505in}{2.804402in}}{\pgfqpoint{2.454405in}{2.801130in}}{\pgfqpoint{2.462641in}{2.801130in}}%
\pgfpathclose%
\pgfusepath{stroke,fill}%
\end{pgfscope}%
\begin{pgfscope}%
\pgfpathrectangle{\pgfqpoint{0.100000in}{0.212622in}}{\pgfqpoint{3.696000in}{3.696000in}}%
\pgfusepath{clip}%
\pgfsetbuttcap%
\pgfsetroundjoin%
\definecolor{currentfill}{rgb}{0.121569,0.466667,0.705882}%
\pgfsetfillcolor{currentfill}%
\pgfsetfillopacity{0.782964}%
\pgfsetlinewidth{1.003750pt}%
\definecolor{currentstroke}{rgb}{0.121569,0.466667,0.705882}%
\pgfsetstrokecolor{currentstroke}%
\pgfsetstrokeopacity{0.782964}%
\pgfsetdash{}{0pt}%
\pgfpathmoveto{\pgfqpoint{2.460459in}{2.798382in}}%
\pgfpathcurveto{\pgfqpoint{2.468696in}{2.798382in}}{\pgfqpoint{2.476596in}{2.801654in}}{\pgfqpoint{2.482420in}{2.807478in}}%
\pgfpathcurveto{\pgfqpoint{2.488244in}{2.813302in}}{\pgfqpoint{2.491516in}{2.821202in}}{\pgfqpoint{2.491516in}{2.829439in}}%
\pgfpathcurveto{\pgfqpoint{2.491516in}{2.837675in}}{\pgfqpoint{2.488244in}{2.845575in}}{\pgfqpoint{2.482420in}{2.851399in}}%
\pgfpathcurveto{\pgfqpoint{2.476596in}{2.857223in}}{\pgfqpoint{2.468696in}{2.860495in}}{\pgfqpoint{2.460459in}{2.860495in}}%
\pgfpathcurveto{\pgfqpoint{2.452223in}{2.860495in}}{\pgfqpoint{2.444323in}{2.857223in}}{\pgfqpoint{2.438499in}{2.851399in}}%
\pgfpathcurveto{\pgfqpoint{2.432675in}{2.845575in}}{\pgfqpoint{2.429403in}{2.837675in}}{\pgfqpoint{2.429403in}{2.829439in}}%
\pgfpathcurveto{\pgfqpoint{2.429403in}{2.821202in}}{\pgfqpoint{2.432675in}{2.813302in}}{\pgfqpoint{2.438499in}{2.807478in}}%
\pgfpathcurveto{\pgfqpoint{2.444323in}{2.801654in}}{\pgfqpoint{2.452223in}{2.798382in}}{\pgfqpoint{2.460459in}{2.798382in}}%
\pgfpathclose%
\pgfusepath{stroke,fill}%
\end{pgfscope}%
\begin{pgfscope}%
\pgfpathrectangle{\pgfqpoint{0.100000in}{0.212622in}}{\pgfqpoint{3.696000in}{3.696000in}}%
\pgfusepath{clip}%
\pgfsetbuttcap%
\pgfsetroundjoin%
\definecolor{currentfill}{rgb}{0.121569,0.466667,0.705882}%
\pgfsetfillcolor{currentfill}%
\pgfsetfillopacity{0.783669}%
\pgfsetlinewidth{1.003750pt}%
\definecolor{currentstroke}{rgb}{0.121569,0.466667,0.705882}%
\pgfsetstrokecolor{currentstroke}%
\pgfsetstrokeopacity{0.783669}%
\pgfsetdash{}{0pt}%
\pgfpathmoveto{\pgfqpoint{2.458827in}{2.796121in}}%
\pgfpathcurveto{\pgfqpoint{2.467064in}{2.796121in}}{\pgfqpoint{2.474964in}{2.799393in}}{\pgfqpoint{2.480788in}{2.805217in}}%
\pgfpathcurveto{\pgfqpoint{2.486612in}{2.811041in}}{\pgfqpoint{2.489884in}{2.818941in}}{\pgfqpoint{2.489884in}{2.827177in}}%
\pgfpathcurveto{\pgfqpoint{2.489884in}{2.835414in}}{\pgfqpoint{2.486612in}{2.843314in}}{\pgfqpoint{2.480788in}{2.849138in}}%
\pgfpathcurveto{\pgfqpoint{2.474964in}{2.854962in}}{\pgfqpoint{2.467064in}{2.858234in}}{\pgfqpoint{2.458827in}{2.858234in}}%
\pgfpathcurveto{\pgfqpoint{2.450591in}{2.858234in}}{\pgfqpoint{2.442691in}{2.854962in}}{\pgfqpoint{2.436867in}{2.849138in}}%
\pgfpathcurveto{\pgfqpoint{2.431043in}{2.843314in}}{\pgfqpoint{2.427771in}{2.835414in}}{\pgfqpoint{2.427771in}{2.827177in}}%
\pgfpathcurveto{\pgfqpoint{2.427771in}{2.818941in}}{\pgfqpoint{2.431043in}{2.811041in}}{\pgfqpoint{2.436867in}{2.805217in}}%
\pgfpathcurveto{\pgfqpoint{2.442691in}{2.799393in}}{\pgfqpoint{2.450591in}{2.796121in}}{\pgfqpoint{2.458827in}{2.796121in}}%
\pgfpathclose%
\pgfusepath{stroke,fill}%
\end{pgfscope}%
\begin{pgfscope}%
\pgfpathrectangle{\pgfqpoint{0.100000in}{0.212622in}}{\pgfqpoint{3.696000in}{3.696000in}}%
\pgfusepath{clip}%
\pgfsetbuttcap%
\pgfsetroundjoin%
\definecolor{currentfill}{rgb}{0.121569,0.466667,0.705882}%
\pgfsetfillcolor{currentfill}%
\pgfsetfillopacity{0.784751}%
\pgfsetlinewidth{1.003750pt}%
\definecolor{currentstroke}{rgb}{0.121569,0.466667,0.705882}%
\pgfsetstrokecolor{currentstroke}%
\pgfsetstrokeopacity{0.784751}%
\pgfsetdash{}{0pt}%
\pgfpathmoveto{\pgfqpoint{1.207713in}{2.440030in}}%
\pgfpathcurveto{\pgfqpoint{1.215950in}{2.440030in}}{\pgfqpoint{1.223850in}{2.443302in}}{\pgfqpoint{1.229674in}{2.449126in}}%
\pgfpathcurveto{\pgfqpoint{1.235497in}{2.454950in}}{\pgfqpoint{1.238770in}{2.462850in}}{\pgfqpoint{1.238770in}{2.471086in}}%
\pgfpathcurveto{\pgfqpoint{1.238770in}{2.479323in}}{\pgfqpoint{1.235497in}{2.487223in}}{\pgfqpoint{1.229674in}{2.493047in}}%
\pgfpathcurveto{\pgfqpoint{1.223850in}{2.498871in}}{\pgfqpoint{1.215950in}{2.502143in}}{\pgfqpoint{1.207713in}{2.502143in}}%
\pgfpathcurveto{\pgfqpoint{1.199477in}{2.502143in}}{\pgfqpoint{1.191577in}{2.498871in}}{\pgfqpoint{1.185753in}{2.493047in}}%
\pgfpathcurveto{\pgfqpoint{1.179929in}{2.487223in}}{\pgfqpoint{1.176657in}{2.479323in}}{\pgfqpoint{1.176657in}{2.471086in}}%
\pgfpathcurveto{\pgfqpoint{1.176657in}{2.462850in}}{\pgfqpoint{1.179929in}{2.454950in}}{\pgfqpoint{1.185753in}{2.449126in}}%
\pgfpathcurveto{\pgfqpoint{1.191577in}{2.443302in}}{\pgfqpoint{1.199477in}{2.440030in}}{\pgfqpoint{1.207713in}{2.440030in}}%
\pgfpathclose%
\pgfusepath{stroke,fill}%
\end{pgfscope}%
\begin{pgfscope}%
\pgfpathrectangle{\pgfqpoint{0.100000in}{0.212622in}}{\pgfqpoint{3.696000in}{3.696000in}}%
\pgfusepath{clip}%
\pgfsetbuttcap%
\pgfsetroundjoin%
\definecolor{currentfill}{rgb}{0.121569,0.466667,0.705882}%
\pgfsetfillcolor{currentfill}%
\pgfsetfillopacity{0.784961}%
\pgfsetlinewidth{1.003750pt}%
\definecolor{currentstroke}{rgb}{0.121569,0.466667,0.705882}%
\pgfsetstrokecolor{currentstroke}%
\pgfsetstrokeopacity{0.784961}%
\pgfsetdash{}{0pt}%
\pgfpathmoveto{\pgfqpoint{2.455800in}{2.792116in}}%
\pgfpathcurveto{\pgfqpoint{2.464036in}{2.792116in}}{\pgfqpoint{2.471936in}{2.795388in}}{\pgfqpoint{2.477760in}{2.801212in}}%
\pgfpathcurveto{\pgfqpoint{2.483584in}{2.807036in}}{\pgfqpoint{2.486856in}{2.814936in}}{\pgfqpoint{2.486856in}{2.823173in}}%
\pgfpathcurveto{\pgfqpoint{2.486856in}{2.831409in}}{\pgfqpoint{2.483584in}{2.839309in}}{\pgfqpoint{2.477760in}{2.845133in}}%
\pgfpathcurveto{\pgfqpoint{2.471936in}{2.850957in}}{\pgfqpoint{2.464036in}{2.854229in}}{\pgfqpoint{2.455800in}{2.854229in}}%
\pgfpathcurveto{\pgfqpoint{2.447564in}{2.854229in}}{\pgfqpoint{2.439664in}{2.850957in}}{\pgfqpoint{2.433840in}{2.845133in}}%
\pgfpathcurveto{\pgfqpoint{2.428016in}{2.839309in}}{\pgfqpoint{2.424743in}{2.831409in}}{\pgfqpoint{2.424743in}{2.823173in}}%
\pgfpathcurveto{\pgfqpoint{2.424743in}{2.814936in}}{\pgfqpoint{2.428016in}{2.807036in}}{\pgfqpoint{2.433840in}{2.801212in}}%
\pgfpathcurveto{\pgfqpoint{2.439664in}{2.795388in}}{\pgfqpoint{2.447564in}{2.792116in}}{\pgfqpoint{2.455800in}{2.792116in}}%
\pgfpathclose%
\pgfusepath{stroke,fill}%
\end{pgfscope}%
\begin{pgfscope}%
\pgfpathrectangle{\pgfqpoint{0.100000in}{0.212622in}}{\pgfqpoint{3.696000in}{3.696000in}}%
\pgfusepath{clip}%
\pgfsetbuttcap%
\pgfsetroundjoin%
\definecolor{currentfill}{rgb}{0.121569,0.466667,0.705882}%
\pgfsetfillcolor{currentfill}%
\pgfsetfillopacity{0.785420}%
\pgfsetlinewidth{1.003750pt}%
\definecolor{currentstroke}{rgb}{0.121569,0.466667,0.705882}%
\pgfsetstrokecolor{currentstroke}%
\pgfsetstrokeopacity{0.785420}%
\pgfsetdash{}{0pt}%
\pgfpathmoveto{\pgfqpoint{2.901057in}{1.577313in}}%
\pgfpathcurveto{\pgfqpoint{2.909293in}{1.577313in}}{\pgfqpoint{2.917193in}{1.580586in}}{\pgfqpoint{2.923017in}{1.586410in}}%
\pgfpathcurveto{\pgfqpoint{2.928841in}{1.592234in}}{\pgfqpoint{2.932114in}{1.600134in}}{\pgfqpoint{2.932114in}{1.608370in}}%
\pgfpathcurveto{\pgfqpoint{2.932114in}{1.616606in}}{\pgfqpoint{2.928841in}{1.624506in}}{\pgfqpoint{2.923017in}{1.630330in}}%
\pgfpathcurveto{\pgfqpoint{2.917193in}{1.636154in}}{\pgfqpoint{2.909293in}{1.639426in}}{\pgfqpoint{2.901057in}{1.639426in}}%
\pgfpathcurveto{\pgfqpoint{2.892821in}{1.639426in}}{\pgfqpoint{2.884921in}{1.636154in}}{\pgfqpoint{2.879097in}{1.630330in}}%
\pgfpathcurveto{\pgfqpoint{2.873273in}{1.624506in}}{\pgfqpoint{2.870001in}{1.616606in}}{\pgfqpoint{2.870001in}{1.608370in}}%
\pgfpathcurveto{\pgfqpoint{2.870001in}{1.600134in}}{\pgfqpoint{2.873273in}{1.592234in}}{\pgfqpoint{2.879097in}{1.586410in}}%
\pgfpathcurveto{\pgfqpoint{2.884921in}{1.580586in}}{\pgfqpoint{2.892821in}{1.577313in}}{\pgfqpoint{2.901057in}{1.577313in}}%
\pgfpathclose%
\pgfusepath{stroke,fill}%
\end{pgfscope}%
\begin{pgfscope}%
\pgfpathrectangle{\pgfqpoint{0.100000in}{0.212622in}}{\pgfqpoint{3.696000in}{3.696000in}}%
\pgfusepath{clip}%
\pgfsetbuttcap%
\pgfsetroundjoin%
\definecolor{currentfill}{rgb}{0.121569,0.466667,0.705882}%
\pgfsetfillcolor{currentfill}%
\pgfsetfillopacity{0.785729}%
\pgfsetlinewidth{1.003750pt}%
\definecolor{currentstroke}{rgb}{0.121569,0.466667,0.705882}%
\pgfsetstrokecolor{currentstroke}%
\pgfsetstrokeopacity{0.785729}%
\pgfsetdash{}{0pt}%
\pgfpathmoveto{\pgfqpoint{2.454075in}{2.790093in}}%
\pgfpathcurveto{\pgfqpoint{2.462311in}{2.790093in}}{\pgfqpoint{2.470211in}{2.793365in}}{\pgfqpoint{2.476035in}{2.799189in}}%
\pgfpathcurveto{\pgfqpoint{2.481859in}{2.805013in}}{\pgfqpoint{2.485131in}{2.812913in}}{\pgfqpoint{2.485131in}{2.821149in}}%
\pgfpathcurveto{\pgfqpoint{2.485131in}{2.829385in}}{\pgfqpoint{2.481859in}{2.837285in}}{\pgfqpoint{2.476035in}{2.843109in}}%
\pgfpathcurveto{\pgfqpoint{2.470211in}{2.848933in}}{\pgfqpoint{2.462311in}{2.852206in}}{\pgfqpoint{2.454075in}{2.852206in}}%
\pgfpathcurveto{\pgfqpoint{2.445838in}{2.852206in}}{\pgfqpoint{2.437938in}{2.848933in}}{\pgfqpoint{2.432115in}{2.843109in}}%
\pgfpathcurveto{\pgfqpoint{2.426291in}{2.837285in}}{\pgfqpoint{2.423018in}{2.829385in}}{\pgfqpoint{2.423018in}{2.821149in}}%
\pgfpathcurveto{\pgfqpoint{2.423018in}{2.812913in}}{\pgfqpoint{2.426291in}{2.805013in}}{\pgfqpoint{2.432115in}{2.799189in}}%
\pgfpathcurveto{\pgfqpoint{2.437938in}{2.793365in}}{\pgfqpoint{2.445838in}{2.790093in}}{\pgfqpoint{2.454075in}{2.790093in}}%
\pgfpathclose%
\pgfusepath{stroke,fill}%
\end{pgfscope}%
\begin{pgfscope}%
\pgfpathrectangle{\pgfqpoint{0.100000in}{0.212622in}}{\pgfqpoint{3.696000in}{3.696000in}}%
\pgfusepath{clip}%
\pgfsetbuttcap%
\pgfsetroundjoin%
\definecolor{currentfill}{rgb}{0.121569,0.466667,0.705882}%
\pgfsetfillcolor{currentfill}%
\pgfsetfillopacity{0.787152}%
\pgfsetlinewidth{1.003750pt}%
\definecolor{currentstroke}{rgb}{0.121569,0.466667,0.705882}%
\pgfsetstrokecolor{currentstroke}%
\pgfsetstrokeopacity{0.787152}%
\pgfsetdash{}{0pt}%
\pgfpathmoveto{\pgfqpoint{2.450988in}{2.786510in}}%
\pgfpathcurveto{\pgfqpoint{2.459224in}{2.786510in}}{\pgfqpoint{2.467124in}{2.789783in}}{\pgfqpoint{2.472948in}{2.795607in}}%
\pgfpathcurveto{\pgfqpoint{2.478772in}{2.801430in}}{\pgfqpoint{2.482044in}{2.809331in}}{\pgfqpoint{2.482044in}{2.817567in}}%
\pgfpathcurveto{\pgfqpoint{2.482044in}{2.825803in}}{\pgfqpoint{2.478772in}{2.833703in}}{\pgfqpoint{2.472948in}{2.839527in}}%
\pgfpathcurveto{\pgfqpoint{2.467124in}{2.845351in}}{\pgfqpoint{2.459224in}{2.848623in}}{\pgfqpoint{2.450988in}{2.848623in}}%
\pgfpathcurveto{\pgfqpoint{2.442752in}{2.848623in}}{\pgfqpoint{2.434851in}{2.845351in}}{\pgfqpoint{2.429028in}{2.839527in}}%
\pgfpathcurveto{\pgfqpoint{2.423204in}{2.833703in}}{\pgfqpoint{2.419931in}{2.825803in}}{\pgfqpoint{2.419931in}{2.817567in}}%
\pgfpathcurveto{\pgfqpoint{2.419931in}{2.809331in}}{\pgfqpoint{2.423204in}{2.801430in}}{\pgfqpoint{2.429028in}{2.795607in}}%
\pgfpathcurveto{\pgfqpoint{2.434851in}{2.789783in}}{\pgfqpoint{2.442752in}{2.786510in}}{\pgfqpoint{2.450988in}{2.786510in}}%
\pgfpathclose%
\pgfusepath{stroke,fill}%
\end{pgfscope}%
\begin{pgfscope}%
\pgfpathrectangle{\pgfqpoint{0.100000in}{0.212622in}}{\pgfqpoint{3.696000in}{3.696000in}}%
\pgfusepath{clip}%
\pgfsetbuttcap%
\pgfsetroundjoin%
\definecolor{currentfill}{rgb}{0.121569,0.466667,0.705882}%
\pgfsetfillcolor{currentfill}%
\pgfsetfillopacity{0.788052}%
\pgfsetlinewidth{1.003750pt}%
\definecolor{currentstroke}{rgb}{0.121569,0.466667,0.705882}%
\pgfsetstrokecolor{currentstroke}%
\pgfsetstrokeopacity{0.788052}%
\pgfsetdash{}{0pt}%
\pgfpathmoveto{\pgfqpoint{2.448962in}{2.783860in}}%
\pgfpathcurveto{\pgfqpoint{2.457198in}{2.783860in}}{\pgfqpoint{2.465098in}{2.787132in}}{\pgfqpoint{2.470922in}{2.792956in}}%
\pgfpathcurveto{\pgfqpoint{2.476746in}{2.798780in}}{\pgfqpoint{2.480018in}{2.806680in}}{\pgfqpoint{2.480018in}{2.814916in}}%
\pgfpathcurveto{\pgfqpoint{2.480018in}{2.823152in}}{\pgfqpoint{2.476746in}{2.831052in}}{\pgfqpoint{2.470922in}{2.836876in}}%
\pgfpathcurveto{\pgfqpoint{2.465098in}{2.842700in}}{\pgfqpoint{2.457198in}{2.845973in}}{\pgfqpoint{2.448962in}{2.845973in}}%
\pgfpathcurveto{\pgfqpoint{2.440726in}{2.845973in}}{\pgfqpoint{2.432826in}{2.842700in}}{\pgfqpoint{2.427002in}{2.836876in}}%
\pgfpathcurveto{\pgfqpoint{2.421178in}{2.831052in}}{\pgfqpoint{2.417905in}{2.823152in}}{\pgfqpoint{2.417905in}{2.814916in}}%
\pgfpathcurveto{\pgfqpoint{2.417905in}{2.806680in}}{\pgfqpoint{2.421178in}{2.798780in}}{\pgfqpoint{2.427002in}{2.792956in}}%
\pgfpathcurveto{\pgfqpoint{2.432826in}{2.787132in}}{\pgfqpoint{2.440726in}{2.783860in}}{\pgfqpoint{2.448962in}{2.783860in}}%
\pgfpathclose%
\pgfusepath{stroke,fill}%
\end{pgfscope}%
\begin{pgfscope}%
\pgfpathrectangle{\pgfqpoint{0.100000in}{0.212622in}}{\pgfqpoint{3.696000in}{3.696000in}}%
\pgfusepath{clip}%
\pgfsetbuttcap%
\pgfsetroundjoin%
\definecolor{currentfill}{rgb}{0.121569,0.466667,0.705882}%
\pgfsetfillcolor{currentfill}%
\pgfsetfillopacity{0.788596}%
\pgfsetlinewidth{1.003750pt}%
\definecolor{currentstroke}{rgb}{0.121569,0.466667,0.705882}%
\pgfsetstrokecolor{currentstroke}%
\pgfsetstrokeopacity{0.788596}%
\pgfsetdash{}{0pt}%
\pgfpathmoveto{\pgfqpoint{1.197085in}{2.430226in}}%
\pgfpathcurveto{\pgfqpoint{1.205321in}{2.430226in}}{\pgfqpoint{1.213221in}{2.433498in}}{\pgfqpoint{1.219045in}{2.439322in}}%
\pgfpathcurveto{\pgfqpoint{1.224869in}{2.445146in}}{\pgfqpoint{1.228141in}{2.453046in}}{\pgfqpoint{1.228141in}{2.461282in}}%
\pgfpathcurveto{\pgfqpoint{1.228141in}{2.469519in}}{\pgfqpoint{1.224869in}{2.477419in}}{\pgfqpoint{1.219045in}{2.483243in}}%
\pgfpathcurveto{\pgfqpoint{1.213221in}{2.489066in}}{\pgfqpoint{1.205321in}{2.492339in}}{\pgfqpoint{1.197085in}{2.492339in}}%
\pgfpathcurveto{\pgfqpoint{1.188849in}{2.492339in}}{\pgfqpoint{1.180949in}{2.489066in}}{\pgfqpoint{1.175125in}{2.483243in}}%
\pgfpathcurveto{\pgfqpoint{1.169301in}{2.477419in}}{\pgfqpoint{1.166028in}{2.469519in}}{\pgfqpoint{1.166028in}{2.461282in}}%
\pgfpathcurveto{\pgfqpoint{1.166028in}{2.453046in}}{\pgfqpoint{1.169301in}{2.445146in}}{\pgfqpoint{1.175125in}{2.439322in}}%
\pgfpathcurveto{\pgfqpoint{1.180949in}{2.433498in}}{\pgfqpoint{1.188849in}{2.430226in}}{\pgfqpoint{1.197085in}{2.430226in}}%
\pgfpathclose%
\pgfusepath{stroke,fill}%
\end{pgfscope}%
\begin{pgfscope}%
\pgfpathrectangle{\pgfqpoint{0.100000in}{0.212622in}}{\pgfqpoint{3.696000in}{3.696000in}}%
\pgfusepath{clip}%
\pgfsetbuttcap%
\pgfsetroundjoin%
\definecolor{currentfill}{rgb}{0.121569,0.466667,0.705882}%
\pgfsetfillcolor{currentfill}%
\pgfsetfillopacity{0.789692}%
\pgfsetlinewidth{1.003750pt}%
\definecolor{currentstroke}{rgb}{0.121569,0.466667,0.705882}%
\pgfsetstrokecolor{currentstroke}%
\pgfsetstrokeopacity{0.789692}%
\pgfsetdash{}{0pt}%
\pgfpathmoveto{\pgfqpoint{2.445261in}{2.779079in}}%
\pgfpathcurveto{\pgfqpoint{2.453497in}{2.779079in}}{\pgfqpoint{2.461397in}{2.782351in}}{\pgfqpoint{2.467221in}{2.788175in}}%
\pgfpathcurveto{\pgfqpoint{2.473045in}{2.793999in}}{\pgfqpoint{2.476317in}{2.801899in}}{\pgfqpoint{2.476317in}{2.810135in}}%
\pgfpathcurveto{\pgfqpoint{2.476317in}{2.818372in}}{\pgfqpoint{2.473045in}{2.826272in}}{\pgfqpoint{2.467221in}{2.832096in}}%
\pgfpathcurveto{\pgfqpoint{2.461397in}{2.837920in}}{\pgfqpoint{2.453497in}{2.841192in}}{\pgfqpoint{2.445261in}{2.841192in}}%
\pgfpathcurveto{\pgfqpoint{2.437025in}{2.841192in}}{\pgfqpoint{2.429125in}{2.837920in}}{\pgfqpoint{2.423301in}{2.832096in}}%
\pgfpathcurveto{\pgfqpoint{2.417477in}{2.826272in}}{\pgfqpoint{2.414204in}{2.818372in}}{\pgfqpoint{2.414204in}{2.810135in}}%
\pgfpathcurveto{\pgfqpoint{2.414204in}{2.801899in}}{\pgfqpoint{2.417477in}{2.793999in}}{\pgfqpoint{2.423301in}{2.788175in}}%
\pgfpathcurveto{\pgfqpoint{2.429125in}{2.782351in}}{\pgfqpoint{2.437025in}{2.779079in}}{\pgfqpoint{2.445261in}{2.779079in}}%
\pgfpathclose%
\pgfusepath{stroke,fill}%
\end{pgfscope}%
\begin{pgfscope}%
\pgfpathrectangle{\pgfqpoint{0.100000in}{0.212622in}}{\pgfqpoint{3.696000in}{3.696000in}}%
\pgfusepath{clip}%
\pgfsetbuttcap%
\pgfsetroundjoin%
\definecolor{currentfill}{rgb}{0.121569,0.466667,0.705882}%
\pgfsetfillcolor{currentfill}%
\pgfsetfillopacity{0.791054}%
\pgfsetlinewidth{1.003750pt}%
\definecolor{currentstroke}{rgb}{0.121569,0.466667,0.705882}%
\pgfsetstrokecolor{currentstroke}%
\pgfsetstrokeopacity{0.791054}%
\pgfsetdash{}{0pt}%
\pgfpathmoveto{\pgfqpoint{2.442370in}{2.775556in}}%
\pgfpathcurveto{\pgfqpoint{2.450606in}{2.775556in}}{\pgfqpoint{2.458506in}{2.778829in}}{\pgfqpoint{2.464330in}{2.784653in}}%
\pgfpathcurveto{\pgfqpoint{2.470154in}{2.790477in}}{\pgfqpoint{2.473427in}{2.798377in}}{\pgfqpoint{2.473427in}{2.806613in}}%
\pgfpathcurveto{\pgfqpoint{2.473427in}{2.814849in}}{\pgfqpoint{2.470154in}{2.822749in}}{\pgfqpoint{2.464330in}{2.828573in}}%
\pgfpathcurveto{\pgfqpoint{2.458506in}{2.834397in}}{\pgfqpoint{2.450606in}{2.837669in}}{\pgfqpoint{2.442370in}{2.837669in}}%
\pgfpathcurveto{\pgfqpoint{2.434134in}{2.837669in}}{\pgfqpoint{2.426234in}{2.834397in}}{\pgfqpoint{2.420410in}{2.828573in}}%
\pgfpathcurveto{\pgfqpoint{2.414586in}{2.822749in}}{\pgfqpoint{2.411314in}{2.814849in}}{\pgfqpoint{2.411314in}{2.806613in}}%
\pgfpathcurveto{\pgfqpoint{2.411314in}{2.798377in}}{\pgfqpoint{2.414586in}{2.790477in}}{\pgfqpoint{2.420410in}{2.784653in}}%
\pgfpathcurveto{\pgfqpoint{2.426234in}{2.778829in}}{\pgfqpoint{2.434134in}{2.775556in}}{\pgfqpoint{2.442370in}{2.775556in}}%
\pgfpathclose%
\pgfusepath{stroke,fill}%
\end{pgfscope}%
\begin{pgfscope}%
\pgfpathrectangle{\pgfqpoint{0.100000in}{0.212622in}}{\pgfqpoint{3.696000in}{3.696000in}}%
\pgfusepath{clip}%
\pgfsetbuttcap%
\pgfsetroundjoin%
\definecolor{currentfill}{rgb}{0.121569,0.466667,0.705882}%
\pgfsetfillcolor{currentfill}%
\pgfsetfillopacity{0.792449}%
\pgfsetlinewidth{1.003750pt}%
\definecolor{currentstroke}{rgb}{0.121569,0.466667,0.705882}%
\pgfsetstrokecolor{currentstroke}%
\pgfsetstrokeopacity{0.792449}%
\pgfsetdash{}{0pt}%
\pgfpathmoveto{\pgfqpoint{1.185768in}{2.419211in}}%
\pgfpathcurveto{\pgfqpoint{1.194005in}{2.419211in}}{\pgfqpoint{1.201905in}{2.422484in}}{\pgfqpoint{1.207729in}{2.428307in}}%
\pgfpathcurveto{\pgfqpoint{1.213553in}{2.434131in}}{\pgfqpoint{1.216825in}{2.442031in}}{\pgfqpoint{1.216825in}{2.450268in}}%
\pgfpathcurveto{\pgfqpoint{1.216825in}{2.458504in}}{\pgfqpoint{1.213553in}{2.466404in}}{\pgfqpoint{1.207729in}{2.472228in}}%
\pgfpathcurveto{\pgfqpoint{1.201905in}{2.478052in}}{\pgfqpoint{1.194005in}{2.481324in}}{\pgfqpoint{1.185768in}{2.481324in}}%
\pgfpathcurveto{\pgfqpoint{1.177532in}{2.481324in}}{\pgfqpoint{1.169632in}{2.478052in}}{\pgfqpoint{1.163808in}{2.472228in}}%
\pgfpathcurveto{\pgfqpoint{1.157984in}{2.466404in}}{\pgfqpoint{1.154712in}{2.458504in}}{\pgfqpoint{1.154712in}{2.450268in}}%
\pgfpathcurveto{\pgfqpoint{1.154712in}{2.442031in}}{\pgfqpoint{1.157984in}{2.434131in}}{\pgfqpoint{1.163808in}{2.428307in}}%
\pgfpathcurveto{\pgfqpoint{1.169632in}{2.422484in}}{\pgfqpoint{1.177532in}{2.419211in}}{\pgfqpoint{1.185768in}{2.419211in}}%
\pgfpathclose%
\pgfusepath{stroke,fill}%
\end{pgfscope}%
\begin{pgfscope}%
\pgfpathrectangle{\pgfqpoint{0.100000in}{0.212622in}}{\pgfqpoint{3.696000in}{3.696000in}}%
\pgfusepath{clip}%
\pgfsetbuttcap%
\pgfsetroundjoin%
\definecolor{currentfill}{rgb}{0.121569,0.466667,0.705882}%
\pgfsetfillcolor{currentfill}%
\pgfsetfillopacity{0.793582}%
\pgfsetlinewidth{1.003750pt}%
\definecolor{currentstroke}{rgb}{0.121569,0.466667,0.705882}%
\pgfsetstrokecolor{currentstroke}%
\pgfsetstrokeopacity{0.793582}%
\pgfsetdash{}{0pt}%
\pgfpathmoveto{\pgfqpoint{2.437128in}{2.769428in}}%
\pgfpathcurveto{\pgfqpoint{2.445364in}{2.769428in}}{\pgfqpoint{2.453264in}{2.772700in}}{\pgfqpoint{2.459088in}{2.778524in}}%
\pgfpathcurveto{\pgfqpoint{2.464912in}{2.784348in}}{\pgfqpoint{2.468184in}{2.792248in}}{\pgfqpoint{2.468184in}{2.800484in}}%
\pgfpathcurveto{\pgfqpoint{2.468184in}{2.808721in}}{\pgfqpoint{2.464912in}{2.816621in}}{\pgfqpoint{2.459088in}{2.822445in}}%
\pgfpathcurveto{\pgfqpoint{2.453264in}{2.828269in}}{\pgfqpoint{2.445364in}{2.831541in}}{\pgfqpoint{2.437128in}{2.831541in}}%
\pgfpathcurveto{\pgfqpoint{2.428891in}{2.831541in}}{\pgfqpoint{2.420991in}{2.828269in}}{\pgfqpoint{2.415167in}{2.822445in}}%
\pgfpathcurveto{\pgfqpoint{2.409343in}{2.816621in}}{\pgfqpoint{2.406071in}{2.808721in}}{\pgfqpoint{2.406071in}{2.800484in}}%
\pgfpathcurveto{\pgfqpoint{2.406071in}{2.792248in}}{\pgfqpoint{2.409343in}{2.784348in}}{\pgfqpoint{2.415167in}{2.778524in}}%
\pgfpathcurveto{\pgfqpoint{2.420991in}{2.772700in}}{\pgfqpoint{2.428891in}{2.769428in}}{\pgfqpoint{2.437128in}{2.769428in}}%
\pgfpathclose%
\pgfusepath{stroke,fill}%
\end{pgfscope}%
\begin{pgfscope}%
\pgfpathrectangle{\pgfqpoint{0.100000in}{0.212622in}}{\pgfqpoint{3.696000in}{3.696000in}}%
\pgfusepath{clip}%
\pgfsetbuttcap%
\pgfsetroundjoin%
\definecolor{currentfill}{rgb}{0.121569,0.466667,0.705882}%
\pgfsetfillcolor{currentfill}%
\pgfsetfillopacity{0.795356}%
\pgfsetlinewidth{1.003750pt}%
\definecolor{currentstroke}{rgb}{0.121569,0.466667,0.705882}%
\pgfsetstrokecolor{currentstroke}%
\pgfsetstrokeopacity{0.795356}%
\pgfsetdash{}{0pt}%
\pgfpathmoveto{\pgfqpoint{2.882016in}{1.556195in}}%
\pgfpathcurveto{\pgfqpoint{2.890252in}{1.556195in}}{\pgfqpoint{2.898152in}{1.559467in}}{\pgfqpoint{2.903976in}{1.565291in}}%
\pgfpathcurveto{\pgfqpoint{2.909800in}{1.571115in}}{\pgfqpoint{2.913072in}{1.579015in}}{\pgfqpoint{2.913072in}{1.587252in}}%
\pgfpathcurveto{\pgfqpoint{2.913072in}{1.595488in}}{\pgfqpoint{2.909800in}{1.603388in}}{\pgfqpoint{2.903976in}{1.609212in}}%
\pgfpathcurveto{\pgfqpoint{2.898152in}{1.615036in}}{\pgfqpoint{2.890252in}{1.618308in}}{\pgfqpoint{2.882016in}{1.618308in}}%
\pgfpathcurveto{\pgfqpoint{2.873780in}{1.618308in}}{\pgfqpoint{2.865880in}{1.615036in}}{\pgfqpoint{2.860056in}{1.609212in}}%
\pgfpathcurveto{\pgfqpoint{2.854232in}{1.603388in}}{\pgfqpoint{2.850959in}{1.595488in}}{\pgfqpoint{2.850959in}{1.587252in}}%
\pgfpathcurveto{\pgfqpoint{2.850959in}{1.579015in}}{\pgfqpoint{2.854232in}{1.571115in}}{\pgfqpoint{2.860056in}{1.565291in}}%
\pgfpathcurveto{\pgfqpoint{2.865880in}{1.559467in}}{\pgfqpoint{2.873780in}{1.556195in}}{\pgfqpoint{2.882016in}{1.556195in}}%
\pgfpathclose%
\pgfusepath{stroke,fill}%
\end{pgfscope}%
\begin{pgfscope}%
\pgfpathrectangle{\pgfqpoint{0.100000in}{0.212622in}}{\pgfqpoint{3.696000in}{3.696000in}}%
\pgfusepath{clip}%
\pgfsetbuttcap%
\pgfsetroundjoin%
\definecolor{currentfill}{rgb}{0.121569,0.466667,0.705882}%
\pgfsetfillcolor{currentfill}%
\pgfsetfillopacity{0.795789}%
\pgfsetlinewidth{1.003750pt}%
\definecolor{currentstroke}{rgb}{0.121569,0.466667,0.705882}%
\pgfsetstrokecolor{currentstroke}%
\pgfsetstrokeopacity{0.795789}%
\pgfsetdash{}{0pt}%
\pgfpathmoveto{\pgfqpoint{2.432345in}{2.763661in}}%
\pgfpathcurveto{\pgfqpoint{2.440581in}{2.763661in}}{\pgfqpoint{2.448481in}{2.766934in}}{\pgfqpoint{2.454305in}{2.772758in}}%
\pgfpathcurveto{\pgfqpoint{2.460129in}{2.778581in}}{\pgfqpoint{2.463402in}{2.786481in}}{\pgfqpoint{2.463402in}{2.794718in}}%
\pgfpathcurveto{\pgfqpoint{2.463402in}{2.802954in}}{\pgfqpoint{2.460129in}{2.810854in}}{\pgfqpoint{2.454305in}{2.816678in}}%
\pgfpathcurveto{\pgfqpoint{2.448481in}{2.822502in}}{\pgfqpoint{2.440581in}{2.825774in}}{\pgfqpoint{2.432345in}{2.825774in}}%
\pgfpathcurveto{\pgfqpoint{2.424109in}{2.825774in}}{\pgfqpoint{2.416209in}{2.822502in}}{\pgfqpoint{2.410385in}{2.816678in}}%
\pgfpathcurveto{\pgfqpoint{2.404561in}{2.810854in}}{\pgfqpoint{2.401289in}{2.802954in}}{\pgfqpoint{2.401289in}{2.794718in}}%
\pgfpathcurveto{\pgfqpoint{2.401289in}{2.786481in}}{\pgfqpoint{2.404561in}{2.778581in}}{\pgfqpoint{2.410385in}{2.772758in}}%
\pgfpathcurveto{\pgfqpoint{2.416209in}{2.766934in}}{\pgfqpoint{2.424109in}{2.763661in}}{\pgfqpoint{2.432345in}{2.763661in}}%
\pgfpathclose%
\pgfusepath{stroke,fill}%
\end{pgfscope}%
\begin{pgfscope}%
\pgfpathrectangle{\pgfqpoint{0.100000in}{0.212622in}}{\pgfqpoint{3.696000in}{3.696000in}}%
\pgfusepath{clip}%
\pgfsetbuttcap%
\pgfsetroundjoin%
\definecolor{currentfill}{rgb}{0.121569,0.466667,0.705882}%
\pgfsetfillcolor{currentfill}%
\pgfsetfillopacity{0.796330}%
\pgfsetlinewidth{1.003750pt}%
\definecolor{currentstroke}{rgb}{0.121569,0.466667,0.705882}%
\pgfsetstrokecolor{currentstroke}%
\pgfsetstrokeopacity{0.796330}%
\pgfsetdash{}{0pt}%
\pgfpathmoveto{\pgfqpoint{1.173600in}{2.406328in}}%
\pgfpathcurveto{\pgfqpoint{1.181836in}{2.406328in}}{\pgfqpoint{1.189736in}{2.409600in}}{\pgfqpoint{1.195560in}{2.415424in}}%
\pgfpathcurveto{\pgfqpoint{1.201384in}{2.421248in}}{\pgfqpoint{1.204656in}{2.429148in}}{\pgfqpoint{1.204656in}{2.437385in}}%
\pgfpathcurveto{\pgfqpoint{1.204656in}{2.445621in}}{\pgfqpoint{1.201384in}{2.453521in}}{\pgfqpoint{1.195560in}{2.459345in}}%
\pgfpathcurveto{\pgfqpoint{1.189736in}{2.465169in}}{\pgfqpoint{1.181836in}{2.468441in}}{\pgfqpoint{1.173600in}{2.468441in}}%
\pgfpathcurveto{\pgfqpoint{1.165363in}{2.468441in}}{\pgfqpoint{1.157463in}{2.465169in}}{\pgfqpoint{1.151639in}{2.459345in}}%
\pgfpathcurveto{\pgfqpoint{1.145815in}{2.453521in}}{\pgfqpoint{1.142543in}{2.445621in}}{\pgfqpoint{1.142543in}{2.437385in}}%
\pgfpathcurveto{\pgfqpoint{1.142543in}{2.429148in}}{\pgfqpoint{1.145815in}{2.421248in}}{\pgfqpoint{1.151639in}{2.415424in}}%
\pgfpathcurveto{\pgfqpoint{1.157463in}{2.409600in}}{\pgfqpoint{1.165363in}{2.406328in}}{\pgfqpoint{1.173600in}{2.406328in}}%
\pgfpathclose%
\pgfusepath{stroke,fill}%
\end{pgfscope}%
\begin{pgfscope}%
\pgfpathrectangle{\pgfqpoint{0.100000in}{0.212622in}}{\pgfqpoint{3.696000in}{3.696000in}}%
\pgfusepath{clip}%
\pgfsetbuttcap%
\pgfsetroundjoin%
\definecolor{currentfill}{rgb}{0.121569,0.466667,0.705882}%
\pgfsetfillcolor{currentfill}%
\pgfsetfillopacity{0.797862}%
\pgfsetlinewidth{1.003750pt}%
\definecolor{currentstroke}{rgb}{0.121569,0.466667,0.705882}%
\pgfsetstrokecolor{currentstroke}%
\pgfsetstrokeopacity{0.797862}%
\pgfsetdash{}{0pt}%
\pgfpathmoveto{\pgfqpoint{2.427828in}{2.758343in}}%
\pgfpathcurveto{\pgfqpoint{2.436064in}{2.758343in}}{\pgfqpoint{2.443964in}{2.761615in}}{\pgfqpoint{2.449788in}{2.767439in}}%
\pgfpathcurveto{\pgfqpoint{2.455612in}{2.773263in}}{\pgfqpoint{2.458885in}{2.781163in}}{\pgfqpoint{2.458885in}{2.789399in}}%
\pgfpathcurveto{\pgfqpoint{2.458885in}{2.797636in}}{\pgfqpoint{2.455612in}{2.805536in}}{\pgfqpoint{2.449788in}{2.811360in}}%
\pgfpathcurveto{\pgfqpoint{2.443964in}{2.817183in}}{\pgfqpoint{2.436064in}{2.820456in}}{\pgfqpoint{2.427828in}{2.820456in}}%
\pgfpathcurveto{\pgfqpoint{2.419592in}{2.820456in}}{\pgfqpoint{2.411692in}{2.817183in}}{\pgfqpoint{2.405868in}{2.811360in}}%
\pgfpathcurveto{\pgfqpoint{2.400044in}{2.805536in}}{\pgfqpoint{2.396772in}{2.797636in}}{\pgfqpoint{2.396772in}{2.789399in}}%
\pgfpathcurveto{\pgfqpoint{2.396772in}{2.781163in}}{\pgfqpoint{2.400044in}{2.773263in}}{\pgfqpoint{2.405868in}{2.767439in}}%
\pgfpathcurveto{\pgfqpoint{2.411692in}{2.761615in}}{\pgfqpoint{2.419592in}{2.758343in}}{\pgfqpoint{2.427828in}{2.758343in}}%
\pgfpathclose%
\pgfusepath{stroke,fill}%
\end{pgfscope}%
\begin{pgfscope}%
\pgfpathrectangle{\pgfqpoint{0.100000in}{0.212622in}}{\pgfqpoint{3.696000in}{3.696000in}}%
\pgfusepath{clip}%
\pgfsetbuttcap%
\pgfsetroundjoin%
\definecolor{currentfill}{rgb}{0.121569,0.466667,0.705882}%
\pgfsetfillcolor{currentfill}%
\pgfsetfillopacity{0.798318}%
\pgfsetlinewidth{1.003750pt}%
\definecolor{currentstroke}{rgb}{0.121569,0.466667,0.705882}%
\pgfsetstrokecolor{currentstroke}%
\pgfsetstrokeopacity{0.798318}%
\pgfsetdash{}{0pt}%
\pgfpathmoveto{\pgfqpoint{1.166939in}{2.398329in}}%
\pgfpathcurveto{\pgfqpoint{1.175176in}{2.398329in}}{\pgfqpoint{1.183076in}{2.401601in}}{\pgfqpoint{1.188900in}{2.407425in}}%
\pgfpathcurveto{\pgfqpoint{1.194724in}{2.413249in}}{\pgfqpoint{1.197996in}{2.421149in}}{\pgfqpoint{1.197996in}{2.429385in}}%
\pgfpathcurveto{\pgfqpoint{1.197996in}{2.437622in}}{\pgfqpoint{1.194724in}{2.445522in}}{\pgfqpoint{1.188900in}{2.451346in}}%
\pgfpathcurveto{\pgfqpoint{1.183076in}{2.457170in}}{\pgfqpoint{1.175176in}{2.460442in}}{\pgfqpoint{1.166939in}{2.460442in}}%
\pgfpathcurveto{\pgfqpoint{1.158703in}{2.460442in}}{\pgfqpoint{1.150803in}{2.457170in}}{\pgfqpoint{1.144979in}{2.451346in}}%
\pgfpathcurveto{\pgfqpoint{1.139155in}{2.445522in}}{\pgfqpoint{1.135883in}{2.437622in}}{\pgfqpoint{1.135883in}{2.429385in}}%
\pgfpathcurveto{\pgfqpoint{1.135883in}{2.421149in}}{\pgfqpoint{1.139155in}{2.413249in}}{\pgfqpoint{1.144979in}{2.407425in}}%
\pgfpathcurveto{\pgfqpoint{1.150803in}{2.401601in}}{\pgfqpoint{1.158703in}{2.398329in}}{\pgfqpoint{1.166939in}{2.398329in}}%
\pgfpathclose%
\pgfusepath{stroke,fill}%
\end{pgfscope}%
\begin{pgfscope}%
\pgfpathrectangle{\pgfqpoint{0.100000in}{0.212622in}}{\pgfqpoint{3.696000in}{3.696000in}}%
\pgfusepath{clip}%
\pgfsetbuttcap%
\pgfsetroundjoin%
\definecolor{currentfill}{rgb}{0.121569,0.466667,0.705882}%
\pgfsetfillcolor{currentfill}%
\pgfsetfillopacity{0.799626}%
\pgfsetlinewidth{1.003750pt}%
\definecolor{currentstroke}{rgb}{0.121569,0.466667,0.705882}%
\pgfsetstrokecolor{currentstroke}%
\pgfsetstrokeopacity{0.799626}%
\pgfsetdash{}{0pt}%
\pgfpathmoveto{\pgfqpoint{2.424083in}{2.754402in}}%
\pgfpathcurveto{\pgfqpoint{2.432320in}{2.754402in}}{\pgfqpoint{2.440220in}{2.757674in}}{\pgfqpoint{2.446044in}{2.763498in}}%
\pgfpathcurveto{\pgfqpoint{2.451867in}{2.769322in}}{\pgfqpoint{2.455140in}{2.777222in}}{\pgfqpoint{2.455140in}{2.785458in}}%
\pgfpathcurveto{\pgfqpoint{2.455140in}{2.793695in}}{\pgfqpoint{2.451867in}{2.801595in}}{\pgfqpoint{2.446044in}{2.807419in}}%
\pgfpathcurveto{\pgfqpoint{2.440220in}{2.813243in}}{\pgfqpoint{2.432320in}{2.816515in}}{\pgfqpoint{2.424083in}{2.816515in}}%
\pgfpathcurveto{\pgfqpoint{2.415847in}{2.816515in}}{\pgfqpoint{2.407947in}{2.813243in}}{\pgfqpoint{2.402123in}{2.807419in}}%
\pgfpathcurveto{\pgfqpoint{2.396299in}{2.801595in}}{\pgfqpoint{2.393027in}{2.793695in}}{\pgfqpoint{2.393027in}{2.785458in}}%
\pgfpathcurveto{\pgfqpoint{2.393027in}{2.777222in}}{\pgfqpoint{2.396299in}{2.769322in}}{\pgfqpoint{2.402123in}{2.763498in}}%
\pgfpathcurveto{\pgfqpoint{2.407947in}{2.757674in}}{\pgfqpoint{2.415847in}{2.754402in}}{\pgfqpoint{2.424083in}{2.754402in}}%
\pgfpathclose%
\pgfusepath{stroke,fill}%
\end{pgfscope}%
\begin{pgfscope}%
\pgfpathrectangle{\pgfqpoint{0.100000in}{0.212622in}}{\pgfqpoint{3.696000in}{3.696000in}}%
\pgfusepath{clip}%
\pgfsetbuttcap%
\pgfsetroundjoin%
\definecolor{currentfill}{rgb}{0.121569,0.466667,0.705882}%
\pgfsetfillcolor{currentfill}%
\pgfsetfillopacity{0.800438}%
\pgfsetlinewidth{1.003750pt}%
\definecolor{currentstroke}{rgb}{0.121569,0.466667,0.705882}%
\pgfsetstrokecolor{currentstroke}%
\pgfsetstrokeopacity{0.800438}%
\pgfsetdash{}{0pt}%
\pgfpathmoveto{\pgfqpoint{1.160117in}{2.389976in}}%
\pgfpathcurveto{\pgfqpoint{1.168353in}{2.389976in}}{\pgfqpoint{1.176253in}{2.393248in}}{\pgfqpoint{1.182077in}{2.399072in}}%
\pgfpathcurveto{\pgfqpoint{1.187901in}{2.404896in}}{\pgfqpoint{1.191173in}{2.412796in}}{\pgfqpoint{1.191173in}{2.421032in}}%
\pgfpathcurveto{\pgfqpoint{1.191173in}{2.429268in}}{\pgfqpoint{1.187901in}{2.437169in}}{\pgfqpoint{1.182077in}{2.442992in}}%
\pgfpathcurveto{\pgfqpoint{1.176253in}{2.448816in}}{\pgfqpoint{1.168353in}{2.452089in}}{\pgfqpoint{1.160117in}{2.452089in}}%
\pgfpathcurveto{\pgfqpoint{1.151880in}{2.452089in}}{\pgfqpoint{1.143980in}{2.448816in}}{\pgfqpoint{1.138156in}{2.442992in}}%
\pgfpathcurveto{\pgfqpoint{1.132332in}{2.437169in}}{\pgfqpoint{1.129060in}{2.429268in}}{\pgfqpoint{1.129060in}{2.421032in}}%
\pgfpathcurveto{\pgfqpoint{1.129060in}{2.412796in}}{\pgfqpoint{1.132332in}{2.404896in}}{\pgfqpoint{1.138156in}{2.399072in}}%
\pgfpathcurveto{\pgfqpoint{1.143980in}{2.393248in}}{\pgfqpoint{1.151880in}{2.389976in}}{\pgfqpoint{1.160117in}{2.389976in}}%
\pgfpathclose%
\pgfusepath{stroke,fill}%
\end{pgfscope}%
\begin{pgfscope}%
\pgfpathrectangle{\pgfqpoint{0.100000in}{0.212622in}}{\pgfqpoint{3.696000in}{3.696000in}}%
\pgfusepath{clip}%
\pgfsetbuttcap%
\pgfsetroundjoin%
\definecolor{currentfill}{rgb}{0.121569,0.466667,0.705882}%
\pgfsetfillcolor{currentfill}%
\pgfsetfillopacity{0.801266}%
\pgfsetlinewidth{1.003750pt}%
\definecolor{currentstroke}{rgb}{0.121569,0.466667,0.705882}%
\pgfsetstrokecolor{currentstroke}%
\pgfsetstrokeopacity{0.801266}%
\pgfsetdash{}{0pt}%
\pgfpathmoveto{\pgfqpoint{2.420671in}{2.750913in}}%
\pgfpathcurveto{\pgfqpoint{2.428907in}{2.750913in}}{\pgfqpoint{2.436807in}{2.754185in}}{\pgfqpoint{2.442631in}{2.760009in}}%
\pgfpathcurveto{\pgfqpoint{2.448455in}{2.765833in}}{\pgfqpoint{2.451728in}{2.773733in}}{\pgfqpoint{2.451728in}{2.781970in}}%
\pgfpathcurveto{\pgfqpoint{2.451728in}{2.790206in}}{\pgfqpoint{2.448455in}{2.798106in}}{\pgfqpoint{2.442631in}{2.803930in}}%
\pgfpathcurveto{\pgfqpoint{2.436807in}{2.809754in}}{\pgfqpoint{2.428907in}{2.813026in}}{\pgfqpoint{2.420671in}{2.813026in}}%
\pgfpathcurveto{\pgfqpoint{2.412435in}{2.813026in}}{\pgfqpoint{2.404535in}{2.809754in}}{\pgfqpoint{2.398711in}{2.803930in}}%
\pgfpathcurveto{\pgfqpoint{2.392887in}{2.798106in}}{\pgfqpoint{2.389615in}{2.790206in}}{\pgfqpoint{2.389615in}{2.781970in}}%
\pgfpathcurveto{\pgfqpoint{2.389615in}{2.773733in}}{\pgfqpoint{2.392887in}{2.765833in}}{\pgfqpoint{2.398711in}{2.760009in}}%
\pgfpathcurveto{\pgfqpoint{2.404535in}{2.754185in}}{\pgfqpoint{2.412435in}{2.750913in}}{\pgfqpoint{2.420671in}{2.750913in}}%
\pgfpathclose%
\pgfusepath{stroke,fill}%
\end{pgfscope}%
\begin{pgfscope}%
\pgfpathrectangle{\pgfqpoint{0.100000in}{0.212622in}}{\pgfqpoint{3.696000in}{3.696000in}}%
\pgfusepath{clip}%
\pgfsetbuttcap%
\pgfsetroundjoin%
\definecolor{currentfill}{rgb}{0.121569,0.466667,0.705882}%
\pgfsetfillcolor{currentfill}%
\pgfsetfillopacity{0.802432}%
\pgfsetlinewidth{1.003750pt}%
\definecolor{currentstroke}{rgb}{0.121569,0.466667,0.705882}%
\pgfsetstrokecolor{currentstroke}%
\pgfsetstrokeopacity{0.802432}%
\pgfsetdash{}{0pt}%
\pgfpathmoveto{\pgfqpoint{2.418102in}{2.747767in}}%
\pgfpathcurveto{\pgfqpoint{2.426338in}{2.747767in}}{\pgfqpoint{2.434238in}{2.751040in}}{\pgfqpoint{2.440062in}{2.756863in}}%
\pgfpathcurveto{\pgfqpoint{2.445886in}{2.762687in}}{\pgfqpoint{2.449158in}{2.770587in}}{\pgfqpoint{2.449158in}{2.778824in}}%
\pgfpathcurveto{\pgfqpoint{2.449158in}{2.787060in}}{\pgfqpoint{2.445886in}{2.794960in}}{\pgfqpoint{2.440062in}{2.800784in}}%
\pgfpathcurveto{\pgfqpoint{2.434238in}{2.806608in}}{\pgfqpoint{2.426338in}{2.809880in}}{\pgfqpoint{2.418102in}{2.809880in}}%
\pgfpathcurveto{\pgfqpoint{2.409865in}{2.809880in}}{\pgfqpoint{2.401965in}{2.806608in}}{\pgfqpoint{2.396141in}{2.800784in}}%
\pgfpathcurveto{\pgfqpoint{2.390317in}{2.794960in}}{\pgfqpoint{2.387045in}{2.787060in}}{\pgfqpoint{2.387045in}{2.778824in}}%
\pgfpathcurveto{\pgfqpoint{2.387045in}{2.770587in}}{\pgfqpoint{2.390317in}{2.762687in}}{\pgfqpoint{2.396141in}{2.756863in}}%
\pgfpathcurveto{\pgfqpoint{2.401965in}{2.751040in}}{\pgfqpoint{2.409865in}{2.747767in}}{\pgfqpoint{2.418102in}{2.747767in}}%
\pgfpathclose%
\pgfusepath{stroke,fill}%
\end{pgfscope}%
\begin{pgfscope}%
\pgfpathrectangle{\pgfqpoint{0.100000in}{0.212622in}}{\pgfqpoint{3.696000in}{3.696000in}}%
\pgfusepath{clip}%
\pgfsetbuttcap%
\pgfsetroundjoin%
\definecolor{currentfill}{rgb}{0.121569,0.466667,0.705882}%
\pgfsetfillcolor{currentfill}%
\pgfsetfillopacity{0.802775}%
\pgfsetlinewidth{1.003750pt}%
\definecolor{currentstroke}{rgb}{0.121569,0.466667,0.705882}%
\pgfsetstrokecolor{currentstroke}%
\pgfsetstrokeopacity{0.802775}%
\pgfsetdash{}{0pt}%
\pgfpathmoveto{\pgfqpoint{1.153061in}{2.381777in}}%
\pgfpathcurveto{\pgfqpoint{1.161297in}{2.381777in}}{\pgfqpoint{1.169197in}{2.385050in}}{\pgfqpoint{1.175021in}{2.390874in}}%
\pgfpathcurveto{\pgfqpoint{1.180845in}{2.396698in}}{\pgfqpoint{1.184118in}{2.404598in}}{\pgfqpoint{1.184118in}{2.412834in}}%
\pgfpathcurveto{\pgfqpoint{1.184118in}{2.421070in}}{\pgfqpoint{1.180845in}{2.428970in}}{\pgfqpoint{1.175021in}{2.434794in}}%
\pgfpathcurveto{\pgfqpoint{1.169197in}{2.440618in}}{\pgfqpoint{1.161297in}{2.443890in}}{\pgfqpoint{1.153061in}{2.443890in}}%
\pgfpathcurveto{\pgfqpoint{1.144825in}{2.443890in}}{\pgfqpoint{1.136925in}{2.440618in}}{\pgfqpoint{1.131101in}{2.434794in}}%
\pgfpathcurveto{\pgfqpoint{1.125277in}{2.428970in}}{\pgfqpoint{1.122005in}{2.421070in}}{\pgfqpoint{1.122005in}{2.412834in}}%
\pgfpathcurveto{\pgfqpoint{1.122005in}{2.404598in}}{\pgfqpoint{1.125277in}{2.396698in}}{\pgfqpoint{1.131101in}{2.390874in}}%
\pgfpathcurveto{\pgfqpoint{1.136925in}{2.385050in}}{\pgfqpoint{1.144825in}{2.381777in}}{\pgfqpoint{1.153061in}{2.381777in}}%
\pgfpathclose%
\pgfusepath{stroke,fill}%
\end{pgfscope}%
\begin{pgfscope}%
\pgfpathrectangle{\pgfqpoint{0.100000in}{0.212622in}}{\pgfqpoint{3.696000in}{3.696000in}}%
\pgfusepath{clip}%
\pgfsetbuttcap%
\pgfsetroundjoin%
\definecolor{currentfill}{rgb}{0.121569,0.466667,0.705882}%
\pgfsetfillcolor{currentfill}%
\pgfsetfillopacity{0.803484}%
\pgfsetlinewidth{1.003750pt}%
\definecolor{currentstroke}{rgb}{0.121569,0.466667,0.705882}%
\pgfsetstrokecolor{currentstroke}%
\pgfsetstrokeopacity{0.803484}%
\pgfsetdash{}{0pt}%
\pgfpathmoveto{\pgfqpoint{2.415734in}{2.744886in}}%
\pgfpathcurveto{\pgfqpoint{2.423971in}{2.744886in}}{\pgfqpoint{2.431871in}{2.748158in}}{\pgfqpoint{2.437695in}{2.753982in}}%
\pgfpathcurveto{\pgfqpoint{2.443519in}{2.759806in}}{\pgfqpoint{2.446791in}{2.767706in}}{\pgfqpoint{2.446791in}{2.775942in}}%
\pgfpathcurveto{\pgfqpoint{2.446791in}{2.784178in}}{\pgfqpoint{2.443519in}{2.792078in}}{\pgfqpoint{2.437695in}{2.797902in}}%
\pgfpathcurveto{\pgfqpoint{2.431871in}{2.803726in}}{\pgfqpoint{2.423971in}{2.806999in}}{\pgfqpoint{2.415734in}{2.806999in}}%
\pgfpathcurveto{\pgfqpoint{2.407498in}{2.806999in}}{\pgfqpoint{2.399598in}{2.803726in}}{\pgfqpoint{2.393774in}{2.797902in}}%
\pgfpathcurveto{\pgfqpoint{2.387950in}{2.792078in}}{\pgfqpoint{2.384678in}{2.784178in}}{\pgfqpoint{2.384678in}{2.775942in}}%
\pgfpathcurveto{\pgfqpoint{2.384678in}{2.767706in}}{\pgfqpoint{2.387950in}{2.759806in}}{\pgfqpoint{2.393774in}{2.753982in}}%
\pgfpathcurveto{\pgfqpoint{2.399598in}{2.748158in}}{\pgfqpoint{2.407498in}{2.744886in}}{\pgfqpoint{2.415734in}{2.744886in}}%
\pgfpathclose%
\pgfusepath{stroke,fill}%
\end{pgfscope}%
\begin{pgfscope}%
\pgfpathrectangle{\pgfqpoint{0.100000in}{0.212622in}}{\pgfqpoint{3.696000in}{3.696000in}}%
\pgfusepath{clip}%
\pgfsetbuttcap%
\pgfsetroundjoin%
\definecolor{currentfill}{rgb}{0.121569,0.466667,0.705882}%
\pgfsetfillcolor{currentfill}%
\pgfsetfillopacity{0.804172}%
\pgfsetlinewidth{1.003750pt}%
\definecolor{currentstroke}{rgb}{0.121569,0.466667,0.705882}%
\pgfsetstrokecolor{currentstroke}%
\pgfsetstrokeopacity{0.804172}%
\pgfsetdash{}{0pt}%
\pgfpathmoveto{\pgfqpoint{1.149163in}{2.377921in}}%
\pgfpathcurveto{\pgfqpoint{1.157399in}{2.377921in}}{\pgfqpoint{1.165299in}{2.381193in}}{\pgfqpoint{1.171123in}{2.387017in}}%
\pgfpathcurveto{\pgfqpoint{1.176947in}{2.392841in}}{\pgfqpoint{1.180219in}{2.400741in}}{\pgfqpoint{1.180219in}{2.408977in}}%
\pgfpathcurveto{\pgfqpoint{1.180219in}{2.417213in}}{\pgfqpoint{1.176947in}{2.425113in}}{\pgfqpoint{1.171123in}{2.430937in}}%
\pgfpathcurveto{\pgfqpoint{1.165299in}{2.436761in}}{\pgfqpoint{1.157399in}{2.440034in}}{\pgfqpoint{1.149163in}{2.440034in}}%
\pgfpathcurveto{\pgfqpoint{1.140927in}{2.440034in}}{\pgfqpoint{1.133027in}{2.436761in}}{\pgfqpoint{1.127203in}{2.430937in}}%
\pgfpathcurveto{\pgfqpoint{1.121379in}{2.425113in}}{\pgfqpoint{1.118106in}{2.417213in}}{\pgfqpoint{1.118106in}{2.408977in}}%
\pgfpathcurveto{\pgfqpoint{1.118106in}{2.400741in}}{\pgfqpoint{1.121379in}{2.392841in}}{\pgfqpoint{1.127203in}{2.387017in}}%
\pgfpathcurveto{\pgfqpoint{1.133027in}{2.381193in}}{\pgfqpoint{1.140927in}{2.377921in}}{\pgfqpoint{1.149163in}{2.377921in}}%
\pgfpathclose%
\pgfusepath{stroke,fill}%
\end{pgfscope}%
\begin{pgfscope}%
\pgfpathrectangle{\pgfqpoint{0.100000in}{0.212622in}}{\pgfqpoint{3.696000in}{3.696000in}}%
\pgfusepath{clip}%
\pgfsetbuttcap%
\pgfsetroundjoin%
\definecolor{currentfill}{rgb}{0.121569,0.466667,0.705882}%
\pgfsetfillcolor{currentfill}%
\pgfsetfillopacity{0.804192}%
\pgfsetlinewidth{1.003750pt}%
\definecolor{currentstroke}{rgb}{0.121569,0.466667,0.705882}%
\pgfsetstrokecolor{currentstroke}%
\pgfsetstrokeopacity{0.804192}%
\pgfsetdash{}{0pt}%
\pgfpathmoveto{\pgfqpoint{2.414139in}{2.743180in}}%
\pgfpathcurveto{\pgfqpoint{2.422376in}{2.743180in}}{\pgfqpoint{2.430276in}{2.746452in}}{\pgfqpoint{2.436100in}{2.752276in}}%
\pgfpathcurveto{\pgfqpoint{2.441924in}{2.758100in}}{\pgfqpoint{2.445196in}{2.766000in}}{\pgfqpoint{2.445196in}{2.774236in}}%
\pgfpathcurveto{\pgfqpoint{2.445196in}{2.782472in}}{\pgfqpoint{2.441924in}{2.790372in}}{\pgfqpoint{2.436100in}{2.796196in}}%
\pgfpathcurveto{\pgfqpoint{2.430276in}{2.802020in}}{\pgfqpoint{2.422376in}{2.805293in}}{\pgfqpoint{2.414139in}{2.805293in}}%
\pgfpathcurveto{\pgfqpoint{2.405903in}{2.805293in}}{\pgfqpoint{2.398003in}{2.802020in}}{\pgfqpoint{2.392179in}{2.796196in}}%
\pgfpathcurveto{\pgfqpoint{2.386355in}{2.790372in}}{\pgfqpoint{2.383083in}{2.782472in}}{\pgfqpoint{2.383083in}{2.774236in}}%
\pgfpathcurveto{\pgfqpoint{2.383083in}{2.766000in}}{\pgfqpoint{2.386355in}{2.758100in}}{\pgfqpoint{2.392179in}{2.752276in}}%
\pgfpathcurveto{\pgfqpoint{2.398003in}{2.746452in}}{\pgfqpoint{2.405903in}{2.743180in}}{\pgfqpoint{2.414139in}{2.743180in}}%
\pgfpathclose%
\pgfusepath{stroke,fill}%
\end{pgfscope}%
\begin{pgfscope}%
\pgfpathrectangle{\pgfqpoint{0.100000in}{0.212622in}}{\pgfqpoint{3.696000in}{3.696000in}}%
\pgfusepath{clip}%
\pgfsetbuttcap%
\pgfsetroundjoin%
\definecolor{currentfill}{rgb}{0.121569,0.466667,0.705882}%
\pgfsetfillcolor{currentfill}%
\pgfsetfillopacity{0.805053}%
\pgfsetlinewidth{1.003750pt}%
\definecolor{currentstroke}{rgb}{0.121569,0.466667,0.705882}%
\pgfsetstrokecolor{currentstroke}%
\pgfsetstrokeopacity{0.805053}%
\pgfsetdash{}{0pt}%
\pgfpathmoveto{\pgfqpoint{2.863841in}{1.536841in}}%
\pgfpathcurveto{\pgfqpoint{2.872077in}{1.536841in}}{\pgfqpoint{2.879977in}{1.540113in}}{\pgfqpoint{2.885801in}{1.545937in}}%
\pgfpathcurveto{\pgfqpoint{2.891625in}{1.551761in}}{\pgfqpoint{2.894897in}{1.559661in}}{\pgfqpoint{2.894897in}{1.567897in}}%
\pgfpathcurveto{\pgfqpoint{2.894897in}{1.576133in}}{\pgfqpoint{2.891625in}{1.584033in}}{\pgfqpoint{2.885801in}{1.589857in}}%
\pgfpathcurveto{\pgfqpoint{2.879977in}{1.595681in}}{\pgfqpoint{2.872077in}{1.598954in}}{\pgfqpoint{2.863841in}{1.598954in}}%
\pgfpathcurveto{\pgfqpoint{2.855605in}{1.598954in}}{\pgfqpoint{2.847705in}{1.595681in}}{\pgfqpoint{2.841881in}{1.589857in}}%
\pgfpathcurveto{\pgfqpoint{2.836057in}{1.584033in}}{\pgfqpoint{2.832784in}{1.576133in}}{\pgfqpoint{2.832784in}{1.567897in}}%
\pgfpathcurveto{\pgfqpoint{2.832784in}{1.559661in}}{\pgfqpoint{2.836057in}{1.551761in}}{\pgfqpoint{2.841881in}{1.545937in}}%
\pgfpathcurveto{\pgfqpoint{2.847705in}{1.540113in}}{\pgfqpoint{2.855605in}{1.536841in}}{\pgfqpoint{2.863841in}{1.536841in}}%
\pgfpathclose%
\pgfusepath{stroke,fill}%
\end{pgfscope}%
\begin{pgfscope}%
\pgfpathrectangle{\pgfqpoint{0.100000in}{0.212622in}}{\pgfqpoint{3.696000in}{3.696000in}}%
\pgfusepath{clip}%
\pgfsetbuttcap%
\pgfsetroundjoin%
\definecolor{currentfill}{rgb}{0.121569,0.466667,0.705882}%
\pgfsetfillcolor{currentfill}%
\pgfsetfillopacity{0.805521}%
\pgfsetlinewidth{1.003750pt}%
\definecolor{currentstroke}{rgb}{0.121569,0.466667,0.705882}%
\pgfsetstrokecolor{currentstroke}%
\pgfsetstrokeopacity{0.805521}%
\pgfsetdash{}{0pt}%
\pgfpathmoveto{\pgfqpoint{2.411318in}{2.740253in}}%
\pgfpathcurveto{\pgfqpoint{2.419554in}{2.740253in}}{\pgfqpoint{2.427454in}{2.743526in}}{\pgfqpoint{2.433278in}{2.749350in}}%
\pgfpathcurveto{\pgfqpoint{2.439102in}{2.755173in}}{\pgfqpoint{2.442374in}{2.763074in}}{\pgfqpoint{2.442374in}{2.771310in}}%
\pgfpathcurveto{\pgfqpoint{2.442374in}{2.779546in}}{\pgfqpoint{2.439102in}{2.787446in}}{\pgfqpoint{2.433278in}{2.793270in}}%
\pgfpathcurveto{\pgfqpoint{2.427454in}{2.799094in}}{\pgfqpoint{2.419554in}{2.802366in}}{\pgfqpoint{2.411318in}{2.802366in}}%
\pgfpathcurveto{\pgfqpoint{2.403081in}{2.802366in}}{\pgfqpoint{2.395181in}{2.799094in}}{\pgfqpoint{2.389357in}{2.793270in}}%
\pgfpathcurveto{\pgfqpoint{2.383534in}{2.787446in}}{\pgfqpoint{2.380261in}{2.779546in}}{\pgfqpoint{2.380261in}{2.771310in}}%
\pgfpathcurveto{\pgfqpoint{2.380261in}{2.763074in}}{\pgfqpoint{2.383534in}{2.755173in}}{\pgfqpoint{2.389357in}{2.749350in}}%
\pgfpathcurveto{\pgfqpoint{2.395181in}{2.743526in}}{\pgfqpoint{2.403081in}{2.740253in}}{\pgfqpoint{2.411318in}{2.740253in}}%
\pgfpathclose%
\pgfusepath{stroke,fill}%
\end{pgfscope}%
\begin{pgfscope}%
\pgfpathrectangle{\pgfqpoint{0.100000in}{0.212622in}}{\pgfqpoint{3.696000in}{3.696000in}}%
\pgfusepath{clip}%
\pgfsetbuttcap%
\pgfsetroundjoin%
\definecolor{currentfill}{rgb}{0.121569,0.466667,0.705882}%
\pgfsetfillcolor{currentfill}%
\pgfsetfillopacity{0.805600}%
\pgfsetlinewidth{1.003750pt}%
\definecolor{currentstroke}{rgb}{0.121569,0.466667,0.705882}%
\pgfsetstrokecolor{currentstroke}%
\pgfsetstrokeopacity{0.805600}%
\pgfsetdash{}{0pt}%
\pgfpathmoveto{\pgfqpoint{1.144796in}{2.373259in}}%
\pgfpathcurveto{\pgfqpoint{1.153032in}{2.373259in}}{\pgfqpoint{1.160933in}{2.376531in}}{\pgfqpoint{1.166756in}{2.382355in}}%
\pgfpathcurveto{\pgfqpoint{1.172580in}{2.388179in}}{\pgfqpoint{1.175853in}{2.396079in}}{\pgfqpoint{1.175853in}{2.404316in}}%
\pgfpathcurveto{\pgfqpoint{1.175853in}{2.412552in}}{\pgfqpoint{1.172580in}{2.420452in}}{\pgfqpoint{1.166756in}{2.426276in}}%
\pgfpathcurveto{\pgfqpoint{1.160933in}{2.432100in}}{\pgfqpoint{1.153032in}{2.435372in}}{\pgfqpoint{1.144796in}{2.435372in}}%
\pgfpathcurveto{\pgfqpoint{1.136560in}{2.435372in}}{\pgfqpoint{1.128660in}{2.432100in}}{\pgfqpoint{1.122836in}{2.426276in}}%
\pgfpathcurveto{\pgfqpoint{1.117012in}{2.420452in}}{\pgfqpoint{1.113740in}{2.412552in}}{\pgfqpoint{1.113740in}{2.404316in}}%
\pgfpathcurveto{\pgfqpoint{1.113740in}{2.396079in}}{\pgfqpoint{1.117012in}{2.388179in}}{\pgfqpoint{1.122836in}{2.382355in}}%
\pgfpathcurveto{\pgfqpoint{1.128660in}{2.376531in}}{\pgfqpoint{1.136560in}{2.373259in}}{\pgfqpoint{1.144796in}{2.373259in}}%
\pgfpathclose%
\pgfusepath{stroke,fill}%
\end{pgfscope}%
\begin{pgfscope}%
\pgfpathrectangle{\pgfqpoint{0.100000in}{0.212622in}}{\pgfqpoint{3.696000in}{3.696000in}}%
\pgfusepath{clip}%
\pgfsetbuttcap%
\pgfsetroundjoin%
\definecolor{currentfill}{rgb}{0.121569,0.466667,0.705882}%
\pgfsetfillcolor{currentfill}%
\pgfsetfillopacity{0.806523}%
\pgfsetlinewidth{1.003750pt}%
\definecolor{currentstroke}{rgb}{0.121569,0.466667,0.705882}%
\pgfsetstrokecolor{currentstroke}%
\pgfsetstrokeopacity{0.806523}%
\pgfsetdash{}{0pt}%
\pgfpathmoveto{\pgfqpoint{2.409088in}{2.737615in}}%
\pgfpathcurveto{\pgfqpoint{2.417324in}{2.737615in}}{\pgfqpoint{2.425224in}{2.740887in}}{\pgfqpoint{2.431048in}{2.746711in}}%
\pgfpathcurveto{\pgfqpoint{2.436872in}{2.752535in}}{\pgfqpoint{2.440144in}{2.760435in}}{\pgfqpoint{2.440144in}{2.768671in}}%
\pgfpathcurveto{\pgfqpoint{2.440144in}{2.776908in}}{\pgfqpoint{2.436872in}{2.784808in}}{\pgfqpoint{2.431048in}{2.790632in}}%
\pgfpathcurveto{\pgfqpoint{2.425224in}{2.796456in}}{\pgfqpoint{2.417324in}{2.799728in}}{\pgfqpoint{2.409088in}{2.799728in}}%
\pgfpathcurveto{\pgfqpoint{2.400851in}{2.799728in}}{\pgfqpoint{2.392951in}{2.796456in}}{\pgfqpoint{2.387127in}{2.790632in}}%
\pgfpathcurveto{\pgfqpoint{2.381303in}{2.784808in}}{\pgfqpoint{2.378031in}{2.776908in}}{\pgfqpoint{2.378031in}{2.768671in}}%
\pgfpathcurveto{\pgfqpoint{2.378031in}{2.760435in}}{\pgfqpoint{2.381303in}{2.752535in}}{\pgfqpoint{2.387127in}{2.746711in}}%
\pgfpathcurveto{\pgfqpoint{2.392951in}{2.740887in}}{\pgfqpoint{2.400851in}{2.737615in}}{\pgfqpoint{2.409088in}{2.737615in}}%
\pgfpathclose%
\pgfusepath{stroke,fill}%
\end{pgfscope}%
\begin{pgfscope}%
\pgfpathrectangle{\pgfqpoint{0.100000in}{0.212622in}}{\pgfqpoint{3.696000in}{3.696000in}}%
\pgfusepath{clip}%
\pgfsetbuttcap%
\pgfsetroundjoin%
\definecolor{currentfill}{rgb}{0.121569,0.466667,0.705882}%
\pgfsetfillcolor{currentfill}%
\pgfsetfillopacity{0.807225}%
\pgfsetlinewidth{1.003750pt}%
\definecolor{currentstroke}{rgb}{0.121569,0.466667,0.705882}%
\pgfsetstrokecolor{currentstroke}%
\pgfsetstrokeopacity{0.807225}%
\pgfsetdash{}{0pt}%
\pgfpathmoveto{\pgfqpoint{1.139481in}{2.367135in}}%
\pgfpathcurveto{\pgfqpoint{1.147717in}{2.367135in}}{\pgfqpoint{1.155617in}{2.370408in}}{\pgfqpoint{1.161441in}{2.376232in}}%
\pgfpathcurveto{\pgfqpoint{1.167265in}{2.382056in}}{\pgfqpoint{1.170537in}{2.389956in}}{\pgfqpoint{1.170537in}{2.398192in}}%
\pgfpathcurveto{\pgfqpoint{1.170537in}{2.406428in}}{\pgfqpoint{1.167265in}{2.414328in}}{\pgfqpoint{1.161441in}{2.420152in}}%
\pgfpathcurveto{\pgfqpoint{1.155617in}{2.425976in}}{\pgfqpoint{1.147717in}{2.429248in}}{\pgfqpoint{1.139481in}{2.429248in}}%
\pgfpathcurveto{\pgfqpoint{1.131245in}{2.429248in}}{\pgfqpoint{1.123345in}{2.425976in}}{\pgfqpoint{1.117521in}{2.420152in}}%
\pgfpathcurveto{\pgfqpoint{1.111697in}{2.414328in}}{\pgfqpoint{1.108424in}{2.406428in}}{\pgfqpoint{1.108424in}{2.398192in}}%
\pgfpathcurveto{\pgfqpoint{1.108424in}{2.389956in}}{\pgfqpoint{1.111697in}{2.382056in}}{\pgfqpoint{1.117521in}{2.376232in}}%
\pgfpathcurveto{\pgfqpoint{1.123345in}{2.370408in}}{\pgfqpoint{1.131245in}{2.367135in}}{\pgfqpoint{1.139481in}{2.367135in}}%
\pgfpathclose%
\pgfusepath{stroke,fill}%
\end{pgfscope}%
\begin{pgfscope}%
\pgfpathrectangle{\pgfqpoint{0.100000in}{0.212622in}}{\pgfqpoint{3.696000in}{3.696000in}}%
\pgfusepath{clip}%
\pgfsetbuttcap%
\pgfsetroundjoin%
\definecolor{currentfill}{rgb}{0.121569,0.466667,0.705882}%
\pgfsetfillcolor{currentfill}%
\pgfsetfillopacity{0.808056}%
\pgfsetlinewidth{1.003750pt}%
\definecolor{currentstroke}{rgb}{0.121569,0.466667,0.705882}%
\pgfsetstrokecolor{currentstroke}%
\pgfsetstrokeopacity{0.808056}%
\pgfsetdash{}{0pt}%
\pgfpathmoveto{\pgfqpoint{1.136567in}{2.363390in}}%
\pgfpathcurveto{\pgfqpoint{1.144803in}{2.363390in}}{\pgfqpoint{1.152703in}{2.366662in}}{\pgfqpoint{1.158527in}{2.372486in}}%
\pgfpathcurveto{\pgfqpoint{1.164351in}{2.378310in}}{\pgfqpoint{1.167623in}{2.386210in}}{\pgfqpoint{1.167623in}{2.394446in}}%
\pgfpathcurveto{\pgfqpoint{1.167623in}{2.402682in}}{\pgfqpoint{1.164351in}{2.410582in}}{\pgfqpoint{1.158527in}{2.416406in}}%
\pgfpathcurveto{\pgfqpoint{1.152703in}{2.422230in}}{\pgfqpoint{1.144803in}{2.425503in}}{\pgfqpoint{1.136567in}{2.425503in}}%
\pgfpathcurveto{\pgfqpoint{1.128331in}{2.425503in}}{\pgfqpoint{1.120431in}{2.422230in}}{\pgfqpoint{1.114607in}{2.416406in}}%
\pgfpathcurveto{\pgfqpoint{1.108783in}{2.410582in}}{\pgfqpoint{1.105510in}{2.402682in}}{\pgfqpoint{1.105510in}{2.394446in}}%
\pgfpathcurveto{\pgfqpoint{1.105510in}{2.386210in}}{\pgfqpoint{1.108783in}{2.378310in}}{\pgfqpoint{1.114607in}{2.372486in}}%
\pgfpathcurveto{\pgfqpoint{1.120431in}{2.366662in}}{\pgfqpoint{1.128331in}{2.363390in}}{\pgfqpoint{1.136567in}{2.363390in}}%
\pgfpathclose%
\pgfusepath{stroke,fill}%
\end{pgfscope}%
\begin{pgfscope}%
\pgfpathrectangle{\pgfqpoint{0.100000in}{0.212622in}}{\pgfqpoint{3.696000in}{3.696000in}}%
\pgfusepath{clip}%
\pgfsetbuttcap%
\pgfsetroundjoin%
\definecolor{currentfill}{rgb}{0.121569,0.466667,0.705882}%
\pgfsetfillcolor{currentfill}%
\pgfsetfillopacity{0.808298}%
\pgfsetlinewidth{1.003750pt}%
\definecolor{currentstroke}{rgb}{0.121569,0.466667,0.705882}%
\pgfsetstrokecolor{currentstroke}%
\pgfsetstrokeopacity{0.808298}%
\pgfsetdash{}{0pt}%
\pgfpathmoveto{\pgfqpoint{2.404925in}{2.732635in}}%
\pgfpathcurveto{\pgfqpoint{2.413162in}{2.732635in}}{\pgfqpoint{2.421062in}{2.735907in}}{\pgfqpoint{2.426886in}{2.741731in}}%
\pgfpathcurveto{\pgfqpoint{2.432710in}{2.747555in}}{\pgfqpoint{2.435982in}{2.755455in}}{\pgfqpoint{2.435982in}{2.763691in}}%
\pgfpathcurveto{\pgfqpoint{2.435982in}{2.771928in}}{\pgfqpoint{2.432710in}{2.779828in}}{\pgfqpoint{2.426886in}{2.785652in}}%
\pgfpathcurveto{\pgfqpoint{2.421062in}{2.791476in}}{\pgfqpoint{2.413162in}{2.794748in}}{\pgfqpoint{2.404925in}{2.794748in}}%
\pgfpathcurveto{\pgfqpoint{2.396689in}{2.794748in}}{\pgfqpoint{2.388789in}{2.791476in}}{\pgfqpoint{2.382965in}{2.785652in}}%
\pgfpathcurveto{\pgfqpoint{2.377141in}{2.779828in}}{\pgfqpoint{2.373869in}{2.771928in}}{\pgfqpoint{2.373869in}{2.763691in}}%
\pgfpathcurveto{\pgfqpoint{2.373869in}{2.755455in}}{\pgfqpoint{2.377141in}{2.747555in}}{\pgfqpoint{2.382965in}{2.741731in}}%
\pgfpathcurveto{\pgfqpoint{2.388789in}{2.735907in}}{\pgfqpoint{2.396689in}{2.732635in}}{\pgfqpoint{2.404925in}{2.732635in}}%
\pgfpathclose%
\pgfusepath{stroke,fill}%
\end{pgfscope}%
\begin{pgfscope}%
\pgfpathrectangle{\pgfqpoint{0.100000in}{0.212622in}}{\pgfqpoint{3.696000in}{3.696000in}}%
\pgfusepath{clip}%
\pgfsetbuttcap%
\pgfsetroundjoin%
\definecolor{currentfill}{rgb}{0.121569,0.466667,0.705882}%
\pgfsetfillcolor{currentfill}%
\pgfsetfillopacity{0.809002}%
\pgfsetlinewidth{1.003750pt}%
\definecolor{currentstroke}{rgb}{0.121569,0.466667,0.705882}%
\pgfsetstrokecolor{currentstroke}%
\pgfsetstrokeopacity{0.809002}%
\pgfsetdash{}{0pt}%
\pgfpathmoveto{\pgfqpoint{1.133458in}{2.359399in}}%
\pgfpathcurveto{\pgfqpoint{1.141694in}{2.359399in}}{\pgfqpoint{1.149594in}{2.362671in}}{\pgfqpoint{1.155418in}{2.368495in}}%
\pgfpathcurveto{\pgfqpoint{1.161242in}{2.374319in}}{\pgfqpoint{1.164515in}{2.382219in}}{\pgfqpoint{1.164515in}{2.390455in}}%
\pgfpathcurveto{\pgfqpoint{1.164515in}{2.398691in}}{\pgfqpoint{1.161242in}{2.406592in}}{\pgfqpoint{1.155418in}{2.412415in}}%
\pgfpathcurveto{\pgfqpoint{1.149594in}{2.418239in}}{\pgfqpoint{1.141694in}{2.421512in}}{\pgfqpoint{1.133458in}{2.421512in}}%
\pgfpathcurveto{\pgfqpoint{1.125222in}{2.421512in}}{\pgfqpoint{1.117322in}{2.418239in}}{\pgfqpoint{1.111498in}{2.412415in}}%
\pgfpathcurveto{\pgfqpoint{1.105674in}{2.406592in}}{\pgfqpoint{1.102402in}{2.398691in}}{\pgfqpoint{1.102402in}{2.390455in}}%
\pgfpathcurveto{\pgfqpoint{1.102402in}{2.382219in}}{\pgfqpoint{1.105674in}{2.374319in}}{\pgfqpoint{1.111498in}{2.368495in}}%
\pgfpathcurveto{\pgfqpoint{1.117322in}{2.362671in}}{\pgfqpoint{1.125222in}{2.359399in}}{\pgfqpoint{1.133458in}{2.359399in}}%
\pgfpathclose%
\pgfusepath{stroke,fill}%
\end{pgfscope}%
\begin{pgfscope}%
\pgfpathrectangle{\pgfqpoint{0.100000in}{0.212622in}}{\pgfqpoint{3.696000in}{3.696000in}}%
\pgfusepath{clip}%
\pgfsetbuttcap%
\pgfsetroundjoin%
\definecolor{currentfill}{rgb}{0.121569,0.466667,0.705882}%
\pgfsetfillcolor{currentfill}%
\pgfsetfillopacity{0.809549}%
\pgfsetlinewidth{1.003750pt}%
\definecolor{currentstroke}{rgb}{0.121569,0.466667,0.705882}%
\pgfsetstrokecolor{currentstroke}%
\pgfsetstrokeopacity{0.809549}%
\pgfsetdash{}{0pt}%
\pgfpathmoveto{\pgfqpoint{1.131773in}{2.357325in}}%
\pgfpathcurveto{\pgfqpoint{1.140009in}{2.357325in}}{\pgfqpoint{1.147909in}{2.360597in}}{\pgfqpoint{1.153733in}{2.366421in}}%
\pgfpathcurveto{\pgfqpoint{1.159557in}{2.372245in}}{\pgfqpoint{1.162830in}{2.380145in}}{\pgfqpoint{1.162830in}{2.388381in}}%
\pgfpathcurveto{\pgfqpoint{1.162830in}{2.396618in}}{\pgfqpoint{1.159557in}{2.404518in}}{\pgfqpoint{1.153733in}{2.410342in}}%
\pgfpathcurveto{\pgfqpoint{1.147909in}{2.416165in}}{\pgfqpoint{1.140009in}{2.419438in}}{\pgfqpoint{1.131773in}{2.419438in}}%
\pgfpathcurveto{\pgfqpoint{1.123537in}{2.419438in}}{\pgfqpoint{1.115637in}{2.416165in}}{\pgfqpoint{1.109813in}{2.410342in}}%
\pgfpathcurveto{\pgfqpoint{1.103989in}{2.404518in}}{\pgfqpoint{1.100717in}{2.396618in}}{\pgfqpoint{1.100717in}{2.388381in}}%
\pgfpathcurveto{\pgfqpoint{1.100717in}{2.380145in}}{\pgfqpoint{1.103989in}{2.372245in}}{\pgfqpoint{1.109813in}{2.366421in}}%
\pgfpathcurveto{\pgfqpoint{1.115637in}{2.360597in}}{\pgfqpoint{1.123537in}{2.357325in}}{\pgfqpoint{1.131773in}{2.357325in}}%
\pgfpathclose%
\pgfusepath{stroke,fill}%
\end{pgfscope}%
\begin{pgfscope}%
\pgfpathrectangle{\pgfqpoint{0.100000in}{0.212622in}}{\pgfqpoint{3.696000in}{3.696000in}}%
\pgfusepath{clip}%
\pgfsetbuttcap%
\pgfsetroundjoin%
\definecolor{currentfill}{rgb}{0.121569,0.466667,0.705882}%
\pgfsetfillcolor{currentfill}%
\pgfsetfillopacity{0.809559}%
\pgfsetlinewidth{1.003750pt}%
\definecolor{currentstroke}{rgb}{0.121569,0.466667,0.705882}%
\pgfsetstrokecolor{currentstroke}%
\pgfsetstrokeopacity{0.809559}%
\pgfsetdash{}{0pt}%
\pgfpathmoveto{\pgfqpoint{2.402076in}{2.729631in}}%
\pgfpathcurveto{\pgfqpoint{2.410312in}{2.729631in}}{\pgfqpoint{2.418212in}{2.732903in}}{\pgfqpoint{2.424036in}{2.738727in}}%
\pgfpathcurveto{\pgfqpoint{2.429860in}{2.744551in}}{\pgfqpoint{2.433132in}{2.752451in}}{\pgfqpoint{2.433132in}{2.760687in}}%
\pgfpathcurveto{\pgfqpoint{2.433132in}{2.768924in}}{\pgfqpoint{2.429860in}{2.776824in}}{\pgfqpoint{2.424036in}{2.782647in}}%
\pgfpathcurveto{\pgfqpoint{2.418212in}{2.788471in}}{\pgfqpoint{2.410312in}{2.791744in}}{\pgfqpoint{2.402076in}{2.791744in}}%
\pgfpathcurveto{\pgfqpoint{2.393840in}{2.791744in}}{\pgfqpoint{2.385940in}{2.788471in}}{\pgfqpoint{2.380116in}{2.782647in}}%
\pgfpathcurveto{\pgfqpoint{2.374292in}{2.776824in}}{\pgfqpoint{2.371019in}{2.768924in}}{\pgfqpoint{2.371019in}{2.760687in}}%
\pgfpathcurveto{\pgfqpoint{2.371019in}{2.752451in}}{\pgfqpoint{2.374292in}{2.744551in}}{\pgfqpoint{2.380116in}{2.738727in}}%
\pgfpathcurveto{\pgfqpoint{2.385940in}{2.732903in}}{\pgfqpoint{2.393840in}{2.729631in}}{\pgfqpoint{2.402076in}{2.729631in}}%
\pgfpathclose%
\pgfusepath{stroke,fill}%
\end{pgfscope}%
\begin{pgfscope}%
\pgfpathrectangle{\pgfqpoint{0.100000in}{0.212622in}}{\pgfqpoint{3.696000in}{3.696000in}}%
\pgfusepath{clip}%
\pgfsetbuttcap%
\pgfsetroundjoin%
\definecolor{currentfill}{rgb}{0.121569,0.466667,0.705882}%
\pgfsetfillcolor{currentfill}%
\pgfsetfillopacity{0.810231}%
\pgfsetlinewidth{1.003750pt}%
\definecolor{currentstroke}{rgb}{0.121569,0.466667,0.705882}%
\pgfsetstrokecolor{currentstroke}%
\pgfsetstrokeopacity{0.810231}%
\pgfsetdash{}{0pt}%
\pgfpathmoveto{\pgfqpoint{1.129863in}{2.355280in}}%
\pgfpathcurveto{\pgfqpoint{1.138099in}{2.355280in}}{\pgfqpoint{1.145999in}{2.358552in}}{\pgfqpoint{1.151823in}{2.364376in}}%
\pgfpathcurveto{\pgfqpoint{1.157647in}{2.370200in}}{\pgfqpoint{1.160919in}{2.378100in}}{\pgfqpoint{1.160919in}{2.386336in}}%
\pgfpathcurveto{\pgfqpoint{1.160919in}{2.394572in}}{\pgfqpoint{1.157647in}{2.402472in}}{\pgfqpoint{1.151823in}{2.408296in}}%
\pgfpathcurveto{\pgfqpoint{1.145999in}{2.414120in}}{\pgfqpoint{1.138099in}{2.417393in}}{\pgfqpoint{1.129863in}{2.417393in}}%
\pgfpathcurveto{\pgfqpoint{1.121627in}{2.417393in}}{\pgfqpoint{1.113727in}{2.414120in}}{\pgfqpoint{1.107903in}{2.408296in}}%
\pgfpathcurveto{\pgfqpoint{1.102079in}{2.402472in}}{\pgfqpoint{1.098806in}{2.394572in}}{\pgfqpoint{1.098806in}{2.386336in}}%
\pgfpathcurveto{\pgfqpoint{1.098806in}{2.378100in}}{\pgfqpoint{1.102079in}{2.370200in}}{\pgfqpoint{1.107903in}{2.364376in}}%
\pgfpathcurveto{\pgfqpoint{1.113727in}{2.358552in}}{\pgfqpoint{1.121627in}{2.355280in}}{\pgfqpoint{1.129863in}{2.355280in}}%
\pgfpathclose%
\pgfusepath{stroke,fill}%
\end{pgfscope}%
\begin{pgfscope}%
\pgfpathrectangle{\pgfqpoint{0.100000in}{0.212622in}}{\pgfqpoint{3.696000in}{3.696000in}}%
\pgfusepath{clip}%
\pgfsetbuttcap%
\pgfsetroundjoin%
\definecolor{currentfill}{rgb}{0.121569,0.466667,0.705882}%
\pgfsetfillcolor{currentfill}%
\pgfsetfillopacity{0.810508}%
\pgfsetlinewidth{1.003750pt}%
\definecolor{currentstroke}{rgb}{0.121569,0.466667,0.705882}%
\pgfsetstrokecolor{currentstroke}%
\pgfsetstrokeopacity{0.810508}%
\pgfsetdash{}{0pt}%
\pgfpathmoveto{\pgfqpoint{2.400006in}{2.727442in}}%
\pgfpathcurveto{\pgfqpoint{2.408242in}{2.727442in}}{\pgfqpoint{2.416142in}{2.730714in}}{\pgfqpoint{2.421966in}{2.736538in}}%
\pgfpathcurveto{\pgfqpoint{2.427790in}{2.742362in}}{\pgfqpoint{2.431063in}{2.750262in}}{\pgfqpoint{2.431063in}{2.758498in}}%
\pgfpathcurveto{\pgfqpoint{2.431063in}{2.766735in}}{\pgfqpoint{2.427790in}{2.774635in}}{\pgfqpoint{2.421966in}{2.780459in}}%
\pgfpathcurveto{\pgfqpoint{2.416142in}{2.786282in}}{\pgfqpoint{2.408242in}{2.789555in}}{\pgfqpoint{2.400006in}{2.789555in}}%
\pgfpathcurveto{\pgfqpoint{2.391770in}{2.789555in}}{\pgfqpoint{2.383870in}{2.786282in}}{\pgfqpoint{2.378046in}{2.780459in}}%
\pgfpathcurveto{\pgfqpoint{2.372222in}{2.774635in}}{\pgfqpoint{2.368950in}{2.766735in}}{\pgfqpoint{2.368950in}{2.758498in}}%
\pgfpathcurveto{\pgfqpoint{2.368950in}{2.750262in}}{\pgfqpoint{2.372222in}{2.742362in}}{\pgfqpoint{2.378046in}{2.736538in}}%
\pgfpathcurveto{\pgfqpoint{2.383870in}{2.730714in}}{\pgfqpoint{2.391770in}{2.727442in}}{\pgfqpoint{2.400006in}{2.727442in}}%
\pgfpathclose%
\pgfusepath{stroke,fill}%
\end{pgfscope}%
\begin{pgfscope}%
\pgfpathrectangle{\pgfqpoint{0.100000in}{0.212622in}}{\pgfqpoint{3.696000in}{3.696000in}}%
\pgfusepath{clip}%
\pgfsetbuttcap%
\pgfsetroundjoin%
\definecolor{currentfill}{rgb}{0.121569,0.466667,0.705882}%
\pgfsetfillcolor{currentfill}%
\pgfsetfillopacity{0.810996}%
\pgfsetlinewidth{1.003750pt}%
\definecolor{currentstroke}{rgb}{0.121569,0.466667,0.705882}%
\pgfsetstrokecolor{currentstroke}%
\pgfsetstrokeopacity{0.810996}%
\pgfsetdash{}{0pt}%
\pgfpathmoveto{\pgfqpoint{1.127598in}{2.352819in}}%
\pgfpathcurveto{\pgfqpoint{1.135834in}{2.352819in}}{\pgfqpoint{1.143734in}{2.356091in}}{\pgfqpoint{1.149558in}{2.361915in}}%
\pgfpathcurveto{\pgfqpoint{1.155382in}{2.367739in}}{\pgfqpoint{1.158654in}{2.375639in}}{\pgfqpoint{1.158654in}{2.383876in}}%
\pgfpathcurveto{\pgfqpoint{1.158654in}{2.392112in}}{\pgfqpoint{1.155382in}{2.400012in}}{\pgfqpoint{1.149558in}{2.405836in}}%
\pgfpathcurveto{\pgfqpoint{1.143734in}{2.411660in}}{\pgfqpoint{1.135834in}{2.414932in}}{\pgfqpoint{1.127598in}{2.414932in}}%
\pgfpathcurveto{\pgfqpoint{1.119362in}{2.414932in}}{\pgfqpoint{1.111461in}{2.411660in}}{\pgfqpoint{1.105638in}{2.405836in}}%
\pgfpathcurveto{\pgfqpoint{1.099814in}{2.400012in}}{\pgfqpoint{1.096541in}{2.392112in}}{\pgfqpoint{1.096541in}{2.383876in}}%
\pgfpathcurveto{\pgfqpoint{1.096541in}{2.375639in}}{\pgfqpoint{1.099814in}{2.367739in}}{\pgfqpoint{1.105638in}{2.361915in}}%
\pgfpathcurveto{\pgfqpoint{1.111461in}{2.356091in}}{\pgfqpoint{1.119362in}{2.352819in}}{\pgfqpoint{1.127598in}{2.352819in}}%
\pgfpathclose%
\pgfusepath{stroke,fill}%
\end{pgfscope}%
\begin{pgfscope}%
\pgfpathrectangle{\pgfqpoint{0.100000in}{0.212622in}}{\pgfqpoint{3.696000in}{3.696000in}}%
\pgfusepath{clip}%
\pgfsetbuttcap%
\pgfsetroundjoin%
\definecolor{currentfill}{rgb}{0.121569,0.466667,0.705882}%
\pgfsetfillcolor{currentfill}%
\pgfsetfillopacity{0.811170}%
\pgfsetlinewidth{1.003750pt}%
\definecolor{currentstroke}{rgb}{0.121569,0.466667,0.705882}%
\pgfsetstrokecolor{currentstroke}%
\pgfsetstrokeopacity{0.811170}%
\pgfsetdash{}{0pt}%
\pgfpathmoveto{\pgfqpoint{2.398464in}{2.725490in}}%
\pgfpathcurveto{\pgfqpoint{2.406701in}{2.725490in}}{\pgfqpoint{2.414601in}{2.728763in}}{\pgfqpoint{2.420425in}{2.734586in}}%
\pgfpathcurveto{\pgfqpoint{2.426248in}{2.740410in}}{\pgfqpoint{2.429521in}{2.748310in}}{\pgfqpoint{2.429521in}{2.756547in}}%
\pgfpathcurveto{\pgfqpoint{2.429521in}{2.764783in}}{\pgfqpoint{2.426248in}{2.772683in}}{\pgfqpoint{2.420425in}{2.778507in}}%
\pgfpathcurveto{\pgfqpoint{2.414601in}{2.784331in}}{\pgfqpoint{2.406701in}{2.787603in}}{\pgfqpoint{2.398464in}{2.787603in}}%
\pgfpathcurveto{\pgfqpoint{2.390228in}{2.787603in}}{\pgfqpoint{2.382328in}{2.784331in}}{\pgfqpoint{2.376504in}{2.778507in}}%
\pgfpathcurveto{\pgfqpoint{2.370680in}{2.772683in}}{\pgfqpoint{2.367408in}{2.764783in}}{\pgfqpoint{2.367408in}{2.756547in}}%
\pgfpathcurveto{\pgfqpoint{2.367408in}{2.748310in}}{\pgfqpoint{2.370680in}{2.740410in}}{\pgfqpoint{2.376504in}{2.734586in}}%
\pgfpathcurveto{\pgfqpoint{2.382328in}{2.728763in}}{\pgfqpoint{2.390228in}{2.725490in}}{\pgfqpoint{2.398464in}{2.725490in}}%
\pgfpathclose%
\pgfusepath{stroke,fill}%
\end{pgfscope}%
\begin{pgfscope}%
\pgfpathrectangle{\pgfqpoint{0.100000in}{0.212622in}}{\pgfqpoint{3.696000in}{3.696000in}}%
\pgfusepath{clip}%
\pgfsetbuttcap%
\pgfsetroundjoin%
\definecolor{currentfill}{rgb}{0.121569,0.466667,0.705882}%
\pgfsetfillcolor{currentfill}%
\pgfsetfillopacity{0.811665}%
\pgfsetlinewidth{1.003750pt}%
\definecolor{currentstroke}{rgb}{0.121569,0.466667,0.705882}%
\pgfsetstrokecolor{currentstroke}%
\pgfsetstrokeopacity{0.811665}%
\pgfsetdash{}{0pt}%
\pgfpathmoveto{\pgfqpoint{2.397266in}{2.723999in}}%
\pgfpathcurveto{\pgfqpoint{2.405502in}{2.723999in}}{\pgfqpoint{2.413402in}{2.727271in}}{\pgfqpoint{2.419226in}{2.733095in}}%
\pgfpathcurveto{\pgfqpoint{2.425050in}{2.738919in}}{\pgfqpoint{2.428322in}{2.746819in}}{\pgfqpoint{2.428322in}{2.755055in}}%
\pgfpathcurveto{\pgfqpoint{2.428322in}{2.763292in}}{\pgfqpoint{2.425050in}{2.771192in}}{\pgfqpoint{2.419226in}{2.777016in}}%
\pgfpathcurveto{\pgfqpoint{2.413402in}{2.782840in}}{\pgfqpoint{2.405502in}{2.786112in}}{\pgfqpoint{2.397266in}{2.786112in}}%
\pgfpathcurveto{\pgfqpoint{2.389030in}{2.786112in}}{\pgfqpoint{2.381129in}{2.782840in}}{\pgfqpoint{2.375306in}{2.777016in}}%
\pgfpathcurveto{\pgfqpoint{2.369482in}{2.771192in}}{\pgfqpoint{2.366209in}{2.763292in}}{\pgfqpoint{2.366209in}{2.755055in}}%
\pgfpathcurveto{\pgfqpoint{2.366209in}{2.746819in}}{\pgfqpoint{2.369482in}{2.738919in}}{\pgfqpoint{2.375306in}{2.733095in}}%
\pgfpathcurveto{\pgfqpoint{2.381129in}{2.727271in}}{\pgfqpoint{2.389030in}{2.723999in}}{\pgfqpoint{2.397266in}{2.723999in}}%
\pgfpathclose%
\pgfusepath{stroke,fill}%
\end{pgfscope}%
\begin{pgfscope}%
\pgfpathrectangle{\pgfqpoint{0.100000in}{0.212622in}}{\pgfqpoint{3.696000in}{3.696000in}}%
\pgfusepath{clip}%
\pgfsetbuttcap%
\pgfsetroundjoin%
\definecolor{currentfill}{rgb}{0.121569,0.466667,0.705882}%
\pgfsetfillcolor{currentfill}%
\pgfsetfillopacity{0.811963}%
\pgfsetlinewidth{1.003750pt}%
\definecolor{currentstroke}{rgb}{0.121569,0.466667,0.705882}%
\pgfsetstrokecolor{currentstroke}%
\pgfsetstrokeopacity{0.811963}%
\pgfsetdash{}{0pt}%
\pgfpathmoveto{\pgfqpoint{1.124532in}{2.349298in}}%
\pgfpathcurveto{\pgfqpoint{1.132768in}{2.349298in}}{\pgfqpoint{1.140668in}{2.352571in}}{\pgfqpoint{1.146492in}{2.358395in}}%
\pgfpathcurveto{\pgfqpoint{1.152316in}{2.364219in}}{\pgfqpoint{1.155588in}{2.372119in}}{\pgfqpoint{1.155588in}{2.380355in}}%
\pgfpathcurveto{\pgfqpoint{1.155588in}{2.388591in}}{\pgfqpoint{1.152316in}{2.396491in}}{\pgfqpoint{1.146492in}{2.402315in}}%
\pgfpathcurveto{\pgfqpoint{1.140668in}{2.408139in}}{\pgfqpoint{1.132768in}{2.411411in}}{\pgfqpoint{1.124532in}{2.411411in}}%
\pgfpathcurveto{\pgfqpoint{1.116296in}{2.411411in}}{\pgfqpoint{1.108396in}{2.408139in}}{\pgfqpoint{1.102572in}{2.402315in}}%
\pgfpathcurveto{\pgfqpoint{1.096748in}{2.396491in}}{\pgfqpoint{1.093475in}{2.388591in}}{\pgfqpoint{1.093475in}{2.380355in}}%
\pgfpathcurveto{\pgfqpoint{1.093475in}{2.372119in}}{\pgfqpoint{1.096748in}{2.364219in}}{\pgfqpoint{1.102572in}{2.358395in}}%
\pgfpathcurveto{\pgfqpoint{1.108396in}{2.352571in}}{\pgfqpoint{1.116296in}{2.349298in}}{\pgfqpoint{1.124532in}{2.349298in}}%
\pgfpathclose%
\pgfusepath{stroke,fill}%
\end{pgfscope}%
\begin{pgfscope}%
\pgfpathrectangle{\pgfqpoint{0.100000in}{0.212622in}}{\pgfqpoint{3.696000in}{3.696000in}}%
\pgfusepath{clip}%
\pgfsetbuttcap%
\pgfsetroundjoin%
\definecolor{currentfill}{rgb}{0.121569,0.466667,0.705882}%
\pgfsetfillcolor{currentfill}%
\pgfsetfillopacity{0.812465}%
\pgfsetlinewidth{1.003750pt}%
\definecolor{currentstroke}{rgb}{0.121569,0.466667,0.705882}%
\pgfsetstrokecolor{currentstroke}%
\pgfsetstrokeopacity{0.812465}%
\pgfsetdash{}{0pt}%
\pgfpathmoveto{\pgfqpoint{1.122902in}{2.347113in}}%
\pgfpathcurveto{\pgfqpoint{1.131139in}{2.347113in}}{\pgfqpoint{1.139039in}{2.350385in}}{\pgfqpoint{1.144863in}{2.356209in}}%
\pgfpathcurveto{\pgfqpoint{1.150687in}{2.362033in}}{\pgfqpoint{1.153959in}{2.369933in}}{\pgfqpoint{1.153959in}{2.378170in}}%
\pgfpathcurveto{\pgfqpoint{1.153959in}{2.386406in}}{\pgfqpoint{1.150687in}{2.394306in}}{\pgfqpoint{1.144863in}{2.400130in}}%
\pgfpathcurveto{\pgfqpoint{1.139039in}{2.405954in}}{\pgfqpoint{1.131139in}{2.409226in}}{\pgfqpoint{1.122902in}{2.409226in}}%
\pgfpathcurveto{\pgfqpoint{1.114666in}{2.409226in}}{\pgfqpoint{1.106766in}{2.405954in}}{\pgfqpoint{1.100942in}{2.400130in}}%
\pgfpathcurveto{\pgfqpoint{1.095118in}{2.394306in}}{\pgfqpoint{1.091846in}{2.386406in}}{\pgfqpoint{1.091846in}{2.378170in}}%
\pgfpathcurveto{\pgfqpoint{1.091846in}{2.369933in}}{\pgfqpoint{1.095118in}{2.362033in}}{\pgfqpoint{1.100942in}{2.356209in}}%
\pgfpathcurveto{\pgfqpoint{1.106766in}{2.350385in}}{\pgfqpoint{1.114666in}{2.347113in}}{\pgfqpoint{1.122902in}{2.347113in}}%
\pgfpathclose%
\pgfusepath{stroke,fill}%
\end{pgfscope}%
\begin{pgfscope}%
\pgfpathrectangle{\pgfqpoint{0.100000in}{0.212622in}}{\pgfqpoint{3.696000in}{3.696000in}}%
\pgfusepath{clip}%
\pgfsetbuttcap%
\pgfsetroundjoin%
\definecolor{currentfill}{rgb}{0.121569,0.466667,0.705882}%
\pgfsetfillcolor{currentfill}%
\pgfsetfillopacity{0.812563}%
\pgfsetlinewidth{1.003750pt}%
\definecolor{currentstroke}{rgb}{0.121569,0.466667,0.705882}%
\pgfsetstrokecolor{currentstroke}%
\pgfsetstrokeopacity{0.812563}%
\pgfsetdash{}{0pt}%
\pgfpathmoveto{\pgfqpoint{2.395066in}{2.721288in}}%
\pgfpathcurveto{\pgfqpoint{2.403302in}{2.721288in}}{\pgfqpoint{2.411202in}{2.724561in}}{\pgfqpoint{2.417026in}{2.730385in}}%
\pgfpathcurveto{\pgfqpoint{2.422850in}{2.736209in}}{\pgfqpoint{2.426122in}{2.744109in}}{\pgfqpoint{2.426122in}{2.752345in}}%
\pgfpathcurveto{\pgfqpoint{2.426122in}{2.760581in}}{\pgfqpoint{2.422850in}{2.768481in}}{\pgfqpoint{2.417026in}{2.774305in}}%
\pgfpathcurveto{\pgfqpoint{2.411202in}{2.780129in}}{\pgfqpoint{2.403302in}{2.783401in}}{\pgfqpoint{2.395066in}{2.783401in}}%
\pgfpathcurveto{\pgfqpoint{2.386830in}{2.783401in}}{\pgfqpoint{2.378930in}{2.780129in}}{\pgfqpoint{2.373106in}{2.774305in}}%
\pgfpathcurveto{\pgfqpoint{2.367282in}{2.768481in}}{\pgfqpoint{2.364009in}{2.760581in}}{\pgfqpoint{2.364009in}{2.752345in}}%
\pgfpathcurveto{\pgfqpoint{2.364009in}{2.744109in}}{\pgfqpoint{2.367282in}{2.736209in}}{\pgfqpoint{2.373106in}{2.730385in}}%
\pgfpathcurveto{\pgfqpoint{2.378930in}{2.724561in}}{\pgfqpoint{2.386830in}{2.721288in}}{\pgfqpoint{2.395066in}{2.721288in}}%
\pgfpathclose%
\pgfusepath{stroke,fill}%
\end{pgfscope}%
\begin{pgfscope}%
\pgfpathrectangle{\pgfqpoint{0.100000in}{0.212622in}}{\pgfqpoint{3.696000in}{3.696000in}}%
\pgfusepath{clip}%
\pgfsetbuttcap%
\pgfsetroundjoin%
\definecolor{currentfill}{rgb}{0.121569,0.466667,0.705882}%
\pgfsetfillcolor{currentfill}%
\pgfsetfillopacity{0.813052}%
\pgfsetlinewidth{1.003750pt}%
\definecolor{currentstroke}{rgb}{0.121569,0.466667,0.705882}%
\pgfsetstrokecolor{currentstroke}%
\pgfsetstrokeopacity{0.813052}%
\pgfsetdash{}{0pt}%
\pgfpathmoveto{\pgfqpoint{2.393913in}{2.720001in}}%
\pgfpathcurveto{\pgfqpoint{2.402149in}{2.720001in}}{\pgfqpoint{2.410049in}{2.723273in}}{\pgfqpoint{2.415873in}{2.729097in}}%
\pgfpathcurveto{\pgfqpoint{2.421697in}{2.734921in}}{\pgfqpoint{2.424970in}{2.742821in}}{\pgfqpoint{2.424970in}{2.751058in}}%
\pgfpathcurveto{\pgfqpoint{2.424970in}{2.759294in}}{\pgfqpoint{2.421697in}{2.767194in}}{\pgfqpoint{2.415873in}{2.773018in}}%
\pgfpathcurveto{\pgfqpoint{2.410049in}{2.778842in}}{\pgfqpoint{2.402149in}{2.782114in}}{\pgfqpoint{2.393913in}{2.782114in}}%
\pgfpathcurveto{\pgfqpoint{2.385677in}{2.782114in}}{\pgfqpoint{2.377777in}{2.778842in}}{\pgfqpoint{2.371953in}{2.773018in}}%
\pgfpathcurveto{\pgfqpoint{2.366129in}{2.767194in}}{\pgfqpoint{2.362857in}{2.759294in}}{\pgfqpoint{2.362857in}{2.751058in}}%
\pgfpathcurveto{\pgfqpoint{2.362857in}{2.742821in}}{\pgfqpoint{2.366129in}{2.734921in}}{\pgfqpoint{2.371953in}{2.729097in}}%
\pgfpathcurveto{\pgfqpoint{2.377777in}{2.723273in}}{\pgfqpoint{2.385677in}{2.720001in}}{\pgfqpoint{2.393913in}{2.720001in}}%
\pgfpathclose%
\pgfusepath{stroke,fill}%
\end{pgfscope}%
\begin{pgfscope}%
\pgfpathrectangle{\pgfqpoint{0.100000in}{0.212622in}}{\pgfqpoint{3.696000in}{3.696000in}}%
\pgfusepath{clip}%
\pgfsetbuttcap%
\pgfsetroundjoin%
\definecolor{currentfill}{rgb}{0.121569,0.466667,0.705882}%
\pgfsetfillcolor{currentfill}%
\pgfsetfillopacity{0.813141}%
\pgfsetlinewidth{1.003750pt}%
\definecolor{currentstroke}{rgb}{0.121569,0.466667,0.705882}%
\pgfsetstrokecolor{currentstroke}%
\pgfsetstrokeopacity{0.813141}%
\pgfsetdash{}{0pt}%
\pgfpathmoveto{\pgfqpoint{1.120829in}{2.344398in}}%
\pgfpathcurveto{\pgfqpoint{1.129065in}{2.344398in}}{\pgfqpoint{1.136965in}{2.347670in}}{\pgfqpoint{1.142789in}{2.353494in}}%
\pgfpathcurveto{\pgfqpoint{1.148613in}{2.359318in}}{\pgfqpoint{1.151886in}{2.367218in}}{\pgfqpoint{1.151886in}{2.375454in}}%
\pgfpathcurveto{\pgfqpoint{1.151886in}{2.383691in}}{\pgfqpoint{1.148613in}{2.391591in}}{\pgfqpoint{1.142789in}{2.397415in}}%
\pgfpathcurveto{\pgfqpoint{1.136965in}{2.403239in}}{\pgfqpoint{1.129065in}{2.406511in}}{\pgfqpoint{1.120829in}{2.406511in}}%
\pgfpathcurveto{\pgfqpoint{1.112593in}{2.406511in}}{\pgfqpoint{1.104693in}{2.403239in}}{\pgfqpoint{1.098869in}{2.397415in}}%
\pgfpathcurveto{\pgfqpoint{1.093045in}{2.391591in}}{\pgfqpoint{1.089773in}{2.383691in}}{\pgfqpoint{1.089773in}{2.375454in}}%
\pgfpathcurveto{\pgfqpoint{1.089773in}{2.367218in}}{\pgfqpoint{1.093045in}{2.359318in}}{\pgfqpoint{1.098869in}{2.353494in}}%
\pgfpathcurveto{\pgfqpoint{1.104693in}{2.347670in}}{\pgfqpoint{1.112593in}{2.344398in}}{\pgfqpoint{1.120829in}{2.344398in}}%
\pgfpathclose%
\pgfusepath{stroke,fill}%
\end{pgfscope}%
\begin{pgfscope}%
\pgfpathrectangle{\pgfqpoint{0.100000in}{0.212622in}}{\pgfqpoint{3.696000in}{3.696000in}}%
\pgfusepath{clip}%
\pgfsetbuttcap%
\pgfsetroundjoin%
\definecolor{currentfill}{rgb}{0.121569,0.466667,0.705882}%
\pgfsetfillcolor{currentfill}%
\pgfsetfillopacity{0.813966}%
\pgfsetlinewidth{1.003750pt}%
\definecolor{currentstroke}{rgb}{0.121569,0.466667,0.705882}%
\pgfsetstrokecolor{currentstroke}%
\pgfsetstrokeopacity{0.813966}%
\pgfsetdash{}{0pt}%
\pgfpathmoveto{\pgfqpoint{2.391851in}{2.717766in}}%
\pgfpathcurveto{\pgfqpoint{2.400087in}{2.717766in}}{\pgfqpoint{2.407987in}{2.721038in}}{\pgfqpoint{2.413811in}{2.726862in}}%
\pgfpathcurveto{\pgfqpoint{2.419635in}{2.732686in}}{\pgfqpoint{2.422908in}{2.740586in}}{\pgfqpoint{2.422908in}{2.748822in}}%
\pgfpathcurveto{\pgfqpoint{2.422908in}{2.757058in}}{\pgfqpoint{2.419635in}{2.764958in}}{\pgfqpoint{2.413811in}{2.770782in}}%
\pgfpathcurveto{\pgfqpoint{2.407987in}{2.776606in}}{\pgfqpoint{2.400087in}{2.779879in}}{\pgfqpoint{2.391851in}{2.779879in}}%
\pgfpathcurveto{\pgfqpoint{2.383615in}{2.779879in}}{\pgfqpoint{2.375715in}{2.776606in}}{\pgfqpoint{2.369891in}{2.770782in}}%
\pgfpathcurveto{\pgfqpoint{2.364067in}{2.764958in}}{\pgfqpoint{2.360795in}{2.757058in}}{\pgfqpoint{2.360795in}{2.748822in}}%
\pgfpathcurveto{\pgfqpoint{2.360795in}{2.740586in}}{\pgfqpoint{2.364067in}{2.732686in}}{\pgfqpoint{2.369891in}{2.726862in}}%
\pgfpathcurveto{\pgfqpoint{2.375715in}{2.721038in}}{\pgfqpoint{2.383615in}{2.717766in}}{\pgfqpoint{2.391851in}{2.717766in}}%
\pgfpathclose%
\pgfusepath{stroke,fill}%
\end{pgfscope}%
\begin{pgfscope}%
\pgfpathrectangle{\pgfqpoint{0.100000in}{0.212622in}}{\pgfqpoint{3.696000in}{3.696000in}}%
\pgfusepath{clip}%
\pgfsetbuttcap%
\pgfsetroundjoin%
\definecolor{currentfill}{rgb}{0.121569,0.466667,0.705882}%
\pgfsetfillcolor{currentfill}%
\pgfsetfillopacity{0.814006}%
\pgfsetlinewidth{1.003750pt}%
\definecolor{currentstroke}{rgb}{0.121569,0.466667,0.705882}%
\pgfsetstrokecolor{currentstroke}%
\pgfsetstrokeopacity{0.814006}%
\pgfsetdash{}{0pt}%
\pgfpathmoveto{\pgfqpoint{1.118319in}{2.341421in}}%
\pgfpathcurveto{\pgfqpoint{1.126556in}{2.341421in}}{\pgfqpoint{1.134456in}{2.344694in}}{\pgfqpoint{1.140280in}{2.350518in}}%
\pgfpathcurveto{\pgfqpoint{1.146103in}{2.356342in}}{\pgfqpoint{1.149376in}{2.364242in}}{\pgfqpoint{1.149376in}{2.372478in}}%
\pgfpathcurveto{\pgfqpoint{1.149376in}{2.380714in}}{\pgfqpoint{1.146103in}{2.388614in}}{\pgfqpoint{1.140280in}{2.394438in}}%
\pgfpathcurveto{\pgfqpoint{1.134456in}{2.400262in}}{\pgfqpoint{1.126556in}{2.403534in}}{\pgfqpoint{1.118319in}{2.403534in}}%
\pgfpathcurveto{\pgfqpoint{1.110083in}{2.403534in}}{\pgfqpoint{1.102183in}{2.400262in}}{\pgfqpoint{1.096359in}{2.394438in}}%
\pgfpathcurveto{\pgfqpoint{1.090535in}{2.388614in}}{\pgfqpoint{1.087263in}{2.380714in}}{\pgfqpoint{1.087263in}{2.372478in}}%
\pgfpathcurveto{\pgfqpoint{1.087263in}{2.364242in}}{\pgfqpoint{1.090535in}{2.356342in}}{\pgfqpoint{1.096359in}{2.350518in}}%
\pgfpathcurveto{\pgfqpoint{1.102183in}{2.344694in}}{\pgfqpoint{1.110083in}{2.341421in}}{\pgfqpoint{1.118319in}{2.341421in}}%
\pgfpathclose%
\pgfusepath{stroke,fill}%
\end{pgfscope}%
\begin{pgfscope}%
\pgfpathrectangle{\pgfqpoint{0.100000in}{0.212622in}}{\pgfqpoint{3.696000in}{3.696000in}}%
\pgfusepath{clip}%
\pgfsetbuttcap%
\pgfsetroundjoin%
\definecolor{currentfill}{rgb}{0.121569,0.466667,0.705882}%
\pgfsetfillcolor{currentfill}%
\pgfsetfillopacity{0.814628}%
\pgfsetlinewidth{1.003750pt}%
\definecolor{currentstroke}{rgb}{0.121569,0.466667,0.705882}%
\pgfsetstrokecolor{currentstroke}%
\pgfsetstrokeopacity{0.814628}%
\pgfsetdash{}{0pt}%
\pgfpathmoveto{\pgfqpoint{2.846288in}{1.518407in}}%
\pgfpathcurveto{\pgfqpoint{2.854525in}{1.518407in}}{\pgfqpoint{2.862425in}{1.521679in}}{\pgfqpoint{2.868249in}{1.527503in}}%
\pgfpathcurveto{\pgfqpoint{2.874073in}{1.533327in}}{\pgfqpoint{2.877345in}{1.541227in}}{\pgfqpoint{2.877345in}{1.549463in}}%
\pgfpathcurveto{\pgfqpoint{2.877345in}{1.557699in}}{\pgfqpoint{2.874073in}{1.565600in}}{\pgfqpoint{2.868249in}{1.571423in}}%
\pgfpathcurveto{\pgfqpoint{2.862425in}{1.577247in}}{\pgfqpoint{2.854525in}{1.580520in}}{\pgfqpoint{2.846288in}{1.580520in}}%
\pgfpathcurveto{\pgfqpoint{2.838052in}{1.580520in}}{\pgfqpoint{2.830152in}{1.577247in}}{\pgfqpoint{2.824328in}{1.571423in}}%
\pgfpathcurveto{\pgfqpoint{2.818504in}{1.565600in}}{\pgfqpoint{2.815232in}{1.557699in}}{\pgfqpoint{2.815232in}{1.549463in}}%
\pgfpathcurveto{\pgfqpoint{2.815232in}{1.541227in}}{\pgfqpoint{2.818504in}{1.533327in}}{\pgfqpoint{2.824328in}{1.527503in}}%
\pgfpathcurveto{\pgfqpoint{2.830152in}{1.521679in}}{\pgfqpoint{2.838052in}{1.518407in}}{\pgfqpoint{2.846288in}{1.518407in}}%
\pgfpathclose%
\pgfusepath{stroke,fill}%
\end{pgfscope}%
\begin{pgfscope}%
\pgfpathrectangle{\pgfqpoint{0.100000in}{0.212622in}}{\pgfqpoint{3.696000in}{3.696000in}}%
\pgfusepath{clip}%
\pgfsetbuttcap%
\pgfsetroundjoin%
\definecolor{currentfill}{rgb}{0.121569,0.466667,0.705882}%
\pgfsetfillcolor{currentfill}%
\pgfsetfillopacity{0.814631}%
\pgfsetlinewidth{1.003750pt}%
\definecolor{currentstroke}{rgb}{0.121569,0.466667,0.705882}%
\pgfsetstrokecolor{currentstroke}%
\pgfsetstrokeopacity{0.814631}%
\pgfsetdash{}{0pt}%
\pgfpathmoveto{\pgfqpoint{2.390297in}{2.715827in}}%
\pgfpathcurveto{\pgfqpoint{2.398533in}{2.715827in}}{\pgfqpoint{2.406433in}{2.719099in}}{\pgfqpoint{2.412257in}{2.724923in}}%
\pgfpathcurveto{\pgfqpoint{2.418081in}{2.730747in}}{\pgfqpoint{2.421353in}{2.738647in}}{\pgfqpoint{2.421353in}{2.746883in}}%
\pgfpathcurveto{\pgfqpoint{2.421353in}{2.755120in}}{\pgfqpoint{2.418081in}{2.763020in}}{\pgfqpoint{2.412257in}{2.768844in}}%
\pgfpathcurveto{\pgfqpoint{2.406433in}{2.774668in}}{\pgfqpoint{2.398533in}{2.777940in}}{\pgfqpoint{2.390297in}{2.777940in}}%
\pgfpathcurveto{\pgfqpoint{2.382061in}{2.777940in}}{\pgfqpoint{2.374161in}{2.774668in}}{\pgfqpoint{2.368337in}{2.768844in}}%
\pgfpathcurveto{\pgfqpoint{2.362513in}{2.763020in}}{\pgfqpoint{2.359240in}{2.755120in}}{\pgfqpoint{2.359240in}{2.746883in}}%
\pgfpathcurveto{\pgfqpoint{2.359240in}{2.738647in}}{\pgfqpoint{2.362513in}{2.730747in}}{\pgfqpoint{2.368337in}{2.724923in}}%
\pgfpathcurveto{\pgfqpoint{2.374161in}{2.719099in}}{\pgfqpoint{2.382061in}{2.715827in}}{\pgfqpoint{2.390297in}{2.715827in}}%
\pgfpathclose%
\pgfusepath{stroke,fill}%
\end{pgfscope}%
\begin{pgfscope}%
\pgfpathrectangle{\pgfqpoint{0.100000in}{0.212622in}}{\pgfqpoint{3.696000in}{3.696000in}}%
\pgfusepath{clip}%
\pgfsetbuttcap%
\pgfsetroundjoin%
\definecolor{currentfill}{rgb}{0.121569,0.466667,0.705882}%
\pgfsetfillcolor{currentfill}%
\pgfsetfillopacity{0.815087}%
\pgfsetlinewidth{1.003750pt}%
\definecolor{currentstroke}{rgb}{0.121569,0.466667,0.705882}%
\pgfsetstrokecolor{currentstroke}%
\pgfsetstrokeopacity{0.815087}%
\pgfsetdash{}{0pt}%
\pgfpathmoveto{\pgfqpoint{1.115253in}{2.338526in}}%
\pgfpathcurveto{\pgfqpoint{1.123489in}{2.338526in}}{\pgfqpoint{1.131389in}{2.341798in}}{\pgfqpoint{1.137213in}{2.347622in}}%
\pgfpathcurveto{\pgfqpoint{1.143037in}{2.353446in}}{\pgfqpoint{1.146309in}{2.361346in}}{\pgfqpoint{1.146309in}{2.369582in}}%
\pgfpathcurveto{\pgfqpoint{1.146309in}{2.377819in}}{\pgfqpoint{1.143037in}{2.385719in}}{\pgfqpoint{1.137213in}{2.391543in}}%
\pgfpathcurveto{\pgfqpoint{1.131389in}{2.397367in}}{\pgfqpoint{1.123489in}{2.400639in}}{\pgfqpoint{1.115253in}{2.400639in}}%
\pgfpathcurveto{\pgfqpoint{1.107017in}{2.400639in}}{\pgfqpoint{1.099117in}{2.397367in}}{\pgfqpoint{1.093293in}{2.391543in}}%
\pgfpathcurveto{\pgfqpoint{1.087469in}{2.385719in}}{\pgfqpoint{1.084196in}{2.377819in}}{\pgfqpoint{1.084196in}{2.369582in}}%
\pgfpathcurveto{\pgfqpoint{1.084196in}{2.361346in}}{\pgfqpoint{1.087469in}{2.353446in}}{\pgfqpoint{1.093293in}{2.347622in}}%
\pgfpathcurveto{\pgfqpoint{1.099117in}{2.341798in}}{\pgfqpoint{1.107017in}{2.338526in}}{\pgfqpoint{1.115253in}{2.338526in}}%
\pgfpathclose%
\pgfusepath{stroke,fill}%
\end{pgfscope}%
\begin{pgfscope}%
\pgfpathrectangle{\pgfqpoint{0.100000in}{0.212622in}}{\pgfqpoint{3.696000in}{3.696000in}}%
\pgfusepath{clip}%
\pgfsetbuttcap%
\pgfsetroundjoin%
\definecolor{currentfill}{rgb}{0.121569,0.466667,0.705882}%
\pgfsetfillcolor{currentfill}%
\pgfsetfillopacity{0.815829}%
\pgfsetlinewidth{1.003750pt}%
\definecolor{currentstroke}{rgb}{0.121569,0.466667,0.705882}%
\pgfsetstrokecolor{currentstroke}%
\pgfsetstrokeopacity{0.815829}%
\pgfsetdash{}{0pt}%
\pgfpathmoveto{\pgfqpoint{2.387416in}{2.712288in}}%
\pgfpathcurveto{\pgfqpoint{2.395653in}{2.712288in}}{\pgfqpoint{2.403553in}{2.715560in}}{\pgfqpoint{2.409377in}{2.721384in}}%
\pgfpathcurveto{\pgfqpoint{2.415201in}{2.727208in}}{\pgfqpoint{2.418473in}{2.735108in}}{\pgfqpoint{2.418473in}{2.743344in}}%
\pgfpathcurveto{\pgfqpoint{2.418473in}{2.751580in}}{\pgfqpoint{2.415201in}{2.759480in}}{\pgfqpoint{2.409377in}{2.765304in}}%
\pgfpathcurveto{\pgfqpoint{2.403553in}{2.771128in}}{\pgfqpoint{2.395653in}{2.774401in}}{\pgfqpoint{2.387416in}{2.774401in}}%
\pgfpathcurveto{\pgfqpoint{2.379180in}{2.774401in}}{\pgfqpoint{2.371280in}{2.771128in}}{\pgfqpoint{2.365456in}{2.765304in}}%
\pgfpathcurveto{\pgfqpoint{2.359632in}{2.759480in}}{\pgfqpoint{2.356360in}{2.751580in}}{\pgfqpoint{2.356360in}{2.743344in}}%
\pgfpathcurveto{\pgfqpoint{2.356360in}{2.735108in}}{\pgfqpoint{2.359632in}{2.727208in}}{\pgfqpoint{2.365456in}{2.721384in}}%
\pgfpathcurveto{\pgfqpoint{2.371280in}{2.715560in}}{\pgfqpoint{2.379180in}{2.712288in}}{\pgfqpoint{2.387416in}{2.712288in}}%
\pgfpathclose%
\pgfusepath{stroke,fill}%
\end{pgfscope}%
\begin{pgfscope}%
\pgfpathrectangle{\pgfqpoint{0.100000in}{0.212622in}}{\pgfqpoint{3.696000in}{3.696000in}}%
\pgfusepath{clip}%
\pgfsetbuttcap%
\pgfsetroundjoin%
\definecolor{currentfill}{rgb}{0.121569,0.466667,0.705882}%
\pgfsetfillcolor{currentfill}%
\pgfsetfillopacity{0.816415}%
\pgfsetlinewidth{1.003750pt}%
\definecolor{currentstroke}{rgb}{0.121569,0.466667,0.705882}%
\pgfsetstrokecolor{currentstroke}%
\pgfsetstrokeopacity{0.816415}%
\pgfsetdash{}{0pt}%
\pgfpathmoveto{\pgfqpoint{1.111279in}{2.334412in}}%
\pgfpathcurveto{\pgfqpoint{1.119515in}{2.334412in}}{\pgfqpoint{1.127415in}{2.337684in}}{\pgfqpoint{1.133239in}{2.343508in}}%
\pgfpathcurveto{\pgfqpoint{1.139063in}{2.349332in}}{\pgfqpoint{1.142335in}{2.357232in}}{\pgfqpoint{1.142335in}{2.365468in}}%
\pgfpathcurveto{\pgfqpoint{1.142335in}{2.373704in}}{\pgfqpoint{1.139063in}{2.381604in}}{\pgfqpoint{1.133239in}{2.387428in}}%
\pgfpathcurveto{\pgfqpoint{1.127415in}{2.393252in}}{\pgfqpoint{1.119515in}{2.396525in}}{\pgfqpoint{1.111279in}{2.396525in}}%
\pgfpathcurveto{\pgfqpoint{1.103043in}{2.396525in}}{\pgfqpoint{1.095143in}{2.393252in}}{\pgfqpoint{1.089319in}{2.387428in}}%
\pgfpathcurveto{\pgfqpoint{1.083495in}{2.381604in}}{\pgfqpoint{1.080222in}{2.373704in}}{\pgfqpoint{1.080222in}{2.365468in}}%
\pgfpathcurveto{\pgfqpoint{1.080222in}{2.357232in}}{\pgfqpoint{1.083495in}{2.349332in}}{\pgfqpoint{1.089319in}{2.343508in}}%
\pgfpathcurveto{\pgfqpoint{1.095143in}{2.337684in}}{\pgfqpoint{1.103043in}{2.334412in}}{\pgfqpoint{1.111279in}{2.334412in}}%
\pgfpathclose%
\pgfusepath{stroke,fill}%
\end{pgfscope}%
\begin{pgfscope}%
\pgfpathrectangle{\pgfqpoint{0.100000in}{0.212622in}}{\pgfqpoint{3.696000in}{3.696000in}}%
\pgfusepath{clip}%
\pgfsetbuttcap%
\pgfsetroundjoin%
\definecolor{currentfill}{rgb}{0.121569,0.466667,0.705882}%
\pgfsetfillcolor{currentfill}%
\pgfsetfillopacity{0.816586}%
\pgfsetlinewidth{1.003750pt}%
\definecolor{currentstroke}{rgb}{0.121569,0.466667,0.705882}%
\pgfsetstrokecolor{currentstroke}%
\pgfsetstrokeopacity{0.816586}%
\pgfsetdash{}{0pt}%
\pgfpathmoveto{\pgfqpoint{2.385678in}{2.710425in}}%
\pgfpathcurveto{\pgfqpoint{2.393915in}{2.710425in}}{\pgfqpoint{2.401815in}{2.713697in}}{\pgfqpoint{2.407639in}{2.719521in}}%
\pgfpathcurveto{\pgfqpoint{2.413463in}{2.725345in}}{\pgfqpoint{2.416735in}{2.733245in}}{\pgfqpoint{2.416735in}{2.741481in}}%
\pgfpathcurveto{\pgfqpoint{2.416735in}{2.749718in}}{\pgfqpoint{2.413463in}{2.757618in}}{\pgfqpoint{2.407639in}{2.763442in}}%
\pgfpathcurveto{\pgfqpoint{2.401815in}{2.769266in}}{\pgfqpoint{2.393915in}{2.772538in}}{\pgfqpoint{2.385678in}{2.772538in}}%
\pgfpathcurveto{\pgfqpoint{2.377442in}{2.772538in}}{\pgfqpoint{2.369542in}{2.769266in}}{\pgfqpoint{2.363718in}{2.763442in}}%
\pgfpathcurveto{\pgfqpoint{2.357894in}{2.757618in}}{\pgfqpoint{2.354622in}{2.749718in}}{\pgfqpoint{2.354622in}{2.741481in}}%
\pgfpathcurveto{\pgfqpoint{2.354622in}{2.733245in}}{\pgfqpoint{2.357894in}{2.725345in}}{\pgfqpoint{2.363718in}{2.719521in}}%
\pgfpathcurveto{\pgfqpoint{2.369542in}{2.713697in}}{\pgfqpoint{2.377442in}{2.710425in}}{\pgfqpoint{2.385678in}{2.710425in}}%
\pgfpathclose%
\pgfusepath{stroke,fill}%
\end{pgfscope}%
\begin{pgfscope}%
\pgfpathrectangle{\pgfqpoint{0.100000in}{0.212622in}}{\pgfqpoint{3.696000in}{3.696000in}}%
\pgfusepath{clip}%
\pgfsetbuttcap%
\pgfsetroundjoin%
\definecolor{currentfill}{rgb}{0.121569,0.466667,0.705882}%
\pgfsetfillcolor{currentfill}%
\pgfsetfillopacity{0.817922}%
\pgfsetlinewidth{1.003750pt}%
\definecolor{currentstroke}{rgb}{0.121569,0.466667,0.705882}%
\pgfsetstrokecolor{currentstroke}%
\pgfsetstrokeopacity{0.817922}%
\pgfsetdash{}{0pt}%
\pgfpathmoveto{\pgfqpoint{1.106383in}{2.328743in}}%
\pgfpathcurveto{\pgfqpoint{1.114619in}{2.328743in}}{\pgfqpoint{1.122519in}{2.332015in}}{\pgfqpoint{1.128343in}{2.337839in}}%
\pgfpathcurveto{\pgfqpoint{1.134167in}{2.343663in}}{\pgfqpoint{1.137439in}{2.351563in}}{\pgfqpoint{1.137439in}{2.359799in}}%
\pgfpathcurveto{\pgfqpoint{1.137439in}{2.368036in}}{\pgfqpoint{1.134167in}{2.375936in}}{\pgfqpoint{1.128343in}{2.381760in}}%
\pgfpathcurveto{\pgfqpoint{1.122519in}{2.387584in}}{\pgfqpoint{1.114619in}{2.390856in}}{\pgfqpoint{1.106383in}{2.390856in}}%
\pgfpathcurveto{\pgfqpoint{1.098147in}{2.390856in}}{\pgfqpoint{1.090247in}{2.387584in}}{\pgfqpoint{1.084423in}{2.381760in}}%
\pgfpathcurveto{\pgfqpoint{1.078599in}{2.375936in}}{\pgfqpoint{1.075326in}{2.368036in}}{\pgfqpoint{1.075326in}{2.359799in}}%
\pgfpathcurveto{\pgfqpoint{1.075326in}{2.351563in}}{\pgfqpoint{1.078599in}{2.343663in}}{\pgfqpoint{1.084423in}{2.337839in}}%
\pgfpathcurveto{\pgfqpoint{1.090247in}{2.332015in}}{\pgfqpoint{1.098147in}{2.328743in}}{\pgfqpoint{1.106383in}{2.328743in}}%
\pgfpathclose%
\pgfusepath{stroke,fill}%
\end{pgfscope}%
\begin{pgfscope}%
\pgfpathrectangle{\pgfqpoint{0.100000in}{0.212622in}}{\pgfqpoint{3.696000in}{3.696000in}}%
\pgfusepath{clip}%
\pgfsetbuttcap%
\pgfsetroundjoin%
\definecolor{currentfill}{rgb}{0.121569,0.466667,0.705882}%
\pgfsetfillcolor{currentfill}%
\pgfsetfillopacity{0.818012}%
\pgfsetlinewidth{1.003750pt}%
\definecolor{currentstroke}{rgb}{0.121569,0.466667,0.705882}%
\pgfsetstrokecolor{currentstroke}%
\pgfsetstrokeopacity{0.818012}%
\pgfsetdash{}{0pt}%
\pgfpathmoveto{\pgfqpoint{2.382591in}{2.707256in}}%
\pgfpathcurveto{\pgfqpoint{2.390827in}{2.707256in}}{\pgfqpoint{2.398727in}{2.710528in}}{\pgfqpoint{2.404551in}{2.716352in}}%
\pgfpathcurveto{\pgfqpoint{2.410375in}{2.722176in}}{\pgfqpoint{2.413647in}{2.730076in}}{\pgfqpoint{2.413647in}{2.738313in}}%
\pgfpathcurveto{\pgfqpoint{2.413647in}{2.746549in}}{\pgfqpoint{2.410375in}{2.754449in}}{\pgfqpoint{2.404551in}{2.760273in}}%
\pgfpathcurveto{\pgfqpoint{2.398727in}{2.766097in}}{\pgfqpoint{2.390827in}{2.769369in}}{\pgfqpoint{2.382591in}{2.769369in}}%
\pgfpathcurveto{\pgfqpoint{2.374355in}{2.769369in}}{\pgfqpoint{2.366455in}{2.766097in}}{\pgfqpoint{2.360631in}{2.760273in}}%
\pgfpathcurveto{\pgfqpoint{2.354807in}{2.754449in}}{\pgfqpoint{2.351534in}{2.746549in}}{\pgfqpoint{2.351534in}{2.738313in}}%
\pgfpathcurveto{\pgfqpoint{2.351534in}{2.730076in}}{\pgfqpoint{2.354807in}{2.722176in}}{\pgfqpoint{2.360631in}{2.716352in}}%
\pgfpathcurveto{\pgfqpoint{2.366455in}{2.710528in}}{\pgfqpoint{2.374355in}{2.707256in}}{\pgfqpoint{2.382591in}{2.707256in}}%
\pgfpathclose%
\pgfusepath{stroke,fill}%
\end{pgfscope}%
\begin{pgfscope}%
\pgfpathrectangle{\pgfqpoint{0.100000in}{0.212622in}}{\pgfqpoint{3.696000in}{3.696000in}}%
\pgfusepath{clip}%
\pgfsetbuttcap%
\pgfsetroundjoin%
\definecolor{currentfill}{rgb}{0.121569,0.466667,0.705882}%
\pgfsetfillcolor{currentfill}%
\pgfsetfillopacity{0.819169}%
\pgfsetlinewidth{1.003750pt}%
\definecolor{currentstroke}{rgb}{0.121569,0.466667,0.705882}%
\pgfsetstrokecolor{currentstroke}%
\pgfsetstrokeopacity{0.819169}%
\pgfsetdash{}{0pt}%
\pgfpathmoveto{\pgfqpoint{2.379958in}{2.704260in}}%
\pgfpathcurveto{\pgfqpoint{2.388194in}{2.704260in}}{\pgfqpoint{2.396094in}{2.707532in}}{\pgfqpoint{2.401918in}{2.713356in}}%
\pgfpathcurveto{\pgfqpoint{2.407742in}{2.719180in}}{\pgfqpoint{2.411015in}{2.727080in}}{\pgfqpoint{2.411015in}{2.735317in}}%
\pgfpathcurveto{\pgfqpoint{2.411015in}{2.743553in}}{\pgfqpoint{2.407742in}{2.751453in}}{\pgfqpoint{2.401918in}{2.757277in}}%
\pgfpathcurveto{\pgfqpoint{2.396094in}{2.763101in}}{\pgfqpoint{2.388194in}{2.766373in}}{\pgfqpoint{2.379958in}{2.766373in}}%
\pgfpathcurveto{\pgfqpoint{2.371722in}{2.766373in}}{\pgfqpoint{2.363822in}{2.763101in}}{\pgfqpoint{2.357998in}{2.757277in}}%
\pgfpathcurveto{\pgfqpoint{2.352174in}{2.751453in}}{\pgfqpoint{2.348902in}{2.743553in}}{\pgfqpoint{2.348902in}{2.735317in}}%
\pgfpathcurveto{\pgfqpoint{2.348902in}{2.727080in}}{\pgfqpoint{2.352174in}{2.719180in}}{\pgfqpoint{2.357998in}{2.713356in}}%
\pgfpathcurveto{\pgfqpoint{2.363822in}{2.707532in}}{\pgfqpoint{2.371722in}{2.704260in}}{\pgfqpoint{2.379958in}{2.704260in}}%
\pgfpathclose%
\pgfusepath{stroke,fill}%
\end{pgfscope}%
\begin{pgfscope}%
\pgfpathrectangle{\pgfqpoint{0.100000in}{0.212622in}}{\pgfqpoint{3.696000in}{3.696000in}}%
\pgfusepath{clip}%
\pgfsetbuttcap%
\pgfsetroundjoin%
\definecolor{currentfill}{rgb}{0.121569,0.466667,0.705882}%
\pgfsetfillcolor{currentfill}%
\pgfsetfillopacity{0.819413}%
\pgfsetlinewidth{1.003750pt}%
\definecolor{currentstroke}{rgb}{0.121569,0.466667,0.705882}%
\pgfsetstrokecolor{currentstroke}%
\pgfsetstrokeopacity{0.819413}%
\pgfsetdash{}{0pt}%
\pgfpathmoveto{\pgfqpoint{1.101466in}{2.321837in}}%
\pgfpathcurveto{\pgfqpoint{1.109702in}{2.321837in}}{\pgfqpoint{1.117602in}{2.325110in}}{\pgfqpoint{1.123426in}{2.330934in}}%
\pgfpathcurveto{\pgfqpoint{1.129250in}{2.336757in}}{\pgfqpoint{1.132523in}{2.344658in}}{\pgfqpoint{1.132523in}{2.352894in}}%
\pgfpathcurveto{\pgfqpoint{1.132523in}{2.361130in}}{\pgfqpoint{1.129250in}{2.369030in}}{\pgfqpoint{1.123426in}{2.374854in}}%
\pgfpathcurveto{\pgfqpoint{1.117602in}{2.380678in}}{\pgfqpoint{1.109702in}{2.383950in}}{\pgfqpoint{1.101466in}{2.383950in}}%
\pgfpathcurveto{\pgfqpoint{1.093230in}{2.383950in}}{\pgfqpoint{1.085330in}{2.380678in}}{\pgfqpoint{1.079506in}{2.374854in}}%
\pgfpathcurveto{\pgfqpoint{1.073682in}{2.369030in}}{\pgfqpoint{1.070410in}{2.361130in}}{\pgfqpoint{1.070410in}{2.352894in}}%
\pgfpathcurveto{\pgfqpoint{1.070410in}{2.344658in}}{\pgfqpoint{1.073682in}{2.336757in}}{\pgfqpoint{1.079506in}{2.330934in}}%
\pgfpathcurveto{\pgfqpoint{1.085330in}{2.325110in}}{\pgfqpoint{1.093230in}{2.321837in}}{\pgfqpoint{1.101466in}{2.321837in}}%
\pgfpathclose%
\pgfusepath{stroke,fill}%
\end{pgfscope}%
\begin{pgfscope}%
\pgfpathrectangle{\pgfqpoint{0.100000in}{0.212622in}}{\pgfqpoint{3.696000in}{3.696000in}}%
\pgfusepath{clip}%
\pgfsetbuttcap%
\pgfsetroundjoin%
\definecolor{currentfill}{rgb}{0.121569,0.466667,0.705882}%
\pgfsetfillcolor{currentfill}%
\pgfsetfillopacity{0.820199}%
\pgfsetlinewidth{1.003750pt}%
\definecolor{currentstroke}{rgb}{0.121569,0.466667,0.705882}%
\pgfsetstrokecolor{currentstroke}%
\pgfsetstrokeopacity{0.820199}%
\pgfsetdash{}{0pt}%
\pgfpathmoveto{\pgfqpoint{2.377548in}{2.701428in}}%
\pgfpathcurveto{\pgfqpoint{2.385784in}{2.701428in}}{\pgfqpoint{2.393684in}{2.704701in}}{\pgfqpoint{2.399508in}{2.710524in}}%
\pgfpathcurveto{\pgfqpoint{2.405332in}{2.716348in}}{\pgfqpoint{2.408604in}{2.724248in}}{\pgfqpoint{2.408604in}{2.732485in}}%
\pgfpathcurveto{\pgfqpoint{2.408604in}{2.740721in}}{\pgfqpoint{2.405332in}{2.748621in}}{\pgfqpoint{2.399508in}{2.754445in}}%
\pgfpathcurveto{\pgfqpoint{2.393684in}{2.760269in}}{\pgfqpoint{2.385784in}{2.763541in}}{\pgfqpoint{2.377548in}{2.763541in}}%
\pgfpathcurveto{\pgfqpoint{2.369312in}{2.763541in}}{\pgfqpoint{2.361412in}{2.760269in}}{\pgfqpoint{2.355588in}{2.754445in}}%
\pgfpathcurveto{\pgfqpoint{2.349764in}{2.748621in}}{\pgfqpoint{2.346491in}{2.740721in}}{\pgfqpoint{2.346491in}{2.732485in}}%
\pgfpathcurveto{\pgfqpoint{2.346491in}{2.724248in}}{\pgfqpoint{2.349764in}{2.716348in}}{\pgfqpoint{2.355588in}{2.710524in}}%
\pgfpathcurveto{\pgfqpoint{2.361412in}{2.704701in}}{\pgfqpoint{2.369312in}{2.701428in}}{\pgfqpoint{2.377548in}{2.701428in}}%
\pgfpathclose%
\pgfusepath{stroke,fill}%
\end{pgfscope}%
\begin{pgfscope}%
\pgfpathrectangle{\pgfqpoint{0.100000in}{0.212622in}}{\pgfqpoint{3.696000in}{3.696000in}}%
\pgfusepath{clip}%
\pgfsetbuttcap%
\pgfsetroundjoin%
\definecolor{currentfill}{rgb}{0.121569,0.466667,0.705882}%
\pgfsetfillcolor{currentfill}%
\pgfsetfillopacity{0.820845}%
\pgfsetlinewidth{1.003750pt}%
\definecolor{currentstroke}{rgb}{0.121569,0.466667,0.705882}%
\pgfsetstrokecolor{currentstroke}%
\pgfsetstrokeopacity{0.820845}%
\pgfsetdash{}{0pt}%
\pgfpathmoveto{\pgfqpoint{2.376037in}{2.699757in}}%
\pgfpathcurveto{\pgfqpoint{2.384273in}{2.699757in}}{\pgfqpoint{2.392173in}{2.703029in}}{\pgfqpoint{2.397997in}{2.708853in}}%
\pgfpathcurveto{\pgfqpoint{2.403821in}{2.714677in}}{\pgfqpoint{2.407093in}{2.722577in}}{\pgfqpoint{2.407093in}{2.730813in}}%
\pgfpathcurveto{\pgfqpoint{2.407093in}{2.739050in}}{\pgfqpoint{2.403821in}{2.746950in}}{\pgfqpoint{2.397997in}{2.752774in}}%
\pgfpathcurveto{\pgfqpoint{2.392173in}{2.758598in}}{\pgfqpoint{2.384273in}{2.761870in}}{\pgfqpoint{2.376037in}{2.761870in}}%
\pgfpathcurveto{\pgfqpoint{2.367801in}{2.761870in}}{\pgfqpoint{2.359901in}{2.758598in}}{\pgfqpoint{2.354077in}{2.752774in}}%
\pgfpathcurveto{\pgfqpoint{2.348253in}{2.746950in}}{\pgfqpoint{2.344980in}{2.739050in}}{\pgfqpoint{2.344980in}{2.730813in}}%
\pgfpathcurveto{\pgfqpoint{2.344980in}{2.722577in}}{\pgfqpoint{2.348253in}{2.714677in}}{\pgfqpoint{2.354077in}{2.708853in}}%
\pgfpathcurveto{\pgfqpoint{2.359901in}{2.703029in}}{\pgfqpoint{2.367801in}{2.699757in}}{\pgfqpoint{2.376037in}{2.699757in}}%
\pgfpathclose%
\pgfusepath{stroke,fill}%
\end{pgfscope}%
\begin{pgfscope}%
\pgfpathrectangle{\pgfqpoint{0.100000in}{0.212622in}}{\pgfqpoint{3.696000in}{3.696000in}}%
\pgfusepath{clip}%
\pgfsetbuttcap%
\pgfsetroundjoin%
\definecolor{currentfill}{rgb}{0.121569,0.466667,0.705882}%
\pgfsetfillcolor{currentfill}%
\pgfsetfillopacity{0.821168}%
\pgfsetlinewidth{1.003750pt}%
\definecolor{currentstroke}{rgb}{0.121569,0.466667,0.705882}%
\pgfsetstrokecolor{currentstroke}%
\pgfsetstrokeopacity{0.821168}%
\pgfsetdash{}{0pt}%
\pgfpathmoveto{\pgfqpoint{1.096004in}{2.314460in}}%
\pgfpathcurveto{\pgfqpoint{1.104240in}{2.314460in}}{\pgfqpoint{1.112140in}{2.317733in}}{\pgfqpoint{1.117964in}{2.323557in}}%
\pgfpathcurveto{\pgfqpoint{1.123788in}{2.329381in}}{\pgfqpoint{1.127060in}{2.337281in}}{\pgfqpoint{1.127060in}{2.345517in}}%
\pgfpathcurveto{\pgfqpoint{1.127060in}{2.353753in}}{\pgfqpoint{1.123788in}{2.361653in}}{\pgfqpoint{1.117964in}{2.367477in}}%
\pgfpathcurveto{\pgfqpoint{1.112140in}{2.373301in}}{\pgfqpoint{1.104240in}{2.376573in}}{\pgfqpoint{1.096004in}{2.376573in}}%
\pgfpathcurveto{\pgfqpoint{1.087767in}{2.376573in}}{\pgfqpoint{1.079867in}{2.373301in}}{\pgfqpoint{1.074043in}{2.367477in}}%
\pgfpathcurveto{\pgfqpoint{1.068220in}{2.361653in}}{\pgfqpoint{1.064947in}{2.353753in}}{\pgfqpoint{1.064947in}{2.345517in}}%
\pgfpathcurveto{\pgfqpoint{1.064947in}{2.337281in}}{\pgfqpoint{1.068220in}{2.329381in}}{\pgfqpoint{1.074043in}{2.323557in}}%
\pgfpathcurveto{\pgfqpoint{1.079867in}{2.317733in}}{\pgfqpoint{1.087767in}{2.314460in}}{\pgfqpoint{1.096004in}{2.314460in}}%
\pgfpathclose%
\pgfusepath{stroke,fill}%
\end{pgfscope}%
\begin{pgfscope}%
\pgfpathrectangle{\pgfqpoint{0.100000in}{0.212622in}}{\pgfqpoint{3.696000in}{3.696000in}}%
\pgfusepath{clip}%
\pgfsetbuttcap%
\pgfsetroundjoin%
\definecolor{currentfill}{rgb}{0.121569,0.466667,0.705882}%
\pgfsetfillcolor{currentfill}%
\pgfsetfillopacity{0.822046}%
\pgfsetlinewidth{1.003750pt}%
\definecolor{currentstroke}{rgb}{0.121569,0.466667,0.705882}%
\pgfsetstrokecolor{currentstroke}%
\pgfsetstrokeopacity{0.822046}%
\pgfsetdash{}{0pt}%
\pgfpathmoveto{\pgfqpoint{2.373298in}{2.696861in}}%
\pgfpathcurveto{\pgfqpoint{2.381534in}{2.696861in}}{\pgfqpoint{2.389435in}{2.700134in}}{\pgfqpoint{2.395258in}{2.705958in}}%
\pgfpathcurveto{\pgfqpoint{2.401082in}{2.711782in}}{\pgfqpoint{2.404355in}{2.719682in}}{\pgfqpoint{2.404355in}{2.727918in}}%
\pgfpathcurveto{\pgfqpoint{2.404355in}{2.736154in}}{\pgfqpoint{2.401082in}{2.744054in}}{\pgfqpoint{2.395258in}{2.749878in}}%
\pgfpathcurveto{\pgfqpoint{2.389435in}{2.755702in}}{\pgfqpoint{2.381534in}{2.758974in}}{\pgfqpoint{2.373298in}{2.758974in}}%
\pgfpathcurveto{\pgfqpoint{2.365062in}{2.758974in}}{\pgfqpoint{2.357162in}{2.755702in}}{\pgfqpoint{2.351338in}{2.749878in}}%
\pgfpathcurveto{\pgfqpoint{2.345514in}{2.744054in}}{\pgfqpoint{2.342242in}{2.736154in}}{\pgfqpoint{2.342242in}{2.727918in}}%
\pgfpathcurveto{\pgfqpoint{2.342242in}{2.719682in}}{\pgfqpoint{2.345514in}{2.711782in}}{\pgfqpoint{2.351338in}{2.705958in}}%
\pgfpathcurveto{\pgfqpoint{2.357162in}{2.700134in}}{\pgfqpoint{2.365062in}{2.696861in}}{\pgfqpoint{2.373298in}{2.696861in}}%
\pgfpathclose%
\pgfusepath{stroke,fill}%
\end{pgfscope}%
\begin{pgfscope}%
\pgfpathrectangle{\pgfqpoint{0.100000in}{0.212622in}}{\pgfqpoint{3.696000in}{3.696000in}}%
\pgfusepath{clip}%
\pgfsetbuttcap%
\pgfsetroundjoin%
\definecolor{currentfill}{rgb}{0.121569,0.466667,0.705882}%
\pgfsetfillcolor{currentfill}%
\pgfsetfillopacity{0.822973}%
\pgfsetlinewidth{1.003750pt}%
\definecolor{currentstroke}{rgb}{0.121569,0.466667,0.705882}%
\pgfsetstrokecolor{currentstroke}%
\pgfsetstrokeopacity{0.822973}%
\pgfsetdash{}{0pt}%
\pgfpathmoveto{\pgfqpoint{2.371008in}{2.694011in}}%
\pgfpathcurveto{\pgfqpoint{2.379245in}{2.694011in}}{\pgfqpoint{2.387145in}{2.697283in}}{\pgfqpoint{2.392969in}{2.703107in}}%
\pgfpathcurveto{\pgfqpoint{2.398793in}{2.708931in}}{\pgfqpoint{2.402065in}{2.716831in}}{\pgfqpoint{2.402065in}{2.725068in}}%
\pgfpathcurveto{\pgfqpoint{2.402065in}{2.733304in}}{\pgfqpoint{2.398793in}{2.741204in}}{\pgfqpoint{2.392969in}{2.747028in}}%
\pgfpathcurveto{\pgfqpoint{2.387145in}{2.752852in}}{\pgfqpoint{2.379245in}{2.756124in}}{\pgfqpoint{2.371008in}{2.756124in}}%
\pgfpathcurveto{\pgfqpoint{2.362772in}{2.756124in}}{\pgfqpoint{2.354872in}{2.752852in}}{\pgfqpoint{2.349048in}{2.747028in}}%
\pgfpathcurveto{\pgfqpoint{2.343224in}{2.741204in}}{\pgfqpoint{2.339952in}{2.733304in}}{\pgfqpoint{2.339952in}{2.725068in}}%
\pgfpathcurveto{\pgfqpoint{2.339952in}{2.716831in}}{\pgfqpoint{2.343224in}{2.708931in}}{\pgfqpoint{2.349048in}{2.703107in}}%
\pgfpathcurveto{\pgfqpoint{2.354872in}{2.697283in}}{\pgfqpoint{2.362772in}{2.694011in}}{\pgfqpoint{2.371008in}{2.694011in}}%
\pgfpathclose%
\pgfusepath{stroke,fill}%
\end{pgfscope}%
\begin{pgfscope}%
\pgfpathrectangle{\pgfqpoint{0.100000in}{0.212622in}}{\pgfqpoint{3.696000in}{3.696000in}}%
\pgfusepath{clip}%
\pgfsetbuttcap%
\pgfsetroundjoin%
\definecolor{currentfill}{rgb}{0.121569,0.466667,0.705882}%
\pgfsetfillcolor{currentfill}%
\pgfsetfillopacity{0.823143}%
\pgfsetlinewidth{1.003750pt}%
\definecolor{currentstroke}{rgb}{0.121569,0.466667,0.705882}%
\pgfsetstrokecolor{currentstroke}%
\pgfsetstrokeopacity{0.823143}%
\pgfsetdash{}{0pt}%
\pgfpathmoveto{\pgfqpoint{1.090342in}{2.307592in}}%
\pgfpathcurveto{\pgfqpoint{1.098579in}{2.307592in}}{\pgfqpoint{1.106479in}{2.310864in}}{\pgfqpoint{1.112302in}{2.316688in}}%
\pgfpathcurveto{\pgfqpoint{1.118126in}{2.322512in}}{\pgfqpoint{1.121399in}{2.330412in}}{\pgfqpoint{1.121399in}{2.338649in}}%
\pgfpathcurveto{\pgfqpoint{1.121399in}{2.346885in}}{\pgfqpoint{1.118126in}{2.354785in}}{\pgfqpoint{1.112302in}{2.360609in}}%
\pgfpathcurveto{\pgfqpoint{1.106479in}{2.366433in}}{\pgfqpoint{1.098579in}{2.369705in}}{\pgfqpoint{1.090342in}{2.369705in}}%
\pgfpathcurveto{\pgfqpoint{1.082106in}{2.369705in}}{\pgfqpoint{1.074206in}{2.366433in}}{\pgfqpoint{1.068382in}{2.360609in}}%
\pgfpathcurveto{\pgfqpoint{1.062558in}{2.354785in}}{\pgfqpoint{1.059286in}{2.346885in}}{\pgfqpoint{1.059286in}{2.338649in}}%
\pgfpathcurveto{\pgfqpoint{1.059286in}{2.330412in}}{\pgfqpoint{1.062558in}{2.322512in}}{\pgfqpoint{1.068382in}{2.316688in}}%
\pgfpathcurveto{\pgfqpoint{1.074206in}{2.310864in}}{\pgfqpoint{1.082106in}{2.307592in}}{\pgfqpoint{1.090342in}{2.307592in}}%
\pgfpathclose%
\pgfusepath{stroke,fill}%
\end{pgfscope}%
\begin{pgfscope}%
\pgfpathrectangle{\pgfqpoint{0.100000in}{0.212622in}}{\pgfqpoint{3.696000in}{3.696000in}}%
\pgfusepath{clip}%
\pgfsetbuttcap%
\pgfsetroundjoin%
\definecolor{currentfill}{rgb}{0.121569,0.466667,0.705882}%
\pgfsetfillcolor{currentfill}%
\pgfsetfillopacity{0.823550}%
\pgfsetlinewidth{1.003750pt}%
\definecolor{currentstroke}{rgb}{0.121569,0.466667,0.705882}%
\pgfsetstrokecolor{currentstroke}%
\pgfsetstrokeopacity{0.823550}%
\pgfsetdash{}{0pt}%
\pgfpathmoveto{\pgfqpoint{2.829225in}{1.497651in}}%
\pgfpathcurveto{\pgfqpoint{2.837462in}{1.497651in}}{\pgfqpoint{2.845362in}{1.500923in}}{\pgfqpoint{2.851186in}{1.506747in}}%
\pgfpathcurveto{\pgfqpoint{2.857010in}{1.512571in}}{\pgfqpoint{2.860282in}{1.520471in}}{\pgfqpoint{2.860282in}{1.528707in}}%
\pgfpathcurveto{\pgfqpoint{2.860282in}{1.536943in}}{\pgfqpoint{2.857010in}{1.544844in}}{\pgfqpoint{2.851186in}{1.550667in}}%
\pgfpathcurveto{\pgfqpoint{2.845362in}{1.556491in}}{\pgfqpoint{2.837462in}{1.559764in}}{\pgfqpoint{2.829225in}{1.559764in}}%
\pgfpathcurveto{\pgfqpoint{2.820989in}{1.559764in}}{\pgfqpoint{2.813089in}{1.556491in}}{\pgfqpoint{2.807265in}{1.550667in}}%
\pgfpathcurveto{\pgfqpoint{2.801441in}{1.544844in}}{\pgfqpoint{2.798169in}{1.536943in}}{\pgfqpoint{2.798169in}{1.528707in}}%
\pgfpathcurveto{\pgfqpoint{2.798169in}{1.520471in}}{\pgfqpoint{2.801441in}{1.512571in}}{\pgfqpoint{2.807265in}{1.506747in}}%
\pgfpathcurveto{\pgfqpoint{2.813089in}{1.500923in}}{\pgfqpoint{2.820989in}{1.497651in}}{\pgfqpoint{2.829225in}{1.497651in}}%
\pgfpathclose%
\pgfusepath{stroke,fill}%
\end{pgfscope}%
\begin{pgfscope}%
\pgfpathrectangle{\pgfqpoint{0.100000in}{0.212622in}}{\pgfqpoint{3.696000in}{3.696000in}}%
\pgfusepath{clip}%
\pgfsetbuttcap%
\pgfsetroundjoin%
\definecolor{currentfill}{rgb}{0.121569,0.466667,0.705882}%
\pgfsetfillcolor{currentfill}%
\pgfsetfillopacity{0.824618}%
\pgfsetlinewidth{1.003750pt}%
\definecolor{currentstroke}{rgb}{0.121569,0.466667,0.705882}%
\pgfsetstrokecolor{currentstroke}%
\pgfsetstrokeopacity{0.824618}%
\pgfsetdash{}{0pt}%
\pgfpathmoveto{\pgfqpoint{2.366733in}{2.688698in}}%
\pgfpathcurveto{\pgfqpoint{2.374970in}{2.688698in}}{\pgfqpoint{2.382870in}{2.691970in}}{\pgfqpoint{2.388694in}{2.697794in}}%
\pgfpathcurveto{\pgfqpoint{2.394518in}{2.703618in}}{\pgfqpoint{2.397790in}{2.711518in}}{\pgfqpoint{2.397790in}{2.719754in}}%
\pgfpathcurveto{\pgfqpoint{2.397790in}{2.727991in}}{\pgfqpoint{2.394518in}{2.735891in}}{\pgfqpoint{2.388694in}{2.741715in}}%
\pgfpathcurveto{\pgfqpoint{2.382870in}{2.747538in}}{\pgfqpoint{2.374970in}{2.750811in}}{\pgfqpoint{2.366733in}{2.750811in}}%
\pgfpathcurveto{\pgfqpoint{2.358497in}{2.750811in}}{\pgfqpoint{2.350597in}{2.747538in}}{\pgfqpoint{2.344773in}{2.741715in}}%
\pgfpathcurveto{\pgfqpoint{2.338949in}{2.735891in}}{\pgfqpoint{2.335677in}{2.727991in}}{\pgfqpoint{2.335677in}{2.719754in}}%
\pgfpathcurveto{\pgfqpoint{2.335677in}{2.711518in}}{\pgfqpoint{2.338949in}{2.703618in}}{\pgfqpoint{2.344773in}{2.697794in}}%
\pgfpathcurveto{\pgfqpoint{2.350597in}{2.691970in}}{\pgfqpoint{2.358497in}{2.688698in}}{\pgfqpoint{2.366733in}{2.688698in}}%
\pgfpathclose%
\pgfusepath{stroke,fill}%
\end{pgfscope}%
\begin{pgfscope}%
\pgfpathrectangle{\pgfqpoint{0.100000in}{0.212622in}}{\pgfqpoint{3.696000in}{3.696000in}}%
\pgfusepath{clip}%
\pgfsetbuttcap%
\pgfsetroundjoin%
\definecolor{currentfill}{rgb}{0.121569,0.466667,0.705882}%
\pgfsetfillcolor{currentfill}%
\pgfsetfillopacity{0.825480}%
\pgfsetlinewidth{1.003750pt}%
\definecolor{currentstroke}{rgb}{0.121569,0.466667,0.705882}%
\pgfsetstrokecolor{currentstroke}%
\pgfsetstrokeopacity{0.825480}%
\pgfsetdash{}{0pt}%
\pgfpathmoveto{\pgfqpoint{1.084142in}{2.301584in}}%
\pgfpathcurveto{\pgfqpoint{1.092378in}{2.301584in}}{\pgfqpoint{1.100278in}{2.304856in}}{\pgfqpoint{1.106102in}{2.310680in}}%
\pgfpathcurveto{\pgfqpoint{1.111926in}{2.316504in}}{\pgfqpoint{1.115198in}{2.324404in}}{\pgfqpoint{1.115198in}{2.332641in}}%
\pgfpathcurveto{\pgfqpoint{1.115198in}{2.340877in}}{\pgfqpoint{1.111926in}{2.348777in}}{\pgfqpoint{1.106102in}{2.354601in}}%
\pgfpathcurveto{\pgfqpoint{1.100278in}{2.360425in}}{\pgfqpoint{1.092378in}{2.363697in}}{\pgfqpoint{1.084142in}{2.363697in}}%
\pgfpathcurveto{\pgfqpoint{1.075905in}{2.363697in}}{\pgfqpoint{1.068005in}{2.360425in}}{\pgfqpoint{1.062181in}{2.354601in}}%
\pgfpathcurveto{\pgfqpoint{1.056357in}{2.348777in}}{\pgfqpoint{1.053085in}{2.340877in}}{\pgfqpoint{1.053085in}{2.332641in}}%
\pgfpathcurveto{\pgfqpoint{1.053085in}{2.324404in}}{\pgfqpoint{1.056357in}{2.316504in}}{\pgfqpoint{1.062181in}{2.310680in}}%
\pgfpathcurveto{\pgfqpoint{1.068005in}{2.304856in}}{\pgfqpoint{1.075905in}{2.301584in}}{\pgfqpoint{1.084142in}{2.301584in}}%
\pgfpathclose%
\pgfusepath{stroke,fill}%
\end{pgfscope}%
\begin{pgfscope}%
\pgfpathrectangle{\pgfqpoint{0.100000in}{0.212622in}}{\pgfqpoint{3.696000in}{3.696000in}}%
\pgfusepath{clip}%
\pgfsetbuttcap%
\pgfsetroundjoin%
\definecolor{currentfill}{rgb}{0.121569,0.466667,0.705882}%
\pgfsetfillcolor{currentfill}%
\pgfsetfillopacity{0.826058}%
\pgfsetlinewidth{1.003750pt}%
\definecolor{currentstroke}{rgb}{0.121569,0.466667,0.705882}%
\pgfsetstrokecolor{currentstroke}%
\pgfsetstrokeopacity{0.826058}%
\pgfsetdash{}{0pt}%
\pgfpathmoveto{\pgfqpoint{2.363296in}{2.684807in}}%
\pgfpathcurveto{\pgfqpoint{2.371532in}{2.684807in}}{\pgfqpoint{2.379432in}{2.688079in}}{\pgfqpoint{2.385256in}{2.693903in}}%
\pgfpathcurveto{\pgfqpoint{2.391080in}{2.699727in}}{\pgfqpoint{2.394352in}{2.707627in}}{\pgfqpoint{2.394352in}{2.715864in}}%
\pgfpathcurveto{\pgfqpoint{2.394352in}{2.724100in}}{\pgfqpoint{2.391080in}{2.732000in}}{\pgfqpoint{2.385256in}{2.737824in}}%
\pgfpathcurveto{\pgfqpoint{2.379432in}{2.743648in}}{\pgfqpoint{2.371532in}{2.746920in}}{\pgfqpoint{2.363296in}{2.746920in}}%
\pgfpathcurveto{\pgfqpoint{2.355060in}{2.746920in}}{\pgfqpoint{2.347160in}{2.743648in}}{\pgfqpoint{2.341336in}{2.737824in}}%
\pgfpathcurveto{\pgfqpoint{2.335512in}{2.732000in}}{\pgfqpoint{2.332239in}{2.724100in}}{\pgfqpoint{2.332239in}{2.715864in}}%
\pgfpathcurveto{\pgfqpoint{2.332239in}{2.707627in}}{\pgfqpoint{2.335512in}{2.699727in}}{\pgfqpoint{2.341336in}{2.693903in}}%
\pgfpathcurveto{\pgfqpoint{2.347160in}{2.688079in}}{\pgfqpoint{2.355060in}{2.684807in}}{\pgfqpoint{2.363296in}{2.684807in}}%
\pgfpathclose%
\pgfusepath{stroke,fill}%
\end{pgfscope}%
\begin{pgfscope}%
\pgfpathrectangle{\pgfqpoint{0.100000in}{0.212622in}}{\pgfqpoint{3.696000in}{3.696000in}}%
\pgfusepath{clip}%
\pgfsetbuttcap%
\pgfsetroundjoin%
\definecolor{currentfill}{rgb}{0.121569,0.466667,0.705882}%
\pgfsetfillcolor{currentfill}%
\pgfsetfillopacity{0.827452}%
\pgfsetlinewidth{1.003750pt}%
\definecolor{currentstroke}{rgb}{0.121569,0.466667,0.705882}%
\pgfsetstrokecolor{currentstroke}%
\pgfsetstrokeopacity{0.827452}%
\pgfsetdash{}{0pt}%
\pgfpathmoveto{\pgfqpoint{2.360201in}{2.681482in}}%
\pgfpathcurveto{\pgfqpoint{2.368437in}{2.681482in}}{\pgfqpoint{2.376337in}{2.684755in}}{\pgfqpoint{2.382161in}{2.690578in}}%
\pgfpathcurveto{\pgfqpoint{2.387985in}{2.696402in}}{\pgfqpoint{2.391257in}{2.704302in}}{\pgfqpoint{2.391257in}{2.712539in}}%
\pgfpathcurveto{\pgfqpoint{2.391257in}{2.720775in}}{\pgfqpoint{2.387985in}{2.728675in}}{\pgfqpoint{2.382161in}{2.734499in}}%
\pgfpathcurveto{\pgfqpoint{2.376337in}{2.740323in}}{\pgfqpoint{2.368437in}{2.743595in}}{\pgfqpoint{2.360201in}{2.743595in}}%
\pgfpathcurveto{\pgfqpoint{2.351965in}{2.743595in}}{\pgfqpoint{2.344064in}{2.740323in}}{\pgfqpoint{2.338241in}{2.734499in}}%
\pgfpathcurveto{\pgfqpoint{2.332417in}{2.728675in}}{\pgfqpoint{2.329144in}{2.720775in}}{\pgfqpoint{2.329144in}{2.712539in}}%
\pgfpathcurveto{\pgfqpoint{2.329144in}{2.704302in}}{\pgfqpoint{2.332417in}{2.696402in}}{\pgfqpoint{2.338241in}{2.690578in}}%
\pgfpathcurveto{\pgfqpoint{2.344064in}{2.684755in}}{\pgfqpoint{2.351965in}{2.681482in}}{\pgfqpoint{2.360201in}{2.681482in}}%
\pgfpathclose%
\pgfusepath{stroke,fill}%
\end{pgfscope}%
\begin{pgfscope}%
\pgfpathrectangle{\pgfqpoint{0.100000in}{0.212622in}}{\pgfqpoint{3.696000in}{3.696000in}}%
\pgfusepath{clip}%
\pgfsetbuttcap%
\pgfsetroundjoin%
\definecolor{currentfill}{rgb}{0.121569,0.466667,0.705882}%
\pgfsetfillcolor{currentfill}%
\pgfsetfillopacity{0.828114}%
\pgfsetlinewidth{1.003750pt}%
\definecolor{currentstroke}{rgb}{0.121569,0.466667,0.705882}%
\pgfsetstrokecolor{currentstroke}%
\pgfsetstrokeopacity{0.828114}%
\pgfsetdash{}{0pt}%
\pgfpathmoveto{\pgfqpoint{1.076808in}{2.293972in}}%
\pgfpathcurveto{\pgfqpoint{1.085044in}{2.293972in}}{\pgfqpoint{1.092944in}{2.297245in}}{\pgfqpoint{1.098768in}{2.303068in}}%
\pgfpathcurveto{\pgfqpoint{1.104592in}{2.308892in}}{\pgfqpoint{1.107864in}{2.316792in}}{\pgfqpoint{1.107864in}{2.325029in}}%
\pgfpathcurveto{\pgfqpoint{1.107864in}{2.333265in}}{\pgfqpoint{1.104592in}{2.341165in}}{\pgfqpoint{1.098768in}{2.346989in}}%
\pgfpathcurveto{\pgfqpoint{1.092944in}{2.352813in}}{\pgfqpoint{1.085044in}{2.356085in}}{\pgfqpoint{1.076808in}{2.356085in}}%
\pgfpathcurveto{\pgfqpoint{1.068571in}{2.356085in}}{\pgfqpoint{1.060671in}{2.352813in}}{\pgfqpoint{1.054847in}{2.346989in}}%
\pgfpathcurveto{\pgfqpoint{1.049024in}{2.341165in}}{\pgfqpoint{1.045751in}{2.333265in}}{\pgfqpoint{1.045751in}{2.325029in}}%
\pgfpathcurveto{\pgfqpoint{1.045751in}{2.316792in}}{\pgfqpoint{1.049024in}{2.308892in}}{\pgfqpoint{1.054847in}{2.303068in}}%
\pgfpathcurveto{\pgfqpoint{1.060671in}{2.297245in}}{\pgfqpoint{1.068571in}{2.293972in}}{\pgfqpoint{1.076808in}{2.293972in}}%
\pgfpathclose%
\pgfusepath{stroke,fill}%
\end{pgfscope}%
\begin{pgfscope}%
\pgfpathrectangle{\pgfqpoint{0.100000in}{0.212622in}}{\pgfqpoint{3.696000in}{3.696000in}}%
\pgfusepath{clip}%
\pgfsetbuttcap%
\pgfsetroundjoin%
\definecolor{currentfill}{rgb}{0.121569,0.466667,0.705882}%
\pgfsetfillcolor{currentfill}%
\pgfsetfillopacity{0.828529}%
\pgfsetlinewidth{1.003750pt}%
\definecolor{currentstroke}{rgb}{0.121569,0.466667,0.705882}%
\pgfsetstrokecolor{currentstroke}%
\pgfsetstrokeopacity{0.828529}%
\pgfsetdash{}{0pt}%
\pgfpathmoveto{\pgfqpoint{2.357576in}{2.678158in}}%
\pgfpathcurveto{\pgfqpoint{2.365812in}{2.678158in}}{\pgfqpoint{2.373712in}{2.681430in}}{\pgfqpoint{2.379536in}{2.687254in}}%
\pgfpathcurveto{\pgfqpoint{2.385360in}{2.693078in}}{\pgfqpoint{2.388632in}{2.700978in}}{\pgfqpoint{2.388632in}{2.709215in}}%
\pgfpathcurveto{\pgfqpoint{2.388632in}{2.717451in}}{\pgfqpoint{2.385360in}{2.725351in}}{\pgfqpoint{2.379536in}{2.731175in}}%
\pgfpathcurveto{\pgfqpoint{2.373712in}{2.736999in}}{\pgfqpoint{2.365812in}{2.740271in}}{\pgfqpoint{2.357576in}{2.740271in}}%
\pgfpathcurveto{\pgfqpoint{2.349340in}{2.740271in}}{\pgfqpoint{2.341440in}{2.736999in}}{\pgfqpoint{2.335616in}{2.731175in}}%
\pgfpathcurveto{\pgfqpoint{2.329792in}{2.725351in}}{\pgfqpoint{2.326519in}{2.717451in}}{\pgfqpoint{2.326519in}{2.709215in}}%
\pgfpathcurveto{\pgfqpoint{2.326519in}{2.700978in}}{\pgfqpoint{2.329792in}{2.693078in}}{\pgfqpoint{2.335616in}{2.687254in}}%
\pgfpathcurveto{\pgfqpoint{2.341440in}{2.681430in}}{\pgfqpoint{2.349340in}{2.678158in}}{\pgfqpoint{2.357576in}{2.678158in}}%
\pgfpathclose%
\pgfusepath{stroke,fill}%
\end{pgfscope}%
\begin{pgfscope}%
\pgfpathrectangle{\pgfqpoint{0.100000in}{0.212622in}}{\pgfqpoint{3.696000in}{3.696000in}}%
\pgfusepath{clip}%
\pgfsetbuttcap%
\pgfsetroundjoin%
\definecolor{currentfill}{rgb}{0.121569,0.466667,0.705882}%
\pgfsetfillcolor{currentfill}%
\pgfsetfillopacity{0.829468}%
\pgfsetlinewidth{1.003750pt}%
\definecolor{currentstroke}{rgb}{0.121569,0.466667,0.705882}%
\pgfsetstrokecolor{currentstroke}%
\pgfsetstrokeopacity{0.829468}%
\pgfsetdash{}{0pt}%
\pgfpathmoveto{\pgfqpoint{2.355193in}{2.675102in}}%
\pgfpathcurveto{\pgfqpoint{2.363429in}{2.675102in}}{\pgfqpoint{2.371329in}{2.678374in}}{\pgfqpoint{2.377153in}{2.684198in}}%
\pgfpathcurveto{\pgfqpoint{2.382977in}{2.690022in}}{\pgfqpoint{2.386250in}{2.697922in}}{\pgfqpoint{2.386250in}{2.706158in}}%
\pgfpathcurveto{\pgfqpoint{2.386250in}{2.714394in}}{\pgfqpoint{2.382977in}{2.722294in}}{\pgfqpoint{2.377153in}{2.728118in}}%
\pgfpathcurveto{\pgfqpoint{2.371329in}{2.733942in}}{\pgfqpoint{2.363429in}{2.737215in}}{\pgfqpoint{2.355193in}{2.737215in}}%
\pgfpathcurveto{\pgfqpoint{2.346957in}{2.737215in}}{\pgfqpoint{2.339057in}{2.733942in}}{\pgfqpoint{2.333233in}{2.728118in}}%
\pgfpathcurveto{\pgfqpoint{2.327409in}{2.722294in}}{\pgfqpoint{2.324137in}{2.714394in}}{\pgfqpoint{2.324137in}{2.706158in}}%
\pgfpathcurveto{\pgfqpoint{2.324137in}{2.697922in}}{\pgfqpoint{2.327409in}{2.690022in}}{\pgfqpoint{2.333233in}{2.684198in}}%
\pgfpathcurveto{\pgfqpoint{2.339057in}{2.678374in}}{\pgfqpoint{2.346957in}{2.675102in}}{\pgfqpoint{2.355193in}{2.675102in}}%
\pgfpathclose%
\pgfusepath{stroke,fill}%
\end{pgfscope}%
\begin{pgfscope}%
\pgfpathrectangle{\pgfqpoint{0.100000in}{0.212622in}}{\pgfqpoint{3.696000in}{3.696000in}}%
\pgfusepath{clip}%
\pgfsetbuttcap%
\pgfsetroundjoin%
\definecolor{currentfill}{rgb}{0.121569,0.466667,0.705882}%
\pgfsetfillcolor{currentfill}%
\pgfsetfillopacity{0.830064}%
\pgfsetlinewidth{1.003750pt}%
\definecolor{currentstroke}{rgb}{0.121569,0.466667,0.705882}%
\pgfsetstrokecolor{currentstroke}%
\pgfsetstrokeopacity{0.830064}%
\pgfsetdash{}{0pt}%
\pgfpathmoveto{\pgfqpoint{2.353754in}{2.673533in}}%
\pgfpathcurveto{\pgfqpoint{2.361990in}{2.673533in}}{\pgfqpoint{2.369890in}{2.676805in}}{\pgfqpoint{2.375714in}{2.682629in}}%
\pgfpathcurveto{\pgfqpoint{2.381538in}{2.688453in}}{\pgfqpoint{2.384810in}{2.696353in}}{\pgfqpoint{2.384810in}{2.704589in}}%
\pgfpathcurveto{\pgfqpoint{2.384810in}{2.712825in}}{\pgfqpoint{2.381538in}{2.720725in}}{\pgfqpoint{2.375714in}{2.726549in}}%
\pgfpathcurveto{\pgfqpoint{2.369890in}{2.732373in}}{\pgfqpoint{2.361990in}{2.735646in}}{\pgfqpoint{2.353754in}{2.735646in}}%
\pgfpathcurveto{\pgfqpoint{2.345517in}{2.735646in}}{\pgfqpoint{2.337617in}{2.732373in}}{\pgfqpoint{2.331793in}{2.726549in}}%
\pgfpathcurveto{\pgfqpoint{2.325969in}{2.720725in}}{\pgfqpoint{2.322697in}{2.712825in}}{\pgfqpoint{2.322697in}{2.704589in}}%
\pgfpathcurveto{\pgfqpoint{2.322697in}{2.696353in}}{\pgfqpoint{2.325969in}{2.688453in}}{\pgfqpoint{2.331793in}{2.682629in}}%
\pgfpathcurveto{\pgfqpoint{2.337617in}{2.676805in}}{\pgfqpoint{2.345517in}{2.673533in}}{\pgfqpoint{2.353754in}{2.673533in}}%
\pgfpathclose%
\pgfusepath{stroke,fill}%
\end{pgfscope}%
\begin{pgfscope}%
\pgfpathrectangle{\pgfqpoint{0.100000in}{0.212622in}}{\pgfqpoint{3.696000in}{3.696000in}}%
\pgfusepath{clip}%
\pgfsetbuttcap%
\pgfsetroundjoin%
\definecolor{currentfill}{rgb}{0.121569,0.466667,0.705882}%
\pgfsetfillcolor{currentfill}%
\pgfsetfillopacity{0.831041}%
\pgfsetlinewidth{1.003750pt}%
\definecolor{currentstroke}{rgb}{0.121569,0.466667,0.705882}%
\pgfsetstrokecolor{currentstroke}%
\pgfsetstrokeopacity{0.831041}%
\pgfsetdash{}{0pt}%
\pgfpathmoveto{\pgfqpoint{1.068123in}{2.283906in}}%
\pgfpathcurveto{\pgfqpoint{1.076359in}{2.283906in}}{\pgfqpoint{1.084259in}{2.287178in}}{\pgfqpoint{1.090083in}{2.293002in}}%
\pgfpathcurveto{\pgfqpoint{1.095907in}{2.298826in}}{\pgfqpoint{1.099179in}{2.306726in}}{\pgfqpoint{1.099179in}{2.314963in}}%
\pgfpathcurveto{\pgfqpoint{1.099179in}{2.323199in}}{\pgfqpoint{1.095907in}{2.331099in}}{\pgfqpoint{1.090083in}{2.336923in}}%
\pgfpathcurveto{\pgfqpoint{1.084259in}{2.342747in}}{\pgfqpoint{1.076359in}{2.346019in}}{\pgfqpoint{1.068123in}{2.346019in}}%
\pgfpathcurveto{\pgfqpoint{1.059887in}{2.346019in}}{\pgfqpoint{1.051986in}{2.342747in}}{\pgfqpoint{1.046163in}{2.336923in}}%
\pgfpathcurveto{\pgfqpoint{1.040339in}{2.331099in}}{\pgfqpoint{1.037066in}{2.323199in}}{\pgfqpoint{1.037066in}{2.314963in}}%
\pgfpathcurveto{\pgfqpoint{1.037066in}{2.306726in}}{\pgfqpoint{1.040339in}{2.298826in}}{\pgfqpoint{1.046163in}{2.293002in}}%
\pgfpathcurveto{\pgfqpoint{1.051986in}{2.287178in}}{\pgfqpoint{1.059887in}{2.283906in}}{\pgfqpoint{1.068123in}{2.283906in}}%
\pgfpathclose%
\pgfusepath{stroke,fill}%
\end{pgfscope}%
\begin{pgfscope}%
\pgfpathrectangle{\pgfqpoint{0.100000in}{0.212622in}}{\pgfqpoint{3.696000in}{3.696000in}}%
\pgfusepath{clip}%
\pgfsetbuttcap%
\pgfsetroundjoin%
\definecolor{currentfill}{rgb}{0.121569,0.466667,0.705882}%
\pgfsetfillcolor{currentfill}%
\pgfsetfillopacity{0.831189}%
\pgfsetlinewidth{1.003750pt}%
\definecolor{currentstroke}{rgb}{0.121569,0.466667,0.705882}%
\pgfsetstrokecolor{currentstroke}%
\pgfsetstrokeopacity{0.831189}%
\pgfsetdash{}{0pt}%
\pgfpathmoveto{\pgfqpoint{2.351185in}{2.670866in}}%
\pgfpathcurveto{\pgfqpoint{2.359421in}{2.670866in}}{\pgfqpoint{2.367321in}{2.674139in}}{\pgfqpoint{2.373145in}{2.679962in}}%
\pgfpathcurveto{\pgfqpoint{2.378969in}{2.685786in}}{\pgfqpoint{2.382242in}{2.693686in}}{\pgfqpoint{2.382242in}{2.701923in}}%
\pgfpathcurveto{\pgfqpoint{2.382242in}{2.710159in}}{\pgfqpoint{2.378969in}{2.718059in}}{\pgfqpoint{2.373145in}{2.723883in}}%
\pgfpathcurveto{\pgfqpoint{2.367321in}{2.729707in}}{\pgfqpoint{2.359421in}{2.732979in}}{\pgfqpoint{2.351185in}{2.732979in}}%
\pgfpathcurveto{\pgfqpoint{2.342949in}{2.732979in}}{\pgfqpoint{2.335049in}{2.729707in}}{\pgfqpoint{2.329225in}{2.723883in}}%
\pgfpathcurveto{\pgfqpoint{2.323401in}{2.718059in}}{\pgfqpoint{2.320129in}{2.710159in}}{\pgfqpoint{2.320129in}{2.701923in}}%
\pgfpathcurveto{\pgfqpoint{2.320129in}{2.693686in}}{\pgfqpoint{2.323401in}{2.685786in}}{\pgfqpoint{2.329225in}{2.679962in}}%
\pgfpathcurveto{\pgfqpoint{2.335049in}{2.674139in}}{\pgfqpoint{2.342949in}{2.670866in}}{\pgfqpoint{2.351185in}{2.670866in}}%
\pgfpathclose%
\pgfusepath{stroke,fill}%
\end{pgfscope}%
\begin{pgfscope}%
\pgfpathrectangle{\pgfqpoint{0.100000in}{0.212622in}}{\pgfqpoint{3.696000in}{3.696000in}}%
\pgfusepath{clip}%
\pgfsetbuttcap%
\pgfsetroundjoin%
\definecolor{currentfill}{rgb}{0.121569,0.466667,0.705882}%
\pgfsetfillcolor{currentfill}%
\pgfsetfillopacity{0.831752}%
\pgfsetlinewidth{1.003750pt}%
\definecolor{currentstroke}{rgb}{0.121569,0.466667,0.705882}%
\pgfsetstrokecolor{currentstroke}%
\pgfsetstrokeopacity{0.831752}%
\pgfsetdash{}{0pt}%
\pgfpathmoveto{\pgfqpoint{2.812790in}{1.476442in}}%
\pgfpathcurveto{\pgfqpoint{2.821027in}{1.476442in}}{\pgfqpoint{2.828927in}{1.479715in}}{\pgfqpoint{2.834751in}{1.485539in}}%
\pgfpathcurveto{\pgfqpoint{2.840574in}{1.491363in}}{\pgfqpoint{2.843847in}{1.499263in}}{\pgfqpoint{2.843847in}{1.507499in}}%
\pgfpathcurveto{\pgfqpoint{2.843847in}{1.515735in}}{\pgfqpoint{2.840574in}{1.523635in}}{\pgfqpoint{2.834751in}{1.529459in}}%
\pgfpathcurveto{\pgfqpoint{2.828927in}{1.535283in}}{\pgfqpoint{2.821027in}{1.538555in}}{\pgfqpoint{2.812790in}{1.538555in}}%
\pgfpathcurveto{\pgfqpoint{2.804554in}{1.538555in}}{\pgfqpoint{2.796654in}{1.535283in}}{\pgfqpoint{2.790830in}{1.529459in}}%
\pgfpathcurveto{\pgfqpoint{2.785006in}{1.523635in}}{\pgfqpoint{2.781734in}{1.515735in}}{\pgfqpoint{2.781734in}{1.507499in}}%
\pgfpathcurveto{\pgfqpoint{2.781734in}{1.499263in}}{\pgfqpoint{2.785006in}{1.491363in}}{\pgfqpoint{2.790830in}{1.485539in}}%
\pgfpathcurveto{\pgfqpoint{2.796654in}{1.479715in}}{\pgfqpoint{2.804554in}{1.476442in}}{\pgfqpoint{2.812790in}{1.476442in}}%
\pgfpathclose%
\pgfusepath{stroke,fill}%
\end{pgfscope}%
\begin{pgfscope}%
\pgfpathrectangle{\pgfqpoint{0.100000in}{0.212622in}}{\pgfqpoint{3.696000in}{3.696000in}}%
\pgfusepath{clip}%
\pgfsetbuttcap%
\pgfsetroundjoin%
\definecolor{currentfill}{rgb}{0.121569,0.466667,0.705882}%
\pgfsetfillcolor{currentfill}%
\pgfsetfillopacity{0.832035}%
\pgfsetlinewidth{1.003750pt}%
\definecolor{currentstroke}{rgb}{0.121569,0.466667,0.705882}%
\pgfsetstrokecolor{currentstroke}%
\pgfsetstrokeopacity{0.832035}%
\pgfsetdash{}{0pt}%
\pgfpathmoveto{\pgfqpoint{2.349104in}{2.668369in}}%
\pgfpathcurveto{\pgfqpoint{2.357340in}{2.668369in}}{\pgfqpoint{2.365240in}{2.671641in}}{\pgfqpoint{2.371064in}{2.677465in}}%
\pgfpathcurveto{\pgfqpoint{2.376888in}{2.683289in}}{\pgfqpoint{2.380160in}{2.691189in}}{\pgfqpoint{2.380160in}{2.699425in}}%
\pgfpathcurveto{\pgfqpoint{2.380160in}{2.707662in}}{\pgfqpoint{2.376888in}{2.715562in}}{\pgfqpoint{2.371064in}{2.721386in}}%
\pgfpathcurveto{\pgfqpoint{2.365240in}{2.727210in}}{\pgfqpoint{2.357340in}{2.730482in}}{\pgfqpoint{2.349104in}{2.730482in}}%
\pgfpathcurveto{\pgfqpoint{2.340867in}{2.730482in}}{\pgfqpoint{2.332967in}{2.727210in}}{\pgfqpoint{2.327143in}{2.721386in}}%
\pgfpathcurveto{\pgfqpoint{2.321319in}{2.715562in}}{\pgfqpoint{2.318047in}{2.707662in}}{\pgfqpoint{2.318047in}{2.699425in}}%
\pgfpathcurveto{\pgfqpoint{2.318047in}{2.691189in}}{\pgfqpoint{2.321319in}{2.683289in}}{\pgfqpoint{2.327143in}{2.677465in}}%
\pgfpathcurveto{\pgfqpoint{2.332967in}{2.671641in}}{\pgfqpoint{2.340867in}{2.668369in}}{\pgfqpoint{2.349104in}{2.668369in}}%
\pgfpathclose%
\pgfusepath{stroke,fill}%
\end{pgfscope}%
\begin{pgfscope}%
\pgfpathrectangle{\pgfqpoint{0.100000in}{0.212622in}}{\pgfqpoint{3.696000in}{3.696000in}}%
\pgfusepath{clip}%
\pgfsetbuttcap%
\pgfsetroundjoin%
\definecolor{currentfill}{rgb}{0.121569,0.466667,0.705882}%
\pgfsetfillcolor{currentfill}%
\pgfsetfillopacity{0.833547}%
\pgfsetlinewidth{1.003750pt}%
\definecolor{currentstroke}{rgb}{0.121569,0.466667,0.705882}%
\pgfsetstrokecolor{currentstroke}%
\pgfsetstrokeopacity{0.833547}%
\pgfsetdash{}{0pt}%
\pgfpathmoveto{\pgfqpoint{2.345271in}{2.663705in}}%
\pgfpathcurveto{\pgfqpoint{2.353507in}{2.663705in}}{\pgfqpoint{2.361407in}{2.666978in}}{\pgfqpoint{2.367231in}{2.672802in}}%
\pgfpathcurveto{\pgfqpoint{2.373055in}{2.678626in}}{\pgfqpoint{2.376328in}{2.686526in}}{\pgfqpoint{2.376328in}{2.694762in}}%
\pgfpathcurveto{\pgfqpoint{2.376328in}{2.702998in}}{\pgfqpoint{2.373055in}{2.710898in}}{\pgfqpoint{2.367231in}{2.716722in}}%
\pgfpathcurveto{\pgfqpoint{2.361407in}{2.722546in}}{\pgfqpoint{2.353507in}{2.725818in}}{\pgfqpoint{2.345271in}{2.725818in}}%
\pgfpathcurveto{\pgfqpoint{2.337035in}{2.725818in}}{\pgfqpoint{2.329135in}{2.722546in}}{\pgfqpoint{2.323311in}{2.716722in}}%
\pgfpathcurveto{\pgfqpoint{2.317487in}{2.710898in}}{\pgfqpoint{2.314215in}{2.702998in}}{\pgfqpoint{2.314215in}{2.694762in}}%
\pgfpathcurveto{\pgfqpoint{2.314215in}{2.686526in}}{\pgfqpoint{2.317487in}{2.678626in}}{\pgfqpoint{2.323311in}{2.672802in}}%
\pgfpathcurveto{\pgfqpoint{2.329135in}{2.666978in}}{\pgfqpoint{2.337035in}{2.663705in}}{\pgfqpoint{2.345271in}{2.663705in}}%
\pgfpathclose%
\pgfusepath{stroke,fill}%
\end{pgfscope}%
\begin{pgfscope}%
\pgfpathrectangle{\pgfqpoint{0.100000in}{0.212622in}}{\pgfqpoint{3.696000in}{3.696000in}}%
\pgfusepath{clip}%
\pgfsetbuttcap%
\pgfsetroundjoin%
\definecolor{currentfill}{rgb}{0.121569,0.466667,0.705882}%
\pgfsetfillcolor{currentfill}%
\pgfsetfillopacity{0.833779}%
\pgfsetlinewidth{1.003750pt}%
\definecolor{currentstroke}{rgb}{0.121569,0.466667,0.705882}%
\pgfsetstrokecolor{currentstroke}%
\pgfsetstrokeopacity{0.833779}%
\pgfsetdash{}{0pt}%
\pgfpathmoveto{\pgfqpoint{1.059382in}{2.271534in}}%
\pgfpathcurveto{\pgfqpoint{1.067619in}{2.271534in}}{\pgfqpoint{1.075519in}{2.274806in}}{\pgfqpoint{1.081343in}{2.280630in}}%
\pgfpathcurveto{\pgfqpoint{1.087167in}{2.286454in}}{\pgfqpoint{1.090439in}{2.294354in}}{\pgfqpoint{1.090439in}{2.302590in}}%
\pgfpathcurveto{\pgfqpoint{1.090439in}{2.310827in}}{\pgfqpoint{1.087167in}{2.318727in}}{\pgfqpoint{1.081343in}{2.324551in}}%
\pgfpathcurveto{\pgfqpoint{1.075519in}{2.330374in}}{\pgfqpoint{1.067619in}{2.333647in}}{\pgfqpoint{1.059382in}{2.333647in}}%
\pgfpathcurveto{\pgfqpoint{1.051146in}{2.333647in}}{\pgfqpoint{1.043246in}{2.330374in}}{\pgfqpoint{1.037422in}{2.324551in}}%
\pgfpathcurveto{\pgfqpoint{1.031598in}{2.318727in}}{\pgfqpoint{1.028326in}{2.310827in}}{\pgfqpoint{1.028326in}{2.302590in}}%
\pgfpathcurveto{\pgfqpoint{1.028326in}{2.294354in}}{\pgfqpoint{1.031598in}{2.286454in}}{\pgfqpoint{1.037422in}{2.280630in}}%
\pgfpathcurveto{\pgfqpoint{1.043246in}{2.274806in}}{\pgfqpoint{1.051146in}{2.271534in}}{\pgfqpoint{1.059382in}{2.271534in}}%
\pgfpathclose%
\pgfusepath{stroke,fill}%
\end{pgfscope}%
\begin{pgfscope}%
\pgfpathrectangle{\pgfqpoint{0.100000in}{0.212622in}}{\pgfqpoint{3.696000in}{3.696000in}}%
\pgfusepath{clip}%
\pgfsetbuttcap%
\pgfsetroundjoin%
\definecolor{currentfill}{rgb}{0.121569,0.466667,0.705882}%
\pgfsetfillcolor{currentfill}%
\pgfsetfillopacity{0.834669}%
\pgfsetlinewidth{1.003750pt}%
\definecolor{currentstroke}{rgb}{0.121569,0.466667,0.705882}%
\pgfsetstrokecolor{currentstroke}%
\pgfsetstrokeopacity{0.834669}%
\pgfsetdash{}{0pt}%
\pgfpathmoveto{\pgfqpoint{2.342571in}{2.660582in}}%
\pgfpathcurveto{\pgfqpoint{2.350807in}{2.660582in}}{\pgfqpoint{2.358707in}{2.663854in}}{\pgfqpoint{2.364531in}{2.669678in}}%
\pgfpathcurveto{\pgfqpoint{2.370355in}{2.675502in}}{\pgfqpoint{2.373627in}{2.683402in}}{\pgfqpoint{2.373627in}{2.691638in}}%
\pgfpathcurveto{\pgfqpoint{2.373627in}{2.699875in}}{\pgfqpoint{2.370355in}{2.707775in}}{\pgfqpoint{2.364531in}{2.713599in}}%
\pgfpathcurveto{\pgfqpoint{2.358707in}{2.719423in}}{\pgfqpoint{2.350807in}{2.722695in}}{\pgfqpoint{2.342571in}{2.722695in}}%
\pgfpathcurveto{\pgfqpoint{2.334335in}{2.722695in}}{\pgfqpoint{2.326434in}{2.719423in}}{\pgfqpoint{2.320611in}{2.713599in}}%
\pgfpathcurveto{\pgfqpoint{2.314787in}{2.707775in}}{\pgfqpoint{2.311514in}{2.699875in}}{\pgfqpoint{2.311514in}{2.691638in}}%
\pgfpathcurveto{\pgfqpoint{2.311514in}{2.683402in}}{\pgfqpoint{2.314787in}{2.675502in}}{\pgfqpoint{2.320611in}{2.669678in}}%
\pgfpathcurveto{\pgfqpoint{2.326434in}{2.663854in}}{\pgfqpoint{2.334335in}{2.660582in}}{\pgfqpoint{2.342571in}{2.660582in}}%
\pgfpathclose%
\pgfusepath{stroke,fill}%
\end{pgfscope}%
\begin{pgfscope}%
\pgfpathrectangle{\pgfqpoint{0.100000in}{0.212622in}}{\pgfqpoint{3.696000in}{3.696000in}}%
\pgfusepath{clip}%
\pgfsetbuttcap%
\pgfsetroundjoin%
\definecolor{currentfill}{rgb}{0.121569,0.466667,0.705882}%
\pgfsetfillcolor{currentfill}%
\pgfsetfillopacity{0.835758}%
\pgfsetlinewidth{1.003750pt}%
\definecolor{currentstroke}{rgb}{0.121569,0.466667,0.705882}%
\pgfsetstrokecolor{currentstroke}%
\pgfsetstrokeopacity{0.835758}%
\pgfsetdash{}{0pt}%
\pgfpathmoveto{\pgfqpoint{2.340125in}{2.657915in}}%
\pgfpathcurveto{\pgfqpoint{2.348361in}{2.657915in}}{\pgfqpoint{2.356261in}{2.661187in}}{\pgfqpoint{2.362085in}{2.667011in}}%
\pgfpathcurveto{\pgfqpoint{2.367909in}{2.672835in}}{\pgfqpoint{2.371181in}{2.680735in}}{\pgfqpoint{2.371181in}{2.688971in}}%
\pgfpathcurveto{\pgfqpoint{2.371181in}{2.697208in}}{\pgfqpoint{2.367909in}{2.705108in}}{\pgfqpoint{2.362085in}{2.710932in}}%
\pgfpathcurveto{\pgfqpoint{2.356261in}{2.716756in}}{\pgfqpoint{2.348361in}{2.720028in}}{\pgfqpoint{2.340125in}{2.720028in}}%
\pgfpathcurveto{\pgfqpoint{2.331889in}{2.720028in}}{\pgfqpoint{2.323989in}{2.716756in}}{\pgfqpoint{2.318165in}{2.710932in}}%
\pgfpathcurveto{\pgfqpoint{2.312341in}{2.705108in}}{\pgfqpoint{2.309068in}{2.697208in}}{\pgfqpoint{2.309068in}{2.688971in}}%
\pgfpathcurveto{\pgfqpoint{2.309068in}{2.680735in}}{\pgfqpoint{2.312341in}{2.672835in}}{\pgfqpoint{2.318165in}{2.667011in}}%
\pgfpathcurveto{\pgfqpoint{2.323989in}{2.661187in}}{\pgfqpoint{2.331889in}{2.657915in}}{\pgfqpoint{2.340125in}{2.657915in}}%
\pgfpathclose%
\pgfusepath{stroke,fill}%
\end{pgfscope}%
\begin{pgfscope}%
\pgfpathrectangle{\pgfqpoint{0.100000in}{0.212622in}}{\pgfqpoint{3.696000in}{3.696000in}}%
\pgfusepath{clip}%
\pgfsetbuttcap%
\pgfsetroundjoin%
\definecolor{currentfill}{rgb}{0.121569,0.466667,0.705882}%
\pgfsetfillcolor{currentfill}%
\pgfsetfillopacity{0.836478}%
\pgfsetlinewidth{1.003750pt}%
\definecolor{currentstroke}{rgb}{0.121569,0.466667,0.705882}%
\pgfsetstrokecolor{currentstroke}%
\pgfsetstrokeopacity{0.836478}%
\pgfsetdash{}{0pt}%
\pgfpathmoveto{\pgfqpoint{2.338429in}{2.655805in}}%
\pgfpathcurveto{\pgfqpoint{2.346665in}{2.655805in}}{\pgfqpoint{2.354565in}{2.659077in}}{\pgfqpoint{2.360389in}{2.664901in}}%
\pgfpathcurveto{\pgfqpoint{2.366213in}{2.670725in}}{\pgfqpoint{2.369486in}{2.678625in}}{\pgfqpoint{2.369486in}{2.686861in}}%
\pgfpathcurveto{\pgfqpoint{2.369486in}{2.695097in}}{\pgfqpoint{2.366213in}{2.702997in}}{\pgfqpoint{2.360389in}{2.708821in}}%
\pgfpathcurveto{\pgfqpoint{2.354565in}{2.714645in}}{\pgfqpoint{2.346665in}{2.717918in}}{\pgfqpoint{2.338429in}{2.717918in}}%
\pgfpathcurveto{\pgfqpoint{2.330193in}{2.717918in}}{\pgfqpoint{2.322293in}{2.714645in}}{\pgfqpoint{2.316469in}{2.708821in}}%
\pgfpathcurveto{\pgfqpoint{2.310645in}{2.702997in}}{\pgfqpoint{2.307373in}{2.695097in}}{\pgfqpoint{2.307373in}{2.686861in}}%
\pgfpathcurveto{\pgfqpoint{2.307373in}{2.678625in}}{\pgfqpoint{2.310645in}{2.670725in}}{\pgfqpoint{2.316469in}{2.664901in}}%
\pgfpathcurveto{\pgfqpoint{2.322293in}{2.659077in}}{\pgfqpoint{2.330193in}{2.655805in}}{\pgfqpoint{2.338429in}{2.655805in}}%
\pgfpathclose%
\pgfusepath{stroke,fill}%
\end{pgfscope}%
\begin{pgfscope}%
\pgfpathrectangle{\pgfqpoint{0.100000in}{0.212622in}}{\pgfqpoint{3.696000in}{3.696000in}}%
\pgfusepath{clip}%
\pgfsetbuttcap%
\pgfsetroundjoin%
\definecolor{currentfill}{rgb}{0.121569,0.466667,0.705882}%
\pgfsetfillcolor{currentfill}%
\pgfsetfillopacity{0.836811}%
\pgfsetlinewidth{1.003750pt}%
\definecolor{currentstroke}{rgb}{0.121569,0.466667,0.705882}%
\pgfsetstrokecolor{currentstroke}%
\pgfsetstrokeopacity{0.836811}%
\pgfsetdash{}{0pt}%
\pgfpathmoveto{\pgfqpoint{1.050208in}{2.258784in}}%
\pgfpathcurveto{\pgfqpoint{1.058444in}{2.258784in}}{\pgfqpoint{1.066345in}{2.262056in}}{\pgfqpoint{1.072168in}{2.267880in}}%
\pgfpathcurveto{\pgfqpoint{1.077992in}{2.273704in}}{\pgfqpoint{1.081265in}{2.281604in}}{\pgfqpoint{1.081265in}{2.289841in}}%
\pgfpathcurveto{\pgfqpoint{1.081265in}{2.298077in}}{\pgfqpoint{1.077992in}{2.305977in}}{\pgfqpoint{1.072168in}{2.311801in}}%
\pgfpathcurveto{\pgfqpoint{1.066345in}{2.317625in}}{\pgfqpoint{1.058444in}{2.320897in}}{\pgfqpoint{1.050208in}{2.320897in}}%
\pgfpathcurveto{\pgfqpoint{1.041972in}{2.320897in}}{\pgfqpoint{1.034072in}{2.317625in}}{\pgfqpoint{1.028248in}{2.311801in}}%
\pgfpathcurveto{\pgfqpoint{1.022424in}{2.305977in}}{\pgfqpoint{1.019152in}{2.298077in}}{\pgfqpoint{1.019152in}{2.289841in}}%
\pgfpathcurveto{\pgfqpoint{1.019152in}{2.281604in}}{\pgfqpoint{1.022424in}{2.273704in}}{\pgfqpoint{1.028248in}{2.267880in}}%
\pgfpathcurveto{\pgfqpoint{1.034072in}{2.262056in}}{\pgfqpoint{1.041972in}{2.258784in}}{\pgfqpoint{1.050208in}{2.258784in}}%
\pgfpathclose%
\pgfusepath{stroke,fill}%
\end{pgfscope}%
\begin{pgfscope}%
\pgfpathrectangle{\pgfqpoint{0.100000in}{0.212622in}}{\pgfqpoint{3.696000in}{3.696000in}}%
\pgfusepath{clip}%
\pgfsetbuttcap%
\pgfsetroundjoin%
\definecolor{currentfill}{rgb}{0.121569,0.466667,0.705882}%
\pgfsetfillcolor{currentfill}%
\pgfsetfillopacity{0.837774}%
\pgfsetlinewidth{1.003750pt}%
\definecolor{currentstroke}{rgb}{0.121569,0.466667,0.705882}%
\pgfsetstrokecolor{currentstroke}%
\pgfsetstrokeopacity{0.837774}%
\pgfsetdash{}{0pt}%
\pgfpathmoveto{\pgfqpoint{2.335274in}{2.651969in}}%
\pgfpathcurveto{\pgfqpoint{2.343511in}{2.651969in}}{\pgfqpoint{2.351411in}{2.655241in}}{\pgfqpoint{2.357235in}{2.661065in}}%
\pgfpathcurveto{\pgfqpoint{2.363059in}{2.666889in}}{\pgfqpoint{2.366331in}{2.674789in}}{\pgfqpoint{2.366331in}{2.683025in}}%
\pgfpathcurveto{\pgfqpoint{2.366331in}{2.691261in}}{\pgfqpoint{2.363059in}{2.699161in}}{\pgfqpoint{2.357235in}{2.704985in}}%
\pgfpathcurveto{\pgfqpoint{2.351411in}{2.710809in}}{\pgfqpoint{2.343511in}{2.714082in}}{\pgfqpoint{2.335274in}{2.714082in}}%
\pgfpathcurveto{\pgfqpoint{2.327038in}{2.714082in}}{\pgfqpoint{2.319138in}{2.710809in}}{\pgfqpoint{2.313314in}{2.704985in}}%
\pgfpathcurveto{\pgfqpoint{2.307490in}{2.699161in}}{\pgfqpoint{2.304218in}{2.691261in}}{\pgfqpoint{2.304218in}{2.683025in}}%
\pgfpathcurveto{\pgfqpoint{2.304218in}{2.674789in}}{\pgfqpoint{2.307490in}{2.666889in}}{\pgfqpoint{2.313314in}{2.661065in}}%
\pgfpathcurveto{\pgfqpoint{2.319138in}{2.655241in}}{\pgfqpoint{2.327038in}{2.651969in}}{\pgfqpoint{2.335274in}{2.651969in}}%
\pgfpathclose%
\pgfusepath{stroke,fill}%
\end{pgfscope}%
\begin{pgfscope}%
\pgfpathrectangle{\pgfqpoint{0.100000in}{0.212622in}}{\pgfqpoint{3.696000in}{3.696000in}}%
\pgfusepath{clip}%
\pgfsetbuttcap%
\pgfsetroundjoin%
\definecolor{currentfill}{rgb}{0.121569,0.466667,0.705882}%
\pgfsetfillcolor{currentfill}%
\pgfsetfillopacity{0.838431}%
\pgfsetlinewidth{1.003750pt}%
\definecolor{currentstroke}{rgb}{0.121569,0.466667,0.705882}%
\pgfsetstrokecolor{currentstroke}%
\pgfsetstrokeopacity{0.838431}%
\pgfsetdash{}{0pt}%
\pgfpathmoveto{\pgfqpoint{2.333770in}{2.650320in}}%
\pgfpathcurveto{\pgfqpoint{2.342007in}{2.650320in}}{\pgfqpoint{2.349907in}{2.653592in}}{\pgfqpoint{2.355731in}{2.659416in}}%
\pgfpathcurveto{\pgfqpoint{2.361555in}{2.665240in}}{\pgfqpoint{2.364827in}{2.673140in}}{\pgfqpoint{2.364827in}{2.681376in}}%
\pgfpathcurveto{\pgfqpoint{2.364827in}{2.689613in}}{\pgfqpoint{2.361555in}{2.697513in}}{\pgfqpoint{2.355731in}{2.703337in}}%
\pgfpathcurveto{\pgfqpoint{2.349907in}{2.709160in}}{\pgfqpoint{2.342007in}{2.712433in}}{\pgfqpoint{2.333770in}{2.712433in}}%
\pgfpathcurveto{\pgfqpoint{2.325534in}{2.712433in}}{\pgfqpoint{2.317634in}{2.709160in}}{\pgfqpoint{2.311810in}{2.703337in}}%
\pgfpathcurveto{\pgfqpoint{2.305986in}{2.697513in}}{\pgfqpoint{2.302714in}{2.689613in}}{\pgfqpoint{2.302714in}{2.681376in}}%
\pgfpathcurveto{\pgfqpoint{2.302714in}{2.673140in}}{\pgfqpoint{2.305986in}{2.665240in}}{\pgfqpoint{2.311810in}{2.659416in}}%
\pgfpathcurveto{\pgfqpoint{2.317634in}{2.653592in}}{\pgfqpoint{2.325534in}{2.650320in}}{\pgfqpoint{2.333770in}{2.650320in}}%
\pgfpathclose%
\pgfusepath{stroke,fill}%
\end{pgfscope}%
\begin{pgfscope}%
\pgfpathrectangle{\pgfqpoint{0.100000in}{0.212622in}}{\pgfqpoint{3.696000in}{3.696000in}}%
\pgfusepath{clip}%
\pgfsetbuttcap%
\pgfsetroundjoin%
\definecolor{currentfill}{rgb}{0.121569,0.466667,0.705882}%
\pgfsetfillcolor{currentfill}%
\pgfsetfillopacity{0.839648}%
\pgfsetlinewidth{1.003750pt}%
\definecolor{currentstroke}{rgb}{0.121569,0.466667,0.705882}%
\pgfsetstrokecolor{currentstroke}%
\pgfsetstrokeopacity{0.839648}%
\pgfsetdash{}{0pt}%
\pgfpathmoveto{\pgfqpoint{2.331121in}{2.647362in}}%
\pgfpathcurveto{\pgfqpoint{2.339357in}{2.647362in}}{\pgfqpoint{2.347257in}{2.650634in}}{\pgfqpoint{2.353081in}{2.656458in}}%
\pgfpathcurveto{\pgfqpoint{2.358905in}{2.662282in}}{\pgfqpoint{2.362177in}{2.670182in}}{\pgfqpoint{2.362177in}{2.678418in}}%
\pgfpathcurveto{\pgfqpoint{2.362177in}{2.686655in}}{\pgfqpoint{2.358905in}{2.694555in}}{\pgfqpoint{2.353081in}{2.700379in}}%
\pgfpathcurveto{\pgfqpoint{2.347257in}{2.706203in}}{\pgfqpoint{2.339357in}{2.709475in}}{\pgfqpoint{2.331121in}{2.709475in}}%
\pgfpathcurveto{\pgfqpoint{2.322885in}{2.709475in}}{\pgfqpoint{2.314985in}{2.706203in}}{\pgfqpoint{2.309161in}{2.700379in}}%
\pgfpathcurveto{\pgfqpoint{2.303337in}{2.694555in}}{\pgfqpoint{2.300064in}{2.686655in}}{\pgfqpoint{2.300064in}{2.678418in}}%
\pgfpathcurveto{\pgfqpoint{2.300064in}{2.670182in}}{\pgfqpoint{2.303337in}{2.662282in}}{\pgfqpoint{2.309161in}{2.656458in}}%
\pgfpathcurveto{\pgfqpoint{2.314985in}{2.650634in}}{\pgfqpoint{2.322885in}{2.647362in}}{\pgfqpoint{2.331121in}{2.647362in}}%
\pgfpathclose%
\pgfusepath{stroke,fill}%
\end{pgfscope}%
\begin{pgfscope}%
\pgfpathrectangle{\pgfqpoint{0.100000in}{0.212622in}}{\pgfqpoint{3.696000in}{3.696000in}}%
\pgfusepath{clip}%
\pgfsetbuttcap%
\pgfsetroundjoin%
\definecolor{currentfill}{rgb}{0.121569,0.466667,0.705882}%
\pgfsetfillcolor{currentfill}%
\pgfsetfillopacity{0.839811}%
\pgfsetlinewidth{1.003750pt}%
\definecolor{currentstroke}{rgb}{0.121569,0.466667,0.705882}%
\pgfsetstrokecolor{currentstroke}%
\pgfsetstrokeopacity{0.839811}%
\pgfsetdash{}{0pt}%
\pgfpathmoveto{\pgfqpoint{2.796098in}{1.455545in}}%
\pgfpathcurveto{\pgfqpoint{2.804334in}{1.455545in}}{\pgfqpoint{2.812234in}{1.458818in}}{\pgfqpoint{2.818058in}{1.464642in}}%
\pgfpathcurveto{\pgfqpoint{2.823882in}{1.470466in}}{\pgfqpoint{2.827154in}{1.478366in}}{\pgfqpoint{2.827154in}{1.486602in}}%
\pgfpathcurveto{\pgfqpoint{2.827154in}{1.494838in}}{\pgfqpoint{2.823882in}{1.502738in}}{\pgfqpoint{2.818058in}{1.508562in}}%
\pgfpathcurveto{\pgfqpoint{2.812234in}{1.514386in}}{\pgfqpoint{2.804334in}{1.517658in}}{\pgfqpoint{2.796098in}{1.517658in}}%
\pgfpathcurveto{\pgfqpoint{2.787861in}{1.517658in}}{\pgfqpoint{2.779961in}{1.514386in}}{\pgfqpoint{2.774137in}{1.508562in}}%
\pgfpathcurveto{\pgfqpoint{2.768314in}{1.502738in}}{\pgfqpoint{2.765041in}{1.494838in}}{\pgfqpoint{2.765041in}{1.486602in}}%
\pgfpathcurveto{\pgfqpoint{2.765041in}{1.478366in}}{\pgfqpoint{2.768314in}{1.470466in}}{\pgfqpoint{2.774137in}{1.464642in}}%
\pgfpathcurveto{\pgfqpoint{2.779961in}{1.458818in}}{\pgfqpoint{2.787861in}{1.455545in}}{\pgfqpoint{2.796098in}{1.455545in}}%
\pgfpathclose%
\pgfusepath{stroke,fill}%
\end{pgfscope}%
\begin{pgfscope}%
\pgfpathrectangle{\pgfqpoint{0.100000in}{0.212622in}}{\pgfqpoint{3.696000in}{3.696000in}}%
\pgfusepath{clip}%
\pgfsetbuttcap%
\pgfsetroundjoin%
\definecolor{currentfill}{rgb}{0.121569,0.466667,0.705882}%
\pgfsetfillcolor{currentfill}%
\pgfsetfillopacity{0.840169}%
\pgfsetlinewidth{1.003750pt}%
\definecolor{currentstroke}{rgb}{0.121569,0.466667,0.705882}%
\pgfsetstrokecolor{currentstroke}%
\pgfsetstrokeopacity{0.840169}%
\pgfsetdash{}{0pt}%
\pgfpathmoveto{\pgfqpoint{1.040619in}{2.246268in}}%
\pgfpathcurveto{\pgfqpoint{1.048856in}{2.246268in}}{\pgfqpoint{1.056756in}{2.249540in}}{\pgfqpoint{1.062580in}{2.255364in}}%
\pgfpathcurveto{\pgfqpoint{1.068404in}{2.261188in}}{\pgfqpoint{1.071676in}{2.269088in}}{\pgfqpoint{1.071676in}{2.277324in}}%
\pgfpathcurveto{\pgfqpoint{1.071676in}{2.285561in}}{\pgfqpoint{1.068404in}{2.293461in}}{\pgfqpoint{1.062580in}{2.299285in}}%
\pgfpathcurveto{\pgfqpoint{1.056756in}{2.305108in}}{\pgfqpoint{1.048856in}{2.308381in}}{\pgfqpoint{1.040619in}{2.308381in}}%
\pgfpathcurveto{\pgfqpoint{1.032383in}{2.308381in}}{\pgfqpoint{1.024483in}{2.305108in}}{\pgfqpoint{1.018659in}{2.299285in}}%
\pgfpathcurveto{\pgfqpoint{1.012835in}{2.293461in}}{\pgfqpoint{1.009563in}{2.285561in}}{\pgfqpoint{1.009563in}{2.277324in}}%
\pgfpathcurveto{\pgfqpoint{1.009563in}{2.269088in}}{\pgfqpoint{1.012835in}{2.261188in}}{\pgfqpoint{1.018659in}{2.255364in}}%
\pgfpathcurveto{\pgfqpoint{1.024483in}{2.249540in}}{\pgfqpoint{1.032383in}{2.246268in}}{\pgfqpoint{1.040619in}{2.246268in}}%
\pgfpathclose%
\pgfusepath{stroke,fill}%
\end{pgfscope}%
\begin{pgfscope}%
\pgfpathrectangle{\pgfqpoint{0.100000in}{0.212622in}}{\pgfqpoint{3.696000in}{3.696000in}}%
\pgfusepath{clip}%
\pgfsetbuttcap%
\pgfsetroundjoin%
\definecolor{currentfill}{rgb}{0.121569,0.466667,0.705882}%
\pgfsetfillcolor{currentfill}%
\pgfsetfillopacity{0.840399}%
\pgfsetlinewidth{1.003750pt}%
\definecolor{currentstroke}{rgb}{0.121569,0.466667,0.705882}%
\pgfsetstrokecolor{currentstroke}%
\pgfsetstrokeopacity{0.840399}%
\pgfsetdash{}{0pt}%
\pgfpathmoveto{\pgfqpoint{2.329416in}{2.645271in}}%
\pgfpathcurveto{\pgfqpoint{2.337652in}{2.645271in}}{\pgfqpoint{2.345552in}{2.648543in}}{\pgfqpoint{2.351376in}{2.654367in}}%
\pgfpathcurveto{\pgfqpoint{2.357200in}{2.660191in}}{\pgfqpoint{2.360472in}{2.668091in}}{\pgfqpoint{2.360472in}{2.676328in}}%
\pgfpathcurveto{\pgfqpoint{2.360472in}{2.684564in}}{\pgfqpoint{2.357200in}{2.692464in}}{\pgfqpoint{2.351376in}{2.698288in}}%
\pgfpathcurveto{\pgfqpoint{2.345552in}{2.704112in}}{\pgfqpoint{2.337652in}{2.707384in}}{\pgfqpoint{2.329416in}{2.707384in}}%
\pgfpathcurveto{\pgfqpoint{2.321180in}{2.707384in}}{\pgfqpoint{2.313280in}{2.704112in}}{\pgfqpoint{2.307456in}{2.698288in}}%
\pgfpathcurveto{\pgfqpoint{2.301632in}{2.692464in}}{\pgfqpoint{2.298359in}{2.684564in}}{\pgfqpoint{2.298359in}{2.676328in}}%
\pgfpathcurveto{\pgfqpoint{2.298359in}{2.668091in}}{\pgfqpoint{2.301632in}{2.660191in}}{\pgfqpoint{2.307456in}{2.654367in}}%
\pgfpathcurveto{\pgfqpoint{2.313280in}{2.648543in}}{\pgfqpoint{2.321180in}{2.645271in}}{\pgfqpoint{2.329416in}{2.645271in}}%
\pgfpathclose%
\pgfusepath{stroke,fill}%
\end{pgfscope}%
\begin{pgfscope}%
\pgfpathrectangle{\pgfqpoint{0.100000in}{0.212622in}}{\pgfqpoint{3.696000in}{3.696000in}}%
\pgfusepath{clip}%
\pgfsetbuttcap%
\pgfsetroundjoin%
\definecolor{currentfill}{rgb}{0.121569,0.466667,0.705882}%
\pgfsetfillcolor{currentfill}%
\pgfsetfillopacity{0.841747}%
\pgfsetlinewidth{1.003750pt}%
\definecolor{currentstroke}{rgb}{0.121569,0.466667,0.705882}%
\pgfsetstrokecolor{currentstroke}%
\pgfsetstrokeopacity{0.841747}%
\pgfsetdash{}{0pt}%
\pgfpathmoveto{\pgfqpoint{2.326167in}{2.641519in}}%
\pgfpathcurveto{\pgfqpoint{2.334404in}{2.641519in}}{\pgfqpoint{2.342304in}{2.644792in}}{\pgfqpoint{2.348128in}{2.650616in}}%
\pgfpathcurveto{\pgfqpoint{2.353952in}{2.656439in}}{\pgfqpoint{2.357224in}{2.664340in}}{\pgfqpoint{2.357224in}{2.672576in}}%
\pgfpathcurveto{\pgfqpoint{2.357224in}{2.680812in}}{\pgfqpoint{2.353952in}{2.688712in}}{\pgfqpoint{2.348128in}{2.694536in}}%
\pgfpathcurveto{\pgfqpoint{2.342304in}{2.700360in}}{\pgfqpoint{2.334404in}{2.703632in}}{\pgfqpoint{2.326167in}{2.703632in}}%
\pgfpathcurveto{\pgfqpoint{2.317931in}{2.703632in}}{\pgfqpoint{2.310031in}{2.700360in}}{\pgfqpoint{2.304207in}{2.694536in}}%
\pgfpathcurveto{\pgfqpoint{2.298383in}{2.688712in}}{\pgfqpoint{2.295111in}{2.680812in}}{\pgfqpoint{2.295111in}{2.672576in}}%
\pgfpathcurveto{\pgfqpoint{2.295111in}{2.664340in}}{\pgfqpoint{2.298383in}{2.656439in}}{\pgfqpoint{2.304207in}{2.650616in}}%
\pgfpathcurveto{\pgfqpoint{2.310031in}{2.644792in}}{\pgfqpoint{2.317931in}{2.641519in}}{\pgfqpoint{2.326167in}{2.641519in}}%
\pgfpathclose%
\pgfusepath{stroke,fill}%
\end{pgfscope}%
\begin{pgfscope}%
\pgfpathrectangle{\pgfqpoint{0.100000in}{0.212622in}}{\pgfqpoint{3.696000in}{3.696000in}}%
\pgfusepath{clip}%
\pgfsetbuttcap%
\pgfsetroundjoin%
\definecolor{currentfill}{rgb}{0.121569,0.466667,0.705882}%
\pgfsetfillcolor{currentfill}%
\pgfsetfillopacity{0.842484}%
\pgfsetlinewidth{1.003750pt}%
\definecolor{currentstroke}{rgb}{0.121569,0.466667,0.705882}%
\pgfsetstrokecolor{currentstroke}%
\pgfsetstrokeopacity{0.842484}%
\pgfsetdash{}{0pt}%
\pgfpathmoveto{\pgfqpoint{2.324481in}{2.639712in}}%
\pgfpathcurveto{\pgfqpoint{2.332717in}{2.639712in}}{\pgfqpoint{2.340617in}{2.642985in}}{\pgfqpoint{2.346441in}{2.648809in}}%
\pgfpathcurveto{\pgfqpoint{2.352265in}{2.654633in}}{\pgfqpoint{2.355537in}{2.662533in}}{\pgfqpoint{2.355537in}{2.670769in}}%
\pgfpathcurveto{\pgfqpoint{2.355537in}{2.679005in}}{\pgfqpoint{2.352265in}{2.686905in}}{\pgfqpoint{2.346441in}{2.692729in}}%
\pgfpathcurveto{\pgfqpoint{2.340617in}{2.698553in}}{\pgfqpoint{2.332717in}{2.701825in}}{\pgfqpoint{2.324481in}{2.701825in}}%
\pgfpathcurveto{\pgfqpoint{2.316245in}{2.701825in}}{\pgfqpoint{2.308345in}{2.698553in}}{\pgfqpoint{2.302521in}{2.692729in}}%
\pgfpathcurveto{\pgfqpoint{2.296697in}{2.686905in}}{\pgfqpoint{2.293424in}{2.679005in}}{\pgfqpoint{2.293424in}{2.670769in}}%
\pgfpathcurveto{\pgfqpoint{2.293424in}{2.662533in}}{\pgfqpoint{2.296697in}{2.654633in}}{\pgfqpoint{2.302521in}{2.648809in}}%
\pgfpathcurveto{\pgfqpoint{2.308345in}{2.642985in}}{\pgfqpoint{2.316245in}{2.639712in}}{\pgfqpoint{2.324481in}{2.639712in}}%
\pgfpathclose%
\pgfusepath{stroke,fill}%
\end{pgfscope}%
\begin{pgfscope}%
\pgfpathrectangle{\pgfqpoint{0.100000in}{0.212622in}}{\pgfqpoint{3.696000in}{3.696000in}}%
\pgfusepath{clip}%
\pgfsetbuttcap%
\pgfsetroundjoin%
\definecolor{currentfill}{rgb}{0.121569,0.466667,0.705882}%
\pgfsetfillcolor{currentfill}%
\pgfsetfillopacity{0.843841}%
\pgfsetlinewidth{1.003750pt}%
\definecolor{currentstroke}{rgb}{0.121569,0.466667,0.705882}%
\pgfsetstrokecolor{currentstroke}%
\pgfsetstrokeopacity{0.843841}%
\pgfsetdash{}{0pt}%
\pgfpathmoveto{\pgfqpoint{2.321500in}{2.636442in}}%
\pgfpathcurveto{\pgfqpoint{2.329737in}{2.636442in}}{\pgfqpoint{2.337637in}{2.639714in}}{\pgfqpoint{2.343461in}{2.645538in}}%
\pgfpathcurveto{\pgfqpoint{2.349285in}{2.651362in}}{\pgfqpoint{2.352557in}{2.659262in}}{\pgfqpoint{2.352557in}{2.667498in}}%
\pgfpathcurveto{\pgfqpoint{2.352557in}{2.675734in}}{\pgfqpoint{2.349285in}{2.683634in}}{\pgfqpoint{2.343461in}{2.689458in}}%
\pgfpathcurveto{\pgfqpoint{2.337637in}{2.695282in}}{\pgfqpoint{2.329737in}{2.698555in}}{\pgfqpoint{2.321500in}{2.698555in}}%
\pgfpathcurveto{\pgfqpoint{2.313264in}{2.698555in}}{\pgfqpoint{2.305364in}{2.695282in}}{\pgfqpoint{2.299540in}{2.689458in}}%
\pgfpathcurveto{\pgfqpoint{2.293716in}{2.683634in}}{\pgfqpoint{2.290444in}{2.675734in}}{\pgfqpoint{2.290444in}{2.667498in}}%
\pgfpathcurveto{\pgfqpoint{2.290444in}{2.659262in}}{\pgfqpoint{2.293716in}{2.651362in}}{\pgfqpoint{2.299540in}{2.645538in}}%
\pgfpathcurveto{\pgfqpoint{2.305364in}{2.639714in}}{\pgfqpoint{2.313264in}{2.636442in}}{\pgfqpoint{2.321500in}{2.636442in}}%
\pgfpathclose%
\pgfusepath{stroke,fill}%
\end{pgfscope}%
\begin{pgfscope}%
\pgfpathrectangle{\pgfqpoint{0.100000in}{0.212622in}}{\pgfqpoint{3.696000in}{3.696000in}}%
\pgfusepath{clip}%
\pgfsetbuttcap%
\pgfsetroundjoin%
\definecolor{currentfill}{rgb}{0.121569,0.466667,0.705882}%
\pgfsetfillcolor{currentfill}%
\pgfsetfillopacity{0.843974}%
\pgfsetlinewidth{1.003750pt}%
\definecolor{currentstroke}{rgb}{0.121569,0.466667,0.705882}%
\pgfsetstrokecolor{currentstroke}%
\pgfsetstrokeopacity{0.843974}%
\pgfsetdash{}{0pt}%
\pgfpathmoveto{\pgfqpoint{1.030550in}{2.235066in}}%
\pgfpathcurveto{\pgfqpoint{1.038786in}{2.235066in}}{\pgfqpoint{1.046687in}{2.238338in}}{\pgfqpoint{1.052510in}{2.244162in}}%
\pgfpathcurveto{\pgfqpoint{1.058334in}{2.249986in}}{\pgfqpoint{1.061607in}{2.257886in}}{\pgfqpoint{1.061607in}{2.266123in}}%
\pgfpathcurveto{\pgfqpoint{1.061607in}{2.274359in}}{\pgfqpoint{1.058334in}{2.282259in}}{\pgfqpoint{1.052510in}{2.288083in}}%
\pgfpathcurveto{\pgfqpoint{1.046687in}{2.293907in}}{\pgfqpoint{1.038786in}{2.297179in}}{\pgfqpoint{1.030550in}{2.297179in}}%
\pgfpathcurveto{\pgfqpoint{1.022314in}{2.297179in}}{\pgfqpoint{1.014414in}{2.293907in}}{\pgfqpoint{1.008590in}{2.288083in}}%
\pgfpathcurveto{\pgfqpoint{1.002766in}{2.282259in}}{\pgfqpoint{0.999494in}{2.274359in}}{\pgfqpoint{0.999494in}{2.266123in}}%
\pgfpathcurveto{\pgfqpoint{0.999494in}{2.257886in}}{\pgfqpoint{1.002766in}{2.249986in}}{\pgfqpoint{1.008590in}{2.244162in}}%
\pgfpathcurveto{\pgfqpoint{1.014414in}{2.238338in}}{\pgfqpoint{1.022314in}{2.235066in}}{\pgfqpoint{1.030550in}{2.235066in}}%
\pgfpathclose%
\pgfusepath{stroke,fill}%
\end{pgfscope}%
\begin{pgfscope}%
\pgfpathrectangle{\pgfqpoint{0.100000in}{0.212622in}}{\pgfqpoint{3.696000in}{3.696000in}}%
\pgfusepath{clip}%
\pgfsetbuttcap%
\pgfsetroundjoin%
\definecolor{currentfill}{rgb}{0.121569,0.466667,0.705882}%
\pgfsetfillcolor{currentfill}%
\pgfsetfillopacity{0.844680}%
\pgfsetlinewidth{1.003750pt}%
\definecolor{currentstroke}{rgb}{0.121569,0.466667,0.705882}%
\pgfsetstrokecolor{currentstroke}%
\pgfsetstrokeopacity{0.844680}%
\pgfsetdash{}{0pt}%
\pgfpathmoveto{\pgfqpoint{2.319527in}{2.634050in}}%
\pgfpathcurveto{\pgfqpoint{2.327763in}{2.634050in}}{\pgfqpoint{2.335663in}{2.637322in}}{\pgfqpoint{2.341487in}{2.643146in}}%
\pgfpathcurveto{\pgfqpoint{2.347311in}{2.648970in}}{\pgfqpoint{2.350583in}{2.656870in}}{\pgfqpoint{2.350583in}{2.665107in}}%
\pgfpathcurveto{\pgfqpoint{2.350583in}{2.673343in}}{\pgfqpoint{2.347311in}{2.681243in}}{\pgfqpoint{2.341487in}{2.687067in}}%
\pgfpathcurveto{\pgfqpoint{2.335663in}{2.692891in}}{\pgfqpoint{2.327763in}{2.696163in}}{\pgfqpoint{2.319527in}{2.696163in}}%
\pgfpathcurveto{\pgfqpoint{2.311290in}{2.696163in}}{\pgfqpoint{2.303390in}{2.692891in}}{\pgfqpoint{2.297566in}{2.687067in}}%
\pgfpathcurveto{\pgfqpoint{2.291743in}{2.681243in}}{\pgfqpoint{2.288470in}{2.673343in}}{\pgfqpoint{2.288470in}{2.665107in}}%
\pgfpathcurveto{\pgfqpoint{2.288470in}{2.656870in}}{\pgfqpoint{2.291743in}{2.648970in}}{\pgfqpoint{2.297566in}{2.643146in}}%
\pgfpathcurveto{\pgfqpoint{2.303390in}{2.637322in}}{\pgfqpoint{2.311290in}{2.634050in}}{\pgfqpoint{2.319527in}{2.634050in}}%
\pgfpathclose%
\pgfusepath{stroke,fill}%
\end{pgfscope}%
\begin{pgfscope}%
\pgfpathrectangle{\pgfqpoint{0.100000in}{0.212622in}}{\pgfqpoint{3.696000in}{3.696000in}}%
\pgfusepath{clip}%
\pgfsetbuttcap%
\pgfsetroundjoin%
\definecolor{currentfill}{rgb}{0.121569,0.466667,0.705882}%
\pgfsetfillcolor{currentfill}%
\pgfsetfillopacity{0.846190}%
\pgfsetlinewidth{1.003750pt}%
\definecolor{currentstroke}{rgb}{0.121569,0.466667,0.705882}%
\pgfsetstrokecolor{currentstroke}%
\pgfsetstrokeopacity{0.846190}%
\pgfsetdash{}{0pt}%
\pgfpathmoveto{\pgfqpoint{2.315865in}{2.629684in}}%
\pgfpathcurveto{\pgfqpoint{2.324101in}{2.629684in}}{\pgfqpoint{2.332001in}{2.632956in}}{\pgfqpoint{2.337825in}{2.638780in}}%
\pgfpathcurveto{\pgfqpoint{2.343649in}{2.644604in}}{\pgfqpoint{2.346921in}{2.652504in}}{\pgfqpoint{2.346921in}{2.660741in}}%
\pgfpathcurveto{\pgfqpoint{2.346921in}{2.668977in}}{\pgfqpoint{2.343649in}{2.676877in}}{\pgfqpoint{2.337825in}{2.682701in}}%
\pgfpathcurveto{\pgfqpoint{2.332001in}{2.688525in}}{\pgfqpoint{2.324101in}{2.691797in}}{\pgfqpoint{2.315865in}{2.691797in}}%
\pgfpathcurveto{\pgfqpoint{2.307629in}{2.691797in}}{\pgfqpoint{2.299729in}{2.688525in}}{\pgfqpoint{2.293905in}{2.682701in}}%
\pgfpathcurveto{\pgfqpoint{2.288081in}{2.676877in}}{\pgfqpoint{2.284808in}{2.668977in}}{\pgfqpoint{2.284808in}{2.660741in}}%
\pgfpathcurveto{\pgfqpoint{2.284808in}{2.652504in}}{\pgfqpoint{2.288081in}{2.644604in}}{\pgfqpoint{2.293905in}{2.638780in}}%
\pgfpathcurveto{\pgfqpoint{2.299729in}{2.632956in}}{\pgfqpoint{2.307629in}{2.629684in}}{\pgfqpoint{2.315865in}{2.629684in}}%
\pgfpathclose%
\pgfusepath{stroke,fill}%
\end{pgfscope}%
\begin{pgfscope}%
\pgfpathrectangle{\pgfqpoint{0.100000in}{0.212622in}}{\pgfqpoint{3.696000in}{3.696000in}}%
\pgfusepath{clip}%
\pgfsetbuttcap%
\pgfsetroundjoin%
\definecolor{currentfill}{rgb}{0.121569,0.466667,0.705882}%
\pgfsetfillcolor{currentfill}%
\pgfsetfillopacity{0.847310}%
\pgfsetlinewidth{1.003750pt}%
\definecolor{currentstroke}{rgb}{0.121569,0.466667,0.705882}%
\pgfsetstrokecolor{currentstroke}%
\pgfsetstrokeopacity{0.847310}%
\pgfsetdash{}{0pt}%
\pgfpathmoveto{\pgfqpoint{2.313266in}{2.626883in}}%
\pgfpathcurveto{\pgfqpoint{2.321502in}{2.626883in}}{\pgfqpoint{2.329402in}{2.630155in}}{\pgfqpoint{2.335226in}{2.635979in}}%
\pgfpathcurveto{\pgfqpoint{2.341050in}{2.641803in}}{\pgfqpoint{2.344322in}{2.649703in}}{\pgfqpoint{2.344322in}{2.657939in}}%
\pgfpathcurveto{\pgfqpoint{2.344322in}{2.666175in}}{\pgfqpoint{2.341050in}{2.674075in}}{\pgfqpoint{2.335226in}{2.679899in}}%
\pgfpathcurveto{\pgfqpoint{2.329402in}{2.685723in}}{\pgfqpoint{2.321502in}{2.688996in}}{\pgfqpoint{2.313266in}{2.688996in}}%
\pgfpathcurveto{\pgfqpoint{2.305030in}{2.688996in}}{\pgfqpoint{2.297130in}{2.685723in}}{\pgfqpoint{2.291306in}{2.679899in}}%
\pgfpathcurveto{\pgfqpoint{2.285482in}{2.674075in}}{\pgfqpoint{2.282209in}{2.666175in}}{\pgfqpoint{2.282209in}{2.657939in}}%
\pgfpathcurveto{\pgfqpoint{2.282209in}{2.649703in}}{\pgfqpoint{2.285482in}{2.641803in}}{\pgfqpoint{2.291306in}{2.635979in}}%
\pgfpathcurveto{\pgfqpoint{2.297130in}{2.630155in}}{\pgfqpoint{2.305030in}{2.626883in}}{\pgfqpoint{2.313266in}{2.626883in}}%
\pgfpathclose%
\pgfusepath{stroke,fill}%
\end{pgfscope}%
\begin{pgfscope}%
\pgfpathrectangle{\pgfqpoint{0.100000in}{0.212622in}}{\pgfqpoint{3.696000in}{3.696000in}}%
\pgfusepath{clip}%
\pgfsetbuttcap%
\pgfsetroundjoin%
\definecolor{currentfill}{rgb}{0.121569,0.466667,0.705882}%
\pgfsetfillcolor{currentfill}%
\pgfsetfillopacity{0.847704}%
\pgfsetlinewidth{1.003750pt}%
\definecolor{currentstroke}{rgb}{0.121569,0.466667,0.705882}%
\pgfsetstrokecolor{currentstroke}%
\pgfsetstrokeopacity{0.847704}%
\pgfsetdash{}{0pt}%
\pgfpathmoveto{\pgfqpoint{2.780390in}{1.437173in}}%
\pgfpathcurveto{\pgfqpoint{2.788626in}{1.437173in}}{\pgfqpoint{2.796526in}{1.440446in}}{\pgfqpoint{2.802350in}{1.446270in}}%
\pgfpathcurveto{\pgfqpoint{2.808174in}{1.452093in}}{\pgfqpoint{2.811447in}{1.459993in}}{\pgfqpoint{2.811447in}{1.468230in}}%
\pgfpathcurveto{\pgfqpoint{2.811447in}{1.476466in}}{\pgfqpoint{2.808174in}{1.484366in}}{\pgfqpoint{2.802350in}{1.490190in}}%
\pgfpathcurveto{\pgfqpoint{2.796526in}{1.496014in}}{\pgfqpoint{2.788626in}{1.499286in}}{\pgfqpoint{2.780390in}{1.499286in}}%
\pgfpathcurveto{\pgfqpoint{2.772154in}{1.499286in}}{\pgfqpoint{2.764254in}{1.496014in}}{\pgfqpoint{2.758430in}{1.490190in}}%
\pgfpathcurveto{\pgfqpoint{2.752606in}{1.484366in}}{\pgfqpoint{2.749334in}{1.476466in}}{\pgfqpoint{2.749334in}{1.468230in}}%
\pgfpathcurveto{\pgfqpoint{2.749334in}{1.459993in}}{\pgfqpoint{2.752606in}{1.452093in}}{\pgfqpoint{2.758430in}{1.446270in}}%
\pgfpathcurveto{\pgfqpoint{2.764254in}{1.440446in}}{\pgfqpoint{2.772154in}{1.437173in}}{\pgfqpoint{2.780390in}{1.437173in}}%
\pgfpathclose%
\pgfusepath{stroke,fill}%
\end{pgfscope}%
\begin{pgfscope}%
\pgfpathrectangle{\pgfqpoint{0.100000in}{0.212622in}}{\pgfqpoint{3.696000in}{3.696000in}}%
\pgfusepath{clip}%
\pgfsetbuttcap%
\pgfsetroundjoin%
\definecolor{currentfill}{rgb}{0.121569,0.466667,0.705882}%
\pgfsetfillcolor{currentfill}%
\pgfsetfillopacity{0.847766}%
\pgfsetlinewidth{1.003750pt}%
\definecolor{currentstroke}{rgb}{0.121569,0.466667,0.705882}%
\pgfsetstrokecolor{currentstroke}%
\pgfsetstrokeopacity{0.847766}%
\pgfsetdash{}{0pt}%
\pgfpathmoveto{\pgfqpoint{1.019961in}{2.223071in}}%
\pgfpathcurveto{\pgfqpoint{1.028197in}{2.223071in}}{\pgfqpoint{1.036097in}{2.226343in}}{\pgfqpoint{1.041921in}{2.232167in}}%
\pgfpathcurveto{\pgfqpoint{1.047745in}{2.237991in}}{\pgfqpoint{1.051017in}{2.245891in}}{\pgfqpoint{1.051017in}{2.254127in}}%
\pgfpathcurveto{\pgfqpoint{1.051017in}{2.262364in}}{\pgfqpoint{1.047745in}{2.270264in}}{\pgfqpoint{1.041921in}{2.276088in}}%
\pgfpathcurveto{\pgfqpoint{1.036097in}{2.281911in}}{\pgfqpoint{1.028197in}{2.285184in}}{\pgfqpoint{1.019961in}{2.285184in}}%
\pgfpathcurveto{\pgfqpoint{1.011725in}{2.285184in}}{\pgfqpoint{1.003825in}{2.281911in}}{\pgfqpoint{0.998001in}{2.276088in}}%
\pgfpathcurveto{\pgfqpoint{0.992177in}{2.270264in}}{\pgfqpoint{0.988904in}{2.262364in}}{\pgfqpoint{0.988904in}{2.254127in}}%
\pgfpathcurveto{\pgfqpoint{0.988904in}{2.245891in}}{\pgfqpoint{0.992177in}{2.237991in}}{\pgfqpoint{0.998001in}{2.232167in}}%
\pgfpathcurveto{\pgfqpoint{1.003825in}{2.226343in}}{\pgfqpoint{1.011725in}{2.223071in}}{\pgfqpoint{1.019961in}{2.223071in}}%
\pgfpathclose%
\pgfusepath{stroke,fill}%
\end{pgfscope}%
\begin{pgfscope}%
\pgfpathrectangle{\pgfqpoint{0.100000in}{0.212622in}}{\pgfqpoint{3.696000in}{3.696000in}}%
\pgfusepath{clip}%
\pgfsetbuttcap%
\pgfsetroundjoin%
\definecolor{currentfill}{rgb}{0.121569,0.466667,0.705882}%
\pgfsetfillcolor{currentfill}%
\pgfsetfillopacity{0.848333}%
\pgfsetlinewidth{1.003750pt}%
\definecolor{currentstroke}{rgb}{0.121569,0.466667,0.705882}%
\pgfsetstrokecolor{currentstroke}%
\pgfsetstrokeopacity{0.848333}%
\pgfsetdash{}{0pt}%
\pgfpathmoveto{\pgfqpoint{2.310950in}{2.624169in}}%
\pgfpathcurveto{\pgfqpoint{2.319186in}{2.624169in}}{\pgfqpoint{2.327086in}{2.627442in}}{\pgfqpoint{2.332910in}{2.633266in}}%
\pgfpathcurveto{\pgfqpoint{2.338734in}{2.639090in}}{\pgfqpoint{2.342006in}{2.646990in}}{\pgfqpoint{2.342006in}{2.655226in}}%
\pgfpathcurveto{\pgfqpoint{2.342006in}{2.663462in}}{\pgfqpoint{2.338734in}{2.671362in}}{\pgfqpoint{2.332910in}{2.677186in}}%
\pgfpathcurveto{\pgfqpoint{2.327086in}{2.683010in}}{\pgfqpoint{2.319186in}{2.686282in}}{\pgfqpoint{2.310950in}{2.686282in}}%
\pgfpathcurveto{\pgfqpoint{2.302714in}{2.686282in}}{\pgfqpoint{2.294813in}{2.683010in}}{\pgfqpoint{2.288990in}{2.677186in}}%
\pgfpathcurveto{\pgfqpoint{2.283166in}{2.671362in}}{\pgfqpoint{2.279893in}{2.663462in}}{\pgfqpoint{2.279893in}{2.655226in}}%
\pgfpathcurveto{\pgfqpoint{2.279893in}{2.646990in}}{\pgfqpoint{2.283166in}{2.639090in}}{\pgfqpoint{2.288990in}{2.633266in}}%
\pgfpathcurveto{\pgfqpoint{2.294813in}{2.627442in}}{\pgfqpoint{2.302714in}{2.624169in}}{\pgfqpoint{2.310950in}{2.624169in}}%
\pgfpathclose%
\pgfusepath{stroke,fill}%
\end{pgfscope}%
\begin{pgfscope}%
\pgfpathrectangle{\pgfqpoint{0.100000in}{0.212622in}}{\pgfqpoint{3.696000in}{3.696000in}}%
\pgfusepath{clip}%
\pgfsetbuttcap%
\pgfsetroundjoin%
\definecolor{currentfill}{rgb}{0.121569,0.466667,0.705882}%
\pgfsetfillcolor{currentfill}%
\pgfsetfillopacity{0.848822}%
\pgfsetlinewidth{1.003750pt}%
\definecolor{currentstroke}{rgb}{0.121569,0.466667,0.705882}%
\pgfsetstrokecolor{currentstroke}%
\pgfsetstrokeopacity{0.848822}%
\pgfsetdash{}{0pt}%
\pgfpathmoveto{\pgfqpoint{2.309771in}{2.622691in}}%
\pgfpathcurveto{\pgfqpoint{2.318007in}{2.622691in}}{\pgfqpoint{2.325907in}{2.625964in}}{\pgfqpoint{2.331731in}{2.631788in}}%
\pgfpathcurveto{\pgfqpoint{2.337555in}{2.637612in}}{\pgfqpoint{2.340827in}{2.645512in}}{\pgfqpoint{2.340827in}{2.653748in}}%
\pgfpathcurveto{\pgfqpoint{2.340827in}{2.661984in}}{\pgfqpoint{2.337555in}{2.669884in}}{\pgfqpoint{2.331731in}{2.675708in}}%
\pgfpathcurveto{\pgfqpoint{2.325907in}{2.681532in}}{\pgfqpoint{2.318007in}{2.684804in}}{\pgfqpoint{2.309771in}{2.684804in}}%
\pgfpathcurveto{\pgfqpoint{2.301535in}{2.684804in}}{\pgfqpoint{2.293634in}{2.681532in}}{\pgfqpoint{2.287811in}{2.675708in}}%
\pgfpathcurveto{\pgfqpoint{2.281987in}{2.669884in}}{\pgfqpoint{2.278714in}{2.661984in}}{\pgfqpoint{2.278714in}{2.653748in}}%
\pgfpathcurveto{\pgfqpoint{2.278714in}{2.645512in}}{\pgfqpoint{2.281987in}{2.637612in}}{\pgfqpoint{2.287811in}{2.631788in}}%
\pgfpathcurveto{\pgfqpoint{2.293634in}{2.625964in}}{\pgfqpoint{2.301535in}{2.622691in}}{\pgfqpoint{2.309771in}{2.622691in}}%
\pgfpathclose%
\pgfusepath{stroke,fill}%
\end{pgfscope}%
\begin{pgfscope}%
\pgfpathrectangle{\pgfqpoint{0.100000in}{0.212622in}}{\pgfqpoint{3.696000in}{3.696000in}}%
\pgfusepath{clip}%
\pgfsetbuttcap%
\pgfsetroundjoin%
\definecolor{currentfill}{rgb}{0.121569,0.466667,0.705882}%
\pgfsetfillcolor{currentfill}%
\pgfsetfillopacity{0.849116}%
\pgfsetlinewidth{1.003750pt}%
\definecolor{currentstroke}{rgb}{0.121569,0.466667,0.705882}%
\pgfsetstrokecolor{currentstroke}%
\pgfsetstrokeopacity{0.849116}%
\pgfsetdash{}{0pt}%
\pgfpathmoveto{\pgfqpoint{1.853752in}{2.745050in}}%
\pgfpathcurveto{\pgfqpoint{1.861988in}{2.745050in}}{\pgfqpoint{1.869888in}{2.748323in}}{\pgfqpoint{1.875712in}{2.754147in}}%
\pgfpathcurveto{\pgfqpoint{1.881536in}{2.759970in}}{\pgfqpoint{1.884808in}{2.767870in}}{\pgfqpoint{1.884808in}{2.776107in}}%
\pgfpathcurveto{\pgfqpoint{1.884808in}{2.784343in}}{\pgfqpoint{1.881536in}{2.792243in}}{\pgfqpoint{1.875712in}{2.798067in}}%
\pgfpathcurveto{\pgfqpoint{1.869888in}{2.803891in}}{\pgfqpoint{1.861988in}{2.807163in}}{\pgfqpoint{1.853752in}{2.807163in}}%
\pgfpathcurveto{\pgfqpoint{1.845516in}{2.807163in}}{\pgfqpoint{1.837616in}{2.803891in}}{\pgfqpoint{1.831792in}{2.798067in}}%
\pgfpathcurveto{\pgfqpoint{1.825968in}{2.792243in}}{\pgfqpoint{1.822695in}{2.784343in}}{\pgfqpoint{1.822695in}{2.776107in}}%
\pgfpathcurveto{\pgfqpoint{1.822695in}{2.767870in}}{\pgfqpoint{1.825968in}{2.759970in}}{\pgfqpoint{1.831792in}{2.754147in}}%
\pgfpathcurveto{\pgfqpoint{1.837616in}{2.748323in}}{\pgfqpoint{1.845516in}{2.745050in}}{\pgfqpoint{1.853752in}{2.745050in}}%
\pgfpathclose%
\pgfusepath{stroke,fill}%
\end{pgfscope}%
\begin{pgfscope}%
\pgfpathrectangle{\pgfqpoint{0.100000in}{0.212622in}}{\pgfqpoint{3.696000in}{3.696000in}}%
\pgfusepath{clip}%
\pgfsetbuttcap%
\pgfsetroundjoin%
\definecolor{currentfill}{rgb}{0.121569,0.466667,0.705882}%
\pgfsetfillcolor{currentfill}%
\pgfsetfillopacity{0.849116}%
\pgfsetlinewidth{1.003750pt}%
\definecolor{currentstroke}{rgb}{0.121569,0.466667,0.705882}%
\pgfsetstrokecolor{currentstroke}%
\pgfsetstrokeopacity{0.849116}%
\pgfsetdash{}{0pt}%
\pgfpathmoveto{\pgfqpoint{1.853752in}{2.745050in}}%
\pgfpathcurveto{\pgfqpoint{1.861988in}{2.745050in}}{\pgfqpoint{1.869888in}{2.748323in}}{\pgfqpoint{1.875712in}{2.754147in}}%
\pgfpathcurveto{\pgfqpoint{1.881536in}{2.759970in}}{\pgfqpoint{1.884808in}{2.767870in}}{\pgfqpoint{1.884808in}{2.776107in}}%
\pgfpathcurveto{\pgfqpoint{1.884808in}{2.784343in}}{\pgfqpoint{1.881536in}{2.792243in}}{\pgfqpoint{1.875712in}{2.798067in}}%
\pgfpathcurveto{\pgfqpoint{1.869888in}{2.803891in}}{\pgfqpoint{1.861988in}{2.807163in}}{\pgfqpoint{1.853752in}{2.807163in}}%
\pgfpathcurveto{\pgfqpoint{1.845516in}{2.807163in}}{\pgfqpoint{1.837616in}{2.803891in}}{\pgfqpoint{1.831792in}{2.798067in}}%
\pgfpathcurveto{\pgfqpoint{1.825968in}{2.792243in}}{\pgfqpoint{1.822695in}{2.784343in}}{\pgfqpoint{1.822695in}{2.776107in}}%
\pgfpathcurveto{\pgfqpoint{1.822695in}{2.767870in}}{\pgfqpoint{1.825968in}{2.759970in}}{\pgfqpoint{1.831792in}{2.754147in}}%
\pgfpathcurveto{\pgfqpoint{1.837616in}{2.748323in}}{\pgfqpoint{1.845516in}{2.745050in}}{\pgfqpoint{1.853752in}{2.745050in}}%
\pgfpathclose%
\pgfusepath{stroke,fill}%
\end{pgfscope}%
\begin{pgfscope}%
\pgfpathrectangle{\pgfqpoint{0.100000in}{0.212622in}}{\pgfqpoint{3.696000in}{3.696000in}}%
\pgfusepath{clip}%
\pgfsetbuttcap%
\pgfsetroundjoin%
\definecolor{currentfill}{rgb}{0.121569,0.466667,0.705882}%
\pgfsetfillcolor{currentfill}%
\pgfsetfillopacity{0.849116}%
\pgfsetlinewidth{1.003750pt}%
\definecolor{currentstroke}{rgb}{0.121569,0.466667,0.705882}%
\pgfsetstrokecolor{currentstroke}%
\pgfsetstrokeopacity{0.849116}%
\pgfsetdash{}{0pt}%
\pgfpathmoveto{\pgfqpoint{1.853752in}{2.745050in}}%
\pgfpathcurveto{\pgfqpoint{1.861988in}{2.745050in}}{\pgfqpoint{1.869888in}{2.748323in}}{\pgfqpoint{1.875712in}{2.754147in}}%
\pgfpathcurveto{\pgfqpoint{1.881536in}{2.759970in}}{\pgfqpoint{1.884808in}{2.767870in}}{\pgfqpoint{1.884808in}{2.776107in}}%
\pgfpathcurveto{\pgfqpoint{1.884808in}{2.784343in}}{\pgfqpoint{1.881536in}{2.792243in}}{\pgfqpoint{1.875712in}{2.798067in}}%
\pgfpathcurveto{\pgfqpoint{1.869888in}{2.803891in}}{\pgfqpoint{1.861988in}{2.807163in}}{\pgfqpoint{1.853752in}{2.807163in}}%
\pgfpathcurveto{\pgfqpoint{1.845516in}{2.807163in}}{\pgfqpoint{1.837616in}{2.803891in}}{\pgfqpoint{1.831792in}{2.798067in}}%
\pgfpathcurveto{\pgfqpoint{1.825968in}{2.792243in}}{\pgfqpoint{1.822695in}{2.784343in}}{\pgfqpoint{1.822695in}{2.776107in}}%
\pgfpathcurveto{\pgfqpoint{1.822695in}{2.767870in}}{\pgfqpoint{1.825968in}{2.759970in}}{\pgfqpoint{1.831792in}{2.754147in}}%
\pgfpathcurveto{\pgfqpoint{1.837616in}{2.748323in}}{\pgfqpoint{1.845516in}{2.745050in}}{\pgfqpoint{1.853752in}{2.745050in}}%
\pgfpathclose%
\pgfusepath{stroke,fill}%
\end{pgfscope}%
\begin{pgfscope}%
\pgfpathrectangle{\pgfqpoint{0.100000in}{0.212622in}}{\pgfqpoint{3.696000in}{3.696000in}}%
\pgfusepath{clip}%
\pgfsetbuttcap%
\pgfsetroundjoin%
\definecolor{currentfill}{rgb}{0.121569,0.466667,0.705882}%
\pgfsetfillcolor{currentfill}%
\pgfsetfillopacity{0.849116}%
\pgfsetlinewidth{1.003750pt}%
\definecolor{currentstroke}{rgb}{0.121569,0.466667,0.705882}%
\pgfsetstrokecolor{currentstroke}%
\pgfsetstrokeopacity{0.849116}%
\pgfsetdash{}{0pt}%
\pgfpathmoveto{\pgfqpoint{1.853752in}{2.745050in}}%
\pgfpathcurveto{\pgfqpoint{1.861988in}{2.745050in}}{\pgfqpoint{1.869888in}{2.748323in}}{\pgfqpoint{1.875712in}{2.754147in}}%
\pgfpathcurveto{\pgfqpoint{1.881536in}{2.759970in}}{\pgfqpoint{1.884808in}{2.767870in}}{\pgfqpoint{1.884808in}{2.776107in}}%
\pgfpathcurveto{\pgfqpoint{1.884808in}{2.784343in}}{\pgfqpoint{1.881536in}{2.792243in}}{\pgfqpoint{1.875712in}{2.798067in}}%
\pgfpathcurveto{\pgfqpoint{1.869888in}{2.803891in}}{\pgfqpoint{1.861988in}{2.807163in}}{\pgfqpoint{1.853752in}{2.807163in}}%
\pgfpathcurveto{\pgfqpoint{1.845516in}{2.807163in}}{\pgfqpoint{1.837616in}{2.803891in}}{\pgfqpoint{1.831792in}{2.798067in}}%
\pgfpathcurveto{\pgfqpoint{1.825968in}{2.792243in}}{\pgfqpoint{1.822695in}{2.784343in}}{\pgfqpoint{1.822695in}{2.776107in}}%
\pgfpathcurveto{\pgfqpoint{1.822695in}{2.767870in}}{\pgfqpoint{1.825968in}{2.759970in}}{\pgfqpoint{1.831792in}{2.754147in}}%
\pgfpathcurveto{\pgfqpoint{1.837616in}{2.748323in}}{\pgfqpoint{1.845516in}{2.745050in}}{\pgfqpoint{1.853752in}{2.745050in}}%
\pgfpathclose%
\pgfusepath{stroke,fill}%
\end{pgfscope}%
\begin{pgfscope}%
\pgfpathrectangle{\pgfqpoint{0.100000in}{0.212622in}}{\pgfqpoint{3.696000in}{3.696000in}}%
\pgfusepath{clip}%
\pgfsetbuttcap%
\pgfsetroundjoin%
\definecolor{currentfill}{rgb}{0.121569,0.466667,0.705882}%
\pgfsetfillcolor{currentfill}%
\pgfsetfillopacity{0.849116}%
\pgfsetlinewidth{1.003750pt}%
\definecolor{currentstroke}{rgb}{0.121569,0.466667,0.705882}%
\pgfsetstrokecolor{currentstroke}%
\pgfsetstrokeopacity{0.849116}%
\pgfsetdash{}{0pt}%
\pgfpathmoveto{\pgfqpoint{1.853752in}{2.745050in}}%
\pgfpathcurveto{\pgfqpoint{1.861988in}{2.745050in}}{\pgfqpoint{1.869888in}{2.748323in}}{\pgfqpoint{1.875712in}{2.754147in}}%
\pgfpathcurveto{\pgfqpoint{1.881536in}{2.759970in}}{\pgfqpoint{1.884808in}{2.767870in}}{\pgfqpoint{1.884808in}{2.776107in}}%
\pgfpathcurveto{\pgfqpoint{1.884808in}{2.784343in}}{\pgfqpoint{1.881536in}{2.792243in}}{\pgfqpoint{1.875712in}{2.798067in}}%
\pgfpathcurveto{\pgfqpoint{1.869888in}{2.803891in}}{\pgfqpoint{1.861988in}{2.807163in}}{\pgfqpoint{1.853752in}{2.807163in}}%
\pgfpathcurveto{\pgfqpoint{1.845516in}{2.807163in}}{\pgfqpoint{1.837616in}{2.803891in}}{\pgfqpoint{1.831792in}{2.798067in}}%
\pgfpathcurveto{\pgfqpoint{1.825968in}{2.792243in}}{\pgfqpoint{1.822695in}{2.784343in}}{\pgfqpoint{1.822695in}{2.776107in}}%
\pgfpathcurveto{\pgfqpoint{1.822695in}{2.767870in}}{\pgfqpoint{1.825968in}{2.759970in}}{\pgfqpoint{1.831792in}{2.754147in}}%
\pgfpathcurveto{\pgfqpoint{1.837616in}{2.748323in}}{\pgfqpoint{1.845516in}{2.745050in}}{\pgfqpoint{1.853752in}{2.745050in}}%
\pgfpathclose%
\pgfusepath{stroke,fill}%
\end{pgfscope}%
\begin{pgfscope}%
\pgfpathrectangle{\pgfqpoint{0.100000in}{0.212622in}}{\pgfqpoint{3.696000in}{3.696000in}}%
\pgfusepath{clip}%
\pgfsetbuttcap%
\pgfsetroundjoin%
\definecolor{currentfill}{rgb}{0.121569,0.466667,0.705882}%
\pgfsetfillcolor{currentfill}%
\pgfsetfillopacity{0.849116}%
\pgfsetlinewidth{1.003750pt}%
\definecolor{currentstroke}{rgb}{0.121569,0.466667,0.705882}%
\pgfsetstrokecolor{currentstroke}%
\pgfsetstrokeopacity{0.849116}%
\pgfsetdash{}{0pt}%
\pgfpathmoveto{\pgfqpoint{1.853752in}{2.745050in}}%
\pgfpathcurveto{\pgfqpoint{1.861988in}{2.745050in}}{\pgfqpoint{1.869888in}{2.748323in}}{\pgfqpoint{1.875712in}{2.754147in}}%
\pgfpathcurveto{\pgfqpoint{1.881536in}{2.759970in}}{\pgfqpoint{1.884808in}{2.767870in}}{\pgfqpoint{1.884808in}{2.776107in}}%
\pgfpathcurveto{\pgfqpoint{1.884808in}{2.784343in}}{\pgfqpoint{1.881536in}{2.792243in}}{\pgfqpoint{1.875712in}{2.798067in}}%
\pgfpathcurveto{\pgfqpoint{1.869888in}{2.803891in}}{\pgfqpoint{1.861988in}{2.807163in}}{\pgfqpoint{1.853752in}{2.807163in}}%
\pgfpathcurveto{\pgfqpoint{1.845516in}{2.807163in}}{\pgfqpoint{1.837616in}{2.803891in}}{\pgfqpoint{1.831792in}{2.798067in}}%
\pgfpathcurveto{\pgfqpoint{1.825968in}{2.792243in}}{\pgfqpoint{1.822695in}{2.784343in}}{\pgfqpoint{1.822695in}{2.776107in}}%
\pgfpathcurveto{\pgfqpoint{1.822695in}{2.767870in}}{\pgfqpoint{1.825968in}{2.759970in}}{\pgfqpoint{1.831792in}{2.754147in}}%
\pgfpathcurveto{\pgfqpoint{1.837616in}{2.748323in}}{\pgfqpoint{1.845516in}{2.745050in}}{\pgfqpoint{1.853752in}{2.745050in}}%
\pgfpathclose%
\pgfusepath{stroke,fill}%
\end{pgfscope}%
\begin{pgfscope}%
\pgfpathrectangle{\pgfqpoint{0.100000in}{0.212622in}}{\pgfqpoint{3.696000in}{3.696000in}}%
\pgfusepath{clip}%
\pgfsetbuttcap%
\pgfsetroundjoin%
\definecolor{currentfill}{rgb}{0.121569,0.466667,0.705882}%
\pgfsetfillcolor{currentfill}%
\pgfsetfillopacity{0.849116}%
\pgfsetlinewidth{1.003750pt}%
\definecolor{currentstroke}{rgb}{0.121569,0.466667,0.705882}%
\pgfsetstrokecolor{currentstroke}%
\pgfsetstrokeopacity{0.849116}%
\pgfsetdash{}{0pt}%
\pgfpathmoveto{\pgfqpoint{1.853752in}{2.745050in}}%
\pgfpathcurveto{\pgfqpoint{1.861988in}{2.745050in}}{\pgfqpoint{1.869888in}{2.748323in}}{\pgfqpoint{1.875712in}{2.754147in}}%
\pgfpathcurveto{\pgfqpoint{1.881536in}{2.759970in}}{\pgfqpoint{1.884808in}{2.767870in}}{\pgfqpoint{1.884808in}{2.776107in}}%
\pgfpathcurveto{\pgfqpoint{1.884808in}{2.784343in}}{\pgfqpoint{1.881536in}{2.792243in}}{\pgfqpoint{1.875712in}{2.798067in}}%
\pgfpathcurveto{\pgfqpoint{1.869888in}{2.803891in}}{\pgfqpoint{1.861988in}{2.807163in}}{\pgfqpoint{1.853752in}{2.807163in}}%
\pgfpathcurveto{\pgfqpoint{1.845516in}{2.807163in}}{\pgfqpoint{1.837616in}{2.803891in}}{\pgfqpoint{1.831792in}{2.798067in}}%
\pgfpathcurveto{\pgfqpoint{1.825968in}{2.792243in}}{\pgfqpoint{1.822695in}{2.784343in}}{\pgfqpoint{1.822695in}{2.776107in}}%
\pgfpathcurveto{\pgfqpoint{1.822695in}{2.767870in}}{\pgfqpoint{1.825968in}{2.759970in}}{\pgfqpoint{1.831792in}{2.754147in}}%
\pgfpathcurveto{\pgfqpoint{1.837616in}{2.748323in}}{\pgfqpoint{1.845516in}{2.745050in}}{\pgfqpoint{1.853752in}{2.745050in}}%
\pgfpathclose%
\pgfusepath{stroke,fill}%
\end{pgfscope}%
\begin{pgfscope}%
\pgfpathrectangle{\pgfqpoint{0.100000in}{0.212622in}}{\pgfqpoint{3.696000in}{3.696000in}}%
\pgfusepath{clip}%
\pgfsetbuttcap%
\pgfsetroundjoin%
\definecolor{currentfill}{rgb}{0.121569,0.466667,0.705882}%
\pgfsetfillcolor{currentfill}%
\pgfsetfillopacity{0.849116}%
\pgfsetlinewidth{1.003750pt}%
\definecolor{currentstroke}{rgb}{0.121569,0.466667,0.705882}%
\pgfsetstrokecolor{currentstroke}%
\pgfsetstrokeopacity{0.849116}%
\pgfsetdash{}{0pt}%
\pgfpathmoveto{\pgfqpoint{1.853752in}{2.745050in}}%
\pgfpathcurveto{\pgfqpoint{1.861988in}{2.745050in}}{\pgfqpoint{1.869888in}{2.748323in}}{\pgfqpoint{1.875712in}{2.754147in}}%
\pgfpathcurveto{\pgfqpoint{1.881536in}{2.759970in}}{\pgfqpoint{1.884808in}{2.767870in}}{\pgfqpoint{1.884808in}{2.776107in}}%
\pgfpathcurveto{\pgfqpoint{1.884808in}{2.784343in}}{\pgfqpoint{1.881536in}{2.792243in}}{\pgfqpoint{1.875712in}{2.798067in}}%
\pgfpathcurveto{\pgfqpoint{1.869888in}{2.803891in}}{\pgfqpoint{1.861988in}{2.807163in}}{\pgfqpoint{1.853752in}{2.807163in}}%
\pgfpathcurveto{\pgfqpoint{1.845516in}{2.807163in}}{\pgfqpoint{1.837616in}{2.803891in}}{\pgfqpoint{1.831792in}{2.798067in}}%
\pgfpathcurveto{\pgfqpoint{1.825968in}{2.792243in}}{\pgfqpoint{1.822695in}{2.784343in}}{\pgfqpoint{1.822695in}{2.776107in}}%
\pgfpathcurveto{\pgfqpoint{1.822695in}{2.767870in}}{\pgfqpoint{1.825968in}{2.759970in}}{\pgfqpoint{1.831792in}{2.754147in}}%
\pgfpathcurveto{\pgfqpoint{1.837616in}{2.748323in}}{\pgfqpoint{1.845516in}{2.745050in}}{\pgfqpoint{1.853752in}{2.745050in}}%
\pgfpathclose%
\pgfusepath{stroke,fill}%
\end{pgfscope}%
\begin{pgfscope}%
\pgfpathrectangle{\pgfqpoint{0.100000in}{0.212622in}}{\pgfqpoint{3.696000in}{3.696000in}}%
\pgfusepath{clip}%
\pgfsetbuttcap%
\pgfsetroundjoin%
\definecolor{currentfill}{rgb}{0.121569,0.466667,0.705882}%
\pgfsetfillcolor{currentfill}%
\pgfsetfillopacity{0.849116}%
\pgfsetlinewidth{1.003750pt}%
\definecolor{currentstroke}{rgb}{0.121569,0.466667,0.705882}%
\pgfsetstrokecolor{currentstroke}%
\pgfsetstrokeopacity{0.849116}%
\pgfsetdash{}{0pt}%
\pgfpathmoveto{\pgfqpoint{1.853752in}{2.745050in}}%
\pgfpathcurveto{\pgfqpoint{1.861988in}{2.745050in}}{\pgfqpoint{1.869888in}{2.748323in}}{\pgfqpoint{1.875712in}{2.754147in}}%
\pgfpathcurveto{\pgfqpoint{1.881536in}{2.759970in}}{\pgfqpoint{1.884808in}{2.767870in}}{\pgfqpoint{1.884808in}{2.776107in}}%
\pgfpathcurveto{\pgfqpoint{1.884808in}{2.784343in}}{\pgfqpoint{1.881536in}{2.792243in}}{\pgfqpoint{1.875712in}{2.798067in}}%
\pgfpathcurveto{\pgfqpoint{1.869888in}{2.803891in}}{\pgfqpoint{1.861988in}{2.807163in}}{\pgfqpoint{1.853752in}{2.807163in}}%
\pgfpathcurveto{\pgfqpoint{1.845516in}{2.807163in}}{\pgfqpoint{1.837616in}{2.803891in}}{\pgfqpoint{1.831792in}{2.798067in}}%
\pgfpathcurveto{\pgfqpoint{1.825968in}{2.792243in}}{\pgfqpoint{1.822695in}{2.784343in}}{\pgfqpoint{1.822695in}{2.776107in}}%
\pgfpathcurveto{\pgfqpoint{1.822695in}{2.767870in}}{\pgfqpoint{1.825968in}{2.759970in}}{\pgfqpoint{1.831792in}{2.754147in}}%
\pgfpathcurveto{\pgfqpoint{1.837616in}{2.748323in}}{\pgfqpoint{1.845516in}{2.745050in}}{\pgfqpoint{1.853752in}{2.745050in}}%
\pgfpathclose%
\pgfusepath{stroke,fill}%
\end{pgfscope}%
\begin{pgfscope}%
\pgfpathrectangle{\pgfqpoint{0.100000in}{0.212622in}}{\pgfqpoint{3.696000in}{3.696000in}}%
\pgfusepath{clip}%
\pgfsetbuttcap%
\pgfsetroundjoin%
\definecolor{currentfill}{rgb}{0.121569,0.466667,0.705882}%
\pgfsetfillcolor{currentfill}%
\pgfsetfillopacity{0.849116}%
\pgfsetlinewidth{1.003750pt}%
\definecolor{currentstroke}{rgb}{0.121569,0.466667,0.705882}%
\pgfsetstrokecolor{currentstroke}%
\pgfsetstrokeopacity{0.849116}%
\pgfsetdash{}{0pt}%
\pgfpathmoveto{\pgfqpoint{1.853752in}{2.745050in}}%
\pgfpathcurveto{\pgfqpoint{1.861988in}{2.745050in}}{\pgfqpoint{1.869888in}{2.748323in}}{\pgfqpoint{1.875712in}{2.754147in}}%
\pgfpathcurveto{\pgfqpoint{1.881536in}{2.759970in}}{\pgfqpoint{1.884808in}{2.767870in}}{\pgfqpoint{1.884808in}{2.776107in}}%
\pgfpathcurveto{\pgfqpoint{1.884808in}{2.784343in}}{\pgfqpoint{1.881536in}{2.792243in}}{\pgfqpoint{1.875712in}{2.798067in}}%
\pgfpathcurveto{\pgfqpoint{1.869888in}{2.803891in}}{\pgfqpoint{1.861988in}{2.807163in}}{\pgfqpoint{1.853752in}{2.807163in}}%
\pgfpathcurveto{\pgfqpoint{1.845516in}{2.807163in}}{\pgfqpoint{1.837616in}{2.803891in}}{\pgfqpoint{1.831792in}{2.798067in}}%
\pgfpathcurveto{\pgfqpoint{1.825968in}{2.792243in}}{\pgfqpoint{1.822695in}{2.784343in}}{\pgfqpoint{1.822695in}{2.776107in}}%
\pgfpathcurveto{\pgfqpoint{1.822695in}{2.767870in}}{\pgfqpoint{1.825968in}{2.759970in}}{\pgfqpoint{1.831792in}{2.754147in}}%
\pgfpathcurveto{\pgfqpoint{1.837616in}{2.748323in}}{\pgfqpoint{1.845516in}{2.745050in}}{\pgfqpoint{1.853752in}{2.745050in}}%
\pgfpathclose%
\pgfusepath{stroke,fill}%
\end{pgfscope}%
\begin{pgfscope}%
\pgfpathrectangle{\pgfqpoint{0.100000in}{0.212622in}}{\pgfqpoint{3.696000in}{3.696000in}}%
\pgfusepath{clip}%
\pgfsetbuttcap%
\pgfsetroundjoin%
\definecolor{currentfill}{rgb}{0.121569,0.466667,0.705882}%
\pgfsetfillcolor{currentfill}%
\pgfsetfillopacity{0.849116}%
\pgfsetlinewidth{1.003750pt}%
\definecolor{currentstroke}{rgb}{0.121569,0.466667,0.705882}%
\pgfsetstrokecolor{currentstroke}%
\pgfsetstrokeopacity{0.849116}%
\pgfsetdash{}{0pt}%
\pgfpathmoveto{\pgfqpoint{1.853752in}{2.745050in}}%
\pgfpathcurveto{\pgfqpoint{1.861988in}{2.745050in}}{\pgfqpoint{1.869888in}{2.748323in}}{\pgfqpoint{1.875712in}{2.754147in}}%
\pgfpathcurveto{\pgfqpoint{1.881536in}{2.759970in}}{\pgfqpoint{1.884808in}{2.767870in}}{\pgfqpoint{1.884808in}{2.776107in}}%
\pgfpathcurveto{\pgfqpoint{1.884808in}{2.784343in}}{\pgfqpoint{1.881536in}{2.792243in}}{\pgfqpoint{1.875712in}{2.798067in}}%
\pgfpathcurveto{\pgfqpoint{1.869888in}{2.803891in}}{\pgfqpoint{1.861988in}{2.807163in}}{\pgfqpoint{1.853752in}{2.807163in}}%
\pgfpathcurveto{\pgfqpoint{1.845516in}{2.807163in}}{\pgfqpoint{1.837616in}{2.803891in}}{\pgfqpoint{1.831792in}{2.798067in}}%
\pgfpathcurveto{\pgfqpoint{1.825968in}{2.792243in}}{\pgfqpoint{1.822695in}{2.784343in}}{\pgfqpoint{1.822695in}{2.776107in}}%
\pgfpathcurveto{\pgfqpoint{1.822695in}{2.767870in}}{\pgfqpoint{1.825968in}{2.759970in}}{\pgfqpoint{1.831792in}{2.754147in}}%
\pgfpathcurveto{\pgfqpoint{1.837616in}{2.748323in}}{\pgfqpoint{1.845516in}{2.745050in}}{\pgfqpoint{1.853752in}{2.745050in}}%
\pgfpathclose%
\pgfusepath{stroke,fill}%
\end{pgfscope}%
\begin{pgfscope}%
\pgfpathrectangle{\pgfqpoint{0.100000in}{0.212622in}}{\pgfqpoint{3.696000in}{3.696000in}}%
\pgfusepath{clip}%
\pgfsetbuttcap%
\pgfsetroundjoin%
\definecolor{currentfill}{rgb}{0.121569,0.466667,0.705882}%
\pgfsetfillcolor{currentfill}%
\pgfsetfillopacity{0.849116}%
\pgfsetlinewidth{1.003750pt}%
\definecolor{currentstroke}{rgb}{0.121569,0.466667,0.705882}%
\pgfsetstrokecolor{currentstroke}%
\pgfsetstrokeopacity{0.849116}%
\pgfsetdash{}{0pt}%
\pgfpathmoveto{\pgfqpoint{1.853752in}{2.745050in}}%
\pgfpathcurveto{\pgfqpoint{1.861988in}{2.745050in}}{\pgfqpoint{1.869888in}{2.748323in}}{\pgfqpoint{1.875712in}{2.754147in}}%
\pgfpathcurveto{\pgfqpoint{1.881536in}{2.759970in}}{\pgfqpoint{1.884808in}{2.767870in}}{\pgfqpoint{1.884808in}{2.776107in}}%
\pgfpathcurveto{\pgfqpoint{1.884808in}{2.784343in}}{\pgfqpoint{1.881536in}{2.792243in}}{\pgfqpoint{1.875712in}{2.798067in}}%
\pgfpathcurveto{\pgfqpoint{1.869888in}{2.803891in}}{\pgfqpoint{1.861988in}{2.807163in}}{\pgfqpoint{1.853752in}{2.807163in}}%
\pgfpathcurveto{\pgfqpoint{1.845516in}{2.807163in}}{\pgfqpoint{1.837616in}{2.803891in}}{\pgfqpoint{1.831792in}{2.798067in}}%
\pgfpathcurveto{\pgfqpoint{1.825968in}{2.792243in}}{\pgfqpoint{1.822695in}{2.784343in}}{\pgfqpoint{1.822695in}{2.776107in}}%
\pgfpathcurveto{\pgfqpoint{1.822695in}{2.767870in}}{\pgfqpoint{1.825968in}{2.759970in}}{\pgfqpoint{1.831792in}{2.754147in}}%
\pgfpathcurveto{\pgfqpoint{1.837616in}{2.748323in}}{\pgfqpoint{1.845516in}{2.745050in}}{\pgfqpoint{1.853752in}{2.745050in}}%
\pgfpathclose%
\pgfusepath{stroke,fill}%
\end{pgfscope}%
\begin{pgfscope}%
\pgfpathrectangle{\pgfqpoint{0.100000in}{0.212622in}}{\pgfqpoint{3.696000in}{3.696000in}}%
\pgfusepath{clip}%
\pgfsetbuttcap%
\pgfsetroundjoin%
\definecolor{currentfill}{rgb}{0.121569,0.466667,0.705882}%
\pgfsetfillcolor{currentfill}%
\pgfsetfillopacity{0.849116}%
\pgfsetlinewidth{1.003750pt}%
\definecolor{currentstroke}{rgb}{0.121569,0.466667,0.705882}%
\pgfsetstrokecolor{currentstroke}%
\pgfsetstrokeopacity{0.849116}%
\pgfsetdash{}{0pt}%
\pgfpathmoveto{\pgfqpoint{1.853752in}{2.745050in}}%
\pgfpathcurveto{\pgfqpoint{1.861988in}{2.745050in}}{\pgfqpoint{1.869888in}{2.748323in}}{\pgfqpoint{1.875712in}{2.754147in}}%
\pgfpathcurveto{\pgfqpoint{1.881536in}{2.759970in}}{\pgfqpoint{1.884808in}{2.767870in}}{\pgfqpoint{1.884808in}{2.776107in}}%
\pgfpathcurveto{\pgfqpoint{1.884808in}{2.784343in}}{\pgfqpoint{1.881536in}{2.792243in}}{\pgfqpoint{1.875712in}{2.798067in}}%
\pgfpathcurveto{\pgfqpoint{1.869888in}{2.803891in}}{\pgfqpoint{1.861988in}{2.807163in}}{\pgfqpoint{1.853752in}{2.807163in}}%
\pgfpathcurveto{\pgfqpoint{1.845516in}{2.807163in}}{\pgfqpoint{1.837616in}{2.803891in}}{\pgfqpoint{1.831792in}{2.798067in}}%
\pgfpathcurveto{\pgfqpoint{1.825968in}{2.792243in}}{\pgfqpoint{1.822695in}{2.784343in}}{\pgfqpoint{1.822695in}{2.776107in}}%
\pgfpathcurveto{\pgfqpoint{1.822695in}{2.767870in}}{\pgfqpoint{1.825968in}{2.759970in}}{\pgfqpoint{1.831792in}{2.754147in}}%
\pgfpathcurveto{\pgfqpoint{1.837616in}{2.748323in}}{\pgfqpoint{1.845516in}{2.745050in}}{\pgfqpoint{1.853752in}{2.745050in}}%
\pgfpathclose%
\pgfusepath{stroke,fill}%
\end{pgfscope}%
\begin{pgfscope}%
\pgfpathrectangle{\pgfqpoint{0.100000in}{0.212622in}}{\pgfqpoint{3.696000in}{3.696000in}}%
\pgfusepath{clip}%
\pgfsetbuttcap%
\pgfsetroundjoin%
\definecolor{currentfill}{rgb}{0.121569,0.466667,0.705882}%
\pgfsetfillcolor{currentfill}%
\pgfsetfillopacity{0.849116}%
\pgfsetlinewidth{1.003750pt}%
\definecolor{currentstroke}{rgb}{0.121569,0.466667,0.705882}%
\pgfsetstrokecolor{currentstroke}%
\pgfsetstrokeopacity{0.849116}%
\pgfsetdash{}{0pt}%
\pgfpathmoveto{\pgfqpoint{1.853752in}{2.745050in}}%
\pgfpathcurveto{\pgfqpoint{1.861988in}{2.745050in}}{\pgfqpoint{1.869888in}{2.748323in}}{\pgfqpoint{1.875712in}{2.754147in}}%
\pgfpathcurveto{\pgfqpoint{1.881536in}{2.759970in}}{\pgfqpoint{1.884808in}{2.767870in}}{\pgfqpoint{1.884808in}{2.776107in}}%
\pgfpathcurveto{\pgfqpoint{1.884808in}{2.784343in}}{\pgfqpoint{1.881536in}{2.792243in}}{\pgfqpoint{1.875712in}{2.798067in}}%
\pgfpathcurveto{\pgfqpoint{1.869888in}{2.803891in}}{\pgfqpoint{1.861988in}{2.807163in}}{\pgfqpoint{1.853752in}{2.807163in}}%
\pgfpathcurveto{\pgfqpoint{1.845516in}{2.807163in}}{\pgfqpoint{1.837616in}{2.803891in}}{\pgfqpoint{1.831792in}{2.798067in}}%
\pgfpathcurveto{\pgfqpoint{1.825968in}{2.792243in}}{\pgfqpoint{1.822695in}{2.784343in}}{\pgfqpoint{1.822695in}{2.776107in}}%
\pgfpathcurveto{\pgfqpoint{1.822695in}{2.767870in}}{\pgfqpoint{1.825968in}{2.759970in}}{\pgfqpoint{1.831792in}{2.754147in}}%
\pgfpathcurveto{\pgfqpoint{1.837616in}{2.748323in}}{\pgfqpoint{1.845516in}{2.745050in}}{\pgfqpoint{1.853752in}{2.745050in}}%
\pgfpathclose%
\pgfusepath{stroke,fill}%
\end{pgfscope}%
\begin{pgfscope}%
\pgfpathrectangle{\pgfqpoint{0.100000in}{0.212622in}}{\pgfqpoint{3.696000in}{3.696000in}}%
\pgfusepath{clip}%
\pgfsetbuttcap%
\pgfsetroundjoin%
\definecolor{currentfill}{rgb}{0.121569,0.466667,0.705882}%
\pgfsetfillcolor{currentfill}%
\pgfsetfillopacity{0.849116}%
\pgfsetlinewidth{1.003750pt}%
\definecolor{currentstroke}{rgb}{0.121569,0.466667,0.705882}%
\pgfsetstrokecolor{currentstroke}%
\pgfsetstrokeopacity{0.849116}%
\pgfsetdash{}{0pt}%
\pgfpathmoveto{\pgfqpoint{1.853752in}{2.745050in}}%
\pgfpathcurveto{\pgfqpoint{1.861988in}{2.745050in}}{\pgfqpoint{1.869888in}{2.748323in}}{\pgfqpoint{1.875712in}{2.754147in}}%
\pgfpathcurveto{\pgfqpoint{1.881536in}{2.759970in}}{\pgfqpoint{1.884808in}{2.767870in}}{\pgfqpoint{1.884808in}{2.776107in}}%
\pgfpathcurveto{\pgfqpoint{1.884808in}{2.784343in}}{\pgfqpoint{1.881536in}{2.792243in}}{\pgfqpoint{1.875712in}{2.798067in}}%
\pgfpathcurveto{\pgfqpoint{1.869888in}{2.803891in}}{\pgfqpoint{1.861988in}{2.807163in}}{\pgfqpoint{1.853752in}{2.807163in}}%
\pgfpathcurveto{\pgfqpoint{1.845516in}{2.807163in}}{\pgfqpoint{1.837616in}{2.803891in}}{\pgfqpoint{1.831792in}{2.798067in}}%
\pgfpathcurveto{\pgfqpoint{1.825968in}{2.792243in}}{\pgfqpoint{1.822695in}{2.784343in}}{\pgfqpoint{1.822695in}{2.776107in}}%
\pgfpathcurveto{\pgfqpoint{1.822695in}{2.767870in}}{\pgfqpoint{1.825968in}{2.759970in}}{\pgfqpoint{1.831792in}{2.754147in}}%
\pgfpathcurveto{\pgfqpoint{1.837616in}{2.748323in}}{\pgfqpoint{1.845516in}{2.745050in}}{\pgfqpoint{1.853752in}{2.745050in}}%
\pgfpathclose%
\pgfusepath{stroke,fill}%
\end{pgfscope}%
\begin{pgfscope}%
\pgfpathrectangle{\pgfqpoint{0.100000in}{0.212622in}}{\pgfqpoint{3.696000in}{3.696000in}}%
\pgfusepath{clip}%
\pgfsetbuttcap%
\pgfsetroundjoin%
\definecolor{currentfill}{rgb}{0.121569,0.466667,0.705882}%
\pgfsetfillcolor{currentfill}%
\pgfsetfillopacity{0.849116}%
\pgfsetlinewidth{1.003750pt}%
\definecolor{currentstroke}{rgb}{0.121569,0.466667,0.705882}%
\pgfsetstrokecolor{currentstroke}%
\pgfsetstrokeopacity{0.849116}%
\pgfsetdash{}{0pt}%
\pgfpathmoveto{\pgfqpoint{1.853752in}{2.745050in}}%
\pgfpathcurveto{\pgfqpoint{1.861988in}{2.745050in}}{\pgfqpoint{1.869888in}{2.748323in}}{\pgfqpoint{1.875712in}{2.754147in}}%
\pgfpathcurveto{\pgfqpoint{1.881536in}{2.759970in}}{\pgfqpoint{1.884808in}{2.767870in}}{\pgfqpoint{1.884808in}{2.776107in}}%
\pgfpathcurveto{\pgfqpoint{1.884808in}{2.784343in}}{\pgfqpoint{1.881536in}{2.792243in}}{\pgfqpoint{1.875712in}{2.798067in}}%
\pgfpathcurveto{\pgfqpoint{1.869888in}{2.803891in}}{\pgfqpoint{1.861988in}{2.807163in}}{\pgfqpoint{1.853752in}{2.807163in}}%
\pgfpathcurveto{\pgfqpoint{1.845516in}{2.807163in}}{\pgfqpoint{1.837616in}{2.803891in}}{\pgfqpoint{1.831792in}{2.798067in}}%
\pgfpathcurveto{\pgfqpoint{1.825968in}{2.792243in}}{\pgfqpoint{1.822695in}{2.784343in}}{\pgfqpoint{1.822695in}{2.776107in}}%
\pgfpathcurveto{\pgfqpoint{1.822695in}{2.767870in}}{\pgfqpoint{1.825968in}{2.759970in}}{\pgfqpoint{1.831792in}{2.754147in}}%
\pgfpathcurveto{\pgfqpoint{1.837616in}{2.748323in}}{\pgfqpoint{1.845516in}{2.745050in}}{\pgfqpoint{1.853752in}{2.745050in}}%
\pgfpathclose%
\pgfusepath{stroke,fill}%
\end{pgfscope}%
\begin{pgfscope}%
\pgfpathrectangle{\pgfqpoint{0.100000in}{0.212622in}}{\pgfqpoint{3.696000in}{3.696000in}}%
\pgfusepath{clip}%
\pgfsetbuttcap%
\pgfsetroundjoin%
\definecolor{currentfill}{rgb}{0.121569,0.466667,0.705882}%
\pgfsetfillcolor{currentfill}%
\pgfsetfillopacity{0.849116}%
\pgfsetlinewidth{1.003750pt}%
\definecolor{currentstroke}{rgb}{0.121569,0.466667,0.705882}%
\pgfsetstrokecolor{currentstroke}%
\pgfsetstrokeopacity{0.849116}%
\pgfsetdash{}{0pt}%
\pgfpathmoveto{\pgfqpoint{1.853752in}{2.745050in}}%
\pgfpathcurveto{\pgfqpoint{1.861988in}{2.745050in}}{\pgfqpoint{1.869888in}{2.748323in}}{\pgfqpoint{1.875712in}{2.754147in}}%
\pgfpathcurveto{\pgfqpoint{1.881536in}{2.759970in}}{\pgfqpoint{1.884808in}{2.767870in}}{\pgfqpoint{1.884808in}{2.776107in}}%
\pgfpathcurveto{\pgfqpoint{1.884808in}{2.784343in}}{\pgfqpoint{1.881536in}{2.792243in}}{\pgfqpoint{1.875712in}{2.798067in}}%
\pgfpathcurveto{\pgfqpoint{1.869888in}{2.803891in}}{\pgfqpoint{1.861988in}{2.807163in}}{\pgfqpoint{1.853752in}{2.807163in}}%
\pgfpathcurveto{\pgfqpoint{1.845516in}{2.807163in}}{\pgfqpoint{1.837616in}{2.803891in}}{\pgfqpoint{1.831792in}{2.798067in}}%
\pgfpathcurveto{\pgfqpoint{1.825968in}{2.792243in}}{\pgfqpoint{1.822695in}{2.784343in}}{\pgfqpoint{1.822695in}{2.776107in}}%
\pgfpathcurveto{\pgfqpoint{1.822695in}{2.767870in}}{\pgfqpoint{1.825968in}{2.759970in}}{\pgfqpoint{1.831792in}{2.754147in}}%
\pgfpathcurveto{\pgfqpoint{1.837616in}{2.748323in}}{\pgfqpoint{1.845516in}{2.745050in}}{\pgfqpoint{1.853752in}{2.745050in}}%
\pgfpathclose%
\pgfusepath{stroke,fill}%
\end{pgfscope}%
\begin{pgfscope}%
\pgfpathrectangle{\pgfqpoint{0.100000in}{0.212622in}}{\pgfqpoint{3.696000in}{3.696000in}}%
\pgfusepath{clip}%
\pgfsetbuttcap%
\pgfsetroundjoin%
\definecolor{currentfill}{rgb}{0.121569,0.466667,0.705882}%
\pgfsetfillcolor{currentfill}%
\pgfsetfillopacity{0.849116}%
\pgfsetlinewidth{1.003750pt}%
\definecolor{currentstroke}{rgb}{0.121569,0.466667,0.705882}%
\pgfsetstrokecolor{currentstroke}%
\pgfsetstrokeopacity{0.849116}%
\pgfsetdash{}{0pt}%
\pgfpathmoveto{\pgfqpoint{1.853752in}{2.745050in}}%
\pgfpathcurveto{\pgfqpoint{1.861988in}{2.745050in}}{\pgfqpoint{1.869888in}{2.748323in}}{\pgfqpoint{1.875712in}{2.754147in}}%
\pgfpathcurveto{\pgfqpoint{1.881536in}{2.759970in}}{\pgfqpoint{1.884808in}{2.767870in}}{\pgfqpoint{1.884808in}{2.776107in}}%
\pgfpathcurveto{\pgfqpoint{1.884808in}{2.784343in}}{\pgfqpoint{1.881536in}{2.792243in}}{\pgfqpoint{1.875712in}{2.798067in}}%
\pgfpathcurveto{\pgfqpoint{1.869888in}{2.803891in}}{\pgfqpoint{1.861988in}{2.807163in}}{\pgfqpoint{1.853752in}{2.807163in}}%
\pgfpathcurveto{\pgfqpoint{1.845516in}{2.807163in}}{\pgfqpoint{1.837616in}{2.803891in}}{\pgfqpoint{1.831792in}{2.798067in}}%
\pgfpathcurveto{\pgfqpoint{1.825968in}{2.792243in}}{\pgfqpoint{1.822695in}{2.784343in}}{\pgfqpoint{1.822695in}{2.776107in}}%
\pgfpathcurveto{\pgfqpoint{1.822695in}{2.767870in}}{\pgfqpoint{1.825968in}{2.759970in}}{\pgfqpoint{1.831792in}{2.754147in}}%
\pgfpathcurveto{\pgfqpoint{1.837616in}{2.748323in}}{\pgfqpoint{1.845516in}{2.745050in}}{\pgfqpoint{1.853752in}{2.745050in}}%
\pgfpathclose%
\pgfusepath{stroke,fill}%
\end{pgfscope}%
\begin{pgfscope}%
\pgfpathrectangle{\pgfqpoint{0.100000in}{0.212622in}}{\pgfqpoint{3.696000in}{3.696000in}}%
\pgfusepath{clip}%
\pgfsetbuttcap%
\pgfsetroundjoin%
\definecolor{currentfill}{rgb}{0.121569,0.466667,0.705882}%
\pgfsetfillcolor{currentfill}%
\pgfsetfillopacity{0.849116}%
\pgfsetlinewidth{1.003750pt}%
\definecolor{currentstroke}{rgb}{0.121569,0.466667,0.705882}%
\pgfsetstrokecolor{currentstroke}%
\pgfsetstrokeopacity{0.849116}%
\pgfsetdash{}{0pt}%
\pgfpathmoveto{\pgfqpoint{1.853752in}{2.745050in}}%
\pgfpathcurveto{\pgfqpoint{1.861988in}{2.745050in}}{\pgfqpoint{1.869888in}{2.748323in}}{\pgfqpoint{1.875712in}{2.754147in}}%
\pgfpathcurveto{\pgfqpoint{1.881536in}{2.759970in}}{\pgfqpoint{1.884808in}{2.767870in}}{\pgfqpoint{1.884808in}{2.776107in}}%
\pgfpathcurveto{\pgfqpoint{1.884808in}{2.784343in}}{\pgfqpoint{1.881536in}{2.792243in}}{\pgfqpoint{1.875712in}{2.798067in}}%
\pgfpathcurveto{\pgfqpoint{1.869888in}{2.803891in}}{\pgfqpoint{1.861988in}{2.807163in}}{\pgfqpoint{1.853752in}{2.807163in}}%
\pgfpathcurveto{\pgfqpoint{1.845516in}{2.807163in}}{\pgfqpoint{1.837616in}{2.803891in}}{\pgfqpoint{1.831792in}{2.798067in}}%
\pgfpathcurveto{\pgfqpoint{1.825968in}{2.792243in}}{\pgfqpoint{1.822695in}{2.784343in}}{\pgfqpoint{1.822695in}{2.776107in}}%
\pgfpathcurveto{\pgfqpoint{1.822695in}{2.767870in}}{\pgfqpoint{1.825968in}{2.759970in}}{\pgfqpoint{1.831792in}{2.754147in}}%
\pgfpathcurveto{\pgfqpoint{1.837616in}{2.748323in}}{\pgfqpoint{1.845516in}{2.745050in}}{\pgfqpoint{1.853752in}{2.745050in}}%
\pgfpathclose%
\pgfusepath{stroke,fill}%
\end{pgfscope}%
\begin{pgfscope}%
\pgfpathrectangle{\pgfqpoint{0.100000in}{0.212622in}}{\pgfqpoint{3.696000in}{3.696000in}}%
\pgfusepath{clip}%
\pgfsetbuttcap%
\pgfsetroundjoin%
\definecolor{currentfill}{rgb}{0.121569,0.466667,0.705882}%
\pgfsetfillcolor{currentfill}%
\pgfsetfillopacity{0.849116}%
\pgfsetlinewidth{1.003750pt}%
\definecolor{currentstroke}{rgb}{0.121569,0.466667,0.705882}%
\pgfsetstrokecolor{currentstroke}%
\pgfsetstrokeopacity{0.849116}%
\pgfsetdash{}{0pt}%
\pgfpathmoveto{\pgfqpoint{1.853752in}{2.745050in}}%
\pgfpathcurveto{\pgfqpoint{1.861988in}{2.745050in}}{\pgfqpoint{1.869888in}{2.748323in}}{\pgfqpoint{1.875712in}{2.754147in}}%
\pgfpathcurveto{\pgfqpoint{1.881536in}{2.759970in}}{\pgfqpoint{1.884808in}{2.767870in}}{\pgfqpoint{1.884808in}{2.776107in}}%
\pgfpathcurveto{\pgfqpoint{1.884808in}{2.784343in}}{\pgfqpoint{1.881536in}{2.792243in}}{\pgfqpoint{1.875712in}{2.798067in}}%
\pgfpathcurveto{\pgfqpoint{1.869888in}{2.803891in}}{\pgfqpoint{1.861988in}{2.807163in}}{\pgfqpoint{1.853752in}{2.807163in}}%
\pgfpathcurveto{\pgfqpoint{1.845516in}{2.807163in}}{\pgfqpoint{1.837616in}{2.803891in}}{\pgfqpoint{1.831792in}{2.798067in}}%
\pgfpathcurveto{\pgfqpoint{1.825968in}{2.792243in}}{\pgfqpoint{1.822695in}{2.784343in}}{\pgfqpoint{1.822695in}{2.776107in}}%
\pgfpathcurveto{\pgfqpoint{1.822695in}{2.767870in}}{\pgfqpoint{1.825968in}{2.759970in}}{\pgfqpoint{1.831792in}{2.754147in}}%
\pgfpathcurveto{\pgfqpoint{1.837616in}{2.748323in}}{\pgfqpoint{1.845516in}{2.745050in}}{\pgfqpoint{1.853752in}{2.745050in}}%
\pgfpathclose%
\pgfusepath{stroke,fill}%
\end{pgfscope}%
\begin{pgfscope}%
\pgfpathrectangle{\pgfqpoint{0.100000in}{0.212622in}}{\pgfqpoint{3.696000in}{3.696000in}}%
\pgfusepath{clip}%
\pgfsetbuttcap%
\pgfsetroundjoin%
\definecolor{currentfill}{rgb}{0.121569,0.466667,0.705882}%
\pgfsetfillcolor{currentfill}%
\pgfsetfillopacity{0.849116}%
\pgfsetlinewidth{1.003750pt}%
\definecolor{currentstroke}{rgb}{0.121569,0.466667,0.705882}%
\pgfsetstrokecolor{currentstroke}%
\pgfsetstrokeopacity{0.849116}%
\pgfsetdash{}{0pt}%
\pgfpathmoveto{\pgfqpoint{1.853752in}{2.745050in}}%
\pgfpathcurveto{\pgfqpoint{1.861988in}{2.745050in}}{\pgfqpoint{1.869888in}{2.748323in}}{\pgfqpoint{1.875712in}{2.754147in}}%
\pgfpathcurveto{\pgfqpoint{1.881536in}{2.759970in}}{\pgfqpoint{1.884808in}{2.767870in}}{\pgfqpoint{1.884808in}{2.776107in}}%
\pgfpathcurveto{\pgfqpoint{1.884808in}{2.784343in}}{\pgfqpoint{1.881536in}{2.792243in}}{\pgfqpoint{1.875712in}{2.798067in}}%
\pgfpathcurveto{\pgfqpoint{1.869888in}{2.803891in}}{\pgfqpoint{1.861988in}{2.807163in}}{\pgfqpoint{1.853752in}{2.807163in}}%
\pgfpathcurveto{\pgfqpoint{1.845516in}{2.807163in}}{\pgfqpoint{1.837616in}{2.803891in}}{\pgfqpoint{1.831792in}{2.798067in}}%
\pgfpathcurveto{\pgfqpoint{1.825968in}{2.792243in}}{\pgfqpoint{1.822695in}{2.784343in}}{\pgfqpoint{1.822695in}{2.776107in}}%
\pgfpathcurveto{\pgfqpoint{1.822695in}{2.767870in}}{\pgfqpoint{1.825968in}{2.759970in}}{\pgfqpoint{1.831792in}{2.754147in}}%
\pgfpathcurveto{\pgfqpoint{1.837616in}{2.748323in}}{\pgfqpoint{1.845516in}{2.745050in}}{\pgfqpoint{1.853752in}{2.745050in}}%
\pgfpathclose%
\pgfusepath{stroke,fill}%
\end{pgfscope}%
\begin{pgfscope}%
\pgfpathrectangle{\pgfqpoint{0.100000in}{0.212622in}}{\pgfqpoint{3.696000in}{3.696000in}}%
\pgfusepath{clip}%
\pgfsetbuttcap%
\pgfsetroundjoin%
\definecolor{currentfill}{rgb}{0.121569,0.466667,0.705882}%
\pgfsetfillcolor{currentfill}%
\pgfsetfillopacity{0.849116}%
\pgfsetlinewidth{1.003750pt}%
\definecolor{currentstroke}{rgb}{0.121569,0.466667,0.705882}%
\pgfsetstrokecolor{currentstroke}%
\pgfsetstrokeopacity{0.849116}%
\pgfsetdash{}{0pt}%
\pgfpathmoveto{\pgfqpoint{1.853752in}{2.745050in}}%
\pgfpathcurveto{\pgfqpoint{1.861988in}{2.745050in}}{\pgfqpoint{1.869888in}{2.748323in}}{\pgfqpoint{1.875712in}{2.754147in}}%
\pgfpathcurveto{\pgfqpoint{1.881536in}{2.759970in}}{\pgfqpoint{1.884808in}{2.767870in}}{\pgfqpoint{1.884808in}{2.776107in}}%
\pgfpathcurveto{\pgfqpoint{1.884808in}{2.784343in}}{\pgfqpoint{1.881536in}{2.792243in}}{\pgfqpoint{1.875712in}{2.798067in}}%
\pgfpathcurveto{\pgfqpoint{1.869888in}{2.803891in}}{\pgfqpoint{1.861988in}{2.807163in}}{\pgfqpoint{1.853752in}{2.807163in}}%
\pgfpathcurveto{\pgfqpoint{1.845516in}{2.807163in}}{\pgfqpoint{1.837616in}{2.803891in}}{\pgfqpoint{1.831792in}{2.798067in}}%
\pgfpathcurveto{\pgfqpoint{1.825968in}{2.792243in}}{\pgfqpoint{1.822695in}{2.784343in}}{\pgfqpoint{1.822695in}{2.776107in}}%
\pgfpathcurveto{\pgfqpoint{1.822695in}{2.767870in}}{\pgfqpoint{1.825968in}{2.759970in}}{\pgfqpoint{1.831792in}{2.754147in}}%
\pgfpathcurveto{\pgfqpoint{1.837616in}{2.748323in}}{\pgfqpoint{1.845516in}{2.745050in}}{\pgfqpoint{1.853752in}{2.745050in}}%
\pgfpathclose%
\pgfusepath{stroke,fill}%
\end{pgfscope}%
\begin{pgfscope}%
\pgfpathrectangle{\pgfqpoint{0.100000in}{0.212622in}}{\pgfqpoint{3.696000in}{3.696000in}}%
\pgfusepath{clip}%
\pgfsetbuttcap%
\pgfsetroundjoin%
\definecolor{currentfill}{rgb}{0.121569,0.466667,0.705882}%
\pgfsetfillcolor{currentfill}%
\pgfsetfillopacity{0.849116}%
\pgfsetlinewidth{1.003750pt}%
\definecolor{currentstroke}{rgb}{0.121569,0.466667,0.705882}%
\pgfsetstrokecolor{currentstroke}%
\pgfsetstrokeopacity{0.849116}%
\pgfsetdash{}{0pt}%
\pgfpathmoveto{\pgfqpoint{1.853752in}{2.745050in}}%
\pgfpathcurveto{\pgfqpoint{1.861988in}{2.745050in}}{\pgfqpoint{1.869888in}{2.748323in}}{\pgfqpoint{1.875712in}{2.754147in}}%
\pgfpathcurveto{\pgfqpoint{1.881536in}{2.759970in}}{\pgfqpoint{1.884808in}{2.767870in}}{\pgfqpoint{1.884808in}{2.776107in}}%
\pgfpathcurveto{\pgfqpoint{1.884808in}{2.784343in}}{\pgfqpoint{1.881536in}{2.792243in}}{\pgfqpoint{1.875712in}{2.798067in}}%
\pgfpathcurveto{\pgfqpoint{1.869888in}{2.803891in}}{\pgfqpoint{1.861988in}{2.807163in}}{\pgfqpoint{1.853752in}{2.807163in}}%
\pgfpathcurveto{\pgfqpoint{1.845516in}{2.807163in}}{\pgfqpoint{1.837616in}{2.803891in}}{\pgfqpoint{1.831792in}{2.798067in}}%
\pgfpathcurveto{\pgfqpoint{1.825968in}{2.792243in}}{\pgfqpoint{1.822695in}{2.784343in}}{\pgfqpoint{1.822695in}{2.776107in}}%
\pgfpathcurveto{\pgfqpoint{1.822695in}{2.767870in}}{\pgfqpoint{1.825968in}{2.759970in}}{\pgfqpoint{1.831792in}{2.754147in}}%
\pgfpathcurveto{\pgfqpoint{1.837616in}{2.748323in}}{\pgfqpoint{1.845516in}{2.745050in}}{\pgfqpoint{1.853752in}{2.745050in}}%
\pgfpathclose%
\pgfusepath{stroke,fill}%
\end{pgfscope}%
\begin{pgfscope}%
\pgfpathrectangle{\pgfqpoint{0.100000in}{0.212622in}}{\pgfqpoint{3.696000in}{3.696000in}}%
\pgfusepath{clip}%
\pgfsetbuttcap%
\pgfsetroundjoin%
\definecolor{currentfill}{rgb}{0.121569,0.466667,0.705882}%
\pgfsetfillcolor{currentfill}%
\pgfsetfillopacity{0.849116}%
\pgfsetlinewidth{1.003750pt}%
\definecolor{currentstroke}{rgb}{0.121569,0.466667,0.705882}%
\pgfsetstrokecolor{currentstroke}%
\pgfsetstrokeopacity{0.849116}%
\pgfsetdash{}{0pt}%
\pgfpathmoveto{\pgfqpoint{1.853752in}{2.745050in}}%
\pgfpathcurveto{\pgfqpoint{1.861988in}{2.745050in}}{\pgfqpoint{1.869888in}{2.748323in}}{\pgfqpoint{1.875712in}{2.754147in}}%
\pgfpathcurveto{\pgfqpoint{1.881536in}{2.759970in}}{\pgfqpoint{1.884808in}{2.767870in}}{\pgfqpoint{1.884808in}{2.776107in}}%
\pgfpathcurveto{\pgfqpoint{1.884808in}{2.784343in}}{\pgfqpoint{1.881536in}{2.792243in}}{\pgfqpoint{1.875712in}{2.798067in}}%
\pgfpathcurveto{\pgfqpoint{1.869888in}{2.803891in}}{\pgfqpoint{1.861988in}{2.807163in}}{\pgfqpoint{1.853752in}{2.807163in}}%
\pgfpathcurveto{\pgfqpoint{1.845516in}{2.807163in}}{\pgfqpoint{1.837616in}{2.803891in}}{\pgfqpoint{1.831792in}{2.798067in}}%
\pgfpathcurveto{\pgfqpoint{1.825968in}{2.792243in}}{\pgfqpoint{1.822695in}{2.784343in}}{\pgfqpoint{1.822695in}{2.776107in}}%
\pgfpathcurveto{\pgfqpoint{1.822695in}{2.767870in}}{\pgfqpoint{1.825968in}{2.759970in}}{\pgfqpoint{1.831792in}{2.754147in}}%
\pgfpathcurveto{\pgfqpoint{1.837616in}{2.748323in}}{\pgfqpoint{1.845516in}{2.745050in}}{\pgfqpoint{1.853752in}{2.745050in}}%
\pgfpathclose%
\pgfusepath{stroke,fill}%
\end{pgfscope}%
\begin{pgfscope}%
\pgfpathrectangle{\pgfqpoint{0.100000in}{0.212622in}}{\pgfqpoint{3.696000in}{3.696000in}}%
\pgfusepath{clip}%
\pgfsetbuttcap%
\pgfsetroundjoin%
\definecolor{currentfill}{rgb}{0.121569,0.466667,0.705882}%
\pgfsetfillcolor{currentfill}%
\pgfsetfillopacity{0.849116}%
\pgfsetlinewidth{1.003750pt}%
\definecolor{currentstroke}{rgb}{0.121569,0.466667,0.705882}%
\pgfsetstrokecolor{currentstroke}%
\pgfsetstrokeopacity{0.849116}%
\pgfsetdash{}{0pt}%
\pgfpathmoveto{\pgfqpoint{1.853752in}{2.745050in}}%
\pgfpathcurveto{\pgfqpoint{1.861988in}{2.745050in}}{\pgfqpoint{1.869888in}{2.748323in}}{\pgfqpoint{1.875712in}{2.754147in}}%
\pgfpathcurveto{\pgfqpoint{1.881536in}{2.759970in}}{\pgfqpoint{1.884808in}{2.767870in}}{\pgfqpoint{1.884808in}{2.776107in}}%
\pgfpathcurveto{\pgfqpoint{1.884808in}{2.784343in}}{\pgfqpoint{1.881536in}{2.792243in}}{\pgfqpoint{1.875712in}{2.798067in}}%
\pgfpathcurveto{\pgfqpoint{1.869888in}{2.803891in}}{\pgfqpoint{1.861988in}{2.807163in}}{\pgfqpoint{1.853752in}{2.807163in}}%
\pgfpathcurveto{\pgfqpoint{1.845516in}{2.807163in}}{\pgfqpoint{1.837616in}{2.803891in}}{\pgfqpoint{1.831792in}{2.798067in}}%
\pgfpathcurveto{\pgfqpoint{1.825968in}{2.792243in}}{\pgfqpoint{1.822695in}{2.784343in}}{\pgfqpoint{1.822695in}{2.776107in}}%
\pgfpathcurveto{\pgfqpoint{1.822695in}{2.767870in}}{\pgfqpoint{1.825968in}{2.759970in}}{\pgfqpoint{1.831792in}{2.754147in}}%
\pgfpathcurveto{\pgfqpoint{1.837616in}{2.748323in}}{\pgfqpoint{1.845516in}{2.745050in}}{\pgfqpoint{1.853752in}{2.745050in}}%
\pgfpathclose%
\pgfusepath{stroke,fill}%
\end{pgfscope}%
\begin{pgfscope}%
\pgfpathrectangle{\pgfqpoint{0.100000in}{0.212622in}}{\pgfqpoint{3.696000in}{3.696000in}}%
\pgfusepath{clip}%
\pgfsetbuttcap%
\pgfsetroundjoin%
\definecolor{currentfill}{rgb}{0.121569,0.466667,0.705882}%
\pgfsetfillcolor{currentfill}%
\pgfsetfillopacity{0.849116}%
\pgfsetlinewidth{1.003750pt}%
\definecolor{currentstroke}{rgb}{0.121569,0.466667,0.705882}%
\pgfsetstrokecolor{currentstroke}%
\pgfsetstrokeopacity{0.849116}%
\pgfsetdash{}{0pt}%
\pgfpathmoveto{\pgfqpoint{1.853752in}{2.745050in}}%
\pgfpathcurveto{\pgfqpoint{1.861988in}{2.745050in}}{\pgfqpoint{1.869888in}{2.748323in}}{\pgfqpoint{1.875712in}{2.754147in}}%
\pgfpathcurveto{\pgfqpoint{1.881536in}{2.759970in}}{\pgfqpoint{1.884808in}{2.767870in}}{\pgfqpoint{1.884808in}{2.776107in}}%
\pgfpathcurveto{\pgfqpoint{1.884808in}{2.784343in}}{\pgfqpoint{1.881536in}{2.792243in}}{\pgfqpoint{1.875712in}{2.798067in}}%
\pgfpathcurveto{\pgfqpoint{1.869888in}{2.803891in}}{\pgfqpoint{1.861988in}{2.807163in}}{\pgfqpoint{1.853752in}{2.807163in}}%
\pgfpathcurveto{\pgfqpoint{1.845516in}{2.807163in}}{\pgfqpoint{1.837616in}{2.803891in}}{\pgfqpoint{1.831792in}{2.798067in}}%
\pgfpathcurveto{\pgfqpoint{1.825968in}{2.792243in}}{\pgfqpoint{1.822695in}{2.784343in}}{\pgfqpoint{1.822695in}{2.776107in}}%
\pgfpathcurveto{\pgfqpoint{1.822695in}{2.767870in}}{\pgfqpoint{1.825968in}{2.759970in}}{\pgfqpoint{1.831792in}{2.754147in}}%
\pgfpathcurveto{\pgfqpoint{1.837616in}{2.748323in}}{\pgfqpoint{1.845516in}{2.745050in}}{\pgfqpoint{1.853752in}{2.745050in}}%
\pgfpathclose%
\pgfusepath{stroke,fill}%
\end{pgfscope}%
\begin{pgfscope}%
\pgfpathrectangle{\pgfqpoint{0.100000in}{0.212622in}}{\pgfqpoint{3.696000in}{3.696000in}}%
\pgfusepath{clip}%
\pgfsetbuttcap%
\pgfsetroundjoin%
\definecolor{currentfill}{rgb}{0.121569,0.466667,0.705882}%
\pgfsetfillcolor{currentfill}%
\pgfsetfillopacity{0.849116}%
\pgfsetlinewidth{1.003750pt}%
\definecolor{currentstroke}{rgb}{0.121569,0.466667,0.705882}%
\pgfsetstrokecolor{currentstroke}%
\pgfsetstrokeopacity{0.849116}%
\pgfsetdash{}{0pt}%
\pgfpathmoveto{\pgfqpoint{1.853752in}{2.745050in}}%
\pgfpathcurveto{\pgfqpoint{1.861988in}{2.745050in}}{\pgfqpoint{1.869888in}{2.748323in}}{\pgfqpoint{1.875712in}{2.754147in}}%
\pgfpathcurveto{\pgfqpoint{1.881536in}{2.759970in}}{\pgfqpoint{1.884808in}{2.767870in}}{\pgfqpoint{1.884808in}{2.776107in}}%
\pgfpathcurveto{\pgfqpoint{1.884808in}{2.784343in}}{\pgfqpoint{1.881536in}{2.792243in}}{\pgfqpoint{1.875712in}{2.798067in}}%
\pgfpathcurveto{\pgfqpoint{1.869888in}{2.803891in}}{\pgfqpoint{1.861988in}{2.807163in}}{\pgfqpoint{1.853752in}{2.807163in}}%
\pgfpathcurveto{\pgfqpoint{1.845516in}{2.807163in}}{\pgfqpoint{1.837616in}{2.803891in}}{\pgfqpoint{1.831792in}{2.798067in}}%
\pgfpathcurveto{\pgfqpoint{1.825968in}{2.792243in}}{\pgfqpoint{1.822695in}{2.784343in}}{\pgfqpoint{1.822695in}{2.776107in}}%
\pgfpathcurveto{\pgfqpoint{1.822695in}{2.767870in}}{\pgfqpoint{1.825968in}{2.759970in}}{\pgfqpoint{1.831792in}{2.754147in}}%
\pgfpathcurveto{\pgfqpoint{1.837616in}{2.748323in}}{\pgfqpoint{1.845516in}{2.745050in}}{\pgfqpoint{1.853752in}{2.745050in}}%
\pgfpathclose%
\pgfusepath{stroke,fill}%
\end{pgfscope}%
\begin{pgfscope}%
\pgfpathrectangle{\pgfqpoint{0.100000in}{0.212622in}}{\pgfqpoint{3.696000in}{3.696000in}}%
\pgfusepath{clip}%
\pgfsetbuttcap%
\pgfsetroundjoin%
\definecolor{currentfill}{rgb}{0.121569,0.466667,0.705882}%
\pgfsetfillcolor{currentfill}%
\pgfsetfillopacity{0.849116}%
\pgfsetlinewidth{1.003750pt}%
\definecolor{currentstroke}{rgb}{0.121569,0.466667,0.705882}%
\pgfsetstrokecolor{currentstroke}%
\pgfsetstrokeopacity{0.849116}%
\pgfsetdash{}{0pt}%
\pgfpathmoveto{\pgfqpoint{1.853752in}{2.745050in}}%
\pgfpathcurveto{\pgfqpoint{1.861988in}{2.745050in}}{\pgfqpoint{1.869888in}{2.748323in}}{\pgfqpoint{1.875712in}{2.754147in}}%
\pgfpathcurveto{\pgfqpoint{1.881536in}{2.759970in}}{\pgfqpoint{1.884808in}{2.767870in}}{\pgfqpoint{1.884808in}{2.776107in}}%
\pgfpathcurveto{\pgfqpoint{1.884808in}{2.784343in}}{\pgfqpoint{1.881536in}{2.792243in}}{\pgfqpoint{1.875712in}{2.798067in}}%
\pgfpathcurveto{\pgfqpoint{1.869888in}{2.803891in}}{\pgfqpoint{1.861988in}{2.807163in}}{\pgfqpoint{1.853752in}{2.807163in}}%
\pgfpathcurveto{\pgfqpoint{1.845516in}{2.807163in}}{\pgfqpoint{1.837616in}{2.803891in}}{\pgfqpoint{1.831792in}{2.798067in}}%
\pgfpathcurveto{\pgfqpoint{1.825968in}{2.792243in}}{\pgfqpoint{1.822695in}{2.784343in}}{\pgfqpoint{1.822695in}{2.776107in}}%
\pgfpathcurveto{\pgfqpoint{1.822695in}{2.767870in}}{\pgfqpoint{1.825968in}{2.759970in}}{\pgfqpoint{1.831792in}{2.754147in}}%
\pgfpathcurveto{\pgfqpoint{1.837616in}{2.748323in}}{\pgfqpoint{1.845516in}{2.745050in}}{\pgfqpoint{1.853752in}{2.745050in}}%
\pgfpathclose%
\pgfusepath{stroke,fill}%
\end{pgfscope}%
\begin{pgfscope}%
\pgfpathrectangle{\pgfqpoint{0.100000in}{0.212622in}}{\pgfqpoint{3.696000in}{3.696000in}}%
\pgfusepath{clip}%
\pgfsetbuttcap%
\pgfsetroundjoin%
\definecolor{currentfill}{rgb}{0.121569,0.466667,0.705882}%
\pgfsetfillcolor{currentfill}%
\pgfsetfillopacity{0.849116}%
\pgfsetlinewidth{1.003750pt}%
\definecolor{currentstroke}{rgb}{0.121569,0.466667,0.705882}%
\pgfsetstrokecolor{currentstroke}%
\pgfsetstrokeopacity{0.849116}%
\pgfsetdash{}{0pt}%
\pgfpathmoveto{\pgfqpoint{1.853752in}{2.745050in}}%
\pgfpathcurveto{\pgfqpoint{1.861988in}{2.745050in}}{\pgfqpoint{1.869888in}{2.748323in}}{\pgfqpoint{1.875712in}{2.754147in}}%
\pgfpathcurveto{\pgfqpoint{1.881536in}{2.759970in}}{\pgfqpoint{1.884808in}{2.767870in}}{\pgfqpoint{1.884808in}{2.776107in}}%
\pgfpathcurveto{\pgfqpoint{1.884808in}{2.784343in}}{\pgfqpoint{1.881536in}{2.792243in}}{\pgfqpoint{1.875712in}{2.798067in}}%
\pgfpathcurveto{\pgfqpoint{1.869888in}{2.803891in}}{\pgfqpoint{1.861988in}{2.807163in}}{\pgfqpoint{1.853752in}{2.807163in}}%
\pgfpathcurveto{\pgfqpoint{1.845516in}{2.807163in}}{\pgfqpoint{1.837616in}{2.803891in}}{\pgfqpoint{1.831792in}{2.798067in}}%
\pgfpathcurveto{\pgfqpoint{1.825968in}{2.792243in}}{\pgfqpoint{1.822695in}{2.784343in}}{\pgfqpoint{1.822695in}{2.776107in}}%
\pgfpathcurveto{\pgfqpoint{1.822695in}{2.767870in}}{\pgfqpoint{1.825968in}{2.759970in}}{\pgfqpoint{1.831792in}{2.754147in}}%
\pgfpathcurveto{\pgfqpoint{1.837616in}{2.748323in}}{\pgfqpoint{1.845516in}{2.745050in}}{\pgfqpoint{1.853752in}{2.745050in}}%
\pgfpathclose%
\pgfusepath{stroke,fill}%
\end{pgfscope}%
\begin{pgfscope}%
\pgfpathrectangle{\pgfqpoint{0.100000in}{0.212622in}}{\pgfqpoint{3.696000in}{3.696000in}}%
\pgfusepath{clip}%
\pgfsetbuttcap%
\pgfsetroundjoin%
\definecolor{currentfill}{rgb}{0.121569,0.466667,0.705882}%
\pgfsetfillcolor{currentfill}%
\pgfsetfillopacity{0.849116}%
\pgfsetlinewidth{1.003750pt}%
\definecolor{currentstroke}{rgb}{0.121569,0.466667,0.705882}%
\pgfsetstrokecolor{currentstroke}%
\pgfsetstrokeopacity{0.849116}%
\pgfsetdash{}{0pt}%
\pgfpathmoveto{\pgfqpoint{1.853752in}{2.745050in}}%
\pgfpathcurveto{\pgfqpoint{1.861988in}{2.745050in}}{\pgfqpoint{1.869888in}{2.748323in}}{\pgfqpoint{1.875712in}{2.754147in}}%
\pgfpathcurveto{\pgfqpoint{1.881536in}{2.759970in}}{\pgfqpoint{1.884808in}{2.767870in}}{\pgfqpoint{1.884808in}{2.776107in}}%
\pgfpathcurveto{\pgfqpoint{1.884808in}{2.784343in}}{\pgfqpoint{1.881536in}{2.792243in}}{\pgfqpoint{1.875712in}{2.798067in}}%
\pgfpathcurveto{\pgfqpoint{1.869888in}{2.803891in}}{\pgfqpoint{1.861988in}{2.807163in}}{\pgfqpoint{1.853752in}{2.807163in}}%
\pgfpathcurveto{\pgfqpoint{1.845516in}{2.807163in}}{\pgfqpoint{1.837616in}{2.803891in}}{\pgfqpoint{1.831792in}{2.798067in}}%
\pgfpathcurveto{\pgfqpoint{1.825968in}{2.792243in}}{\pgfqpoint{1.822695in}{2.784343in}}{\pgfqpoint{1.822695in}{2.776107in}}%
\pgfpathcurveto{\pgfqpoint{1.822695in}{2.767870in}}{\pgfqpoint{1.825968in}{2.759970in}}{\pgfqpoint{1.831792in}{2.754147in}}%
\pgfpathcurveto{\pgfqpoint{1.837616in}{2.748323in}}{\pgfqpoint{1.845516in}{2.745050in}}{\pgfqpoint{1.853752in}{2.745050in}}%
\pgfpathclose%
\pgfusepath{stroke,fill}%
\end{pgfscope}%
\begin{pgfscope}%
\pgfpathrectangle{\pgfqpoint{0.100000in}{0.212622in}}{\pgfqpoint{3.696000in}{3.696000in}}%
\pgfusepath{clip}%
\pgfsetbuttcap%
\pgfsetroundjoin%
\definecolor{currentfill}{rgb}{0.121569,0.466667,0.705882}%
\pgfsetfillcolor{currentfill}%
\pgfsetfillopacity{0.849116}%
\pgfsetlinewidth{1.003750pt}%
\definecolor{currentstroke}{rgb}{0.121569,0.466667,0.705882}%
\pgfsetstrokecolor{currentstroke}%
\pgfsetstrokeopacity{0.849116}%
\pgfsetdash{}{0pt}%
\pgfpathmoveto{\pgfqpoint{1.853752in}{2.745050in}}%
\pgfpathcurveto{\pgfqpoint{1.861988in}{2.745050in}}{\pgfqpoint{1.869888in}{2.748323in}}{\pgfqpoint{1.875712in}{2.754147in}}%
\pgfpathcurveto{\pgfqpoint{1.881536in}{2.759970in}}{\pgfqpoint{1.884808in}{2.767870in}}{\pgfqpoint{1.884808in}{2.776107in}}%
\pgfpathcurveto{\pgfqpoint{1.884808in}{2.784343in}}{\pgfqpoint{1.881536in}{2.792243in}}{\pgfqpoint{1.875712in}{2.798067in}}%
\pgfpathcurveto{\pgfqpoint{1.869888in}{2.803891in}}{\pgfqpoint{1.861988in}{2.807163in}}{\pgfqpoint{1.853752in}{2.807163in}}%
\pgfpathcurveto{\pgfqpoint{1.845516in}{2.807163in}}{\pgfqpoint{1.837616in}{2.803891in}}{\pgfqpoint{1.831792in}{2.798067in}}%
\pgfpathcurveto{\pgfqpoint{1.825968in}{2.792243in}}{\pgfqpoint{1.822695in}{2.784343in}}{\pgfqpoint{1.822695in}{2.776107in}}%
\pgfpathcurveto{\pgfqpoint{1.822695in}{2.767870in}}{\pgfqpoint{1.825968in}{2.759970in}}{\pgfqpoint{1.831792in}{2.754147in}}%
\pgfpathcurveto{\pgfqpoint{1.837616in}{2.748323in}}{\pgfqpoint{1.845516in}{2.745050in}}{\pgfqpoint{1.853752in}{2.745050in}}%
\pgfpathclose%
\pgfusepath{stroke,fill}%
\end{pgfscope}%
\begin{pgfscope}%
\pgfpathrectangle{\pgfqpoint{0.100000in}{0.212622in}}{\pgfqpoint{3.696000in}{3.696000in}}%
\pgfusepath{clip}%
\pgfsetbuttcap%
\pgfsetroundjoin%
\definecolor{currentfill}{rgb}{0.121569,0.466667,0.705882}%
\pgfsetfillcolor{currentfill}%
\pgfsetfillopacity{0.849116}%
\pgfsetlinewidth{1.003750pt}%
\definecolor{currentstroke}{rgb}{0.121569,0.466667,0.705882}%
\pgfsetstrokecolor{currentstroke}%
\pgfsetstrokeopacity{0.849116}%
\pgfsetdash{}{0pt}%
\pgfpathmoveto{\pgfqpoint{1.853752in}{2.745050in}}%
\pgfpathcurveto{\pgfqpoint{1.861988in}{2.745050in}}{\pgfqpoint{1.869888in}{2.748323in}}{\pgfqpoint{1.875712in}{2.754147in}}%
\pgfpathcurveto{\pgfqpoint{1.881536in}{2.759970in}}{\pgfqpoint{1.884808in}{2.767870in}}{\pgfqpoint{1.884808in}{2.776107in}}%
\pgfpathcurveto{\pgfqpoint{1.884808in}{2.784343in}}{\pgfqpoint{1.881536in}{2.792243in}}{\pgfqpoint{1.875712in}{2.798067in}}%
\pgfpathcurveto{\pgfqpoint{1.869888in}{2.803891in}}{\pgfqpoint{1.861988in}{2.807163in}}{\pgfqpoint{1.853752in}{2.807163in}}%
\pgfpathcurveto{\pgfqpoint{1.845516in}{2.807163in}}{\pgfqpoint{1.837616in}{2.803891in}}{\pgfqpoint{1.831792in}{2.798067in}}%
\pgfpathcurveto{\pgfqpoint{1.825968in}{2.792243in}}{\pgfqpoint{1.822695in}{2.784343in}}{\pgfqpoint{1.822695in}{2.776107in}}%
\pgfpathcurveto{\pgfqpoint{1.822695in}{2.767870in}}{\pgfqpoint{1.825968in}{2.759970in}}{\pgfqpoint{1.831792in}{2.754147in}}%
\pgfpathcurveto{\pgfqpoint{1.837616in}{2.748323in}}{\pgfqpoint{1.845516in}{2.745050in}}{\pgfqpoint{1.853752in}{2.745050in}}%
\pgfpathclose%
\pgfusepath{stroke,fill}%
\end{pgfscope}%
\begin{pgfscope}%
\pgfpathrectangle{\pgfqpoint{0.100000in}{0.212622in}}{\pgfqpoint{3.696000in}{3.696000in}}%
\pgfusepath{clip}%
\pgfsetbuttcap%
\pgfsetroundjoin%
\definecolor{currentfill}{rgb}{0.121569,0.466667,0.705882}%
\pgfsetfillcolor{currentfill}%
\pgfsetfillopacity{0.849116}%
\pgfsetlinewidth{1.003750pt}%
\definecolor{currentstroke}{rgb}{0.121569,0.466667,0.705882}%
\pgfsetstrokecolor{currentstroke}%
\pgfsetstrokeopacity{0.849116}%
\pgfsetdash{}{0pt}%
\pgfpathmoveto{\pgfqpoint{1.853752in}{2.745050in}}%
\pgfpathcurveto{\pgfqpoint{1.861988in}{2.745050in}}{\pgfqpoint{1.869888in}{2.748323in}}{\pgfqpoint{1.875712in}{2.754147in}}%
\pgfpathcurveto{\pgfqpoint{1.881536in}{2.759970in}}{\pgfqpoint{1.884808in}{2.767870in}}{\pgfqpoint{1.884808in}{2.776107in}}%
\pgfpathcurveto{\pgfqpoint{1.884808in}{2.784343in}}{\pgfqpoint{1.881536in}{2.792243in}}{\pgfqpoint{1.875712in}{2.798067in}}%
\pgfpathcurveto{\pgfqpoint{1.869888in}{2.803891in}}{\pgfqpoint{1.861988in}{2.807163in}}{\pgfqpoint{1.853752in}{2.807163in}}%
\pgfpathcurveto{\pgfqpoint{1.845516in}{2.807163in}}{\pgfqpoint{1.837616in}{2.803891in}}{\pgfqpoint{1.831792in}{2.798067in}}%
\pgfpathcurveto{\pgfqpoint{1.825968in}{2.792243in}}{\pgfqpoint{1.822695in}{2.784343in}}{\pgfqpoint{1.822695in}{2.776107in}}%
\pgfpathcurveto{\pgfqpoint{1.822695in}{2.767870in}}{\pgfqpoint{1.825968in}{2.759970in}}{\pgfqpoint{1.831792in}{2.754147in}}%
\pgfpathcurveto{\pgfqpoint{1.837616in}{2.748323in}}{\pgfqpoint{1.845516in}{2.745050in}}{\pgfqpoint{1.853752in}{2.745050in}}%
\pgfpathclose%
\pgfusepath{stroke,fill}%
\end{pgfscope}%
\begin{pgfscope}%
\pgfpathrectangle{\pgfqpoint{0.100000in}{0.212622in}}{\pgfqpoint{3.696000in}{3.696000in}}%
\pgfusepath{clip}%
\pgfsetbuttcap%
\pgfsetroundjoin%
\definecolor{currentfill}{rgb}{0.121569,0.466667,0.705882}%
\pgfsetfillcolor{currentfill}%
\pgfsetfillopacity{0.849116}%
\pgfsetlinewidth{1.003750pt}%
\definecolor{currentstroke}{rgb}{0.121569,0.466667,0.705882}%
\pgfsetstrokecolor{currentstroke}%
\pgfsetstrokeopacity{0.849116}%
\pgfsetdash{}{0pt}%
\pgfpathmoveto{\pgfqpoint{1.853752in}{2.745050in}}%
\pgfpathcurveto{\pgfqpoint{1.861988in}{2.745050in}}{\pgfqpoint{1.869888in}{2.748323in}}{\pgfqpoint{1.875712in}{2.754147in}}%
\pgfpathcurveto{\pgfqpoint{1.881536in}{2.759970in}}{\pgfqpoint{1.884808in}{2.767870in}}{\pgfqpoint{1.884808in}{2.776107in}}%
\pgfpathcurveto{\pgfqpoint{1.884808in}{2.784343in}}{\pgfqpoint{1.881536in}{2.792243in}}{\pgfqpoint{1.875712in}{2.798067in}}%
\pgfpathcurveto{\pgfqpoint{1.869888in}{2.803891in}}{\pgfqpoint{1.861988in}{2.807163in}}{\pgfqpoint{1.853752in}{2.807163in}}%
\pgfpathcurveto{\pgfqpoint{1.845516in}{2.807163in}}{\pgfqpoint{1.837616in}{2.803891in}}{\pgfqpoint{1.831792in}{2.798067in}}%
\pgfpathcurveto{\pgfqpoint{1.825968in}{2.792243in}}{\pgfqpoint{1.822695in}{2.784343in}}{\pgfqpoint{1.822695in}{2.776107in}}%
\pgfpathcurveto{\pgfqpoint{1.822695in}{2.767870in}}{\pgfqpoint{1.825968in}{2.759970in}}{\pgfqpoint{1.831792in}{2.754147in}}%
\pgfpathcurveto{\pgfqpoint{1.837616in}{2.748323in}}{\pgfqpoint{1.845516in}{2.745050in}}{\pgfqpoint{1.853752in}{2.745050in}}%
\pgfpathclose%
\pgfusepath{stroke,fill}%
\end{pgfscope}%
\begin{pgfscope}%
\pgfpathrectangle{\pgfqpoint{0.100000in}{0.212622in}}{\pgfqpoint{3.696000in}{3.696000in}}%
\pgfusepath{clip}%
\pgfsetbuttcap%
\pgfsetroundjoin%
\definecolor{currentfill}{rgb}{0.121569,0.466667,0.705882}%
\pgfsetfillcolor{currentfill}%
\pgfsetfillopacity{0.849116}%
\pgfsetlinewidth{1.003750pt}%
\definecolor{currentstroke}{rgb}{0.121569,0.466667,0.705882}%
\pgfsetstrokecolor{currentstroke}%
\pgfsetstrokeopacity{0.849116}%
\pgfsetdash{}{0pt}%
\pgfpathmoveto{\pgfqpoint{1.853752in}{2.745050in}}%
\pgfpathcurveto{\pgfqpoint{1.861988in}{2.745050in}}{\pgfqpoint{1.869888in}{2.748323in}}{\pgfqpoint{1.875712in}{2.754147in}}%
\pgfpathcurveto{\pgfqpoint{1.881536in}{2.759970in}}{\pgfqpoint{1.884808in}{2.767870in}}{\pgfqpoint{1.884808in}{2.776107in}}%
\pgfpathcurveto{\pgfqpoint{1.884808in}{2.784343in}}{\pgfqpoint{1.881536in}{2.792243in}}{\pgfqpoint{1.875712in}{2.798067in}}%
\pgfpathcurveto{\pgfqpoint{1.869888in}{2.803891in}}{\pgfqpoint{1.861988in}{2.807163in}}{\pgfqpoint{1.853752in}{2.807163in}}%
\pgfpathcurveto{\pgfqpoint{1.845516in}{2.807163in}}{\pgfqpoint{1.837616in}{2.803891in}}{\pgfqpoint{1.831792in}{2.798067in}}%
\pgfpathcurveto{\pgfqpoint{1.825968in}{2.792243in}}{\pgfqpoint{1.822695in}{2.784343in}}{\pgfqpoint{1.822695in}{2.776107in}}%
\pgfpathcurveto{\pgfqpoint{1.822695in}{2.767870in}}{\pgfqpoint{1.825968in}{2.759970in}}{\pgfqpoint{1.831792in}{2.754147in}}%
\pgfpathcurveto{\pgfqpoint{1.837616in}{2.748323in}}{\pgfqpoint{1.845516in}{2.745050in}}{\pgfqpoint{1.853752in}{2.745050in}}%
\pgfpathclose%
\pgfusepath{stroke,fill}%
\end{pgfscope}%
\begin{pgfscope}%
\pgfpathrectangle{\pgfqpoint{0.100000in}{0.212622in}}{\pgfqpoint{3.696000in}{3.696000in}}%
\pgfusepath{clip}%
\pgfsetbuttcap%
\pgfsetroundjoin%
\definecolor{currentfill}{rgb}{0.121569,0.466667,0.705882}%
\pgfsetfillcolor{currentfill}%
\pgfsetfillopacity{0.849116}%
\pgfsetlinewidth{1.003750pt}%
\definecolor{currentstroke}{rgb}{0.121569,0.466667,0.705882}%
\pgfsetstrokecolor{currentstroke}%
\pgfsetstrokeopacity{0.849116}%
\pgfsetdash{}{0pt}%
\pgfpathmoveto{\pgfqpoint{1.853752in}{2.745050in}}%
\pgfpathcurveto{\pgfqpoint{1.861988in}{2.745050in}}{\pgfqpoint{1.869888in}{2.748323in}}{\pgfqpoint{1.875712in}{2.754147in}}%
\pgfpathcurveto{\pgfqpoint{1.881536in}{2.759970in}}{\pgfqpoint{1.884808in}{2.767870in}}{\pgfqpoint{1.884808in}{2.776107in}}%
\pgfpathcurveto{\pgfqpoint{1.884808in}{2.784343in}}{\pgfqpoint{1.881536in}{2.792243in}}{\pgfqpoint{1.875712in}{2.798067in}}%
\pgfpathcurveto{\pgfqpoint{1.869888in}{2.803891in}}{\pgfqpoint{1.861988in}{2.807163in}}{\pgfqpoint{1.853752in}{2.807163in}}%
\pgfpathcurveto{\pgfqpoint{1.845516in}{2.807163in}}{\pgfqpoint{1.837616in}{2.803891in}}{\pgfqpoint{1.831792in}{2.798067in}}%
\pgfpathcurveto{\pgfqpoint{1.825968in}{2.792243in}}{\pgfqpoint{1.822695in}{2.784343in}}{\pgfqpoint{1.822695in}{2.776107in}}%
\pgfpathcurveto{\pgfqpoint{1.822695in}{2.767870in}}{\pgfqpoint{1.825968in}{2.759970in}}{\pgfqpoint{1.831792in}{2.754147in}}%
\pgfpathcurveto{\pgfqpoint{1.837616in}{2.748323in}}{\pgfqpoint{1.845516in}{2.745050in}}{\pgfqpoint{1.853752in}{2.745050in}}%
\pgfpathclose%
\pgfusepath{stroke,fill}%
\end{pgfscope}%
\begin{pgfscope}%
\pgfpathrectangle{\pgfqpoint{0.100000in}{0.212622in}}{\pgfqpoint{3.696000in}{3.696000in}}%
\pgfusepath{clip}%
\pgfsetbuttcap%
\pgfsetroundjoin%
\definecolor{currentfill}{rgb}{0.121569,0.466667,0.705882}%
\pgfsetfillcolor{currentfill}%
\pgfsetfillopacity{0.849116}%
\pgfsetlinewidth{1.003750pt}%
\definecolor{currentstroke}{rgb}{0.121569,0.466667,0.705882}%
\pgfsetstrokecolor{currentstroke}%
\pgfsetstrokeopacity{0.849116}%
\pgfsetdash{}{0pt}%
\pgfpathmoveto{\pgfqpoint{1.853752in}{2.745050in}}%
\pgfpathcurveto{\pgfqpoint{1.861988in}{2.745050in}}{\pgfqpoint{1.869888in}{2.748323in}}{\pgfqpoint{1.875712in}{2.754147in}}%
\pgfpathcurveto{\pgfqpoint{1.881536in}{2.759970in}}{\pgfqpoint{1.884808in}{2.767870in}}{\pgfqpoint{1.884808in}{2.776107in}}%
\pgfpathcurveto{\pgfqpoint{1.884808in}{2.784343in}}{\pgfqpoint{1.881536in}{2.792243in}}{\pgfqpoint{1.875712in}{2.798067in}}%
\pgfpathcurveto{\pgfqpoint{1.869888in}{2.803891in}}{\pgfqpoint{1.861988in}{2.807163in}}{\pgfqpoint{1.853752in}{2.807163in}}%
\pgfpathcurveto{\pgfqpoint{1.845516in}{2.807163in}}{\pgfqpoint{1.837616in}{2.803891in}}{\pgfqpoint{1.831792in}{2.798067in}}%
\pgfpathcurveto{\pgfqpoint{1.825968in}{2.792243in}}{\pgfqpoint{1.822695in}{2.784343in}}{\pgfqpoint{1.822695in}{2.776107in}}%
\pgfpathcurveto{\pgfqpoint{1.822695in}{2.767870in}}{\pgfqpoint{1.825968in}{2.759970in}}{\pgfqpoint{1.831792in}{2.754147in}}%
\pgfpathcurveto{\pgfqpoint{1.837616in}{2.748323in}}{\pgfqpoint{1.845516in}{2.745050in}}{\pgfqpoint{1.853752in}{2.745050in}}%
\pgfpathclose%
\pgfusepath{stroke,fill}%
\end{pgfscope}%
\begin{pgfscope}%
\pgfpathrectangle{\pgfqpoint{0.100000in}{0.212622in}}{\pgfqpoint{3.696000in}{3.696000in}}%
\pgfusepath{clip}%
\pgfsetbuttcap%
\pgfsetroundjoin%
\definecolor{currentfill}{rgb}{0.121569,0.466667,0.705882}%
\pgfsetfillcolor{currentfill}%
\pgfsetfillopacity{0.849223}%
\pgfsetlinewidth{1.003750pt}%
\definecolor{currentstroke}{rgb}{0.121569,0.466667,0.705882}%
\pgfsetstrokecolor{currentstroke}%
\pgfsetstrokeopacity{0.849223}%
\pgfsetdash{}{0pt}%
\pgfpathmoveto{\pgfqpoint{1.853567in}{2.744853in}}%
\pgfpathcurveto{\pgfqpoint{1.861803in}{2.744853in}}{\pgfqpoint{1.869704in}{2.748125in}}{\pgfqpoint{1.875527in}{2.753949in}}%
\pgfpathcurveto{\pgfqpoint{1.881351in}{2.759773in}}{\pgfqpoint{1.884624in}{2.767673in}}{\pgfqpoint{1.884624in}{2.775910in}}%
\pgfpathcurveto{\pgfqpoint{1.884624in}{2.784146in}}{\pgfqpoint{1.881351in}{2.792046in}}{\pgfqpoint{1.875527in}{2.797870in}}%
\pgfpathcurveto{\pgfqpoint{1.869704in}{2.803694in}}{\pgfqpoint{1.861803in}{2.806966in}}{\pgfqpoint{1.853567in}{2.806966in}}%
\pgfpathcurveto{\pgfqpoint{1.845331in}{2.806966in}}{\pgfqpoint{1.837431in}{2.803694in}}{\pgfqpoint{1.831607in}{2.797870in}}%
\pgfpathcurveto{\pgfqpoint{1.825783in}{2.792046in}}{\pgfqpoint{1.822511in}{2.784146in}}{\pgfqpoint{1.822511in}{2.775910in}}%
\pgfpathcurveto{\pgfqpoint{1.822511in}{2.767673in}}{\pgfqpoint{1.825783in}{2.759773in}}{\pgfqpoint{1.831607in}{2.753949in}}%
\pgfpathcurveto{\pgfqpoint{1.837431in}{2.748125in}}{\pgfqpoint{1.845331in}{2.744853in}}{\pgfqpoint{1.853567in}{2.744853in}}%
\pgfpathclose%
\pgfusepath{stroke,fill}%
\end{pgfscope}%
\begin{pgfscope}%
\pgfpathrectangle{\pgfqpoint{0.100000in}{0.212622in}}{\pgfqpoint{3.696000in}{3.696000in}}%
\pgfusepath{clip}%
\pgfsetbuttcap%
\pgfsetroundjoin%
\definecolor{currentfill}{rgb}{0.121569,0.466667,0.705882}%
\pgfsetfillcolor{currentfill}%
\pgfsetfillopacity{0.849282}%
\pgfsetlinewidth{1.003750pt}%
\definecolor{currentstroke}{rgb}{0.121569,0.466667,0.705882}%
\pgfsetstrokecolor{currentstroke}%
\pgfsetstrokeopacity{0.849282}%
\pgfsetdash{}{0pt}%
\pgfpathmoveto{\pgfqpoint{1.853466in}{2.744740in}}%
\pgfpathcurveto{\pgfqpoint{1.861702in}{2.744740in}}{\pgfqpoint{1.869602in}{2.748012in}}{\pgfqpoint{1.875426in}{2.753836in}}%
\pgfpathcurveto{\pgfqpoint{1.881250in}{2.759660in}}{\pgfqpoint{1.884522in}{2.767560in}}{\pgfqpoint{1.884522in}{2.775797in}}%
\pgfpathcurveto{\pgfqpoint{1.884522in}{2.784033in}}{\pgfqpoint{1.881250in}{2.791933in}}{\pgfqpoint{1.875426in}{2.797757in}}%
\pgfpathcurveto{\pgfqpoint{1.869602in}{2.803581in}}{\pgfqpoint{1.861702in}{2.806853in}}{\pgfqpoint{1.853466in}{2.806853in}}%
\pgfpathcurveto{\pgfqpoint{1.845230in}{2.806853in}}{\pgfqpoint{1.837330in}{2.803581in}}{\pgfqpoint{1.831506in}{2.797757in}}%
\pgfpathcurveto{\pgfqpoint{1.825682in}{2.791933in}}{\pgfqpoint{1.822409in}{2.784033in}}{\pgfqpoint{1.822409in}{2.775797in}}%
\pgfpathcurveto{\pgfqpoint{1.822409in}{2.767560in}}{\pgfqpoint{1.825682in}{2.759660in}}{\pgfqpoint{1.831506in}{2.753836in}}%
\pgfpathcurveto{\pgfqpoint{1.837330in}{2.748012in}}{\pgfqpoint{1.845230in}{2.744740in}}{\pgfqpoint{1.853466in}{2.744740in}}%
\pgfpathclose%
\pgfusepath{stroke,fill}%
\end{pgfscope}%
\begin{pgfscope}%
\pgfpathrectangle{\pgfqpoint{0.100000in}{0.212622in}}{\pgfqpoint{3.696000in}{3.696000in}}%
\pgfusepath{clip}%
\pgfsetbuttcap%
\pgfsetroundjoin%
\definecolor{currentfill}{rgb}{0.121569,0.466667,0.705882}%
\pgfsetfillcolor{currentfill}%
\pgfsetfillopacity{0.849316}%
\pgfsetlinewidth{1.003750pt}%
\definecolor{currentstroke}{rgb}{0.121569,0.466667,0.705882}%
\pgfsetstrokecolor{currentstroke}%
\pgfsetstrokeopacity{0.849316}%
\pgfsetdash{}{0pt}%
\pgfpathmoveto{\pgfqpoint{1.853410in}{2.744687in}}%
\pgfpathcurveto{\pgfqpoint{1.861646in}{2.744687in}}{\pgfqpoint{1.869546in}{2.747959in}}{\pgfqpoint{1.875370in}{2.753783in}}%
\pgfpathcurveto{\pgfqpoint{1.881194in}{2.759607in}}{\pgfqpoint{1.884466in}{2.767507in}}{\pgfqpoint{1.884466in}{2.775743in}}%
\pgfpathcurveto{\pgfqpoint{1.884466in}{2.783980in}}{\pgfqpoint{1.881194in}{2.791880in}}{\pgfqpoint{1.875370in}{2.797704in}}%
\pgfpathcurveto{\pgfqpoint{1.869546in}{2.803528in}}{\pgfqpoint{1.861646in}{2.806800in}}{\pgfqpoint{1.853410in}{2.806800in}}%
\pgfpathcurveto{\pgfqpoint{1.845174in}{2.806800in}}{\pgfqpoint{1.837274in}{2.803528in}}{\pgfqpoint{1.831450in}{2.797704in}}%
\pgfpathcurveto{\pgfqpoint{1.825626in}{2.791880in}}{\pgfqpoint{1.822353in}{2.783980in}}{\pgfqpoint{1.822353in}{2.775743in}}%
\pgfpathcurveto{\pgfqpoint{1.822353in}{2.767507in}}{\pgfqpoint{1.825626in}{2.759607in}}{\pgfqpoint{1.831450in}{2.753783in}}%
\pgfpathcurveto{\pgfqpoint{1.837274in}{2.747959in}}{\pgfqpoint{1.845174in}{2.744687in}}{\pgfqpoint{1.853410in}{2.744687in}}%
\pgfpathclose%
\pgfusepath{stroke,fill}%
\end{pgfscope}%
\begin{pgfscope}%
\pgfpathrectangle{\pgfqpoint{0.100000in}{0.212622in}}{\pgfqpoint{3.696000in}{3.696000in}}%
\pgfusepath{clip}%
\pgfsetbuttcap%
\pgfsetroundjoin%
\definecolor{currentfill}{rgb}{0.121569,0.466667,0.705882}%
\pgfsetfillcolor{currentfill}%
\pgfsetfillopacity{0.849334}%
\pgfsetlinewidth{1.003750pt}%
\definecolor{currentstroke}{rgb}{0.121569,0.466667,0.705882}%
\pgfsetstrokecolor{currentstroke}%
\pgfsetstrokeopacity{0.849334}%
\pgfsetdash{}{0pt}%
\pgfpathmoveto{\pgfqpoint{1.853382in}{2.744655in}}%
\pgfpathcurveto{\pgfqpoint{1.861618in}{2.744655in}}{\pgfqpoint{1.869519in}{2.747927in}}{\pgfqpoint{1.875342in}{2.753751in}}%
\pgfpathcurveto{\pgfqpoint{1.881166in}{2.759575in}}{\pgfqpoint{1.884439in}{2.767475in}}{\pgfqpoint{1.884439in}{2.775711in}}%
\pgfpathcurveto{\pgfqpoint{1.884439in}{2.783948in}}{\pgfqpoint{1.881166in}{2.791848in}}{\pgfqpoint{1.875342in}{2.797671in}}%
\pgfpathcurveto{\pgfqpoint{1.869519in}{2.803495in}}{\pgfqpoint{1.861618in}{2.806768in}}{\pgfqpoint{1.853382in}{2.806768in}}%
\pgfpathcurveto{\pgfqpoint{1.845146in}{2.806768in}}{\pgfqpoint{1.837246in}{2.803495in}}{\pgfqpoint{1.831422in}{2.797671in}}%
\pgfpathcurveto{\pgfqpoint{1.825598in}{2.791848in}}{\pgfqpoint{1.822326in}{2.783948in}}{\pgfqpoint{1.822326in}{2.775711in}}%
\pgfpathcurveto{\pgfqpoint{1.822326in}{2.767475in}}{\pgfqpoint{1.825598in}{2.759575in}}{\pgfqpoint{1.831422in}{2.753751in}}%
\pgfpathcurveto{\pgfqpoint{1.837246in}{2.747927in}}{\pgfqpoint{1.845146in}{2.744655in}}{\pgfqpoint{1.853382in}{2.744655in}}%
\pgfpathclose%
\pgfusepath{stroke,fill}%
\end{pgfscope}%
\begin{pgfscope}%
\pgfpathrectangle{\pgfqpoint{0.100000in}{0.212622in}}{\pgfqpoint{3.696000in}{3.696000in}}%
\pgfusepath{clip}%
\pgfsetbuttcap%
\pgfsetroundjoin%
\definecolor{currentfill}{rgb}{0.121569,0.466667,0.705882}%
\pgfsetfillcolor{currentfill}%
\pgfsetfillopacity{0.849344}%
\pgfsetlinewidth{1.003750pt}%
\definecolor{currentstroke}{rgb}{0.121569,0.466667,0.705882}%
\pgfsetstrokecolor{currentstroke}%
\pgfsetstrokeopacity{0.849344}%
\pgfsetdash{}{0pt}%
\pgfpathmoveto{\pgfqpoint{1.853366in}{2.744636in}}%
\pgfpathcurveto{\pgfqpoint{1.861602in}{2.744636in}}{\pgfqpoint{1.869502in}{2.747908in}}{\pgfqpoint{1.875326in}{2.753732in}}%
\pgfpathcurveto{\pgfqpoint{1.881150in}{2.759556in}}{\pgfqpoint{1.884422in}{2.767456in}}{\pgfqpoint{1.884422in}{2.775693in}}%
\pgfpathcurveto{\pgfqpoint{1.884422in}{2.783929in}}{\pgfqpoint{1.881150in}{2.791829in}}{\pgfqpoint{1.875326in}{2.797653in}}%
\pgfpathcurveto{\pgfqpoint{1.869502in}{2.803477in}}{\pgfqpoint{1.861602in}{2.806749in}}{\pgfqpoint{1.853366in}{2.806749in}}%
\pgfpathcurveto{\pgfqpoint{1.845129in}{2.806749in}}{\pgfqpoint{1.837229in}{2.803477in}}{\pgfqpoint{1.831405in}{2.797653in}}%
\pgfpathcurveto{\pgfqpoint{1.825582in}{2.791829in}}{\pgfqpoint{1.822309in}{2.783929in}}{\pgfqpoint{1.822309in}{2.775693in}}%
\pgfpathcurveto{\pgfqpoint{1.822309in}{2.767456in}}{\pgfqpoint{1.825582in}{2.759556in}}{\pgfqpoint{1.831405in}{2.753732in}}%
\pgfpathcurveto{\pgfqpoint{1.837229in}{2.747908in}}{\pgfqpoint{1.845129in}{2.744636in}}{\pgfqpoint{1.853366in}{2.744636in}}%
\pgfpathclose%
\pgfusepath{stroke,fill}%
\end{pgfscope}%
\begin{pgfscope}%
\pgfpathrectangle{\pgfqpoint{0.100000in}{0.212622in}}{\pgfqpoint{3.696000in}{3.696000in}}%
\pgfusepath{clip}%
\pgfsetbuttcap%
\pgfsetroundjoin%
\definecolor{currentfill}{rgb}{0.121569,0.466667,0.705882}%
\pgfsetfillcolor{currentfill}%
\pgfsetfillopacity{0.849349}%
\pgfsetlinewidth{1.003750pt}%
\definecolor{currentstroke}{rgb}{0.121569,0.466667,0.705882}%
\pgfsetstrokecolor{currentstroke}%
\pgfsetstrokeopacity{0.849349}%
\pgfsetdash{}{0pt}%
\pgfpathmoveto{\pgfqpoint{1.853357in}{2.744626in}}%
\pgfpathcurveto{\pgfqpoint{1.861594in}{2.744626in}}{\pgfqpoint{1.869494in}{2.747899in}}{\pgfqpoint{1.875318in}{2.753723in}}%
\pgfpathcurveto{\pgfqpoint{1.881141in}{2.759547in}}{\pgfqpoint{1.884414in}{2.767447in}}{\pgfqpoint{1.884414in}{2.775683in}}%
\pgfpathcurveto{\pgfqpoint{1.884414in}{2.783919in}}{\pgfqpoint{1.881141in}{2.791819in}}{\pgfqpoint{1.875318in}{2.797643in}}%
\pgfpathcurveto{\pgfqpoint{1.869494in}{2.803467in}}{\pgfqpoint{1.861594in}{2.806739in}}{\pgfqpoint{1.853357in}{2.806739in}}%
\pgfpathcurveto{\pgfqpoint{1.845121in}{2.806739in}}{\pgfqpoint{1.837221in}{2.803467in}}{\pgfqpoint{1.831397in}{2.797643in}}%
\pgfpathcurveto{\pgfqpoint{1.825573in}{2.791819in}}{\pgfqpoint{1.822301in}{2.783919in}}{\pgfqpoint{1.822301in}{2.775683in}}%
\pgfpathcurveto{\pgfqpoint{1.822301in}{2.767447in}}{\pgfqpoint{1.825573in}{2.759547in}}{\pgfqpoint{1.831397in}{2.753723in}}%
\pgfpathcurveto{\pgfqpoint{1.837221in}{2.747899in}}{\pgfqpoint{1.845121in}{2.744626in}}{\pgfqpoint{1.853357in}{2.744626in}}%
\pgfpathclose%
\pgfusepath{stroke,fill}%
\end{pgfscope}%
\begin{pgfscope}%
\pgfpathrectangle{\pgfqpoint{0.100000in}{0.212622in}}{\pgfqpoint{3.696000in}{3.696000in}}%
\pgfusepath{clip}%
\pgfsetbuttcap%
\pgfsetroundjoin%
\definecolor{currentfill}{rgb}{0.121569,0.466667,0.705882}%
\pgfsetfillcolor{currentfill}%
\pgfsetfillopacity{0.849352}%
\pgfsetlinewidth{1.003750pt}%
\definecolor{currentstroke}{rgb}{0.121569,0.466667,0.705882}%
\pgfsetstrokecolor{currentstroke}%
\pgfsetstrokeopacity{0.849352}%
\pgfsetdash{}{0pt}%
\pgfpathmoveto{\pgfqpoint{1.853353in}{2.744620in}}%
\pgfpathcurveto{\pgfqpoint{1.861589in}{2.744620in}}{\pgfqpoint{1.869489in}{2.747892in}}{\pgfqpoint{1.875313in}{2.753716in}}%
\pgfpathcurveto{\pgfqpoint{1.881137in}{2.759540in}}{\pgfqpoint{1.884409in}{2.767440in}}{\pgfqpoint{1.884409in}{2.775676in}}%
\pgfpathcurveto{\pgfqpoint{1.884409in}{2.783913in}}{\pgfqpoint{1.881137in}{2.791813in}}{\pgfqpoint{1.875313in}{2.797637in}}%
\pgfpathcurveto{\pgfqpoint{1.869489in}{2.803461in}}{\pgfqpoint{1.861589in}{2.806733in}}{\pgfqpoint{1.853353in}{2.806733in}}%
\pgfpathcurveto{\pgfqpoint{1.845116in}{2.806733in}}{\pgfqpoint{1.837216in}{2.803461in}}{\pgfqpoint{1.831392in}{2.797637in}}%
\pgfpathcurveto{\pgfqpoint{1.825568in}{2.791813in}}{\pgfqpoint{1.822296in}{2.783913in}}{\pgfqpoint{1.822296in}{2.775676in}}%
\pgfpathcurveto{\pgfqpoint{1.822296in}{2.767440in}}{\pgfqpoint{1.825568in}{2.759540in}}{\pgfqpoint{1.831392in}{2.753716in}}%
\pgfpathcurveto{\pgfqpoint{1.837216in}{2.747892in}}{\pgfqpoint{1.845116in}{2.744620in}}{\pgfqpoint{1.853353in}{2.744620in}}%
\pgfpathclose%
\pgfusepath{stroke,fill}%
\end{pgfscope}%
\begin{pgfscope}%
\pgfpathrectangle{\pgfqpoint{0.100000in}{0.212622in}}{\pgfqpoint{3.696000in}{3.696000in}}%
\pgfusepath{clip}%
\pgfsetbuttcap%
\pgfsetroundjoin%
\definecolor{currentfill}{rgb}{0.121569,0.466667,0.705882}%
\pgfsetfillcolor{currentfill}%
\pgfsetfillopacity{0.849354}%
\pgfsetlinewidth{1.003750pt}%
\definecolor{currentstroke}{rgb}{0.121569,0.466667,0.705882}%
\pgfsetstrokecolor{currentstroke}%
\pgfsetstrokeopacity{0.849354}%
\pgfsetdash{}{0pt}%
\pgfpathmoveto{\pgfqpoint{1.853350in}{2.744617in}}%
\pgfpathcurveto{\pgfqpoint{1.861586in}{2.744617in}}{\pgfqpoint{1.869486in}{2.747889in}}{\pgfqpoint{1.875310in}{2.753713in}}%
\pgfpathcurveto{\pgfqpoint{1.881134in}{2.759537in}}{\pgfqpoint{1.884406in}{2.767437in}}{\pgfqpoint{1.884406in}{2.775673in}}%
\pgfpathcurveto{\pgfqpoint{1.884406in}{2.783910in}}{\pgfqpoint{1.881134in}{2.791810in}}{\pgfqpoint{1.875310in}{2.797633in}}%
\pgfpathcurveto{\pgfqpoint{1.869486in}{2.803457in}}{\pgfqpoint{1.861586in}{2.806730in}}{\pgfqpoint{1.853350in}{2.806730in}}%
\pgfpathcurveto{\pgfqpoint{1.845114in}{2.806730in}}{\pgfqpoint{1.837214in}{2.803457in}}{\pgfqpoint{1.831390in}{2.797633in}}%
\pgfpathcurveto{\pgfqpoint{1.825566in}{2.791810in}}{\pgfqpoint{1.822293in}{2.783910in}}{\pgfqpoint{1.822293in}{2.775673in}}%
\pgfpathcurveto{\pgfqpoint{1.822293in}{2.767437in}}{\pgfqpoint{1.825566in}{2.759537in}}{\pgfqpoint{1.831390in}{2.753713in}}%
\pgfpathcurveto{\pgfqpoint{1.837214in}{2.747889in}}{\pgfqpoint{1.845114in}{2.744617in}}{\pgfqpoint{1.853350in}{2.744617in}}%
\pgfpathclose%
\pgfusepath{stroke,fill}%
\end{pgfscope}%
\begin{pgfscope}%
\pgfpathrectangle{\pgfqpoint{0.100000in}{0.212622in}}{\pgfqpoint{3.696000in}{3.696000in}}%
\pgfusepath{clip}%
\pgfsetbuttcap%
\pgfsetroundjoin%
\definecolor{currentfill}{rgb}{0.121569,0.466667,0.705882}%
\pgfsetfillcolor{currentfill}%
\pgfsetfillopacity{0.849355}%
\pgfsetlinewidth{1.003750pt}%
\definecolor{currentstroke}{rgb}{0.121569,0.466667,0.705882}%
\pgfsetstrokecolor{currentstroke}%
\pgfsetstrokeopacity{0.849355}%
\pgfsetdash{}{0pt}%
\pgfpathmoveto{\pgfqpoint{1.853349in}{2.744615in}}%
\pgfpathcurveto{\pgfqpoint{1.861585in}{2.744615in}}{\pgfqpoint{1.869485in}{2.747887in}}{\pgfqpoint{1.875309in}{2.753711in}}%
\pgfpathcurveto{\pgfqpoint{1.881133in}{2.759535in}}{\pgfqpoint{1.884405in}{2.767435in}}{\pgfqpoint{1.884405in}{2.775671in}}%
\pgfpathcurveto{\pgfqpoint{1.884405in}{2.783908in}}{\pgfqpoint{1.881133in}{2.791808in}}{\pgfqpoint{1.875309in}{2.797632in}}%
\pgfpathcurveto{\pgfqpoint{1.869485in}{2.803456in}}{\pgfqpoint{1.861585in}{2.806728in}}{\pgfqpoint{1.853349in}{2.806728in}}%
\pgfpathcurveto{\pgfqpoint{1.845112in}{2.806728in}}{\pgfqpoint{1.837212in}{2.803456in}}{\pgfqpoint{1.831388in}{2.797632in}}%
\pgfpathcurveto{\pgfqpoint{1.825564in}{2.791808in}}{\pgfqpoint{1.822292in}{2.783908in}}{\pgfqpoint{1.822292in}{2.775671in}}%
\pgfpathcurveto{\pgfqpoint{1.822292in}{2.767435in}}{\pgfqpoint{1.825564in}{2.759535in}}{\pgfqpoint{1.831388in}{2.753711in}}%
\pgfpathcurveto{\pgfqpoint{1.837212in}{2.747887in}}{\pgfqpoint{1.845112in}{2.744615in}}{\pgfqpoint{1.853349in}{2.744615in}}%
\pgfpathclose%
\pgfusepath{stroke,fill}%
\end{pgfscope}%
\begin{pgfscope}%
\pgfpathrectangle{\pgfqpoint{0.100000in}{0.212622in}}{\pgfqpoint{3.696000in}{3.696000in}}%
\pgfusepath{clip}%
\pgfsetbuttcap%
\pgfsetroundjoin%
\definecolor{currentfill}{rgb}{0.121569,0.466667,0.705882}%
\pgfsetfillcolor{currentfill}%
\pgfsetfillopacity{0.849355}%
\pgfsetlinewidth{1.003750pt}%
\definecolor{currentstroke}{rgb}{0.121569,0.466667,0.705882}%
\pgfsetstrokecolor{currentstroke}%
\pgfsetstrokeopacity{0.849355}%
\pgfsetdash{}{0pt}%
\pgfpathmoveto{\pgfqpoint{1.853348in}{2.744614in}}%
\pgfpathcurveto{\pgfqpoint{1.861584in}{2.744614in}}{\pgfqpoint{1.869484in}{2.747886in}}{\pgfqpoint{1.875308in}{2.753710in}}%
\pgfpathcurveto{\pgfqpoint{1.881132in}{2.759534in}}{\pgfqpoint{1.884404in}{2.767434in}}{\pgfqpoint{1.884404in}{2.775670in}}%
\pgfpathcurveto{\pgfqpoint{1.884404in}{2.783907in}}{\pgfqpoint{1.881132in}{2.791807in}}{\pgfqpoint{1.875308in}{2.797631in}}%
\pgfpathcurveto{\pgfqpoint{1.869484in}{2.803455in}}{\pgfqpoint{1.861584in}{2.806727in}}{\pgfqpoint{1.853348in}{2.806727in}}%
\pgfpathcurveto{\pgfqpoint{1.845112in}{2.806727in}}{\pgfqpoint{1.837212in}{2.803455in}}{\pgfqpoint{1.831388in}{2.797631in}}%
\pgfpathcurveto{\pgfqpoint{1.825564in}{2.791807in}}{\pgfqpoint{1.822291in}{2.783907in}}{\pgfqpoint{1.822291in}{2.775670in}}%
\pgfpathcurveto{\pgfqpoint{1.822291in}{2.767434in}}{\pgfqpoint{1.825564in}{2.759534in}}{\pgfqpoint{1.831388in}{2.753710in}}%
\pgfpathcurveto{\pgfqpoint{1.837212in}{2.747886in}}{\pgfqpoint{1.845112in}{2.744614in}}{\pgfqpoint{1.853348in}{2.744614in}}%
\pgfpathclose%
\pgfusepath{stroke,fill}%
\end{pgfscope}%
\begin{pgfscope}%
\pgfpathrectangle{\pgfqpoint{0.100000in}{0.212622in}}{\pgfqpoint{3.696000in}{3.696000in}}%
\pgfusepath{clip}%
\pgfsetbuttcap%
\pgfsetroundjoin%
\definecolor{currentfill}{rgb}{0.121569,0.466667,0.705882}%
\pgfsetfillcolor{currentfill}%
\pgfsetfillopacity{0.849355}%
\pgfsetlinewidth{1.003750pt}%
\definecolor{currentstroke}{rgb}{0.121569,0.466667,0.705882}%
\pgfsetstrokecolor{currentstroke}%
\pgfsetstrokeopacity{0.849355}%
\pgfsetdash{}{0pt}%
\pgfpathmoveto{\pgfqpoint{1.853347in}{2.744613in}}%
\pgfpathcurveto{\pgfqpoint{1.861584in}{2.744613in}}{\pgfqpoint{1.869484in}{2.747886in}}{\pgfqpoint{1.875308in}{2.753710in}}%
\pgfpathcurveto{\pgfqpoint{1.881132in}{2.759534in}}{\pgfqpoint{1.884404in}{2.767434in}}{\pgfqpoint{1.884404in}{2.775670in}}%
\pgfpathcurveto{\pgfqpoint{1.884404in}{2.783906in}}{\pgfqpoint{1.881132in}{2.791806in}}{\pgfqpoint{1.875308in}{2.797630in}}%
\pgfpathcurveto{\pgfqpoint{1.869484in}{2.803454in}}{\pgfqpoint{1.861584in}{2.806726in}}{\pgfqpoint{1.853347in}{2.806726in}}%
\pgfpathcurveto{\pgfqpoint{1.845111in}{2.806726in}}{\pgfqpoint{1.837211in}{2.803454in}}{\pgfqpoint{1.831387in}{2.797630in}}%
\pgfpathcurveto{\pgfqpoint{1.825563in}{2.791806in}}{\pgfqpoint{1.822291in}{2.783906in}}{\pgfqpoint{1.822291in}{2.775670in}}%
\pgfpathcurveto{\pgfqpoint{1.822291in}{2.767434in}}{\pgfqpoint{1.825563in}{2.759534in}}{\pgfqpoint{1.831387in}{2.753710in}}%
\pgfpathcurveto{\pgfqpoint{1.837211in}{2.747886in}}{\pgfqpoint{1.845111in}{2.744613in}}{\pgfqpoint{1.853347in}{2.744613in}}%
\pgfpathclose%
\pgfusepath{stroke,fill}%
\end{pgfscope}%
\begin{pgfscope}%
\pgfpathrectangle{\pgfqpoint{0.100000in}{0.212622in}}{\pgfqpoint{3.696000in}{3.696000in}}%
\pgfusepath{clip}%
\pgfsetbuttcap%
\pgfsetroundjoin%
\definecolor{currentfill}{rgb}{0.121569,0.466667,0.705882}%
\pgfsetfillcolor{currentfill}%
\pgfsetfillopacity{0.849447}%
\pgfsetlinewidth{1.003750pt}%
\definecolor{currentstroke}{rgb}{0.121569,0.466667,0.705882}%
\pgfsetstrokecolor{currentstroke}%
\pgfsetstrokeopacity{0.849447}%
\pgfsetdash{}{0pt}%
\pgfpathmoveto{\pgfqpoint{1.853188in}{2.744420in}}%
\pgfpathcurveto{\pgfqpoint{1.861424in}{2.744420in}}{\pgfqpoint{1.869324in}{2.747692in}}{\pgfqpoint{1.875148in}{2.753516in}}%
\pgfpathcurveto{\pgfqpoint{1.880972in}{2.759340in}}{\pgfqpoint{1.884245in}{2.767240in}}{\pgfqpoint{1.884245in}{2.775477in}}%
\pgfpathcurveto{\pgfqpoint{1.884245in}{2.783713in}}{\pgfqpoint{1.880972in}{2.791613in}}{\pgfqpoint{1.875148in}{2.797437in}}%
\pgfpathcurveto{\pgfqpoint{1.869324in}{2.803261in}}{\pgfqpoint{1.861424in}{2.806533in}}{\pgfqpoint{1.853188in}{2.806533in}}%
\pgfpathcurveto{\pgfqpoint{1.844952in}{2.806533in}}{\pgfqpoint{1.837052in}{2.803261in}}{\pgfqpoint{1.831228in}{2.797437in}}%
\pgfpathcurveto{\pgfqpoint{1.825404in}{2.791613in}}{\pgfqpoint{1.822132in}{2.783713in}}{\pgfqpoint{1.822132in}{2.775477in}}%
\pgfpathcurveto{\pgfqpoint{1.822132in}{2.767240in}}{\pgfqpoint{1.825404in}{2.759340in}}{\pgfqpoint{1.831228in}{2.753516in}}%
\pgfpathcurveto{\pgfqpoint{1.837052in}{2.747692in}}{\pgfqpoint{1.844952in}{2.744420in}}{\pgfqpoint{1.853188in}{2.744420in}}%
\pgfpathclose%
\pgfusepath{stroke,fill}%
\end{pgfscope}%
\begin{pgfscope}%
\pgfpathrectangle{\pgfqpoint{0.100000in}{0.212622in}}{\pgfqpoint{3.696000in}{3.696000in}}%
\pgfusepath{clip}%
\pgfsetbuttcap%
\pgfsetroundjoin%
\definecolor{currentfill}{rgb}{0.121569,0.466667,0.705882}%
\pgfsetfillcolor{currentfill}%
\pgfsetfillopacity{0.849677}%
\pgfsetlinewidth{1.003750pt}%
\definecolor{currentstroke}{rgb}{0.121569,0.466667,0.705882}%
\pgfsetstrokecolor{currentstroke}%
\pgfsetstrokeopacity{0.849677}%
\pgfsetdash{}{0pt}%
\pgfpathmoveto{\pgfqpoint{1.852808in}{2.743968in}}%
\pgfpathcurveto{\pgfqpoint{1.861045in}{2.743968in}}{\pgfqpoint{1.868945in}{2.747241in}}{\pgfqpoint{1.874769in}{2.753064in}}%
\pgfpathcurveto{\pgfqpoint{1.880593in}{2.758888in}}{\pgfqpoint{1.883865in}{2.766788in}}{\pgfqpoint{1.883865in}{2.775025in}}%
\pgfpathcurveto{\pgfqpoint{1.883865in}{2.783261in}}{\pgfqpoint{1.880593in}{2.791161in}}{\pgfqpoint{1.874769in}{2.796985in}}%
\pgfpathcurveto{\pgfqpoint{1.868945in}{2.802809in}}{\pgfqpoint{1.861045in}{2.806081in}}{\pgfqpoint{1.852808in}{2.806081in}}%
\pgfpathcurveto{\pgfqpoint{1.844572in}{2.806081in}}{\pgfqpoint{1.836672in}{2.802809in}}{\pgfqpoint{1.830848in}{2.796985in}}%
\pgfpathcurveto{\pgfqpoint{1.825024in}{2.791161in}}{\pgfqpoint{1.821752in}{2.783261in}}{\pgfqpoint{1.821752in}{2.775025in}}%
\pgfpathcurveto{\pgfqpoint{1.821752in}{2.766788in}}{\pgfqpoint{1.825024in}{2.758888in}}{\pgfqpoint{1.830848in}{2.753064in}}%
\pgfpathcurveto{\pgfqpoint{1.836672in}{2.747241in}}{\pgfqpoint{1.844572in}{2.743968in}}{\pgfqpoint{1.852808in}{2.743968in}}%
\pgfpathclose%
\pgfusepath{stroke,fill}%
\end{pgfscope}%
\begin{pgfscope}%
\pgfpathrectangle{\pgfqpoint{0.100000in}{0.212622in}}{\pgfqpoint{3.696000in}{3.696000in}}%
\pgfusepath{clip}%
\pgfsetbuttcap%
\pgfsetroundjoin%
\definecolor{currentfill}{rgb}{0.121569,0.466667,0.705882}%
\pgfsetfillcolor{currentfill}%
\pgfsetfillopacity{0.849697}%
\pgfsetlinewidth{1.003750pt}%
\definecolor{currentstroke}{rgb}{0.121569,0.466667,0.705882}%
\pgfsetstrokecolor{currentstroke}%
\pgfsetstrokeopacity{0.849697}%
\pgfsetdash{}{0pt}%
\pgfpathmoveto{\pgfqpoint{2.307552in}{2.619994in}}%
\pgfpathcurveto{\pgfqpoint{2.315788in}{2.619994in}}{\pgfqpoint{2.323688in}{2.623266in}}{\pgfqpoint{2.329512in}{2.629090in}}%
\pgfpathcurveto{\pgfqpoint{2.335336in}{2.634914in}}{\pgfqpoint{2.338609in}{2.642814in}}{\pgfqpoint{2.338609in}{2.651050in}}%
\pgfpathcurveto{\pgfqpoint{2.338609in}{2.659287in}}{\pgfqpoint{2.335336in}{2.667187in}}{\pgfqpoint{2.329512in}{2.673011in}}%
\pgfpathcurveto{\pgfqpoint{2.323688in}{2.678835in}}{\pgfqpoint{2.315788in}{2.682107in}}{\pgfqpoint{2.307552in}{2.682107in}}%
\pgfpathcurveto{\pgfqpoint{2.299316in}{2.682107in}}{\pgfqpoint{2.291416in}{2.678835in}}{\pgfqpoint{2.285592in}{2.673011in}}%
\pgfpathcurveto{\pgfqpoint{2.279768in}{2.667187in}}{\pgfqpoint{2.276496in}{2.659287in}}{\pgfqpoint{2.276496in}{2.651050in}}%
\pgfpathcurveto{\pgfqpoint{2.276496in}{2.642814in}}{\pgfqpoint{2.279768in}{2.634914in}}{\pgfqpoint{2.285592in}{2.629090in}}%
\pgfpathcurveto{\pgfqpoint{2.291416in}{2.623266in}}{\pgfqpoint{2.299316in}{2.619994in}}{\pgfqpoint{2.307552in}{2.619994in}}%
\pgfpathclose%
\pgfusepath{stroke,fill}%
\end{pgfscope}%
\begin{pgfscope}%
\pgfpathrectangle{\pgfqpoint{0.100000in}{0.212622in}}{\pgfqpoint{3.696000in}{3.696000in}}%
\pgfusepath{clip}%
\pgfsetbuttcap%
\pgfsetroundjoin%
\definecolor{currentfill}{rgb}{0.121569,0.466667,0.705882}%
\pgfsetfillcolor{currentfill}%
\pgfsetfillopacity{0.850363}%
\pgfsetlinewidth{1.003750pt}%
\definecolor{currentstroke}{rgb}{0.121569,0.466667,0.705882}%
\pgfsetstrokecolor{currentstroke}%
\pgfsetstrokeopacity{0.850363}%
\pgfsetdash{}{0pt}%
\pgfpathmoveto{\pgfqpoint{2.305969in}{2.618248in}}%
\pgfpathcurveto{\pgfqpoint{2.314206in}{2.618248in}}{\pgfqpoint{2.322106in}{2.621520in}}{\pgfqpoint{2.327930in}{2.627344in}}%
\pgfpathcurveto{\pgfqpoint{2.333754in}{2.633168in}}{\pgfqpoint{2.337026in}{2.641068in}}{\pgfqpoint{2.337026in}{2.649304in}}%
\pgfpathcurveto{\pgfqpoint{2.337026in}{2.657540in}}{\pgfqpoint{2.333754in}{2.665440in}}{\pgfqpoint{2.327930in}{2.671264in}}%
\pgfpathcurveto{\pgfqpoint{2.322106in}{2.677088in}}{\pgfqpoint{2.314206in}{2.680361in}}{\pgfqpoint{2.305969in}{2.680361in}}%
\pgfpathcurveto{\pgfqpoint{2.297733in}{2.680361in}}{\pgfqpoint{2.289833in}{2.677088in}}{\pgfqpoint{2.284009in}{2.671264in}}%
\pgfpathcurveto{\pgfqpoint{2.278185in}{2.665440in}}{\pgfqpoint{2.274913in}{2.657540in}}{\pgfqpoint{2.274913in}{2.649304in}}%
\pgfpathcurveto{\pgfqpoint{2.274913in}{2.641068in}}{\pgfqpoint{2.278185in}{2.633168in}}{\pgfqpoint{2.284009in}{2.627344in}}%
\pgfpathcurveto{\pgfqpoint{2.289833in}{2.621520in}}{\pgfqpoint{2.297733in}{2.618248in}}{\pgfqpoint{2.305969in}{2.618248in}}%
\pgfpathclose%
\pgfusepath{stroke,fill}%
\end{pgfscope}%
\begin{pgfscope}%
\pgfpathrectangle{\pgfqpoint{0.100000in}{0.212622in}}{\pgfqpoint{3.696000in}{3.696000in}}%
\pgfusepath{clip}%
\pgfsetbuttcap%
\pgfsetroundjoin%
\definecolor{currentfill}{rgb}{0.121569,0.466667,0.705882}%
\pgfsetfillcolor{currentfill}%
\pgfsetfillopacity{0.850367}%
\pgfsetlinewidth{1.003750pt}%
\definecolor{currentstroke}{rgb}{0.121569,0.466667,0.705882}%
\pgfsetstrokecolor{currentstroke}%
\pgfsetstrokeopacity{0.850367}%
\pgfsetdash{}{0pt}%
\pgfpathmoveto{\pgfqpoint{1.851633in}{2.742608in}}%
\pgfpathcurveto{\pgfqpoint{1.859870in}{2.742608in}}{\pgfqpoint{1.867770in}{2.745880in}}{\pgfqpoint{1.873594in}{2.751704in}}%
\pgfpathcurveto{\pgfqpoint{1.879417in}{2.757528in}}{\pgfqpoint{1.882690in}{2.765428in}}{\pgfqpoint{1.882690in}{2.773665in}}%
\pgfpathcurveto{\pgfqpoint{1.882690in}{2.781901in}}{\pgfqpoint{1.879417in}{2.789801in}}{\pgfqpoint{1.873594in}{2.795625in}}%
\pgfpathcurveto{\pgfqpoint{1.867770in}{2.801449in}}{\pgfqpoint{1.859870in}{2.804721in}}{\pgfqpoint{1.851633in}{2.804721in}}%
\pgfpathcurveto{\pgfqpoint{1.843397in}{2.804721in}}{\pgfqpoint{1.835497in}{2.801449in}}{\pgfqpoint{1.829673in}{2.795625in}}%
\pgfpathcurveto{\pgfqpoint{1.823849in}{2.789801in}}{\pgfqpoint{1.820577in}{2.781901in}}{\pgfqpoint{1.820577in}{2.773665in}}%
\pgfpathcurveto{\pgfqpoint{1.820577in}{2.765428in}}{\pgfqpoint{1.823849in}{2.757528in}}{\pgfqpoint{1.829673in}{2.751704in}}%
\pgfpathcurveto{\pgfqpoint{1.835497in}{2.745880in}}{\pgfqpoint{1.843397in}{2.742608in}}{\pgfqpoint{1.851633in}{2.742608in}}%
\pgfpathclose%
\pgfusepath{stroke,fill}%
\end{pgfscope}%
\begin{pgfscope}%
\pgfpathrectangle{\pgfqpoint{0.100000in}{0.212622in}}{\pgfqpoint{3.696000in}{3.696000in}}%
\pgfusepath{clip}%
\pgfsetbuttcap%
\pgfsetroundjoin%
\definecolor{currentfill}{rgb}{0.121569,0.466667,0.705882}%
\pgfsetfillcolor{currentfill}%
\pgfsetfillopacity{0.850741}%
\pgfsetlinewidth{1.003750pt}%
\definecolor{currentstroke}{rgb}{0.121569,0.466667,0.705882}%
\pgfsetstrokecolor{currentstroke}%
\pgfsetstrokeopacity{0.850741}%
\pgfsetdash{}{0pt}%
\pgfpathmoveto{\pgfqpoint{1.851022in}{2.741809in}}%
\pgfpathcurveto{\pgfqpoint{1.859258in}{2.741809in}}{\pgfqpoint{1.867158in}{2.745081in}}{\pgfqpoint{1.872982in}{2.750905in}}%
\pgfpathcurveto{\pgfqpoint{1.878806in}{2.756729in}}{\pgfqpoint{1.882078in}{2.764629in}}{\pgfqpoint{1.882078in}{2.772865in}}%
\pgfpathcurveto{\pgfqpoint{1.882078in}{2.781101in}}{\pgfqpoint{1.878806in}{2.789001in}}{\pgfqpoint{1.872982in}{2.794825in}}%
\pgfpathcurveto{\pgfqpoint{1.867158in}{2.800649in}}{\pgfqpoint{1.859258in}{2.803922in}}{\pgfqpoint{1.851022in}{2.803922in}}%
\pgfpathcurveto{\pgfqpoint{1.842786in}{2.803922in}}{\pgfqpoint{1.834886in}{2.800649in}}{\pgfqpoint{1.829062in}{2.794825in}}%
\pgfpathcurveto{\pgfqpoint{1.823238in}{2.789001in}}{\pgfqpoint{1.819965in}{2.781101in}}{\pgfqpoint{1.819965in}{2.772865in}}%
\pgfpathcurveto{\pgfqpoint{1.819965in}{2.764629in}}{\pgfqpoint{1.823238in}{2.756729in}}{\pgfqpoint{1.829062in}{2.750905in}}%
\pgfpathcurveto{\pgfqpoint{1.834886in}{2.745081in}}{\pgfqpoint{1.842786in}{2.741809in}}{\pgfqpoint{1.851022in}{2.741809in}}%
\pgfpathclose%
\pgfusepath{stroke,fill}%
\end{pgfscope}%
\begin{pgfscope}%
\pgfpathrectangle{\pgfqpoint{0.100000in}{0.212622in}}{\pgfqpoint{3.696000in}{3.696000in}}%
\pgfusepath{clip}%
\pgfsetbuttcap%
\pgfsetroundjoin%
\definecolor{currentfill}{rgb}{0.121569,0.466667,0.705882}%
\pgfsetfillcolor{currentfill}%
\pgfsetfillopacity{0.850948}%
\pgfsetlinewidth{1.003750pt}%
\definecolor{currentstroke}{rgb}{0.121569,0.466667,0.705882}%
\pgfsetstrokecolor{currentstroke}%
\pgfsetstrokeopacity{0.850948}%
\pgfsetdash{}{0pt}%
\pgfpathmoveto{\pgfqpoint{1.850698in}{2.741371in}}%
\pgfpathcurveto{\pgfqpoint{1.858935in}{2.741371in}}{\pgfqpoint{1.866835in}{2.744644in}}{\pgfqpoint{1.872659in}{2.750468in}}%
\pgfpathcurveto{\pgfqpoint{1.878483in}{2.756292in}}{\pgfqpoint{1.881755in}{2.764192in}}{\pgfqpoint{1.881755in}{2.772428in}}%
\pgfpathcurveto{\pgfqpoint{1.881755in}{2.780664in}}{\pgfqpoint{1.878483in}{2.788564in}}{\pgfqpoint{1.872659in}{2.794388in}}%
\pgfpathcurveto{\pgfqpoint{1.866835in}{2.800212in}}{\pgfqpoint{1.858935in}{2.803484in}}{\pgfqpoint{1.850698in}{2.803484in}}%
\pgfpathcurveto{\pgfqpoint{1.842462in}{2.803484in}}{\pgfqpoint{1.834562in}{2.800212in}}{\pgfqpoint{1.828738in}{2.794388in}}%
\pgfpathcurveto{\pgfqpoint{1.822914in}{2.788564in}}{\pgfqpoint{1.819642in}{2.780664in}}{\pgfqpoint{1.819642in}{2.772428in}}%
\pgfpathcurveto{\pgfqpoint{1.819642in}{2.764192in}}{\pgfqpoint{1.822914in}{2.756292in}}{\pgfqpoint{1.828738in}{2.750468in}}%
\pgfpathcurveto{\pgfqpoint{1.834562in}{2.744644in}}{\pgfqpoint{1.842462in}{2.741371in}}{\pgfqpoint{1.850698in}{2.741371in}}%
\pgfpathclose%
\pgfusepath{stroke,fill}%
\end{pgfscope}%
\begin{pgfscope}%
\pgfpathrectangle{\pgfqpoint{0.100000in}{0.212622in}}{\pgfqpoint{3.696000in}{3.696000in}}%
\pgfusepath{clip}%
\pgfsetbuttcap%
\pgfsetroundjoin%
\definecolor{currentfill}{rgb}{0.121569,0.466667,0.705882}%
\pgfsetfillcolor{currentfill}%
\pgfsetfillopacity{0.851246}%
\pgfsetlinewidth{1.003750pt}%
\definecolor{currentstroke}{rgb}{0.121569,0.466667,0.705882}%
\pgfsetstrokecolor{currentstroke}%
\pgfsetstrokeopacity{0.851246}%
\pgfsetdash{}{0pt}%
\pgfpathmoveto{\pgfqpoint{1.850228in}{2.740738in}}%
\pgfpathcurveto{\pgfqpoint{1.858464in}{2.740738in}}{\pgfqpoint{1.866364in}{2.744011in}}{\pgfqpoint{1.872188in}{2.749834in}}%
\pgfpathcurveto{\pgfqpoint{1.878012in}{2.755658in}}{\pgfqpoint{1.881284in}{2.763558in}}{\pgfqpoint{1.881284in}{2.771795in}}%
\pgfpathcurveto{\pgfqpoint{1.881284in}{2.780031in}}{\pgfqpoint{1.878012in}{2.787931in}}{\pgfqpoint{1.872188in}{2.793755in}}%
\pgfpathcurveto{\pgfqpoint{1.866364in}{2.799579in}}{\pgfqpoint{1.858464in}{2.802851in}}{\pgfqpoint{1.850228in}{2.802851in}}%
\pgfpathcurveto{\pgfqpoint{1.841991in}{2.802851in}}{\pgfqpoint{1.834091in}{2.799579in}}{\pgfqpoint{1.828267in}{2.793755in}}%
\pgfpathcurveto{\pgfqpoint{1.822444in}{2.787931in}}{\pgfqpoint{1.819171in}{2.780031in}}{\pgfqpoint{1.819171in}{2.771795in}}%
\pgfpathcurveto{\pgfqpoint{1.819171in}{2.763558in}}{\pgfqpoint{1.822444in}{2.755658in}}{\pgfqpoint{1.828267in}{2.749834in}}%
\pgfpathcurveto{\pgfqpoint{1.834091in}{2.744011in}}{\pgfqpoint{1.841991in}{2.740738in}}{\pgfqpoint{1.850228in}{2.740738in}}%
\pgfpathclose%
\pgfusepath{stroke,fill}%
\end{pgfscope}%
\begin{pgfscope}%
\pgfpathrectangle{\pgfqpoint{0.100000in}{0.212622in}}{\pgfqpoint{3.696000in}{3.696000in}}%
\pgfusepath{clip}%
\pgfsetbuttcap%
\pgfsetroundjoin%
\definecolor{currentfill}{rgb}{0.121569,0.466667,0.705882}%
\pgfsetfillcolor{currentfill}%
\pgfsetfillopacity{0.851599}%
\pgfsetlinewidth{1.003750pt}%
\definecolor{currentstroke}{rgb}{0.121569,0.466667,0.705882}%
\pgfsetstrokecolor{currentstroke}%
\pgfsetstrokeopacity{0.851599}%
\pgfsetdash{}{0pt}%
\pgfpathmoveto{\pgfqpoint{2.303154in}{2.615150in}}%
\pgfpathcurveto{\pgfqpoint{2.311390in}{2.615150in}}{\pgfqpoint{2.319290in}{2.618422in}}{\pgfqpoint{2.325114in}{2.624246in}}%
\pgfpathcurveto{\pgfqpoint{2.330938in}{2.630070in}}{\pgfqpoint{2.334211in}{2.637970in}}{\pgfqpoint{2.334211in}{2.646207in}}%
\pgfpathcurveto{\pgfqpoint{2.334211in}{2.654443in}}{\pgfqpoint{2.330938in}{2.662343in}}{\pgfqpoint{2.325114in}{2.668167in}}%
\pgfpathcurveto{\pgfqpoint{2.319290in}{2.673991in}}{\pgfqpoint{2.311390in}{2.677263in}}{\pgfqpoint{2.303154in}{2.677263in}}%
\pgfpathcurveto{\pgfqpoint{2.294918in}{2.677263in}}{\pgfqpoint{2.287018in}{2.673991in}}{\pgfqpoint{2.281194in}{2.668167in}}%
\pgfpathcurveto{\pgfqpoint{2.275370in}{2.662343in}}{\pgfqpoint{2.272098in}{2.654443in}}{\pgfqpoint{2.272098in}{2.646207in}}%
\pgfpathcurveto{\pgfqpoint{2.272098in}{2.637970in}}{\pgfqpoint{2.275370in}{2.630070in}}{\pgfqpoint{2.281194in}{2.624246in}}%
\pgfpathcurveto{\pgfqpoint{2.287018in}{2.618422in}}{\pgfqpoint{2.294918in}{2.615150in}}{\pgfqpoint{2.303154in}{2.615150in}}%
\pgfpathclose%
\pgfusepath{stroke,fill}%
\end{pgfscope}%
\begin{pgfscope}%
\pgfpathrectangle{\pgfqpoint{0.100000in}{0.212622in}}{\pgfqpoint{3.696000in}{3.696000in}}%
\pgfusepath{clip}%
\pgfsetbuttcap%
\pgfsetroundjoin%
\definecolor{currentfill}{rgb}{0.121569,0.466667,0.705882}%
\pgfsetfillcolor{currentfill}%
\pgfsetfillopacity{0.851633}%
\pgfsetlinewidth{1.003750pt}%
\definecolor{currentstroke}{rgb}{0.121569,0.466667,0.705882}%
\pgfsetstrokecolor{currentstroke}%
\pgfsetstrokeopacity{0.851633}%
\pgfsetdash{}{0pt}%
\pgfpathmoveto{\pgfqpoint{1.008415in}{2.209195in}}%
\pgfpathcurveto{\pgfqpoint{1.016651in}{2.209195in}}{\pgfqpoint{1.024551in}{2.212467in}}{\pgfqpoint{1.030375in}{2.218291in}}%
\pgfpathcurveto{\pgfqpoint{1.036199in}{2.224115in}}{\pgfqpoint{1.039471in}{2.232015in}}{\pgfqpoint{1.039471in}{2.240252in}}%
\pgfpathcurveto{\pgfqpoint{1.039471in}{2.248488in}}{\pgfqpoint{1.036199in}{2.256388in}}{\pgfqpoint{1.030375in}{2.262212in}}%
\pgfpathcurveto{\pgfqpoint{1.024551in}{2.268036in}}{\pgfqpoint{1.016651in}{2.271308in}}{\pgfqpoint{1.008415in}{2.271308in}}%
\pgfpathcurveto{\pgfqpoint{1.000178in}{2.271308in}}{\pgfqpoint{0.992278in}{2.268036in}}{\pgfqpoint{0.986454in}{2.262212in}}%
\pgfpathcurveto{\pgfqpoint{0.980630in}{2.256388in}}{\pgfqpoint{0.977358in}{2.248488in}}{\pgfqpoint{0.977358in}{2.240252in}}%
\pgfpathcurveto{\pgfqpoint{0.977358in}{2.232015in}}{\pgfqpoint{0.980630in}{2.224115in}}{\pgfqpoint{0.986454in}{2.218291in}}%
\pgfpathcurveto{\pgfqpoint{0.992278in}{2.212467in}}{\pgfqpoint{1.000178in}{2.209195in}}{\pgfqpoint{1.008415in}{2.209195in}}%
\pgfpathclose%
\pgfusepath{stroke,fill}%
\end{pgfscope}%
\begin{pgfscope}%
\pgfpathrectangle{\pgfqpoint{0.100000in}{0.212622in}}{\pgfqpoint{3.696000in}{3.696000in}}%
\pgfusepath{clip}%
\pgfsetbuttcap%
\pgfsetroundjoin%
\definecolor{currentfill}{rgb}{0.121569,0.466667,0.705882}%
\pgfsetfillcolor{currentfill}%
\pgfsetfillopacity{0.851814}%
\pgfsetlinewidth{1.003750pt}%
\definecolor{currentstroke}{rgb}{0.121569,0.466667,0.705882}%
\pgfsetstrokecolor{currentstroke}%
\pgfsetstrokeopacity{0.851814}%
\pgfsetdash{}{0pt}%
\pgfpathmoveto{\pgfqpoint{1.849435in}{2.739741in}}%
\pgfpathcurveto{\pgfqpoint{1.857672in}{2.739741in}}{\pgfqpoint{1.865572in}{2.743013in}}{\pgfqpoint{1.871396in}{2.748837in}}%
\pgfpathcurveto{\pgfqpoint{1.877220in}{2.754661in}}{\pgfqpoint{1.880492in}{2.762561in}}{\pgfqpoint{1.880492in}{2.770798in}}%
\pgfpathcurveto{\pgfqpoint{1.880492in}{2.779034in}}{\pgfqpoint{1.877220in}{2.786934in}}{\pgfqpoint{1.871396in}{2.792758in}}%
\pgfpathcurveto{\pgfqpoint{1.865572in}{2.798582in}}{\pgfqpoint{1.857672in}{2.801854in}}{\pgfqpoint{1.849435in}{2.801854in}}%
\pgfpathcurveto{\pgfqpoint{1.841199in}{2.801854in}}{\pgfqpoint{1.833299in}{2.798582in}}{\pgfqpoint{1.827475in}{2.792758in}}%
\pgfpathcurveto{\pgfqpoint{1.821651in}{2.786934in}}{\pgfqpoint{1.818379in}{2.779034in}}{\pgfqpoint{1.818379in}{2.770798in}}%
\pgfpathcurveto{\pgfqpoint{1.818379in}{2.762561in}}{\pgfqpoint{1.821651in}{2.754661in}}{\pgfqpoint{1.827475in}{2.748837in}}%
\pgfpathcurveto{\pgfqpoint{1.833299in}{2.743013in}}{\pgfqpoint{1.841199in}{2.739741in}}{\pgfqpoint{1.849435in}{2.739741in}}%
\pgfpathclose%
\pgfusepath{stroke,fill}%
\end{pgfscope}%
\begin{pgfscope}%
\pgfpathrectangle{\pgfqpoint{0.100000in}{0.212622in}}{\pgfqpoint{3.696000in}{3.696000in}}%
\pgfusepath{clip}%
\pgfsetbuttcap%
\pgfsetroundjoin%
\definecolor{currentfill}{rgb}{0.121569,0.466667,0.705882}%
\pgfsetfillcolor{currentfill}%
\pgfsetfillopacity{0.852326}%
\pgfsetlinewidth{1.003750pt}%
\definecolor{currentstroke}{rgb}{0.121569,0.466667,0.705882}%
\pgfsetstrokecolor{currentstroke}%
\pgfsetstrokeopacity{0.852326}%
\pgfsetdash{}{0pt}%
\pgfpathmoveto{\pgfqpoint{2.301377in}{2.612921in}}%
\pgfpathcurveto{\pgfqpoint{2.309613in}{2.612921in}}{\pgfqpoint{2.317513in}{2.616193in}}{\pgfqpoint{2.323337in}{2.622017in}}%
\pgfpathcurveto{\pgfqpoint{2.329161in}{2.627841in}}{\pgfqpoint{2.332433in}{2.635741in}}{\pgfqpoint{2.332433in}{2.643978in}}%
\pgfpathcurveto{\pgfqpoint{2.332433in}{2.652214in}}{\pgfqpoint{2.329161in}{2.660114in}}{\pgfqpoint{2.323337in}{2.665938in}}%
\pgfpathcurveto{\pgfqpoint{2.317513in}{2.671762in}}{\pgfqpoint{2.309613in}{2.675034in}}{\pgfqpoint{2.301377in}{2.675034in}}%
\pgfpathcurveto{\pgfqpoint{2.293141in}{2.675034in}}{\pgfqpoint{2.285240in}{2.671762in}}{\pgfqpoint{2.279417in}{2.665938in}}%
\pgfpathcurveto{\pgfqpoint{2.273593in}{2.660114in}}{\pgfqpoint{2.270320in}{2.652214in}}{\pgfqpoint{2.270320in}{2.643978in}}%
\pgfpathcurveto{\pgfqpoint{2.270320in}{2.635741in}}{\pgfqpoint{2.273593in}{2.627841in}}{\pgfqpoint{2.279417in}{2.622017in}}%
\pgfpathcurveto{\pgfqpoint{2.285240in}{2.616193in}}{\pgfqpoint{2.293141in}{2.612921in}}{\pgfqpoint{2.301377in}{2.612921in}}%
\pgfpathclose%
\pgfusepath{stroke,fill}%
\end{pgfscope}%
\begin{pgfscope}%
\pgfpathrectangle{\pgfqpoint{0.100000in}{0.212622in}}{\pgfqpoint{3.696000in}{3.696000in}}%
\pgfusepath{clip}%
\pgfsetbuttcap%
\pgfsetroundjoin%
\definecolor{currentfill}{rgb}{0.121569,0.466667,0.705882}%
\pgfsetfillcolor{currentfill}%
\pgfsetfillopacity{0.852477}%
\pgfsetlinewidth{1.003750pt}%
\definecolor{currentstroke}{rgb}{0.121569,0.466667,0.705882}%
\pgfsetstrokecolor{currentstroke}%
\pgfsetstrokeopacity{0.852477}%
\pgfsetdash{}{0pt}%
\pgfpathmoveto{\pgfqpoint{1.848507in}{2.738496in}}%
\pgfpathcurveto{\pgfqpoint{1.856743in}{2.738496in}}{\pgfqpoint{1.864643in}{2.741769in}}{\pgfqpoint{1.870467in}{2.747593in}}%
\pgfpathcurveto{\pgfqpoint{1.876291in}{2.753417in}}{\pgfqpoint{1.879563in}{2.761317in}}{\pgfqpoint{1.879563in}{2.769553in}}%
\pgfpathcurveto{\pgfqpoint{1.879563in}{2.777789in}}{\pgfqpoint{1.876291in}{2.785689in}}{\pgfqpoint{1.870467in}{2.791513in}}%
\pgfpathcurveto{\pgfqpoint{1.864643in}{2.797337in}}{\pgfqpoint{1.856743in}{2.800609in}}{\pgfqpoint{1.848507in}{2.800609in}}%
\pgfpathcurveto{\pgfqpoint{1.840271in}{2.800609in}}{\pgfqpoint{1.832371in}{2.797337in}}{\pgfqpoint{1.826547in}{2.791513in}}%
\pgfpathcurveto{\pgfqpoint{1.820723in}{2.785689in}}{\pgfqpoint{1.817450in}{2.777789in}}{\pgfqpoint{1.817450in}{2.769553in}}%
\pgfpathcurveto{\pgfqpoint{1.817450in}{2.761317in}}{\pgfqpoint{1.820723in}{2.753417in}}{\pgfqpoint{1.826547in}{2.747593in}}%
\pgfpathcurveto{\pgfqpoint{1.832371in}{2.741769in}}{\pgfqpoint{1.840271in}{2.738496in}}{\pgfqpoint{1.848507in}{2.738496in}}%
\pgfpathclose%
\pgfusepath{stroke,fill}%
\end{pgfscope}%
\begin{pgfscope}%
\pgfpathrectangle{\pgfqpoint{0.100000in}{0.212622in}}{\pgfqpoint{3.696000in}{3.696000in}}%
\pgfusepath{clip}%
\pgfsetbuttcap%
\pgfsetroundjoin%
\definecolor{currentfill}{rgb}{0.121569,0.466667,0.705882}%
\pgfsetfillcolor{currentfill}%
\pgfsetfillopacity{0.852832}%
\pgfsetlinewidth{1.003750pt}%
\definecolor{currentstroke}{rgb}{0.121569,0.466667,0.705882}%
\pgfsetstrokecolor{currentstroke}%
\pgfsetstrokeopacity{0.852832}%
\pgfsetdash{}{0pt}%
\pgfpathmoveto{\pgfqpoint{1.848026in}{2.737750in}}%
\pgfpathcurveto{\pgfqpoint{1.856262in}{2.737750in}}{\pgfqpoint{1.864162in}{2.741022in}}{\pgfqpoint{1.869986in}{2.746846in}}%
\pgfpathcurveto{\pgfqpoint{1.875810in}{2.752670in}}{\pgfqpoint{1.879082in}{2.760570in}}{\pgfqpoint{1.879082in}{2.768806in}}%
\pgfpathcurveto{\pgfqpoint{1.879082in}{2.777043in}}{\pgfqpoint{1.875810in}{2.784943in}}{\pgfqpoint{1.869986in}{2.790767in}}%
\pgfpathcurveto{\pgfqpoint{1.864162in}{2.796591in}}{\pgfqpoint{1.856262in}{2.799863in}}{\pgfqpoint{1.848026in}{2.799863in}}%
\pgfpathcurveto{\pgfqpoint{1.839790in}{2.799863in}}{\pgfqpoint{1.831889in}{2.796591in}}{\pgfqpoint{1.826066in}{2.790767in}}%
\pgfpathcurveto{\pgfqpoint{1.820242in}{2.784943in}}{\pgfqpoint{1.816969in}{2.777043in}}{\pgfqpoint{1.816969in}{2.768806in}}%
\pgfpathcurveto{\pgfqpoint{1.816969in}{2.760570in}}{\pgfqpoint{1.820242in}{2.752670in}}{\pgfqpoint{1.826066in}{2.746846in}}%
\pgfpathcurveto{\pgfqpoint{1.831889in}{2.741022in}}{\pgfqpoint{1.839790in}{2.737750in}}{\pgfqpoint{1.848026in}{2.737750in}}%
\pgfpathclose%
\pgfusepath{stroke,fill}%
\end{pgfscope}%
\begin{pgfscope}%
\pgfpathrectangle{\pgfqpoint{0.100000in}{0.212622in}}{\pgfqpoint{3.696000in}{3.696000in}}%
\pgfusepath{clip}%
\pgfsetbuttcap%
\pgfsetroundjoin%
\definecolor{currentfill}{rgb}{0.121569,0.466667,0.705882}%
\pgfsetfillcolor{currentfill}%
\pgfsetfillopacity{0.853572}%
\pgfsetlinewidth{1.003750pt}%
\definecolor{currentstroke}{rgb}{0.121569,0.466667,0.705882}%
\pgfsetstrokecolor{currentstroke}%
\pgfsetstrokeopacity{0.853572}%
\pgfsetdash{}{0pt}%
\pgfpathmoveto{\pgfqpoint{1.847149in}{2.736147in}}%
\pgfpathcurveto{\pgfqpoint{1.855386in}{2.736147in}}{\pgfqpoint{1.863286in}{2.739420in}}{\pgfqpoint{1.869110in}{2.745243in}}%
\pgfpathcurveto{\pgfqpoint{1.874934in}{2.751067in}}{\pgfqpoint{1.878206in}{2.758967in}}{\pgfqpoint{1.878206in}{2.767204in}}%
\pgfpathcurveto{\pgfqpoint{1.878206in}{2.775440in}}{\pgfqpoint{1.874934in}{2.783340in}}{\pgfqpoint{1.869110in}{2.789164in}}%
\pgfpathcurveto{\pgfqpoint{1.863286in}{2.794988in}}{\pgfqpoint{1.855386in}{2.798260in}}{\pgfqpoint{1.847149in}{2.798260in}}%
\pgfpathcurveto{\pgfqpoint{1.838913in}{2.798260in}}{\pgfqpoint{1.831013in}{2.794988in}}{\pgfqpoint{1.825189in}{2.789164in}}%
\pgfpathcurveto{\pgfqpoint{1.819365in}{2.783340in}}{\pgfqpoint{1.816093in}{2.775440in}}{\pgfqpoint{1.816093in}{2.767204in}}%
\pgfpathcurveto{\pgfqpoint{1.816093in}{2.758967in}}{\pgfqpoint{1.819365in}{2.751067in}}{\pgfqpoint{1.825189in}{2.745243in}}%
\pgfpathcurveto{\pgfqpoint{1.831013in}{2.739420in}}{\pgfqpoint{1.838913in}{2.736147in}}{\pgfqpoint{1.847149in}{2.736147in}}%
\pgfpathclose%
\pgfusepath{stroke,fill}%
\end{pgfscope}%
\begin{pgfscope}%
\pgfpathrectangle{\pgfqpoint{0.100000in}{0.212622in}}{\pgfqpoint{3.696000in}{3.696000in}}%
\pgfusepath{clip}%
\pgfsetbuttcap%
\pgfsetroundjoin%
\definecolor{currentfill}{rgb}{0.121569,0.466667,0.705882}%
\pgfsetfillcolor{currentfill}%
\pgfsetfillopacity{0.853612}%
\pgfsetlinewidth{1.003750pt}%
\definecolor{currentstroke}{rgb}{0.121569,0.466667,0.705882}%
\pgfsetstrokecolor{currentstroke}%
\pgfsetstrokeopacity{0.853612}%
\pgfsetdash{}{0pt}%
\pgfpathmoveto{\pgfqpoint{1.002066in}{2.200706in}}%
\pgfpathcurveto{\pgfqpoint{1.010302in}{2.200706in}}{\pgfqpoint{1.018202in}{2.203978in}}{\pgfqpoint{1.024026in}{2.209802in}}%
\pgfpathcurveto{\pgfqpoint{1.029850in}{2.215626in}}{\pgfqpoint{1.033122in}{2.223526in}}{\pgfqpoint{1.033122in}{2.231762in}}%
\pgfpathcurveto{\pgfqpoint{1.033122in}{2.239999in}}{\pgfqpoint{1.029850in}{2.247899in}}{\pgfqpoint{1.024026in}{2.253723in}}%
\pgfpathcurveto{\pgfqpoint{1.018202in}{2.259547in}}{\pgfqpoint{1.010302in}{2.262819in}}{\pgfqpoint{1.002066in}{2.262819in}}%
\pgfpathcurveto{\pgfqpoint{0.993829in}{2.262819in}}{\pgfqpoint{0.985929in}{2.259547in}}{\pgfqpoint{0.980105in}{2.253723in}}%
\pgfpathcurveto{\pgfqpoint{0.974281in}{2.247899in}}{\pgfqpoint{0.971009in}{2.239999in}}{\pgfqpoint{0.971009in}{2.231762in}}%
\pgfpathcurveto{\pgfqpoint{0.971009in}{2.223526in}}{\pgfqpoint{0.974281in}{2.215626in}}{\pgfqpoint{0.980105in}{2.209802in}}%
\pgfpathcurveto{\pgfqpoint{0.985929in}{2.203978in}}{\pgfqpoint{0.993829in}{2.200706in}}{\pgfqpoint{1.002066in}{2.200706in}}%
\pgfpathclose%
\pgfusepath{stroke,fill}%
\end{pgfscope}%
\begin{pgfscope}%
\pgfpathrectangle{\pgfqpoint{0.100000in}{0.212622in}}{\pgfqpoint{3.696000in}{3.696000in}}%
\pgfusepath{clip}%
\pgfsetbuttcap%
\pgfsetroundjoin%
\definecolor{currentfill}{rgb}{0.121569,0.466667,0.705882}%
\pgfsetfillcolor{currentfill}%
\pgfsetfillopacity{0.853616}%
\pgfsetlinewidth{1.003750pt}%
\definecolor{currentstroke}{rgb}{0.121569,0.466667,0.705882}%
\pgfsetstrokecolor{currentstroke}%
\pgfsetstrokeopacity{0.853616}%
\pgfsetdash{}{0pt}%
\pgfpathmoveto{\pgfqpoint{2.298020in}{2.608807in}}%
\pgfpathcurveto{\pgfqpoint{2.306256in}{2.608807in}}{\pgfqpoint{2.314156in}{2.612079in}}{\pgfqpoint{2.319980in}{2.617903in}}%
\pgfpathcurveto{\pgfqpoint{2.325804in}{2.623727in}}{\pgfqpoint{2.329076in}{2.631627in}}{\pgfqpoint{2.329076in}{2.639864in}}%
\pgfpathcurveto{\pgfqpoint{2.329076in}{2.648100in}}{\pgfqpoint{2.325804in}{2.656000in}}{\pgfqpoint{2.319980in}{2.661824in}}%
\pgfpathcurveto{\pgfqpoint{2.314156in}{2.667648in}}{\pgfqpoint{2.306256in}{2.670920in}}{\pgfqpoint{2.298020in}{2.670920in}}%
\pgfpathcurveto{\pgfqpoint{2.289783in}{2.670920in}}{\pgfqpoint{2.281883in}{2.667648in}}{\pgfqpoint{2.276059in}{2.661824in}}%
\pgfpathcurveto{\pgfqpoint{2.270235in}{2.656000in}}{\pgfqpoint{2.266963in}{2.648100in}}{\pgfqpoint{2.266963in}{2.639864in}}%
\pgfpathcurveto{\pgfqpoint{2.266963in}{2.631627in}}{\pgfqpoint{2.270235in}{2.623727in}}{\pgfqpoint{2.276059in}{2.617903in}}%
\pgfpathcurveto{\pgfqpoint{2.281883in}{2.612079in}}{\pgfqpoint{2.289783in}{2.608807in}}{\pgfqpoint{2.298020in}{2.608807in}}%
\pgfpathclose%
\pgfusepath{stroke,fill}%
\end{pgfscope}%
\begin{pgfscope}%
\pgfpathrectangle{\pgfqpoint{0.100000in}{0.212622in}}{\pgfqpoint{3.696000in}{3.696000in}}%
\pgfusepath{clip}%
\pgfsetbuttcap%
\pgfsetroundjoin%
\definecolor{currentfill}{rgb}{0.121569,0.466667,0.705882}%
\pgfsetfillcolor{currentfill}%
\pgfsetfillopacity{0.853982}%
\pgfsetlinewidth{1.003750pt}%
\definecolor{currentstroke}{rgb}{0.121569,0.466667,0.705882}%
\pgfsetstrokecolor{currentstroke}%
\pgfsetstrokeopacity{0.853982}%
\pgfsetdash{}{0pt}%
\pgfpathmoveto{\pgfqpoint{1.846708in}{2.735280in}}%
\pgfpathcurveto{\pgfqpoint{1.854944in}{2.735280in}}{\pgfqpoint{1.862844in}{2.738552in}}{\pgfqpoint{1.868668in}{2.744376in}}%
\pgfpathcurveto{\pgfqpoint{1.874492in}{2.750200in}}{\pgfqpoint{1.877765in}{2.758100in}}{\pgfqpoint{1.877765in}{2.766336in}}%
\pgfpathcurveto{\pgfqpoint{1.877765in}{2.774573in}}{\pgfqpoint{1.874492in}{2.782473in}}{\pgfqpoint{1.868668in}{2.788297in}}%
\pgfpathcurveto{\pgfqpoint{1.862844in}{2.794120in}}{\pgfqpoint{1.854944in}{2.797393in}}{\pgfqpoint{1.846708in}{2.797393in}}%
\pgfpathcurveto{\pgfqpoint{1.838472in}{2.797393in}}{\pgfqpoint{1.830572in}{2.794120in}}{\pgfqpoint{1.824748in}{2.788297in}}%
\pgfpathcurveto{\pgfqpoint{1.818924in}{2.782473in}}{\pgfqpoint{1.815652in}{2.774573in}}{\pgfqpoint{1.815652in}{2.766336in}}%
\pgfpathcurveto{\pgfqpoint{1.815652in}{2.758100in}}{\pgfqpoint{1.818924in}{2.750200in}}{\pgfqpoint{1.824748in}{2.744376in}}%
\pgfpathcurveto{\pgfqpoint{1.830572in}{2.738552in}}{\pgfqpoint{1.838472in}{2.735280in}}{\pgfqpoint{1.846708in}{2.735280in}}%
\pgfpathclose%
\pgfusepath{stroke,fill}%
\end{pgfscope}%
\begin{pgfscope}%
\pgfpathrectangle{\pgfqpoint{0.100000in}{0.212622in}}{\pgfqpoint{3.696000in}{3.696000in}}%
\pgfusepath{clip}%
\pgfsetbuttcap%
\pgfsetroundjoin%
\definecolor{currentfill}{rgb}{0.121569,0.466667,0.705882}%
\pgfsetfillcolor{currentfill}%
\pgfsetfillopacity{0.854529}%
\pgfsetlinewidth{1.003750pt}%
\definecolor{currentstroke}{rgb}{0.121569,0.466667,0.705882}%
\pgfsetstrokecolor{currentstroke}%
\pgfsetstrokeopacity{0.854529}%
\pgfsetdash{}{0pt}%
\pgfpathmoveto{\pgfqpoint{2.295764in}{2.606327in}}%
\pgfpathcurveto{\pgfqpoint{2.304000in}{2.606327in}}{\pgfqpoint{2.311900in}{2.609599in}}{\pgfqpoint{2.317724in}{2.615423in}}%
\pgfpathcurveto{\pgfqpoint{2.323548in}{2.621247in}}{\pgfqpoint{2.326820in}{2.629147in}}{\pgfqpoint{2.326820in}{2.637384in}}%
\pgfpathcurveto{\pgfqpoint{2.326820in}{2.645620in}}{\pgfqpoint{2.323548in}{2.653520in}}{\pgfqpoint{2.317724in}{2.659344in}}%
\pgfpathcurveto{\pgfqpoint{2.311900in}{2.665168in}}{\pgfqpoint{2.304000in}{2.668440in}}{\pgfqpoint{2.295764in}{2.668440in}}%
\pgfpathcurveto{\pgfqpoint{2.287528in}{2.668440in}}{\pgfqpoint{2.279628in}{2.665168in}}{\pgfqpoint{2.273804in}{2.659344in}}%
\pgfpathcurveto{\pgfqpoint{2.267980in}{2.653520in}}{\pgfqpoint{2.264707in}{2.645620in}}{\pgfqpoint{2.264707in}{2.637384in}}%
\pgfpathcurveto{\pgfqpoint{2.264707in}{2.629147in}}{\pgfqpoint{2.267980in}{2.621247in}}{\pgfqpoint{2.273804in}{2.615423in}}%
\pgfpathcurveto{\pgfqpoint{2.279628in}{2.609599in}}{\pgfqpoint{2.287528in}{2.606327in}}{\pgfqpoint{2.295764in}{2.606327in}}%
\pgfpathclose%
\pgfusepath{stroke,fill}%
\end{pgfscope}%
\begin{pgfscope}%
\pgfpathrectangle{\pgfqpoint{0.100000in}{0.212622in}}{\pgfqpoint{3.696000in}{3.696000in}}%
\pgfusepath{clip}%
\pgfsetbuttcap%
\pgfsetroundjoin%
\definecolor{currentfill}{rgb}{0.121569,0.466667,0.705882}%
\pgfsetfillcolor{currentfill}%
\pgfsetfillopacity{0.854622}%
\pgfsetlinewidth{1.003750pt}%
\definecolor{currentstroke}{rgb}{0.121569,0.466667,0.705882}%
\pgfsetstrokecolor{currentstroke}%
\pgfsetstrokeopacity{0.854622}%
\pgfsetdash{}{0pt}%
\pgfpathmoveto{\pgfqpoint{1.846056in}{2.733965in}}%
\pgfpathcurveto{\pgfqpoint{1.854292in}{2.733965in}}{\pgfqpoint{1.862192in}{2.737238in}}{\pgfqpoint{1.868016in}{2.743062in}}%
\pgfpathcurveto{\pgfqpoint{1.873840in}{2.748885in}}{\pgfqpoint{1.877113in}{2.756786in}}{\pgfqpoint{1.877113in}{2.765022in}}%
\pgfpathcurveto{\pgfqpoint{1.877113in}{2.773258in}}{\pgfqpoint{1.873840in}{2.781158in}}{\pgfqpoint{1.868016in}{2.786982in}}%
\pgfpathcurveto{\pgfqpoint{1.862192in}{2.792806in}}{\pgfqpoint{1.854292in}{2.796078in}}{\pgfqpoint{1.846056in}{2.796078in}}%
\pgfpathcurveto{\pgfqpoint{1.837820in}{2.796078in}}{\pgfqpoint{1.829920in}{2.792806in}}{\pgfqpoint{1.824096in}{2.786982in}}%
\pgfpathcurveto{\pgfqpoint{1.818272in}{2.781158in}}{\pgfqpoint{1.815000in}{2.773258in}}{\pgfqpoint{1.815000in}{2.765022in}}%
\pgfpathcurveto{\pgfqpoint{1.815000in}{2.756786in}}{\pgfqpoint{1.818272in}{2.748885in}}{\pgfqpoint{1.824096in}{2.743062in}}%
\pgfpathcurveto{\pgfqpoint{1.829920in}{2.737238in}}{\pgfqpoint{1.837820in}{2.733965in}}{\pgfqpoint{1.846056in}{2.733965in}}%
\pgfpathclose%
\pgfusepath{stroke,fill}%
\end{pgfscope}%
\begin{pgfscope}%
\pgfpathrectangle{\pgfqpoint{0.100000in}{0.212622in}}{\pgfqpoint{3.696000in}{3.696000in}}%
\pgfusepath{clip}%
\pgfsetbuttcap%
\pgfsetroundjoin%
\definecolor{currentfill}{rgb}{0.121569,0.466667,0.705882}%
\pgfsetfillcolor{currentfill}%
\pgfsetfillopacity{0.854686}%
\pgfsetlinewidth{1.003750pt}%
\definecolor{currentstroke}{rgb}{0.121569,0.466667,0.705882}%
\pgfsetstrokecolor{currentstroke}%
\pgfsetstrokeopacity{0.854686}%
\pgfsetdash{}{0pt}%
\pgfpathmoveto{\pgfqpoint{0.998770in}{2.195718in}}%
\pgfpathcurveto{\pgfqpoint{1.007007in}{2.195718in}}{\pgfqpoint{1.014907in}{2.198990in}}{\pgfqpoint{1.020731in}{2.204814in}}%
\pgfpathcurveto{\pgfqpoint{1.026554in}{2.210638in}}{\pgfqpoint{1.029827in}{2.218538in}}{\pgfqpoint{1.029827in}{2.226774in}}%
\pgfpathcurveto{\pgfqpoint{1.029827in}{2.235010in}}{\pgfqpoint{1.026554in}{2.242910in}}{\pgfqpoint{1.020731in}{2.248734in}}%
\pgfpathcurveto{\pgfqpoint{1.014907in}{2.254558in}}{\pgfqpoint{1.007007in}{2.257831in}}{\pgfqpoint{0.998770in}{2.257831in}}%
\pgfpathcurveto{\pgfqpoint{0.990534in}{2.257831in}}{\pgfqpoint{0.982634in}{2.254558in}}{\pgfqpoint{0.976810in}{2.248734in}}%
\pgfpathcurveto{\pgfqpoint{0.970986in}{2.242910in}}{\pgfqpoint{0.967714in}{2.235010in}}{\pgfqpoint{0.967714in}{2.226774in}}%
\pgfpathcurveto{\pgfqpoint{0.967714in}{2.218538in}}{\pgfqpoint{0.970986in}{2.210638in}}{\pgfqpoint{0.976810in}{2.204814in}}%
\pgfpathcurveto{\pgfqpoint{0.982634in}{2.198990in}}{\pgfqpoint{0.990534in}{2.195718in}}{\pgfqpoint{0.998770in}{2.195718in}}%
\pgfpathclose%
\pgfusepath{stroke,fill}%
\end{pgfscope}%
\begin{pgfscope}%
\pgfpathrectangle{\pgfqpoint{0.100000in}{0.212622in}}{\pgfqpoint{3.696000in}{3.696000in}}%
\pgfusepath{clip}%
\pgfsetbuttcap%
\pgfsetroundjoin%
\definecolor{currentfill}{rgb}{0.121569,0.466667,0.705882}%
\pgfsetfillcolor{currentfill}%
\pgfsetfillopacity{0.855293}%
\pgfsetlinewidth{1.003750pt}%
\definecolor{currentstroke}{rgb}{0.121569,0.466667,0.705882}%
\pgfsetstrokecolor{currentstroke}%
\pgfsetstrokeopacity{0.855293}%
\pgfsetdash{}{0pt}%
\pgfpathmoveto{\pgfqpoint{2.765926in}{1.420673in}}%
\pgfpathcurveto{\pgfqpoint{2.774163in}{1.420673in}}{\pgfqpoint{2.782063in}{1.423945in}}{\pgfqpoint{2.787887in}{1.429769in}}%
\pgfpathcurveto{\pgfqpoint{2.793710in}{1.435593in}}{\pgfqpoint{2.796983in}{1.443493in}}{\pgfqpoint{2.796983in}{1.451729in}}%
\pgfpathcurveto{\pgfqpoint{2.796983in}{1.459965in}}{\pgfqpoint{2.793710in}{1.467866in}}{\pgfqpoint{2.787887in}{1.473689in}}%
\pgfpathcurveto{\pgfqpoint{2.782063in}{1.479513in}}{\pgfqpoint{2.774163in}{1.482786in}}{\pgfqpoint{2.765926in}{1.482786in}}%
\pgfpathcurveto{\pgfqpoint{2.757690in}{1.482786in}}{\pgfqpoint{2.749790in}{1.479513in}}{\pgfqpoint{2.743966in}{1.473689in}}%
\pgfpathcurveto{\pgfqpoint{2.738142in}{1.467866in}}{\pgfqpoint{2.734870in}{1.459965in}}{\pgfqpoint{2.734870in}{1.451729in}}%
\pgfpathcurveto{\pgfqpoint{2.734870in}{1.443493in}}{\pgfqpoint{2.738142in}{1.435593in}}{\pgfqpoint{2.743966in}{1.429769in}}%
\pgfpathcurveto{\pgfqpoint{2.749790in}{1.423945in}}{\pgfqpoint{2.757690in}{1.420673in}}{\pgfqpoint{2.765926in}{1.420673in}}%
\pgfpathclose%
\pgfusepath{stroke,fill}%
\end{pgfscope}%
\begin{pgfscope}%
\pgfpathrectangle{\pgfqpoint{0.100000in}{0.212622in}}{\pgfqpoint{3.696000in}{3.696000in}}%
\pgfusepath{clip}%
\pgfsetbuttcap%
\pgfsetroundjoin%
\definecolor{currentfill}{rgb}{0.121569,0.466667,0.705882}%
\pgfsetfillcolor{currentfill}%
\pgfsetfillopacity{0.855380}%
\pgfsetlinewidth{1.003750pt}%
\definecolor{currentstroke}{rgb}{0.121569,0.466667,0.705882}%
\pgfsetstrokecolor{currentstroke}%
\pgfsetstrokeopacity{0.855380}%
\pgfsetdash{}{0pt}%
\pgfpathmoveto{\pgfqpoint{2.293764in}{2.604112in}}%
\pgfpathcurveto{\pgfqpoint{2.302001in}{2.604112in}}{\pgfqpoint{2.309901in}{2.607385in}}{\pgfqpoint{2.315725in}{2.613209in}}%
\pgfpathcurveto{\pgfqpoint{2.321548in}{2.619033in}}{\pgfqpoint{2.324821in}{2.626933in}}{\pgfqpoint{2.324821in}{2.635169in}}%
\pgfpathcurveto{\pgfqpoint{2.324821in}{2.643405in}}{\pgfqpoint{2.321548in}{2.651305in}}{\pgfqpoint{2.315725in}{2.657129in}}%
\pgfpathcurveto{\pgfqpoint{2.309901in}{2.662953in}}{\pgfqpoint{2.302001in}{2.666225in}}{\pgfqpoint{2.293764in}{2.666225in}}%
\pgfpathcurveto{\pgfqpoint{2.285528in}{2.666225in}}{\pgfqpoint{2.277628in}{2.662953in}}{\pgfqpoint{2.271804in}{2.657129in}}%
\pgfpathcurveto{\pgfqpoint{2.265980in}{2.651305in}}{\pgfqpoint{2.262708in}{2.643405in}}{\pgfqpoint{2.262708in}{2.635169in}}%
\pgfpathcurveto{\pgfqpoint{2.262708in}{2.626933in}}{\pgfqpoint{2.265980in}{2.619033in}}{\pgfqpoint{2.271804in}{2.613209in}}%
\pgfpathcurveto{\pgfqpoint{2.277628in}{2.607385in}}{\pgfqpoint{2.285528in}{2.604112in}}{\pgfqpoint{2.293764in}{2.604112in}}%
\pgfpathclose%
\pgfusepath{stroke,fill}%
\end{pgfscope}%
\begin{pgfscope}%
\pgfpathrectangle{\pgfqpoint{0.100000in}{0.212622in}}{\pgfqpoint{3.696000in}{3.696000in}}%
\pgfusepath{clip}%
\pgfsetbuttcap%
\pgfsetroundjoin%
\definecolor{currentfill}{rgb}{0.121569,0.466667,0.705882}%
\pgfsetfillcolor{currentfill}%
\pgfsetfillopacity{0.855590}%
\pgfsetlinewidth{1.003750pt}%
\definecolor{currentstroke}{rgb}{0.121569,0.466667,0.705882}%
\pgfsetstrokecolor{currentstroke}%
\pgfsetstrokeopacity{0.855590}%
\pgfsetdash{}{0pt}%
\pgfpathmoveto{\pgfqpoint{1.844916in}{2.731710in}}%
\pgfpathcurveto{\pgfqpoint{1.853152in}{2.731710in}}{\pgfqpoint{1.861052in}{2.734982in}}{\pgfqpoint{1.866876in}{2.740806in}}%
\pgfpathcurveto{\pgfqpoint{1.872700in}{2.746630in}}{\pgfqpoint{1.875972in}{2.754530in}}{\pgfqpoint{1.875972in}{2.762766in}}%
\pgfpathcurveto{\pgfqpoint{1.875972in}{2.771002in}}{\pgfqpoint{1.872700in}{2.778902in}}{\pgfqpoint{1.866876in}{2.784726in}}%
\pgfpathcurveto{\pgfqpoint{1.861052in}{2.790550in}}{\pgfqpoint{1.853152in}{2.793823in}}{\pgfqpoint{1.844916in}{2.793823in}}%
\pgfpathcurveto{\pgfqpoint{1.836680in}{2.793823in}}{\pgfqpoint{1.828780in}{2.790550in}}{\pgfqpoint{1.822956in}{2.784726in}}%
\pgfpathcurveto{\pgfqpoint{1.817132in}{2.778902in}}{\pgfqpoint{1.813859in}{2.771002in}}{\pgfqpoint{1.813859in}{2.762766in}}%
\pgfpathcurveto{\pgfqpoint{1.813859in}{2.754530in}}{\pgfqpoint{1.817132in}{2.746630in}}{\pgfqpoint{1.822956in}{2.740806in}}%
\pgfpathcurveto{\pgfqpoint{1.828780in}{2.734982in}}{\pgfqpoint{1.836680in}{2.731710in}}{\pgfqpoint{1.844916in}{2.731710in}}%
\pgfpathclose%
\pgfusepath{stroke,fill}%
\end{pgfscope}%
\begin{pgfscope}%
\pgfpathrectangle{\pgfqpoint{0.100000in}{0.212622in}}{\pgfqpoint{3.696000in}{3.696000in}}%
\pgfusepath{clip}%
\pgfsetbuttcap%
\pgfsetroundjoin%
\definecolor{currentfill}{rgb}{0.121569,0.466667,0.705882}%
\pgfsetfillcolor{currentfill}%
\pgfsetfillopacity{0.855880}%
\pgfsetlinewidth{1.003750pt}%
\definecolor{currentstroke}{rgb}{0.121569,0.466667,0.705882}%
\pgfsetstrokecolor{currentstroke}%
\pgfsetstrokeopacity{0.855880}%
\pgfsetdash{}{0pt}%
\pgfpathmoveto{\pgfqpoint{2.292513in}{2.602614in}}%
\pgfpathcurveto{\pgfqpoint{2.300749in}{2.602614in}}{\pgfqpoint{2.308649in}{2.605886in}}{\pgfqpoint{2.314473in}{2.611710in}}%
\pgfpathcurveto{\pgfqpoint{2.320297in}{2.617534in}}{\pgfqpoint{2.323569in}{2.625434in}}{\pgfqpoint{2.323569in}{2.633670in}}%
\pgfpathcurveto{\pgfqpoint{2.323569in}{2.641907in}}{\pgfqpoint{2.320297in}{2.649807in}}{\pgfqpoint{2.314473in}{2.655631in}}%
\pgfpathcurveto{\pgfqpoint{2.308649in}{2.661455in}}{\pgfqpoint{2.300749in}{2.664727in}}{\pgfqpoint{2.292513in}{2.664727in}}%
\pgfpathcurveto{\pgfqpoint{2.284276in}{2.664727in}}{\pgfqpoint{2.276376in}{2.661455in}}{\pgfqpoint{2.270552in}{2.655631in}}%
\pgfpathcurveto{\pgfqpoint{2.264729in}{2.649807in}}{\pgfqpoint{2.261456in}{2.641907in}}{\pgfqpoint{2.261456in}{2.633670in}}%
\pgfpathcurveto{\pgfqpoint{2.261456in}{2.625434in}}{\pgfqpoint{2.264729in}{2.617534in}}{\pgfqpoint{2.270552in}{2.611710in}}%
\pgfpathcurveto{\pgfqpoint{2.276376in}{2.605886in}}{\pgfqpoint{2.284276in}{2.602614in}}{\pgfqpoint{2.292513in}{2.602614in}}%
\pgfpathclose%
\pgfusepath{stroke,fill}%
\end{pgfscope}%
\begin{pgfscope}%
\pgfpathrectangle{\pgfqpoint{0.100000in}{0.212622in}}{\pgfqpoint{3.696000in}{3.696000in}}%
\pgfusepath{clip}%
\pgfsetbuttcap%
\pgfsetroundjoin%
\definecolor{currentfill}{rgb}{0.121569,0.466667,0.705882}%
\pgfsetfillcolor{currentfill}%
\pgfsetfillopacity{0.855911}%
\pgfsetlinewidth{1.003750pt}%
\definecolor{currentstroke}{rgb}{0.121569,0.466667,0.705882}%
\pgfsetstrokecolor{currentstroke}%
\pgfsetstrokeopacity{0.855911}%
\pgfsetdash{}{0pt}%
\pgfpathmoveto{\pgfqpoint{0.995285in}{2.190496in}}%
\pgfpathcurveto{\pgfqpoint{1.003521in}{2.190496in}}{\pgfqpoint{1.011421in}{2.193768in}}{\pgfqpoint{1.017245in}{2.199592in}}%
\pgfpathcurveto{\pgfqpoint{1.023069in}{2.205416in}}{\pgfqpoint{1.026341in}{2.213316in}}{\pgfqpoint{1.026341in}{2.221552in}}%
\pgfpathcurveto{\pgfqpoint{1.026341in}{2.229789in}}{\pgfqpoint{1.023069in}{2.237689in}}{\pgfqpoint{1.017245in}{2.243513in}}%
\pgfpathcurveto{\pgfqpoint{1.011421in}{2.249337in}}{\pgfqpoint{1.003521in}{2.252609in}}{\pgfqpoint{0.995285in}{2.252609in}}%
\pgfpathcurveto{\pgfqpoint{0.987049in}{2.252609in}}{\pgfqpoint{0.979149in}{2.249337in}}{\pgfqpoint{0.973325in}{2.243513in}}%
\pgfpathcurveto{\pgfqpoint{0.967501in}{2.237689in}}{\pgfqpoint{0.964228in}{2.229789in}}{\pgfqpoint{0.964228in}{2.221552in}}%
\pgfpathcurveto{\pgfqpoint{0.964228in}{2.213316in}}{\pgfqpoint{0.967501in}{2.205416in}}{\pgfqpoint{0.973325in}{2.199592in}}%
\pgfpathcurveto{\pgfqpoint{0.979149in}{2.193768in}}{\pgfqpoint{0.987049in}{2.190496in}}{\pgfqpoint{0.995285in}{2.190496in}}%
\pgfpathclose%
\pgfusepath{stroke,fill}%
\end{pgfscope}%
\begin{pgfscope}%
\pgfpathrectangle{\pgfqpoint{0.100000in}{0.212622in}}{\pgfqpoint{3.696000in}{3.696000in}}%
\pgfusepath{clip}%
\pgfsetbuttcap%
\pgfsetroundjoin%
\definecolor{currentfill}{rgb}{0.121569,0.466667,0.705882}%
\pgfsetfillcolor{currentfill}%
\pgfsetfillopacity{0.856752}%
\pgfsetlinewidth{1.003750pt}%
\definecolor{currentstroke}{rgb}{0.121569,0.466667,0.705882}%
\pgfsetstrokecolor{currentstroke}%
\pgfsetstrokeopacity{0.856752}%
\pgfsetdash{}{0pt}%
\pgfpathmoveto{\pgfqpoint{1.843490in}{2.729117in}}%
\pgfpathcurveto{\pgfqpoint{1.851727in}{2.729117in}}{\pgfqpoint{1.859627in}{2.732389in}}{\pgfqpoint{1.865451in}{2.738213in}}%
\pgfpathcurveto{\pgfqpoint{1.871274in}{2.744037in}}{\pgfqpoint{1.874547in}{2.751937in}}{\pgfqpoint{1.874547in}{2.760174in}}%
\pgfpathcurveto{\pgfqpoint{1.874547in}{2.768410in}}{\pgfqpoint{1.871274in}{2.776310in}}{\pgfqpoint{1.865451in}{2.782134in}}%
\pgfpathcurveto{\pgfqpoint{1.859627in}{2.787958in}}{\pgfqpoint{1.851727in}{2.791230in}}{\pgfqpoint{1.843490in}{2.791230in}}%
\pgfpathcurveto{\pgfqpoint{1.835254in}{2.791230in}}{\pgfqpoint{1.827354in}{2.787958in}}{\pgfqpoint{1.821530in}{2.782134in}}%
\pgfpathcurveto{\pgfqpoint{1.815706in}{2.776310in}}{\pgfqpoint{1.812434in}{2.768410in}}{\pgfqpoint{1.812434in}{2.760174in}}%
\pgfpathcurveto{\pgfqpoint{1.812434in}{2.751937in}}{\pgfqpoint{1.815706in}{2.744037in}}{\pgfqpoint{1.821530in}{2.738213in}}%
\pgfpathcurveto{\pgfqpoint{1.827354in}{2.732389in}}{\pgfqpoint{1.835254in}{2.729117in}}{\pgfqpoint{1.843490in}{2.729117in}}%
\pgfpathclose%
\pgfusepath{stroke,fill}%
\end{pgfscope}%
\begin{pgfscope}%
\pgfpathrectangle{\pgfqpoint{0.100000in}{0.212622in}}{\pgfqpoint{3.696000in}{3.696000in}}%
\pgfusepath{clip}%
\pgfsetbuttcap%
\pgfsetroundjoin%
\definecolor{currentfill}{rgb}{0.121569,0.466667,0.705882}%
\pgfsetfillcolor{currentfill}%
\pgfsetfillopacity{0.856785}%
\pgfsetlinewidth{1.003750pt}%
\definecolor{currentstroke}{rgb}{0.121569,0.466667,0.705882}%
\pgfsetstrokecolor{currentstroke}%
\pgfsetstrokeopacity{0.856785}%
\pgfsetdash{}{0pt}%
\pgfpathmoveto{\pgfqpoint{2.290216in}{2.599898in}}%
\pgfpathcurveto{\pgfqpoint{2.298453in}{2.599898in}}{\pgfqpoint{2.306353in}{2.603170in}}{\pgfqpoint{2.312177in}{2.608994in}}%
\pgfpathcurveto{\pgfqpoint{2.318001in}{2.614818in}}{\pgfqpoint{2.321273in}{2.622718in}}{\pgfqpoint{2.321273in}{2.630954in}}%
\pgfpathcurveto{\pgfqpoint{2.321273in}{2.639191in}}{\pgfqpoint{2.318001in}{2.647091in}}{\pgfqpoint{2.312177in}{2.652914in}}%
\pgfpathcurveto{\pgfqpoint{2.306353in}{2.658738in}}{\pgfqpoint{2.298453in}{2.662011in}}{\pgfqpoint{2.290216in}{2.662011in}}%
\pgfpathcurveto{\pgfqpoint{2.281980in}{2.662011in}}{\pgfqpoint{2.274080in}{2.658738in}}{\pgfqpoint{2.268256in}{2.652914in}}%
\pgfpathcurveto{\pgfqpoint{2.262432in}{2.647091in}}{\pgfqpoint{2.259160in}{2.639191in}}{\pgfqpoint{2.259160in}{2.630954in}}%
\pgfpathcurveto{\pgfqpoint{2.259160in}{2.622718in}}{\pgfqpoint{2.262432in}{2.614818in}}{\pgfqpoint{2.268256in}{2.608994in}}%
\pgfpathcurveto{\pgfqpoint{2.274080in}{2.603170in}}{\pgfqpoint{2.281980in}{2.599898in}}{\pgfqpoint{2.290216in}{2.599898in}}%
\pgfpathclose%
\pgfusepath{stroke,fill}%
\end{pgfscope}%
\begin{pgfscope}%
\pgfpathrectangle{\pgfqpoint{0.100000in}{0.212622in}}{\pgfqpoint{3.696000in}{3.696000in}}%
\pgfusepath{clip}%
\pgfsetbuttcap%
\pgfsetroundjoin%
\definecolor{currentfill}{rgb}{0.121569,0.466667,0.705882}%
\pgfsetfillcolor{currentfill}%
\pgfsetfillopacity{0.857271}%
\pgfsetlinewidth{1.003750pt}%
\definecolor{currentstroke}{rgb}{0.121569,0.466667,0.705882}%
\pgfsetstrokecolor{currentstroke}%
\pgfsetstrokeopacity{0.857271}%
\pgfsetdash{}{0pt}%
\pgfpathmoveto{\pgfqpoint{2.289049in}{2.598624in}}%
\pgfpathcurveto{\pgfqpoint{2.297286in}{2.598624in}}{\pgfqpoint{2.305186in}{2.601896in}}{\pgfqpoint{2.311010in}{2.607720in}}%
\pgfpathcurveto{\pgfqpoint{2.316833in}{2.613544in}}{\pgfqpoint{2.320106in}{2.621444in}}{\pgfqpoint{2.320106in}{2.629680in}}%
\pgfpathcurveto{\pgfqpoint{2.320106in}{2.637917in}}{\pgfqpoint{2.316833in}{2.645817in}}{\pgfqpoint{2.311010in}{2.651641in}}%
\pgfpathcurveto{\pgfqpoint{2.305186in}{2.657464in}}{\pgfqpoint{2.297286in}{2.660737in}}{\pgfqpoint{2.289049in}{2.660737in}}%
\pgfpathcurveto{\pgfqpoint{2.280813in}{2.660737in}}{\pgfqpoint{2.272913in}{2.657464in}}{\pgfqpoint{2.267089in}{2.651641in}}%
\pgfpathcurveto{\pgfqpoint{2.261265in}{2.645817in}}{\pgfqpoint{2.257993in}{2.637917in}}{\pgfqpoint{2.257993in}{2.629680in}}%
\pgfpathcurveto{\pgfqpoint{2.257993in}{2.621444in}}{\pgfqpoint{2.261265in}{2.613544in}}{\pgfqpoint{2.267089in}{2.607720in}}%
\pgfpathcurveto{\pgfqpoint{2.272913in}{2.601896in}}{\pgfqpoint{2.280813in}{2.598624in}}{\pgfqpoint{2.289049in}{2.598624in}}%
\pgfpathclose%
\pgfusepath{stroke,fill}%
\end{pgfscope}%
\begin{pgfscope}%
\pgfpathrectangle{\pgfqpoint{0.100000in}{0.212622in}}{\pgfqpoint{3.696000in}{3.696000in}}%
\pgfusepath{clip}%
\pgfsetbuttcap%
\pgfsetroundjoin%
\definecolor{currentfill}{rgb}{0.121569,0.466667,0.705882}%
\pgfsetfillcolor{currentfill}%
\pgfsetfillopacity{0.857285}%
\pgfsetlinewidth{1.003750pt}%
\definecolor{currentstroke}{rgb}{0.121569,0.466667,0.705882}%
\pgfsetstrokecolor{currentstroke}%
\pgfsetstrokeopacity{0.857285}%
\pgfsetdash{}{0pt}%
\pgfpathmoveto{\pgfqpoint{0.991509in}{2.185202in}}%
\pgfpathcurveto{\pgfqpoint{0.999745in}{2.185202in}}{\pgfqpoint{1.007645in}{2.188474in}}{\pgfqpoint{1.013469in}{2.194298in}}%
\pgfpathcurveto{\pgfqpoint{1.019293in}{2.200122in}}{\pgfqpoint{1.022565in}{2.208022in}}{\pgfqpoint{1.022565in}{2.216258in}}%
\pgfpathcurveto{\pgfqpoint{1.022565in}{2.224495in}}{\pgfqpoint{1.019293in}{2.232395in}}{\pgfqpoint{1.013469in}{2.238219in}}%
\pgfpathcurveto{\pgfqpoint{1.007645in}{2.244043in}}{\pgfqpoint{0.999745in}{2.247315in}}{\pgfqpoint{0.991509in}{2.247315in}}%
\pgfpathcurveto{\pgfqpoint{0.983273in}{2.247315in}}{\pgfqpoint{0.975373in}{2.244043in}}{\pgfqpoint{0.969549in}{2.238219in}}%
\pgfpathcurveto{\pgfqpoint{0.963725in}{2.232395in}}{\pgfqpoint{0.960452in}{2.224495in}}{\pgfqpoint{0.960452in}{2.216258in}}%
\pgfpathcurveto{\pgfqpoint{0.960452in}{2.208022in}}{\pgfqpoint{0.963725in}{2.200122in}}{\pgfqpoint{0.969549in}{2.194298in}}%
\pgfpathcurveto{\pgfqpoint{0.975373in}{2.188474in}}{\pgfqpoint{0.983273in}{2.185202in}}{\pgfqpoint{0.991509in}{2.185202in}}%
\pgfpathclose%
\pgfusepath{stroke,fill}%
\end{pgfscope}%
\begin{pgfscope}%
\pgfpathrectangle{\pgfqpoint{0.100000in}{0.212622in}}{\pgfqpoint{3.696000in}{3.696000in}}%
\pgfusepath{clip}%
\pgfsetbuttcap%
\pgfsetroundjoin%
\definecolor{currentfill}{rgb}{0.121569,0.466667,0.705882}%
\pgfsetfillcolor{currentfill}%
\pgfsetfillopacity{0.857393}%
\pgfsetlinewidth{1.003750pt}%
\definecolor{currentstroke}{rgb}{0.121569,0.466667,0.705882}%
\pgfsetstrokecolor{currentstroke}%
\pgfsetstrokeopacity{0.857393}%
\pgfsetdash{}{0pt}%
\pgfpathmoveto{\pgfqpoint{1.842598in}{2.727720in}}%
\pgfpathcurveto{\pgfqpoint{1.850834in}{2.727720in}}{\pgfqpoint{1.858734in}{2.730992in}}{\pgfqpoint{1.864558in}{2.736816in}}%
\pgfpathcurveto{\pgfqpoint{1.870382in}{2.742640in}}{\pgfqpoint{1.873654in}{2.750540in}}{\pgfqpoint{1.873654in}{2.758776in}}%
\pgfpathcurveto{\pgfqpoint{1.873654in}{2.767012in}}{\pgfqpoint{1.870382in}{2.774912in}}{\pgfqpoint{1.864558in}{2.780736in}}%
\pgfpathcurveto{\pgfqpoint{1.858734in}{2.786560in}}{\pgfqpoint{1.850834in}{2.789833in}}{\pgfqpoint{1.842598in}{2.789833in}}%
\pgfpathcurveto{\pgfqpoint{1.834362in}{2.789833in}}{\pgfqpoint{1.826462in}{2.786560in}}{\pgfqpoint{1.820638in}{2.780736in}}%
\pgfpathcurveto{\pgfqpoint{1.814814in}{2.774912in}}{\pgfqpoint{1.811541in}{2.767012in}}{\pgfqpoint{1.811541in}{2.758776in}}%
\pgfpathcurveto{\pgfqpoint{1.811541in}{2.750540in}}{\pgfqpoint{1.814814in}{2.742640in}}{\pgfqpoint{1.820638in}{2.736816in}}%
\pgfpathcurveto{\pgfqpoint{1.826462in}{2.730992in}}{\pgfqpoint{1.834362in}{2.727720in}}{\pgfqpoint{1.842598in}{2.727720in}}%
\pgfpathclose%
\pgfusepath{stroke,fill}%
\end{pgfscope}%
\begin{pgfscope}%
\pgfpathrectangle{\pgfqpoint{0.100000in}{0.212622in}}{\pgfqpoint{3.696000in}{3.696000in}}%
\pgfusepath{clip}%
\pgfsetbuttcap%
\pgfsetroundjoin%
\definecolor{currentfill}{rgb}{0.121569,0.466667,0.705882}%
\pgfsetfillcolor{currentfill}%
\pgfsetfillopacity{0.858148}%
\pgfsetlinewidth{1.003750pt}%
\definecolor{currentstroke}{rgb}{0.121569,0.466667,0.705882}%
\pgfsetstrokecolor{currentstroke}%
\pgfsetstrokeopacity{0.858148}%
\pgfsetdash{}{0pt}%
\pgfpathmoveto{\pgfqpoint{1.841418in}{2.726052in}}%
\pgfpathcurveto{\pgfqpoint{1.849655in}{2.726052in}}{\pgfqpoint{1.857555in}{2.729324in}}{\pgfqpoint{1.863379in}{2.735148in}}%
\pgfpathcurveto{\pgfqpoint{1.869203in}{2.740972in}}{\pgfqpoint{1.872475in}{2.748872in}}{\pgfqpoint{1.872475in}{2.757108in}}%
\pgfpathcurveto{\pgfqpoint{1.872475in}{2.765345in}}{\pgfqpoint{1.869203in}{2.773245in}}{\pgfqpoint{1.863379in}{2.779068in}}%
\pgfpathcurveto{\pgfqpoint{1.857555in}{2.784892in}}{\pgfqpoint{1.849655in}{2.788165in}}{\pgfqpoint{1.841418in}{2.788165in}}%
\pgfpathcurveto{\pgfqpoint{1.833182in}{2.788165in}}{\pgfqpoint{1.825282in}{2.784892in}}{\pgfqpoint{1.819458in}{2.779068in}}%
\pgfpathcurveto{\pgfqpoint{1.813634in}{2.773245in}}{\pgfqpoint{1.810362in}{2.765345in}}{\pgfqpoint{1.810362in}{2.757108in}}%
\pgfpathcurveto{\pgfqpoint{1.810362in}{2.748872in}}{\pgfqpoint{1.813634in}{2.740972in}}{\pgfqpoint{1.819458in}{2.735148in}}%
\pgfpathcurveto{\pgfqpoint{1.825282in}{2.729324in}}{\pgfqpoint{1.833182in}{2.726052in}}{\pgfqpoint{1.841418in}{2.726052in}}%
\pgfpathclose%
\pgfusepath{stroke,fill}%
\end{pgfscope}%
\begin{pgfscope}%
\pgfpathrectangle{\pgfqpoint{0.100000in}{0.212622in}}{\pgfqpoint{3.696000in}{3.696000in}}%
\pgfusepath{clip}%
\pgfsetbuttcap%
\pgfsetroundjoin%
\definecolor{currentfill}{rgb}{0.121569,0.466667,0.705882}%
\pgfsetfillcolor{currentfill}%
\pgfsetfillopacity{0.858150}%
\pgfsetlinewidth{1.003750pt}%
\definecolor{currentstroke}{rgb}{0.121569,0.466667,0.705882}%
\pgfsetstrokecolor{currentstroke}%
\pgfsetstrokeopacity{0.858150}%
\pgfsetdash{}{0pt}%
\pgfpathmoveto{\pgfqpoint{2.286971in}{2.596231in}}%
\pgfpathcurveto{\pgfqpoint{2.295207in}{2.596231in}}{\pgfqpoint{2.303107in}{2.599503in}}{\pgfqpoint{2.308931in}{2.605327in}}%
\pgfpathcurveto{\pgfqpoint{2.314755in}{2.611151in}}{\pgfqpoint{2.318028in}{2.619051in}}{\pgfqpoint{2.318028in}{2.627287in}}%
\pgfpathcurveto{\pgfqpoint{2.318028in}{2.635523in}}{\pgfqpoint{2.314755in}{2.643424in}}{\pgfqpoint{2.308931in}{2.649247in}}%
\pgfpathcurveto{\pgfqpoint{2.303107in}{2.655071in}}{\pgfqpoint{2.295207in}{2.658344in}}{\pgfqpoint{2.286971in}{2.658344in}}%
\pgfpathcurveto{\pgfqpoint{2.278735in}{2.658344in}}{\pgfqpoint{2.270835in}{2.655071in}}{\pgfqpoint{2.265011in}{2.649247in}}%
\pgfpathcurveto{\pgfqpoint{2.259187in}{2.643424in}}{\pgfqpoint{2.255915in}{2.635523in}}{\pgfqpoint{2.255915in}{2.627287in}}%
\pgfpathcurveto{\pgfqpoint{2.255915in}{2.619051in}}{\pgfqpoint{2.259187in}{2.611151in}}{\pgfqpoint{2.265011in}{2.605327in}}%
\pgfpathcurveto{\pgfqpoint{2.270835in}{2.599503in}}{\pgfqpoint{2.278735in}{2.596231in}}{\pgfqpoint{2.286971in}{2.596231in}}%
\pgfpathclose%
\pgfusepath{stroke,fill}%
\end{pgfscope}%
\begin{pgfscope}%
\pgfpathrectangle{\pgfqpoint{0.100000in}{0.212622in}}{\pgfqpoint{3.696000in}{3.696000in}}%
\pgfusepath{clip}%
\pgfsetbuttcap%
\pgfsetroundjoin%
\definecolor{currentfill}{rgb}{0.121569,0.466667,0.705882}%
\pgfsetfillcolor{currentfill}%
\pgfsetfillopacity{0.858663}%
\pgfsetlinewidth{1.003750pt}%
\definecolor{currentstroke}{rgb}{0.121569,0.466667,0.705882}%
\pgfsetstrokecolor{currentstroke}%
\pgfsetstrokeopacity{0.858663}%
\pgfsetdash{}{0pt}%
\pgfpathmoveto{\pgfqpoint{2.285714in}{2.594702in}}%
\pgfpathcurveto{\pgfqpoint{2.293950in}{2.594702in}}{\pgfqpoint{2.301850in}{2.597975in}}{\pgfqpoint{2.307674in}{2.603798in}}%
\pgfpathcurveto{\pgfqpoint{2.313498in}{2.609622in}}{\pgfqpoint{2.316771in}{2.617522in}}{\pgfqpoint{2.316771in}{2.625759in}}%
\pgfpathcurveto{\pgfqpoint{2.316771in}{2.633995in}}{\pgfqpoint{2.313498in}{2.641895in}}{\pgfqpoint{2.307674in}{2.647719in}}%
\pgfpathcurveto{\pgfqpoint{2.301850in}{2.653543in}}{\pgfqpoint{2.293950in}{2.656815in}}{\pgfqpoint{2.285714in}{2.656815in}}%
\pgfpathcurveto{\pgfqpoint{2.277478in}{2.656815in}}{\pgfqpoint{2.269578in}{2.653543in}}{\pgfqpoint{2.263754in}{2.647719in}}%
\pgfpathcurveto{\pgfqpoint{2.257930in}{2.641895in}}{\pgfqpoint{2.254658in}{2.633995in}}{\pgfqpoint{2.254658in}{2.625759in}}%
\pgfpathcurveto{\pgfqpoint{2.254658in}{2.617522in}}{\pgfqpoint{2.257930in}{2.609622in}}{\pgfqpoint{2.263754in}{2.603798in}}%
\pgfpathcurveto{\pgfqpoint{2.269578in}{2.597975in}}{\pgfqpoint{2.277478in}{2.594702in}}{\pgfqpoint{2.285714in}{2.594702in}}%
\pgfpathclose%
\pgfusepath{stroke,fill}%
\end{pgfscope}%
\begin{pgfscope}%
\pgfpathrectangle{\pgfqpoint{0.100000in}{0.212622in}}{\pgfqpoint{3.696000in}{3.696000in}}%
\pgfusepath{clip}%
\pgfsetbuttcap%
\pgfsetroundjoin%
\definecolor{currentfill}{rgb}{0.121569,0.466667,0.705882}%
\pgfsetfillcolor{currentfill}%
\pgfsetfillopacity{0.858797}%
\pgfsetlinewidth{1.003750pt}%
\definecolor{currentstroke}{rgb}{0.121569,0.466667,0.705882}%
\pgfsetstrokecolor{currentstroke}%
\pgfsetstrokeopacity{0.858797}%
\pgfsetdash{}{0pt}%
\pgfpathmoveto{\pgfqpoint{0.987428in}{2.179940in}}%
\pgfpathcurveto{\pgfqpoint{0.995664in}{2.179940in}}{\pgfqpoint{1.003564in}{2.183212in}}{\pgfqpoint{1.009388in}{2.189036in}}%
\pgfpathcurveto{\pgfqpoint{1.015212in}{2.194860in}}{\pgfqpoint{1.018484in}{2.202760in}}{\pgfqpoint{1.018484in}{2.210997in}}%
\pgfpathcurveto{\pgfqpoint{1.018484in}{2.219233in}}{\pgfqpoint{1.015212in}{2.227133in}}{\pgfqpoint{1.009388in}{2.232957in}}%
\pgfpathcurveto{\pgfqpoint{1.003564in}{2.238781in}}{\pgfqpoint{0.995664in}{2.242053in}}{\pgfqpoint{0.987428in}{2.242053in}}%
\pgfpathcurveto{\pgfqpoint{0.979191in}{2.242053in}}{\pgfqpoint{0.971291in}{2.238781in}}{\pgfqpoint{0.965467in}{2.232957in}}%
\pgfpathcurveto{\pgfqpoint{0.959643in}{2.227133in}}{\pgfqpoint{0.956371in}{2.219233in}}{\pgfqpoint{0.956371in}{2.210997in}}%
\pgfpathcurveto{\pgfqpoint{0.956371in}{2.202760in}}{\pgfqpoint{0.959643in}{2.194860in}}{\pgfqpoint{0.965467in}{2.189036in}}%
\pgfpathcurveto{\pgfqpoint{0.971291in}{2.183212in}}{\pgfqpoint{0.979191in}{2.179940in}}{\pgfqpoint{0.987428in}{2.179940in}}%
\pgfpathclose%
\pgfusepath{stroke,fill}%
\end{pgfscope}%
\begin{pgfscope}%
\pgfpathrectangle{\pgfqpoint{0.100000in}{0.212622in}}{\pgfqpoint{3.696000in}{3.696000in}}%
\pgfusepath{clip}%
\pgfsetbuttcap%
\pgfsetroundjoin%
\definecolor{currentfill}{rgb}{0.121569,0.466667,0.705882}%
\pgfsetfillcolor{currentfill}%
\pgfsetfillopacity{0.858993}%
\pgfsetlinewidth{1.003750pt}%
\definecolor{currentstroke}{rgb}{0.121569,0.466667,0.705882}%
\pgfsetstrokecolor{currentstroke}%
\pgfsetstrokeopacity{0.858993}%
\pgfsetdash{}{0pt}%
\pgfpathmoveto{\pgfqpoint{1.840015in}{2.724181in}}%
\pgfpathcurveto{\pgfqpoint{1.848251in}{2.724181in}}{\pgfqpoint{1.856151in}{2.727453in}}{\pgfqpoint{1.861975in}{2.733277in}}%
\pgfpathcurveto{\pgfqpoint{1.867799in}{2.739101in}}{\pgfqpoint{1.871072in}{2.747001in}}{\pgfqpoint{1.871072in}{2.755238in}}%
\pgfpathcurveto{\pgfqpoint{1.871072in}{2.763474in}}{\pgfqpoint{1.867799in}{2.771374in}}{\pgfqpoint{1.861975in}{2.777198in}}%
\pgfpathcurveto{\pgfqpoint{1.856151in}{2.783022in}}{\pgfqpoint{1.848251in}{2.786294in}}{\pgfqpoint{1.840015in}{2.786294in}}%
\pgfpathcurveto{\pgfqpoint{1.831779in}{2.786294in}}{\pgfqpoint{1.823879in}{2.783022in}}{\pgfqpoint{1.818055in}{2.777198in}}%
\pgfpathcurveto{\pgfqpoint{1.812231in}{2.771374in}}{\pgfqpoint{1.808959in}{2.763474in}}{\pgfqpoint{1.808959in}{2.755238in}}%
\pgfpathcurveto{\pgfqpoint{1.808959in}{2.747001in}}{\pgfqpoint{1.812231in}{2.739101in}}{\pgfqpoint{1.818055in}{2.733277in}}%
\pgfpathcurveto{\pgfqpoint{1.823879in}{2.727453in}}{\pgfqpoint{1.831779in}{2.724181in}}{\pgfqpoint{1.840015in}{2.724181in}}%
\pgfpathclose%
\pgfusepath{stroke,fill}%
\end{pgfscope}%
\begin{pgfscope}%
\pgfpathrectangle{\pgfqpoint{0.100000in}{0.212622in}}{\pgfqpoint{3.696000in}{3.696000in}}%
\pgfusepath{clip}%
\pgfsetbuttcap%
\pgfsetroundjoin%
\definecolor{currentfill}{rgb}{0.121569,0.466667,0.705882}%
\pgfsetfillcolor{currentfill}%
\pgfsetfillopacity{0.859590}%
\pgfsetlinewidth{1.003750pt}%
\definecolor{currentstroke}{rgb}{0.121569,0.466667,0.705882}%
\pgfsetstrokecolor{currentstroke}%
\pgfsetstrokeopacity{0.859590}%
\pgfsetdash{}{0pt}%
\pgfpathmoveto{\pgfqpoint{2.283392in}{2.591937in}}%
\pgfpathcurveto{\pgfqpoint{2.291628in}{2.591937in}}{\pgfqpoint{2.299528in}{2.595209in}}{\pgfqpoint{2.305352in}{2.601033in}}%
\pgfpathcurveto{\pgfqpoint{2.311176in}{2.606857in}}{\pgfqpoint{2.314448in}{2.614757in}}{\pgfqpoint{2.314448in}{2.622993in}}%
\pgfpathcurveto{\pgfqpoint{2.314448in}{2.631229in}}{\pgfqpoint{2.311176in}{2.639130in}}{\pgfqpoint{2.305352in}{2.644953in}}%
\pgfpathcurveto{\pgfqpoint{2.299528in}{2.650777in}}{\pgfqpoint{2.291628in}{2.654050in}}{\pgfqpoint{2.283392in}{2.654050in}}%
\pgfpathcurveto{\pgfqpoint{2.275156in}{2.654050in}}{\pgfqpoint{2.267256in}{2.650777in}}{\pgfqpoint{2.261432in}{2.644953in}}%
\pgfpathcurveto{\pgfqpoint{2.255608in}{2.639130in}}{\pgfqpoint{2.252335in}{2.631229in}}{\pgfqpoint{2.252335in}{2.622993in}}%
\pgfpathcurveto{\pgfqpoint{2.252335in}{2.614757in}}{\pgfqpoint{2.255608in}{2.606857in}}{\pgfqpoint{2.261432in}{2.601033in}}%
\pgfpathcurveto{\pgfqpoint{2.267256in}{2.595209in}}{\pgfqpoint{2.275156in}{2.591937in}}{\pgfqpoint{2.283392in}{2.591937in}}%
\pgfpathclose%
\pgfusepath{stroke,fill}%
\end{pgfscope}%
\begin{pgfscope}%
\pgfpathrectangle{\pgfqpoint{0.100000in}{0.212622in}}{\pgfqpoint{3.696000in}{3.696000in}}%
\pgfusepath{clip}%
\pgfsetbuttcap%
\pgfsetroundjoin%
\definecolor{currentfill}{rgb}{0.121569,0.466667,0.705882}%
\pgfsetfillcolor{currentfill}%
\pgfsetfillopacity{0.859916}%
\pgfsetlinewidth{1.003750pt}%
\definecolor{currentstroke}{rgb}{0.121569,0.466667,0.705882}%
\pgfsetstrokecolor{currentstroke}%
\pgfsetstrokeopacity{0.859916}%
\pgfsetdash{}{0pt}%
\pgfpathmoveto{\pgfqpoint{1.838297in}{2.722076in}}%
\pgfpathcurveto{\pgfqpoint{1.846533in}{2.722076in}}{\pgfqpoint{1.854434in}{2.725349in}}{\pgfqpoint{1.860257in}{2.731172in}}%
\pgfpathcurveto{\pgfqpoint{1.866081in}{2.736996in}}{\pgfqpoint{1.869354in}{2.744896in}}{\pgfqpoint{1.869354in}{2.753133in}}%
\pgfpathcurveto{\pgfqpoint{1.869354in}{2.761369in}}{\pgfqpoint{1.866081in}{2.769269in}}{\pgfqpoint{1.860257in}{2.775093in}}%
\pgfpathcurveto{\pgfqpoint{1.854434in}{2.780917in}}{\pgfqpoint{1.846533in}{2.784189in}}{\pgfqpoint{1.838297in}{2.784189in}}%
\pgfpathcurveto{\pgfqpoint{1.830061in}{2.784189in}}{\pgfqpoint{1.822161in}{2.780917in}}{\pgfqpoint{1.816337in}{2.775093in}}%
\pgfpathcurveto{\pgfqpoint{1.810513in}{2.769269in}}{\pgfqpoint{1.807241in}{2.761369in}}{\pgfqpoint{1.807241in}{2.753133in}}%
\pgfpathcurveto{\pgfqpoint{1.807241in}{2.744896in}}{\pgfqpoint{1.810513in}{2.736996in}}{\pgfqpoint{1.816337in}{2.731172in}}%
\pgfpathcurveto{\pgfqpoint{1.822161in}{2.725349in}}{\pgfqpoint{1.830061in}{2.722076in}}{\pgfqpoint{1.838297in}{2.722076in}}%
\pgfpathclose%
\pgfusepath{stroke,fill}%
\end{pgfscope}%
\begin{pgfscope}%
\pgfpathrectangle{\pgfqpoint{0.100000in}{0.212622in}}{\pgfqpoint{3.696000in}{3.696000in}}%
\pgfusepath{clip}%
\pgfsetbuttcap%
\pgfsetroundjoin%
\definecolor{currentfill}{rgb}{0.121569,0.466667,0.705882}%
\pgfsetfillcolor{currentfill}%
\pgfsetfillopacity{0.860179}%
\pgfsetlinewidth{1.003750pt}%
\definecolor{currentstroke}{rgb}{0.121569,0.466667,0.705882}%
\pgfsetstrokecolor{currentstroke}%
\pgfsetstrokeopacity{0.860179}%
\pgfsetdash{}{0pt}%
\pgfpathmoveto{\pgfqpoint{2.281967in}{2.590338in}}%
\pgfpathcurveto{\pgfqpoint{2.290203in}{2.590338in}}{\pgfqpoint{2.298103in}{2.593611in}}{\pgfqpoint{2.303927in}{2.599434in}}%
\pgfpathcurveto{\pgfqpoint{2.309751in}{2.605258in}}{\pgfqpoint{2.313024in}{2.613158in}}{\pgfqpoint{2.313024in}{2.621395in}}%
\pgfpathcurveto{\pgfqpoint{2.313024in}{2.629631in}}{\pgfqpoint{2.309751in}{2.637531in}}{\pgfqpoint{2.303927in}{2.643355in}}%
\pgfpathcurveto{\pgfqpoint{2.298103in}{2.649179in}}{\pgfqpoint{2.290203in}{2.652451in}}{\pgfqpoint{2.281967in}{2.652451in}}%
\pgfpathcurveto{\pgfqpoint{2.273731in}{2.652451in}}{\pgfqpoint{2.265831in}{2.649179in}}{\pgfqpoint{2.260007in}{2.643355in}}%
\pgfpathcurveto{\pgfqpoint{2.254183in}{2.637531in}}{\pgfqpoint{2.250911in}{2.629631in}}{\pgfqpoint{2.250911in}{2.621395in}}%
\pgfpathcurveto{\pgfqpoint{2.250911in}{2.613158in}}{\pgfqpoint{2.254183in}{2.605258in}}{\pgfqpoint{2.260007in}{2.599434in}}%
\pgfpathcurveto{\pgfqpoint{2.265831in}{2.593611in}}{\pgfqpoint{2.273731in}{2.590338in}}{\pgfqpoint{2.281967in}{2.590338in}}%
\pgfpathclose%
\pgfusepath{stroke,fill}%
\end{pgfscope}%
\begin{pgfscope}%
\pgfpathrectangle{\pgfqpoint{0.100000in}{0.212622in}}{\pgfqpoint{3.696000in}{3.696000in}}%
\pgfusepath{clip}%
\pgfsetbuttcap%
\pgfsetroundjoin%
\definecolor{currentfill}{rgb}{0.121569,0.466667,0.705882}%
\pgfsetfillcolor{currentfill}%
\pgfsetfillopacity{0.860635}%
\pgfsetlinewidth{1.003750pt}%
\definecolor{currentstroke}{rgb}{0.121569,0.466667,0.705882}%
\pgfsetstrokecolor{currentstroke}%
\pgfsetstrokeopacity{0.860635}%
\pgfsetdash{}{0pt}%
\pgfpathmoveto{\pgfqpoint{0.982684in}{2.174871in}}%
\pgfpathcurveto{\pgfqpoint{0.990920in}{2.174871in}}{\pgfqpoint{0.998821in}{2.178143in}}{\pgfqpoint{1.004644in}{2.183967in}}%
\pgfpathcurveto{\pgfqpoint{1.010468in}{2.189791in}}{\pgfqpoint{1.013741in}{2.197691in}}{\pgfqpoint{1.013741in}{2.205927in}}%
\pgfpathcurveto{\pgfqpoint{1.013741in}{2.214164in}}{\pgfqpoint{1.010468in}{2.222064in}}{\pgfqpoint{1.004644in}{2.227888in}}%
\pgfpathcurveto{\pgfqpoint{0.998821in}{2.233711in}}{\pgfqpoint{0.990920in}{2.236984in}}{\pgfqpoint{0.982684in}{2.236984in}}%
\pgfpathcurveto{\pgfqpoint{0.974448in}{2.236984in}}{\pgfqpoint{0.966548in}{2.233711in}}{\pgfqpoint{0.960724in}{2.227888in}}%
\pgfpathcurveto{\pgfqpoint{0.954900in}{2.222064in}}{\pgfqpoint{0.951628in}{2.214164in}}{\pgfqpoint{0.951628in}{2.205927in}}%
\pgfpathcurveto{\pgfqpoint{0.951628in}{2.197691in}}{\pgfqpoint{0.954900in}{2.189791in}}{\pgfqpoint{0.960724in}{2.183967in}}%
\pgfpathcurveto{\pgfqpoint{0.966548in}{2.178143in}}{\pgfqpoint{0.974448in}{2.174871in}}{\pgfqpoint{0.982684in}{2.174871in}}%
\pgfpathclose%
\pgfusepath{stroke,fill}%
\end{pgfscope}%
\begin{pgfscope}%
\pgfpathrectangle{\pgfqpoint{0.100000in}{0.212622in}}{\pgfqpoint{3.696000in}{3.696000in}}%
\pgfusepath{clip}%
\pgfsetbuttcap%
\pgfsetroundjoin%
\definecolor{currentfill}{rgb}{0.121569,0.466667,0.705882}%
\pgfsetfillcolor{currentfill}%
\pgfsetfillopacity{0.860942}%
\pgfsetlinewidth{1.003750pt}%
\definecolor{currentstroke}{rgb}{0.121569,0.466667,0.705882}%
\pgfsetstrokecolor{currentstroke}%
\pgfsetstrokeopacity{0.860942}%
\pgfsetdash{}{0pt}%
\pgfpathmoveto{\pgfqpoint{1.836166in}{2.719458in}}%
\pgfpathcurveto{\pgfqpoint{1.844403in}{2.719458in}}{\pgfqpoint{1.852303in}{2.722730in}}{\pgfqpoint{1.858127in}{2.728554in}}%
\pgfpathcurveto{\pgfqpoint{1.863951in}{2.734378in}}{\pgfqpoint{1.867223in}{2.742278in}}{\pgfqpoint{1.867223in}{2.750514in}}%
\pgfpathcurveto{\pgfqpoint{1.867223in}{2.758751in}}{\pgfqpoint{1.863951in}{2.766651in}}{\pgfqpoint{1.858127in}{2.772475in}}%
\pgfpathcurveto{\pgfqpoint{1.852303in}{2.778298in}}{\pgfqpoint{1.844403in}{2.781571in}}{\pgfqpoint{1.836166in}{2.781571in}}%
\pgfpathcurveto{\pgfqpoint{1.827930in}{2.781571in}}{\pgfqpoint{1.820030in}{2.778298in}}{\pgfqpoint{1.814206in}{2.772475in}}%
\pgfpathcurveto{\pgfqpoint{1.808382in}{2.766651in}}{\pgfqpoint{1.805110in}{2.758751in}}{\pgfqpoint{1.805110in}{2.750514in}}%
\pgfpathcurveto{\pgfqpoint{1.805110in}{2.742278in}}{\pgfqpoint{1.808382in}{2.734378in}}{\pgfqpoint{1.814206in}{2.728554in}}%
\pgfpathcurveto{\pgfqpoint{1.820030in}{2.722730in}}{\pgfqpoint{1.827930in}{2.719458in}}{\pgfqpoint{1.836166in}{2.719458in}}%
\pgfpathclose%
\pgfusepath{stroke,fill}%
\end{pgfscope}%
\begin{pgfscope}%
\pgfpathrectangle{\pgfqpoint{0.100000in}{0.212622in}}{\pgfqpoint{3.696000in}{3.696000in}}%
\pgfusepath{clip}%
\pgfsetbuttcap%
\pgfsetroundjoin%
\definecolor{currentfill}{rgb}{0.121569,0.466667,0.705882}%
\pgfsetfillcolor{currentfill}%
\pgfsetfillopacity{0.861260}%
\pgfsetlinewidth{1.003750pt}%
\definecolor{currentstroke}{rgb}{0.121569,0.466667,0.705882}%
\pgfsetstrokecolor{currentstroke}%
\pgfsetstrokeopacity{0.861260}%
\pgfsetdash{}{0pt}%
\pgfpathmoveto{\pgfqpoint{2.279436in}{2.587430in}}%
\pgfpathcurveto{\pgfqpoint{2.287673in}{2.587430in}}{\pgfqpoint{2.295573in}{2.590702in}}{\pgfqpoint{2.301397in}{2.596526in}}%
\pgfpathcurveto{\pgfqpoint{2.307221in}{2.602350in}}{\pgfqpoint{2.310493in}{2.610250in}}{\pgfqpoint{2.310493in}{2.618487in}}%
\pgfpathcurveto{\pgfqpoint{2.310493in}{2.626723in}}{\pgfqpoint{2.307221in}{2.634623in}}{\pgfqpoint{2.301397in}{2.640447in}}%
\pgfpathcurveto{\pgfqpoint{2.295573in}{2.646271in}}{\pgfqpoint{2.287673in}{2.649543in}}{\pgfqpoint{2.279436in}{2.649543in}}%
\pgfpathcurveto{\pgfqpoint{2.271200in}{2.649543in}}{\pgfqpoint{2.263300in}{2.646271in}}{\pgfqpoint{2.257476in}{2.640447in}}%
\pgfpathcurveto{\pgfqpoint{2.251652in}{2.634623in}}{\pgfqpoint{2.248380in}{2.626723in}}{\pgfqpoint{2.248380in}{2.618487in}}%
\pgfpathcurveto{\pgfqpoint{2.248380in}{2.610250in}}{\pgfqpoint{2.251652in}{2.602350in}}{\pgfqpoint{2.257476in}{2.596526in}}%
\pgfpathcurveto{\pgfqpoint{2.263300in}{2.590702in}}{\pgfqpoint{2.271200in}{2.587430in}}{\pgfqpoint{2.279436in}{2.587430in}}%
\pgfpathclose%
\pgfusepath{stroke,fill}%
\end{pgfscope}%
\begin{pgfscope}%
\pgfpathrectangle{\pgfqpoint{0.100000in}{0.212622in}}{\pgfqpoint{3.696000in}{3.696000in}}%
\pgfusepath{clip}%
\pgfsetbuttcap%
\pgfsetroundjoin%
\definecolor{currentfill}{rgb}{0.121569,0.466667,0.705882}%
\pgfsetfillcolor{currentfill}%
\pgfsetfillopacity{0.861493}%
\pgfsetlinewidth{1.003750pt}%
\definecolor{currentstroke}{rgb}{0.121569,0.466667,0.705882}%
\pgfsetstrokecolor{currentstroke}%
\pgfsetstrokeopacity{0.861493}%
\pgfsetdash{}{0pt}%
\pgfpathmoveto{\pgfqpoint{1.834990in}{2.717943in}}%
\pgfpathcurveto{\pgfqpoint{1.843226in}{2.717943in}}{\pgfqpoint{1.851126in}{2.721215in}}{\pgfqpoint{1.856950in}{2.727039in}}%
\pgfpathcurveto{\pgfqpoint{1.862774in}{2.732863in}}{\pgfqpoint{1.866046in}{2.740763in}}{\pgfqpoint{1.866046in}{2.748999in}}%
\pgfpathcurveto{\pgfqpoint{1.866046in}{2.757236in}}{\pgfqpoint{1.862774in}{2.765136in}}{\pgfqpoint{1.856950in}{2.770960in}}%
\pgfpathcurveto{\pgfqpoint{1.851126in}{2.776783in}}{\pgfqpoint{1.843226in}{2.780056in}}{\pgfqpoint{1.834990in}{2.780056in}}%
\pgfpathcurveto{\pgfqpoint{1.826753in}{2.780056in}}{\pgfqpoint{1.818853in}{2.776783in}}{\pgfqpoint{1.813029in}{2.770960in}}%
\pgfpathcurveto{\pgfqpoint{1.807205in}{2.765136in}}{\pgfqpoint{1.803933in}{2.757236in}}{\pgfqpoint{1.803933in}{2.748999in}}%
\pgfpathcurveto{\pgfqpoint{1.803933in}{2.740763in}}{\pgfqpoint{1.807205in}{2.732863in}}{\pgfqpoint{1.813029in}{2.727039in}}%
\pgfpathcurveto{\pgfqpoint{1.818853in}{2.721215in}}{\pgfqpoint{1.826753in}{2.717943in}}{\pgfqpoint{1.834990in}{2.717943in}}%
\pgfpathclose%
\pgfusepath{stroke,fill}%
\end{pgfscope}%
\begin{pgfscope}%
\pgfpathrectangle{\pgfqpoint{0.100000in}{0.212622in}}{\pgfqpoint{3.696000in}{3.696000in}}%
\pgfusepath{clip}%
\pgfsetbuttcap%
\pgfsetroundjoin%
\definecolor{currentfill}{rgb}{0.121569,0.466667,0.705882}%
\pgfsetfillcolor{currentfill}%
\pgfsetfillopacity{0.861633}%
\pgfsetlinewidth{1.003750pt}%
\definecolor{currentstroke}{rgb}{0.121569,0.466667,0.705882}%
\pgfsetstrokecolor{currentstroke}%
\pgfsetstrokeopacity{0.861633}%
\pgfsetdash{}{0pt}%
\pgfpathmoveto{\pgfqpoint{0.979972in}{2.172125in}}%
\pgfpathcurveto{\pgfqpoint{0.988208in}{2.172125in}}{\pgfqpoint{0.996109in}{2.175397in}}{\pgfqpoint{1.001932in}{2.181221in}}%
\pgfpathcurveto{\pgfqpoint{1.007756in}{2.187045in}}{\pgfqpoint{1.011029in}{2.194945in}}{\pgfqpoint{1.011029in}{2.203182in}}%
\pgfpathcurveto{\pgfqpoint{1.011029in}{2.211418in}}{\pgfqpoint{1.007756in}{2.219318in}}{\pgfqpoint{1.001932in}{2.225142in}}%
\pgfpathcurveto{\pgfqpoint{0.996109in}{2.230966in}}{\pgfqpoint{0.988208in}{2.234238in}}{\pgfqpoint{0.979972in}{2.234238in}}%
\pgfpathcurveto{\pgfqpoint{0.971736in}{2.234238in}}{\pgfqpoint{0.963836in}{2.230966in}}{\pgfqpoint{0.958012in}{2.225142in}}%
\pgfpathcurveto{\pgfqpoint{0.952188in}{2.219318in}}{\pgfqpoint{0.948916in}{2.211418in}}{\pgfqpoint{0.948916in}{2.203182in}}%
\pgfpathcurveto{\pgfqpoint{0.948916in}{2.194945in}}{\pgfqpoint{0.952188in}{2.187045in}}{\pgfqpoint{0.958012in}{2.181221in}}%
\pgfpathcurveto{\pgfqpoint{0.963836in}{2.175397in}}{\pgfqpoint{0.971736in}{2.172125in}}{\pgfqpoint{0.979972in}{2.172125in}}%
\pgfpathclose%
\pgfusepath{stroke,fill}%
\end{pgfscope}%
\begin{pgfscope}%
\pgfpathrectangle{\pgfqpoint{0.100000in}{0.212622in}}{\pgfqpoint{3.696000in}{3.696000in}}%
\pgfusepath{clip}%
\pgfsetbuttcap%
\pgfsetroundjoin%
\definecolor{currentfill}{rgb}{0.121569,0.466667,0.705882}%
\pgfsetfillcolor{currentfill}%
\pgfsetfillopacity{0.861796}%
\pgfsetlinewidth{1.003750pt}%
\definecolor{currentstroke}{rgb}{0.121569,0.466667,0.705882}%
\pgfsetstrokecolor{currentstroke}%
\pgfsetstrokeopacity{0.861796}%
\pgfsetdash{}{0pt}%
\pgfpathmoveto{\pgfqpoint{1.834303in}{2.717138in}}%
\pgfpathcurveto{\pgfqpoint{1.842540in}{2.717138in}}{\pgfqpoint{1.850440in}{2.720410in}}{\pgfqpoint{1.856264in}{2.726234in}}%
\pgfpathcurveto{\pgfqpoint{1.862088in}{2.732058in}}{\pgfqpoint{1.865360in}{2.739958in}}{\pgfqpoint{1.865360in}{2.748194in}}%
\pgfpathcurveto{\pgfqpoint{1.865360in}{2.756430in}}{\pgfqpoint{1.862088in}{2.764330in}}{\pgfqpoint{1.856264in}{2.770154in}}%
\pgfpathcurveto{\pgfqpoint{1.850440in}{2.775978in}}{\pgfqpoint{1.842540in}{2.779251in}}{\pgfqpoint{1.834303in}{2.779251in}}%
\pgfpathcurveto{\pgfqpoint{1.826067in}{2.779251in}}{\pgfqpoint{1.818167in}{2.775978in}}{\pgfqpoint{1.812343in}{2.770154in}}%
\pgfpathcurveto{\pgfqpoint{1.806519in}{2.764330in}}{\pgfqpoint{1.803247in}{2.756430in}}{\pgfqpoint{1.803247in}{2.748194in}}%
\pgfpathcurveto{\pgfqpoint{1.803247in}{2.739958in}}{\pgfqpoint{1.806519in}{2.732058in}}{\pgfqpoint{1.812343in}{2.726234in}}%
\pgfpathcurveto{\pgfqpoint{1.818167in}{2.720410in}}{\pgfqpoint{1.826067in}{2.717138in}}{\pgfqpoint{1.834303in}{2.717138in}}%
\pgfpathclose%
\pgfusepath{stroke,fill}%
\end{pgfscope}%
\begin{pgfscope}%
\pgfpathrectangle{\pgfqpoint{0.100000in}{0.212622in}}{\pgfqpoint{3.696000in}{3.696000in}}%
\pgfusepath{clip}%
\pgfsetbuttcap%
\pgfsetroundjoin%
\definecolor{currentfill}{rgb}{0.121569,0.466667,0.705882}%
\pgfsetfillcolor{currentfill}%
\pgfsetfillopacity{0.861979}%
\pgfsetlinewidth{1.003750pt}%
\definecolor{currentstroke}{rgb}{0.121569,0.466667,0.705882}%
\pgfsetstrokecolor{currentstroke}%
\pgfsetstrokeopacity{0.861979}%
\pgfsetdash{}{0pt}%
\pgfpathmoveto{\pgfqpoint{2.277675in}{2.585188in}}%
\pgfpathcurveto{\pgfqpoint{2.285911in}{2.585188in}}{\pgfqpoint{2.293811in}{2.588460in}}{\pgfqpoint{2.299635in}{2.594284in}}%
\pgfpathcurveto{\pgfqpoint{2.305459in}{2.600108in}}{\pgfqpoint{2.308732in}{2.608008in}}{\pgfqpoint{2.308732in}{2.616244in}}%
\pgfpathcurveto{\pgfqpoint{2.308732in}{2.624481in}}{\pgfqpoint{2.305459in}{2.632381in}}{\pgfqpoint{2.299635in}{2.638205in}}%
\pgfpathcurveto{\pgfqpoint{2.293811in}{2.644028in}}{\pgfqpoint{2.285911in}{2.647301in}}{\pgfqpoint{2.277675in}{2.647301in}}%
\pgfpathcurveto{\pgfqpoint{2.269439in}{2.647301in}}{\pgfqpoint{2.261539in}{2.644028in}}{\pgfqpoint{2.255715in}{2.638205in}}%
\pgfpathcurveto{\pgfqpoint{2.249891in}{2.632381in}}{\pgfqpoint{2.246619in}{2.624481in}}{\pgfqpoint{2.246619in}{2.616244in}}%
\pgfpathcurveto{\pgfqpoint{2.246619in}{2.608008in}}{\pgfqpoint{2.249891in}{2.600108in}}{\pgfqpoint{2.255715in}{2.594284in}}%
\pgfpathcurveto{\pgfqpoint{2.261539in}{2.588460in}}{\pgfqpoint{2.269439in}{2.585188in}}{\pgfqpoint{2.277675in}{2.585188in}}%
\pgfpathclose%
\pgfusepath{stroke,fill}%
\end{pgfscope}%
\begin{pgfscope}%
\pgfpathrectangle{\pgfqpoint{0.100000in}{0.212622in}}{\pgfqpoint{3.696000in}{3.696000in}}%
\pgfusepath{clip}%
\pgfsetbuttcap%
\pgfsetroundjoin%
\definecolor{currentfill}{rgb}{0.121569,0.466667,0.705882}%
\pgfsetfillcolor{currentfill}%
\pgfsetfillopacity{0.862358}%
\pgfsetlinewidth{1.003750pt}%
\definecolor{currentstroke}{rgb}{0.121569,0.466667,0.705882}%
\pgfsetstrokecolor{currentstroke}%
\pgfsetstrokeopacity{0.862358}%
\pgfsetdash{}{0pt}%
\pgfpathmoveto{\pgfqpoint{2.752565in}{1.401440in}}%
\pgfpathcurveto{\pgfqpoint{2.760801in}{1.401440in}}{\pgfqpoint{2.768702in}{1.404712in}}{\pgfqpoint{2.774525in}{1.410536in}}%
\pgfpathcurveto{\pgfqpoint{2.780349in}{1.416360in}}{\pgfqpoint{2.783622in}{1.424260in}}{\pgfqpoint{2.783622in}{1.432496in}}%
\pgfpathcurveto{\pgfqpoint{2.783622in}{1.440733in}}{\pgfqpoint{2.780349in}{1.448633in}}{\pgfqpoint{2.774525in}{1.454457in}}%
\pgfpathcurveto{\pgfqpoint{2.768702in}{1.460281in}}{\pgfqpoint{2.760801in}{1.463553in}}{\pgfqpoint{2.752565in}{1.463553in}}%
\pgfpathcurveto{\pgfqpoint{2.744329in}{1.463553in}}{\pgfqpoint{2.736429in}{1.460281in}}{\pgfqpoint{2.730605in}{1.454457in}}%
\pgfpathcurveto{\pgfqpoint{2.724781in}{1.448633in}}{\pgfqpoint{2.721509in}{1.440733in}}{\pgfqpoint{2.721509in}{1.432496in}}%
\pgfpathcurveto{\pgfqpoint{2.721509in}{1.424260in}}{\pgfqpoint{2.724781in}{1.416360in}}{\pgfqpoint{2.730605in}{1.410536in}}%
\pgfpathcurveto{\pgfqpoint{2.736429in}{1.404712in}}{\pgfqpoint{2.744329in}{1.401440in}}{\pgfqpoint{2.752565in}{1.401440in}}%
\pgfpathclose%
\pgfusepath{stroke,fill}%
\end{pgfscope}%
\begin{pgfscope}%
\pgfpathrectangle{\pgfqpoint{0.100000in}{0.212622in}}{\pgfqpoint{3.696000in}{3.696000in}}%
\pgfusepath{clip}%
\pgfsetbuttcap%
\pgfsetroundjoin%
\definecolor{currentfill}{rgb}{0.121569,0.466667,0.705882}%
\pgfsetfillcolor{currentfill}%
\pgfsetfillopacity{0.862384}%
\pgfsetlinewidth{1.003750pt}%
\definecolor{currentstroke}{rgb}{0.121569,0.466667,0.705882}%
\pgfsetstrokecolor{currentstroke}%
\pgfsetstrokeopacity{0.862384}%
\pgfsetdash{}{0pt}%
\pgfpathmoveto{\pgfqpoint{1.832939in}{2.715157in}}%
\pgfpathcurveto{\pgfqpoint{1.841175in}{2.715157in}}{\pgfqpoint{1.849075in}{2.718429in}}{\pgfqpoint{1.854899in}{2.724253in}}%
\pgfpathcurveto{\pgfqpoint{1.860723in}{2.730077in}}{\pgfqpoint{1.863995in}{2.737977in}}{\pgfqpoint{1.863995in}{2.746213in}}%
\pgfpathcurveto{\pgfqpoint{1.863995in}{2.754449in}}{\pgfqpoint{1.860723in}{2.762349in}}{\pgfqpoint{1.854899in}{2.768173in}}%
\pgfpathcurveto{\pgfqpoint{1.849075in}{2.773997in}}{\pgfqpoint{1.841175in}{2.777270in}}{\pgfqpoint{1.832939in}{2.777270in}}%
\pgfpathcurveto{\pgfqpoint{1.824702in}{2.777270in}}{\pgfqpoint{1.816802in}{2.773997in}}{\pgfqpoint{1.810978in}{2.768173in}}%
\pgfpathcurveto{\pgfqpoint{1.805154in}{2.762349in}}{\pgfqpoint{1.801882in}{2.754449in}}{\pgfqpoint{1.801882in}{2.746213in}}%
\pgfpathcurveto{\pgfqpoint{1.801882in}{2.737977in}}{\pgfqpoint{1.805154in}{2.730077in}}{\pgfqpoint{1.810978in}{2.724253in}}%
\pgfpathcurveto{\pgfqpoint{1.816802in}{2.718429in}}{\pgfqpoint{1.824702in}{2.715157in}}{\pgfqpoint{1.832939in}{2.715157in}}%
\pgfpathclose%
\pgfusepath{stroke,fill}%
\end{pgfscope}%
\begin{pgfscope}%
\pgfpathrectangle{\pgfqpoint{0.100000in}{0.212622in}}{\pgfqpoint{3.696000in}{3.696000in}}%
\pgfusepath{clip}%
\pgfsetbuttcap%
\pgfsetroundjoin%
\definecolor{currentfill}{rgb}{0.121569,0.466667,0.705882}%
\pgfsetfillcolor{currentfill}%
\pgfsetfillopacity{0.862712}%
\pgfsetlinewidth{1.003750pt}%
\definecolor{currentstroke}{rgb}{0.121569,0.466667,0.705882}%
\pgfsetstrokecolor{currentstroke}%
\pgfsetstrokeopacity{0.862712}%
\pgfsetdash{}{0pt}%
\pgfpathmoveto{\pgfqpoint{1.832194in}{2.714089in}}%
\pgfpathcurveto{\pgfqpoint{1.840430in}{2.714089in}}{\pgfqpoint{1.848330in}{2.717361in}}{\pgfqpoint{1.854154in}{2.723185in}}%
\pgfpathcurveto{\pgfqpoint{1.859978in}{2.729009in}}{\pgfqpoint{1.863250in}{2.736909in}}{\pgfqpoint{1.863250in}{2.745146in}}%
\pgfpathcurveto{\pgfqpoint{1.863250in}{2.753382in}}{\pgfqpoint{1.859978in}{2.761282in}}{\pgfqpoint{1.854154in}{2.767106in}}%
\pgfpathcurveto{\pgfqpoint{1.848330in}{2.772930in}}{\pgfqpoint{1.840430in}{2.776202in}}{\pgfqpoint{1.832194in}{2.776202in}}%
\pgfpathcurveto{\pgfqpoint{1.823957in}{2.776202in}}{\pgfqpoint{1.816057in}{2.772930in}}{\pgfqpoint{1.810233in}{2.767106in}}%
\pgfpathcurveto{\pgfqpoint{1.804409in}{2.761282in}}{\pgfqpoint{1.801137in}{2.753382in}}{\pgfqpoint{1.801137in}{2.745146in}}%
\pgfpathcurveto{\pgfqpoint{1.801137in}{2.736909in}}{\pgfqpoint{1.804409in}{2.729009in}}{\pgfqpoint{1.810233in}{2.723185in}}%
\pgfpathcurveto{\pgfqpoint{1.816057in}{2.717361in}}{\pgfqpoint{1.823957in}{2.714089in}}{\pgfqpoint{1.832194in}{2.714089in}}%
\pgfpathclose%
\pgfusepath{stroke,fill}%
\end{pgfscope}%
\begin{pgfscope}%
\pgfpathrectangle{\pgfqpoint{0.100000in}{0.212622in}}{\pgfqpoint{3.696000in}{3.696000in}}%
\pgfusepath{clip}%
\pgfsetbuttcap%
\pgfsetroundjoin%
\definecolor{currentfill}{rgb}{0.121569,0.466667,0.705882}%
\pgfsetfillcolor{currentfill}%
\pgfsetfillopacity{0.862907}%
\pgfsetlinewidth{1.003750pt}%
\definecolor{currentstroke}{rgb}{0.121569,0.466667,0.705882}%
\pgfsetstrokecolor{currentstroke}%
\pgfsetstrokeopacity{0.862907}%
\pgfsetdash{}{0pt}%
\pgfpathmoveto{\pgfqpoint{1.831771in}{2.713602in}}%
\pgfpathcurveto{\pgfqpoint{1.840007in}{2.713602in}}{\pgfqpoint{1.847907in}{2.716875in}}{\pgfqpoint{1.853731in}{2.722699in}}%
\pgfpathcurveto{\pgfqpoint{1.859555in}{2.728523in}}{\pgfqpoint{1.862827in}{2.736423in}}{\pgfqpoint{1.862827in}{2.744659in}}%
\pgfpathcurveto{\pgfqpoint{1.862827in}{2.752895in}}{\pgfqpoint{1.859555in}{2.760795in}}{\pgfqpoint{1.853731in}{2.766619in}}%
\pgfpathcurveto{\pgfqpoint{1.847907in}{2.772443in}}{\pgfqpoint{1.840007in}{2.775715in}}{\pgfqpoint{1.831771in}{2.775715in}}%
\pgfpathcurveto{\pgfqpoint{1.823534in}{2.775715in}}{\pgfqpoint{1.815634in}{2.772443in}}{\pgfqpoint{1.809811in}{2.766619in}}%
\pgfpathcurveto{\pgfqpoint{1.803987in}{2.760795in}}{\pgfqpoint{1.800714in}{2.752895in}}{\pgfqpoint{1.800714in}{2.744659in}}%
\pgfpathcurveto{\pgfqpoint{1.800714in}{2.736423in}}{\pgfqpoint{1.803987in}{2.728523in}}{\pgfqpoint{1.809811in}{2.722699in}}%
\pgfpathcurveto{\pgfqpoint{1.815634in}{2.716875in}}{\pgfqpoint{1.823534in}{2.713602in}}{\pgfqpoint{1.831771in}{2.713602in}}%
\pgfpathclose%
\pgfusepath{stroke,fill}%
\end{pgfscope}%
\begin{pgfscope}%
\pgfpathrectangle{\pgfqpoint{0.100000in}{0.212622in}}{\pgfqpoint{3.696000in}{3.696000in}}%
\pgfusepath{clip}%
\pgfsetbuttcap%
\pgfsetroundjoin%
\definecolor{currentfill}{rgb}{0.121569,0.466667,0.705882}%
\pgfsetfillcolor{currentfill}%
\pgfsetfillopacity{0.863013}%
\pgfsetlinewidth{1.003750pt}%
\definecolor{currentstroke}{rgb}{0.121569,0.466667,0.705882}%
\pgfsetstrokecolor{currentstroke}%
\pgfsetstrokeopacity{0.863013}%
\pgfsetdash{}{0pt}%
\pgfpathmoveto{\pgfqpoint{0.976044in}{2.168194in}}%
\pgfpathcurveto{\pgfqpoint{0.984280in}{2.168194in}}{\pgfqpoint{0.992180in}{2.171467in}}{\pgfqpoint{0.998004in}{2.177290in}}%
\pgfpathcurveto{\pgfqpoint{1.003828in}{2.183114in}}{\pgfqpoint{1.007100in}{2.191014in}}{\pgfqpoint{1.007100in}{2.199251in}}%
\pgfpathcurveto{\pgfqpoint{1.007100in}{2.207487in}}{\pgfqpoint{1.003828in}{2.215387in}}{\pgfqpoint{0.998004in}{2.221211in}}%
\pgfpathcurveto{\pgfqpoint{0.992180in}{2.227035in}}{\pgfqpoint{0.984280in}{2.230307in}}{\pgfqpoint{0.976044in}{2.230307in}}%
\pgfpathcurveto{\pgfqpoint{0.967808in}{2.230307in}}{\pgfqpoint{0.959908in}{2.227035in}}{\pgfqpoint{0.954084in}{2.221211in}}%
\pgfpathcurveto{\pgfqpoint{0.948260in}{2.215387in}}{\pgfqpoint{0.944987in}{2.207487in}}{\pgfqpoint{0.944987in}{2.199251in}}%
\pgfpathcurveto{\pgfqpoint{0.944987in}{2.191014in}}{\pgfqpoint{0.948260in}{2.183114in}}{\pgfqpoint{0.954084in}{2.177290in}}%
\pgfpathcurveto{\pgfqpoint{0.959908in}{2.171467in}}{\pgfqpoint{0.967808in}{2.168194in}}{\pgfqpoint{0.976044in}{2.168194in}}%
\pgfpathclose%
\pgfusepath{stroke,fill}%
\end{pgfscope}%
\begin{pgfscope}%
\pgfpathrectangle{\pgfqpoint{0.100000in}{0.212622in}}{\pgfqpoint{3.696000in}{3.696000in}}%
\pgfusepath{clip}%
\pgfsetbuttcap%
\pgfsetroundjoin%
\definecolor{currentfill}{rgb}{0.121569,0.466667,0.705882}%
\pgfsetfillcolor{currentfill}%
\pgfsetfillopacity{0.863296}%
\pgfsetlinewidth{1.003750pt}%
\definecolor{currentstroke}{rgb}{0.121569,0.466667,0.705882}%
\pgfsetstrokecolor{currentstroke}%
\pgfsetstrokeopacity{0.863296}%
\pgfsetdash{}{0pt}%
\pgfpathmoveto{\pgfqpoint{2.274385in}{2.581249in}}%
\pgfpathcurveto{\pgfqpoint{2.282621in}{2.581249in}}{\pgfqpoint{2.290521in}{2.584521in}}{\pgfqpoint{2.296345in}{2.590345in}}%
\pgfpathcurveto{\pgfqpoint{2.302169in}{2.596169in}}{\pgfqpoint{2.305441in}{2.604069in}}{\pgfqpoint{2.305441in}{2.612305in}}%
\pgfpathcurveto{\pgfqpoint{2.305441in}{2.620541in}}{\pgfqpoint{2.302169in}{2.628441in}}{\pgfqpoint{2.296345in}{2.634265in}}%
\pgfpathcurveto{\pgfqpoint{2.290521in}{2.640089in}}{\pgfqpoint{2.282621in}{2.643362in}}{\pgfqpoint{2.274385in}{2.643362in}}%
\pgfpathcurveto{\pgfqpoint{2.266149in}{2.643362in}}{\pgfqpoint{2.258249in}{2.640089in}}{\pgfqpoint{2.252425in}{2.634265in}}%
\pgfpathcurveto{\pgfqpoint{2.246601in}{2.628441in}}{\pgfqpoint{2.243328in}{2.620541in}}{\pgfqpoint{2.243328in}{2.612305in}}%
\pgfpathcurveto{\pgfqpoint{2.243328in}{2.604069in}}{\pgfqpoint{2.246601in}{2.596169in}}{\pgfqpoint{2.252425in}{2.590345in}}%
\pgfpathcurveto{\pgfqpoint{2.258249in}{2.584521in}}{\pgfqpoint{2.266149in}{2.581249in}}{\pgfqpoint{2.274385in}{2.581249in}}%
\pgfpathclose%
\pgfusepath{stroke,fill}%
\end{pgfscope}%
\begin{pgfscope}%
\pgfpathrectangle{\pgfqpoint{0.100000in}{0.212622in}}{\pgfqpoint{3.696000in}{3.696000in}}%
\pgfusepath{clip}%
\pgfsetbuttcap%
\pgfsetroundjoin%
\definecolor{currentfill}{rgb}{0.121569,0.466667,0.705882}%
\pgfsetfillcolor{currentfill}%
\pgfsetfillopacity{0.863376}%
\pgfsetlinewidth{1.003750pt}%
\definecolor{currentstroke}{rgb}{0.121569,0.466667,0.705882}%
\pgfsetstrokecolor{currentstroke}%
\pgfsetstrokeopacity{0.863376}%
\pgfsetdash{}{0pt}%
\pgfpathmoveto{\pgfqpoint{1.830721in}{2.712266in}}%
\pgfpathcurveto{\pgfqpoint{1.838957in}{2.712266in}}{\pgfqpoint{1.846857in}{2.715539in}}{\pgfqpoint{1.852681in}{2.721363in}}%
\pgfpathcurveto{\pgfqpoint{1.858505in}{2.727186in}}{\pgfqpoint{1.861777in}{2.735087in}}{\pgfqpoint{1.861777in}{2.743323in}}%
\pgfpathcurveto{\pgfqpoint{1.861777in}{2.751559in}}{\pgfqpoint{1.858505in}{2.759459in}}{\pgfqpoint{1.852681in}{2.765283in}}%
\pgfpathcurveto{\pgfqpoint{1.846857in}{2.771107in}}{\pgfqpoint{1.838957in}{2.774379in}}{\pgfqpoint{1.830721in}{2.774379in}}%
\pgfpathcurveto{\pgfqpoint{1.822484in}{2.774379in}}{\pgfqpoint{1.814584in}{2.771107in}}{\pgfqpoint{1.808760in}{2.765283in}}%
\pgfpathcurveto{\pgfqpoint{1.802937in}{2.759459in}}{\pgfqpoint{1.799664in}{2.751559in}}{\pgfqpoint{1.799664in}{2.743323in}}%
\pgfpathcurveto{\pgfqpoint{1.799664in}{2.735087in}}{\pgfqpoint{1.802937in}{2.727186in}}{\pgfqpoint{1.808760in}{2.721363in}}%
\pgfpathcurveto{\pgfqpoint{1.814584in}{2.715539in}}{\pgfqpoint{1.822484in}{2.712266in}}{\pgfqpoint{1.830721in}{2.712266in}}%
\pgfpathclose%
\pgfusepath{stroke,fill}%
\end{pgfscope}%
\begin{pgfscope}%
\pgfpathrectangle{\pgfqpoint{0.100000in}{0.212622in}}{\pgfqpoint{3.696000in}{3.696000in}}%
\pgfusepath{clip}%
\pgfsetbuttcap%
\pgfsetroundjoin%
\definecolor{currentfill}{rgb}{0.121569,0.466667,0.705882}%
\pgfsetfillcolor{currentfill}%
\pgfsetfillopacity{0.863636}%
\pgfsetlinewidth{1.003750pt}%
\definecolor{currentstroke}{rgb}{0.121569,0.466667,0.705882}%
\pgfsetstrokecolor{currentstroke}%
\pgfsetstrokeopacity{0.863636}%
\pgfsetdash{}{0pt}%
\pgfpathmoveto{\pgfqpoint{1.830156in}{2.711529in}}%
\pgfpathcurveto{\pgfqpoint{1.838393in}{2.711529in}}{\pgfqpoint{1.846293in}{2.714802in}}{\pgfqpoint{1.852117in}{2.720626in}}%
\pgfpathcurveto{\pgfqpoint{1.857941in}{2.726450in}}{\pgfqpoint{1.861213in}{2.734350in}}{\pgfqpoint{1.861213in}{2.742586in}}%
\pgfpathcurveto{\pgfqpoint{1.861213in}{2.750822in}}{\pgfqpoint{1.857941in}{2.758722in}}{\pgfqpoint{1.852117in}{2.764546in}}%
\pgfpathcurveto{\pgfqpoint{1.846293in}{2.770370in}}{\pgfqpoint{1.838393in}{2.773642in}}{\pgfqpoint{1.830156in}{2.773642in}}%
\pgfpathcurveto{\pgfqpoint{1.821920in}{2.773642in}}{\pgfqpoint{1.814020in}{2.770370in}}{\pgfqpoint{1.808196in}{2.764546in}}%
\pgfpathcurveto{\pgfqpoint{1.802372in}{2.758722in}}{\pgfqpoint{1.799100in}{2.750822in}}{\pgfqpoint{1.799100in}{2.742586in}}%
\pgfpathcurveto{\pgfqpoint{1.799100in}{2.734350in}}{\pgfqpoint{1.802372in}{2.726450in}}{\pgfqpoint{1.808196in}{2.720626in}}%
\pgfpathcurveto{\pgfqpoint{1.814020in}{2.714802in}}{\pgfqpoint{1.821920in}{2.711529in}}{\pgfqpoint{1.830156in}{2.711529in}}%
\pgfpathclose%
\pgfusepath{stroke,fill}%
\end{pgfscope}%
\begin{pgfscope}%
\pgfpathrectangle{\pgfqpoint{0.100000in}{0.212622in}}{\pgfqpoint{3.696000in}{3.696000in}}%
\pgfusepath{clip}%
\pgfsetbuttcap%
\pgfsetroundjoin%
\definecolor{currentfill}{rgb}{0.121569,0.466667,0.705882}%
\pgfsetfillcolor{currentfill}%
\pgfsetfillopacity{0.863745}%
\pgfsetlinewidth{1.003750pt}%
\definecolor{currentstroke}{rgb}{0.121569,0.466667,0.705882}%
\pgfsetstrokecolor{currentstroke}%
\pgfsetstrokeopacity{0.863745}%
\pgfsetdash{}{0pt}%
\pgfpathmoveto{\pgfqpoint{0.973924in}{2.165825in}}%
\pgfpathcurveto{\pgfqpoint{0.982161in}{2.165825in}}{\pgfqpoint{0.990061in}{2.169097in}}{\pgfqpoint{0.995885in}{2.174921in}}%
\pgfpathcurveto{\pgfqpoint{1.001708in}{2.180745in}}{\pgfqpoint{1.004981in}{2.188645in}}{\pgfqpoint{1.004981in}{2.196881in}}%
\pgfpathcurveto{\pgfqpoint{1.004981in}{2.205118in}}{\pgfqpoint{1.001708in}{2.213018in}}{\pgfqpoint{0.995885in}{2.218842in}}%
\pgfpathcurveto{\pgfqpoint{0.990061in}{2.224666in}}{\pgfqpoint{0.982161in}{2.227938in}}{\pgfqpoint{0.973924in}{2.227938in}}%
\pgfpathcurveto{\pgfqpoint{0.965688in}{2.227938in}}{\pgfqpoint{0.957788in}{2.224666in}}{\pgfqpoint{0.951964in}{2.218842in}}%
\pgfpathcurveto{\pgfqpoint{0.946140in}{2.213018in}}{\pgfqpoint{0.942868in}{2.205118in}}{\pgfqpoint{0.942868in}{2.196881in}}%
\pgfpathcurveto{\pgfqpoint{0.942868in}{2.188645in}}{\pgfqpoint{0.946140in}{2.180745in}}{\pgfqpoint{0.951964in}{2.174921in}}%
\pgfpathcurveto{\pgfqpoint{0.957788in}{2.169097in}}{\pgfqpoint{0.965688in}{2.165825in}}{\pgfqpoint{0.973924in}{2.165825in}}%
\pgfpathclose%
\pgfusepath{stroke,fill}%
\end{pgfscope}%
\begin{pgfscope}%
\pgfpathrectangle{\pgfqpoint{0.100000in}{0.212622in}}{\pgfqpoint{3.696000in}{3.696000in}}%
\pgfusepath{clip}%
\pgfsetbuttcap%
\pgfsetroundjoin%
\definecolor{currentfill}{rgb}{0.121569,0.466667,0.705882}%
\pgfsetfillcolor{currentfill}%
\pgfsetfillopacity{0.864111}%
\pgfsetlinewidth{1.003750pt}%
\definecolor{currentstroke}{rgb}{0.121569,0.466667,0.705882}%
\pgfsetstrokecolor{currentstroke}%
\pgfsetstrokeopacity{0.864111}%
\pgfsetdash{}{0pt}%
\pgfpathmoveto{\pgfqpoint{1.829208in}{2.710376in}}%
\pgfpathcurveto{\pgfqpoint{1.837444in}{2.710376in}}{\pgfqpoint{1.845344in}{2.713648in}}{\pgfqpoint{1.851168in}{2.719472in}}%
\pgfpathcurveto{\pgfqpoint{1.856992in}{2.725296in}}{\pgfqpoint{1.860264in}{2.733196in}}{\pgfqpoint{1.860264in}{2.741432in}}%
\pgfpathcurveto{\pgfqpoint{1.860264in}{2.749669in}}{\pgfqpoint{1.856992in}{2.757569in}}{\pgfqpoint{1.851168in}{2.763393in}}%
\pgfpathcurveto{\pgfqpoint{1.845344in}{2.769216in}}{\pgfqpoint{1.837444in}{2.772489in}}{\pgfqpoint{1.829208in}{2.772489in}}%
\pgfpathcurveto{\pgfqpoint{1.820972in}{2.772489in}}{\pgfqpoint{1.813072in}{2.769216in}}{\pgfqpoint{1.807248in}{2.763393in}}%
\pgfpathcurveto{\pgfqpoint{1.801424in}{2.757569in}}{\pgfqpoint{1.798151in}{2.749669in}}{\pgfqpoint{1.798151in}{2.741432in}}%
\pgfpathcurveto{\pgfqpoint{1.798151in}{2.733196in}}{\pgfqpoint{1.801424in}{2.725296in}}{\pgfqpoint{1.807248in}{2.719472in}}%
\pgfpathcurveto{\pgfqpoint{1.813072in}{2.713648in}}{\pgfqpoint{1.820972in}{2.710376in}}{\pgfqpoint{1.829208in}{2.710376in}}%
\pgfpathclose%
\pgfusepath{stroke,fill}%
\end{pgfscope}%
\begin{pgfscope}%
\pgfpathrectangle{\pgfqpoint{0.100000in}{0.212622in}}{\pgfqpoint{3.696000in}{3.696000in}}%
\pgfusepath{clip}%
\pgfsetbuttcap%
\pgfsetroundjoin%
\definecolor{currentfill}{rgb}{0.121569,0.466667,0.705882}%
\pgfsetfillcolor{currentfill}%
\pgfsetfillopacity{0.864154}%
\pgfsetlinewidth{1.003750pt}%
\definecolor{currentstroke}{rgb}{0.121569,0.466667,0.705882}%
\pgfsetstrokecolor{currentstroke}%
\pgfsetstrokeopacity{0.864154}%
\pgfsetdash{}{0pt}%
\pgfpathmoveto{\pgfqpoint{0.972791in}{2.164512in}}%
\pgfpathcurveto{\pgfqpoint{0.981028in}{2.164512in}}{\pgfqpoint{0.988928in}{2.167784in}}{\pgfqpoint{0.994752in}{2.173608in}}%
\pgfpathcurveto{\pgfqpoint{1.000576in}{2.179432in}}{\pgfqpoint{1.003848in}{2.187332in}}{\pgfqpoint{1.003848in}{2.195568in}}%
\pgfpathcurveto{\pgfqpoint{1.003848in}{2.203804in}}{\pgfqpoint{1.000576in}{2.211704in}}{\pgfqpoint{0.994752in}{2.217528in}}%
\pgfpathcurveto{\pgfqpoint{0.988928in}{2.223352in}}{\pgfqpoint{0.981028in}{2.226625in}}{\pgfqpoint{0.972791in}{2.226625in}}%
\pgfpathcurveto{\pgfqpoint{0.964555in}{2.226625in}}{\pgfqpoint{0.956655in}{2.223352in}}{\pgfqpoint{0.950831in}{2.217528in}}%
\pgfpathcurveto{\pgfqpoint{0.945007in}{2.211704in}}{\pgfqpoint{0.941735in}{2.203804in}}{\pgfqpoint{0.941735in}{2.195568in}}%
\pgfpathcurveto{\pgfqpoint{0.941735in}{2.187332in}}{\pgfqpoint{0.945007in}{2.179432in}}{\pgfqpoint{0.950831in}{2.173608in}}%
\pgfpathcurveto{\pgfqpoint{0.956655in}{2.167784in}}{\pgfqpoint{0.964555in}{2.164512in}}{\pgfqpoint{0.972791in}{2.164512in}}%
\pgfpathclose%
\pgfusepath{stroke,fill}%
\end{pgfscope}%
\begin{pgfscope}%
\pgfpathrectangle{\pgfqpoint{0.100000in}{0.212622in}}{\pgfqpoint{3.696000in}{3.696000in}}%
\pgfusepath{clip}%
\pgfsetbuttcap%
\pgfsetroundjoin%
\definecolor{currentfill}{rgb}{0.121569,0.466667,0.705882}%
\pgfsetfillcolor{currentfill}%
\pgfsetfillopacity{0.864216}%
\pgfsetlinewidth{1.003750pt}%
\definecolor{currentstroke}{rgb}{0.121569,0.466667,0.705882}%
\pgfsetstrokecolor{currentstroke}%
\pgfsetstrokeopacity{0.864216}%
\pgfsetdash{}{0pt}%
\pgfpathmoveto{\pgfqpoint{2.272175in}{2.578850in}}%
\pgfpathcurveto{\pgfqpoint{2.280412in}{2.578850in}}{\pgfqpoint{2.288312in}{2.582123in}}{\pgfqpoint{2.294136in}{2.587947in}}%
\pgfpathcurveto{\pgfqpoint{2.299960in}{2.593771in}}{\pgfqpoint{2.303232in}{2.601671in}}{\pgfqpoint{2.303232in}{2.609907in}}%
\pgfpathcurveto{\pgfqpoint{2.303232in}{2.618143in}}{\pgfqpoint{2.299960in}{2.626043in}}{\pgfqpoint{2.294136in}{2.631867in}}%
\pgfpathcurveto{\pgfqpoint{2.288312in}{2.637691in}}{\pgfqpoint{2.280412in}{2.640963in}}{\pgfqpoint{2.272175in}{2.640963in}}%
\pgfpathcurveto{\pgfqpoint{2.263939in}{2.640963in}}{\pgfqpoint{2.256039in}{2.637691in}}{\pgfqpoint{2.250215in}{2.631867in}}%
\pgfpathcurveto{\pgfqpoint{2.244391in}{2.626043in}}{\pgfqpoint{2.241119in}{2.618143in}}{\pgfqpoint{2.241119in}{2.609907in}}%
\pgfpathcurveto{\pgfqpoint{2.241119in}{2.601671in}}{\pgfqpoint{2.244391in}{2.593771in}}{\pgfqpoint{2.250215in}{2.587947in}}%
\pgfpathcurveto{\pgfqpoint{2.256039in}{2.582123in}}{\pgfqpoint{2.263939in}{2.578850in}}{\pgfqpoint{2.272175in}{2.578850in}}%
\pgfpathclose%
\pgfusepath{stroke,fill}%
\end{pgfscope}%
\begin{pgfscope}%
\pgfpathrectangle{\pgfqpoint{0.100000in}{0.212622in}}{\pgfqpoint{3.696000in}{3.696000in}}%
\pgfusepath{clip}%
\pgfsetbuttcap%
\pgfsetroundjoin%
\definecolor{currentfill}{rgb}{0.121569,0.466667,0.705882}%
\pgfsetfillcolor{currentfill}%
\pgfsetfillopacity{0.864384}%
\pgfsetlinewidth{1.003750pt}%
\definecolor{currentstroke}{rgb}{0.121569,0.466667,0.705882}%
\pgfsetstrokecolor{currentstroke}%
\pgfsetstrokeopacity{0.864384}%
\pgfsetdash{}{0pt}%
\pgfpathmoveto{\pgfqpoint{0.972174in}{2.163817in}}%
\pgfpathcurveto{\pgfqpoint{0.980411in}{2.163817in}}{\pgfqpoint{0.988311in}{2.167089in}}{\pgfqpoint{0.994135in}{2.172913in}}%
\pgfpathcurveto{\pgfqpoint{0.999958in}{2.178737in}}{\pgfqpoint{1.003231in}{2.186637in}}{\pgfqpoint{1.003231in}{2.194873in}}%
\pgfpathcurveto{\pgfqpoint{1.003231in}{2.203110in}}{\pgfqpoint{0.999958in}{2.211010in}}{\pgfqpoint{0.994135in}{2.216834in}}%
\pgfpathcurveto{\pgfqpoint{0.988311in}{2.222657in}}{\pgfqpoint{0.980411in}{2.225930in}}{\pgfqpoint{0.972174in}{2.225930in}}%
\pgfpathcurveto{\pgfqpoint{0.963938in}{2.225930in}}{\pgfqpoint{0.956038in}{2.222657in}}{\pgfqpoint{0.950214in}{2.216834in}}%
\pgfpathcurveto{\pgfqpoint{0.944390in}{2.211010in}}{\pgfqpoint{0.941118in}{2.203110in}}{\pgfqpoint{0.941118in}{2.194873in}}%
\pgfpathcurveto{\pgfqpoint{0.941118in}{2.186637in}}{\pgfqpoint{0.944390in}{2.178737in}}{\pgfqpoint{0.950214in}{2.172913in}}%
\pgfpathcurveto{\pgfqpoint{0.956038in}{2.167089in}}{\pgfqpoint{0.963938in}{2.163817in}}{\pgfqpoint{0.972174in}{2.163817in}}%
\pgfpathclose%
\pgfusepath{stroke,fill}%
\end{pgfscope}%
\begin{pgfscope}%
\pgfpathrectangle{\pgfqpoint{0.100000in}{0.212622in}}{\pgfqpoint{3.696000in}{3.696000in}}%
\pgfusepath{clip}%
\pgfsetbuttcap%
\pgfsetroundjoin%
\definecolor{currentfill}{rgb}{0.121569,0.466667,0.705882}%
\pgfsetfillcolor{currentfill}%
\pgfsetfillopacity{0.864519}%
\pgfsetlinewidth{1.003750pt}%
\definecolor{currentstroke}{rgb}{0.121569,0.466667,0.705882}%
\pgfsetstrokecolor{currentstroke}%
\pgfsetstrokeopacity{0.864519}%
\pgfsetdash{}{0pt}%
\pgfpathmoveto{\pgfqpoint{0.971836in}{2.163479in}}%
\pgfpathcurveto{\pgfqpoint{0.980072in}{2.163479in}}{\pgfqpoint{0.987972in}{2.166751in}}{\pgfqpoint{0.993796in}{2.172575in}}%
\pgfpathcurveto{\pgfqpoint{0.999620in}{2.178399in}}{\pgfqpoint{1.002892in}{2.186299in}}{\pgfqpoint{1.002892in}{2.194536in}}%
\pgfpathcurveto{\pgfqpoint{1.002892in}{2.202772in}}{\pgfqpoint{0.999620in}{2.210672in}}{\pgfqpoint{0.993796in}{2.216496in}}%
\pgfpathcurveto{\pgfqpoint{0.987972in}{2.222320in}}{\pgfqpoint{0.980072in}{2.225592in}}{\pgfqpoint{0.971836in}{2.225592in}}%
\pgfpathcurveto{\pgfqpoint{0.963599in}{2.225592in}}{\pgfqpoint{0.955699in}{2.222320in}}{\pgfqpoint{0.949875in}{2.216496in}}%
\pgfpathcurveto{\pgfqpoint{0.944051in}{2.210672in}}{\pgfqpoint{0.940779in}{2.202772in}}{\pgfqpoint{0.940779in}{2.194536in}}%
\pgfpathcurveto{\pgfqpoint{0.940779in}{2.186299in}}{\pgfqpoint{0.944051in}{2.178399in}}{\pgfqpoint{0.949875in}{2.172575in}}%
\pgfpathcurveto{\pgfqpoint{0.955699in}{2.166751in}}{\pgfqpoint{0.963599in}{2.163479in}}{\pgfqpoint{0.971836in}{2.163479in}}%
\pgfpathclose%
\pgfusepath{stroke,fill}%
\end{pgfscope}%
\begin{pgfscope}%
\pgfpathrectangle{\pgfqpoint{0.100000in}{0.212622in}}{\pgfqpoint{3.696000in}{3.696000in}}%
\pgfusepath{clip}%
\pgfsetbuttcap%
\pgfsetroundjoin%
\definecolor{currentfill}{rgb}{0.121569,0.466667,0.705882}%
\pgfsetfillcolor{currentfill}%
\pgfsetfillopacity{0.864757}%
\pgfsetlinewidth{1.003750pt}%
\definecolor{currentstroke}{rgb}{0.121569,0.466667,0.705882}%
\pgfsetstrokecolor{currentstroke}%
\pgfsetstrokeopacity{0.864757}%
\pgfsetdash{}{0pt}%
\pgfpathmoveto{\pgfqpoint{1.827854in}{2.708608in}}%
\pgfpathcurveto{\pgfqpoint{1.836090in}{2.708608in}}{\pgfqpoint{1.843990in}{2.711880in}}{\pgfqpoint{1.849814in}{2.717704in}}%
\pgfpathcurveto{\pgfqpoint{1.855638in}{2.723528in}}{\pgfqpoint{1.858910in}{2.731428in}}{\pgfqpoint{1.858910in}{2.739665in}}%
\pgfpathcurveto{\pgfqpoint{1.858910in}{2.747901in}}{\pgfqpoint{1.855638in}{2.755801in}}{\pgfqpoint{1.849814in}{2.761625in}}%
\pgfpathcurveto{\pgfqpoint{1.843990in}{2.767449in}}{\pgfqpoint{1.836090in}{2.770721in}}{\pgfqpoint{1.827854in}{2.770721in}}%
\pgfpathcurveto{\pgfqpoint{1.819617in}{2.770721in}}{\pgfqpoint{1.811717in}{2.767449in}}{\pgfqpoint{1.805894in}{2.761625in}}%
\pgfpathcurveto{\pgfqpoint{1.800070in}{2.755801in}}{\pgfqpoint{1.796797in}{2.747901in}}{\pgfqpoint{1.796797in}{2.739665in}}%
\pgfpathcurveto{\pgfqpoint{1.796797in}{2.731428in}}{\pgfqpoint{1.800070in}{2.723528in}}{\pgfqpoint{1.805894in}{2.717704in}}%
\pgfpathcurveto{\pgfqpoint{1.811717in}{2.711880in}}{\pgfqpoint{1.819617in}{2.708608in}}{\pgfqpoint{1.827854in}{2.708608in}}%
\pgfpathclose%
\pgfusepath{stroke,fill}%
\end{pgfscope}%
\begin{pgfscope}%
\pgfpathrectangle{\pgfqpoint{0.100000in}{0.212622in}}{\pgfqpoint{3.696000in}{3.696000in}}%
\pgfusepath{clip}%
\pgfsetbuttcap%
\pgfsetroundjoin%
\definecolor{currentfill}{rgb}{0.121569,0.466667,0.705882}%
\pgfsetfillcolor{currentfill}%
\pgfsetfillopacity{0.864856}%
\pgfsetlinewidth{1.003750pt}%
\definecolor{currentstroke}{rgb}{0.121569,0.466667,0.705882}%
\pgfsetstrokecolor{currentstroke}%
\pgfsetstrokeopacity{0.864856}%
\pgfsetdash{}{0pt}%
\pgfpathmoveto{\pgfqpoint{0.971033in}{2.162836in}}%
\pgfpathcurveto{\pgfqpoint{0.979269in}{2.162836in}}{\pgfqpoint{0.987169in}{2.166109in}}{\pgfqpoint{0.992993in}{2.171933in}}%
\pgfpathcurveto{\pgfqpoint{0.998817in}{2.177757in}}{\pgfqpoint{1.002089in}{2.185657in}}{\pgfqpoint{1.002089in}{2.193893in}}%
\pgfpathcurveto{\pgfqpoint{1.002089in}{2.202129in}}{\pgfqpoint{0.998817in}{2.210029in}}{\pgfqpoint{0.992993in}{2.215853in}}%
\pgfpathcurveto{\pgfqpoint{0.987169in}{2.221677in}}{\pgfqpoint{0.979269in}{2.224949in}}{\pgfqpoint{0.971033in}{2.224949in}}%
\pgfpathcurveto{\pgfqpoint{0.962796in}{2.224949in}}{\pgfqpoint{0.954896in}{2.221677in}}{\pgfqpoint{0.949072in}{2.215853in}}%
\pgfpathcurveto{\pgfqpoint{0.943248in}{2.210029in}}{\pgfqpoint{0.939976in}{2.202129in}}{\pgfqpoint{0.939976in}{2.193893in}}%
\pgfpathcurveto{\pgfqpoint{0.939976in}{2.185657in}}{\pgfqpoint{0.943248in}{2.177757in}}{\pgfqpoint{0.949072in}{2.171933in}}%
\pgfpathcurveto{\pgfqpoint{0.954896in}{2.166109in}}{\pgfqpoint{0.962796in}{2.162836in}}{\pgfqpoint{0.971033in}{2.162836in}}%
\pgfpathclose%
\pgfusepath{stroke,fill}%
\end{pgfscope}%
\begin{pgfscope}%
\pgfpathrectangle{\pgfqpoint{0.100000in}{0.212622in}}{\pgfqpoint{3.696000in}{3.696000in}}%
\pgfusepath{clip}%
\pgfsetbuttcap%
\pgfsetroundjoin%
\definecolor{currentfill}{rgb}{0.121569,0.466667,0.705882}%
\pgfsetfillcolor{currentfill}%
\pgfsetfillopacity{0.865041}%
\pgfsetlinewidth{1.003750pt}%
\definecolor{currentstroke}{rgb}{0.121569,0.466667,0.705882}%
\pgfsetstrokecolor{currentstroke}%
\pgfsetstrokeopacity{0.865041}%
\pgfsetdash{}{0pt}%
\pgfpathmoveto{\pgfqpoint{2.270238in}{2.576729in}}%
\pgfpathcurveto{\pgfqpoint{2.278475in}{2.576729in}}{\pgfqpoint{2.286375in}{2.580001in}}{\pgfqpoint{2.292199in}{2.585825in}}%
\pgfpathcurveto{\pgfqpoint{2.298023in}{2.591649in}}{\pgfqpoint{2.301295in}{2.599549in}}{\pgfqpoint{2.301295in}{2.607785in}}%
\pgfpathcurveto{\pgfqpoint{2.301295in}{2.616021in}}{\pgfqpoint{2.298023in}{2.623921in}}{\pgfqpoint{2.292199in}{2.629745in}}%
\pgfpathcurveto{\pgfqpoint{2.286375in}{2.635569in}}{\pgfqpoint{2.278475in}{2.638842in}}{\pgfqpoint{2.270238in}{2.638842in}}%
\pgfpathcurveto{\pgfqpoint{2.262002in}{2.638842in}}{\pgfqpoint{2.254102in}{2.635569in}}{\pgfqpoint{2.248278in}{2.629745in}}%
\pgfpathcurveto{\pgfqpoint{2.242454in}{2.623921in}}{\pgfqpoint{2.239182in}{2.616021in}}{\pgfqpoint{2.239182in}{2.607785in}}%
\pgfpathcurveto{\pgfqpoint{2.239182in}{2.599549in}}{\pgfqpoint{2.242454in}{2.591649in}}{\pgfqpoint{2.248278in}{2.585825in}}%
\pgfpathcurveto{\pgfqpoint{2.254102in}{2.580001in}}{\pgfqpoint{2.262002in}{2.576729in}}{\pgfqpoint{2.270238in}{2.576729in}}%
\pgfpathclose%
\pgfusepath{stroke,fill}%
\end{pgfscope}%
\begin{pgfscope}%
\pgfpathrectangle{\pgfqpoint{0.100000in}{0.212622in}}{\pgfqpoint{3.696000in}{3.696000in}}%
\pgfusepath{clip}%
\pgfsetbuttcap%
\pgfsetroundjoin%
\definecolor{currentfill}{rgb}{0.121569,0.466667,0.705882}%
\pgfsetfillcolor{currentfill}%
\pgfsetfillopacity{0.865299}%
\pgfsetlinewidth{1.003750pt}%
\definecolor{currentstroke}{rgb}{0.121569,0.466667,0.705882}%
\pgfsetstrokecolor{currentstroke}%
\pgfsetstrokeopacity{0.865299}%
\pgfsetdash{}{0pt}%
\pgfpathmoveto{\pgfqpoint{0.969929in}{2.161893in}}%
\pgfpathcurveto{\pgfqpoint{0.978165in}{2.161893in}}{\pgfqpoint{0.986065in}{2.165165in}}{\pgfqpoint{0.991889in}{2.170989in}}%
\pgfpathcurveto{\pgfqpoint{0.997713in}{2.176813in}}{\pgfqpoint{1.000985in}{2.184713in}}{\pgfqpoint{1.000985in}{2.192949in}}%
\pgfpathcurveto{\pgfqpoint{1.000985in}{2.201185in}}{\pgfqpoint{0.997713in}{2.209086in}}{\pgfqpoint{0.991889in}{2.214909in}}%
\pgfpathcurveto{\pgfqpoint{0.986065in}{2.220733in}}{\pgfqpoint{0.978165in}{2.224006in}}{\pgfqpoint{0.969929in}{2.224006in}}%
\pgfpathcurveto{\pgfqpoint{0.961693in}{2.224006in}}{\pgfqpoint{0.953792in}{2.220733in}}{\pgfqpoint{0.947969in}{2.214909in}}%
\pgfpathcurveto{\pgfqpoint{0.942145in}{2.209086in}}{\pgfqpoint{0.938872in}{2.201185in}}{\pgfqpoint{0.938872in}{2.192949in}}%
\pgfpathcurveto{\pgfqpoint{0.938872in}{2.184713in}}{\pgfqpoint{0.942145in}{2.176813in}}{\pgfqpoint{0.947969in}{2.170989in}}%
\pgfpathcurveto{\pgfqpoint{0.953792in}{2.165165in}}{\pgfqpoint{0.961693in}{2.161893in}}{\pgfqpoint{0.969929in}{2.161893in}}%
\pgfpathclose%
\pgfusepath{stroke,fill}%
\end{pgfscope}%
\begin{pgfscope}%
\pgfpathrectangle{\pgfqpoint{0.100000in}{0.212622in}}{\pgfqpoint{3.696000in}{3.696000in}}%
\pgfusepath{clip}%
\pgfsetbuttcap%
\pgfsetroundjoin%
\definecolor{currentfill}{rgb}{0.121569,0.466667,0.705882}%
\pgfsetfillcolor{currentfill}%
\pgfsetfillopacity{0.865499}%
\pgfsetlinewidth{1.003750pt}%
\definecolor{currentstroke}{rgb}{0.121569,0.466667,0.705882}%
\pgfsetstrokecolor{currentstroke}%
\pgfsetstrokeopacity{0.865499}%
\pgfsetdash{}{0pt}%
\pgfpathmoveto{\pgfqpoint{1.826325in}{2.706645in}}%
\pgfpathcurveto{\pgfqpoint{1.834561in}{2.706645in}}{\pgfqpoint{1.842462in}{2.709917in}}{\pgfqpoint{1.848285in}{2.715741in}}%
\pgfpathcurveto{\pgfqpoint{1.854109in}{2.721565in}}{\pgfqpoint{1.857382in}{2.729465in}}{\pgfqpoint{1.857382in}{2.737701in}}%
\pgfpathcurveto{\pgfqpoint{1.857382in}{2.745937in}}{\pgfqpoint{1.854109in}{2.753838in}}{\pgfqpoint{1.848285in}{2.759661in}}%
\pgfpathcurveto{\pgfqpoint{1.842462in}{2.765485in}}{\pgfqpoint{1.834561in}{2.768758in}}{\pgfqpoint{1.826325in}{2.768758in}}%
\pgfpathcurveto{\pgfqpoint{1.818089in}{2.768758in}}{\pgfqpoint{1.810189in}{2.765485in}}{\pgfqpoint{1.804365in}{2.759661in}}%
\pgfpathcurveto{\pgfqpoint{1.798541in}{2.753838in}}{\pgfqpoint{1.795269in}{2.745937in}}{\pgfqpoint{1.795269in}{2.737701in}}%
\pgfpathcurveto{\pgfqpoint{1.795269in}{2.729465in}}{\pgfqpoint{1.798541in}{2.721565in}}{\pgfqpoint{1.804365in}{2.715741in}}%
\pgfpathcurveto{\pgfqpoint{1.810189in}{2.709917in}}{\pgfqpoint{1.818089in}{2.706645in}}{\pgfqpoint{1.826325in}{2.706645in}}%
\pgfpathclose%
\pgfusepath{stroke,fill}%
\end{pgfscope}%
\begin{pgfscope}%
\pgfpathrectangle{\pgfqpoint{0.100000in}{0.212622in}}{\pgfqpoint{3.696000in}{3.696000in}}%
\pgfusepath{clip}%
\pgfsetbuttcap%
\pgfsetroundjoin%
\definecolor{currentfill}{rgb}{0.121569,0.466667,0.705882}%
\pgfsetfillcolor{currentfill}%
\pgfsetfillopacity{0.865522}%
\pgfsetlinewidth{1.003750pt}%
\definecolor{currentstroke}{rgb}{0.121569,0.466667,0.705882}%
\pgfsetstrokecolor{currentstroke}%
\pgfsetstrokeopacity{0.865522}%
\pgfsetdash{}{0pt}%
\pgfpathmoveto{\pgfqpoint{2.269047in}{2.575311in}}%
\pgfpathcurveto{\pgfqpoint{2.277284in}{2.575311in}}{\pgfqpoint{2.285184in}{2.578584in}}{\pgfqpoint{2.291008in}{2.584408in}}%
\pgfpathcurveto{\pgfqpoint{2.296832in}{2.590231in}}{\pgfqpoint{2.300104in}{2.598132in}}{\pgfqpoint{2.300104in}{2.606368in}}%
\pgfpathcurveto{\pgfqpoint{2.300104in}{2.614604in}}{\pgfqpoint{2.296832in}{2.622504in}}{\pgfqpoint{2.291008in}{2.628328in}}%
\pgfpathcurveto{\pgfqpoint{2.285184in}{2.634152in}}{\pgfqpoint{2.277284in}{2.637424in}}{\pgfqpoint{2.269047in}{2.637424in}}%
\pgfpathcurveto{\pgfqpoint{2.260811in}{2.637424in}}{\pgfqpoint{2.252911in}{2.634152in}}{\pgfqpoint{2.247087in}{2.628328in}}%
\pgfpathcurveto{\pgfqpoint{2.241263in}{2.622504in}}{\pgfqpoint{2.237991in}{2.614604in}}{\pgfqpoint{2.237991in}{2.606368in}}%
\pgfpathcurveto{\pgfqpoint{2.237991in}{2.598132in}}{\pgfqpoint{2.241263in}{2.590231in}}{\pgfqpoint{2.247087in}{2.584408in}}%
\pgfpathcurveto{\pgfqpoint{2.252911in}{2.578584in}}{\pgfqpoint{2.260811in}{2.575311in}}{\pgfqpoint{2.269047in}{2.575311in}}%
\pgfpathclose%
\pgfusepath{stroke,fill}%
\end{pgfscope}%
\begin{pgfscope}%
\pgfpathrectangle{\pgfqpoint{0.100000in}{0.212622in}}{\pgfqpoint{3.696000in}{3.696000in}}%
\pgfusepath{clip}%
\pgfsetbuttcap%
\pgfsetroundjoin%
\definecolor{currentfill}{rgb}{0.121569,0.466667,0.705882}%
\pgfsetfillcolor{currentfill}%
\pgfsetfillopacity{0.865916}%
\pgfsetlinewidth{1.003750pt}%
\definecolor{currentstroke}{rgb}{0.121569,0.466667,0.705882}%
\pgfsetstrokecolor{currentstroke}%
\pgfsetstrokeopacity{0.865916}%
\pgfsetdash{}{0pt}%
\pgfpathmoveto{\pgfqpoint{0.968301in}{2.160371in}}%
\pgfpathcurveto{\pgfqpoint{0.976538in}{2.160371in}}{\pgfqpoint{0.984438in}{2.163644in}}{\pgfqpoint{0.990262in}{2.169468in}}%
\pgfpathcurveto{\pgfqpoint{0.996086in}{2.175291in}}{\pgfqpoint{0.999358in}{2.183192in}}{\pgfqpoint{0.999358in}{2.191428in}}%
\pgfpathcurveto{\pgfqpoint{0.999358in}{2.199664in}}{\pgfqpoint{0.996086in}{2.207564in}}{\pgfqpoint{0.990262in}{2.213388in}}%
\pgfpathcurveto{\pgfqpoint{0.984438in}{2.219212in}}{\pgfqpoint{0.976538in}{2.222484in}}{\pgfqpoint{0.968301in}{2.222484in}}%
\pgfpathcurveto{\pgfqpoint{0.960065in}{2.222484in}}{\pgfqpoint{0.952165in}{2.219212in}}{\pgfqpoint{0.946341in}{2.213388in}}%
\pgfpathcurveto{\pgfqpoint{0.940517in}{2.207564in}}{\pgfqpoint{0.937245in}{2.199664in}}{\pgfqpoint{0.937245in}{2.191428in}}%
\pgfpathcurveto{\pgfqpoint{0.937245in}{2.183192in}}{\pgfqpoint{0.940517in}{2.175291in}}{\pgfqpoint{0.946341in}{2.169468in}}%
\pgfpathcurveto{\pgfqpoint{0.952165in}{2.163644in}}{\pgfqpoint{0.960065in}{2.160371in}}{\pgfqpoint{0.968301in}{2.160371in}}%
\pgfpathclose%
\pgfusepath{stroke,fill}%
\end{pgfscope}%
\begin{pgfscope}%
\pgfpathrectangle{\pgfqpoint{0.100000in}{0.212622in}}{\pgfqpoint{3.696000in}{3.696000in}}%
\pgfusepath{clip}%
\pgfsetbuttcap%
\pgfsetroundjoin%
\definecolor{currentfill}{rgb}{0.121569,0.466667,0.705882}%
\pgfsetfillcolor{currentfill}%
\pgfsetfillopacity{0.866396}%
\pgfsetlinewidth{1.003750pt}%
\definecolor{currentstroke}{rgb}{0.121569,0.466667,0.705882}%
\pgfsetstrokecolor{currentstroke}%
\pgfsetstrokeopacity{0.866396}%
\pgfsetdash{}{0pt}%
\pgfpathmoveto{\pgfqpoint{2.266848in}{2.572769in}}%
\pgfpathcurveto{\pgfqpoint{2.275084in}{2.572769in}}{\pgfqpoint{2.282984in}{2.576041in}}{\pgfqpoint{2.288808in}{2.581865in}}%
\pgfpathcurveto{\pgfqpoint{2.294632in}{2.587689in}}{\pgfqpoint{2.297904in}{2.595589in}}{\pgfqpoint{2.297904in}{2.603825in}}%
\pgfpathcurveto{\pgfqpoint{2.297904in}{2.612062in}}{\pgfqpoint{2.294632in}{2.619962in}}{\pgfqpoint{2.288808in}{2.625786in}}%
\pgfpathcurveto{\pgfqpoint{2.282984in}{2.631609in}}{\pgfqpoint{2.275084in}{2.634882in}}{\pgfqpoint{2.266848in}{2.634882in}}%
\pgfpathcurveto{\pgfqpoint{2.258611in}{2.634882in}}{\pgfqpoint{2.250711in}{2.631609in}}{\pgfqpoint{2.244887in}{2.625786in}}%
\pgfpathcurveto{\pgfqpoint{2.239063in}{2.619962in}}{\pgfqpoint{2.235791in}{2.612062in}}{\pgfqpoint{2.235791in}{2.603825in}}%
\pgfpathcurveto{\pgfqpoint{2.235791in}{2.595589in}}{\pgfqpoint{2.239063in}{2.587689in}}{\pgfqpoint{2.244887in}{2.581865in}}%
\pgfpathcurveto{\pgfqpoint{2.250711in}{2.576041in}}{\pgfqpoint{2.258611in}{2.572769in}}{\pgfqpoint{2.266848in}{2.572769in}}%
\pgfpathclose%
\pgfusepath{stroke,fill}%
\end{pgfscope}%
\begin{pgfscope}%
\pgfpathrectangle{\pgfqpoint{0.100000in}{0.212622in}}{\pgfqpoint{3.696000in}{3.696000in}}%
\pgfusepath{clip}%
\pgfsetbuttcap%
\pgfsetroundjoin%
\definecolor{currentfill}{rgb}{0.121569,0.466667,0.705882}%
\pgfsetfillcolor{currentfill}%
\pgfsetfillopacity{0.866621}%
\pgfsetlinewidth{1.003750pt}%
\definecolor{currentstroke}{rgb}{0.121569,0.466667,0.705882}%
\pgfsetstrokecolor{currentstroke}%
\pgfsetstrokeopacity{0.866621}%
\pgfsetdash{}{0pt}%
\pgfpathmoveto{\pgfqpoint{1.824057in}{2.703938in}}%
\pgfpathcurveto{\pgfqpoint{1.832293in}{2.703938in}}{\pgfqpoint{1.840193in}{2.707210in}}{\pgfqpoint{1.846017in}{2.713034in}}%
\pgfpathcurveto{\pgfqpoint{1.851841in}{2.718858in}}{\pgfqpoint{1.855113in}{2.726758in}}{\pgfqpoint{1.855113in}{2.734994in}}%
\pgfpathcurveto{\pgfqpoint{1.855113in}{2.743231in}}{\pgfqpoint{1.851841in}{2.751131in}}{\pgfqpoint{1.846017in}{2.756955in}}%
\pgfpathcurveto{\pgfqpoint{1.840193in}{2.762779in}}{\pgfqpoint{1.832293in}{2.766051in}}{\pgfqpoint{1.824057in}{2.766051in}}%
\pgfpathcurveto{\pgfqpoint{1.815821in}{2.766051in}}{\pgfqpoint{1.807921in}{2.762779in}}{\pgfqpoint{1.802097in}{2.756955in}}%
\pgfpathcurveto{\pgfqpoint{1.796273in}{2.751131in}}{\pgfqpoint{1.793000in}{2.743231in}}{\pgfqpoint{1.793000in}{2.734994in}}%
\pgfpathcurveto{\pgfqpoint{1.793000in}{2.726758in}}{\pgfqpoint{1.796273in}{2.718858in}}{\pgfqpoint{1.802097in}{2.713034in}}%
\pgfpathcurveto{\pgfqpoint{1.807921in}{2.707210in}}{\pgfqpoint{1.815821in}{2.703938in}}{\pgfqpoint{1.824057in}{2.703938in}}%
\pgfpathclose%
\pgfusepath{stroke,fill}%
\end{pgfscope}%
\begin{pgfscope}%
\pgfpathrectangle{\pgfqpoint{0.100000in}{0.212622in}}{\pgfqpoint{3.696000in}{3.696000in}}%
\pgfusepath{clip}%
\pgfsetbuttcap%
\pgfsetroundjoin%
\definecolor{currentfill}{rgb}{0.121569,0.466667,0.705882}%
\pgfsetfillcolor{currentfill}%
\pgfsetfillopacity{0.866632}%
\pgfsetlinewidth{1.003750pt}%
\definecolor{currentstroke}{rgb}{0.121569,0.466667,0.705882}%
\pgfsetstrokecolor{currentstroke}%
\pgfsetstrokeopacity{0.866632}%
\pgfsetdash{}{0pt}%
\pgfpathmoveto{\pgfqpoint{0.966277in}{2.158261in}}%
\pgfpathcurveto{\pgfqpoint{0.974514in}{2.158261in}}{\pgfqpoint{0.982414in}{2.161533in}}{\pgfqpoint{0.988238in}{2.167357in}}%
\pgfpathcurveto{\pgfqpoint{0.994062in}{2.173181in}}{\pgfqpoint{0.997334in}{2.181081in}}{\pgfqpoint{0.997334in}{2.189317in}}%
\pgfpathcurveto{\pgfqpoint{0.997334in}{2.197553in}}{\pgfqpoint{0.994062in}{2.205453in}}{\pgfqpoint{0.988238in}{2.211277in}}%
\pgfpathcurveto{\pgfqpoint{0.982414in}{2.217101in}}{\pgfqpoint{0.974514in}{2.220374in}}{\pgfqpoint{0.966277in}{2.220374in}}%
\pgfpathcurveto{\pgfqpoint{0.958041in}{2.220374in}}{\pgfqpoint{0.950141in}{2.217101in}}{\pgfqpoint{0.944317in}{2.211277in}}%
\pgfpathcurveto{\pgfqpoint{0.938493in}{2.205453in}}{\pgfqpoint{0.935221in}{2.197553in}}{\pgfqpoint{0.935221in}{2.189317in}}%
\pgfpathcurveto{\pgfqpoint{0.935221in}{2.181081in}}{\pgfqpoint{0.938493in}{2.173181in}}{\pgfqpoint{0.944317in}{2.167357in}}%
\pgfpathcurveto{\pgfqpoint{0.950141in}{2.161533in}}{\pgfqpoint{0.958041in}{2.158261in}}{\pgfqpoint{0.966277in}{2.158261in}}%
\pgfpathclose%
\pgfusepath{stroke,fill}%
\end{pgfscope}%
\begin{pgfscope}%
\pgfpathrectangle{\pgfqpoint{0.100000in}{0.212622in}}{\pgfqpoint{3.696000in}{3.696000in}}%
\pgfusepath{clip}%
\pgfsetbuttcap%
\pgfsetroundjoin%
\definecolor{currentfill}{rgb}{0.121569,0.466667,0.705882}%
\pgfsetfillcolor{currentfill}%
\pgfsetfillopacity{0.866837}%
\pgfsetlinewidth{1.003750pt}%
\definecolor{currentstroke}{rgb}{0.121569,0.466667,0.705882}%
\pgfsetstrokecolor{currentstroke}%
\pgfsetstrokeopacity{0.866837}%
\pgfsetdash{}{0pt}%
\pgfpathmoveto{\pgfqpoint{2.265793in}{2.571661in}}%
\pgfpathcurveto{\pgfqpoint{2.274029in}{2.571661in}}{\pgfqpoint{2.281930in}{2.574934in}}{\pgfqpoint{2.287753in}{2.580758in}}%
\pgfpathcurveto{\pgfqpoint{2.293577in}{2.586582in}}{\pgfqpoint{2.296850in}{2.594482in}}{\pgfqpoint{2.296850in}{2.602718in}}%
\pgfpathcurveto{\pgfqpoint{2.296850in}{2.610954in}}{\pgfqpoint{2.293577in}{2.618854in}}{\pgfqpoint{2.287753in}{2.624678in}}%
\pgfpathcurveto{\pgfqpoint{2.281930in}{2.630502in}}{\pgfqpoint{2.274029in}{2.633774in}}{\pgfqpoint{2.265793in}{2.633774in}}%
\pgfpathcurveto{\pgfqpoint{2.257557in}{2.633774in}}{\pgfqpoint{2.249657in}{2.630502in}}{\pgfqpoint{2.243833in}{2.624678in}}%
\pgfpathcurveto{\pgfqpoint{2.238009in}{2.618854in}}{\pgfqpoint{2.234737in}{2.610954in}}{\pgfqpoint{2.234737in}{2.602718in}}%
\pgfpathcurveto{\pgfqpoint{2.234737in}{2.594482in}}{\pgfqpoint{2.238009in}{2.586582in}}{\pgfqpoint{2.243833in}{2.580758in}}%
\pgfpathcurveto{\pgfqpoint{2.249657in}{2.574934in}}{\pgfqpoint{2.257557in}{2.571661in}}{\pgfqpoint{2.265793in}{2.571661in}}%
\pgfpathclose%
\pgfusepath{stroke,fill}%
\end{pgfscope}%
\begin{pgfscope}%
\pgfpathrectangle{\pgfqpoint{0.100000in}{0.212622in}}{\pgfqpoint{3.696000in}{3.696000in}}%
\pgfusepath{clip}%
\pgfsetbuttcap%
\pgfsetroundjoin%
\definecolor{currentfill}{rgb}{0.121569,0.466667,0.705882}%
\pgfsetfillcolor{currentfill}%
\pgfsetfillopacity{0.867221}%
\pgfsetlinewidth{1.003750pt}%
\definecolor{currentstroke}{rgb}{0.121569,0.466667,0.705882}%
\pgfsetstrokecolor{currentstroke}%
\pgfsetstrokeopacity{0.867221}%
\pgfsetdash{}{0pt}%
\pgfpathmoveto{\pgfqpoint{1.822800in}{2.702349in}}%
\pgfpathcurveto{\pgfqpoint{1.831037in}{2.702349in}}{\pgfqpoint{1.838937in}{2.705621in}}{\pgfqpoint{1.844761in}{2.711445in}}%
\pgfpathcurveto{\pgfqpoint{1.850585in}{2.717269in}}{\pgfqpoint{1.853857in}{2.725169in}}{\pgfqpoint{1.853857in}{2.733405in}}%
\pgfpathcurveto{\pgfqpoint{1.853857in}{2.741642in}}{\pgfqpoint{1.850585in}{2.749542in}}{\pgfqpoint{1.844761in}{2.755366in}}%
\pgfpathcurveto{\pgfqpoint{1.838937in}{2.761190in}}{\pgfqpoint{1.831037in}{2.764462in}}{\pgfqpoint{1.822800in}{2.764462in}}%
\pgfpathcurveto{\pgfqpoint{1.814564in}{2.764462in}}{\pgfqpoint{1.806664in}{2.761190in}}{\pgfqpoint{1.800840in}{2.755366in}}%
\pgfpathcurveto{\pgfqpoint{1.795016in}{2.749542in}}{\pgfqpoint{1.791744in}{2.741642in}}{\pgfqpoint{1.791744in}{2.733405in}}%
\pgfpathcurveto{\pgfqpoint{1.791744in}{2.725169in}}{\pgfqpoint{1.795016in}{2.717269in}}{\pgfqpoint{1.800840in}{2.711445in}}%
\pgfpathcurveto{\pgfqpoint{1.806664in}{2.705621in}}{\pgfqpoint{1.814564in}{2.702349in}}{\pgfqpoint{1.822800in}{2.702349in}}%
\pgfpathclose%
\pgfusepath{stroke,fill}%
\end{pgfscope}%
\begin{pgfscope}%
\pgfpathrectangle{\pgfqpoint{0.100000in}{0.212622in}}{\pgfqpoint{3.696000in}{3.696000in}}%
\pgfusepath{clip}%
\pgfsetbuttcap%
\pgfsetroundjoin%
\definecolor{currentfill}{rgb}{0.121569,0.466667,0.705882}%
\pgfsetfillcolor{currentfill}%
\pgfsetfillopacity{0.867411}%
\pgfsetlinewidth{1.003750pt}%
\definecolor{currentstroke}{rgb}{0.121569,0.466667,0.705882}%
\pgfsetstrokecolor{currentstroke}%
\pgfsetstrokeopacity{0.867411}%
\pgfsetdash{}{0pt}%
\pgfpathmoveto{\pgfqpoint{0.963910in}{2.155373in}}%
\pgfpathcurveto{\pgfqpoint{0.972146in}{2.155373in}}{\pgfqpoint{0.980046in}{2.158645in}}{\pgfqpoint{0.985870in}{2.164469in}}%
\pgfpathcurveto{\pgfqpoint{0.991694in}{2.170293in}}{\pgfqpoint{0.994966in}{2.178193in}}{\pgfqpoint{0.994966in}{2.186430in}}%
\pgfpathcurveto{\pgfqpoint{0.994966in}{2.194666in}}{\pgfqpoint{0.991694in}{2.202566in}}{\pgfqpoint{0.985870in}{2.208390in}}%
\pgfpathcurveto{\pgfqpoint{0.980046in}{2.214214in}}{\pgfqpoint{0.972146in}{2.217486in}}{\pgfqpoint{0.963910in}{2.217486in}}%
\pgfpathcurveto{\pgfqpoint{0.955673in}{2.217486in}}{\pgfqpoint{0.947773in}{2.214214in}}{\pgfqpoint{0.941949in}{2.208390in}}%
\pgfpathcurveto{\pgfqpoint{0.936126in}{2.202566in}}{\pgfqpoint{0.932853in}{2.194666in}}{\pgfqpoint{0.932853in}{2.186430in}}%
\pgfpathcurveto{\pgfqpoint{0.932853in}{2.178193in}}{\pgfqpoint{0.936126in}{2.170293in}}{\pgfqpoint{0.941949in}{2.164469in}}%
\pgfpathcurveto{\pgfqpoint{0.947773in}{2.158645in}}{\pgfqpoint{0.955673in}{2.155373in}}{\pgfqpoint{0.963910in}{2.155373in}}%
\pgfpathclose%
\pgfusepath{stroke,fill}%
\end{pgfscope}%
\begin{pgfscope}%
\pgfpathrectangle{\pgfqpoint{0.100000in}{0.212622in}}{\pgfqpoint{3.696000in}{3.696000in}}%
\pgfusepath{clip}%
\pgfsetbuttcap%
\pgfsetroundjoin%
\definecolor{currentfill}{rgb}{0.121569,0.466667,0.705882}%
\pgfsetfillcolor{currentfill}%
\pgfsetfillopacity{0.867657}%
\pgfsetlinewidth{1.003750pt}%
\definecolor{currentstroke}{rgb}{0.121569,0.466667,0.705882}%
\pgfsetstrokecolor{currentstroke}%
\pgfsetstrokeopacity{0.867657}%
\pgfsetdash{}{0pt}%
\pgfpathmoveto{\pgfqpoint{2.263915in}{2.569713in}}%
\pgfpathcurveto{\pgfqpoint{2.272151in}{2.569713in}}{\pgfqpoint{2.280051in}{2.572985in}}{\pgfqpoint{2.285875in}{2.578809in}}%
\pgfpathcurveto{\pgfqpoint{2.291699in}{2.584633in}}{\pgfqpoint{2.294972in}{2.592533in}}{\pgfqpoint{2.294972in}{2.600769in}}%
\pgfpathcurveto{\pgfqpoint{2.294972in}{2.609006in}}{\pgfqpoint{2.291699in}{2.616906in}}{\pgfqpoint{2.285875in}{2.622730in}}%
\pgfpathcurveto{\pgfqpoint{2.280051in}{2.628554in}}{\pgfqpoint{2.272151in}{2.631826in}}{\pgfqpoint{2.263915in}{2.631826in}}%
\pgfpathcurveto{\pgfqpoint{2.255679in}{2.631826in}}{\pgfqpoint{2.247779in}{2.628554in}}{\pgfqpoint{2.241955in}{2.622730in}}%
\pgfpathcurveto{\pgfqpoint{2.236131in}{2.616906in}}{\pgfqpoint{2.232859in}{2.609006in}}{\pgfqpoint{2.232859in}{2.600769in}}%
\pgfpathcurveto{\pgfqpoint{2.232859in}{2.592533in}}{\pgfqpoint{2.236131in}{2.584633in}}{\pgfqpoint{2.241955in}{2.578809in}}%
\pgfpathcurveto{\pgfqpoint{2.247779in}{2.572985in}}{\pgfqpoint{2.255679in}{2.569713in}}{\pgfqpoint{2.263915in}{2.569713in}}%
\pgfpathclose%
\pgfusepath{stroke,fill}%
\end{pgfscope}%
\begin{pgfscope}%
\pgfpathrectangle{\pgfqpoint{0.100000in}{0.212622in}}{\pgfqpoint{3.696000in}{3.696000in}}%
\pgfusepath{clip}%
\pgfsetbuttcap%
\pgfsetroundjoin%
\definecolor{currentfill}{rgb}{0.121569,0.466667,0.705882}%
\pgfsetfillcolor{currentfill}%
\pgfsetfillopacity{0.868120}%
\pgfsetlinewidth{1.003750pt}%
\definecolor{currentstroke}{rgb}{0.121569,0.466667,0.705882}%
\pgfsetstrokecolor{currentstroke}%
\pgfsetstrokeopacity{0.868120}%
\pgfsetdash{}{0pt}%
\pgfpathmoveto{\pgfqpoint{1.821094in}{2.700023in}}%
\pgfpathcurveto{\pgfqpoint{1.829330in}{2.700023in}}{\pgfqpoint{1.837230in}{2.703295in}}{\pgfqpoint{1.843054in}{2.709119in}}%
\pgfpathcurveto{\pgfqpoint{1.848878in}{2.714943in}}{\pgfqpoint{1.852150in}{2.722843in}}{\pgfqpoint{1.852150in}{2.731080in}}%
\pgfpathcurveto{\pgfqpoint{1.852150in}{2.739316in}}{\pgfqpoint{1.848878in}{2.747216in}}{\pgfqpoint{1.843054in}{2.753040in}}%
\pgfpathcurveto{\pgfqpoint{1.837230in}{2.758864in}}{\pgfqpoint{1.829330in}{2.762136in}}{\pgfqpoint{1.821094in}{2.762136in}}%
\pgfpathcurveto{\pgfqpoint{1.812858in}{2.762136in}}{\pgfqpoint{1.804958in}{2.758864in}}{\pgfqpoint{1.799134in}{2.753040in}}%
\pgfpathcurveto{\pgfqpoint{1.793310in}{2.747216in}}{\pgfqpoint{1.790037in}{2.739316in}}{\pgfqpoint{1.790037in}{2.731080in}}%
\pgfpathcurveto{\pgfqpoint{1.790037in}{2.722843in}}{\pgfqpoint{1.793310in}{2.714943in}}{\pgfqpoint{1.799134in}{2.709119in}}%
\pgfpathcurveto{\pgfqpoint{1.804958in}{2.703295in}}{\pgfqpoint{1.812858in}{2.700023in}}{\pgfqpoint{1.821094in}{2.700023in}}%
\pgfpathclose%
\pgfusepath{stroke,fill}%
\end{pgfscope}%
\begin{pgfscope}%
\pgfpathrectangle{\pgfqpoint{0.100000in}{0.212622in}}{\pgfqpoint{3.696000in}{3.696000in}}%
\pgfusepath{clip}%
\pgfsetbuttcap%
\pgfsetroundjoin%
\definecolor{currentfill}{rgb}{0.121569,0.466667,0.705882}%
\pgfsetfillcolor{currentfill}%
\pgfsetfillopacity{0.868261}%
\pgfsetlinewidth{1.003750pt}%
\definecolor{currentstroke}{rgb}{0.121569,0.466667,0.705882}%
\pgfsetstrokecolor{currentstroke}%
\pgfsetstrokeopacity{0.868261}%
\pgfsetdash{}{0pt}%
\pgfpathmoveto{\pgfqpoint{2.262461in}{2.568024in}}%
\pgfpathcurveto{\pgfqpoint{2.270697in}{2.568024in}}{\pgfqpoint{2.278597in}{2.571296in}}{\pgfqpoint{2.284421in}{2.577120in}}%
\pgfpathcurveto{\pgfqpoint{2.290245in}{2.582944in}}{\pgfqpoint{2.293517in}{2.590844in}}{\pgfqpoint{2.293517in}{2.599081in}}%
\pgfpathcurveto{\pgfqpoint{2.293517in}{2.607317in}}{\pgfqpoint{2.290245in}{2.615217in}}{\pgfqpoint{2.284421in}{2.621041in}}%
\pgfpathcurveto{\pgfqpoint{2.278597in}{2.626865in}}{\pgfqpoint{2.270697in}{2.630137in}}{\pgfqpoint{2.262461in}{2.630137in}}%
\pgfpathcurveto{\pgfqpoint{2.254224in}{2.630137in}}{\pgfqpoint{2.246324in}{2.626865in}}{\pgfqpoint{2.240500in}{2.621041in}}%
\pgfpathcurveto{\pgfqpoint{2.234677in}{2.615217in}}{\pgfqpoint{2.231404in}{2.607317in}}{\pgfqpoint{2.231404in}{2.599081in}}%
\pgfpathcurveto{\pgfqpoint{2.231404in}{2.590844in}}{\pgfqpoint{2.234677in}{2.582944in}}{\pgfqpoint{2.240500in}{2.577120in}}%
\pgfpathcurveto{\pgfqpoint{2.246324in}{2.571296in}}{\pgfqpoint{2.254224in}{2.568024in}}{\pgfqpoint{2.262461in}{2.568024in}}%
\pgfpathclose%
\pgfusepath{stroke,fill}%
\end{pgfscope}%
\begin{pgfscope}%
\pgfpathrectangle{\pgfqpoint{0.100000in}{0.212622in}}{\pgfqpoint{3.696000in}{3.696000in}}%
\pgfusepath{clip}%
\pgfsetbuttcap%
\pgfsetroundjoin%
\definecolor{currentfill}{rgb}{0.121569,0.466667,0.705882}%
\pgfsetfillcolor{currentfill}%
\pgfsetfillopacity{0.868415}%
\pgfsetlinewidth{1.003750pt}%
\definecolor{currentstroke}{rgb}{0.121569,0.466667,0.705882}%
\pgfsetstrokecolor{currentstroke}%
\pgfsetstrokeopacity{0.868415}%
\pgfsetdash{}{0pt}%
\pgfpathmoveto{\pgfqpoint{0.960833in}{2.151655in}}%
\pgfpathcurveto{\pgfqpoint{0.969069in}{2.151655in}}{\pgfqpoint{0.976969in}{2.154927in}}{\pgfqpoint{0.982793in}{2.160751in}}%
\pgfpathcurveto{\pgfqpoint{0.988617in}{2.166575in}}{\pgfqpoint{0.991889in}{2.174475in}}{\pgfqpoint{0.991889in}{2.182711in}}%
\pgfpathcurveto{\pgfqpoint{0.991889in}{2.190947in}}{\pgfqpoint{0.988617in}{2.198847in}}{\pgfqpoint{0.982793in}{2.204671in}}%
\pgfpathcurveto{\pgfqpoint{0.976969in}{2.210495in}}{\pgfqpoint{0.969069in}{2.213768in}}{\pgfqpoint{0.960833in}{2.213768in}}%
\pgfpathcurveto{\pgfqpoint{0.952597in}{2.213768in}}{\pgfqpoint{0.944697in}{2.210495in}}{\pgfqpoint{0.938873in}{2.204671in}}%
\pgfpathcurveto{\pgfqpoint{0.933049in}{2.198847in}}{\pgfqpoint{0.929776in}{2.190947in}}{\pgfqpoint{0.929776in}{2.182711in}}%
\pgfpathcurveto{\pgfqpoint{0.929776in}{2.174475in}}{\pgfqpoint{0.933049in}{2.166575in}}{\pgfqpoint{0.938873in}{2.160751in}}%
\pgfpathcurveto{\pgfqpoint{0.944697in}{2.154927in}}{\pgfqpoint{0.952597in}{2.151655in}}{\pgfqpoint{0.960833in}{2.151655in}}%
\pgfpathclose%
\pgfusepath{stroke,fill}%
\end{pgfscope}%
\begin{pgfscope}%
\pgfpathrectangle{\pgfqpoint{0.100000in}{0.212622in}}{\pgfqpoint{3.696000in}{3.696000in}}%
\pgfusepath{clip}%
\pgfsetbuttcap%
\pgfsetroundjoin%
\definecolor{currentfill}{rgb}{0.121569,0.466667,0.705882}%
\pgfsetfillcolor{currentfill}%
\pgfsetfillopacity{0.868911}%
\pgfsetlinewidth{1.003750pt}%
\definecolor{currentstroke}{rgb}{0.121569,0.466667,0.705882}%
\pgfsetstrokecolor{currentstroke}%
\pgfsetstrokeopacity{0.868911}%
\pgfsetdash{}{0pt}%
\pgfpathmoveto{\pgfqpoint{2.739995in}{1.382187in}}%
\pgfpathcurveto{\pgfqpoint{2.748231in}{1.382187in}}{\pgfqpoint{2.756131in}{1.385459in}}{\pgfqpoint{2.761955in}{1.391283in}}%
\pgfpathcurveto{\pgfqpoint{2.767779in}{1.397107in}}{\pgfqpoint{2.771051in}{1.405007in}}{\pgfqpoint{2.771051in}{1.413243in}}%
\pgfpathcurveto{\pgfqpoint{2.771051in}{1.421480in}}{\pgfqpoint{2.767779in}{1.429380in}}{\pgfqpoint{2.761955in}{1.435204in}}%
\pgfpathcurveto{\pgfqpoint{2.756131in}{1.441028in}}{\pgfqpoint{2.748231in}{1.444300in}}{\pgfqpoint{2.739995in}{1.444300in}}%
\pgfpathcurveto{\pgfqpoint{2.731758in}{1.444300in}}{\pgfqpoint{2.723858in}{1.441028in}}{\pgfqpoint{2.718034in}{1.435204in}}%
\pgfpathcurveto{\pgfqpoint{2.712210in}{1.429380in}}{\pgfqpoint{2.708938in}{1.421480in}}{\pgfqpoint{2.708938in}{1.413243in}}%
\pgfpathcurveto{\pgfqpoint{2.708938in}{1.405007in}}{\pgfqpoint{2.712210in}{1.397107in}}{\pgfqpoint{2.718034in}{1.391283in}}%
\pgfpathcurveto{\pgfqpoint{2.723858in}{1.385459in}}{\pgfqpoint{2.731758in}{1.382187in}}{\pgfqpoint{2.739995in}{1.382187in}}%
\pgfpathclose%
\pgfusepath{stroke,fill}%
\end{pgfscope}%
\begin{pgfscope}%
\pgfpathrectangle{\pgfqpoint{0.100000in}{0.212622in}}{\pgfqpoint{3.696000in}{3.696000in}}%
\pgfusepath{clip}%
\pgfsetbuttcap%
\pgfsetroundjoin%
\definecolor{currentfill}{rgb}{0.121569,0.466667,0.705882}%
\pgfsetfillcolor{currentfill}%
\pgfsetfillopacity{0.869144}%
\pgfsetlinewidth{1.003750pt}%
\definecolor{currentstroke}{rgb}{0.121569,0.466667,0.705882}%
\pgfsetstrokecolor{currentstroke}%
\pgfsetstrokeopacity{0.869144}%
\pgfsetdash{}{0pt}%
\pgfpathmoveto{\pgfqpoint{1.819136in}{2.697367in}}%
\pgfpathcurveto{\pgfqpoint{1.827372in}{2.697367in}}{\pgfqpoint{1.835272in}{2.700639in}}{\pgfqpoint{1.841096in}{2.706463in}}%
\pgfpathcurveto{\pgfqpoint{1.846920in}{2.712287in}}{\pgfqpoint{1.850192in}{2.720187in}}{\pgfqpoint{1.850192in}{2.728423in}}%
\pgfpathcurveto{\pgfqpoint{1.850192in}{2.736659in}}{\pgfqpoint{1.846920in}{2.744559in}}{\pgfqpoint{1.841096in}{2.750383in}}%
\pgfpathcurveto{\pgfqpoint{1.835272in}{2.756207in}}{\pgfqpoint{1.827372in}{2.759480in}}{\pgfqpoint{1.819136in}{2.759480in}}%
\pgfpathcurveto{\pgfqpoint{1.810899in}{2.759480in}}{\pgfqpoint{1.802999in}{2.756207in}}{\pgfqpoint{1.797175in}{2.750383in}}%
\pgfpathcurveto{\pgfqpoint{1.791352in}{2.744559in}}{\pgfqpoint{1.788079in}{2.736659in}}{\pgfqpoint{1.788079in}{2.728423in}}%
\pgfpathcurveto{\pgfqpoint{1.788079in}{2.720187in}}{\pgfqpoint{1.791352in}{2.712287in}}{\pgfqpoint{1.797175in}{2.706463in}}%
\pgfpathcurveto{\pgfqpoint{1.802999in}{2.700639in}}{\pgfqpoint{1.810899in}{2.697367in}}{\pgfqpoint{1.819136in}{2.697367in}}%
\pgfpathclose%
\pgfusepath{stroke,fill}%
\end{pgfscope}%
\begin{pgfscope}%
\pgfpathrectangle{\pgfqpoint{0.100000in}{0.212622in}}{\pgfqpoint{3.696000in}{3.696000in}}%
\pgfusepath{clip}%
\pgfsetbuttcap%
\pgfsetroundjoin%
\definecolor{currentfill}{rgb}{0.121569,0.466667,0.705882}%
\pgfsetfillcolor{currentfill}%
\pgfsetfillopacity{0.869348}%
\pgfsetlinewidth{1.003750pt}%
\definecolor{currentstroke}{rgb}{0.121569,0.466667,0.705882}%
\pgfsetstrokecolor{currentstroke}%
\pgfsetstrokeopacity{0.869348}%
\pgfsetdash{}{0pt}%
\pgfpathmoveto{\pgfqpoint{2.259768in}{2.564942in}}%
\pgfpathcurveto{\pgfqpoint{2.268004in}{2.564942in}}{\pgfqpoint{2.275904in}{2.568214in}}{\pgfqpoint{2.281728in}{2.574038in}}%
\pgfpathcurveto{\pgfqpoint{2.287552in}{2.579862in}}{\pgfqpoint{2.290825in}{2.587762in}}{\pgfqpoint{2.290825in}{2.595998in}}%
\pgfpathcurveto{\pgfqpoint{2.290825in}{2.604235in}}{\pgfqpoint{2.287552in}{2.612135in}}{\pgfqpoint{2.281728in}{2.617958in}}%
\pgfpathcurveto{\pgfqpoint{2.275904in}{2.623782in}}{\pgfqpoint{2.268004in}{2.627055in}}{\pgfqpoint{2.259768in}{2.627055in}}%
\pgfpathcurveto{\pgfqpoint{2.251532in}{2.627055in}}{\pgfqpoint{2.243632in}{2.623782in}}{\pgfqpoint{2.237808in}{2.617958in}}%
\pgfpathcurveto{\pgfqpoint{2.231984in}{2.612135in}}{\pgfqpoint{2.228712in}{2.604235in}}{\pgfqpoint{2.228712in}{2.595998in}}%
\pgfpathcurveto{\pgfqpoint{2.228712in}{2.587762in}}{\pgfqpoint{2.231984in}{2.579862in}}{\pgfqpoint{2.237808in}{2.574038in}}%
\pgfpathcurveto{\pgfqpoint{2.243632in}{2.568214in}}{\pgfqpoint{2.251532in}{2.564942in}}{\pgfqpoint{2.259768in}{2.564942in}}%
\pgfpathclose%
\pgfusepath{stroke,fill}%
\end{pgfscope}%
\begin{pgfscope}%
\pgfpathrectangle{\pgfqpoint{0.100000in}{0.212622in}}{\pgfqpoint{3.696000in}{3.696000in}}%
\pgfusepath{clip}%
\pgfsetbuttcap%
\pgfsetroundjoin%
\definecolor{currentfill}{rgb}{0.121569,0.466667,0.705882}%
\pgfsetfillcolor{currentfill}%
\pgfsetfillopacity{0.869604}%
\pgfsetlinewidth{1.003750pt}%
\definecolor{currentstroke}{rgb}{0.121569,0.466667,0.705882}%
\pgfsetstrokecolor{currentstroke}%
\pgfsetstrokeopacity{0.869604}%
\pgfsetdash{}{0pt}%
\pgfpathmoveto{\pgfqpoint{0.957462in}{2.147725in}}%
\pgfpathcurveto{\pgfqpoint{0.965698in}{2.147725in}}{\pgfqpoint{0.973598in}{2.150997in}}{\pgfqpoint{0.979422in}{2.156821in}}%
\pgfpathcurveto{\pgfqpoint{0.985246in}{2.162645in}}{\pgfqpoint{0.988519in}{2.170545in}}{\pgfqpoint{0.988519in}{2.178781in}}%
\pgfpathcurveto{\pgfqpoint{0.988519in}{2.187018in}}{\pgfqpoint{0.985246in}{2.194918in}}{\pgfqpoint{0.979422in}{2.200742in}}%
\pgfpathcurveto{\pgfqpoint{0.973598in}{2.206565in}}{\pgfqpoint{0.965698in}{2.209838in}}{\pgfqpoint{0.957462in}{2.209838in}}%
\pgfpathcurveto{\pgfqpoint{0.949226in}{2.209838in}}{\pgfqpoint{0.941326in}{2.206565in}}{\pgfqpoint{0.935502in}{2.200742in}}%
\pgfpathcurveto{\pgfqpoint{0.929678in}{2.194918in}}{\pgfqpoint{0.926406in}{2.187018in}}{\pgfqpoint{0.926406in}{2.178781in}}%
\pgfpathcurveto{\pgfqpoint{0.926406in}{2.170545in}}{\pgfqpoint{0.929678in}{2.162645in}}{\pgfqpoint{0.935502in}{2.156821in}}%
\pgfpathcurveto{\pgfqpoint{0.941326in}{2.150997in}}{\pgfqpoint{0.949226in}{2.147725in}}{\pgfqpoint{0.957462in}{2.147725in}}%
\pgfpathclose%
\pgfusepath{stroke,fill}%
\end{pgfscope}%
\begin{pgfscope}%
\pgfpathrectangle{\pgfqpoint{0.100000in}{0.212622in}}{\pgfqpoint{3.696000in}{3.696000in}}%
\pgfusepath{clip}%
\pgfsetbuttcap%
\pgfsetroundjoin%
\definecolor{currentfill}{rgb}{0.121569,0.466667,0.705882}%
\pgfsetfillcolor{currentfill}%
\pgfsetfillopacity{0.870162}%
\pgfsetlinewidth{1.003750pt}%
\definecolor{currentstroke}{rgb}{0.121569,0.466667,0.705882}%
\pgfsetstrokecolor{currentstroke}%
\pgfsetstrokeopacity{0.870162}%
\pgfsetdash{}{0pt}%
\pgfpathmoveto{\pgfqpoint{2.257875in}{2.562990in}}%
\pgfpathcurveto{\pgfqpoint{2.266112in}{2.562990in}}{\pgfqpoint{2.274012in}{2.566262in}}{\pgfqpoint{2.279836in}{2.572086in}}%
\pgfpathcurveto{\pgfqpoint{2.285660in}{2.577910in}}{\pgfqpoint{2.288932in}{2.585810in}}{\pgfqpoint{2.288932in}{2.594046in}}%
\pgfpathcurveto{\pgfqpoint{2.288932in}{2.602282in}}{\pgfqpoint{2.285660in}{2.610183in}}{\pgfqpoint{2.279836in}{2.616006in}}%
\pgfpathcurveto{\pgfqpoint{2.274012in}{2.621830in}}{\pgfqpoint{2.266112in}{2.625103in}}{\pgfqpoint{2.257875in}{2.625103in}}%
\pgfpathcurveto{\pgfqpoint{2.249639in}{2.625103in}}{\pgfqpoint{2.241739in}{2.621830in}}{\pgfqpoint{2.235915in}{2.616006in}}%
\pgfpathcurveto{\pgfqpoint{2.230091in}{2.610183in}}{\pgfqpoint{2.226819in}{2.602282in}}{\pgfqpoint{2.226819in}{2.594046in}}%
\pgfpathcurveto{\pgfqpoint{2.226819in}{2.585810in}}{\pgfqpoint{2.230091in}{2.577910in}}{\pgfqpoint{2.235915in}{2.572086in}}%
\pgfpathcurveto{\pgfqpoint{2.241739in}{2.566262in}}{\pgfqpoint{2.249639in}{2.562990in}}{\pgfqpoint{2.257875in}{2.562990in}}%
\pgfpathclose%
\pgfusepath{stroke,fill}%
\end{pgfscope}%
\begin{pgfscope}%
\pgfpathrectangle{\pgfqpoint{0.100000in}{0.212622in}}{\pgfqpoint{3.696000in}{3.696000in}}%
\pgfusepath{clip}%
\pgfsetbuttcap%
\pgfsetroundjoin%
\definecolor{currentfill}{rgb}{0.121569,0.466667,0.705882}%
\pgfsetfillcolor{currentfill}%
\pgfsetfillopacity{0.870256}%
\pgfsetlinewidth{1.003750pt}%
\definecolor{currentstroke}{rgb}{0.121569,0.466667,0.705882}%
\pgfsetstrokecolor{currentstroke}%
\pgfsetstrokeopacity{0.870256}%
\pgfsetdash{}{0pt}%
\pgfpathmoveto{\pgfqpoint{1.816921in}{2.694393in}}%
\pgfpathcurveto{\pgfqpoint{1.825157in}{2.694393in}}{\pgfqpoint{1.833057in}{2.697665in}}{\pgfqpoint{1.838881in}{2.703489in}}%
\pgfpathcurveto{\pgfqpoint{1.844705in}{2.709313in}}{\pgfqpoint{1.847977in}{2.717213in}}{\pgfqpoint{1.847977in}{2.725449in}}%
\pgfpathcurveto{\pgfqpoint{1.847977in}{2.733686in}}{\pgfqpoint{1.844705in}{2.741586in}}{\pgfqpoint{1.838881in}{2.747410in}}%
\pgfpathcurveto{\pgfqpoint{1.833057in}{2.753234in}}{\pgfqpoint{1.825157in}{2.756506in}}{\pgfqpoint{1.816921in}{2.756506in}}%
\pgfpathcurveto{\pgfqpoint{1.808685in}{2.756506in}}{\pgfqpoint{1.800784in}{2.753234in}}{\pgfqpoint{1.794961in}{2.747410in}}%
\pgfpathcurveto{\pgfqpoint{1.789137in}{2.741586in}}{\pgfqpoint{1.785864in}{2.733686in}}{\pgfqpoint{1.785864in}{2.725449in}}%
\pgfpathcurveto{\pgfqpoint{1.785864in}{2.717213in}}{\pgfqpoint{1.789137in}{2.709313in}}{\pgfqpoint{1.794961in}{2.703489in}}%
\pgfpathcurveto{\pgfqpoint{1.800784in}{2.697665in}}{\pgfqpoint{1.808685in}{2.694393in}}{\pgfqpoint{1.816921in}{2.694393in}}%
\pgfpathclose%
\pgfusepath{stroke,fill}%
\end{pgfscope}%
\begin{pgfscope}%
\pgfpathrectangle{\pgfqpoint{0.100000in}{0.212622in}}{\pgfqpoint{3.696000in}{3.696000in}}%
\pgfusepath{clip}%
\pgfsetbuttcap%
\pgfsetroundjoin%
\definecolor{currentfill}{rgb}{0.121569,0.466667,0.705882}%
\pgfsetfillcolor{currentfill}%
\pgfsetfillopacity{0.870312}%
\pgfsetlinewidth{1.003750pt}%
\definecolor{currentstroke}{rgb}{0.121569,0.466667,0.705882}%
\pgfsetstrokecolor{currentstroke}%
\pgfsetstrokeopacity{0.870312}%
\pgfsetdash{}{0pt}%
\pgfpathmoveto{\pgfqpoint{0.955626in}{2.145842in}}%
\pgfpathcurveto{\pgfqpoint{0.963863in}{2.145842in}}{\pgfqpoint{0.971763in}{2.149114in}}{\pgfqpoint{0.977587in}{2.154938in}}%
\pgfpathcurveto{\pgfqpoint{0.983411in}{2.160762in}}{\pgfqpoint{0.986683in}{2.168662in}}{\pgfqpoint{0.986683in}{2.176898in}}%
\pgfpathcurveto{\pgfqpoint{0.986683in}{2.185134in}}{\pgfqpoint{0.983411in}{2.193035in}}{\pgfqpoint{0.977587in}{2.198858in}}%
\pgfpathcurveto{\pgfqpoint{0.971763in}{2.204682in}}{\pgfqpoint{0.963863in}{2.207955in}}{\pgfqpoint{0.955626in}{2.207955in}}%
\pgfpathcurveto{\pgfqpoint{0.947390in}{2.207955in}}{\pgfqpoint{0.939490in}{2.204682in}}{\pgfqpoint{0.933666in}{2.198858in}}%
\pgfpathcurveto{\pgfqpoint{0.927842in}{2.193035in}}{\pgfqpoint{0.924570in}{2.185134in}}{\pgfqpoint{0.924570in}{2.176898in}}%
\pgfpathcurveto{\pgfqpoint{0.924570in}{2.168662in}}{\pgfqpoint{0.927842in}{2.160762in}}{\pgfqpoint{0.933666in}{2.154938in}}%
\pgfpathcurveto{\pgfqpoint{0.939490in}{2.149114in}}{\pgfqpoint{0.947390in}{2.145842in}}{\pgfqpoint{0.955626in}{2.145842in}}%
\pgfpathclose%
\pgfusepath{stroke,fill}%
\end{pgfscope}%
\begin{pgfscope}%
\pgfpathrectangle{\pgfqpoint{0.100000in}{0.212622in}}{\pgfqpoint{3.696000in}{3.696000in}}%
\pgfusepath{clip}%
\pgfsetbuttcap%
\pgfsetroundjoin%
\definecolor{currentfill}{rgb}{0.121569,0.466667,0.705882}%
\pgfsetfillcolor{currentfill}%
\pgfsetfillopacity{0.870702}%
\pgfsetlinewidth{1.003750pt}%
\definecolor{currentstroke}{rgb}{0.121569,0.466667,0.705882}%
\pgfsetstrokecolor{currentstroke}%
\pgfsetstrokeopacity{0.870702}%
\pgfsetdash{}{0pt}%
\pgfpathmoveto{\pgfqpoint{0.954600in}{2.144832in}}%
\pgfpathcurveto{\pgfqpoint{0.962836in}{2.144832in}}{\pgfqpoint{0.970736in}{2.148104in}}{\pgfqpoint{0.976560in}{2.153928in}}%
\pgfpathcurveto{\pgfqpoint{0.982384in}{2.159752in}}{\pgfqpoint{0.985657in}{2.167652in}}{\pgfqpoint{0.985657in}{2.175888in}}%
\pgfpathcurveto{\pgfqpoint{0.985657in}{2.184124in}}{\pgfqpoint{0.982384in}{2.192024in}}{\pgfqpoint{0.976560in}{2.197848in}}%
\pgfpathcurveto{\pgfqpoint{0.970736in}{2.203672in}}{\pgfqpoint{0.962836in}{2.206945in}}{\pgfqpoint{0.954600in}{2.206945in}}%
\pgfpathcurveto{\pgfqpoint{0.946364in}{2.206945in}}{\pgfqpoint{0.938464in}{2.203672in}}{\pgfqpoint{0.932640in}{2.197848in}}%
\pgfpathcurveto{\pgfqpoint{0.926816in}{2.192024in}}{\pgfqpoint{0.923544in}{2.184124in}}{\pgfqpoint{0.923544in}{2.175888in}}%
\pgfpathcurveto{\pgfqpoint{0.923544in}{2.167652in}}{\pgfqpoint{0.926816in}{2.159752in}}{\pgfqpoint{0.932640in}{2.153928in}}%
\pgfpathcurveto{\pgfqpoint{0.938464in}{2.148104in}}{\pgfqpoint{0.946364in}{2.144832in}}{\pgfqpoint{0.954600in}{2.144832in}}%
\pgfpathclose%
\pgfusepath{stroke,fill}%
\end{pgfscope}%
\begin{pgfscope}%
\pgfpathrectangle{\pgfqpoint{0.100000in}{0.212622in}}{\pgfqpoint{3.696000in}{3.696000in}}%
\pgfusepath{clip}%
\pgfsetbuttcap%
\pgfsetroundjoin%
\definecolor{currentfill}{rgb}{0.121569,0.466667,0.705882}%
\pgfsetfillcolor{currentfill}%
\pgfsetfillopacity{0.870766}%
\pgfsetlinewidth{1.003750pt}%
\definecolor{currentstroke}{rgb}{0.121569,0.466667,0.705882}%
\pgfsetstrokecolor{currentstroke}%
\pgfsetstrokeopacity{0.870766}%
\pgfsetdash{}{0pt}%
\pgfpathmoveto{\pgfqpoint{2.256546in}{2.561688in}}%
\pgfpathcurveto{\pgfqpoint{2.264782in}{2.561688in}}{\pgfqpoint{2.272682in}{2.564960in}}{\pgfqpoint{2.278506in}{2.570784in}}%
\pgfpathcurveto{\pgfqpoint{2.284330in}{2.576608in}}{\pgfqpoint{2.287603in}{2.584508in}}{\pgfqpoint{2.287603in}{2.592744in}}%
\pgfpathcurveto{\pgfqpoint{2.287603in}{2.600980in}}{\pgfqpoint{2.284330in}{2.608880in}}{\pgfqpoint{2.278506in}{2.614704in}}%
\pgfpathcurveto{\pgfqpoint{2.272682in}{2.620528in}}{\pgfqpoint{2.264782in}{2.623801in}}{\pgfqpoint{2.256546in}{2.623801in}}%
\pgfpathcurveto{\pgfqpoint{2.248310in}{2.623801in}}{\pgfqpoint{2.240410in}{2.620528in}}{\pgfqpoint{2.234586in}{2.614704in}}%
\pgfpathcurveto{\pgfqpoint{2.228762in}{2.608880in}}{\pgfqpoint{2.225490in}{2.600980in}}{\pgfqpoint{2.225490in}{2.592744in}}%
\pgfpathcurveto{\pgfqpoint{2.225490in}{2.584508in}}{\pgfqpoint{2.228762in}{2.576608in}}{\pgfqpoint{2.234586in}{2.570784in}}%
\pgfpathcurveto{\pgfqpoint{2.240410in}{2.564960in}}{\pgfqpoint{2.248310in}{2.561688in}}{\pgfqpoint{2.256546in}{2.561688in}}%
\pgfpathclose%
\pgfusepath{stroke,fill}%
\end{pgfscope}%
\begin{pgfscope}%
\pgfpathrectangle{\pgfqpoint{0.100000in}{0.212622in}}{\pgfqpoint{3.696000in}{3.696000in}}%
\pgfusepath{clip}%
\pgfsetbuttcap%
\pgfsetroundjoin%
\definecolor{currentfill}{rgb}{0.121569,0.466667,0.705882}%
\pgfsetfillcolor{currentfill}%
\pgfsetfillopacity{0.871248}%
\pgfsetlinewidth{1.003750pt}%
\definecolor{currentstroke}{rgb}{0.121569,0.466667,0.705882}%
\pgfsetstrokecolor{currentstroke}%
\pgfsetstrokeopacity{0.871248}%
\pgfsetdash{}{0pt}%
\pgfpathmoveto{\pgfqpoint{0.953086in}{2.143224in}}%
\pgfpathcurveto{\pgfqpoint{0.961322in}{2.143224in}}{\pgfqpoint{0.969222in}{2.146496in}}{\pgfqpoint{0.975046in}{2.152320in}}%
\pgfpathcurveto{\pgfqpoint{0.980870in}{2.158144in}}{\pgfqpoint{0.984142in}{2.166044in}}{\pgfqpoint{0.984142in}{2.174280in}}%
\pgfpathcurveto{\pgfqpoint{0.984142in}{2.182516in}}{\pgfqpoint{0.980870in}{2.190416in}}{\pgfqpoint{0.975046in}{2.196240in}}%
\pgfpathcurveto{\pgfqpoint{0.969222in}{2.202064in}}{\pgfqpoint{0.961322in}{2.205337in}}{\pgfqpoint{0.953086in}{2.205337in}}%
\pgfpathcurveto{\pgfqpoint{0.944849in}{2.205337in}}{\pgfqpoint{0.936949in}{2.202064in}}{\pgfqpoint{0.931125in}{2.196240in}}%
\pgfpathcurveto{\pgfqpoint{0.925301in}{2.190416in}}{\pgfqpoint{0.922029in}{2.182516in}}{\pgfqpoint{0.922029in}{2.174280in}}%
\pgfpathcurveto{\pgfqpoint{0.922029in}{2.166044in}}{\pgfqpoint{0.925301in}{2.158144in}}{\pgfqpoint{0.931125in}{2.152320in}}%
\pgfpathcurveto{\pgfqpoint{0.936949in}{2.146496in}}{\pgfqpoint{0.944849in}{2.143224in}}{\pgfqpoint{0.953086in}{2.143224in}}%
\pgfpathclose%
\pgfusepath{stroke,fill}%
\end{pgfscope}%
\begin{pgfscope}%
\pgfpathrectangle{\pgfqpoint{0.100000in}{0.212622in}}{\pgfqpoint{3.696000in}{3.696000in}}%
\pgfusepath{clip}%
\pgfsetbuttcap%
\pgfsetroundjoin%
\definecolor{currentfill}{rgb}{0.121569,0.466667,0.705882}%
\pgfsetfillcolor{currentfill}%
\pgfsetfillopacity{0.871785}%
\pgfsetlinewidth{1.003750pt}%
\definecolor{currentstroke}{rgb}{0.121569,0.466667,0.705882}%
\pgfsetstrokecolor{currentstroke}%
\pgfsetstrokeopacity{0.871785}%
\pgfsetdash{}{0pt}%
\pgfpathmoveto{\pgfqpoint{1.813861in}{2.689956in}}%
\pgfpathcurveto{\pgfqpoint{1.822097in}{2.689956in}}{\pgfqpoint{1.829997in}{2.693228in}}{\pgfqpoint{1.835821in}{2.699052in}}%
\pgfpathcurveto{\pgfqpoint{1.841645in}{2.704876in}}{\pgfqpoint{1.844917in}{2.712776in}}{\pgfqpoint{1.844917in}{2.721012in}}%
\pgfpathcurveto{\pgfqpoint{1.844917in}{2.729248in}}{\pgfqpoint{1.841645in}{2.737149in}}{\pgfqpoint{1.835821in}{2.742972in}}%
\pgfpathcurveto{\pgfqpoint{1.829997in}{2.748796in}}{\pgfqpoint{1.822097in}{2.752069in}}{\pgfqpoint{1.813861in}{2.752069in}}%
\pgfpathcurveto{\pgfqpoint{1.805625in}{2.752069in}}{\pgfqpoint{1.797725in}{2.748796in}}{\pgfqpoint{1.791901in}{2.742972in}}%
\pgfpathcurveto{\pgfqpoint{1.786077in}{2.737149in}}{\pgfqpoint{1.782804in}{2.729248in}}{\pgfqpoint{1.782804in}{2.721012in}}%
\pgfpathcurveto{\pgfqpoint{1.782804in}{2.712776in}}{\pgfqpoint{1.786077in}{2.704876in}}{\pgfqpoint{1.791901in}{2.699052in}}%
\pgfpathcurveto{\pgfqpoint{1.797725in}{2.693228in}}{\pgfqpoint{1.805625in}{2.689956in}}{\pgfqpoint{1.813861in}{2.689956in}}%
\pgfpathclose%
\pgfusepath{stroke,fill}%
\end{pgfscope}%
\begin{pgfscope}%
\pgfpathrectangle{\pgfqpoint{0.100000in}{0.212622in}}{\pgfqpoint{3.696000in}{3.696000in}}%
\pgfusepath{clip}%
\pgfsetbuttcap%
\pgfsetroundjoin%
\definecolor{currentfill}{rgb}{0.121569,0.466667,0.705882}%
\pgfsetfillcolor{currentfill}%
\pgfsetfillopacity{0.871793}%
\pgfsetlinewidth{1.003750pt}%
\definecolor{currentstroke}{rgb}{0.121569,0.466667,0.705882}%
\pgfsetstrokecolor{currentstroke}%
\pgfsetstrokeopacity{0.871793}%
\pgfsetdash{}{0pt}%
\pgfpathmoveto{\pgfqpoint{2.254207in}{2.558820in}}%
\pgfpathcurveto{\pgfqpoint{2.262444in}{2.558820in}}{\pgfqpoint{2.270344in}{2.562092in}}{\pgfqpoint{2.276168in}{2.567916in}}%
\pgfpathcurveto{\pgfqpoint{2.281992in}{2.573740in}}{\pgfqpoint{2.285264in}{2.581640in}}{\pgfqpoint{2.285264in}{2.589876in}}%
\pgfpathcurveto{\pgfqpoint{2.285264in}{2.598113in}}{\pgfqpoint{2.281992in}{2.606013in}}{\pgfqpoint{2.276168in}{2.611837in}}%
\pgfpathcurveto{\pgfqpoint{2.270344in}{2.617661in}}{\pgfqpoint{2.262444in}{2.620933in}}{\pgfqpoint{2.254207in}{2.620933in}}%
\pgfpathcurveto{\pgfqpoint{2.245971in}{2.620933in}}{\pgfqpoint{2.238071in}{2.617661in}}{\pgfqpoint{2.232247in}{2.611837in}}%
\pgfpathcurveto{\pgfqpoint{2.226423in}{2.606013in}}{\pgfqpoint{2.223151in}{2.598113in}}{\pgfqpoint{2.223151in}{2.589876in}}%
\pgfpathcurveto{\pgfqpoint{2.223151in}{2.581640in}}{\pgfqpoint{2.226423in}{2.573740in}}{\pgfqpoint{2.232247in}{2.567916in}}%
\pgfpathcurveto{\pgfqpoint{2.238071in}{2.562092in}}{\pgfqpoint{2.245971in}{2.558820in}}{\pgfqpoint{2.254207in}{2.558820in}}%
\pgfpathclose%
\pgfusepath{stroke,fill}%
\end{pgfscope}%
\begin{pgfscope}%
\pgfpathrectangle{\pgfqpoint{0.100000in}{0.212622in}}{\pgfqpoint{3.696000in}{3.696000in}}%
\pgfusepath{clip}%
\pgfsetbuttcap%
\pgfsetroundjoin%
\definecolor{currentfill}{rgb}{0.121569,0.466667,0.705882}%
\pgfsetfillcolor{currentfill}%
\pgfsetfillopacity{0.871932}%
\pgfsetlinewidth{1.003750pt}%
\definecolor{currentstroke}{rgb}{0.121569,0.466667,0.705882}%
\pgfsetstrokecolor{currentstroke}%
\pgfsetstrokeopacity{0.871932}%
\pgfsetdash{}{0pt}%
\pgfpathmoveto{\pgfqpoint{0.951094in}{2.140881in}}%
\pgfpathcurveto{\pgfqpoint{0.959331in}{2.140881in}}{\pgfqpoint{0.967231in}{2.144153in}}{\pgfqpoint{0.973055in}{2.149977in}}%
\pgfpathcurveto{\pgfqpoint{0.978878in}{2.155801in}}{\pgfqpoint{0.982151in}{2.163701in}}{\pgfqpoint{0.982151in}{2.171938in}}%
\pgfpathcurveto{\pgfqpoint{0.982151in}{2.180174in}}{\pgfqpoint{0.978878in}{2.188074in}}{\pgfqpoint{0.973055in}{2.193898in}}%
\pgfpathcurveto{\pgfqpoint{0.967231in}{2.199722in}}{\pgfqpoint{0.959331in}{2.202994in}}{\pgfqpoint{0.951094in}{2.202994in}}%
\pgfpathcurveto{\pgfqpoint{0.942858in}{2.202994in}}{\pgfqpoint{0.934958in}{2.199722in}}{\pgfqpoint{0.929134in}{2.193898in}}%
\pgfpathcurveto{\pgfqpoint{0.923310in}{2.188074in}}{\pgfqpoint{0.920038in}{2.180174in}}{\pgfqpoint{0.920038in}{2.171938in}}%
\pgfpathcurveto{\pgfqpoint{0.920038in}{2.163701in}}{\pgfqpoint{0.923310in}{2.155801in}}{\pgfqpoint{0.929134in}{2.149977in}}%
\pgfpathcurveto{\pgfqpoint{0.934958in}{2.144153in}}{\pgfqpoint{0.942858in}{2.140881in}}{\pgfqpoint{0.951094in}{2.140881in}}%
\pgfpathclose%
\pgfusepath{stroke,fill}%
\end{pgfscope}%
\begin{pgfscope}%
\pgfpathrectangle{\pgfqpoint{0.100000in}{0.212622in}}{\pgfqpoint{3.696000in}{3.696000in}}%
\pgfusepath{clip}%
\pgfsetbuttcap%
\pgfsetroundjoin%
\definecolor{currentfill}{rgb}{0.121569,0.466667,0.705882}%
\pgfsetfillcolor{currentfill}%
\pgfsetfillopacity{0.872592}%
\pgfsetlinewidth{1.003750pt}%
\definecolor{currentstroke}{rgb}{0.121569,0.466667,0.705882}%
\pgfsetstrokecolor{currentstroke}%
\pgfsetstrokeopacity{0.872592}%
\pgfsetdash{}{0pt}%
\pgfpathmoveto{\pgfqpoint{2.252317in}{2.556391in}}%
\pgfpathcurveto{\pgfqpoint{2.260553in}{2.556391in}}{\pgfqpoint{2.268453in}{2.559664in}}{\pgfqpoint{2.274277in}{2.565487in}}%
\pgfpathcurveto{\pgfqpoint{2.280101in}{2.571311in}}{\pgfqpoint{2.283374in}{2.579211in}}{\pgfqpoint{2.283374in}{2.587448in}}%
\pgfpathcurveto{\pgfqpoint{2.283374in}{2.595684in}}{\pgfqpoint{2.280101in}{2.603584in}}{\pgfqpoint{2.274277in}{2.609408in}}%
\pgfpathcurveto{\pgfqpoint{2.268453in}{2.615232in}}{\pgfqpoint{2.260553in}{2.618504in}}{\pgfqpoint{2.252317in}{2.618504in}}%
\pgfpathcurveto{\pgfqpoint{2.244081in}{2.618504in}}{\pgfqpoint{2.236181in}{2.615232in}}{\pgfqpoint{2.230357in}{2.609408in}}%
\pgfpathcurveto{\pgfqpoint{2.224533in}{2.603584in}}{\pgfqpoint{2.221261in}{2.595684in}}{\pgfqpoint{2.221261in}{2.587448in}}%
\pgfpathcurveto{\pgfqpoint{2.221261in}{2.579211in}}{\pgfqpoint{2.224533in}{2.571311in}}{\pgfqpoint{2.230357in}{2.565487in}}%
\pgfpathcurveto{\pgfqpoint{2.236181in}{2.559664in}}{\pgfqpoint{2.244081in}{2.556391in}}{\pgfqpoint{2.252317in}{2.556391in}}%
\pgfpathclose%
\pgfusepath{stroke,fill}%
\end{pgfscope}%
\begin{pgfscope}%
\pgfpathrectangle{\pgfqpoint{0.100000in}{0.212622in}}{\pgfqpoint{3.696000in}{3.696000in}}%
\pgfusepath{clip}%
\pgfsetbuttcap%
\pgfsetroundjoin%
\definecolor{currentfill}{rgb}{0.121569,0.466667,0.705882}%
\pgfsetfillcolor{currentfill}%
\pgfsetfillopacity{0.872646}%
\pgfsetlinewidth{1.003750pt}%
\definecolor{currentstroke}{rgb}{0.121569,0.466667,0.705882}%
\pgfsetstrokecolor{currentstroke}%
\pgfsetstrokeopacity{0.872646}%
\pgfsetdash{}{0pt}%
\pgfpathmoveto{\pgfqpoint{0.948905in}{2.137925in}}%
\pgfpathcurveto{\pgfqpoint{0.957141in}{2.137925in}}{\pgfqpoint{0.965041in}{2.141197in}}{\pgfqpoint{0.970865in}{2.147021in}}%
\pgfpathcurveto{\pgfqpoint{0.976689in}{2.152845in}}{\pgfqpoint{0.979961in}{2.160745in}}{\pgfqpoint{0.979961in}{2.168981in}}%
\pgfpathcurveto{\pgfqpoint{0.979961in}{2.177217in}}{\pgfqpoint{0.976689in}{2.185117in}}{\pgfqpoint{0.970865in}{2.190941in}}%
\pgfpathcurveto{\pgfqpoint{0.965041in}{2.196765in}}{\pgfqpoint{0.957141in}{2.200038in}}{\pgfqpoint{0.948905in}{2.200038in}}%
\pgfpathcurveto{\pgfqpoint{0.940668in}{2.200038in}}{\pgfqpoint{0.932768in}{2.196765in}}{\pgfqpoint{0.926944in}{2.190941in}}%
\pgfpathcurveto{\pgfqpoint{0.921120in}{2.185117in}}{\pgfqpoint{0.917848in}{2.177217in}}{\pgfqpoint{0.917848in}{2.168981in}}%
\pgfpathcurveto{\pgfqpoint{0.917848in}{2.160745in}}{\pgfqpoint{0.921120in}{2.152845in}}{\pgfqpoint{0.926944in}{2.147021in}}%
\pgfpathcurveto{\pgfqpoint{0.932768in}{2.141197in}}{\pgfqpoint{0.940668in}{2.137925in}}{\pgfqpoint{0.948905in}{2.137925in}}%
\pgfpathclose%
\pgfusepath{stroke,fill}%
\end{pgfscope}%
\begin{pgfscope}%
\pgfpathrectangle{\pgfqpoint{0.100000in}{0.212622in}}{\pgfqpoint{3.696000in}{3.696000in}}%
\pgfusepath{clip}%
\pgfsetbuttcap%
\pgfsetroundjoin%
\definecolor{currentfill}{rgb}{0.121569,0.466667,0.705882}%
\pgfsetfillcolor{currentfill}%
\pgfsetfillopacity{0.873290}%
\pgfsetlinewidth{1.003750pt}%
\definecolor{currentstroke}{rgb}{0.121569,0.466667,0.705882}%
\pgfsetstrokecolor{currentstroke}%
\pgfsetstrokeopacity{0.873290}%
\pgfsetdash{}{0pt}%
\pgfpathmoveto{\pgfqpoint{2.250635in}{2.554571in}}%
\pgfpathcurveto{\pgfqpoint{2.258871in}{2.554571in}}{\pgfqpoint{2.266772in}{2.557843in}}{\pgfqpoint{2.272595in}{2.563667in}}%
\pgfpathcurveto{\pgfqpoint{2.278419in}{2.569491in}}{\pgfqpoint{2.281692in}{2.577391in}}{\pgfqpoint{2.281692in}{2.585627in}}%
\pgfpathcurveto{\pgfqpoint{2.281692in}{2.593863in}}{\pgfqpoint{2.278419in}{2.601763in}}{\pgfqpoint{2.272595in}{2.607587in}}%
\pgfpathcurveto{\pgfqpoint{2.266772in}{2.613411in}}{\pgfqpoint{2.258871in}{2.616684in}}{\pgfqpoint{2.250635in}{2.616684in}}%
\pgfpathcurveto{\pgfqpoint{2.242399in}{2.616684in}}{\pgfqpoint{2.234499in}{2.613411in}}{\pgfqpoint{2.228675in}{2.607587in}}%
\pgfpathcurveto{\pgfqpoint{2.222851in}{2.601763in}}{\pgfqpoint{2.219579in}{2.593863in}}{\pgfqpoint{2.219579in}{2.585627in}}%
\pgfpathcurveto{\pgfqpoint{2.219579in}{2.577391in}}{\pgfqpoint{2.222851in}{2.569491in}}{\pgfqpoint{2.228675in}{2.563667in}}%
\pgfpathcurveto{\pgfqpoint{2.234499in}{2.557843in}}{\pgfqpoint{2.242399in}{2.554571in}}{\pgfqpoint{2.250635in}{2.554571in}}%
\pgfpathclose%
\pgfusepath{stroke,fill}%
\end{pgfscope}%
\begin{pgfscope}%
\pgfpathrectangle{\pgfqpoint{0.100000in}{0.212622in}}{\pgfqpoint{3.696000in}{3.696000in}}%
\pgfusepath{clip}%
\pgfsetbuttcap%
\pgfsetroundjoin%
\definecolor{currentfill}{rgb}{0.121569,0.466667,0.705882}%
\pgfsetfillcolor{currentfill}%
\pgfsetfillopacity{0.873399}%
\pgfsetlinewidth{1.003750pt}%
\definecolor{currentstroke}{rgb}{0.121569,0.466667,0.705882}%
\pgfsetstrokecolor{currentstroke}%
\pgfsetstrokeopacity{0.873399}%
\pgfsetdash{}{0pt}%
\pgfpathmoveto{\pgfqpoint{1.810659in}{2.685099in}}%
\pgfpathcurveto{\pgfqpoint{1.818895in}{2.685099in}}{\pgfqpoint{1.826795in}{2.688371in}}{\pgfqpoint{1.832619in}{2.694195in}}%
\pgfpathcurveto{\pgfqpoint{1.838443in}{2.700019in}}{\pgfqpoint{1.841715in}{2.707919in}}{\pgfqpoint{1.841715in}{2.716156in}}%
\pgfpathcurveto{\pgfqpoint{1.841715in}{2.724392in}}{\pgfqpoint{1.838443in}{2.732292in}}{\pgfqpoint{1.832619in}{2.738116in}}%
\pgfpathcurveto{\pgfqpoint{1.826795in}{2.743940in}}{\pgfqpoint{1.818895in}{2.747212in}}{\pgfqpoint{1.810659in}{2.747212in}}%
\pgfpathcurveto{\pgfqpoint{1.802423in}{2.747212in}}{\pgfqpoint{1.794522in}{2.743940in}}{\pgfqpoint{1.788699in}{2.738116in}}%
\pgfpathcurveto{\pgfqpoint{1.782875in}{2.732292in}}{\pgfqpoint{1.779602in}{2.724392in}}{\pgfqpoint{1.779602in}{2.716156in}}%
\pgfpathcurveto{\pgfqpoint{1.779602in}{2.707919in}}{\pgfqpoint{1.782875in}{2.700019in}}{\pgfqpoint{1.788699in}{2.694195in}}%
\pgfpathcurveto{\pgfqpoint{1.794522in}{2.688371in}}{\pgfqpoint{1.802423in}{2.685099in}}{\pgfqpoint{1.810659in}{2.685099in}}%
\pgfpathclose%
\pgfusepath{stroke,fill}%
\end{pgfscope}%
\begin{pgfscope}%
\pgfpathrectangle{\pgfqpoint{0.100000in}{0.212622in}}{\pgfqpoint{3.696000in}{3.696000in}}%
\pgfusepath{clip}%
\pgfsetbuttcap%
\pgfsetroundjoin%
\definecolor{currentfill}{rgb}{0.121569,0.466667,0.705882}%
\pgfsetfillcolor{currentfill}%
\pgfsetfillopacity{0.873424}%
\pgfsetlinewidth{1.003750pt}%
\definecolor{currentstroke}{rgb}{0.121569,0.466667,0.705882}%
\pgfsetstrokecolor{currentstroke}%
\pgfsetstrokeopacity{0.873424}%
\pgfsetdash{}{0pt}%
\pgfpathmoveto{\pgfqpoint{0.946577in}{2.134509in}}%
\pgfpathcurveto{\pgfqpoint{0.954814in}{2.134509in}}{\pgfqpoint{0.962714in}{2.137781in}}{\pgfqpoint{0.968538in}{2.143605in}}%
\pgfpathcurveto{\pgfqpoint{0.974361in}{2.149429in}}{\pgfqpoint{0.977634in}{2.157329in}}{\pgfqpoint{0.977634in}{2.165565in}}%
\pgfpathcurveto{\pgfqpoint{0.977634in}{2.173801in}}{\pgfqpoint{0.974361in}{2.181701in}}{\pgfqpoint{0.968538in}{2.187525in}}%
\pgfpathcurveto{\pgfqpoint{0.962714in}{2.193349in}}{\pgfqpoint{0.954814in}{2.196622in}}{\pgfqpoint{0.946577in}{2.196622in}}%
\pgfpathcurveto{\pgfqpoint{0.938341in}{2.196622in}}{\pgfqpoint{0.930441in}{2.193349in}}{\pgfqpoint{0.924617in}{2.187525in}}%
\pgfpathcurveto{\pgfqpoint{0.918793in}{2.181701in}}{\pgfqpoint{0.915521in}{2.173801in}}{\pgfqpoint{0.915521in}{2.165565in}}%
\pgfpathcurveto{\pgfqpoint{0.915521in}{2.157329in}}{\pgfqpoint{0.918793in}{2.149429in}}{\pgfqpoint{0.924617in}{2.143605in}}%
\pgfpathcurveto{\pgfqpoint{0.930441in}{2.137781in}}{\pgfqpoint{0.938341in}{2.134509in}}{\pgfqpoint{0.946577in}{2.134509in}}%
\pgfpathclose%
\pgfusepath{stroke,fill}%
\end{pgfscope}%
\begin{pgfscope}%
\pgfpathrectangle{\pgfqpoint{0.100000in}{0.212622in}}{\pgfqpoint{3.696000in}{3.696000in}}%
\pgfusepath{clip}%
\pgfsetbuttcap%
\pgfsetroundjoin%
\definecolor{currentfill}{rgb}{0.121569,0.466667,0.705882}%
\pgfsetfillcolor{currentfill}%
\pgfsetfillopacity{0.873578}%
\pgfsetlinewidth{1.003750pt}%
\definecolor{currentstroke}{rgb}{0.121569,0.466667,0.705882}%
\pgfsetstrokecolor{currentstroke}%
\pgfsetstrokeopacity{0.873578}%
\pgfsetdash{}{0pt}%
\pgfpathmoveto{\pgfqpoint{2.249971in}{2.553920in}}%
\pgfpathcurveto{\pgfqpoint{2.258207in}{2.553920in}}{\pgfqpoint{2.266107in}{2.557193in}}{\pgfqpoint{2.271931in}{2.563016in}}%
\pgfpathcurveto{\pgfqpoint{2.277755in}{2.568840in}}{\pgfqpoint{2.281027in}{2.576740in}}{\pgfqpoint{2.281027in}{2.584977in}}%
\pgfpathcurveto{\pgfqpoint{2.281027in}{2.593213in}}{\pgfqpoint{2.277755in}{2.601113in}}{\pgfqpoint{2.271931in}{2.606937in}}%
\pgfpathcurveto{\pgfqpoint{2.266107in}{2.612761in}}{\pgfqpoint{2.258207in}{2.616033in}}{\pgfqpoint{2.249971in}{2.616033in}}%
\pgfpathcurveto{\pgfqpoint{2.241734in}{2.616033in}}{\pgfqpoint{2.233834in}{2.612761in}}{\pgfqpoint{2.228010in}{2.606937in}}%
\pgfpathcurveto{\pgfqpoint{2.222186in}{2.601113in}}{\pgfqpoint{2.218914in}{2.593213in}}{\pgfqpoint{2.218914in}{2.584977in}}%
\pgfpathcurveto{\pgfqpoint{2.218914in}{2.576740in}}{\pgfqpoint{2.222186in}{2.568840in}}{\pgfqpoint{2.228010in}{2.563016in}}%
\pgfpathcurveto{\pgfqpoint{2.233834in}{2.557193in}}{\pgfqpoint{2.241734in}{2.553920in}}{\pgfqpoint{2.249971in}{2.553920in}}%
\pgfpathclose%
\pgfusepath{stroke,fill}%
\end{pgfscope}%
\begin{pgfscope}%
\pgfpathrectangle{\pgfqpoint{0.100000in}{0.212622in}}{\pgfqpoint{3.696000in}{3.696000in}}%
\pgfusepath{clip}%
\pgfsetbuttcap%
\pgfsetroundjoin%
\definecolor{currentfill}{rgb}{0.121569,0.466667,0.705882}%
\pgfsetfillcolor{currentfill}%
\pgfsetfillopacity{0.873875}%
\pgfsetlinewidth{1.003750pt}%
\definecolor{currentstroke}{rgb}{0.121569,0.466667,0.705882}%
\pgfsetstrokecolor{currentstroke}%
\pgfsetstrokeopacity{0.873875}%
\pgfsetdash{}{0pt}%
\pgfpathmoveto{\pgfqpoint{0.945309in}{2.132746in}}%
\pgfpathcurveto{\pgfqpoint{0.953545in}{2.132746in}}{\pgfqpoint{0.961445in}{2.136019in}}{\pgfqpoint{0.967269in}{2.141842in}}%
\pgfpathcurveto{\pgfqpoint{0.973093in}{2.147666in}}{\pgfqpoint{0.976366in}{2.155566in}}{\pgfqpoint{0.976366in}{2.163803in}}%
\pgfpathcurveto{\pgfqpoint{0.976366in}{2.172039in}}{\pgfqpoint{0.973093in}{2.179939in}}{\pgfqpoint{0.967269in}{2.185763in}}%
\pgfpathcurveto{\pgfqpoint{0.961445in}{2.191587in}}{\pgfqpoint{0.953545in}{2.194859in}}{\pgfqpoint{0.945309in}{2.194859in}}%
\pgfpathcurveto{\pgfqpoint{0.937073in}{2.194859in}}{\pgfqpoint{0.929173in}{2.191587in}}{\pgfqpoint{0.923349in}{2.185763in}}%
\pgfpathcurveto{\pgfqpoint{0.917525in}{2.179939in}}{\pgfqpoint{0.914253in}{2.172039in}}{\pgfqpoint{0.914253in}{2.163803in}}%
\pgfpathcurveto{\pgfqpoint{0.914253in}{2.155566in}}{\pgfqpoint{0.917525in}{2.147666in}}{\pgfqpoint{0.923349in}{2.141842in}}%
\pgfpathcurveto{\pgfqpoint{0.929173in}{2.136019in}}{\pgfqpoint{0.937073in}{2.132746in}}{\pgfqpoint{0.945309in}{2.132746in}}%
\pgfpathclose%
\pgfusepath{stroke,fill}%
\end{pgfscope}%
\begin{pgfscope}%
\pgfpathrectangle{\pgfqpoint{0.100000in}{0.212622in}}{\pgfqpoint{3.696000in}{3.696000in}}%
\pgfusepath{clip}%
\pgfsetbuttcap%
\pgfsetroundjoin%
\definecolor{currentfill}{rgb}{0.121569,0.466667,0.705882}%
\pgfsetfillcolor{currentfill}%
\pgfsetfillopacity{0.874084}%
\pgfsetlinewidth{1.003750pt}%
\definecolor{currentstroke}{rgb}{0.121569,0.466667,0.705882}%
\pgfsetstrokecolor{currentstroke}%
\pgfsetstrokeopacity{0.874084}%
\pgfsetdash{}{0pt}%
\pgfpathmoveto{\pgfqpoint{2.248804in}{2.552576in}}%
\pgfpathcurveto{\pgfqpoint{2.257041in}{2.552576in}}{\pgfqpoint{2.264941in}{2.555848in}}{\pgfqpoint{2.270765in}{2.561672in}}%
\pgfpathcurveto{\pgfqpoint{2.276589in}{2.567496in}}{\pgfqpoint{2.279861in}{2.575396in}}{\pgfqpoint{2.279861in}{2.583632in}}%
\pgfpathcurveto{\pgfqpoint{2.279861in}{2.591869in}}{\pgfqpoint{2.276589in}{2.599769in}}{\pgfqpoint{2.270765in}{2.605592in}}%
\pgfpathcurveto{\pgfqpoint{2.264941in}{2.611416in}}{\pgfqpoint{2.257041in}{2.614689in}}{\pgfqpoint{2.248804in}{2.614689in}}%
\pgfpathcurveto{\pgfqpoint{2.240568in}{2.614689in}}{\pgfqpoint{2.232668in}{2.611416in}}{\pgfqpoint{2.226844in}{2.605592in}}%
\pgfpathcurveto{\pgfqpoint{2.221020in}{2.599769in}}{\pgfqpoint{2.217748in}{2.591869in}}{\pgfqpoint{2.217748in}{2.583632in}}%
\pgfpathcurveto{\pgfqpoint{2.217748in}{2.575396in}}{\pgfqpoint{2.221020in}{2.567496in}}{\pgfqpoint{2.226844in}{2.561672in}}%
\pgfpathcurveto{\pgfqpoint{2.232668in}{2.555848in}}{\pgfqpoint{2.240568in}{2.552576in}}{\pgfqpoint{2.248804in}{2.552576in}}%
\pgfpathclose%
\pgfusepath{stroke,fill}%
\end{pgfscope}%
\begin{pgfscope}%
\pgfpathrectangle{\pgfqpoint{0.100000in}{0.212622in}}{\pgfqpoint{3.696000in}{3.696000in}}%
\pgfusepath{clip}%
\pgfsetbuttcap%
\pgfsetroundjoin%
\definecolor{currentfill}{rgb}{0.121569,0.466667,0.705882}%
\pgfsetfillcolor{currentfill}%
\pgfsetfillopacity{0.874316}%
\pgfsetlinewidth{1.003750pt}%
\definecolor{currentstroke}{rgb}{0.121569,0.466667,0.705882}%
\pgfsetstrokecolor{currentstroke}%
\pgfsetstrokeopacity{0.874316}%
\pgfsetdash{}{0pt}%
\pgfpathmoveto{\pgfqpoint{2.248250in}{2.551896in}}%
\pgfpathcurveto{\pgfqpoint{2.256487in}{2.551896in}}{\pgfqpoint{2.264387in}{2.555168in}}{\pgfqpoint{2.270211in}{2.560992in}}%
\pgfpathcurveto{\pgfqpoint{2.276034in}{2.566816in}}{\pgfqpoint{2.279307in}{2.574716in}}{\pgfqpoint{2.279307in}{2.582953in}}%
\pgfpathcurveto{\pgfqpoint{2.279307in}{2.591189in}}{\pgfqpoint{2.276034in}{2.599089in}}{\pgfqpoint{2.270211in}{2.604913in}}%
\pgfpathcurveto{\pgfqpoint{2.264387in}{2.610737in}}{\pgfqpoint{2.256487in}{2.614009in}}{\pgfqpoint{2.248250in}{2.614009in}}%
\pgfpathcurveto{\pgfqpoint{2.240014in}{2.614009in}}{\pgfqpoint{2.232114in}{2.610737in}}{\pgfqpoint{2.226290in}{2.604913in}}%
\pgfpathcurveto{\pgfqpoint{2.220466in}{2.599089in}}{\pgfqpoint{2.217194in}{2.591189in}}{\pgfqpoint{2.217194in}{2.582953in}}%
\pgfpathcurveto{\pgfqpoint{2.217194in}{2.574716in}}{\pgfqpoint{2.220466in}{2.566816in}}{\pgfqpoint{2.226290in}{2.560992in}}%
\pgfpathcurveto{\pgfqpoint{2.232114in}{2.555168in}}{\pgfqpoint{2.240014in}{2.551896in}}{\pgfqpoint{2.248250in}{2.551896in}}%
\pgfpathclose%
\pgfusepath{stroke,fill}%
\end{pgfscope}%
\begin{pgfscope}%
\pgfpathrectangle{\pgfqpoint{0.100000in}{0.212622in}}{\pgfqpoint{3.696000in}{3.696000in}}%
\pgfusepath{clip}%
\pgfsetbuttcap%
\pgfsetroundjoin%
\definecolor{currentfill}{rgb}{0.121569,0.466667,0.705882}%
\pgfsetfillcolor{currentfill}%
\pgfsetfillopacity{0.874326}%
\pgfsetlinewidth{1.003750pt}%
\definecolor{currentstroke}{rgb}{0.121569,0.466667,0.705882}%
\pgfsetstrokecolor{currentstroke}%
\pgfsetstrokeopacity{0.874326}%
\pgfsetdash{}{0pt}%
\pgfpathmoveto{\pgfqpoint{1.808894in}{2.682656in}}%
\pgfpathcurveto{\pgfqpoint{1.817130in}{2.682656in}}{\pgfqpoint{1.825030in}{2.685928in}}{\pgfqpoint{1.830854in}{2.691752in}}%
\pgfpathcurveto{\pgfqpoint{1.836678in}{2.697576in}}{\pgfqpoint{1.839950in}{2.705476in}}{\pgfqpoint{1.839950in}{2.713712in}}%
\pgfpathcurveto{\pgfqpoint{1.839950in}{2.721948in}}{\pgfqpoint{1.836678in}{2.729848in}}{\pgfqpoint{1.830854in}{2.735672in}}%
\pgfpathcurveto{\pgfqpoint{1.825030in}{2.741496in}}{\pgfqpoint{1.817130in}{2.744769in}}{\pgfqpoint{1.808894in}{2.744769in}}%
\pgfpathcurveto{\pgfqpoint{1.800657in}{2.744769in}}{\pgfqpoint{1.792757in}{2.741496in}}{\pgfqpoint{1.786933in}{2.735672in}}%
\pgfpathcurveto{\pgfqpoint{1.781109in}{2.729848in}}{\pgfqpoint{1.777837in}{2.721948in}}{\pgfqpoint{1.777837in}{2.713712in}}%
\pgfpathcurveto{\pgfqpoint{1.777837in}{2.705476in}}{\pgfqpoint{1.781109in}{2.697576in}}{\pgfqpoint{1.786933in}{2.691752in}}%
\pgfpathcurveto{\pgfqpoint{1.792757in}{2.685928in}}{\pgfqpoint{1.800657in}{2.682656in}}{\pgfqpoint{1.808894in}{2.682656in}}%
\pgfpathclose%
\pgfusepath{stroke,fill}%
\end{pgfscope}%
\begin{pgfscope}%
\pgfpathrectangle{\pgfqpoint{0.100000in}{0.212622in}}{\pgfqpoint{3.696000in}{3.696000in}}%
\pgfusepath{clip}%
\pgfsetbuttcap%
\pgfsetroundjoin%
\definecolor{currentfill}{rgb}{0.121569,0.466667,0.705882}%
\pgfsetfillcolor{currentfill}%
\pgfsetfillopacity{0.874440}%
\pgfsetlinewidth{1.003750pt}%
\definecolor{currentstroke}{rgb}{0.121569,0.466667,0.705882}%
\pgfsetstrokecolor{currentstroke}%
\pgfsetstrokeopacity{0.874440}%
\pgfsetdash{}{0pt}%
\pgfpathmoveto{\pgfqpoint{0.943845in}{2.130839in}}%
\pgfpathcurveto{\pgfqpoint{0.952081in}{2.130839in}}{\pgfqpoint{0.959982in}{2.134111in}}{\pgfqpoint{0.965805in}{2.139935in}}%
\pgfpathcurveto{\pgfqpoint{0.971629in}{2.145759in}}{\pgfqpoint{0.974902in}{2.153659in}}{\pgfqpoint{0.974902in}{2.161895in}}%
\pgfpathcurveto{\pgfqpoint{0.974902in}{2.170131in}}{\pgfqpoint{0.971629in}{2.178032in}}{\pgfqpoint{0.965805in}{2.183855in}}%
\pgfpathcurveto{\pgfqpoint{0.959982in}{2.189679in}}{\pgfqpoint{0.952081in}{2.192952in}}{\pgfqpoint{0.943845in}{2.192952in}}%
\pgfpathcurveto{\pgfqpoint{0.935609in}{2.192952in}}{\pgfqpoint{0.927709in}{2.189679in}}{\pgfqpoint{0.921885in}{2.183855in}}%
\pgfpathcurveto{\pgfqpoint{0.916061in}{2.178032in}}{\pgfqpoint{0.912789in}{2.170131in}}{\pgfqpoint{0.912789in}{2.161895in}}%
\pgfpathcurveto{\pgfqpoint{0.912789in}{2.153659in}}{\pgfqpoint{0.916061in}{2.145759in}}{\pgfqpoint{0.921885in}{2.139935in}}%
\pgfpathcurveto{\pgfqpoint{0.927709in}{2.134111in}}{\pgfqpoint{0.935609in}{2.130839in}}{\pgfqpoint{0.943845in}{2.130839in}}%
\pgfpathclose%
\pgfusepath{stroke,fill}%
\end{pgfscope}%
\begin{pgfscope}%
\pgfpathrectangle{\pgfqpoint{0.100000in}{0.212622in}}{\pgfqpoint{3.696000in}{3.696000in}}%
\pgfusepath{clip}%
\pgfsetbuttcap%
\pgfsetroundjoin%
\definecolor{currentfill}{rgb}{0.121569,0.466667,0.705882}%
\pgfsetfillcolor{currentfill}%
\pgfsetfillopacity{0.874744}%
\pgfsetlinewidth{1.003750pt}%
\definecolor{currentstroke}{rgb}{0.121569,0.466667,0.705882}%
\pgfsetstrokecolor{currentstroke}%
\pgfsetstrokeopacity{0.874744}%
\pgfsetdash{}{0pt}%
\pgfpathmoveto{\pgfqpoint{2.247215in}{2.550724in}}%
\pgfpathcurveto{\pgfqpoint{2.255451in}{2.550724in}}{\pgfqpoint{2.263351in}{2.553996in}}{\pgfqpoint{2.269175in}{2.559820in}}%
\pgfpathcurveto{\pgfqpoint{2.274999in}{2.565644in}}{\pgfqpoint{2.278272in}{2.573544in}}{\pgfqpoint{2.278272in}{2.581780in}}%
\pgfpathcurveto{\pgfqpoint{2.278272in}{2.590017in}}{\pgfqpoint{2.274999in}{2.597917in}}{\pgfqpoint{2.269175in}{2.603741in}}%
\pgfpathcurveto{\pgfqpoint{2.263351in}{2.609565in}}{\pgfqpoint{2.255451in}{2.612837in}}{\pgfqpoint{2.247215in}{2.612837in}}%
\pgfpathcurveto{\pgfqpoint{2.238979in}{2.612837in}}{\pgfqpoint{2.231079in}{2.609565in}}{\pgfqpoint{2.225255in}{2.603741in}}%
\pgfpathcurveto{\pgfqpoint{2.219431in}{2.597917in}}{\pgfqpoint{2.216159in}{2.590017in}}{\pgfqpoint{2.216159in}{2.581780in}}%
\pgfpathcurveto{\pgfqpoint{2.216159in}{2.573544in}}{\pgfqpoint{2.219431in}{2.565644in}}{\pgfqpoint{2.225255in}{2.559820in}}%
\pgfpathcurveto{\pgfqpoint{2.231079in}{2.553996in}}{\pgfqpoint{2.238979in}{2.550724in}}{\pgfqpoint{2.247215in}{2.550724in}}%
\pgfpathclose%
\pgfusepath{stroke,fill}%
\end{pgfscope}%
\begin{pgfscope}%
\pgfpathrectangle{\pgfqpoint{0.100000in}{0.212622in}}{\pgfqpoint{3.696000in}{3.696000in}}%
\pgfusepath{clip}%
\pgfsetbuttcap%
\pgfsetroundjoin%
\definecolor{currentfill}{rgb}{0.121569,0.466667,0.705882}%
\pgfsetfillcolor{currentfill}%
\pgfsetfillopacity{0.874826}%
\pgfsetlinewidth{1.003750pt}%
\definecolor{currentstroke}{rgb}{0.121569,0.466667,0.705882}%
\pgfsetstrokecolor{currentstroke}%
\pgfsetstrokeopacity{0.874826}%
\pgfsetdash{}{0pt}%
\pgfpathmoveto{\pgfqpoint{2.247024in}{2.550531in}}%
\pgfpathcurveto{\pgfqpoint{2.255260in}{2.550531in}}{\pgfqpoint{2.263160in}{2.553803in}}{\pgfqpoint{2.268984in}{2.559627in}}%
\pgfpathcurveto{\pgfqpoint{2.274808in}{2.565451in}}{\pgfqpoint{2.278080in}{2.573351in}}{\pgfqpoint{2.278080in}{2.581588in}}%
\pgfpathcurveto{\pgfqpoint{2.278080in}{2.589824in}}{\pgfqpoint{2.274808in}{2.597724in}}{\pgfqpoint{2.268984in}{2.603548in}}%
\pgfpathcurveto{\pgfqpoint{2.263160in}{2.609372in}}{\pgfqpoint{2.255260in}{2.612644in}}{\pgfqpoint{2.247024in}{2.612644in}}%
\pgfpathcurveto{\pgfqpoint{2.238787in}{2.612644in}}{\pgfqpoint{2.230887in}{2.609372in}}{\pgfqpoint{2.225064in}{2.603548in}}%
\pgfpathcurveto{\pgfqpoint{2.219240in}{2.597724in}}{\pgfqpoint{2.215967in}{2.589824in}}{\pgfqpoint{2.215967in}{2.581588in}}%
\pgfpathcurveto{\pgfqpoint{2.215967in}{2.573351in}}{\pgfqpoint{2.219240in}{2.565451in}}{\pgfqpoint{2.225064in}{2.559627in}}%
\pgfpathcurveto{\pgfqpoint{2.230887in}{2.553803in}}{\pgfqpoint{2.238787in}{2.550531in}}{\pgfqpoint{2.247024in}{2.550531in}}%
\pgfpathclose%
\pgfusepath{stroke,fill}%
\end{pgfscope}%
\begin{pgfscope}%
\pgfpathrectangle{\pgfqpoint{0.100000in}{0.212622in}}{\pgfqpoint{3.696000in}{3.696000in}}%
\pgfusepath{clip}%
\pgfsetbuttcap%
\pgfsetroundjoin%
\definecolor{currentfill}{rgb}{0.121569,0.466667,0.705882}%
\pgfsetfillcolor{currentfill}%
\pgfsetfillopacity{0.874977}%
\pgfsetlinewidth{1.003750pt}%
\definecolor{currentstroke}{rgb}{0.121569,0.466667,0.705882}%
\pgfsetstrokecolor{currentstroke}%
\pgfsetstrokeopacity{0.874977}%
\pgfsetdash{}{0pt}%
\pgfpathmoveto{\pgfqpoint{2.246682in}{2.550175in}}%
\pgfpathcurveto{\pgfqpoint{2.254918in}{2.550175in}}{\pgfqpoint{2.262818in}{2.553448in}}{\pgfqpoint{2.268642in}{2.559272in}}%
\pgfpathcurveto{\pgfqpoint{2.274466in}{2.565096in}}{\pgfqpoint{2.277738in}{2.572996in}}{\pgfqpoint{2.277738in}{2.581232in}}%
\pgfpathcurveto{\pgfqpoint{2.277738in}{2.589468in}}{\pgfqpoint{2.274466in}{2.597368in}}{\pgfqpoint{2.268642in}{2.603192in}}%
\pgfpathcurveto{\pgfqpoint{2.262818in}{2.609016in}}{\pgfqpoint{2.254918in}{2.612288in}}{\pgfqpoint{2.246682in}{2.612288in}}%
\pgfpathcurveto{\pgfqpoint{2.238445in}{2.612288in}}{\pgfqpoint{2.230545in}{2.609016in}}{\pgfqpoint{2.224721in}{2.603192in}}%
\pgfpathcurveto{\pgfqpoint{2.218898in}{2.597368in}}{\pgfqpoint{2.215625in}{2.589468in}}{\pgfqpoint{2.215625in}{2.581232in}}%
\pgfpathcurveto{\pgfqpoint{2.215625in}{2.572996in}}{\pgfqpoint{2.218898in}{2.565096in}}{\pgfqpoint{2.224721in}{2.559272in}}%
\pgfpathcurveto{\pgfqpoint{2.230545in}{2.553448in}}{\pgfqpoint{2.238445in}{2.550175in}}{\pgfqpoint{2.246682in}{2.550175in}}%
\pgfpathclose%
\pgfusepath{stroke,fill}%
\end{pgfscope}%
\begin{pgfscope}%
\pgfpathrectangle{\pgfqpoint{0.100000in}{0.212622in}}{\pgfqpoint{3.696000in}{3.696000in}}%
\pgfusepath{clip}%
\pgfsetbuttcap%
\pgfsetroundjoin%
\definecolor{currentfill}{rgb}{0.121569,0.466667,0.705882}%
\pgfsetfillcolor{currentfill}%
\pgfsetfillopacity{0.874991}%
\pgfsetlinewidth{1.003750pt}%
\definecolor{currentstroke}{rgb}{0.121569,0.466667,0.705882}%
\pgfsetstrokecolor{currentstroke}%
\pgfsetstrokeopacity{0.874991}%
\pgfsetdash{}{0pt}%
\pgfpathmoveto{\pgfqpoint{2.728260in}{1.364846in}}%
\pgfpathcurveto{\pgfqpoint{2.736496in}{1.364846in}}{\pgfqpoint{2.744396in}{1.368118in}}{\pgfqpoint{2.750220in}{1.373942in}}%
\pgfpathcurveto{\pgfqpoint{2.756044in}{1.379766in}}{\pgfqpoint{2.759317in}{1.387666in}}{\pgfqpoint{2.759317in}{1.395903in}}%
\pgfpathcurveto{\pgfqpoint{2.759317in}{1.404139in}}{\pgfqpoint{2.756044in}{1.412039in}}{\pgfqpoint{2.750220in}{1.417863in}}%
\pgfpathcurveto{\pgfqpoint{2.744396in}{1.423687in}}{\pgfqpoint{2.736496in}{1.426959in}}{\pgfqpoint{2.728260in}{1.426959in}}%
\pgfpathcurveto{\pgfqpoint{2.720024in}{1.426959in}}{\pgfqpoint{2.712124in}{1.423687in}}{\pgfqpoint{2.706300in}{1.417863in}}%
\pgfpathcurveto{\pgfqpoint{2.700476in}{1.412039in}}{\pgfqpoint{2.697204in}{1.404139in}}{\pgfqpoint{2.697204in}{1.395903in}}%
\pgfpathcurveto{\pgfqpoint{2.697204in}{1.387666in}}{\pgfqpoint{2.700476in}{1.379766in}}{\pgfqpoint{2.706300in}{1.373942in}}%
\pgfpathcurveto{\pgfqpoint{2.712124in}{1.368118in}}{\pgfqpoint{2.720024in}{1.364846in}}{\pgfqpoint{2.728260in}{1.364846in}}%
\pgfpathclose%
\pgfusepath{stroke,fill}%
\end{pgfscope}%
\begin{pgfscope}%
\pgfpathrectangle{\pgfqpoint{0.100000in}{0.212622in}}{\pgfqpoint{3.696000in}{3.696000in}}%
\pgfusepath{clip}%
\pgfsetbuttcap%
\pgfsetroundjoin%
\definecolor{currentfill}{rgb}{0.121569,0.466667,0.705882}%
\pgfsetfillcolor{currentfill}%
\pgfsetfillopacity{0.875118}%
\pgfsetlinewidth{1.003750pt}%
\definecolor{currentstroke}{rgb}{0.121569,0.466667,0.705882}%
\pgfsetstrokecolor{currentstroke}%
\pgfsetstrokeopacity{0.875118}%
\pgfsetdash{}{0pt}%
\pgfpathmoveto{\pgfqpoint{0.941914in}{2.128386in}}%
\pgfpathcurveto{\pgfqpoint{0.950150in}{2.128386in}}{\pgfqpoint{0.958050in}{2.131658in}}{\pgfqpoint{0.963874in}{2.137482in}}%
\pgfpathcurveto{\pgfqpoint{0.969698in}{2.143306in}}{\pgfqpoint{0.972970in}{2.151206in}}{\pgfqpoint{0.972970in}{2.159442in}}%
\pgfpathcurveto{\pgfqpoint{0.972970in}{2.167679in}}{\pgfqpoint{0.969698in}{2.175579in}}{\pgfqpoint{0.963874in}{2.181403in}}%
\pgfpathcurveto{\pgfqpoint{0.958050in}{2.187227in}}{\pgfqpoint{0.950150in}{2.190499in}}{\pgfqpoint{0.941914in}{2.190499in}}%
\pgfpathcurveto{\pgfqpoint{0.933677in}{2.190499in}}{\pgfqpoint{0.925777in}{2.187227in}}{\pgfqpoint{0.919953in}{2.181403in}}%
\pgfpathcurveto{\pgfqpoint{0.914130in}{2.175579in}}{\pgfqpoint{0.910857in}{2.167679in}}{\pgfqpoint{0.910857in}{2.159442in}}%
\pgfpathcurveto{\pgfqpoint{0.910857in}{2.151206in}}{\pgfqpoint{0.914130in}{2.143306in}}{\pgfqpoint{0.919953in}{2.137482in}}%
\pgfpathcurveto{\pgfqpoint{0.925777in}{2.131658in}}{\pgfqpoint{0.933677in}{2.128386in}}{\pgfqpoint{0.941914in}{2.128386in}}%
\pgfpathclose%
\pgfusepath{stroke,fill}%
\end{pgfscope}%
\begin{pgfscope}%
\pgfpathrectangle{\pgfqpoint{0.100000in}{0.212622in}}{\pgfqpoint{3.696000in}{3.696000in}}%
\pgfusepath{clip}%
\pgfsetbuttcap%
\pgfsetroundjoin%
\definecolor{currentfill}{rgb}{0.121569,0.466667,0.705882}%
\pgfsetfillcolor{currentfill}%
\pgfsetfillopacity{0.875239}%
\pgfsetlinewidth{1.003750pt}%
\definecolor{currentstroke}{rgb}{0.121569,0.466667,0.705882}%
\pgfsetstrokecolor{currentstroke}%
\pgfsetstrokeopacity{0.875239}%
\pgfsetdash{}{0pt}%
\pgfpathmoveto{\pgfqpoint{2.246061in}{2.549462in}}%
\pgfpathcurveto{\pgfqpoint{2.254297in}{2.549462in}}{\pgfqpoint{2.262197in}{2.552734in}}{\pgfqpoint{2.268021in}{2.558558in}}%
\pgfpathcurveto{\pgfqpoint{2.273845in}{2.564382in}}{\pgfqpoint{2.277118in}{2.572282in}}{\pgfqpoint{2.277118in}{2.580518in}}%
\pgfpathcurveto{\pgfqpoint{2.277118in}{2.588754in}}{\pgfqpoint{2.273845in}{2.596655in}}{\pgfqpoint{2.268021in}{2.602478in}}%
\pgfpathcurveto{\pgfqpoint{2.262197in}{2.608302in}}{\pgfqpoint{2.254297in}{2.611575in}}{\pgfqpoint{2.246061in}{2.611575in}}%
\pgfpathcurveto{\pgfqpoint{2.237825in}{2.611575in}}{\pgfqpoint{2.229925in}{2.608302in}}{\pgfqpoint{2.224101in}{2.602478in}}%
\pgfpathcurveto{\pgfqpoint{2.218277in}{2.596655in}}{\pgfqpoint{2.215005in}{2.588754in}}{\pgfqpoint{2.215005in}{2.580518in}}%
\pgfpathcurveto{\pgfqpoint{2.215005in}{2.572282in}}{\pgfqpoint{2.218277in}{2.564382in}}{\pgfqpoint{2.224101in}{2.558558in}}%
\pgfpathcurveto{\pgfqpoint{2.229925in}{2.552734in}}{\pgfqpoint{2.237825in}{2.549462in}}{\pgfqpoint{2.246061in}{2.549462in}}%
\pgfpathclose%
\pgfusepath{stroke,fill}%
\end{pgfscope}%
\begin{pgfscope}%
\pgfpathrectangle{\pgfqpoint{0.100000in}{0.212622in}}{\pgfqpoint{3.696000in}{3.696000in}}%
\pgfusepath{clip}%
\pgfsetbuttcap%
\pgfsetroundjoin%
\definecolor{currentfill}{rgb}{0.121569,0.466667,0.705882}%
\pgfsetfillcolor{currentfill}%
\pgfsetfillopacity{0.875484}%
\pgfsetlinewidth{1.003750pt}%
\definecolor{currentstroke}{rgb}{0.121569,0.466667,0.705882}%
\pgfsetstrokecolor{currentstroke}%
\pgfsetstrokeopacity{0.875484}%
\pgfsetdash{}{0pt}%
\pgfpathmoveto{\pgfqpoint{0.940882in}{2.126968in}}%
\pgfpathcurveto{\pgfqpoint{0.949118in}{2.126968in}}{\pgfqpoint{0.957018in}{2.130240in}}{\pgfqpoint{0.962842in}{2.136064in}}%
\pgfpathcurveto{\pgfqpoint{0.968666in}{2.141888in}}{\pgfqpoint{0.971938in}{2.149788in}}{\pgfqpoint{0.971938in}{2.158024in}}%
\pgfpathcurveto{\pgfqpoint{0.971938in}{2.166260in}}{\pgfqpoint{0.968666in}{2.174160in}}{\pgfqpoint{0.962842in}{2.179984in}}%
\pgfpathcurveto{\pgfqpoint{0.957018in}{2.185808in}}{\pgfqpoint{0.949118in}{2.189081in}}{\pgfqpoint{0.940882in}{2.189081in}}%
\pgfpathcurveto{\pgfqpoint{0.932645in}{2.189081in}}{\pgfqpoint{0.924745in}{2.185808in}}{\pgfqpoint{0.918921in}{2.179984in}}%
\pgfpathcurveto{\pgfqpoint{0.913098in}{2.174160in}}{\pgfqpoint{0.909825in}{2.166260in}}{\pgfqpoint{0.909825in}{2.158024in}}%
\pgfpathcurveto{\pgfqpoint{0.909825in}{2.149788in}}{\pgfqpoint{0.913098in}{2.141888in}}{\pgfqpoint{0.918921in}{2.136064in}}%
\pgfpathcurveto{\pgfqpoint{0.924745in}{2.130240in}}{\pgfqpoint{0.932645in}{2.126968in}}{\pgfqpoint{0.940882in}{2.126968in}}%
\pgfpathclose%
\pgfusepath{stroke,fill}%
\end{pgfscope}%
\begin{pgfscope}%
\pgfpathrectangle{\pgfqpoint{0.100000in}{0.212622in}}{\pgfqpoint{3.696000in}{3.696000in}}%
\pgfusepath{clip}%
\pgfsetbuttcap%
\pgfsetroundjoin%
\definecolor{currentfill}{rgb}{0.121569,0.466667,0.705882}%
\pgfsetfillcolor{currentfill}%
\pgfsetfillopacity{0.875512}%
\pgfsetlinewidth{1.003750pt}%
\definecolor{currentstroke}{rgb}{0.121569,0.466667,0.705882}%
\pgfsetstrokecolor{currentstroke}%
\pgfsetstrokeopacity{0.875512}%
\pgfsetdash{}{0pt}%
\pgfpathmoveto{\pgfqpoint{1.806566in}{2.679098in}}%
\pgfpathcurveto{\pgfqpoint{1.814802in}{2.679098in}}{\pgfqpoint{1.822702in}{2.682370in}}{\pgfqpoint{1.828526in}{2.688194in}}%
\pgfpathcurveto{\pgfqpoint{1.834350in}{2.694018in}}{\pgfqpoint{1.837622in}{2.701918in}}{\pgfqpoint{1.837622in}{2.710154in}}%
\pgfpathcurveto{\pgfqpoint{1.837622in}{2.718390in}}{\pgfqpoint{1.834350in}{2.726290in}}{\pgfqpoint{1.828526in}{2.732114in}}%
\pgfpathcurveto{\pgfqpoint{1.822702in}{2.737938in}}{\pgfqpoint{1.814802in}{2.741211in}}{\pgfqpoint{1.806566in}{2.741211in}}%
\pgfpathcurveto{\pgfqpoint{1.798330in}{2.741211in}}{\pgfqpoint{1.790430in}{2.737938in}}{\pgfqpoint{1.784606in}{2.732114in}}%
\pgfpathcurveto{\pgfqpoint{1.778782in}{2.726290in}}{\pgfqpoint{1.775509in}{2.718390in}}{\pgfqpoint{1.775509in}{2.710154in}}%
\pgfpathcurveto{\pgfqpoint{1.775509in}{2.701918in}}{\pgfqpoint{1.778782in}{2.694018in}}{\pgfqpoint{1.784606in}{2.688194in}}%
\pgfpathcurveto{\pgfqpoint{1.790430in}{2.682370in}}{\pgfqpoint{1.798330in}{2.679098in}}{\pgfqpoint{1.806566in}{2.679098in}}%
\pgfpathclose%
\pgfusepath{stroke,fill}%
\end{pgfscope}%
\begin{pgfscope}%
\pgfpathrectangle{\pgfqpoint{0.100000in}{0.212622in}}{\pgfqpoint{3.696000in}{3.696000in}}%
\pgfusepath{clip}%
\pgfsetbuttcap%
\pgfsetroundjoin%
\definecolor{currentfill}{rgb}{0.121569,0.466667,0.705882}%
\pgfsetfillcolor{currentfill}%
\pgfsetfillopacity{0.875698}%
\pgfsetlinewidth{1.003750pt}%
\definecolor{currentstroke}{rgb}{0.121569,0.466667,0.705882}%
\pgfsetstrokecolor{currentstroke}%
\pgfsetstrokeopacity{0.875698}%
\pgfsetdash{}{0pt}%
\pgfpathmoveto{\pgfqpoint{0.940324in}{2.126250in}}%
\pgfpathcurveto{\pgfqpoint{0.948560in}{2.126250in}}{\pgfqpoint{0.956460in}{2.129522in}}{\pgfqpoint{0.962284in}{2.135346in}}%
\pgfpathcurveto{\pgfqpoint{0.968108in}{2.141170in}}{\pgfqpoint{0.971380in}{2.149070in}}{\pgfqpoint{0.971380in}{2.157306in}}%
\pgfpathcurveto{\pgfqpoint{0.971380in}{2.165542in}}{\pgfqpoint{0.968108in}{2.173442in}}{\pgfqpoint{0.962284in}{2.179266in}}%
\pgfpathcurveto{\pgfqpoint{0.956460in}{2.185090in}}{\pgfqpoint{0.948560in}{2.188363in}}{\pgfqpoint{0.940324in}{2.188363in}}%
\pgfpathcurveto{\pgfqpoint{0.932088in}{2.188363in}}{\pgfqpoint{0.924187in}{2.185090in}}{\pgfqpoint{0.918364in}{2.179266in}}%
\pgfpathcurveto{\pgfqpoint{0.912540in}{2.173442in}}{\pgfqpoint{0.909267in}{2.165542in}}{\pgfqpoint{0.909267in}{2.157306in}}%
\pgfpathcurveto{\pgfqpoint{0.909267in}{2.149070in}}{\pgfqpoint{0.912540in}{2.141170in}}{\pgfqpoint{0.918364in}{2.135346in}}%
\pgfpathcurveto{\pgfqpoint{0.924187in}{2.129522in}}{\pgfqpoint{0.932088in}{2.126250in}}{\pgfqpoint{0.940324in}{2.126250in}}%
\pgfpathclose%
\pgfusepath{stroke,fill}%
\end{pgfscope}%
\begin{pgfscope}%
\pgfpathrectangle{\pgfqpoint{0.100000in}{0.212622in}}{\pgfqpoint{3.696000in}{3.696000in}}%
\pgfusepath{clip}%
\pgfsetbuttcap%
\pgfsetroundjoin%
\definecolor{currentfill}{rgb}{0.121569,0.466667,0.705882}%
\pgfsetfillcolor{currentfill}%
\pgfsetfillopacity{0.875718}%
\pgfsetlinewidth{1.003750pt}%
\definecolor{currentstroke}{rgb}{0.121569,0.466667,0.705882}%
\pgfsetstrokecolor{currentstroke}%
\pgfsetstrokeopacity{0.875718}%
\pgfsetdash{}{0pt}%
\pgfpathmoveto{\pgfqpoint{2.244908in}{2.548204in}}%
\pgfpathcurveto{\pgfqpoint{2.253144in}{2.548204in}}{\pgfqpoint{2.261044in}{2.551477in}}{\pgfqpoint{2.266868in}{2.557300in}}%
\pgfpathcurveto{\pgfqpoint{2.272692in}{2.563124in}}{\pgfqpoint{2.275964in}{2.571024in}}{\pgfqpoint{2.275964in}{2.579261in}}%
\pgfpathcurveto{\pgfqpoint{2.275964in}{2.587497in}}{\pgfqpoint{2.272692in}{2.595397in}}{\pgfqpoint{2.266868in}{2.601221in}}%
\pgfpathcurveto{\pgfqpoint{2.261044in}{2.607045in}}{\pgfqpoint{2.253144in}{2.610317in}}{\pgfqpoint{2.244908in}{2.610317in}}%
\pgfpathcurveto{\pgfqpoint{2.236671in}{2.610317in}}{\pgfqpoint{2.228771in}{2.607045in}}{\pgfqpoint{2.222947in}{2.601221in}}%
\pgfpathcurveto{\pgfqpoint{2.217124in}{2.595397in}}{\pgfqpoint{2.213851in}{2.587497in}}{\pgfqpoint{2.213851in}{2.579261in}}%
\pgfpathcurveto{\pgfqpoint{2.213851in}{2.571024in}}{\pgfqpoint{2.217124in}{2.563124in}}{\pgfqpoint{2.222947in}{2.557300in}}%
\pgfpathcurveto{\pgfqpoint{2.228771in}{2.551477in}}{\pgfqpoint{2.236671in}{2.548204in}}{\pgfqpoint{2.244908in}{2.548204in}}%
\pgfpathclose%
\pgfusepath{stroke,fill}%
\end{pgfscope}%
\begin{pgfscope}%
\pgfpathrectangle{\pgfqpoint{0.100000in}{0.212622in}}{\pgfqpoint{3.696000in}{3.696000in}}%
\pgfusepath{clip}%
\pgfsetbuttcap%
\pgfsetroundjoin%
\definecolor{currentfill}{rgb}{0.121569,0.466667,0.705882}%
\pgfsetfillcolor{currentfill}%
\pgfsetfillopacity{0.875814}%
\pgfsetlinewidth{1.003750pt}%
\definecolor{currentstroke}{rgb}{0.121569,0.466667,0.705882}%
\pgfsetstrokecolor{currentstroke}%
\pgfsetstrokeopacity{0.875814}%
\pgfsetdash{}{0pt}%
\pgfpathmoveto{\pgfqpoint{0.940037in}{2.125828in}}%
\pgfpathcurveto{\pgfqpoint{0.948273in}{2.125828in}}{\pgfqpoint{0.956173in}{2.129100in}}{\pgfqpoint{0.961997in}{2.134924in}}%
\pgfpathcurveto{\pgfqpoint{0.967821in}{2.140748in}}{\pgfqpoint{0.971093in}{2.148648in}}{\pgfqpoint{0.971093in}{2.156884in}}%
\pgfpathcurveto{\pgfqpoint{0.971093in}{2.165121in}}{\pgfqpoint{0.967821in}{2.173021in}}{\pgfqpoint{0.961997in}{2.178845in}}%
\pgfpathcurveto{\pgfqpoint{0.956173in}{2.184668in}}{\pgfqpoint{0.948273in}{2.187941in}}{\pgfqpoint{0.940037in}{2.187941in}}%
\pgfpathcurveto{\pgfqpoint{0.931800in}{2.187941in}}{\pgfqpoint{0.923900in}{2.184668in}}{\pgfqpoint{0.918076in}{2.178845in}}%
\pgfpathcurveto{\pgfqpoint{0.912253in}{2.173021in}}{\pgfqpoint{0.908980in}{2.165121in}}{\pgfqpoint{0.908980in}{2.156884in}}%
\pgfpathcurveto{\pgfqpoint{0.908980in}{2.148648in}}{\pgfqpoint{0.912253in}{2.140748in}}{\pgfqpoint{0.918076in}{2.134924in}}%
\pgfpathcurveto{\pgfqpoint{0.923900in}{2.129100in}}{\pgfqpoint{0.931800in}{2.125828in}}{\pgfqpoint{0.940037in}{2.125828in}}%
\pgfpathclose%
\pgfusepath{stroke,fill}%
\end{pgfscope}%
\begin{pgfscope}%
\pgfpathrectangle{\pgfqpoint{0.100000in}{0.212622in}}{\pgfqpoint{3.696000in}{3.696000in}}%
\pgfusepath{clip}%
\pgfsetbuttcap%
\pgfsetroundjoin%
\definecolor{currentfill}{rgb}{0.121569,0.466667,0.705882}%
\pgfsetfillcolor{currentfill}%
\pgfsetfillopacity{0.875876}%
\pgfsetlinewidth{1.003750pt}%
\definecolor{currentstroke}{rgb}{0.121569,0.466667,0.705882}%
\pgfsetstrokecolor{currentstroke}%
\pgfsetstrokeopacity{0.875876}%
\pgfsetdash{}{0pt}%
\pgfpathmoveto{\pgfqpoint{0.939870in}{2.125595in}}%
\pgfpathcurveto{\pgfqpoint{0.948106in}{2.125595in}}{\pgfqpoint{0.956006in}{2.128867in}}{\pgfqpoint{0.961830in}{2.134691in}}%
\pgfpathcurveto{\pgfqpoint{0.967654in}{2.140515in}}{\pgfqpoint{0.970926in}{2.148415in}}{\pgfqpoint{0.970926in}{2.156651in}}%
\pgfpathcurveto{\pgfqpoint{0.970926in}{2.164887in}}{\pgfqpoint{0.967654in}{2.172788in}}{\pgfqpoint{0.961830in}{2.178611in}}%
\pgfpathcurveto{\pgfqpoint{0.956006in}{2.184435in}}{\pgfqpoint{0.948106in}{2.187708in}}{\pgfqpoint{0.939870in}{2.187708in}}%
\pgfpathcurveto{\pgfqpoint{0.931633in}{2.187708in}}{\pgfqpoint{0.923733in}{2.184435in}}{\pgfqpoint{0.917909in}{2.178611in}}%
\pgfpathcurveto{\pgfqpoint{0.912085in}{2.172788in}}{\pgfqpoint{0.908813in}{2.164887in}}{\pgfqpoint{0.908813in}{2.156651in}}%
\pgfpathcurveto{\pgfqpoint{0.908813in}{2.148415in}}{\pgfqpoint{0.912085in}{2.140515in}}{\pgfqpoint{0.917909in}{2.134691in}}%
\pgfpathcurveto{\pgfqpoint{0.923733in}{2.128867in}}{\pgfqpoint{0.931633in}{2.125595in}}{\pgfqpoint{0.939870in}{2.125595in}}%
\pgfpathclose%
\pgfusepath{stroke,fill}%
\end{pgfscope}%
\begin{pgfscope}%
\pgfpathrectangle{\pgfqpoint{0.100000in}{0.212622in}}{\pgfqpoint{3.696000in}{3.696000in}}%
\pgfusepath{clip}%
\pgfsetbuttcap%
\pgfsetroundjoin%
\definecolor{currentfill}{rgb}{0.121569,0.466667,0.705882}%
\pgfsetfillcolor{currentfill}%
\pgfsetfillopacity{0.875912}%
\pgfsetlinewidth{1.003750pt}%
\definecolor{currentstroke}{rgb}{0.121569,0.466667,0.705882}%
\pgfsetstrokecolor{currentstroke}%
\pgfsetstrokeopacity{0.875912}%
\pgfsetdash{}{0pt}%
\pgfpathmoveto{\pgfqpoint{0.939771in}{2.125478in}}%
\pgfpathcurveto{\pgfqpoint{0.948008in}{2.125478in}}{\pgfqpoint{0.955908in}{2.128750in}}{\pgfqpoint{0.961732in}{2.134574in}}%
\pgfpathcurveto{\pgfqpoint{0.967556in}{2.140398in}}{\pgfqpoint{0.970828in}{2.148298in}}{\pgfqpoint{0.970828in}{2.156534in}}%
\pgfpathcurveto{\pgfqpoint{0.970828in}{2.164770in}}{\pgfqpoint{0.967556in}{2.172670in}}{\pgfqpoint{0.961732in}{2.178494in}}%
\pgfpathcurveto{\pgfqpoint{0.955908in}{2.184318in}}{\pgfqpoint{0.948008in}{2.187591in}}{\pgfqpoint{0.939771in}{2.187591in}}%
\pgfpathcurveto{\pgfqpoint{0.931535in}{2.187591in}}{\pgfqpoint{0.923635in}{2.184318in}}{\pgfqpoint{0.917811in}{2.178494in}}%
\pgfpathcurveto{\pgfqpoint{0.911987in}{2.172670in}}{\pgfqpoint{0.908715in}{2.164770in}}{\pgfqpoint{0.908715in}{2.156534in}}%
\pgfpathcurveto{\pgfqpoint{0.908715in}{2.148298in}}{\pgfqpoint{0.911987in}{2.140398in}}{\pgfqpoint{0.917811in}{2.134574in}}%
\pgfpathcurveto{\pgfqpoint{0.923635in}{2.128750in}}{\pgfqpoint{0.931535in}{2.125478in}}{\pgfqpoint{0.939771in}{2.125478in}}%
\pgfpathclose%
\pgfusepath{stroke,fill}%
\end{pgfscope}%
\begin{pgfscope}%
\pgfpathrectangle{\pgfqpoint{0.100000in}{0.212622in}}{\pgfqpoint{3.696000in}{3.696000in}}%
\pgfusepath{clip}%
\pgfsetbuttcap%
\pgfsetroundjoin%
\definecolor{currentfill}{rgb}{0.121569,0.466667,0.705882}%
\pgfsetfillcolor{currentfill}%
\pgfsetfillopacity{0.875930}%
\pgfsetlinewidth{1.003750pt}%
\definecolor{currentstroke}{rgb}{0.121569,0.466667,0.705882}%
\pgfsetstrokecolor{currentstroke}%
\pgfsetstrokeopacity{0.875930}%
\pgfsetdash{}{0pt}%
\pgfpathmoveto{\pgfqpoint{0.939715in}{2.125410in}}%
\pgfpathcurveto{\pgfqpoint{0.947951in}{2.125410in}}{\pgfqpoint{0.955851in}{2.128682in}}{\pgfqpoint{0.961675in}{2.134506in}}%
\pgfpathcurveto{\pgfqpoint{0.967499in}{2.140330in}}{\pgfqpoint{0.970772in}{2.148230in}}{\pgfqpoint{0.970772in}{2.156466in}}%
\pgfpathcurveto{\pgfqpoint{0.970772in}{2.164702in}}{\pgfqpoint{0.967499in}{2.172602in}}{\pgfqpoint{0.961675in}{2.178426in}}%
\pgfpathcurveto{\pgfqpoint{0.955851in}{2.184250in}}{\pgfqpoint{0.947951in}{2.187523in}}{\pgfqpoint{0.939715in}{2.187523in}}%
\pgfpathcurveto{\pgfqpoint{0.931479in}{2.187523in}}{\pgfqpoint{0.923579in}{2.184250in}}{\pgfqpoint{0.917755in}{2.178426in}}%
\pgfpathcurveto{\pgfqpoint{0.911931in}{2.172602in}}{\pgfqpoint{0.908659in}{2.164702in}}{\pgfqpoint{0.908659in}{2.156466in}}%
\pgfpathcurveto{\pgfqpoint{0.908659in}{2.148230in}}{\pgfqpoint{0.911931in}{2.140330in}}{\pgfqpoint{0.917755in}{2.134506in}}%
\pgfpathcurveto{\pgfqpoint{0.923579in}{2.128682in}}{\pgfqpoint{0.931479in}{2.125410in}}{\pgfqpoint{0.939715in}{2.125410in}}%
\pgfpathclose%
\pgfusepath{stroke,fill}%
\end{pgfscope}%
\begin{pgfscope}%
\pgfpathrectangle{\pgfqpoint{0.100000in}{0.212622in}}{\pgfqpoint{3.696000in}{3.696000in}}%
\pgfusepath{clip}%
\pgfsetbuttcap%
\pgfsetroundjoin%
\definecolor{currentfill}{rgb}{0.121569,0.466667,0.705882}%
\pgfsetfillcolor{currentfill}%
\pgfsetfillopacity{0.875940}%
\pgfsetlinewidth{1.003750pt}%
\definecolor{currentstroke}{rgb}{0.121569,0.466667,0.705882}%
\pgfsetstrokecolor{currentstroke}%
\pgfsetstrokeopacity{0.875940}%
\pgfsetdash{}{0pt}%
\pgfpathmoveto{\pgfqpoint{0.939686in}{2.125372in}}%
\pgfpathcurveto{\pgfqpoint{0.947922in}{2.125372in}}{\pgfqpoint{0.955822in}{2.128644in}}{\pgfqpoint{0.961646in}{2.134468in}}%
\pgfpathcurveto{\pgfqpoint{0.967470in}{2.140292in}}{\pgfqpoint{0.970742in}{2.148192in}}{\pgfqpoint{0.970742in}{2.156428in}}%
\pgfpathcurveto{\pgfqpoint{0.970742in}{2.164664in}}{\pgfqpoint{0.967470in}{2.172564in}}{\pgfqpoint{0.961646in}{2.178388in}}%
\pgfpathcurveto{\pgfqpoint{0.955822in}{2.184212in}}{\pgfqpoint{0.947922in}{2.187485in}}{\pgfqpoint{0.939686in}{2.187485in}}%
\pgfpathcurveto{\pgfqpoint{0.931450in}{2.187485in}}{\pgfqpoint{0.923550in}{2.184212in}}{\pgfqpoint{0.917726in}{2.178388in}}%
\pgfpathcurveto{\pgfqpoint{0.911902in}{2.172564in}}{\pgfqpoint{0.908629in}{2.164664in}}{\pgfqpoint{0.908629in}{2.156428in}}%
\pgfpathcurveto{\pgfqpoint{0.908629in}{2.148192in}}{\pgfqpoint{0.911902in}{2.140292in}}{\pgfqpoint{0.917726in}{2.134468in}}%
\pgfpathcurveto{\pgfqpoint{0.923550in}{2.128644in}}{\pgfqpoint{0.931450in}{2.125372in}}{\pgfqpoint{0.939686in}{2.125372in}}%
\pgfpathclose%
\pgfusepath{stroke,fill}%
\end{pgfscope}%
\begin{pgfscope}%
\pgfpathrectangle{\pgfqpoint{0.100000in}{0.212622in}}{\pgfqpoint{3.696000in}{3.696000in}}%
\pgfusepath{clip}%
\pgfsetbuttcap%
\pgfsetroundjoin%
\definecolor{currentfill}{rgb}{0.121569,0.466667,0.705882}%
\pgfsetfillcolor{currentfill}%
\pgfsetfillopacity{0.875946}%
\pgfsetlinewidth{1.003750pt}%
\definecolor{currentstroke}{rgb}{0.121569,0.466667,0.705882}%
\pgfsetstrokecolor{currentstroke}%
\pgfsetstrokeopacity{0.875946}%
\pgfsetdash{}{0pt}%
\pgfpathmoveto{\pgfqpoint{0.939671in}{2.125351in}}%
\pgfpathcurveto{\pgfqpoint{0.947907in}{2.125351in}}{\pgfqpoint{0.955807in}{2.128624in}}{\pgfqpoint{0.961631in}{2.134448in}}%
\pgfpathcurveto{\pgfqpoint{0.967455in}{2.140272in}}{\pgfqpoint{0.970727in}{2.148172in}}{\pgfqpoint{0.970727in}{2.156408in}}%
\pgfpathcurveto{\pgfqpoint{0.970727in}{2.164644in}}{\pgfqpoint{0.967455in}{2.172544in}}{\pgfqpoint{0.961631in}{2.178368in}}%
\pgfpathcurveto{\pgfqpoint{0.955807in}{2.184192in}}{\pgfqpoint{0.947907in}{2.187464in}}{\pgfqpoint{0.939671in}{2.187464in}}%
\pgfpathcurveto{\pgfqpoint{0.931434in}{2.187464in}}{\pgfqpoint{0.923534in}{2.184192in}}{\pgfqpoint{0.917710in}{2.178368in}}%
\pgfpathcurveto{\pgfqpoint{0.911886in}{2.172544in}}{\pgfqpoint{0.908614in}{2.164644in}}{\pgfqpoint{0.908614in}{2.156408in}}%
\pgfpathcurveto{\pgfqpoint{0.908614in}{2.148172in}}{\pgfqpoint{0.911886in}{2.140272in}}{\pgfqpoint{0.917710in}{2.134448in}}%
\pgfpathcurveto{\pgfqpoint{0.923534in}{2.128624in}}{\pgfqpoint{0.931434in}{2.125351in}}{\pgfqpoint{0.939671in}{2.125351in}}%
\pgfpathclose%
\pgfusepath{stroke,fill}%
\end{pgfscope}%
\begin{pgfscope}%
\pgfpathrectangle{\pgfqpoint{0.100000in}{0.212622in}}{\pgfqpoint{3.696000in}{3.696000in}}%
\pgfusepath{clip}%
\pgfsetbuttcap%
\pgfsetroundjoin%
\definecolor{currentfill}{rgb}{0.121569,0.466667,0.705882}%
\pgfsetfillcolor{currentfill}%
\pgfsetfillopacity{0.875949}%
\pgfsetlinewidth{1.003750pt}%
\definecolor{currentstroke}{rgb}{0.121569,0.466667,0.705882}%
\pgfsetstrokecolor{currentstroke}%
\pgfsetstrokeopacity{0.875949}%
\pgfsetdash{}{0pt}%
\pgfpathmoveto{\pgfqpoint{0.939662in}{2.125340in}}%
\pgfpathcurveto{\pgfqpoint{0.947898in}{2.125340in}}{\pgfqpoint{0.955798in}{2.128612in}}{\pgfqpoint{0.961622in}{2.134436in}}%
\pgfpathcurveto{\pgfqpoint{0.967446in}{2.140260in}}{\pgfqpoint{0.970718in}{2.148160in}}{\pgfqpoint{0.970718in}{2.156397in}}%
\pgfpathcurveto{\pgfqpoint{0.970718in}{2.164633in}}{\pgfqpoint{0.967446in}{2.172533in}}{\pgfqpoint{0.961622in}{2.178357in}}%
\pgfpathcurveto{\pgfqpoint{0.955798in}{2.184181in}}{\pgfqpoint{0.947898in}{2.187453in}}{\pgfqpoint{0.939662in}{2.187453in}}%
\pgfpathcurveto{\pgfqpoint{0.931426in}{2.187453in}}{\pgfqpoint{0.923526in}{2.184181in}}{\pgfqpoint{0.917702in}{2.178357in}}%
\pgfpathcurveto{\pgfqpoint{0.911878in}{2.172533in}}{\pgfqpoint{0.908605in}{2.164633in}}{\pgfqpoint{0.908605in}{2.156397in}}%
\pgfpathcurveto{\pgfqpoint{0.908605in}{2.148160in}}{\pgfqpoint{0.911878in}{2.140260in}}{\pgfqpoint{0.917702in}{2.134436in}}%
\pgfpathcurveto{\pgfqpoint{0.923526in}{2.128612in}}{\pgfqpoint{0.931426in}{2.125340in}}{\pgfqpoint{0.939662in}{2.125340in}}%
\pgfpathclose%
\pgfusepath{stroke,fill}%
\end{pgfscope}%
\begin{pgfscope}%
\pgfpathrectangle{\pgfqpoint{0.100000in}{0.212622in}}{\pgfqpoint{3.696000in}{3.696000in}}%
\pgfusepath{clip}%
\pgfsetbuttcap%
\pgfsetroundjoin%
\definecolor{currentfill}{rgb}{0.121569,0.466667,0.705882}%
\pgfsetfillcolor{currentfill}%
\pgfsetfillopacity{0.875951}%
\pgfsetlinewidth{1.003750pt}%
\definecolor{currentstroke}{rgb}{0.121569,0.466667,0.705882}%
\pgfsetstrokecolor{currentstroke}%
\pgfsetstrokeopacity{0.875951}%
\pgfsetdash{}{0pt}%
\pgfpathmoveto{\pgfqpoint{0.939657in}{2.125335in}}%
\pgfpathcurveto{\pgfqpoint{0.947893in}{2.125335in}}{\pgfqpoint{0.955793in}{2.128607in}}{\pgfqpoint{0.961617in}{2.134431in}}%
\pgfpathcurveto{\pgfqpoint{0.967441in}{2.140255in}}{\pgfqpoint{0.970713in}{2.148155in}}{\pgfqpoint{0.970713in}{2.156391in}}%
\pgfpathcurveto{\pgfqpoint{0.970713in}{2.164628in}}{\pgfqpoint{0.967441in}{2.172528in}}{\pgfqpoint{0.961617in}{2.178352in}}%
\pgfpathcurveto{\pgfqpoint{0.955793in}{2.184175in}}{\pgfqpoint{0.947893in}{2.187448in}}{\pgfqpoint{0.939657in}{2.187448in}}%
\pgfpathcurveto{\pgfqpoint{0.931421in}{2.187448in}}{\pgfqpoint{0.923521in}{2.184175in}}{\pgfqpoint{0.917697in}{2.178352in}}%
\pgfpathcurveto{\pgfqpoint{0.911873in}{2.172528in}}{\pgfqpoint{0.908600in}{2.164628in}}{\pgfqpoint{0.908600in}{2.156391in}}%
\pgfpathcurveto{\pgfqpoint{0.908600in}{2.148155in}}{\pgfqpoint{0.911873in}{2.140255in}}{\pgfqpoint{0.917697in}{2.134431in}}%
\pgfpathcurveto{\pgfqpoint{0.923521in}{2.128607in}}{\pgfqpoint{0.931421in}{2.125335in}}{\pgfqpoint{0.939657in}{2.125335in}}%
\pgfpathclose%
\pgfusepath{stroke,fill}%
\end{pgfscope}%
\begin{pgfscope}%
\pgfpathrectangle{\pgfqpoint{0.100000in}{0.212622in}}{\pgfqpoint{3.696000in}{3.696000in}}%
\pgfusepath{clip}%
\pgfsetbuttcap%
\pgfsetroundjoin%
\definecolor{currentfill}{rgb}{0.121569,0.466667,0.705882}%
\pgfsetfillcolor{currentfill}%
\pgfsetfillopacity{0.875952}%
\pgfsetlinewidth{1.003750pt}%
\definecolor{currentstroke}{rgb}{0.121569,0.466667,0.705882}%
\pgfsetstrokecolor{currentstroke}%
\pgfsetstrokeopacity{0.875952}%
\pgfsetdash{}{0pt}%
\pgfpathmoveto{\pgfqpoint{0.939654in}{2.125331in}}%
\pgfpathcurveto{\pgfqpoint{0.947891in}{2.125331in}}{\pgfqpoint{0.955791in}{2.128604in}}{\pgfqpoint{0.961615in}{2.134428in}}%
\pgfpathcurveto{\pgfqpoint{0.967439in}{2.140252in}}{\pgfqpoint{0.970711in}{2.148152in}}{\pgfqpoint{0.970711in}{2.156388in}}%
\pgfpathcurveto{\pgfqpoint{0.970711in}{2.164624in}}{\pgfqpoint{0.967439in}{2.172524in}}{\pgfqpoint{0.961615in}{2.178348in}}%
\pgfpathcurveto{\pgfqpoint{0.955791in}{2.184172in}}{\pgfqpoint{0.947891in}{2.187444in}}{\pgfqpoint{0.939654in}{2.187444in}}%
\pgfpathcurveto{\pgfqpoint{0.931418in}{2.187444in}}{\pgfqpoint{0.923518in}{2.184172in}}{\pgfqpoint{0.917694in}{2.178348in}}%
\pgfpathcurveto{\pgfqpoint{0.911870in}{2.172524in}}{\pgfqpoint{0.908598in}{2.164624in}}{\pgfqpoint{0.908598in}{2.156388in}}%
\pgfpathcurveto{\pgfqpoint{0.908598in}{2.148152in}}{\pgfqpoint{0.911870in}{2.140252in}}{\pgfqpoint{0.917694in}{2.134428in}}%
\pgfpathcurveto{\pgfqpoint{0.923518in}{2.128604in}}{\pgfqpoint{0.931418in}{2.125331in}}{\pgfqpoint{0.939654in}{2.125331in}}%
\pgfpathclose%
\pgfusepath{stroke,fill}%
\end{pgfscope}%
\begin{pgfscope}%
\pgfpathrectangle{\pgfqpoint{0.100000in}{0.212622in}}{\pgfqpoint{3.696000in}{3.696000in}}%
\pgfusepath{clip}%
\pgfsetbuttcap%
\pgfsetroundjoin%
\definecolor{currentfill}{rgb}{0.121569,0.466667,0.705882}%
\pgfsetfillcolor{currentfill}%
\pgfsetfillopacity{0.875952}%
\pgfsetlinewidth{1.003750pt}%
\definecolor{currentstroke}{rgb}{0.121569,0.466667,0.705882}%
\pgfsetstrokecolor{currentstroke}%
\pgfsetstrokeopacity{0.875952}%
\pgfsetdash{}{0pt}%
\pgfpathmoveto{\pgfqpoint{0.939653in}{2.125329in}}%
\pgfpathcurveto{\pgfqpoint{0.947889in}{2.125329in}}{\pgfqpoint{0.955789in}{2.128602in}}{\pgfqpoint{0.961613in}{2.134426in}}%
\pgfpathcurveto{\pgfqpoint{0.967437in}{2.140250in}}{\pgfqpoint{0.970709in}{2.148150in}}{\pgfqpoint{0.970709in}{2.156386in}}%
\pgfpathcurveto{\pgfqpoint{0.970709in}{2.164622in}}{\pgfqpoint{0.967437in}{2.172522in}}{\pgfqpoint{0.961613in}{2.178346in}}%
\pgfpathcurveto{\pgfqpoint{0.955789in}{2.184170in}}{\pgfqpoint{0.947889in}{2.187442in}}{\pgfqpoint{0.939653in}{2.187442in}}%
\pgfpathcurveto{\pgfqpoint{0.931417in}{2.187442in}}{\pgfqpoint{0.923517in}{2.184170in}}{\pgfqpoint{0.917693in}{2.178346in}}%
\pgfpathcurveto{\pgfqpoint{0.911869in}{2.172522in}}{\pgfqpoint{0.908596in}{2.164622in}}{\pgfqpoint{0.908596in}{2.156386in}}%
\pgfpathcurveto{\pgfqpoint{0.908596in}{2.148150in}}{\pgfqpoint{0.911869in}{2.140250in}}{\pgfqpoint{0.917693in}{2.134426in}}%
\pgfpathcurveto{\pgfqpoint{0.923517in}{2.128602in}}{\pgfqpoint{0.931417in}{2.125329in}}{\pgfqpoint{0.939653in}{2.125329in}}%
\pgfpathclose%
\pgfusepath{stroke,fill}%
\end{pgfscope}%
\begin{pgfscope}%
\pgfpathrectangle{\pgfqpoint{0.100000in}{0.212622in}}{\pgfqpoint{3.696000in}{3.696000in}}%
\pgfusepath{clip}%
\pgfsetbuttcap%
\pgfsetroundjoin%
\definecolor{currentfill}{rgb}{0.121569,0.466667,0.705882}%
\pgfsetfillcolor{currentfill}%
\pgfsetfillopacity{0.875953}%
\pgfsetlinewidth{1.003750pt}%
\definecolor{currentstroke}{rgb}{0.121569,0.466667,0.705882}%
\pgfsetstrokecolor{currentstroke}%
\pgfsetstrokeopacity{0.875953}%
\pgfsetdash{}{0pt}%
\pgfpathmoveto{\pgfqpoint{0.939652in}{2.125328in}}%
\pgfpathcurveto{\pgfqpoint{0.947888in}{2.125328in}}{\pgfqpoint{0.955788in}{2.128601in}}{\pgfqpoint{0.961612in}{2.134425in}}%
\pgfpathcurveto{\pgfqpoint{0.967436in}{2.140249in}}{\pgfqpoint{0.970709in}{2.148149in}}{\pgfqpoint{0.970709in}{2.156385in}}%
\pgfpathcurveto{\pgfqpoint{0.970709in}{2.164621in}}{\pgfqpoint{0.967436in}{2.172521in}}{\pgfqpoint{0.961612in}{2.178345in}}%
\pgfpathcurveto{\pgfqpoint{0.955788in}{2.184169in}}{\pgfqpoint{0.947888in}{2.187441in}}{\pgfqpoint{0.939652in}{2.187441in}}%
\pgfpathcurveto{\pgfqpoint{0.931416in}{2.187441in}}{\pgfqpoint{0.923516in}{2.184169in}}{\pgfqpoint{0.917692in}{2.178345in}}%
\pgfpathcurveto{\pgfqpoint{0.911868in}{2.172521in}}{\pgfqpoint{0.908596in}{2.164621in}}{\pgfqpoint{0.908596in}{2.156385in}}%
\pgfpathcurveto{\pgfqpoint{0.908596in}{2.148149in}}{\pgfqpoint{0.911868in}{2.140249in}}{\pgfqpoint{0.917692in}{2.134425in}}%
\pgfpathcurveto{\pgfqpoint{0.923516in}{2.128601in}}{\pgfqpoint{0.931416in}{2.125328in}}{\pgfqpoint{0.939652in}{2.125328in}}%
\pgfpathclose%
\pgfusepath{stroke,fill}%
\end{pgfscope}%
\begin{pgfscope}%
\pgfpathrectangle{\pgfqpoint{0.100000in}{0.212622in}}{\pgfqpoint{3.696000in}{3.696000in}}%
\pgfusepath{clip}%
\pgfsetbuttcap%
\pgfsetroundjoin%
\definecolor{currentfill}{rgb}{0.121569,0.466667,0.705882}%
\pgfsetfillcolor{currentfill}%
\pgfsetfillopacity{0.875953}%
\pgfsetlinewidth{1.003750pt}%
\definecolor{currentstroke}{rgb}{0.121569,0.466667,0.705882}%
\pgfsetstrokecolor{currentstroke}%
\pgfsetstrokeopacity{0.875953}%
\pgfsetdash{}{0pt}%
\pgfpathmoveto{\pgfqpoint{0.939652in}{2.125328in}}%
\pgfpathcurveto{\pgfqpoint{0.947888in}{2.125328in}}{\pgfqpoint{0.955788in}{2.128600in}}{\pgfqpoint{0.961612in}{2.134424in}}%
\pgfpathcurveto{\pgfqpoint{0.967436in}{2.140248in}}{\pgfqpoint{0.970708in}{2.148148in}}{\pgfqpoint{0.970708in}{2.156384in}}%
\pgfpathcurveto{\pgfqpoint{0.970708in}{2.164621in}}{\pgfqpoint{0.967436in}{2.172521in}}{\pgfqpoint{0.961612in}{2.178345in}}%
\pgfpathcurveto{\pgfqpoint{0.955788in}{2.184169in}}{\pgfqpoint{0.947888in}{2.187441in}}{\pgfqpoint{0.939652in}{2.187441in}}%
\pgfpathcurveto{\pgfqpoint{0.931415in}{2.187441in}}{\pgfqpoint{0.923515in}{2.184169in}}{\pgfqpoint{0.917691in}{2.178345in}}%
\pgfpathcurveto{\pgfqpoint{0.911867in}{2.172521in}}{\pgfqpoint{0.908595in}{2.164621in}}{\pgfqpoint{0.908595in}{2.156384in}}%
\pgfpathcurveto{\pgfqpoint{0.908595in}{2.148148in}}{\pgfqpoint{0.911867in}{2.140248in}}{\pgfqpoint{0.917691in}{2.134424in}}%
\pgfpathcurveto{\pgfqpoint{0.923515in}{2.128600in}}{\pgfqpoint{0.931415in}{2.125328in}}{\pgfqpoint{0.939652in}{2.125328in}}%
\pgfpathclose%
\pgfusepath{stroke,fill}%
\end{pgfscope}%
\begin{pgfscope}%
\pgfpathrectangle{\pgfqpoint{0.100000in}{0.212622in}}{\pgfqpoint{3.696000in}{3.696000in}}%
\pgfusepath{clip}%
\pgfsetbuttcap%
\pgfsetroundjoin%
\definecolor{currentfill}{rgb}{0.121569,0.466667,0.705882}%
\pgfsetfillcolor{currentfill}%
\pgfsetfillopacity{0.875953}%
\pgfsetlinewidth{1.003750pt}%
\definecolor{currentstroke}{rgb}{0.121569,0.466667,0.705882}%
\pgfsetstrokecolor{currentstroke}%
\pgfsetstrokeopacity{0.875953}%
\pgfsetdash{}{0pt}%
\pgfpathmoveto{\pgfqpoint{0.939651in}{2.125328in}}%
\pgfpathcurveto{\pgfqpoint{0.947888in}{2.125328in}}{\pgfqpoint{0.955788in}{2.128600in}}{\pgfqpoint{0.961612in}{2.134424in}}%
\pgfpathcurveto{\pgfqpoint{0.967436in}{2.140248in}}{\pgfqpoint{0.970708in}{2.148148in}}{\pgfqpoint{0.970708in}{2.156384in}}%
\pgfpathcurveto{\pgfqpoint{0.970708in}{2.164620in}}{\pgfqpoint{0.967436in}{2.172520in}}{\pgfqpoint{0.961612in}{2.178344in}}%
\pgfpathcurveto{\pgfqpoint{0.955788in}{2.184168in}}{\pgfqpoint{0.947888in}{2.187441in}}{\pgfqpoint{0.939651in}{2.187441in}}%
\pgfpathcurveto{\pgfqpoint{0.931415in}{2.187441in}}{\pgfqpoint{0.923515in}{2.184168in}}{\pgfqpoint{0.917691in}{2.178344in}}%
\pgfpathcurveto{\pgfqpoint{0.911867in}{2.172520in}}{\pgfqpoint{0.908595in}{2.164620in}}{\pgfqpoint{0.908595in}{2.156384in}}%
\pgfpathcurveto{\pgfqpoint{0.908595in}{2.148148in}}{\pgfqpoint{0.911867in}{2.140248in}}{\pgfqpoint{0.917691in}{2.134424in}}%
\pgfpathcurveto{\pgfqpoint{0.923515in}{2.128600in}}{\pgfqpoint{0.931415in}{2.125328in}}{\pgfqpoint{0.939651in}{2.125328in}}%
\pgfpathclose%
\pgfusepath{stroke,fill}%
\end{pgfscope}%
\begin{pgfscope}%
\pgfpathrectangle{\pgfqpoint{0.100000in}{0.212622in}}{\pgfqpoint{3.696000in}{3.696000in}}%
\pgfusepath{clip}%
\pgfsetbuttcap%
\pgfsetroundjoin%
\definecolor{currentfill}{rgb}{0.121569,0.466667,0.705882}%
\pgfsetfillcolor{currentfill}%
\pgfsetfillopacity{0.875953}%
\pgfsetlinewidth{1.003750pt}%
\definecolor{currentstroke}{rgb}{0.121569,0.466667,0.705882}%
\pgfsetstrokecolor{currentstroke}%
\pgfsetstrokeopacity{0.875953}%
\pgfsetdash{}{0pt}%
\pgfpathmoveto{\pgfqpoint{0.939651in}{2.125327in}}%
\pgfpathcurveto{\pgfqpoint{0.947887in}{2.125327in}}{\pgfqpoint{0.955788in}{2.128600in}}{\pgfqpoint{0.961611in}{2.134424in}}%
\pgfpathcurveto{\pgfqpoint{0.967435in}{2.140248in}}{\pgfqpoint{0.970708in}{2.148148in}}{\pgfqpoint{0.970708in}{2.156384in}}%
\pgfpathcurveto{\pgfqpoint{0.970708in}{2.164620in}}{\pgfqpoint{0.967435in}{2.172520in}}{\pgfqpoint{0.961611in}{2.178344in}}%
\pgfpathcurveto{\pgfqpoint{0.955788in}{2.184168in}}{\pgfqpoint{0.947887in}{2.187440in}}{\pgfqpoint{0.939651in}{2.187440in}}%
\pgfpathcurveto{\pgfqpoint{0.931415in}{2.187440in}}{\pgfqpoint{0.923515in}{2.184168in}}{\pgfqpoint{0.917691in}{2.178344in}}%
\pgfpathcurveto{\pgfqpoint{0.911867in}{2.172520in}}{\pgfqpoint{0.908595in}{2.164620in}}{\pgfqpoint{0.908595in}{2.156384in}}%
\pgfpathcurveto{\pgfqpoint{0.908595in}{2.148148in}}{\pgfqpoint{0.911867in}{2.140248in}}{\pgfqpoint{0.917691in}{2.134424in}}%
\pgfpathcurveto{\pgfqpoint{0.923515in}{2.128600in}}{\pgfqpoint{0.931415in}{2.125327in}}{\pgfqpoint{0.939651in}{2.125327in}}%
\pgfpathclose%
\pgfusepath{stroke,fill}%
\end{pgfscope}%
\begin{pgfscope}%
\pgfpathrectangle{\pgfqpoint{0.100000in}{0.212622in}}{\pgfqpoint{3.696000in}{3.696000in}}%
\pgfusepath{clip}%
\pgfsetbuttcap%
\pgfsetroundjoin%
\definecolor{currentfill}{rgb}{0.121569,0.466667,0.705882}%
\pgfsetfillcolor{currentfill}%
\pgfsetfillopacity{0.875953}%
\pgfsetlinewidth{1.003750pt}%
\definecolor{currentstroke}{rgb}{0.121569,0.466667,0.705882}%
\pgfsetstrokecolor{currentstroke}%
\pgfsetstrokeopacity{0.875953}%
\pgfsetdash{}{0pt}%
\pgfpathmoveto{\pgfqpoint{0.939651in}{2.125327in}}%
\pgfpathcurveto{\pgfqpoint{0.947887in}{2.125327in}}{\pgfqpoint{0.955787in}{2.128600in}}{\pgfqpoint{0.961611in}{2.134424in}}%
\pgfpathcurveto{\pgfqpoint{0.967435in}{2.140248in}}{\pgfqpoint{0.970708in}{2.148148in}}{\pgfqpoint{0.970708in}{2.156384in}}%
\pgfpathcurveto{\pgfqpoint{0.970708in}{2.164620in}}{\pgfqpoint{0.967435in}{2.172520in}}{\pgfqpoint{0.961611in}{2.178344in}}%
\pgfpathcurveto{\pgfqpoint{0.955787in}{2.184168in}}{\pgfqpoint{0.947887in}{2.187440in}}{\pgfqpoint{0.939651in}{2.187440in}}%
\pgfpathcurveto{\pgfqpoint{0.931415in}{2.187440in}}{\pgfqpoint{0.923515in}{2.184168in}}{\pgfqpoint{0.917691in}{2.178344in}}%
\pgfpathcurveto{\pgfqpoint{0.911867in}{2.172520in}}{\pgfqpoint{0.908595in}{2.164620in}}{\pgfqpoint{0.908595in}{2.156384in}}%
\pgfpathcurveto{\pgfqpoint{0.908595in}{2.148148in}}{\pgfqpoint{0.911867in}{2.140248in}}{\pgfqpoint{0.917691in}{2.134424in}}%
\pgfpathcurveto{\pgfqpoint{0.923515in}{2.128600in}}{\pgfqpoint{0.931415in}{2.125327in}}{\pgfqpoint{0.939651in}{2.125327in}}%
\pgfpathclose%
\pgfusepath{stroke,fill}%
\end{pgfscope}%
\begin{pgfscope}%
\pgfpathrectangle{\pgfqpoint{0.100000in}{0.212622in}}{\pgfqpoint{3.696000in}{3.696000in}}%
\pgfusepath{clip}%
\pgfsetbuttcap%
\pgfsetroundjoin%
\definecolor{currentfill}{rgb}{0.121569,0.466667,0.705882}%
\pgfsetfillcolor{currentfill}%
\pgfsetfillopacity{0.875953}%
\pgfsetlinewidth{1.003750pt}%
\definecolor{currentstroke}{rgb}{0.121569,0.466667,0.705882}%
\pgfsetstrokecolor{currentstroke}%
\pgfsetstrokeopacity{0.875953}%
\pgfsetdash{}{0pt}%
\pgfpathmoveto{\pgfqpoint{0.939651in}{2.125327in}}%
\pgfpathcurveto{\pgfqpoint{0.947887in}{2.125327in}}{\pgfqpoint{0.955787in}{2.128600in}}{\pgfqpoint{0.961611in}{2.134424in}}%
\pgfpathcurveto{\pgfqpoint{0.967435in}{2.140247in}}{\pgfqpoint{0.970708in}{2.148148in}}{\pgfqpoint{0.970708in}{2.156384in}}%
\pgfpathcurveto{\pgfqpoint{0.970708in}{2.164620in}}{\pgfqpoint{0.967435in}{2.172520in}}{\pgfqpoint{0.961611in}{2.178344in}}%
\pgfpathcurveto{\pgfqpoint{0.955787in}{2.184168in}}{\pgfqpoint{0.947887in}{2.187440in}}{\pgfqpoint{0.939651in}{2.187440in}}%
\pgfpathcurveto{\pgfqpoint{0.931415in}{2.187440in}}{\pgfqpoint{0.923515in}{2.184168in}}{\pgfqpoint{0.917691in}{2.178344in}}%
\pgfpathcurveto{\pgfqpoint{0.911867in}{2.172520in}}{\pgfqpoint{0.908595in}{2.164620in}}{\pgfqpoint{0.908595in}{2.156384in}}%
\pgfpathcurveto{\pgfqpoint{0.908595in}{2.148148in}}{\pgfqpoint{0.911867in}{2.140247in}}{\pgfqpoint{0.917691in}{2.134424in}}%
\pgfpathcurveto{\pgfqpoint{0.923515in}{2.128600in}}{\pgfqpoint{0.931415in}{2.125327in}}{\pgfqpoint{0.939651in}{2.125327in}}%
\pgfpathclose%
\pgfusepath{stroke,fill}%
\end{pgfscope}%
\begin{pgfscope}%
\pgfpathrectangle{\pgfqpoint{0.100000in}{0.212622in}}{\pgfqpoint{3.696000in}{3.696000in}}%
\pgfusepath{clip}%
\pgfsetbuttcap%
\pgfsetroundjoin%
\definecolor{currentfill}{rgb}{0.121569,0.466667,0.705882}%
\pgfsetfillcolor{currentfill}%
\pgfsetfillopacity{0.875953}%
\pgfsetlinewidth{1.003750pt}%
\definecolor{currentstroke}{rgb}{0.121569,0.466667,0.705882}%
\pgfsetstrokecolor{currentstroke}%
\pgfsetstrokeopacity{0.875953}%
\pgfsetdash{}{0pt}%
\pgfpathmoveto{\pgfqpoint{0.939651in}{2.125327in}}%
\pgfpathcurveto{\pgfqpoint{0.947887in}{2.125327in}}{\pgfqpoint{0.955787in}{2.128600in}}{\pgfqpoint{0.961611in}{2.134424in}}%
\pgfpathcurveto{\pgfqpoint{0.967435in}{2.140247in}}{\pgfqpoint{0.970708in}{2.148147in}}{\pgfqpoint{0.970708in}{2.156384in}}%
\pgfpathcurveto{\pgfqpoint{0.970708in}{2.164620in}}{\pgfqpoint{0.967435in}{2.172520in}}{\pgfqpoint{0.961611in}{2.178344in}}%
\pgfpathcurveto{\pgfqpoint{0.955787in}{2.184168in}}{\pgfqpoint{0.947887in}{2.187440in}}{\pgfqpoint{0.939651in}{2.187440in}}%
\pgfpathcurveto{\pgfqpoint{0.931415in}{2.187440in}}{\pgfqpoint{0.923515in}{2.184168in}}{\pgfqpoint{0.917691in}{2.178344in}}%
\pgfpathcurveto{\pgfqpoint{0.911867in}{2.172520in}}{\pgfqpoint{0.908595in}{2.164620in}}{\pgfqpoint{0.908595in}{2.156384in}}%
\pgfpathcurveto{\pgfqpoint{0.908595in}{2.148147in}}{\pgfqpoint{0.911867in}{2.140247in}}{\pgfqpoint{0.917691in}{2.134424in}}%
\pgfpathcurveto{\pgfqpoint{0.923515in}{2.128600in}}{\pgfqpoint{0.931415in}{2.125327in}}{\pgfqpoint{0.939651in}{2.125327in}}%
\pgfpathclose%
\pgfusepath{stroke,fill}%
\end{pgfscope}%
\begin{pgfscope}%
\pgfpathrectangle{\pgfqpoint{0.100000in}{0.212622in}}{\pgfqpoint{3.696000in}{3.696000in}}%
\pgfusepath{clip}%
\pgfsetbuttcap%
\pgfsetroundjoin%
\definecolor{currentfill}{rgb}{0.121569,0.466667,0.705882}%
\pgfsetfillcolor{currentfill}%
\pgfsetfillopacity{0.875953}%
\pgfsetlinewidth{1.003750pt}%
\definecolor{currentstroke}{rgb}{0.121569,0.466667,0.705882}%
\pgfsetstrokecolor{currentstroke}%
\pgfsetstrokeopacity{0.875953}%
\pgfsetdash{}{0pt}%
\pgfpathmoveto{\pgfqpoint{0.939651in}{2.125327in}}%
\pgfpathcurveto{\pgfqpoint{0.947887in}{2.125327in}}{\pgfqpoint{0.955787in}{2.128600in}}{\pgfqpoint{0.961611in}{2.134423in}}%
\pgfpathcurveto{\pgfqpoint{0.967435in}{2.140247in}}{\pgfqpoint{0.970708in}{2.148147in}}{\pgfqpoint{0.970708in}{2.156384in}}%
\pgfpathcurveto{\pgfqpoint{0.970708in}{2.164620in}}{\pgfqpoint{0.967435in}{2.172520in}}{\pgfqpoint{0.961611in}{2.178344in}}%
\pgfpathcurveto{\pgfqpoint{0.955787in}{2.184168in}}{\pgfqpoint{0.947887in}{2.187440in}}{\pgfqpoint{0.939651in}{2.187440in}}%
\pgfpathcurveto{\pgfqpoint{0.931415in}{2.187440in}}{\pgfqpoint{0.923515in}{2.184168in}}{\pgfqpoint{0.917691in}{2.178344in}}%
\pgfpathcurveto{\pgfqpoint{0.911867in}{2.172520in}}{\pgfqpoint{0.908595in}{2.164620in}}{\pgfqpoint{0.908595in}{2.156384in}}%
\pgfpathcurveto{\pgfqpoint{0.908595in}{2.148147in}}{\pgfqpoint{0.911867in}{2.140247in}}{\pgfqpoint{0.917691in}{2.134423in}}%
\pgfpathcurveto{\pgfqpoint{0.923515in}{2.128600in}}{\pgfqpoint{0.931415in}{2.125327in}}{\pgfqpoint{0.939651in}{2.125327in}}%
\pgfpathclose%
\pgfusepath{stroke,fill}%
\end{pgfscope}%
\begin{pgfscope}%
\pgfpathrectangle{\pgfqpoint{0.100000in}{0.212622in}}{\pgfqpoint{3.696000in}{3.696000in}}%
\pgfusepath{clip}%
\pgfsetbuttcap%
\pgfsetroundjoin%
\definecolor{currentfill}{rgb}{0.121569,0.466667,0.705882}%
\pgfsetfillcolor{currentfill}%
\pgfsetfillopacity{0.875953}%
\pgfsetlinewidth{1.003750pt}%
\definecolor{currentstroke}{rgb}{0.121569,0.466667,0.705882}%
\pgfsetstrokecolor{currentstroke}%
\pgfsetstrokeopacity{0.875953}%
\pgfsetdash{}{0pt}%
\pgfpathmoveto{\pgfqpoint{0.939651in}{2.125327in}}%
\pgfpathcurveto{\pgfqpoint{0.947887in}{2.125327in}}{\pgfqpoint{0.955787in}{2.128600in}}{\pgfqpoint{0.961611in}{2.134423in}}%
\pgfpathcurveto{\pgfqpoint{0.967435in}{2.140247in}}{\pgfqpoint{0.970708in}{2.148147in}}{\pgfqpoint{0.970708in}{2.156384in}}%
\pgfpathcurveto{\pgfqpoint{0.970708in}{2.164620in}}{\pgfqpoint{0.967435in}{2.172520in}}{\pgfqpoint{0.961611in}{2.178344in}}%
\pgfpathcurveto{\pgfqpoint{0.955787in}{2.184168in}}{\pgfqpoint{0.947887in}{2.187440in}}{\pgfqpoint{0.939651in}{2.187440in}}%
\pgfpathcurveto{\pgfqpoint{0.931415in}{2.187440in}}{\pgfqpoint{0.923515in}{2.184168in}}{\pgfqpoint{0.917691in}{2.178344in}}%
\pgfpathcurveto{\pgfqpoint{0.911867in}{2.172520in}}{\pgfqpoint{0.908595in}{2.164620in}}{\pgfqpoint{0.908595in}{2.156384in}}%
\pgfpathcurveto{\pgfqpoint{0.908595in}{2.148147in}}{\pgfqpoint{0.911867in}{2.140247in}}{\pgfqpoint{0.917691in}{2.134423in}}%
\pgfpathcurveto{\pgfqpoint{0.923515in}{2.128600in}}{\pgfqpoint{0.931415in}{2.125327in}}{\pgfqpoint{0.939651in}{2.125327in}}%
\pgfpathclose%
\pgfusepath{stroke,fill}%
\end{pgfscope}%
\begin{pgfscope}%
\pgfpathrectangle{\pgfqpoint{0.100000in}{0.212622in}}{\pgfqpoint{3.696000in}{3.696000in}}%
\pgfusepath{clip}%
\pgfsetbuttcap%
\pgfsetroundjoin%
\definecolor{currentfill}{rgb}{0.121569,0.466667,0.705882}%
\pgfsetfillcolor{currentfill}%
\pgfsetfillopacity{0.875953}%
\pgfsetlinewidth{1.003750pt}%
\definecolor{currentstroke}{rgb}{0.121569,0.466667,0.705882}%
\pgfsetstrokecolor{currentstroke}%
\pgfsetstrokeopacity{0.875953}%
\pgfsetdash{}{0pt}%
\pgfpathmoveto{\pgfqpoint{0.939651in}{2.125327in}}%
\pgfpathcurveto{\pgfqpoint{0.947887in}{2.125327in}}{\pgfqpoint{0.955787in}{2.128600in}}{\pgfqpoint{0.961611in}{2.134423in}}%
\pgfpathcurveto{\pgfqpoint{0.967435in}{2.140247in}}{\pgfqpoint{0.970708in}{2.148147in}}{\pgfqpoint{0.970708in}{2.156384in}}%
\pgfpathcurveto{\pgfqpoint{0.970708in}{2.164620in}}{\pgfqpoint{0.967435in}{2.172520in}}{\pgfqpoint{0.961611in}{2.178344in}}%
\pgfpathcurveto{\pgfqpoint{0.955787in}{2.184168in}}{\pgfqpoint{0.947887in}{2.187440in}}{\pgfqpoint{0.939651in}{2.187440in}}%
\pgfpathcurveto{\pgfqpoint{0.931415in}{2.187440in}}{\pgfqpoint{0.923515in}{2.184168in}}{\pgfqpoint{0.917691in}{2.178344in}}%
\pgfpathcurveto{\pgfqpoint{0.911867in}{2.172520in}}{\pgfqpoint{0.908595in}{2.164620in}}{\pgfqpoint{0.908595in}{2.156384in}}%
\pgfpathcurveto{\pgfqpoint{0.908595in}{2.148147in}}{\pgfqpoint{0.911867in}{2.140247in}}{\pgfqpoint{0.917691in}{2.134423in}}%
\pgfpathcurveto{\pgfqpoint{0.923515in}{2.128600in}}{\pgfqpoint{0.931415in}{2.125327in}}{\pgfqpoint{0.939651in}{2.125327in}}%
\pgfpathclose%
\pgfusepath{stroke,fill}%
\end{pgfscope}%
\begin{pgfscope}%
\pgfpathrectangle{\pgfqpoint{0.100000in}{0.212622in}}{\pgfqpoint{3.696000in}{3.696000in}}%
\pgfusepath{clip}%
\pgfsetbuttcap%
\pgfsetroundjoin%
\definecolor{currentfill}{rgb}{0.121569,0.466667,0.705882}%
\pgfsetfillcolor{currentfill}%
\pgfsetfillopacity{0.875953}%
\pgfsetlinewidth{1.003750pt}%
\definecolor{currentstroke}{rgb}{0.121569,0.466667,0.705882}%
\pgfsetstrokecolor{currentstroke}%
\pgfsetstrokeopacity{0.875953}%
\pgfsetdash{}{0pt}%
\pgfpathmoveto{\pgfqpoint{0.939651in}{2.125327in}}%
\pgfpathcurveto{\pgfqpoint{0.947887in}{2.125327in}}{\pgfqpoint{0.955787in}{2.128600in}}{\pgfqpoint{0.961611in}{2.134423in}}%
\pgfpathcurveto{\pgfqpoint{0.967435in}{2.140247in}}{\pgfqpoint{0.970708in}{2.148147in}}{\pgfqpoint{0.970708in}{2.156384in}}%
\pgfpathcurveto{\pgfqpoint{0.970708in}{2.164620in}}{\pgfqpoint{0.967435in}{2.172520in}}{\pgfqpoint{0.961611in}{2.178344in}}%
\pgfpathcurveto{\pgfqpoint{0.955787in}{2.184168in}}{\pgfqpoint{0.947887in}{2.187440in}}{\pgfqpoint{0.939651in}{2.187440in}}%
\pgfpathcurveto{\pgfqpoint{0.931415in}{2.187440in}}{\pgfqpoint{0.923515in}{2.184168in}}{\pgfqpoint{0.917691in}{2.178344in}}%
\pgfpathcurveto{\pgfqpoint{0.911867in}{2.172520in}}{\pgfqpoint{0.908595in}{2.164620in}}{\pgfqpoint{0.908595in}{2.156384in}}%
\pgfpathcurveto{\pgfqpoint{0.908595in}{2.148147in}}{\pgfqpoint{0.911867in}{2.140247in}}{\pgfqpoint{0.917691in}{2.134423in}}%
\pgfpathcurveto{\pgfqpoint{0.923515in}{2.128600in}}{\pgfqpoint{0.931415in}{2.125327in}}{\pgfqpoint{0.939651in}{2.125327in}}%
\pgfpathclose%
\pgfusepath{stroke,fill}%
\end{pgfscope}%
\begin{pgfscope}%
\pgfpathrectangle{\pgfqpoint{0.100000in}{0.212622in}}{\pgfqpoint{3.696000in}{3.696000in}}%
\pgfusepath{clip}%
\pgfsetbuttcap%
\pgfsetroundjoin%
\definecolor{currentfill}{rgb}{0.121569,0.466667,0.705882}%
\pgfsetfillcolor{currentfill}%
\pgfsetfillopacity{0.875953}%
\pgfsetlinewidth{1.003750pt}%
\definecolor{currentstroke}{rgb}{0.121569,0.466667,0.705882}%
\pgfsetstrokecolor{currentstroke}%
\pgfsetstrokeopacity{0.875953}%
\pgfsetdash{}{0pt}%
\pgfpathmoveto{\pgfqpoint{0.939651in}{2.125327in}}%
\pgfpathcurveto{\pgfqpoint{0.947887in}{2.125327in}}{\pgfqpoint{0.955787in}{2.128600in}}{\pgfqpoint{0.961611in}{2.134423in}}%
\pgfpathcurveto{\pgfqpoint{0.967435in}{2.140247in}}{\pgfqpoint{0.970708in}{2.148147in}}{\pgfqpoint{0.970708in}{2.156384in}}%
\pgfpathcurveto{\pgfqpoint{0.970708in}{2.164620in}}{\pgfqpoint{0.967435in}{2.172520in}}{\pgfqpoint{0.961611in}{2.178344in}}%
\pgfpathcurveto{\pgfqpoint{0.955787in}{2.184168in}}{\pgfqpoint{0.947887in}{2.187440in}}{\pgfqpoint{0.939651in}{2.187440in}}%
\pgfpathcurveto{\pgfqpoint{0.931415in}{2.187440in}}{\pgfqpoint{0.923515in}{2.184168in}}{\pgfqpoint{0.917691in}{2.178344in}}%
\pgfpathcurveto{\pgfqpoint{0.911867in}{2.172520in}}{\pgfqpoint{0.908595in}{2.164620in}}{\pgfqpoint{0.908595in}{2.156384in}}%
\pgfpathcurveto{\pgfqpoint{0.908595in}{2.148147in}}{\pgfqpoint{0.911867in}{2.140247in}}{\pgfqpoint{0.917691in}{2.134423in}}%
\pgfpathcurveto{\pgfqpoint{0.923515in}{2.128600in}}{\pgfqpoint{0.931415in}{2.125327in}}{\pgfqpoint{0.939651in}{2.125327in}}%
\pgfpathclose%
\pgfusepath{stroke,fill}%
\end{pgfscope}%
\begin{pgfscope}%
\pgfpathrectangle{\pgfqpoint{0.100000in}{0.212622in}}{\pgfqpoint{3.696000in}{3.696000in}}%
\pgfusepath{clip}%
\pgfsetbuttcap%
\pgfsetroundjoin%
\definecolor{currentfill}{rgb}{0.121569,0.466667,0.705882}%
\pgfsetfillcolor{currentfill}%
\pgfsetfillopacity{0.875953}%
\pgfsetlinewidth{1.003750pt}%
\definecolor{currentstroke}{rgb}{0.121569,0.466667,0.705882}%
\pgfsetstrokecolor{currentstroke}%
\pgfsetstrokeopacity{0.875953}%
\pgfsetdash{}{0pt}%
\pgfpathmoveto{\pgfqpoint{0.939651in}{2.125327in}}%
\pgfpathcurveto{\pgfqpoint{0.947887in}{2.125327in}}{\pgfqpoint{0.955787in}{2.128600in}}{\pgfqpoint{0.961611in}{2.134423in}}%
\pgfpathcurveto{\pgfqpoint{0.967435in}{2.140247in}}{\pgfqpoint{0.970708in}{2.148147in}}{\pgfqpoint{0.970708in}{2.156384in}}%
\pgfpathcurveto{\pgfqpoint{0.970708in}{2.164620in}}{\pgfqpoint{0.967435in}{2.172520in}}{\pgfqpoint{0.961611in}{2.178344in}}%
\pgfpathcurveto{\pgfqpoint{0.955787in}{2.184168in}}{\pgfqpoint{0.947887in}{2.187440in}}{\pgfqpoint{0.939651in}{2.187440in}}%
\pgfpathcurveto{\pgfqpoint{0.931415in}{2.187440in}}{\pgfqpoint{0.923515in}{2.184168in}}{\pgfqpoint{0.917691in}{2.178344in}}%
\pgfpathcurveto{\pgfqpoint{0.911867in}{2.172520in}}{\pgfqpoint{0.908595in}{2.164620in}}{\pgfqpoint{0.908595in}{2.156384in}}%
\pgfpathcurveto{\pgfqpoint{0.908595in}{2.148147in}}{\pgfqpoint{0.911867in}{2.140247in}}{\pgfqpoint{0.917691in}{2.134423in}}%
\pgfpathcurveto{\pgfqpoint{0.923515in}{2.128600in}}{\pgfqpoint{0.931415in}{2.125327in}}{\pgfqpoint{0.939651in}{2.125327in}}%
\pgfpathclose%
\pgfusepath{stroke,fill}%
\end{pgfscope}%
\begin{pgfscope}%
\pgfpathrectangle{\pgfqpoint{0.100000in}{0.212622in}}{\pgfqpoint{3.696000in}{3.696000in}}%
\pgfusepath{clip}%
\pgfsetbuttcap%
\pgfsetroundjoin%
\definecolor{currentfill}{rgb}{0.121569,0.466667,0.705882}%
\pgfsetfillcolor{currentfill}%
\pgfsetfillopacity{0.875953}%
\pgfsetlinewidth{1.003750pt}%
\definecolor{currentstroke}{rgb}{0.121569,0.466667,0.705882}%
\pgfsetstrokecolor{currentstroke}%
\pgfsetstrokeopacity{0.875953}%
\pgfsetdash{}{0pt}%
\pgfpathmoveto{\pgfqpoint{0.939651in}{2.125327in}}%
\pgfpathcurveto{\pgfqpoint{0.947887in}{2.125327in}}{\pgfqpoint{0.955787in}{2.128600in}}{\pgfqpoint{0.961611in}{2.134423in}}%
\pgfpathcurveto{\pgfqpoint{0.967435in}{2.140247in}}{\pgfqpoint{0.970708in}{2.148147in}}{\pgfqpoint{0.970708in}{2.156384in}}%
\pgfpathcurveto{\pgfqpoint{0.970708in}{2.164620in}}{\pgfqpoint{0.967435in}{2.172520in}}{\pgfqpoint{0.961611in}{2.178344in}}%
\pgfpathcurveto{\pgfqpoint{0.955787in}{2.184168in}}{\pgfqpoint{0.947887in}{2.187440in}}{\pgfqpoint{0.939651in}{2.187440in}}%
\pgfpathcurveto{\pgfqpoint{0.931415in}{2.187440in}}{\pgfqpoint{0.923515in}{2.184168in}}{\pgfqpoint{0.917691in}{2.178344in}}%
\pgfpathcurveto{\pgfqpoint{0.911867in}{2.172520in}}{\pgfqpoint{0.908595in}{2.164620in}}{\pgfqpoint{0.908595in}{2.156384in}}%
\pgfpathcurveto{\pgfqpoint{0.908595in}{2.148147in}}{\pgfqpoint{0.911867in}{2.140247in}}{\pgfqpoint{0.917691in}{2.134423in}}%
\pgfpathcurveto{\pgfqpoint{0.923515in}{2.128600in}}{\pgfqpoint{0.931415in}{2.125327in}}{\pgfqpoint{0.939651in}{2.125327in}}%
\pgfpathclose%
\pgfusepath{stroke,fill}%
\end{pgfscope}%
\begin{pgfscope}%
\pgfpathrectangle{\pgfqpoint{0.100000in}{0.212622in}}{\pgfqpoint{3.696000in}{3.696000in}}%
\pgfusepath{clip}%
\pgfsetbuttcap%
\pgfsetroundjoin%
\definecolor{currentfill}{rgb}{0.121569,0.466667,0.705882}%
\pgfsetfillcolor{currentfill}%
\pgfsetfillopacity{0.875953}%
\pgfsetlinewidth{1.003750pt}%
\definecolor{currentstroke}{rgb}{0.121569,0.466667,0.705882}%
\pgfsetstrokecolor{currentstroke}%
\pgfsetstrokeopacity{0.875953}%
\pgfsetdash{}{0pt}%
\pgfpathmoveto{\pgfqpoint{0.939651in}{2.125327in}}%
\pgfpathcurveto{\pgfqpoint{0.947887in}{2.125327in}}{\pgfqpoint{0.955787in}{2.128600in}}{\pgfqpoint{0.961611in}{2.134423in}}%
\pgfpathcurveto{\pgfqpoint{0.967435in}{2.140247in}}{\pgfqpoint{0.970708in}{2.148147in}}{\pgfqpoint{0.970708in}{2.156384in}}%
\pgfpathcurveto{\pgfqpoint{0.970708in}{2.164620in}}{\pgfqpoint{0.967435in}{2.172520in}}{\pgfqpoint{0.961611in}{2.178344in}}%
\pgfpathcurveto{\pgfqpoint{0.955787in}{2.184168in}}{\pgfqpoint{0.947887in}{2.187440in}}{\pgfqpoint{0.939651in}{2.187440in}}%
\pgfpathcurveto{\pgfqpoint{0.931415in}{2.187440in}}{\pgfqpoint{0.923515in}{2.184168in}}{\pgfqpoint{0.917691in}{2.178344in}}%
\pgfpathcurveto{\pgfqpoint{0.911867in}{2.172520in}}{\pgfqpoint{0.908595in}{2.164620in}}{\pgfqpoint{0.908595in}{2.156384in}}%
\pgfpathcurveto{\pgfqpoint{0.908595in}{2.148147in}}{\pgfqpoint{0.911867in}{2.140247in}}{\pgfqpoint{0.917691in}{2.134423in}}%
\pgfpathcurveto{\pgfqpoint{0.923515in}{2.128600in}}{\pgfqpoint{0.931415in}{2.125327in}}{\pgfqpoint{0.939651in}{2.125327in}}%
\pgfpathclose%
\pgfusepath{stroke,fill}%
\end{pgfscope}%
\begin{pgfscope}%
\pgfpathrectangle{\pgfqpoint{0.100000in}{0.212622in}}{\pgfqpoint{3.696000in}{3.696000in}}%
\pgfusepath{clip}%
\pgfsetbuttcap%
\pgfsetroundjoin%
\definecolor{currentfill}{rgb}{0.121569,0.466667,0.705882}%
\pgfsetfillcolor{currentfill}%
\pgfsetfillopacity{0.875953}%
\pgfsetlinewidth{1.003750pt}%
\definecolor{currentstroke}{rgb}{0.121569,0.466667,0.705882}%
\pgfsetstrokecolor{currentstroke}%
\pgfsetstrokeopacity{0.875953}%
\pgfsetdash{}{0pt}%
\pgfpathmoveto{\pgfqpoint{0.939651in}{2.125327in}}%
\pgfpathcurveto{\pgfqpoint{0.947887in}{2.125327in}}{\pgfqpoint{0.955787in}{2.128600in}}{\pgfqpoint{0.961611in}{2.134423in}}%
\pgfpathcurveto{\pgfqpoint{0.967435in}{2.140247in}}{\pgfqpoint{0.970708in}{2.148147in}}{\pgfqpoint{0.970708in}{2.156384in}}%
\pgfpathcurveto{\pgfqpoint{0.970708in}{2.164620in}}{\pgfqpoint{0.967435in}{2.172520in}}{\pgfqpoint{0.961611in}{2.178344in}}%
\pgfpathcurveto{\pgfqpoint{0.955787in}{2.184168in}}{\pgfqpoint{0.947887in}{2.187440in}}{\pgfqpoint{0.939651in}{2.187440in}}%
\pgfpathcurveto{\pgfqpoint{0.931415in}{2.187440in}}{\pgfqpoint{0.923515in}{2.184168in}}{\pgfqpoint{0.917691in}{2.178344in}}%
\pgfpathcurveto{\pgfqpoint{0.911867in}{2.172520in}}{\pgfqpoint{0.908595in}{2.164620in}}{\pgfqpoint{0.908595in}{2.156384in}}%
\pgfpathcurveto{\pgfqpoint{0.908595in}{2.148147in}}{\pgfqpoint{0.911867in}{2.140247in}}{\pgfqpoint{0.917691in}{2.134423in}}%
\pgfpathcurveto{\pgfqpoint{0.923515in}{2.128600in}}{\pgfqpoint{0.931415in}{2.125327in}}{\pgfqpoint{0.939651in}{2.125327in}}%
\pgfpathclose%
\pgfusepath{stroke,fill}%
\end{pgfscope}%
\begin{pgfscope}%
\pgfpathrectangle{\pgfqpoint{0.100000in}{0.212622in}}{\pgfqpoint{3.696000in}{3.696000in}}%
\pgfusepath{clip}%
\pgfsetbuttcap%
\pgfsetroundjoin%
\definecolor{currentfill}{rgb}{0.121569,0.466667,0.705882}%
\pgfsetfillcolor{currentfill}%
\pgfsetfillopacity{0.875953}%
\pgfsetlinewidth{1.003750pt}%
\definecolor{currentstroke}{rgb}{0.121569,0.466667,0.705882}%
\pgfsetstrokecolor{currentstroke}%
\pgfsetstrokeopacity{0.875953}%
\pgfsetdash{}{0pt}%
\pgfpathmoveto{\pgfqpoint{0.939651in}{2.125327in}}%
\pgfpathcurveto{\pgfqpoint{0.947887in}{2.125327in}}{\pgfqpoint{0.955787in}{2.128600in}}{\pgfqpoint{0.961611in}{2.134423in}}%
\pgfpathcurveto{\pgfqpoint{0.967435in}{2.140247in}}{\pgfqpoint{0.970708in}{2.148147in}}{\pgfqpoint{0.970708in}{2.156384in}}%
\pgfpathcurveto{\pgfqpoint{0.970708in}{2.164620in}}{\pgfqpoint{0.967435in}{2.172520in}}{\pgfqpoint{0.961611in}{2.178344in}}%
\pgfpathcurveto{\pgfqpoint{0.955787in}{2.184168in}}{\pgfqpoint{0.947887in}{2.187440in}}{\pgfqpoint{0.939651in}{2.187440in}}%
\pgfpathcurveto{\pgfqpoint{0.931415in}{2.187440in}}{\pgfqpoint{0.923515in}{2.184168in}}{\pgfqpoint{0.917691in}{2.178344in}}%
\pgfpathcurveto{\pgfqpoint{0.911867in}{2.172520in}}{\pgfqpoint{0.908595in}{2.164620in}}{\pgfqpoint{0.908595in}{2.156384in}}%
\pgfpathcurveto{\pgfqpoint{0.908595in}{2.148147in}}{\pgfqpoint{0.911867in}{2.140247in}}{\pgfqpoint{0.917691in}{2.134423in}}%
\pgfpathcurveto{\pgfqpoint{0.923515in}{2.128600in}}{\pgfqpoint{0.931415in}{2.125327in}}{\pgfqpoint{0.939651in}{2.125327in}}%
\pgfpathclose%
\pgfusepath{stroke,fill}%
\end{pgfscope}%
\begin{pgfscope}%
\pgfpathrectangle{\pgfqpoint{0.100000in}{0.212622in}}{\pgfqpoint{3.696000in}{3.696000in}}%
\pgfusepath{clip}%
\pgfsetbuttcap%
\pgfsetroundjoin%
\definecolor{currentfill}{rgb}{0.121569,0.466667,0.705882}%
\pgfsetfillcolor{currentfill}%
\pgfsetfillopacity{0.875953}%
\pgfsetlinewidth{1.003750pt}%
\definecolor{currentstroke}{rgb}{0.121569,0.466667,0.705882}%
\pgfsetstrokecolor{currentstroke}%
\pgfsetstrokeopacity{0.875953}%
\pgfsetdash{}{0pt}%
\pgfpathmoveto{\pgfqpoint{0.939651in}{2.125327in}}%
\pgfpathcurveto{\pgfqpoint{0.947887in}{2.125327in}}{\pgfqpoint{0.955787in}{2.128600in}}{\pgfqpoint{0.961611in}{2.134423in}}%
\pgfpathcurveto{\pgfqpoint{0.967435in}{2.140247in}}{\pgfqpoint{0.970708in}{2.148147in}}{\pgfqpoint{0.970708in}{2.156384in}}%
\pgfpathcurveto{\pgfqpoint{0.970708in}{2.164620in}}{\pgfqpoint{0.967435in}{2.172520in}}{\pgfqpoint{0.961611in}{2.178344in}}%
\pgfpathcurveto{\pgfqpoint{0.955787in}{2.184168in}}{\pgfqpoint{0.947887in}{2.187440in}}{\pgfqpoint{0.939651in}{2.187440in}}%
\pgfpathcurveto{\pgfqpoint{0.931415in}{2.187440in}}{\pgfqpoint{0.923515in}{2.184168in}}{\pgfqpoint{0.917691in}{2.178344in}}%
\pgfpathcurveto{\pgfqpoint{0.911867in}{2.172520in}}{\pgfqpoint{0.908595in}{2.164620in}}{\pgfqpoint{0.908595in}{2.156384in}}%
\pgfpathcurveto{\pgfqpoint{0.908595in}{2.148147in}}{\pgfqpoint{0.911867in}{2.140247in}}{\pgfqpoint{0.917691in}{2.134423in}}%
\pgfpathcurveto{\pgfqpoint{0.923515in}{2.128600in}}{\pgfqpoint{0.931415in}{2.125327in}}{\pgfqpoint{0.939651in}{2.125327in}}%
\pgfpathclose%
\pgfusepath{stroke,fill}%
\end{pgfscope}%
\begin{pgfscope}%
\pgfpathrectangle{\pgfqpoint{0.100000in}{0.212622in}}{\pgfqpoint{3.696000in}{3.696000in}}%
\pgfusepath{clip}%
\pgfsetbuttcap%
\pgfsetroundjoin%
\definecolor{currentfill}{rgb}{0.121569,0.466667,0.705882}%
\pgfsetfillcolor{currentfill}%
\pgfsetfillopacity{0.875953}%
\pgfsetlinewidth{1.003750pt}%
\definecolor{currentstroke}{rgb}{0.121569,0.466667,0.705882}%
\pgfsetstrokecolor{currentstroke}%
\pgfsetstrokeopacity{0.875953}%
\pgfsetdash{}{0pt}%
\pgfpathmoveto{\pgfqpoint{0.939651in}{2.125327in}}%
\pgfpathcurveto{\pgfqpoint{0.947887in}{2.125327in}}{\pgfqpoint{0.955787in}{2.128600in}}{\pgfqpoint{0.961611in}{2.134423in}}%
\pgfpathcurveto{\pgfqpoint{0.967435in}{2.140247in}}{\pgfqpoint{0.970708in}{2.148147in}}{\pgfqpoint{0.970708in}{2.156384in}}%
\pgfpathcurveto{\pgfqpoint{0.970708in}{2.164620in}}{\pgfqpoint{0.967435in}{2.172520in}}{\pgfqpoint{0.961611in}{2.178344in}}%
\pgfpathcurveto{\pgfqpoint{0.955787in}{2.184168in}}{\pgfqpoint{0.947887in}{2.187440in}}{\pgfqpoint{0.939651in}{2.187440in}}%
\pgfpathcurveto{\pgfqpoint{0.931415in}{2.187440in}}{\pgfqpoint{0.923515in}{2.184168in}}{\pgfqpoint{0.917691in}{2.178344in}}%
\pgfpathcurveto{\pgfqpoint{0.911867in}{2.172520in}}{\pgfqpoint{0.908595in}{2.164620in}}{\pgfqpoint{0.908595in}{2.156384in}}%
\pgfpathcurveto{\pgfqpoint{0.908595in}{2.148147in}}{\pgfqpoint{0.911867in}{2.140247in}}{\pgfqpoint{0.917691in}{2.134423in}}%
\pgfpathcurveto{\pgfqpoint{0.923515in}{2.128600in}}{\pgfqpoint{0.931415in}{2.125327in}}{\pgfqpoint{0.939651in}{2.125327in}}%
\pgfpathclose%
\pgfusepath{stroke,fill}%
\end{pgfscope}%
\begin{pgfscope}%
\pgfpathrectangle{\pgfqpoint{0.100000in}{0.212622in}}{\pgfqpoint{3.696000in}{3.696000in}}%
\pgfusepath{clip}%
\pgfsetbuttcap%
\pgfsetroundjoin%
\definecolor{currentfill}{rgb}{0.121569,0.466667,0.705882}%
\pgfsetfillcolor{currentfill}%
\pgfsetfillopacity{0.875953}%
\pgfsetlinewidth{1.003750pt}%
\definecolor{currentstroke}{rgb}{0.121569,0.466667,0.705882}%
\pgfsetstrokecolor{currentstroke}%
\pgfsetstrokeopacity{0.875953}%
\pgfsetdash{}{0pt}%
\pgfpathmoveto{\pgfqpoint{0.939651in}{2.125327in}}%
\pgfpathcurveto{\pgfqpoint{0.947887in}{2.125327in}}{\pgfqpoint{0.955787in}{2.128600in}}{\pgfqpoint{0.961611in}{2.134423in}}%
\pgfpathcurveto{\pgfqpoint{0.967435in}{2.140247in}}{\pgfqpoint{0.970708in}{2.148147in}}{\pgfqpoint{0.970708in}{2.156384in}}%
\pgfpathcurveto{\pgfqpoint{0.970708in}{2.164620in}}{\pgfqpoint{0.967435in}{2.172520in}}{\pgfqpoint{0.961611in}{2.178344in}}%
\pgfpathcurveto{\pgfqpoint{0.955787in}{2.184168in}}{\pgfqpoint{0.947887in}{2.187440in}}{\pgfqpoint{0.939651in}{2.187440in}}%
\pgfpathcurveto{\pgfqpoint{0.931415in}{2.187440in}}{\pgfqpoint{0.923515in}{2.184168in}}{\pgfqpoint{0.917691in}{2.178344in}}%
\pgfpathcurveto{\pgfqpoint{0.911867in}{2.172520in}}{\pgfqpoint{0.908595in}{2.164620in}}{\pgfqpoint{0.908595in}{2.156384in}}%
\pgfpathcurveto{\pgfqpoint{0.908595in}{2.148147in}}{\pgfqpoint{0.911867in}{2.140247in}}{\pgfqpoint{0.917691in}{2.134423in}}%
\pgfpathcurveto{\pgfqpoint{0.923515in}{2.128600in}}{\pgfqpoint{0.931415in}{2.125327in}}{\pgfqpoint{0.939651in}{2.125327in}}%
\pgfpathclose%
\pgfusepath{stroke,fill}%
\end{pgfscope}%
\begin{pgfscope}%
\pgfpathrectangle{\pgfqpoint{0.100000in}{0.212622in}}{\pgfqpoint{3.696000in}{3.696000in}}%
\pgfusepath{clip}%
\pgfsetbuttcap%
\pgfsetroundjoin%
\definecolor{currentfill}{rgb}{0.121569,0.466667,0.705882}%
\pgfsetfillcolor{currentfill}%
\pgfsetfillopacity{0.875953}%
\pgfsetlinewidth{1.003750pt}%
\definecolor{currentstroke}{rgb}{0.121569,0.466667,0.705882}%
\pgfsetstrokecolor{currentstroke}%
\pgfsetstrokeopacity{0.875953}%
\pgfsetdash{}{0pt}%
\pgfpathmoveto{\pgfqpoint{0.939651in}{2.125327in}}%
\pgfpathcurveto{\pgfqpoint{0.947887in}{2.125327in}}{\pgfqpoint{0.955787in}{2.128600in}}{\pgfqpoint{0.961611in}{2.134423in}}%
\pgfpathcurveto{\pgfqpoint{0.967435in}{2.140247in}}{\pgfqpoint{0.970708in}{2.148147in}}{\pgfqpoint{0.970708in}{2.156384in}}%
\pgfpathcurveto{\pgfqpoint{0.970708in}{2.164620in}}{\pgfqpoint{0.967435in}{2.172520in}}{\pgfqpoint{0.961611in}{2.178344in}}%
\pgfpathcurveto{\pgfqpoint{0.955787in}{2.184168in}}{\pgfqpoint{0.947887in}{2.187440in}}{\pgfqpoint{0.939651in}{2.187440in}}%
\pgfpathcurveto{\pgfqpoint{0.931415in}{2.187440in}}{\pgfqpoint{0.923515in}{2.184168in}}{\pgfqpoint{0.917691in}{2.178344in}}%
\pgfpathcurveto{\pgfqpoint{0.911867in}{2.172520in}}{\pgfqpoint{0.908595in}{2.164620in}}{\pgfqpoint{0.908595in}{2.156384in}}%
\pgfpathcurveto{\pgfqpoint{0.908595in}{2.148147in}}{\pgfqpoint{0.911867in}{2.140247in}}{\pgfqpoint{0.917691in}{2.134423in}}%
\pgfpathcurveto{\pgfqpoint{0.923515in}{2.128600in}}{\pgfqpoint{0.931415in}{2.125327in}}{\pgfqpoint{0.939651in}{2.125327in}}%
\pgfpathclose%
\pgfusepath{stroke,fill}%
\end{pgfscope}%
\begin{pgfscope}%
\pgfpathrectangle{\pgfqpoint{0.100000in}{0.212622in}}{\pgfqpoint{3.696000in}{3.696000in}}%
\pgfusepath{clip}%
\pgfsetbuttcap%
\pgfsetroundjoin%
\definecolor{currentfill}{rgb}{0.121569,0.466667,0.705882}%
\pgfsetfillcolor{currentfill}%
\pgfsetfillopacity{0.875953}%
\pgfsetlinewidth{1.003750pt}%
\definecolor{currentstroke}{rgb}{0.121569,0.466667,0.705882}%
\pgfsetstrokecolor{currentstroke}%
\pgfsetstrokeopacity{0.875953}%
\pgfsetdash{}{0pt}%
\pgfpathmoveto{\pgfqpoint{0.939651in}{2.125327in}}%
\pgfpathcurveto{\pgfqpoint{0.947887in}{2.125327in}}{\pgfqpoint{0.955787in}{2.128600in}}{\pgfqpoint{0.961611in}{2.134423in}}%
\pgfpathcurveto{\pgfqpoint{0.967435in}{2.140247in}}{\pgfqpoint{0.970708in}{2.148147in}}{\pgfqpoint{0.970708in}{2.156384in}}%
\pgfpathcurveto{\pgfqpoint{0.970708in}{2.164620in}}{\pgfqpoint{0.967435in}{2.172520in}}{\pgfqpoint{0.961611in}{2.178344in}}%
\pgfpathcurveto{\pgfqpoint{0.955787in}{2.184168in}}{\pgfqpoint{0.947887in}{2.187440in}}{\pgfqpoint{0.939651in}{2.187440in}}%
\pgfpathcurveto{\pgfqpoint{0.931415in}{2.187440in}}{\pgfqpoint{0.923515in}{2.184168in}}{\pgfqpoint{0.917691in}{2.178344in}}%
\pgfpathcurveto{\pgfqpoint{0.911867in}{2.172520in}}{\pgfqpoint{0.908595in}{2.164620in}}{\pgfqpoint{0.908595in}{2.156384in}}%
\pgfpathcurveto{\pgfqpoint{0.908595in}{2.148147in}}{\pgfqpoint{0.911867in}{2.140247in}}{\pgfqpoint{0.917691in}{2.134423in}}%
\pgfpathcurveto{\pgfqpoint{0.923515in}{2.128600in}}{\pgfqpoint{0.931415in}{2.125327in}}{\pgfqpoint{0.939651in}{2.125327in}}%
\pgfpathclose%
\pgfusepath{stroke,fill}%
\end{pgfscope}%
\begin{pgfscope}%
\pgfpathrectangle{\pgfqpoint{0.100000in}{0.212622in}}{\pgfqpoint{3.696000in}{3.696000in}}%
\pgfusepath{clip}%
\pgfsetbuttcap%
\pgfsetroundjoin%
\definecolor{currentfill}{rgb}{0.121569,0.466667,0.705882}%
\pgfsetfillcolor{currentfill}%
\pgfsetfillopacity{0.875953}%
\pgfsetlinewidth{1.003750pt}%
\definecolor{currentstroke}{rgb}{0.121569,0.466667,0.705882}%
\pgfsetstrokecolor{currentstroke}%
\pgfsetstrokeopacity{0.875953}%
\pgfsetdash{}{0pt}%
\pgfpathmoveto{\pgfqpoint{0.939651in}{2.125327in}}%
\pgfpathcurveto{\pgfqpoint{0.947887in}{2.125327in}}{\pgfqpoint{0.955787in}{2.128600in}}{\pgfqpoint{0.961611in}{2.134423in}}%
\pgfpathcurveto{\pgfqpoint{0.967435in}{2.140247in}}{\pgfqpoint{0.970708in}{2.148147in}}{\pgfqpoint{0.970708in}{2.156384in}}%
\pgfpathcurveto{\pgfqpoint{0.970708in}{2.164620in}}{\pgfqpoint{0.967435in}{2.172520in}}{\pgfqpoint{0.961611in}{2.178344in}}%
\pgfpathcurveto{\pgfqpoint{0.955787in}{2.184168in}}{\pgfqpoint{0.947887in}{2.187440in}}{\pgfqpoint{0.939651in}{2.187440in}}%
\pgfpathcurveto{\pgfqpoint{0.931415in}{2.187440in}}{\pgfqpoint{0.923515in}{2.184168in}}{\pgfqpoint{0.917691in}{2.178344in}}%
\pgfpathcurveto{\pgfqpoint{0.911867in}{2.172520in}}{\pgfqpoint{0.908595in}{2.164620in}}{\pgfqpoint{0.908595in}{2.156384in}}%
\pgfpathcurveto{\pgfqpoint{0.908595in}{2.148147in}}{\pgfqpoint{0.911867in}{2.140247in}}{\pgfqpoint{0.917691in}{2.134423in}}%
\pgfpathcurveto{\pgfqpoint{0.923515in}{2.128600in}}{\pgfqpoint{0.931415in}{2.125327in}}{\pgfqpoint{0.939651in}{2.125327in}}%
\pgfpathclose%
\pgfusepath{stroke,fill}%
\end{pgfscope}%
\begin{pgfscope}%
\pgfpathrectangle{\pgfqpoint{0.100000in}{0.212622in}}{\pgfqpoint{3.696000in}{3.696000in}}%
\pgfusepath{clip}%
\pgfsetbuttcap%
\pgfsetroundjoin%
\definecolor{currentfill}{rgb}{0.121569,0.466667,0.705882}%
\pgfsetfillcolor{currentfill}%
\pgfsetfillopacity{0.875953}%
\pgfsetlinewidth{1.003750pt}%
\definecolor{currentstroke}{rgb}{0.121569,0.466667,0.705882}%
\pgfsetstrokecolor{currentstroke}%
\pgfsetstrokeopacity{0.875953}%
\pgfsetdash{}{0pt}%
\pgfpathmoveto{\pgfqpoint{0.939651in}{2.125327in}}%
\pgfpathcurveto{\pgfqpoint{0.947887in}{2.125327in}}{\pgfqpoint{0.955787in}{2.128600in}}{\pgfqpoint{0.961611in}{2.134423in}}%
\pgfpathcurveto{\pgfqpoint{0.967435in}{2.140247in}}{\pgfqpoint{0.970708in}{2.148147in}}{\pgfqpoint{0.970708in}{2.156384in}}%
\pgfpathcurveto{\pgfqpoint{0.970708in}{2.164620in}}{\pgfqpoint{0.967435in}{2.172520in}}{\pgfqpoint{0.961611in}{2.178344in}}%
\pgfpathcurveto{\pgfqpoint{0.955787in}{2.184168in}}{\pgfqpoint{0.947887in}{2.187440in}}{\pgfqpoint{0.939651in}{2.187440in}}%
\pgfpathcurveto{\pgfqpoint{0.931415in}{2.187440in}}{\pgfqpoint{0.923515in}{2.184168in}}{\pgfqpoint{0.917691in}{2.178344in}}%
\pgfpathcurveto{\pgfqpoint{0.911867in}{2.172520in}}{\pgfqpoint{0.908595in}{2.164620in}}{\pgfqpoint{0.908595in}{2.156384in}}%
\pgfpathcurveto{\pgfqpoint{0.908595in}{2.148147in}}{\pgfqpoint{0.911867in}{2.140247in}}{\pgfqpoint{0.917691in}{2.134423in}}%
\pgfpathcurveto{\pgfqpoint{0.923515in}{2.128600in}}{\pgfqpoint{0.931415in}{2.125327in}}{\pgfqpoint{0.939651in}{2.125327in}}%
\pgfpathclose%
\pgfusepath{stroke,fill}%
\end{pgfscope}%
\begin{pgfscope}%
\pgfpathrectangle{\pgfqpoint{0.100000in}{0.212622in}}{\pgfqpoint{3.696000in}{3.696000in}}%
\pgfusepath{clip}%
\pgfsetbuttcap%
\pgfsetroundjoin%
\definecolor{currentfill}{rgb}{0.121569,0.466667,0.705882}%
\pgfsetfillcolor{currentfill}%
\pgfsetfillopacity{0.875953}%
\pgfsetlinewidth{1.003750pt}%
\definecolor{currentstroke}{rgb}{0.121569,0.466667,0.705882}%
\pgfsetstrokecolor{currentstroke}%
\pgfsetstrokeopacity{0.875953}%
\pgfsetdash{}{0pt}%
\pgfpathmoveto{\pgfqpoint{0.939651in}{2.125327in}}%
\pgfpathcurveto{\pgfqpoint{0.947887in}{2.125327in}}{\pgfqpoint{0.955787in}{2.128600in}}{\pgfqpoint{0.961611in}{2.134423in}}%
\pgfpathcurveto{\pgfqpoint{0.967435in}{2.140247in}}{\pgfqpoint{0.970708in}{2.148147in}}{\pgfqpoint{0.970708in}{2.156384in}}%
\pgfpathcurveto{\pgfqpoint{0.970708in}{2.164620in}}{\pgfqpoint{0.967435in}{2.172520in}}{\pgfqpoint{0.961611in}{2.178344in}}%
\pgfpathcurveto{\pgfqpoint{0.955787in}{2.184168in}}{\pgfqpoint{0.947887in}{2.187440in}}{\pgfqpoint{0.939651in}{2.187440in}}%
\pgfpathcurveto{\pgfqpoint{0.931415in}{2.187440in}}{\pgfqpoint{0.923515in}{2.184168in}}{\pgfqpoint{0.917691in}{2.178344in}}%
\pgfpathcurveto{\pgfqpoint{0.911867in}{2.172520in}}{\pgfqpoint{0.908595in}{2.164620in}}{\pgfqpoint{0.908595in}{2.156384in}}%
\pgfpathcurveto{\pgfqpoint{0.908595in}{2.148147in}}{\pgfqpoint{0.911867in}{2.140247in}}{\pgfqpoint{0.917691in}{2.134423in}}%
\pgfpathcurveto{\pgfqpoint{0.923515in}{2.128600in}}{\pgfqpoint{0.931415in}{2.125327in}}{\pgfqpoint{0.939651in}{2.125327in}}%
\pgfpathclose%
\pgfusepath{stroke,fill}%
\end{pgfscope}%
\begin{pgfscope}%
\pgfpathrectangle{\pgfqpoint{0.100000in}{0.212622in}}{\pgfqpoint{3.696000in}{3.696000in}}%
\pgfusepath{clip}%
\pgfsetbuttcap%
\pgfsetroundjoin%
\definecolor{currentfill}{rgb}{0.121569,0.466667,0.705882}%
\pgfsetfillcolor{currentfill}%
\pgfsetfillopacity{0.875953}%
\pgfsetlinewidth{1.003750pt}%
\definecolor{currentstroke}{rgb}{0.121569,0.466667,0.705882}%
\pgfsetstrokecolor{currentstroke}%
\pgfsetstrokeopacity{0.875953}%
\pgfsetdash{}{0pt}%
\pgfpathmoveto{\pgfqpoint{0.939651in}{2.125327in}}%
\pgfpathcurveto{\pgfqpoint{0.947887in}{2.125327in}}{\pgfqpoint{0.955787in}{2.128600in}}{\pgfqpoint{0.961611in}{2.134423in}}%
\pgfpathcurveto{\pgfqpoint{0.967435in}{2.140247in}}{\pgfqpoint{0.970708in}{2.148147in}}{\pgfqpoint{0.970708in}{2.156384in}}%
\pgfpathcurveto{\pgfqpoint{0.970708in}{2.164620in}}{\pgfqpoint{0.967435in}{2.172520in}}{\pgfqpoint{0.961611in}{2.178344in}}%
\pgfpathcurveto{\pgfqpoint{0.955787in}{2.184168in}}{\pgfqpoint{0.947887in}{2.187440in}}{\pgfqpoint{0.939651in}{2.187440in}}%
\pgfpathcurveto{\pgfqpoint{0.931415in}{2.187440in}}{\pgfqpoint{0.923515in}{2.184168in}}{\pgfqpoint{0.917691in}{2.178344in}}%
\pgfpathcurveto{\pgfqpoint{0.911867in}{2.172520in}}{\pgfqpoint{0.908595in}{2.164620in}}{\pgfqpoint{0.908595in}{2.156384in}}%
\pgfpathcurveto{\pgfqpoint{0.908595in}{2.148147in}}{\pgfqpoint{0.911867in}{2.140247in}}{\pgfqpoint{0.917691in}{2.134423in}}%
\pgfpathcurveto{\pgfqpoint{0.923515in}{2.128600in}}{\pgfqpoint{0.931415in}{2.125327in}}{\pgfqpoint{0.939651in}{2.125327in}}%
\pgfpathclose%
\pgfusepath{stroke,fill}%
\end{pgfscope}%
\begin{pgfscope}%
\pgfpathrectangle{\pgfqpoint{0.100000in}{0.212622in}}{\pgfqpoint{3.696000in}{3.696000in}}%
\pgfusepath{clip}%
\pgfsetbuttcap%
\pgfsetroundjoin%
\definecolor{currentfill}{rgb}{0.121569,0.466667,0.705882}%
\pgfsetfillcolor{currentfill}%
\pgfsetfillopacity{0.875953}%
\pgfsetlinewidth{1.003750pt}%
\definecolor{currentstroke}{rgb}{0.121569,0.466667,0.705882}%
\pgfsetstrokecolor{currentstroke}%
\pgfsetstrokeopacity{0.875953}%
\pgfsetdash{}{0pt}%
\pgfpathmoveto{\pgfqpoint{0.939651in}{2.125327in}}%
\pgfpathcurveto{\pgfqpoint{0.947887in}{2.125327in}}{\pgfqpoint{0.955787in}{2.128600in}}{\pgfqpoint{0.961611in}{2.134423in}}%
\pgfpathcurveto{\pgfqpoint{0.967435in}{2.140247in}}{\pgfqpoint{0.970708in}{2.148147in}}{\pgfqpoint{0.970708in}{2.156384in}}%
\pgfpathcurveto{\pgfqpoint{0.970708in}{2.164620in}}{\pgfqpoint{0.967435in}{2.172520in}}{\pgfqpoint{0.961611in}{2.178344in}}%
\pgfpathcurveto{\pgfqpoint{0.955787in}{2.184168in}}{\pgfqpoint{0.947887in}{2.187440in}}{\pgfqpoint{0.939651in}{2.187440in}}%
\pgfpathcurveto{\pgfqpoint{0.931415in}{2.187440in}}{\pgfqpoint{0.923515in}{2.184168in}}{\pgfqpoint{0.917691in}{2.178344in}}%
\pgfpathcurveto{\pgfqpoint{0.911867in}{2.172520in}}{\pgfqpoint{0.908595in}{2.164620in}}{\pgfqpoint{0.908595in}{2.156384in}}%
\pgfpathcurveto{\pgfqpoint{0.908595in}{2.148147in}}{\pgfqpoint{0.911867in}{2.140247in}}{\pgfqpoint{0.917691in}{2.134423in}}%
\pgfpathcurveto{\pgfqpoint{0.923515in}{2.128600in}}{\pgfqpoint{0.931415in}{2.125327in}}{\pgfqpoint{0.939651in}{2.125327in}}%
\pgfpathclose%
\pgfusepath{stroke,fill}%
\end{pgfscope}%
\begin{pgfscope}%
\pgfpathrectangle{\pgfqpoint{0.100000in}{0.212622in}}{\pgfqpoint{3.696000in}{3.696000in}}%
\pgfusepath{clip}%
\pgfsetbuttcap%
\pgfsetroundjoin%
\definecolor{currentfill}{rgb}{0.121569,0.466667,0.705882}%
\pgfsetfillcolor{currentfill}%
\pgfsetfillopacity{0.875953}%
\pgfsetlinewidth{1.003750pt}%
\definecolor{currentstroke}{rgb}{0.121569,0.466667,0.705882}%
\pgfsetstrokecolor{currentstroke}%
\pgfsetstrokeopacity{0.875953}%
\pgfsetdash{}{0pt}%
\pgfpathmoveto{\pgfqpoint{0.939651in}{2.125327in}}%
\pgfpathcurveto{\pgfqpoint{0.947887in}{2.125327in}}{\pgfqpoint{0.955787in}{2.128600in}}{\pgfqpoint{0.961611in}{2.134423in}}%
\pgfpathcurveto{\pgfqpoint{0.967435in}{2.140247in}}{\pgfqpoint{0.970708in}{2.148147in}}{\pgfqpoint{0.970708in}{2.156384in}}%
\pgfpathcurveto{\pgfqpoint{0.970708in}{2.164620in}}{\pgfqpoint{0.967435in}{2.172520in}}{\pgfqpoint{0.961611in}{2.178344in}}%
\pgfpathcurveto{\pgfqpoint{0.955787in}{2.184168in}}{\pgfqpoint{0.947887in}{2.187440in}}{\pgfqpoint{0.939651in}{2.187440in}}%
\pgfpathcurveto{\pgfqpoint{0.931415in}{2.187440in}}{\pgfqpoint{0.923515in}{2.184168in}}{\pgfqpoint{0.917691in}{2.178344in}}%
\pgfpathcurveto{\pgfqpoint{0.911867in}{2.172520in}}{\pgfqpoint{0.908595in}{2.164620in}}{\pgfqpoint{0.908595in}{2.156384in}}%
\pgfpathcurveto{\pgfqpoint{0.908595in}{2.148147in}}{\pgfqpoint{0.911867in}{2.140247in}}{\pgfqpoint{0.917691in}{2.134423in}}%
\pgfpathcurveto{\pgfqpoint{0.923515in}{2.128600in}}{\pgfqpoint{0.931415in}{2.125327in}}{\pgfqpoint{0.939651in}{2.125327in}}%
\pgfpathclose%
\pgfusepath{stroke,fill}%
\end{pgfscope}%
\begin{pgfscope}%
\pgfpathrectangle{\pgfqpoint{0.100000in}{0.212622in}}{\pgfqpoint{3.696000in}{3.696000in}}%
\pgfusepath{clip}%
\pgfsetbuttcap%
\pgfsetroundjoin%
\definecolor{currentfill}{rgb}{0.121569,0.466667,0.705882}%
\pgfsetfillcolor{currentfill}%
\pgfsetfillopacity{0.875953}%
\pgfsetlinewidth{1.003750pt}%
\definecolor{currentstroke}{rgb}{0.121569,0.466667,0.705882}%
\pgfsetstrokecolor{currentstroke}%
\pgfsetstrokeopacity{0.875953}%
\pgfsetdash{}{0pt}%
\pgfpathmoveto{\pgfqpoint{0.939651in}{2.125327in}}%
\pgfpathcurveto{\pgfqpoint{0.947887in}{2.125327in}}{\pgfqpoint{0.955787in}{2.128600in}}{\pgfqpoint{0.961611in}{2.134423in}}%
\pgfpathcurveto{\pgfqpoint{0.967435in}{2.140247in}}{\pgfqpoint{0.970708in}{2.148147in}}{\pgfqpoint{0.970708in}{2.156384in}}%
\pgfpathcurveto{\pgfqpoint{0.970708in}{2.164620in}}{\pgfqpoint{0.967435in}{2.172520in}}{\pgfqpoint{0.961611in}{2.178344in}}%
\pgfpathcurveto{\pgfqpoint{0.955787in}{2.184168in}}{\pgfqpoint{0.947887in}{2.187440in}}{\pgfqpoint{0.939651in}{2.187440in}}%
\pgfpathcurveto{\pgfqpoint{0.931415in}{2.187440in}}{\pgfqpoint{0.923515in}{2.184168in}}{\pgfqpoint{0.917691in}{2.178344in}}%
\pgfpathcurveto{\pgfqpoint{0.911867in}{2.172520in}}{\pgfqpoint{0.908595in}{2.164620in}}{\pgfqpoint{0.908595in}{2.156384in}}%
\pgfpathcurveto{\pgfqpoint{0.908595in}{2.148147in}}{\pgfqpoint{0.911867in}{2.140247in}}{\pgfqpoint{0.917691in}{2.134423in}}%
\pgfpathcurveto{\pgfqpoint{0.923515in}{2.128600in}}{\pgfqpoint{0.931415in}{2.125327in}}{\pgfqpoint{0.939651in}{2.125327in}}%
\pgfpathclose%
\pgfusepath{stroke,fill}%
\end{pgfscope}%
\begin{pgfscope}%
\pgfpathrectangle{\pgfqpoint{0.100000in}{0.212622in}}{\pgfqpoint{3.696000in}{3.696000in}}%
\pgfusepath{clip}%
\pgfsetbuttcap%
\pgfsetroundjoin%
\definecolor{currentfill}{rgb}{0.121569,0.466667,0.705882}%
\pgfsetfillcolor{currentfill}%
\pgfsetfillopacity{0.875953}%
\pgfsetlinewidth{1.003750pt}%
\definecolor{currentstroke}{rgb}{0.121569,0.466667,0.705882}%
\pgfsetstrokecolor{currentstroke}%
\pgfsetstrokeopacity{0.875953}%
\pgfsetdash{}{0pt}%
\pgfpathmoveto{\pgfqpoint{0.939651in}{2.125327in}}%
\pgfpathcurveto{\pgfqpoint{0.947887in}{2.125327in}}{\pgfqpoint{0.955787in}{2.128600in}}{\pgfqpoint{0.961611in}{2.134423in}}%
\pgfpathcurveto{\pgfqpoint{0.967435in}{2.140247in}}{\pgfqpoint{0.970708in}{2.148147in}}{\pgfqpoint{0.970708in}{2.156384in}}%
\pgfpathcurveto{\pgfqpoint{0.970708in}{2.164620in}}{\pgfqpoint{0.967435in}{2.172520in}}{\pgfqpoint{0.961611in}{2.178344in}}%
\pgfpathcurveto{\pgfqpoint{0.955787in}{2.184168in}}{\pgfqpoint{0.947887in}{2.187440in}}{\pgfqpoint{0.939651in}{2.187440in}}%
\pgfpathcurveto{\pgfqpoint{0.931415in}{2.187440in}}{\pgfqpoint{0.923515in}{2.184168in}}{\pgfqpoint{0.917691in}{2.178344in}}%
\pgfpathcurveto{\pgfqpoint{0.911867in}{2.172520in}}{\pgfqpoint{0.908595in}{2.164620in}}{\pgfqpoint{0.908595in}{2.156384in}}%
\pgfpathcurveto{\pgfqpoint{0.908595in}{2.148147in}}{\pgfqpoint{0.911867in}{2.140247in}}{\pgfqpoint{0.917691in}{2.134423in}}%
\pgfpathcurveto{\pgfqpoint{0.923515in}{2.128600in}}{\pgfqpoint{0.931415in}{2.125327in}}{\pgfqpoint{0.939651in}{2.125327in}}%
\pgfpathclose%
\pgfusepath{stroke,fill}%
\end{pgfscope}%
\begin{pgfscope}%
\pgfpathrectangle{\pgfqpoint{0.100000in}{0.212622in}}{\pgfqpoint{3.696000in}{3.696000in}}%
\pgfusepath{clip}%
\pgfsetbuttcap%
\pgfsetroundjoin%
\definecolor{currentfill}{rgb}{0.121569,0.466667,0.705882}%
\pgfsetfillcolor{currentfill}%
\pgfsetfillopacity{0.875953}%
\pgfsetlinewidth{1.003750pt}%
\definecolor{currentstroke}{rgb}{0.121569,0.466667,0.705882}%
\pgfsetstrokecolor{currentstroke}%
\pgfsetstrokeopacity{0.875953}%
\pgfsetdash{}{0pt}%
\pgfpathmoveto{\pgfqpoint{0.939651in}{2.125327in}}%
\pgfpathcurveto{\pgfqpoint{0.947887in}{2.125327in}}{\pgfqpoint{0.955787in}{2.128600in}}{\pgfqpoint{0.961611in}{2.134423in}}%
\pgfpathcurveto{\pgfqpoint{0.967435in}{2.140247in}}{\pgfqpoint{0.970708in}{2.148147in}}{\pgfqpoint{0.970708in}{2.156384in}}%
\pgfpathcurveto{\pgfqpoint{0.970708in}{2.164620in}}{\pgfqpoint{0.967435in}{2.172520in}}{\pgfqpoint{0.961611in}{2.178344in}}%
\pgfpathcurveto{\pgfqpoint{0.955787in}{2.184168in}}{\pgfqpoint{0.947887in}{2.187440in}}{\pgfqpoint{0.939651in}{2.187440in}}%
\pgfpathcurveto{\pgfqpoint{0.931415in}{2.187440in}}{\pgfqpoint{0.923515in}{2.184168in}}{\pgfqpoint{0.917691in}{2.178344in}}%
\pgfpathcurveto{\pgfqpoint{0.911867in}{2.172520in}}{\pgfqpoint{0.908595in}{2.164620in}}{\pgfqpoint{0.908595in}{2.156384in}}%
\pgfpathcurveto{\pgfqpoint{0.908595in}{2.148147in}}{\pgfqpoint{0.911867in}{2.140247in}}{\pgfqpoint{0.917691in}{2.134423in}}%
\pgfpathcurveto{\pgfqpoint{0.923515in}{2.128600in}}{\pgfqpoint{0.931415in}{2.125327in}}{\pgfqpoint{0.939651in}{2.125327in}}%
\pgfpathclose%
\pgfusepath{stroke,fill}%
\end{pgfscope}%
\begin{pgfscope}%
\pgfpathrectangle{\pgfqpoint{0.100000in}{0.212622in}}{\pgfqpoint{3.696000in}{3.696000in}}%
\pgfusepath{clip}%
\pgfsetbuttcap%
\pgfsetroundjoin%
\definecolor{currentfill}{rgb}{0.121569,0.466667,0.705882}%
\pgfsetfillcolor{currentfill}%
\pgfsetfillopacity{0.875953}%
\pgfsetlinewidth{1.003750pt}%
\definecolor{currentstroke}{rgb}{0.121569,0.466667,0.705882}%
\pgfsetstrokecolor{currentstroke}%
\pgfsetstrokeopacity{0.875953}%
\pgfsetdash{}{0pt}%
\pgfpathmoveto{\pgfqpoint{0.939651in}{2.125327in}}%
\pgfpathcurveto{\pgfqpoint{0.947887in}{2.125327in}}{\pgfqpoint{0.955787in}{2.128600in}}{\pgfqpoint{0.961611in}{2.134423in}}%
\pgfpathcurveto{\pgfqpoint{0.967435in}{2.140247in}}{\pgfqpoint{0.970708in}{2.148147in}}{\pgfqpoint{0.970708in}{2.156384in}}%
\pgfpathcurveto{\pgfqpoint{0.970708in}{2.164620in}}{\pgfqpoint{0.967435in}{2.172520in}}{\pgfqpoint{0.961611in}{2.178344in}}%
\pgfpathcurveto{\pgfqpoint{0.955787in}{2.184168in}}{\pgfqpoint{0.947887in}{2.187440in}}{\pgfqpoint{0.939651in}{2.187440in}}%
\pgfpathcurveto{\pgfqpoint{0.931415in}{2.187440in}}{\pgfqpoint{0.923515in}{2.184168in}}{\pgfqpoint{0.917691in}{2.178344in}}%
\pgfpathcurveto{\pgfqpoint{0.911867in}{2.172520in}}{\pgfqpoint{0.908595in}{2.164620in}}{\pgfqpoint{0.908595in}{2.156384in}}%
\pgfpathcurveto{\pgfqpoint{0.908595in}{2.148147in}}{\pgfqpoint{0.911867in}{2.140247in}}{\pgfqpoint{0.917691in}{2.134423in}}%
\pgfpathcurveto{\pgfqpoint{0.923515in}{2.128600in}}{\pgfqpoint{0.931415in}{2.125327in}}{\pgfqpoint{0.939651in}{2.125327in}}%
\pgfpathclose%
\pgfusepath{stroke,fill}%
\end{pgfscope}%
\begin{pgfscope}%
\pgfpathrectangle{\pgfqpoint{0.100000in}{0.212622in}}{\pgfqpoint{3.696000in}{3.696000in}}%
\pgfusepath{clip}%
\pgfsetbuttcap%
\pgfsetroundjoin%
\definecolor{currentfill}{rgb}{0.121569,0.466667,0.705882}%
\pgfsetfillcolor{currentfill}%
\pgfsetfillopacity{0.875953}%
\pgfsetlinewidth{1.003750pt}%
\definecolor{currentstroke}{rgb}{0.121569,0.466667,0.705882}%
\pgfsetstrokecolor{currentstroke}%
\pgfsetstrokeopacity{0.875953}%
\pgfsetdash{}{0pt}%
\pgfpathmoveto{\pgfqpoint{0.939651in}{2.125327in}}%
\pgfpathcurveto{\pgfqpoint{0.947887in}{2.125327in}}{\pgfqpoint{0.955787in}{2.128600in}}{\pgfqpoint{0.961611in}{2.134423in}}%
\pgfpathcurveto{\pgfqpoint{0.967435in}{2.140247in}}{\pgfqpoint{0.970708in}{2.148147in}}{\pgfqpoint{0.970708in}{2.156384in}}%
\pgfpathcurveto{\pgfqpoint{0.970708in}{2.164620in}}{\pgfqpoint{0.967435in}{2.172520in}}{\pgfqpoint{0.961611in}{2.178344in}}%
\pgfpathcurveto{\pgfqpoint{0.955787in}{2.184168in}}{\pgfqpoint{0.947887in}{2.187440in}}{\pgfqpoint{0.939651in}{2.187440in}}%
\pgfpathcurveto{\pgfqpoint{0.931415in}{2.187440in}}{\pgfqpoint{0.923515in}{2.184168in}}{\pgfqpoint{0.917691in}{2.178344in}}%
\pgfpathcurveto{\pgfqpoint{0.911867in}{2.172520in}}{\pgfqpoint{0.908595in}{2.164620in}}{\pgfqpoint{0.908595in}{2.156384in}}%
\pgfpathcurveto{\pgfqpoint{0.908595in}{2.148147in}}{\pgfqpoint{0.911867in}{2.140247in}}{\pgfqpoint{0.917691in}{2.134423in}}%
\pgfpathcurveto{\pgfqpoint{0.923515in}{2.128600in}}{\pgfqpoint{0.931415in}{2.125327in}}{\pgfqpoint{0.939651in}{2.125327in}}%
\pgfpathclose%
\pgfusepath{stroke,fill}%
\end{pgfscope}%
\begin{pgfscope}%
\pgfpathrectangle{\pgfqpoint{0.100000in}{0.212622in}}{\pgfqpoint{3.696000in}{3.696000in}}%
\pgfusepath{clip}%
\pgfsetbuttcap%
\pgfsetroundjoin%
\definecolor{currentfill}{rgb}{0.121569,0.466667,0.705882}%
\pgfsetfillcolor{currentfill}%
\pgfsetfillopacity{0.875953}%
\pgfsetlinewidth{1.003750pt}%
\definecolor{currentstroke}{rgb}{0.121569,0.466667,0.705882}%
\pgfsetstrokecolor{currentstroke}%
\pgfsetstrokeopacity{0.875953}%
\pgfsetdash{}{0pt}%
\pgfpathmoveto{\pgfqpoint{0.939651in}{2.125327in}}%
\pgfpathcurveto{\pgfqpoint{0.947887in}{2.125327in}}{\pgfqpoint{0.955787in}{2.128600in}}{\pgfqpoint{0.961611in}{2.134423in}}%
\pgfpathcurveto{\pgfqpoint{0.967435in}{2.140247in}}{\pgfqpoint{0.970708in}{2.148147in}}{\pgfqpoint{0.970708in}{2.156384in}}%
\pgfpathcurveto{\pgfqpoint{0.970708in}{2.164620in}}{\pgfqpoint{0.967435in}{2.172520in}}{\pgfqpoint{0.961611in}{2.178344in}}%
\pgfpathcurveto{\pgfqpoint{0.955787in}{2.184168in}}{\pgfqpoint{0.947887in}{2.187440in}}{\pgfqpoint{0.939651in}{2.187440in}}%
\pgfpathcurveto{\pgfqpoint{0.931415in}{2.187440in}}{\pgfqpoint{0.923515in}{2.184168in}}{\pgfqpoint{0.917691in}{2.178344in}}%
\pgfpathcurveto{\pgfqpoint{0.911867in}{2.172520in}}{\pgfqpoint{0.908595in}{2.164620in}}{\pgfqpoint{0.908595in}{2.156384in}}%
\pgfpathcurveto{\pgfqpoint{0.908595in}{2.148147in}}{\pgfqpoint{0.911867in}{2.140247in}}{\pgfqpoint{0.917691in}{2.134423in}}%
\pgfpathcurveto{\pgfqpoint{0.923515in}{2.128600in}}{\pgfqpoint{0.931415in}{2.125327in}}{\pgfqpoint{0.939651in}{2.125327in}}%
\pgfpathclose%
\pgfusepath{stroke,fill}%
\end{pgfscope}%
\begin{pgfscope}%
\pgfpathrectangle{\pgfqpoint{0.100000in}{0.212622in}}{\pgfqpoint{3.696000in}{3.696000in}}%
\pgfusepath{clip}%
\pgfsetbuttcap%
\pgfsetroundjoin%
\definecolor{currentfill}{rgb}{0.121569,0.466667,0.705882}%
\pgfsetfillcolor{currentfill}%
\pgfsetfillopacity{0.876011}%
\pgfsetlinewidth{1.003750pt}%
\definecolor{currentstroke}{rgb}{0.121569,0.466667,0.705882}%
\pgfsetstrokecolor{currentstroke}%
\pgfsetstrokeopacity{0.876011}%
\pgfsetdash{}{0pt}%
\pgfpathmoveto{\pgfqpoint{2.244212in}{2.547533in}}%
\pgfpathcurveto{\pgfqpoint{2.252449in}{2.547533in}}{\pgfqpoint{2.260349in}{2.550806in}}{\pgfqpoint{2.266173in}{2.556630in}}%
\pgfpathcurveto{\pgfqpoint{2.271996in}{2.562454in}}{\pgfqpoint{2.275269in}{2.570354in}}{\pgfqpoint{2.275269in}{2.578590in}}%
\pgfpathcurveto{\pgfqpoint{2.275269in}{2.586826in}}{\pgfqpoint{2.271996in}{2.594726in}}{\pgfqpoint{2.266173in}{2.600550in}}%
\pgfpathcurveto{\pgfqpoint{2.260349in}{2.606374in}}{\pgfqpoint{2.252449in}{2.609646in}}{\pgfqpoint{2.244212in}{2.609646in}}%
\pgfpathcurveto{\pgfqpoint{2.235976in}{2.609646in}}{\pgfqpoint{2.228076in}{2.606374in}}{\pgfqpoint{2.222252in}{2.600550in}}%
\pgfpathcurveto{\pgfqpoint{2.216428in}{2.594726in}}{\pgfqpoint{2.213156in}{2.586826in}}{\pgfqpoint{2.213156in}{2.578590in}}%
\pgfpathcurveto{\pgfqpoint{2.213156in}{2.570354in}}{\pgfqpoint{2.216428in}{2.562454in}}{\pgfqpoint{2.222252in}{2.556630in}}%
\pgfpathcurveto{\pgfqpoint{2.228076in}{2.550806in}}{\pgfqpoint{2.235976in}{2.547533in}}{\pgfqpoint{2.244212in}{2.547533in}}%
\pgfpathclose%
\pgfusepath{stroke,fill}%
\end{pgfscope}%
\begin{pgfscope}%
\pgfpathrectangle{\pgfqpoint{0.100000in}{0.212622in}}{\pgfqpoint{3.696000in}{3.696000in}}%
\pgfusepath{clip}%
\pgfsetbuttcap%
\pgfsetroundjoin%
\definecolor{currentfill}{rgb}{0.121569,0.466667,0.705882}%
\pgfsetfillcolor{currentfill}%
\pgfsetfillopacity{0.876083}%
\pgfsetlinewidth{1.003750pt}%
\definecolor{currentstroke}{rgb}{0.121569,0.466667,0.705882}%
\pgfsetstrokecolor{currentstroke}%
\pgfsetstrokeopacity{0.876083}%
\pgfsetdash{}{0pt}%
\pgfpathmoveto{\pgfqpoint{0.939432in}{2.125089in}}%
\pgfpathcurveto{\pgfqpoint{0.947668in}{2.125089in}}{\pgfqpoint{0.955568in}{2.128362in}}{\pgfqpoint{0.961392in}{2.134186in}}%
\pgfpathcurveto{\pgfqpoint{0.967216in}{2.140010in}}{\pgfqpoint{0.970488in}{2.147910in}}{\pgfqpoint{0.970488in}{2.156146in}}%
\pgfpathcurveto{\pgfqpoint{0.970488in}{2.164382in}}{\pgfqpoint{0.967216in}{2.172282in}}{\pgfqpoint{0.961392in}{2.178106in}}%
\pgfpathcurveto{\pgfqpoint{0.955568in}{2.183930in}}{\pgfqpoint{0.947668in}{2.187202in}}{\pgfqpoint{0.939432in}{2.187202in}}%
\pgfpathcurveto{\pgfqpoint{0.931196in}{2.187202in}}{\pgfqpoint{0.923296in}{2.183930in}}{\pgfqpoint{0.917472in}{2.178106in}}%
\pgfpathcurveto{\pgfqpoint{0.911648in}{2.172282in}}{\pgfqpoint{0.908375in}{2.164382in}}{\pgfqpoint{0.908375in}{2.156146in}}%
\pgfpathcurveto{\pgfqpoint{0.908375in}{2.147910in}}{\pgfqpoint{0.911648in}{2.140010in}}{\pgfqpoint{0.917472in}{2.134186in}}%
\pgfpathcurveto{\pgfqpoint{0.923296in}{2.128362in}}{\pgfqpoint{0.931196in}{2.125089in}}{\pgfqpoint{0.939432in}{2.125089in}}%
\pgfpathclose%
\pgfusepath{stroke,fill}%
\end{pgfscope}%
\begin{pgfscope}%
\pgfpathrectangle{\pgfqpoint{0.100000in}{0.212622in}}{\pgfqpoint{3.696000in}{3.696000in}}%
\pgfusepath{clip}%
\pgfsetbuttcap%
\pgfsetroundjoin%
\definecolor{currentfill}{rgb}{0.121569,0.466667,0.705882}%
\pgfsetfillcolor{currentfill}%
\pgfsetfillopacity{0.876154}%
\pgfsetlinewidth{1.003750pt}%
\definecolor{currentstroke}{rgb}{0.121569,0.466667,0.705882}%
\pgfsetstrokecolor{currentstroke}%
\pgfsetstrokeopacity{0.876154}%
\pgfsetdash{}{0pt}%
\pgfpathmoveto{\pgfqpoint{0.939311in}{2.124961in}}%
\pgfpathcurveto{\pgfqpoint{0.947547in}{2.124961in}}{\pgfqpoint{0.955447in}{2.128233in}}{\pgfqpoint{0.961271in}{2.134057in}}%
\pgfpathcurveto{\pgfqpoint{0.967095in}{2.139881in}}{\pgfqpoint{0.970367in}{2.147781in}}{\pgfqpoint{0.970367in}{2.156017in}}%
\pgfpathcurveto{\pgfqpoint{0.970367in}{2.164254in}}{\pgfqpoint{0.967095in}{2.172154in}}{\pgfqpoint{0.961271in}{2.177978in}}%
\pgfpathcurveto{\pgfqpoint{0.955447in}{2.183802in}}{\pgfqpoint{0.947547in}{2.187074in}}{\pgfqpoint{0.939311in}{2.187074in}}%
\pgfpathcurveto{\pgfqpoint{0.931074in}{2.187074in}}{\pgfqpoint{0.923174in}{2.183802in}}{\pgfqpoint{0.917350in}{2.177978in}}%
\pgfpathcurveto{\pgfqpoint{0.911527in}{2.172154in}}{\pgfqpoint{0.908254in}{2.164254in}}{\pgfqpoint{0.908254in}{2.156017in}}%
\pgfpathcurveto{\pgfqpoint{0.908254in}{2.147781in}}{\pgfqpoint{0.911527in}{2.139881in}}{\pgfqpoint{0.917350in}{2.134057in}}%
\pgfpathcurveto{\pgfqpoint{0.923174in}{2.128233in}}{\pgfqpoint{0.931074in}{2.124961in}}{\pgfqpoint{0.939311in}{2.124961in}}%
\pgfpathclose%
\pgfusepath{stroke,fill}%
\end{pgfscope}%
\begin{pgfscope}%
\pgfpathrectangle{\pgfqpoint{0.100000in}{0.212622in}}{\pgfqpoint{3.696000in}{3.696000in}}%
\pgfusepath{clip}%
\pgfsetbuttcap%
\pgfsetroundjoin%
\definecolor{currentfill}{rgb}{0.121569,0.466667,0.705882}%
\pgfsetfillcolor{currentfill}%
\pgfsetfillopacity{0.876353}%
\pgfsetlinewidth{1.003750pt}%
\definecolor{currentstroke}{rgb}{0.121569,0.466667,0.705882}%
\pgfsetstrokecolor{currentstroke}%
\pgfsetstrokeopacity{0.876353}%
\pgfsetdash{}{0pt}%
\pgfpathmoveto{\pgfqpoint{0.939001in}{2.124599in}}%
\pgfpathcurveto{\pgfqpoint{0.947237in}{2.124599in}}{\pgfqpoint{0.955137in}{2.127872in}}{\pgfqpoint{0.960961in}{2.133696in}}%
\pgfpathcurveto{\pgfqpoint{0.966785in}{2.139520in}}{\pgfqpoint{0.970057in}{2.147420in}}{\pgfqpoint{0.970057in}{2.155656in}}%
\pgfpathcurveto{\pgfqpoint{0.970057in}{2.163892in}}{\pgfqpoint{0.966785in}{2.171792in}}{\pgfqpoint{0.960961in}{2.177616in}}%
\pgfpathcurveto{\pgfqpoint{0.955137in}{2.183440in}}{\pgfqpoint{0.947237in}{2.186712in}}{\pgfqpoint{0.939001in}{2.186712in}}%
\pgfpathcurveto{\pgfqpoint{0.930764in}{2.186712in}}{\pgfqpoint{0.922864in}{2.183440in}}{\pgfqpoint{0.917041in}{2.177616in}}%
\pgfpathcurveto{\pgfqpoint{0.911217in}{2.171792in}}{\pgfqpoint{0.907944in}{2.163892in}}{\pgfqpoint{0.907944in}{2.155656in}}%
\pgfpathcurveto{\pgfqpoint{0.907944in}{2.147420in}}{\pgfqpoint{0.911217in}{2.139520in}}{\pgfqpoint{0.917041in}{2.133696in}}%
\pgfpathcurveto{\pgfqpoint{0.922864in}{2.127872in}}{\pgfqpoint{0.930764in}{2.124599in}}{\pgfqpoint{0.939001in}{2.124599in}}%
\pgfpathclose%
\pgfusepath{stroke,fill}%
\end{pgfscope}%
\begin{pgfscope}%
\pgfpathrectangle{\pgfqpoint{0.100000in}{0.212622in}}{\pgfqpoint{3.696000in}{3.696000in}}%
\pgfusepath{clip}%
\pgfsetbuttcap%
\pgfsetroundjoin%
\definecolor{currentfill}{rgb}{0.121569,0.466667,0.705882}%
\pgfsetfillcolor{currentfill}%
\pgfsetfillopacity{0.876462}%
\pgfsetlinewidth{1.003750pt}%
\definecolor{currentstroke}{rgb}{0.121569,0.466667,0.705882}%
\pgfsetstrokecolor{currentstroke}%
\pgfsetstrokeopacity{0.876462}%
\pgfsetdash{}{0pt}%
\pgfpathmoveto{\pgfqpoint{0.938845in}{2.124400in}}%
\pgfpathcurveto{\pgfqpoint{0.947082in}{2.124400in}}{\pgfqpoint{0.954982in}{2.127673in}}{\pgfqpoint{0.960806in}{2.133496in}}%
\pgfpathcurveto{\pgfqpoint{0.966629in}{2.139320in}}{\pgfqpoint{0.969902in}{2.147220in}}{\pgfqpoint{0.969902in}{2.155457in}}%
\pgfpathcurveto{\pgfqpoint{0.969902in}{2.163693in}}{\pgfqpoint{0.966629in}{2.171593in}}{\pgfqpoint{0.960806in}{2.177417in}}%
\pgfpathcurveto{\pgfqpoint{0.954982in}{2.183241in}}{\pgfqpoint{0.947082in}{2.186513in}}{\pgfqpoint{0.938845in}{2.186513in}}%
\pgfpathcurveto{\pgfqpoint{0.930609in}{2.186513in}}{\pgfqpoint{0.922709in}{2.183241in}}{\pgfqpoint{0.916885in}{2.177417in}}%
\pgfpathcurveto{\pgfqpoint{0.911061in}{2.171593in}}{\pgfqpoint{0.907789in}{2.163693in}}{\pgfqpoint{0.907789in}{2.155457in}}%
\pgfpathcurveto{\pgfqpoint{0.907789in}{2.147220in}}{\pgfqpoint{0.911061in}{2.139320in}}{\pgfqpoint{0.916885in}{2.133496in}}%
\pgfpathcurveto{\pgfqpoint{0.922709in}{2.127673in}}{\pgfqpoint{0.930609in}{2.124400in}}{\pgfqpoint{0.938845in}{2.124400in}}%
\pgfpathclose%
\pgfusepath{stroke,fill}%
\end{pgfscope}%
\begin{pgfscope}%
\pgfpathrectangle{\pgfqpoint{0.100000in}{0.212622in}}{\pgfqpoint{3.696000in}{3.696000in}}%
\pgfusepath{clip}%
\pgfsetbuttcap%
\pgfsetroundjoin%
\definecolor{currentfill}{rgb}{0.121569,0.466667,0.705882}%
\pgfsetfillcolor{currentfill}%
\pgfsetfillopacity{0.876523}%
\pgfsetlinewidth{1.003750pt}%
\definecolor{currentstroke}{rgb}{0.121569,0.466667,0.705882}%
\pgfsetstrokecolor{currentstroke}%
\pgfsetstrokeopacity{0.876523}%
\pgfsetdash{}{0pt}%
\pgfpathmoveto{\pgfqpoint{0.938767in}{2.124293in}}%
\pgfpathcurveto{\pgfqpoint{0.947003in}{2.124293in}}{\pgfqpoint{0.954903in}{2.127565in}}{\pgfqpoint{0.960727in}{2.133389in}}%
\pgfpathcurveto{\pgfqpoint{0.966551in}{2.139213in}}{\pgfqpoint{0.969823in}{2.147113in}}{\pgfqpoint{0.969823in}{2.155349in}}%
\pgfpathcurveto{\pgfqpoint{0.969823in}{2.163585in}}{\pgfqpoint{0.966551in}{2.171485in}}{\pgfqpoint{0.960727in}{2.177309in}}%
\pgfpathcurveto{\pgfqpoint{0.954903in}{2.183133in}}{\pgfqpoint{0.947003in}{2.186406in}}{\pgfqpoint{0.938767in}{2.186406in}}%
\pgfpathcurveto{\pgfqpoint{0.930531in}{2.186406in}}{\pgfqpoint{0.922631in}{2.183133in}}{\pgfqpoint{0.916807in}{2.177309in}}%
\pgfpathcurveto{\pgfqpoint{0.910983in}{2.171485in}}{\pgfqpoint{0.907710in}{2.163585in}}{\pgfqpoint{0.907710in}{2.155349in}}%
\pgfpathcurveto{\pgfqpoint{0.907710in}{2.147113in}}{\pgfqpoint{0.910983in}{2.139213in}}{\pgfqpoint{0.916807in}{2.133389in}}%
\pgfpathcurveto{\pgfqpoint{0.922631in}{2.127565in}}{\pgfqpoint{0.930531in}{2.124293in}}{\pgfqpoint{0.938767in}{2.124293in}}%
\pgfpathclose%
\pgfusepath{stroke,fill}%
\end{pgfscope}%
\begin{pgfscope}%
\pgfpathrectangle{\pgfqpoint{0.100000in}{0.212622in}}{\pgfqpoint{3.696000in}{3.696000in}}%
\pgfusepath{clip}%
\pgfsetbuttcap%
\pgfsetroundjoin%
\definecolor{currentfill}{rgb}{0.121569,0.466667,0.705882}%
\pgfsetfillcolor{currentfill}%
\pgfsetfillopacity{0.876560}%
\pgfsetlinewidth{1.003750pt}%
\definecolor{currentstroke}{rgb}{0.121569,0.466667,0.705882}%
\pgfsetstrokecolor{currentstroke}%
\pgfsetstrokeopacity{0.876560}%
\pgfsetdash{}{0pt}%
\pgfpathmoveto{\pgfqpoint{2.242971in}{2.546385in}}%
\pgfpathcurveto{\pgfqpoint{2.251207in}{2.546385in}}{\pgfqpoint{2.259107in}{2.549658in}}{\pgfqpoint{2.264931in}{2.555481in}}%
\pgfpathcurveto{\pgfqpoint{2.270755in}{2.561305in}}{\pgfqpoint{2.274027in}{2.569205in}}{\pgfqpoint{2.274027in}{2.577442in}}%
\pgfpathcurveto{\pgfqpoint{2.274027in}{2.585678in}}{\pgfqpoint{2.270755in}{2.593578in}}{\pgfqpoint{2.264931in}{2.599402in}}%
\pgfpathcurveto{\pgfqpoint{2.259107in}{2.605226in}}{\pgfqpoint{2.251207in}{2.608498in}}{\pgfqpoint{2.242971in}{2.608498in}}%
\pgfpathcurveto{\pgfqpoint{2.234735in}{2.608498in}}{\pgfqpoint{2.226835in}{2.605226in}}{\pgfqpoint{2.221011in}{2.599402in}}%
\pgfpathcurveto{\pgfqpoint{2.215187in}{2.593578in}}{\pgfqpoint{2.211914in}{2.585678in}}{\pgfqpoint{2.211914in}{2.577442in}}%
\pgfpathcurveto{\pgfqpoint{2.211914in}{2.569205in}}{\pgfqpoint{2.215187in}{2.561305in}}{\pgfqpoint{2.221011in}{2.555481in}}%
\pgfpathcurveto{\pgfqpoint{2.226835in}{2.549658in}}{\pgfqpoint{2.234735in}{2.546385in}}{\pgfqpoint{2.242971in}{2.546385in}}%
\pgfpathclose%
\pgfusepath{stroke,fill}%
\end{pgfscope}%
\begin{pgfscope}%
\pgfpathrectangle{\pgfqpoint{0.100000in}{0.212622in}}{\pgfqpoint{3.696000in}{3.696000in}}%
\pgfusepath{clip}%
\pgfsetbuttcap%
\pgfsetroundjoin%
\definecolor{currentfill}{rgb}{0.121569,0.466667,0.705882}%
\pgfsetfillcolor{currentfill}%
\pgfsetfillopacity{0.876756}%
\pgfsetlinewidth{1.003750pt}%
\definecolor{currentstroke}{rgb}{0.121569,0.466667,0.705882}%
\pgfsetstrokecolor{currentstroke}%
\pgfsetstrokeopacity{0.876756}%
\pgfsetdash{}{0pt}%
\pgfpathmoveto{\pgfqpoint{0.938478in}{2.123883in}}%
\pgfpathcurveto{\pgfqpoint{0.946714in}{2.123883in}}{\pgfqpoint{0.954614in}{2.127155in}}{\pgfqpoint{0.960438in}{2.132979in}}%
\pgfpathcurveto{\pgfqpoint{0.966262in}{2.138803in}}{\pgfqpoint{0.969534in}{2.146703in}}{\pgfqpoint{0.969534in}{2.154939in}}%
\pgfpathcurveto{\pgfqpoint{0.969534in}{2.163175in}}{\pgfqpoint{0.966262in}{2.171075in}}{\pgfqpoint{0.960438in}{2.176899in}}%
\pgfpathcurveto{\pgfqpoint{0.954614in}{2.182723in}}{\pgfqpoint{0.946714in}{2.185996in}}{\pgfqpoint{0.938478in}{2.185996in}}%
\pgfpathcurveto{\pgfqpoint{0.930241in}{2.185996in}}{\pgfqpoint{0.922341in}{2.182723in}}{\pgfqpoint{0.916517in}{2.176899in}}%
\pgfpathcurveto{\pgfqpoint{0.910694in}{2.171075in}}{\pgfqpoint{0.907421in}{2.163175in}}{\pgfqpoint{0.907421in}{2.154939in}}%
\pgfpathcurveto{\pgfqpoint{0.907421in}{2.146703in}}{\pgfqpoint{0.910694in}{2.138803in}}{\pgfqpoint{0.916517in}{2.132979in}}%
\pgfpathcurveto{\pgfqpoint{0.922341in}{2.127155in}}{\pgfqpoint{0.930241in}{2.123883in}}{\pgfqpoint{0.938478in}{2.123883in}}%
\pgfpathclose%
\pgfusepath{stroke,fill}%
\end{pgfscope}%
\begin{pgfscope}%
\pgfpathrectangle{\pgfqpoint{0.100000in}{0.212622in}}{\pgfqpoint{3.696000in}{3.696000in}}%
\pgfusepath{clip}%
\pgfsetbuttcap%
\pgfsetroundjoin%
\definecolor{currentfill}{rgb}{0.121569,0.466667,0.705882}%
\pgfsetfillcolor{currentfill}%
\pgfsetfillopacity{0.876836}%
\pgfsetlinewidth{1.003750pt}%
\definecolor{currentstroke}{rgb}{0.121569,0.466667,0.705882}%
\pgfsetstrokecolor{currentstroke}%
\pgfsetstrokeopacity{0.876836}%
\pgfsetdash{}{0pt}%
\pgfpathmoveto{\pgfqpoint{1.804073in}{2.675300in}}%
\pgfpathcurveto{\pgfqpoint{1.812309in}{2.675300in}}{\pgfqpoint{1.820209in}{2.678572in}}{\pgfqpoint{1.826033in}{2.684396in}}%
\pgfpathcurveto{\pgfqpoint{1.831857in}{2.690220in}}{\pgfqpoint{1.835129in}{2.698120in}}{\pgfqpoint{1.835129in}{2.706356in}}%
\pgfpathcurveto{\pgfqpoint{1.835129in}{2.714593in}}{\pgfqpoint{1.831857in}{2.722493in}}{\pgfqpoint{1.826033in}{2.728317in}}%
\pgfpathcurveto{\pgfqpoint{1.820209in}{2.734141in}}{\pgfqpoint{1.812309in}{2.737413in}}{\pgfqpoint{1.804073in}{2.737413in}}%
\pgfpathcurveto{\pgfqpoint{1.795837in}{2.737413in}}{\pgfqpoint{1.787937in}{2.734141in}}{\pgfqpoint{1.782113in}{2.728317in}}%
\pgfpathcurveto{\pgfqpoint{1.776289in}{2.722493in}}{\pgfqpoint{1.773016in}{2.714593in}}{\pgfqpoint{1.773016in}{2.706356in}}%
\pgfpathcurveto{\pgfqpoint{1.773016in}{2.698120in}}{\pgfqpoint{1.776289in}{2.690220in}}{\pgfqpoint{1.782113in}{2.684396in}}%
\pgfpathcurveto{\pgfqpoint{1.787937in}{2.678572in}}{\pgfqpoint{1.795837in}{2.675300in}}{\pgfqpoint{1.804073in}{2.675300in}}%
\pgfpathclose%
\pgfusepath{stroke,fill}%
\end{pgfscope}%
\begin{pgfscope}%
\pgfpathrectangle{\pgfqpoint{0.100000in}{0.212622in}}{\pgfqpoint{3.696000in}{3.696000in}}%
\pgfusepath{clip}%
\pgfsetbuttcap%
\pgfsetroundjoin%
\definecolor{currentfill}{rgb}{0.121569,0.466667,0.705882}%
\pgfsetfillcolor{currentfill}%
\pgfsetfillopacity{0.877259}%
\pgfsetlinewidth{1.003750pt}%
\definecolor{currentstroke}{rgb}{0.121569,0.466667,0.705882}%
\pgfsetstrokecolor{currentstroke}%
\pgfsetstrokeopacity{0.877259}%
\pgfsetdash{}{0pt}%
\pgfpathmoveto{\pgfqpoint{0.937948in}{2.122997in}}%
\pgfpathcurveto{\pgfqpoint{0.946184in}{2.122997in}}{\pgfqpoint{0.954084in}{2.126269in}}{\pgfqpoint{0.959908in}{2.132093in}}%
\pgfpathcurveto{\pgfqpoint{0.965732in}{2.137917in}}{\pgfqpoint{0.969004in}{2.145817in}}{\pgfqpoint{0.969004in}{2.154053in}}%
\pgfpathcurveto{\pgfqpoint{0.969004in}{2.162290in}}{\pgfqpoint{0.965732in}{2.170190in}}{\pgfqpoint{0.959908in}{2.176014in}}%
\pgfpathcurveto{\pgfqpoint{0.954084in}{2.181838in}}{\pgfqpoint{0.946184in}{2.185110in}}{\pgfqpoint{0.937948in}{2.185110in}}%
\pgfpathcurveto{\pgfqpoint{0.929711in}{2.185110in}}{\pgfqpoint{0.921811in}{2.181838in}}{\pgfqpoint{0.915988in}{2.176014in}}%
\pgfpathcurveto{\pgfqpoint{0.910164in}{2.170190in}}{\pgfqpoint{0.906891in}{2.162290in}}{\pgfqpoint{0.906891in}{2.154053in}}%
\pgfpathcurveto{\pgfqpoint{0.906891in}{2.145817in}}{\pgfqpoint{0.910164in}{2.137917in}}{\pgfqpoint{0.915988in}{2.132093in}}%
\pgfpathcurveto{\pgfqpoint{0.921811in}{2.126269in}}{\pgfqpoint{0.929711in}{2.122997in}}{\pgfqpoint{0.937948in}{2.122997in}}%
\pgfpathclose%
\pgfusepath{stroke,fill}%
\end{pgfscope}%
\begin{pgfscope}%
\pgfpathrectangle{\pgfqpoint{0.100000in}{0.212622in}}{\pgfqpoint{3.696000in}{3.696000in}}%
\pgfusepath{clip}%
\pgfsetbuttcap%
\pgfsetroundjoin%
\definecolor{currentfill}{rgb}{0.121569,0.466667,0.705882}%
\pgfsetfillcolor{currentfill}%
\pgfsetfillopacity{0.877505}%
\pgfsetlinewidth{1.003750pt}%
\definecolor{currentstroke}{rgb}{0.121569,0.466667,0.705882}%
\pgfsetstrokecolor{currentstroke}%
\pgfsetstrokeopacity{0.877505}%
\pgfsetdash{}{0pt}%
\pgfpathmoveto{\pgfqpoint{2.240746in}{2.543947in}}%
\pgfpathcurveto{\pgfqpoint{2.248982in}{2.543947in}}{\pgfqpoint{2.256882in}{2.547220in}}{\pgfqpoint{2.262706in}{2.553044in}}%
\pgfpathcurveto{\pgfqpoint{2.268530in}{2.558867in}}{\pgfqpoint{2.271802in}{2.566767in}}{\pgfqpoint{2.271802in}{2.575004in}}%
\pgfpathcurveto{\pgfqpoint{2.271802in}{2.583240in}}{\pgfqpoint{2.268530in}{2.591140in}}{\pgfqpoint{2.262706in}{2.596964in}}%
\pgfpathcurveto{\pgfqpoint{2.256882in}{2.602788in}}{\pgfqpoint{2.248982in}{2.606060in}}{\pgfqpoint{2.240746in}{2.606060in}}%
\pgfpathcurveto{\pgfqpoint{2.232509in}{2.606060in}}{\pgfqpoint{2.224609in}{2.602788in}}{\pgfqpoint{2.218785in}{2.596964in}}%
\pgfpathcurveto{\pgfqpoint{2.212962in}{2.591140in}}{\pgfqpoint{2.209689in}{2.583240in}}{\pgfqpoint{2.209689in}{2.575004in}}%
\pgfpathcurveto{\pgfqpoint{2.209689in}{2.566767in}}{\pgfqpoint{2.212962in}{2.558867in}}{\pgfqpoint{2.218785in}{2.553044in}}%
\pgfpathcurveto{\pgfqpoint{2.224609in}{2.547220in}}{\pgfqpoint{2.232509in}{2.543947in}}{\pgfqpoint{2.240746in}{2.543947in}}%
\pgfpathclose%
\pgfusepath{stroke,fill}%
\end{pgfscope}%
\begin{pgfscope}%
\pgfpathrectangle{\pgfqpoint{0.100000in}{0.212622in}}{\pgfqpoint{3.696000in}{3.696000in}}%
\pgfusepath{clip}%
\pgfsetbuttcap%
\pgfsetroundjoin%
\definecolor{currentfill}{rgb}{0.121569,0.466667,0.705882}%
\pgfsetfillcolor{currentfill}%
\pgfsetfillopacity{0.877893}%
\pgfsetlinewidth{1.003750pt}%
\definecolor{currentstroke}{rgb}{0.121569,0.466667,0.705882}%
\pgfsetstrokecolor{currentstroke}%
\pgfsetstrokeopacity{0.877893}%
\pgfsetdash{}{0pt}%
\pgfpathmoveto{\pgfqpoint{0.937397in}{2.121844in}}%
\pgfpathcurveto{\pgfqpoint{0.945633in}{2.121844in}}{\pgfqpoint{0.953533in}{2.125116in}}{\pgfqpoint{0.959357in}{2.130940in}}%
\pgfpathcurveto{\pgfqpoint{0.965181in}{2.136764in}}{\pgfqpoint{0.968453in}{2.144664in}}{\pgfqpoint{0.968453in}{2.152900in}}%
\pgfpathcurveto{\pgfqpoint{0.968453in}{2.161137in}}{\pgfqpoint{0.965181in}{2.169037in}}{\pgfqpoint{0.959357in}{2.174861in}}%
\pgfpathcurveto{\pgfqpoint{0.953533in}{2.180684in}}{\pgfqpoint{0.945633in}{2.183957in}}{\pgfqpoint{0.937397in}{2.183957in}}%
\pgfpathcurveto{\pgfqpoint{0.929160in}{2.183957in}}{\pgfqpoint{0.921260in}{2.180684in}}{\pgfqpoint{0.915436in}{2.174861in}}%
\pgfpathcurveto{\pgfqpoint{0.909612in}{2.169037in}}{\pgfqpoint{0.906340in}{2.161137in}}{\pgfqpoint{0.906340in}{2.152900in}}%
\pgfpathcurveto{\pgfqpoint{0.906340in}{2.144664in}}{\pgfqpoint{0.909612in}{2.136764in}}{\pgfqpoint{0.915436in}{2.130940in}}%
\pgfpathcurveto{\pgfqpoint{0.921260in}{2.125116in}}{\pgfqpoint{0.929160in}{2.121844in}}{\pgfqpoint{0.937397in}{2.121844in}}%
\pgfpathclose%
\pgfusepath{stroke,fill}%
\end{pgfscope}%
\begin{pgfscope}%
\pgfpathrectangle{\pgfqpoint{0.100000in}{0.212622in}}{\pgfqpoint{3.696000in}{3.696000in}}%
\pgfusepath{clip}%
\pgfsetbuttcap%
\pgfsetroundjoin%
\definecolor{currentfill}{rgb}{0.121569,0.466667,0.705882}%
\pgfsetfillcolor{currentfill}%
\pgfsetfillopacity{0.878461}%
\pgfsetlinewidth{1.003750pt}%
\definecolor{currentstroke}{rgb}{0.121569,0.466667,0.705882}%
\pgfsetstrokecolor{currentstroke}%
\pgfsetstrokeopacity{0.878461}%
\pgfsetdash{}{0pt}%
\pgfpathmoveto{\pgfqpoint{1.801082in}{2.671267in}}%
\pgfpathcurveto{\pgfqpoint{1.809318in}{2.671267in}}{\pgfqpoint{1.817218in}{2.674539in}}{\pgfqpoint{1.823042in}{2.680363in}}%
\pgfpathcurveto{\pgfqpoint{1.828866in}{2.686187in}}{\pgfqpoint{1.832138in}{2.694087in}}{\pgfqpoint{1.832138in}{2.702324in}}%
\pgfpathcurveto{\pgfqpoint{1.832138in}{2.710560in}}{\pgfqpoint{1.828866in}{2.718460in}}{\pgfqpoint{1.823042in}{2.724284in}}%
\pgfpathcurveto{\pgfqpoint{1.817218in}{2.730108in}}{\pgfqpoint{1.809318in}{2.733380in}}{\pgfqpoint{1.801082in}{2.733380in}}%
\pgfpathcurveto{\pgfqpoint{1.792846in}{2.733380in}}{\pgfqpoint{1.784946in}{2.730108in}}{\pgfqpoint{1.779122in}{2.724284in}}%
\pgfpathcurveto{\pgfqpoint{1.773298in}{2.718460in}}{\pgfqpoint{1.770025in}{2.710560in}}{\pgfqpoint{1.770025in}{2.702324in}}%
\pgfpathcurveto{\pgfqpoint{1.770025in}{2.694087in}}{\pgfqpoint{1.773298in}{2.686187in}}{\pgfqpoint{1.779122in}{2.680363in}}%
\pgfpathcurveto{\pgfqpoint{1.784946in}{2.674539in}}{\pgfqpoint{1.792846in}{2.671267in}}{\pgfqpoint{1.801082in}{2.671267in}}%
\pgfpathclose%
\pgfusepath{stroke,fill}%
\end{pgfscope}%
\begin{pgfscope}%
\pgfpathrectangle{\pgfqpoint{0.100000in}{0.212622in}}{\pgfqpoint{3.696000in}{3.696000in}}%
\pgfusepath{clip}%
\pgfsetbuttcap%
\pgfsetroundjoin%
\definecolor{currentfill}{rgb}{0.121569,0.466667,0.705882}%
\pgfsetfillcolor{currentfill}%
\pgfsetfillopacity{0.878675}%
\pgfsetlinewidth{1.003750pt}%
\definecolor{currentstroke}{rgb}{0.121569,0.466667,0.705882}%
\pgfsetstrokecolor{currentstroke}%
\pgfsetstrokeopacity{0.878675}%
\pgfsetdash{}{0pt}%
\pgfpathmoveto{\pgfqpoint{0.936821in}{2.120456in}}%
\pgfpathcurveto{\pgfqpoint{0.945058in}{2.120456in}}{\pgfqpoint{0.952958in}{2.123729in}}{\pgfqpoint{0.958782in}{2.129553in}}%
\pgfpathcurveto{\pgfqpoint{0.964605in}{2.135376in}}{\pgfqpoint{0.967878in}{2.143277in}}{\pgfqpoint{0.967878in}{2.151513in}}%
\pgfpathcurveto{\pgfqpoint{0.967878in}{2.159749in}}{\pgfqpoint{0.964605in}{2.167649in}}{\pgfqpoint{0.958782in}{2.173473in}}%
\pgfpathcurveto{\pgfqpoint{0.952958in}{2.179297in}}{\pgfqpoint{0.945058in}{2.182569in}}{\pgfqpoint{0.936821in}{2.182569in}}%
\pgfpathcurveto{\pgfqpoint{0.928585in}{2.182569in}}{\pgfqpoint{0.920685in}{2.179297in}}{\pgfqpoint{0.914861in}{2.173473in}}%
\pgfpathcurveto{\pgfqpoint{0.909037in}{2.167649in}}{\pgfqpoint{0.905765in}{2.159749in}}{\pgfqpoint{0.905765in}{2.151513in}}%
\pgfpathcurveto{\pgfqpoint{0.905765in}{2.143277in}}{\pgfqpoint{0.909037in}{2.135376in}}{\pgfqpoint{0.914861in}{2.129553in}}%
\pgfpathcurveto{\pgfqpoint{0.920685in}{2.123729in}}{\pgfqpoint{0.928585in}{2.120456in}}{\pgfqpoint{0.936821in}{2.120456in}}%
\pgfpathclose%
\pgfusepath{stroke,fill}%
\end{pgfscope}%
\begin{pgfscope}%
\pgfpathrectangle{\pgfqpoint{0.100000in}{0.212622in}}{\pgfqpoint{3.696000in}{3.696000in}}%
\pgfusepath{clip}%
\pgfsetbuttcap%
\pgfsetroundjoin%
\definecolor{currentfill}{rgb}{0.121569,0.466667,0.705882}%
\pgfsetfillcolor{currentfill}%
\pgfsetfillopacity{0.879203}%
\pgfsetlinewidth{1.003750pt}%
\definecolor{currentstroke}{rgb}{0.121569,0.466667,0.705882}%
\pgfsetstrokecolor{currentstroke}%
\pgfsetstrokeopacity{0.879203}%
\pgfsetdash{}{0pt}%
\pgfpathmoveto{\pgfqpoint{2.236596in}{2.539509in}}%
\pgfpathcurveto{\pgfqpoint{2.244832in}{2.539509in}}{\pgfqpoint{2.252732in}{2.542782in}}{\pgfqpoint{2.258556in}{2.548606in}}%
\pgfpathcurveto{\pgfqpoint{2.264380in}{2.554430in}}{\pgfqpoint{2.267653in}{2.562330in}}{\pgfqpoint{2.267653in}{2.570566in}}%
\pgfpathcurveto{\pgfqpoint{2.267653in}{2.578802in}}{\pgfqpoint{2.264380in}{2.586702in}}{\pgfqpoint{2.258556in}{2.592526in}}%
\pgfpathcurveto{\pgfqpoint{2.252732in}{2.598350in}}{\pgfqpoint{2.244832in}{2.601622in}}{\pgfqpoint{2.236596in}{2.601622in}}%
\pgfpathcurveto{\pgfqpoint{2.228360in}{2.601622in}}{\pgfqpoint{2.220460in}{2.598350in}}{\pgfqpoint{2.214636in}{2.592526in}}%
\pgfpathcurveto{\pgfqpoint{2.208812in}{2.586702in}}{\pgfqpoint{2.205540in}{2.578802in}}{\pgfqpoint{2.205540in}{2.570566in}}%
\pgfpathcurveto{\pgfqpoint{2.205540in}{2.562330in}}{\pgfqpoint{2.208812in}{2.554430in}}{\pgfqpoint{2.214636in}{2.548606in}}%
\pgfpathcurveto{\pgfqpoint{2.220460in}{2.542782in}}{\pgfqpoint{2.228360in}{2.539509in}}{\pgfqpoint{2.236596in}{2.539509in}}%
\pgfpathclose%
\pgfusepath{stroke,fill}%
\end{pgfscope}%
\begin{pgfscope}%
\pgfpathrectangle{\pgfqpoint{0.100000in}{0.212622in}}{\pgfqpoint{3.696000in}{3.696000in}}%
\pgfusepath{clip}%
\pgfsetbuttcap%
\pgfsetroundjoin%
\definecolor{currentfill}{rgb}{0.121569,0.466667,0.705882}%
\pgfsetfillcolor{currentfill}%
\pgfsetfillopacity{0.879567}%
\pgfsetlinewidth{1.003750pt}%
\definecolor{currentstroke}{rgb}{0.121569,0.466667,0.705882}%
\pgfsetstrokecolor{currentstroke}%
\pgfsetstrokeopacity{0.879567}%
\pgfsetdash{}{0pt}%
\pgfpathmoveto{\pgfqpoint{0.936331in}{2.118802in}}%
\pgfpathcurveto{\pgfqpoint{0.944568in}{2.118802in}}{\pgfqpoint{0.952468in}{2.122074in}}{\pgfqpoint{0.958292in}{2.127898in}}%
\pgfpathcurveto{\pgfqpoint{0.964116in}{2.133722in}}{\pgfqpoint{0.967388in}{2.141622in}}{\pgfqpoint{0.967388in}{2.149859in}}%
\pgfpathcurveto{\pgfqpoint{0.967388in}{2.158095in}}{\pgfqpoint{0.964116in}{2.165995in}}{\pgfqpoint{0.958292in}{2.171819in}}%
\pgfpathcurveto{\pgfqpoint{0.952468in}{2.177643in}}{\pgfqpoint{0.944568in}{2.180915in}}{\pgfqpoint{0.936331in}{2.180915in}}%
\pgfpathcurveto{\pgfqpoint{0.928095in}{2.180915in}}{\pgfqpoint{0.920195in}{2.177643in}}{\pgfqpoint{0.914371in}{2.171819in}}%
\pgfpathcurveto{\pgfqpoint{0.908547in}{2.165995in}}{\pgfqpoint{0.905275in}{2.158095in}}{\pgfqpoint{0.905275in}{2.149859in}}%
\pgfpathcurveto{\pgfqpoint{0.905275in}{2.141622in}}{\pgfqpoint{0.908547in}{2.133722in}}{\pgfqpoint{0.914371in}{2.127898in}}%
\pgfpathcurveto{\pgfqpoint{0.920195in}{2.122074in}}{\pgfqpoint{0.928095in}{2.118802in}}{\pgfqpoint{0.936331in}{2.118802in}}%
\pgfpathclose%
\pgfusepath{stroke,fill}%
\end{pgfscope}%
\begin{pgfscope}%
\pgfpathrectangle{\pgfqpoint{0.100000in}{0.212622in}}{\pgfqpoint{3.696000in}{3.696000in}}%
\pgfusepath{clip}%
\pgfsetbuttcap%
\pgfsetroundjoin%
\definecolor{currentfill}{rgb}{0.121569,0.466667,0.705882}%
\pgfsetfillcolor{currentfill}%
\pgfsetfillopacity{0.880154}%
\pgfsetlinewidth{1.003750pt}%
\definecolor{currentstroke}{rgb}{0.121569,0.466667,0.705882}%
\pgfsetstrokecolor{currentstroke}%
\pgfsetstrokeopacity{0.880154}%
\pgfsetdash{}{0pt}%
\pgfpathmoveto{\pgfqpoint{1.797825in}{2.666720in}}%
\pgfpathcurveto{\pgfqpoint{1.806062in}{2.666720in}}{\pgfqpoint{1.813962in}{2.669992in}}{\pgfqpoint{1.819786in}{2.675816in}}%
\pgfpathcurveto{\pgfqpoint{1.825610in}{2.681640in}}{\pgfqpoint{1.828882in}{2.689540in}}{\pgfqpoint{1.828882in}{2.697777in}}%
\pgfpathcurveto{\pgfqpoint{1.828882in}{2.706013in}}{\pgfqpoint{1.825610in}{2.713913in}}{\pgfqpoint{1.819786in}{2.719737in}}%
\pgfpathcurveto{\pgfqpoint{1.813962in}{2.725561in}}{\pgfqpoint{1.806062in}{2.728833in}}{\pgfqpoint{1.797825in}{2.728833in}}%
\pgfpathcurveto{\pgfqpoint{1.789589in}{2.728833in}}{\pgfqpoint{1.781689in}{2.725561in}}{\pgfqpoint{1.775865in}{2.719737in}}%
\pgfpathcurveto{\pgfqpoint{1.770041in}{2.713913in}}{\pgfqpoint{1.766769in}{2.706013in}}{\pgfqpoint{1.766769in}{2.697777in}}%
\pgfpathcurveto{\pgfqpoint{1.766769in}{2.689540in}}{\pgfqpoint{1.770041in}{2.681640in}}{\pgfqpoint{1.775865in}{2.675816in}}%
\pgfpathcurveto{\pgfqpoint{1.781689in}{2.669992in}}{\pgfqpoint{1.789589in}{2.666720in}}{\pgfqpoint{1.797825in}{2.666720in}}%
\pgfpathclose%
\pgfusepath{stroke,fill}%
\end{pgfscope}%
\begin{pgfscope}%
\pgfpathrectangle{\pgfqpoint{0.100000in}{0.212622in}}{\pgfqpoint{3.696000in}{3.696000in}}%
\pgfusepath{clip}%
\pgfsetbuttcap%
\pgfsetroundjoin%
\definecolor{currentfill}{rgb}{0.121569,0.466667,0.705882}%
\pgfsetfillcolor{currentfill}%
\pgfsetfillopacity{0.880760}%
\pgfsetlinewidth{1.003750pt}%
\definecolor{currentstroke}{rgb}{0.121569,0.466667,0.705882}%
\pgfsetstrokecolor{currentstroke}%
\pgfsetstrokeopacity{0.880760}%
\pgfsetdash{}{0pt}%
\pgfpathmoveto{\pgfqpoint{2.232806in}{2.536006in}}%
\pgfpathcurveto{\pgfqpoint{2.241042in}{2.536006in}}{\pgfqpoint{2.248942in}{2.539278in}}{\pgfqpoint{2.254766in}{2.545102in}}%
\pgfpathcurveto{\pgfqpoint{2.260590in}{2.550926in}}{\pgfqpoint{2.263863in}{2.558826in}}{\pgfqpoint{2.263863in}{2.567062in}}%
\pgfpathcurveto{\pgfqpoint{2.263863in}{2.575299in}}{\pgfqpoint{2.260590in}{2.583199in}}{\pgfqpoint{2.254766in}{2.589023in}}%
\pgfpathcurveto{\pgfqpoint{2.248942in}{2.594847in}}{\pgfqpoint{2.241042in}{2.598119in}}{\pgfqpoint{2.232806in}{2.598119in}}%
\pgfpathcurveto{\pgfqpoint{2.224570in}{2.598119in}}{\pgfqpoint{2.216670in}{2.594847in}}{\pgfqpoint{2.210846in}{2.589023in}}%
\pgfpathcurveto{\pgfqpoint{2.205022in}{2.583199in}}{\pgfqpoint{2.201750in}{2.575299in}}{\pgfqpoint{2.201750in}{2.567062in}}%
\pgfpathcurveto{\pgfqpoint{2.201750in}{2.558826in}}{\pgfqpoint{2.205022in}{2.550926in}}{\pgfqpoint{2.210846in}{2.545102in}}%
\pgfpathcurveto{\pgfqpoint{2.216670in}{2.539278in}}{\pgfqpoint{2.224570in}{2.536006in}}{\pgfqpoint{2.232806in}{2.536006in}}%
\pgfpathclose%
\pgfusepath{stroke,fill}%
\end{pgfscope}%
\begin{pgfscope}%
\pgfpathrectangle{\pgfqpoint{0.100000in}{0.212622in}}{\pgfqpoint{3.696000in}{3.696000in}}%
\pgfusepath{clip}%
\pgfsetbuttcap%
\pgfsetroundjoin%
\definecolor{currentfill}{rgb}{0.121569,0.466667,0.705882}%
\pgfsetfillcolor{currentfill}%
\pgfsetfillopacity{0.881017}%
\pgfsetlinewidth{1.003750pt}%
\definecolor{currentstroke}{rgb}{0.121569,0.466667,0.705882}%
\pgfsetstrokecolor{currentstroke}%
\pgfsetstrokeopacity{0.881017}%
\pgfsetdash{}{0pt}%
\pgfpathmoveto{\pgfqpoint{0.935816in}{2.116191in}}%
\pgfpathcurveto{\pgfqpoint{0.944052in}{2.116191in}}{\pgfqpoint{0.951952in}{2.119463in}}{\pgfqpoint{0.957776in}{2.125287in}}%
\pgfpathcurveto{\pgfqpoint{0.963600in}{2.131111in}}{\pgfqpoint{0.966873in}{2.139011in}}{\pgfqpoint{0.966873in}{2.147247in}}%
\pgfpathcurveto{\pgfqpoint{0.966873in}{2.155484in}}{\pgfqpoint{0.963600in}{2.163384in}}{\pgfqpoint{0.957776in}{2.169208in}}%
\pgfpathcurveto{\pgfqpoint{0.951952in}{2.175032in}}{\pgfqpoint{0.944052in}{2.178304in}}{\pgfqpoint{0.935816in}{2.178304in}}%
\pgfpathcurveto{\pgfqpoint{0.927580in}{2.178304in}}{\pgfqpoint{0.919680in}{2.175032in}}{\pgfqpoint{0.913856in}{2.169208in}}%
\pgfpathcurveto{\pgfqpoint{0.908032in}{2.163384in}}{\pgfqpoint{0.904760in}{2.155484in}}{\pgfqpoint{0.904760in}{2.147247in}}%
\pgfpathcurveto{\pgfqpoint{0.904760in}{2.139011in}}{\pgfqpoint{0.908032in}{2.131111in}}{\pgfqpoint{0.913856in}{2.125287in}}%
\pgfpathcurveto{\pgfqpoint{0.919680in}{2.119463in}}{\pgfqpoint{0.927580in}{2.116191in}}{\pgfqpoint{0.935816in}{2.116191in}}%
\pgfpathclose%
\pgfusepath{stroke,fill}%
\end{pgfscope}%
\begin{pgfscope}%
\pgfpathrectangle{\pgfqpoint{0.100000in}{0.212622in}}{\pgfqpoint{3.696000in}{3.696000in}}%
\pgfusepath{clip}%
\pgfsetbuttcap%
\pgfsetroundjoin%
\definecolor{currentfill}{rgb}{0.121569,0.466667,0.705882}%
\pgfsetfillcolor{currentfill}%
\pgfsetfillopacity{0.881094}%
\pgfsetlinewidth{1.003750pt}%
\definecolor{currentstroke}{rgb}{0.121569,0.466667,0.705882}%
\pgfsetstrokecolor{currentstroke}%
\pgfsetstrokeopacity{0.881094}%
\pgfsetdash{}{0pt}%
\pgfpathmoveto{\pgfqpoint{2.715736in}{1.349580in}}%
\pgfpathcurveto{\pgfqpoint{2.723972in}{1.349580in}}{\pgfqpoint{2.731872in}{1.352852in}}{\pgfqpoint{2.737696in}{1.358676in}}%
\pgfpathcurveto{\pgfqpoint{2.743520in}{1.364500in}}{\pgfqpoint{2.746792in}{1.372400in}}{\pgfqpoint{2.746792in}{1.380636in}}%
\pgfpathcurveto{\pgfqpoint{2.746792in}{1.388873in}}{\pgfqpoint{2.743520in}{1.396773in}}{\pgfqpoint{2.737696in}{1.402597in}}%
\pgfpathcurveto{\pgfqpoint{2.731872in}{1.408421in}}{\pgfqpoint{2.723972in}{1.411693in}}{\pgfqpoint{2.715736in}{1.411693in}}%
\pgfpathcurveto{\pgfqpoint{2.707499in}{1.411693in}}{\pgfqpoint{2.699599in}{1.408421in}}{\pgfqpoint{2.693775in}{1.402597in}}%
\pgfpathcurveto{\pgfqpoint{2.687951in}{1.396773in}}{\pgfqpoint{2.684679in}{1.388873in}}{\pgfqpoint{2.684679in}{1.380636in}}%
\pgfpathcurveto{\pgfqpoint{2.684679in}{1.372400in}}{\pgfqpoint{2.687951in}{1.364500in}}{\pgfqpoint{2.693775in}{1.358676in}}%
\pgfpathcurveto{\pgfqpoint{2.699599in}{1.352852in}}{\pgfqpoint{2.707499in}{1.349580in}}{\pgfqpoint{2.715736in}{1.349580in}}%
\pgfpathclose%
\pgfusepath{stroke,fill}%
\end{pgfscope}%
\begin{pgfscope}%
\pgfpathrectangle{\pgfqpoint{0.100000in}{0.212622in}}{\pgfqpoint{3.696000in}{3.696000in}}%
\pgfusepath{clip}%
\pgfsetbuttcap%
\pgfsetroundjoin%
\definecolor{currentfill}{rgb}{0.121569,0.466667,0.705882}%
\pgfsetfillcolor{currentfill}%
\pgfsetfillopacity{0.882078}%
\pgfsetlinewidth{1.003750pt}%
\definecolor{currentstroke}{rgb}{0.121569,0.466667,0.705882}%
\pgfsetstrokecolor{currentstroke}%
\pgfsetstrokeopacity{0.882078}%
\pgfsetdash{}{0pt}%
\pgfpathmoveto{\pgfqpoint{1.794436in}{2.662170in}}%
\pgfpathcurveto{\pgfqpoint{1.802672in}{2.662170in}}{\pgfqpoint{1.810572in}{2.665442in}}{\pgfqpoint{1.816396in}{2.671266in}}%
\pgfpathcurveto{\pgfqpoint{1.822220in}{2.677090in}}{\pgfqpoint{1.825492in}{2.684990in}}{\pgfqpoint{1.825492in}{2.693226in}}%
\pgfpathcurveto{\pgfqpoint{1.825492in}{2.701462in}}{\pgfqpoint{1.822220in}{2.709363in}}{\pgfqpoint{1.816396in}{2.715186in}}%
\pgfpathcurveto{\pgfqpoint{1.810572in}{2.721010in}}{\pgfqpoint{1.802672in}{2.724283in}}{\pgfqpoint{1.794436in}{2.724283in}}%
\pgfpathcurveto{\pgfqpoint{1.786200in}{2.724283in}}{\pgfqpoint{1.778300in}{2.721010in}}{\pgfqpoint{1.772476in}{2.715186in}}%
\pgfpathcurveto{\pgfqpoint{1.766652in}{2.709363in}}{\pgfqpoint{1.763379in}{2.701462in}}{\pgfqpoint{1.763379in}{2.693226in}}%
\pgfpathcurveto{\pgfqpoint{1.763379in}{2.684990in}}{\pgfqpoint{1.766652in}{2.677090in}}{\pgfqpoint{1.772476in}{2.671266in}}%
\pgfpathcurveto{\pgfqpoint{1.778300in}{2.665442in}}{\pgfqpoint{1.786200in}{2.662170in}}{\pgfqpoint{1.794436in}{2.662170in}}%
\pgfpathclose%
\pgfusepath{stroke,fill}%
\end{pgfscope}%
\begin{pgfscope}%
\pgfpathrectangle{\pgfqpoint{0.100000in}{0.212622in}}{\pgfqpoint{3.696000in}{3.696000in}}%
\pgfusepath{clip}%
\pgfsetbuttcap%
\pgfsetroundjoin%
\definecolor{currentfill}{rgb}{0.121569,0.466667,0.705882}%
\pgfsetfillcolor{currentfill}%
\pgfsetfillopacity{0.882211}%
\pgfsetlinewidth{1.003750pt}%
\definecolor{currentstroke}{rgb}{0.121569,0.466667,0.705882}%
\pgfsetstrokecolor{currentstroke}%
\pgfsetstrokeopacity{0.882211}%
\pgfsetdash{}{0pt}%
\pgfpathmoveto{\pgfqpoint{2.229432in}{2.533084in}}%
\pgfpathcurveto{\pgfqpoint{2.237669in}{2.533084in}}{\pgfqpoint{2.245569in}{2.536356in}}{\pgfqpoint{2.251393in}{2.542180in}}%
\pgfpathcurveto{\pgfqpoint{2.257217in}{2.548004in}}{\pgfqpoint{2.260489in}{2.555904in}}{\pgfqpoint{2.260489in}{2.564141in}}%
\pgfpathcurveto{\pgfqpoint{2.260489in}{2.572377in}}{\pgfqpoint{2.257217in}{2.580277in}}{\pgfqpoint{2.251393in}{2.586101in}}%
\pgfpathcurveto{\pgfqpoint{2.245569in}{2.591925in}}{\pgfqpoint{2.237669in}{2.595197in}}{\pgfqpoint{2.229432in}{2.595197in}}%
\pgfpathcurveto{\pgfqpoint{2.221196in}{2.595197in}}{\pgfqpoint{2.213296in}{2.591925in}}{\pgfqpoint{2.207472in}{2.586101in}}%
\pgfpathcurveto{\pgfqpoint{2.201648in}{2.580277in}}{\pgfqpoint{2.198376in}{2.572377in}}{\pgfqpoint{2.198376in}{2.564141in}}%
\pgfpathcurveto{\pgfqpoint{2.198376in}{2.555904in}}{\pgfqpoint{2.201648in}{2.548004in}}{\pgfqpoint{2.207472in}{2.542180in}}%
\pgfpathcurveto{\pgfqpoint{2.213296in}{2.536356in}}{\pgfqpoint{2.221196in}{2.533084in}}{\pgfqpoint{2.229432in}{2.533084in}}%
\pgfpathclose%
\pgfusepath{stroke,fill}%
\end{pgfscope}%
\begin{pgfscope}%
\pgfpathrectangle{\pgfqpoint{0.100000in}{0.212622in}}{\pgfqpoint{3.696000in}{3.696000in}}%
\pgfusepath{clip}%
\pgfsetbuttcap%
\pgfsetroundjoin%
\definecolor{currentfill}{rgb}{0.121569,0.466667,0.705882}%
\pgfsetfillcolor{currentfill}%
\pgfsetfillopacity{0.882588}%
\pgfsetlinewidth{1.003750pt}%
\definecolor{currentstroke}{rgb}{0.121569,0.466667,0.705882}%
\pgfsetstrokecolor{currentstroke}%
\pgfsetstrokeopacity{0.882588}%
\pgfsetdash{}{0pt}%
\pgfpathmoveto{\pgfqpoint{0.935461in}{2.113287in}}%
\pgfpathcurveto{\pgfqpoint{0.943698in}{2.113287in}}{\pgfqpoint{0.951598in}{2.116559in}}{\pgfqpoint{0.957421in}{2.122383in}}%
\pgfpathcurveto{\pgfqpoint{0.963245in}{2.128207in}}{\pgfqpoint{0.966518in}{2.136107in}}{\pgfqpoint{0.966518in}{2.144343in}}%
\pgfpathcurveto{\pgfqpoint{0.966518in}{2.152579in}}{\pgfqpoint{0.963245in}{2.160480in}}{\pgfqpoint{0.957421in}{2.166303in}}%
\pgfpathcurveto{\pgfqpoint{0.951598in}{2.172127in}}{\pgfqpoint{0.943698in}{2.175400in}}{\pgfqpoint{0.935461in}{2.175400in}}%
\pgfpathcurveto{\pgfqpoint{0.927225in}{2.175400in}}{\pgfqpoint{0.919325in}{2.172127in}}{\pgfqpoint{0.913501in}{2.166303in}}%
\pgfpathcurveto{\pgfqpoint{0.907677in}{2.160480in}}{\pgfqpoint{0.904405in}{2.152579in}}{\pgfqpoint{0.904405in}{2.144343in}}%
\pgfpathcurveto{\pgfqpoint{0.904405in}{2.136107in}}{\pgfqpoint{0.907677in}{2.128207in}}{\pgfqpoint{0.913501in}{2.122383in}}%
\pgfpathcurveto{\pgfqpoint{0.919325in}{2.116559in}}{\pgfqpoint{0.927225in}{2.113287in}}{\pgfqpoint{0.935461in}{2.113287in}}%
\pgfpathclose%
\pgfusepath{stroke,fill}%
\end{pgfscope}%
\begin{pgfscope}%
\pgfpathrectangle{\pgfqpoint{0.100000in}{0.212622in}}{\pgfqpoint{3.696000in}{3.696000in}}%
\pgfusepath{clip}%
\pgfsetbuttcap%
\pgfsetroundjoin%
\definecolor{currentfill}{rgb}{0.121569,0.466667,0.705882}%
\pgfsetfillcolor{currentfill}%
\pgfsetfillopacity{0.883443}%
\pgfsetlinewidth{1.003750pt}%
\definecolor{currentstroke}{rgb}{0.121569,0.466667,0.705882}%
\pgfsetstrokecolor{currentstroke}%
\pgfsetstrokeopacity{0.883443}%
\pgfsetdash{}{0pt}%
\pgfpathmoveto{\pgfqpoint{2.226388in}{2.529687in}}%
\pgfpathcurveto{\pgfqpoint{2.234624in}{2.529687in}}{\pgfqpoint{2.242525in}{2.532959in}}{\pgfqpoint{2.248348in}{2.538783in}}%
\pgfpathcurveto{\pgfqpoint{2.254172in}{2.544607in}}{\pgfqpoint{2.257445in}{2.552507in}}{\pgfqpoint{2.257445in}{2.560743in}}%
\pgfpathcurveto{\pgfqpoint{2.257445in}{2.568980in}}{\pgfqpoint{2.254172in}{2.576880in}}{\pgfqpoint{2.248348in}{2.582704in}}%
\pgfpathcurveto{\pgfqpoint{2.242525in}{2.588528in}}{\pgfqpoint{2.234624in}{2.591800in}}{\pgfqpoint{2.226388in}{2.591800in}}%
\pgfpathcurveto{\pgfqpoint{2.218152in}{2.591800in}}{\pgfqpoint{2.210252in}{2.588528in}}{\pgfqpoint{2.204428in}{2.582704in}}%
\pgfpathcurveto{\pgfqpoint{2.198604in}{2.576880in}}{\pgfqpoint{2.195332in}{2.568980in}}{\pgfqpoint{2.195332in}{2.560743in}}%
\pgfpathcurveto{\pgfqpoint{2.195332in}{2.552507in}}{\pgfqpoint{2.198604in}{2.544607in}}{\pgfqpoint{2.204428in}{2.538783in}}%
\pgfpathcurveto{\pgfqpoint{2.210252in}{2.532959in}}{\pgfqpoint{2.218152in}{2.529687in}}{\pgfqpoint{2.226388in}{2.529687in}}%
\pgfpathclose%
\pgfusepath{stroke,fill}%
\end{pgfscope}%
\begin{pgfscope}%
\pgfpathrectangle{\pgfqpoint{0.100000in}{0.212622in}}{\pgfqpoint{3.696000in}{3.696000in}}%
\pgfusepath{clip}%
\pgfsetbuttcap%
\pgfsetroundjoin%
\definecolor{currentfill}{rgb}{0.121569,0.466667,0.705882}%
\pgfsetfillcolor{currentfill}%
\pgfsetfillopacity{0.884129}%
\pgfsetlinewidth{1.003750pt}%
\definecolor{currentstroke}{rgb}{0.121569,0.466667,0.705882}%
\pgfsetstrokecolor{currentstroke}%
\pgfsetstrokeopacity{0.884129}%
\pgfsetdash{}{0pt}%
\pgfpathmoveto{\pgfqpoint{1.790614in}{2.656966in}}%
\pgfpathcurveto{\pgfqpoint{1.798851in}{2.656966in}}{\pgfqpoint{1.806751in}{2.660238in}}{\pgfqpoint{1.812575in}{2.666062in}}%
\pgfpathcurveto{\pgfqpoint{1.818398in}{2.671886in}}{\pgfqpoint{1.821671in}{2.679786in}}{\pgfqpoint{1.821671in}{2.688022in}}%
\pgfpathcurveto{\pgfqpoint{1.821671in}{2.696258in}}{\pgfqpoint{1.818398in}{2.704158in}}{\pgfqpoint{1.812575in}{2.709982in}}%
\pgfpathcurveto{\pgfqpoint{1.806751in}{2.715806in}}{\pgfqpoint{1.798851in}{2.719079in}}{\pgfqpoint{1.790614in}{2.719079in}}%
\pgfpathcurveto{\pgfqpoint{1.782378in}{2.719079in}}{\pgfqpoint{1.774478in}{2.715806in}}{\pgfqpoint{1.768654in}{2.709982in}}%
\pgfpathcurveto{\pgfqpoint{1.762830in}{2.704158in}}{\pgfqpoint{1.759558in}{2.696258in}}{\pgfqpoint{1.759558in}{2.688022in}}%
\pgfpathcurveto{\pgfqpoint{1.759558in}{2.679786in}}{\pgfqpoint{1.762830in}{2.671886in}}{\pgfqpoint{1.768654in}{2.666062in}}%
\pgfpathcurveto{\pgfqpoint{1.774478in}{2.660238in}}{\pgfqpoint{1.782378in}{2.656966in}}{\pgfqpoint{1.790614in}{2.656966in}}%
\pgfpathclose%
\pgfusepath{stroke,fill}%
\end{pgfscope}%
\begin{pgfscope}%
\pgfpathrectangle{\pgfqpoint{0.100000in}{0.212622in}}{\pgfqpoint{3.696000in}{3.696000in}}%
\pgfusepath{clip}%
\pgfsetbuttcap%
\pgfsetroundjoin%
\definecolor{currentfill}{rgb}{0.121569,0.466667,0.705882}%
\pgfsetfillcolor{currentfill}%
\pgfsetfillopacity{0.884508}%
\pgfsetlinewidth{1.003750pt}%
\definecolor{currentstroke}{rgb}{0.121569,0.466667,0.705882}%
\pgfsetstrokecolor{currentstroke}%
\pgfsetstrokeopacity{0.884508}%
\pgfsetdash{}{0pt}%
\pgfpathmoveto{\pgfqpoint{2.223615in}{2.526479in}}%
\pgfpathcurveto{\pgfqpoint{2.231852in}{2.526479in}}{\pgfqpoint{2.239752in}{2.529751in}}{\pgfqpoint{2.245576in}{2.535575in}}%
\pgfpathcurveto{\pgfqpoint{2.251400in}{2.541399in}}{\pgfqpoint{2.254672in}{2.549299in}}{\pgfqpoint{2.254672in}{2.557535in}}%
\pgfpathcurveto{\pgfqpoint{2.254672in}{2.565772in}}{\pgfqpoint{2.251400in}{2.573672in}}{\pgfqpoint{2.245576in}{2.579496in}}%
\pgfpathcurveto{\pgfqpoint{2.239752in}{2.585320in}}{\pgfqpoint{2.231852in}{2.588592in}}{\pgfqpoint{2.223615in}{2.588592in}}%
\pgfpathcurveto{\pgfqpoint{2.215379in}{2.588592in}}{\pgfqpoint{2.207479in}{2.585320in}}{\pgfqpoint{2.201655in}{2.579496in}}%
\pgfpathcurveto{\pgfqpoint{2.195831in}{2.573672in}}{\pgfqpoint{2.192559in}{2.565772in}}{\pgfqpoint{2.192559in}{2.557535in}}%
\pgfpathcurveto{\pgfqpoint{2.192559in}{2.549299in}}{\pgfqpoint{2.195831in}{2.541399in}}{\pgfqpoint{2.201655in}{2.535575in}}%
\pgfpathcurveto{\pgfqpoint{2.207479in}{2.529751in}}{\pgfqpoint{2.215379in}{2.526479in}}{\pgfqpoint{2.223615in}{2.526479in}}%
\pgfpathclose%
\pgfusepath{stroke,fill}%
\end{pgfscope}%
\begin{pgfscope}%
\pgfpathrectangle{\pgfqpoint{0.100000in}{0.212622in}}{\pgfqpoint{3.696000in}{3.696000in}}%
\pgfusepath{clip}%
\pgfsetbuttcap%
\pgfsetroundjoin%
\definecolor{currentfill}{rgb}{0.121569,0.466667,0.705882}%
\pgfsetfillcolor{currentfill}%
\pgfsetfillopacity{0.885236}%
\pgfsetlinewidth{1.003750pt}%
\definecolor{currentstroke}{rgb}{0.121569,0.466667,0.705882}%
\pgfsetstrokecolor{currentstroke}%
\pgfsetstrokeopacity{0.885236}%
\pgfsetdash{}{0pt}%
\pgfpathmoveto{\pgfqpoint{1.788394in}{2.654049in}}%
\pgfpathcurveto{\pgfqpoint{1.796630in}{2.654049in}}{\pgfqpoint{1.804531in}{2.657321in}}{\pgfqpoint{1.810354in}{2.663145in}}%
\pgfpathcurveto{\pgfqpoint{1.816178in}{2.668969in}}{\pgfqpoint{1.819451in}{2.676869in}}{\pgfqpoint{1.819451in}{2.685106in}}%
\pgfpathcurveto{\pgfqpoint{1.819451in}{2.693342in}}{\pgfqpoint{1.816178in}{2.701242in}}{\pgfqpoint{1.810354in}{2.707066in}}%
\pgfpathcurveto{\pgfqpoint{1.804531in}{2.712890in}}{\pgfqpoint{1.796630in}{2.716162in}}{\pgfqpoint{1.788394in}{2.716162in}}%
\pgfpathcurveto{\pgfqpoint{1.780158in}{2.716162in}}{\pgfqpoint{1.772258in}{2.712890in}}{\pgfqpoint{1.766434in}{2.707066in}}%
\pgfpathcurveto{\pgfqpoint{1.760610in}{2.701242in}}{\pgfqpoint{1.757338in}{2.693342in}}{\pgfqpoint{1.757338in}{2.685106in}}%
\pgfpathcurveto{\pgfqpoint{1.757338in}{2.676869in}}{\pgfqpoint{1.760610in}{2.668969in}}{\pgfqpoint{1.766434in}{2.663145in}}%
\pgfpathcurveto{\pgfqpoint{1.772258in}{2.657321in}}{\pgfqpoint{1.780158in}{2.654049in}}{\pgfqpoint{1.788394in}{2.654049in}}%
\pgfpathclose%
\pgfusepath{stroke,fill}%
\end{pgfscope}%
\begin{pgfscope}%
\pgfpathrectangle{\pgfqpoint{0.100000in}{0.212622in}}{\pgfqpoint{3.696000in}{3.696000in}}%
\pgfusepath{clip}%
\pgfsetbuttcap%
\pgfsetroundjoin%
\definecolor{currentfill}{rgb}{0.121569,0.466667,0.705882}%
\pgfsetfillcolor{currentfill}%
\pgfsetfillopacity{0.885483}%
\pgfsetlinewidth{1.003750pt}%
\definecolor{currentstroke}{rgb}{0.121569,0.466667,0.705882}%
\pgfsetstrokecolor{currentstroke}%
\pgfsetstrokeopacity{0.885483}%
\pgfsetdash{}{0pt}%
\pgfpathmoveto{\pgfqpoint{2.221260in}{2.524292in}}%
\pgfpathcurveto{\pgfqpoint{2.229497in}{2.524292in}}{\pgfqpoint{2.237397in}{2.527564in}}{\pgfqpoint{2.243221in}{2.533388in}}%
\pgfpathcurveto{\pgfqpoint{2.249045in}{2.539212in}}{\pgfqpoint{2.252317in}{2.547112in}}{\pgfqpoint{2.252317in}{2.555348in}}%
\pgfpathcurveto{\pgfqpoint{2.252317in}{2.563585in}}{\pgfqpoint{2.249045in}{2.571485in}}{\pgfqpoint{2.243221in}{2.577309in}}%
\pgfpathcurveto{\pgfqpoint{2.237397in}{2.583132in}}{\pgfqpoint{2.229497in}{2.586405in}}{\pgfqpoint{2.221260in}{2.586405in}}%
\pgfpathcurveto{\pgfqpoint{2.213024in}{2.586405in}}{\pgfqpoint{2.205124in}{2.583132in}}{\pgfqpoint{2.199300in}{2.577309in}}%
\pgfpathcurveto{\pgfqpoint{2.193476in}{2.571485in}}{\pgfqpoint{2.190204in}{2.563585in}}{\pgfqpoint{2.190204in}{2.555348in}}%
\pgfpathcurveto{\pgfqpoint{2.190204in}{2.547112in}}{\pgfqpoint{2.193476in}{2.539212in}}{\pgfqpoint{2.199300in}{2.533388in}}%
\pgfpathcurveto{\pgfqpoint{2.205124in}{2.527564in}}{\pgfqpoint{2.213024in}{2.524292in}}{\pgfqpoint{2.221260in}{2.524292in}}%
\pgfpathclose%
\pgfusepath{stroke,fill}%
\end{pgfscope}%
\begin{pgfscope}%
\pgfpathrectangle{\pgfqpoint{0.100000in}{0.212622in}}{\pgfqpoint{3.696000in}{3.696000in}}%
\pgfusepath{clip}%
\pgfsetbuttcap%
\pgfsetroundjoin%
\definecolor{currentfill}{rgb}{0.121569,0.466667,0.705882}%
\pgfsetfillcolor{currentfill}%
\pgfsetfillopacity{0.885619}%
\pgfsetlinewidth{1.003750pt}%
\definecolor{currentstroke}{rgb}{0.121569,0.466667,0.705882}%
\pgfsetstrokecolor{currentstroke}%
\pgfsetstrokeopacity{0.885619}%
\pgfsetdash{}{0pt}%
\pgfpathmoveto{\pgfqpoint{0.935214in}{2.107510in}}%
\pgfpathcurveto{\pgfqpoint{0.943450in}{2.107510in}}{\pgfqpoint{0.951350in}{2.110782in}}{\pgfqpoint{0.957174in}{2.116606in}}%
\pgfpathcurveto{\pgfqpoint{0.962998in}{2.122430in}}{\pgfqpoint{0.966270in}{2.130330in}}{\pgfqpoint{0.966270in}{2.138567in}}%
\pgfpathcurveto{\pgfqpoint{0.966270in}{2.146803in}}{\pgfqpoint{0.962998in}{2.154703in}}{\pgfqpoint{0.957174in}{2.160527in}}%
\pgfpathcurveto{\pgfqpoint{0.951350in}{2.166351in}}{\pgfqpoint{0.943450in}{2.169623in}}{\pgfqpoint{0.935214in}{2.169623in}}%
\pgfpathcurveto{\pgfqpoint{0.926978in}{2.169623in}}{\pgfqpoint{0.919078in}{2.166351in}}{\pgfqpoint{0.913254in}{2.160527in}}%
\pgfpathcurveto{\pgfqpoint{0.907430in}{2.154703in}}{\pgfqpoint{0.904157in}{2.146803in}}{\pgfqpoint{0.904157in}{2.138567in}}%
\pgfpathcurveto{\pgfqpoint{0.904157in}{2.130330in}}{\pgfqpoint{0.907430in}{2.122430in}}{\pgfqpoint{0.913254in}{2.116606in}}%
\pgfpathcurveto{\pgfqpoint{0.919078in}{2.110782in}}{\pgfqpoint{0.926978in}{2.107510in}}{\pgfqpoint{0.935214in}{2.107510in}}%
\pgfpathclose%
\pgfusepath{stroke,fill}%
\end{pgfscope}%
\begin{pgfscope}%
\pgfpathrectangle{\pgfqpoint{0.100000in}{0.212622in}}{\pgfqpoint{3.696000in}{3.696000in}}%
\pgfusepath{clip}%
\pgfsetbuttcap%
\pgfsetroundjoin%
\definecolor{currentfill}{rgb}{0.121569,0.466667,0.705882}%
\pgfsetfillcolor{currentfill}%
\pgfsetfillopacity{0.886132}%
\pgfsetlinewidth{1.003750pt}%
\definecolor{currentstroke}{rgb}{0.121569,0.466667,0.705882}%
\pgfsetstrokecolor{currentstroke}%
\pgfsetstrokeopacity{0.886132}%
\pgfsetdash{}{0pt}%
\pgfpathmoveto{\pgfqpoint{2.219771in}{2.523034in}}%
\pgfpathcurveto{\pgfqpoint{2.228007in}{2.523034in}}{\pgfqpoint{2.235907in}{2.526306in}}{\pgfqpoint{2.241731in}{2.532130in}}%
\pgfpathcurveto{\pgfqpoint{2.247555in}{2.537954in}}{\pgfqpoint{2.250828in}{2.545854in}}{\pgfqpoint{2.250828in}{2.554090in}}%
\pgfpathcurveto{\pgfqpoint{2.250828in}{2.562327in}}{\pgfqpoint{2.247555in}{2.570227in}}{\pgfqpoint{2.241731in}{2.576051in}}%
\pgfpathcurveto{\pgfqpoint{2.235907in}{2.581874in}}{\pgfqpoint{2.228007in}{2.585147in}}{\pgfqpoint{2.219771in}{2.585147in}}%
\pgfpathcurveto{\pgfqpoint{2.211535in}{2.585147in}}{\pgfqpoint{2.203635in}{2.581874in}}{\pgfqpoint{2.197811in}{2.576051in}}%
\pgfpathcurveto{\pgfqpoint{2.191987in}{2.570227in}}{\pgfqpoint{2.188715in}{2.562327in}}{\pgfqpoint{2.188715in}{2.554090in}}%
\pgfpathcurveto{\pgfqpoint{2.188715in}{2.545854in}}{\pgfqpoint{2.191987in}{2.537954in}}{\pgfqpoint{2.197811in}{2.532130in}}%
\pgfpathcurveto{\pgfqpoint{2.203635in}{2.526306in}}{\pgfqpoint{2.211535in}{2.523034in}}{\pgfqpoint{2.219771in}{2.523034in}}%
\pgfpathclose%
\pgfusepath{stroke,fill}%
\end{pgfscope}%
\begin{pgfscope}%
\pgfpathrectangle{\pgfqpoint{0.100000in}{0.212622in}}{\pgfqpoint{3.696000in}{3.696000in}}%
\pgfusepath{clip}%
\pgfsetbuttcap%
\pgfsetroundjoin%
\definecolor{currentfill}{rgb}{0.121569,0.466667,0.705882}%
\pgfsetfillcolor{currentfill}%
\pgfsetfillopacity{0.886550}%
\pgfsetlinewidth{1.003750pt}%
\definecolor{currentstroke}{rgb}{0.121569,0.466667,0.705882}%
\pgfsetstrokecolor{currentstroke}%
\pgfsetstrokeopacity{0.886550}%
\pgfsetdash{}{0pt}%
\pgfpathmoveto{\pgfqpoint{1.785545in}{2.650165in}}%
\pgfpathcurveto{\pgfqpoint{1.793782in}{2.650165in}}{\pgfqpoint{1.801682in}{2.653437in}}{\pgfqpoint{1.807506in}{2.659261in}}%
\pgfpathcurveto{\pgfqpoint{1.813330in}{2.665085in}}{\pgfqpoint{1.816602in}{2.672985in}}{\pgfqpoint{1.816602in}{2.681221in}}%
\pgfpathcurveto{\pgfqpoint{1.816602in}{2.689458in}}{\pgfqpoint{1.813330in}{2.697358in}}{\pgfqpoint{1.807506in}{2.703182in}}%
\pgfpathcurveto{\pgfqpoint{1.801682in}{2.709006in}}{\pgfqpoint{1.793782in}{2.712278in}}{\pgfqpoint{1.785545in}{2.712278in}}%
\pgfpathcurveto{\pgfqpoint{1.777309in}{2.712278in}}{\pgfqpoint{1.769409in}{2.709006in}}{\pgfqpoint{1.763585in}{2.703182in}}%
\pgfpathcurveto{\pgfqpoint{1.757761in}{2.697358in}}{\pgfqpoint{1.754489in}{2.689458in}}{\pgfqpoint{1.754489in}{2.681221in}}%
\pgfpathcurveto{\pgfqpoint{1.754489in}{2.672985in}}{\pgfqpoint{1.757761in}{2.665085in}}{\pgfqpoint{1.763585in}{2.659261in}}%
\pgfpathcurveto{\pgfqpoint{1.769409in}{2.653437in}}{\pgfqpoint{1.777309in}{2.650165in}}{\pgfqpoint{1.785545in}{2.650165in}}%
\pgfpathclose%
\pgfusepath{stroke,fill}%
\end{pgfscope}%
\begin{pgfscope}%
\pgfpathrectangle{\pgfqpoint{0.100000in}{0.212622in}}{\pgfqpoint{3.696000in}{3.696000in}}%
\pgfusepath{clip}%
\pgfsetbuttcap%
\pgfsetroundjoin%
\definecolor{currentfill}{rgb}{0.121569,0.466667,0.705882}%
\pgfsetfillcolor{currentfill}%
\pgfsetfillopacity{0.887016}%
\pgfsetlinewidth{1.003750pt}%
\definecolor{currentstroke}{rgb}{0.121569,0.466667,0.705882}%
\pgfsetstrokecolor{currentstroke}%
\pgfsetstrokeopacity{0.887016}%
\pgfsetdash{}{0pt}%
\pgfpathmoveto{\pgfqpoint{2.703909in}{1.336201in}}%
\pgfpathcurveto{\pgfqpoint{2.712145in}{1.336201in}}{\pgfqpoint{2.720045in}{1.339473in}}{\pgfqpoint{2.725869in}{1.345297in}}%
\pgfpathcurveto{\pgfqpoint{2.731693in}{1.351121in}}{\pgfqpoint{2.734965in}{1.359021in}}{\pgfqpoint{2.734965in}{1.367257in}}%
\pgfpathcurveto{\pgfqpoint{2.734965in}{1.375494in}}{\pgfqpoint{2.731693in}{1.383394in}}{\pgfqpoint{2.725869in}{1.389218in}}%
\pgfpathcurveto{\pgfqpoint{2.720045in}{1.395042in}}{\pgfqpoint{2.712145in}{1.398314in}}{\pgfqpoint{2.703909in}{1.398314in}}%
\pgfpathcurveto{\pgfqpoint{2.695672in}{1.398314in}}{\pgfqpoint{2.687772in}{1.395042in}}{\pgfqpoint{2.681948in}{1.389218in}}%
\pgfpathcurveto{\pgfqpoint{2.676125in}{1.383394in}}{\pgfqpoint{2.672852in}{1.375494in}}{\pgfqpoint{2.672852in}{1.367257in}}%
\pgfpathcurveto{\pgfqpoint{2.672852in}{1.359021in}}{\pgfqpoint{2.676125in}{1.351121in}}{\pgfqpoint{2.681948in}{1.345297in}}%
\pgfpathcurveto{\pgfqpoint{2.687772in}{1.339473in}}{\pgfqpoint{2.695672in}{1.336201in}}{\pgfqpoint{2.703909in}{1.336201in}}%
\pgfpathclose%
\pgfusepath{stroke,fill}%
\end{pgfscope}%
\begin{pgfscope}%
\pgfpathrectangle{\pgfqpoint{0.100000in}{0.212622in}}{\pgfqpoint{3.696000in}{3.696000in}}%
\pgfusepath{clip}%
\pgfsetbuttcap%
\pgfsetroundjoin%
\definecolor{currentfill}{rgb}{0.121569,0.466667,0.705882}%
\pgfsetfillcolor{currentfill}%
\pgfsetfillopacity{0.887234}%
\pgfsetlinewidth{1.003750pt}%
\definecolor{currentstroke}{rgb}{0.121569,0.466667,0.705882}%
\pgfsetstrokecolor{currentstroke}%
\pgfsetstrokeopacity{0.887234}%
\pgfsetdash{}{0pt}%
\pgfpathmoveto{\pgfqpoint{2.217084in}{2.520268in}}%
\pgfpathcurveto{\pgfqpoint{2.225320in}{2.520268in}}{\pgfqpoint{2.233220in}{2.523540in}}{\pgfqpoint{2.239044in}{2.529364in}}%
\pgfpathcurveto{\pgfqpoint{2.244868in}{2.535188in}}{\pgfqpoint{2.248140in}{2.543088in}}{\pgfqpoint{2.248140in}{2.551324in}}%
\pgfpathcurveto{\pgfqpoint{2.248140in}{2.559560in}}{\pgfqpoint{2.244868in}{2.567461in}}{\pgfqpoint{2.239044in}{2.573284in}}%
\pgfpathcurveto{\pgfqpoint{2.233220in}{2.579108in}}{\pgfqpoint{2.225320in}{2.582381in}}{\pgfqpoint{2.217084in}{2.582381in}}%
\pgfpathcurveto{\pgfqpoint{2.208847in}{2.582381in}}{\pgfqpoint{2.200947in}{2.579108in}}{\pgfqpoint{2.195123in}{2.573284in}}%
\pgfpathcurveto{\pgfqpoint{2.189299in}{2.567461in}}{\pgfqpoint{2.186027in}{2.559560in}}{\pgfqpoint{2.186027in}{2.551324in}}%
\pgfpathcurveto{\pgfqpoint{2.186027in}{2.543088in}}{\pgfqpoint{2.189299in}{2.535188in}}{\pgfqpoint{2.195123in}{2.529364in}}%
\pgfpathcurveto{\pgfqpoint{2.200947in}{2.523540in}}{\pgfqpoint{2.208847in}{2.520268in}}{\pgfqpoint{2.217084in}{2.520268in}}%
\pgfpathclose%
\pgfusepath{stroke,fill}%
\end{pgfscope}%
\begin{pgfscope}%
\pgfpathrectangle{\pgfqpoint{0.100000in}{0.212622in}}{\pgfqpoint{3.696000in}{3.696000in}}%
\pgfusepath{clip}%
\pgfsetbuttcap%
\pgfsetroundjoin%
\definecolor{currentfill}{rgb}{0.121569,0.466667,0.705882}%
\pgfsetfillcolor{currentfill}%
\pgfsetfillopacity{0.887979}%
\pgfsetlinewidth{1.003750pt}%
\definecolor{currentstroke}{rgb}{0.121569,0.466667,0.705882}%
\pgfsetstrokecolor{currentstroke}%
\pgfsetstrokeopacity{0.887979}%
\pgfsetdash{}{0pt}%
\pgfpathmoveto{\pgfqpoint{1.782445in}{2.645946in}}%
\pgfpathcurveto{\pgfqpoint{1.790682in}{2.645946in}}{\pgfqpoint{1.798582in}{2.649218in}}{\pgfqpoint{1.804406in}{2.655042in}}%
\pgfpathcurveto{\pgfqpoint{1.810230in}{2.660866in}}{\pgfqpoint{1.813502in}{2.668766in}}{\pgfqpoint{1.813502in}{2.677002in}}%
\pgfpathcurveto{\pgfqpoint{1.813502in}{2.685238in}}{\pgfqpoint{1.810230in}{2.693138in}}{\pgfqpoint{1.804406in}{2.698962in}}%
\pgfpathcurveto{\pgfqpoint{1.798582in}{2.704786in}}{\pgfqpoint{1.790682in}{2.708059in}}{\pgfqpoint{1.782445in}{2.708059in}}%
\pgfpathcurveto{\pgfqpoint{1.774209in}{2.708059in}}{\pgfqpoint{1.766309in}{2.704786in}}{\pgfqpoint{1.760485in}{2.698962in}}%
\pgfpathcurveto{\pgfqpoint{1.754661in}{2.693138in}}{\pgfqpoint{1.751389in}{2.685238in}}{\pgfqpoint{1.751389in}{2.677002in}}%
\pgfpathcurveto{\pgfqpoint{1.751389in}{2.668766in}}{\pgfqpoint{1.754661in}{2.660866in}}{\pgfqpoint{1.760485in}{2.655042in}}%
\pgfpathcurveto{\pgfqpoint{1.766309in}{2.649218in}}{\pgfqpoint{1.774209in}{2.645946in}}{\pgfqpoint{1.782445in}{2.645946in}}%
\pgfpathclose%
\pgfusepath{stroke,fill}%
\end{pgfscope}%
\begin{pgfscope}%
\pgfpathrectangle{\pgfqpoint{0.100000in}{0.212622in}}{\pgfqpoint{3.696000in}{3.696000in}}%
\pgfusepath{clip}%
\pgfsetbuttcap%
\pgfsetroundjoin%
\definecolor{currentfill}{rgb}{0.121569,0.466667,0.705882}%
\pgfsetfillcolor{currentfill}%
\pgfsetfillopacity{0.888066}%
\pgfsetlinewidth{1.003750pt}%
\definecolor{currentstroke}{rgb}{0.121569,0.466667,0.705882}%
\pgfsetstrokecolor{currentstroke}%
\pgfsetstrokeopacity{0.888066}%
\pgfsetdash{}{0pt}%
\pgfpathmoveto{\pgfqpoint{2.214961in}{2.517914in}}%
\pgfpathcurveto{\pgfqpoint{2.223198in}{2.517914in}}{\pgfqpoint{2.231098in}{2.521187in}}{\pgfqpoint{2.236922in}{2.527010in}}%
\pgfpathcurveto{\pgfqpoint{2.242746in}{2.532834in}}{\pgfqpoint{2.246018in}{2.540734in}}{\pgfqpoint{2.246018in}{2.548971in}}%
\pgfpathcurveto{\pgfqpoint{2.246018in}{2.557207in}}{\pgfqpoint{2.242746in}{2.565107in}}{\pgfqpoint{2.236922in}{2.570931in}}%
\pgfpathcurveto{\pgfqpoint{2.231098in}{2.576755in}}{\pgfqpoint{2.223198in}{2.580027in}}{\pgfqpoint{2.214961in}{2.580027in}}%
\pgfpathcurveto{\pgfqpoint{2.206725in}{2.580027in}}{\pgfqpoint{2.198825in}{2.576755in}}{\pgfqpoint{2.193001in}{2.570931in}}%
\pgfpathcurveto{\pgfqpoint{2.187177in}{2.565107in}}{\pgfqpoint{2.183905in}{2.557207in}}{\pgfqpoint{2.183905in}{2.548971in}}%
\pgfpathcurveto{\pgfqpoint{2.183905in}{2.540734in}}{\pgfqpoint{2.187177in}{2.532834in}}{\pgfqpoint{2.193001in}{2.527010in}}%
\pgfpathcurveto{\pgfqpoint{2.198825in}{2.521187in}}{\pgfqpoint{2.206725in}{2.517914in}}{\pgfqpoint{2.214961in}{2.517914in}}%
\pgfpathclose%
\pgfusepath{stroke,fill}%
\end{pgfscope}%
\begin{pgfscope}%
\pgfpathrectangle{\pgfqpoint{0.100000in}{0.212622in}}{\pgfqpoint{3.696000in}{3.696000in}}%
\pgfusepath{clip}%
\pgfsetbuttcap%
\pgfsetroundjoin%
\definecolor{currentfill}{rgb}{0.121569,0.466667,0.705882}%
\pgfsetfillcolor{currentfill}%
\pgfsetfillopacity{0.888790}%
\pgfsetlinewidth{1.003750pt}%
\definecolor{currentstroke}{rgb}{0.121569,0.466667,0.705882}%
\pgfsetstrokecolor{currentstroke}%
\pgfsetstrokeopacity{0.888790}%
\pgfsetdash{}{0pt}%
\pgfpathmoveto{\pgfqpoint{2.213152in}{2.516130in}}%
\pgfpathcurveto{\pgfqpoint{2.221388in}{2.516130in}}{\pgfqpoint{2.229288in}{2.519402in}}{\pgfqpoint{2.235112in}{2.525226in}}%
\pgfpathcurveto{\pgfqpoint{2.240936in}{2.531050in}}{\pgfqpoint{2.244208in}{2.538950in}}{\pgfqpoint{2.244208in}{2.547187in}}%
\pgfpathcurveto{\pgfqpoint{2.244208in}{2.555423in}}{\pgfqpoint{2.240936in}{2.563323in}}{\pgfqpoint{2.235112in}{2.569147in}}%
\pgfpathcurveto{\pgfqpoint{2.229288in}{2.574971in}}{\pgfqpoint{2.221388in}{2.578243in}}{\pgfqpoint{2.213152in}{2.578243in}}%
\pgfpathcurveto{\pgfqpoint{2.204916in}{2.578243in}}{\pgfqpoint{2.197015in}{2.574971in}}{\pgfqpoint{2.191192in}{2.569147in}}%
\pgfpathcurveto{\pgfqpoint{2.185368in}{2.563323in}}{\pgfqpoint{2.182095in}{2.555423in}}{\pgfqpoint{2.182095in}{2.547187in}}%
\pgfpathcurveto{\pgfqpoint{2.182095in}{2.538950in}}{\pgfqpoint{2.185368in}{2.531050in}}{\pgfqpoint{2.191192in}{2.525226in}}%
\pgfpathcurveto{\pgfqpoint{2.197015in}{2.519402in}}{\pgfqpoint{2.204916in}{2.516130in}}{\pgfqpoint{2.213152in}{2.516130in}}%
\pgfpathclose%
\pgfusepath{stroke,fill}%
\end{pgfscope}%
\begin{pgfscope}%
\pgfpathrectangle{\pgfqpoint{0.100000in}{0.212622in}}{\pgfqpoint{3.696000in}{3.696000in}}%
\pgfusepath{clip}%
\pgfsetbuttcap%
\pgfsetroundjoin%
\definecolor{currentfill}{rgb}{0.121569,0.466667,0.705882}%
\pgfsetfillcolor{currentfill}%
\pgfsetfillopacity{0.889080}%
\pgfsetlinewidth{1.003750pt}%
\definecolor{currentstroke}{rgb}{0.121569,0.466667,0.705882}%
\pgfsetstrokecolor{currentstroke}%
\pgfsetstrokeopacity{0.889080}%
\pgfsetdash{}{0pt}%
\pgfpathmoveto{\pgfqpoint{2.212459in}{2.515515in}}%
\pgfpathcurveto{\pgfqpoint{2.220695in}{2.515515in}}{\pgfqpoint{2.228595in}{2.518787in}}{\pgfqpoint{2.234419in}{2.524611in}}%
\pgfpathcurveto{\pgfqpoint{2.240243in}{2.530435in}}{\pgfqpoint{2.243515in}{2.538335in}}{\pgfqpoint{2.243515in}{2.546571in}}%
\pgfpathcurveto{\pgfqpoint{2.243515in}{2.554808in}}{\pgfqpoint{2.240243in}{2.562708in}}{\pgfqpoint{2.234419in}{2.568532in}}%
\pgfpathcurveto{\pgfqpoint{2.228595in}{2.574356in}}{\pgfqpoint{2.220695in}{2.577628in}}{\pgfqpoint{2.212459in}{2.577628in}}%
\pgfpathcurveto{\pgfqpoint{2.204222in}{2.577628in}}{\pgfqpoint{2.196322in}{2.574356in}}{\pgfqpoint{2.190498in}{2.568532in}}%
\pgfpathcurveto{\pgfqpoint{2.184674in}{2.562708in}}{\pgfqpoint{2.181402in}{2.554808in}}{\pgfqpoint{2.181402in}{2.546571in}}%
\pgfpathcurveto{\pgfqpoint{2.181402in}{2.538335in}}{\pgfqpoint{2.184674in}{2.530435in}}{\pgfqpoint{2.190498in}{2.524611in}}%
\pgfpathcurveto{\pgfqpoint{2.196322in}{2.518787in}}{\pgfqpoint{2.204222in}{2.515515in}}{\pgfqpoint{2.212459in}{2.515515in}}%
\pgfpathclose%
\pgfusepath{stroke,fill}%
\end{pgfscope}%
\begin{pgfscope}%
\pgfpathrectangle{\pgfqpoint{0.100000in}{0.212622in}}{\pgfqpoint{3.696000in}{3.696000in}}%
\pgfusepath{clip}%
\pgfsetbuttcap%
\pgfsetroundjoin%
\definecolor{currentfill}{rgb}{0.121569,0.466667,0.705882}%
\pgfsetfillcolor{currentfill}%
\pgfsetfillopacity{0.889125}%
\pgfsetlinewidth{1.003750pt}%
\definecolor{currentstroke}{rgb}{0.121569,0.466667,0.705882}%
\pgfsetstrokecolor{currentstroke}%
\pgfsetstrokeopacity{0.889125}%
\pgfsetdash{}{0pt}%
\pgfpathmoveto{\pgfqpoint{0.935716in}{2.100551in}}%
\pgfpathcurveto{\pgfqpoint{0.943953in}{2.100551in}}{\pgfqpoint{0.951853in}{2.103823in}}{\pgfqpoint{0.957677in}{2.109647in}}%
\pgfpathcurveto{\pgfqpoint{0.963500in}{2.115471in}}{\pgfqpoint{0.966773in}{2.123371in}}{\pgfqpoint{0.966773in}{2.131607in}}%
\pgfpathcurveto{\pgfqpoint{0.966773in}{2.139844in}}{\pgfqpoint{0.963500in}{2.147744in}}{\pgfqpoint{0.957677in}{2.153568in}}%
\pgfpathcurveto{\pgfqpoint{0.951853in}{2.159391in}}{\pgfqpoint{0.943953in}{2.162664in}}{\pgfqpoint{0.935716in}{2.162664in}}%
\pgfpathcurveto{\pgfqpoint{0.927480in}{2.162664in}}{\pgfqpoint{0.919580in}{2.159391in}}{\pgfqpoint{0.913756in}{2.153568in}}%
\pgfpathcurveto{\pgfqpoint{0.907932in}{2.147744in}}{\pgfqpoint{0.904660in}{2.139844in}}{\pgfqpoint{0.904660in}{2.131607in}}%
\pgfpathcurveto{\pgfqpoint{0.904660in}{2.123371in}}{\pgfqpoint{0.907932in}{2.115471in}}{\pgfqpoint{0.913756in}{2.109647in}}%
\pgfpathcurveto{\pgfqpoint{0.919580in}{2.103823in}}{\pgfqpoint{0.927480in}{2.100551in}}{\pgfqpoint{0.935716in}{2.100551in}}%
\pgfpathclose%
\pgfusepath{stroke,fill}%
\end{pgfscope}%
\begin{pgfscope}%
\pgfpathrectangle{\pgfqpoint{0.100000in}{0.212622in}}{\pgfqpoint{3.696000in}{3.696000in}}%
\pgfusepath{clip}%
\pgfsetbuttcap%
\pgfsetroundjoin%
\definecolor{currentfill}{rgb}{0.121569,0.466667,0.705882}%
\pgfsetfillcolor{currentfill}%
\pgfsetfillopacity{0.889614}%
\pgfsetlinewidth{1.003750pt}%
\definecolor{currentstroke}{rgb}{0.121569,0.466667,0.705882}%
\pgfsetstrokecolor{currentstroke}%
\pgfsetstrokeopacity{0.889614}%
\pgfsetdash{}{0pt}%
\pgfpathmoveto{\pgfqpoint{2.211183in}{2.514451in}}%
\pgfpathcurveto{\pgfqpoint{2.219420in}{2.514451in}}{\pgfqpoint{2.227320in}{2.517724in}}{\pgfqpoint{2.233144in}{2.523547in}}%
\pgfpathcurveto{\pgfqpoint{2.238967in}{2.529371in}}{\pgfqpoint{2.242240in}{2.537271in}}{\pgfqpoint{2.242240in}{2.545508in}}%
\pgfpathcurveto{\pgfqpoint{2.242240in}{2.553744in}}{\pgfqpoint{2.238967in}{2.561644in}}{\pgfqpoint{2.233144in}{2.567468in}}%
\pgfpathcurveto{\pgfqpoint{2.227320in}{2.573292in}}{\pgfqpoint{2.219420in}{2.576564in}}{\pgfqpoint{2.211183in}{2.576564in}}%
\pgfpathcurveto{\pgfqpoint{2.202947in}{2.576564in}}{\pgfqpoint{2.195047in}{2.573292in}}{\pgfqpoint{2.189223in}{2.567468in}}%
\pgfpathcurveto{\pgfqpoint{2.183399in}{2.561644in}}{\pgfqpoint{2.180127in}{2.553744in}}{\pgfqpoint{2.180127in}{2.545508in}}%
\pgfpathcurveto{\pgfqpoint{2.180127in}{2.537271in}}{\pgfqpoint{2.183399in}{2.529371in}}{\pgfqpoint{2.189223in}{2.523547in}}%
\pgfpathcurveto{\pgfqpoint{2.195047in}{2.517724in}}{\pgfqpoint{2.202947in}{2.514451in}}{\pgfqpoint{2.211183in}{2.514451in}}%
\pgfpathclose%
\pgfusepath{stroke,fill}%
\end{pgfscope}%
\begin{pgfscope}%
\pgfpathrectangle{\pgfqpoint{0.100000in}{0.212622in}}{\pgfqpoint{3.696000in}{3.696000in}}%
\pgfusepath{clip}%
\pgfsetbuttcap%
\pgfsetroundjoin%
\definecolor{currentfill}{rgb}{0.121569,0.466667,0.705882}%
\pgfsetfillcolor{currentfill}%
\pgfsetfillopacity{0.889626}%
\pgfsetlinewidth{1.003750pt}%
\definecolor{currentstroke}{rgb}{0.121569,0.466667,0.705882}%
\pgfsetstrokecolor{currentstroke}%
\pgfsetstrokeopacity{0.889626}%
\pgfsetdash{}{0pt}%
\pgfpathmoveto{\pgfqpoint{1.778799in}{2.641876in}}%
\pgfpathcurveto{\pgfqpoint{1.787035in}{2.641876in}}{\pgfqpoint{1.794935in}{2.645149in}}{\pgfqpoint{1.800759in}{2.650973in}}%
\pgfpathcurveto{\pgfqpoint{1.806583in}{2.656796in}}{\pgfqpoint{1.809855in}{2.664697in}}{\pgfqpoint{1.809855in}{2.672933in}}%
\pgfpathcurveto{\pgfqpoint{1.809855in}{2.681169in}}{\pgfqpoint{1.806583in}{2.689069in}}{\pgfqpoint{1.800759in}{2.694893in}}%
\pgfpathcurveto{\pgfqpoint{1.794935in}{2.700717in}}{\pgfqpoint{1.787035in}{2.703989in}}{\pgfqpoint{1.778799in}{2.703989in}}%
\pgfpathcurveto{\pgfqpoint{1.770563in}{2.703989in}}{\pgfqpoint{1.762663in}{2.700717in}}{\pgfqpoint{1.756839in}{2.694893in}}%
\pgfpathcurveto{\pgfqpoint{1.751015in}{2.689069in}}{\pgfqpoint{1.747742in}{2.681169in}}{\pgfqpoint{1.747742in}{2.672933in}}%
\pgfpathcurveto{\pgfqpoint{1.747742in}{2.664697in}}{\pgfqpoint{1.751015in}{2.656796in}}{\pgfqpoint{1.756839in}{2.650973in}}%
\pgfpathcurveto{\pgfqpoint{1.762663in}{2.645149in}}{\pgfqpoint{1.770563in}{2.641876in}}{\pgfqpoint{1.778799in}{2.641876in}}%
\pgfpathclose%
\pgfusepath{stroke,fill}%
\end{pgfscope}%
\begin{pgfscope}%
\pgfpathrectangle{\pgfqpoint{0.100000in}{0.212622in}}{\pgfqpoint{3.696000in}{3.696000in}}%
\pgfusepath{clip}%
\pgfsetbuttcap%
\pgfsetroundjoin%
\definecolor{currentfill}{rgb}{0.121569,0.466667,0.705882}%
\pgfsetfillcolor{currentfill}%
\pgfsetfillopacity{0.889904}%
\pgfsetlinewidth{1.003750pt}%
\definecolor{currentstroke}{rgb}{0.121569,0.466667,0.705882}%
\pgfsetstrokecolor{currentstroke}%
\pgfsetstrokeopacity{0.889904}%
\pgfsetdash{}{0pt}%
\pgfpathmoveto{\pgfqpoint{2.210472in}{2.513830in}}%
\pgfpathcurveto{\pgfqpoint{2.218708in}{2.513830in}}{\pgfqpoint{2.226608in}{2.517102in}}{\pgfqpoint{2.232432in}{2.522926in}}%
\pgfpathcurveto{\pgfqpoint{2.238256in}{2.528750in}}{\pgfqpoint{2.241529in}{2.536650in}}{\pgfqpoint{2.241529in}{2.544886in}}%
\pgfpathcurveto{\pgfqpoint{2.241529in}{2.553123in}}{\pgfqpoint{2.238256in}{2.561023in}}{\pgfqpoint{2.232432in}{2.566847in}}%
\pgfpathcurveto{\pgfqpoint{2.226608in}{2.572670in}}{\pgfqpoint{2.218708in}{2.575943in}}{\pgfqpoint{2.210472in}{2.575943in}}%
\pgfpathcurveto{\pgfqpoint{2.202236in}{2.575943in}}{\pgfqpoint{2.194336in}{2.572670in}}{\pgfqpoint{2.188512in}{2.566847in}}%
\pgfpathcurveto{\pgfqpoint{2.182688in}{2.561023in}}{\pgfqpoint{2.179416in}{2.553123in}}{\pgfqpoint{2.179416in}{2.544886in}}%
\pgfpathcurveto{\pgfqpoint{2.179416in}{2.536650in}}{\pgfqpoint{2.182688in}{2.528750in}}{\pgfqpoint{2.188512in}{2.522926in}}%
\pgfpathcurveto{\pgfqpoint{2.194336in}{2.517102in}}{\pgfqpoint{2.202236in}{2.513830in}}{\pgfqpoint{2.210472in}{2.513830in}}%
\pgfpathclose%
\pgfusepath{stroke,fill}%
\end{pgfscope}%
\begin{pgfscope}%
\pgfpathrectangle{\pgfqpoint{0.100000in}{0.212622in}}{\pgfqpoint{3.696000in}{3.696000in}}%
\pgfusepath{clip}%
\pgfsetbuttcap%
\pgfsetroundjoin%
\definecolor{currentfill}{rgb}{0.121569,0.466667,0.705882}%
\pgfsetfillcolor{currentfill}%
\pgfsetfillopacity{0.890425}%
\pgfsetlinewidth{1.003750pt}%
\definecolor{currentstroke}{rgb}{0.121569,0.466667,0.705882}%
\pgfsetstrokecolor{currentstroke}%
\pgfsetstrokeopacity{0.890425}%
\pgfsetdash{}{0pt}%
\pgfpathmoveto{\pgfqpoint{2.209149in}{2.512706in}}%
\pgfpathcurveto{\pgfqpoint{2.217386in}{2.512706in}}{\pgfqpoint{2.225286in}{2.515978in}}{\pgfqpoint{2.231110in}{2.521802in}}%
\pgfpathcurveto{\pgfqpoint{2.236933in}{2.527626in}}{\pgfqpoint{2.240206in}{2.535526in}}{\pgfqpoint{2.240206in}{2.543763in}}%
\pgfpathcurveto{\pgfqpoint{2.240206in}{2.551999in}}{\pgfqpoint{2.236933in}{2.559899in}}{\pgfqpoint{2.231110in}{2.565723in}}%
\pgfpathcurveto{\pgfqpoint{2.225286in}{2.571547in}}{\pgfqpoint{2.217386in}{2.574819in}}{\pgfqpoint{2.209149in}{2.574819in}}%
\pgfpathcurveto{\pgfqpoint{2.200913in}{2.574819in}}{\pgfqpoint{2.193013in}{2.571547in}}{\pgfqpoint{2.187189in}{2.565723in}}%
\pgfpathcurveto{\pgfqpoint{2.181365in}{2.559899in}}{\pgfqpoint{2.178093in}{2.551999in}}{\pgfqpoint{2.178093in}{2.543763in}}%
\pgfpathcurveto{\pgfqpoint{2.178093in}{2.535526in}}{\pgfqpoint{2.181365in}{2.527626in}}{\pgfqpoint{2.187189in}{2.521802in}}%
\pgfpathcurveto{\pgfqpoint{2.193013in}{2.515978in}}{\pgfqpoint{2.200913in}{2.512706in}}{\pgfqpoint{2.209149in}{2.512706in}}%
\pgfpathclose%
\pgfusepath{stroke,fill}%
\end{pgfscope}%
\begin{pgfscope}%
\pgfpathrectangle{\pgfqpoint{0.100000in}{0.212622in}}{\pgfqpoint{3.696000in}{3.696000in}}%
\pgfusepath{clip}%
\pgfsetbuttcap%
\pgfsetroundjoin%
\definecolor{currentfill}{rgb}{0.121569,0.466667,0.705882}%
\pgfsetfillcolor{currentfill}%
\pgfsetfillopacity{0.890483}%
\pgfsetlinewidth{1.003750pt}%
\definecolor{currentstroke}{rgb}{0.121569,0.466667,0.705882}%
\pgfsetstrokecolor{currentstroke}%
\pgfsetstrokeopacity{0.890483}%
\pgfsetdash{}{0pt}%
\pgfpathmoveto{\pgfqpoint{1.776759in}{2.639383in}}%
\pgfpathcurveto{\pgfqpoint{1.784995in}{2.639383in}}{\pgfqpoint{1.792896in}{2.642655in}}{\pgfqpoint{1.798719in}{2.648479in}}%
\pgfpathcurveto{\pgfqpoint{1.804543in}{2.654303in}}{\pgfqpoint{1.807816in}{2.662203in}}{\pgfqpoint{1.807816in}{2.670440in}}%
\pgfpathcurveto{\pgfqpoint{1.807816in}{2.678676in}}{\pgfqpoint{1.804543in}{2.686576in}}{\pgfqpoint{1.798719in}{2.692400in}}%
\pgfpathcurveto{\pgfqpoint{1.792896in}{2.698224in}}{\pgfqpoint{1.784995in}{2.701496in}}{\pgfqpoint{1.776759in}{2.701496in}}%
\pgfpathcurveto{\pgfqpoint{1.768523in}{2.701496in}}{\pgfqpoint{1.760623in}{2.698224in}}{\pgfqpoint{1.754799in}{2.692400in}}%
\pgfpathcurveto{\pgfqpoint{1.748975in}{2.686576in}}{\pgfqpoint{1.745703in}{2.678676in}}{\pgfqpoint{1.745703in}{2.670440in}}%
\pgfpathcurveto{\pgfqpoint{1.745703in}{2.662203in}}{\pgfqpoint{1.748975in}{2.654303in}}{\pgfqpoint{1.754799in}{2.648479in}}%
\pgfpathcurveto{\pgfqpoint{1.760623in}{2.642655in}}{\pgfqpoint{1.768523in}{2.639383in}}{\pgfqpoint{1.776759in}{2.639383in}}%
\pgfpathclose%
\pgfusepath{stroke,fill}%
\end{pgfscope}%
\begin{pgfscope}%
\pgfpathrectangle{\pgfqpoint{0.100000in}{0.212622in}}{\pgfqpoint{3.696000in}{3.696000in}}%
\pgfusepath{clip}%
\pgfsetbuttcap%
\pgfsetroundjoin%
\definecolor{currentfill}{rgb}{0.121569,0.466667,0.705882}%
\pgfsetfillcolor{currentfill}%
\pgfsetfillopacity{0.890724}%
\pgfsetlinewidth{1.003750pt}%
\definecolor{currentstroke}{rgb}{0.121569,0.466667,0.705882}%
\pgfsetstrokecolor{currentstroke}%
\pgfsetstrokeopacity{0.890724}%
\pgfsetdash{}{0pt}%
\pgfpathmoveto{\pgfqpoint{2.208415in}{2.512174in}}%
\pgfpathcurveto{\pgfqpoint{2.216651in}{2.512174in}}{\pgfqpoint{2.224551in}{2.515446in}}{\pgfqpoint{2.230375in}{2.521270in}}%
\pgfpathcurveto{\pgfqpoint{2.236199in}{2.527094in}}{\pgfqpoint{2.239471in}{2.534994in}}{\pgfqpoint{2.239471in}{2.543230in}}%
\pgfpathcurveto{\pgfqpoint{2.239471in}{2.551467in}}{\pgfqpoint{2.236199in}{2.559367in}}{\pgfqpoint{2.230375in}{2.565191in}}%
\pgfpathcurveto{\pgfqpoint{2.224551in}{2.571015in}}{\pgfqpoint{2.216651in}{2.574287in}}{\pgfqpoint{2.208415in}{2.574287in}}%
\pgfpathcurveto{\pgfqpoint{2.200178in}{2.574287in}}{\pgfqpoint{2.192278in}{2.571015in}}{\pgfqpoint{2.186454in}{2.565191in}}%
\pgfpathcurveto{\pgfqpoint{2.180630in}{2.559367in}}{\pgfqpoint{2.177358in}{2.551467in}}{\pgfqpoint{2.177358in}{2.543230in}}%
\pgfpathcurveto{\pgfqpoint{2.177358in}{2.534994in}}{\pgfqpoint{2.180630in}{2.527094in}}{\pgfqpoint{2.186454in}{2.521270in}}%
\pgfpathcurveto{\pgfqpoint{2.192278in}{2.515446in}}{\pgfqpoint{2.200178in}{2.512174in}}{\pgfqpoint{2.208415in}{2.512174in}}%
\pgfpathclose%
\pgfusepath{stroke,fill}%
\end{pgfscope}%
\begin{pgfscope}%
\pgfpathrectangle{\pgfqpoint{0.100000in}{0.212622in}}{\pgfqpoint{3.696000in}{3.696000in}}%
\pgfusepath{clip}%
\pgfsetbuttcap%
\pgfsetroundjoin%
\definecolor{currentfill}{rgb}{0.121569,0.466667,0.705882}%
\pgfsetfillcolor{currentfill}%
\pgfsetfillopacity{0.890975}%
\pgfsetlinewidth{1.003750pt}%
\definecolor{currentstroke}{rgb}{0.121569,0.466667,0.705882}%
\pgfsetstrokecolor{currentstroke}%
\pgfsetstrokeopacity{0.890975}%
\pgfsetdash{}{0pt}%
\pgfpathmoveto{\pgfqpoint{0.936495in}{2.096770in}}%
\pgfpathcurveto{\pgfqpoint{0.944731in}{2.096770in}}{\pgfqpoint{0.952631in}{2.100042in}}{\pgfqpoint{0.958455in}{2.105866in}}%
\pgfpathcurveto{\pgfqpoint{0.964279in}{2.111690in}}{\pgfqpoint{0.967551in}{2.119590in}}{\pgfqpoint{0.967551in}{2.127826in}}%
\pgfpathcurveto{\pgfqpoint{0.967551in}{2.136062in}}{\pgfqpoint{0.964279in}{2.143962in}}{\pgfqpoint{0.958455in}{2.149786in}}%
\pgfpathcurveto{\pgfqpoint{0.952631in}{2.155610in}}{\pgfqpoint{0.944731in}{2.158883in}}{\pgfqpoint{0.936495in}{2.158883in}}%
\pgfpathcurveto{\pgfqpoint{0.928259in}{2.158883in}}{\pgfqpoint{0.920359in}{2.155610in}}{\pgfqpoint{0.914535in}{2.149786in}}%
\pgfpathcurveto{\pgfqpoint{0.908711in}{2.143962in}}{\pgfqpoint{0.905438in}{2.136062in}}{\pgfqpoint{0.905438in}{2.127826in}}%
\pgfpathcurveto{\pgfqpoint{0.905438in}{2.119590in}}{\pgfqpoint{0.908711in}{2.111690in}}{\pgfqpoint{0.914535in}{2.105866in}}%
\pgfpathcurveto{\pgfqpoint{0.920359in}{2.100042in}}{\pgfqpoint{0.928259in}{2.096770in}}{\pgfqpoint{0.936495in}{2.096770in}}%
\pgfpathclose%
\pgfusepath{stroke,fill}%
\end{pgfscope}%
\begin{pgfscope}%
\pgfpathrectangle{\pgfqpoint{0.100000in}{0.212622in}}{\pgfqpoint{3.696000in}{3.696000in}}%
\pgfusepath{clip}%
\pgfsetbuttcap%
\pgfsetroundjoin%
\definecolor{currentfill}{rgb}{0.121569,0.466667,0.705882}%
\pgfsetfillcolor{currentfill}%
\pgfsetfillopacity{0.891287}%
\pgfsetlinewidth{1.003750pt}%
\definecolor{currentstroke}{rgb}{0.121569,0.466667,0.705882}%
\pgfsetstrokecolor{currentstroke}%
\pgfsetstrokeopacity{0.891287}%
\pgfsetdash{}{0pt}%
\pgfpathmoveto{\pgfqpoint{2.207097in}{2.511284in}}%
\pgfpathcurveto{\pgfqpoint{2.215333in}{2.511284in}}{\pgfqpoint{2.223233in}{2.514556in}}{\pgfqpoint{2.229057in}{2.520380in}}%
\pgfpathcurveto{\pgfqpoint{2.234881in}{2.526204in}}{\pgfqpoint{2.238153in}{2.534104in}}{\pgfqpoint{2.238153in}{2.542341in}}%
\pgfpathcurveto{\pgfqpoint{2.238153in}{2.550577in}}{\pgfqpoint{2.234881in}{2.558477in}}{\pgfqpoint{2.229057in}{2.564301in}}%
\pgfpathcurveto{\pgfqpoint{2.223233in}{2.570125in}}{\pgfqpoint{2.215333in}{2.573397in}}{\pgfqpoint{2.207097in}{2.573397in}}%
\pgfpathcurveto{\pgfqpoint{2.198861in}{2.573397in}}{\pgfqpoint{2.190961in}{2.570125in}}{\pgfqpoint{2.185137in}{2.564301in}}%
\pgfpathcurveto{\pgfqpoint{2.179313in}{2.558477in}}{\pgfqpoint{2.176040in}{2.550577in}}{\pgfqpoint{2.176040in}{2.542341in}}%
\pgfpathcurveto{\pgfqpoint{2.176040in}{2.534104in}}{\pgfqpoint{2.179313in}{2.526204in}}{\pgfqpoint{2.185137in}{2.520380in}}%
\pgfpathcurveto{\pgfqpoint{2.190961in}{2.514556in}}{\pgfqpoint{2.198861in}{2.511284in}}{\pgfqpoint{2.207097in}{2.511284in}}%
\pgfpathclose%
\pgfusepath{stroke,fill}%
\end{pgfscope}%
\begin{pgfscope}%
\pgfpathrectangle{\pgfqpoint{0.100000in}{0.212622in}}{\pgfqpoint{3.696000in}{3.696000in}}%
\pgfusepath{clip}%
\pgfsetbuttcap%
\pgfsetroundjoin%
\definecolor{currentfill}{rgb}{0.121569,0.466667,0.705882}%
\pgfsetfillcolor{currentfill}%
\pgfsetfillopacity{0.891488}%
\pgfsetlinewidth{1.003750pt}%
\definecolor{currentstroke}{rgb}{0.121569,0.466667,0.705882}%
\pgfsetstrokecolor{currentstroke}%
\pgfsetstrokeopacity{0.891488}%
\pgfsetdash{}{0pt}%
\pgfpathmoveto{\pgfqpoint{1.774480in}{2.636606in}}%
\pgfpathcurveto{\pgfqpoint{1.782716in}{2.636606in}}{\pgfqpoint{1.790616in}{2.639878in}}{\pgfqpoint{1.796440in}{2.645702in}}%
\pgfpathcurveto{\pgfqpoint{1.802264in}{2.651526in}}{\pgfqpoint{1.805537in}{2.659426in}}{\pgfqpoint{1.805537in}{2.667662in}}%
\pgfpathcurveto{\pgfqpoint{1.805537in}{2.675899in}}{\pgfqpoint{1.802264in}{2.683799in}}{\pgfqpoint{1.796440in}{2.689623in}}%
\pgfpathcurveto{\pgfqpoint{1.790616in}{2.695447in}}{\pgfqpoint{1.782716in}{2.698719in}}{\pgfqpoint{1.774480in}{2.698719in}}%
\pgfpathcurveto{\pgfqpoint{1.766244in}{2.698719in}}{\pgfqpoint{1.758344in}{2.695447in}}{\pgfqpoint{1.752520in}{2.689623in}}%
\pgfpathcurveto{\pgfqpoint{1.746696in}{2.683799in}}{\pgfqpoint{1.743424in}{2.675899in}}{\pgfqpoint{1.743424in}{2.667662in}}%
\pgfpathcurveto{\pgfqpoint{1.743424in}{2.659426in}}{\pgfqpoint{1.746696in}{2.651526in}}{\pgfqpoint{1.752520in}{2.645702in}}%
\pgfpathcurveto{\pgfqpoint{1.758344in}{2.639878in}}{\pgfqpoint{1.766244in}{2.636606in}}{\pgfqpoint{1.774480in}{2.636606in}}%
\pgfpathclose%
\pgfusepath{stroke,fill}%
\end{pgfscope}%
\begin{pgfscope}%
\pgfpathrectangle{\pgfqpoint{0.100000in}{0.212622in}}{\pgfqpoint{3.696000in}{3.696000in}}%
\pgfusepath{clip}%
\pgfsetbuttcap%
\pgfsetroundjoin%
\definecolor{currentfill}{rgb}{0.121569,0.466667,0.705882}%
\pgfsetfillcolor{currentfill}%
\pgfsetfillopacity{0.892263}%
\pgfsetlinewidth{1.003750pt}%
\definecolor{currentstroke}{rgb}{0.121569,0.466667,0.705882}%
\pgfsetstrokecolor{currentstroke}%
\pgfsetstrokeopacity{0.892263}%
\pgfsetdash{}{0pt}%
\pgfpathmoveto{\pgfqpoint{2.204674in}{2.509434in}}%
\pgfpathcurveto{\pgfqpoint{2.212910in}{2.509434in}}{\pgfqpoint{2.220810in}{2.512707in}}{\pgfqpoint{2.226634in}{2.518531in}}%
\pgfpathcurveto{\pgfqpoint{2.232458in}{2.524355in}}{\pgfqpoint{2.235731in}{2.532255in}}{\pgfqpoint{2.235731in}{2.540491in}}%
\pgfpathcurveto{\pgfqpoint{2.235731in}{2.548727in}}{\pgfqpoint{2.232458in}{2.556627in}}{\pgfqpoint{2.226634in}{2.562451in}}%
\pgfpathcurveto{\pgfqpoint{2.220810in}{2.568275in}}{\pgfqpoint{2.212910in}{2.571547in}}{\pgfqpoint{2.204674in}{2.571547in}}%
\pgfpathcurveto{\pgfqpoint{2.196438in}{2.571547in}}{\pgfqpoint{2.188538in}{2.568275in}}{\pgfqpoint{2.182714in}{2.562451in}}%
\pgfpathcurveto{\pgfqpoint{2.176890in}{2.556627in}}{\pgfqpoint{2.173618in}{2.548727in}}{\pgfqpoint{2.173618in}{2.540491in}}%
\pgfpathcurveto{\pgfqpoint{2.173618in}{2.532255in}}{\pgfqpoint{2.176890in}{2.524355in}}{\pgfqpoint{2.182714in}{2.518531in}}%
\pgfpathcurveto{\pgfqpoint{2.188538in}{2.512707in}}{\pgfqpoint{2.196438in}{2.509434in}}{\pgfqpoint{2.204674in}{2.509434in}}%
\pgfpathclose%
\pgfusepath{stroke,fill}%
\end{pgfscope}%
\begin{pgfscope}%
\pgfpathrectangle{\pgfqpoint{0.100000in}{0.212622in}}{\pgfqpoint{3.696000in}{3.696000in}}%
\pgfusepath{clip}%
\pgfsetbuttcap%
\pgfsetroundjoin%
\definecolor{currentfill}{rgb}{0.121569,0.466667,0.705882}%
\pgfsetfillcolor{currentfill}%
\pgfsetfillopacity{0.892650}%
\pgfsetlinewidth{1.003750pt}%
\definecolor{currentstroke}{rgb}{0.121569,0.466667,0.705882}%
\pgfsetstrokecolor{currentstroke}%
\pgfsetstrokeopacity{0.892650}%
\pgfsetdash{}{0pt}%
\pgfpathmoveto{\pgfqpoint{2.693210in}{1.324711in}}%
\pgfpathcurveto{\pgfqpoint{2.701446in}{1.324711in}}{\pgfqpoint{2.709346in}{1.327983in}}{\pgfqpoint{2.715170in}{1.333807in}}%
\pgfpathcurveto{\pgfqpoint{2.720994in}{1.339631in}}{\pgfqpoint{2.724266in}{1.347531in}}{\pgfqpoint{2.724266in}{1.355767in}}%
\pgfpathcurveto{\pgfqpoint{2.724266in}{1.364004in}}{\pgfqpoint{2.720994in}{1.371904in}}{\pgfqpoint{2.715170in}{1.377728in}}%
\pgfpathcurveto{\pgfqpoint{2.709346in}{1.383552in}}{\pgfqpoint{2.701446in}{1.386824in}}{\pgfqpoint{2.693210in}{1.386824in}}%
\pgfpathcurveto{\pgfqpoint{2.684973in}{1.386824in}}{\pgfqpoint{2.677073in}{1.383552in}}{\pgfqpoint{2.671249in}{1.377728in}}%
\pgfpathcurveto{\pgfqpoint{2.665426in}{1.371904in}}{\pgfqpoint{2.662153in}{1.364004in}}{\pgfqpoint{2.662153in}{1.355767in}}%
\pgfpathcurveto{\pgfqpoint{2.662153in}{1.347531in}}{\pgfqpoint{2.665426in}{1.339631in}}{\pgfqpoint{2.671249in}{1.333807in}}%
\pgfpathcurveto{\pgfqpoint{2.677073in}{1.327983in}}{\pgfqpoint{2.684973in}{1.324711in}}{\pgfqpoint{2.693210in}{1.324711in}}%
\pgfpathclose%
\pgfusepath{stroke,fill}%
\end{pgfscope}%
\begin{pgfscope}%
\pgfpathrectangle{\pgfqpoint{0.100000in}{0.212622in}}{\pgfqpoint{3.696000in}{3.696000in}}%
\pgfusepath{clip}%
\pgfsetbuttcap%
\pgfsetroundjoin%
\definecolor{currentfill}{rgb}{0.121569,0.466667,0.705882}%
\pgfsetfillcolor{currentfill}%
\pgfsetfillopacity{0.892845}%
\pgfsetlinewidth{1.003750pt}%
\definecolor{currentstroke}{rgb}{0.121569,0.466667,0.705882}%
\pgfsetstrokecolor{currentstroke}%
\pgfsetstrokeopacity{0.892845}%
\pgfsetdash{}{0pt}%
\pgfpathmoveto{\pgfqpoint{1.771568in}{2.633484in}}%
\pgfpathcurveto{\pgfqpoint{1.779805in}{2.633484in}}{\pgfqpoint{1.787705in}{2.636757in}}{\pgfqpoint{1.793529in}{2.642581in}}%
\pgfpathcurveto{\pgfqpoint{1.799353in}{2.648404in}}{\pgfqpoint{1.802625in}{2.656305in}}{\pgfqpoint{1.802625in}{2.664541in}}%
\pgfpathcurveto{\pgfqpoint{1.802625in}{2.672777in}}{\pgfqpoint{1.799353in}{2.680677in}}{\pgfqpoint{1.793529in}{2.686501in}}%
\pgfpathcurveto{\pgfqpoint{1.787705in}{2.692325in}}{\pgfqpoint{1.779805in}{2.695597in}}{\pgfqpoint{1.771568in}{2.695597in}}%
\pgfpathcurveto{\pgfqpoint{1.763332in}{2.695597in}}{\pgfqpoint{1.755432in}{2.692325in}}{\pgfqpoint{1.749608in}{2.686501in}}%
\pgfpathcurveto{\pgfqpoint{1.743784in}{2.680677in}}{\pgfqpoint{1.740512in}{2.672777in}}{\pgfqpoint{1.740512in}{2.664541in}}%
\pgfpathcurveto{\pgfqpoint{1.740512in}{2.656305in}}{\pgfqpoint{1.743784in}{2.648404in}}{\pgfqpoint{1.749608in}{2.642581in}}%
\pgfpathcurveto{\pgfqpoint{1.755432in}{2.636757in}}{\pgfqpoint{1.763332in}{2.633484in}}{\pgfqpoint{1.771568in}{2.633484in}}%
\pgfpathclose%
\pgfusepath{stroke,fill}%
\end{pgfscope}%
\begin{pgfscope}%
\pgfpathrectangle{\pgfqpoint{0.100000in}{0.212622in}}{\pgfqpoint{3.696000in}{3.696000in}}%
\pgfusepath{clip}%
\pgfsetbuttcap%
\pgfsetroundjoin%
\definecolor{currentfill}{rgb}{0.121569,0.466667,0.705882}%
\pgfsetfillcolor{currentfill}%
\pgfsetfillopacity{0.892899}%
\pgfsetlinewidth{1.003750pt}%
\definecolor{currentstroke}{rgb}{0.121569,0.466667,0.705882}%
\pgfsetstrokecolor{currentstroke}%
\pgfsetstrokeopacity{0.892899}%
\pgfsetdash{}{0pt}%
\pgfpathmoveto{\pgfqpoint{0.937722in}{2.092896in}}%
\pgfpathcurveto{\pgfqpoint{0.945958in}{2.092896in}}{\pgfqpoint{0.953858in}{2.096168in}}{\pgfqpoint{0.959682in}{2.101992in}}%
\pgfpathcurveto{\pgfqpoint{0.965506in}{2.107816in}}{\pgfqpoint{0.968778in}{2.115716in}}{\pgfqpoint{0.968778in}{2.123952in}}%
\pgfpathcurveto{\pgfqpoint{0.968778in}{2.132189in}}{\pgfqpoint{0.965506in}{2.140089in}}{\pgfqpoint{0.959682in}{2.145913in}}%
\pgfpathcurveto{\pgfqpoint{0.953858in}{2.151736in}}{\pgfqpoint{0.945958in}{2.155009in}}{\pgfqpoint{0.937722in}{2.155009in}}%
\pgfpathcurveto{\pgfqpoint{0.929485in}{2.155009in}}{\pgfqpoint{0.921585in}{2.151736in}}{\pgfqpoint{0.915761in}{2.145913in}}%
\pgfpathcurveto{\pgfqpoint{0.909937in}{2.140089in}}{\pgfqpoint{0.906665in}{2.132189in}}{\pgfqpoint{0.906665in}{2.123952in}}%
\pgfpathcurveto{\pgfqpoint{0.906665in}{2.115716in}}{\pgfqpoint{0.909937in}{2.107816in}}{\pgfqpoint{0.915761in}{2.101992in}}%
\pgfpathcurveto{\pgfqpoint{0.921585in}{2.096168in}}{\pgfqpoint{0.929485in}{2.092896in}}{\pgfqpoint{0.937722in}{2.092896in}}%
\pgfpathclose%
\pgfusepath{stroke,fill}%
\end{pgfscope}%
\begin{pgfscope}%
\pgfpathrectangle{\pgfqpoint{0.100000in}{0.212622in}}{\pgfqpoint{3.696000in}{3.696000in}}%
\pgfusepath{clip}%
\pgfsetbuttcap%
\pgfsetroundjoin%
\definecolor{currentfill}{rgb}{0.121569,0.466667,0.705882}%
\pgfsetfillcolor{currentfill}%
\pgfsetfillopacity{0.893126}%
\pgfsetlinewidth{1.003750pt}%
\definecolor{currentstroke}{rgb}{0.121569,0.466667,0.705882}%
\pgfsetstrokecolor{currentstroke}%
\pgfsetstrokeopacity{0.893126}%
\pgfsetdash{}{0pt}%
\pgfpathmoveto{\pgfqpoint{2.202474in}{2.507813in}}%
\pgfpathcurveto{\pgfqpoint{2.210711in}{2.507813in}}{\pgfqpoint{2.218611in}{2.511086in}}{\pgfqpoint{2.224435in}{2.516910in}}%
\pgfpathcurveto{\pgfqpoint{2.230258in}{2.522734in}}{\pgfqpoint{2.233531in}{2.530634in}}{\pgfqpoint{2.233531in}{2.538870in}}%
\pgfpathcurveto{\pgfqpoint{2.233531in}{2.547106in}}{\pgfqpoint{2.230258in}{2.555006in}}{\pgfqpoint{2.224435in}{2.560830in}}%
\pgfpathcurveto{\pgfqpoint{2.218611in}{2.566654in}}{\pgfqpoint{2.210711in}{2.569926in}}{\pgfqpoint{2.202474in}{2.569926in}}%
\pgfpathcurveto{\pgfqpoint{2.194238in}{2.569926in}}{\pgfqpoint{2.186338in}{2.566654in}}{\pgfqpoint{2.180514in}{2.560830in}}%
\pgfpathcurveto{\pgfqpoint{2.174690in}{2.555006in}}{\pgfqpoint{2.171418in}{2.547106in}}{\pgfqpoint{2.171418in}{2.538870in}}%
\pgfpathcurveto{\pgfqpoint{2.171418in}{2.530634in}}{\pgfqpoint{2.174690in}{2.522734in}}{\pgfqpoint{2.180514in}{2.516910in}}%
\pgfpathcurveto{\pgfqpoint{2.186338in}{2.511086in}}{\pgfqpoint{2.194238in}{2.507813in}}{\pgfqpoint{2.202474in}{2.507813in}}%
\pgfpathclose%
\pgfusepath{stroke,fill}%
\end{pgfscope}%
\begin{pgfscope}%
\pgfpathrectangle{\pgfqpoint{0.100000in}{0.212622in}}{\pgfqpoint{3.696000in}{3.696000in}}%
\pgfusepath{clip}%
\pgfsetbuttcap%
\pgfsetroundjoin%
\definecolor{currentfill}{rgb}{0.121569,0.466667,0.705882}%
\pgfsetfillcolor{currentfill}%
\pgfsetfillopacity{0.893560}%
\pgfsetlinewidth{1.003750pt}%
\definecolor{currentstroke}{rgb}{0.121569,0.466667,0.705882}%
\pgfsetstrokecolor{currentstroke}%
\pgfsetstrokeopacity{0.893560}%
\pgfsetdash{}{0pt}%
\pgfpathmoveto{\pgfqpoint{1.769937in}{2.631613in}}%
\pgfpathcurveto{\pgfqpoint{1.778174in}{2.631613in}}{\pgfqpoint{1.786074in}{2.634886in}}{\pgfqpoint{1.791898in}{2.640710in}}%
\pgfpathcurveto{\pgfqpoint{1.797722in}{2.646534in}}{\pgfqpoint{1.800994in}{2.654434in}}{\pgfqpoint{1.800994in}{2.662670in}}%
\pgfpathcurveto{\pgfqpoint{1.800994in}{2.670906in}}{\pgfqpoint{1.797722in}{2.678806in}}{\pgfqpoint{1.791898in}{2.684630in}}%
\pgfpathcurveto{\pgfqpoint{1.786074in}{2.690454in}}{\pgfqpoint{1.778174in}{2.693726in}}{\pgfqpoint{1.769937in}{2.693726in}}%
\pgfpathcurveto{\pgfqpoint{1.761701in}{2.693726in}}{\pgfqpoint{1.753801in}{2.690454in}}{\pgfqpoint{1.747977in}{2.684630in}}%
\pgfpathcurveto{\pgfqpoint{1.742153in}{2.678806in}}{\pgfqpoint{1.738881in}{2.670906in}}{\pgfqpoint{1.738881in}{2.662670in}}%
\pgfpathcurveto{\pgfqpoint{1.738881in}{2.654434in}}{\pgfqpoint{1.742153in}{2.646534in}}{\pgfqpoint{1.747977in}{2.640710in}}%
\pgfpathcurveto{\pgfqpoint{1.753801in}{2.634886in}}{\pgfqpoint{1.761701in}{2.631613in}}{\pgfqpoint{1.769937in}{2.631613in}}%
\pgfpathclose%
\pgfusepath{stroke,fill}%
\end{pgfscope}%
\begin{pgfscope}%
\pgfpathrectangle{\pgfqpoint{0.100000in}{0.212622in}}{\pgfqpoint{3.696000in}{3.696000in}}%
\pgfusepath{clip}%
\pgfsetbuttcap%
\pgfsetroundjoin%
\definecolor{currentfill}{rgb}{0.121569,0.466667,0.705882}%
\pgfsetfillcolor{currentfill}%
\pgfsetfillopacity{0.893839}%
\pgfsetlinewidth{1.003750pt}%
\definecolor{currentstroke}{rgb}{0.121569,0.466667,0.705882}%
\pgfsetstrokecolor{currentstroke}%
\pgfsetstrokeopacity{0.893839}%
\pgfsetdash{}{0pt}%
\pgfpathmoveto{\pgfqpoint{2.200738in}{2.506686in}}%
\pgfpathcurveto{\pgfqpoint{2.208974in}{2.506686in}}{\pgfqpoint{2.216874in}{2.509958in}}{\pgfqpoint{2.222698in}{2.515782in}}%
\pgfpathcurveto{\pgfqpoint{2.228522in}{2.521606in}}{\pgfqpoint{2.231794in}{2.529506in}}{\pgfqpoint{2.231794in}{2.537742in}}%
\pgfpathcurveto{\pgfqpoint{2.231794in}{2.545979in}}{\pgfqpoint{2.228522in}{2.553879in}}{\pgfqpoint{2.222698in}{2.559703in}}%
\pgfpathcurveto{\pgfqpoint{2.216874in}{2.565526in}}{\pgfqpoint{2.208974in}{2.568799in}}{\pgfqpoint{2.200738in}{2.568799in}}%
\pgfpathcurveto{\pgfqpoint{2.192502in}{2.568799in}}{\pgfqpoint{2.184602in}{2.565526in}}{\pgfqpoint{2.178778in}{2.559703in}}%
\pgfpathcurveto{\pgfqpoint{2.172954in}{2.553879in}}{\pgfqpoint{2.169681in}{2.545979in}}{\pgfqpoint{2.169681in}{2.537742in}}%
\pgfpathcurveto{\pgfqpoint{2.169681in}{2.529506in}}{\pgfqpoint{2.172954in}{2.521606in}}{\pgfqpoint{2.178778in}{2.515782in}}%
\pgfpathcurveto{\pgfqpoint{2.184602in}{2.509958in}}{\pgfqpoint{2.192502in}{2.506686in}}{\pgfqpoint{2.200738in}{2.506686in}}%
\pgfpathclose%
\pgfusepath{stroke,fill}%
\end{pgfscope}%
\begin{pgfscope}%
\pgfpathrectangle{\pgfqpoint{0.100000in}{0.212622in}}{\pgfqpoint{3.696000in}{3.696000in}}%
\pgfusepath{clip}%
\pgfsetbuttcap%
\pgfsetroundjoin%
\definecolor{currentfill}{rgb}{0.121569,0.466667,0.705882}%
\pgfsetfillcolor{currentfill}%
\pgfsetfillopacity{0.893990}%
\pgfsetlinewidth{1.003750pt}%
\definecolor{currentstroke}{rgb}{0.121569,0.466667,0.705882}%
\pgfsetstrokecolor{currentstroke}%
\pgfsetstrokeopacity{0.893990}%
\pgfsetdash{}{0pt}%
\pgfpathmoveto{\pgfqpoint{2.200387in}{2.506487in}}%
\pgfpathcurveto{\pgfqpoint{2.208623in}{2.506487in}}{\pgfqpoint{2.216523in}{2.509760in}}{\pgfqpoint{2.222347in}{2.515584in}}%
\pgfpathcurveto{\pgfqpoint{2.228171in}{2.521408in}}{\pgfqpoint{2.231444in}{2.529308in}}{\pgfqpoint{2.231444in}{2.537544in}}%
\pgfpathcurveto{\pgfqpoint{2.231444in}{2.545780in}}{\pgfqpoint{2.228171in}{2.553680in}}{\pgfqpoint{2.222347in}{2.559504in}}%
\pgfpathcurveto{\pgfqpoint{2.216523in}{2.565328in}}{\pgfqpoint{2.208623in}{2.568600in}}{\pgfqpoint{2.200387in}{2.568600in}}%
\pgfpathcurveto{\pgfqpoint{2.192151in}{2.568600in}}{\pgfqpoint{2.184251in}{2.565328in}}{\pgfqpoint{2.178427in}{2.559504in}}%
\pgfpathcurveto{\pgfqpoint{2.172603in}{2.553680in}}{\pgfqpoint{2.169331in}{2.545780in}}{\pgfqpoint{2.169331in}{2.537544in}}%
\pgfpathcurveto{\pgfqpoint{2.169331in}{2.529308in}}{\pgfqpoint{2.172603in}{2.521408in}}{\pgfqpoint{2.178427in}{2.515584in}}%
\pgfpathcurveto{\pgfqpoint{2.184251in}{2.509760in}}{\pgfqpoint{2.192151in}{2.506487in}}{\pgfqpoint{2.200387in}{2.506487in}}%
\pgfpathclose%
\pgfusepath{stroke,fill}%
\end{pgfscope}%
\begin{pgfscope}%
\pgfpathrectangle{\pgfqpoint{0.100000in}{0.212622in}}{\pgfqpoint{3.696000in}{3.696000in}}%
\pgfusepath{clip}%
\pgfsetbuttcap%
\pgfsetroundjoin%
\definecolor{currentfill}{rgb}{0.121569,0.466667,0.705882}%
\pgfsetfillcolor{currentfill}%
\pgfsetfillopacity{0.894259}%
\pgfsetlinewidth{1.003750pt}%
\definecolor{currentstroke}{rgb}{0.121569,0.466667,0.705882}%
\pgfsetstrokecolor{currentstroke}%
\pgfsetstrokeopacity{0.894259}%
\pgfsetdash{}{0pt}%
\pgfpathmoveto{\pgfqpoint{2.199757in}{2.506088in}}%
\pgfpathcurveto{\pgfqpoint{2.207993in}{2.506088in}}{\pgfqpoint{2.215893in}{2.509360in}}{\pgfqpoint{2.221717in}{2.515184in}}%
\pgfpathcurveto{\pgfqpoint{2.227541in}{2.521008in}}{\pgfqpoint{2.230813in}{2.528908in}}{\pgfqpoint{2.230813in}{2.537144in}}%
\pgfpathcurveto{\pgfqpoint{2.230813in}{2.545381in}}{\pgfqpoint{2.227541in}{2.553281in}}{\pgfqpoint{2.221717in}{2.559105in}}%
\pgfpathcurveto{\pgfqpoint{2.215893in}{2.564929in}}{\pgfqpoint{2.207993in}{2.568201in}}{\pgfqpoint{2.199757in}{2.568201in}}%
\pgfpathcurveto{\pgfqpoint{2.191520in}{2.568201in}}{\pgfqpoint{2.183620in}{2.564929in}}{\pgfqpoint{2.177796in}{2.559105in}}%
\pgfpathcurveto{\pgfqpoint{2.171972in}{2.553281in}}{\pgfqpoint{2.168700in}{2.545381in}}{\pgfqpoint{2.168700in}{2.537144in}}%
\pgfpathcurveto{\pgfqpoint{2.168700in}{2.528908in}}{\pgfqpoint{2.171972in}{2.521008in}}{\pgfqpoint{2.177796in}{2.515184in}}%
\pgfpathcurveto{\pgfqpoint{2.183620in}{2.509360in}}{\pgfqpoint{2.191520in}{2.506088in}}{\pgfqpoint{2.199757in}{2.506088in}}%
\pgfpathclose%
\pgfusepath{stroke,fill}%
\end{pgfscope}%
\begin{pgfscope}%
\pgfpathrectangle{\pgfqpoint{0.100000in}{0.212622in}}{\pgfqpoint{3.696000in}{3.696000in}}%
\pgfusepath{clip}%
\pgfsetbuttcap%
\pgfsetroundjoin%
\definecolor{currentfill}{rgb}{0.121569,0.466667,0.705882}%
\pgfsetfillcolor{currentfill}%
\pgfsetfillopacity{0.894378}%
\pgfsetlinewidth{1.003750pt}%
\definecolor{currentstroke}{rgb}{0.121569,0.466667,0.705882}%
\pgfsetstrokecolor{currentstroke}%
\pgfsetstrokeopacity{0.894378}%
\pgfsetdash{}{0pt}%
\pgfpathmoveto{\pgfqpoint{2.199462in}{2.505874in}}%
\pgfpathcurveto{\pgfqpoint{2.207698in}{2.505874in}}{\pgfqpoint{2.215599in}{2.509146in}}{\pgfqpoint{2.221422in}{2.514970in}}%
\pgfpathcurveto{\pgfqpoint{2.227246in}{2.520794in}}{\pgfqpoint{2.230519in}{2.528694in}}{\pgfqpoint{2.230519in}{2.536931in}}%
\pgfpathcurveto{\pgfqpoint{2.230519in}{2.545167in}}{\pgfqpoint{2.227246in}{2.553067in}}{\pgfqpoint{2.221422in}{2.558891in}}%
\pgfpathcurveto{\pgfqpoint{2.215599in}{2.564715in}}{\pgfqpoint{2.207698in}{2.567987in}}{\pgfqpoint{2.199462in}{2.567987in}}%
\pgfpathcurveto{\pgfqpoint{2.191226in}{2.567987in}}{\pgfqpoint{2.183326in}{2.564715in}}{\pgfqpoint{2.177502in}{2.558891in}}%
\pgfpathcurveto{\pgfqpoint{2.171678in}{2.553067in}}{\pgfqpoint{2.168406in}{2.545167in}}{\pgfqpoint{2.168406in}{2.536931in}}%
\pgfpathcurveto{\pgfqpoint{2.168406in}{2.528694in}}{\pgfqpoint{2.171678in}{2.520794in}}{\pgfqpoint{2.177502in}{2.514970in}}%
\pgfpathcurveto{\pgfqpoint{2.183326in}{2.509146in}}{\pgfqpoint{2.191226in}{2.505874in}}{\pgfqpoint{2.199462in}{2.505874in}}%
\pgfpathclose%
\pgfusepath{stroke,fill}%
\end{pgfscope}%
\begin{pgfscope}%
\pgfpathrectangle{\pgfqpoint{0.100000in}{0.212622in}}{\pgfqpoint{3.696000in}{3.696000in}}%
\pgfusepath{clip}%
\pgfsetbuttcap%
\pgfsetroundjoin%
\definecolor{currentfill}{rgb}{0.121569,0.466667,0.705882}%
\pgfsetfillcolor{currentfill}%
\pgfsetfillopacity{0.894503}%
\pgfsetlinewidth{1.003750pt}%
\definecolor{currentstroke}{rgb}{0.121569,0.466667,0.705882}%
\pgfsetstrokecolor{currentstroke}%
\pgfsetstrokeopacity{0.894503}%
\pgfsetdash{}{0pt}%
\pgfpathmoveto{\pgfqpoint{1.767626in}{2.628620in}}%
\pgfpathcurveto{\pgfqpoint{1.775862in}{2.628620in}}{\pgfqpoint{1.783762in}{2.631892in}}{\pgfqpoint{1.789586in}{2.637716in}}%
\pgfpathcurveto{\pgfqpoint{1.795410in}{2.643540in}}{\pgfqpoint{1.798683in}{2.651440in}}{\pgfqpoint{1.798683in}{2.659676in}}%
\pgfpathcurveto{\pgfqpoint{1.798683in}{2.667913in}}{\pgfqpoint{1.795410in}{2.675813in}}{\pgfqpoint{1.789586in}{2.681637in}}%
\pgfpathcurveto{\pgfqpoint{1.783762in}{2.687461in}}{\pgfqpoint{1.775862in}{2.690733in}}{\pgfqpoint{1.767626in}{2.690733in}}%
\pgfpathcurveto{\pgfqpoint{1.759390in}{2.690733in}}{\pgfqpoint{1.751490in}{2.687461in}}{\pgfqpoint{1.745666in}{2.681637in}}%
\pgfpathcurveto{\pgfqpoint{1.739842in}{2.675813in}}{\pgfqpoint{1.736570in}{2.667913in}}{\pgfqpoint{1.736570in}{2.659676in}}%
\pgfpathcurveto{\pgfqpoint{1.736570in}{2.651440in}}{\pgfqpoint{1.739842in}{2.643540in}}{\pgfqpoint{1.745666in}{2.637716in}}%
\pgfpathcurveto{\pgfqpoint{1.751490in}{2.631892in}}{\pgfqpoint{1.759390in}{2.628620in}}{\pgfqpoint{1.767626in}{2.628620in}}%
\pgfpathclose%
\pgfusepath{stroke,fill}%
\end{pgfscope}%
\begin{pgfscope}%
\pgfpathrectangle{\pgfqpoint{0.100000in}{0.212622in}}{\pgfqpoint{3.696000in}{3.696000in}}%
\pgfusepath{clip}%
\pgfsetbuttcap%
\pgfsetroundjoin%
\definecolor{currentfill}{rgb}{0.121569,0.466667,0.705882}%
\pgfsetfillcolor{currentfill}%
\pgfsetfillopacity{0.894595}%
\pgfsetlinewidth{1.003750pt}%
\definecolor{currentstroke}{rgb}{0.121569,0.466667,0.705882}%
\pgfsetstrokecolor{currentstroke}%
\pgfsetstrokeopacity{0.894595}%
\pgfsetdash{}{0pt}%
\pgfpathmoveto{\pgfqpoint{2.198919in}{2.505498in}}%
\pgfpathcurveto{\pgfqpoint{2.207155in}{2.505498in}}{\pgfqpoint{2.215055in}{2.508770in}}{\pgfqpoint{2.220879in}{2.514594in}}%
\pgfpathcurveto{\pgfqpoint{2.226703in}{2.520418in}}{\pgfqpoint{2.229976in}{2.528318in}}{\pgfqpoint{2.229976in}{2.536554in}}%
\pgfpathcurveto{\pgfqpoint{2.229976in}{2.544791in}}{\pgfqpoint{2.226703in}{2.552691in}}{\pgfqpoint{2.220879in}{2.558515in}}%
\pgfpathcurveto{\pgfqpoint{2.215055in}{2.564338in}}{\pgfqpoint{2.207155in}{2.567611in}}{\pgfqpoint{2.198919in}{2.567611in}}%
\pgfpathcurveto{\pgfqpoint{2.190683in}{2.567611in}}{\pgfqpoint{2.182783in}{2.564338in}}{\pgfqpoint{2.176959in}{2.558515in}}%
\pgfpathcurveto{\pgfqpoint{2.171135in}{2.552691in}}{\pgfqpoint{2.167863in}{2.544791in}}{\pgfqpoint{2.167863in}{2.536554in}}%
\pgfpathcurveto{\pgfqpoint{2.167863in}{2.528318in}}{\pgfqpoint{2.171135in}{2.520418in}}{\pgfqpoint{2.176959in}{2.514594in}}%
\pgfpathcurveto{\pgfqpoint{2.182783in}{2.508770in}}{\pgfqpoint{2.190683in}{2.505498in}}{\pgfqpoint{2.198919in}{2.505498in}}%
\pgfpathclose%
\pgfusepath{stroke,fill}%
\end{pgfscope}%
\begin{pgfscope}%
\pgfpathrectangle{\pgfqpoint{0.100000in}{0.212622in}}{\pgfqpoint{3.696000in}{3.696000in}}%
\pgfusepath{clip}%
\pgfsetbuttcap%
\pgfsetroundjoin%
\definecolor{currentfill}{rgb}{0.121569,0.466667,0.705882}%
\pgfsetfillcolor{currentfill}%
\pgfsetfillopacity{0.894693}%
\pgfsetlinewidth{1.003750pt}%
\definecolor{currentstroke}{rgb}{0.121569,0.466667,0.705882}%
\pgfsetstrokecolor{currentstroke}%
\pgfsetstrokeopacity{0.894693}%
\pgfsetdash{}{0pt}%
\pgfpathmoveto{\pgfqpoint{2.198685in}{2.505354in}}%
\pgfpathcurveto{\pgfqpoint{2.206921in}{2.505354in}}{\pgfqpoint{2.214821in}{2.508627in}}{\pgfqpoint{2.220645in}{2.514450in}}%
\pgfpathcurveto{\pgfqpoint{2.226469in}{2.520274in}}{\pgfqpoint{2.229741in}{2.528174in}}{\pgfqpoint{2.229741in}{2.536411in}}%
\pgfpathcurveto{\pgfqpoint{2.229741in}{2.544647in}}{\pgfqpoint{2.226469in}{2.552547in}}{\pgfqpoint{2.220645in}{2.558371in}}%
\pgfpathcurveto{\pgfqpoint{2.214821in}{2.564195in}}{\pgfqpoint{2.206921in}{2.567467in}}{\pgfqpoint{2.198685in}{2.567467in}}%
\pgfpathcurveto{\pgfqpoint{2.190448in}{2.567467in}}{\pgfqpoint{2.182548in}{2.564195in}}{\pgfqpoint{2.176724in}{2.558371in}}%
\pgfpathcurveto{\pgfqpoint{2.170900in}{2.552547in}}{\pgfqpoint{2.167628in}{2.544647in}}{\pgfqpoint{2.167628in}{2.536411in}}%
\pgfpathcurveto{\pgfqpoint{2.167628in}{2.528174in}}{\pgfqpoint{2.170900in}{2.520274in}}{\pgfqpoint{2.176724in}{2.514450in}}%
\pgfpathcurveto{\pgfqpoint{2.182548in}{2.508627in}}{\pgfqpoint{2.190448in}{2.505354in}}{\pgfqpoint{2.198685in}{2.505354in}}%
\pgfpathclose%
\pgfusepath{stroke,fill}%
\end{pgfscope}%
\begin{pgfscope}%
\pgfpathrectangle{\pgfqpoint{0.100000in}{0.212622in}}{\pgfqpoint{3.696000in}{3.696000in}}%
\pgfusepath{clip}%
\pgfsetbuttcap%
\pgfsetroundjoin%
\definecolor{currentfill}{rgb}{0.121569,0.466667,0.705882}%
\pgfsetfillcolor{currentfill}%
\pgfsetfillopacity{0.894876}%
\pgfsetlinewidth{1.003750pt}%
\definecolor{currentstroke}{rgb}{0.121569,0.466667,0.705882}%
\pgfsetstrokecolor{currentstroke}%
\pgfsetstrokeopacity{0.894876}%
\pgfsetdash{}{0pt}%
\pgfpathmoveto{\pgfqpoint{2.198258in}{2.505115in}}%
\pgfpathcurveto{\pgfqpoint{2.206494in}{2.505115in}}{\pgfqpoint{2.214394in}{2.508387in}}{\pgfqpoint{2.220218in}{2.514211in}}%
\pgfpathcurveto{\pgfqpoint{2.226042in}{2.520035in}}{\pgfqpoint{2.229315in}{2.527935in}}{\pgfqpoint{2.229315in}{2.536172in}}%
\pgfpathcurveto{\pgfqpoint{2.229315in}{2.544408in}}{\pgfqpoint{2.226042in}{2.552308in}}{\pgfqpoint{2.220218in}{2.558132in}}%
\pgfpathcurveto{\pgfqpoint{2.214394in}{2.563956in}}{\pgfqpoint{2.206494in}{2.567228in}}{\pgfqpoint{2.198258in}{2.567228in}}%
\pgfpathcurveto{\pgfqpoint{2.190022in}{2.567228in}}{\pgfqpoint{2.182122in}{2.563956in}}{\pgfqpoint{2.176298in}{2.558132in}}%
\pgfpathcurveto{\pgfqpoint{2.170474in}{2.552308in}}{\pgfqpoint{2.167202in}{2.544408in}}{\pgfqpoint{2.167202in}{2.536172in}}%
\pgfpathcurveto{\pgfqpoint{2.167202in}{2.527935in}}{\pgfqpoint{2.170474in}{2.520035in}}{\pgfqpoint{2.176298in}{2.514211in}}%
\pgfpathcurveto{\pgfqpoint{2.182122in}{2.508387in}}{\pgfqpoint{2.190022in}{2.505115in}}{\pgfqpoint{2.198258in}{2.505115in}}%
\pgfpathclose%
\pgfusepath{stroke,fill}%
\end{pgfscope}%
\begin{pgfscope}%
\pgfpathrectangle{\pgfqpoint{0.100000in}{0.212622in}}{\pgfqpoint{3.696000in}{3.696000in}}%
\pgfusepath{clip}%
\pgfsetbuttcap%
\pgfsetroundjoin%
\definecolor{currentfill}{rgb}{0.121569,0.466667,0.705882}%
\pgfsetfillcolor{currentfill}%
\pgfsetfillopacity{0.895052}%
\pgfsetlinewidth{1.003750pt}%
\definecolor{currentstroke}{rgb}{0.121569,0.466667,0.705882}%
\pgfsetstrokecolor{currentstroke}%
\pgfsetstrokeopacity{0.895052}%
\pgfsetdash{}{0pt}%
\pgfpathmoveto{\pgfqpoint{1.766370in}{2.627136in}}%
\pgfpathcurveto{\pgfqpoint{1.774607in}{2.627136in}}{\pgfqpoint{1.782507in}{2.630408in}}{\pgfqpoint{1.788331in}{2.636232in}}%
\pgfpathcurveto{\pgfqpoint{1.794155in}{2.642056in}}{\pgfqpoint{1.797427in}{2.649956in}}{\pgfqpoint{1.797427in}{2.658192in}}%
\pgfpathcurveto{\pgfqpoint{1.797427in}{2.666428in}}{\pgfqpoint{1.794155in}{2.674328in}}{\pgfqpoint{1.788331in}{2.680152in}}%
\pgfpathcurveto{\pgfqpoint{1.782507in}{2.685976in}}{\pgfqpoint{1.774607in}{2.689249in}}{\pgfqpoint{1.766370in}{2.689249in}}%
\pgfpathcurveto{\pgfqpoint{1.758134in}{2.689249in}}{\pgfqpoint{1.750234in}{2.685976in}}{\pgfqpoint{1.744410in}{2.680152in}}%
\pgfpathcurveto{\pgfqpoint{1.738586in}{2.674328in}}{\pgfqpoint{1.735314in}{2.666428in}}{\pgfqpoint{1.735314in}{2.658192in}}%
\pgfpathcurveto{\pgfqpoint{1.735314in}{2.649956in}}{\pgfqpoint{1.738586in}{2.642056in}}{\pgfqpoint{1.744410in}{2.636232in}}%
\pgfpathcurveto{\pgfqpoint{1.750234in}{2.630408in}}{\pgfqpoint{1.758134in}{2.627136in}}{\pgfqpoint{1.766370in}{2.627136in}}%
\pgfpathclose%
\pgfusepath{stroke,fill}%
\end{pgfscope}%
\begin{pgfscope}%
\pgfpathrectangle{\pgfqpoint{0.100000in}{0.212622in}}{\pgfqpoint{3.696000in}{3.696000in}}%
\pgfusepath{clip}%
\pgfsetbuttcap%
\pgfsetroundjoin%
\definecolor{currentfill}{rgb}{0.121569,0.466667,0.705882}%
\pgfsetfillcolor{currentfill}%
\pgfsetfillopacity{0.895187}%
\pgfsetlinewidth{1.003750pt}%
\definecolor{currentstroke}{rgb}{0.121569,0.466667,0.705882}%
\pgfsetstrokecolor{currentstroke}%
\pgfsetstrokeopacity{0.895187}%
\pgfsetdash{}{0pt}%
\pgfpathmoveto{\pgfqpoint{2.197483in}{2.504564in}}%
\pgfpathcurveto{\pgfqpoint{2.205720in}{2.504564in}}{\pgfqpoint{2.213620in}{2.507837in}}{\pgfqpoint{2.219443in}{2.513661in}}%
\pgfpathcurveto{\pgfqpoint{2.225267in}{2.519485in}}{\pgfqpoint{2.228540in}{2.527385in}}{\pgfqpoint{2.228540in}{2.535621in}}%
\pgfpathcurveto{\pgfqpoint{2.228540in}{2.543857in}}{\pgfqpoint{2.225267in}{2.551757in}}{\pgfqpoint{2.219443in}{2.557581in}}%
\pgfpathcurveto{\pgfqpoint{2.213620in}{2.563405in}}{\pgfqpoint{2.205720in}{2.566677in}}{\pgfqpoint{2.197483in}{2.566677in}}%
\pgfpathcurveto{\pgfqpoint{2.189247in}{2.566677in}}{\pgfqpoint{2.181347in}{2.563405in}}{\pgfqpoint{2.175523in}{2.557581in}}%
\pgfpathcurveto{\pgfqpoint{2.169699in}{2.551757in}}{\pgfqpoint{2.166427in}{2.543857in}}{\pgfqpoint{2.166427in}{2.535621in}}%
\pgfpathcurveto{\pgfqpoint{2.166427in}{2.527385in}}{\pgfqpoint{2.169699in}{2.519485in}}{\pgfqpoint{2.175523in}{2.513661in}}%
\pgfpathcurveto{\pgfqpoint{2.181347in}{2.507837in}}{\pgfqpoint{2.189247in}{2.504564in}}{\pgfqpoint{2.197483in}{2.504564in}}%
\pgfpathclose%
\pgfusepath{stroke,fill}%
\end{pgfscope}%
\begin{pgfscope}%
\pgfpathrectangle{\pgfqpoint{0.100000in}{0.212622in}}{\pgfqpoint{3.696000in}{3.696000in}}%
\pgfusepath{clip}%
\pgfsetbuttcap%
\pgfsetroundjoin%
\definecolor{currentfill}{rgb}{0.121569,0.466667,0.705882}%
\pgfsetfillcolor{currentfill}%
\pgfsetfillopacity{0.895349}%
\pgfsetlinewidth{1.003750pt}%
\definecolor{currentstroke}{rgb}{0.121569,0.466667,0.705882}%
\pgfsetstrokecolor{currentstroke}%
\pgfsetstrokeopacity{0.895349}%
\pgfsetdash{}{0pt}%
\pgfpathmoveto{\pgfqpoint{1.765665in}{2.626302in}}%
\pgfpathcurveto{\pgfqpoint{1.773901in}{2.626302in}}{\pgfqpoint{1.781801in}{2.629575in}}{\pgfqpoint{1.787625in}{2.635399in}}%
\pgfpathcurveto{\pgfqpoint{1.793449in}{2.641223in}}{\pgfqpoint{1.796722in}{2.649123in}}{\pgfqpoint{1.796722in}{2.657359in}}%
\pgfpathcurveto{\pgfqpoint{1.796722in}{2.665595in}}{\pgfqpoint{1.793449in}{2.673495in}}{\pgfqpoint{1.787625in}{2.679319in}}%
\pgfpathcurveto{\pgfqpoint{1.781801in}{2.685143in}}{\pgfqpoint{1.773901in}{2.688415in}}{\pgfqpoint{1.765665in}{2.688415in}}%
\pgfpathcurveto{\pgfqpoint{1.757429in}{2.688415in}}{\pgfqpoint{1.749529in}{2.685143in}}{\pgfqpoint{1.743705in}{2.679319in}}%
\pgfpathcurveto{\pgfqpoint{1.737881in}{2.673495in}}{\pgfqpoint{1.734609in}{2.665595in}}{\pgfqpoint{1.734609in}{2.657359in}}%
\pgfpathcurveto{\pgfqpoint{1.734609in}{2.649123in}}{\pgfqpoint{1.737881in}{2.641223in}}{\pgfqpoint{1.743705in}{2.635399in}}%
\pgfpathcurveto{\pgfqpoint{1.749529in}{2.629575in}}{\pgfqpoint{1.757429in}{2.626302in}}{\pgfqpoint{1.765665in}{2.626302in}}%
\pgfpathclose%
\pgfusepath{stroke,fill}%
\end{pgfscope}%
\begin{pgfscope}%
\pgfpathrectangle{\pgfqpoint{0.100000in}{0.212622in}}{\pgfqpoint{3.696000in}{3.696000in}}%
\pgfusepath{clip}%
\pgfsetbuttcap%
\pgfsetroundjoin%
\definecolor{currentfill}{rgb}{0.121569,0.466667,0.705882}%
\pgfsetfillcolor{currentfill}%
\pgfsetfillopacity{0.895356}%
\pgfsetlinewidth{1.003750pt}%
\definecolor{currentstroke}{rgb}{0.121569,0.466667,0.705882}%
\pgfsetstrokecolor{currentstroke}%
\pgfsetstrokeopacity{0.895356}%
\pgfsetdash{}{0pt}%
\pgfpathmoveto{\pgfqpoint{0.939864in}{2.087322in}}%
\pgfpathcurveto{\pgfqpoint{0.948100in}{2.087322in}}{\pgfqpoint{0.956000in}{2.090595in}}{\pgfqpoint{0.961824in}{2.096419in}}%
\pgfpathcurveto{\pgfqpoint{0.967648in}{2.102243in}}{\pgfqpoint{0.970921in}{2.110143in}}{\pgfqpoint{0.970921in}{2.118379in}}%
\pgfpathcurveto{\pgfqpoint{0.970921in}{2.126615in}}{\pgfqpoint{0.967648in}{2.134515in}}{\pgfqpoint{0.961824in}{2.140339in}}%
\pgfpathcurveto{\pgfqpoint{0.956000in}{2.146163in}}{\pgfqpoint{0.948100in}{2.149435in}}{\pgfqpoint{0.939864in}{2.149435in}}%
\pgfpathcurveto{\pgfqpoint{0.931628in}{2.149435in}}{\pgfqpoint{0.923728in}{2.146163in}}{\pgfqpoint{0.917904in}{2.140339in}}%
\pgfpathcurveto{\pgfqpoint{0.912080in}{2.134515in}}{\pgfqpoint{0.908808in}{2.126615in}}{\pgfqpoint{0.908808in}{2.118379in}}%
\pgfpathcurveto{\pgfqpoint{0.908808in}{2.110143in}}{\pgfqpoint{0.912080in}{2.102243in}}{\pgfqpoint{0.917904in}{2.096419in}}%
\pgfpathcurveto{\pgfqpoint{0.923728in}{2.090595in}}{\pgfqpoint{0.931628in}{2.087322in}}{\pgfqpoint{0.939864in}{2.087322in}}%
\pgfpathclose%
\pgfusepath{stroke,fill}%
\end{pgfscope}%
\begin{pgfscope}%
\pgfpathrectangle{\pgfqpoint{0.100000in}{0.212622in}}{\pgfqpoint{3.696000in}{3.696000in}}%
\pgfusepath{clip}%
\pgfsetbuttcap%
\pgfsetroundjoin%
\definecolor{currentfill}{rgb}{0.121569,0.466667,0.705882}%
\pgfsetfillcolor{currentfill}%
\pgfsetfillopacity{0.895394}%
\pgfsetlinewidth{1.003750pt}%
\definecolor{currentstroke}{rgb}{0.121569,0.466667,0.705882}%
\pgfsetstrokecolor{currentstroke}%
\pgfsetstrokeopacity{0.895394}%
\pgfsetdash{}{0pt}%
\pgfpathmoveto{\pgfqpoint{2.196950in}{2.504156in}}%
\pgfpathcurveto{\pgfqpoint{2.205186in}{2.504156in}}{\pgfqpoint{2.213086in}{2.507428in}}{\pgfqpoint{2.218910in}{2.513252in}}%
\pgfpathcurveto{\pgfqpoint{2.224734in}{2.519076in}}{\pgfqpoint{2.228006in}{2.526976in}}{\pgfqpoint{2.228006in}{2.535213in}}%
\pgfpathcurveto{\pgfqpoint{2.228006in}{2.543449in}}{\pgfqpoint{2.224734in}{2.551349in}}{\pgfqpoint{2.218910in}{2.557173in}}%
\pgfpathcurveto{\pgfqpoint{2.213086in}{2.562997in}}{\pgfqpoint{2.205186in}{2.566269in}}{\pgfqpoint{2.196950in}{2.566269in}}%
\pgfpathcurveto{\pgfqpoint{2.188714in}{2.566269in}}{\pgfqpoint{2.180813in}{2.562997in}}{\pgfqpoint{2.174990in}{2.557173in}}%
\pgfpathcurveto{\pgfqpoint{2.169166in}{2.551349in}}{\pgfqpoint{2.165893in}{2.543449in}}{\pgfqpoint{2.165893in}{2.535213in}}%
\pgfpathcurveto{\pgfqpoint{2.165893in}{2.526976in}}{\pgfqpoint{2.169166in}{2.519076in}}{\pgfqpoint{2.174990in}{2.513252in}}%
\pgfpathcurveto{\pgfqpoint{2.180813in}{2.507428in}}{\pgfqpoint{2.188714in}{2.504156in}}{\pgfqpoint{2.196950in}{2.504156in}}%
\pgfpathclose%
\pgfusepath{stroke,fill}%
\end{pgfscope}%
\begin{pgfscope}%
\pgfpathrectangle{\pgfqpoint{0.100000in}{0.212622in}}{\pgfqpoint{3.696000in}{3.696000in}}%
\pgfusepath{clip}%
\pgfsetbuttcap%
\pgfsetroundjoin%
\definecolor{currentfill}{rgb}{0.121569,0.466667,0.705882}%
\pgfsetfillcolor{currentfill}%
\pgfsetfillopacity{0.895462}%
\pgfsetlinewidth{1.003750pt}%
\definecolor{currentstroke}{rgb}{0.121569,0.466667,0.705882}%
\pgfsetstrokecolor{currentstroke}%
\pgfsetstrokeopacity{0.895462}%
\pgfsetdash{}{0pt}%
\pgfpathmoveto{\pgfqpoint{2.196786in}{2.504065in}}%
\pgfpathcurveto{\pgfqpoint{2.205022in}{2.504065in}}{\pgfqpoint{2.212922in}{2.507337in}}{\pgfqpoint{2.218746in}{2.513161in}}%
\pgfpathcurveto{\pgfqpoint{2.224570in}{2.518985in}}{\pgfqpoint{2.227842in}{2.526885in}}{\pgfqpoint{2.227842in}{2.535121in}}%
\pgfpathcurveto{\pgfqpoint{2.227842in}{2.543357in}}{\pgfqpoint{2.224570in}{2.551257in}}{\pgfqpoint{2.218746in}{2.557081in}}%
\pgfpathcurveto{\pgfqpoint{2.212922in}{2.562905in}}{\pgfqpoint{2.205022in}{2.566178in}}{\pgfqpoint{2.196786in}{2.566178in}}%
\pgfpathcurveto{\pgfqpoint{2.188549in}{2.566178in}}{\pgfqpoint{2.180649in}{2.562905in}}{\pgfqpoint{2.174825in}{2.557081in}}%
\pgfpathcurveto{\pgfqpoint{2.169001in}{2.551257in}}{\pgfqpoint{2.165729in}{2.543357in}}{\pgfqpoint{2.165729in}{2.535121in}}%
\pgfpathcurveto{\pgfqpoint{2.165729in}{2.526885in}}{\pgfqpoint{2.169001in}{2.518985in}}{\pgfqpoint{2.174825in}{2.513161in}}%
\pgfpathcurveto{\pgfqpoint{2.180649in}{2.507337in}}{\pgfqpoint{2.188549in}{2.504065in}}{\pgfqpoint{2.196786in}{2.504065in}}%
\pgfpathclose%
\pgfusepath{stroke,fill}%
\end{pgfscope}%
\begin{pgfscope}%
\pgfpathrectangle{\pgfqpoint{0.100000in}{0.212622in}}{\pgfqpoint{3.696000in}{3.696000in}}%
\pgfusepath{clip}%
\pgfsetbuttcap%
\pgfsetroundjoin%
\definecolor{currentfill}{rgb}{0.121569,0.466667,0.705882}%
\pgfsetfillcolor{currentfill}%
\pgfsetfillopacity{0.895592}%
\pgfsetlinewidth{1.003750pt}%
\definecolor{currentstroke}{rgb}{0.121569,0.466667,0.705882}%
\pgfsetstrokecolor{currentstroke}%
\pgfsetstrokeopacity{0.895592}%
\pgfsetdash{}{0pt}%
\pgfpathmoveto{\pgfqpoint{2.196487in}{2.503927in}}%
\pgfpathcurveto{\pgfqpoint{2.204723in}{2.503927in}}{\pgfqpoint{2.212623in}{2.507199in}}{\pgfqpoint{2.218447in}{2.513023in}}%
\pgfpathcurveto{\pgfqpoint{2.224271in}{2.518847in}}{\pgfqpoint{2.227544in}{2.526747in}}{\pgfqpoint{2.227544in}{2.534983in}}%
\pgfpathcurveto{\pgfqpoint{2.227544in}{2.543219in}}{\pgfqpoint{2.224271in}{2.551119in}}{\pgfqpoint{2.218447in}{2.556943in}}%
\pgfpathcurveto{\pgfqpoint{2.212623in}{2.562767in}}{\pgfqpoint{2.204723in}{2.566040in}}{\pgfqpoint{2.196487in}{2.566040in}}%
\pgfpathcurveto{\pgfqpoint{2.188251in}{2.566040in}}{\pgfqpoint{2.180351in}{2.562767in}}{\pgfqpoint{2.174527in}{2.556943in}}%
\pgfpathcurveto{\pgfqpoint{2.168703in}{2.551119in}}{\pgfqpoint{2.165431in}{2.543219in}}{\pgfqpoint{2.165431in}{2.534983in}}%
\pgfpathcurveto{\pgfqpoint{2.165431in}{2.526747in}}{\pgfqpoint{2.168703in}{2.518847in}}{\pgfqpoint{2.174527in}{2.513023in}}%
\pgfpathcurveto{\pgfqpoint{2.180351in}{2.507199in}}{\pgfqpoint{2.188251in}{2.503927in}}{\pgfqpoint{2.196487in}{2.503927in}}%
\pgfpathclose%
\pgfusepath{stroke,fill}%
\end{pgfscope}%
\begin{pgfscope}%
\pgfpathrectangle{\pgfqpoint{0.100000in}{0.212622in}}{\pgfqpoint{3.696000in}{3.696000in}}%
\pgfusepath{clip}%
\pgfsetbuttcap%
\pgfsetroundjoin%
\definecolor{currentfill}{rgb}{0.121569,0.466667,0.705882}%
\pgfsetfillcolor{currentfill}%
\pgfsetfillopacity{0.895834}%
\pgfsetlinewidth{1.003750pt}%
\definecolor{currentstroke}{rgb}{0.121569,0.466667,0.705882}%
\pgfsetstrokecolor{currentstroke}%
\pgfsetstrokeopacity{0.895834}%
\pgfsetdash{}{0pt}%
\pgfpathmoveto{\pgfqpoint{2.195954in}{2.503700in}}%
\pgfpathcurveto{\pgfqpoint{2.204190in}{2.503700in}}{\pgfqpoint{2.212090in}{2.506972in}}{\pgfqpoint{2.217914in}{2.512796in}}%
\pgfpathcurveto{\pgfqpoint{2.223738in}{2.518620in}}{\pgfqpoint{2.227010in}{2.526520in}}{\pgfqpoint{2.227010in}{2.534756in}}%
\pgfpathcurveto{\pgfqpoint{2.227010in}{2.542993in}}{\pgfqpoint{2.223738in}{2.550893in}}{\pgfqpoint{2.217914in}{2.556717in}}%
\pgfpathcurveto{\pgfqpoint{2.212090in}{2.562541in}}{\pgfqpoint{2.204190in}{2.565813in}}{\pgfqpoint{2.195954in}{2.565813in}}%
\pgfpathcurveto{\pgfqpoint{2.187718in}{2.565813in}}{\pgfqpoint{2.179818in}{2.562541in}}{\pgfqpoint{2.173994in}{2.556717in}}%
\pgfpathcurveto{\pgfqpoint{2.168170in}{2.550893in}}{\pgfqpoint{2.164897in}{2.542993in}}{\pgfqpoint{2.164897in}{2.534756in}}%
\pgfpathcurveto{\pgfqpoint{2.164897in}{2.526520in}}{\pgfqpoint{2.168170in}{2.518620in}}{\pgfqpoint{2.173994in}{2.512796in}}%
\pgfpathcurveto{\pgfqpoint{2.179818in}{2.506972in}}{\pgfqpoint{2.187718in}{2.503700in}}{\pgfqpoint{2.195954in}{2.503700in}}%
\pgfpathclose%
\pgfusepath{stroke,fill}%
\end{pgfscope}%
\begin{pgfscope}%
\pgfpathrectangle{\pgfqpoint{0.100000in}{0.212622in}}{\pgfqpoint{3.696000in}{3.696000in}}%
\pgfusepath{clip}%
\pgfsetbuttcap%
\pgfsetroundjoin%
\definecolor{currentfill}{rgb}{0.121569,0.466667,0.705882}%
\pgfsetfillcolor{currentfill}%
\pgfsetfillopacity{0.895898}%
\pgfsetlinewidth{1.003750pt}%
\definecolor{currentstroke}{rgb}{0.121569,0.466667,0.705882}%
\pgfsetstrokecolor{currentstroke}%
\pgfsetstrokeopacity{0.895898}%
\pgfsetdash{}{0pt}%
\pgfpathmoveto{\pgfqpoint{1.764314in}{2.624530in}}%
\pgfpathcurveto{\pgfqpoint{1.772550in}{2.624530in}}{\pgfqpoint{1.780450in}{2.627802in}}{\pgfqpoint{1.786274in}{2.633626in}}%
\pgfpathcurveto{\pgfqpoint{1.792098in}{2.639450in}}{\pgfqpoint{1.795370in}{2.647350in}}{\pgfqpoint{1.795370in}{2.655586in}}%
\pgfpathcurveto{\pgfqpoint{1.795370in}{2.663823in}}{\pgfqpoint{1.792098in}{2.671723in}}{\pgfqpoint{1.786274in}{2.677547in}}%
\pgfpathcurveto{\pgfqpoint{1.780450in}{2.683371in}}{\pgfqpoint{1.772550in}{2.686643in}}{\pgfqpoint{1.764314in}{2.686643in}}%
\pgfpathcurveto{\pgfqpoint{1.756077in}{2.686643in}}{\pgfqpoint{1.748177in}{2.683371in}}{\pgfqpoint{1.742353in}{2.677547in}}%
\pgfpathcurveto{\pgfqpoint{1.736529in}{2.671723in}}{\pgfqpoint{1.733257in}{2.663823in}}{\pgfqpoint{1.733257in}{2.655586in}}%
\pgfpathcurveto{\pgfqpoint{1.733257in}{2.647350in}}{\pgfqpoint{1.736529in}{2.639450in}}{\pgfqpoint{1.742353in}{2.633626in}}%
\pgfpathcurveto{\pgfqpoint{1.748177in}{2.627802in}}{\pgfqpoint{1.756077in}{2.624530in}}{\pgfqpoint{1.764314in}{2.624530in}}%
\pgfpathclose%
\pgfusepath{stroke,fill}%
\end{pgfscope}%
\begin{pgfscope}%
\pgfpathrectangle{\pgfqpoint{0.100000in}{0.212622in}}{\pgfqpoint{3.696000in}{3.696000in}}%
\pgfusepath{clip}%
\pgfsetbuttcap%
\pgfsetroundjoin%
\definecolor{currentfill}{rgb}{0.121569,0.466667,0.705882}%
\pgfsetfillcolor{currentfill}%
\pgfsetfillopacity{0.896253}%
\pgfsetlinewidth{1.003750pt}%
\definecolor{currentstroke}{rgb}{0.121569,0.466667,0.705882}%
\pgfsetstrokecolor{currentstroke}%
\pgfsetstrokeopacity{0.896253}%
\pgfsetdash{}{0pt}%
\pgfpathmoveto{\pgfqpoint{2.194974in}{2.503176in}}%
\pgfpathcurveto{\pgfqpoint{2.203211in}{2.503176in}}{\pgfqpoint{2.211111in}{2.506448in}}{\pgfqpoint{2.216935in}{2.512272in}}%
\pgfpathcurveto{\pgfqpoint{2.222759in}{2.518096in}}{\pgfqpoint{2.226031in}{2.525996in}}{\pgfqpoint{2.226031in}{2.534232in}}%
\pgfpathcurveto{\pgfqpoint{2.226031in}{2.542469in}}{\pgfqpoint{2.222759in}{2.550369in}}{\pgfqpoint{2.216935in}{2.556193in}}%
\pgfpathcurveto{\pgfqpoint{2.211111in}{2.562017in}}{\pgfqpoint{2.203211in}{2.565289in}}{\pgfqpoint{2.194974in}{2.565289in}}%
\pgfpathcurveto{\pgfqpoint{2.186738in}{2.565289in}}{\pgfqpoint{2.178838in}{2.562017in}}{\pgfqpoint{2.173014in}{2.556193in}}%
\pgfpathcurveto{\pgfqpoint{2.167190in}{2.550369in}}{\pgfqpoint{2.163918in}{2.542469in}}{\pgfqpoint{2.163918in}{2.534232in}}%
\pgfpathcurveto{\pgfqpoint{2.163918in}{2.525996in}}{\pgfqpoint{2.167190in}{2.518096in}}{\pgfqpoint{2.173014in}{2.512272in}}%
\pgfpathcurveto{\pgfqpoint{2.178838in}{2.506448in}}{\pgfqpoint{2.186738in}{2.503176in}}{\pgfqpoint{2.194974in}{2.503176in}}%
\pgfpathclose%
\pgfusepath{stroke,fill}%
\end{pgfscope}%
\begin{pgfscope}%
\pgfpathrectangle{\pgfqpoint{0.100000in}{0.212622in}}{\pgfqpoint{3.696000in}{3.696000in}}%
\pgfusepath{clip}%
\pgfsetbuttcap%
\pgfsetroundjoin%
\definecolor{currentfill}{rgb}{0.121569,0.466667,0.705882}%
\pgfsetfillcolor{currentfill}%
\pgfsetfillopacity{0.896553}%
\pgfsetlinewidth{1.003750pt}%
\definecolor{currentstroke}{rgb}{0.121569,0.466667,0.705882}%
\pgfsetstrokecolor{currentstroke}%
\pgfsetstrokeopacity{0.896553}%
\pgfsetdash{}{0pt}%
\pgfpathmoveto{\pgfqpoint{1.762790in}{2.622543in}}%
\pgfpathcurveto{\pgfqpoint{1.771026in}{2.622543in}}{\pgfqpoint{1.778926in}{2.625816in}}{\pgfqpoint{1.784750in}{2.631640in}}%
\pgfpathcurveto{\pgfqpoint{1.790574in}{2.637464in}}{\pgfqpoint{1.793846in}{2.645364in}}{\pgfqpoint{1.793846in}{2.653600in}}%
\pgfpathcurveto{\pgfqpoint{1.793846in}{2.661836in}}{\pgfqpoint{1.790574in}{2.669736in}}{\pgfqpoint{1.784750in}{2.675560in}}%
\pgfpathcurveto{\pgfqpoint{1.778926in}{2.681384in}}{\pgfqpoint{1.771026in}{2.684656in}}{\pgfqpoint{1.762790in}{2.684656in}}%
\pgfpathcurveto{\pgfqpoint{1.754554in}{2.684656in}}{\pgfqpoint{1.746654in}{2.681384in}}{\pgfqpoint{1.740830in}{2.675560in}}%
\pgfpathcurveto{\pgfqpoint{1.735006in}{2.669736in}}{\pgfqpoint{1.731733in}{2.661836in}}{\pgfqpoint{1.731733in}{2.653600in}}%
\pgfpathcurveto{\pgfqpoint{1.731733in}{2.645364in}}{\pgfqpoint{1.735006in}{2.637464in}}{\pgfqpoint{1.740830in}{2.631640in}}%
\pgfpathcurveto{\pgfqpoint{1.746654in}{2.625816in}}{\pgfqpoint{1.754554in}{2.622543in}}{\pgfqpoint{1.762790in}{2.622543in}}%
\pgfpathclose%
\pgfusepath{stroke,fill}%
\end{pgfscope}%
\begin{pgfscope}%
\pgfpathrectangle{\pgfqpoint{0.100000in}{0.212622in}}{\pgfqpoint{3.696000in}{3.696000in}}%
\pgfusepath{clip}%
\pgfsetbuttcap%
\pgfsetroundjoin%
\definecolor{currentfill}{rgb}{0.121569,0.466667,0.705882}%
\pgfsetfillcolor{currentfill}%
\pgfsetfillopacity{0.897001}%
\pgfsetlinewidth{1.003750pt}%
\definecolor{currentstroke}{rgb}{0.121569,0.466667,0.705882}%
\pgfsetstrokecolor{currentstroke}%
\pgfsetstrokeopacity{0.897001}%
\pgfsetdash{}{0pt}%
\pgfpathmoveto{\pgfqpoint{2.193192in}{2.502149in}}%
\pgfpathcurveto{\pgfqpoint{2.201428in}{2.502149in}}{\pgfqpoint{2.209328in}{2.505421in}}{\pgfqpoint{2.215152in}{2.511245in}}%
\pgfpathcurveto{\pgfqpoint{2.220976in}{2.517069in}}{\pgfqpoint{2.224248in}{2.524969in}}{\pgfqpoint{2.224248in}{2.533205in}}%
\pgfpathcurveto{\pgfqpoint{2.224248in}{2.541441in}}{\pgfqpoint{2.220976in}{2.549341in}}{\pgfqpoint{2.215152in}{2.555165in}}%
\pgfpathcurveto{\pgfqpoint{2.209328in}{2.560989in}}{\pgfqpoint{2.201428in}{2.564262in}}{\pgfqpoint{2.193192in}{2.564262in}}%
\pgfpathcurveto{\pgfqpoint{2.184956in}{2.564262in}}{\pgfqpoint{2.177056in}{2.560989in}}{\pgfqpoint{2.171232in}{2.555165in}}%
\pgfpathcurveto{\pgfqpoint{2.165408in}{2.549341in}}{\pgfqpoint{2.162135in}{2.541441in}}{\pgfqpoint{2.162135in}{2.533205in}}%
\pgfpathcurveto{\pgfqpoint{2.162135in}{2.524969in}}{\pgfqpoint{2.165408in}{2.517069in}}{\pgfqpoint{2.171232in}{2.511245in}}%
\pgfpathcurveto{\pgfqpoint{2.177056in}{2.505421in}}{\pgfqpoint{2.184956in}{2.502149in}}{\pgfqpoint{2.193192in}{2.502149in}}%
\pgfpathclose%
\pgfusepath{stroke,fill}%
\end{pgfscope}%
\begin{pgfscope}%
\pgfpathrectangle{\pgfqpoint{0.100000in}{0.212622in}}{\pgfqpoint{3.696000in}{3.696000in}}%
\pgfusepath{clip}%
\pgfsetbuttcap%
\pgfsetroundjoin%
\definecolor{currentfill}{rgb}{0.121569,0.466667,0.705882}%
\pgfsetfillcolor{currentfill}%
\pgfsetfillopacity{0.897450}%
\pgfsetlinewidth{1.003750pt}%
\definecolor{currentstroke}{rgb}{0.121569,0.466667,0.705882}%
\pgfsetstrokecolor{currentstroke}%
\pgfsetstrokeopacity{0.897450}%
\pgfsetdash{}{0pt}%
\pgfpathmoveto{\pgfqpoint{1.760893in}{2.620531in}}%
\pgfpathcurveto{\pgfqpoint{1.769129in}{2.620531in}}{\pgfqpoint{1.777030in}{2.623803in}}{\pgfqpoint{1.782853in}{2.629627in}}%
\pgfpathcurveto{\pgfqpoint{1.788677in}{2.635451in}}{\pgfqpoint{1.791950in}{2.643351in}}{\pgfqpoint{1.791950in}{2.651587in}}%
\pgfpathcurveto{\pgfqpoint{1.791950in}{2.659824in}}{\pgfqpoint{1.788677in}{2.667724in}}{\pgfqpoint{1.782853in}{2.673548in}}%
\pgfpathcurveto{\pgfqpoint{1.777030in}{2.679372in}}{\pgfqpoint{1.769129in}{2.682644in}}{\pgfqpoint{1.760893in}{2.682644in}}%
\pgfpathcurveto{\pgfqpoint{1.752657in}{2.682644in}}{\pgfqpoint{1.744757in}{2.679372in}}{\pgfqpoint{1.738933in}{2.673548in}}%
\pgfpathcurveto{\pgfqpoint{1.733109in}{2.667724in}}{\pgfqpoint{1.729837in}{2.659824in}}{\pgfqpoint{1.729837in}{2.651587in}}%
\pgfpathcurveto{\pgfqpoint{1.729837in}{2.643351in}}{\pgfqpoint{1.733109in}{2.635451in}}{\pgfqpoint{1.738933in}{2.629627in}}%
\pgfpathcurveto{\pgfqpoint{1.744757in}{2.623803in}}{\pgfqpoint{1.752657in}{2.620531in}}{\pgfqpoint{1.760893in}{2.620531in}}%
\pgfpathclose%
\pgfusepath{stroke,fill}%
\end{pgfscope}%
\begin{pgfscope}%
\pgfpathrectangle{\pgfqpoint{0.100000in}{0.212622in}}{\pgfqpoint{3.696000in}{3.696000in}}%
\pgfusepath{clip}%
\pgfsetbuttcap%
\pgfsetroundjoin%
\definecolor{currentfill}{rgb}{0.121569,0.466667,0.705882}%
\pgfsetfillcolor{currentfill}%
\pgfsetfillopacity{0.897644}%
\pgfsetlinewidth{1.003750pt}%
\definecolor{currentstroke}{rgb}{0.121569,0.466667,0.705882}%
\pgfsetstrokecolor{currentstroke}%
\pgfsetstrokeopacity{0.897644}%
\pgfsetdash{}{0pt}%
\pgfpathmoveto{\pgfqpoint{2.191635in}{2.501203in}}%
\pgfpathcurveto{\pgfqpoint{2.199871in}{2.501203in}}{\pgfqpoint{2.207771in}{2.504475in}}{\pgfqpoint{2.213595in}{2.510299in}}%
\pgfpathcurveto{\pgfqpoint{2.219419in}{2.516123in}}{\pgfqpoint{2.222691in}{2.524023in}}{\pgfqpoint{2.222691in}{2.532259in}}%
\pgfpathcurveto{\pgfqpoint{2.222691in}{2.540496in}}{\pgfqpoint{2.219419in}{2.548396in}}{\pgfqpoint{2.213595in}{2.554220in}}%
\pgfpathcurveto{\pgfqpoint{2.207771in}{2.560044in}}{\pgfqpoint{2.199871in}{2.563316in}}{\pgfqpoint{2.191635in}{2.563316in}}%
\pgfpathcurveto{\pgfqpoint{2.183398in}{2.563316in}}{\pgfqpoint{2.175498in}{2.560044in}}{\pgfqpoint{2.169674in}{2.554220in}}%
\pgfpathcurveto{\pgfqpoint{2.163851in}{2.548396in}}{\pgfqpoint{2.160578in}{2.540496in}}{\pgfqpoint{2.160578in}{2.532259in}}%
\pgfpathcurveto{\pgfqpoint{2.160578in}{2.524023in}}{\pgfqpoint{2.163851in}{2.516123in}}{\pgfqpoint{2.169674in}{2.510299in}}%
\pgfpathcurveto{\pgfqpoint{2.175498in}{2.504475in}}{\pgfqpoint{2.183398in}{2.501203in}}{\pgfqpoint{2.191635in}{2.501203in}}%
\pgfpathclose%
\pgfusepath{stroke,fill}%
\end{pgfscope}%
\begin{pgfscope}%
\pgfpathrectangle{\pgfqpoint{0.100000in}{0.212622in}}{\pgfqpoint{3.696000in}{3.696000in}}%
\pgfusepath{clip}%
\pgfsetbuttcap%
\pgfsetroundjoin%
\definecolor{currentfill}{rgb}{0.121569,0.466667,0.705882}%
\pgfsetfillcolor{currentfill}%
\pgfsetfillopacity{0.897922}%
\pgfsetlinewidth{1.003750pt}%
\definecolor{currentstroke}{rgb}{0.121569,0.466667,0.705882}%
\pgfsetstrokecolor{currentstroke}%
\pgfsetstrokeopacity{0.897922}%
\pgfsetdash{}{0pt}%
\pgfpathmoveto{\pgfqpoint{1.759840in}{2.619309in}}%
\pgfpathcurveto{\pgfqpoint{1.768076in}{2.619309in}}{\pgfqpoint{1.775976in}{2.622581in}}{\pgfqpoint{1.781800in}{2.628405in}}%
\pgfpathcurveto{\pgfqpoint{1.787624in}{2.634229in}}{\pgfqpoint{1.790896in}{2.642129in}}{\pgfqpoint{1.790896in}{2.650365in}}%
\pgfpathcurveto{\pgfqpoint{1.790896in}{2.658601in}}{\pgfqpoint{1.787624in}{2.666502in}}{\pgfqpoint{1.781800in}{2.672325in}}%
\pgfpathcurveto{\pgfqpoint{1.775976in}{2.678149in}}{\pgfqpoint{1.768076in}{2.681422in}}{\pgfqpoint{1.759840in}{2.681422in}}%
\pgfpathcurveto{\pgfqpoint{1.751603in}{2.681422in}}{\pgfqpoint{1.743703in}{2.678149in}}{\pgfqpoint{1.737879in}{2.672325in}}%
\pgfpathcurveto{\pgfqpoint{1.732056in}{2.666502in}}{\pgfqpoint{1.728783in}{2.658601in}}{\pgfqpoint{1.728783in}{2.650365in}}%
\pgfpathcurveto{\pgfqpoint{1.728783in}{2.642129in}}{\pgfqpoint{1.732056in}{2.634229in}}{\pgfqpoint{1.737879in}{2.628405in}}%
\pgfpathcurveto{\pgfqpoint{1.743703in}{2.622581in}}{\pgfqpoint{1.751603in}{2.619309in}}{\pgfqpoint{1.759840in}{2.619309in}}%
\pgfpathclose%
\pgfusepath{stroke,fill}%
\end{pgfscope}%
\begin{pgfscope}%
\pgfpathrectangle{\pgfqpoint{0.100000in}{0.212622in}}{\pgfqpoint{3.696000in}{3.696000in}}%
\pgfusepath{clip}%
\pgfsetbuttcap%
\pgfsetroundjoin%
\definecolor{currentfill}{rgb}{0.121569,0.466667,0.705882}%
\pgfsetfillcolor{currentfill}%
\pgfsetfillopacity{0.898022}%
\pgfsetlinewidth{1.003750pt}%
\definecolor{currentstroke}{rgb}{0.121569,0.466667,0.705882}%
\pgfsetstrokecolor{currentstroke}%
\pgfsetstrokeopacity{0.898022}%
\pgfsetdash{}{0pt}%
\pgfpathmoveto{\pgfqpoint{0.942960in}{2.080807in}}%
\pgfpathcurveto{\pgfqpoint{0.951196in}{2.080807in}}{\pgfqpoint{0.959096in}{2.084080in}}{\pgfqpoint{0.964920in}{2.089904in}}%
\pgfpathcurveto{\pgfqpoint{0.970744in}{2.095728in}}{\pgfqpoint{0.974016in}{2.103628in}}{\pgfqpoint{0.974016in}{2.111864in}}%
\pgfpathcurveto{\pgfqpoint{0.974016in}{2.120100in}}{\pgfqpoint{0.970744in}{2.128000in}}{\pgfqpoint{0.964920in}{2.133824in}}%
\pgfpathcurveto{\pgfqpoint{0.959096in}{2.139648in}}{\pgfqpoint{0.951196in}{2.142920in}}{\pgfqpoint{0.942960in}{2.142920in}}%
\pgfpathcurveto{\pgfqpoint{0.934723in}{2.142920in}}{\pgfqpoint{0.926823in}{2.139648in}}{\pgfqpoint{0.921000in}{2.133824in}}%
\pgfpathcurveto{\pgfqpoint{0.915176in}{2.128000in}}{\pgfqpoint{0.911903in}{2.120100in}}{\pgfqpoint{0.911903in}{2.111864in}}%
\pgfpathcurveto{\pgfqpoint{0.911903in}{2.103628in}}{\pgfqpoint{0.915176in}{2.095728in}}{\pgfqpoint{0.921000in}{2.089904in}}%
\pgfpathcurveto{\pgfqpoint{0.926823in}{2.084080in}}{\pgfqpoint{0.934723in}{2.080807in}}{\pgfqpoint{0.942960in}{2.080807in}}%
\pgfpathclose%
\pgfusepath{stroke,fill}%
\end{pgfscope}%
\begin{pgfscope}%
\pgfpathrectangle{\pgfqpoint{0.100000in}{0.212622in}}{\pgfqpoint{3.696000in}{3.696000in}}%
\pgfusepath{clip}%
\pgfsetbuttcap%
\pgfsetroundjoin%
\definecolor{currentfill}{rgb}{0.121569,0.466667,0.705882}%
\pgfsetfillcolor{currentfill}%
\pgfsetfillopacity{0.898037}%
\pgfsetlinewidth{1.003750pt}%
\definecolor{currentstroke}{rgb}{0.121569,0.466667,0.705882}%
\pgfsetstrokecolor{currentstroke}%
\pgfsetstrokeopacity{0.898037}%
\pgfsetdash{}{0pt}%
\pgfpathmoveto{\pgfqpoint{2.683299in}{1.312212in}}%
\pgfpathcurveto{\pgfqpoint{2.691536in}{1.312212in}}{\pgfqpoint{2.699436in}{1.315485in}}{\pgfqpoint{2.705260in}{1.321308in}}%
\pgfpathcurveto{\pgfqpoint{2.711084in}{1.327132in}}{\pgfqpoint{2.714356in}{1.335032in}}{\pgfqpoint{2.714356in}{1.343269in}}%
\pgfpathcurveto{\pgfqpoint{2.714356in}{1.351505in}}{\pgfqpoint{2.711084in}{1.359405in}}{\pgfqpoint{2.705260in}{1.365229in}}%
\pgfpathcurveto{\pgfqpoint{2.699436in}{1.371053in}}{\pgfqpoint{2.691536in}{1.374325in}}{\pgfqpoint{2.683299in}{1.374325in}}%
\pgfpathcurveto{\pgfqpoint{2.675063in}{1.374325in}}{\pgfqpoint{2.667163in}{1.371053in}}{\pgfqpoint{2.661339in}{1.365229in}}%
\pgfpathcurveto{\pgfqpoint{2.655515in}{1.359405in}}{\pgfqpoint{2.652243in}{1.351505in}}{\pgfqpoint{2.652243in}{1.343269in}}%
\pgfpathcurveto{\pgfqpoint{2.652243in}{1.335032in}}{\pgfqpoint{2.655515in}{1.327132in}}{\pgfqpoint{2.661339in}{1.321308in}}%
\pgfpathcurveto{\pgfqpoint{2.667163in}{1.315485in}}{\pgfqpoint{2.675063in}{1.312212in}}{\pgfqpoint{2.683299in}{1.312212in}}%
\pgfpathclose%
\pgfusepath{stroke,fill}%
\end{pgfscope}%
\begin{pgfscope}%
\pgfpathrectangle{\pgfqpoint{0.100000in}{0.212622in}}{\pgfqpoint{3.696000in}{3.696000in}}%
\pgfusepath{clip}%
\pgfsetbuttcap%
\pgfsetroundjoin%
\definecolor{currentfill}{rgb}{0.121569,0.466667,0.705882}%
\pgfsetfillcolor{currentfill}%
\pgfsetfillopacity{0.898188}%
\pgfsetlinewidth{1.003750pt}%
\definecolor{currentstroke}{rgb}{0.121569,0.466667,0.705882}%
\pgfsetstrokecolor{currentstroke}%
\pgfsetstrokeopacity{0.898188}%
\pgfsetdash{}{0pt}%
\pgfpathmoveto{\pgfqpoint{1.759274in}{2.618660in}}%
\pgfpathcurveto{\pgfqpoint{1.767510in}{2.618660in}}{\pgfqpoint{1.775410in}{2.621932in}}{\pgfqpoint{1.781234in}{2.627756in}}%
\pgfpathcurveto{\pgfqpoint{1.787058in}{2.633580in}}{\pgfqpoint{1.790331in}{2.641480in}}{\pgfqpoint{1.790331in}{2.649716in}}%
\pgfpathcurveto{\pgfqpoint{1.790331in}{2.657952in}}{\pgfqpoint{1.787058in}{2.665852in}}{\pgfqpoint{1.781234in}{2.671676in}}%
\pgfpathcurveto{\pgfqpoint{1.775410in}{2.677500in}}{\pgfqpoint{1.767510in}{2.680773in}}{\pgfqpoint{1.759274in}{2.680773in}}%
\pgfpathcurveto{\pgfqpoint{1.751038in}{2.680773in}}{\pgfqpoint{1.743138in}{2.677500in}}{\pgfqpoint{1.737314in}{2.671676in}}%
\pgfpathcurveto{\pgfqpoint{1.731490in}{2.665852in}}{\pgfqpoint{1.728218in}{2.657952in}}{\pgfqpoint{1.728218in}{2.649716in}}%
\pgfpathcurveto{\pgfqpoint{1.728218in}{2.641480in}}{\pgfqpoint{1.731490in}{2.633580in}}{\pgfqpoint{1.737314in}{2.627756in}}%
\pgfpathcurveto{\pgfqpoint{1.743138in}{2.621932in}}{\pgfqpoint{1.751038in}{2.618660in}}{\pgfqpoint{1.759274in}{2.618660in}}%
\pgfpathclose%
\pgfusepath{stroke,fill}%
\end{pgfscope}%
\begin{pgfscope}%
\pgfpathrectangle{\pgfqpoint{0.100000in}{0.212622in}}{\pgfqpoint{3.696000in}{3.696000in}}%
\pgfusepath{clip}%
\pgfsetbuttcap%
\pgfsetroundjoin%
\definecolor{currentfill}{rgb}{0.121569,0.466667,0.705882}%
\pgfsetfillcolor{currentfill}%
\pgfsetfillopacity{0.898205}%
\pgfsetlinewidth{1.003750pt}%
\definecolor{currentstroke}{rgb}{0.121569,0.466667,0.705882}%
\pgfsetstrokecolor{currentstroke}%
\pgfsetstrokeopacity{0.898205}%
\pgfsetdash{}{0pt}%
\pgfpathmoveto{\pgfqpoint{2.190363in}{2.500390in}}%
\pgfpathcurveto{\pgfqpoint{2.198599in}{2.500390in}}{\pgfqpoint{2.206499in}{2.503662in}}{\pgfqpoint{2.212323in}{2.509486in}}%
\pgfpathcurveto{\pgfqpoint{2.218147in}{2.515310in}}{\pgfqpoint{2.221419in}{2.523210in}}{\pgfqpoint{2.221419in}{2.531446in}}%
\pgfpathcurveto{\pgfqpoint{2.221419in}{2.539682in}}{\pgfqpoint{2.218147in}{2.547582in}}{\pgfqpoint{2.212323in}{2.553406in}}%
\pgfpathcurveto{\pgfqpoint{2.206499in}{2.559230in}}{\pgfqpoint{2.198599in}{2.562503in}}{\pgfqpoint{2.190363in}{2.562503in}}%
\pgfpathcurveto{\pgfqpoint{2.182127in}{2.562503in}}{\pgfqpoint{2.174227in}{2.559230in}}{\pgfqpoint{2.168403in}{2.553406in}}%
\pgfpathcurveto{\pgfqpoint{2.162579in}{2.547582in}}{\pgfqpoint{2.159306in}{2.539682in}}{\pgfqpoint{2.159306in}{2.531446in}}%
\pgfpathcurveto{\pgfqpoint{2.159306in}{2.523210in}}{\pgfqpoint{2.162579in}{2.515310in}}{\pgfqpoint{2.168403in}{2.509486in}}%
\pgfpathcurveto{\pgfqpoint{2.174227in}{2.503662in}}{\pgfqpoint{2.182127in}{2.500390in}}{\pgfqpoint{2.190363in}{2.500390in}}%
\pgfpathclose%
\pgfusepath{stroke,fill}%
\end{pgfscope}%
\begin{pgfscope}%
\pgfpathrectangle{\pgfqpoint{0.100000in}{0.212622in}}{\pgfqpoint{3.696000in}{3.696000in}}%
\pgfusepath{clip}%
\pgfsetbuttcap%
\pgfsetroundjoin%
\definecolor{currentfill}{rgb}{0.121569,0.466667,0.705882}%
\pgfsetfillcolor{currentfill}%
\pgfsetfillopacity{0.898492}%
\pgfsetlinewidth{1.003750pt}%
\definecolor{currentstroke}{rgb}{0.121569,0.466667,0.705882}%
\pgfsetstrokecolor{currentstroke}%
\pgfsetstrokeopacity{0.898492}%
\pgfsetdash{}{0pt}%
\pgfpathmoveto{\pgfqpoint{2.189744in}{2.500044in}}%
\pgfpathcurveto{\pgfqpoint{2.197981in}{2.500044in}}{\pgfqpoint{2.205881in}{2.503316in}}{\pgfqpoint{2.211705in}{2.509140in}}%
\pgfpathcurveto{\pgfqpoint{2.217529in}{2.514964in}}{\pgfqpoint{2.220801in}{2.522864in}}{\pgfqpoint{2.220801in}{2.531100in}}%
\pgfpathcurveto{\pgfqpoint{2.220801in}{2.539336in}}{\pgfqpoint{2.217529in}{2.547236in}}{\pgfqpoint{2.211705in}{2.553060in}}%
\pgfpathcurveto{\pgfqpoint{2.205881in}{2.558884in}}{\pgfqpoint{2.197981in}{2.562157in}}{\pgfqpoint{2.189744in}{2.562157in}}%
\pgfpathcurveto{\pgfqpoint{2.181508in}{2.562157in}}{\pgfqpoint{2.173608in}{2.558884in}}{\pgfqpoint{2.167784in}{2.553060in}}%
\pgfpathcurveto{\pgfqpoint{2.161960in}{2.547236in}}{\pgfqpoint{2.158688in}{2.539336in}}{\pgfqpoint{2.158688in}{2.531100in}}%
\pgfpathcurveto{\pgfqpoint{2.158688in}{2.522864in}}{\pgfqpoint{2.161960in}{2.514964in}}{\pgfqpoint{2.167784in}{2.509140in}}%
\pgfpathcurveto{\pgfqpoint{2.173608in}{2.503316in}}{\pgfqpoint{2.181508in}{2.500044in}}{\pgfqpoint{2.189744in}{2.500044in}}%
\pgfpathclose%
\pgfusepath{stroke,fill}%
\end{pgfscope}%
\begin{pgfscope}%
\pgfpathrectangle{\pgfqpoint{0.100000in}{0.212622in}}{\pgfqpoint{3.696000in}{3.696000in}}%
\pgfusepath{clip}%
\pgfsetbuttcap%
\pgfsetroundjoin%
\definecolor{currentfill}{rgb}{0.121569,0.466667,0.705882}%
\pgfsetfillcolor{currentfill}%
\pgfsetfillopacity{0.898742}%
\pgfsetlinewidth{1.003750pt}%
\definecolor{currentstroke}{rgb}{0.121569,0.466667,0.705882}%
\pgfsetstrokecolor{currentstroke}%
\pgfsetstrokeopacity{0.898742}%
\pgfsetdash{}{0pt}%
\pgfpathmoveto{\pgfqpoint{1.758157in}{2.617461in}}%
\pgfpathcurveto{\pgfqpoint{1.766393in}{2.617461in}}{\pgfqpoint{1.774293in}{2.620733in}}{\pgfqpoint{1.780117in}{2.626557in}}%
\pgfpathcurveto{\pgfqpoint{1.785941in}{2.632381in}}{\pgfqpoint{1.789213in}{2.640281in}}{\pgfqpoint{1.789213in}{2.648517in}}%
\pgfpathcurveto{\pgfqpoint{1.789213in}{2.656753in}}{\pgfqpoint{1.785941in}{2.664654in}}{\pgfqpoint{1.780117in}{2.670477in}}%
\pgfpathcurveto{\pgfqpoint{1.774293in}{2.676301in}}{\pgfqpoint{1.766393in}{2.679574in}}{\pgfqpoint{1.758157in}{2.679574in}}%
\pgfpathcurveto{\pgfqpoint{1.749920in}{2.679574in}}{\pgfqpoint{1.742020in}{2.676301in}}{\pgfqpoint{1.736196in}{2.670477in}}%
\pgfpathcurveto{\pgfqpoint{1.730372in}{2.664654in}}{\pgfqpoint{1.727100in}{2.656753in}}{\pgfqpoint{1.727100in}{2.648517in}}%
\pgfpathcurveto{\pgfqpoint{1.727100in}{2.640281in}}{\pgfqpoint{1.730372in}{2.632381in}}{\pgfqpoint{1.736196in}{2.626557in}}%
\pgfpathcurveto{\pgfqpoint{1.742020in}{2.620733in}}{\pgfqpoint{1.749920in}{2.617461in}}{\pgfqpoint{1.758157in}{2.617461in}}%
\pgfpathclose%
\pgfusepath{stroke,fill}%
\end{pgfscope}%
\begin{pgfscope}%
\pgfpathrectangle{\pgfqpoint{0.100000in}{0.212622in}}{\pgfqpoint{3.696000in}{3.696000in}}%
\pgfusepath{clip}%
\pgfsetbuttcap%
\pgfsetroundjoin%
\definecolor{currentfill}{rgb}{0.121569,0.466667,0.705882}%
\pgfsetfillcolor{currentfill}%
\pgfsetfillopacity{0.899013}%
\pgfsetlinewidth{1.003750pt}%
\definecolor{currentstroke}{rgb}{0.121569,0.466667,0.705882}%
\pgfsetstrokecolor{currentstroke}%
\pgfsetstrokeopacity{0.899013}%
\pgfsetdash{}{0pt}%
\pgfpathmoveto{\pgfqpoint{2.188653in}{2.499363in}}%
\pgfpathcurveto{\pgfqpoint{2.196889in}{2.499363in}}{\pgfqpoint{2.204789in}{2.502635in}}{\pgfqpoint{2.210613in}{2.508459in}}%
\pgfpathcurveto{\pgfqpoint{2.216437in}{2.514283in}}{\pgfqpoint{2.219710in}{2.522183in}}{\pgfqpoint{2.219710in}{2.530419in}}%
\pgfpathcurveto{\pgfqpoint{2.219710in}{2.538656in}}{\pgfqpoint{2.216437in}{2.546556in}}{\pgfqpoint{2.210613in}{2.552380in}}%
\pgfpathcurveto{\pgfqpoint{2.204789in}{2.558204in}}{\pgfqpoint{2.196889in}{2.561476in}}{\pgfqpoint{2.188653in}{2.561476in}}%
\pgfpathcurveto{\pgfqpoint{2.180417in}{2.561476in}}{\pgfqpoint{2.172517in}{2.558204in}}{\pgfqpoint{2.166693in}{2.552380in}}%
\pgfpathcurveto{\pgfqpoint{2.160869in}{2.546556in}}{\pgfqpoint{2.157597in}{2.538656in}}{\pgfqpoint{2.157597in}{2.530419in}}%
\pgfpathcurveto{\pgfqpoint{2.157597in}{2.522183in}}{\pgfqpoint{2.160869in}{2.514283in}}{\pgfqpoint{2.166693in}{2.508459in}}%
\pgfpathcurveto{\pgfqpoint{2.172517in}{2.502635in}}{\pgfqpoint{2.180417in}{2.499363in}}{\pgfqpoint{2.188653in}{2.499363in}}%
\pgfpathclose%
\pgfusepath{stroke,fill}%
\end{pgfscope}%
\begin{pgfscope}%
\pgfpathrectangle{\pgfqpoint{0.100000in}{0.212622in}}{\pgfqpoint{3.696000in}{3.696000in}}%
\pgfusepath{clip}%
\pgfsetbuttcap%
\pgfsetroundjoin%
\definecolor{currentfill}{rgb}{0.121569,0.466667,0.705882}%
\pgfsetfillcolor{currentfill}%
\pgfsetfillopacity{0.899037}%
\pgfsetlinewidth{1.003750pt}%
\definecolor{currentstroke}{rgb}{0.121569,0.466667,0.705882}%
\pgfsetstrokecolor{currentstroke}%
\pgfsetstrokeopacity{0.899037}%
\pgfsetdash{}{0pt}%
\pgfpathmoveto{\pgfqpoint{1.757525in}{2.616757in}}%
\pgfpathcurveto{\pgfqpoint{1.765761in}{2.616757in}}{\pgfqpoint{1.773661in}{2.620029in}}{\pgfqpoint{1.779485in}{2.625853in}}%
\pgfpathcurveto{\pgfqpoint{1.785309in}{2.631677in}}{\pgfqpoint{1.788581in}{2.639577in}}{\pgfqpoint{1.788581in}{2.647814in}}%
\pgfpathcurveto{\pgfqpoint{1.788581in}{2.656050in}}{\pgfqpoint{1.785309in}{2.663950in}}{\pgfqpoint{1.779485in}{2.669774in}}%
\pgfpathcurveto{\pgfqpoint{1.773661in}{2.675598in}}{\pgfqpoint{1.765761in}{2.678870in}}{\pgfqpoint{1.757525in}{2.678870in}}%
\pgfpathcurveto{\pgfqpoint{1.749288in}{2.678870in}}{\pgfqpoint{1.741388in}{2.675598in}}{\pgfqpoint{1.735564in}{2.669774in}}%
\pgfpathcurveto{\pgfqpoint{1.729740in}{2.663950in}}{\pgfqpoint{1.726468in}{2.656050in}}{\pgfqpoint{1.726468in}{2.647814in}}%
\pgfpathcurveto{\pgfqpoint{1.726468in}{2.639577in}}{\pgfqpoint{1.729740in}{2.631677in}}{\pgfqpoint{1.735564in}{2.625853in}}%
\pgfpathcurveto{\pgfqpoint{1.741388in}{2.620029in}}{\pgfqpoint{1.749288in}{2.616757in}}{\pgfqpoint{1.757525in}{2.616757in}}%
\pgfpathclose%
\pgfusepath{stroke,fill}%
\end{pgfscope}%
\begin{pgfscope}%
\pgfpathrectangle{\pgfqpoint{0.100000in}{0.212622in}}{\pgfqpoint{3.696000in}{3.696000in}}%
\pgfusepath{clip}%
\pgfsetbuttcap%
\pgfsetroundjoin%
\definecolor{currentfill}{rgb}{0.121569,0.466667,0.705882}%
\pgfsetfillcolor{currentfill}%
\pgfsetfillopacity{0.899349}%
\pgfsetlinewidth{1.003750pt}%
\definecolor{currentstroke}{rgb}{0.121569,0.466667,0.705882}%
\pgfsetstrokecolor{currentstroke}%
\pgfsetstrokeopacity{0.899349}%
\pgfsetdash{}{0pt}%
\pgfpathmoveto{\pgfqpoint{2.187904in}{2.498852in}}%
\pgfpathcurveto{\pgfqpoint{2.196140in}{2.498852in}}{\pgfqpoint{2.204041in}{2.502124in}}{\pgfqpoint{2.209864in}{2.507948in}}%
\pgfpathcurveto{\pgfqpoint{2.215688in}{2.513772in}}{\pgfqpoint{2.218961in}{2.521672in}}{\pgfqpoint{2.218961in}{2.529908in}}%
\pgfpathcurveto{\pgfqpoint{2.218961in}{2.538145in}}{\pgfqpoint{2.215688in}{2.546045in}}{\pgfqpoint{2.209864in}{2.551869in}}%
\pgfpathcurveto{\pgfqpoint{2.204041in}{2.557693in}}{\pgfqpoint{2.196140in}{2.560965in}}{\pgfqpoint{2.187904in}{2.560965in}}%
\pgfpathcurveto{\pgfqpoint{2.179668in}{2.560965in}}{\pgfqpoint{2.171768in}{2.557693in}}{\pgfqpoint{2.165944in}{2.551869in}}%
\pgfpathcurveto{\pgfqpoint{2.160120in}{2.546045in}}{\pgfqpoint{2.156848in}{2.538145in}}{\pgfqpoint{2.156848in}{2.529908in}}%
\pgfpathcurveto{\pgfqpoint{2.156848in}{2.521672in}}{\pgfqpoint{2.160120in}{2.513772in}}{\pgfqpoint{2.165944in}{2.507948in}}%
\pgfpathcurveto{\pgfqpoint{2.171768in}{2.502124in}}{\pgfqpoint{2.179668in}{2.498852in}}{\pgfqpoint{2.187904in}{2.498852in}}%
\pgfpathclose%
\pgfusepath{stroke,fill}%
\end{pgfscope}%
\begin{pgfscope}%
\pgfpathrectangle{\pgfqpoint{0.100000in}{0.212622in}}{\pgfqpoint{3.696000in}{3.696000in}}%
\pgfusepath{clip}%
\pgfsetbuttcap%
\pgfsetroundjoin%
\definecolor{currentfill}{rgb}{0.121569,0.466667,0.705882}%
\pgfsetfillcolor{currentfill}%
\pgfsetfillopacity{0.899389}%
\pgfsetlinewidth{1.003750pt}%
\definecolor{currentstroke}{rgb}{0.121569,0.466667,0.705882}%
\pgfsetstrokecolor{currentstroke}%
\pgfsetstrokeopacity{0.899389}%
\pgfsetdash{}{0pt}%
\pgfpathmoveto{\pgfqpoint{0.945106in}{2.077319in}}%
\pgfpathcurveto{\pgfqpoint{0.953342in}{2.077319in}}{\pgfqpoint{0.961242in}{2.080592in}}{\pgfqpoint{0.967066in}{2.086416in}}%
\pgfpathcurveto{\pgfqpoint{0.972890in}{2.092240in}}{\pgfqpoint{0.976162in}{2.100140in}}{\pgfqpoint{0.976162in}{2.108376in}}%
\pgfpathcurveto{\pgfqpoint{0.976162in}{2.116612in}}{\pgfqpoint{0.972890in}{2.124512in}}{\pgfqpoint{0.967066in}{2.130336in}}%
\pgfpathcurveto{\pgfqpoint{0.961242in}{2.136160in}}{\pgfqpoint{0.953342in}{2.139432in}}{\pgfqpoint{0.945106in}{2.139432in}}%
\pgfpathcurveto{\pgfqpoint{0.936869in}{2.139432in}}{\pgfqpoint{0.928969in}{2.136160in}}{\pgfqpoint{0.923145in}{2.130336in}}%
\pgfpathcurveto{\pgfqpoint{0.917322in}{2.124512in}}{\pgfqpoint{0.914049in}{2.116612in}}{\pgfqpoint{0.914049in}{2.108376in}}%
\pgfpathcurveto{\pgfqpoint{0.914049in}{2.100140in}}{\pgfqpoint{0.917322in}{2.092240in}}{\pgfqpoint{0.923145in}{2.086416in}}%
\pgfpathcurveto{\pgfqpoint{0.928969in}{2.080592in}}{\pgfqpoint{0.936869in}{2.077319in}}{\pgfqpoint{0.945106in}{2.077319in}}%
\pgfpathclose%
\pgfusepath{stroke,fill}%
\end{pgfscope}%
\begin{pgfscope}%
\pgfpathrectangle{\pgfqpoint{0.100000in}{0.212622in}}{\pgfqpoint{3.696000in}{3.696000in}}%
\pgfusepath{clip}%
\pgfsetbuttcap%
\pgfsetroundjoin%
\definecolor{currentfill}{rgb}{0.121569,0.466667,0.705882}%
\pgfsetfillcolor{currentfill}%
\pgfsetfillopacity{0.899567}%
\pgfsetlinewidth{1.003750pt}%
\definecolor{currentstroke}{rgb}{0.121569,0.466667,0.705882}%
\pgfsetstrokecolor{currentstroke}%
\pgfsetstrokeopacity{0.899567}%
\pgfsetdash{}{0pt}%
\pgfpathmoveto{\pgfqpoint{2.187408in}{2.498502in}}%
\pgfpathcurveto{\pgfqpoint{2.195644in}{2.498502in}}{\pgfqpoint{2.203544in}{2.501775in}}{\pgfqpoint{2.209368in}{2.507599in}}%
\pgfpathcurveto{\pgfqpoint{2.215192in}{2.513422in}}{\pgfqpoint{2.218465in}{2.521323in}}{\pgfqpoint{2.218465in}{2.529559in}}%
\pgfpathcurveto{\pgfqpoint{2.218465in}{2.537795in}}{\pgfqpoint{2.215192in}{2.545695in}}{\pgfqpoint{2.209368in}{2.551519in}}%
\pgfpathcurveto{\pgfqpoint{2.203544in}{2.557343in}}{\pgfqpoint{2.195644in}{2.560615in}}{\pgfqpoint{2.187408in}{2.560615in}}%
\pgfpathcurveto{\pgfqpoint{2.179172in}{2.560615in}}{\pgfqpoint{2.171272in}{2.557343in}}{\pgfqpoint{2.165448in}{2.551519in}}%
\pgfpathcurveto{\pgfqpoint{2.159624in}{2.545695in}}{\pgfqpoint{2.156352in}{2.537795in}}{\pgfqpoint{2.156352in}{2.529559in}}%
\pgfpathcurveto{\pgfqpoint{2.156352in}{2.521323in}}{\pgfqpoint{2.159624in}{2.513422in}}{\pgfqpoint{2.165448in}{2.507599in}}%
\pgfpathcurveto{\pgfqpoint{2.171272in}{2.501775in}}{\pgfqpoint{2.179172in}{2.498502in}}{\pgfqpoint{2.187408in}{2.498502in}}%
\pgfpathclose%
\pgfusepath{stroke,fill}%
\end{pgfscope}%
\begin{pgfscope}%
\pgfpathrectangle{\pgfqpoint{0.100000in}{0.212622in}}{\pgfqpoint{3.696000in}{3.696000in}}%
\pgfusepath{clip}%
\pgfsetbuttcap%
\pgfsetroundjoin%
\definecolor{currentfill}{rgb}{0.121569,0.466667,0.705882}%
\pgfsetfillcolor{currentfill}%
\pgfsetfillopacity{0.899648}%
\pgfsetlinewidth{1.003750pt}%
\definecolor{currentstroke}{rgb}{0.121569,0.466667,0.705882}%
\pgfsetstrokecolor{currentstroke}%
\pgfsetstrokeopacity{0.899648}%
\pgfsetdash{}{0pt}%
\pgfpathmoveto{\pgfqpoint{1.756123in}{2.615089in}}%
\pgfpathcurveto{\pgfqpoint{1.764359in}{2.615089in}}{\pgfqpoint{1.772260in}{2.618361in}}{\pgfqpoint{1.778083in}{2.624185in}}%
\pgfpathcurveto{\pgfqpoint{1.783907in}{2.630009in}}{\pgfqpoint{1.787180in}{2.637909in}}{\pgfqpoint{1.787180in}{2.646145in}}%
\pgfpathcurveto{\pgfqpoint{1.787180in}{2.654382in}}{\pgfqpoint{1.783907in}{2.662282in}}{\pgfqpoint{1.778083in}{2.668106in}}%
\pgfpathcurveto{\pgfqpoint{1.772260in}{2.673930in}}{\pgfqpoint{1.764359in}{2.677202in}}{\pgfqpoint{1.756123in}{2.677202in}}%
\pgfpathcurveto{\pgfqpoint{1.747887in}{2.677202in}}{\pgfqpoint{1.739987in}{2.673930in}}{\pgfqpoint{1.734163in}{2.668106in}}%
\pgfpathcurveto{\pgfqpoint{1.728339in}{2.662282in}}{\pgfqpoint{1.725067in}{2.654382in}}{\pgfqpoint{1.725067in}{2.646145in}}%
\pgfpathcurveto{\pgfqpoint{1.725067in}{2.637909in}}{\pgfqpoint{1.728339in}{2.630009in}}{\pgfqpoint{1.734163in}{2.624185in}}%
\pgfpathcurveto{\pgfqpoint{1.739987in}{2.618361in}}{\pgfqpoint{1.747887in}{2.615089in}}{\pgfqpoint{1.756123in}{2.615089in}}%
\pgfpathclose%
\pgfusepath{stroke,fill}%
\end{pgfscope}%
\begin{pgfscope}%
\pgfpathrectangle{\pgfqpoint{0.100000in}{0.212622in}}{\pgfqpoint{3.696000in}{3.696000in}}%
\pgfusepath{clip}%
\pgfsetbuttcap%
\pgfsetroundjoin%
\definecolor{currentfill}{rgb}{0.121569,0.466667,0.705882}%
\pgfsetfillcolor{currentfill}%
\pgfsetfillopacity{0.899677}%
\pgfsetlinewidth{1.003750pt}%
\definecolor{currentstroke}{rgb}{0.121569,0.466667,0.705882}%
\pgfsetstrokecolor{currentstroke}%
\pgfsetstrokeopacity{0.899677}%
\pgfsetdash{}{0pt}%
\pgfpathmoveto{\pgfqpoint{2.187165in}{2.498351in}}%
\pgfpathcurveto{\pgfqpoint{2.195402in}{2.498351in}}{\pgfqpoint{2.203302in}{2.501623in}}{\pgfqpoint{2.209126in}{2.507447in}}%
\pgfpathcurveto{\pgfqpoint{2.214949in}{2.513271in}}{\pgfqpoint{2.218222in}{2.521171in}}{\pgfqpoint{2.218222in}{2.529407in}}%
\pgfpathcurveto{\pgfqpoint{2.218222in}{2.537644in}}{\pgfqpoint{2.214949in}{2.545544in}}{\pgfqpoint{2.209126in}{2.551368in}}%
\pgfpathcurveto{\pgfqpoint{2.203302in}{2.557192in}}{\pgfqpoint{2.195402in}{2.560464in}}{\pgfqpoint{2.187165in}{2.560464in}}%
\pgfpathcurveto{\pgfqpoint{2.178929in}{2.560464in}}{\pgfqpoint{2.171029in}{2.557192in}}{\pgfqpoint{2.165205in}{2.551368in}}%
\pgfpathcurveto{\pgfqpoint{2.159381in}{2.545544in}}{\pgfqpoint{2.156109in}{2.537644in}}{\pgfqpoint{2.156109in}{2.529407in}}%
\pgfpathcurveto{\pgfqpoint{2.156109in}{2.521171in}}{\pgfqpoint{2.159381in}{2.513271in}}{\pgfqpoint{2.165205in}{2.507447in}}%
\pgfpathcurveto{\pgfqpoint{2.171029in}{2.501623in}}{\pgfqpoint{2.178929in}{2.498351in}}{\pgfqpoint{2.187165in}{2.498351in}}%
\pgfpathclose%
\pgfusepath{stroke,fill}%
\end{pgfscope}%
\begin{pgfscope}%
\pgfpathrectangle{\pgfqpoint{0.100000in}{0.212622in}}{\pgfqpoint{3.696000in}{3.696000in}}%
\pgfusepath{clip}%
\pgfsetbuttcap%
\pgfsetroundjoin%
\definecolor{currentfill}{rgb}{0.121569,0.466667,0.705882}%
\pgfsetfillcolor{currentfill}%
\pgfsetfillopacity{0.899881}%
\pgfsetlinewidth{1.003750pt}%
\definecolor{currentstroke}{rgb}{0.121569,0.466667,0.705882}%
\pgfsetstrokecolor{currentstroke}%
\pgfsetstrokeopacity{0.899881}%
\pgfsetdash{}{0pt}%
\pgfpathmoveto{\pgfqpoint{2.186730in}{2.498085in}}%
\pgfpathcurveto{\pgfqpoint{2.194967in}{2.498085in}}{\pgfqpoint{2.202867in}{2.501357in}}{\pgfqpoint{2.208691in}{2.507181in}}%
\pgfpathcurveto{\pgfqpoint{2.214515in}{2.513005in}}{\pgfqpoint{2.217787in}{2.520905in}}{\pgfqpoint{2.217787in}{2.529141in}}%
\pgfpathcurveto{\pgfqpoint{2.217787in}{2.537377in}}{\pgfqpoint{2.214515in}{2.545277in}}{\pgfqpoint{2.208691in}{2.551101in}}%
\pgfpathcurveto{\pgfqpoint{2.202867in}{2.556925in}}{\pgfqpoint{2.194967in}{2.560198in}}{\pgfqpoint{2.186730in}{2.560198in}}%
\pgfpathcurveto{\pgfqpoint{2.178494in}{2.560198in}}{\pgfqpoint{2.170594in}{2.556925in}}{\pgfqpoint{2.164770in}{2.551101in}}%
\pgfpathcurveto{\pgfqpoint{2.158946in}{2.545277in}}{\pgfqpoint{2.155674in}{2.537377in}}{\pgfqpoint{2.155674in}{2.529141in}}%
\pgfpathcurveto{\pgfqpoint{2.155674in}{2.520905in}}{\pgfqpoint{2.158946in}{2.513005in}}{\pgfqpoint{2.164770in}{2.507181in}}%
\pgfpathcurveto{\pgfqpoint{2.170594in}{2.501357in}}{\pgfqpoint{2.178494in}{2.498085in}}{\pgfqpoint{2.186730in}{2.498085in}}%
\pgfpathclose%
\pgfusepath{stroke,fill}%
\end{pgfscope}%
\begin{pgfscope}%
\pgfpathrectangle{\pgfqpoint{0.100000in}{0.212622in}}{\pgfqpoint{3.696000in}{3.696000in}}%
\pgfusepath{clip}%
\pgfsetbuttcap%
\pgfsetroundjoin%
\definecolor{currentfill}{rgb}{0.121569,0.466667,0.705882}%
\pgfsetfillcolor{currentfill}%
\pgfsetfillopacity{0.899994}%
\pgfsetlinewidth{1.003750pt}%
\definecolor{currentstroke}{rgb}{0.121569,0.466667,0.705882}%
\pgfsetstrokecolor{currentstroke}%
\pgfsetstrokeopacity{0.899994}%
\pgfsetdash{}{0pt}%
\pgfpathmoveto{\pgfqpoint{1.755364in}{2.614222in}}%
\pgfpathcurveto{\pgfqpoint{1.763600in}{2.614222in}}{\pgfqpoint{1.771500in}{2.617494in}}{\pgfqpoint{1.777324in}{2.623318in}}%
\pgfpathcurveto{\pgfqpoint{1.783148in}{2.629142in}}{\pgfqpoint{1.786420in}{2.637042in}}{\pgfqpoint{1.786420in}{2.645279in}}%
\pgfpathcurveto{\pgfqpoint{1.786420in}{2.653515in}}{\pgfqpoint{1.783148in}{2.661415in}}{\pgfqpoint{1.777324in}{2.667239in}}%
\pgfpathcurveto{\pgfqpoint{1.771500in}{2.673063in}}{\pgfqpoint{1.763600in}{2.676335in}}{\pgfqpoint{1.755364in}{2.676335in}}%
\pgfpathcurveto{\pgfqpoint{1.747127in}{2.676335in}}{\pgfqpoint{1.739227in}{2.673063in}}{\pgfqpoint{1.733403in}{2.667239in}}%
\pgfpathcurveto{\pgfqpoint{1.727579in}{2.661415in}}{\pgfqpoint{1.724307in}{2.653515in}}{\pgfqpoint{1.724307in}{2.645279in}}%
\pgfpathcurveto{\pgfqpoint{1.724307in}{2.637042in}}{\pgfqpoint{1.727579in}{2.629142in}}{\pgfqpoint{1.733403in}{2.623318in}}%
\pgfpathcurveto{\pgfqpoint{1.739227in}{2.617494in}}{\pgfqpoint{1.747127in}{2.614222in}}{\pgfqpoint{1.755364in}{2.614222in}}%
\pgfpathclose%
\pgfusepath{stroke,fill}%
\end{pgfscope}%
\begin{pgfscope}%
\pgfpathrectangle{\pgfqpoint{0.100000in}{0.212622in}}{\pgfqpoint{3.696000in}{3.696000in}}%
\pgfusepath{clip}%
\pgfsetbuttcap%
\pgfsetroundjoin%
\definecolor{currentfill}{rgb}{0.121569,0.466667,0.705882}%
\pgfsetfillcolor{currentfill}%
\pgfsetfillopacity{0.900177}%
\pgfsetlinewidth{1.003750pt}%
\definecolor{currentstroke}{rgb}{0.121569,0.466667,0.705882}%
\pgfsetstrokecolor{currentstroke}%
\pgfsetstrokeopacity{0.900177}%
\pgfsetdash{}{0pt}%
\pgfpathmoveto{\pgfqpoint{1.754938in}{2.613705in}}%
\pgfpathcurveto{\pgfqpoint{1.763174in}{2.613705in}}{\pgfqpoint{1.771074in}{2.616978in}}{\pgfqpoint{1.776898in}{2.622802in}}%
\pgfpathcurveto{\pgfqpoint{1.782722in}{2.628625in}}{\pgfqpoint{1.785994in}{2.636525in}}{\pgfqpoint{1.785994in}{2.644762in}}%
\pgfpathcurveto{\pgfqpoint{1.785994in}{2.652998in}}{\pgfqpoint{1.782722in}{2.660898in}}{\pgfqpoint{1.776898in}{2.666722in}}%
\pgfpathcurveto{\pgfqpoint{1.771074in}{2.672546in}}{\pgfqpoint{1.763174in}{2.675818in}}{\pgfqpoint{1.754938in}{2.675818in}}%
\pgfpathcurveto{\pgfqpoint{1.746702in}{2.675818in}}{\pgfqpoint{1.738802in}{2.672546in}}{\pgfqpoint{1.732978in}{2.666722in}}%
\pgfpathcurveto{\pgfqpoint{1.727154in}{2.660898in}}{\pgfqpoint{1.723881in}{2.652998in}}{\pgfqpoint{1.723881in}{2.644762in}}%
\pgfpathcurveto{\pgfqpoint{1.723881in}{2.636525in}}{\pgfqpoint{1.727154in}{2.628625in}}{\pgfqpoint{1.732978in}{2.622802in}}%
\pgfpathcurveto{\pgfqpoint{1.738802in}{2.616978in}}{\pgfqpoint{1.746702in}{2.613705in}}{\pgfqpoint{1.754938in}{2.613705in}}%
\pgfpathclose%
\pgfusepath{stroke,fill}%
\end{pgfscope}%
\begin{pgfscope}%
\pgfpathrectangle{\pgfqpoint{0.100000in}{0.212622in}}{\pgfqpoint{3.696000in}{3.696000in}}%
\pgfusepath{clip}%
\pgfsetbuttcap%
\pgfsetroundjoin%
\definecolor{currentfill}{rgb}{0.121569,0.466667,0.705882}%
\pgfsetfillcolor{currentfill}%
\pgfsetfillopacity{0.900237}%
\pgfsetlinewidth{1.003750pt}%
\definecolor{currentstroke}{rgb}{0.121569,0.466667,0.705882}%
\pgfsetstrokecolor{currentstroke}%
\pgfsetstrokeopacity{0.900237}%
\pgfsetdash{}{0pt}%
\pgfpathmoveto{\pgfqpoint{2.185932in}{2.497522in}}%
\pgfpathcurveto{\pgfqpoint{2.194168in}{2.497522in}}{\pgfqpoint{2.202068in}{2.500794in}}{\pgfqpoint{2.207892in}{2.506618in}}%
\pgfpathcurveto{\pgfqpoint{2.213716in}{2.512442in}}{\pgfqpoint{2.216988in}{2.520342in}}{\pgfqpoint{2.216988in}{2.528578in}}%
\pgfpathcurveto{\pgfqpoint{2.216988in}{2.536815in}}{\pgfqpoint{2.213716in}{2.544715in}}{\pgfqpoint{2.207892in}{2.550539in}}%
\pgfpathcurveto{\pgfqpoint{2.202068in}{2.556362in}}{\pgfqpoint{2.194168in}{2.559635in}}{\pgfqpoint{2.185932in}{2.559635in}}%
\pgfpathcurveto{\pgfqpoint{2.177695in}{2.559635in}}{\pgfqpoint{2.169795in}{2.556362in}}{\pgfqpoint{2.163971in}{2.550539in}}%
\pgfpathcurveto{\pgfqpoint{2.158148in}{2.544715in}}{\pgfqpoint{2.154875in}{2.536815in}}{\pgfqpoint{2.154875in}{2.528578in}}%
\pgfpathcurveto{\pgfqpoint{2.154875in}{2.520342in}}{\pgfqpoint{2.158148in}{2.512442in}}{\pgfqpoint{2.163971in}{2.506618in}}%
\pgfpathcurveto{\pgfqpoint{2.169795in}{2.500794in}}{\pgfqpoint{2.177695in}{2.497522in}}{\pgfqpoint{2.185932in}{2.497522in}}%
\pgfpathclose%
\pgfusepath{stroke,fill}%
\end{pgfscope}%
\begin{pgfscope}%
\pgfpathrectangle{\pgfqpoint{0.100000in}{0.212622in}}{\pgfqpoint{3.696000in}{3.696000in}}%
\pgfusepath{clip}%
\pgfsetbuttcap%
\pgfsetroundjoin%
\definecolor{currentfill}{rgb}{0.121569,0.466667,0.705882}%
\pgfsetfillcolor{currentfill}%
\pgfsetfillopacity{0.900490}%
\pgfsetlinewidth{1.003750pt}%
\definecolor{currentstroke}{rgb}{0.121569,0.466667,0.705882}%
\pgfsetstrokecolor{currentstroke}%
\pgfsetstrokeopacity{0.900490}%
\pgfsetdash{}{0pt}%
\pgfpathmoveto{\pgfqpoint{2.185351in}{2.497118in}}%
\pgfpathcurveto{\pgfqpoint{2.193588in}{2.497118in}}{\pgfqpoint{2.201488in}{2.500390in}}{\pgfqpoint{2.207311in}{2.506214in}}%
\pgfpathcurveto{\pgfqpoint{2.213135in}{2.512038in}}{\pgfqpoint{2.216408in}{2.519938in}}{\pgfqpoint{2.216408in}{2.528174in}}%
\pgfpathcurveto{\pgfqpoint{2.216408in}{2.536411in}}{\pgfqpoint{2.213135in}{2.544311in}}{\pgfqpoint{2.207311in}{2.550135in}}%
\pgfpathcurveto{\pgfqpoint{2.201488in}{2.555959in}}{\pgfqpoint{2.193588in}{2.559231in}}{\pgfqpoint{2.185351in}{2.559231in}}%
\pgfpathcurveto{\pgfqpoint{2.177115in}{2.559231in}}{\pgfqpoint{2.169215in}{2.555959in}}{\pgfqpoint{2.163391in}{2.550135in}}%
\pgfpathcurveto{\pgfqpoint{2.157567in}{2.544311in}}{\pgfqpoint{2.154295in}{2.536411in}}{\pgfqpoint{2.154295in}{2.528174in}}%
\pgfpathcurveto{\pgfqpoint{2.154295in}{2.519938in}}{\pgfqpoint{2.157567in}{2.512038in}}{\pgfqpoint{2.163391in}{2.506214in}}%
\pgfpathcurveto{\pgfqpoint{2.169215in}{2.500390in}}{\pgfqpoint{2.177115in}{2.497118in}}{\pgfqpoint{2.185351in}{2.497118in}}%
\pgfpathclose%
\pgfusepath{stroke,fill}%
\end{pgfscope}%
\begin{pgfscope}%
\pgfpathrectangle{\pgfqpoint{0.100000in}{0.212622in}}{\pgfqpoint{3.696000in}{3.696000in}}%
\pgfusepath{clip}%
\pgfsetbuttcap%
\pgfsetroundjoin%
\definecolor{currentfill}{rgb}{0.121569,0.466667,0.705882}%
\pgfsetfillcolor{currentfill}%
\pgfsetfillopacity{0.900579}%
\pgfsetlinewidth{1.003750pt}%
\definecolor{currentstroke}{rgb}{0.121569,0.466667,0.705882}%
\pgfsetstrokecolor{currentstroke}%
\pgfsetstrokeopacity{0.900579}%
\pgfsetdash{}{0pt}%
\pgfpathmoveto{\pgfqpoint{1.753941in}{2.612399in}}%
\pgfpathcurveto{\pgfqpoint{1.762177in}{2.612399in}}{\pgfqpoint{1.770077in}{2.615671in}}{\pgfqpoint{1.775901in}{2.621495in}}%
\pgfpathcurveto{\pgfqpoint{1.781725in}{2.627319in}}{\pgfqpoint{1.784998in}{2.635219in}}{\pgfqpoint{1.784998in}{2.643455in}}%
\pgfpathcurveto{\pgfqpoint{1.784998in}{2.651691in}}{\pgfqpoint{1.781725in}{2.659592in}}{\pgfqpoint{1.775901in}{2.665415in}}%
\pgfpathcurveto{\pgfqpoint{1.770077in}{2.671239in}}{\pgfqpoint{1.762177in}{2.674512in}}{\pgfqpoint{1.753941in}{2.674512in}}%
\pgfpathcurveto{\pgfqpoint{1.745705in}{2.674512in}}{\pgfqpoint{1.737805in}{2.671239in}}{\pgfqpoint{1.731981in}{2.665415in}}%
\pgfpathcurveto{\pgfqpoint{1.726157in}{2.659592in}}{\pgfqpoint{1.722885in}{2.651691in}}{\pgfqpoint{1.722885in}{2.643455in}}%
\pgfpathcurveto{\pgfqpoint{1.722885in}{2.635219in}}{\pgfqpoint{1.726157in}{2.627319in}}{\pgfqpoint{1.731981in}{2.621495in}}%
\pgfpathcurveto{\pgfqpoint{1.737805in}{2.615671in}}{\pgfqpoint{1.745705in}{2.612399in}}{\pgfqpoint{1.753941in}{2.612399in}}%
\pgfpathclose%
\pgfusepath{stroke,fill}%
\end{pgfscope}%
\begin{pgfscope}%
\pgfpathrectangle{\pgfqpoint{0.100000in}{0.212622in}}{\pgfqpoint{3.696000in}{3.696000in}}%
\pgfusepath{clip}%
\pgfsetbuttcap%
\pgfsetroundjoin%
\definecolor{currentfill}{rgb}{0.121569,0.466667,0.705882}%
\pgfsetfillcolor{currentfill}%
\pgfsetfillopacity{0.900582}%
\pgfsetlinewidth{1.003750pt}%
\definecolor{currentstroke}{rgb}{0.121569,0.466667,0.705882}%
\pgfsetstrokecolor{currentstroke}%
\pgfsetstrokeopacity{0.900582}%
\pgfsetdash{}{0pt}%
\pgfpathmoveto{\pgfqpoint{2.185154in}{2.497004in}}%
\pgfpathcurveto{\pgfqpoint{2.193390in}{2.497004in}}{\pgfqpoint{2.201290in}{2.500276in}}{\pgfqpoint{2.207114in}{2.506100in}}%
\pgfpathcurveto{\pgfqpoint{2.212938in}{2.511924in}}{\pgfqpoint{2.216210in}{2.519824in}}{\pgfqpoint{2.216210in}{2.528060in}}%
\pgfpathcurveto{\pgfqpoint{2.216210in}{2.536297in}}{\pgfqpoint{2.212938in}{2.544197in}}{\pgfqpoint{2.207114in}{2.550021in}}%
\pgfpathcurveto{\pgfqpoint{2.201290in}{2.555844in}}{\pgfqpoint{2.193390in}{2.559117in}}{\pgfqpoint{2.185154in}{2.559117in}}%
\pgfpathcurveto{\pgfqpoint{2.176917in}{2.559117in}}{\pgfqpoint{2.169017in}{2.555844in}}{\pgfqpoint{2.163193in}{2.550021in}}%
\pgfpathcurveto{\pgfqpoint{2.157369in}{2.544197in}}{\pgfqpoint{2.154097in}{2.536297in}}{\pgfqpoint{2.154097in}{2.528060in}}%
\pgfpathcurveto{\pgfqpoint{2.154097in}{2.519824in}}{\pgfqpoint{2.157369in}{2.511924in}}{\pgfqpoint{2.163193in}{2.506100in}}%
\pgfpathcurveto{\pgfqpoint{2.169017in}{2.500276in}}{\pgfqpoint{2.176917in}{2.497004in}}{\pgfqpoint{2.185154in}{2.497004in}}%
\pgfpathclose%
\pgfusepath{stroke,fill}%
\end{pgfscope}%
\begin{pgfscope}%
\pgfpathrectangle{\pgfqpoint{0.100000in}{0.212622in}}{\pgfqpoint{3.696000in}{3.696000in}}%
\pgfusepath{clip}%
\pgfsetbuttcap%
\pgfsetroundjoin%
\definecolor{currentfill}{rgb}{0.121569,0.466667,0.705882}%
\pgfsetfillcolor{currentfill}%
\pgfsetfillopacity{0.900754}%
\pgfsetlinewidth{1.003750pt}%
\definecolor{currentstroke}{rgb}{0.121569,0.466667,0.705882}%
\pgfsetstrokecolor{currentstroke}%
\pgfsetstrokeopacity{0.900754}%
\pgfsetdash{}{0pt}%
\pgfpathmoveto{\pgfqpoint{2.184798in}{2.496822in}}%
\pgfpathcurveto{\pgfqpoint{2.193035in}{2.496822in}}{\pgfqpoint{2.200935in}{2.500095in}}{\pgfqpoint{2.206759in}{2.505919in}}%
\pgfpathcurveto{\pgfqpoint{2.212583in}{2.511742in}}{\pgfqpoint{2.215855in}{2.519642in}}{\pgfqpoint{2.215855in}{2.527879in}}%
\pgfpathcurveto{\pgfqpoint{2.215855in}{2.536115in}}{\pgfqpoint{2.212583in}{2.544015in}}{\pgfqpoint{2.206759in}{2.549839in}}%
\pgfpathcurveto{\pgfqpoint{2.200935in}{2.555663in}}{\pgfqpoint{2.193035in}{2.558935in}}{\pgfqpoint{2.184798in}{2.558935in}}%
\pgfpathcurveto{\pgfqpoint{2.176562in}{2.558935in}}{\pgfqpoint{2.168662in}{2.555663in}}{\pgfqpoint{2.162838in}{2.549839in}}%
\pgfpathcurveto{\pgfqpoint{2.157014in}{2.544015in}}{\pgfqpoint{2.153742in}{2.536115in}}{\pgfqpoint{2.153742in}{2.527879in}}%
\pgfpathcurveto{\pgfqpoint{2.153742in}{2.519642in}}{\pgfqpoint{2.157014in}{2.511742in}}{\pgfqpoint{2.162838in}{2.505919in}}%
\pgfpathcurveto{\pgfqpoint{2.168662in}{2.500095in}}{\pgfqpoint{2.176562in}{2.496822in}}{\pgfqpoint{2.184798in}{2.496822in}}%
\pgfpathclose%
\pgfusepath{stroke,fill}%
\end{pgfscope}%
\begin{pgfscope}%
\pgfpathrectangle{\pgfqpoint{0.100000in}{0.212622in}}{\pgfqpoint{3.696000in}{3.696000in}}%
\pgfusepath{clip}%
\pgfsetbuttcap%
\pgfsetroundjoin%
\definecolor{currentfill}{rgb}{0.121569,0.466667,0.705882}%
\pgfsetfillcolor{currentfill}%
\pgfsetfillopacity{0.901046}%
\pgfsetlinewidth{1.003750pt}%
\definecolor{currentstroke}{rgb}{0.121569,0.466667,0.705882}%
\pgfsetstrokecolor{currentstroke}%
\pgfsetstrokeopacity{0.901046}%
\pgfsetdash{}{0pt}%
\pgfpathmoveto{\pgfqpoint{2.184133in}{2.496391in}}%
\pgfpathcurveto{\pgfqpoint{2.192369in}{2.496391in}}{\pgfqpoint{2.200269in}{2.499663in}}{\pgfqpoint{2.206093in}{2.505487in}}%
\pgfpathcurveto{\pgfqpoint{2.211917in}{2.511311in}}{\pgfqpoint{2.215189in}{2.519211in}}{\pgfqpoint{2.215189in}{2.527447in}}%
\pgfpathcurveto{\pgfqpoint{2.215189in}{2.535684in}}{\pgfqpoint{2.211917in}{2.543584in}}{\pgfqpoint{2.206093in}{2.549408in}}%
\pgfpathcurveto{\pgfqpoint{2.200269in}{2.555231in}}{\pgfqpoint{2.192369in}{2.558504in}}{\pgfqpoint{2.184133in}{2.558504in}}%
\pgfpathcurveto{\pgfqpoint{2.175896in}{2.558504in}}{\pgfqpoint{2.167996in}{2.555231in}}{\pgfqpoint{2.162172in}{2.549408in}}%
\pgfpathcurveto{\pgfqpoint{2.156348in}{2.543584in}}{\pgfqpoint{2.153076in}{2.535684in}}{\pgfqpoint{2.153076in}{2.527447in}}%
\pgfpathcurveto{\pgfqpoint{2.153076in}{2.519211in}}{\pgfqpoint{2.156348in}{2.511311in}}{\pgfqpoint{2.162172in}{2.505487in}}%
\pgfpathcurveto{\pgfqpoint{2.167996in}{2.499663in}}{\pgfqpoint{2.175896in}{2.496391in}}{\pgfqpoint{2.184133in}{2.496391in}}%
\pgfpathclose%
\pgfusepath{stroke,fill}%
\end{pgfscope}%
\begin{pgfscope}%
\pgfpathrectangle{\pgfqpoint{0.100000in}{0.212622in}}{\pgfqpoint{3.696000in}{3.696000in}}%
\pgfusepath{clip}%
\pgfsetbuttcap%
\pgfsetroundjoin%
\definecolor{currentfill}{rgb}{0.121569,0.466667,0.705882}%
\pgfsetfillcolor{currentfill}%
\pgfsetfillopacity{0.901084}%
\pgfsetlinewidth{1.003750pt}%
\definecolor{currentstroke}{rgb}{0.121569,0.466667,0.705882}%
\pgfsetstrokecolor{currentstroke}%
\pgfsetstrokeopacity{0.901084}%
\pgfsetdash{}{0pt}%
\pgfpathmoveto{\pgfqpoint{1.752747in}{2.610900in}}%
\pgfpathcurveto{\pgfqpoint{1.760984in}{2.610900in}}{\pgfqpoint{1.768884in}{2.614173in}}{\pgfqpoint{1.774708in}{2.619997in}}%
\pgfpathcurveto{\pgfqpoint{1.780532in}{2.625821in}}{\pgfqpoint{1.783804in}{2.633721in}}{\pgfqpoint{1.783804in}{2.641957in}}%
\pgfpathcurveto{\pgfqpoint{1.783804in}{2.650193in}}{\pgfqpoint{1.780532in}{2.658093in}}{\pgfqpoint{1.774708in}{2.663917in}}%
\pgfpathcurveto{\pgfqpoint{1.768884in}{2.669741in}}{\pgfqpoint{1.760984in}{2.673013in}}{\pgfqpoint{1.752747in}{2.673013in}}%
\pgfpathcurveto{\pgfqpoint{1.744511in}{2.673013in}}{\pgfqpoint{1.736611in}{2.669741in}}{\pgfqpoint{1.730787in}{2.663917in}}%
\pgfpathcurveto{\pgfqpoint{1.724963in}{2.658093in}}{\pgfqpoint{1.721691in}{2.650193in}}{\pgfqpoint{1.721691in}{2.641957in}}%
\pgfpathcurveto{\pgfqpoint{1.721691in}{2.633721in}}{\pgfqpoint{1.724963in}{2.625821in}}{\pgfqpoint{1.730787in}{2.619997in}}%
\pgfpathcurveto{\pgfqpoint{1.736611in}{2.614173in}}{\pgfqpoint{1.744511in}{2.610900in}}{\pgfqpoint{1.752747in}{2.610900in}}%
\pgfpathclose%
\pgfusepath{stroke,fill}%
\end{pgfscope}%
\begin{pgfscope}%
\pgfpathrectangle{\pgfqpoint{0.100000in}{0.212622in}}{\pgfqpoint{3.696000in}{3.696000in}}%
\pgfusepath{clip}%
\pgfsetbuttcap%
\pgfsetroundjoin%
\definecolor{currentfill}{rgb}{0.121569,0.466667,0.705882}%
\pgfsetfillcolor{currentfill}%
\pgfsetfillopacity{0.901225}%
\pgfsetlinewidth{1.003750pt}%
\definecolor{currentstroke}{rgb}{0.121569,0.466667,0.705882}%
\pgfsetstrokecolor{currentstroke}%
\pgfsetstrokeopacity{0.901225}%
\pgfsetdash{}{0pt}%
\pgfpathmoveto{\pgfqpoint{0.948868in}{2.072557in}}%
\pgfpathcurveto{\pgfqpoint{0.957104in}{2.072557in}}{\pgfqpoint{0.965004in}{2.075830in}}{\pgfqpoint{0.970828in}{2.081654in}}%
\pgfpathcurveto{\pgfqpoint{0.976652in}{2.087478in}}{\pgfqpoint{0.979924in}{2.095378in}}{\pgfqpoint{0.979924in}{2.103614in}}%
\pgfpathcurveto{\pgfqpoint{0.979924in}{2.111850in}}{\pgfqpoint{0.976652in}{2.119750in}}{\pgfqpoint{0.970828in}{2.125574in}}%
\pgfpathcurveto{\pgfqpoint{0.965004in}{2.131398in}}{\pgfqpoint{0.957104in}{2.134670in}}{\pgfqpoint{0.948868in}{2.134670in}}%
\pgfpathcurveto{\pgfqpoint{0.940631in}{2.134670in}}{\pgfqpoint{0.932731in}{2.131398in}}{\pgfqpoint{0.926907in}{2.125574in}}%
\pgfpathcurveto{\pgfqpoint{0.921083in}{2.119750in}}{\pgfqpoint{0.917811in}{2.111850in}}{\pgfqpoint{0.917811in}{2.103614in}}%
\pgfpathcurveto{\pgfqpoint{0.917811in}{2.095378in}}{\pgfqpoint{0.921083in}{2.087478in}}{\pgfqpoint{0.926907in}{2.081654in}}%
\pgfpathcurveto{\pgfqpoint{0.932731in}{2.075830in}}{\pgfqpoint{0.940631in}{2.072557in}}{\pgfqpoint{0.948868in}{2.072557in}}%
\pgfpathclose%
\pgfusepath{stroke,fill}%
\end{pgfscope}%
\begin{pgfscope}%
\pgfpathrectangle{\pgfqpoint{0.100000in}{0.212622in}}{\pgfqpoint{3.696000in}{3.696000in}}%
\pgfusepath{clip}%
\pgfsetbuttcap%
\pgfsetroundjoin%
\definecolor{currentfill}{rgb}{0.121569,0.466667,0.705882}%
\pgfsetfillcolor{currentfill}%
\pgfsetfillopacity{0.901382}%
\pgfsetlinewidth{1.003750pt}%
\definecolor{currentstroke}{rgb}{0.121569,0.466667,0.705882}%
\pgfsetstrokecolor{currentstroke}%
\pgfsetstrokeopacity{0.901382}%
\pgfsetdash{}{0pt}%
\pgfpathmoveto{\pgfqpoint{1.752066in}{2.610219in}}%
\pgfpathcurveto{\pgfqpoint{1.760302in}{2.610219in}}{\pgfqpoint{1.768202in}{2.613492in}}{\pgfqpoint{1.774026in}{2.619316in}}%
\pgfpathcurveto{\pgfqpoint{1.779850in}{2.625140in}}{\pgfqpoint{1.783122in}{2.633040in}}{\pgfqpoint{1.783122in}{2.641276in}}%
\pgfpathcurveto{\pgfqpoint{1.783122in}{2.649512in}}{\pgfqpoint{1.779850in}{2.657412in}}{\pgfqpoint{1.774026in}{2.663236in}}%
\pgfpathcurveto{\pgfqpoint{1.768202in}{2.669060in}}{\pgfqpoint{1.760302in}{2.672332in}}{\pgfqpoint{1.752066in}{2.672332in}}%
\pgfpathcurveto{\pgfqpoint{1.743830in}{2.672332in}}{\pgfqpoint{1.735930in}{2.669060in}}{\pgfqpoint{1.730106in}{2.663236in}}%
\pgfpathcurveto{\pgfqpoint{1.724282in}{2.657412in}}{\pgfqpoint{1.721009in}{2.649512in}}{\pgfqpoint{1.721009in}{2.641276in}}%
\pgfpathcurveto{\pgfqpoint{1.721009in}{2.633040in}}{\pgfqpoint{1.724282in}{2.625140in}}{\pgfqpoint{1.730106in}{2.619316in}}%
\pgfpathcurveto{\pgfqpoint{1.735930in}{2.613492in}}{\pgfqpoint{1.743830in}{2.610219in}}{\pgfqpoint{1.752066in}{2.610219in}}%
\pgfpathclose%
\pgfusepath{stroke,fill}%
\end{pgfscope}%
\begin{pgfscope}%
\pgfpathrectangle{\pgfqpoint{0.100000in}{0.212622in}}{\pgfqpoint{3.696000in}{3.696000in}}%
\pgfusepath{clip}%
\pgfsetbuttcap%
\pgfsetroundjoin%
\definecolor{currentfill}{rgb}{0.121569,0.466667,0.705882}%
\pgfsetfillcolor{currentfill}%
\pgfsetfillopacity{0.901536}%
\pgfsetlinewidth{1.003750pt}%
\definecolor{currentstroke}{rgb}{0.121569,0.466667,0.705882}%
\pgfsetstrokecolor{currentstroke}%
\pgfsetstrokeopacity{0.901536}%
\pgfsetdash{}{0pt}%
\pgfpathmoveto{\pgfqpoint{1.751691in}{2.609791in}}%
\pgfpathcurveto{\pgfqpoint{1.759928in}{2.609791in}}{\pgfqpoint{1.767828in}{2.613063in}}{\pgfqpoint{1.773652in}{2.618887in}}%
\pgfpathcurveto{\pgfqpoint{1.779475in}{2.624711in}}{\pgfqpoint{1.782748in}{2.632611in}}{\pgfqpoint{1.782748in}{2.640847in}}%
\pgfpathcurveto{\pgfqpoint{1.782748in}{2.649083in}}{\pgfqpoint{1.779475in}{2.656983in}}{\pgfqpoint{1.773652in}{2.662807in}}%
\pgfpathcurveto{\pgfqpoint{1.767828in}{2.668631in}}{\pgfqpoint{1.759928in}{2.671904in}}{\pgfqpoint{1.751691in}{2.671904in}}%
\pgfpathcurveto{\pgfqpoint{1.743455in}{2.671904in}}{\pgfqpoint{1.735555in}{2.668631in}}{\pgfqpoint{1.729731in}{2.662807in}}%
\pgfpathcurveto{\pgfqpoint{1.723907in}{2.656983in}}{\pgfqpoint{1.720635in}{2.649083in}}{\pgfqpoint{1.720635in}{2.640847in}}%
\pgfpathcurveto{\pgfqpoint{1.720635in}{2.632611in}}{\pgfqpoint{1.723907in}{2.624711in}}{\pgfqpoint{1.729731in}{2.618887in}}%
\pgfpathcurveto{\pgfqpoint{1.735555in}{2.613063in}}{\pgfqpoint{1.743455in}{2.609791in}}{\pgfqpoint{1.751691in}{2.609791in}}%
\pgfpathclose%
\pgfusepath{stroke,fill}%
\end{pgfscope}%
\begin{pgfscope}%
\pgfpathrectangle{\pgfqpoint{0.100000in}{0.212622in}}{\pgfqpoint{3.696000in}{3.696000in}}%
\pgfusepath{clip}%
\pgfsetbuttcap%
\pgfsetroundjoin%
\definecolor{currentfill}{rgb}{0.121569,0.466667,0.705882}%
\pgfsetfillcolor{currentfill}%
\pgfsetfillopacity{0.901569}%
\pgfsetlinewidth{1.003750pt}%
\definecolor{currentstroke}{rgb}{0.121569,0.466667,0.705882}%
\pgfsetstrokecolor{currentstroke}%
\pgfsetstrokeopacity{0.901569}%
\pgfsetdash{}{0pt}%
\pgfpathmoveto{\pgfqpoint{2.182909in}{2.495577in}}%
\pgfpathcurveto{\pgfqpoint{2.191145in}{2.495577in}}{\pgfqpoint{2.199045in}{2.498849in}}{\pgfqpoint{2.204869in}{2.504673in}}%
\pgfpathcurveto{\pgfqpoint{2.210693in}{2.510497in}}{\pgfqpoint{2.213965in}{2.518397in}}{\pgfqpoint{2.213965in}{2.526633in}}%
\pgfpathcurveto{\pgfqpoint{2.213965in}{2.534870in}}{\pgfqpoint{2.210693in}{2.542770in}}{\pgfqpoint{2.204869in}{2.548594in}}%
\pgfpathcurveto{\pgfqpoint{2.199045in}{2.554418in}}{\pgfqpoint{2.191145in}{2.557690in}}{\pgfqpoint{2.182909in}{2.557690in}}%
\pgfpathcurveto{\pgfqpoint{2.174672in}{2.557690in}}{\pgfqpoint{2.166772in}{2.554418in}}{\pgfqpoint{2.160948in}{2.548594in}}%
\pgfpathcurveto{\pgfqpoint{2.155124in}{2.542770in}}{\pgfqpoint{2.151852in}{2.534870in}}{\pgfqpoint{2.151852in}{2.526633in}}%
\pgfpathcurveto{\pgfqpoint{2.151852in}{2.518397in}}{\pgfqpoint{2.155124in}{2.510497in}}{\pgfqpoint{2.160948in}{2.504673in}}%
\pgfpathcurveto{\pgfqpoint{2.166772in}{2.498849in}}{\pgfqpoint{2.174672in}{2.495577in}}{\pgfqpoint{2.182909in}{2.495577in}}%
\pgfpathclose%
\pgfusepath{stroke,fill}%
\end{pgfscope}%
\begin{pgfscope}%
\pgfpathrectangle{\pgfqpoint{0.100000in}{0.212622in}}{\pgfqpoint{3.696000in}{3.696000in}}%
\pgfusepath{clip}%
\pgfsetbuttcap%
\pgfsetroundjoin%
\definecolor{currentfill}{rgb}{0.121569,0.466667,0.705882}%
\pgfsetfillcolor{currentfill}%
\pgfsetfillopacity{0.901621}%
\pgfsetlinewidth{1.003750pt}%
\definecolor{currentstroke}{rgb}{0.121569,0.466667,0.705882}%
\pgfsetstrokecolor{currentstroke}%
\pgfsetstrokeopacity{0.901621}%
\pgfsetdash{}{0pt}%
\pgfpathmoveto{\pgfqpoint{1.751486in}{2.609555in}}%
\pgfpathcurveto{\pgfqpoint{1.759722in}{2.609555in}}{\pgfqpoint{1.767622in}{2.612827in}}{\pgfqpoint{1.773446in}{2.618651in}}%
\pgfpathcurveto{\pgfqpoint{1.779270in}{2.624475in}}{\pgfqpoint{1.782543in}{2.632375in}}{\pgfqpoint{1.782543in}{2.640611in}}%
\pgfpathcurveto{\pgfqpoint{1.782543in}{2.648847in}}{\pgfqpoint{1.779270in}{2.656748in}}{\pgfqpoint{1.773446in}{2.662571in}}%
\pgfpathcurveto{\pgfqpoint{1.767622in}{2.668395in}}{\pgfqpoint{1.759722in}{2.671668in}}{\pgfqpoint{1.751486in}{2.671668in}}%
\pgfpathcurveto{\pgfqpoint{1.743250in}{2.671668in}}{\pgfqpoint{1.735350in}{2.668395in}}{\pgfqpoint{1.729526in}{2.662571in}}%
\pgfpathcurveto{\pgfqpoint{1.723702in}{2.656748in}}{\pgfqpoint{1.720430in}{2.648847in}}{\pgfqpoint{1.720430in}{2.640611in}}%
\pgfpathcurveto{\pgfqpoint{1.720430in}{2.632375in}}{\pgfqpoint{1.723702in}{2.624475in}}{\pgfqpoint{1.729526in}{2.618651in}}%
\pgfpathcurveto{\pgfqpoint{1.735350in}{2.612827in}}{\pgfqpoint{1.743250in}{2.609555in}}{\pgfqpoint{1.751486in}{2.609555in}}%
\pgfpathclose%
\pgfusepath{stroke,fill}%
\end{pgfscope}%
\begin{pgfscope}%
\pgfpathrectangle{\pgfqpoint{0.100000in}{0.212622in}}{\pgfqpoint{3.696000in}{3.696000in}}%
\pgfusepath{clip}%
\pgfsetbuttcap%
\pgfsetroundjoin%
\definecolor{currentfill}{rgb}{0.121569,0.466667,0.705882}%
\pgfsetfillcolor{currentfill}%
\pgfsetfillopacity{0.901794}%
\pgfsetlinewidth{1.003750pt}%
\definecolor{currentstroke}{rgb}{0.121569,0.466667,0.705882}%
\pgfsetstrokecolor{currentstroke}%
\pgfsetstrokeopacity{0.901794}%
\pgfsetdash{}{0pt}%
\pgfpathmoveto{\pgfqpoint{2.182399in}{2.495283in}}%
\pgfpathcurveto{\pgfqpoint{2.190636in}{2.495283in}}{\pgfqpoint{2.198536in}{2.498555in}}{\pgfqpoint{2.204360in}{2.504379in}}%
\pgfpathcurveto{\pgfqpoint{2.210184in}{2.510203in}}{\pgfqpoint{2.213456in}{2.518103in}}{\pgfqpoint{2.213456in}{2.526340in}}%
\pgfpathcurveto{\pgfqpoint{2.213456in}{2.534576in}}{\pgfqpoint{2.210184in}{2.542476in}}{\pgfqpoint{2.204360in}{2.548300in}}%
\pgfpathcurveto{\pgfqpoint{2.198536in}{2.554124in}}{\pgfqpoint{2.190636in}{2.557396in}}{\pgfqpoint{2.182399in}{2.557396in}}%
\pgfpathcurveto{\pgfqpoint{2.174163in}{2.557396in}}{\pgfqpoint{2.166263in}{2.554124in}}{\pgfqpoint{2.160439in}{2.548300in}}%
\pgfpathcurveto{\pgfqpoint{2.154615in}{2.542476in}}{\pgfqpoint{2.151343in}{2.534576in}}{\pgfqpoint{2.151343in}{2.526340in}}%
\pgfpathcurveto{\pgfqpoint{2.151343in}{2.518103in}}{\pgfqpoint{2.154615in}{2.510203in}}{\pgfqpoint{2.160439in}{2.504379in}}%
\pgfpathcurveto{\pgfqpoint{2.166263in}{2.498555in}}{\pgfqpoint{2.174163in}{2.495283in}}{\pgfqpoint{2.182399in}{2.495283in}}%
\pgfpathclose%
\pgfusepath{stroke,fill}%
\end{pgfscope}%
\begin{pgfscope}%
\pgfpathrectangle{\pgfqpoint{0.100000in}{0.212622in}}{\pgfqpoint{3.696000in}{3.696000in}}%
\pgfusepath{clip}%
\pgfsetbuttcap%
\pgfsetroundjoin%
\definecolor{currentfill}{rgb}{0.121569,0.466667,0.705882}%
\pgfsetfillcolor{currentfill}%
\pgfsetfillopacity{0.901861}%
\pgfsetlinewidth{1.003750pt}%
\definecolor{currentstroke}{rgb}{0.121569,0.466667,0.705882}%
\pgfsetstrokecolor{currentstroke}%
\pgfsetstrokeopacity{0.901861}%
\pgfsetdash{}{0pt}%
\pgfpathmoveto{\pgfqpoint{1.750912in}{2.608967in}}%
\pgfpathcurveto{\pgfqpoint{1.759148in}{2.608967in}}{\pgfqpoint{1.767048in}{2.612239in}}{\pgfqpoint{1.772872in}{2.618063in}}%
\pgfpathcurveto{\pgfqpoint{1.778696in}{2.623887in}}{\pgfqpoint{1.781968in}{2.631787in}}{\pgfqpoint{1.781968in}{2.640023in}}%
\pgfpathcurveto{\pgfqpoint{1.781968in}{2.648259in}}{\pgfqpoint{1.778696in}{2.656159in}}{\pgfqpoint{1.772872in}{2.661983in}}%
\pgfpathcurveto{\pgfqpoint{1.767048in}{2.667807in}}{\pgfqpoint{1.759148in}{2.671080in}}{\pgfqpoint{1.750912in}{2.671080in}}%
\pgfpathcurveto{\pgfqpoint{1.742676in}{2.671080in}}{\pgfqpoint{1.734776in}{2.667807in}}{\pgfqpoint{1.728952in}{2.661983in}}%
\pgfpathcurveto{\pgfqpoint{1.723128in}{2.656159in}}{\pgfqpoint{1.719855in}{2.648259in}}{\pgfqpoint{1.719855in}{2.640023in}}%
\pgfpathcurveto{\pgfqpoint{1.719855in}{2.631787in}}{\pgfqpoint{1.723128in}{2.623887in}}{\pgfqpoint{1.728952in}{2.618063in}}%
\pgfpathcurveto{\pgfqpoint{1.734776in}{2.612239in}}{\pgfqpoint{1.742676in}{2.608967in}}{\pgfqpoint{1.750912in}{2.608967in}}%
\pgfpathclose%
\pgfusepath{stroke,fill}%
\end{pgfscope}%
\begin{pgfscope}%
\pgfpathrectangle{\pgfqpoint{0.100000in}{0.212622in}}{\pgfqpoint{3.696000in}{3.696000in}}%
\pgfusepath{clip}%
\pgfsetbuttcap%
\pgfsetroundjoin%
\definecolor{currentfill}{rgb}{0.121569,0.466667,0.705882}%
\pgfsetfillcolor{currentfill}%
\pgfsetfillopacity{0.901988}%
\pgfsetlinewidth{1.003750pt}%
\definecolor{currentstroke}{rgb}{0.121569,0.466667,0.705882}%
\pgfsetstrokecolor{currentstroke}%
\pgfsetstrokeopacity{0.901988}%
\pgfsetdash{}{0pt}%
\pgfpathmoveto{\pgfqpoint{1.750588in}{2.608625in}}%
\pgfpathcurveto{\pgfqpoint{1.758824in}{2.608625in}}{\pgfqpoint{1.766724in}{2.611897in}}{\pgfqpoint{1.772548in}{2.617721in}}%
\pgfpathcurveto{\pgfqpoint{1.778372in}{2.623545in}}{\pgfqpoint{1.781645in}{2.631445in}}{\pgfqpoint{1.781645in}{2.639681in}}%
\pgfpathcurveto{\pgfqpoint{1.781645in}{2.647918in}}{\pgfqpoint{1.778372in}{2.655818in}}{\pgfqpoint{1.772548in}{2.661642in}}%
\pgfpathcurveto{\pgfqpoint{1.766724in}{2.667466in}}{\pgfqpoint{1.758824in}{2.670738in}}{\pgfqpoint{1.750588in}{2.670738in}}%
\pgfpathcurveto{\pgfqpoint{1.742352in}{2.670738in}}{\pgfqpoint{1.734452in}{2.667466in}}{\pgfqpoint{1.728628in}{2.661642in}}%
\pgfpathcurveto{\pgfqpoint{1.722804in}{2.655818in}}{\pgfqpoint{1.719532in}{2.647918in}}{\pgfqpoint{1.719532in}{2.639681in}}%
\pgfpathcurveto{\pgfqpoint{1.719532in}{2.631445in}}{\pgfqpoint{1.722804in}{2.623545in}}{\pgfqpoint{1.728628in}{2.617721in}}%
\pgfpathcurveto{\pgfqpoint{1.734452in}{2.611897in}}{\pgfqpoint{1.742352in}{2.608625in}}{\pgfqpoint{1.750588in}{2.608625in}}%
\pgfpathclose%
\pgfusepath{stroke,fill}%
\end{pgfscope}%
\begin{pgfscope}%
\pgfpathrectangle{\pgfqpoint{0.100000in}{0.212622in}}{\pgfqpoint{3.696000in}{3.696000in}}%
\pgfusepath{clip}%
\pgfsetbuttcap%
\pgfsetroundjoin%
\definecolor{currentfill}{rgb}{0.121569,0.466667,0.705882}%
\pgfsetfillcolor{currentfill}%
\pgfsetfillopacity{0.902217}%
\pgfsetlinewidth{1.003750pt}%
\definecolor{currentstroke}{rgb}{0.121569,0.466667,0.705882}%
\pgfsetstrokecolor{currentstroke}%
\pgfsetstrokeopacity{0.902217}%
\pgfsetdash{}{0pt}%
\pgfpathmoveto{\pgfqpoint{2.181497in}{2.494793in}}%
\pgfpathcurveto{\pgfqpoint{2.189733in}{2.494793in}}{\pgfqpoint{2.197633in}{2.498065in}}{\pgfqpoint{2.203457in}{2.503889in}}%
\pgfpathcurveto{\pgfqpoint{2.209281in}{2.509713in}}{\pgfqpoint{2.212553in}{2.517613in}}{\pgfqpoint{2.212553in}{2.525849in}}%
\pgfpathcurveto{\pgfqpoint{2.212553in}{2.534086in}}{\pgfqpoint{2.209281in}{2.541986in}}{\pgfqpoint{2.203457in}{2.547809in}}%
\pgfpathcurveto{\pgfqpoint{2.197633in}{2.553633in}}{\pgfqpoint{2.189733in}{2.556906in}}{\pgfqpoint{2.181497in}{2.556906in}}%
\pgfpathcurveto{\pgfqpoint{2.173260in}{2.556906in}}{\pgfqpoint{2.165360in}{2.553633in}}{\pgfqpoint{2.159536in}{2.547809in}}%
\pgfpathcurveto{\pgfqpoint{2.153712in}{2.541986in}}{\pgfqpoint{2.150440in}{2.534086in}}{\pgfqpoint{2.150440in}{2.525849in}}%
\pgfpathcurveto{\pgfqpoint{2.150440in}{2.517613in}}{\pgfqpoint{2.153712in}{2.509713in}}{\pgfqpoint{2.159536in}{2.503889in}}%
\pgfpathcurveto{\pgfqpoint{2.165360in}{2.498065in}}{\pgfqpoint{2.173260in}{2.494793in}}{\pgfqpoint{2.181497in}{2.494793in}}%
\pgfpathclose%
\pgfusepath{stroke,fill}%
\end{pgfscope}%
\begin{pgfscope}%
\pgfpathrectangle{\pgfqpoint{0.100000in}{0.212622in}}{\pgfqpoint{3.696000in}{3.696000in}}%
\pgfusepath{clip}%
\pgfsetbuttcap%
\pgfsetroundjoin%
\definecolor{currentfill}{rgb}{0.121569,0.466667,0.705882}%
\pgfsetfillcolor{currentfill}%
\pgfsetfillopacity{0.902335}%
\pgfsetlinewidth{1.003750pt}%
\definecolor{currentstroke}{rgb}{0.121569,0.466667,0.705882}%
\pgfsetstrokecolor{currentstroke}%
\pgfsetstrokeopacity{0.902335}%
\pgfsetdash{}{0pt}%
\pgfpathmoveto{\pgfqpoint{1.749663in}{2.607565in}}%
\pgfpathcurveto{\pgfqpoint{1.757900in}{2.607565in}}{\pgfqpoint{1.765800in}{2.610837in}}{\pgfqpoint{1.771624in}{2.616661in}}%
\pgfpathcurveto{\pgfqpoint{1.777448in}{2.622485in}}{\pgfqpoint{1.780720in}{2.630385in}}{\pgfqpoint{1.780720in}{2.638621in}}%
\pgfpathcurveto{\pgfqpoint{1.780720in}{2.646858in}}{\pgfqpoint{1.777448in}{2.654758in}}{\pgfqpoint{1.771624in}{2.660581in}}%
\pgfpathcurveto{\pgfqpoint{1.765800in}{2.666405in}}{\pgfqpoint{1.757900in}{2.669678in}}{\pgfqpoint{1.749663in}{2.669678in}}%
\pgfpathcurveto{\pgfqpoint{1.741427in}{2.669678in}}{\pgfqpoint{1.733527in}{2.666405in}}{\pgfqpoint{1.727703in}{2.660581in}}%
\pgfpathcurveto{\pgfqpoint{1.721879in}{2.654758in}}{\pgfqpoint{1.718607in}{2.646858in}}{\pgfqpoint{1.718607in}{2.638621in}}%
\pgfpathcurveto{\pgfqpoint{1.718607in}{2.630385in}}{\pgfqpoint{1.721879in}{2.622485in}}{\pgfqpoint{1.727703in}{2.616661in}}%
\pgfpathcurveto{\pgfqpoint{1.733527in}{2.610837in}}{\pgfqpoint{1.741427in}{2.607565in}}{\pgfqpoint{1.749663in}{2.607565in}}%
\pgfpathclose%
\pgfusepath{stroke,fill}%
\end{pgfscope}%
\begin{pgfscope}%
\pgfpathrectangle{\pgfqpoint{0.100000in}{0.212622in}}{\pgfqpoint{3.696000in}{3.696000in}}%
\pgfusepath{clip}%
\pgfsetbuttcap%
\pgfsetroundjoin%
\definecolor{currentfill}{rgb}{0.121569,0.466667,0.705882}%
\pgfsetfillcolor{currentfill}%
\pgfsetfillopacity{0.902447}%
\pgfsetlinewidth{1.003750pt}%
\definecolor{currentstroke}{rgb}{0.121569,0.466667,0.705882}%
\pgfsetstrokecolor{currentstroke}%
\pgfsetstrokeopacity{0.902447}%
\pgfsetdash{}{0pt}%
\pgfpathmoveto{\pgfqpoint{2.180979in}{2.494431in}}%
\pgfpathcurveto{\pgfqpoint{2.189216in}{2.494431in}}{\pgfqpoint{2.197116in}{2.497704in}}{\pgfqpoint{2.202940in}{2.503527in}}%
\pgfpathcurveto{\pgfqpoint{2.208764in}{2.509351in}}{\pgfqpoint{2.212036in}{2.517251in}}{\pgfqpoint{2.212036in}{2.525488in}}%
\pgfpathcurveto{\pgfqpoint{2.212036in}{2.533724in}}{\pgfqpoint{2.208764in}{2.541624in}}{\pgfqpoint{2.202940in}{2.547448in}}%
\pgfpathcurveto{\pgfqpoint{2.197116in}{2.553272in}}{\pgfqpoint{2.189216in}{2.556544in}}{\pgfqpoint{2.180979in}{2.556544in}}%
\pgfpathcurveto{\pgfqpoint{2.172743in}{2.556544in}}{\pgfqpoint{2.164843in}{2.553272in}}{\pgfqpoint{2.159019in}{2.547448in}}%
\pgfpathcurveto{\pgfqpoint{2.153195in}{2.541624in}}{\pgfqpoint{2.149923in}{2.533724in}}{\pgfqpoint{2.149923in}{2.525488in}}%
\pgfpathcurveto{\pgfqpoint{2.149923in}{2.517251in}}{\pgfqpoint{2.153195in}{2.509351in}}{\pgfqpoint{2.159019in}{2.503527in}}%
\pgfpathcurveto{\pgfqpoint{2.164843in}{2.497704in}}{\pgfqpoint{2.172743in}{2.494431in}}{\pgfqpoint{2.180979in}{2.494431in}}%
\pgfpathclose%
\pgfusepath{stroke,fill}%
\end{pgfscope}%
\begin{pgfscope}%
\pgfpathrectangle{\pgfqpoint{0.100000in}{0.212622in}}{\pgfqpoint{3.696000in}{3.696000in}}%
\pgfusepath{clip}%
\pgfsetbuttcap%
\pgfsetroundjoin%
\definecolor{currentfill}{rgb}{0.121569,0.466667,0.705882}%
\pgfsetfillcolor{currentfill}%
\pgfsetfillopacity{0.902534}%
\pgfsetlinewidth{1.003750pt}%
\definecolor{currentstroke}{rgb}{0.121569,0.466667,0.705882}%
\pgfsetstrokecolor{currentstroke}%
\pgfsetstrokeopacity{0.902534}%
\pgfsetdash{}{0pt}%
\pgfpathmoveto{\pgfqpoint{1.749157in}{2.607025in}}%
\pgfpathcurveto{\pgfqpoint{1.757394in}{2.607025in}}{\pgfqpoint{1.765294in}{2.610298in}}{\pgfqpoint{1.771118in}{2.616122in}}%
\pgfpathcurveto{\pgfqpoint{1.776942in}{2.621946in}}{\pgfqpoint{1.780214in}{2.629846in}}{\pgfqpoint{1.780214in}{2.638082in}}%
\pgfpathcurveto{\pgfqpoint{1.780214in}{2.646318in}}{\pgfqpoint{1.776942in}{2.654218in}}{\pgfqpoint{1.771118in}{2.660042in}}%
\pgfpathcurveto{\pgfqpoint{1.765294in}{2.665866in}}{\pgfqpoint{1.757394in}{2.669138in}}{\pgfqpoint{1.749157in}{2.669138in}}%
\pgfpathcurveto{\pgfqpoint{1.740921in}{2.669138in}}{\pgfqpoint{1.733021in}{2.665866in}}{\pgfqpoint{1.727197in}{2.660042in}}%
\pgfpathcurveto{\pgfqpoint{1.721373in}{2.654218in}}{\pgfqpoint{1.718101in}{2.646318in}}{\pgfqpoint{1.718101in}{2.638082in}}%
\pgfpathcurveto{\pgfqpoint{1.718101in}{2.629846in}}{\pgfqpoint{1.721373in}{2.621946in}}{\pgfqpoint{1.727197in}{2.616122in}}%
\pgfpathcurveto{\pgfqpoint{1.733021in}{2.610298in}}{\pgfqpoint{1.740921in}{2.607025in}}{\pgfqpoint{1.749157in}{2.607025in}}%
\pgfpathclose%
\pgfusepath{stroke,fill}%
\end{pgfscope}%
\begin{pgfscope}%
\pgfpathrectangle{\pgfqpoint{0.100000in}{0.212622in}}{\pgfqpoint{3.696000in}{3.696000in}}%
\pgfusepath{clip}%
\pgfsetbuttcap%
\pgfsetroundjoin%
\definecolor{currentfill}{rgb}{0.121569,0.466667,0.705882}%
\pgfsetfillcolor{currentfill}%
\pgfsetfillopacity{0.902652}%
\pgfsetlinewidth{1.003750pt}%
\definecolor{currentstroke}{rgb}{0.121569,0.466667,0.705882}%
\pgfsetstrokecolor{currentstroke}%
\pgfsetstrokeopacity{0.902652}%
\pgfsetdash{}{0pt}%
\pgfpathmoveto{\pgfqpoint{1.748872in}{2.606792in}}%
\pgfpathcurveto{\pgfqpoint{1.757109in}{2.606792in}}{\pgfqpoint{1.765009in}{2.610064in}}{\pgfqpoint{1.770833in}{2.615888in}}%
\pgfpathcurveto{\pgfqpoint{1.776657in}{2.621712in}}{\pgfqpoint{1.779929in}{2.629612in}}{\pgfqpoint{1.779929in}{2.637848in}}%
\pgfpathcurveto{\pgfqpoint{1.779929in}{2.646084in}}{\pgfqpoint{1.776657in}{2.653985in}}{\pgfqpoint{1.770833in}{2.659808in}}%
\pgfpathcurveto{\pgfqpoint{1.765009in}{2.665632in}}{\pgfqpoint{1.757109in}{2.668905in}}{\pgfqpoint{1.748872in}{2.668905in}}%
\pgfpathcurveto{\pgfqpoint{1.740636in}{2.668905in}}{\pgfqpoint{1.732736in}{2.665632in}}{\pgfqpoint{1.726912in}{2.659808in}}%
\pgfpathcurveto{\pgfqpoint{1.721088in}{2.653985in}}{\pgfqpoint{1.717816in}{2.646084in}}{\pgfqpoint{1.717816in}{2.637848in}}%
\pgfpathcurveto{\pgfqpoint{1.717816in}{2.629612in}}{\pgfqpoint{1.721088in}{2.621712in}}{\pgfqpoint{1.726912in}{2.615888in}}%
\pgfpathcurveto{\pgfqpoint{1.732736in}{2.610064in}}{\pgfqpoint{1.740636in}{2.606792in}}{\pgfqpoint{1.748872in}{2.606792in}}%
\pgfpathclose%
\pgfusepath{stroke,fill}%
\end{pgfscope}%
\begin{pgfscope}%
\pgfpathrectangle{\pgfqpoint{0.100000in}{0.212622in}}{\pgfqpoint{3.696000in}{3.696000in}}%
\pgfusepath{clip}%
\pgfsetbuttcap%
\pgfsetroundjoin%
\definecolor{currentfill}{rgb}{0.121569,0.466667,0.705882}%
\pgfsetfillcolor{currentfill}%
\pgfsetfillopacity{0.902714}%
\pgfsetlinewidth{1.003750pt}%
\definecolor{currentstroke}{rgb}{0.121569,0.466667,0.705882}%
\pgfsetstrokecolor{currentstroke}%
\pgfsetstrokeopacity{0.902714}%
\pgfsetdash{}{0pt}%
\pgfpathmoveto{\pgfqpoint{1.748715in}{2.606641in}}%
\pgfpathcurveto{\pgfqpoint{1.756952in}{2.606641in}}{\pgfqpoint{1.764852in}{2.609913in}}{\pgfqpoint{1.770676in}{2.615737in}}%
\pgfpathcurveto{\pgfqpoint{1.776500in}{2.621561in}}{\pgfqpoint{1.779772in}{2.629461in}}{\pgfqpoint{1.779772in}{2.637697in}}%
\pgfpathcurveto{\pgfqpoint{1.779772in}{2.645933in}}{\pgfqpoint{1.776500in}{2.653834in}}{\pgfqpoint{1.770676in}{2.659657in}}%
\pgfpathcurveto{\pgfqpoint{1.764852in}{2.665481in}}{\pgfqpoint{1.756952in}{2.668754in}}{\pgfqpoint{1.748715in}{2.668754in}}%
\pgfpathcurveto{\pgfqpoint{1.740479in}{2.668754in}}{\pgfqpoint{1.732579in}{2.665481in}}{\pgfqpoint{1.726755in}{2.659657in}}%
\pgfpathcurveto{\pgfqpoint{1.720931in}{2.653834in}}{\pgfqpoint{1.717659in}{2.645933in}}{\pgfqpoint{1.717659in}{2.637697in}}%
\pgfpathcurveto{\pgfqpoint{1.717659in}{2.629461in}}{\pgfqpoint{1.720931in}{2.621561in}}{\pgfqpoint{1.726755in}{2.615737in}}%
\pgfpathcurveto{\pgfqpoint{1.732579in}{2.609913in}}{\pgfqpoint{1.740479in}{2.606641in}}{\pgfqpoint{1.748715in}{2.606641in}}%
\pgfpathclose%
\pgfusepath{stroke,fill}%
\end{pgfscope}%
\begin{pgfscope}%
\pgfpathrectangle{\pgfqpoint{0.100000in}{0.212622in}}{\pgfqpoint{3.696000in}{3.696000in}}%
\pgfusepath{clip}%
\pgfsetbuttcap%
\pgfsetroundjoin%
\definecolor{currentfill}{rgb}{0.121569,0.466667,0.705882}%
\pgfsetfillcolor{currentfill}%
\pgfsetfillopacity{0.902746}%
\pgfsetlinewidth{1.003750pt}%
\definecolor{currentstroke}{rgb}{0.121569,0.466667,0.705882}%
\pgfsetstrokecolor{currentstroke}%
\pgfsetstrokeopacity{0.902746}%
\pgfsetdash{}{0pt}%
\pgfpathmoveto{\pgfqpoint{1.748627in}{2.606554in}}%
\pgfpathcurveto{\pgfqpoint{1.756864in}{2.606554in}}{\pgfqpoint{1.764764in}{2.609826in}}{\pgfqpoint{1.770588in}{2.615650in}}%
\pgfpathcurveto{\pgfqpoint{1.776412in}{2.621474in}}{\pgfqpoint{1.779684in}{2.629374in}}{\pgfqpoint{1.779684in}{2.637610in}}%
\pgfpathcurveto{\pgfqpoint{1.779684in}{2.645846in}}{\pgfqpoint{1.776412in}{2.653746in}}{\pgfqpoint{1.770588in}{2.659570in}}%
\pgfpathcurveto{\pgfqpoint{1.764764in}{2.665394in}}{\pgfqpoint{1.756864in}{2.668667in}}{\pgfqpoint{1.748627in}{2.668667in}}%
\pgfpathcurveto{\pgfqpoint{1.740391in}{2.668667in}}{\pgfqpoint{1.732491in}{2.665394in}}{\pgfqpoint{1.726667in}{2.659570in}}%
\pgfpathcurveto{\pgfqpoint{1.720843in}{2.653746in}}{\pgfqpoint{1.717571in}{2.645846in}}{\pgfqpoint{1.717571in}{2.637610in}}%
\pgfpathcurveto{\pgfqpoint{1.717571in}{2.629374in}}{\pgfqpoint{1.720843in}{2.621474in}}{\pgfqpoint{1.726667in}{2.615650in}}%
\pgfpathcurveto{\pgfqpoint{1.732491in}{2.609826in}}{\pgfqpoint{1.740391in}{2.606554in}}{\pgfqpoint{1.748627in}{2.606554in}}%
\pgfpathclose%
\pgfusepath{stroke,fill}%
\end{pgfscope}%
\begin{pgfscope}%
\pgfpathrectangle{\pgfqpoint{0.100000in}{0.212622in}}{\pgfqpoint{3.696000in}{3.696000in}}%
\pgfusepath{clip}%
\pgfsetbuttcap%
\pgfsetroundjoin%
\definecolor{currentfill}{rgb}{0.121569,0.466667,0.705882}%
\pgfsetfillcolor{currentfill}%
\pgfsetfillopacity{0.902859}%
\pgfsetlinewidth{1.003750pt}%
\definecolor{currentstroke}{rgb}{0.121569,0.466667,0.705882}%
\pgfsetstrokecolor{currentstroke}%
\pgfsetstrokeopacity{0.902859}%
\pgfsetdash{}{0pt}%
\pgfpathmoveto{\pgfqpoint{2.180019in}{2.493767in}}%
\pgfpathcurveto{\pgfqpoint{2.188255in}{2.493767in}}{\pgfqpoint{2.196155in}{2.497039in}}{\pgfqpoint{2.201979in}{2.502863in}}%
\pgfpathcurveto{\pgfqpoint{2.207803in}{2.508687in}}{\pgfqpoint{2.211076in}{2.516587in}}{\pgfqpoint{2.211076in}{2.524823in}}%
\pgfpathcurveto{\pgfqpoint{2.211076in}{2.533060in}}{\pgfqpoint{2.207803in}{2.540960in}}{\pgfqpoint{2.201979in}{2.546784in}}%
\pgfpathcurveto{\pgfqpoint{2.196155in}{2.552607in}}{\pgfqpoint{2.188255in}{2.555880in}}{\pgfqpoint{2.180019in}{2.555880in}}%
\pgfpathcurveto{\pgfqpoint{2.171783in}{2.555880in}}{\pgfqpoint{2.163883in}{2.552607in}}{\pgfqpoint{2.158059in}{2.546784in}}%
\pgfpathcurveto{\pgfqpoint{2.152235in}{2.540960in}}{\pgfqpoint{2.148963in}{2.533060in}}{\pgfqpoint{2.148963in}{2.524823in}}%
\pgfpathcurveto{\pgfqpoint{2.148963in}{2.516587in}}{\pgfqpoint{2.152235in}{2.508687in}}{\pgfqpoint{2.158059in}{2.502863in}}%
\pgfpathcurveto{\pgfqpoint{2.163883in}{2.497039in}}{\pgfqpoint{2.171783in}{2.493767in}}{\pgfqpoint{2.180019in}{2.493767in}}%
\pgfpathclose%
\pgfusepath{stroke,fill}%
\end{pgfscope}%
\begin{pgfscope}%
\pgfpathrectangle{\pgfqpoint{0.100000in}{0.212622in}}{\pgfqpoint{3.696000in}{3.696000in}}%
\pgfusepath{clip}%
\pgfsetbuttcap%
\pgfsetroundjoin%
\definecolor{currentfill}{rgb}{0.121569,0.466667,0.705882}%
\pgfsetfillcolor{currentfill}%
\pgfsetfillopacity{0.902867}%
\pgfsetlinewidth{1.003750pt}%
\definecolor{currentstroke}{rgb}{0.121569,0.466667,0.705882}%
\pgfsetstrokecolor{currentstroke}%
\pgfsetstrokeopacity{0.902867}%
\pgfsetdash{}{0pt}%
\pgfpathmoveto{\pgfqpoint{2.674467in}{1.300033in}}%
\pgfpathcurveto{\pgfqpoint{2.682703in}{1.300033in}}{\pgfqpoint{2.690603in}{1.303306in}}{\pgfqpoint{2.696427in}{1.309130in}}%
\pgfpathcurveto{\pgfqpoint{2.702251in}{1.314954in}}{\pgfqpoint{2.705523in}{1.322854in}}{\pgfqpoint{2.705523in}{1.331090in}}%
\pgfpathcurveto{\pgfqpoint{2.705523in}{1.339326in}}{\pgfqpoint{2.702251in}{1.347226in}}{\pgfqpoint{2.696427in}{1.353050in}}%
\pgfpathcurveto{\pgfqpoint{2.690603in}{1.358874in}}{\pgfqpoint{2.682703in}{1.362146in}}{\pgfqpoint{2.674467in}{1.362146in}}%
\pgfpathcurveto{\pgfqpoint{2.666231in}{1.362146in}}{\pgfqpoint{2.658331in}{1.358874in}}{\pgfqpoint{2.652507in}{1.353050in}}%
\pgfpathcurveto{\pgfqpoint{2.646683in}{1.347226in}}{\pgfqpoint{2.643410in}{1.339326in}}{\pgfqpoint{2.643410in}{1.331090in}}%
\pgfpathcurveto{\pgfqpoint{2.643410in}{1.322854in}}{\pgfqpoint{2.646683in}{1.314954in}}{\pgfqpoint{2.652507in}{1.309130in}}%
\pgfpathcurveto{\pgfqpoint{2.658331in}{1.303306in}}{\pgfqpoint{2.666231in}{1.300033in}}{\pgfqpoint{2.674467in}{1.300033in}}%
\pgfpathclose%
\pgfusepath{stroke,fill}%
\end{pgfscope}%
\begin{pgfscope}%
\pgfpathrectangle{\pgfqpoint{0.100000in}{0.212622in}}{\pgfqpoint{3.696000in}{3.696000in}}%
\pgfusepath{clip}%
\pgfsetbuttcap%
\pgfsetroundjoin%
\definecolor{currentfill}{rgb}{0.121569,0.466667,0.705882}%
\pgfsetfillcolor{currentfill}%
\pgfsetfillopacity{0.902897}%
\pgfsetlinewidth{1.003750pt}%
\definecolor{currentstroke}{rgb}{0.121569,0.466667,0.705882}%
\pgfsetstrokecolor{currentstroke}%
\pgfsetstrokeopacity{0.902897}%
\pgfsetdash{}{0pt}%
\pgfpathmoveto{\pgfqpoint{1.748249in}{2.606211in}}%
\pgfpathcurveto{\pgfqpoint{1.756485in}{2.606211in}}{\pgfqpoint{1.764385in}{2.609483in}}{\pgfqpoint{1.770209in}{2.615307in}}%
\pgfpathcurveto{\pgfqpoint{1.776033in}{2.621131in}}{\pgfqpoint{1.779305in}{2.629031in}}{\pgfqpoint{1.779305in}{2.637267in}}%
\pgfpathcurveto{\pgfqpoint{1.779305in}{2.645503in}}{\pgfqpoint{1.776033in}{2.653403in}}{\pgfqpoint{1.770209in}{2.659227in}}%
\pgfpathcurveto{\pgfqpoint{1.764385in}{2.665051in}}{\pgfqpoint{1.756485in}{2.668324in}}{\pgfqpoint{1.748249in}{2.668324in}}%
\pgfpathcurveto{\pgfqpoint{1.740012in}{2.668324in}}{\pgfqpoint{1.732112in}{2.665051in}}{\pgfqpoint{1.726289in}{2.659227in}}%
\pgfpathcurveto{\pgfqpoint{1.720465in}{2.653403in}}{\pgfqpoint{1.717192in}{2.645503in}}{\pgfqpoint{1.717192in}{2.637267in}}%
\pgfpathcurveto{\pgfqpoint{1.717192in}{2.629031in}}{\pgfqpoint{1.720465in}{2.621131in}}{\pgfqpoint{1.726289in}{2.615307in}}%
\pgfpathcurveto{\pgfqpoint{1.732112in}{2.609483in}}{\pgfqpoint{1.740012in}{2.606211in}}{\pgfqpoint{1.748249in}{2.606211in}}%
\pgfpathclose%
\pgfusepath{stroke,fill}%
\end{pgfscope}%
\begin{pgfscope}%
\pgfpathrectangle{\pgfqpoint{0.100000in}{0.212622in}}{\pgfqpoint{3.696000in}{3.696000in}}%
\pgfusepath{clip}%
\pgfsetbuttcap%
\pgfsetroundjoin%
\definecolor{currentfill}{rgb}{0.121569,0.466667,0.705882}%
\pgfsetfillcolor{currentfill}%
\pgfsetfillopacity{0.902977}%
\pgfsetlinewidth{1.003750pt}%
\definecolor{currentstroke}{rgb}{0.121569,0.466667,0.705882}%
\pgfsetstrokecolor{currentstroke}%
\pgfsetstrokeopacity{0.902977}%
\pgfsetdash{}{0pt}%
\pgfpathmoveto{\pgfqpoint{1.748034in}{2.606019in}}%
\pgfpathcurveto{\pgfqpoint{1.756271in}{2.606019in}}{\pgfqpoint{1.764171in}{2.609292in}}{\pgfqpoint{1.769995in}{2.615116in}}%
\pgfpathcurveto{\pgfqpoint{1.775819in}{2.620939in}}{\pgfqpoint{1.779091in}{2.628840in}}{\pgfqpoint{1.779091in}{2.637076in}}%
\pgfpathcurveto{\pgfqpoint{1.779091in}{2.645312in}}{\pgfqpoint{1.775819in}{2.653212in}}{\pgfqpoint{1.769995in}{2.659036in}}%
\pgfpathcurveto{\pgfqpoint{1.764171in}{2.664860in}}{\pgfqpoint{1.756271in}{2.668132in}}{\pgfqpoint{1.748034in}{2.668132in}}%
\pgfpathcurveto{\pgfqpoint{1.739798in}{2.668132in}}{\pgfqpoint{1.731898in}{2.664860in}}{\pgfqpoint{1.726074in}{2.659036in}}%
\pgfpathcurveto{\pgfqpoint{1.720250in}{2.653212in}}{\pgfqpoint{1.716978in}{2.645312in}}{\pgfqpoint{1.716978in}{2.637076in}}%
\pgfpathcurveto{\pgfqpoint{1.716978in}{2.628840in}}{\pgfqpoint{1.720250in}{2.620939in}}{\pgfqpoint{1.726074in}{2.615116in}}%
\pgfpathcurveto{\pgfqpoint{1.731898in}{2.609292in}}{\pgfqpoint{1.739798in}{2.606019in}}{\pgfqpoint{1.748034in}{2.606019in}}%
\pgfpathclose%
\pgfusepath{stroke,fill}%
\end{pgfscope}%
\begin{pgfscope}%
\pgfpathrectangle{\pgfqpoint{0.100000in}{0.212622in}}{\pgfqpoint{3.696000in}{3.696000in}}%
\pgfusepath{clip}%
\pgfsetbuttcap%
\pgfsetroundjoin%
\definecolor{currentfill}{rgb}{0.121569,0.466667,0.705882}%
\pgfsetfillcolor{currentfill}%
\pgfsetfillopacity{0.903113}%
\pgfsetlinewidth{1.003750pt}%
\definecolor{currentstroke}{rgb}{0.121569,0.466667,0.705882}%
\pgfsetstrokecolor{currentstroke}%
\pgfsetstrokeopacity{0.903113}%
\pgfsetdash{}{0pt}%
\pgfpathmoveto{\pgfqpoint{2.179455in}{2.493427in}}%
\pgfpathcurveto{\pgfqpoint{2.187692in}{2.493427in}}{\pgfqpoint{2.195592in}{2.496700in}}{\pgfqpoint{2.201416in}{2.502524in}}%
\pgfpathcurveto{\pgfqpoint{2.207240in}{2.508347in}}{\pgfqpoint{2.210512in}{2.516248in}}{\pgfqpoint{2.210512in}{2.524484in}}%
\pgfpathcurveto{\pgfqpoint{2.210512in}{2.532720in}}{\pgfqpoint{2.207240in}{2.540620in}}{\pgfqpoint{2.201416in}{2.546444in}}%
\pgfpathcurveto{\pgfqpoint{2.195592in}{2.552268in}}{\pgfqpoint{2.187692in}{2.555540in}}{\pgfqpoint{2.179455in}{2.555540in}}%
\pgfpathcurveto{\pgfqpoint{2.171219in}{2.555540in}}{\pgfqpoint{2.163319in}{2.552268in}}{\pgfqpoint{2.157495in}{2.546444in}}%
\pgfpathcurveto{\pgfqpoint{2.151671in}{2.540620in}}{\pgfqpoint{2.148399in}{2.532720in}}{\pgfqpoint{2.148399in}{2.524484in}}%
\pgfpathcurveto{\pgfqpoint{2.148399in}{2.516248in}}{\pgfqpoint{2.151671in}{2.508347in}}{\pgfqpoint{2.157495in}{2.502524in}}%
\pgfpathcurveto{\pgfqpoint{2.163319in}{2.496700in}}{\pgfqpoint{2.171219in}{2.493427in}}{\pgfqpoint{2.179455in}{2.493427in}}%
\pgfpathclose%
\pgfusepath{stroke,fill}%
\end{pgfscope}%
\begin{pgfscope}%
\pgfpathrectangle{\pgfqpoint{0.100000in}{0.212622in}}{\pgfqpoint{3.696000in}{3.696000in}}%
\pgfusepath{clip}%
\pgfsetbuttcap%
\pgfsetroundjoin%
\definecolor{currentfill}{rgb}{0.121569,0.466667,0.705882}%
\pgfsetfillcolor{currentfill}%
\pgfsetfillopacity{0.903152}%
\pgfsetlinewidth{1.003750pt}%
\definecolor{currentstroke}{rgb}{0.121569,0.466667,0.705882}%
\pgfsetstrokecolor{currentstroke}%
\pgfsetstrokeopacity{0.903152}%
\pgfsetdash{}{0pt}%
\pgfpathmoveto{\pgfqpoint{0.953679in}{2.067248in}}%
\pgfpathcurveto{\pgfqpoint{0.961915in}{2.067248in}}{\pgfqpoint{0.969815in}{2.070521in}}{\pgfqpoint{0.975639in}{2.076345in}}%
\pgfpathcurveto{\pgfqpoint{0.981463in}{2.082169in}}{\pgfqpoint{0.984735in}{2.090069in}}{\pgfqpoint{0.984735in}{2.098305in}}%
\pgfpathcurveto{\pgfqpoint{0.984735in}{2.106541in}}{\pgfqpoint{0.981463in}{2.114441in}}{\pgfqpoint{0.975639in}{2.120265in}}%
\pgfpathcurveto{\pgfqpoint{0.969815in}{2.126089in}}{\pgfqpoint{0.961915in}{2.129361in}}{\pgfqpoint{0.953679in}{2.129361in}}%
\pgfpathcurveto{\pgfqpoint{0.945443in}{2.129361in}}{\pgfqpoint{0.937543in}{2.126089in}}{\pgfqpoint{0.931719in}{2.120265in}}%
\pgfpathcurveto{\pgfqpoint{0.925895in}{2.114441in}}{\pgfqpoint{0.922622in}{2.106541in}}{\pgfqpoint{0.922622in}{2.098305in}}%
\pgfpathcurveto{\pgfqpoint{0.922622in}{2.090069in}}{\pgfqpoint{0.925895in}{2.082169in}}{\pgfqpoint{0.931719in}{2.076345in}}%
\pgfpathcurveto{\pgfqpoint{0.937543in}{2.070521in}}{\pgfqpoint{0.945443in}{2.067248in}}{\pgfqpoint{0.953679in}{2.067248in}}%
\pgfpathclose%
\pgfusepath{stroke,fill}%
\end{pgfscope}%
\begin{pgfscope}%
\pgfpathrectangle{\pgfqpoint{0.100000in}{0.212622in}}{\pgfqpoint{3.696000in}{3.696000in}}%
\pgfusepath{clip}%
\pgfsetbuttcap%
\pgfsetroundjoin%
\definecolor{currentfill}{rgb}{0.121569,0.466667,0.705882}%
\pgfsetfillcolor{currentfill}%
\pgfsetfillopacity{0.903190}%
\pgfsetlinewidth{1.003750pt}%
\definecolor{currentstroke}{rgb}{0.121569,0.466667,0.705882}%
\pgfsetstrokecolor{currentstroke}%
\pgfsetstrokeopacity{0.903190}%
\pgfsetdash{}{0pt}%
\pgfpathmoveto{\pgfqpoint{1.747429in}{2.605408in}}%
\pgfpathcurveto{\pgfqpoint{1.755666in}{2.605408in}}{\pgfqpoint{1.763566in}{2.608680in}}{\pgfqpoint{1.769390in}{2.614504in}}%
\pgfpathcurveto{\pgfqpoint{1.775213in}{2.620328in}}{\pgfqpoint{1.778486in}{2.628228in}}{\pgfqpoint{1.778486in}{2.636464in}}%
\pgfpathcurveto{\pgfqpoint{1.778486in}{2.644700in}}{\pgfqpoint{1.775213in}{2.652600in}}{\pgfqpoint{1.769390in}{2.658424in}}%
\pgfpathcurveto{\pgfqpoint{1.763566in}{2.664248in}}{\pgfqpoint{1.755666in}{2.667521in}}{\pgfqpoint{1.747429in}{2.667521in}}%
\pgfpathcurveto{\pgfqpoint{1.739193in}{2.667521in}}{\pgfqpoint{1.731293in}{2.664248in}}{\pgfqpoint{1.725469in}{2.658424in}}%
\pgfpathcurveto{\pgfqpoint{1.719645in}{2.652600in}}{\pgfqpoint{1.716373in}{2.644700in}}{\pgfqpoint{1.716373in}{2.636464in}}%
\pgfpathcurveto{\pgfqpoint{1.716373in}{2.628228in}}{\pgfqpoint{1.719645in}{2.620328in}}{\pgfqpoint{1.725469in}{2.614504in}}%
\pgfpathcurveto{\pgfqpoint{1.731293in}{2.608680in}}{\pgfqpoint{1.739193in}{2.605408in}}{\pgfqpoint{1.747429in}{2.605408in}}%
\pgfpathclose%
\pgfusepath{stroke,fill}%
\end{pgfscope}%
\begin{pgfscope}%
\pgfpathrectangle{\pgfqpoint{0.100000in}{0.212622in}}{\pgfqpoint{3.696000in}{3.696000in}}%
\pgfusepath{clip}%
\pgfsetbuttcap%
\pgfsetroundjoin%
\definecolor{currentfill}{rgb}{0.121569,0.466667,0.705882}%
\pgfsetfillcolor{currentfill}%
\pgfsetfillopacity{0.903490}%
\pgfsetlinewidth{1.003750pt}%
\definecolor{currentstroke}{rgb}{0.121569,0.466667,0.705882}%
\pgfsetstrokecolor{currentstroke}%
\pgfsetstrokeopacity{0.903490}%
\pgfsetdash{}{0pt}%
\pgfpathmoveto{\pgfqpoint{1.746637in}{2.604558in}}%
\pgfpathcurveto{\pgfqpoint{1.754873in}{2.604558in}}{\pgfqpoint{1.762774in}{2.607831in}}{\pgfqpoint{1.768597in}{2.613655in}}%
\pgfpathcurveto{\pgfqpoint{1.774421in}{2.619478in}}{\pgfqpoint{1.777694in}{2.627379in}}{\pgfqpoint{1.777694in}{2.635615in}}%
\pgfpathcurveto{\pgfqpoint{1.777694in}{2.643851in}}{\pgfqpoint{1.774421in}{2.651751in}}{\pgfqpoint{1.768597in}{2.657575in}}%
\pgfpathcurveto{\pgfqpoint{1.762774in}{2.663399in}}{\pgfqpoint{1.754873in}{2.666671in}}{\pgfqpoint{1.746637in}{2.666671in}}%
\pgfpathcurveto{\pgfqpoint{1.738401in}{2.666671in}}{\pgfqpoint{1.730501in}{2.663399in}}{\pgfqpoint{1.724677in}{2.657575in}}%
\pgfpathcurveto{\pgfqpoint{1.718853in}{2.651751in}}{\pgfqpoint{1.715581in}{2.643851in}}{\pgfqpoint{1.715581in}{2.635615in}}%
\pgfpathcurveto{\pgfqpoint{1.715581in}{2.627379in}}{\pgfqpoint{1.718853in}{2.619478in}}{\pgfqpoint{1.724677in}{2.613655in}}%
\pgfpathcurveto{\pgfqpoint{1.730501in}{2.607831in}}{\pgfqpoint{1.738401in}{2.604558in}}{\pgfqpoint{1.746637in}{2.604558in}}%
\pgfpathclose%
\pgfusepath{stroke,fill}%
\end{pgfscope}%
\begin{pgfscope}%
\pgfpathrectangle{\pgfqpoint{0.100000in}{0.212622in}}{\pgfqpoint{3.696000in}{3.696000in}}%
\pgfusepath{clip}%
\pgfsetbuttcap%
\pgfsetroundjoin%
\definecolor{currentfill}{rgb}{0.121569,0.466667,0.705882}%
\pgfsetfillcolor{currentfill}%
\pgfsetfillopacity{0.903587}%
\pgfsetlinewidth{1.003750pt}%
\definecolor{currentstroke}{rgb}{0.121569,0.466667,0.705882}%
\pgfsetstrokecolor{currentstroke}%
\pgfsetstrokeopacity{0.903587}%
\pgfsetdash{}{0pt}%
\pgfpathmoveto{\pgfqpoint{2.178448in}{2.492861in}}%
\pgfpathcurveto{\pgfqpoint{2.186684in}{2.492861in}}{\pgfqpoint{2.194584in}{2.496133in}}{\pgfqpoint{2.200408in}{2.501957in}}%
\pgfpathcurveto{\pgfqpoint{2.206232in}{2.507781in}}{\pgfqpoint{2.209504in}{2.515681in}}{\pgfqpoint{2.209504in}{2.523918in}}%
\pgfpathcurveto{\pgfqpoint{2.209504in}{2.532154in}}{\pgfqpoint{2.206232in}{2.540054in}}{\pgfqpoint{2.200408in}{2.545878in}}%
\pgfpathcurveto{\pgfqpoint{2.194584in}{2.551702in}}{\pgfqpoint{2.186684in}{2.554974in}}{\pgfqpoint{2.178448in}{2.554974in}}%
\pgfpathcurveto{\pgfqpoint{2.170212in}{2.554974in}}{\pgfqpoint{2.162311in}{2.551702in}}{\pgfqpoint{2.156488in}{2.545878in}}%
\pgfpathcurveto{\pgfqpoint{2.150664in}{2.540054in}}{\pgfqpoint{2.147391in}{2.532154in}}{\pgfqpoint{2.147391in}{2.523918in}}%
\pgfpathcurveto{\pgfqpoint{2.147391in}{2.515681in}}{\pgfqpoint{2.150664in}{2.507781in}}{\pgfqpoint{2.156488in}{2.501957in}}%
\pgfpathcurveto{\pgfqpoint{2.162311in}{2.496133in}}{\pgfqpoint{2.170212in}{2.492861in}}{\pgfqpoint{2.178448in}{2.492861in}}%
\pgfpathclose%
\pgfusepath{stroke,fill}%
\end{pgfscope}%
\begin{pgfscope}%
\pgfpathrectangle{\pgfqpoint{0.100000in}{0.212622in}}{\pgfqpoint{3.696000in}{3.696000in}}%
\pgfusepath{clip}%
\pgfsetbuttcap%
\pgfsetroundjoin%
\definecolor{currentfill}{rgb}{0.121569,0.466667,0.705882}%
\pgfsetfillcolor{currentfill}%
\pgfsetfillopacity{0.903660}%
\pgfsetlinewidth{1.003750pt}%
\definecolor{currentstroke}{rgb}{0.121569,0.466667,0.705882}%
\pgfsetstrokecolor{currentstroke}%
\pgfsetstrokeopacity{0.903660}%
\pgfsetdash{}{0pt}%
\pgfpathmoveto{\pgfqpoint{1.746193in}{2.604133in}}%
\pgfpathcurveto{\pgfqpoint{1.754430in}{2.604133in}}{\pgfqpoint{1.762330in}{2.607406in}}{\pgfqpoint{1.768154in}{2.613229in}}%
\pgfpathcurveto{\pgfqpoint{1.773977in}{2.619053in}}{\pgfqpoint{1.777250in}{2.626953in}}{\pgfqpoint{1.777250in}{2.635190in}}%
\pgfpathcurveto{\pgfqpoint{1.777250in}{2.643426in}}{\pgfqpoint{1.773977in}{2.651326in}}{\pgfqpoint{1.768154in}{2.657150in}}%
\pgfpathcurveto{\pgfqpoint{1.762330in}{2.662974in}}{\pgfqpoint{1.754430in}{2.666246in}}{\pgfqpoint{1.746193in}{2.666246in}}%
\pgfpathcurveto{\pgfqpoint{1.737957in}{2.666246in}}{\pgfqpoint{1.730057in}{2.662974in}}{\pgfqpoint{1.724233in}{2.657150in}}%
\pgfpathcurveto{\pgfqpoint{1.718409in}{2.651326in}}{\pgfqpoint{1.715137in}{2.643426in}}{\pgfqpoint{1.715137in}{2.635190in}}%
\pgfpathcurveto{\pgfqpoint{1.715137in}{2.626953in}}{\pgfqpoint{1.718409in}{2.619053in}}{\pgfqpoint{1.724233in}{2.613229in}}%
\pgfpathcurveto{\pgfqpoint{1.730057in}{2.607406in}}{\pgfqpoint{1.737957in}{2.604133in}}{\pgfqpoint{1.746193in}{2.604133in}}%
\pgfpathclose%
\pgfusepath{stroke,fill}%
\end{pgfscope}%
\begin{pgfscope}%
\pgfpathrectangle{\pgfqpoint{0.100000in}{0.212622in}}{\pgfqpoint{3.696000in}{3.696000in}}%
\pgfusepath{clip}%
\pgfsetbuttcap%
\pgfsetroundjoin%
\definecolor{currentfill}{rgb}{0.121569,0.466667,0.705882}%
\pgfsetfillcolor{currentfill}%
\pgfsetfillopacity{0.903752}%
\pgfsetlinewidth{1.003750pt}%
\definecolor{currentstroke}{rgb}{0.121569,0.466667,0.705882}%
\pgfsetstrokecolor{currentstroke}%
\pgfsetstrokeopacity{0.903752}%
\pgfsetdash{}{0pt}%
\pgfpathmoveto{\pgfqpoint{1.745943in}{2.603894in}}%
\pgfpathcurveto{\pgfqpoint{1.754180in}{2.603894in}}{\pgfqpoint{1.762080in}{2.607167in}}{\pgfqpoint{1.767904in}{2.612991in}}%
\pgfpathcurveto{\pgfqpoint{1.773728in}{2.618815in}}{\pgfqpoint{1.777000in}{2.626715in}}{\pgfqpoint{1.777000in}{2.634951in}}%
\pgfpathcurveto{\pgfqpoint{1.777000in}{2.643187in}}{\pgfqpoint{1.773728in}{2.651087in}}{\pgfqpoint{1.767904in}{2.656911in}}%
\pgfpathcurveto{\pgfqpoint{1.762080in}{2.662735in}}{\pgfqpoint{1.754180in}{2.666007in}}{\pgfqpoint{1.745943in}{2.666007in}}%
\pgfpathcurveto{\pgfqpoint{1.737707in}{2.666007in}}{\pgfqpoint{1.729807in}{2.662735in}}{\pgfqpoint{1.723983in}{2.656911in}}%
\pgfpathcurveto{\pgfqpoint{1.718159in}{2.651087in}}{\pgfqpoint{1.714887in}{2.643187in}}{\pgfqpoint{1.714887in}{2.634951in}}%
\pgfpathcurveto{\pgfqpoint{1.714887in}{2.626715in}}{\pgfqpoint{1.718159in}{2.618815in}}{\pgfqpoint{1.723983in}{2.612991in}}%
\pgfpathcurveto{\pgfqpoint{1.729807in}{2.607167in}}{\pgfqpoint{1.737707in}{2.603894in}}{\pgfqpoint{1.745943in}{2.603894in}}%
\pgfpathclose%
\pgfusepath{stroke,fill}%
\end{pgfscope}%
\begin{pgfscope}%
\pgfpathrectangle{\pgfqpoint{0.100000in}{0.212622in}}{\pgfqpoint{3.696000in}{3.696000in}}%
\pgfusepath{clip}%
\pgfsetbuttcap%
\pgfsetroundjoin%
\definecolor{currentfill}{rgb}{0.121569,0.466667,0.705882}%
\pgfsetfillcolor{currentfill}%
\pgfsetfillopacity{0.903980}%
\pgfsetlinewidth{1.003750pt}%
\definecolor{currentstroke}{rgb}{0.121569,0.466667,0.705882}%
\pgfsetstrokecolor{currentstroke}%
\pgfsetstrokeopacity{0.903980}%
\pgfsetdash{}{0pt}%
\pgfpathmoveto{\pgfqpoint{1.745290in}{2.603187in}}%
\pgfpathcurveto{\pgfqpoint{1.753526in}{2.603187in}}{\pgfqpoint{1.761426in}{2.606460in}}{\pgfqpoint{1.767250in}{2.612284in}}%
\pgfpathcurveto{\pgfqpoint{1.773074in}{2.618108in}}{\pgfqpoint{1.776346in}{2.626008in}}{\pgfqpoint{1.776346in}{2.634244in}}%
\pgfpathcurveto{\pgfqpoint{1.776346in}{2.642480in}}{\pgfqpoint{1.773074in}{2.650380in}}{\pgfqpoint{1.767250in}{2.656204in}}%
\pgfpathcurveto{\pgfqpoint{1.761426in}{2.662028in}}{\pgfqpoint{1.753526in}{2.665300in}}{\pgfqpoint{1.745290in}{2.665300in}}%
\pgfpathcurveto{\pgfqpoint{1.737053in}{2.665300in}}{\pgfqpoint{1.729153in}{2.662028in}}{\pgfqpoint{1.723329in}{2.656204in}}%
\pgfpathcurveto{\pgfqpoint{1.717505in}{2.650380in}}{\pgfqpoint{1.714233in}{2.642480in}}{\pgfqpoint{1.714233in}{2.634244in}}%
\pgfpathcurveto{\pgfqpoint{1.714233in}{2.626008in}}{\pgfqpoint{1.717505in}{2.618108in}}{\pgfqpoint{1.723329in}{2.612284in}}%
\pgfpathcurveto{\pgfqpoint{1.729153in}{2.606460in}}{\pgfqpoint{1.737053in}{2.603187in}}{\pgfqpoint{1.745290in}{2.603187in}}%
\pgfpathclose%
\pgfusepath{stroke,fill}%
\end{pgfscope}%
\begin{pgfscope}%
\pgfpathrectangle{\pgfqpoint{0.100000in}{0.212622in}}{\pgfqpoint{3.696000in}{3.696000in}}%
\pgfusepath{clip}%
\pgfsetbuttcap%
\pgfsetroundjoin%
\definecolor{currentfill}{rgb}{0.121569,0.466667,0.705882}%
\pgfsetfillcolor{currentfill}%
\pgfsetfillopacity{0.904323}%
\pgfsetlinewidth{1.003750pt}%
\definecolor{currentstroke}{rgb}{0.121569,0.466667,0.705882}%
\pgfsetstrokecolor{currentstroke}%
\pgfsetstrokeopacity{0.904323}%
\pgfsetdash{}{0pt}%
\pgfpathmoveto{\pgfqpoint{1.744392in}{2.602255in}}%
\pgfpathcurveto{\pgfqpoint{1.752628in}{2.602255in}}{\pgfqpoint{1.760529in}{2.605527in}}{\pgfqpoint{1.766352in}{2.611351in}}%
\pgfpathcurveto{\pgfqpoint{1.772176in}{2.617175in}}{\pgfqpoint{1.775449in}{2.625075in}}{\pgfqpoint{1.775449in}{2.633311in}}%
\pgfpathcurveto{\pgfqpoint{1.775449in}{2.641548in}}{\pgfqpoint{1.772176in}{2.649448in}}{\pgfqpoint{1.766352in}{2.655272in}}%
\pgfpathcurveto{\pgfqpoint{1.760529in}{2.661096in}}{\pgfqpoint{1.752628in}{2.664368in}}{\pgfqpoint{1.744392in}{2.664368in}}%
\pgfpathcurveto{\pgfqpoint{1.736156in}{2.664368in}}{\pgfqpoint{1.728256in}{2.661096in}}{\pgfqpoint{1.722432in}{2.655272in}}%
\pgfpathcurveto{\pgfqpoint{1.716608in}{2.649448in}}{\pgfqpoint{1.713336in}{2.641548in}}{\pgfqpoint{1.713336in}{2.633311in}}%
\pgfpathcurveto{\pgfqpoint{1.713336in}{2.625075in}}{\pgfqpoint{1.716608in}{2.617175in}}{\pgfqpoint{1.722432in}{2.611351in}}%
\pgfpathcurveto{\pgfqpoint{1.728256in}{2.605527in}}{\pgfqpoint{1.736156in}{2.602255in}}{\pgfqpoint{1.744392in}{2.602255in}}%
\pgfpathclose%
\pgfusepath{stroke,fill}%
\end{pgfscope}%
\begin{pgfscope}%
\pgfpathrectangle{\pgfqpoint{0.100000in}{0.212622in}}{\pgfqpoint{3.696000in}{3.696000in}}%
\pgfusepath{clip}%
\pgfsetbuttcap%
\pgfsetroundjoin%
\definecolor{currentfill}{rgb}{0.121569,0.466667,0.705882}%
\pgfsetfillcolor{currentfill}%
\pgfsetfillopacity{0.904411}%
\pgfsetlinewidth{1.003750pt}%
\definecolor{currentstroke}{rgb}{0.121569,0.466667,0.705882}%
\pgfsetstrokecolor{currentstroke}%
\pgfsetstrokeopacity{0.904411}%
\pgfsetdash{}{0pt}%
\pgfpathmoveto{\pgfqpoint{2.176647in}{2.491570in}}%
\pgfpathcurveto{\pgfqpoint{2.184883in}{2.491570in}}{\pgfqpoint{2.192783in}{2.494842in}}{\pgfqpoint{2.198607in}{2.500666in}}%
\pgfpathcurveto{\pgfqpoint{2.204431in}{2.506490in}}{\pgfqpoint{2.207704in}{2.514390in}}{\pgfqpoint{2.207704in}{2.522626in}}%
\pgfpathcurveto{\pgfqpoint{2.207704in}{2.530862in}}{\pgfqpoint{2.204431in}{2.538762in}}{\pgfqpoint{2.198607in}{2.544586in}}%
\pgfpathcurveto{\pgfqpoint{2.192783in}{2.550410in}}{\pgfqpoint{2.184883in}{2.553683in}}{\pgfqpoint{2.176647in}{2.553683in}}%
\pgfpathcurveto{\pgfqpoint{2.168411in}{2.553683in}}{\pgfqpoint{2.160511in}{2.550410in}}{\pgfqpoint{2.154687in}{2.544586in}}%
\pgfpathcurveto{\pgfqpoint{2.148863in}{2.538762in}}{\pgfqpoint{2.145591in}{2.530862in}}{\pgfqpoint{2.145591in}{2.522626in}}%
\pgfpathcurveto{\pgfqpoint{2.145591in}{2.514390in}}{\pgfqpoint{2.148863in}{2.506490in}}{\pgfqpoint{2.154687in}{2.500666in}}%
\pgfpathcurveto{\pgfqpoint{2.160511in}{2.494842in}}{\pgfqpoint{2.168411in}{2.491570in}}{\pgfqpoint{2.176647in}{2.491570in}}%
\pgfpathclose%
\pgfusepath{stroke,fill}%
\end{pgfscope}%
\begin{pgfscope}%
\pgfpathrectangle{\pgfqpoint{0.100000in}{0.212622in}}{\pgfqpoint{3.696000in}{3.696000in}}%
\pgfusepath{clip}%
\pgfsetbuttcap%
\pgfsetroundjoin%
\definecolor{currentfill}{rgb}{0.121569,0.466667,0.705882}%
\pgfsetfillcolor{currentfill}%
\pgfsetfillopacity{0.904783}%
\pgfsetlinewidth{1.003750pt}%
\definecolor{currentstroke}{rgb}{0.121569,0.466667,0.705882}%
\pgfsetstrokecolor{currentstroke}%
\pgfsetstrokeopacity{0.904783}%
\pgfsetdash{}{0pt}%
\pgfpathmoveto{\pgfqpoint{1.743263in}{2.601260in}}%
\pgfpathcurveto{\pgfqpoint{1.751499in}{2.601260in}}{\pgfqpoint{1.759399in}{2.604532in}}{\pgfqpoint{1.765223in}{2.610356in}}%
\pgfpathcurveto{\pgfqpoint{1.771047in}{2.616180in}}{\pgfqpoint{1.774319in}{2.624080in}}{\pgfqpoint{1.774319in}{2.632316in}}%
\pgfpathcurveto{\pgfqpoint{1.774319in}{2.640552in}}{\pgfqpoint{1.771047in}{2.648452in}}{\pgfqpoint{1.765223in}{2.654276in}}%
\pgfpathcurveto{\pgfqpoint{1.759399in}{2.660100in}}{\pgfqpoint{1.751499in}{2.663373in}}{\pgfqpoint{1.743263in}{2.663373in}}%
\pgfpathcurveto{\pgfqpoint{1.735027in}{2.663373in}}{\pgfqpoint{1.727126in}{2.660100in}}{\pgfqpoint{1.721303in}{2.654276in}}%
\pgfpathcurveto{\pgfqpoint{1.715479in}{2.648452in}}{\pgfqpoint{1.712206in}{2.640552in}}{\pgfqpoint{1.712206in}{2.632316in}}%
\pgfpathcurveto{\pgfqpoint{1.712206in}{2.624080in}}{\pgfqpoint{1.715479in}{2.616180in}}{\pgfqpoint{1.721303in}{2.610356in}}%
\pgfpathcurveto{\pgfqpoint{1.727126in}{2.604532in}}{\pgfqpoint{1.735027in}{2.601260in}}{\pgfqpoint{1.743263in}{2.601260in}}%
\pgfpathclose%
\pgfusepath{stroke,fill}%
\end{pgfscope}%
\begin{pgfscope}%
\pgfpathrectangle{\pgfqpoint{0.100000in}{0.212622in}}{\pgfqpoint{3.696000in}{3.696000in}}%
\pgfusepath{clip}%
\pgfsetbuttcap%
\pgfsetroundjoin%
\definecolor{currentfill}{rgb}{0.121569,0.466667,0.705882}%
\pgfsetfillcolor{currentfill}%
\pgfsetfillopacity{0.905020}%
\pgfsetlinewidth{1.003750pt}%
\definecolor{currentstroke}{rgb}{0.121569,0.466667,0.705882}%
\pgfsetstrokecolor{currentstroke}%
\pgfsetstrokeopacity{0.905020}%
\pgfsetdash{}{0pt}%
\pgfpathmoveto{\pgfqpoint{2.175334in}{2.490585in}}%
\pgfpathcurveto{\pgfqpoint{2.183570in}{2.490585in}}{\pgfqpoint{2.191470in}{2.493857in}}{\pgfqpoint{2.197294in}{2.499681in}}%
\pgfpathcurveto{\pgfqpoint{2.203118in}{2.505505in}}{\pgfqpoint{2.206391in}{2.513405in}}{\pgfqpoint{2.206391in}{2.521641in}}%
\pgfpathcurveto{\pgfqpoint{2.206391in}{2.529878in}}{\pgfqpoint{2.203118in}{2.537778in}}{\pgfqpoint{2.197294in}{2.543601in}}%
\pgfpathcurveto{\pgfqpoint{2.191470in}{2.549425in}}{\pgfqpoint{2.183570in}{2.552698in}}{\pgfqpoint{2.175334in}{2.552698in}}%
\pgfpathcurveto{\pgfqpoint{2.167098in}{2.552698in}}{\pgfqpoint{2.159198in}{2.549425in}}{\pgfqpoint{2.153374in}{2.543601in}}%
\pgfpathcurveto{\pgfqpoint{2.147550in}{2.537778in}}{\pgfqpoint{2.144278in}{2.529878in}}{\pgfqpoint{2.144278in}{2.521641in}}%
\pgfpathcurveto{\pgfqpoint{2.144278in}{2.513405in}}{\pgfqpoint{2.147550in}{2.505505in}}{\pgfqpoint{2.153374in}{2.499681in}}%
\pgfpathcurveto{\pgfqpoint{2.159198in}{2.493857in}}{\pgfqpoint{2.167098in}{2.490585in}}{\pgfqpoint{2.175334in}{2.490585in}}%
\pgfpathclose%
\pgfusepath{stroke,fill}%
\end{pgfscope}%
\begin{pgfscope}%
\pgfpathrectangle{\pgfqpoint{0.100000in}{0.212622in}}{\pgfqpoint{3.696000in}{3.696000in}}%
\pgfusepath{clip}%
\pgfsetbuttcap%
\pgfsetroundjoin%
\definecolor{currentfill}{rgb}{0.121569,0.466667,0.705882}%
\pgfsetfillcolor{currentfill}%
\pgfsetfillopacity{0.905025}%
\pgfsetlinewidth{1.003750pt}%
\definecolor{currentstroke}{rgb}{0.121569,0.466667,0.705882}%
\pgfsetstrokecolor{currentstroke}%
\pgfsetstrokeopacity{0.905025}%
\pgfsetdash{}{0pt}%
\pgfpathmoveto{\pgfqpoint{1.742636in}{2.600657in}}%
\pgfpathcurveto{\pgfqpoint{1.750872in}{2.600657in}}{\pgfqpoint{1.758772in}{2.603929in}}{\pgfqpoint{1.764596in}{2.609753in}}%
\pgfpathcurveto{\pgfqpoint{1.770420in}{2.615577in}}{\pgfqpoint{1.773692in}{2.623477in}}{\pgfqpoint{1.773692in}{2.631713in}}%
\pgfpathcurveto{\pgfqpoint{1.773692in}{2.639949in}}{\pgfqpoint{1.770420in}{2.647849in}}{\pgfqpoint{1.764596in}{2.653673in}}%
\pgfpathcurveto{\pgfqpoint{1.758772in}{2.659497in}}{\pgfqpoint{1.750872in}{2.662770in}}{\pgfqpoint{1.742636in}{2.662770in}}%
\pgfpathcurveto{\pgfqpoint{1.734399in}{2.662770in}}{\pgfqpoint{1.726499in}{2.659497in}}{\pgfqpoint{1.720675in}{2.653673in}}%
\pgfpathcurveto{\pgfqpoint{1.714852in}{2.647849in}}{\pgfqpoint{1.711579in}{2.639949in}}{\pgfqpoint{1.711579in}{2.631713in}}%
\pgfpathcurveto{\pgfqpoint{1.711579in}{2.623477in}}{\pgfqpoint{1.714852in}{2.615577in}}{\pgfqpoint{1.720675in}{2.609753in}}%
\pgfpathcurveto{\pgfqpoint{1.726499in}{2.603929in}}{\pgfqpoint{1.734399in}{2.600657in}}{\pgfqpoint{1.742636in}{2.600657in}}%
\pgfpathclose%
\pgfusepath{stroke,fill}%
\end{pgfscope}%
\begin{pgfscope}%
\pgfpathrectangle{\pgfqpoint{0.100000in}{0.212622in}}{\pgfqpoint{3.696000in}{3.696000in}}%
\pgfusepath{clip}%
\pgfsetbuttcap%
\pgfsetroundjoin%
\definecolor{currentfill}{rgb}{0.121569,0.466667,0.705882}%
\pgfsetfillcolor{currentfill}%
\pgfsetfillopacity{0.905308}%
\pgfsetlinewidth{1.003750pt}%
\definecolor{currentstroke}{rgb}{0.121569,0.466667,0.705882}%
\pgfsetstrokecolor{currentstroke}%
\pgfsetstrokeopacity{0.905308}%
\pgfsetdash{}{0pt}%
\pgfpathmoveto{\pgfqpoint{0.959921in}{2.060879in}}%
\pgfpathcurveto{\pgfqpoint{0.968157in}{2.060879in}}{\pgfqpoint{0.976057in}{2.064151in}}{\pgfqpoint{0.981881in}{2.069975in}}%
\pgfpathcurveto{\pgfqpoint{0.987705in}{2.075799in}}{\pgfqpoint{0.990977in}{2.083699in}}{\pgfqpoint{0.990977in}{2.091936in}}%
\pgfpathcurveto{\pgfqpoint{0.990977in}{2.100172in}}{\pgfqpoint{0.987705in}{2.108072in}}{\pgfqpoint{0.981881in}{2.113896in}}%
\pgfpathcurveto{\pgfqpoint{0.976057in}{2.119720in}}{\pgfqpoint{0.968157in}{2.122992in}}{\pgfqpoint{0.959921in}{2.122992in}}%
\pgfpathcurveto{\pgfqpoint{0.951685in}{2.122992in}}{\pgfqpoint{0.943785in}{2.119720in}}{\pgfqpoint{0.937961in}{2.113896in}}%
\pgfpathcurveto{\pgfqpoint{0.932137in}{2.108072in}}{\pgfqpoint{0.928864in}{2.100172in}}{\pgfqpoint{0.928864in}{2.091936in}}%
\pgfpathcurveto{\pgfqpoint{0.928864in}{2.083699in}}{\pgfqpoint{0.932137in}{2.075799in}}{\pgfqpoint{0.937961in}{2.069975in}}%
\pgfpathcurveto{\pgfqpoint{0.943785in}{2.064151in}}{\pgfqpoint{0.951685in}{2.060879in}}{\pgfqpoint{0.959921in}{2.060879in}}%
\pgfpathclose%
\pgfusepath{stroke,fill}%
\end{pgfscope}%
\begin{pgfscope}%
\pgfpathrectangle{\pgfqpoint{0.100000in}{0.212622in}}{\pgfqpoint{3.696000in}{3.696000in}}%
\pgfusepath{clip}%
\pgfsetbuttcap%
\pgfsetroundjoin%
\definecolor{currentfill}{rgb}{0.121569,0.466667,0.705882}%
\pgfsetfillcolor{currentfill}%
\pgfsetfillopacity{0.905465}%
\pgfsetlinewidth{1.003750pt}%
\definecolor{currentstroke}{rgb}{0.121569,0.466667,0.705882}%
\pgfsetstrokecolor{currentstroke}%
\pgfsetstrokeopacity{0.905465}%
\pgfsetdash{}{0pt}%
\pgfpathmoveto{\pgfqpoint{1.741444in}{2.599405in}}%
\pgfpathcurveto{\pgfqpoint{1.749680in}{2.599405in}}{\pgfqpoint{1.757580in}{2.602677in}}{\pgfqpoint{1.763404in}{2.608501in}}%
\pgfpathcurveto{\pgfqpoint{1.769228in}{2.614325in}}{\pgfqpoint{1.772500in}{2.622225in}}{\pgfqpoint{1.772500in}{2.630462in}}%
\pgfpathcurveto{\pgfqpoint{1.772500in}{2.638698in}}{\pgfqpoint{1.769228in}{2.646598in}}{\pgfqpoint{1.763404in}{2.652422in}}%
\pgfpathcurveto{\pgfqpoint{1.757580in}{2.658246in}}{\pgfqpoint{1.749680in}{2.661518in}}{\pgfqpoint{1.741444in}{2.661518in}}%
\pgfpathcurveto{\pgfqpoint{1.733208in}{2.661518in}}{\pgfqpoint{1.725308in}{2.658246in}}{\pgfqpoint{1.719484in}{2.652422in}}%
\pgfpathcurveto{\pgfqpoint{1.713660in}{2.646598in}}{\pgfqpoint{1.710387in}{2.638698in}}{\pgfqpoint{1.710387in}{2.630462in}}%
\pgfpathcurveto{\pgfqpoint{1.710387in}{2.622225in}}{\pgfqpoint{1.713660in}{2.614325in}}{\pgfqpoint{1.719484in}{2.608501in}}%
\pgfpathcurveto{\pgfqpoint{1.725308in}{2.602677in}}{\pgfqpoint{1.733208in}{2.599405in}}{\pgfqpoint{1.741444in}{2.599405in}}%
\pgfpathclose%
\pgfusepath{stroke,fill}%
\end{pgfscope}%
\begin{pgfscope}%
\pgfpathrectangle{\pgfqpoint{0.100000in}{0.212622in}}{\pgfqpoint{3.696000in}{3.696000in}}%
\pgfusepath{clip}%
\pgfsetbuttcap%
\pgfsetroundjoin%
\definecolor{currentfill}{rgb}{0.121569,0.466667,0.705882}%
\pgfsetfillcolor{currentfill}%
\pgfsetfillopacity{0.906011}%
\pgfsetlinewidth{1.003750pt}%
\definecolor{currentstroke}{rgb}{0.121569,0.466667,0.705882}%
\pgfsetstrokecolor{currentstroke}%
\pgfsetstrokeopacity{0.906011}%
\pgfsetdash{}{0pt}%
\pgfpathmoveto{\pgfqpoint{1.740073in}{2.598017in}}%
\pgfpathcurveto{\pgfqpoint{1.748310in}{2.598017in}}{\pgfqpoint{1.756210in}{2.601289in}}{\pgfqpoint{1.762033in}{2.607113in}}%
\pgfpathcurveto{\pgfqpoint{1.767857in}{2.612937in}}{\pgfqpoint{1.771130in}{2.620837in}}{\pgfqpoint{1.771130in}{2.629073in}}%
\pgfpathcurveto{\pgfqpoint{1.771130in}{2.637310in}}{\pgfqpoint{1.767857in}{2.645210in}}{\pgfqpoint{1.762033in}{2.651034in}}%
\pgfpathcurveto{\pgfqpoint{1.756210in}{2.656858in}}{\pgfqpoint{1.748310in}{2.660130in}}{\pgfqpoint{1.740073in}{2.660130in}}%
\pgfpathcurveto{\pgfqpoint{1.731837in}{2.660130in}}{\pgfqpoint{1.723937in}{2.656858in}}{\pgfqpoint{1.718113in}{2.651034in}}%
\pgfpathcurveto{\pgfqpoint{1.712289in}{2.645210in}}{\pgfqpoint{1.709017in}{2.637310in}}{\pgfqpoint{1.709017in}{2.629073in}}%
\pgfpathcurveto{\pgfqpoint{1.709017in}{2.620837in}}{\pgfqpoint{1.712289in}{2.612937in}}{\pgfqpoint{1.718113in}{2.607113in}}%
\pgfpathcurveto{\pgfqpoint{1.723937in}{2.601289in}}{\pgfqpoint{1.731837in}{2.598017in}}{\pgfqpoint{1.740073in}{2.598017in}}%
\pgfpathclose%
\pgfusepath{stroke,fill}%
\end{pgfscope}%
\begin{pgfscope}%
\pgfpathrectangle{\pgfqpoint{0.100000in}{0.212622in}}{\pgfqpoint{3.696000in}{3.696000in}}%
\pgfusepath{clip}%
\pgfsetbuttcap%
\pgfsetroundjoin%
\definecolor{currentfill}{rgb}{0.121569,0.466667,0.705882}%
\pgfsetfillcolor{currentfill}%
\pgfsetfillopacity{0.906139}%
\pgfsetlinewidth{1.003750pt}%
\definecolor{currentstroke}{rgb}{0.121569,0.466667,0.705882}%
\pgfsetstrokecolor{currentstroke}%
\pgfsetstrokeopacity{0.906139}%
\pgfsetdash{}{0pt}%
\pgfpathmoveto{\pgfqpoint{2.173056in}{2.488747in}}%
\pgfpathcurveto{\pgfqpoint{2.181292in}{2.488747in}}{\pgfqpoint{2.189192in}{2.492019in}}{\pgfqpoint{2.195016in}{2.497843in}}%
\pgfpathcurveto{\pgfqpoint{2.200840in}{2.503667in}}{\pgfqpoint{2.204113in}{2.511567in}}{\pgfqpoint{2.204113in}{2.519803in}}%
\pgfpathcurveto{\pgfqpoint{2.204113in}{2.528040in}}{\pgfqpoint{2.200840in}{2.535940in}}{\pgfqpoint{2.195016in}{2.541764in}}%
\pgfpathcurveto{\pgfqpoint{2.189192in}{2.547588in}}{\pgfqpoint{2.181292in}{2.550860in}}{\pgfqpoint{2.173056in}{2.550860in}}%
\pgfpathcurveto{\pgfqpoint{2.164820in}{2.550860in}}{\pgfqpoint{2.156920in}{2.547588in}}{\pgfqpoint{2.151096in}{2.541764in}}%
\pgfpathcurveto{\pgfqpoint{2.145272in}{2.535940in}}{\pgfqpoint{2.142000in}{2.528040in}}{\pgfqpoint{2.142000in}{2.519803in}}%
\pgfpathcurveto{\pgfqpoint{2.142000in}{2.511567in}}{\pgfqpoint{2.145272in}{2.503667in}}{\pgfqpoint{2.151096in}{2.497843in}}%
\pgfpathcurveto{\pgfqpoint{2.156920in}{2.492019in}}{\pgfqpoint{2.164820in}{2.488747in}}{\pgfqpoint{2.173056in}{2.488747in}}%
\pgfpathclose%
\pgfusepath{stroke,fill}%
\end{pgfscope}%
\begin{pgfscope}%
\pgfpathrectangle{\pgfqpoint{0.100000in}{0.212622in}}{\pgfqpoint{3.696000in}{3.696000in}}%
\pgfusepath{clip}%
\pgfsetbuttcap%
\pgfsetroundjoin%
\definecolor{currentfill}{rgb}{0.121569,0.466667,0.705882}%
\pgfsetfillcolor{currentfill}%
\pgfsetfillopacity{0.906336}%
\pgfsetlinewidth{1.003750pt}%
\definecolor{currentstroke}{rgb}{0.121569,0.466667,0.705882}%
\pgfsetstrokecolor{currentstroke}%
\pgfsetstrokeopacity{0.906336}%
\pgfsetdash{}{0pt}%
\pgfpathmoveto{\pgfqpoint{1.739298in}{2.597423in}}%
\pgfpathcurveto{\pgfqpoint{1.747534in}{2.597423in}}{\pgfqpoint{1.755434in}{2.600696in}}{\pgfqpoint{1.761258in}{2.606520in}}%
\pgfpathcurveto{\pgfqpoint{1.767082in}{2.612343in}}{\pgfqpoint{1.770355in}{2.620243in}}{\pgfqpoint{1.770355in}{2.628480in}}%
\pgfpathcurveto{\pgfqpoint{1.770355in}{2.636716in}}{\pgfqpoint{1.767082in}{2.644616in}}{\pgfqpoint{1.761258in}{2.650440in}}%
\pgfpathcurveto{\pgfqpoint{1.755434in}{2.656264in}}{\pgfqpoint{1.747534in}{2.659536in}}{\pgfqpoint{1.739298in}{2.659536in}}%
\pgfpathcurveto{\pgfqpoint{1.731062in}{2.659536in}}{\pgfqpoint{1.723162in}{2.656264in}}{\pgfqpoint{1.717338in}{2.650440in}}%
\pgfpathcurveto{\pgfqpoint{1.711514in}{2.644616in}}{\pgfqpoint{1.708242in}{2.636716in}}{\pgfqpoint{1.708242in}{2.628480in}}%
\pgfpathcurveto{\pgfqpoint{1.708242in}{2.620243in}}{\pgfqpoint{1.711514in}{2.612343in}}{\pgfqpoint{1.717338in}{2.606520in}}%
\pgfpathcurveto{\pgfqpoint{1.723162in}{2.600696in}}{\pgfqpoint{1.731062in}{2.597423in}}{\pgfqpoint{1.739298in}{2.597423in}}%
\pgfpathclose%
\pgfusepath{stroke,fill}%
\end{pgfscope}%
\begin{pgfscope}%
\pgfpathrectangle{\pgfqpoint{0.100000in}{0.212622in}}{\pgfqpoint{3.696000in}{3.696000in}}%
\pgfusepath{clip}%
\pgfsetbuttcap%
\pgfsetroundjoin%
\definecolor{currentfill}{rgb}{0.121569,0.466667,0.705882}%
\pgfsetfillcolor{currentfill}%
\pgfsetfillopacity{0.906816}%
\pgfsetlinewidth{1.003750pt}%
\definecolor{currentstroke}{rgb}{0.121569,0.466667,0.705882}%
\pgfsetstrokecolor{currentstroke}%
\pgfsetstrokeopacity{0.906816}%
\pgfsetdash{}{0pt}%
\pgfpathmoveto{\pgfqpoint{1.738073in}{2.596319in}}%
\pgfpathcurveto{\pgfqpoint{1.746309in}{2.596319in}}{\pgfqpoint{1.754209in}{2.599591in}}{\pgfqpoint{1.760033in}{2.605415in}}%
\pgfpathcurveto{\pgfqpoint{1.765857in}{2.611239in}}{\pgfqpoint{1.769129in}{2.619139in}}{\pgfqpoint{1.769129in}{2.627375in}}%
\pgfpathcurveto{\pgfqpoint{1.769129in}{2.635612in}}{\pgfqpoint{1.765857in}{2.643512in}}{\pgfqpoint{1.760033in}{2.649336in}}%
\pgfpathcurveto{\pgfqpoint{1.754209in}{2.655160in}}{\pgfqpoint{1.746309in}{2.658432in}}{\pgfqpoint{1.738073in}{2.658432in}}%
\pgfpathcurveto{\pgfqpoint{1.729836in}{2.658432in}}{\pgfqpoint{1.721936in}{2.655160in}}{\pgfqpoint{1.716112in}{2.649336in}}%
\pgfpathcurveto{\pgfqpoint{1.710288in}{2.643512in}}{\pgfqpoint{1.707016in}{2.635612in}}{\pgfqpoint{1.707016in}{2.627375in}}%
\pgfpathcurveto{\pgfqpoint{1.707016in}{2.619139in}}{\pgfqpoint{1.710288in}{2.611239in}}{\pgfqpoint{1.716112in}{2.605415in}}%
\pgfpathcurveto{\pgfqpoint{1.721936in}{2.599591in}}{\pgfqpoint{1.729836in}{2.596319in}}{\pgfqpoint{1.738073in}{2.596319in}}%
\pgfpathclose%
\pgfusepath{stroke,fill}%
\end{pgfscope}%
\begin{pgfscope}%
\pgfpathrectangle{\pgfqpoint{0.100000in}{0.212622in}}{\pgfqpoint{3.696000in}{3.696000in}}%
\pgfusepath{clip}%
\pgfsetbuttcap%
\pgfsetroundjoin%
\definecolor{currentfill}{rgb}{0.121569,0.466667,0.705882}%
\pgfsetfillcolor{currentfill}%
\pgfsetfillopacity{0.907078}%
\pgfsetlinewidth{1.003750pt}%
\definecolor{currentstroke}{rgb}{0.121569,0.466667,0.705882}%
\pgfsetstrokecolor{currentstroke}%
\pgfsetstrokeopacity{0.907078}%
\pgfsetdash{}{0pt}%
\pgfpathmoveto{\pgfqpoint{1.737430in}{2.595658in}}%
\pgfpathcurveto{\pgfqpoint{1.745667in}{2.595658in}}{\pgfqpoint{1.753567in}{2.598930in}}{\pgfqpoint{1.759390in}{2.604754in}}%
\pgfpathcurveto{\pgfqpoint{1.765214in}{2.610578in}}{\pgfqpoint{1.768487in}{2.618478in}}{\pgfqpoint{1.768487in}{2.626714in}}%
\pgfpathcurveto{\pgfqpoint{1.768487in}{2.634951in}}{\pgfqpoint{1.765214in}{2.642851in}}{\pgfqpoint{1.759390in}{2.648675in}}%
\pgfpathcurveto{\pgfqpoint{1.753567in}{2.654499in}}{\pgfqpoint{1.745667in}{2.657771in}}{\pgfqpoint{1.737430in}{2.657771in}}%
\pgfpathcurveto{\pgfqpoint{1.729194in}{2.657771in}}{\pgfqpoint{1.721294in}{2.654499in}}{\pgfqpoint{1.715470in}{2.648675in}}%
\pgfpathcurveto{\pgfqpoint{1.709646in}{2.642851in}}{\pgfqpoint{1.706374in}{2.634951in}}{\pgfqpoint{1.706374in}{2.626714in}}%
\pgfpathcurveto{\pgfqpoint{1.706374in}{2.618478in}}{\pgfqpoint{1.709646in}{2.610578in}}{\pgfqpoint{1.715470in}{2.604754in}}%
\pgfpathcurveto{\pgfqpoint{1.721294in}{2.598930in}}{\pgfqpoint{1.729194in}{2.595658in}}{\pgfqpoint{1.737430in}{2.595658in}}%
\pgfpathclose%
\pgfusepath{stroke,fill}%
\end{pgfscope}%
\begin{pgfscope}%
\pgfpathrectangle{\pgfqpoint{0.100000in}{0.212622in}}{\pgfqpoint{3.696000in}{3.696000in}}%
\pgfusepath{clip}%
\pgfsetbuttcap%
\pgfsetroundjoin%
\definecolor{currentfill}{rgb}{0.121569,0.466667,0.705882}%
\pgfsetfillcolor{currentfill}%
\pgfsetfillopacity{0.907109}%
\pgfsetlinewidth{1.003750pt}%
\definecolor{currentstroke}{rgb}{0.121569,0.466667,0.705882}%
\pgfsetstrokecolor{currentstroke}%
\pgfsetstrokeopacity{0.907109}%
\pgfsetdash{}{0pt}%
\pgfpathmoveto{\pgfqpoint{2.666599in}{1.289121in}}%
\pgfpathcurveto{\pgfqpoint{2.674835in}{1.289121in}}{\pgfqpoint{2.682736in}{1.292393in}}{\pgfqpoint{2.688559in}{1.298217in}}%
\pgfpathcurveto{\pgfqpoint{2.694383in}{1.304041in}}{\pgfqpoint{2.697656in}{1.311941in}}{\pgfqpoint{2.697656in}{1.320177in}}%
\pgfpathcurveto{\pgfqpoint{2.697656in}{1.328414in}}{\pgfqpoint{2.694383in}{1.336314in}}{\pgfqpoint{2.688559in}{1.342138in}}%
\pgfpathcurveto{\pgfqpoint{2.682736in}{1.347962in}}{\pgfqpoint{2.674835in}{1.351234in}}{\pgfqpoint{2.666599in}{1.351234in}}%
\pgfpathcurveto{\pgfqpoint{2.658363in}{1.351234in}}{\pgfqpoint{2.650463in}{1.347962in}}{\pgfqpoint{2.644639in}{1.342138in}}%
\pgfpathcurveto{\pgfqpoint{2.638815in}{1.336314in}}{\pgfqpoint{2.635543in}{1.328414in}}{\pgfqpoint{2.635543in}{1.320177in}}%
\pgfpathcurveto{\pgfqpoint{2.635543in}{1.311941in}}{\pgfqpoint{2.638815in}{1.304041in}}{\pgfqpoint{2.644639in}{1.298217in}}%
\pgfpathcurveto{\pgfqpoint{2.650463in}{1.292393in}}{\pgfqpoint{2.658363in}{1.289121in}}{\pgfqpoint{2.666599in}{1.289121in}}%
\pgfpathclose%
\pgfusepath{stroke,fill}%
\end{pgfscope}%
\begin{pgfscope}%
\pgfpathrectangle{\pgfqpoint{0.100000in}{0.212622in}}{\pgfqpoint{3.696000in}{3.696000in}}%
\pgfusepath{clip}%
\pgfsetbuttcap%
\pgfsetroundjoin%
\definecolor{currentfill}{rgb}{0.121569,0.466667,0.705882}%
\pgfsetfillcolor{currentfill}%
\pgfsetfillopacity{0.907129}%
\pgfsetlinewidth{1.003750pt}%
\definecolor{currentstroke}{rgb}{0.121569,0.466667,0.705882}%
\pgfsetstrokecolor{currentstroke}%
\pgfsetstrokeopacity{0.907129}%
\pgfsetdash{}{0pt}%
\pgfpathmoveto{\pgfqpoint{2.171198in}{2.487339in}}%
\pgfpathcurveto{\pgfqpoint{2.179434in}{2.487339in}}{\pgfqpoint{2.187334in}{2.490612in}}{\pgfqpoint{2.193158in}{2.496436in}}%
\pgfpathcurveto{\pgfqpoint{2.198982in}{2.502259in}}{\pgfqpoint{2.202255in}{2.510160in}}{\pgfqpoint{2.202255in}{2.518396in}}%
\pgfpathcurveto{\pgfqpoint{2.202255in}{2.526632in}}{\pgfqpoint{2.198982in}{2.534532in}}{\pgfqpoint{2.193158in}{2.540356in}}%
\pgfpathcurveto{\pgfqpoint{2.187334in}{2.546180in}}{\pgfqpoint{2.179434in}{2.549452in}}{\pgfqpoint{2.171198in}{2.549452in}}%
\pgfpathcurveto{\pgfqpoint{2.162962in}{2.549452in}}{\pgfqpoint{2.155062in}{2.546180in}}{\pgfqpoint{2.149238in}{2.540356in}}%
\pgfpathcurveto{\pgfqpoint{2.143414in}{2.534532in}}{\pgfqpoint{2.140142in}{2.526632in}}{\pgfqpoint{2.140142in}{2.518396in}}%
\pgfpathcurveto{\pgfqpoint{2.140142in}{2.510160in}}{\pgfqpoint{2.143414in}{2.502259in}}{\pgfqpoint{2.149238in}{2.496436in}}%
\pgfpathcurveto{\pgfqpoint{2.155062in}{2.490612in}}{\pgfqpoint{2.162962in}{2.487339in}}{\pgfqpoint{2.171198in}{2.487339in}}%
\pgfpathclose%
\pgfusepath{stroke,fill}%
\end{pgfscope}%
\begin{pgfscope}%
\pgfpathrectangle{\pgfqpoint{0.100000in}{0.212622in}}{\pgfqpoint{3.696000in}{3.696000in}}%
\pgfusepath{clip}%
\pgfsetbuttcap%
\pgfsetroundjoin%
\definecolor{currentfill}{rgb}{0.121569,0.466667,0.705882}%
\pgfsetfillcolor{currentfill}%
\pgfsetfillopacity{0.907228}%
\pgfsetlinewidth{1.003750pt}%
\definecolor{currentstroke}{rgb}{0.121569,0.466667,0.705882}%
\pgfsetstrokecolor{currentstroke}%
\pgfsetstrokeopacity{0.907228}%
\pgfsetdash{}{0pt}%
\pgfpathmoveto{\pgfqpoint{1.737080in}{2.595322in}}%
\pgfpathcurveto{\pgfqpoint{1.745317in}{2.595322in}}{\pgfqpoint{1.753217in}{2.598594in}}{\pgfqpoint{1.759041in}{2.604418in}}%
\pgfpathcurveto{\pgfqpoint{1.764865in}{2.610242in}}{\pgfqpoint{1.768137in}{2.618142in}}{\pgfqpoint{1.768137in}{2.626379in}}%
\pgfpathcurveto{\pgfqpoint{1.768137in}{2.634615in}}{\pgfqpoint{1.764865in}{2.642515in}}{\pgfqpoint{1.759041in}{2.648339in}}%
\pgfpathcurveto{\pgfqpoint{1.753217in}{2.654163in}}{\pgfqpoint{1.745317in}{2.657435in}}{\pgfqpoint{1.737080in}{2.657435in}}%
\pgfpathcurveto{\pgfqpoint{1.728844in}{2.657435in}}{\pgfqpoint{1.720944in}{2.654163in}}{\pgfqpoint{1.715120in}{2.648339in}}%
\pgfpathcurveto{\pgfqpoint{1.709296in}{2.642515in}}{\pgfqpoint{1.706024in}{2.634615in}}{\pgfqpoint{1.706024in}{2.626379in}}%
\pgfpathcurveto{\pgfqpoint{1.706024in}{2.618142in}}{\pgfqpoint{1.709296in}{2.610242in}}{\pgfqpoint{1.715120in}{2.604418in}}%
\pgfpathcurveto{\pgfqpoint{1.720944in}{2.598594in}}{\pgfqpoint{1.728844in}{2.595322in}}{\pgfqpoint{1.737080in}{2.595322in}}%
\pgfpathclose%
\pgfusepath{stroke,fill}%
\end{pgfscope}%
\begin{pgfscope}%
\pgfpathrectangle{\pgfqpoint{0.100000in}{0.212622in}}{\pgfqpoint{3.696000in}{3.696000in}}%
\pgfusepath{clip}%
\pgfsetbuttcap%
\pgfsetroundjoin%
\definecolor{currentfill}{rgb}{0.121569,0.466667,0.705882}%
\pgfsetfillcolor{currentfill}%
\pgfsetfillopacity{0.907308}%
\pgfsetlinewidth{1.003750pt}%
\definecolor{currentstroke}{rgb}{0.121569,0.466667,0.705882}%
\pgfsetstrokecolor{currentstroke}%
\pgfsetstrokeopacity{0.907308}%
\pgfsetdash{}{0pt}%
\pgfpathmoveto{\pgfqpoint{1.736884in}{2.595130in}}%
\pgfpathcurveto{\pgfqpoint{1.745121in}{2.595130in}}{\pgfqpoint{1.753021in}{2.598402in}}{\pgfqpoint{1.758845in}{2.604226in}}%
\pgfpathcurveto{\pgfqpoint{1.764669in}{2.610050in}}{\pgfqpoint{1.767941in}{2.617950in}}{\pgfqpoint{1.767941in}{2.626186in}}%
\pgfpathcurveto{\pgfqpoint{1.767941in}{2.634422in}}{\pgfqpoint{1.764669in}{2.642323in}}{\pgfqpoint{1.758845in}{2.648146in}}%
\pgfpathcurveto{\pgfqpoint{1.753021in}{2.653970in}}{\pgfqpoint{1.745121in}{2.657243in}}{\pgfqpoint{1.736884in}{2.657243in}}%
\pgfpathcurveto{\pgfqpoint{1.728648in}{2.657243in}}{\pgfqpoint{1.720748in}{2.653970in}}{\pgfqpoint{1.714924in}{2.648146in}}%
\pgfpathcurveto{\pgfqpoint{1.709100in}{2.642323in}}{\pgfqpoint{1.705828in}{2.634422in}}{\pgfqpoint{1.705828in}{2.626186in}}%
\pgfpathcurveto{\pgfqpoint{1.705828in}{2.617950in}}{\pgfqpoint{1.709100in}{2.610050in}}{\pgfqpoint{1.714924in}{2.604226in}}%
\pgfpathcurveto{\pgfqpoint{1.720748in}{2.598402in}}{\pgfqpoint{1.728648in}{2.595130in}}{\pgfqpoint{1.736884in}{2.595130in}}%
\pgfpathclose%
\pgfusepath{stroke,fill}%
\end{pgfscope}%
\begin{pgfscope}%
\pgfpathrectangle{\pgfqpoint{0.100000in}{0.212622in}}{\pgfqpoint{3.696000in}{3.696000in}}%
\pgfusepath{clip}%
\pgfsetbuttcap%
\pgfsetroundjoin%
\definecolor{currentfill}{rgb}{0.121569,0.466667,0.705882}%
\pgfsetfillcolor{currentfill}%
\pgfsetfillopacity{0.907444}%
\pgfsetlinewidth{1.003750pt}%
\definecolor{currentstroke}{rgb}{0.121569,0.466667,0.705882}%
\pgfsetstrokecolor{currentstroke}%
\pgfsetstrokeopacity{0.907444}%
\pgfsetdash{}{0pt}%
\pgfpathmoveto{\pgfqpoint{0.967207in}{2.054093in}}%
\pgfpathcurveto{\pgfqpoint{0.975443in}{2.054093in}}{\pgfqpoint{0.983343in}{2.057365in}}{\pgfqpoint{0.989167in}{2.063189in}}%
\pgfpathcurveto{\pgfqpoint{0.994991in}{2.069013in}}{\pgfqpoint{0.998263in}{2.076913in}}{\pgfqpoint{0.998263in}{2.085149in}}%
\pgfpathcurveto{\pgfqpoint{0.998263in}{2.093385in}}{\pgfqpoint{0.994991in}{2.101285in}}{\pgfqpoint{0.989167in}{2.107109in}}%
\pgfpathcurveto{\pgfqpoint{0.983343in}{2.112933in}}{\pgfqpoint{0.975443in}{2.116206in}}{\pgfqpoint{0.967207in}{2.116206in}}%
\pgfpathcurveto{\pgfqpoint{0.958970in}{2.116206in}}{\pgfqpoint{0.951070in}{2.112933in}}{\pgfqpoint{0.945246in}{2.107109in}}%
\pgfpathcurveto{\pgfqpoint{0.939422in}{2.101285in}}{\pgfqpoint{0.936150in}{2.093385in}}{\pgfqpoint{0.936150in}{2.085149in}}%
\pgfpathcurveto{\pgfqpoint{0.936150in}{2.076913in}}{\pgfqpoint{0.939422in}{2.069013in}}{\pgfqpoint{0.945246in}{2.063189in}}%
\pgfpathcurveto{\pgfqpoint{0.951070in}{2.057365in}}{\pgfqpoint{0.958970in}{2.054093in}}{\pgfqpoint{0.967207in}{2.054093in}}%
\pgfpathclose%
\pgfusepath{stroke,fill}%
\end{pgfscope}%
\begin{pgfscope}%
\pgfpathrectangle{\pgfqpoint{0.100000in}{0.212622in}}{\pgfqpoint{3.696000in}{3.696000in}}%
\pgfusepath{clip}%
\pgfsetbuttcap%
\pgfsetroundjoin%
\definecolor{currentfill}{rgb}{0.121569,0.466667,0.705882}%
\pgfsetfillcolor{currentfill}%
\pgfsetfillopacity{0.907571}%
\pgfsetlinewidth{1.003750pt}%
\definecolor{currentstroke}{rgb}{0.121569,0.466667,0.705882}%
\pgfsetstrokecolor{currentstroke}%
\pgfsetstrokeopacity{0.907571}%
\pgfsetdash{}{0pt}%
\pgfpathmoveto{\pgfqpoint{1.736200in}{2.594349in}}%
\pgfpathcurveto{\pgfqpoint{1.744436in}{2.594349in}}{\pgfqpoint{1.752336in}{2.597621in}}{\pgfqpoint{1.758160in}{2.603445in}}%
\pgfpathcurveto{\pgfqpoint{1.763984in}{2.609269in}}{\pgfqpoint{1.767257in}{2.617169in}}{\pgfqpoint{1.767257in}{2.625405in}}%
\pgfpathcurveto{\pgfqpoint{1.767257in}{2.633641in}}{\pgfqpoint{1.763984in}{2.641542in}}{\pgfqpoint{1.758160in}{2.647365in}}%
\pgfpathcurveto{\pgfqpoint{1.752336in}{2.653189in}}{\pgfqpoint{1.744436in}{2.656462in}}{\pgfqpoint{1.736200in}{2.656462in}}%
\pgfpathcurveto{\pgfqpoint{1.727964in}{2.656462in}}{\pgfqpoint{1.720064in}{2.653189in}}{\pgfqpoint{1.714240in}{2.647365in}}%
\pgfpathcurveto{\pgfqpoint{1.708416in}{2.641542in}}{\pgfqpoint{1.705144in}{2.633641in}}{\pgfqpoint{1.705144in}{2.625405in}}%
\pgfpathcurveto{\pgfqpoint{1.705144in}{2.617169in}}{\pgfqpoint{1.708416in}{2.609269in}}{\pgfqpoint{1.714240in}{2.603445in}}%
\pgfpathcurveto{\pgfqpoint{1.720064in}{2.597621in}}{\pgfqpoint{1.727964in}{2.594349in}}{\pgfqpoint{1.736200in}{2.594349in}}%
\pgfpathclose%
\pgfusepath{stroke,fill}%
\end{pgfscope}%
\begin{pgfscope}%
\pgfpathrectangle{\pgfqpoint{0.100000in}{0.212622in}}{\pgfqpoint{3.696000in}{3.696000in}}%
\pgfusepath{clip}%
\pgfsetbuttcap%
\pgfsetroundjoin%
\definecolor{currentfill}{rgb}{0.121569,0.466667,0.705882}%
\pgfsetfillcolor{currentfill}%
\pgfsetfillopacity{0.907724}%
\pgfsetlinewidth{1.003750pt}%
\definecolor{currentstroke}{rgb}{0.121569,0.466667,0.705882}%
\pgfsetstrokecolor{currentstroke}%
\pgfsetstrokeopacity{0.907724}%
\pgfsetdash{}{0pt}%
\pgfpathmoveto{\pgfqpoint{1.735830in}{2.593958in}}%
\pgfpathcurveto{\pgfqpoint{1.744066in}{2.593958in}}{\pgfqpoint{1.751966in}{2.597230in}}{\pgfqpoint{1.757790in}{2.603054in}}%
\pgfpathcurveto{\pgfqpoint{1.763614in}{2.608878in}}{\pgfqpoint{1.766886in}{2.616778in}}{\pgfqpoint{1.766886in}{2.625015in}}%
\pgfpathcurveto{\pgfqpoint{1.766886in}{2.633251in}}{\pgfqpoint{1.763614in}{2.641151in}}{\pgfqpoint{1.757790in}{2.646975in}}%
\pgfpathcurveto{\pgfqpoint{1.751966in}{2.652799in}}{\pgfqpoint{1.744066in}{2.656071in}}{\pgfqpoint{1.735830in}{2.656071in}}%
\pgfpathcurveto{\pgfqpoint{1.727594in}{2.656071in}}{\pgfqpoint{1.719693in}{2.652799in}}{\pgfqpoint{1.713870in}{2.646975in}}%
\pgfpathcurveto{\pgfqpoint{1.708046in}{2.641151in}}{\pgfqpoint{1.704773in}{2.633251in}}{\pgfqpoint{1.704773in}{2.625015in}}%
\pgfpathcurveto{\pgfqpoint{1.704773in}{2.616778in}}{\pgfqpoint{1.708046in}{2.608878in}}{\pgfqpoint{1.713870in}{2.603054in}}%
\pgfpathcurveto{\pgfqpoint{1.719693in}{2.597230in}}{\pgfqpoint{1.727594in}{2.593958in}}{\pgfqpoint{1.735830in}{2.593958in}}%
\pgfpathclose%
\pgfusepath{stroke,fill}%
\end{pgfscope}%
\begin{pgfscope}%
\pgfpathrectangle{\pgfqpoint{0.100000in}{0.212622in}}{\pgfqpoint{3.696000in}{3.696000in}}%
\pgfusepath{clip}%
\pgfsetbuttcap%
\pgfsetroundjoin%
\definecolor{currentfill}{rgb}{0.121569,0.466667,0.705882}%
\pgfsetfillcolor{currentfill}%
\pgfsetfillopacity{0.907812}%
\pgfsetlinewidth{1.003750pt}%
\definecolor{currentstroke}{rgb}{0.121569,0.466667,0.705882}%
\pgfsetstrokecolor{currentstroke}%
\pgfsetstrokeopacity{0.907812}%
\pgfsetdash{}{0pt}%
\pgfpathmoveto{\pgfqpoint{1.735624in}{2.593773in}}%
\pgfpathcurveto{\pgfqpoint{1.743860in}{2.593773in}}{\pgfqpoint{1.751760in}{2.597046in}}{\pgfqpoint{1.757584in}{2.602870in}}%
\pgfpathcurveto{\pgfqpoint{1.763408in}{2.608693in}}{\pgfqpoint{1.766681in}{2.616593in}}{\pgfqpoint{1.766681in}{2.624830in}}%
\pgfpathcurveto{\pgfqpoint{1.766681in}{2.633066in}}{\pgfqpoint{1.763408in}{2.640966in}}{\pgfqpoint{1.757584in}{2.646790in}}%
\pgfpathcurveto{\pgfqpoint{1.751760in}{2.652614in}}{\pgfqpoint{1.743860in}{2.655886in}}{\pgfqpoint{1.735624in}{2.655886in}}%
\pgfpathcurveto{\pgfqpoint{1.727388in}{2.655886in}}{\pgfqpoint{1.719488in}{2.652614in}}{\pgfqpoint{1.713664in}{2.646790in}}%
\pgfpathcurveto{\pgfqpoint{1.707840in}{2.640966in}}{\pgfqpoint{1.704568in}{2.633066in}}{\pgfqpoint{1.704568in}{2.624830in}}%
\pgfpathcurveto{\pgfqpoint{1.704568in}{2.616593in}}{\pgfqpoint{1.707840in}{2.608693in}}{\pgfqpoint{1.713664in}{2.602870in}}%
\pgfpathcurveto{\pgfqpoint{1.719488in}{2.597046in}}{\pgfqpoint{1.727388in}{2.593773in}}{\pgfqpoint{1.735624in}{2.593773in}}%
\pgfpathclose%
\pgfusepath{stroke,fill}%
\end{pgfscope}%
\begin{pgfscope}%
\pgfpathrectangle{\pgfqpoint{0.100000in}{0.212622in}}{\pgfqpoint{3.696000in}{3.696000in}}%
\pgfusepath{clip}%
\pgfsetbuttcap%
\pgfsetroundjoin%
\definecolor{currentfill}{rgb}{0.121569,0.466667,0.705882}%
\pgfsetfillcolor{currentfill}%
\pgfsetfillopacity{0.907859}%
\pgfsetlinewidth{1.003750pt}%
\definecolor{currentstroke}{rgb}{0.121569,0.466667,0.705882}%
\pgfsetstrokecolor{currentstroke}%
\pgfsetstrokeopacity{0.907859}%
\pgfsetdash{}{0pt}%
\pgfpathmoveto{\pgfqpoint{1.735508in}{2.593665in}}%
\pgfpathcurveto{\pgfqpoint{1.743744in}{2.593665in}}{\pgfqpoint{1.751644in}{2.596937in}}{\pgfqpoint{1.757468in}{2.602761in}}%
\pgfpathcurveto{\pgfqpoint{1.763292in}{2.608585in}}{\pgfqpoint{1.766564in}{2.616485in}}{\pgfqpoint{1.766564in}{2.624721in}}%
\pgfpathcurveto{\pgfqpoint{1.766564in}{2.632957in}}{\pgfqpoint{1.763292in}{2.640857in}}{\pgfqpoint{1.757468in}{2.646681in}}%
\pgfpathcurveto{\pgfqpoint{1.751644in}{2.652505in}}{\pgfqpoint{1.743744in}{2.655778in}}{\pgfqpoint{1.735508in}{2.655778in}}%
\pgfpathcurveto{\pgfqpoint{1.727271in}{2.655778in}}{\pgfqpoint{1.719371in}{2.652505in}}{\pgfqpoint{1.713547in}{2.646681in}}%
\pgfpathcurveto{\pgfqpoint{1.707723in}{2.640857in}}{\pgfqpoint{1.704451in}{2.632957in}}{\pgfqpoint{1.704451in}{2.624721in}}%
\pgfpathcurveto{\pgfqpoint{1.704451in}{2.616485in}}{\pgfqpoint{1.707723in}{2.608585in}}{\pgfqpoint{1.713547in}{2.602761in}}%
\pgfpathcurveto{\pgfqpoint{1.719371in}{2.596937in}}{\pgfqpoint{1.727271in}{2.593665in}}{\pgfqpoint{1.735508in}{2.593665in}}%
\pgfpathclose%
\pgfusepath{stroke,fill}%
\end{pgfscope}%
\begin{pgfscope}%
\pgfpathrectangle{\pgfqpoint{0.100000in}{0.212622in}}{\pgfqpoint{3.696000in}{3.696000in}}%
\pgfusepath{clip}%
\pgfsetbuttcap%
\pgfsetroundjoin%
\definecolor{currentfill}{rgb}{0.121569,0.466667,0.705882}%
\pgfsetfillcolor{currentfill}%
\pgfsetfillopacity{0.907884}%
\pgfsetlinewidth{1.003750pt}%
\definecolor{currentstroke}{rgb}{0.121569,0.466667,0.705882}%
\pgfsetstrokecolor{currentstroke}%
\pgfsetstrokeopacity{0.907884}%
\pgfsetdash{}{0pt}%
\pgfpathmoveto{\pgfqpoint{1.735445in}{2.593596in}}%
\pgfpathcurveto{\pgfqpoint{1.743681in}{2.593596in}}{\pgfqpoint{1.751581in}{2.596868in}}{\pgfqpoint{1.757405in}{2.602692in}}%
\pgfpathcurveto{\pgfqpoint{1.763229in}{2.608516in}}{\pgfqpoint{1.766502in}{2.616416in}}{\pgfqpoint{1.766502in}{2.624653in}}%
\pgfpathcurveto{\pgfqpoint{1.766502in}{2.632889in}}{\pgfqpoint{1.763229in}{2.640789in}}{\pgfqpoint{1.757405in}{2.646613in}}%
\pgfpathcurveto{\pgfqpoint{1.751581in}{2.652437in}}{\pgfqpoint{1.743681in}{2.655709in}}{\pgfqpoint{1.735445in}{2.655709in}}%
\pgfpathcurveto{\pgfqpoint{1.727209in}{2.655709in}}{\pgfqpoint{1.719309in}{2.652437in}}{\pgfqpoint{1.713485in}{2.646613in}}%
\pgfpathcurveto{\pgfqpoint{1.707661in}{2.640789in}}{\pgfqpoint{1.704389in}{2.632889in}}{\pgfqpoint{1.704389in}{2.624653in}}%
\pgfpathcurveto{\pgfqpoint{1.704389in}{2.616416in}}{\pgfqpoint{1.707661in}{2.608516in}}{\pgfqpoint{1.713485in}{2.602692in}}%
\pgfpathcurveto{\pgfqpoint{1.719309in}{2.596868in}}{\pgfqpoint{1.727209in}{2.593596in}}{\pgfqpoint{1.735445in}{2.593596in}}%
\pgfpathclose%
\pgfusepath{stroke,fill}%
\end{pgfscope}%
\begin{pgfscope}%
\pgfpathrectangle{\pgfqpoint{0.100000in}{0.212622in}}{\pgfqpoint{3.696000in}{3.696000in}}%
\pgfusepath{clip}%
\pgfsetbuttcap%
\pgfsetroundjoin%
\definecolor{currentfill}{rgb}{0.121569,0.466667,0.705882}%
\pgfsetfillcolor{currentfill}%
\pgfsetfillopacity{0.907897}%
\pgfsetlinewidth{1.003750pt}%
\definecolor{currentstroke}{rgb}{0.121569,0.466667,0.705882}%
\pgfsetstrokecolor{currentstroke}%
\pgfsetstrokeopacity{0.907897}%
\pgfsetdash{}{0pt}%
\pgfpathmoveto{\pgfqpoint{1.735413in}{2.593552in}}%
\pgfpathcurveto{\pgfqpoint{1.743649in}{2.593552in}}{\pgfqpoint{1.751549in}{2.596824in}}{\pgfqpoint{1.757373in}{2.602648in}}%
\pgfpathcurveto{\pgfqpoint{1.763197in}{2.608472in}}{\pgfqpoint{1.766469in}{2.616372in}}{\pgfqpoint{1.766469in}{2.624609in}}%
\pgfpathcurveto{\pgfqpoint{1.766469in}{2.632845in}}{\pgfqpoint{1.763197in}{2.640745in}}{\pgfqpoint{1.757373in}{2.646569in}}%
\pgfpathcurveto{\pgfqpoint{1.751549in}{2.652393in}}{\pgfqpoint{1.743649in}{2.655665in}}{\pgfqpoint{1.735413in}{2.655665in}}%
\pgfpathcurveto{\pgfqpoint{1.727177in}{2.655665in}}{\pgfqpoint{1.719276in}{2.652393in}}{\pgfqpoint{1.713453in}{2.646569in}}%
\pgfpathcurveto{\pgfqpoint{1.707629in}{2.640745in}}{\pgfqpoint{1.704356in}{2.632845in}}{\pgfqpoint{1.704356in}{2.624609in}}%
\pgfpathcurveto{\pgfqpoint{1.704356in}{2.616372in}}{\pgfqpoint{1.707629in}{2.608472in}}{\pgfqpoint{1.713453in}{2.602648in}}%
\pgfpathcurveto{\pgfqpoint{1.719276in}{2.596824in}}{\pgfqpoint{1.727177in}{2.593552in}}{\pgfqpoint{1.735413in}{2.593552in}}%
\pgfpathclose%
\pgfusepath{stroke,fill}%
\end{pgfscope}%
\begin{pgfscope}%
\pgfpathrectangle{\pgfqpoint{0.100000in}{0.212622in}}{\pgfqpoint{3.696000in}{3.696000in}}%
\pgfusepath{clip}%
\pgfsetbuttcap%
\pgfsetroundjoin%
\definecolor{currentfill}{rgb}{0.121569,0.466667,0.705882}%
\pgfsetfillcolor{currentfill}%
\pgfsetfillopacity{0.907904}%
\pgfsetlinewidth{1.003750pt}%
\definecolor{currentstroke}{rgb}{0.121569,0.466667,0.705882}%
\pgfsetstrokecolor{currentstroke}%
\pgfsetstrokeopacity{0.907904}%
\pgfsetdash{}{0pt}%
\pgfpathmoveto{\pgfqpoint{1.735395in}{2.593531in}}%
\pgfpathcurveto{\pgfqpoint{1.743632in}{2.593531in}}{\pgfqpoint{1.751532in}{2.596804in}}{\pgfqpoint{1.757356in}{2.602627in}}%
\pgfpathcurveto{\pgfqpoint{1.763180in}{2.608451in}}{\pgfqpoint{1.766452in}{2.616351in}}{\pgfqpoint{1.766452in}{2.624588in}}%
\pgfpathcurveto{\pgfqpoint{1.766452in}{2.632824in}}{\pgfqpoint{1.763180in}{2.640724in}}{\pgfqpoint{1.757356in}{2.646548in}}%
\pgfpathcurveto{\pgfqpoint{1.751532in}{2.652372in}}{\pgfqpoint{1.743632in}{2.655644in}}{\pgfqpoint{1.735395in}{2.655644in}}%
\pgfpathcurveto{\pgfqpoint{1.727159in}{2.655644in}}{\pgfqpoint{1.719259in}{2.652372in}}{\pgfqpoint{1.713435in}{2.646548in}}%
\pgfpathcurveto{\pgfqpoint{1.707611in}{2.640724in}}{\pgfqpoint{1.704339in}{2.632824in}}{\pgfqpoint{1.704339in}{2.624588in}}%
\pgfpathcurveto{\pgfqpoint{1.704339in}{2.616351in}}{\pgfqpoint{1.707611in}{2.608451in}}{\pgfqpoint{1.713435in}{2.602627in}}%
\pgfpathcurveto{\pgfqpoint{1.719259in}{2.596804in}}{\pgfqpoint{1.727159in}{2.593531in}}{\pgfqpoint{1.735395in}{2.593531in}}%
\pgfpathclose%
\pgfusepath{stroke,fill}%
\end{pgfscope}%
\begin{pgfscope}%
\pgfpathrectangle{\pgfqpoint{0.100000in}{0.212622in}}{\pgfqpoint{3.696000in}{3.696000in}}%
\pgfusepath{clip}%
\pgfsetbuttcap%
\pgfsetroundjoin%
\definecolor{currentfill}{rgb}{0.121569,0.466667,0.705882}%
\pgfsetfillcolor{currentfill}%
\pgfsetfillopacity{0.907908}%
\pgfsetlinewidth{1.003750pt}%
\definecolor{currentstroke}{rgb}{0.121569,0.466667,0.705882}%
\pgfsetstrokecolor{currentstroke}%
\pgfsetstrokeopacity{0.907908}%
\pgfsetdash{}{0pt}%
\pgfpathmoveto{\pgfqpoint{1.735385in}{2.593519in}}%
\pgfpathcurveto{\pgfqpoint{1.743622in}{2.593519in}}{\pgfqpoint{1.751522in}{2.596791in}}{\pgfqpoint{1.757346in}{2.602615in}}%
\pgfpathcurveto{\pgfqpoint{1.763170in}{2.608439in}}{\pgfqpoint{1.766442in}{2.616339in}}{\pgfqpoint{1.766442in}{2.624575in}}%
\pgfpathcurveto{\pgfqpoint{1.766442in}{2.632812in}}{\pgfqpoint{1.763170in}{2.640712in}}{\pgfqpoint{1.757346in}{2.646536in}}%
\pgfpathcurveto{\pgfqpoint{1.751522in}{2.652360in}}{\pgfqpoint{1.743622in}{2.655632in}}{\pgfqpoint{1.735385in}{2.655632in}}%
\pgfpathcurveto{\pgfqpoint{1.727149in}{2.655632in}}{\pgfqpoint{1.719249in}{2.652360in}}{\pgfqpoint{1.713425in}{2.646536in}}%
\pgfpathcurveto{\pgfqpoint{1.707601in}{2.640712in}}{\pgfqpoint{1.704329in}{2.632812in}}{\pgfqpoint{1.704329in}{2.624575in}}%
\pgfpathcurveto{\pgfqpoint{1.704329in}{2.616339in}}{\pgfqpoint{1.707601in}{2.608439in}}{\pgfqpoint{1.713425in}{2.602615in}}%
\pgfpathcurveto{\pgfqpoint{1.719249in}{2.596791in}}{\pgfqpoint{1.727149in}{2.593519in}}{\pgfqpoint{1.735385in}{2.593519in}}%
\pgfpathclose%
\pgfusepath{stroke,fill}%
\end{pgfscope}%
\begin{pgfscope}%
\pgfpathrectangle{\pgfqpoint{0.100000in}{0.212622in}}{\pgfqpoint{3.696000in}{3.696000in}}%
\pgfusepath{clip}%
\pgfsetbuttcap%
\pgfsetroundjoin%
\definecolor{currentfill}{rgb}{0.121569,0.466667,0.705882}%
\pgfsetfillcolor{currentfill}%
\pgfsetfillopacity{0.907911}%
\pgfsetlinewidth{1.003750pt}%
\definecolor{currentstroke}{rgb}{0.121569,0.466667,0.705882}%
\pgfsetstrokecolor{currentstroke}%
\pgfsetstrokeopacity{0.907911}%
\pgfsetdash{}{0pt}%
\pgfpathmoveto{\pgfqpoint{1.735380in}{2.593512in}}%
\pgfpathcurveto{\pgfqpoint{1.743617in}{2.593512in}}{\pgfqpoint{1.751517in}{2.596785in}}{\pgfqpoint{1.757341in}{2.602609in}}%
\pgfpathcurveto{\pgfqpoint{1.763164in}{2.608433in}}{\pgfqpoint{1.766437in}{2.616333in}}{\pgfqpoint{1.766437in}{2.624569in}}%
\pgfpathcurveto{\pgfqpoint{1.766437in}{2.632805in}}{\pgfqpoint{1.763164in}{2.640705in}}{\pgfqpoint{1.757341in}{2.646529in}}%
\pgfpathcurveto{\pgfqpoint{1.751517in}{2.652353in}}{\pgfqpoint{1.743617in}{2.655625in}}{\pgfqpoint{1.735380in}{2.655625in}}%
\pgfpathcurveto{\pgfqpoint{1.727144in}{2.655625in}}{\pgfqpoint{1.719244in}{2.652353in}}{\pgfqpoint{1.713420in}{2.646529in}}%
\pgfpathcurveto{\pgfqpoint{1.707596in}{2.640705in}}{\pgfqpoint{1.704324in}{2.632805in}}{\pgfqpoint{1.704324in}{2.624569in}}%
\pgfpathcurveto{\pgfqpoint{1.704324in}{2.616333in}}{\pgfqpoint{1.707596in}{2.608433in}}{\pgfqpoint{1.713420in}{2.602609in}}%
\pgfpathcurveto{\pgfqpoint{1.719244in}{2.596785in}}{\pgfqpoint{1.727144in}{2.593512in}}{\pgfqpoint{1.735380in}{2.593512in}}%
\pgfpathclose%
\pgfusepath{stroke,fill}%
\end{pgfscope}%
\begin{pgfscope}%
\pgfpathrectangle{\pgfqpoint{0.100000in}{0.212622in}}{\pgfqpoint{3.696000in}{3.696000in}}%
\pgfusepath{clip}%
\pgfsetbuttcap%
\pgfsetroundjoin%
\definecolor{currentfill}{rgb}{0.121569,0.466667,0.705882}%
\pgfsetfillcolor{currentfill}%
\pgfsetfillopacity{0.907912}%
\pgfsetlinewidth{1.003750pt}%
\definecolor{currentstroke}{rgb}{0.121569,0.466667,0.705882}%
\pgfsetstrokecolor{currentstroke}%
\pgfsetstrokeopacity{0.907912}%
\pgfsetdash{}{0pt}%
\pgfpathmoveto{\pgfqpoint{1.735378in}{2.593509in}}%
\pgfpathcurveto{\pgfqpoint{1.743614in}{2.593509in}}{\pgfqpoint{1.751514in}{2.596781in}}{\pgfqpoint{1.757338in}{2.602605in}}%
\pgfpathcurveto{\pgfqpoint{1.763162in}{2.608429in}}{\pgfqpoint{1.766434in}{2.616329in}}{\pgfqpoint{1.766434in}{2.624566in}}%
\pgfpathcurveto{\pgfqpoint{1.766434in}{2.632802in}}{\pgfqpoint{1.763162in}{2.640702in}}{\pgfqpoint{1.757338in}{2.646526in}}%
\pgfpathcurveto{\pgfqpoint{1.751514in}{2.652350in}}{\pgfqpoint{1.743614in}{2.655622in}}{\pgfqpoint{1.735378in}{2.655622in}}%
\pgfpathcurveto{\pgfqpoint{1.727141in}{2.655622in}}{\pgfqpoint{1.719241in}{2.652350in}}{\pgfqpoint{1.713417in}{2.646526in}}%
\pgfpathcurveto{\pgfqpoint{1.707593in}{2.640702in}}{\pgfqpoint{1.704321in}{2.632802in}}{\pgfqpoint{1.704321in}{2.624566in}}%
\pgfpathcurveto{\pgfqpoint{1.704321in}{2.616329in}}{\pgfqpoint{1.707593in}{2.608429in}}{\pgfqpoint{1.713417in}{2.602605in}}%
\pgfpathcurveto{\pgfqpoint{1.719241in}{2.596781in}}{\pgfqpoint{1.727141in}{2.593509in}}{\pgfqpoint{1.735378in}{2.593509in}}%
\pgfpathclose%
\pgfusepath{stroke,fill}%
\end{pgfscope}%
\begin{pgfscope}%
\pgfpathrectangle{\pgfqpoint{0.100000in}{0.212622in}}{\pgfqpoint{3.696000in}{3.696000in}}%
\pgfusepath{clip}%
\pgfsetbuttcap%
\pgfsetroundjoin%
\definecolor{currentfill}{rgb}{0.121569,0.466667,0.705882}%
\pgfsetfillcolor{currentfill}%
\pgfsetfillopacity{0.907913}%
\pgfsetlinewidth{1.003750pt}%
\definecolor{currentstroke}{rgb}{0.121569,0.466667,0.705882}%
\pgfsetstrokecolor{currentstroke}%
\pgfsetstrokeopacity{0.907913}%
\pgfsetdash{}{0pt}%
\pgfpathmoveto{\pgfqpoint{1.735376in}{2.593507in}}%
\pgfpathcurveto{\pgfqpoint{1.743612in}{2.593507in}}{\pgfqpoint{1.751512in}{2.596779in}}{\pgfqpoint{1.757336in}{2.602603in}}%
\pgfpathcurveto{\pgfqpoint{1.763160in}{2.608427in}}{\pgfqpoint{1.766433in}{2.616327in}}{\pgfqpoint{1.766433in}{2.624563in}}%
\pgfpathcurveto{\pgfqpoint{1.766433in}{2.632800in}}{\pgfqpoint{1.763160in}{2.640700in}}{\pgfqpoint{1.757336in}{2.646524in}}%
\pgfpathcurveto{\pgfqpoint{1.751512in}{2.652348in}}{\pgfqpoint{1.743612in}{2.655620in}}{\pgfqpoint{1.735376in}{2.655620in}}%
\pgfpathcurveto{\pgfqpoint{1.727140in}{2.655620in}}{\pgfqpoint{1.719240in}{2.652348in}}{\pgfqpoint{1.713416in}{2.646524in}}%
\pgfpathcurveto{\pgfqpoint{1.707592in}{2.640700in}}{\pgfqpoint{1.704320in}{2.632800in}}{\pgfqpoint{1.704320in}{2.624563in}}%
\pgfpathcurveto{\pgfqpoint{1.704320in}{2.616327in}}{\pgfqpoint{1.707592in}{2.608427in}}{\pgfqpoint{1.713416in}{2.602603in}}%
\pgfpathcurveto{\pgfqpoint{1.719240in}{2.596779in}}{\pgfqpoint{1.727140in}{2.593507in}}{\pgfqpoint{1.735376in}{2.593507in}}%
\pgfpathclose%
\pgfusepath{stroke,fill}%
\end{pgfscope}%
\begin{pgfscope}%
\pgfpathrectangle{\pgfqpoint{0.100000in}{0.212622in}}{\pgfqpoint{3.696000in}{3.696000in}}%
\pgfusepath{clip}%
\pgfsetbuttcap%
\pgfsetroundjoin%
\definecolor{currentfill}{rgb}{0.121569,0.466667,0.705882}%
\pgfsetfillcolor{currentfill}%
\pgfsetfillopacity{0.907913}%
\pgfsetlinewidth{1.003750pt}%
\definecolor{currentstroke}{rgb}{0.121569,0.466667,0.705882}%
\pgfsetstrokecolor{currentstroke}%
\pgfsetstrokeopacity{0.907913}%
\pgfsetdash{}{0pt}%
\pgfpathmoveto{\pgfqpoint{1.735375in}{2.593506in}}%
\pgfpathcurveto{\pgfqpoint{1.743612in}{2.593506in}}{\pgfqpoint{1.751512in}{2.596778in}}{\pgfqpoint{1.757336in}{2.602602in}}%
\pgfpathcurveto{\pgfqpoint{1.763160in}{2.608426in}}{\pgfqpoint{1.766432in}{2.616326in}}{\pgfqpoint{1.766432in}{2.624562in}}%
\pgfpathcurveto{\pgfqpoint{1.766432in}{2.632799in}}{\pgfqpoint{1.763160in}{2.640699in}}{\pgfqpoint{1.757336in}{2.646523in}}%
\pgfpathcurveto{\pgfqpoint{1.751512in}{2.652346in}}{\pgfqpoint{1.743612in}{2.655619in}}{\pgfqpoint{1.735375in}{2.655619in}}%
\pgfpathcurveto{\pgfqpoint{1.727139in}{2.655619in}}{\pgfqpoint{1.719239in}{2.652346in}}{\pgfqpoint{1.713415in}{2.646523in}}%
\pgfpathcurveto{\pgfqpoint{1.707591in}{2.640699in}}{\pgfqpoint{1.704319in}{2.632799in}}{\pgfqpoint{1.704319in}{2.624562in}}%
\pgfpathcurveto{\pgfqpoint{1.704319in}{2.616326in}}{\pgfqpoint{1.707591in}{2.608426in}}{\pgfqpoint{1.713415in}{2.602602in}}%
\pgfpathcurveto{\pgfqpoint{1.719239in}{2.596778in}}{\pgfqpoint{1.727139in}{2.593506in}}{\pgfqpoint{1.735375in}{2.593506in}}%
\pgfpathclose%
\pgfusepath{stroke,fill}%
\end{pgfscope}%
\begin{pgfscope}%
\pgfpathrectangle{\pgfqpoint{0.100000in}{0.212622in}}{\pgfqpoint{3.696000in}{3.696000in}}%
\pgfusepath{clip}%
\pgfsetbuttcap%
\pgfsetroundjoin%
\definecolor{currentfill}{rgb}{0.121569,0.466667,0.705882}%
\pgfsetfillcolor{currentfill}%
\pgfsetfillopacity{0.907913}%
\pgfsetlinewidth{1.003750pt}%
\definecolor{currentstroke}{rgb}{0.121569,0.466667,0.705882}%
\pgfsetstrokecolor{currentstroke}%
\pgfsetstrokeopacity{0.907913}%
\pgfsetdash{}{0pt}%
\pgfpathmoveto{\pgfqpoint{1.735375in}{2.593505in}}%
\pgfpathcurveto{\pgfqpoint{1.743611in}{2.593505in}}{\pgfqpoint{1.751511in}{2.596777in}}{\pgfqpoint{1.757335in}{2.602601in}}%
\pgfpathcurveto{\pgfqpoint{1.763159in}{2.608425in}}{\pgfqpoint{1.766431in}{2.616325in}}{\pgfqpoint{1.766431in}{2.624562in}}%
\pgfpathcurveto{\pgfqpoint{1.766431in}{2.632798in}}{\pgfqpoint{1.763159in}{2.640698in}}{\pgfqpoint{1.757335in}{2.646522in}}%
\pgfpathcurveto{\pgfqpoint{1.751511in}{2.652346in}}{\pgfqpoint{1.743611in}{2.655618in}}{\pgfqpoint{1.735375in}{2.655618in}}%
\pgfpathcurveto{\pgfqpoint{1.727139in}{2.655618in}}{\pgfqpoint{1.719239in}{2.652346in}}{\pgfqpoint{1.713415in}{2.646522in}}%
\pgfpathcurveto{\pgfqpoint{1.707591in}{2.640698in}}{\pgfqpoint{1.704318in}{2.632798in}}{\pgfqpoint{1.704318in}{2.624562in}}%
\pgfpathcurveto{\pgfqpoint{1.704318in}{2.616325in}}{\pgfqpoint{1.707591in}{2.608425in}}{\pgfqpoint{1.713415in}{2.602601in}}%
\pgfpathcurveto{\pgfqpoint{1.719239in}{2.596777in}}{\pgfqpoint{1.727139in}{2.593505in}}{\pgfqpoint{1.735375in}{2.593505in}}%
\pgfpathclose%
\pgfusepath{stroke,fill}%
\end{pgfscope}%
\begin{pgfscope}%
\pgfpathrectangle{\pgfqpoint{0.100000in}{0.212622in}}{\pgfqpoint{3.696000in}{3.696000in}}%
\pgfusepath{clip}%
\pgfsetbuttcap%
\pgfsetroundjoin%
\definecolor{currentfill}{rgb}{0.121569,0.466667,0.705882}%
\pgfsetfillcolor{currentfill}%
\pgfsetfillopacity{0.908021}%
\pgfsetlinewidth{1.003750pt}%
\definecolor{currentstroke}{rgb}{0.121569,0.466667,0.705882}%
\pgfsetstrokecolor{currentstroke}%
\pgfsetstrokeopacity{0.908021}%
\pgfsetdash{}{0pt}%
\pgfpathmoveto{\pgfqpoint{2.169630in}{2.486012in}}%
\pgfpathcurveto{\pgfqpoint{2.177866in}{2.486012in}}{\pgfqpoint{2.185767in}{2.489284in}}{\pgfqpoint{2.191590in}{2.495108in}}%
\pgfpathcurveto{\pgfqpoint{2.197414in}{2.500932in}}{\pgfqpoint{2.200687in}{2.508832in}}{\pgfqpoint{2.200687in}{2.517068in}}%
\pgfpathcurveto{\pgfqpoint{2.200687in}{2.525304in}}{\pgfqpoint{2.197414in}{2.533205in}}{\pgfqpoint{2.191590in}{2.539028in}}%
\pgfpathcurveto{\pgfqpoint{2.185767in}{2.544852in}}{\pgfqpoint{2.177866in}{2.548125in}}{\pgfqpoint{2.169630in}{2.548125in}}%
\pgfpathcurveto{\pgfqpoint{2.161394in}{2.548125in}}{\pgfqpoint{2.153494in}{2.544852in}}{\pgfqpoint{2.147670in}{2.539028in}}%
\pgfpathcurveto{\pgfqpoint{2.141846in}{2.533205in}}{\pgfqpoint{2.138574in}{2.525304in}}{\pgfqpoint{2.138574in}{2.517068in}}%
\pgfpathcurveto{\pgfqpoint{2.138574in}{2.508832in}}{\pgfqpoint{2.141846in}{2.500932in}}{\pgfqpoint{2.147670in}{2.495108in}}%
\pgfpathcurveto{\pgfqpoint{2.153494in}{2.489284in}}{\pgfqpoint{2.161394in}{2.486012in}}{\pgfqpoint{2.169630in}{2.486012in}}%
\pgfpathclose%
\pgfusepath{stroke,fill}%
\end{pgfscope}%
\begin{pgfscope}%
\pgfpathrectangle{\pgfqpoint{0.100000in}{0.212622in}}{\pgfqpoint{3.696000in}{3.696000in}}%
\pgfusepath{clip}%
\pgfsetbuttcap%
\pgfsetroundjoin%
\definecolor{currentfill}{rgb}{0.121569,0.466667,0.705882}%
\pgfsetfillcolor{currentfill}%
\pgfsetfillopacity{0.908022}%
\pgfsetlinewidth{1.003750pt}%
\definecolor{currentstroke}{rgb}{0.121569,0.466667,0.705882}%
\pgfsetstrokecolor{currentstroke}%
\pgfsetstrokeopacity{0.908022}%
\pgfsetdash{}{0pt}%
\pgfpathmoveto{\pgfqpoint{1.735169in}{2.593176in}}%
\pgfpathcurveto{\pgfqpoint{1.743405in}{2.593176in}}{\pgfqpoint{1.751305in}{2.596448in}}{\pgfqpoint{1.757129in}{2.602272in}}%
\pgfpathcurveto{\pgfqpoint{1.762953in}{2.608096in}}{\pgfqpoint{1.766225in}{2.615996in}}{\pgfqpoint{1.766225in}{2.624232in}}%
\pgfpathcurveto{\pgfqpoint{1.766225in}{2.632468in}}{\pgfqpoint{1.762953in}{2.640368in}}{\pgfqpoint{1.757129in}{2.646192in}}%
\pgfpathcurveto{\pgfqpoint{1.751305in}{2.652016in}}{\pgfqpoint{1.743405in}{2.655289in}}{\pgfqpoint{1.735169in}{2.655289in}}%
\pgfpathcurveto{\pgfqpoint{1.726933in}{2.655289in}}{\pgfqpoint{1.719032in}{2.652016in}}{\pgfqpoint{1.713209in}{2.646192in}}%
\pgfpathcurveto{\pgfqpoint{1.707385in}{2.640368in}}{\pgfqpoint{1.704112in}{2.632468in}}{\pgfqpoint{1.704112in}{2.624232in}}%
\pgfpathcurveto{\pgfqpoint{1.704112in}{2.615996in}}{\pgfqpoint{1.707385in}{2.608096in}}{\pgfqpoint{1.713209in}{2.602272in}}%
\pgfpathcurveto{\pgfqpoint{1.719032in}{2.596448in}}{\pgfqpoint{1.726933in}{2.593176in}}{\pgfqpoint{1.735169in}{2.593176in}}%
\pgfpathclose%
\pgfusepath{stroke,fill}%
\end{pgfscope}%
\begin{pgfscope}%
\pgfpathrectangle{\pgfqpoint{0.100000in}{0.212622in}}{\pgfqpoint{3.696000in}{3.696000in}}%
\pgfusepath{clip}%
\pgfsetbuttcap%
\pgfsetroundjoin%
\definecolor{currentfill}{rgb}{0.121569,0.466667,0.705882}%
\pgfsetfillcolor{currentfill}%
\pgfsetfillopacity{0.908205}%
\pgfsetlinewidth{1.003750pt}%
\definecolor{currentstroke}{rgb}{0.121569,0.466667,0.705882}%
\pgfsetstrokecolor{currentstroke}%
\pgfsetstrokeopacity{0.908205}%
\pgfsetdash{}{0pt}%
\pgfpathmoveto{\pgfqpoint{1.734829in}{2.592495in}}%
\pgfpathcurveto{\pgfqpoint{1.743065in}{2.592495in}}{\pgfqpoint{1.750965in}{2.595767in}}{\pgfqpoint{1.756789in}{2.601591in}}%
\pgfpathcurveto{\pgfqpoint{1.762613in}{2.607415in}}{\pgfqpoint{1.765886in}{2.615315in}}{\pgfqpoint{1.765886in}{2.623552in}}%
\pgfpathcurveto{\pgfqpoint{1.765886in}{2.631788in}}{\pgfqpoint{1.762613in}{2.639688in}}{\pgfqpoint{1.756789in}{2.645512in}}%
\pgfpathcurveto{\pgfqpoint{1.750965in}{2.651336in}}{\pgfqpoint{1.743065in}{2.654608in}}{\pgfqpoint{1.734829in}{2.654608in}}%
\pgfpathcurveto{\pgfqpoint{1.726593in}{2.654608in}}{\pgfqpoint{1.718693in}{2.651336in}}{\pgfqpoint{1.712869in}{2.645512in}}%
\pgfpathcurveto{\pgfqpoint{1.707045in}{2.639688in}}{\pgfqpoint{1.703773in}{2.631788in}}{\pgfqpoint{1.703773in}{2.623552in}}%
\pgfpathcurveto{\pgfqpoint{1.703773in}{2.615315in}}{\pgfqpoint{1.707045in}{2.607415in}}{\pgfqpoint{1.712869in}{2.601591in}}%
\pgfpathcurveto{\pgfqpoint{1.718693in}{2.595767in}}{\pgfqpoint{1.726593in}{2.592495in}}{\pgfqpoint{1.734829in}{2.592495in}}%
\pgfpathclose%
\pgfusepath{stroke,fill}%
\end{pgfscope}%
\begin{pgfscope}%
\pgfpathrectangle{\pgfqpoint{0.100000in}{0.212622in}}{\pgfqpoint{3.696000in}{3.696000in}}%
\pgfusepath{clip}%
\pgfsetbuttcap%
\pgfsetroundjoin%
\definecolor{currentfill}{rgb}{0.121569,0.466667,0.705882}%
\pgfsetfillcolor{currentfill}%
\pgfsetfillopacity{0.908499}%
\pgfsetlinewidth{1.003750pt}%
\definecolor{currentstroke}{rgb}{0.121569,0.466667,0.705882}%
\pgfsetstrokecolor{currentstroke}%
\pgfsetstrokeopacity{0.908499}%
\pgfsetdash{}{0pt}%
\pgfpathmoveto{\pgfqpoint{0.971590in}{2.050525in}}%
\pgfpathcurveto{\pgfqpoint{0.979826in}{2.050525in}}{\pgfqpoint{0.987726in}{2.053797in}}{\pgfqpoint{0.993550in}{2.059621in}}%
\pgfpathcurveto{\pgfqpoint{0.999374in}{2.065445in}}{\pgfqpoint{1.002646in}{2.073345in}}{\pgfqpoint{1.002646in}{2.081581in}}%
\pgfpathcurveto{\pgfqpoint{1.002646in}{2.089818in}}{\pgfqpoint{0.999374in}{2.097718in}}{\pgfqpoint{0.993550in}{2.103542in}}%
\pgfpathcurveto{\pgfqpoint{0.987726in}{2.109366in}}{\pgfqpoint{0.979826in}{2.112638in}}{\pgfqpoint{0.971590in}{2.112638in}}%
\pgfpathcurveto{\pgfqpoint{0.963354in}{2.112638in}}{\pgfqpoint{0.955453in}{2.109366in}}{\pgfqpoint{0.949630in}{2.103542in}}%
\pgfpathcurveto{\pgfqpoint{0.943806in}{2.097718in}}{\pgfqpoint{0.940533in}{2.089818in}}{\pgfqpoint{0.940533in}{2.081581in}}%
\pgfpathcurveto{\pgfqpoint{0.940533in}{2.073345in}}{\pgfqpoint{0.943806in}{2.065445in}}{\pgfqpoint{0.949630in}{2.059621in}}%
\pgfpathcurveto{\pgfqpoint{0.955453in}{2.053797in}}{\pgfqpoint{0.963354in}{2.050525in}}{\pgfqpoint{0.971590in}{2.050525in}}%
\pgfpathclose%
\pgfusepath{stroke,fill}%
\end{pgfscope}%
\begin{pgfscope}%
\pgfpathrectangle{\pgfqpoint{0.100000in}{0.212622in}}{\pgfqpoint{3.696000in}{3.696000in}}%
\pgfusepath{clip}%
\pgfsetbuttcap%
\pgfsetroundjoin%
\definecolor{currentfill}{rgb}{0.121569,0.466667,0.705882}%
\pgfsetfillcolor{currentfill}%
\pgfsetfillopacity{0.908522}%
\pgfsetlinewidth{1.003750pt}%
\definecolor{currentstroke}{rgb}{0.121569,0.466667,0.705882}%
\pgfsetstrokecolor{currentstroke}%
\pgfsetstrokeopacity{0.908522}%
\pgfsetdash{}{0pt}%
\pgfpathmoveto{\pgfqpoint{1.734312in}{2.591450in}}%
\pgfpathcurveto{\pgfqpoint{1.742548in}{2.591450in}}{\pgfqpoint{1.750448in}{2.594723in}}{\pgfqpoint{1.756272in}{2.600547in}}%
\pgfpathcurveto{\pgfqpoint{1.762096in}{2.606371in}}{\pgfqpoint{1.765368in}{2.614271in}}{\pgfqpoint{1.765368in}{2.622507in}}%
\pgfpathcurveto{\pgfqpoint{1.765368in}{2.630743in}}{\pgfqpoint{1.762096in}{2.638643in}}{\pgfqpoint{1.756272in}{2.644467in}}%
\pgfpathcurveto{\pgfqpoint{1.750448in}{2.650291in}}{\pgfqpoint{1.742548in}{2.653563in}}{\pgfqpoint{1.734312in}{2.653563in}}%
\pgfpathcurveto{\pgfqpoint{1.726076in}{2.653563in}}{\pgfqpoint{1.718176in}{2.650291in}}{\pgfqpoint{1.712352in}{2.644467in}}%
\pgfpathcurveto{\pgfqpoint{1.706528in}{2.638643in}}{\pgfqpoint{1.703255in}{2.630743in}}{\pgfqpoint{1.703255in}{2.622507in}}%
\pgfpathcurveto{\pgfqpoint{1.703255in}{2.614271in}}{\pgfqpoint{1.706528in}{2.606371in}}{\pgfqpoint{1.712352in}{2.600547in}}%
\pgfpathcurveto{\pgfqpoint{1.718176in}{2.594723in}}{\pgfqpoint{1.726076in}{2.591450in}}{\pgfqpoint{1.734312in}{2.591450in}}%
\pgfpathclose%
\pgfusepath{stroke,fill}%
\end{pgfscope}%
\begin{pgfscope}%
\pgfpathrectangle{\pgfqpoint{0.100000in}{0.212622in}}{\pgfqpoint{3.696000in}{3.696000in}}%
\pgfusepath{clip}%
\pgfsetbuttcap%
\pgfsetroundjoin%
\definecolor{currentfill}{rgb}{0.121569,0.466667,0.705882}%
\pgfsetfillcolor{currentfill}%
\pgfsetfillopacity{0.908792}%
\pgfsetlinewidth{1.003750pt}%
\definecolor{currentstroke}{rgb}{0.121569,0.466667,0.705882}%
\pgfsetstrokecolor{currentstroke}%
\pgfsetstrokeopacity{0.908792}%
\pgfsetdash{}{0pt}%
\pgfpathmoveto{\pgfqpoint{2.168253in}{2.484714in}}%
\pgfpathcurveto{\pgfqpoint{2.176489in}{2.484714in}}{\pgfqpoint{2.184389in}{2.487987in}}{\pgfqpoint{2.190213in}{2.493811in}}%
\pgfpathcurveto{\pgfqpoint{2.196037in}{2.499635in}}{\pgfqpoint{2.199309in}{2.507535in}}{\pgfqpoint{2.199309in}{2.515771in}}%
\pgfpathcurveto{\pgfqpoint{2.199309in}{2.524007in}}{\pgfqpoint{2.196037in}{2.531907in}}{\pgfqpoint{2.190213in}{2.537731in}}%
\pgfpathcurveto{\pgfqpoint{2.184389in}{2.543555in}}{\pgfqpoint{2.176489in}{2.546827in}}{\pgfqpoint{2.168253in}{2.546827in}}%
\pgfpathcurveto{\pgfqpoint{2.160016in}{2.546827in}}{\pgfqpoint{2.152116in}{2.543555in}}{\pgfqpoint{2.146292in}{2.537731in}}%
\pgfpathcurveto{\pgfqpoint{2.140468in}{2.531907in}}{\pgfqpoint{2.137196in}{2.524007in}}{\pgfqpoint{2.137196in}{2.515771in}}%
\pgfpathcurveto{\pgfqpoint{2.137196in}{2.507535in}}{\pgfqpoint{2.140468in}{2.499635in}}{\pgfqpoint{2.146292in}{2.493811in}}%
\pgfpathcurveto{\pgfqpoint{2.152116in}{2.487987in}}{\pgfqpoint{2.160016in}{2.484714in}}{\pgfqpoint{2.168253in}{2.484714in}}%
\pgfpathclose%
\pgfusepath{stroke,fill}%
\end{pgfscope}%
\begin{pgfscope}%
\pgfpathrectangle{\pgfqpoint{0.100000in}{0.212622in}}{\pgfqpoint{3.696000in}{3.696000in}}%
\pgfusepath{clip}%
\pgfsetbuttcap%
\pgfsetroundjoin%
\definecolor{currentfill}{rgb}{0.121569,0.466667,0.705882}%
\pgfsetfillcolor{currentfill}%
\pgfsetfillopacity{0.909015}%
\pgfsetlinewidth{1.003750pt}%
\definecolor{currentstroke}{rgb}{0.121569,0.466667,0.705882}%
\pgfsetstrokecolor{currentstroke}%
\pgfsetstrokeopacity{0.909015}%
\pgfsetdash{}{0pt}%
\pgfpathmoveto{\pgfqpoint{1.733593in}{2.590137in}}%
\pgfpathcurveto{\pgfqpoint{1.741829in}{2.590137in}}{\pgfqpoint{1.749729in}{2.593409in}}{\pgfqpoint{1.755553in}{2.599233in}}%
\pgfpathcurveto{\pgfqpoint{1.761377in}{2.605057in}}{\pgfqpoint{1.764649in}{2.612957in}}{\pgfqpoint{1.764649in}{2.621193in}}%
\pgfpathcurveto{\pgfqpoint{1.764649in}{2.629430in}}{\pgfqpoint{1.761377in}{2.637330in}}{\pgfqpoint{1.755553in}{2.643154in}}%
\pgfpathcurveto{\pgfqpoint{1.749729in}{2.648978in}}{\pgfqpoint{1.741829in}{2.652250in}}{\pgfqpoint{1.733593in}{2.652250in}}%
\pgfpathcurveto{\pgfqpoint{1.725357in}{2.652250in}}{\pgfqpoint{1.717457in}{2.648978in}}{\pgfqpoint{1.711633in}{2.643154in}}%
\pgfpathcurveto{\pgfqpoint{1.705809in}{2.637330in}}{\pgfqpoint{1.702536in}{2.629430in}}{\pgfqpoint{1.702536in}{2.621193in}}%
\pgfpathcurveto{\pgfqpoint{1.702536in}{2.612957in}}{\pgfqpoint{1.705809in}{2.605057in}}{\pgfqpoint{1.711633in}{2.599233in}}%
\pgfpathcurveto{\pgfqpoint{1.717457in}{2.593409in}}{\pgfqpoint{1.725357in}{2.590137in}}{\pgfqpoint{1.733593in}{2.590137in}}%
\pgfpathclose%
\pgfusepath{stroke,fill}%
\end{pgfscope}%
\begin{pgfscope}%
\pgfpathrectangle{\pgfqpoint{0.100000in}{0.212622in}}{\pgfqpoint{3.696000in}{3.696000in}}%
\pgfusepath{clip}%
\pgfsetbuttcap%
\pgfsetroundjoin%
\definecolor{currentfill}{rgb}{0.121569,0.466667,0.705882}%
\pgfsetfillcolor{currentfill}%
\pgfsetfillopacity{0.909644}%
\pgfsetlinewidth{1.003750pt}%
\definecolor{currentstroke}{rgb}{0.121569,0.466667,0.705882}%
\pgfsetstrokecolor{currentstroke}%
\pgfsetstrokeopacity{0.909644}%
\pgfsetdash{}{0pt}%
\pgfpathmoveto{\pgfqpoint{1.732753in}{2.588208in}}%
\pgfpathcurveto{\pgfqpoint{1.740989in}{2.588208in}}{\pgfqpoint{1.748889in}{2.591480in}}{\pgfqpoint{1.754713in}{2.597304in}}%
\pgfpathcurveto{\pgfqpoint{1.760537in}{2.603128in}}{\pgfqpoint{1.763809in}{2.611028in}}{\pgfqpoint{1.763809in}{2.619265in}}%
\pgfpathcurveto{\pgfqpoint{1.763809in}{2.627501in}}{\pgfqpoint{1.760537in}{2.635401in}}{\pgfqpoint{1.754713in}{2.641225in}}%
\pgfpathcurveto{\pgfqpoint{1.748889in}{2.647049in}}{\pgfqpoint{1.740989in}{2.650321in}}{\pgfqpoint{1.732753in}{2.650321in}}%
\pgfpathcurveto{\pgfqpoint{1.724516in}{2.650321in}}{\pgfqpoint{1.716616in}{2.647049in}}{\pgfqpoint{1.710792in}{2.641225in}}%
\pgfpathcurveto{\pgfqpoint{1.704969in}{2.635401in}}{\pgfqpoint{1.701696in}{2.627501in}}{\pgfqpoint{1.701696in}{2.619265in}}%
\pgfpathcurveto{\pgfqpoint{1.701696in}{2.611028in}}{\pgfqpoint{1.704969in}{2.603128in}}{\pgfqpoint{1.710792in}{2.597304in}}%
\pgfpathcurveto{\pgfqpoint{1.716616in}{2.591480in}}{\pgfqpoint{1.724516in}{2.588208in}}{\pgfqpoint{1.732753in}{2.588208in}}%
\pgfpathclose%
\pgfusepath{stroke,fill}%
\end{pgfscope}%
\begin{pgfscope}%
\pgfpathrectangle{\pgfqpoint{0.100000in}{0.212622in}}{\pgfqpoint{3.696000in}{3.696000in}}%
\pgfusepath{clip}%
\pgfsetbuttcap%
\pgfsetroundjoin%
\definecolor{currentfill}{rgb}{0.121569,0.466667,0.705882}%
\pgfsetfillcolor{currentfill}%
\pgfsetfillopacity{0.909737}%
\pgfsetlinewidth{1.003750pt}%
\definecolor{currentstroke}{rgb}{0.121569,0.466667,0.705882}%
\pgfsetstrokecolor{currentstroke}%
\pgfsetstrokeopacity{0.909737}%
\pgfsetdash{}{0pt}%
\pgfpathmoveto{\pgfqpoint{0.977525in}{2.046420in}}%
\pgfpathcurveto{\pgfqpoint{0.985761in}{2.046420in}}{\pgfqpoint{0.993662in}{2.049692in}}{\pgfqpoint{0.999485in}{2.055516in}}%
\pgfpathcurveto{\pgfqpoint{1.005309in}{2.061340in}}{\pgfqpoint{1.008582in}{2.069240in}}{\pgfqpoint{1.008582in}{2.077476in}}%
\pgfpathcurveto{\pgfqpoint{1.008582in}{2.085713in}}{\pgfqpoint{1.005309in}{2.093613in}}{\pgfqpoint{0.999485in}{2.099437in}}%
\pgfpathcurveto{\pgfqpoint{0.993662in}{2.105261in}}{\pgfqpoint{0.985761in}{2.108533in}}{\pgfqpoint{0.977525in}{2.108533in}}%
\pgfpathcurveto{\pgfqpoint{0.969289in}{2.108533in}}{\pgfqpoint{0.961389in}{2.105261in}}{\pgfqpoint{0.955565in}{2.099437in}}%
\pgfpathcurveto{\pgfqpoint{0.949741in}{2.093613in}}{\pgfqpoint{0.946469in}{2.085713in}}{\pgfqpoint{0.946469in}{2.077476in}}%
\pgfpathcurveto{\pgfqpoint{0.946469in}{2.069240in}}{\pgfqpoint{0.949741in}{2.061340in}}{\pgfqpoint{0.955565in}{2.055516in}}%
\pgfpathcurveto{\pgfqpoint{0.961389in}{2.049692in}}{\pgfqpoint{0.969289in}{2.046420in}}{\pgfqpoint{0.977525in}{2.046420in}}%
\pgfpathclose%
\pgfusepath{stroke,fill}%
\end{pgfscope}%
\begin{pgfscope}%
\pgfpathrectangle{\pgfqpoint{0.100000in}{0.212622in}}{\pgfqpoint{3.696000in}{3.696000in}}%
\pgfusepath{clip}%
\pgfsetbuttcap%
\pgfsetroundjoin%
\definecolor{currentfill}{rgb}{0.121569,0.466667,0.705882}%
\pgfsetfillcolor{currentfill}%
\pgfsetfillopacity{0.909998}%
\pgfsetlinewidth{1.003750pt}%
\definecolor{currentstroke}{rgb}{0.121569,0.466667,0.705882}%
\pgfsetstrokecolor{currentstroke}%
\pgfsetstrokeopacity{0.909998}%
\pgfsetdash{}{0pt}%
\pgfpathmoveto{\pgfqpoint{1.732354in}{2.587194in}}%
\pgfpathcurveto{\pgfqpoint{1.740590in}{2.587194in}}{\pgfqpoint{1.748490in}{2.590467in}}{\pgfqpoint{1.754314in}{2.596290in}}%
\pgfpathcurveto{\pgfqpoint{1.760138in}{2.602114in}}{\pgfqpoint{1.763410in}{2.610014in}}{\pgfqpoint{1.763410in}{2.618251in}}%
\pgfpathcurveto{\pgfqpoint{1.763410in}{2.626487in}}{\pgfqpoint{1.760138in}{2.634387in}}{\pgfqpoint{1.754314in}{2.640211in}}%
\pgfpathcurveto{\pgfqpoint{1.748490in}{2.646035in}}{\pgfqpoint{1.740590in}{2.649307in}}{\pgfqpoint{1.732354in}{2.649307in}}%
\pgfpathcurveto{\pgfqpoint{1.724117in}{2.649307in}}{\pgfqpoint{1.716217in}{2.646035in}}{\pgfqpoint{1.710393in}{2.640211in}}%
\pgfpathcurveto{\pgfqpoint{1.704569in}{2.634387in}}{\pgfqpoint{1.701297in}{2.626487in}}{\pgfqpoint{1.701297in}{2.618251in}}%
\pgfpathcurveto{\pgfqpoint{1.701297in}{2.610014in}}{\pgfqpoint{1.704569in}{2.602114in}}{\pgfqpoint{1.710393in}{2.596290in}}%
\pgfpathcurveto{\pgfqpoint{1.716217in}{2.590467in}}{\pgfqpoint{1.724117in}{2.587194in}}{\pgfqpoint{1.732354in}{2.587194in}}%
\pgfpathclose%
\pgfusepath{stroke,fill}%
\end{pgfscope}%
\begin{pgfscope}%
\pgfpathrectangle{\pgfqpoint{0.100000in}{0.212622in}}{\pgfqpoint{3.696000in}{3.696000in}}%
\pgfusepath{clip}%
\pgfsetbuttcap%
\pgfsetroundjoin%
\definecolor{currentfill}{rgb}{0.121569,0.466667,0.705882}%
\pgfsetfillcolor{currentfill}%
\pgfsetfillopacity{0.910177}%
\pgfsetlinewidth{1.003750pt}%
\definecolor{currentstroke}{rgb}{0.121569,0.466667,0.705882}%
\pgfsetstrokecolor{currentstroke}%
\pgfsetstrokeopacity{0.910177}%
\pgfsetdash{}{0pt}%
\pgfpathmoveto{\pgfqpoint{2.165696in}{2.482288in}}%
\pgfpathcurveto{\pgfqpoint{2.173932in}{2.482288in}}{\pgfqpoint{2.181832in}{2.485560in}}{\pgfqpoint{2.187656in}{2.491384in}}%
\pgfpathcurveto{\pgfqpoint{2.193480in}{2.497208in}}{\pgfqpoint{2.196752in}{2.505108in}}{\pgfqpoint{2.196752in}{2.513345in}}%
\pgfpathcurveto{\pgfqpoint{2.196752in}{2.521581in}}{\pgfqpoint{2.193480in}{2.529481in}}{\pgfqpoint{2.187656in}{2.535305in}}%
\pgfpathcurveto{\pgfqpoint{2.181832in}{2.541129in}}{\pgfqpoint{2.173932in}{2.544401in}}{\pgfqpoint{2.165696in}{2.544401in}}%
\pgfpathcurveto{\pgfqpoint{2.157459in}{2.544401in}}{\pgfqpoint{2.149559in}{2.541129in}}{\pgfqpoint{2.143735in}{2.535305in}}%
\pgfpathcurveto{\pgfqpoint{2.137912in}{2.529481in}}{\pgfqpoint{2.134639in}{2.521581in}}{\pgfqpoint{2.134639in}{2.513345in}}%
\pgfpathcurveto{\pgfqpoint{2.134639in}{2.505108in}}{\pgfqpoint{2.137912in}{2.497208in}}{\pgfqpoint{2.143735in}{2.491384in}}%
\pgfpathcurveto{\pgfqpoint{2.149559in}{2.485560in}}{\pgfqpoint{2.157459in}{2.482288in}}{\pgfqpoint{2.165696in}{2.482288in}}%
\pgfpathclose%
\pgfusepath{stroke,fill}%
\end{pgfscope}%
\begin{pgfscope}%
\pgfpathrectangle{\pgfqpoint{0.100000in}{0.212622in}}{\pgfqpoint{3.696000in}{3.696000in}}%
\pgfusepath{clip}%
\pgfsetbuttcap%
\pgfsetroundjoin%
\definecolor{currentfill}{rgb}{0.121569,0.466667,0.705882}%
\pgfsetfillcolor{currentfill}%
\pgfsetfillopacity{0.910484}%
\pgfsetlinewidth{1.003750pt}%
\definecolor{currentstroke}{rgb}{0.121569,0.466667,0.705882}%
\pgfsetstrokecolor{currentstroke}%
\pgfsetstrokeopacity{0.910484}%
\pgfsetdash{}{0pt}%
\pgfpathmoveto{\pgfqpoint{1.731891in}{2.585905in}}%
\pgfpathcurveto{\pgfqpoint{1.740128in}{2.585905in}}{\pgfqpoint{1.748028in}{2.589178in}}{\pgfqpoint{1.753852in}{2.595002in}}%
\pgfpathcurveto{\pgfqpoint{1.759676in}{2.600825in}}{\pgfqpoint{1.762948in}{2.608726in}}{\pgfqpoint{1.762948in}{2.616962in}}%
\pgfpathcurveto{\pgfqpoint{1.762948in}{2.625198in}}{\pgfqpoint{1.759676in}{2.633098in}}{\pgfqpoint{1.753852in}{2.638922in}}%
\pgfpathcurveto{\pgfqpoint{1.748028in}{2.644746in}}{\pgfqpoint{1.740128in}{2.648018in}}{\pgfqpoint{1.731891in}{2.648018in}}%
\pgfpathcurveto{\pgfqpoint{1.723655in}{2.648018in}}{\pgfqpoint{1.715755in}{2.644746in}}{\pgfqpoint{1.709931in}{2.638922in}}%
\pgfpathcurveto{\pgfqpoint{1.704107in}{2.633098in}}{\pgfqpoint{1.700835in}{2.625198in}}{\pgfqpoint{1.700835in}{2.616962in}}%
\pgfpathcurveto{\pgfqpoint{1.700835in}{2.608726in}}{\pgfqpoint{1.704107in}{2.600825in}}{\pgfqpoint{1.709931in}{2.595002in}}%
\pgfpathcurveto{\pgfqpoint{1.715755in}{2.589178in}}{\pgfqpoint{1.723655in}{2.585905in}}{\pgfqpoint{1.731891in}{2.585905in}}%
\pgfpathclose%
\pgfusepath{stroke,fill}%
\end{pgfscope}%
\begin{pgfscope}%
\pgfpathrectangle{\pgfqpoint{0.100000in}{0.212622in}}{\pgfqpoint{3.696000in}{3.696000in}}%
\pgfusepath{clip}%
\pgfsetbuttcap%
\pgfsetroundjoin%
\definecolor{currentfill}{rgb}{0.121569,0.466667,0.705882}%
\pgfsetfillcolor{currentfill}%
\pgfsetfillopacity{0.910953}%
\pgfsetlinewidth{1.003750pt}%
\definecolor{currentstroke}{rgb}{0.121569,0.466667,0.705882}%
\pgfsetstrokecolor{currentstroke}%
\pgfsetstrokeopacity{0.910953}%
\pgfsetdash{}{0pt}%
\pgfpathmoveto{\pgfqpoint{2.164310in}{2.481069in}}%
\pgfpathcurveto{\pgfqpoint{2.172546in}{2.481069in}}{\pgfqpoint{2.180446in}{2.484341in}}{\pgfqpoint{2.186270in}{2.490165in}}%
\pgfpathcurveto{\pgfqpoint{2.192094in}{2.495989in}}{\pgfqpoint{2.195367in}{2.503889in}}{\pgfqpoint{2.195367in}{2.512126in}}%
\pgfpathcurveto{\pgfqpoint{2.195367in}{2.520362in}}{\pgfqpoint{2.192094in}{2.528262in}}{\pgfqpoint{2.186270in}{2.534086in}}%
\pgfpathcurveto{\pgfqpoint{2.180446in}{2.539910in}}{\pgfqpoint{2.172546in}{2.543182in}}{\pgfqpoint{2.164310in}{2.543182in}}%
\pgfpathcurveto{\pgfqpoint{2.156074in}{2.543182in}}{\pgfqpoint{2.148174in}{2.539910in}}{\pgfqpoint{2.142350in}{2.534086in}}%
\pgfpathcurveto{\pgfqpoint{2.136526in}{2.528262in}}{\pgfqpoint{2.133254in}{2.520362in}}{\pgfqpoint{2.133254in}{2.512126in}}%
\pgfpathcurveto{\pgfqpoint{2.133254in}{2.503889in}}{\pgfqpoint{2.136526in}{2.495989in}}{\pgfqpoint{2.142350in}{2.490165in}}%
\pgfpathcurveto{\pgfqpoint{2.148174in}{2.484341in}}{\pgfqpoint{2.156074in}{2.481069in}}{\pgfqpoint{2.164310in}{2.481069in}}%
\pgfpathclose%
\pgfusepath{stroke,fill}%
\end{pgfscope}%
\begin{pgfscope}%
\pgfpathrectangle{\pgfqpoint{0.100000in}{0.212622in}}{\pgfqpoint{3.696000in}{3.696000in}}%
\pgfusepath{clip}%
\pgfsetbuttcap%
\pgfsetroundjoin%
\definecolor{currentfill}{rgb}{0.121569,0.466667,0.705882}%
\pgfsetfillcolor{currentfill}%
\pgfsetfillopacity{0.910995}%
\pgfsetlinewidth{1.003750pt}%
\definecolor{currentstroke}{rgb}{0.121569,0.466667,0.705882}%
\pgfsetstrokecolor{currentstroke}%
\pgfsetstrokeopacity{0.910995}%
\pgfsetdash{}{0pt}%
\pgfpathmoveto{\pgfqpoint{0.984322in}{2.042211in}}%
\pgfpathcurveto{\pgfqpoint{0.992558in}{2.042211in}}{\pgfqpoint{1.000458in}{2.045483in}}{\pgfqpoint{1.006282in}{2.051307in}}%
\pgfpathcurveto{\pgfqpoint{1.012106in}{2.057131in}}{\pgfqpoint{1.015378in}{2.065031in}}{\pgfqpoint{1.015378in}{2.073267in}}%
\pgfpathcurveto{\pgfqpoint{1.015378in}{2.081504in}}{\pgfqpoint{1.012106in}{2.089404in}}{\pgfqpoint{1.006282in}{2.095227in}}%
\pgfpathcurveto{\pgfqpoint{1.000458in}{2.101051in}}{\pgfqpoint{0.992558in}{2.104324in}}{\pgfqpoint{0.984322in}{2.104324in}}%
\pgfpathcurveto{\pgfqpoint{0.976086in}{2.104324in}}{\pgfqpoint{0.968186in}{2.101051in}}{\pgfqpoint{0.962362in}{2.095227in}}%
\pgfpathcurveto{\pgfqpoint{0.956538in}{2.089404in}}{\pgfqpoint{0.953265in}{2.081504in}}{\pgfqpoint{0.953265in}{2.073267in}}%
\pgfpathcurveto{\pgfqpoint{0.953265in}{2.065031in}}{\pgfqpoint{0.956538in}{2.057131in}}{\pgfqpoint{0.962362in}{2.051307in}}%
\pgfpathcurveto{\pgfqpoint{0.968186in}{2.045483in}}{\pgfqpoint{0.976086in}{2.042211in}}{\pgfqpoint{0.984322in}{2.042211in}}%
\pgfpathclose%
\pgfusepath{stroke,fill}%
\end{pgfscope}%
\begin{pgfscope}%
\pgfpathrectangle{\pgfqpoint{0.100000in}{0.212622in}}{\pgfqpoint{3.696000in}{3.696000in}}%
\pgfusepath{clip}%
\pgfsetbuttcap%
\pgfsetroundjoin%
\definecolor{currentfill}{rgb}{0.121569,0.466667,0.705882}%
\pgfsetfillcolor{currentfill}%
\pgfsetfillopacity{0.911154}%
\pgfsetlinewidth{1.003750pt}%
\definecolor{currentstroke}{rgb}{0.121569,0.466667,0.705882}%
\pgfsetstrokecolor{currentstroke}%
\pgfsetstrokeopacity{0.911154}%
\pgfsetdash{}{0pt}%
\pgfpathmoveto{\pgfqpoint{1.731355in}{2.584221in}}%
\pgfpathcurveto{\pgfqpoint{1.739591in}{2.584221in}}{\pgfqpoint{1.747491in}{2.587493in}}{\pgfqpoint{1.753315in}{2.593317in}}%
\pgfpathcurveto{\pgfqpoint{1.759139in}{2.599141in}}{\pgfqpoint{1.762411in}{2.607041in}}{\pgfqpoint{1.762411in}{2.615277in}}%
\pgfpathcurveto{\pgfqpoint{1.762411in}{2.623514in}}{\pgfqpoint{1.759139in}{2.631414in}}{\pgfqpoint{1.753315in}{2.637238in}}%
\pgfpathcurveto{\pgfqpoint{1.747491in}{2.643062in}}{\pgfqpoint{1.739591in}{2.646334in}}{\pgfqpoint{1.731355in}{2.646334in}}%
\pgfpathcurveto{\pgfqpoint{1.723119in}{2.646334in}}{\pgfqpoint{1.715219in}{2.643062in}}{\pgfqpoint{1.709395in}{2.637238in}}%
\pgfpathcurveto{\pgfqpoint{1.703571in}{2.631414in}}{\pgfqpoint{1.700298in}{2.623514in}}{\pgfqpoint{1.700298in}{2.615277in}}%
\pgfpathcurveto{\pgfqpoint{1.700298in}{2.607041in}}{\pgfqpoint{1.703571in}{2.599141in}}{\pgfqpoint{1.709395in}{2.593317in}}%
\pgfpathcurveto{\pgfqpoint{1.715219in}{2.587493in}}{\pgfqpoint{1.723119in}{2.584221in}}{\pgfqpoint{1.731355in}{2.584221in}}%
\pgfpathclose%
\pgfusepath{stroke,fill}%
\end{pgfscope}%
\begin{pgfscope}%
\pgfpathrectangle{\pgfqpoint{0.100000in}{0.212622in}}{\pgfqpoint{3.696000in}{3.696000in}}%
\pgfusepath{clip}%
\pgfsetbuttcap%
\pgfsetroundjoin%
\definecolor{currentfill}{rgb}{0.121569,0.466667,0.705882}%
\pgfsetfillcolor{currentfill}%
\pgfsetfillopacity{0.911487}%
\pgfsetlinewidth{1.003750pt}%
\definecolor{currentstroke}{rgb}{0.121569,0.466667,0.705882}%
\pgfsetstrokecolor{currentstroke}%
\pgfsetstrokeopacity{0.911487}%
\pgfsetdash{}{0pt}%
\pgfpathmoveto{\pgfqpoint{2.658628in}{1.280332in}}%
\pgfpathcurveto{\pgfqpoint{2.666864in}{1.280332in}}{\pgfqpoint{2.674764in}{1.283605in}}{\pgfqpoint{2.680588in}{1.289429in}}%
\pgfpathcurveto{\pgfqpoint{2.686412in}{1.295253in}}{\pgfqpoint{2.689684in}{1.303153in}}{\pgfqpoint{2.689684in}{1.311389in}}%
\pgfpathcurveto{\pgfqpoint{2.689684in}{1.319625in}}{\pgfqpoint{2.686412in}{1.327525in}}{\pgfqpoint{2.680588in}{1.333349in}}%
\pgfpathcurveto{\pgfqpoint{2.674764in}{1.339173in}}{\pgfqpoint{2.666864in}{1.342445in}}{\pgfqpoint{2.658628in}{1.342445in}}%
\pgfpathcurveto{\pgfqpoint{2.650391in}{1.342445in}}{\pgfqpoint{2.642491in}{1.339173in}}{\pgfqpoint{2.636667in}{1.333349in}}%
\pgfpathcurveto{\pgfqpoint{2.630843in}{1.327525in}}{\pgfqpoint{2.627571in}{1.319625in}}{\pgfqpoint{2.627571in}{1.311389in}}%
\pgfpathcurveto{\pgfqpoint{2.627571in}{1.303153in}}{\pgfqpoint{2.630843in}{1.295253in}}{\pgfqpoint{2.636667in}{1.289429in}}%
\pgfpathcurveto{\pgfqpoint{2.642491in}{1.283605in}}{\pgfqpoint{2.650391in}{1.280332in}}{\pgfqpoint{2.658628in}{1.280332in}}%
\pgfpathclose%
\pgfusepath{stroke,fill}%
\end{pgfscope}%
\begin{pgfscope}%
\pgfpathrectangle{\pgfqpoint{0.100000in}{0.212622in}}{\pgfqpoint{3.696000in}{3.696000in}}%
\pgfusepath{clip}%
\pgfsetbuttcap%
\pgfsetroundjoin%
\definecolor{currentfill}{rgb}{0.121569,0.466667,0.705882}%
\pgfsetfillcolor{currentfill}%
\pgfsetfillopacity{0.912014}%
\pgfsetlinewidth{1.003750pt}%
\definecolor{currentstroke}{rgb}{0.121569,0.466667,0.705882}%
\pgfsetstrokecolor{currentstroke}%
\pgfsetstrokeopacity{0.912014}%
\pgfsetdash{}{0pt}%
\pgfpathmoveto{\pgfqpoint{1.730817in}{2.582113in}}%
\pgfpathcurveto{\pgfqpoint{1.739053in}{2.582113in}}{\pgfqpoint{1.746953in}{2.585385in}}{\pgfqpoint{1.752777in}{2.591209in}}%
\pgfpathcurveto{\pgfqpoint{1.758601in}{2.597033in}}{\pgfqpoint{1.761873in}{2.604933in}}{\pgfqpoint{1.761873in}{2.613169in}}%
\pgfpathcurveto{\pgfqpoint{1.761873in}{2.621405in}}{\pgfqpoint{1.758601in}{2.629305in}}{\pgfqpoint{1.752777in}{2.635129in}}%
\pgfpathcurveto{\pgfqpoint{1.746953in}{2.640953in}}{\pgfqpoint{1.739053in}{2.644226in}}{\pgfqpoint{1.730817in}{2.644226in}}%
\pgfpathcurveto{\pgfqpoint{1.722581in}{2.644226in}}{\pgfqpoint{1.714681in}{2.640953in}}{\pgfqpoint{1.708857in}{2.635129in}}%
\pgfpathcurveto{\pgfqpoint{1.703033in}{2.629305in}}{\pgfqpoint{1.699760in}{2.621405in}}{\pgfqpoint{1.699760in}{2.613169in}}%
\pgfpathcurveto{\pgfqpoint{1.699760in}{2.604933in}}{\pgfqpoint{1.703033in}{2.597033in}}{\pgfqpoint{1.708857in}{2.591209in}}%
\pgfpathcurveto{\pgfqpoint{1.714681in}{2.585385in}}{\pgfqpoint{1.722581in}{2.582113in}}{\pgfqpoint{1.730817in}{2.582113in}}%
\pgfpathclose%
\pgfusepath{stroke,fill}%
\end{pgfscope}%
\begin{pgfscope}%
\pgfpathrectangle{\pgfqpoint{0.100000in}{0.212622in}}{\pgfqpoint{3.696000in}{3.696000in}}%
\pgfusepath{clip}%
\pgfsetbuttcap%
\pgfsetroundjoin%
\definecolor{currentfill}{rgb}{0.121569,0.466667,0.705882}%
\pgfsetfillcolor{currentfill}%
\pgfsetfillopacity{0.912380}%
\pgfsetlinewidth{1.003750pt}%
\definecolor{currentstroke}{rgb}{0.121569,0.466667,0.705882}%
\pgfsetstrokecolor{currentstroke}%
\pgfsetstrokeopacity{0.912380}%
\pgfsetdash{}{0pt}%
\pgfpathmoveto{\pgfqpoint{2.161908in}{2.478849in}}%
\pgfpathcurveto{\pgfqpoint{2.170145in}{2.478849in}}{\pgfqpoint{2.178045in}{2.482122in}}{\pgfqpoint{2.183869in}{2.487946in}}%
\pgfpathcurveto{\pgfqpoint{2.189693in}{2.493770in}}{\pgfqpoint{2.192965in}{2.501670in}}{\pgfqpoint{2.192965in}{2.509906in}}%
\pgfpathcurveto{\pgfqpoint{2.192965in}{2.518142in}}{\pgfqpoint{2.189693in}{2.526042in}}{\pgfqpoint{2.183869in}{2.531866in}}%
\pgfpathcurveto{\pgfqpoint{2.178045in}{2.537690in}}{\pgfqpoint{2.170145in}{2.540962in}}{\pgfqpoint{2.161908in}{2.540962in}}%
\pgfpathcurveto{\pgfqpoint{2.153672in}{2.540962in}}{\pgfqpoint{2.145772in}{2.537690in}}{\pgfqpoint{2.139948in}{2.531866in}}%
\pgfpathcurveto{\pgfqpoint{2.134124in}{2.526042in}}{\pgfqpoint{2.130852in}{2.518142in}}{\pgfqpoint{2.130852in}{2.509906in}}%
\pgfpathcurveto{\pgfqpoint{2.130852in}{2.501670in}}{\pgfqpoint{2.134124in}{2.493770in}}{\pgfqpoint{2.139948in}{2.487946in}}%
\pgfpathcurveto{\pgfqpoint{2.145772in}{2.482122in}}{\pgfqpoint{2.153672in}{2.478849in}}{\pgfqpoint{2.161908in}{2.478849in}}%
\pgfpathclose%
\pgfusepath{stroke,fill}%
\end{pgfscope}%
\begin{pgfscope}%
\pgfpathrectangle{\pgfqpoint{0.100000in}{0.212622in}}{\pgfqpoint{3.696000in}{3.696000in}}%
\pgfusepath{clip}%
\pgfsetbuttcap%
\pgfsetroundjoin%
\definecolor{currentfill}{rgb}{0.121569,0.466667,0.705882}%
\pgfsetfillcolor{currentfill}%
\pgfsetfillopacity{0.912507}%
\pgfsetlinewidth{1.003750pt}%
\definecolor{currentstroke}{rgb}{0.121569,0.466667,0.705882}%
\pgfsetstrokecolor{currentstroke}%
\pgfsetstrokeopacity{0.912507}%
\pgfsetdash{}{0pt}%
\pgfpathmoveto{\pgfqpoint{0.993333in}{2.037342in}}%
\pgfpathcurveto{\pgfqpoint{1.001569in}{2.037342in}}{\pgfqpoint{1.009469in}{2.040614in}}{\pgfqpoint{1.015293in}{2.046438in}}%
\pgfpathcurveto{\pgfqpoint{1.021117in}{2.052262in}}{\pgfqpoint{1.024389in}{2.060162in}}{\pgfqpoint{1.024389in}{2.068398in}}%
\pgfpathcurveto{\pgfqpoint{1.024389in}{2.076635in}}{\pgfqpoint{1.021117in}{2.084535in}}{\pgfqpoint{1.015293in}{2.090359in}}%
\pgfpathcurveto{\pgfqpoint{1.009469in}{2.096183in}}{\pgfqpoint{1.001569in}{2.099455in}}{\pgfqpoint{0.993333in}{2.099455in}}%
\pgfpathcurveto{\pgfqpoint{0.985096in}{2.099455in}}{\pgfqpoint{0.977196in}{2.096183in}}{\pgfqpoint{0.971372in}{2.090359in}}%
\pgfpathcurveto{\pgfqpoint{0.965548in}{2.084535in}}{\pgfqpoint{0.962276in}{2.076635in}}{\pgfqpoint{0.962276in}{2.068398in}}%
\pgfpathcurveto{\pgfqpoint{0.962276in}{2.060162in}}{\pgfqpoint{0.965548in}{2.052262in}}{\pgfqpoint{0.971372in}{2.046438in}}%
\pgfpathcurveto{\pgfqpoint{0.977196in}{2.040614in}}{\pgfqpoint{0.985096in}{2.037342in}}{\pgfqpoint{0.993333in}{2.037342in}}%
\pgfpathclose%
\pgfusepath{stroke,fill}%
\end{pgfscope}%
\begin{pgfscope}%
\pgfpathrectangle{\pgfqpoint{0.100000in}{0.212622in}}{\pgfqpoint{3.696000in}{3.696000in}}%
\pgfusepath{clip}%
\pgfsetbuttcap%
\pgfsetroundjoin%
\definecolor{currentfill}{rgb}{0.121569,0.466667,0.705882}%
\pgfsetfillcolor{currentfill}%
\pgfsetfillopacity{0.912976}%
\pgfsetlinewidth{1.003750pt}%
\definecolor{currentstroke}{rgb}{0.121569,0.466667,0.705882}%
\pgfsetstrokecolor{currentstroke}%
\pgfsetstrokeopacity{0.912976}%
\pgfsetdash{}{0pt}%
\pgfpathmoveto{\pgfqpoint{1.730388in}{2.579856in}}%
\pgfpathcurveto{\pgfqpoint{1.738624in}{2.579856in}}{\pgfqpoint{1.746524in}{2.583128in}}{\pgfqpoint{1.752348in}{2.588952in}}%
\pgfpathcurveto{\pgfqpoint{1.758172in}{2.594776in}}{\pgfqpoint{1.761445in}{2.602676in}}{\pgfqpoint{1.761445in}{2.610913in}}%
\pgfpathcurveto{\pgfqpoint{1.761445in}{2.619149in}}{\pgfqpoint{1.758172in}{2.627049in}}{\pgfqpoint{1.752348in}{2.632873in}}%
\pgfpathcurveto{\pgfqpoint{1.746524in}{2.638697in}}{\pgfqpoint{1.738624in}{2.641969in}}{\pgfqpoint{1.730388in}{2.641969in}}%
\pgfpathcurveto{\pgfqpoint{1.722152in}{2.641969in}}{\pgfqpoint{1.714252in}{2.638697in}}{\pgfqpoint{1.708428in}{2.632873in}}%
\pgfpathcurveto{\pgfqpoint{1.702604in}{2.627049in}}{\pgfqpoint{1.699332in}{2.619149in}}{\pgfqpoint{1.699332in}{2.610913in}}%
\pgfpathcurveto{\pgfqpoint{1.699332in}{2.602676in}}{\pgfqpoint{1.702604in}{2.594776in}}{\pgfqpoint{1.708428in}{2.588952in}}%
\pgfpathcurveto{\pgfqpoint{1.714252in}{2.583128in}}{\pgfqpoint{1.722152in}{2.579856in}}{\pgfqpoint{1.730388in}{2.579856in}}%
\pgfpathclose%
\pgfusepath{stroke,fill}%
\end{pgfscope}%
\begin{pgfscope}%
\pgfpathrectangle{\pgfqpoint{0.100000in}{0.212622in}}{\pgfqpoint{3.696000in}{3.696000in}}%
\pgfusepath{clip}%
\pgfsetbuttcap%
\pgfsetroundjoin%
\definecolor{currentfill}{rgb}{0.121569,0.466667,0.705882}%
\pgfsetfillcolor{currentfill}%
\pgfsetfillopacity{0.913137}%
\pgfsetlinewidth{1.003750pt}%
\definecolor{currentstroke}{rgb}{0.121569,0.466667,0.705882}%
\pgfsetstrokecolor{currentstroke}%
\pgfsetstrokeopacity{0.913137}%
\pgfsetdash{}{0pt}%
\pgfpathmoveto{\pgfqpoint{2.160586in}{2.477438in}}%
\pgfpathcurveto{\pgfqpoint{2.168822in}{2.477438in}}{\pgfqpoint{2.176722in}{2.480711in}}{\pgfqpoint{2.182546in}{2.486535in}}%
\pgfpathcurveto{\pgfqpoint{2.188370in}{2.492359in}}{\pgfqpoint{2.191642in}{2.500259in}}{\pgfqpoint{2.191642in}{2.508495in}}%
\pgfpathcurveto{\pgfqpoint{2.191642in}{2.516731in}}{\pgfqpoint{2.188370in}{2.524631in}}{\pgfqpoint{2.182546in}{2.530455in}}%
\pgfpathcurveto{\pgfqpoint{2.176722in}{2.536279in}}{\pgfqpoint{2.168822in}{2.539551in}}{\pgfqpoint{2.160586in}{2.539551in}}%
\pgfpathcurveto{\pgfqpoint{2.152350in}{2.539551in}}{\pgfqpoint{2.144450in}{2.536279in}}{\pgfqpoint{2.138626in}{2.530455in}}%
\pgfpathcurveto{\pgfqpoint{2.132802in}{2.524631in}}{\pgfqpoint{2.129529in}{2.516731in}}{\pgfqpoint{2.129529in}{2.508495in}}%
\pgfpathcurveto{\pgfqpoint{2.129529in}{2.500259in}}{\pgfqpoint{2.132802in}{2.492359in}}{\pgfqpoint{2.138626in}{2.486535in}}%
\pgfpathcurveto{\pgfqpoint{2.144450in}{2.480711in}}{\pgfqpoint{2.152350in}{2.477438in}}{\pgfqpoint{2.160586in}{2.477438in}}%
\pgfpathclose%
\pgfusepath{stroke,fill}%
\end{pgfscope}%
\begin{pgfscope}%
\pgfpathrectangle{\pgfqpoint{0.100000in}{0.212622in}}{\pgfqpoint{3.696000in}{3.696000in}}%
\pgfusepath{clip}%
\pgfsetbuttcap%
\pgfsetroundjoin%
\definecolor{currentfill}{rgb}{0.121569,0.466667,0.705882}%
\pgfsetfillcolor{currentfill}%
\pgfsetfillopacity{0.913682}%
\pgfsetlinewidth{1.003750pt}%
\definecolor{currentstroke}{rgb}{0.121569,0.466667,0.705882}%
\pgfsetstrokecolor{currentstroke}%
\pgfsetstrokeopacity{0.913682}%
\pgfsetdash{}{0pt}%
\pgfpathmoveto{\pgfqpoint{1.003261in}{2.031328in}}%
\pgfpathcurveto{\pgfqpoint{1.011497in}{2.031328in}}{\pgfqpoint{1.019397in}{2.034601in}}{\pgfqpoint{1.025221in}{2.040425in}}%
\pgfpathcurveto{\pgfqpoint{1.031045in}{2.046249in}}{\pgfqpoint{1.034317in}{2.054149in}}{\pgfqpoint{1.034317in}{2.062385in}}%
\pgfpathcurveto{\pgfqpoint{1.034317in}{2.070621in}}{\pgfqpoint{1.031045in}{2.078521in}}{\pgfqpoint{1.025221in}{2.084345in}}%
\pgfpathcurveto{\pgfqpoint{1.019397in}{2.090169in}}{\pgfqpoint{1.011497in}{2.093441in}}{\pgfqpoint{1.003261in}{2.093441in}}%
\pgfpathcurveto{\pgfqpoint{0.995024in}{2.093441in}}{\pgfqpoint{0.987124in}{2.090169in}}{\pgfqpoint{0.981300in}{2.084345in}}%
\pgfpathcurveto{\pgfqpoint{0.975477in}{2.078521in}}{\pgfqpoint{0.972204in}{2.070621in}}{\pgfqpoint{0.972204in}{2.062385in}}%
\pgfpathcurveto{\pgfqpoint{0.972204in}{2.054149in}}{\pgfqpoint{0.975477in}{2.046249in}}{\pgfqpoint{0.981300in}{2.040425in}}%
\pgfpathcurveto{\pgfqpoint{0.987124in}{2.034601in}}{\pgfqpoint{0.995024in}{2.031328in}}{\pgfqpoint{1.003261in}{2.031328in}}%
\pgfpathclose%
\pgfusepath{stroke,fill}%
\end{pgfscope}%
\begin{pgfscope}%
\pgfpathrectangle{\pgfqpoint{0.100000in}{0.212622in}}{\pgfqpoint{3.696000in}{3.696000in}}%
\pgfusepath{clip}%
\pgfsetbuttcap%
\pgfsetroundjoin%
\definecolor{currentfill}{rgb}{0.121569,0.466667,0.705882}%
\pgfsetfillcolor{currentfill}%
\pgfsetfillopacity{0.914096}%
\pgfsetlinewidth{1.003750pt}%
\definecolor{currentstroke}{rgb}{0.121569,0.466667,0.705882}%
\pgfsetstrokecolor{currentstroke}%
\pgfsetstrokeopacity{0.914096}%
\pgfsetdash{}{0pt}%
\pgfpathmoveto{\pgfqpoint{1.730080in}{2.577306in}}%
\pgfpathcurveto{\pgfqpoint{1.738317in}{2.577306in}}{\pgfqpoint{1.746217in}{2.580578in}}{\pgfqpoint{1.752041in}{2.586402in}}%
\pgfpathcurveto{\pgfqpoint{1.757865in}{2.592226in}}{\pgfqpoint{1.761137in}{2.600126in}}{\pgfqpoint{1.761137in}{2.608362in}}%
\pgfpathcurveto{\pgfqpoint{1.761137in}{2.616599in}}{\pgfqpoint{1.757865in}{2.624499in}}{\pgfqpoint{1.752041in}{2.630323in}}%
\pgfpathcurveto{\pgfqpoint{1.746217in}{2.636147in}}{\pgfqpoint{1.738317in}{2.639419in}}{\pgfqpoint{1.730080in}{2.639419in}}%
\pgfpathcurveto{\pgfqpoint{1.721844in}{2.639419in}}{\pgfqpoint{1.713944in}{2.636147in}}{\pgfqpoint{1.708120in}{2.630323in}}%
\pgfpathcurveto{\pgfqpoint{1.702296in}{2.624499in}}{\pgfqpoint{1.699024in}{2.616599in}}{\pgfqpoint{1.699024in}{2.608362in}}%
\pgfpathcurveto{\pgfqpoint{1.699024in}{2.600126in}}{\pgfqpoint{1.702296in}{2.592226in}}{\pgfqpoint{1.708120in}{2.586402in}}%
\pgfpathcurveto{\pgfqpoint{1.713944in}{2.580578in}}{\pgfqpoint{1.721844in}{2.577306in}}{\pgfqpoint{1.730080in}{2.577306in}}%
\pgfpathclose%
\pgfusepath{stroke,fill}%
\end{pgfscope}%
\begin{pgfscope}%
\pgfpathrectangle{\pgfqpoint{0.100000in}{0.212622in}}{\pgfqpoint{3.696000in}{3.696000in}}%
\pgfusepath{clip}%
\pgfsetbuttcap%
\pgfsetroundjoin%
\definecolor{currentfill}{rgb}{0.121569,0.466667,0.705882}%
\pgfsetfillcolor{currentfill}%
\pgfsetfillopacity{0.914513}%
\pgfsetlinewidth{1.003750pt}%
\definecolor{currentstroke}{rgb}{0.121569,0.466667,0.705882}%
\pgfsetstrokecolor{currentstroke}%
\pgfsetstrokeopacity{0.914513}%
\pgfsetdash{}{0pt}%
\pgfpathmoveto{\pgfqpoint{2.158148in}{2.474888in}}%
\pgfpathcurveto{\pgfqpoint{2.166384in}{2.474888in}}{\pgfqpoint{2.174284in}{2.478160in}}{\pgfqpoint{2.180108in}{2.483984in}}%
\pgfpathcurveto{\pgfqpoint{2.185932in}{2.489808in}}{\pgfqpoint{2.189204in}{2.497708in}}{\pgfqpoint{2.189204in}{2.505944in}}%
\pgfpathcurveto{\pgfqpoint{2.189204in}{2.514181in}}{\pgfqpoint{2.185932in}{2.522081in}}{\pgfqpoint{2.180108in}{2.527904in}}%
\pgfpathcurveto{\pgfqpoint{2.174284in}{2.533728in}}{\pgfqpoint{2.166384in}{2.537001in}}{\pgfqpoint{2.158148in}{2.537001in}}%
\pgfpathcurveto{\pgfqpoint{2.149911in}{2.537001in}}{\pgfqpoint{2.142011in}{2.533728in}}{\pgfqpoint{2.136187in}{2.527904in}}%
\pgfpathcurveto{\pgfqpoint{2.130364in}{2.522081in}}{\pgfqpoint{2.127091in}{2.514181in}}{\pgfqpoint{2.127091in}{2.505944in}}%
\pgfpathcurveto{\pgfqpoint{2.127091in}{2.497708in}}{\pgfqpoint{2.130364in}{2.489808in}}{\pgfqpoint{2.136187in}{2.483984in}}%
\pgfpathcurveto{\pgfqpoint{2.142011in}{2.478160in}}{\pgfqpoint{2.149911in}{2.474888in}}{\pgfqpoint{2.158148in}{2.474888in}}%
\pgfpathclose%
\pgfusepath{stroke,fill}%
\end{pgfscope}%
\begin{pgfscope}%
\pgfpathrectangle{\pgfqpoint{0.100000in}{0.212622in}}{\pgfqpoint{3.696000in}{3.696000in}}%
\pgfusepath{clip}%
\pgfsetbuttcap%
\pgfsetroundjoin%
\definecolor{currentfill}{rgb}{0.121569,0.466667,0.705882}%
\pgfsetfillcolor{currentfill}%
\pgfsetfillopacity{0.914815}%
\pgfsetlinewidth{1.003750pt}%
\definecolor{currentstroke}{rgb}{0.121569,0.466667,0.705882}%
\pgfsetstrokecolor{currentstroke}%
\pgfsetstrokeopacity{0.914815}%
\pgfsetdash{}{0pt}%
\pgfpathmoveto{\pgfqpoint{1.013693in}{2.025509in}}%
\pgfpathcurveto{\pgfqpoint{1.021929in}{2.025509in}}{\pgfqpoint{1.029829in}{2.028782in}}{\pgfqpoint{1.035653in}{2.034606in}}%
\pgfpathcurveto{\pgfqpoint{1.041477in}{2.040429in}}{\pgfqpoint{1.044749in}{2.048330in}}{\pgfqpoint{1.044749in}{2.056566in}}%
\pgfpathcurveto{\pgfqpoint{1.044749in}{2.064802in}}{\pgfqpoint{1.041477in}{2.072702in}}{\pgfqpoint{1.035653in}{2.078526in}}%
\pgfpathcurveto{\pgfqpoint{1.029829in}{2.084350in}}{\pgfqpoint{1.021929in}{2.087622in}}{\pgfqpoint{1.013693in}{2.087622in}}%
\pgfpathcurveto{\pgfqpoint{1.005457in}{2.087622in}}{\pgfqpoint{0.997556in}{2.084350in}}{\pgfqpoint{0.991733in}{2.078526in}}%
\pgfpathcurveto{\pgfqpoint{0.985909in}{2.072702in}}{\pgfqpoint{0.982636in}{2.064802in}}{\pgfqpoint{0.982636in}{2.056566in}}%
\pgfpathcurveto{\pgfqpoint{0.982636in}{2.048330in}}{\pgfqpoint{0.985909in}{2.040429in}}{\pgfqpoint{0.991733in}{2.034606in}}%
\pgfpathcurveto{\pgfqpoint{0.997556in}{2.028782in}}{\pgfqpoint{1.005457in}{2.025509in}}{\pgfqpoint{1.013693in}{2.025509in}}%
\pgfpathclose%
\pgfusepath{stroke,fill}%
\end{pgfscope}%
\begin{pgfscope}%
\pgfpathrectangle{\pgfqpoint{0.100000in}{0.212622in}}{\pgfqpoint{3.696000in}{3.696000in}}%
\pgfusepath{clip}%
\pgfsetbuttcap%
\pgfsetroundjoin%
\definecolor{currentfill}{rgb}{0.121569,0.466667,0.705882}%
\pgfsetfillcolor{currentfill}%
\pgfsetfillopacity{0.915307}%
\pgfsetlinewidth{1.003750pt}%
\definecolor{currentstroke}{rgb}{0.121569,0.466667,0.705882}%
\pgfsetstrokecolor{currentstroke}%
\pgfsetstrokeopacity{0.915307}%
\pgfsetdash{}{0pt}%
\pgfpathmoveto{\pgfqpoint{1.729984in}{2.574564in}}%
\pgfpathcurveto{\pgfqpoint{1.738221in}{2.574564in}}{\pgfqpoint{1.746121in}{2.577837in}}{\pgfqpoint{1.751945in}{2.583661in}}%
\pgfpathcurveto{\pgfqpoint{1.757768in}{2.589485in}}{\pgfqpoint{1.761041in}{2.597385in}}{\pgfqpoint{1.761041in}{2.605621in}}%
\pgfpathcurveto{\pgfqpoint{1.761041in}{2.613857in}}{\pgfqpoint{1.757768in}{2.621757in}}{\pgfqpoint{1.751945in}{2.627581in}}%
\pgfpathcurveto{\pgfqpoint{1.746121in}{2.633405in}}{\pgfqpoint{1.738221in}{2.636677in}}{\pgfqpoint{1.729984in}{2.636677in}}%
\pgfpathcurveto{\pgfqpoint{1.721748in}{2.636677in}}{\pgfqpoint{1.713848in}{2.633405in}}{\pgfqpoint{1.708024in}{2.627581in}}%
\pgfpathcurveto{\pgfqpoint{1.702200in}{2.621757in}}{\pgfqpoint{1.698928in}{2.613857in}}{\pgfqpoint{1.698928in}{2.605621in}}%
\pgfpathcurveto{\pgfqpoint{1.698928in}{2.597385in}}{\pgfqpoint{1.702200in}{2.589485in}}{\pgfqpoint{1.708024in}{2.583661in}}%
\pgfpathcurveto{\pgfqpoint{1.713848in}{2.577837in}}{\pgfqpoint{1.721748in}{2.574564in}}{\pgfqpoint{1.729984in}{2.574564in}}%
\pgfpathclose%
\pgfusepath{stroke,fill}%
\end{pgfscope}%
\begin{pgfscope}%
\pgfpathrectangle{\pgfqpoint{0.100000in}{0.212622in}}{\pgfqpoint{3.696000in}{3.696000in}}%
\pgfusepath{clip}%
\pgfsetbuttcap%
\pgfsetroundjoin%
\definecolor{currentfill}{rgb}{0.121569,0.466667,0.705882}%
\pgfsetfillcolor{currentfill}%
\pgfsetfillopacity{0.915426}%
\pgfsetlinewidth{1.003750pt}%
\definecolor{currentstroke}{rgb}{0.121569,0.466667,0.705882}%
\pgfsetstrokecolor{currentstroke}%
\pgfsetstrokeopacity{0.915426}%
\pgfsetdash{}{0pt}%
\pgfpathmoveto{\pgfqpoint{1.019574in}{2.022726in}}%
\pgfpathcurveto{\pgfqpoint{1.027810in}{2.022726in}}{\pgfqpoint{1.035710in}{2.025998in}}{\pgfqpoint{1.041534in}{2.031822in}}%
\pgfpathcurveto{\pgfqpoint{1.047358in}{2.037646in}}{\pgfqpoint{1.050630in}{2.045546in}}{\pgfqpoint{1.050630in}{2.053782in}}%
\pgfpathcurveto{\pgfqpoint{1.050630in}{2.062019in}}{\pgfqpoint{1.047358in}{2.069919in}}{\pgfqpoint{1.041534in}{2.075743in}}%
\pgfpathcurveto{\pgfqpoint{1.035710in}{2.081567in}}{\pgfqpoint{1.027810in}{2.084839in}}{\pgfqpoint{1.019574in}{2.084839in}}%
\pgfpathcurveto{\pgfqpoint{1.011338in}{2.084839in}}{\pgfqpoint{1.003438in}{2.081567in}}{\pgfqpoint{0.997614in}{2.075743in}}%
\pgfpathcurveto{\pgfqpoint{0.991790in}{2.069919in}}{\pgfqpoint{0.988517in}{2.062019in}}{\pgfqpoint{0.988517in}{2.053782in}}%
\pgfpathcurveto{\pgfqpoint{0.988517in}{2.045546in}}{\pgfqpoint{0.991790in}{2.037646in}}{\pgfqpoint{0.997614in}{2.031822in}}%
\pgfpathcurveto{\pgfqpoint{1.003438in}{2.025998in}}{\pgfqpoint{1.011338in}{2.022726in}}{\pgfqpoint{1.019574in}{2.022726in}}%
\pgfpathclose%
\pgfusepath{stroke,fill}%
\end{pgfscope}%
\begin{pgfscope}%
\pgfpathrectangle{\pgfqpoint{0.100000in}{0.212622in}}{\pgfqpoint{3.696000in}{3.696000in}}%
\pgfusepath{clip}%
\pgfsetbuttcap%
\pgfsetroundjoin%
\definecolor{currentfill}{rgb}{0.121569,0.466667,0.705882}%
\pgfsetfillcolor{currentfill}%
\pgfsetfillopacity{0.915453}%
\pgfsetlinewidth{1.003750pt}%
\definecolor{currentstroke}{rgb}{0.121569,0.466667,0.705882}%
\pgfsetstrokecolor{currentstroke}%
\pgfsetstrokeopacity{0.915453}%
\pgfsetdash{}{0pt}%
\pgfpathmoveto{\pgfqpoint{2.156567in}{2.473341in}}%
\pgfpathcurveto{\pgfqpoint{2.164803in}{2.473341in}}{\pgfqpoint{2.172704in}{2.476613in}}{\pgfqpoint{2.178527in}{2.482437in}}%
\pgfpathcurveto{\pgfqpoint{2.184351in}{2.488261in}}{\pgfqpoint{2.187624in}{2.496161in}}{\pgfqpoint{2.187624in}{2.504397in}}%
\pgfpathcurveto{\pgfqpoint{2.187624in}{2.512633in}}{\pgfqpoint{2.184351in}{2.520533in}}{\pgfqpoint{2.178527in}{2.526357in}}%
\pgfpathcurveto{\pgfqpoint{2.172704in}{2.532181in}}{\pgfqpoint{2.164803in}{2.535454in}}{\pgfqpoint{2.156567in}{2.535454in}}%
\pgfpathcurveto{\pgfqpoint{2.148331in}{2.535454in}}{\pgfqpoint{2.140431in}{2.532181in}}{\pgfqpoint{2.134607in}{2.526357in}}%
\pgfpathcurveto{\pgfqpoint{2.128783in}{2.520533in}}{\pgfqpoint{2.125511in}{2.512633in}}{\pgfqpoint{2.125511in}{2.504397in}}%
\pgfpathcurveto{\pgfqpoint{2.125511in}{2.496161in}}{\pgfqpoint{2.128783in}{2.488261in}}{\pgfqpoint{2.134607in}{2.482437in}}%
\pgfpathcurveto{\pgfqpoint{2.140431in}{2.476613in}}{\pgfqpoint{2.148331in}{2.473341in}}{\pgfqpoint{2.156567in}{2.473341in}}%
\pgfpathclose%
\pgfusepath{stroke,fill}%
\end{pgfscope}%
\begin{pgfscope}%
\pgfpathrectangle{\pgfqpoint{0.100000in}{0.212622in}}{\pgfqpoint{3.696000in}{3.696000in}}%
\pgfusepath{clip}%
\pgfsetbuttcap%
\pgfsetroundjoin%
\definecolor{currentfill}{rgb}{0.121569,0.466667,0.705882}%
\pgfsetfillcolor{currentfill}%
\pgfsetfillopacity{0.915666}%
\pgfsetlinewidth{1.003750pt}%
\definecolor{currentstroke}{rgb}{0.121569,0.466667,0.705882}%
\pgfsetstrokecolor{currentstroke}%
\pgfsetstrokeopacity{0.915666}%
\pgfsetdash{}{0pt}%
\pgfpathmoveto{\pgfqpoint{2.651111in}{1.272933in}}%
\pgfpathcurveto{\pgfqpoint{2.659347in}{1.272933in}}{\pgfqpoint{2.667247in}{1.276205in}}{\pgfqpoint{2.673071in}{1.282029in}}%
\pgfpathcurveto{\pgfqpoint{2.678895in}{1.287853in}}{\pgfqpoint{2.682168in}{1.295753in}}{\pgfqpoint{2.682168in}{1.303989in}}%
\pgfpathcurveto{\pgfqpoint{2.682168in}{1.312226in}}{\pgfqpoint{2.678895in}{1.320126in}}{\pgfqpoint{2.673071in}{1.325950in}}%
\pgfpathcurveto{\pgfqpoint{2.667247in}{1.331774in}}{\pgfqpoint{2.659347in}{1.335046in}}{\pgfqpoint{2.651111in}{1.335046in}}%
\pgfpathcurveto{\pgfqpoint{2.642875in}{1.335046in}}{\pgfqpoint{2.634975in}{1.331774in}}{\pgfqpoint{2.629151in}{1.325950in}}%
\pgfpathcurveto{\pgfqpoint{2.623327in}{1.320126in}}{\pgfqpoint{2.620055in}{1.312226in}}{\pgfqpoint{2.620055in}{1.303989in}}%
\pgfpathcurveto{\pgfqpoint{2.620055in}{1.295753in}}{\pgfqpoint{2.623327in}{1.287853in}}{\pgfqpoint{2.629151in}{1.282029in}}%
\pgfpathcurveto{\pgfqpoint{2.634975in}{1.276205in}}{\pgfqpoint{2.642875in}{1.272933in}}{\pgfqpoint{2.651111in}{1.272933in}}%
\pgfpathclose%
\pgfusepath{stroke,fill}%
\end{pgfscope}%
\begin{pgfscope}%
\pgfpathrectangle{\pgfqpoint{0.100000in}{0.212622in}}{\pgfqpoint{3.696000in}{3.696000in}}%
\pgfusepath{clip}%
\pgfsetbuttcap%
\pgfsetroundjoin%
\definecolor{currentfill}{rgb}{0.121569,0.466667,0.705882}%
\pgfsetfillcolor{currentfill}%
\pgfsetfillopacity{0.916087}%
\pgfsetlinewidth{1.003750pt}%
\definecolor{currentstroke}{rgb}{0.121569,0.466667,0.705882}%
\pgfsetstrokecolor{currentstroke}%
\pgfsetstrokeopacity{0.916087}%
\pgfsetdash{}{0pt}%
\pgfpathmoveto{\pgfqpoint{1.025832in}{2.019688in}}%
\pgfpathcurveto{\pgfqpoint{1.034068in}{2.019688in}}{\pgfqpoint{1.041968in}{2.022960in}}{\pgfqpoint{1.047792in}{2.028784in}}%
\pgfpathcurveto{\pgfqpoint{1.053616in}{2.034608in}}{\pgfqpoint{1.056888in}{2.042508in}}{\pgfqpoint{1.056888in}{2.050745in}}%
\pgfpathcurveto{\pgfqpoint{1.056888in}{2.058981in}}{\pgfqpoint{1.053616in}{2.066881in}}{\pgfqpoint{1.047792in}{2.072705in}}%
\pgfpathcurveto{\pgfqpoint{1.041968in}{2.078529in}}{\pgfqpoint{1.034068in}{2.081801in}}{\pgfqpoint{1.025832in}{2.081801in}}%
\pgfpathcurveto{\pgfqpoint{1.017595in}{2.081801in}}{\pgfqpoint{1.009695in}{2.078529in}}{\pgfqpoint{1.003871in}{2.072705in}}%
\pgfpathcurveto{\pgfqpoint{0.998048in}{2.066881in}}{\pgfqpoint{0.994775in}{2.058981in}}{\pgfqpoint{0.994775in}{2.050745in}}%
\pgfpathcurveto{\pgfqpoint{0.994775in}{2.042508in}}{\pgfqpoint{0.998048in}{2.034608in}}{\pgfqpoint{1.003871in}{2.028784in}}%
\pgfpathcurveto{\pgfqpoint{1.009695in}{2.022960in}}{\pgfqpoint{1.017595in}{2.019688in}}{\pgfqpoint{1.025832in}{2.019688in}}%
\pgfpathclose%
\pgfusepath{stroke,fill}%
\end{pgfscope}%
\begin{pgfscope}%
\pgfpathrectangle{\pgfqpoint{0.100000in}{0.212622in}}{\pgfqpoint{3.696000in}{3.696000in}}%
\pgfusepath{clip}%
\pgfsetbuttcap%
\pgfsetroundjoin%
\definecolor{currentfill}{rgb}{0.121569,0.466667,0.705882}%
\pgfsetfillcolor{currentfill}%
\pgfsetfillopacity{0.916625}%
\pgfsetlinewidth{1.003750pt}%
\definecolor{currentstroke}{rgb}{0.121569,0.466667,0.705882}%
\pgfsetstrokecolor{currentstroke}%
\pgfsetstrokeopacity{0.916625}%
\pgfsetdash{}{0pt}%
\pgfpathmoveto{\pgfqpoint{1.730130in}{2.571581in}}%
\pgfpathcurveto{\pgfqpoint{1.738366in}{2.571581in}}{\pgfqpoint{1.746266in}{2.574853in}}{\pgfqpoint{1.752090in}{2.580677in}}%
\pgfpathcurveto{\pgfqpoint{1.757914in}{2.586501in}}{\pgfqpoint{1.761186in}{2.594401in}}{\pgfqpoint{1.761186in}{2.602637in}}%
\pgfpathcurveto{\pgfqpoint{1.761186in}{2.610873in}}{\pgfqpoint{1.757914in}{2.618774in}}{\pgfqpoint{1.752090in}{2.624597in}}%
\pgfpathcurveto{\pgfqpoint{1.746266in}{2.630421in}}{\pgfqpoint{1.738366in}{2.633694in}}{\pgfqpoint{1.730130in}{2.633694in}}%
\pgfpathcurveto{\pgfqpoint{1.721894in}{2.633694in}}{\pgfqpoint{1.713994in}{2.630421in}}{\pgfqpoint{1.708170in}{2.624597in}}%
\pgfpathcurveto{\pgfqpoint{1.702346in}{2.618774in}}{\pgfqpoint{1.699073in}{2.610873in}}{\pgfqpoint{1.699073in}{2.602637in}}%
\pgfpathcurveto{\pgfqpoint{1.699073in}{2.594401in}}{\pgfqpoint{1.702346in}{2.586501in}}{\pgfqpoint{1.708170in}{2.580677in}}%
\pgfpathcurveto{\pgfqpoint{1.713994in}{2.574853in}}{\pgfqpoint{1.721894in}{2.571581in}}{\pgfqpoint{1.730130in}{2.571581in}}%
\pgfpathclose%
\pgfusepath{stroke,fill}%
\end{pgfscope}%
\begin{pgfscope}%
\pgfpathrectangle{\pgfqpoint{0.100000in}{0.212622in}}{\pgfqpoint{3.696000in}{3.696000in}}%
\pgfusepath{clip}%
\pgfsetbuttcap%
\pgfsetroundjoin%
\definecolor{currentfill}{rgb}{0.121569,0.466667,0.705882}%
\pgfsetfillcolor{currentfill}%
\pgfsetfillopacity{0.916834}%
\pgfsetlinewidth{1.003750pt}%
\definecolor{currentstroke}{rgb}{0.121569,0.466667,0.705882}%
\pgfsetstrokecolor{currentstroke}%
\pgfsetstrokeopacity{0.916834}%
\pgfsetdash{}{0pt}%
\pgfpathmoveto{\pgfqpoint{1.032568in}{2.016511in}}%
\pgfpathcurveto{\pgfqpoint{1.040804in}{2.016511in}}{\pgfqpoint{1.048704in}{2.019783in}}{\pgfqpoint{1.054528in}{2.025607in}}%
\pgfpathcurveto{\pgfqpoint{1.060352in}{2.031431in}}{\pgfqpoint{1.063624in}{2.039331in}}{\pgfqpoint{1.063624in}{2.047567in}}%
\pgfpathcurveto{\pgfqpoint{1.063624in}{2.055804in}}{\pgfqpoint{1.060352in}{2.063704in}}{\pgfqpoint{1.054528in}{2.069528in}}%
\pgfpathcurveto{\pgfqpoint{1.048704in}{2.075352in}}{\pgfqpoint{1.040804in}{2.078624in}}{\pgfqpoint{1.032568in}{2.078624in}}%
\pgfpathcurveto{\pgfqpoint{1.024331in}{2.078624in}}{\pgfqpoint{1.016431in}{2.075352in}}{\pgfqpoint{1.010607in}{2.069528in}}%
\pgfpathcurveto{\pgfqpoint{1.004784in}{2.063704in}}{\pgfqpoint{1.001511in}{2.055804in}}{\pgfqpoint{1.001511in}{2.047567in}}%
\pgfpathcurveto{\pgfqpoint{1.001511in}{2.039331in}}{\pgfqpoint{1.004784in}{2.031431in}}{\pgfqpoint{1.010607in}{2.025607in}}%
\pgfpathcurveto{\pgfqpoint{1.016431in}{2.019783in}}{\pgfqpoint{1.024331in}{2.016511in}}{\pgfqpoint{1.032568in}{2.016511in}}%
\pgfpathclose%
\pgfusepath{stroke,fill}%
\end{pgfscope}%
\begin{pgfscope}%
\pgfpathrectangle{\pgfqpoint{0.100000in}{0.212622in}}{\pgfqpoint{3.696000in}{3.696000in}}%
\pgfusepath{clip}%
\pgfsetbuttcap%
\pgfsetroundjoin%
\definecolor{currentfill}{rgb}{0.121569,0.466667,0.705882}%
\pgfsetfillcolor{currentfill}%
\pgfsetfillopacity{0.917132}%
\pgfsetlinewidth{1.003750pt}%
\definecolor{currentstroke}{rgb}{0.121569,0.466667,0.705882}%
\pgfsetstrokecolor{currentstroke}%
\pgfsetstrokeopacity{0.917132}%
\pgfsetdash{}{0pt}%
\pgfpathmoveto{\pgfqpoint{2.153797in}{2.470278in}}%
\pgfpathcurveto{\pgfqpoint{2.162033in}{2.470278in}}{\pgfqpoint{2.169933in}{2.473550in}}{\pgfqpoint{2.175757in}{2.479374in}}%
\pgfpathcurveto{\pgfqpoint{2.181581in}{2.485198in}}{\pgfqpoint{2.184853in}{2.493098in}}{\pgfqpoint{2.184853in}{2.501334in}}%
\pgfpathcurveto{\pgfqpoint{2.184853in}{2.509571in}}{\pgfqpoint{2.181581in}{2.517471in}}{\pgfqpoint{2.175757in}{2.523295in}}%
\pgfpathcurveto{\pgfqpoint{2.169933in}{2.529119in}}{\pgfqpoint{2.162033in}{2.532391in}}{\pgfqpoint{2.153797in}{2.532391in}}%
\pgfpathcurveto{\pgfqpoint{2.145561in}{2.532391in}}{\pgfqpoint{2.137661in}{2.529119in}}{\pgfqpoint{2.131837in}{2.523295in}}%
\pgfpathcurveto{\pgfqpoint{2.126013in}{2.517471in}}{\pgfqpoint{2.122740in}{2.509571in}}{\pgfqpoint{2.122740in}{2.501334in}}%
\pgfpathcurveto{\pgfqpoint{2.122740in}{2.493098in}}{\pgfqpoint{2.126013in}{2.485198in}}{\pgfqpoint{2.131837in}{2.479374in}}%
\pgfpathcurveto{\pgfqpoint{2.137661in}{2.473550in}}{\pgfqpoint{2.145561in}{2.470278in}}{\pgfqpoint{2.153797in}{2.470278in}}%
\pgfpathclose%
\pgfusepath{stroke,fill}%
\end{pgfscope}%
\begin{pgfscope}%
\pgfpathrectangle{\pgfqpoint{0.100000in}{0.212622in}}{\pgfqpoint{3.696000in}{3.696000in}}%
\pgfusepath{clip}%
\pgfsetbuttcap%
\pgfsetroundjoin%
\definecolor{currentfill}{rgb}{0.121569,0.466667,0.705882}%
\pgfsetfillcolor{currentfill}%
\pgfsetfillopacity{0.917333}%
\pgfsetlinewidth{1.003750pt}%
\definecolor{currentstroke}{rgb}{0.121569,0.466667,0.705882}%
\pgfsetstrokecolor{currentstroke}%
\pgfsetstrokeopacity{0.917333}%
\pgfsetdash{}{0pt}%
\pgfpathmoveto{\pgfqpoint{1.040329in}{2.012220in}}%
\pgfpathcurveto{\pgfqpoint{1.048565in}{2.012220in}}{\pgfqpoint{1.056465in}{2.015492in}}{\pgfqpoint{1.062289in}{2.021316in}}%
\pgfpathcurveto{\pgfqpoint{1.068113in}{2.027140in}}{\pgfqpoint{1.071385in}{2.035040in}}{\pgfqpoint{1.071385in}{2.043276in}}%
\pgfpathcurveto{\pgfqpoint{1.071385in}{2.051512in}}{\pgfqpoint{1.068113in}{2.059413in}}{\pgfqpoint{1.062289in}{2.065236in}}%
\pgfpathcurveto{\pgfqpoint{1.056465in}{2.071060in}}{\pgfqpoint{1.048565in}{2.074333in}}{\pgfqpoint{1.040329in}{2.074333in}}%
\pgfpathcurveto{\pgfqpoint{1.032092in}{2.074333in}}{\pgfqpoint{1.024192in}{2.071060in}}{\pgfqpoint{1.018368in}{2.065236in}}%
\pgfpathcurveto{\pgfqpoint{1.012544in}{2.059413in}}{\pgfqpoint{1.009272in}{2.051512in}}{\pgfqpoint{1.009272in}{2.043276in}}%
\pgfpathcurveto{\pgfqpoint{1.009272in}{2.035040in}}{\pgfqpoint{1.012544in}{2.027140in}}{\pgfqpoint{1.018368in}{2.021316in}}%
\pgfpathcurveto{\pgfqpoint{1.024192in}{2.015492in}}{\pgfqpoint{1.032092in}{2.012220in}}{\pgfqpoint{1.040329in}{2.012220in}}%
\pgfpathclose%
\pgfusepath{stroke,fill}%
\end{pgfscope}%
\begin{pgfscope}%
\pgfpathrectangle{\pgfqpoint{0.100000in}{0.212622in}}{\pgfqpoint{3.696000in}{3.696000in}}%
\pgfusepath{clip}%
\pgfsetbuttcap%
\pgfsetroundjoin%
\definecolor{currentfill}{rgb}{0.121569,0.466667,0.705882}%
\pgfsetfillcolor{currentfill}%
\pgfsetfillopacity{0.917906}%
\pgfsetlinewidth{1.003750pt}%
\definecolor{currentstroke}{rgb}{0.121569,0.466667,0.705882}%
\pgfsetstrokecolor{currentstroke}%
\pgfsetstrokeopacity{0.917906}%
\pgfsetdash{}{0pt}%
\pgfpathmoveto{\pgfqpoint{1.049115in}{2.007387in}}%
\pgfpathcurveto{\pgfqpoint{1.057351in}{2.007387in}}{\pgfqpoint{1.065251in}{2.010659in}}{\pgfqpoint{1.071075in}{2.016483in}}%
\pgfpathcurveto{\pgfqpoint{1.076899in}{2.022307in}}{\pgfqpoint{1.080171in}{2.030207in}}{\pgfqpoint{1.080171in}{2.038444in}}%
\pgfpathcurveto{\pgfqpoint{1.080171in}{2.046680in}}{\pgfqpoint{1.076899in}{2.054580in}}{\pgfqpoint{1.071075in}{2.060404in}}%
\pgfpathcurveto{\pgfqpoint{1.065251in}{2.066228in}}{\pgfqpoint{1.057351in}{2.069500in}}{\pgfqpoint{1.049115in}{2.069500in}}%
\pgfpathcurveto{\pgfqpoint{1.040879in}{2.069500in}}{\pgfqpoint{1.032979in}{2.066228in}}{\pgfqpoint{1.027155in}{2.060404in}}%
\pgfpathcurveto{\pgfqpoint{1.021331in}{2.054580in}}{\pgfqpoint{1.018058in}{2.046680in}}{\pgfqpoint{1.018058in}{2.038444in}}%
\pgfpathcurveto{\pgfqpoint{1.018058in}{2.030207in}}{\pgfqpoint{1.021331in}{2.022307in}}{\pgfqpoint{1.027155in}{2.016483in}}%
\pgfpathcurveto{\pgfqpoint{1.032979in}{2.010659in}}{\pgfqpoint{1.040879in}{2.007387in}}{\pgfqpoint{1.049115in}{2.007387in}}%
\pgfpathclose%
\pgfusepath{stroke,fill}%
\end{pgfscope}%
\begin{pgfscope}%
\pgfpathrectangle{\pgfqpoint{0.100000in}{0.212622in}}{\pgfqpoint{3.696000in}{3.696000in}}%
\pgfusepath{clip}%
\pgfsetbuttcap%
\pgfsetroundjoin%
\definecolor{currentfill}{rgb}{0.121569,0.466667,0.705882}%
\pgfsetfillcolor{currentfill}%
\pgfsetfillopacity{0.918016}%
\pgfsetlinewidth{1.003750pt}%
\definecolor{currentstroke}{rgb}{0.121569,0.466667,0.705882}%
\pgfsetstrokecolor{currentstroke}%
\pgfsetstrokeopacity{0.918016}%
\pgfsetdash{}{0pt}%
\pgfpathmoveto{\pgfqpoint{1.730585in}{2.568413in}}%
\pgfpathcurveto{\pgfqpoint{1.738821in}{2.568413in}}{\pgfqpoint{1.746721in}{2.571686in}}{\pgfqpoint{1.752545in}{2.577510in}}%
\pgfpathcurveto{\pgfqpoint{1.758369in}{2.583333in}}{\pgfqpoint{1.761641in}{2.591233in}}{\pgfqpoint{1.761641in}{2.599470in}}%
\pgfpathcurveto{\pgfqpoint{1.761641in}{2.607706in}}{\pgfqpoint{1.758369in}{2.615606in}}{\pgfqpoint{1.752545in}{2.621430in}}%
\pgfpathcurveto{\pgfqpoint{1.746721in}{2.627254in}}{\pgfqpoint{1.738821in}{2.630526in}}{\pgfqpoint{1.730585in}{2.630526in}}%
\pgfpathcurveto{\pgfqpoint{1.722349in}{2.630526in}}{\pgfqpoint{1.714448in}{2.627254in}}{\pgfqpoint{1.708625in}{2.621430in}}%
\pgfpathcurveto{\pgfqpoint{1.702801in}{2.615606in}}{\pgfqpoint{1.699528in}{2.607706in}}{\pgfqpoint{1.699528in}{2.599470in}}%
\pgfpathcurveto{\pgfqpoint{1.699528in}{2.591233in}}{\pgfqpoint{1.702801in}{2.583333in}}{\pgfqpoint{1.708625in}{2.577510in}}%
\pgfpathcurveto{\pgfqpoint{1.714448in}{2.571686in}}{\pgfqpoint{1.722349in}{2.568413in}}{\pgfqpoint{1.730585in}{2.568413in}}%
\pgfpathclose%
\pgfusepath{stroke,fill}%
\end{pgfscope}%
\begin{pgfscope}%
\pgfpathrectangle{\pgfqpoint{0.100000in}{0.212622in}}{\pgfqpoint{3.696000in}{3.696000in}}%
\pgfusepath{clip}%
\pgfsetbuttcap%
\pgfsetroundjoin%
\definecolor{currentfill}{rgb}{0.121569,0.466667,0.705882}%
\pgfsetfillcolor{currentfill}%
\pgfsetfillopacity{0.918372}%
\pgfsetlinewidth{1.003750pt}%
\definecolor{currentstroke}{rgb}{0.121569,0.466667,0.705882}%
\pgfsetstrokecolor{currentstroke}%
\pgfsetstrokeopacity{0.918372}%
\pgfsetdash{}{0pt}%
\pgfpathmoveto{\pgfqpoint{2.151703in}{2.467980in}}%
\pgfpathcurveto{\pgfqpoint{2.159939in}{2.467980in}}{\pgfqpoint{2.167839in}{2.471253in}}{\pgfqpoint{2.173663in}{2.477076in}}%
\pgfpathcurveto{\pgfqpoint{2.179487in}{2.482900in}}{\pgfqpoint{2.182759in}{2.490800in}}{\pgfqpoint{2.182759in}{2.499037in}}%
\pgfpathcurveto{\pgfqpoint{2.182759in}{2.507273in}}{\pgfqpoint{2.179487in}{2.515173in}}{\pgfqpoint{2.173663in}{2.520997in}}%
\pgfpathcurveto{\pgfqpoint{2.167839in}{2.526821in}}{\pgfqpoint{2.159939in}{2.530093in}}{\pgfqpoint{2.151703in}{2.530093in}}%
\pgfpathcurveto{\pgfqpoint{2.143467in}{2.530093in}}{\pgfqpoint{2.135567in}{2.526821in}}{\pgfqpoint{2.129743in}{2.520997in}}%
\pgfpathcurveto{\pgfqpoint{2.123919in}{2.515173in}}{\pgfqpoint{2.120646in}{2.507273in}}{\pgfqpoint{2.120646in}{2.499037in}}%
\pgfpathcurveto{\pgfqpoint{2.120646in}{2.490800in}}{\pgfqpoint{2.123919in}{2.482900in}}{\pgfqpoint{2.129743in}{2.477076in}}%
\pgfpathcurveto{\pgfqpoint{2.135567in}{2.471253in}}{\pgfqpoint{2.143467in}{2.467980in}}{\pgfqpoint{2.151703in}{2.467980in}}%
\pgfpathclose%
\pgfusepath{stroke,fill}%
\end{pgfscope}%
\begin{pgfscope}%
\pgfpathrectangle{\pgfqpoint{0.100000in}{0.212622in}}{\pgfqpoint{3.696000in}{3.696000in}}%
\pgfusepath{clip}%
\pgfsetbuttcap%
\pgfsetroundjoin%
\definecolor{currentfill}{rgb}{0.121569,0.466667,0.705882}%
\pgfsetfillcolor{currentfill}%
\pgfsetfillopacity{0.918699}%
\pgfsetlinewidth{1.003750pt}%
\definecolor{currentstroke}{rgb}{0.121569,0.466667,0.705882}%
\pgfsetstrokecolor{currentstroke}%
\pgfsetstrokeopacity{0.918699}%
\pgfsetdash{}{0pt}%
\pgfpathmoveto{\pgfqpoint{1.059023in}{2.002291in}}%
\pgfpathcurveto{\pgfqpoint{1.067259in}{2.002291in}}{\pgfqpoint{1.075159in}{2.005563in}}{\pgfqpoint{1.080983in}{2.011387in}}%
\pgfpathcurveto{\pgfqpoint{1.086807in}{2.017211in}}{\pgfqpoint{1.090080in}{2.025111in}}{\pgfqpoint{1.090080in}{2.033348in}}%
\pgfpathcurveto{\pgfqpoint{1.090080in}{2.041584in}}{\pgfqpoint{1.086807in}{2.049484in}}{\pgfqpoint{1.080983in}{2.055308in}}%
\pgfpathcurveto{\pgfqpoint{1.075159in}{2.061132in}}{\pgfqpoint{1.067259in}{2.064404in}}{\pgfqpoint{1.059023in}{2.064404in}}%
\pgfpathcurveto{\pgfqpoint{1.050787in}{2.064404in}}{\pgfqpoint{1.042887in}{2.061132in}}{\pgfqpoint{1.037063in}{2.055308in}}%
\pgfpathcurveto{\pgfqpoint{1.031239in}{2.049484in}}{\pgfqpoint{1.027967in}{2.041584in}}{\pgfqpoint{1.027967in}{2.033348in}}%
\pgfpathcurveto{\pgfqpoint{1.027967in}{2.025111in}}{\pgfqpoint{1.031239in}{2.017211in}}{\pgfqpoint{1.037063in}{2.011387in}}%
\pgfpathcurveto{\pgfqpoint{1.042887in}{2.005563in}}{\pgfqpoint{1.050787in}{2.002291in}}{\pgfqpoint{1.059023in}{2.002291in}}%
\pgfpathclose%
\pgfusepath{stroke,fill}%
\end{pgfscope}%
\begin{pgfscope}%
\pgfpathrectangle{\pgfqpoint{0.100000in}{0.212622in}}{\pgfqpoint{3.696000in}{3.696000in}}%
\pgfusepath{clip}%
\pgfsetbuttcap%
\pgfsetroundjoin%
\definecolor{currentfill}{rgb}{0.121569,0.466667,0.705882}%
\pgfsetfillcolor{currentfill}%
\pgfsetfillopacity{0.919425}%
\pgfsetlinewidth{1.003750pt}%
\definecolor{currentstroke}{rgb}{0.121569,0.466667,0.705882}%
\pgfsetstrokecolor{currentstroke}%
\pgfsetstrokeopacity{0.919425}%
\pgfsetdash{}{0pt}%
\pgfpathmoveto{\pgfqpoint{2.644480in}{1.266285in}}%
\pgfpathcurveto{\pgfqpoint{2.652716in}{1.266285in}}{\pgfqpoint{2.660616in}{1.269557in}}{\pgfqpoint{2.666440in}{1.275381in}}%
\pgfpathcurveto{\pgfqpoint{2.672264in}{1.281205in}}{\pgfqpoint{2.675536in}{1.289105in}}{\pgfqpoint{2.675536in}{1.297341in}}%
\pgfpathcurveto{\pgfqpoint{2.675536in}{1.305578in}}{\pgfqpoint{2.672264in}{1.313478in}}{\pgfqpoint{2.666440in}{1.319302in}}%
\pgfpathcurveto{\pgfqpoint{2.660616in}{1.325126in}}{\pgfqpoint{2.652716in}{1.328398in}}{\pgfqpoint{2.644480in}{1.328398in}}%
\pgfpathcurveto{\pgfqpoint{2.636243in}{1.328398in}}{\pgfqpoint{2.628343in}{1.325126in}}{\pgfqpoint{2.622519in}{1.319302in}}%
\pgfpathcurveto{\pgfqpoint{2.616696in}{1.313478in}}{\pgfqpoint{2.613423in}{1.305578in}}{\pgfqpoint{2.613423in}{1.297341in}}%
\pgfpathcurveto{\pgfqpoint{2.613423in}{1.289105in}}{\pgfqpoint{2.616696in}{1.281205in}}{\pgfqpoint{2.622519in}{1.275381in}}%
\pgfpathcurveto{\pgfqpoint{2.628343in}{1.269557in}}{\pgfqpoint{2.636243in}{1.266285in}}{\pgfqpoint{2.644480in}{1.266285in}}%
\pgfpathclose%
\pgfusepath{stroke,fill}%
\end{pgfscope}%
\begin{pgfscope}%
\pgfpathrectangle{\pgfqpoint{0.100000in}{0.212622in}}{\pgfqpoint{3.696000in}{3.696000in}}%
\pgfusepath{clip}%
\pgfsetbuttcap%
\pgfsetroundjoin%
\definecolor{currentfill}{rgb}{0.121569,0.466667,0.705882}%
\pgfsetfillcolor{currentfill}%
\pgfsetfillopacity{0.919485}%
\pgfsetlinewidth{1.003750pt}%
\definecolor{currentstroke}{rgb}{0.121569,0.466667,0.705882}%
\pgfsetstrokecolor{currentstroke}%
\pgfsetstrokeopacity{0.919485}%
\pgfsetdash{}{0pt}%
\pgfpathmoveto{\pgfqpoint{1.731435in}{2.565061in}}%
\pgfpathcurveto{\pgfqpoint{1.739671in}{2.565061in}}{\pgfqpoint{1.747571in}{2.568333in}}{\pgfqpoint{1.753395in}{2.574157in}}%
\pgfpathcurveto{\pgfqpoint{1.759219in}{2.579981in}}{\pgfqpoint{1.762491in}{2.587881in}}{\pgfqpoint{1.762491in}{2.596118in}}%
\pgfpathcurveto{\pgfqpoint{1.762491in}{2.604354in}}{\pgfqpoint{1.759219in}{2.612254in}}{\pgfqpoint{1.753395in}{2.618078in}}%
\pgfpathcurveto{\pgfqpoint{1.747571in}{2.623902in}}{\pgfqpoint{1.739671in}{2.627174in}}{\pgfqpoint{1.731435in}{2.627174in}}%
\pgfpathcurveto{\pgfqpoint{1.723198in}{2.627174in}}{\pgfqpoint{1.715298in}{2.623902in}}{\pgfqpoint{1.709474in}{2.618078in}}%
\pgfpathcurveto{\pgfqpoint{1.703650in}{2.612254in}}{\pgfqpoint{1.700378in}{2.604354in}}{\pgfqpoint{1.700378in}{2.596118in}}%
\pgfpathcurveto{\pgfqpoint{1.700378in}{2.587881in}}{\pgfqpoint{1.703650in}{2.579981in}}{\pgfqpoint{1.709474in}{2.574157in}}%
\pgfpathcurveto{\pgfqpoint{1.715298in}{2.568333in}}{\pgfqpoint{1.723198in}{2.565061in}}{\pgfqpoint{1.731435in}{2.565061in}}%
\pgfpathclose%
\pgfusepath{stroke,fill}%
\end{pgfscope}%
\begin{pgfscope}%
\pgfpathrectangle{\pgfqpoint{0.100000in}{0.212622in}}{\pgfqpoint{3.696000in}{3.696000in}}%
\pgfusepath{clip}%
\pgfsetbuttcap%
\pgfsetroundjoin%
\definecolor{currentfill}{rgb}{0.121569,0.466667,0.705882}%
\pgfsetfillcolor{currentfill}%
\pgfsetfillopacity{0.919492}%
\pgfsetlinewidth{1.003750pt}%
\definecolor{currentstroke}{rgb}{0.121569,0.466667,0.705882}%
\pgfsetstrokecolor{currentstroke}%
\pgfsetstrokeopacity{0.919492}%
\pgfsetdash{}{0pt}%
\pgfpathmoveto{\pgfqpoint{2.149713in}{2.466044in}}%
\pgfpathcurveto{\pgfqpoint{2.157949in}{2.466044in}}{\pgfqpoint{2.165849in}{2.469316in}}{\pgfqpoint{2.171673in}{2.475140in}}%
\pgfpathcurveto{\pgfqpoint{2.177497in}{2.480964in}}{\pgfqpoint{2.180770in}{2.488864in}}{\pgfqpoint{2.180770in}{2.497101in}}%
\pgfpathcurveto{\pgfqpoint{2.180770in}{2.505337in}}{\pgfqpoint{2.177497in}{2.513237in}}{\pgfqpoint{2.171673in}{2.519061in}}%
\pgfpathcurveto{\pgfqpoint{2.165849in}{2.524885in}}{\pgfqpoint{2.157949in}{2.528157in}}{\pgfqpoint{2.149713in}{2.528157in}}%
\pgfpathcurveto{\pgfqpoint{2.141477in}{2.528157in}}{\pgfqpoint{2.133577in}{2.524885in}}{\pgfqpoint{2.127753in}{2.519061in}}%
\pgfpathcurveto{\pgfqpoint{2.121929in}{2.513237in}}{\pgfqpoint{2.118657in}{2.505337in}}{\pgfqpoint{2.118657in}{2.497101in}}%
\pgfpathcurveto{\pgfqpoint{2.118657in}{2.488864in}}{\pgfqpoint{2.121929in}{2.480964in}}{\pgfqpoint{2.127753in}{2.475140in}}%
\pgfpathcurveto{\pgfqpoint{2.133577in}{2.469316in}}{\pgfqpoint{2.141477in}{2.466044in}}{\pgfqpoint{2.149713in}{2.466044in}}%
\pgfpathclose%
\pgfusepath{stroke,fill}%
\end{pgfscope}%
\begin{pgfscope}%
\pgfpathrectangle{\pgfqpoint{0.100000in}{0.212622in}}{\pgfqpoint{3.696000in}{3.696000in}}%
\pgfusepath{clip}%
\pgfsetbuttcap%
\pgfsetroundjoin%
\definecolor{currentfill}{rgb}{0.121569,0.466667,0.705882}%
\pgfsetfillcolor{currentfill}%
\pgfsetfillopacity{0.919786}%
\pgfsetlinewidth{1.003750pt}%
\definecolor{currentstroke}{rgb}{0.121569,0.466667,0.705882}%
\pgfsetstrokecolor{currentstroke}%
\pgfsetstrokeopacity{0.919786}%
\pgfsetdash{}{0pt}%
\pgfpathmoveto{\pgfqpoint{1.068901in}{1.996987in}}%
\pgfpathcurveto{\pgfqpoint{1.077138in}{1.996987in}}{\pgfqpoint{1.085038in}{2.000259in}}{\pgfqpoint{1.090862in}{2.006083in}}%
\pgfpathcurveto{\pgfqpoint{1.096685in}{2.011907in}}{\pgfqpoint{1.099958in}{2.019807in}}{\pgfqpoint{1.099958in}{2.028043in}}%
\pgfpathcurveto{\pgfqpoint{1.099958in}{2.036279in}}{\pgfqpoint{1.096685in}{2.044179in}}{\pgfqpoint{1.090862in}{2.050003in}}%
\pgfpathcurveto{\pgfqpoint{1.085038in}{2.055827in}}{\pgfqpoint{1.077138in}{2.059100in}}{\pgfqpoint{1.068901in}{2.059100in}}%
\pgfpathcurveto{\pgfqpoint{1.060665in}{2.059100in}}{\pgfqpoint{1.052765in}{2.055827in}}{\pgfqpoint{1.046941in}{2.050003in}}%
\pgfpathcurveto{\pgfqpoint{1.041117in}{2.044179in}}{\pgfqpoint{1.037845in}{2.036279in}}{\pgfqpoint{1.037845in}{2.028043in}}%
\pgfpathcurveto{\pgfqpoint{1.037845in}{2.019807in}}{\pgfqpoint{1.041117in}{2.011907in}}{\pgfqpoint{1.046941in}{2.006083in}}%
\pgfpathcurveto{\pgfqpoint{1.052765in}{2.000259in}}{\pgfqpoint{1.060665in}{1.996987in}}{\pgfqpoint{1.068901in}{1.996987in}}%
\pgfpathclose%
\pgfusepath{stroke,fill}%
\end{pgfscope}%
\begin{pgfscope}%
\pgfpathrectangle{\pgfqpoint{0.100000in}{0.212622in}}{\pgfqpoint{3.696000in}{3.696000in}}%
\pgfusepath{clip}%
\pgfsetbuttcap%
\pgfsetroundjoin%
\definecolor{currentfill}{rgb}{0.121569,0.466667,0.705882}%
\pgfsetfillcolor{currentfill}%
\pgfsetfillopacity{0.920278}%
\pgfsetlinewidth{1.003750pt}%
\definecolor{currentstroke}{rgb}{0.121569,0.466667,0.705882}%
\pgfsetstrokecolor{currentstroke}%
\pgfsetstrokeopacity{0.920278}%
\pgfsetdash{}{0pt}%
\pgfpathmoveto{\pgfqpoint{2.148356in}{2.464774in}}%
\pgfpathcurveto{\pgfqpoint{2.156592in}{2.464774in}}{\pgfqpoint{2.164493in}{2.468046in}}{\pgfqpoint{2.170316in}{2.473870in}}%
\pgfpathcurveto{\pgfqpoint{2.176140in}{2.479694in}}{\pgfqpoint{2.179413in}{2.487594in}}{\pgfqpoint{2.179413in}{2.495830in}}%
\pgfpathcurveto{\pgfqpoint{2.179413in}{2.504066in}}{\pgfqpoint{2.176140in}{2.511966in}}{\pgfqpoint{2.170316in}{2.517790in}}%
\pgfpathcurveto{\pgfqpoint{2.164493in}{2.523614in}}{\pgfqpoint{2.156592in}{2.526887in}}{\pgfqpoint{2.148356in}{2.526887in}}%
\pgfpathcurveto{\pgfqpoint{2.140120in}{2.526887in}}{\pgfqpoint{2.132220in}{2.523614in}}{\pgfqpoint{2.126396in}{2.517790in}}%
\pgfpathcurveto{\pgfqpoint{2.120572in}{2.511966in}}{\pgfqpoint{2.117300in}{2.504066in}}{\pgfqpoint{2.117300in}{2.495830in}}%
\pgfpathcurveto{\pgfqpoint{2.117300in}{2.487594in}}{\pgfqpoint{2.120572in}{2.479694in}}{\pgfqpoint{2.126396in}{2.473870in}}%
\pgfpathcurveto{\pgfqpoint{2.132220in}{2.468046in}}{\pgfqpoint{2.140120in}{2.464774in}}{\pgfqpoint{2.148356in}{2.464774in}}%
\pgfpathclose%
\pgfusepath{stroke,fill}%
\end{pgfscope}%
\begin{pgfscope}%
\pgfpathrectangle{\pgfqpoint{0.100000in}{0.212622in}}{\pgfqpoint{3.696000in}{3.696000in}}%
\pgfusepath{clip}%
\pgfsetbuttcap%
\pgfsetroundjoin%
\definecolor{currentfill}{rgb}{0.121569,0.466667,0.705882}%
\pgfsetfillcolor{currentfill}%
\pgfsetfillopacity{0.920540}%
\pgfsetlinewidth{1.003750pt}%
\definecolor{currentstroke}{rgb}{0.121569,0.466667,0.705882}%
\pgfsetstrokecolor{currentstroke}%
\pgfsetstrokeopacity{0.920540}%
\pgfsetdash{}{0pt}%
\pgfpathmoveto{\pgfqpoint{1.074140in}{1.994384in}}%
\pgfpathcurveto{\pgfqpoint{1.082376in}{1.994384in}}{\pgfqpoint{1.090276in}{1.997656in}}{\pgfqpoint{1.096100in}{2.003480in}}%
\pgfpathcurveto{\pgfqpoint{1.101924in}{2.009304in}}{\pgfqpoint{1.105196in}{2.017204in}}{\pgfqpoint{1.105196in}{2.025440in}}%
\pgfpathcurveto{\pgfqpoint{1.105196in}{2.033676in}}{\pgfqpoint{1.101924in}{2.041577in}}{\pgfqpoint{1.096100in}{2.047400in}}%
\pgfpathcurveto{\pgfqpoint{1.090276in}{2.053224in}}{\pgfqpoint{1.082376in}{2.056497in}}{\pgfqpoint{1.074140in}{2.056497in}}%
\pgfpathcurveto{\pgfqpoint{1.065903in}{2.056497in}}{\pgfqpoint{1.058003in}{2.053224in}}{\pgfqpoint{1.052179in}{2.047400in}}%
\pgfpathcurveto{\pgfqpoint{1.046356in}{2.041577in}}{\pgfqpoint{1.043083in}{2.033676in}}{\pgfqpoint{1.043083in}{2.025440in}}%
\pgfpathcurveto{\pgfqpoint{1.043083in}{2.017204in}}{\pgfqpoint{1.046356in}{2.009304in}}{\pgfqpoint{1.052179in}{2.003480in}}%
\pgfpathcurveto{\pgfqpoint{1.058003in}{1.997656in}}{\pgfqpoint{1.065903in}{1.994384in}}{\pgfqpoint{1.074140in}{1.994384in}}%
\pgfpathclose%
\pgfusepath{stroke,fill}%
\end{pgfscope}%
\begin{pgfscope}%
\pgfpathrectangle{\pgfqpoint{0.100000in}{0.212622in}}{\pgfqpoint{3.696000in}{3.696000in}}%
\pgfusepath{clip}%
\pgfsetbuttcap%
\pgfsetroundjoin%
\definecolor{currentfill}{rgb}{0.121569,0.466667,0.705882}%
\pgfsetfillcolor{currentfill}%
\pgfsetfillopacity{0.921037}%
\pgfsetlinewidth{1.003750pt}%
\definecolor{currentstroke}{rgb}{0.121569,0.466667,0.705882}%
\pgfsetstrokecolor{currentstroke}%
\pgfsetstrokeopacity{0.921037}%
\pgfsetdash{}{0pt}%
\pgfpathmoveto{\pgfqpoint{1.732686in}{2.561407in}}%
\pgfpathcurveto{\pgfqpoint{1.740923in}{2.561407in}}{\pgfqpoint{1.748823in}{2.564679in}}{\pgfqpoint{1.754647in}{2.570503in}}%
\pgfpathcurveto{\pgfqpoint{1.760471in}{2.576327in}}{\pgfqpoint{1.763743in}{2.584227in}}{\pgfqpoint{1.763743in}{2.592463in}}%
\pgfpathcurveto{\pgfqpoint{1.763743in}{2.600699in}}{\pgfqpoint{1.760471in}{2.608599in}}{\pgfqpoint{1.754647in}{2.614423in}}%
\pgfpathcurveto{\pgfqpoint{1.748823in}{2.620247in}}{\pgfqpoint{1.740923in}{2.623520in}}{\pgfqpoint{1.732686in}{2.623520in}}%
\pgfpathcurveto{\pgfqpoint{1.724450in}{2.623520in}}{\pgfqpoint{1.716550in}{2.620247in}}{\pgfqpoint{1.710726in}{2.614423in}}%
\pgfpathcurveto{\pgfqpoint{1.704902in}{2.608599in}}{\pgfqpoint{1.701630in}{2.600699in}}{\pgfqpoint{1.701630in}{2.592463in}}%
\pgfpathcurveto{\pgfqpoint{1.701630in}{2.584227in}}{\pgfqpoint{1.704902in}{2.576327in}}{\pgfqpoint{1.710726in}{2.570503in}}%
\pgfpathcurveto{\pgfqpoint{1.716550in}{2.564679in}}{\pgfqpoint{1.724450in}{2.561407in}}{\pgfqpoint{1.732686in}{2.561407in}}%
\pgfpathclose%
\pgfusepath{stroke,fill}%
\end{pgfscope}%
\begin{pgfscope}%
\pgfpathrectangle{\pgfqpoint{0.100000in}{0.212622in}}{\pgfqpoint{3.696000in}{3.696000in}}%
\pgfusepath{clip}%
\pgfsetbuttcap%
\pgfsetroundjoin%
\definecolor{currentfill}{rgb}{0.121569,0.466667,0.705882}%
\pgfsetfillcolor{currentfill}%
\pgfsetfillopacity{0.921160}%
\pgfsetlinewidth{1.003750pt}%
\definecolor{currentstroke}{rgb}{0.121569,0.466667,0.705882}%
\pgfsetstrokecolor{currentstroke}%
\pgfsetstrokeopacity{0.921160}%
\pgfsetdash{}{0pt}%
\pgfpathmoveto{\pgfqpoint{1.080206in}{1.990860in}}%
\pgfpathcurveto{\pgfqpoint{1.088442in}{1.990860in}}{\pgfqpoint{1.096342in}{1.994132in}}{\pgfqpoint{1.102166in}{1.999956in}}%
\pgfpathcurveto{\pgfqpoint{1.107990in}{2.005780in}}{\pgfqpoint{1.111262in}{2.013680in}}{\pgfqpoint{1.111262in}{2.021916in}}%
\pgfpathcurveto{\pgfqpoint{1.111262in}{2.030152in}}{\pgfqpoint{1.107990in}{2.038052in}}{\pgfqpoint{1.102166in}{2.043876in}}%
\pgfpathcurveto{\pgfqpoint{1.096342in}{2.049700in}}{\pgfqpoint{1.088442in}{2.052973in}}{\pgfqpoint{1.080206in}{2.052973in}}%
\pgfpathcurveto{\pgfqpoint{1.071969in}{2.052973in}}{\pgfqpoint{1.064069in}{2.049700in}}{\pgfqpoint{1.058245in}{2.043876in}}%
\pgfpathcurveto{\pgfqpoint{1.052422in}{2.038052in}}{\pgfqpoint{1.049149in}{2.030152in}}{\pgfqpoint{1.049149in}{2.021916in}}%
\pgfpathcurveto{\pgfqpoint{1.049149in}{2.013680in}}{\pgfqpoint{1.052422in}{2.005780in}}{\pgfqpoint{1.058245in}{1.999956in}}%
\pgfpathcurveto{\pgfqpoint{1.064069in}{1.994132in}}{\pgfqpoint{1.071969in}{1.990860in}}{\pgfqpoint{1.080206in}{1.990860in}}%
\pgfpathclose%
\pgfusepath{stroke,fill}%
\end{pgfscope}%
\begin{pgfscope}%
\pgfpathrectangle{\pgfqpoint{0.100000in}{0.212622in}}{\pgfqpoint{3.696000in}{3.696000in}}%
\pgfusepath{clip}%
\pgfsetbuttcap%
\pgfsetroundjoin%
\definecolor{currentfill}{rgb}{0.121569,0.466667,0.705882}%
\pgfsetfillcolor{currentfill}%
\pgfsetfillopacity{0.921511}%
\pgfsetlinewidth{1.003750pt}%
\definecolor{currentstroke}{rgb}{0.121569,0.466667,0.705882}%
\pgfsetstrokecolor{currentstroke}%
\pgfsetstrokeopacity{0.921511}%
\pgfsetdash{}{0pt}%
\pgfpathmoveto{\pgfqpoint{1.083510in}{1.988865in}}%
\pgfpathcurveto{\pgfqpoint{1.091746in}{1.988865in}}{\pgfqpoint{1.099646in}{1.992138in}}{\pgfqpoint{1.105470in}{1.997962in}}%
\pgfpathcurveto{\pgfqpoint{1.111294in}{2.003786in}}{\pgfqpoint{1.114566in}{2.011686in}}{\pgfqpoint{1.114566in}{2.019922in}}%
\pgfpathcurveto{\pgfqpoint{1.114566in}{2.028158in}}{\pgfqpoint{1.111294in}{2.036058in}}{\pgfqpoint{1.105470in}{2.041882in}}%
\pgfpathcurveto{\pgfqpoint{1.099646in}{2.047706in}}{\pgfqpoint{1.091746in}{2.050978in}}{\pgfqpoint{1.083510in}{2.050978in}}%
\pgfpathcurveto{\pgfqpoint{1.075274in}{2.050978in}}{\pgfqpoint{1.067373in}{2.047706in}}{\pgfqpoint{1.061550in}{2.041882in}}%
\pgfpathcurveto{\pgfqpoint{1.055726in}{2.036058in}}{\pgfqpoint{1.052453in}{2.028158in}}{\pgfqpoint{1.052453in}{2.019922in}}%
\pgfpathcurveto{\pgfqpoint{1.052453in}{2.011686in}}{\pgfqpoint{1.055726in}{2.003786in}}{\pgfqpoint{1.061550in}{1.997962in}}%
\pgfpathcurveto{\pgfqpoint{1.067373in}{1.992138in}}{\pgfqpoint{1.075274in}{1.988865in}}{\pgfqpoint{1.083510in}{1.988865in}}%
\pgfpathclose%
\pgfusepath{stroke,fill}%
\end{pgfscope}%
\begin{pgfscope}%
\pgfpathrectangle{\pgfqpoint{0.100000in}{0.212622in}}{\pgfqpoint{3.696000in}{3.696000in}}%
\pgfusepath{clip}%
\pgfsetbuttcap%
\pgfsetroundjoin%
\definecolor{currentfill}{rgb}{0.121569,0.466667,0.705882}%
\pgfsetfillcolor{currentfill}%
\pgfsetfillopacity{0.921713}%
\pgfsetlinewidth{1.003750pt}%
\definecolor{currentstroke}{rgb}{0.121569,0.466667,0.705882}%
\pgfsetstrokecolor{currentstroke}%
\pgfsetstrokeopacity{0.921713}%
\pgfsetdash{}{0pt}%
\pgfpathmoveto{\pgfqpoint{1.085327in}{1.987829in}}%
\pgfpathcurveto{\pgfqpoint{1.093563in}{1.987829in}}{\pgfqpoint{1.101463in}{1.991101in}}{\pgfqpoint{1.107287in}{1.996925in}}%
\pgfpathcurveto{\pgfqpoint{1.113111in}{2.002749in}}{\pgfqpoint{1.116383in}{2.010649in}}{\pgfqpoint{1.116383in}{2.018886in}}%
\pgfpathcurveto{\pgfqpoint{1.116383in}{2.027122in}}{\pgfqpoint{1.113111in}{2.035022in}}{\pgfqpoint{1.107287in}{2.040846in}}%
\pgfpathcurveto{\pgfqpoint{1.101463in}{2.046670in}}{\pgfqpoint{1.093563in}{2.049942in}}{\pgfqpoint{1.085327in}{2.049942in}}%
\pgfpathcurveto{\pgfqpoint{1.077091in}{2.049942in}}{\pgfqpoint{1.069191in}{2.046670in}}{\pgfqpoint{1.063367in}{2.040846in}}%
\pgfpathcurveto{\pgfqpoint{1.057543in}{2.035022in}}{\pgfqpoint{1.054270in}{2.027122in}}{\pgfqpoint{1.054270in}{2.018886in}}%
\pgfpathcurveto{\pgfqpoint{1.054270in}{2.010649in}}{\pgfqpoint{1.057543in}{2.002749in}}{\pgfqpoint{1.063367in}{1.996925in}}%
\pgfpathcurveto{\pgfqpoint{1.069191in}{1.991101in}}{\pgfqpoint{1.077091in}{1.987829in}}{\pgfqpoint{1.085327in}{1.987829in}}%
\pgfpathclose%
\pgfusepath{stroke,fill}%
\end{pgfscope}%
\begin{pgfscope}%
\pgfpathrectangle{\pgfqpoint{0.100000in}{0.212622in}}{\pgfqpoint{3.696000in}{3.696000in}}%
\pgfusepath{clip}%
\pgfsetbuttcap%
\pgfsetroundjoin%
\definecolor{currentfill}{rgb}{0.121569,0.466667,0.705882}%
\pgfsetfillcolor{currentfill}%
\pgfsetfillopacity{0.921734}%
\pgfsetlinewidth{1.003750pt}%
\definecolor{currentstroke}{rgb}{0.121569,0.466667,0.705882}%
\pgfsetstrokecolor{currentstroke}%
\pgfsetstrokeopacity{0.921734}%
\pgfsetdash{}{0pt}%
\pgfpathmoveto{\pgfqpoint{2.145942in}{2.462586in}}%
\pgfpathcurveto{\pgfqpoint{2.154179in}{2.462586in}}{\pgfqpoint{2.162079in}{2.465858in}}{\pgfqpoint{2.167903in}{2.471682in}}%
\pgfpathcurveto{\pgfqpoint{2.173727in}{2.477506in}}{\pgfqpoint{2.176999in}{2.485406in}}{\pgfqpoint{2.176999in}{2.493642in}}%
\pgfpathcurveto{\pgfqpoint{2.176999in}{2.501879in}}{\pgfqpoint{2.173727in}{2.509779in}}{\pgfqpoint{2.167903in}{2.515603in}}%
\pgfpathcurveto{\pgfqpoint{2.162079in}{2.521427in}}{\pgfqpoint{2.154179in}{2.524699in}}{\pgfqpoint{2.145942in}{2.524699in}}%
\pgfpathcurveto{\pgfqpoint{2.137706in}{2.524699in}}{\pgfqpoint{2.129806in}{2.521427in}}{\pgfqpoint{2.123982in}{2.515603in}}%
\pgfpathcurveto{\pgfqpoint{2.118158in}{2.509779in}}{\pgfqpoint{2.114886in}{2.501879in}}{\pgfqpoint{2.114886in}{2.493642in}}%
\pgfpathcurveto{\pgfqpoint{2.114886in}{2.485406in}}{\pgfqpoint{2.118158in}{2.477506in}}{\pgfqpoint{2.123982in}{2.471682in}}%
\pgfpathcurveto{\pgfqpoint{2.129806in}{2.465858in}}{\pgfqpoint{2.137706in}{2.462586in}}{\pgfqpoint{2.145942in}{2.462586in}}%
\pgfpathclose%
\pgfusepath{stroke,fill}%
\end{pgfscope}%
\begin{pgfscope}%
\pgfpathrectangle{\pgfqpoint{0.100000in}{0.212622in}}{\pgfqpoint{3.696000in}{3.696000in}}%
\pgfusepath{clip}%
\pgfsetbuttcap%
\pgfsetroundjoin%
\definecolor{currentfill}{rgb}{0.121569,0.466667,0.705882}%
\pgfsetfillcolor{currentfill}%
\pgfsetfillopacity{0.921834}%
\pgfsetlinewidth{1.003750pt}%
\definecolor{currentstroke}{rgb}{0.121569,0.466667,0.705882}%
\pgfsetstrokecolor{currentstroke}%
\pgfsetstrokeopacity{0.921834}%
\pgfsetdash{}{0pt}%
\pgfpathmoveto{\pgfqpoint{1.086305in}{1.987241in}}%
\pgfpathcurveto{\pgfqpoint{1.094541in}{1.987241in}}{\pgfqpoint{1.102441in}{1.990513in}}{\pgfqpoint{1.108265in}{1.996337in}}%
\pgfpathcurveto{\pgfqpoint{1.114089in}{2.002161in}}{\pgfqpoint{1.117361in}{2.010061in}}{\pgfqpoint{1.117361in}{2.018297in}}%
\pgfpathcurveto{\pgfqpoint{1.117361in}{2.026534in}}{\pgfqpoint{1.114089in}{2.034434in}}{\pgfqpoint{1.108265in}{2.040258in}}%
\pgfpathcurveto{\pgfqpoint{1.102441in}{2.046082in}}{\pgfqpoint{1.094541in}{2.049354in}}{\pgfqpoint{1.086305in}{2.049354in}}%
\pgfpathcurveto{\pgfqpoint{1.078068in}{2.049354in}}{\pgfqpoint{1.070168in}{2.046082in}}{\pgfqpoint{1.064344in}{2.040258in}}%
\pgfpathcurveto{\pgfqpoint{1.058520in}{2.034434in}}{\pgfqpoint{1.055248in}{2.026534in}}{\pgfqpoint{1.055248in}{2.018297in}}%
\pgfpathcurveto{\pgfqpoint{1.055248in}{2.010061in}}{\pgfqpoint{1.058520in}{2.002161in}}{\pgfqpoint{1.064344in}{1.996337in}}%
\pgfpathcurveto{\pgfqpoint{1.070168in}{1.990513in}}{\pgfqpoint{1.078068in}{1.987241in}}{\pgfqpoint{1.086305in}{1.987241in}}%
\pgfpathclose%
\pgfusepath{stroke,fill}%
\end{pgfscope}%
\begin{pgfscope}%
\pgfpathrectangle{\pgfqpoint{0.100000in}{0.212622in}}{\pgfqpoint{3.696000in}{3.696000in}}%
\pgfusepath{clip}%
\pgfsetbuttcap%
\pgfsetroundjoin%
\definecolor{currentfill}{rgb}{0.121569,0.466667,0.705882}%
\pgfsetfillcolor{currentfill}%
\pgfsetfillopacity{0.921907}%
\pgfsetlinewidth{1.003750pt}%
\definecolor{currentstroke}{rgb}{0.121569,0.466667,0.705882}%
\pgfsetstrokecolor{currentstroke}%
\pgfsetstrokeopacity{0.921907}%
\pgfsetdash{}{0pt}%
\pgfpathmoveto{\pgfqpoint{1.086830in}{1.986922in}}%
\pgfpathcurveto{\pgfqpoint{1.095066in}{1.986922in}}{\pgfqpoint{1.102966in}{1.990194in}}{\pgfqpoint{1.108790in}{1.996018in}}%
\pgfpathcurveto{\pgfqpoint{1.114614in}{2.001842in}}{\pgfqpoint{1.117886in}{2.009742in}}{\pgfqpoint{1.117886in}{2.017978in}}%
\pgfpathcurveto{\pgfqpoint{1.117886in}{2.026215in}}{\pgfqpoint{1.114614in}{2.034115in}}{\pgfqpoint{1.108790in}{2.039939in}}%
\pgfpathcurveto{\pgfqpoint{1.102966in}{2.045763in}}{\pgfqpoint{1.095066in}{2.049035in}}{\pgfqpoint{1.086830in}{2.049035in}}%
\pgfpathcurveto{\pgfqpoint{1.078594in}{2.049035in}}{\pgfqpoint{1.070694in}{2.045763in}}{\pgfqpoint{1.064870in}{2.039939in}}%
\pgfpathcurveto{\pgfqpoint{1.059046in}{2.034115in}}{\pgfqpoint{1.055773in}{2.026215in}}{\pgfqpoint{1.055773in}{2.017978in}}%
\pgfpathcurveto{\pgfqpoint{1.055773in}{2.009742in}}{\pgfqpoint{1.059046in}{2.001842in}}{\pgfqpoint{1.064870in}{1.996018in}}%
\pgfpathcurveto{\pgfqpoint{1.070694in}{1.990194in}}{\pgfqpoint{1.078594in}{1.986922in}}{\pgfqpoint{1.086830in}{1.986922in}}%
\pgfpathclose%
\pgfusepath{stroke,fill}%
\end{pgfscope}%
\begin{pgfscope}%
\pgfpathrectangle{\pgfqpoint{0.100000in}{0.212622in}}{\pgfqpoint{3.696000in}{3.696000in}}%
\pgfusepath{clip}%
\pgfsetbuttcap%
\pgfsetroundjoin%
\definecolor{currentfill}{rgb}{0.121569,0.466667,0.705882}%
\pgfsetfillcolor{currentfill}%
\pgfsetfillopacity{0.922067}%
\pgfsetlinewidth{1.003750pt}%
\definecolor{currentstroke}{rgb}{0.121569,0.466667,0.705882}%
\pgfsetstrokecolor{currentstroke}%
\pgfsetstrokeopacity{0.922067}%
\pgfsetdash{}{0pt}%
\pgfpathmoveto{\pgfqpoint{1.087836in}{1.986229in}}%
\pgfpathcurveto{\pgfqpoint{1.096072in}{1.986229in}}{\pgfqpoint{1.103972in}{1.989501in}}{\pgfqpoint{1.109796in}{1.995325in}}%
\pgfpathcurveto{\pgfqpoint{1.115620in}{2.001149in}}{\pgfqpoint{1.118892in}{2.009049in}}{\pgfqpoint{1.118892in}{2.017285in}}%
\pgfpathcurveto{\pgfqpoint{1.118892in}{2.025521in}}{\pgfqpoint{1.115620in}{2.033422in}}{\pgfqpoint{1.109796in}{2.039245in}}%
\pgfpathcurveto{\pgfqpoint{1.103972in}{2.045069in}}{\pgfqpoint{1.096072in}{2.048342in}}{\pgfqpoint{1.087836in}{2.048342in}}%
\pgfpathcurveto{\pgfqpoint{1.079600in}{2.048342in}}{\pgfqpoint{1.071700in}{2.045069in}}{\pgfqpoint{1.065876in}{2.039245in}}%
\pgfpathcurveto{\pgfqpoint{1.060052in}{2.033422in}}{\pgfqpoint{1.056779in}{2.025521in}}{\pgfqpoint{1.056779in}{2.017285in}}%
\pgfpathcurveto{\pgfqpoint{1.056779in}{2.009049in}}{\pgfqpoint{1.060052in}{2.001149in}}{\pgfqpoint{1.065876in}{1.995325in}}%
\pgfpathcurveto{\pgfqpoint{1.071700in}{1.989501in}}{\pgfqpoint{1.079600in}{1.986229in}}{\pgfqpoint{1.087836in}{1.986229in}}%
\pgfpathclose%
\pgfusepath{stroke,fill}%
\end{pgfscope}%
\begin{pgfscope}%
\pgfpathrectangle{\pgfqpoint{0.100000in}{0.212622in}}{\pgfqpoint{3.696000in}{3.696000in}}%
\pgfusepath{clip}%
\pgfsetbuttcap%
\pgfsetroundjoin%
\definecolor{currentfill}{rgb}{0.121569,0.466667,0.705882}%
\pgfsetfillcolor{currentfill}%
\pgfsetfillopacity{0.922252}%
\pgfsetlinewidth{1.003750pt}%
\definecolor{currentstroke}{rgb}{0.121569,0.466667,0.705882}%
\pgfsetstrokecolor{currentstroke}%
\pgfsetstrokeopacity{0.922252}%
\pgfsetdash{}{0pt}%
\pgfpathmoveto{\pgfqpoint{1.089050in}{1.985304in}}%
\pgfpathcurveto{\pgfqpoint{1.097286in}{1.985304in}}{\pgfqpoint{1.105186in}{1.988577in}}{\pgfqpoint{1.111010in}{1.994400in}}%
\pgfpathcurveto{\pgfqpoint{1.116834in}{2.000224in}}{\pgfqpoint{1.120106in}{2.008124in}}{\pgfqpoint{1.120106in}{2.016361in}}%
\pgfpathcurveto{\pgfqpoint{1.120106in}{2.024597in}}{\pgfqpoint{1.116834in}{2.032497in}}{\pgfqpoint{1.111010in}{2.038321in}}%
\pgfpathcurveto{\pgfqpoint{1.105186in}{2.044145in}}{\pgfqpoint{1.097286in}{2.047417in}}{\pgfqpoint{1.089050in}{2.047417in}}%
\pgfpathcurveto{\pgfqpoint{1.080813in}{2.047417in}}{\pgfqpoint{1.072913in}{2.044145in}}{\pgfqpoint{1.067089in}{2.038321in}}%
\pgfpathcurveto{\pgfqpoint{1.061266in}{2.032497in}}{\pgfqpoint{1.057993in}{2.024597in}}{\pgfqpoint{1.057993in}{2.016361in}}%
\pgfpathcurveto{\pgfqpoint{1.057993in}{2.008124in}}{\pgfqpoint{1.061266in}{2.000224in}}{\pgfqpoint{1.067089in}{1.994400in}}%
\pgfpathcurveto{\pgfqpoint{1.072913in}{1.988577in}}{\pgfqpoint{1.080813in}{1.985304in}}{\pgfqpoint{1.089050in}{1.985304in}}%
\pgfpathclose%
\pgfusepath{stroke,fill}%
\end{pgfscope}%
\begin{pgfscope}%
\pgfpathrectangle{\pgfqpoint{0.100000in}{0.212622in}}{\pgfqpoint{3.696000in}{3.696000in}}%
\pgfusepath{clip}%
\pgfsetbuttcap%
\pgfsetroundjoin%
\definecolor{currentfill}{rgb}{0.121569,0.466667,0.705882}%
\pgfsetfillcolor{currentfill}%
\pgfsetfillopacity{0.922564}%
\pgfsetlinewidth{1.003750pt}%
\definecolor{currentstroke}{rgb}{0.121569,0.466667,0.705882}%
\pgfsetstrokecolor{currentstroke}%
\pgfsetstrokeopacity{0.922564}%
\pgfsetdash{}{0pt}%
\pgfpathmoveto{\pgfqpoint{1.090934in}{1.983906in}}%
\pgfpathcurveto{\pgfqpoint{1.099170in}{1.983906in}}{\pgfqpoint{1.107070in}{1.987178in}}{\pgfqpoint{1.112894in}{1.993002in}}%
\pgfpathcurveto{\pgfqpoint{1.118718in}{1.998826in}}{\pgfqpoint{1.121990in}{2.006726in}}{\pgfqpoint{1.121990in}{2.014963in}}%
\pgfpathcurveto{\pgfqpoint{1.121990in}{2.023199in}}{\pgfqpoint{1.118718in}{2.031099in}}{\pgfqpoint{1.112894in}{2.036923in}}%
\pgfpathcurveto{\pgfqpoint{1.107070in}{2.042747in}}{\pgfqpoint{1.099170in}{2.046019in}}{\pgfqpoint{1.090934in}{2.046019in}}%
\pgfpathcurveto{\pgfqpoint{1.082698in}{2.046019in}}{\pgfqpoint{1.074798in}{2.042747in}}{\pgfqpoint{1.068974in}{2.036923in}}%
\pgfpathcurveto{\pgfqpoint{1.063150in}{2.031099in}}{\pgfqpoint{1.059877in}{2.023199in}}{\pgfqpoint{1.059877in}{2.014963in}}%
\pgfpathcurveto{\pgfqpoint{1.059877in}{2.006726in}}{\pgfqpoint{1.063150in}{1.998826in}}{\pgfqpoint{1.068974in}{1.993002in}}%
\pgfpathcurveto{\pgfqpoint{1.074798in}{1.987178in}}{\pgfqpoint{1.082698in}{1.983906in}}{\pgfqpoint{1.090934in}{1.983906in}}%
\pgfpathclose%
\pgfusepath{stroke,fill}%
\end{pgfscope}%
\begin{pgfscope}%
\pgfpathrectangle{\pgfqpoint{0.100000in}{0.212622in}}{\pgfqpoint{3.696000in}{3.696000in}}%
\pgfusepath{clip}%
\pgfsetbuttcap%
\pgfsetroundjoin%
\definecolor{currentfill}{rgb}{0.121569,0.466667,0.705882}%
\pgfsetfillcolor{currentfill}%
\pgfsetfillopacity{0.922707}%
\pgfsetlinewidth{1.003750pt}%
\definecolor{currentstroke}{rgb}{0.121569,0.466667,0.705882}%
\pgfsetstrokecolor{currentstroke}%
\pgfsetstrokeopacity{0.922707}%
\pgfsetdash{}{0pt}%
\pgfpathmoveto{\pgfqpoint{1.734523in}{2.557367in}}%
\pgfpathcurveto{\pgfqpoint{1.742759in}{2.557367in}}{\pgfqpoint{1.750659in}{2.560640in}}{\pgfqpoint{1.756483in}{2.566464in}}%
\pgfpathcurveto{\pgfqpoint{1.762307in}{2.572288in}}{\pgfqpoint{1.765579in}{2.580188in}}{\pgfqpoint{1.765579in}{2.588424in}}%
\pgfpathcurveto{\pgfqpoint{1.765579in}{2.596660in}}{\pgfqpoint{1.762307in}{2.604560in}}{\pgfqpoint{1.756483in}{2.610384in}}%
\pgfpathcurveto{\pgfqpoint{1.750659in}{2.616208in}}{\pgfqpoint{1.742759in}{2.619480in}}{\pgfqpoint{1.734523in}{2.619480in}}%
\pgfpathcurveto{\pgfqpoint{1.726286in}{2.619480in}}{\pgfqpoint{1.718386in}{2.616208in}}{\pgfqpoint{1.712562in}{2.610384in}}%
\pgfpathcurveto{\pgfqpoint{1.706739in}{2.604560in}}{\pgfqpoint{1.703466in}{2.596660in}}{\pgfqpoint{1.703466in}{2.588424in}}%
\pgfpathcurveto{\pgfqpoint{1.703466in}{2.580188in}}{\pgfqpoint{1.706739in}{2.572288in}}{\pgfqpoint{1.712562in}{2.566464in}}%
\pgfpathcurveto{\pgfqpoint{1.718386in}{2.560640in}}{\pgfqpoint{1.726286in}{2.557367in}}{\pgfqpoint{1.734523in}{2.557367in}}%
\pgfpathclose%
\pgfusepath{stroke,fill}%
\end{pgfscope}%
\begin{pgfscope}%
\pgfpathrectangle{\pgfqpoint{0.100000in}{0.212622in}}{\pgfqpoint{3.696000in}{3.696000in}}%
\pgfusepath{clip}%
\pgfsetbuttcap%
\pgfsetroundjoin%
\definecolor{currentfill}{rgb}{0.121569,0.466667,0.705882}%
\pgfsetfillcolor{currentfill}%
\pgfsetfillopacity{0.922841}%
\pgfsetlinewidth{1.003750pt}%
\definecolor{currentstroke}{rgb}{0.121569,0.466667,0.705882}%
\pgfsetstrokecolor{currentstroke}%
\pgfsetstrokeopacity{0.922841}%
\pgfsetdash{}{0pt}%
\pgfpathmoveto{\pgfqpoint{2.638270in}{1.258876in}}%
\pgfpathcurveto{\pgfqpoint{2.646506in}{1.258876in}}{\pgfqpoint{2.654406in}{1.262148in}}{\pgfqpoint{2.660230in}{1.267972in}}%
\pgfpathcurveto{\pgfqpoint{2.666054in}{1.273796in}}{\pgfqpoint{2.669326in}{1.281696in}}{\pgfqpoint{2.669326in}{1.289932in}}%
\pgfpathcurveto{\pgfqpoint{2.669326in}{1.298169in}}{\pgfqpoint{2.666054in}{1.306069in}}{\pgfqpoint{2.660230in}{1.311893in}}%
\pgfpathcurveto{\pgfqpoint{2.654406in}{1.317717in}}{\pgfqpoint{2.646506in}{1.320989in}}{\pgfqpoint{2.638270in}{1.320989in}}%
\pgfpathcurveto{\pgfqpoint{2.630034in}{1.320989in}}{\pgfqpoint{2.622134in}{1.317717in}}{\pgfqpoint{2.616310in}{1.311893in}}%
\pgfpathcurveto{\pgfqpoint{2.610486in}{1.306069in}}{\pgfqpoint{2.607213in}{1.298169in}}{\pgfqpoint{2.607213in}{1.289932in}}%
\pgfpathcurveto{\pgfqpoint{2.607213in}{1.281696in}}{\pgfqpoint{2.610486in}{1.273796in}}{\pgfqpoint{2.616310in}{1.267972in}}%
\pgfpathcurveto{\pgfqpoint{2.622134in}{1.262148in}}{\pgfqpoint{2.630034in}{1.258876in}}{\pgfqpoint{2.638270in}{1.258876in}}%
\pgfpathclose%
\pgfusepath{stroke,fill}%
\end{pgfscope}%
\begin{pgfscope}%
\pgfpathrectangle{\pgfqpoint{0.100000in}{0.212622in}}{\pgfqpoint{3.696000in}{3.696000in}}%
\pgfusepath{clip}%
\pgfsetbuttcap%
\pgfsetroundjoin%
\definecolor{currentfill}{rgb}{0.121569,0.466667,0.705882}%
\pgfsetfillcolor{currentfill}%
\pgfsetfillopacity{0.923062}%
\pgfsetlinewidth{1.003750pt}%
\definecolor{currentstroke}{rgb}{0.121569,0.466667,0.705882}%
\pgfsetstrokecolor{currentstroke}%
\pgfsetstrokeopacity{0.923062}%
\pgfsetdash{}{0pt}%
\pgfpathmoveto{\pgfqpoint{2.143678in}{2.460565in}}%
\pgfpathcurveto{\pgfqpoint{2.151914in}{2.460565in}}{\pgfqpoint{2.159814in}{2.463838in}}{\pgfqpoint{2.165638in}{2.469662in}}%
\pgfpathcurveto{\pgfqpoint{2.171462in}{2.475486in}}{\pgfqpoint{2.174734in}{2.483386in}}{\pgfqpoint{2.174734in}{2.491622in}}%
\pgfpathcurveto{\pgfqpoint{2.174734in}{2.499858in}}{\pgfqpoint{2.171462in}{2.507758in}}{\pgfqpoint{2.165638in}{2.513582in}}%
\pgfpathcurveto{\pgfqpoint{2.159814in}{2.519406in}}{\pgfqpoint{2.151914in}{2.522678in}}{\pgfqpoint{2.143678in}{2.522678in}}%
\pgfpathcurveto{\pgfqpoint{2.135441in}{2.522678in}}{\pgfqpoint{2.127541in}{2.519406in}}{\pgfqpoint{2.121717in}{2.513582in}}%
\pgfpathcurveto{\pgfqpoint{2.115894in}{2.507758in}}{\pgfqpoint{2.112621in}{2.499858in}}{\pgfqpoint{2.112621in}{2.491622in}}%
\pgfpathcurveto{\pgfqpoint{2.112621in}{2.483386in}}{\pgfqpoint{2.115894in}{2.475486in}}{\pgfqpoint{2.121717in}{2.469662in}}%
\pgfpathcurveto{\pgfqpoint{2.127541in}{2.463838in}}{\pgfqpoint{2.135441in}{2.460565in}}{\pgfqpoint{2.143678in}{2.460565in}}%
\pgfpathclose%
\pgfusepath{stroke,fill}%
\end{pgfscope}%
\begin{pgfscope}%
\pgfpathrectangle{\pgfqpoint{0.100000in}{0.212622in}}{\pgfqpoint{3.696000in}{3.696000in}}%
\pgfusepath{clip}%
\pgfsetbuttcap%
\pgfsetroundjoin%
\definecolor{currentfill}{rgb}{0.121569,0.466667,0.705882}%
\pgfsetfillcolor{currentfill}%
\pgfsetfillopacity{0.923211}%
\pgfsetlinewidth{1.003750pt}%
\definecolor{currentstroke}{rgb}{0.121569,0.466667,0.705882}%
\pgfsetstrokecolor{currentstroke}%
\pgfsetstrokeopacity{0.923211}%
\pgfsetdash{}{0pt}%
\pgfpathmoveto{\pgfqpoint{1.094850in}{1.981016in}}%
\pgfpathcurveto{\pgfqpoint{1.103086in}{1.981016in}}{\pgfqpoint{1.110986in}{1.984288in}}{\pgfqpoint{1.116810in}{1.990112in}}%
\pgfpathcurveto{\pgfqpoint{1.122634in}{1.995936in}}{\pgfqpoint{1.125906in}{2.003836in}}{\pgfqpoint{1.125906in}{2.012072in}}%
\pgfpathcurveto{\pgfqpoint{1.125906in}{2.020308in}}{\pgfqpoint{1.122634in}{2.028209in}}{\pgfqpoint{1.116810in}{2.034032in}}%
\pgfpathcurveto{\pgfqpoint{1.110986in}{2.039856in}}{\pgfqpoint{1.103086in}{2.043129in}}{\pgfqpoint{1.094850in}{2.043129in}}%
\pgfpathcurveto{\pgfqpoint{1.086614in}{2.043129in}}{\pgfqpoint{1.078714in}{2.039856in}}{\pgfqpoint{1.072890in}{2.034032in}}%
\pgfpathcurveto{\pgfqpoint{1.067066in}{2.028209in}}{\pgfqpoint{1.063793in}{2.020308in}}{\pgfqpoint{1.063793in}{2.012072in}}%
\pgfpathcurveto{\pgfqpoint{1.063793in}{2.003836in}}{\pgfqpoint{1.067066in}{1.995936in}}{\pgfqpoint{1.072890in}{1.990112in}}%
\pgfpathcurveto{\pgfqpoint{1.078714in}{1.984288in}}{\pgfqpoint{1.086614in}{1.981016in}}{\pgfqpoint{1.094850in}{1.981016in}}%
\pgfpathclose%
\pgfusepath{stroke,fill}%
\end{pgfscope}%
\begin{pgfscope}%
\pgfpathrectangle{\pgfqpoint{0.100000in}{0.212622in}}{\pgfqpoint{3.696000in}{3.696000in}}%
\pgfusepath{clip}%
\pgfsetbuttcap%
\pgfsetroundjoin%
\definecolor{currentfill}{rgb}{0.121569,0.466667,0.705882}%
\pgfsetfillcolor{currentfill}%
\pgfsetfillopacity{0.923705}%
\pgfsetlinewidth{1.003750pt}%
\definecolor{currentstroke}{rgb}{0.121569,0.466667,0.705882}%
\pgfsetstrokecolor{currentstroke}%
\pgfsetstrokeopacity{0.923705}%
\pgfsetdash{}{0pt}%
\pgfpathmoveto{\pgfqpoint{2.142449in}{2.459323in}}%
\pgfpathcurveto{\pgfqpoint{2.150685in}{2.459323in}}{\pgfqpoint{2.158585in}{2.462595in}}{\pgfqpoint{2.164409in}{2.468419in}}%
\pgfpathcurveto{\pgfqpoint{2.170233in}{2.474243in}}{\pgfqpoint{2.173505in}{2.482143in}}{\pgfqpoint{2.173505in}{2.490379in}}%
\pgfpathcurveto{\pgfqpoint{2.173505in}{2.498615in}}{\pgfqpoint{2.170233in}{2.506515in}}{\pgfqpoint{2.164409in}{2.512339in}}%
\pgfpathcurveto{\pgfqpoint{2.158585in}{2.518163in}}{\pgfqpoint{2.150685in}{2.521436in}}{\pgfqpoint{2.142449in}{2.521436in}}%
\pgfpathcurveto{\pgfqpoint{2.134213in}{2.521436in}}{\pgfqpoint{2.126313in}{2.518163in}}{\pgfqpoint{2.120489in}{2.512339in}}%
\pgfpathcurveto{\pgfqpoint{2.114665in}{2.506515in}}{\pgfqpoint{2.111392in}{2.498615in}}{\pgfqpoint{2.111392in}{2.490379in}}%
\pgfpathcurveto{\pgfqpoint{2.111392in}{2.482143in}}{\pgfqpoint{2.114665in}{2.474243in}}{\pgfqpoint{2.120489in}{2.468419in}}%
\pgfpathcurveto{\pgfqpoint{2.126313in}{2.462595in}}{\pgfqpoint{2.134213in}{2.459323in}}{\pgfqpoint{2.142449in}{2.459323in}}%
\pgfpathclose%
\pgfusepath{stroke,fill}%
\end{pgfscope}%
\begin{pgfscope}%
\pgfpathrectangle{\pgfqpoint{0.100000in}{0.212622in}}{\pgfqpoint{3.696000in}{3.696000in}}%
\pgfusepath{clip}%
\pgfsetbuttcap%
\pgfsetroundjoin%
\definecolor{currentfill}{rgb}{0.121569,0.466667,0.705882}%
\pgfsetfillcolor{currentfill}%
\pgfsetfillopacity{0.924039}%
\pgfsetlinewidth{1.003750pt}%
\definecolor{currentstroke}{rgb}{0.121569,0.466667,0.705882}%
\pgfsetstrokecolor{currentstroke}%
\pgfsetstrokeopacity{0.924039}%
\pgfsetdash{}{0pt}%
\pgfpathmoveto{\pgfqpoint{1.099753in}{1.977691in}}%
\pgfpathcurveto{\pgfqpoint{1.107989in}{1.977691in}}{\pgfqpoint{1.115889in}{1.980963in}}{\pgfqpoint{1.121713in}{1.986787in}}%
\pgfpathcurveto{\pgfqpoint{1.127537in}{1.992611in}}{\pgfqpoint{1.130810in}{2.000511in}}{\pgfqpoint{1.130810in}{2.008747in}}%
\pgfpathcurveto{\pgfqpoint{1.130810in}{2.016984in}}{\pgfqpoint{1.127537in}{2.024884in}}{\pgfqpoint{1.121713in}{2.030707in}}%
\pgfpathcurveto{\pgfqpoint{1.115889in}{2.036531in}}{\pgfqpoint{1.107989in}{2.039804in}}{\pgfqpoint{1.099753in}{2.039804in}}%
\pgfpathcurveto{\pgfqpoint{1.091517in}{2.039804in}}{\pgfqpoint{1.083617in}{2.036531in}}{\pgfqpoint{1.077793in}{2.030707in}}%
\pgfpathcurveto{\pgfqpoint{1.071969in}{2.024884in}}{\pgfqpoint{1.068697in}{2.016984in}}{\pgfqpoint{1.068697in}{2.008747in}}%
\pgfpathcurveto{\pgfqpoint{1.068697in}{2.000511in}}{\pgfqpoint{1.071969in}{1.992611in}}{\pgfqpoint{1.077793in}{1.986787in}}%
\pgfpathcurveto{\pgfqpoint{1.083617in}{1.980963in}}{\pgfqpoint{1.091517in}{1.977691in}}{\pgfqpoint{1.099753in}{1.977691in}}%
\pgfpathclose%
\pgfusepath{stroke,fill}%
\end{pgfscope}%
\begin{pgfscope}%
\pgfpathrectangle{\pgfqpoint{0.100000in}{0.212622in}}{\pgfqpoint{3.696000in}{3.696000in}}%
\pgfusepath{clip}%
\pgfsetbuttcap%
\pgfsetroundjoin%
\definecolor{currentfill}{rgb}{0.121569,0.466667,0.705882}%
\pgfsetfillcolor{currentfill}%
\pgfsetfillopacity{0.924470}%
\pgfsetlinewidth{1.003750pt}%
\definecolor{currentstroke}{rgb}{0.121569,0.466667,0.705882}%
\pgfsetstrokecolor{currentstroke}%
\pgfsetstrokeopacity{0.924470}%
\pgfsetdash{}{0pt}%
\pgfpathmoveto{\pgfqpoint{1.737134in}{2.553013in}}%
\pgfpathcurveto{\pgfqpoint{1.745370in}{2.553013in}}{\pgfqpoint{1.753270in}{2.556286in}}{\pgfqpoint{1.759094in}{2.562110in}}%
\pgfpathcurveto{\pgfqpoint{1.764918in}{2.567934in}}{\pgfqpoint{1.768190in}{2.575834in}}{\pgfqpoint{1.768190in}{2.584070in}}%
\pgfpathcurveto{\pgfqpoint{1.768190in}{2.592306in}}{\pgfqpoint{1.764918in}{2.600206in}}{\pgfqpoint{1.759094in}{2.606030in}}%
\pgfpathcurveto{\pgfqpoint{1.753270in}{2.611854in}}{\pgfqpoint{1.745370in}{2.615126in}}{\pgfqpoint{1.737134in}{2.615126in}}%
\pgfpathcurveto{\pgfqpoint{1.728897in}{2.615126in}}{\pgfqpoint{1.720997in}{2.611854in}}{\pgfqpoint{1.715173in}{2.606030in}}%
\pgfpathcurveto{\pgfqpoint{1.709349in}{2.600206in}}{\pgfqpoint{1.706077in}{2.592306in}}{\pgfqpoint{1.706077in}{2.584070in}}%
\pgfpathcurveto{\pgfqpoint{1.706077in}{2.575834in}}{\pgfqpoint{1.709349in}{2.567934in}}{\pgfqpoint{1.715173in}{2.562110in}}%
\pgfpathcurveto{\pgfqpoint{1.720997in}{2.556286in}}{\pgfqpoint{1.728897in}{2.553013in}}{\pgfqpoint{1.737134in}{2.553013in}}%
\pgfpathclose%
\pgfusepath{stroke,fill}%
\end{pgfscope}%
\begin{pgfscope}%
\pgfpathrectangle{\pgfqpoint{0.100000in}{0.212622in}}{\pgfqpoint{3.696000in}{3.696000in}}%
\pgfusepath{clip}%
\pgfsetbuttcap%
\pgfsetroundjoin%
\definecolor{currentfill}{rgb}{0.121569,0.466667,0.705882}%
\pgfsetfillcolor{currentfill}%
\pgfsetfillopacity{0.924867}%
\pgfsetlinewidth{1.003750pt}%
\definecolor{currentstroke}{rgb}{0.121569,0.466667,0.705882}%
\pgfsetstrokecolor{currentstroke}%
\pgfsetstrokeopacity{0.924867}%
\pgfsetdash{}{0pt}%
\pgfpathmoveto{\pgfqpoint{1.105180in}{1.974044in}}%
\pgfpathcurveto{\pgfqpoint{1.113417in}{1.974044in}}{\pgfqpoint{1.121317in}{1.977316in}}{\pgfqpoint{1.127141in}{1.983140in}}%
\pgfpathcurveto{\pgfqpoint{1.132964in}{1.988964in}}{\pgfqpoint{1.136237in}{1.996864in}}{\pgfqpoint{1.136237in}{2.005100in}}%
\pgfpathcurveto{\pgfqpoint{1.136237in}{2.013336in}}{\pgfqpoint{1.132964in}{2.021236in}}{\pgfqpoint{1.127141in}{2.027060in}}%
\pgfpathcurveto{\pgfqpoint{1.121317in}{2.032884in}}{\pgfqpoint{1.113417in}{2.036157in}}{\pgfqpoint{1.105180in}{2.036157in}}%
\pgfpathcurveto{\pgfqpoint{1.096944in}{2.036157in}}{\pgfqpoint{1.089044in}{2.032884in}}{\pgfqpoint{1.083220in}{2.027060in}}%
\pgfpathcurveto{\pgfqpoint{1.077396in}{2.021236in}}{\pgfqpoint{1.074124in}{2.013336in}}{\pgfqpoint{1.074124in}{2.005100in}}%
\pgfpathcurveto{\pgfqpoint{1.074124in}{1.996864in}}{\pgfqpoint{1.077396in}{1.988964in}}{\pgfqpoint{1.083220in}{1.983140in}}%
\pgfpathcurveto{\pgfqpoint{1.089044in}{1.977316in}}{\pgfqpoint{1.096944in}{1.974044in}}{\pgfqpoint{1.105180in}{1.974044in}}%
\pgfpathclose%
\pgfusepath{stroke,fill}%
\end{pgfscope}%
\begin{pgfscope}%
\pgfpathrectangle{\pgfqpoint{0.100000in}{0.212622in}}{\pgfqpoint{3.696000in}{3.696000in}}%
\pgfusepath{clip}%
\pgfsetbuttcap%
\pgfsetroundjoin%
\definecolor{currentfill}{rgb}{0.121569,0.466667,0.705882}%
\pgfsetfillcolor{currentfill}%
\pgfsetfillopacity{0.924880}%
\pgfsetlinewidth{1.003750pt}%
\definecolor{currentstroke}{rgb}{0.121569,0.466667,0.705882}%
\pgfsetstrokecolor{currentstroke}%
\pgfsetstrokeopacity{0.924880}%
\pgfsetdash{}{0pt}%
\pgfpathmoveto{\pgfqpoint{2.140253in}{2.457067in}}%
\pgfpathcurveto{\pgfqpoint{2.148489in}{2.457067in}}{\pgfqpoint{2.156389in}{2.460339in}}{\pgfqpoint{2.162213in}{2.466163in}}%
\pgfpathcurveto{\pgfqpoint{2.168037in}{2.471987in}}{\pgfqpoint{2.171309in}{2.479887in}}{\pgfqpoint{2.171309in}{2.488123in}}%
\pgfpathcurveto{\pgfqpoint{2.171309in}{2.496359in}}{\pgfqpoint{2.168037in}{2.504259in}}{\pgfqpoint{2.162213in}{2.510083in}}%
\pgfpathcurveto{\pgfqpoint{2.156389in}{2.515907in}}{\pgfqpoint{2.148489in}{2.519180in}}{\pgfqpoint{2.140253in}{2.519180in}}%
\pgfpathcurveto{\pgfqpoint{2.132017in}{2.519180in}}{\pgfqpoint{2.124117in}{2.515907in}}{\pgfqpoint{2.118293in}{2.510083in}}%
\pgfpathcurveto{\pgfqpoint{2.112469in}{2.504259in}}{\pgfqpoint{2.109196in}{2.496359in}}{\pgfqpoint{2.109196in}{2.488123in}}%
\pgfpathcurveto{\pgfqpoint{2.109196in}{2.479887in}}{\pgfqpoint{2.112469in}{2.471987in}}{\pgfqpoint{2.118293in}{2.466163in}}%
\pgfpathcurveto{\pgfqpoint{2.124117in}{2.460339in}}{\pgfqpoint{2.132017in}{2.457067in}}{\pgfqpoint{2.140253in}{2.457067in}}%
\pgfpathclose%
\pgfusepath{stroke,fill}%
\end{pgfscope}%
\begin{pgfscope}%
\pgfpathrectangle{\pgfqpoint{0.100000in}{0.212622in}}{\pgfqpoint{3.696000in}{3.696000in}}%
\pgfusepath{clip}%
\pgfsetbuttcap%
\pgfsetroundjoin%
\definecolor{currentfill}{rgb}{0.121569,0.466667,0.705882}%
\pgfsetfillcolor{currentfill}%
\pgfsetfillopacity{0.925318}%
\pgfsetlinewidth{1.003750pt}%
\definecolor{currentstroke}{rgb}{0.121569,0.466667,0.705882}%
\pgfsetstrokecolor{currentstroke}%
\pgfsetstrokeopacity{0.925318}%
\pgfsetdash{}{0pt}%
\pgfpathmoveto{\pgfqpoint{1.108153in}{1.971970in}}%
\pgfpathcurveto{\pgfqpoint{1.116389in}{1.971970in}}{\pgfqpoint{1.124289in}{1.975242in}}{\pgfqpoint{1.130113in}{1.981066in}}%
\pgfpathcurveto{\pgfqpoint{1.135937in}{1.986890in}}{\pgfqpoint{1.139210in}{1.994790in}}{\pgfqpoint{1.139210in}{2.003026in}}%
\pgfpathcurveto{\pgfqpoint{1.139210in}{2.011263in}}{\pgfqpoint{1.135937in}{2.019163in}}{\pgfqpoint{1.130113in}{2.024987in}}%
\pgfpathcurveto{\pgfqpoint{1.124289in}{2.030811in}}{\pgfqpoint{1.116389in}{2.034083in}}{\pgfqpoint{1.108153in}{2.034083in}}%
\pgfpathcurveto{\pgfqpoint{1.099917in}{2.034083in}}{\pgfqpoint{1.092017in}{2.030811in}}{\pgfqpoint{1.086193in}{2.024987in}}%
\pgfpathcurveto{\pgfqpoint{1.080369in}{2.019163in}}{\pgfqpoint{1.077097in}{2.011263in}}{\pgfqpoint{1.077097in}{2.003026in}}%
\pgfpathcurveto{\pgfqpoint{1.077097in}{1.994790in}}{\pgfqpoint{1.080369in}{1.986890in}}{\pgfqpoint{1.086193in}{1.981066in}}%
\pgfpathcurveto{\pgfqpoint{1.092017in}{1.975242in}}{\pgfqpoint{1.099917in}{1.971970in}}{\pgfqpoint{1.108153in}{1.971970in}}%
\pgfpathclose%
\pgfusepath{stroke,fill}%
\end{pgfscope}%
\begin{pgfscope}%
\pgfpathrectangle{\pgfqpoint{0.100000in}{0.212622in}}{\pgfqpoint{3.696000in}{3.696000in}}%
\pgfusepath{clip}%
\pgfsetbuttcap%
\pgfsetroundjoin%
\definecolor{currentfill}{rgb}{0.121569,0.466667,0.705882}%
\pgfsetfillcolor{currentfill}%
\pgfsetfillopacity{0.925634}%
\pgfsetlinewidth{1.003750pt}%
\definecolor{currentstroke}{rgb}{0.121569,0.466667,0.705882}%
\pgfsetstrokecolor{currentstroke}%
\pgfsetstrokeopacity{0.925634}%
\pgfsetdash{}{0pt}%
\pgfpathmoveto{\pgfqpoint{2.138766in}{2.455386in}}%
\pgfpathcurveto{\pgfqpoint{2.147002in}{2.455386in}}{\pgfqpoint{2.154902in}{2.458659in}}{\pgfqpoint{2.160726in}{2.464483in}}%
\pgfpathcurveto{\pgfqpoint{2.166550in}{2.470306in}}{\pgfqpoint{2.169822in}{2.478206in}}{\pgfqpoint{2.169822in}{2.486443in}}%
\pgfpathcurveto{\pgfqpoint{2.169822in}{2.494679in}}{\pgfqpoint{2.166550in}{2.502579in}}{\pgfqpoint{2.160726in}{2.508403in}}%
\pgfpathcurveto{\pgfqpoint{2.154902in}{2.514227in}}{\pgfqpoint{2.147002in}{2.517499in}}{\pgfqpoint{2.138766in}{2.517499in}}%
\pgfpathcurveto{\pgfqpoint{2.130530in}{2.517499in}}{\pgfqpoint{2.122630in}{2.514227in}}{\pgfqpoint{2.116806in}{2.508403in}}%
\pgfpathcurveto{\pgfqpoint{2.110982in}{2.502579in}}{\pgfqpoint{2.107709in}{2.494679in}}{\pgfqpoint{2.107709in}{2.486443in}}%
\pgfpathcurveto{\pgfqpoint{2.107709in}{2.478206in}}{\pgfqpoint{2.110982in}{2.470306in}}{\pgfqpoint{2.116806in}{2.464483in}}%
\pgfpathcurveto{\pgfqpoint{2.122630in}{2.458659in}}{\pgfqpoint{2.130530in}{2.455386in}}{\pgfqpoint{2.138766in}{2.455386in}}%
\pgfpathclose%
\pgfusepath{stroke,fill}%
\end{pgfscope}%
\begin{pgfscope}%
\pgfpathrectangle{\pgfqpoint{0.100000in}{0.212622in}}{\pgfqpoint{3.696000in}{3.696000in}}%
\pgfusepath{clip}%
\pgfsetbuttcap%
\pgfsetroundjoin%
\definecolor{currentfill}{rgb}{0.121569,0.466667,0.705882}%
\pgfsetfillcolor{currentfill}%
\pgfsetfillopacity{0.925640}%
\pgfsetlinewidth{1.003750pt}%
\definecolor{currentstroke}{rgb}{0.121569,0.466667,0.705882}%
\pgfsetstrokecolor{currentstroke}%
\pgfsetstrokeopacity{0.925640}%
\pgfsetdash{}{0pt}%
\pgfpathmoveto{\pgfqpoint{2.633083in}{1.252413in}}%
\pgfpathcurveto{\pgfqpoint{2.641319in}{1.252413in}}{\pgfqpoint{2.649219in}{1.255686in}}{\pgfqpoint{2.655043in}{1.261510in}}%
\pgfpathcurveto{\pgfqpoint{2.660867in}{1.267333in}}{\pgfqpoint{2.664139in}{1.275233in}}{\pgfqpoint{2.664139in}{1.283470in}}%
\pgfpathcurveto{\pgfqpoint{2.664139in}{1.291706in}}{\pgfqpoint{2.660867in}{1.299606in}}{\pgfqpoint{2.655043in}{1.305430in}}%
\pgfpathcurveto{\pgfqpoint{2.649219in}{1.311254in}}{\pgfqpoint{2.641319in}{1.314526in}}{\pgfqpoint{2.633083in}{1.314526in}}%
\pgfpathcurveto{\pgfqpoint{2.624847in}{1.314526in}}{\pgfqpoint{2.616947in}{1.311254in}}{\pgfqpoint{2.611123in}{1.305430in}}%
\pgfpathcurveto{\pgfqpoint{2.605299in}{1.299606in}}{\pgfqpoint{2.602026in}{1.291706in}}{\pgfqpoint{2.602026in}{1.283470in}}%
\pgfpathcurveto{\pgfqpoint{2.602026in}{1.275233in}}{\pgfqpoint{2.605299in}{1.267333in}}{\pgfqpoint{2.611123in}{1.261510in}}%
\pgfpathcurveto{\pgfqpoint{2.616947in}{1.255686in}}{\pgfqpoint{2.624847in}{1.252413in}}{\pgfqpoint{2.633083in}{1.252413in}}%
\pgfpathclose%
\pgfusepath{stroke,fill}%
\end{pgfscope}%
\begin{pgfscope}%
\pgfpathrectangle{\pgfqpoint{0.100000in}{0.212622in}}{\pgfqpoint{3.696000in}{3.696000in}}%
\pgfusepath{clip}%
\pgfsetbuttcap%
\pgfsetroundjoin%
\definecolor{currentfill}{rgb}{0.121569,0.466667,0.705882}%
\pgfsetfillcolor{currentfill}%
\pgfsetfillopacity{0.925921}%
\pgfsetlinewidth{1.003750pt}%
\definecolor{currentstroke}{rgb}{0.121569,0.466667,0.705882}%
\pgfsetstrokecolor{currentstroke}%
\pgfsetstrokeopacity{0.925921}%
\pgfsetdash{}{0pt}%
\pgfpathmoveto{\pgfqpoint{1.111844in}{1.969305in}}%
\pgfpathcurveto{\pgfqpoint{1.120080in}{1.969305in}}{\pgfqpoint{1.127980in}{1.972577in}}{\pgfqpoint{1.133804in}{1.978401in}}%
\pgfpathcurveto{\pgfqpoint{1.139628in}{1.984225in}}{\pgfqpoint{1.142900in}{1.992125in}}{\pgfqpoint{1.142900in}{2.000361in}}%
\pgfpathcurveto{\pgfqpoint{1.142900in}{2.008597in}}{\pgfqpoint{1.139628in}{2.016497in}}{\pgfqpoint{1.133804in}{2.022321in}}%
\pgfpathcurveto{\pgfqpoint{1.127980in}{2.028145in}}{\pgfqpoint{1.120080in}{2.031418in}}{\pgfqpoint{1.111844in}{2.031418in}}%
\pgfpathcurveto{\pgfqpoint{1.103608in}{2.031418in}}{\pgfqpoint{1.095708in}{2.028145in}}{\pgfqpoint{1.089884in}{2.022321in}}%
\pgfpathcurveto{\pgfqpoint{1.084060in}{2.016497in}}{\pgfqpoint{1.080787in}{2.008597in}}{\pgfqpoint{1.080787in}{2.000361in}}%
\pgfpathcurveto{\pgfqpoint{1.080787in}{1.992125in}}{\pgfqpoint{1.084060in}{1.984225in}}{\pgfqpoint{1.089884in}{1.978401in}}%
\pgfpathcurveto{\pgfqpoint{1.095708in}{1.972577in}}{\pgfqpoint{1.103608in}{1.969305in}}{\pgfqpoint{1.111844in}{1.969305in}}%
\pgfpathclose%
\pgfusepath{stroke,fill}%
\end{pgfscope}%
\begin{pgfscope}%
\pgfpathrectangle{\pgfqpoint{0.100000in}{0.212622in}}{\pgfqpoint{3.696000in}{3.696000in}}%
\pgfusepath{clip}%
\pgfsetbuttcap%
\pgfsetroundjoin%
\definecolor{currentfill}{rgb}{0.121569,0.466667,0.705882}%
\pgfsetfillcolor{currentfill}%
\pgfsetfillopacity{0.926180}%
\pgfsetlinewidth{1.003750pt}%
\definecolor{currentstroke}{rgb}{0.121569,0.466667,0.705882}%
\pgfsetstrokecolor{currentstroke}%
\pgfsetstrokeopacity{0.926180}%
\pgfsetdash{}{0pt}%
\pgfpathmoveto{\pgfqpoint{1.740639in}{2.548718in}}%
\pgfpathcurveto{\pgfqpoint{1.748875in}{2.548718in}}{\pgfqpoint{1.756775in}{2.551991in}}{\pgfqpoint{1.762599in}{2.557814in}}%
\pgfpathcurveto{\pgfqpoint{1.768423in}{2.563638in}}{\pgfqpoint{1.771695in}{2.571538in}}{\pgfqpoint{1.771695in}{2.579775in}}%
\pgfpathcurveto{\pgfqpoint{1.771695in}{2.588011in}}{\pgfqpoint{1.768423in}{2.595911in}}{\pgfqpoint{1.762599in}{2.601735in}}%
\pgfpathcurveto{\pgfqpoint{1.756775in}{2.607559in}}{\pgfqpoint{1.748875in}{2.610831in}}{\pgfqpoint{1.740639in}{2.610831in}}%
\pgfpathcurveto{\pgfqpoint{1.732402in}{2.610831in}}{\pgfqpoint{1.724502in}{2.607559in}}{\pgfqpoint{1.718678in}{2.601735in}}%
\pgfpathcurveto{\pgfqpoint{1.712854in}{2.595911in}}{\pgfqpoint{1.709582in}{2.588011in}}{\pgfqpoint{1.709582in}{2.579775in}}%
\pgfpathcurveto{\pgfqpoint{1.709582in}{2.571538in}}{\pgfqpoint{1.712854in}{2.563638in}}{\pgfqpoint{1.718678in}{2.557814in}}%
\pgfpathcurveto{\pgfqpoint{1.724502in}{2.551991in}}{\pgfqpoint{1.732402in}{2.548718in}}{\pgfqpoint{1.740639in}{2.548718in}}%
\pgfpathclose%
\pgfusepath{stroke,fill}%
\end{pgfscope}%
\begin{pgfscope}%
\pgfpathrectangle{\pgfqpoint{0.100000in}{0.212622in}}{\pgfqpoint{3.696000in}{3.696000in}}%
\pgfusepath{clip}%
\pgfsetbuttcap%
\pgfsetroundjoin%
\definecolor{currentfill}{rgb}{0.121569,0.466667,0.705882}%
\pgfsetfillcolor{currentfill}%
\pgfsetfillopacity{0.926196}%
\pgfsetlinewidth{1.003750pt}%
\definecolor{currentstroke}{rgb}{0.121569,0.466667,0.705882}%
\pgfsetstrokecolor{currentstroke}%
\pgfsetstrokeopacity{0.926196}%
\pgfsetdash{}{0pt}%
\pgfpathmoveto{\pgfqpoint{2.137632in}{2.454127in}}%
\pgfpathcurveto{\pgfqpoint{2.145868in}{2.454127in}}{\pgfqpoint{2.153768in}{2.457400in}}{\pgfqpoint{2.159592in}{2.463224in}}%
\pgfpathcurveto{\pgfqpoint{2.165416in}{2.469048in}}{\pgfqpoint{2.168688in}{2.476948in}}{\pgfqpoint{2.168688in}{2.485184in}}%
\pgfpathcurveto{\pgfqpoint{2.168688in}{2.493420in}}{\pgfqpoint{2.165416in}{2.501320in}}{\pgfqpoint{2.159592in}{2.507144in}}%
\pgfpathcurveto{\pgfqpoint{2.153768in}{2.512968in}}{\pgfqpoint{2.145868in}{2.516240in}}{\pgfqpoint{2.137632in}{2.516240in}}%
\pgfpathcurveto{\pgfqpoint{2.129395in}{2.516240in}}{\pgfqpoint{2.121495in}{2.512968in}}{\pgfqpoint{2.115671in}{2.507144in}}%
\pgfpathcurveto{\pgfqpoint{2.109847in}{2.501320in}}{\pgfqpoint{2.106575in}{2.493420in}}{\pgfqpoint{2.106575in}{2.485184in}}%
\pgfpathcurveto{\pgfqpoint{2.106575in}{2.476948in}}{\pgfqpoint{2.109847in}{2.469048in}}{\pgfqpoint{2.115671in}{2.463224in}}%
\pgfpathcurveto{\pgfqpoint{2.121495in}{2.457400in}}{\pgfqpoint{2.129395in}{2.454127in}}{\pgfqpoint{2.137632in}{2.454127in}}%
\pgfpathclose%
\pgfusepath{stroke,fill}%
\end{pgfscope}%
\begin{pgfscope}%
\pgfpathrectangle{\pgfqpoint{0.100000in}{0.212622in}}{\pgfqpoint{3.696000in}{3.696000in}}%
\pgfusepath{clip}%
\pgfsetbuttcap%
\pgfsetroundjoin%
\definecolor{currentfill}{rgb}{0.121569,0.466667,0.705882}%
\pgfsetfillcolor{currentfill}%
\pgfsetfillopacity{0.926917}%
\pgfsetlinewidth{1.003750pt}%
\definecolor{currentstroke}{rgb}{0.121569,0.466667,0.705882}%
\pgfsetstrokecolor{currentstroke}%
\pgfsetstrokeopacity{0.926917}%
\pgfsetdash{}{0pt}%
\pgfpathmoveto{\pgfqpoint{1.117774in}{1.965076in}}%
\pgfpathcurveto{\pgfqpoint{1.126011in}{1.965076in}}{\pgfqpoint{1.133911in}{1.968349in}}{\pgfqpoint{1.139735in}{1.974173in}}%
\pgfpathcurveto{\pgfqpoint{1.145559in}{1.979996in}}{\pgfqpoint{1.148831in}{1.987897in}}{\pgfqpoint{1.148831in}{1.996133in}}%
\pgfpathcurveto{\pgfqpoint{1.148831in}{2.004369in}}{\pgfqpoint{1.145559in}{2.012269in}}{\pgfqpoint{1.139735in}{2.018093in}}%
\pgfpathcurveto{\pgfqpoint{1.133911in}{2.023917in}}{\pgfqpoint{1.126011in}{2.027189in}}{\pgfqpoint{1.117774in}{2.027189in}}%
\pgfpathcurveto{\pgfqpoint{1.109538in}{2.027189in}}{\pgfqpoint{1.101638in}{2.023917in}}{\pgfqpoint{1.095814in}{2.018093in}}%
\pgfpathcurveto{\pgfqpoint{1.089990in}{2.012269in}}{\pgfqpoint{1.086718in}{2.004369in}}{\pgfqpoint{1.086718in}{1.996133in}}%
\pgfpathcurveto{\pgfqpoint{1.086718in}{1.987897in}}{\pgfqpoint{1.089990in}{1.979996in}}{\pgfqpoint{1.095814in}{1.974173in}}%
\pgfpathcurveto{\pgfqpoint{1.101638in}{1.968349in}}{\pgfqpoint{1.109538in}{1.965076in}}{\pgfqpoint{1.117774in}{1.965076in}}%
\pgfpathclose%
\pgfusepath{stroke,fill}%
\end{pgfscope}%
\begin{pgfscope}%
\pgfpathrectangle{\pgfqpoint{0.100000in}{0.212622in}}{\pgfqpoint{3.696000in}{3.696000in}}%
\pgfusepath{clip}%
\pgfsetbuttcap%
\pgfsetroundjoin%
\definecolor{currentfill}{rgb}{0.121569,0.466667,0.705882}%
\pgfsetfillcolor{currentfill}%
\pgfsetfillopacity{0.927028}%
\pgfsetlinewidth{1.003750pt}%
\definecolor{currentstroke}{rgb}{0.121569,0.466667,0.705882}%
\pgfsetstrokecolor{currentstroke}%
\pgfsetstrokeopacity{0.927028}%
\pgfsetdash{}{0pt}%
\pgfpathmoveto{\pgfqpoint{1.742966in}{2.546601in}}%
\pgfpathcurveto{\pgfqpoint{1.751202in}{2.546601in}}{\pgfqpoint{1.759102in}{2.549874in}}{\pgfqpoint{1.764926in}{2.555698in}}%
\pgfpathcurveto{\pgfqpoint{1.770750in}{2.561522in}}{\pgfqpoint{1.774023in}{2.569422in}}{\pgfqpoint{1.774023in}{2.577658in}}%
\pgfpathcurveto{\pgfqpoint{1.774023in}{2.585894in}}{\pgfqpoint{1.770750in}{2.593794in}}{\pgfqpoint{1.764926in}{2.599618in}}%
\pgfpathcurveto{\pgfqpoint{1.759102in}{2.605442in}}{\pgfqpoint{1.751202in}{2.608714in}}{\pgfqpoint{1.742966in}{2.608714in}}%
\pgfpathcurveto{\pgfqpoint{1.734730in}{2.608714in}}{\pgfqpoint{1.726830in}{2.605442in}}{\pgfqpoint{1.721006in}{2.599618in}}%
\pgfpathcurveto{\pgfqpoint{1.715182in}{2.593794in}}{\pgfqpoint{1.711910in}{2.585894in}}{\pgfqpoint{1.711910in}{2.577658in}}%
\pgfpathcurveto{\pgfqpoint{1.711910in}{2.569422in}}{\pgfqpoint{1.715182in}{2.561522in}}{\pgfqpoint{1.721006in}{2.555698in}}%
\pgfpathcurveto{\pgfqpoint{1.726830in}{2.549874in}}{\pgfqpoint{1.734730in}{2.546601in}}{\pgfqpoint{1.742966in}{2.546601in}}%
\pgfpathclose%
\pgfusepath{stroke,fill}%
\end{pgfscope}%
\begin{pgfscope}%
\pgfpathrectangle{\pgfqpoint{0.100000in}{0.212622in}}{\pgfqpoint{3.696000in}{3.696000in}}%
\pgfusepath{clip}%
\pgfsetbuttcap%
\pgfsetroundjoin%
\definecolor{currentfill}{rgb}{0.121569,0.466667,0.705882}%
\pgfsetfillcolor{currentfill}%
\pgfsetfillopacity{0.927243}%
\pgfsetlinewidth{1.003750pt}%
\definecolor{currentstroke}{rgb}{0.121569,0.466667,0.705882}%
\pgfsetstrokecolor{currentstroke}%
\pgfsetstrokeopacity{0.927243}%
\pgfsetdash{}{0pt}%
\pgfpathmoveto{\pgfqpoint{2.135579in}{2.451971in}}%
\pgfpathcurveto{\pgfqpoint{2.143815in}{2.451971in}}{\pgfqpoint{2.151715in}{2.455244in}}{\pgfqpoint{2.157539in}{2.461067in}}%
\pgfpathcurveto{\pgfqpoint{2.163363in}{2.466891in}}{\pgfqpoint{2.166636in}{2.474791in}}{\pgfqpoint{2.166636in}{2.483028in}}%
\pgfpathcurveto{\pgfqpoint{2.166636in}{2.491264in}}{\pgfqpoint{2.163363in}{2.499164in}}{\pgfqpoint{2.157539in}{2.504988in}}%
\pgfpathcurveto{\pgfqpoint{2.151715in}{2.510812in}}{\pgfqpoint{2.143815in}{2.514084in}}{\pgfqpoint{2.135579in}{2.514084in}}%
\pgfpathcurveto{\pgfqpoint{2.127343in}{2.514084in}}{\pgfqpoint{2.119443in}{2.510812in}}{\pgfqpoint{2.113619in}{2.504988in}}%
\pgfpathcurveto{\pgfqpoint{2.107795in}{2.499164in}}{\pgfqpoint{2.104523in}{2.491264in}}{\pgfqpoint{2.104523in}{2.483028in}}%
\pgfpathcurveto{\pgfqpoint{2.104523in}{2.474791in}}{\pgfqpoint{2.107795in}{2.466891in}}{\pgfqpoint{2.113619in}{2.461067in}}%
\pgfpathcurveto{\pgfqpoint{2.119443in}{2.455244in}}{\pgfqpoint{2.127343in}{2.451971in}}{\pgfqpoint{2.135579in}{2.451971in}}%
\pgfpathclose%
\pgfusepath{stroke,fill}%
\end{pgfscope}%
\begin{pgfscope}%
\pgfpathrectangle{\pgfqpoint{0.100000in}{0.212622in}}{\pgfqpoint{3.696000in}{3.696000in}}%
\pgfusepath{clip}%
\pgfsetbuttcap%
\pgfsetroundjoin%
\definecolor{currentfill}{rgb}{0.121569,0.466667,0.705882}%
\pgfsetfillcolor{currentfill}%
\pgfsetfillopacity{0.927431}%
\pgfsetlinewidth{1.003750pt}%
\definecolor{currentstroke}{rgb}{0.121569,0.466667,0.705882}%
\pgfsetstrokecolor{currentstroke}%
\pgfsetstrokeopacity{0.927431}%
\pgfsetdash{}{0pt}%
\pgfpathmoveto{\pgfqpoint{1.744453in}{2.545531in}}%
\pgfpathcurveto{\pgfqpoint{1.752690in}{2.545531in}}{\pgfqpoint{1.760590in}{2.548804in}}{\pgfqpoint{1.766414in}{2.554628in}}%
\pgfpathcurveto{\pgfqpoint{1.772238in}{2.560452in}}{\pgfqpoint{1.775510in}{2.568352in}}{\pgfqpoint{1.775510in}{2.576588in}}%
\pgfpathcurveto{\pgfqpoint{1.775510in}{2.584824in}}{\pgfqpoint{1.772238in}{2.592724in}}{\pgfqpoint{1.766414in}{2.598548in}}%
\pgfpathcurveto{\pgfqpoint{1.760590in}{2.604372in}}{\pgfqpoint{1.752690in}{2.607644in}}{\pgfqpoint{1.744453in}{2.607644in}}%
\pgfpathcurveto{\pgfqpoint{1.736217in}{2.607644in}}{\pgfqpoint{1.728317in}{2.604372in}}{\pgfqpoint{1.722493in}{2.598548in}}%
\pgfpathcurveto{\pgfqpoint{1.716669in}{2.592724in}}{\pgfqpoint{1.713397in}{2.584824in}}{\pgfqpoint{1.713397in}{2.576588in}}%
\pgfpathcurveto{\pgfqpoint{1.713397in}{2.568352in}}{\pgfqpoint{1.716669in}{2.560452in}}{\pgfqpoint{1.722493in}{2.554628in}}%
\pgfpathcurveto{\pgfqpoint{1.728317in}{2.548804in}}{\pgfqpoint{1.736217in}{2.545531in}}{\pgfqpoint{1.744453in}{2.545531in}}%
\pgfpathclose%
\pgfusepath{stroke,fill}%
\end{pgfscope}%
\begin{pgfscope}%
\pgfpathrectangle{\pgfqpoint{0.100000in}{0.212622in}}{\pgfqpoint{3.696000in}{3.696000in}}%
\pgfusepath{clip}%
\pgfsetbuttcap%
\pgfsetroundjoin%
\definecolor{currentfill}{rgb}{0.121569,0.466667,0.705882}%
\pgfsetfillcolor{currentfill}%
\pgfsetfillopacity{0.927622}%
\pgfsetlinewidth{1.003750pt}%
\definecolor{currentstroke}{rgb}{0.121569,0.466667,0.705882}%
\pgfsetstrokecolor{currentstroke}%
\pgfsetstrokeopacity{0.927622}%
\pgfsetdash{}{0pt}%
\pgfpathmoveto{\pgfqpoint{1.745357in}{2.544984in}}%
\pgfpathcurveto{\pgfqpoint{1.753593in}{2.544984in}}{\pgfqpoint{1.761493in}{2.548256in}}{\pgfqpoint{1.767317in}{2.554080in}}%
\pgfpathcurveto{\pgfqpoint{1.773141in}{2.559904in}}{\pgfqpoint{1.776413in}{2.567804in}}{\pgfqpoint{1.776413in}{2.576040in}}%
\pgfpathcurveto{\pgfqpoint{1.776413in}{2.584277in}}{\pgfqpoint{1.773141in}{2.592177in}}{\pgfqpoint{1.767317in}{2.598001in}}%
\pgfpathcurveto{\pgfqpoint{1.761493in}{2.603825in}}{\pgfqpoint{1.753593in}{2.607097in}}{\pgfqpoint{1.745357in}{2.607097in}}%
\pgfpathcurveto{\pgfqpoint{1.737120in}{2.607097in}}{\pgfqpoint{1.729220in}{2.603825in}}{\pgfqpoint{1.723396in}{2.598001in}}%
\pgfpathcurveto{\pgfqpoint{1.717572in}{2.592177in}}{\pgfqpoint{1.714300in}{2.584277in}}{\pgfqpoint{1.714300in}{2.576040in}}%
\pgfpathcurveto{\pgfqpoint{1.714300in}{2.567804in}}{\pgfqpoint{1.717572in}{2.559904in}}{\pgfqpoint{1.723396in}{2.554080in}}%
\pgfpathcurveto{\pgfqpoint{1.729220in}{2.548256in}}{\pgfqpoint{1.737120in}{2.544984in}}{\pgfqpoint{1.745357in}{2.544984in}}%
\pgfpathclose%
\pgfusepath{stroke,fill}%
\end{pgfscope}%
\begin{pgfscope}%
\pgfpathrectangle{\pgfqpoint{0.100000in}{0.212622in}}{\pgfqpoint{3.696000in}{3.696000in}}%
\pgfusepath{clip}%
\pgfsetbuttcap%
\pgfsetroundjoin%
\definecolor{currentfill}{rgb}{0.121569,0.466667,0.705882}%
\pgfsetfillcolor{currentfill}%
\pgfsetfillopacity{0.927669}%
\pgfsetlinewidth{1.003750pt}%
\definecolor{currentstroke}{rgb}{0.121569,0.466667,0.705882}%
\pgfsetstrokecolor{currentstroke}%
\pgfsetstrokeopacity{0.927669}%
\pgfsetdash{}{0pt}%
\pgfpathmoveto{\pgfqpoint{2.134710in}{2.450947in}}%
\pgfpathcurveto{\pgfqpoint{2.142946in}{2.450947in}}{\pgfqpoint{2.150846in}{2.454219in}}{\pgfqpoint{2.156670in}{2.460043in}}%
\pgfpathcurveto{\pgfqpoint{2.162494in}{2.465867in}}{\pgfqpoint{2.165767in}{2.473767in}}{\pgfqpoint{2.165767in}{2.482003in}}%
\pgfpathcurveto{\pgfqpoint{2.165767in}{2.490240in}}{\pgfqpoint{2.162494in}{2.498140in}}{\pgfqpoint{2.156670in}{2.503964in}}%
\pgfpathcurveto{\pgfqpoint{2.150846in}{2.509788in}}{\pgfqpoint{2.142946in}{2.513060in}}{\pgfqpoint{2.134710in}{2.513060in}}%
\pgfpathcurveto{\pgfqpoint{2.126474in}{2.513060in}}{\pgfqpoint{2.118574in}{2.509788in}}{\pgfqpoint{2.112750in}{2.503964in}}%
\pgfpathcurveto{\pgfqpoint{2.106926in}{2.498140in}}{\pgfqpoint{2.103654in}{2.490240in}}{\pgfqpoint{2.103654in}{2.482003in}}%
\pgfpathcurveto{\pgfqpoint{2.103654in}{2.473767in}}{\pgfqpoint{2.106926in}{2.465867in}}{\pgfqpoint{2.112750in}{2.460043in}}%
\pgfpathcurveto{\pgfqpoint{2.118574in}{2.454219in}}{\pgfqpoint{2.126474in}{2.450947in}}{\pgfqpoint{2.134710in}{2.450947in}}%
\pgfpathclose%
\pgfusepath{stroke,fill}%
\end{pgfscope}%
\begin{pgfscope}%
\pgfpathrectangle{\pgfqpoint{0.100000in}{0.212622in}}{\pgfqpoint{3.696000in}{3.696000in}}%
\pgfusepath{clip}%
\pgfsetbuttcap%
\pgfsetroundjoin%
\definecolor{currentfill}{rgb}{0.121569,0.466667,0.705882}%
\pgfsetfillcolor{currentfill}%
\pgfsetfillopacity{0.927729}%
\pgfsetlinewidth{1.003750pt}%
\definecolor{currentstroke}{rgb}{0.121569,0.466667,0.705882}%
\pgfsetstrokecolor{currentstroke}%
\pgfsetstrokeopacity{0.927729}%
\pgfsetdash{}{0pt}%
\pgfpathmoveto{\pgfqpoint{1.745842in}{2.544661in}}%
\pgfpathcurveto{\pgfqpoint{1.754079in}{2.544661in}}{\pgfqpoint{1.761979in}{2.547934in}}{\pgfqpoint{1.767803in}{2.553757in}}%
\pgfpathcurveto{\pgfqpoint{1.773627in}{2.559581in}}{\pgfqpoint{1.776899in}{2.567481in}}{\pgfqpoint{1.776899in}{2.575718in}}%
\pgfpathcurveto{\pgfqpoint{1.776899in}{2.583954in}}{\pgfqpoint{1.773627in}{2.591854in}}{\pgfqpoint{1.767803in}{2.597678in}}%
\pgfpathcurveto{\pgfqpoint{1.761979in}{2.603502in}}{\pgfqpoint{1.754079in}{2.606774in}}{\pgfqpoint{1.745842in}{2.606774in}}%
\pgfpathcurveto{\pgfqpoint{1.737606in}{2.606774in}}{\pgfqpoint{1.729706in}{2.603502in}}{\pgfqpoint{1.723882in}{2.597678in}}%
\pgfpathcurveto{\pgfqpoint{1.718058in}{2.591854in}}{\pgfqpoint{1.714786in}{2.583954in}}{\pgfqpoint{1.714786in}{2.575718in}}%
\pgfpathcurveto{\pgfqpoint{1.714786in}{2.567481in}}{\pgfqpoint{1.718058in}{2.559581in}}{\pgfqpoint{1.723882in}{2.553757in}}%
\pgfpathcurveto{\pgfqpoint{1.729706in}{2.547934in}}{\pgfqpoint{1.737606in}{2.544661in}}{\pgfqpoint{1.745842in}{2.544661in}}%
\pgfpathclose%
\pgfusepath{stroke,fill}%
\end{pgfscope}%
\begin{pgfscope}%
\pgfpathrectangle{\pgfqpoint{0.100000in}{0.212622in}}{\pgfqpoint{3.696000in}{3.696000in}}%
\pgfusepath{clip}%
\pgfsetbuttcap%
\pgfsetroundjoin%
\definecolor{currentfill}{rgb}{0.121569,0.466667,0.705882}%
\pgfsetfillcolor{currentfill}%
\pgfsetfillopacity{0.927916}%
\pgfsetlinewidth{1.003750pt}%
\definecolor{currentstroke}{rgb}{0.121569,0.466667,0.705882}%
\pgfsetstrokecolor{currentstroke}%
\pgfsetstrokeopacity{0.927916}%
\pgfsetdash{}{0pt}%
\pgfpathmoveto{\pgfqpoint{1.746898in}{2.543953in}}%
\pgfpathcurveto{\pgfqpoint{1.755134in}{2.543953in}}{\pgfqpoint{1.763034in}{2.547226in}}{\pgfqpoint{1.768858in}{2.553049in}}%
\pgfpathcurveto{\pgfqpoint{1.774682in}{2.558873in}}{\pgfqpoint{1.777954in}{2.566773in}}{\pgfqpoint{1.777954in}{2.575010in}}%
\pgfpathcurveto{\pgfqpoint{1.777954in}{2.583246in}}{\pgfqpoint{1.774682in}{2.591146in}}{\pgfqpoint{1.768858in}{2.596970in}}%
\pgfpathcurveto{\pgfqpoint{1.763034in}{2.602794in}}{\pgfqpoint{1.755134in}{2.606066in}}{\pgfqpoint{1.746898in}{2.606066in}}%
\pgfpathcurveto{\pgfqpoint{1.738661in}{2.606066in}}{\pgfqpoint{1.730761in}{2.602794in}}{\pgfqpoint{1.724937in}{2.596970in}}%
\pgfpathcurveto{\pgfqpoint{1.719113in}{2.591146in}}{\pgfqpoint{1.715841in}{2.583246in}}{\pgfqpoint{1.715841in}{2.575010in}}%
\pgfpathcurveto{\pgfqpoint{1.715841in}{2.566773in}}{\pgfqpoint{1.719113in}{2.558873in}}{\pgfqpoint{1.724937in}{2.553049in}}%
\pgfpathcurveto{\pgfqpoint{1.730761in}{2.547226in}}{\pgfqpoint{1.738661in}{2.543953in}}{\pgfqpoint{1.746898in}{2.543953in}}%
\pgfpathclose%
\pgfusepath{stroke,fill}%
\end{pgfscope}%
\begin{pgfscope}%
\pgfpathrectangle{\pgfqpoint{0.100000in}{0.212622in}}{\pgfqpoint{3.696000in}{3.696000in}}%
\pgfusepath{clip}%
\pgfsetbuttcap%
\pgfsetroundjoin%
\definecolor{currentfill}{rgb}{0.121569,0.466667,0.705882}%
\pgfsetfillcolor{currentfill}%
\pgfsetfillopacity{0.927932}%
\pgfsetlinewidth{1.003750pt}%
\definecolor{currentstroke}{rgb}{0.121569,0.466667,0.705882}%
\pgfsetstrokecolor{currentstroke}%
\pgfsetstrokeopacity{0.927932}%
\pgfsetdash{}{0pt}%
\pgfpathmoveto{\pgfqpoint{1.125613in}{1.959647in}}%
\pgfpathcurveto{\pgfqpoint{1.133849in}{1.959647in}}{\pgfqpoint{1.141749in}{1.962919in}}{\pgfqpoint{1.147573in}{1.968743in}}%
\pgfpathcurveto{\pgfqpoint{1.153397in}{1.974567in}}{\pgfqpoint{1.156669in}{1.982467in}}{\pgfqpoint{1.156669in}{1.990704in}}%
\pgfpathcurveto{\pgfqpoint{1.156669in}{1.998940in}}{\pgfqpoint{1.153397in}{2.006840in}}{\pgfqpoint{1.147573in}{2.012664in}}%
\pgfpathcurveto{\pgfqpoint{1.141749in}{2.018488in}}{\pgfqpoint{1.133849in}{2.021760in}}{\pgfqpoint{1.125613in}{2.021760in}}%
\pgfpathcurveto{\pgfqpoint{1.117376in}{2.021760in}}{\pgfqpoint{1.109476in}{2.018488in}}{\pgfqpoint{1.103652in}{2.012664in}}%
\pgfpathcurveto{\pgfqpoint{1.097828in}{2.006840in}}{\pgfqpoint{1.094556in}{1.998940in}}{\pgfqpoint{1.094556in}{1.990704in}}%
\pgfpathcurveto{\pgfqpoint{1.094556in}{1.982467in}}{\pgfqpoint{1.097828in}{1.974567in}}{\pgfqpoint{1.103652in}{1.968743in}}%
\pgfpathcurveto{\pgfqpoint{1.109476in}{1.962919in}}{\pgfqpoint{1.117376in}{1.959647in}}{\pgfqpoint{1.125613in}{1.959647in}}%
\pgfpathclose%
\pgfusepath{stroke,fill}%
\end{pgfscope}%
\begin{pgfscope}%
\pgfpathrectangle{\pgfqpoint{0.100000in}{0.212622in}}{\pgfqpoint{3.696000in}{3.696000in}}%
\pgfusepath{clip}%
\pgfsetbuttcap%
\pgfsetroundjoin%
\definecolor{currentfill}{rgb}{0.121569,0.466667,0.705882}%
\pgfsetfillcolor{currentfill}%
\pgfsetfillopacity{0.928126}%
\pgfsetlinewidth{1.003750pt}%
\definecolor{currentstroke}{rgb}{0.121569,0.466667,0.705882}%
\pgfsetstrokecolor{currentstroke}%
\pgfsetstrokeopacity{0.928126}%
\pgfsetdash{}{0pt}%
\pgfpathmoveto{\pgfqpoint{1.748405in}{2.543051in}}%
\pgfpathcurveto{\pgfqpoint{1.756641in}{2.543051in}}{\pgfqpoint{1.764541in}{2.546323in}}{\pgfqpoint{1.770365in}{2.552147in}}%
\pgfpathcurveto{\pgfqpoint{1.776189in}{2.557971in}}{\pgfqpoint{1.779461in}{2.565871in}}{\pgfqpoint{1.779461in}{2.574108in}}%
\pgfpathcurveto{\pgfqpoint{1.779461in}{2.582344in}}{\pgfqpoint{1.776189in}{2.590244in}}{\pgfqpoint{1.770365in}{2.596068in}}%
\pgfpathcurveto{\pgfqpoint{1.764541in}{2.601892in}}{\pgfqpoint{1.756641in}{2.605164in}}{\pgfqpoint{1.748405in}{2.605164in}}%
\pgfpathcurveto{\pgfqpoint{1.740169in}{2.605164in}}{\pgfqpoint{1.732269in}{2.601892in}}{\pgfqpoint{1.726445in}{2.596068in}}%
\pgfpathcurveto{\pgfqpoint{1.720621in}{2.590244in}}{\pgfqpoint{1.717348in}{2.582344in}}{\pgfqpoint{1.717348in}{2.574108in}}%
\pgfpathcurveto{\pgfqpoint{1.717348in}{2.565871in}}{\pgfqpoint{1.720621in}{2.557971in}}{\pgfqpoint{1.726445in}{2.552147in}}%
\pgfpathcurveto{\pgfqpoint{1.732269in}{2.546323in}}{\pgfqpoint{1.740169in}{2.543051in}}{\pgfqpoint{1.748405in}{2.543051in}}%
\pgfpathclose%
\pgfusepath{stroke,fill}%
\end{pgfscope}%
\begin{pgfscope}%
\pgfpathrectangle{\pgfqpoint{0.100000in}{0.212622in}}{\pgfqpoint{3.696000in}{3.696000in}}%
\pgfusepath{clip}%
\pgfsetbuttcap%
\pgfsetroundjoin%
\definecolor{currentfill}{rgb}{0.121569,0.466667,0.705882}%
\pgfsetfillcolor{currentfill}%
\pgfsetfillopacity{0.928377}%
\pgfsetlinewidth{1.003750pt}%
\definecolor{currentstroke}{rgb}{0.121569,0.466667,0.705882}%
\pgfsetstrokecolor{currentstroke}%
\pgfsetstrokeopacity{0.928377}%
\pgfsetdash{}{0pt}%
\pgfpathmoveto{\pgfqpoint{1.750389in}{2.541906in}}%
\pgfpathcurveto{\pgfqpoint{1.758626in}{2.541906in}}{\pgfqpoint{1.766526in}{2.545178in}}{\pgfqpoint{1.772349in}{2.551002in}}%
\pgfpathcurveto{\pgfqpoint{1.778173in}{2.556826in}}{\pgfqpoint{1.781446in}{2.564726in}}{\pgfqpoint{1.781446in}{2.572962in}}%
\pgfpathcurveto{\pgfqpoint{1.781446in}{2.581199in}}{\pgfqpoint{1.778173in}{2.589099in}}{\pgfqpoint{1.772349in}{2.594923in}}%
\pgfpathcurveto{\pgfqpoint{1.766526in}{2.600747in}}{\pgfqpoint{1.758626in}{2.604019in}}{\pgfqpoint{1.750389in}{2.604019in}}%
\pgfpathcurveto{\pgfqpoint{1.742153in}{2.604019in}}{\pgfqpoint{1.734253in}{2.600747in}}{\pgfqpoint{1.728429in}{2.594923in}}%
\pgfpathcurveto{\pgfqpoint{1.722605in}{2.589099in}}{\pgfqpoint{1.719333in}{2.581199in}}{\pgfqpoint{1.719333in}{2.572962in}}%
\pgfpathcurveto{\pgfqpoint{1.719333in}{2.564726in}}{\pgfqpoint{1.722605in}{2.556826in}}{\pgfqpoint{1.728429in}{2.551002in}}%
\pgfpathcurveto{\pgfqpoint{1.734253in}{2.545178in}}{\pgfqpoint{1.742153in}{2.541906in}}{\pgfqpoint{1.750389in}{2.541906in}}%
\pgfpathclose%
\pgfusepath{stroke,fill}%
\end{pgfscope}%
\begin{pgfscope}%
\pgfpathrectangle{\pgfqpoint{0.100000in}{0.212622in}}{\pgfqpoint{3.696000in}{3.696000in}}%
\pgfusepath{clip}%
\pgfsetbuttcap%
\pgfsetroundjoin%
\definecolor{currentfill}{rgb}{0.121569,0.466667,0.705882}%
\pgfsetfillcolor{currentfill}%
\pgfsetfillopacity{0.928379}%
\pgfsetlinewidth{1.003750pt}%
\definecolor{currentstroke}{rgb}{0.121569,0.466667,0.705882}%
\pgfsetstrokecolor{currentstroke}%
\pgfsetstrokeopacity{0.928379}%
\pgfsetdash{}{0pt}%
\pgfpathmoveto{\pgfqpoint{2.627890in}{1.246528in}}%
\pgfpathcurveto{\pgfqpoint{2.636126in}{1.246528in}}{\pgfqpoint{2.644026in}{1.249800in}}{\pgfqpoint{2.649850in}{1.255624in}}%
\pgfpathcurveto{\pgfqpoint{2.655674in}{1.261448in}}{\pgfqpoint{2.658946in}{1.269348in}}{\pgfqpoint{2.658946in}{1.277584in}}%
\pgfpathcurveto{\pgfqpoint{2.658946in}{1.285820in}}{\pgfqpoint{2.655674in}{1.293720in}}{\pgfqpoint{2.649850in}{1.299544in}}%
\pgfpathcurveto{\pgfqpoint{2.644026in}{1.305368in}}{\pgfqpoint{2.636126in}{1.308641in}}{\pgfqpoint{2.627890in}{1.308641in}}%
\pgfpathcurveto{\pgfqpoint{2.619653in}{1.308641in}}{\pgfqpoint{2.611753in}{1.305368in}}{\pgfqpoint{2.605929in}{1.299544in}}%
\pgfpathcurveto{\pgfqpoint{2.600105in}{1.293720in}}{\pgfqpoint{2.596833in}{1.285820in}}{\pgfqpoint{2.596833in}{1.277584in}}%
\pgfpathcurveto{\pgfqpoint{2.596833in}{1.269348in}}{\pgfqpoint{2.600105in}{1.261448in}}{\pgfqpoint{2.605929in}{1.255624in}}%
\pgfpathcurveto{\pgfqpoint{2.611753in}{1.249800in}}{\pgfqpoint{2.619653in}{1.246528in}}{\pgfqpoint{2.627890in}{1.246528in}}%
\pgfpathclose%
\pgfusepath{stroke,fill}%
\end{pgfscope}%
\begin{pgfscope}%
\pgfpathrectangle{\pgfqpoint{0.100000in}{0.212622in}}{\pgfqpoint{3.696000in}{3.696000in}}%
\pgfusepath{clip}%
\pgfsetbuttcap%
\pgfsetroundjoin%
\definecolor{currentfill}{rgb}{0.121569,0.466667,0.705882}%
\pgfsetfillcolor{currentfill}%
\pgfsetfillopacity{0.928435}%
\pgfsetlinewidth{1.003750pt}%
\definecolor{currentstroke}{rgb}{0.121569,0.466667,0.705882}%
\pgfsetstrokecolor{currentstroke}%
\pgfsetstrokeopacity{0.928435}%
\pgfsetdash{}{0pt}%
\pgfpathmoveto{\pgfqpoint{2.133073in}{2.449095in}}%
\pgfpathcurveto{\pgfqpoint{2.141309in}{2.449095in}}{\pgfqpoint{2.149209in}{2.452368in}}{\pgfqpoint{2.155033in}{2.458191in}}%
\pgfpathcurveto{\pgfqpoint{2.160857in}{2.464015in}}{\pgfqpoint{2.164129in}{2.471915in}}{\pgfqpoint{2.164129in}{2.480152in}}%
\pgfpathcurveto{\pgfqpoint{2.164129in}{2.488388in}}{\pgfqpoint{2.160857in}{2.496288in}}{\pgfqpoint{2.155033in}{2.502112in}}%
\pgfpathcurveto{\pgfqpoint{2.149209in}{2.507936in}}{\pgfqpoint{2.141309in}{2.511208in}}{\pgfqpoint{2.133073in}{2.511208in}}%
\pgfpathcurveto{\pgfqpoint{2.124836in}{2.511208in}}{\pgfqpoint{2.116936in}{2.507936in}}{\pgfqpoint{2.111112in}{2.502112in}}%
\pgfpathcurveto{\pgfqpoint{2.105288in}{2.496288in}}{\pgfqpoint{2.102016in}{2.488388in}}{\pgfqpoint{2.102016in}{2.480152in}}%
\pgfpathcurveto{\pgfqpoint{2.102016in}{2.471915in}}{\pgfqpoint{2.105288in}{2.464015in}}{\pgfqpoint{2.111112in}{2.458191in}}%
\pgfpathcurveto{\pgfqpoint{2.116936in}{2.452368in}}{\pgfqpoint{2.124836in}{2.449095in}}{\pgfqpoint{2.133073in}{2.449095in}}%
\pgfpathclose%
\pgfusepath{stroke,fill}%
\end{pgfscope}%
\begin{pgfscope}%
\pgfpathrectangle{\pgfqpoint{0.100000in}{0.212622in}}{\pgfqpoint{3.696000in}{3.696000in}}%
\pgfusepath{clip}%
\pgfsetbuttcap%
\pgfsetroundjoin%
\definecolor{currentfill}{rgb}{0.121569,0.466667,0.705882}%
\pgfsetfillcolor{currentfill}%
\pgfsetfillopacity{0.928682}%
\pgfsetlinewidth{1.003750pt}%
\definecolor{currentstroke}{rgb}{0.121569,0.466667,0.705882}%
\pgfsetstrokecolor{currentstroke}%
\pgfsetstrokeopacity{0.928682}%
\pgfsetdash{}{0pt}%
\pgfpathmoveto{\pgfqpoint{1.753133in}{2.540667in}}%
\pgfpathcurveto{\pgfqpoint{1.761369in}{2.540667in}}{\pgfqpoint{1.769269in}{2.543939in}}{\pgfqpoint{1.775093in}{2.549763in}}%
\pgfpathcurveto{\pgfqpoint{1.780917in}{2.555587in}}{\pgfqpoint{1.784190in}{2.563487in}}{\pgfqpoint{1.784190in}{2.571723in}}%
\pgfpathcurveto{\pgfqpoint{1.784190in}{2.579960in}}{\pgfqpoint{1.780917in}{2.587860in}}{\pgfqpoint{1.775093in}{2.593683in}}%
\pgfpathcurveto{\pgfqpoint{1.769269in}{2.599507in}}{\pgfqpoint{1.761369in}{2.602780in}}{\pgfqpoint{1.753133in}{2.602780in}}%
\pgfpathcurveto{\pgfqpoint{1.744897in}{2.602780in}}{\pgfqpoint{1.736997in}{2.599507in}}{\pgfqpoint{1.731173in}{2.593683in}}%
\pgfpathcurveto{\pgfqpoint{1.725349in}{2.587860in}}{\pgfqpoint{1.722077in}{2.579960in}}{\pgfqpoint{1.722077in}{2.571723in}}%
\pgfpathcurveto{\pgfqpoint{1.722077in}{2.563487in}}{\pgfqpoint{1.725349in}{2.555587in}}{\pgfqpoint{1.731173in}{2.549763in}}%
\pgfpathcurveto{\pgfqpoint{1.736997in}{2.543939in}}{\pgfqpoint{1.744897in}{2.540667in}}{\pgfqpoint{1.753133in}{2.540667in}}%
\pgfpathclose%
\pgfusepath{stroke,fill}%
\end{pgfscope}%
\begin{pgfscope}%
\pgfpathrectangle{\pgfqpoint{0.100000in}{0.212622in}}{\pgfqpoint{3.696000in}{3.696000in}}%
\pgfusepath{clip}%
\pgfsetbuttcap%
\pgfsetroundjoin%
\definecolor{currentfill}{rgb}{0.121569,0.466667,0.705882}%
\pgfsetfillcolor{currentfill}%
\pgfsetfillopacity{0.928755}%
\pgfsetlinewidth{1.003750pt}%
\definecolor{currentstroke}{rgb}{0.121569,0.466667,0.705882}%
\pgfsetstrokecolor{currentstroke}%
\pgfsetstrokeopacity{0.928755}%
\pgfsetdash{}{0pt}%
\pgfpathmoveto{\pgfqpoint{2.132407in}{2.448395in}}%
\pgfpathcurveto{\pgfqpoint{2.140644in}{2.448395in}}{\pgfqpoint{2.148544in}{2.451667in}}{\pgfqpoint{2.154368in}{2.457491in}}%
\pgfpathcurveto{\pgfqpoint{2.160192in}{2.463315in}}{\pgfqpoint{2.163464in}{2.471215in}}{\pgfqpoint{2.163464in}{2.479451in}}%
\pgfpathcurveto{\pgfqpoint{2.163464in}{2.487688in}}{\pgfqpoint{2.160192in}{2.495588in}}{\pgfqpoint{2.154368in}{2.501412in}}%
\pgfpathcurveto{\pgfqpoint{2.148544in}{2.507236in}}{\pgfqpoint{2.140644in}{2.510508in}}{\pgfqpoint{2.132407in}{2.510508in}}%
\pgfpathcurveto{\pgfqpoint{2.124171in}{2.510508in}}{\pgfqpoint{2.116271in}{2.507236in}}{\pgfqpoint{2.110447in}{2.501412in}}%
\pgfpathcurveto{\pgfqpoint{2.104623in}{2.495588in}}{\pgfqpoint{2.101351in}{2.487688in}}{\pgfqpoint{2.101351in}{2.479451in}}%
\pgfpathcurveto{\pgfqpoint{2.101351in}{2.471215in}}{\pgfqpoint{2.104623in}{2.463315in}}{\pgfqpoint{2.110447in}{2.457491in}}%
\pgfpathcurveto{\pgfqpoint{2.116271in}{2.451667in}}{\pgfqpoint{2.124171in}{2.448395in}}{\pgfqpoint{2.132407in}{2.448395in}}%
\pgfpathclose%
\pgfusepath{stroke,fill}%
\end{pgfscope}%
\begin{pgfscope}%
\pgfpathrectangle{\pgfqpoint{0.100000in}{0.212622in}}{\pgfqpoint{3.696000in}{3.696000in}}%
\pgfusepath{clip}%
\pgfsetbuttcap%
\pgfsetroundjoin%
\definecolor{currentfill}{rgb}{0.121569,0.466667,0.705882}%
\pgfsetfillcolor{currentfill}%
\pgfsetfillopacity{0.928887}%
\pgfsetlinewidth{1.003750pt}%
\definecolor{currentstroke}{rgb}{0.121569,0.466667,0.705882}%
\pgfsetstrokecolor{currentstroke}%
\pgfsetstrokeopacity{0.928887}%
\pgfsetdash{}{0pt}%
\pgfpathmoveto{\pgfqpoint{1.756611in}{2.538841in}}%
\pgfpathcurveto{\pgfqpoint{1.764847in}{2.538841in}}{\pgfqpoint{1.772747in}{2.542114in}}{\pgfqpoint{1.778571in}{2.547938in}}%
\pgfpathcurveto{\pgfqpoint{1.784395in}{2.553761in}}{\pgfqpoint{1.787668in}{2.561662in}}{\pgfqpoint{1.787668in}{2.569898in}}%
\pgfpathcurveto{\pgfqpoint{1.787668in}{2.578134in}}{\pgfqpoint{1.784395in}{2.586034in}}{\pgfqpoint{1.778571in}{2.591858in}}%
\pgfpathcurveto{\pgfqpoint{1.772747in}{2.597682in}}{\pgfqpoint{1.764847in}{2.600954in}}{\pgfqpoint{1.756611in}{2.600954in}}%
\pgfpathcurveto{\pgfqpoint{1.748375in}{2.600954in}}{\pgfqpoint{1.740475in}{2.597682in}}{\pgfqpoint{1.734651in}{2.591858in}}%
\pgfpathcurveto{\pgfqpoint{1.728827in}{2.586034in}}{\pgfqpoint{1.725555in}{2.578134in}}{\pgfqpoint{1.725555in}{2.569898in}}%
\pgfpathcurveto{\pgfqpoint{1.725555in}{2.561662in}}{\pgfqpoint{1.728827in}{2.553761in}}{\pgfqpoint{1.734651in}{2.547938in}}%
\pgfpathcurveto{\pgfqpoint{1.740475in}{2.542114in}}{\pgfqpoint{1.748375in}{2.538841in}}{\pgfqpoint{1.756611in}{2.538841in}}%
\pgfpathclose%
\pgfusepath{stroke,fill}%
\end{pgfscope}%
\begin{pgfscope}%
\pgfpathrectangle{\pgfqpoint{0.100000in}{0.212622in}}{\pgfqpoint{3.696000in}{3.696000in}}%
\pgfusepath{clip}%
\pgfsetbuttcap%
\pgfsetroundjoin%
\definecolor{currentfill}{rgb}{0.121569,0.466667,0.705882}%
\pgfsetfillcolor{currentfill}%
\pgfsetfillopacity{0.929019}%
\pgfsetlinewidth{1.003750pt}%
\definecolor{currentstroke}{rgb}{0.121569,0.466667,0.705882}%
\pgfsetstrokecolor{currentstroke}%
\pgfsetstrokeopacity{0.929019}%
\pgfsetdash{}{0pt}%
\pgfpathmoveto{\pgfqpoint{1.134160in}{1.954008in}}%
\pgfpathcurveto{\pgfqpoint{1.142397in}{1.954008in}}{\pgfqpoint{1.150297in}{1.957281in}}{\pgfqpoint{1.156121in}{1.963105in}}%
\pgfpathcurveto{\pgfqpoint{1.161945in}{1.968928in}}{\pgfqpoint{1.165217in}{1.976828in}}{\pgfqpoint{1.165217in}{1.985065in}}%
\pgfpathcurveto{\pgfqpoint{1.165217in}{1.993301in}}{\pgfqpoint{1.161945in}{2.001201in}}{\pgfqpoint{1.156121in}{2.007025in}}%
\pgfpathcurveto{\pgfqpoint{1.150297in}{2.012849in}}{\pgfqpoint{1.142397in}{2.016121in}}{\pgfqpoint{1.134160in}{2.016121in}}%
\pgfpathcurveto{\pgfqpoint{1.125924in}{2.016121in}}{\pgfqpoint{1.118024in}{2.012849in}}{\pgfqpoint{1.112200in}{2.007025in}}%
\pgfpathcurveto{\pgfqpoint{1.106376in}{2.001201in}}{\pgfqpoint{1.103104in}{1.993301in}}{\pgfqpoint{1.103104in}{1.985065in}}%
\pgfpathcurveto{\pgfqpoint{1.103104in}{1.976828in}}{\pgfqpoint{1.106376in}{1.968928in}}{\pgfqpoint{1.112200in}{1.963105in}}%
\pgfpathcurveto{\pgfqpoint{1.118024in}{1.957281in}}{\pgfqpoint{1.125924in}{1.954008in}}{\pgfqpoint{1.134160in}{1.954008in}}%
\pgfpathclose%
\pgfusepath{stroke,fill}%
\end{pgfscope}%
\begin{pgfscope}%
\pgfpathrectangle{\pgfqpoint{0.100000in}{0.212622in}}{\pgfqpoint{3.696000in}{3.696000in}}%
\pgfusepath{clip}%
\pgfsetbuttcap%
\pgfsetroundjoin%
\definecolor{currentfill}{rgb}{0.121569,0.466667,0.705882}%
\pgfsetfillcolor{currentfill}%
\pgfsetfillopacity{0.929207}%
\pgfsetlinewidth{1.003750pt}%
\definecolor{currentstroke}{rgb}{0.121569,0.466667,0.705882}%
\pgfsetstrokecolor{currentstroke}%
\pgfsetstrokeopacity{0.929207}%
\pgfsetdash{}{0pt}%
\pgfpathmoveto{\pgfqpoint{1.760773in}{2.536664in}}%
\pgfpathcurveto{\pgfqpoint{1.769009in}{2.536664in}}{\pgfqpoint{1.776909in}{2.539937in}}{\pgfqpoint{1.782733in}{2.545761in}}%
\pgfpathcurveto{\pgfqpoint{1.788557in}{2.551585in}}{\pgfqpoint{1.791829in}{2.559485in}}{\pgfqpoint{1.791829in}{2.567721in}}%
\pgfpathcurveto{\pgfqpoint{1.791829in}{2.575957in}}{\pgfqpoint{1.788557in}{2.583857in}}{\pgfqpoint{1.782733in}{2.589681in}}%
\pgfpathcurveto{\pgfqpoint{1.776909in}{2.595505in}}{\pgfqpoint{1.769009in}{2.598777in}}{\pgfqpoint{1.760773in}{2.598777in}}%
\pgfpathcurveto{\pgfqpoint{1.752536in}{2.598777in}}{\pgfqpoint{1.744636in}{2.595505in}}{\pgfqpoint{1.738812in}{2.589681in}}%
\pgfpathcurveto{\pgfqpoint{1.732988in}{2.583857in}}{\pgfqpoint{1.729716in}{2.575957in}}{\pgfqpoint{1.729716in}{2.567721in}}%
\pgfpathcurveto{\pgfqpoint{1.729716in}{2.559485in}}{\pgfqpoint{1.732988in}{2.551585in}}{\pgfqpoint{1.738812in}{2.545761in}}%
\pgfpathcurveto{\pgfqpoint{1.744636in}{2.539937in}}{\pgfqpoint{1.752536in}{2.536664in}}{\pgfqpoint{1.760773in}{2.536664in}}%
\pgfpathclose%
\pgfusepath{stroke,fill}%
\end{pgfscope}%
\begin{pgfscope}%
\pgfpathrectangle{\pgfqpoint{0.100000in}{0.212622in}}{\pgfqpoint{3.696000in}{3.696000in}}%
\pgfusepath{clip}%
\pgfsetbuttcap%
\pgfsetroundjoin%
\definecolor{currentfill}{rgb}{0.121569,0.466667,0.705882}%
\pgfsetfillcolor{currentfill}%
\pgfsetfillopacity{0.929317}%
\pgfsetlinewidth{1.003750pt}%
\definecolor{currentstroke}{rgb}{0.121569,0.466667,0.705882}%
\pgfsetstrokecolor{currentstroke}%
\pgfsetstrokeopacity{0.929317}%
\pgfsetdash{}{0pt}%
\pgfpathmoveto{\pgfqpoint{2.131215in}{2.446995in}}%
\pgfpathcurveto{\pgfqpoint{2.139451in}{2.446995in}}{\pgfqpoint{2.147351in}{2.450268in}}{\pgfqpoint{2.153175in}{2.456092in}}%
\pgfpathcurveto{\pgfqpoint{2.158999in}{2.461915in}}{\pgfqpoint{2.162271in}{2.469815in}}{\pgfqpoint{2.162271in}{2.478052in}}%
\pgfpathcurveto{\pgfqpoint{2.162271in}{2.486288in}}{\pgfqpoint{2.158999in}{2.494188in}}{\pgfqpoint{2.153175in}{2.500012in}}%
\pgfpathcurveto{\pgfqpoint{2.147351in}{2.505836in}}{\pgfqpoint{2.139451in}{2.509108in}}{\pgfqpoint{2.131215in}{2.509108in}}%
\pgfpathcurveto{\pgfqpoint{2.122979in}{2.509108in}}{\pgfqpoint{2.115079in}{2.505836in}}{\pgfqpoint{2.109255in}{2.500012in}}%
\pgfpathcurveto{\pgfqpoint{2.103431in}{2.494188in}}{\pgfqpoint{2.100158in}{2.486288in}}{\pgfqpoint{2.100158in}{2.478052in}}%
\pgfpathcurveto{\pgfqpoint{2.100158in}{2.469815in}}{\pgfqpoint{2.103431in}{2.461915in}}{\pgfqpoint{2.109255in}{2.456092in}}%
\pgfpathcurveto{\pgfqpoint{2.115079in}{2.450268in}}{\pgfqpoint{2.122979in}{2.446995in}}{\pgfqpoint{2.131215in}{2.446995in}}%
\pgfpathclose%
\pgfusepath{stroke,fill}%
\end{pgfscope}%
\begin{pgfscope}%
\pgfpathrectangle{\pgfqpoint{0.100000in}{0.212622in}}{\pgfqpoint{3.696000in}{3.696000in}}%
\pgfusepath{clip}%
\pgfsetbuttcap%
\pgfsetroundjoin%
\definecolor{currentfill}{rgb}{0.121569,0.466667,0.705882}%
\pgfsetfillcolor{currentfill}%
\pgfsetfillopacity{0.929664}%
\pgfsetlinewidth{1.003750pt}%
\definecolor{currentstroke}{rgb}{0.121569,0.466667,0.705882}%
\pgfsetstrokecolor{currentstroke}%
\pgfsetstrokeopacity{0.929664}%
\pgfsetdash{}{0pt}%
\pgfpathmoveto{\pgfqpoint{1.765188in}{2.534504in}}%
\pgfpathcurveto{\pgfqpoint{1.773425in}{2.534504in}}{\pgfqpoint{1.781325in}{2.537776in}}{\pgfqpoint{1.787149in}{2.543600in}}%
\pgfpathcurveto{\pgfqpoint{1.792973in}{2.549424in}}{\pgfqpoint{1.796245in}{2.557324in}}{\pgfqpoint{1.796245in}{2.565560in}}%
\pgfpathcurveto{\pgfqpoint{1.796245in}{2.573796in}}{\pgfqpoint{1.792973in}{2.581697in}}{\pgfqpoint{1.787149in}{2.587520in}}%
\pgfpathcurveto{\pgfqpoint{1.781325in}{2.593344in}}{\pgfqpoint{1.773425in}{2.596617in}}{\pgfqpoint{1.765188in}{2.596617in}}%
\pgfpathcurveto{\pgfqpoint{1.756952in}{2.596617in}}{\pgfqpoint{1.749052in}{2.593344in}}{\pgfqpoint{1.743228in}{2.587520in}}%
\pgfpathcurveto{\pgfqpoint{1.737404in}{2.581697in}}{\pgfqpoint{1.734132in}{2.573796in}}{\pgfqpoint{1.734132in}{2.565560in}}%
\pgfpathcurveto{\pgfqpoint{1.734132in}{2.557324in}}{\pgfqpoint{1.737404in}{2.549424in}}{\pgfqpoint{1.743228in}{2.543600in}}%
\pgfpathcurveto{\pgfqpoint{1.749052in}{2.537776in}}{\pgfqpoint{1.756952in}{2.534504in}}{\pgfqpoint{1.765188in}{2.534504in}}%
\pgfpathclose%
\pgfusepath{stroke,fill}%
\end{pgfscope}%
\begin{pgfscope}%
\pgfpathrectangle{\pgfqpoint{0.100000in}{0.212622in}}{\pgfqpoint{3.696000in}{3.696000in}}%
\pgfusepath{clip}%
\pgfsetbuttcap%
\pgfsetroundjoin%
\definecolor{currentfill}{rgb}{0.121569,0.466667,0.705882}%
\pgfsetfillcolor{currentfill}%
\pgfsetfillopacity{0.930142}%
\pgfsetlinewidth{1.003750pt}%
\definecolor{currentstroke}{rgb}{0.121569,0.466667,0.705882}%
\pgfsetstrokecolor{currentstroke}%
\pgfsetstrokeopacity{0.930142}%
\pgfsetdash{}{0pt}%
\pgfpathmoveto{\pgfqpoint{1.143193in}{1.948760in}}%
\pgfpathcurveto{\pgfqpoint{1.151430in}{1.948760in}}{\pgfqpoint{1.159330in}{1.952032in}}{\pgfqpoint{1.165154in}{1.957856in}}%
\pgfpathcurveto{\pgfqpoint{1.170978in}{1.963680in}}{\pgfqpoint{1.174250in}{1.971580in}}{\pgfqpoint{1.174250in}{1.979816in}}%
\pgfpathcurveto{\pgfqpoint{1.174250in}{1.988052in}}{\pgfqpoint{1.170978in}{1.995952in}}{\pgfqpoint{1.165154in}{2.001776in}}%
\pgfpathcurveto{\pgfqpoint{1.159330in}{2.007600in}}{\pgfqpoint{1.151430in}{2.010873in}}{\pgfqpoint{1.143193in}{2.010873in}}%
\pgfpathcurveto{\pgfqpoint{1.134957in}{2.010873in}}{\pgfqpoint{1.127057in}{2.007600in}}{\pgfqpoint{1.121233in}{2.001776in}}%
\pgfpathcurveto{\pgfqpoint{1.115409in}{1.995952in}}{\pgfqpoint{1.112137in}{1.988052in}}{\pgfqpoint{1.112137in}{1.979816in}}%
\pgfpathcurveto{\pgfqpoint{1.112137in}{1.971580in}}{\pgfqpoint{1.115409in}{1.963680in}}{\pgfqpoint{1.121233in}{1.957856in}}%
\pgfpathcurveto{\pgfqpoint{1.127057in}{1.952032in}}{\pgfqpoint{1.134957in}{1.948760in}}{\pgfqpoint{1.143193in}{1.948760in}}%
\pgfpathclose%
\pgfusepath{stroke,fill}%
\end{pgfscope}%
\begin{pgfscope}%
\pgfpathrectangle{\pgfqpoint{0.100000in}{0.212622in}}{\pgfqpoint{3.696000in}{3.696000in}}%
\pgfusepath{clip}%
\pgfsetbuttcap%
\pgfsetroundjoin%
\definecolor{currentfill}{rgb}{0.121569,0.466667,0.705882}%
\pgfsetfillcolor{currentfill}%
\pgfsetfillopacity{0.930216}%
\pgfsetlinewidth{1.003750pt}%
\definecolor{currentstroke}{rgb}{0.121569,0.466667,0.705882}%
\pgfsetstrokecolor{currentstroke}%
\pgfsetstrokeopacity{0.930216}%
\pgfsetdash{}{0pt}%
\pgfpathmoveto{\pgfqpoint{1.769922in}{2.531945in}}%
\pgfpathcurveto{\pgfqpoint{1.778158in}{2.531945in}}{\pgfqpoint{1.786058in}{2.535218in}}{\pgfqpoint{1.791882in}{2.541042in}}%
\pgfpathcurveto{\pgfqpoint{1.797706in}{2.546866in}}{\pgfqpoint{1.800979in}{2.554766in}}{\pgfqpoint{1.800979in}{2.563002in}}%
\pgfpathcurveto{\pgfqpoint{1.800979in}{2.571238in}}{\pgfqpoint{1.797706in}{2.579138in}}{\pgfqpoint{1.791882in}{2.584962in}}%
\pgfpathcurveto{\pgfqpoint{1.786058in}{2.590786in}}{\pgfqpoint{1.778158in}{2.594058in}}{\pgfqpoint{1.769922in}{2.594058in}}%
\pgfpathcurveto{\pgfqpoint{1.761686in}{2.594058in}}{\pgfqpoint{1.753786in}{2.590786in}}{\pgfqpoint{1.747962in}{2.584962in}}%
\pgfpathcurveto{\pgfqpoint{1.742138in}{2.579138in}}{\pgfqpoint{1.738866in}{2.571238in}}{\pgfqpoint{1.738866in}{2.563002in}}%
\pgfpathcurveto{\pgfqpoint{1.738866in}{2.554766in}}{\pgfqpoint{1.742138in}{2.546866in}}{\pgfqpoint{1.747962in}{2.541042in}}%
\pgfpathcurveto{\pgfqpoint{1.753786in}{2.535218in}}{\pgfqpoint{1.761686in}{2.531945in}}{\pgfqpoint{1.769922in}{2.531945in}}%
\pgfpathclose%
\pgfusepath{stroke,fill}%
\end{pgfscope}%
\begin{pgfscope}%
\pgfpathrectangle{\pgfqpoint{0.100000in}{0.212622in}}{\pgfqpoint{3.696000in}{3.696000in}}%
\pgfusepath{clip}%
\pgfsetbuttcap%
\pgfsetroundjoin%
\definecolor{currentfill}{rgb}{0.121569,0.466667,0.705882}%
\pgfsetfillcolor{currentfill}%
\pgfsetfillopacity{0.930341}%
\pgfsetlinewidth{1.003750pt}%
\definecolor{currentstroke}{rgb}{0.121569,0.466667,0.705882}%
\pgfsetstrokecolor{currentstroke}%
\pgfsetstrokeopacity{0.930341}%
\pgfsetdash{}{0pt}%
\pgfpathmoveto{\pgfqpoint{2.129039in}{2.444465in}}%
\pgfpathcurveto{\pgfqpoint{2.137275in}{2.444465in}}{\pgfqpoint{2.145175in}{2.447737in}}{\pgfqpoint{2.150999in}{2.453561in}}%
\pgfpathcurveto{\pgfqpoint{2.156823in}{2.459385in}}{\pgfqpoint{2.160095in}{2.467285in}}{\pgfqpoint{2.160095in}{2.475521in}}%
\pgfpathcurveto{\pgfqpoint{2.160095in}{2.483758in}}{\pgfqpoint{2.156823in}{2.491658in}}{\pgfqpoint{2.150999in}{2.497482in}}%
\pgfpathcurveto{\pgfqpoint{2.145175in}{2.503306in}}{\pgfqpoint{2.137275in}{2.506578in}}{\pgfqpoint{2.129039in}{2.506578in}}%
\pgfpathcurveto{\pgfqpoint{2.120802in}{2.506578in}}{\pgfqpoint{2.112902in}{2.503306in}}{\pgfqpoint{2.107078in}{2.497482in}}%
\pgfpathcurveto{\pgfqpoint{2.101254in}{2.491658in}}{\pgfqpoint{2.097982in}{2.483758in}}{\pgfqpoint{2.097982in}{2.475521in}}%
\pgfpathcurveto{\pgfqpoint{2.097982in}{2.467285in}}{\pgfqpoint{2.101254in}{2.459385in}}{\pgfqpoint{2.107078in}{2.453561in}}%
\pgfpathcurveto{\pgfqpoint{2.112902in}{2.447737in}}{\pgfqpoint{2.120802in}{2.444465in}}{\pgfqpoint{2.129039in}{2.444465in}}%
\pgfpathclose%
\pgfusepath{stroke,fill}%
\end{pgfscope}%
\begin{pgfscope}%
\pgfpathrectangle{\pgfqpoint{0.100000in}{0.212622in}}{\pgfqpoint{3.696000in}{3.696000in}}%
\pgfusepath{clip}%
\pgfsetbuttcap%
\pgfsetroundjoin%
\definecolor{currentfill}{rgb}{0.121569,0.466667,0.705882}%
\pgfsetfillcolor{currentfill}%
\pgfsetfillopacity{0.930755}%
\pgfsetlinewidth{1.003750pt}%
\definecolor{currentstroke}{rgb}{0.121569,0.466667,0.705882}%
\pgfsetstrokecolor{currentstroke}%
\pgfsetstrokeopacity{0.930755}%
\pgfsetdash{}{0pt}%
\pgfpathmoveto{\pgfqpoint{1.148191in}{1.945931in}}%
\pgfpathcurveto{\pgfqpoint{1.156427in}{1.945931in}}{\pgfqpoint{1.164327in}{1.949203in}}{\pgfqpoint{1.170151in}{1.955027in}}%
\pgfpathcurveto{\pgfqpoint{1.175975in}{1.960851in}}{\pgfqpoint{1.179247in}{1.968751in}}{\pgfqpoint{1.179247in}{1.976987in}}%
\pgfpathcurveto{\pgfqpoint{1.179247in}{1.985224in}}{\pgfqpoint{1.175975in}{1.993124in}}{\pgfqpoint{1.170151in}{1.998948in}}%
\pgfpathcurveto{\pgfqpoint{1.164327in}{2.004772in}}{\pgfqpoint{1.156427in}{2.008044in}}{\pgfqpoint{1.148191in}{2.008044in}}%
\pgfpathcurveto{\pgfqpoint{1.139955in}{2.008044in}}{\pgfqpoint{1.132054in}{2.004772in}}{\pgfqpoint{1.126231in}{1.998948in}}%
\pgfpathcurveto{\pgfqpoint{1.120407in}{1.993124in}}{\pgfqpoint{1.117134in}{1.985224in}}{\pgfqpoint{1.117134in}{1.976987in}}%
\pgfpathcurveto{\pgfqpoint{1.117134in}{1.968751in}}{\pgfqpoint{1.120407in}{1.960851in}}{\pgfqpoint{1.126231in}{1.955027in}}%
\pgfpathcurveto{\pgfqpoint{1.132054in}{1.949203in}}{\pgfqpoint{1.139955in}{1.945931in}}{\pgfqpoint{1.148191in}{1.945931in}}%
\pgfpathclose%
\pgfusepath{stroke,fill}%
\end{pgfscope}%
\begin{pgfscope}%
\pgfpathrectangle{\pgfqpoint{0.100000in}{0.212622in}}{\pgfqpoint{3.696000in}{3.696000in}}%
\pgfusepath{clip}%
\pgfsetbuttcap%
\pgfsetroundjoin%
\definecolor{currentfill}{rgb}{0.121569,0.466667,0.705882}%
\pgfsetfillcolor{currentfill}%
\pgfsetfillopacity{0.930799}%
\pgfsetlinewidth{1.003750pt}%
\definecolor{currentstroke}{rgb}{0.121569,0.466667,0.705882}%
\pgfsetstrokecolor{currentstroke}%
\pgfsetstrokeopacity{0.930799}%
\pgfsetdash{}{0pt}%
\pgfpathmoveto{\pgfqpoint{2.623351in}{1.241830in}}%
\pgfpathcurveto{\pgfqpoint{2.631587in}{1.241830in}}{\pgfqpoint{2.639487in}{1.245103in}}{\pgfqpoint{2.645311in}{1.250927in}}%
\pgfpathcurveto{\pgfqpoint{2.651135in}{1.256751in}}{\pgfqpoint{2.654407in}{1.264651in}}{\pgfqpoint{2.654407in}{1.272887in}}%
\pgfpathcurveto{\pgfqpoint{2.654407in}{1.281123in}}{\pgfqpoint{2.651135in}{1.289023in}}{\pgfqpoint{2.645311in}{1.294847in}}%
\pgfpathcurveto{\pgfqpoint{2.639487in}{1.300671in}}{\pgfqpoint{2.631587in}{1.303943in}}{\pgfqpoint{2.623351in}{1.303943in}}%
\pgfpathcurveto{\pgfqpoint{2.615115in}{1.303943in}}{\pgfqpoint{2.607215in}{1.300671in}}{\pgfqpoint{2.601391in}{1.294847in}}%
\pgfpathcurveto{\pgfqpoint{2.595567in}{1.289023in}}{\pgfqpoint{2.592294in}{1.281123in}}{\pgfqpoint{2.592294in}{1.272887in}}%
\pgfpathcurveto{\pgfqpoint{2.592294in}{1.264651in}}{\pgfqpoint{2.595567in}{1.256751in}}{\pgfqpoint{2.601391in}{1.250927in}}%
\pgfpathcurveto{\pgfqpoint{2.607215in}{1.245103in}}{\pgfqpoint{2.615115in}{1.241830in}}{\pgfqpoint{2.623351in}{1.241830in}}%
\pgfpathclose%
\pgfusepath{stroke,fill}%
\end{pgfscope}%
\begin{pgfscope}%
\pgfpathrectangle{\pgfqpoint{0.100000in}{0.212622in}}{\pgfqpoint{3.696000in}{3.696000in}}%
\pgfusepath{clip}%
\pgfsetbuttcap%
\pgfsetroundjoin%
\definecolor{currentfill}{rgb}{0.121569,0.466667,0.705882}%
\pgfsetfillcolor{currentfill}%
\pgfsetfillopacity{0.930851}%
\pgfsetlinewidth{1.003750pt}%
\definecolor{currentstroke}{rgb}{0.121569,0.466667,0.705882}%
\pgfsetstrokecolor{currentstroke}%
\pgfsetstrokeopacity{0.930851}%
\pgfsetdash{}{0pt}%
\pgfpathmoveto{\pgfqpoint{2.127932in}{2.443266in}}%
\pgfpathcurveto{\pgfqpoint{2.136168in}{2.443266in}}{\pgfqpoint{2.144068in}{2.446539in}}{\pgfqpoint{2.149892in}{2.452363in}}%
\pgfpathcurveto{\pgfqpoint{2.155716in}{2.458187in}}{\pgfqpoint{2.158988in}{2.466087in}}{\pgfqpoint{2.158988in}{2.474323in}}%
\pgfpathcurveto{\pgfqpoint{2.158988in}{2.482559in}}{\pgfqpoint{2.155716in}{2.490459in}}{\pgfqpoint{2.149892in}{2.496283in}}%
\pgfpathcurveto{\pgfqpoint{2.144068in}{2.502107in}}{\pgfqpoint{2.136168in}{2.505379in}}{\pgfqpoint{2.127932in}{2.505379in}}%
\pgfpathcurveto{\pgfqpoint{2.119695in}{2.505379in}}{\pgfqpoint{2.111795in}{2.502107in}}{\pgfqpoint{2.105972in}{2.496283in}}%
\pgfpathcurveto{\pgfqpoint{2.100148in}{2.490459in}}{\pgfqpoint{2.096875in}{2.482559in}}{\pgfqpoint{2.096875in}{2.474323in}}%
\pgfpathcurveto{\pgfqpoint{2.096875in}{2.466087in}}{\pgfqpoint{2.100148in}{2.458187in}}{\pgfqpoint{2.105972in}{2.452363in}}%
\pgfpathcurveto{\pgfqpoint{2.111795in}{2.446539in}}{\pgfqpoint{2.119695in}{2.443266in}}{\pgfqpoint{2.127932in}{2.443266in}}%
\pgfpathclose%
\pgfusepath{stroke,fill}%
\end{pgfscope}%
\begin{pgfscope}%
\pgfpathrectangle{\pgfqpoint{0.100000in}{0.212622in}}{\pgfqpoint{3.696000in}{3.696000in}}%
\pgfusepath{clip}%
\pgfsetbuttcap%
\pgfsetroundjoin%
\definecolor{currentfill}{rgb}{0.121569,0.466667,0.705882}%
\pgfsetfillcolor{currentfill}%
\pgfsetfillopacity{0.930953}%
\pgfsetlinewidth{1.003750pt}%
\definecolor{currentstroke}{rgb}{0.121569,0.466667,0.705882}%
\pgfsetstrokecolor{currentstroke}%
\pgfsetstrokeopacity{0.930953}%
\pgfsetdash{}{0pt}%
\pgfpathmoveto{\pgfqpoint{1.775112in}{2.529307in}}%
\pgfpathcurveto{\pgfqpoint{1.783349in}{2.529307in}}{\pgfqpoint{1.791249in}{2.532580in}}{\pgfqpoint{1.797073in}{2.538404in}}%
\pgfpathcurveto{\pgfqpoint{1.802896in}{2.544228in}}{\pgfqpoint{1.806169in}{2.552128in}}{\pgfqpoint{1.806169in}{2.560364in}}%
\pgfpathcurveto{\pgfqpoint{1.806169in}{2.568600in}}{\pgfqpoint{1.802896in}{2.576500in}}{\pgfqpoint{1.797073in}{2.582324in}}%
\pgfpathcurveto{\pgfqpoint{1.791249in}{2.588148in}}{\pgfqpoint{1.783349in}{2.591420in}}{\pgfqpoint{1.775112in}{2.591420in}}%
\pgfpathcurveto{\pgfqpoint{1.766876in}{2.591420in}}{\pgfqpoint{1.758976in}{2.588148in}}{\pgfqpoint{1.753152in}{2.582324in}}%
\pgfpathcurveto{\pgfqpoint{1.747328in}{2.576500in}}{\pgfqpoint{1.744056in}{2.568600in}}{\pgfqpoint{1.744056in}{2.560364in}}%
\pgfpathcurveto{\pgfqpoint{1.744056in}{2.552128in}}{\pgfqpoint{1.747328in}{2.544228in}}{\pgfqpoint{1.753152in}{2.538404in}}%
\pgfpathcurveto{\pgfqpoint{1.758976in}{2.532580in}}{\pgfqpoint{1.766876in}{2.529307in}}{\pgfqpoint{1.775112in}{2.529307in}}%
\pgfpathclose%
\pgfusepath{stroke,fill}%
\end{pgfscope}%
\begin{pgfscope}%
\pgfpathrectangle{\pgfqpoint{0.100000in}{0.212622in}}{\pgfqpoint{3.696000in}{3.696000in}}%
\pgfusepath{clip}%
\pgfsetbuttcap%
\pgfsetroundjoin%
\definecolor{currentfill}{rgb}{0.121569,0.466667,0.705882}%
\pgfsetfillcolor{currentfill}%
\pgfsetfillopacity{0.931445}%
\pgfsetlinewidth{1.003750pt}%
\definecolor{currentstroke}{rgb}{0.121569,0.466667,0.705882}%
\pgfsetstrokecolor{currentstroke}%
\pgfsetstrokeopacity{0.931445}%
\pgfsetdash{}{0pt}%
\pgfpathmoveto{\pgfqpoint{1.154068in}{1.942535in}}%
\pgfpathcurveto{\pgfqpoint{1.162304in}{1.942535in}}{\pgfqpoint{1.170204in}{1.945808in}}{\pgfqpoint{1.176028in}{1.951631in}}%
\pgfpathcurveto{\pgfqpoint{1.181852in}{1.957455in}}{\pgfqpoint{1.185124in}{1.965355in}}{\pgfqpoint{1.185124in}{1.973592in}}%
\pgfpathcurveto{\pgfqpoint{1.185124in}{1.981828in}}{\pgfqpoint{1.181852in}{1.989728in}}{\pgfqpoint{1.176028in}{1.995552in}}%
\pgfpathcurveto{\pgfqpoint{1.170204in}{2.001376in}}{\pgfqpoint{1.162304in}{2.004648in}}{\pgfqpoint{1.154068in}{2.004648in}}%
\pgfpathcurveto{\pgfqpoint{1.145831in}{2.004648in}}{\pgfqpoint{1.137931in}{2.001376in}}{\pgfqpoint{1.132107in}{1.995552in}}%
\pgfpathcurveto{\pgfqpoint{1.126283in}{1.989728in}}{\pgfqpoint{1.123011in}{1.981828in}}{\pgfqpoint{1.123011in}{1.973592in}}%
\pgfpathcurveto{\pgfqpoint{1.123011in}{1.965355in}}{\pgfqpoint{1.126283in}{1.957455in}}{\pgfqpoint{1.132107in}{1.951631in}}%
\pgfpathcurveto{\pgfqpoint{1.137931in}{1.945808in}}{\pgfqpoint{1.145831in}{1.942535in}}{\pgfqpoint{1.154068in}{1.942535in}}%
\pgfpathclose%
\pgfusepath{stroke,fill}%
\end{pgfscope}%
\begin{pgfscope}%
\pgfpathrectangle{\pgfqpoint{0.100000in}{0.212622in}}{\pgfqpoint{3.696000in}{3.696000in}}%
\pgfusepath{clip}%
\pgfsetbuttcap%
\pgfsetroundjoin%
\definecolor{currentfill}{rgb}{0.121569,0.466667,0.705882}%
\pgfsetfillcolor{currentfill}%
\pgfsetfillopacity{0.931801}%
\pgfsetlinewidth{1.003750pt}%
\definecolor{currentstroke}{rgb}{0.121569,0.466667,0.705882}%
\pgfsetstrokecolor{currentstroke}%
\pgfsetstrokeopacity{0.931801}%
\pgfsetdash{}{0pt}%
\pgfpathmoveto{\pgfqpoint{2.125995in}{2.441148in}}%
\pgfpathcurveto{\pgfqpoint{2.134231in}{2.441148in}}{\pgfqpoint{2.142131in}{2.444420in}}{\pgfqpoint{2.147955in}{2.450244in}}%
\pgfpathcurveto{\pgfqpoint{2.153779in}{2.456068in}}{\pgfqpoint{2.157052in}{2.463968in}}{\pgfqpoint{2.157052in}{2.472205in}}%
\pgfpathcurveto{\pgfqpoint{2.157052in}{2.480441in}}{\pgfqpoint{2.153779in}{2.488341in}}{\pgfqpoint{2.147955in}{2.494165in}}%
\pgfpathcurveto{\pgfqpoint{2.142131in}{2.499989in}}{\pgfqpoint{2.134231in}{2.503261in}}{\pgfqpoint{2.125995in}{2.503261in}}%
\pgfpathcurveto{\pgfqpoint{2.117759in}{2.503261in}}{\pgfqpoint{2.109859in}{2.499989in}}{\pgfqpoint{2.104035in}{2.494165in}}%
\pgfpathcurveto{\pgfqpoint{2.098211in}{2.488341in}}{\pgfqpoint{2.094939in}{2.480441in}}{\pgfqpoint{2.094939in}{2.472205in}}%
\pgfpathcurveto{\pgfqpoint{2.094939in}{2.463968in}}{\pgfqpoint{2.098211in}{2.456068in}}{\pgfqpoint{2.104035in}{2.450244in}}%
\pgfpathcurveto{\pgfqpoint{2.109859in}{2.444420in}}{\pgfqpoint{2.117759in}{2.441148in}}{\pgfqpoint{2.125995in}{2.441148in}}%
\pgfpathclose%
\pgfusepath{stroke,fill}%
\end{pgfscope}%
\begin{pgfscope}%
\pgfpathrectangle{\pgfqpoint{0.100000in}{0.212622in}}{\pgfqpoint{3.696000in}{3.696000in}}%
\pgfusepath{clip}%
\pgfsetbuttcap%
\pgfsetroundjoin%
\definecolor{currentfill}{rgb}{0.121569,0.466667,0.705882}%
\pgfsetfillcolor{currentfill}%
\pgfsetfillopacity{0.931934}%
\pgfsetlinewidth{1.003750pt}%
\definecolor{currentstroke}{rgb}{0.121569,0.466667,0.705882}%
\pgfsetstrokecolor{currentstroke}%
\pgfsetstrokeopacity{0.931934}%
\pgfsetdash{}{0pt}%
\pgfpathmoveto{\pgfqpoint{1.780679in}{2.526703in}}%
\pgfpathcurveto{\pgfqpoint{1.788915in}{2.526703in}}{\pgfqpoint{1.796815in}{2.529975in}}{\pgfqpoint{1.802639in}{2.535799in}}%
\pgfpathcurveto{\pgfqpoint{1.808463in}{2.541623in}}{\pgfqpoint{1.811735in}{2.549523in}}{\pgfqpoint{1.811735in}{2.557759in}}%
\pgfpathcurveto{\pgfqpoint{1.811735in}{2.565996in}}{\pgfqpoint{1.808463in}{2.573896in}}{\pgfqpoint{1.802639in}{2.579720in}}%
\pgfpathcurveto{\pgfqpoint{1.796815in}{2.585544in}}{\pgfqpoint{1.788915in}{2.588816in}}{\pgfqpoint{1.780679in}{2.588816in}}%
\pgfpathcurveto{\pgfqpoint{1.772443in}{2.588816in}}{\pgfqpoint{1.764543in}{2.585544in}}{\pgfqpoint{1.758719in}{2.579720in}}%
\pgfpathcurveto{\pgfqpoint{1.752895in}{2.573896in}}{\pgfqpoint{1.749622in}{2.565996in}}{\pgfqpoint{1.749622in}{2.557759in}}%
\pgfpathcurveto{\pgfqpoint{1.749622in}{2.549523in}}{\pgfqpoint{1.752895in}{2.541623in}}{\pgfqpoint{1.758719in}{2.535799in}}%
\pgfpathcurveto{\pgfqpoint{1.764543in}{2.529975in}}{\pgfqpoint{1.772443in}{2.526703in}}{\pgfqpoint{1.780679in}{2.526703in}}%
\pgfpathclose%
\pgfusepath{stroke,fill}%
\end{pgfscope}%
\begin{pgfscope}%
\pgfpathrectangle{\pgfqpoint{0.100000in}{0.212622in}}{\pgfqpoint{3.696000in}{3.696000in}}%
\pgfusepath{clip}%
\pgfsetbuttcap%
\pgfsetroundjoin%
\definecolor{currentfill}{rgb}{0.121569,0.466667,0.705882}%
\pgfsetfillcolor{currentfill}%
\pgfsetfillopacity{0.932175}%
\pgfsetlinewidth{1.003750pt}%
\definecolor{currentstroke}{rgb}{0.121569,0.466667,0.705882}%
\pgfsetstrokecolor{currentstroke}%
\pgfsetstrokeopacity{0.932175}%
\pgfsetdash{}{0pt}%
\pgfpathmoveto{\pgfqpoint{1.161111in}{1.938501in}}%
\pgfpathcurveto{\pgfqpoint{1.169347in}{1.938501in}}{\pgfqpoint{1.177247in}{1.941773in}}{\pgfqpoint{1.183071in}{1.947597in}}%
\pgfpathcurveto{\pgfqpoint{1.188895in}{1.953421in}}{\pgfqpoint{1.192168in}{1.961321in}}{\pgfqpoint{1.192168in}{1.969558in}}%
\pgfpathcurveto{\pgfqpoint{1.192168in}{1.977794in}}{\pgfqpoint{1.188895in}{1.985694in}}{\pgfqpoint{1.183071in}{1.991518in}}%
\pgfpathcurveto{\pgfqpoint{1.177247in}{1.997342in}}{\pgfqpoint{1.169347in}{2.000614in}}{\pgfqpoint{1.161111in}{2.000614in}}%
\pgfpathcurveto{\pgfqpoint{1.152875in}{2.000614in}}{\pgfqpoint{1.144975in}{1.997342in}}{\pgfqpoint{1.139151in}{1.991518in}}%
\pgfpathcurveto{\pgfqpoint{1.133327in}{1.985694in}}{\pgfqpoint{1.130055in}{1.977794in}}{\pgfqpoint{1.130055in}{1.969558in}}%
\pgfpathcurveto{\pgfqpoint{1.130055in}{1.961321in}}{\pgfqpoint{1.133327in}{1.953421in}}{\pgfqpoint{1.139151in}{1.947597in}}%
\pgfpathcurveto{\pgfqpoint{1.144975in}{1.941773in}}{\pgfqpoint{1.152875in}{1.938501in}}{\pgfqpoint{1.161111in}{1.938501in}}%
\pgfpathclose%
\pgfusepath{stroke,fill}%
\end{pgfscope}%
\begin{pgfscope}%
\pgfpathrectangle{\pgfqpoint{0.100000in}{0.212622in}}{\pgfqpoint{3.696000in}{3.696000in}}%
\pgfusepath{clip}%
\pgfsetbuttcap%
\pgfsetroundjoin%
\definecolor{currentfill}{rgb}{0.121569,0.466667,0.705882}%
\pgfsetfillcolor{currentfill}%
\pgfsetfillopacity{0.932407}%
\pgfsetlinewidth{1.003750pt}%
\definecolor{currentstroke}{rgb}{0.121569,0.466667,0.705882}%
\pgfsetstrokecolor{currentstroke}%
\pgfsetstrokeopacity{0.932407}%
\pgfsetdash{}{0pt}%
\pgfpathmoveto{\pgfqpoint{2.124784in}{2.439802in}}%
\pgfpathcurveto{\pgfqpoint{2.133020in}{2.439802in}}{\pgfqpoint{2.140921in}{2.443075in}}{\pgfqpoint{2.146744in}{2.448899in}}%
\pgfpathcurveto{\pgfqpoint{2.152568in}{2.454723in}}{\pgfqpoint{2.155841in}{2.462623in}}{\pgfqpoint{2.155841in}{2.470859in}}%
\pgfpathcurveto{\pgfqpoint{2.155841in}{2.479095in}}{\pgfqpoint{2.152568in}{2.486995in}}{\pgfqpoint{2.146744in}{2.492819in}}%
\pgfpathcurveto{\pgfqpoint{2.140921in}{2.498643in}}{\pgfqpoint{2.133020in}{2.501915in}}{\pgfqpoint{2.124784in}{2.501915in}}%
\pgfpathcurveto{\pgfqpoint{2.116548in}{2.501915in}}{\pgfqpoint{2.108648in}{2.498643in}}{\pgfqpoint{2.102824in}{2.492819in}}%
\pgfpathcurveto{\pgfqpoint{2.097000in}{2.486995in}}{\pgfqpoint{2.093728in}{2.479095in}}{\pgfqpoint{2.093728in}{2.470859in}}%
\pgfpathcurveto{\pgfqpoint{2.093728in}{2.462623in}}{\pgfqpoint{2.097000in}{2.454723in}}{\pgfqpoint{2.102824in}{2.448899in}}%
\pgfpathcurveto{\pgfqpoint{2.108648in}{2.443075in}}{\pgfqpoint{2.116548in}{2.439802in}}{\pgfqpoint{2.124784in}{2.439802in}}%
\pgfpathclose%
\pgfusepath{stroke,fill}%
\end{pgfscope}%
\begin{pgfscope}%
\pgfpathrectangle{\pgfqpoint{0.100000in}{0.212622in}}{\pgfqpoint{3.696000in}{3.696000in}}%
\pgfusepath{clip}%
\pgfsetbuttcap%
\pgfsetroundjoin%
\definecolor{currentfill}{rgb}{0.121569,0.466667,0.705882}%
\pgfsetfillcolor{currentfill}%
\pgfsetfillopacity{0.932548}%
\pgfsetlinewidth{1.003750pt}%
\definecolor{currentstroke}{rgb}{0.121569,0.466667,0.705882}%
\pgfsetstrokecolor{currentstroke}%
\pgfsetstrokeopacity{0.932548}%
\pgfsetdash{}{0pt}%
\pgfpathmoveto{\pgfqpoint{2.620153in}{1.238774in}}%
\pgfpathcurveto{\pgfqpoint{2.628389in}{1.238774in}}{\pgfqpoint{2.636289in}{1.242046in}}{\pgfqpoint{2.642113in}{1.247870in}}%
\pgfpathcurveto{\pgfqpoint{2.647937in}{1.253694in}}{\pgfqpoint{2.651209in}{1.261594in}}{\pgfqpoint{2.651209in}{1.269831in}}%
\pgfpathcurveto{\pgfqpoint{2.651209in}{1.278067in}}{\pgfqpoint{2.647937in}{1.285967in}}{\pgfqpoint{2.642113in}{1.291791in}}%
\pgfpathcurveto{\pgfqpoint{2.636289in}{1.297615in}}{\pgfqpoint{2.628389in}{1.300887in}}{\pgfqpoint{2.620153in}{1.300887in}}%
\pgfpathcurveto{\pgfqpoint{2.611916in}{1.300887in}}{\pgfqpoint{2.604016in}{1.297615in}}{\pgfqpoint{2.598192in}{1.291791in}}%
\pgfpathcurveto{\pgfqpoint{2.592368in}{1.285967in}}{\pgfqpoint{2.589096in}{1.278067in}}{\pgfqpoint{2.589096in}{1.269831in}}%
\pgfpathcurveto{\pgfqpoint{2.589096in}{1.261594in}}{\pgfqpoint{2.592368in}{1.253694in}}{\pgfqpoint{2.598192in}{1.247870in}}%
\pgfpathcurveto{\pgfqpoint{2.604016in}{1.242046in}}{\pgfqpoint{2.611916in}{1.238774in}}{\pgfqpoint{2.620153in}{1.238774in}}%
\pgfpathclose%
\pgfusepath{stroke,fill}%
\end{pgfscope}%
\begin{pgfscope}%
\pgfpathrectangle{\pgfqpoint{0.100000in}{0.212622in}}{\pgfqpoint{3.696000in}{3.696000in}}%
\pgfusepath{clip}%
\pgfsetbuttcap%
\pgfsetroundjoin%
\definecolor{currentfill}{rgb}{0.121569,0.466667,0.705882}%
\pgfsetfillcolor{currentfill}%
\pgfsetfillopacity{0.932908}%
\pgfsetlinewidth{1.003750pt}%
\definecolor{currentstroke}{rgb}{0.121569,0.466667,0.705882}%
\pgfsetstrokecolor{currentstroke}%
\pgfsetstrokeopacity{0.932908}%
\pgfsetdash{}{0pt}%
\pgfpathmoveto{\pgfqpoint{2.123750in}{2.438803in}}%
\pgfpathcurveto{\pgfqpoint{2.131986in}{2.438803in}}{\pgfqpoint{2.139886in}{2.442075in}}{\pgfqpoint{2.145710in}{2.447899in}}%
\pgfpathcurveto{\pgfqpoint{2.151534in}{2.453723in}}{\pgfqpoint{2.154806in}{2.461623in}}{\pgfqpoint{2.154806in}{2.469860in}}%
\pgfpathcurveto{\pgfqpoint{2.154806in}{2.478096in}}{\pgfqpoint{2.151534in}{2.485996in}}{\pgfqpoint{2.145710in}{2.491820in}}%
\pgfpathcurveto{\pgfqpoint{2.139886in}{2.497644in}}{\pgfqpoint{2.131986in}{2.500916in}}{\pgfqpoint{2.123750in}{2.500916in}}%
\pgfpathcurveto{\pgfqpoint{2.115513in}{2.500916in}}{\pgfqpoint{2.107613in}{2.497644in}}{\pgfqpoint{2.101789in}{2.491820in}}%
\pgfpathcurveto{\pgfqpoint{2.095966in}{2.485996in}}{\pgfqpoint{2.092693in}{2.478096in}}{\pgfqpoint{2.092693in}{2.469860in}}%
\pgfpathcurveto{\pgfqpoint{2.092693in}{2.461623in}}{\pgfqpoint{2.095966in}{2.453723in}}{\pgfqpoint{2.101789in}{2.447899in}}%
\pgfpathcurveto{\pgfqpoint{2.107613in}{2.442075in}}{\pgfqpoint{2.115513in}{2.438803in}}{\pgfqpoint{2.123750in}{2.438803in}}%
\pgfpathclose%
\pgfusepath{stroke,fill}%
\end{pgfscope}%
\begin{pgfscope}%
\pgfpathrectangle{\pgfqpoint{0.100000in}{0.212622in}}{\pgfqpoint{3.696000in}{3.696000in}}%
\pgfusepath{clip}%
\pgfsetbuttcap%
\pgfsetroundjoin%
\definecolor{currentfill}{rgb}{0.121569,0.466667,0.705882}%
\pgfsetfillcolor{currentfill}%
\pgfsetfillopacity{0.932933}%
\pgfsetlinewidth{1.003750pt}%
\definecolor{currentstroke}{rgb}{0.121569,0.466667,0.705882}%
\pgfsetstrokecolor{currentstroke}%
\pgfsetstrokeopacity{0.932933}%
\pgfsetdash{}{0pt}%
\pgfpathmoveto{\pgfqpoint{1.786858in}{2.523512in}}%
\pgfpathcurveto{\pgfqpoint{1.795094in}{2.523512in}}{\pgfqpoint{1.802994in}{2.526784in}}{\pgfqpoint{1.808818in}{2.532608in}}%
\pgfpathcurveto{\pgfqpoint{1.814642in}{2.538432in}}{\pgfqpoint{1.817915in}{2.546332in}}{\pgfqpoint{1.817915in}{2.554569in}}%
\pgfpathcurveto{\pgfqpoint{1.817915in}{2.562805in}}{\pgfqpoint{1.814642in}{2.570705in}}{\pgfqpoint{1.808818in}{2.576529in}}%
\pgfpathcurveto{\pgfqpoint{1.802994in}{2.582353in}}{\pgfqpoint{1.795094in}{2.585625in}}{\pgfqpoint{1.786858in}{2.585625in}}%
\pgfpathcurveto{\pgfqpoint{1.778622in}{2.585625in}}{\pgfqpoint{1.770722in}{2.582353in}}{\pgfqpoint{1.764898in}{2.576529in}}%
\pgfpathcurveto{\pgfqpoint{1.759074in}{2.570705in}}{\pgfqpoint{1.755802in}{2.562805in}}{\pgfqpoint{1.755802in}{2.554569in}}%
\pgfpathcurveto{\pgfqpoint{1.755802in}{2.546332in}}{\pgfqpoint{1.759074in}{2.538432in}}{\pgfqpoint{1.764898in}{2.532608in}}%
\pgfpathcurveto{\pgfqpoint{1.770722in}{2.526784in}}{\pgfqpoint{1.778622in}{2.523512in}}{\pgfqpoint{1.786858in}{2.523512in}}%
\pgfpathclose%
\pgfusepath{stroke,fill}%
\end{pgfscope}%
\begin{pgfscope}%
\pgfpathrectangle{\pgfqpoint{0.100000in}{0.212622in}}{\pgfqpoint{3.696000in}{3.696000in}}%
\pgfusepath{clip}%
\pgfsetbuttcap%
\pgfsetroundjoin%
\definecolor{currentfill}{rgb}{0.121569,0.466667,0.705882}%
\pgfsetfillcolor{currentfill}%
\pgfsetfillopacity{0.933050}%
\pgfsetlinewidth{1.003750pt}%
\definecolor{currentstroke}{rgb}{0.121569,0.466667,0.705882}%
\pgfsetstrokecolor{currentstroke}%
\pgfsetstrokeopacity{0.933050}%
\pgfsetdash{}{0pt}%
\pgfpathmoveto{\pgfqpoint{1.170502in}{1.933327in}}%
\pgfpathcurveto{\pgfqpoint{1.178738in}{1.933327in}}{\pgfqpoint{1.186638in}{1.936599in}}{\pgfqpoint{1.192462in}{1.942423in}}%
\pgfpathcurveto{\pgfqpoint{1.198286in}{1.948247in}}{\pgfqpoint{1.201558in}{1.956147in}}{\pgfqpoint{1.201558in}{1.964383in}}%
\pgfpathcurveto{\pgfqpoint{1.201558in}{1.972620in}}{\pgfqpoint{1.198286in}{1.980520in}}{\pgfqpoint{1.192462in}{1.986344in}}%
\pgfpathcurveto{\pgfqpoint{1.186638in}{1.992168in}}{\pgfqpoint{1.178738in}{1.995440in}}{\pgfqpoint{1.170502in}{1.995440in}}%
\pgfpathcurveto{\pgfqpoint{1.162266in}{1.995440in}}{\pgfqpoint{1.154365in}{1.992168in}}{\pgfqpoint{1.148542in}{1.986344in}}%
\pgfpathcurveto{\pgfqpoint{1.142718in}{1.980520in}}{\pgfqpoint{1.139445in}{1.972620in}}{\pgfqpoint{1.139445in}{1.964383in}}%
\pgfpathcurveto{\pgfqpoint{1.139445in}{1.956147in}}{\pgfqpoint{1.142718in}{1.948247in}}{\pgfqpoint{1.148542in}{1.942423in}}%
\pgfpathcurveto{\pgfqpoint{1.154365in}{1.936599in}}{\pgfqpoint{1.162266in}{1.933327in}}{\pgfqpoint{1.170502in}{1.933327in}}%
\pgfpathclose%
\pgfusepath{stroke,fill}%
\end{pgfscope}%
\begin{pgfscope}%
\pgfpathrectangle{\pgfqpoint{0.100000in}{0.212622in}}{\pgfqpoint{3.696000in}{3.696000in}}%
\pgfusepath{clip}%
\pgfsetbuttcap%
\pgfsetroundjoin%
\definecolor{currentfill}{rgb}{0.121569,0.466667,0.705882}%
\pgfsetfillcolor{currentfill}%
\pgfsetfillopacity{0.933834}%
\pgfsetlinewidth{1.003750pt}%
\definecolor{currentstroke}{rgb}{0.121569,0.466667,0.705882}%
\pgfsetstrokecolor{currentstroke}%
\pgfsetstrokeopacity{0.933834}%
\pgfsetdash{}{0pt}%
\pgfpathmoveto{\pgfqpoint{2.121887in}{2.437057in}}%
\pgfpathcurveto{\pgfqpoint{2.130123in}{2.437057in}}{\pgfqpoint{2.138024in}{2.440329in}}{\pgfqpoint{2.143847in}{2.446153in}}%
\pgfpathcurveto{\pgfqpoint{2.149671in}{2.451977in}}{\pgfqpoint{2.152944in}{2.459877in}}{\pgfqpoint{2.152944in}{2.468114in}}%
\pgfpathcurveto{\pgfqpoint{2.152944in}{2.476350in}}{\pgfqpoint{2.149671in}{2.484250in}}{\pgfqpoint{2.143847in}{2.490074in}}%
\pgfpathcurveto{\pgfqpoint{2.138024in}{2.495898in}}{\pgfqpoint{2.130123in}{2.499170in}}{\pgfqpoint{2.121887in}{2.499170in}}%
\pgfpathcurveto{\pgfqpoint{2.113651in}{2.499170in}}{\pgfqpoint{2.105751in}{2.495898in}}{\pgfqpoint{2.099927in}{2.490074in}}%
\pgfpathcurveto{\pgfqpoint{2.094103in}{2.484250in}}{\pgfqpoint{2.090831in}{2.476350in}}{\pgfqpoint{2.090831in}{2.468114in}}%
\pgfpathcurveto{\pgfqpoint{2.090831in}{2.459877in}}{\pgfqpoint{2.094103in}{2.451977in}}{\pgfqpoint{2.099927in}{2.446153in}}%
\pgfpathcurveto{\pgfqpoint{2.105751in}{2.440329in}}{\pgfqpoint{2.113651in}{2.437057in}}{\pgfqpoint{2.121887in}{2.437057in}}%
\pgfpathclose%
\pgfusepath{stroke,fill}%
\end{pgfscope}%
\begin{pgfscope}%
\pgfpathrectangle{\pgfqpoint{0.100000in}{0.212622in}}{\pgfqpoint{3.696000in}{3.696000in}}%
\pgfusepath{clip}%
\pgfsetbuttcap%
\pgfsetroundjoin%
\definecolor{currentfill}{rgb}{0.121569,0.466667,0.705882}%
\pgfsetfillcolor{currentfill}%
\pgfsetfillopacity{0.933964}%
\pgfsetlinewidth{1.003750pt}%
\definecolor{currentstroke}{rgb}{0.121569,0.466667,0.705882}%
\pgfsetstrokecolor{currentstroke}%
\pgfsetstrokeopacity{0.933964}%
\pgfsetdash{}{0pt}%
\pgfpathmoveto{\pgfqpoint{1.793774in}{2.519880in}}%
\pgfpathcurveto{\pgfqpoint{1.802011in}{2.519880in}}{\pgfqpoint{1.809911in}{2.523153in}}{\pgfqpoint{1.815735in}{2.528977in}}%
\pgfpathcurveto{\pgfqpoint{1.821559in}{2.534801in}}{\pgfqpoint{1.824831in}{2.542701in}}{\pgfqpoint{1.824831in}{2.550937in}}%
\pgfpathcurveto{\pgfqpoint{1.824831in}{2.559173in}}{\pgfqpoint{1.821559in}{2.567073in}}{\pgfqpoint{1.815735in}{2.572897in}}%
\pgfpathcurveto{\pgfqpoint{1.809911in}{2.578721in}}{\pgfqpoint{1.802011in}{2.581993in}}{\pgfqpoint{1.793774in}{2.581993in}}%
\pgfpathcurveto{\pgfqpoint{1.785538in}{2.581993in}}{\pgfqpoint{1.777638in}{2.578721in}}{\pgfqpoint{1.771814in}{2.572897in}}%
\pgfpathcurveto{\pgfqpoint{1.765990in}{2.567073in}}{\pgfqpoint{1.762718in}{2.559173in}}{\pgfqpoint{1.762718in}{2.550937in}}%
\pgfpathcurveto{\pgfqpoint{1.762718in}{2.542701in}}{\pgfqpoint{1.765990in}{2.534801in}}{\pgfqpoint{1.771814in}{2.528977in}}%
\pgfpathcurveto{\pgfqpoint{1.777638in}{2.523153in}}{\pgfqpoint{1.785538in}{2.519880in}}{\pgfqpoint{1.793774in}{2.519880in}}%
\pgfpathclose%
\pgfusepath{stroke,fill}%
\end{pgfscope}%
\begin{pgfscope}%
\pgfpathrectangle{\pgfqpoint{0.100000in}{0.212622in}}{\pgfqpoint{3.696000in}{3.696000in}}%
\pgfusepath{clip}%
\pgfsetbuttcap%
\pgfsetroundjoin%
\definecolor{currentfill}{rgb}{0.121569,0.466667,0.705882}%
\pgfsetfillcolor{currentfill}%
\pgfsetfillopacity{0.934051}%
\pgfsetlinewidth{1.003750pt}%
\definecolor{currentstroke}{rgb}{0.121569,0.466667,0.705882}%
\pgfsetstrokecolor{currentstroke}%
\pgfsetstrokeopacity{0.934051}%
\pgfsetdash{}{0pt}%
\pgfpathmoveto{\pgfqpoint{1.180710in}{1.927183in}}%
\pgfpathcurveto{\pgfqpoint{1.188946in}{1.927183in}}{\pgfqpoint{1.196846in}{1.930456in}}{\pgfqpoint{1.202670in}{1.936280in}}%
\pgfpathcurveto{\pgfqpoint{1.208494in}{1.942104in}}{\pgfqpoint{1.211766in}{1.950004in}}{\pgfqpoint{1.211766in}{1.958240in}}%
\pgfpathcurveto{\pgfqpoint{1.211766in}{1.966476in}}{\pgfqpoint{1.208494in}{1.974376in}}{\pgfqpoint{1.202670in}{1.980200in}}%
\pgfpathcurveto{\pgfqpoint{1.196846in}{1.986024in}}{\pgfqpoint{1.188946in}{1.989296in}}{\pgfqpoint{1.180710in}{1.989296in}}%
\pgfpathcurveto{\pgfqpoint{1.172474in}{1.989296in}}{\pgfqpoint{1.164574in}{1.986024in}}{\pgfqpoint{1.158750in}{1.980200in}}%
\pgfpathcurveto{\pgfqpoint{1.152926in}{1.974376in}}{\pgfqpoint{1.149653in}{1.966476in}}{\pgfqpoint{1.149653in}{1.958240in}}%
\pgfpathcurveto{\pgfqpoint{1.149653in}{1.950004in}}{\pgfqpoint{1.152926in}{1.942104in}}{\pgfqpoint{1.158750in}{1.936280in}}%
\pgfpathcurveto{\pgfqpoint{1.164574in}{1.930456in}}{\pgfqpoint{1.172474in}{1.927183in}}{\pgfqpoint{1.180710in}{1.927183in}}%
\pgfpathclose%
\pgfusepath{stroke,fill}%
\end{pgfscope}%
\begin{pgfscope}%
\pgfpathrectangle{\pgfqpoint{0.100000in}{0.212622in}}{\pgfqpoint{3.696000in}{3.696000in}}%
\pgfusepath{clip}%
\pgfsetbuttcap%
\pgfsetroundjoin%
\definecolor{currentfill}{rgb}{0.121569,0.466667,0.705882}%
\pgfsetfillcolor{currentfill}%
\pgfsetfillopacity{0.934109}%
\pgfsetlinewidth{1.003750pt}%
\definecolor{currentstroke}{rgb}{0.121569,0.466667,0.705882}%
\pgfsetstrokecolor{currentstroke}%
\pgfsetstrokeopacity{0.934109}%
\pgfsetdash{}{0pt}%
\pgfpathmoveto{\pgfqpoint{2.617182in}{1.235619in}}%
\pgfpathcurveto{\pgfqpoint{2.625418in}{1.235619in}}{\pgfqpoint{2.633318in}{1.238891in}}{\pgfqpoint{2.639142in}{1.244715in}}%
\pgfpathcurveto{\pgfqpoint{2.644966in}{1.250539in}}{\pgfqpoint{2.648238in}{1.258439in}}{\pgfqpoint{2.648238in}{1.266675in}}%
\pgfpathcurveto{\pgfqpoint{2.648238in}{1.274911in}}{\pgfqpoint{2.644966in}{1.282811in}}{\pgfqpoint{2.639142in}{1.288635in}}%
\pgfpathcurveto{\pgfqpoint{2.633318in}{1.294459in}}{\pgfqpoint{2.625418in}{1.297732in}}{\pgfqpoint{2.617182in}{1.297732in}}%
\pgfpathcurveto{\pgfqpoint{2.608946in}{1.297732in}}{\pgfqpoint{2.601046in}{1.294459in}}{\pgfqpoint{2.595222in}{1.288635in}}%
\pgfpathcurveto{\pgfqpoint{2.589398in}{1.282811in}}{\pgfqpoint{2.586125in}{1.274911in}}{\pgfqpoint{2.586125in}{1.266675in}}%
\pgfpathcurveto{\pgfqpoint{2.586125in}{1.258439in}}{\pgfqpoint{2.589398in}{1.250539in}}{\pgfqpoint{2.595222in}{1.244715in}}%
\pgfpathcurveto{\pgfqpoint{2.601046in}{1.238891in}}{\pgfqpoint{2.608946in}{1.235619in}}{\pgfqpoint{2.617182in}{1.235619in}}%
\pgfpathclose%
\pgfusepath{stroke,fill}%
\end{pgfscope}%
\begin{pgfscope}%
\pgfpathrectangle{\pgfqpoint{0.100000in}{0.212622in}}{\pgfqpoint{3.696000in}{3.696000in}}%
\pgfusepath{clip}%
\pgfsetbuttcap%
\pgfsetroundjoin%
\definecolor{currentfill}{rgb}{0.121569,0.466667,0.705882}%
\pgfsetfillcolor{currentfill}%
\pgfsetfillopacity{0.934401}%
\pgfsetlinewidth{1.003750pt}%
\definecolor{currentstroke}{rgb}{0.121569,0.466667,0.705882}%
\pgfsetstrokecolor{currentstroke}%
\pgfsetstrokeopacity{0.934401}%
\pgfsetdash{}{0pt}%
\pgfpathmoveto{\pgfqpoint{2.120759in}{2.436245in}}%
\pgfpathcurveto{\pgfqpoint{2.128995in}{2.436245in}}{\pgfqpoint{2.136895in}{2.439517in}}{\pgfqpoint{2.142719in}{2.445341in}}%
\pgfpathcurveto{\pgfqpoint{2.148543in}{2.451165in}}{\pgfqpoint{2.151815in}{2.459065in}}{\pgfqpoint{2.151815in}{2.467301in}}%
\pgfpathcurveto{\pgfqpoint{2.151815in}{2.475537in}}{\pgfqpoint{2.148543in}{2.483437in}}{\pgfqpoint{2.142719in}{2.489261in}}%
\pgfpathcurveto{\pgfqpoint{2.136895in}{2.495085in}}{\pgfqpoint{2.128995in}{2.498358in}}{\pgfqpoint{2.120759in}{2.498358in}}%
\pgfpathcurveto{\pgfqpoint{2.112523in}{2.498358in}}{\pgfqpoint{2.104623in}{2.495085in}}{\pgfqpoint{2.098799in}{2.489261in}}%
\pgfpathcurveto{\pgfqpoint{2.092975in}{2.483437in}}{\pgfqpoint{2.089702in}{2.475537in}}{\pgfqpoint{2.089702in}{2.467301in}}%
\pgfpathcurveto{\pgfqpoint{2.089702in}{2.459065in}}{\pgfqpoint{2.092975in}{2.451165in}}{\pgfqpoint{2.098799in}{2.445341in}}%
\pgfpathcurveto{\pgfqpoint{2.104623in}{2.439517in}}{\pgfqpoint{2.112523in}{2.436245in}}{\pgfqpoint{2.120759in}{2.436245in}}%
\pgfpathclose%
\pgfusepath{stroke,fill}%
\end{pgfscope}%
\begin{pgfscope}%
\pgfpathrectangle{\pgfqpoint{0.100000in}{0.212622in}}{\pgfqpoint{3.696000in}{3.696000in}}%
\pgfusepath{clip}%
\pgfsetbuttcap%
\pgfsetroundjoin%
\definecolor{currentfill}{rgb}{0.121569,0.466667,0.705882}%
\pgfsetfillcolor{currentfill}%
\pgfsetfillopacity{0.934910}%
\pgfsetlinewidth{1.003750pt}%
\definecolor{currentstroke}{rgb}{0.121569,0.466667,0.705882}%
\pgfsetstrokecolor{currentstroke}%
\pgfsetstrokeopacity{0.934910}%
\pgfsetdash{}{0pt}%
\pgfpathmoveto{\pgfqpoint{1.801383in}{2.515965in}}%
\pgfpathcurveto{\pgfqpoint{1.809619in}{2.515965in}}{\pgfqpoint{1.817519in}{2.519238in}}{\pgfqpoint{1.823343in}{2.525061in}}%
\pgfpathcurveto{\pgfqpoint{1.829167in}{2.530885in}}{\pgfqpoint{1.832439in}{2.538785in}}{\pgfqpoint{1.832439in}{2.547022in}}%
\pgfpathcurveto{\pgfqpoint{1.832439in}{2.555258in}}{\pgfqpoint{1.829167in}{2.563158in}}{\pgfqpoint{1.823343in}{2.568982in}}%
\pgfpathcurveto{\pgfqpoint{1.817519in}{2.574806in}}{\pgfqpoint{1.809619in}{2.578078in}}{\pgfqpoint{1.801383in}{2.578078in}}%
\pgfpathcurveto{\pgfqpoint{1.793147in}{2.578078in}}{\pgfqpoint{1.785247in}{2.574806in}}{\pgfqpoint{1.779423in}{2.568982in}}%
\pgfpathcurveto{\pgfqpoint{1.773599in}{2.563158in}}{\pgfqpoint{1.770326in}{2.555258in}}{\pgfqpoint{1.770326in}{2.547022in}}%
\pgfpathcurveto{\pgfqpoint{1.770326in}{2.538785in}}{\pgfqpoint{1.773599in}{2.530885in}}{\pgfqpoint{1.779423in}{2.525061in}}%
\pgfpathcurveto{\pgfqpoint{1.785247in}{2.519238in}}{\pgfqpoint{1.793147in}{2.515965in}}{\pgfqpoint{1.801383in}{2.515965in}}%
\pgfpathclose%
\pgfusepath{stroke,fill}%
\end{pgfscope}%
\begin{pgfscope}%
\pgfpathrectangle{\pgfqpoint{0.100000in}{0.212622in}}{\pgfqpoint{3.696000in}{3.696000in}}%
\pgfusepath{clip}%
\pgfsetbuttcap%
\pgfsetroundjoin%
\definecolor{currentfill}{rgb}{0.121569,0.466667,0.705882}%
\pgfsetfillcolor{currentfill}%
\pgfsetfillopacity{0.935181}%
\pgfsetlinewidth{1.003750pt}%
\definecolor{currentstroke}{rgb}{0.121569,0.466667,0.705882}%
\pgfsetstrokecolor{currentstroke}%
\pgfsetstrokeopacity{0.935181}%
\pgfsetdash{}{0pt}%
\pgfpathmoveto{\pgfqpoint{1.191773in}{1.920258in}}%
\pgfpathcurveto{\pgfqpoint{1.200009in}{1.920258in}}{\pgfqpoint{1.207909in}{1.923530in}}{\pgfqpoint{1.213733in}{1.929354in}}%
\pgfpathcurveto{\pgfqpoint{1.219557in}{1.935178in}}{\pgfqpoint{1.222829in}{1.943078in}}{\pgfqpoint{1.222829in}{1.951314in}}%
\pgfpathcurveto{\pgfqpoint{1.222829in}{1.959550in}}{\pgfqpoint{1.219557in}{1.967450in}}{\pgfqpoint{1.213733in}{1.973274in}}%
\pgfpathcurveto{\pgfqpoint{1.207909in}{1.979098in}}{\pgfqpoint{1.200009in}{1.982371in}}{\pgfqpoint{1.191773in}{1.982371in}}%
\pgfpathcurveto{\pgfqpoint{1.183536in}{1.982371in}}{\pgfqpoint{1.175636in}{1.979098in}}{\pgfqpoint{1.169812in}{1.973274in}}%
\pgfpathcurveto{\pgfqpoint{1.163988in}{1.967450in}}{\pgfqpoint{1.160716in}{1.959550in}}{\pgfqpoint{1.160716in}{1.951314in}}%
\pgfpathcurveto{\pgfqpoint{1.160716in}{1.943078in}}{\pgfqpoint{1.163988in}{1.935178in}}{\pgfqpoint{1.169812in}{1.929354in}}%
\pgfpathcurveto{\pgfqpoint{1.175636in}{1.923530in}}{\pgfqpoint{1.183536in}{1.920258in}}{\pgfqpoint{1.191773in}{1.920258in}}%
\pgfpathclose%
\pgfusepath{stroke,fill}%
\end{pgfscope}%
\begin{pgfscope}%
\pgfpathrectangle{\pgfqpoint{0.100000in}{0.212622in}}{\pgfqpoint{3.696000in}{3.696000in}}%
\pgfusepath{clip}%
\pgfsetbuttcap%
\pgfsetroundjoin%
\definecolor{currentfill}{rgb}{0.121569,0.466667,0.705882}%
\pgfsetfillcolor{currentfill}%
\pgfsetfillopacity{0.935196}%
\pgfsetlinewidth{1.003750pt}%
\definecolor{currentstroke}{rgb}{0.121569,0.466667,0.705882}%
\pgfsetstrokecolor{currentstroke}%
\pgfsetstrokeopacity{0.935196}%
\pgfsetdash{}{0pt}%
\pgfpathmoveto{\pgfqpoint{2.615004in}{1.233185in}}%
\pgfpathcurveto{\pgfqpoint{2.623240in}{1.233185in}}{\pgfqpoint{2.631140in}{1.236458in}}{\pgfqpoint{2.636964in}{1.242282in}}%
\pgfpathcurveto{\pgfqpoint{2.642788in}{1.248106in}}{\pgfqpoint{2.646060in}{1.256006in}}{\pgfqpoint{2.646060in}{1.264242in}}%
\pgfpathcurveto{\pgfqpoint{2.646060in}{1.272478in}}{\pgfqpoint{2.642788in}{1.280378in}}{\pgfqpoint{2.636964in}{1.286202in}}%
\pgfpathcurveto{\pgfqpoint{2.631140in}{1.292026in}}{\pgfqpoint{2.623240in}{1.295298in}}{\pgfqpoint{2.615004in}{1.295298in}}%
\pgfpathcurveto{\pgfqpoint{2.606768in}{1.295298in}}{\pgfqpoint{2.598868in}{1.292026in}}{\pgfqpoint{2.593044in}{1.286202in}}%
\pgfpathcurveto{\pgfqpoint{2.587220in}{1.280378in}}{\pgfqpoint{2.583947in}{1.272478in}}{\pgfqpoint{2.583947in}{1.264242in}}%
\pgfpathcurveto{\pgfqpoint{2.583947in}{1.256006in}}{\pgfqpoint{2.587220in}{1.248106in}}{\pgfqpoint{2.593044in}{1.242282in}}%
\pgfpathcurveto{\pgfqpoint{2.598868in}{1.236458in}}{\pgfqpoint{2.606768in}{1.233185in}}{\pgfqpoint{2.615004in}{1.233185in}}%
\pgfpathclose%
\pgfusepath{stroke,fill}%
\end{pgfscope}%
\begin{pgfscope}%
\pgfpathrectangle{\pgfqpoint{0.100000in}{0.212622in}}{\pgfqpoint{3.696000in}{3.696000in}}%
\pgfusepath{clip}%
\pgfsetbuttcap%
\pgfsetroundjoin%
\definecolor{currentfill}{rgb}{0.121569,0.466667,0.705882}%
\pgfsetfillcolor{currentfill}%
\pgfsetfillopacity{0.935459}%
\pgfsetlinewidth{1.003750pt}%
\definecolor{currentstroke}{rgb}{0.121569,0.466667,0.705882}%
\pgfsetstrokecolor{currentstroke}%
\pgfsetstrokeopacity{0.935459}%
\pgfsetdash{}{0pt}%
\pgfpathmoveto{\pgfqpoint{2.118681in}{2.434955in}}%
\pgfpathcurveto{\pgfqpoint{2.126917in}{2.434955in}}{\pgfqpoint{2.134817in}{2.438228in}}{\pgfqpoint{2.140641in}{2.444052in}}%
\pgfpathcurveto{\pgfqpoint{2.146465in}{2.449876in}}{\pgfqpoint{2.149737in}{2.457776in}}{\pgfqpoint{2.149737in}{2.466012in}}%
\pgfpathcurveto{\pgfqpoint{2.149737in}{2.474248in}}{\pgfqpoint{2.146465in}{2.482148in}}{\pgfqpoint{2.140641in}{2.487972in}}%
\pgfpathcurveto{\pgfqpoint{2.134817in}{2.493796in}}{\pgfqpoint{2.126917in}{2.497068in}}{\pgfqpoint{2.118681in}{2.497068in}}%
\pgfpathcurveto{\pgfqpoint{2.110444in}{2.497068in}}{\pgfqpoint{2.102544in}{2.493796in}}{\pgfqpoint{2.096720in}{2.487972in}}%
\pgfpathcurveto{\pgfqpoint{2.090896in}{2.482148in}}{\pgfqpoint{2.087624in}{2.474248in}}{\pgfqpoint{2.087624in}{2.466012in}}%
\pgfpathcurveto{\pgfqpoint{2.087624in}{2.457776in}}{\pgfqpoint{2.090896in}{2.449876in}}{\pgfqpoint{2.096720in}{2.444052in}}%
\pgfpathcurveto{\pgfqpoint{2.102544in}{2.438228in}}{\pgfqpoint{2.110444in}{2.434955in}}{\pgfqpoint{2.118681in}{2.434955in}}%
\pgfpathclose%
\pgfusepath{stroke,fill}%
\end{pgfscope}%
\begin{pgfscope}%
\pgfpathrectangle{\pgfqpoint{0.100000in}{0.212622in}}{\pgfqpoint{3.696000in}{3.696000in}}%
\pgfusepath{clip}%
\pgfsetbuttcap%
\pgfsetroundjoin%
\definecolor{currentfill}{rgb}{0.121569,0.466667,0.705882}%
\pgfsetfillcolor{currentfill}%
\pgfsetfillopacity{0.935843}%
\pgfsetlinewidth{1.003750pt}%
\definecolor{currentstroke}{rgb}{0.121569,0.466667,0.705882}%
\pgfsetstrokecolor{currentstroke}%
\pgfsetstrokeopacity{0.935843}%
\pgfsetdash{}{0pt}%
\pgfpathmoveto{\pgfqpoint{2.117935in}{2.434510in}}%
\pgfpathcurveto{\pgfqpoint{2.126171in}{2.434510in}}{\pgfqpoint{2.134071in}{2.437782in}}{\pgfqpoint{2.139895in}{2.443606in}}%
\pgfpathcurveto{\pgfqpoint{2.145719in}{2.449430in}}{\pgfqpoint{2.148991in}{2.457330in}}{\pgfqpoint{2.148991in}{2.465566in}}%
\pgfpathcurveto{\pgfqpoint{2.148991in}{2.473803in}}{\pgfqpoint{2.145719in}{2.481703in}}{\pgfqpoint{2.139895in}{2.487527in}}%
\pgfpathcurveto{\pgfqpoint{2.134071in}{2.493350in}}{\pgfqpoint{2.126171in}{2.496623in}}{\pgfqpoint{2.117935in}{2.496623in}}%
\pgfpathcurveto{\pgfqpoint{2.109698in}{2.496623in}}{\pgfqpoint{2.101798in}{2.493350in}}{\pgfqpoint{2.095974in}{2.487527in}}%
\pgfpathcurveto{\pgfqpoint{2.090150in}{2.481703in}}{\pgfqpoint{2.086878in}{2.473803in}}{\pgfqpoint{2.086878in}{2.465566in}}%
\pgfpathcurveto{\pgfqpoint{2.086878in}{2.457330in}}{\pgfqpoint{2.090150in}{2.449430in}}{\pgfqpoint{2.095974in}{2.443606in}}%
\pgfpathcurveto{\pgfqpoint{2.101798in}{2.437782in}}{\pgfqpoint{2.109698in}{2.434510in}}{\pgfqpoint{2.117935in}{2.434510in}}%
\pgfpathclose%
\pgfusepath{stroke,fill}%
\end{pgfscope}%
\begin{pgfscope}%
\pgfpathrectangle{\pgfqpoint{0.100000in}{0.212622in}}{\pgfqpoint{3.696000in}{3.696000in}}%
\pgfusepath{clip}%
\pgfsetbuttcap%
\pgfsetroundjoin%
\definecolor{currentfill}{rgb}{0.121569,0.466667,0.705882}%
\pgfsetfillcolor{currentfill}%
\pgfsetfillopacity{0.936016}%
\pgfsetlinewidth{1.003750pt}%
\definecolor{currentstroke}{rgb}{0.121569,0.466667,0.705882}%
\pgfsetstrokecolor{currentstroke}%
\pgfsetstrokeopacity{0.936016}%
\pgfsetdash{}{0pt}%
\pgfpathmoveto{\pgfqpoint{1.809601in}{2.511335in}}%
\pgfpathcurveto{\pgfqpoint{1.817838in}{2.511335in}}{\pgfqpoint{1.825738in}{2.514607in}}{\pgfqpoint{1.831562in}{2.520431in}}%
\pgfpathcurveto{\pgfqpoint{1.837386in}{2.526255in}}{\pgfqpoint{1.840658in}{2.534155in}}{\pgfqpoint{1.840658in}{2.542391in}}%
\pgfpathcurveto{\pgfqpoint{1.840658in}{2.550628in}}{\pgfqpoint{1.837386in}{2.558528in}}{\pgfqpoint{1.831562in}{2.564352in}}%
\pgfpathcurveto{\pgfqpoint{1.825738in}{2.570176in}}{\pgfqpoint{1.817838in}{2.573448in}}{\pgfqpoint{1.809601in}{2.573448in}}%
\pgfpathcurveto{\pgfqpoint{1.801365in}{2.573448in}}{\pgfqpoint{1.793465in}{2.570176in}}{\pgfqpoint{1.787641in}{2.564352in}}%
\pgfpathcurveto{\pgfqpoint{1.781817in}{2.558528in}}{\pgfqpoint{1.778545in}{2.550628in}}{\pgfqpoint{1.778545in}{2.542391in}}%
\pgfpathcurveto{\pgfqpoint{1.778545in}{2.534155in}}{\pgfqpoint{1.781817in}{2.526255in}}{\pgfqpoint{1.787641in}{2.520431in}}%
\pgfpathcurveto{\pgfqpoint{1.793465in}{2.514607in}}{\pgfqpoint{1.801365in}{2.511335in}}{\pgfqpoint{1.809601in}{2.511335in}}%
\pgfpathclose%
\pgfusepath{stroke,fill}%
\end{pgfscope}%
\begin{pgfscope}%
\pgfpathrectangle{\pgfqpoint{0.100000in}{0.212622in}}{\pgfqpoint{3.696000in}{3.696000in}}%
\pgfusepath{clip}%
\pgfsetbuttcap%
\pgfsetroundjoin%
\definecolor{currentfill}{rgb}{0.121569,0.466667,0.705882}%
\pgfsetfillcolor{currentfill}%
\pgfsetfillopacity{0.936067}%
\pgfsetlinewidth{1.003750pt}%
\definecolor{currentstroke}{rgb}{0.121569,0.466667,0.705882}%
\pgfsetstrokecolor{currentstroke}%
\pgfsetstrokeopacity{0.936067}%
\pgfsetdash{}{0pt}%
\pgfpathmoveto{\pgfqpoint{2.117497in}{2.434226in}}%
\pgfpathcurveto{\pgfqpoint{2.125733in}{2.434226in}}{\pgfqpoint{2.133633in}{2.437499in}}{\pgfqpoint{2.139457in}{2.443323in}}%
\pgfpathcurveto{\pgfqpoint{2.145281in}{2.449147in}}{\pgfqpoint{2.148554in}{2.457047in}}{\pgfqpoint{2.148554in}{2.465283in}}%
\pgfpathcurveto{\pgfqpoint{2.148554in}{2.473519in}}{\pgfqpoint{2.145281in}{2.481419in}}{\pgfqpoint{2.139457in}{2.487243in}}%
\pgfpathcurveto{\pgfqpoint{2.133633in}{2.493067in}}{\pgfqpoint{2.125733in}{2.496339in}}{\pgfqpoint{2.117497in}{2.496339in}}%
\pgfpathcurveto{\pgfqpoint{2.109261in}{2.496339in}}{\pgfqpoint{2.101361in}{2.493067in}}{\pgfqpoint{2.095537in}{2.487243in}}%
\pgfpathcurveto{\pgfqpoint{2.089713in}{2.481419in}}{\pgfqpoint{2.086441in}{2.473519in}}{\pgfqpoint{2.086441in}{2.465283in}}%
\pgfpathcurveto{\pgfqpoint{2.086441in}{2.457047in}}{\pgfqpoint{2.089713in}{2.449147in}}{\pgfqpoint{2.095537in}{2.443323in}}%
\pgfpathcurveto{\pgfqpoint{2.101361in}{2.437499in}}{\pgfqpoint{2.109261in}{2.434226in}}{\pgfqpoint{2.117497in}{2.434226in}}%
\pgfpathclose%
\pgfusepath{stroke,fill}%
\end{pgfscope}%
\begin{pgfscope}%
\pgfpathrectangle{\pgfqpoint{0.100000in}{0.212622in}}{\pgfqpoint{3.696000in}{3.696000in}}%
\pgfusepath{clip}%
\pgfsetbuttcap%
\pgfsetroundjoin%
\definecolor{currentfill}{rgb}{0.121569,0.466667,0.705882}%
\pgfsetfillcolor{currentfill}%
\pgfsetfillopacity{0.936118}%
\pgfsetlinewidth{1.003750pt}%
\definecolor{currentstroke}{rgb}{0.121569,0.466667,0.705882}%
\pgfsetstrokecolor{currentstroke}%
\pgfsetstrokeopacity{0.936118}%
\pgfsetdash{}{0pt}%
\pgfpathmoveto{\pgfqpoint{2.117398in}{2.434166in}}%
\pgfpathcurveto{\pgfqpoint{2.125634in}{2.434166in}}{\pgfqpoint{2.133534in}{2.437438in}}{\pgfqpoint{2.139358in}{2.443262in}}%
\pgfpathcurveto{\pgfqpoint{2.145182in}{2.449086in}}{\pgfqpoint{2.148454in}{2.456986in}}{\pgfqpoint{2.148454in}{2.465222in}}%
\pgfpathcurveto{\pgfqpoint{2.148454in}{2.473458in}}{\pgfqpoint{2.145182in}{2.481358in}}{\pgfqpoint{2.139358in}{2.487182in}}%
\pgfpathcurveto{\pgfqpoint{2.133534in}{2.493006in}}{\pgfqpoint{2.125634in}{2.496279in}}{\pgfqpoint{2.117398in}{2.496279in}}%
\pgfpathcurveto{\pgfqpoint{2.109162in}{2.496279in}}{\pgfqpoint{2.101262in}{2.493006in}}{\pgfqpoint{2.095438in}{2.487182in}}%
\pgfpathcurveto{\pgfqpoint{2.089614in}{2.481358in}}{\pgfqpoint{2.086341in}{2.473458in}}{\pgfqpoint{2.086341in}{2.465222in}}%
\pgfpathcurveto{\pgfqpoint{2.086341in}{2.456986in}}{\pgfqpoint{2.089614in}{2.449086in}}{\pgfqpoint{2.095438in}{2.443262in}}%
\pgfpathcurveto{\pgfqpoint{2.101262in}{2.437438in}}{\pgfqpoint{2.109162in}{2.434166in}}{\pgfqpoint{2.117398in}{2.434166in}}%
\pgfpathclose%
\pgfusepath{stroke,fill}%
\end{pgfscope}%
\begin{pgfscope}%
\pgfpathrectangle{\pgfqpoint{0.100000in}{0.212622in}}{\pgfqpoint{3.696000in}{3.696000in}}%
\pgfusepath{clip}%
\pgfsetbuttcap%
\pgfsetroundjoin%
\definecolor{currentfill}{rgb}{0.121569,0.466667,0.705882}%
\pgfsetfillcolor{currentfill}%
\pgfsetfillopacity{0.936141}%
\pgfsetlinewidth{1.003750pt}%
\definecolor{currentstroke}{rgb}{0.121569,0.466667,0.705882}%
\pgfsetstrokecolor{currentstroke}%
\pgfsetstrokeopacity{0.936141}%
\pgfsetdash{}{0pt}%
\pgfpathmoveto{\pgfqpoint{2.613071in}{1.231167in}}%
\pgfpathcurveto{\pgfqpoint{2.621307in}{1.231167in}}{\pgfqpoint{2.629207in}{1.234439in}}{\pgfqpoint{2.635031in}{1.240263in}}%
\pgfpathcurveto{\pgfqpoint{2.640855in}{1.246087in}}{\pgfqpoint{2.644127in}{1.253987in}}{\pgfqpoint{2.644127in}{1.262223in}}%
\pgfpathcurveto{\pgfqpoint{2.644127in}{1.270459in}}{\pgfqpoint{2.640855in}{1.278359in}}{\pgfqpoint{2.635031in}{1.284183in}}%
\pgfpathcurveto{\pgfqpoint{2.629207in}{1.290007in}}{\pgfqpoint{2.621307in}{1.293280in}}{\pgfqpoint{2.613071in}{1.293280in}}%
\pgfpathcurveto{\pgfqpoint{2.604835in}{1.293280in}}{\pgfqpoint{2.596935in}{1.290007in}}{\pgfqpoint{2.591111in}{1.284183in}}%
\pgfpathcurveto{\pgfqpoint{2.585287in}{1.278359in}}{\pgfqpoint{2.582014in}{1.270459in}}{\pgfqpoint{2.582014in}{1.262223in}}%
\pgfpathcurveto{\pgfqpoint{2.582014in}{1.253987in}}{\pgfqpoint{2.585287in}{1.246087in}}{\pgfqpoint{2.591111in}{1.240263in}}%
\pgfpathcurveto{\pgfqpoint{2.596935in}{1.234439in}}{\pgfqpoint{2.604835in}{1.231167in}}{\pgfqpoint{2.613071in}{1.231167in}}%
\pgfpathclose%
\pgfusepath{stroke,fill}%
\end{pgfscope}%
\begin{pgfscope}%
\pgfpathrectangle{\pgfqpoint{0.100000in}{0.212622in}}{\pgfqpoint{3.696000in}{3.696000in}}%
\pgfusepath{clip}%
\pgfsetbuttcap%
\pgfsetroundjoin%
\definecolor{currentfill}{rgb}{0.121569,0.466667,0.705882}%
\pgfsetfillcolor{currentfill}%
\pgfsetfillopacity{0.936203}%
\pgfsetlinewidth{1.003750pt}%
\definecolor{currentstroke}{rgb}{0.121569,0.466667,0.705882}%
\pgfsetstrokecolor{currentstroke}%
\pgfsetstrokeopacity{0.936203}%
\pgfsetdash{}{0pt}%
\pgfpathmoveto{\pgfqpoint{2.117217in}{2.434020in}}%
\pgfpathcurveto{\pgfqpoint{2.125454in}{2.434020in}}{\pgfqpoint{2.133354in}{2.437292in}}{\pgfqpoint{2.139178in}{2.443116in}}%
\pgfpathcurveto{\pgfqpoint{2.145001in}{2.448940in}}{\pgfqpoint{2.148274in}{2.456840in}}{\pgfqpoint{2.148274in}{2.465076in}}%
\pgfpathcurveto{\pgfqpoint{2.148274in}{2.473312in}}{\pgfqpoint{2.145001in}{2.481212in}}{\pgfqpoint{2.139178in}{2.487036in}}%
\pgfpathcurveto{\pgfqpoint{2.133354in}{2.492860in}}{\pgfqpoint{2.125454in}{2.496133in}}{\pgfqpoint{2.117217in}{2.496133in}}%
\pgfpathcurveto{\pgfqpoint{2.108981in}{2.496133in}}{\pgfqpoint{2.101081in}{2.492860in}}{\pgfqpoint{2.095257in}{2.487036in}}%
\pgfpathcurveto{\pgfqpoint{2.089433in}{2.481212in}}{\pgfqpoint{2.086161in}{2.473312in}}{\pgfqpoint{2.086161in}{2.465076in}}%
\pgfpathcurveto{\pgfqpoint{2.086161in}{2.456840in}}{\pgfqpoint{2.089433in}{2.448940in}}{\pgfqpoint{2.095257in}{2.443116in}}%
\pgfpathcurveto{\pgfqpoint{2.101081in}{2.437292in}}{\pgfqpoint{2.108981in}{2.434020in}}{\pgfqpoint{2.117217in}{2.434020in}}%
\pgfpathclose%
\pgfusepath{stroke,fill}%
\end{pgfscope}%
\begin{pgfscope}%
\pgfpathrectangle{\pgfqpoint{0.100000in}{0.212622in}}{\pgfqpoint{3.696000in}{3.696000in}}%
\pgfusepath{clip}%
\pgfsetbuttcap%
\pgfsetroundjoin%
\definecolor{currentfill}{rgb}{0.121569,0.466667,0.705882}%
\pgfsetfillcolor{currentfill}%
\pgfsetfillopacity{0.936357}%
\pgfsetlinewidth{1.003750pt}%
\definecolor{currentstroke}{rgb}{0.121569,0.466667,0.705882}%
\pgfsetstrokecolor{currentstroke}%
\pgfsetstrokeopacity{0.936357}%
\pgfsetdash{}{0pt}%
\pgfpathmoveto{\pgfqpoint{2.116886in}{2.433743in}}%
\pgfpathcurveto{\pgfqpoint{2.125122in}{2.433743in}}{\pgfqpoint{2.133022in}{2.437015in}}{\pgfqpoint{2.138846in}{2.442839in}}%
\pgfpathcurveto{\pgfqpoint{2.144670in}{2.448663in}}{\pgfqpoint{2.147942in}{2.456563in}}{\pgfqpoint{2.147942in}{2.464800in}}%
\pgfpathcurveto{\pgfqpoint{2.147942in}{2.473036in}}{\pgfqpoint{2.144670in}{2.480936in}}{\pgfqpoint{2.138846in}{2.486760in}}%
\pgfpathcurveto{\pgfqpoint{2.133022in}{2.492584in}}{\pgfqpoint{2.125122in}{2.495856in}}{\pgfqpoint{2.116886in}{2.495856in}}%
\pgfpathcurveto{\pgfqpoint{2.108649in}{2.495856in}}{\pgfqpoint{2.100749in}{2.492584in}}{\pgfqpoint{2.094925in}{2.486760in}}%
\pgfpathcurveto{\pgfqpoint{2.089101in}{2.480936in}}{\pgfqpoint{2.085829in}{2.473036in}}{\pgfqpoint{2.085829in}{2.464800in}}%
\pgfpathcurveto{\pgfqpoint{2.085829in}{2.456563in}}{\pgfqpoint{2.089101in}{2.448663in}}{\pgfqpoint{2.094925in}{2.442839in}}%
\pgfpathcurveto{\pgfqpoint{2.100749in}{2.437015in}}{\pgfqpoint{2.108649in}{2.433743in}}{\pgfqpoint{2.116886in}{2.433743in}}%
\pgfpathclose%
\pgfusepath{stroke,fill}%
\end{pgfscope}%
\begin{pgfscope}%
\pgfpathrectangle{\pgfqpoint{0.100000in}{0.212622in}}{\pgfqpoint{3.696000in}{3.696000in}}%
\pgfusepath{clip}%
\pgfsetbuttcap%
\pgfsetroundjoin%
\definecolor{currentfill}{rgb}{0.121569,0.466667,0.705882}%
\pgfsetfillcolor{currentfill}%
\pgfsetfillopacity{0.936442}%
\pgfsetlinewidth{1.003750pt}%
\definecolor{currentstroke}{rgb}{0.121569,0.466667,0.705882}%
\pgfsetstrokecolor{currentstroke}%
\pgfsetstrokeopacity{0.936442}%
\pgfsetdash{}{0pt}%
\pgfpathmoveto{\pgfqpoint{1.203036in}{1.913556in}}%
\pgfpathcurveto{\pgfqpoint{1.211272in}{1.913556in}}{\pgfqpoint{1.219172in}{1.916829in}}{\pgfqpoint{1.224996in}{1.922653in}}%
\pgfpathcurveto{\pgfqpoint{1.230820in}{1.928476in}}{\pgfqpoint{1.234092in}{1.936377in}}{\pgfqpoint{1.234092in}{1.944613in}}%
\pgfpathcurveto{\pgfqpoint{1.234092in}{1.952849in}}{\pgfqpoint{1.230820in}{1.960749in}}{\pgfqpoint{1.224996in}{1.966573in}}%
\pgfpathcurveto{\pgfqpoint{1.219172in}{1.972397in}}{\pgfqpoint{1.211272in}{1.975669in}}{\pgfqpoint{1.203036in}{1.975669in}}%
\pgfpathcurveto{\pgfqpoint{1.194799in}{1.975669in}}{\pgfqpoint{1.186899in}{1.972397in}}{\pgfqpoint{1.181075in}{1.966573in}}%
\pgfpathcurveto{\pgfqpoint{1.175251in}{1.960749in}}{\pgfqpoint{1.171979in}{1.952849in}}{\pgfqpoint{1.171979in}{1.944613in}}%
\pgfpathcurveto{\pgfqpoint{1.171979in}{1.936377in}}{\pgfqpoint{1.175251in}{1.928476in}}{\pgfqpoint{1.181075in}{1.922653in}}%
\pgfpathcurveto{\pgfqpoint{1.186899in}{1.916829in}}{\pgfqpoint{1.194799in}{1.913556in}}{\pgfqpoint{1.203036in}{1.913556in}}%
\pgfpathclose%
\pgfusepath{stroke,fill}%
\end{pgfscope}%
\begin{pgfscope}%
\pgfpathrectangle{\pgfqpoint{0.100000in}{0.212622in}}{\pgfqpoint{3.696000in}{3.696000in}}%
\pgfusepath{clip}%
\pgfsetbuttcap%
\pgfsetroundjoin%
\definecolor{currentfill}{rgb}{0.121569,0.466667,0.705882}%
\pgfsetfillcolor{currentfill}%
\pgfsetfillopacity{0.936643}%
\pgfsetlinewidth{1.003750pt}%
\definecolor{currentstroke}{rgb}{0.121569,0.466667,0.705882}%
\pgfsetstrokecolor{currentstroke}%
\pgfsetstrokeopacity{0.936643}%
\pgfsetdash{}{0pt}%
\pgfpathmoveto{\pgfqpoint{2.116269in}{2.433303in}}%
\pgfpathcurveto{\pgfqpoint{2.124506in}{2.433303in}}{\pgfqpoint{2.132406in}{2.436575in}}{\pgfqpoint{2.138229in}{2.442399in}}%
\pgfpathcurveto{\pgfqpoint{2.144053in}{2.448223in}}{\pgfqpoint{2.147326in}{2.456123in}}{\pgfqpoint{2.147326in}{2.464359in}}%
\pgfpathcurveto{\pgfqpoint{2.147326in}{2.472595in}}{\pgfqpoint{2.144053in}{2.480496in}}{\pgfqpoint{2.138229in}{2.486319in}}%
\pgfpathcurveto{\pgfqpoint{2.132406in}{2.492143in}}{\pgfqpoint{2.124506in}{2.495416in}}{\pgfqpoint{2.116269in}{2.495416in}}%
\pgfpathcurveto{\pgfqpoint{2.108033in}{2.495416in}}{\pgfqpoint{2.100133in}{2.492143in}}{\pgfqpoint{2.094309in}{2.486319in}}%
\pgfpathcurveto{\pgfqpoint{2.088485in}{2.480496in}}{\pgfqpoint{2.085213in}{2.472595in}}{\pgfqpoint{2.085213in}{2.464359in}}%
\pgfpathcurveto{\pgfqpoint{2.085213in}{2.456123in}}{\pgfqpoint{2.088485in}{2.448223in}}{\pgfqpoint{2.094309in}{2.442399in}}%
\pgfpathcurveto{\pgfqpoint{2.100133in}{2.436575in}}{\pgfqpoint{2.108033in}{2.433303in}}{\pgfqpoint{2.116269in}{2.433303in}}%
\pgfpathclose%
\pgfusepath{stroke,fill}%
\end{pgfscope}%
\begin{pgfscope}%
\pgfpathrectangle{\pgfqpoint{0.100000in}{0.212622in}}{\pgfqpoint{3.696000in}{3.696000in}}%
\pgfusepath{clip}%
\pgfsetbuttcap%
\pgfsetroundjoin%
\definecolor{currentfill}{rgb}{0.121569,0.466667,0.705882}%
\pgfsetfillcolor{currentfill}%
\pgfsetfillopacity{0.937087}%
\pgfsetlinewidth{1.003750pt}%
\definecolor{currentstroke}{rgb}{0.121569,0.466667,0.705882}%
\pgfsetstrokecolor{currentstroke}%
\pgfsetstrokeopacity{0.937087}%
\pgfsetdash{}{0pt}%
\pgfpathmoveto{\pgfqpoint{1.209404in}{1.910165in}}%
\pgfpathcurveto{\pgfqpoint{1.217641in}{1.910165in}}{\pgfqpoint{1.225541in}{1.913437in}}{\pgfqpoint{1.231365in}{1.919261in}}%
\pgfpathcurveto{\pgfqpoint{1.237189in}{1.925085in}}{\pgfqpoint{1.240461in}{1.932985in}}{\pgfqpoint{1.240461in}{1.941221in}}%
\pgfpathcurveto{\pgfqpoint{1.240461in}{1.949457in}}{\pgfqpoint{1.237189in}{1.957357in}}{\pgfqpoint{1.231365in}{1.963181in}}%
\pgfpathcurveto{\pgfqpoint{1.225541in}{1.969005in}}{\pgfqpoint{1.217641in}{1.972278in}}{\pgfqpoint{1.209404in}{1.972278in}}%
\pgfpathcurveto{\pgfqpoint{1.201168in}{1.972278in}}{\pgfqpoint{1.193268in}{1.969005in}}{\pgfqpoint{1.187444in}{1.963181in}}%
\pgfpathcurveto{\pgfqpoint{1.181620in}{1.957357in}}{\pgfqpoint{1.178348in}{1.949457in}}{\pgfqpoint{1.178348in}{1.941221in}}%
\pgfpathcurveto{\pgfqpoint{1.178348in}{1.932985in}}{\pgfqpoint{1.181620in}{1.925085in}}{\pgfqpoint{1.187444in}{1.919261in}}%
\pgfpathcurveto{\pgfqpoint{1.193268in}{1.913437in}}{\pgfqpoint{1.201168in}{1.910165in}}{\pgfqpoint{1.209404in}{1.910165in}}%
\pgfpathclose%
\pgfusepath{stroke,fill}%
\end{pgfscope}%
\begin{pgfscope}%
\pgfpathrectangle{\pgfqpoint{0.100000in}{0.212622in}}{\pgfqpoint{3.696000in}{3.696000in}}%
\pgfusepath{clip}%
\pgfsetbuttcap%
\pgfsetroundjoin%
\definecolor{currentfill}{rgb}{0.121569,0.466667,0.705882}%
\pgfsetfillcolor{currentfill}%
\pgfsetfillopacity{0.937130}%
\pgfsetlinewidth{1.003750pt}%
\definecolor{currentstroke}{rgb}{0.121569,0.466667,0.705882}%
\pgfsetstrokecolor{currentstroke}%
\pgfsetstrokeopacity{0.937130}%
\pgfsetdash{}{0pt}%
\pgfpathmoveto{\pgfqpoint{2.115109in}{2.432354in}}%
\pgfpathcurveto{\pgfqpoint{2.123346in}{2.432354in}}{\pgfqpoint{2.131246in}{2.435626in}}{\pgfqpoint{2.137070in}{2.441450in}}%
\pgfpathcurveto{\pgfqpoint{2.142894in}{2.447274in}}{\pgfqpoint{2.146166in}{2.455174in}}{\pgfqpoint{2.146166in}{2.463410in}}%
\pgfpathcurveto{\pgfqpoint{2.146166in}{2.471647in}}{\pgfqpoint{2.142894in}{2.479547in}}{\pgfqpoint{2.137070in}{2.485371in}}%
\pgfpathcurveto{\pgfqpoint{2.131246in}{2.491195in}}{\pgfqpoint{2.123346in}{2.494467in}}{\pgfqpoint{2.115109in}{2.494467in}}%
\pgfpathcurveto{\pgfqpoint{2.106873in}{2.494467in}}{\pgfqpoint{2.098973in}{2.491195in}}{\pgfqpoint{2.093149in}{2.485371in}}%
\pgfpathcurveto{\pgfqpoint{2.087325in}{2.479547in}}{\pgfqpoint{2.084053in}{2.471647in}}{\pgfqpoint{2.084053in}{2.463410in}}%
\pgfpathcurveto{\pgfqpoint{2.084053in}{2.455174in}}{\pgfqpoint{2.087325in}{2.447274in}}{\pgfqpoint{2.093149in}{2.441450in}}%
\pgfpathcurveto{\pgfqpoint{2.098973in}{2.435626in}}{\pgfqpoint{2.106873in}{2.432354in}}{\pgfqpoint{2.115109in}{2.432354in}}%
\pgfpathclose%
\pgfusepath{stroke,fill}%
\end{pgfscope}%
\begin{pgfscope}%
\pgfpathrectangle{\pgfqpoint{0.100000in}{0.212622in}}{\pgfqpoint{3.696000in}{3.696000in}}%
\pgfusepath{clip}%
\pgfsetbuttcap%
\pgfsetroundjoin%
\definecolor{currentfill}{rgb}{0.121569,0.466667,0.705882}%
\pgfsetfillcolor{currentfill}%
\pgfsetfillopacity{0.937212}%
\pgfsetlinewidth{1.003750pt}%
\definecolor{currentstroke}{rgb}{0.121569,0.466667,0.705882}%
\pgfsetstrokecolor{currentstroke}%
\pgfsetstrokeopacity{0.937212}%
\pgfsetdash{}{0pt}%
\pgfpathmoveto{\pgfqpoint{1.818416in}{2.507086in}}%
\pgfpathcurveto{\pgfqpoint{1.826652in}{2.507086in}}{\pgfqpoint{1.834552in}{2.510358in}}{\pgfqpoint{1.840376in}{2.516182in}}%
\pgfpathcurveto{\pgfqpoint{1.846200in}{2.522006in}}{\pgfqpoint{1.849472in}{2.529906in}}{\pgfqpoint{1.849472in}{2.538142in}}%
\pgfpathcurveto{\pgfqpoint{1.849472in}{2.546379in}}{\pgfqpoint{1.846200in}{2.554279in}}{\pgfqpoint{1.840376in}{2.560103in}}%
\pgfpathcurveto{\pgfqpoint{1.834552in}{2.565927in}}{\pgfqpoint{1.826652in}{2.569199in}}{\pgfqpoint{1.818416in}{2.569199in}}%
\pgfpathcurveto{\pgfqpoint{1.810179in}{2.569199in}}{\pgfqpoint{1.802279in}{2.565927in}}{\pgfqpoint{1.796456in}{2.560103in}}%
\pgfpathcurveto{\pgfqpoint{1.790632in}{2.554279in}}{\pgfqpoint{1.787359in}{2.546379in}}{\pgfqpoint{1.787359in}{2.538142in}}%
\pgfpathcurveto{\pgfqpoint{1.787359in}{2.529906in}}{\pgfqpoint{1.790632in}{2.522006in}}{\pgfqpoint{1.796456in}{2.516182in}}%
\pgfpathcurveto{\pgfqpoint{1.802279in}{2.510358in}}{\pgfqpoint{1.810179in}{2.507086in}}{\pgfqpoint{1.818416in}{2.507086in}}%
\pgfpathclose%
\pgfusepath{stroke,fill}%
\end{pgfscope}%
\begin{pgfscope}%
\pgfpathrectangle{\pgfqpoint{0.100000in}{0.212622in}}{\pgfqpoint{3.696000in}{3.696000in}}%
\pgfusepath{clip}%
\pgfsetbuttcap%
\pgfsetroundjoin%
\definecolor{currentfill}{rgb}{0.121569,0.466667,0.705882}%
\pgfsetfillcolor{currentfill}%
\pgfsetfillopacity{0.937487}%
\pgfsetlinewidth{1.003750pt}%
\definecolor{currentstroke}{rgb}{0.121569,0.466667,0.705882}%
\pgfsetstrokecolor{currentstroke}%
\pgfsetstrokeopacity{0.937487}%
\pgfsetdash{}{0pt}%
\pgfpathmoveto{\pgfqpoint{2.114290in}{2.431823in}}%
\pgfpathcurveto{\pgfqpoint{2.122527in}{2.431823in}}{\pgfqpoint{2.130427in}{2.435095in}}{\pgfqpoint{2.136250in}{2.440919in}}%
\pgfpathcurveto{\pgfqpoint{2.142074in}{2.446743in}}{\pgfqpoint{2.145347in}{2.454643in}}{\pgfqpoint{2.145347in}{2.462880in}}%
\pgfpathcurveto{\pgfqpoint{2.145347in}{2.471116in}}{\pgfqpoint{2.142074in}{2.479016in}}{\pgfqpoint{2.136250in}{2.484840in}}%
\pgfpathcurveto{\pgfqpoint{2.130427in}{2.490664in}}{\pgfqpoint{2.122527in}{2.493936in}}{\pgfqpoint{2.114290in}{2.493936in}}%
\pgfpathcurveto{\pgfqpoint{2.106054in}{2.493936in}}{\pgfqpoint{2.098154in}{2.490664in}}{\pgfqpoint{2.092330in}{2.484840in}}%
\pgfpathcurveto{\pgfqpoint{2.086506in}{2.479016in}}{\pgfqpoint{2.083234in}{2.471116in}}{\pgfqpoint{2.083234in}{2.462880in}}%
\pgfpathcurveto{\pgfqpoint{2.083234in}{2.454643in}}{\pgfqpoint{2.086506in}{2.446743in}}{\pgfqpoint{2.092330in}{2.440919in}}%
\pgfpathcurveto{\pgfqpoint{2.098154in}{2.435095in}}{\pgfqpoint{2.106054in}{2.431823in}}{\pgfqpoint{2.114290in}{2.431823in}}%
\pgfpathclose%
\pgfusepath{stroke,fill}%
\end{pgfscope}%
\begin{pgfscope}%
\pgfpathrectangle{\pgfqpoint{0.100000in}{0.212622in}}{\pgfqpoint{3.696000in}{3.696000in}}%
\pgfusepath{clip}%
\pgfsetbuttcap%
\pgfsetroundjoin%
\definecolor{currentfill}{rgb}{0.121569,0.466667,0.705882}%
\pgfsetfillcolor{currentfill}%
\pgfsetfillopacity{0.937658}%
\pgfsetlinewidth{1.003750pt}%
\definecolor{currentstroke}{rgb}{0.121569,0.466667,0.705882}%
\pgfsetstrokecolor{currentstroke}%
\pgfsetstrokeopacity{0.937658}%
\pgfsetdash{}{0pt}%
\pgfpathmoveto{\pgfqpoint{2.113884in}{2.431598in}}%
\pgfpathcurveto{\pgfqpoint{2.122120in}{2.431598in}}{\pgfqpoint{2.130020in}{2.434870in}}{\pgfqpoint{2.135844in}{2.440694in}}%
\pgfpathcurveto{\pgfqpoint{2.141668in}{2.446518in}}{\pgfqpoint{2.144941in}{2.454418in}}{\pgfqpoint{2.144941in}{2.462654in}}%
\pgfpathcurveto{\pgfqpoint{2.144941in}{2.470890in}}{\pgfqpoint{2.141668in}{2.478790in}}{\pgfqpoint{2.135844in}{2.484614in}}%
\pgfpathcurveto{\pgfqpoint{2.130020in}{2.490438in}}{\pgfqpoint{2.122120in}{2.493711in}}{\pgfqpoint{2.113884in}{2.493711in}}%
\pgfpathcurveto{\pgfqpoint{2.105648in}{2.493711in}}{\pgfqpoint{2.097748in}{2.490438in}}{\pgfqpoint{2.091924in}{2.484614in}}%
\pgfpathcurveto{\pgfqpoint{2.086100in}{2.478790in}}{\pgfqpoint{2.082828in}{2.470890in}}{\pgfqpoint{2.082828in}{2.462654in}}%
\pgfpathcurveto{\pgfqpoint{2.082828in}{2.454418in}}{\pgfqpoint{2.086100in}{2.446518in}}{\pgfqpoint{2.091924in}{2.440694in}}%
\pgfpathcurveto{\pgfqpoint{2.097748in}{2.434870in}}{\pgfqpoint{2.105648in}{2.431598in}}{\pgfqpoint{2.113884in}{2.431598in}}%
\pgfpathclose%
\pgfusepath{stroke,fill}%
\end{pgfscope}%
\begin{pgfscope}%
\pgfpathrectangle{\pgfqpoint{0.100000in}{0.212622in}}{\pgfqpoint{3.696000in}{3.696000in}}%
\pgfusepath{clip}%
\pgfsetbuttcap%
\pgfsetroundjoin%
\definecolor{currentfill}{rgb}{0.121569,0.466667,0.705882}%
\pgfsetfillcolor{currentfill}%
\pgfsetfillopacity{0.937807}%
\pgfsetlinewidth{1.003750pt}%
\definecolor{currentstroke}{rgb}{0.121569,0.466667,0.705882}%
\pgfsetstrokecolor{currentstroke}%
\pgfsetstrokeopacity{0.937807}%
\pgfsetdash{}{0pt}%
\pgfpathmoveto{\pgfqpoint{1.216746in}{1.906461in}}%
\pgfpathcurveto{\pgfqpoint{1.224982in}{1.906461in}}{\pgfqpoint{1.232882in}{1.909733in}}{\pgfqpoint{1.238706in}{1.915557in}}%
\pgfpathcurveto{\pgfqpoint{1.244530in}{1.921381in}}{\pgfqpoint{1.247802in}{1.929281in}}{\pgfqpoint{1.247802in}{1.937517in}}%
\pgfpathcurveto{\pgfqpoint{1.247802in}{1.945754in}}{\pgfqpoint{1.244530in}{1.953654in}}{\pgfqpoint{1.238706in}{1.959478in}}%
\pgfpathcurveto{\pgfqpoint{1.232882in}{1.965302in}}{\pgfqpoint{1.224982in}{1.968574in}}{\pgfqpoint{1.216746in}{1.968574in}}%
\pgfpathcurveto{\pgfqpoint{1.208509in}{1.968574in}}{\pgfqpoint{1.200609in}{1.965302in}}{\pgfqpoint{1.194785in}{1.959478in}}%
\pgfpathcurveto{\pgfqpoint{1.188962in}{1.953654in}}{\pgfqpoint{1.185689in}{1.945754in}}{\pgfqpoint{1.185689in}{1.937517in}}%
\pgfpathcurveto{\pgfqpoint{1.185689in}{1.929281in}}{\pgfqpoint{1.188962in}{1.921381in}}{\pgfqpoint{1.194785in}{1.915557in}}%
\pgfpathcurveto{\pgfqpoint{1.200609in}{1.909733in}}{\pgfqpoint{1.208509in}{1.906461in}}{\pgfqpoint{1.216746in}{1.906461in}}%
\pgfpathclose%
\pgfusepath{stroke,fill}%
\end{pgfscope}%
\begin{pgfscope}%
\pgfpathrectangle{\pgfqpoint{0.100000in}{0.212622in}}{\pgfqpoint{3.696000in}{3.696000in}}%
\pgfusepath{clip}%
\pgfsetbuttcap%
\pgfsetroundjoin%
\definecolor{currentfill}{rgb}{0.121569,0.466667,0.705882}%
\pgfsetfillcolor{currentfill}%
\pgfsetfillopacity{0.937897}%
\pgfsetlinewidth{1.003750pt}%
\definecolor{currentstroke}{rgb}{0.121569,0.466667,0.705882}%
\pgfsetstrokecolor{currentstroke}%
\pgfsetstrokeopacity{0.937897}%
\pgfsetdash{}{0pt}%
\pgfpathmoveto{\pgfqpoint{2.609534in}{1.227730in}}%
\pgfpathcurveto{\pgfqpoint{2.617770in}{1.227730in}}{\pgfqpoint{2.625671in}{1.231002in}}{\pgfqpoint{2.631494in}{1.236826in}}%
\pgfpathcurveto{\pgfqpoint{2.637318in}{1.242650in}}{\pgfqpoint{2.640591in}{1.250550in}}{\pgfqpoint{2.640591in}{1.258786in}}%
\pgfpathcurveto{\pgfqpoint{2.640591in}{1.267022in}}{\pgfqpoint{2.637318in}{1.274922in}}{\pgfqpoint{2.631494in}{1.280746in}}%
\pgfpathcurveto{\pgfqpoint{2.625671in}{1.286570in}}{\pgfqpoint{2.617770in}{1.289843in}}{\pgfqpoint{2.609534in}{1.289843in}}%
\pgfpathcurveto{\pgfqpoint{2.601298in}{1.289843in}}{\pgfqpoint{2.593398in}{1.286570in}}{\pgfqpoint{2.587574in}{1.280746in}}%
\pgfpathcurveto{\pgfqpoint{2.581750in}{1.274922in}}{\pgfqpoint{2.578478in}{1.267022in}}{\pgfqpoint{2.578478in}{1.258786in}}%
\pgfpathcurveto{\pgfqpoint{2.578478in}{1.250550in}}{\pgfqpoint{2.581750in}{1.242650in}}{\pgfqpoint{2.587574in}{1.236826in}}%
\pgfpathcurveto{\pgfqpoint{2.593398in}{1.231002in}}{\pgfqpoint{2.601298in}{1.227730in}}{\pgfqpoint{2.609534in}{1.227730in}}%
\pgfpathclose%
\pgfusepath{stroke,fill}%
\end{pgfscope}%
\begin{pgfscope}%
\pgfpathrectangle{\pgfqpoint{0.100000in}{0.212622in}}{\pgfqpoint{3.696000in}{3.696000in}}%
\pgfusepath{clip}%
\pgfsetbuttcap%
\pgfsetroundjoin%
\definecolor{currentfill}{rgb}{0.121569,0.466667,0.705882}%
\pgfsetfillcolor{currentfill}%
\pgfsetfillopacity{0.937969}%
\pgfsetlinewidth{1.003750pt}%
\definecolor{currentstroke}{rgb}{0.121569,0.466667,0.705882}%
\pgfsetstrokecolor{currentstroke}%
\pgfsetstrokeopacity{0.937969}%
\pgfsetdash{}{0pt}%
\pgfpathmoveto{\pgfqpoint{2.113137in}{2.431202in}}%
\pgfpathcurveto{\pgfqpoint{2.121373in}{2.431202in}}{\pgfqpoint{2.129273in}{2.434475in}}{\pgfqpoint{2.135097in}{2.440299in}}%
\pgfpathcurveto{\pgfqpoint{2.140921in}{2.446123in}}{\pgfqpoint{2.144193in}{2.454023in}}{\pgfqpoint{2.144193in}{2.462259in}}%
\pgfpathcurveto{\pgfqpoint{2.144193in}{2.470495in}}{\pgfqpoint{2.140921in}{2.478395in}}{\pgfqpoint{2.135097in}{2.484219in}}%
\pgfpathcurveto{\pgfqpoint{2.129273in}{2.490043in}}{\pgfqpoint{2.121373in}{2.493315in}}{\pgfqpoint{2.113137in}{2.493315in}}%
\pgfpathcurveto{\pgfqpoint{2.104901in}{2.493315in}}{\pgfqpoint{2.097001in}{2.490043in}}{\pgfqpoint{2.091177in}{2.484219in}}%
\pgfpathcurveto{\pgfqpoint{2.085353in}{2.478395in}}{\pgfqpoint{2.082080in}{2.470495in}}{\pgfqpoint{2.082080in}{2.462259in}}%
\pgfpathcurveto{\pgfqpoint{2.082080in}{2.454023in}}{\pgfqpoint{2.085353in}{2.446123in}}{\pgfqpoint{2.091177in}{2.440299in}}%
\pgfpathcurveto{\pgfqpoint{2.097001in}{2.434475in}}{\pgfqpoint{2.104901in}{2.431202in}}{\pgfqpoint{2.113137in}{2.431202in}}%
\pgfpathclose%
\pgfusepath{stroke,fill}%
\end{pgfscope}%
\begin{pgfscope}%
\pgfpathrectangle{\pgfqpoint{0.100000in}{0.212622in}}{\pgfqpoint{3.696000in}{3.696000in}}%
\pgfusepath{clip}%
\pgfsetbuttcap%
\pgfsetroundjoin%
\definecolor{currentfill}{rgb}{0.121569,0.466667,0.705882}%
\pgfsetfillcolor{currentfill}%
\pgfsetfillopacity{0.938446}%
\pgfsetlinewidth{1.003750pt}%
\definecolor{currentstroke}{rgb}{0.121569,0.466667,0.705882}%
\pgfsetstrokecolor{currentstroke}%
\pgfsetstrokeopacity{0.938446}%
\pgfsetdash{}{0pt}%
\pgfpathmoveto{\pgfqpoint{1.827879in}{2.501985in}}%
\pgfpathcurveto{\pgfqpoint{1.836115in}{2.501985in}}{\pgfqpoint{1.844015in}{2.505257in}}{\pgfqpoint{1.849839in}{2.511081in}}%
\pgfpathcurveto{\pgfqpoint{1.855663in}{2.516905in}}{\pgfqpoint{1.858935in}{2.524805in}}{\pgfqpoint{1.858935in}{2.533041in}}%
\pgfpathcurveto{\pgfqpoint{1.858935in}{2.541278in}}{\pgfqpoint{1.855663in}{2.549178in}}{\pgfqpoint{1.849839in}{2.555002in}}%
\pgfpathcurveto{\pgfqpoint{1.844015in}{2.560825in}}{\pgfqpoint{1.836115in}{2.564098in}}{\pgfqpoint{1.827879in}{2.564098in}}%
\pgfpathcurveto{\pgfqpoint{1.819642in}{2.564098in}}{\pgfqpoint{1.811742in}{2.560825in}}{\pgfqpoint{1.805918in}{2.555002in}}%
\pgfpathcurveto{\pgfqpoint{1.800095in}{2.549178in}}{\pgfqpoint{1.796822in}{2.541278in}}{\pgfqpoint{1.796822in}{2.533041in}}%
\pgfpathcurveto{\pgfqpoint{1.796822in}{2.524805in}}{\pgfqpoint{1.800095in}{2.516905in}}{\pgfqpoint{1.805918in}{2.511081in}}%
\pgfpathcurveto{\pgfqpoint{1.811742in}{2.505257in}}{\pgfqpoint{1.819642in}{2.501985in}}{\pgfqpoint{1.827879in}{2.501985in}}%
\pgfpathclose%
\pgfusepath{stroke,fill}%
\end{pgfscope}%
\begin{pgfscope}%
\pgfpathrectangle{\pgfqpoint{0.100000in}{0.212622in}}{\pgfqpoint{3.696000in}{3.696000in}}%
\pgfusepath{clip}%
\pgfsetbuttcap%
\pgfsetroundjoin%
\definecolor{currentfill}{rgb}{0.121569,0.466667,0.705882}%
\pgfsetfillcolor{currentfill}%
\pgfsetfillopacity{0.938544}%
\pgfsetlinewidth{1.003750pt}%
\definecolor{currentstroke}{rgb}{0.121569,0.466667,0.705882}%
\pgfsetstrokecolor{currentstroke}%
\pgfsetstrokeopacity{0.938544}%
\pgfsetdash{}{0pt}%
\pgfpathmoveto{\pgfqpoint{2.111821in}{2.430470in}}%
\pgfpathcurveto{\pgfqpoint{2.120057in}{2.430470in}}{\pgfqpoint{2.127957in}{2.433743in}}{\pgfqpoint{2.133781in}{2.439567in}}%
\pgfpathcurveto{\pgfqpoint{2.139605in}{2.445391in}}{\pgfqpoint{2.142877in}{2.453291in}}{\pgfqpoint{2.142877in}{2.461527in}}%
\pgfpathcurveto{\pgfqpoint{2.142877in}{2.469763in}}{\pgfqpoint{2.139605in}{2.477663in}}{\pgfqpoint{2.133781in}{2.483487in}}%
\pgfpathcurveto{\pgfqpoint{2.127957in}{2.489311in}}{\pgfqpoint{2.120057in}{2.492583in}}{\pgfqpoint{2.111821in}{2.492583in}}%
\pgfpathcurveto{\pgfqpoint{2.103585in}{2.492583in}}{\pgfqpoint{2.095685in}{2.489311in}}{\pgfqpoint{2.089861in}{2.483487in}}%
\pgfpathcurveto{\pgfqpoint{2.084037in}{2.477663in}}{\pgfqpoint{2.080764in}{2.469763in}}{\pgfqpoint{2.080764in}{2.461527in}}%
\pgfpathcurveto{\pgfqpoint{2.080764in}{2.453291in}}{\pgfqpoint{2.084037in}{2.445391in}}{\pgfqpoint{2.089861in}{2.439567in}}%
\pgfpathcurveto{\pgfqpoint{2.095685in}{2.433743in}}{\pgfqpoint{2.103585in}{2.430470in}}{\pgfqpoint{2.111821in}{2.430470in}}%
\pgfpathclose%
\pgfusepath{stroke,fill}%
\end{pgfscope}%
\begin{pgfscope}%
\pgfpathrectangle{\pgfqpoint{0.100000in}{0.212622in}}{\pgfqpoint{3.696000in}{3.696000in}}%
\pgfusepath{clip}%
\pgfsetbuttcap%
\pgfsetroundjoin%
\definecolor{currentfill}{rgb}{0.121569,0.466667,0.705882}%
\pgfsetfillcolor{currentfill}%
\pgfsetfillopacity{0.938715}%
\pgfsetlinewidth{1.003750pt}%
\definecolor{currentstroke}{rgb}{0.121569,0.466667,0.705882}%
\pgfsetstrokecolor{currentstroke}%
\pgfsetstrokeopacity{0.938715}%
\pgfsetdash{}{0pt}%
\pgfpathmoveto{\pgfqpoint{1.224299in}{1.902959in}}%
\pgfpathcurveto{\pgfqpoint{1.232535in}{1.902959in}}{\pgfqpoint{1.240435in}{1.906232in}}{\pgfqpoint{1.246259in}{1.912055in}}%
\pgfpathcurveto{\pgfqpoint{1.252083in}{1.917879in}}{\pgfqpoint{1.255355in}{1.925779in}}{\pgfqpoint{1.255355in}{1.934016in}}%
\pgfpathcurveto{\pgfqpoint{1.255355in}{1.942252in}}{\pgfqpoint{1.252083in}{1.950152in}}{\pgfqpoint{1.246259in}{1.955976in}}%
\pgfpathcurveto{\pgfqpoint{1.240435in}{1.961800in}}{\pgfqpoint{1.232535in}{1.965072in}}{\pgfqpoint{1.224299in}{1.965072in}}%
\pgfpathcurveto{\pgfqpoint{1.216062in}{1.965072in}}{\pgfqpoint{1.208162in}{1.961800in}}{\pgfqpoint{1.202338in}{1.955976in}}%
\pgfpathcurveto{\pgfqpoint{1.196515in}{1.950152in}}{\pgfqpoint{1.193242in}{1.942252in}}{\pgfqpoint{1.193242in}{1.934016in}}%
\pgfpathcurveto{\pgfqpoint{1.193242in}{1.925779in}}{\pgfqpoint{1.196515in}{1.917879in}}{\pgfqpoint{1.202338in}{1.912055in}}%
\pgfpathcurveto{\pgfqpoint{1.208162in}{1.906232in}}{\pgfqpoint{1.216062in}{1.902959in}}{\pgfqpoint{1.224299in}{1.902959in}}%
\pgfpathclose%
\pgfusepath{stroke,fill}%
\end{pgfscope}%
\begin{pgfscope}%
\pgfpathrectangle{\pgfqpoint{0.100000in}{0.212622in}}{\pgfqpoint{3.696000in}{3.696000in}}%
\pgfusepath{clip}%
\pgfsetbuttcap%
\pgfsetroundjoin%
\definecolor{currentfill}{rgb}{0.121569,0.466667,0.705882}%
\pgfsetfillcolor{currentfill}%
\pgfsetfillopacity{0.938922}%
\pgfsetlinewidth{1.003750pt}%
\definecolor{currentstroke}{rgb}{0.121569,0.466667,0.705882}%
\pgfsetstrokecolor{currentstroke}%
\pgfsetstrokeopacity{0.938922}%
\pgfsetdash{}{0pt}%
\pgfpathmoveto{\pgfqpoint{2.110820in}{2.429867in}}%
\pgfpathcurveto{\pgfqpoint{2.119056in}{2.429867in}}{\pgfqpoint{2.126956in}{2.433139in}}{\pgfqpoint{2.132780in}{2.438963in}}%
\pgfpathcurveto{\pgfqpoint{2.138604in}{2.444787in}}{\pgfqpoint{2.141877in}{2.452687in}}{\pgfqpoint{2.141877in}{2.460923in}}%
\pgfpathcurveto{\pgfqpoint{2.141877in}{2.469160in}}{\pgfqpoint{2.138604in}{2.477060in}}{\pgfqpoint{2.132780in}{2.482884in}}%
\pgfpathcurveto{\pgfqpoint{2.126956in}{2.488708in}}{\pgfqpoint{2.119056in}{2.491980in}}{\pgfqpoint{2.110820in}{2.491980in}}%
\pgfpathcurveto{\pgfqpoint{2.102584in}{2.491980in}}{\pgfqpoint{2.094684in}{2.488708in}}{\pgfqpoint{2.088860in}{2.482884in}}%
\pgfpathcurveto{\pgfqpoint{2.083036in}{2.477060in}}{\pgfqpoint{2.079764in}{2.469160in}}{\pgfqpoint{2.079764in}{2.460923in}}%
\pgfpathcurveto{\pgfqpoint{2.079764in}{2.452687in}}{\pgfqpoint{2.083036in}{2.444787in}}{\pgfqpoint{2.088860in}{2.438963in}}%
\pgfpathcurveto{\pgfqpoint{2.094684in}{2.433139in}}{\pgfqpoint{2.102584in}{2.429867in}}{\pgfqpoint{2.110820in}{2.429867in}}%
\pgfpathclose%
\pgfusepath{stroke,fill}%
\end{pgfscope}%
\begin{pgfscope}%
\pgfpathrectangle{\pgfqpoint{0.100000in}{0.212622in}}{\pgfqpoint{3.696000in}{3.696000in}}%
\pgfusepath{clip}%
\pgfsetbuttcap%
\pgfsetroundjoin%
\definecolor{currentfill}{rgb}{0.121569,0.466667,0.705882}%
\pgfsetfillcolor{currentfill}%
\pgfsetfillopacity{0.939059}%
\pgfsetlinewidth{1.003750pt}%
\definecolor{currentstroke}{rgb}{0.121569,0.466667,0.705882}%
\pgfsetstrokecolor{currentstroke}%
\pgfsetstrokeopacity{0.939059}%
\pgfsetdash{}{0pt}%
\pgfpathmoveto{\pgfqpoint{2.110499in}{2.429574in}}%
\pgfpathcurveto{\pgfqpoint{2.118735in}{2.429574in}}{\pgfqpoint{2.126635in}{2.432846in}}{\pgfqpoint{2.132459in}{2.438670in}}%
\pgfpathcurveto{\pgfqpoint{2.138283in}{2.444494in}}{\pgfqpoint{2.141555in}{2.452394in}}{\pgfqpoint{2.141555in}{2.460630in}}%
\pgfpathcurveto{\pgfqpoint{2.141555in}{2.468867in}}{\pgfqpoint{2.138283in}{2.476767in}}{\pgfqpoint{2.132459in}{2.482591in}}%
\pgfpathcurveto{\pgfqpoint{2.126635in}{2.488414in}}{\pgfqpoint{2.118735in}{2.491687in}}{\pgfqpoint{2.110499in}{2.491687in}}%
\pgfpathcurveto{\pgfqpoint{2.102262in}{2.491687in}}{\pgfqpoint{2.094362in}{2.488414in}}{\pgfqpoint{2.088538in}{2.482591in}}%
\pgfpathcurveto{\pgfqpoint{2.082715in}{2.476767in}}{\pgfqpoint{2.079442in}{2.468867in}}{\pgfqpoint{2.079442in}{2.460630in}}%
\pgfpathcurveto{\pgfqpoint{2.079442in}{2.452394in}}{\pgfqpoint{2.082715in}{2.444494in}}{\pgfqpoint{2.088538in}{2.438670in}}%
\pgfpathcurveto{\pgfqpoint{2.094362in}{2.432846in}}{\pgfqpoint{2.102262in}{2.429574in}}{\pgfqpoint{2.110499in}{2.429574in}}%
\pgfpathclose%
\pgfusepath{stroke,fill}%
\end{pgfscope}%
\begin{pgfscope}%
\pgfpathrectangle{\pgfqpoint{0.100000in}{0.212622in}}{\pgfqpoint{3.696000in}{3.696000in}}%
\pgfusepath{clip}%
\pgfsetbuttcap%
\pgfsetroundjoin%
\definecolor{currentfill}{rgb}{0.121569,0.466667,0.705882}%
\pgfsetfillcolor{currentfill}%
\pgfsetfillopacity{0.939308}%
\pgfsetlinewidth{1.003750pt}%
\definecolor{currentstroke}{rgb}{0.121569,0.466667,0.705882}%
\pgfsetstrokecolor{currentstroke}%
\pgfsetstrokeopacity{0.939308}%
\pgfsetdash{}{0pt}%
\pgfpathmoveto{\pgfqpoint{2.109908in}{2.429040in}}%
\pgfpathcurveto{\pgfqpoint{2.118144in}{2.429040in}}{\pgfqpoint{2.126044in}{2.432313in}}{\pgfqpoint{2.131868in}{2.438137in}}%
\pgfpathcurveto{\pgfqpoint{2.137692in}{2.443961in}}{\pgfqpoint{2.140964in}{2.451861in}}{\pgfqpoint{2.140964in}{2.460097in}}%
\pgfpathcurveto{\pgfqpoint{2.140964in}{2.468333in}}{\pgfqpoint{2.137692in}{2.476233in}}{\pgfqpoint{2.131868in}{2.482057in}}%
\pgfpathcurveto{\pgfqpoint{2.126044in}{2.487881in}}{\pgfqpoint{2.118144in}{2.491153in}}{\pgfqpoint{2.109908in}{2.491153in}}%
\pgfpathcurveto{\pgfqpoint{2.101671in}{2.491153in}}{\pgfqpoint{2.093771in}{2.487881in}}{\pgfqpoint{2.087947in}{2.482057in}}%
\pgfpathcurveto{\pgfqpoint{2.082123in}{2.476233in}}{\pgfqpoint{2.078851in}{2.468333in}}{\pgfqpoint{2.078851in}{2.460097in}}%
\pgfpathcurveto{\pgfqpoint{2.078851in}{2.451861in}}{\pgfqpoint{2.082123in}{2.443961in}}{\pgfqpoint{2.087947in}{2.438137in}}%
\pgfpathcurveto{\pgfqpoint{2.093771in}{2.432313in}}{\pgfqpoint{2.101671in}{2.429040in}}{\pgfqpoint{2.109908in}{2.429040in}}%
\pgfpathclose%
\pgfusepath{stroke,fill}%
\end{pgfscope}%
\begin{pgfscope}%
\pgfpathrectangle{\pgfqpoint{0.100000in}{0.212622in}}{\pgfqpoint{3.696000in}{3.696000in}}%
\pgfusepath{clip}%
\pgfsetbuttcap%
\pgfsetroundjoin%
\definecolor{currentfill}{rgb}{0.121569,0.466667,0.705882}%
\pgfsetfillcolor{currentfill}%
\pgfsetfillopacity{0.939343}%
\pgfsetlinewidth{1.003750pt}%
\definecolor{currentstroke}{rgb}{0.121569,0.466667,0.705882}%
\pgfsetstrokecolor{currentstroke}%
\pgfsetstrokeopacity{0.939343}%
\pgfsetdash{}{0pt}%
\pgfpathmoveto{\pgfqpoint{2.606713in}{1.225095in}}%
\pgfpathcurveto{\pgfqpoint{2.614949in}{1.225095in}}{\pgfqpoint{2.622849in}{1.228367in}}{\pgfqpoint{2.628673in}{1.234191in}}%
\pgfpathcurveto{\pgfqpoint{2.634497in}{1.240015in}}{\pgfqpoint{2.637769in}{1.247915in}}{\pgfqpoint{2.637769in}{1.256152in}}%
\pgfpathcurveto{\pgfqpoint{2.637769in}{1.264388in}}{\pgfqpoint{2.634497in}{1.272288in}}{\pgfqpoint{2.628673in}{1.278112in}}%
\pgfpathcurveto{\pgfqpoint{2.622849in}{1.283936in}}{\pgfqpoint{2.614949in}{1.287208in}}{\pgfqpoint{2.606713in}{1.287208in}}%
\pgfpathcurveto{\pgfqpoint{2.598477in}{1.287208in}}{\pgfqpoint{2.590577in}{1.283936in}}{\pgfqpoint{2.584753in}{1.278112in}}%
\pgfpathcurveto{\pgfqpoint{2.578929in}{1.272288in}}{\pgfqpoint{2.575656in}{1.264388in}}{\pgfqpoint{2.575656in}{1.256152in}}%
\pgfpathcurveto{\pgfqpoint{2.575656in}{1.247915in}}{\pgfqpoint{2.578929in}{1.240015in}}{\pgfqpoint{2.584753in}{1.234191in}}%
\pgfpathcurveto{\pgfqpoint{2.590577in}{1.228367in}}{\pgfqpoint{2.598477in}{1.225095in}}{\pgfqpoint{2.606713in}{1.225095in}}%
\pgfpathclose%
\pgfusepath{stroke,fill}%
\end{pgfscope}%
\begin{pgfscope}%
\pgfpathrectangle{\pgfqpoint{0.100000in}{0.212622in}}{\pgfqpoint{3.696000in}{3.696000in}}%
\pgfusepath{clip}%
\pgfsetbuttcap%
\pgfsetroundjoin%
\definecolor{currentfill}{rgb}{0.121569,0.466667,0.705882}%
\pgfsetfillcolor{currentfill}%
\pgfsetfillopacity{0.939703}%
\pgfsetlinewidth{1.003750pt}%
\definecolor{currentstroke}{rgb}{0.121569,0.466667,0.705882}%
\pgfsetstrokecolor{currentstroke}%
\pgfsetstrokeopacity{0.939703}%
\pgfsetdash{}{0pt}%
\pgfpathmoveto{\pgfqpoint{1.838931in}{2.495889in}}%
\pgfpathcurveto{\pgfqpoint{1.847167in}{2.495889in}}{\pgfqpoint{1.855067in}{2.499161in}}{\pgfqpoint{1.860891in}{2.504985in}}%
\pgfpathcurveto{\pgfqpoint{1.866715in}{2.510809in}}{\pgfqpoint{1.869988in}{2.518709in}}{\pgfqpoint{1.869988in}{2.526945in}}%
\pgfpathcurveto{\pgfqpoint{1.869988in}{2.535181in}}{\pgfqpoint{1.866715in}{2.543081in}}{\pgfqpoint{1.860891in}{2.548905in}}%
\pgfpathcurveto{\pgfqpoint{1.855067in}{2.554729in}}{\pgfqpoint{1.847167in}{2.558002in}}{\pgfqpoint{1.838931in}{2.558002in}}%
\pgfpathcurveto{\pgfqpoint{1.830695in}{2.558002in}}{\pgfqpoint{1.822795in}{2.554729in}}{\pgfqpoint{1.816971in}{2.548905in}}%
\pgfpathcurveto{\pgfqpoint{1.811147in}{2.543081in}}{\pgfqpoint{1.807875in}{2.535181in}}{\pgfqpoint{1.807875in}{2.526945in}}%
\pgfpathcurveto{\pgfqpoint{1.807875in}{2.518709in}}{\pgfqpoint{1.811147in}{2.510809in}}{\pgfqpoint{1.816971in}{2.504985in}}%
\pgfpathcurveto{\pgfqpoint{1.822795in}{2.499161in}}{\pgfqpoint{1.830695in}{2.495889in}}{\pgfqpoint{1.838931in}{2.495889in}}%
\pgfpathclose%
\pgfusepath{stroke,fill}%
\end{pgfscope}%
\begin{pgfscope}%
\pgfpathrectangle{\pgfqpoint{0.100000in}{0.212622in}}{\pgfqpoint{3.696000in}{3.696000in}}%
\pgfusepath{clip}%
\pgfsetbuttcap%
\pgfsetroundjoin%
\definecolor{currentfill}{rgb}{0.121569,0.466667,0.705882}%
\pgfsetfillcolor{currentfill}%
\pgfsetfillopacity{0.939769}%
\pgfsetlinewidth{1.003750pt}%
\definecolor{currentstroke}{rgb}{0.121569,0.466667,0.705882}%
\pgfsetstrokecolor{currentstroke}%
\pgfsetstrokeopacity{0.939769}%
\pgfsetdash{}{0pt}%
\pgfpathmoveto{\pgfqpoint{2.108858in}{2.428094in}}%
\pgfpathcurveto{\pgfqpoint{2.117095in}{2.428094in}}{\pgfqpoint{2.124995in}{2.431367in}}{\pgfqpoint{2.130818in}{2.437191in}}%
\pgfpathcurveto{\pgfqpoint{2.136642in}{2.443015in}}{\pgfqpoint{2.139915in}{2.450915in}}{\pgfqpoint{2.139915in}{2.459151in}}%
\pgfpathcurveto{\pgfqpoint{2.139915in}{2.467387in}}{\pgfqpoint{2.136642in}{2.475287in}}{\pgfqpoint{2.130818in}{2.481111in}}%
\pgfpathcurveto{\pgfqpoint{2.124995in}{2.486935in}}{\pgfqpoint{2.117095in}{2.490207in}}{\pgfqpoint{2.108858in}{2.490207in}}%
\pgfpathcurveto{\pgfqpoint{2.100622in}{2.490207in}}{\pgfqpoint{2.092722in}{2.486935in}}{\pgfqpoint{2.086898in}{2.481111in}}%
\pgfpathcurveto{\pgfqpoint{2.081074in}{2.475287in}}{\pgfqpoint{2.077802in}{2.467387in}}{\pgfqpoint{2.077802in}{2.459151in}}%
\pgfpathcurveto{\pgfqpoint{2.077802in}{2.450915in}}{\pgfqpoint{2.081074in}{2.443015in}}{\pgfqpoint{2.086898in}{2.437191in}}%
\pgfpathcurveto{\pgfqpoint{2.092722in}{2.431367in}}{\pgfqpoint{2.100622in}{2.428094in}}{\pgfqpoint{2.108858in}{2.428094in}}%
\pgfpathclose%
\pgfusepath{stroke,fill}%
\end{pgfscope}%
\begin{pgfscope}%
\pgfpathrectangle{\pgfqpoint{0.100000in}{0.212622in}}{\pgfqpoint{3.696000in}{3.696000in}}%
\pgfusepath{clip}%
\pgfsetbuttcap%
\pgfsetroundjoin%
\definecolor{currentfill}{rgb}{0.121569,0.466667,0.705882}%
\pgfsetfillcolor{currentfill}%
\pgfsetfillopacity{0.939845}%
\pgfsetlinewidth{1.003750pt}%
\definecolor{currentstroke}{rgb}{0.121569,0.466667,0.705882}%
\pgfsetstrokecolor{currentstroke}%
\pgfsetstrokeopacity{0.939845}%
\pgfsetdash{}{0pt}%
\pgfpathmoveto{\pgfqpoint{1.233727in}{1.897836in}}%
\pgfpathcurveto{\pgfqpoint{1.241963in}{1.897836in}}{\pgfqpoint{1.249863in}{1.901109in}}{\pgfqpoint{1.255687in}{1.906932in}}%
\pgfpathcurveto{\pgfqpoint{1.261511in}{1.912756in}}{\pgfqpoint{1.264784in}{1.920656in}}{\pgfqpoint{1.264784in}{1.928893in}}%
\pgfpathcurveto{\pgfqpoint{1.264784in}{1.937129in}}{\pgfqpoint{1.261511in}{1.945029in}}{\pgfqpoint{1.255687in}{1.950853in}}%
\pgfpathcurveto{\pgfqpoint{1.249863in}{1.956677in}}{\pgfqpoint{1.241963in}{1.959949in}}{\pgfqpoint{1.233727in}{1.959949in}}%
\pgfpathcurveto{\pgfqpoint{1.225491in}{1.959949in}}{\pgfqpoint{1.217591in}{1.956677in}}{\pgfqpoint{1.211767in}{1.950853in}}%
\pgfpathcurveto{\pgfqpoint{1.205943in}{1.945029in}}{\pgfqpoint{1.202671in}{1.937129in}}{\pgfqpoint{1.202671in}{1.928893in}}%
\pgfpathcurveto{\pgfqpoint{1.202671in}{1.920656in}}{\pgfqpoint{1.205943in}{1.912756in}}{\pgfqpoint{1.211767in}{1.906932in}}%
\pgfpathcurveto{\pgfqpoint{1.217591in}{1.901109in}}{\pgfqpoint{1.225491in}{1.897836in}}{\pgfqpoint{1.233727in}{1.897836in}}%
\pgfpathclose%
\pgfusepath{stroke,fill}%
\end{pgfscope}%
\begin{pgfscope}%
\pgfpathrectangle{\pgfqpoint{0.100000in}{0.212622in}}{\pgfqpoint{3.696000in}{3.696000in}}%
\pgfusepath{clip}%
\pgfsetbuttcap%
\pgfsetroundjoin%
\definecolor{currentfill}{rgb}{0.121569,0.466667,0.705882}%
\pgfsetfillcolor{currentfill}%
\pgfsetfillopacity{0.940146}%
\pgfsetlinewidth{1.003750pt}%
\definecolor{currentstroke}{rgb}{0.121569,0.466667,0.705882}%
\pgfsetstrokecolor{currentstroke}%
\pgfsetstrokeopacity{0.940146}%
\pgfsetdash{}{0pt}%
\pgfpathmoveto{\pgfqpoint{2.108094in}{2.427527in}}%
\pgfpathcurveto{\pgfqpoint{2.116330in}{2.427527in}}{\pgfqpoint{2.124230in}{2.430799in}}{\pgfqpoint{2.130054in}{2.436623in}}%
\pgfpathcurveto{\pgfqpoint{2.135878in}{2.442447in}}{\pgfqpoint{2.139151in}{2.450347in}}{\pgfqpoint{2.139151in}{2.458584in}}%
\pgfpathcurveto{\pgfqpoint{2.139151in}{2.466820in}}{\pgfqpoint{2.135878in}{2.474720in}}{\pgfqpoint{2.130054in}{2.480544in}}%
\pgfpathcurveto{\pgfqpoint{2.124230in}{2.486368in}}{\pgfqpoint{2.116330in}{2.489640in}}{\pgfqpoint{2.108094in}{2.489640in}}%
\pgfpathcurveto{\pgfqpoint{2.099858in}{2.489640in}}{\pgfqpoint{2.091958in}{2.486368in}}{\pgfqpoint{2.086134in}{2.480544in}}%
\pgfpathcurveto{\pgfqpoint{2.080310in}{2.474720in}}{\pgfqpoint{2.077038in}{2.466820in}}{\pgfqpoint{2.077038in}{2.458584in}}%
\pgfpathcurveto{\pgfqpoint{2.077038in}{2.450347in}}{\pgfqpoint{2.080310in}{2.442447in}}{\pgfqpoint{2.086134in}{2.436623in}}%
\pgfpathcurveto{\pgfqpoint{2.091958in}{2.430799in}}{\pgfqpoint{2.099858in}{2.427527in}}{\pgfqpoint{2.108094in}{2.427527in}}%
\pgfpathclose%
\pgfusepath{stroke,fill}%
\end{pgfscope}%
\begin{pgfscope}%
\pgfpathrectangle{\pgfqpoint{0.100000in}{0.212622in}}{\pgfqpoint{3.696000in}{3.696000in}}%
\pgfusepath{clip}%
\pgfsetbuttcap%
\pgfsetroundjoin%
\definecolor{currentfill}{rgb}{0.121569,0.466667,0.705882}%
\pgfsetfillcolor{currentfill}%
\pgfsetfillopacity{0.940814}%
\pgfsetlinewidth{1.003750pt}%
\definecolor{currentstroke}{rgb}{0.121569,0.466667,0.705882}%
\pgfsetstrokecolor{currentstroke}%
\pgfsetstrokeopacity{0.940814}%
\pgfsetdash{}{0pt}%
\pgfpathmoveto{\pgfqpoint{2.106672in}{2.426418in}}%
\pgfpathcurveto{\pgfqpoint{2.114908in}{2.426418in}}{\pgfqpoint{2.122808in}{2.429690in}}{\pgfqpoint{2.128632in}{2.435514in}}%
\pgfpathcurveto{\pgfqpoint{2.134456in}{2.441338in}}{\pgfqpoint{2.137729in}{2.449238in}}{\pgfqpoint{2.137729in}{2.457475in}}%
\pgfpathcurveto{\pgfqpoint{2.137729in}{2.465711in}}{\pgfqpoint{2.134456in}{2.473611in}}{\pgfqpoint{2.128632in}{2.479435in}}%
\pgfpathcurveto{\pgfqpoint{2.122808in}{2.485259in}}{\pgfqpoint{2.114908in}{2.488531in}}{\pgfqpoint{2.106672in}{2.488531in}}%
\pgfpathcurveto{\pgfqpoint{2.098436in}{2.488531in}}{\pgfqpoint{2.090536in}{2.485259in}}{\pgfqpoint{2.084712in}{2.479435in}}%
\pgfpathcurveto{\pgfqpoint{2.078888in}{2.473611in}}{\pgfqpoint{2.075616in}{2.465711in}}{\pgfqpoint{2.075616in}{2.457475in}}%
\pgfpathcurveto{\pgfqpoint{2.075616in}{2.449238in}}{\pgfqpoint{2.078888in}{2.441338in}}{\pgfqpoint{2.084712in}{2.435514in}}%
\pgfpathcurveto{\pgfqpoint{2.090536in}{2.429690in}}{\pgfqpoint{2.098436in}{2.426418in}}{\pgfqpoint{2.106672in}{2.426418in}}%
\pgfpathclose%
\pgfusepath{stroke,fill}%
\end{pgfscope}%
\begin{pgfscope}%
\pgfpathrectangle{\pgfqpoint{0.100000in}{0.212622in}}{\pgfqpoint{3.696000in}{3.696000in}}%
\pgfusepath{clip}%
\pgfsetbuttcap%
\pgfsetroundjoin%
\definecolor{currentfill}{rgb}{0.121569,0.466667,0.705882}%
\pgfsetfillcolor{currentfill}%
\pgfsetfillopacity{0.940850}%
\pgfsetlinewidth{1.003750pt}%
\definecolor{currentstroke}{rgb}{0.121569,0.466667,0.705882}%
\pgfsetstrokecolor{currentstroke}%
\pgfsetstrokeopacity{0.940850}%
\pgfsetdash{}{0pt}%
\pgfpathmoveto{\pgfqpoint{1.244459in}{1.891633in}}%
\pgfpathcurveto{\pgfqpoint{1.252696in}{1.891633in}}{\pgfqpoint{1.260596in}{1.894905in}}{\pgfqpoint{1.266420in}{1.900729in}}%
\pgfpathcurveto{\pgfqpoint{1.272243in}{1.906553in}}{\pgfqpoint{1.275516in}{1.914453in}}{\pgfqpoint{1.275516in}{1.922689in}}%
\pgfpathcurveto{\pgfqpoint{1.275516in}{1.930925in}}{\pgfqpoint{1.272243in}{1.938825in}}{\pgfqpoint{1.266420in}{1.944649in}}%
\pgfpathcurveto{\pgfqpoint{1.260596in}{1.950473in}}{\pgfqpoint{1.252696in}{1.953746in}}{\pgfqpoint{1.244459in}{1.953746in}}%
\pgfpathcurveto{\pgfqpoint{1.236223in}{1.953746in}}{\pgfqpoint{1.228323in}{1.950473in}}{\pgfqpoint{1.222499in}{1.944649in}}%
\pgfpathcurveto{\pgfqpoint{1.216675in}{1.938825in}}{\pgfqpoint{1.213403in}{1.930925in}}{\pgfqpoint{1.213403in}{1.922689in}}%
\pgfpathcurveto{\pgfqpoint{1.213403in}{1.914453in}}{\pgfqpoint{1.216675in}{1.906553in}}{\pgfqpoint{1.222499in}{1.900729in}}%
\pgfpathcurveto{\pgfqpoint{1.228323in}{1.894905in}}{\pgfqpoint{1.236223in}{1.891633in}}{\pgfqpoint{1.244459in}{1.891633in}}%
\pgfpathclose%
\pgfusepath{stroke,fill}%
\end{pgfscope}%
\begin{pgfscope}%
\pgfpathrectangle{\pgfqpoint{0.100000in}{0.212622in}}{\pgfqpoint{3.696000in}{3.696000in}}%
\pgfusepath{clip}%
\pgfsetbuttcap%
\pgfsetroundjoin%
\definecolor{currentfill}{rgb}{0.121569,0.466667,0.705882}%
\pgfsetfillcolor{currentfill}%
\pgfsetfillopacity{0.941079}%
\pgfsetlinewidth{1.003750pt}%
\definecolor{currentstroke}{rgb}{0.121569,0.466667,0.705882}%
\pgfsetstrokecolor{currentstroke}%
\pgfsetstrokeopacity{0.941079}%
\pgfsetdash{}{0pt}%
\pgfpathmoveto{\pgfqpoint{1.850387in}{2.489887in}}%
\pgfpathcurveto{\pgfqpoint{1.858623in}{2.489887in}}{\pgfqpoint{1.866524in}{2.493159in}}{\pgfqpoint{1.872347in}{2.498983in}}%
\pgfpathcurveto{\pgfqpoint{1.878171in}{2.504807in}}{\pgfqpoint{1.881444in}{2.512707in}}{\pgfqpoint{1.881444in}{2.520943in}}%
\pgfpathcurveto{\pgfqpoint{1.881444in}{2.529179in}}{\pgfqpoint{1.878171in}{2.537079in}}{\pgfqpoint{1.872347in}{2.542903in}}%
\pgfpathcurveto{\pgfqpoint{1.866524in}{2.548727in}}{\pgfqpoint{1.858623in}{2.552000in}}{\pgfqpoint{1.850387in}{2.552000in}}%
\pgfpathcurveto{\pgfqpoint{1.842151in}{2.552000in}}{\pgfqpoint{1.834251in}{2.548727in}}{\pgfqpoint{1.828427in}{2.542903in}}%
\pgfpathcurveto{\pgfqpoint{1.822603in}{2.537079in}}{\pgfqpoint{1.819331in}{2.529179in}}{\pgfqpoint{1.819331in}{2.520943in}}%
\pgfpathcurveto{\pgfqpoint{1.819331in}{2.512707in}}{\pgfqpoint{1.822603in}{2.504807in}}{\pgfqpoint{1.828427in}{2.498983in}}%
\pgfpathcurveto{\pgfqpoint{1.834251in}{2.493159in}}{\pgfqpoint{1.842151in}{2.489887in}}{\pgfqpoint{1.850387in}{2.489887in}}%
\pgfpathclose%
\pgfusepath{stroke,fill}%
\end{pgfscope}%
\begin{pgfscope}%
\pgfpathrectangle{\pgfqpoint{0.100000in}{0.212622in}}{\pgfqpoint{3.696000in}{3.696000in}}%
\pgfusepath{clip}%
\pgfsetbuttcap%
\pgfsetroundjoin%
\definecolor{currentfill}{rgb}{0.121569,0.466667,0.705882}%
\pgfsetfillcolor{currentfill}%
\pgfsetfillopacity{0.941831}%
\pgfsetlinewidth{1.003750pt}%
\definecolor{currentstroke}{rgb}{0.121569,0.466667,0.705882}%
\pgfsetstrokecolor{currentstroke}%
\pgfsetstrokeopacity{0.941831}%
\pgfsetdash{}{0pt}%
\pgfpathmoveto{\pgfqpoint{1.856569in}{2.486158in}}%
\pgfpathcurveto{\pgfqpoint{1.864806in}{2.486158in}}{\pgfqpoint{1.872706in}{2.489430in}}{\pgfqpoint{1.878530in}{2.495254in}}%
\pgfpathcurveto{\pgfqpoint{1.884354in}{2.501078in}}{\pgfqpoint{1.887626in}{2.508978in}}{\pgfqpoint{1.887626in}{2.517214in}}%
\pgfpathcurveto{\pgfqpoint{1.887626in}{2.525450in}}{\pgfqpoint{1.884354in}{2.533350in}}{\pgfqpoint{1.878530in}{2.539174in}}%
\pgfpathcurveto{\pgfqpoint{1.872706in}{2.544998in}}{\pgfqpoint{1.864806in}{2.548271in}}{\pgfqpoint{1.856569in}{2.548271in}}%
\pgfpathcurveto{\pgfqpoint{1.848333in}{2.548271in}}{\pgfqpoint{1.840433in}{2.544998in}}{\pgfqpoint{1.834609in}{2.539174in}}%
\pgfpathcurveto{\pgfqpoint{1.828785in}{2.533350in}}{\pgfqpoint{1.825513in}{2.525450in}}{\pgfqpoint{1.825513in}{2.517214in}}%
\pgfpathcurveto{\pgfqpoint{1.825513in}{2.508978in}}{\pgfqpoint{1.828785in}{2.501078in}}{\pgfqpoint{1.834609in}{2.495254in}}%
\pgfpathcurveto{\pgfqpoint{1.840433in}{2.489430in}}{\pgfqpoint{1.848333in}{2.486158in}}{\pgfqpoint{1.856569in}{2.486158in}}%
\pgfpathclose%
\pgfusepath{stroke,fill}%
\end{pgfscope}%
\begin{pgfscope}%
\pgfpathrectangle{\pgfqpoint{0.100000in}{0.212622in}}{\pgfqpoint{3.696000in}{3.696000in}}%
\pgfusepath{clip}%
\pgfsetbuttcap%
\pgfsetroundjoin%
\definecolor{currentfill}{rgb}{0.121569,0.466667,0.705882}%
\pgfsetfillcolor{currentfill}%
\pgfsetfillopacity{0.941839}%
\pgfsetlinewidth{1.003750pt}%
\definecolor{currentstroke}{rgb}{0.121569,0.466667,0.705882}%
\pgfsetstrokecolor{currentstroke}%
\pgfsetstrokeopacity{0.941839}%
\pgfsetdash{}{0pt}%
\pgfpathmoveto{\pgfqpoint{2.601808in}{1.219300in}}%
\pgfpathcurveto{\pgfqpoint{2.610044in}{1.219300in}}{\pgfqpoint{2.617944in}{1.222572in}}{\pgfqpoint{2.623768in}{1.228396in}}%
\pgfpathcurveto{\pgfqpoint{2.629592in}{1.234220in}}{\pgfqpoint{2.632864in}{1.242120in}}{\pgfqpoint{2.632864in}{1.250357in}}%
\pgfpathcurveto{\pgfqpoint{2.632864in}{1.258593in}}{\pgfqpoint{2.629592in}{1.266493in}}{\pgfqpoint{2.623768in}{1.272317in}}%
\pgfpathcurveto{\pgfqpoint{2.617944in}{1.278141in}}{\pgfqpoint{2.610044in}{1.281413in}}{\pgfqpoint{2.601808in}{1.281413in}}%
\pgfpathcurveto{\pgfqpoint{2.593571in}{1.281413in}}{\pgfqpoint{2.585671in}{1.278141in}}{\pgfqpoint{2.579847in}{1.272317in}}%
\pgfpathcurveto{\pgfqpoint{2.574023in}{1.266493in}}{\pgfqpoint{2.570751in}{1.258593in}}{\pgfqpoint{2.570751in}{1.250357in}}%
\pgfpathcurveto{\pgfqpoint{2.570751in}{1.242120in}}{\pgfqpoint{2.574023in}{1.234220in}}{\pgfqpoint{2.579847in}{1.228396in}}%
\pgfpathcurveto{\pgfqpoint{2.585671in}{1.222572in}}{\pgfqpoint{2.593571in}{1.219300in}}{\pgfqpoint{2.601808in}{1.219300in}}%
\pgfpathclose%
\pgfusepath{stroke,fill}%
\end{pgfscope}%
\begin{pgfscope}%
\pgfpathrectangle{\pgfqpoint{0.100000in}{0.212622in}}{\pgfqpoint{3.696000in}{3.696000in}}%
\pgfusepath{clip}%
\pgfsetbuttcap%
\pgfsetroundjoin%
\definecolor{currentfill}{rgb}{0.121569,0.466667,0.705882}%
\pgfsetfillcolor{currentfill}%
\pgfsetfillopacity{0.941917}%
\pgfsetlinewidth{1.003750pt}%
\definecolor{currentstroke}{rgb}{0.121569,0.466667,0.705882}%
\pgfsetstrokecolor{currentstroke}%
\pgfsetstrokeopacity{0.941917}%
\pgfsetdash{}{0pt}%
\pgfpathmoveto{\pgfqpoint{1.255686in}{1.885173in}}%
\pgfpathcurveto{\pgfqpoint{1.263922in}{1.885173in}}{\pgfqpoint{1.271822in}{1.888445in}}{\pgfqpoint{1.277646in}{1.894269in}}%
\pgfpathcurveto{\pgfqpoint{1.283470in}{1.900093in}}{\pgfqpoint{1.286742in}{1.907993in}}{\pgfqpoint{1.286742in}{1.916230in}}%
\pgfpathcurveto{\pgfqpoint{1.286742in}{1.924466in}}{\pgfqpoint{1.283470in}{1.932366in}}{\pgfqpoint{1.277646in}{1.938190in}}%
\pgfpathcurveto{\pgfqpoint{1.271822in}{1.944014in}}{\pgfqpoint{1.263922in}{1.947286in}}{\pgfqpoint{1.255686in}{1.947286in}}%
\pgfpathcurveto{\pgfqpoint{1.247450in}{1.947286in}}{\pgfqpoint{1.239550in}{1.944014in}}{\pgfqpoint{1.233726in}{1.938190in}}%
\pgfpathcurveto{\pgfqpoint{1.227902in}{1.932366in}}{\pgfqpoint{1.224629in}{1.924466in}}{\pgfqpoint{1.224629in}{1.916230in}}%
\pgfpathcurveto{\pgfqpoint{1.224629in}{1.907993in}}{\pgfqpoint{1.227902in}{1.900093in}}{\pgfqpoint{1.233726in}{1.894269in}}%
\pgfpathcurveto{\pgfqpoint{1.239550in}{1.888445in}}{\pgfqpoint{1.247450in}{1.885173in}}{\pgfqpoint{1.255686in}{1.885173in}}%
\pgfpathclose%
\pgfusepath{stroke,fill}%
\end{pgfscope}%
\begin{pgfscope}%
\pgfpathrectangle{\pgfqpoint{0.100000in}{0.212622in}}{\pgfqpoint{3.696000in}{3.696000in}}%
\pgfusepath{clip}%
\pgfsetbuttcap%
\pgfsetroundjoin%
\definecolor{currentfill}{rgb}{0.121569,0.466667,0.705882}%
\pgfsetfillcolor{currentfill}%
\pgfsetfillopacity{0.942080}%
\pgfsetlinewidth{1.003750pt}%
\definecolor{currentstroke}{rgb}{0.121569,0.466667,0.705882}%
\pgfsetstrokecolor{currentstroke}%
\pgfsetstrokeopacity{0.942080}%
\pgfsetdash{}{0pt}%
\pgfpathmoveto{\pgfqpoint{2.104155in}{2.424629in}}%
\pgfpathcurveto{\pgfqpoint{2.112392in}{2.424629in}}{\pgfqpoint{2.120292in}{2.427901in}}{\pgfqpoint{2.126116in}{2.433725in}}%
\pgfpathcurveto{\pgfqpoint{2.131940in}{2.439549in}}{\pgfqpoint{2.135212in}{2.447449in}}{\pgfqpoint{2.135212in}{2.455685in}}%
\pgfpathcurveto{\pgfqpoint{2.135212in}{2.463922in}}{\pgfqpoint{2.131940in}{2.471822in}}{\pgfqpoint{2.126116in}{2.477646in}}%
\pgfpathcurveto{\pgfqpoint{2.120292in}{2.483470in}}{\pgfqpoint{2.112392in}{2.486742in}}{\pgfqpoint{2.104155in}{2.486742in}}%
\pgfpathcurveto{\pgfqpoint{2.095919in}{2.486742in}}{\pgfqpoint{2.088019in}{2.483470in}}{\pgfqpoint{2.082195in}{2.477646in}}%
\pgfpathcurveto{\pgfqpoint{2.076371in}{2.471822in}}{\pgfqpoint{2.073099in}{2.463922in}}{\pgfqpoint{2.073099in}{2.455685in}}%
\pgfpathcurveto{\pgfqpoint{2.073099in}{2.447449in}}{\pgfqpoint{2.076371in}{2.439549in}}{\pgfqpoint{2.082195in}{2.433725in}}%
\pgfpathcurveto{\pgfqpoint{2.088019in}{2.427901in}}{\pgfqpoint{2.095919in}{2.424629in}}{\pgfqpoint{2.104155in}{2.424629in}}%
\pgfpathclose%
\pgfusepath{stroke,fill}%
\end{pgfscope}%
\begin{pgfscope}%
\pgfpathrectangle{\pgfqpoint{0.100000in}{0.212622in}}{\pgfqpoint{3.696000in}{3.696000in}}%
\pgfusepath{clip}%
\pgfsetbuttcap%
\pgfsetroundjoin%
\definecolor{currentfill}{rgb}{0.121569,0.466667,0.705882}%
\pgfsetfillcolor{currentfill}%
\pgfsetfillopacity{0.942242}%
\pgfsetlinewidth{1.003750pt}%
\definecolor{currentstroke}{rgb}{0.121569,0.466667,0.705882}%
\pgfsetstrokecolor{currentstroke}%
\pgfsetstrokeopacity{0.942242}%
\pgfsetdash{}{0pt}%
\pgfpathmoveto{\pgfqpoint{1.859972in}{2.484090in}}%
\pgfpathcurveto{\pgfqpoint{1.868208in}{2.484090in}}{\pgfqpoint{1.876108in}{2.487362in}}{\pgfqpoint{1.881932in}{2.493186in}}%
\pgfpathcurveto{\pgfqpoint{1.887756in}{2.499010in}}{\pgfqpoint{1.891028in}{2.506910in}}{\pgfqpoint{1.891028in}{2.515146in}}%
\pgfpathcurveto{\pgfqpoint{1.891028in}{2.523382in}}{\pgfqpoint{1.887756in}{2.531283in}}{\pgfqpoint{1.881932in}{2.537106in}}%
\pgfpathcurveto{\pgfqpoint{1.876108in}{2.542930in}}{\pgfqpoint{1.868208in}{2.546203in}}{\pgfqpoint{1.859972in}{2.546203in}}%
\pgfpathcurveto{\pgfqpoint{1.851736in}{2.546203in}}{\pgfqpoint{1.843836in}{2.542930in}}{\pgfqpoint{1.838012in}{2.537106in}}%
\pgfpathcurveto{\pgfqpoint{1.832188in}{2.531283in}}{\pgfqpoint{1.828915in}{2.523382in}}{\pgfqpoint{1.828915in}{2.515146in}}%
\pgfpathcurveto{\pgfqpoint{1.828915in}{2.506910in}}{\pgfqpoint{1.832188in}{2.499010in}}{\pgfqpoint{1.838012in}{2.493186in}}%
\pgfpathcurveto{\pgfqpoint{1.843836in}{2.487362in}}{\pgfqpoint{1.851736in}{2.484090in}}{\pgfqpoint{1.859972in}{2.484090in}}%
\pgfpathclose%
\pgfusepath{stroke,fill}%
\end{pgfscope}%
\begin{pgfscope}%
\pgfpathrectangle{\pgfqpoint{0.100000in}{0.212622in}}{\pgfqpoint{3.696000in}{3.696000in}}%
\pgfusepath{clip}%
\pgfsetbuttcap%
\pgfsetroundjoin%
\definecolor{currentfill}{rgb}{0.121569,0.466667,0.705882}%
\pgfsetfillcolor{currentfill}%
\pgfsetfillopacity{0.942497}%
\pgfsetlinewidth{1.003750pt}%
\definecolor{currentstroke}{rgb}{0.121569,0.466667,0.705882}%
\pgfsetstrokecolor{currentstroke}%
\pgfsetstrokeopacity{0.942497}%
\pgfsetdash{}{0pt}%
\pgfpathmoveto{\pgfqpoint{1.261945in}{1.881872in}}%
\pgfpathcurveto{\pgfqpoint{1.270181in}{1.881872in}}{\pgfqpoint{1.278081in}{1.885144in}}{\pgfqpoint{1.283905in}{1.890968in}}%
\pgfpathcurveto{\pgfqpoint{1.289729in}{1.896792in}}{\pgfqpoint{1.293002in}{1.904692in}}{\pgfqpoint{1.293002in}{1.912928in}}%
\pgfpathcurveto{\pgfqpoint{1.293002in}{1.921165in}}{\pgfqpoint{1.289729in}{1.929065in}}{\pgfqpoint{1.283905in}{1.934889in}}%
\pgfpathcurveto{\pgfqpoint{1.278081in}{1.940712in}}{\pgfqpoint{1.270181in}{1.943985in}}{\pgfqpoint{1.261945in}{1.943985in}}%
\pgfpathcurveto{\pgfqpoint{1.253709in}{1.943985in}}{\pgfqpoint{1.245809in}{1.940712in}}{\pgfqpoint{1.239985in}{1.934889in}}%
\pgfpathcurveto{\pgfqpoint{1.234161in}{1.929065in}}{\pgfqpoint{1.230889in}{1.921165in}}{\pgfqpoint{1.230889in}{1.912928in}}%
\pgfpathcurveto{\pgfqpoint{1.230889in}{1.904692in}}{\pgfqpoint{1.234161in}{1.896792in}}{\pgfqpoint{1.239985in}{1.890968in}}%
\pgfpathcurveto{\pgfqpoint{1.245809in}{1.885144in}}{\pgfqpoint{1.253709in}{1.881872in}}{\pgfqpoint{1.261945in}{1.881872in}}%
\pgfpathclose%
\pgfusepath{stroke,fill}%
\end{pgfscope}%
\begin{pgfscope}%
\pgfpathrectangle{\pgfqpoint{0.100000in}{0.212622in}}{\pgfqpoint{3.696000in}{3.696000in}}%
\pgfusepath{clip}%
\pgfsetbuttcap%
\pgfsetroundjoin%
\definecolor{currentfill}{rgb}{0.121569,0.466667,0.705882}%
\pgfsetfillcolor{currentfill}%
\pgfsetfillopacity{0.942499}%
\pgfsetlinewidth{1.003750pt}%
\definecolor{currentstroke}{rgb}{0.121569,0.466667,0.705882}%
\pgfsetstrokecolor{currentstroke}%
\pgfsetstrokeopacity{0.942499}%
\pgfsetdash{}{0pt}%
\pgfpathmoveto{\pgfqpoint{1.861815in}{2.483048in}}%
\pgfpathcurveto{\pgfqpoint{1.870052in}{2.483048in}}{\pgfqpoint{1.877952in}{2.486320in}}{\pgfqpoint{1.883776in}{2.492144in}}%
\pgfpathcurveto{\pgfqpoint{1.889600in}{2.497968in}}{\pgfqpoint{1.892872in}{2.505868in}}{\pgfqpoint{1.892872in}{2.514104in}}%
\pgfpathcurveto{\pgfqpoint{1.892872in}{2.522341in}}{\pgfqpoint{1.889600in}{2.530241in}}{\pgfqpoint{1.883776in}{2.536065in}}%
\pgfpathcurveto{\pgfqpoint{1.877952in}{2.541889in}}{\pgfqpoint{1.870052in}{2.545161in}}{\pgfqpoint{1.861815in}{2.545161in}}%
\pgfpathcurveto{\pgfqpoint{1.853579in}{2.545161in}}{\pgfqpoint{1.845679in}{2.541889in}}{\pgfqpoint{1.839855in}{2.536065in}}%
\pgfpathcurveto{\pgfqpoint{1.834031in}{2.530241in}}{\pgfqpoint{1.830759in}{2.522341in}}{\pgfqpoint{1.830759in}{2.514104in}}%
\pgfpathcurveto{\pgfqpoint{1.830759in}{2.505868in}}{\pgfqpoint{1.834031in}{2.497968in}}{\pgfqpoint{1.839855in}{2.492144in}}%
\pgfpathcurveto{\pgfqpoint{1.845679in}{2.486320in}}{\pgfqpoint{1.853579in}{2.483048in}}{\pgfqpoint{1.861815in}{2.483048in}}%
\pgfpathclose%
\pgfusepath{stroke,fill}%
\end{pgfscope}%
\begin{pgfscope}%
\pgfpathrectangle{\pgfqpoint{0.100000in}{0.212622in}}{\pgfqpoint{3.696000in}{3.696000in}}%
\pgfusepath{clip}%
\pgfsetbuttcap%
\pgfsetroundjoin%
\definecolor{currentfill}{rgb}{0.121569,0.466667,0.705882}%
\pgfsetfillcolor{currentfill}%
\pgfsetfillopacity{0.942786}%
\pgfsetlinewidth{1.003750pt}%
\definecolor{currentstroke}{rgb}{0.121569,0.466667,0.705882}%
\pgfsetstrokecolor{currentstroke}%
\pgfsetstrokeopacity{0.942786}%
\pgfsetdash{}{0pt}%
\pgfpathmoveto{\pgfqpoint{1.863934in}{2.481765in}}%
\pgfpathcurveto{\pgfqpoint{1.872171in}{2.481765in}}{\pgfqpoint{1.880071in}{2.485037in}}{\pgfqpoint{1.885895in}{2.490861in}}%
\pgfpathcurveto{\pgfqpoint{1.891719in}{2.496685in}}{\pgfqpoint{1.894991in}{2.504585in}}{\pgfqpoint{1.894991in}{2.512822in}}%
\pgfpathcurveto{\pgfqpoint{1.894991in}{2.521058in}}{\pgfqpoint{1.891719in}{2.528958in}}{\pgfqpoint{1.885895in}{2.534782in}}%
\pgfpathcurveto{\pgfqpoint{1.880071in}{2.540606in}}{\pgfqpoint{1.872171in}{2.543878in}}{\pgfqpoint{1.863934in}{2.543878in}}%
\pgfpathcurveto{\pgfqpoint{1.855698in}{2.543878in}}{\pgfqpoint{1.847798in}{2.540606in}}{\pgfqpoint{1.841974in}{2.534782in}}%
\pgfpathcurveto{\pgfqpoint{1.836150in}{2.528958in}}{\pgfqpoint{1.832878in}{2.521058in}}{\pgfqpoint{1.832878in}{2.512822in}}%
\pgfpathcurveto{\pgfqpoint{1.832878in}{2.504585in}}{\pgfqpoint{1.836150in}{2.496685in}}{\pgfqpoint{1.841974in}{2.490861in}}%
\pgfpathcurveto{\pgfqpoint{1.847798in}{2.485037in}}{\pgfqpoint{1.855698in}{2.481765in}}{\pgfqpoint{1.863934in}{2.481765in}}%
\pgfpathclose%
\pgfusepath{stroke,fill}%
\end{pgfscope}%
\begin{pgfscope}%
\pgfpathrectangle{\pgfqpoint{0.100000in}{0.212622in}}{\pgfqpoint{3.696000in}{3.696000in}}%
\pgfusepath{clip}%
\pgfsetbuttcap%
\pgfsetroundjoin%
\definecolor{currentfill}{rgb}{0.121569,0.466667,0.705882}%
\pgfsetfillcolor{currentfill}%
\pgfsetfillopacity{0.943040}%
\pgfsetlinewidth{1.003750pt}%
\definecolor{currentstroke}{rgb}{0.121569,0.466667,0.705882}%
\pgfsetstrokecolor{currentstroke}%
\pgfsetstrokeopacity{0.943040}%
\pgfsetdash{}{0pt}%
\pgfpathmoveto{\pgfqpoint{1.866420in}{2.480005in}}%
\pgfpathcurveto{\pgfqpoint{1.874656in}{2.480005in}}{\pgfqpoint{1.882556in}{2.483277in}}{\pgfqpoint{1.888380in}{2.489101in}}%
\pgfpathcurveto{\pgfqpoint{1.894204in}{2.494925in}}{\pgfqpoint{1.897477in}{2.502825in}}{\pgfqpoint{1.897477in}{2.511061in}}%
\pgfpathcurveto{\pgfqpoint{1.897477in}{2.519298in}}{\pgfqpoint{1.894204in}{2.527198in}}{\pgfqpoint{1.888380in}{2.533022in}}%
\pgfpathcurveto{\pgfqpoint{1.882556in}{2.538845in}}{\pgfqpoint{1.874656in}{2.542118in}}{\pgfqpoint{1.866420in}{2.542118in}}%
\pgfpathcurveto{\pgfqpoint{1.858184in}{2.542118in}}{\pgfqpoint{1.850284in}{2.538845in}}{\pgfqpoint{1.844460in}{2.533022in}}%
\pgfpathcurveto{\pgfqpoint{1.838636in}{2.527198in}}{\pgfqpoint{1.835364in}{2.519298in}}{\pgfqpoint{1.835364in}{2.511061in}}%
\pgfpathcurveto{\pgfqpoint{1.835364in}{2.502825in}}{\pgfqpoint{1.838636in}{2.494925in}}{\pgfqpoint{1.844460in}{2.489101in}}%
\pgfpathcurveto{\pgfqpoint{1.850284in}{2.483277in}}{\pgfqpoint{1.858184in}{2.480005in}}{\pgfqpoint{1.866420in}{2.480005in}}%
\pgfpathclose%
\pgfusepath{stroke,fill}%
\end{pgfscope}%
\begin{pgfscope}%
\pgfpathrectangle{\pgfqpoint{0.100000in}{0.212622in}}{\pgfqpoint{3.696000in}{3.696000in}}%
\pgfusepath{clip}%
\pgfsetbuttcap%
\pgfsetroundjoin%
\definecolor{currentfill}{rgb}{0.121569,0.466667,0.705882}%
\pgfsetfillcolor{currentfill}%
\pgfsetfillopacity{0.943080}%
\pgfsetlinewidth{1.003750pt}%
\definecolor{currentstroke}{rgb}{0.121569,0.466667,0.705882}%
\pgfsetstrokecolor{currentstroke}%
\pgfsetstrokeopacity{0.943080}%
\pgfsetdash{}{0pt}%
\pgfpathmoveto{\pgfqpoint{2.102044in}{2.422727in}}%
\pgfpathcurveto{\pgfqpoint{2.110280in}{2.422727in}}{\pgfqpoint{2.118180in}{2.425999in}}{\pgfqpoint{2.124004in}{2.431823in}}%
\pgfpathcurveto{\pgfqpoint{2.129828in}{2.437647in}}{\pgfqpoint{2.133100in}{2.445547in}}{\pgfqpoint{2.133100in}{2.453783in}}%
\pgfpathcurveto{\pgfqpoint{2.133100in}{2.462019in}}{\pgfqpoint{2.129828in}{2.469919in}}{\pgfqpoint{2.124004in}{2.475743in}}%
\pgfpathcurveto{\pgfqpoint{2.118180in}{2.481567in}}{\pgfqpoint{2.110280in}{2.484840in}}{\pgfqpoint{2.102044in}{2.484840in}}%
\pgfpathcurveto{\pgfqpoint{2.093808in}{2.484840in}}{\pgfqpoint{2.085908in}{2.481567in}}{\pgfqpoint{2.080084in}{2.475743in}}%
\pgfpathcurveto{\pgfqpoint{2.074260in}{2.469919in}}{\pgfqpoint{2.070987in}{2.462019in}}{\pgfqpoint{2.070987in}{2.453783in}}%
\pgfpathcurveto{\pgfqpoint{2.070987in}{2.445547in}}{\pgfqpoint{2.074260in}{2.437647in}}{\pgfqpoint{2.080084in}{2.431823in}}%
\pgfpathcurveto{\pgfqpoint{2.085908in}{2.425999in}}{\pgfqpoint{2.093808in}{2.422727in}}{\pgfqpoint{2.102044in}{2.422727in}}%
\pgfpathclose%
\pgfusepath{stroke,fill}%
\end{pgfscope}%
\begin{pgfscope}%
\pgfpathrectangle{\pgfqpoint{0.100000in}{0.212622in}}{\pgfqpoint{3.696000in}{3.696000in}}%
\pgfusepath{clip}%
\pgfsetbuttcap%
\pgfsetroundjoin%
\definecolor{currentfill}{rgb}{0.121569,0.466667,0.705882}%
\pgfsetfillcolor{currentfill}%
\pgfsetfillopacity{0.943130}%
\pgfsetlinewidth{1.003750pt}%
\definecolor{currentstroke}{rgb}{0.121569,0.466667,0.705882}%
\pgfsetstrokecolor{currentstroke}%
\pgfsetstrokeopacity{0.943130}%
\pgfsetdash{}{0pt}%
\pgfpathmoveto{\pgfqpoint{1.269327in}{1.877507in}}%
\pgfpathcurveto{\pgfqpoint{1.277563in}{1.877507in}}{\pgfqpoint{1.285463in}{1.880779in}}{\pgfqpoint{1.291287in}{1.886603in}}%
\pgfpathcurveto{\pgfqpoint{1.297111in}{1.892427in}}{\pgfqpoint{1.300383in}{1.900327in}}{\pgfqpoint{1.300383in}{1.908563in}}%
\pgfpathcurveto{\pgfqpoint{1.300383in}{1.916800in}}{\pgfqpoint{1.297111in}{1.924700in}}{\pgfqpoint{1.291287in}{1.930524in}}%
\pgfpathcurveto{\pgfqpoint{1.285463in}{1.936348in}}{\pgfqpoint{1.277563in}{1.939620in}}{\pgfqpoint{1.269327in}{1.939620in}}%
\pgfpathcurveto{\pgfqpoint{1.261091in}{1.939620in}}{\pgfqpoint{1.253191in}{1.936348in}}{\pgfqpoint{1.247367in}{1.930524in}}%
\pgfpathcurveto{\pgfqpoint{1.241543in}{1.924700in}}{\pgfqpoint{1.238270in}{1.916800in}}{\pgfqpoint{1.238270in}{1.908563in}}%
\pgfpathcurveto{\pgfqpoint{1.238270in}{1.900327in}}{\pgfqpoint{1.241543in}{1.892427in}}{\pgfqpoint{1.247367in}{1.886603in}}%
\pgfpathcurveto{\pgfqpoint{1.253191in}{1.880779in}}{\pgfqpoint{1.261091in}{1.877507in}}{\pgfqpoint{1.269327in}{1.877507in}}%
\pgfpathclose%
\pgfusepath{stroke,fill}%
\end{pgfscope}%
\begin{pgfscope}%
\pgfpathrectangle{\pgfqpoint{0.100000in}{0.212622in}}{\pgfqpoint{3.696000in}{3.696000in}}%
\pgfusepath{clip}%
\pgfsetbuttcap%
\pgfsetroundjoin%
\definecolor{currentfill}{rgb}{0.121569,0.466667,0.705882}%
\pgfsetfillcolor{currentfill}%
\pgfsetfillopacity{0.943190}%
\pgfsetlinewidth{1.003750pt}%
\definecolor{currentstroke}{rgb}{0.121569,0.466667,0.705882}%
\pgfsetstrokecolor{currentstroke}%
\pgfsetstrokeopacity{0.943190}%
\pgfsetdash{}{0pt}%
\pgfpathmoveto{\pgfqpoint{1.867783in}{2.479079in}}%
\pgfpathcurveto{\pgfqpoint{1.876019in}{2.479079in}}{\pgfqpoint{1.883919in}{2.482351in}}{\pgfqpoint{1.889743in}{2.488175in}}%
\pgfpathcurveto{\pgfqpoint{1.895567in}{2.493999in}}{\pgfqpoint{1.898839in}{2.501899in}}{\pgfqpoint{1.898839in}{2.510135in}}%
\pgfpathcurveto{\pgfqpoint{1.898839in}{2.518371in}}{\pgfqpoint{1.895567in}{2.526271in}}{\pgfqpoint{1.889743in}{2.532095in}}%
\pgfpathcurveto{\pgfqpoint{1.883919in}{2.537919in}}{\pgfqpoint{1.876019in}{2.541192in}}{\pgfqpoint{1.867783in}{2.541192in}}%
\pgfpathcurveto{\pgfqpoint{1.859546in}{2.541192in}}{\pgfqpoint{1.851646in}{2.537919in}}{\pgfqpoint{1.845822in}{2.532095in}}%
\pgfpathcurveto{\pgfqpoint{1.839998in}{2.526271in}}{\pgfqpoint{1.836726in}{2.518371in}}{\pgfqpoint{1.836726in}{2.510135in}}%
\pgfpathcurveto{\pgfqpoint{1.836726in}{2.501899in}}{\pgfqpoint{1.839998in}{2.493999in}}{\pgfqpoint{1.845822in}{2.488175in}}%
\pgfpathcurveto{\pgfqpoint{1.851646in}{2.482351in}}{\pgfqpoint{1.859546in}{2.479079in}}{\pgfqpoint{1.867783in}{2.479079in}}%
\pgfpathclose%
\pgfusepath{stroke,fill}%
\end{pgfscope}%
\begin{pgfscope}%
\pgfpathrectangle{\pgfqpoint{0.100000in}{0.212622in}}{\pgfqpoint{3.696000in}{3.696000in}}%
\pgfusepath{clip}%
\pgfsetbuttcap%
\pgfsetroundjoin%
\definecolor{currentfill}{rgb}{0.121569,0.466667,0.705882}%
\pgfsetfillcolor{currentfill}%
\pgfsetfillopacity{0.943428}%
\pgfsetlinewidth{1.003750pt}%
\definecolor{currentstroke}{rgb}{0.121569,0.466667,0.705882}%
\pgfsetstrokecolor{currentstroke}%
\pgfsetstrokeopacity{0.943428}%
\pgfsetdash{}{0pt}%
\pgfpathmoveto{\pgfqpoint{1.869800in}{2.477763in}}%
\pgfpathcurveto{\pgfqpoint{1.878036in}{2.477763in}}{\pgfqpoint{1.885936in}{2.481035in}}{\pgfqpoint{1.891760in}{2.486859in}}%
\pgfpathcurveto{\pgfqpoint{1.897584in}{2.492683in}}{\pgfqpoint{1.900856in}{2.500583in}}{\pgfqpoint{1.900856in}{2.508820in}}%
\pgfpathcurveto{\pgfqpoint{1.900856in}{2.517056in}}{\pgfqpoint{1.897584in}{2.524956in}}{\pgfqpoint{1.891760in}{2.530780in}}%
\pgfpathcurveto{\pgfqpoint{1.885936in}{2.536604in}}{\pgfqpoint{1.878036in}{2.539876in}}{\pgfqpoint{1.869800in}{2.539876in}}%
\pgfpathcurveto{\pgfqpoint{1.861564in}{2.539876in}}{\pgfqpoint{1.853664in}{2.536604in}}{\pgfqpoint{1.847840in}{2.530780in}}%
\pgfpathcurveto{\pgfqpoint{1.842016in}{2.524956in}}{\pgfqpoint{1.838743in}{2.517056in}}{\pgfqpoint{1.838743in}{2.508820in}}%
\pgfpathcurveto{\pgfqpoint{1.838743in}{2.500583in}}{\pgfqpoint{1.842016in}{2.492683in}}{\pgfqpoint{1.847840in}{2.486859in}}%
\pgfpathcurveto{\pgfqpoint{1.853664in}{2.481035in}}{\pgfqpoint{1.861564in}{2.477763in}}{\pgfqpoint{1.869800in}{2.477763in}}%
\pgfpathclose%
\pgfusepath{stroke,fill}%
\end{pgfscope}%
\begin{pgfscope}%
\pgfpathrectangle{\pgfqpoint{0.100000in}{0.212622in}}{\pgfqpoint{3.696000in}{3.696000in}}%
\pgfusepath{clip}%
\pgfsetbuttcap%
\pgfsetroundjoin%
\definecolor{currentfill}{rgb}{0.121569,0.466667,0.705882}%
\pgfsetfillcolor{currentfill}%
\pgfsetfillopacity{0.943804}%
\pgfsetlinewidth{1.003750pt}%
\definecolor{currentstroke}{rgb}{0.121569,0.466667,0.705882}%
\pgfsetstrokecolor{currentstroke}%
\pgfsetstrokeopacity{0.943804}%
\pgfsetdash{}{0pt}%
\pgfpathmoveto{\pgfqpoint{2.100531in}{2.421334in}}%
\pgfpathcurveto{\pgfqpoint{2.108767in}{2.421334in}}{\pgfqpoint{2.116667in}{2.424606in}}{\pgfqpoint{2.122491in}{2.430430in}}%
\pgfpathcurveto{\pgfqpoint{2.128315in}{2.436254in}}{\pgfqpoint{2.131587in}{2.444154in}}{\pgfqpoint{2.131587in}{2.452390in}}%
\pgfpathcurveto{\pgfqpoint{2.131587in}{2.460626in}}{\pgfqpoint{2.128315in}{2.468527in}}{\pgfqpoint{2.122491in}{2.474350in}}%
\pgfpathcurveto{\pgfqpoint{2.116667in}{2.480174in}}{\pgfqpoint{2.108767in}{2.483447in}}{\pgfqpoint{2.100531in}{2.483447in}}%
\pgfpathcurveto{\pgfqpoint{2.092295in}{2.483447in}}{\pgfqpoint{2.084394in}{2.480174in}}{\pgfqpoint{2.078571in}{2.474350in}}%
\pgfpathcurveto{\pgfqpoint{2.072747in}{2.468527in}}{\pgfqpoint{2.069474in}{2.460626in}}{\pgfqpoint{2.069474in}{2.452390in}}%
\pgfpathcurveto{\pgfqpoint{2.069474in}{2.444154in}}{\pgfqpoint{2.072747in}{2.436254in}}{\pgfqpoint{2.078571in}{2.430430in}}%
\pgfpathcurveto{\pgfqpoint{2.084394in}{2.424606in}}{\pgfqpoint{2.092295in}{2.421334in}}{\pgfqpoint{2.100531in}{2.421334in}}%
\pgfpathclose%
\pgfusepath{stroke,fill}%
\end{pgfscope}%
\begin{pgfscope}%
\pgfpathrectangle{\pgfqpoint{0.100000in}{0.212622in}}{\pgfqpoint{3.696000in}{3.696000in}}%
\pgfusepath{clip}%
\pgfsetbuttcap%
\pgfsetroundjoin%
\definecolor{currentfill}{rgb}{0.121569,0.466667,0.705882}%
\pgfsetfillcolor{currentfill}%
\pgfsetfillopacity{0.943848}%
\pgfsetlinewidth{1.003750pt}%
\definecolor{currentstroke}{rgb}{0.121569,0.466667,0.705882}%
\pgfsetstrokecolor{currentstroke}%
\pgfsetstrokeopacity{0.943848}%
\pgfsetdash{}{0pt}%
\pgfpathmoveto{\pgfqpoint{2.597667in}{1.213529in}}%
\pgfpathcurveto{\pgfqpoint{2.605903in}{1.213529in}}{\pgfqpoint{2.613803in}{1.216801in}}{\pgfqpoint{2.619627in}{1.222625in}}%
\pgfpathcurveto{\pgfqpoint{2.625451in}{1.228449in}}{\pgfqpoint{2.628723in}{1.236349in}}{\pgfqpoint{2.628723in}{1.244586in}}%
\pgfpathcurveto{\pgfqpoint{2.628723in}{1.252822in}}{\pgfqpoint{2.625451in}{1.260722in}}{\pgfqpoint{2.619627in}{1.266546in}}%
\pgfpathcurveto{\pgfqpoint{2.613803in}{1.272370in}}{\pgfqpoint{2.605903in}{1.275642in}}{\pgfqpoint{2.597667in}{1.275642in}}%
\pgfpathcurveto{\pgfqpoint{2.589430in}{1.275642in}}{\pgfqpoint{2.581530in}{1.272370in}}{\pgfqpoint{2.575706in}{1.266546in}}%
\pgfpathcurveto{\pgfqpoint{2.569883in}{1.260722in}}{\pgfqpoint{2.566610in}{1.252822in}}{\pgfqpoint{2.566610in}{1.244586in}}%
\pgfpathcurveto{\pgfqpoint{2.566610in}{1.236349in}}{\pgfqpoint{2.569883in}{1.228449in}}{\pgfqpoint{2.575706in}{1.222625in}}%
\pgfpathcurveto{\pgfqpoint{2.581530in}{1.216801in}}{\pgfqpoint{2.589430in}{1.213529in}}{\pgfqpoint{2.597667in}{1.213529in}}%
\pgfpathclose%
\pgfusepath{stroke,fill}%
\end{pgfscope}%
\begin{pgfscope}%
\pgfpathrectangle{\pgfqpoint{0.100000in}{0.212622in}}{\pgfqpoint{3.696000in}{3.696000in}}%
\pgfusepath{clip}%
\pgfsetbuttcap%
\pgfsetroundjoin%
\definecolor{currentfill}{rgb}{0.121569,0.466667,0.705882}%
\pgfsetfillcolor{currentfill}%
\pgfsetfillopacity{0.943890}%
\pgfsetlinewidth{1.003750pt}%
\definecolor{currentstroke}{rgb}{0.121569,0.466667,0.705882}%
\pgfsetstrokecolor{currentstroke}%
\pgfsetstrokeopacity{0.943890}%
\pgfsetdash{}{0pt}%
\pgfpathmoveto{\pgfqpoint{1.873765in}{2.475362in}}%
\pgfpathcurveto{\pgfqpoint{1.882002in}{2.475362in}}{\pgfqpoint{1.889902in}{2.478634in}}{\pgfqpoint{1.895726in}{2.484458in}}%
\pgfpathcurveto{\pgfqpoint{1.901550in}{2.490282in}}{\pgfqpoint{1.904822in}{2.498182in}}{\pgfqpoint{1.904822in}{2.506418in}}%
\pgfpathcurveto{\pgfqpoint{1.904822in}{2.514655in}}{\pgfqpoint{1.901550in}{2.522555in}}{\pgfqpoint{1.895726in}{2.528379in}}%
\pgfpathcurveto{\pgfqpoint{1.889902in}{2.534202in}}{\pgfqpoint{1.882002in}{2.537475in}}{\pgfqpoint{1.873765in}{2.537475in}}%
\pgfpathcurveto{\pgfqpoint{1.865529in}{2.537475in}}{\pgfqpoint{1.857629in}{2.534202in}}{\pgfqpoint{1.851805in}{2.528379in}}%
\pgfpathcurveto{\pgfqpoint{1.845981in}{2.522555in}}{\pgfqpoint{1.842709in}{2.514655in}}{\pgfqpoint{1.842709in}{2.506418in}}%
\pgfpathcurveto{\pgfqpoint{1.842709in}{2.498182in}}{\pgfqpoint{1.845981in}{2.490282in}}{\pgfqpoint{1.851805in}{2.484458in}}%
\pgfpathcurveto{\pgfqpoint{1.857629in}{2.478634in}}{\pgfqpoint{1.865529in}{2.475362in}}{\pgfqpoint{1.873765in}{2.475362in}}%
\pgfpathclose%
\pgfusepath{stroke,fill}%
\end{pgfscope}%
\begin{pgfscope}%
\pgfpathrectangle{\pgfqpoint{0.100000in}{0.212622in}}{\pgfqpoint{3.696000in}{3.696000in}}%
\pgfusepath{clip}%
\pgfsetbuttcap%
\pgfsetroundjoin%
\definecolor{currentfill}{rgb}{0.121569,0.466667,0.705882}%
\pgfsetfillcolor{currentfill}%
\pgfsetfillopacity{0.944242}%
\pgfsetlinewidth{1.003750pt}%
\definecolor{currentstroke}{rgb}{0.121569,0.466667,0.705882}%
\pgfsetstrokecolor{currentstroke}%
\pgfsetstrokeopacity{0.944242}%
\pgfsetdash{}{0pt}%
\pgfpathmoveto{\pgfqpoint{1.279038in}{1.871511in}}%
\pgfpathcurveto{\pgfqpoint{1.287274in}{1.871511in}}{\pgfqpoint{1.295174in}{1.874784in}}{\pgfqpoint{1.300998in}{1.880608in}}%
\pgfpathcurveto{\pgfqpoint{1.306822in}{1.886432in}}{\pgfqpoint{1.310094in}{1.894332in}}{\pgfqpoint{1.310094in}{1.902568in}}%
\pgfpathcurveto{\pgfqpoint{1.310094in}{1.910804in}}{\pgfqpoint{1.306822in}{1.918704in}}{\pgfqpoint{1.300998in}{1.924528in}}%
\pgfpathcurveto{\pgfqpoint{1.295174in}{1.930352in}}{\pgfqpoint{1.287274in}{1.933624in}}{\pgfqpoint{1.279038in}{1.933624in}}%
\pgfpathcurveto{\pgfqpoint{1.270802in}{1.933624in}}{\pgfqpoint{1.262902in}{1.930352in}}{\pgfqpoint{1.257078in}{1.924528in}}%
\pgfpathcurveto{\pgfqpoint{1.251254in}{1.918704in}}{\pgfqpoint{1.247981in}{1.910804in}}{\pgfqpoint{1.247981in}{1.902568in}}%
\pgfpathcurveto{\pgfqpoint{1.247981in}{1.894332in}}{\pgfqpoint{1.251254in}{1.886432in}}{\pgfqpoint{1.257078in}{1.880608in}}%
\pgfpathcurveto{\pgfqpoint{1.262902in}{1.874784in}}{\pgfqpoint{1.270802in}{1.871511in}}{\pgfqpoint{1.279038in}{1.871511in}}%
\pgfpathclose%
\pgfusepath{stroke,fill}%
\end{pgfscope}%
\begin{pgfscope}%
\pgfpathrectangle{\pgfqpoint{0.100000in}{0.212622in}}{\pgfqpoint{3.696000in}{3.696000in}}%
\pgfusepath{clip}%
\pgfsetbuttcap%
\pgfsetroundjoin%
\definecolor{currentfill}{rgb}{0.121569,0.466667,0.705882}%
\pgfsetfillcolor{currentfill}%
\pgfsetfillopacity{0.944393}%
\pgfsetlinewidth{1.003750pt}%
\definecolor{currentstroke}{rgb}{0.121569,0.466667,0.705882}%
\pgfsetstrokecolor{currentstroke}%
\pgfsetstrokeopacity{0.944393}%
\pgfsetdash{}{0pt}%
\pgfpathmoveto{\pgfqpoint{2.099258in}{2.420120in}}%
\pgfpathcurveto{\pgfqpoint{2.107494in}{2.420120in}}{\pgfqpoint{2.115394in}{2.423392in}}{\pgfqpoint{2.121218in}{2.429216in}}%
\pgfpathcurveto{\pgfqpoint{2.127042in}{2.435040in}}{\pgfqpoint{2.130314in}{2.442940in}}{\pgfqpoint{2.130314in}{2.451176in}}%
\pgfpathcurveto{\pgfqpoint{2.130314in}{2.459412in}}{\pgfqpoint{2.127042in}{2.467312in}}{\pgfqpoint{2.121218in}{2.473136in}}%
\pgfpathcurveto{\pgfqpoint{2.115394in}{2.478960in}}{\pgfqpoint{2.107494in}{2.482233in}}{\pgfqpoint{2.099258in}{2.482233in}}%
\pgfpathcurveto{\pgfqpoint{2.091022in}{2.482233in}}{\pgfqpoint{2.083122in}{2.478960in}}{\pgfqpoint{2.077298in}{2.473136in}}%
\pgfpathcurveto{\pgfqpoint{2.071474in}{2.467312in}}{\pgfqpoint{2.068201in}{2.459412in}}{\pgfqpoint{2.068201in}{2.451176in}}%
\pgfpathcurveto{\pgfqpoint{2.068201in}{2.442940in}}{\pgfqpoint{2.071474in}{2.435040in}}{\pgfqpoint{2.077298in}{2.429216in}}%
\pgfpathcurveto{\pgfqpoint{2.083122in}{2.423392in}}{\pgfqpoint{2.091022in}{2.420120in}}{\pgfqpoint{2.099258in}{2.420120in}}%
\pgfpathclose%
\pgfusepath{stroke,fill}%
\end{pgfscope}%
\begin{pgfscope}%
\pgfpathrectangle{\pgfqpoint{0.100000in}{0.212622in}}{\pgfqpoint{3.696000in}{3.696000in}}%
\pgfusepath{clip}%
\pgfsetbuttcap%
\pgfsetroundjoin%
\definecolor{currentfill}{rgb}{0.121569,0.466667,0.705882}%
\pgfsetfillcolor{currentfill}%
\pgfsetfillopacity{0.944488}%
\pgfsetlinewidth{1.003750pt}%
\definecolor{currentstroke}{rgb}{0.121569,0.466667,0.705882}%
\pgfsetstrokecolor{currentstroke}%
\pgfsetstrokeopacity{0.944488}%
\pgfsetdash{}{0pt}%
\pgfpathmoveto{\pgfqpoint{1.878601in}{2.472852in}}%
\pgfpathcurveto{\pgfqpoint{1.886837in}{2.472852in}}{\pgfqpoint{1.894737in}{2.476124in}}{\pgfqpoint{1.900561in}{2.481948in}}%
\pgfpathcurveto{\pgfqpoint{1.906385in}{2.487772in}}{\pgfqpoint{1.909657in}{2.495672in}}{\pgfqpoint{1.909657in}{2.503908in}}%
\pgfpathcurveto{\pgfqpoint{1.909657in}{2.512145in}}{\pgfqpoint{1.906385in}{2.520045in}}{\pgfqpoint{1.900561in}{2.525869in}}%
\pgfpathcurveto{\pgfqpoint{1.894737in}{2.531692in}}{\pgfqpoint{1.886837in}{2.534965in}}{\pgfqpoint{1.878601in}{2.534965in}}%
\pgfpathcurveto{\pgfqpoint{1.870365in}{2.534965in}}{\pgfqpoint{1.862464in}{2.531692in}}{\pgfqpoint{1.856641in}{2.525869in}}%
\pgfpathcurveto{\pgfqpoint{1.850817in}{2.520045in}}{\pgfqpoint{1.847544in}{2.512145in}}{\pgfqpoint{1.847544in}{2.503908in}}%
\pgfpathcurveto{\pgfqpoint{1.847544in}{2.495672in}}{\pgfqpoint{1.850817in}{2.487772in}}{\pgfqpoint{1.856641in}{2.481948in}}%
\pgfpathcurveto{\pgfqpoint{1.862464in}{2.476124in}}{\pgfqpoint{1.870365in}{2.472852in}}{\pgfqpoint{1.878601in}{2.472852in}}%
\pgfpathclose%
\pgfusepath{stroke,fill}%
\end{pgfscope}%
\begin{pgfscope}%
\pgfpathrectangle{\pgfqpoint{0.100000in}{0.212622in}}{\pgfqpoint{3.696000in}{3.696000in}}%
\pgfusepath{clip}%
\pgfsetbuttcap%
\pgfsetroundjoin%
\definecolor{currentfill}{rgb}{0.121569,0.466667,0.705882}%
\pgfsetfillcolor{currentfill}%
\pgfsetfillopacity{0.944822}%
\pgfsetlinewidth{1.003750pt}%
\definecolor{currentstroke}{rgb}{0.121569,0.466667,0.705882}%
\pgfsetstrokecolor{currentstroke}%
\pgfsetstrokeopacity{0.944822}%
\pgfsetdash{}{0pt}%
\pgfpathmoveto{\pgfqpoint{1.881238in}{2.471426in}}%
\pgfpathcurveto{\pgfqpoint{1.889474in}{2.471426in}}{\pgfqpoint{1.897374in}{2.474698in}}{\pgfqpoint{1.903198in}{2.480522in}}%
\pgfpathcurveto{\pgfqpoint{1.909022in}{2.486346in}}{\pgfqpoint{1.912294in}{2.494246in}}{\pgfqpoint{1.912294in}{2.502483in}}%
\pgfpathcurveto{\pgfqpoint{1.912294in}{2.510719in}}{\pgfqpoint{1.909022in}{2.518619in}}{\pgfqpoint{1.903198in}{2.524443in}}%
\pgfpathcurveto{\pgfqpoint{1.897374in}{2.530267in}}{\pgfqpoint{1.889474in}{2.533539in}}{\pgfqpoint{1.881238in}{2.533539in}}%
\pgfpathcurveto{\pgfqpoint{1.873002in}{2.533539in}}{\pgfqpoint{1.865101in}{2.530267in}}{\pgfqpoint{1.859278in}{2.524443in}}%
\pgfpathcurveto{\pgfqpoint{1.853454in}{2.518619in}}{\pgfqpoint{1.850181in}{2.510719in}}{\pgfqpoint{1.850181in}{2.502483in}}%
\pgfpathcurveto{\pgfqpoint{1.850181in}{2.494246in}}{\pgfqpoint{1.853454in}{2.486346in}}{\pgfqpoint{1.859278in}{2.480522in}}%
\pgfpathcurveto{\pgfqpoint{1.865101in}{2.474698in}}{\pgfqpoint{1.873002in}{2.471426in}}{\pgfqpoint{1.881238in}{2.471426in}}%
\pgfpathclose%
\pgfusepath{stroke,fill}%
\end{pgfscope}%
\begin{pgfscope}%
\pgfpathrectangle{\pgfqpoint{0.100000in}{0.212622in}}{\pgfqpoint{3.696000in}{3.696000in}}%
\pgfusepath{clip}%
\pgfsetbuttcap%
\pgfsetroundjoin%
\definecolor{currentfill}{rgb}{0.121569,0.466667,0.705882}%
\pgfsetfillcolor{currentfill}%
\pgfsetfillopacity{0.944865}%
\pgfsetlinewidth{1.003750pt}%
\definecolor{currentstroke}{rgb}{0.121569,0.466667,0.705882}%
\pgfsetstrokecolor{currentstroke}%
\pgfsetstrokeopacity{0.944865}%
\pgfsetdash{}{0pt}%
\pgfpathmoveto{\pgfqpoint{2.098282in}{2.419167in}}%
\pgfpathcurveto{\pgfqpoint{2.106518in}{2.419167in}}{\pgfqpoint{2.114418in}{2.422440in}}{\pgfqpoint{2.120242in}{2.428264in}}%
\pgfpathcurveto{\pgfqpoint{2.126066in}{2.434087in}}{\pgfqpoint{2.129338in}{2.441988in}}{\pgfqpoint{2.129338in}{2.450224in}}%
\pgfpathcurveto{\pgfqpoint{2.129338in}{2.458460in}}{\pgfqpoint{2.126066in}{2.466360in}}{\pgfqpoint{2.120242in}{2.472184in}}%
\pgfpathcurveto{\pgfqpoint{2.114418in}{2.478008in}}{\pgfqpoint{2.106518in}{2.481280in}}{\pgfqpoint{2.098282in}{2.481280in}}%
\pgfpathcurveto{\pgfqpoint{2.090046in}{2.481280in}}{\pgfqpoint{2.082146in}{2.478008in}}{\pgfqpoint{2.076322in}{2.472184in}}%
\pgfpathcurveto{\pgfqpoint{2.070498in}{2.466360in}}{\pgfqpoint{2.067225in}{2.458460in}}{\pgfqpoint{2.067225in}{2.450224in}}%
\pgfpathcurveto{\pgfqpoint{2.067225in}{2.441988in}}{\pgfqpoint{2.070498in}{2.434087in}}{\pgfqpoint{2.076322in}{2.428264in}}%
\pgfpathcurveto{\pgfqpoint{2.082146in}{2.422440in}}{\pgfqpoint{2.090046in}{2.419167in}}{\pgfqpoint{2.098282in}{2.419167in}}%
\pgfpathclose%
\pgfusepath{stroke,fill}%
\end{pgfscope}%
\begin{pgfscope}%
\pgfpathrectangle{\pgfqpoint{0.100000in}{0.212622in}}{\pgfqpoint{3.696000in}{3.696000in}}%
\pgfusepath{clip}%
\pgfsetbuttcap%
\pgfsetroundjoin%
\definecolor{currentfill}{rgb}{0.121569,0.466667,0.705882}%
\pgfsetfillcolor{currentfill}%
\pgfsetfillopacity{0.944935}%
\pgfsetlinewidth{1.003750pt}%
\definecolor{currentstroke}{rgb}{0.121569,0.466667,0.705882}%
\pgfsetstrokecolor{currentstroke}%
\pgfsetstrokeopacity{0.944935}%
\pgfsetdash{}{0pt}%
\pgfpathmoveto{\pgfqpoint{1.882748in}{2.470430in}}%
\pgfpathcurveto{\pgfqpoint{1.890984in}{2.470430in}}{\pgfqpoint{1.898884in}{2.473703in}}{\pgfqpoint{1.904708in}{2.479527in}}%
\pgfpathcurveto{\pgfqpoint{1.910532in}{2.485351in}}{\pgfqpoint{1.913804in}{2.493251in}}{\pgfqpoint{1.913804in}{2.501487in}}%
\pgfpathcurveto{\pgfqpoint{1.913804in}{2.509723in}}{\pgfqpoint{1.910532in}{2.517623in}}{\pgfqpoint{1.904708in}{2.523447in}}%
\pgfpathcurveto{\pgfqpoint{1.898884in}{2.529271in}}{\pgfqpoint{1.890984in}{2.532543in}}{\pgfqpoint{1.882748in}{2.532543in}}%
\pgfpathcurveto{\pgfqpoint{1.874512in}{2.532543in}}{\pgfqpoint{1.866612in}{2.529271in}}{\pgfqpoint{1.860788in}{2.523447in}}%
\pgfpathcurveto{\pgfqpoint{1.854964in}{2.517623in}}{\pgfqpoint{1.851691in}{2.509723in}}{\pgfqpoint{1.851691in}{2.501487in}}%
\pgfpathcurveto{\pgfqpoint{1.851691in}{2.493251in}}{\pgfqpoint{1.854964in}{2.485351in}}{\pgfqpoint{1.860788in}{2.479527in}}%
\pgfpathcurveto{\pgfqpoint{1.866612in}{2.473703in}}{\pgfqpoint{1.874512in}{2.470430in}}{\pgfqpoint{1.882748in}{2.470430in}}%
\pgfpathclose%
\pgfusepath{stroke,fill}%
\end{pgfscope}%
\begin{pgfscope}%
\pgfpathrectangle{\pgfqpoint{0.100000in}{0.212622in}}{\pgfqpoint{3.696000in}{3.696000in}}%
\pgfusepath{clip}%
\pgfsetbuttcap%
\pgfsetroundjoin%
\definecolor{currentfill}{rgb}{0.121569,0.466667,0.705882}%
\pgfsetfillcolor{currentfill}%
\pgfsetfillopacity{0.945129}%
\pgfsetlinewidth{1.003750pt}%
\definecolor{currentstroke}{rgb}{0.121569,0.466667,0.705882}%
\pgfsetstrokecolor{currentstroke}%
\pgfsetstrokeopacity{0.945129}%
\pgfsetdash{}{0pt}%
\pgfpathmoveto{\pgfqpoint{1.884654in}{2.469306in}}%
\pgfpathcurveto{\pgfqpoint{1.892890in}{2.469306in}}{\pgfqpoint{1.900790in}{2.472578in}}{\pgfqpoint{1.906614in}{2.478402in}}%
\pgfpathcurveto{\pgfqpoint{1.912438in}{2.484226in}}{\pgfqpoint{1.915711in}{2.492126in}}{\pgfqpoint{1.915711in}{2.500362in}}%
\pgfpathcurveto{\pgfqpoint{1.915711in}{2.508599in}}{\pgfqpoint{1.912438in}{2.516499in}}{\pgfqpoint{1.906614in}{2.522323in}}%
\pgfpathcurveto{\pgfqpoint{1.900790in}{2.528147in}}{\pgfqpoint{1.892890in}{2.531419in}}{\pgfqpoint{1.884654in}{2.531419in}}%
\pgfpathcurveto{\pgfqpoint{1.876418in}{2.531419in}}{\pgfqpoint{1.868518in}{2.528147in}}{\pgfqpoint{1.862694in}{2.522323in}}%
\pgfpathcurveto{\pgfqpoint{1.856870in}{2.516499in}}{\pgfqpoint{1.853598in}{2.508599in}}{\pgfqpoint{1.853598in}{2.500362in}}%
\pgfpathcurveto{\pgfqpoint{1.853598in}{2.492126in}}{\pgfqpoint{1.856870in}{2.484226in}}{\pgfqpoint{1.862694in}{2.478402in}}%
\pgfpathcurveto{\pgfqpoint{1.868518in}{2.472578in}}{\pgfqpoint{1.876418in}{2.469306in}}{\pgfqpoint{1.884654in}{2.469306in}}%
\pgfpathclose%
\pgfusepath{stroke,fill}%
\end{pgfscope}%
\begin{pgfscope}%
\pgfpathrectangle{\pgfqpoint{0.100000in}{0.212622in}}{\pgfqpoint{3.696000in}{3.696000in}}%
\pgfusepath{clip}%
\pgfsetbuttcap%
\pgfsetroundjoin%
\definecolor{currentfill}{rgb}{0.121569,0.466667,0.705882}%
\pgfsetfillcolor{currentfill}%
\pgfsetfillopacity{0.945228}%
\pgfsetlinewidth{1.003750pt}%
\definecolor{currentstroke}{rgb}{0.121569,0.466667,0.705882}%
\pgfsetstrokecolor{currentstroke}%
\pgfsetstrokeopacity{0.945228}%
\pgfsetdash{}{0pt}%
\pgfpathmoveto{\pgfqpoint{2.097499in}{2.418373in}}%
\pgfpathcurveto{\pgfqpoint{2.105736in}{2.418373in}}{\pgfqpoint{2.113636in}{2.421645in}}{\pgfqpoint{2.119460in}{2.427469in}}%
\pgfpathcurveto{\pgfqpoint{2.125284in}{2.433293in}}{\pgfqpoint{2.128556in}{2.441193in}}{\pgfqpoint{2.128556in}{2.449429in}}%
\pgfpathcurveto{\pgfqpoint{2.128556in}{2.457665in}}{\pgfqpoint{2.125284in}{2.465566in}}{\pgfqpoint{2.119460in}{2.471389in}}%
\pgfpathcurveto{\pgfqpoint{2.113636in}{2.477213in}}{\pgfqpoint{2.105736in}{2.480486in}}{\pgfqpoint{2.097499in}{2.480486in}}%
\pgfpathcurveto{\pgfqpoint{2.089263in}{2.480486in}}{\pgfqpoint{2.081363in}{2.477213in}}{\pgfqpoint{2.075539in}{2.471389in}}%
\pgfpathcurveto{\pgfqpoint{2.069715in}{2.465566in}}{\pgfqpoint{2.066443in}{2.457665in}}{\pgfqpoint{2.066443in}{2.449429in}}%
\pgfpathcurveto{\pgfqpoint{2.066443in}{2.441193in}}{\pgfqpoint{2.069715in}{2.433293in}}{\pgfqpoint{2.075539in}{2.427469in}}%
\pgfpathcurveto{\pgfqpoint{2.081363in}{2.421645in}}{\pgfqpoint{2.089263in}{2.418373in}}{\pgfqpoint{2.097499in}{2.418373in}}%
\pgfpathclose%
\pgfusepath{stroke,fill}%
\end{pgfscope}%
\begin{pgfscope}%
\pgfpathrectangle{\pgfqpoint{0.100000in}{0.212622in}}{\pgfqpoint{3.696000in}{3.696000in}}%
\pgfusepath{clip}%
\pgfsetbuttcap%
\pgfsetroundjoin%
\definecolor{currentfill}{rgb}{0.121569,0.466667,0.705882}%
\pgfsetfillcolor{currentfill}%
\pgfsetfillopacity{0.945370}%
\pgfsetlinewidth{1.003750pt}%
\definecolor{currentstroke}{rgb}{0.121569,0.466667,0.705882}%
\pgfsetstrokecolor{currentstroke}%
\pgfsetstrokeopacity{0.945370}%
\pgfsetdash{}{0pt}%
\pgfpathmoveto{\pgfqpoint{2.594613in}{1.209236in}}%
\pgfpathcurveto{\pgfqpoint{2.602849in}{1.209236in}}{\pgfqpoint{2.610749in}{1.212509in}}{\pgfqpoint{2.616573in}{1.218333in}}%
\pgfpathcurveto{\pgfqpoint{2.622397in}{1.224157in}}{\pgfqpoint{2.625670in}{1.232057in}}{\pgfqpoint{2.625670in}{1.240293in}}%
\pgfpathcurveto{\pgfqpoint{2.625670in}{1.248529in}}{\pgfqpoint{2.622397in}{1.256429in}}{\pgfqpoint{2.616573in}{1.262253in}}%
\pgfpathcurveto{\pgfqpoint{2.610749in}{1.268077in}}{\pgfqpoint{2.602849in}{1.271349in}}{\pgfqpoint{2.594613in}{1.271349in}}%
\pgfpathcurveto{\pgfqpoint{2.586377in}{1.271349in}}{\pgfqpoint{2.578477in}{1.268077in}}{\pgfqpoint{2.572653in}{1.262253in}}%
\pgfpathcurveto{\pgfqpoint{2.566829in}{1.256429in}}{\pgfqpoint{2.563557in}{1.248529in}}{\pgfqpoint{2.563557in}{1.240293in}}%
\pgfpathcurveto{\pgfqpoint{2.563557in}{1.232057in}}{\pgfqpoint{2.566829in}{1.224157in}}{\pgfqpoint{2.572653in}{1.218333in}}%
\pgfpathcurveto{\pgfqpoint{2.578477in}{1.212509in}}{\pgfqpoint{2.586377in}{1.209236in}}{\pgfqpoint{2.594613in}{1.209236in}}%
\pgfpathclose%
\pgfusepath{stroke,fill}%
\end{pgfscope}%
\begin{pgfscope}%
\pgfpathrectangle{\pgfqpoint{0.100000in}{0.212622in}}{\pgfqpoint{3.696000in}{3.696000in}}%
\pgfusepath{clip}%
\pgfsetbuttcap%
\pgfsetroundjoin%
\definecolor{currentfill}{rgb}{0.121569,0.466667,0.705882}%
\pgfsetfillcolor{currentfill}%
\pgfsetfillopacity{0.945487}%
\pgfsetlinewidth{1.003750pt}%
\definecolor{currentstroke}{rgb}{0.121569,0.466667,0.705882}%
\pgfsetstrokecolor{currentstroke}%
\pgfsetstrokeopacity{0.945487}%
\pgfsetdash{}{0pt}%
\pgfpathmoveto{\pgfqpoint{1.290079in}{1.863965in}}%
\pgfpathcurveto{\pgfqpoint{1.298316in}{1.863965in}}{\pgfqpoint{1.306216in}{1.867237in}}{\pgfqpoint{1.312040in}{1.873061in}}%
\pgfpathcurveto{\pgfqpoint{1.317863in}{1.878885in}}{\pgfqpoint{1.321136in}{1.886785in}}{\pgfqpoint{1.321136in}{1.895021in}}%
\pgfpathcurveto{\pgfqpoint{1.321136in}{1.903258in}}{\pgfqpoint{1.317863in}{1.911158in}}{\pgfqpoint{1.312040in}{1.916982in}}%
\pgfpathcurveto{\pgfqpoint{1.306216in}{1.922806in}}{\pgfqpoint{1.298316in}{1.926078in}}{\pgfqpoint{1.290079in}{1.926078in}}%
\pgfpathcurveto{\pgfqpoint{1.281843in}{1.926078in}}{\pgfqpoint{1.273943in}{1.922806in}}{\pgfqpoint{1.268119in}{1.916982in}}%
\pgfpathcurveto{\pgfqpoint{1.262295in}{1.911158in}}{\pgfqpoint{1.259023in}{1.903258in}}{\pgfqpoint{1.259023in}{1.895021in}}%
\pgfpathcurveto{\pgfqpoint{1.259023in}{1.886785in}}{\pgfqpoint{1.262295in}{1.878885in}}{\pgfqpoint{1.268119in}{1.873061in}}%
\pgfpathcurveto{\pgfqpoint{1.273943in}{1.867237in}}{\pgfqpoint{1.281843in}{1.863965in}}{\pgfqpoint{1.290079in}{1.863965in}}%
\pgfpathclose%
\pgfusepath{stroke,fill}%
\end{pgfscope}%
\begin{pgfscope}%
\pgfpathrectangle{\pgfqpoint{0.100000in}{0.212622in}}{\pgfqpoint{3.696000in}{3.696000in}}%
\pgfusepath{clip}%
\pgfsetbuttcap%
\pgfsetroundjoin%
\definecolor{currentfill}{rgb}{0.121569,0.466667,0.705882}%
\pgfsetfillcolor{currentfill}%
\pgfsetfillopacity{0.945493}%
\pgfsetlinewidth{1.003750pt}%
\definecolor{currentstroke}{rgb}{0.121569,0.466667,0.705882}%
\pgfsetstrokecolor{currentstroke}%
\pgfsetstrokeopacity{0.945493}%
\pgfsetdash{}{0pt}%
\pgfpathmoveto{\pgfqpoint{1.887550in}{2.467568in}}%
\pgfpathcurveto{\pgfqpoint{1.895786in}{2.467568in}}{\pgfqpoint{1.903686in}{2.470840in}}{\pgfqpoint{1.909510in}{2.476664in}}%
\pgfpathcurveto{\pgfqpoint{1.915334in}{2.482488in}}{\pgfqpoint{1.918606in}{2.490388in}}{\pgfqpoint{1.918606in}{2.498625in}}%
\pgfpathcurveto{\pgfqpoint{1.918606in}{2.506861in}}{\pgfqpoint{1.915334in}{2.514761in}}{\pgfqpoint{1.909510in}{2.520585in}}%
\pgfpathcurveto{\pgfqpoint{1.903686in}{2.526409in}}{\pgfqpoint{1.895786in}{2.529681in}}{\pgfqpoint{1.887550in}{2.529681in}}%
\pgfpathcurveto{\pgfqpoint{1.879313in}{2.529681in}}{\pgfqpoint{1.871413in}{2.526409in}}{\pgfqpoint{1.865589in}{2.520585in}}%
\pgfpathcurveto{\pgfqpoint{1.859765in}{2.514761in}}{\pgfqpoint{1.856493in}{2.506861in}}{\pgfqpoint{1.856493in}{2.498625in}}%
\pgfpathcurveto{\pgfqpoint{1.856493in}{2.490388in}}{\pgfqpoint{1.859765in}{2.482488in}}{\pgfqpoint{1.865589in}{2.476664in}}%
\pgfpathcurveto{\pgfqpoint{1.871413in}{2.470840in}}{\pgfqpoint{1.879313in}{2.467568in}}{\pgfqpoint{1.887550in}{2.467568in}}%
\pgfpathclose%
\pgfusepath{stroke,fill}%
\end{pgfscope}%
\begin{pgfscope}%
\pgfpathrectangle{\pgfqpoint{0.100000in}{0.212622in}}{\pgfqpoint{3.696000in}{3.696000in}}%
\pgfusepath{clip}%
\pgfsetbuttcap%
\pgfsetroundjoin%
\definecolor{currentfill}{rgb}{0.121569,0.466667,0.705882}%
\pgfsetfillcolor{currentfill}%
\pgfsetfillopacity{0.945889}%
\pgfsetlinewidth{1.003750pt}%
\definecolor{currentstroke}{rgb}{0.121569,0.466667,0.705882}%
\pgfsetstrokecolor{currentstroke}%
\pgfsetstrokeopacity{0.945889}%
\pgfsetdash{}{0pt}%
\pgfpathmoveto{\pgfqpoint{2.096097in}{2.416914in}}%
\pgfpathcurveto{\pgfqpoint{2.104333in}{2.416914in}}{\pgfqpoint{2.112233in}{2.420186in}}{\pgfqpoint{2.118057in}{2.426010in}}%
\pgfpathcurveto{\pgfqpoint{2.123881in}{2.431834in}}{\pgfqpoint{2.127153in}{2.439734in}}{\pgfqpoint{2.127153in}{2.447970in}}%
\pgfpathcurveto{\pgfqpoint{2.127153in}{2.456206in}}{\pgfqpoint{2.123881in}{2.464107in}}{\pgfqpoint{2.118057in}{2.469930in}}%
\pgfpathcurveto{\pgfqpoint{2.112233in}{2.475754in}}{\pgfqpoint{2.104333in}{2.479027in}}{\pgfqpoint{2.096097in}{2.479027in}}%
\pgfpathcurveto{\pgfqpoint{2.087860in}{2.479027in}}{\pgfqpoint{2.079960in}{2.475754in}}{\pgfqpoint{2.074136in}{2.469930in}}%
\pgfpathcurveto{\pgfqpoint{2.068312in}{2.464107in}}{\pgfqpoint{2.065040in}{2.456206in}}{\pgfqpoint{2.065040in}{2.447970in}}%
\pgfpathcurveto{\pgfqpoint{2.065040in}{2.439734in}}{\pgfqpoint{2.068312in}{2.431834in}}{\pgfqpoint{2.074136in}{2.426010in}}%
\pgfpathcurveto{\pgfqpoint{2.079960in}{2.420186in}}{\pgfqpoint{2.087860in}{2.416914in}}{\pgfqpoint{2.096097in}{2.416914in}}%
\pgfpathclose%
\pgfusepath{stroke,fill}%
\end{pgfscope}%
\begin{pgfscope}%
\pgfpathrectangle{\pgfqpoint{0.100000in}{0.212622in}}{\pgfqpoint{3.696000in}{3.696000in}}%
\pgfusepath{clip}%
\pgfsetbuttcap%
\pgfsetroundjoin%
\definecolor{currentfill}{rgb}{0.121569,0.466667,0.705882}%
\pgfsetfillcolor{currentfill}%
\pgfsetfillopacity{0.945895}%
\pgfsetlinewidth{1.003750pt}%
\definecolor{currentstroke}{rgb}{0.121569,0.466667,0.705882}%
\pgfsetstrokecolor{currentstroke}%
\pgfsetstrokeopacity{0.945895}%
\pgfsetdash{}{0pt}%
\pgfpathmoveto{\pgfqpoint{1.891600in}{2.465308in}}%
\pgfpathcurveto{\pgfqpoint{1.899836in}{2.465308in}}{\pgfqpoint{1.907736in}{2.468580in}}{\pgfqpoint{1.913560in}{2.474404in}}%
\pgfpathcurveto{\pgfqpoint{1.919384in}{2.480228in}}{\pgfqpoint{1.922657in}{2.488128in}}{\pgfqpoint{1.922657in}{2.496364in}}%
\pgfpathcurveto{\pgfqpoint{1.922657in}{2.504601in}}{\pgfqpoint{1.919384in}{2.512501in}}{\pgfqpoint{1.913560in}{2.518325in}}%
\pgfpathcurveto{\pgfqpoint{1.907736in}{2.524149in}}{\pgfqpoint{1.899836in}{2.527421in}}{\pgfqpoint{1.891600in}{2.527421in}}%
\pgfpathcurveto{\pgfqpoint{1.883364in}{2.527421in}}{\pgfqpoint{1.875464in}{2.524149in}}{\pgfqpoint{1.869640in}{2.518325in}}%
\pgfpathcurveto{\pgfqpoint{1.863816in}{2.512501in}}{\pgfqpoint{1.860544in}{2.504601in}}{\pgfqpoint{1.860544in}{2.496364in}}%
\pgfpathcurveto{\pgfqpoint{1.860544in}{2.488128in}}{\pgfqpoint{1.863816in}{2.480228in}}{\pgfqpoint{1.869640in}{2.474404in}}%
\pgfpathcurveto{\pgfqpoint{1.875464in}{2.468580in}}{\pgfqpoint{1.883364in}{2.465308in}}{\pgfqpoint{1.891600in}{2.465308in}}%
\pgfpathclose%
\pgfusepath{stroke,fill}%
\end{pgfscope}%
\begin{pgfscope}%
\pgfpathrectangle{\pgfqpoint{0.100000in}{0.212622in}}{\pgfqpoint{3.696000in}{3.696000in}}%
\pgfusepath{clip}%
\pgfsetbuttcap%
\pgfsetroundjoin%
\definecolor{currentfill}{rgb}{0.121569,0.466667,0.705882}%
\pgfsetfillcolor{currentfill}%
\pgfsetfillopacity{0.945969}%
\pgfsetlinewidth{1.003750pt}%
\definecolor{currentstroke}{rgb}{0.121569,0.466667,0.705882}%
\pgfsetstrokecolor{currentstroke}%
\pgfsetstrokeopacity{0.945969}%
\pgfsetdash{}{0pt}%
\pgfpathmoveto{\pgfqpoint{1.296599in}{1.860118in}}%
\pgfpathcurveto{\pgfqpoint{1.304835in}{1.860118in}}{\pgfqpoint{1.312735in}{1.863390in}}{\pgfqpoint{1.318559in}{1.869214in}}%
\pgfpathcurveto{\pgfqpoint{1.324383in}{1.875038in}}{\pgfqpoint{1.327655in}{1.882938in}}{\pgfqpoint{1.327655in}{1.891174in}}%
\pgfpathcurveto{\pgfqpoint{1.327655in}{1.899410in}}{\pgfqpoint{1.324383in}{1.907310in}}{\pgfqpoint{1.318559in}{1.913134in}}%
\pgfpathcurveto{\pgfqpoint{1.312735in}{1.918958in}}{\pgfqpoint{1.304835in}{1.922231in}}{\pgfqpoint{1.296599in}{1.922231in}}%
\pgfpathcurveto{\pgfqpoint{1.288362in}{1.922231in}}{\pgfqpoint{1.280462in}{1.918958in}}{\pgfqpoint{1.274638in}{1.913134in}}%
\pgfpathcurveto{\pgfqpoint{1.268814in}{1.907310in}}{\pgfqpoint{1.265542in}{1.899410in}}{\pgfqpoint{1.265542in}{1.891174in}}%
\pgfpathcurveto{\pgfqpoint{1.265542in}{1.882938in}}{\pgfqpoint{1.268814in}{1.875038in}}{\pgfqpoint{1.274638in}{1.869214in}}%
\pgfpathcurveto{\pgfqpoint{1.280462in}{1.863390in}}{\pgfqpoint{1.288362in}{1.860118in}}{\pgfqpoint{1.296599in}{1.860118in}}%
\pgfpathclose%
\pgfusepath{stroke,fill}%
\end{pgfscope}%
\begin{pgfscope}%
\pgfpathrectangle{\pgfqpoint{0.100000in}{0.212622in}}{\pgfqpoint{3.696000in}{3.696000in}}%
\pgfusepath{clip}%
\pgfsetbuttcap%
\pgfsetroundjoin%
\definecolor{currentfill}{rgb}{0.121569,0.466667,0.705882}%
\pgfsetfillcolor{currentfill}%
\pgfsetfillopacity{0.946519}%
\pgfsetlinewidth{1.003750pt}%
\definecolor{currentstroke}{rgb}{0.121569,0.466667,0.705882}%
\pgfsetstrokecolor{currentstroke}%
\pgfsetstrokeopacity{0.946519}%
\pgfsetdash{}{0pt}%
\pgfpathmoveto{\pgfqpoint{1.896333in}{2.462981in}}%
\pgfpathcurveto{\pgfqpoint{1.904569in}{2.462981in}}{\pgfqpoint{1.912469in}{2.466254in}}{\pgfqpoint{1.918293in}{2.472078in}}%
\pgfpathcurveto{\pgfqpoint{1.924117in}{2.477902in}}{\pgfqpoint{1.927389in}{2.485802in}}{\pgfqpoint{1.927389in}{2.494038in}}%
\pgfpathcurveto{\pgfqpoint{1.927389in}{2.502274in}}{\pgfqpoint{1.924117in}{2.510174in}}{\pgfqpoint{1.918293in}{2.515998in}}%
\pgfpathcurveto{\pgfqpoint{1.912469in}{2.521822in}}{\pgfqpoint{1.904569in}{2.525094in}}{\pgfqpoint{1.896333in}{2.525094in}}%
\pgfpathcurveto{\pgfqpoint{1.888096in}{2.525094in}}{\pgfqpoint{1.880196in}{2.521822in}}{\pgfqpoint{1.874372in}{2.515998in}}%
\pgfpathcurveto{\pgfqpoint{1.868548in}{2.510174in}}{\pgfqpoint{1.865276in}{2.502274in}}{\pgfqpoint{1.865276in}{2.494038in}}%
\pgfpathcurveto{\pgfqpoint{1.865276in}{2.485802in}}{\pgfqpoint{1.868548in}{2.477902in}}{\pgfqpoint{1.874372in}{2.472078in}}%
\pgfpathcurveto{\pgfqpoint{1.880196in}{2.466254in}}{\pgfqpoint{1.888096in}{2.462981in}}{\pgfqpoint{1.896333in}{2.462981in}}%
\pgfpathclose%
\pgfusepath{stroke,fill}%
\end{pgfscope}%
\begin{pgfscope}%
\pgfpathrectangle{\pgfqpoint{0.100000in}{0.212622in}}{\pgfqpoint{3.696000in}{3.696000in}}%
\pgfusepath{clip}%
\pgfsetbuttcap%
\pgfsetroundjoin%
\definecolor{currentfill}{rgb}{0.121569,0.466667,0.705882}%
\pgfsetfillcolor{currentfill}%
\pgfsetfillopacity{0.946526}%
\pgfsetlinewidth{1.003750pt}%
\definecolor{currentstroke}{rgb}{0.121569,0.466667,0.705882}%
\pgfsetstrokecolor{currentstroke}%
\pgfsetstrokeopacity{0.946526}%
\pgfsetdash{}{0pt}%
\pgfpathmoveto{\pgfqpoint{1.303729in}{1.855943in}}%
\pgfpathcurveto{\pgfqpoint{1.311965in}{1.855943in}}{\pgfqpoint{1.319865in}{1.859215in}}{\pgfqpoint{1.325689in}{1.865039in}}%
\pgfpathcurveto{\pgfqpoint{1.331513in}{1.870863in}}{\pgfqpoint{1.334786in}{1.878763in}}{\pgfqpoint{1.334786in}{1.886999in}}%
\pgfpathcurveto{\pgfqpoint{1.334786in}{1.895235in}}{\pgfqpoint{1.331513in}{1.903135in}}{\pgfqpoint{1.325689in}{1.908959in}}%
\pgfpathcurveto{\pgfqpoint{1.319865in}{1.914783in}}{\pgfqpoint{1.311965in}{1.918056in}}{\pgfqpoint{1.303729in}{1.918056in}}%
\pgfpathcurveto{\pgfqpoint{1.295493in}{1.918056in}}{\pgfqpoint{1.287593in}{1.914783in}}{\pgfqpoint{1.281769in}{1.908959in}}%
\pgfpathcurveto{\pgfqpoint{1.275945in}{1.903135in}}{\pgfqpoint{1.272673in}{1.895235in}}{\pgfqpoint{1.272673in}{1.886999in}}%
\pgfpathcurveto{\pgfqpoint{1.272673in}{1.878763in}}{\pgfqpoint{1.275945in}{1.870863in}}{\pgfqpoint{1.281769in}{1.865039in}}%
\pgfpathcurveto{\pgfqpoint{1.287593in}{1.859215in}}{\pgfqpoint{1.295493in}{1.855943in}}{\pgfqpoint{1.303729in}{1.855943in}}%
\pgfpathclose%
\pgfusepath{stroke,fill}%
\end{pgfscope}%
\begin{pgfscope}%
\pgfpathrectangle{\pgfqpoint{0.100000in}{0.212622in}}{\pgfqpoint{3.696000in}{3.696000in}}%
\pgfusepath{clip}%
\pgfsetbuttcap%
\pgfsetroundjoin%
\definecolor{currentfill}{rgb}{0.121569,0.466667,0.705882}%
\pgfsetfillcolor{currentfill}%
\pgfsetfillopacity{0.946999}%
\pgfsetlinewidth{1.003750pt}%
\definecolor{currentstroke}{rgb}{0.121569,0.466667,0.705882}%
\pgfsetstrokecolor{currentstroke}%
\pgfsetstrokeopacity{0.946999}%
\pgfsetdash{}{0pt}%
\pgfpathmoveto{\pgfqpoint{1.901787in}{2.459974in}}%
\pgfpathcurveto{\pgfqpoint{1.910023in}{2.459974in}}{\pgfqpoint{1.917923in}{2.463247in}}{\pgfqpoint{1.923747in}{2.469071in}}%
\pgfpathcurveto{\pgfqpoint{1.929571in}{2.474895in}}{\pgfqpoint{1.932843in}{2.482795in}}{\pgfqpoint{1.932843in}{2.491031in}}%
\pgfpathcurveto{\pgfqpoint{1.932843in}{2.499267in}}{\pgfqpoint{1.929571in}{2.507167in}}{\pgfqpoint{1.923747in}{2.512991in}}%
\pgfpathcurveto{\pgfqpoint{1.917923in}{2.518815in}}{\pgfqpoint{1.910023in}{2.522087in}}{\pgfqpoint{1.901787in}{2.522087in}}%
\pgfpathcurveto{\pgfqpoint{1.893551in}{2.522087in}}{\pgfqpoint{1.885651in}{2.518815in}}{\pgfqpoint{1.879827in}{2.512991in}}%
\pgfpathcurveto{\pgfqpoint{1.874003in}{2.507167in}}{\pgfqpoint{1.870730in}{2.499267in}}{\pgfqpoint{1.870730in}{2.491031in}}%
\pgfpathcurveto{\pgfqpoint{1.870730in}{2.482795in}}{\pgfqpoint{1.874003in}{2.474895in}}{\pgfqpoint{1.879827in}{2.469071in}}%
\pgfpathcurveto{\pgfqpoint{1.885651in}{2.463247in}}{\pgfqpoint{1.893551in}{2.459974in}}{\pgfqpoint{1.901787in}{2.459974in}}%
\pgfpathclose%
\pgfusepath{stroke,fill}%
\end{pgfscope}%
\begin{pgfscope}%
\pgfpathrectangle{\pgfqpoint{0.100000in}{0.212622in}}{\pgfqpoint{3.696000in}{3.696000in}}%
\pgfusepath{clip}%
\pgfsetbuttcap%
\pgfsetroundjoin%
\definecolor{currentfill}{rgb}{0.121569,0.466667,0.705882}%
\pgfsetfillcolor{currentfill}%
\pgfsetfillopacity{0.947095}%
\pgfsetlinewidth{1.003750pt}%
\definecolor{currentstroke}{rgb}{0.121569,0.466667,0.705882}%
\pgfsetstrokecolor{currentstroke}%
\pgfsetstrokeopacity{0.947095}%
\pgfsetdash{}{0pt}%
\pgfpathmoveto{\pgfqpoint{2.093475in}{2.414346in}}%
\pgfpathcurveto{\pgfqpoint{2.101712in}{2.414346in}}{\pgfqpoint{2.109612in}{2.417618in}}{\pgfqpoint{2.115436in}{2.423442in}}%
\pgfpathcurveto{\pgfqpoint{2.121260in}{2.429266in}}{\pgfqpoint{2.124532in}{2.437166in}}{\pgfqpoint{2.124532in}{2.445403in}}%
\pgfpathcurveto{\pgfqpoint{2.124532in}{2.453639in}}{\pgfqpoint{2.121260in}{2.461539in}}{\pgfqpoint{2.115436in}{2.467363in}}%
\pgfpathcurveto{\pgfqpoint{2.109612in}{2.473187in}}{\pgfqpoint{2.101712in}{2.476459in}}{\pgfqpoint{2.093475in}{2.476459in}}%
\pgfpathcurveto{\pgfqpoint{2.085239in}{2.476459in}}{\pgfqpoint{2.077339in}{2.473187in}}{\pgfqpoint{2.071515in}{2.467363in}}%
\pgfpathcurveto{\pgfqpoint{2.065691in}{2.461539in}}{\pgfqpoint{2.062419in}{2.453639in}}{\pgfqpoint{2.062419in}{2.445403in}}%
\pgfpathcurveto{\pgfqpoint{2.062419in}{2.437166in}}{\pgfqpoint{2.065691in}{2.429266in}}{\pgfqpoint{2.071515in}{2.423442in}}%
\pgfpathcurveto{\pgfqpoint{2.077339in}{2.417618in}}{\pgfqpoint{2.085239in}{2.414346in}}{\pgfqpoint{2.093475in}{2.414346in}}%
\pgfpathclose%
\pgfusepath{stroke,fill}%
\end{pgfscope}%
\begin{pgfscope}%
\pgfpathrectangle{\pgfqpoint{0.100000in}{0.212622in}}{\pgfqpoint{3.696000in}{3.696000in}}%
\pgfusepath{clip}%
\pgfsetbuttcap%
\pgfsetroundjoin%
\definecolor{currentfill}{rgb}{0.121569,0.466667,0.705882}%
\pgfsetfillcolor{currentfill}%
\pgfsetfillopacity{0.947240}%
\pgfsetlinewidth{1.003750pt}%
\definecolor{currentstroke}{rgb}{0.121569,0.466667,0.705882}%
\pgfsetstrokecolor{currentstroke}%
\pgfsetstrokeopacity{0.947240}%
\pgfsetdash{}{0pt}%
\pgfpathmoveto{\pgfqpoint{1.312386in}{1.851032in}}%
\pgfpathcurveto{\pgfqpoint{1.320622in}{1.851032in}}{\pgfqpoint{1.328522in}{1.854304in}}{\pgfqpoint{1.334346in}{1.860128in}}%
\pgfpathcurveto{\pgfqpoint{1.340170in}{1.865952in}}{\pgfqpoint{1.343442in}{1.873852in}}{\pgfqpoint{1.343442in}{1.882089in}}%
\pgfpathcurveto{\pgfqpoint{1.343442in}{1.890325in}}{\pgfqpoint{1.340170in}{1.898225in}}{\pgfqpoint{1.334346in}{1.904049in}}%
\pgfpathcurveto{\pgfqpoint{1.328522in}{1.909873in}}{\pgfqpoint{1.320622in}{1.913145in}}{\pgfqpoint{1.312386in}{1.913145in}}%
\pgfpathcurveto{\pgfqpoint{1.304150in}{1.913145in}}{\pgfqpoint{1.296250in}{1.909873in}}{\pgfqpoint{1.290426in}{1.904049in}}%
\pgfpathcurveto{\pgfqpoint{1.284602in}{1.898225in}}{\pgfqpoint{1.281329in}{1.890325in}}{\pgfqpoint{1.281329in}{1.882089in}}%
\pgfpathcurveto{\pgfqpoint{1.281329in}{1.873852in}}{\pgfqpoint{1.284602in}{1.865952in}}{\pgfqpoint{1.290426in}{1.860128in}}%
\pgfpathcurveto{\pgfqpoint{1.296250in}{1.854304in}}{\pgfqpoint{1.304150in}{1.851032in}}{\pgfqpoint{1.312386in}{1.851032in}}%
\pgfpathclose%
\pgfusepath{stroke,fill}%
\end{pgfscope}%
\begin{pgfscope}%
\pgfpathrectangle{\pgfqpoint{0.100000in}{0.212622in}}{\pgfqpoint{3.696000in}{3.696000in}}%
\pgfusepath{clip}%
\pgfsetbuttcap%
\pgfsetroundjoin%
\definecolor{currentfill}{rgb}{0.121569,0.466667,0.705882}%
\pgfsetfillcolor{currentfill}%
\pgfsetfillopacity{0.947528}%
\pgfsetlinewidth{1.003750pt}%
\definecolor{currentstroke}{rgb}{0.121569,0.466667,0.705882}%
\pgfsetstrokecolor{currentstroke}%
\pgfsetstrokeopacity{0.947528}%
\pgfsetdash{}{0pt}%
\pgfpathmoveto{\pgfqpoint{1.907692in}{2.456385in}}%
\pgfpathcurveto{\pgfqpoint{1.915928in}{2.456385in}}{\pgfqpoint{1.923828in}{2.459657in}}{\pgfqpoint{1.929652in}{2.465481in}}%
\pgfpathcurveto{\pgfqpoint{1.935476in}{2.471305in}}{\pgfqpoint{1.938748in}{2.479205in}}{\pgfqpoint{1.938748in}{2.487441in}}%
\pgfpathcurveto{\pgfqpoint{1.938748in}{2.495678in}}{\pgfqpoint{1.935476in}{2.503578in}}{\pgfqpoint{1.929652in}{2.509402in}}%
\pgfpathcurveto{\pgfqpoint{1.923828in}{2.515226in}}{\pgfqpoint{1.915928in}{2.518498in}}{\pgfqpoint{1.907692in}{2.518498in}}%
\pgfpathcurveto{\pgfqpoint{1.899455in}{2.518498in}}{\pgfqpoint{1.891555in}{2.515226in}}{\pgfqpoint{1.885731in}{2.509402in}}%
\pgfpathcurveto{\pgfqpoint{1.879907in}{2.503578in}}{\pgfqpoint{1.876635in}{2.495678in}}{\pgfqpoint{1.876635in}{2.487441in}}%
\pgfpathcurveto{\pgfqpoint{1.876635in}{2.479205in}}{\pgfqpoint{1.879907in}{2.471305in}}{\pgfqpoint{1.885731in}{2.465481in}}%
\pgfpathcurveto{\pgfqpoint{1.891555in}{2.459657in}}{\pgfqpoint{1.899455in}{2.456385in}}{\pgfqpoint{1.907692in}{2.456385in}}%
\pgfpathclose%
\pgfusepath{stroke,fill}%
\end{pgfscope}%
\begin{pgfscope}%
\pgfpathrectangle{\pgfqpoint{0.100000in}{0.212622in}}{\pgfqpoint{3.696000in}{3.696000in}}%
\pgfusepath{clip}%
\pgfsetbuttcap%
\pgfsetroundjoin%
\definecolor{currentfill}{rgb}{0.121569,0.466667,0.705882}%
\pgfsetfillcolor{currentfill}%
\pgfsetfillopacity{0.947962}%
\pgfsetlinewidth{1.003750pt}%
\definecolor{currentstroke}{rgb}{0.121569,0.466667,0.705882}%
\pgfsetstrokecolor{currentstroke}%
\pgfsetstrokeopacity{0.947962}%
\pgfsetdash{}{0pt}%
\pgfpathmoveto{\pgfqpoint{1.321811in}{1.845077in}}%
\pgfpathcurveto{\pgfqpoint{1.330047in}{1.845077in}}{\pgfqpoint{1.337947in}{1.848349in}}{\pgfqpoint{1.343771in}{1.854173in}}%
\pgfpathcurveto{\pgfqpoint{1.349595in}{1.859997in}}{\pgfqpoint{1.352867in}{1.867897in}}{\pgfqpoint{1.352867in}{1.876134in}}%
\pgfpathcurveto{\pgfqpoint{1.352867in}{1.884370in}}{\pgfqpoint{1.349595in}{1.892270in}}{\pgfqpoint{1.343771in}{1.898094in}}%
\pgfpathcurveto{\pgfqpoint{1.337947in}{1.903918in}}{\pgfqpoint{1.330047in}{1.907190in}}{\pgfqpoint{1.321811in}{1.907190in}}%
\pgfpathcurveto{\pgfqpoint{1.313575in}{1.907190in}}{\pgfqpoint{1.305675in}{1.903918in}}{\pgfqpoint{1.299851in}{1.898094in}}%
\pgfpathcurveto{\pgfqpoint{1.294027in}{1.892270in}}{\pgfqpoint{1.290754in}{1.884370in}}{\pgfqpoint{1.290754in}{1.876134in}}%
\pgfpathcurveto{\pgfqpoint{1.290754in}{1.867897in}}{\pgfqpoint{1.294027in}{1.859997in}}{\pgfqpoint{1.299851in}{1.854173in}}%
\pgfpathcurveto{\pgfqpoint{1.305675in}{1.848349in}}{\pgfqpoint{1.313575in}{1.845077in}}{\pgfqpoint{1.321811in}{1.845077in}}%
\pgfpathclose%
\pgfusepath{stroke,fill}%
\end{pgfscope}%
\begin{pgfscope}%
\pgfpathrectangle{\pgfqpoint{0.100000in}{0.212622in}}{\pgfqpoint{3.696000in}{3.696000in}}%
\pgfusepath{clip}%
\pgfsetbuttcap%
\pgfsetroundjoin%
\definecolor{currentfill}{rgb}{0.121569,0.466667,0.705882}%
\pgfsetfillcolor{currentfill}%
\pgfsetfillopacity{0.948217}%
\pgfsetlinewidth{1.003750pt}%
\definecolor{currentstroke}{rgb}{0.121569,0.466667,0.705882}%
\pgfsetstrokecolor{currentstroke}%
\pgfsetstrokeopacity{0.948217}%
\pgfsetdash{}{0pt}%
\pgfpathmoveto{\pgfqpoint{2.091098in}{2.412034in}}%
\pgfpathcurveto{\pgfqpoint{2.099334in}{2.412034in}}{\pgfqpoint{2.107234in}{2.415306in}}{\pgfqpoint{2.113058in}{2.421130in}}%
\pgfpathcurveto{\pgfqpoint{2.118882in}{2.426954in}}{\pgfqpoint{2.122154in}{2.434854in}}{\pgfqpoint{2.122154in}{2.443090in}}%
\pgfpathcurveto{\pgfqpoint{2.122154in}{2.451327in}}{\pgfqpoint{2.118882in}{2.459227in}}{\pgfqpoint{2.113058in}{2.465051in}}%
\pgfpathcurveto{\pgfqpoint{2.107234in}{2.470875in}}{\pgfqpoint{2.099334in}{2.474147in}}{\pgfqpoint{2.091098in}{2.474147in}}%
\pgfpathcurveto{\pgfqpoint{2.082862in}{2.474147in}}{\pgfqpoint{2.074962in}{2.470875in}}{\pgfqpoint{2.069138in}{2.465051in}}%
\pgfpathcurveto{\pgfqpoint{2.063314in}{2.459227in}}{\pgfqpoint{2.060041in}{2.451327in}}{\pgfqpoint{2.060041in}{2.443090in}}%
\pgfpathcurveto{\pgfqpoint{2.060041in}{2.434854in}}{\pgfqpoint{2.063314in}{2.426954in}}{\pgfqpoint{2.069138in}{2.421130in}}%
\pgfpathcurveto{\pgfqpoint{2.074962in}{2.415306in}}{\pgfqpoint{2.082862in}{2.412034in}}{\pgfqpoint{2.091098in}{2.412034in}}%
\pgfpathclose%
\pgfusepath{stroke,fill}%
\end{pgfscope}%
\begin{pgfscope}%
\pgfpathrectangle{\pgfqpoint{0.100000in}{0.212622in}}{\pgfqpoint{3.696000in}{3.696000in}}%
\pgfusepath{clip}%
\pgfsetbuttcap%
\pgfsetroundjoin%
\definecolor{currentfill}{rgb}{0.121569,0.466667,0.705882}%
\pgfsetfillcolor{currentfill}%
\pgfsetfillopacity{0.948254}%
\pgfsetlinewidth{1.003750pt}%
\definecolor{currentstroke}{rgb}{0.121569,0.466667,0.705882}%
\pgfsetstrokecolor{currentstroke}%
\pgfsetstrokeopacity{0.948254}%
\pgfsetdash{}{0pt}%
\pgfpathmoveto{\pgfqpoint{2.588821in}{1.202279in}}%
\pgfpathcurveto{\pgfqpoint{2.597057in}{1.202279in}}{\pgfqpoint{2.604957in}{1.205552in}}{\pgfqpoint{2.610781in}{1.211376in}}%
\pgfpathcurveto{\pgfqpoint{2.616605in}{1.217200in}}{\pgfqpoint{2.619878in}{1.225100in}}{\pgfqpoint{2.619878in}{1.233336in}}%
\pgfpathcurveto{\pgfqpoint{2.619878in}{1.241572in}}{\pgfqpoint{2.616605in}{1.249472in}}{\pgfqpoint{2.610781in}{1.255296in}}%
\pgfpathcurveto{\pgfqpoint{2.604957in}{1.261120in}}{\pgfqpoint{2.597057in}{1.264392in}}{\pgfqpoint{2.588821in}{1.264392in}}%
\pgfpathcurveto{\pgfqpoint{2.580585in}{1.264392in}}{\pgfqpoint{2.572685in}{1.261120in}}{\pgfqpoint{2.566861in}{1.255296in}}%
\pgfpathcurveto{\pgfqpoint{2.561037in}{1.249472in}}{\pgfqpoint{2.557765in}{1.241572in}}{\pgfqpoint{2.557765in}{1.233336in}}%
\pgfpathcurveto{\pgfqpoint{2.557765in}{1.225100in}}{\pgfqpoint{2.561037in}{1.217200in}}{\pgfqpoint{2.566861in}{1.211376in}}%
\pgfpathcurveto{\pgfqpoint{2.572685in}{1.205552in}}{\pgfqpoint{2.580585in}{1.202279in}}{\pgfqpoint{2.588821in}{1.202279in}}%
\pgfpathclose%
\pgfusepath{stroke,fill}%
\end{pgfscope}%
\begin{pgfscope}%
\pgfpathrectangle{\pgfqpoint{0.100000in}{0.212622in}}{\pgfqpoint{3.696000in}{3.696000in}}%
\pgfusepath{clip}%
\pgfsetbuttcap%
\pgfsetroundjoin%
\definecolor{currentfill}{rgb}{0.121569,0.466667,0.705882}%
\pgfsetfillcolor{currentfill}%
\pgfsetfillopacity{0.948306}%
\pgfsetlinewidth{1.003750pt}%
\definecolor{currentstroke}{rgb}{0.121569,0.466667,0.705882}%
\pgfsetstrokecolor{currentstroke}%
\pgfsetstrokeopacity{0.948306}%
\pgfsetdash{}{0pt}%
\pgfpathmoveto{\pgfqpoint{1.914138in}{2.452811in}}%
\pgfpathcurveto{\pgfqpoint{1.922375in}{2.452811in}}{\pgfqpoint{1.930275in}{2.456084in}}{\pgfqpoint{1.936099in}{2.461908in}}%
\pgfpathcurveto{\pgfqpoint{1.941923in}{2.467731in}}{\pgfqpoint{1.945195in}{2.475632in}}{\pgfqpoint{1.945195in}{2.483868in}}%
\pgfpathcurveto{\pgfqpoint{1.945195in}{2.492104in}}{\pgfqpoint{1.941923in}{2.500004in}}{\pgfqpoint{1.936099in}{2.505828in}}%
\pgfpathcurveto{\pgfqpoint{1.930275in}{2.511652in}}{\pgfqpoint{1.922375in}{2.514924in}}{\pgfqpoint{1.914138in}{2.514924in}}%
\pgfpathcurveto{\pgfqpoint{1.905902in}{2.514924in}}{\pgfqpoint{1.898002in}{2.511652in}}{\pgfqpoint{1.892178in}{2.505828in}}%
\pgfpathcurveto{\pgfqpoint{1.886354in}{2.500004in}}{\pgfqpoint{1.883082in}{2.492104in}}{\pgfqpoint{1.883082in}{2.483868in}}%
\pgfpathcurveto{\pgfqpoint{1.883082in}{2.475632in}}{\pgfqpoint{1.886354in}{2.467731in}}{\pgfqpoint{1.892178in}{2.461908in}}%
\pgfpathcurveto{\pgfqpoint{1.898002in}{2.456084in}}{\pgfqpoint{1.905902in}{2.452811in}}{\pgfqpoint{1.914138in}{2.452811in}}%
\pgfpathclose%
\pgfusepath{stroke,fill}%
\end{pgfscope}%
\begin{pgfscope}%
\pgfpathrectangle{\pgfqpoint{0.100000in}{0.212622in}}{\pgfqpoint{3.696000in}{3.696000in}}%
\pgfusepath{clip}%
\pgfsetbuttcap%
\pgfsetroundjoin%
\definecolor{currentfill}{rgb}{0.121569,0.466667,0.705882}%
\pgfsetfillcolor{currentfill}%
\pgfsetfillopacity{0.948363}%
\pgfsetlinewidth{1.003750pt}%
\definecolor{currentstroke}{rgb}{0.121569,0.466667,0.705882}%
\pgfsetstrokecolor{currentstroke}%
\pgfsetstrokeopacity{0.948363}%
\pgfsetdash{}{0pt}%
\pgfpathmoveto{\pgfqpoint{1.326912in}{1.841530in}}%
\pgfpathcurveto{\pgfqpoint{1.335148in}{1.841530in}}{\pgfqpoint{1.343048in}{1.844802in}}{\pgfqpoint{1.348872in}{1.850626in}}%
\pgfpathcurveto{\pgfqpoint{1.354696in}{1.856450in}}{\pgfqpoint{1.357969in}{1.864350in}}{\pgfqpoint{1.357969in}{1.872586in}}%
\pgfpathcurveto{\pgfqpoint{1.357969in}{1.880823in}}{\pgfqpoint{1.354696in}{1.888723in}}{\pgfqpoint{1.348872in}{1.894547in}}%
\pgfpathcurveto{\pgfqpoint{1.343048in}{1.900371in}}{\pgfqpoint{1.335148in}{1.903643in}}{\pgfqpoint{1.326912in}{1.903643in}}%
\pgfpathcurveto{\pgfqpoint{1.318676in}{1.903643in}}{\pgfqpoint{1.310776in}{1.900371in}}{\pgfqpoint{1.304952in}{1.894547in}}%
\pgfpathcurveto{\pgfqpoint{1.299128in}{1.888723in}}{\pgfqpoint{1.295856in}{1.880823in}}{\pgfqpoint{1.295856in}{1.872586in}}%
\pgfpathcurveto{\pgfqpoint{1.295856in}{1.864350in}}{\pgfqpoint{1.299128in}{1.856450in}}{\pgfqpoint{1.304952in}{1.850626in}}%
\pgfpathcurveto{\pgfqpoint{1.310776in}{1.844802in}}{\pgfqpoint{1.318676in}{1.841530in}}{\pgfqpoint{1.326912in}{1.841530in}}%
\pgfpathclose%
\pgfusepath{stroke,fill}%
\end{pgfscope}%
\begin{pgfscope}%
\pgfpathrectangle{\pgfqpoint{0.100000in}{0.212622in}}{\pgfqpoint{3.696000in}{3.696000in}}%
\pgfusepath{clip}%
\pgfsetbuttcap%
\pgfsetroundjoin%
\definecolor{currentfill}{rgb}{0.121569,0.466667,0.705882}%
\pgfsetfillcolor{currentfill}%
\pgfsetfillopacity{0.948883}%
\pgfsetlinewidth{1.003750pt}%
\definecolor{currentstroke}{rgb}{0.121569,0.466667,0.705882}%
\pgfsetstrokecolor{currentstroke}%
\pgfsetstrokeopacity{0.948883}%
\pgfsetdash{}{0pt}%
\pgfpathmoveto{\pgfqpoint{1.333155in}{1.837238in}}%
\pgfpathcurveto{\pgfqpoint{1.341392in}{1.837238in}}{\pgfqpoint{1.349292in}{1.840510in}}{\pgfqpoint{1.355116in}{1.846334in}}%
\pgfpathcurveto{\pgfqpoint{1.360939in}{1.852158in}}{\pgfqpoint{1.364212in}{1.860058in}}{\pgfqpoint{1.364212in}{1.868294in}}%
\pgfpathcurveto{\pgfqpoint{1.364212in}{1.876531in}}{\pgfqpoint{1.360939in}{1.884431in}}{\pgfqpoint{1.355116in}{1.890255in}}%
\pgfpathcurveto{\pgfqpoint{1.349292in}{1.896079in}}{\pgfqpoint{1.341392in}{1.899351in}}{\pgfqpoint{1.333155in}{1.899351in}}%
\pgfpathcurveto{\pgfqpoint{1.324919in}{1.899351in}}{\pgfqpoint{1.317019in}{1.896079in}}{\pgfqpoint{1.311195in}{1.890255in}}%
\pgfpathcurveto{\pgfqpoint{1.305371in}{1.884431in}}{\pgfqpoint{1.302099in}{1.876531in}}{\pgfqpoint{1.302099in}{1.868294in}}%
\pgfpathcurveto{\pgfqpoint{1.302099in}{1.860058in}}{\pgfqpoint{1.305371in}{1.852158in}}{\pgfqpoint{1.311195in}{1.846334in}}%
\pgfpathcurveto{\pgfqpoint{1.317019in}{1.840510in}}{\pgfqpoint{1.324919in}{1.837238in}}{\pgfqpoint{1.333155in}{1.837238in}}%
\pgfpathclose%
\pgfusepath{stroke,fill}%
\end{pgfscope}%
\begin{pgfscope}%
\pgfpathrectangle{\pgfqpoint{0.100000in}{0.212622in}}{\pgfqpoint{3.696000in}{3.696000in}}%
\pgfusepath{clip}%
\pgfsetbuttcap%
\pgfsetroundjoin%
\definecolor{currentfill}{rgb}{0.121569,0.466667,0.705882}%
\pgfsetfillcolor{currentfill}%
\pgfsetfillopacity{0.949197}%
\pgfsetlinewidth{1.003750pt}%
\definecolor{currentstroke}{rgb}{0.121569,0.466667,0.705882}%
\pgfsetstrokecolor{currentstroke}%
\pgfsetstrokeopacity{0.949197}%
\pgfsetdash{}{0pt}%
\pgfpathmoveto{\pgfqpoint{1.922289in}{2.449532in}}%
\pgfpathcurveto{\pgfqpoint{1.930525in}{2.449532in}}{\pgfqpoint{1.938425in}{2.452804in}}{\pgfqpoint{1.944249in}{2.458628in}}%
\pgfpathcurveto{\pgfqpoint{1.950073in}{2.464452in}}{\pgfqpoint{1.953346in}{2.472352in}}{\pgfqpoint{1.953346in}{2.480589in}}%
\pgfpathcurveto{\pgfqpoint{1.953346in}{2.488825in}}{\pgfqpoint{1.950073in}{2.496725in}}{\pgfqpoint{1.944249in}{2.502549in}}%
\pgfpathcurveto{\pgfqpoint{1.938425in}{2.508373in}}{\pgfqpoint{1.930525in}{2.511645in}}{\pgfqpoint{1.922289in}{2.511645in}}%
\pgfpathcurveto{\pgfqpoint{1.914053in}{2.511645in}}{\pgfqpoint{1.906153in}{2.508373in}}{\pgfqpoint{1.900329in}{2.502549in}}%
\pgfpathcurveto{\pgfqpoint{1.894505in}{2.496725in}}{\pgfqpoint{1.891233in}{2.488825in}}{\pgfqpoint{1.891233in}{2.480589in}}%
\pgfpathcurveto{\pgfqpoint{1.891233in}{2.472352in}}{\pgfqpoint{1.894505in}{2.464452in}}{\pgfqpoint{1.900329in}{2.458628in}}%
\pgfpathcurveto{\pgfqpoint{1.906153in}{2.452804in}}{\pgfqpoint{1.914053in}{2.449532in}}{\pgfqpoint{1.922289in}{2.449532in}}%
\pgfpathclose%
\pgfusepath{stroke,fill}%
\end{pgfscope}%
\begin{pgfscope}%
\pgfpathrectangle{\pgfqpoint{0.100000in}{0.212622in}}{\pgfqpoint{3.696000in}{3.696000in}}%
\pgfusepath{clip}%
\pgfsetbuttcap%
\pgfsetroundjoin%
\definecolor{currentfill}{rgb}{0.121569,0.466667,0.705882}%
\pgfsetfillcolor{currentfill}%
\pgfsetfillopacity{0.949233}%
\pgfsetlinewidth{1.003750pt}%
\definecolor{currentstroke}{rgb}{0.121569,0.466667,0.705882}%
\pgfsetstrokecolor{currentstroke}%
\pgfsetstrokeopacity{0.949233}%
\pgfsetdash{}{0pt}%
\pgfpathmoveto{\pgfqpoint{2.088836in}{2.410020in}}%
\pgfpathcurveto{\pgfqpoint{2.097073in}{2.410020in}}{\pgfqpoint{2.104973in}{2.413292in}}{\pgfqpoint{2.110797in}{2.419116in}}%
\pgfpathcurveto{\pgfqpoint{2.116620in}{2.424940in}}{\pgfqpoint{2.119893in}{2.432840in}}{\pgfqpoint{2.119893in}{2.441077in}}%
\pgfpathcurveto{\pgfqpoint{2.119893in}{2.449313in}}{\pgfqpoint{2.116620in}{2.457213in}}{\pgfqpoint{2.110797in}{2.463037in}}%
\pgfpathcurveto{\pgfqpoint{2.104973in}{2.468861in}}{\pgfqpoint{2.097073in}{2.472133in}}{\pgfqpoint{2.088836in}{2.472133in}}%
\pgfpathcurveto{\pgfqpoint{2.080600in}{2.472133in}}{\pgfqpoint{2.072700in}{2.468861in}}{\pgfqpoint{2.066876in}{2.463037in}}%
\pgfpathcurveto{\pgfqpoint{2.061052in}{2.457213in}}{\pgfqpoint{2.057780in}{2.449313in}}{\pgfqpoint{2.057780in}{2.441077in}}%
\pgfpathcurveto{\pgfqpoint{2.057780in}{2.432840in}}{\pgfqpoint{2.061052in}{2.424940in}}{\pgfqpoint{2.066876in}{2.419116in}}%
\pgfpathcurveto{\pgfqpoint{2.072700in}{2.413292in}}{\pgfqpoint{2.080600in}{2.410020in}}{\pgfqpoint{2.088836in}{2.410020in}}%
\pgfpathclose%
\pgfusepath{stroke,fill}%
\end{pgfscope}%
\begin{pgfscope}%
\pgfpathrectangle{\pgfqpoint{0.100000in}{0.212622in}}{\pgfqpoint{3.696000in}{3.696000in}}%
\pgfusepath{clip}%
\pgfsetbuttcap%
\pgfsetroundjoin%
\definecolor{currentfill}{rgb}{0.121569,0.466667,0.705882}%
\pgfsetfillcolor{currentfill}%
\pgfsetfillopacity{0.949547}%
\pgfsetlinewidth{1.003750pt}%
\definecolor{currentstroke}{rgb}{0.121569,0.466667,0.705882}%
\pgfsetstrokecolor{currentstroke}%
\pgfsetstrokeopacity{0.949547}%
\pgfsetdash{}{0pt}%
\pgfpathmoveto{\pgfqpoint{1.341705in}{1.831544in}}%
\pgfpathcurveto{\pgfqpoint{1.349941in}{1.831544in}}{\pgfqpoint{1.357841in}{1.834816in}}{\pgfqpoint{1.363665in}{1.840640in}}%
\pgfpathcurveto{\pgfqpoint{1.369489in}{1.846464in}}{\pgfqpoint{1.372761in}{1.854364in}}{\pgfqpoint{1.372761in}{1.862601in}}%
\pgfpathcurveto{\pgfqpoint{1.372761in}{1.870837in}}{\pgfqpoint{1.369489in}{1.878737in}}{\pgfqpoint{1.363665in}{1.884561in}}%
\pgfpathcurveto{\pgfqpoint{1.357841in}{1.890385in}}{\pgfqpoint{1.349941in}{1.893657in}}{\pgfqpoint{1.341705in}{1.893657in}}%
\pgfpathcurveto{\pgfqpoint{1.333468in}{1.893657in}}{\pgfqpoint{1.325568in}{1.890385in}}{\pgfqpoint{1.319744in}{1.884561in}}%
\pgfpathcurveto{\pgfqpoint{1.313921in}{1.878737in}}{\pgfqpoint{1.310648in}{1.870837in}}{\pgfqpoint{1.310648in}{1.862601in}}%
\pgfpathcurveto{\pgfqpoint{1.310648in}{1.854364in}}{\pgfqpoint{1.313921in}{1.846464in}}{\pgfqpoint{1.319744in}{1.840640in}}%
\pgfpathcurveto{\pgfqpoint{1.325568in}{1.834816in}}{\pgfqpoint{1.333468in}{1.831544in}}{\pgfqpoint{1.341705in}{1.831544in}}%
\pgfpathclose%
\pgfusepath{stroke,fill}%
\end{pgfscope}%
\begin{pgfscope}%
\pgfpathrectangle{\pgfqpoint{0.100000in}{0.212622in}}{\pgfqpoint{3.696000in}{3.696000in}}%
\pgfusepath{clip}%
\pgfsetbuttcap%
\pgfsetroundjoin%
\definecolor{currentfill}{rgb}{0.121569,0.466667,0.705882}%
\pgfsetfillcolor{currentfill}%
\pgfsetfillopacity{0.949956}%
\pgfsetlinewidth{1.003750pt}%
\definecolor{currentstroke}{rgb}{0.121569,0.466667,0.705882}%
\pgfsetstrokecolor{currentstroke}%
\pgfsetstrokeopacity{0.949956}%
\pgfsetdash{}{0pt}%
\pgfpathmoveto{\pgfqpoint{2.087234in}{2.408355in}}%
\pgfpathcurveto{\pgfqpoint{2.095470in}{2.408355in}}{\pgfqpoint{2.103370in}{2.411627in}}{\pgfqpoint{2.109194in}{2.417451in}}%
\pgfpathcurveto{\pgfqpoint{2.115018in}{2.423275in}}{\pgfqpoint{2.118291in}{2.431175in}}{\pgfqpoint{2.118291in}{2.439412in}}%
\pgfpathcurveto{\pgfqpoint{2.118291in}{2.447648in}}{\pgfqpoint{2.115018in}{2.455548in}}{\pgfqpoint{2.109194in}{2.461372in}}%
\pgfpathcurveto{\pgfqpoint{2.103370in}{2.467196in}}{\pgfqpoint{2.095470in}{2.470468in}}{\pgfqpoint{2.087234in}{2.470468in}}%
\pgfpathcurveto{\pgfqpoint{2.078998in}{2.470468in}}{\pgfqpoint{2.071098in}{2.467196in}}{\pgfqpoint{2.065274in}{2.461372in}}%
\pgfpathcurveto{\pgfqpoint{2.059450in}{2.455548in}}{\pgfqpoint{2.056178in}{2.447648in}}{\pgfqpoint{2.056178in}{2.439412in}}%
\pgfpathcurveto{\pgfqpoint{2.056178in}{2.431175in}}{\pgfqpoint{2.059450in}{2.423275in}}{\pgfqpoint{2.065274in}{2.417451in}}%
\pgfpathcurveto{\pgfqpoint{2.071098in}{2.411627in}}{\pgfqpoint{2.078998in}{2.408355in}}{\pgfqpoint{2.087234in}{2.408355in}}%
\pgfpathclose%
\pgfusepath{stroke,fill}%
\end{pgfscope}%
\begin{pgfscope}%
\pgfpathrectangle{\pgfqpoint{0.100000in}{0.212622in}}{\pgfqpoint{3.696000in}{3.696000in}}%
\pgfusepath{clip}%
\pgfsetbuttcap%
\pgfsetroundjoin%
\definecolor{currentfill}{rgb}{0.121569,0.466667,0.705882}%
\pgfsetfillcolor{currentfill}%
\pgfsetfillopacity{0.950173}%
\pgfsetlinewidth{1.003750pt}%
\definecolor{currentstroke}{rgb}{0.121569,0.466667,0.705882}%
\pgfsetstrokecolor{currentstroke}%
\pgfsetstrokeopacity{0.950173}%
\pgfsetdash{}{0pt}%
\pgfpathmoveto{\pgfqpoint{2.086759in}{2.407826in}}%
\pgfpathcurveto{\pgfqpoint{2.094995in}{2.407826in}}{\pgfqpoint{2.102896in}{2.411098in}}{\pgfqpoint{2.108719in}{2.416922in}}%
\pgfpathcurveto{\pgfqpoint{2.114543in}{2.422746in}}{\pgfqpoint{2.117816in}{2.430646in}}{\pgfqpoint{2.117816in}{2.438883in}}%
\pgfpathcurveto{\pgfqpoint{2.117816in}{2.447119in}}{\pgfqpoint{2.114543in}{2.455019in}}{\pgfqpoint{2.108719in}{2.460843in}}%
\pgfpathcurveto{\pgfqpoint{2.102896in}{2.466667in}}{\pgfqpoint{2.094995in}{2.469939in}}{\pgfqpoint{2.086759in}{2.469939in}}%
\pgfpathcurveto{\pgfqpoint{2.078523in}{2.469939in}}{\pgfqpoint{2.070623in}{2.466667in}}{\pgfqpoint{2.064799in}{2.460843in}}%
\pgfpathcurveto{\pgfqpoint{2.058975in}{2.455019in}}{\pgfqpoint{2.055703in}{2.447119in}}{\pgfqpoint{2.055703in}{2.438883in}}%
\pgfpathcurveto{\pgfqpoint{2.055703in}{2.430646in}}{\pgfqpoint{2.058975in}{2.422746in}}{\pgfqpoint{2.064799in}{2.416922in}}%
\pgfpathcurveto{\pgfqpoint{2.070623in}{2.411098in}}{\pgfqpoint{2.078523in}{2.407826in}}{\pgfqpoint{2.086759in}{2.407826in}}%
\pgfpathclose%
\pgfusepath{stroke,fill}%
\end{pgfscope}%
\begin{pgfscope}%
\pgfpathrectangle{\pgfqpoint{0.100000in}{0.212622in}}{\pgfqpoint{3.696000in}{3.696000in}}%
\pgfusepath{clip}%
\pgfsetbuttcap%
\pgfsetroundjoin%
\definecolor{currentfill}{rgb}{0.121569,0.466667,0.705882}%
\pgfsetfillcolor{currentfill}%
\pgfsetfillopacity{0.950174}%
\pgfsetlinewidth{1.003750pt}%
\definecolor{currentstroke}{rgb}{0.121569,0.466667,0.705882}%
\pgfsetstrokecolor{currentstroke}%
\pgfsetstrokeopacity{0.950174}%
\pgfsetdash{}{0pt}%
\pgfpathmoveto{\pgfqpoint{1.931133in}{2.445626in}}%
\pgfpathcurveto{\pgfqpoint{1.939369in}{2.445626in}}{\pgfqpoint{1.947269in}{2.448898in}}{\pgfqpoint{1.953093in}{2.454722in}}%
\pgfpathcurveto{\pgfqpoint{1.958917in}{2.460546in}}{\pgfqpoint{1.962189in}{2.468446in}}{\pgfqpoint{1.962189in}{2.476682in}}%
\pgfpathcurveto{\pgfqpoint{1.962189in}{2.484919in}}{\pgfqpoint{1.958917in}{2.492819in}}{\pgfqpoint{1.953093in}{2.498643in}}%
\pgfpathcurveto{\pgfqpoint{1.947269in}{2.504467in}}{\pgfqpoint{1.939369in}{2.507739in}}{\pgfqpoint{1.931133in}{2.507739in}}%
\pgfpathcurveto{\pgfqpoint{1.922896in}{2.507739in}}{\pgfqpoint{1.914996in}{2.504467in}}{\pgfqpoint{1.909172in}{2.498643in}}%
\pgfpathcurveto{\pgfqpoint{1.903348in}{2.492819in}}{\pgfqpoint{1.900076in}{2.484919in}}{\pgfqpoint{1.900076in}{2.476682in}}%
\pgfpathcurveto{\pgfqpoint{1.900076in}{2.468446in}}{\pgfqpoint{1.903348in}{2.460546in}}{\pgfqpoint{1.909172in}{2.454722in}}%
\pgfpathcurveto{\pgfqpoint{1.914996in}{2.448898in}}{\pgfqpoint{1.922896in}{2.445626in}}{\pgfqpoint{1.931133in}{2.445626in}}%
\pgfpathclose%
\pgfusepath{stroke,fill}%
\end{pgfscope}%
\begin{pgfscope}%
\pgfpathrectangle{\pgfqpoint{0.100000in}{0.212622in}}{\pgfqpoint{3.696000in}{3.696000in}}%
\pgfusepath{clip}%
\pgfsetbuttcap%
\pgfsetroundjoin%
\definecolor{currentfill}{rgb}{0.121569,0.466667,0.705882}%
\pgfsetfillcolor{currentfill}%
\pgfsetfillopacity{0.950208}%
\pgfsetlinewidth{1.003750pt}%
\definecolor{currentstroke}{rgb}{0.121569,0.466667,0.705882}%
\pgfsetstrokecolor{currentstroke}%
\pgfsetstrokeopacity{0.950208}%
\pgfsetdash{}{0pt}%
\pgfpathmoveto{\pgfqpoint{2.086682in}{2.407747in}}%
\pgfpathcurveto{\pgfqpoint{2.094918in}{2.407747in}}{\pgfqpoint{2.102818in}{2.411019in}}{\pgfqpoint{2.108642in}{2.416843in}}%
\pgfpathcurveto{\pgfqpoint{2.114466in}{2.422667in}}{\pgfqpoint{2.117738in}{2.430567in}}{\pgfqpoint{2.117738in}{2.438803in}}%
\pgfpathcurveto{\pgfqpoint{2.117738in}{2.447040in}}{\pgfqpoint{2.114466in}{2.454940in}}{\pgfqpoint{2.108642in}{2.460764in}}%
\pgfpathcurveto{\pgfqpoint{2.102818in}{2.466588in}}{\pgfqpoint{2.094918in}{2.469860in}}{\pgfqpoint{2.086682in}{2.469860in}}%
\pgfpathcurveto{\pgfqpoint{2.078446in}{2.469860in}}{\pgfqpoint{2.070546in}{2.466588in}}{\pgfqpoint{2.064722in}{2.460764in}}%
\pgfpathcurveto{\pgfqpoint{2.058898in}{2.454940in}}{\pgfqpoint{2.055625in}{2.447040in}}{\pgfqpoint{2.055625in}{2.438803in}}%
\pgfpathcurveto{\pgfqpoint{2.055625in}{2.430567in}}{\pgfqpoint{2.058898in}{2.422667in}}{\pgfqpoint{2.064722in}{2.416843in}}%
\pgfpathcurveto{\pgfqpoint{2.070546in}{2.411019in}}{\pgfqpoint{2.078446in}{2.407747in}}{\pgfqpoint{2.086682in}{2.407747in}}%
\pgfpathclose%
\pgfusepath{stroke,fill}%
\end{pgfscope}%
\begin{pgfscope}%
\pgfpathrectangle{\pgfqpoint{0.100000in}{0.212622in}}{\pgfqpoint{3.696000in}{3.696000in}}%
\pgfusepath{clip}%
\pgfsetbuttcap%
\pgfsetroundjoin%
\definecolor{currentfill}{rgb}{0.121569,0.466667,0.705882}%
\pgfsetfillcolor{currentfill}%
\pgfsetfillopacity{0.950274}%
\pgfsetlinewidth{1.003750pt}%
\definecolor{currentstroke}{rgb}{0.121569,0.466667,0.705882}%
\pgfsetstrokecolor{currentstroke}%
\pgfsetstrokeopacity{0.950274}%
\pgfsetdash{}{0pt}%
\pgfpathmoveto{\pgfqpoint{2.086543in}{2.407613in}}%
\pgfpathcurveto{\pgfqpoint{2.094780in}{2.407613in}}{\pgfqpoint{2.102680in}{2.410886in}}{\pgfqpoint{2.108504in}{2.416709in}}%
\pgfpathcurveto{\pgfqpoint{2.114328in}{2.422533in}}{\pgfqpoint{2.117600in}{2.430433in}}{\pgfqpoint{2.117600in}{2.438670in}}%
\pgfpathcurveto{\pgfqpoint{2.117600in}{2.446906in}}{\pgfqpoint{2.114328in}{2.454806in}}{\pgfqpoint{2.108504in}{2.460630in}}%
\pgfpathcurveto{\pgfqpoint{2.102680in}{2.466454in}}{\pgfqpoint{2.094780in}{2.469726in}}{\pgfqpoint{2.086543in}{2.469726in}}%
\pgfpathcurveto{\pgfqpoint{2.078307in}{2.469726in}}{\pgfqpoint{2.070407in}{2.466454in}}{\pgfqpoint{2.064583in}{2.460630in}}%
\pgfpathcurveto{\pgfqpoint{2.058759in}{2.454806in}}{\pgfqpoint{2.055487in}{2.446906in}}{\pgfqpoint{2.055487in}{2.438670in}}%
\pgfpathcurveto{\pgfqpoint{2.055487in}{2.430433in}}{\pgfqpoint{2.058759in}{2.422533in}}{\pgfqpoint{2.064583in}{2.416709in}}%
\pgfpathcurveto{\pgfqpoint{2.070407in}{2.410886in}}{\pgfqpoint{2.078307in}{2.407613in}}{\pgfqpoint{2.086543in}{2.407613in}}%
\pgfpathclose%
\pgfusepath{stroke,fill}%
\end{pgfscope}%
\begin{pgfscope}%
\pgfpathrectangle{\pgfqpoint{0.100000in}{0.212622in}}{\pgfqpoint{3.696000in}{3.696000in}}%
\pgfusepath{clip}%
\pgfsetbuttcap%
\pgfsetroundjoin%
\definecolor{currentfill}{rgb}{0.121569,0.466667,0.705882}%
\pgfsetfillcolor{currentfill}%
\pgfsetfillopacity{0.950392}%
\pgfsetlinewidth{1.003750pt}%
\definecolor{currentstroke}{rgb}{0.121569,0.466667,0.705882}%
\pgfsetstrokecolor{currentstroke}%
\pgfsetstrokeopacity{0.950392}%
\pgfsetdash{}{0pt}%
\pgfpathmoveto{\pgfqpoint{2.086285in}{2.407365in}}%
\pgfpathcurveto{\pgfqpoint{2.094522in}{2.407365in}}{\pgfqpoint{2.102422in}{2.410638in}}{\pgfqpoint{2.108245in}{2.416462in}}%
\pgfpathcurveto{\pgfqpoint{2.114069in}{2.422285in}}{\pgfqpoint{2.117342in}{2.430185in}}{\pgfqpoint{2.117342in}{2.438422in}}%
\pgfpathcurveto{\pgfqpoint{2.117342in}{2.446658in}}{\pgfqpoint{2.114069in}{2.454558in}}{\pgfqpoint{2.108245in}{2.460382in}}%
\pgfpathcurveto{\pgfqpoint{2.102422in}{2.466206in}}{\pgfqpoint{2.094522in}{2.469478in}}{\pgfqpoint{2.086285in}{2.469478in}}%
\pgfpathcurveto{\pgfqpoint{2.078049in}{2.469478in}}{\pgfqpoint{2.070149in}{2.466206in}}{\pgfqpoint{2.064325in}{2.460382in}}%
\pgfpathcurveto{\pgfqpoint{2.058501in}{2.454558in}}{\pgfqpoint{2.055229in}{2.446658in}}{\pgfqpoint{2.055229in}{2.438422in}}%
\pgfpathcurveto{\pgfqpoint{2.055229in}{2.430185in}}{\pgfqpoint{2.058501in}{2.422285in}}{\pgfqpoint{2.064325in}{2.416462in}}%
\pgfpathcurveto{\pgfqpoint{2.070149in}{2.410638in}}{\pgfqpoint{2.078049in}{2.407365in}}{\pgfqpoint{2.086285in}{2.407365in}}%
\pgfpathclose%
\pgfusepath{stroke,fill}%
\end{pgfscope}%
\begin{pgfscope}%
\pgfpathrectangle{\pgfqpoint{0.100000in}{0.212622in}}{\pgfqpoint{3.696000in}{3.696000in}}%
\pgfusepath{clip}%
\pgfsetbuttcap%
\pgfsetroundjoin%
\definecolor{currentfill}{rgb}{0.121569,0.466667,0.705882}%
\pgfsetfillcolor{currentfill}%
\pgfsetfillopacity{0.950486}%
\pgfsetlinewidth{1.003750pt}%
\definecolor{currentstroke}{rgb}{0.121569,0.466667,0.705882}%
\pgfsetstrokecolor{currentstroke}%
\pgfsetstrokeopacity{0.950486}%
\pgfsetdash{}{0pt}%
\pgfpathmoveto{\pgfqpoint{1.353271in}{1.824186in}}%
\pgfpathcurveto{\pgfqpoint{1.361507in}{1.824186in}}{\pgfqpoint{1.369407in}{1.827458in}}{\pgfqpoint{1.375231in}{1.833282in}}%
\pgfpathcurveto{\pgfqpoint{1.381055in}{1.839106in}}{\pgfqpoint{1.384328in}{1.847006in}}{\pgfqpoint{1.384328in}{1.855242in}}%
\pgfpathcurveto{\pgfqpoint{1.384328in}{1.863478in}}{\pgfqpoint{1.381055in}{1.871378in}}{\pgfqpoint{1.375231in}{1.877202in}}%
\pgfpathcurveto{\pgfqpoint{1.369407in}{1.883026in}}{\pgfqpoint{1.361507in}{1.886299in}}{\pgfqpoint{1.353271in}{1.886299in}}%
\pgfpathcurveto{\pgfqpoint{1.345035in}{1.886299in}}{\pgfqpoint{1.337135in}{1.883026in}}{\pgfqpoint{1.331311in}{1.877202in}}%
\pgfpathcurveto{\pgfqpoint{1.325487in}{1.871378in}}{\pgfqpoint{1.322215in}{1.863478in}}{\pgfqpoint{1.322215in}{1.855242in}}%
\pgfpathcurveto{\pgfqpoint{1.322215in}{1.847006in}}{\pgfqpoint{1.325487in}{1.839106in}}{\pgfqpoint{1.331311in}{1.833282in}}%
\pgfpathcurveto{\pgfqpoint{1.337135in}{1.827458in}}{\pgfqpoint{1.345035in}{1.824186in}}{\pgfqpoint{1.353271in}{1.824186in}}%
\pgfpathclose%
\pgfusepath{stroke,fill}%
\end{pgfscope}%
\begin{pgfscope}%
\pgfpathrectangle{\pgfqpoint{0.100000in}{0.212622in}}{\pgfqpoint{3.696000in}{3.696000in}}%
\pgfusepath{clip}%
\pgfsetbuttcap%
\pgfsetroundjoin%
\definecolor{currentfill}{rgb}{0.121569,0.466667,0.705882}%
\pgfsetfillcolor{currentfill}%
\pgfsetfillopacity{0.950562}%
\pgfsetlinewidth{1.003750pt}%
\definecolor{currentstroke}{rgb}{0.121569,0.466667,0.705882}%
\pgfsetstrokecolor{currentstroke}%
\pgfsetstrokeopacity{0.950562}%
\pgfsetdash{}{0pt}%
\pgfpathmoveto{\pgfqpoint{2.584197in}{1.197302in}}%
\pgfpathcurveto{\pgfqpoint{2.592434in}{1.197302in}}{\pgfqpoint{2.600334in}{1.200574in}}{\pgfqpoint{2.606158in}{1.206398in}}%
\pgfpathcurveto{\pgfqpoint{2.611982in}{1.212222in}}{\pgfqpoint{2.615254in}{1.220122in}}{\pgfqpoint{2.615254in}{1.228358in}}%
\pgfpathcurveto{\pgfqpoint{2.615254in}{1.236595in}}{\pgfqpoint{2.611982in}{1.244495in}}{\pgfqpoint{2.606158in}{1.250319in}}%
\pgfpathcurveto{\pgfqpoint{2.600334in}{1.256142in}}{\pgfqpoint{2.592434in}{1.259415in}}{\pgfqpoint{2.584197in}{1.259415in}}%
\pgfpathcurveto{\pgfqpoint{2.575961in}{1.259415in}}{\pgfqpoint{2.568061in}{1.256142in}}{\pgfqpoint{2.562237in}{1.250319in}}%
\pgfpathcurveto{\pgfqpoint{2.556413in}{1.244495in}}{\pgfqpoint{2.553141in}{1.236595in}}{\pgfqpoint{2.553141in}{1.228358in}}%
\pgfpathcurveto{\pgfqpoint{2.553141in}{1.220122in}}{\pgfqpoint{2.556413in}{1.212222in}}{\pgfqpoint{2.562237in}{1.206398in}}%
\pgfpathcurveto{\pgfqpoint{2.568061in}{1.200574in}}{\pgfqpoint{2.575961in}{1.197302in}}{\pgfqpoint{2.584197in}{1.197302in}}%
\pgfpathclose%
\pgfusepath{stroke,fill}%
\end{pgfscope}%
\begin{pgfscope}%
\pgfpathrectangle{\pgfqpoint{0.100000in}{0.212622in}}{\pgfqpoint{3.696000in}{3.696000in}}%
\pgfusepath{clip}%
\pgfsetbuttcap%
\pgfsetroundjoin%
\definecolor{currentfill}{rgb}{0.121569,0.466667,0.705882}%
\pgfsetfillcolor{currentfill}%
\pgfsetfillopacity{0.950607}%
\pgfsetlinewidth{1.003750pt}%
\definecolor{currentstroke}{rgb}{0.121569,0.466667,0.705882}%
\pgfsetstrokecolor{currentstroke}%
\pgfsetstrokeopacity{0.950607}%
\pgfsetdash{}{0pt}%
\pgfpathmoveto{\pgfqpoint{2.085823in}{2.406907in}}%
\pgfpathcurveto{\pgfqpoint{2.094059in}{2.406907in}}{\pgfqpoint{2.101959in}{2.410179in}}{\pgfqpoint{2.107783in}{2.416003in}}%
\pgfpathcurveto{\pgfqpoint{2.113607in}{2.421827in}}{\pgfqpoint{2.116879in}{2.429727in}}{\pgfqpoint{2.116879in}{2.437963in}}%
\pgfpathcurveto{\pgfqpoint{2.116879in}{2.446199in}}{\pgfqpoint{2.113607in}{2.454099in}}{\pgfqpoint{2.107783in}{2.459923in}}%
\pgfpathcurveto{\pgfqpoint{2.101959in}{2.465747in}}{\pgfqpoint{2.094059in}{2.469020in}}{\pgfqpoint{2.085823in}{2.469020in}}%
\pgfpathcurveto{\pgfqpoint{2.077586in}{2.469020in}}{\pgfqpoint{2.069686in}{2.465747in}}{\pgfqpoint{2.063862in}{2.459923in}}%
\pgfpathcurveto{\pgfqpoint{2.058039in}{2.454099in}}{\pgfqpoint{2.054766in}{2.446199in}}{\pgfqpoint{2.054766in}{2.437963in}}%
\pgfpathcurveto{\pgfqpoint{2.054766in}{2.429727in}}{\pgfqpoint{2.058039in}{2.421827in}}{\pgfqpoint{2.063862in}{2.416003in}}%
\pgfpathcurveto{\pgfqpoint{2.069686in}{2.410179in}}{\pgfqpoint{2.077586in}{2.406907in}}{\pgfqpoint{2.085823in}{2.406907in}}%
\pgfpathclose%
\pgfusepath{stroke,fill}%
\end{pgfscope}%
\begin{pgfscope}%
\pgfpathrectangle{\pgfqpoint{0.100000in}{0.212622in}}{\pgfqpoint{3.696000in}{3.696000in}}%
\pgfusepath{clip}%
\pgfsetbuttcap%
\pgfsetroundjoin%
\definecolor{currentfill}{rgb}{0.121569,0.466667,0.705882}%
\pgfsetfillcolor{currentfill}%
\pgfsetfillopacity{0.950663}%
\pgfsetlinewidth{1.003750pt}%
\definecolor{currentstroke}{rgb}{0.121569,0.466667,0.705882}%
\pgfsetstrokecolor{currentstroke}%
\pgfsetstrokeopacity{0.950663}%
\pgfsetdash{}{0pt}%
\pgfpathmoveto{\pgfqpoint{1.936064in}{2.443435in}}%
\pgfpathcurveto{\pgfqpoint{1.944301in}{2.443435in}}{\pgfqpoint{1.952201in}{2.446708in}}{\pgfqpoint{1.958025in}{2.452532in}}%
\pgfpathcurveto{\pgfqpoint{1.963849in}{2.458355in}}{\pgfqpoint{1.967121in}{2.466256in}}{\pgfqpoint{1.967121in}{2.474492in}}%
\pgfpathcurveto{\pgfqpoint{1.967121in}{2.482728in}}{\pgfqpoint{1.963849in}{2.490628in}}{\pgfqpoint{1.958025in}{2.496452in}}%
\pgfpathcurveto{\pgfqpoint{1.952201in}{2.502276in}}{\pgfqpoint{1.944301in}{2.505548in}}{\pgfqpoint{1.936064in}{2.505548in}}%
\pgfpathcurveto{\pgfqpoint{1.927828in}{2.505548in}}{\pgfqpoint{1.919928in}{2.502276in}}{\pgfqpoint{1.914104in}{2.496452in}}%
\pgfpathcurveto{\pgfqpoint{1.908280in}{2.490628in}}{\pgfqpoint{1.905008in}{2.482728in}}{\pgfqpoint{1.905008in}{2.474492in}}%
\pgfpathcurveto{\pgfqpoint{1.905008in}{2.466256in}}{\pgfqpoint{1.908280in}{2.458355in}}{\pgfqpoint{1.914104in}{2.452532in}}%
\pgfpathcurveto{\pgfqpoint{1.919928in}{2.446708in}}{\pgfqpoint{1.927828in}{2.443435in}}{\pgfqpoint{1.936064in}{2.443435in}}%
\pgfpathclose%
\pgfusepath{stroke,fill}%
\end{pgfscope}%
\begin{pgfscope}%
\pgfpathrectangle{\pgfqpoint{0.100000in}{0.212622in}}{\pgfqpoint{3.696000in}{3.696000in}}%
\pgfusepath{clip}%
\pgfsetbuttcap%
\pgfsetroundjoin%
\definecolor{currentfill}{rgb}{0.121569,0.466667,0.705882}%
\pgfsetfillcolor{currentfill}%
\pgfsetfillopacity{0.950958}%
\pgfsetlinewidth{1.003750pt}%
\definecolor{currentstroke}{rgb}{0.121569,0.466667,0.705882}%
\pgfsetstrokecolor{currentstroke}%
\pgfsetstrokeopacity{0.950958}%
\pgfsetdash{}{0pt}%
\pgfpathmoveto{\pgfqpoint{1.941643in}{2.440825in}}%
\pgfpathcurveto{\pgfqpoint{1.949879in}{2.440825in}}{\pgfqpoint{1.957779in}{2.444097in}}{\pgfqpoint{1.963603in}{2.449921in}}%
\pgfpathcurveto{\pgfqpoint{1.969427in}{2.455745in}}{\pgfqpoint{1.972700in}{2.463645in}}{\pgfqpoint{1.972700in}{2.471881in}}%
\pgfpathcurveto{\pgfqpoint{1.972700in}{2.480117in}}{\pgfqpoint{1.969427in}{2.488018in}}{\pgfqpoint{1.963603in}{2.493841in}}%
\pgfpathcurveto{\pgfqpoint{1.957779in}{2.499665in}}{\pgfqpoint{1.949879in}{2.502938in}}{\pgfqpoint{1.941643in}{2.502938in}}%
\pgfpathcurveto{\pgfqpoint{1.933407in}{2.502938in}}{\pgfqpoint{1.925507in}{2.499665in}}{\pgfqpoint{1.919683in}{2.493841in}}%
\pgfpathcurveto{\pgfqpoint{1.913859in}{2.488018in}}{\pgfqpoint{1.910587in}{2.480117in}}{\pgfqpoint{1.910587in}{2.471881in}}%
\pgfpathcurveto{\pgfqpoint{1.910587in}{2.463645in}}{\pgfqpoint{1.913859in}{2.455745in}}{\pgfqpoint{1.919683in}{2.449921in}}%
\pgfpathcurveto{\pgfqpoint{1.925507in}{2.444097in}}{\pgfqpoint{1.933407in}{2.440825in}}{\pgfqpoint{1.941643in}{2.440825in}}%
\pgfpathclose%
\pgfusepath{stroke,fill}%
\end{pgfscope}%
\begin{pgfscope}%
\pgfpathrectangle{\pgfqpoint{0.100000in}{0.212622in}}{\pgfqpoint{3.696000in}{3.696000in}}%
\pgfusepath{clip}%
\pgfsetbuttcap%
\pgfsetroundjoin%
\definecolor{currentfill}{rgb}{0.121569,0.466667,0.705882}%
\pgfsetfillcolor{currentfill}%
\pgfsetfillopacity{0.951001}%
\pgfsetlinewidth{1.003750pt}%
\definecolor{currentstroke}{rgb}{0.121569,0.466667,0.705882}%
\pgfsetstrokecolor{currentstroke}%
\pgfsetstrokeopacity{0.951001}%
\pgfsetdash{}{0pt}%
\pgfpathmoveto{\pgfqpoint{2.084997in}{2.406074in}}%
\pgfpathcurveto{\pgfqpoint{2.093234in}{2.406074in}}{\pgfqpoint{2.101134in}{2.409347in}}{\pgfqpoint{2.106958in}{2.415171in}}%
\pgfpathcurveto{\pgfqpoint{2.112782in}{2.420995in}}{\pgfqpoint{2.116054in}{2.428895in}}{\pgfqpoint{2.116054in}{2.437131in}}%
\pgfpathcurveto{\pgfqpoint{2.116054in}{2.445367in}}{\pgfqpoint{2.112782in}{2.453267in}}{\pgfqpoint{2.106958in}{2.459091in}}%
\pgfpathcurveto{\pgfqpoint{2.101134in}{2.464915in}}{\pgfqpoint{2.093234in}{2.468187in}}{\pgfqpoint{2.084997in}{2.468187in}}%
\pgfpathcurveto{\pgfqpoint{2.076761in}{2.468187in}}{\pgfqpoint{2.068861in}{2.464915in}}{\pgfqpoint{2.063037in}{2.459091in}}%
\pgfpathcurveto{\pgfqpoint{2.057213in}{2.453267in}}{\pgfqpoint{2.053941in}{2.445367in}}{\pgfqpoint{2.053941in}{2.437131in}}%
\pgfpathcurveto{\pgfqpoint{2.053941in}{2.428895in}}{\pgfqpoint{2.057213in}{2.420995in}}{\pgfqpoint{2.063037in}{2.415171in}}%
\pgfpathcurveto{\pgfqpoint{2.068861in}{2.409347in}}{\pgfqpoint{2.076761in}{2.406074in}}{\pgfqpoint{2.084997in}{2.406074in}}%
\pgfpathclose%
\pgfusepath{stroke,fill}%
\end{pgfscope}%
\begin{pgfscope}%
\pgfpathrectangle{\pgfqpoint{0.100000in}{0.212622in}}{\pgfqpoint{3.696000in}{3.696000in}}%
\pgfusepath{clip}%
\pgfsetbuttcap%
\pgfsetroundjoin%
\definecolor{currentfill}{rgb}{0.121569,0.466667,0.705882}%
\pgfsetfillcolor{currentfill}%
\pgfsetfillopacity{0.951291}%
\pgfsetlinewidth{1.003750pt}%
\definecolor{currentstroke}{rgb}{0.121569,0.466667,0.705882}%
\pgfsetstrokecolor{currentstroke}%
\pgfsetstrokeopacity{0.951291}%
\pgfsetdash{}{0pt}%
\pgfpathmoveto{\pgfqpoint{2.084406in}{2.405545in}}%
\pgfpathcurveto{\pgfqpoint{2.092643in}{2.405545in}}{\pgfqpoint{2.100543in}{2.408817in}}{\pgfqpoint{2.106367in}{2.414641in}}%
\pgfpathcurveto{\pgfqpoint{2.112191in}{2.420465in}}{\pgfqpoint{2.115463in}{2.428365in}}{\pgfqpoint{2.115463in}{2.436601in}}%
\pgfpathcurveto{\pgfqpoint{2.115463in}{2.444838in}}{\pgfqpoint{2.112191in}{2.452738in}}{\pgfqpoint{2.106367in}{2.458562in}}%
\pgfpathcurveto{\pgfqpoint{2.100543in}{2.464386in}}{\pgfqpoint{2.092643in}{2.467658in}}{\pgfqpoint{2.084406in}{2.467658in}}%
\pgfpathcurveto{\pgfqpoint{2.076170in}{2.467658in}}{\pgfqpoint{2.068270in}{2.464386in}}{\pgfqpoint{2.062446in}{2.458562in}}%
\pgfpathcurveto{\pgfqpoint{2.056622in}{2.452738in}}{\pgfqpoint{2.053350in}{2.444838in}}{\pgfqpoint{2.053350in}{2.436601in}}%
\pgfpathcurveto{\pgfqpoint{2.053350in}{2.428365in}}{\pgfqpoint{2.056622in}{2.420465in}}{\pgfqpoint{2.062446in}{2.414641in}}%
\pgfpathcurveto{\pgfqpoint{2.068270in}{2.408817in}}{\pgfqpoint{2.076170in}{2.405545in}}{\pgfqpoint{2.084406in}{2.405545in}}%
\pgfpathclose%
\pgfusepath{stroke,fill}%
\end{pgfscope}%
\begin{pgfscope}%
\pgfpathrectangle{\pgfqpoint{0.100000in}{0.212622in}}{\pgfqpoint{3.696000in}{3.696000in}}%
\pgfusepath{clip}%
\pgfsetbuttcap%
\pgfsetroundjoin%
\definecolor{currentfill}{rgb}{0.121569,0.466667,0.705882}%
\pgfsetfillcolor{currentfill}%
\pgfsetfillopacity{0.951348}%
\pgfsetlinewidth{1.003750pt}%
\definecolor{currentstroke}{rgb}{0.121569,0.466667,0.705882}%
\pgfsetstrokecolor{currentstroke}%
\pgfsetstrokeopacity{0.951348}%
\pgfsetdash{}{0pt}%
\pgfpathmoveto{\pgfqpoint{1.947898in}{2.438007in}}%
\pgfpathcurveto{\pgfqpoint{1.956134in}{2.438007in}}{\pgfqpoint{1.964034in}{2.441279in}}{\pgfqpoint{1.969858in}{2.447103in}}%
\pgfpathcurveto{\pgfqpoint{1.975682in}{2.452927in}}{\pgfqpoint{1.978954in}{2.460827in}}{\pgfqpoint{1.978954in}{2.469063in}}%
\pgfpathcurveto{\pgfqpoint{1.978954in}{2.477300in}}{\pgfqpoint{1.975682in}{2.485200in}}{\pgfqpoint{1.969858in}{2.491024in}}%
\pgfpathcurveto{\pgfqpoint{1.964034in}{2.496847in}}{\pgfqpoint{1.956134in}{2.500120in}}{\pgfqpoint{1.947898in}{2.500120in}}%
\pgfpathcurveto{\pgfqpoint{1.939661in}{2.500120in}}{\pgfqpoint{1.931761in}{2.496847in}}{\pgfqpoint{1.925937in}{2.491024in}}%
\pgfpathcurveto{\pgfqpoint{1.920114in}{2.485200in}}{\pgfqpoint{1.916841in}{2.477300in}}{\pgfqpoint{1.916841in}{2.469063in}}%
\pgfpathcurveto{\pgfqpoint{1.916841in}{2.460827in}}{\pgfqpoint{1.920114in}{2.452927in}}{\pgfqpoint{1.925937in}{2.447103in}}%
\pgfpathcurveto{\pgfqpoint{1.931761in}{2.441279in}}{\pgfqpoint{1.939661in}{2.438007in}}{\pgfqpoint{1.947898in}{2.438007in}}%
\pgfpathclose%
\pgfusepath{stroke,fill}%
\end{pgfscope}%
\begin{pgfscope}%
\pgfpathrectangle{\pgfqpoint{0.100000in}{0.212622in}}{\pgfqpoint{3.696000in}{3.696000in}}%
\pgfusepath{clip}%
\pgfsetbuttcap%
\pgfsetroundjoin%
\definecolor{currentfill}{rgb}{0.121569,0.466667,0.705882}%
\pgfsetfillcolor{currentfill}%
\pgfsetfillopacity{0.951396}%
\pgfsetlinewidth{1.003750pt}%
\definecolor{currentstroke}{rgb}{0.121569,0.466667,0.705882}%
\pgfsetstrokecolor{currentstroke}%
\pgfsetstrokeopacity{0.951396}%
\pgfsetdash{}{0pt}%
\pgfpathmoveto{\pgfqpoint{2.084195in}{2.405349in}}%
\pgfpathcurveto{\pgfqpoint{2.092431in}{2.405349in}}{\pgfqpoint{2.100331in}{2.408621in}}{\pgfqpoint{2.106155in}{2.414445in}}%
\pgfpathcurveto{\pgfqpoint{2.111979in}{2.420269in}}{\pgfqpoint{2.115251in}{2.428169in}}{\pgfqpoint{2.115251in}{2.436405in}}%
\pgfpathcurveto{\pgfqpoint{2.115251in}{2.444641in}}{\pgfqpoint{2.111979in}{2.452542in}}{\pgfqpoint{2.106155in}{2.458365in}}%
\pgfpathcurveto{\pgfqpoint{2.100331in}{2.464189in}}{\pgfqpoint{2.092431in}{2.467462in}}{\pgfqpoint{2.084195in}{2.467462in}}%
\pgfpathcurveto{\pgfqpoint{2.075959in}{2.467462in}}{\pgfqpoint{2.068058in}{2.464189in}}{\pgfqpoint{2.062235in}{2.458365in}}%
\pgfpathcurveto{\pgfqpoint{2.056411in}{2.452542in}}{\pgfqpoint{2.053138in}{2.444641in}}{\pgfqpoint{2.053138in}{2.436405in}}%
\pgfpathcurveto{\pgfqpoint{2.053138in}{2.428169in}}{\pgfqpoint{2.056411in}{2.420269in}}{\pgfqpoint{2.062235in}{2.414445in}}%
\pgfpathcurveto{\pgfqpoint{2.068058in}{2.408621in}}{\pgfqpoint{2.075959in}{2.405349in}}{\pgfqpoint{2.084195in}{2.405349in}}%
\pgfpathclose%
\pgfusepath{stroke,fill}%
\end{pgfscope}%
\begin{pgfscope}%
\pgfpathrectangle{\pgfqpoint{0.100000in}{0.212622in}}{\pgfqpoint{3.696000in}{3.696000in}}%
\pgfusepath{clip}%
\pgfsetbuttcap%
\pgfsetroundjoin%
\definecolor{currentfill}{rgb}{0.121569,0.466667,0.705882}%
\pgfsetfillcolor{currentfill}%
\pgfsetfillopacity{0.951402}%
\pgfsetlinewidth{1.003750pt}%
\definecolor{currentstroke}{rgb}{0.121569,0.466667,0.705882}%
\pgfsetstrokecolor{currentstroke}%
\pgfsetstrokeopacity{0.951402}%
\pgfsetdash{}{0pt}%
\pgfpathmoveto{\pgfqpoint{1.365703in}{1.815566in}}%
\pgfpathcurveto{\pgfqpoint{1.373940in}{1.815566in}}{\pgfqpoint{1.381840in}{1.818839in}}{\pgfqpoint{1.387664in}{1.824663in}}%
\pgfpathcurveto{\pgfqpoint{1.393488in}{1.830487in}}{\pgfqpoint{1.396760in}{1.838387in}}{\pgfqpoint{1.396760in}{1.846623in}}%
\pgfpathcurveto{\pgfqpoint{1.396760in}{1.854859in}}{\pgfqpoint{1.393488in}{1.862759in}}{\pgfqpoint{1.387664in}{1.868583in}}%
\pgfpathcurveto{\pgfqpoint{1.381840in}{1.874407in}}{\pgfqpoint{1.373940in}{1.877679in}}{\pgfqpoint{1.365703in}{1.877679in}}%
\pgfpathcurveto{\pgfqpoint{1.357467in}{1.877679in}}{\pgfqpoint{1.349567in}{1.874407in}}{\pgfqpoint{1.343743in}{1.868583in}}%
\pgfpathcurveto{\pgfqpoint{1.337919in}{1.862759in}}{\pgfqpoint{1.334647in}{1.854859in}}{\pgfqpoint{1.334647in}{1.846623in}}%
\pgfpathcurveto{\pgfqpoint{1.334647in}{1.838387in}}{\pgfqpoint{1.337919in}{1.830487in}}{\pgfqpoint{1.343743in}{1.824663in}}%
\pgfpathcurveto{\pgfqpoint{1.349567in}{1.818839in}}{\pgfqpoint{1.357467in}{1.815566in}}{\pgfqpoint{1.365703in}{1.815566in}}%
\pgfpathclose%
\pgfusepath{stroke,fill}%
\end{pgfscope}%
\begin{pgfscope}%
\pgfpathrectangle{\pgfqpoint{0.100000in}{0.212622in}}{\pgfqpoint{3.696000in}{3.696000in}}%
\pgfusepath{clip}%
\pgfsetbuttcap%
\pgfsetroundjoin%
\definecolor{currentfill}{rgb}{0.121569,0.466667,0.705882}%
\pgfsetfillcolor{currentfill}%
\pgfsetfillopacity{0.951587}%
\pgfsetlinewidth{1.003750pt}%
\definecolor{currentstroke}{rgb}{0.121569,0.466667,0.705882}%
\pgfsetstrokecolor{currentstroke}%
\pgfsetstrokeopacity{0.951587}%
\pgfsetdash{}{0pt}%
\pgfpathmoveto{\pgfqpoint{2.083803in}{2.404998in}}%
\pgfpathcurveto{\pgfqpoint{2.092039in}{2.404998in}}{\pgfqpoint{2.099939in}{2.408270in}}{\pgfqpoint{2.105763in}{2.414094in}}%
\pgfpathcurveto{\pgfqpoint{2.111587in}{2.419918in}}{\pgfqpoint{2.114859in}{2.427818in}}{\pgfqpoint{2.114859in}{2.436055in}}%
\pgfpathcurveto{\pgfqpoint{2.114859in}{2.444291in}}{\pgfqpoint{2.111587in}{2.452191in}}{\pgfqpoint{2.105763in}{2.458015in}}%
\pgfpathcurveto{\pgfqpoint{2.099939in}{2.463839in}}{\pgfqpoint{2.092039in}{2.467111in}}{\pgfqpoint{2.083803in}{2.467111in}}%
\pgfpathcurveto{\pgfqpoint{2.075567in}{2.467111in}}{\pgfqpoint{2.067666in}{2.463839in}}{\pgfqpoint{2.061843in}{2.458015in}}%
\pgfpathcurveto{\pgfqpoint{2.056019in}{2.452191in}}{\pgfqpoint{2.052746in}{2.444291in}}{\pgfqpoint{2.052746in}{2.436055in}}%
\pgfpathcurveto{\pgfqpoint{2.052746in}{2.427818in}}{\pgfqpoint{2.056019in}{2.419918in}}{\pgfqpoint{2.061843in}{2.414094in}}%
\pgfpathcurveto{\pgfqpoint{2.067666in}{2.408270in}}{\pgfqpoint{2.075567in}{2.404998in}}{\pgfqpoint{2.083803in}{2.404998in}}%
\pgfpathclose%
\pgfusepath{stroke,fill}%
\end{pgfscope}%
\begin{pgfscope}%
\pgfpathrectangle{\pgfqpoint{0.100000in}{0.212622in}}{\pgfqpoint{3.696000in}{3.696000in}}%
\pgfusepath{clip}%
\pgfsetbuttcap%
\pgfsetroundjoin%
\definecolor{currentfill}{rgb}{0.121569,0.466667,0.705882}%
\pgfsetfillcolor{currentfill}%
\pgfsetfillopacity{0.951916}%
\pgfsetlinewidth{1.003750pt}%
\definecolor{currentstroke}{rgb}{0.121569,0.466667,0.705882}%
\pgfsetstrokecolor{currentstroke}%
\pgfsetstrokeopacity{0.951916}%
\pgfsetdash{}{0pt}%
\pgfpathmoveto{\pgfqpoint{1.372454in}{1.810576in}}%
\pgfpathcurveto{\pgfqpoint{1.380691in}{1.810576in}}{\pgfqpoint{1.388591in}{1.813848in}}{\pgfqpoint{1.394415in}{1.819672in}}%
\pgfpathcurveto{\pgfqpoint{1.400239in}{1.825496in}}{\pgfqpoint{1.403511in}{1.833396in}}{\pgfqpoint{1.403511in}{1.841632in}}%
\pgfpathcurveto{\pgfqpoint{1.403511in}{1.849868in}}{\pgfqpoint{1.400239in}{1.857768in}}{\pgfqpoint{1.394415in}{1.863592in}}%
\pgfpathcurveto{\pgfqpoint{1.388591in}{1.869416in}}{\pgfqpoint{1.380691in}{1.872689in}}{\pgfqpoint{1.372454in}{1.872689in}}%
\pgfpathcurveto{\pgfqpoint{1.364218in}{1.872689in}}{\pgfqpoint{1.356318in}{1.869416in}}{\pgfqpoint{1.350494in}{1.863592in}}%
\pgfpathcurveto{\pgfqpoint{1.344670in}{1.857768in}}{\pgfqpoint{1.341398in}{1.849868in}}{\pgfqpoint{1.341398in}{1.841632in}}%
\pgfpathcurveto{\pgfqpoint{1.341398in}{1.833396in}}{\pgfqpoint{1.344670in}{1.825496in}}{\pgfqpoint{1.350494in}{1.819672in}}%
\pgfpathcurveto{\pgfqpoint{1.356318in}{1.813848in}}{\pgfqpoint{1.364218in}{1.810576in}}{\pgfqpoint{1.372454in}{1.810576in}}%
\pgfpathclose%
\pgfusepath{stroke,fill}%
\end{pgfscope}%
\begin{pgfscope}%
\pgfpathrectangle{\pgfqpoint{0.100000in}{0.212622in}}{\pgfqpoint{3.696000in}{3.696000in}}%
\pgfusepath{clip}%
\pgfsetbuttcap%
\pgfsetroundjoin%
\definecolor{currentfill}{rgb}{0.121569,0.466667,0.705882}%
\pgfsetfillcolor{currentfill}%
\pgfsetfillopacity{0.951942}%
\pgfsetlinewidth{1.003750pt}%
\definecolor{currentstroke}{rgb}{0.121569,0.466667,0.705882}%
\pgfsetstrokecolor{currentstroke}%
\pgfsetstrokeopacity{0.951942}%
\pgfsetdash{}{0pt}%
\pgfpathmoveto{\pgfqpoint{2.083096in}{2.404396in}}%
\pgfpathcurveto{\pgfqpoint{2.091333in}{2.404396in}}{\pgfqpoint{2.099233in}{2.407669in}}{\pgfqpoint{2.105057in}{2.413493in}}%
\pgfpathcurveto{\pgfqpoint{2.110881in}{2.419316in}}{\pgfqpoint{2.114153in}{2.427217in}}{\pgfqpoint{2.114153in}{2.435453in}}%
\pgfpathcurveto{\pgfqpoint{2.114153in}{2.443689in}}{\pgfqpoint{2.110881in}{2.451589in}}{\pgfqpoint{2.105057in}{2.457413in}}%
\pgfpathcurveto{\pgfqpoint{2.099233in}{2.463237in}}{\pgfqpoint{2.091333in}{2.466509in}}{\pgfqpoint{2.083096in}{2.466509in}}%
\pgfpathcurveto{\pgfqpoint{2.074860in}{2.466509in}}{\pgfqpoint{2.066960in}{2.463237in}}{\pgfqpoint{2.061136in}{2.457413in}}%
\pgfpathcurveto{\pgfqpoint{2.055312in}{2.451589in}}{\pgfqpoint{2.052040in}{2.443689in}}{\pgfqpoint{2.052040in}{2.435453in}}%
\pgfpathcurveto{\pgfqpoint{2.052040in}{2.427217in}}{\pgfqpoint{2.055312in}{2.419316in}}{\pgfqpoint{2.061136in}{2.413493in}}%
\pgfpathcurveto{\pgfqpoint{2.066960in}{2.407669in}}{\pgfqpoint{2.074860in}{2.404396in}}{\pgfqpoint{2.083096in}{2.404396in}}%
\pgfpathclose%
\pgfusepath{stroke,fill}%
\end{pgfscope}%
\begin{pgfscope}%
\pgfpathrectangle{\pgfqpoint{0.100000in}{0.212622in}}{\pgfqpoint{3.696000in}{3.696000in}}%
\pgfusepath{clip}%
\pgfsetbuttcap%
\pgfsetroundjoin%
\definecolor{currentfill}{rgb}{0.121569,0.466667,0.705882}%
\pgfsetfillcolor{currentfill}%
\pgfsetfillopacity{0.952075}%
\pgfsetlinewidth{1.003750pt}%
\definecolor{currentstroke}{rgb}{0.121569,0.466667,0.705882}%
\pgfsetstrokecolor{currentstroke}%
\pgfsetstrokeopacity{0.952075}%
\pgfsetdash{}{0pt}%
\pgfpathmoveto{\pgfqpoint{1.954366in}{2.435420in}}%
\pgfpathcurveto{\pgfqpoint{1.962602in}{2.435420in}}{\pgfqpoint{1.970502in}{2.438692in}}{\pgfqpoint{1.976326in}{2.444516in}}%
\pgfpathcurveto{\pgfqpoint{1.982150in}{2.450340in}}{\pgfqpoint{1.985422in}{2.458240in}}{\pgfqpoint{1.985422in}{2.466476in}}%
\pgfpathcurveto{\pgfqpoint{1.985422in}{2.474712in}}{\pgfqpoint{1.982150in}{2.482612in}}{\pgfqpoint{1.976326in}{2.488436in}}%
\pgfpathcurveto{\pgfqpoint{1.970502in}{2.494260in}}{\pgfqpoint{1.962602in}{2.497533in}}{\pgfqpoint{1.954366in}{2.497533in}}%
\pgfpathcurveto{\pgfqpoint{1.946130in}{2.497533in}}{\pgfqpoint{1.938230in}{2.494260in}}{\pgfqpoint{1.932406in}{2.488436in}}%
\pgfpathcurveto{\pgfqpoint{1.926582in}{2.482612in}}{\pgfqpoint{1.923309in}{2.474712in}}{\pgfqpoint{1.923309in}{2.466476in}}%
\pgfpathcurveto{\pgfqpoint{1.923309in}{2.458240in}}{\pgfqpoint{1.926582in}{2.450340in}}{\pgfqpoint{1.932406in}{2.444516in}}%
\pgfpathcurveto{\pgfqpoint{1.938230in}{2.438692in}}{\pgfqpoint{1.946130in}{2.435420in}}{\pgfqpoint{1.954366in}{2.435420in}}%
\pgfpathclose%
\pgfusepath{stroke,fill}%
\end{pgfscope}%
\begin{pgfscope}%
\pgfpathrectangle{\pgfqpoint{0.100000in}{0.212622in}}{\pgfqpoint{3.696000in}{3.696000in}}%
\pgfusepath{clip}%
\pgfsetbuttcap%
\pgfsetroundjoin%
\definecolor{currentfill}{rgb}{0.121569,0.466667,0.705882}%
\pgfsetfillcolor{currentfill}%
\pgfsetfillopacity{0.952228}%
\pgfsetlinewidth{1.003750pt}%
\definecolor{currentstroke}{rgb}{0.121569,0.466667,0.705882}%
\pgfsetstrokecolor{currentstroke}%
\pgfsetstrokeopacity{0.952228}%
\pgfsetdash{}{0pt}%
\pgfpathmoveto{\pgfqpoint{2.580915in}{1.193888in}}%
\pgfpathcurveto{\pgfqpoint{2.589152in}{1.193888in}}{\pgfqpoint{2.597052in}{1.197160in}}{\pgfqpoint{2.602876in}{1.202984in}}%
\pgfpathcurveto{\pgfqpoint{2.608700in}{1.208808in}}{\pgfqpoint{2.611972in}{1.216708in}}{\pgfqpoint{2.611972in}{1.224944in}}%
\pgfpathcurveto{\pgfqpoint{2.611972in}{1.233181in}}{\pgfqpoint{2.608700in}{1.241081in}}{\pgfqpoint{2.602876in}{1.246905in}}%
\pgfpathcurveto{\pgfqpoint{2.597052in}{1.252729in}}{\pgfqpoint{2.589152in}{1.256001in}}{\pgfqpoint{2.580915in}{1.256001in}}%
\pgfpathcurveto{\pgfqpoint{2.572679in}{1.256001in}}{\pgfqpoint{2.564779in}{1.252729in}}{\pgfqpoint{2.558955in}{1.246905in}}%
\pgfpathcurveto{\pgfqpoint{2.553131in}{1.241081in}}{\pgfqpoint{2.549859in}{1.233181in}}{\pgfqpoint{2.549859in}{1.224944in}}%
\pgfpathcurveto{\pgfqpoint{2.549859in}{1.216708in}}{\pgfqpoint{2.553131in}{1.208808in}}{\pgfqpoint{2.558955in}{1.202984in}}%
\pgfpathcurveto{\pgfqpoint{2.564779in}{1.197160in}}{\pgfqpoint{2.572679in}{1.193888in}}{\pgfqpoint{2.580915in}{1.193888in}}%
\pgfpathclose%
\pgfusepath{stroke,fill}%
\end{pgfscope}%
\begin{pgfscope}%
\pgfpathrectangle{\pgfqpoint{0.100000in}{0.212622in}}{\pgfqpoint{3.696000in}{3.696000in}}%
\pgfusepath{clip}%
\pgfsetbuttcap%
\pgfsetroundjoin%
\definecolor{currentfill}{rgb}{0.121569,0.466667,0.705882}%
\pgfsetfillcolor{currentfill}%
\pgfsetfillopacity{0.952346}%
\pgfsetlinewidth{1.003750pt}%
\definecolor{currentstroke}{rgb}{0.121569,0.466667,0.705882}%
\pgfsetstrokecolor{currentstroke}%
\pgfsetstrokeopacity{0.952346}%
\pgfsetdash{}{0pt}%
\pgfpathmoveto{\pgfqpoint{1.380302in}{1.804709in}}%
\pgfpathcurveto{\pgfqpoint{1.388539in}{1.804709in}}{\pgfqpoint{1.396439in}{1.807981in}}{\pgfqpoint{1.402263in}{1.813805in}}%
\pgfpathcurveto{\pgfqpoint{1.408087in}{1.819629in}}{\pgfqpoint{1.411359in}{1.827529in}}{\pgfqpoint{1.411359in}{1.835765in}}%
\pgfpathcurveto{\pgfqpoint{1.411359in}{1.844002in}}{\pgfqpoint{1.408087in}{1.851902in}}{\pgfqpoint{1.402263in}{1.857726in}}%
\pgfpathcurveto{\pgfqpoint{1.396439in}{1.863549in}}{\pgfqpoint{1.388539in}{1.866822in}}{\pgfqpoint{1.380302in}{1.866822in}}%
\pgfpathcurveto{\pgfqpoint{1.372066in}{1.866822in}}{\pgfqpoint{1.364166in}{1.863549in}}{\pgfqpoint{1.358342in}{1.857726in}}%
\pgfpathcurveto{\pgfqpoint{1.352518in}{1.851902in}}{\pgfqpoint{1.349246in}{1.844002in}}{\pgfqpoint{1.349246in}{1.835765in}}%
\pgfpathcurveto{\pgfqpoint{1.349246in}{1.827529in}}{\pgfqpoint{1.352518in}{1.819629in}}{\pgfqpoint{1.358342in}{1.813805in}}%
\pgfpathcurveto{\pgfqpoint{1.364166in}{1.807981in}}{\pgfqpoint{1.372066in}{1.804709in}}{\pgfqpoint{1.380302in}{1.804709in}}%
\pgfpathclose%
\pgfusepath{stroke,fill}%
\end{pgfscope}%
\begin{pgfscope}%
\pgfpathrectangle{\pgfqpoint{0.100000in}{0.212622in}}{\pgfqpoint{3.696000in}{3.696000in}}%
\pgfusepath{clip}%
\pgfsetbuttcap%
\pgfsetroundjoin%
\definecolor{currentfill}{rgb}{0.121569,0.466667,0.705882}%
\pgfsetfillcolor{currentfill}%
\pgfsetfillopacity{0.952350}%
\pgfsetlinewidth{1.003750pt}%
\definecolor{currentstroke}{rgb}{0.121569,0.466667,0.705882}%
\pgfsetstrokecolor{currentstroke}%
\pgfsetstrokeopacity{0.952350}%
\pgfsetdash{}{0pt}%
\pgfpathmoveto{\pgfqpoint{1.958053in}{2.433752in}}%
\pgfpathcurveto{\pgfqpoint{1.966289in}{2.433752in}}{\pgfqpoint{1.974189in}{2.437024in}}{\pgfqpoint{1.980013in}{2.442848in}}%
\pgfpathcurveto{\pgfqpoint{1.985837in}{2.448672in}}{\pgfqpoint{1.989109in}{2.456572in}}{\pgfqpoint{1.989109in}{2.464809in}}%
\pgfpathcurveto{\pgfqpoint{1.989109in}{2.473045in}}{\pgfqpoint{1.985837in}{2.480945in}}{\pgfqpoint{1.980013in}{2.486769in}}%
\pgfpathcurveto{\pgfqpoint{1.974189in}{2.492593in}}{\pgfqpoint{1.966289in}{2.495865in}}{\pgfqpoint{1.958053in}{2.495865in}}%
\pgfpathcurveto{\pgfqpoint{1.949816in}{2.495865in}}{\pgfqpoint{1.941916in}{2.492593in}}{\pgfqpoint{1.936092in}{2.486769in}}%
\pgfpathcurveto{\pgfqpoint{1.930268in}{2.480945in}}{\pgfqpoint{1.926996in}{2.473045in}}{\pgfqpoint{1.926996in}{2.464809in}}%
\pgfpathcurveto{\pgfqpoint{1.926996in}{2.456572in}}{\pgfqpoint{1.930268in}{2.448672in}}{\pgfqpoint{1.936092in}{2.442848in}}%
\pgfpathcurveto{\pgfqpoint{1.941916in}{2.437024in}}{\pgfqpoint{1.949816in}{2.433752in}}{\pgfqpoint{1.958053in}{2.433752in}}%
\pgfpathclose%
\pgfusepath{stroke,fill}%
\end{pgfscope}%
\begin{pgfscope}%
\pgfpathrectangle{\pgfqpoint{0.100000in}{0.212622in}}{\pgfqpoint{3.696000in}{3.696000in}}%
\pgfusepath{clip}%
\pgfsetbuttcap%
\pgfsetroundjoin%
\definecolor{currentfill}{rgb}{0.121569,0.466667,0.705882}%
\pgfsetfillcolor{currentfill}%
\pgfsetfillopacity{0.952582}%
\pgfsetlinewidth{1.003750pt}%
\definecolor{currentstroke}{rgb}{0.121569,0.466667,0.705882}%
\pgfsetstrokecolor{currentstroke}%
\pgfsetstrokeopacity{0.952582}%
\pgfsetdash{}{0pt}%
\pgfpathmoveto{\pgfqpoint{2.081746in}{2.403349in}}%
\pgfpathcurveto{\pgfqpoint{2.089982in}{2.403349in}}{\pgfqpoint{2.097882in}{2.406621in}}{\pgfqpoint{2.103706in}{2.412445in}}%
\pgfpathcurveto{\pgfqpoint{2.109530in}{2.418269in}}{\pgfqpoint{2.112802in}{2.426169in}}{\pgfqpoint{2.112802in}{2.434405in}}%
\pgfpathcurveto{\pgfqpoint{2.112802in}{2.442641in}}{\pgfqpoint{2.109530in}{2.450541in}}{\pgfqpoint{2.103706in}{2.456365in}}%
\pgfpathcurveto{\pgfqpoint{2.097882in}{2.462189in}}{\pgfqpoint{2.089982in}{2.465462in}}{\pgfqpoint{2.081746in}{2.465462in}}%
\pgfpathcurveto{\pgfqpoint{2.073510in}{2.465462in}}{\pgfqpoint{2.065609in}{2.462189in}}{\pgfqpoint{2.059786in}{2.456365in}}%
\pgfpathcurveto{\pgfqpoint{2.053962in}{2.450541in}}{\pgfqpoint{2.050689in}{2.442641in}}{\pgfqpoint{2.050689in}{2.434405in}}%
\pgfpathcurveto{\pgfqpoint{2.050689in}{2.426169in}}{\pgfqpoint{2.053962in}{2.418269in}}{\pgfqpoint{2.059786in}{2.412445in}}%
\pgfpathcurveto{\pgfqpoint{2.065609in}{2.406621in}}{\pgfqpoint{2.073510in}{2.403349in}}{\pgfqpoint{2.081746in}{2.403349in}}%
\pgfpathclose%
\pgfusepath{stroke,fill}%
\end{pgfscope}%
\begin{pgfscope}%
\pgfpathrectangle{\pgfqpoint{0.100000in}{0.212622in}}{\pgfqpoint{3.696000in}{3.696000in}}%
\pgfusepath{clip}%
\pgfsetbuttcap%
\pgfsetroundjoin%
\definecolor{currentfill}{rgb}{0.121569,0.466667,0.705882}%
\pgfsetfillcolor{currentfill}%
\pgfsetfillopacity{0.952686}%
\pgfsetlinewidth{1.003750pt}%
\definecolor{currentstroke}{rgb}{0.121569,0.466667,0.705882}%
\pgfsetstrokecolor{currentstroke}%
\pgfsetstrokeopacity{0.952686}%
\pgfsetdash{}{0pt}%
\pgfpathmoveto{\pgfqpoint{1.962667in}{2.431657in}}%
\pgfpathcurveto{\pgfqpoint{1.970903in}{2.431657in}}{\pgfqpoint{1.978803in}{2.434929in}}{\pgfqpoint{1.984627in}{2.440753in}}%
\pgfpathcurveto{\pgfqpoint{1.990451in}{2.446577in}}{\pgfqpoint{1.993723in}{2.454477in}}{\pgfqpoint{1.993723in}{2.462713in}}%
\pgfpathcurveto{\pgfqpoint{1.993723in}{2.470949in}}{\pgfqpoint{1.990451in}{2.478849in}}{\pgfqpoint{1.984627in}{2.484673in}}%
\pgfpathcurveto{\pgfqpoint{1.978803in}{2.490497in}}{\pgfqpoint{1.970903in}{2.493770in}}{\pgfqpoint{1.962667in}{2.493770in}}%
\pgfpathcurveto{\pgfqpoint{1.954430in}{2.493770in}}{\pgfqpoint{1.946530in}{2.490497in}}{\pgfqpoint{1.940706in}{2.484673in}}%
\pgfpathcurveto{\pgfqpoint{1.934882in}{2.478849in}}{\pgfqpoint{1.931610in}{2.470949in}}{\pgfqpoint{1.931610in}{2.462713in}}%
\pgfpathcurveto{\pgfqpoint{1.931610in}{2.454477in}}{\pgfqpoint{1.934882in}{2.446577in}}{\pgfqpoint{1.940706in}{2.440753in}}%
\pgfpathcurveto{\pgfqpoint{1.946530in}{2.434929in}}{\pgfqpoint{1.954430in}{2.431657in}}{\pgfqpoint{1.962667in}{2.431657in}}%
\pgfpathclose%
\pgfusepath{stroke,fill}%
\end{pgfscope}%
\begin{pgfscope}%
\pgfpathrectangle{\pgfqpoint{0.100000in}{0.212622in}}{\pgfqpoint{3.696000in}{3.696000in}}%
\pgfusepath{clip}%
\pgfsetbuttcap%
\pgfsetroundjoin%
\definecolor{currentfill}{rgb}{0.121569,0.466667,0.705882}%
\pgfsetfillcolor{currentfill}%
\pgfsetfillopacity{0.952902}%
\pgfsetlinewidth{1.003750pt}%
\definecolor{currentstroke}{rgb}{0.121569,0.466667,0.705882}%
\pgfsetstrokecolor{currentstroke}%
\pgfsetstrokeopacity{0.952902}%
\pgfsetdash{}{0pt}%
\pgfpathmoveto{\pgfqpoint{1.390705in}{1.797492in}}%
\pgfpathcurveto{\pgfqpoint{1.398941in}{1.797492in}}{\pgfqpoint{1.406841in}{1.800764in}}{\pgfqpoint{1.412665in}{1.806588in}}%
\pgfpathcurveto{\pgfqpoint{1.418489in}{1.812412in}}{\pgfqpoint{1.421761in}{1.820312in}}{\pgfqpoint{1.421761in}{1.828549in}}%
\pgfpathcurveto{\pgfqpoint{1.421761in}{1.836785in}}{\pgfqpoint{1.418489in}{1.844685in}}{\pgfqpoint{1.412665in}{1.850509in}}%
\pgfpathcurveto{\pgfqpoint{1.406841in}{1.856333in}}{\pgfqpoint{1.398941in}{1.859605in}}{\pgfqpoint{1.390705in}{1.859605in}}%
\pgfpathcurveto{\pgfqpoint{1.382468in}{1.859605in}}{\pgfqpoint{1.374568in}{1.856333in}}{\pgfqpoint{1.368744in}{1.850509in}}%
\pgfpathcurveto{\pgfqpoint{1.362921in}{1.844685in}}{\pgfqpoint{1.359648in}{1.836785in}}{\pgfqpoint{1.359648in}{1.828549in}}%
\pgfpathcurveto{\pgfqpoint{1.359648in}{1.820312in}}{\pgfqpoint{1.362921in}{1.812412in}}{\pgfqpoint{1.368744in}{1.806588in}}%
\pgfpathcurveto{\pgfqpoint{1.374568in}{1.800764in}}{\pgfqpoint{1.382468in}{1.797492in}}{\pgfqpoint{1.390705in}{1.797492in}}%
\pgfpathclose%
\pgfusepath{stroke,fill}%
\end{pgfscope}%
\begin{pgfscope}%
\pgfpathrectangle{\pgfqpoint{0.100000in}{0.212622in}}{\pgfqpoint{3.696000in}{3.696000in}}%
\pgfusepath{clip}%
\pgfsetbuttcap%
\pgfsetroundjoin%
\definecolor{currentfill}{rgb}{0.121569,0.466667,0.705882}%
\pgfsetfillcolor{currentfill}%
\pgfsetfillopacity{0.952915}%
\pgfsetlinewidth{1.003750pt}%
\definecolor{currentstroke}{rgb}{0.121569,0.466667,0.705882}%
\pgfsetstrokecolor{currentstroke}%
\pgfsetstrokeopacity{0.952915}%
\pgfsetdash{}{0pt}%
\pgfpathmoveto{\pgfqpoint{1.967829in}{2.429657in}}%
\pgfpathcurveto{\pgfqpoint{1.976066in}{2.429657in}}{\pgfqpoint{1.983966in}{2.432930in}}{\pgfqpoint{1.989790in}{2.438754in}}%
\pgfpathcurveto{\pgfqpoint{1.995613in}{2.444578in}}{\pgfqpoint{1.998886in}{2.452478in}}{\pgfqpoint{1.998886in}{2.460714in}}%
\pgfpathcurveto{\pgfqpoint{1.998886in}{2.468950in}}{\pgfqpoint{1.995613in}{2.476850in}}{\pgfqpoint{1.989790in}{2.482674in}}%
\pgfpathcurveto{\pgfqpoint{1.983966in}{2.488498in}}{\pgfqpoint{1.976066in}{2.491770in}}{\pgfqpoint{1.967829in}{2.491770in}}%
\pgfpathcurveto{\pgfqpoint{1.959593in}{2.491770in}}{\pgfqpoint{1.951693in}{2.488498in}}{\pgfqpoint{1.945869in}{2.482674in}}%
\pgfpathcurveto{\pgfqpoint{1.940045in}{2.476850in}}{\pgfqpoint{1.936773in}{2.468950in}}{\pgfqpoint{1.936773in}{2.460714in}}%
\pgfpathcurveto{\pgfqpoint{1.936773in}{2.452478in}}{\pgfqpoint{1.940045in}{2.444578in}}{\pgfqpoint{1.945869in}{2.438754in}}%
\pgfpathcurveto{\pgfqpoint{1.951693in}{2.432930in}}{\pgfqpoint{1.959593in}{2.429657in}}{\pgfqpoint{1.967829in}{2.429657in}}%
\pgfpathclose%
\pgfusepath{stroke,fill}%
\end{pgfscope}%
\begin{pgfscope}%
\pgfpathrectangle{\pgfqpoint{0.100000in}{0.212622in}}{\pgfqpoint{3.696000in}{3.696000in}}%
\pgfusepath{clip}%
\pgfsetbuttcap%
\pgfsetroundjoin%
\definecolor{currentfill}{rgb}{0.121569,0.466667,0.705882}%
\pgfsetfillcolor{currentfill}%
\pgfsetfillopacity{0.953055}%
\pgfsetlinewidth{1.003750pt}%
\definecolor{currentstroke}{rgb}{0.121569,0.466667,0.705882}%
\pgfsetstrokecolor{currentstroke}%
\pgfsetstrokeopacity{0.953055}%
\pgfsetdash{}{0pt}%
\pgfpathmoveto{\pgfqpoint{1.970658in}{2.428590in}}%
\pgfpathcurveto{\pgfqpoint{1.978894in}{2.428590in}}{\pgfqpoint{1.986794in}{2.431862in}}{\pgfqpoint{1.992618in}{2.437686in}}%
\pgfpathcurveto{\pgfqpoint{1.998442in}{2.443510in}}{\pgfqpoint{2.001714in}{2.451410in}}{\pgfqpoint{2.001714in}{2.459647in}}%
\pgfpathcurveto{\pgfqpoint{2.001714in}{2.467883in}}{\pgfqpoint{1.998442in}{2.475783in}}{\pgfqpoint{1.992618in}{2.481607in}}%
\pgfpathcurveto{\pgfqpoint{1.986794in}{2.487431in}}{\pgfqpoint{1.978894in}{2.490703in}}{\pgfqpoint{1.970658in}{2.490703in}}%
\pgfpathcurveto{\pgfqpoint{1.962421in}{2.490703in}}{\pgfqpoint{1.954521in}{2.487431in}}{\pgfqpoint{1.948697in}{2.481607in}}%
\pgfpathcurveto{\pgfqpoint{1.942874in}{2.475783in}}{\pgfqpoint{1.939601in}{2.467883in}}{\pgfqpoint{1.939601in}{2.459647in}}%
\pgfpathcurveto{\pgfqpoint{1.939601in}{2.451410in}}{\pgfqpoint{1.942874in}{2.443510in}}{\pgfqpoint{1.948697in}{2.437686in}}%
\pgfpathcurveto{\pgfqpoint{1.954521in}{2.431862in}}{\pgfqpoint{1.962421in}{2.428590in}}{\pgfqpoint{1.970658in}{2.428590in}}%
\pgfpathclose%
\pgfusepath{stroke,fill}%
\end{pgfscope}%
\begin{pgfscope}%
\pgfpathrectangle{\pgfqpoint{0.100000in}{0.212622in}}{\pgfqpoint{3.696000in}{3.696000in}}%
\pgfusepath{clip}%
\pgfsetbuttcap%
\pgfsetroundjoin%
\definecolor{currentfill}{rgb}{0.121569,0.466667,0.705882}%
\pgfsetfillcolor{currentfill}%
\pgfsetfillopacity{0.953087}%
\pgfsetlinewidth{1.003750pt}%
\definecolor{currentstroke}{rgb}{0.121569,0.466667,0.705882}%
\pgfsetstrokecolor{currentstroke}%
\pgfsetstrokeopacity{0.953087}%
\pgfsetdash{}{0pt}%
\pgfpathmoveto{\pgfqpoint{2.080655in}{2.402247in}}%
\pgfpathcurveto{\pgfqpoint{2.088891in}{2.402247in}}{\pgfqpoint{2.096791in}{2.405519in}}{\pgfqpoint{2.102615in}{2.411343in}}%
\pgfpathcurveto{\pgfqpoint{2.108439in}{2.417167in}}{\pgfqpoint{2.111711in}{2.425067in}}{\pgfqpoint{2.111711in}{2.433303in}}%
\pgfpathcurveto{\pgfqpoint{2.111711in}{2.441540in}}{\pgfqpoint{2.108439in}{2.449440in}}{\pgfqpoint{2.102615in}{2.455264in}}%
\pgfpathcurveto{\pgfqpoint{2.096791in}{2.461088in}}{\pgfqpoint{2.088891in}{2.464360in}}{\pgfqpoint{2.080655in}{2.464360in}}%
\pgfpathcurveto{\pgfqpoint{2.072418in}{2.464360in}}{\pgfqpoint{2.064518in}{2.461088in}}{\pgfqpoint{2.058694in}{2.455264in}}%
\pgfpathcurveto{\pgfqpoint{2.052870in}{2.449440in}}{\pgfqpoint{2.049598in}{2.441540in}}{\pgfqpoint{2.049598in}{2.433303in}}%
\pgfpathcurveto{\pgfqpoint{2.049598in}{2.425067in}}{\pgfqpoint{2.052870in}{2.417167in}}{\pgfqpoint{2.058694in}{2.411343in}}%
\pgfpathcurveto{\pgfqpoint{2.064518in}{2.405519in}}{\pgfqpoint{2.072418in}{2.402247in}}{\pgfqpoint{2.080655in}{2.402247in}}%
\pgfpathclose%
\pgfusepath{stroke,fill}%
\end{pgfscope}%
\begin{pgfscope}%
\pgfpathrectangle{\pgfqpoint{0.100000in}{0.212622in}}{\pgfqpoint{3.696000in}{3.696000in}}%
\pgfusepath{clip}%
\pgfsetbuttcap%
\pgfsetroundjoin%
\definecolor{currentfill}{rgb}{0.121569,0.466667,0.705882}%
\pgfsetfillcolor{currentfill}%
\pgfsetfillopacity{0.953239}%
\pgfsetlinewidth{1.003750pt}%
\definecolor{currentstroke}{rgb}{0.121569,0.466667,0.705882}%
\pgfsetstrokecolor{currentstroke}%
\pgfsetstrokeopacity{0.953239}%
\pgfsetdash{}{0pt}%
\pgfpathmoveto{\pgfqpoint{1.974433in}{2.427169in}}%
\pgfpathcurveto{\pgfqpoint{1.982669in}{2.427169in}}{\pgfqpoint{1.990569in}{2.430441in}}{\pgfqpoint{1.996393in}{2.436265in}}%
\pgfpathcurveto{\pgfqpoint{2.002217in}{2.442089in}}{\pgfqpoint{2.005490in}{2.449989in}}{\pgfqpoint{2.005490in}{2.458225in}}%
\pgfpathcurveto{\pgfqpoint{2.005490in}{2.466462in}}{\pgfqpoint{2.002217in}{2.474362in}}{\pgfqpoint{1.996393in}{2.480186in}}%
\pgfpathcurveto{\pgfqpoint{1.990569in}{2.486010in}}{\pgfqpoint{1.982669in}{2.489282in}}{\pgfqpoint{1.974433in}{2.489282in}}%
\pgfpathcurveto{\pgfqpoint{1.966197in}{2.489282in}}{\pgfqpoint{1.958297in}{2.486010in}}{\pgfqpoint{1.952473in}{2.480186in}}%
\pgfpathcurveto{\pgfqpoint{1.946649in}{2.474362in}}{\pgfqpoint{1.943377in}{2.466462in}}{\pgfqpoint{1.943377in}{2.458225in}}%
\pgfpathcurveto{\pgfqpoint{1.943377in}{2.449989in}}{\pgfqpoint{1.946649in}{2.442089in}}{\pgfqpoint{1.952473in}{2.436265in}}%
\pgfpathcurveto{\pgfqpoint{1.958297in}{2.430441in}}{\pgfqpoint{1.966197in}{2.427169in}}{\pgfqpoint{1.974433in}{2.427169in}}%
\pgfpathclose%
\pgfusepath{stroke,fill}%
\end{pgfscope}%
\begin{pgfscope}%
\pgfpathrectangle{\pgfqpoint{0.100000in}{0.212622in}}{\pgfqpoint{3.696000in}{3.696000in}}%
\pgfusepath{clip}%
\pgfsetbuttcap%
\pgfsetroundjoin%
\definecolor{currentfill}{rgb}{0.121569,0.466667,0.705882}%
\pgfsetfillcolor{currentfill}%
\pgfsetfillopacity{0.953398}%
\pgfsetlinewidth{1.003750pt}%
\definecolor{currentstroke}{rgb}{0.121569,0.466667,0.705882}%
\pgfsetstrokecolor{currentstroke}%
\pgfsetstrokeopacity{0.953398}%
\pgfsetdash{}{0pt}%
\pgfpathmoveto{\pgfqpoint{2.079988in}{2.401546in}}%
\pgfpathcurveto{\pgfqpoint{2.088224in}{2.401546in}}{\pgfqpoint{2.096124in}{2.404818in}}{\pgfqpoint{2.101948in}{2.410642in}}%
\pgfpathcurveto{\pgfqpoint{2.107772in}{2.416466in}}{\pgfqpoint{2.111044in}{2.424366in}}{\pgfqpoint{2.111044in}{2.432602in}}%
\pgfpathcurveto{\pgfqpoint{2.111044in}{2.440839in}}{\pgfqpoint{2.107772in}{2.448739in}}{\pgfqpoint{2.101948in}{2.454563in}}%
\pgfpathcurveto{\pgfqpoint{2.096124in}{2.460387in}}{\pgfqpoint{2.088224in}{2.463659in}}{\pgfqpoint{2.079988in}{2.463659in}}%
\pgfpathcurveto{\pgfqpoint{2.071752in}{2.463659in}}{\pgfqpoint{2.063851in}{2.460387in}}{\pgfqpoint{2.058028in}{2.454563in}}%
\pgfpathcurveto{\pgfqpoint{2.052204in}{2.448739in}}{\pgfqpoint{2.048931in}{2.440839in}}{\pgfqpoint{2.048931in}{2.432602in}}%
\pgfpathcurveto{\pgfqpoint{2.048931in}{2.424366in}}{\pgfqpoint{2.052204in}{2.416466in}}{\pgfqpoint{2.058028in}{2.410642in}}%
\pgfpathcurveto{\pgfqpoint{2.063851in}{2.404818in}}{\pgfqpoint{2.071752in}{2.401546in}}{\pgfqpoint{2.079988in}{2.401546in}}%
\pgfpathclose%
\pgfusepath{stroke,fill}%
\end{pgfscope}%
\begin{pgfscope}%
\pgfpathrectangle{\pgfqpoint{0.100000in}{0.212622in}}{\pgfqpoint{3.696000in}{3.696000in}}%
\pgfusepath{clip}%
\pgfsetbuttcap%
\pgfsetroundjoin%
\definecolor{currentfill}{rgb}{0.121569,0.466667,0.705882}%
\pgfsetfillcolor{currentfill}%
\pgfsetfillopacity{0.953466}%
\pgfsetlinewidth{1.003750pt}%
\definecolor{currentstroke}{rgb}{0.121569,0.466667,0.705882}%
\pgfsetstrokecolor{currentstroke}%
\pgfsetstrokeopacity{0.953466}%
\pgfsetdash{}{0pt}%
\pgfpathmoveto{\pgfqpoint{1.979483in}{2.425773in}}%
\pgfpathcurveto{\pgfqpoint{1.987719in}{2.425773in}}{\pgfqpoint{1.995619in}{2.429045in}}{\pgfqpoint{2.001443in}{2.434869in}}%
\pgfpathcurveto{\pgfqpoint{2.007267in}{2.440693in}}{\pgfqpoint{2.010540in}{2.448593in}}{\pgfqpoint{2.010540in}{2.456829in}}%
\pgfpathcurveto{\pgfqpoint{2.010540in}{2.465065in}}{\pgfqpoint{2.007267in}{2.472965in}}{\pgfqpoint{2.001443in}{2.478789in}}%
\pgfpathcurveto{\pgfqpoint{1.995619in}{2.484613in}}{\pgfqpoint{1.987719in}{2.487886in}}{\pgfqpoint{1.979483in}{2.487886in}}%
\pgfpathcurveto{\pgfqpoint{1.971247in}{2.487886in}}{\pgfqpoint{1.963347in}{2.484613in}}{\pgfqpoint{1.957523in}{2.478789in}}%
\pgfpathcurveto{\pgfqpoint{1.951699in}{2.472965in}}{\pgfqpoint{1.948427in}{2.465065in}}{\pgfqpoint{1.948427in}{2.456829in}}%
\pgfpathcurveto{\pgfqpoint{1.948427in}{2.448593in}}{\pgfqpoint{1.951699in}{2.440693in}}{\pgfqpoint{1.957523in}{2.434869in}}%
\pgfpathcurveto{\pgfqpoint{1.963347in}{2.429045in}}{\pgfqpoint{1.971247in}{2.425773in}}{\pgfqpoint{1.979483in}{2.425773in}}%
\pgfpathclose%
\pgfusepath{stroke,fill}%
\end{pgfscope}%
\begin{pgfscope}%
\pgfpathrectangle{\pgfqpoint{0.100000in}{0.212622in}}{\pgfqpoint{3.696000in}{3.696000in}}%
\pgfusepath{clip}%
\pgfsetbuttcap%
\pgfsetroundjoin%
\definecolor{currentfill}{rgb}{0.121569,0.466667,0.705882}%
\pgfsetfillcolor{currentfill}%
\pgfsetfillopacity{0.953572}%
\pgfsetlinewidth{1.003750pt}%
\definecolor{currentstroke}{rgb}{0.121569,0.466667,0.705882}%
\pgfsetstrokecolor{currentstroke}%
\pgfsetstrokeopacity{0.953572}%
\pgfsetdash{}{0pt}%
\pgfpathmoveto{\pgfqpoint{2.079600in}{2.401166in}}%
\pgfpathcurveto{\pgfqpoint{2.087837in}{2.401166in}}{\pgfqpoint{2.095737in}{2.404438in}}{\pgfqpoint{2.101561in}{2.410262in}}%
\pgfpathcurveto{\pgfqpoint{2.107384in}{2.416086in}}{\pgfqpoint{2.110657in}{2.423986in}}{\pgfqpoint{2.110657in}{2.432222in}}%
\pgfpathcurveto{\pgfqpoint{2.110657in}{2.440459in}}{\pgfqpoint{2.107384in}{2.448359in}}{\pgfqpoint{2.101561in}{2.454183in}}%
\pgfpathcurveto{\pgfqpoint{2.095737in}{2.460006in}}{\pgfqpoint{2.087837in}{2.463279in}}{\pgfqpoint{2.079600in}{2.463279in}}%
\pgfpathcurveto{\pgfqpoint{2.071364in}{2.463279in}}{\pgfqpoint{2.063464in}{2.460006in}}{\pgfqpoint{2.057640in}{2.454183in}}%
\pgfpathcurveto{\pgfqpoint{2.051816in}{2.448359in}}{\pgfqpoint{2.048544in}{2.440459in}}{\pgfqpoint{2.048544in}{2.432222in}}%
\pgfpathcurveto{\pgfqpoint{2.048544in}{2.423986in}}{\pgfqpoint{2.051816in}{2.416086in}}{\pgfqpoint{2.057640in}{2.410262in}}%
\pgfpathcurveto{\pgfqpoint{2.063464in}{2.404438in}}{\pgfqpoint{2.071364in}{2.401166in}}{\pgfqpoint{2.079600in}{2.401166in}}%
\pgfpathclose%
\pgfusepath{stroke,fill}%
\end{pgfscope}%
\begin{pgfscope}%
\pgfpathrectangle{\pgfqpoint{0.100000in}{0.212622in}}{\pgfqpoint{3.696000in}{3.696000in}}%
\pgfusepath{clip}%
\pgfsetbuttcap%
\pgfsetroundjoin%
\definecolor{currentfill}{rgb}{0.121569,0.466667,0.705882}%
\pgfsetfillcolor{currentfill}%
\pgfsetfillopacity{0.953582}%
\pgfsetlinewidth{1.003750pt}%
\definecolor{currentstroke}{rgb}{0.121569,0.466667,0.705882}%
\pgfsetstrokecolor{currentstroke}%
\pgfsetstrokeopacity{0.953582}%
\pgfsetdash{}{0pt}%
\pgfpathmoveto{\pgfqpoint{1.403746in}{1.788786in}}%
\pgfpathcurveto{\pgfqpoint{1.411982in}{1.788786in}}{\pgfqpoint{1.419882in}{1.792058in}}{\pgfqpoint{1.425706in}{1.797882in}}%
\pgfpathcurveto{\pgfqpoint{1.431530in}{1.803706in}}{\pgfqpoint{1.434802in}{1.811606in}}{\pgfqpoint{1.434802in}{1.819842in}}%
\pgfpathcurveto{\pgfqpoint{1.434802in}{1.828078in}}{\pgfqpoint{1.431530in}{1.835978in}}{\pgfqpoint{1.425706in}{1.841802in}}%
\pgfpathcurveto{\pgfqpoint{1.419882in}{1.847626in}}{\pgfqpoint{1.411982in}{1.850899in}}{\pgfqpoint{1.403746in}{1.850899in}}%
\pgfpathcurveto{\pgfqpoint{1.395509in}{1.850899in}}{\pgfqpoint{1.387609in}{1.847626in}}{\pgfqpoint{1.381785in}{1.841802in}}%
\pgfpathcurveto{\pgfqpoint{1.375962in}{1.835978in}}{\pgfqpoint{1.372689in}{1.828078in}}{\pgfqpoint{1.372689in}{1.819842in}}%
\pgfpathcurveto{\pgfqpoint{1.372689in}{1.811606in}}{\pgfqpoint{1.375962in}{1.803706in}}{\pgfqpoint{1.381785in}{1.797882in}}%
\pgfpathcurveto{\pgfqpoint{1.387609in}{1.792058in}}{\pgfqpoint{1.395509in}{1.788786in}}{\pgfqpoint{1.403746in}{1.788786in}}%
\pgfpathclose%
\pgfusepath{stroke,fill}%
\end{pgfscope}%
\begin{pgfscope}%
\pgfpathrectangle{\pgfqpoint{0.100000in}{0.212622in}}{\pgfqpoint{3.696000in}{3.696000in}}%
\pgfusepath{clip}%
\pgfsetbuttcap%
\pgfsetroundjoin%
\definecolor{currentfill}{rgb}{0.121569,0.466667,0.705882}%
\pgfsetfillcolor{currentfill}%
\pgfsetfillopacity{0.953635}%
\pgfsetlinewidth{1.003750pt}%
\definecolor{currentstroke}{rgb}{0.121569,0.466667,0.705882}%
\pgfsetstrokecolor{currentstroke}%
\pgfsetstrokeopacity{0.953635}%
\pgfsetdash{}{0pt}%
\pgfpathmoveto{\pgfqpoint{2.079464in}{2.401028in}}%
\pgfpathcurveto{\pgfqpoint{2.087700in}{2.401028in}}{\pgfqpoint{2.095600in}{2.404301in}}{\pgfqpoint{2.101424in}{2.410125in}}%
\pgfpathcurveto{\pgfqpoint{2.107248in}{2.415948in}}{\pgfqpoint{2.110520in}{2.423849in}}{\pgfqpoint{2.110520in}{2.432085in}}%
\pgfpathcurveto{\pgfqpoint{2.110520in}{2.440321in}}{\pgfqpoint{2.107248in}{2.448221in}}{\pgfqpoint{2.101424in}{2.454045in}}%
\pgfpathcurveto{\pgfqpoint{2.095600in}{2.459869in}}{\pgfqpoint{2.087700in}{2.463141in}}{\pgfqpoint{2.079464in}{2.463141in}}%
\pgfpathcurveto{\pgfqpoint{2.071227in}{2.463141in}}{\pgfqpoint{2.063327in}{2.459869in}}{\pgfqpoint{2.057503in}{2.454045in}}%
\pgfpathcurveto{\pgfqpoint{2.051679in}{2.448221in}}{\pgfqpoint{2.048407in}{2.440321in}}{\pgfqpoint{2.048407in}{2.432085in}}%
\pgfpathcurveto{\pgfqpoint{2.048407in}{2.423849in}}{\pgfqpoint{2.051679in}{2.415948in}}{\pgfqpoint{2.057503in}{2.410125in}}%
\pgfpathcurveto{\pgfqpoint{2.063327in}{2.404301in}}{\pgfqpoint{2.071227in}{2.401028in}}{\pgfqpoint{2.079464in}{2.401028in}}%
\pgfpathclose%
\pgfusepath{stroke,fill}%
\end{pgfscope}%
\begin{pgfscope}%
\pgfpathrectangle{\pgfqpoint{0.100000in}{0.212622in}}{\pgfqpoint{3.696000in}{3.696000in}}%
\pgfusepath{clip}%
\pgfsetbuttcap%
\pgfsetroundjoin%
\definecolor{currentfill}{rgb}{0.121569,0.466667,0.705882}%
\pgfsetfillcolor{currentfill}%
\pgfsetfillopacity{0.953684}%
\pgfsetlinewidth{1.003750pt}%
\definecolor{currentstroke}{rgb}{0.121569,0.466667,0.705882}%
\pgfsetstrokecolor{currentstroke}%
\pgfsetstrokeopacity{0.953684}%
\pgfsetdash{}{0pt}%
\pgfpathmoveto{\pgfqpoint{2.578073in}{1.190370in}}%
\pgfpathcurveto{\pgfqpoint{2.586310in}{1.190370in}}{\pgfqpoint{2.594210in}{1.193642in}}{\pgfqpoint{2.600034in}{1.199466in}}%
\pgfpathcurveto{\pgfqpoint{2.605858in}{1.205290in}}{\pgfqpoint{2.609130in}{1.213190in}}{\pgfqpoint{2.609130in}{1.221427in}}%
\pgfpathcurveto{\pgfqpoint{2.609130in}{1.229663in}}{\pgfqpoint{2.605858in}{1.237563in}}{\pgfqpoint{2.600034in}{1.243387in}}%
\pgfpathcurveto{\pgfqpoint{2.594210in}{1.249211in}}{\pgfqpoint{2.586310in}{1.252483in}}{\pgfqpoint{2.578073in}{1.252483in}}%
\pgfpathcurveto{\pgfqpoint{2.569837in}{1.252483in}}{\pgfqpoint{2.561937in}{1.249211in}}{\pgfqpoint{2.556113in}{1.243387in}}%
\pgfpathcurveto{\pgfqpoint{2.550289in}{1.237563in}}{\pgfqpoint{2.547017in}{1.229663in}}{\pgfqpoint{2.547017in}{1.221427in}}%
\pgfpathcurveto{\pgfqpoint{2.547017in}{1.213190in}}{\pgfqpoint{2.550289in}{1.205290in}}{\pgfqpoint{2.556113in}{1.199466in}}%
\pgfpathcurveto{\pgfqpoint{2.561937in}{1.193642in}}{\pgfqpoint{2.569837in}{1.190370in}}{\pgfqpoint{2.578073in}{1.190370in}}%
\pgfpathclose%
\pgfusepath{stroke,fill}%
\end{pgfscope}%
\begin{pgfscope}%
\pgfpathrectangle{\pgfqpoint{0.100000in}{0.212622in}}{\pgfqpoint{3.696000in}{3.696000in}}%
\pgfusepath{clip}%
\pgfsetbuttcap%
\pgfsetroundjoin%
\definecolor{currentfill}{rgb}{0.121569,0.466667,0.705882}%
\pgfsetfillcolor{currentfill}%
\pgfsetfillopacity{0.953746}%
\pgfsetlinewidth{1.003750pt}%
\definecolor{currentstroke}{rgb}{0.121569,0.466667,0.705882}%
\pgfsetstrokecolor{currentstroke}%
\pgfsetstrokeopacity{0.953746}%
\pgfsetdash{}{0pt}%
\pgfpathmoveto{\pgfqpoint{2.079211in}{2.400773in}}%
\pgfpathcurveto{\pgfqpoint{2.087447in}{2.400773in}}{\pgfqpoint{2.095347in}{2.404045in}}{\pgfqpoint{2.101171in}{2.409869in}}%
\pgfpathcurveto{\pgfqpoint{2.106995in}{2.415693in}}{\pgfqpoint{2.110267in}{2.423593in}}{\pgfqpoint{2.110267in}{2.431829in}}%
\pgfpathcurveto{\pgfqpoint{2.110267in}{2.440065in}}{\pgfqpoint{2.106995in}{2.447965in}}{\pgfqpoint{2.101171in}{2.453789in}}%
\pgfpathcurveto{\pgfqpoint{2.095347in}{2.459613in}}{\pgfqpoint{2.087447in}{2.462886in}}{\pgfqpoint{2.079211in}{2.462886in}}%
\pgfpathcurveto{\pgfqpoint{2.070974in}{2.462886in}}{\pgfqpoint{2.063074in}{2.459613in}}{\pgfqpoint{2.057250in}{2.453789in}}%
\pgfpathcurveto{\pgfqpoint{2.051426in}{2.447965in}}{\pgfqpoint{2.048154in}{2.440065in}}{\pgfqpoint{2.048154in}{2.431829in}}%
\pgfpathcurveto{\pgfqpoint{2.048154in}{2.423593in}}{\pgfqpoint{2.051426in}{2.415693in}}{\pgfqpoint{2.057250in}{2.409869in}}%
\pgfpathcurveto{\pgfqpoint{2.063074in}{2.404045in}}{\pgfqpoint{2.070974in}{2.400773in}}{\pgfqpoint{2.079211in}{2.400773in}}%
\pgfpathclose%
\pgfusepath{stroke,fill}%
\end{pgfscope}%
\begin{pgfscope}%
\pgfpathrectangle{\pgfqpoint{0.100000in}{0.212622in}}{\pgfqpoint{3.696000in}{3.696000in}}%
\pgfusepath{clip}%
\pgfsetbuttcap%
\pgfsetroundjoin%
\definecolor{currentfill}{rgb}{0.121569,0.466667,0.705882}%
\pgfsetfillcolor{currentfill}%
\pgfsetfillopacity{0.953841}%
\pgfsetlinewidth{1.003750pt}%
\definecolor{currentstroke}{rgb}{0.121569,0.466667,0.705882}%
\pgfsetstrokecolor{currentstroke}%
\pgfsetstrokeopacity{0.953841}%
\pgfsetdash{}{0pt}%
\pgfpathmoveto{\pgfqpoint{1.985212in}{2.424410in}}%
\pgfpathcurveto{\pgfqpoint{1.993448in}{2.424410in}}{\pgfqpoint{2.001348in}{2.427682in}}{\pgfqpoint{2.007172in}{2.433506in}}%
\pgfpathcurveto{\pgfqpoint{2.012996in}{2.439330in}}{\pgfqpoint{2.016268in}{2.447230in}}{\pgfqpoint{2.016268in}{2.455466in}}%
\pgfpathcurveto{\pgfqpoint{2.016268in}{2.463703in}}{\pgfqpoint{2.012996in}{2.471603in}}{\pgfqpoint{2.007172in}{2.477427in}}%
\pgfpathcurveto{\pgfqpoint{2.001348in}{2.483251in}}{\pgfqpoint{1.993448in}{2.486523in}}{\pgfqpoint{1.985212in}{2.486523in}}%
\pgfpathcurveto{\pgfqpoint{1.976976in}{2.486523in}}{\pgfqpoint{1.969076in}{2.483251in}}{\pgfqpoint{1.963252in}{2.477427in}}%
\pgfpathcurveto{\pgfqpoint{1.957428in}{2.471603in}}{\pgfqpoint{1.954155in}{2.463703in}}{\pgfqpoint{1.954155in}{2.455466in}}%
\pgfpathcurveto{\pgfqpoint{1.954155in}{2.447230in}}{\pgfqpoint{1.957428in}{2.439330in}}{\pgfqpoint{1.963252in}{2.433506in}}%
\pgfpathcurveto{\pgfqpoint{1.969076in}{2.427682in}}{\pgfqpoint{1.976976in}{2.424410in}}{\pgfqpoint{1.985212in}{2.424410in}}%
\pgfpathclose%
\pgfusepath{stroke,fill}%
\end{pgfscope}%
\begin{pgfscope}%
\pgfpathrectangle{\pgfqpoint{0.100000in}{0.212622in}}{\pgfqpoint{3.696000in}{3.696000in}}%
\pgfusepath{clip}%
\pgfsetbuttcap%
\pgfsetroundjoin%
\definecolor{currentfill}{rgb}{0.121569,0.466667,0.705882}%
\pgfsetfillcolor{currentfill}%
\pgfsetfillopacity{0.953954}%
\pgfsetlinewidth{1.003750pt}%
\definecolor{currentstroke}{rgb}{0.121569,0.466667,0.705882}%
\pgfsetstrokecolor{currentstroke}%
\pgfsetstrokeopacity{0.953954}%
\pgfsetdash{}{0pt}%
\pgfpathmoveto{\pgfqpoint{2.078754in}{2.400330in}}%
\pgfpathcurveto{\pgfqpoint{2.086991in}{2.400330in}}{\pgfqpoint{2.094891in}{2.403602in}}{\pgfqpoint{2.100715in}{2.409426in}}%
\pgfpathcurveto{\pgfqpoint{2.106539in}{2.415250in}}{\pgfqpoint{2.109811in}{2.423150in}}{\pgfqpoint{2.109811in}{2.431387in}}%
\pgfpathcurveto{\pgfqpoint{2.109811in}{2.439623in}}{\pgfqpoint{2.106539in}{2.447523in}}{\pgfqpoint{2.100715in}{2.453347in}}%
\pgfpathcurveto{\pgfqpoint{2.094891in}{2.459171in}}{\pgfqpoint{2.086991in}{2.462443in}}{\pgfqpoint{2.078754in}{2.462443in}}%
\pgfpathcurveto{\pgfqpoint{2.070518in}{2.462443in}}{\pgfqpoint{2.062618in}{2.459171in}}{\pgfqpoint{2.056794in}{2.453347in}}%
\pgfpathcurveto{\pgfqpoint{2.050970in}{2.447523in}}{\pgfqpoint{2.047698in}{2.439623in}}{\pgfqpoint{2.047698in}{2.431387in}}%
\pgfpathcurveto{\pgfqpoint{2.047698in}{2.423150in}}{\pgfqpoint{2.050970in}{2.415250in}}{\pgfqpoint{2.056794in}{2.409426in}}%
\pgfpathcurveto{\pgfqpoint{2.062618in}{2.403602in}}{\pgfqpoint{2.070518in}{2.400330in}}{\pgfqpoint{2.078754in}{2.400330in}}%
\pgfpathclose%
\pgfusepath{stroke,fill}%
\end{pgfscope}%
\begin{pgfscope}%
\pgfpathrectangle{\pgfqpoint{0.100000in}{0.212622in}}{\pgfqpoint{3.696000in}{3.696000in}}%
\pgfusepath{clip}%
\pgfsetbuttcap%
\pgfsetroundjoin%
\definecolor{currentfill}{rgb}{0.121569,0.466667,0.705882}%
\pgfsetfillcolor{currentfill}%
\pgfsetfillopacity{0.954346}%
\pgfsetlinewidth{1.003750pt}%
\definecolor{currentstroke}{rgb}{0.121569,0.466667,0.705882}%
\pgfsetstrokecolor{currentstroke}%
\pgfsetstrokeopacity{0.954346}%
\pgfsetdash{}{0pt}%
\pgfpathmoveto{\pgfqpoint{2.077922in}{2.399613in}}%
\pgfpathcurveto{\pgfqpoint{2.086159in}{2.399613in}}{\pgfqpoint{2.094059in}{2.402886in}}{\pgfqpoint{2.099883in}{2.408709in}}%
\pgfpathcurveto{\pgfqpoint{2.105707in}{2.414533in}}{\pgfqpoint{2.108979in}{2.422433in}}{\pgfqpoint{2.108979in}{2.430670in}}%
\pgfpathcurveto{\pgfqpoint{2.108979in}{2.438906in}}{\pgfqpoint{2.105707in}{2.446806in}}{\pgfqpoint{2.099883in}{2.452630in}}%
\pgfpathcurveto{\pgfqpoint{2.094059in}{2.458454in}}{\pgfqpoint{2.086159in}{2.461726in}}{\pgfqpoint{2.077922in}{2.461726in}}%
\pgfpathcurveto{\pgfqpoint{2.069686in}{2.461726in}}{\pgfqpoint{2.061786in}{2.458454in}}{\pgfqpoint{2.055962in}{2.452630in}}%
\pgfpathcurveto{\pgfqpoint{2.050138in}{2.446806in}}{\pgfqpoint{2.046866in}{2.438906in}}{\pgfqpoint{2.046866in}{2.430670in}}%
\pgfpathcurveto{\pgfqpoint{2.046866in}{2.422433in}}{\pgfqpoint{2.050138in}{2.414533in}}{\pgfqpoint{2.055962in}{2.408709in}}%
\pgfpathcurveto{\pgfqpoint{2.061786in}{2.402886in}}{\pgfqpoint{2.069686in}{2.399613in}}{\pgfqpoint{2.077922in}{2.399613in}}%
\pgfpathclose%
\pgfusepath{stroke,fill}%
\end{pgfscope}%
\begin{pgfscope}%
\pgfpathrectangle{\pgfqpoint{0.100000in}{0.212622in}}{\pgfqpoint{3.696000in}{3.696000in}}%
\pgfusepath{clip}%
\pgfsetbuttcap%
\pgfsetroundjoin%
\definecolor{currentfill}{rgb}{0.121569,0.466667,0.705882}%
\pgfsetfillcolor{currentfill}%
\pgfsetfillopacity{0.954391}%
\pgfsetlinewidth{1.003750pt}%
\definecolor{currentstroke}{rgb}{0.121569,0.466667,0.705882}%
\pgfsetstrokecolor{currentstroke}%
\pgfsetstrokeopacity{0.954391}%
\pgfsetdash{}{0pt}%
\pgfpathmoveto{\pgfqpoint{1.991646in}{2.422272in}}%
\pgfpathcurveto{\pgfqpoint{1.999882in}{2.422272in}}{\pgfqpoint{2.007782in}{2.425544in}}{\pgfqpoint{2.013606in}{2.431368in}}%
\pgfpathcurveto{\pgfqpoint{2.019430in}{2.437192in}}{\pgfqpoint{2.022703in}{2.445092in}}{\pgfqpoint{2.022703in}{2.453328in}}%
\pgfpathcurveto{\pgfqpoint{2.022703in}{2.461565in}}{\pgfqpoint{2.019430in}{2.469465in}}{\pgfqpoint{2.013606in}{2.475289in}}%
\pgfpathcurveto{\pgfqpoint{2.007782in}{2.481113in}}{\pgfqpoint{1.999882in}{2.484385in}}{\pgfqpoint{1.991646in}{2.484385in}}%
\pgfpathcurveto{\pgfqpoint{1.983410in}{2.484385in}}{\pgfqpoint{1.975510in}{2.481113in}}{\pgfqpoint{1.969686in}{2.475289in}}%
\pgfpathcurveto{\pgfqpoint{1.963862in}{2.469465in}}{\pgfqpoint{1.960590in}{2.461565in}}{\pgfqpoint{1.960590in}{2.453328in}}%
\pgfpathcurveto{\pgfqpoint{1.960590in}{2.445092in}}{\pgfqpoint{1.963862in}{2.437192in}}{\pgfqpoint{1.969686in}{2.431368in}}%
\pgfpathcurveto{\pgfqpoint{1.975510in}{2.425544in}}{\pgfqpoint{1.983410in}{2.422272in}}{\pgfqpoint{1.991646in}{2.422272in}}%
\pgfpathclose%
\pgfusepath{stroke,fill}%
\end{pgfscope}%
\begin{pgfscope}%
\pgfpathrectangle{\pgfqpoint{0.100000in}{0.212622in}}{\pgfqpoint{3.696000in}{3.696000in}}%
\pgfusepath{clip}%
\pgfsetbuttcap%
\pgfsetroundjoin%
\definecolor{currentfill}{rgb}{0.121569,0.466667,0.705882}%
\pgfsetfillcolor{currentfill}%
\pgfsetfillopacity{0.954537}%
\pgfsetlinewidth{1.003750pt}%
\definecolor{currentstroke}{rgb}{0.121569,0.466667,0.705882}%
\pgfsetstrokecolor{currentstroke}%
\pgfsetstrokeopacity{0.954537}%
\pgfsetdash{}{0pt}%
\pgfpathmoveto{\pgfqpoint{1.417442in}{1.779709in}}%
\pgfpathcurveto{\pgfqpoint{1.425678in}{1.779709in}}{\pgfqpoint{1.433578in}{1.782981in}}{\pgfqpoint{1.439402in}{1.788805in}}%
\pgfpathcurveto{\pgfqpoint{1.445226in}{1.794629in}}{\pgfqpoint{1.448499in}{1.802529in}}{\pgfqpoint{1.448499in}{1.810765in}}%
\pgfpathcurveto{\pgfqpoint{1.448499in}{1.819001in}}{\pgfqpoint{1.445226in}{1.826902in}}{\pgfqpoint{1.439402in}{1.832725in}}%
\pgfpathcurveto{\pgfqpoint{1.433578in}{1.838549in}}{\pgfqpoint{1.425678in}{1.841822in}}{\pgfqpoint{1.417442in}{1.841822in}}%
\pgfpathcurveto{\pgfqpoint{1.409206in}{1.841822in}}{\pgfqpoint{1.401306in}{1.838549in}}{\pgfqpoint{1.395482in}{1.832725in}}%
\pgfpathcurveto{\pgfqpoint{1.389658in}{1.826902in}}{\pgfqpoint{1.386386in}{1.819001in}}{\pgfqpoint{1.386386in}{1.810765in}}%
\pgfpathcurveto{\pgfqpoint{1.386386in}{1.802529in}}{\pgfqpoint{1.389658in}{1.794629in}}{\pgfqpoint{1.395482in}{1.788805in}}%
\pgfpathcurveto{\pgfqpoint{1.401306in}{1.782981in}}{\pgfqpoint{1.409206in}{1.779709in}}{\pgfqpoint{1.417442in}{1.779709in}}%
\pgfpathclose%
\pgfusepath{stroke,fill}%
\end{pgfscope}%
\begin{pgfscope}%
\pgfpathrectangle{\pgfqpoint{0.100000in}{0.212622in}}{\pgfqpoint{3.696000in}{3.696000in}}%
\pgfusepath{clip}%
\pgfsetbuttcap%
\pgfsetroundjoin%
\definecolor{currentfill}{rgb}{0.121569,0.466667,0.705882}%
\pgfsetfillcolor{currentfill}%
\pgfsetfillopacity{0.954610}%
\pgfsetlinewidth{1.003750pt}%
\definecolor{currentstroke}{rgb}{0.121569,0.466667,0.705882}%
\pgfsetstrokecolor{currentstroke}%
\pgfsetstrokeopacity{0.954610}%
\pgfsetdash{}{0pt}%
\pgfpathmoveto{\pgfqpoint{2.077323in}{2.399078in}}%
\pgfpathcurveto{\pgfqpoint{2.085559in}{2.399078in}}{\pgfqpoint{2.093459in}{2.402351in}}{\pgfqpoint{2.099283in}{2.408175in}}%
\pgfpathcurveto{\pgfqpoint{2.105107in}{2.413999in}}{\pgfqpoint{2.108379in}{2.421899in}}{\pgfqpoint{2.108379in}{2.430135in}}%
\pgfpathcurveto{\pgfqpoint{2.108379in}{2.438371in}}{\pgfqpoint{2.105107in}{2.446271in}}{\pgfqpoint{2.099283in}{2.452095in}}%
\pgfpathcurveto{\pgfqpoint{2.093459in}{2.457919in}}{\pgfqpoint{2.085559in}{2.461191in}}{\pgfqpoint{2.077323in}{2.461191in}}%
\pgfpathcurveto{\pgfqpoint{2.069087in}{2.461191in}}{\pgfqpoint{2.061187in}{2.457919in}}{\pgfqpoint{2.055363in}{2.452095in}}%
\pgfpathcurveto{\pgfqpoint{2.049539in}{2.446271in}}{\pgfqpoint{2.046266in}{2.438371in}}{\pgfqpoint{2.046266in}{2.430135in}}%
\pgfpathcurveto{\pgfqpoint{2.046266in}{2.421899in}}{\pgfqpoint{2.049539in}{2.413999in}}{\pgfqpoint{2.055363in}{2.408175in}}%
\pgfpathcurveto{\pgfqpoint{2.061187in}{2.402351in}}{\pgfqpoint{2.069087in}{2.399078in}}{\pgfqpoint{2.077323in}{2.399078in}}%
\pgfpathclose%
\pgfusepath{stroke,fill}%
\end{pgfscope}%
\begin{pgfscope}%
\pgfpathrectangle{\pgfqpoint{0.100000in}{0.212622in}}{\pgfqpoint{3.696000in}{3.696000in}}%
\pgfusepath{clip}%
\pgfsetbuttcap%
\pgfsetroundjoin%
\definecolor{currentfill}{rgb}{0.121569,0.466667,0.705882}%
\pgfsetfillcolor{currentfill}%
\pgfsetfillopacity{0.954716}%
\pgfsetlinewidth{1.003750pt}%
\definecolor{currentstroke}{rgb}{0.121569,0.466667,0.705882}%
\pgfsetstrokecolor{currentstroke}%
\pgfsetstrokeopacity{0.954716}%
\pgfsetdash{}{0pt}%
\pgfpathmoveto{\pgfqpoint{2.077087in}{2.398889in}}%
\pgfpathcurveto{\pgfqpoint{2.085324in}{2.398889in}}{\pgfqpoint{2.093224in}{2.402162in}}{\pgfqpoint{2.099048in}{2.407986in}}%
\pgfpathcurveto{\pgfqpoint{2.104871in}{2.413810in}}{\pgfqpoint{2.108144in}{2.421710in}}{\pgfqpoint{2.108144in}{2.429946in}}%
\pgfpathcurveto{\pgfqpoint{2.108144in}{2.438182in}}{\pgfqpoint{2.104871in}{2.446082in}}{\pgfqpoint{2.099048in}{2.451906in}}%
\pgfpathcurveto{\pgfqpoint{2.093224in}{2.457730in}}{\pgfqpoint{2.085324in}{2.461002in}}{\pgfqpoint{2.077087in}{2.461002in}}%
\pgfpathcurveto{\pgfqpoint{2.068851in}{2.461002in}}{\pgfqpoint{2.060951in}{2.457730in}}{\pgfqpoint{2.055127in}{2.451906in}}%
\pgfpathcurveto{\pgfqpoint{2.049303in}{2.446082in}}{\pgfqpoint{2.046031in}{2.438182in}}{\pgfqpoint{2.046031in}{2.429946in}}%
\pgfpathcurveto{\pgfqpoint{2.046031in}{2.421710in}}{\pgfqpoint{2.049303in}{2.413810in}}{\pgfqpoint{2.055127in}{2.407986in}}%
\pgfpathcurveto{\pgfqpoint{2.060951in}{2.402162in}}{\pgfqpoint{2.068851in}{2.398889in}}{\pgfqpoint{2.077087in}{2.398889in}}%
\pgfpathclose%
\pgfusepath{stroke,fill}%
\end{pgfscope}%
\begin{pgfscope}%
\pgfpathrectangle{\pgfqpoint{0.100000in}{0.212622in}}{\pgfqpoint{3.696000in}{3.696000in}}%
\pgfusepath{clip}%
\pgfsetbuttcap%
\pgfsetroundjoin%
\definecolor{currentfill}{rgb}{0.121569,0.466667,0.705882}%
\pgfsetfillcolor{currentfill}%
\pgfsetfillopacity{0.954721}%
\pgfsetlinewidth{1.003750pt}%
\definecolor{currentstroke}{rgb}{0.121569,0.466667,0.705882}%
\pgfsetstrokecolor{currentstroke}%
\pgfsetstrokeopacity{0.954721}%
\pgfsetdash{}{0pt}%
\pgfpathmoveto{\pgfqpoint{2.077074in}{2.398880in}}%
\pgfpathcurveto{\pgfqpoint{2.085311in}{2.398880in}}{\pgfqpoint{2.093211in}{2.402152in}}{\pgfqpoint{2.099035in}{2.407976in}}%
\pgfpathcurveto{\pgfqpoint{2.104859in}{2.413800in}}{\pgfqpoint{2.108131in}{2.421700in}}{\pgfqpoint{2.108131in}{2.429936in}}%
\pgfpathcurveto{\pgfqpoint{2.108131in}{2.438173in}}{\pgfqpoint{2.104859in}{2.446073in}}{\pgfqpoint{2.099035in}{2.451896in}}%
\pgfpathcurveto{\pgfqpoint{2.093211in}{2.457720in}}{\pgfqpoint{2.085311in}{2.460993in}}{\pgfqpoint{2.077074in}{2.460993in}}%
\pgfpathcurveto{\pgfqpoint{2.068838in}{2.460993in}}{\pgfqpoint{2.060938in}{2.457720in}}{\pgfqpoint{2.055114in}{2.451896in}}%
\pgfpathcurveto{\pgfqpoint{2.049290in}{2.446073in}}{\pgfqpoint{2.046018in}{2.438173in}}{\pgfqpoint{2.046018in}{2.429936in}}%
\pgfpathcurveto{\pgfqpoint{2.046018in}{2.421700in}}{\pgfqpoint{2.049290in}{2.413800in}}{\pgfqpoint{2.055114in}{2.407976in}}%
\pgfpathcurveto{\pgfqpoint{2.060938in}{2.402152in}}{\pgfqpoint{2.068838in}{2.398880in}}{\pgfqpoint{2.077074in}{2.398880in}}%
\pgfpathclose%
\pgfusepath{stroke,fill}%
\end{pgfscope}%
\begin{pgfscope}%
\pgfpathrectangle{\pgfqpoint{0.100000in}{0.212622in}}{\pgfqpoint{3.696000in}{3.696000in}}%
\pgfusepath{clip}%
\pgfsetbuttcap%
\pgfsetroundjoin%
\definecolor{currentfill}{rgb}{0.121569,0.466667,0.705882}%
\pgfsetfillcolor{currentfill}%
\pgfsetfillopacity{0.954732}%
\pgfsetlinewidth{1.003750pt}%
\definecolor{currentstroke}{rgb}{0.121569,0.466667,0.705882}%
\pgfsetstrokecolor{currentstroke}%
\pgfsetstrokeopacity{0.954732}%
\pgfsetdash{}{0pt}%
\pgfpathmoveto{\pgfqpoint{2.077051in}{2.398860in}}%
\pgfpathcurveto{\pgfqpoint{2.085288in}{2.398860in}}{\pgfqpoint{2.093188in}{2.402133in}}{\pgfqpoint{2.099012in}{2.407957in}}%
\pgfpathcurveto{\pgfqpoint{2.104836in}{2.413781in}}{\pgfqpoint{2.108108in}{2.421681in}}{\pgfqpoint{2.108108in}{2.429917in}}%
\pgfpathcurveto{\pgfqpoint{2.108108in}{2.438153in}}{\pgfqpoint{2.104836in}{2.446053in}}{\pgfqpoint{2.099012in}{2.451877in}}%
\pgfpathcurveto{\pgfqpoint{2.093188in}{2.457701in}}{\pgfqpoint{2.085288in}{2.460973in}}{\pgfqpoint{2.077051in}{2.460973in}}%
\pgfpathcurveto{\pgfqpoint{2.068815in}{2.460973in}}{\pgfqpoint{2.060915in}{2.457701in}}{\pgfqpoint{2.055091in}{2.451877in}}%
\pgfpathcurveto{\pgfqpoint{2.049267in}{2.446053in}}{\pgfqpoint{2.045995in}{2.438153in}}{\pgfqpoint{2.045995in}{2.429917in}}%
\pgfpathcurveto{\pgfqpoint{2.045995in}{2.421681in}}{\pgfqpoint{2.049267in}{2.413781in}}{\pgfqpoint{2.055091in}{2.407957in}}%
\pgfpathcurveto{\pgfqpoint{2.060915in}{2.402133in}}{\pgfqpoint{2.068815in}{2.398860in}}{\pgfqpoint{2.077051in}{2.398860in}}%
\pgfpathclose%
\pgfusepath{stroke,fill}%
\end{pgfscope}%
\begin{pgfscope}%
\pgfpathrectangle{\pgfqpoint{0.100000in}{0.212622in}}{\pgfqpoint{3.696000in}{3.696000in}}%
\pgfusepath{clip}%
\pgfsetbuttcap%
\pgfsetroundjoin%
\definecolor{currentfill}{rgb}{0.121569,0.466667,0.705882}%
\pgfsetfillcolor{currentfill}%
\pgfsetfillopacity{0.954750}%
\pgfsetlinewidth{1.003750pt}%
\definecolor{currentstroke}{rgb}{0.121569,0.466667,0.705882}%
\pgfsetstrokecolor{currentstroke}%
\pgfsetstrokeopacity{0.954750}%
\pgfsetdash{}{0pt}%
\pgfpathmoveto{\pgfqpoint{2.077010in}{2.398824in}}%
\pgfpathcurveto{\pgfqpoint{2.085246in}{2.398824in}}{\pgfqpoint{2.093146in}{2.402097in}}{\pgfqpoint{2.098970in}{2.407921in}}%
\pgfpathcurveto{\pgfqpoint{2.104794in}{2.413744in}}{\pgfqpoint{2.108066in}{2.421645in}}{\pgfqpoint{2.108066in}{2.429881in}}%
\pgfpathcurveto{\pgfqpoint{2.108066in}{2.438117in}}{\pgfqpoint{2.104794in}{2.446017in}}{\pgfqpoint{2.098970in}{2.451841in}}%
\pgfpathcurveto{\pgfqpoint{2.093146in}{2.457665in}}{\pgfqpoint{2.085246in}{2.460937in}}{\pgfqpoint{2.077010in}{2.460937in}}%
\pgfpathcurveto{\pgfqpoint{2.068773in}{2.460937in}}{\pgfqpoint{2.060873in}{2.457665in}}{\pgfqpoint{2.055049in}{2.451841in}}%
\pgfpathcurveto{\pgfqpoint{2.049225in}{2.446017in}}{\pgfqpoint{2.045953in}{2.438117in}}{\pgfqpoint{2.045953in}{2.429881in}}%
\pgfpathcurveto{\pgfqpoint{2.045953in}{2.421645in}}{\pgfqpoint{2.049225in}{2.413744in}}{\pgfqpoint{2.055049in}{2.407921in}}%
\pgfpathcurveto{\pgfqpoint{2.060873in}{2.402097in}}{\pgfqpoint{2.068773in}{2.398824in}}{\pgfqpoint{2.077010in}{2.398824in}}%
\pgfpathclose%
\pgfusepath{stroke,fill}%
\end{pgfscope}%
\begin{pgfscope}%
\pgfpathrectangle{\pgfqpoint{0.100000in}{0.212622in}}{\pgfqpoint{3.696000in}{3.696000in}}%
\pgfusepath{clip}%
\pgfsetbuttcap%
\pgfsetroundjoin%
\definecolor{currentfill}{rgb}{0.121569,0.466667,0.705882}%
\pgfsetfillcolor{currentfill}%
\pgfsetfillopacity{0.954783}%
\pgfsetlinewidth{1.003750pt}%
\definecolor{currentstroke}{rgb}{0.121569,0.466667,0.705882}%
\pgfsetstrokecolor{currentstroke}%
\pgfsetstrokeopacity{0.954783}%
\pgfsetdash{}{0pt}%
\pgfpathmoveto{\pgfqpoint{2.076935in}{2.398755in}}%
\pgfpathcurveto{\pgfqpoint{2.085171in}{2.398755in}}{\pgfqpoint{2.093071in}{2.402027in}}{\pgfqpoint{2.098895in}{2.407851in}}%
\pgfpathcurveto{\pgfqpoint{2.104719in}{2.413675in}}{\pgfqpoint{2.107991in}{2.421575in}}{\pgfqpoint{2.107991in}{2.429811in}}%
\pgfpathcurveto{\pgfqpoint{2.107991in}{2.438048in}}{\pgfqpoint{2.104719in}{2.445948in}}{\pgfqpoint{2.098895in}{2.451772in}}%
\pgfpathcurveto{\pgfqpoint{2.093071in}{2.457596in}}{\pgfqpoint{2.085171in}{2.460868in}}{\pgfqpoint{2.076935in}{2.460868in}}%
\pgfpathcurveto{\pgfqpoint{2.068698in}{2.460868in}}{\pgfqpoint{2.060798in}{2.457596in}}{\pgfqpoint{2.054974in}{2.451772in}}%
\pgfpathcurveto{\pgfqpoint{2.049150in}{2.445948in}}{\pgfqpoint{2.045878in}{2.438048in}}{\pgfqpoint{2.045878in}{2.429811in}}%
\pgfpathcurveto{\pgfqpoint{2.045878in}{2.421575in}}{\pgfqpoint{2.049150in}{2.413675in}}{\pgfqpoint{2.054974in}{2.407851in}}%
\pgfpathcurveto{\pgfqpoint{2.060798in}{2.402027in}}{\pgfqpoint{2.068698in}{2.398755in}}{\pgfqpoint{2.076935in}{2.398755in}}%
\pgfpathclose%
\pgfusepath{stroke,fill}%
\end{pgfscope}%
\begin{pgfscope}%
\pgfpathrectangle{\pgfqpoint{0.100000in}{0.212622in}}{\pgfqpoint{3.696000in}{3.696000in}}%
\pgfusepath{clip}%
\pgfsetbuttcap%
\pgfsetroundjoin%
\definecolor{currentfill}{rgb}{0.121569,0.466667,0.705882}%
\pgfsetfillcolor{currentfill}%
\pgfsetfillopacity{0.954787}%
\pgfsetlinewidth{1.003750pt}%
\definecolor{currentstroke}{rgb}{0.121569,0.466667,0.705882}%
\pgfsetstrokecolor{currentstroke}%
\pgfsetstrokeopacity{0.954787}%
\pgfsetdash{}{0pt}%
\pgfpathmoveto{\pgfqpoint{1.995016in}{2.420992in}}%
\pgfpathcurveto{\pgfqpoint{2.003252in}{2.420992in}}{\pgfqpoint{2.011152in}{2.424264in}}{\pgfqpoint{2.016976in}{2.430088in}}%
\pgfpathcurveto{\pgfqpoint{2.022800in}{2.435912in}}{\pgfqpoint{2.026072in}{2.443812in}}{\pgfqpoint{2.026072in}{2.452048in}}%
\pgfpathcurveto{\pgfqpoint{2.026072in}{2.460285in}}{\pgfqpoint{2.022800in}{2.468185in}}{\pgfqpoint{2.016976in}{2.474009in}}%
\pgfpathcurveto{\pgfqpoint{2.011152in}{2.479833in}}{\pgfqpoint{2.003252in}{2.483105in}}{\pgfqpoint{1.995016in}{2.483105in}}%
\pgfpathcurveto{\pgfqpoint{1.986780in}{2.483105in}}{\pgfqpoint{1.978880in}{2.479833in}}{\pgfqpoint{1.973056in}{2.474009in}}%
\pgfpathcurveto{\pgfqpoint{1.967232in}{2.468185in}}{\pgfqpoint{1.963959in}{2.460285in}}{\pgfqpoint{1.963959in}{2.452048in}}%
\pgfpathcurveto{\pgfqpoint{1.963959in}{2.443812in}}{\pgfqpoint{1.967232in}{2.435912in}}{\pgfqpoint{1.973056in}{2.430088in}}%
\pgfpathcurveto{\pgfqpoint{1.978880in}{2.424264in}}{\pgfqpoint{1.986780in}{2.420992in}}{\pgfqpoint{1.995016in}{2.420992in}}%
\pgfpathclose%
\pgfusepath{stroke,fill}%
\end{pgfscope}%
\begin{pgfscope}%
\pgfpathrectangle{\pgfqpoint{0.100000in}{0.212622in}}{\pgfqpoint{3.696000in}{3.696000in}}%
\pgfusepath{clip}%
\pgfsetbuttcap%
\pgfsetroundjoin%
\definecolor{currentfill}{rgb}{0.121569,0.466667,0.705882}%
\pgfsetfillcolor{currentfill}%
\pgfsetfillopacity{0.954810}%
\pgfsetlinewidth{1.003750pt}%
\definecolor{currentstroke}{rgb}{0.121569,0.466667,0.705882}%
\pgfsetstrokecolor{currentstroke}%
\pgfsetstrokeopacity{0.954810}%
\pgfsetdash{}{0pt}%
\pgfpathmoveto{\pgfqpoint{2.575807in}{1.187249in}}%
\pgfpathcurveto{\pgfqpoint{2.584043in}{1.187249in}}{\pgfqpoint{2.591943in}{1.190521in}}{\pgfqpoint{2.597767in}{1.196345in}}%
\pgfpathcurveto{\pgfqpoint{2.603591in}{1.202169in}}{\pgfqpoint{2.606863in}{1.210069in}}{\pgfqpoint{2.606863in}{1.218305in}}%
\pgfpathcurveto{\pgfqpoint{2.606863in}{1.226541in}}{\pgfqpoint{2.603591in}{1.234441in}}{\pgfqpoint{2.597767in}{1.240265in}}%
\pgfpathcurveto{\pgfqpoint{2.591943in}{1.246089in}}{\pgfqpoint{2.584043in}{1.249362in}}{\pgfqpoint{2.575807in}{1.249362in}}%
\pgfpathcurveto{\pgfqpoint{2.567570in}{1.249362in}}{\pgfqpoint{2.559670in}{1.246089in}}{\pgfqpoint{2.553846in}{1.240265in}}%
\pgfpathcurveto{\pgfqpoint{2.548022in}{1.234441in}}{\pgfqpoint{2.544750in}{1.226541in}}{\pgfqpoint{2.544750in}{1.218305in}}%
\pgfpathcurveto{\pgfqpoint{2.544750in}{1.210069in}}{\pgfqpoint{2.548022in}{1.202169in}}{\pgfqpoint{2.553846in}{1.196345in}}%
\pgfpathcurveto{\pgfqpoint{2.559670in}{1.190521in}}{\pgfqpoint{2.567570in}{1.187249in}}{\pgfqpoint{2.575807in}{1.187249in}}%
\pgfpathclose%
\pgfusepath{stroke,fill}%
\end{pgfscope}%
\begin{pgfscope}%
\pgfpathrectangle{\pgfqpoint{0.100000in}{0.212622in}}{\pgfqpoint{3.696000in}{3.696000in}}%
\pgfusepath{clip}%
\pgfsetbuttcap%
\pgfsetroundjoin%
\definecolor{currentfill}{rgb}{0.121569,0.466667,0.705882}%
\pgfsetfillcolor{currentfill}%
\pgfsetfillopacity{0.954842}%
\pgfsetlinewidth{1.003750pt}%
\definecolor{currentstroke}{rgb}{0.121569,0.466667,0.705882}%
\pgfsetstrokecolor{currentstroke}%
\pgfsetstrokeopacity{0.954842}%
\pgfsetdash{}{0pt}%
\pgfpathmoveto{\pgfqpoint{2.076797in}{2.398625in}}%
\pgfpathcurveto{\pgfqpoint{2.085033in}{2.398625in}}{\pgfqpoint{2.092933in}{2.401897in}}{\pgfqpoint{2.098757in}{2.407721in}}%
\pgfpathcurveto{\pgfqpoint{2.104581in}{2.413545in}}{\pgfqpoint{2.107853in}{2.421445in}}{\pgfqpoint{2.107853in}{2.429681in}}%
\pgfpathcurveto{\pgfqpoint{2.107853in}{2.437917in}}{\pgfqpoint{2.104581in}{2.445817in}}{\pgfqpoint{2.098757in}{2.451641in}}%
\pgfpathcurveto{\pgfqpoint{2.092933in}{2.457465in}}{\pgfqpoint{2.085033in}{2.460738in}}{\pgfqpoint{2.076797in}{2.460738in}}%
\pgfpathcurveto{\pgfqpoint{2.068561in}{2.460738in}}{\pgfqpoint{2.060661in}{2.457465in}}{\pgfqpoint{2.054837in}{2.451641in}}%
\pgfpathcurveto{\pgfqpoint{2.049013in}{2.445817in}}{\pgfqpoint{2.045740in}{2.437917in}}{\pgfqpoint{2.045740in}{2.429681in}}%
\pgfpathcurveto{\pgfqpoint{2.045740in}{2.421445in}}{\pgfqpoint{2.049013in}{2.413545in}}{\pgfqpoint{2.054837in}{2.407721in}}%
\pgfpathcurveto{\pgfqpoint{2.060661in}{2.401897in}}{\pgfqpoint{2.068561in}{2.398625in}}{\pgfqpoint{2.076797in}{2.398625in}}%
\pgfpathclose%
\pgfusepath{stroke,fill}%
\end{pgfscope}%
\begin{pgfscope}%
\pgfpathrectangle{\pgfqpoint{0.100000in}{0.212622in}}{\pgfqpoint{3.696000in}{3.696000in}}%
\pgfusepath{clip}%
\pgfsetbuttcap%
\pgfsetroundjoin%
\definecolor{currentfill}{rgb}{0.121569,0.466667,0.705882}%
\pgfsetfillcolor{currentfill}%
\pgfsetfillopacity{0.954950}%
\pgfsetlinewidth{1.003750pt}%
\definecolor{currentstroke}{rgb}{0.121569,0.466667,0.705882}%
\pgfsetstrokecolor{currentstroke}%
\pgfsetstrokeopacity{0.954950}%
\pgfsetdash{}{0pt}%
\pgfpathmoveto{\pgfqpoint{2.076548in}{2.398390in}}%
\pgfpathcurveto{\pgfqpoint{2.084785in}{2.398390in}}{\pgfqpoint{2.092685in}{2.401663in}}{\pgfqpoint{2.098509in}{2.407486in}}%
\pgfpathcurveto{\pgfqpoint{2.104333in}{2.413310in}}{\pgfqpoint{2.107605in}{2.421210in}}{\pgfqpoint{2.107605in}{2.429447in}}%
\pgfpathcurveto{\pgfqpoint{2.107605in}{2.437683in}}{\pgfqpoint{2.104333in}{2.445583in}}{\pgfqpoint{2.098509in}{2.451407in}}%
\pgfpathcurveto{\pgfqpoint{2.092685in}{2.457231in}}{\pgfqpoint{2.084785in}{2.460503in}}{\pgfqpoint{2.076548in}{2.460503in}}%
\pgfpathcurveto{\pgfqpoint{2.068312in}{2.460503in}}{\pgfqpoint{2.060412in}{2.457231in}}{\pgfqpoint{2.054588in}{2.451407in}}%
\pgfpathcurveto{\pgfqpoint{2.048764in}{2.445583in}}{\pgfqpoint{2.045492in}{2.437683in}}{\pgfqpoint{2.045492in}{2.429447in}}%
\pgfpathcurveto{\pgfqpoint{2.045492in}{2.421210in}}{\pgfqpoint{2.048764in}{2.413310in}}{\pgfqpoint{2.054588in}{2.407486in}}%
\pgfpathcurveto{\pgfqpoint{2.060412in}{2.401663in}}{\pgfqpoint{2.068312in}{2.398390in}}{\pgfqpoint{2.076548in}{2.398390in}}%
\pgfpathclose%
\pgfusepath{stroke,fill}%
\end{pgfscope}%
\begin{pgfscope}%
\pgfpathrectangle{\pgfqpoint{0.100000in}{0.212622in}}{\pgfqpoint{3.696000in}{3.696000in}}%
\pgfusepath{clip}%
\pgfsetbuttcap%
\pgfsetroundjoin%
\definecolor{currentfill}{rgb}{0.121569,0.466667,0.705882}%
\pgfsetfillcolor{currentfill}%
\pgfsetfillopacity{0.954999}%
\pgfsetlinewidth{1.003750pt}%
\definecolor{currentstroke}{rgb}{0.121569,0.466667,0.705882}%
\pgfsetstrokecolor{currentstroke}%
\pgfsetstrokeopacity{0.954999}%
\pgfsetdash{}{0pt}%
\pgfpathmoveto{\pgfqpoint{1.996886in}{2.420314in}}%
\pgfpathcurveto{\pgfqpoint{2.005122in}{2.420314in}}{\pgfqpoint{2.013022in}{2.423587in}}{\pgfqpoint{2.018846in}{2.429411in}}%
\pgfpathcurveto{\pgfqpoint{2.024670in}{2.435235in}}{\pgfqpoint{2.027943in}{2.443135in}}{\pgfqpoint{2.027943in}{2.451371in}}%
\pgfpathcurveto{\pgfqpoint{2.027943in}{2.459607in}}{\pgfqpoint{2.024670in}{2.467507in}}{\pgfqpoint{2.018846in}{2.473331in}}%
\pgfpathcurveto{\pgfqpoint{2.013022in}{2.479155in}}{\pgfqpoint{2.005122in}{2.482427in}}{\pgfqpoint{1.996886in}{2.482427in}}%
\pgfpathcurveto{\pgfqpoint{1.988650in}{2.482427in}}{\pgfqpoint{1.980750in}{2.479155in}}{\pgfqpoint{1.974926in}{2.473331in}}%
\pgfpathcurveto{\pgfqpoint{1.969102in}{2.467507in}}{\pgfqpoint{1.965830in}{2.459607in}}{\pgfqpoint{1.965830in}{2.451371in}}%
\pgfpathcurveto{\pgfqpoint{1.965830in}{2.443135in}}{\pgfqpoint{1.969102in}{2.435235in}}{\pgfqpoint{1.974926in}{2.429411in}}%
\pgfpathcurveto{\pgfqpoint{1.980750in}{2.423587in}}{\pgfqpoint{1.988650in}{2.420314in}}{\pgfqpoint{1.996886in}{2.420314in}}%
\pgfpathclose%
\pgfusepath{stroke,fill}%
\end{pgfscope}%
\begin{pgfscope}%
\pgfpathrectangle{\pgfqpoint{0.100000in}{0.212622in}}{\pgfqpoint{3.696000in}{3.696000in}}%
\pgfusepath{clip}%
\pgfsetbuttcap%
\pgfsetroundjoin%
\definecolor{currentfill}{rgb}{0.121569,0.466667,0.705882}%
\pgfsetfillcolor{currentfill}%
\pgfsetfillopacity{0.955160}%
\pgfsetlinewidth{1.003750pt}%
\definecolor{currentstroke}{rgb}{0.121569,0.466667,0.705882}%
\pgfsetstrokecolor{currentstroke}%
\pgfsetstrokeopacity{0.955160}%
\pgfsetdash{}{0pt}%
\pgfpathmoveto{\pgfqpoint{2.076095in}{2.398050in}}%
\pgfpathcurveto{\pgfqpoint{2.084331in}{2.398050in}}{\pgfqpoint{2.092231in}{2.401322in}}{\pgfqpoint{2.098055in}{2.407146in}}%
\pgfpathcurveto{\pgfqpoint{2.103879in}{2.412970in}}{\pgfqpoint{2.107151in}{2.420870in}}{\pgfqpoint{2.107151in}{2.429106in}}%
\pgfpathcurveto{\pgfqpoint{2.107151in}{2.437343in}}{\pgfqpoint{2.103879in}{2.445243in}}{\pgfqpoint{2.098055in}{2.451067in}}%
\pgfpathcurveto{\pgfqpoint{2.092231in}{2.456891in}}{\pgfqpoint{2.084331in}{2.460163in}}{\pgfqpoint{2.076095in}{2.460163in}}%
\pgfpathcurveto{\pgfqpoint{2.067859in}{2.460163in}}{\pgfqpoint{2.059959in}{2.456891in}}{\pgfqpoint{2.054135in}{2.451067in}}%
\pgfpathcurveto{\pgfqpoint{2.048311in}{2.445243in}}{\pgfqpoint{2.045038in}{2.437343in}}{\pgfqpoint{2.045038in}{2.429106in}}%
\pgfpathcurveto{\pgfqpoint{2.045038in}{2.420870in}}{\pgfqpoint{2.048311in}{2.412970in}}{\pgfqpoint{2.054135in}{2.407146in}}%
\pgfpathcurveto{\pgfqpoint{2.059959in}{2.401322in}}{\pgfqpoint{2.067859in}{2.398050in}}{\pgfqpoint{2.076095in}{2.398050in}}%
\pgfpathclose%
\pgfusepath{stroke,fill}%
\end{pgfscope}%
\begin{pgfscope}%
\pgfpathrectangle{\pgfqpoint{0.100000in}{0.212622in}}{\pgfqpoint{3.696000in}{3.696000in}}%
\pgfusepath{clip}%
\pgfsetbuttcap%
\pgfsetroundjoin%
\definecolor{currentfill}{rgb}{0.121569,0.466667,0.705882}%
\pgfsetfillcolor{currentfill}%
\pgfsetfillopacity{0.955293}%
\pgfsetlinewidth{1.003750pt}%
\definecolor{currentstroke}{rgb}{0.121569,0.466667,0.705882}%
\pgfsetstrokecolor{currentstroke}%
\pgfsetstrokeopacity{0.955293}%
\pgfsetdash{}{0pt}%
\pgfpathmoveto{\pgfqpoint{1.999745in}{2.418745in}}%
\pgfpathcurveto{\pgfqpoint{2.007981in}{2.418745in}}{\pgfqpoint{2.015881in}{2.422017in}}{\pgfqpoint{2.021705in}{2.427841in}}%
\pgfpathcurveto{\pgfqpoint{2.027529in}{2.433665in}}{\pgfqpoint{2.030801in}{2.441565in}}{\pgfqpoint{2.030801in}{2.449802in}}%
\pgfpathcurveto{\pgfqpoint{2.030801in}{2.458038in}}{\pgfqpoint{2.027529in}{2.465938in}}{\pgfqpoint{2.021705in}{2.471762in}}%
\pgfpathcurveto{\pgfqpoint{2.015881in}{2.477586in}}{\pgfqpoint{2.007981in}{2.480858in}}{\pgfqpoint{1.999745in}{2.480858in}}%
\pgfpathcurveto{\pgfqpoint{1.991508in}{2.480858in}}{\pgfqpoint{1.983608in}{2.477586in}}{\pgfqpoint{1.977784in}{2.471762in}}%
\pgfpathcurveto{\pgfqpoint{1.971960in}{2.465938in}}{\pgfqpoint{1.968688in}{2.458038in}}{\pgfqpoint{1.968688in}{2.449802in}}%
\pgfpathcurveto{\pgfqpoint{1.968688in}{2.441565in}}{\pgfqpoint{1.971960in}{2.433665in}}{\pgfqpoint{1.977784in}{2.427841in}}%
\pgfpathcurveto{\pgfqpoint{1.983608in}{2.422017in}}{\pgfqpoint{1.991508in}{2.418745in}}{\pgfqpoint{1.999745in}{2.418745in}}%
\pgfpathclose%
\pgfusepath{stroke,fill}%
\end{pgfscope}%
\begin{pgfscope}%
\pgfpathrectangle{\pgfqpoint{0.100000in}{0.212622in}}{\pgfqpoint{3.696000in}{3.696000in}}%
\pgfusepath{clip}%
\pgfsetbuttcap%
\pgfsetroundjoin%
\definecolor{currentfill}{rgb}{0.121569,0.466667,0.705882}%
\pgfsetfillcolor{currentfill}%
\pgfsetfillopacity{0.955449}%
\pgfsetlinewidth{1.003750pt}%
\definecolor{currentstroke}{rgb}{0.121569,0.466667,0.705882}%
\pgfsetstrokecolor{currentstroke}%
\pgfsetstrokeopacity{0.955449}%
\pgfsetdash{}{0pt}%
\pgfpathmoveto{\pgfqpoint{2.001329in}{2.417889in}}%
\pgfpathcurveto{\pgfqpoint{2.009565in}{2.417889in}}{\pgfqpoint{2.017465in}{2.421161in}}{\pgfqpoint{2.023289in}{2.426985in}}%
\pgfpathcurveto{\pgfqpoint{2.029113in}{2.432809in}}{\pgfqpoint{2.032385in}{2.440709in}}{\pgfqpoint{2.032385in}{2.448945in}}%
\pgfpathcurveto{\pgfqpoint{2.032385in}{2.457181in}}{\pgfqpoint{2.029113in}{2.465081in}}{\pgfqpoint{2.023289in}{2.470905in}}%
\pgfpathcurveto{\pgfqpoint{2.017465in}{2.476729in}}{\pgfqpoint{2.009565in}{2.480002in}}{\pgfqpoint{2.001329in}{2.480002in}}%
\pgfpathcurveto{\pgfqpoint{1.993093in}{2.480002in}}{\pgfqpoint{1.985193in}{2.476729in}}{\pgfqpoint{1.979369in}{2.470905in}}%
\pgfpathcurveto{\pgfqpoint{1.973545in}{2.465081in}}{\pgfqpoint{1.970272in}{2.457181in}}{\pgfqpoint{1.970272in}{2.448945in}}%
\pgfpathcurveto{\pgfqpoint{1.970272in}{2.440709in}}{\pgfqpoint{1.973545in}{2.432809in}}{\pgfqpoint{1.979369in}{2.426985in}}%
\pgfpathcurveto{\pgfqpoint{1.985193in}{2.421161in}}{\pgfqpoint{1.993093in}{2.417889in}}{\pgfqpoint{2.001329in}{2.417889in}}%
\pgfpathclose%
\pgfusepath{stroke,fill}%
\end{pgfscope}%
\begin{pgfscope}%
\pgfpathrectangle{\pgfqpoint{0.100000in}{0.212622in}}{\pgfqpoint{3.696000in}{3.696000in}}%
\pgfusepath{clip}%
\pgfsetbuttcap%
\pgfsetroundjoin%
\definecolor{currentfill}{rgb}{0.121569,0.466667,0.705882}%
\pgfsetfillcolor{currentfill}%
\pgfsetfillopacity{0.955460}%
\pgfsetlinewidth{1.003750pt}%
\definecolor{currentstroke}{rgb}{0.121569,0.466667,0.705882}%
\pgfsetstrokecolor{currentstroke}%
\pgfsetstrokeopacity{0.955460}%
\pgfsetdash{}{0pt}%
\pgfpathmoveto{\pgfqpoint{1.432934in}{1.768805in}}%
\pgfpathcurveto{\pgfqpoint{1.441171in}{1.768805in}}{\pgfqpoint{1.449071in}{1.772077in}}{\pgfqpoint{1.454895in}{1.777901in}}%
\pgfpathcurveto{\pgfqpoint{1.460719in}{1.783725in}}{\pgfqpoint{1.463991in}{1.791625in}}{\pgfqpoint{1.463991in}{1.799862in}}%
\pgfpathcurveto{\pgfqpoint{1.463991in}{1.808098in}}{\pgfqpoint{1.460719in}{1.815998in}}{\pgfqpoint{1.454895in}{1.821822in}}%
\pgfpathcurveto{\pgfqpoint{1.449071in}{1.827646in}}{\pgfqpoint{1.441171in}{1.830918in}}{\pgfqpoint{1.432934in}{1.830918in}}%
\pgfpathcurveto{\pgfqpoint{1.424698in}{1.830918in}}{\pgfqpoint{1.416798in}{1.827646in}}{\pgfqpoint{1.410974in}{1.821822in}}%
\pgfpathcurveto{\pgfqpoint{1.405150in}{1.815998in}}{\pgfqpoint{1.401878in}{1.808098in}}{\pgfqpoint{1.401878in}{1.799862in}}%
\pgfpathcurveto{\pgfqpoint{1.401878in}{1.791625in}}{\pgfqpoint{1.405150in}{1.783725in}}{\pgfqpoint{1.410974in}{1.777901in}}%
\pgfpathcurveto{\pgfqpoint{1.416798in}{1.772077in}}{\pgfqpoint{1.424698in}{1.768805in}}{\pgfqpoint{1.432934in}{1.768805in}}%
\pgfpathclose%
\pgfusepath{stroke,fill}%
\end{pgfscope}%
\begin{pgfscope}%
\pgfpathrectangle{\pgfqpoint{0.100000in}{0.212622in}}{\pgfqpoint{3.696000in}{3.696000in}}%
\pgfusepath{clip}%
\pgfsetbuttcap%
\pgfsetroundjoin%
\definecolor{currentfill}{rgb}{0.121569,0.466667,0.705882}%
\pgfsetfillcolor{currentfill}%
\pgfsetfillopacity{0.955541}%
\pgfsetlinewidth{1.003750pt}%
\definecolor{currentstroke}{rgb}{0.121569,0.466667,0.705882}%
\pgfsetstrokecolor{currentstroke}%
\pgfsetstrokeopacity{0.955541}%
\pgfsetdash{}{0pt}%
\pgfpathmoveto{\pgfqpoint{2.075292in}{2.397389in}}%
\pgfpathcurveto{\pgfqpoint{2.083528in}{2.397389in}}{\pgfqpoint{2.091429in}{2.400662in}}{\pgfqpoint{2.097252in}{2.406486in}}%
\pgfpathcurveto{\pgfqpoint{2.103076in}{2.412309in}}{\pgfqpoint{2.106349in}{2.420209in}}{\pgfqpoint{2.106349in}{2.428446in}}%
\pgfpathcurveto{\pgfqpoint{2.106349in}{2.436682in}}{\pgfqpoint{2.103076in}{2.444582in}}{\pgfqpoint{2.097252in}{2.450406in}}%
\pgfpathcurveto{\pgfqpoint{2.091429in}{2.456230in}}{\pgfqpoint{2.083528in}{2.459502in}}{\pgfqpoint{2.075292in}{2.459502in}}%
\pgfpathcurveto{\pgfqpoint{2.067056in}{2.459502in}}{\pgfqpoint{2.059156in}{2.456230in}}{\pgfqpoint{2.053332in}{2.450406in}}%
\pgfpathcurveto{\pgfqpoint{2.047508in}{2.444582in}}{\pgfqpoint{2.044236in}{2.436682in}}{\pgfqpoint{2.044236in}{2.428446in}}%
\pgfpathcurveto{\pgfqpoint{2.044236in}{2.420209in}}{\pgfqpoint{2.047508in}{2.412309in}}{\pgfqpoint{2.053332in}{2.406486in}}%
\pgfpathcurveto{\pgfqpoint{2.059156in}{2.400662in}}{\pgfqpoint{2.067056in}{2.397389in}}{\pgfqpoint{2.075292in}{2.397389in}}%
\pgfpathclose%
\pgfusepath{stroke,fill}%
\end{pgfscope}%
\begin{pgfscope}%
\pgfpathrectangle{\pgfqpoint{0.100000in}{0.212622in}}{\pgfqpoint{3.696000in}{3.696000in}}%
\pgfusepath{clip}%
\pgfsetbuttcap%
\pgfsetroundjoin%
\definecolor{currentfill}{rgb}{0.121569,0.466667,0.705882}%
\pgfsetfillcolor{currentfill}%
\pgfsetfillopacity{0.955541}%
\pgfsetlinewidth{1.003750pt}%
\definecolor{currentstroke}{rgb}{0.121569,0.466667,0.705882}%
\pgfsetstrokecolor{currentstroke}%
\pgfsetstrokeopacity{0.955541}%
\pgfsetdash{}{0pt}%
\pgfpathmoveto{\pgfqpoint{2.075292in}{2.397389in}}%
\pgfpathcurveto{\pgfqpoint{2.083528in}{2.397389in}}{\pgfqpoint{2.091429in}{2.400662in}}{\pgfqpoint{2.097252in}{2.406486in}}%
\pgfpathcurveto{\pgfqpoint{2.103076in}{2.412309in}}{\pgfqpoint{2.106349in}{2.420209in}}{\pgfqpoint{2.106349in}{2.428446in}}%
\pgfpathcurveto{\pgfqpoint{2.106349in}{2.436682in}}{\pgfqpoint{2.103076in}{2.444582in}}{\pgfqpoint{2.097252in}{2.450406in}}%
\pgfpathcurveto{\pgfqpoint{2.091429in}{2.456230in}}{\pgfqpoint{2.083528in}{2.459502in}}{\pgfqpoint{2.075292in}{2.459502in}}%
\pgfpathcurveto{\pgfqpoint{2.067056in}{2.459502in}}{\pgfqpoint{2.059156in}{2.456230in}}{\pgfqpoint{2.053332in}{2.450406in}}%
\pgfpathcurveto{\pgfqpoint{2.047508in}{2.444582in}}{\pgfqpoint{2.044236in}{2.436682in}}{\pgfqpoint{2.044236in}{2.428446in}}%
\pgfpathcurveto{\pgfqpoint{2.044236in}{2.420209in}}{\pgfqpoint{2.047508in}{2.412309in}}{\pgfqpoint{2.053332in}{2.406486in}}%
\pgfpathcurveto{\pgfqpoint{2.059156in}{2.400662in}}{\pgfqpoint{2.067056in}{2.397389in}}{\pgfqpoint{2.075292in}{2.397389in}}%
\pgfpathclose%
\pgfusepath{stroke,fill}%
\end{pgfscope}%
\begin{pgfscope}%
\pgfpathrectangle{\pgfqpoint{0.100000in}{0.212622in}}{\pgfqpoint{3.696000in}{3.696000in}}%
\pgfusepath{clip}%
\pgfsetbuttcap%
\pgfsetroundjoin%
\definecolor{currentfill}{rgb}{0.121569,0.466667,0.705882}%
\pgfsetfillcolor{currentfill}%
\pgfsetfillopacity{0.955541}%
\pgfsetlinewidth{1.003750pt}%
\definecolor{currentstroke}{rgb}{0.121569,0.466667,0.705882}%
\pgfsetstrokecolor{currentstroke}%
\pgfsetstrokeopacity{0.955541}%
\pgfsetdash{}{0pt}%
\pgfpathmoveto{\pgfqpoint{2.075292in}{2.397389in}}%
\pgfpathcurveto{\pgfqpoint{2.083528in}{2.397389in}}{\pgfqpoint{2.091429in}{2.400662in}}{\pgfqpoint{2.097252in}{2.406486in}}%
\pgfpathcurveto{\pgfqpoint{2.103076in}{2.412309in}}{\pgfqpoint{2.106349in}{2.420209in}}{\pgfqpoint{2.106349in}{2.428446in}}%
\pgfpathcurveto{\pgfqpoint{2.106349in}{2.436682in}}{\pgfqpoint{2.103076in}{2.444582in}}{\pgfqpoint{2.097252in}{2.450406in}}%
\pgfpathcurveto{\pgfqpoint{2.091429in}{2.456230in}}{\pgfqpoint{2.083528in}{2.459502in}}{\pgfqpoint{2.075292in}{2.459502in}}%
\pgfpathcurveto{\pgfqpoint{2.067056in}{2.459502in}}{\pgfqpoint{2.059156in}{2.456230in}}{\pgfqpoint{2.053332in}{2.450406in}}%
\pgfpathcurveto{\pgfqpoint{2.047508in}{2.444582in}}{\pgfqpoint{2.044236in}{2.436682in}}{\pgfqpoint{2.044236in}{2.428446in}}%
\pgfpathcurveto{\pgfqpoint{2.044236in}{2.420209in}}{\pgfqpoint{2.047508in}{2.412309in}}{\pgfqpoint{2.053332in}{2.406486in}}%
\pgfpathcurveto{\pgfqpoint{2.059156in}{2.400662in}}{\pgfqpoint{2.067056in}{2.397389in}}{\pgfqpoint{2.075292in}{2.397389in}}%
\pgfpathclose%
\pgfusepath{stroke,fill}%
\end{pgfscope}%
\begin{pgfscope}%
\pgfpathrectangle{\pgfqpoint{0.100000in}{0.212622in}}{\pgfqpoint{3.696000in}{3.696000in}}%
\pgfusepath{clip}%
\pgfsetbuttcap%
\pgfsetroundjoin%
\definecolor{currentfill}{rgb}{0.121569,0.466667,0.705882}%
\pgfsetfillcolor{currentfill}%
\pgfsetfillopacity{0.955541}%
\pgfsetlinewidth{1.003750pt}%
\definecolor{currentstroke}{rgb}{0.121569,0.466667,0.705882}%
\pgfsetstrokecolor{currentstroke}%
\pgfsetstrokeopacity{0.955541}%
\pgfsetdash{}{0pt}%
\pgfpathmoveto{\pgfqpoint{2.075292in}{2.397389in}}%
\pgfpathcurveto{\pgfqpoint{2.083528in}{2.397389in}}{\pgfqpoint{2.091429in}{2.400662in}}{\pgfqpoint{2.097252in}{2.406486in}}%
\pgfpathcurveto{\pgfqpoint{2.103076in}{2.412309in}}{\pgfqpoint{2.106349in}{2.420209in}}{\pgfqpoint{2.106349in}{2.428446in}}%
\pgfpathcurveto{\pgfqpoint{2.106349in}{2.436682in}}{\pgfqpoint{2.103076in}{2.444582in}}{\pgfqpoint{2.097252in}{2.450406in}}%
\pgfpathcurveto{\pgfqpoint{2.091429in}{2.456230in}}{\pgfqpoint{2.083528in}{2.459502in}}{\pgfqpoint{2.075292in}{2.459502in}}%
\pgfpathcurveto{\pgfqpoint{2.067056in}{2.459502in}}{\pgfqpoint{2.059156in}{2.456230in}}{\pgfqpoint{2.053332in}{2.450406in}}%
\pgfpathcurveto{\pgfqpoint{2.047508in}{2.444582in}}{\pgfqpoint{2.044236in}{2.436682in}}{\pgfqpoint{2.044236in}{2.428446in}}%
\pgfpathcurveto{\pgfqpoint{2.044236in}{2.420209in}}{\pgfqpoint{2.047508in}{2.412309in}}{\pgfqpoint{2.053332in}{2.406486in}}%
\pgfpathcurveto{\pgfqpoint{2.059156in}{2.400662in}}{\pgfqpoint{2.067056in}{2.397389in}}{\pgfqpoint{2.075292in}{2.397389in}}%
\pgfpathclose%
\pgfusepath{stroke,fill}%
\end{pgfscope}%
\begin{pgfscope}%
\pgfpathrectangle{\pgfqpoint{0.100000in}{0.212622in}}{\pgfqpoint{3.696000in}{3.696000in}}%
\pgfusepath{clip}%
\pgfsetbuttcap%
\pgfsetroundjoin%
\definecolor{currentfill}{rgb}{0.121569,0.466667,0.705882}%
\pgfsetfillcolor{currentfill}%
\pgfsetfillopacity{0.955541}%
\pgfsetlinewidth{1.003750pt}%
\definecolor{currentstroke}{rgb}{0.121569,0.466667,0.705882}%
\pgfsetstrokecolor{currentstroke}%
\pgfsetstrokeopacity{0.955541}%
\pgfsetdash{}{0pt}%
\pgfpathmoveto{\pgfqpoint{2.075292in}{2.397389in}}%
\pgfpathcurveto{\pgfqpoint{2.083528in}{2.397389in}}{\pgfqpoint{2.091429in}{2.400662in}}{\pgfqpoint{2.097252in}{2.406486in}}%
\pgfpathcurveto{\pgfqpoint{2.103076in}{2.412309in}}{\pgfqpoint{2.106349in}{2.420209in}}{\pgfqpoint{2.106349in}{2.428446in}}%
\pgfpathcurveto{\pgfqpoint{2.106349in}{2.436682in}}{\pgfqpoint{2.103076in}{2.444582in}}{\pgfqpoint{2.097252in}{2.450406in}}%
\pgfpathcurveto{\pgfqpoint{2.091429in}{2.456230in}}{\pgfqpoint{2.083528in}{2.459502in}}{\pgfqpoint{2.075292in}{2.459502in}}%
\pgfpathcurveto{\pgfqpoint{2.067056in}{2.459502in}}{\pgfqpoint{2.059156in}{2.456230in}}{\pgfqpoint{2.053332in}{2.450406in}}%
\pgfpathcurveto{\pgfqpoint{2.047508in}{2.444582in}}{\pgfqpoint{2.044236in}{2.436682in}}{\pgfqpoint{2.044236in}{2.428446in}}%
\pgfpathcurveto{\pgfqpoint{2.044236in}{2.420209in}}{\pgfqpoint{2.047508in}{2.412309in}}{\pgfqpoint{2.053332in}{2.406486in}}%
\pgfpathcurveto{\pgfqpoint{2.059156in}{2.400662in}}{\pgfqpoint{2.067056in}{2.397389in}}{\pgfqpoint{2.075292in}{2.397389in}}%
\pgfpathclose%
\pgfusepath{stroke,fill}%
\end{pgfscope}%
\begin{pgfscope}%
\pgfpathrectangle{\pgfqpoint{0.100000in}{0.212622in}}{\pgfqpoint{3.696000in}{3.696000in}}%
\pgfusepath{clip}%
\pgfsetbuttcap%
\pgfsetroundjoin%
\definecolor{currentfill}{rgb}{0.121569,0.466667,0.705882}%
\pgfsetfillcolor{currentfill}%
\pgfsetfillopacity{0.955541}%
\pgfsetlinewidth{1.003750pt}%
\definecolor{currentstroke}{rgb}{0.121569,0.466667,0.705882}%
\pgfsetstrokecolor{currentstroke}%
\pgfsetstrokeopacity{0.955541}%
\pgfsetdash{}{0pt}%
\pgfpathmoveto{\pgfqpoint{2.075292in}{2.397389in}}%
\pgfpathcurveto{\pgfqpoint{2.083528in}{2.397389in}}{\pgfqpoint{2.091429in}{2.400662in}}{\pgfqpoint{2.097252in}{2.406486in}}%
\pgfpathcurveto{\pgfqpoint{2.103076in}{2.412309in}}{\pgfqpoint{2.106349in}{2.420209in}}{\pgfqpoint{2.106349in}{2.428446in}}%
\pgfpathcurveto{\pgfqpoint{2.106349in}{2.436682in}}{\pgfqpoint{2.103076in}{2.444582in}}{\pgfqpoint{2.097252in}{2.450406in}}%
\pgfpathcurveto{\pgfqpoint{2.091429in}{2.456230in}}{\pgfqpoint{2.083528in}{2.459502in}}{\pgfqpoint{2.075292in}{2.459502in}}%
\pgfpathcurveto{\pgfqpoint{2.067056in}{2.459502in}}{\pgfqpoint{2.059156in}{2.456230in}}{\pgfqpoint{2.053332in}{2.450406in}}%
\pgfpathcurveto{\pgfqpoint{2.047508in}{2.444582in}}{\pgfqpoint{2.044236in}{2.436682in}}{\pgfqpoint{2.044236in}{2.428446in}}%
\pgfpathcurveto{\pgfqpoint{2.044236in}{2.420209in}}{\pgfqpoint{2.047508in}{2.412309in}}{\pgfqpoint{2.053332in}{2.406486in}}%
\pgfpathcurveto{\pgfqpoint{2.059156in}{2.400662in}}{\pgfqpoint{2.067056in}{2.397389in}}{\pgfqpoint{2.075292in}{2.397389in}}%
\pgfpathclose%
\pgfusepath{stroke,fill}%
\end{pgfscope}%
\begin{pgfscope}%
\pgfpathrectangle{\pgfqpoint{0.100000in}{0.212622in}}{\pgfqpoint{3.696000in}{3.696000in}}%
\pgfusepath{clip}%
\pgfsetbuttcap%
\pgfsetroundjoin%
\definecolor{currentfill}{rgb}{0.121569,0.466667,0.705882}%
\pgfsetfillcolor{currentfill}%
\pgfsetfillopacity{0.955541}%
\pgfsetlinewidth{1.003750pt}%
\definecolor{currentstroke}{rgb}{0.121569,0.466667,0.705882}%
\pgfsetstrokecolor{currentstroke}%
\pgfsetstrokeopacity{0.955541}%
\pgfsetdash{}{0pt}%
\pgfpathmoveto{\pgfqpoint{2.075292in}{2.397389in}}%
\pgfpathcurveto{\pgfqpoint{2.083528in}{2.397389in}}{\pgfqpoint{2.091429in}{2.400662in}}{\pgfqpoint{2.097252in}{2.406486in}}%
\pgfpathcurveto{\pgfqpoint{2.103076in}{2.412309in}}{\pgfqpoint{2.106349in}{2.420209in}}{\pgfqpoint{2.106349in}{2.428446in}}%
\pgfpathcurveto{\pgfqpoint{2.106349in}{2.436682in}}{\pgfqpoint{2.103076in}{2.444582in}}{\pgfqpoint{2.097252in}{2.450406in}}%
\pgfpathcurveto{\pgfqpoint{2.091429in}{2.456230in}}{\pgfqpoint{2.083528in}{2.459502in}}{\pgfqpoint{2.075292in}{2.459502in}}%
\pgfpathcurveto{\pgfqpoint{2.067056in}{2.459502in}}{\pgfqpoint{2.059156in}{2.456230in}}{\pgfqpoint{2.053332in}{2.450406in}}%
\pgfpathcurveto{\pgfqpoint{2.047508in}{2.444582in}}{\pgfqpoint{2.044236in}{2.436682in}}{\pgfqpoint{2.044236in}{2.428446in}}%
\pgfpathcurveto{\pgfqpoint{2.044236in}{2.420209in}}{\pgfqpoint{2.047508in}{2.412309in}}{\pgfqpoint{2.053332in}{2.406486in}}%
\pgfpathcurveto{\pgfqpoint{2.059156in}{2.400662in}}{\pgfqpoint{2.067056in}{2.397389in}}{\pgfqpoint{2.075292in}{2.397389in}}%
\pgfpathclose%
\pgfusepath{stroke,fill}%
\end{pgfscope}%
\begin{pgfscope}%
\pgfpathrectangle{\pgfqpoint{0.100000in}{0.212622in}}{\pgfqpoint{3.696000in}{3.696000in}}%
\pgfusepath{clip}%
\pgfsetbuttcap%
\pgfsetroundjoin%
\definecolor{currentfill}{rgb}{0.121569,0.466667,0.705882}%
\pgfsetfillcolor{currentfill}%
\pgfsetfillopacity{0.955541}%
\pgfsetlinewidth{1.003750pt}%
\definecolor{currentstroke}{rgb}{0.121569,0.466667,0.705882}%
\pgfsetstrokecolor{currentstroke}%
\pgfsetstrokeopacity{0.955541}%
\pgfsetdash{}{0pt}%
\pgfpathmoveto{\pgfqpoint{2.075292in}{2.397389in}}%
\pgfpathcurveto{\pgfqpoint{2.083528in}{2.397389in}}{\pgfqpoint{2.091429in}{2.400662in}}{\pgfqpoint{2.097252in}{2.406486in}}%
\pgfpathcurveto{\pgfqpoint{2.103076in}{2.412309in}}{\pgfqpoint{2.106349in}{2.420209in}}{\pgfqpoint{2.106349in}{2.428446in}}%
\pgfpathcurveto{\pgfqpoint{2.106349in}{2.436682in}}{\pgfqpoint{2.103076in}{2.444582in}}{\pgfqpoint{2.097252in}{2.450406in}}%
\pgfpathcurveto{\pgfqpoint{2.091429in}{2.456230in}}{\pgfqpoint{2.083528in}{2.459502in}}{\pgfqpoint{2.075292in}{2.459502in}}%
\pgfpathcurveto{\pgfqpoint{2.067056in}{2.459502in}}{\pgfqpoint{2.059156in}{2.456230in}}{\pgfqpoint{2.053332in}{2.450406in}}%
\pgfpathcurveto{\pgfqpoint{2.047508in}{2.444582in}}{\pgfqpoint{2.044236in}{2.436682in}}{\pgfqpoint{2.044236in}{2.428446in}}%
\pgfpathcurveto{\pgfqpoint{2.044236in}{2.420209in}}{\pgfqpoint{2.047508in}{2.412309in}}{\pgfqpoint{2.053332in}{2.406486in}}%
\pgfpathcurveto{\pgfqpoint{2.059156in}{2.400662in}}{\pgfqpoint{2.067056in}{2.397389in}}{\pgfqpoint{2.075292in}{2.397389in}}%
\pgfpathclose%
\pgfusepath{stroke,fill}%
\end{pgfscope}%
\begin{pgfscope}%
\pgfpathrectangle{\pgfqpoint{0.100000in}{0.212622in}}{\pgfqpoint{3.696000in}{3.696000in}}%
\pgfusepath{clip}%
\pgfsetbuttcap%
\pgfsetroundjoin%
\definecolor{currentfill}{rgb}{0.121569,0.466667,0.705882}%
\pgfsetfillcolor{currentfill}%
\pgfsetfillopacity{0.955541}%
\pgfsetlinewidth{1.003750pt}%
\definecolor{currentstroke}{rgb}{0.121569,0.466667,0.705882}%
\pgfsetstrokecolor{currentstroke}%
\pgfsetstrokeopacity{0.955541}%
\pgfsetdash{}{0pt}%
\pgfpathmoveto{\pgfqpoint{2.075292in}{2.397389in}}%
\pgfpathcurveto{\pgfqpoint{2.083528in}{2.397389in}}{\pgfqpoint{2.091429in}{2.400662in}}{\pgfqpoint{2.097252in}{2.406486in}}%
\pgfpathcurveto{\pgfqpoint{2.103076in}{2.412309in}}{\pgfqpoint{2.106349in}{2.420209in}}{\pgfqpoint{2.106349in}{2.428446in}}%
\pgfpathcurveto{\pgfqpoint{2.106349in}{2.436682in}}{\pgfqpoint{2.103076in}{2.444582in}}{\pgfqpoint{2.097252in}{2.450406in}}%
\pgfpathcurveto{\pgfqpoint{2.091429in}{2.456230in}}{\pgfqpoint{2.083528in}{2.459502in}}{\pgfqpoint{2.075292in}{2.459502in}}%
\pgfpathcurveto{\pgfqpoint{2.067056in}{2.459502in}}{\pgfqpoint{2.059156in}{2.456230in}}{\pgfqpoint{2.053332in}{2.450406in}}%
\pgfpathcurveto{\pgfqpoint{2.047508in}{2.444582in}}{\pgfqpoint{2.044236in}{2.436682in}}{\pgfqpoint{2.044236in}{2.428446in}}%
\pgfpathcurveto{\pgfqpoint{2.044236in}{2.420209in}}{\pgfqpoint{2.047508in}{2.412309in}}{\pgfqpoint{2.053332in}{2.406486in}}%
\pgfpathcurveto{\pgfqpoint{2.059156in}{2.400662in}}{\pgfqpoint{2.067056in}{2.397389in}}{\pgfqpoint{2.075292in}{2.397389in}}%
\pgfpathclose%
\pgfusepath{stroke,fill}%
\end{pgfscope}%
\begin{pgfscope}%
\pgfpathrectangle{\pgfqpoint{0.100000in}{0.212622in}}{\pgfqpoint{3.696000in}{3.696000in}}%
\pgfusepath{clip}%
\pgfsetbuttcap%
\pgfsetroundjoin%
\definecolor{currentfill}{rgb}{0.121569,0.466667,0.705882}%
\pgfsetfillcolor{currentfill}%
\pgfsetfillopacity{0.955541}%
\pgfsetlinewidth{1.003750pt}%
\definecolor{currentstroke}{rgb}{0.121569,0.466667,0.705882}%
\pgfsetstrokecolor{currentstroke}%
\pgfsetstrokeopacity{0.955541}%
\pgfsetdash{}{0pt}%
\pgfpathmoveto{\pgfqpoint{2.075292in}{2.397389in}}%
\pgfpathcurveto{\pgfqpoint{2.083528in}{2.397389in}}{\pgfqpoint{2.091429in}{2.400662in}}{\pgfqpoint{2.097252in}{2.406486in}}%
\pgfpathcurveto{\pgfqpoint{2.103076in}{2.412309in}}{\pgfqpoint{2.106349in}{2.420209in}}{\pgfqpoint{2.106349in}{2.428446in}}%
\pgfpathcurveto{\pgfqpoint{2.106349in}{2.436682in}}{\pgfqpoint{2.103076in}{2.444582in}}{\pgfqpoint{2.097252in}{2.450406in}}%
\pgfpathcurveto{\pgfqpoint{2.091429in}{2.456230in}}{\pgfqpoint{2.083528in}{2.459502in}}{\pgfqpoint{2.075292in}{2.459502in}}%
\pgfpathcurveto{\pgfqpoint{2.067056in}{2.459502in}}{\pgfqpoint{2.059156in}{2.456230in}}{\pgfqpoint{2.053332in}{2.450406in}}%
\pgfpathcurveto{\pgfqpoint{2.047508in}{2.444582in}}{\pgfqpoint{2.044236in}{2.436682in}}{\pgfqpoint{2.044236in}{2.428446in}}%
\pgfpathcurveto{\pgfqpoint{2.044236in}{2.420209in}}{\pgfqpoint{2.047508in}{2.412309in}}{\pgfqpoint{2.053332in}{2.406486in}}%
\pgfpathcurveto{\pgfqpoint{2.059156in}{2.400662in}}{\pgfqpoint{2.067056in}{2.397389in}}{\pgfqpoint{2.075292in}{2.397389in}}%
\pgfpathclose%
\pgfusepath{stroke,fill}%
\end{pgfscope}%
\begin{pgfscope}%
\pgfpathrectangle{\pgfqpoint{0.100000in}{0.212622in}}{\pgfqpoint{3.696000in}{3.696000in}}%
\pgfusepath{clip}%
\pgfsetbuttcap%
\pgfsetroundjoin%
\definecolor{currentfill}{rgb}{0.121569,0.466667,0.705882}%
\pgfsetfillcolor{currentfill}%
\pgfsetfillopacity{0.955541}%
\pgfsetlinewidth{1.003750pt}%
\definecolor{currentstroke}{rgb}{0.121569,0.466667,0.705882}%
\pgfsetstrokecolor{currentstroke}%
\pgfsetstrokeopacity{0.955541}%
\pgfsetdash{}{0pt}%
\pgfpathmoveto{\pgfqpoint{2.075292in}{2.397389in}}%
\pgfpathcurveto{\pgfqpoint{2.083528in}{2.397389in}}{\pgfqpoint{2.091429in}{2.400662in}}{\pgfqpoint{2.097252in}{2.406486in}}%
\pgfpathcurveto{\pgfqpoint{2.103076in}{2.412309in}}{\pgfqpoint{2.106349in}{2.420209in}}{\pgfqpoint{2.106349in}{2.428446in}}%
\pgfpathcurveto{\pgfqpoint{2.106349in}{2.436682in}}{\pgfqpoint{2.103076in}{2.444582in}}{\pgfqpoint{2.097252in}{2.450406in}}%
\pgfpathcurveto{\pgfqpoint{2.091429in}{2.456230in}}{\pgfqpoint{2.083528in}{2.459502in}}{\pgfqpoint{2.075292in}{2.459502in}}%
\pgfpathcurveto{\pgfqpoint{2.067056in}{2.459502in}}{\pgfqpoint{2.059156in}{2.456230in}}{\pgfqpoint{2.053332in}{2.450406in}}%
\pgfpathcurveto{\pgfqpoint{2.047508in}{2.444582in}}{\pgfqpoint{2.044236in}{2.436682in}}{\pgfqpoint{2.044236in}{2.428446in}}%
\pgfpathcurveto{\pgfqpoint{2.044236in}{2.420209in}}{\pgfqpoint{2.047508in}{2.412309in}}{\pgfqpoint{2.053332in}{2.406486in}}%
\pgfpathcurveto{\pgfqpoint{2.059156in}{2.400662in}}{\pgfqpoint{2.067056in}{2.397389in}}{\pgfqpoint{2.075292in}{2.397389in}}%
\pgfpathclose%
\pgfusepath{stroke,fill}%
\end{pgfscope}%
\begin{pgfscope}%
\pgfpathrectangle{\pgfqpoint{0.100000in}{0.212622in}}{\pgfqpoint{3.696000in}{3.696000in}}%
\pgfusepath{clip}%
\pgfsetbuttcap%
\pgfsetroundjoin%
\definecolor{currentfill}{rgb}{0.121569,0.466667,0.705882}%
\pgfsetfillcolor{currentfill}%
\pgfsetfillopacity{0.955541}%
\pgfsetlinewidth{1.003750pt}%
\definecolor{currentstroke}{rgb}{0.121569,0.466667,0.705882}%
\pgfsetstrokecolor{currentstroke}%
\pgfsetstrokeopacity{0.955541}%
\pgfsetdash{}{0pt}%
\pgfpathmoveto{\pgfqpoint{2.075292in}{2.397389in}}%
\pgfpathcurveto{\pgfqpoint{2.083528in}{2.397389in}}{\pgfqpoint{2.091429in}{2.400662in}}{\pgfqpoint{2.097252in}{2.406486in}}%
\pgfpathcurveto{\pgfqpoint{2.103076in}{2.412309in}}{\pgfqpoint{2.106349in}{2.420209in}}{\pgfqpoint{2.106349in}{2.428446in}}%
\pgfpathcurveto{\pgfqpoint{2.106349in}{2.436682in}}{\pgfqpoint{2.103076in}{2.444582in}}{\pgfqpoint{2.097252in}{2.450406in}}%
\pgfpathcurveto{\pgfqpoint{2.091429in}{2.456230in}}{\pgfqpoint{2.083528in}{2.459502in}}{\pgfqpoint{2.075292in}{2.459502in}}%
\pgfpathcurveto{\pgfqpoint{2.067056in}{2.459502in}}{\pgfqpoint{2.059156in}{2.456230in}}{\pgfqpoint{2.053332in}{2.450406in}}%
\pgfpathcurveto{\pgfqpoint{2.047508in}{2.444582in}}{\pgfqpoint{2.044236in}{2.436682in}}{\pgfqpoint{2.044236in}{2.428446in}}%
\pgfpathcurveto{\pgfqpoint{2.044236in}{2.420209in}}{\pgfqpoint{2.047508in}{2.412309in}}{\pgfqpoint{2.053332in}{2.406486in}}%
\pgfpathcurveto{\pgfqpoint{2.059156in}{2.400662in}}{\pgfqpoint{2.067056in}{2.397389in}}{\pgfqpoint{2.075292in}{2.397389in}}%
\pgfpathclose%
\pgfusepath{stroke,fill}%
\end{pgfscope}%
\begin{pgfscope}%
\pgfpathrectangle{\pgfqpoint{0.100000in}{0.212622in}}{\pgfqpoint{3.696000in}{3.696000in}}%
\pgfusepath{clip}%
\pgfsetbuttcap%
\pgfsetroundjoin%
\definecolor{currentfill}{rgb}{0.121569,0.466667,0.705882}%
\pgfsetfillcolor{currentfill}%
\pgfsetfillopacity{0.955541}%
\pgfsetlinewidth{1.003750pt}%
\definecolor{currentstroke}{rgb}{0.121569,0.466667,0.705882}%
\pgfsetstrokecolor{currentstroke}%
\pgfsetstrokeopacity{0.955541}%
\pgfsetdash{}{0pt}%
\pgfpathmoveto{\pgfqpoint{2.075292in}{2.397389in}}%
\pgfpathcurveto{\pgfqpoint{2.083528in}{2.397389in}}{\pgfqpoint{2.091429in}{2.400662in}}{\pgfqpoint{2.097252in}{2.406486in}}%
\pgfpathcurveto{\pgfqpoint{2.103076in}{2.412309in}}{\pgfqpoint{2.106349in}{2.420209in}}{\pgfqpoint{2.106349in}{2.428446in}}%
\pgfpathcurveto{\pgfqpoint{2.106349in}{2.436682in}}{\pgfqpoint{2.103076in}{2.444582in}}{\pgfqpoint{2.097252in}{2.450406in}}%
\pgfpathcurveto{\pgfqpoint{2.091429in}{2.456230in}}{\pgfqpoint{2.083528in}{2.459502in}}{\pgfqpoint{2.075292in}{2.459502in}}%
\pgfpathcurveto{\pgfqpoint{2.067056in}{2.459502in}}{\pgfqpoint{2.059156in}{2.456230in}}{\pgfqpoint{2.053332in}{2.450406in}}%
\pgfpathcurveto{\pgfqpoint{2.047508in}{2.444582in}}{\pgfqpoint{2.044236in}{2.436682in}}{\pgfqpoint{2.044236in}{2.428446in}}%
\pgfpathcurveto{\pgfqpoint{2.044236in}{2.420209in}}{\pgfqpoint{2.047508in}{2.412309in}}{\pgfqpoint{2.053332in}{2.406486in}}%
\pgfpathcurveto{\pgfqpoint{2.059156in}{2.400662in}}{\pgfqpoint{2.067056in}{2.397389in}}{\pgfqpoint{2.075292in}{2.397389in}}%
\pgfpathclose%
\pgfusepath{stroke,fill}%
\end{pgfscope}%
\begin{pgfscope}%
\pgfpathrectangle{\pgfqpoint{0.100000in}{0.212622in}}{\pgfqpoint{3.696000in}{3.696000in}}%
\pgfusepath{clip}%
\pgfsetbuttcap%
\pgfsetroundjoin%
\definecolor{currentfill}{rgb}{0.121569,0.466667,0.705882}%
\pgfsetfillcolor{currentfill}%
\pgfsetfillopacity{0.955541}%
\pgfsetlinewidth{1.003750pt}%
\definecolor{currentstroke}{rgb}{0.121569,0.466667,0.705882}%
\pgfsetstrokecolor{currentstroke}%
\pgfsetstrokeopacity{0.955541}%
\pgfsetdash{}{0pt}%
\pgfpathmoveto{\pgfqpoint{2.075292in}{2.397389in}}%
\pgfpathcurveto{\pgfqpoint{2.083528in}{2.397389in}}{\pgfqpoint{2.091429in}{2.400662in}}{\pgfqpoint{2.097252in}{2.406486in}}%
\pgfpathcurveto{\pgfqpoint{2.103076in}{2.412309in}}{\pgfqpoint{2.106349in}{2.420209in}}{\pgfqpoint{2.106349in}{2.428446in}}%
\pgfpathcurveto{\pgfqpoint{2.106349in}{2.436682in}}{\pgfqpoint{2.103076in}{2.444582in}}{\pgfqpoint{2.097252in}{2.450406in}}%
\pgfpathcurveto{\pgfqpoint{2.091429in}{2.456230in}}{\pgfqpoint{2.083528in}{2.459502in}}{\pgfqpoint{2.075292in}{2.459502in}}%
\pgfpathcurveto{\pgfqpoint{2.067056in}{2.459502in}}{\pgfqpoint{2.059156in}{2.456230in}}{\pgfqpoint{2.053332in}{2.450406in}}%
\pgfpathcurveto{\pgfqpoint{2.047508in}{2.444582in}}{\pgfqpoint{2.044236in}{2.436682in}}{\pgfqpoint{2.044236in}{2.428446in}}%
\pgfpathcurveto{\pgfqpoint{2.044236in}{2.420209in}}{\pgfqpoint{2.047508in}{2.412309in}}{\pgfqpoint{2.053332in}{2.406486in}}%
\pgfpathcurveto{\pgfqpoint{2.059156in}{2.400662in}}{\pgfqpoint{2.067056in}{2.397389in}}{\pgfqpoint{2.075292in}{2.397389in}}%
\pgfpathclose%
\pgfusepath{stroke,fill}%
\end{pgfscope}%
\begin{pgfscope}%
\pgfpathrectangle{\pgfqpoint{0.100000in}{0.212622in}}{\pgfqpoint{3.696000in}{3.696000in}}%
\pgfusepath{clip}%
\pgfsetbuttcap%
\pgfsetroundjoin%
\definecolor{currentfill}{rgb}{0.121569,0.466667,0.705882}%
\pgfsetfillcolor{currentfill}%
\pgfsetfillopacity{0.955541}%
\pgfsetlinewidth{1.003750pt}%
\definecolor{currentstroke}{rgb}{0.121569,0.466667,0.705882}%
\pgfsetstrokecolor{currentstroke}%
\pgfsetstrokeopacity{0.955541}%
\pgfsetdash{}{0pt}%
\pgfpathmoveto{\pgfqpoint{2.075292in}{2.397389in}}%
\pgfpathcurveto{\pgfqpoint{2.083528in}{2.397389in}}{\pgfqpoint{2.091429in}{2.400662in}}{\pgfqpoint{2.097252in}{2.406486in}}%
\pgfpathcurveto{\pgfqpoint{2.103076in}{2.412309in}}{\pgfqpoint{2.106349in}{2.420209in}}{\pgfqpoint{2.106349in}{2.428446in}}%
\pgfpathcurveto{\pgfqpoint{2.106349in}{2.436682in}}{\pgfqpoint{2.103076in}{2.444582in}}{\pgfqpoint{2.097252in}{2.450406in}}%
\pgfpathcurveto{\pgfqpoint{2.091429in}{2.456230in}}{\pgfqpoint{2.083528in}{2.459502in}}{\pgfqpoint{2.075292in}{2.459502in}}%
\pgfpathcurveto{\pgfqpoint{2.067056in}{2.459502in}}{\pgfqpoint{2.059156in}{2.456230in}}{\pgfqpoint{2.053332in}{2.450406in}}%
\pgfpathcurveto{\pgfqpoint{2.047508in}{2.444582in}}{\pgfqpoint{2.044236in}{2.436682in}}{\pgfqpoint{2.044236in}{2.428446in}}%
\pgfpathcurveto{\pgfqpoint{2.044236in}{2.420209in}}{\pgfqpoint{2.047508in}{2.412309in}}{\pgfqpoint{2.053332in}{2.406486in}}%
\pgfpathcurveto{\pgfqpoint{2.059156in}{2.400662in}}{\pgfqpoint{2.067056in}{2.397389in}}{\pgfqpoint{2.075292in}{2.397389in}}%
\pgfpathclose%
\pgfusepath{stroke,fill}%
\end{pgfscope}%
\begin{pgfscope}%
\pgfpathrectangle{\pgfqpoint{0.100000in}{0.212622in}}{\pgfqpoint{3.696000in}{3.696000in}}%
\pgfusepath{clip}%
\pgfsetbuttcap%
\pgfsetroundjoin%
\definecolor{currentfill}{rgb}{0.121569,0.466667,0.705882}%
\pgfsetfillcolor{currentfill}%
\pgfsetfillopacity{0.955541}%
\pgfsetlinewidth{1.003750pt}%
\definecolor{currentstroke}{rgb}{0.121569,0.466667,0.705882}%
\pgfsetstrokecolor{currentstroke}%
\pgfsetstrokeopacity{0.955541}%
\pgfsetdash{}{0pt}%
\pgfpathmoveto{\pgfqpoint{2.075292in}{2.397389in}}%
\pgfpathcurveto{\pgfqpoint{2.083528in}{2.397389in}}{\pgfqpoint{2.091429in}{2.400662in}}{\pgfqpoint{2.097252in}{2.406486in}}%
\pgfpathcurveto{\pgfqpoint{2.103076in}{2.412309in}}{\pgfqpoint{2.106349in}{2.420209in}}{\pgfqpoint{2.106349in}{2.428446in}}%
\pgfpathcurveto{\pgfqpoint{2.106349in}{2.436682in}}{\pgfqpoint{2.103076in}{2.444582in}}{\pgfqpoint{2.097252in}{2.450406in}}%
\pgfpathcurveto{\pgfqpoint{2.091429in}{2.456230in}}{\pgfqpoint{2.083528in}{2.459502in}}{\pgfqpoint{2.075292in}{2.459502in}}%
\pgfpathcurveto{\pgfqpoint{2.067056in}{2.459502in}}{\pgfqpoint{2.059156in}{2.456230in}}{\pgfqpoint{2.053332in}{2.450406in}}%
\pgfpathcurveto{\pgfqpoint{2.047508in}{2.444582in}}{\pgfqpoint{2.044236in}{2.436682in}}{\pgfqpoint{2.044236in}{2.428446in}}%
\pgfpathcurveto{\pgfqpoint{2.044236in}{2.420209in}}{\pgfqpoint{2.047508in}{2.412309in}}{\pgfqpoint{2.053332in}{2.406486in}}%
\pgfpathcurveto{\pgfqpoint{2.059156in}{2.400662in}}{\pgfqpoint{2.067056in}{2.397389in}}{\pgfqpoint{2.075292in}{2.397389in}}%
\pgfpathclose%
\pgfusepath{stroke,fill}%
\end{pgfscope}%
\begin{pgfscope}%
\pgfpathrectangle{\pgfqpoint{0.100000in}{0.212622in}}{\pgfqpoint{3.696000in}{3.696000in}}%
\pgfusepath{clip}%
\pgfsetbuttcap%
\pgfsetroundjoin%
\definecolor{currentfill}{rgb}{0.121569,0.466667,0.705882}%
\pgfsetfillcolor{currentfill}%
\pgfsetfillopacity{0.955541}%
\pgfsetlinewidth{1.003750pt}%
\definecolor{currentstroke}{rgb}{0.121569,0.466667,0.705882}%
\pgfsetstrokecolor{currentstroke}%
\pgfsetstrokeopacity{0.955541}%
\pgfsetdash{}{0pt}%
\pgfpathmoveto{\pgfqpoint{2.075292in}{2.397389in}}%
\pgfpathcurveto{\pgfqpoint{2.083528in}{2.397389in}}{\pgfqpoint{2.091429in}{2.400662in}}{\pgfqpoint{2.097252in}{2.406486in}}%
\pgfpathcurveto{\pgfqpoint{2.103076in}{2.412309in}}{\pgfqpoint{2.106349in}{2.420209in}}{\pgfqpoint{2.106349in}{2.428446in}}%
\pgfpathcurveto{\pgfqpoint{2.106349in}{2.436682in}}{\pgfqpoint{2.103076in}{2.444582in}}{\pgfqpoint{2.097252in}{2.450406in}}%
\pgfpathcurveto{\pgfqpoint{2.091429in}{2.456230in}}{\pgfqpoint{2.083528in}{2.459502in}}{\pgfqpoint{2.075292in}{2.459502in}}%
\pgfpathcurveto{\pgfqpoint{2.067056in}{2.459502in}}{\pgfqpoint{2.059156in}{2.456230in}}{\pgfqpoint{2.053332in}{2.450406in}}%
\pgfpathcurveto{\pgfqpoint{2.047508in}{2.444582in}}{\pgfqpoint{2.044236in}{2.436682in}}{\pgfqpoint{2.044236in}{2.428446in}}%
\pgfpathcurveto{\pgfqpoint{2.044236in}{2.420209in}}{\pgfqpoint{2.047508in}{2.412309in}}{\pgfqpoint{2.053332in}{2.406486in}}%
\pgfpathcurveto{\pgfqpoint{2.059156in}{2.400662in}}{\pgfqpoint{2.067056in}{2.397389in}}{\pgfqpoint{2.075292in}{2.397389in}}%
\pgfpathclose%
\pgfusepath{stroke,fill}%
\end{pgfscope}%
\begin{pgfscope}%
\pgfpathrectangle{\pgfqpoint{0.100000in}{0.212622in}}{\pgfqpoint{3.696000in}{3.696000in}}%
\pgfusepath{clip}%
\pgfsetbuttcap%
\pgfsetroundjoin%
\definecolor{currentfill}{rgb}{0.121569,0.466667,0.705882}%
\pgfsetfillcolor{currentfill}%
\pgfsetfillopacity{0.955541}%
\pgfsetlinewidth{1.003750pt}%
\definecolor{currentstroke}{rgb}{0.121569,0.466667,0.705882}%
\pgfsetstrokecolor{currentstroke}%
\pgfsetstrokeopacity{0.955541}%
\pgfsetdash{}{0pt}%
\pgfpathmoveto{\pgfqpoint{2.075292in}{2.397389in}}%
\pgfpathcurveto{\pgfqpoint{2.083528in}{2.397389in}}{\pgfqpoint{2.091429in}{2.400662in}}{\pgfqpoint{2.097252in}{2.406486in}}%
\pgfpathcurveto{\pgfqpoint{2.103076in}{2.412309in}}{\pgfqpoint{2.106349in}{2.420209in}}{\pgfqpoint{2.106349in}{2.428446in}}%
\pgfpathcurveto{\pgfqpoint{2.106349in}{2.436682in}}{\pgfqpoint{2.103076in}{2.444582in}}{\pgfqpoint{2.097252in}{2.450406in}}%
\pgfpathcurveto{\pgfqpoint{2.091429in}{2.456230in}}{\pgfqpoint{2.083528in}{2.459502in}}{\pgfqpoint{2.075292in}{2.459502in}}%
\pgfpathcurveto{\pgfqpoint{2.067056in}{2.459502in}}{\pgfqpoint{2.059156in}{2.456230in}}{\pgfqpoint{2.053332in}{2.450406in}}%
\pgfpathcurveto{\pgfqpoint{2.047508in}{2.444582in}}{\pgfqpoint{2.044236in}{2.436682in}}{\pgfqpoint{2.044236in}{2.428446in}}%
\pgfpathcurveto{\pgfqpoint{2.044236in}{2.420209in}}{\pgfqpoint{2.047508in}{2.412309in}}{\pgfqpoint{2.053332in}{2.406486in}}%
\pgfpathcurveto{\pgfqpoint{2.059156in}{2.400662in}}{\pgfqpoint{2.067056in}{2.397389in}}{\pgfqpoint{2.075292in}{2.397389in}}%
\pgfpathclose%
\pgfusepath{stroke,fill}%
\end{pgfscope}%
\begin{pgfscope}%
\pgfpathrectangle{\pgfqpoint{0.100000in}{0.212622in}}{\pgfqpoint{3.696000in}{3.696000in}}%
\pgfusepath{clip}%
\pgfsetbuttcap%
\pgfsetroundjoin%
\definecolor{currentfill}{rgb}{0.121569,0.466667,0.705882}%
\pgfsetfillcolor{currentfill}%
\pgfsetfillopacity{0.955541}%
\pgfsetlinewidth{1.003750pt}%
\definecolor{currentstroke}{rgb}{0.121569,0.466667,0.705882}%
\pgfsetstrokecolor{currentstroke}%
\pgfsetstrokeopacity{0.955541}%
\pgfsetdash{}{0pt}%
\pgfpathmoveto{\pgfqpoint{2.075292in}{2.397389in}}%
\pgfpathcurveto{\pgfqpoint{2.083528in}{2.397389in}}{\pgfqpoint{2.091429in}{2.400662in}}{\pgfqpoint{2.097252in}{2.406486in}}%
\pgfpathcurveto{\pgfqpoint{2.103076in}{2.412309in}}{\pgfqpoint{2.106349in}{2.420209in}}{\pgfqpoint{2.106349in}{2.428446in}}%
\pgfpathcurveto{\pgfqpoint{2.106349in}{2.436682in}}{\pgfqpoint{2.103076in}{2.444582in}}{\pgfqpoint{2.097252in}{2.450406in}}%
\pgfpathcurveto{\pgfqpoint{2.091429in}{2.456230in}}{\pgfqpoint{2.083528in}{2.459502in}}{\pgfqpoint{2.075292in}{2.459502in}}%
\pgfpathcurveto{\pgfqpoint{2.067056in}{2.459502in}}{\pgfqpoint{2.059156in}{2.456230in}}{\pgfqpoint{2.053332in}{2.450406in}}%
\pgfpathcurveto{\pgfqpoint{2.047508in}{2.444582in}}{\pgfqpoint{2.044236in}{2.436682in}}{\pgfqpoint{2.044236in}{2.428446in}}%
\pgfpathcurveto{\pgfqpoint{2.044236in}{2.420209in}}{\pgfqpoint{2.047508in}{2.412309in}}{\pgfqpoint{2.053332in}{2.406486in}}%
\pgfpathcurveto{\pgfqpoint{2.059156in}{2.400662in}}{\pgfqpoint{2.067056in}{2.397389in}}{\pgfqpoint{2.075292in}{2.397389in}}%
\pgfpathclose%
\pgfusepath{stroke,fill}%
\end{pgfscope}%
\begin{pgfscope}%
\pgfpathrectangle{\pgfqpoint{0.100000in}{0.212622in}}{\pgfqpoint{3.696000in}{3.696000in}}%
\pgfusepath{clip}%
\pgfsetbuttcap%
\pgfsetroundjoin%
\definecolor{currentfill}{rgb}{0.121569,0.466667,0.705882}%
\pgfsetfillcolor{currentfill}%
\pgfsetfillopacity{0.955541}%
\pgfsetlinewidth{1.003750pt}%
\definecolor{currentstroke}{rgb}{0.121569,0.466667,0.705882}%
\pgfsetstrokecolor{currentstroke}%
\pgfsetstrokeopacity{0.955541}%
\pgfsetdash{}{0pt}%
\pgfpathmoveto{\pgfqpoint{2.075292in}{2.397389in}}%
\pgfpathcurveto{\pgfqpoint{2.083528in}{2.397389in}}{\pgfqpoint{2.091429in}{2.400662in}}{\pgfqpoint{2.097252in}{2.406486in}}%
\pgfpathcurveto{\pgfqpoint{2.103076in}{2.412309in}}{\pgfqpoint{2.106349in}{2.420209in}}{\pgfqpoint{2.106349in}{2.428446in}}%
\pgfpathcurveto{\pgfqpoint{2.106349in}{2.436682in}}{\pgfqpoint{2.103076in}{2.444582in}}{\pgfqpoint{2.097252in}{2.450406in}}%
\pgfpathcurveto{\pgfqpoint{2.091429in}{2.456230in}}{\pgfqpoint{2.083528in}{2.459502in}}{\pgfqpoint{2.075292in}{2.459502in}}%
\pgfpathcurveto{\pgfqpoint{2.067056in}{2.459502in}}{\pgfqpoint{2.059156in}{2.456230in}}{\pgfqpoint{2.053332in}{2.450406in}}%
\pgfpathcurveto{\pgfqpoint{2.047508in}{2.444582in}}{\pgfqpoint{2.044236in}{2.436682in}}{\pgfqpoint{2.044236in}{2.428446in}}%
\pgfpathcurveto{\pgfqpoint{2.044236in}{2.420209in}}{\pgfqpoint{2.047508in}{2.412309in}}{\pgfqpoint{2.053332in}{2.406486in}}%
\pgfpathcurveto{\pgfqpoint{2.059156in}{2.400662in}}{\pgfqpoint{2.067056in}{2.397389in}}{\pgfqpoint{2.075292in}{2.397389in}}%
\pgfpathclose%
\pgfusepath{stroke,fill}%
\end{pgfscope}%
\begin{pgfscope}%
\pgfpathrectangle{\pgfqpoint{0.100000in}{0.212622in}}{\pgfqpoint{3.696000in}{3.696000in}}%
\pgfusepath{clip}%
\pgfsetbuttcap%
\pgfsetroundjoin%
\definecolor{currentfill}{rgb}{0.121569,0.466667,0.705882}%
\pgfsetfillcolor{currentfill}%
\pgfsetfillopacity{0.955541}%
\pgfsetlinewidth{1.003750pt}%
\definecolor{currentstroke}{rgb}{0.121569,0.466667,0.705882}%
\pgfsetstrokecolor{currentstroke}%
\pgfsetstrokeopacity{0.955541}%
\pgfsetdash{}{0pt}%
\pgfpathmoveto{\pgfqpoint{2.075292in}{2.397389in}}%
\pgfpathcurveto{\pgfqpoint{2.083528in}{2.397389in}}{\pgfqpoint{2.091428in}{2.400662in}}{\pgfqpoint{2.097252in}{2.406485in}}%
\pgfpathcurveto{\pgfqpoint{2.103076in}{2.412309in}}{\pgfqpoint{2.106349in}{2.420209in}}{\pgfqpoint{2.106349in}{2.428446in}}%
\pgfpathcurveto{\pgfqpoint{2.106349in}{2.436682in}}{\pgfqpoint{2.103076in}{2.444582in}}{\pgfqpoint{2.097252in}{2.450406in}}%
\pgfpathcurveto{\pgfqpoint{2.091428in}{2.456230in}}{\pgfqpoint{2.083528in}{2.459502in}}{\pgfqpoint{2.075292in}{2.459502in}}%
\pgfpathcurveto{\pgfqpoint{2.067056in}{2.459502in}}{\pgfqpoint{2.059156in}{2.456230in}}{\pgfqpoint{2.053332in}{2.450406in}}%
\pgfpathcurveto{\pgfqpoint{2.047508in}{2.444582in}}{\pgfqpoint{2.044236in}{2.436682in}}{\pgfqpoint{2.044236in}{2.428446in}}%
\pgfpathcurveto{\pgfqpoint{2.044236in}{2.420209in}}{\pgfqpoint{2.047508in}{2.412309in}}{\pgfqpoint{2.053332in}{2.406485in}}%
\pgfpathcurveto{\pgfqpoint{2.059156in}{2.400662in}}{\pgfqpoint{2.067056in}{2.397389in}}{\pgfqpoint{2.075292in}{2.397389in}}%
\pgfpathclose%
\pgfusepath{stroke,fill}%
\end{pgfscope}%
\begin{pgfscope}%
\pgfpathrectangle{\pgfqpoint{0.100000in}{0.212622in}}{\pgfqpoint{3.696000in}{3.696000in}}%
\pgfusepath{clip}%
\pgfsetbuttcap%
\pgfsetroundjoin%
\definecolor{currentfill}{rgb}{0.121569,0.466667,0.705882}%
\pgfsetfillcolor{currentfill}%
\pgfsetfillopacity{0.955541}%
\pgfsetlinewidth{1.003750pt}%
\definecolor{currentstroke}{rgb}{0.121569,0.466667,0.705882}%
\pgfsetstrokecolor{currentstroke}%
\pgfsetstrokeopacity{0.955541}%
\pgfsetdash{}{0pt}%
\pgfpathmoveto{\pgfqpoint{2.075292in}{2.397389in}}%
\pgfpathcurveto{\pgfqpoint{2.083528in}{2.397389in}}{\pgfqpoint{2.091428in}{2.400662in}}{\pgfqpoint{2.097252in}{2.406485in}}%
\pgfpathcurveto{\pgfqpoint{2.103076in}{2.412309in}}{\pgfqpoint{2.106349in}{2.420209in}}{\pgfqpoint{2.106349in}{2.428446in}}%
\pgfpathcurveto{\pgfqpoint{2.106349in}{2.436682in}}{\pgfqpoint{2.103076in}{2.444582in}}{\pgfqpoint{2.097252in}{2.450406in}}%
\pgfpathcurveto{\pgfqpoint{2.091428in}{2.456230in}}{\pgfqpoint{2.083528in}{2.459502in}}{\pgfqpoint{2.075292in}{2.459502in}}%
\pgfpathcurveto{\pgfqpoint{2.067056in}{2.459502in}}{\pgfqpoint{2.059156in}{2.456230in}}{\pgfqpoint{2.053332in}{2.450406in}}%
\pgfpathcurveto{\pgfqpoint{2.047508in}{2.444582in}}{\pgfqpoint{2.044236in}{2.436682in}}{\pgfqpoint{2.044236in}{2.428446in}}%
\pgfpathcurveto{\pgfqpoint{2.044236in}{2.420209in}}{\pgfqpoint{2.047508in}{2.412309in}}{\pgfqpoint{2.053332in}{2.406485in}}%
\pgfpathcurveto{\pgfqpoint{2.059156in}{2.400662in}}{\pgfqpoint{2.067056in}{2.397389in}}{\pgfqpoint{2.075292in}{2.397389in}}%
\pgfpathclose%
\pgfusepath{stroke,fill}%
\end{pgfscope}%
\begin{pgfscope}%
\pgfpathrectangle{\pgfqpoint{0.100000in}{0.212622in}}{\pgfqpoint{3.696000in}{3.696000in}}%
\pgfusepath{clip}%
\pgfsetbuttcap%
\pgfsetroundjoin%
\definecolor{currentfill}{rgb}{0.121569,0.466667,0.705882}%
\pgfsetfillcolor{currentfill}%
\pgfsetfillopacity{0.955541}%
\pgfsetlinewidth{1.003750pt}%
\definecolor{currentstroke}{rgb}{0.121569,0.466667,0.705882}%
\pgfsetstrokecolor{currentstroke}%
\pgfsetstrokeopacity{0.955541}%
\pgfsetdash{}{0pt}%
\pgfpathmoveto{\pgfqpoint{2.075292in}{2.397389in}}%
\pgfpathcurveto{\pgfqpoint{2.083528in}{2.397389in}}{\pgfqpoint{2.091428in}{2.400662in}}{\pgfqpoint{2.097252in}{2.406485in}}%
\pgfpathcurveto{\pgfqpoint{2.103076in}{2.412309in}}{\pgfqpoint{2.106349in}{2.420209in}}{\pgfqpoint{2.106349in}{2.428446in}}%
\pgfpathcurveto{\pgfqpoint{2.106349in}{2.436682in}}{\pgfqpoint{2.103076in}{2.444582in}}{\pgfqpoint{2.097252in}{2.450406in}}%
\pgfpathcurveto{\pgfqpoint{2.091428in}{2.456230in}}{\pgfqpoint{2.083528in}{2.459502in}}{\pgfqpoint{2.075292in}{2.459502in}}%
\pgfpathcurveto{\pgfqpoint{2.067056in}{2.459502in}}{\pgfqpoint{2.059156in}{2.456230in}}{\pgfqpoint{2.053332in}{2.450406in}}%
\pgfpathcurveto{\pgfqpoint{2.047508in}{2.444582in}}{\pgfqpoint{2.044236in}{2.436682in}}{\pgfqpoint{2.044236in}{2.428446in}}%
\pgfpathcurveto{\pgfqpoint{2.044236in}{2.420209in}}{\pgfqpoint{2.047508in}{2.412309in}}{\pgfqpoint{2.053332in}{2.406485in}}%
\pgfpathcurveto{\pgfqpoint{2.059156in}{2.400662in}}{\pgfqpoint{2.067056in}{2.397389in}}{\pgfqpoint{2.075292in}{2.397389in}}%
\pgfpathclose%
\pgfusepath{stroke,fill}%
\end{pgfscope}%
\begin{pgfscope}%
\pgfpathrectangle{\pgfqpoint{0.100000in}{0.212622in}}{\pgfqpoint{3.696000in}{3.696000in}}%
\pgfusepath{clip}%
\pgfsetbuttcap%
\pgfsetroundjoin%
\definecolor{currentfill}{rgb}{0.121569,0.466667,0.705882}%
\pgfsetfillcolor{currentfill}%
\pgfsetfillopacity{0.955541}%
\pgfsetlinewidth{1.003750pt}%
\definecolor{currentstroke}{rgb}{0.121569,0.466667,0.705882}%
\pgfsetstrokecolor{currentstroke}%
\pgfsetstrokeopacity{0.955541}%
\pgfsetdash{}{0pt}%
\pgfpathmoveto{\pgfqpoint{2.075292in}{2.397389in}}%
\pgfpathcurveto{\pgfqpoint{2.083528in}{2.397389in}}{\pgfqpoint{2.091428in}{2.400662in}}{\pgfqpoint{2.097252in}{2.406485in}}%
\pgfpathcurveto{\pgfqpoint{2.103076in}{2.412309in}}{\pgfqpoint{2.106349in}{2.420209in}}{\pgfqpoint{2.106349in}{2.428446in}}%
\pgfpathcurveto{\pgfqpoint{2.106349in}{2.436682in}}{\pgfqpoint{2.103076in}{2.444582in}}{\pgfqpoint{2.097252in}{2.450406in}}%
\pgfpathcurveto{\pgfqpoint{2.091428in}{2.456230in}}{\pgfqpoint{2.083528in}{2.459502in}}{\pgfqpoint{2.075292in}{2.459502in}}%
\pgfpathcurveto{\pgfqpoint{2.067056in}{2.459502in}}{\pgfqpoint{2.059156in}{2.456230in}}{\pgfqpoint{2.053332in}{2.450406in}}%
\pgfpathcurveto{\pgfqpoint{2.047508in}{2.444582in}}{\pgfqpoint{2.044236in}{2.436682in}}{\pgfqpoint{2.044236in}{2.428446in}}%
\pgfpathcurveto{\pgfqpoint{2.044236in}{2.420209in}}{\pgfqpoint{2.047508in}{2.412309in}}{\pgfqpoint{2.053332in}{2.406485in}}%
\pgfpathcurveto{\pgfqpoint{2.059156in}{2.400662in}}{\pgfqpoint{2.067056in}{2.397389in}}{\pgfqpoint{2.075292in}{2.397389in}}%
\pgfpathclose%
\pgfusepath{stroke,fill}%
\end{pgfscope}%
\begin{pgfscope}%
\pgfpathrectangle{\pgfqpoint{0.100000in}{0.212622in}}{\pgfqpoint{3.696000in}{3.696000in}}%
\pgfusepath{clip}%
\pgfsetbuttcap%
\pgfsetroundjoin%
\definecolor{currentfill}{rgb}{0.121569,0.466667,0.705882}%
\pgfsetfillcolor{currentfill}%
\pgfsetfillopacity{0.955541}%
\pgfsetlinewidth{1.003750pt}%
\definecolor{currentstroke}{rgb}{0.121569,0.466667,0.705882}%
\pgfsetstrokecolor{currentstroke}%
\pgfsetstrokeopacity{0.955541}%
\pgfsetdash{}{0pt}%
\pgfpathmoveto{\pgfqpoint{2.075292in}{2.397389in}}%
\pgfpathcurveto{\pgfqpoint{2.083528in}{2.397389in}}{\pgfqpoint{2.091428in}{2.400661in}}{\pgfqpoint{2.097252in}{2.406485in}}%
\pgfpathcurveto{\pgfqpoint{2.103076in}{2.412309in}}{\pgfqpoint{2.106348in}{2.420209in}}{\pgfqpoint{2.106348in}{2.428446in}}%
\pgfpathcurveto{\pgfqpoint{2.106348in}{2.436682in}}{\pgfqpoint{2.103076in}{2.444582in}}{\pgfqpoint{2.097252in}{2.450406in}}%
\pgfpathcurveto{\pgfqpoint{2.091428in}{2.456230in}}{\pgfqpoint{2.083528in}{2.459502in}}{\pgfqpoint{2.075292in}{2.459502in}}%
\pgfpathcurveto{\pgfqpoint{2.067056in}{2.459502in}}{\pgfqpoint{2.059156in}{2.456230in}}{\pgfqpoint{2.053332in}{2.450406in}}%
\pgfpathcurveto{\pgfqpoint{2.047508in}{2.444582in}}{\pgfqpoint{2.044235in}{2.436682in}}{\pgfqpoint{2.044235in}{2.428446in}}%
\pgfpathcurveto{\pgfqpoint{2.044235in}{2.420209in}}{\pgfqpoint{2.047508in}{2.412309in}}{\pgfqpoint{2.053332in}{2.406485in}}%
\pgfpathcurveto{\pgfqpoint{2.059156in}{2.400661in}}{\pgfqpoint{2.067056in}{2.397389in}}{\pgfqpoint{2.075292in}{2.397389in}}%
\pgfpathclose%
\pgfusepath{stroke,fill}%
\end{pgfscope}%
\begin{pgfscope}%
\pgfpathrectangle{\pgfqpoint{0.100000in}{0.212622in}}{\pgfqpoint{3.696000in}{3.696000in}}%
\pgfusepath{clip}%
\pgfsetbuttcap%
\pgfsetroundjoin%
\definecolor{currentfill}{rgb}{0.121569,0.466667,0.705882}%
\pgfsetfillcolor{currentfill}%
\pgfsetfillopacity{0.955541}%
\pgfsetlinewidth{1.003750pt}%
\definecolor{currentstroke}{rgb}{0.121569,0.466667,0.705882}%
\pgfsetstrokecolor{currentstroke}%
\pgfsetstrokeopacity{0.955541}%
\pgfsetdash{}{0pt}%
\pgfpathmoveto{\pgfqpoint{2.075292in}{2.397389in}}%
\pgfpathcurveto{\pgfqpoint{2.083528in}{2.397389in}}{\pgfqpoint{2.091428in}{2.400661in}}{\pgfqpoint{2.097252in}{2.406485in}}%
\pgfpathcurveto{\pgfqpoint{2.103076in}{2.412309in}}{\pgfqpoint{2.106348in}{2.420209in}}{\pgfqpoint{2.106348in}{2.428446in}}%
\pgfpathcurveto{\pgfqpoint{2.106348in}{2.436682in}}{\pgfqpoint{2.103076in}{2.444582in}}{\pgfqpoint{2.097252in}{2.450406in}}%
\pgfpathcurveto{\pgfqpoint{2.091428in}{2.456230in}}{\pgfqpoint{2.083528in}{2.459502in}}{\pgfqpoint{2.075292in}{2.459502in}}%
\pgfpathcurveto{\pgfqpoint{2.067056in}{2.459502in}}{\pgfqpoint{2.059156in}{2.456230in}}{\pgfqpoint{2.053332in}{2.450406in}}%
\pgfpathcurveto{\pgfqpoint{2.047508in}{2.444582in}}{\pgfqpoint{2.044235in}{2.436682in}}{\pgfqpoint{2.044235in}{2.428446in}}%
\pgfpathcurveto{\pgfqpoint{2.044235in}{2.420209in}}{\pgfqpoint{2.047508in}{2.412309in}}{\pgfqpoint{2.053332in}{2.406485in}}%
\pgfpathcurveto{\pgfqpoint{2.059156in}{2.400661in}}{\pgfqpoint{2.067056in}{2.397389in}}{\pgfqpoint{2.075292in}{2.397389in}}%
\pgfpathclose%
\pgfusepath{stroke,fill}%
\end{pgfscope}%
\begin{pgfscope}%
\pgfpathrectangle{\pgfqpoint{0.100000in}{0.212622in}}{\pgfqpoint{3.696000in}{3.696000in}}%
\pgfusepath{clip}%
\pgfsetbuttcap%
\pgfsetroundjoin%
\definecolor{currentfill}{rgb}{0.121569,0.466667,0.705882}%
\pgfsetfillcolor{currentfill}%
\pgfsetfillopacity{0.955541}%
\pgfsetlinewidth{1.003750pt}%
\definecolor{currentstroke}{rgb}{0.121569,0.466667,0.705882}%
\pgfsetstrokecolor{currentstroke}%
\pgfsetstrokeopacity{0.955541}%
\pgfsetdash{}{0pt}%
\pgfpathmoveto{\pgfqpoint{2.075292in}{2.397389in}}%
\pgfpathcurveto{\pgfqpoint{2.083528in}{2.397389in}}{\pgfqpoint{2.091428in}{2.400661in}}{\pgfqpoint{2.097252in}{2.406485in}}%
\pgfpathcurveto{\pgfqpoint{2.103076in}{2.412309in}}{\pgfqpoint{2.106348in}{2.420209in}}{\pgfqpoint{2.106348in}{2.428445in}}%
\pgfpathcurveto{\pgfqpoint{2.106348in}{2.436682in}}{\pgfqpoint{2.103076in}{2.444582in}}{\pgfqpoint{2.097252in}{2.450406in}}%
\pgfpathcurveto{\pgfqpoint{2.091428in}{2.456230in}}{\pgfqpoint{2.083528in}{2.459502in}}{\pgfqpoint{2.075292in}{2.459502in}}%
\pgfpathcurveto{\pgfqpoint{2.067055in}{2.459502in}}{\pgfqpoint{2.059155in}{2.456230in}}{\pgfqpoint{2.053331in}{2.450406in}}%
\pgfpathcurveto{\pgfqpoint{2.047507in}{2.444582in}}{\pgfqpoint{2.044235in}{2.436682in}}{\pgfqpoint{2.044235in}{2.428445in}}%
\pgfpathcurveto{\pgfqpoint{2.044235in}{2.420209in}}{\pgfqpoint{2.047507in}{2.412309in}}{\pgfqpoint{2.053331in}{2.406485in}}%
\pgfpathcurveto{\pgfqpoint{2.059155in}{2.400661in}}{\pgfqpoint{2.067055in}{2.397389in}}{\pgfqpoint{2.075292in}{2.397389in}}%
\pgfpathclose%
\pgfusepath{stroke,fill}%
\end{pgfscope}%
\begin{pgfscope}%
\pgfpathrectangle{\pgfqpoint{0.100000in}{0.212622in}}{\pgfqpoint{3.696000in}{3.696000in}}%
\pgfusepath{clip}%
\pgfsetbuttcap%
\pgfsetroundjoin%
\definecolor{currentfill}{rgb}{0.121569,0.466667,0.705882}%
\pgfsetfillcolor{currentfill}%
\pgfsetfillopacity{0.955541}%
\pgfsetlinewidth{1.003750pt}%
\definecolor{currentstroke}{rgb}{0.121569,0.466667,0.705882}%
\pgfsetstrokecolor{currentstroke}%
\pgfsetstrokeopacity{0.955541}%
\pgfsetdash{}{0pt}%
\pgfpathmoveto{\pgfqpoint{2.075291in}{2.397389in}}%
\pgfpathcurveto{\pgfqpoint{2.083527in}{2.397389in}}{\pgfqpoint{2.091427in}{2.400661in}}{\pgfqpoint{2.097251in}{2.406485in}}%
\pgfpathcurveto{\pgfqpoint{2.103075in}{2.412309in}}{\pgfqpoint{2.106348in}{2.420209in}}{\pgfqpoint{2.106348in}{2.428445in}}%
\pgfpathcurveto{\pgfqpoint{2.106348in}{2.436681in}}{\pgfqpoint{2.103075in}{2.444582in}}{\pgfqpoint{2.097251in}{2.450405in}}%
\pgfpathcurveto{\pgfqpoint{2.091427in}{2.456229in}}{\pgfqpoint{2.083527in}{2.459502in}}{\pgfqpoint{2.075291in}{2.459502in}}%
\pgfpathcurveto{\pgfqpoint{2.067055in}{2.459502in}}{\pgfqpoint{2.059155in}{2.456229in}}{\pgfqpoint{2.053331in}{2.450405in}}%
\pgfpathcurveto{\pgfqpoint{2.047507in}{2.444582in}}{\pgfqpoint{2.044235in}{2.436681in}}{\pgfqpoint{2.044235in}{2.428445in}}%
\pgfpathcurveto{\pgfqpoint{2.044235in}{2.420209in}}{\pgfqpoint{2.047507in}{2.412309in}}{\pgfqpoint{2.053331in}{2.406485in}}%
\pgfpathcurveto{\pgfqpoint{2.059155in}{2.400661in}}{\pgfqpoint{2.067055in}{2.397389in}}{\pgfqpoint{2.075291in}{2.397389in}}%
\pgfpathclose%
\pgfusepath{stroke,fill}%
\end{pgfscope}%
\begin{pgfscope}%
\pgfpathrectangle{\pgfqpoint{0.100000in}{0.212622in}}{\pgfqpoint{3.696000in}{3.696000in}}%
\pgfusepath{clip}%
\pgfsetbuttcap%
\pgfsetroundjoin%
\definecolor{currentfill}{rgb}{0.121569,0.466667,0.705882}%
\pgfsetfillcolor{currentfill}%
\pgfsetfillopacity{0.955542}%
\pgfsetlinewidth{1.003750pt}%
\definecolor{currentstroke}{rgb}{0.121569,0.466667,0.705882}%
\pgfsetstrokecolor{currentstroke}%
\pgfsetstrokeopacity{0.955542}%
\pgfsetdash{}{0pt}%
\pgfpathmoveto{\pgfqpoint{2.075290in}{2.397388in}}%
\pgfpathcurveto{\pgfqpoint{2.083527in}{2.397388in}}{\pgfqpoint{2.091427in}{2.400661in}}{\pgfqpoint{2.097251in}{2.406484in}}%
\pgfpathcurveto{\pgfqpoint{2.103074in}{2.412308in}}{\pgfqpoint{2.106347in}{2.420208in}}{\pgfqpoint{2.106347in}{2.428445in}}%
\pgfpathcurveto{\pgfqpoint{2.106347in}{2.436681in}}{\pgfqpoint{2.103074in}{2.444581in}}{\pgfqpoint{2.097251in}{2.450405in}}%
\pgfpathcurveto{\pgfqpoint{2.091427in}{2.456229in}}{\pgfqpoint{2.083527in}{2.459501in}}{\pgfqpoint{2.075290in}{2.459501in}}%
\pgfpathcurveto{\pgfqpoint{2.067054in}{2.459501in}}{\pgfqpoint{2.059154in}{2.456229in}}{\pgfqpoint{2.053330in}{2.450405in}}%
\pgfpathcurveto{\pgfqpoint{2.047506in}{2.444581in}}{\pgfqpoint{2.044234in}{2.436681in}}{\pgfqpoint{2.044234in}{2.428445in}}%
\pgfpathcurveto{\pgfqpoint{2.044234in}{2.420208in}}{\pgfqpoint{2.047506in}{2.412308in}}{\pgfqpoint{2.053330in}{2.406484in}}%
\pgfpathcurveto{\pgfqpoint{2.059154in}{2.400661in}}{\pgfqpoint{2.067054in}{2.397388in}}{\pgfqpoint{2.075290in}{2.397388in}}%
\pgfpathclose%
\pgfusepath{stroke,fill}%
\end{pgfscope}%
\begin{pgfscope}%
\pgfpathrectangle{\pgfqpoint{0.100000in}{0.212622in}}{\pgfqpoint{3.696000in}{3.696000in}}%
\pgfusepath{clip}%
\pgfsetbuttcap%
\pgfsetroundjoin%
\definecolor{currentfill}{rgb}{0.121569,0.466667,0.705882}%
\pgfsetfillcolor{currentfill}%
\pgfsetfillopacity{0.955543}%
\pgfsetlinewidth{1.003750pt}%
\definecolor{currentstroke}{rgb}{0.121569,0.466667,0.705882}%
\pgfsetstrokecolor{currentstroke}%
\pgfsetstrokeopacity{0.955543}%
\pgfsetdash{}{0pt}%
\pgfpathmoveto{\pgfqpoint{2.075289in}{2.397387in}}%
\pgfpathcurveto{\pgfqpoint{2.083525in}{2.397387in}}{\pgfqpoint{2.091425in}{2.400660in}}{\pgfqpoint{2.097249in}{2.406484in}}%
\pgfpathcurveto{\pgfqpoint{2.103073in}{2.412308in}}{\pgfqpoint{2.106345in}{2.420208in}}{\pgfqpoint{2.106345in}{2.428444in}}%
\pgfpathcurveto{\pgfqpoint{2.106345in}{2.436680in}}{\pgfqpoint{2.103073in}{2.444580in}}{\pgfqpoint{2.097249in}{2.450404in}}%
\pgfpathcurveto{\pgfqpoint{2.091425in}{2.456228in}}{\pgfqpoint{2.083525in}{2.459500in}}{\pgfqpoint{2.075289in}{2.459500in}}%
\pgfpathcurveto{\pgfqpoint{2.067052in}{2.459500in}}{\pgfqpoint{2.059152in}{2.456228in}}{\pgfqpoint{2.053328in}{2.450404in}}%
\pgfpathcurveto{\pgfqpoint{2.047504in}{2.444580in}}{\pgfqpoint{2.044232in}{2.436680in}}{\pgfqpoint{2.044232in}{2.428444in}}%
\pgfpathcurveto{\pgfqpoint{2.044232in}{2.420208in}}{\pgfqpoint{2.047504in}{2.412308in}}{\pgfqpoint{2.053328in}{2.406484in}}%
\pgfpathcurveto{\pgfqpoint{2.059152in}{2.400660in}}{\pgfqpoint{2.067052in}{2.397387in}}{\pgfqpoint{2.075289in}{2.397387in}}%
\pgfpathclose%
\pgfusepath{stroke,fill}%
\end{pgfscope}%
\begin{pgfscope}%
\pgfpathrectangle{\pgfqpoint{0.100000in}{0.212622in}}{\pgfqpoint{3.696000in}{3.696000in}}%
\pgfusepath{clip}%
\pgfsetbuttcap%
\pgfsetroundjoin%
\definecolor{currentfill}{rgb}{0.121569,0.466667,0.705882}%
\pgfsetfillcolor{currentfill}%
\pgfsetfillopacity{0.955544}%
\pgfsetlinewidth{1.003750pt}%
\definecolor{currentstroke}{rgb}{0.121569,0.466667,0.705882}%
\pgfsetstrokecolor{currentstroke}%
\pgfsetstrokeopacity{0.955544}%
\pgfsetdash{}{0pt}%
\pgfpathmoveto{\pgfqpoint{2.075286in}{2.397386in}}%
\pgfpathcurveto{\pgfqpoint{2.083522in}{2.397386in}}{\pgfqpoint{2.091422in}{2.400658in}}{\pgfqpoint{2.097246in}{2.406482in}}%
\pgfpathcurveto{\pgfqpoint{2.103070in}{2.412306in}}{\pgfqpoint{2.106342in}{2.420206in}}{\pgfqpoint{2.106342in}{2.428442in}}%
\pgfpathcurveto{\pgfqpoint{2.106342in}{2.436679in}}{\pgfqpoint{2.103070in}{2.444579in}}{\pgfqpoint{2.097246in}{2.450403in}}%
\pgfpathcurveto{\pgfqpoint{2.091422in}{2.456226in}}{\pgfqpoint{2.083522in}{2.459499in}}{\pgfqpoint{2.075286in}{2.459499in}}%
\pgfpathcurveto{\pgfqpoint{2.067049in}{2.459499in}}{\pgfqpoint{2.059149in}{2.456226in}}{\pgfqpoint{2.053325in}{2.450403in}}%
\pgfpathcurveto{\pgfqpoint{2.047501in}{2.444579in}}{\pgfqpoint{2.044229in}{2.436679in}}{\pgfqpoint{2.044229in}{2.428442in}}%
\pgfpathcurveto{\pgfqpoint{2.044229in}{2.420206in}}{\pgfqpoint{2.047501in}{2.412306in}}{\pgfqpoint{2.053325in}{2.406482in}}%
\pgfpathcurveto{\pgfqpoint{2.059149in}{2.400658in}}{\pgfqpoint{2.067049in}{2.397386in}}{\pgfqpoint{2.075286in}{2.397386in}}%
\pgfpathclose%
\pgfusepath{stroke,fill}%
\end{pgfscope}%
\begin{pgfscope}%
\pgfpathrectangle{\pgfqpoint{0.100000in}{0.212622in}}{\pgfqpoint{3.696000in}{3.696000in}}%
\pgfusepath{clip}%
\pgfsetbuttcap%
\pgfsetroundjoin%
\definecolor{currentfill}{rgb}{0.121569,0.466667,0.705882}%
\pgfsetfillcolor{currentfill}%
\pgfsetfillopacity{0.955546}%
\pgfsetlinewidth{1.003750pt}%
\definecolor{currentstroke}{rgb}{0.121569,0.466667,0.705882}%
\pgfsetstrokecolor{currentstroke}%
\pgfsetstrokeopacity{0.955546}%
\pgfsetdash{}{0pt}%
\pgfpathmoveto{\pgfqpoint{2.002196in}{2.417468in}}%
\pgfpathcurveto{\pgfqpoint{2.010433in}{2.417468in}}{\pgfqpoint{2.018333in}{2.420740in}}{\pgfqpoint{2.024157in}{2.426564in}}%
\pgfpathcurveto{\pgfqpoint{2.029980in}{2.432388in}}{\pgfqpoint{2.033253in}{2.440288in}}{\pgfqpoint{2.033253in}{2.448525in}}%
\pgfpathcurveto{\pgfqpoint{2.033253in}{2.456761in}}{\pgfqpoint{2.029980in}{2.464661in}}{\pgfqpoint{2.024157in}{2.470485in}}%
\pgfpathcurveto{\pgfqpoint{2.018333in}{2.476309in}}{\pgfqpoint{2.010433in}{2.479581in}}{\pgfqpoint{2.002196in}{2.479581in}}%
\pgfpathcurveto{\pgfqpoint{1.993960in}{2.479581in}}{\pgfqpoint{1.986060in}{2.476309in}}{\pgfqpoint{1.980236in}{2.470485in}}%
\pgfpathcurveto{\pgfqpoint{1.974412in}{2.464661in}}{\pgfqpoint{1.971140in}{2.456761in}}{\pgfqpoint{1.971140in}{2.448525in}}%
\pgfpathcurveto{\pgfqpoint{1.971140in}{2.440288in}}{\pgfqpoint{1.974412in}{2.432388in}}{\pgfqpoint{1.980236in}{2.426564in}}%
\pgfpathcurveto{\pgfqpoint{1.986060in}{2.420740in}}{\pgfqpoint{1.993960in}{2.417468in}}{\pgfqpoint{2.002196in}{2.417468in}}%
\pgfpathclose%
\pgfusepath{stroke,fill}%
\end{pgfscope}%
\begin{pgfscope}%
\pgfpathrectangle{\pgfqpoint{0.100000in}{0.212622in}}{\pgfqpoint{3.696000in}{3.696000in}}%
\pgfusepath{clip}%
\pgfsetbuttcap%
\pgfsetroundjoin%
\definecolor{currentfill}{rgb}{0.121569,0.466667,0.705882}%
\pgfsetfillcolor{currentfill}%
\pgfsetfillopacity{0.955547}%
\pgfsetlinewidth{1.003750pt}%
\definecolor{currentstroke}{rgb}{0.121569,0.466667,0.705882}%
\pgfsetstrokecolor{currentstroke}%
\pgfsetstrokeopacity{0.955547}%
\pgfsetdash{}{0pt}%
\pgfpathmoveto{\pgfqpoint{2.075279in}{2.397383in}}%
\pgfpathcurveto{\pgfqpoint{2.083516in}{2.397383in}}{\pgfqpoint{2.091416in}{2.400655in}}{\pgfqpoint{2.097240in}{2.406479in}}%
\pgfpathcurveto{\pgfqpoint{2.103064in}{2.412303in}}{\pgfqpoint{2.106336in}{2.420203in}}{\pgfqpoint{2.106336in}{2.428439in}}%
\pgfpathcurveto{\pgfqpoint{2.106336in}{2.436676in}}{\pgfqpoint{2.103064in}{2.444576in}}{\pgfqpoint{2.097240in}{2.450400in}}%
\pgfpathcurveto{\pgfqpoint{2.091416in}{2.456224in}}{\pgfqpoint{2.083516in}{2.459496in}}{\pgfqpoint{2.075279in}{2.459496in}}%
\pgfpathcurveto{\pgfqpoint{2.067043in}{2.459496in}}{\pgfqpoint{2.059143in}{2.456224in}}{\pgfqpoint{2.053319in}{2.450400in}}%
\pgfpathcurveto{\pgfqpoint{2.047495in}{2.444576in}}{\pgfqpoint{2.044223in}{2.436676in}}{\pgfqpoint{2.044223in}{2.428439in}}%
\pgfpathcurveto{\pgfqpoint{2.044223in}{2.420203in}}{\pgfqpoint{2.047495in}{2.412303in}}{\pgfqpoint{2.053319in}{2.406479in}}%
\pgfpathcurveto{\pgfqpoint{2.059143in}{2.400655in}}{\pgfqpoint{2.067043in}{2.397383in}}{\pgfqpoint{2.075279in}{2.397383in}}%
\pgfpathclose%
\pgfusepath{stroke,fill}%
\end{pgfscope}%
\begin{pgfscope}%
\pgfpathrectangle{\pgfqpoint{0.100000in}{0.212622in}}{\pgfqpoint{3.696000in}{3.696000in}}%
\pgfusepath{clip}%
\pgfsetbuttcap%
\pgfsetroundjoin%
\definecolor{currentfill}{rgb}{0.121569,0.466667,0.705882}%
\pgfsetfillcolor{currentfill}%
\pgfsetfillopacity{0.955552}%
\pgfsetlinewidth{1.003750pt}%
\definecolor{currentstroke}{rgb}{0.121569,0.466667,0.705882}%
\pgfsetstrokecolor{currentstroke}%
\pgfsetstrokeopacity{0.955552}%
\pgfsetdash{}{0pt}%
\pgfpathmoveto{\pgfqpoint{2.075268in}{2.397378in}}%
\pgfpathcurveto{\pgfqpoint{2.083504in}{2.397378in}}{\pgfqpoint{2.091404in}{2.400650in}}{\pgfqpoint{2.097228in}{2.406474in}}%
\pgfpathcurveto{\pgfqpoint{2.103052in}{2.412298in}}{\pgfqpoint{2.106324in}{2.420198in}}{\pgfqpoint{2.106324in}{2.428434in}}%
\pgfpathcurveto{\pgfqpoint{2.106324in}{2.436671in}}{\pgfqpoint{2.103052in}{2.444571in}}{\pgfqpoint{2.097228in}{2.450395in}}%
\pgfpathcurveto{\pgfqpoint{2.091404in}{2.456219in}}{\pgfqpoint{2.083504in}{2.459491in}}{\pgfqpoint{2.075268in}{2.459491in}}%
\pgfpathcurveto{\pgfqpoint{2.067032in}{2.459491in}}{\pgfqpoint{2.059132in}{2.456219in}}{\pgfqpoint{2.053308in}{2.450395in}}%
\pgfpathcurveto{\pgfqpoint{2.047484in}{2.444571in}}{\pgfqpoint{2.044211in}{2.436671in}}{\pgfqpoint{2.044211in}{2.428434in}}%
\pgfpathcurveto{\pgfqpoint{2.044211in}{2.420198in}}{\pgfqpoint{2.047484in}{2.412298in}}{\pgfqpoint{2.053308in}{2.406474in}}%
\pgfpathcurveto{\pgfqpoint{2.059132in}{2.400650in}}{\pgfqpoint{2.067032in}{2.397378in}}{\pgfqpoint{2.075268in}{2.397378in}}%
\pgfpathclose%
\pgfusepath{stroke,fill}%
\end{pgfscope}%
\begin{pgfscope}%
\pgfpathrectangle{\pgfqpoint{0.100000in}{0.212622in}}{\pgfqpoint{3.696000in}{3.696000in}}%
\pgfusepath{clip}%
\pgfsetbuttcap%
\pgfsetroundjoin%
\definecolor{currentfill}{rgb}{0.121569,0.466667,0.705882}%
\pgfsetfillcolor{currentfill}%
\pgfsetfillopacity{0.955560}%
\pgfsetlinewidth{1.003750pt}%
\definecolor{currentstroke}{rgb}{0.121569,0.466667,0.705882}%
\pgfsetstrokecolor{currentstroke}%
\pgfsetstrokeopacity{0.955560}%
\pgfsetdash{}{0pt}%
\pgfpathmoveto{\pgfqpoint{2.075246in}{2.397369in}}%
\pgfpathcurveto{\pgfqpoint{2.083482in}{2.397369in}}{\pgfqpoint{2.091382in}{2.400641in}}{\pgfqpoint{2.097206in}{2.406465in}}%
\pgfpathcurveto{\pgfqpoint{2.103030in}{2.412289in}}{\pgfqpoint{2.106302in}{2.420189in}}{\pgfqpoint{2.106302in}{2.428425in}}%
\pgfpathcurveto{\pgfqpoint{2.106302in}{2.436661in}}{\pgfqpoint{2.103030in}{2.444561in}}{\pgfqpoint{2.097206in}{2.450385in}}%
\pgfpathcurveto{\pgfqpoint{2.091382in}{2.456209in}}{\pgfqpoint{2.083482in}{2.459482in}}{\pgfqpoint{2.075246in}{2.459482in}}%
\pgfpathcurveto{\pgfqpoint{2.067009in}{2.459482in}}{\pgfqpoint{2.059109in}{2.456209in}}{\pgfqpoint{2.053285in}{2.450385in}}%
\pgfpathcurveto{\pgfqpoint{2.047462in}{2.444561in}}{\pgfqpoint{2.044189in}{2.436661in}}{\pgfqpoint{2.044189in}{2.428425in}}%
\pgfpathcurveto{\pgfqpoint{2.044189in}{2.420189in}}{\pgfqpoint{2.047462in}{2.412289in}}{\pgfqpoint{2.053285in}{2.406465in}}%
\pgfpathcurveto{\pgfqpoint{2.059109in}{2.400641in}}{\pgfqpoint{2.067009in}{2.397369in}}{\pgfqpoint{2.075246in}{2.397369in}}%
\pgfpathclose%
\pgfusepath{stroke,fill}%
\end{pgfscope}%
\begin{pgfscope}%
\pgfpathrectangle{\pgfqpoint{0.100000in}{0.212622in}}{\pgfqpoint{3.696000in}{3.696000in}}%
\pgfusepath{clip}%
\pgfsetbuttcap%
\pgfsetroundjoin%
\definecolor{currentfill}{rgb}{0.121569,0.466667,0.705882}%
\pgfsetfillcolor{currentfill}%
\pgfsetfillopacity{0.955575}%
\pgfsetlinewidth{1.003750pt}%
\definecolor{currentstroke}{rgb}{0.121569,0.466667,0.705882}%
\pgfsetstrokecolor{currentstroke}%
\pgfsetstrokeopacity{0.955575}%
\pgfsetdash{}{0pt}%
\pgfpathmoveto{\pgfqpoint{2.075204in}{2.397352in}}%
\pgfpathcurveto{\pgfqpoint{2.083440in}{2.397352in}}{\pgfqpoint{2.091340in}{2.400624in}}{\pgfqpoint{2.097164in}{2.406448in}}%
\pgfpathcurveto{\pgfqpoint{2.102988in}{2.412272in}}{\pgfqpoint{2.106260in}{2.420172in}}{\pgfqpoint{2.106260in}{2.428408in}}%
\pgfpathcurveto{\pgfqpoint{2.106260in}{2.436645in}}{\pgfqpoint{2.102988in}{2.444545in}}{\pgfqpoint{2.097164in}{2.450369in}}%
\pgfpathcurveto{\pgfqpoint{2.091340in}{2.456193in}}{\pgfqpoint{2.083440in}{2.459465in}}{\pgfqpoint{2.075204in}{2.459465in}}%
\pgfpathcurveto{\pgfqpoint{2.066967in}{2.459465in}}{\pgfqpoint{2.059067in}{2.456193in}}{\pgfqpoint{2.053243in}{2.450369in}}%
\pgfpathcurveto{\pgfqpoint{2.047419in}{2.444545in}}{\pgfqpoint{2.044147in}{2.436645in}}{\pgfqpoint{2.044147in}{2.428408in}}%
\pgfpathcurveto{\pgfqpoint{2.044147in}{2.420172in}}{\pgfqpoint{2.047419in}{2.412272in}}{\pgfqpoint{2.053243in}{2.406448in}}%
\pgfpathcurveto{\pgfqpoint{2.059067in}{2.400624in}}{\pgfqpoint{2.066967in}{2.397352in}}{\pgfqpoint{2.075204in}{2.397352in}}%
\pgfpathclose%
\pgfusepath{stroke,fill}%
\end{pgfscope}%
\begin{pgfscope}%
\pgfpathrectangle{\pgfqpoint{0.100000in}{0.212622in}}{\pgfqpoint{3.696000in}{3.696000in}}%
\pgfusepath{clip}%
\pgfsetbuttcap%
\pgfsetroundjoin%
\definecolor{currentfill}{rgb}{0.121569,0.466667,0.705882}%
\pgfsetfillcolor{currentfill}%
\pgfsetfillopacity{0.955600}%
\pgfsetlinewidth{1.003750pt}%
\definecolor{currentstroke}{rgb}{0.121569,0.466667,0.705882}%
\pgfsetstrokecolor{currentstroke}%
\pgfsetstrokeopacity{0.955600}%
\pgfsetdash{}{0pt}%
\pgfpathmoveto{\pgfqpoint{2.075124in}{2.397323in}}%
\pgfpathcurveto{\pgfqpoint{2.083360in}{2.397323in}}{\pgfqpoint{2.091260in}{2.400595in}}{\pgfqpoint{2.097084in}{2.406419in}}%
\pgfpathcurveto{\pgfqpoint{2.102908in}{2.412243in}}{\pgfqpoint{2.106180in}{2.420143in}}{\pgfqpoint{2.106180in}{2.428379in}}%
\pgfpathcurveto{\pgfqpoint{2.106180in}{2.436615in}}{\pgfqpoint{2.102908in}{2.444515in}}{\pgfqpoint{2.097084in}{2.450339in}}%
\pgfpathcurveto{\pgfqpoint{2.091260in}{2.456163in}}{\pgfqpoint{2.083360in}{2.459436in}}{\pgfqpoint{2.075124in}{2.459436in}}%
\pgfpathcurveto{\pgfqpoint{2.066887in}{2.459436in}}{\pgfqpoint{2.058987in}{2.456163in}}{\pgfqpoint{2.053163in}{2.450339in}}%
\pgfpathcurveto{\pgfqpoint{2.047339in}{2.444515in}}{\pgfqpoint{2.044067in}{2.436615in}}{\pgfqpoint{2.044067in}{2.428379in}}%
\pgfpathcurveto{\pgfqpoint{2.044067in}{2.420143in}}{\pgfqpoint{2.047339in}{2.412243in}}{\pgfqpoint{2.053163in}{2.406419in}}%
\pgfpathcurveto{\pgfqpoint{2.058987in}{2.400595in}}{\pgfqpoint{2.066887in}{2.397323in}}{\pgfqpoint{2.075124in}{2.397323in}}%
\pgfpathclose%
\pgfusepath{stroke,fill}%
\end{pgfscope}%
\begin{pgfscope}%
\pgfpathrectangle{\pgfqpoint{0.100000in}{0.212622in}}{\pgfqpoint{3.696000in}{3.696000in}}%
\pgfusepath{clip}%
\pgfsetbuttcap%
\pgfsetroundjoin%
\definecolor{currentfill}{rgb}{0.121569,0.466667,0.705882}%
\pgfsetfillcolor{currentfill}%
\pgfsetfillopacity{0.955613}%
\pgfsetlinewidth{1.003750pt}%
\definecolor{currentstroke}{rgb}{0.121569,0.466667,0.705882}%
\pgfsetstrokecolor{currentstroke}%
\pgfsetstrokeopacity{0.955613}%
\pgfsetdash{}{0pt}%
\pgfpathmoveto{\pgfqpoint{2.002645in}{2.417222in}}%
\pgfpathcurveto{\pgfqpoint{2.010881in}{2.417222in}}{\pgfqpoint{2.018781in}{2.420494in}}{\pgfqpoint{2.024605in}{2.426318in}}%
\pgfpathcurveto{\pgfqpoint{2.030429in}{2.432142in}}{\pgfqpoint{2.033702in}{2.440042in}}{\pgfqpoint{2.033702in}{2.448278in}}%
\pgfpathcurveto{\pgfqpoint{2.033702in}{2.456514in}}{\pgfqpoint{2.030429in}{2.464415in}}{\pgfqpoint{2.024605in}{2.470238in}}%
\pgfpathcurveto{\pgfqpoint{2.018781in}{2.476062in}}{\pgfqpoint{2.010881in}{2.479335in}}{\pgfqpoint{2.002645in}{2.479335in}}%
\pgfpathcurveto{\pgfqpoint{1.994409in}{2.479335in}}{\pgfqpoint{1.986509in}{2.476062in}}{\pgfqpoint{1.980685in}{2.470238in}}%
\pgfpathcurveto{\pgfqpoint{1.974861in}{2.464415in}}{\pgfqpoint{1.971589in}{2.456514in}}{\pgfqpoint{1.971589in}{2.448278in}}%
\pgfpathcurveto{\pgfqpoint{1.971589in}{2.440042in}}{\pgfqpoint{1.974861in}{2.432142in}}{\pgfqpoint{1.980685in}{2.426318in}}%
\pgfpathcurveto{\pgfqpoint{1.986509in}{2.420494in}}{\pgfqpoint{1.994409in}{2.417222in}}{\pgfqpoint{2.002645in}{2.417222in}}%
\pgfpathclose%
\pgfusepath{stroke,fill}%
\end{pgfscope}%
\begin{pgfscope}%
\pgfpathrectangle{\pgfqpoint{0.100000in}{0.212622in}}{\pgfqpoint{3.696000in}{3.696000in}}%
\pgfusepath{clip}%
\pgfsetbuttcap%
\pgfsetroundjoin%
\definecolor{currentfill}{rgb}{0.121569,0.466667,0.705882}%
\pgfsetfillcolor{currentfill}%
\pgfsetfillopacity{0.955643}%
\pgfsetlinewidth{1.003750pt}%
\definecolor{currentstroke}{rgb}{0.121569,0.466667,0.705882}%
\pgfsetstrokecolor{currentstroke}%
\pgfsetstrokeopacity{0.955643}%
\pgfsetdash{}{0pt}%
\pgfpathmoveto{\pgfqpoint{2.074973in}{2.397272in}}%
\pgfpathcurveto{\pgfqpoint{2.083209in}{2.397272in}}{\pgfqpoint{2.091109in}{2.400544in}}{\pgfqpoint{2.096933in}{2.406368in}}%
\pgfpathcurveto{\pgfqpoint{2.102757in}{2.412192in}}{\pgfqpoint{2.106029in}{2.420092in}}{\pgfqpoint{2.106029in}{2.428329in}}%
\pgfpathcurveto{\pgfqpoint{2.106029in}{2.436565in}}{\pgfqpoint{2.102757in}{2.444465in}}{\pgfqpoint{2.096933in}{2.450289in}}%
\pgfpathcurveto{\pgfqpoint{2.091109in}{2.456113in}}{\pgfqpoint{2.083209in}{2.459385in}}{\pgfqpoint{2.074973in}{2.459385in}}%
\pgfpathcurveto{\pgfqpoint{2.066736in}{2.459385in}}{\pgfqpoint{2.058836in}{2.456113in}}{\pgfqpoint{2.053012in}{2.450289in}}%
\pgfpathcurveto{\pgfqpoint{2.047189in}{2.444465in}}{\pgfqpoint{2.043916in}{2.436565in}}{\pgfqpoint{2.043916in}{2.428329in}}%
\pgfpathcurveto{\pgfqpoint{2.043916in}{2.420092in}}{\pgfqpoint{2.047189in}{2.412192in}}{\pgfqpoint{2.053012in}{2.406368in}}%
\pgfpathcurveto{\pgfqpoint{2.058836in}{2.400544in}}{\pgfqpoint{2.066736in}{2.397272in}}{\pgfqpoint{2.074973in}{2.397272in}}%
\pgfpathclose%
\pgfusepath{stroke,fill}%
\end{pgfscope}%
\begin{pgfscope}%
\pgfpathrectangle{\pgfqpoint{0.100000in}{0.212622in}}{\pgfqpoint{3.696000in}{3.696000in}}%
\pgfusepath{clip}%
\pgfsetbuttcap%
\pgfsetroundjoin%
\definecolor{currentfill}{rgb}{0.121569,0.466667,0.705882}%
\pgfsetfillcolor{currentfill}%
\pgfsetfillopacity{0.955651}%
\pgfsetlinewidth{1.003750pt}%
\definecolor{currentstroke}{rgb}{0.121569,0.466667,0.705882}%
\pgfsetstrokecolor{currentstroke}%
\pgfsetstrokeopacity{0.955651}%
\pgfsetdash{}{0pt}%
\pgfpathmoveto{\pgfqpoint{2.574098in}{1.184949in}}%
\pgfpathcurveto{\pgfqpoint{2.582335in}{1.184949in}}{\pgfqpoint{2.590235in}{1.188221in}}{\pgfqpoint{2.596059in}{1.194045in}}%
\pgfpathcurveto{\pgfqpoint{2.601882in}{1.199869in}}{\pgfqpoint{2.605155in}{1.207769in}}{\pgfqpoint{2.605155in}{1.216005in}}%
\pgfpathcurveto{\pgfqpoint{2.605155in}{1.224242in}}{\pgfqpoint{2.601882in}{1.232142in}}{\pgfqpoint{2.596059in}{1.237966in}}%
\pgfpathcurveto{\pgfqpoint{2.590235in}{1.243789in}}{\pgfqpoint{2.582335in}{1.247062in}}{\pgfqpoint{2.574098in}{1.247062in}}%
\pgfpathcurveto{\pgfqpoint{2.565862in}{1.247062in}}{\pgfqpoint{2.557962in}{1.243789in}}{\pgfqpoint{2.552138in}{1.237966in}}%
\pgfpathcurveto{\pgfqpoint{2.546314in}{1.232142in}}{\pgfqpoint{2.543042in}{1.224242in}}{\pgfqpoint{2.543042in}{1.216005in}}%
\pgfpathcurveto{\pgfqpoint{2.543042in}{1.207769in}}{\pgfqpoint{2.546314in}{1.199869in}}{\pgfqpoint{2.552138in}{1.194045in}}%
\pgfpathcurveto{\pgfqpoint{2.557962in}{1.188221in}}{\pgfqpoint{2.565862in}{1.184949in}}{\pgfqpoint{2.574098in}{1.184949in}}%
\pgfpathclose%
\pgfusepath{stroke,fill}%
\end{pgfscope}%
\begin{pgfscope}%
\pgfpathrectangle{\pgfqpoint{0.100000in}{0.212622in}}{\pgfqpoint{3.696000in}{3.696000in}}%
\pgfusepath{clip}%
\pgfsetbuttcap%
\pgfsetroundjoin%
\definecolor{currentfill}{rgb}{0.121569,0.466667,0.705882}%
\pgfsetfillcolor{currentfill}%
\pgfsetfillopacity{0.955717}%
\pgfsetlinewidth{1.003750pt}%
\definecolor{currentstroke}{rgb}{0.121569,0.466667,0.705882}%
\pgfsetstrokecolor{currentstroke}%
\pgfsetstrokeopacity{0.955717}%
\pgfsetdash{}{0pt}%
\pgfpathmoveto{\pgfqpoint{2.074689in}{2.397188in}}%
\pgfpathcurveto{\pgfqpoint{2.082926in}{2.397188in}}{\pgfqpoint{2.090826in}{2.400460in}}{\pgfqpoint{2.096650in}{2.406284in}}%
\pgfpathcurveto{\pgfqpoint{2.102474in}{2.412108in}}{\pgfqpoint{2.105746in}{2.420008in}}{\pgfqpoint{2.105746in}{2.428244in}}%
\pgfpathcurveto{\pgfqpoint{2.105746in}{2.436481in}}{\pgfqpoint{2.102474in}{2.444381in}}{\pgfqpoint{2.096650in}{2.450205in}}%
\pgfpathcurveto{\pgfqpoint{2.090826in}{2.456029in}}{\pgfqpoint{2.082926in}{2.459301in}}{\pgfqpoint{2.074689in}{2.459301in}}%
\pgfpathcurveto{\pgfqpoint{2.066453in}{2.459301in}}{\pgfqpoint{2.058553in}{2.456029in}}{\pgfqpoint{2.052729in}{2.450205in}}%
\pgfpathcurveto{\pgfqpoint{2.046905in}{2.444381in}}{\pgfqpoint{2.043633in}{2.436481in}}{\pgfqpoint{2.043633in}{2.428244in}}%
\pgfpathcurveto{\pgfqpoint{2.043633in}{2.420008in}}{\pgfqpoint{2.046905in}{2.412108in}}{\pgfqpoint{2.052729in}{2.406284in}}%
\pgfpathcurveto{\pgfqpoint{2.058553in}{2.400460in}}{\pgfqpoint{2.066453in}{2.397188in}}{\pgfqpoint{2.074689in}{2.397188in}}%
\pgfpathclose%
\pgfusepath{stroke,fill}%
\end{pgfscope}%
\begin{pgfscope}%
\pgfpathrectangle{\pgfqpoint{0.100000in}{0.212622in}}{\pgfqpoint{3.696000in}{3.696000in}}%
\pgfusepath{clip}%
\pgfsetbuttcap%
\pgfsetroundjoin%
\definecolor{currentfill}{rgb}{0.121569,0.466667,0.705882}%
\pgfsetfillcolor{currentfill}%
\pgfsetfillopacity{0.955737}%
\pgfsetlinewidth{1.003750pt}%
\definecolor{currentstroke}{rgb}{0.121569,0.466667,0.705882}%
\pgfsetstrokecolor{currentstroke}%
\pgfsetstrokeopacity{0.955737}%
\pgfsetdash{}{0pt}%
\pgfpathmoveto{\pgfqpoint{2.003699in}{2.416624in}}%
\pgfpathcurveto{\pgfqpoint{2.011935in}{2.416624in}}{\pgfqpoint{2.019835in}{2.419896in}}{\pgfqpoint{2.025659in}{2.425720in}}%
\pgfpathcurveto{\pgfqpoint{2.031483in}{2.431544in}}{\pgfqpoint{2.034755in}{2.439444in}}{\pgfqpoint{2.034755in}{2.447680in}}%
\pgfpathcurveto{\pgfqpoint{2.034755in}{2.455916in}}{\pgfqpoint{2.031483in}{2.463816in}}{\pgfqpoint{2.025659in}{2.469640in}}%
\pgfpathcurveto{\pgfqpoint{2.019835in}{2.475464in}}{\pgfqpoint{2.011935in}{2.478737in}}{\pgfqpoint{2.003699in}{2.478737in}}%
\pgfpathcurveto{\pgfqpoint{1.995462in}{2.478737in}}{\pgfqpoint{1.987562in}{2.475464in}}{\pgfqpoint{1.981738in}{2.469640in}}%
\pgfpathcurveto{\pgfqpoint{1.975915in}{2.463816in}}{\pgfqpoint{1.972642in}{2.455916in}}{\pgfqpoint{1.972642in}{2.447680in}}%
\pgfpathcurveto{\pgfqpoint{1.972642in}{2.439444in}}{\pgfqpoint{1.975915in}{2.431544in}}{\pgfqpoint{1.981738in}{2.425720in}}%
\pgfpathcurveto{\pgfqpoint{1.987562in}{2.419896in}}{\pgfqpoint{1.995462in}{2.416624in}}{\pgfqpoint{2.003699in}{2.416624in}}%
\pgfpathclose%
\pgfusepath{stroke,fill}%
\end{pgfscope}%
\begin{pgfscope}%
\pgfpathrectangle{\pgfqpoint{0.100000in}{0.212622in}}{\pgfqpoint{3.696000in}{3.696000in}}%
\pgfusepath{clip}%
\pgfsetbuttcap%
\pgfsetroundjoin%
\definecolor{currentfill}{rgb}{0.121569,0.466667,0.705882}%
\pgfsetfillcolor{currentfill}%
\pgfsetfillopacity{0.955837}%
\pgfsetlinewidth{1.003750pt}%
\definecolor{currentstroke}{rgb}{0.121569,0.466667,0.705882}%
\pgfsetstrokecolor{currentstroke}%
\pgfsetstrokeopacity{0.955837}%
\pgfsetdash{}{0pt}%
\pgfpathmoveto{\pgfqpoint{2.005179in}{2.415711in}}%
\pgfpathcurveto{\pgfqpoint{2.013416in}{2.415711in}}{\pgfqpoint{2.021316in}{2.418984in}}{\pgfqpoint{2.027140in}{2.424807in}}%
\pgfpathcurveto{\pgfqpoint{2.032964in}{2.430631in}}{\pgfqpoint{2.036236in}{2.438531in}}{\pgfqpoint{2.036236in}{2.446768in}}%
\pgfpathcurveto{\pgfqpoint{2.036236in}{2.455004in}}{\pgfqpoint{2.032964in}{2.462904in}}{\pgfqpoint{2.027140in}{2.468728in}}%
\pgfpathcurveto{\pgfqpoint{2.021316in}{2.474552in}}{\pgfqpoint{2.013416in}{2.477824in}}{\pgfqpoint{2.005179in}{2.477824in}}%
\pgfpathcurveto{\pgfqpoint{1.996943in}{2.477824in}}{\pgfqpoint{1.989043in}{2.474552in}}{\pgfqpoint{1.983219in}{2.468728in}}%
\pgfpathcurveto{\pgfqpoint{1.977395in}{2.462904in}}{\pgfqpoint{1.974123in}{2.455004in}}{\pgfqpoint{1.974123in}{2.446768in}}%
\pgfpathcurveto{\pgfqpoint{1.974123in}{2.438531in}}{\pgfqpoint{1.977395in}{2.430631in}}{\pgfqpoint{1.983219in}{2.424807in}}%
\pgfpathcurveto{\pgfqpoint{1.989043in}{2.418984in}}{\pgfqpoint{1.996943in}{2.415711in}}{\pgfqpoint{2.005179in}{2.415711in}}%
\pgfpathclose%
\pgfusepath{stroke,fill}%
\end{pgfscope}%
\begin{pgfscope}%
\pgfpathrectangle{\pgfqpoint{0.100000in}{0.212622in}}{\pgfqpoint{3.696000in}{3.696000in}}%
\pgfusepath{clip}%
\pgfsetbuttcap%
\pgfsetroundjoin%
\definecolor{currentfill}{rgb}{0.121569,0.466667,0.705882}%
\pgfsetfillcolor{currentfill}%
\pgfsetfillopacity{0.955843}%
\pgfsetlinewidth{1.003750pt}%
\definecolor{currentstroke}{rgb}{0.121569,0.466667,0.705882}%
\pgfsetstrokecolor{currentstroke}%
\pgfsetstrokeopacity{0.955843}%
\pgfsetdash{}{0pt}%
\pgfpathmoveto{\pgfqpoint{2.074159in}{2.397052in}}%
\pgfpathcurveto{\pgfqpoint{2.082396in}{2.397052in}}{\pgfqpoint{2.090296in}{2.400324in}}{\pgfqpoint{2.096120in}{2.406148in}}%
\pgfpathcurveto{\pgfqpoint{2.101944in}{2.411972in}}{\pgfqpoint{2.105216in}{2.419872in}}{\pgfqpoint{2.105216in}{2.428108in}}%
\pgfpathcurveto{\pgfqpoint{2.105216in}{2.436344in}}{\pgfqpoint{2.101944in}{2.444244in}}{\pgfqpoint{2.096120in}{2.450068in}}%
\pgfpathcurveto{\pgfqpoint{2.090296in}{2.455892in}}{\pgfqpoint{2.082396in}{2.459165in}}{\pgfqpoint{2.074159in}{2.459165in}}%
\pgfpathcurveto{\pgfqpoint{2.065923in}{2.459165in}}{\pgfqpoint{2.058023in}{2.455892in}}{\pgfqpoint{2.052199in}{2.450068in}}%
\pgfpathcurveto{\pgfqpoint{2.046375in}{2.444244in}}{\pgfqpoint{2.043103in}{2.436344in}}{\pgfqpoint{2.043103in}{2.428108in}}%
\pgfpathcurveto{\pgfqpoint{2.043103in}{2.419872in}}{\pgfqpoint{2.046375in}{2.411972in}}{\pgfqpoint{2.052199in}{2.406148in}}%
\pgfpathcurveto{\pgfqpoint{2.058023in}{2.400324in}}{\pgfqpoint{2.065923in}{2.397052in}}{\pgfqpoint{2.074159in}{2.397052in}}%
\pgfpathclose%
\pgfusepath{stroke,fill}%
\end{pgfscope}%
\begin{pgfscope}%
\pgfpathrectangle{\pgfqpoint{0.100000in}{0.212622in}}{\pgfqpoint{3.696000in}{3.696000in}}%
\pgfusepath{clip}%
\pgfsetbuttcap%
\pgfsetroundjoin%
\definecolor{currentfill}{rgb}{0.121569,0.466667,0.705882}%
\pgfsetfillcolor{currentfill}%
\pgfsetfillopacity{0.955895}%
\pgfsetlinewidth{1.003750pt}%
\definecolor{currentstroke}{rgb}{0.121569,0.466667,0.705882}%
\pgfsetstrokecolor{currentstroke}%
\pgfsetstrokeopacity{0.955895}%
\pgfsetdash{}{0pt}%
\pgfpathmoveto{\pgfqpoint{2.005998in}{2.415247in}}%
\pgfpathcurveto{\pgfqpoint{2.014234in}{2.415247in}}{\pgfqpoint{2.022134in}{2.418519in}}{\pgfqpoint{2.027958in}{2.424343in}}%
\pgfpathcurveto{\pgfqpoint{2.033782in}{2.430167in}}{\pgfqpoint{2.037054in}{2.438067in}}{\pgfqpoint{2.037054in}{2.446304in}}%
\pgfpathcurveto{\pgfqpoint{2.037054in}{2.454540in}}{\pgfqpoint{2.033782in}{2.462440in}}{\pgfqpoint{2.027958in}{2.468264in}}%
\pgfpathcurveto{\pgfqpoint{2.022134in}{2.474088in}}{\pgfqpoint{2.014234in}{2.477360in}}{\pgfqpoint{2.005998in}{2.477360in}}%
\pgfpathcurveto{\pgfqpoint{1.997761in}{2.477360in}}{\pgfqpoint{1.989861in}{2.474088in}}{\pgfqpoint{1.984038in}{2.468264in}}%
\pgfpathcurveto{\pgfqpoint{1.978214in}{2.462440in}}{\pgfqpoint{1.974941in}{2.454540in}}{\pgfqpoint{1.974941in}{2.446304in}}%
\pgfpathcurveto{\pgfqpoint{1.974941in}{2.438067in}}{\pgfqpoint{1.978214in}{2.430167in}}{\pgfqpoint{1.984038in}{2.424343in}}%
\pgfpathcurveto{\pgfqpoint{1.989861in}{2.418519in}}{\pgfqpoint{1.997761in}{2.415247in}}{\pgfqpoint{2.005998in}{2.415247in}}%
\pgfpathclose%
\pgfusepath{stroke,fill}%
\end{pgfscope}%
\begin{pgfscope}%
\pgfpathrectangle{\pgfqpoint{0.100000in}{0.212622in}}{\pgfqpoint{3.696000in}{3.696000in}}%
\pgfusepath{clip}%
\pgfsetbuttcap%
\pgfsetroundjoin%
\definecolor{currentfill}{rgb}{0.121569,0.466667,0.705882}%
\pgfsetfillcolor{currentfill}%
\pgfsetfillopacity{0.956041}%
\pgfsetlinewidth{1.003750pt}%
\definecolor{currentstroke}{rgb}{0.121569,0.466667,0.705882}%
\pgfsetstrokecolor{currentstroke}%
\pgfsetstrokeopacity{0.956041}%
\pgfsetdash{}{0pt}%
\pgfpathmoveto{\pgfqpoint{2.007297in}{2.414591in}}%
\pgfpathcurveto{\pgfqpoint{2.015534in}{2.414591in}}{\pgfqpoint{2.023434in}{2.417863in}}{\pgfqpoint{2.029258in}{2.423687in}}%
\pgfpathcurveto{\pgfqpoint{2.035081in}{2.429511in}}{\pgfqpoint{2.038354in}{2.437411in}}{\pgfqpoint{2.038354in}{2.445647in}}%
\pgfpathcurveto{\pgfqpoint{2.038354in}{2.453884in}}{\pgfqpoint{2.035081in}{2.461784in}}{\pgfqpoint{2.029258in}{2.467608in}}%
\pgfpathcurveto{\pgfqpoint{2.023434in}{2.473432in}}{\pgfqpoint{2.015534in}{2.476704in}}{\pgfqpoint{2.007297in}{2.476704in}}%
\pgfpathcurveto{\pgfqpoint{1.999061in}{2.476704in}}{\pgfqpoint{1.991161in}{2.473432in}}{\pgfqpoint{1.985337in}{2.467608in}}%
\pgfpathcurveto{\pgfqpoint{1.979513in}{2.461784in}}{\pgfqpoint{1.976241in}{2.453884in}}{\pgfqpoint{1.976241in}{2.445647in}}%
\pgfpathcurveto{\pgfqpoint{1.976241in}{2.437411in}}{\pgfqpoint{1.979513in}{2.429511in}}{\pgfqpoint{1.985337in}{2.423687in}}%
\pgfpathcurveto{\pgfqpoint{1.991161in}{2.417863in}}{\pgfqpoint{1.999061in}{2.414591in}}{\pgfqpoint{2.007297in}{2.414591in}}%
\pgfpathclose%
\pgfusepath{stroke,fill}%
\end{pgfscope}%
\begin{pgfscope}%
\pgfpathrectangle{\pgfqpoint{0.100000in}{0.212622in}}{\pgfqpoint{3.696000in}{3.696000in}}%
\pgfusepath{clip}%
\pgfsetbuttcap%
\pgfsetroundjoin%
\definecolor{currentfill}{rgb}{0.121569,0.466667,0.705882}%
\pgfsetfillcolor{currentfill}%
\pgfsetfillopacity{0.956053}%
\pgfsetlinewidth{1.003750pt}%
\definecolor{currentstroke}{rgb}{0.121569,0.466667,0.705882}%
\pgfsetstrokecolor{currentstroke}%
\pgfsetstrokeopacity{0.956053}%
\pgfsetdash{}{0pt}%
\pgfpathmoveto{\pgfqpoint{2.073173in}{2.396826in}}%
\pgfpathcurveto{\pgfqpoint{2.081410in}{2.396826in}}{\pgfqpoint{2.089310in}{2.400099in}}{\pgfqpoint{2.095134in}{2.405923in}}%
\pgfpathcurveto{\pgfqpoint{2.100958in}{2.411747in}}{\pgfqpoint{2.104230in}{2.419647in}}{\pgfqpoint{2.104230in}{2.427883in}}%
\pgfpathcurveto{\pgfqpoint{2.104230in}{2.436119in}}{\pgfqpoint{2.100958in}{2.444019in}}{\pgfqpoint{2.095134in}{2.449843in}}%
\pgfpathcurveto{\pgfqpoint{2.089310in}{2.455667in}}{\pgfqpoint{2.081410in}{2.458939in}}{\pgfqpoint{2.073173in}{2.458939in}}%
\pgfpathcurveto{\pgfqpoint{2.064937in}{2.458939in}}{\pgfqpoint{2.057037in}{2.455667in}}{\pgfqpoint{2.051213in}{2.449843in}}%
\pgfpathcurveto{\pgfqpoint{2.045389in}{2.444019in}}{\pgfqpoint{2.042117in}{2.436119in}}{\pgfqpoint{2.042117in}{2.427883in}}%
\pgfpathcurveto{\pgfqpoint{2.042117in}{2.419647in}}{\pgfqpoint{2.045389in}{2.411747in}}{\pgfqpoint{2.051213in}{2.405923in}}%
\pgfpathcurveto{\pgfqpoint{2.057037in}{2.400099in}}{\pgfqpoint{2.064937in}{2.396826in}}{\pgfqpoint{2.073173in}{2.396826in}}%
\pgfpathclose%
\pgfusepath{stroke,fill}%
\end{pgfscope}%
\begin{pgfscope}%
\pgfpathrectangle{\pgfqpoint{0.100000in}{0.212622in}}{\pgfqpoint{3.696000in}{3.696000in}}%
\pgfusepath{clip}%
\pgfsetbuttcap%
\pgfsetroundjoin%
\definecolor{currentfill}{rgb}{0.121569,0.466667,0.705882}%
\pgfsetfillcolor{currentfill}%
\pgfsetfillopacity{0.956072}%
\pgfsetlinewidth{1.003750pt}%
\definecolor{currentstroke}{rgb}{0.121569,0.466667,0.705882}%
\pgfsetstrokecolor{currentstroke}%
\pgfsetstrokeopacity{0.956072}%
\pgfsetdash{}{0pt}%
\pgfpathmoveto{\pgfqpoint{1.449244in}{1.756528in}}%
\pgfpathcurveto{\pgfqpoint{1.457481in}{1.756528in}}{\pgfqpoint{1.465381in}{1.759801in}}{\pgfqpoint{1.471204in}{1.765625in}}%
\pgfpathcurveto{\pgfqpoint{1.477028in}{1.771449in}}{\pgfqpoint{1.480301in}{1.779349in}}{\pgfqpoint{1.480301in}{1.787585in}}%
\pgfpathcurveto{\pgfqpoint{1.480301in}{1.795821in}}{\pgfqpoint{1.477028in}{1.803721in}}{\pgfqpoint{1.471204in}{1.809545in}}%
\pgfpathcurveto{\pgfqpoint{1.465381in}{1.815369in}}{\pgfqpoint{1.457481in}{1.818641in}}{\pgfqpoint{1.449244in}{1.818641in}}%
\pgfpathcurveto{\pgfqpoint{1.441008in}{1.818641in}}{\pgfqpoint{1.433108in}{1.815369in}}{\pgfqpoint{1.427284in}{1.809545in}}%
\pgfpathcurveto{\pgfqpoint{1.421460in}{1.803721in}}{\pgfqpoint{1.418188in}{1.795821in}}{\pgfqpoint{1.418188in}{1.787585in}}%
\pgfpathcurveto{\pgfqpoint{1.418188in}{1.779349in}}{\pgfqpoint{1.421460in}{1.771449in}}{\pgfqpoint{1.427284in}{1.765625in}}%
\pgfpathcurveto{\pgfqpoint{1.433108in}{1.759801in}}{\pgfqpoint{1.441008in}{1.756528in}}{\pgfqpoint{1.449244in}{1.756528in}}%
\pgfpathclose%
\pgfusepath{stroke,fill}%
\end{pgfscope}%
\begin{pgfscope}%
\pgfpathrectangle{\pgfqpoint{0.100000in}{0.212622in}}{\pgfqpoint{3.696000in}{3.696000in}}%
\pgfusepath{clip}%
\pgfsetbuttcap%
\pgfsetroundjoin%
\definecolor{currentfill}{rgb}{0.121569,0.466667,0.705882}%
\pgfsetfillcolor{currentfill}%
\pgfsetfillopacity{0.956212}%
\pgfsetlinewidth{1.003750pt}%
\definecolor{currentstroke}{rgb}{0.121569,0.466667,0.705882}%
\pgfsetstrokecolor{currentstroke}%
\pgfsetstrokeopacity{0.956212}%
\pgfsetdash{}{0pt}%
\pgfpathmoveto{\pgfqpoint{2.008907in}{2.413714in}}%
\pgfpathcurveto{\pgfqpoint{2.017143in}{2.413714in}}{\pgfqpoint{2.025043in}{2.416986in}}{\pgfqpoint{2.030867in}{2.422810in}}%
\pgfpathcurveto{\pgfqpoint{2.036691in}{2.428634in}}{\pgfqpoint{2.039963in}{2.436534in}}{\pgfqpoint{2.039963in}{2.444770in}}%
\pgfpathcurveto{\pgfqpoint{2.039963in}{2.453007in}}{\pgfqpoint{2.036691in}{2.460907in}}{\pgfqpoint{2.030867in}{2.466731in}}%
\pgfpathcurveto{\pgfqpoint{2.025043in}{2.472555in}}{\pgfqpoint{2.017143in}{2.475827in}}{\pgfqpoint{2.008907in}{2.475827in}}%
\pgfpathcurveto{\pgfqpoint{2.000670in}{2.475827in}}{\pgfqpoint{1.992770in}{2.472555in}}{\pgfqpoint{1.986946in}{2.466731in}}%
\pgfpathcurveto{\pgfqpoint{1.981122in}{2.460907in}}{\pgfqpoint{1.977850in}{2.453007in}}{\pgfqpoint{1.977850in}{2.444770in}}%
\pgfpathcurveto{\pgfqpoint{1.977850in}{2.436534in}}{\pgfqpoint{1.981122in}{2.428634in}}{\pgfqpoint{1.986946in}{2.422810in}}%
\pgfpathcurveto{\pgfqpoint{1.992770in}{2.416986in}}{\pgfqpoint{2.000670in}{2.413714in}}{\pgfqpoint{2.008907in}{2.413714in}}%
\pgfpathclose%
\pgfusepath{stroke,fill}%
\end{pgfscope}%
\begin{pgfscope}%
\pgfpathrectangle{\pgfqpoint{0.100000in}{0.212622in}}{\pgfqpoint{3.696000in}{3.696000in}}%
\pgfusepath{clip}%
\pgfsetbuttcap%
\pgfsetroundjoin%
\definecolor{currentfill}{rgb}{0.121569,0.466667,0.705882}%
\pgfsetfillcolor{currentfill}%
\pgfsetfillopacity{0.956370}%
\pgfsetlinewidth{1.003750pt}%
\definecolor{currentstroke}{rgb}{0.121569,0.466667,0.705882}%
\pgfsetstrokecolor{currentstroke}%
\pgfsetstrokeopacity{0.956370}%
\pgfsetdash{}{0pt}%
\pgfpathmoveto{\pgfqpoint{2.011174in}{2.412318in}}%
\pgfpathcurveto{\pgfqpoint{2.019411in}{2.412318in}}{\pgfqpoint{2.027311in}{2.415590in}}{\pgfqpoint{2.033135in}{2.421414in}}%
\pgfpathcurveto{\pgfqpoint{2.038959in}{2.427238in}}{\pgfqpoint{2.042231in}{2.435138in}}{\pgfqpoint{2.042231in}{2.443374in}}%
\pgfpathcurveto{\pgfqpoint{2.042231in}{2.451611in}}{\pgfqpoint{2.038959in}{2.459511in}}{\pgfqpoint{2.033135in}{2.465335in}}%
\pgfpathcurveto{\pgfqpoint{2.027311in}{2.471159in}}{\pgfqpoint{2.019411in}{2.474431in}}{\pgfqpoint{2.011174in}{2.474431in}}%
\pgfpathcurveto{\pgfqpoint{2.002938in}{2.474431in}}{\pgfqpoint{1.995038in}{2.471159in}}{\pgfqpoint{1.989214in}{2.465335in}}%
\pgfpathcurveto{\pgfqpoint{1.983390in}{2.459511in}}{\pgfqpoint{1.980118in}{2.451611in}}{\pgfqpoint{1.980118in}{2.443374in}}%
\pgfpathcurveto{\pgfqpoint{1.980118in}{2.435138in}}{\pgfqpoint{1.983390in}{2.427238in}}{\pgfqpoint{1.989214in}{2.421414in}}%
\pgfpathcurveto{\pgfqpoint{1.995038in}{2.415590in}}{\pgfqpoint{2.002938in}{2.412318in}}{\pgfqpoint{2.011174in}{2.412318in}}%
\pgfpathclose%
\pgfusepath{stroke,fill}%
\end{pgfscope}%
\begin{pgfscope}%
\pgfpathrectangle{\pgfqpoint{0.100000in}{0.212622in}}{\pgfqpoint{3.696000in}{3.696000in}}%
\pgfusepath{clip}%
\pgfsetbuttcap%
\pgfsetroundjoin%
\definecolor{currentfill}{rgb}{0.121569,0.466667,0.705882}%
\pgfsetfillcolor{currentfill}%
\pgfsetfillopacity{0.956402}%
\pgfsetlinewidth{1.003750pt}%
\definecolor{currentstroke}{rgb}{0.121569,0.466667,0.705882}%
\pgfsetstrokecolor{currentstroke}%
\pgfsetstrokeopacity{0.956402}%
\pgfsetdash{}{0pt}%
\pgfpathmoveto{\pgfqpoint{2.071347in}{2.396487in}}%
\pgfpathcurveto{\pgfqpoint{2.079583in}{2.396487in}}{\pgfqpoint{2.087483in}{2.399759in}}{\pgfqpoint{2.093307in}{2.405583in}}%
\pgfpathcurveto{\pgfqpoint{2.099131in}{2.411407in}}{\pgfqpoint{2.102403in}{2.419307in}}{\pgfqpoint{2.102403in}{2.427543in}}%
\pgfpathcurveto{\pgfqpoint{2.102403in}{2.435780in}}{\pgfqpoint{2.099131in}{2.443680in}}{\pgfqpoint{2.093307in}{2.449504in}}%
\pgfpathcurveto{\pgfqpoint{2.087483in}{2.455328in}}{\pgfqpoint{2.079583in}{2.458600in}}{\pgfqpoint{2.071347in}{2.458600in}}%
\pgfpathcurveto{\pgfqpoint{2.063110in}{2.458600in}}{\pgfqpoint{2.055210in}{2.455328in}}{\pgfqpoint{2.049386in}{2.449504in}}%
\pgfpathcurveto{\pgfqpoint{2.043563in}{2.443680in}}{\pgfqpoint{2.040290in}{2.435780in}}{\pgfqpoint{2.040290in}{2.427543in}}%
\pgfpathcurveto{\pgfqpoint{2.040290in}{2.419307in}}{\pgfqpoint{2.043563in}{2.411407in}}{\pgfqpoint{2.049386in}{2.405583in}}%
\pgfpathcurveto{\pgfqpoint{2.055210in}{2.399759in}}{\pgfqpoint{2.063110in}{2.396487in}}{\pgfqpoint{2.071347in}{2.396487in}}%
\pgfpathclose%
\pgfusepath{stroke,fill}%
\end{pgfscope}%
\begin{pgfscope}%
\pgfpathrectangle{\pgfqpoint{0.100000in}{0.212622in}}{\pgfqpoint{3.696000in}{3.696000in}}%
\pgfusepath{clip}%
\pgfsetbuttcap%
\pgfsetroundjoin%
\definecolor{currentfill}{rgb}{0.121569,0.466667,0.705882}%
\pgfsetfillcolor{currentfill}%
\pgfsetfillopacity{0.956464}%
\pgfsetlinewidth{1.003750pt}%
\definecolor{currentstroke}{rgb}{0.121569,0.466667,0.705882}%
\pgfsetstrokecolor{currentstroke}%
\pgfsetstrokeopacity{0.956464}%
\pgfsetdash{}{0pt}%
\pgfpathmoveto{\pgfqpoint{2.012432in}{2.411629in}}%
\pgfpathcurveto{\pgfqpoint{2.020668in}{2.411629in}}{\pgfqpoint{2.028568in}{2.414901in}}{\pgfqpoint{2.034392in}{2.420725in}}%
\pgfpathcurveto{\pgfqpoint{2.040216in}{2.426549in}}{\pgfqpoint{2.043488in}{2.434449in}}{\pgfqpoint{2.043488in}{2.442686in}}%
\pgfpathcurveto{\pgfqpoint{2.043488in}{2.450922in}}{\pgfqpoint{2.040216in}{2.458822in}}{\pgfqpoint{2.034392in}{2.464646in}}%
\pgfpathcurveto{\pgfqpoint{2.028568in}{2.470470in}}{\pgfqpoint{2.020668in}{2.473742in}}{\pgfqpoint{2.012432in}{2.473742in}}%
\pgfpathcurveto{\pgfqpoint{2.004195in}{2.473742in}}{\pgfqpoint{1.996295in}{2.470470in}}{\pgfqpoint{1.990471in}{2.464646in}}%
\pgfpathcurveto{\pgfqpoint{1.984648in}{2.458822in}}{\pgfqpoint{1.981375in}{2.450922in}}{\pgfqpoint{1.981375in}{2.442686in}}%
\pgfpathcurveto{\pgfqpoint{1.981375in}{2.434449in}}{\pgfqpoint{1.984648in}{2.426549in}}{\pgfqpoint{1.990471in}{2.420725in}}%
\pgfpathcurveto{\pgfqpoint{1.996295in}{2.414901in}}{\pgfqpoint{2.004195in}{2.411629in}}{\pgfqpoint{2.012432in}{2.411629in}}%
\pgfpathclose%
\pgfusepath{stroke,fill}%
\end{pgfscope}%
\begin{pgfscope}%
\pgfpathrectangle{\pgfqpoint{0.100000in}{0.212622in}}{\pgfqpoint{3.696000in}{3.696000in}}%
\pgfusepath{clip}%
\pgfsetbuttcap%
\pgfsetroundjoin%
\definecolor{currentfill}{rgb}{0.121569,0.466667,0.705882}%
\pgfsetfillcolor{currentfill}%
\pgfsetfillopacity{0.956622}%
\pgfsetlinewidth{1.003750pt}%
\definecolor{currentstroke}{rgb}{0.121569,0.466667,0.705882}%
\pgfsetstrokecolor{currentstroke}%
\pgfsetstrokeopacity{0.956622}%
\pgfsetdash{}{0pt}%
\pgfpathmoveto{\pgfqpoint{2.014495in}{2.410815in}}%
\pgfpathcurveto{\pgfqpoint{2.022731in}{2.410815in}}{\pgfqpoint{2.030631in}{2.414087in}}{\pgfqpoint{2.036455in}{2.419911in}}%
\pgfpathcurveto{\pgfqpoint{2.042279in}{2.425735in}}{\pgfqpoint{2.045551in}{2.433635in}}{\pgfqpoint{2.045551in}{2.441872in}}%
\pgfpathcurveto{\pgfqpoint{2.045551in}{2.450108in}}{\pgfqpoint{2.042279in}{2.458008in}}{\pgfqpoint{2.036455in}{2.463832in}}%
\pgfpathcurveto{\pgfqpoint{2.030631in}{2.469656in}}{\pgfqpoint{2.022731in}{2.472928in}}{\pgfqpoint{2.014495in}{2.472928in}}%
\pgfpathcurveto{\pgfqpoint{2.006258in}{2.472928in}}{\pgfqpoint{1.998358in}{2.469656in}}{\pgfqpoint{1.992534in}{2.463832in}}%
\pgfpathcurveto{\pgfqpoint{1.986710in}{2.458008in}}{\pgfqpoint{1.983438in}{2.450108in}}{\pgfqpoint{1.983438in}{2.441872in}}%
\pgfpathcurveto{\pgfqpoint{1.983438in}{2.433635in}}{\pgfqpoint{1.986710in}{2.425735in}}{\pgfqpoint{1.992534in}{2.419911in}}%
\pgfpathcurveto{\pgfqpoint{1.998358in}{2.414087in}}{\pgfqpoint{2.006258in}{2.410815in}}{\pgfqpoint{2.014495in}{2.410815in}}%
\pgfpathclose%
\pgfusepath{stroke,fill}%
\end{pgfscope}%
\begin{pgfscope}%
\pgfpathrectangle{\pgfqpoint{0.100000in}{0.212622in}}{\pgfqpoint{3.696000in}{3.696000in}}%
\pgfusepath{clip}%
\pgfsetbuttcap%
\pgfsetroundjoin%
\definecolor{currentfill}{rgb}{0.121569,0.466667,0.705882}%
\pgfsetfillcolor{currentfill}%
\pgfsetfillopacity{0.956804}%
\pgfsetlinewidth{1.003750pt}%
\definecolor{currentstroke}{rgb}{0.121569,0.466667,0.705882}%
\pgfsetstrokecolor{currentstroke}%
\pgfsetstrokeopacity{0.956804}%
\pgfsetdash{}{0pt}%
\pgfpathmoveto{\pgfqpoint{2.016905in}{2.409833in}}%
\pgfpathcurveto{\pgfqpoint{2.025141in}{2.409833in}}{\pgfqpoint{2.033041in}{2.413106in}}{\pgfqpoint{2.038865in}{2.418930in}}%
\pgfpathcurveto{\pgfqpoint{2.044689in}{2.424754in}}{\pgfqpoint{2.047961in}{2.432654in}}{\pgfqpoint{2.047961in}{2.440890in}}%
\pgfpathcurveto{\pgfqpoint{2.047961in}{2.449126in}}{\pgfqpoint{2.044689in}{2.457026in}}{\pgfqpoint{2.038865in}{2.462850in}}%
\pgfpathcurveto{\pgfqpoint{2.033041in}{2.468674in}}{\pgfqpoint{2.025141in}{2.471946in}}{\pgfqpoint{2.016905in}{2.471946in}}%
\pgfpathcurveto{\pgfqpoint{2.008669in}{2.471946in}}{\pgfqpoint{2.000769in}{2.468674in}}{\pgfqpoint{1.994945in}{2.462850in}}%
\pgfpathcurveto{\pgfqpoint{1.989121in}{2.457026in}}{\pgfqpoint{1.985848in}{2.449126in}}{\pgfqpoint{1.985848in}{2.440890in}}%
\pgfpathcurveto{\pgfqpoint{1.985848in}{2.432654in}}{\pgfqpoint{1.989121in}{2.424754in}}{\pgfqpoint{1.994945in}{2.418930in}}%
\pgfpathcurveto{\pgfqpoint{2.000769in}{2.413106in}}{\pgfqpoint{2.008669in}{2.409833in}}{\pgfqpoint{2.016905in}{2.409833in}}%
\pgfpathclose%
\pgfusepath{stroke,fill}%
\end{pgfscope}%
\begin{pgfscope}%
\pgfpathrectangle{\pgfqpoint{0.100000in}{0.212622in}}{\pgfqpoint{3.696000in}{3.696000in}}%
\pgfusepath{clip}%
\pgfsetbuttcap%
\pgfsetroundjoin%
\definecolor{currentfill}{rgb}{0.121569,0.466667,0.705882}%
\pgfsetfillcolor{currentfill}%
\pgfsetfillopacity{0.956855}%
\pgfsetlinewidth{1.003750pt}%
\definecolor{currentstroke}{rgb}{0.121569,0.466667,0.705882}%
\pgfsetstrokecolor{currentstroke}%
\pgfsetstrokeopacity{0.956855}%
\pgfsetdash{}{0pt}%
\pgfpathmoveto{\pgfqpoint{1.466053in}{1.744702in}}%
\pgfpathcurveto{\pgfqpoint{1.474290in}{1.744702in}}{\pgfqpoint{1.482190in}{1.747975in}}{\pgfqpoint{1.488014in}{1.753798in}}%
\pgfpathcurveto{\pgfqpoint{1.493838in}{1.759622in}}{\pgfqpoint{1.497110in}{1.767522in}}{\pgfqpoint{1.497110in}{1.775759in}}%
\pgfpathcurveto{\pgfqpoint{1.497110in}{1.783995in}}{\pgfqpoint{1.493838in}{1.791895in}}{\pgfqpoint{1.488014in}{1.797719in}}%
\pgfpathcurveto{\pgfqpoint{1.482190in}{1.803543in}}{\pgfqpoint{1.474290in}{1.806815in}}{\pgfqpoint{1.466053in}{1.806815in}}%
\pgfpathcurveto{\pgfqpoint{1.457817in}{1.806815in}}{\pgfqpoint{1.449917in}{1.803543in}}{\pgfqpoint{1.444093in}{1.797719in}}%
\pgfpathcurveto{\pgfqpoint{1.438269in}{1.791895in}}{\pgfqpoint{1.434997in}{1.783995in}}{\pgfqpoint{1.434997in}{1.775759in}}%
\pgfpathcurveto{\pgfqpoint{1.434997in}{1.767522in}}{\pgfqpoint{1.438269in}{1.759622in}}{\pgfqpoint{1.444093in}{1.753798in}}%
\pgfpathcurveto{\pgfqpoint{1.449917in}{1.747975in}}{\pgfqpoint{1.457817in}{1.744702in}}{\pgfqpoint{1.466053in}{1.744702in}}%
\pgfpathclose%
\pgfusepath{stroke,fill}%
\end{pgfscope}%
\begin{pgfscope}%
\pgfpathrectangle{\pgfqpoint{0.100000in}{0.212622in}}{\pgfqpoint{3.696000in}{3.696000in}}%
\pgfusepath{clip}%
\pgfsetbuttcap%
\pgfsetroundjoin%
\definecolor{currentfill}{rgb}{0.121569,0.466667,0.705882}%
\pgfsetfillcolor{currentfill}%
\pgfsetfillopacity{0.956971}%
\pgfsetlinewidth{1.003750pt}%
\definecolor{currentstroke}{rgb}{0.121569,0.466667,0.705882}%
\pgfsetstrokecolor{currentstroke}%
\pgfsetstrokeopacity{0.956971}%
\pgfsetdash{}{0pt}%
\pgfpathmoveto{\pgfqpoint{2.067977in}{2.395999in}}%
\pgfpathcurveto{\pgfqpoint{2.076214in}{2.395999in}}{\pgfqpoint{2.084114in}{2.399271in}}{\pgfqpoint{2.089938in}{2.405095in}}%
\pgfpathcurveto{\pgfqpoint{2.095762in}{2.410919in}}{\pgfqpoint{2.099034in}{2.418819in}}{\pgfqpoint{2.099034in}{2.427055in}}%
\pgfpathcurveto{\pgfqpoint{2.099034in}{2.435292in}}{\pgfqpoint{2.095762in}{2.443192in}}{\pgfqpoint{2.089938in}{2.449016in}}%
\pgfpathcurveto{\pgfqpoint{2.084114in}{2.454840in}}{\pgfqpoint{2.076214in}{2.458112in}}{\pgfqpoint{2.067977in}{2.458112in}}%
\pgfpathcurveto{\pgfqpoint{2.059741in}{2.458112in}}{\pgfqpoint{2.051841in}{2.454840in}}{\pgfqpoint{2.046017in}{2.449016in}}%
\pgfpathcurveto{\pgfqpoint{2.040193in}{2.443192in}}{\pgfqpoint{2.036921in}{2.435292in}}{\pgfqpoint{2.036921in}{2.427055in}}%
\pgfpathcurveto{\pgfqpoint{2.036921in}{2.418819in}}{\pgfqpoint{2.040193in}{2.410919in}}{\pgfqpoint{2.046017in}{2.405095in}}%
\pgfpathcurveto{\pgfqpoint{2.051841in}{2.399271in}}{\pgfqpoint{2.059741in}{2.395999in}}{\pgfqpoint{2.067977in}{2.395999in}}%
\pgfpathclose%
\pgfusepath{stroke,fill}%
\end{pgfscope}%
\begin{pgfscope}%
\pgfpathrectangle{\pgfqpoint{0.100000in}{0.212622in}}{\pgfqpoint{3.696000in}{3.696000in}}%
\pgfusepath{clip}%
\pgfsetbuttcap%
\pgfsetroundjoin%
\definecolor{currentfill}{rgb}{0.121569,0.466667,0.705882}%
\pgfsetfillcolor{currentfill}%
\pgfsetfillopacity{0.957000}%
\pgfsetlinewidth{1.003750pt}%
\definecolor{currentstroke}{rgb}{0.121569,0.466667,0.705882}%
\pgfsetstrokecolor{currentstroke}%
\pgfsetstrokeopacity{0.957000}%
\pgfsetdash{}{0pt}%
\pgfpathmoveto{\pgfqpoint{2.019613in}{2.408656in}}%
\pgfpathcurveto{\pgfqpoint{2.027849in}{2.408656in}}{\pgfqpoint{2.035749in}{2.411928in}}{\pgfqpoint{2.041573in}{2.417752in}}%
\pgfpathcurveto{\pgfqpoint{2.047397in}{2.423576in}}{\pgfqpoint{2.050669in}{2.431476in}}{\pgfqpoint{2.050669in}{2.439713in}}%
\pgfpathcurveto{\pgfqpoint{2.050669in}{2.447949in}}{\pgfqpoint{2.047397in}{2.455849in}}{\pgfqpoint{2.041573in}{2.461673in}}%
\pgfpathcurveto{\pgfqpoint{2.035749in}{2.467497in}}{\pgfqpoint{2.027849in}{2.470769in}}{\pgfqpoint{2.019613in}{2.470769in}}%
\pgfpathcurveto{\pgfqpoint{2.011376in}{2.470769in}}{\pgfqpoint{2.003476in}{2.467497in}}{\pgfqpoint{1.997652in}{2.461673in}}%
\pgfpathcurveto{\pgfqpoint{1.991828in}{2.455849in}}{\pgfqpoint{1.988556in}{2.447949in}}{\pgfqpoint{1.988556in}{2.439713in}}%
\pgfpathcurveto{\pgfqpoint{1.988556in}{2.431476in}}{\pgfqpoint{1.991828in}{2.423576in}}{\pgfqpoint{1.997652in}{2.417752in}}%
\pgfpathcurveto{\pgfqpoint{2.003476in}{2.411928in}}{\pgfqpoint{2.011376in}{2.408656in}}{\pgfqpoint{2.019613in}{2.408656in}}%
\pgfpathclose%
\pgfusepath{stroke,fill}%
\end{pgfscope}%
\begin{pgfscope}%
\pgfpathrectangle{\pgfqpoint{0.100000in}{0.212622in}}{\pgfqpoint{3.696000in}{3.696000in}}%
\pgfusepath{clip}%
\pgfsetbuttcap%
\pgfsetroundjoin%
\definecolor{currentfill}{rgb}{0.121569,0.466667,0.705882}%
\pgfsetfillcolor{currentfill}%
\pgfsetfillopacity{0.957106}%
\pgfsetlinewidth{1.003750pt}%
\definecolor{currentstroke}{rgb}{0.121569,0.466667,0.705882}%
\pgfsetstrokecolor{currentstroke}%
\pgfsetstrokeopacity{0.957106}%
\pgfsetdash{}{0pt}%
\pgfpathmoveto{\pgfqpoint{2.021090in}{2.407955in}}%
\pgfpathcurveto{\pgfqpoint{2.029326in}{2.407955in}}{\pgfqpoint{2.037226in}{2.411227in}}{\pgfqpoint{2.043050in}{2.417051in}}%
\pgfpathcurveto{\pgfqpoint{2.048874in}{2.422875in}}{\pgfqpoint{2.052147in}{2.430775in}}{\pgfqpoint{2.052147in}{2.439011in}}%
\pgfpathcurveto{\pgfqpoint{2.052147in}{2.447247in}}{\pgfqpoint{2.048874in}{2.455147in}}{\pgfqpoint{2.043050in}{2.460971in}}%
\pgfpathcurveto{\pgfqpoint{2.037226in}{2.466795in}}{\pgfqpoint{2.029326in}{2.470068in}}{\pgfqpoint{2.021090in}{2.470068in}}%
\pgfpathcurveto{\pgfqpoint{2.012854in}{2.470068in}}{\pgfqpoint{2.004954in}{2.466795in}}{\pgfqpoint{1.999130in}{2.460971in}}%
\pgfpathcurveto{\pgfqpoint{1.993306in}{2.455147in}}{\pgfqpoint{1.990034in}{2.447247in}}{\pgfqpoint{1.990034in}{2.439011in}}%
\pgfpathcurveto{\pgfqpoint{1.990034in}{2.430775in}}{\pgfqpoint{1.993306in}{2.422875in}}{\pgfqpoint{1.999130in}{2.417051in}}%
\pgfpathcurveto{\pgfqpoint{2.004954in}{2.411227in}}{\pgfqpoint{2.012854in}{2.407955in}}{\pgfqpoint{2.021090in}{2.407955in}}%
\pgfpathclose%
\pgfusepath{stroke,fill}%
\end{pgfscope}%
\begin{pgfscope}%
\pgfpathrectangle{\pgfqpoint{0.100000in}{0.212622in}}{\pgfqpoint{3.696000in}{3.696000in}}%
\pgfusepath{clip}%
\pgfsetbuttcap%
\pgfsetroundjoin%
\definecolor{currentfill}{rgb}{0.121569,0.466667,0.705882}%
\pgfsetfillcolor{currentfill}%
\pgfsetfillopacity{0.957208}%
\pgfsetlinewidth{1.003750pt}%
\definecolor{currentstroke}{rgb}{0.121569,0.466667,0.705882}%
\pgfsetstrokecolor{currentstroke}%
\pgfsetstrokeopacity{0.957208}%
\pgfsetdash{}{0pt}%
\pgfpathmoveto{\pgfqpoint{2.023470in}{2.406596in}}%
\pgfpathcurveto{\pgfqpoint{2.031706in}{2.406596in}}{\pgfqpoint{2.039606in}{2.409868in}}{\pgfqpoint{2.045430in}{2.415692in}}%
\pgfpathcurveto{\pgfqpoint{2.051254in}{2.421516in}}{\pgfqpoint{2.054526in}{2.429416in}}{\pgfqpoint{2.054526in}{2.437652in}}%
\pgfpathcurveto{\pgfqpoint{2.054526in}{2.445888in}}{\pgfqpoint{2.051254in}{2.453789in}}{\pgfqpoint{2.045430in}{2.459612in}}%
\pgfpathcurveto{\pgfqpoint{2.039606in}{2.465436in}}{\pgfqpoint{2.031706in}{2.468709in}}{\pgfqpoint{2.023470in}{2.468709in}}%
\pgfpathcurveto{\pgfqpoint{2.015233in}{2.468709in}}{\pgfqpoint{2.007333in}{2.465436in}}{\pgfqpoint{2.001509in}{2.459612in}}%
\pgfpathcurveto{\pgfqpoint{1.995685in}{2.453789in}}{\pgfqpoint{1.992413in}{2.445888in}}{\pgfqpoint{1.992413in}{2.437652in}}%
\pgfpathcurveto{\pgfqpoint{1.992413in}{2.429416in}}{\pgfqpoint{1.995685in}{2.421516in}}{\pgfqpoint{2.001509in}{2.415692in}}%
\pgfpathcurveto{\pgfqpoint{2.007333in}{2.409868in}}{\pgfqpoint{2.015233in}{2.406596in}}{\pgfqpoint{2.023470in}{2.406596in}}%
\pgfpathclose%
\pgfusepath{stroke,fill}%
\end{pgfscope}%
\begin{pgfscope}%
\pgfpathrectangle{\pgfqpoint{0.100000in}{0.212622in}}{\pgfqpoint{3.696000in}{3.696000in}}%
\pgfusepath{clip}%
\pgfsetbuttcap%
\pgfsetroundjoin%
\definecolor{currentfill}{rgb}{0.121569,0.466667,0.705882}%
\pgfsetfillcolor{currentfill}%
\pgfsetfillopacity{0.957238}%
\pgfsetlinewidth{1.003750pt}%
\definecolor{currentstroke}{rgb}{0.121569,0.466667,0.705882}%
\pgfsetstrokecolor{currentstroke}%
\pgfsetstrokeopacity{0.957238}%
\pgfsetdash{}{0pt}%
\pgfpathmoveto{\pgfqpoint{2.570874in}{1.181193in}}%
\pgfpathcurveto{\pgfqpoint{2.579111in}{1.181193in}}{\pgfqpoint{2.587011in}{1.184466in}}{\pgfqpoint{2.592835in}{1.190290in}}%
\pgfpathcurveto{\pgfqpoint{2.598659in}{1.196114in}}{\pgfqpoint{2.601931in}{1.204014in}}{\pgfqpoint{2.601931in}{1.212250in}}%
\pgfpathcurveto{\pgfqpoint{2.601931in}{1.220486in}}{\pgfqpoint{2.598659in}{1.228386in}}{\pgfqpoint{2.592835in}{1.234210in}}%
\pgfpathcurveto{\pgfqpoint{2.587011in}{1.240034in}}{\pgfqpoint{2.579111in}{1.243306in}}{\pgfqpoint{2.570874in}{1.243306in}}%
\pgfpathcurveto{\pgfqpoint{2.562638in}{1.243306in}}{\pgfqpoint{2.554738in}{1.240034in}}{\pgfqpoint{2.548914in}{1.234210in}}%
\pgfpathcurveto{\pgfqpoint{2.543090in}{1.228386in}}{\pgfqpoint{2.539818in}{1.220486in}}{\pgfqpoint{2.539818in}{1.212250in}}%
\pgfpathcurveto{\pgfqpoint{2.539818in}{1.204014in}}{\pgfqpoint{2.543090in}{1.196114in}}{\pgfqpoint{2.548914in}{1.190290in}}%
\pgfpathcurveto{\pgfqpoint{2.554738in}{1.184466in}}{\pgfqpoint{2.562638in}{1.181193in}}{\pgfqpoint{2.570874in}{1.181193in}}%
\pgfpathclose%
\pgfusepath{stroke,fill}%
\end{pgfscope}%
\begin{pgfscope}%
\pgfpathrectangle{\pgfqpoint{0.100000in}{0.212622in}}{\pgfqpoint{3.696000in}{3.696000in}}%
\pgfusepath{clip}%
\pgfsetbuttcap%
\pgfsetroundjoin%
\definecolor{currentfill}{rgb}{0.121569,0.466667,0.705882}%
\pgfsetfillcolor{currentfill}%
\pgfsetfillopacity{0.957279}%
\pgfsetlinewidth{1.003750pt}%
\definecolor{currentstroke}{rgb}{0.121569,0.466667,0.705882}%
\pgfsetstrokecolor{currentstroke}%
\pgfsetstrokeopacity{0.957279}%
\pgfsetdash{}{0pt}%
\pgfpathmoveto{\pgfqpoint{2.024767in}{2.405891in}}%
\pgfpathcurveto{\pgfqpoint{2.033004in}{2.405891in}}{\pgfqpoint{2.040904in}{2.409163in}}{\pgfqpoint{2.046728in}{2.414987in}}%
\pgfpathcurveto{\pgfqpoint{2.052552in}{2.420811in}}{\pgfqpoint{2.055824in}{2.428711in}}{\pgfqpoint{2.055824in}{2.436947in}}%
\pgfpathcurveto{\pgfqpoint{2.055824in}{2.445183in}}{\pgfqpoint{2.052552in}{2.453083in}}{\pgfqpoint{2.046728in}{2.458907in}}%
\pgfpathcurveto{\pgfqpoint{2.040904in}{2.464731in}}{\pgfqpoint{2.033004in}{2.468004in}}{\pgfqpoint{2.024767in}{2.468004in}}%
\pgfpathcurveto{\pgfqpoint{2.016531in}{2.468004in}}{\pgfqpoint{2.008631in}{2.464731in}}{\pgfqpoint{2.002807in}{2.458907in}}%
\pgfpathcurveto{\pgfqpoint{1.996983in}{2.453083in}}{\pgfqpoint{1.993711in}{2.445183in}}{\pgfqpoint{1.993711in}{2.436947in}}%
\pgfpathcurveto{\pgfqpoint{1.993711in}{2.428711in}}{\pgfqpoint{1.996983in}{2.420811in}}{\pgfqpoint{2.002807in}{2.414987in}}%
\pgfpathcurveto{\pgfqpoint{2.008631in}{2.409163in}}{\pgfqpoint{2.016531in}{2.405891in}}{\pgfqpoint{2.024767in}{2.405891in}}%
\pgfpathclose%
\pgfusepath{stroke,fill}%
\end{pgfscope}%
\begin{pgfscope}%
\pgfpathrectangle{\pgfqpoint{0.100000in}{0.212622in}}{\pgfqpoint{3.696000in}{3.696000in}}%
\pgfusepath{clip}%
\pgfsetbuttcap%
\pgfsetroundjoin%
\definecolor{currentfill}{rgb}{0.121569,0.466667,0.705882}%
\pgfsetfillcolor{currentfill}%
\pgfsetfillopacity{0.957422}%
\pgfsetlinewidth{1.003750pt}%
\definecolor{currentstroke}{rgb}{0.121569,0.466667,0.705882}%
\pgfsetstrokecolor{currentstroke}%
\pgfsetstrokeopacity{0.957422}%
\pgfsetdash{}{0pt}%
\pgfpathmoveto{\pgfqpoint{2.064993in}{2.395731in}}%
\pgfpathcurveto{\pgfqpoint{2.073229in}{2.395731in}}{\pgfqpoint{2.081129in}{2.399004in}}{\pgfqpoint{2.086953in}{2.404828in}}%
\pgfpathcurveto{\pgfqpoint{2.092777in}{2.410652in}}{\pgfqpoint{2.096049in}{2.418552in}}{\pgfqpoint{2.096049in}{2.426788in}}%
\pgfpathcurveto{\pgfqpoint{2.096049in}{2.435024in}}{\pgfqpoint{2.092777in}{2.442924in}}{\pgfqpoint{2.086953in}{2.448748in}}%
\pgfpathcurveto{\pgfqpoint{2.081129in}{2.454572in}}{\pgfqpoint{2.073229in}{2.457844in}}{\pgfqpoint{2.064993in}{2.457844in}}%
\pgfpathcurveto{\pgfqpoint{2.056757in}{2.457844in}}{\pgfqpoint{2.048857in}{2.454572in}}{\pgfqpoint{2.043033in}{2.448748in}}%
\pgfpathcurveto{\pgfqpoint{2.037209in}{2.442924in}}{\pgfqpoint{2.033936in}{2.435024in}}{\pgfqpoint{2.033936in}{2.426788in}}%
\pgfpathcurveto{\pgfqpoint{2.033936in}{2.418552in}}{\pgfqpoint{2.037209in}{2.410652in}}{\pgfqpoint{2.043033in}{2.404828in}}%
\pgfpathcurveto{\pgfqpoint{2.048857in}{2.399004in}}{\pgfqpoint{2.056757in}{2.395731in}}{\pgfqpoint{2.064993in}{2.395731in}}%
\pgfpathclose%
\pgfusepath{stroke,fill}%
\end{pgfscope}%
\begin{pgfscope}%
\pgfpathrectangle{\pgfqpoint{0.100000in}{0.212622in}}{\pgfqpoint{3.696000in}{3.696000in}}%
\pgfusepath{clip}%
\pgfsetbuttcap%
\pgfsetroundjoin%
\definecolor{currentfill}{rgb}{0.121569,0.466667,0.705882}%
\pgfsetfillcolor{currentfill}%
\pgfsetfillopacity{0.957428}%
\pgfsetlinewidth{1.003750pt}%
\definecolor{currentstroke}{rgb}{0.121569,0.466667,0.705882}%
\pgfsetstrokecolor{currentstroke}%
\pgfsetstrokeopacity{0.957428}%
\pgfsetdash{}{0pt}%
\pgfpathmoveto{\pgfqpoint{2.026350in}{2.405127in}}%
\pgfpathcurveto{\pgfqpoint{2.034586in}{2.405127in}}{\pgfqpoint{2.042486in}{2.408399in}}{\pgfqpoint{2.048310in}{2.414223in}}%
\pgfpathcurveto{\pgfqpoint{2.054134in}{2.420047in}}{\pgfqpoint{2.057406in}{2.427947in}}{\pgfqpoint{2.057406in}{2.436183in}}%
\pgfpathcurveto{\pgfqpoint{2.057406in}{2.444419in}}{\pgfqpoint{2.054134in}{2.452319in}}{\pgfqpoint{2.048310in}{2.458143in}}%
\pgfpathcurveto{\pgfqpoint{2.042486in}{2.463967in}}{\pgfqpoint{2.034586in}{2.467240in}}{\pgfqpoint{2.026350in}{2.467240in}}%
\pgfpathcurveto{\pgfqpoint{2.018114in}{2.467240in}}{\pgfqpoint{2.010214in}{2.463967in}}{\pgfqpoint{2.004390in}{2.458143in}}%
\pgfpathcurveto{\pgfqpoint{1.998566in}{2.452319in}}{\pgfqpoint{1.995293in}{2.444419in}}{\pgfqpoint{1.995293in}{2.436183in}}%
\pgfpathcurveto{\pgfqpoint{1.995293in}{2.427947in}}{\pgfqpoint{1.998566in}{2.420047in}}{\pgfqpoint{2.004390in}{2.414223in}}%
\pgfpathcurveto{\pgfqpoint{2.010214in}{2.408399in}}{\pgfqpoint{2.018114in}{2.405127in}}{\pgfqpoint{2.026350in}{2.405127in}}%
\pgfpathclose%
\pgfusepath{stroke,fill}%
\end{pgfscope}%
\begin{pgfscope}%
\pgfpathrectangle{\pgfqpoint{0.100000in}{0.212622in}}{\pgfqpoint{3.696000in}{3.696000in}}%
\pgfusepath{clip}%
\pgfsetbuttcap%
\pgfsetroundjoin%
\definecolor{currentfill}{rgb}{0.121569,0.466667,0.705882}%
\pgfsetfillcolor{currentfill}%
\pgfsetfillopacity{0.957492}%
\pgfsetlinewidth{1.003750pt}%
\definecolor{currentstroke}{rgb}{0.121569,0.466667,0.705882}%
\pgfsetstrokecolor{currentstroke}%
\pgfsetstrokeopacity{0.957492}%
\pgfsetdash{}{0pt}%
\pgfpathmoveto{\pgfqpoint{2.027233in}{2.404652in}}%
\pgfpathcurveto{\pgfqpoint{2.035469in}{2.404652in}}{\pgfqpoint{2.043369in}{2.407924in}}{\pgfqpoint{2.049193in}{2.413748in}}%
\pgfpathcurveto{\pgfqpoint{2.055017in}{2.419572in}}{\pgfqpoint{2.058289in}{2.427472in}}{\pgfqpoint{2.058289in}{2.435708in}}%
\pgfpathcurveto{\pgfqpoint{2.058289in}{2.443945in}}{\pgfqpoint{2.055017in}{2.451845in}}{\pgfqpoint{2.049193in}{2.457669in}}%
\pgfpathcurveto{\pgfqpoint{2.043369in}{2.463493in}}{\pgfqpoint{2.035469in}{2.466765in}}{\pgfqpoint{2.027233in}{2.466765in}}%
\pgfpathcurveto{\pgfqpoint{2.018997in}{2.466765in}}{\pgfqpoint{2.011097in}{2.463493in}}{\pgfqpoint{2.005273in}{2.457669in}}%
\pgfpathcurveto{\pgfqpoint{1.999449in}{2.451845in}}{\pgfqpoint{1.996176in}{2.443945in}}{\pgfqpoint{1.996176in}{2.435708in}}%
\pgfpathcurveto{\pgfqpoint{1.996176in}{2.427472in}}{\pgfqpoint{1.999449in}{2.419572in}}{\pgfqpoint{2.005273in}{2.413748in}}%
\pgfpathcurveto{\pgfqpoint{2.011097in}{2.407924in}}{\pgfqpoint{2.018997in}{2.404652in}}{\pgfqpoint{2.027233in}{2.404652in}}%
\pgfpathclose%
\pgfusepath{stroke,fill}%
\end{pgfscope}%
\begin{pgfscope}%
\pgfpathrectangle{\pgfqpoint{0.100000in}{0.212622in}}{\pgfqpoint{3.696000in}{3.696000in}}%
\pgfusepath{clip}%
\pgfsetbuttcap%
\pgfsetroundjoin%
\definecolor{currentfill}{rgb}{0.121569,0.466667,0.705882}%
\pgfsetfillcolor{currentfill}%
\pgfsetfillopacity{0.957601}%
\pgfsetlinewidth{1.003750pt}%
\definecolor{currentstroke}{rgb}{0.121569,0.466667,0.705882}%
\pgfsetstrokecolor{currentstroke}%
\pgfsetstrokeopacity{0.957601}%
\pgfsetdash{}{0pt}%
\pgfpathmoveto{\pgfqpoint{2.028637in}{2.403888in}}%
\pgfpathcurveto{\pgfqpoint{2.036874in}{2.403888in}}{\pgfqpoint{2.044774in}{2.407160in}}{\pgfqpoint{2.050598in}{2.412984in}}%
\pgfpathcurveto{\pgfqpoint{2.056422in}{2.418808in}}{\pgfqpoint{2.059694in}{2.426708in}}{\pgfqpoint{2.059694in}{2.434944in}}%
\pgfpathcurveto{\pgfqpoint{2.059694in}{2.443181in}}{\pgfqpoint{2.056422in}{2.451081in}}{\pgfqpoint{2.050598in}{2.456905in}}%
\pgfpathcurveto{\pgfqpoint{2.044774in}{2.462729in}}{\pgfqpoint{2.036874in}{2.466001in}}{\pgfqpoint{2.028637in}{2.466001in}}%
\pgfpathcurveto{\pgfqpoint{2.020401in}{2.466001in}}{\pgfqpoint{2.012501in}{2.462729in}}{\pgfqpoint{2.006677in}{2.456905in}}%
\pgfpathcurveto{\pgfqpoint{2.000853in}{2.451081in}}{\pgfqpoint{1.997581in}{2.443181in}}{\pgfqpoint{1.997581in}{2.434944in}}%
\pgfpathcurveto{\pgfqpoint{1.997581in}{2.426708in}}{\pgfqpoint{2.000853in}{2.418808in}}{\pgfqpoint{2.006677in}{2.412984in}}%
\pgfpathcurveto{\pgfqpoint{2.012501in}{2.407160in}}{\pgfqpoint{2.020401in}{2.403888in}}{\pgfqpoint{2.028637in}{2.403888in}}%
\pgfpathclose%
\pgfusepath{stroke,fill}%
\end{pgfscope}%
\begin{pgfscope}%
\pgfpathrectangle{\pgfqpoint{0.100000in}{0.212622in}}{\pgfqpoint{3.696000in}{3.696000in}}%
\pgfusepath{clip}%
\pgfsetbuttcap%
\pgfsetroundjoin%
\definecolor{currentfill}{rgb}{0.121569,0.466667,0.705882}%
\pgfsetfillcolor{currentfill}%
\pgfsetfillopacity{0.957658}%
\pgfsetlinewidth{1.003750pt}%
\definecolor{currentstroke}{rgb}{0.121569,0.466667,0.705882}%
\pgfsetstrokecolor{currentstroke}%
\pgfsetstrokeopacity{0.957658}%
\pgfsetdash{}{0pt}%
\pgfpathmoveto{\pgfqpoint{2.030451in}{2.402880in}}%
\pgfpathcurveto{\pgfqpoint{2.038687in}{2.402880in}}{\pgfqpoint{2.046587in}{2.406152in}}{\pgfqpoint{2.052411in}{2.411976in}}%
\pgfpathcurveto{\pgfqpoint{2.058235in}{2.417800in}}{\pgfqpoint{2.061508in}{2.425700in}}{\pgfqpoint{2.061508in}{2.433937in}}%
\pgfpathcurveto{\pgfqpoint{2.061508in}{2.442173in}}{\pgfqpoint{2.058235in}{2.450073in}}{\pgfqpoint{2.052411in}{2.455897in}}%
\pgfpathcurveto{\pgfqpoint{2.046587in}{2.461721in}}{\pgfqpoint{2.038687in}{2.464993in}}{\pgfqpoint{2.030451in}{2.464993in}}%
\pgfpathcurveto{\pgfqpoint{2.022215in}{2.464993in}}{\pgfqpoint{2.014315in}{2.461721in}}{\pgfqpoint{2.008491in}{2.455897in}}%
\pgfpathcurveto{\pgfqpoint{2.002667in}{2.450073in}}{\pgfqpoint{1.999395in}{2.442173in}}{\pgfqpoint{1.999395in}{2.433937in}}%
\pgfpathcurveto{\pgfqpoint{1.999395in}{2.425700in}}{\pgfqpoint{2.002667in}{2.417800in}}{\pgfqpoint{2.008491in}{2.411976in}}%
\pgfpathcurveto{\pgfqpoint{2.014315in}{2.406152in}}{\pgfqpoint{2.022215in}{2.402880in}}{\pgfqpoint{2.030451in}{2.402880in}}%
\pgfpathclose%
\pgfusepath{stroke,fill}%
\end{pgfscope}%
\begin{pgfscope}%
\pgfpathrectangle{\pgfqpoint{0.100000in}{0.212622in}}{\pgfqpoint{3.696000in}{3.696000in}}%
\pgfusepath{clip}%
\pgfsetbuttcap%
\pgfsetroundjoin%
\definecolor{currentfill}{rgb}{0.121569,0.466667,0.705882}%
\pgfsetfillcolor{currentfill}%
\pgfsetfillopacity{0.957669}%
\pgfsetlinewidth{1.003750pt}%
\definecolor{currentstroke}{rgb}{0.121569,0.466667,0.705882}%
\pgfsetstrokecolor{currentstroke}%
\pgfsetstrokeopacity{0.957669}%
\pgfsetdash{}{0pt}%
\pgfpathmoveto{\pgfqpoint{1.483147in}{1.732865in}}%
\pgfpathcurveto{\pgfqpoint{1.491383in}{1.732865in}}{\pgfqpoint{1.499283in}{1.736137in}}{\pgfqpoint{1.505107in}{1.741961in}}%
\pgfpathcurveto{\pgfqpoint{1.510931in}{1.747785in}}{\pgfqpoint{1.514203in}{1.755685in}}{\pgfqpoint{1.514203in}{1.763922in}}%
\pgfpathcurveto{\pgfqpoint{1.514203in}{1.772158in}}{\pgfqpoint{1.510931in}{1.780058in}}{\pgfqpoint{1.505107in}{1.785882in}}%
\pgfpathcurveto{\pgfqpoint{1.499283in}{1.791706in}}{\pgfqpoint{1.491383in}{1.794978in}}{\pgfqpoint{1.483147in}{1.794978in}}%
\pgfpathcurveto{\pgfqpoint{1.474911in}{1.794978in}}{\pgfqpoint{1.467011in}{1.791706in}}{\pgfqpoint{1.461187in}{1.785882in}}%
\pgfpathcurveto{\pgfqpoint{1.455363in}{1.780058in}}{\pgfqpoint{1.452090in}{1.772158in}}{\pgfqpoint{1.452090in}{1.763922in}}%
\pgfpathcurveto{\pgfqpoint{1.452090in}{1.755685in}}{\pgfqpoint{1.455363in}{1.747785in}}{\pgfqpoint{1.461187in}{1.741961in}}%
\pgfpathcurveto{\pgfqpoint{1.467011in}{1.736137in}}{\pgfqpoint{1.474911in}{1.732865in}}{\pgfqpoint{1.483147in}{1.732865in}}%
\pgfpathclose%
\pgfusepath{stroke,fill}%
\end{pgfscope}%
\begin{pgfscope}%
\pgfpathrectangle{\pgfqpoint{0.100000in}{0.212622in}}{\pgfqpoint{3.696000in}{3.696000in}}%
\pgfusepath{clip}%
\pgfsetbuttcap%
\pgfsetroundjoin%
\definecolor{currentfill}{rgb}{0.121569,0.466667,0.705882}%
\pgfsetfillcolor{currentfill}%
\pgfsetfillopacity{0.957733}%
\pgfsetlinewidth{1.003750pt}%
\definecolor{currentstroke}{rgb}{0.121569,0.466667,0.705882}%
\pgfsetstrokecolor{currentstroke}%
\pgfsetstrokeopacity{0.957733}%
\pgfsetdash{}{0pt}%
\pgfpathmoveto{\pgfqpoint{2.062684in}{2.395641in}}%
\pgfpathcurveto{\pgfqpoint{2.070921in}{2.395641in}}{\pgfqpoint{2.078821in}{2.398913in}}{\pgfqpoint{2.084645in}{2.404737in}}%
\pgfpathcurveto{\pgfqpoint{2.090468in}{2.410561in}}{\pgfqpoint{2.093741in}{2.418461in}}{\pgfqpoint{2.093741in}{2.426697in}}%
\pgfpathcurveto{\pgfqpoint{2.093741in}{2.434934in}}{\pgfqpoint{2.090468in}{2.442834in}}{\pgfqpoint{2.084645in}{2.448658in}}%
\pgfpathcurveto{\pgfqpoint{2.078821in}{2.454482in}}{\pgfqpoint{2.070921in}{2.457754in}}{\pgfqpoint{2.062684in}{2.457754in}}%
\pgfpathcurveto{\pgfqpoint{2.054448in}{2.457754in}}{\pgfqpoint{2.046548in}{2.454482in}}{\pgfqpoint{2.040724in}{2.448658in}}%
\pgfpathcurveto{\pgfqpoint{2.034900in}{2.442834in}}{\pgfqpoint{2.031628in}{2.434934in}}{\pgfqpoint{2.031628in}{2.426697in}}%
\pgfpathcurveto{\pgfqpoint{2.031628in}{2.418461in}}{\pgfqpoint{2.034900in}{2.410561in}}{\pgfqpoint{2.040724in}{2.404737in}}%
\pgfpathcurveto{\pgfqpoint{2.046548in}{2.398913in}}{\pgfqpoint{2.054448in}{2.395641in}}{\pgfqpoint{2.062684in}{2.395641in}}%
\pgfpathclose%
\pgfusepath{stroke,fill}%
\end{pgfscope}%
\begin{pgfscope}%
\pgfpathrectangle{\pgfqpoint{0.100000in}{0.212622in}}{\pgfqpoint{3.696000in}{3.696000in}}%
\pgfusepath{clip}%
\pgfsetbuttcap%
\pgfsetroundjoin%
\definecolor{currentfill}{rgb}{0.121569,0.466667,0.705882}%
\pgfsetfillcolor{currentfill}%
\pgfsetfillopacity{0.957812}%
\pgfsetlinewidth{1.003750pt}%
\definecolor{currentstroke}{rgb}{0.121569,0.466667,0.705882}%
\pgfsetstrokecolor{currentstroke}%
\pgfsetstrokeopacity{0.957812}%
\pgfsetdash{}{0pt}%
\pgfpathmoveto{\pgfqpoint{2.032526in}{2.401789in}}%
\pgfpathcurveto{\pgfqpoint{2.040763in}{2.401789in}}{\pgfqpoint{2.048663in}{2.405061in}}{\pgfqpoint{2.054487in}{2.410885in}}%
\pgfpathcurveto{\pgfqpoint{2.060311in}{2.416709in}}{\pgfqpoint{2.063583in}{2.424609in}}{\pgfqpoint{2.063583in}{2.432845in}}%
\pgfpathcurveto{\pgfqpoint{2.063583in}{2.441081in}}{\pgfqpoint{2.060311in}{2.448981in}}{\pgfqpoint{2.054487in}{2.454805in}}%
\pgfpathcurveto{\pgfqpoint{2.048663in}{2.460629in}}{\pgfqpoint{2.040763in}{2.463902in}}{\pgfqpoint{2.032526in}{2.463902in}}%
\pgfpathcurveto{\pgfqpoint{2.024290in}{2.463902in}}{\pgfqpoint{2.016390in}{2.460629in}}{\pgfqpoint{2.010566in}{2.454805in}}%
\pgfpathcurveto{\pgfqpoint{2.004742in}{2.448981in}}{\pgfqpoint{2.001470in}{2.441081in}}{\pgfqpoint{2.001470in}{2.432845in}}%
\pgfpathcurveto{\pgfqpoint{2.001470in}{2.424609in}}{\pgfqpoint{2.004742in}{2.416709in}}{\pgfqpoint{2.010566in}{2.410885in}}%
\pgfpathcurveto{\pgfqpoint{2.016390in}{2.405061in}}{\pgfqpoint{2.024290in}{2.401789in}}{\pgfqpoint{2.032526in}{2.401789in}}%
\pgfpathclose%
\pgfusepath{stroke,fill}%
\end{pgfscope}%
\begin{pgfscope}%
\pgfpathrectangle{\pgfqpoint{0.100000in}{0.212622in}}{\pgfqpoint{3.696000in}{3.696000in}}%
\pgfusepath{clip}%
\pgfsetbuttcap%
\pgfsetroundjoin%
\definecolor{currentfill}{rgb}{0.121569,0.466667,0.705882}%
\pgfsetfillcolor{currentfill}%
\pgfsetfillopacity{0.957897}%
\pgfsetlinewidth{1.003750pt}%
\definecolor{currentstroke}{rgb}{0.121569,0.466667,0.705882}%
\pgfsetstrokecolor{currentstroke}%
\pgfsetstrokeopacity{0.957897}%
\pgfsetdash{}{0pt}%
\pgfpathmoveto{\pgfqpoint{2.033664in}{2.401184in}}%
\pgfpathcurveto{\pgfqpoint{2.041901in}{2.401184in}}{\pgfqpoint{2.049801in}{2.404456in}}{\pgfqpoint{2.055625in}{2.410280in}}%
\pgfpathcurveto{\pgfqpoint{2.061449in}{2.416104in}}{\pgfqpoint{2.064721in}{2.424004in}}{\pgfqpoint{2.064721in}{2.432240in}}%
\pgfpathcurveto{\pgfqpoint{2.064721in}{2.440477in}}{\pgfqpoint{2.061449in}{2.448377in}}{\pgfqpoint{2.055625in}{2.454201in}}%
\pgfpathcurveto{\pgfqpoint{2.049801in}{2.460025in}}{\pgfqpoint{2.041901in}{2.463297in}}{\pgfqpoint{2.033664in}{2.463297in}}%
\pgfpathcurveto{\pgfqpoint{2.025428in}{2.463297in}}{\pgfqpoint{2.017528in}{2.460025in}}{\pgfqpoint{2.011704in}{2.454201in}}%
\pgfpathcurveto{\pgfqpoint{2.005880in}{2.448377in}}{\pgfqpoint{2.002608in}{2.440477in}}{\pgfqpoint{2.002608in}{2.432240in}}%
\pgfpathcurveto{\pgfqpoint{2.002608in}{2.424004in}}{\pgfqpoint{2.005880in}{2.416104in}}{\pgfqpoint{2.011704in}{2.410280in}}%
\pgfpathcurveto{\pgfqpoint{2.017528in}{2.404456in}}{\pgfqpoint{2.025428in}{2.401184in}}{\pgfqpoint{2.033664in}{2.401184in}}%
\pgfpathclose%
\pgfusepath{stroke,fill}%
\end{pgfscope}%
\begin{pgfscope}%
\pgfpathrectangle{\pgfqpoint{0.100000in}{0.212622in}}{\pgfqpoint{3.696000in}{3.696000in}}%
\pgfusepath{clip}%
\pgfsetbuttcap%
\pgfsetroundjoin%
\definecolor{currentfill}{rgb}{0.121569,0.466667,0.705882}%
\pgfsetfillcolor{currentfill}%
\pgfsetfillopacity{0.957919}%
\pgfsetlinewidth{1.003750pt}%
\definecolor{currentstroke}{rgb}{0.121569,0.466667,0.705882}%
\pgfsetstrokecolor{currentstroke}%
\pgfsetstrokeopacity{0.957919}%
\pgfsetdash{}{0pt}%
\pgfpathmoveto{\pgfqpoint{2.060959in}{2.395510in}}%
\pgfpathcurveto{\pgfqpoint{2.069196in}{2.395510in}}{\pgfqpoint{2.077096in}{2.398782in}}{\pgfqpoint{2.082920in}{2.404606in}}%
\pgfpathcurveto{\pgfqpoint{2.088744in}{2.410430in}}{\pgfqpoint{2.092016in}{2.418330in}}{\pgfqpoint{2.092016in}{2.426567in}}%
\pgfpathcurveto{\pgfqpoint{2.092016in}{2.434803in}}{\pgfqpoint{2.088744in}{2.442703in}}{\pgfqpoint{2.082920in}{2.448527in}}%
\pgfpathcurveto{\pgfqpoint{2.077096in}{2.454351in}}{\pgfqpoint{2.069196in}{2.457623in}}{\pgfqpoint{2.060959in}{2.457623in}}%
\pgfpathcurveto{\pgfqpoint{2.052723in}{2.457623in}}{\pgfqpoint{2.044823in}{2.454351in}}{\pgfqpoint{2.038999in}{2.448527in}}%
\pgfpathcurveto{\pgfqpoint{2.033175in}{2.442703in}}{\pgfqpoint{2.029903in}{2.434803in}}{\pgfqpoint{2.029903in}{2.426567in}}%
\pgfpathcurveto{\pgfqpoint{2.029903in}{2.418330in}}{\pgfqpoint{2.033175in}{2.410430in}}{\pgfqpoint{2.038999in}{2.404606in}}%
\pgfpathcurveto{\pgfqpoint{2.044823in}{2.398782in}}{\pgfqpoint{2.052723in}{2.395510in}}{\pgfqpoint{2.060959in}{2.395510in}}%
\pgfpathclose%
\pgfusepath{stroke,fill}%
\end{pgfscope}%
\begin{pgfscope}%
\pgfpathrectangle{\pgfqpoint{0.100000in}{0.212622in}}{\pgfqpoint{3.696000in}{3.696000in}}%
\pgfusepath{clip}%
\pgfsetbuttcap%
\pgfsetroundjoin%
\definecolor{currentfill}{rgb}{0.121569,0.466667,0.705882}%
\pgfsetfillcolor{currentfill}%
\pgfsetfillopacity{0.957983}%
\pgfsetlinewidth{1.003750pt}%
\definecolor{currentstroke}{rgb}{0.121569,0.466667,0.705882}%
\pgfsetstrokecolor{currentstroke}%
\pgfsetstrokeopacity{0.957983}%
\pgfsetdash{}{0pt}%
\pgfpathmoveto{\pgfqpoint{2.035409in}{2.400208in}}%
\pgfpathcurveto{\pgfqpoint{2.043645in}{2.400208in}}{\pgfqpoint{2.051545in}{2.403481in}}{\pgfqpoint{2.057369in}{2.409305in}}%
\pgfpathcurveto{\pgfqpoint{2.063193in}{2.415129in}}{\pgfqpoint{2.066465in}{2.423029in}}{\pgfqpoint{2.066465in}{2.431265in}}%
\pgfpathcurveto{\pgfqpoint{2.066465in}{2.439501in}}{\pgfqpoint{2.063193in}{2.447401in}}{\pgfqpoint{2.057369in}{2.453225in}}%
\pgfpathcurveto{\pgfqpoint{2.051545in}{2.459049in}}{\pgfqpoint{2.043645in}{2.462321in}}{\pgfqpoint{2.035409in}{2.462321in}}%
\pgfpathcurveto{\pgfqpoint{2.027172in}{2.462321in}}{\pgfqpoint{2.019272in}{2.459049in}}{\pgfqpoint{2.013448in}{2.453225in}}%
\pgfpathcurveto{\pgfqpoint{2.007624in}{2.447401in}}{\pgfqpoint{2.004352in}{2.439501in}}{\pgfqpoint{2.004352in}{2.431265in}}%
\pgfpathcurveto{\pgfqpoint{2.004352in}{2.423029in}}{\pgfqpoint{2.007624in}{2.415129in}}{\pgfqpoint{2.013448in}{2.409305in}}%
\pgfpathcurveto{\pgfqpoint{2.019272in}{2.403481in}}{\pgfqpoint{2.027172in}{2.400208in}}{\pgfqpoint{2.035409in}{2.400208in}}%
\pgfpathclose%
\pgfusepath{stroke,fill}%
\end{pgfscope}%
\begin{pgfscope}%
\pgfpathrectangle{\pgfqpoint{0.100000in}{0.212622in}}{\pgfqpoint{3.696000in}{3.696000in}}%
\pgfusepath{clip}%
\pgfsetbuttcap%
\pgfsetroundjoin%
\definecolor{currentfill}{rgb}{0.121569,0.466667,0.705882}%
\pgfsetfillcolor{currentfill}%
\pgfsetfillopacity{0.958112}%
\pgfsetlinewidth{1.003750pt}%
\definecolor{currentstroke}{rgb}{0.121569,0.466667,0.705882}%
\pgfsetstrokecolor{currentstroke}%
\pgfsetstrokeopacity{0.958112}%
\pgfsetdash{}{0pt}%
\pgfpathmoveto{\pgfqpoint{1.492673in}{1.726787in}}%
\pgfpathcurveto{\pgfqpoint{1.500909in}{1.726787in}}{\pgfqpoint{1.508809in}{1.730059in}}{\pgfqpoint{1.514633in}{1.735883in}}%
\pgfpathcurveto{\pgfqpoint{1.520457in}{1.741707in}}{\pgfqpoint{1.523729in}{1.749607in}}{\pgfqpoint{1.523729in}{1.757843in}}%
\pgfpathcurveto{\pgfqpoint{1.523729in}{1.766079in}}{\pgfqpoint{1.520457in}{1.773979in}}{\pgfqpoint{1.514633in}{1.779803in}}%
\pgfpathcurveto{\pgfqpoint{1.508809in}{1.785627in}}{\pgfqpoint{1.500909in}{1.788900in}}{\pgfqpoint{1.492673in}{1.788900in}}%
\pgfpathcurveto{\pgfqpoint{1.484437in}{1.788900in}}{\pgfqpoint{1.476536in}{1.785627in}}{\pgfqpoint{1.470713in}{1.779803in}}%
\pgfpathcurveto{\pgfqpoint{1.464889in}{1.773979in}}{\pgfqpoint{1.461616in}{1.766079in}}{\pgfqpoint{1.461616in}{1.757843in}}%
\pgfpathcurveto{\pgfqpoint{1.461616in}{1.749607in}}{\pgfqpoint{1.464889in}{1.741707in}}{\pgfqpoint{1.470713in}{1.735883in}}%
\pgfpathcurveto{\pgfqpoint{1.476536in}{1.730059in}}{\pgfqpoint{1.484437in}{1.726787in}}{\pgfqpoint{1.492673in}{1.726787in}}%
\pgfpathclose%
\pgfusepath{stroke,fill}%
\end{pgfscope}%
\begin{pgfscope}%
\pgfpathrectangle{\pgfqpoint{0.100000in}{0.212622in}}{\pgfqpoint{3.696000in}{3.696000in}}%
\pgfusepath{clip}%
\pgfsetbuttcap%
\pgfsetroundjoin%
\definecolor{currentfill}{rgb}{0.121569,0.466667,0.705882}%
\pgfsetfillcolor{currentfill}%
\pgfsetfillopacity{0.958172}%
\pgfsetlinewidth{1.003750pt}%
\definecolor{currentstroke}{rgb}{0.121569,0.466667,0.705882}%
\pgfsetstrokecolor{currentstroke}%
\pgfsetstrokeopacity{0.958172}%
\pgfsetdash{}{0pt}%
\pgfpathmoveto{\pgfqpoint{2.057837in}{2.395330in}}%
\pgfpathcurveto{\pgfqpoint{2.066074in}{2.395330in}}{\pgfqpoint{2.073974in}{2.398602in}}{\pgfqpoint{2.079798in}{2.404426in}}%
\pgfpathcurveto{\pgfqpoint{2.085621in}{2.410250in}}{\pgfqpoint{2.088894in}{2.418150in}}{\pgfqpoint{2.088894in}{2.426386in}}%
\pgfpathcurveto{\pgfqpoint{2.088894in}{2.434622in}}{\pgfqpoint{2.085621in}{2.442522in}}{\pgfqpoint{2.079798in}{2.448346in}}%
\pgfpathcurveto{\pgfqpoint{2.073974in}{2.454170in}}{\pgfqpoint{2.066074in}{2.457443in}}{\pgfqpoint{2.057837in}{2.457443in}}%
\pgfpathcurveto{\pgfqpoint{2.049601in}{2.457443in}}{\pgfqpoint{2.041701in}{2.454170in}}{\pgfqpoint{2.035877in}{2.448346in}}%
\pgfpathcurveto{\pgfqpoint{2.030053in}{2.442522in}}{\pgfqpoint{2.026781in}{2.434622in}}{\pgfqpoint{2.026781in}{2.426386in}}%
\pgfpathcurveto{\pgfqpoint{2.026781in}{2.418150in}}{\pgfqpoint{2.030053in}{2.410250in}}{\pgfqpoint{2.035877in}{2.404426in}}%
\pgfpathcurveto{\pgfqpoint{2.041701in}{2.398602in}}{\pgfqpoint{2.049601in}{2.395330in}}{\pgfqpoint{2.057837in}{2.395330in}}%
\pgfpathclose%
\pgfusepath{stroke,fill}%
\end{pgfscope}%
\begin{pgfscope}%
\pgfpathrectangle{\pgfqpoint{0.100000in}{0.212622in}}{\pgfqpoint{3.696000in}{3.696000in}}%
\pgfusepath{clip}%
\pgfsetbuttcap%
\pgfsetroundjoin%
\definecolor{currentfill}{rgb}{0.121569,0.466667,0.705882}%
\pgfsetfillcolor{currentfill}%
\pgfsetfillopacity{0.958195}%
\pgfsetlinewidth{1.003750pt}%
\definecolor{currentstroke}{rgb}{0.121569,0.466667,0.705882}%
\pgfsetstrokecolor{currentstroke}%
\pgfsetstrokeopacity{0.958195}%
\pgfsetdash{}{0pt}%
\pgfpathmoveto{\pgfqpoint{2.037960in}{2.399166in}}%
\pgfpathcurveto{\pgfqpoint{2.046196in}{2.399166in}}{\pgfqpoint{2.054096in}{2.402438in}}{\pgfqpoint{2.059920in}{2.408262in}}%
\pgfpathcurveto{\pgfqpoint{2.065744in}{2.414086in}}{\pgfqpoint{2.069016in}{2.421986in}}{\pgfqpoint{2.069016in}{2.430223in}}%
\pgfpathcurveto{\pgfqpoint{2.069016in}{2.438459in}}{\pgfqpoint{2.065744in}{2.446359in}}{\pgfqpoint{2.059920in}{2.452183in}}%
\pgfpathcurveto{\pgfqpoint{2.054096in}{2.458007in}}{\pgfqpoint{2.046196in}{2.461279in}}{\pgfqpoint{2.037960in}{2.461279in}}%
\pgfpathcurveto{\pgfqpoint{2.029723in}{2.461279in}}{\pgfqpoint{2.021823in}{2.458007in}}{\pgfqpoint{2.015999in}{2.452183in}}%
\pgfpathcurveto{\pgfqpoint{2.010175in}{2.446359in}}{\pgfqpoint{2.006903in}{2.438459in}}{\pgfqpoint{2.006903in}{2.430223in}}%
\pgfpathcurveto{\pgfqpoint{2.006903in}{2.421986in}}{\pgfqpoint{2.010175in}{2.414086in}}{\pgfqpoint{2.015999in}{2.408262in}}%
\pgfpathcurveto{\pgfqpoint{2.021823in}{2.402438in}}{\pgfqpoint{2.029723in}{2.399166in}}{\pgfqpoint{2.037960in}{2.399166in}}%
\pgfpathclose%
\pgfusepath{stroke,fill}%
\end{pgfscope}%
\begin{pgfscope}%
\pgfpathrectangle{\pgfqpoint{0.100000in}{0.212622in}}{\pgfqpoint{3.696000in}{3.696000in}}%
\pgfusepath{clip}%
\pgfsetbuttcap%
\pgfsetroundjoin%
\definecolor{currentfill}{rgb}{0.121569,0.466667,0.705882}%
\pgfsetfillcolor{currentfill}%
\pgfsetfillopacity{0.958358}%
\pgfsetlinewidth{1.003750pt}%
\definecolor{currentstroke}{rgb}{0.121569,0.466667,0.705882}%
\pgfsetstrokecolor{currentstroke}%
\pgfsetstrokeopacity{0.958358}%
\pgfsetdash{}{0pt}%
\pgfpathmoveto{\pgfqpoint{2.055130in}{2.395352in}}%
\pgfpathcurveto{\pgfqpoint{2.063366in}{2.395352in}}{\pgfqpoint{2.071267in}{2.398624in}}{\pgfqpoint{2.077090in}{2.404448in}}%
\pgfpathcurveto{\pgfqpoint{2.082914in}{2.410272in}}{\pgfqpoint{2.086187in}{2.418172in}}{\pgfqpoint{2.086187in}{2.426408in}}%
\pgfpathcurveto{\pgfqpoint{2.086187in}{2.434644in}}{\pgfqpoint{2.082914in}{2.442544in}}{\pgfqpoint{2.077090in}{2.448368in}}%
\pgfpathcurveto{\pgfqpoint{2.071267in}{2.454192in}}{\pgfqpoint{2.063366in}{2.457465in}}{\pgfqpoint{2.055130in}{2.457465in}}%
\pgfpathcurveto{\pgfqpoint{2.046894in}{2.457465in}}{\pgfqpoint{2.038994in}{2.454192in}}{\pgfqpoint{2.033170in}{2.448368in}}%
\pgfpathcurveto{\pgfqpoint{2.027346in}{2.442544in}}{\pgfqpoint{2.024074in}{2.434644in}}{\pgfqpoint{2.024074in}{2.426408in}}%
\pgfpathcurveto{\pgfqpoint{2.024074in}{2.418172in}}{\pgfqpoint{2.027346in}{2.410272in}}{\pgfqpoint{2.033170in}{2.404448in}}%
\pgfpathcurveto{\pgfqpoint{2.038994in}{2.398624in}}{\pgfqpoint{2.046894in}{2.395352in}}{\pgfqpoint{2.055130in}{2.395352in}}%
\pgfpathclose%
\pgfusepath{stroke,fill}%
\end{pgfscope}%
\begin{pgfscope}%
\pgfpathrectangle{\pgfqpoint{0.100000in}{0.212622in}}{\pgfqpoint{3.696000in}{3.696000in}}%
\pgfusepath{clip}%
\pgfsetbuttcap%
\pgfsetroundjoin%
\definecolor{currentfill}{rgb}{0.121569,0.466667,0.705882}%
\pgfsetfillcolor{currentfill}%
\pgfsetfillopacity{0.958391}%
\pgfsetlinewidth{1.003750pt}%
\definecolor{currentstroke}{rgb}{0.121569,0.466667,0.705882}%
\pgfsetstrokecolor{currentstroke}%
\pgfsetstrokeopacity{0.958391}%
\pgfsetdash{}{0pt}%
\pgfpathmoveto{\pgfqpoint{2.040902in}{2.397974in}}%
\pgfpathcurveto{\pgfqpoint{2.049138in}{2.397974in}}{\pgfqpoint{2.057038in}{2.401246in}}{\pgfqpoint{2.062862in}{2.407070in}}%
\pgfpathcurveto{\pgfqpoint{2.068686in}{2.412894in}}{\pgfqpoint{2.071958in}{2.420794in}}{\pgfqpoint{2.071958in}{2.429030in}}%
\pgfpathcurveto{\pgfqpoint{2.071958in}{2.437267in}}{\pgfqpoint{2.068686in}{2.445167in}}{\pgfqpoint{2.062862in}{2.450991in}}%
\pgfpathcurveto{\pgfqpoint{2.057038in}{2.456814in}}{\pgfqpoint{2.049138in}{2.460087in}}{\pgfqpoint{2.040902in}{2.460087in}}%
\pgfpathcurveto{\pgfqpoint{2.032666in}{2.460087in}}{\pgfqpoint{2.024766in}{2.456814in}}{\pgfqpoint{2.018942in}{2.450991in}}%
\pgfpathcurveto{\pgfqpoint{2.013118in}{2.445167in}}{\pgfqpoint{2.009845in}{2.437267in}}{\pgfqpoint{2.009845in}{2.429030in}}%
\pgfpathcurveto{\pgfqpoint{2.009845in}{2.420794in}}{\pgfqpoint{2.013118in}{2.412894in}}{\pgfqpoint{2.018942in}{2.407070in}}%
\pgfpathcurveto{\pgfqpoint{2.024766in}{2.401246in}}{\pgfqpoint{2.032666in}{2.397974in}}{\pgfqpoint{2.040902in}{2.397974in}}%
\pgfpathclose%
\pgfusepath{stroke,fill}%
\end{pgfscope}%
\begin{pgfscope}%
\pgfpathrectangle{\pgfqpoint{0.100000in}{0.212622in}}{\pgfqpoint{3.696000in}{3.696000in}}%
\pgfusepath{clip}%
\pgfsetbuttcap%
\pgfsetroundjoin%
\definecolor{currentfill}{rgb}{0.121569,0.466667,0.705882}%
\pgfsetfillcolor{currentfill}%
\pgfsetfillopacity{0.958435}%
\pgfsetlinewidth{1.003750pt}%
\definecolor{currentstroke}{rgb}{0.121569,0.466667,0.705882}%
\pgfsetstrokecolor{currentstroke}%
\pgfsetstrokeopacity{0.958435}%
\pgfsetdash{}{0pt}%
\pgfpathmoveto{\pgfqpoint{2.042604in}{2.397326in}}%
\pgfpathcurveto{\pgfqpoint{2.050840in}{2.397326in}}{\pgfqpoint{2.058740in}{2.400599in}}{\pgfqpoint{2.064564in}{2.406422in}}%
\pgfpathcurveto{\pgfqpoint{2.070388in}{2.412246in}}{\pgfqpoint{2.073661in}{2.420146in}}{\pgfqpoint{2.073661in}{2.428383in}}%
\pgfpathcurveto{\pgfqpoint{2.073661in}{2.436619in}}{\pgfqpoint{2.070388in}{2.444519in}}{\pgfqpoint{2.064564in}{2.450343in}}%
\pgfpathcurveto{\pgfqpoint{2.058740in}{2.456167in}}{\pgfqpoint{2.050840in}{2.459439in}}{\pgfqpoint{2.042604in}{2.459439in}}%
\pgfpathcurveto{\pgfqpoint{2.034368in}{2.459439in}}{\pgfqpoint{2.026468in}{2.456167in}}{\pgfqpoint{2.020644in}{2.450343in}}%
\pgfpathcurveto{\pgfqpoint{2.014820in}{2.444519in}}{\pgfqpoint{2.011548in}{2.436619in}}{\pgfqpoint{2.011548in}{2.428383in}}%
\pgfpathcurveto{\pgfqpoint{2.011548in}{2.420146in}}{\pgfqpoint{2.014820in}{2.412246in}}{\pgfqpoint{2.020644in}{2.406422in}}%
\pgfpathcurveto{\pgfqpoint{2.026468in}{2.400599in}}{\pgfqpoint{2.034368in}{2.397326in}}{\pgfqpoint{2.042604in}{2.397326in}}%
\pgfpathclose%
\pgfusepath{stroke,fill}%
\end{pgfscope}%
\begin{pgfscope}%
\pgfpathrectangle{\pgfqpoint{0.100000in}{0.212622in}}{\pgfqpoint{3.696000in}{3.696000in}}%
\pgfusepath{clip}%
\pgfsetbuttcap%
\pgfsetroundjoin%
\definecolor{currentfill}{rgb}{0.121569,0.466667,0.705882}%
\pgfsetfillcolor{currentfill}%
\pgfsetfillopacity{0.958458}%
\pgfsetlinewidth{1.003750pt}%
\definecolor{currentstroke}{rgb}{0.121569,0.466667,0.705882}%
\pgfsetstrokecolor{currentstroke}%
\pgfsetstrokeopacity{0.958458}%
\pgfsetdash{}{0pt}%
\pgfpathmoveto{\pgfqpoint{2.053120in}{2.395432in}}%
\pgfpathcurveto{\pgfqpoint{2.061357in}{2.395432in}}{\pgfqpoint{2.069257in}{2.398704in}}{\pgfqpoint{2.075081in}{2.404528in}}%
\pgfpathcurveto{\pgfqpoint{2.080904in}{2.410352in}}{\pgfqpoint{2.084177in}{2.418252in}}{\pgfqpoint{2.084177in}{2.426489in}}%
\pgfpathcurveto{\pgfqpoint{2.084177in}{2.434725in}}{\pgfqpoint{2.080904in}{2.442625in}}{\pgfqpoint{2.075081in}{2.448449in}}%
\pgfpathcurveto{\pgfqpoint{2.069257in}{2.454273in}}{\pgfqpoint{2.061357in}{2.457545in}}{\pgfqpoint{2.053120in}{2.457545in}}%
\pgfpathcurveto{\pgfqpoint{2.044884in}{2.457545in}}{\pgfqpoint{2.036984in}{2.454273in}}{\pgfqpoint{2.031160in}{2.448449in}}%
\pgfpathcurveto{\pgfqpoint{2.025336in}{2.442625in}}{\pgfqpoint{2.022064in}{2.434725in}}{\pgfqpoint{2.022064in}{2.426489in}}%
\pgfpathcurveto{\pgfqpoint{2.022064in}{2.418252in}}{\pgfqpoint{2.025336in}{2.410352in}}{\pgfqpoint{2.031160in}{2.404528in}}%
\pgfpathcurveto{\pgfqpoint{2.036984in}{2.398704in}}{\pgfqpoint{2.044884in}{2.395432in}}{\pgfqpoint{2.053120in}{2.395432in}}%
\pgfpathclose%
\pgfusepath{stroke,fill}%
\end{pgfscope}%
\begin{pgfscope}%
\pgfpathrectangle{\pgfqpoint{0.100000in}{0.212622in}}{\pgfqpoint{3.696000in}{3.696000in}}%
\pgfusepath{clip}%
\pgfsetbuttcap%
\pgfsetroundjoin%
\definecolor{currentfill}{rgb}{0.121569,0.466667,0.705882}%
\pgfsetfillcolor{currentfill}%
\pgfsetfillopacity{0.958470}%
\pgfsetlinewidth{1.003750pt}%
\definecolor{currentstroke}{rgb}{0.121569,0.466667,0.705882}%
\pgfsetstrokecolor{currentstroke}%
\pgfsetstrokeopacity{0.958470}%
\pgfsetdash{}{0pt}%
\pgfpathmoveto{\pgfqpoint{2.051662in}{2.395514in}}%
\pgfpathcurveto{\pgfqpoint{2.059898in}{2.395514in}}{\pgfqpoint{2.067798in}{2.398786in}}{\pgfqpoint{2.073622in}{2.404610in}}%
\pgfpathcurveto{\pgfqpoint{2.079446in}{2.410434in}}{\pgfqpoint{2.082718in}{2.418334in}}{\pgfqpoint{2.082718in}{2.426570in}}%
\pgfpathcurveto{\pgfqpoint{2.082718in}{2.434807in}}{\pgfqpoint{2.079446in}{2.442707in}}{\pgfqpoint{2.073622in}{2.448531in}}%
\pgfpathcurveto{\pgfqpoint{2.067798in}{2.454354in}}{\pgfqpoint{2.059898in}{2.457627in}}{\pgfqpoint{2.051662in}{2.457627in}}%
\pgfpathcurveto{\pgfqpoint{2.043425in}{2.457627in}}{\pgfqpoint{2.035525in}{2.454354in}}{\pgfqpoint{2.029701in}{2.448531in}}%
\pgfpathcurveto{\pgfqpoint{2.023877in}{2.442707in}}{\pgfqpoint{2.020605in}{2.434807in}}{\pgfqpoint{2.020605in}{2.426570in}}%
\pgfpathcurveto{\pgfqpoint{2.020605in}{2.418334in}}{\pgfqpoint{2.023877in}{2.410434in}}{\pgfqpoint{2.029701in}{2.404610in}}%
\pgfpathcurveto{\pgfqpoint{2.035525in}{2.398786in}}{\pgfqpoint{2.043425in}{2.395514in}}{\pgfqpoint{2.051662in}{2.395514in}}%
\pgfpathclose%
\pgfusepath{stroke,fill}%
\end{pgfscope}%
\begin{pgfscope}%
\pgfpathrectangle{\pgfqpoint{0.100000in}{0.212622in}}{\pgfqpoint{3.696000in}{3.696000in}}%
\pgfusepath{clip}%
\pgfsetbuttcap%
\pgfsetroundjoin%
\definecolor{currentfill}{rgb}{0.121569,0.466667,0.705882}%
\pgfsetfillcolor{currentfill}%
\pgfsetfillopacity{0.958554}%
\pgfsetlinewidth{1.003750pt}%
\definecolor{currentstroke}{rgb}{0.121569,0.466667,0.705882}%
\pgfsetstrokecolor{currentstroke}%
\pgfsetstrokeopacity{0.958554}%
\pgfsetdash{}{0pt}%
\pgfpathmoveto{\pgfqpoint{2.048993in}{2.395880in}}%
\pgfpathcurveto{\pgfqpoint{2.057229in}{2.395880in}}{\pgfqpoint{2.065129in}{2.399153in}}{\pgfqpoint{2.070953in}{2.404977in}}%
\pgfpathcurveto{\pgfqpoint{2.076777in}{2.410801in}}{\pgfqpoint{2.080049in}{2.418701in}}{\pgfqpoint{2.080049in}{2.426937in}}%
\pgfpathcurveto{\pgfqpoint{2.080049in}{2.435173in}}{\pgfqpoint{2.076777in}{2.443073in}}{\pgfqpoint{2.070953in}{2.448897in}}%
\pgfpathcurveto{\pgfqpoint{2.065129in}{2.454721in}}{\pgfqpoint{2.057229in}{2.457993in}}{\pgfqpoint{2.048993in}{2.457993in}}%
\pgfpathcurveto{\pgfqpoint{2.040756in}{2.457993in}}{\pgfqpoint{2.032856in}{2.454721in}}{\pgfqpoint{2.027032in}{2.448897in}}%
\pgfpathcurveto{\pgfqpoint{2.021209in}{2.443073in}}{\pgfqpoint{2.017936in}{2.435173in}}{\pgfqpoint{2.017936in}{2.426937in}}%
\pgfpathcurveto{\pgfqpoint{2.017936in}{2.418701in}}{\pgfqpoint{2.021209in}{2.410801in}}{\pgfqpoint{2.027032in}{2.404977in}}%
\pgfpathcurveto{\pgfqpoint{2.032856in}{2.399153in}}{\pgfqpoint{2.040756in}{2.395880in}}{\pgfqpoint{2.048993in}{2.395880in}}%
\pgfpathclose%
\pgfusepath{stroke,fill}%
\end{pgfscope}%
\begin{pgfscope}%
\pgfpathrectangle{\pgfqpoint{0.100000in}{0.212622in}}{\pgfqpoint{3.696000in}{3.696000in}}%
\pgfusepath{clip}%
\pgfsetbuttcap%
\pgfsetroundjoin%
\definecolor{currentfill}{rgb}{0.121569,0.466667,0.705882}%
\pgfsetfillcolor{currentfill}%
\pgfsetfillopacity{0.958567}%
\pgfsetlinewidth{1.003750pt}%
\definecolor{currentstroke}{rgb}{0.121569,0.466667,0.705882}%
\pgfsetstrokecolor{currentstroke}%
\pgfsetstrokeopacity{0.958567}%
\pgfsetdash{}{0pt}%
\pgfpathmoveto{\pgfqpoint{2.045403in}{2.396651in}}%
\pgfpathcurveto{\pgfqpoint{2.053639in}{2.396651in}}{\pgfqpoint{2.061539in}{2.399923in}}{\pgfqpoint{2.067363in}{2.405747in}}%
\pgfpathcurveto{\pgfqpoint{2.073187in}{2.411571in}}{\pgfqpoint{2.076459in}{2.419471in}}{\pgfqpoint{2.076459in}{2.427707in}}%
\pgfpathcurveto{\pgfqpoint{2.076459in}{2.435943in}}{\pgfqpoint{2.073187in}{2.443843in}}{\pgfqpoint{2.067363in}{2.449667in}}%
\pgfpathcurveto{\pgfqpoint{2.061539in}{2.455491in}}{\pgfqpoint{2.053639in}{2.458764in}}{\pgfqpoint{2.045403in}{2.458764in}}%
\pgfpathcurveto{\pgfqpoint{2.037166in}{2.458764in}}{\pgfqpoint{2.029266in}{2.455491in}}{\pgfqpoint{2.023442in}{2.449667in}}%
\pgfpathcurveto{\pgfqpoint{2.017618in}{2.443843in}}{\pgfqpoint{2.014346in}{2.435943in}}{\pgfqpoint{2.014346in}{2.427707in}}%
\pgfpathcurveto{\pgfqpoint{2.014346in}{2.419471in}}{\pgfqpoint{2.017618in}{2.411571in}}{\pgfqpoint{2.023442in}{2.405747in}}%
\pgfpathcurveto{\pgfqpoint{2.029266in}{2.399923in}}{\pgfqpoint{2.037166in}{2.396651in}}{\pgfqpoint{2.045403in}{2.396651in}}%
\pgfpathclose%
\pgfusepath{stroke,fill}%
\end{pgfscope}%
\begin{pgfscope}%
\pgfpathrectangle{\pgfqpoint{0.100000in}{0.212622in}}{\pgfqpoint{3.696000in}{3.696000in}}%
\pgfusepath{clip}%
\pgfsetbuttcap%
\pgfsetroundjoin%
\definecolor{currentfill}{rgb}{0.121569,0.466667,0.705882}%
\pgfsetfillcolor{currentfill}%
\pgfsetfillopacity{0.958579}%
\pgfsetlinewidth{1.003750pt}%
\definecolor{currentstroke}{rgb}{0.121569,0.466667,0.705882}%
\pgfsetstrokecolor{currentstroke}%
\pgfsetstrokeopacity{0.958579}%
\pgfsetdash{}{0pt}%
\pgfpathmoveto{\pgfqpoint{2.047001in}{2.396246in}}%
\pgfpathcurveto{\pgfqpoint{2.055237in}{2.396246in}}{\pgfqpoint{2.063137in}{2.399518in}}{\pgfqpoint{2.068961in}{2.405342in}}%
\pgfpathcurveto{\pgfqpoint{2.074785in}{2.411166in}}{\pgfqpoint{2.078058in}{2.419066in}}{\pgfqpoint{2.078058in}{2.427302in}}%
\pgfpathcurveto{\pgfqpoint{2.078058in}{2.435539in}}{\pgfqpoint{2.074785in}{2.443439in}}{\pgfqpoint{2.068961in}{2.449262in}}%
\pgfpathcurveto{\pgfqpoint{2.063137in}{2.455086in}}{\pgfqpoint{2.055237in}{2.458359in}}{\pgfqpoint{2.047001in}{2.458359in}}%
\pgfpathcurveto{\pgfqpoint{2.038765in}{2.458359in}}{\pgfqpoint{2.030865in}{2.455086in}}{\pgfqpoint{2.025041in}{2.449262in}}%
\pgfpathcurveto{\pgfqpoint{2.019217in}{2.443439in}}{\pgfqpoint{2.015945in}{2.435539in}}{\pgfqpoint{2.015945in}{2.427302in}}%
\pgfpathcurveto{\pgfqpoint{2.015945in}{2.419066in}}{\pgfqpoint{2.019217in}{2.411166in}}{\pgfqpoint{2.025041in}{2.405342in}}%
\pgfpathcurveto{\pgfqpoint{2.030865in}{2.399518in}}{\pgfqpoint{2.038765in}{2.396246in}}{\pgfqpoint{2.047001in}{2.396246in}}%
\pgfpathclose%
\pgfusepath{stroke,fill}%
\end{pgfscope}%
\begin{pgfscope}%
\pgfpathrectangle{\pgfqpoint{0.100000in}{0.212622in}}{\pgfqpoint{3.696000in}{3.696000in}}%
\pgfusepath{clip}%
\pgfsetbuttcap%
\pgfsetroundjoin%
\definecolor{currentfill}{rgb}{0.121569,0.466667,0.705882}%
\pgfsetfillcolor{currentfill}%
\pgfsetfillopacity{0.958646}%
\pgfsetlinewidth{1.003750pt}%
\definecolor{currentstroke}{rgb}{0.121569,0.466667,0.705882}%
\pgfsetstrokecolor{currentstroke}%
\pgfsetstrokeopacity{0.958646}%
\pgfsetdash{}{0pt}%
\pgfpathmoveto{\pgfqpoint{1.504347in}{1.719057in}}%
\pgfpathcurveto{\pgfqpoint{1.512583in}{1.719057in}}{\pgfqpoint{1.520484in}{1.722329in}}{\pgfqpoint{1.526307in}{1.728153in}}%
\pgfpathcurveto{\pgfqpoint{1.532131in}{1.733977in}}{\pgfqpoint{1.535404in}{1.741877in}}{\pgfqpoint{1.535404in}{1.750114in}}%
\pgfpathcurveto{\pgfqpoint{1.535404in}{1.758350in}}{\pgfqpoint{1.532131in}{1.766250in}}{\pgfqpoint{1.526307in}{1.772074in}}%
\pgfpathcurveto{\pgfqpoint{1.520484in}{1.777898in}}{\pgfqpoint{1.512583in}{1.781170in}}{\pgfqpoint{1.504347in}{1.781170in}}%
\pgfpathcurveto{\pgfqpoint{1.496111in}{1.781170in}}{\pgfqpoint{1.488211in}{1.777898in}}{\pgfqpoint{1.482387in}{1.772074in}}%
\pgfpathcurveto{\pgfqpoint{1.476563in}{1.766250in}}{\pgfqpoint{1.473291in}{1.758350in}}{\pgfqpoint{1.473291in}{1.750114in}}%
\pgfpathcurveto{\pgfqpoint{1.473291in}{1.741877in}}{\pgfqpoint{1.476563in}{1.733977in}}{\pgfqpoint{1.482387in}{1.728153in}}%
\pgfpathcurveto{\pgfqpoint{1.488211in}{1.722329in}}{\pgfqpoint{1.496111in}{1.719057in}}{\pgfqpoint{1.504347in}{1.719057in}}%
\pgfpathclose%
\pgfusepath{stroke,fill}%
\end{pgfscope}%
\begin{pgfscope}%
\pgfpathrectangle{\pgfqpoint{0.100000in}{0.212622in}}{\pgfqpoint{3.696000in}{3.696000in}}%
\pgfusepath{clip}%
\pgfsetbuttcap%
\pgfsetroundjoin%
\definecolor{currentfill}{rgb}{0.121569,0.466667,0.705882}%
\pgfsetfillcolor{currentfill}%
\pgfsetfillopacity{0.958692}%
\pgfsetlinewidth{1.003750pt}%
\definecolor{currentstroke}{rgb}{0.121569,0.466667,0.705882}%
\pgfsetstrokecolor{currentstroke}%
\pgfsetstrokeopacity{0.958692}%
\pgfsetdash{}{0pt}%
\pgfpathmoveto{\pgfqpoint{2.567986in}{1.178136in}}%
\pgfpathcurveto{\pgfqpoint{2.576222in}{1.178136in}}{\pgfqpoint{2.584122in}{1.181408in}}{\pgfqpoint{2.589946in}{1.187232in}}%
\pgfpathcurveto{\pgfqpoint{2.595770in}{1.193056in}}{\pgfqpoint{2.599042in}{1.200956in}}{\pgfqpoint{2.599042in}{1.209193in}}%
\pgfpathcurveto{\pgfqpoint{2.599042in}{1.217429in}}{\pgfqpoint{2.595770in}{1.225329in}}{\pgfqpoint{2.589946in}{1.231153in}}%
\pgfpathcurveto{\pgfqpoint{2.584122in}{1.236977in}}{\pgfqpoint{2.576222in}{1.240249in}}{\pgfqpoint{2.567986in}{1.240249in}}%
\pgfpathcurveto{\pgfqpoint{2.559749in}{1.240249in}}{\pgfqpoint{2.551849in}{1.236977in}}{\pgfqpoint{2.546025in}{1.231153in}}%
\pgfpathcurveto{\pgfqpoint{2.540201in}{1.225329in}}{\pgfqpoint{2.536929in}{1.217429in}}{\pgfqpoint{2.536929in}{1.209193in}}%
\pgfpathcurveto{\pgfqpoint{2.536929in}{1.200956in}}{\pgfqpoint{2.540201in}{1.193056in}}{\pgfqpoint{2.546025in}{1.187232in}}%
\pgfpathcurveto{\pgfqpoint{2.551849in}{1.181408in}}{\pgfqpoint{2.559749in}{1.178136in}}{\pgfqpoint{2.567986in}{1.178136in}}%
\pgfpathclose%
\pgfusepath{stroke,fill}%
\end{pgfscope}%
\begin{pgfscope}%
\pgfpathrectangle{\pgfqpoint{0.100000in}{0.212622in}}{\pgfqpoint{3.696000in}{3.696000in}}%
\pgfusepath{clip}%
\pgfsetbuttcap%
\pgfsetroundjoin%
\definecolor{currentfill}{rgb}{0.121569,0.466667,0.705882}%
\pgfsetfillcolor{currentfill}%
\pgfsetfillopacity{0.959303}%
\pgfsetlinewidth{1.003750pt}%
\definecolor{currentstroke}{rgb}{0.121569,0.466667,0.705882}%
\pgfsetstrokecolor{currentstroke}%
\pgfsetstrokeopacity{0.959303}%
\pgfsetdash{}{0pt}%
\pgfpathmoveto{\pgfqpoint{1.517516in}{1.709629in}}%
\pgfpathcurveto{\pgfqpoint{1.525753in}{1.709629in}}{\pgfqpoint{1.533653in}{1.712901in}}{\pgfqpoint{1.539477in}{1.718725in}}%
\pgfpathcurveto{\pgfqpoint{1.545301in}{1.724549in}}{\pgfqpoint{1.548573in}{1.732449in}}{\pgfqpoint{1.548573in}{1.740685in}}%
\pgfpathcurveto{\pgfqpoint{1.548573in}{1.748921in}}{\pgfqpoint{1.545301in}{1.756822in}}{\pgfqpoint{1.539477in}{1.762645in}}%
\pgfpathcurveto{\pgfqpoint{1.533653in}{1.768469in}}{\pgfqpoint{1.525753in}{1.771742in}}{\pgfqpoint{1.517516in}{1.771742in}}%
\pgfpathcurveto{\pgfqpoint{1.509280in}{1.771742in}}{\pgfqpoint{1.501380in}{1.768469in}}{\pgfqpoint{1.495556in}{1.762645in}}%
\pgfpathcurveto{\pgfqpoint{1.489732in}{1.756822in}}{\pgfqpoint{1.486460in}{1.748921in}}{\pgfqpoint{1.486460in}{1.740685in}}%
\pgfpathcurveto{\pgfqpoint{1.486460in}{1.732449in}}{\pgfqpoint{1.489732in}{1.724549in}}{\pgfqpoint{1.495556in}{1.718725in}}%
\pgfpathcurveto{\pgfqpoint{1.501380in}{1.712901in}}{\pgfqpoint{1.509280in}{1.709629in}}{\pgfqpoint{1.517516in}{1.709629in}}%
\pgfpathclose%
\pgfusepath{stroke,fill}%
\end{pgfscope}%
\begin{pgfscope}%
\pgfpathrectangle{\pgfqpoint{0.100000in}{0.212622in}}{\pgfqpoint{3.696000in}{3.696000in}}%
\pgfusepath{clip}%
\pgfsetbuttcap%
\pgfsetroundjoin%
\definecolor{currentfill}{rgb}{0.121569,0.466667,0.705882}%
\pgfsetfillcolor{currentfill}%
\pgfsetfillopacity{0.959816}%
\pgfsetlinewidth{1.003750pt}%
\definecolor{currentstroke}{rgb}{0.121569,0.466667,0.705882}%
\pgfsetstrokecolor{currentstroke}%
\pgfsetstrokeopacity{0.959816}%
\pgfsetdash{}{0pt}%
\pgfpathmoveto{\pgfqpoint{2.565820in}{1.175922in}}%
\pgfpathcurveto{\pgfqpoint{2.574056in}{1.175922in}}{\pgfqpoint{2.581956in}{1.179195in}}{\pgfqpoint{2.587780in}{1.185019in}}%
\pgfpathcurveto{\pgfqpoint{2.593604in}{1.190843in}}{\pgfqpoint{2.596876in}{1.198743in}}{\pgfqpoint{2.596876in}{1.206979in}}%
\pgfpathcurveto{\pgfqpoint{2.596876in}{1.215215in}}{\pgfqpoint{2.593604in}{1.223115in}}{\pgfqpoint{2.587780in}{1.228939in}}%
\pgfpathcurveto{\pgfqpoint{2.581956in}{1.234763in}}{\pgfqpoint{2.574056in}{1.238035in}}{\pgfqpoint{2.565820in}{1.238035in}}%
\pgfpathcurveto{\pgfqpoint{2.557583in}{1.238035in}}{\pgfqpoint{2.549683in}{1.234763in}}{\pgfqpoint{2.543859in}{1.228939in}}%
\pgfpathcurveto{\pgfqpoint{2.538036in}{1.223115in}}{\pgfqpoint{2.534763in}{1.215215in}}{\pgfqpoint{2.534763in}{1.206979in}}%
\pgfpathcurveto{\pgfqpoint{2.534763in}{1.198743in}}{\pgfqpoint{2.538036in}{1.190843in}}{\pgfqpoint{2.543859in}{1.185019in}}%
\pgfpathcurveto{\pgfqpoint{2.549683in}{1.179195in}}{\pgfqpoint{2.557583in}{1.175922in}}{\pgfqpoint{2.565820in}{1.175922in}}%
\pgfpathclose%
\pgfusepath{stroke,fill}%
\end{pgfscope}%
\begin{pgfscope}%
\pgfpathrectangle{\pgfqpoint{0.100000in}{0.212622in}}{\pgfqpoint{3.696000in}{3.696000in}}%
\pgfusepath{clip}%
\pgfsetbuttcap%
\pgfsetroundjoin%
\definecolor{currentfill}{rgb}{0.121569,0.466667,0.705882}%
\pgfsetfillcolor{currentfill}%
\pgfsetfillopacity{0.959817}%
\pgfsetlinewidth{1.003750pt}%
\definecolor{currentstroke}{rgb}{0.121569,0.466667,0.705882}%
\pgfsetstrokecolor{currentstroke}%
\pgfsetstrokeopacity{0.959817}%
\pgfsetdash{}{0pt}%
\pgfpathmoveto{\pgfqpoint{1.530832in}{1.698595in}}%
\pgfpathcurveto{\pgfqpoint{1.539068in}{1.698595in}}{\pgfqpoint{1.546968in}{1.701867in}}{\pgfqpoint{1.552792in}{1.707691in}}%
\pgfpathcurveto{\pgfqpoint{1.558616in}{1.713515in}}{\pgfqpoint{1.561888in}{1.721415in}}{\pgfqpoint{1.561888in}{1.729651in}}%
\pgfpathcurveto{\pgfqpoint{1.561888in}{1.737888in}}{\pgfqpoint{1.558616in}{1.745788in}}{\pgfqpoint{1.552792in}{1.751612in}}%
\pgfpathcurveto{\pgfqpoint{1.546968in}{1.757436in}}{\pgfqpoint{1.539068in}{1.760708in}}{\pgfqpoint{1.530832in}{1.760708in}}%
\pgfpathcurveto{\pgfqpoint{1.522596in}{1.760708in}}{\pgfqpoint{1.514696in}{1.757436in}}{\pgfqpoint{1.508872in}{1.751612in}}%
\pgfpathcurveto{\pgfqpoint{1.503048in}{1.745788in}}{\pgfqpoint{1.499775in}{1.737888in}}{\pgfqpoint{1.499775in}{1.729651in}}%
\pgfpathcurveto{\pgfqpoint{1.499775in}{1.721415in}}{\pgfqpoint{1.503048in}{1.713515in}}{\pgfqpoint{1.508872in}{1.707691in}}%
\pgfpathcurveto{\pgfqpoint{1.514696in}{1.701867in}}{\pgfqpoint{1.522596in}{1.698595in}}{\pgfqpoint{1.530832in}{1.698595in}}%
\pgfpathclose%
\pgfusepath{stroke,fill}%
\end{pgfscope}%
\begin{pgfscope}%
\pgfpathrectangle{\pgfqpoint{0.100000in}{0.212622in}}{\pgfqpoint{3.696000in}{3.696000in}}%
\pgfusepath{clip}%
\pgfsetbuttcap%
\pgfsetroundjoin%
\definecolor{currentfill}{rgb}{0.121569,0.466667,0.705882}%
\pgfsetfillcolor{currentfill}%
\pgfsetfillopacity{0.960443}%
\pgfsetlinewidth{1.003750pt}%
\definecolor{currentstroke}{rgb}{0.121569,0.466667,0.705882}%
\pgfsetstrokecolor{currentstroke}%
\pgfsetstrokeopacity{0.960443}%
\pgfsetdash{}{0pt}%
\pgfpathmoveto{\pgfqpoint{1.544826in}{1.687556in}}%
\pgfpathcurveto{\pgfqpoint{1.553062in}{1.687556in}}{\pgfqpoint{1.560962in}{1.690829in}}{\pgfqpoint{1.566786in}{1.696653in}}%
\pgfpathcurveto{\pgfqpoint{1.572610in}{1.702477in}}{\pgfqpoint{1.575883in}{1.710377in}}{\pgfqpoint{1.575883in}{1.718613in}}%
\pgfpathcurveto{\pgfqpoint{1.575883in}{1.726849in}}{\pgfqpoint{1.572610in}{1.734749in}}{\pgfqpoint{1.566786in}{1.740573in}}%
\pgfpathcurveto{\pgfqpoint{1.560962in}{1.746397in}}{\pgfqpoint{1.553062in}{1.749669in}}{\pgfqpoint{1.544826in}{1.749669in}}%
\pgfpathcurveto{\pgfqpoint{1.536590in}{1.749669in}}{\pgfqpoint{1.528690in}{1.746397in}}{\pgfqpoint{1.522866in}{1.740573in}}%
\pgfpathcurveto{\pgfqpoint{1.517042in}{1.734749in}}{\pgfqpoint{1.513770in}{1.726849in}}{\pgfqpoint{1.513770in}{1.718613in}}%
\pgfpathcurveto{\pgfqpoint{1.513770in}{1.710377in}}{\pgfqpoint{1.517042in}{1.702477in}}{\pgfqpoint{1.522866in}{1.696653in}}%
\pgfpathcurveto{\pgfqpoint{1.528690in}{1.690829in}}{\pgfqpoint{1.536590in}{1.687556in}}{\pgfqpoint{1.544826in}{1.687556in}}%
\pgfpathclose%
\pgfusepath{stroke,fill}%
\end{pgfscope}%
\begin{pgfscope}%
\pgfpathrectangle{\pgfqpoint{0.100000in}{0.212622in}}{\pgfqpoint{3.696000in}{3.696000in}}%
\pgfusepath{clip}%
\pgfsetbuttcap%
\pgfsetroundjoin%
\definecolor{currentfill}{rgb}{0.121569,0.466667,0.705882}%
\pgfsetfillcolor{currentfill}%
\pgfsetfillopacity{0.960746}%
\pgfsetlinewidth{1.003750pt}%
\definecolor{currentstroke}{rgb}{0.121569,0.466667,0.705882}%
\pgfsetstrokecolor{currentstroke}%
\pgfsetstrokeopacity{0.960746}%
\pgfsetdash{}{0pt}%
\pgfpathmoveto{\pgfqpoint{2.564101in}{1.173963in}}%
\pgfpathcurveto{\pgfqpoint{2.572337in}{1.173963in}}{\pgfqpoint{2.580237in}{1.177235in}}{\pgfqpoint{2.586061in}{1.183059in}}%
\pgfpathcurveto{\pgfqpoint{2.591885in}{1.188883in}}{\pgfqpoint{2.595157in}{1.196783in}}{\pgfqpoint{2.595157in}{1.205020in}}%
\pgfpathcurveto{\pgfqpoint{2.595157in}{1.213256in}}{\pgfqpoint{2.591885in}{1.221156in}}{\pgfqpoint{2.586061in}{1.226980in}}%
\pgfpathcurveto{\pgfqpoint{2.580237in}{1.232804in}}{\pgfqpoint{2.572337in}{1.236076in}}{\pgfqpoint{2.564101in}{1.236076in}}%
\pgfpathcurveto{\pgfqpoint{2.555864in}{1.236076in}}{\pgfqpoint{2.547964in}{1.232804in}}{\pgfqpoint{2.542140in}{1.226980in}}%
\pgfpathcurveto{\pgfqpoint{2.536317in}{1.221156in}}{\pgfqpoint{2.533044in}{1.213256in}}{\pgfqpoint{2.533044in}{1.205020in}}%
\pgfpathcurveto{\pgfqpoint{2.533044in}{1.196783in}}{\pgfqpoint{2.536317in}{1.188883in}}{\pgfqpoint{2.542140in}{1.183059in}}%
\pgfpathcurveto{\pgfqpoint{2.547964in}{1.177235in}}{\pgfqpoint{2.555864in}{1.173963in}}{\pgfqpoint{2.564101in}{1.173963in}}%
\pgfpathclose%
\pgfusepath{stroke,fill}%
\end{pgfscope}%
\begin{pgfscope}%
\pgfpathrectangle{\pgfqpoint{0.100000in}{0.212622in}}{\pgfqpoint{3.696000in}{3.696000in}}%
\pgfusepath{clip}%
\pgfsetbuttcap%
\pgfsetroundjoin%
\definecolor{currentfill}{rgb}{0.121569,0.466667,0.705882}%
\pgfsetfillcolor{currentfill}%
\pgfsetfillopacity{0.960765}%
\pgfsetlinewidth{1.003750pt}%
\definecolor{currentstroke}{rgb}{0.121569,0.466667,0.705882}%
\pgfsetstrokecolor{currentstroke}%
\pgfsetstrokeopacity{0.960765}%
\pgfsetdash{}{0pt}%
\pgfpathmoveto{\pgfqpoint{1.552608in}{1.681659in}}%
\pgfpathcurveto{\pgfqpoint{1.560844in}{1.681659in}}{\pgfqpoint{1.568744in}{1.684931in}}{\pgfqpoint{1.574568in}{1.690755in}}%
\pgfpathcurveto{\pgfqpoint{1.580392in}{1.696579in}}{\pgfqpoint{1.583664in}{1.704479in}}{\pgfqpoint{1.583664in}{1.712715in}}%
\pgfpathcurveto{\pgfqpoint{1.583664in}{1.720952in}}{\pgfqpoint{1.580392in}{1.728852in}}{\pgfqpoint{1.574568in}{1.734676in}}%
\pgfpathcurveto{\pgfqpoint{1.568744in}{1.740500in}}{\pgfqpoint{1.560844in}{1.743772in}}{\pgfqpoint{1.552608in}{1.743772in}}%
\pgfpathcurveto{\pgfqpoint{1.544371in}{1.743772in}}{\pgfqpoint{1.536471in}{1.740500in}}{\pgfqpoint{1.530647in}{1.734676in}}%
\pgfpathcurveto{\pgfqpoint{1.524824in}{1.728852in}}{\pgfqpoint{1.521551in}{1.720952in}}{\pgfqpoint{1.521551in}{1.712715in}}%
\pgfpathcurveto{\pgfqpoint{1.521551in}{1.704479in}}{\pgfqpoint{1.524824in}{1.696579in}}{\pgfqpoint{1.530647in}{1.690755in}}%
\pgfpathcurveto{\pgfqpoint{1.536471in}{1.684931in}}{\pgfqpoint{1.544371in}{1.681659in}}{\pgfqpoint{1.552608in}{1.681659in}}%
\pgfpathclose%
\pgfusepath{stroke,fill}%
\end{pgfscope}%
\begin{pgfscope}%
\pgfpathrectangle{\pgfqpoint{0.100000in}{0.212622in}}{\pgfqpoint{3.696000in}{3.696000in}}%
\pgfusepath{clip}%
\pgfsetbuttcap%
\pgfsetroundjoin%
\definecolor{currentfill}{rgb}{0.121569,0.466667,0.705882}%
\pgfsetfillcolor{currentfill}%
\pgfsetfillopacity{0.961201}%
\pgfsetlinewidth{1.003750pt}%
\definecolor{currentstroke}{rgb}{0.121569,0.466667,0.705882}%
\pgfsetstrokecolor{currentstroke}%
\pgfsetstrokeopacity{0.961201}%
\pgfsetdash{}{0pt}%
\pgfpathmoveto{\pgfqpoint{1.560850in}{1.676365in}}%
\pgfpathcurveto{\pgfqpoint{1.569086in}{1.676365in}}{\pgfqpoint{1.576986in}{1.679638in}}{\pgfqpoint{1.582810in}{1.685462in}}%
\pgfpathcurveto{\pgfqpoint{1.588634in}{1.691285in}}{\pgfqpoint{1.591906in}{1.699186in}}{\pgfqpoint{1.591906in}{1.707422in}}%
\pgfpathcurveto{\pgfqpoint{1.591906in}{1.715658in}}{\pgfqpoint{1.588634in}{1.723558in}}{\pgfqpoint{1.582810in}{1.729382in}}%
\pgfpathcurveto{\pgfqpoint{1.576986in}{1.735206in}}{\pgfqpoint{1.569086in}{1.738478in}}{\pgfqpoint{1.560850in}{1.738478in}}%
\pgfpathcurveto{\pgfqpoint{1.552614in}{1.738478in}}{\pgfqpoint{1.544713in}{1.735206in}}{\pgfqpoint{1.538890in}{1.729382in}}%
\pgfpathcurveto{\pgfqpoint{1.533066in}{1.723558in}}{\pgfqpoint{1.529793in}{1.715658in}}{\pgfqpoint{1.529793in}{1.707422in}}%
\pgfpathcurveto{\pgfqpoint{1.529793in}{1.699186in}}{\pgfqpoint{1.533066in}{1.691285in}}{\pgfqpoint{1.538890in}{1.685462in}}%
\pgfpathcurveto{\pgfqpoint{1.544713in}{1.679638in}}{\pgfqpoint{1.552614in}{1.676365in}}{\pgfqpoint{1.560850in}{1.676365in}}%
\pgfpathclose%
\pgfusepath{stroke,fill}%
\end{pgfscope}%
\begin{pgfscope}%
\pgfpathrectangle{\pgfqpoint{0.100000in}{0.212622in}}{\pgfqpoint{3.696000in}{3.696000in}}%
\pgfusepath{clip}%
\pgfsetbuttcap%
\pgfsetroundjoin%
\definecolor{currentfill}{rgb}{0.121569,0.466667,0.705882}%
\pgfsetfillcolor{currentfill}%
\pgfsetfillopacity{0.961812}%
\pgfsetlinewidth{1.003750pt}%
\definecolor{currentstroke}{rgb}{0.121569,0.466667,0.705882}%
\pgfsetstrokecolor{currentstroke}%
\pgfsetstrokeopacity{0.961812}%
\pgfsetdash{}{0pt}%
\pgfpathmoveto{\pgfqpoint{1.571211in}{1.669194in}}%
\pgfpathcurveto{\pgfqpoint{1.579447in}{1.669194in}}{\pgfqpoint{1.587347in}{1.672466in}}{\pgfqpoint{1.593171in}{1.678290in}}%
\pgfpathcurveto{\pgfqpoint{1.598995in}{1.684114in}}{\pgfqpoint{1.602268in}{1.692014in}}{\pgfqpoint{1.602268in}{1.700251in}}%
\pgfpathcurveto{\pgfqpoint{1.602268in}{1.708487in}}{\pgfqpoint{1.598995in}{1.716387in}}{\pgfqpoint{1.593171in}{1.722211in}}%
\pgfpathcurveto{\pgfqpoint{1.587347in}{1.728035in}}{\pgfqpoint{1.579447in}{1.731307in}}{\pgfqpoint{1.571211in}{1.731307in}}%
\pgfpathcurveto{\pgfqpoint{1.562975in}{1.731307in}}{\pgfqpoint{1.555075in}{1.728035in}}{\pgfqpoint{1.549251in}{1.722211in}}%
\pgfpathcurveto{\pgfqpoint{1.543427in}{1.716387in}}{\pgfqpoint{1.540155in}{1.708487in}}{\pgfqpoint{1.540155in}{1.700251in}}%
\pgfpathcurveto{\pgfqpoint{1.540155in}{1.692014in}}{\pgfqpoint{1.543427in}{1.684114in}}{\pgfqpoint{1.549251in}{1.678290in}}%
\pgfpathcurveto{\pgfqpoint{1.555075in}{1.672466in}}{\pgfqpoint{1.562975in}{1.669194in}}{\pgfqpoint{1.571211in}{1.669194in}}%
\pgfpathclose%
\pgfusepath{stroke,fill}%
\end{pgfscope}%
\begin{pgfscope}%
\pgfpathrectangle{\pgfqpoint{0.100000in}{0.212622in}}{\pgfqpoint{3.696000in}{3.696000in}}%
\pgfusepath{clip}%
\pgfsetbuttcap%
\pgfsetroundjoin%
\definecolor{currentfill}{rgb}{0.121569,0.466667,0.705882}%
\pgfsetfillcolor{currentfill}%
\pgfsetfillopacity{0.962366}%
\pgfsetlinewidth{1.003750pt}%
\definecolor{currentstroke}{rgb}{0.121569,0.466667,0.705882}%
\pgfsetstrokecolor{currentstroke}%
\pgfsetstrokeopacity{0.962366}%
\pgfsetdash{}{0pt}%
\pgfpathmoveto{\pgfqpoint{2.561070in}{1.169911in}}%
\pgfpathcurveto{\pgfqpoint{2.569306in}{1.169911in}}{\pgfqpoint{2.577206in}{1.173184in}}{\pgfqpoint{2.583030in}{1.179008in}}%
\pgfpathcurveto{\pgfqpoint{2.588854in}{1.184832in}}{\pgfqpoint{2.592126in}{1.192732in}}{\pgfqpoint{2.592126in}{1.200968in}}%
\pgfpathcurveto{\pgfqpoint{2.592126in}{1.209204in}}{\pgfqpoint{2.588854in}{1.217104in}}{\pgfqpoint{2.583030in}{1.222928in}}%
\pgfpathcurveto{\pgfqpoint{2.577206in}{1.228752in}}{\pgfqpoint{2.569306in}{1.232024in}}{\pgfqpoint{2.561070in}{1.232024in}}%
\pgfpathcurveto{\pgfqpoint{2.552833in}{1.232024in}}{\pgfqpoint{2.544933in}{1.228752in}}{\pgfqpoint{2.539109in}{1.222928in}}%
\pgfpathcurveto{\pgfqpoint{2.533285in}{1.217104in}}{\pgfqpoint{2.530013in}{1.209204in}}{\pgfqpoint{2.530013in}{1.200968in}}%
\pgfpathcurveto{\pgfqpoint{2.530013in}{1.192732in}}{\pgfqpoint{2.533285in}{1.184832in}}{\pgfqpoint{2.539109in}{1.179008in}}%
\pgfpathcurveto{\pgfqpoint{2.544933in}{1.173184in}}{\pgfqpoint{2.552833in}{1.169911in}}{\pgfqpoint{2.561070in}{1.169911in}}%
\pgfpathclose%
\pgfusepath{stroke,fill}%
\end{pgfscope}%
\begin{pgfscope}%
\pgfpathrectangle{\pgfqpoint{0.100000in}{0.212622in}}{\pgfqpoint{3.696000in}{3.696000in}}%
\pgfusepath{clip}%
\pgfsetbuttcap%
\pgfsetroundjoin%
\definecolor{currentfill}{rgb}{0.121569,0.466667,0.705882}%
\pgfsetfillcolor{currentfill}%
\pgfsetfillopacity{0.962499}%
\pgfsetlinewidth{1.003750pt}%
\definecolor{currentstroke}{rgb}{0.121569,0.466667,0.705882}%
\pgfsetstrokecolor{currentstroke}%
\pgfsetstrokeopacity{0.962499}%
\pgfsetdash{}{0pt}%
\pgfpathmoveto{\pgfqpoint{1.583224in}{1.660590in}}%
\pgfpathcurveto{\pgfqpoint{1.591460in}{1.660590in}}{\pgfqpoint{1.599360in}{1.663862in}}{\pgfqpoint{1.605184in}{1.669686in}}%
\pgfpathcurveto{\pgfqpoint{1.611008in}{1.675510in}}{\pgfqpoint{1.614280in}{1.683410in}}{\pgfqpoint{1.614280in}{1.691646in}}%
\pgfpathcurveto{\pgfqpoint{1.614280in}{1.699882in}}{\pgfqpoint{1.611008in}{1.707782in}}{\pgfqpoint{1.605184in}{1.713606in}}%
\pgfpathcurveto{\pgfqpoint{1.599360in}{1.719430in}}{\pgfqpoint{1.591460in}{1.722703in}}{\pgfqpoint{1.583224in}{1.722703in}}%
\pgfpathcurveto{\pgfqpoint{1.574988in}{1.722703in}}{\pgfqpoint{1.567088in}{1.719430in}}{\pgfqpoint{1.561264in}{1.713606in}}%
\pgfpathcurveto{\pgfqpoint{1.555440in}{1.707782in}}{\pgfqpoint{1.552167in}{1.699882in}}{\pgfqpoint{1.552167in}{1.691646in}}%
\pgfpathcurveto{\pgfqpoint{1.552167in}{1.683410in}}{\pgfqpoint{1.555440in}{1.675510in}}{\pgfqpoint{1.561264in}{1.669686in}}%
\pgfpathcurveto{\pgfqpoint{1.567088in}{1.663862in}}{\pgfqpoint{1.574988in}{1.660590in}}{\pgfqpoint{1.583224in}{1.660590in}}%
\pgfpathclose%
\pgfusepath{stroke,fill}%
\end{pgfscope}%
\begin{pgfscope}%
\pgfpathrectangle{\pgfqpoint{0.100000in}{0.212622in}}{\pgfqpoint{3.696000in}{3.696000in}}%
\pgfusepath{clip}%
\pgfsetbuttcap%
\pgfsetroundjoin%
\definecolor{currentfill}{rgb}{0.121569,0.466667,0.705882}%
\pgfsetfillcolor{currentfill}%
\pgfsetfillopacity{0.962862}%
\pgfsetlinewidth{1.003750pt}%
\definecolor{currentstroke}{rgb}{0.121569,0.466667,0.705882}%
\pgfsetstrokecolor{currentstroke}%
\pgfsetstrokeopacity{0.962862}%
\pgfsetdash{}{0pt}%
\pgfpathmoveto{\pgfqpoint{1.595885in}{1.649713in}}%
\pgfpathcurveto{\pgfqpoint{1.604121in}{1.649713in}}{\pgfqpoint{1.612021in}{1.652986in}}{\pgfqpoint{1.617845in}{1.658810in}}%
\pgfpathcurveto{\pgfqpoint{1.623669in}{1.664634in}}{\pgfqpoint{1.626941in}{1.672534in}}{\pgfqpoint{1.626941in}{1.680770in}}%
\pgfpathcurveto{\pgfqpoint{1.626941in}{1.689006in}}{\pgfqpoint{1.623669in}{1.696906in}}{\pgfqpoint{1.617845in}{1.702730in}}%
\pgfpathcurveto{\pgfqpoint{1.612021in}{1.708554in}}{\pgfqpoint{1.604121in}{1.711826in}}{\pgfqpoint{1.595885in}{1.711826in}}%
\pgfpathcurveto{\pgfqpoint{1.587649in}{1.711826in}}{\pgfqpoint{1.579749in}{1.708554in}}{\pgfqpoint{1.573925in}{1.702730in}}%
\pgfpathcurveto{\pgfqpoint{1.568101in}{1.696906in}}{\pgfqpoint{1.564828in}{1.689006in}}{\pgfqpoint{1.564828in}{1.680770in}}%
\pgfpathcurveto{\pgfqpoint{1.564828in}{1.672534in}}{\pgfqpoint{1.568101in}{1.664634in}}{\pgfqpoint{1.573925in}{1.658810in}}%
\pgfpathcurveto{\pgfqpoint{1.579749in}{1.652986in}}{\pgfqpoint{1.587649in}{1.649713in}}{\pgfqpoint{1.595885in}{1.649713in}}%
\pgfpathclose%
\pgfusepath{stroke,fill}%
\end{pgfscope}%
\begin{pgfscope}%
\pgfpathrectangle{\pgfqpoint{0.100000in}{0.212622in}}{\pgfqpoint{3.696000in}{3.696000in}}%
\pgfusepath{clip}%
\pgfsetbuttcap%
\pgfsetroundjoin%
\definecolor{currentfill}{rgb}{0.121569,0.466667,0.705882}%
\pgfsetfillcolor{currentfill}%
\pgfsetfillopacity{0.963250}%
\pgfsetlinewidth{1.003750pt}%
\definecolor{currentstroke}{rgb}{0.121569,0.466667,0.705882}%
\pgfsetstrokecolor{currentstroke}%
\pgfsetstrokeopacity{0.963250}%
\pgfsetdash{}{0pt}%
\pgfpathmoveto{\pgfqpoint{1.609894in}{1.638732in}}%
\pgfpathcurveto{\pgfqpoint{1.618130in}{1.638732in}}{\pgfqpoint{1.626030in}{1.642004in}}{\pgfqpoint{1.631854in}{1.647828in}}%
\pgfpathcurveto{\pgfqpoint{1.637678in}{1.653652in}}{\pgfqpoint{1.640951in}{1.661552in}}{\pgfqpoint{1.640951in}{1.669789in}}%
\pgfpathcurveto{\pgfqpoint{1.640951in}{1.678025in}}{\pgfqpoint{1.637678in}{1.685925in}}{\pgfqpoint{1.631854in}{1.691749in}}%
\pgfpathcurveto{\pgfqpoint{1.626030in}{1.697573in}}{\pgfqpoint{1.618130in}{1.700845in}}{\pgfqpoint{1.609894in}{1.700845in}}%
\pgfpathcurveto{\pgfqpoint{1.601658in}{1.700845in}}{\pgfqpoint{1.593758in}{1.697573in}}{\pgfqpoint{1.587934in}{1.691749in}}%
\pgfpathcurveto{\pgfqpoint{1.582110in}{1.685925in}}{\pgfqpoint{1.578838in}{1.678025in}}{\pgfqpoint{1.578838in}{1.669789in}}%
\pgfpathcurveto{\pgfqpoint{1.578838in}{1.661552in}}{\pgfqpoint{1.582110in}{1.653652in}}{\pgfqpoint{1.587934in}{1.647828in}}%
\pgfpathcurveto{\pgfqpoint{1.593758in}{1.642004in}}{\pgfqpoint{1.601658in}{1.638732in}}{\pgfqpoint{1.609894in}{1.638732in}}%
\pgfpathclose%
\pgfusepath{stroke,fill}%
\end{pgfscope}%
\begin{pgfscope}%
\pgfpathrectangle{\pgfqpoint{0.100000in}{0.212622in}}{\pgfqpoint{3.696000in}{3.696000in}}%
\pgfusepath{clip}%
\pgfsetbuttcap%
\pgfsetroundjoin%
\definecolor{currentfill}{rgb}{0.121569,0.466667,0.705882}%
\pgfsetfillcolor{currentfill}%
\pgfsetfillopacity{0.963624}%
\pgfsetlinewidth{1.003750pt}%
\definecolor{currentstroke}{rgb}{0.121569,0.466667,0.705882}%
\pgfsetstrokecolor{currentstroke}%
\pgfsetstrokeopacity{0.963624}%
\pgfsetdash{}{0pt}%
\pgfpathmoveto{\pgfqpoint{2.558716in}{1.166560in}}%
\pgfpathcurveto{\pgfqpoint{2.566952in}{1.166560in}}{\pgfqpoint{2.574852in}{1.169832in}}{\pgfqpoint{2.580676in}{1.175656in}}%
\pgfpathcurveto{\pgfqpoint{2.586500in}{1.181480in}}{\pgfqpoint{2.589772in}{1.189380in}}{\pgfqpoint{2.589772in}{1.197616in}}%
\pgfpathcurveto{\pgfqpoint{2.589772in}{1.205853in}}{\pgfqpoint{2.586500in}{1.213753in}}{\pgfqpoint{2.580676in}{1.219577in}}%
\pgfpathcurveto{\pgfqpoint{2.574852in}{1.225400in}}{\pgfqpoint{2.566952in}{1.228673in}}{\pgfqpoint{2.558716in}{1.228673in}}%
\pgfpathcurveto{\pgfqpoint{2.550480in}{1.228673in}}{\pgfqpoint{2.542580in}{1.225400in}}{\pgfqpoint{2.536756in}{1.219577in}}%
\pgfpathcurveto{\pgfqpoint{2.530932in}{1.213753in}}{\pgfqpoint{2.527659in}{1.205853in}}{\pgfqpoint{2.527659in}{1.197616in}}%
\pgfpathcurveto{\pgfqpoint{2.527659in}{1.189380in}}{\pgfqpoint{2.530932in}{1.181480in}}{\pgfqpoint{2.536756in}{1.175656in}}%
\pgfpathcurveto{\pgfqpoint{2.542580in}{1.169832in}}{\pgfqpoint{2.550480in}{1.166560in}}{\pgfqpoint{2.558716in}{1.166560in}}%
\pgfpathclose%
\pgfusepath{stroke,fill}%
\end{pgfscope}%
\begin{pgfscope}%
\pgfpathrectangle{\pgfqpoint{0.100000in}{0.212622in}}{\pgfqpoint{3.696000in}{3.696000in}}%
\pgfusepath{clip}%
\pgfsetbuttcap%
\pgfsetroundjoin%
\definecolor{currentfill}{rgb}{0.121569,0.466667,0.705882}%
\pgfsetfillcolor{currentfill}%
\pgfsetfillopacity{0.963765}%
\pgfsetlinewidth{1.003750pt}%
\definecolor{currentstroke}{rgb}{0.121569,0.466667,0.705882}%
\pgfsetstrokecolor{currentstroke}%
\pgfsetstrokeopacity{0.963765}%
\pgfsetdash{}{0pt}%
\pgfpathmoveto{\pgfqpoint{1.625996in}{1.626540in}}%
\pgfpathcurveto{\pgfqpoint{1.634232in}{1.626540in}}{\pgfqpoint{1.642132in}{1.629813in}}{\pgfqpoint{1.647956in}{1.635637in}}%
\pgfpathcurveto{\pgfqpoint{1.653780in}{1.641461in}}{\pgfqpoint{1.657052in}{1.649361in}}{\pgfqpoint{1.657052in}{1.657597in}}%
\pgfpathcurveto{\pgfqpoint{1.657052in}{1.665833in}}{\pgfqpoint{1.653780in}{1.673733in}}{\pgfqpoint{1.647956in}{1.679557in}}%
\pgfpathcurveto{\pgfqpoint{1.642132in}{1.685381in}}{\pgfqpoint{1.634232in}{1.688653in}}{\pgfqpoint{1.625996in}{1.688653in}}%
\pgfpathcurveto{\pgfqpoint{1.617759in}{1.688653in}}{\pgfqpoint{1.609859in}{1.685381in}}{\pgfqpoint{1.604035in}{1.679557in}}%
\pgfpathcurveto{\pgfqpoint{1.598211in}{1.673733in}}{\pgfqpoint{1.594939in}{1.665833in}}{\pgfqpoint{1.594939in}{1.657597in}}%
\pgfpathcurveto{\pgfqpoint{1.594939in}{1.649361in}}{\pgfqpoint{1.598211in}{1.641461in}}{\pgfqpoint{1.604035in}{1.635637in}}%
\pgfpathcurveto{\pgfqpoint{1.609859in}{1.629813in}}{\pgfqpoint{1.617759in}{1.626540in}}{\pgfqpoint{1.625996in}{1.626540in}}%
\pgfpathclose%
\pgfusepath{stroke,fill}%
\end{pgfscope}%
\begin{pgfscope}%
\pgfpathrectangle{\pgfqpoint{0.100000in}{0.212622in}}{\pgfqpoint{3.696000in}{3.696000in}}%
\pgfusepath{clip}%
\pgfsetbuttcap%
\pgfsetroundjoin%
\definecolor{currentfill}{rgb}{0.121569,0.466667,0.705882}%
\pgfsetfillcolor{currentfill}%
\pgfsetfillopacity{0.964463}%
\pgfsetlinewidth{1.003750pt}%
\definecolor{currentstroke}{rgb}{0.121569,0.466667,0.705882}%
\pgfsetstrokecolor{currentstroke}%
\pgfsetstrokeopacity{0.964463}%
\pgfsetdash{}{0pt}%
\pgfpathmoveto{\pgfqpoint{1.642637in}{1.613916in}}%
\pgfpathcurveto{\pgfqpoint{1.650874in}{1.613916in}}{\pgfqpoint{1.658774in}{1.617188in}}{\pgfqpoint{1.664598in}{1.623012in}}%
\pgfpathcurveto{\pgfqpoint{1.670421in}{1.628836in}}{\pgfqpoint{1.673694in}{1.636736in}}{\pgfqpoint{1.673694in}{1.644973in}}%
\pgfpathcurveto{\pgfqpoint{1.673694in}{1.653209in}}{\pgfqpoint{1.670421in}{1.661109in}}{\pgfqpoint{1.664598in}{1.666933in}}%
\pgfpathcurveto{\pgfqpoint{1.658774in}{1.672757in}}{\pgfqpoint{1.650874in}{1.676029in}}{\pgfqpoint{1.642637in}{1.676029in}}%
\pgfpathcurveto{\pgfqpoint{1.634401in}{1.676029in}}{\pgfqpoint{1.626501in}{1.672757in}}{\pgfqpoint{1.620677in}{1.666933in}}%
\pgfpathcurveto{\pgfqpoint{1.614853in}{1.661109in}}{\pgfqpoint{1.611581in}{1.653209in}}{\pgfqpoint{1.611581in}{1.644973in}}%
\pgfpathcurveto{\pgfqpoint{1.611581in}{1.636736in}}{\pgfqpoint{1.614853in}{1.628836in}}{\pgfqpoint{1.620677in}{1.623012in}}%
\pgfpathcurveto{\pgfqpoint{1.626501in}{1.617188in}}{\pgfqpoint{1.634401in}{1.613916in}}{\pgfqpoint{1.642637in}{1.613916in}}%
\pgfpathclose%
\pgfusepath{stroke,fill}%
\end{pgfscope}%
\begin{pgfscope}%
\pgfpathrectangle{\pgfqpoint{0.100000in}{0.212622in}}{\pgfqpoint{3.696000in}{3.696000in}}%
\pgfusepath{clip}%
\pgfsetbuttcap%
\pgfsetroundjoin%
\definecolor{currentfill}{rgb}{0.121569,0.466667,0.705882}%
\pgfsetfillcolor{currentfill}%
\pgfsetfillopacity{0.964649}%
\pgfsetlinewidth{1.003750pt}%
\definecolor{currentstroke}{rgb}{0.121569,0.466667,0.705882}%
\pgfsetstrokecolor{currentstroke}%
\pgfsetstrokeopacity{0.964649}%
\pgfsetdash{}{0pt}%
\pgfpathmoveto{\pgfqpoint{2.556821in}{1.163981in}}%
\pgfpathcurveto{\pgfqpoint{2.565058in}{1.163981in}}{\pgfqpoint{2.572958in}{1.167253in}}{\pgfqpoint{2.578782in}{1.173077in}}%
\pgfpathcurveto{\pgfqpoint{2.584606in}{1.178901in}}{\pgfqpoint{2.587878in}{1.186801in}}{\pgfqpoint{2.587878in}{1.195038in}}%
\pgfpathcurveto{\pgfqpoint{2.587878in}{1.203274in}}{\pgfqpoint{2.584606in}{1.211174in}}{\pgfqpoint{2.578782in}{1.216998in}}%
\pgfpathcurveto{\pgfqpoint{2.572958in}{1.222822in}}{\pgfqpoint{2.565058in}{1.226094in}}{\pgfqpoint{2.556821in}{1.226094in}}%
\pgfpathcurveto{\pgfqpoint{2.548585in}{1.226094in}}{\pgfqpoint{2.540685in}{1.222822in}}{\pgfqpoint{2.534861in}{1.216998in}}%
\pgfpathcurveto{\pgfqpoint{2.529037in}{1.211174in}}{\pgfqpoint{2.525765in}{1.203274in}}{\pgfqpoint{2.525765in}{1.195038in}}%
\pgfpathcurveto{\pgfqpoint{2.525765in}{1.186801in}}{\pgfqpoint{2.529037in}{1.178901in}}{\pgfqpoint{2.534861in}{1.173077in}}%
\pgfpathcurveto{\pgfqpoint{2.540685in}{1.167253in}}{\pgfqpoint{2.548585in}{1.163981in}}{\pgfqpoint{2.556821in}{1.163981in}}%
\pgfpathclose%
\pgfusepath{stroke,fill}%
\end{pgfscope}%
\begin{pgfscope}%
\pgfpathrectangle{\pgfqpoint{0.100000in}{0.212622in}}{\pgfqpoint{3.696000in}{3.696000in}}%
\pgfusepath{clip}%
\pgfsetbuttcap%
\pgfsetroundjoin%
\definecolor{currentfill}{rgb}{0.121569,0.466667,0.705882}%
\pgfsetfillcolor{currentfill}%
\pgfsetfillopacity{0.965137}%
\pgfsetlinewidth{1.003750pt}%
\definecolor{currentstroke}{rgb}{0.121569,0.466667,0.705882}%
\pgfsetstrokecolor{currentstroke}%
\pgfsetstrokeopacity{0.965137}%
\pgfsetdash{}{0pt}%
\pgfpathmoveto{\pgfqpoint{1.661977in}{1.599345in}}%
\pgfpathcurveto{\pgfqpoint{1.670213in}{1.599345in}}{\pgfqpoint{1.678113in}{1.602617in}}{\pgfqpoint{1.683937in}{1.608441in}}%
\pgfpathcurveto{\pgfqpoint{1.689761in}{1.614265in}}{\pgfqpoint{1.693034in}{1.622165in}}{\pgfqpoint{1.693034in}{1.630402in}}%
\pgfpathcurveto{\pgfqpoint{1.693034in}{1.638638in}}{\pgfqpoint{1.689761in}{1.646538in}}{\pgfqpoint{1.683937in}{1.652362in}}%
\pgfpathcurveto{\pgfqpoint{1.678113in}{1.658186in}}{\pgfqpoint{1.670213in}{1.661458in}}{\pgfqpoint{1.661977in}{1.661458in}}%
\pgfpathcurveto{\pgfqpoint{1.653741in}{1.661458in}}{\pgfqpoint{1.645841in}{1.658186in}}{\pgfqpoint{1.640017in}{1.652362in}}%
\pgfpathcurveto{\pgfqpoint{1.634193in}{1.646538in}}{\pgfqpoint{1.630921in}{1.638638in}}{\pgfqpoint{1.630921in}{1.630402in}}%
\pgfpathcurveto{\pgfqpoint{1.630921in}{1.622165in}}{\pgfqpoint{1.634193in}{1.614265in}}{\pgfqpoint{1.640017in}{1.608441in}}%
\pgfpathcurveto{\pgfqpoint{1.645841in}{1.602617in}}{\pgfqpoint{1.653741in}{1.599345in}}{\pgfqpoint{1.661977in}{1.599345in}}%
\pgfpathclose%
\pgfusepath{stroke,fill}%
\end{pgfscope}%
\begin{pgfscope}%
\pgfpathrectangle{\pgfqpoint{0.100000in}{0.212622in}}{\pgfqpoint{3.696000in}{3.696000in}}%
\pgfusepath{clip}%
\pgfsetbuttcap%
\pgfsetroundjoin%
\definecolor{currentfill}{rgb}{0.121569,0.466667,0.705882}%
\pgfsetfillcolor{currentfill}%
\pgfsetfillopacity{0.965700}%
\pgfsetlinewidth{1.003750pt}%
\definecolor{currentstroke}{rgb}{0.121569,0.466667,0.705882}%
\pgfsetstrokecolor{currentstroke}%
\pgfsetstrokeopacity{0.965700}%
\pgfsetdash{}{0pt}%
\pgfpathmoveto{\pgfqpoint{1.683009in}{1.582192in}}%
\pgfpathcurveto{\pgfqpoint{1.691245in}{1.582192in}}{\pgfqpoint{1.699145in}{1.585464in}}{\pgfqpoint{1.704969in}{1.591288in}}%
\pgfpathcurveto{\pgfqpoint{1.710793in}{1.597112in}}{\pgfqpoint{1.714066in}{1.605012in}}{\pgfqpoint{1.714066in}{1.613249in}}%
\pgfpathcurveto{\pgfqpoint{1.714066in}{1.621485in}}{\pgfqpoint{1.710793in}{1.629385in}}{\pgfqpoint{1.704969in}{1.635209in}}%
\pgfpathcurveto{\pgfqpoint{1.699145in}{1.641033in}}{\pgfqpoint{1.691245in}{1.644305in}}{\pgfqpoint{1.683009in}{1.644305in}}%
\pgfpathcurveto{\pgfqpoint{1.674773in}{1.644305in}}{\pgfqpoint{1.666873in}{1.641033in}}{\pgfqpoint{1.661049in}{1.635209in}}%
\pgfpathcurveto{\pgfqpoint{1.655225in}{1.629385in}}{\pgfqpoint{1.651953in}{1.621485in}}{\pgfqpoint{1.651953in}{1.613249in}}%
\pgfpathcurveto{\pgfqpoint{1.651953in}{1.605012in}}{\pgfqpoint{1.655225in}{1.597112in}}{\pgfqpoint{1.661049in}{1.591288in}}%
\pgfpathcurveto{\pgfqpoint{1.666873in}{1.585464in}}{\pgfqpoint{1.674773in}{1.582192in}}{\pgfqpoint{1.683009in}{1.582192in}}%
\pgfpathclose%
\pgfusepath{stroke,fill}%
\end{pgfscope}%
\begin{pgfscope}%
\pgfpathrectangle{\pgfqpoint{0.100000in}{0.212622in}}{\pgfqpoint{3.696000in}{3.696000in}}%
\pgfusepath{clip}%
\pgfsetbuttcap%
\pgfsetroundjoin%
\definecolor{currentfill}{rgb}{0.121569,0.466667,0.705882}%
\pgfsetfillcolor{currentfill}%
\pgfsetfillopacity{0.966408}%
\pgfsetlinewidth{1.003750pt}%
\definecolor{currentstroke}{rgb}{0.121569,0.466667,0.705882}%
\pgfsetstrokecolor{currentstroke}%
\pgfsetstrokeopacity{0.966408}%
\pgfsetdash{}{0pt}%
\pgfpathmoveto{\pgfqpoint{1.704860in}{1.565628in}}%
\pgfpathcurveto{\pgfqpoint{1.713096in}{1.565628in}}{\pgfqpoint{1.720996in}{1.568900in}}{\pgfqpoint{1.726820in}{1.574724in}}%
\pgfpathcurveto{\pgfqpoint{1.732644in}{1.580548in}}{\pgfqpoint{1.735916in}{1.588448in}}{\pgfqpoint{1.735916in}{1.596684in}}%
\pgfpathcurveto{\pgfqpoint{1.735916in}{1.604920in}}{\pgfqpoint{1.732644in}{1.612820in}}{\pgfqpoint{1.726820in}{1.618644in}}%
\pgfpathcurveto{\pgfqpoint{1.720996in}{1.624468in}}{\pgfqpoint{1.713096in}{1.627741in}}{\pgfqpoint{1.704860in}{1.627741in}}%
\pgfpathcurveto{\pgfqpoint{1.696623in}{1.627741in}}{\pgfqpoint{1.688723in}{1.624468in}}{\pgfqpoint{1.682899in}{1.618644in}}%
\pgfpathcurveto{\pgfqpoint{1.677075in}{1.612820in}}{\pgfqpoint{1.673803in}{1.604920in}}{\pgfqpoint{1.673803in}{1.596684in}}%
\pgfpathcurveto{\pgfqpoint{1.673803in}{1.588448in}}{\pgfqpoint{1.677075in}{1.580548in}}{\pgfqpoint{1.682899in}{1.574724in}}%
\pgfpathcurveto{\pgfqpoint{1.688723in}{1.568900in}}{\pgfqpoint{1.696623in}{1.565628in}}{\pgfqpoint{1.704860in}{1.565628in}}%
\pgfpathclose%
\pgfusepath{stroke,fill}%
\end{pgfscope}%
\begin{pgfscope}%
\pgfpathrectangle{\pgfqpoint{0.100000in}{0.212622in}}{\pgfqpoint{3.696000in}{3.696000in}}%
\pgfusepath{clip}%
\pgfsetbuttcap%
\pgfsetroundjoin%
\definecolor{currentfill}{rgb}{0.121569,0.466667,0.705882}%
\pgfsetfillcolor{currentfill}%
\pgfsetfillopacity{0.966574}%
\pgfsetlinewidth{1.003750pt}%
\definecolor{currentstroke}{rgb}{0.121569,0.466667,0.705882}%
\pgfsetstrokecolor{currentstroke}%
\pgfsetstrokeopacity{0.966574}%
\pgfsetdash{}{0pt}%
\pgfpathmoveto{\pgfqpoint{2.553307in}{1.159682in}}%
\pgfpathcurveto{\pgfqpoint{2.561544in}{1.159682in}}{\pgfqpoint{2.569444in}{1.162955in}}{\pgfqpoint{2.575268in}{1.168779in}}%
\pgfpathcurveto{\pgfqpoint{2.581092in}{1.174603in}}{\pgfqpoint{2.584364in}{1.182503in}}{\pgfqpoint{2.584364in}{1.190739in}}%
\pgfpathcurveto{\pgfqpoint{2.584364in}{1.198975in}}{\pgfqpoint{2.581092in}{1.206875in}}{\pgfqpoint{2.575268in}{1.212699in}}%
\pgfpathcurveto{\pgfqpoint{2.569444in}{1.218523in}}{\pgfqpoint{2.561544in}{1.221795in}}{\pgfqpoint{2.553307in}{1.221795in}}%
\pgfpathcurveto{\pgfqpoint{2.545071in}{1.221795in}}{\pgfqpoint{2.537171in}{1.218523in}}{\pgfqpoint{2.531347in}{1.212699in}}%
\pgfpathcurveto{\pgfqpoint{2.525523in}{1.206875in}}{\pgfqpoint{2.522251in}{1.198975in}}{\pgfqpoint{2.522251in}{1.190739in}}%
\pgfpathcurveto{\pgfqpoint{2.522251in}{1.182503in}}{\pgfqpoint{2.525523in}{1.174603in}}{\pgfqpoint{2.531347in}{1.168779in}}%
\pgfpathcurveto{\pgfqpoint{2.537171in}{1.162955in}}{\pgfqpoint{2.545071in}{1.159682in}}{\pgfqpoint{2.553307in}{1.159682in}}%
\pgfpathclose%
\pgfusepath{stroke,fill}%
\end{pgfscope}%
\begin{pgfscope}%
\pgfpathrectangle{\pgfqpoint{0.100000in}{0.212622in}}{\pgfqpoint{3.696000in}{3.696000in}}%
\pgfusepath{clip}%
\pgfsetbuttcap%
\pgfsetroundjoin%
\definecolor{currentfill}{rgb}{0.121569,0.466667,0.705882}%
\pgfsetfillcolor{currentfill}%
\pgfsetfillopacity{0.966806}%
\pgfsetlinewidth{1.003750pt}%
\definecolor{currentstroke}{rgb}{0.121569,0.466667,0.705882}%
\pgfsetstrokecolor{currentstroke}%
\pgfsetstrokeopacity{0.966806}%
\pgfsetdash{}{0pt}%
\pgfpathmoveto{\pgfqpoint{1.716962in}{1.556886in}}%
\pgfpathcurveto{\pgfqpoint{1.725199in}{1.556886in}}{\pgfqpoint{1.733099in}{1.560158in}}{\pgfqpoint{1.738923in}{1.565982in}}%
\pgfpathcurveto{\pgfqpoint{1.744747in}{1.571806in}}{\pgfqpoint{1.748019in}{1.579706in}}{\pgfqpoint{1.748019in}{1.587942in}}%
\pgfpathcurveto{\pgfqpoint{1.748019in}{1.596178in}}{\pgfqpoint{1.744747in}{1.604078in}}{\pgfqpoint{1.738923in}{1.609902in}}%
\pgfpathcurveto{\pgfqpoint{1.733099in}{1.615726in}}{\pgfqpoint{1.725199in}{1.618999in}}{\pgfqpoint{1.716962in}{1.618999in}}%
\pgfpathcurveto{\pgfqpoint{1.708726in}{1.618999in}}{\pgfqpoint{1.700826in}{1.615726in}}{\pgfqpoint{1.695002in}{1.609902in}}%
\pgfpathcurveto{\pgfqpoint{1.689178in}{1.604078in}}{\pgfqpoint{1.685906in}{1.596178in}}{\pgfqpoint{1.685906in}{1.587942in}}%
\pgfpathcurveto{\pgfqpoint{1.685906in}{1.579706in}}{\pgfqpoint{1.689178in}{1.571806in}}{\pgfqpoint{1.695002in}{1.565982in}}%
\pgfpathcurveto{\pgfqpoint{1.700826in}{1.560158in}}{\pgfqpoint{1.708726in}{1.556886in}}{\pgfqpoint{1.716962in}{1.556886in}}%
\pgfpathclose%
\pgfusepath{stroke,fill}%
\end{pgfscope}%
\begin{pgfscope}%
\pgfpathrectangle{\pgfqpoint{0.100000in}{0.212622in}}{\pgfqpoint{3.696000in}{3.696000in}}%
\pgfusepath{clip}%
\pgfsetbuttcap%
\pgfsetroundjoin%
\definecolor{currentfill}{rgb}{0.121569,0.466667,0.705882}%
\pgfsetfillcolor{currentfill}%
\pgfsetfillopacity{0.967028}%
\pgfsetlinewidth{1.003750pt}%
\definecolor{currentstroke}{rgb}{0.121569,0.466667,0.705882}%
\pgfsetstrokecolor{currentstroke}%
\pgfsetstrokeopacity{0.967028}%
\pgfsetdash{}{0pt}%
\pgfpathmoveto{\pgfqpoint{1.723732in}{1.552535in}}%
\pgfpathcurveto{\pgfqpoint{1.731968in}{1.552535in}}{\pgfqpoint{1.739868in}{1.555807in}}{\pgfqpoint{1.745692in}{1.561631in}}%
\pgfpathcurveto{\pgfqpoint{1.751516in}{1.567455in}}{\pgfqpoint{1.754788in}{1.575355in}}{\pgfqpoint{1.754788in}{1.583591in}}%
\pgfpathcurveto{\pgfqpoint{1.754788in}{1.591827in}}{\pgfqpoint{1.751516in}{1.599727in}}{\pgfqpoint{1.745692in}{1.605551in}}%
\pgfpathcurveto{\pgfqpoint{1.739868in}{1.611375in}}{\pgfqpoint{1.731968in}{1.614648in}}{\pgfqpoint{1.723732in}{1.614648in}}%
\pgfpathcurveto{\pgfqpoint{1.715496in}{1.614648in}}{\pgfqpoint{1.707596in}{1.611375in}}{\pgfqpoint{1.701772in}{1.605551in}}%
\pgfpathcurveto{\pgfqpoint{1.695948in}{1.599727in}}{\pgfqpoint{1.692675in}{1.591827in}}{\pgfqpoint{1.692675in}{1.583591in}}%
\pgfpathcurveto{\pgfqpoint{1.692675in}{1.575355in}}{\pgfqpoint{1.695948in}{1.567455in}}{\pgfqpoint{1.701772in}{1.561631in}}%
\pgfpathcurveto{\pgfqpoint{1.707596in}{1.555807in}}{\pgfqpoint{1.715496in}{1.552535in}}{\pgfqpoint{1.723732in}{1.552535in}}%
\pgfpathclose%
\pgfusepath{stroke,fill}%
\end{pgfscope}%
\begin{pgfscope}%
\pgfpathrectangle{\pgfqpoint{0.100000in}{0.212622in}}{\pgfqpoint{3.696000in}{3.696000in}}%
\pgfusepath{clip}%
\pgfsetbuttcap%
\pgfsetroundjoin%
\definecolor{currentfill}{rgb}{0.121569,0.466667,0.705882}%
\pgfsetfillcolor{currentfill}%
\pgfsetfillopacity{0.967298}%
\pgfsetlinewidth{1.003750pt}%
\definecolor{currentstroke}{rgb}{0.121569,0.466667,0.705882}%
\pgfsetstrokecolor{currentstroke}%
\pgfsetstrokeopacity{0.967298}%
\pgfsetdash{}{0pt}%
\pgfpathmoveto{\pgfqpoint{1.733823in}{1.545575in}}%
\pgfpathcurveto{\pgfqpoint{1.742060in}{1.545575in}}{\pgfqpoint{1.749960in}{1.548847in}}{\pgfqpoint{1.755784in}{1.554671in}}%
\pgfpathcurveto{\pgfqpoint{1.761607in}{1.560495in}}{\pgfqpoint{1.764880in}{1.568395in}}{\pgfqpoint{1.764880in}{1.576631in}}%
\pgfpathcurveto{\pgfqpoint{1.764880in}{1.584868in}}{\pgfqpoint{1.761607in}{1.592768in}}{\pgfqpoint{1.755784in}{1.598592in}}%
\pgfpathcurveto{\pgfqpoint{1.749960in}{1.604416in}}{\pgfqpoint{1.742060in}{1.607688in}}{\pgfqpoint{1.733823in}{1.607688in}}%
\pgfpathcurveto{\pgfqpoint{1.725587in}{1.607688in}}{\pgfqpoint{1.717687in}{1.604416in}}{\pgfqpoint{1.711863in}{1.598592in}}%
\pgfpathcurveto{\pgfqpoint{1.706039in}{1.592768in}}{\pgfqpoint{1.702767in}{1.584868in}}{\pgfqpoint{1.702767in}{1.576631in}}%
\pgfpathcurveto{\pgfqpoint{1.702767in}{1.568395in}}{\pgfqpoint{1.706039in}{1.560495in}}{\pgfqpoint{1.711863in}{1.554671in}}%
\pgfpathcurveto{\pgfqpoint{1.717687in}{1.548847in}}{\pgfqpoint{1.725587in}{1.545575in}}{\pgfqpoint{1.733823in}{1.545575in}}%
\pgfpathclose%
\pgfusepath{stroke,fill}%
\end{pgfscope}%
\begin{pgfscope}%
\pgfpathrectangle{\pgfqpoint{0.100000in}{0.212622in}}{\pgfqpoint{3.696000in}{3.696000in}}%
\pgfusepath{clip}%
\pgfsetbuttcap%
\pgfsetroundjoin%
\definecolor{currentfill}{rgb}{0.121569,0.466667,0.705882}%
\pgfsetfillcolor{currentfill}%
\pgfsetfillopacity{0.967607}%
\pgfsetlinewidth{1.003750pt}%
\definecolor{currentstroke}{rgb}{0.121569,0.466667,0.705882}%
\pgfsetstrokecolor{currentstroke}%
\pgfsetstrokeopacity{0.967607}%
\pgfsetdash{}{0pt}%
\pgfpathmoveto{\pgfqpoint{1.746336in}{1.536101in}}%
\pgfpathcurveto{\pgfqpoint{1.754572in}{1.536101in}}{\pgfqpoint{1.762472in}{1.539374in}}{\pgfqpoint{1.768296in}{1.545197in}}%
\pgfpathcurveto{\pgfqpoint{1.774120in}{1.551021in}}{\pgfqpoint{1.777392in}{1.558921in}}{\pgfqpoint{1.777392in}{1.567158in}}%
\pgfpathcurveto{\pgfqpoint{1.777392in}{1.575394in}}{\pgfqpoint{1.774120in}{1.583294in}}{\pgfqpoint{1.768296in}{1.589118in}}%
\pgfpathcurveto{\pgfqpoint{1.762472in}{1.594942in}}{\pgfqpoint{1.754572in}{1.598214in}}{\pgfqpoint{1.746336in}{1.598214in}}%
\pgfpathcurveto{\pgfqpoint{1.738099in}{1.598214in}}{\pgfqpoint{1.730199in}{1.594942in}}{\pgfqpoint{1.724375in}{1.589118in}}%
\pgfpathcurveto{\pgfqpoint{1.718551in}{1.583294in}}{\pgfqpoint{1.715279in}{1.575394in}}{\pgfqpoint{1.715279in}{1.567158in}}%
\pgfpathcurveto{\pgfqpoint{1.715279in}{1.558921in}}{\pgfqpoint{1.718551in}{1.551021in}}{\pgfqpoint{1.724375in}{1.545197in}}%
\pgfpathcurveto{\pgfqpoint{1.730199in}{1.539374in}}{\pgfqpoint{1.738099in}{1.536101in}}{\pgfqpoint{1.746336in}{1.536101in}}%
\pgfpathclose%
\pgfusepath{stroke,fill}%
\end{pgfscope}%
\begin{pgfscope}%
\pgfpathrectangle{\pgfqpoint{0.100000in}{0.212622in}}{\pgfqpoint{3.696000in}{3.696000in}}%
\pgfusepath{clip}%
\pgfsetbuttcap%
\pgfsetroundjoin%
\definecolor{currentfill}{rgb}{0.121569,0.466667,0.705882}%
\pgfsetfillcolor{currentfill}%
\pgfsetfillopacity{0.967967}%
\pgfsetlinewidth{1.003750pt}%
\definecolor{currentstroke}{rgb}{0.121569,0.466667,0.705882}%
\pgfsetstrokecolor{currentstroke}%
\pgfsetstrokeopacity{0.967967}%
\pgfsetdash{}{0pt}%
\pgfpathmoveto{\pgfqpoint{2.550884in}{1.156800in}}%
\pgfpathcurveto{\pgfqpoint{2.559120in}{1.156800in}}{\pgfqpoint{2.567020in}{1.160073in}}{\pgfqpoint{2.572844in}{1.165897in}}%
\pgfpathcurveto{\pgfqpoint{2.578668in}{1.171720in}}{\pgfqpoint{2.581941in}{1.179621in}}{\pgfqpoint{2.581941in}{1.187857in}}%
\pgfpathcurveto{\pgfqpoint{2.581941in}{1.196093in}}{\pgfqpoint{2.578668in}{1.203993in}}{\pgfqpoint{2.572844in}{1.209817in}}%
\pgfpathcurveto{\pgfqpoint{2.567020in}{1.215641in}}{\pgfqpoint{2.559120in}{1.218913in}}{\pgfqpoint{2.550884in}{1.218913in}}%
\pgfpathcurveto{\pgfqpoint{2.542648in}{1.218913in}}{\pgfqpoint{2.534748in}{1.215641in}}{\pgfqpoint{2.528924in}{1.209817in}}%
\pgfpathcurveto{\pgfqpoint{2.523100in}{1.203993in}}{\pgfqpoint{2.519828in}{1.196093in}}{\pgfqpoint{2.519828in}{1.187857in}}%
\pgfpathcurveto{\pgfqpoint{2.519828in}{1.179621in}}{\pgfqpoint{2.523100in}{1.171720in}}{\pgfqpoint{2.528924in}{1.165897in}}%
\pgfpathcurveto{\pgfqpoint{2.534748in}{1.160073in}}{\pgfqpoint{2.542648in}{1.156800in}}{\pgfqpoint{2.550884in}{1.156800in}}%
\pgfpathclose%
\pgfusepath{stroke,fill}%
\end{pgfscope}%
\begin{pgfscope}%
\pgfpathrectangle{\pgfqpoint{0.100000in}{0.212622in}}{\pgfqpoint{3.696000in}{3.696000in}}%
\pgfusepath{clip}%
\pgfsetbuttcap%
\pgfsetroundjoin%
\definecolor{currentfill}{rgb}{0.121569,0.466667,0.705882}%
\pgfsetfillcolor{currentfill}%
\pgfsetfillopacity{0.968123}%
\pgfsetlinewidth{1.003750pt}%
\definecolor{currentstroke}{rgb}{0.121569,0.466667,0.705882}%
\pgfsetstrokecolor{currentstroke}%
\pgfsetstrokeopacity{0.968123}%
\pgfsetdash{}{0pt}%
\pgfpathmoveto{\pgfqpoint{1.759569in}{1.526992in}}%
\pgfpathcurveto{\pgfqpoint{1.767805in}{1.526992in}}{\pgfqpoint{1.775705in}{1.530264in}}{\pgfqpoint{1.781529in}{1.536088in}}%
\pgfpathcurveto{\pgfqpoint{1.787353in}{1.541912in}}{\pgfqpoint{1.790625in}{1.549812in}}{\pgfqpoint{1.790625in}{1.558048in}}%
\pgfpathcurveto{\pgfqpoint{1.790625in}{1.566284in}}{\pgfqpoint{1.787353in}{1.574184in}}{\pgfqpoint{1.781529in}{1.580008in}}%
\pgfpathcurveto{\pgfqpoint{1.775705in}{1.585832in}}{\pgfqpoint{1.767805in}{1.589105in}}{\pgfqpoint{1.759569in}{1.589105in}}%
\pgfpathcurveto{\pgfqpoint{1.751332in}{1.589105in}}{\pgfqpoint{1.743432in}{1.585832in}}{\pgfqpoint{1.737608in}{1.580008in}}%
\pgfpathcurveto{\pgfqpoint{1.731784in}{1.574184in}}{\pgfqpoint{1.728512in}{1.566284in}}{\pgfqpoint{1.728512in}{1.558048in}}%
\pgfpathcurveto{\pgfqpoint{1.728512in}{1.549812in}}{\pgfqpoint{1.731784in}{1.541912in}}{\pgfqpoint{1.737608in}{1.536088in}}%
\pgfpathcurveto{\pgfqpoint{1.743432in}{1.530264in}}{\pgfqpoint{1.751332in}{1.526992in}}{\pgfqpoint{1.759569in}{1.526992in}}%
\pgfpathclose%
\pgfusepath{stroke,fill}%
\end{pgfscope}%
\begin{pgfscope}%
\pgfpathrectangle{\pgfqpoint{0.100000in}{0.212622in}}{\pgfqpoint{3.696000in}{3.696000in}}%
\pgfusepath{clip}%
\pgfsetbuttcap%
\pgfsetroundjoin%
\definecolor{currentfill}{rgb}{0.121569,0.466667,0.705882}%
\pgfsetfillcolor{currentfill}%
\pgfsetfillopacity{0.968640}%
\pgfsetlinewidth{1.003750pt}%
\definecolor{currentstroke}{rgb}{0.121569,0.466667,0.705882}%
\pgfsetstrokecolor{currentstroke}%
\pgfsetstrokeopacity{0.968640}%
\pgfsetdash{}{0pt}%
\pgfpathmoveto{\pgfqpoint{2.549765in}{1.155439in}}%
\pgfpathcurveto{\pgfqpoint{2.558002in}{1.155439in}}{\pgfqpoint{2.565902in}{1.158712in}}{\pgfqpoint{2.571726in}{1.164536in}}%
\pgfpathcurveto{\pgfqpoint{2.577550in}{1.170360in}}{\pgfqpoint{2.580822in}{1.178260in}}{\pgfqpoint{2.580822in}{1.186496in}}%
\pgfpathcurveto{\pgfqpoint{2.580822in}{1.194732in}}{\pgfqpoint{2.577550in}{1.202632in}}{\pgfqpoint{2.571726in}{1.208456in}}%
\pgfpathcurveto{\pgfqpoint{2.565902in}{1.214280in}}{\pgfqpoint{2.558002in}{1.217552in}}{\pgfqpoint{2.549765in}{1.217552in}}%
\pgfpathcurveto{\pgfqpoint{2.541529in}{1.217552in}}{\pgfqpoint{2.533629in}{1.214280in}}{\pgfqpoint{2.527805in}{1.208456in}}%
\pgfpathcurveto{\pgfqpoint{2.521981in}{1.202632in}}{\pgfqpoint{2.518709in}{1.194732in}}{\pgfqpoint{2.518709in}{1.186496in}}%
\pgfpathcurveto{\pgfqpoint{2.518709in}{1.178260in}}{\pgfqpoint{2.521981in}{1.170360in}}{\pgfqpoint{2.527805in}{1.164536in}}%
\pgfpathcurveto{\pgfqpoint{2.533629in}{1.158712in}}{\pgfqpoint{2.541529in}{1.155439in}}{\pgfqpoint{2.549765in}{1.155439in}}%
\pgfpathclose%
\pgfusepath{stroke,fill}%
\end{pgfscope}%
\begin{pgfscope}%
\pgfpathrectangle{\pgfqpoint{0.100000in}{0.212622in}}{\pgfqpoint{3.696000in}{3.696000in}}%
\pgfusepath{clip}%
\pgfsetbuttcap%
\pgfsetroundjoin%
\definecolor{currentfill}{rgb}{0.121569,0.466667,0.705882}%
\pgfsetfillcolor{currentfill}%
\pgfsetfillopacity{0.968704}%
\pgfsetlinewidth{1.003750pt}%
\definecolor{currentstroke}{rgb}{0.121569,0.466667,0.705882}%
\pgfsetstrokecolor{currentstroke}%
\pgfsetstrokeopacity{0.968704}%
\pgfsetdash{}{0pt}%
\pgfpathmoveto{\pgfqpoint{1.774338in}{1.517254in}}%
\pgfpathcurveto{\pgfqpoint{1.782574in}{1.517254in}}{\pgfqpoint{1.790474in}{1.520527in}}{\pgfqpoint{1.796298in}{1.526351in}}%
\pgfpathcurveto{\pgfqpoint{1.802122in}{1.532175in}}{\pgfqpoint{1.805394in}{1.540075in}}{\pgfqpoint{1.805394in}{1.548311in}}%
\pgfpathcurveto{\pgfqpoint{1.805394in}{1.556547in}}{\pgfqpoint{1.802122in}{1.564447in}}{\pgfqpoint{1.796298in}{1.570271in}}%
\pgfpathcurveto{\pgfqpoint{1.790474in}{1.576095in}}{\pgfqpoint{1.782574in}{1.579367in}}{\pgfqpoint{1.774338in}{1.579367in}}%
\pgfpathcurveto{\pgfqpoint{1.766101in}{1.579367in}}{\pgfqpoint{1.758201in}{1.576095in}}{\pgfqpoint{1.752377in}{1.570271in}}%
\pgfpathcurveto{\pgfqpoint{1.746554in}{1.564447in}}{\pgfqpoint{1.743281in}{1.556547in}}{\pgfqpoint{1.743281in}{1.548311in}}%
\pgfpathcurveto{\pgfqpoint{1.743281in}{1.540075in}}{\pgfqpoint{1.746554in}{1.532175in}}{\pgfqpoint{1.752377in}{1.526351in}}%
\pgfpathcurveto{\pgfqpoint{1.758201in}{1.520527in}}{\pgfqpoint{1.766101in}{1.517254in}}{\pgfqpoint{1.774338in}{1.517254in}}%
\pgfpathclose%
\pgfusepath{stroke,fill}%
\end{pgfscope}%
\begin{pgfscope}%
\pgfpathrectangle{\pgfqpoint{0.100000in}{0.212622in}}{\pgfqpoint{3.696000in}{3.696000in}}%
\pgfusepath{clip}%
\pgfsetbuttcap%
\pgfsetroundjoin%
\definecolor{currentfill}{rgb}{0.121569,0.466667,0.705882}%
\pgfsetfillcolor{currentfill}%
\pgfsetfillopacity{0.969338}%
\pgfsetlinewidth{1.003750pt}%
\definecolor{currentstroke}{rgb}{0.121569,0.466667,0.705882}%
\pgfsetstrokecolor{currentstroke}%
\pgfsetstrokeopacity{0.969338}%
\pgfsetdash{}{0pt}%
\pgfpathmoveto{\pgfqpoint{1.790729in}{1.507263in}}%
\pgfpathcurveto{\pgfqpoint{1.798965in}{1.507263in}}{\pgfqpoint{1.806865in}{1.510535in}}{\pgfqpoint{1.812689in}{1.516359in}}%
\pgfpathcurveto{\pgfqpoint{1.818513in}{1.522183in}}{\pgfqpoint{1.821785in}{1.530083in}}{\pgfqpoint{1.821785in}{1.538319in}}%
\pgfpathcurveto{\pgfqpoint{1.821785in}{1.546556in}}{\pgfqpoint{1.818513in}{1.554456in}}{\pgfqpoint{1.812689in}{1.560279in}}%
\pgfpathcurveto{\pgfqpoint{1.806865in}{1.566103in}}{\pgfqpoint{1.798965in}{1.569376in}}{\pgfqpoint{1.790729in}{1.569376in}}%
\pgfpathcurveto{\pgfqpoint{1.782493in}{1.569376in}}{\pgfqpoint{1.774593in}{1.566103in}}{\pgfqpoint{1.768769in}{1.560279in}}%
\pgfpathcurveto{\pgfqpoint{1.762945in}{1.554456in}}{\pgfqpoint{1.759672in}{1.546556in}}{\pgfqpoint{1.759672in}{1.538319in}}%
\pgfpathcurveto{\pgfqpoint{1.759672in}{1.530083in}}{\pgfqpoint{1.762945in}{1.522183in}}{\pgfqpoint{1.768769in}{1.516359in}}%
\pgfpathcurveto{\pgfqpoint{1.774593in}{1.510535in}}{\pgfqpoint{1.782493in}{1.507263in}}{\pgfqpoint{1.790729in}{1.507263in}}%
\pgfpathclose%
\pgfusepath{stroke,fill}%
\end{pgfscope}%
\begin{pgfscope}%
\pgfpathrectangle{\pgfqpoint{0.100000in}{0.212622in}}{\pgfqpoint{3.696000in}{3.696000in}}%
\pgfusepath{clip}%
\pgfsetbuttcap%
\pgfsetroundjoin%
\definecolor{currentfill}{rgb}{0.121569,0.466667,0.705882}%
\pgfsetfillcolor{currentfill}%
\pgfsetfillopacity{0.969804}%
\pgfsetlinewidth{1.003750pt}%
\definecolor{currentstroke}{rgb}{0.121569,0.466667,0.705882}%
\pgfsetstrokecolor{currentstroke}%
\pgfsetstrokeopacity{0.969804}%
\pgfsetdash{}{0pt}%
\pgfpathmoveto{\pgfqpoint{2.547772in}{1.152610in}}%
\pgfpathcurveto{\pgfqpoint{2.556008in}{1.152610in}}{\pgfqpoint{2.563908in}{1.155882in}}{\pgfqpoint{2.569732in}{1.161706in}}%
\pgfpathcurveto{\pgfqpoint{2.575556in}{1.167530in}}{\pgfqpoint{2.578828in}{1.175430in}}{\pgfqpoint{2.578828in}{1.183666in}}%
\pgfpathcurveto{\pgfqpoint{2.578828in}{1.191903in}}{\pgfqpoint{2.575556in}{1.199803in}}{\pgfqpoint{2.569732in}{1.205627in}}%
\pgfpathcurveto{\pgfqpoint{2.563908in}{1.211451in}}{\pgfqpoint{2.556008in}{1.214723in}}{\pgfqpoint{2.547772in}{1.214723in}}%
\pgfpathcurveto{\pgfqpoint{2.539535in}{1.214723in}}{\pgfqpoint{2.531635in}{1.211451in}}{\pgfqpoint{2.525811in}{1.205627in}}%
\pgfpathcurveto{\pgfqpoint{2.519987in}{1.199803in}}{\pgfqpoint{2.516715in}{1.191903in}}{\pgfqpoint{2.516715in}{1.183666in}}%
\pgfpathcurveto{\pgfqpoint{2.516715in}{1.175430in}}{\pgfqpoint{2.519987in}{1.167530in}}{\pgfqpoint{2.525811in}{1.161706in}}%
\pgfpathcurveto{\pgfqpoint{2.531635in}{1.155882in}}{\pgfqpoint{2.539535in}{1.152610in}}{\pgfqpoint{2.547772in}{1.152610in}}%
\pgfpathclose%
\pgfusepath{stroke,fill}%
\end{pgfscope}%
\begin{pgfscope}%
\pgfpathrectangle{\pgfqpoint{0.100000in}{0.212622in}}{\pgfqpoint{3.696000in}{3.696000in}}%
\pgfusepath{clip}%
\pgfsetbuttcap%
\pgfsetroundjoin%
\definecolor{currentfill}{rgb}{0.121569,0.466667,0.705882}%
\pgfsetfillcolor{currentfill}%
\pgfsetfillopacity{0.970516}%
\pgfsetlinewidth{1.003750pt}%
\definecolor{currentstroke}{rgb}{0.121569,0.466667,0.705882}%
\pgfsetstrokecolor{currentstroke}%
\pgfsetstrokeopacity{0.970516}%
\pgfsetdash{}{0pt}%
\pgfpathmoveto{\pgfqpoint{1.807687in}{1.499648in}}%
\pgfpathcurveto{\pgfqpoint{1.815923in}{1.499648in}}{\pgfqpoint{1.823823in}{1.502920in}}{\pgfqpoint{1.829647in}{1.508744in}}%
\pgfpathcurveto{\pgfqpoint{1.835471in}{1.514568in}}{\pgfqpoint{1.838744in}{1.522468in}}{\pgfqpoint{1.838744in}{1.530704in}}%
\pgfpathcurveto{\pgfqpoint{1.838744in}{1.538940in}}{\pgfqpoint{1.835471in}{1.546840in}}{\pgfqpoint{1.829647in}{1.552664in}}%
\pgfpathcurveto{\pgfqpoint{1.823823in}{1.558488in}}{\pgfqpoint{1.815923in}{1.561761in}}{\pgfqpoint{1.807687in}{1.561761in}}%
\pgfpathcurveto{\pgfqpoint{1.799451in}{1.561761in}}{\pgfqpoint{1.791551in}{1.558488in}}{\pgfqpoint{1.785727in}{1.552664in}}%
\pgfpathcurveto{\pgfqpoint{1.779903in}{1.546840in}}{\pgfqpoint{1.776631in}{1.538940in}}{\pgfqpoint{1.776631in}{1.530704in}}%
\pgfpathcurveto{\pgfqpoint{1.776631in}{1.522468in}}{\pgfqpoint{1.779903in}{1.514568in}}{\pgfqpoint{1.785727in}{1.508744in}}%
\pgfpathcurveto{\pgfqpoint{1.791551in}{1.502920in}}{\pgfqpoint{1.799451in}{1.499648in}}{\pgfqpoint{1.807687in}{1.499648in}}%
\pgfpathclose%
\pgfusepath{stroke,fill}%
\end{pgfscope}%
\begin{pgfscope}%
\pgfpathrectangle{\pgfqpoint{0.100000in}{0.212622in}}{\pgfqpoint{3.696000in}{3.696000in}}%
\pgfusepath{clip}%
\pgfsetbuttcap%
\pgfsetroundjoin%
\definecolor{currentfill}{rgb}{0.121569,0.466667,0.705882}%
\pgfsetfillcolor{currentfill}%
\pgfsetfillopacity{0.970598}%
\pgfsetlinewidth{1.003750pt}%
\definecolor{currentstroke}{rgb}{0.121569,0.466667,0.705882}%
\pgfsetstrokecolor{currentstroke}%
\pgfsetstrokeopacity{0.970598}%
\pgfsetdash{}{0pt}%
\pgfpathmoveto{\pgfqpoint{2.546423in}{1.150579in}}%
\pgfpathcurveto{\pgfqpoint{2.554659in}{1.150579in}}{\pgfqpoint{2.562559in}{1.153851in}}{\pgfqpoint{2.568383in}{1.159675in}}%
\pgfpathcurveto{\pgfqpoint{2.574207in}{1.165499in}}{\pgfqpoint{2.577479in}{1.173399in}}{\pgfqpoint{2.577479in}{1.181635in}}%
\pgfpathcurveto{\pgfqpoint{2.577479in}{1.189871in}}{\pgfqpoint{2.574207in}{1.197771in}}{\pgfqpoint{2.568383in}{1.203595in}}%
\pgfpathcurveto{\pgfqpoint{2.562559in}{1.209419in}}{\pgfqpoint{2.554659in}{1.212692in}}{\pgfqpoint{2.546423in}{1.212692in}}%
\pgfpathcurveto{\pgfqpoint{2.538187in}{1.212692in}}{\pgfqpoint{2.530287in}{1.209419in}}{\pgfqpoint{2.524463in}{1.203595in}}%
\pgfpathcurveto{\pgfqpoint{2.518639in}{1.197771in}}{\pgfqpoint{2.515366in}{1.189871in}}{\pgfqpoint{2.515366in}{1.181635in}}%
\pgfpathcurveto{\pgfqpoint{2.515366in}{1.173399in}}{\pgfqpoint{2.518639in}{1.165499in}}{\pgfqpoint{2.524463in}{1.159675in}}%
\pgfpathcurveto{\pgfqpoint{2.530287in}{1.153851in}}{\pgfqpoint{2.538187in}{1.150579in}}{\pgfqpoint{2.546423in}{1.150579in}}%
\pgfpathclose%
\pgfusepath{stroke,fill}%
\end{pgfscope}%
\begin{pgfscope}%
\pgfpathrectangle{\pgfqpoint{0.100000in}{0.212622in}}{\pgfqpoint{3.696000in}{3.696000in}}%
\pgfusepath{clip}%
\pgfsetbuttcap%
\pgfsetroundjoin%
\definecolor{currentfill}{rgb}{0.121569,0.466667,0.705882}%
\pgfsetfillcolor{currentfill}%
\pgfsetfillopacity{0.970917}%
\pgfsetlinewidth{1.003750pt}%
\definecolor{currentstroke}{rgb}{0.121569,0.466667,0.705882}%
\pgfsetstrokecolor{currentstroke}%
\pgfsetstrokeopacity{0.970917}%
\pgfsetdash{}{0pt}%
\pgfpathmoveto{\pgfqpoint{2.545886in}{1.149808in}}%
\pgfpathcurveto{\pgfqpoint{2.554123in}{1.149808in}}{\pgfqpoint{2.562023in}{1.153080in}}{\pgfqpoint{2.567847in}{1.158904in}}%
\pgfpathcurveto{\pgfqpoint{2.573671in}{1.164728in}}{\pgfqpoint{2.576943in}{1.172628in}}{\pgfqpoint{2.576943in}{1.180864in}}%
\pgfpathcurveto{\pgfqpoint{2.576943in}{1.189101in}}{\pgfqpoint{2.573671in}{1.197001in}}{\pgfqpoint{2.567847in}{1.202825in}}%
\pgfpathcurveto{\pgfqpoint{2.562023in}{1.208649in}}{\pgfqpoint{2.554123in}{1.211921in}}{\pgfqpoint{2.545886in}{1.211921in}}%
\pgfpathcurveto{\pgfqpoint{2.537650in}{1.211921in}}{\pgfqpoint{2.529750in}{1.208649in}}{\pgfqpoint{2.523926in}{1.202825in}}%
\pgfpathcurveto{\pgfqpoint{2.518102in}{1.197001in}}{\pgfqpoint{2.514830in}{1.189101in}}{\pgfqpoint{2.514830in}{1.180864in}}%
\pgfpathcurveto{\pgfqpoint{2.514830in}{1.172628in}}{\pgfqpoint{2.518102in}{1.164728in}}{\pgfqpoint{2.523926in}{1.158904in}}%
\pgfpathcurveto{\pgfqpoint{2.529750in}{1.153080in}}{\pgfqpoint{2.537650in}{1.149808in}}{\pgfqpoint{2.545886in}{1.149808in}}%
\pgfpathclose%
\pgfusepath{stroke,fill}%
\end{pgfscope}%
\begin{pgfscope}%
\pgfpathrectangle{\pgfqpoint{0.100000in}{0.212622in}}{\pgfqpoint{3.696000in}{3.696000in}}%
\pgfusepath{clip}%
\pgfsetbuttcap%
\pgfsetroundjoin%
\definecolor{currentfill}{rgb}{0.121569,0.466667,0.705882}%
\pgfsetfillcolor{currentfill}%
\pgfsetfillopacity{0.971510}%
\pgfsetlinewidth{1.003750pt}%
\definecolor{currentstroke}{rgb}{0.121569,0.466667,0.705882}%
\pgfsetstrokecolor{currentstroke}%
\pgfsetstrokeopacity{0.971510}%
\pgfsetdash{}{0pt}%
\pgfpathmoveto{\pgfqpoint{2.544833in}{1.148522in}}%
\pgfpathcurveto{\pgfqpoint{2.553069in}{1.148522in}}{\pgfqpoint{2.560969in}{1.151795in}}{\pgfqpoint{2.566793in}{1.157618in}}%
\pgfpathcurveto{\pgfqpoint{2.572617in}{1.163442in}}{\pgfqpoint{2.575890in}{1.171342in}}{\pgfqpoint{2.575890in}{1.179579in}}%
\pgfpathcurveto{\pgfqpoint{2.575890in}{1.187815in}}{\pgfqpoint{2.572617in}{1.195715in}}{\pgfqpoint{2.566793in}{1.201539in}}%
\pgfpathcurveto{\pgfqpoint{2.560969in}{1.207363in}}{\pgfqpoint{2.553069in}{1.210635in}}{\pgfqpoint{2.544833in}{1.210635in}}%
\pgfpathcurveto{\pgfqpoint{2.536597in}{1.210635in}}{\pgfqpoint{2.528697in}{1.207363in}}{\pgfqpoint{2.522873in}{1.201539in}}%
\pgfpathcurveto{\pgfqpoint{2.517049in}{1.195715in}}{\pgfqpoint{2.513777in}{1.187815in}}{\pgfqpoint{2.513777in}{1.179579in}}%
\pgfpathcurveto{\pgfqpoint{2.513777in}{1.171342in}}{\pgfqpoint{2.517049in}{1.163442in}}{\pgfqpoint{2.522873in}{1.157618in}}%
\pgfpathcurveto{\pgfqpoint{2.528697in}{1.151795in}}{\pgfqpoint{2.536597in}{1.148522in}}{\pgfqpoint{2.544833in}{1.148522in}}%
\pgfpathclose%
\pgfusepath{stroke,fill}%
\end{pgfscope}%
\begin{pgfscope}%
\pgfpathrectangle{\pgfqpoint{0.100000in}{0.212622in}}{\pgfqpoint{3.696000in}{3.696000in}}%
\pgfusepath{clip}%
\pgfsetbuttcap%
\pgfsetroundjoin%
\definecolor{currentfill}{rgb}{0.121569,0.466667,0.705882}%
\pgfsetfillcolor{currentfill}%
\pgfsetfillopacity{0.971790}%
\pgfsetlinewidth{1.003750pt}%
\definecolor{currentstroke}{rgb}{0.121569,0.466667,0.705882}%
\pgfsetstrokecolor{currentstroke}%
\pgfsetstrokeopacity{0.971790}%
\pgfsetdash{}{0pt}%
\pgfpathmoveto{\pgfqpoint{1.828202in}{1.489465in}}%
\pgfpathcurveto{\pgfqpoint{1.836439in}{1.489465in}}{\pgfqpoint{1.844339in}{1.492737in}}{\pgfqpoint{1.850163in}{1.498561in}}%
\pgfpathcurveto{\pgfqpoint{1.855987in}{1.504385in}}{\pgfqpoint{1.859259in}{1.512285in}}{\pgfqpoint{1.859259in}{1.520522in}}%
\pgfpathcurveto{\pgfqpoint{1.859259in}{1.528758in}}{\pgfqpoint{1.855987in}{1.536658in}}{\pgfqpoint{1.850163in}{1.542482in}}%
\pgfpathcurveto{\pgfqpoint{1.844339in}{1.548306in}}{\pgfqpoint{1.836439in}{1.551578in}}{\pgfqpoint{1.828202in}{1.551578in}}%
\pgfpathcurveto{\pgfqpoint{1.819966in}{1.551578in}}{\pgfqpoint{1.812066in}{1.548306in}}{\pgfqpoint{1.806242in}{1.542482in}}%
\pgfpathcurveto{\pgfqpoint{1.800418in}{1.536658in}}{\pgfqpoint{1.797146in}{1.528758in}}{\pgfqpoint{1.797146in}{1.520522in}}%
\pgfpathcurveto{\pgfqpoint{1.797146in}{1.512285in}}{\pgfqpoint{1.800418in}{1.504385in}}{\pgfqpoint{1.806242in}{1.498561in}}%
\pgfpathcurveto{\pgfqpoint{1.812066in}{1.492737in}}{\pgfqpoint{1.819966in}{1.489465in}}{\pgfqpoint{1.828202in}{1.489465in}}%
\pgfpathclose%
\pgfusepath{stroke,fill}%
\end{pgfscope}%
\begin{pgfscope}%
\pgfpathrectangle{\pgfqpoint{0.100000in}{0.212622in}}{\pgfqpoint{3.696000in}{3.696000in}}%
\pgfusepath{clip}%
\pgfsetbuttcap%
\pgfsetroundjoin%
\definecolor{currentfill}{rgb}{0.121569,0.466667,0.705882}%
\pgfsetfillcolor{currentfill}%
\pgfsetfillopacity{0.971997}%
\pgfsetlinewidth{1.003750pt}%
\definecolor{currentstroke}{rgb}{0.121569,0.466667,0.705882}%
\pgfsetstrokecolor{currentstroke}%
\pgfsetstrokeopacity{0.971997}%
\pgfsetdash{}{0pt}%
\pgfpathmoveto{\pgfqpoint{2.543972in}{1.147602in}}%
\pgfpathcurveto{\pgfqpoint{2.552208in}{1.147602in}}{\pgfqpoint{2.560108in}{1.150874in}}{\pgfqpoint{2.565932in}{1.156698in}}%
\pgfpathcurveto{\pgfqpoint{2.571756in}{1.162522in}}{\pgfqpoint{2.575028in}{1.170422in}}{\pgfqpoint{2.575028in}{1.178659in}}%
\pgfpathcurveto{\pgfqpoint{2.575028in}{1.186895in}}{\pgfqpoint{2.571756in}{1.194795in}}{\pgfqpoint{2.565932in}{1.200619in}}%
\pgfpathcurveto{\pgfqpoint{2.560108in}{1.206443in}}{\pgfqpoint{2.552208in}{1.209715in}}{\pgfqpoint{2.543972in}{1.209715in}}%
\pgfpathcurveto{\pgfqpoint{2.535736in}{1.209715in}}{\pgfqpoint{2.527836in}{1.206443in}}{\pgfqpoint{2.522012in}{1.200619in}}%
\pgfpathcurveto{\pgfqpoint{2.516188in}{1.194795in}}{\pgfqpoint{2.512915in}{1.186895in}}{\pgfqpoint{2.512915in}{1.178659in}}%
\pgfpathcurveto{\pgfqpoint{2.512915in}{1.170422in}}{\pgfqpoint{2.516188in}{1.162522in}}{\pgfqpoint{2.522012in}{1.156698in}}%
\pgfpathcurveto{\pgfqpoint{2.527836in}{1.150874in}}{\pgfqpoint{2.535736in}{1.147602in}}{\pgfqpoint{2.543972in}{1.147602in}}%
\pgfpathclose%
\pgfusepath{stroke,fill}%
\end{pgfscope}%
\begin{pgfscope}%
\pgfpathrectangle{\pgfqpoint{0.100000in}{0.212622in}}{\pgfqpoint{3.696000in}{3.696000in}}%
\pgfusepath{clip}%
\pgfsetbuttcap%
\pgfsetroundjoin%
\definecolor{currentfill}{rgb}{0.121569,0.466667,0.705882}%
\pgfsetfillcolor{currentfill}%
\pgfsetfillopacity{0.972903}%
\pgfsetlinewidth{1.003750pt}%
\definecolor{currentstroke}{rgb}{0.121569,0.466667,0.705882}%
\pgfsetstrokecolor{currentstroke}%
\pgfsetstrokeopacity{0.972903}%
\pgfsetdash{}{0pt}%
\pgfpathmoveto{\pgfqpoint{1.851353in}{1.476524in}}%
\pgfpathcurveto{\pgfqpoint{1.859589in}{1.476524in}}{\pgfqpoint{1.867489in}{1.479796in}}{\pgfqpoint{1.873313in}{1.485620in}}%
\pgfpathcurveto{\pgfqpoint{1.879137in}{1.491444in}}{\pgfqpoint{1.882409in}{1.499344in}}{\pgfqpoint{1.882409in}{1.507580in}}%
\pgfpathcurveto{\pgfqpoint{1.882409in}{1.515816in}}{\pgfqpoint{1.879137in}{1.523716in}}{\pgfqpoint{1.873313in}{1.529540in}}%
\pgfpathcurveto{\pgfqpoint{1.867489in}{1.535364in}}{\pgfqpoint{1.859589in}{1.538637in}}{\pgfqpoint{1.851353in}{1.538637in}}%
\pgfpathcurveto{\pgfqpoint{1.843116in}{1.538637in}}{\pgfqpoint{1.835216in}{1.535364in}}{\pgfqpoint{1.829392in}{1.529540in}}%
\pgfpathcurveto{\pgfqpoint{1.823568in}{1.523716in}}{\pgfqpoint{1.820296in}{1.515816in}}{\pgfqpoint{1.820296in}{1.507580in}}%
\pgfpathcurveto{\pgfqpoint{1.820296in}{1.499344in}}{\pgfqpoint{1.823568in}{1.491444in}}{\pgfqpoint{1.829392in}{1.485620in}}%
\pgfpathcurveto{\pgfqpoint{1.835216in}{1.479796in}}{\pgfqpoint{1.843116in}{1.476524in}}{\pgfqpoint{1.851353in}{1.476524in}}%
\pgfpathclose%
\pgfusepath{stroke,fill}%
\end{pgfscope}%
\begin{pgfscope}%
\pgfpathrectangle{\pgfqpoint{0.100000in}{0.212622in}}{\pgfqpoint{3.696000in}{3.696000in}}%
\pgfusepath{clip}%
\pgfsetbuttcap%
\pgfsetroundjoin%
\definecolor{currentfill}{rgb}{0.121569,0.466667,0.705882}%
\pgfsetfillcolor{currentfill}%
\pgfsetfillopacity{0.972910}%
\pgfsetlinewidth{1.003750pt}%
\definecolor{currentstroke}{rgb}{0.121569,0.466667,0.705882}%
\pgfsetstrokecolor{currentstroke}%
\pgfsetstrokeopacity{0.972910}%
\pgfsetdash{}{0pt}%
\pgfpathmoveto{\pgfqpoint{2.542418in}{1.146063in}}%
\pgfpathcurveto{\pgfqpoint{2.550654in}{1.146063in}}{\pgfqpoint{2.558554in}{1.149335in}}{\pgfqpoint{2.564378in}{1.155159in}}%
\pgfpathcurveto{\pgfqpoint{2.570202in}{1.160983in}}{\pgfqpoint{2.573475in}{1.168883in}}{\pgfqpoint{2.573475in}{1.177120in}}%
\pgfpathcurveto{\pgfqpoint{2.573475in}{1.185356in}}{\pgfqpoint{2.570202in}{1.193256in}}{\pgfqpoint{2.564378in}{1.199080in}}%
\pgfpathcurveto{\pgfqpoint{2.558554in}{1.204904in}}{\pgfqpoint{2.550654in}{1.208176in}}{\pgfqpoint{2.542418in}{1.208176in}}%
\pgfpathcurveto{\pgfqpoint{2.534182in}{1.208176in}}{\pgfqpoint{2.526282in}{1.204904in}}{\pgfqpoint{2.520458in}{1.199080in}}%
\pgfpathcurveto{\pgfqpoint{2.514634in}{1.193256in}}{\pgfqpoint{2.511362in}{1.185356in}}{\pgfqpoint{2.511362in}{1.177120in}}%
\pgfpathcurveto{\pgfqpoint{2.511362in}{1.168883in}}{\pgfqpoint{2.514634in}{1.160983in}}{\pgfqpoint{2.520458in}{1.155159in}}%
\pgfpathcurveto{\pgfqpoint{2.526282in}{1.149335in}}{\pgfqpoint{2.534182in}{1.146063in}}{\pgfqpoint{2.542418in}{1.146063in}}%
\pgfpathclose%
\pgfusepath{stroke,fill}%
\end{pgfscope}%
\begin{pgfscope}%
\pgfpathrectangle{\pgfqpoint{0.100000in}{0.212622in}}{\pgfqpoint{3.696000in}{3.696000in}}%
\pgfusepath{clip}%
\pgfsetbuttcap%
\pgfsetroundjoin%
\definecolor{currentfill}{rgb}{0.121569,0.466667,0.705882}%
\pgfsetfillcolor{currentfill}%
\pgfsetfillopacity{0.973324}%
\pgfsetlinewidth{1.003750pt}%
\definecolor{currentstroke}{rgb}{0.121569,0.466667,0.705882}%
\pgfsetstrokecolor{currentstroke}%
\pgfsetstrokeopacity{0.973324}%
\pgfsetdash{}{0pt}%
\pgfpathmoveto{\pgfqpoint{1.864049in}{1.468150in}}%
\pgfpathcurveto{\pgfqpoint{1.872285in}{1.468150in}}{\pgfqpoint{1.880185in}{1.471422in}}{\pgfqpoint{1.886009in}{1.477246in}}%
\pgfpathcurveto{\pgfqpoint{1.891833in}{1.483070in}}{\pgfqpoint{1.895105in}{1.490970in}}{\pgfqpoint{1.895105in}{1.499206in}}%
\pgfpathcurveto{\pgfqpoint{1.895105in}{1.507443in}}{\pgfqpoint{1.891833in}{1.515343in}}{\pgfqpoint{1.886009in}{1.521166in}}%
\pgfpathcurveto{\pgfqpoint{1.880185in}{1.526990in}}{\pgfqpoint{1.872285in}{1.530263in}}{\pgfqpoint{1.864049in}{1.530263in}}%
\pgfpathcurveto{\pgfqpoint{1.855813in}{1.530263in}}{\pgfqpoint{1.847913in}{1.526990in}}{\pgfqpoint{1.842089in}{1.521166in}}%
\pgfpathcurveto{\pgfqpoint{1.836265in}{1.515343in}}{\pgfqpoint{1.832992in}{1.507443in}}{\pgfqpoint{1.832992in}{1.499206in}}%
\pgfpathcurveto{\pgfqpoint{1.832992in}{1.490970in}}{\pgfqpoint{1.836265in}{1.483070in}}{\pgfqpoint{1.842089in}{1.477246in}}%
\pgfpathcurveto{\pgfqpoint{1.847913in}{1.471422in}}{\pgfqpoint{1.855813in}{1.468150in}}{\pgfqpoint{1.864049in}{1.468150in}}%
\pgfpathclose%
\pgfusepath{stroke,fill}%
\end{pgfscope}%
\begin{pgfscope}%
\pgfpathrectangle{\pgfqpoint{0.100000in}{0.212622in}}{\pgfqpoint{3.696000in}{3.696000in}}%
\pgfusepath{clip}%
\pgfsetbuttcap%
\pgfsetroundjoin%
\definecolor{currentfill}{rgb}{0.121569,0.466667,0.705882}%
\pgfsetfillcolor{currentfill}%
\pgfsetfillopacity{0.973509}%
\pgfsetlinewidth{1.003750pt}%
\definecolor{currentstroke}{rgb}{0.121569,0.466667,0.705882}%
\pgfsetstrokecolor{currentstroke}%
\pgfsetstrokeopacity{0.973509}%
\pgfsetdash{}{0pt}%
\pgfpathmoveto{\pgfqpoint{2.541433in}{1.145121in}}%
\pgfpathcurveto{\pgfqpoint{2.549669in}{1.145121in}}{\pgfqpoint{2.557569in}{1.148394in}}{\pgfqpoint{2.563393in}{1.154218in}}%
\pgfpathcurveto{\pgfqpoint{2.569217in}{1.160042in}}{\pgfqpoint{2.572489in}{1.167942in}}{\pgfqpoint{2.572489in}{1.176178in}}%
\pgfpathcurveto{\pgfqpoint{2.572489in}{1.184414in}}{\pgfqpoint{2.569217in}{1.192314in}}{\pgfqpoint{2.563393in}{1.198138in}}%
\pgfpathcurveto{\pgfqpoint{2.557569in}{1.203962in}}{\pgfqpoint{2.549669in}{1.207234in}}{\pgfqpoint{2.541433in}{1.207234in}}%
\pgfpathcurveto{\pgfqpoint{2.533197in}{1.207234in}}{\pgfqpoint{2.525297in}{1.203962in}}{\pgfqpoint{2.519473in}{1.198138in}}%
\pgfpathcurveto{\pgfqpoint{2.513649in}{1.192314in}}{\pgfqpoint{2.510376in}{1.184414in}}{\pgfqpoint{2.510376in}{1.176178in}}%
\pgfpathcurveto{\pgfqpoint{2.510376in}{1.167942in}}{\pgfqpoint{2.513649in}{1.160042in}}{\pgfqpoint{2.519473in}{1.154218in}}%
\pgfpathcurveto{\pgfqpoint{2.525297in}{1.148394in}}{\pgfqpoint{2.533197in}{1.145121in}}{\pgfqpoint{2.541433in}{1.145121in}}%
\pgfpathclose%
\pgfusepath{stroke,fill}%
\end{pgfscope}%
\begin{pgfscope}%
\pgfpathrectangle{\pgfqpoint{0.100000in}{0.212622in}}{\pgfqpoint{3.696000in}{3.696000in}}%
\pgfusepath{clip}%
\pgfsetbuttcap%
\pgfsetroundjoin%
\definecolor{currentfill}{rgb}{0.121569,0.466667,0.705882}%
\pgfsetfillcolor{currentfill}%
\pgfsetfillopacity{0.973780}%
\pgfsetlinewidth{1.003750pt}%
\definecolor{currentstroke}{rgb}{0.121569,0.466667,0.705882}%
\pgfsetstrokecolor{currentstroke}%
\pgfsetstrokeopacity{0.973780}%
\pgfsetdash{}{0pt}%
\pgfpathmoveto{\pgfqpoint{1.877651in}{1.459677in}}%
\pgfpathcurveto{\pgfqpoint{1.885887in}{1.459677in}}{\pgfqpoint{1.893787in}{1.462949in}}{\pgfqpoint{1.899611in}{1.468773in}}%
\pgfpathcurveto{\pgfqpoint{1.905435in}{1.474597in}}{\pgfqpoint{1.908707in}{1.482497in}}{\pgfqpoint{1.908707in}{1.490734in}}%
\pgfpathcurveto{\pgfqpoint{1.908707in}{1.498970in}}{\pgfqpoint{1.905435in}{1.506870in}}{\pgfqpoint{1.899611in}{1.512694in}}%
\pgfpathcurveto{\pgfqpoint{1.893787in}{1.518518in}}{\pgfqpoint{1.885887in}{1.521790in}}{\pgfqpoint{1.877651in}{1.521790in}}%
\pgfpathcurveto{\pgfqpoint{1.869414in}{1.521790in}}{\pgfqpoint{1.861514in}{1.518518in}}{\pgfqpoint{1.855690in}{1.512694in}}%
\pgfpathcurveto{\pgfqpoint{1.849867in}{1.506870in}}{\pgfqpoint{1.846594in}{1.498970in}}{\pgfqpoint{1.846594in}{1.490734in}}%
\pgfpathcurveto{\pgfqpoint{1.846594in}{1.482497in}}{\pgfqpoint{1.849867in}{1.474597in}}{\pgfqpoint{1.855690in}{1.468773in}}%
\pgfpathcurveto{\pgfqpoint{1.861514in}{1.462949in}}{\pgfqpoint{1.869414in}{1.459677in}}{\pgfqpoint{1.877651in}{1.459677in}}%
\pgfpathclose%
\pgfusepath{stroke,fill}%
\end{pgfscope}%
\begin{pgfscope}%
\pgfpathrectangle{\pgfqpoint{0.100000in}{0.212622in}}{\pgfqpoint{3.696000in}{3.696000in}}%
\pgfusepath{clip}%
\pgfsetbuttcap%
\pgfsetroundjoin%
\definecolor{currentfill}{rgb}{0.121569,0.466667,0.705882}%
\pgfsetfillcolor{currentfill}%
\pgfsetfillopacity{0.974280}%
\pgfsetlinewidth{1.003750pt}%
\definecolor{currentstroke}{rgb}{0.121569,0.466667,0.705882}%
\pgfsetstrokecolor{currentstroke}%
\pgfsetstrokeopacity{0.974280}%
\pgfsetdash{}{0pt}%
\pgfpathmoveto{\pgfqpoint{1.892052in}{1.451186in}}%
\pgfpathcurveto{\pgfqpoint{1.900288in}{1.451186in}}{\pgfqpoint{1.908189in}{1.454458in}}{\pgfqpoint{1.914012in}{1.460282in}}%
\pgfpathcurveto{\pgfqpoint{1.919836in}{1.466106in}}{\pgfqpoint{1.923109in}{1.474006in}}{\pgfqpoint{1.923109in}{1.482242in}}%
\pgfpathcurveto{\pgfqpoint{1.923109in}{1.490479in}}{\pgfqpoint{1.919836in}{1.498379in}}{\pgfqpoint{1.914012in}{1.504203in}}%
\pgfpathcurveto{\pgfqpoint{1.908189in}{1.510027in}}{\pgfqpoint{1.900288in}{1.513299in}}{\pgfqpoint{1.892052in}{1.513299in}}%
\pgfpathcurveto{\pgfqpoint{1.883816in}{1.513299in}}{\pgfqpoint{1.875916in}{1.510027in}}{\pgfqpoint{1.870092in}{1.504203in}}%
\pgfpathcurveto{\pgfqpoint{1.864268in}{1.498379in}}{\pgfqpoint{1.860996in}{1.490479in}}{\pgfqpoint{1.860996in}{1.482242in}}%
\pgfpathcurveto{\pgfqpoint{1.860996in}{1.474006in}}{\pgfqpoint{1.864268in}{1.466106in}}{\pgfqpoint{1.870092in}{1.460282in}}%
\pgfpathcurveto{\pgfqpoint{1.875916in}{1.454458in}}{\pgfqpoint{1.883816in}{1.451186in}}{\pgfqpoint{1.892052in}{1.451186in}}%
\pgfpathclose%
\pgfusepath{stroke,fill}%
\end{pgfscope}%
\begin{pgfscope}%
\pgfpathrectangle{\pgfqpoint{0.100000in}{0.212622in}}{\pgfqpoint{3.696000in}{3.696000in}}%
\pgfusepath{clip}%
\pgfsetbuttcap%
\pgfsetroundjoin%
\definecolor{currentfill}{rgb}{0.121569,0.466667,0.705882}%
\pgfsetfillcolor{currentfill}%
\pgfsetfillopacity{0.974521}%
\pgfsetlinewidth{1.003750pt}%
\definecolor{currentstroke}{rgb}{0.121569,0.466667,0.705882}%
\pgfsetstrokecolor{currentstroke}%
\pgfsetstrokeopacity{0.974521}%
\pgfsetdash{}{0pt}%
\pgfpathmoveto{\pgfqpoint{2.539703in}{1.142931in}}%
\pgfpathcurveto{\pgfqpoint{2.547939in}{1.142931in}}{\pgfqpoint{2.555839in}{1.146203in}}{\pgfqpoint{2.561663in}{1.152027in}}%
\pgfpathcurveto{\pgfqpoint{2.567487in}{1.157851in}}{\pgfqpoint{2.570759in}{1.165751in}}{\pgfqpoint{2.570759in}{1.173987in}}%
\pgfpathcurveto{\pgfqpoint{2.570759in}{1.182223in}}{\pgfqpoint{2.567487in}{1.190124in}}{\pgfqpoint{2.561663in}{1.195947in}}%
\pgfpathcurveto{\pgfqpoint{2.555839in}{1.201771in}}{\pgfqpoint{2.547939in}{1.205044in}}{\pgfqpoint{2.539703in}{1.205044in}}%
\pgfpathcurveto{\pgfqpoint{2.531467in}{1.205044in}}{\pgfqpoint{2.523566in}{1.201771in}}{\pgfqpoint{2.517743in}{1.195947in}}%
\pgfpathcurveto{\pgfqpoint{2.511919in}{1.190124in}}{\pgfqpoint{2.508646in}{1.182223in}}{\pgfqpoint{2.508646in}{1.173987in}}%
\pgfpathcurveto{\pgfqpoint{2.508646in}{1.165751in}}{\pgfqpoint{2.511919in}{1.157851in}}{\pgfqpoint{2.517743in}{1.152027in}}%
\pgfpathcurveto{\pgfqpoint{2.523566in}{1.146203in}}{\pgfqpoint{2.531467in}{1.142931in}}{\pgfqpoint{2.539703in}{1.142931in}}%
\pgfpathclose%
\pgfusepath{stroke,fill}%
\end{pgfscope}%
\begin{pgfscope}%
\pgfpathrectangle{\pgfqpoint{0.100000in}{0.212622in}}{\pgfqpoint{3.696000in}{3.696000in}}%
\pgfusepath{clip}%
\pgfsetbuttcap%
\pgfsetroundjoin%
\definecolor{currentfill}{rgb}{0.121569,0.466667,0.705882}%
\pgfsetfillcolor{currentfill}%
\pgfsetfillopacity{0.974643}%
\pgfsetlinewidth{1.003750pt}%
\definecolor{currentstroke}{rgb}{0.121569,0.466667,0.705882}%
\pgfsetstrokecolor{currentstroke}%
\pgfsetstrokeopacity{0.974643}%
\pgfsetdash{}{0pt}%
\pgfpathmoveto{\pgfqpoint{1.900050in}{1.447333in}}%
\pgfpathcurveto{\pgfqpoint{1.908286in}{1.447333in}}{\pgfqpoint{1.916186in}{1.450605in}}{\pgfqpoint{1.922010in}{1.456429in}}%
\pgfpathcurveto{\pgfqpoint{1.927834in}{1.462253in}}{\pgfqpoint{1.931106in}{1.470153in}}{\pgfqpoint{1.931106in}{1.478389in}}%
\pgfpathcurveto{\pgfqpoint{1.931106in}{1.486626in}}{\pgfqpoint{1.927834in}{1.494526in}}{\pgfqpoint{1.922010in}{1.500350in}}%
\pgfpathcurveto{\pgfqpoint{1.916186in}{1.506174in}}{\pgfqpoint{1.908286in}{1.509446in}}{\pgfqpoint{1.900050in}{1.509446in}}%
\pgfpathcurveto{\pgfqpoint{1.891814in}{1.509446in}}{\pgfqpoint{1.883914in}{1.506174in}}{\pgfqpoint{1.878090in}{1.500350in}}%
\pgfpathcurveto{\pgfqpoint{1.872266in}{1.494526in}}{\pgfqpoint{1.868993in}{1.486626in}}{\pgfqpoint{1.868993in}{1.478389in}}%
\pgfpathcurveto{\pgfqpoint{1.868993in}{1.470153in}}{\pgfqpoint{1.872266in}{1.462253in}}{\pgfqpoint{1.878090in}{1.456429in}}%
\pgfpathcurveto{\pgfqpoint{1.883914in}{1.450605in}}{\pgfqpoint{1.891814in}{1.447333in}}{\pgfqpoint{1.900050in}{1.447333in}}%
\pgfpathclose%
\pgfusepath{stroke,fill}%
\end{pgfscope}%
\begin{pgfscope}%
\pgfpathrectangle{\pgfqpoint{0.100000in}{0.212622in}}{\pgfqpoint{3.696000in}{3.696000in}}%
\pgfusepath{clip}%
\pgfsetbuttcap%
\pgfsetroundjoin%
\definecolor{currentfill}{rgb}{0.121569,0.466667,0.705882}%
\pgfsetfillcolor{currentfill}%
\pgfsetfillopacity{0.975008}%
\pgfsetlinewidth{1.003750pt}%
\definecolor{currentstroke}{rgb}{0.121569,0.466667,0.705882}%
\pgfsetstrokecolor{currentstroke}%
\pgfsetstrokeopacity{0.975008}%
\pgfsetdash{}{0pt}%
\pgfpathmoveto{\pgfqpoint{1.909572in}{1.442030in}}%
\pgfpathcurveto{\pgfqpoint{1.917808in}{1.442030in}}{\pgfqpoint{1.925708in}{1.445303in}}{\pgfqpoint{1.931532in}{1.451126in}}%
\pgfpathcurveto{\pgfqpoint{1.937356in}{1.456950in}}{\pgfqpoint{1.940628in}{1.464850in}}{\pgfqpoint{1.940628in}{1.473087in}}%
\pgfpathcurveto{\pgfqpoint{1.940628in}{1.481323in}}{\pgfqpoint{1.937356in}{1.489223in}}{\pgfqpoint{1.931532in}{1.495047in}}%
\pgfpathcurveto{\pgfqpoint{1.925708in}{1.500871in}}{\pgfqpoint{1.917808in}{1.504143in}}{\pgfqpoint{1.909572in}{1.504143in}}%
\pgfpathcurveto{\pgfqpoint{1.901335in}{1.504143in}}{\pgfqpoint{1.893435in}{1.500871in}}{\pgfqpoint{1.887611in}{1.495047in}}%
\pgfpathcurveto{\pgfqpoint{1.881787in}{1.489223in}}{\pgfqpoint{1.878515in}{1.481323in}}{\pgfqpoint{1.878515in}{1.473087in}}%
\pgfpathcurveto{\pgfqpoint{1.878515in}{1.464850in}}{\pgfqpoint{1.881787in}{1.456950in}}{\pgfqpoint{1.887611in}{1.451126in}}%
\pgfpathcurveto{\pgfqpoint{1.893435in}{1.445303in}}{\pgfqpoint{1.901335in}{1.442030in}}{\pgfqpoint{1.909572in}{1.442030in}}%
\pgfpathclose%
\pgfusepath{stroke,fill}%
\end{pgfscope}%
\begin{pgfscope}%
\pgfpathrectangle{\pgfqpoint{0.100000in}{0.212622in}}{\pgfqpoint{3.696000in}{3.696000in}}%
\pgfusepath{clip}%
\pgfsetbuttcap%
\pgfsetroundjoin%
\definecolor{currentfill}{rgb}{0.121569,0.466667,0.705882}%
\pgfsetfillcolor{currentfill}%
\pgfsetfillopacity{0.975399}%
\pgfsetlinewidth{1.003750pt}%
\definecolor{currentstroke}{rgb}{0.121569,0.466667,0.705882}%
\pgfsetstrokecolor{currentstroke}%
\pgfsetstrokeopacity{0.975399}%
\pgfsetdash{}{0pt}%
\pgfpathmoveto{\pgfqpoint{2.538129in}{1.140758in}}%
\pgfpathcurveto{\pgfqpoint{2.546365in}{1.140758in}}{\pgfqpoint{2.554265in}{1.144030in}}{\pgfqpoint{2.560089in}{1.149854in}}%
\pgfpathcurveto{\pgfqpoint{2.565913in}{1.155678in}}{\pgfqpoint{2.569185in}{1.163578in}}{\pgfqpoint{2.569185in}{1.171814in}}%
\pgfpathcurveto{\pgfqpoint{2.569185in}{1.180050in}}{\pgfqpoint{2.565913in}{1.187950in}}{\pgfqpoint{2.560089in}{1.193774in}}%
\pgfpathcurveto{\pgfqpoint{2.554265in}{1.199598in}}{\pgfqpoint{2.546365in}{1.202871in}}{\pgfqpoint{2.538129in}{1.202871in}}%
\pgfpathcurveto{\pgfqpoint{2.529893in}{1.202871in}}{\pgfqpoint{2.521993in}{1.199598in}}{\pgfqpoint{2.516169in}{1.193774in}}%
\pgfpathcurveto{\pgfqpoint{2.510345in}{1.187950in}}{\pgfqpoint{2.507072in}{1.180050in}}{\pgfqpoint{2.507072in}{1.171814in}}%
\pgfpathcurveto{\pgfqpoint{2.507072in}{1.163578in}}{\pgfqpoint{2.510345in}{1.155678in}}{\pgfqpoint{2.516169in}{1.149854in}}%
\pgfpathcurveto{\pgfqpoint{2.521993in}{1.144030in}}{\pgfqpoint{2.529893in}{1.140758in}}{\pgfqpoint{2.538129in}{1.140758in}}%
\pgfpathclose%
\pgfusepath{stroke,fill}%
\end{pgfscope}%
\begin{pgfscope}%
\pgfpathrectangle{\pgfqpoint{0.100000in}{0.212622in}}{\pgfqpoint{3.696000in}{3.696000in}}%
\pgfusepath{clip}%
\pgfsetbuttcap%
\pgfsetroundjoin%
\definecolor{currentfill}{rgb}{0.121569,0.466667,0.705882}%
\pgfsetfillcolor{currentfill}%
\pgfsetfillopacity{0.975492}%
\pgfsetlinewidth{1.003750pt}%
\definecolor{currentstroke}{rgb}{0.121569,0.466667,0.705882}%
\pgfsetstrokecolor{currentstroke}%
\pgfsetstrokeopacity{0.975492}%
\pgfsetdash{}{0pt}%
\pgfpathmoveto{\pgfqpoint{1.920900in}{1.434980in}}%
\pgfpathcurveto{\pgfqpoint{1.929137in}{1.434980in}}{\pgfqpoint{1.937037in}{1.438252in}}{\pgfqpoint{1.942861in}{1.444076in}}%
\pgfpathcurveto{\pgfqpoint{1.948685in}{1.449900in}}{\pgfqpoint{1.951957in}{1.457800in}}{\pgfqpoint{1.951957in}{1.466037in}}%
\pgfpathcurveto{\pgfqpoint{1.951957in}{1.474273in}}{\pgfqpoint{1.948685in}{1.482173in}}{\pgfqpoint{1.942861in}{1.487997in}}%
\pgfpathcurveto{\pgfqpoint{1.937037in}{1.493821in}}{\pgfqpoint{1.929137in}{1.497093in}}{\pgfqpoint{1.920900in}{1.497093in}}%
\pgfpathcurveto{\pgfqpoint{1.912664in}{1.497093in}}{\pgfqpoint{1.904764in}{1.493821in}}{\pgfqpoint{1.898940in}{1.487997in}}%
\pgfpathcurveto{\pgfqpoint{1.893116in}{1.482173in}}{\pgfqpoint{1.889844in}{1.474273in}}{\pgfqpoint{1.889844in}{1.466037in}}%
\pgfpathcurveto{\pgfqpoint{1.889844in}{1.457800in}}{\pgfqpoint{1.893116in}{1.449900in}}{\pgfqpoint{1.898940in}{1.444076in}}%
\pgfpathcurveto{\pgfqpoint{1.904764in}{1.438252in}}{\pgfqpoint{1.912664in}{1.434980in}}{\pgfqpoint{1.920900in}{1.434980in}}%
\pgfpathclose%
\pgfusepath{stroke,fill}%
\end{pgfscope}%
\begin{pgfscope}%
\pgfpathrectangle{\pgfqpoint{0.100000in}{0.212622in}}{\pgfqpoint{3.696000in}{3.696000in}}%
\pgfusepath{clip}%
\pgfsetbuttcap%
\pgfsetroundjoin%
\definecolor{currentfill}{rgb}{0.121569,0.466667,0.705882}%
\pgfsetfillcolor{currentfill}%
\pgfsetfillopacity{0.975962}%
\pgfsetlinewidth{1.003750pt}%
\definecolor{currentstroke}{rgb}{0.121569,0.466667,0.705882}%
\pgfsetstrokecolor{currentstroke}%
\pgfsetstrokeopacity{0.975962}%
\pgfsetdash{}{0pt}%
\pgfpathmoveto{\pgfqpoint{2.537086in}{1.139312in}}%
\pgfpathcurveto{\pgfqpoint{2.545323in}{1.139312in}}{\pgfqpoint{2.553223in}{1.142584in}}{\pgfqpoint{2.559047in}{1.148408in}}%
\pgfpathcurveto{\pgfqpoint{2.564871in}{1.154232in}}{\pgfqpoint{2.568143in}{1.162132in}}{\pgfqpoint{2.568143in}{1.170368in}}%
\pgfpathcurveto{\pgfqpoint{2.568143in}{1.178605in}}{\pgfqpoint{2.564871in}{1.186505in}}{\pgfqpoint{2.559047in}{1.192329in}}%
\pgfpathcurveto{\pgfqpoint{2.553223in}{1.198152in}}{\pgfqpoint{2.545323in}{1.201425in}}{\pgfqpoint{2.537086in}{1.201425in}}%
\pgfpathcurveto{\pgfqpoint{2.528850in}{1.201425in}}{\pgfqpoint{2.520950in}{1.198152in}}{\pgfqpoint{2.515126in}{1.192329in}}%
\pgfpathcurveto{\pgfqpoint{2.509302in}{1.186505in}}{\pgfqpoint{2.506030in}{1.178605in}}{\pgfqpoint{2.506030in}{1.170368in}}%
\pgfpathcurveto{\pgfqpoint{2.506030in}{1.162132in}}{\pgfqpoint{2.509302in}{1.154232in}}{\pgfqpoint{2.515126in}{1.148408in}}%
\pgfpathcurveto{\pgfqpoint{2.520950in}{1.142584in}}{\pgfqpoint{2.528850in}{1.139312in}}{\pgfqpoint{2.537086in}{1.139312in}}%
\pgfpathclose%
\pgfusepath{stroke,fill}%
\end{pgfscope}%
\begin{pgfscope}%
\pgfpathrectangle{\pgfqpoint{0.100000in}{0.212622in}}{\pgfqpoint{3.696000in}{3.696000in}}%
\pgfusepath{clip}%
\pgfsetbuttcap%
\pgfsetroundjoin%
\definecolor{currentfill}{rgb}{0.121569,0.466667,0.705882}%
\pgfsetfillcolor{currentfill}%
\pgfsetfillopacity{0.975985}%
\pgfsetlinewidth{1.003750pt}%
\definecolor{currentstroke}{rgb}{0.121569,0.466667,0.705882}%
\pgfsetstrokecolor{currentstroke}%
\pgfsetstrokeopacity{0.975985}%
\pgfsetdash{}{0pt}%
\pgfpathmoveto{\pgfqpoint{1.932191in}{1.425944in}}%
\pgfpathcurveto{\pgfqpoint{1.940428in}{1.425944in}}{\pgfqpoint{1.948328in}{1.429216in}}{\pgfqpoint{1.954152in}{1.435040in}}%
\pgfpathcurveto{\pgfqpoint{1.959976in}{1.440864in}}{\pgfqpoint{1.963248in}{1.448764in}}{\pgfqpoint{1.963248in}{1.457001in}}%
\pgfpathcurveto{\pgfqpoint{1.963248in}{1.465237in}}{\pgfqpoint{1.959976in}{1.473137in}}{\pgfqpoint{1.954152in}{1.478961in}}%
\pgfpathcurveto{\pgfqpoint{1.948328in}{1.484785in}}{\pgfqpoint{1.940428in}{1.488057in}}{\pgfqpoint{1.932191in}{1.488057in}}%
\pgfpathcurveto{\pgfqpoint{1.923955in}{1.488057in}}{\pgfqpoint{1.916055in}{1.484785in}}{\pgfqpoint{1.910231in}{1.478961in}}%
\pgfpathcurveto{\pgfqpoint{1.904407in}{1.473137in}}{\pgfqpoint{1.901135in}{1.465237in}}{\pgfqpoint{1.901135in}{1.457001in}}%
\pgfpathcurveto{\pgfqpoint{1.901135in}{1.448764in}}{\pgfqpoint{1.904407in}{1.440864in}}{\pgfqpoint{1.910231in}{1.435040in}}%
\pgfpathcurveto{\pgfqpoint{1.916055in}{1.429216in}}{\pgfqpoint{1.923955in}{1.425944in}}{\pgfqpoint{1.932191in}{1.425944in}}%
\pgfpathclose%
\pgfusepath{stroke,fill}%
\end{pgfscope}%
\begin{pgfscope}%
\pgfpathrectangle{\pgfqpoint{0.100000in}{0.212622in}}{\pgfqpoint{3.696000in}{3.696000in}}%
\pgfusepath{clip}%
\pgfsetbuttcap%
\pgfsetroundjoin%
\definecolor{currentfill}{rgb}{0.121569,0.466667,0.705882}%
\pgfsetfillcolor{currentfill}%
\pgfsetfillopacity{0.976253}%
\pgfsetlinewidth{1.003750pt}%
\definecolor{currentstroke}{rgb}{0.121569,0.466667,0.705882}%
\pgfsetstrokecolor{currentstroke}%
\pgfsetstrokeopacity{0.976253}%
\pgfsetdash{}{0pt}%
\pgfpathmoveto{\pgfqpoint{1.938500in}{1.421316in}}%
\pgfpathcurveto{\pgfqpoint{1.946736in}{1.421316in}}{\pgfqpoint{1.954636in}{1.424589in}}{\pgfqpoint{1.960460in}{1.430412in}}%
\pgfpathcurveto{\pgfqpoint{1.966284in}{1.436236in}}{\pgfqpoint{1.969556in}{1.444136in}}{\pgfqpoint{1.969556in}{1.452373in}}%
\pgfpathcurveto{\pgfqpoint{1.969556in}{1.460609in}}{\pgfqpoint{1.966284in}{1.468509in}}{\pgfqpoint{1.960460in}{1.474333in}}%
\pgfpathcurveto{\pgfqpoint{1.954636in}{1.480157in}}{\pgfqpoint{1.946736in}{1.483429in}}{\pgfqpoint{1.938500in}{1.483429in}}%
\pgfpathcurveto{\pgfqpoint{1.930263in}{1.483429in}}{\pgfqpoint{1.922363in}{1.480157in}}{\pgfqpoint{1.916539in}{1.474333in}}%
\pgfpathcurveto{\pgfqpoint{1.910715in}{1.468509in}}{\pgfqpoint{1.907443in}{1.460609in}}{\pgfqpoint{1.907443in}{1.452373in}}%
\pgfpathcurveto{\pgfqpoint{1.907443in}{1.444136in}}{\pgfqpoint{1.910715in}{1.436236in}}{\pgfqpoint{1.916539in}{1.430412in}}%
\pgfpathcurveto{\pgfqpoint{1.922363in}{1.424589in}}{\pgfqpoint{1.930263in}{1.421316in}}{\pgfqpoint{1.938500in}{1.421316in}}%
\pgfpathclose%
\pgfusepath{stroke,fill}%
\end{pgfscope}%
\begin{pgfscope}%
\pgfpathrectangle{\pgfqpoint{0.100000in}{0.212622in}}{\pgfqpoint{3.696000in}{3.696000in}}%
\pgfusepath{clip}%
\pgfsetbuttcap%
\pgfsetroundjoin%
\definecolor{currentfill}{rgb}{0.121569,0.466667,0.705882}%
\pgfsetfillcolor{currentfill}%
\pgfsetfillopacity{0.976557}%
\pgfsetlinewidth{1.003750pt}%
\definecolor{currentstroke}{rgb}{0.121569,0.466667,0.705882}%
\pgfsetstrokecolor{currentstroke}%
\pgfsetstrokeopacity{0.976557}%
\pgfsetdash{}{0pt}%
\pgfpathmoveto{\pgfqpoint{1.946032in}{1.416153in}}%
\pgfpathcurveto{\pgfqpoint{1.954268in}{1.416153in}}{\pgfqpoint{1.962168in}{1.419426in}}{\pgfqpoint{1.967992in}{1.425249in}}%
\pgfpathcurveto{\pgfqpoint{1.973816in}{1.431073in}}{\pgfqpoint{1.977088in}{1.438973in}}{\pgfqpoint{1.977088in}{1.447210in}}%
\pgfpathcurveto{\pgfqpoint{1.977088in}{1.455446in}}{\pgfqpoint{1.973816in}{1.463346in}}{\pgfqpoint{1.967992in}{1.469170in}}%
\pgfpathcurveto{\pgfqpoint{1.962168in}{1.474994in}}{\pgfqpoint{1.954268in}{1.478266in}}{\pgfqpoint{1.946032in}{1.478266in}}%
\pgfpathcurveto{\pgfqpoint{1.937795in}{1.478266in}}{\pgfqpoint{1.929895in}{1.474994in}}{\pgfqpoint{1.924071in}{1.469170in}}%
\pgfpathcurveto{\pgfqpoint{1.918247in}{1.463346in}}{\pgfqpoint{1.914975in}{1.455446in}}{\pgfqpoint{1.914975in}{1.447210in}}%
\pgfpathcurveto{\pgfqpoint{1.914975in}{1.438973in}}{\pgfqpoint{1.918247in}{1.431073in}}{\pgfqpoint{1.924071in}{1.425249in}}%
\pgfpathcurveto{\pgfqpoint{1.929895in}{1.419426in}}{\pgfqpoint{1.937795in}{1.416153in}}{\pgfqpoint{1.946032in}{1.416153in}}%
\pgfpathclose%
\pgfusepath{stroke,fill}%
\end{pgfscope}%
\begin{pgfscope}%
\pgfpathrectangle{\pgfqpoint{0.100000in}{0.212622in}}{\pgfqpoint{3.696000in}{3.696000in}}%
\pgfusepath{clip}%
\pgfsetbuttcap%
\pgfsetroundjoin%
\definecolor{currentfill}{rgb}{0.121569,0.466667,0.705882}%
\pgfsetfillcolor{currentfill}%
\pgfsetfillopacity{0.976761}%
\pgfsetlinewidth{1.003750pt}%
\definecolor{currentstroke}{rgb}{0.121569,0.466667,0.705882}%
\pgfsetstrokecolor{currentstroke}%
\pgfsetstrokeopacity{0.976761}%
\pgfsetdash{}{0pt}%
\pgfpathmoveto{\pgfqpoint{1.950286in}{1.413970in}}%
\pgfpathcurveto{\pgfqpoint{1.958522in}{1.413970in}}{\pgfqpoint{1.966422in}{1.417243in}}{\pgfqpoint{1.972246in}{1.423067in}}%
\pgfpathcurveto{\pgfqpoint{1.978070in}{1.428891in}}{\pgfqpoint{1.981343in}{1.436791in}}{\pgfqpoint{1.981343in}{1.445027in}}%
\pgfpathcurveto{\pgfqpoint{1.981343in}{1.453263in}}{\pgfqpoint{1.978070in}{1.461163in}}{\pgfqpoint{1.972246in}{1.466987in}}%
\pgfpathcurveto{\pgfqpoint{1.966422in}{1.472811in}}{\pgfqpoint{1.958522in}{1.476083in}}{\pgfqpoint{1.950286in}{1.476083in}}%
\pgfpathcurveto{\pgfqpoint{1.942050in}{1.476083in}}{\pgfqpoint{1.934150in}{1.472811in}}{\pgfqpoint{1.928326in}{1.466987in}}%
\pgfpathcurveto{\pgfqpoint{1.922502in}{1.461163in}}{\pgfqpoint{1.919230in}{1.453263in}}{\pgfqpoint{1.919230in}{1.445027in}}%
\pgfpathcurveto{\pgfqpoint{1.919230in}{1.436791in}}{\pgfqpoint{1.922502in}{1.428891in}}{\pgfqpoint{1.928326in}{1.423067in}}%
\pgfpathcurveto{\pgfqpoint{1.934150in}{1.417243in}}{\pgfqpoint{1.942050in}{1.413970in}}{\pgfqpoint{1.950286in}{1.413970in}}%
\pgfpathclose%
\pgfusepath{stroke,fill}%
\end{pgfscope}%
\begin{pgfscope}%
\pgfpathrectangle{\pgfqpoint{0.100000in}{0.212622in}}{\pgfqpoint{3.696000in}{3.696000in}}%
\pgfusepath{clip}%
\pgfsetbuttcap%
\pgfsetroundjoin%
\definecolor{currentfill}{rgb}{0.121569,0.466667,0.705882}%
\pgfsetfillcolor{currentfill}%
\pgfsetfillopacity{0.976984}%
\pgfsetlinewidth{1.003750pt}%
\definecolor{currentstroke}{rgb}{0.121569,0.466667,0.705882}%
\pgfsetstrokecolor{currentstroke}%
\pgfsetstrokeopacity{0.976984}%
\pgfsetdash{}{0pt}%
\pgfpathmoveto{\pgfqpoint{2.534987in}{1.136849in}}%
\pgfpathcurveto{\pgfqpoint{2.543224in}{1.136849in}}{\pgfqpoint{2.551124in}{1.140121in}}{\pgfqpoint{2.556948in}{1.145945in}}%
\pgfpathcurveto{\pgfqpoint{2.562772in}{1.151769in}}{\pgfqpoint{2.566044in}{1.159669in}}{\pgfqpoint{2.566044in}{1.167905in}}%
\pgfpathcurveto{\pgfqpoint{2.566044in}{1.176142in}}{\pgfqpoint{2.562772in}{1.184042in}}{\pgfqpoint{2.556948in}{1.189866in}}%
\pgfpathcurveto{\pgfqpoint{2.551124in}{1.195690in}}{\pgfqpoint{2.543224in}{1.198962in}}{\pgfqpoint{2.534987in}{1.198962in}}%
\pgfpathcurveto{\pgfqpoint{2.526751in}{1.198962in}}{\pgfqpoint{2.518851in}{1.195690in}}{\pgfqpoint{2.513027in}{1.189866in}}%
\pgfpathcurveto{\pgfqpoint{2.507203in}{1.184042in}}{\pgfqpoint{2.503931in}{1.176142in}}{\pgfqpoint{2.503931in}{1.167905in}}%
\pgfpathcurveto{\pgfqpoint{2.503931in}{1.159669in}}{\pgfqpoint{2.507203in}{1.151769in}}{\pgfqpoint{2.513027in}{1.145945in}}%
\pgfpathcurveto{\pgfqpoint{2.518851in}{1.140121in}}{\pgfqpoint{2.526751in}{1.136849in}}{\pgfqpoint{2.534987in}{1.136849in}}%
\pgfpathclose%
\pgfusepath{stroke,fill}%
\end{pgfscope}%
\begin{pgfscope}%
\pgfpathrectangle{\pgfqpoint{0.100000in}{0.212622in}}{\pgfqpoint{3.696000in}{3.696000in}}%
\pgfusepath{clip}%
\pgfsetbuttcap%
\pgfsetroundjoin%
\definecolor{currentfill}{rgb}{0.121569,0.466667,0.705882}%
\pgfsetfillcolor{currentfill}%
\pgfsetfillopacity{0.977089}%
\pgfsetlinewidth{1.003750pt}%
\definecolor{currentstroke}{rgb}{0.121569,0.466667,0.705882}%
\pgfsetstrokecolor{currentstroke}%
\pgfsetstrokeopacity{0.977089}%
\pgfsetdash{}{0pt}%
\pgfpathmoveto{\pgfqpoint{1.957764in}{1.409786in}}%
\pgfpathcurveto{\pgfqpoint{1.966000in}{1.409786in}}{\pgfqpoint{1.973900in}{1.413058in}}{\pgfqpoint{1.979724in}{1.418882in}}%
\pgfpathcurveto{\pgfqpoint{1.985548in}{1.424706in}}{\pgfqpoint{1.988821in}{1.432606in}}{\pgfqpoint{1.988821in}{1.440843in}}%
\pgfpathcurveto{\pgfqpoint{1.988821in}{1.449079in}}{\pgfqpoint{1.985548in}{1.456979in}}{\pgfqpoint{1.979724in}{1.462803in}}%
\pgfpathcurveto{\pgfqpoint{1.973900in}{1.468627in}}{\pgfqpoint{1.966000in}{1.471899in}}{\pgfqpoint{1.957764in}{1.471899in}}%
\pgfpathcurveto{\pgfqpoint{1.949528in}{1.471899in}}{\pgfqpoint{1.941628in}{1.468627in}}{\pgfqpoint{1.935804in}{1.462803in}}%
\pgfpathcurveto{\pgfqpoint{1.929980in}{1.456979in}}{\pgfqpoint{1.926708in}{1.449079in}}{\pgfqpoint{1.926708in}{1.440843in}}%
\pgfpathcurveto{\pgfqpoint{1.926708in}{1.432606in}}{\pgfqpoint{1.929980in}{1.424706in}}{\pgfqpoint{1.935804in}{1.418882in}}%
\pgfpathcurveto{\pgfqpoint{1.941628in}{1.413058in}}{\pgfqpoint{1.949528in}{1.409786in}}{\pgfqpoint{1.957764in}{1.409786in}}%
\pgfpathclose%
\pgfusepath{stroke,fill}%
\end{pgfscope}%
\begin{pgfscope}%
\pgfpathrectangle{\pgfqpoint{0.100000in}{0.212622in}}{\pgfqpoint{3.696000in}{3.696000in}}%
\pgfusepath{clip}%
\pgfsetbuttcap%
\pgfsetroundjoin%
\definecolor{currentfill}{rgb}{0.121569,0.466667,0.705882}%
\pgfsetfillcolor{currentfill}%
\pgfsetfillopacity{0.977485}%
\pgfsetlinewidth{1.003750pt}%
\definecolor{currentstroke}{rgb}{0.121569,0.466667,0.705882}%
\pgfsetstrokecolor{currentstroke}%
\pgfsetstrokeopacity{0.977485}%
\pgfsetdash{}{0pt}%
\pgfpathmoveto{\pgfqpoint{1.967637in}{1.403736in}}%
\pgfpathcurveto{\pgfqpoint{1.975873in}{1.403736in}}{\pgfqpoint{1.983773in}{1.407009in}}{\pgfqpoint{1.989597in}{1.412833in}}%
\pgfpathcurveto{\pgfqpoint{1.995421in}{1.418656in}}{\pgfqpoint{1.998693in}{1.426556in}}{\pgfqpoint{1.998693in}{1.434793in}}%
\pgfpathcurveto{\pgfqpoint{1.998693in}{1.443029in}}{\pgfqpoint{1.995421in}{1.450929in}}{\pgfqpoint{1.989597in}{1.456753in}}%
\pgfpathcurveto{\pgfqpoint{1.983773in}{1.462577in}}{\pgfqpoint{1.975873in}{1.465849in}}{\pgfqpoint{1.967637in}{1.465849in}}%
\pgfpathcurveto{\pgfqpoint{1.959400in}{1.465849in}}{\pgfqpoint{1.951500in}{1.462577in}}{\pgfqpoint{1.945676in}{1.456753in}}%
\pgfpathcurveto{\pgfqpoint{1.939853in}{1.450929in}}{\pgfqpoint{1.936580in}{1.443029in}}{\pgfqpoint{1.936580in}{1.434793in}}%
\pgfpathcurveto{\pgfqpoint{1.936580in}{1.426556in}}{\pgfqpoint{1.939853in}{1.418656in}}{\pgfqpoint{1.945676in}{1.412833in}}%
\pgfpathcurveto{\pgfqpoint{1.951500in}{1.407009in}}{\pgfqpoint{1.959400in}{1.403736in}}{\pgfqpoint{1.967637in}{1.403736in}}%
\pgfpathclose%
\pgfusepath{stroke,fill}%
\end{pgfscope}%
\begin{pgfscope}%
\pgfpathrectangle{\pgfqpoint{0.100000in}{0.212622in}}{\pgfqpoint{3.696000in}{3.696000in}}%
\pgfusepath{clip}%
\pgfsetbuttcap%
\pgfsetroundjoin%
\definecolor{currentfill}{rgb}{0.121569,0.466667,0.705882}%
\pgfsetfillcolor{currentfill}%
\pgfsetfillopacity{0.977752}%
\pgfsetlinewidth{1.003750pt}%
\definecolor{currentstroke}{rgb}{0.121569,0.466667,0.705882}%
\pgfsetstrokecolor{currentstroke}%
\pgfsetstrokeopacity{0.977752}%
\pgfsetdash{}{0pt}%
\pgfpathmoveto{\pgfqpoint{1.978173in}{1.396180in}}%
\pgfpathcurveto{\pgfqpoint{1.986409in}{1.396180in}}{\pgfqpoint{1.994309in}{1.399452in}}{\pgfqpoint{2.000133in}{1.405276in}}%
\pgfpathcurveto{\pgfqpoint{2.005957in}{1.411100in}}{\pgfqpoint{2.009229in}{1.419000in}}{\pgfqpoint{2.009229in}{1.427237in}}%
\pgfpathcurveto{\pgfqpoint{2.009229in}{1.435473in}}{\pgfqpoint{2.005957in}{1.443373in}}{\pgfqpoint{2.000133in}{1.449197in}}%
\pgfpathcurveto{\pgfqpoint{1.994309in}{1.455021in}}{\pgfqpoint{1.986409in}{1.458293in}}{\pgfqpoint{1.978173in}{1.458293in}}%
\pgfpathcurveto{\pgfqpoint{1.969936in}{1.458293in}}{\pgfqpoint{1.962036in}{1.455021in}}{\pgfqpoint{1.956212in}{1.449197in}}%
\pgfpathcurveto{\pgfqpoint{1.950388in}{1.443373in}}{\pgfqpoint{1.947116in}{1.435473in}}{\pgfqpoint{1.947116in}{1.427237in}}%
\pgfpathcurveto{\pgfqpoint{1.947116in}{1.419000in}}{\pgfqpoint{1.950388in}{1.411100in}}{\pgfqpoint{1.956212in}{1.405276in}}%
\pgfpathcurveto{\pgfqpoint{1.962036in}{1.399452in}}{\pgfqpoint{1.969936in}{1.396180in}}{\pgfqpoint{1.978173in}{1.396180in}}%
\pgfpathclose%
\pgfusepath{stroke,fill}%
\end{pgfscope}%
\begin{pgfscope}%
\pgfpathrectangle{\pgfqpoint{0.100000in}{0.212622in}}{\pgfqpoint{3.696000in}{3.696000in}}%
\pgfusepath{clip}%
\pgfsetbuttcap%
\pgfsetroundjoin%
\definecolor{currentfill}{rgb}{0.121569,0.466667,0.705882}%
\pgfsetfillcolor{currentfill}%
\pgfsetfillopacity{0.977890}%
\pgfsetlinewidth{1.003750pt}%
\definecolor{currentstroke}{rgb}{0.121569,0.466667,0.705882}%
\pgfsetstrokecolor{currentstroke}%
\pgfsetstrokeopacity{0.977890}%
\pgfsetdash{}{0pt}%
\pgfpathmoveto{\pgfqpoint{2.533125in}{1.134821in}}%
\pgfpathcurveto{\pgfqpoint{2.541362in}{1.134821in}}{\pgfqpoint{2.549262in}{1.138093in}}{\pgfqpoint{2.555086in}{1.143917in}}%
\pgfpathcurveto{\pgfqpoint{2.560910in}{1.149741in}}{\pgfqpoint{2.564182in}{1.157641in}}{\pgfqpoint{2.564182in}{1.165877in}}%
\pgfpathcurveto{\pgfqpoint{2.564182in}{1.174114in}}{\pgfqpoint{2.560910in}{1.182014in}}{\pgfqpoint{2.555086in}{1.187838in}}%
\pgfpathcurveto{\pgfqpoint{2.549262in}{1.193661in}}{\pgfqpoint{2.541362in}{1.196934in}}{\pgfqpoint{2.533125in}{1.196934in}}%
\pgfpathcurveto{\pgfqpoint{2.524889in}{1.196934in}}{\pgfqpoint{2.516989in}{1.193661in}}{\pgfqpoint{2.511165in}{1.187838in}}%
\pgfpathcurveto{\pgfqpoint{2.505341in}{1.182014in}}{\pgfqpoint{2.502069in}{1.174114in}}{\pgfqpoint{2.502069in}{1.165877in}}%
\pgfpathcurveto{\pgfqpoint{2.502069in}{1.157641in}}{\pgfqpoint{2.505341in}{1.149741in}}{\pgfqpoint{2.511165in}{1.143917in}}%
\pgfpathcurveto{\pgfqpoint{2.516989in}{1.138093in}}{\pgfqpoint{2.524889in}{1.134821in}}{\pgfqpoint{2.533125in}{1.134821in}}%
\pgfpathclose%
\pgfusepath{stroke,fill}%
\end{pgfscope}%
\begin{pgfscope}%
\pgfpathrectangle{\pgfqpoint{0.100000in}{0.212622in}}{\pgfqpoint{3.696000in}{3.696000in}}%
\pgfusepath{clip}%
\pgfsetbuttcap%
\pgfsetroundjoin%
\definecolor{currentfill}{rgb}{0.121569,0.466667,0.705882}%
\pgfsetfillcolor{currentfill}%
\pgfsetfillopacity{0.978073}%
\pgfsetlinewidth{1.003750pt}%
\definecolor{currentstroke}{rgb}{0.121569,0.466667,0.705882}%
\pgfsetstrokecolor{currentstroke}%
\pgfsetstrokeopacity{0.978073}%
\pgfsetdash{}{0pt}%
\pgfpathmoveto{\pgfqpoint{1.989773in}{1.388523in}}%
\pgfpathcurveto{\pgfqpoint{1.998009in}{1.388523in}}{\pgfqpoint{2.005909in}{1.391795in}}{\pgfqpoint{2.011733in}{1.397619in}}%
\pgfpathcurveto{\pgfqpoint{2.017557in}{1.403443in}}{\pgfqpoint{2.020829in}{1.411343in}}{\pgfqpoint{2.020829in}{1.419579in}}%
\pgfpathcurveto{\pgfqpoint{2.020829in}{1.427815in}}{\pgfqpoint{2.017557in}{1.435716in}}{\pgfqpoint{2.011733in}{1.441539in}}%
\pgfpathcurveto{\pgfqpoint{2.005909in}{1.447363in}}{\pgfqpoint{1.998009in}{1.450636in}}{\pgfqpoint{1.989773in}{1.450636in}}%
\pgfpathcurveto{\pgfqpoint{1.981536in}{1.450636in}}{\pgfqpoint{1.973636in}{1.447363in}}{\pgfqpoint{1.967812in}{1.441539in}}%
\pgfpathcurveto{\pgfqpoint{1.961988in}{1.435716in}}{\pgfqpoint{1.958716in}{1.427815in}}{\pgfqpoint{1.958716in}{1.419579in}}%
\pgfpathcurveto{\pgfqpoint{1.958716in}{1.411343in}}{\pgfqpoint{1.961988in}{1.403443in}}{\pgfqpoint{1.967812in}{1.397619in}}%
\pgfpathcurveto{\pgfqpoint{1.973636in}{1.391795in}}{\pgfqpoint{1.981536in}{1.388523in}}{\pgfqpoint{1.989773in}{1.388523in}}%
\pgfpathclose%
\pgfusepath{stroke,fill}%
\end{pgfscope}%
\begin{pgfscope}%
\pgfpathrectangle{\pgfqpoint{0.100000in}{0.212622in}}{\pgfqpoint{3.696000in}{3.696000in}}%
\pgfusepath{clip}%
\pgfsetbuttcap%
\pgfsetroundjoin%
\definecolor{currentfill}{rgb}{0.121569,0.466667,0.705882}%
\pgfsetfillcolor{currentfill}%
\pgfsetfillopacity{0.978433}%
\pgfsetlinewidth{1.003750pt}%
\definecolor{currentstroke}{rgb}{0.121569,0.466667,0.705882}%
\pgfsetstrokecolor{currentstroke}%
\pgfsetstrokeopacity{0.978433}%
\pgfsetdash{}{0pt}%
\pgfpathmoveto{\pgfqpoint{2.002741in}{1.380459in}}%
\pgfpathcurveto{\pgfqpoint{2.010978in}{1.380459in}}{\pgfqpoint{2.018878in}{1.383731in}}{\pgfqpoint{2.024702in}{1.389555in}}%
\pgfpathcurveto{\pgfqpoint{2.030526in}{1.395379in}}{\pgfqpoint{2.033798in}{1.403279in}}{\pgfqpoint{2.033798in}{1.411515in}}%
\pgfpathcurveto{\pgfqpoint{2.033798in}{1.419752in}}{\pgfqpoint{2.030526in}{1.427652in}}{\pgfqpoint{2.024702in}{1.433476in}}%
\pgfpathcurveto{\pgfqpoint{2.018878in}{1.439299in}}{\pgfqpoint{2.010978in}{1.442572in}}{\pgfqpoint{2.002741in}{1.442572in}}%
\pgfpathcurveto{\pgfqpoint{1.994505in}{1.442572in}}{\pgfqpoint{1.986605in}{1.439299in}}{\pgfqpoint{1.980781in}{1.433476in}}%
\pgfpathcurveto{\pgfqpoint{1.974957in}{1.427652in}}{\pgfqpoint{1.971685in}{1.419752in}}{\pgfqpoint{1.971685in}{1.411515in}}%
\pgfpathcurveto{\pgfqpoint{1.971685in}{1.403279in}}{\pgfqpoint{1.974957in}{1.395379in}}{\pgfqpoint{1.980781in}{1.389555in}}%
\pgfpathcurveto{\pgfqpoint{1.986605in}{1.383731in}}{\pgfqpoint{1.994505in}{1.380459in}}{\pgfqpoint{2.002741in}{1.380459in}}%
\pgfpathclose%
\pgfusepath{stroke,fill}%
\end{pgfscope}%
\begin{pgfscope}%
\pgfpathrectangle{\pgfqpoint{0.100000in}{0.212622in}}{\pgfqpoint{3.696000in}{3.696000in}}%
\pgfusepath{clip}%
\pgfsetbuttcap%
\pgfsetroundjoin%
\definecolor{currentfill}{rgb}{0.121569,0.466667,0.705882}%
\pgfsetfillcolor{currentfill}%
\pgfsetfillopacity{0.978925}%
\pgfsetlinewidth{1.003750pt}%
\definecolor{currentstroke}{rgb}{0.121569,0.466667,0.705882}%
\pgfsetstrokecolor{currentstroke}%
\pgfsetstrokeopacity{0.978925}%
\pgfsetdash{}{0pt}%
\pgfpathmoveto{\pgfqpoint{2.016071in}{1.372819in}}%
\pgfpathcurveto{\pgfqpoint{2.024307in}{1.372819in}}{\pgfqpoint{2.032207in}{1.376091in}}{\pgfqpoint{2.038031in}{1.381915in}}%
\pgfpathcurveto{\pgfqpoint{2.043855in}{1.387739in}}{\pgfqpoint{2.047127in}{1.395639in}}{\pgfqpoint{2.047127in}{1.403875in}}%
\pgfpathcurveto{\pgfqpoint{2.047127in}{1.412111in}}{\pgfqpoint{2.043855in}{1.420011in}}{\pgfqpoint{2.038031in}{1.425835in}}%
\pgfpathcurveto{\pgfqpoint{2.032207in}{1.431659in}}{\pgfqpoint{2.024307in}{1.434932in}}{\pgfqpoint{2.016071in}{1.434932in}}%
\pgfpathcurveto{\pgfqpoint{2.007834in}{1.434932in}}{\pgfqpoint{1.999934in}{1.431659in}}{\pgfqpoint{1.994110in}{1.425835in}}%
\pgfpathcurveto{\pgfqpoint{1.988286in}{1.420011in}}{\pgfqpoint{1.985014in}{1.412111in}}{\pgfqpoint{1.985014in}{1.403875in}}%
\pgfpathcurveto{\pgfqpoint{1.985014in}{1.395639in}}{\pgfqpoint{1.988286in}{1.387739in}}{\pgfqpoint{1.994110in}{1.381915in}}%
\pgfpathcurveto{\pgfqpoint{1.999934in}{1.376091in}}{\pgfqpoint{2.007834in}{1.372819in}}{\pgfqpoint{2.016071in}{1.372819in}}%
\pgfpathclose%
\pgfusepath{stroke,fill}%
\end{pgfscope}%
\begin{pgfscope}%
\pgfpathrectangle{\pgfqpoint{0.100000in}{0.212622in}}{\pgfqpoint{3.696000in}{3.696000in}}%
\pgfusepath{clip}%
\pgfsetbuttcap%
\pgfsetroundjoin%
\definecolor{currentfill}{rgb}{0.121569,0.466667,0.705882}%
\pgfsetfillcolor{currentfill}%
\pgfsetfillopacity{0.979470}%
\pgfsetlinewidth{1.003750pt}%
\definecolor{currentstroke}{rgb}{0.121569,0.466667,0.705882}%
\pgfsetstrokecolor{currentstroke}%
\pgfsetstrokeopacity{0.979470}%
\pgfsetdash{}{0pt}%
\pgfpathmoveto{\pgfqpoint{2.030944in}{1.363588in}}%
\pgfpathcurveto{\pgfqpoint{2.039181in}{1.363588in}}{\pgfqpoint{2.047081in}{1.366860in}}{\pgfqpoint{2.052905in}{1.372684in}}%
\pgfpathcurveto{\pgfqpoint{2.058728in}{1.378508in}}{\pgfqpoint{2.062001in}{1.386408in}}{\pgfqpoint{2.062001in}{1.394645in}}%
\pgfpathcurveto{\pgfqpoint{2.062001in}{1.402881in}}{\pgfqpoint{2.058728in}{1.410781in}}{\pgfqpoint{2.052905in}{1.416605in}}%
\pgfpathcurveto{\pgfqpoint{2.047081in}{1.422429in}}{\pgfqpoint{2.039181in}{1.425701in}}{\pgfqpoint{2.030944in}{1.425701in}}%
\pgfpathcurveto{\pgfqpoint{2.022708in}{1.425701in}}{\pgfqpoint{2.014808in}{1.422429in}}{\pgfqpoint{2.008984in}{1.416605in}}%
\pgfpathcurveto{\pgfqpoint{2.003160in}{1.410781in}}{\pgfqpoint{1.999888in}{1.402881in}}{\pgfqpoint{1.999888in}{1.394645in}}%
\pgfpathcurveto{\pgfqpoint{1.999888in}{1.386408in}}{\pgfqpoint{2.003160in}{1.378508in}}{\pgfqpoint{2.008984in}{1.372684in}}%
\pgfpathcurveto{\pgfqpoint{2.014808in}{1.366860in}}{\pgfqpoint{2.022708in}{1.363588in}}{\pgfqpoint{2.030944in}{1.363588in}}%
\pgfpathclose%
\pgfusepath{stroke,fill}%
\end{pgfscope}%
\begin{pgfscope}%
\pgfpathrectangle{\pgfqpoint{0.100000in}{0.212622in}}{\pgfqpoint{3.696000in}{3.696000in}}%
\pgfusepath{clip}%
\pgfsetbuttcap%
\pgfsetroundjoin%
\definecolor{currentfill}{rgb}{0.121569,0.466667,0.705882}%
\pgfsetfillcolor{currentfill}%
\pgfsetfillopacity{0.979579}%
\pgfsetlinewidth{1.003750pt}%
\definecolor{currentstroke}{rgb}{0.121569,0.466667,0.705882}%
\pgfsetstrokecolor{currentstroke}%
\pgfsetstrokeopacity{0.979579}%
\pgfsetdash{}{0pt}%
\pgfpathmoveto{\pgfqpoint{2.529759in}{1.131349in}}%
\pgfpathcurveto{\pgfqpoint{2.537996in}{1.131349in}}{\pgfqpoint{2.545896in}{1.134621in}}{\pgfqpoint{2.551720in}{1.140445in}}%
\pgfpathcurveto{\pgfqpoint{2.557544in}{1.146269in}}{\pgfqpoint{2.560816in}{1.154169in}}{\pgfqpoint{2.560816in}{1.162405in}}%
\pgfpathcurveto{\pgfqpoint{2.560816in}{1.170642in}}{\pgfqpoint{2.557544in}{1.178542in}}{\pgfqpoint{2.551720in}{1.184366in}}%
\pgfpathcurveto{\pgfqpoint{2.545896in}{1.190190in}}{\pgfqpoint{2.537996in}{1.193462in}}{\pgfqpoint{2.529759in}{1.193462in}}%
\pgfpathcurveto{\pgfqpoint{2.521523in}{1.193462in}}{\pgfqpoint{2.513623in}{1.190190in}}{\pgfqpoint{2.507799in}{1.184366in}}%
\pgfpathcurveto{\pgfqpoint{2.501975in}{1.178542in}}{\pgfqpoint{2.498703in}{1.170642in}}{\pgfqpoint{2.498703in}{1.162405in}}%
\pgfpathcurveto{\pgfqpoint{2.498703in}{1.154169in}}{\pgfqpoint{2.501975in}{1.146269in}}{\pgfqpoint{2.507799in}{1.140445in}}%
\pgfpathcurveto{\pgfqpoint{2.513623in}{1.134621in}}{\pgfqpoint{2.521523in}{1.131349in}}{\pgfqpoint{2.529759in}{1.131349in}}%
\pgfpathclose%
\pgfusepath{stroke,fill}%
\end{pgfscope}%
\begin{pgfscope}%
\pgfpathrectangle{\pgfqpoint{0.100000in}{0.212622in}}{\pgfqpoint{3.696000in}{3.696000in}}%
\pgfusepath{clip}%
\pgfsetbuttcap%
\pgfsetroundjoin%
\definecolor{currentfill}{rgb}{0.121569,0.466667,0.705882}%
\pgfsetfillcolor{currentfill}%
\pgfsetfillopacity{0.980030}%
\pgfsetlinewidth{1.003750pt}%
\definecolor{currentstroke}{rgb}{0.121569,0.466667,0.705882}%
\pgfsetstrokecolor{currentstroke}%
\pgfsetstrokeopacity{0.980030}%
\pgfsetdash{}{0pt}%
\pgfpathmoveto{\pgfqpoint{2.047408in}{1.352656in}}%
\pgfpathcurveto{\pgfqpoint{2.055644in}{1.352656in}}{\pgfqpoint{2.063544in}{1.355928in}}{\pgfqpoint{2.069368in}{1.361752in}}%
\pgfpathcurveto{\pgfqpoint{2.075192in}{1.367576in}}{\pgfqpoint{2.078464in}{1.375476in}}{\pgfqpoint{2.078464in}{1.383712in}}%
\pgfpathcurveto{\pgfqpoint{2.078464in}{1.391949in}}{\pgfqpoint{2.075192in}{1.399849in}}{\pgfqpoint{2.069368in}{1.405673in}}%
\pgfpathcurveto{\pgfqpoint{2.063544in}{1.411497in}}{\pgfqpoint{2.055644in}{1.414769in}}{\pgfqpoint{2.047408in}{1.414769in}}%
\pgfpathcurveto{\pgfqpoint{2.039172in}{1.414769in}}{\pgfqpoint{2.031272in}{1.411497in}}{\pgfqpoint{2.025448in}{1.405673in}}%
\pgfpathcurveto{\pgfqpoint{2.019624in}{1.399849in}}{\pgfqpoint{2.016351in}{1.391949in}}{\pgfqpoint{2.016351in}{1.383712in}}%
\pgfpathcurveto{\pgfqpoint{2.016351in}{1.375476in}}{\pgfqpoint{2.019624in}{1.367576in}}{\pgfqpoint{2.025448in}{1.361752in}}%
\pgfpathcurveto{\pgfqpoint{2.031272in}{1.355928in}}{\pgfqpoint{2.039172in}{1.352656in}}{\pgfqpoint{2.047408in}{1.352656in}}%
\pgfpathclose%
\pgfusepath{stroke,fill}%
\end{pgfscope}%
\begin{pgfscope}%
\pgfpathrectangle{\pgfqpoint{0.100000in}{0.212622in}}{\pgfqpoint{3.696000in}{3.696000in}}%
\pgfusepath{clip}%
\pgfsetbuttcap%
\pgfsetroundjoin%
\definecolor{currentfill}{rgb}{0.121569,0.466667,0.705882}%
\pgfsetfillcolor{currentfill}%
\pgfsetfillopacity{0.980359}%
\pgfsetlinewidth{1.003750pt}%
\definecolor{currentstroke}{rgb}{0.121569,0.466667,0.705882}%
\pgfsetstrokecolor{currentstroke}%
\pgfsetstrokeopacity{0.980359}%
\pgfsetdash{}{0pt}%
\pgfpathmoveto{\pgfqpoint{2.056161in}{1.345604in}}%
\pgfpathcurveto{\pgfqpoint{2.064397in}{1.345604in}}{\pgfqpoint{2.072297in}{1.348876in}}{\pgfqpoint{2.078121in}{1.354700in}}%
\pgfpathcurveto{\pgfqpoint{2.083945in}{1.360524in}}{\pgfqpoint{2.087217in}{1.368424in}}{\pgfqpoint{2.087217in}{1.376660in}}%
\pgfpathcurveto{\pgfqpoint{2.087217in}{1.384896in}}{\pgfqpoint{2.083945in}{1.392797in}}{\pgfqpoint{2.078121in}{1.398620in}}%
\pgfpathcurveto{\pgfqpoint{2.072297in}{1.404444in}}{\pgfqpoint{2.064397in}{1.407717in}}{\pgfqpoint{2.056161in}{1.407717in}}%
\pgfpathcurveto{\pgfqpoint{2.047924in}{1.407717in}}{\pgfqpoint{2.040024in}{1.404444in}}{\pgfqpoint{2.034200in}{1.398620in}}%
\pgfpathcurveto{\pgfqpoint{2.028376in}{1.392797in}}{\pgfqpoint{2.025104in}{1.384896in}}{\pgfqpoint{2.025104in}{1.376660in}}%
\pgfpathcurveto{\pgfqpoint{2.025104in}{1.368424in}}{\pgfqpoint{2.028376in}{1.360524in}}{\pgfqpoint{2.034200in}{1.354700in}}%
\pgfpathcurveto{\pgfqpoint{2.040024in}{1.348876in}}{\pgfqpoint{2.047924in}{1.345604in}}{\pgfqpoint{2.056161in}{1.345604in}}%
\pgfpathclose%
\pgfusepath{stroke,fill}%
\end{pgfscope}%
\begin{pgfscope}%
\pgfpathrectangle{\pgfqpoint{0.100000in}{0.212622in}}{\pgfqpoint{3.696000in}{3.696000in}}%
\pgfusepath{clip}%
\pgfsetbuttcap%
\pgfsetroundjoin%
\definecolor{currentfill}{rgb}{0.121569,0.466667,0.705882}%
\pgfsetfillcolor{currentfill}%
\pgfsetfillopacity{0.980593}%
\pgfsetlinewidth{1.003750pt}%
\definecolor{currentstroke}{rgb}{0.121569,0.466667,0.705882}%
\pgfsetstrokecolor{currentstroke}%
\pgfsetstrokeopacity{0.980593}%
\pgfsetdash{}{0pt}%
\pgfpathmoveto{\pgfqpoint{2.060938in}{1.341885in}}%
\pgfpathcurveto{\pgfqpoint{2.069174in}{1.341885in}}{\pgfqpoint{2.077074in}{1.345157in}}{\pgfqpoint{2.082898in}{1.350981in}}%
\pgfpathcurveto{\pgfqpoint{2.088722in}{1.356805in}}{\pgfqpoint{2.091994in}{1.364705in}}{\pgfqpoint{2.091994in}{1.372941in}}%
\pgfpathcurveto{\pgfqpoint{2.091994in}{1.381178in}}{\pgfqpoint{2.088722in}{1.389078in}}{\pgfqpoint{2.082898in}{1.394901in}}%
\pgfpathcurveto{\pgfqpoint{2.077074in}{1.400725in}}{\pgfqpoint{2.069174in}{1.403998in}}{\pgfqpoint{2.060938in}{1.403998in}}%
\pgfpathcurveto{\pgfqpoint{2.052702in}{1.403998in}}{\pgfqpoint{2.044802in}{1.400725in}}{\pgfqpoint{2.038978in}{1.394901in}}%
\pgfpathcurveto{\pgfqpoint{2.033154in}{1.389078in}}{\pgfqpoint{2.029881in}{1.381178in}}{\pgfqpoint{2.029881in}{1.372941in}}%
\pgfpathcurveto{\pgfqpoint{2.029881in}{1.364705in}}{\pgfqpoint{2.033154in}{1.356805in}}{\pgfqpoint{2.038978in}{1.350981in}}%
\pgfpathcurveto{\pgfqpoint{2.044802in}{1.345157in}}{\pgfqpoint{2.052702in}{1.341885in}}{\pgfqpoint{2.060938in}{1.341885in}}%
\pgfpathclose%
\pgfusepath{stroke,fill}%
\end{pgfscope}%
\begin{pgfscope}%
\pgfpathrectangle{\pgfqpoint{0.100000in}{0.212622in}}{\pgfqpoint{3.696000in}{3.696000in}}%
\pgfusepath{clip}%
\pgfsetbuttcap%
\pgfsetroundjoin%
\definecolor{currentfill}{rgb}{0.121569,0.466667,0.705882}%
\pgfsetfillcolor{currentfill}%
\pgfsetfillopacity{0.980864}%
\pgfsetlinewidth{1.003750pt}%
\definecolor{currentstroke}{rgb}{0.121569,0.466667,0.705882}%
\pgfsetstrokecolor{currentstroke}%
\pgfsetstrokeopacity{0.980864}%
\pgfsetdash{}{0pt}%
\pgfpathmoveto{\pgfqpoint{2.066281in}{1.337873in}}%
\pgfpathcurveto{\pgfqpoint{2.074518in}{1.337873in}}{\pgfqpoint{2.082418in}{1.341145in}}{\pgfqpoint{2.088242in}{1.346969in}}%
\pgfpathcurveto{\pgfqpoint{2.094066in}{1.352793in}}{\pgfqpoint{2.097338in}{1.360693in}}{\pgfqpoint{2.097338in}{1.368930in}}%
\pgfpathcurveto{\pgfqpoint{2.097338in}{1.377166in}}{\pgfqpoint{2.094066in}{1.385066in}}{\pgfqpoint{2.088242in}{1.390890in}}%
\pgfpathcurveto{\pgfqpoint{2.082418in}{1.396714in}}{\pgfqpoint{2.074518in}{1.399986in}}{\pgfqpoint{2.066281in}{1.399986in}}%
\pgfpathcurveto{\pgfqpoint{2.058045in}{1.399986in}}{\pgfqpoint{2.050145in}{1.396714in}}{\pgfqpoint{2.044321in}{1.390890in}}%
\pgfpathcurveto{\pgfqpoint{2.038497in}{1.385066in}}{\pgfqpoint{2.035225in}{1.377166in}}{\pgfqpoint{2.035225in}{1.368930in}}%
\pgfpathcurveto{\pgfqpoint{2.035225in}{1.360693in}}{\pgfqpoint{2.038497in}{1.352793in}}{\pgfqpoint{2.044321in}{1.346969in}}%
\pgfpathcurveto{\pgfqpoint{2.050145in}{1.341145in}}{\pgfqpoint{2.058045in}{1.337873in}}{\pgfqpoint{2.066281in}{1.337873in}}%
\pgfpathclose%
\pgfusepath{stroke,fill}%
\end{pgfscope}%
\begin{pgfscope}%
\pgfpathrectangle{\pgfqpoint{0.100000in}{0.212622in}}{\pgfqpoint{3.696000in}{3.696000in}}%
\pgfusepath{clip}%
\pgfsetbuttcap%
\pgfsetroundjoin%
\definecolor{currentfill}{rgb}{0.121569,0.466667,0.705882}%
\pgfsetfillcolor{currentfill}%
\pgfsetfillopacity{0.980982}%
\pgfsetlinewidth{1.003750pt}%
\definecolor{currentstroke}{rgb}{0.121569,0.466667,0.705882}%
\pgfsetstrokecolor{currentstroke}%
\pgfsetstrokeopacity{0.980982}%
\pgfsetdash{}{0pt}%
\pgfpathmoveto{\pgfqpoint{2.527018in}{1.128501in}}%
\pgfpathcurveto{\pgfqpoint{2.535255in}{1.128501in}}{\pgfqpoint{2.543155in}{1.131774in}}{\pgfqpoint{2.548979in}{1.137598in}}%
\pgfpathcurveto{\pgfqpoint{2.554802in}{1.143422in}}{\pgfqpoint{2.558075in}{1.151322in}}{\pgfqpoint{2.558075in}{1.159558in}}%
\pgfpathcurveto{\pgfqpoint{2.558075in}{1.167794in}}{\pgfqpoint{2.554802in}{1.175694in}}{\pgfqpoint{2.548979in}{1.181518in}}%
\pgfpathcurveto{\pgfqpoint{2.543155in}{1.187342in}}{\pgfqpoint{2.535255in}{1.190614in}}{\pgfqpoint{2.527018in}{1.190614in}}%
\pgfpathcurveto{\pgfqpoint{2.518782in}{1.190614in}}{\pgfqpoint{2.510882in}{1.187342in}}{\pgfqpoint{2.505058in}{1.181518in}}%
\pgfpathcurveto{\pgfqpoint{2.499234in}{1.175694in}}{\pgfqpoint{2.495962in}{1.167794in}}{\pgfqpoint{2.495962in}{1.159558in}}%
\pgfpathcurveto{\pgfqpoint{2.495962in}{1.151322in}}{\pgfqpoint{2.499234in}{1.143422in}}{\pgfqpoint{2.505058in}{1.137598in}}%
\pgfpathcurveto{\pgfqpoint{2.510882in}{1.131774in}}{\pgfqpoint{2.518782in}{1.128501in}}{\pgfqpoint{2.527018in}{1.128501in}}%
\pgfpathclose%
\pgfusepath{stroke,fill}%
\end{pgfscope}%
\begin{pgfscope}%
\pgfpathrectangle{\pgfqpoint{0.100000in}{0.212622in}}{\pgfqpoint{3.696000in}{3.696000in}}%
\pgfusepath{clip}%
\pgfsetbuttcap%
\pgfsetroundjoin%
\definecolor{currentfill}{rgb}{0.121569,0.466667,0.705882}%
\pgfsetfillcolor{currentfill}%
\pgfsetfillopacity{0.981197}%
\pgfsetlinewidth{1.003750pt}%
\definecolor{currentstroke}{rgb}{0.121569,0.466667,0.705882}%
\pgfsetstrokecolor{currentstroke}%
\pgfsetstrokeopacity{0.981197}%
\pgfsetdash{}{0pt}%
\pgfpathmoveto{\pgfqpoint{2.072067in}{1.334044in}}%
\pgfpathcurveto{\pgfqpoint{2.080303in}{1.334044in}}{\pgfqpoint{2.088203in}{1.337316in}}{\pgfqpoint{2.094027in}{1.343140in}}%
\pgfpathcurveto{\pgfqpoint{2.099851in}{1.348964in}}{\pgfqpoint{2.103124in}{1.356864in}}{\pgfqpoint{2.103124in}{1.365100in}}%
\pgfpathcurveto{\pgfqpoint{2.103124in}{1.373336in}}{\pgfqpoint{2.099851in}{1.381236in}}{\pgfqpoint{2.094027in}{1.387060in}}%
\pgfpathcurveto{\pgfqpoint{2.088203in}{1.392884in}}{\pgfqpoint{2.080303in}{1.396157in}}{\pgfqpoint{2.072067in}{1.396157in}}%
\pgfpathcurveto{\pgfqpoint{2.063831in}{1.396157in}}{\pgfqpoint{2.055931in}{1.392884in}}{\pgfqpoint{2.050107in}{1.387060in}}%
\pgfpathcurveto{\pgfqpoint{2.044283in}{1.381236in}}{\pgfqpoint{2.041011in}{1.373336in}}{\pgfqpoint{2.041011in}{1.365100in}}%
\pgfpathcurveto{\pgfqpoint{2.041011in}{1.356864in}}{\pgfqpoint{2.044283in}{1.348964in}}{\pgfqpoint{2.050107in}{1.343140in}}%
\pgfpathcurveto{\pgfqpoint{2.055931in}{1.337316in}}{\pgfqpoint{2.063831in}{1.334044in}}{\pgfqpoint{2.072067in}{1.334044in}}%
\pgfpathclose%
\pgfusepath{stroke,fill}%
\end{pgfscope}%
\begin{pgfscope}%
\pgfpathrectangle{\pgfqpoint{0.100000in}{0.212622in}}{\pgfqpoint{3.696000in}{3.696000in}}%
\pgfusepath{clip}%
\pgfsetbuttcap%
\pgfsetroundjoin%
\definecolor{currentfill}{rgb}{0.121569,0.466667,0.705882}%
\pgfsetfillcolor{currentfill}%
\pgfsetfillopacity{0.981632}%
\pgfsetlinewidth{1.003750pt}%
\definecolor{currentstroke}{rgb}{0.121569,0.466667,0.705882}%
\pgfsetstrokecolor{currentstroke}%
\pgfsetstrokeopacity{0.981632}%
\pgfsetdash{}{0pt}%
\pgfpathmoveto{\pgfqpoint{2.079755in}{1.328612in}}%
\pgfpathcurveto{\pgfqpoint{2.087991in}{1.328612in}}{\pgfqpoint{2.095891in}{1.331884in}}{\pgfqpoint{2.101715in}{1.337708in}}%
\pgfpathcurveto{\pgfqpoint{2.107539in}{1.343532in}}{\pgfqpoint{2.110811in}{1.351432in}}{\pgfqpoint{2.110811in}{1.359668in}}%
\pgfpathcurveto{\pgfqpoint{2.110811in}{1.367904in}}{\pgfqpoint{2.107539in}{1.375805in}}{\pgfqpoint{2.101715in}{1.381628in}}%
\pgfpathcurveto{\pgfqpoint{2.095891in}{1.387452in}}{\pgfqpoint{2.087991in}{1.390725in}}{\pgfqpoint{2.079755in}{1.390725in}}%
\pgfpathcurveto{\pgfqpoint{2.071519in}{1.390725in}}{\pgfqpoint{2.063619in}{1.387452in}}{\pgfqpoint{2.057795in}{1.381628in}}%
\pgfpathcurveto{\pgfqpoint{2.051971in}{1.375805in}}{\pgfqpoint{2.048698in}{1.367904in}}{\pgfqpoint{2.048698in}{1.359668in}}%
\pgfpathcurveto{\pgfqpoint{2.048698in}{1.351432in}}{\pgfqpoint{2.051971in}{1.343532in}}{\pgfqpoint{2.057795in}{1.337708in}}%
\pgfpathcurveto{\pgfqpoint{2.063619in}{1.331884in}}{\pgfqpoint{2.071519in}{1.328612in}}{\pgfqpoint{2.079755in}{1.328612in}}%
\pgfpathclose%
\pgfusepath{stroke,fill}%
\end{pgfscope}%
\begin{pgfscope}%
\pgfpathrectangle{\pgfqpoint{0.100000in}{0.212622in}}{\pgfqpoint{3.696000in}{3.696000in}}%
\pgfusepath{clip}%
\pgfsetbuttcap%
\pgfsetroundjoin%
\definecolor{currentfill}{rgb}{0.121569,0.466667,0.705882}%
\pgfsetfillcolor{currentfill}%
\pgfsetfillopacity{0.981986}%
\pgfsetlinewidth{1.003750pt}%
\definecolor{currentstroke}{rgb}{0.121569,0.466667,0.705882}%
\pgfsetstrokecolor{currentstroke}%
\pgfsetstrokeopacity{0.981986}%
\pgfsetdash{}{0pt}%
\pgfpathmoveto{\pgfqpoint{2.525080in}{1.126374in}}%
\pgfpathcurveto{\pgfqpoint{2.533316in}{1.126374in}}{\pgfqpoint{2.541216in}{1.129646in}}{\pgfqpoint{2.547040in}{1.135470in}}%
\pgfpathcurveto{\pgfqpoint{2.552864in}{1.141294in}}{\pgfqpoint{2.556136in}{1.149194in}}{\pgfqpoint{2.556136in}{1.157430in}}%
\pgfpathcurveto{\pgfqpoint{2.556136in}{1.165666in}}{\pgfqpoint{2.552864in}{1.173566in}}{\pgfqpoint{2.547040in}{1.179390in}}%
\pgfpathcurveto{\pgfqpoint{2.541216in}{1.185214in}}{\pgfqpoint{2.533316in}{1.188487in}}{\pgfqpoint{2.525080in}{1.188487in}}%
\pgfpathcurveto{\pgfqpoint{2.516844in}{1.188487in}}{\pgfqpoint{2.508944in}{1.185214in}}{\pgfqpoint{2.503120in}{1.179390in}}%
\pgfpathcurveto{\pgfqpoint{2.497296in}{1.173566in}}{\pgfqpoint{2.494023in}{1.165666in}}{\pgfqpoint{2.494023in}{1.157430in}}%
\pgfpathcurveto{\pgfqpoint{2.494023in}{1.149194in}}{\pgfqpoint{2.497296in}{1.141294in}}{\pgfqpoint{2.503120in}{1.135470in}}%
\pgfpathcurveto{\pgfqpoint{2.508944in}{1.129646in}}{\pgfqpoint{2.516844in}{1.126374in}}{\pgfqpoint{2.525080in}{1.126374in}}%
\pgfpathclose%
\pgfusepath{stroke,fill}%
\end{pgfscope}%
\begin{pgfscope}%
\pgfpathrectangle{\pgfqpoint{0.100000in}{0.212622in}}{\pgfqpoint{3.696000in}{3.696000in}}%
\pgfusepath{clip}%
\pgfsetbuttcap%
\pgfsetroundjoin%
\definecolor{currentfill}{rgb}{0.121569,0.466667,0.705882}%
\pgfsetfillcolor{currentfill}%
\pgfsetfillopacity{0.982144}%
\pgfsetlinewidth{1.003750pt}%
\definecolor{currentstroke}{rgb}{0.121569,0.466667,0.705882}%
\pgfsetstrokecolor{currentstroke}%
\pgfsetstrokeopacity{0.982144}%
\pgfsetdash{}{0pt}%
\pgfpathmoveto{\pgfqpoint{2.089238in}{1.321482in}}%
\pgfpathcurveto{\pgfqpoint{2.097475in}{1.321482in}}{\pgfqpoint{2.105375in}{1.324754in}}{\pgfqpoint{2.111199in}{1.330578in}}%
\pgfpathcurveto{\pgfqpoint{2.117023in}{1.336402in}}{\pgfqpoint{2.120295in}{1.344302in}}{\pgfqpoint{2.120295in}{1.352538in}}%
\pgfpathcurveto{\pgfqpoint{2.120295in}{1.360775in}}{\pgfqpoint{2.117023in}{1.368675in}}{\pgfqpoint{2.111199in}{1.374499in}}%
\pgfpathcurveto{\pgfqpoint{2.105375in}{1.380322in}}{\pgfqpoint{2.097475in}{1.383595in}}{\pgfqpoint{2.089238in}{1.383595in}}%
\pgfpathcurveto{\pgfqpoint{2.081002in}{1.383595in}}{\pgfqpoint{2.073102in}{1.380322in}}{\pgfqpoint{2.067278in}{1.374499in}}%
\pgfpathcurveto{\pgfqpoint{2.061454in}{1.368675in}}{\pgfqpoint{2.058182in}{1.360775in}}{\pgfqpoint{2.058182in}{1.352538in}}%
\pgfpathcurveto{\pgfqpoint{2.058182in}{1.344302in}}{\pgfqpoint{2.061454in}{1.336402in}}{\pgfqpoint{2.067278in}{1.330578in}}%
\pgfpathcurveto{\pgfqpoint{2.073102in}{1.324754in}}{\pgfqpoint{2.081002in}{1.321482in}}{\pgfqpoint{2.089238in}{1.321482in}}%
\pgfpathclose%
\pgfusepath{stroke,fill}%
\end{pgfscope}%
\begin{pgfscope}%
\pgfpathrectangle{\pgfqpoint{0.100000in}{0.212622in}}{\pgfqpoint{3.696000in}{3.696000in}}%
\pgfusepath{clip}%
\pgfsetbuttcap%
\pgfsetroundjoin%
\definecolor{currentfill}{rgb}{0.121569,0.466667,0.705882}%
\pgfsetfillcolor{currentfill}%
\pgfsetfillopacity{0.982371}%
\pgfsetlinewidth{1.003750pt}%
\definecolor{currentstroke}{rgb}{0.121569,0.466667,0.705882}%
\pgfsetstrokecolor{currentstroke}%
\pgfsetstrokeopacity{0.982371}%
\pgfsetdash{}{0pt}%
\pgfpathmoveto{\pgfqpoint{2.094459in}{1.317277in}}%
\pgfpathcurveto{\pgfqpoint{2.102695in}{1.317277in}}{\pgfqpoint{2.110595in}{1.320549in}}{\pgfqpoint{2.116419in}{1.326373in}}%
\pgfpathcurveto{\pgfqpoint{2.122243in}{1.332197in}}{\pgfqpoint{2.125515in}{1.340097in}}{\pgfqpoint{2.125515in}{1.348333in}}%
\pgfpathcurveto{\pgfqpoint{2.125515in}{1.356569in}}{\pgfqpoint{2.122243in}{1.364470in}}{\pgfqpoint{2.116419in}{1.370293in}}%
\pgfpathcurveto{\pgfqpoint{2.110595in}{1.376117in}}{\pgfqpoint{2.102695in}{1.379390in}}{\pgfqpoint{2.094459in}{1.379390in}}%
\pgfpathcurveto{\pgfqpoint{2.086222in}{1.379390in}}{\pgfqpoint{2.078322in}{1.376117in}}{\pgfqpoint{2.072498in}{1.370293in}}%
\pgfpathcurveto{\pgfqpoint{2.066675in}{1.364470in}}{\pgfqpoint{2.063402in}{1.356569in}}{\pgfqpoint{2.063402in}{1.348333in}}%
\pgfpathcurveto{\pgfqpoint{2.063402in}{1.340097in}}{\pgfqpoint{2.066675in}{1.332197in}}{\pgfqpoint{2.072498in}{1.326373in}}%
\pgfpathcurveto{\pgfqpoint{2.078322in}{1.320549in}}{\pgfqpoint{2.086222in}{1.317277in}}{\pgfqpoint{2.094459in}{1.317277in}}%
\pgfpathclose%
\pgfusepath{stroke,fill}%
\end{pgfscope}%
\begin{pgfscope}%
\pgfpathrectangle{\pgfqpoint{0.100000in}{0.212622in}}{\pgfqpoint{3.696000in}{3.696000in}}%
\pgfusepath{clip}%
\pgfsetbuttcap%
\pgfsetroundjoin%
\definecolor{currentfill}{rgb}{0.121569,0.466667,0.705882}%
\pgfsetfillcolor{currentfill}%
\pgfsetfillopacity{0.982605}%
\pgfsetlinewidth{1.003750pt}%
\definecolor{currentstroke}{rgb}{0.121569,0.466667,0.705882}%
\pgfsetstrokecolor{currentstroke}%
\pgfsetstrokeopacity{0.982605}%
\pgfsetdash{}{0pt}%
\pgfpathmoveto{\pgfqpoint{2.100152in}{1.313009in}}%
\pgfpathcurveto{\pgfqpoint{2.108388in}{1.313009in}}{\pgfqpoint{2.116288in}{1.316281in}}{\pgfqpoint{2.122112in}{1.322105in}}%
\pgfpathcurveto{\pgfqpoint{2.127936in}{1.327929in}}{\pgfqpoint{2.131209in}{1.335829in}}{\pgfqpoint{2.131209in}{1.344066in}}%
\pgfpathcurveto{\pgfqpoint{2.131209in}{1.352302in}}{\pgfqpoint{2.127936in}{1.360202in}}{\pgfqpoint{2.122112in}{1.366026in}}%
\pgfpathcurveto{\pgfqpoint{2.116288in}{1.371850in}}{\pgfqpoint{2.108388in}{1.375122in}}{\pgfqpoint{2.100152in}{1.375122in}}%
\pgfpathcurveto{\pgfqpoint{2.091916in}{1.375122in}}{\pgfqpoint{2.084016in}{1.371850in}}{\pgfqpoint{2.078192in}{1.366026in}}%
\pgfpathcurveto{\pgfqpoint{2.072368in}{1.360202in}}{\pgfqpoint{2.069096in}{1.352302in}}{\pgfqpoint{2.069096in}{1.344066in}}%
\pgfpathcurveto{\pgfqpoint{2.069096in}{1.335829in}}{\pgfqpoint{2.072368in}{1.327929in}}{\pgfqpoint{2.078192in}{1.322105in}}%
\pgfpathcurveto{\pgfqpoint{2.084016in}{1.316281in}}{\pgfqpoint{2.091916in}{1.313009in}}{\pgfqpoint{2.100152in}{1.313009in}}%
\pgfpathclose%
\pgfusepath{stroke,fill}%
\end{pgfscope}%
\begin{pgfscope}%
\pgfpathrectangle{\pgfqpoint{0.100000in}{0.212622in}}{\pgfqpoint{3.696000in}{3.696000in}}%
\pgfusepath{clip}%
\pgfsetbuttcap%
\pgfsetroundjoin%
\definecolor{currentfill}{rgb}{0.121569,0.466667,0.705882}%
\pgfsetfillcolor{currentfill}%
\pgfsetfillopacity{0.982885}%
\pgfsetlinewidth{1.003750pt}%
\definecolor{currentstroke}{rgb}{0.121569,0.466667,0.705882}%
\pgfsetstrokecolor{currentstroke}%
\pgfsetstrokeopacity{0.982885}%
\pgfsetdash{}{0pt}%
\pgfpathmoveto{\pgfqpoint{2.107663in}{1.307678in}}%
\pgfpathcurveto{\pgfqpoint{2.115899in}{1.307678in}}{\pgfqpoint{2.123799in}{1.310950in}}{\pgfqpoint{2.129623in}{1.316774in}}%
\pgfpathcurveto{\pgfqpoint{2.135447in}{1.322598in}}{\pgfqpoint{2.138719in}{1.330498in}}{\pgfqpoint{2.138719in}{1.338735in}}%
\pgfpathcurveto{\pgfqpoint{2.138719in}{1.346971in}}{\pgfqpoint{2.135447in}{1.354871in}}{\pgfqpoint{2.129623in}{1.360695in}}%
\pgfpathcurveto{\pgfqpoint{2.123799in}{1.366519in}}{\pgfqpoint{2.115899in}{1.369791in}}{\pgfqpoint{2.107663in}{1.369791in}}%
\pgfpathcurveto{\pgfqpoint{2.099426in}{1.369791in}}{\pgfqpoint{2.091526in}{1.366519in}}{\pgfqpoint{2.085702in}{1.360695in}}%
\pgfpathcurveto{\pgfqpoint{2.079878in}{1.354871in}}{\pgfqpoint{2.076606in}{1.346971in}}{\pgfqpoint{2.076606in}{1.338735in}}%
\pgfpathcurveto{\pgfqpoint{2.076606in}{1.330498in}}{\pgfqpoint{2.079878in}{1.322598in}}{\pgfqpoint{2.085702in}{1.316774in}}%
\pgfpathcurveto{\pgfqpoint{2.091526in}{1.310950in}}{\pgfqpoint{2.099426in}{1.307678in}}{\pgfqpoint{2.107663in}{1.307678in}}%
\pgfpathclose%
\pgfusepath{stroke,fill}%
\end{pgfscope}%
\begin{pgfscope}%
\pgfpathrectangle{\pgfqpoint{0.100000in}{0.212622in}}{\pgfqpoint{3.696000in}{3.696000in}}%
\pgfusepath{clip}%
\pgfsetbuttcap%
\pgfsetroundjoin%
\definecolor{currentfill}{rgb}{0.121569,0.466667,0.705882}%
\pgfsetfillcolor{currentfill}%
\pgfsetfillopacity{0.983284}%
\pgfsetlinewidth{1.003750pt}%
\definecolor{currentstroke}{rgb}{0.121569,0.466667,0.705882}%
\pgfsetstrokecolor{currentstroke}%
\pgfsetstrokeopacity{0.983284}%
\pgfsetdash{}{0pt}%
\pgfpathmoveto{\pgfqpoint{2.115740in}{1.302322in}}%
\pgfpathcurveto{\pgfqpoint{2.123976in}{1.302322in}}{\pgfqpoint{2.131876in}{1.305595in}}{\pgfqpoint{2.137700in}{1.311419in}}%
\pgfpathcurveto{\pgfqpoint{2.143524in}{1.317243in}}{\pgfqpoint{2.146796in}{1.325143in}}{\pgfqpoint{2.146796in}{1.333379in}}%
\pgfpathcurveto{\pgfqpoint{2.146796in}{1.341615in}}{\pgfqpoint{2.143524in}{1.349515in}}{\pgfqpoint{2.137700in}{1.355339in}}%
\pgfpathcurveto{\pgfqpoint{2.131876in}{1.361163in}}{\pgfqpoint{2.123976in}{1.364435in}}{\pgfqpoint{2.115740in}{1.364435in}}%
\pgfpathcurveto{\pgfqpoint{2.107503in}{1.364435in}}{\pgfqpoint{2.099603in}{1.361163in}}{\pgfqpoint{2.093779in}{1.355339in}}%
\pgfpathcurveto{\pgfqpoint{2.087956in}{1.349515in}}{\pgfqpoint{2.084683in}{1.341615in}}{\pgfqpoint{2.084683in}{1.333379in}}%
\pgfpathcurveto{\pgfqpoint{2.084683in}{1.325143in}}{\pgfqpoint{2.087956in}{1.317243in}}{\pgfqpoint{2.093779in}{1.311419in}}%
\pgfpathcurveto{\pgfqpoint{2.099603in}{1.305595in}}{\pgfqpoint{2.107503in}{1.302322in}}{\pgfqpoint{2.115740in}{1.302322in}}%
\pgfpathclose%
\pgfusepath{stroke,fill}%
\end{pgfscope}%
\begin{pgfscope}%
\pgfpathrectangle{\pgfqpoint{0.100000in}{0.212622in}}{\pgfqpoint{3.696000in}{3.696000in}}%
\pgfusepath{clip}%
\pgfsetbuttcap%
\pgfsetroundjoin%
\definecolor{currentfill}{rgb}{0.121569,0.466667,0.705882}%
\pgfsetfillcolor{currentfill}%
\pgfsetfillopacity{0.983573}%
\pgfsetlinewidth{1.003750pt}%
\definecolor{currentstroke}{rgb}{0.121569,0.466667,0.705882}%
\pgfsetstrokecolor{currentstroke}%
\pgfsetstrokeopacity{0.983573}%
\pgfsetdash{}{0pt}%
\pgfpathmoveto{\pgfqpoint{2.120048in}{1.299266in}}%
\pgfpathcurveto{\pgfqpoint{2.128285in}{1.299266in}}{\pgfqpoint{2.136185in}{1.302538in}}{\pgfqpoint{2.142009in}{1.308362in}}%
\pgfpathcurveto{\pgfqpoint{2.147833in}{1.314186in}}{\pgfqpoint{2.151105in}{1.322086in}}{\pgfqpoint{2.151105in}{1.330322in}}%
\pgfpathcurveto{\pgfqpoint{2.151105in}{1.338558in}}{\pgfqpoint{2.147833in}{1.346458in}}{\pgfqpoint{2.142009in}{1.352282in}}%
\pgfpathcurveto{\pgfqpoint{2.136185in}{1.358106in}}{\pgfqpoint{2.128285in}{1.361379in}}{\pgfqpoint{2.120048in}{1.361379in}}%
\pgfpathcurveto{\pgfqpoint{2.111812in}{1.361379in}}{\pgfqpoint{2.103912in}{1.358106in}}{\pgfqpoint{2.098088in}{1.352282in}}%
\pgfpathcurveto{\pgfqpoint{2.092264in}{1.346458in}}{\pgfqpoint{2.088992in}{1.338558in}}{\pgfqpoint{2.088992in}{1.330322in}}%
\pgfpathcurveto{\pgfqpoint{2.088992in}{1.322086in}}{\pgfqpoint{2.092264in}{1.314186in}}{\pgfqpoint{2.098088in}{1.308362in}}%
\pgfpathcurveto{\pgfqpoint{2.103912in}{1.302538in}}{\pgfqpoint{2.111812in}{1.299266in}}{\pgfqpoint{2.120048in}{1.299266in}}%
\pgfpathclose%
\pgfusepath{stroke,fill}%
\end{pgfscope}%
\begin{pgfscope}%
\pgfpathrectangle{\pgfqpoint{0.100000in}{0.212622in}}{\pgfqpoint{3.696000in}{3.696000in}}%
\pgfusepath{clip}%
\pgfsetbuttcap%
\pgfsetroundjoin%
\definecolor{currentfill}{rgb}{0.121569,0.466667,0.705882}%
\pgfsetfillcolor{currentfill}%
\pgfsetfillopacity{0.983711}%
\pgfsetlinewidth{1.003750pt}%
\definecolor{currentstroke}{rgb}{0.121569,0.466667,0.705882}%
\pgfsetstrokecolor{currentstroke}%
\pgfsetstrokeopacity{0.983711}%
\pgfsetdash{}{0pt}%
\pgfpathmoveto{\pgfqpoint{2.521718in}{1.121797in}}%
\pgfpathcurveto{\pgfqpoint{2.529954in}{1.121797in}}{\pgfqpoint{2.537854in}{1.125070in}}{\pgfqpoint{2.543678in}{1.130894in}}%
\pgfpathcurveto{\pgfqpoint{2.549502in}{1.136717in}}{\pgfqpoint{2.552774in}{1.144618in}}{\pgfqpoint{2.552774in}{1.152854in}}%
\pgfpathcurveto{\pgfqpoint{2.552774in}{1.161090in}}{\pgfqpoint{2.549502in}{1.168990in}}{\pgfqpoint{2.543678in}{1.174814in}}%
\pgfpathcurveto{\pgfqpoint{2.537854in}{1.180638in}}{\pgfqpoint{2.529954in}{1.183910in}}{\pgfqpoint{2.521718in}{1.183910in}}%
\pgfpathcurveto{\pgfqpoint{2.513481in}{1.183910in}}{\pgfqpoint{2.505581in}{1.180638in}}{\pgfqpoint{2.499757in}{1.174814in}}%
\pgfpathcurveto{\pgfqpoint{2.493933in}{1.168990in}}{\pgfqpoint{2.490661in}{1.161090in}}{\pgfqpoint{2.490661in}{1.152854in}}%
\pgfpathcurveto{\pgfqpoint{2.490661in}{1.144618in}}{\pgfqpoint{2.493933in}{1.136717in}}{\pgfqpoint{2.499757in}{1.130894in}}%
\pgfpathcurveto{\pgfqpoint{2.505581in}{1.125070in}}{\pgfqpoint{2.513481in}{1.121797in}}{\pgfqpoint{2.521718in}{1.121797in}}%
\pgfpathclose%
\pgfusepath{stroke,fill}%
\end{pgfscope}%
\begin{pgfscope}%
\pgfpathrectangle{\pgfqpoint{0.100000in}{0.212622in}}{\pgfqpoint{3.696000in}{3.696000in}}%
\pgfusepath{clip}%
\pgfsetbuttcap%
\pgfsetroundjoin%
\definecolor{currentfill}{rgb}{0.121569,0.466667,0.705882}%
\pgfsetfillcolor{currentfill}%
\pgfsetfillopacity{0.983971}%
\pgfsetlinewidth{1.003750pt}%
\definecolor{currentstroke}{rgb}{0.121569,0.466667,0.705882}%
\pgfsetstrokecolor{currentstroke}%
\pgfsetstrokeopacity{0.983971}%
\pgfsetdash{}{0pt}%
\pgfpathmoveto{\pgfqpoint{2.126383in}{1.294493in}}%
\pgfpathcurveto{\pgfqpoint{2.134619in}{1.294493in}}{\pgfqpoint{2.142519in}{1.297765in}}{\pgfqpoint{2.148343in}{1.303589in}}%
\pgfpathcurveto{\pgfqpoint{2.154167in}{1.309413in}}{\pgfqpoint{2.157439in}{1.317313in}}{\pgfqpoint{2.157439in}{1.325549in}}%
\pgfpathcurveto{\pgfqpoint{2.157439in}{1.333785in}}{\pgfqpoint{2.154167in}{1.341685in}}{\pgfqpoint{2.148343in}{1.347509in}}%
\pgfpathcurveto{\pgfqpoint{2.142519in}{1.353333in}}{\pgfqpoint{2.134619in}{1.356606in}}{\pgfqpoint{2.126383in}{1.356606in}}%
\pgfpathcurveto{\pgfqpoint{2.118146in}{1.356606in}}{\pgfqpoint{2.110246in}{1.353333in}}{\pgfqpoint{2.104422in}{1.347509in}}%
\pgfpathcurveto{\pgfqpoint{2.098598in}{1.341685in}}{\pgfqpoint{2.095326in}{1.333785in}}{\pgfqpoint{2.095326in}{1.325549in}}%
\pgfpathcurveto{\pgfqpoint{2.095326in}{1.317313in}}{\pgfqpoint{2.098598in}{1.309413in}}{\pgfqpoint{2.104422in}{1.303589in}}%
\pgfpathcurveto{\pgfqpoint{2.110246in}{1.297765in}}{\pgfqpoint{2.118146in}{1.294493in}}{\pgfqpoint{2.126383in}{1.294493in}}%
\pgfpathclose%
\pgfusepath{stroke,fill}%
\end{pgfscope}%
\begin{pgfscope}%
\pgfpathrectangle{\pgfqpoint{0.100000in}{0.212622in}}{\pgfqpoint{3.696000in}{3.696000in}}%
\pgfusepath{clip}%
\pgfsetbuttcap%
\pgfsetroundjoin%
\definecolor{currentfill}{rgb}{0.121569,0.466667,0.705882}%
\pgfsetfillcolor{currentfill}%
\pgfsetfillopacity{0.984340}%
\pgfsetlinewidth{1.003750pt}%
\definecolor{currentstroke}{rgb}{0.121569,0.466667,0.705882}%
\pgfsetstrokecolor{currentstroke}%
\pgfsetstrokeopacity{0.984340}%
\pgfsetdash{}{0pt}%
\pgfpathmoveto{\pgfqpoint{2.134060in}{1.288247in}}%
\pgfpathcurveto{\pgfqpoint{2.142297in}{1.288247in}}{\pgfqpoint{2.150197in}{1.291519in}}{\pgfqpoint{2.156021in}{1.297343in}}%
\pgfpathcurveto{\pgfqpoint{2.161845in}{1.303167in}}{\pgfqpoint{2.165117in}{1.311067in}}{\pgfqpoint{2.165117in}{1.319304in}}%
\pgfpathcurveto{\pgfqpoint{2.165117in}{1.327540in}}{\pgfqpoint{2.161845in}{1.335440in}}{\pgfqpoint{2.156021in}{1.341264in}}%
\pgfpathcurveto{\pgfqpoint{2.150197in}{1.347088in}}{\pgfqpoint{2.142297in}{1.350360in}}{\pgfqpoint{2.134060in}{1.350360in}}%
\pgfpathcurveto{\pgfqpoint{2.125824in}{1.350360in}}{\pgfqpoint{2.117924in}{1.347088in}}{\pgfqpoint{2.112100in}{1.341264in}}%
\pgfpathcurveto{\pgfqpoint{2.106276in}{1.335440in}}{\pgfqpoint{2.103004in}{1.327540in}}{\pgfqpoint{2.103004in}{1.319304in}}%
\pgfpathcurveto{\pgfqpoint{2.103004in}{1.311067in}}{\pgfqpoint{2.106276in}{1.303167in}}{\pgfqpoint{2.112100in}{1.297343in}}%
\pgfpathcurveto{\pgfqpoint{2.117924in}{1.291519in}}{\pgfqpoint{2.125824in}{1.288247in}}{\pgfqpoint{2.134060in}{1.288247in}}%
\pgfpathclose%
\pgfusepath{stroke,fill}%
\end{pgfscope}%
\begin{pgfscope}%
\pgfpathrectangle{\pgfqpoint{0.100000in}{0.212622in}}{\pgfqpoint{3.696000in}{3.696000in}}%
\pgfusepath{clip}%
\pgfsetbuttcap%
\pgfsetroundjoin%
\definecolor{currentfill}{rgb}{0.121569,0.466667,0.705882}%
\pgfsetfillcolor{currentfill}%
\pgfsetfillopacity{0.984868}%
\pgfsetlinewidth{1.003750pt}%
\definecolor{currentstroke}{rgb}{0.121569,0.466667,0.705882}%
\pgfsetstrokecolor{currentstroke}%
\pgfsetstrokeopacity{0.984868}%
\pgfsetdash{}{0pt}%
\pgfpathmoveto{\pgfqpoint{2.142981in}{1.281478in}}%
\pgfpathcurveto{\pgfqpoint{2.151218in}{1.281478in}}{\pgfqpoint{2.159118in}{1.284750in}}{\pgfqpoint{2.164941in}{1.290574in}}%
\pgfpathcurveto{\pgfqpoint{2.170765in}{1.296398in}}{\pgfqpoint{2.174038in}{1.304298in}}{\pgfqpoint{2.174038in}{1.312535in}}%
\pgfpathcurveto{\pgfqpoint{2.174038in}{1.320771in}}{\pgfqpoint{2.170765in}{1.328671in}}{\pgfqpoint{2.164941in}{1.334495in}}%
\pgfpathcurveto{\pgfqpoint{2.159118in}{1.340319in}}{\pgfqpoint{2.151218in}{1.343591in}}{\pgfqpoint{2.142981in}{1.343591in}}%
\pgfpathcurveto{\pgfqpoint{2.134745in}{1.343591in}}{\pgfqpoint{2.126845in}{1.340319in}}{\pgfqpoint{2.121021in}{1.334495in}}%
\pgfpathcurveto{\pgfqpoint{2.115197in}{1.328671in}}{\pgfqpoint{2.111925in}{1.320771in}}{\pgfqpoint{2.111925in}{1.312535in}}%
\pgfpathcurveto{\pgfqpoint{2.111925in}{1.304298in}}{\pgfqpoint{2.115197in}{1.296398in}}{\pgfqpoint{2.121021in}{1.290574in}}%
\pgfpathcurveto{\pgfqpoint{2.126845in}{1.284750in}}{\pgfqpoint{2.134745in}{1.281478in}}{\pgfqpoint{2.142981in}{1.281478in}}%
\pgfpathclose%
\pgfusepath{stroke,fill}%
\end{pgfscope}%
\begin{pgfscope}%
\pgfpathrectangle{\pgfqpoint{0.100000in}{0.212622in}}{\pgfqpoint{3.696000in}{3.696000in}}%
\pgfusepath{clip}%
\pgfsetbuttcap%
\pgfsetroundjoin%
\definecolor{currentfill}{rgb}{0.121569,0.466667,0.705882}%
\pgfsetfillcolor{currentfill}%
\pgfsetfillopacity{0.985100}%
\pgfsetlinewidth{1.003750pt}%
\definecolor{currentstroke}{rgb}{0.121569,0.466667,0.705882}%
\pgfsetstrokecolor{currentstroke}%
\pgfsetstrokeopacity{0.985100}%
\pgfsetdash{}{0pt}%
\pgfpathmoveto{\pgfqpoint{2.518926in}{1.117695in}}%
\pgfpathcurveto{\pgfqpoint{2.527162in}{1.117695in}}{\pgfqpoint{2.535062in}{1.120967in}}{\pgfqpoint{2.540886in}{1.126791in}}%
\pgfpathcurveto{\pgfqpoint{2.546710in}{1.132615in}}{\pgfqpoint{2.549983in}{1.140515in}}{\pgfqpoint{2.549983in}{1.148751in}}%
\pgfpathcurveto{\pgfqpoint{2.549983in}{1.156987in}}{\pgfqpoint{2.546710in}{1.164887in}}{\pgfqpoint{2.540886in}{1.170711in}}%
\pgfpathcurveto{\pgfqpoint{2.535062in}{1.176535in}}{\pgfqpoint{2.527162in}{1.179808in}}{\pgfqpoint{2.518926in}{1.179808in}}%
\pgfpathcurveto{\pgfqpoint{2.510690in}{1.179808in}}{\pgfqpoint{2.502790in}{1.176535in}}{\pgfqpoint{2.496966in}{1.170711in}}%
\pgfpathcurveto{\pgfqpoint{2.491142in}{1.164887in}}{\pgfqpoint{2.487870in}{1.156987in}}{\pgfqpoint{2.487870in}{1.148751in}}%
\pgfpathcurveto{\pgfqpoint{2.487870in}{1.140515in}}{\pgfqpoint{2.491142in}{1.132615in}}{\pgfqpoint{2.496966in}{1.126791in}}%
\pgfpathcurveto{\pgfqpoint{2.502790in}{1.120967in}}{\pgfqpoint{2.510690in}{1.117695in}}{\pgfqpoint{2.518926in}{1.117695in}}%
\pgfpathclose%
\pgfusepath{stroke,fill}%
\end{pgfscope}%
\begin{pgfscope}%
\pgfpathrectangle{\pgfqpoint{0.100000in}{0.212622in}}{\pgfqpoint{3.696000in}{3.696000in}}%
\pgfusepath{clip}%
\pgfsetbuttcap%
\pgfsetroundjoin%
\definecolor{currentfill}{rgb}{0.121569,0.466667,0.705882}%
\pgfsetfillcolor{currentfill}%
\pgfsetfillopacity{0.985527}%
\pgfsetlinewidth{1.003750pt}%
\definecolor{currentstroke}{rgb}{0.121569,0.466667,0.705882}%
\pgfsetstrokecolor{currentstroke}%
\pgfsetstrokeopacity{0.985527}%
\pgfsetdash{}{0pt}%
\pgfpathmoveto{\pgfqpoint{2.154465in}{1.273100in}}%
\pgfpathcurveto{\pgfqpoint{2.162701in}{1.273100in}}{\pgfqpoint{2.170601in}{1.276373in}}{\pgfqpoint{2.176425in}{1.282196in}}%
\pgfpathcurveto{\pgfqpoint{2.182249in}{1.288020in}}{\pgfqpoint{2.185522in}{1.295920in}}{\pgfqpoint{2.185522in}{1.304157in}}%
\pgfpathcurveto{\pgfqpoint{2.185522in}{1.312393in}}{\pgfqpoint{2.182249in}{1.320293in}}{\pgfqpoint{2.176425in}{1.326117in}}%
\pgfpathcurveto{\pgfqpoint{2.170601in}{1.331941in}}{\pgfqpoint{2.162701in}{1.335213in}}{\pgfqpoint{2.154465in}{1.335213in}}%
\pgfpathcurveto{\pgfqpoint{2.146229in}{1.335213in}}{\pgfqpoint{2.138329in}{1.331941in}}{\pgfqpoint{2.132505in}{1.326117in}}%
\pgfpathcurveto{\pgfqpoint{2.126681in}{1.320293in}}{\pgfqpoint{2.123409in}{1.312393in}}{\pgfqpoint{2.123409in}{1.304157in}}%
\pgfpathcurveto{\pgfqpoint{2.123409in}{1.295920in}}{\pgfqpoint{2.126681in}{1.288020in}}{\pgfqpoint{2.132505in}{1.282196in}}%
\pgfpathcurveto{\pgfqpoint{2.138329in}{1.276373in}}{\pgfqpoint{2.146229in}{1.273100in}}{\pgfqpoint{2.154465in}{1.273100in}}%
\pgfpathclose%
\pgfusepath{stroke,fill}%
\end{pgfscope}%
\begin{pgfscope}%
\pgfpathrectangle{\pgfqpoint{0.100000in}{0.212622in}}{\pgfqpoint{3.696000in}{3.696000in}}%
\pgfusepath{clip}%
\pgfsetbuttcap%
\pgfsetroundjoin%
\definecolor{currentfill}{rgb}{0.121569,0.466667,0.705882}%
\pgfsetfillcolor{currentfill}%
\pgfsetfillopacity{0.986256}%
\pgfsetlinewidth{1.003750pt}%
\definecolor{currentstroke}{rgb}{0.121569,0.466667,0.705882}%
\pgfsetstrokecolor{currentstroke}%
\pgfsetstrokeopacity{0.986256}%
\pgfsetdash{}{0pt}%
\pgfpathmoveto{\pgfqpoint{2.516601in}{1.114439in}}%
\pgfpathcurveto{\pgfqpoint{2.524837in}{1.114439in}}{\pgfqpoint{2.532737in}{1.117711in}}{\pgfqpoint{2.538561in}{1.123535in}}%
\pgfpathcurveto{\pgfqpoint{2.544385in}{1.129359in}}{\pgfqpoint{2.547657in}{1.137259in}}{\pgfqpoint{2.547657in}{1.145496in}}%
\pgfpathcurveto{\pgfqpoint{2.547657in}{1.153732in}}{\pgfqpoint{2.544385in}{1.161632in}}{\pgfqpoint{2.538561in}{1.167456in}}%
\pgfpathcurveto{\pgfqpoint{2.532737in}{1.173280in}}{\pgfqpoint{2.524837in}{1.176552in}}{\pgfqpoint{2.516601in}{1.176552in}}%
\pgfpathcurveto{\pgfqpoint{2.508364in}{1.176552in}}{\pgfqpoint{2.500464in}{1.173280in}}{\pgfqpoint{2.494640in}{1.167456in}}%
\pgfpathcurveto{\pgfqpoint{2.488816in}{1.161632in}}{\pgfqpoint{2.485544in}{1.153732in}}{\pgfqpoint{2.485544in}{1.145496in}}%
\pgfpathcurveto{\pgfqpoint{2.485544in}{1.137259in}}{\pgfqpoint{2.488816in}{1.129359in}}{\pgfqpoint{2.494640in}{1.123535in}}%
\pgfpathcurveto{\pgfqpoint{2.500464in}{1.117711in}}{\pgfqpoint{2.508364in}{1.114439in}}{\pgfqpoint{2.516601in}{1.114439in}}%
\pgfpathclose%
\pgfusepath{stroke,fill}%
\end{pgfscope}%
\begin{pgfscope}%
\pgfpathrectangle{\pgfqpoint{0.100000in}{0.212622in}}{\pgfqpoint{3.696000in}{3.696000in}}%
\pgfusepath{clip}%
\pgfsetbuttcap%
\pgfsetroundjoin%
\definecolor{currentfill}{rgb}{0.121569,0.466667,0.705882}%
\pgfsetfillcolor{currentfill}%
\pgfsetfillopacity{0.986381}%
\pgfsetlinewidth{1.003750pt}%
\definecolor{currentstroke}{rgb}{0.121569,0.466667,0.705882}%
\pgfsetstrokecolor{currentstroke}%
\pgfsetstrokeopacity{0.986381}%
\pgfsetdash{}{0pt}%
\pgfpathmoveto{\pgfqpoint{2.168284in}{1.263752in}}%
\pgfpathcurveto{\pgfqpoint{2.176520in}{1.263752in}}{\pgfqpoint{2.184420in}{1.267025in}}{\pgfqpoint{2.190244in}{1.272849in}}%
\pgfpathcurveto{\pgfqpoint{2.196068in}{1.278673in}}{\pgfqpoint{2.199340in}{1.286573in}}{\pgfqpoint{2.199340in}{1.294809in}}%
\pgfpathcurveto{\pgfqpoint{2.199340in}{1.303045in}}{\pgfqpoint{2.196068in}{1.310945in}}{\pgfqpoint{2.190244in}{1.316769in}}%
\pgfpathcurveto{\pgfqpoint{2.184420in}{1.322593in}}{\pgfqpoint{2.176520in}{1.325865in}}{\pgfqpoint{2.168284in}{1.325865in}}%
\pgfpathcurveto{\pgfqpoint{2.160047in}{1.325865in}}{\pgfqpoint{2.152147in}{1.322593in}}{\pgfqpoint{2.146323in}{1.316769in}}%
\pgfpathcurveto{\pgfqpoint{2.140499in}{1.310945in}}{\pgfqpoint{2.137227in}{1.303045in}}{\pgfqpoint{2.137227in}{1.294809in}}%
\pgfpathcurveto{\pgfqpoint{2.137227in}{1.286573in}}{\pgfqpoint{2.140499in}{1.278673in}}{\pgfqpoint{2.146323in}{1.272849in}}%
\pgfpathcurveto{\pgfqpoint{2.152147in}{1.267025in}}{\pgfqpoint{2.160047in}{1.263752in}}{\pgfqpoint{2.168284in}{1.263752in}}%
\pgfpathclose%
\pgfusepath{stroke,fill}%
\end{pgfscope}%
\begin{pgfscope}%
\pgfpathrectangle{\pgfqpoint{0.100000in}{0.212622in}}{\pgfqpoint{3.696000in}{3.696000in}}%
\pgfusepath{clip}%
\pgfsetbuttcap%
\pgfsetroundjoin%
\definecolor{currentfill}{rgb}{0.121569,0.466667,0.705882}%
\pgfsetfillcolor{currentfill}%
\pgfsetfillopacity{0.986757}%
\pgfsetlinewidth{1.003750pt}%
\definecolor{currentstroke}{rgb}{0.121569,0.466667,0.705882}%
\pgfsetstrokecolor{currentstroke}%
\pgfsetstrokeopacity{0.986757}%
\pgfsetdash{}{0pt}%
\pgfpathmoveto{\pgfqpoint{2.175890in}{1.258115in}}%
\pgfpathcurveto{\pgfqpoint{2.184126in}{1.258115in}}{\pgfqpoint{2.192026in}{1.261388in}}{\pgfqpoint{2.197850in}{1.267212in}}%
\pgfpathcurveto{\pgfqpoint{2.203674in}{1.273035in}}{\pgfqpoint{2.206946in}{1.280935in}}{\pgfqpoint{2.206946in}{1.289172in}}%
\pgfpathcurveto{\pgfqpoint{2.206946in}{1.297408in}}{\pgfqpoint{2.203674in}{1.305308in}}{\pgfqpoint{2.197850in}{1.311132in}}%
\pgfpathcurveto{\pgfqpoint{2.192026in}{1.316956in}}{\pgfqpoint{2.184126in}{1.320228in}}{\pgfqpoint{2.175890in}{1.320228in}}%
\pgfpathcurveto{\pgfqpoint{2.167653in}{1.320228in}}{\pgfqpoint{2.159753in}{1.316956in}}{\pgfqpoint{2.153929in}{1.311132in}}%
\pgfpathcurveto{\pgfqpoint{2.148105in}{1.305308in}}{\pgfqpoint{2.144833in}{1.297408in}}{\pgfqpoint{2.144833in}{1.289172in}}%
\pgfpathcurveto{\pgfqpoint{2.144833in}{1.280935in}}{\pgfqpoint{2.148105in}{1.273035in}}{\pgfqpoint{2.153929in}{1.267212in}}%
\pgfpathcurveto{\pgfqpoint{2.159753in}{1.261388in}}{\pgfqpoint{2.167653in}{1.258115in}}{\pgfqpoint{2.175890in}{1.258115in}}%
\pgfpathclose%
\pgfusepath{stroke,fill}%
\end{pgfscope}%
\begin{pgfscope}%
\pgfpathrectangle{\pgfqpoint{0.100000in}{0.212622in}}{\pgfqpoint{3.696000in}{3.696000in}}%
\pgfusepath{clip}%
\pgfsetbuttcap%
\pgfsetroundjoin%
\definecolor{currentfill}{rgb}{0.121569,0.466667,0.705882}%
\pgfsetfillcolor{currentfill}%
\pgfsetfillopacity{0.987087}%
\pgfsetlinewidth{1.003750pt}%
\definecolor{currentstroke}{rgb}{0.121569,0.466667,0.705882}%
\pgfsetstrokecolor{currentstroke}%
\pgfsetstrokeopacity{0.987087}%
\pgfsetdash{}{0pt}%
\pgfpathmoveto{\pgfqpoint{2.514842in}{1.111933in}}%
\pgfpathcurveto{\pgfqpoint{2.523078in}{1.111933in}}{\pgfqpoint{2.530978in}{1.115205in}}{\pgfqpoint{2.536802in}{1.121029in}}%
\pgfpathcurveto{\pgfqpoint{2.542626in}{1.126853in}}{\pgfqpoint{2.545898in}{1.134753in}}{\pgfqpoint{2.545898in}{1.142989in}}%
\pgfpathcurveto{\pgfqpoint{2.545898in}{1.151226in}}{\pgfqpoint{2.542626in}{1.159126in}}{\pgfqpoint{2.536802in}{1.164950in}}%
\pgfpathcurveto{\pgfqpoint{2.530978in}{1.170774in}}{\pgfqpoint{2.523078in}{1.174046in}}{\pgfqpoint{2.514842in}{1.174046in}}%
\pgfpathcurveto{\pgfqpoint{2.506605in}{1.174046in}}{\pgfqpoint{2.498705in}{1.170774in}}{\pgfqpoint{2.492882in}{1.164950in}}%
\pgfpathcurveto{\pgfqpoint{2.487058in}{1.159126in}}{\pgfqpoint{2.483785in}{1.151226in}}{\pgfqpoint{2.483785in}{1.142989in}}%
\pgfpathcurveto{\pgfqpoint{2.483785in}{1.134753in}}{\pgfqpoint{2.487058in}{1.126853in}}{\pgfqpoint{2.492882in}{1.121029in}}%
\pgfpathcurveto{\pgfqpoint{2.498705in}{1.115205in}}{\pgfqpoint{2.506605in}{1.111933in}}{\pgfqpoint{2.514842in}{1.111933in}}%
\pgfpathclose%
\pgfusepath{stroke,fill}%
\end{pgfscope}%
\begin{pgfscope}%
\pgfpathrectangle{\pgfqpoint{0.100000in}{0.212622in}}{\pgfqpoint{3.696000in}{3.696000in}}%
\pgfusepath{clip}%
\pgfsetbuttcap%
\pgfsetroundjoin%
\definecolor{currentfill}{rgb}{0.121569,0.466667,0.705882}%
\pgfsetfillcolor{currentfill}%
\pgfsetfillopacity{0.987141}%
\pgfsetlinewidth{1.003750pt}%
\definecolor{currentstroke}{rgb}{0.121569,0.466667,0.705882}%
\pgfsetstrokecolor{currentstroke}%
\pgfsetstrokeopacity{0.987141}%
\pgfsetdash{}{0pt}%
\pgfpathmoveto{\pgfqpoint{2.184501in}{1.252065in}}%
\pgfpathcurveto{\pgfqpoint{2.192737in}{1.252065in}}{\pgfqpoint{2.200637in}{1.255337in}}{\pgfqpoint{2.206461in}{1.261161in}}%
\pgfpathcurveto{\pgfqpoint{2.212285in}{1.266985in}}{\pgfqpoint{2.215557in}{1.274885in}}{\pgfqpoint{2.215557in}{1.283121in}}%
\pgfpathcurveto{\pgfqpoint{2.215557in}{1.291357in}}{\pgfqpoint{2.212285in}{1.299258in}}{\pgfqpoint{2.206461in}{1.305081in}}%
\pgfpathcurveto{\pgfqpoint{2.200637in}{1.310905in}}{\pgfqpoint{2.192737in}{1.314178in}}{\pgfqpoint{2.184501in}{1.314178in}}%
\pgfpathcurveto{\pgfqpoint{2.176264in}{1.314178in}}{\pgfqpoint{2.168364in}{1.310905in}}{\pgfqpoint{2.162540in}{1.305081in}}%
\pgfpathcurveto{\pgfqpoint{2.156716in}{1.299258in}}{\pgfqpoint{2.153444in}{1.291357in}}{\pgfqpoint{2.153444in}{1.283121in}}%
\pgfpathcurveto{\pgfqpoint{2.153444in}{1.274885in}}{\pgfqpoint{2.156716in}{1.266985in}}{\pgfqpoint{2.162540in}{1.261161in}}%
\pgfpathcurveto{\pgfqpoint{2.168364in}{1.255337in}}{\pgfqpoint{2.176264in}{1.252065in}}{\pgfqpoint{2.184501in}{1.252065in}}%
\pgfpathclose%
\pgfusepath{stroke,fill}%
\end{pgfscope}%
\begin{pgfscope}%
\pgfpathrectangle{\pgfqpoint{0.100000in}{0.212622in}}{\pgfqpoint{3.696000in}{3.696000in}}%
\pgfusepath{clip}%
\pgfsetbuttcap%
\pgfsetroundjoin%
\definecolor{currentfill}{rgb}{0.121569,0.466667,0.705882}%
\pgfsetfillcolor{currentfill}%
\pgfsetfillopacity{0.987546}%
\pgfsetlinewidth{1.003750pt}%
\definecolor{currentstroke}{rgb}{0.121569,0.466667,0.705882}%
\pgfsetstrokecolor{currentstroke}%
\pgfsetstrokeopacity{0.987546}%
\pgfsetdash{}{0pt}%
\pgfpathmoveto{\pgfqpoint{2.193623in}{1.244804in}}%
\pgfpathcurveto{\pgfqpoint{2.201860in}{1.244804in}}{\pgfqpoint{2.209760in}{1.248077in}}{\pgfqpoint{2.215584in}{1.253900in}}%
\pgfpathcurveto{\pgfqpoint{2.221407in}{1.259724in}}{\pgfqpoint{2.224680in}{1.267624in}}{\pgfqpoint{2.224680in}{1.275861in}}%
\pgfpathcurveto{\pgfqpoint{2.224680in}{1.284097in}}{\pgfqpoint{2.221407in}{1.291997in}}{\pgfqpoint{2.215584in}{1.297821in}}%
\pgfpathcurveto{\pgfqpoint{2.209760in}{1.303645in}}{\pgfqpoint{2.201860in}{1.306917in}}{\pgfqpoint{2.193623in}{1.306917in}}%
\pgfpathcurveto{\pgfqpoint{2.185387in}{1.306917in}}{\pgfqpoint{2.177487in}{1.303645in}}{\pgfqpoint{2.171663in}{1.297821in}}%
\pgfpathcurveto{\pgfqpoint{2.165839in}{1.291997in}}{\pgfqpoint{2.162567in}{1.284097in}}{\pgfqpoint{2.162567in}{1.275861in}}%
\pgfpathcurveto{\pgfqpoint{2.162567in}{1.267624in}}{\pgfqpoint{2.165839in}{1.259724in}}{\pgfqpoint{2.171663in}{1.253900in}}%
\pgfpathcurveto{\pgfqpoint{2.177487in}{1.248077in}}{\pgfqpoint{2.185387in}{1.244804in}}{\pgfqpoint{2.193623in}{1.244804in}}%
\pgfpathclose%
\pgfusepath{stroke,fill}%
\end{pgfscope}%
\begin{pgfscope}%
\pgfpathrectangle{\pgfqpoint{0.100000in}{0.212622in}}{\pgfqpoint{3.696000in}{3.696000in}}%
\pgfusepath{clip}%
\pgfsetbuttcap%
\pgfsetroundjoin%
\definecolor{currentfill}{rgb}{0.121569,0.466667,0.705882}%
\pgfsetfillcolor{currentfill}%
\pgfsetfillopacity{0.987758}%
\pgfsetlinewidth{1.003750pt}%
\definecolor{currentstroke}{rgb}{0.121569,0.466667,0.705882}%
\pgfsetstrokecolor{currentstroke}%
\pgfsetstrokeopacity{0.987758}%
\pgfsetdash{}{0pt}%
\pgfpathmoveto{\pgfqpoint{2.198730in}{1.241085in}}%
\pgfpathcurveto{\pgfqpoint{2.206966in}{1.241085in}}{\pgfqpoint{2.214866in}{1.244357in}}{\pgfqpoint{2.220690in}{1.250181in}}%
\pgfpathcurveto{\pgfqpoint{2.226514in}{1.256005in}}{\pgfqpoint{2.229786in}{1.263905in}}{\pgfqpoint{2.229786in}{1.272141in}}%
\pgfpathcurveto{\pgfqpoint{2.229786in}{1.280378in}}{\pgfqpoint{2.226514in}{1.288278in}}{\pgfqpoint{2.220690in}{1.294102in}}%
\pgfpathcurveto{\pgfqpoint{2.214866in}{1.299926in}}{\pgfqpoint{2.206966in}{1.303198in}}{\pgfqpoint{2.198730in}{1.303198in}}%
\pgfpathcurveto{\pgfqpoint{2.190494in}{1.303198in}}{\pgfqpoint{2.182594in}{1.299926in}}{\pgfqpoint{2.176770in}{1.294102in}}%
\pgfpathcurveto{\pgfqpoint{2.170946in}{1.288278in}}{\pgfqpoint{2.167673in}{1.280378in}}{\pgfqpoint{2.167673in}{1.272141in}}%
\pgfpathcurveto{\pgfqpoint{2.167673in}{1.263905in}}{\pgfqpoint{2.170946in}{1.256005in}}{\pgfqpoint{2.176770in}{1.250181in}}%
\pgfpathcurveto{\pgfqpoint{2.182594in}{1.244357in}}{\pgfqpoint{2.190494in}{1.241085in}}{\pgfqpoint{2.198730in}{1.241085in}}%
\pgfpathclose%
\pgfusepath{stroke,fill}%
\end{pgfscope}%
\begin{pgfscope}%
\pgfpathrectangle{\pgfqpoint{0.100000in}{0.212622in}}{\pgfqpoint{3.696000in}{3.696000in}}%
\pgfusepath{clip}%
\pgfsetbuttcap%
\pgfsetroundjoin%
\definecolor{currentfill}{rgb}{0.121569,0.466667,0.705882}%
\pgfsetfillcolor{currentfill}%
\pgfsetfillopacity{0.988018}%
\pgfsetlinewidth{1.003750pt}%
\definecolor{currentstroke}{rgb}{0.121569,0.466667,0.705882}%
\pgfsetstrokecolor{currentstroke}%
\pgfsetstrokeopacity{0.988018}%
\pgfsetdash{}{0pt}%
\pgfpathmoveto{\pgfqpoint{2.204887in}{1.236866in}}%
\pgfpathcurveto{\pgfqpoint{2.213123in}{1.236866in}}{\pgfqpoint{2.221023in}{1.240138in}}{\pgfqpoint{2.226847in}{1.245962in}}%
\pgfpathcurveto{\pgfqpoint{2.232671in}{1.251786in}}{\pgfqpoint{2.235943in}{1.259686in}}{\pgfqpoint{2.235943in}{1.267923in}}%
\pgfpathcurveto{\pgfqpoint{2.235943in}{1.276159in}}{\pgfqpoint{2.232671in}{1.284059in}}{\pgfqpoint{2.226847in}{1.289883in}}%
\pgfpathcurveto{\pgfqpoint{2.221023in}{1.295707in}}{\pgfqpoint{2.213123in}{1.298979in}}{\pgfqpoint{2.204887in}{1.298979in}}%
\pgfpathcurveto{\pgfqpoint{2.196650in}{1.298979in}}{\pgfqpoint{2.188750in}{1.295707in}}{\pgfqpoint{2.182926in}{1.289883in}}%
\pgfpathcurveto{\pgfqpoint{2.177103in}{1.284059in}}{\pgfqpoint{2.173830in}{1.276159in}}{\pgfqpoint{2.173830in}{1.267923in}}%
\pgfpathcurveto{\pgfqpoint{2.173830in}{1.259686in}}{\pgfqpoint{2.177103in}{1.251786in}}{\pgfqpoint{2.182926in}{1.245962in}}%
\pgfpathcurveto{\pgfqpoint{2.188750in}{1.240138in}}{\pgfqpoint{2.196650in}{1.236866in}}{\pgfqpoint{2.204887in}{1.236866in}}%
\pgfpathclose%
\pgfusepath{stroke,fill}%
\end{pgfscope}%
\begin{pgfscope}%
\pgfpathrectangle{\pgfqpoint{0.100000in}{0.212622in}}{\pgfqpoint{3.696000in}{3.696000in}}%
\pgfusepath{clip}%
\pgfsetbuttcap%
\pgfsetroundjoin%
\definecolor{currentfill}{rgb}{0.121569,0.466667,0.705882}%
\pgfsetfillcolor{currentfill}%
\pgfsetfillopacity{0.988179}%
\pgfsetlinewidth{1.003750pt}%
\definecolor{currentstroke}{rgb}{0.121569,0.466667,0.705882}%
\pgfsetstrokecolor{currentstroke}%
\pgfsetstrokeopacity{0.988179}%
\pgfsetdash{}{0pt}%
\pgfpathmoveto{\pgfqpoint{2.208286in}{1.234697in}}%
\pgfpathcurveto{\pgfqpoint{2.216522in}{1.234697in}}{\pgfqpoint{2.224422in}{1.237969in}}{\pgfqpoint{2.230246in}{1.243793in}}%
\pgfpathcurveto{\pgfqpoint{2.236070in}{1.249617in}}{\pgfqpoint{2.239342in}{1.257517in}}{\pgfqpoint{2.239342in}{1.265754in}}%
\pgfpathcurveto{\pgfqpoint{2.239342in}{1.273990in}}{\pgfqpoint{2.236070in}{1.281890in}}{\pgfqpoint{2.230246in}{1.287714in}}%
\pgfpathcurveto{\pgfqpoint{2.224422in}{1.293538in}}{\pgfqpoint{2.216522in}{1.296810in}}{\pgfqpoint{2.208286in}{1.296810in}}%
\pgfpathcurveto{\pgfqpoint{2.200049in}{1.296810in}}{\pgfqpoint{2.192149in}{1.293538in}}{\pgfqpoint{2.186325in}{1.287714in}}%
\pgfpathcurveto{\pgfqpoint{2.180501in}{1.281890in}}{\pgfqpoint{2.177229in}{1.273990in}}{\pgfqpoint{2.177229in}{1.265754in}}%
\pgfpathcurveto{\pgfqpoint{2.177229in}{1.257517in}}{\pgfqpoint{2.180501in}{1.249617in}}{\pgfqpoint{2.186325in}{1.243793in}}%
\pgfpathcurveto{\pgfqpoint{2.192149in}{1.237969in}}{\pgfqpoint{2.200049in}{1.234697in}}{\pgfqpoint{2.208286in}{1.234697in}}%
\pgfpathclose%
\pgfusepath{stroke,fill}%
\end{pgfscope}%
\begin{pgfscope}%
\pgfpathrectangle{\pgfqpoint{0.100000in}{0.212622in}}{\pgfqpoint{3.696000in}{3.696000in}}%
\pgfusepath{clip}%
\pgfsetbuttcap%
\pgfsetroundjoin%
\definecolor{currentfill}{rgb}{0.121569,0.466667,0.705882}%
\pgfsetfillcolor{currentfill}%
\pgfsetfillopacity{0.988452}%
\pgfsetlinewidth{1.003750pt}%
\definecolor{currentstroke}{rgb}{0.121569,0.466667,0.705882}%
\pgfsetstrokecolor{currentstroke}%
\pgfsetstrokeopacity{0.988452}%
\pgfsetdash{}{0pt}%
\pgfpathmoveto{\pgfqpoint{2.212374in}{1.232624in}}%
\pgfpathcurveto{\pgfqpoint{2.220611in}{1.232624in}}{\pgfqpoint{2.228511in}{1.235896in}}{\pgfqpoint{2.234335in}{1.241720in}}%
\pgfpathcurveto{\pgfqpoint{2.240159in}{1.247544in}}{\pgfqpoint{2.243431in}{1.255444in}}{\pgfqpoint{2.243431in}{1.263680in}}%
\pgfpathcurveto{\pgfqpoint{2.243431in}{1.271917in}}{\pgfqpoint{2.240159in}{1.279817in}}{\pgfqpoint{2.234335in}{1.285641in}}%
\pgfpathcurveto{\pgfqpoint{2.228511in}{1.291465in}}{\pgfqpoint{2.220611in}{1.294737in}}{\pgfqpoint{2.212374in}{1.294737in}}%
\pgfpathcurveto{\pgfqpoint{2.204138in}{1.294737in}}{\pgfqpoint{2.196238in}{1.291465in}}{\pgfqpoint{2.190414in}{1.285641in}}%
\pgfpathcurveto{\pgfqpoint{2.184590in}{1.279817in}}{\pgfqpoint{2.181318in}{1.271917in}}{\pgfqpoint{2.181318in}{1.263680in}}%
\pgfpathcurveto{\pgfqpoint{2.181318in}{1.255444in}}{\pgfqpoint{2.184590in}{1.247544in}}{\pgfqpoint{2.190414in}{1.241720in}}%
\pgfpathcurveto{\pgfqpoint{2.196238in}{1.235896in}}{\pgfqpoint{2.204138in}{1.232624in}}{\pgfqpoint{2.212374in}{1.232624in}}%
\pgfpathclose%
\pgfusepath{stroke,fill}%
\end{pgfscope}%
\begin{pgfscope}%
\pgfpathrectangle{\pgfqpoint{0.100000in}{0.212622in}}{\pgfqpoint{3.696000in}{3.696000in}}%
\pgfusepath{clip}%
\pgfsetbuttcap%
\pgfsetroundjoin%
\definecolor{currentfill}{rgb}{0.121569,0.466667,0.705882}%
\pgfsetfillcolor{currentfill}%
\pgfsetfillopacity{0.988700}%
\pgfsetlinewidth{1.003750pt}%
\definecolor{currentstroke}{rgb}{0.121569,0.466667,0.705882}%
\pgfsetstrokecolor{currentstroke}%
\pgfsetstrokeopacity{0.988700}%
\pgfsetdash{}{0pt}%
\pgfpathmoveto{\pgfqpoint{2.511459in}{1.108126in}}%
\pgfpathcurveto{\pgfqpoint{2.519696in}{1.108126in}}{\pgfqpoint{2.527596in}{1.111398in}}{\pgfqpoint{2.533420in}{1.117222in}}%
\pgfpathcurveto{\pgfqpoint{2.539244in}{1.123046in}}{\pgfqpoint{2.542516in}{1.130946in}}{\pgfqpoint{2.542516in}{1.139183in}}%
\pgfpathcurveto{\pgfqpoint{2.542516in}{1.147419in}}{\pgfqpoint{2.539244in}{1.155319in}}{\pgfqpoint{2.533420in}{1.161143in}}%
\pgfpathcurveto{\pgfqpoint{2.527596in}{1.166967in}}{\pgfqpoint{2.519696in}{1.170239in}}{\pgfqpoint{2.511459in}{1.170239in}}%
\pgfpathcurveto{\pgfqpoint{2.503223in}{1.170239in}}{\pgfqpoint{2.495323in}{1.166967in}}{\pgfqpoint{2.489499in}{1.161143in}}%
\pgfpathcurveto{\pgfqpoint{2.483675in}{1.155319in}}{\pgfqpoint{2.480403in}{1.147419in}}{\pgfqpoint{2.480403in}{1.139183in}}%
\pgfpathcurveto{\pgfqpoint{2.480403in}{1.130946in}}{\pgfqpoint{2.483675in}{1.123046in}}{\pgfqpoint{2.489499in}{1.117222in}}%
\pgfpathcurveto{\pgfqpoint{2.495323in}{1.111398in}}{\pgfqpoint{2.503223in}{1.108126in}}{\pgfqpoint{2.511459in}{1.108126in}}%
\pgfpathclose%
\pgfusepath{stroke,fill}%
\end{pgfscope}%
\begin{pgfscope}%
\pgfpathrectangle{\pgfqpoint{0.100000in}{0.212622in}}{\pgfqpoint{3.696000in}{3.696000in}}%
\pgfusepath{clip}%
\pgfsetbuttcap%
\pgfsetroundjoin%
\definecolor{currentfill}{rgb}{0.121569,0.466667,0.705882}%
\pgfsetfillcolor{currentfill}%
\pgfsetfillopacity{0.988781}%
\pgfsetlinewidth{1.003750pt}%
\definecolor{currentstroke}{rgb}{0.121569,0.466667,0.705882}%
\pgfsetstrokecolor{currentstroke}%
\pgfsetstrokeopacity{0.988781}%
\pgfsetdash{}{0pt}%
\pgfpathmoveto{\pgfqpoint{2.217577in}{1.229503in}}%
\pgfpathcurveto{\pgfqpoint{2.225813in}{1.229503in}}{\pgfqpoint{2.233713in}{1.232775in}}{\pgfqpoint{2.239537in}{1.238599in}}%
\pgfpathcurveto{\pgfqpoint{2.245361in}{1.244423in}}{\pgfqpoint{2.248633in}{1.252323in}}{\pgfqpoint{2.248633in}{1.260559in}}%
\pgfpathcurveto{\pgfqpoint{2.248633in}{1.268795in}}{\pgfqpoint{2.245361in}{1.276695in}}{\pgfqpoint{2.239537in}{1.282519in}}%
\pgfpathcurveto{\pgfqpoint{2.233713in}{1.288343in}}{\pgfqpoint{2.225813in}{1.291616in}}{\pgfqpoint{2.217577in}{1.291616in}}%
\pgfpathcurveto{\pgfqpoint{2.209340in}{1.291616in}}{\pgfqpoint{2.201440in}{1.288343in}}{\pgfqpoint{2.195616in}{1.282519in}}%
\pgfpathcurveto{\pgfqpoint{2.189792in}{1.276695in}}{\pgfqpoint{2.186520in}{1.268795in}}{\pgfqpoint{2.186520in}{1.260559in}}%
\pgfpathcurveto{\pgfqpoint{2.186520in}{1.252323in}}{\pgfqpoint{2.189792in}{1.244423in}}{\pgfqpoint{2.195616in}{1.238599in}}%
\pgfpathcurveto{\pgfqpoint{2.201440in}{1.232775in}}{\pgfqpoint{2.209340in}{1.229503in}}{\pgfqpoint{2.217577in}{1.229503in}}%
\pgfpathclose%
\pgfusepath{stroke,fill}%
\end{pgfscope}%
\begin{pgfscope}%
\pgfpathrectangle{\pgfqpoint{0.100000in}{0.212622in}}{\pgfqpoint{3.696000in}{3.696000in}}%
\pgfusepath{clip}%
\pgfsetbuttcap%
\pgfsetroundjoin%
\definecolor{currentfill}{rgb}{0.121569,0.466667,0.705882}%
\pgfsetfillcolor{currentfill}%
\pgfsetfillopacity{0.988944}%
\pgfsetlinewidth{1.003750pt}%
\definecolor{currentstroke}{rgb}{0.121569,0.466667,0.705882}%
\pgfsetstrokecolor{currentstroke}%
\pgfsetstrokeopacity{0.988944}%
\pgfsetdash{}{0pt}%
\pgfpathmoveto{\pgfqpoint{2.220403in}{1.227563in}}%
\pgfpathcurveto{\pgfqpoint{2.228640in}{1.227563in}}{\pgfqpoint{2.236540in}{1.230835in}}{\pgfqpoint{2.242364in}{1.236659in}}%
\pgfpathcurveto{\pgfqpoint{2.248188in}{1.242483in}}{\pgfqpoint{2.251460in}{1.250383in}}{\pgfqpoint{2.251460in}{1.258619in}}%
\pgfpathcurveto{\pgfqpoint{2.251460in}{1.266855in}}{\pgfqpoint{2.248188in}{1.274755in}}{\pgfqpoint{2.242364in}{1.280579in}}%
\pgfpathcurveto{\pgfqpoint{2.236540in}{1.286403in}}{\pgfqpoint{2.228640in}{1.289676in}}{\pgfqpoint{2.220403in}{1.289676in}}%
\pgfpathcurveto{\pgfqpoint{2.212167in}{1.289676in}}{\pgfqpoint{2.204267in}{1.286403in}}{\pgfqpoint{2.198443in}{1.280579in}}%
\pgfpathcurveto{\pgfqpoint{2.192619in}{1.274755in}}{\pgfqpoint{2.189347in}{1.266855in}}{\pgfqpoint{2.189347in}{1.258619in}}%
\pgfpathcurveto{\pgfqpoint{2.189347in}{1.250383in}}{\pgfqpoint{2.192619in}{1.242483in}}{\pgfqpoint{2.198443in}{1.236659in}}%
\pgfpathcurveto{\pgfqpoint{2.204267in}{1.230835in}}{\pgfqpoint{2.212167in}{1.227563in}}{\pgfqpoint{2.220403in}{1.227563in}}%
\pgfpathclose%
\pgfusepath{stroke,fill}%
\end{pgfscope}%
\begin{pgfscope}%
\pgfpathrectangle{\pgfqpoint{0.100000in}{0.212622in}}{\pgfqpoint{3.696000in}{3.696000in}}%
\pgfusepath{clip}%
\pgfsetbuttcap%
\pgfsetroundjoin%
\definecolor{currentfill}{rgb}{0.121569,0.466667,0.705882}%
\pgfsetfillcolor{currentfill}%
\pgfsetfillopacity{0.989046}%
\pgfsetlinewidth{1.003750pt}%
\definecolor{currentstroke}{rgb}{0.121569,0.466667,0.705882}%
\pgfsetstrokecolor{currentstroke}%
\pgfsetstrokeopacity{0.989046}%
\pgfsetdash{}{0pt}%
\pgfpathmoveto{\pgfqpoint{2.224432in}{1.224594in}}%
\pgfpathcurveto{\pgfqpoint{2.232668in}{1.224594in}}{\pgfqpoint{2.240568in}{1.227867in}}{\pgfqpoint{2.246392in}{1.233690in}}%
\pgfpathcurveto{\pgfqpoint{2.252216in}{1.239514in}}{\pgfqpoint{2.255488in}{1.247414in}}{\pgfqpoint{2.255488in}{1.255651in}}%
\pgfpathcurveto{\pgfqpoint{2.255488in}{1.263887in}}{\pgfqpoint{2.252216in}{1.271787in}}{\pgfqpoint{2.246392in}{1.277611in}}%
\pgfpathcurveto{\pgfqpoint{2.240568in}{1.283435in}}{\pgfqpoint{2.232668in}{1.286707in}}{\pgfqpoint{2.224432in}{1.286707in}}%
\pgfpathcurveto{\pgfqpoint{2.216196in}{1.286707in}}{\pgfqpoint{2.208295in}{1.283435in}}{\pgfqpoint{2.202472in}{1.277611in}}%
\pgfpathcurveto{\pgfqpoint{2.196648in}{1.271787in}}{\pgfqpoint{2.193375in}{1.263887in}}{\pgfqpoint{2.193375in}{1.255651in}}%
\pgfpathcurveto{\pgfqpoint{2.193375in}{1.247414in}}{\pgfqpoint{2.196648in}{1.239514in}}{\pgfqpoint{2.202472in}{1.233690in}}%
\pgfpathcurveto{\pgfqpoint{2.208295in}{1.227867in}}{\pgfqpoint{2.216196in}{1.224594in}}{\pgfqpoint{2.224432in}{1.224594in}}%
\pgfpathclose%
\pgfusepath{stroke,fill}%
\end{pgfscope}%
\begin{pgfscope}%
\pgfpathrectangle{\pgfqpoint{0.100000in}{0.212622in}}{\pgfqpoint{3.696000in}{3.696000in}}%
\pgfusepath{clip}%
\pgfsetbuttcap%
\pgfsetroundjoin%
\definecolor{currentfill}{rgb}{0.121569,0.466667,0.705882}%
\pgfsetfillcolor{currentfill}%
\pgfsetfillopacity{0.989119}%
\pgfsetlinewidth{1.003750pt}%
\definecolor{currentstroke}{rgb}{0.121569,0.466667,0.705882}%
\pgfsetstrokecolor{currentstroke}%
\pgfsetstrokeopacity{0.989119}%
\pgfsetdash{}{0pt}%
\pgfpathmoveto{\pgfqpoint{2.226659in}{1.223096in}}%
\pgfpathcurveto{\pgfqpoint{2.234895in}{1.223096in}}{\pgfqpoint{2.242796in}{1.226369in}}{\pgfqpoint{2.248619in}{1.232192in}}%
\pgfpathcurveto{\pgfqpoint{2.254443in}{1.238016in}}{\pgfqpoint{2.257716in}{1.245916in}}{\pgfqpoint{2.257716in}{1.254153in}}%
\pgfpathcurveto{\pgfqpoint{2.257716in}{1.262389in}}{\pgfqpoint{2.254443in}{1.270289in}}{\pgfqpoint{2.248619in}{1.276113in}}%
\pgfpathcurveto{\pgfqpoint{2.242796in}{1.281937in}}{\pgfqpoint{2.234895in}{1.285209in}}{\pgfqpoint{2.226659in}{1.285209in}}%
\pgfpathcurveto{\pgfqpoint{2.218423in}{1.285209in}}{\pgfqpoint{2.210523in}{1.281937in}}{\pgfqpoint{2.204699in}{1.276113in}}%
\pgfpathcurveto{\pgfqpoint{2.198875in}{1.270289in}}{\pgfqpoint{2.195603in}{1.262389in}}{\pgfqpoint{2.195603in}{1.254153in}}%
\pgfpathcurveto{\pgfqpoint{2.195603in}{1.245916in}}{\pgfqpoint{2.198875in}{1.238016in}}{\pgfqpoint{2.204699in}{1.232192in}}%
\pgfpathcurveto{\pgfqpoint{2.210523in}{1.226369in}}{\pgfqpoint{2.218423in}{1.223096in}}{\pgfqpoint{2.226659in}{1.223096in}}%
\pgfpathclose%
\pgfusepath{stroke,fill}%
\end{pgfscope}%
\begin{pgfscope}%
\pgfpathrectangle{\pgfqpoint{0.100000in}{0.212622in}}{\pgfqpoint{3.696000in}{3.696000in}}%
\pgfusepath{clip}%
\pgfsetbuttcap%
\pgfsetroundjoin%
\definecolor{currentfill}{rgb}{0.121569,0.466667,0.705882}%
\pgfsetfillcolor{currentfill}%
\pgfsetfillopacity{0.989215}%
\pgfsetlinewidth{1.003750pt}%
\definecolor{currentstroke}{rgb}{0.121569,0.466667,0.705882}%
\pgfsetstrokecolor{currentstroke}%
\pgfsetstrokeopacity{0.989215}%
\pgfsetdash{}{0pt}%
\pgfpathmoveto{\pgfqpoint{2.229303in}{1.221386in}}%
\pgfpathcurveto{\pgfqpoint{2.237539in}{1.221386in}}{\pgfqpoint{2.245439in}{1.224659in}}{\pgfqpoint{2.251263in}{1.230483in}}%
\pgfpathcurveto{\pgfqpoint{2.257087in}{1.236307in}}{\pgfqpoint{2.260359in}{1.244207in}}{\pgfqpoint{2.260359in}{1.252443in}}%
\pgfpathcurveto{\pgfqpoint{2.260359in}{1.260679in}}{\pgfqpoint{2.257087in}{1.268579in}}{\pgfqpoint{2.251263in}{1.274403in}}%
\pgfpathcurveto{\pgfqpoint{2.245439in}{1.280227in}}{\pgfqpoint{2.237539in}{1.283499in}}{\pgfqpoint{2.229303in}{1.283499in}}%
\pgfpathcurveto{\pgfqpoint{2.221067in}{1.283499in}}{\pgfqpoint{2.213166in}{1.280227in}}{\pgfqpoint{2.207343in}{1.274403in}}%
\pgfpathcurveto{\pgfqpoint{2.201519in}{1.268579in}}{\pgfqpoint{2.198246in}{1.260679in}}{\pgfqpoint{2.198246in}{1.252443in}}%
\pgfpathcurveto{\pgfqpoint{2.198246in}{1.244207in}}{\pgfqpoint{2.201519in}{1.236307in}}{\pgfqpoint{2.207343in}{1.230483in}}%
\pgfpathcurveto{\pgfqpoint{2.213166in}{1.224659in}}{\pgfqpoint{2.221067in}{1.221386in}}{\pgfqpoint{2.229303in}{1.221386in}}%
\pgfpathclose%
\pgfusepath{stroke,fill}%
\end{pgfscope}%
\begin{pgfscope}%
\pgfpathrectangle{\pgfqpoint{0.100000in}{0.212622in}}{\pgfqpoint{3.696000in}{3.696000in}}%
\pgfusepath{clip}%
\pgfsetbuttcap%
\pgfsetroundjoin%
\definecolor{currentfill}{rgb}{0.121569,0.466667,0.705882}%
\pgfsetfillcolor{currentfill}%
\pgfsetfillopacity{0.989375}%
\pgfsetlinewidth{1.003750pt}%
\definecolor{currentstroke}{rgb}{0.121569,0.466667,0.705882}%
\pgfsetstrokecolor{currentstroke}%
\pgfsetstrokeopacity{0.989375}%
\pgfsetdash{}{0pt}%
\pgfpathmoveto{\pgfqpoint{2.232514in}{1.219495in}}%
\pgfpathcurveto{\pgfqpoint{2.240751in}{1.219495in}}{\pgfqpoint{2.248651in}{1.222767in}}{\pgfqpoint{2.254475in}{1.228591in}}%
\pgfpathcurveto{\pgfqpoint{2.260299in}{1.234415in}}{\pgfqpoint{2.263571in}{1.242315in}}{\pgfqpoint{2.263571in}{1.250551in}}%
\pgfpathcurveto{\pgfqpoint{2.263571in}{1.258787in}}{\pgfqpoint{2.260299in}{1.266687in}}{\pgfqpoint{2.254475in}{1.272511in}}%
\pgfpathcurveto{\pgfqpoint{2.248651in}{1.278335in}}{\pgfqpoint{2.240751in}{1.281608in}}{\pgfqpoint{2.232514in}{1.281608in}}%
\pgfpathcurveto{\pgfqpoint{2.224278in}{1.281608in}}{\pgfqpoint{2.216378in}{1.278335in}}{\pgfqpoint{2.210554in}{1.272511in}}%
\pgfpathcurveto{\pgfqpoint{2.204730in}{1.266687in}}{\pgfqpoint{2.201458in}{1.258787in}}{\pgfqpoint{2.201458in}{1.250551in}}%
\pgfpathcurveto{\pgfqpoint{2.201458in}{1.242315in}}{\pgfqpoint{2.204730in}{1.234415in}}{\pgfqpoint{2.210554in}{1.228591in}}%
\pgfpathcurveto{\pgfqpoint{2.216378in}{1.222767in}}{\pgfqpoint{2.224278in}{1.219495in}}{\pgfqpoint{2.232514in}{1.219495in}}%
\pgfpathclose%
\pgfusepath{stroke,fill}%
\end{pgfscope}%
\begin{pgfscope}%
\pgfpathrectangle{\pgfqpoint{0.100000in}{0.212622in}}{\pgfqpoint{3.696000in}{3.696000in}}%
\pgfusepath{clip}%
\pgfsetbuttcap%
\pgfsetroundjoin%
\definecolor{currentfill}{rgb}{0.121569,0.466667,0.705882}%
\pgfsetfillcolor{currentfill}%
\pgfsetfillopacity{0.989661}%
\pgfsetlinewidth{1.003750pt}%
\definecolor{currentstroke}{rgb}{0.121569,0.466667,0.705882}%
\pgfsetstrokecolor{currentstroke}%
\pgfsetstrokeopacity{0.989661}%
\pgfsetdash{}{0pt}%
\pgfpathmoveto{\pgfqpoint{2.238514in}{1.215611in}}%
\pgfpathcurveto{\pgfqpoint{2.246750in}{1.215611in}}{\pgfqpoint{2.254650in}{1.218883in}}{\pgfqpoint{2.260474in}{1.224707in}}%
\pgfpathcurveto{\pgfqpoint{2.266298in}{1.230531in}}{\pgfqpoint{2.269571in}{1.238431in}}{\pgfqpoint{2.269571in}{1.246667in}}%
\pgfpathcurveto{\pgfqpoint{2.269571in}{1.254903in}}{\pgfqpoint{2.266298in}{1.262803in}}{\pgfqpoint{2.260474in}{1.268627in}}%
\pgfpathcurveto{\pgfqpoint{2.254650in}{1.274451in}}{\pgfqpoint{2.246750in}{1.277724in}}{\pgfqpoint{2.238514in}{1.277724in}}%
\pgfpathcurveto{\pgfqpoint{2.230278in}{1.277724in}}{\pgfqpoint{2.222378in}{1.274451in}}{\pgfqpoint{2.216554in}{1.268627in}}%
\pgfpathcurveto{\pgfqpoint{2.210730in}{1.262803in}}{\pgfqpoint{2.207458in}{1.254903in}}{\pgfqpoint{2.207458in}{1.246667in}}%
\pgfpathcurveto{\pgfqpoint{2.207458in}{1.238431in}}{\pgfqpoint{2.210730in}{1.230531in}}{\pgfqpoint{2.216554in}{1.224707in}}%
\pgfpathcurveto{\pgfqpoint{2.222378in}{1.218883in}}{\pgfqpoint{2.230278in}{1.215611in}}{\pgfqpoint{2.238514in}{1.215611in}}%
\pgfpathclose%
\pgfusepath{stroke,fill}%
\end{pgfscope}%
\begin{pgfscope}%
\pgfpathrectangle{\pgfqpoint{0.100000in}{0.212622in}}{\pgfqpoint{3.696000in}{3.696000in}}%
\pgfusepath{clip}%
\pgfsetbuttcap%
\pgfsetroundjoin%
\definecolor{currentfill}{rgb}{0.121569,0.466667,0.705882}%
\pgfsetfillcolor{currentfill}%
\pgfsetfillopacity{0.989806}%
\pgfsetlinewidth{1.003750pt}%
\definecolor{currentstroke}{rgb}{0.121569,0.466667,0.705882}%
\pgfsetstrokecolor{currentstroke}%
\pgfsetstrokeopacity{0.989806}%
\pgfsetdash{}{0pt}%
\pgfpathmoveto{\pgfqpoint{2.509317in}{1.105943in}}%
\pgfpathcurveto{\pgfqpoint{2.517554in}{1.105943in}}{\pgfqpoint{2.525454in}{1.109216in}}{\pgfqpoint{2.531278in}{1.115039in}}%
\pgfpathcurveto{\pgfqpoint{2.537101in}{1.120863in}}{\pgfqpoint{2.540374in}{1.128763in}}{\pgfqpoint{2.540374in}{1.137000in}}%
\pgfpathcurveto{\pgfqpoint{2.540374in}{1.145236in}}{\pgfqpoint{2.537101in}{1.153136in}}{\pgfqpoint{2.531278in}{1.158960in}}%
\pgfpathcurveto{\pgfqpoint{2.525454in}{1.164784in}}{\pgfqpoint{2.517554in}{1.168056in}}{\pgfqpoint{2.509317in}{1.168056in}}%
\pgfpathcurveto{\pgfqpoint{2.501081in}{1.168056in}}{\pgfqpoint{2.493181in}{1.164784in}}{\pgfqpoint{2.487357in}{1.158960in}}%
\pgfpathcurveto{\pgfqpoint{2.481533in}{1.153136in}}{\pgfqpoint{2.478261in}{1.145236in}}{\pgfqpoint{2.478261in}{1.137000in}}%
\pgfpathcurveto{\pgfqpoint{2.478261in}{1.128763in}}{\pgfqpoint{2.481533in}{1.120863in}}{\pgfqpoint{2.487357in}{1.115039in}}%
\pgfpathcurveto{\pgfqpoint{2.493181in}{1.109216in}}{\pgfqpoint{2.501081in}{1.105943in}}{\pgfqpoint{2.509317in}{1.105943in}}%
\pgfpathclose%
\pgfusepath{stroke,fill}%
\end{pgfscope}%
\begin{pgfscope}%
\pgfpathrectangle{\pgfqpoint{0.100000in}{0.212622in}}{\pgfqpoint{3.696000in}{3.696000in}}%
\pgfusepath{clip}%
\pgfsetbuttcap%
\pgfsetroundjoin%
\definecolor{currentfill}{rgb}{0.121569,0.466667,0.705882}%
\pgfsetfillcolor{currentfill}%
\pgfsetfillopacity{0.990079}%
\pgfsetlinewidth{1.003750pt}%
\definecolor{currentstroke}{rgb}{0.121569,0.466667,0.705882}%
\pgfsetstrokecolor{currentstroke}%
\pgfsetstrokeopacity{0.990079}%
\pgfsetdash{}{0pt}%
\pgfpathmoveto{\pgfqpoint{2.246573in}{1.210339in}}%
\pgfpathcurveto{\pgfqpoint{2.254809in}{1.210339in}}{\pgfqpoint{2.262709in}{1.213611in}}{\pgfqpoint{2.268533in}{1.219435in}}%
\pgfpathcurveto{\pgfqpoint{2.274357in}{1.225259in}}{\pgfqpoint{2.277629in}{1.233159in}}{\pgfqpoint{2.277629in}{1.241395in}}%
\pgfpathcurveto{\pgfqpoint{2.277629in}{1.249632in}}{\pgfqpoint{2.274357in}{1.257532in}}{\pgfqpoint{2.268533in}{1.263355in}}%
\pgfpathcurveto{\pgfqpoint{2.262709in}{1.269179in}}{\pgfqpoint{2.254809in}{1.272452in}}{\pgfqpoint{2.246573in}{1.272452in}}%
\pgfpathcurveto{\pgfqpoint{2.238337in}{1.272452in}}{\pgfqpoint{2.230437in}{1.269179in}}{\pgfqpoint{2.224613in}{1.263355in}}%
\pgfpathcurveto{\pgfqpoint{2.218789in}{1.257532in}}{\pgfqpoint{2.215516in}{1.249632in}}{\pgfqpoint{2.215516in}{1.241395in}}%
\pgfpathcurveto{\pgfqpoint{2.215516in}{1.233159in}}{\pgfqpoint{2.218789in}{1.225259in}}{\pgfqpoint{2.224613in}{1.219435in}}%
\pgfpathcurveto{\pgfqpoint{2.230437in}{1.213611in}}{\pgfqpoint{2.238337in}{1.210339in}}{\pgfqpoint{2.246573in}{1.210339in}}%
\pgfpathclose%
\pgfusepath{stroke,fill}%
\end{pgfscope}%
\begin{pgfscope}%
\pgfpathrectangle{\pgfqpoint{0.100000in}{0.212622in}}{\pgfqpoint{3.696000in}{3.696000in}}%
\pgfusepath{clip}%
\pgfsetbuttcap%
\pgfsetroundjoin%
\definecolor{currentfill}{rgb}{0.121569,0.466667,0.705882}%
\pgfsetfillcolor{currentfill}%
\pgfsetfillopacity{0.990486}%
\pgfsetlinewidth{1.003750pt}%
\definecolor{currentstroke}{rgb}{0.121569,0.466667,0.705882}%
\pgfsetstrokecolor{currentstroke}%
\pgfsetstrokeopacity{0.990486}%
\pgfsetdash{}{0pt}%
\pgfpathmoveto{\pgfqpoint{2.255812in}{1.203767in}}%
\pgfpathcurveto{\pgfqpoint{2.264048in}{1.203767in}}{\pgfqpoint{2.271948in}{1.207039in}}{\pgfqpoint{2.277772in}{1.212863in}}%
\pgfpathcurveto{\pgfqpoint{2.283596in}{1.218687in}}{\pgfqpoint{2.286868in}{1.226587in}}{\pgfqpoint{2.286868in}{1.234823in}}%
\pgfpathcurveto{\pgfqpoint{2.286868in}{1.243059in}}{\pgfqpoint{2.283596in}{1.250959in}}{\pgfqpoint{2.277772in}{1.256783in}}%
\pgfpathcurveto{\pgfqpoint{2.271948in}{1.262607in}}{\pgfqpoint{2.264048in}{1.265880in}}{\pgfqpoint{2.255812in}{1.265880in}}%
\pgfpathcurveto{\pgfqpoint{2.247576in}{1.265880in}}{\pgfqpoint{2.239676in}{1.262607in}}{\pgfqpoint{2.233852in}{1.256783in}}%
\pgfpathcurveto{\pgfqpoint{2.228028in}{1.250959in}}{\pgfqpoint{2.224755in}{1.243059in}}{\pgfqpoint{2.224755in}{1.234823in}}%
\pgfpathcurveto{\pgfqpoint{2.224755in}{1.226587in}}{\pgfqpoint{2.228028in}{1.218687in}}{\pgfqpoint{2.233852in}{1.212863in}}%
\pgfpathcurveto{\pgfqpoint{2.239676in}{1.207039in}}{\pgfqpoint{2.247576in}{1.203767in}}{\pgfqpoint{2.255812in}{1.203767in}}%
\pgfpathclose%
\pgfusepath{stroke,fill}%
\end{pgfscope}%
\begin{pgfscope}%
\pgfpathrectangle{\pgfqpoint{0.100000in}{0.212622in}}{\pgfqpoint{3.696000in}{3.696000in}}%
\pgfusepath{clip}%
\pgfsetbuttcap%
\pgfsetroundjoin%
\definecolor{currentfill}{rgb}{0.121569,0.466667,0.705882}%
\pgfsetfillcolor{currentfill}%
\pgfsetfillopacity{0.990993}%
\pgfsetlinewidth{1.003750pt}%
\definecolor{currentstroke}{rgb}{0.121569,0.466667,0.705882}%
\pgfsetstrokecolor{currentstroke}%
\pgfsetstrokeopacity{0.990993}%
\pgfsetdash{}{0pt}%
\pgfpathmoveto{\pgfqpoint{2.265526in}{1.197209in}}%
\pgfpathcurveto{\pgfqpoint{2.273762in}{1.197209in}}{\pgfqpoint{2.281662in}{1.200481in}}{\pgfqpoint{2.287486in}{1.206305in}}%
\pgfpathcurveto{\pgfqpoint{2.293310in}{1.212129in}}{\pgfqpoint{2.296582in}{1.220029in}}{\pgfqpoint{2.296582in}{1.228265in}}%
\pgfpathcurveto{\pgfqpoint{2.296582in}{1.236501in}}{\pgfqpoint{2.293310in}{1.244401in}}{\pgfqpoint{2.287486in}{1.250225in}}%
\pgfpathcurveto{\pgfqpoint{2.281662in}{1.256049in}}{\pgfqpoint{2.273762in}{1.259322in}}{\pgfqpoint{2.265526in}{1.259322in}}%
\pgfpathcurveto{\pgfqpoint{2.257290in}{1.259322in}}{\pgfqpoint{2.249389in}{1.256049in}}{\pgfqpoint{2.243566in}{1.250225in}}%
\pgfpathcurveto{\pgfqpoint{2.237742in}{1.244401in}}{\pgfqpoint{2.234469in}{1.236501in}}{\pgfqpoint{2.234469in}{1.228265in}}%
\pgfpathcurveto{\pgfqpoint{2.234469in}{1.220029in}}{\pgfqpoint{2.237742in}{1.212129in}}{\pgfqpoint{2.243566in}{1.206305in}}%
\pgfpathcurveto{\pgfqpoint{2.249389in}{1.200481in}}{\pgfqpoint{2.257290in}{1.197209in}}{\pgfqpoint{2.265526in}{1.197209in}}%
\pgfpathclose%
\pgfusepath{stroke,fill}%
\end{pgfscope}%
\begin{pgfscope}%
\pgfpathrectangle{\pgfqpoint{0.100000in}{0.212622in}}{\pgfqpoint{3.696000in}{3.696000in}}%
\pgfusepath{clip}%
\pgfsetbuttcap%
\pgfsetroundjoin%
\definecolor{currentfill}{rgb}{0.121569,0.466667,0.705882}%
\pgfsetfillcolor{currentfill}%
\pgfsetfillopacity{0.991546}%
\pgfsetlinewidth{1.003750pt}%
\definecolor{currentstroke}{rgb}{0.121569,0.466667,0.705882}%
\pgfsetstrokecolor{currentstroke}%
\pgfsetstrokeopacity{0.991546}%
\pgfsetdash{}{0pt}%
\pgfpathmoveto{\pgfqpoint{2.275677in}{1.190751in}}%
\pgfpathcurveto{\pgfqpoint{2.283913in}{1.190751in}}{\pgfqpoint{2.291813in}{1.194023in}}{\pgfqpoint{2.297637in}{1.199847in}}%
\pgfpathcurveto{\pgfqpoint{2.303461in}{1.205671in}}{\pgfqpoint{2.306733in}{1.213571in}}{\pgfqpoint{2.306733in}{1.221807in}}%
\pgfpathcurveto{\pgfqpoint{2.306733in}{1.230043in}}{\pgfqpoint{2.303461in}{1.237943in}}{\pgfqpoint{2.297637in}{1.243767in}}%
\pgfpathcurveto{\pgfqpoint{2.291813in}{1.249591in}}{\pgfqpoint{2.283913in}{1.252864in}}{\pgfqpoint{2.275677in}{1.252864in}}%
\pgfpathcurveto{\pgfqpoint{2.267441in}{1.252864in}}{\pgfqpoint{2.259541in}{1.249591in}}{\pgfqpoint{2.253717in}{1.243767in}}%
\pgfpathcurveto{\pgfqpoint{2.247893in}{1.237943in}}{\pgfqpoint{2.244620in}{1.230043in}}{\pgfqpoint{2.244620in}{1.221807in}}%
\pgfpathcurveto{\pgfqpoint{2.244620in}{1.213571in}}{\pgfqpoint{2.247893in}{1.205671in}}{\pgfqpoint{2.253717in}{1.199847in}}%
\pgfpathcurveto{\pgfqpoint{2.259541in}{1.194023in}}{\pgfqpoint{2.267441in}{1.190751in}}{\pgfqpoint{2.275677in}{1.190751in}}%
\pgfpathclose%
\pgfusepath{stroke,fill}%
\end{pgfscope}%
\begin{pgfscope}%
\pgfpathrectangle{\pgfqpoint{0.100000in}{0.212622in}}{\pgfqpoint{3.696000in}{3.696000in}}%
\pgfusepath{clip}%
\pgfsetbuttcap%
\pgfsetroundjoin%
\definecolor{currentfill}{rgb}{0.121569,0.466667,0.705882}%
\pgfsetfillcolor{currentfill}%
\pgfsetfillopacity{0.991842}%
\pgfsetlinewidth{1.003750pt}%
\definecolor{currentstroke}{rgb}{0.121569,0.466667,0.705882}%
\pgfsetstrokecolor{currentstroke}%
\pgfsetstrokeopacity{0.991842}%
\pgfsetdash{}{0pt}%
\pgfpathmoveto{\pgfqpoint{2.505521in}{1.102009in}}%
\pgfpathcurveto{\pgfqpoint{2.513757in}{1.102009in}}{\pgfqpoint{2.521657in}{1.105281in}}{\pgfqpoint{2.527481in}{1.111105in}}%
\pgfpathcurveto{\pgfqpoint{2.533305in}{1.116929in}}{\pgfqpoint{2.536577in}{1.124829in}}{\pgfqpoint{2.536577in}{1.133065in}}%
\pgfpathcurveto{\pgfqpoint{2.536577in}{1.141302in}}{\pgfqpoint{2.533305in}{1.149202in}}{\pgfqpoint{2.527481in}{1.155026in}}%
\pgfpathcurveto{\pgfqpoint{2.521657in}{1.160850in}}{\pgfqpoint{2.513757in}{1.164122in}}{\pgfqpoint{2.505521in}{1.164122in}}%
\pgfpathcurveto{\pgfqpoint{2.497284in}{1.164122in}}{\pgfqpoint{2.489384in}{1.160850in}}{\pgfqpoint{2.483560in}{1.155026in}}%
\pgfpathcurveto{\pgfqpoint{2.477737in}{1.149202in}}{\pgfqpoint{2.474464in}{1.141302in}}{\pgfqpoint{2.474464in}{1.133065in}}%
\pgfpathcurveto{\pgfqpoint{2.474464in}{1.124829in}}{\pgfqpoint{2.477737in}{1.116929in}}{\pgfqpoint{2.483560in}{1.111105in}}%
\pgfpathcurveto{\pgfqpoint{2.489384in}{1.105281in}}{\pgfqpoint{2.497284in}{1.102009in}}{\pgfqpoint{2.505521in}{1.102009in}}%
\pgfpathclose%
\pgfusepath{stroke,fill}%
\end{pgfscope}%
\begin{pgfscope}%
\pgfpathrectangle{\pgfqpoint{0.100000in}{0.212622in}}{\pgfqpoint{3.696000in}{3.696000in}}%
\pgfusepath{clip}%
\pgfsetbuttcap%
\pgfsetroundjoin%
\definecolor{currentfill}{rgb}{0.121569,0.466667,0.705882}%
\pgfsetfillcolor{currentfill}%
\pgfsetfillopacity{0.991876}%
\pgfsetlinewidth{1.003750pt}%
\definecolor{currentstroke}{rgb}{0.121569,0.466667,0.705882}%
\pgfsetstrokecolor{currentstroke}%
\pgfsetstrokeopacity{0.991876}%
\pgfsetdash{}{0pt}%
\pgfpathmoveto{\pgfqpoint{2.281332in}{1.187613in}}%
\pgfpathcurveto{\pgfqpoint{2.289569in}{1.187613in}}{\pgfqpoint{2.297469in}{1.190885in}}{\pgfqpoint{2.303293in}{1.196709in}}%
\pgfpathcurveto{\pgfqpoint{2.309116in}{1.202533in}}{\pgfqpoint{2.312389in}{1.210433in}}{\pgfqpoint{2.312389in}{1.218669in}}%
\pgfpathcurveto{\pgfqpoint{2.312389in}{1.226905in}}{\pgfqpoint{2.309116in}{1.234805in}}{\pgfqpoint{2.303293in}{1.240629in}}%
\pgfpathcurveto{\pgfqpoint{2.297469in}{1.246453in}}{\pgfqpoint{2.289569in}{1.249726in}}{\pgfqpoint{2.281332in}{1.249726in}}%
\pgfpathcurveto{\pgfqpoint{2.273096in}{1.249726in}}{\pgfqpoint{2.265196in}{1.246453in}}{\pgfqpoint{2.259372in}{1.240629in}}%
\pgfpathcurveto{\pgfqpoint{2.253548in}{1.234805in}}{\pgfqpoint{2.250276in}{1.226905in}}{\pgfqpoint{2.250276in}{1.218669in}}%
\pgfpathcurveto{\pgfqpoint{2.250276in}{1.210433in}}{\pgfqpoint{2.253548in}{1.202533in}}{\pgfqpoint{2.259372in}{1.196709in}}%
\pgfpathcurveto{\pgfqpoint{2.265196in}{1.190885in}}{\pgfqpoint{2.273096in}{1.187613in}}{\pgfqpoint{2.281332in}{1.187613in}}%
\pgfpathclose%
\pgfusepath{stroke,fill}%
\end{pgfscope}%
\begin{pgfscope}%
\pgfpathrectangle{\pgfqpoint{0.100000in}{0.212622in}}{\pgfqpoint{3.696000in}{3.696000in}}%
\pgfusepath{clip}%
\pgfsetbuttcap%
\pgfsetroundjoin%
\definecolor{currentfill}{rgb}{0.121569,0.466667,0.705882}%
\pgfsetfillcolor{currentfill}%
\pgfsetfillopacity{0.992355}%
\pgfsetlinewidth{1.003750pt}%
\definecolor{currentstroke}{rgb}{0.121569,0.466667,0.705882}%
\pgfsetstrokecolor{currentstroke}%
\pgfsetstrokeopacity{0.992355}%
\pgfsetdash{}{0pt}%
\pgfpathmoveto{\pgfqpoint{2.289909in}{1.182451in}}%
\pgfpathcurveto{\pgfqpoint{2.298145in}{1.182451in}}{\pgfqpoint{2.306045in}{1.185723in}}{\pgfqpoint{2.311869in}{1.191547in}}%
\pgfpathcurveto{\pgfqpoint{2.317693in}{1.197371in}}{\pgfqpoint{2.320965in}{1.205271in}}{\pgfqpoint{2.320965in}{1.213507in}}%
\pgfpathcurveto{\pgfqpoint{2.320965in}{1.221744in}}{\pgfqpoint{2.317693in}{1.229644in}}{\pgfqpoint{2.311869in}{1.235468in}}%
\pgfpathcurveto{\pgfqpoint{2.306045in}{1.241292in}}{\pgfqpoint{2.298145in}{1.244564in}}{\pgfqpoint{2.289909in}{1.244564in}}%
\pgfpathcurveto{\pgfqpoint{2.281672in}{1.244564in}}{\pgfqpoint{2.273772in}{1.241292in}}{\pgfqpoint{2.267948in}{1.235468in}}%
\pgfpathcurveto{\pgfqpoint{2.262124in}{1.229644in}}{\pgfqpoint{2.258852in}{1.221744in}}{\pgfqpoint{2.258852in}{1.213507in}}%
\pgfpathcurveto{\pgfqpoint{2.258852in}{1.205271in}}{\pgfqpoint{2.262124in}{1.197371in}}{\pgfqpoint{2.267948in}{1.191547in}}%
\pgfpathcurveto{\pgfqpoint{2.273772in}{1.185723in}}{\pgfqpoint{2.281672in}{1.182451in}}{\pgfqpoint{2.289909in}{1.182451in}}%
\pgfpathclose%
\pgfusepath{stroke,fill}%
\end{pgfscope}%
\begin{pgfscope}%
\pgfpathrectangle{\pgfqpoint{0.100000in}{0.212622in}}{\pgfqpoint{3.696000in}{3.696000in}}%
\pgfusepath{clip}%
\pgfsetbuttcap%
\pgfsetroundjoin%
\definecolor{currentfill}{rgb}{0.121569,0.466667,0.705882}%
\pgfsetfillcolor{currentfill}%
\pgfsetfillopacity{0.992877}%
\pgfsetlinewidth{1.003750pt}%
\definecolor{currentstroke}{rgb}{0.121569,0.466667,0.705882}%
\pgfsetstrokecolor{currentstroke}%
\pgfsetstrokeopacity{0.992877}%
\pgfsetdash{}{0pt}%
\pgfpathmoveto{\pgfqpoint{2.300662in}{1.175367in}}%
\pgfpathcurveto{\pgfqpoint{2.308899in}{1.175367in}}{\pgfqpoint{2.316799in}{1.178640in}}{\pgfqpoint{2.322622in}{1.184464in}}%
\pgfpathcurveto{\pgfqpoint{2.328446in}{1.190288in}}{\pgfqpoint{2.331719in}{1.198188in}}{\pgfqpoint{2.331719in}{1.206424in}}%
\pgfpathcurveto{\pgfqpoint{2.331719in}{1.214660in}}{\pgfqpoint{2.328446in}{1.222560in}}{\pgfqpoint{2.322622in}{1.228384in}}%
\pgfpathcurveto{\pgfqpoint{2.316799in}{1.234208in}}{\pgfqpoint{2.308899in}{1.237480in}}{\pgfqpoint{2.300662in}{1.237480in}}%
\pgfpathcurveto{\pgfqpoint{2.292426in}{1.237480in}}{\pgfqpoint{2.284526in}{1.234208in}}{\pgfqpoint{2.278702in}{1.228384in}}%
\pgfpathcurveto{\pgfqpoint{2.272878in}{1.222560in}}{\pgfqpoint{2.269606in}{1.214660in}}{\pgfqpoint{2.269606in}{1.206424in}}%
\pgfpathcurveto{\pgfqpoint{2.269606in}{1.198188in}}{\pgfqpoint{2.272878in}{1.190288in}}{\pgfqpoint{2.278702in}{1.184464in}}%
\pgfpathcurveto{\pgfqpoint{2.284526in}{1.178640in}}{\pgfqpoint{2.292426in}{1.175367in}}{\pgfqpoint{2.300662in}{1.175367in}}%
\pgfpathclose%
\pgfusepath{stroke,fill}%
\end{pgfscope}%
\begin{pgfscope}%
\pgfpathrectangle{\pgfqpoint{0.100000in}{0.212622in}}{\pgfqpoint{3.696000in}{3.696000in}}%
\pgfusepath{clip}%
\pgfsetbuttcap%
\pgfsetroundjoin%
\definecolor{currentfill}{rgb}{0.121569,0.466667,0.705882}%
\pgfsetfillcolor{currentfill}%
\pgfsetfillopacity{0.993085}%
\pgfsetlinewidth{1.003750pt}%
\definecolor{currentstroke}{rgb}{0.121569,0.466667,0.705882}%
\pgfsetstrokecolor{currentstroke}%
\pgfsetstrokeopacity{0.993085}%
\pgfsetdash{}{0pt}%
\pgfpathmoveto{\pgfqpoint{2.306683in}{1.171458in}}%
\pgfpathcurveto{\pgfqpoint{2.314919in}{1.171458in}}{\pgfqpoint{2.322819in}{1.174730in}}{\pgfqpoint{2.328643in}{1.180554in}}%
\pgfpathcurveto{\pgfqpoint{2.334467in}{1.186378in}}{\pgfqpoint{2.337739in}{1.194278in}}{\pgfqpoint{2.337739in}{1.202514in}}%
\pgfpathcurveto{\pgfqpoint{2.337739in}{1.210750in}}{\pgfqpoint{2.334467in}{1.218650in}}{\pgfqpoint{2.328643in}{1.224474in}}%
\pgfpathcurveto{\pgfqpoint{2.322819in}{1.230298in}}{\pgfqpoint{2.314919in}{1.233571in}}{\pgfqpoint{2.306683in}{1.233571in}}%
\pgfpathcurveto{\pgfqpoint{2.298446in}{1.233571in}}{\pgfqpoint{2.290546in}{1.230298in}}{\pgfqpoint{2.284722in}{1.224474in}}%
\pgfpathcurveto{\pgfqpoint{2.278898in}{1.218650in}}{\pgfqpoint{2.275626in}{1.210750in}}{\pgfqpoint{2.275626in}{1.202514in}}%
\pgfpathcurveto{\pgfqpoint{2.275626in}{1.194278in}}{\pgfqpoint{2.278898in}{1.186378in}}{\pgfqpoint{2.284722in}{1.180554in}}%
\pgfpathcurveto{\pgfqpoint{2.290546in}{1.174730in}}{\pgfqpoint{2.298446in}{1.171458in}}{\pgfqpoint{2.306683in}{1.171458in}}%
\pgfpathclose%
\pgfusepath{stroke,fill}%
\end{pgfscope}%
\begin{pgfscope}%
\pgfpathrectangle{\pgfqpoint{0.100000in}{0.212622in}}{\pgfqpoint{3.696000in}{3.696000in}}%
\pgfusepath{clip}%
\pgfsetbuttcap%
\pgfsetroundjoin%
\definecolor{currentfill}{rgb}{0.121569,0.466667,0.705882}%
\pgfsetfillcolor{currentfill}%
\pgfsetfillopacity{0.993313}%
\pgfsetlinewidth{1.003750pt}%
\definecolor{currentstroke}{rgb}{0.121569,0.466667,0.705882}%
\pgfsetstrokecolor{currentstroke}%
\pgfsetstrokeopacity{0.993313}%
\pgfsetdash{}{0pt}%
\pgfpathmoveto{\pgfqpoint{2.313196in}{1.167363in}}%
\pgfpathcurveto{\pgfqpoint{2.321432in}{1.167363in}}{\pgfqpoint{2.329332in}{1.170636in}}{\pgfqpoint{2.335156in}{1.176459in}}%
\pgfpathcurveto{\pgfqpoint{2.340980in}{1.182283in}}{\pgfqpoint{2.344252in}{1.190183in}}{\pgfqpoint{2.344252in}{1.198420in}}%
\pgfpathcurveto{\pgfqpoint{2.344252in}{1.206656in}}{\pgfqpoint{2.340980in}{1.214556in}}{\pgfqpoint{2.335156in}{1.220380in}}%
\pgfpathcurveto{\pgfqpoint{2.329332in}{1.226204in}}{\pgfqpoint{2.321432in}{1.229476in}}{\pgfqpoint{2.313196in}{1.229476in}}%
\pgfpathcurveto{\pgfqpoint{2.304960in}{1.229476in}}{\pgfqpoint{2.297060in}{1.226204in}}{\pgfqpoint{2.291236in}{1.220380in}}%
\pgfpathcurveto{\pgfqpoint{2.285412in}{1.214556in}}{\pgfqpoint{2.282139in}{1.206656in}}{\pgfqpoint{2.282139in}{1.198420in}}%
\pgfpathcurveto{\pgfqpoint{2.282139in}{1.190183in}}{\pgfqpoint{2.285412in}{1.182283in}}{\pgfqpoint{2.291236in}{1.176459in}}%
\pgfpathcurveto{\pgfqpoint{2.297060in}{1.170636in}}{\pgfqpoint{2.304960in}{1.167363in}}{\pgfqpoint{2.313196in}{1.167363in}}%
\pgfpathclose%
\pgfusepath{stroke,fill}%
\end{pgfscope}%
\begin{pgfscope}%
\pgfpathrectangle{\pgfqpoint{0.100000in}{0.212622in}}{\pgfqpoint{3.696000in}{3.696000in}}%
\pgfusepath{clip}%
\pgfsetbuttcap%
\pgfsetroundjoin%
\definecolor{currentfill}{rgb}{0.121569,0.466667,0.705882}%
\pgfsetfillcolor{currentfill}%
\pgfsetfillopacity{0.993458}%
\pgfsetlinewidth{1.003750pt}%
\definecolor{currentstroke}{rgb}{0.121569,0.466667,0.705882}%
\pgfsetstrokecolor{currentstroke}%
\pgfsetstrokeopacity{0.993458}%
\pgfsetdash{}{0pt}%
\pgfpathmoveto{\pgfqpoint{2.502483in}{1.098617in}}%
\pgfpathcurveto{\pgfqpoint{2.510719in}{1.098617in}}{\pgfqpoint{2.518619in}{1.101890in}}{\pgfqpoint{2.524443in}{1.107713in}}%
\pgfpathcurveto{\pgfqpoint{2.530267in}{1.113537in}}{\pgfqpoint{2.533539in}{1.121437in}}{\pgfqpoint{2.533539in}{1.129674in}}%
\pgfpathcurveto{\pgfqpoint{2.533539in}{1.137910in}}{\pgfqpoint{2.530267in}{1.145810in}}{\pgfqpoint{2.524443in}{1.151634in}}%
\pgfpathcurveto{\pgfqpoint{2.518619in}{1.157458in}}{\pgfqpoint{2.510719in}{1.160730in}}{\pgfqpoint{2.502483in}{1.160730in}}%
\pgfpathcurveto{\pgfqpoint{2.494247in}{1.160730in}}{\pgfqpoint{2.486346in}{1.157458in}}{\pgfqpoint{2.480523in}{1.151634in}}%
\pgfpathcurveto{\pgfqpoint{2.474699in}{1.145810in}}{\pgfqpoint{2.471426in}{1.137910in}}{\pgfqpoint{2.471426in}{1.129674in}}%
\pgfpathcurveto{\pgfqpoint{2.471426in}{1.121437in}}{\pgfqpoint{2.474699in}{1.113537in}}{\pgfqpoint{2.480523in}{1.107713in}}%
\pgfpathcurveto{\pgfqpoint{2.486346in}{1.101890in}}{\pgfqpoint{2.494247in}{1.098617in}}{\pgfqpoint{2.502483in}{1.098617in}}%
\pgfpathclose%
\pgfusepath{stroke,fill}%
\end{pgfscope}%
\begin{pgfscope}%
\pgfpathrectangle{\pgfqpoint{0.100000in}{0.212622in}}{\pgfqpoint{3.696000in}{3.696000in}}%
\pgfusepath{clip}%
\pgfsetbuttcap%
\pgfsetroundjoin%
\definecolor{currentfill}{rgb}{0.121569,0.466667,0.705882}%
\pgfsetfillcolor{currentfill}%
\pgfsetfillopacity{0.993603}%
\pgfsetlinewidth{1.003750pt}%
\definecolor{currentstroke}{rgb}{0.121569,0.466667,0.705882}%
\pgfsetstrokecolor{currentstroke}%
\pgfsetstrokeopacity{0.993603}%
\pgfsetdash{}{0pt}%
\pgfpathmoveto{\pgfqpoint{2.320354in}{1.163142in}}%
\pgfpathcurveto{\pgfqpoint{2.328590in}{1.163142in}}{\pgfqpoint{2.336490in}{1.166414in}}{\pgfqpoint{2.342314in}{1.172238in}}%
\pgfpathcurveto{\pgfqpoint{2.348138in}{1.178062in}}{\pgfqpoint{2.351410in}{1.185962in}}{\pgfqpoint{2.351410in}{1.194199in}}%
\pgfpathcurveto{\pgfqpoint{2.351410in}{1.202435in}}{\pgfqpoint{2.348138in}{1.210335in}}{\pgfqpoint{2.342314in}{1.216159in}}%
\pgfpathcurveto{\pgfqpoint{2.336490in}{1.221983in}}{\pgfqpoint{2.328590in}{1.225255in}}{\pgfqpoint{2.320354in}{1.225255in}}%
\pgfpathcurveto{\pgfqpoint{2.312118in}{1.225255in}}{\pgfqpoint{2.304217in}{1.221983in}}{\pgfqpoint{2.298394in}{1.216159in}}%
\pgfpathcurveto{\pgfqpoint{2.292570in}{1.210335in}}{\pgfqpoint{2.289297in}{1.202435in}}{\pgfqpoint{2.289297in}{1.194199in}}%
\pgfpathcurveto{\pgfqpoint{2.289297in}{1.185962in}}{\pgfqpoint{2.292570in}{1.178062in}}{\pgfqpoint{2.298394in}{1.172238in}}%
\pgfpathcurveto{\pgfqpoint{2.304217in}{1.166414in}}{\pgfqpoint{2.312118in}{1.163142in}}{\pgfqpoint{2.320354in}{1.163142in}}%
\pgfpathclose%
\pgfusepath{stroke,fill}%
\end{pgfscope}%
\begin{pgfscope}%
\pgfpathrectangle{\pgfqpoint{0.100000in}{0.212622in}}{\pgfqpoint{3.696000in}{3.696000in}}%
\pgfusepath{clip}%
\pgfsetbuttcap%
\pgfsetroundjoin%
\definecolor{currentfill}{rgb}{0.121569,0.466667,0.705882}%
\pgfsetfillcolor{currentfill}%
\pgfsetfillopacity{0.993816}%
\pgfsetlinewidth{1.003750pt}%
\definecolor{currentstroke}{rgb}{0.121569,0.466667,0.705882}%
\pgfsetstrokecolor{currentstroke}%
\pgfsetstrokeopacity{0.993816}%
\pgfsetdash{}{0pt}%
\pgfpathmoveto{\pgfqpoint{2.324264in}{1.161002in}}%
\pgfpathcurveto{\pgfqpoint{2.332501in}{1.161002in}}{\pgfqpoint{2.340401in}{1.164274in}}{\pgfqpoint{2.346225in}{1.170098in}}%
\pgfpathcurveto{\pgfqpoint{2.352049in}{1.175922in}}{\pgfqpoint{2.355321in}{1.183822in}}{\pgfqpoint{2.355321in}{1.192059in}}%
\pgfpathcurveto{\pgfqpoint{2.355321in}{1.200295in}}{\pgfqpoint{2.352049in}{1.208195in}}{\pgfqpoint{2.346225in}{1.214019in}}%
\pgfpathcurveto{\pgfqpoint{2.340401in}{1.219843in}}{\pgfqpoint{2.332501in}{1.223115in}}{\pgfqpoint{2.324264in}{1.223115in}}%
\pgfpathcurveto{\pgfqpoint{2.316028in}{1.223115in}}{\pgfqpoint{2.308128in}{1.219843in}}{\pgfqpoint{2.302304in}{1.214019in}}%
\pgfpathcurveto{\pgfqpoint{2.296480in}{1.208195in}}{\pgfqpoint{2.293208in}{1.200295in}}{\pgfqpoint{2.293208in}{1.192059in}}%
\pgfpathcurveto{\pgfqpoint{2.293208in}{1.183822in}}{\pgfqpoint{2.296480in}{1.175922in}}{\pgfqpoint{2.302304in}{1.170098in}}%
\pgfpathcurveto{\pgfqpoint{2.308128in}{1.164274in}}{\pgfqpoint{2.316028in}{1.161002in}}{\pgfqpoint{2.324264in}{1.161002in}}%
\pgfpathclose%
\pgfusepath{stroke,fill}%
\end{pgfscope}%
\begin{pgfscope}%
\pgfpathrectangle{\pgfqpoint{0.100000in}{0.212622in}}{\pgfqpoint{3.696000in}{3.696000in}}%
\pgfusepath{clip}%
\pgfsetbuttcap%
\pgfsetroundjoin%
\definecolor{currentfill}{rgb}{0.121569,0.466667,0.705882}%
\pgfsetfillcolor{currentfill}%
\pgfsetfillopacity{0.994187}%
\pgfsetlinewidth{1.003750pt}%
\definecolor{currentstroke}{rgb}{0.121569,0.466667,0.705882}%
\pgfsetstrokecolor{currentstroke}%
\pgfsetstrokeopacity{0.994187}%
\pgfsetdash{}{0pt}%
\pgfpathmoveto{\pgfqpoint{2.329347in}{1.158463in}}%
\pgfpathcurveto{\pgfqpoint{2.337583in}{1.158463in}}{\pgfqpoint{2.345483in}{1.161735in}}{\pgfqpoint{2.351307in}{1.167559in}}%
\pgfpathcurveto{\pgfqpoint{2.357131in}{1.173383in}}{\pgfqpoint{2.360403in}{1.181283in}}{\pgfqpoint{2.360403in}{1.189520in}}%
\pgfpathcurveto{\pgfqpoint{2.360403in}{1.197756in}}{\pgfqpoint{2.357131in}{1.205656in}}{\pgfqpoint{2.351307in}{1.211480in}}%
\pgfpathcurveto{\pgfqpoint{2.345483in}{1.217304in}}{\pgfqpoint{2.337583in}{1.220576in}}{\pgfqpoint{2.329347in}{1.220576in}}%
\pgfpathcurveto{\pgfqpoint{2.321110in}{1.220576in}}{\pgfqpoint{2.313210in}{1.217304in}}{\pgfqpoint{2.307386in}{1.211480in}}%
\pgfpathcurveto{\pgfqpoint{2.301562in}{1.205656in}}{\pgfqpoint{2.298290in}{1.197756in}}{\pgfqpoint{2.298290in}{1.189520in}}%
\pgfpathcurveto{\pgfqpoint{2.298290in}{1.181283in}}{\pgfqpoint{2.301562in}{1.173383in}}{\pgfqpoint{2.307386in}{1.167559in}}%
\pgfpathcurveto{\pgfqpoint{2.313210in}{1.161735in}}{\pgfqpoint{2.321110in}{1.158463in}}{\pgfqpoint{2.329347in}{1.158463in}}%
\pgfpathclose%
\pgfusepath{stroke,fill}%
\end{pgfscope}%
\begin{pgfscope}%
\pgfpathrectangle{\pgfqpoint{0.100000in}{0.212622in}}{\pgfqpoint{3.696000in}{3.696000in}}%
\pgfusepath{clip}%
\pgfsetbuttcap%
\pgfsetroundjoin%
\definecolor{currentfill}{rgb}{0.121569,0.466667,0.705882}%
\pgfsetfillcolor{currentfill}%
\pgfsetfillopacity{0.994655}%
\pgfsetlinewidth{1.003750pt}%
\definecolor{currentstroke}{rgb}{0.121569,0.466667,0.705882}%
\pgfsetstrokecolor{currentstroke}%
\pgfsetstrokeopacity{0.994655}%
\pgfsetdash{}{0pt}%
\pgfpathmoveto{\pgfqpoint{2.334590in}{1.155562in}}%
\pgfpathcurveto{\pgfqpoint{2.342826in}{1.155562in}}{\pgfqpoint{2.350726in}{1.158834in}}{\pgfqpoint{2.356550in}{1.164658in}}%
\pgfpathcurveto{\pgfqpoint{2.362374in}{1.170482in}}{\pgfqpoint{2.365646in}{1.178382in}}{\pgfqpoint{2.365646in}{1.186618in}}%
\pgfpathcurveto{\pgfqpoint{2.365646in}{1.194855in}}{\pgfqpoint{2.362374in}{1.202755in}}{\pgfqpoint{2.356550in}{1.208579in}}%
\pgfpathcurveto{\pgfqpoint{2.350726in}{1.214403in}}{\pgfqpoint{2.342826in}{1.217675in}}{\pgfqpoint{2.334590in}{1.217675in}}%
\pgfpathcurveto{\pgfqpoint{2.326353in}{1.217675in}}{\pgfqpoint{2.318453in}{1.214403in}}{\pgfqpoint{2.312629in}{1.208579in}}%
\pgfpathcurveto{\pgfqpoint{2.306806in}{1.202755in}}{\pgfqpoint{2.303533in}{1.194855in}}{\pgfqpoint{2.303533in}{1.186618in}}%
\pgfpathcurveto{\pgfqpoint{2.303533in}{1.178382in}}{\pgfqpoint{2.306806in}{1.170482in}}{\pgfqpoint{2.312629in}{1.164658in}}%
\pgfpathcurveto{\pgfqpoint{2.318453in}{1.158834in}}{\pgfqpoint{2.326353in}{1.155562in}}{\pgfqpoint{2.334590in}{1.155562in}}%
\pgfpathclose%
\pgfusepath{stroke,fill}%
\end{pgfscope}%
\begin{pgfscope}%
\pgfpathrectangle{\pgfqpoint{0.100000in}{0.212622in}}{\pgfqpoint{3.696000in}{3.696000in}}%
\pgfusepath{clip}%
\pgfsetbuttcap%
\pgfsetroundjoin%
\definecolor{currentfill}{rgb}{0.121569,0.466667,0.705882}%
\pgfsetfillcolor{currentfill}%
\pgfsetfillopacity{0.994985}%
\pgfsetlinewidth{1.003750pt}%
\definecolor{currentstroke}{rgb}{0.121569,0.466667,0.705882}%
\pgfsetstrokecolor{currentstroke}%
\pgfsetstrokeopacity{0.994985}%
\pgfsetdash{}{0pt}%
\pgfpathmoveto{\pgfqpoint{2.499595in}{1.095599in}}%
\pgfpathcurveto{\pgfqpoint{2.507831in}{1.095599in}}{\pgfqpoint{2.515731in}{1.098871in}}{\pgfqpoint{2.521555in}{1.104695in}}%
\pgfpathcurveto{\pgfqpoint{2.527379in}{1.110519in}}{\pgfqpoint{2.530651in}{1.118419in}}{\pgfqpoint{2.530651in}{1.126656in}}%
\pgfpathcurveto{\pgfqpoint{2.530651in}{1.134892in}}{\pgfqpoint{2.527379in}{1.142792in}}{\pgfqpoint{2.521555in}{1.148616in}}%
\pgfpathcurveto{\pgfqpoint{2.515731in}{1.154440in}}{\pgfqpoint{2.507831in}{1.157712in}}{\pgfqpoint{2.499595in}{1.157712in}}%
\pgfpathcurveto{\pgfqpoint{2.491359in}{1.157712in}}{\pgfqpoint{2.483459in}{1.154440in}}{\pgfqpoint{2.477635in}{1.148616in}}%
\pgfpathcurveto{\pgfqpoint{2.471811in}{1.142792in}}{\pgfqpoint{2.468538in}{1.134892in}}{\pgfqpoint{2.468538in}{1.126656in}}%
\pgfpathcurveto{\pgfqpoint{2.468538in}{1.118419in}}{\pgfqpoint{2.471811in}{1.110519in}}{\pgfqpoint{2.477635in}{1.104695in}}%
\pgfpathcurveto{\pgfqpoint{2.483459in}{1.098871in}}{\pgfqpoint{2.491359in}{1.095599in}}{\pgfqpoint{2.499595in}{1.095599in}}%
\pgfpathclose%
\pgfusepath{stroke,fill}%
\end{pgfscope}%
\begin{pgfscope}%
\pgfpathrectangle{\pgfqpoint{0.100000in}{0.212622in}}{\pgfqpoint{3.696000in}{3.696000in}}%
\pgfusepath{clip}%
\pgfsetbuttcap%
\pgfsetroundjoin%
\definecolor{currentfill}{rgb}{0.121569,0.466667,0.705882}%
\pgfsetfillcolor{currentfill}%
\pgfsetfillopacity{0.995091}%
\pgfsetlinewidth{1.003750pt}%
\definecolor{currentstroke}{rgb}{0.121569,0.466667,0.705882}%
\pgfsetstrokecolor{currentstroke}%
\pgfsetstrokeopacity{0.995091}%
\pgfsetdash{}{0pt}%
\pgfpathmoveto{\pgfqpoint{2.340895in}{1.151511in}}%
\pgfpathcurveto{\pgfqpoint{2.349132in}{1.151511in}}{\pgfqpoint{2.357032in}{1.154784in}}{\pgfqpoint{2.362856in}{1.160608in}}%
\pgfpathcurveto{\pgfqpoint{2.368680in}{1.166432in}}{\pgfqpoint{2.371952in}{1.174332in}}{\pgfqpoint{2.371952in}{1.182568in}}%
\pgfpathcurveto{\pgfqpoint{2.371952in}{1.190804in}}{\pgfqpoint{2.368680in}{1.198704in}}{\pgfqpoint{2.362856in}{1.204528in}}%
\pgfpathcurveto{\pgfqpoint{2.357032in}{1.210352in}}{\pgfqpoint{2.349132in}{1.213624in}}{\pgfqpoint{2.340895in}{1.213624in}}%
\pgfpathcurveto{\pgfqpoint{2.332659in}{1.213624in}}{\pgfqpoint{2.324759in}{1.210352in}}{\pgfqpoint{2.318935in}{1.204528in}}%
\pgfpathcurveto{\pgfqpoint{2.313111in}{1.198704in}}{\pgfqpoint{2.309839in}{1.190804in}}{\pgfqpoint{2.309839in}{1.182568in}}%
\pgfpathcurveto{\pgfqpoint{2.309839in}{1.174332in}}{\pgfqpoint{2.313111in}{1.166432in}}{\pgfqpoint{2.318935in}{1.160608in}}%
\pgfpathcurveto{\pgfqpoint{2.324759in}{1.154784in}}{\pgfqpoint{2.332659in}{1.151511in}}{\pgfqpoint{2.340895in}{1.151511in}}%
\pgfpathclose%
\pgfusepath{stroke,fill}%
\end{pgfscope}%
\begin{pgfscope}%
\pgfpathrectangle{\pgfqpoint{0.100000in}{0.212622in}}{\pgfqpoint{3.696000in}{3.696000in}}%
\pgfusepath{clip}%
\pgfsetbuttcap%
\pgfsetroundjoin%
\definecolor{currentfill}{rgb}{0.121569,0.466667,0.705882}%
\pgfsetfillcolor{currentfill}%
\pgfsetfillopacity{0.995410}%
\pgfsetlinewidth{1.003750pt}%
\definecolor{currentstroke}{rgb}{0.121569,0.466667,0.705882}%
\pgfsetstrokecolor{currentstroke}%
\pgfsetstrokeopacity{0.995410}%
\pgfsetdash{}{0pt}%
\pgfpathmoveto{\pgfqpoint{2.347768in}{1.147566in}}%
\pgfpathcurveto{\pgfqpoint{2.356004in}{1.147566in}}{\pgfqpoint{2.363904in}{1.150838in}}{\pgfqpoint{2.369728in}{1.156662in}}%
\pgfpathcurveto{\pgfqpoint{2.375552in}{1.162486in}}{\pgfqpoint{2.378824in}{1.170386in}}{\pgfqpoint{2.378824in}{1.178622in}}%
\pgfpathcurveto{\pgfqpoint{2.378824in}{1.186859in}}{\pgfqpoint{2.375552in}{1.194759in}}{\pgfqpoint{2.369728in}{1.200583in}}%
\pgfpathcurveto{\pgfqpoint{2.363904in}{1.206407in}}{\pgfqpoint{2.356004in}{1.209679in}}{\pgfqpoint{2.347768in}{1.209679in}}%
\pgfpathcurveto{\pgfqpoint{2.339531in}{1.209679in}}{\pgfqpoint{2.331631in}{1.206407in}}{\pgfqpoint{2.325807in}{1.200583in}}%
\pgfpathcurveto{\pgfqpoint{2.319983in}{1.194759in}}{\pgfqpoint{2.316711in}{1.186859in}}{\pgfqpoint{2.316711in}{1.178622in}}%
\pgfpathcurveto{\pgfqpoint{2.316711in}{1.170386in}}{\pgfqpoint{2.319983in}{1.162486in}}{\pgfqpoint{2.325807in}{1.156662in}}%
\pgfpathcurveto{\pgfqpoint{2.331631in}{1.150838in}}{\pgfqpoint{2.339531in}{1.147566in}}{\pgfqpoint{2.347768in}{1.147566in}}%
\pgfpathclose%
\pgfusepath{stroke,fill}%
\end{pgfscope}%
\begin{pgfscope}%
\pgfpathrectangle{\pgfqpoint{0.100000in}{0.212622in}}{\pgfqpoint{3.696000in}{3.696000in}}%
\pgfusepath{clip}%
\pgfsetbuttcap%
\pgfsetroundjoin%
\definecolor{currentfill}{rgb}{0.121569,0.466667,0.705882}%
\pgfsetfillcolor{currentfill}%
\pgfsetfillopacity{0.995650}%
\pgfsetlinewidth{1.003750pt}%
\definecolor{currentstroke}{rgb}{0.121569,0.466667,0.705882}%
\pgfsetstrokecolor{currentstroke}%
\pgfsetstrokeopacity{0.995650}%
\pgfsetdash{}{0pt}%
\pgfpathmoveto{\pgfqpoint{2.355910in}{1.142605in}}%
\pgfpathcurveto{\pgfqpoint{2.364146in}{1.142605in}}{\pgfqpoint{2.372046in}{1.145877in}}{\pgfqpoint{2.377870in}{1.151701in}}%
\pgfpathcurveto{\pgfqpoint{2.383694in}{1.157525in}}{\pgfqpoint{2.386967in}{1.165425in}}{\pgfqpoint{2.386967in}{1.173661in}}%
\pgfpathcurveto{\pgfqpoint{2.386967in}{1.181898in}}{\pgfqpoint{2.383694in}{1.189798in}}{\pgfqpoint{2.377870in}{1.195622in}}%
\pgfpathcurveto{\pgfqpoint{2.372046in}{1.201445in}}{\pgfqpoint{2.364146in}{1.204718in}}{\pgfqpoint{2.355910in}{1.204718in}}%
\pgfpathcurveto{\pgfqpoint{2.347674in}{1.204718in}}{\pgfqpoint{2.339774in}{1.201445in}}{\pgfqpoint{2.333950in}{1.195622in}}%
\pgfpathcurveto{\pgfqpoint{2.328126in}{1.189798in}}{\pgfqpoint{2.324854in}{1.181898in}}{\pgfqpoint{2.324854in}{1.173661in}}%
\pgfpathcurveto{\pgfqpoint{2.324854in}{1.165425in}}{\pgfqpoint{2.328126in}{1.157525in}}{\pgfqpoint{2.333950in}{1.151701in}}%
\pgfpathcurveto{\pgfqpoint{2.339774in}{1.145877in}}{\pgfqpoint{2.347674in}{1.142605in}}{\pgfqpoint{2.355910in}{1.142605in}}%
\pgfpathclose%
\pgfusepath{stroke,fill}%
\end{pgfscope}%
\begin{pgfscope}%
\pgfpathrectangle{\pgfqpoint{0.100000in}{0.212622in}}{\pgfqpoint{3.696000in}{3.696000in}}%
\pgfusepath{clip}%
\pgfsetbuttcap%
\pgfsetroundjoin%
\definecolor{currentfill}{rgb}{0.121569,0.466667,0.705882}%
\pgfsetfillcolor{currentfill}%
\pgfsetfillopacity{0.996210}%
\pgfsetlinewidth{1.003750pt}%
\definecolor{currentstroke}{rgb}{0.121569,0.466667,0.705882}%
\pgfsetstrokecolor{currentstroke}%
\pgfsetstrokeopacity{0.996210}%
\pgfsetdash{}{0pt}%
\pgfpathmoveto{\pgfqpoint{2.363948in}{1.137619in}}%
\pgfpathcurveto{\pgfqpoint{2.372184in}{1.137619in}}{\pgfqpoint{2.380084in}{1.140892in}}{\pgfqpoint{2.385908in}{1.146715in}}%
\pgfpathcurveto{\pgfqpoint{2.391732in}{1.152539in}}{\pgfqpoint{2.395004in}{1.160439in}}{\pgfqpoint{2.395004in}{1.168676in}}%
\pgfpathcurveto{\pgfqpoint{2.395004in}{1.176912in}}{\pgfqpoint{2.391732in}{1.184812in}}{\pgfqpoint{2.385908in}{1.190636in}}%
\pgfpathcurveto{\pgfqpoint{2.380084in}{1.196460in}}{\pgfqpoint{2.372184in}{1.199732in}}{\pgfqpoint{2.363948in}{1.199732in}}%
\pgfpathcurveto{\pgfqpoint{2.355712in}{1.199732in}}{\pgfqpoint{2.347812in}{1.196460in}}{\pgfqpoint{2.341988in}{1.190636in}}%
\pgfpathcurveto{\pgfqpoint{2.336164in}{1.184812in}}{\pgfqpoint{2.332891in}{1.176912in}}{\pgfqpoint{2.332891in}{1.168676in}}%
\pgfpathcurveto{\pgfqpoint{2.332891in}{1.160439in}}{\pgfqpoint{2.336164in}{1.152539in}}{\pgfqpoint{2.341988in}{1.146715in}}%
\pgfpathcurveto{\pgfqpoint{2.347812in}{1.140892in}}{\pgfqpoint{2.355712in}{1.137619in}}{\pgfqpoint{2.363948in}{1.137619in}}%
\pgfpathclose%
\pgfusepath{stroke,fill}%
\end{pgfscope}%
\begin{pgfscope}%
\pgfpathrectangle{\pgfqpoint{0.100000in}{0.212622in}}{\pgfqpoint{3.696000in}{3.696000in}}%
\pgfusepath{clip}%
\pgfsetbuttcap%
\pgfsetroundjoin%
\definecolor{currentfill}{rgb}{0.121569,0.466667,0.705882}%
\pgfsetfillcolor{currentfill}%
\pgfsetfillopacity{0.996361}%
\pgfsetlinewidth{1.003750pt}%
\definecolor{currentstroke}{rgb}{0.121569,0.466667,0.705882}%
\pgfsetstrokecolor{currentstroke}%
\pgfsetstrokeopacity{0.996361}%
\pgfsetdash{}{0pt}%
\pgfpathmoveto{\pgfqpoint{2.497086in}{1.093096in}}%
\pgfpathcurveto{\pgfqpoint{2.505322in}{1.093096in}}{\pgfqpoint{2.513222in}{1.096369in}}{\pgfqpoint{2.519046in}{1.102193in}}%
\pgfpathcurveto{\pgfqpoint{2.524870in}{1.108017in}}{\pgfqpoint{2.528143in}{1.115917in}}{\pgfqpoint{2.528143in}{1.124153in}}%
\pgfpathcurveto{\pgfqpoint{2.528143in}{1.132389in}}{\pgfqpoint{2.524870in}{1.140289in}}{\pgfqpoint{2.519046in}{1.146113in}}%
\pgfpathcurveto{\pgfqpoint{2.513222in}{1.151937in}}{\pgfqpoint{2.505322in}{1.155209in}}{\pgfqpoint{2.497086in}{1.155209in}}%
\pgfpathcurveto{\pgfqpoint{2.488850in}{1.155209in}}{\pgfqpoint{2.480950in}{1.151937in}}{\pgfqpoint{2.475126in}{1.146113in}}%
\pgfpathcurveto{\pgfqpoint{2.469302in}{1.140289in}}{\pgfqpoint{2.466030in}{1.132389in}}{\pgfqpoint{2.466030in}{1.124153in}}%
\pgfpathcurveto{\pgfqpoint{2.466030in}{1.115917in}}{\pgfqpoint{2.469302in}{1.108017in}}{\pgfqpoint{2.475126in}{1.102193in}}%
\pgfpathcurveto{\pgfqpoint{2.480950in}{1.096369in}}{\pgfqpoint{2.488850in}{1.093096in}}{\pgfqpoint{2.497086in}{1.093096in}}%
\pgfpathclose%
\pgfusepath{stroke,fill}%
\end{pgfscope}%
\begin{pgfscope}%
\pgfpathrectangle{\pgfqpoint{0.100000in}{0.212622in}}{\pgfqpoint{3.696000in}{3.696000in}}%
\pgfusepath{clip}%
\pgfsetbuttcap%
\pgfsetroundjoin%
\definecolor{currentfill}{rgb}{0.121569,0.466667,0.705882}%
\pgfsetfillcolor{currentfill}%
\pgfsetfillopacity{0.996582}%
\pgfsetlinewidth{1.003750pt}%
\definecolor{currentstroke}{rgb}{0.121569,0.466667,0.705882}%
\pgfsetstrokecolor{currentstroke}%
\pgfsetstrokeopacity{0.996582}%
\pgfsetdash{}{0pt}%
\pgfpathmoveto{\pgfqpoint{2.373721in}{1.132561in}}%
\pgfpathcurveto{\pgfqpoint{2.381957in}{1.132561in}}{\pgfqpoint{2.389857in}{1.135833in}}{\pgfqpoint{2.395681in}{1.141657in}}%
\pgfpathcurveto{\pgfqpoint{2.401505in}{1.147481in}}{\pgfqpoint{2.404778in}{1.155381in}}{\pgfqpoint{2.404778in}{1.163617in}}%
\pgfpathcurveto{\pgfqpoint{2.404778in}{1.171853in}}{\pgfqpoint{2.401505in}{1.179753in}}{\pgfqpoint{2.395681in}{1.185577in}}%
\pgfpathcurveto{\pgfqpoint{2.389857in}{1.191401in}}{\pgfqpoint{2.381957in}{1.194674in}}{\pgfqpoint{2.373721in}{1.194674in}}%
\pgfpathcurveto{\pgfqpoint{2.365485in}{1.194674in}}{\pgfqpoint{2.357585in}{1.191401in}}{\pgfqpoint{2.351761in}{1.185577in}}%
\pgfpathcurveto{\pgfqpoint{2.345937in}{1.179753in}}{\pgfqpoint{2.342665in}{1.171853in}}{\pgfqpoint{2.342665in}{1.163617in}}%
\pgfpathcurveto{\pgfqpoint{2.342665in}{1.155381in}}{\pgfqpoint{2.345937in}{1.147481in}}{\pgfqpoint{2.351761in}{1.141657in}}%
\pgfpathcurveto{\pgfqpoint{2.357585in}{1.135833in}}{\pgfqpoint{2.365485in}{1.132561in}}{\pgfqpoint{2.373721in}{1.132561in}}%
\pgfpathclose%
\pgfusepath{stroke,fill}%
\end{pgfscope}%
\begin{pgfscope}%
\pgfpathrectangle{\pgfqpoint{0.100000in}{0.212622in}}{\pgfqpoint{3.696000in}{3.696000in}}%
\pgfusepath{clip}%
\pgfsetbuttcap%
\pgfsetroundjoin%
\definecolor{currentfill}{rgb}{0.121569,0.466667,0.705882}%
\pgfsetfillcolor{currentfill}%
\pgfsetfillopacity{0.997343}%
\pgfsetlinewidth{1.003750pt}%
\definecolor{currentstroke}{rgb}{0.121569,0.466667,0.705882}%
\pgfsetstrokecolor{currentstroke}%
\pgfsetstrokeopacity{0.997343}%
\pgfsetdash{}{0pt}%
\pgfpathmoveto{\pgfqpoint{2.383428in}{1.127419in}}%
\pgfpathcurveto{\pgfqpoint{2.391665in}{1.127419in}}{\pgfqpoint{2.399565in}{1.130691in}}{\pgfqpoint{2.405389in}{1.136515in}}%
\pgfpathcurveto{\pgfqpoint{2.411213in}{1.142339in}}{\pgfqpoint{2.414485in}{1.150239in}}{\pgfqpoint{2.414485in}{1.158475in}}%
\pgfpathcurveto{\pgfqpoint{2.414485in}{1.166711in}}{\pgfqpoint{2.411213in}{1.174611in}}{\pgfqpoint{2.405389in}{1.180435in}}%
\pgfpathcurveto{\pgfqpoint{2.399565in}{1.186259in}}{\pgfqpoint{2.391665in}{1.189532in}}{\pgfqpoint{2.383428in}{1.189532in}}%
\pgfpathcurveto{\pgfqpoint{2.375192in}{1.189532in}}{\pgfqpoint{2.367292in}{1.186259in}}{\pgfqpoint{2.361468in}{1.180435in}}%
\pgfpathcurveto{\pgfqpoint{2.355644in}{1.174611in}}{\pgfqpoint{2.352372in}{1.166711in}}{\pgfqpoint{2.352372in}{1.158475in}}%
\pgfpathcurveto{\pgfqpoint{2.352372in}{1.150239in}}{\pgfqpoint{2.355644in}{1.142339in}}{\pgfqpoint{2.361468in}{1.136515in}}%
\pgfpathcurveto{\pgfqpoint{2.367292in}{1.130691in}}{\pgfqpoint{2.375192in}{1.127419in}}{\pgfqpoint{2.383428in}{1.127419in}}%
\pgfpathclose%
\pgfusepath{stroke,fill}%
\end{pgfscope}%
\begin{pgfscope}%
\pgfpathrectangle{\pgfqpoint{0.100000in}{0.212622in}}{\pgfqpoint{3.696000in}{3.696000in}}%
\pgfusepath{clip}%
\pgfsetbuttcap%
\pgfsetroundjoin%
\definecolor{currentfill}{rgb}{0.121569,0.466667,0.705882}%
\pgfsetfillcolor{currentfill}%
\pgfsetfillopacity{0.997346}%
\pgfsetlinewidth{1.003750pt}%
\definecolor{currentstroke}{rgb}{0.121569,0.466667,0.705882}%
\pgfsetstrokecolor{currentstroke}%
\pgfsetstrokeopacity{0.997346}%
\pgfsetdash{}{0pt}%
\pgfpathmoveto{\pgfqpoint{2.495368in}{1.091466in}}%
\pgfpathcurveto{\pgfqpoint{2.503604in}{1.091466in}}{\pgfqpoint{2.511504in}{1.094738in}}{\pgfqpoint{2.517328in}{1.100562in}}%
\pgfpathcurveto{\pgfqpoint{2.523152in}{1.106386in}}{\pgfqpoint{2.526425in}{1.114286in}}{\pgfqpoint{2.526425in}{1.122522in}}%
\pgfpathcurveto{\pgfqpoint{2.526425in}{1.130759in}}{\pgfqpoint{2.523152in}{1.138659in}}{\pgfqpoint{2.517328in}{1.144483in}}%
\pgfpathcurveto{\pgfqpoint{2.511504in}{1.150307in}}{\pgfqpoint{2.503604in}{1.153579in}}{\pgfqpoint{2.495368in}{1.153579in}}%
\pgfpathcurveto{\pgfqpoint{2.487132in}{1.153579in}}{\pgfqpoint{2.479232in}{1.150307in}}{\pgfqpoint{2.473408in}{1.144483in}}%
\pgfpathcurveto{\pgfqpoint{2.467584in}{1.138659in}}{\pgfqpoint{2.464312in}{1.130759in}}{\pgfqpoint{2.464312in}{1.122522in}}%
\pgfpathcurveto{\pgfqpoint{2.464312in}{1.114286in}}{\pgfqpoint{2.467584in}{1.106386in}}{\pgfqpoint{2.473408in}{1.100562in}}%
\pgfpathcurveto{\pgfqpoint{2.479232in}{1.094738in}}{\pgfqpoint{2.487132in}{1.091466in}}{\pgfqpoint{2.495368in}{1.091466in}}%
\pgfpathclose%
\pgfusepath{stroke,fill}%
\end{pgfscope}%
\begin{pgfscope}%
\pgfpathrectangle{\pgfqpoint{0.100000in}{0.212622in}}{\pgfqpoint{3.696000in}{3.696000in}}%
\pgfusepath{clip}%
\pgfsetbuttcap%
\pgfsetroundjoin%
\definecolor{currentfill}{rgb}{0.121569,0.466667,0.705882}%
\pgfsetfillcolor{currentfill}%
\pgfsetfillopacity{0.997894}%
\pgfsetlinewidth{1.003750pt}%
\definecolor{currentstroke}{rgb}{0.121569,0.466667,0.705882}%
\pgfsetstrokecolor{currentstroke}%
\pgfsetstrokeopacity{0.997894}%
\pgfsetdash{}{0pt}%
\pgfpathmoveto{\pgfqpoint{2.394669in}{1.121323in}}%
\pgfpathcurveto{\pgfqpoint{2.402905in}{1.121323in}}{\pgfqpoint{2.410805in}{1.124595in}}{\pgfqpoint{2.416629in}{1.130419in}}%
\pgfpathcurveto{\pgfqpoint{2.422453in}{1.136243in}}{\pgfqpoint{2.425726in}{1.144143in}}{\pgfqpoint{2.425726in}{1.152380in}}%
\pgfpathcurveto{\pgfqpoint{2.425726in}{1.160616in}}{\pgfqpoint{2.422453in}{1.168516in}}{\pgfqpoint{2.416629in}{1.174340in}}%
\pgfpathcurveto{\pgfqpoint{2.410805in}{1.180164in}}{\pgfqpoint{2.402905in}{1.183436in}}{\pgfqpoint{2.394669in}{1.183436in}}%
\pgfpathcurveto{\pgfqpoint{2.386433in}{1.183436in}}{\pgfqpoint{2.378533in}{1.180164in}}{\pgfqpoint{2.372709in}{1.174340in}}%
\pgfpathcurveto{\pgfqpoint{2.366885in}{1.168516in}}{\pgfqpoint{2.363613in}{1.160616in}}{\pgfqpoint{2.363613in}{1.152380in}}%
\pgfpathcurveto{\pgfqpoint{2.363613in}{1.144143in}}{\pgfqpoint{2.366885in}{1.136243in}}{\pgfqpoint{2.372709in}{1.130419in}}%
\pgfpathcurveto{\pgfqpoint{2.378533in}{1.124595in}}{\pgfqpoint{2.386433in}{1.121323in}}{\pgfqpoint{2.394669in}{1.121323in}}%
\pgfpathclose%
\pgfusepath{stroke,fill}%
\end{pgfscope}%
\begin{pgfscope}%
\pgfpathrectangle{\pgfqpoint{0.100000in}{0.212622in}}{\pgfqpoint{3.696000in}{3.696000in}}%
\pgfusepath{clip}%
\pgfsetbuttcap%
\pgfsetroundjoin%
\definecolor{currentfill}{rgb}{0.121569,0.466667,0.705882}%
\pgfsetfillcolor{currentfill}%
\pgfsetfillopacity{0.997932}%
\pgfsetlinewidth{1.003750pt}%
\definecolor{currentstroke}{rgb}{0.121569,0.466667,0.705882}%
\pgfsetstrokecolor{currentstroke}%
\pgfsetstrokeopacity{0.997932}%
\pgfsetdash{}{0pt}%
\pgfpathmoveto{\pgfqpoint{2.494260in}{1.090490in}}%
\pgfpathcurveto{\pgfqpoint{2.502496in}{1.090490in}}{\pgfqpoint{2.510396in}{1.093762in}}{\pgfqpoint{2.516220in}{1.099586in}}%
\pgfpathcurveto{\pgfqpoint{2.522044in}{1.105410in}}{\pgfqpoint{2.525316in}{1.113310in}}{\pgfqpoint{2.525316in}{1.121546in}}%
\pgfpathcurveto{\pgfqpoint{2.525316in}{1.129782in}}{\pgfqpoint{2.522044in}{1.137682in}}{\pgfqpoint{2.516220in}{1.143506in}}%
\pgfpathcurveto{\pgfqpoint{2.510396in}{1.149330in}}{\pgfqpoint{2.502496in}{1.152603in}}{\pgfqpoint{2.494260in}{1.152603in}}%
\pgfpathcurveto{\pgfqpoint{2.486024in}{1.152603in}}{\pgfqpoint{2.478124in}{1.149330in}}{\pgfqpoint{2.472300in}{1.143506in}}%
\pgfpathcurveto{\pgfqpoint{2.466476in}{1.137682in}}{\pgfqpoint{2.463203in}{1.129782in}}{\pgfqpoint{2.463203in}{1.121546in}}%
\pgfpathcurveto{\pgfqpoint{2.463203in}{1.113310in}}{\pgfqpoint{2.466476in}{1.105410in}}{\pgfqpoint{2.472300in}{1.099586in}}%
\pgfpathcurveto{\pgfqpoint{2.478124in}{1.093762in}}{\pgfqpoint{2.486024in}{1.090490in}}{\pgfqpoint{2.494260in}{1.090490in}}%
\pgfpathclose%
\pgfusepath{stroke,fill}%
\end{pgfscope}%
\begin{pgfscope}%
\pgfpathrectangle{\pgfqpoint{0.100000in}{0.212622in}}{\pgfqpoint{3.696000in}{3.696000in}}%
\pgfusepath{clip}%
\pgfsetbuttcap%
\pgfsetroundjoin%
\definecolor{currentfill}{rgb}{0.121569,0.466667,0.705882}%
\pgfsetfillcolor{currentfill}%
\pgfsetfillopacity{0.998033}%
\pgfsetlinewidth{1.003750pt}%
\definecolor{currentstroke}{rgb}{0.121569,0.466667,0.705882}%
\pgfsetstrokecolor{currentstroke}%
\pgfsetstrokeopacity{0.998033}%
\pgfsetdash{}{0pt}%
\pgfpathmoveto{\pgfqpoint{2.494064in}{1.090316in}}%
\pgfpathcurveto{\pgfqpoint{2.502300in}{1.090316in}}{\pgfqpoint{2.510200in}{1.093588in}}{\pgfqpoint{2.516024in}{1.099412in}}%
\pgfpathcurveto{\pgfqpoint{2.521848in}{1.105236in}}{\pgfqpoint{2.525120in}{1.113136in}}{\pgfqpoint{2.525120in}{1.121372in}}%
\pgfpathcurveto{\pgfqpoint{2.525120in}{1.129609in}}{\pgfqpoint{2.521848in}{1.137509in}}{\pgfqpoint{2.516024in}{1.143333in}}%
\pgfpathcurveto{\pgfqpoint{2.510200in}{1.149157in}}{\pgfqpoint{2.502300in}{1.152429in}}{\pgfqpoint{2.494064in}{1.152429in}}%
\pgfpathcurveto{\pgfqpoint{2.485827in}{1.152429in}}{\pgfqpoint{2.477927in}{1.149157in}}{\pgfqpoint{2.472103in}{1.143333in}}%
\pgfpathcurveto{\pgfqpoint{2.466280in}{1.137509in}}{\pgfqpoint{2.463007in}{1.129609in}}{\pgfqpoint{2.463007in}{1.121372in}}%
\pgfpathcurveto{\pgfqpoint{2.463007in}{1.113136in}}{\pgfqpoint{2.466280in}{1.105236in}}{\pgfqpoint{2.472103in}{1.099412in}}%
\pgfpathcurveto{\pgfqpoint{2.477927in}{1.093588in}}{\pgfqpoint{2.485827in}{1.090316in}}{\pgfqpoint{2.494064in}{1.090316in}}%
\pgfpathclose%
\pgfusepath{stroke,fill}%
\end{pgfscope}%
\begin{pgfscope}%
\pgfpathrectangle{\pgfqpoint{0.100000in}{0.212622in}}{\pgfqpoint{3.696000in}{3.696000in}}%
\pgfusepath{clip}%
\pgfsetbuttcap%
\pgfsetroundjoin%
\definecolor{currentfill}{rgb}{0.121569,0.466667,0.705882}%
\pgfsetfillcolor{currentfill}%
\pgfsetfillopacity{0.998214}%
\pgfsetlinewidth{1.003750pt}%
\definecolor{currentstroke}{rgb}{0.121569,0.466667,0.705882}%
\pgfsetstrokecolor{currentstroke}%
\pgfsetstrokeopacity{0.998214}%
\pgfsetdash{}{0pt}%
\pgfpathmoveto{\pgfqpoint{2.493692in}{1.089996in}}%
\pgfpathcurveto{\pgfqpoint{2.501928in}{1.089996in}}{\pgfqpoint{2.509828in}{1.093268in}}{\pgfqpoint{2.515652in}{1.099092in}}%
\pgfpathcurveto{\pgfqpoint{2.521476in}{1.104916in}}{\pgfqpoint{2.524748in}{1.112816in}}{\pgfqpoint{2.524748in}{1.121052in}}%
\pgfpathcurveto{\pgfqpoint{2.524748in}{1.129288in}}{\pgfqpoint{2.521476in}{1.137188in}}{\pgfqpoint{2.515652in}{1.143012in}}%
\pgfpathcurveto{\pgfqpoint{2.509828in}{1.148836in}}{\pgfqpoint{2.501928in}{1.152109in}}{\pgfqpoint{2.493692in}{1.152109in}}%
\pgfpathcurveto{\pgfqpoint{2.485455in}{1.152109in}}{\pgfqpoint{2.477555in}{1.148836in}}{\pgfqpoint{2.471731in}{1.143012in}}%
\pgfpathcurveto{\pgfqpoint{2.465908in}{1.137188in}}{\pgfqpoint{2.462635in}{1.129288in}}{\pgfqpoint{2.462635in}{1.121052in}}%
\pgfpathcurveto{\pgfqpoint{2.462635in}{1.112816in}}{\pgfqpoint{2.465908in}{1.104916in}}{\pgfqpoint{2.471731in}{1.099092in}}%
\pgfpathcurveto{\pgfqpoint{2.477555in}{1.093268in}}{\pgfqpoint{2.485455in}{1.089996in}}{\pgfqpoint{2.493692in}{1.089996in}}%
\pgfpathclose%
\pgfusepath{stroke,fill}%
\end{pgfscope}%
\begin{pgfscope}%
\pgfpathrectangle{\pgfqpoint{0.100000in}{0.212622in}}{\pgfqpoint{3.696000in}{3.696000in}}%
\pgfusepath{clip}%
\pgfsetbuttcap%
\pgfsetroundjoin%
\definecolor{currentfill}{rgb}{0.121569,0.466667,0.705882}%
\pgfsetfillcolor{currentfill}%
\pgfsetfillopacity{0.998214}%
\pgfsetlinewidth{1.003750pt}%
\definecolor{currentstroke}{rgb}{0.121569,0.466667,0.705882}%
\pgfsetstrokecolor{currentstroke}%
\pgfsetstrokeopacity{0.998214}%
\pgfsetdash{}{0pt}%
\pgfpathmoveto{\pgfqpoint{2.493692in}{1.089996in}}%
\pgfpathcurveto{\pgfqpoint{2.501928in}{1.089996in}}{\pgfqpoint{2.509828in}{1.093268in}}{\pgfqpoint{2.515652in}{1.099092in}}%
\pgfpathcurveto{\pgfqpoint{2.521476in}{1.104916in}}{\pgfqpoint{2.524748in}{1.112816in}}{\pgfqpoint{2.524748in}{1.121052in}}%
\pgfpathcurveto{\pgfqpoint{2.524748in}{1.129288in}}{\pgfqpoint{2.521476in}{1.137188in}}{\pgfqpoint{2.515652in}{1.143012in}}%
\pgfpathcurveto{\pgfqpoint{2.509828in}{1.148836in}}{\pgfqpoint{2.501928in}{1.152109in}}{\pgfqpoint{2.493692in}{1.152109in}}%
\pgfpathcurveto{\pgfqpoint{2.485455in}{1.152109in}}{\pgfqpoint{2.477555in}{1.148836in}}{\pgfqpoint{2.471731in}{1.143012in}}%
\pgfpathcurveto{\pgfqpoint{2.465908in}{1.137188in}}{\pgfqpoint{2.462635in}{1.129288in}}{\pgfqpoint{2.462635in}{1.121052in}}%
\pgfpathcurveto{\pgfqpoint{2.462635in}{1.112816in}}{\pgfqpoint{2.465908in}{1.104916in}}{\pgfqpoint{2.471731in}{1.099092in}}%
\pgfpathcurveto{\pgfqpoint{2.477555in}{1.093268in}}{\pgfqpoint{2.485455in}{1.089996in}}{\pgfqpoint{2.493692in}{1.089996in}}%
\pgfpathclose%
\pgfusepath{stroke,fill}%
\end{pgfscope}%
\begin{pgfscope}%
\pgfpathrectangle{\pgfqpoint{0.100000in}{0.212622in}}{\pgfqpoint{3.696000in}{3.696000in}}%
\pgfusepath{clip}%
\pgfsetbuttcap%
\pgfsetroundjoin%
\definecolor{currentfill}{rgb}{0.121569,0.466667,0.705882}%
\pgfsetfillcolor{currentfill}%
\pgfsetfillopacity{0.998214}%
\pgfsetlinewidth{1.003750pt}%
\definecolor{currentstroke}{rgb}{0.121569,0.466667,0.705882}%
\pgfsetstrokecolor{currentstroke}%
\pgfsetstrokeopacity{0.998214}%
\pgfsetdash{}{0pt}%
\pgfpathmoveto{\pgfqpoint{2.493692in}{1.089996in}}%
\pgfpathcurveto{\pgfqpoint{2.501928in}{1.089996in}}{\pgfqpoint{2.509828in}{1.093268in}}{\pgfqpoint{2.515652in}{1.099092in}}%
\pgfpathcurveto{\pgfqpoint{2.521476in}{1.104916in}}{\pgfqpoint{2.524748in}{1.112816in}}{\pgfqpoint{2.524748in}{1.121052in}}%
\pgfpathcurveto{\pgfqpoint{2.524748in}{1.129288in}}{\pgfqpoint{2.521476in}{1.137188in}}{\pgfqpoint{2.515652in}{1.143012in}}%
\pgfpathcurveto{\pgfqpoint{2.509828in}{1.148836in}}{\pgfqpoint{2.501928in}{1.152109in}}{\pgfqpoint{2.493692in}{1.152109in}}%
\pgfpathcurveto{\pgfqpoint{2.485455in}{1.152109in}}{\pgfqpoint{2.477555in}{1.148836in}}{\pgfqpoint{2.471731in}{1.143012in}}%
\pgfpathcurveto{\pgfqpoint{2.465908in}{1.137188in}}{\pgfqpoint{2.462635in}{1.129288in}}{\pgfqpoint{2.462635in}{1.121052in}}%
\pgfpathcurveto{\pgfqpoint{2.462635in}{1.112816in}}{\pgfqpoint{2.465908in}{1.104916in}}{\pgfqpoint{2.471731in}{1.099092in}}%
\pgfpathcurveto{\pgfqpoint{2.477555in}{1.093268in}}{\pgfqpoint{2.485455in}{1.089996in}}{\pgfqpoint{2.493692in}{1.089996in}}%
\pgfpathclose%
\pgfusepath{stroke,fill}%
\end{pgfscope}%
\begin{pgfscope}%
\pgfpathrectangle{\pgfqpoint{0.100000in}{0.212622in}}{\pgfqpoint{3.696000in}{3.696000in}}%
\pgfusepath{clip}%
\pgfsetbuttcap%
\pgfsetroundjoin%
\definecolor{currentfill}{rgb}{0.121569,0.466667,0.705882}%
\pgfsetfillcolor{currentfill}%
\pgfsetfillopacity{0.998214}%
\pgfsetlinewidth{1.003750pt}%
\definecolor{currentstroke}{rgb}{0.121569,0.466667,0.705882}%
\pgfsetstrokecolor{currentstroke}%
\pgfsetstrokeopacity{0.998214}%
\pgfsetdash{}{0pt}%
\pgfpathmoveto{\pgfqpoint{2.493692in}{1.089996in}}%
\pgfpathcurveto{\pgfqpoint{2.501928in}{1.089996in}}{\pgfqpoint{2.509828in}{1.093268in}}{\pgfqpoint{2.515652in}{1.099092in}}%
\pgfpathcurveto{\pgfqpoint{2.521476in}{1.104916in}}{\pgfqpoint{2.524748in}{1.112816in}}{\pgfqpoint{2.524748in}{1.121052in}}%
\pgfpathcurveto{\pgfqpoint{2.524748in}{1.129288in}}{\pgfqpoint{2.521476in}{1.137188in}}{\pgfqpoint{2.515652in}{1.143012in}}%
\pgfpathcurveto{\pgfqpoint{2.509828in}{1.148836in}}{\pgfqpoint{2.501928in}{1.152109in}}{\pgfqpoint{2.493692in}{1.152109in}}%
\pgfpathcurveto{\pgfqpoint{2.485455in}{1.152109in}}{\pgfqpoint{2.477555in}{1.148836in}}{\pgfqpoint{2.471731in}{1.143012in}}%
\pgfpathcurveto{\pgfqpoint{2.465908in}{1.137188in}}{\pgfqpoint{2.462635in}{1.129288in}}{\pgfqpoint{2.462635in}{1.121052in}}%
\pgfpathcurveto{\pgfqpoint{2.462635in}{1.112816in}}{\pgfqpoint{2.465908in}{1.104916in}}{\pgfqpoint{2.471731in}{1.099092in}}%
\pgfpathcurveto{\pgfqpoint{2.477555in}{1.093268in}}{\pgfqpoint{2.485455in}{1.089996in}}{\pgfqpoint{2.493692in}{1.089996in}}%
\pgfpathclose%
\pgfusepath{stroke,fill}%
\end{pgfscope}%
\begin{pgfscope}%
\pgfpathrectangle{\pgfqpoint{0.100000in}{0.212622in}}{\pgfqpoint{3.696000in}{3.696000in}}%
\pgfusepath{clip}%
\pgfsetbuttcap%
\pgfsetroundjoin%
\definecolor{currentfill}{rgb}{0.121569,0.466667,0.705882}%
\pgfsetfillcolor{currentfill}%
\pgfsetfillopacity{0.998214}%
\pgfsetlinewidth{1.003750pt}%
\definecolor{currentstroke}{rgb}{0.121569,0.466667,0.705882}%
\pgfsetstrokecolor{currentstroke}%
\pgfsetstrokeopacity{0.998214}%
\pgfsetdash{}{0pt}%
\pgfpathmoveto{\pgfqpoint{2.493692in}{1.089996in}}%
\pgfpathcurveto{\pgfqpoint{2.501928in}{1.089996in}}{\pgfqpoint{2.509828in}{1.093268in}}{\pgfqpoint{2.515652in}{1.099092in}}%
\pgfpathcurveto{\pgfqpoint{2.521476in}{1.104916in}}{\pgfqpoint{2.524748in}{1.112816in}}{\pgfqpoint{2.524748in}{1.121052in}}%
\pgfpathcurveto{\pgfqpoint{2.524748in}{1.129288in}}{\pgfqpoint{2.521476in}{1.137188in}}{\pgfqpoint{2.515652in}{1.143012in}}%
\pgfpathcurveto{\pgfqpoint{2.509828in}{1.148836in}}{\pgfqpoint{2.501928in}{1.152109in}}{\pgfqpoint{2.493692in}{1.152109in}}%
\pgfpathcurveto{\pgfqpoint{2.485455in}{1.152109in}}{\pgfqpoint{2.477555in}{1.148836in}}{\pgfqpoint{2.471731in}{1.143012in}}%
\pgfpathcurveto{\pgfqpoint{2.465908in}{1.137188in}}{\pgfqpoint{2.462635in}{1.129288in}}{\pgfqpoint{2.462635in}{1.121052in}}%
\pgfpathcurveto{\pgfqpoint{2.462635in}{1.112816in}}{\pgfqpoint{2.465908in}{1.104916in}}{\pgfqpoint{2.471731in}{1.099092in}}%
\pgfpathcurveto{\pgfqpoint{2.477555in}{1.093268in}}{\pgfqpoint{2.485455in}{1.089996in}}{\pgfqpoint{2.493692in}{1.089996in}}%
\pgfpathclose%
\pgfusepath{stroke,fill}%
\end{pgfscope}%
\begin{pgfscope}%
\pgfpathrectangle{\pgfqpoint{0.100000in}{0.212622in}}{\pgfqpoint{3.696000in}{3.696000in}}%
\pgfusepath{clip}%
\pgfsetbuttcap%
\pgfsetroundjoin%
\definecolor{currentfill}{rgb}{0.121569,0.466667,0.705882}%
\pgfsetfillcolor{currentfill}%
\pgfsetfillopacity{0.998214}%
\pgfsetlinewidth{1.003750pt}%
\definecolor{currentstroke}{rgb}{0.121569,0.466667,0.705882}%
\pgfsetstrokecolor{currentstroke}%
\pgfsetstrokeopacity{0.998214}%
\pgfsetdash{}{0pt}%
\pgfpathmoveto{\pgfqpoint{2.493692in}{1.089996in}}%
\pgfpathcurveto{\pgfqpoint{2.501928in}{1.089996in}}{\pgfqpoint{2.509828in}{1.093268in}}{\pgfqpoint{2.515652in}{1.099092in}}%
\pgfpathcurveto{\pgfqpoint{2.521476in}{1.104916in}}{\pgfqpoint{2.524748in}{1.112816in}}{\pgfqpoint{2.524748in}{1.121052in}}%
\pgfpathcurveto{\pgfqpoint{2.524748in}{1.129288in}}{\pgfqpoint{2.521476in}{1.137188in}}{\pgfqpoint{2.515652in}{1.143012in}}%
\pgfpathcurveto{\pgfqpoint{2.509828in}{1.148836in}}{\pgfqpoint{2.501928in}{1.152109in}}{\pgfqpoint{2.493692in}{1.152109in}}%
\pgfpathcurveto{\pgfqpoint{2.485455in}{1.152109in}}{\pgfqpoint{2.477555in}{1.148836in}}{\pgfqpoint{2.471731in}{1.143012in}}%
\pgfpathcurveto{\pgfqpoint{2.465908in}{1.137188in}}{\pgfqpoint{2.462635in}{1.129288in}}{\pgfqpoint{2.462635in}{1.121052in}}%
\pgfpathcurveto{\pgfqpoint{2.462635in}{1.112816in}}{\pgfqpoint{2.465908in}{1.104916in}}{\pgfqpoint{2.471731in}{1.099092in}}%
\pgfpathcurveto{\pgfqpoint{2.477555in}{1.093268in}}{\pgfqpoint{2.485455in}{1.089996in}}{\pgfqpoint{2.493692in}{1.089996in}}%
\pgfpathclose%
\pgfusepath{stroke,fill}%
\end{pgfscope}%
\begin{pgfscope}%
\pgfpathrectangle{\pgfqpoint{0.100000in}{0.212622in}}{\pgfqpoint{3.696000in}{3.696000in}}%
\pgfusepath{clip}%
\pgfsetbuttcap%
\pgfsetroundjoin%
\definecolor{currentfill}{rgb}{0.121569,0.466667,0.705882}%
\pgfsetfillcolor{currentfill}%
\pgfsetfillopacity{0.998214}%
\pgfsetlinewidth{1.003750pt}%
\definecolor{currentstroke}{rgb}{0.121569,0.466667,0.705882}%
\pgfsetstrokecolor{currentstroke}%
\pgfsetstrokeopacity{0.998214}%
\pgfsetdash{}{0pt}%
\pgfpathmoveto{\pgfqpoint{2.493692in}{1.089996in}}%
\pgfpathcurveto{\pgfqpoint{2.501928in}{1.089996in}}{\pgfqpoint{2.509828in}{1.093268in}}{\pgfqpoint{2.515652in}{1.099092in}}%
\pgfpathcurveto{\pgfqpoint{2.521476in}{1.104916in}}{\pgfqpoint{2.524748in}{1.112816in}}{\pgfqpoint{2.524748in}{1.121052in}}%
\pgfpathcurveto{\pgfqpoint{2.524748in}{1.129288in}}{\pgfqpoint{2.521476in}{1.137188in}}{\pgfqpoint{2.515652in}{1.143012in}}%
\pgfpathcurveto{\pgfqpoint{2.509828in}{1.148836in}}{\pgfqpoint{2.501928in}{1.152109in}}{\pgfqpoint{2.493692in}{1.152109in}}%
\pgfpathcurveto{\pgfqpoint{2.485455in}{1.152109in}}{\pgfqpoint{2.477555in}{1.148836in}}{\pgfqpoint{2.471731in}{1.143012in}}%
\pgfpathcurveto{\pgfqpoint{2.465908in}{1.137188in}}{\pgfqpoint{2.462635in}{1.129288in}}{\pgfqpoint{2.462635in}{1.121052in}}%
\pgfpathcurveto{\pgfqpoint{2.462635in}{1.112816in}}{\pgfqpoint{2.465908in}{1.104916in}}{\pgfqpoint{2.471731in}{1.099092in}}%
\pgfpathcurveto{\pgfqpoint{2.477555in}{1.093268in}}{\pgfqpoint{2.485455in}{1.089996in}}{\pgfqpoint{2.493692in}{1.089996in}}%
\pgfpathclose%
\pgfusepath{stroke,fill}%
\end{pgfscope}%
\begin{pgfscope}%
\pgfpathrectangle{\pgfqpoint{0.100000in}{0.212622in}}{\pgfqpoint{3.696000in}{3.696000in}}%
\pgfusepath{clip}%
\pgfsetbuttcap%
\pgfsetroundjoin%
\definecolor{currentfill}{rgb}{0.121569,0.466667,0.705882}%
\pgfsetfillcolor{currentfill}%
\pgfsetfillopacity{0.998214}%
\pgfsetlinewidth{1.003750pt}%
\definecolor{currentstroke}{rgb}{0.121569,0.466667,0.705882}%
\pgfsetstrokecolor{currentstroke}%
\pgfsetstrokeopacity{0.998214}%
\pgfsetdash{}{0pt}%
\pgfpathmoveto{\pgfqpoint{2.493692in}{1.089996in}}%
\pgfpathcurveto{\pgfqpoint{2.501928in}{1.089996in}}{\pgfqpoint{2.509828in}{1.093268in}}{\pgfqpoint{2.515652in}{1.099092in}}%
\pgfpathcurveto{\pgfqpoint{2.521476in}{1.104916in}}{\pgfqpoint{2.524748in}{1.112816in}}{\pgfqpoint{2.524748in}{1.121052in}}%
\pgfpathcurveto{\pgfqpoint{2.524748in}{1.129288in}}{\pgfqpoint{2.521476in}{1.137188in}}{\pgfqpoint{2.515652in}{1.143012in}}%
\pgfpathcurveto{\pgfqpoint{2.509828in}{1.148836in}}{\pgfqpoint{2.501928in}{1.152109in}}{\pgfqpoint{2.493692in}{1.152109in}}%
\pgfpathcurveto{\pgfqpoint{2.485455in}{1.152109in}}{\pgfqpoint{2.477555in}{1.148836in}}{\pgfqpoint{2.471731in}{1.143012in}}%
\pgfpathcurveto{\pgfqpoint{2.465908in}{1.137188in}}{\pgfqpoint{2.462635in}{1.129288in}}{\pgfqpoint{2.462635in}{1.121052in}}%
\pgfpathcurveto{\pgfqpoint{2.462635in}{1.112816in}}{\pgfqpoint{2.465908in}{1.104916in}}{\pgfqpoint{2.471731in}{1.099092in}}%
\pgfpathcurveto{\pgfqpoint{2.477555in}{1.093268in}}{\pgfqpoint{2.485455in}{1.089996in}}{\pgfqpoint{2.493692in}{1.089996in}}%
\pgfpathclose%
\pgfusepath{stroke,fill}%
\end{pgfscope}%
\begin{pgfscope}%
\pgfpathrectangle{\pgfqpoint{0.100000in}{0.212622in}}{\pgfqpoint{3.696000in}{3.696000in}}%
\pgfusepath{clip}%
\pgfsetbuttcap%
\pgfsetroundjoin%
\definecolor{currentfill}{rgb}{0.121569,0.466667,0.705882}%
\pgfsetfillcolor{currentfill}%
\pgfsetfillopacity{0.998214}%
\pgfsetlinewidth{1.003750pt}%
\definecolor{currentstroke}{rgb}{0.121569,0.466667,0.705882}%
\pgfsetstrokecolor{currentstroke}%
\pgfsetstrokeopacity{0.998214}%
\pgfsetdash{}{0pt}%
\pgfpathmoveto{\pgfqpoint{2.493692in}{1.089996in}}%
\pgfpathcurveto{\pgfqpoint{2.501928in}{1.089996in}}{\pgfqpoint{2.509828in}{1.093268in}}{\pgfqpoint{2.515652in}{1.099092in}}%
\pgfpathcurveto{\pgfqpoint{2.521476in}{1.104916in}}{\pgfqpoint{2.524748in}{1.112816in}}{\pgfqpoint{2.524748in}{1.121052in}}%
\pgfpathcurveto{\pgfqpoint{2.524748in}{1.129288in}}{\pgfqpoint{2.521476in}{1.137188in}}{\pgfqpoint{2.515652in}{1.143012in}}%
\pgfpathcurveto{\pgfqpoint{2.509828in}{1.148836in}}{\pgfqpoint{2.501928in}{1.152109in}}{\pgfqpoint{2.493692in}{1.152109in}}%
\pgfpathcurveto{\pgfqpoint{2.485455in}{1.152109in}}{\pgfqpoint{2.477555in}{1.148836in}}{\pgfqpoint{2.471731in}{1.143012in}}%
\pgfpathcurveto{\pgfqpoint{2.465908in}{1.137188in}}{\pgfqpoint{2.462635in}{1.129288in}}{\pgfqpoint{2.462635in}{1.121052in}}%
\pgfpathcurveto{\pgfqpoint{2.462635in}{1.112816in}}{\pgfqpoint{2.465908in}{1.104916in}}{\pgfqpoint{2.471731in}{1.099092in}}%
\pgfpathcurveto{\pgfqpoint{2.477555in}{1.093268in}}{\pgfqpoint{2.485455in}{1.089996in}}{\pgfqpoint{2.493692in}{1.089996in}}%
\pgfpathclose%
\pgfusepath{stroke,fill}%
\end{pgfscope}%
\begin{pgfscope}%
\pgfpathrectangle{\pgfqpoint{0.100000in}{0.212622in}}{\pgfqpoint{3.696000in}{3.696000in}}%
\pgfusepath{clip}%
\pgfsetbuttcap%
\pgfsetroundjoin%
\definecolor{currentfill}{rgb}{0.121569,0.466667,0.705882}%
\pgfsetfillcolor{currentfill}%
\pgfsetfillopacity{0.998214}%
\pgfsetlinewidth{1.003750pt}%
\definecolor{currentstroke}{rgb}{0.121569,0.466667,0.705882}%
\pgfsetstrokecolor{currentstroke}%
\pgfsetstrokeopacity{0.998214}%
\pgfsetdash{}{0pt}%
\pgfpathmoveto{\pgfqpoint{2.493692in}{1.089996in}}%
\pgfpathcurveto{\pgfqpoint{2.501928in}{1.089996in}}{\pgfqpoint{2.509828in}{1.093268in}}{\pgfqpoint{2.515652in}{1.099092in}}%
\pgfpathcurveto{\pgfqpoint{2.521476in}{1.104916in}}{\pgfqpoint{2.524748in}{1.112816in}}{\pgfqpoint{2.524748in}{1.121052in}}%
\pgfpathcurveto{\pgfqpoint{2.524748in}{1.129288in}}{\pgfqpoint{2.521476in}{1.137188in}}{\pgfqpoint{2.515652in}{1.143012in}}%
\pgfpathcurveto{\pgfqpoint{2.509828in}{1.148836in}}{\pgfqpoint{2.501928in}{1.152109in}}{\pgfqpoint{2.493692in}{1.152109in}}%
\pgfpathcurveto{\pgfqpoint{2.485455in}{1.152109in}}{\pgfqpoint{2.477555in}{1.148836in}}{\pgfqpoint{2.471731in}{1.143012in}}%
\pgfpathcurveto{\pgfqpoint{2.465908in}{1.137188in}}{\pgfqpoint{2.462635in}{1.129288in}}{\pgfqpoint{2.462635in}{1.121052in}}%
\pgfpathcurveto{\pgfqpoint{2.462635in}{1.112816in}}{\pgfqpoint{2.465908in}{1.104916in}}{\pgfqpoint{2.471731in}{1.099092in}}%
\pgfpathcurveto{\pgfqpoint{2.477555in}{1.093268in}}{\pgfqpoint{2.485455in}{1.089996in}}{\pgfqpoint{2.493692in}{1.089996in}}%
\pgfpathclose%
\pgfusepath{stroke,fill}%
\end{pgfscope}%
\begin{pgfscope}%
\pgfpathrectangle{\pgfqpoint{0.100000in}{0.212622in}}{\pgfqpoint{3.696000in}{3.696000in}}%
\pgfusepath{clip}%
\pgfsetbuttcap%
\pgfsetroundjoin%
\definecolor{currentfill}{rgb}{0.121569,0.466667,0.705882}%
\pgfsetfillcolor{currentfill}%
\pgfsetfillopacity{0.998214}%
\pgfsetlinewidth{1.003750pt}%
\definecolor{currentstroke}{rgb}{0.121569,0.466667,0.705882}%
\pgfsetstrokecolor{currentstroke}%
\pgfsetstrokeopacity{0.998214}%
\pgfsetdash{}{0pt}%
\pgfpathmoveto{\pgfqpoint{2.493692in}{1.089996in}}%
\pgfpathcurveto{\pgfqpoint{2.501928in}{1.089996in}}{\pgfqpoint{2.509828in}{1.093268in}}{\pgfqpoint{2.515652in}{1.099092in}}%
\pgfpathcurveto{\pgfqpoint{2.521476in}{1.104916in}}{\pgfqpoint{2.524748in}{1.112816in}}{\pgfqpoint{2.524748in}{1.121052in}}%
\pgfpathcurveto{\pgfqpoint{2.524748in}{1.129288in}}{\pgfqpoint{2.521476in}{1.137188in}}{\pgfqpoint{2.515652in}{1.143012in}}%
\pgfpathcurveto{\pgfqpoint{2.509828in}{1.148836in}}{\pgfqpoint{2.501928in}{1.152109in}}{\pgfqpoint{2.493692in}{1.152109in}}%
\pgfpathcurveto{\pgfqpoint{2.485455in}{1.152109in}}{\pgfqpoint{2.477555in}{1.148836in}}{\pgfqpoint{2.471731in}{1.143012in}}%
\pgfpathcurveto{\pgfqpoint{2.465908in}{1.137188in}}{\pgfqpoint{2.462635in}{1.129288in}}{\pgfqpoint{2.462635in}{1.121052in}}%
\pgfpathcurveto{\pgfqpoint{2.462635in}{1.112816in}}{\pgfqpoint{2.465908in}{1.104916in}}{\pgfqpoint{2.471731in}{1.099092in}}%
\pgfpathcurveto{\pgfqpoint{2.477555in}{1.093268in}}{\pgfqpoint{2.485455in}{1.089996in}}{\pgfqpoint{2.493692in}{1.089996in}}%
\pgfpathclose%
\pgfusepath{stroke,fill}%
\end{pgfscope}%
\begin{pgfscope}%
\pgfpathrectangle{\pgfqpoint{0.100000in}{0.212622in}}{\pgfqpoint{3.696000in}{3.696000in}}%
\pgfusepath{clip}%
\pgfsetbuttcap%
\pgfsetroundjoin%
\definecolor{currentfill}{rgb}{0.121569,0.466667,0.705882}%
\pgfsetfillcolor{currentfill}%
\pgfsetfillopacity{0.998214}%
\pgfsetlinewidth{1.003750pt}%
\definecolor{currentstroke}{rgb}{0.121569,0.466667,0.705882}%
\pgfsetstrokecolor{currentstroke}%
\pgfsetstrokeopacity{0.998214}%
\pgfsetdash{}{0pt}%
\pgfpathmoveto{\pgfqpoint{2.493692in}{1.089996in}}%
\pgfpathcurveto{\pgfqpoint{2.501928in}{1.089996in}}{\pgfqpoint{2.509828in}{1.093268in}}{\pgfqpoint{2.515652in}{1.099092in}}%
\pgfpathcurveto{\pgfqpoint{2.521476in}{1.104916in}}{\pgfqpoint{2.524748in}{1.112816in}}{\pgfqpoint{2.524748in}{1.121052in}}%
\pgfpathcurveto{\pgfqpoint{2.524748in}{1.129288in}}{\pgfqpoint{2.521476in}{1.137188in}}{\pgfqpoint{2.515652in}{1.143012in}}%
\pgfpathcurveto{\pgfqpoint{2.509828in}{1.148836in}}{\pgfqpoint{2.501928in}{1.152109in}}{\pgfqpoint{2.493692in}{1.152109in}}%
\pgfpathcurveto{\pgfqpoint{2.485455in}{1.152109in}}{\pgfqpoint{2.477555in}{1.148836in}}{\pgfqpoint{2.471731in}{1.143012in}}%
\pgfpathcurveto{\pgfqpoint{2.465908in}{1.137188in}}{\pgfqpoint{2.462635in}{1.129288in}}{\pgfqpoint{2.462635in}{1.121052in}}%
\pgfpathcurveto{\pgfqpoint{2.462635in}{1.112816in}}{\pgfqpoint{2.465908in}{1.104916in}}{\pgfqpoint{2.471731in}{1.099092in}}%
\pgfpathcurveto{\pgfqpoint{2.477555in}{1.093268in}}{\pgfqpoint{2.485455in}{1.089996in}}{\pgfqpoint{2.493692in}{1.089996in}}%
\pgfpathclose%
\pgfusepath{stroke,fill}%
\end{pgfscope}%
\begin{pgfscope}%
\pgfpathrectangle{\pgfqpoint{0.100000in}{0.212622in}}{\pgfqpoint{3.696000in}{3.696000in}}%
\pgfusepath{clip}%
\pgfsetbuttcap%
\pgfsetroundjoin%
\definecolor{currentfill}{rgb}{0.121569,0.466667,0.705882}%
\pgfsetfillcolor{currentfill}%
\pgfsetfillopacity{0.998214}%
\pgfsetlinewidth{1.003750pt}%
\definecolor{currentstroke}{rgb}{0.121569,0.466667,0.705882}%
\pgfsetstrokecolor{currentstroke}%
\pgfsetstrokeopacity{0.998214}%
\pgfsetdash{}{0pt}%
\pgfpathmoveto{\pgfqpoint{2.493692in}{1.089996in}}%
\pgfpathcurveto{\pgfqpoint{2.501928in}{1.089996in}}{\pgfqpoint{2.509828in}{1.093268in}}{\pgfqpoint{2.515652in}{1.099092in}}%
\pgfpathcurveto{\pgfqpoint{2.521476in}{1.104916in}}{\pgfqpoint{2.524748in}{1.112816in}}{\pgfqpoint{2.524748in}{1.121052in}}%
\pgfpathcurveto{\pgfqpoint{2.524748in}{1.129288in}}{\pgfqpoint{2.521476in}{1.137188in}}{\pgfqpoint{2.515652in}{1.143012in}}%
\pgfpathcurveto{\pgfqpoint{2.509828in}{1.148836in}}{\pgfqpoint{2.501928in}{1.152109in}}{\pgfqpoint{2.493692in}{1.152109in}}%
\pgfpathcurveto{\pgfqpoint{2.485455in}{1.152109in}}{\pgfqpoint{2.477555in}{1.148836in}}{\pgfqpoint{2.471731in}{1.143012in}}%
\pgfpathcurveto{\pgfqpoint{2.465908in}{1.137188in}}{\pgfqpoint{2.462635in}{1.129288in}}{\pgfqpoint{2.462635in}{1.121052in}}%
\pgfpathcurveto{\pgfqpoint{2.462635in}{1.112816in}}{\pgfqpoint{2.465908in}{1.104916in}}{\pgfqpoint{2.471731in}{1.099092in}}%
\pgfpathcurveto{\pgfqpoint{2.477555in}{1.093268in}}{\pgfqpoint{2.485455in}{1.089996in}}{\pgfqpoint{2.493692in}{1.089996in}}%
\pgfpathclose%
\pgfusepath{stroke,fill}%
\end{pgfscope}%
\begin{pgfscope}%
\pgfpathrectangle{\pgfqpoint{0.100000in}{0.212622in}}{\pgfqpoint{3.696000in}{3.696000in}}%
\pgfusepath{clip}%
\pgfsetbuttcap%
\pgfsetroundjoin%
\definecolor{currentfill}{rgb}{0.121569,0.466667,0.705882}%
\pgfsetfillcolor{currentfill}%
\pgfsetfillopacity{0.998214}%
\pgfsetlinewidth{1.003750pt}%
\definecolor{currentstroke}{rgb}{0.121569,0.466667,0.705882}%
\pgfsetstrokecolor{currentstroke}%
\pgfsetstrokeopacity{0.998214}%
\pgfsetdash{}{0pt}%
\pgfpathmoveto{\pgfqpoint{2.493692in}{1.089996in}}%
\pgfpathcurveto{\pgfqpoint{2.501928in}{1.089996in}}{\pgfqpoint{2.509828in}{1.093268in}}{\pgfqpoint{2.515652in}{1.099092in}}%
\pgfpathcurveto{\pgfqpoint{2.521476in}{1.104916in}}{\pgfqpoint{2.524748in}{1.112816in}}{\pgfqpoint{2.524748in}{1.121052in}}%
\pgfpathcurveto{\pgfqpoint{2.524748in}{1.129288in}}{\pgfqpoint{2.521476in}{1.137188in}}{\pgfqpoint{2.515652in}{1.143012in}}%
\pgfpathcurveto{\pgfqpoint{2.509828in}{1.148836in}}{\pgfqpoint{2.501928in}{1.152109in}}{\pgfqpoint{2.493692in}{1.152109in}}%
\pgfpathcurveto{\pgfqpoint{2.485455in}{1.152109in}}{\pgfqpoint{2.477555in}{1.148836in}}{\pgfqpoint{2.471731in}{1.143012in}}%
\pgfpathcurveto{\pgfqpoint{2.465908in}{1.137188in}}{\pgfqpoint{2.462635in}{1.129288in}}{\pgfqpoint{2.462635in}{1.121052in}}%
\pgfpathcurveto{\pgfqpoint{2.462635in}{1.112816in}}{\pgfqpoint{2.465908in}{1.104916in}}{\pgfqpoint{2.471731in}{1.099092in}}%
\pgfpathcurveto{\pgfqpoint{2.477555in}{1.093268in}}{\pgfqpoint{2.485455in}{1.089996in}}{\pgfqpoint{2.493692in}{1.089996in}}%
\pgfpathclose%
\pgfusepath{stroke,fill}%
\end{pgfscope}%
\begin{pgfscope}%
\pgfpathrectangle{\pgfqpoint{0.100000in}{0.212622in}}{\pgfqpoint{3.696000in}{3.696000in}}%
\pgfusepath{clip}%
\pgfsetbuttcap%
\pgfsetroundjoin%
\definecolor{currentfill}{rgb}{0.121569,0.466667,0.705882}%
\pgfsetfillcolor{currentfill}%
\pgfsetfillopacity{0.998214}%
\pgfsetlinewidth{1.003750pt}%
\definecolor{currentstroke}{rgb}{0.121569,0.466667,0.705882}%
\pgfsetstrokecolor{currentstroke}%
\pgfsetstrokeopacity{0.998214}%
\pgfsetdash{}{0pt}%
\pgfpathmoveto{\pgfqpoint{2.493692in}{1.089996in}}%
\pgfpathcurveto{\pgfqpoint{2.501928in}{1.089996in}}{\pgfqpoint{2.509828in}{1.093268in}}{\pgfqpoint{2.515652in}{1.099092in}}%
\pgfpathcurveto{\pgfqpoint{2.521476in}{1.104916in}}{\pgfqpoint{2.524748in}{1.112816in}}{\pgfqpoint{2.524748in}{1.121052in}}%
\pgfpathcurveto{\pgfqpoint{2.524748in}{1.129288in}}{\pgfqpoint{2.521476in}{1.137188in}}{\pgfqpoint{2.515652in}{1.143012in}}%
\pgfpathcurveto{\pgfqpoint{2.509828in}{1.148836in}}{\pgfqpoint{2.501928in}{1.152109in}}{\pgfqpoint{2.493692in}{1.152109in}}%
\pgfpathcurveto{\pgfqpoint{2.485455in}{1.152109in}}{\pgfqpoint{2.477555in}{1.148836in}}{\pgfqpoint{2.471731in}{1.143012in}}%
\pgfpathcurveto{\pgfqpoint{2.465908in}{1.137188in}}{\pgfqpoint{2.462635in}{1.129288in}}{\pgfqpoint{2.462635in}{1.121052in}}%
\pgfpathcurveto{\pgfqpoint{2.462635in}{1.112816in}}{\pgfqpoint{2.465908in}{1.104916in}}{\pgfqpoint{2.471731in}{1.099092in}}%
\pgfpathcurveto{\pgfqpoint{2.477555in}{1.093268in}}{\pgfqpoint{2.485455in}{1.089996in}}{\pgfqpoint{2.493692in}{1.089996in}}%
\pgfpathclose%
\pgfusepath{stroke,fill}%
\end{pgfscope}%
\begin{pgfscope}%
\pgfpathrectangle{\pgfqpoint{0.100000in}{0.212622in}}{\pgfqpoint{3.696000in}{3.696000in}}%
\pgfusepath{clip}%
\pgfsetbuttcap%
\pgfsetroundjoin%
\definecolor{currentfill}{rgb}{0.121569,0.466667,0.705882}%
\pgfsetfillcolor{currentfill}%
\pgfsetfillopacity{0.998214}%
\pgfsetlinewidth{1.003750pt}%
\definecolor{currentstroke}{rgb}{0.121569,0.466667,0.705882}%
\pgfsetstrokecolor{currentstroke}%
\pgfsetstrokeopacity{0.998214}%
\pgfsetdash{}{0pt}%
\pgfpathmoveto{\pgfqpoint{2.493692in}{1.089996in}}%
\pgfpathcurveto{\pgfqpoint{2.501928in}{1.089996in}}{\pgfqpoint{2.509828in}{1.093268in}}{\pgfqpoint{2.515652in}{1.099092in}}%
\pgfpathcurveto{\pgfqpoint{2.521476in}{1.104916in}}{\pgfqpoint{2.524748in}{1.112816in}}{\pgfqpoint{2.524748in}{1.121052in}}%
\pgfpathcurveto{\pgfqpoint{2.524748in}{1.129288in}}{\pgfqpoint{2.521476in}{1.137188in}}{\pgfqpoint{2.515652in}{1.143012in}}%
\pgfpathcurveto{\pgfqpoint{2.509828in}{1.148836in}}{\pgfqpoint{2.501928in}{1.152109in}}{\pgfqpoint{2.493692in}{1.152109in}}%
\pgfpathcurveto{\pgfqpoint{2.485455in}{1.152109in}}{\pgfqpoint{2.477555in}{1.148836in}}{\pgfqpoint{2.471731in}{1.143012in}}%
\pgfpathcurveto{\pgfqpoint{2.465908in}{1.137188in}}{\pgfqpoint{2.462635in}{1.129288in}}{\pgfqpoint{2.462635in}{1.121052in}}%
\pgfpathcurveto{\pgfqpoint{2.462635in}{1.112816in}}{\pgfqpoint{2.465908in}{1.104916in}}{\pgfqpoint{2.471731in}{1.099092in}}%
\pgfpathcurveto{\pgfqpoint{2.477555in}{1.093268in}}{\pgfqpoint{2.485455in}{1.089996in}}{\pgfqpoint{2.493692in}{1.089996in}}%
\pgfpathclose%
\pgfusepath{stroke,fill}%
\end{pgfscope}%
\begin{pgfscope}%
\pgfpathrectangle{\pgfqpoint{0.100000in}{0.212622in}}{\pgfqpoint{3.696000in}{3.696000in}}%
\pgfusepath{clip}%
\pgfsetbuttcap%
\pgfsetroundjoin%
\definecolor{currentfill}{rgb}{0.121569,0.466667,0.705882}%
\pgfsetfillcolor{currentfill}%
\pgfsetfillopacity{0.998214}%
\pgfsetlinewidth{1.003750pt}%
\definecolor{currentstroke}{rgb}{0.121569,0.466667,0.705882}%
\pgfsetstrokecolor{currentstroke}%
\pgfsetstrokeopacity{0.998214}%
\pgfsetdash{}{0pt}%
\pgfpathmoveto{\pgfqpoint{2.493692in}{1.089996in}}%
\pgfpathcurveto{\pgfqpoint{2.501928in}{1.089996in}}{\pgfqpoint{2.509828in}{1.093268in}}{\pgfqpoint{2.515652in}{1.099092in}}%
\pgfpathcurveto{\pgfqpoint{2.521476in}{1.104916in}}{\pgfqpoint{2.524748in}{1.112816in}}{\pgfqpoint{2.524748in}{1.121052in}}%
\pgfpathcurveto{\pgfqpoint{2.524748in}{1.129288in}}{\pgfqpoint{2.521476in}{1.137188in}}{\pgfqpoint{2.515652in}{1.143012in}}%
\pgfpathcurveto{\pgfqpoint{2.509828in}{1.148836in}}{\pgfqpoint{2.501928in}{1.152109in}}{\pgfqpoint{2.493692in}{1.152109in}}%
\pgfpathcurveto{\pgfqpoint{2.485455in}{1.152109in}}{\pgfqpoint{2.477555in}{1.148836in}}{\pgfqpoint{2.471731in}{1.143012in}}%
\pgfpathcurveto{\pgfqpoint{2.465908in}{1.137188in}}{\pgfqpoint{2.462635in}{1.129288in}}{\pgfqpoint{2.462635in}{1.121052in}}%
\pgfpathcurveto{\pgfqpoint{2.462635in}{1.112816in}}{\pgfqpoint{2.465908in}{1.104916in}}{\pgfqpoint{2.471731in}{1.099092in}}%
\pgfpathcurveto{\pgfqpoint{2.477555in}{1.093268in}}{\pgfqpoint{2.485455in}{1.089996in}}{\pgfqpoint{2.493692in}{1.089996in}}%
\pgfpathclose%
\pgfusepath{stroke,fill}%
\end{pgfscope}%
\begin{pgfscope}%
\pgfpathrectangle{\pgfqpoint{0.100000in}{0.212622in}}{\pgfqpoint{3.696000in}{3.696000in}}%
\pgfusepath{clip}%
\pgfsetbuttcap%
\pgfsetroundjoin%
\definecolor{currentfill}{rgb}{0.121569,0.466667,0.705882}%
\pgfsetfillcolor{currentfill}%
\pgfsetfillopacity{0.998214}%
\pgfsetlinewidth{1.003750pt}%
\definecolor{currentstroke}{rgb}{0.121569,0.466667,0.705882}%
\pgfsetstrokecolor{currentstroke}%
\pgfsetstrokeopacity{0.998214}%
\pgfsetdash{}{0pt}%
\pgfpathmoveto{\pgfqpoint{2.493692in}{1.089996in}}%
\pgfpathcurveto{\pgfqpoint{2.501928in}{1.089996in}}{\pgfqpoint{2.509828in}{1.093268in}}{\pgfqpoint{2.515652in}{1.099092in}}%
\pgfpathcurveto{\pgfqpoint{2.521476in}{1.104916in}}{\pgfqpoint{2.524748in}{1.112816in}}{\pgfqpoint{2.524748in}{1.121052in}}%
\pgfpathcurveto{\pgfqpoint{2.524748in}{1.129288in}}{\pgfqpoint{2.521476in}{1.137188in}}{\pgfqpoint{2.515652in}{1.143012in}}%
\pgfpathcurveto{\pgfqpoint{2.509828in}{1.148836in}}{\pgfqpoint{2.501928in}{1.152109in}}{\pgfqpoint{2.493692in}{1.152109in}}%
\pgfpathcurveto{\pgfqpoint{2.485455in}{1.152109in}}{\pgfqpoint{2.477555in}{1.148836in}}{\pgfqpoint{2.471731in}{1.143012in}}%
\pgfpathcurveto{\pgfqpoint{2.465908in}{1.137188in}}{\pgfqpoint{2.462635in}{1.129288in}}{\pgfqpoint{2.462635in}{1.121052in}}%
\pgfpathcurveto{\pgfqpoint{2.462635in}{1.112816in}}{\pgfqpoint{2.465908in}{1.104916in}}{\pgfqpoint{2.471731in}{1.099092in}}%
\pgfpathcurveto{\pgfqpoint{2.477555in}{1.093268in}}{\pgfqpoint{2.485455in}{1.089996in}}{\pgfqpoint{2.493692in}{1.089996in}}%
\pgfpathclose%
\pgfusepath{stroke,fill}%
\end{pgfscope}%
\begin{pgfscope}%
\pgfpathrectangle{\pgfqpoint{0.100000in}{0.212622in}}{\pgfqpoint{3.696000in}{3.696000in}}%
\pgfusepath{clip}%
\pgfsetbuttcap%
\pgfsetroundjoin%
\definecolor{currentfill}{rgb}{0.121569,0.466667,0.705882}%
\pgfsetfillcolor{currentfill}%
\pgfsetfillopacity{0.998214}%
\pgfsetlinewidth{1.003750pt}%
\definecolor{currentstroke}{rgb}{0.121569,0.466667,0.705882}%
\pgfsetstrokecolor{currentstroke}%
\pgfsetstrokeopacity{0.998214}%
\pgfsetdash{}{0pt}%
\pgfpathmoveto{\pgfqpoint{2.493692in}{1.089996in}}%
\pgfpathcurveto{\pgfqpoint{2.501928in}{1.089996in}}{\pgfqpoint{2.509828in}{1.093268in}}{\pgfqpoint{2.515652in}{1.099092in}}%
\pgfpathcurveto{\pgfqpoint{2.521476in}{1.104916in}}{\pgfqpoint{2.524748in}{1.112816in}}{\pgfqpoint{2.524748in}{1.121052in}}%
\pgfpathcurveto{\pgfqpoint{2.524748in}{1.129288in}}{\pgfqpoint{2.521476in}{1.137188in}}{\pgfqpoint{2.515652in}{1.143012in}}%
\pgfpathcurveto{\pgfqpoint{2.509828in}{1.148836in}}{\pgfqpoint{2.501928in}{1.152109in}}{\pgfqpoint{2.493692in}{1.152109in}}%
\pgfpathcurveto{\pgfqpoint{2.485455in}{1.152109in}}{\pgfqpoint{2.477555in}{1.148836in}}{\pgfqpoint{2.471731in}{1.143012in}}%
\pgfpathcurveto{\pgfqpoint{2.465908in}{1.137188in}}{\pgfqpoint{2.462635in}{1.129288in}}{\pgfqpoint{2.462635in}{1.121052in}}%
\pgfpathcurveto{\pgfqpoint{2.462635in}{1.112816in}}{\pgfqpoint{2.465908in}{1.104916in}}{\pgfqpoint{2.471731in}{1.099092in}}%
\pgfpathcurveto{\pgfqpoint{2.477555in}{1.093268in}}{\pgfqpoint{2.485455in}{1.089996in}}{\pgfqpoint{2.493692in}{1.089996in}}%
\pgfpathclose%
\pgfusepath{stroke,fill}%
\end{pgfscope}%
\begin{pgfscope}%
\pgfpathrectangle{\pgfqpoint{0.100000in}{0.212622in}}{\pgfqpoint{3.696000in}{3.696000in}}%
\pgfusepath{clip}%
\pgfsetbuttcap%
\pgfsetroundjoin%
\definecolor{currentfill}{rgb}{0.121569,0.466667,0.705882}%
\pgfsetfillcolor{currentfill}%
\pgfsetfillopacity{0.998214}%
\pgfsetlinewidth{1.003750pt}%
\definecolor{currentstroke}{rgb}{0.121569,0.466667,0.705882}%
\pgfsetstrokecolor{currentstroke}%
\pgfsetstrokeopacity{0.998214}%
\pgfsetdash{}{0pt}%
\pgfpathmoveto{\pgfqpoint{2.493692in}{1.089996in}}%
\pgfpathcurveto{\pgfqpoint{2.501928in}{1.089996in}}{\pgfqpoint{2.509828in}{1.093268in}}{\pgfqpoint{2.515652in}{1.099092in}}%
\pgfpathcurveto{\pgfqpoint{2.521476in}{1.104916in}}{\pgfqpoint{2.524748in}{1.112816in}}{\pgfqpoint{2.524748in}{1.121052in}}%
\pgfpathcurveto{\pgfqpoint{2.524748in}{1.129288in}}{\pgfqpoint{2.521476in}{1.137188in}}{\pgfqpoint{2.515652in}{1.143012in}}%
\pgfpathcurveto{\pgfqpoint{2.509828in}{1.148836in}}{\pgfqpoint{2.501928in}{1.152109in}}{\pgfqpoint{2.493692in}{1.152109in}}%
\pgfpathcurveto{\pgfqpoint{2.485455in}{1.152109in}}{\pgfqpoint{2.477555in}{1.148836in}}{\pgfqpoint{2.471731in}{1.143012in}}%
\pgfpathcurveto{\pgfqpoint{2.465907in}{1.137188in}}{\pgfqpoint{2.462635in}{1.129288in}}{\pgfqpoint{2.462635in}{1.121052in}}%
\pgfpathcurveto{\pgfqpoint{2.462635in}{1.112816in}}{\pgfqpoint{2.465907in}{1.104916in}}{\pgfqpoint{2.471731in}{1.099092in}}%
\pgfpathcurveto{\pgfqpoint{2.477555in}{1.093268in}}{\pgfqpoint{2.485455in}{1.089996in}}{\pgfqpoint{2.493692in}{1.089996in}}%
\pgfpathclose%
\pgfusepath{stroke,fill}%
\end{pgfscope}%
\begin{pgfscope}%
\pgfpathrectangle{\pgfqpoint{0.100000in}{0.212622in}}{\pgfqpoint{3.696000in}{3.696000in}}%
\pgfusepath{clip}%
\pgfsetbuttcap%
\pgfsetroundjoin%
\definecolor{currentfill}{rgb}{0.121569,0.466667,0.705882}%
\pgfsetfillcolor{currentfill}%
\pgfsetfillopacity{0.998214}%
\pgfsetlinewidth{1.003750pt}%
\definecolor{currentstroke}{rgb}{0.121569,0.466667,0.705882}%
\pgfsetstrokecolor{currentstroke}%
\pgfsetstrokeopacity{0.998214}%
\pgfsetdash{}{0pt}%
\pgfpathmoveto{\pgfqpoint{2.493692in}{1.089996in}}%
\pgfpathcurveto{\pgfqpoint{2.501928in}{1.089996in}}{\pgfqpoint{2.509828in}{1.093268in}}{\pgfqpoint{2.515652in}{1.099092in}}%
\pgfpathcurveto{\pgfqpoint{2.521476in}{1.104916in}}{\pgfqpoint{2.524748in}{1.112816in}}{\pgfqpoint{2.524748in}{1.121052in}}%
\pgfpathcurveto{\pgfqpoint{2.524748in}{1.129288in}}{\pgfqpoint{2.521476in}{1.137188in}}{\pgfqpoint{2.515652in}{1.143012in}}%
\pgfpathcurveto{\pgfqpoint{2.509828in}{1.148836in}}{\pgfqpoint{2.501928in}{1.152109in}}{\pgfqpoint{2.493692in}{1.152109in}}%
\pgfpathcurveto{\pgfqpoint{2.485455in}{1.152109in}}{\pgfqpoint{2.477555in}{1.148836in}}{\pgfqpoint{2.471731in}{1.143012in}}%
\pgfpathcurveto{\pgfqpoint{2.465907in}{1.137188in}}{\pgfqpoint{2.462635in}{1.129288in}}{\pgfqpoint{2.462635in}{1.121052in}}%
\pgfpathcurveto{\pgfqpoint{2.462635in}{1.112816in}}{\pgfqpoint{2.465907in}{1.104916in}}{\pgfqpoint{2.471731in}{1.099092in}}%
\pgfpathcurveto{\pgfqpoint{2.477555in}{1.093268in}}{\pgfqpoint{2.485455in}{1.089996in}}{\pgfqpoint{2.493692in}{1.089996in}}%
\pgfpathclose%
\pgfusepath{stroke,fill}%
\end{pgfscope}%
\begin{pgfscope}%
\pgfpathrectangle{\pgfqpoint{0.100000in}{0.212622in}}{\pgfqpoint{3.696000in}{3.696000in}}%
\pgfusepath{clip}%
\pgfsetbuttcap%
\pgfsetroundjoin%
\definecolor{currentfill}{rgb}{0.121569,0.466667,0.705882}%
\pgfsetfillcolor{currentfill}%
\pgfsetfillopacity{0.998214}%
\pgfsetlinewidth{1.003750pt}%
\definecolor{currentstroke}{rgb}{0.121569,0.466667,0.705882}%
\pgfsetstrokecolor{currentstroke}%
\pgfsetstrokeopacity{0.998214}%
\pgfsetdash{}{0pt}%
\pgfpathmoveto{\pgfqpoint{2.493692in}{1.089996in}}%
\pgfpathcurveto{\pgfqpoint{2.501928in}{1.089996in}}{\pgfqpoint{2.509828in}{1.093268in}}{\pgfqpoint{2.515652in}{1.099092in}}%
\pgfpathcurveto{\pgfqpoint{2.521476in}{1.104916in}}{\pgfqpoint{2.524748in}{1.112816in}}{\pgfqpoint{2.524748in}{1.121052in}}%
\pgfpathcurveto{\pgfqpoint{2.524748in}{1.129288in}}{\pgfqpoint{2.521476in}{1.137188in}}{\pgfqpoint{2.515652in}{1.143012in}}%
\pgfpathcurveto{\pgfqpoint{2.509828in}{1.148836in}}{\pgfqpoint{2.501928in}{1.152109in}}{\pgfqpoint{2.493692in}{1.152109in}}%
\pgfpathcurveto{\pgfqpoint{2.485455in}{1.152109in}}{\pgfqpoint{2.477555in}{1.148836in}}{\pgfqpoint{2.471731in}{1.143012in}}%
\pgfpathcurveto{\pgfqpoint{2.465907in}{1.137188in}}{\pgfqpoint{2.462635in}{1.129288in}}{\pgfqpoint{2.462635in}{1.121052in}}%
\pgfpathcurveto{\pgfqpoint{2.462635in}{1.112816in}}{\pgfqpoint{2.465907in}{1.104916in}}{\pgfqpoint{2.471731in}{1.099092in}}%
\pgfpathcurveto{\pgfqpoint{2.477555in}{1.093268in}}{\pgfqpoint{2.485455in}{1.089996in}}{\pgfqpoint{2.493692in}{1.089996in}}%
\pgfpathclose%
\pgfusepath{stroke,fill}%
\end{pgfscope}%
\begin{pgfscope}%
\pgfpathrectangle{\pgfqpoint{0.100000in}{0.212622in}}{\pgfqpoint{3.696000in}{3.696000in}}%
\pgfusepath{clip}%
\pgfsetbuttcap%
\pgfsetroundjoin%
\definecolor{currentfill}{rgb}{0.121569,0.466667,0.705882}%
\pgfsetfillcolor{currentfill}%
\pgfsetfillopacity{0.998214}%
\pgfsetlinewidth{1.003750pt}%
\definecolor{currentstroke}{rgb}{0.121569,0.466667,0.705882}%
\pgfsetstrokecolor{currentstroke}%
\pgfsetstrokeopacity{0.998214}%
\pgfsetdash{}{0pt}%
\pgfpathmoveto{\pgfqpoint{2.493692in}{1.089996in}}%
\pgfpathcurveto{\pgfqpoint{2.501928in}{1.089996in}}{\pgfqpoint{2.509828in}{1.093268in}}{\pgfqpoint{2.515652in}{1.099092in}}%
\pgfpathcurveto{\pgfqpoint{2.521476in}{1.104916in}}{\pgfqpoint{2.524748in}{1.112816in}}{\pgfqpoint{2.524748in}{1.121052in}}%
\pgfpathcurveto{\pgfqpoint{2.524748in}{1.129288in}}{\pgfqpoint{2.521476in}{1.137188in}}{\pgfqpoint{2.515652in}{1.143012in}}%
\pgfpathcurveto{\pgfqpoint{2.509828in}{1.148836in}}{\pgfqpoint{2.501928in}{1.152109in}}{\pgfqpoint{2.493692in}{1.152109in}}%
\pgfpathcurveto{\pgfqpoint{2.485455in}{1.152109in}}{\pgfqpoint{2.477555in}{1.148836in}}{\pgfqpoint{2.471731in}{1.143012in}}%
\pgfpathcurveto{\pgfqpoint{2.465907in}{1.137188in}}{\pgfqpoint{2.462635in}{1.129288in}}{\pgfqpoint{2.462635in}{1.121052in}}%
\pgfpathcurveto{\pgfqpoint{2.462635in}{1.112816in}}{\pgfqpoint{2.465907in}{1.104916in}}{\pgfqpoint{2.471731in}{1.099092in}}%
\pgfpathcurveto{\pgfqpoint{2.477555in}{1.093268in}}{\pgfqpoint{2.485455in}{1.089996in}}{\pgfqpoint{2.493692in}{1.089996in}}%
\pgfpathclose%
\pgfusepath{stroke,fill}%
\end{pgfscope}%
\begin{pgfscope}%
\pgfpathrectangle{\pgfqpoint{0.100000in}{0.212622in}}{\pgfqpoint{3.696000in}{3.696000in}}%
\pgfusepath{clip}%
\pgfsetbuttcap%
\pgfsetroundjoin%
\definecolor{currentfill}{rgb}{0.121569,0.466667,0.705882}%
\pgfsetfillcolor{currentfill}%
\pgfsetfillopacity{0.998214}%
\pgfsetlinewidth{1.003750pt}%
\definecolor{currentstroke}{rgb}{0.121569,0.466667,0.705882}%
\pgfsetstrokecolor{currentstroke}%
\pgfsetstrokeopacity{0.998214}%
\pgfsetdash{}{0pt}%
\pgfpathmoveto{\pgfqpoint{2.493692in}{1.089996in}}%
\pgfpathcurveto{\pgfqpoint{2.501928in}{1.089996in}}{\pgfqpoint{2.509828in}{1.093268in}}{\pgfqpoint{2.515652in}{1.099092in}}%
\pgfpathcurveto{\pgfqpoint{2.521476in}{1.104916in}}{\pgfqpoint{2.524748in}{1.112816in}}{\pgfqpoint{2.524748in}{1.121052in}}%
\pgfpathcurveto{\pgfqpoint{2.524748in}{1.129288in}}{\pgfqpoint{2.521476in}{1.137188in}}{\pgfqpoint{2.515652in}{1.143012in}}%
\pgfpathcurveto{\pgfqpoint{2.509828in}{1.148836in}}{\pgfqpoint{2.501928in}{1.152109in}}{\pgfqpoint{2.493692in}{1.152109in}}%
\pgfpathcurveto{\pgfqpoint{2.485455in}{1.152109in}}{\pgfqpoint{2.477555in}{1.148836in}}{\pgfqpoint{2.471731in}{1.143012in}}%
\pgfpathcurveto{\pgfqpoint{2.465907in}{1.137188in}}{\pgfqpoint{2.462635in}{1.129288in}}{\pgfqpoint{2.462635in}{1.121052in}}%
\pgfpathcurveto{\pgfqpoint{2.462635in}{1.112816in}}{\pgfqpoint{2.465907in}{1.104916in}}{\pgfqpoint{2.471731in}{1.099092in}}%
\pgfpathcurveto{\pgfqpoint{2.477555in}{1.093268in}}{\pgfqpoint{2.485455in}{1.089996in}}{\pgfqpoint{2.493692in}{1.089996in}}%
\pgfpathclose%
\pgfusepath{stroke,fill}%
\end{pgfscope}%
\begin{pgfscope}%
\pgfpathrectangle{\pgfqpoint{0.100000in}{0.212622in}}{\pgfqpoint{3.696000in}{3.696000in}}%
\pgfusepath{clip}%
\pgfsetbuttcap%
\pgfsetroundjoin%
\definecolor{currentfill}{rgb}{0.121569,0.466667,0.705882}%
\pgfsetfillcolor{currentfill}%
\pgfsetfillopacity{0.998214}%
\pgfsetlinewidth{1.003750pt}%
\definecolor{currentstroke}{rgb}{0.121569,0.466667,0.705882}%
\pgfsetstrokecolor{currentstroke}%
\pgfsetstrokeopacity{0.998214}%
\pgfsetdash{}{0pt}%
\pgfpathmoveto{\pgfqpoint{2.493692in}{1.089996in}}%
\pgfpathcurveto{\pgfqpoint{2.501928in}{1.089996in}}{\pgfqpoint{2.509828in}{1.093268in}}{\pgfqpoint{2.515652in}{1.099092in}}%
\pgfpathcurveto{\pgfqpoint{2.521476in}{1.104916in}}{\pgfqpoint{2.524748in}{1.112816in}}{\pgfqpoint{2.524748in}{1.121052in}}%
\pgfpathcurveto{\pgfqpoint{2.524748in}{1.129288in}}{\pgfqpoint{2.521476in}{1.137188in}}{\pgfqpoint{2.515652in}{1.143012in}}%
\pgfpathcurveto{\pgfqpoint{2.509828in}{1.148836in}}{\pgfqpoint{2.501928in}{1.152109in}}{\pgfqpoint{2.493692in}{1.152109in}}%
\pgfpathcurveto{\pgfqpoint{2.485455in}{1.152109in}}{\pgfqpoint{2.477555in}{1.148836in}}{\pgfqpoint{2.471731in}{1.143012in}}%
\pgfpathcurveto{\pgfqpoint{2.465907in}{1.137188in}}{\pgfqpoint{2.462635in}{1.129288in}}{\pgfqpoint{2.462635in}{1.121052in}}%
\pgfpathcurveto{\pgfqpoint{2.462635in}{1.112816in}}{\pgfqpoint{2.465907in}{1.104916in}}{\pgfqpoint{2.471731in}{1.099092in}}%
\pgfpathcurveto{\pgfqpoint{2.477555in}{1.093268in}}{\pgfqpoint{2.485455in}{1.089996in}}{\pgfqpoint{2.493692in}{1.089996in}}%
\pgfpathclose%
\pgfusepath{stroke,fill}%
\end{pgfscope}%
\begin{pgfscope}%
\pgfpathrectangle{\pgfqpoint{0.100000in}{0.212622in}}{\pgfqpoint{3.696000in}{3.696000in}}%
\pgfusepath{clip}%
\pgfsetbuttcap%
\pgfsetroundjoin%
\definecolor{currentfill}{rgb}{0.121569,0.466667,0.705882}%
\pgfsetfillcolor{currentfill}%
\pgfsetfillopacity{0.998214}%
\pgfsetlinewidth{1.003750pt}%
\definecolor{currentstroke}{rgb}{0.121569,0.466667,0.705882}%
\pgfsetstrokecolor{currentstroke}%
\pgfsetstrokeopacity{0.998214}%
\pgfsetdash{}{0pt}%
\pgfpathmoveto{\pgfqpoint{2.493692in}{1.089996in}}%
\pgfpathcurveto{\pgfqpoint{2.501928in}{1.089996in}}{\pgfqpoint{2.509828in}{1.093268in}}{\pgfqpoint{2.515652in}{1.099092in}}%
\pgfpathcurveto{\pgfqpoint{2.521476in}{1.104916in}}{\pgfqpoint{2.524748in}{1.112816in}}{\pgfqpoint{2.524748in}{1.121052in}}%
\pgfpathcurveto{\pgfqpoint{2.524748in}{1.129288in}}{\pgfqpoint{2.521476in}{1.137188in}}{\pgfqpoint{2.515652in}{1.143012in}}%
\pgfpathcurveto{\pgfqpoint{2.509828in}{1.148836in}}{\pgfqpoint{2.501928in}{1.152109in}}{\pgfqpoint{2.493692in}{1.152109in}}%
\pgfpathcurveto{\pgfqpoint{2.485455in}{1.152109in}}{\pgfqpoint{2.477555in}{1.148836in}}{\pgfqpoint{2.471731in}{1.143012in}}%
\pgfpathcurveto{\pgfqpoint{2.465907in}{1.137188in}}{\pgfqpoint{2.462635in}{1.129288in}}{\pgfqpoint{2.462635in}{1.121052in}}%
\pgfpathcurveto{\pgfqpoint{2.462635in}{1.112816in}}{\pgfqpoint{2.465907in}{1.104916in}}{\pgfqpoint{2.471731in}{1.099092in}}%
\pgfpathcurveto{\pgfqpoint{2.477555in}{1.093268in}}{\pgfqpoint{2.485455in}{1.089996in}}{\pgfqpoint{2.493692in}{1.089996in}}%
\pgfpathclose%
\pgfusepath{stroke,fill}%
\end{pgfscope}%
\begin{pgfscope}%
\pgfpathrectangle{\pgfqpoint{0.100000in}{0.212622in}}{\pgfqpoint{3.696000in}{3.696000in}}%
\pgfusepath{clip}%
\pgfsetbuttcap%
\pgfsetroundjoin%
\definecolor{currentfill}{rgb}{0.121569,0.466667,0.705882}%
\pgfsetfillcolor{currentfill}%
\pgfsetfillopacity{0.998214}%
\pgfsetlinewidth{1.003750pt}%
\definecolor{currentstroke}{rgb}{0.121569,0.466667,0.705882}%
\pgfsetstrokecolor{currentstroke}%
\pgfsetstrokeopacity{0.998214}%
\pgfsetdash{}{0pt}%
\pgfpathmoveto{\pgfqpoint{2.493692in}{1.089996in}}%
\pgfpathcurveto{\pgfqpoint{2.501928in}{1.089996in}}{\pgfqpoint{2.509828in}{1.093268in}}{\pgfqpoint{2.515652in}{1.099092in}}%
\pgfpathcurveto{\pgfqpoint{2.521476in}{1.104916in}}{\pgfqpoint{2.524748in}{1.112816in}}{\pgfqpoint{2.524748in}{1.121052in}}%
\pgfpathcurveto{\pgfqpoint{2.524748in}{1.129288in}}{\pgfqpoint{2.521476in}{1.137188in}}{\pgfqpoint{2.515652in}{1.143012in}}%
\pgfpathcurveto{\pgfqpoint{2.509828in}{1.148836in}}{\pgfqpoint{2.501928in}{1.152109in}}{\pgfqpoint{2.493692in}{1.152109in}}%
\pgfpathcurveto{\pgfqpoint{2.485455in}{1.152109in}}{\pgfqpoint{2.477555in}{1.148836in}}{\pgfqpoint{2.471731in}{1.143012in}}%
\pgfpathcurveto{\pgfqpoint{2.465907in}{1.137188in}}{\pgfqpoint{2.462635in}{1.129288in}}{\pgfqpoint{2.462635in}{1.121052in}}%
\pgfpathcurveto{\pgfqpoint{2.462635in}{1.112816in}}{\pgfqpoint{2.465907in}{1.104916in}}{\pgfqpoint{2.471731in}{1.099092in}}%
\pgfpathcurveto{\pgfqpoint{2.477555in}{1.093268in}}{\pgfqpoint{2.485455in}{1.089996in}}{\pgfqpoint{2.493692in}{1.089996in}}%
\pgfpathclose%
\pgfusepath{stroke,fill}%
\end{pgfscope}%
\begin{pgfscope}%
\pgfpathrectangle{\pgfqpoint{0.100000in}{0.212622in}}{\pgfqpoint{3.696000in}{3.696000in}}%
\pgfusepath{clip}%
\pgfsetbuttcap%
\pgfsetroundjoin%
\definecolor{currentfill}{rgb}{0.121569,0.466667,0.705882}%
\pgfsetfillcolor{currentfill}%
\pgfsetfillopacity{0.998214}%
\pgfsetlinewidth{1.003750pt}%
\definecolor{currentstroke}{rgb}{0.121569,0.466667,0.705882}%
\pgfsetstrokecolor{currentstroke}%
\pgfsetstrokeopacity{0.998214}%
\pgfsetdash{}{0pt}%
\pgfpathmoveto{\pgfqpoint{2.493691in}{1.089995in}}%
\pgfpathcurveto{\pgfqpoint{2.501928in}{1.089995in}}{\pgfqpoint{2.509828in}{1.093268in}}{\pgfqpoint{2.515652in}{1.099092in}}%
\pgfpathcurveto{\pgfqpoint{2.521476in}{1.104916in}}{\pgfqpoint{2.524748in}{1.112816in}}{\pgfqpoint{2.524748in}{1.121052in}}%
\pgfpathcurveto{\pgfqpoint{2.524748in}{1.129288in}}{\pgfqpoint{2.521476in}{1.137188in}}{\pgfqpoint{2.515652in}{1.143012in}}%
\pgfpathcurveto{\pgfqpoint{2.509828in}{1.148836in}}{\pgfqpoint{2.501928in}{1.152108in}}{\pgfqpoint{2.493691in}{1.152108in}}%
\pgfpathcurveto{\pgfqpoint{2.485455in}{1.152108in}}{\pgfqpoint{2.477555in}{1.148836in}}{\pgfqpoint{2.471731in}{1.143012in}}%
\pgfpathcurveto{\pgfqpoint{2.465907in}{1.137188in}}{\pgfqpoint{2.462635in}{1.129288in}}{\pgfqpoint{2.462635in}{1.121052in}}%
\pgfpathcurveto{\pgfqpoint{2.462635in}{1.112816in}}{\pgfqpoint{2.465907in}{1.104916in}}{\pgfqpoint{2.471731in}{1.099092in}}%
\pgfpathcurveto{\pgfqpoint{2.477555in}{1.093268in}}{\pgfqpoint{2.485455in}{1.089995in}}{\pgfqpoint{2.493691in}{1.089995in}}%
\pgfpathclose%
\pgfusepath{stroke,fill}%
\end{pgfscope}%
\begin{pgfscope}%
\pgfpathrectangle{\pgfqpoint{0.100000in}{0.212622in}}{\pgfqpoint{3.696000in}{3.696000in}}%
\pgfusepath{clip}%
\pgfsetbuttcap%
\pgfsetroundjoin%
\definecolor{currentfill}{rgb}{0.121569,0.466667,0.705882}%
\pgfsetfillcolor{currentfill}%
\pgfsetfillopacity{0.998214}%
\pgfsetlinewidth{1.003750pt}%
\definecolor{currentstroke}{rgb}{0.121569,0.466667,0.705882}%
\pgfsetstrokecolor{currentstroke}%
\pgfsetstrokeopacity{0.998214}%
\pgfsetdash{}{0pt}%
\pgfpathmoveto{\pgfqpoint{2.493691in}{1.089995in}}%
\pgfpathcurveto{\pgfqpoint{2.501927in}{1.089995in}}{\pgfqpoint{2.509828in}{1.093268in}}{\pgfqpoint{2.515651in}{1.099092in}}%
\pgfpathcurveto{\pgfqpoint{2.521475in}{1.104915in}}{\pgfqpoint{2.524748in}{1.112816in}}{\pgfqpoint{2.524748in}{1.121052in}}%
\pgfpathcurveto{\pgfqpoint{2.524748in}{1.129288in}}{\pgfqpoint{2.521475in}{1.137188in}}{\pgfqpoint{2.515651in}{1.143012in}}%
\pgfpathcurveto{\pgfqpoint{2.509828in}{1.148836in}}{\pgfqpoint{2.501927in}{1.152108in}}{\pgfqpoint{2.493691in}{1.152108in}}%
\pgfpathcurveto{\pgfqpoint{2.485455in}{1.152108in}}{\pgfqpoint{2.477555in}{1.148836in}}{\pgfqpoint{2.471731in}{1.143012in}}%
\pgfpathcurveto{\pgfqpoint{2.465907in}{1.137188in}}{\pgfqpoint{2.462635in}{1.129288in}}{\pgfqpoint{2.462635in}{1.121052in}}%
\pgfpathcurveto{\pgfqpoint{2.462635in}{1.112816in}}{\pgfqpoint{2.465907in}{1.104915in}}{\pgfqpoint{2.471731in}{1.099092in}}%
\pgfpathcurveto{\pgfqpoint{2.477555in}{1.093268in}}{\pgfqpoint{2.485455in}{1.089995in}}{\pgfqpoint{2.493691in}{1.089995in}}%
\pgfpathclose%
\pgfusepath{stroke,fill}%
\end{pgfscope}%
\begin{pgfscope}%
\pgfpathrectangle{\pgfqpoint{0.100000in}{0.212622in}}{\pgfqpoint{3.696000in}{3.696000in}}%
\pgfusepath{clip}%
\pgfsetbuttcap%
\pgfsetroundjoin%
\definecolor{currentfill}{rgb}{0.121569,0.466667,0.705882}%
\pgfsetfillcolor{currentfill}%
\pgfsetfillopacity{0.998214}%
\pgfsetlinewidth{1.003750pt}%
\definecolor{currentstroke}{rgb}{0.121569,0.466667,0.705882}%
\pgfsetstrokecolor{currentstroke}%
\pgfsetstrokeopacity{0.998214}%
\pgfsetdash{}{0pt}%
\pgfpathmoveto{\pgfqpoint{2.493691in}{1.089995in}}%
\pgfpathcurveto{\pgfqpoint{2.501927in}{1.089995in}}{\pgfqpoint{2.509827in}{1.093267in}}{\pgfqpoint{2.515651in}{1.099091in}}%
\pgfpathcurveto{\pgfqpoint{2.521475in}{1.104915in}}{\pgfqpoint{2.524747in}{1.112815in}}{\pgfqpoint{2.524747in}{1.121052in}}%
\pgfpathcurveto{\pgfqpoint{2.524747in}{1.129288in}}{\pgfqpoint{2.521475in}{1.137188in}}{\pgfqpoint{2.515651in}{1.143012in}}%
\pgfpathcurveto{\pgfqpoint{2.509827in}{1.148836in}}{\pgfqpoint{2.501927in}{1.152108in}}{\pgfqpoint{2.493691in}{1.152108in}}%
\pgfpathcurveto{\pgfqpoint{2.485454in}{1.152108in}}{\pgfqpoint{2.477554in}{1.148836in}}{\pgfqpoint{2.471730in}{1.143012in}}%
\pgfpathcurveto{\pgfqpoint{2.465906in}{1.137188in}}{\pgfqpoint{2.462634in}{1.129288in}}{\pgfqpoint{2.462634in}{1.121052in}}%
\pgfpathcurveto{\pgfqpoint{2.462634in}{1.112815in}}{\pgfqpoint{2.465906in}{1.104915in}}{\pgfqpoint{2.471730in}{1.099091in}}%
\pgfpathcurveto{\pgfqpoint{2.477554in}{1.093267in}}{\pgfqpoint{2.485454in}{1.089995in}}{\pgfqpoint{2.493691in}{1.089995in}}%
\pgfpathclose%
\pgfusepath{stroke,fill}%
\end{pgfscope}%
\begin{pgfscope}%
\pgfpathrectangle{\pgfqpoint{0.100000in}{0.212622in}}{\pgfqpoint{3.696000in}{3.696000in}}%
\pgfusepath{clip}%
\pgfsetbuttcap%
\pgfsetroundjoin%
\definecolor{currentfill}{rgb}{0.121569,0.466667,0.705882}%
\pgfsetfillcolor{currentfill}%
\pgfsetfillopacity{0.998215}%
\pgfsetlinewidth{1.003750pt}%
\definecolor{currentstroke}{rgb}{0.121569,0.466667,0.705882}%
\pgfsetstrokecolor{currentstroke}%
\pgfsetstrokeopacity{0.998215}%
\pgfsetdash{}{0pt}%
\pgfpathmoveto{\pgfqpoint{2.493690in}{1.089995in}}%
\pgfpathcurveto{\pgfqpoint{2.501926in}{1.089995in}}{\pgfqpoint{2.509826in}{1.093267in}}{\pgfqpoint{2.515650in}{1.099091in}}%
\pgfpathcurveto{\pgfqpoint{2.521474in}{1.104915in}}{\pgfqpoint{2.524746in}{1.112815in}}{\pgfqpoint{2.524746in}{1.121051in}}%
\pgfpathcurveto{\pgfqpoint{2.524746in}{1.129287in}}{\pgfqpoint{2.521474in}{1.137187in}}{\pgfqpoint{2.515650in}{1.143011in}}%
\pgfpathcurveto{\pgfqpoint{2.509826in}{1.148835in}}{\pgfqpoint{2.501926in}{1.152108in}}{\pgfqpoint{2.493690in}{1.152108in}}%
\pgfpathcurveto{\pgfqpoint{2.485453in}{1.152108in}}{\pgfqpoint{2.477553in}{1.148835in}}{\pgfqpoint{2.471729in}{1.143011in}}%
\pgfpathcurveto{\pgfqpoint{2.465906in}{1.137187in}}{\pgfqpoint{2.462633in}{1.129287in}}{\pgfqpoint{2.462633in}{1.121051in}}%
\pgfpathcurveto{\pgfqpoint{2.462633in}{1.112815in}}{\pgfqpoint{2.465906in}{1.104915in}}{\pgfqpoint{2.471729in}{1.099091in}}%
\pgfpathcurveto{\pgfqpoint{2.477553in}{1.093267in}}{\pgfqpoint{2.485453in}{1.089995in}}{\pgfqpoint{2.493690in}{1.089995in}}%
\pgfpathclose%
\pgfusepath{stroke,fill}%
\end{pgfscope}%
\begin{pgfscope}%
\pgfpathrectangle{\pgfqpoint{0.100000in}{0.212622in}}{\pgfqpoint{3.696000in}{3.696000in}}%
\pgfusepath{clip}%
\pgfsetbuttcap%
\pgfsetroundjoin%
\definecolor{currentfill}{rgb}{0.121569,0.466667,0.705882}%
\pgfsetfillcolor{currentfill}%
\pgfsetfillopacity{0.998215}%
\pgfsetlinewidth{1.003750pt}%
\definecolor{currentstroke}{rgb}{0.121569,0.466667,0.705882}%
\pgfsetstrokecolor{currentstroke}%
\pgfsetstrokeopacity{0.998215}%
\pgfsetdash{}{0pt}%
\pgfpathmoveto{\pgfqpoint{2.493688in}{1.089994in}}%
\pgfpathcurveto{\pgfqpoint{2.501924in}{1.089994in}}{\pgfqpoint{2.509824in}{1.093266in}}{\pgfqpoint{2.515648in}{1.099090in}}%
\pgfpathcurveto{\pgfqpoint{2.521472in}{1.104914in}}{\pgfqpoint{2.524744in}{1.112814in}}{\pgfqpoint{2.524744in}{1.121050in}}%
\pgfpathcurveto{\pgfqpoint{2.524744in}{1.129287in}}{\pgfqpoint{2.521472in}{1.137187in}}{\pgfqpoint{2.515648in}{1.143011in}}%
\pgfpathcurveto{\pgfqpoint{2.509824in}{1.148835in}}{\pgfqpoint{2.501924in}{1.152107in}}{\pgfqpoint{2.493688in}{1.152107in}}%
\pgfpathcurveto{\pgfqpoint{2.485452in}{1.152107in}}{\pgfqpoint{2.477552in}{1.148835in}}{\pgfqpoint{2.471728in}{1.143011in}}%
\pgfpathcurveto{\pgfqpoint{2.465904in}{1.137187in}}{\pgfqpoint{2.462631in}{1.129287in}}{\pgfqpoint{2.462631in}{1.121050in}}%
\pgfpathcurveto{\pgfqpoint{2.462631in}{1.112814in}}{\pgfqpoint{2.465904in}{1.104914in}}{\pgfqpoint{2.471728in}{1.099090in}}%
\pgfpathcurveto{\pgfqpoint{2.477552in}{1.093266in}}{\pgfqpoint{2.485452in}{1.089994in}}{\pgfqpoint{2.493688in}{1.089994in}}%
\pgfpathclose%
\pgfusepath{stroke,fill}%
\end{pgfscope}%
\begin{pgfscope}%
\pgfpathrectangle{\pgfqpoint{0.100000in}{0.212622in}}{\pgfqpoint{3.696000in}{3.696000in}}%
\pgfusepath{clip}%
\pgfsetbuttcap%
\pgfsetroundjoin%
\definecolor{currentfill}{rgb}{0.121569,0.466667,0.705882}%
\pgfsetfillcolor{currentfill}%
\pgfsetfillopacity{0.998217}%
\pgfsetlinewidth{1.003750pt}%
\definecolor{currentstroke}{rgb}{0.121569,0.466667,0.705882}%
\pgfsetstrokecolor{currentstroke}%
\pgfsetstrokeopacity{0.998217}%
\pgfsetdash{}{0pt}%
\pgfpathmoveto{\pgfqpoint{2.493685in}{1.089993in}}%
\pgfpathcurveto{\pgfqpoint{2.501921in}{1.089993in}}{\pgfqpoint{2.509821in}{1.093265in}}{\pgfqpoint{2.515645in}{1.099089in}}%
\pgfpathcurveto{\pgfqpoint{2.521469in}{1.104913in}}{\pgfqpoint{2.524741in}{1.112813in}}{\pgfqpoint{2.524741in}{1.121049in}}%
\pgfpathcurveto{\pgfqpoint{2.524741in}{1.129286in}}{\pgfqpoint{2.521469in}{1.137186in}}{\pgfqpoint{2.515645in}{1.143009in}}%
\pgfpathcurveto{\pgfqpoint{2.509821in}{1.148833in}}{\pgfqpoint{2.501921in}{1.152106in}}{\pgfqpoint{2.493685in}{1.152106in}}%
\pgfpathcurveto{\pgfqpoint{2.485448in}{1.152106in}}{\pgfqpoint{2.477548in}{1.148833in}}{\pgfqpoint{2.471724in}{1.143009in}}%
\pgfpathcurveto{\pgfqpoint{2.465900in}{1.137186in}}{\pgfqpoint{2.462628in}{1.129286in}}{\pgfqpoint{2.462628in}{1.121049in}}%
\pgfpathcurveto{\pgfqpoint{2.462628in}{1.112813in}}{\pgfqpoint{2.465900in}{1.104913in}}{\pgfqpoint{2.471724in}{1.099089in}}%
\pgfpathcurveto{\pgfqpoint{2.477548in}{1.093265in}}{\pgfqpoint{2.485448in}{1.089993in}}{\pgfqpoint{2.493685in}{1.089993in}}%
\pgfpathclose%
\pgfusepath{stroke,fill}%
\end{pgfscope}%
\begin{pgfscope}%
\pgfpathrectangle{\pgfqpoint{0.100000in}{0.212622in}}{\pgfqpoint{3.696000in}{3.696000in}}%
\pgfusepath{clip}%
\pgfsetbuttcap%
\pgfsetroundjoin%
\definecolor{currentfill}{rgb}{0.121569,0.466667,0.705882}%
\pgfsetfillcolor{currentfill}%
\pgfsetfillopacity{0.998219}%
\pgfsetlinewidth{1.003750pt}%
\definecolor{currentstroke}{rgb}{0.121569,0.466667,0.705882}%
\pgfsetstrokecolor{currentstroke}%
\pgfsetstrokeopacity{0.998219}%
\pgfsetdash{}{0pt}%
\pgfpathmoveto{\pgfqpoint{2.493678in}{1.089991in}}%
\pgfpathcurveto{\pgfqpoint{2.501914in}{1.089991in}}{\pgfqpoint{2.509814in}{1.093263in}}{\pgfqpoint{2.515638in}{1.099087in}}%
\pgfpathcurveto{\pgfqpoint{2.521462in}{1.104911in}}{\pgfqpoint{2.524735in}{1.112811in}}{\pgfqpoint{2.524735in}{1.121047in}}%
\pgfpathcurveto{\pgfqpoint{2.524735in}{1.129284in}}{\pgfqpoint{2.521462in}{1.137184in}}{\pgfqpoint{2.515638in}{1.143008in}}%
\pgfpathcurveto{\pgfqpoint{2.509814in}{1.148832in}}{\pgfqpoint{2.501914in}{1.152104in}}{\pgfqpoint{2.493678in}{1.152104in}}%
\pgfpathcurveto{\pgfqpoint{2.485442in}{1.152104in}}{\pgfqpoint{2.477542in}{1.148832in}}{\pgfqpoint{2.471718in}{1.143008in}}%
\pgfpathcurveto{\pgfqpoint{2.465894in}{1.137184in}}{\pgfqpoint{2.462622in}{1.129284in}}{\pgfqpoint{2.462622in}{1.121047in}}%
\pgfpathcurveto{\pgfqpoint{2.462622in}{1.112811in}}{\pgfqpoint{2.465894in}{1.104911in}}{\pgfqpoint{2.471718in}{1.099087in}}%
\pgfpathcurveto{\pgfqpoint{2.477542in}{1.093263in}}{\pgfqpoint{2.485442in}{1.089991in}}{\pgfqpoint{2.493678in}{1.089991in}}%
\pgfpathclose%
\pgfusepath{stroke,fill}%
\end{pgfscope}%
\begin{pgfscope}%
\pgfpathrectangle{\pgfqpoint{0.100000in}{0.212622in}}{\pgfqpoint{3.696000in}{3.696000in}}%
\pgfusepath{clip}%
\pgfsetbuttcap%
\pgfsetroundjoin%
\definecolor{currentfill}{rgb}{0.121569,0.466667,0.705882}%
\pgfsetfillcolor{currentfill}%
\pgfsetfillopacity{0.998222}%
\pgfsetlinewidth{1.003750pt}%
\definecolor{currentstroke}{rgb}{0.121569,0.466667,0.705882}%
\pgfsetstrokecolor{currentstroke}%
\pgfsetstrokeopacity{0.998222}%
\pgfsetdash{}{0pt}%
\pgfpathmoveto{\pgfqpoint{2.493666in}{1.089988in}}%
\pgfpathcurveto{\pgfqpoint{2.501902in}{1.089988in}}{\pgfqpoint{2.509802in}{1.093261in}}{\pgfqpoint{2.515626in}{1.099084in}}%
\pgfpathcurveto{\pgfqpoint{2.521450in}{1.104908in}}{\pgfqpoint{2.524722in}{1.112808in}}{\pgfqpoint{2.524722in}{1.121045in}}%
\pgfpathcurveto{\pgfqpoint{2.524722in}{1.129281in}}{\pgfqpoint{2.521450in}{1.137181in}}{\pgfqpoint{2.515626in}{1.143005in}}%
\pgfpathcurveto{\pgfqpoint{2.509802in}{1.148829in}}{\pgfqpoint{2.501902in}{1.152101in}}{\pgfqpoint{2.493666in}{1.152101in}}%
\pgfpathcurveto{\pgfqpoint{2.485430in}{1.152101in}}{\pgfqpoint{2.477530in}{1.148829in}}{\pgfqpoint{2.471706in}{1.143005in}}%
\pgfpathcurveto{\pgfqpoint{2.465882in}{1.137181in}}{\pgfqpoint{2.462609in}{1.129281in}}{\pgfqpoint{2.462609in}{1.121045in}}%
\pgfpathcurveto{\pgfqpoint{2.462609in}{1.112808in}}{\pgfqpoint{2.465882in}{1.104908in}}{\pgfqpoint{2.471706in}{1.099084in}}%
\pgfpathcurveto{\pgfqpoint{2.477530in}{1.093261in}}{\pgfqpoint{2.485430in}{1.089988in}}{\pgfqpoint{2.493666in}{1.089988in}}%
\pgfpathclose%
\pgfusepath{stroke,fill}%
\end{pgfscope}%
\begin{pgfscope}%
\pgfpathrectangle{\pgfqpoint{0.100000in}{0.212622in}}{\pgfqpoint{3.696000in}{3.696000in}}%
\pgfusepath{clip}%
\pgfsetbuttcap%
\pgfsetroundjoin%
\definecolor{currentfill}{rgb}{0.121569,0.466667,0.705882}%
\pgfsetfillcolor{currentfill}%
\pgfsetfillopacity{0.998229}%
\pgfsetlinewidth{1.003750pt}%
\definecolor{currentstroke}{rgb}{0.121569,0.466667,0.705882}%
\pgfsetstrokecolor{currentstroke}%
\pgfsetstrokeopacity{0.998229}%
\pgfsetdash{}{0pt}%
\pgfpathmoveto{\pgfqpoint{2.493643in}{1.089985in}}%
\pgfpathcurveto{\pgfqpoint{2.501880in}{1.089985in}}{\pgfqpoint{2.509780in}{1.093257in}}{\pgfqpoint{2.515603in}{1.099081in}}%
\pgfpathcurveto{\pgfqpoint{2.521427in}{1.104905in}}{\pgfqpoint{2.524700in}{1.112805in}}{\pgfqpoint{2.524700in}{1.121041in}}%
\pgfpathcurveto{\pgfqpoint{2.524700in}{1.129277in}}{\pgfqpoint{2.521427in}{1.137177in}}{\pgfqpoint{2.515603in}{1.143001in}}%
\pgfpathcurveto{\pgfqpoint{2.509780in}{1.148825in}}{\pgfqpoint{2.501880in}{1.152098in}}{\pgfqpoint{2.493643in}{1.152098in}}%
\pgfpathcurveto{\pgfqpoint{2.485407in}{1.152098in}}{\pgfqpoint{2.477507in}{1.148825in}}{\pgfqpoint{2.471683in}{1.143001in}}%
\pgfpathcurveto{\pgfqpoint{2.465859in}{1.137177in}}{\pgfqpoint{2.462587in}{1.129277in}}{\pgfqpoint{2.462587in}{1.121041in}}%
\pgfpathcurveto{\pgfqpoint{2.462587in}{1.112805in}}{\pgfqpoint{2.465859in}{1.104905in}}{\pgfqpoint{2.471683in}{1.099081in}}%
\pgfpathcurveto{\pgfqpoint{2.477507in}{1.093257in}}{\pgfqpoint{2.485407in}{1.089985in}}{\pgfqpoint{2.493643in}{1.089985in}}%
\pgfpathclose%
\pgfusepath{stroke,fill}%
\end{pgfscope}%
\begin{pgfscope}%
\pgfpathrectangle{\pgfqpoint{0.100000in}{0.212622in}}{\pgfqpoint{3.696000in}{3.696000in}}%
\pgfusepath{clip}%
\pgfsetbuttcap%
\pgfsetroundjoin%
\definecolor{currentfill}{rgb}{0.121569,0.466667,0.705882}%
\pgfsetfillcolor{currentfill}%
\pgfsetfillopacity{0.998240}%
\pgfsetlinewidth{1.003750pt}%
\definecolor{currentstroke}{rgb}{0.121569,0.466667,0.705882}%
\pgfsetstrokecolor{currentstroke}%
\pgfsetstrokeopacity{0.998240}%
\pgfsetdash{}{0pt}%
\pgfpathmoveto{\pgfqpoint{2.493601in}{1.089981in}}%
\pgfpathcurveto{\pgfqpoint{2.501837in}{1.089981in}}{\pgfqpoint{2.509737in}{1.093253in}}{\pgfqpoint{2.515561in}{1.099077in}}%
\pgfpathcurveto{\pgfqpoint{2.521385in}{1.104901in}}{\pgfqpoint{2.524657in}{1.112801in}}{\pgfqpoint{2.524657in}{1.121037in}}%
\pgfpathcurveto{\pgfqpoint{2.524657in}{1.129273in}}{\pgfqpoint{2.521385in}{1.137173in}}{\pgfqpoint{2.515561in}{1.142997in}}%
\pgfpathcurveto{\pgfqpoint{2.509737in}{1.148821in}}{\pgfqpoint{2.501837in}{1.152094in}}{\pgfqpoint{2.493601in}{1.152094in}}%
\pgfpathcurveto{\pgfqpoint{2.485364in}{1.152094in}}{\pgfqpoint{2.477464in}{1.148821in}}{\pgfqpoint{2.471640in}{1.142997in}}%
\pgfpathcurveto{\pgfqpoint{2.465817in}{1.137173in}}{\pgfqpoint{2.462544in}{1.129273in}}{\pgfqpoint{2.462544in}{1.121037in}}%
\pgfpathcurveto{\pgfqpoint{2.462544in}{1.112801in}}{\pgfqpoint{2.465817in}{1.104901in}}{\pgfqpoint{2.471640in}{1.099077in}}%
\pgfpathcurveto{\pgfqpoint{2.477464in}{1.093253in}}{\pgfqpoint{2.485364in}{1.089981in}}{\pgfqpoint{2.493601in}{1.089981in}}%
\pgfpathclose%
\pgfusepath{stroke,fill}%
\end{pgfscope}%
\begin{pgfscope}%
\pgfpathrectangle{\pgfqpoint{0.100000in}{0.212622in}}{\pgfqpoint{3.696000in}{3.696000in}}%
\pgfusepath{clip}%
\pgfsetbuttcap%
\pgfsetroundjoin%
\definecolor{currentfill}{rgb}{0.121569,0.466667,0.705882}%
\pgfsetfillcolor{currentfill}%
\pgfsetfillopacity{0.998258}%
\pgfsetlinewidth{1.003750pt}%
\definecolor{currentstroke}{rgb}{0.121569,0.466667,0.705882}%
\pgfsetstrokecolor{currentstroke}%
\pgfsetstrokeopacity{0.998258}%
\pgfsetdash{}{0pt}%
\pgfpathmoveto{\pgfqpoint{2.493522in}{1.089977in}}%
\pgfpathcurveto{\pgfqpoint{2.501758in}{1.089977in}}{\pgfqpoint{2.509658in}{1.093249in}}{\pgfqpoint{2.515482in}{1.099073in}}%
\pgfpathcurveto{\pgfqpoint{2.521306in}{1.104897in}}{\pgfqpoint{2.524578in}{1.112797in}}{\pgfqpoint{2.524578in}{1.121033in}}%
\pgfpathcurveto{\pgfqpoint{2.524578in}{1.129270in}}{\pgfqpoint{2.521306in}{1.137170in}}{\pgfqpoint{2.515482in}{1.142994in}}%
\pgfpathcurveto{\pgfqpoint{2.509658in}{1.148817in}}{\pgfqpoint{2.501758in}{1.152090in}}{\pgfqpoint{2.493522in}{1.152090in}}%
\pgfpathcurveto{\pgfqpoint{2.485286in}{1.152090in}}{\pgfqpoint{2.477386in}{1.148817in}}{\pgfqpoint{2.471562in}{1.142994in}}%
\pgfpathcurveto{\pgfqpoint{2.465738in}{1.137170in}}{\pgfqpoint{2.462465in}{1.129270in}}{\pgfqpoint{2.462465in}{1.121033in}}%
\pgfpathcurveto{\pgfqpoint{2.462465in}{1.112797in}}{\pgfqpoint{2.465738in}{1.104897in}}{\pgfqpoint{2.471562in}{1.099073in}}%
\pgfpathcurveto{\pgfqpoint{2.477386in}{1.093249in}}{\pgfqpoint{2.485286in}{1.089977in}}{\pgfqpoint{2.493522in}{1.089977in}}%
\pgfpathclose%
\pgfusepath{stroke,fill}%
\end{pgfscope}%
\begin{pgfscope}%
\pgfpathrectangle{\pgfqpoint{0.100000in}{0.212622in}}{\pgfqpoint{3.696000in}{3.696000in}}%
\pgfusepath{clip}%
\pgfsetbuttcap%
\pgfsetroundjoin%
\definecolor{currentfill}{rgb}{0.121569,0.466667,0.705882}%
\pgfsetfillcolor{currentfill}%
\pgfsetfillopacity{0.998289}%
\pgfsetlinewidth{1.003750pt}%
\definecolor{currentstroke}{rgb}{0.121569,0.466667,0.705882}%
\pgfsetstrokecolor{currentstroke}%
\pgfsetstrokeopacity{0.998289}%
\pgfsetdash{}{0pt}%
\pgfpathmoveto{\pgfqpoint{2.493376in}{1.089977in}}%
\pgfpathcurveto{\pgfqpoint{2.501612in}{1.089977in}}{\pgfqpoint{2.509513in}{1.093249in}}{\pgfqpoint{2.515336in}{1.099073in}}%
\pgfpathcurveto{\pgfqpoint{2.521160in}{1.104897in}}{\pgfqpoint{2.524433in}{1.112797in}}{\pgfqpoint{2.524433in}{1.121033in}}%
\pgfpathcurveto{\pgfqpoint{2.524433in}{1.129270in}}{\pgfqpoint{2.521160in}{1.137170in}}{\pgfqpoint{2.515336in}{1.142994in}}%
\pgfpathcurveto{\pgfqpoint{2.509513in}{1.148818in}}{\pgfqpoint{2.501612in}{1.152090in}}{\pgfqpoint{2.493376in}{1.152090in}}%
\pgfpathcurveto{\pgfqpoint{2.485140in}{1.152090in}}{\pgfqpoint{2.477240in}{1.148818in}}{\pgfqpoint{2.471416in}{1.142994in}}%
\pgfpathcurveto{\pgfqpoint{2.465592in}{1.137170in}}{\pgfqpoint{2.462320in}{1.129270in}}{\pgfqpoint{2.462320in}{1.121033in}}%
\pgfpathcurveto{\pgfqpoint{2.462320in}{1.112797in}}{\pgfqpoint{2.465592in}{1.104897in}}{\pgfqpoint{2.471416in}{1.099073in}}%
\pgfpathcurveto{\pgfqpoint{2.477240in}{1.093249in}}{\pgfqpoint{2.485140in}{1.089977in}}{\pgfqpoint{2.493376in}{1.089977in}}%
\pgfpathclose%
\pgfusepath{stroke,fill}%
\end{pgfscope}%
\begin{pgfscope}%
\pgfpathrectangle{\pgfqpoint{0.100000in}{0.212622in}}{\pgfqpoint{3.696000in}{3.696000in}}%
\pgfusepath{clip}%
\pgfsetbuttcap%
\pgfsetroundjoin%
\definecolor{currentfill}{rgb}{0.121569,0.466667,0.705882}%
\pgfsetfillcolor{currentfill}%
\pgfsetfillopacity{0.998340}%
\pgfsetlinewidth{1.003750pt}%
\definecolor{currentstroke}{rgb}{0.121569,0.466667,0.705882}%
\pgfsetstrokecolor{currentstroke}%
\pgfsetstrokeopacity{0.998340}%
\pgfsetdash{}{0pt}%
\pgfpathmoveto{\pgfqpoint{2.493108in}{1.089987in}}%
\pgfpathcurveto{\pgfqpoint{2.501345in}{1.089987in}}{\pgfqpoint{2.509245in}{1.093259in}}{\pgfqpoint{2.515068in}{1.099083in}}%
\pgfpathcurveto{\pgfqpoint{2.520892in}{1.104907in}}{\pgfqpoint{2.524165in}{1.112807in}}{\pgfqpoint{2.524165in}{1.121044in}}%
\pgfpathcurveto{\pgfqpoint{2.524165in}{1.129280in}}{\pgfqpoint{2.520892in}{1.137180in}}{\pgfqpoint{2.515068in}{1.143004in}}%
\pgfpathcurveto{\pgfqpoint{2.509245in}{1.148828in}}{\pgfqpoint{2.501345in}{1.152100in}}{\pgfqpoint{2.493108in}{1.152100in}}%
\pgfpathcurveto{\pgfqpoint{2.484872in}{1.152100in}}{\pgfqpoint{2.476972in}{1.148828in}}{\pgfqpoint{2.471148in}{1.143004in}}%
\pgfpathcurveto{\pgfqpoint{2.465324in}{1.137180in}}{\pgfqpoint{2.462052in}{1.129280in}}{\pgfqpoint{2.462052in}{1.121044in}}%
\pgfpathcurveto{\pgfqpoint{2.462052in}{1.112807in}}{\pgfqpoint{2.465324in}{1.104907in}}{\pgfqpoint{2.471148in}{1.099083in}}%
\pgfpathcurveto{\pgfqpoint{2.476972in}{1.093259in}}{\pgfqpoint{2.484872in}{1.089987in}}{\pgfqpoint{2.493108in}{1.089987in}}%
\pgfpathclose%
\pgfusepath{stroke,fill}%
\end{pgfscope}%
\begin{pgfscope}%
\pgfpathrectangle{\pgfqpoint{0.100000in}{0.212622in}}{\pgfqpoint{3.696000in}{3.696000in}}%
\pgfusepath{clip}%
\pgfsetbuttcap%
\pgfsetroundjoin%
\definecolor{currentfill}{rgb}{0.121569,0.466667,0.705882}%
\pgfsetfillcolor{currentfill}%
\pgfsetfillopacity{0.998425}%
\pgfsetlinewidth{1.003750pt}%
\definecolor{currentstroke}{rgb}{0.121569,0.466667,0.705882}%
\pgfsetstrokecolor{currentstroke}%
\pgfsetstrokeopacity{0.998425}%
\pgfsetdash{}{0pt}%
\pgfpathmoveto{\pgfqpoint{2.492618in}{1.090027in}}%
\pgfpathcurveto{\pgfqpoint{2.500854in}{1.090027in}}{\pgfqpoint{2.508754in}{1.093299in}}{\pgfqpoint{2.514578in}{1.099123in}}%
\pgfpathcurveto{\pgfqpoint{2.520402in}{1.104947in}}{\pgfqpoint{2.523674in}{1.112847in}}{\pgfqpoint{2.523674in}{1.121083in}}%
\pgfpathcurveto{\pgfqpoint{2.523674in}{1.129319in}}{\pgfqpoint{2.520402in}{1.137219in}}{\pgfqpoint{2.514578in}{1.143043in}}%
\pgfpathcurveto{\pgfqpoint{2.508754in}{1.148867in}}{\pgfqpoint{2.500854in}{1.152140in}}{\pgfqpoint{2.492618in}{1.152140in}}%
\pgfpathcurveto{\pgfqpoint{2.484382in}{1.152140in}}{\pgfqpoint{2.476482in}{1.148867in}}{\pgfqpoint{2.470658in}{1.143043in}}%
\pgfpathcurveto{\pgfqpoint{2.464834in}{1.137219in}}{\pgfqpoint{2.461561in}{1.129319in}}{\pgfqpoint{2.461561in}{1.121083in}}%
\pgfpathcurveto{\pgfqpoint{2.461561in}{1.112847in}}{\pgfqpoint{2.464834in}{1.104947in}}{\pgfqpoint{2.470658in}{1.099123in}}%
\pgfpathcurveto{\pgfqpoint{2.476482in}{1.093299in}}{\pgfqpoint{2.484382in}{1.090027in}}{\pgfqpoint{2.492618in}{1.090027in}}%
\pgfpathclose%
\pgfusepath{stroke,fill}%
\end{pgfscope}%
\begin{pgfscope}%
\pgfpathrectangle{\pgfqpoint{0.100000in}{0.212622in}}{\pgfqpoint{3.696000in}{3.696000in}}%
\pgfusepath{clip}%
\pgfsetbuttcap%
\pgfsetroundjoin%
\definecolor{currentfill}{rgb}{0.121569,0.466667,0.705882}%
\pgfsetfillcolor{currentfill}%
\pgfsetfillopacity{0.998558}%
\pgfsetlinewidth{1.003750pt}%
\definecolor{currentstroke}{rgb}{0.121569,0.466667,0.705882}%
\pgfsetstrokecolor{currentstroke}%
\pgfsetstrokeopacity{0.998558}%
\pgfsetdash{}{0pt}%
\pgfpathmoveto{\pgfqpoint{2.491724in}{1.090142in}}%
\pgfpathcurveto{\pgfqpoint{2.499961in}{1.090142in}}{\pgfqpoint{2.507861in}{1.093415in}}{\pgfqpoint{2.513685in}{1.099238in}}%
\pgfpathcurveto{\pgfqpoint{2.519509in}{1.105062in}}{\pgfqpoint{2.522781in}{1.112962in}}{\pgfqpoint{2.522781in}{1.121199in}}%
\pgfpathcurveto{\pgfqpoint{2.522781in}{1.129435in}}{\pgfqpoint{2.519509in}{1.137335in}}{\pgfqpoint{2.513685in}{1.143159in}}%
\pgfpathcurveto{\pgfqpoint{2.507861in}{1.148983in}}{\pgfqpoint{2.499961in}{1.152255in}}{\pgfqpoint{2.491724in}{1.152255in}}%
\pgfpathcurveto{\pgfqpoint{2.483488in}{1.152255in}}{\pgfqpoint{2.475588in}{1.148983in}}{\pgfqpoint{2.469764in}{1.143159in}}%
\pgfpathcurveto{\pgfqpoint{2.463940in}{1.137335in}}{\pgfqpoint{2.460668in}{1.129435in}}{\pgfqpoint{2.460668in}{1.121199in}}%
\pgfpathcurveto{\pgfqpoint{2.460668in}{1.112962in}}{\pgfqpoint{2.463940in}{1.105062in}}{\pgfqpoint{2.469764in}{1.099238in}}%
\pgfpathcurveto{\pgfqpoint{2.475588in}{1.093415in}}{\pgfqpoint{2.483488in}{1.090142in}}{\pgfqpoint{2.491724in}{1.090142in}}%
\pgfpathclose%
\pgfusepath{stroke,fill}%
\end{pgfscope}%
\begin{pgfscope}%
\pgfpathrectangle{\pgfqpoint{0.100000in}{0.212622in}}{\pgfqpoint{3.696000in}{3.696000in}}%
\pgfusepath{clip}%
\pgfsetbuttcap%
\pgfsetroundjoin%
\definecolor{currentfill}{rgb}{0.121569,0.466667,0.705882}%
\pgfsetfillcolor{currentfill}%
\pgfsetfillopacity{0.998559}%
\pgfsetlinewidth{1.003750pt}%
\definecolor{currentstroke}{rgb}{0.121569,0.466667,0.705882}%
\pgfsetstrokecolor{currentstroke}%
\pgfsetstrokeopacity{0.998559}%
\pgfsetdash{}{0pt}%
\pgfpathmoveto{\pgfqpoint{2.406606in}{1.116974in}}%
\pgfpathcurveto{\pgfqpoint{2.414843in}{1.116974in}}{\pgfqpoint{2.422743in}{1.120246in}}{\pgfqpoint{2.428567in}{1.126070in}}%
\pgfpathcurveto{\pgfqpoint{2.434391in}{1.131894in}}{\pgfqpoint{2.437663in}{1.139794in}}{\pgfqpoint{2.437663in}{1.148030in}}%
\pgfpathcurveto{\pgfqpoint{2.437663in}{1.156266in}}{\pgfqpoint{2.434391in}{1.164166in}}{\pgfqpoint{2.428567in}{1.169990in}}%
\pgfpathcurveto{\pgfqpoint{2.422743in}{1.175814in}}{\pgfqpoint{2.414843in}{1.179087in}}{\pgfqpoint{2.406606in}{1.179087in}}%
\pgfpathcurveto{\pgfqpoint{2.398370in}{1.179087in}}{\pgfqpoint{2.390470in}{1.175814in}}{\pgfqpoint{2.384646in}{1.169990in}}%
\pgfpathcurveto{\pgfqpoint{2.378822in}{1.164166in}}{\pgfqpoint{2.375550in}{1.156266in}}{\pgfqpoint{2.375550in}{1.148030in}}%
\pgfpathcurveto{\pgfqpoint{2.375550in}{1.139794in}}{\pgfqpoint{2.378822in}{1.131894in}}{\pgfqpoint{2.384646in}{1.126070in}}%
\pgfpathcurveto{\pgfqpoint{2.390470in}{1.120246in}}{\pgfqpoint{2.398370in}{1.116974in}}{\pgfqpoint{2.406606in}{1.116974in}}%
\pgfpathclose%
\pgfusepath{stroke,fill}%
\end{pgfscope}%
\begin{pgfscope}%
\pgfpathrectangle{\pgfqpoint{0.100000in}{0.212622in}}{\pgfqpoint{3.696000in}{3.696000in}}%
\pgfusepath{clip}%
\pgfsetbuttcap%
\pgfsetroundjoin%
\definecolor{currentfill}{rgb}{0.121569,0.466667,0.705882}%
\pgfsetfillcolor{currentfill}%
\pgfsetfillopacity{0.998776}%
\pgfsetlinewidth{1.003750pt}%
\definecolor{currentstroke}{rgb}{0.121569,0.466667,0.705882}%
\pgfsetstrokecolor{currentstroke}%
\pgfsetstrokeopacity{0.998776}%
\pgfsetdash{}{0pt}%
\pgfpathmoveto{\pgfqpoint{2.490103in}{1.090434in}}%
\pgfpathcurveto{\pgfqpoint{2.498340in}{1.090434in}}{\pgfqpoint{2.506240in}{1.093707in}}{\pgfqpoint{2.512064in}{1.099531in}}%
\pgfpathcurveto{\pgfqpoint{2.517887in}{1.105355in}}{\pgfqpoint{2.521160in}{1.113255in}}{\pgfqpoint{2.521160in}{1.121491in}}%
\pgfpathcurveto{\pgfqpoint{2.521160in}{1.129727in}}{\pgfqpoint{2.517887in}{1.137627in}}{\pgfqpoint{2.512064in}{1.143451in}}%
\pgfpathcurveto{\pgfqpoint{2.506240in}{1.149275in}}{\pgfqpoint{2.498340in}{1.152547in}}{\pgfqpoint{2.490103in}{1.152547in}}%
\pgfpathcurveto{\pgfqpoint{2.481867in}{1.152547in}}{\pgfqpoint{2.473967in}{1.149275in}}{\pgfqpoint{2.468143in}{1.143451in}}%
\pgfpathcurveto{\pgfqpoint{2.462319in}{1.137627in}}{\pgfqpoint{2.459047in}{1.129727in}}{\pgfqpoint{2.459047in}{1.121491in}}%
\pgfpathcurveto{\pgfqpoint{2.459047in}{1.113255in}}{\pgfqpoint{2.462319in}{1.105355in}}{\pgfqpoint{2.468143in}{1.099531in}}%
\pgfpathcurveto{\pgfqpoint{2.473967in}{1.093707in}}{\pgfqpoint{2.481867in}{1.090434in}}{\pgfqpoint{2.490103in}{1.090434in}}%
\pgfpathclose%
\pgfusepath{stroke,fill}%
\end{pgfscope}%
\begin{pgfscope}%
\pgfpathrectangle{\pgfqpoint{0.100000in}{0.212622in}}{\pgfqpoint{3.696000in}{3.696000in}}%
\pgfusepath{clip}%
\pgfsetbuttcap%
\pgfsetroundjoin%
\definecolor{currentfill}{rgb}{0.121569,0.466667,0.705882}%
\pgfsetfillcolor{currentfill}%
\pgfsetfillopacity{0.999085}%
\pgfsetlinewidth{1.003750pt}%
\definecolor{currentstroke}{rgb}{0.121569,0.466667,0.705882}%
\pgfsetstrokecolor{currentstroke}%
\pgfsetstrokeopacity{0.999085}%
\pgfsetdash{}{0pt}%
\pgfpathmoveto{\pgfqpoint{2.487179in}{1.091047in}}%
\pgfpathcurveto{\pgfqpoint{2.495416in}{1.091047in}}{\pgfqpoint{2.503316in}{1.094319in}}{\pgfqpoint{2.509139in}{1.100143in}}%
\pgfpathcurveto{\pgfqpoint{2.514963in}{1.105967in}}{\pgfqpoint{2.518236in}{1.113867in}}{\pgfqpoint{2.518236in}{1.122103in}}%
\pgfpathcurveto{\pgfqpoint{2.518236in}{1.130339in}}{\pgfqpoint{2.514963in}{1.138239in}}{\pgfqpoint{2.509139in}{1.144063in}}%
\pgfpathcurveto{\pgfqpoint{2.503316in}{1.149887in}}{\pgfqpoint{2.495416in}{1.153160in}}{\pgfqpoint{2.487179in}{1.153160in}}%
\pgfpathcurveto{\pgfqpoint{2.478943in}{1.153160in}}{\pgfqpoint{2.471043in}{1.149887in}}{\pgfqpoint{2.465219in}{1.144063in}}%
\pgfpathcurveto{\pgfqpoint{2.459395in}{1.138239in}}{\pgfqpoint{2.456123in}{1.130339in}}{\pgfqpoint{2.456123in}{1.122103in}}%
\pgfpathcurveto{\pgfqpoint{2.456123in}{1.113867in}}{\pgfqpoint{2.459395in}{1.105967in}}{\pgfqpoint{2.465219in}{1.100143in}}%
\pgfpathcurveto{\pgfqpoint{2.471043in}{1.094319in}}{\pgfqpoint{2.478943in}{1.091047in}}{\pgfqpoint{2.487179in}{1.091047in}}%
\pgfpathclose%
\pgfusepath{stroke,fill}%
\end{pgfscope}%
\begin{pgfscope}%
\pgfpathrectangle{\pgfqpoint{0.100000in}{0.212622in}}{\pgfqpoint{3.696000in}{3.696000in}}%
\pgfusepath{clip}%
\pgfsetbuttcap%
\pgfsetroundjoin%
\definecolor{currentfill}{rgb}{0.121569,0.466667,0.705882}%
\pgfsetfillcolor{currentfill}%
\pgfsetfillopacity{0.999113}%
\pgfsetlinewidth{1.003750pt}%
\definecolor{currentstroke}{rgb}{0.121569,0.466667,0.705882}%
\pgfsetstrokecolor{currentstroke}%
\pgfsetstrokeopacity{0.999113}%
\pgfsetdash{}{0pt}%
\pgfpathmoveto{\pgfqpoint{2.419716in}{1.111436in}}%
\pgfpathcurveto{\pgfqpoint{2.427952in}{1.111436in}}{\pgfqpoint{2.435852in}{1.114708in}}{\pgfqpoint{2.441676in}{1.120532in}}%
\pgfpathcurveto{\pgfqpoint{2.447500in}{1.126356in}}{\pgfqpoint{2.450772in}{1.134256in}}{\pgfqpoint{2.450772in}{1.142493in}}%
\pgfpathcurveto{\pgfqpoint{2.450772in}{1.150729in}}{\pgfqpoint{2.447500in}{1.158629in}}{\pgfqpoint{2.441676in}{1.164453in}}%
\pgfpathcurveto{\pgfqpoint{2.435852in}{1.170277in}}{\pgfqpoint{2.427952in}{1.173549in}}{\pgfqpoint{2.419716in}{1.173549in}}%
\pgfpathcurveto{\pgfqpoint{2.411479in}{1.173549in}}{\pgfqpoint{2.403579in}{1.170277in}}{\pgfqpoint{2.397756in}{1.164453in}}%
\pgfpathcurveto{\pgfqpoint{2.391932in}{1.158629in}}{\pgfqpoint{2.388659in}{1.150729in}}{\pgfqpoint{2.388659in}{1.142493in}}%
\pgfpathcurveto{\pgfqpoint{2.388659in}{1.134256in}}{\pgfqpoint{2.391932in}{1.126356in}}{\pgfqpoint{2.397756in}{1.120532in}}%
\pgfpathcurveto{\pgfqpoint{2.403579in}{1.114708in}}{\pgfqpoint{2.411479in}{1.111436in}}{\pgfqpoint{2.419716in}{1.111436in}}%
\pgfpathclose%
\pgfusepath{stroke,fill}%
\end{pgfscope}%
\begin{pgfscope}%
\pgfpathrectangle{\pgfqpoint{0.100000in}{0.212622in}}{\pgfqpoint{3.696000in}{3.696000in}}%
\pgfusepath{clip}%
\pgfsetbuttcap%
\pgfsetroundjoin%
\definecolor{currentfill}{rgb}{0.121569,0.466667,0.705882}%
\pgfsetfillcolor{currentfill}%
\pgfsetfillopacity{0.999486}%
\pgfsetlinewidth{1.003750pt}%
\definecolor{currentstroke}{rgb}{0.121569,0.466667,0.705882}%
\pgfsetstrokecolor{currentstroke}%
\pgfsetstrokeopacity{0.999486}%
\pgfsetdash{}{0pt}%
\pgfpathmoveto{\pgfqpoint{2.481920in}{1.092116in}}%
\pgfpathcurveto{\pgfqpoint{2.490156in}{1.092116in}}{\pgfqpoint{2.498056in}{1.095388in}}{\pgfqpoint{2.503880in}{1.101212in}}%
\pgfpathcurveto{\pgfqpoint{2.509704in}{1.107036in}}{\pgfqpoint{2.512976in}{1.114936in}}{\pgfqpoint{2.512976in}{1.123172in}}%
\pgfpathcurveto{\pgfqpoint{2.512976in}{1.131409in}}{\pgfqpoint{2.509704in}{1.139309in}}{\pgfqpoint{2.503880in}{1.145133in}}%
\pgfpathcurveto{\pgfqpoint{2.498056in}{1.150957in}}{\pgfqpoint{2.490156in}{1.154229in}}{\pgfqpoint{2.481920in}{1.154229in}}%
\pgfpathcurveto{\pgfqpoint{2.473683in}{1.154229in}}{\pgfqpoint{2.465783in}{1.150957in}}{\pgfqpoint{2.459959in}{1.145133in}}%
\pgfpathcurveto{\pgfqpoint{2.454135in}{1.139309in}}{\pgfqpoint{2.450863in}{1.131409in}}{\pgfqpoint{2.450863in}{1.123172in}}%
\pgfpathcurveto{\pgfqpoint{2.450863in}{1.114936in}}{\pgfqpoint{2.454135in}{1.107036in}}{\pgfqpoint{2.459959in}{1.101212in}}%
\pgfpathcurveto{\pgfqpoint{2.465783in}{1.095388in}}{\pgfqpoint{2.473683in}{1.092116in}}{\pgfqpoint{2.481920in}{1.092116in}}%
\pgfpathclose%
\pgfusepath{stroke,fill}%
\end{pgfscope}%
\begin{pgfscope}%
\pgfpathrectangle{\pgfqpoint{0.100000in}{0.212622in}}{\pgfqpoint{3.696000in}{3.696000in}}%
\pgfusepath{clip}%
\pgfsetbuttcap%
\pgfsetroundjoin%
\definecolor{currentfill}{rgb}{0.121569,0.466667,0.705882}%
\pgfsetfillcolor{currentfill}%
\pgfsetfillopacity{0.999587}%
\pgfsetlinewidth{1.003750pt}%
\definecolor{currentstroke}{rgb}{0.121569,0.466667,0.705882}%
\pgfsetstrokecolor{currentstroke}%
\pgfsetstrokeopacity{0.999587}%
\pgfsetdash{}{0pt}%
\pgfpathmoveto{\pgfqpoint{2.433578in}{1.107184in}}%
\pgfpathcurveto{\pgfqpoint{2.441814in}{1.107184in}}{\pgfqpoint{2.449714in}{1.110456in}}{\pgfqpoint{2.455538in}{1.116280in}}%
\pgfpathcurveto{\pgfqpoint{2.461362in}{1.122104in}}{\pgfqpoint{2.464634in}{1.130004in}}{\pgfqpoint{2.464634in}{1.138240in}}%
\pgfpathcurveto{\pgfqpoint{2.464634in}{1.146476in}}{\pgfqpoint{2.461362in}{1.154377in}}{\pgfqpoint{2.455538in}{1.160200in}}%
\pgfpathcurveto{\pgfqpoint{2.449714in}{1.166024in}}{\pgfqpoint{2.441814in}{1.169297in}}{\pgfqpoint{2.433578in}{1.169297in}}%
\pgfpathcurveto{\pgfqpoint{2.425342in}{1.169297in}}{\pgfqpoint{2.417441in}{1.166024in}}{\pgfqpoint{2.411618in}{1.160200in}}%
\pgfpathcurveto{\pgfqpoint{2.405794in}{1.154377in}}{\pgfqpoint{2.402521in}{1.146476in}}{\pgfqpoint{2.402521in}{1.138240in}}%
\pgfpathcurveto{\pgfqpoint{2.402521in}{1.130004in}}{\pgfqpoint{2.405794in}{1.122104in}}{\pgfqpoint{2.411618in}{1.116280in}}%
\pgfpathcurveto{\pgfqpoint{2.417441in}{1.110456in}}{\pgfqpoint{2.425342in}{1.107184in}}{\pgfqpoint{2.433578in}{1.107184in}}%
\pgfpathclose%
\pgfusepath{stroke,fill}%
\end{pgfscope}%
\begin{pgfscope}%
\pgfpathrectangle{\pgfqpoint{0.100000in}{0.212622in}}{\pgfqpoint{3.696000in}{3.696000in}}%
\pgfusepath{clip}%
\pgfsetbuttcap%
\pgfsetroundjoin%
\definecolor{currentfill}{rgb}{0.121569,0.466667,0.705882}%
\pgfsetfillcolor{currentfill}%
\pgfsetfillopacity{0.999794}%
\pgfsetlinewidth{1.003750pt}%
\definecolor{currentstroke}{rgb}{0.121569,0.466667,0.705882}%
\pgfsetstrokecolor{currentstroke}%
\pgfsetstrokeopacity{0.999794}%
\pgfsetdash{}{0pt}%
\pgfpathmoveto{\pgfqpoint{2.448572in}{1.102176in}}%
\pgfpathcurveto{\pgfqpoint{2.456808in}{1.102176in}}{\pgfqpoint{2.464708in}{1.105448in}}{\pgfqpoint{2.470532in}{1.111272in}}%
\pgfpathcurveto{\pgfqpoint{2.476356in}{1.117096in}}{\pgfqpoint{2.479629in}{1.124996in}}{\pgfqpoint{2.479629in}{1.133232in}}%
\pgfpathcurveto{\pgfqpoint{2.479629in}{1.141469in}}{\pgfqpoint{2.476356in}{1.149369in}}{\pgfqpoint{2.470532in}{1.155193in}}%
\pgfpathcurveto{\pgfqpoint{2.464708in}{1.161017in}}{\pgfqpoint{2.456808in}{1.164289in}}{\pgfqpoint{2.448572in}{1.164289in}}%
\pgfpathcurveto{\pgfqpoint{2.440336in}{1.164289in}}{\pgfqpoint{2.432436in}{1.161017in}}{\pgfqpoint{2.426612in}{1.155193in}}%
\pgfpathcurveto{\pgfqpoint{2.420788in}{1.149369in}}{\pgfqpoint{2.417516in}{1.141469in}}{\pgfqpoint{2.417516in}{1.133232in}}%
\pgfpathcurveto{\pgfqpoint{2.417516in}{1.124996in}}{\pgfqpoint{2.420788in}{1.117096in}}{\pgfqpoint{2.426612in}{1.111272in}}%
\pgfpathcurveto{\pgfqpoint{2.432436in}{1.105448in}}{\pgfqpoint{2.440336in}{1.102176in}}{\pgfqpoint{2.448572in}{1.102176in}}%
\pgfpathclose%
\pgfusepath{stroke,fill}%
\end{pgfscope}%
\begin{pgfscope}%
\pgfpathrectangle{\pgfqpoint{0.100000in}{0.212622in}}{\pgfqpoint{3.696000in}{3.696000in}}%
\pgfusepath{clip}%
\pgfsetbuttcap%
\pgfsetroundjoin%
\definecolor{currentfill}{rgb}{0.121569,0.466667,0.705882}%
\pgfsetfillcolor{currentfill}%
\pgfsetfillopacity{0.999889}%
\pgfsetlinewidth{1.003750pt}%
\definecolor{currentstroke}{rgb}{0.121569,0.466667,0.705882}%
\pgfsetstrokecolor{currentstroke}%
\pgfsetstrokeopacity{0.999889}%
\pgfsetdash{}{0pt}%
\pgfpathmoveto{\pgfqpoint{2.472565in}{1.094323in}}%
\pgfpathcurveto{\pgfqpoint{2.480801in}{1.094323in}}{\pgfqpoint{2.488701in}{1.097596in}}{\pgfqpoint{2.494525in}{1.103420in}}%
\pgfpathcurveto{\pgfqpoint{2.500349in}{1.109243in}}{\pgfqpoint{2.503621in}{1.117144in}}{\pgfqpoint{2.503621in}{1.125380in}}%
\pgfpathcurveto{\pgfqpoint{2.503621in}{1.133616in}}{\pgfqpoint{2.500349in}{1.141516in}}{\pgfqpoint{2.494525in}{1.147340in}}%
\pgfpathcurveto{\pgfqpoint{2.488701in}{1.153164in}}{\pgfqpoint{2.480801in}{1.156436in}}{\pgfqpoint{2.472565in}{1.156436in}}%
\pgfpathcurveto{\pgfqpoint{2.464329in}{1.156436in}}{\pgfqpoint{2.456429in}{1.153164in}}{\pgfqpoint{2.450605in}{1.147340in}}%
\pgfpathcurveto{\pgfqpoint{2.444781in}{1.141516in}}{\pgfqpoint{2.441509in}{1.133616in}}{\pgfqpoint{2.441509in}{1.125380in}}%
\pgfpathcurveto{\pgfqpoint{2.441509in}{1.117144in}}{\pgfqpoint{2.444781in}{1.109243in}}{\pgfqpoint{2.450605in}{1.103420in}}%
\pgfpathcurveto{\pgfqpoint{2.456429in}{1.097596in}}{\pgfqpoint{2.464329in}{1.094323in}}{\pgfqpoint{2.472565in}{1.094323in}}%
\pgfpathclose%
\pgfusepath{stroke,fill}%
\end{pgfscope}%
\begin{pgfscope}%
\pgfpathrectangle{\pgfqpoint{0.100000in}{0.212622in}}{\pgfqpoint{3.696000in}{3.696000in}}%
\pgfusepath{clip}%
\pgfsetbuttcap%
\pgfsetroundjoin%
\definecolor{currentfill}{rgb}{0.121569,0.466667,0.705882}%
\pgfsetfillcolor{currentfill}%
\pgfsetlinewidth{1.003750pt}%
\definecolor{currentstroke}{rgb}{0.121569,0.466667,0.705882}%
\pgfsetstrokecolor{currentstroke}%
\pgfsetdash{}{0pt}%
\pgfpathmoveto{\pgfqpoint{2.463906in}{1.097039in}}%
\pgfpathcurveto{\pgfqpoint{2.472142in}{1.097039in}}{\pgfqpoint{2.480042in}{1.100311in}}{\pgfqpoint{2.485866in}{1.106135in}}%
\pgfpathcurveto{\pgfqpoint{2.491690in}{1.111959in}}{\pgfqpoint{2.494962in}{1.119859in}}{\pgfqpoint{2.494962in}{1.128095in}}%
\pgfpathcurveto{\pgfqpoint{2.494962in}{1.136331in}}{\pgfqpoint{2.491690in}{1.144232in}}{\pgfqpoint{2.485866in}{1.150055in}}%
\pgfpathcurveto{\pgfqpoint{2.480042in}{1.155879in}}{\pgfqpoint{2.472142in}{1.159152in}}{\pgfqpoint{2.463906in}{1.159152in}}%
\pgfpathcurveto{\pgfqpoint{2.455669in}{1.159152in}}{\pgfqpoint{2.447769in}{1.155879in}}{\pgfqpoint{2.441945in}{1.150055in}}%
\pgfpathcurveto{\pgfqpoint{2.436121in}{1.144232in}}{\pgfqpoint{2.432849in}{1.136331in}}{\pgfqpoint{2.432849in}{1.128095in}}%
\pgfpathcurveto{\pgfqpoint{2.432849in}{1.119859in}}{\pgfqpoint{2.436121in}{1.111959in}}{\pgfqpoint{2.441945in}{1.106135in}}%
\pgfpathcurveto{\pgfqpoint{2.447769in}{1.100311in}}{\pgfqpoint{2.455669in}{1.097039in}}{\pgfqpoint{2.463906in}{1.097039in}}%
\pgfpathclose%
\pgfusepath{stroke,fill}%
\end{pgfscope}%
\begin{pgfscope}%
\pgfsetbuttcap%
\pgfsetmiterjoin%
\definecolor{currentfill}{rgb}{1.000000,1.000000,1.000000}%
\pgfsetfillcolor{currentfill}%
\pgfsetfillopacity{0.800000}%
\pgfsetlinewidth{1.003750pt}%
\definecolor{currentstroke}{rgb}{0.800000,0.800000,0.800000}%
\pgfsetstrokecolor{currentstroke}%
\pgfsetstrokeopacity{0.800000}%
\pgfsetdash{}{0pt}%
\pgfpathmoveto{\pgfqpoint{2.104889in}{3.216678in}}%
\pgfpathlineto{\pgfqpoint{3.698778in}{3.216678in}}%
\pgfpathquadraticcurveto{\pgfqpoint{3.726556in}{3.216678in}}{\pgfqpoint{3.726556in}{3.244456in}}%
\pgfpathlineto{\pgfqpoint{3.726556in}{3.811400in}}%
\pgfpathquadraticcurveto{\pgfqpoint{3.726556in}{3.839178in}}{\pgfqpoint{3.698778in}{3.839178in}}%
\pgfpathlineto{\pgfqpoint{2.104889in}{3.839178in}}%
\pgfpathquadraticcurveto{\pgfqpoint{2.077111in}{3.839178in}}{\pgfqpoint{2.077111in}{3.811400in}}%
\pgfpathlineto{\pgfqpoint{2.077111in}{3.244456in}}%
\pgfpathquadraticcurveto{\pgfqpoint{2.077111in}{3.216678in}}{\pgfqpoint{2.104889in}{3.216678in}}%
\pgfpathclose%
\pgfusepath{stroke,fill}%
\end{pgfscope}%
\begin{pgfscope}%
\pgfsetrectcap%
\pgfsetroundjoin%
\pgfsetlinewidth{1.505625pt}%
\definecolor{currentstroke}{rgb}{0.121569,0.466667,0.705882}%
\pgfsetstrokecolor{currentstroke}%
\pgfsetdash{}{0pt}%
\pgfpathmoveto{\pgfqpoint{2.132667in}{3.735011in}}%
\pgfpathlineto{\pgfqpoint{2.410444in}{3.735011in}}%
\pgfusepath{stroke}%
\end{pgfscope}%
\begin{pgfscope}%
\definecolor{textcolor}{rgb}{0.000000,0.000000,0.000000}%
\pgfsetstrokecolor{textcolor}%
\pgfsetfillcolor{textcolor}%
\pgftext[x=2.521555in,y=3.686400in,left,base]{\color{textcolor}\rmfamily\fontsize{10.000000}{12.000000}\selectfont Ground truth}%
\end{pgfscope}%
\begin{pgfscope}%
\pgfsetbuttcap%
\pgfsetroundjoin%
\definecolor{currentfill}{rgb}{0.121569,0.466667,0.705882}%
\pgfsetfillcolor{currentfill}%
\pgfsetlinewidth{1.003750pt}%
\definecolor{currentstroke}{rgb}{0.121569,0.466667,0.705882}%
\pgfsetstrokecolor{currentstroke}%
\pgfsetdash{}{0pt}%
\pgfsys@defobject{currentmarker}{\pgfqpoint{-0.031056in}{-0.031056in}}{\pgfqpoint{0.031056in}{0.031056in}}{%
\pgfpathmoveto{\pgfqpoint{0.000000in}{-0.031056in}}%
\pgfpathcurveto{\pgfqpoint{0.008236in}{-0.031056in}}{\pgfqpoint{0.016136in}{-0.027784in}}{\pgfqpoint{0.021960in}{-0.021960in}}%
\pgfpathcurveto{\pgfqpoint{0.027784in}{-0.016136in}}{\pgfqpoint{0.031056in}{-0.008236in}}{\pgfqpoint{0.031056in}{0.000000in}}%
\pgfpathcurveto{\pgfqpoint{0.031056in}{0.008236in}}{\pgfqpoint{0.027784in}{0.016136in}}{\pgfqpoint{0.021960in}{0.021960in}}%
\pgfpathcurveto{\pgfqpoint{0.016136in}{0.027784in}}{\pgfqpoint{0.008236in}{0.031056in}}{\pgfqpoint{0.000000in}{0.031056in}}%
\pgfpathcurveto{\pgfqpoint{-0.008236in}{0.031056in}}{\pgfqpoint{-0.016136in}{0.027784in}}{\pgfqpoint{-0.021960in}{0.021960in}}%
\pgfpathcurveto{\pgfqpoint{-0.027784in}{0.016136in}}{\pgfqpoint{-0.031056in}{0.008236in}}{\pgfqpoint{-0.031056in}{0.000000in}}%
\pgfpathcurveto{\pgfqpoint{-0.031056in}{-0.008236in}}{\pgfqpoint{-0.027784in}{-0.016136in}}{\pgfqpoint{-0.021960in}{-0.021960in}}%
\pgfpathcurveto{\pgfqpoint{-0.016136in}{-0.027784in}}{\pgfqpoint{-0.008236in}{-0.031056in}}{\pgfqpoint{0.000000in}{-0.031056in}}%
\pgfpathclose%
\pgfusepath{stroke,fill}%
}%
\begin{pgfscope}%
\pgfsys@transformshift{2.271555in}{3.529248in}%
\pgfsys@useobject{currentmarker}{}%
\end{pgfscope}%
\end{pgfscope}%
\begin{pgfscope}%
\definecolor{textcolor}{rgb}{0.000000,0.000000,0.000000}%
\pgfsetstrokecolor{textcolor}%
\pgfsetfillcolor{textcolor}%
\pgftext[x=2.521555in,y=3.492789in,left,base]{\color{textcolor}\rmfamily\fontsize{10.000000}{12.000000}\selectfont Estimated position}%
\end{pgfscope}%
\begin{pgfscope}%
\pgfsetbuttcap%
\pgfsetroundjoin%
\definecolor{currentfill}{rgb}{1.000000,0.498039,0.054902}%
\pgfsetfillcolor{currentfill}%
\pgfsetlinewidth{1.003750pt}%
\definecolor{currentstroke}{rgb}{1.000000,0.498039,0.054902}%
\pgfsetstrokecolor{currentstroke}%
\pgfsetdash{}{0pt}%
\pgfsys@defobject{currentmarker}{\pgfqpoint{-0.031056in}{-0.031056in}}{\pgfqpoint{0.031056in}{0.031056in}}{%
\pgfpathmoveto{\pgfqpoint{0.000000in}{-0.031056in}}%
\pgfpathcurveto{\pgfqpoint{0.008236in}{-0.031056in}}{\pgfqpoint{0.016136in}{-0.027784in}}{\pgfqpoint{0.021960in}{-0.021960in}}%
\pgfpathcurveto{\pgfqpoint{0.027784in}{-0.016136in}}{\pgfqpoint{0.031056in}{-0.008236in}}{\pgfqpoint{0.031056in}{0.000000in}}%
\pgfpathcurveto{\pgfqpoint{0.031056in}{0.008236in}}{\pgfqpoint{0.027784in}{0.016136in}}{\pgfqpoint{0.021960in}{0.021960in}}%
\pgfpathcurveto{\pgfqpoint{0.016136in}{0.027784in}}{\pgfqpoint{0.008236in}{0.031056in}}{\pgfqpoint{0.000000in}{0.031056in}}%
\pgfpathcurveto{\pgfqpoint{-0.008236in}{0.031056in}}{\pgfqpoint{-0.016136in}{0.027784in}}{\pgfqpoint{-0.021960in}{0.021960in}}%
\pgfpathcurveto{\pgfqpoint{-0.027784in}{0.016136in}}{\pgfqpoint{-0.031056in}{0.008236in}}{\pgfqpoint{-0.031056in}{0.000000in}}%
\pgfpathcurveto{\pgfqpoint{-0.031056in}{-0.008236in}}{\pgfqpoint{-0.027784in}{-0.016136in}}{\pgfqpoint{-0.021960in}{-0.021960in}}%
\pgfpathcurveto{\pgfqpoint{-0.016136in}{-0.027784in}}{\pgfqpoint{-0.008236in}{-0.031056in}}{\pgfqpoint{0.000000in}{-0.031056in}}%
\pgfpathclose%
\pgfusepath{stroke,fill}%
}%
\begin{pgfscope}%
\pgfsys@transformshift{2.271555in}{3.335637in}%
\pgfsys@useobject{currentmarker}{}%
\end{pgfscope}%
\end{pgfscope}%
\begin{pgfscope}%
\definecolor{textcolor}{rgb}{0.000000,0.000000,0.000000}%
\pgfsetstrokecolor{textcolor}%
\pgfsetfillcolor{textcolor}%
\pgftext[x=2.521555in,y=3.299178in,left,base]{\color{textcolor}\rmfamily\fontsize{10.000000}{12.000000}\selectfont Estimated turn}%
\end{pgfscope}%
\end{pgfpicture}%
\makeatother%
\endgroup%
}
%         \caption{Madgwick's 3D position estimation had the lowest turn error for the spiral experiment.}
%         \label{fig:spiral3D}
%     \end{subfigure}
%     \caption{Position estimation by the best performing algorithms in the spiral experiment.}
%     \label{fig:spiral}
% \end{figure}

% \subsection{Squares}

% For the 28-meter line experiment, the Mahony algorithm which had the lowest displacement error with an average of 6.18 meters (4.17\% of error margin), and ROLEQ with an average of 7.06 meters of turn error (4.77\% of error margin).

% \begin{figure}[!h]
%     \centering
%     \begin{table}[H]
    \begin{center}
        \resizebox{1\linewidth}{!}{
            \begin{tabular}[t]{lcccc}
                \hline
                Algorithm   & Displacement Error[$m$] & Displacement Error[\%] & Turn Error[$m$] & Turn Error[\%] \\
                \hline
                AngularRate & 112.95                  & 134.46                 & 122.64          & 146.00         \\            AQUA            & 26.02  & 30.97 & 35.28 & 42.00              \\            Complementary            & 43.81  & 52.16 & 64.79 & 77.13              \\            Davenport            & 10.97  & 13.05 & 13.38 & 15.93              \\            EKF            & 1.70  & 2.02 & 5.16 & 6.15              \\            FAMC            & 152.41  & 181.44 & 188.05 & 223.87              \\            FLAE            & 12.06  & 14.36 & 17.28 & 20.58              \\            Fourati            & 90.63  & 107.89 & 100.72 & 119.90              \\            Madgwick            & 8.15  & 9.70 & 8.18 & 9.74              \\            Mahony            & 1.46  & 1.73 & 4.27 & 5.09              \\            OLEQ            & 1.69  & 2.01 & 5.05 & 6.01              \\            QUEST            & 82.11  & 97.75 & 152.58 & 181.64              \\            ROLEQ            & 1.13  & 1.35 & 3.64 & 4.34              \\            SAAM            & 11.94  & 14.21 & 14.15 & 16.85              \\            Tilt            & 11.94  & 14.21 & 14.15 & 16.85              \\
                \hline
                Average     & 37.93                   & 45.15                  & 49.95           & 59.47
            \end{tabular}
        }
        \caption{Squares position estimation error (displacement and turn) of the sensor fusion algorithms. }
        \label{tab:squares}
    \end{center}
\end{table}
% \end{figure}

% \begin{figure}[!h]
%     \centering
%     \begin{subfigure}{0.49\textwidth}
%         \centering
%         \resizebox{1\linewidth}{!}{%% Creator: Matplotlib, PGF backend
%%
%% To include the figure in your LaTeX document, write
%%   \input{<filename>.pgf}
%%
%% Make sure the required packages are loaded in your preamble
%%   \usepackage{pgf}
%%
%% and, on pdftex
%%   \usepackage[utf8]{inputenc}\DeclareUnicodeCharacter{2212}{-}
%%
%% or, on luatex and xetex
%%   \usepackage{unicode-math}
%%
%% Figures using additional raster images can only be included by \input if
%% they are in the same directory as the main LaTeX file. For loading figures
%% from other directories you can use the `import` package
%%   \usepackage{import}
%%
%% and then include the figures with
%%   \import{<path to file>}{<filename>.pgf}
%%
%% Matplotlib used the following preamble
%%   \usepackage{fontspec}
%%
\begingroup%
\makeatletter%
\begin{pgfpicture}%
\pgfpathrectangle{\pgfpointorigin}{\pgfqpoint{5.590556in}{4.311000in}}%
\pgfusepath{use as bounding box, clip}%
\begin{pgfscope}%
\pgfsetbuttcap%
\pgfsetmiterjoin%
\definecolor{currentfill}{rgb}{1.000000,1.000000,1.000000}%
\pgfsetfillcolor{currentfill}%
\pgfsetlinewidth{0.000000pt}%
\definecolor{currentstroke}{rgb}{1.000000,1.000000,1.000000}%
\pgfsetstrokecolor{currentstroke}%
\pgfsetdash{}{0pt}%
\pgfpathmoveto{\pgfqpoint{0.000000in}{0.000000in}}%
\pgfpathlineto{\pgfqpoint{5.590556in}{0.000000in}}%
\pgfpathlineto{\pgfqpoint{5.590556in}{4.311000in}}%
\pgfpathlineto{\pgfqpoint{0.000000in}{4.311000in}}%
\pgfpathclose%
\pgfusepath{fill}%
\end{pgfscope}%
\begin{pgfscope}%
\pgfsetbuttcap%
\pgfsetmiterjoin%
\definecolor{currentfill}{rgb}{1.000000,1.000000,1.000000}%
\pgfsetfillcolor{currentfill}%
\pgfsetlinewidth{0.000000pt}%
\definecolor{currentstroke}{rgb}{0.000000,0.000000,0.000000}%
\pgfsetstrokecolor{currentstroke}%
\pgfsetstrokeopacity{0.000000}%
\pgfsetdash{}{0pt}%
\pgfpathmoveto{\pgfqpoint{0.530556in}{0.515000in}}%
\pgfpathlineto{\pgfqpoint{5.490556in}{0.515000in}}%
\pgfpathlineto{\pgfqpoint{5.490556in}{4.211000in}}%
\pgfpathlineto{\pgfqpoint{0.530556in}{4.211000in}}%
\pgfpathclose%
\pgfusepath{fill}%
\end{pgfscope}%
\begin{pgfscope}%
\pgfpathrectangle{\pgfqpoint{0.530556in}{0.515000in}}{\pgfqpoint{4.960000in}{3.696000in}}%
\pgfusepath{clip}%
\pgfsetbuttcap%
\pgfsetroundjoin%
\definecolor{currentfill}{rgb}{0.121569,0.466667,0.705882}%
\pgfsetfillcolor{currentfill}%
\pgfsetlinewidth{1.003750pt}%
\definecolor{currentstroke}{rgb}{0.121569,0.466667,0.705882}%
\pgfsetstrokecolor{currentstroke}%
\pgfsetdash{}{0pt}%
\pgfsys@defobject{currentmarker}{\pgfqpoint{-0.041667in}{-0.041667in}}{\pgfqpoint{0.041667in}{0.041667in}}{%
\pgfpathmoveto{\pgfqpoint{0.000000in}{-0.041667in}}%
\pgfpathcurveto{\pgfqpoint{0.011050in}{-0.041667in}}{\pgfqpoint{0.021649in}{-0.037276in}}{\pgfqpoint{0.029463in}{-0.029463in}}%
\pgfpathcurveto{\pgfqpoint{0.037276in}{-0.021649in}}{\pgfqpoint{0.041667in}{-0.011050in}}{\pgfqpoint{0.041667in}{0.000000in}}%
\pgfpathcurveto{\pgfqpoint{0.041667in}{0.011050in}}{\pgfqpoint{0.037276in}{0.021649in}}{\pgfqpoint{0.029463in}{0.029463in}}%
\pgfpathcurveto{\pgfqpoint{0.021649in}{0.037276in}}{\pgfqpoint{0.011050in}{0.041667in}}{\pgfqpoint{0.000000in}{0.041667in}}%
\pgfpathcurveto{\pgfqpoint{-0.011050in}{0.041667in}}{\pgfqpoint{-0.021649in}{0.037276in}}{\pgfqpoint{-0.029463in}{0.029463in}}%
\pgfpathcurveto{\pgfqpoint{-0.037276in}{0.021649in}}{\pgfqpoint{-0.041667in}{0.011050in}}{\pgfqpoint{-0.041667in}{0.000000in}}%
\pgfpathcurveto{\pgfqpoint{-0.041667in}{-0.011050in}}{\pgfqpoint{-0.037276in}{-0.021649in}}{\pgfqpoint{-0.029463in}{-0.029463in}}%
\pgfpathcurveto{\pgfqpoint{-0.021649in}{-0.037276in}}{\pgfqpoint{-0.011050in}{-0.041667in}}{\pgfqpoint{0.000000in}{-0.041667in}}%
\pgfpathclose%
\pgfusepath{stroke,fill}%
}%
\begin{pgfscope}%
\pgfsys@transformshift{1.244158in}{0.884903in}%
\pgfsys@useobject{currentmarker}{}%
\end{pgfscope}%
\begin{pgfscope}%
\pgfsys@transformshift{1.244158in}{0.884903in}%
\pgfsys@useobject{currentmarker}{}%
\end{pgfscope}%
\begin{pgfscope}%
\pgfsys@transformshift{1.244158in}{0.884903in}%
\pgfsys@useobject{currentmarker}{}%
\end{pgfscope}%
\begin{pgfscope}%
\pgfsys@transformshift{1.244158in}{0.884903in}%
\pgfsys@useobject{currentmarker}{}%
\end{pgfscope}%
\begin{pgfscope}%
\pgfsys@transformshift{1.244158in}{0.884903in}%
\pgfsys@useobject{currentmarker}{}%
\end{pgfscope}%
\begin{pgfscope}%
\pgfsys@transformshift{1.244158in}{0.884903in}%
\pgfsys@useobject{currentmarker}{}%
\end{pgfscope}%
\begin{pgfscope}%
\pgfsys@transformshift{1.244158in}{0.884903in}%
\pgfsys@useobject{currentmarker}{}%
\end{pgfscope}%
\begin{pgfscope}%
\pgfsys@transformshift{1.244158in}{0.884903in}%
\pgfsys@useobject{currentmarker}{}%
\end{pgfscope}%
\begin{pgfscope}%
\pgfsys@transformshift{1.244158in}{0.884903in}%
\pgfsys@useobject{currentmarker}{}%
\end{pgfscope}%
\begin{pgfscope}%
\pgfsys@transformshift{1.244158in}{0.884903in}%
\pgfsys@useobject{currentmarker}{}%
\end{pgfscope}%
\begin{pgfscope}%
\pgfsys@transformshift{1.244158in}{0.884903in}%
\pgfsys@useobject{currentmarker}{}%
\end{pgfscope}%
\begin{pgfscope}%
\pgfsys@transformshift{1.244158in}{0.884903in}%
\pgfsys@useobject{currentmarker}{}%
\end{pgfscope}%
\begin{pgfscope}%
\pgfsys@transformshift{1.244158in}{0.884903in}%
\pgfsys@useobject{currentmarker}{}%
\end{pgfscope}%
\begin{pgfscope}%
\pgfsys@transformshift{1.244158in}{0.884903in}%
\pgfsys@useobject{currentmarker}{}%
\end{pgfscope}%
\begin{pgfscope}%
\pgfsys@transformshift{1.244158in}{0.884903in}%
\pgfsys@useobject{currentmarker}{}%
\end{pgfscope}%
\begin{pgfscope}%
\pgfsys@transformshift{1.244158in}{0.884903in}%
\pgfsys@useobject{currentmarker}{}%
\end{pgfscope}%
\begin{pgfscope}%
\pgfsys@transformshift{1.244158in}{0.884903in}%
\pgfsys@useobject{currentmarker}{}%
\end{pgfscope}%
\begin{pgfscope}%
\pgfsys@transformshift{1.244158in}{0.884903in}%
\pgfsys@useobject{currentmarker}{}%
\end{pgfscope}%
\begin{pgfscope}%
\pgfsys@transformshift{1.244158in}{0.884903in}%
\pgfsys@useobject{currentmarker}{}%
\end{pgfscope}%
\begin{pgfscope}%
\pgfsys@transformshift{1.244158in}{0.884903in}%
\pgfsys@useobject{currentmarker}{}%
\end{pgfscope}%
\begin{pgfscope}%
\pgfsys@transformshift{1.244158in}{0.884903in}%
\pgfsys@useobject{currentmarker}{}%
\end{pgfscope}%
\begin{pgfscope}%
\pgfsys@transformshift{1.244158in}{0.884903in}%
\pgfsys@useobject{currentmarker}{}%
\end{pgfscope}%
\begin{pgfscope}%
\pgfsys@transformshift{1.244158in}{0.884903in}%
\pgfsys@useobject{currentmarker}{}%
\end{pgfscope}%
\begin{pgfscope}%
\pgfsys@transformshift{1.244158in}{0.884903in}%
\pgfsys@useobject{currentmarker}{}%
\end{pgfscope}%
\begin{pgfscope}%
\pgfsys@transformshift{1.244158in}{0.884903in}%
\pgfsys@useobject{currentmarker}{}%
\end{pgfscope}%
\begin{pgfscope}%
\pgfsys@transformshift{1.244158in}{0.884903in}%
\pgfsys@useobject{currentmarker}{}%
\end{pgfscope}%
\begin{pgfscope}%
\pgfsys@transformshift{1.244158in}{0.884903in}%
\pgfsys@useobject{currentmarker}{}%
\end{pgfscope}%
\begin{pgfscope}%
\pgfsys@transformshift{1.244158in}{0.884903in}%
\pgfsys@useobject{currentmarker}{}%
\end{pgfscope}%
\begin{pgfscope}%
\pgfsys@transformshift{1.244158in}{0.884903in}%
\pgfsys@useobject{currentmarker}{}%
\end{pgfscope}%
\begin{pgfscope}%
\pgfsys@transformshift{1.244158in}{0.884903in}%
\pgfsys@useobject{currentmarker}{}%
\end{pgfscope}%
\begin{pgfscope}%
\pgfsys@transformshift{1.244158in}{0.884903in}%
\pgfsys@useobject{currentmarker}{}%
\end{pgfscope}%
\begin{pgfscope}%
\pgfsys@transformshift{1.244158in}{0.884903in}%
\pgfsys@useobject{currentmarker}{}%
\end{pgfscope}%
\begin{pgfscope}%
\pgfsys@transformshift{1.244158in}{0.884903in}%
\pgfsys@useobject{currentmarker}{}%
\end{pgfscope}%
\begin{pgfscope}%
\pgfsys@transformshift{1.244158in}{0.884903in}%
\pgfsys@useobject{currentmarker}{}%
\end{pgfscope}%
\begin{pgfscope}%
\pgfsys@transformshift{1.244158in}{0.884903in}%
\pgfsys@useobject{currentmarker}{}%
\end{pgfscope}%
\begin{pgfscope}%
\pgfsys@transformshift{1.244158in}{0.884903in}%
\pgfsys@useobject{currentmarker}{}%
\end{pgfscope}%
\begin{pgfscope}%
\pgfsys@transformshift{1.244158in}{0.884903in}%
\pgfsys@useobject{currentmarker}{}%
\end{pgfscope}%
\begin{pgfscope}%
\pgfsys@transformshift{1.244158in}{0.884903in}%
\pgfsys@useobject{currentmarker}{}%
\end{pgfscope}%
\begin{pgfscope}%
\pgfsys@transformshift{1.244158in}{0.884903in}%
\pgfsys@useobject{currentmarker}{}%
\end{pgfscope}%
\begin{pgfscope}%
\pgfsys@transformshift{1.244158in}{0.884903in}%
\pgfsys@useobject{currentmarker}{}%
\end{pgfscope}%
\begin{pgfscope}%
\pgfsys@transformshift{1.244158in}{0.884903in}%
\pgfsys@useobject{currentmarker}{}%
\end{pgfscope}%
\begin{pgfscope}%
\pgfsys@transformshift{1.244158in}{0.884903in}%
\pgfsys@useobject{currentmarker}{}%
\end{pgfscope}%
\begin{pgfscope}%
\pgfsys@transformshift{1.244158in}{0.884903in}%
\pgfsys@useobject{currentmarker}{}%
\end{pgfscope}%
\begin{pgfscope}%
\pgfsys@transformshift{1.244158in}{0.884903in}%
\pgfsys@useobject{currentmarker}{}%
\end{pgfscope}%
\begin{pgfscope}%
\pgfsys@transformshift{1.244158in}{0.884903in}%
\pgfsys@useobject{currentmarker}{}%
\end{pgfscope}%
\begin{pgfscope}%
\pgfsys@transformshift{1.244158in}{0.884903in}%
\pgfsys@useobject{currentmarker}{}%
\end{pgfscope}%
\begin{pgfscope}%
\pgfsys@transformshift{1.244158in}{0.884903in}%
\pgfsys@useobject{currentmarker}{}%
\end{pgfscope}%
\begin{pgfscope}%
\pgfsys@transformshift{1.244158in}{0.884903in}%
\pgfsys@useobject{currentmarker}{}%
\end{pgfscope}%
\begin{pgfscope}%
\pgfsys@transformshift{1.244158in}{0.884903in}%
\pgfsys@useobject{currentmarker}{}%
\end{pgfscope}%
\begin{pgfscope}%
\pgfsys@transformshift{1.244158in}{0.884903in}%
\pgfsys@useobject{currentmarker}{}%
\end{pgfscope}%
\begin{pgfscope}%
\pgfsys@transformshift{1.244158in}{0.884903in}%
\pgfsys@useobject{currentmarker}{}%
\end{pgfscope}%
\begin{pgfscope}%
\pgfsys@transformshift{1.244158in}{0.884903in}%
\pgfsys@useobject{currentmarker}{}%
\end{pgfscope}%
\begin{pgfscope}%
\pgfsys@transformshift{1.244158in}{0.884903in}%
\pgfsys@useobject{currentmarker}{}%
\end{pgfscope}%
\begin{pgfscope}%
\pgfsys@transformshift{1.244158in}{0.884903in}%
\pgfsys@useobject{currentmarker}{}%
\end{pgfscope}%
\begin{pgfscope}%
\pgfsys@transformshift{1.244158in}{0.884903in}%
\pgfsys@useobject{currentmarker}{}%
\end{pgfscope}%
\begin{pgfscope}%
\pgfsys@transformshift{1.244158in}{0.884903in}%
\pgfsys@useobject{currentmarker}{}%
\end{pgfscope}%
\begin{pgfscope}%
\pgfsys@transformshift{1.244158in}{0.884903in}%
\pgfsys@useobject{currentmarker}{}%
\end{pgfscope}%
\begin{pgfscope}%
\pgfsys@transformshift{1.245039in}{0.884753in}%
\pgfsys@useobject{currentmarker}{}%
\end{pgfscope}%
\begin{pgfscope}%
\pgfsys@transformshift{1.246738in}{0.885405in}%
\pgfsys@useobject{currentmarker}{}%
\end{pgfscope}%
\begin{pgfscope}%
\pgfsys@transformshift{1.248801in}{0.886470in}%
\pgfsys@useobject{currentmarker}{}%
\end{pgfscope}%
\begin{pgfscope}%
\pgfsys@transformshift{1.249227in}{0.887674in}%
\pgfsys@useobject{currentmarker}{}%
\end{pgfscope}%
\begin{pgfscope}%
\pgfsys@transformshift{1.249803in}{0.890722in}%
\pgfsys@useobject{currentmarker}{}%
\end{pgfscope}%
\begin{pgfscope}%
\pgfsys@transformshift{1.249230in}{0.895005in}%
\pgfsys@useobject{currentmarker}{}%
\end{pgfscope}%
\begin{pgfscope}%
\pgfsys@transformshift{1.248548in}{0.901400in}%
\pgfsys@useobject{currentmarker}{}%
\end{pgfscope}%
\begin{pgfscope}%
\pgfsys@transformshift{1.247337in}{0.909709in}%
\pgfsys@useobject{currentmarker}{}%
\end{pgfscope}%
\begin{pgfscope}%
\pgfsys@transformshift{1.246264in}{0.919982in}%
\pgfsys@useobject{currentmarker}{}%
\end{pgfscope}%
\begin{pgfscope}%
\pgfsys@transformshift{1.244473in}{0.931810in}%
\pgfsys@useobject{currentmarker}{}%
\end{pgfscope}%
\begin{pgfscope}%
\pgfsys@transformshift{1.245726in}{0.945440in}%
\pgfsys@useobject{currentmarker}{}%
\end{pgfscope}%
\begin{pgfscope}%
\pgfsys@transformshift{1.247132in}{0.960593in}%
\pgfsys@useobject{currentmarker}{}%
\end{pgfscope}%
\begin{pgfscope}%
\pgfsys@transformshift{1.248232in}{0.976474in}%
\pgfsys@useobject{currentmarker}{}%
\end{pgfscope}%
\begin{pgfscope}%
\pgfsys@transformshift{1.248929in}{0.992874in}%
\pgfsys@useobject{currentmarker}{}%
\end{pgfscope}%
\begin{pgfscope}%
\pgfsys@transformshift{1.249212in}{1.001898in}%
\pgfsys@useobject{currentmarker}{}%
\end{pgfscope}%
\begin{pgfscope}%
\pgfsys@transformshift{1.249679in}{1.006842in}%
\pgfsys@useobject{currentmarker}{}%
\end{pgfscope}%
\begin{pgfscope}%
\pgfsys@transformshift{1.249825in}{1.009569in}%
\pgfsys@useobject{currentmarker}{}%
\end{pgfscope}%
\begin{pgfscope}%
\pgfsys@transformshift{1.249916in}{1.011069in}%
\pgfsys@useobject{currentmarker}{}%
\end{pgfscope}%
\begin{pgfscope}%
\pgfsys@transformshift{1.249867in}{1.011893in}%
\pgfsys@useobject{currentmarker}{}%
\end{pgfscope}%
\begin{pgfscope}%
\pgfsys@transformshift{1.249878in}{1.012348in}%
\pgfsys@useobject{currentmarker}{}%
\end{pgfscope}%
\begin{pgfscope}%
\pgfsys@transformshift{1.249876in}{1.012598in}%
\pgfsys@useobject{currentmarker}{}%
\end{pgfscope}%
\begin{pgfscope}%
\pgfsys@transformshift{1.249872in}{1.012735in}%
\pgfsys@useobject{currentmarker}{}%
\end{pgfscope}%
\begin{pgfscope}%
\pgfsys@transformshift{1.249873in}{1.012811in}%
\pgfsys@useobject{currentmarker}{}%
\end{pgfscope}%
\begin{pgfscope}%
\pgfsys@transformshift{1.249871in}{1.012852in}%
\pgfsys@useobject{currentmarker}{}%
\end{pgfscope}%
\begin{pgfscope}%
\pgfsys@transformshift{1.249871in}{1.012875in}%
\pgfsys@useobject{currentmarker}{}%
\end{pgfscope}%
\begin{pgfscope}%
\pgfsys@transformshift{1.249871in}{1.012888in}%
\pgfsys@useobject{currentmarker}{}%
\end{pgfscope}%
\begin{pgfscope}%
\pgfsys@transformshift{1.249871in}{1.012894in}%
\pgfsys@useobject{currentmarker}{}%
\end{pgfscope}%
\begin{pgfscope}%
\pgfsys@transformshift{1.249871in}{1.012898in}%
\pgfsys@useobject{currentmarker}{}%
\end{pgfscope}%
\begin{pgfscope}%
\pgfsys@transformshift{1.249871in}{1.012900in}%
\pgfsys@useobject{currentmarker}{}%
\end{pgfscope}%
\begin{pgfscope}%
\pgfsys@transformshift{1.249871in}{1.012901in}%
\pgfsys@useobject{currentmarker}{}%
\end{pgfscope}%
\begin{pgfscope}%
\pgfsys@transformshift{1.249871in}{1.012902in}%
\pgfsys@useobject{currentmarker}{}%
\end{pgfscope}%
\begin{pgfscope}%
\pgfsys@transformshift{1.249871in}{1.012902in}%
\pgfsys@useobject{currentmarker}{}%
\end{pgfscope}%
\begin{pgfscope}%
\pgfsys@transformshift{1.249871in}{1.012903in}%
\pgfsys@useobject{currentmarker}{}%
\end{pgfscope}%
\begin{pgfscope}%
\pgfsys@transformshift{1.249871in}{1.012903in}%
\pgfsys@useobject{currentmarker}{}%
\end{pgfscope}%
\begin{pgfscope}%
\pgfsys@transformshift{1.249871in}{1.012903in}%
\pgfsys@useobject{currentmarker}{}%
\end{pgfscope}%
\begin{pgfscope}%
\pgfsys@transformshift{1.249871in}{1.012903in}%
\pgfsys@useobject{currentmarker}{}%
\end{pgfscope}%
\begin{pgfscope}%
\pgfsys@transformshift{1.249871in}{1.012903in}%
\pgfsys@useobject{currentmarker}{}%
\end{pgfscope}%
\begin{pgfscope}%
\pgfsys@transformshift{1.249871in}{1.012903in}%
\pgfsys@useobject{currentmarker}{}%
\end{pgfscope}%
\begin{pgfscope}%
\pgfsys@transformshift{1.249871in}{1.012903in}%
\pgfsys@useobject{currentmarker}{}%
\end{pgfscope}%
\begin{pgfscope}%
\pgfsys@transformshift{1.249871in}{1.012903in}%
\pgfsys@useobject{currentmarker}{}%
\end{pgfscope}%
\begin{pgfscope}%
\pgfsys@transformshift{1.249871in}{1.012903in}%
\pgfsys@useobject{currentmarker}{}%
\end{pgfscope}%
\begin{pgfscope}%
\pgfsys@transformshift{1.249871in}{1.012903in}%
\pgfsys@useobject{currentmarker}{}%
\end{pgfscope}%
\begin{pgfscope}%
\pgfsys@transformshift{1.249871in}{1.012903in}%
\pgfsys@useobject{currentmarker}{}%
\end{pgfscope}%
\begin{pgfscope}%
\pgfsys@transformshift{1.249871in}{1.012903in}%
\pgfsys@useobject{currentmarker}{}%
\end{pgfscope}%
\begin{pgfscope}%
\pgfsys@transformshift{1.249871in}{1.012903in}%
\pgfsys@useobject{currentmarker}{}%
\end{pgfscope}%
\begin{pgfscope}%
\pgfsys@transformshift{1.249871in}{1.012903in}%
\pgfsys@useobject{currentmarker}{}%
\end{pgfscope}%
\begin{pgfscope}%
\pgfsys@transformshift{1.249871in}{1.012903in}%
\pgfsys@useobject{currentmarker}{}%
\end{pgfscope}%
\begin{pgfscope}%
\pgfsys@transformshift{1.249871in}{1.012903in}%
\pgfsys@useobject{currentmarker}{}%
\end{pgfscope}%
\begin{pgfscope}%
\pgfsys@transformshift{1.249871in}{1.012903in}%
\pgfsys@useobject{currentmarker}{}%
\end{pgfscope}%
\begin{pgfscope}%
\pgfsys@transformshift{1.249871in}{1.012903in}%
\pgfsys@useobject{currentmarker}{}%
\end{pgfscope}%
\begin{pgfscope}%
\pgfsys@transformshift{1.249871in}{1.012903in}%
\pgfsys@useobject{currentmarker}{}%
\end{pgfscope}%
\begin{pgfscope}%
\pgfsys@transformshift{1.249871in}{1.012903in}%
\pgfsys@useobject{currentmarker}{}%
\end{pgfscope}%
\begin{pgfscope}%
\pgfsys@transformshift{1.249871in}{1.012903in}%
\pgfsys@useobject{currentmarker}{}%
\end{pgfscope}%
\begin{pgfscope}%
\pgfsys@transformshift{1.249871in}{1.012903in}%
\pgfsys@useobject{currentmarker}{}%
\end{pgfscope}%
\begin{pgfscope}%
\pgfsys@transformshift{1.249871in}{1.012903in}%
\pgfsys@useobject{currentmarker}{}%
\end{pgfscope}%
\begin{pgfscope}%
\pgfsys@transformshift{1.249871in}{1.012903in}%
\pgfsys@useobject{currentmarker}{}%
\end{pgfscope}%
\begin{pgfscope}%
\pgfsys@transformshift{1.249871in}{1.012903in}%
\pgfsys@useobject{currentmarker}{}%
\end{pgfscope}%
\begin{pgfscope}%
\pgfsys@transformshift{1.249871in}{1.012903in}%
\pgfsys@useobject{currentmarker}{}%
\end{pgfscope}%
\begin{pgfscope}%
\pgfsys@transformshift{1.249871in}{1.012903in}%
\pgfsys@useobject{currentmarker}{}%
\end{pgfscope}%
\begin{pgfscope}%
\pgfsys@transformshift{1.249871in}{1.012903in}%
\pgfsys@useobject{currentmarker}{}%
\end{pgfscope}%
\begin{pgfscope}%
\pgfsys@transformshift{1.249871in}{1.012903in}%
\pgfsys@useobject{currentmarker}{}%
\end{pgfscope}%
\begin{pgfscope}%
\pgfsys@transformshift{1.249871in}{1.012903in}%
\pgfsys@useobject{currentmarker}{}%
\end{pgfscope}%
\begin{pgfscope}%
\pgfsys@transformshift{1.249871in}{1.012903in}%
\pgfsys@useobject{currentmarker}{}%
\end{pgfscope}%
\begin{pgfscope}%
\pgfsys@transformshift{1.249871in}{1.012903in}%
\pgfsys@useobject{currentmarker}{}%
\end{pgfscope}%
\begin{pgfscope}%
\pgfsys@transformshift{1.249871in}{1.012903in}%
\pgfsys@useobject{currentmarker}{}%
\end{pgfscope}%
\begin{pgfscope}%
\pgfsys@transformshift{1.249871in}{1.012903in}%
\pgfsys@useobject{currentmarker}{}%
\end{pgfscope}%
\begin{pgfscope}%
\pgfsys@transformshift{1.249871in}{1.012903in}%
\pgfsys@useobject{currentmarker}{}%
\end{pgfscope}%
\begin{pgfscope}%
\pgfsys@transformshift{1.249871in}{1.012903in}%
\pgfsys@useobject{currentmarker}{}%
\end{pgfscope}%
\begin{pgfscope}%
\pgfsys@transformshift{1.249871in}{1.012903in}%
\pgfsys@useobject{currentmarker}{}%
\end{pgfscope}%
\begin{pgfscope}%
\pgfsys@transformshift{1.249871in}{1.012903in}%
\pgfsys@useobject{currentmarker}{}%
\end{pgfscope}%
\begin{pgfscope}%
\pgfsys@transformshift{1.249871in}{1.012903in}%
\pgfsys@useobject{currentmarker}{}%
\end{pgfscope}%
\begin{pgfscope}%
\pgfsys@transformshift{1.249871in}{1.012903in}%
\pgfsys@useobject{currentmarker}{}%
\end{pgfscope}%
\begin{pgfscope}%
\pgfsys@transformshift{1.249871in}{1.012903in}%
\pgfsys@useobject{currentmarker}{}%
\end{pgfscope}%
\begin{pgfscope}%
\pgfsys@transformshift{1.249871in}{1.012903in}%
\pgfsys@useobject{currentmarker}{}%
\end{pgfscope}%
\begin{pgfscope}%
\pgfsys@transformshift{1.249871in}{1.012903in}%
\pgfsys@useobject{currentmarker}{}%
\end{pgfscope}%
\begin{pgfscope}%
\pgfsys@transformshift{1.249871in}{1.012903in}%
\pgfsys@useobject{currentmarker}{}%
\end{pgfscope}%
\begin{pgfscope}%
\pgfsys@transformshift{1.249871in}{1.012903in}%
\pgfsys@useobject{currentmarker}{}%
\end{pgfscope}%
\begin{pgfscope}%
\pgfsys@transformshift{1.249871in}{1.012903in}%
\pgfsys@useobject{currentmarker}{}%
\end{pgfscope}%
\begin{pgfscope}%
\pgfsys@transformshift{1.249871in}{1.012903in}%
\pgfsys@useobject{currentmarker}{}%
\end{pgfscope}%
\begin{pgfscope}%
\pgfsys@transformshift{1.249871in}{1.012903in}%
\pgfsys@useobject{currentmarker}{}%
\end{pgfscope}%
\begin{pgfscope}%
\pgfsys@transformshift{1.249871in}{1.012903in}%
\pgfsys@useobject{currentmarker}{}%
\end{pgfscope}%
\begin{pgfscope}%
\pgfsys@transformshift{1.249871in}{1.012903in}%
\pgfsys@useobject{currentmarker}{}%
\end{pgfscope}%
\begin{pgfscope}%
\pgfsys@transformshift{1.249871in}{1.012903in}%
\pgfsys@useobject{currentmarker}{}%
\end{pgfscope}%
\begin{pgfscope}%
\pgfsys@transformshift{1.249871in}{1.012903in}%
\pgfsys@useobject{currentmarker}{}%
\end{pgfscope}%
\begin{pgfscope}%
\pgfsys@transformshift{1.249871in}{1.012903in}%
\pgfsys@useobject{currentmarker}{}%
\end{pgfscope}%
\begin{pgfscope}%
\pgfsys@transformshift{1.249871in}{1.012903in}%
\pgfsys@useobject{currentmarker}{}%
\end{pgfscope}%
\begin{pgfscope}%
\pgfsys@transformshift{1.249871in}{1.012903in}%
\pgfsys@useobject{currentmarker}{}%
\end{pgfscope}%
\begin{pgfscope}%
\pgfsys@transformshift{1.249871in}{1.012903in}%
\pgfsys@useobject{currentmarker}{}%
\end{pgfscope}%
\begin{pgfscope}%
\pgfsys@transformshift{1.249871in}{1.012903in}%
\pgfsys@useobject{currentmarker}{}%
\end{pgfscope}%
\begin{pgfscope}%
\pgfsys@transformshift{1.249871in}{1.012903in}%
\pgfsys@useobject{currentmarker}{}%
\end{pgfscope}%
\begin{pgfscope}%
\pgfsys@transformshift{1.249871in}{1.012903in}%
\pgfsys@useobject{currentmarker}{}%
\end{pgfscope}%
\begin{pgfscope}%
\pgfsys@transformshift{1.249871in}{1.012903in}%
\pgfsys@useobject{currentmarker}{}%
\end{pgfscope}%
\begin{pgfscope}%
\pgfsys@transformshift{1.249871in}{1.012903in}%
\pgfsys@useobject{currentmarker}{}%
\end{pgfscope}%
\begin{pgfscope}%
\pgfsys@transformshift{1.249871in}{1.012903in}%
\pgfsys@useobject{currentmarker}{}%
\end{pgfscope}%
\begin{pgfscope}%
\pgfsys@transformshift{1.249871in}{1.012903in}%
\pgfsys@useobject{currentmarker}{}%
\end{pgfscope}%
\begin{pgfscope}%
\pgfsys@transformshift{1.249871in}{1.012903in}%
\pgfsys@useobject{currentmarker}{}%
\end{pgfscope}%
\begin{pgfscope}%
\pgfsys@transformshift{1.249871in}{1.012903in}%
\pgfsys@useobject{currentmarker}{}%
\end{pgfscope}%
\begin{pgfscope}%
\pgfsys@transformshift{1.249871in}{1.012903in}%
\pgfsys@useobject{currentmarker}{}%
\end{pgfscope}%
\begin{pgfscope}%
\pgfsys@transformshift{1.249871in}{1.012903in}%
\pgfsys@useobject{currentmarker}{}%
\end{pgfscope}%
\begin{pgfscope}%
\pgfsys@transformshift{1.249871in}{1.012903in}%
\pgfsys@useobject{currentmarker}{}%
\end{pgfscope}%
\begin{pgfscope}%
\pgfsys@transformshift{1.249871in}{1.012903in}%
\pgfsys@useobject{currentmarker}{}%
\end{pgfscope}%
\begin{pgfscope}%
\pgfsys@transformshift{1.249871in}{1.012903in}%
\pgfsys@useobject{currentmarker}{}%
\end{pgfscope}%
\begin{pgfscope}%
\pgfsys@transformshift{1.249871in}{1.012903in}%
\pgfsys@useobject{currentmarker}{}%
\end{pgfscope}%
\begin{pgfscope}%
\pgfsys@transformshift{1.249871in}{1.012903in}%
\pgfsys@useobject{currentmarker}{}%
\end{pgfscope}%
\begin{pgfscope}%
\pgfsys@transformshift{1.249871in}{1.012903in}%
\pgfsys@useobject{currentmarker}{}%
\end{pgfscope}%
\begin{pgfscope}%
\pgfsys@transformshift{1.249871in}{1.012903in}%
\pgfsys@useobject{currentmarker}{}%
\end{pgfscope}%
\begin{pgfscope}%
\pgfsys@transformshift{1.249871in}{1.012903in}%
\pgfsys@useobject{currentmarker}{}%
\end{pgfscope}%
\begin{pgfscope}%
\pgfsys@transformshift{1.249871in}{1.012903in}%
\pgfsys@useobject{currentmarker}{}%
\end{pgfscope}%
\begin{pgfscope}%
\pgfsys@transformshift{1.249871in}{1.012903in}%
\pgfsys@useobject{currentmarker}{}%
\end{pgfscope}%
\begin{pgfscope}%
\pgfsys@transformshift{1.249871in}{1.012903in}%
\pgfsys@useobject{currentmarker}{}%
\end{pgfscope}%
\begin{pgfscope}%
\pgfsys@transformshift{1.249871in}{1.012903in}%
\pgfsys@useobject{currentmarker}{}%
\end{pgfscope}%
\begin{pgfscope}%
\pgfsys@transformshift{1.249871in}{1.012903in}%
\pgfsys@useobject{currentmarker}{}%
\end{pgfscope}%
\begin{pgfscope}%
\pgfsys@transformshift{1.249871in}{1.012903in}%
\pgfsys@useobject{currentmarker}{}%
\end{pgfscope}%
\begin{pgfscope}%
\pgfsys@transformshift{1.249871in}{1.012903in}%
\pgfsys@useobject{currentmarker}{}%
\end{pgfscope}%
\begin{pgfscope}%
\pgfsys@transformshift{1.249871in}{1.012903in}%
\pgfsys@useobject{currentmarker}{}%
\end{pgfscope}%
\begin{pgfscope}%
\pgfsys@transformshift{1.249871in}{1.012903in}%
\pgfsys@useobject{currentmarker}{}%
\end{pgfscope}%
\begin{pgfscope}%
\pgfsys@transformshift{1.249871in}{1.012903in}%
\pgfsys@useobject{currentmarker}{}%
\end{pgfscope}%
\begin{pgfscope}%
\pgfsys@transformshift{1.249871in}{1.012903in}%
\pgfsys@useobject{currentmarker}{}%
\end{pgfscope}%
\begin{pgfscope}%
\pgfsys@transformshift{1.249871in}{1.012903in}%
\pgfsys@useobject{currentmarker}{}%
\end{pgfscope}%
\begin{pgfscope}%
\pgfsys@transformshift{1.249871in}{1.012903in}%
\pgfsys@useobject{currentmarker}{}%
\end{pgfscope}%
\begin{pgfscope}%
\pgfsys@transformshift{1.249871in}{1.012903in}%
\pgfsys@useobject{currentmarker}{}%
\end{pgfscope}%
\begin{pgfscope}%
\pgfsys@transformshift{1.249871in}{1.012903in}%
\pgfsys@useobject{currentmarker}{}%
\end{pgfscope}%
\begin{pgfscope}%
\pgfsys@transformshift{1.249871in}{1.012903in}%
\pgfsys@useobject{currentmarker}{}%
\end{pgfscope}%
\begin{pgfscope}%
\pgfsys@transformshift{1.249871in}{1.012903in}%
\pgfsys@useobject{currentmarker}{}%
\end{pgfscope}%
\begin{pgfscope}%
\pgfsys@transformshift{1.249871in}{1.012903in}%
\pgfsys@useobject{currentmarker}{}%
\end{pgfscope}%
\begin{pgfscope}%
\pgfsys@transformshift{1.249871in}{1.012903in}%
\pgfsys@useobject{currentmarker}{}%
\end{pgfscope}%
\begin{pgfscope}%
\pgfsys@transformshift{1.249871in}{1.012903in}%
\pgfsys@useobject{currentmarker}{}%
\end{pgfscope}%
\begin{pgfscope}%
\pgfsys@transformshift{1.249871in}{1.012903in}%
\pgfsys@useobject{currentmarker}{}%
\end{pgfscope}%
\begin{pgfscope}%
\pgfsys@transformshift{1.249871in}{1.012903in}%
\pgfsys@useobject{currentmarker}{}%
\end{pgfscope}%
\begin{pgfscope}%
\pgfsys@transformshift{1.249871in}{1.012903in}%
\pgfsys@useobject{currentmarker}{}%
\end{pgfscope}%
\begin{pgfscope}%
\pgfsys@transformshift{1.249871in}{1.012903in}%
\pgfsys@useobject{currentmarker}{}%
\end{pgfscope}%
\begin{pgfscope}%
\pgfsys@transformshift{1.249871in}{1.012903in}%
\pgfsys@useobject{currentmarker}{}%
\end{pgfscope}%
\begin{pgfscope}%
\pgfsys@transformshift{1.249871in}{1.012903in}%
\pgfsys@useobject{currentmarker}{}%
\end{pgfscope}%
\begin{pgfscope}%
\pgfsys@transformshift{1.249645in}{1.013746in}%
\pgfsys@useobject{currentmarker}{}%
\end{pgfscope}%
\begin{pgfscope}%
\pgfsys@transformshift{1.250246in}{1.015899in}%
\pgfsys@useobject{currentmarker}{}%
\end{pgfscope}%
\begin{pgfscope}%
\pgfsys@transformshift{1.249659in}{1.018680in}%
\pgfsys@useobject{currentmarker}{}%
\end{pgfscope}%
\begin{pgfscope}%
\pgfsys@transformshift{1.250076in}{1.020187in}%
\pgfsys@useobject{currentmarker}{}%
\end{pgfscope}%
\begin{pgfscope}%
\pgfsys@transformshift{1.249449in}{1.022512in}%
\pgfsys@useobject{currentmarker}{}%
\end{pgfscope}%
\begin{pgfscope}%
\pgfsys@transformshift{1.250118in}{1.025680in}%
\pgfsys@useobject{currentmarker}{}%
\end{pgfscope}%
\begin{pgfscope}%
\pgfsys@transformshift{1.249254in}{1.029425in}%
\pgfsys@useobject{currentmarker}{}%
\end{pgfscope}%
\begin{pgfscope}%
\pgfsys@transformshift{1.249781in}{1.031472in}%
\pgfsys@useobject{currentmarker}{}%
\end{pgfscope}%
\begin{pgfscope}%
\pgfsys@transformshift{1.248786in}{1.035146in}%
\pgfsys@useobject{currentmarker}{}%
\end{pgfscope}%
\begin{pgfscope}%
\pgfsys@transformshift{1.249097in}{1.039428in}%
\pgfsys@useobject{currentmarker}{}%
\end{pgfscope}%
\begin{pgfscope}%
\pgfsys@transformshift{1.248737in}{1.041761in}%
\pgfsys@useobject{currentmarker}{}%
\end{pgfscope}%
\begin{pgfscope}%
\pgfsys@transformshift{1.248289in}{1.042981in}%
\pgfsys@useobject{currentmarker}{}%
\end{pgfscope}%
\begin{pgfscope}%
\pgfsys@transformshift{1.248585in}{1.045637in}%
\pgfsys@useobject{currentmarker}{}%
\end{pgfscope}%
\begin{pgfscope}%
\pgfsys@transformshift{1.247268in}{1.049565in}%
\pgfsys@useobject{currentmarker}{}%
\end{pgfscope}%
\begin{pgfscope}%
\pgfsys@transformshift{1.247292in}{1.054724in}%
\pgfsys@useobject{currentmarker}{}%
\end{pgfscope}%
\begin{pgfscope}%
\pgfsys@transformshift{1.247007in}{1.057546in}%
\pgfsys@useobject{currentmarker}{}%
\end{pgfscope}%
\begin{pgfscope}%
\pgfsys@transformshift{1.245919in}{1.060892in}%
\pgfsys@useobject{currentmarker}{}%
\end{pgfscope}%
\begin{pgfscope}%
\pgfsys@transformshift{1.246941in}{1.065859in}%
\pgfsys@useobject{currentmarker}{}%
\end{pgfscope}%
\begin{pgfscope}%
\pgfsys@transformshift{1.246257in}{1.068563in}%
\pgfsys@useobject{currentmarker}{}%
\end{pgfscope}%
\begin{pgfscope}%
\pgfsys@transformshift{1.246438in}{1.072001in}%
\pgfsys@useobject{currentmarker}{}%
\end{pgfscope}%
\begin{pgfscope}%
\pgfsys@transformshift{1.247092in}{1.075901in}%
\pgfsys@useobject{currentmarker}{}%
\end{pgfscope}%
\begin{pgfscope}%
\pgfsys@transformshift{1.247256in}{1.080404in}%
\pgfsys@useobject{currentmarker}{}%
\end{pgfscope}%
\begin{pgfscope}%
\pgfsys@transformshift{1.247367in}{1.082881in}%
\pgfsys@useobject{currentmarker}{}%
\end{pgfscope}%
\begin{pgfscope}%
\pgfsys@transformshift{1.247513in}{1.084236in}%
\pgfsys@useobject{currentmarker}{}%
\end{pgfscope}%
\begin{pgfscope}%
\pgfsys@transformshift{1.247289in}{1.087432in}%
\pgfsys@useobject{currentmarker}{}%
\end{pgfscope}%
\begin{pgfscope}%
\pgfsys@transformshift{1.247645in}{1.091104in}%
\pgfsys@useobject{currentmarker}{}%
\end{pgfscope}%
\begin{pgfscope}%
\pgfsys@transformshift{1.248353in}{1.095486in}%
\pgfsys@useobject{currentmarker}{}%
\end{pgfscope}%
\begin{pgfscope}%
\pgfsys@transformshift{1.248008in}{1.101414in}%
\pgfsys@useobject{currentmarker}{}%
\end{pgfscope}%
\begin{pgfscope}%
\pgfsys@transformshift{1.248168in}{1.104676in}%
\pgfsys@useobject{currentmarker}{}%
\end{pgfscope}%
\begin{pgfscope}%
\pgfsys@transformshift{1.248810in}{1.108381in}%
\pgfsys@useobject{currentmarker}{}%
\end{pgfscope}%
\begin{pgfscope}%
\pgfsys@transformshift{1.248284in}{1.113545in}%
\pgfsys@useobject{currentmarker}{}%
\end{pgfscope}%
\begin{pgfscope}%
\pgfsys@transformshift{1.248372in}{1.116400in}%
\pgfsys@useobject{currentmarker}{}%
\end{pgfscope}%
\begin{pgfscope}%
\pgfsys@transformshift{1.248633in}{1.117948in}%
\pgfsys@useobject{currentmarker}{}%
\end{pgfscope}%
\begin{pgfscope}%
\pgfsys@transformshift{1.248371in}{1.120810in}%
\pgfsys@useobject{currentmarker}{}%
\end{pgfscope}%
\begin{pgfscope}%
\pgfsys@transformshift{1.248242in}{1.122386in}%
\pgfsys@useobject{currentmarker}{}%
\end{pgfscope}%
\begin{pgfscope}%
\pgfsys@transformshift{1.249184in}{1.126152in}%
\pgfsys@useobject{currentmarker}{}%
\end{pgfscope}%
\begin{pgfscope}%
\pgfsys@transformshift{1.248934in}{1.128272in}%
\pgfsys@useobject{currentmarker}{}%
\end{pgfscope}%
\begin{pgfscope}%
\pgfsys@transformshift{1.248804in}{1.129439in}%
\pgfsys@useobject{currentmarker}{}%
\end{pgfscope}%
\begin{pgfscope}%
\pgfsys@transformshift{1.249524in}{1.132500in}%
\pgfsys@useobject{currentmarker}{}%
\end{pgfscope}%
\begin{pgfscope}%
\pgfsys@transformshift{1.249345in}{1.134220in}%
\pgfsys@useobject{currentmarker}{}%
\end{pgfscope}%
\begin{pgfscope}%
\pgfsys@transformshift{1.249338in}{1.135172in}%
\pgfsys@useobject{currentmarker}{}%
\end{pgfscope}%
\begin{pgfscope}%
\pgfsys@transformshift{1.249675in}{1.137092in}%
\pgfsys@useobject{currentmarker}{}%
\end{pgfscope}%
\begin{pgfscope}%
\pgfsys@transformshift{1.249607in}{1.139583in}%
\pgfsys@useobject{currentmarker}{}%
\end{pgfscope}%
\begin{pgfscope}%
\pgfsys@transformshift{1.249443in}{1.140944in}%
\pgfsys@useobject{currentmarker}{}%
\end{pgfscope}%
\begin{pgfscope}%
\pgfsys@transformshift{1.250084in}{1.143134in}%
\pgfsys@useobject{currentmarker}{}%
\end{pgfscope}%
\begin{pgfscope}%
\pgfsys@transformshift{1.249384in}{1.146889in}%
\pgfsys@useobject{currentmarker}{}%
\end{pgfscope}%
\begin{pgfscope}%
\pgfsys@transformshift{1.249489in}{1.148988in}%
\pgfsys@useobject{currentmarker}{}%
\end{pgfscope}%
\begin{pgfscope}%
\pgfsys@transformshift{1.249416in}{1.154235in}%
\pgfsys@useobject{currentmarker}{}%
\end{pgfscope}%
\begin{pgfscope}%
\pgfsys@transformshift{1.246870in}{1.161830in}%
\pgfsys@useobject{currentmarker}{}%
\end{pgfscope}%
\begin{pgfscope}%
\pgfsys@transformshift{1.247274in}{1.166218in}%
\pgfsys@useobject{currentmarker}{}%
\end{pgfscope}%
\begin{pgfscope}%
\pgfsys@transformshift{1.247537in}{1.171323in}%
\pgfsys@useobject{currentmarker}{}%
\end{pgfscope}%
\begin{pgfscope}%
\pgfsys@transformshift{1.246722in}{1.174013in}%
\pgfsys@useobject{currentmarker}{}%
\end{pgfscope}%
\begin{pgfscope}%
\pgfsys@transformshift{1.247584in}{1.178624in}%
\pgfsys@useobject{currentmarker}{}%
\end{pgfscope}%
\begin{pgfscope}%
\pgfsys@transformshift{1.246734in}{1.183991in}%
\pgfsys@useobject{currentmarker}{}%
\end{pgfscope}%
\begin{pgfscope}%
\pgfsys@transformshift{1.246480in}{1.186968in}%
\pgfsys@useobject{currentmarker}{}%
\end{pgfscope}%
\begin{pgfscope}%
\pgfsys@transformshift{1.246764in}{1.188587in}%
\pgfsys@useobject{currentmarker}{}%
\end{pgfscope}%
\begin{pgfscope}%
\pgfsys@transformshift{1.245786in}{1.191610in}%
\pgfsys@useobject{currentmarker}{}%
\end{pgfscope}%
\begin{pgfscope}%
\pgfsys@transformshift{1.245887in}{1.193355in}%
\pgfsys@useobject{currentmarker}{}%
\end{pgfscope}%
\begin{pgfscope}%
\pgfsys@transformshift{1.245504in}{1.196800in}%
\pgfsys@useobject{currentmarker}{}%
\end{pgfscope}%
\begin{pgfscope}%
\pgfsys@transformshift{1.244319in}{1.202993in}%
\pgfsys@useobject{currentmarker}{}%
\end{pgfscope}%
\begin{pgfscope}%
\pgfsys@transformshift{1.245258in}{1.209802in}%
\pgfsys@useobject{currentmarker}{}%
\end{pgfscope}%
\begin{pgfscope}%
\pgfsys@transformshift{1.249036in}{1.216434in}%
\pgfsys@useobject{currentmarker}{}%
\end{pgfscope}%
\begin{pgfscope}%
\pgfsys@transformshift{1.246576in}{1.225992in}%
\pgfsys@useobject{currentmarker}{}%
\end{pgfscope}%
\begin{pgfscope}%
\pgfsys@transformshift{1.245785in}{1.231362in}%
\pgfsys@useobject{currentmarker}{}%
\end{pgfscope}%
\begin{pgfscope}%
\pgfsys@transformshift{1.246798in}{1.234171in}%
\pgfsys@useobject{currentmarker}{}%
\end{pgfscope}%
\begin{pgfscope}%
\pgfsys@transformshift{1.246564in}{1.235796in}%
\pgfsys@useobject{currentmarker}{}%
\end{pgfscope}%
\begin{pgfscope}%
\pgfsys@transformshift{1.247483in}{1.239459in}%
\pgfsys@useobject{currentmarker}{}%
\end{pgfscope}%
\begin{pgfscope}%
\pgfsys@transformshift{1.247127in}{1.241505in}%
\pgfsys@useobject{currentmarker}{}%
\end{pgfscope}%
\begin{pgfscope}%
\pgfsys@transformshift{1.246979in}{1.242638in}%
\pgfsys@useobject{currentmarker}{}%
\end{pgfscope}%
\begin{pgfscope}%
\pgfsys@transformshift{1.247284in}{1.245487in}%
\pgfsys@useobject{currentmarker}{}%
\end{pgfscope}%
\begin{pgfscope}%
\pgfsys@transformshift{1.246219in}{1.248968in}%
\pgfsys@useobject{currentmarker}{}%
\end{pgfscope}%
\begin{pgfscope}%
\pgfsys@transformshift{1.246394in}{1.250962in}%
\pgfsys@useobject{currentmarker}{}%
\end{pgfscope}%
\begin{pgfscope}%
\pgfsys@transformshift{1.246592in}{1.252045in}%
\pgfsys@useobject{currentmarker}{}%
\end{pgfscope}%
\begin{pgfscope}%
\pgfsys@transformshift{1.246381in}{1.253669in}%
\pgfsys@useobject{currentmarker}{}%
\end{pgfscope}%
\begin{pgfscope}%
\pgfsys@transformshift{1.246170in}{1.255870in}%
\pgfsys@useobject{currentmarker}{}%
\end{pgfscope}%
\begin{pgfscope}%
\pgfsys@transformshift{1.246465in}{1.260102in}%
\pgfsys@useobject{currentmarker}{}%
\end{pgfscope}%
\begin{pgfscope}%
\pgfsys@transformshift{1.244667in}{1.265908in}%
\pgfsys@useobject{currentmarker}{}%
\end{pgfscope}%
\begin{pgfscope}%
\pgfsys@transformshift{1.244807in}{1.272360in}%
\pgfsys@useobject{currentmarker}{}%
\end{pgfscope}%
\begin{pgfscope}%
\pgfsys@transformshift{1.244236in}{1.282175in}%
\pgfsys@useobject{currentmarker}{}%
\end{pgfscope}%
\begin{pgfscope}%
\pgfsys@transformshift{1.246645in}{1.292927in}%
\pgfsys@useobject{currentmarker}{}%
\end{pgfscope}%
\begin{pgfscope}%
\pgfsys@transformshift{1.250360in}{1.304155in}%
\pgfsys@useobject{currentmarker}{}%
\end{pgfscope}%
\begin{pgfscope}%
\pgfsys@transformshift{1.244800in}{1.317140in}%
\pgfsys@useobject{currentmarker}{}%
\end{pgfscope}%
\begin{pgfscope}%
\pgfsys@transformshift{1.252342in}{1.329683in}%
\pgfsys@useobject{currentmarker}{}%
\end{pgfscope}%
\begin{pgfscope}%
\pgfsys@transformshift{1.252271in}{1.345990in}%
\pgfsys@useobject{currentmarker}{}%
\end{pgfscope}%
\begin{pgfscope}%
\pgfsys@transformshift{1.254560in}{1.363904in}%
\pgfsys@useobject{currentmarker}{}%
\end{pgfscope}%
\begin{pgfscope}%
\pgfsys@transformshift{1.256548in}{1.382396in}%
\pgfsys@useobject{currentmarker}{}%
\end{pgfscope}%
\begin{pgfscope}%
\pgfsys@transformshift{1.254125in}{1.401460in}%
\pgfsys@useobject{currentmarker}{}%
\end{pgfscope}%
\begin{pgfscope}%
\pgfsys@transformshift{1.256249in}{1.411813in}%
\pgfsys@useobject{currentmarker}{}%
\end{pgfscope}%
\begin{pgfscope}%
\pgfsys@transformshift{1.255861in}{1.417614in}%
\pgfsys@useobject{currentmarker}{}%
\end{pgfscope}%
\begin{pgfscope}%
\pgfsys@transformshift{1.256280in}{1.420783in}%
\pgfsys@useobject{currentmarker}{}%
\end{pgfscope}%
\begin{pgfscope}%
\pgfsys@transformshift{1.256138in}{1.422536in}%
\pgfsys@useobject{currentmarker}{}%
\end{pgfscope}%
\begin{pgfscope}%
\pgfsys@transformshift{1.256284in}{1.423492in}%
\pgfsys@useobject{currentmarker}{}%
\end{pgfscope}%
\begin{pgfscope}%
\pgfsys@transformshift{1.256250in}{1.424023in}%
\pgfsys@useobject{currentmarker}{}%
\end{pgfscope}%
\begin{pgfscope}%
\pgfsys@transformshift{1.256492in}{1.425073in}%
\pgfsys@useobject{currentmarker}{}%
\end{pgfscope}%
\begin{pgfscope}%
\pgfsys@transformshift{1.256568in}{1.426973in}%
\pgfsys@useobject{currentmarker}{}%
\end{pgfscope}%
\begin{pgfscope}%
\pgfsys@transformshift{1.256861in}{1.427978in}%
\pgfsys@useobject{currentmarker}{}%
\end{pgfscope}%
\begin{pgfscope}%
\pgfsys@transformshift{1.257066in}{1.429691in}%
\pgfsys@useobject{currentmarker}{}%
\end{pgfscope}%
\begin{pgfscope}%
\pgfsys@transformshift{1.257716in}{1.430383in}%
\pgfsys@useobject{currentmarker}{}%
\end{pgfscope}%
\begin{pgfscope}%
\pgfsys@transformshift{1.258552in}{1.431511in}%
\pgfsys@useobject{currentmarker}{}%
\end{pgfscope}%
\begin{pgfscope}%
\pgfsys@transformshift{1.261098in}{1.431996in}%
\pgfsys@useobject{currentmarker}{}%
\end{pgfscope}%
\begin{pgfscope}%
\pgfsys@transformshift{1.262478in}{1.432354in}%
\pgfsys@useobject{currentmarker}{}%
\end{pgfscope}%
\begin{pgfscope}%
\pgfsys@transformshift{1.263260in}{1.432288in}%
\pgfsys@useobject{currentmarker}{}%
\end{pgfscope}%
\begin{pgfscope}%
\pgfsys@transformshift{1.263691in}{1.432301in}%
\pgfsys@useobject{currentmarker}{}%
\end{pgfscope}%
\begin{pgfscope}%
\pgfsys@transformshift{1.264688in}{1.432294in}%
\pgfsys@useobject{currentmarker}{}%
\end{pgfscope}%
\begin{pgfscope}%
\pgfsys@transformshift{1.265237in}{1.432307in}%
\pgfsys@useobject{currentmarker}{}%
\end{pgfscope}%
\begin{pgfscope}%
\pgfsys@transformshift{1.265537in}{1.432336in}%
\pgfsys@useobject{currentmarker}{}%
\end{pgfscope}%
\begin{pgfscope}%
\pgfsys@transformshift{1.266432in}{1.432352in}%
\pgfsys@useobject{currentmarker}{}%
\end{pgfscope}%
\begin{pgfscope}%
\pgfsys@transformshift{1.266924in}{1.432358in}%
\pgfsys@useobject{currentmarker}{}%
\end{pgfscope}%
\begin{pgfscope}%
\pgfsys@transformshift{1.267870in}{1.432527in}%
\pgfsys@useobject{currentmarker}{}%
\end{pgfscope}%
\begin{pgfscope}%
\pgfsys@transformshift{1.268397in}{1.432529in}%
\pgfsys@useobject{currentmarker}{}%
\end{pgfscope}%
\begin{pgfscope}%
\pgfsys@transformshift{1.270207in}{1.432851in}%
\pgfsys@useobject{currentmarker}{}%
\end{pgfscope}%
\begin{pgfscope}%
\pgfsys@transformshift{1.274837in}{1.433984in}%
\pgfsys@useobject{currentmarker}{}%
\end{pgfscope}%
\begin{pgfscope}%
\pgfsys@transformshift{1.277453in}{1.434167in}%
\pgfsys@useobject{currentmarker}{}%
\end{pgfscope}%
\begin{pgfscope}%
\pgfsys@transformshift{1.281996in}{1.435140in}%
\pgfsys@useobject{currentmarker}{}%
\end{pgfscope}%
\begin{pgfscope}%
\pgfsys@transformshift{1.284490in}{1.435699in}%
\pgfsys@useobject{currentmarker}{}%
\end{pgfscope}%
\begin{pgfscope}%
\pgfsys@transformshift{1.288538in}{1.436053in}%
\pgfsys@useobject{currentmarker}{}%
\end{pgfscope}%
\begin{pgfscope}%
\pgfsys@transformshift{1.293721in}{1.436449in}%
\pgfsys@useobject{currentmarker}{}%
\end{pgfscope}%
\begin{pgfscope}%
\pgfsys@transformshift{1.299699in}{1.436336in}%
\pgfsys@useobject{currentmarker}{}%
\end{pgfscope}%
\begin{pgfscope}%
\pgfsys@transformshift{1.308984in}{1.438265in}%
\pgfsys@useobject{currentmarker}{}%
\end{pgfscope}%
\begin{pgfscope}%
\pgfsys@transformshift{1.318975in}{1.439859in}%
\pgfsys@useobject{currentmarker}{}%
\end{pgfscope}%
\begin{pgfscope}%
\pgfsys@transformshift{1.331663in}{1.440828in}%
\pgfsys@useobject{currentmarker}{}%
\end{pgfscope}%
\begin{pgfscope}%
\pgfsys@transformshift{1.344857in}{1.441507in}%
\pgfsys@useobject{currentmarker}{}%
\end{pgfscope}%
\begin{pgfscope}%
\pgfsys@transformshift{1.359935in}{1.443429in}%
\pgfsys@useobject{currentmarker}{}%
\end{pgfscope}%
\begin{pgfscope}%
\pgfsys@transformshift{1.375712in}{1.442928in}%
\pgfsys@useobject{currentmarker}{}%
\end{pgfscope}%
\begin{pgfscope}%
\pgfsys@transformshift{1.392108in}{1.445873in}%
\pgfsys@useobject{currentmarker}{}%
\end{pgfscope}%
\begin{pgfscope}%
\pgfsys@transformshift{1.409449in}{1.447493in}%
\pgfsys@useobject{currentmarker}{}%
\end{pgfscope}%
\begin{pgfscope}%
\pgfsys@transformshift{1.428661in}{1.447489in}%
\pgfsys@useobject{currentmarker}{}%
\end{pgfscope}%
\begin{pgfscope}%
\pgfsys@transformshift{1.439191in}{1.448371in}%
\pgfsys@useobject{currentmarker}{}%
\end{pgfscope}%
\begin{pgfscope}%
\pgfsys@transformshift{1.450249in}{1.448208in}%
\pgfsys@useobject{currentmarker}{}%
\end{pgfscope}%
\begin{pgfscope}%
\pgfsys@transformshift{1.456332in}{1.448279in}%
\pgfsys@useobject{currentmarker}{}%
\end{pgfscope}%
\begin{pgfscope}%
\pgfsys@transformshift{1.465012in}{1.447865in}%
\pgfsys@useobject{currentmarker}{}%
\end{pgfscope}%
\begin{pgfscope}%
\pgfsys@transformshift{1.474286in}{1.447168in}%
\pgfsys@useobject{currentmarker}{}%
\end{pgfscope}%
\begin{pgfscope}%
\pgfsys@transformshift{1.486009in}{1.446200in}%
\pgfsys@useobject{currentmarker}{}%
\end{pgfscope}%
\begin{pgfscope}%
\pgfsys@transformshift{1.492467in}{1.446581in}%
\pgfsys@useobject{currentmarker}{}%
\end{pgfscope}%
\begin{pgfscope}%
\pgfsys@transformshift{1.499809in}{1.446422in}%
\pgfsys@useobject{currentmarker}{}%
\end{pgfscope}%
\begin{pgfscope}%
\pgfsys@transformshift{1.503846in}{1.446506in}%
\pgfsys@useobject{currentmarker}{}%
\end{pgfscope}%
\begin{pgfscope}%
\pgfsys@transformshift{1.508458in}{1.445957in}%
\pgfsys@useobject{currentmarker}{}%
\end{pgfscope}%
\begin{pgfscope}%
\pgfsys@transformshift{1.514456in}{1.446156in}%
\pgfsys@useobject{currentmarker}{}%
\end{pgfscope}%
\begin{pgfscope}%
\pgfsys@transformshift{1.521763in}{1.445686in}%
\pgfsys@useobject{currentmarker}{}%
\end{pgfscope}%
\begin{pgfscope}%
\pgfsys@transformshift{1.530587in}{1.446294in}%
\pgfsys@useobject{currentmarker}{}%
\end{pgfscope}%
\begin{pgfscope}%
\pgfsys@transformshift{1.542175in}{1.446550in}%
\pgfsys@useobject{currentmarker}{}%
\end{pgfscope}%
\begin{pgfscope}%
\pgfsys@transformshift{1.548549in}{1.446459in}%
\pgfsys@useobject{currentmarker}{}%
\end{pgfscope}%
\begin{pgfscope}%
\pgfsys@transformshift{1.555484in}{1.446142in}%
\pgfsys@useobject{currentmarker}{}%
\end{pgfscope}%
\begin{pgfscope}%
\pgfsys@transformshift{1.565978in}{1.446290in}%
\pgfsys@useobject{currentmarker}{}%
\end{pgfscope}%
\begin{pgfscope}%
\pgfsys@transformshift{1.577102in}{1.446554in}%
\pgfsys@useobject{currentmarker}{}%
\end{pgfscope}%
\begin{pgfscope}%
\pgfsys@transformshift{1.589333in}{1.445925in}%
\pgfsys@useobject{currentmarker}{}%
\end{pgfscope}%
\begin{pgfscope}%
\pgfsys@transformshift{1.602662in}{1.446241in}%
\pgfsys@useobject{currentmarker}{}%
\end{pgfscope}%
\begin{pgfscope}%
\pgfsys@transformshift{1.609966in}{1.446899in}%
\pgfsys@useobject{currentmarker}{}%
\end{pgfscope}%
\begin{pgfscope}%
\pgfsys@transformshift{1.618332in}{1.447023in}%
\pgfsys@useobject{currentmarker}{}%
\end{pgfscope}%
\begin{pgfscope}%
\pgfsys@transformshift{1.622919in}{1.447400in}%
\pgfsys@useobject{currentmarker}{}%
\end{pgfscope}%
\begin{pgfscope}%
\pgfsys@transformshift{1.628955in}{1.446914in}%
\pgfsys@useobject{currentmarker}{}%
\end{pgfscope}%
\begin{pgfscope}%
\pgfsys@transformshift{1.635885in}{1.447736in}%
\pgfsys@useobject{currentmarker}{}%
\end{pgfscope}%
\begin{pgfscope}%
\pgfsys@transformshift{1.644486in}{1.447164in}%
\pgfsys@useobject{currentmarker}{}%
\end{pgfscope}%
\begin{pgfscope}%
\pgfsys@transformshift{1.649201in}{1.447651in}%
\pgfsys@useobject{currentmarker}{}%
\end{pgfscope}%
\begin{pgfscope}%
\pgfsys@transformshift{1.654481in}{1.447270in}%
\pgfsys@useobject{currentmarker}{}%
\end{pgfscope}%
\begin{pgfscope}%
\pgfsys@transformshift{1.660568in}{1.448016in}%
\pgfsys@useobject{currentmarker}{}%
\end{pgfscope}%
\begin{pgfscope}%
\pgfsys@transformshift{1.668480in}{1.447885in}%
\pgfsys@useobject{currentmarker}{}%
\end{pgfscope}%
\begin{pgfscope}%
\pgfsys@transformshift{1.672815in}{1.448277in}%
\pgfsys@useobject{currentmarker}{}%
\end{pgfscope}%
\begin{pgfscope}%
\pgfsys@transformshift{1.675202in}{1.448448in}%
\pgfsys@useobject{currentmarker}{}%
\end{pgfscope}%
\begin{pgfscope}%
\pgfsys@transformshift{1.678407in}{1.448507in}%
\pgfsys@useobject{currentmarker}{}%
\end{pgfscope}%
\begin{pgfscope}%
\pgfsys@transformshift{1.680170in}{1.448486in}%
\pgfsys@useobject{currentmarker}{}%
\end{pgfscope}%
\begin{pgfscope}%
\pgfsys@transformshift{1.683312in}{1.448822in}%
\pgfsys@useobject{currentmarker}{}%
\end{pgfscope}%
\begin{pgfscope}%
\pgfsys@transformshift{1.687606in}{1.448709in}%
\pgfsys@useobject{currentmarker}{}%
\end{pgfscope}%
\begin{pgfscope}%
\pgfsys@transformshift{1.689957in}{1.448933in}%
\pgfsys@useobject{currentmarker}{}%
\end{pgfscope}%
\begin{pgfscope}%
\pgfsys@transformshift{1.694023in}{1.448694in}%
\pgfsys@useobject{currentmarker}{}%
\end{pgfscope}%
\begin{pgfscope}%
\pgfsys@transformshift{1.696253in}{1.448914in}%
\pgfsys@useobject{currentmarker}{}%
\end{pgfscope}%
\begin{pgfscope}%
\pgfsys@transformshift{1.697483in}{1.448837in}%
\pgfsys@useobject{currentmarker}{}%
\end{pgfscope}%
\begin{pgfscope}%
\pgfsys@transformshift{1.700077in}{1.449357in}%
\pgfsys@useobject{currentmarker}{}%
\end{pgfscope}%
\begin{pgfscope}%
\pgfsys@transformshift{1.703267in}{1.449212in}%
\pgfsys@useobject{currentmarker}{}%
\end{pgfscope}%
\begin{pgfscope}%
\pgfsys@transformshift{1.707354in}{1.449632in}%
\pgfsys@useobject{currentmarker}{}%
\end{pgfscope}%
\begin{pgfscope}%
\pgfsys@transformshift{1.709611in}{1.449532in}%
\pgfsys@useobject{currentmarker}{}%
\end{pgfscope}%
\begin{pgfscope}%
\pgfsys@transformshift{1.714115in}{1.450128in}%
\pgfsys@useobject{currentmarker}{}%
\end{pgfscope}%
\begin{pgfscope}%
\pgfsys@transformshift{1.716606in}{1.449938in}%
\pgfsys@useobject{currentmarker}{}%
\end{pgfscope}%
\begin{pgfscope}%
\pgfsys@transformshift{1.720608in}{1.450359in}%
\pgfsys@useobject{currentmarker}{}%
\end{pgfscope}%
\begin{pgfscope}%
\pgfsys@transformshift{1.722809in}{1.450130in}%
\pgfsys@useobject{currentmarker}{}%
\end{pgfscope}%
\begin{pgfscope}%
\pgfsys@transformshift{1.725783in}{1.450291in}%
\pgfsys@useobject{currentmarker}{}%
\end{pgfscope}%
\begin{pgfscope}%
\pgfsys@transformshift{1.727417in}{1.450167in}%
\pgfsys@useobject{currentmarker}{}%
\end{pgfscope}%
\begin{pgfscope}%
\pgfsys@transformshift{1.728317in}{1.450165in}%
\pgfsys@useobject{currentmarker}{}%
\end{pgfscope}%
\begin{pgfscope}%
\pgfsys@transformshift{1.728771in}{1.449966in}%
\pgfsys@useobject{currentmarker}{}%
\end{pgfscope}%
\begin{pgfscope}%
\pgfsys@transformshift{1.729692in}{1.449304in}%
\pgfsys@useobject{currentmarker}{}%
\end{pgfscope}%
\begin{pgfscope}%
\pgfsys@transformshift{1.730683in}{1.447593in}%
\pgfsys@useobject{currentmarker}{}%
\end{pgfscope}%
\begin{pgfscope}%
\pgfsys@transformshift{1.731578in}{1.444522in}%
\pgfsys@useobject{currentmarker}{}%
\end{pgfscope}%
\begin{pgfscope}%
\pgfsys@transformshift{1.731960in}{1.439794in}%
\pgfsys@useobject{currentmarker}{}%
\end{pgfscope}%
\begin{pgfscope}%
\pgfsys@transformshift{1.732980in}{1.433837in}%
\pgfsys@useobject{currentmarker}{}%
\end{pgfscope}%
\begin{pgfscope}%
\pgfsys@transformshift{1.732247in}{1.427014in}%
\pgfsys@useobject{currentmarker}{}%
\end{pgfscope}%
\begin{pgfscope}%
\pgfsys@transformshift{1.732681in}{1.419362in}%
\pgfsys@useobject{currentmarker}{}%
\end{pgfscope}%
\begin{pgfscope}%
\pgfsys@transformshift{1.731305in}{1.411151in}%
\pgfsys@useobject{currentmarker}{}%
\end{pgfscope}%
\begin{pgfscope}%
\pgfsys@transformshift{1.730994in}{1.402373in}%
\pgfsys@useobject{currentmarker}{}%
\end{pgfscope}%
\begin{pgfscope}%
\pgfsys@transformshift{1.729658in}{1.393230in}%
\pgfsys@useobject{currentmarker}{}%
\end{pgfscope}%
\begin{pgfscope}%
\pgfsys@transformshift{1.727680in}{1.383654in}%
\pgfsys@useobject{currentmarker}{}%
\end{pgfscope}%
\begin{pgfscope}%
\pgfsys@transformshift{1.727291in}{1.372631in}%
\pgfsys@useobject{currentmarker}{}%
\end{pgfscope}%
\begin{pgfscope}%
\pgfsys@transformshift{1.727161in}{1.360972in}%
\pgfsys@useobject{currentmarker}{}%
\end{pgfscope}%
\begin{pgfscope}%
\pgfsys@transformshift{1.723314in}{1.347831in}%
\pgfsys@useobject{currentmarker}{}%
\end{pgfscope}%
\begin{pgfscope}%
\pgfsys@transformshift{1.723586in}{1.340304in}%
\pgfsys@useobject{currentmarker}{}%
\end{pgfscope}%
\begin{pgfscope}%
\pgfsys@transformshift{1.723926in}{1.336176in}%
\pgfsys@useobject{currentmarker}{}%
\end{pgfscope}%
\begin{pgfscope}%
\pgfsys@transformshift{1.722893in}{1.329743in}%
\pgfsys@useobject{currentmarker}{}%
\end{pgfscope}%
\begin{pgfscope}%
\pgfsys@transformshift{1.722215in}{1.326224in}%
\pgfsys@useobject{currentmarker}{}%
\end{pgfscope}%
\begin{pgfscope}%
\pgfsys@transformshift{1.722732in}{1.321286in}%
\pgfsys@useobject{currentmarker}{}%
\end{pgfscope}%
\begin{pgfscope}%
\pgfsys@transformshift{1.722398in}{1.318576in}%
\pgfsys@useobject{currentmarker}{}%
\end{pgfscope}%
\begin{pgfscope}%
\pgfsys@transformshift{1.722120in}{1.317100in}%
\pgfsys@useobject{currentmarker}{}%
\end{pgfscope}%
\begin{pgfscope}%
\pgfsys@transformshift{1.722266in}{1.314247in}%
\pgfsys@useobject{currentmarker}{}%
\end{pgfscope}%
\begin{pgfscope}%
\pgfsys@transformshift{1.722164in}{1.312680in}%
\pgfsys@useobject{currentmarker}{}%
\end{pgfscope}%
\begin{pgfscope}%
\pgfsys@transformshift{1.722038in}{1.311825in}%
\pgfsys@useobject{currentmarker}{}%
\end{pgfscope}%
\begin{pgfscope}%
\pgfsys@transformshift{1.722043in}{1.311350in}%
\pgfsys@useobject{currentmarker}{}%
\end{pgfscope}%
\begin{pgfscope}%
\pgfsys@transformshift{1.722086in}{1.311092in}%
\pgfsys@useobject{currentmarker}{}%
\end{pgfscope}%
\begin{pgfscope}%
\pgfsys@transformshift{1.721937in}{1.309773in}%
\pgfsys@useobject{currentmarker}{}%
\end{pgfscope}%
\begin{pgfscope}%
\pgfsys@transformshift{1.721744in}{1.307681in}%
\pgfsys@useobject{currentmarker}{}%
\end{pgfscope}%
\begin{pgfscope}%
\pgfsys@transformshift{1.721979in}{1.304215in}%
\pgfsys@useobject{currentmarker}{}%
\end{pgfscope}%
\begin{pgfscope}%
\pgfsys@transformshift{1.721984in}{1.302304in}%
\pgfsys@useobject{currentmarker}{}%
\end{pgfscope}%
\begin{pgfscope}%
\pgfsys@transformshift{1.721855in}{1.301261in}%
\pgfsys@useobject{currentmarker}{}%
\end{pgfscope}%
\begin{pgfscope}%
\pgfsys@transformshift{1.721847in}{1.300684in}%
\pgfsys@useobject{currentmarker}{}%
\end{pgfscope}%
\begin{pgfscope}%
\pgfsys@transformshift{1.721886in}{1.300368in}%
\pgfsys@useobject{currentmarker}{}%
\end{pgfscope}%
\begin{pgfscope}%
\pgfsys@transformshift{1.721754in}{1.299020in}%
\pgfsys@useobject{currentmarker}{}%
\end{pgfscope}%
\begin{pgfscope}%
\pgfsys@transformshift{1.721673in}{1.297119in}%
\pgfsys@useobject{currentmarker}{}%
\end{pgfscope}%
\begin{pgfscope}%
\pgfsys@transformshift{1.722045in}{1.293646in}%
\pgfsys@useobject{currentmarker}{}%
\end{pgfscope}%
\begin{pgfscope}%
\pgfsys@transformshift{1.721792in}{1.289621in}%
\pgfsys@useobject{currentmarker}{}%
\end{pgfscope}%
\begin{pgfscope}%
\pgfsys@transformshift{1.721469in}{1.287426in}%
\pgfsys@useobject{currentmarker}{}%
\end{pgfscope}%
\begin{pgfscope}%
\pgfsys@transformshift{1.721493in}{1.283555in}%
\pgfsys@useobject{currentmarker}{}%
\end{pgfscope}%
\begin{pgfscope}%
\pgfsys@transformshift{1.721678in}{1.281434in}%
\pgfsys@useobject{currentmarker}{}%
\end{pgfscope}%
\begin{pgfscope}%
\pgfsys@transformshift{1.721463in}{1.278067in}%
\pgfsys@useobject{currentmarker}{}%
\end{pgfscope}%
\begin{pgfscope}%
\pgfsys@transformshift{1.720781in}{1.274196in}%
\pgfsys@useobject{currentmarker}{}%
\end{pgfscope}%
\begin{pgfscope}%
\pgfsys@transformshift{1.721042in}{1.268737in}%
\pgfsys@useobject{currentmarker}{}%
\end{pgfscope}%
\begin{pgfscope}%
\pgfsys@transformshift{1.721247in}{1.265738in}%
\pgfsys@useobject{currentmarker}{}%
\end{pgfscope}%
\begin{pgfscope}%
\pgfsys@transformshift{1.720880in}{1.261724in}%
\pgfsys@useobject{currentmarker}{}%
\end{pgfscope}%
\begin{pgfscope}%
\pgfsys@transformshift{1.720676in}{1.259517in}%
\pgfsys@useobject{currentmarker}{}%
\end{pgfscope}%
\begin{pgfscope}%
\pgfsys@transformshift{1.721114in}{1.256323in}%
\pgfsys@useobject{currentmarker}{}%
\end{pgfscope}%
\begin{pgfscope}%
\pgfsys@transformshift{1.720794in}{1.252391in}%
\pgfsys@useobject{currentmarker}{}%
\end{pgfscope}%
\begin{pgfscope}%
\pgfsys@transformshift{1.720431in}{1.247990in}%
\pgfsys@useobject{currentmarker}{}%
\end{pgfscope}%
\begin{pgfscope}%
\pgfsys@transformshift{1.720845in}{1.242631in}%
\pgfsys@useobject{currentmarker}{}%
\end{pgfscope}%
\begin{pgfscope}%
\pgfsys@transformshift{1.721195in}{1.239696in}%
\pgfsys@useobject{currentmarker}{}%
\end{pgfscope}%
\begin{pgfscope}%
\pgfsys@transformshift{1.721128in}{1.235996in}%
\pgfsys@useobject{currentmarker}{}%
\end{pgfscope}%
\begin{pgfscope}%
\pgfsys@transformshift{1.720615in}{1.231723in}%
\pgfsys@useobject{currentmarker}{}%
\end{pgfscope}%
\begin{pgfscope}%
\pgfsys@transformshift{1.721733in}{1.225917in}%
\pgfsys@useobject{currentmarker}{}%
\end{pgfscope}%
\begin{pgfscope}%
\pgfsys@transformshift{1.721919in}{1.222670in}%
\pgfsys@useobject{currentmarker}{}%
\end{pgfscope}%
\begin{pgfscope}%
\pgfsys@transformshift{1.721711in}{1.220894in}%
\pgfsys@useobject{currentmarker}{}%
\end{pgfscope}%
\begin{pgfscope}%
\pgfsys@transformshift{1.721669in}{1.219911in}%
\pgfsys@useobject{currentmarker}{}%
\end{pgfscope}%
\begin{pgfscope}%
\pgfsys@transformshift{1.721749in}{1.219376in}%
\pgfsys@useobject{currentmarker}{}%
\end{pgfscope}%
\begin{pgfscope}%
\pgfsys@transformshift{1.721549in}{1.218013in}%
\pgfsys@useobject{currentmarker}{}%
\end{pgfscope}%
\begin{pgfscope}%
\pgfsys@transformshift{1.721427in}{1.216070in}%
\pgfsys@useobject{currentmarker}{}%
\end{pgfscope}%
\begin{pgfscope}%
\pgfsys@transformshift{1.721991in}{1.211917in}%
\pgfsys@useobject{currentmarker}{}%
\end{pgfscope}%
\begin{pgfscope}%
\pgfsys@transformshift{1.722065in}{1.206821in}%
\pgfsys@useobject{currentmarker}{}%
\end{pgfscope}%
\begin{pgfscope}%
\pgfsys@transformshift{1.721525in}{1.204071in}%
\pgfsys@useobject{currentmarker}{}%
\end{pgfscope}%
\begin{pgfscope}%
\pgfsys@transformshift{1.722095in}{1.199272in}%
\pgfsys@useobject{currentmarker}{}%
\end{pgfscope}%
\begin{pgfscope}%
\pgfsys@transformshift{1.722493in}{1.196644in}%
\pgfsys@useobject{currentmarker}{}%
\end{pgfscope}%
\begin{pgfscope}%
\pgfsys@transformshift{1.721632in}{1.192174in}%
\pgfsys@useobject{currentmarker}{}%
\end{pgfscope}%
\begin{pgfscope}%
\pgfsys@transformshift{1.721686in}{1.189670in}%
\pgfsys@useobject{currentmarker}{}%
\end{pgfscope}%
\begin{pgfscope}%
\pgfsys@transformshift{1.722064in}{1.185947in}%
\pgfsys@useobject{currentmarker}{}%
\end{pgfscope}%
\begin{pgfscope}%
\pgfsys@transformshift{1.722256in}{1.183897in}%
\pgfsys@useobject{currentmarker}{}%
\end{pgfscope}%
\begin{pgfscope}%
\pgfsys@transformshift{1.721985in}{1.182798in}%
\pgfsys@useobject{currentmarker}{}%
\end{pgfscope}%
\begin{pgfscope}%
\pgfsys@transformshift{1.722539in}{1.180009in}%
\pgfsys@useobject{currentmarker}{}%
\end{pgfscope}%
\begin{pgfscope}%
\pgfsys@transformshift{1.722592in}{1.178446in}%
\pgfsys@useobject{currentmarker}{}%
\end{pgfscope}%
\begin{pgfscope}%
\pgfsys@transformshift{1.722307in}{1.174814in}%
\pgfsys@useobject{currentmarker}{}%
\end{pgfscope}%
\begin{pgfscope}%
\pgfsys@transformshift{1.721658in}{1.170686in}%
\pgfsys@useobject{currentmarker}{}%
\end{pgfscope}%
\begin{pgfscope}%
\pgfsys@transformshift{1.722773in}{1.164562in}%
\pgfsys@useobject{currentmarker}{}%
\end{pgfscope}%
\begin{pgfscope}%
\pgfsys@transformshift{1.722626in}{1.161142in}%
\pgfsys@useobject{currentmarker}{}%
\end{pgfscope}%
\begin{pgfscope}%
\pgfsys@transformshift{1.722303in}{1.159287in}%
\pgfsys@useobject{currentmarker}{}%
\end{pgfscope}%
\begin{pgfscope}%
\pgfsys@transformshift{1.722256in}{1.158252in}%
\pgfsys@useobject{currentmarker}{}%
\end{pgfscope}%
\begin{pgfscope}%
\pgfsys@transformshift{1.722339in}{1.157689in}%
\pgfsys@useobject{currentmarker}{}%
\end{pgfscope}%
\begin{pgfscope}%
\pgfsys@transformshift{1.722279in}{1.156611in}%
\pgfsys@useobject{currentmarker}{}%
\end{pgfscope}%
\begin{pgfscope}%
\pgfsys@transformshift{1.722063in}{1.155027in}%
\pgfsys@useobject{currentmarker}{}%
\end{pgfscope}%
\begin{pgfscope}%
\pgfsys@transformshift{1.722575in}{1.151063in}%
\pgfsys@useobject{currentmarker}{}%
\end{pgfscope}%
\begin{pgfscope}%
\pgfsys@transformshift{1.722707in}{1.148869in}%
\pgfsys@useobject{currentmarker}{}%
\end{pgfscope}%
\begin{pgfscope}%
\pgfsys@transformshift{1.722643in}{1.145532in}%
\pgfsys@useobject{currentmarker}{}%
\end{pgfscope}%
\begin{pgfscope}%
\pgfsys@transformshift{1.722295in}{1.143729in}%
\pgfsys@useobject{currentmarker}{}%
\end{pgfscope}%
\begin{pgfscope}%
\pgfsys@transformshift{1.723107in}{1.139940in}%
\pgfsys@useobject{currentmarker}{}%
\end{pgfscope}%
\begin{pgfscope}%
\pgfsys@transformshift{1.723203in}{1.135468in}%
\pgfsys@useobject{currentmarker}{}%
\end{pgfscope}%
\begin{pgfscope}%
\pgfsys@transformshift{1.723083in}{1.129391in}%
\pgfsys@useobject{currentmarker}{}%
\end{pgfscope}%
\begin{pgfscope}%
\pgfsys@transformshift{1.722765in}{1.126063in}%
\pgfsys@useobject{currentmarker}{}%
\end{pgfscope}%
\begin{pgfscope}%
\pgfsys@transformshift{1.723515in}{1.121014in}%
\pgfsys@useobject{currentmarker}{}%
\end{pgfscope}%
\begin{pgfscope}%
\pgfsys@transformshift{1.723588in}{1.118207in}%
\pgfsys@useobject{currentmarker}{}%
\end{pgfscope}%
\begin{pgfscope}%
\pgfsys@transformshift{1.723357in}{1.114086in}%
\pgfsys@useobject{currentmarker}{}%
\end{pgfscope}%
\begin{pgfscope}%
\pgfsys@transformshift{1.723205in}{1.111821in}%
\pgfsys@useobject{currentmarker}{}%
\end{pgfscope}%
\begin{pgfscope}%
\pgfsys@transformshift{1.723700in}{1.108727in}%
\pgfsys@useobject{currentmarker}{}%
\end{pgfscope}%
\begin{pgfscope}%
\pgfsys@transformshift{1.723753in}{1.107005in}%
\pgfsys@useobject{currentmarker}{}%
\end{pgfscope}%
\begin{pgfscope}%
\pgfsys@transformshift{1.723644in}{1.104458in}%
\pgfsys@useobject{currentmarker}{}%
\end{pgfscope}%
\begin{pgfscope}%
\pgfsys@transformshift{1.723639in}{1.103057in}%
\pgfsys@useobject{currentmarker}{}%
\end{pgfscope}%
\begin{pgfscope}%
\pgfsys@transformshift{1.723796in}{1.102302in}%
\pgfsys@useobject{currentmarker}{}%
\end{pgfscope}%
\begin{pgfscope}%
\pgfsys@transformshift{1.723814in}{1.100933in}%
\pgfsys@useobject{currentmarker}{}%
\end{pgfscope}%
\begin{pgfscope}%
\pgfsys@transformshift{1.723773in}{1.100181in}%
\pgfsys@useobject{currentmarker}{}%
\end{pgfscope}%
\begin{pgfscope}%
\pgfsys@transformshift{1.723757in}{1.099768in}%
\pgfsys@useobject{currentmarker}{}%
\end{pgfscope}%
\begin{pgfscope}%
\pgfsys@transformshift{1.723790in}{1.099542in}%
\pgfsys@useobject{currentmarker}{}%
\end{pgfscope}%
\begin{pgfscope}%
\pgfsys@transformshift{1.723809in}{1.099418in}%
\pgfsys@useobject{currentmarker}{}%
\end{pgfscope}%
\begin{pgfscope}%
\pgfsys@transformshift{1.723704in}{1.098543in}%
\pgfsys@useobject{currentmarker}{}%
\end{pgfscope}%
\begin{pgfscope}%
\pgfsys@transformshift{1.723747in}{1.098061in}%
\pgfsys@useobject{currentmarker}{}%
\end{pgfscope}%
\begin{pgfscope}%
\pgfsys@transformshift{1.723784in}{1.097797in}%
\pgfsys@useobject{currentmarker}{}%
\end{pgfscope}%
\begin{pgfscope}%
\pgfsys@transformshift{1.723799in}{1.097651in}%
\pgfsys@useobject{currentmarker}{}%
\end{pgfscope}%
\begin{pgfscope}%
\pgfsys@transformshift{1.723688in}{1.096967in}%
\pgfsys@useobject{currentmarker}{}%
\end{pgfscope}%
\begin{pgfscope}%
\pgfsys@transformshift{1.724274in}{1.094650in}%
\pgfsys@useobject{currentmarker}{}%
\end{pgfscope}%
\begin{pgfscope}%
\pgfsys@transformshift{1.724470in}{1.093350in}%
\pgfsys@useobject{currentmarker}{}%
\end{pgfscope}%
\begin{pgfscope}%
\pgfsys@transformshift{1.724769in}{1.090374in}%
\pgfsys@useobject{currentmarker}{}%
\end{pgfscope}%
\begin{pgfscope}%
\pgfsys@transformshift{1.724530in}{1.088747in}%
\pgfsys@useobject{currentmarker}{}%
\end{pgfscope}%
\begin{pgfscope}%
\pgfsys@transformshift{1.725471in}{1.085364in}%
\pgfsys@useobject{currentmarker}{}%
\end{pgfscope}%
\begin{pgfscope}%
\pgfsys@transformshift{1.726114in}{1.081347in}%
\pgfsys@useobject{currentmarker}{}%
\end{pgfscope}%
\begin{pgfscope}%
\pgfsys@transformshift{1.726511in}{1.075339in}%
\pgfsys@useobject{currentmarker}{}%
\end{pgfscope}%
\begin{pgfscope}%
\pgfsys@transformshift{1.726227in}{1.072040in}%
\pgfsys@useobject{currentmarker}{}%
\end{pgfscope}%
\begin{pgfscope}%
\pgfsys@transformshift{1.727142in}{1.066904in}%
\pgfsys@useobject{currentmarker}{}%
\end{pgfscope}%
\begin{pgfscope}%
\pgfsys@transformshift{1.728091in}{1.061114in}%
\pgfsys@useobject{currentmarker}{}%
\end{pgfscope}%
\begin{pgfscope}%
\pgfsys@transformshift{1.727598in}{1.053733in}%
\pgfsys@useobject{currentmarker}{}%
\end{pgfscope}%
\begin{pgfscope}%
\pgfsys@transformshift{1.727507in}{1.049665in}%
\pgfsys@useobject{currentmarker}{}%
\end{pgfscope}%
\begin{pgfscope}%
\pgfsys@transformshift{1.728848in}{1.044701in}%
\pgfsys@useobject{currentmarker}{}%
\end{pgfscope}%
\begin{pgfscope}%
\pgfsys@transformshift{1.729250in}{1.039110in}%
\pgfsys@useobject{currentmarker}{}%
\end{pgfscope}%
\begin{pgfscope}%
\pgfsys@transformshift{1.729245in}{1.033020in}%
\pgfsys@useobject{currentmarker}{}%
\end{pgfscope}%
\begin{pgfscope}%
\pgfsys@transformshift{1.728342in}{1.026307in}%
\pgfsys@useobject{currentmarker}{}%
\end{pgfscope}%
\begin{pgfscope}%
\pgfsys@transformshift{1.729423in}{1.022742in}%
\pgfsys@useobject{currentmarker}{}%
\end{pgfscope}%
\begin{pgfscope}%
\pgfsys@transformshift{1.729800in}{1.018306in}%
\pgfsys@useobject{currentmarker}{}%
\end{pgfscope}%
\begin{pgfscope}%
\pgfsys@transformshift{1.730218in}{1.013268in}%
\pgfsys@useobject{currentmarker}{}%
\end{pgfscope}%
\begin{pgfscope}%
\pgfsys@transformshift{1.730198in}{1.010488in}%
\pgfsys@useobject{currentmarker}{}%
\end{pgfscope}%
\begin{pgfscope}%
\pgfsys@transformshift{1.731187in}{1.006799in}%
\pgfsys@useobject{currentmarker}{}%
\end{pgfscope}%
\begin{pgfscope}%
\pgfsys@transformshift{1.732061in}{1.002410in}%
\pgfsys@useobject{currentmarker}{}%
\end{pgfscope}%
\begin{pgfscope}%
\pgfsys@transformshift{1.732397in}{0.999971in}%
\pgfsys@useobject{currentmarker}{}%
\end{pgfscope}%
\begin{pgfscope}%
\pgfsys@transformshift{1.732349in}{0.998618in}%
\pgfsys@useobject{currentmarker}{}%
\end{pgfscope}%
\begin{pgfscope}%
\pgfsys@transformshift{1.732840in}{0.995889in}%
\pgfsys@useobject{currentmarker}{}%
\end{pgfscope}%
\begin{pgfscope}%
\pgfsys@transformshift{1.733100in}{0.994386in}%
\pgfsys@useobject{currentmarker}{}%
\end{pgfscope}%
\begin{pgfscope}%
\pgfsys@transformshift{1.733234in}{0.993558in}%
\pgfsys@useobject{currentmarker}{}%
\end{pgfscope}%
\begin{pgfscope}%
\pgfsys@transformshift{1.733155in}{0.992200in}%
\pgfsys@useobject{currentmarker}{}%
\end{pgfscope}%
\begin{pgfscope}%
\pgfsys@transformshift{1.733229in}{0.991456in}%
\pgfsys@useobject{currentmarker}{}%
\end{pgfscope}%
\begin{pgfscope}%
\pgfsys@transformshift{1.733291in}{0.991049in}%
\pgfsys@useobject{currentmarker}{}%
\end{pgfscope}%
\begin{pgfscope}%
\pgfsys@transformshift{1.733527in}{0.989882in}%
\pgfsys@useobject{currentmarker}{}%
\end{pgfscope}%
\begin{pgfscope}%
\pgfsys@transformshift{1.733576in}{0.988061in}%
\pgfsys@useobject{currentmarker}{}%
\end{pgfscope}%
\begin{pgfscope}%
\pgfsys@transformshift{1.733667in}{0.987064in}%
\pgfsys@useobject{currentmarker}{}%
\end{pgfscope}%
\begin{pgfscope}%
\pgfsys@transformshift{1.733992in}{0.984892in}%
\pgfsys@useobject{currentmarker}{}%
\end{pgfscope}%
\begin{pgfscope}%
\pgfsys@transformshift{1.734787in}{0.982230in}%
\pgfsys@useobject{currentmarker}{}%
\end{pgfscope}%
\begin{pgfscope}%
\pgfsys@transformshift{1.734968in}{0.980712in}%
\pgfsys@useobject{currentmarker}{}%
\end{pgfscope}%
\begin{pgfscope}%
\pgfsys@transformshift{1.734993in}{0.979872in}%
\pgfsys@useobject{currentmarker}{}%
\end{pgfscope}%
\begin{pgfscope}%
\pgfsys@transformshift{1.735038in}{0.979412in}%
\pgfsys@useobject{currentmarker}{}%
\end{pgfscope}%
\begin{pgfscope}%
\pgfsys@transformshift{1.735350in}{0.978321in}%
\pgfsys@useobject{currentmarker}{}%
\end{pgfscope}%
\begin{pgfscope}%
\pgfsys@transformshift{1.735702in}{0.976464in}%
\pgfsys@useobject{currentmarker}{}%
\end{pgfscope}%
\begin{pgfscope}%
\pgfsys@transformshift{1.735891in}{0.973924in}%
\pgfsys@useobject{currentmarker}{}%
\end{pgfscope}%
\begin{pgfscope}%
\pgfsys@transformshift{1.736165in}{0.970627in}%
\pgfsys@useobject{currentmarker}{}%
\end{pgfscope}%
\begin{pgfscope}%
\pgfsys@transformshift{1.736693in}{0.968886in}%
\pgfsys@useobject{currentmarker}{}%
\end{pgfscope}%
\begin{pgfscope}%
\pgfsys@transformshift{1.736934in}{0.967914in}%
\pgfsys@useobject{currentmarker}{}%
\end{pgfscope}%
\begin{pgfscope}%
\pgfsys@transformshift{1.737036in}{0.967373in}%
\pgfsys@useobject{currentmarker}{}%
\end{pgfscope}%
\begin{pgfscope}%
\pgfsys@transformshift{1.737001in}{0.966333in}%
\pgfsys@useobject{currentmarker}{}%
\end{pgfscope}%
\begin{pgfscope}%
\pgfsys@transformshift{1.737060in}{0.965763in}%
\pgfsys@useobject{currentmarker}{}%
\end{pgfscope}%
\begin{pgfscope}%
\pgfsys@transformshift{1.737272in}{0.964725in}%
\pgfsys@useobject{currentmarker}{}%
\end{pgfscope}%
\begin{pgfscope}%
\pgfsys@transformshift{1.737755in}{0.962824in}%
\pgfsys@useobject{currentmarker}{}%
\end{pgfscope}%
\begin{pgfscope}%
\pgfsys@transformshift{1.737755in}{0.961746in}%
\pgfsys@useobject{currentmarker}{}%
\end{pgfscope}%
\begin{pgfscope}%
\pgfsys@transformshift{1.738085in}{0.960003in}%
\pgfsys@useobject{currentmarker}{}%
\end{pgfscope}%
\begin{pgfscope}%
\pgfsys@transformshift{1.737981in}{0.957435in}%
\pgfsys@useobject{currentmarker}{}%
\end{pgfscope}%
\begin{pgfscope}%
\pgfsys@transformshift{1.738341in}{0.953894in}%
\pgfsys@useobject{currentmarker}{}%
\end{pgfscope}%
\begin{pgfscope}%
\pgfsys@transformshift{1.738043in}{0.951959in}%
\pgfsys@useobject{currentmarker}{}%
\end{pgfscope}%
\begin{pgfscope}%
\pgfsys@transformshift{1.737704in}{0.950938in}%
\pgfsys@useobject{currentmarker}{}%
\end{pgfscope}%
\begin{pgfscope}%
\pgfsys@transformshift{1.736498in}{0.949735in}%
\pgfsys@useobject{currentmarker}{}%
\end{pgfscope}%
\begin{pgfscope}%
\pgfsys@transformshift{1.734152in}{0.948245in}%
\pgfsys@useobject{currentmarker}{}%
\end{pgfscope}%
\begin{pgfscope}%
\pgfsys@transformshift{1.732648in}{0.947969in}%
\pgfsys@useobject{currentmarker}{}%
\end{pgfscope}%
\begin{pgfscope}%
\pgfsys@transformshift{1.730137in}{0.947651in}%
\pgfsys@useobject{currentmarker}{}%
\end{pgfscope}%
\begin{pgfscope}%
\pgfsys@transformshift{1.726136in}{0.948075in}%
\pgfsys@useobject{currentmarker}{}%
\end{pgfscope}%
\begin{pgfscope}%
\pgfsys@transformshift{1.721055in}{0.948636in}%
\pgfsys@useobject{currentmarker}{}%
\end{pgfscope}%
\begin{pgfscope}%
\pgfsys@transformshift{1.714778in}{0.949504in}%
\pgfsys@useobject{currentmarker}{}%
\end{pgfscope}%
\begin{pgfscope}%
\pgfsys@transformshift{1.707596in}{0.949177in}%
\pgfsys@useobject{currentmarker}{}%
\end{pgfscope}%
\begin{pgfscope}%
\pgfsys@transformshift{1.700056in}{0.951107in}%
\pgfsys@useobject{currentmarker}{}%
\end{pgfscope}%
\begin{pgfscope}%
\pgfsys@transformshift{1.691662in}{0.951673in}%
\pgfsys@useobject{currentmarker}{}%
\end{pgfscope}%
\begin{pgfscope}%
\pgfsys@transformshift{1.687146in}{0.952680in}%
\pgfsys@useobject{currentmarker}{}%
\end{pgfscope}%
\begin{pgfscope}%
\pgfsys@transformshift{1.684603in}{0.952775in}%
\pgfsys@useobject{currentmarker}{}%
\end{pgfscope}%
\begin{pgfscope}%
\pgfsys@transformshift{1.681343in}{0.953357in}%
\pgfsys@useobject{currentmarker}{}%
\end{pgfscope}%
\begin{pgfscope}%
\pgfsys@transformshift{1.679536in}{0.953132in}%
\pgfsys@useobject{currentmarker}{}%
\end{pgfscope}%
\begin{pgfscope}%
\pgfsys@transformshift{1.677138in}{0.953268in}%
\pgfsys@useobject{currentmarker}{}%
\end{pgfscope}%
\begin{pgfscope}%
\pgfsys@transformshift{1.672310in}{0.953220in}%
\pgfsys@useobject{currentmarker}{}%
\end{pgfscope}%
\begin{pgfscope}%
\pgfsys@transformshift{1.669666in}{0.953472in}%
\pgfsys@useobject{currentmarker}{}%
\end{pgfscope}%
\begin{pgfscope}%
\pgfsys@transformshift{1.665544in}{0.953700in}%
\pgfsys@useobject{currentmarker}{}%
\end{pgfscope}%
\begin{pgfscope}%
\pgfsys@transformshift{1.659916in}{0.953965in}%
\pgfsys@useobject{currentmarker}{}%
\end{pgfscope}%
\begin{pgfscope}%
\pgfsys@transformshift{1.656822in}{0.953797in}%
\pgfsys@useobject{currentmarker}{}%
\end{pgfscope}%
\begin{pgfscope}%
\pgfsys@transformshift{1.651684in}{0.953790in}%
\pgfsys@useobject{currentmarker}{}%
\end{pgfscope}%
\begin{pgfscope}%
\pgfsys@transformshift{1.645951in}{0.953765in}%
\pgfsys@useobject{currentmarker}{}%
\end{pgfscope}%
\begin{pgfscope}%
\pgfsys@transformshift{1.638096in}{0.953528in}%
\pgfsys@useobject{currentmarker}{}%
\end{pgfscope}%
\begin{pgfscope}%
\pgfsys@transformshift{1.627734in}{0.952571in}%
\pgfsys@useobject{currentmarker}{}%
\end{pgfscope}%
\begin{pgfscope}%
\pgfsys@transformshift{1.622014in}{0.952365in}%
\pgfsys@useobject{currentmarker}{}%
\end{pgfscope}%
\begin{pgfscope}%
\pgfsys@transformshift{1.613574in}{0.952051in}%
\pgfsys@useobject{currentmarker}{}%
\end{pgfscope}%
\begin{pgfscope}%
\pgfsys@transformshift{1.608933in}{0.951844in}%
\pgfsys@useobject{currentmarker}{}%
\end{pgfscope}%
\begin{pgfscope}%
\pgfsys@transformshift{1.602733in}{0.951082in}%
\pgfsys@useobject{currentmarker}{}%
\end{pgfscope}%
\begin{pgfscope}%
\pgfsys@transformshift{1.599300in}{0.950928in}%
\pgfsys@useobject{currentmarker}{}%
\end{pgfscope}%
\begin{pgfscope}%
\pgfsys@transformshift{1.594080in}{0.950885in}%
\pgfsys@useobject{currentmarker}{}%
\end{pgfscope}%
\begin{pgfscope}%
\pgfsys@transformshift{1.586279in}{0.949569in}%
\pgfsys@useobject{currentmarker}{}%
\end{pgfscope}%
\begin{pgfscope}%
\pgfsys@transformshift{1.581940in}{0.949248in}%
\pgfsys@useobject{currentmarker}{}%
\end{pgfscope}%
\begin{pgfscope}%
\pgfsys@transformshift{1.575208in}{0.949233in}%
\pgfsys@useobject{currentmarker}{}%
\end{pgfscope}%
\begin{pgfscope}%
\pgfsys@transformshift{1.571507in}{0.949170in}%
\pgfsys@useobject{currentmarker}{}%
\end{pgfscope}%
\begin{pgfscope}%
\pgfsys@transformshift{1.565967in}{0.948493in}%
\pgfsys@useobject{currentmarker}{}%
\end{pgfscope}%
\begin{pgfscope}%
\pgfsys@transformshift{1.562909in}{0.948232in}%
\pgfsys@useobject{currentmarker}{}%
\end{pgfscope}%
\begin{pgfscope}%
\pgfsys@transformshift{1.556864in}{0.947712in}%
\pgfsys@useobject{currentmarker}{}%
\end{pgfscope}%
\begin{pgfscope}%
\pgfsys@transformshift{1.549573in}{0.946663in}%
\pgfsys@useobject{currentmarker}{}%
\end{pgfscope}%
\begin{pgfscope}%
\pgfsys@transformshift{1.541235in}{0.946023in}%
\pgfsys@useobject{currentmarker}{}%
\end{pgfscope}%
\begin{pgfscope}%
\pgfsys@transformshift{1.531155in}{0.946018in}%
\pgfsys@useobject{currentmarker}{}%
\end{pgfscope}%
\begin{pgfscope}%
\pgfsys@transformshift{1.520081in}{0.945662in}%
\pgfsys@useobject{currentmarker}{}%
\end{pgfscope}%
\begin{pgfscope}%
\pgfsys@transformshift{1.506840in}{0.944631in}%
\pgfsys@useobject{currentmarker}{}%
\end{pgfscope}%
\begin{pgfscope}%
\pgfsys@transformshift{1.499552in}{0.944141in}%
\pgfsys@useobject{currentmarker}{}%
\end{pgfscope}%
\begin{pgfscope}%
\pgfsys@transformshift{1.490451in}{0.943085in}%
\pgfsys@useobject{currentmarker}{}%
\end{pgfscope}%
\begin{pgfscope}%
\pgfsys@transformshift{1.480106in}{0.942085in}%
\pgfsys@useobject{currentmarker}{}%
\end{pgfscope}%
\begin{pgfscope}%
\pgfsys@transformshift{1.469046in}{0.941007in}%
\pgfsys@useobject{currentmarker}{}%
\end{pgfscope}%
\begin{pgfscope}%
\pgfsys@transformshift{1.455454in}{0.939795in}%
\pgfsys@useobject{currentmarker}{}%
\end{pgfscope}%
\begin{pgfscope}%
\pgfsys@transformshift{1.447949in}{0.939770in}%
\pgfsys@useobject{currentmarker}{}%
\end{pgfscope}%
\begin{pgfscope}%
\pgfsys@transformshift{1.437838in}{0.938306in}%
\pgfsys@useobject{currentmarker}{}%
\end{pgfscope}%
\begin{pgfscope}%
\pgfsys@transformshift{1.432231in}{0.937945in}%
\pgfsys@useobject{currentmarker}{}%
\end{pgfscope}%
\begin{pgfscope}%
\pgfsys@transformshift{1.423706in}{0.937484in}%
\pgfsys@useobject{currentmarker}{}%
\end{pgfscope}%
\begin{pgfscope}%
\pgfsys@transformshift{1.414139in}{0.936513in}%
\pgfsys@useobject{currentmarker}{}%
\end{pgfscope}%
\begin{pgfscope}%
\pgfsys@transformshift{1.403821in}{0.935798in}%
\pgfsys@useobject{currentmarker}{}%
\end{pgfscope}%
\begin{pgfscope}%
\pgfsys@transformshift{1.392002in}{0.936524in}%
\pgfsys@useobject{currentmarker}{}%
\end{pgfscope}%
\begin{pgfscope}%
\pgfsys@transformshift{1.385500in}{0.936169in}%
\pgfsys@useobject{currentmarker}{}%
\end{pgfscope}%
\begin{pgfscope}%
\pgfsys@transformshift{1.376538in}{0.935646in}%
\pgfsys@useobject{currentmarker}{}%
\end{pgfscope}%
\begin{pgfscope}%
\pgfsys@transformshift{1.366832in}{0.935266in}%
\pgfsys@useobject{currentmarker}{}%
\end{pgfscope}%
\begin{pgfscope}%
\pgfsys@transformshift{1.355192in}{0.934713in}%
\pgfsys@useobject{currentmarker}{}%
\end{pgfscope}%
\begin{pgfscope}%
\pgfsys@transformshift{1.348783in}{0.934698in}%
\pgfsys@useobject{currentmarker}{}%
\end{pgfscope}%
\begin{pgfscope}%
\pgfsys@transformshift{1.340942in}{0.933750in}%
\pgfsys@useobject{currentmarker}{}%
\end{pgfscope}%
\begin{pgfscope}%
\pgfsys@transformshift{1.336598in}{0.933699in}%
\pgfsys@useobject{currentmarker}{}%
\end{pgfscope}%
\begin{pgfscope}%
\pgfsys@transformshift{1.330128in}{0.933526in}%
\pgfsys@useobject{currentmarker}{}%
\end{pgfscope}%
\begin{pgfscope}%
\pgfsys@transformshift{1.323065in}{0.933354in}%
\pgfsys@useobject{currentmarker}{}%
\end{pgfscope}%
\begin{pgfscope}%
\pgfsys@transformshift{1.314337in}{0.932901in}%
\pgfsys@useobject{currentmarker}{}%
\end{pgfscope}%
\begin{pgfscope}%
\pgfsys@transformshift{1.309531in}{0.932987in}%
\pgfsys@useobject{currentmarker}{}%
\end{pgfscope}%
\begin{pgfscope}%
\pgfsys@transformshift{1.304058in}{0.933057in}%
\pgfsys@useobject{currentmarker}{}%
\end{pgfscope}%
\begin{pgfscope}%
\pgfsys@transformshift{1.298069in}{0.933442in}%
\pgfsys@useobject{currentmarker}{}%
\end{pgfscope}%
\begin{pgfscope}%
\pgfsys@transformshift{1.294850in}{0.934171in}%
\pgfsys@useobject{currentmarker}{}%
\end{pgfscope}%
\begin{pgfscope}%
\pgfsys@transformshift{1.293245in}{0.935019in}%
\pgfsys@useobject{currentmarker}{}%
\end{pgfscope}%
\begin{pgfscope}%
\pgfsys@transformshift{1.291496in}{0.936932in}%
\pgfsys@useobject{currentmarker}{}%
\end{pgfscope}%
\begin{pgfscope}%
\pgfsys@transformshift{1.290488in}{0.937941in}%
\pgfsys@useobject{currentmarker}{}%
\end{pgfscope}%
\begin{pgfscope}%
\pgfsys@transformshift{1.290528in}{0.939965in}%
\pgfsys@useobject{currentmarker}{}%
\end{pgfscope}%
\begin{pgfscope}%
\pgfsys@transformshift{1.290522in}{0.942583in}%
\pgfsys@useobject{currentmarker}{}%
\end{pgfscope}%
\begin{pgfscope}%
\pgfsys@transformshift{1.290950in}{0.946324in}%
\pgfsys@useobject{currentmarker}{}%
\end{pgfscope}%
\begin{pgfscope}%
\pgfsys@transformshift{1.291622in}{0.951031in}%
\pgfsys@useobject{currentmarker}{}%
\end{pgfscope}%
\begin{pgfscope}%
\pgfsys@transformshift{1.292534in}{0.956827in}%
\pgfsys@useobject{currentmarker}{}%
\end{pgfscope}%
\begin{pgfscope}%
\pgfsys@transformshift{1.292883in}{0.963643in}%
\pgfsys@useobject{currentmarker}{}%
\end{pgfscope}%
\begin{pgfscope}%
\pgfsys@transformshift{1.293338in}{0.967368in}%
\pgfsys@useobject{currentmarker}{}%
\end{pgfscope}%
\begin{pgfscope}%
\pgfsys@transformshift{1.292324in}{0.971535in}%
\pgfsys@useobject{currentmarker}{}%
\end{pgfscope}%
\begin{pgfscope}%
\pgfsys@transformshift{1.294812in}{0.978915in}%
\pgfsys@useobject{currentmarker}{}%
\end{pgfscope}%
\begin{pgfscope}%
\pgfsys@transformshift{1.291569in}{0.989001in}%
\pgfsys@useobject{currentmarker}{}%
\end{pgfscope}%
\begin{pgfscope}%
\pgfsys@transformshift{1.291195in}{0.994816in}%
\pgfsys@useobject{currentmarker}{}%
\end{pgfscope}%
\begin{pgfscope}%
\pgfsys@transformshift{1.293223in}{1.003690in}%
\pgfsys@useobject{currentmarker}{}%
\end{pgfscope}%
\begin{pgfscope}%
\pgfsys@transformshift{1.289310in}{1.014408in}%
\pgfsys@useobject{currentmarker}{}%
\end{pgfscope}%
\begin{pgfscope}%
\pgfsys@transformshift{1.290004in}{1.026417in}%
\pgfsys@useobject{currentmarker}{}%
\end{pgfscope}%
\begin{pgfscope}%
\pgfsys@transformshift{1.293668in}{1.039683in}%
\pgfsys@useobject{currentmarker}{}%
\end{pgfscope}%
\begin{pgfscope}%
\pgfsys@transformshift{1.289090in}{1.055889in}%
\pgfsys@useobject{currentmarker}{}%
\end{pgfscope}%
\begin{pgfscope}%
\pgfsys@transformshift{1.288941in}{1.065149in}%
\pgfsys@useobject{currentmarker}{}%
\end{pgfscope}%
\begin{pgfscope}%
\pgfsys@transformshift{1.290632in}{1.077386in}%
\pgfsys@useobject{currentmarker}{}%
\end{pgfscope}%
\begin{pgfscope}%
\pgfsys@transformshift{1.287835in}{1.090662in}%
\pgfsys@useobject{currentmarker}{}%
\end{pgfscope}%
\begin{pgfscope}%
\pgfsys@transformshift{1.283836in}{1.104148in}%
\pgfsys@useobject{currentmarker}{}%
\end{pgfscope}%
\begin{pgfscope}%
\pgfsys@transformshift{1.280112in}{1.119594in}%
\pgfsys@useobject{currentmarker}{}%
\end{pgfscope}%
\begin{pgfscope}%
\pgfsys@transformshift{1.283713in}{1.135605in}%
\pgfsys@useobject{currentmarker}{}%
\end{pgfscope}%
\begin{pgfscope}%
\pgfsys@transformshift{1.277377in}{1.153528in}%
\pgfsys@useobject{currentmarker}{}%
\end{pgfscope}%
\begin{pgfscope}%
\pgfsys@transformshift{1.275880in}{1.173299in}%
\pgfsys@useobject{currentmarker}{}%
\end{pgfscope}%
\begin{pgfscope}%
\pgfsys@transformshift{1.278101in}{1.196543in}%
\pgfsys@useobject{currentmarker}{}%
\end{pgfscope}%
\begin{pgfscope}%
\pgfsys@transformshift{1.275878in}{1.220663in}%
\pgfsys@useobject{currentmarker}{}%
\end{pgfscope}%
\begin{pgfscope}%
\pgfsys@transformshift{1.267280in}{1.243868in}%
\pgfsys@useobject{currentmarker}{}%
\end{pgfscope}%
\begin{pgfscope}%
\pgfsys@transformshift{1.266822in}{1.269202in}%
\pgfsys@useobject{currentmarker}{}%
\end{pgfscope}%
\begin{pgfscope}%
\pgfsys@transformshift{1.272240in}{1.295098in}%
\pgfsys@useobject{currentmarker}{}%
\end{pgfscope}%
\begin{pgfscope}%
\pgfsys@transformshift{1.267617in}{1.322928in}%
\pgfsys@useobject{currentmarker}{}%
\end{pgfscope}%
\begin{pgfscope}%
\pgfsys@transformshift{1.263985in}{1.338014in}%
\pgfsys@useobject{currentmarker}{}%
\end{pgfscope}%
\begin{pgfscope}%
\pgfsys@transformshift{1.260232in}{1.354000in}%
\pgfsys@useobject{currentmarker}{}%
\end{pgfscope}%
\begin{pgfscope}%
\pgfsys@transformshift{1.265004in}{1.371422in}%
\pgfsys@useobject{currentmarker}{}%
\end{pgfscope}%
\begin{pgfscope}%
\pgfsys@transformshift{1.258631in}{1.392495in}%
\pgfsys@useobject{currentmarker}{}%
\end{pgfscope}%
\begin{pgfscope}%
\pgfsys@transformshift{1.256845in}{1.404471in}%
\pgfsys@useobject{currentmarker}{}%
\end{pgfscope}%
\begin{pgfscope}%
\pgfsys@transformshift{1.254336in}{1.419241in}%
\pgfsys@useobject{currentmarker}{}%
\end{pgfscope}%
\begin{pgfscope}%
\pgfsys@transformshift{1.255501in}{1.427398in}%
\pgfsys@useobject{currentmarker}{}%
\end{pgfscope}%
\begin{pgfscope}%
\pgfsys@transformshift{1.252223in}{1.437920in}%
\pgfsys@useobject{currentmarker}{}%
\end{pgfscope}%
\begin{pgfscope}%
\pgfsys@transformshift{1.252870in}{1.443947in}%
\pgfsys@useobject{currentmarker}{}%
\end{pgfscope}%
\begin{pgfscope}%
\pgfsys@transformshift{1.253750in}{1.451961in}%
\pgfsys@useobject{currentmarker}{}%
\end{pgfscope}%
\begin{pgfscope}%
\pgfsys@transformshift{1.253298in}{1.461006in}%
\pgfsys@useobject{currentmarker}{}%
\end{pgfscope}%
\begin{pgfscope}%
\pgfsys@transformshift{1.252043in}{1.465827in}%
\pgfsys@useobject{currentmarker}{}%
\end{pgfscope}%
\begin{pgfscope}%
\pgfsys@transformshift{1.252124in}{1.468565in}%
\pgfsys@useobject{currentmarker}{}%
\end{pgfscope}%
\begin{pgfscope}%
\pgfsys@transformshift{1.252724in}{1.473087in}%
\pgfsys@useobject{currentmarker}{}%
\end{pgfscope}%
\begin{pgfscope}%
\pgfsys@transformshift{1.251619in}{1.479495in}%
\pgfsys@useobject{currentmarker}{}%
\end{pgfscope}%
\begin{pgfscope}%
\pgfsys@transformshift{1.250649in}{1.482937in}%
\pgfsys@useobject{currentmarker}{}%
\end{pgfscope}%
\begin{pgfscope}%
\pgfsys@transformshift{1.250535in}{1.486997in}%
\pgfsys@useobject{currentmarker}{}%
\end{pgfscope}%
\begin{pgfscope}%
\pgfsys@transformshift{1.251631in}{1.492253in}%
\pgfsys@useobject{currentmarker}{}%
\end{pgfscope}%
\begin{pgfscope}%
\pgfsys@transformshift{1.248813in}{1.501161in}%
\pgfsys@useobject{currentmarker}{}%
\end{pgfscope}%
\begin{pgfscope}%
\pgfsys@transformshift{1.247934in}{1.506224in}%
\pgfsys@useobject{currentmarker}{}%
\end{pgfscope}%
\begin{pgfscope}%
\pgfsys@transformshift{1.246476in}{1.512556in}%
\pgfsys@useobject{currentmarker}{}%
\end{pgfscope}%
\begin{pgfscope}%
\pgfsys@transformshift{1.248453in}{1.519561in}%
\pgfsys@useobject{currentmarker}{}%
\end{pgfscope}%
\begin{pgfscope}%
\pgfsys@transformshift{1.245637in}{1.530022in}%
\pgfsys@useobject{currentmarker}{}%
\end{pgfscope}%
\begin{pgfscope}%
\pgfsys@transformshift{1.244823in}{1.535925in}%
\pgfsys@useobject{currentmarker}{}%
\end{pgfscope}%
\begin{pgfscope}%
\pgfsys@transformshift{1.244432in}{1.543417in}%
\pgfsys@useobject{currentmarker}{}%
\end{pgfscope}%
\begin{pgfscope}%
\pgfsys@transformshift{1.246193in}{1.551369in}%
\pgfsys@useobject{currentmarker}{}%
\end{pgfscope}%
\begin{pgfscope}%
\pgfsys@transformshift{1.246251in}{1.563047in}%
\pgfsys@useobject{currentmarker}{}%
\end{pgfscope}%
\begin{pgfscope}%
\pgfsys@transformshift{1.247997in}{1.575985in}%
\pgfsys@useobject{currentmarker}{}%
\end{pgfscope}%
\begin{pgfscope}%
\pgfsys@transformshift{1.251132in}{1.590771in}%
\pgfsys@useobject{currentmarker}{}%
\end{pgfscope}%
\begin{pgfscope}%
\pgfsys@transformshift{1.247214in}{1.606289in}%
\pgfsys@useobject{currentmarker}{}%
\end{pgfscope}%
\begin{pgfscope}%
\pgfsys@transformshift{1.253594in}{1.625306in}%
\pgfsys@useobject{currentmarker}{}%
\end{pgfscope}%
\begin{pgfscope}%
\pgfsys@transformshift{1.254548in}{1.636296in}%
\pgfsys@useobject{currentmarker}{}%
\end{pgfscope}%
\begin{pgfscope}%
\pgfsys@transformshift{1.257333in}{1.649227in}%
\pgfsys@useobject{currentmarker}{}%
\end{pgfscope}%
\begin{pgfscope}%
\pgfsys@transformshift{1.255700in}{1.656316in}%
\pgfsys@useobject{currentmarker}{}%
\end{pgfscope}%
\begin{pgfscope}%
\pgfsys@transformshift{1.258361in}{1.665815in}%
\pgfsys@useobject{currentmarker}{}%
\end{pgfscope}%
\begin{pgfscope}%
\pgfsys@transformshift{1.258968in}{1.671206in}%
\pgfsys@useobject{currentmarker}{}%
\end{pgfscope}%
\begin{pgfscope}%
\pgfsys@transformshift{1.259672in}{1.674106in}%
\pgfsys@useobject{currentmarker}{}%
\end{pgfscope}%
\begin{pgfscope}%
\pgfsys@transformshift{1.259161in}{1.677553in}%
\pgfsys@useobject{currentmarker}{}%
\end{pgfscope}%
\begin{pgfscope}%
\pgfsys@transformshift{1.261352in}{1.684108in}%
\pgfsys@useobject{currentmarker}{}%
\end{pgfscope}%
\begin{pgfscope}%
\pgfsys@transformshift{1.261821in}{1.687881in}%
\pgfsys@useobject{currentmarker}{}%
\end{pgfscope}%
\begin{pgfscope}%
\pgfsys@transformshift{1.262785in}{1.693687in}%
\pgfsys@useobject{currentmarker}{}%
\end{pgfscope}%
\begin{pgfscope}%
\pgfsys@transformshift{1.261696in}{1.700054in}%
\pgfsys@useobject{currentmarker}{}%
\end{pgfscope}%
\begin{pgfscope}%
\pgfsys@transformshift{1.264740in}{1.709484in}%
\pgfsys@useobject{currentmarker}{}%
\end{pgfscope}%
\begin{pgfscope}%
\pgfsys@transformshift{1.265434in}{1.714890in}%
\pgfsys@useobject{currentmarker}{}%
\end{pgfscope}%
\begin{pgfscope}%
\pgfsys@transformshift{1.265784in}{1.722379in}%
\pgfsys@useobject{currentmarker}{}%
\end{pgfscope}%
\begin{pgfscope}%
\pgfsys@transformshift{1.265618in}{1.726498in}%
\pgfsys@useobject{currentmarker}{}%
\end{pgfscope}%
\begin{pgfscope}%
\pgfsys@transformshift{1.265953in}{1.728741in}%
\pgfsys@useobject{currentmarker}{}%
\end{pgfscope}%
\begin{pgfscope}%
\pgfsys@transformshift{1.265586in}{1.731481in}%
\pgfsys@useobject{currentmarker}{}%
\end{pgfscope}%
\begin{pgfscope}%
\pgfsys@transformshift{1.264234in}{1.734640in}%
\pgfsys@useobject{currentmarker}{}%
\end{pgfscope}%
\begin{pgfscope}%
\pgfsys@transformshift{1.261027in}{1.737455in}%
\pgfsys@useobject{currentmarker}{}%
\end{pgfscope}%
\begin{pgfscope}%
\pgfsys@transformshift{1.256048in}{1.739616in}%
\pgfsys@useobject{currentmarker}{}%
\end{pgfscope}%
\begin{pgfscope}%
\pgfsys@transformshift{1.249644in}{1.740985in}%
\pgfsys@useobject{currentmarker}{}%
\end{pgfscope}%
\begin{pgfscope}%
\pgfsys@transformshift{1.245947in}{1.741255in}%
\pgfsys@useobject{currentmarker}{}%
\end{pgfscope}%
\begin{pgfscope}%
\pgfsys@transformshift{1.249939in}{1.741574in}%
\pgfsys@useobject{currentmarker}{}%
\end{pgfscope}%
\begin{pgfscope}%
\pgfsys@transformshift{1.252141in}{1.741619in}%
\pgfsys@useobject{currentmarker}{}%
\end{pgfscope}%
\begin{pgfscope}%
\pgfsys@transformshift{1.255081in}{1.741677in}%
\pgfsys@useobject{currentmarker}{}%
\end{pgfscope}%
\begin{pgfscope}%
\pgfsys@transformshift{1.259220in}{1.741576in}%
\pgfsys@useobject{currentmarker}{}%
\end{pgfscope}%
\begin{pgfscope}%
\pgfsys@transformshift{1.264255in}{1.741730in}%
\pgfsys@useobject{currentmarker}{}%
\end{pgfscope}%
\begin{pgfscope}%
\pgfsys@transformshift{1.267014in}{1.741476in}%
\pgfsys@useobject{currentmarker}{}%
\end{pgfscope}%
\begin{pgfscope}%
\pgfsys@transformshift{1.268526in}{1.741660in}%
\pgfsys@useobject{currentmarker}{}%
\end{pgfscope}%
\begin{pgfscope}%
\pgfsys@transformshift{1.273809in}{1.741344in}%
\pgfsys@useobject{currentmarker}{}%
\end{pgfscope}%
\begin{pgfscope}%
\pgfsys@transformshift{1.280156in}{1.742140in}%
\pgfsys@useobject{currentmarker}{}%
\end{pgfscope}%
\begin{pgfscope}%
\pgfsys@transformshift{1.287679in}{1.742052in}%
\pgfsys@useobject{currentmarker}{}%
\end{pgfscope}%
\begin{pgfscope}%
\pgfsys@transformshift{1.297025in}{1.743247in}%
\pgfsys@useobject{currentmarker}{}%
\end{pgfscope}%
\begin{pgfscope}%
\pgfsys@transformshift{1.306922in}{1.742161in}%
\pgfsys@useobject{currentmarker}{}%
\end{pgfscope}%
\begin{pgfscope}%
\pgfsys@transformshift{1.317320in}{1.742768in}%
\pgfsys@useobject{currentmarker}{}%
\end{pgfscope}%
\begin{pgfscope}%
\pgfsys@transformshift{1.329650in}{1.744477in}%
\pgfsys@useobject{currentmarker}{}%
\end{pgfscope}%
\begin{pgfscope}%
\pgfsys@transformshift{1.336495in}{1.744626in}%
\pgfsys@useobject{currentmarker}{}%
\end{pgfscope}%
\begin{pgfscope}%
\pgfsys@transformshift{1.340261in}{1.744635in}%
\pgfsys@useobject{currentmarker}{}%
\end{pgfscope}%
\begin{pgfscope}%
\pgfsys@transformshift{1.345989in}{1.744863in}%
\pgfsys@useobject{currentmarker}{}%
\end{pgfscope}%
\begin{pgfscope}%
\pgfsys@transformshift{1.354234in}{1.746145in}%
\pgfsys@useobject{currentmarker}{}%
\end{pgfscope}%
\begin{pgfscope}%
\pgfsys@transformshift{1.363374in}{1.746307in}%
\pgfsys@useobject{currentmarker}{}%
\end{pgfscope}%
\begin{pgfscope}%
\pgfsys@transformshift{1.368403in}{1.746317in}%
\pgfsys@useobject{currentmarker}{}%
\end{pgfscope}%
\begin{pgfscope}%
\pgfsys@transformshift{1.375593in}{1.746706in}%
\pgfsys@useobject{currentmarker}{}%
\end{pgfscope}%
\begin{pgfscope}%
\pgfsys@transformshift{1.384032in}{1.747745in}%
\pgfsys@useobject{currentmarker}{}%
\end{pgfscope}%
\begin{pgfscope}%
\pgfsys@transformshift{1.393145in}{1.748604in}%
\pgfsys@useobject{currentmarker}{}%
\end{pgfscope}%
\begin{pgfscope}%
\pgfsys@transformshift{1.402926in}{1.747968in}%
\pgfsys@useobject{currentmarker}{}%
\end{pgfscope}%
\begin{pgfscope}%
\pgfsys@transformshift{1.408313in}{1.747738in}%
\pgfsys@useobject{currentmarker}{}%
\end{pgfscope}%
\begin{pgfscope}%
\pgfsys@transformshift{1.415732in}{1.748418in}%
\pgfsys@useobject{currentmarker}{}%
\end{pgfscope}%
\begin{pgfscope}%
\pgfsys@transformshift{1.425121in}{1.748319in}%
\pgfsys@useobject{currentmarker}{}%
\end{pgfscope}%
\begin{pgfscope}%
\pgfsys@transformshift{1.435341in}{1.748311in}%
\pgfsys@useobject{currentmarker}{}%
\end{pgfscope}%
\begin{pgfscope}%
\pgfsys@transformshift{1.446389in}{1.744098in}%
\pgfsys@useobject{currentmarker}{}%
\end{pgfscope}%
\begin{pgfscope}%
\pgfsys@transformshift{1.460547in}{1.746329in}%
\pgfsys@useobject{currentmarker}{}%
\end{pgfscope}%
\begin{pgfscope}%
\pgfsys@transformshift{1.468422in}{1.746668in}%
\pgfsys@useobject{currentmarker}{}%
\end{pgfscope}%
\begin{pgfscope}%
\pgfsys@transformshift{1.476811in}{1.746796in}%
\pgfsys@useobject{currentmarker}{}%
\end{pgfscope}%
\begin{pgfscope}%
\pgfsys@transformshift{1.486090in}{1.746304in}%
\pgfsys@useobject{currentmarker}{}%
\end{pgfscope}%
\begin{pgfscope}%
\pgfsys@transformshift{1.496601in}{1.749136in}%
\pgfsys@useobject{currentmarker}{}%
\end{pgfscope}%
\begin{pgfscope}%
\pgfsys@transformshift{1.502580in}{1.749463in}%
\pgfsys@useobject{currentmarker}{}%
\end{pgfscope}%
\begin{pgfscope}%
\pgfsys@transformshift{1.511510in}{1.750712in}%
\pgfsys@useobject{currentmarker}{}%
\end{pgfscope}%
\begin{pgfscope}%
\pgfsys@transformshift{1.522544in}{1.751475in}%
\pgfsys@useobject{currentmarker}{}%
\end{pgfscope}%
\begin{pgfscope}%
\pgfsys@transformshift{1.528508in}{1.752672in}%
\pgfsys@useobject{currentmarker}{}%
\end{pgfscope}%
\begin{pgfscope}%
\pgfsys@transformshift{1.535238in}{1.754310in}%
\pgfsys@useobject{currentmarker}{}%
\end{pgfscope}%
\begin{pgfscope}%
\pgfsys@transformshift{1.543862in}{1.755610in}%
\pgfsys@useobject{currentmarker}{}%
\end{pgfscope}%
\begin{pgfscope}%
\pgfsys@transformshift{1.554114in}{1.755192in}%
\pgfsys@useobject{currentmarker}{}%
\end{pgfscope}%
\begin{pgfscope}%
\pgfsys@transformshift{1.565263in}{1.755106in}%
\pgfsys@useobject{currentmarker}{}%
\end{pgfscope}%
\begin{pgfscope}%
\pgfsys@transformshift{1.571386in}{1.755448in}%
\pgfsys@useobject{currentmarker}{}%
\end{pgfscope}%
\begin{pgfscope}%
\pgfsys@transformshift{1.578115in}{1.756493in}%
\pgfsys@useobject{currentmarker}{}%
\end{pgfscope}%
\begin{pgfscope}%
\pgfsys@transformshift{1.585765in}{1.756342in}%
\pgfsys@useobject{currentmarker}{}%
\end{pgfscope}%
\begin{pgfscope}%
\pgfsys@transformshift{1.595139in}{1.756878in}%
\pgfsys@useobject{currentmarker}{}%
\end{pgfscope}%
\begin{pgfscope}%
\pgfsys@transformshift{1.600266in}{1.757496in}%
\pgfsys@useobject{currentmarker}{}%
\end{pgfscope}%
\begin{pgfscope}%
\pgfsys@transformshift{1.605781in}{1.758769in}%
\pgfsys@useobject{currentmarker}{}%
\end{pgfscope}%
\begin{pgfscope}%
\pgfsys@transformshift{1.613972in}{1.759211in}%
\pgfsys@useobject{currentmarker}{}%
\end{pgfscope}%
\begin{pgfscope}%
\pgfsys@transformshift{1.618457in}{1.759705in}%
\pgfsys@useobject{currentmarker}{}%
\end{pgfscope}%
\begin{pgfscope}%
\pgfsys@transformshift{1.624863in}{1.761466in}%
\pgfsys@useobject{currentmarker}{}%
\end{pgfscope}%
\begin{pgfscope}%
\pgfsys@transformshift{1.628475in}{1.762016in}%
\pgfsys@useobject{currentmarker}{}%
\end{pgfscope}%
\begin{pgfscope}%
\pgfsys@transformshift{1.630426in}{1.762502in}%
\pgfsys@useobject{currentmarker}{}%
\end{pgfscope}%
\begin{pgfscope}%
\pgfsys@transformshift{1.631531in}{1.762521in}%
\pgfsys@useobject{currentmarker}{}%
\end{pgfscope}%
\begin{pgfscope}%
\pgfsys@transformshift{1.633447in}{1.762499in}%
\pgfsys@useobject{currentmarker}{}%
\end{pgfscope}%
\begin{pgfscope}%
\pgfsys@transformshift{1.638440in}{1.763877in}%
\pgfsys@useobject{currentmarker}{}%
\end{pgfscope}%
\begin{pgfscope}%
\pgfsys@transformshift{1.641196in}{1.764594in}%
\pgfsys@useobject{currentmarker}{}%
\end{pgfscope}%
\begin{pgfscope}%
\pgfsys@transformshift{1.646507in}{1.764752in}%
\pgfsys@useobject{currentmarker}{}%
\end{pgfscope}%
\begin{pgfscope}%
\pgfsys@transformshift{1.649429in}{1.764817in}%
\pgfsys@useobject{currentmarker}{}%
\end{pgfscope}%
\begin{pgfscope}%
\pgfsys@transformshift{1.654427in}{1.764608in}%
\pgfsys@useobject{currentmarker}{}%
\end{pgfscope}%
\begin{pgfscope}%
\pgfsys@transformshift{1.660498in}{1.765934in}%
\pgfsys@useobject{currentmarker}{}%
\end{pgfscope}%
\begin{pgfscope}%
\pgfsys@transformshift{1.663849in}{1.766611in}%
\pgfsys@useobject{currentmarker}{}%
\end{pgfscope}%
\begin{pgfscope}%
\pgfsys@transformshift{1.665705in}{1.766907in}%
\pgfsys@useobject{currentmarker}{}%
\end{pgfscope}%
\begin{pgfscope}%
\pgfsys@transformshift{1.668443in}{1.766847in}%
\pgfsys@useobject{currentmarker}{}%
\end{pgfscope}%
\begin{pgfscope}%
\pgfsys@transformshift{1.673330in}{1.766894in}%
\pgfsys@useobject{currentmarker}{}%
\end{pgfscope}%
\begin{pgfscope}%
\pgfsys@transformshift{1.680236in}{1.768736in}%
\pgfsys@useobject{currentmarker}{}%
\end{pgfscope}%
\begin{pgfscope}%
\pgfsys@transformshift{1.688166in}{1.770650in}%
\pgfsys@useobject{currentmarker}{}%
\end{pgfscope}%
\begin{pgfscope}%
\pgfsys@transformshift{1.699391in}{1.771596in}%
\pgfsys@useobject{currentmarker}{}%
\end{pgfscope}%
\begin{pgfscope}%
\pgfsys@transformshift{1.711866in}{1.770704in}%
\pgfsys@useobject{currentmarker}{}%
\end{pgfscope}%
\begin{pgfscope}%
\pgfsys@transformshift{1.725583in}{1.769020in}%
\pgfsys@useobject{currentmarker}{}%
\end{pgfscope}%
\begin{pgfscope}%
\pgfsys@transformshift{1.733045in}{1.770468in}%
\pgfsys@useobject{currentmarker}{}%
\end{pgfscope}%
\begin{pgfscope}%
\pgfsys@transformshift{1.741156in}{1.772298in}%
\pgfsys@useobject{currentmarker}{}%
\end{pgfscope}%
\begin{pgfscope}%
\pgfsys@transformshift{1.745703in}{1.772780in}%
\pgfsys@useobject{currentmarker}{}%
\end{pgfscope}%
\begin{pgfscope}%
\pgfsys@transformshift{1.752276in}{1.772458in}%
\pgfsys@useobject{currentmarker}{}%
\end{pgfscope}%
\begin{pgfscope}%
\pgfsys@transformshift{1.762176in}{1.772576in}%
\pgfsys@useobject{currentmarker}{}%
\end{pgfscope}%
\begin{pgfscope}%
\pgfsys@transformshift{1.775073in}{1.773452in}%
\pgfsys@useobject{currentmarker}{}%
\end{pgfscope}%
\begin{pgfscope}%
\pgfsys@transformshift{1.789856in}{1.775293in}%
\pgfsys@useobject{currentmarker}{}%
\end{pgfscope}%
\begin{pgfscope}%
\pgfsys@transformshift{1.807360in}{1.776090in}%
\pgfsys@useobject{currentmarker}{}%
\end{pgfscope}%
\begin{pgfscope}%
\pgfsys@transformshift{1.825444in}{1.775924in}%
\pgfsys@useobject{currentmarker}{}%
\end{pgfscope}%
\begin{pgfscope}%
\pgfsys@transformshift{1.835333in}{1.774847in}%
\pgfsys@useobject{currentmarker}{}%
\end{pgfscope}%
\begin{pgfscope}%
\pgfsys@transformshift{1.847021in}{1.776560in}%
\pgfsys@useobject{currentmarker}{}%
\end{pgfscope}%
\begin{pgfscope}%
\pgfsys@transformshift{1.859556in}{1.777142in}%
\pgfsys@useobject{currentmarker}{}%
\end{pgfscope}%
\begin{pgfscope}%
\pgfsys@transformshift{1.866433in}{1.777727in}%
\pgfsys@useobject{currentmarker}{}%
\end{pgfscope}%
\begin{pgfscope}%
\pgfsys@transformshift{1.874046in}{1.778207in}%
\pgfsys@useobject{currentmarker}{}%
\end{pgfscope}%
\begin{pgfscope}%
\pgfsys@transformshift{1.883573in}{1.779584in}%
\pgfsys@useobject{currentmarker}{}%
\end{pgfscope}%
\begin{pgfscope}%
\pgfsys@transformshift{1.888822in}{1.780272in}%
\pgfsys@useobject{currentmarker}{}%
\end{pgfscope}%
\begin{pgfscope}%
\pgfsys@transformshift{1.894365in}{1.781884in}%
\pgfsys@useobject{currentmarker}{}%
\end{pgfscope}%
\begin{pgfscope}%
\pgfsys@transformshift{1.897420in}{1.782746in}%
\pgfsys@useobject{currentmarker}{}%
\end{pgfscope}%
\begin{pgfscope}%
\pgfsys@transformshift{1.899160in}{1.782894in}%
\pgfsys@useobject{currentmarker}{}%
\end{pgfscope}%
\begin{pgfscope}%
\pgfsys@transformshift{1.901883in}{1.783115in}%
\pgfsys@useobject{currentmarker}{}%
\end{pgfscope}%
\begin{pgfscope}%
\pgfsys@transformshift{1.906285in}{1.783269in}%
\pgfsys@useobject{currentmarker}{}%
\end{pgfscope}%
\begin{pgfscope}%
\pgfsys@transformshift{1.912338in}{1.785674in}%
\pgfsys@useobject{currentmarker}{}%
\end{pgfscope}%
\begin{pgfscope}%
\pgfsys@transformshift{1.919651in}{1.787949in}%
\pgfsys@useobject{currentmarker}{}%
\end{pgfscope}%
\begin{pgfscope}%
\pgfsys@transformshift{1.929298in}{1.789365in}%
\pgfsys@useobject{currentmarker}{}%
\end{pgfscope}%
\begin{pgfscope}%
\pgfsys@transformshift{1.940599in}{1.789929in}%
\pgfsys@useobject{currentmarker}{}%
\end{pgfscope}%
\begin{pgfscope}%
\pgfsys@transformshift{1.946822in}{1.789893in}%
\pgfsys@useobject{currentmarker}{}%
\end{pgfscope}%
\begin{pgfscope}%
\pgfsys@transformshift{1.954860in}{1.791551in}%
\pgfsys@useobject{currentmarker}{}%
\end{pgfscope}%
\begin{pgfscope}%
\pgfsys@transformshift{1.964038in}{1.790686in}%
\pgfsys@useobject{currentmarker}{}%
\end{pgfscope}%
\begin{pgfscope}%
\pgfsys@transformshift{1.974716in}{1.792046in}%
\pgfsys@useobject{currentmarker}{}%
\end{pgfscope}%
\begin{pgfscope}%
\pgfsys@transformshift{1.986536in}{1.791931in}%
\pgfsys@useobject{currentmarker}{}%
\end{pgfscope}%
\begin{pgfscope}%
\pgfsys@transformshift{1.999421in}{1.792172in}%
\pgfsys@useobject{currentmarker}{}%
\end{pgfscope}%
\begin{pgfscope}%
\pgfsys@transformshift{2.014151in}{1.793141in}%
\pgfsys@useobject{currentmarker}{}%
\end{pgfscope}%
\begin{pgfscope}%
\pgfsys@transformshift{2.029640in}{1.792043in}%
\pgfsys@useobject{currentmarker}{}%
\end{pgfscope}%
\begin{pgfscope}%
\pgfsys@transformshift{2.037865in}{1.794341in}%
\pgfsys@useobject{currentmarker}{}%
\end{pgfscope}%
\begin{pgfscope}%
\pgfsys@transformshift{2.042558in}{1.794152in}%
\pgfsys@useobject{currentmarker}{}%
\end{pgfscope}%
\begin{pgfscope}%
\pgfsys@transformshift{2.048195in}{1.795074in}%
\pgfsys@useobject{currentmarker}{}%
\end{pgfscope}%
\begin{pgfscope}%
\pgfsys@transformshift{2.051337in}{1.795078in}%
\pgfsys@useobject{currentmarker}{}%
\end{pgfscope}%
\begin{pgfscope}%
\pgfsys@transformshift{2.055329in}{1.795564in}%
\pgfsys@useobject{currentmarker}{}%
\end{pgfscope}%
\begin{pgfscope}%
\pgfsys@transformshift{2.057540in}{1.795628in}%
\pgfsys@useobject{currentmarker}{}%
\end{pgfscope}%
\begin{pgfscope}%
\pgfsys@transformshift{2.061377in}{1.796298in}%
\pgfsys@useobject{currentmarker}{}%
\end{pgfscope}%
\begin{pgfscope}%
\pgfsys@transformshift{2.063517in}{1.796195in}%
\pgfsys@useobject{currentmarker}{}%
\end{pgfscope}%
\begin{pgfscope}%
\pgfsys@transformshift{2.066712in}{1.796454in}%
\pgfsys@useobject{currentmarker}{}%
\end{pgfscope}%
\begin{pgfscope}%
\pgfsys@transformshift{2.068418in}{1.796009in}%
\pgfsys@useobject{currentmarker}{}%
\end{pgfscope}%
\begin{pgfscope}%
\pgfsys@transformshift{2.071241in}{1.795342in}%
\pgfsys@useobject{currentmarker}{}%
\end{pgfscope}%
\begin{pgfscope}%
\pgfsys@transformshift{2.074913in}{1.794143in}%
\pgfsys@useobject{currentmarker}{}%
\end{pgfscope}%
\begin{pgfscope}%
\pgfsys@transformshift{2.078561in}{1.791282in}%
\pgfsys@useobject{currentmarker}{}%
\end{pgfscope}%
\begin{pgfscope}%
\pgfsys@transformshift{2.081423in}{1.784795in}%
\pgfsys@useobject{currentmarker}{}%
\end{pgfscope}%
\begin{pgfscope}%
\pgfsys@transformshift{2.085270in}{1.777631in}%
\pgfsys@useobject{currentmarker}{}%
\end{pgfscope}%
\begin{pgfscope}%
\pgfsys@transformshift{2.085784in}{1.768299in}%
\pgfsys@useobject{currentmarker}{}%
\end{pgfscope}%
\begin{pgfscope}%
\pgfsys@transformshift{2.088544in}{1.758295in}%
\pgfsys@useobject{currentmarker}{}%
\end{pgfscope}%
\begin{pgfscope}%
\pgfsys@transformshift{2.091229in}{1.746974in}%
\pgfsys@useobject{currentmarker}{}%
\end{pgfscope}%
\begin{pgfscope}%
\pgfsys@transformshift{2.093036in}{1.734571in}%
\pgfsys@useobject{currentmarker}{}%
\end{pgfscope}%
\begin{pgfscope}%
\pgfsys@transformshift{2.095515in}{1.728139in}%
\pgfsys@useobject{currentmarker}{}%
\end{pgfscope}%
\begin{pgfscope}%
\pgfsys@transformshift{2.094625in}{1.724454in}%
\pgfsys@useobject{currentmarker}{}%
\end{pgfscope}%
\begin{pgfscope}%
\pgfsys@transformshift{2.095105in}{1.722425in}%
\pgfsys@useobject{currentmarker}{}%
\end{pgfscope}%
\begin{pgfscope}%
\pgfsys@transformshift{2.094622in}{1.719560in}%
\pgfsys@useobject{currentmarker}{}%
\end{pgfscope}%
\begin{pgfscope}%
\pgfsys@transformshift{2.094061in}{1.716144in}%
\pgfsys@useobject{currentmarker}{}%
\end{pgfscope}%
\begin{pgfscope}%
\pgfsys@transformshift{2.094995in}{1.710676in}%
\pgfsys@useobject{currentmarker}{}%
\end{pgfscope}%
\begin{pgfscope}%
\pgfsys@transformshift{2.095759in}{1.704551in}%
\pgfsys@useobject{currentmarker}{}%
\end{pgfscope}%
\begin{pgfscope}%
\pgfsys@transformshift{2.094207in}{1.696702in}%
\pgfsys@useobject{currentmarker}{}%
\end{pgfscope}%
\begin{pgfscope}%
\pgfsys@transformshift{2.093967in}{1.692308in}%
\pgfsys@useobject{currentmarker}{}%
\end{pgfscope}%
\begin{pgfscope}%
\pgfsys@transformshift{2.094548in}{1.684835in}%
\pgfsys@useobject{currentmarker}{}%
\end{pgfscope}%
\begin{pgfscope}%
\pgfsys@transformshift{2.095126in}{1.680754in}%
\pgfsys@useobject{currentmarker}{}%
\end{pgfscope}%
\begin{pgfscope}%
\pgfsys@transformshift{2.094237in}{1.675148in}%
\pgfsys@useobject{currentmarker}{}%
\end{pgfscope}%
\begin{pgfscope}%
\pgfsys@transformshift{2.093896in}{1.672045in}%
\pgfsys@useobject{currentmarker}{}%
\end{pgfscope}%
\begin{pgfscope}%
\pgfsys@transformshift{2.093562in}{1.670361in}%
\pgfsys@useobject{currentmarker}{}%
\end{pgfscope}%
\begin{pgfscope}%
\pgfsys@transformshift{2.093914in}{1.668214in}%
\pgfsys@useobject{currentmarker}{}%
\end{pgfscope}%
\begin{pgfscope}%
\pgfsys@transformshift{2.093700in}{1.667037in}%
\pgfsys@useobject{currentmarker}{}%
\end{pgfscope}%
\begin{pgfscope}%
\pgfsys@transformshift{2.093655in}{1.666380in}%
\pgfsys@useobject{currentmarker}{}%
\end{pgfscope}%
\begin{pgfscope}%
\pgfsys@transformshift{2.093598in}{1.666023in}%
\pgfsys@useobject{currentmarker}{}%
\end{pgfscope}%
\begin{pgfscope}%
\pgfsys@transformshift{2.093768in}{1.665214in}%
\pgfsys@useobject{currentmarker}{}%
\end{pgfscope}%
\begin{pgfscope}%
\pgfsys@transformshift{2.093745in}{1.664760in}%
\pgfsys@useobject{currentmarker}{}%
\end{pgfscope}%
\begin{pgfscope}%
\pgfsys@transformshift{2.093594in}{1.663394in}%
\pgfsys@useobject{currentmarker}{}%
\end{pgfscope}%
\begin{pgfscope}%
\pgfsys@transformshift{2.093562in}{1.662638in}%
\pgfsys@useobject{currentmarker}{}%
\end{pgfscope}%
\begin{pgfscope}%
\pgfsys@transformshift{2.093773in}{1.659568in}%
\pgfsys@useobject{currentmarker}{}%
\end{pgfscope}%
\begin{pgfscope}%
\pgfsys@transformshift{2.094620in}{1.656083in}%
\pgfsys@useobject{currentmarker}{}%
\end{pgfscope}%
\begin{pgfscope}%
\pgfsys@transformshift{2.093654in}{1.649932in}%
\pgfsys@useobject{currentmarker}{}%
\end{pgfscope}%
\begin{pgfscope}%
\pgfsys@transformshift{2.093042in}{1.642358in}%
\pgfsys@useobject{currentmarker}{}%
\end{pgfscope}%
\begin{pgfscope}%
\pgfsys@transformshift{2.091828in}{1.633539in}%
\pgfsys@useobject{currentmarker}{}%
\end{pgfscope}%
\begin{pgfscope}%
\pgfsys@transformshift{2.093146in}{1.628823in}%
\pgfsys@useobject{currentmarker}{}%
\end{pgfscope}%
\begin{pgfscope}%
\pgfsys@transformshift{2.092288in}{1.623391in}%
\pgfsys@useobject{currentmarker}{}%
\end{pgfscope}%
\begin{pgfscope}%
\pgfsys@transformshift{2.091420in}{1.617400in}%
\pgfsys@useobject{currentmarker}{}%
\end{pgfscope}%
\begin{pgfscope}%
\pgfsys@transformshift{2.090990in}{1.610180in}%
\pgfsys@useobject{currentmarker}{}%
\end{pgfscope}%
\begin{pgfscope}%
\pgfsys@transformshift{2.092159in}{1.606378in}%
\pgfsys@useobject{currentmarker}{}%
\end{pgfscope}%
\begin{pgfscope}%
\pgfsys@transformshift{2.091997in}{1.601634in}%
\pgfsys@useobject{currentmarker}{}%
\end{pgfscope}%
\begin{pgfscope}%
\pgfsys@transformshift{2.091580in}{1.596441in}%
\pgfsys@useobject{currentmarker}{}%
\end{pgfscope}%
\begin{pgfscope}%
\pgfsys@transformshift{2.091066in}{1.593622in}%
\pgfsys@useobject{currentmarker}{}%
\end{pgfscope}%
\begin{pgfscope}%
\pgfsys@transformshift{2.092270in}{1.588201in}%
\pgfsys@useobject{currentmarker}{}%
\end{pgfscope}%
\begin{pgfscope}%
\pgfsys@transformshift{2.092499in}{1.585155in}%
\pgfsys@useobject{currentmarker}{}%
\end{pgfscope}%
\begin{pgfscope}%
\pgfsys@transformshift{2.092758in}{1.583495in}%
\pgfsys@useobject{currentmarker}{}%
\end{pgfscope}%
\begin{pgfscope}%
\pgfsys@transformshift{2.092582in}{1.582588in}%
\pgfsys@useobject{currentmarker}{}%
\end{pgfscope}%
\begin{pgfscope}%
\pgfsys@transformshift{2.092587in}{1.582080in}%
\pgfsys@useobject{currentmarker}{}%
\end{pgfscope}%
\begin{pgfscope}%
\pgfsys@transformshift{2.092663in}{1.581811in}%
\pgfsys@useobject{currentmarker}{}%
\end{pgfscope}%
\begin{pgfscope}%
\pgfsys@transformshift{2.092676in}{1.581658in}%
\pgfsys@useobject{currentmarker}{}%
\end{pgfscope}%
\begin{pgfscope}%
\pgfsys@transformshift{2.092743in}{1.580431in}%
\pgfsys@useobject{currentmarker}{}%
\end{pgfscope}%
\begin{pgfscope}%
\pgfsys@transformshift{2.092664in}{1.579760in}%
\pgfsys@useobject{currentmarker}{}%
\end{pgfscope}%
\begin{pgfscope}%
\pgfsys@transformshift{2.093715in}{1.576057in}%
\pgfsys@useobject{currentmarker}{}%
\end{pgfscope}%
\begin{pgfscope}%
\pgfsys@transformshift{2.094336in}{1.574032in}%
\pgfsys@useobject{currentmarker}{}%
\end{pgfscope}%
\begin{pgfscope}%
\pgfsys@transformshift{2.094383in}{1.570429in}%
\pgfsys@useobject{currentmarker}{}%
\end{pgfscope}%
\begin{pgfscope}%
\pgfsys@transformshift{2.094454in}{1.568448in}%
\pgfsys@useobject{currentmarker}{}%
\end{pgfscope}%
\begin{pgfscope}%
\pgfsys@transformshift{2.094398in}{1.565321in}%
\pgfsys@useobject{currentmarker}{}%
\end{pgfscope}%
\begin{pgfscope}%
\pgfsys@transformshift{2.095278in}{1.561729in}%
\pgfsys@useobject{currentmarker}{}%
\end{pgfscope}%
\begin{pgfscope}%
\pgfsys@transformshift{2.096266in}{1.557446in}%
\pgfsys@useobject{currentmarker}{}%
\end{pgfscope}%
\begin{pgfscope}%
\pgfsys@transformshift{2.095628in}{1.552458in}%
\pgfsys@useobject{currentmarker}{}%
\end{pgfscope}%
\begin{pgfscope}%
\pgfsys@transformshift{2.095604in}{1.549692in}%
\pgfsys@useobject{currentmarker}{}%
\end{pgfscope}%
\begin{pgfscope}%
\pgfsys@transformshift{2.096372in}{1.545715in}%
\pgfsys@useobject{currentmarker}{}%
\end{pgfscope}%
\begin{pgfscope}%
\pgfsys@transformshift{2.096926in}{1.541090in}%
\pgfsys@useobject{currentmarker}{}%
\end{pgfscope}%
\begin{pgfscope}%
\pgfsys@transformshift{2.097451in}{1.534563in}%
\pgfsys@useobject{currentmarker}{}%
\end{pgfscope}%
\begin{pgfscope}%
\pgfsys@transformshift{2.097121in}{1.530976in}%
\pgfsys@useobject{currentmarker}{}%
\end{pgfscope}%
\begin{pgfscope}%
\pgfsys@transformshift{2.098367in}{1.525713in}%
\pgfsys@useobject{currentmarker}{}%
\end{pgfscope}%
\begin{pgfscope}%
\pgfsys@transformshift{2.099603in}{1.519701in}%
\pgfsys@useobject{currentmarker}{}%
\end{pgfscope}%
\begin{pgfscope}%
\pgfsys@transformshift{2.100885in}{1.512541in}%
\pgfsys@useobject{currentmarker}{}%
\end{pgfscope}%
\begin{pgfscope}%
\pgfsys@transformshift{2.100646in}{1.508547in}%
\pgfsys@useobject{currentmarker}{}%
\end{pgfscope}%
\begin{pgfscope}%
\pgfsys@transformshift{2.101460in}{1.503612in}%
\pgfsys@useobject{currentmarker}{}%
\end{pgfscope}%
\begin{pgfscope}%
\pgfsys@transformshift{2.102111in}{1.500939in}%
\pgfsys@useobject{currentmarker}{}%
\end{pgfscope}%
\begin{pgfscope}%
\pgfsys@transformshift{2.102279in}{1.499435in}%
\pgfsys@useobject{currentmarker}{}%
\end{pgfscope}%
\begin{pgfscope}%
\pgfsys@transformshift{2.102195in}{1.498607in}%
\pgfsys@useobject{currentmarker}{}%
\end{pgfscope}%
\begin{pgfscope}%
\pgfsys@transformshift{2.102167in}{1.498150in}%
\pgfsys@useobject{currentmarker}{}%
\end{pgfscope}%
\begin{pgfscope}%
\pgfsys@transformshift{2.102594in}{1.496291in}%
\pgfsys@useobject{currentmarker}{}%
\end{pgfscope}%
\begin{pgfscope}%
\pgfsys@transformshift{2.102890in}{1.493889in}%
\pgfsys@useobject{currentmarker}{}%
\end{pgfscope}%
\begin{pgfscope}%
\pgfsys@transformshift{2.102780in}{1.489904in}%
\pgfsys@useobject{currentmarker}{}%
\end{pgfscope}%
\begin{pgfscope}%
\pgfsys@transformshift{2.102440in}{1.485349in}%
\pgfsys@useobject{currentmarker}{}%
\end{pgfscope}%
\begin{pgfscope}%
\pgfsys@transformshift{2.104250in}{1.478347in}%
\pgfsys@useobject{currentmarker}{}%
\end{pgfscope}%
\begin{pgfscope}%
\pgfsys@transformshift{2.105975in}{1.470788in}%
\pgfsys@useobject{currentmarker}{}%
\end{pgfscope}%
\begin{pgfscope}%
\pgfsys@transformshift{2.106690in}{1.461474in}%
\pgfsys@useobject{currentmarker}{}%
\end{pgfscope}%
\begin{pgfscope}%
\pgfsys@transformshift{2.106380in}{1.456346in}%
\pgfsys@useobject{currentmarker}{}%
\end{pgfscope}%
\begin{pgfscope}%
\pgfsys@transformshift{2.106179in}{1.448854in}%
\pgfsys@useobject{currentmarker}{}%
\end{pgfscope}%
\begin{pgfscope}%
\pgfsys@transformshift{2.107101in}{1.440341in}%
\pgfsys@useobject{currentmarker}{}%
\end{pgfscope}%
\begin{pgfscope}%
\pgfsys@transformshift{2.110988in}{1.431556in}%
\pgfsys@useobject{currentmarker}{}%
\end{pgfscope}%
\begin{pgfscope}%
\pgfsys@transformshift{2.111105in}{1.420108in}%
\pgfsys@useobject{currentmarker}{}%
\end{pgfscope}%
\begin{pgfscope}%
\pgfsys@transformshift{2.111932in}{1.407934in}%
\pgfsys@useobject{currentmarker}{}%
\end{pgfscope}%
\begin{pgfscope}%
\pgfsys@transformshift{2.111934in}{1.401223in}%
\pgfsys@useobject{currentmarker}{}%
\end{pgfscope}%
\begin{pgfscope}%
\pgfsys@transformshift{2.112974in}{1.397681in}%
\pgfsys@useobject{currentmarker}{}%
\end{pgfscope}%
\begin{pgfscope}%
\pgfsys@transformshift{2.113335in}{1.395684in}%
\pgfsys@useobject{currentmarker}{}%
\end{pgfscope}%
\begin{pgfscope}%
\pgfsys@transformshift{2.113248in}{1.393032in}%
\pgfsys@useobject{currentmarker}{}%
\end{pgfscope}%
\begin{pgfscope}%
\pgfsys@transformshift{2.113184in}{1.391573in}%
\pgfsys@useobject{currentmarker}{}%
\end{pgfscope}%
\begin{pgfscope}%
\pgfsys@transformshift{2.114142in}{1.387513in}%
\pgfsys@useobject{currentmarker}{}%
\end{pgfscope}%
\begin{pgfscope}%
\pgfsys@transformshift{2.115146in}{1.382926in}%
\pgfsys@useobject{currentmarker}{}%
\end{pgfscope}%
\begin{pgfscope}%
\pgfsys@transformshift{2.115603in}{1.380385in}%
\pgfsys@useobject{currentmarker}{}%
\end{pgfscope}%
\begin{pgfscope}%
\pgfsys@transformshift{2.115411in}{1.378978in}%
\pgfsys@useobject{currentmarker}{}%
\end{pgfscope}%
\begin{pgfscope}%
\pgfsys@transformshift{2.115491in}{1.377069in}%
\pgfsys@useobject{currentmarker}{}%
\end{pgfscope}%
\begin{pgfscope}%
\pgfsys@transformshift{2.116117in}{1.374267in}%
\pgfsys@useobject{currentmarker}{}%
\end{pgfscope}%
\begin{pgfscope}%
\pgfsys@transformshift{2.116211in}{1.370630in}%
\pgfsys@useobject{currentmarker}{}%
\end{pgfscope}%
\begin{pgfscope}%
\pgfsys@transformshift{2.115997in}{1.366374in}%
\pgfsys@useobject{currentmarker}{}%
\end{pgfscope}%
\begin{pgfscope}%
\pgfsys@transformshift{2.115889in}{1.364032in}%
\pgfsys@useobject{currentmarker}{}%
\end{pgfscope}%
\begin{pgfscope}%
\pgfsys@transformshift{2.117136in}{1.358910in}%
\pgfsys@useobject{currentmarker}{}%
\end{pgfscope}%
\begin{pgfscope}%
\pgfsys@transformshift{2.119273in}{1.353386in}%
\pgfsys@useobject{currentmarker}{}%
\end{pgfscope}%
\begin{pgfscope}%
\pgfsys@transformshift{2.119172in}{1.345267in}%
\pgfsys@useobject{currentmarker}{}%
\end{pgfscope}%
\begin{pgfscope}%
\pgfsys@transformshift{2.119561in}{1.340818in}%
\pgfsys@useobject{currentmarker}{}%
\end{pgfscope}%
\begin{pgfscope}%
\pgfsys@transformshift{2.119329in}{1.335597in}%
\pgfsys@useobject{currentmarker}{}%
\end{pgfscope}%
\begin{pgfscope}%
\pgfsys@transformshift{2.121869in}{1.329489in}%
\pgfsys@useobject{currentmarker}{}%
\end{pgfscope}%
\begin{pgfscope}%
\pgfsys@transformshift{2.121574in}{1.322314in}%
\pgfsys@useobject{currentmarker}{}%
\end{pgfscope}%
\begin{pgfscope}%
\pgfsys@transformshift{2.121375in}{1.318370in}%
\pgfsys@useobject{currentmarker}{}%
\end{pgfscope}%
\begin{pgfscope}%
\pgfsys@transformshift{2.121039in}{1.316224in}%
\pgfsys@useobject{currentmarker}{}%
\end{pgfscope}%
\begin{pgfscope}%
\pgfsys@transformshift{2.122350in}{1.312448in}%
\pgfsys@useobject{currentmarker}{}%
\end{pgfscope}%
\begin{pgfscope}%
\pgfsys@transformshift{2.122391in}{1.310251in}%
\pgfsys@useobject{currentmarker}{}%
\end{pgfscope}%
\begin{pgfscope}%
\pgfsys@transformshift{2.122732in}{1.306753in}%
\pgfsys@useobject{currentmarker}{}%
\end{pgfscope}%
\begin{pgfscope}%
\pgfsys@transformshift{2.122741in}{1.304820in}%
\pgfsys@useobject{currentmarker}{}%
\end{pgfscope}%
\begin{pgfscope}%
\pgfsys@transformshift{2.124037in}{1.301942in}%
\pgfsys@useobject{currentmarker}{}%
\end{pgfscope}%
\begin{pgfscope}%
\pgfsys@transformshift{2.124435in}{1.300252in}%
\pgfsys@useobject{currentmarker}{}%
\end{pgfscope}%
\begin{pgfscope}%
\pgfsys@transformshift{2.125142in}{1.297783in}%
\pgfsys@useobject{currentmarker}{}%
\end{pgfscope}%
\begin{pgfscope}%
\pgfsys@transformshift{2.125190in}{1.296371in}%
\pgfsys@useobject{currentmarker}{}%
\end{pgfscope}%
\begin{pgfscope}%
\pgfsys@transformshift{2.126183in}{1.293603in}%
\pgfsys@useobject{currentmarker}{}%
\end{pgfscope}%
\begin{pgfscope}%
\pgfsys@transformshift{2.127566in}{1.290310in}%
\pgfsys@useobject{currentmarker}{}%
\end{pgfscope}%
\begin{pgfscope}%
\pgfsys@transformshift{2.127901in}{1.284196in}%
\pgfsys@useobject{currentmarker}{}%
\end{pgfscope}%
\begin{pgfscope}%
\pgfsys@transformshift{2.128040in}{1.280831in}%
\pgfsys@useobject{currentmarker}{}%
\end{pgfscope}%
\begin{pgfscope}%
\pgfsys@transformshift{2.129216in}{1.275626in}%
\pgfsys@useobject{currentmarker}{}%
\end{pgfscope}%
\begin{pgfscope}%
\pgfsys@transformshift{2.131156in}{1.269940in}%
\pgfsys@useobject{currentmarker}{}%
\end{pgfscope}%
\begin{pgfscope}%
\pgfsys@transformshift{2.131863in}{1.266712in}%
\pgfsys@useobject{currentmarker}{}%
\end{pgfscope}%
\begin{pgfscope}%
\pgfsys@transformshift{2.131969in}{1.264897in}%
\pgfsys@useobject{currentmarker}{}%
\end{pgfscope}%
\begin{pgfscope}%
\pgfsys@transformshift{2.132011in}{1.263899in}%
\pgfsys@useobject{currentmarker}{}%
\end{pgfscope}%
\begin{pgfscope}%
\pgfsys@transformshift{2.132496in}{1.262482in}%
\pgfsys@useobject{currentmarker}{}%
\end{pgfscope}%
\begin{pgfscope}%
\pgfsys@transformshift{2.132684in}{1.261680in}%
\pgfsys@useobject{currentmarker}{}%
\end{pgfscope}%
\begin{pgfscope}%
\pgfsys@transformshift{2.132782in}{1.261238in}%
\pgfsys@useobject{currentmarker}{}%
\end{pgfscope}%
\begin{pgfscope}%
\pgfsys@transformshift{2.132778in}{1.260318in}%
\pgfsys@useobject{currentmarker}{}%
\end{pgfscope}%
\begin{pgfscope}%
\pgfsys@transformshift{2.132884in}{1.259823in}%
\pgfsys@useobject{currentmarker}{}%
\end{pgfscope}%
\begin{pgfscope}%
\pgfsys@transformshift{2.132967in}{1.259557in}%
\pgfsys@useobject{currentmarker}{}%
\end{pgfscope}%
\begin{pgfscope}%
\pgfsys@transformshift{2.133334in}{1.258493in}%
\pgfsys@useobject{currentmarker}{}%
\end{pgfscope}%
\begin{pgfscope}%
\pgfsys@transformshift{2.133374in}{1.256905in}%
\pgfsys@useobject{currentmarker}{}%
\end{pgfscope}%
\begin{pgfscope}%
\pgfsys@transformshift{2.133433in}{1.256033in}%
\pgfsys@useobject{currentmarker}{}%
\end{pgfscope}%
\begin{pgfscope}%
\pgfsys@transformshift{2.134091in}{1.253566in}%
\pgfsys@useobject{currentmarker}{}%
\end{pgfscope}%
\begin{pgfscope}%
\pgfsys@transformshift{2.135218in}{1.250339in}%
\pgfsys@useobject{currentmarker}{}%
\end{pgfscope}%
\begin{pgfscope}%
\pgfsys@transformshift{2.135728in}{1.245956in}%
\pgfsys@useobject{currentmarker}{}%
\end{pgfscope}%
\begin{pgfscope}%
\pgfsys@transformshift{2.135673in}{1.243529in}%
\pgfsys@useobject{currentmarker}{}%
\end{pgfscope}%
\begin{pgfscope}%
\pgfsys@transformshift{2.135752in}{1.242197in}%
\pgfsys@useobject{currentmarker}{}%
\end{pgfscope}%
\begin{pgfscope}%
\pgfsys@transformshift{2.135934in}{1.241486in}%
\pgfsys@useobject{currentmarker}{}%
\end{pgfscope}%
\begin{pgfscope}%
\pgfsys@transformshift{2.136024in}{1.241092in}%
\pgfsys@useobject{currentmarker}{}%
\end{pgfscope}%
\begin{pgfscope}%
\pgfsys@transformshift{2.136267in}{1.239586in}%
\pgfsys@useobject{currentmarker}{}%
\end{pgfscope}%
\begin{pgfscope}%
\pgfsys@transformshift{2.136208in}{1.238748in}%
\pgfsys@useobject{currentmarker}{}%
\end{pgfscope}%
\begin{pgfscope}%
\pgfsys@transformshift{2.136580in}{1.236512in}%
\pgfsys@useobject{currentmarker}{}%
\end{pgfscope}%
\begin{pgfscope}%
\pgfsys@transformshift{2.136815in}{1.235287in}%
\pgfsys@useobject{currentmarker}{}%
\end{pgfscope}%
\begin{pgfscope}%
\pgfsys@transformshift{2.137021in}{1.234633in}%
\pgfsys@useobject{currentmarker}{}%
\end{pgfscope}%
\begin{pgfscope}%
\pgfsys@transformshift{2.137025in}{1.234256in}%
\pgfsys@useobject{currentmarker}{}%
\end{pgfscope}%
\begin{pgfscope}%
\pgfsys@transformshift{2.137039in}{1.234049in}%
\pgfsys@useobject{currentmarker}{}%
\end{pgfscope}%
\begin{pgfscope}%
\pgfsys@transformshift{2.137225in}{1.233081in}%
\pgfsys@useobject{currentmarker}{}%
\end{pgfscope}%
\begin{pgfscope}%
\pgfsys@transformshift{2.137407in}{1.232570in}%
\pgfsys@useobject{currentmarker}{}%
\end{pgfscope}%
\begin{pgfscope}%
\pgfsys@transformshift{2.137709in}{1.231062in}%
\pgfsys@useobject{currentmarker}{}%
\end{pgfscope}%
\begin{pgfscope}%
\pgfsys@transformshift{2.137764in}{1.228604in}%
\pgfsys@useobject{currentmarker}{}%
\end{pgfscope}%
\begin{pgfscope}%
\pgfsys@transformshift{2.137574in}{1.225515in}%
\pgfsys@useobject{currentmarker}{}%
\end{pgfscope}%
\begin{pgfscope}%
\pgfsys@transformshift{2.138111in}{1.223900in}%
\pgfsys@useobject{currentmarker}{}%
\end{pgfscope}%
\begin{pgfscope}%
\pgfsys@transformshift{2.138597in}{1.221300in}%
\pgfsys@useobject{currentmarker}{}%
\end{pgfscope}%
\begin{pgfscope}%
\pgfsys@transformshift{2.139399in}{1.217295in}%
\pgfsys@useobject{currentmarker}{}%
\end{pgfscope}%
\begin{pgfscope}%
\pgfsys@transformshift{2.138879in}{1.212452in}%
\pgfsys@useobject{currentmarker}{}%
\end{pgfscope}%
\begin{pgfscope}%
\pgfsys@transformshift{2.139957in}{1.206976in}%
\pgfsys@useobject{currentmarker}{}%
\end{pgfscope}%
\begin{pgfscope}%
\pgfsys@transformshift{2.141855in}{1.200335in}%
\pgfsys@useobject{currentmarker}{}%
\end{pgfscope}%
\begin{pgfscope}%
\pgfsys@transformshift{2.143109in}{1.192812in}%
\pgfsys@useobject{currentmarker}{}%
\end{pgfscope}%
\begin{pgfscope}%
\pgfsys@transformshift{2.143348in}{1.188624in}%
\pgfsys@useobject{currentmarker}{}%
\end{pgfscope}%
\begin{pgfscope}%
\pgfsys@transformshift{2.142774in}{1.186389in}%
\pgfsys@useobject{currentmarker}{}%
\end{pgfscope}%
\begin{pgfscope}%
\pgfsys@transformshift{2.142606in}{1.183552in}%
\pgfsys@useobject{currentmarker}{}%
\end{pgfscope}%
\begin{pgfscope}%
\pgfsys@transformshift{2.142658in}{1.181989in}%
\pgfsys@useobject{currentmarker}{}%
\end{pgfscope}%
\begin{pgfscope}%
\pgfsys@transformshift{2.142758in}{1.181135in}%
\pgfsys@useobject{currentmarker}{}%
\end{pgfscope}%
\begin{pgfscope}%
\pgfsys@transformshift{2.142459in}{1.179794in}%
\pgfsys@useobject{currentmarker}{}%
\end{pgfscope}%
\begin{pgfscope}%
\pgfsys@transformshift{2.142451in}{1.179039in}%
\pgfsys@useobject{currentmarker}{}%
\end{pgfscope}%
\begin{pgfscope}%
\pgfsys@transformshift{2.142622in}{1.177805in}%
\pgfsys@useobject{currentmarker}{}%
\end{pgfscope}%
\begin{pgfscope}%
\pgfsys@transformshift{2.143125in}{1.176001in}%
\pgfsys@useobject{currentmarker}{}%
\end{pgfscope}%
\begin{pgfscope}%
\pgfsys@transformshift{2.142628in}{1.170989in}%
\pgfsys@useobject{currentmarker}{}%
\end{pgfscope}%
\begin{pgfscope}%
\pgfsys@transformshift{2.142040in}{1.164364in}%
\pgfsys@useobject{currentmarker}{}%
\end{pgfscope}%
\begin{pgfscope}%
\pgfsys@transformshift{2.141551in}{1.160738in}%
\pgfsys@useobject{currentmarker}{}%
\end{pgfscope}%
\begin{pgfscope}%
\pgfsys@transformshift{2.142694in}{1.155914in}%
\pgfsys@useobject{currentmarker}{}%
\end{pgfscope}%
\begin{pgfscope}%
\pgfsys@transformshift{2.143459in}{1.153297in}%
\pgfsys@useobject{currentmarker}{}%
\end{pgfscope}%
\begin{pgfscope}%
\pgfsys@transformshift{2.143338in}{1.148478in}%
\pgfsys@useobject{currentmarker}{}%
\end{pgfscope}%
\begin{pgfscope}%
\pgfsys@transformshift{2.143159in}{1.142746in}%
\pgfsys@useobject{currentmarker}{}%
\end{pgfscope}%
\begin{pgfscope}%
\pgfsys@transformshift{2.142055in}{1.136636in}%
\pgfsys@useobject{currentmarker}{}%
\end{pgfscope}%
\begin{pgfscope}%
\pgfsys@transformshift{2.143887in}{1.128397in}%
\pgfsys@useobject{currentmarker}{}%
\end{pgfscope}%
\begin{pgfscope}%
\pgfsys@transformshift{2.145661in}{1.118603in}%
\pgfsys@useobject{currentmarker}{}%
\end{pgfscope}%
\begin{pgfscope}%
\pgfsys@transformshift{2.146788in}{1.108142in}%
\pgfsys@useobject{currentmarker}{}%
\end{pgfscope}%
\begin{pgfscope}%
\pgfsys@transformshift{2.145799in}{1.102440in}%
\pgfsys@useobject{currentmarker}{}%
\end{pgfscope}%
\begin{pgfscope}%
\pgfsys@transformshift{2.145538in}{1.099268in}%
\pgfsys@useobject{currentmarker}{}%
\end{pgfscope}%
\begin{pgfscope}%
\pgfsys@transformshift{2.146035in}{1.094829in}%
\pgfsys@useobject{currentmarker}{}%
\end{pgfscope}%
\begin{pgfscope}%
\pgfsys@transformshift{2.147769in}{1.090148in}%
\pgfsys@useobject{currentmarker}{}%
\end{pgfscope}%
\begin{pgfscope}%
\pgfsys@transformshift{2.146504in}{1.083011in}%
\pgfsys@useobject{currentmarker}{}%
\end{pgfscope}%
\begin{pgfscope}%
\pgfsys@transformshift{2.146502in}{1.074923in}%
\pgfsys@useobject{currentmarker}{}%
\end{pgfscope}%
\begin{pgfscope}%
\pgfsys@transformshift{2.145868in}{1.070521in}%
\pgfsys@useobject{currentmarker}{}%
\end{pgfscope}%
\begin{pgfscope}%
\pgfsys@transformshift{2.147072in}{1.065319in}%
\pgfsys@useobject{currentmarker}{}%
\end{pgfscope}%
\begin{pgfscope}%
\pgfsys@transformshift{2.147456in}{1.059520in}%
\pgfsys@useobject{currentmarker}{}%
\end{pgfscope}%
\begin{pgfscope}%
\pgfsys@transformshift{2.147239in}{1.052970in}%
\pgfsys@useobject{currentmarker}{}%
\end{pgfscope}%
\begin{pgfscope}%
\pgfsys@transformshift{2.146878in}{1.049383in}%
\pgfsys@useobject{currentmarker}{}%
\end{pgfscope}%
\begin{pgfscope}%
\pgfsys@transformshift{2.146451in}{1.047447in}%
\pgfsys@useobject{currentmarker}{}%
\end{pgfscope}%
\begin{pgfscope}%
\pgfsys@transformshift{2.147118in}{1.043946in}%
\pgfsys@useobject{currentmarker}{}%
\end{pgfscope}%
\begin{pgfscope}%
\pgfsys@transformshift{2.147307in}{1.041995in}%
\pgfsys@useobject{currentmarker}{}%
\end{pgfscope}%
\begin{pgfscope}%
\pgfsys@transformshift{2.146840in}{1.038018in}%
\pgfsys@useobject{currentmarker}{}%
\end{pgfscope}%
\begin{pgfscope}%
\pgfsys@transformshift{2.146711in}{1.035819in}%
\pgfsys@useobject{currentmarker}{}%
\end{pgfscope}%
\begin{pgfscope}%
\pgfsys@transformshift{2.146427in}{1.034641in}%
\pgfsys@useobject{currentmarker}{}%
\end{pgfscope}%
\begin{pgfscope}%
\pgfsys@transformshift{2.146996in}{1.031656in}%
\pgfsys@useobject{currentmarker}{}%
\end{pgfscope}%
\begin{pgfscope}%
\pgfsys@transformshift{2.147838in}{1.028047in}%
\pgfsys@useobject{currentmarker}{}%
\end{pgfscope}%
\begin{pgfscope}%
\pgfsys@transformshift{2.147118in}{1.023933in}%
\pgfsys@useobject{currentmarker}{}%
\end{pgfscope}%
\begin{pgfscope}%
\pgfsys@transformshift{2.147076in}{1.019042in}%
\pgfsys@useobject{currentmarker}{}%
\end{pgfscope}%
\begin{pgfscope}%
\pgfsys@transformshift{2.146997in}{1.013075in}%
\pgfsys@useobject{currentmarker}{}%
\end{pgfscope}%
\begin{pgfscope}%
\pgfsys@transformshift{2.149152in}{1.005900in}%
\pgfsys@useobject{currentmarker}{}%
\end{pgfscope}%
\begin{pgfscope}%
\pgfsys@transformshift{2.151398in}{0.998102in}%
\pgfsys@useobject{currentmarker}{}%
\end{pgfscope}%
\begin{pgfscope}%
\pgfsys@transformshift{2.150746in}{0.989286in}%
\pgfsys@useobject{currentmarker}{}%
\end{pgfscope}%
\begin{pgfscope}%
\pgfsys@transformshift{2.150556in}{0.984428in}%
\pgfsys@useobject{currentmarker}{}%
\end{pgfscope}%
\begin{pgfscope}%
\pgfsys@transformshift{2.150906in}{0.978375in}%
\pgfsys@useobject{currentmarker}{}%
\end{pgfscope}%
\begin{pgfscope}%
\pgfsys@transformshift{2.151414in}{0.971242in}%
\pgfsys@useobject{currentmarker}{}%
\end{pgfscope}%
\begin{pgfscope}%
\pgfsys@transformshift{2.152492in}{0.967460in}%
\pgfsys@useobject{currentmarker}{}%
\end{pgfscope}%
\begin{pgfscope}%
\pgfsys@transformshift{2.152282in}{0.965307in}%
\pgfsys@useobject{currentmarker}{}%
\end{pgfscope}%
\begin{pgfscope}%
\pgfsys@transformshift{2.152358in}{0.964120in}%
\pgfsys@useobject{currentmarker}{}%
\end{pgfscope}%
\begin{pgfscope}%
\pgfsys@transformshift{2.152174in}{0.963492in}%
\pgfsys@useobject{currentmarker}{}%
\end{pgfscope}%
\begin{pgfscope}%
\pgfsys@transformshift{2.152013in}{0.963170in}%
\pgfsys@useobject{currentmarker}{}%
\end{pgfscope}%
\begin{pgfscope}%
\pgfsys@transformshift{2.150841in}{0.962433in}%
\pgfsys@useobject{currentmarker}{}%
\end{pgfscope}%
\begin{pgfscope}%
\pgfsys@transformshift{2.148183in}{0.961783in}%
\pgfsys@useobject{currentmarker}{}%
\end{pgfscope}%
\begin{pgfscope}%
\pgfsys@transformshift{2.146687in}{0.961615in}%
\pgfsys@useobject{currentmarker}{}%
\end{pgfscope}%
\begin{pgfscope}%
\pgfsys@transformshift{2.143862in}{0.961192in}%
\pgfsys@useobject{currentmarker}{}%
\end{pgfscope}%
\begin{pgfscope}%
\pgfsys@transformshift{2.140506in}{0.961339in}%
\pgfsys@useobject{currentmarker}{}%
\end{pgfscope}%
\begin{pgfscope}%
\pgfsys@transformshift{2.136430in}{0.960977in}%
\pgfsys@useobject{currentmarker}{}%
\end{pgfscope}%
\begin{pgfscope}%
\pgfsys@transformshift{2.129977in}{0.960680in}%
\pgfsys@useobject{currentmarker}{}%
\end{pgfscope}%
\begin{pgfscope}%
\pgfsys@transformshift{2.123157in}{0.959504in}%
\pgfsys@useobject{currentmarker}{}%
\end{pgfscope}%
\begin{pgfscope}%
\pgfsys@transformshift{2.115236in}{0.960350in}%
\pgfsys@useobject{currentmarker}{}%
\end{pgfscope}%
\begin{pgfscope}%
\pgfsys@transformshift{2.110877in}{0.959914in}%
\pgfsys@useobject{currentmarker}{}%
\end{pgfscope}%
\begin{pgfscope}%
\pgfsys@transformshift{2.108482in}{0.960177in}%
\pgfsys@useobject{currentmarker}{}%
\end{pgfscope}%
\begin{pgfscope}%
\pgfsys@transformshift{2.105368in}{0.959840in}%
\pgfsys@useobject{currentmarker}{}%
\end{pgfscope}%
\begin{pgfscope}%
\pgfsys@transformshift{2.103648in}{0.959943in}%
\pgfsys@useobject{currentmarker}{}%
\end{pgfscope}%
\begin{pgfscope}%
\pgfsys@transformshift{2.102701in}{0.959979in}%
\pgfsys@useobject{currentmarker}{}%
\end{pgfscope}%
\begin{pgfscope}%
\pgfsys@transformshift{2.102181in}{0.959993in}%
\pgfsys@useobject{currentmarker}{}%
\end{pgfscope}%
\begin{pgfscope}%
\pgfsys@transformshift{2.100372in}{0.959968in}%
\pgfsys@useobject{currentmarker}{}%
\end{pgfscope}%
\begin{pgfscope}%
\pgfsys@transformshift{2.099378in}{0.960017in}%
\pgfsys@useobject{currentmarker}{}%
\end{pgfscope}%
\begin{pgfscope}%
\pgfsys@transformshift{2.093643in}{0.959481in}%
\pgfsys@useobject{currentmarker}{}%
\end{pgfscope}%
\begin{pgfscope}%
\pgfsys@transformshift{2.086939in}{0.958776in}%
\pgfsys@useobject{currentmarker}{}%
\end{pgfscope}%
\begin{pgfscope}%
\pgfsys@transformshift{2.078728in}{0.957584in}%
\pgfsys@useobject{currentmarker}{}%
\end{pgfscope}%
\begin{pgfscope}%
\pgfsys@transformshift{2.074165in}{0.957627in}%
\pgfsys@useobject{currentmarker}{}%
\end{pgfscope}%
\begin{pgfscope}%
\pgfsys@transformshift{2.066158in}{0.958466in}%
\pgfsys@useobject{currentmarker}{}%
\end{pgfscope}%
\begin{pgfscope}%
\pgfsys@transformshift{2.056389in}{0.957985in}%
\pgfsys@useobject{currentmarker}{}%
\end{pgfscope}%
\begin{pgfscope}%
\pgfsys@transformshift{2.043086in}{0.956679in}%
\pgfsys@useobject{currentmarker}{}%
\end{pgfscope}%
\begin{pgfscope}%
\pgfsys@transformshift{2.027734in}{0.954220in}%
\pgfsys@useobject{currentmarker}{}%
\end{pgfscope}%
\begin{pgfscope}%
\pgfsys@transformshift{2.011668in}{0.953088in}%
\pgfsys@useobject{currentmarker}{}%
\end{pgfscope}%
\begin{pgfscope}%
\pgfsys@transformshift{1.990926in}{0.954314in}%
\pgfsys@useobject{currentmarker}{}%
\end{pgfscope}%
\begin{pgfscope}%
\pgfsys@transformshift{1.968785in}{0.954505in}%
\pgfsys@useobject{currentmarker}{}%
\end{pgfscope}%
\begin{pgfscope}%
\pgfsys@transformshift{1.943682in}{0.953720in}%
\pgfsys@useobject{currentmarker}{}%
\end{pgfscope}%
\begin{pgfscope}%
\pgfsys@transformshift{1.917203in}{0.948297in}%
\pgfsys@useobject{currentmarker}{}%
\end{pgfscope}%
\begin{pgfscope}%
\pgfsys@transformshift{1.889640in}{0.946826in}%
\pgfsys@useobject{currentmarker}{}%
\end{pgfscope}%
\begin{pgfscope}%
\pgfsys@transformshift{1.858215in}{0.950451in}%
\pgfsys@useobject{currentmarker}{}%
\end{pgfscope}%
\begin{pgfscope}%
\pgfsys@transformshift{1.826024in}{0.949524in}%
\pgfsys@useobject{currentmarker}{}%
\end{pgfscope}%
\begin{pgfscope}%
\pgfsys@transformshift{1.791998in}{0.944869in}%
\pgfsys@useobject{currentmarker}{}%
\end{pgfscope}%
\begin{pgfscope}%
\pgfsys@transformshift{1.755932in}{0.941612in}%
\pgfsys@useobject{currentmarker}{}%
\end{pgfscope}%
\begin{pgfscope}%
\pgfsys@transformshift{1.718974in}{0.940898in}%
\pgfsys@useobject{currentmarker}{}%
\end{pgfscope}%
\begin{pgfscope}%
\pgfsys@transformshift{1.677336in}{0.938182in}%
\pgfsys@useobject{currentmarker}{}%
\end{pgfscope}%
\begin{pgfscope}%
\pgfsys@transformshift{1.635186in}{0.934564in}%
\pgfsys@useobject{currentmarker}{}%
\end{pgfscope}%
\begin{pgfscope}%
\pgfsys@transformshift{1.589457in}{0.930190in}%
\pgfsys@useobject{currentmarker}{}%
\end{pgfscope}%
\begin{pgfscope}%
\pgfsys@transformshift{1.542641in}{0.925899in}%
\pgfsys@useobject{currentmarker}{}%
\end{pgfscope}%
\begin{pgfscope}%
\pgfsys@transformshift{1.492977in}{0.927226in}%
\pgfsys@useobject{currentmarker}{}%
\end{pgfscope}%
\begin{pgfscope}%
\pgfsys@transformshift{1.439513in}{0.927372in}%
\pgfsys@useobject{currentmarker}{}%
\end{pgfscope}%
\begin{pgfscope}%
\pgfsys@transformshift{1.410315in}{0.923890in}%
\pgfsys@useobject{currentmarker}{}%
\end{pgfscope}%
\begin{pgfscope}%
\pgfsys@transformshift{1.376490in}{0.921871in}%
\pgfsys@useobject{currentmarker}{}%
\end{pgfscope}%
\begin{pgfscope}%
\pgfsys@transformshift{1.341958in}{0.919188in}%
\pgfsys@useobject{currentmarker}{}%
\end{pgfscope}%
\begin{pgfscope}%
\pgfsys@transformshift{1.322940in}{0.920305in}%
\pgfsys@useobject{currentmarker}{}%
\end{pgfscope}%
\begin{pgfscope}%
\pgfsys@transformshift{1.303042in}{0.920276in}%
\pgfsys@useobject{currentmarker}{}%
\end{pgfscope}%
\begin{pgfscope}%
\pgfsys@transformshift{1.292243in}{0.922052in}%
\pgfsys@useobject{currentmarker}{}%
\end{pgfscope}%
\begin{pgfscope}%
\pgfsys@transformshift{1.286989in}{0.924989in}%
\pgfsys@useobject{currentmarker}{}%
\end{pgfscope}%
\begin{pgfscope}%
\pgfsys@transformshift{1.284291in}{0.926909in}%
\pgfsys@useobject{currentmarker}{}%
\end{pgfscope}%
\begin{pgfscope}%
\pgfsys@transformshift{1.283816in}{0.930987in}%
\pgfsys@useobject{currentmarker}{}%
\end{pgfscope}%
\begin{pgfscope}%
\pgfsys@transformshift{1.284118in}{0.937965in}%
\pgfsys@useobject{currentmarker}{}%
\end{pgfscope}%
\begin{pgfscope}%
\pgfsys@transformshift{1.286027in}{0.946600in}%
\pgfsys@useobject{currentmarker}{}%
\end{pgfscope}%
\begin{pgfscope}%
\pgfsys@transformshift{1.287595in}{0.956269in}%
\pgfsys@useobject{currentmarker}{}%
\end{pgfscope}%
\begin{pgfscope}%
\pgfsys@transformshift{1.289927in}{0.967336in}%
\pgfsys@useobject{currentmarker}{}%
\end{pgfscope}%
\begin{pgfscope}%
\pgfsys@transformshift{1.291334in}{0.979481in}%
\pgfsys@useobject{currentmarker}{}%
\end{pgfscope}%
\begin{pgfscope}%
\pgfsys@transformshift{1.293536in}{0.992794in}%
\pgfsys@useobject{currentmarker}{}%
\end{pgfscope}%
\begin{pgfscope}%
\pgfsys@transformshift{1.293682in}{1.000214in}%
\pgfsys@useobject{currentmarker}{}%
\end{pgfscope}%
\begin{pgfscope}%
\pgfsys@transformshift{1.294342in}{1.004242in}%
\pgfsys@useobject{currentmarker}{}%
\end{pgfscope}%
\begin{pgfscope}%
\pgfsys@transformshift{1.292874in}{1.010289in}%
\pgfsys@useobject{currentmarker}{}%
\end{pgfscope}%
\begin{pgfscope}%
\pgfsys@transformshift{1.292415in}{1.013681in}%
\pgfsys@useobject{currentmarker}{}%
\end{pgfscope}%
\begin{pgfscope}%
\pgfsys@transformshift{1.293407in}{1.017695in}%
\pgfsys@useobject{currentmarker}{}%
\end{pgfscope}%
\begin{pgfscope}%
\pgfsys@transformshift{1.290635in}{1.026164in}%
\pgfsys@useobject{currentmarker}{}%
\end{pgfscope}%
\begin{pgfscope}%
\pgfsys@transformshift{1.290765in}{1.031063in}%
\pgfsys@useobject{currentmarker}{}%
\end{pgfscope}%
\begin{pgfscope}%
\pgfsys@transformshift{1.292605in}{1.038051in}%
\pgfsys@useobject{currentmarker}{}%
\end{pgfscope}%
\begin{pgfscope}%
\pgfsys@transformshift{1.289487in}{1.049034in}%
\pgfsys@useobject{currentmarker}{}%
\end{pgfscope}%
\begin{pgfscope}%
\pgfsys@transformshift{1.286550in}{1.060965in}%
\pgfsys@useobject{currentmarker}{}%
\end{pgfscope}%
\begin{pgfscope}%
\pgfsys@transformshift{1.287847in}{1.078244in}%
\pgfsys@useobject{currentmarker}{}%
\end{pgfscope}%
\begin{pgfscope}%
\pgfsys@transformshift{1.284918in}{1.096074in}%
\pgfsys@useobject{currentmarker}{}%
\end{pgfscope}%
\begin{pgfscope}%
\pgfsys@transformshift{1.275048in}{1.114913in}%
\pgfsys@useobject{currentmarker}{}%
\end{pgfscope}%
\begin{pgfscope}%
\pgfsys@transformshift{1.274651in}{1.136730in}%
\pgfsys@useobject{currentmarker}{}%
\end{pgfscope}%
\begin{pgfscope}%
\pgfsys@transformshift{1.279124in}{1.158994in}%
\pgfsys@useobject{currentmarker}{}%
\end{pgfscope}%
\begin{pgfscope}%
\pgfsys@transformshift{1.270614in}{1.185160in}%
\pgfsys@useobject{currentmarker}{}%
\end{pgfscope}%
\begin{pgfscope}%
\pgfsys@transformshift{1.268523in}{1.213322in}%
\pgfsys@useobject{currentmarker}{}%
\end{pgfscope}%
\begin{pgfscope}%
\pgfsys@transformshift{1.269027in}{1.245518in}%
\pgfsys@useobject{currentmarker}{}%
\end{pgfscope}%
\begin{pgfscope}%
\pgfsys@transformshift{1.268216in}{1.278653in}%
\pgfsys@useobject{currentmarker}{}%
\end{pgfscope}%
\begin{pgfscope}%
\pgfsys@transformshift{1.255914in}{1.310088in}%
\pgfsys@useobject{currentmarker}{}%
\end{pgfscope}%
\begin{pgfscope}%
\pgfsys@transformshift{1.255277in}{1.328643in}%
\pgfsys@useobject{currentmarker}{}%
\end{pgfscope}%
\begin{pgfscope}%
\pgfsys@transformshift{1.258083in}{1.350170in}%
\pgfsys@useobject{currentmarker}{}%
\end{pgfscope}%
\begin{pgfscope}%
\pgfsys@transformshift{1.254817in}{1.372269in}%
\pgfsys@useobject{currentmarker}{}%
\end{pgfscope}%
\begin{pgfscope}%
\pgfsys@transformshift{1.250544in}{1.383789in}%
\pgfsys@useobject{currentmarker}{}%
\end{pgfscope}%
\begin{pgfscope}%
\pgfsys@transformshift{1.249510in}{1.390467in}%
\pgfsys@useobject{currentmarker}{}%
\end{pgfscope}%
\begin{pgfscope}%
\pgfsys@transformshift{1.251921in}{1.399796in}%
\pgfsys@useobject{currentmarker}{}%
\end{pgfscope}%
\begin{pgfscope}%
\pgfsys@transformshift{1.249555in}{1.410343in}%
\pgfsys@useobject{currentmarker}{}%
\end{pgfscope}%
\begin{pgfscope}%
\pgfsys@transformshift{1.248298in}{1.416154in}%
\pgfsys@useobject{currentmarker}{}%
\end{pgfscope}%
\begin{pgfscope}%
\pgfsys@transformshift{1.246849in}{1.423273in}%
\pgfsys@useobject{currentmarker}{}%
\end{pgfscope}%
\begin{pgfscope}%
\pgfsys@transformshift{1.249483in}{1.431380in}%
\pgfsys@useobject{currentmarker}{}%
\end{pgfscope}%
\begin{pgfscope}%
\pgfsys@transformshift{1.247074in}{1.443862in}%
\pgfsys@useobject{currentmarker}{}%
\end{pgfscope}%
\begin{pgfscope}%
\pgfsys@transformshift{1.247047in}{1.450854in}%
\pgfsys@useobject{currentmarker}{}%
\end{pgfscope}%
\begin{pgfscope}%
\pgfsys@transformshift{1.246354in}{1.461022in}%
\pgfsys@useobject{currentmarker}{}%
\end{pgfscope}%
\begin{pgfscope}%
\pgfsys@transformshift{1.250954in}{1.471831in}%
\pgfsys@useobject{currentmarker}{}%
\end{pgfscope}%
\begin{pgfscope}%
\pgfsys@transformshift{1.248709in}{1.487108in}%
\pgfsys@useobject{currentmarker}{}%
\end{pgfscope}%
\begin{pgfscope}%
\pgfsys@transformshift{1.247841in}{1.495556in}%
\pgfsys@useobject{currentmarker}{}%
\end{pgfscope}%
\begin{pgfscope}%
\pgfsys@transformshift{1.246270in}{1.505385in}%
\pgfsys@useobject{currentmarker}{}%
\end{pgfscope}%
\begin{pgfscope}%
\pgfsys@transformshift{1.249507in}{1.515966in}%
\pgfsys@useobject{currentmarker}{}%
\end{pgfscope}%
\begin{pgfscope}%
\pgfsys@transformshift{1.245949in}{1.531290in}%
\pgfsys@useobject{currentmarker}{}%
\end{pgfscope}%
\begin{pgfscope}%
\pgfsys@transformshift{1.246586in}{1.547269in}%
\pgfsys@useobject{currentmarker}{}%
\end{pgfscope}%
\begin{pgfscope}%
\pgfsys@transformshift{1.247285in}{1.566375in}%
\pgfsys@useobject{currentmarker}{}%
\end{pgfscope}%
\begin{pgfscope}%
\pgfsys@transformshift{1.244723in}{1.576574in}%
\pgfsys@useobject{currentmarker}{}%
\end{pgfscope}%
\begin{pgfscope}%
\pgfsys@transformshift{1.248752in}{1.590644in}%
\pgfsys@useobject{currentmarker}{}%
\end{pgfscope}%
\begin{pgfscope}%
\pgfsys@transformshift{1.251118in}{1.605783in}%
\pgfsys@useobject{currentmarker}{}%
\end{pgfscope}%
\begin{pgfscope}%
\pgfsys@transformshift{1.256968in}{1.623299in}%
\pgfsys@useobject{currentmarker}{}%
\end{pgfscope}%
\begin{pgfscope}%
\pgfsys@transformshift{1.251925in}{1.642101in}%
\pgfsys@useobject{currentmarker}{}%
\end{pgfscope}%
\begin{pgfscope}%
\pgfsys@transformshift{1.259428in}{1.664630in}%
\pgfsys@useobject{currentmarker}{}%
\end{pgfscope}%
\begin{pgfscope}%
\pgfsys@transformshift{1.262153in}{1.677403in}%
\pgfsys@useobject{currentmarker}{}%
\end{pgfscope}%
\begin{pgfscope}%
\pgfsys@transformshift{1.262960in}{1.693597in}%
\pgfsys@useobject{currentmarker}{}%
\end{pgfscope}%
\begin{pgfscope}%
\pgfsys@transformshift{1.261713in}{1.702428in}%
\pgfsys@useobject{currentmarker}{}%
\end{pgfscope}%
\begin{pgfscope}%
\pgfsys@transformshift{1.265590in}{1.715308in}%
\pgfsys@useobject{currentmarker}{}%
\end{pgfscope}%
\begin{pgfscope}%
\pgfsys@transformshift{1.266161in}{1.722684in}%
\pgfsys@useobject{currentmarker}{}%
\end{pgfscope}%
\begin{pgfscope}%
\pgfsys@transformshift{1.265807in}{1.733055in}%
\pgfsys@useobject{currentmarker}{}%
\end{pgfscope}%
\begin{pgfscope}%
\pgfsys@transformshift{1.264410in}{1.744092in}%
\pgfsys@useobject{currentmarker}{}%
\end{pgfscope}%
\begin{pgfscope}%
\pgfsys@transformshift{1.267744in}{1.758156in}%
\pgfsys@useobject{currentmarker}{}%
\end{pgfscope}%
\begin{pgfscope}%
\pgfsys@transformshift{1.268229in}{1.766091in}%
\pgfsys@useobject{currentmarker}{}%
\end{pgfscope}%
\begin{pgfscope}%
\pgfsys@transformshift{1.270479in}{1.776475in}%
\pgfsys@useobject{currentmarker}{}%
\end{pgfscope}%
\begin{pgfscope}%
\pgfsys@transformshift{1.269037in}{1.782139in}%
\pgfsys@useobject{currentmarker}{}%
\end{pgfscope}%
\begin{pgfscope}%
\pgfsys@transformshift{1.271304in}{1.792021in}%
\pgfsys@useobject{currentmarker}{}%
\end{pgfscope}%
\begin{pgfscope}%
\pgfsys@transformshift{1.274388in}{1.802865in}%
\pgfsys@useobject{currentmarker}{}%
\end{pgfscope}%
\begin{pgfscope}%
\pgfsys@transformshift{1.277192in}{1.814827in}%
\pgfsys@useobject{currentmarker}{}%
\end{pgfscope}%
\begin{pgfscope}%
\pgfsys@transformshift{1.273899in}{1.828563in}%
\pgfsys@useobject{currentmarker}{}%
\end{pgfscope}%
\begin{pgfscope}%
\pgfsys@transformshift{1.278308in}{1.844657in}%
\pgfsys@useobject{currentmarker}{}%
\end{pgfscope}%
\begin{pgfscope}%
\pgfsys@transformshift{1.284044in}{1.861793in}%
\pgfsys@useobject{currentmarker}{}%
\end{pgfscope}%
\begin{pgfscope}%
\pgfsys@transformshift{1.284902in}{1.871695in}%
\pgfsys@useobject{currentmarker}{}%
\end{pgfscope}%
\begin{pgfscope}%
\pgfsys@transformshift{1.282531in}{1.883327in}%
\pgfsys@useobject{currentmarker}{}%
\end{pgfscope}%
\begin{pgfscope}%
\pgfsys@transformshift{1.282975in}{1.889842in}%
\pgfsys@useobject{currentmarker}{}%
\end{pgfscope}%
\begin{pgfscope}%
\pgfsys@transformshift{1.284931in}{1.896955in}%
\pgfsys@useobject{currentmarker}{}%
\end{pgfscope}%
\begin{pgfscope}%
\pgfsys@transformshift{1.284807in}{1.901011in}%
\pgfsys@useobject{currentmarker}{}%
\end{pgfscope}%
\begin{pgfscope}%
\pgfsys@transformshift{1.285073in}{1.908181in}%
\pgfsys@useobject{currentmarker}{}%
\end{pgfscope}%
\begin{pgfscope}%
\pgfsys@transformshift{1.284058in}{1.915788in}%
\pgfsys@useobject{currentmarker}{}%
\end{pgfscope}%
\begin{pgfscope}%
\pgfsys@transformshift{1.287048in}{1.926678in}%
\pgfsys@useobject{currentmarker}{}%
\end{pgfscope}%
\begin{pgfscope}%
\pgfsys@transformshift{1.288425in}{1.932735in}%
\pgfsys@useobject{currentmarker}{}%
\end{pgfscope}%
\begin{pgfscope}%
\pgfsys@transformshift{1.289396in}{1.941280in}%
\pgfsys@useobject{currentmarker}{}%
\end{pgfscope}%
\begin{pgfscope}%
\pgfsys@transformshift{1.288624in}{1.945947in}%
\pgfsys@useobject{currentmarker}{}%
\end{pgfscope}%
\begin{pgfscope}%
\pgfsys@transformshift{1.291901in}{1.954471in}%
\pgfsys@useobject{currentmarker}{}%
\end{pgfscope}%
\begin{pgfscope}%
\pgfsys@transformshift{1.293982in}{1.963866in}%
\pgfsys@useobject{currentmarker}{}%
\end{pgfscope}%
\begin{pgfscope}%
\pgfsys@transformshift{1.295561in}{1.975793in}%
\pgfsys@useobject{currentmarker}{}%
\end{pgfscope}%
\begin{pgfscope}%
\pgfsys@transformshift{1.293204in}{1.988271in}%
\pgfsys@useobject{currentmarker}{}%
\end{pgfscope}%
\begin{pgfscope}%
\pgfsys@transformshift{1.297894in}{2.004335in}%
\pgfsys@useobject{currentmarker}{}%
\end{pgfscope}%
\begin{pgfscope}%
\pgfsys@transformshift{1.301502in}{2.021359in}%
\pgfsys@useobject{currentmarker}{}%
\end{pgfscope}%
\begin{pgfscope}%
\pgfsys@transformshift{1.307878in}{2.038488in}%
\pgfsys@useobject{currentmarker}{}%
\end{pgfscope}%
\begin{pgfscope}%
\pgfsys@transformshift{1.303849in}{2.058167in}%
\pgfsys@useobject{currentmarker}{}%
\end{pgfscope}%
\begin{pgfscope}%
\pgfsys@transformshift{1.309957in}{2.080792in}%
\pgfsys@useobject{currentmarker}{}%
\end{pgfscope}%
\begin{pgfscope}%
\pgfsys@transformshift{1.316504in}{2.103824in}%
\pgfsys@useobject{currentmarker}{}%
\end{pgfscope}%
\begin{pgfscope}%
\pgfsys@transformshift{1.317509in}{2.116956in}%
\pgfsys@useobject{currentmarker}{}%
\end{pgfscope}%
\begin{pgfscope}%
\pgfsys@transformshift{1.314785in}{2.123667in}%
\pgfsys@useobject{currentmarker}{}%
\end{pgfscope}%
\begin{pgfscope}%
\pgfsys@transformshift{1.308017in}{2.128453in}%
\pgfsys@useobject{currentmarker}{}%
\end{pgfscope}%
\begin{pgfscope}%
\pgfsys@transformshift{1.299294in}{2.132248in}%
\pgfsys@useobject{currentmarker}{}%
\end{pgfscope}%
\begin{pgfscope}%
\pgfsys@transformshift{1.289417in}{2.133739in}%
\pgfsys@useobject{currentmarker}{}%
\end{pgfscope}%
\begin{pgfscope}%
\pgfsys@transformshift{1.278410in}{2.133472in}%
\pgfsys@useobject{currentmarker}{}%
\end{pgfscope}%
\begin{pgfscope}%
\pgfsys@transformshift{1.266809in}{2.133110in}%
\pgfsys@useobject{currentmarker}{}%
\end{pgfscope}%
\begin{pgfscope}%
\pgfsys@transformshift{1.254504in}{2.131728in}%
\pgfsys@useobject{currentmarker}{}%
\end{pgfscope}%
\begin{pgfscope}%
\pgfsys@transformshift{1.247076in}{2.130810in}%
\pgfsys@useobject{currentmarker}{}%
\end{pgfscope}%
\begin{pgfscope}%
\pgfsys@transformshift{1.260836in}{2.129035in}%
\pgfsys@useobject{currentmarker}{}%
\end{pgfscope}%
\begin{pgfscope}%
\pgfsys@transformshift{1.268437in}{2.128371in}%
\pgfsys@useobject{currentmarker}{}%
\end{pgfscope}%
\begin{pgfscope}%
\pgfsys@transformshift{1.272599in}{2.127828in}%
\pgfsys@useobject{currentmarker}{}%
\end{pgfscope}%
\begin{pgfscope}%
\pgfsys@transformshift{1.277323in}{2.127670in}%
\pgfsys@useobject{currentmarker}{}%
\end{pgfscope}%
\begin{pgfscope}%
\pgfsys@transformshift{1.283477in}{2.126909in}%
\pgfsys@useobject{currentmarker}{}%
\end{pgfscope}%
\begin{pgfscope}%
\pgfsys@transformshift{1.291843in}{2.128373in}%
\pgfsys@useobject{currentmarker}{}%
\end{pgfscope}%
\begin{pgfscope}%
\pgfsys@transformshift{1.301670in}{2.128680in}%
\pgfsys@useobject{currentmarker}{}%
\end{pgfscope}%
\begin{pgfscope}%
\pgfsys@transformshift{1.314753in}{2.127444in}%
\pgfsys@useobject{currentmarker}{}%
\end{pgfscope}%
\begin{pgfscope}%
\pgfsys@transformshift{1.328450in}{2.125584in}%
\pgfsys@useobject{currentmarker}{}%
\end{pgfscope}%
\begin{pgfscope}%
\pgfsys@transformshift{1.336027in}{2.124965in}%
\pgfsys@useobject{currentmarker}{}%
\end{pgfscope}%
\begin{pgfscope}%
\pgfsys@transformshift{1.340199in}{2.124699in}%
\pgfsys@useobject{currentmarker}{}%
\end{pgfscope}%
\begin{pgfscope}%
\pgfsys@transformshift{1.346215in}{2.123696in}%
\pgfsys@useobject{currentmarker}{}%
\end{pgfscope}%
\begin{pgfscope}%
\pgfsys@transformshift{1.349570in}{2.123687in}%
\pgfsys@useobject{currentmarker}{}%
\end{pgfscope}%
\begin{pgfscope}%
\pgfsys@transformshift{1.354924in}{2.122337in}%
\pgfsys@useobject{currentmarker}{}%
\end{pgfscope}%
\begin{pgfscope}%
\pgfsys@transformshift{1.357960in}{2.122362in}%
\pgfsys@useobject{currentmarker}{}%
\end{pgfscope}%
\begin{pgfscope}%
\pgfsys@transformshift{1.363396in}{2.121598in}%
\pgfsys@useobject{currentmarker}{}%
\end{pgfscope}%
\begin{pgfscope}%
\pgfsys@transformshift{1.369462in}{2.121669in}%
\pgfsys@useobject{currentmarker}{}%
\end{pgfscope}%
\begin{pgfscope}%
\pgfsys@transformshift{1.378627in}{2.121501in}%
\pgfsys@useobject{currentmarker}{}%
\end{pgfscope}%
\begin{pgfscope}%
\pgfsys@transformshift{1.388334in}{2.120660in}%
\pgfsys@useobject{currentmarker}{}%
\end{pgfscope}%
\begin{pgfscope}%
\pgfsys@transformshift{1.393692in}{2.120517in}%
\pgfsys@useobject{currentmarker}{}%
\end{pgfscope}%
\begin{pgfscope}%
\pgfsys@transformshift{1.396612in}{2.120117in}%
\pgfsys@useobject{currentmarker}{}%
\end{pgfscope}%
\begin{pgfscope}%
\pgfsys@transformshift{1.400829in}{2.121023in}%
\pgfsys@useobject{currentmarker}{}%
\end{pgfscope}%
\begin{pgfscope}%
\pgfsys@transformshift{1.403191in}{2.120805in}%
\pgfsys@useobject{currentmarker}{}%
\end{pgfscope}%
\begin{pgfscope}%
\pgfsys@transformshift{1.406077in}{2.120962in}%
\pgfsys@useobject{currentmarker}{}%
\end{pgfscope}%
\begin{pgfscope}%
\pgfsys@transformshift{1.409416in}{2.120660in}%
\pgfsys@useobject{currentmarker}{}%
\end{pgfscope}%
\begin{pgfscope}%
\pgfsys@transformshift{1.414374in}{2.122216in}%
\pgfsys@useobject{currentmarker}{}%
\end{pgfscope}%
\begin{pgfscope}%
\pgfsys@transformshift{1.420098in}{2.121550in}%
\pgfsys@useobject{currentmarker}{}%
\end{pgfscope}%
\begin{pgfscope}%
\pgfsys@transformshift{1.427691in}{2.121295in}%
\pgfsys@useobject{currentmarker}{}%
\end{pgfscope}%
\begin{pgfscope}%
\pgfsys@transformshift{1.436820in}{2.119135in}%
\pgfsys@useobject{currentmarker}{}%
\end{pgfscope}%
\begin{pgfscope}%
\pgfsys@transformshift{1.441979in}{2.119102in}%
\pgfsys@useobject{currentmarker}{}%
\end{pgfscope}%
\begin{pgfscope}%
\pgfsys@transformshift{1.448803in}{2.118757in}%
\pgfsys@useobject{currentmarker}{}%
\end{pgfscope}%
\begin{pgfscope}%
\pgfsys@transformshift{1.456917in}{2.119806in}%
\pgfsys@useobject{currentmarker}{}%
\end{pgfscope}%
\begin{pgfscope}%
\pgfsys@transformshift{1.467193in}{2.120759in}%
\pgfsys@useobject{currentmarker}{}%
\end{pgfscope}%
\begin{pgfscope}%
\pgfsys@transformshift{1.478314in}{2.120988in}%
\pgfsys@useobject{currentmarker}{}%
\end{pgfscope}%
\begin{pgfscope}%
\pgfsys@transformshift{1.493718in}{2.119968in}%
\pgfsys@useobject{currentmarker}{}%
\end{pgfscope}%
\begin{pgfscope}%
\pgfsys@transformshift{1.510087in}{2.119348in}%
\pgfsys@useobject{currentmarker}{}%
\end{pgfscope}%
\begin{pgfscope}%
\pgfsys@transformshift{1.528763in}{2.119374in}%
\pgfsys@useobject{currentmarker}{}%
\end{pgfscope}%
\begin{pgfscope}%
\pgfsys@transformshift{1.547657in}{2.115991in}%
\pgfsys@useobject{currentmarker}{}%
\end{pgfscope}%
\begin{pgfscope}%
\pgfsys@transformshift{1.567756in}{2.115956in}%
\pgfsys@useobject{currentmarker}{}%
\end{pgfscope}%
\begin{pgfscope}%
\pgfsys@transformshift{1.592195in}{2.119506in}%
\pgfsys@useobject{currentmarker}{}%
\end{pgfscope}%
\begin{pgfscope}%
\pgfsys@transformshift{1.617706in}{2.118499in}%
\pgfsys@useobject{currentmarker}{}%
\end{pgfscope}%
\begin{pgfscope}%
\pgfsys@transformshift{1.644096in}{2.123379in}%
\pgfsys@useobject{currentmarker}{}%
\end{pgfscope}%
\begin{pgfscope}%
\pgfsys@transformshift{1.672882in}{2.123610in}%
\pgfsys@useobject{currentmarker}{}%
\end{pgfscope}%
\begin{pgfscope}%
\pgfsys@transformshift{1.702158in}{2.126993in}%
\pgfsys@useobject{currentmarker}{}%
\end{pgfscope}%
\begin{pgfscope}%
\pgfsys@transformshift{1.734346in}{2.127234in}%
\pgfsys@useobject{currentmarker}{}%
\end{pgfscope}%
\begin{pgfscope}%
\pgfsys@transformshift{1.767056in}{2.129031in}%
\pgfsys@useobject{currentmarker}{}%
\end{pgfscope}%
\begin{pgfscope}%
\pgfsys@transformshift{1.801723in}{2.125039in}%
\pgfsys@useobject{currentmarker}{}%
\end{pgfscope}%
\begin{pgfscope}%
\pgfsys@transformshift{1.820906in}{2.125692in}%
\pgfsys@useobject{currentmarker}{}%
\end{pgfscope}%
\begin{pgfscope}%
\pgfsys@transformshift{1.843172in}{2.125522in}%
\pgfsys@useobject{currentmarker}{}%
\end{pgfscope}%
\begin{pgfscope}%
\pgfsys@transformshift{1.865531in}{2.130409in}%
\pgfsys@useobject{currentmarker}{}%
\end{pgfscope}%
\begin{pgfscope}%
\pgfsys@transformshift{1.889884in}{2.133050in}%
\pgfsys@useobject{currentmarker}{}%
\end{pgfscope}%
\begin{pgfscope}%
\pgfsys@transformshift{1.903339in}{2.132360in}%
\pgfsys@useobject{currentmarker}{}%
\end{pgfscope}%
\begin{pgfscope}%
\pgfsys@transformshift{1.910734in}{2.131897in}%
\pgfsys@useobject{currentmarker}{}%
\end{pgfscope}%
\begin{pgfscope}%
\pgfsys@transformshift{1.920403in}{2.131859in}%
\pgfsys@useobject{currentmarker}{}%
\end{pgfscope}%
\begin{pgfscope}%
\pgfsys@transformshift{1.930922in}{2.130351in}%
\pgfsys@useobject{currentmarker}{}%
\end{pgfscope}%
\begin{pgfscope}%
\pgfsys@transformshift{1.942026in}{2.131100in}%
\pgfsys@useobject{currentmarker}{}%
\end{pgfscope}%
\begin{pgfscope}%
\pgfsys@transformshift{1.953907in}{2.129285in}%
\pgfsys@useobject{currentmarker}{}%
\end{pgfscope}%
\begin{pgfscope}%
\pgfsys@transformshift{1.966214in}{2.133675in}%
\pgfsys@useobject{currentmarker}{}%
\end{pgfscope}%
\begin{pgfscope}%
\pgfsys@transformshift{1.973401in}{2.133650in}%
\pgfsys@useobject{currentmarker}{}%
\end{pgfscope}%
\begin{pgfscope}%
\pgfsys@transformshift{1.983493in}{2.133035in}%
\pgfsys@useobject{currentmarker}{}%
\end{pgfscope}%
\begin{pgfscope}%
\pgfsys@transformshift{1.988916in}{2.131807in}%
\pgfsys@useobject{currentmarker}{}%
\end{pgfscope}%
\begin{pgfscope}%
\pgfsys@transformshift{1.996144in}{2.132838in}%
\pgfsys@useobject{currentmarker}{}%
\end{pgfscope}%
\begin{pgfscope}%
\pgfsys@transformshift{2.004246in}{2.132296in}%
\pgfsys@useobject{currentmarker}{}%
\end{pgfscope}%
\begin{pgfscope}%
\pgfsys@transformshift{2.016273in}{2.133231in}%
\pgfsys@useobject{currentmarker}{}%
\end{pgfscope}%
\begin{pgfscope}%
\pgfsys@transformshift{2.028987in}{2.132507in}%
\pgfsys@useobject{currentmarker}{}%
\end{pgfscope}%
\begin{pgfscope}%
\pgfsys@transformshift{2.043715in}{2.135498in}%
\pgfsys@useobject{currentmarker}{}%
\end{pgfscope}%
\begin{pgfscope}%
\pgfsys@transformshift{2.060069in}{2.134358in}%
\pgfsys@useobject{currentmarker}{}%
\end{pgfscope}%
\begin{pgfscope}%
\pgfsys@transformshift{2.077178in}{2.135341in}%
\pgfsys@useobject{currentmarker}{}%
\end{pgfscope}%
\begin{pgfscope}%
\pgfsys@transformshift{2.096607in}{2.135337in}%
\pgfsys@useobject{currentmarker}{}%
\end{pgfscope}%
\begin{pgfscope}%
\pgfsys@transformshift{2.119600in}{2.135003in}%
\pgfsys@useobject{currentmarker}{}%
\end{pgfscope}%
\begin{pgfscope}%
\pgfsys@transformshift{2.143775in}{2.136577in}%
\pgfsys@useobject{currentmarker}{}%
\end{pgfscope}%
\begin{pgfscope}%
\pgfsys@transformshift{2.170038in}{2.137350in}%
\pgfsys@useobject{currentmarker}{}%
\end{pgfscope}%
\begin{pgfscope}%
\pgfsys@transformshift{2.199431in}{2.139715in}%
\pgfsys@useobject{currentmarker}{}%
\end{pgfscope}%
\begin{pgfscope}%
\pgfsys@transformshift{2.228988in}{2.146374in}%
\pgfsys@useobject{currentmarker}{}%
\end{pgfscope}%
\begin{pgfscope}%
\pgfsys@transformshift{2.245487in}{2.148711in}%
\pgfsys@useobject{currentmarker}{}%
\end{pgfscope}%
\begin{pgfscope}%
\pgfsys@transformshift{2.263306in}{2.149524in}%
\pgfsys@useobject{currentmarker}{}%
\end{pgfscope}%
\begin{pgfscope}%
\pgfsys@transformshift{2.273075in}{2.150430in}%
\pgfsys@useobject{currentmarker}{}%
\end{pgfscope}%
\begin{pgfscope}%
\pgfsys@transformshift{2.285163in}{2.152714in}%
\pgfsys@useobject{currentmarker}{}%
\end{pgfscope}%
\begin{pgfscope}%
\pgfsys@transformshift{2.298264in}{2.153761in}%
\pgfsys@useobject{currentmarker}{}%
\end{pgfscope}%
\begin{pgfscope}%
\pgfsys@transformshift{2.312668in}{2.153192in}%
\pgfsys@useobject{currentmarker}{}%
\end{pgfscope}%
\begin{pgfscope}%
\pgfsys@transformshift{2.320596in}{2.153159in}%
\pgfsys@useobject{currentmarker}{}%
\end{pgfscope}%
\begin{pgfscope}%
\pgfsys@transformshift{2.330603in}{2.154898in}%
\pgfsys@useobject{currentmarker}{}%
\end{pgfscope}%
\begin{pgfscope}%
\pgfsys@transformshift{2.336179in}{2.155229in}%
\pgfsys@useobject{currentmarker}{}%
\end{pgfscope}%
\begin{pgfscope}%
\pgfsys@transformshift{2.339250in}{2.155306in}%
\pgfsys@useobject{currentmarker}{}%
\end{pgfscope}%
\begin{pgfscope}%
\pgfsys@transformshift{2.342804in}{2.155255in}%
\pgfsys@useobject{currentmarker}{}%
\end{pgfscope}%
\begin{pgfscope}%
\pgfsys@transformshift{2.347177in}{2.156414in}%
\pgfsys@useobject{currentmarker}{}%
\end{pgfscope}%
\begin{pgfscope}%
\pgfsys@transformshift{2.349660in}{2.156578in}%
\pgfsys@useobject{currentmarker}{}%
\end{pgfscope}%
\begin{pgfscope}%
\pgfsys@transformshift{2.351028in}{2.156565in}%
\pgfsys@useobject{currentmarker}{}%
\end{pgfscope}%
\begin{pgfscope}%
\pgfsys@transformshift{2.351781in}{2.156567in}%
\pgfsys@useobject{currentmarker}{}%
\end{pgfscope}%
\begin{pgfscope}%
\pgfsys@transformshift{2.353876in}{2.157198in}%
\pgfsys@useobject{currentmarker}{}%
\end{pgfscope}%
\begin{pgfscope}%
\pgfsys@transformshift{2.356970in}{2.157516in}%
\pgfsys@useobject{currentmarker}{}%
\end{pgfscope}%
\begin{pgfscope}%
\pgfsys@transformshift{2.362043in}{2.157508in}%
\pgfsys@useobject{currentmarker}{}%
\end{pgfscope}%
\begin{pgfscope}%
\pgfsys@transformshift{2.364829in}{2.157660in}%
\pgfsys@useobject{currentmarker}{}%
\end{pgfscope}%
\begin{pgfscope}%
\pgfsys@transformshift{2.369958in}{2.160389in}%
\pgfsys@useobject{currentmarker}{}%
\end{pgfscope}%
\begin{pgfscope}%
\pgfsys@transformshift{2.373144in}{2.160639in}%
\pgfsys@useobject{currentmarker}{}%
\end{pgfscope}%
\begin{pgfscope}%
\pgfsys@transformshift{2.378656in}{2.160648in}%
\pgfsys@useobject{currentmarker}{}%
\end{pgfscope}%
\begin{pgfscope}%
\pgfsys@transformshift{2.385036in}{2.159945in}%
\pgfsys@useobject{currentmarker}{}%
\end{pgfscope}%
\begin{pgfscope}%
\pgfsys@transformshift{2.388517in}{2.160531in}%
\pgfsys@useobject{currentmarker}{}%
\end{pgfscope}%
\begin{pgfscope}%
\pgfsys@transformshift{2.393267in}{2.160478in}%
\pgfsys@useobject{currentmarker}{}%
\end{pgfscope}%
\begin{pgfscope}%
\pgfsys@transformshift{2.398460in}{2.161350in}%
\pgfsys@useobject{currentmarker}{}%
\end{pgfscope}%
\begin{pgfscope}%
\pgfsys@transformshift{2.405054in}{2.162751in}%
\pgfsys@useobject{currentmarker}{}%
\end{pgfscope}%
\begin{pgfscope}%
\pgfsys@transformshift{2.412831in}{2.163524in}%
\pgfsys@useobject{currentmarker}{}%
\end{pgfscope}%
\begin{pgfscope}%
\pgfsys@transformshift{2.422117in}{2.164080in}%
\pgfsys@useobject{currentmarker}{}%
\end{pgfscope}%
\begin{pgfscope}%
\pgfsys@transformshift{2.427201in}{2.164660in}%
\pgfsys@useobject{currentmarker}{}%
\end{pgfscope}%
\begin{pgfscope}%
\pgfsys@transformshift{2.433102in}{2.165289in}%
\pgfsys@useobject{currentmarker}{}%
\end{pgfscope}%
\begin{pgfscope}%
\pgfsys@transformshift{2.436366in}{2.165368in}%
\pgfsys@useobject{currentmarker}{}%
\end{pgfscope}%
\begin{pgfscope}%
\pgfsys@transformshift{2.438121in}{2.165749in}%
\pgfsys@useobject{currentmarker}{}%
\end{pgfscope}%
\begin{pgfscope}%
\pgfsys@transformshift{2.440384in}{2.166116in}%
\pgfsys@useobject{currentmarker}{}%
\end{pgfscope}%
\begin{pgfscope}%
\pgfsys@transformshift{2.443392in}{2.166985in}%
\pgfsys@useobject{currentmarker}{}%
\end{pgfscope}%
\begin{pgfscope}%
\pgfsys@transformshift{2.448258in}{2.167286in}%
\pgfsys@useobject{currentmarker}{}%
\end{pgfscope}%
\begin{pgfscope}%
\pgfsys@transformshift{2.450937in}{2.167383in}%
\pgfsys@useobject{currentmarker}{}%
\end{pgfscope}%
\begin{pgfscope}%
\pgfsys@transformshift{2.454454in}{2.167041in}%
\pgfsys@useobject{currentmarker}{}%
\end{pgfscope}%
\begin{pgfscope}%
\pgfsys@transformshift{2.458618in}{2.166684in}%
\pgfsys@useobject{currentmarker}{}%
\end{pgfscope}%
\begin{pgfscope}%
\pgfsys@transformshift{2.460811in}{2.165995in}%
\pgfsys@useobject{currentmarker}{}%
\end{pgfscope}%
\begin{pgfscope}%
\pgfsys@transformshift{2.461928in}{2.165403in}%
\pgfsys@useobject{currentmarker}{}%
\end{pgfscope}%
\begin{pgfscope}%
\pgfsys@transformshift{2.463104in}{2.163254in}%
\pgfsys@useobject{currentmarker}{}%
\end{pgfscope}%
\begin{pgfscope}%
\pgfsys@transformshift{2.463962in}{2.162216in}%
\pgfsys@useobject{currentmarker}{}%
\end{pgfscope}%
\begin{pgfscope}%
\pgfsys@transformshift{2.464564in}{2.159807in}%
\pgfsys@useobject{currentmarker}{}%
\end{pgfscope}%
\begin{pgfscope}%
\pgfsys@transformshift{2.466283in}{2.156821in}%
\pgfsys@useobject{currentmarker}{}%
\end{pgfscope}%
\begin{pgfscope}%
\pgfsys@transformshift{2.466195in}{2.154928in}%
\pgfsys@useobject{currentmarker}{}%
\end{pgfscope}%
\begin{pgfscope}%
\pgfsys@transformshift{2.466840in}{2.151957in}%
\pgfsys@useobject{currentmarker}{}%
\end{pgfscope}%
\begin{pgfscope}%
\pgfsys@transformshift{2.467437in}{2.148295in}%
\pgfsys@useobject{currentmarker}{}%
\end{pgfscope}%
\begin{pgfscope}%
\pgfsys@transformshift{2.467952in}{2.146321in}%
\pgfsys@useobject{currentmarker}{}%
\end{pgfscope}%
\begin{pgfscope}%
\pgfsys@transformshift{2.468486in}{2.143163in}%
\pgfsys@useobject{currentmarker}{}%
\end{pgfscope}%
\begin{pgfscope}%
\pgfsys@transformshift{2.469151in}{2.139365in}%
\pgfsys@useobject{currentmarker}{}%
\end{pgfscope}%
\begin{pgfscope}%
\pgfsys@transformshift{2.469285in}{2.137249in}%
\pgfsys@useobject{currentmarker}{}%
\end{pgfscope}%
\begin{pgfscope}%
\pgfsys@transformshift{2.469300in}{2.136082in}%
\pgfsys@useobject{currentmarker}{}%
\end{pgfscope}%
\begin{pgfscope}%
\pgfsys@transformshift{2.470140in}{2.133777in}%
\pgfsys@useobject{currentmarker}{}%
\end{pgfscope}%
\begin{pgfscope}%
\pgfsys@transformshift{2.469929in}{2.130225in}%
\pgfsys@useobject{currentmarker}{}%
\end{pgfscope}%
\begin{pgfscope}%
\pgfsys@transformshift{2.470285in}{2.126128in}%
\pgfsys@useobject{currentmarker}{}%
\end{pgfscope}%
\begin{pgfscope}%
\pgfsys@transformshift{2.472569in}{2.119963in}%
\pgfsys@useobject{currentmarker}{}%
\end{pgfscope}%
\begin{pgfscope}%
\pgfsys@transformshift{2.472446in}{2.112684in}%
\pgfsys@useobject{currentmarker}{}%
\end{pgfscope}%
\begin{pgfscope}%
\pgfsys@transformshift{2.471759in}{2.108740in}%
\pgfsys@useobject{currentmarker}{}%
\end{pgfscope}%
\begin{pgfscope}%
\pgfsys@transformshift{2.473506in}{2.102347in}%
\pgfsys@useobject{currentmarker}{}%
\end{pgfscope}%
\begin{pgfscope}%
\pgfsys@transformshift{2.473465in}{2.098702in}%
\pgfsys@useobject{currentmarker}{}%
\end{pgfscope}%
\begin{pgfscope}%
\pgfsys@transformshift{2.472738in}{2.094027in}%
\pgfsys@useobject{currentmarker}{}%
\end{pgfscope}%
\begin{pgfscope}%
\pgfsys@transformshift{2.474731in}{2.087101in}%
\pgfsys@useobject{currentmarker}{}%
\end{pgfscope}%
\begin{pgfscope}%
\pgfsys@transformshift{2.476924in}{2.079160in}%
\pgfsys@useobject{currentmarker}{}%
\end{pgfscope}%
\begin{pgfscope}%
\pgfsys@transformshift{2.476818in}{2.067038in}%
\pgfsys@useobject{currentmarker}{}%
\end{pgfscope}%
\begin{pgfscope}%
\pgfsys@transformshift{2.479933in}{2.054672in}%
\pgfsys@useobject{currentmarker}{}%
\end{pgfscope}%
\begin{pgfscope}%
\pgfsys@transformshift{2.483863in}{2.041586in}%
\pgfsys@useobject{currentmarker}{}%
\end{pgfscope}%
\begin{pgfscope}%
\pgfsys@transformshift{2.486623in}{2.026154in}%
\pgfsys@useobject{currentmarker}{}%
\end{pgfscope}%
\begin{pgfscope}%
\pgfsys@transformshift{2.486872in}{2.017535in}%
\pgfsys@useobject{currentmarker}{}%
\end{pgfscope}%
\begin{pgfscope}%
\pgfsys@transformshift{2.490343in}{2.006691in}%
\pgfsys@useobject{currentmarker}{}%
\end{pgfscope}%
\begin{pgfscope}%
\pgfsys@transformshift{2.494003in}{1.995405in}%
\pgfsys@useobject{currentmarker}{}%
\end{pgfscope}%
\begin{pgfscope}%
\pgfsys@transformshift{2.495850in}{1.981396in}%
\pgfsys@useobject{currentmarker}{}%
\end{pgfscope}%
\begin{pgfscope}%
\pgfsys@transformshift{2.496102in}{1.973628in}%
\pgfsys@useobject{currentmarker}{}%
\end{pgfscope}%
\begin{pgfscope}%
\pgfsys@transformshift{2.499611in}{1.964707in}%
\pgfsys@useobject{currentmarker}{}%
\end{pgfscope}%
\begin{pgfscope}%
\pgfsys@transformshift{2.500563in}{1.954381in}%
\pgfsys@useobject{currentmarker}{}%
\end{pgfscope}%
\begin{pgfscope}%
\pgfsys@transformshift{2.500952in}{1.948692in}%
\pgfsys@useobject{currentmarker}{}%
\end{pgfscope}%
\begin{pgfscope}%
\pgfsys@transformshift{2.501874in}{1.941701in}%
\pgfsys@useobject{currentmarker}{}%
\end{pgfscope}%
\begin{pgfscope}%
\pgfsys@transformshift{2.503145in}{1.938037in}%
\pgfsys@useobject{currentmarker}{}%
\end{pgfscope}%
\begin{pgfscope}%
\pgfsys@transformshift{2.503209in}{1.932910in}%
\pgfsys@useobject{currentmarker}{}%
\end{pgfscope}%
\begin{pgfscope}%
\pgfsys@transformshift{2.503103in}{1.930092in}%
\pgfsys@useobject{currentmarker}{}%
\end{pgfscope}%
\begin{pgfscope}%
\pgfsys@transformshift{2.504412in}{1.924625in}%
\pgfsys@useobject{currentmarker}{}%
\end{pgfscope}%
\begin{pgfscope}%
\pgfsys@transformshift{2.505820in}{1.918697in}%
\pgfsys@useobject{currentmarker}{}%
\end{pgfscope}%
\begin{pgfscope}%
\pgfsys@transformshift{2.505439in}{1.909927in}%
\pgfsys@useobject{currentmarker}{}%
\end{pgfscope}%
\begin{pgfscope}%
\pgfsys@transformshift{2.506011in}{1.905133in}%
\pgfsys@useobject{currentmarker}{}%
\end{pgfscope}%
\begin{pgfscope}%
\pgfsys@transformshift{2.508757in}{1.899162in}%
\pgfsys@useobject{currentmarker}{}%
\end{pgfscope}%
\begin{pgfscope}%
\pgfsys@transformshift{2.509770in}{1.891640in}%
\pgfsys@useobject{currentmarker}{}%
\end{pgfscope}%
\begin{pgfscope}%
\pgfsys@transformshift{2.510546in}{1.887539in}%
\pgfsys@useobject{currentmarker}{}%
\end{pgfscope}%
\begin{pgfscope}%
\pgfsys@transformshift{2.512826in}{1.880105in}%
\pgfsys@useobject{currentmarker}{}%
\end{pgfscope}%
\begin{pgfscope}%
\pgfsys@transformshift{2.516907in}{1.872434in}%
\pgfsys@useobject{currentmarker}{}%
\end{pgfscope}%
\begin{pgfscope}%
\pgfsys@transformshift{2.517076in}{1.861332in}%
\pgfsys@useobject{currentmarker}{}%
\end{pgfscope}%
\begin{pgfscope}%
\pgfsys@transformshift{2.519585in}{1.849767in}%
\pgfsys@useobject{currentmarker}{}%
\end{pgfscope}%
\begin{pgfscope}%
\pgfsys@transformshift{2.524222in}{1.835461in}%
\pgfsys@useobject{currentmarker}{}%
\end{pgfscope}%
\begin{pgfscope}%
\pgfsys@transformshift{2.525864in}{1.827355in}%
\pgfsys@useobject{currentmarker}{}%
\end{pgfscope}%
\begin{pgfscope}%
\pgfsys@transformshift{2.525965in}{1.818151in}%
\pgfsys@useobject{currentmarker}{}%
\end{pgfscope}%
\begin{pgfscope}%
\pgfsys@transformshift{2.529972in}{1.806855in}%
\pgfsys@useobject{currentmarker}{}%
\end{pgfscope}%
\begin{pgfscope}%
\pgfsys@transformshift{2.532345in}{1.800705in}%
\pgfsys@useobject{currentmarker}{}%
\end{pgfscope}%
\begin{pgfscope}%
\pgfsys@transformshift{2.533023in}{1.791550in}%
\pgfsys@useobject{currentmarker}{}%
\end{pgfscope}%
\begin{pgfscope}%
\pgfsys@transformshift{2.532857in}{1.781582in}%
\pgfsys@useobject{currentmarker}{}%
\end{pgfscope}%
\begin{pgfscope}%
\pgfsys@transformshift{2.537492in}{1.770178in}%
\pgfsys@useobject{currentmarker}{}%
\end{pgfscope}%
\begin{pgfscope}%
\pgfsys@transformshift{2.538796in}{1.763533in}%
\pgfsys@useobject{currentmarker}{}%
\end{pgfscope}%
\begin{pgfscope}%
\pgfsys@transformshift{2.539302in}{1.754129in}%
\pgfsys@useobject{currentmarker}{}%
\end{pgfscope}%
\begin{pgfscope}%
\pgfsys@transformshift{2.541392in}{1.744007in}%
\pgfsys@useobject{currentmarker}{}%
\end{pgfscope}%
\begin{pgfscope}%
\pgfsys@transformshift{2.546867in}{1.733974in}%
\pgfsys@useobject{currentmarker}{}%
\end{pgfscope}%
\begin{pgfscope}%
\pgfsys@transformshift{2.547951in}{1.727782in}%
\pgfsys@useobject{currentmarker}{}%
\end{pgfscope}%
\begin{pgfscope}%
\pgfsys@transformshift{2.549662in}{1.720897in}%
\pgfsys@useobject{currentmarker}{}%
\end{pgfscope}%
\begin{pgfscope}%
\pgfsys@transformshift{2.551334in}{1.717371in}%
\pgfsys@useobject{currentmarker}{}%
\end{pgfscope}%
\begin{pgfscope}%
\pgfsys@transformshift{2.552808in}{1.712829in}%
\pgfsys@useobject{currentmarker}{}%
\end{pgfscope}%
\begin{pgfscope}%
\pgfsys@transformshift{2.555398in}{1.707972in}%
\pgfsys@useobject{currentmarker}{}%
\end{pgfscope}%
\begin{pgfscope}%
\pgfsys@transformshift{2.556209in}{1.701889in}%
\pgfsys@useobject{currentmarker}{}%
\end{pgfscope}%
\begin{pgfscope}%
\pgfsys@transformshift{2.558982in}{1.695517in}%
\pgfsys@useobject{currentmarker}{}%
\end{pgfscope}%
\begin{pgfscope}%
\pgfsys@transformshift{2.561057in}{1.688160in}%
\pgfsys@useobject{currentmarker}{}%
\end{pgfscope}%
\begin{pgfscope}%
\pgfsys@transformshift{2.562665in}{1.684275in}%
\pgfsys@useobject{currentmarker}{}%
\end{pgfscope}%
\begin{pgfscope}%
\pgfsys@transformshift{2.562823in}{1.679605in}%
\pgfsys@useobject{currentmarker}{}%
\end{pgfscope}%
\begin{pgfscope}%
\pgfsys@transformshift{2.564214in}{1.673571in}%
\pgfsys@useobject{currentmarker}{}%
\end{pgfscope}%
\begin{pgfscope}%
\pgfsys@transformshift{2.565006in}{1.670259in}%
\pgfsys@useobject{currentmarker}{}%
\end{pgfscope}%
\begin{pgfscope}%
\pgfsys@transformshift{2.566045in}{1.666152in}%
\pgfsys@useobject{currentmarker}{}%
\end{pgfscope}%
\begin{pgfscope}%
\pgfsys@transformshift{2.566157in}{1.663826in}%
\pgfsys@useobject{currentmarker}{}%
\end{pgfscope}%
\begin{pgfscope}%
\pgfsys@transformshift{2.566203in}{1.662545in}%
\pgfsys@useobject{currentmarker}{}%
\end{pgfscope}%
\begin{pgfscope}%
\pgfsys@transformshift{2.567485in}{1.658764in}%
\pgfsys@useobject{currentmarker}{}%
\end{pgfscope}%
\begin{pgfscope}%
\pgfsys@transformshift{2.567643in}{1.656573in}%
\pgfsys@useobject{currentmarker}{}%
\end{pgfscope}%
\begin{pgfscope}%
\pgfsys@transformshift{2.567245in}{1.652452in}%
\pgfsys@useobject{currentmarker}{}%
\end{pgfscope}%
\begin{pgfscope}%
\pgfsys@transformshift{2.567634in}{1.647586in}%
\pgfsys@useobject{currentmarker}{}%
\end{pgfscope}%
\begin{pgfscope}%
\pgfsys@transformshift{2.568085in}{1.644939in}%
\pgfsys@useobject{currentmarker}{}%
\end{pgfscope}%
\begin{pgfscope}%
\pgfsys@transformshift{2.568577in}{1.643547in}%
\pgfsys@useobject{currentmarker}{}%
\end{pgfscope}%
\begin{pgfscope}%
\pgfsys@transformshift{2.568264in}{1.639449in}%
\pgfsys@useobject{currentmarker}{}%
\end{pgfscope}%
\begin{pgfscope}%
\pgfsys@transformshift{2.568433in}{1.634817in}%
\pgfsys@useobject{currentmarker}{}%
\end{pgfscope}%
\begin{pgfscope}%
\pgfsys@transformshift{2.569965in}{1.629963in}%
\pgfsys@useobject{currentmarker}{}%
\end{pgfscope}%
\begin{pgfscope}%
\pgfsys@transformshift{2.570821in}{1.624040in}%
\pgfsys@useobject{currentmarker}{}%
\end{pgfscope}%
\begin{pgfscope}%
\pgfsys@transformshift{2.570435in}{1.617218in}%
\pgfsys@useobject{currentmarker}{}%
\end{pgfscope}%
\begin{pgfscope}%
\pgfsys@transformshift{2.570309in}{1.613462in}%
\pgfsys@useobject{currentmarker}{}%
\end{pgfscope}%
\begin{pgfscope}%
\pgfsys@transformshift{2.572318in}{1.607714in}%
\pgfsys@useobject{currentmarker}{}%
\end{pgfscope}%
\begin{pgfscope}%
\pgfsys@transformshift{2.573796in}{1.601133in}%
\pgfsys@useobject{currentmarker}{}%
\end{pgfscope}%
\begin{pgfscope}%
\pgfsys@transformshift{2.572989in}{1.593506in}%
\pgfsys@useobject{currentmarker}{}%
\end{pgfscope}%
\begin{pgfscope}%
\pgfsys@transformshift{2.572724in}{1.589296in}%
\pgfsys@useobject{currentmarker}{}%
\end{pgfscope}%
\begin{pgfscope}%
\pgfsys@transformshift{2.574661in}{1.583662in}%
\pgfsys@useobject{currentmarker}{}%
\end{pgfscope}%
\begin{pgfscope}%
\pgfsys@transformshift{2.576404in}{1.577415in}%
\pgfsys@useobject{currentmarker}{}%
\end{pgfscope}%
\begin{pgfscope}%
\pgfsys@transformshift{2.576953in}{1.568055in}%
\pgfsys@useobject{currentmarker}{}%
\end{pgfscope}%
\begin{pgfscope}%
\pgfsys@transformshift{2.577283in}{1.562908in}%
\pgfsys@useobject{currentmarker}{}%
\end{pgfscope}%
\begin{pgfscope}%
\pgfsys@transformshift{2.580046in}{1.555133in}%
\pgfsys@useobject{currentmarker}{}%
\end{pgfscope}%
\begin{pgfscope}%
\pgfsys@transformshift{2.580955in}{1.550687in}%
\pgfsys@useobject{currentmarker}{}%
\end{pgfscope}%
\begin{pgfscope}%
\pgfsys@transformshift{2.582127in}{1.543973in}%
\pgfsys@useobject{currentmarker}{}%
\end{pgfscope}%
\begin{pgfscope}%
\pgfsys@transformshift{2.582286in}{1.540227in}%
\pgfsys@useobject{currentmarker}{}%
\end{pgfscope}%
\begin{pgfscope}%
\pgfsys@transformshift{2.584457in}{1.534626in}%
\pgfsys@useobject{currentmarker}{}%
\end{pgfscope}%
\begin{pgfscope}%
\pgfsys@transformshift{2.584971in}{1.528011in}%
\pgfsys@useobject{currentmarker}{}%
\end{pgfscope}%
\begin{pgfscope}%
\pgfsys@transformshift{2.584903in}{1.517815in}%
\pgfsys@useobject{currentmarker}{}%
\end{pgfscope}%
\begin{pgfscope}%
\pgfsys@transformshift{2.583466in}{1.506209in}%
\pgfsys@useobject{currentmarker}{}%
\end{pgfscope}%
\begin{pgfscope}%
\pgfsys@transformshift{2.585397in}{1.493483in}%
\pgfsys@useobject{currentmarker}{}%
\end{pgfscope}%
\begin{pgfscope}%
\pgfsys@transformshift{2.587195in}{1.480046in}%
\pgfsys@useobject{currentmarker}{}%
\end{pgfscope}%
\begin{pgfscope}%
\pgfsys@transformshift{2.586219in}{1.464825in}%
\pgfsys@useobject{currentmarker}{}%
\end{pgfscope}%
\begin{pgfscope}%
\pgfsys@transformshift{2.585286in}{1.456488in}%
\pgfsys@useobject{currentmarker}{}%
\end{pgfscope}%
\begin{pgfscope}%
\pgfsys@transformshift{2.586883in}{1.446044in}%
\pgfsys@useobject{currentmarker}{}%
\end{pgfscope}%
\begin{pgfscope}%
\pgfsys@transformshift{2.588657in}{1.440511in}%
\pgfsys@useobject{currentmarker}{}%
\end{pgfscope}%
\begin{pgfscope}%
\pgfsys@transformshift{2.588170in}{1.433240in}%
\pgfsys@useobject{currentmarker}{}%
\end{pgfscope}%
\begin{pgfscope}%
\pgfsys@transformshift{2.586748in}{1.425373in}%
\pgfsys@useobject{currentmarker}{}%
\end{pgfscope}%
\begin{pgfscope}%
\pgfsys@transformshift{2.585426in}{1.417001in}%
\pgfsys@useobject{currentmarker}{}%
\end{pgfscope}%
\begin{pgfscope}%
\pgfsys@transformshift{2.587544in}{1.406690in}%
\pgfsys@useobject{currentmarker}{}%
\end{pgfscope}%
\begin{pgfscope}%
\pgfsys@transformshift{2.587259in}{1.395710in}%
\pgfsys@useobject{currentmarker}{}%
\end{pgfscope}%
\begin{pgfscope}%
\pgfsys@transformshift{2.585053in}{1.384338in}%
\pgfsys@useobject{currentmarker}{}%
\end{pgfscope}%
\begin{pgfscope}%
\pgfsys@transformshift{2.584857in}{1.372139in}%
\pgfsys@useobject{currentmarker}{}%
\end{pgfscope}%
\begin{pgfscope}%
\pgfsys@transformshift{2.586948in}{1.356358in}%
\pgfsys@useobject{currentmarker}{}%
\end{pgfscope}%
\begin{pgfscope}%
\pgfsys@transformshift{2.588371in}{1.347719in}%
\pgfsys@useobject{currentmarker}{}%
\end{pgfscope}%
\begin{pgfscope}%
\pgfsys@transformshift{2.586963in}{1.336938in}%
\pgfsys@useobject{currentmarker}{}%
\end{pgfscope}%
\begin{pgfscope}%
\pgfsys@transformshift{2.586327in}{1.330992in}%
\pgfsys@useobject{currentmarker}{}%
\end{pgfscope}%
\begin{pgfscope}%
\pgfsys@transformshift{2.585694in}{1.322199in}%
\pgfsys@useobject{currentmarker}{}%
\end{pgfscope}%
\begin{pgfscope}%
\pgfsys@transformshift{2.588550in}{1.312679in}%
\pgfsys@useobject{currentmarker}{}%
\end{pgfscope}%
\begin{pgfscope}%
\pgfsys@transformshift{2.586918in}{1.300849in}%
\pgfsys@useobject{currentmarker}{}%
\end{pgfscope}%
\begin{pgfscope}%
\pgfsys@transformshift{2.585049in}{1.288328in}%
\pgfsys@useobject{currentmarker}{}%
\end{pgfscope}%
\begin{pgfscope}%
\pgfsys@transformshift{2.583643in}{1.281508in}%
\pgfsys@useobject{currentmarker}{}%
\end{pgfscope}%
\begin{pgfscope}%
\pgfsys@transformshift{2.585863in}{1.272582in}%
\pgfsys@useobject{currentmarker}{}%
\end{pgfscope}%
\begin{pgfscope}%
\pgfsys@transformshift{2.586941in}{1.262691in}%
\pgfsys@useobject{currentmarker}{}%
\end{pgfscope}%
\begin{pgfscope}%
\pgfsys@transformshift{2.586426in}{1.251964in}%
\pgfsys@useobject{currentmarker}{}%
\end{pgfscope}%
\begin{pgfscope}%
\pgfsys@transformshift{2.585278in}{1.240524in}%
\pgfsys@useobject{currentmarker}{}%
\end{pgfscope}%
\begin{pgfscope}%
\pgfsys@transformshift{2.582170in}{1.226251in}%
\pgfsys@useobject{currentmarker}{}%
\end{pgfscope}%
\begin{pgfscope}%
\pgfsys@transformshift{2.585433in}{1.211431in}%
\pgfsys@useobject{currentmarker}{}%
\end{pgfscope}%
\begin{pgfscope}%
\pgfsys@transformshift{2.590245in}{1.196421in}%
\pgfsys@useobject{currentmarker}{}%
\end{pgfscope}%
\begin{pgfscope}%
\pgfsys@transformshift{2.588499in}{1.177585in}%
\pgfsys@useobject{currentmarker}{}%
\end{pgfscope}%
\begin{pgfscope}%
\pgfsys@transformshift{2.587022in}{1.158140in}%
\pgfsys@useobject{currentmarker}{}%
\end{pgfscope}%
\begin{pgfscope}%
\pgfsys@transformshift{2.583001in}{1.138227in}%
\pgfsys@useobject{currentmarker}{}%
\end{pgfscope}%
\begin{pgfscope}%
\pgfsys@transformshift{2.589231in}{1.117412in}%
\pgfsys@useobject{currentmarker}{}%
\end{pgfscope}%
\begin{pgfscope}%
\pgfsys@transformshift{2.586755in}{1.094583in}%
\pgfsys@useobject{currentmarker}{}%
\end{pgfscope}%
\begin{pgfscope}%
\pgfsys@transformshift{2.586003in}{1.081976in}%
\pgfsys@useobject{currentmarker}{}%
\end{pgfscope}%
\begin{pgfscope}%
\pgfsys@transformshift{2.585305in}{1.075064in}%
\pgfsys@useobject{currentmarker}{}%
\end{pgfscope}%
\begin{pgfscope}%
\pgfsys@transformshift{2.585623in}{1.066757in}%
\pgfsys@useobject{currentmarker}{}%
\end{pgfscope}%
\begin{pgfscope}%
\pgfsys@transformshift{2.587019in}{1.058084in}%
\pgfsys@useobject{currentmarker}{}%
\end{pgfscope}%
\begin{pgfscope}%
\pgfsys@transformshift{2.585540in}{1.047718in}%
\pgfsys@useobject{currentmarker}{}%
\end{pgfscope}%
\begin{pgfscope}%
\pgfsys@transformshift{2.585364in}{1.036736in}%
\pgfsys@useobject{currentmarker}{}%
\end{pgfscope}%
\begin{pgfscope}%
\pgfsys@transformshift{2.584079in}{1.030833in}%
\pgfsys@useobject{currentmarker}{}%
\end{pgfscope}%
\begin{pgfscope}%
\pgfsys@transformshift{2.584236in}{1.027514in}%
\pgfsys@useobject{currentmarker}{}%
\end{pgfscope}%
\begin{pgfscope}%
\pgfsys@transformshift{2.583474in}{1.023794in}%
\pgfsys@useobject{currentmarker}{}%
\end{pgfscope}%
\begin{pgfscope}%
\pgfsys@transformshift{2.584958in}{1.018490in}%
\pgfsys@useobject{currentmarker}{}%
\end{pgfscope}%
\begin{pgfscope}%
\pgfsys@transformshift{2.584721in}{1.012482in}%
\pgfsys@useobject{currentmarker}{}%
\end{pgfscope}%
\begin{pgfscope}%
\pgfsys@transformshift{2.583980in}{1.005004in}%
\pgfsys@useobject{currentmarker}{}%
\end{pgfscope}%
\begin{pgfscope}%
\pgfsys@transformshift{2.583366in}{1.000917in}%
\pgfsys@useobject{currentmarker}{}%
\end{pgfscope}%
\begin{pgfscope}%
\pgfsys@transformshift{2.583250in}{0.996239in}%
\pgfsys@useobject{currentmarker}{}%
\end{pgfscope}%
\begin{pgfscope}%
\pgfsys@transformshift{2.582932in}{0.991066in}%
\pgfsys@useobject{currentmarker}{}%
\end{pgfscope}%
\begin{pgfscope}%
\pgfsys@transformshift{2.582817in}{0.988219in}%
\pgfsys@useobject{currentmarker}{}%
\end{pgfscope}%
\begin{pgfscope}%
\pgfsys@transformshift{2.582892in}{0.986653in}%
\pgfsys@useobject{currentmarker}{}%
\end{pgfscope}%
\begin{pgfscope}%
\pgfsys@transformshift{2.583525in}{0.982919in}%
\pgfsys@useobject{currentmarker}{}%
\end{pgfscope}%
\begin{pgfscope}%
\pgfsys@transformshift{2.584139in}{0.978716in}%
\pgfsys@useobject{currentmarker}{}%
\end{pgfscope}%
\begin{pgfscope}%
\pgfsys@transformshift{2.583446in}{0.976485in}%
\pgfsys@useobject{currentmarker}{}%
\end{pgfscope}%
\begin{pgfscope}%
\pgfsys@transformshift{2.582604in}{0.975515in}%
\pgfsys@useobject{currentmarker}{}%
\end{pgfscope}%
\begin{pgfscope}%
\pgfsys@transformshift{2.580037in}{0.974618in}%
\pgfsys@useobject{currentmarker}{}%
\end{pgfscope}%
\begin{pgfscope}%
\pgfsys@transformshift{2.575434in}{0.973132in}%
\pgfsys@useobject{currentmarker}{}%
\end{pgfscope}%
\begin{pgfscope}%
\pgfsys@transformshift{2.569037in}{0.972548in}%
\pgfsys@useobject{currentmarker}{}%
\end{pgfscope}%
\begin{pgfscope}%
\pgfsys@transformshift{2.562114in}{0.969885in}%
\pgfsys@useobject{currentmarker}{}%
\end{pgfscope}%
\begin{pgfscope}%
\pgfsys@transformshift{2.553721in}{0.970777in}%
\pgfsys@useobject{currentmarker}{}%
\end{pgfscope}%
\begin{pgfscope}%
\pgfsys@transformshift{2.544745in}{0.969658in}%
\pgfsys@useobject{currentmarker}{}%
\end{pgfscope}%
\begin{pgfscope}%
\pgfsys@transformshift{2.534581in}{0.970645in}%
\pgfsys@useobject{currentmarker}{}%
\end{pgfscope}%
\begin{pgfscope}%
\pgfsys@transformshift{2.523807in}{0.970941in}%
\pgfsys@useobject{currentmarker}{}%
\end{pgfscope}%
\begin{pgfscope}%
\pgfsys@transformshift{2.512283in}{0.971898in}%
\pgfsys@useobject{currentmarker}{}%
\end{pgfscope}%
\begin{pgfscope}%
\pgfsys@transformshift{2.505943in}{0.972398in}%
\pgfsys@useobject{currentmarker}{}%
\end{pgfscope}%
\begin{pgfscope}%
\pgfsys@transformshift{2.502461in}{0.972732in}%
\pgfsys@useobject{currentmarker}{}%
\end{pgfscope}%
\begin{pgfscope}%
\pgfsys@transformshift{2.500544in}{0.972893in}%
\pgfsys@useobject{currentmarker}{}%
\end{pgfscope}%
\begin{pgfscope}%
\pgfsys@transformshift{2.499487in}{0.972944in}%
\pgfsys@useobject{currentmarker}{}%
\end{pgfscope}%
\begin{pgfscope}%
\pgfsys@transformshift{2.498909in}{0.973010in}%
\pgfsys@useobject{currentmarker}{}%
\end{pgfscope}%
\begin{pgfscope}%
\pgfsys@transformshift{2.495425in}{0.972731in}%
\pgfsys@useobject{currentmarker}{}%
\end{pgfscope}%
\begin{pgfscope}%
\pgfsys@transformshift{2.491249in}{0.972549in}%
\pgfsys@useobject{currentmarker}{}%
\end{pgfscope}%
\begin{pgfscope}%
\pgfsys@transformshift{2.485661in}{0.971809in}%
\pgfsys@useobject{currentmarker}{}%
\end{pgfscope}%
\begin{pgfscope}%
\pgfsys@transformshift{2.479330in}{0.972692in}%
\pgfsys@useobject{currentmarker}{}%
\end{pgfscope}%
\begin{pgfscope}%
\pgfsys@transformshift{2.468893in}{0.971939in}%
\pgfsys@useobject{currentmarker}{}%
\end{pgfscope}%
\begin{pgfscope}%
\pgfsys@transformshift{2.455434in}{0.971640in}%
\pgfsys@useobject{currentmarker}{}%
\end{pgfscope}%
\begin{pgfscope}%
\pgfsys@transformshift{2.439777in}{0.969676in}%
\pgfsys@useobject{currentmarker}{}%
\end{pgfscope}%
\begin{pgfscope}%
\pgfsys@transformshift{2.423395in}{0.970028in}%
\pgfsys@useobject{currentmarker}{}%
\end{pgfscope}%
\begin{pgfscope}%
\pgfsys@transformshift{2.403723in}{0.969264in}%
\pgfsys@useobject{currentmarker}{}%
\end{pgfscope}%
\begin{pgfscope}%
\pgfsys@transformshift{2.383272in}{0.968462in}%
\pgfsys@useobject{currentmarker}{}%
\end{pgfscope}%
\begin{pgfscope}%
\pgfsys@transformshift{2.360415in}{0.965860in}%
\pgfsys@useobject{currentmarker}{}%
\end{pgfscope}%
\begin{pgfscope}%
\pgfsys@transformshift{2.347783in}{0.965131in}%
\pgfsys@useobject{currentmarker}{}%
\end{pgfscope}%
\begin{pgfscope}%
\pgfsys@transformshift{2.331617in}{0.965843in}%
\pgfsys@useobject{currentmarker}{}%
\end{pgfscope}%
\begin{pgfscope}%
\pgfsys@transformshift{2.314275in}{0.964833in}%
\pgfsys@useobject{currentmarker}{}%
\end{pgfscope}%
\begin{pgfscope}%
\pgfsys@transformshift{2.294573in}{0.962072in}%
\pgfsys@useobject{currentmarker}{}%
\end{pgfscope}%
\begin{pgfscope}%
\pgfsys@transformshift{2.272646in}{0.959338in}%
\pgfsys@useobject{currentmarker}{}%
\end{pgfscope}%
\begin{pgfscope}%
\pgfsys@transformshift{2.249603in}{0.960320in}%
\pgfsys@useobject{currentmarker}{}%
\end{pgfscope}%
\begin{pgfscope}%
\pgfsys@transformshift{2.223591in}{0.959070in}%
\pgfsys@useobject{currentmarker}{}%
\end{pgfscope}%
\begin{pgfscope}%
\pgfsys@transformshift{2.196767in}{0.956803in}%
\pgfsys@useobject{currentmarker}{}%
\end{pgfscope}%
\begin{pgfscope}%
\pgfsys@transformshift{2.167266in}{0.956397in}%
\pgfsys@useobject{currentmarker}{}%
\end{pgfscope}%
\begin{pgfscope}%
\pgfsys@transformshift{2.151081in}{0.957570in}%
\pgfsys@useobject{currentmarker}{}%
\end{pgfscope}%
\begin{pgfscope}%
\pgfsys@transformshift{2.132113in}{0.956792in}%
\pgfsys@useobject{currentmarker}{}%
\end{pgfscope}%
\begin{pgfscope}%
\pgfsys@transformshift{2.111828in}{0.955196in}%
\pgfsys@useobject{currentmarker}{}%
\end{pgfscope}%
\begin{pgfscope}%
\pgfsys@transformshift{2.089545in}{0.952303in}%
\pgfsys@useobject{currentmarker}{}%
\end{pgfscope}%
\begin{pgfscope}%
\pgfsys@transformshift{2.065299in}{0.953823in}%
\pgfsys@useobject{currentmarker}{}%
\end{pgfscope}%
\begin{pgfscope}%
\pgfsys@transformshift{2.039914in}{0.950769in}%
\pgfsys@useobject{currentmarker}{}%
\end{pgfscope}%
\begin{pgfscope}%
\pgfsys@transformshift{2.012796in}{0.950743in}%
\pgfsys@useobject{currentmarker}{}%
\end{pgfscope}%
\begin{pgfscope}%
\pgfsys@transformshift{1.985249in}{0.944935in}%
\pgfsys@useobject{currentmarker}{}%
\end{pgfscope}%
\begin{pgfscope}%
\pgfsys@transformshift{1.956090in}{0.944536in}%
\pgfsys@useobject{currentmarker}{}%
\end{pgfscope}%
\begin{pgfscope}%
\pgfsys@transformshift{1.925545in}{0.944756in}%
\pgfsys@useobject{currentmarker}{}%
\end{pgfscope}%
\begin{pgfscope}%
\pgfsys@transformshift{1.892608in}{0.944697in}%
\pgfsys@useobject{currentmarker}{}%
\end{pgfscope}%
\begin{pgfscope}%
\pgfsys@transformshift{1.858104in}{0.941895in}%
\pgfsys@useobject{currentmarker}{}%
\end{pgfscope}%
\begin{pgfscope}%
\pgfsys@transformshift{1.822104in}{0.943732in}%
\pgfsys@useobject{currentmarker}{}%
\end{pgfscope}%
\begin{pgfscope}%
\pgfsys@transformshift{1.783839in}{0.943749in}%
\pgfsys@useobject{currentmarker}{}%
\end{pgfscope}%
\begin{pgfscope}%
\pgfsys@transformshift{1.744586in}{0.940652in}%
\pgfsys@useobject{currentmarker}{}%
\end{pgfscope}%
\begin{pgfscope}%
\pgfsys@transformshift{1.703047in}{0.933945in}%
\pgfsys@useobject{currentmarker}{}%
\end{pgfscope}%
\begin{pgfscope}%
\pgfsys@transformshift{1.658380in}{0.933101in}%
\pgfsys@useobject{currentmarker}{}%
\end{pgfscope}%
\begin{pgfscope}%
\pgfsys@transformshift{1.611017in}{0.935056in}%
\pgfsys@useobject{currentmarker}{}%
\end{pgfscope}%
\begin{pgfscope}%
\pgfsys@transformshift{1.562626in}{0.930276in}%
\pgfsys@useobject{currentmarker}{}%
\end{pgfscope}%
\begin{pgfscope}%
\pgfsys@transformshift{1.513138in}{0.926729in}%
\pgfsys@useobject{currentmarker}{}%
\end{pgfscope}%
\begin{pgfscope}%
\pgfsys@transformshift{1.460933in}{0.921745in}%
\pgfsys@useobject{currentmarker}{}%
\end{pgfscope}%
\begin{pgfscope}%
\pgfsys@transformshift{1.407937in}{0.922703in}%
\pgfsys@useobject{currentmarker}{}%
\end{pgfscope}%
\begin{pgfscope}%
\pgfsys@transformshift{1.355705in}{0.937050in}%
\pgfsys@useobject{currentmarker}{}%
\end{pgfscope}%
\begin{pgfscope}%
\pgfsys@transformshift{1.303858in}{0.961151in}%
\pgfsys@useobject{currentmarker}{}%
\end{pgfscope}%
\begin{pgfscope}%
\pgfsys@transformshift{1.263409in}{1.004115in}%
\pgfsys@useobject{currentmarker}{}%
\end{pgfscope}%
\begin{pgfscope}%
\pgfsys@transformshift{1.260641in}{1.036451in}%
\pgfsys@useobject{currentmarker}{}%
\end{pgfscope}%
\begin{pgfscope}%
\pgfsys@transformshift{1.260899in}{1.070463in}%
\pgfsys@useobject{currentmarker}{}%
\end{pgfscope}%
\begin{pgfscope}%
\pgfsys@transformshift{1.265291in}{1.107064in}%
\pgfsys@useobject{currentmarker}{}%
\end{pgfscope}%
\begin{pgfscope}%
\pgfsys@transformshift{1.274493in}{1.145106in}%
\pgfsys@useobject{currentmarker}{}%
\end{pgfscope}%
\begin{pgfscope}%
\pgfsys@transformshift{1.282024in}{1.184290in}%
\pgfsys@useobject{currentmarker}{}%
\end{pgfscope}%
\begin{pgfscope}%
\pgfsys@transformshift{1.286752in}{1.225269in}%
\pgfsys@useobject{currentmarker}{}%
\end{pgfscope}%
\begin{pgfscope}%
\pgfsys@transformshift{1.289061in}{1.267793in}%
\pgfsys@useobject{currentmarker}{}%
\end{pgfscope}%
\begin{pgfscope}%
\pgfsys@transformshift{1.291312in}{1.311949in}%
\pgfsys@useobject{currentmarker}{}%
\end{pgfscope}%
\begin{pgfscope}%
\pgfsys@transformshift{1.283729in}{1.335053in}%
\pgfsys@useobject{currentmarker}{}%
\end{pgfscope}%
\begin{pgfscope}%
\pgfsys@transformshift{1.288217in}{1.361371in}%
\pgfsys@useobject{currentmarker}{}%
\end{pgfscope}%
\begin{pgfscope}%
\pgfsys@transformshift{1.286175in}{1.388674in}%
\pgfsys@useobject{currentmarker}{}%
\end{pgfscope}%
\begin{pgfscope}%
\pgfsys@transformshift{1.276434in}{1.415072in}%
\pgfsys@useobject{currentmarker}{}%
\end{pgfscope}%
\begin{pgfscope}%
\pgfsys@transformshift{1.277226in}{1.430528in}%
\pgfsys@useobject{currentmarker}{}%
\end{pgfscope}%
\begin{pgfscope}%
\pgfsys@transformshift{1.280762in}{1.446980in}%
\pgfsys@useobject{currentmarker}{}%
\end{pgfscope}%
\begin{pgfscope}%
\pgfsys@transformshift{1.274124in}{1.467329in}%
\pgfsys@useobject{currentmarker}{}%
\end{pgfscope}%
\begin{pgfscope}%
\pgfsys@transformshift{1.274410in}{1.479097in}%
\pgfsys@useobject{currentmarker}{}%
\end{pgfscope}%
\begin{pgfscope}%
\pgfsys@transformshift{1.278724in}{1.494073in}%
\pgfsys@useobject{currentmarker}{}%
\end{pgfscope}%
\begin{pgfscope}%
\pgfsys@transformshift{1.275391in}{1.510248in}%
\pgfsys@useobject{currentmarker}{}%
\end{pgfscope}%
\begin{pgfscope}%
\pgfsys@transformshift{1.275204in}{1.519329in}%
\pgfsys@useobject{currentmarker}{}%
\end{pgfscope}%
\begin{pgfscope}%
\pgfsys@transformshift{1.273326in}{1.528996in}%
\pgfsys@useobject{currentmarker}{}%
\end{pgfscope}%
\begin{pgfscope}%
\pgfsys@transformshift{1.275605in}{1.539234in}%
\pgfsys@useobject{currentmarker}{}%
\end{pgfscope}%
\begin{pgfscope}%
\pgfsys@transformshift{1.270669in}{1.552326in}%
\pgfsys@useobject{currentmarker}{}%
\end{pgfscope}%
\begin{pgfscope}%
\pgfsys@transformshift{1.270571in}{1.560021in}%
\pgfsys@useobject{currentmarker}{}%
\end{pgfscope}%
\begin{pgfscope}%
\pgfsys@transformshift{1.267618in}{1.569333in}%
\pgfsys@useobject{currentmarker}{}%
\end{pgfscope}%
\begin{pgfscope}%
\pgfsys@transformshift{1.268785in}{1.574578in}%
\pgfsys@useobject{currentmarker}{}%
\end{pgfscope}%
\begin{pgfscope}%
\pgfsys@transformshift{1.266339in}{1.583068in}%
\pgfsys@useobject{currentmarker}{}%
\end{pgfscope}%
\begin{pgfscope}%
\pgfsys@transformshift{1.265792in}{1.587896in}%
\pgfsys@useobject{currentmarker}{}%
\end{pgfscope}%
\begin{pgfscope}%
\pgfsys@transformshift{1.267518in}{1.596225in}%
\pgfsys@useobject{currentmarker}{}%
\end{pgfscope}%
\begin{pgfscope}%
\pgfsys@transformshift{1.266728in}{1.605217in}%
\pgfsys@useobject{currentmarker}{}%
\end{pgfscope}%
\begin{pgfscope}%
\pgfsys@transformshift{1.263974in}{1.614512in}%
\pgfsys@useobject{currentmarker}{}%
\end{pgfscope}%
\begin{pgfscope}%
\pgfsys@transformshift{1.264110in}{1.619842in}%
\pgfsys@useobject{currentmarker}{}%
\end{pgfscope}%
\begin{pgfscope}%
\pgfsys@transformshift{1.264602in}{1.627500in}%
\pgfsys@useobject{currentmarker}{}%
\end{pgfscope}%
\begin{pgfscope}%
\pgfsys@transformshift{1.266340in}{1.635441in}%
\pgfsys@useobject{currentmarker}{}%
\end{pgfscope}%
\begin{pgfscope}%
\pgfsys@transformshift{1.263123in}{1.646799in}%
\pgfsys@useobject{currentmarker}{}%
\end{pgfscope}%
\begin{pgfscope}%
\pgfsys@transformshift{1.262891in}{1.653288in}%
\pgfsys@useobject{currentmarker}{}%
\end{pgfscope}%
\begin{pgfscope}%
\pgfsys@transformshift{1.264196in}{1.663568in}%
\pgfsys@useobject{currentmarker}{}%
\end{pgfscope}%
\begin{pgfscope}%
\pgfsys@transformshift{1.265370in}{1.674620in}%
\pgfsys@useobject{currentmarker}{}%
\end{pgfscope}%
\begin{pgfscope}%
\pgfsys@transformshift{1.262577in}{1.689185in}%
\pgfsys@useobject{currentmarker}{}%
\end{pgfscope}%
\begin{pgfscope}%
\pgfsys@transformshift{1.262783in}{1.697339in}%
\pgfsys@useobject{currentmarker}{}%
\end{pgfscope}%
\begin{pgfscope}%
\pgfsys@transformshift{1.260086in}{1.707440in}%
\pgfsys@useobject{currentmarker}{}%
\end{pgfscope}%
\begin{pgfscope}%
\pgfsys@transformshift{1.261330in}{1.713054in}%
\pgfsys@useobject{currentmarker}{}%
\end{pgfscope}%
\begin{pgfscope}%
\pgfsys@transformshift{1.259076in}{1.722036in}%
\pgfsys@useobject{currentmarker}{}%
\end{pgfscope}%
\begin{pgfscope}%
\pgfsys@transformshift{1.258062in}{1.727028in}%
\pgfsys@useobject{currentmarker}{}%
\end{pgfscope}%
\begin{pgfscope}%
\pgfsys@transformshift{1.256245in}{1.734570in}%
\pgfsys@useobject{currentmarker}{}%
\end{pgfscope}%
\begin{pgfscope}%
\pgfsys@transformshift{1.259246in}{1.744402in}%
\pgfsys@useobject{currentmarker}{}%
\end{pgfscope}%
\begin{pgfscope}%
\pgfsys@transformshift{1.256167in}{1.757808in}%
\pgfsys@useobject{currentmarker}{}%
\end{pgfscope}%
\begin{pgfscope}%
\pgfsys@transformshift{1.253218in}{1.771902in}%
\pgfsys@useobject{currentmarker}{}%
\end{pgfscope}%
\begin{pgfscope}%
\pgfsys@transformshift{1.251746in}{1.779683in}%
\pgfsys@useobject{currentmarker}{}%
\end{pgfscope}%
\begin{pgfscope}%
\pgfsys@transformshift{1.253695in}{1.789731in}%
\pgfsys@useobject{currentmarker}{}%
\end{pgfscope}%
\begin{pgfscope}%
\pgfsys@transformshift{1.253305in}{1.795347in}%
\pgfsys@useobject{currentmarker}{}%
\end{pgfscope}%
\begin{pgfscope}%
\pgfsys@transformshift{1.251557in}{1.802200in}%
\pgfsys@useobject{currentmarker}{}%
\end{pgfscope}%
\begin{pgfscope}%
\pgfsys@transformshift{1.251170in}{1.806071in}%
\pgfsys@useobject{currentmarker}{}%
\end{pgfscope}%
\begin{pgfscope}%
\pgfsys@transformshift{1.250714in}{1.813184in}%
\pgfsys@useobject{currentmarker}{}%
\end{pgfscope}%
\begin{pgfscope}%
\pgfsys@transformshift{1.251549in}{1.817014in}%
\pgfsys@useobject{currentmarker}{}%
\end{pgfscope}%
\begin{pgfscope}%
\pgfsys@transformshift{1.249275in}{1.825144in}%
\pgfsys@useobject{currentmarker}{}%
\end{pgfscope}%
\begin{pgfscope}%
\pgfsys@transformshift{1.249005in}{1.829780in}%
\pgfsys@useobject{currentmarker}{}%
\end{pgfscope}%
\begin{pgfscope}%
\pgfsys@transformshift{1.248086in}{1.837535in}%
\pgfsys@useobject{currentmarker}{}%
\end{pgfscope}%
\begin{pgfscope}%
\pgfsys@transformshift{1.249373in}{1.841633in}%
\pgfsys@useobject{currentmarker}{}%
\end{pgfscope}%
\begin{pgfscope}%
\pgfsys@transformshift{1.247369in}{1.850658in}%
\pgfsys@useobject{currentmarker}{}%
\end{pgfscope}%
\begin{pgfscope}%
\pgfsys@transformshift{1.247065in}{1.855733in}%
\pgfsys@useobject{currentmarker}{}%
\end{pgfscope}%
\begin{pgfscope}%
\pgfsys@transformshift{1.244464in}{1.864249in}%
\pgfsys@useobject{currentmarker}{}%
\end{pgfscope}%
\begin{pgfscope}%
\pgfsys@transformshift{1.247240in}{1.874733in}%
\pgfsys@useobject{currentmarker}{}%
\end{pgfscope}%
\begin{pgfscope}%
\pgfsys@transformshift{1.244686in}{1.889819in}%
\pgfsys@useobject{currentmarker}{}%
\end{pgfscope}%
\begin{pgfscope}%
\pgfsys@transformshift{1.247559in}{1.906367in}%
\pgfsys@useobject{currentmarker}{}%
\end{pgfscope}%
\begin{pgfscope}%
\pgfsys@transformshift{1.254101in}{1.922576in}%
\pgfsys@useobject{currentmarker}{}%
\end{pgfscope}%
\begin{pgfscope}%
\pgfsys@transformshift{1.250301in}{1.941555in}%
\pgfsys@useobject{currentmarker}{}%
\end{pgfscope}%
\begin{pgfscope}%
\pgfsys@transformshift{1.251529in}{1.961733in}%
\pgfsys@useobject{currentmarker}{}%
\end{pgfscope}%
\begin{pgfscope}%
\pgfsys@transformshift{1.256847in}{1.983122in}%
\pgfsys@useobject{currentmarker}{}%
\end{pgfscope}%
\begin{pgfscope}%
\pgfsys@transformshift{1.258195in}{1.995170in}%
\pgfsys@useobject{currentmarker}{}%
\end{pgfscope}%
\begin{pgfscope}%
\pgfsys@transformshift{1.257483in}{2.011395in}%
\pgfsys@useobject{currentmarker}{}%
\end{pgfscope}%
\begin{pgfscope}%
\pgfsys@transformshift{1.256142in}{2.020226in}%
\pgfsys@useobject{currentmarker}{}%
\end{pgfscope}%
\begin{pgfscope}%
\pgfsys@transformshift{1.259448in}{2.033461in}%
\pgfsys@useobject{currentmarker}{}%
\end{pgfscope}%
\begin{pgfscope}%
\pgfsys@transformshift{1.261784in}{2.048048in}%
\pgfsys@useobject{currentmarker}{}%
\end{pgfscope}%
\begin{pgfscope}%
\pgfsys@transformshift{1.266790in}{2.063787in}%
\pgfsys@useobject{currentmarker}{}%
\end{pgfscope}%
\begin{pgfscope}%
\pgfsys@transformshift{1.262446in}{2.081697in}%
\pgfsys@useobject{currentmarker}{}%
\end{pgfscope}%
\begin{pgfscope}%
\pgfsys@transformshift{1.266875in}{2.103359in}%
\pgfsys@useobject{currentmarker}{}%
\end{pgfscope}%
\begin{pgfscope}%
\pgfsys@transformshift{1.273233in}{2.125424in}%
\pgfsys@useobject{currentmarker}{}%
\end{pgfscope}%
\begin{pgfscope}%
\pgfsys@transformshift{1.276783in}{2.148650in}%
\pgfsys@useobject{currentmarker}{}%
\end{pgfscope}%
\begin{pgfscope}%
\pgfsys@transformshift{1.276752in}{2.173925in}%
\pgfsys@useobject{currentmarker}{}%
\end{pgfscope}%
\begin{pgfscope}%
\pgfsys@transformshift{1.273126in}{2.199452in}%
\pgfsys@useobject{currentmarker}{}%
\end{pgfscope}%
\begin{pgfscope}%
\pgfsys@transformshift{1.282180in}{2.229866in}%
\pgfsys@useobject{currentmarker}{}%
\end{pgfscope}%
\begin{pgfscope}%
\pgfsys@transformshift{1.290778in}{2.263069in}%
\pgfsys@useobject{currentmarker}{}%
\end{pgfscope}%
\begin{pgfscope}%
\pgfsys@transformshift{1.301421in}{2.297587in}%
\pgfsys@useobject{currentmarker}{}%
\end{pgfscope}%
\begin{pgfscope}%
\pgfsys@transformshift{1.294882in}{2.334610in}%
\pgfsys@useobject{currentmarker}{}%
\end{pgfscope}%
\begin{pgfscope}%
\pgfsys@transformshift{1.309632in}{2.373554in}%
\pgfsys@useobject{currentmarker}{}%
\end{pgfscope}%
\begin{pgfscope}%
\pgfsys@transformshift{1.326937in}{2.413815in}%
\pgfsys@useobject{currentmarker}{}%
\end{pgfscope}%
\begin{pgfscope}%
\pgfsys@transformshift{1.339302in}{2.456609in}%
\pgfsys@useobject{currentmarker}{}%
\end{pgfscope}%
\begin{pgfscope}%
\pgfsys@transformshift{1.334544in}{2.502648in}%
\pgfsys@useobject{currentmarker}{}%
\end{pgfscope}%
\begin{pgfscope}%
\pgfsys@transformshift{1.330003in}{2.551067in}%
\pgfsys@useobject{currentmarker}{}%
\end{pgfscope}%
\begin{pgfscope}%
\pgfsys@transformshift{1.329046in}{2.577798in}%
\pgfsys@useobject{currentmarker}{}%
\end{pgfscope}%
\begin{pgfscope}%
\pgfsys@transformshift{1.333278in}{2.591887in}%
\pgfsys@useobject{currentmarker}{}%
\end{pgfscope}%
\begin{pgfscope}%
\pgfsys@transformshift{1.334505in}{2.599885in}%
\pgfsys@useobject{currentmarker}{}%
\end{pgfscope}%
\begin{pgfscope}%
\pgfsys@transformshift{1.333832in}{2.611300in}%
\pgfsys@useobject{currentmarker}{}%
\end{pgfscope}%
\begin{pgfscope}%
\pgfsys@transformshift{1.333232in}{2.617560in}%
\pgfsys@useobject{currentmarker}{}%
\end{pgfscope}%
\begin{pgfscope}%
\pgfsys@transformshift{1.335710in}{2.627440in}%
\pgfsys@useobject{currentmarker}{}%
\end{pgfscope}%
\begin{pgfscope}%
\pgfsys@transformshift{1.336563in}{2.638231in}%
\pgfsys@useobject{currentmarker}{}%
\end{pgfscope}%
\begin{pgfscope}%
\pgfsys@transformshift{1.340070in}{2.650561in}%
\pgfsys@useobject{currentmarker}{}%
\end{pgfscope}%
\begin{pgfscope}%
\pgfsys@transformshift{1.336949in}{2.664967in}%
\pgfsys@useobject{currentmarker}{}%
\end{pgfscope}%
\begin{pgfscope}%
\pgfsys@transformshift{1.336699in}{2.673071in}%
\pgfsys@useobject{currentmarker}{}%
\end{pgfscope}%
\begin{pgfscope}%
\pgfsys@transformshift{1.337932in}{2.682734in}%
\pgfsys@useobject{currentmarker}{}%
\end{pgfscope}%
\begin{pgfscope}%
\pgfsys@transformshift{1.337884in}{2.688091in}%
\pgfsys@useobject{currentmarker}{}%
\end{pgfscope}%
\begin{pgfscope}%
\pgfsys@transformshift{1.338054in}{2.693995in}%
\pgfsys@useobject{currentmarker}{}%
\end{pgfscope}%
\begin{pgfscope}%
\pgfsys@transformshift{1.337724in}{2.700713in}%
\pgfsys@useobject{currentmarker}{}%
\end{pgfscope}%
\begin{pgfscope}%
\pgfsys@transformshift{1.333730in}{2.706789in}%
\pgfsys@useobject{currentmarker}{}%
\end{pgfscope}%
\begin{pgfscope}%
\pgfsys@transformshift{1.330174in}{2.708620in}%
\pgfsys@useobject{currentmarker}{}%
\end{pgfscope}%
\begin{pgfscope}%
\pgfsys@transformshift{1.322881in}{2.708712in}%
\pgfsys@useobject{currentmarker}{}%
\end{pgfscope}%
\begin{pgfscope}%
\pgfsys@transformshift{1.313114in}{2.708781in}%
\pgfsys@useobject{currentmarker}{}%
\end{pgfscope}%
\begin{pgfscope}%
\pgfsys@transformshift{1.307742in}{2.708766in}%
\pgfsys@useobject{currentmarker}{}%
\end{pgfscope}%
\begin{pgfscope}%
\pgfsys@transformshift{1.304800in}{2.708491in}%
\pgfsys@useobject{currentmarker}{}%
\end{pgfscope}%
\begin{pgfscope}%
\pgfsys@transformshift{1.303175in}{2.708446in}%
\pgfsys@useobject{currentmarker}{}%
\end{pgfscope}%
\begin{pgfscope}%
\pgfsys@transformshift{1.302283in}{2.708399in}%
\pgfsys@useobject{currentmarker}{}%
\end{pgfscope}%
\begin{pgfscope}%
\pgfsys@transformshift{1.300093in}{2.708501in}%
\pgfsys@useobject{currentmarker}{}%
\end{pgfscope}%
\begin{pgfscope}%
\pgfsys@transformshift{1.296588in}{2.708384in}%
\pgfsys@useobject{currentmarker}{}%
\end{pgfscope}%
\begin{pgfscope}%
\pgfsys@transformshift{1.291783in}{2.708150in}%
\pgfsys@useobject{currentmarker}{}%
\end{pgfscope}%
\begin{pgfscope}%
\pgfsys@transformshift{1.285399in}{2.708258in}%
\pgfsys@useobject{currentmarker}{}%
\end{pgfscope}%
\begin{pgfscope}%
\pgfsys@transformshift{1.281887in}{2.708256in}%
\pgfsys@useobject{currentmarker}{}%
\end{pgfscope}%
\begin{pgfscope}%
\pgfsys@transformshift{1.276857in}{2.707439in}%
\pgfsys@useobject{currentmarker}{}%
\end{pgfscope}%
\begin{pgfscope}%
\pgfsys@transformshift{1.270973in}{2.707124in}%
\pgfsys@useobject{currentmarker}{}%
\end{pgfscope}%
\begin{pgfscope}%
\pgfsys@transformshift{1.264554in}{2.708075in}%
\pgfsys@useobject{currentmarker}{}%
\end{pgfscope}%
\begin{pgfscope}%
\pgfsys@transformshift{1.257324in}{2.708279in}%
\pgfsys@useobject{currentmarker}{}%
\end{pgfscope}%
\begin{pgfscope}%
\pgfsys@transformshift{1.249159in}{2.707724in}%
\pgfsys@useobject{currentmarker}{}%
\end{pgfscope}%
\begin{pgfscope}%
\pgfsys@transformshift{1.248627in}{2.707245in}%
\pgfsys@useobject{currentmarker}{}%
\end{pgfscope}%
\begin{pgfscope}%
\pgfsys@transformshift{1.259977in}{2.706358in}%
\pgfsys@useobject{currentmarker}{}%
\end{pgfscope}%
\begin{pgfscope}%
\pgfsys@transformshift{1.273190in}{2.704350in}%
\pgfsys@useobject{currentmarker}{}%
\end{pgfscope}%
\begin{pgfscope}%
\pgfsys@transformshift{1.287184in}{2.703106in}%
\pgfsys@useobject{currentmarker}{}%
\end{pgfscope}%
\begin{pgfscope}%
\pgfsys@transformshift{1.294896in}{2.702627in}%
\pgfsys@useobject{currentmarker}{}%
\end{pgfscope}%
\begin{pgfscope}%
\pgfsys@transformshift{1.307090in}{2.702868in}%
\pgfsys@useobject{currentmarker}{}%
\end{pgfscope}%
\begin{pgfscope}%
\pgfsys@transformshift{1.320122in}{2.703698in}%
\pgfsys@useobject{currentmarker}{}%
\end{pgfscope}%
\begin{pgfscope}%
\pgfsys@transformshift{1.334853in}{2.702411in}%
\pgfsys@useobject{currentmarker}{}%
\end{pgfscope}%
\begin{pgfscope}%
\pgfsys@transformshift{1.350280in}{2.703244in}%
\pgfsys@useobject{currentmarker}{}%
\end{pgfscope}%
\begin{pgfscope}%
\pgfsys@transformshift{1.369990in}{2.701700in}%
\pgfsys@useobject{currentmarker}{}%
\end{pgfscope}%
\begin{pgfscope}%
\pgfsys@transformshift{1.390911in}{2.703714in}%
\pgfsys@useobject{currentmarker}{}%
\end{pgfscope}%
\begin{pgfscope}%
\pgfsys@transformshift{1.413935in}{2.701834in}%
\pgfsys@useobject{currentmarker}{}%
\end{pgfscope}%
\begin{pgfscope}%
\pgfsys@transformshift{1.437490in}{2.702684in}%
\pgfsys@useobject{currentmarker}{}%
\end{pgfscope}%
\begin{pgfscope}%
\pgfsys@transformshift{1.462358in}{2.699391in}%
\pgfsys@useobject{currentmarker}{}%
\end{pgfscope}%
\begin{pgfscope}%
\pgfsys@transformshift{1.488867in}{2.706382in}%
\pgfsys@useobject{currentmarker}{}%
\end{pgfscope}%
\begin{pgfscope}%
\pgfsys@transformshift{1.516680in}{2.708489in}%
\pgfsys@useobject{currentmarker}{}%
\end{pgfscope}%
\begin{pgfscope}%
\pgfsys@transformshift{1.532019in}{2.708791in}%
\pgfsys@useobject{currentmarker}{}%
\end{pgfscope}%
\begin{pgfscope}%
\pgfsys@transformshift{1.548433in}{2.708783in}%
\pgfsys@useobject{currentmarker}{}%
\end{pgfscope}%
\begin{pgfscope}%
\pgfsys@transformshift{1.566440in}{2.709017in}%
\pgfsys@useobject{currentmarker}{}%
\end{pgfscope}%
\begin{pgfscope}%
\pgfsys@transformshift{1.584735in}{2.712044in}%
\pgfsys@useobject{currentmarker}{}%
\end{pgfscope}%
\begin{pgfscope}%
\pgfsys@transformshift{1.606576in}{2.710559in}%
\pgfsys@useobject{currentmarker}{}%
\end{pgfscope}%
\begin{pgfscope}%
\pgfsys@transformshift{1.630842in}{2.709540in}%
\pgfsys@useobject{currentmarker}{}%
\end{pgfscope}%
\begin{pgfscope}%
\pgfsys@transformshift{1.655850in}{2.702765in}%
\pgfsys@useobject{currentmarker}{}%
\end{pgfscope}%
\begin{pgfscope}%
\pgfsys@transformshift{1.669946in}{2.704863in}%
\pgfsys@useobject{currentmarker}{}%
\end{pgfscope}%
\begin{pgfscope}%
\pgfsys@transformshift{1.684909in}{2.704812in}%
\pgfsys@useobject{currentmarker}{}%
\end{pgfscope}%
\begin{pgfscope}%
\pgfsys@transformshift{1.701827in}{2.704315in}%
\pgfsys@useobject{currentmarker}{}%
\end{pgfscope}%
\begin{pgfscope}%
\pgfsys@transformshift{1.719961in}{2.702798in}%
\pgfsys@useobject{currentmarker}{}%
\end{pgfscope}%
\begin{pgfscope}%
\pgfsys@transformshift{1.741828in}{2.700832in}%
\pgfsys@useobject{currentmarker}{}%
\end{pgfscope}%
\begin{pgfscope}%
\pgfsys@transformshift{1.766788in}{2.700336in}%
\pgfsys@useobject{currentmarker}{}%
\end{pgfscope}%
\begin{pgfscope}%
\pgfsys@transformshift{1.792278in}{2.700238in}%
\pgfsys@useobject{currentmarker}{}%
\end{pgfscope}%
\begin{pgfscope}%
\pgfsys@transformshift{1.806292in}{2.699847in}%
\pgfsys@useobject{currentmarker}{}%
\end{pgfscope}%
\begin{pgfscope}%
\pgfsys@transformshift{1.813976in}{2.700491in}%
\pgfsys@useobject{currentmarker}{}%
\end{pgfscope}%
\begin{pgfscope}%
\pgfsys@transformshift{1.823320in}{2.700305in}%
\pgfsys@useobject{currentmarker}{}%
\end{pgfscope}%
\begin{pgfscope}%
\pgfsys@transformshift{1.834477in}{2.702632in}%
\pgfsys@useobject{currentmarker}{}%
\end{pgfscope}%
\begin{pgfscope}%
\pgfsys@transformshift{1.848695in}{2.702714in}%
\pgfsys@useobject{currentmarker}{}%
\end{pgfscope}%
\begin{pgfscope}%
\pgfsys@transformshift{1.865062in}{2.703030in}%
\pgfsys@useobject{currentmarker}{}%
\end{pgfscope}%
\begin{pgfscope}%
\pgfsys@transformshift{1.874052in}{2.702553in}%
\pgfsys@useobject{currentmarker}{}%
\end{pgfscope}%
\begin{pgfscope}%
\pgfsys@transformshift{1.884793in}{2.703610in}%
\pgfsys@useobject{currentmarker}{}%
\end{pgfscope}%
\begin{pgfscope}%
\pgfsys@transformshift{1.896673in}{2.705335in}%
\pgfsys@useobject{currentmarker}{}%
\end{pgfscope}%
\begin{pgfscope}%
\pgfsys@transformshift{1.912164in}{2.705633in}%
\pgfsys@useobject{currentmarker}{}%
\end{pgfscope}%
\begin{pgfscope}%
\pgfsys@transformshift{1.929652in}{2.705350in}%
\pgfsys@useobject{currentmarker}{}%
\end{pgfscope}%
\begin{pgfscope}%
\pgfsys@transformshift{1.948025in}{2.706474in}%
\pgfsys@useobject{currentmarker}{}%
\end{pgfscope}%
\begin{pgfscope}%
\pgfsys@transformshift{1.967040in}{2.709820in}%
\pgfsys@useobject{currentmarker}{}%
\end{pgfscope}%
\begin{pgfscope}%
\pgfsys@transformshift{1.987718in}{2.713051in}%
\pgfsys@useobject{currentmarker}{}%
\end{pgfscope}%
\begin{pgfscope}%
\pgfsys@transformshift{1.999226in}{2.713315in}%
\pgfsys@useobject{currentmarker}{}%
\end{pgfscope}%
\begin{pgfscope}%
\pgfsys@transformshift{2.005504in}{2.712493in}%
\pgfsys@useobject{currentmarker}{}%
\end{pgfscope}%
\begin{pgfscope}%
\pgfsys@transformshift{2.013970in}{2.711529in}%
\pgfsys@useobject{currentmarker}{}%
\end{pgfscope}%
\begin{pgfscope}%
\pgfsys@transformshift{2.025162in}{2.712764in}%
\pgfsys@useobject{currentmarker}{}%
\end{pgfscope}%
\begin{pgfscope}%
\pgfsys@transformshift{2.038133in}{2.713743in}%
\pgfsys@useobject{currentmarker}{}%
\end{pgfscope}%
\begin{pgfscope}%
\pgfsys@transformshift{2.053863in}{2.713991in}%
\pgfsys@useobject{currentmarker}{}%
\end{pgfscope}%
\begin{pgfscope}%
\pgfsys@transformshift{2.072482in}{2.713224in}%
\pgfsys@useobject{currentmarker}{}%
\end{pgfscope}%
\begin{pgfscope}%
\pgfsys@transformshift{2.092037in}{2.712898in}%
\pgfsys@useobject{currentmarker}{}%
\end{pgfscope}%
\begin{pgfscope}%
\pgfsys@transformshift{2.102790in}{2.713186in}%
\pgfsys@useobject{currentmarker}{}%
\end{pgfscope}%
\begin{pgfscope}%
\pgfsys@transformshift{2.115144in}{2.714420in}%
\pgfsys@useobject{currentmarker}{}%
\end{pgfscope}%
\begin{pgfscope}%
\pgfsys@transformshift{2.121967in}{2.714696in}%
\pgfsys@useobject{currentmarker}{}%
\end{pgfscope}%
\begin{pgfscope}%
\pgfsys@transformshift{2.129781in}{2.714207in}%
\pgfsys@useobject{currentmarker}{}%
\end{pgfscope}%
\begin{pgfscope}%
\pgfsys@transformshift{2.134083in}{2.714003in}%
\pgfsys@useobject{currentmarker}{}%
\end{pgfscope}%
\begin{pgfscope}%
\pgfsys@transformshift{2.140168in}{2.714086in}%
\pgfsys@useobject{currentmarker}{}%
\end{pgfscope}%
\begin{pgfscope}%
\pgfsys@transformshift{2.147082in}{2.715292in}%
\pgfsys@useobject{currentmarker}{}%
\end{pgfscope}%
\begin{pgfscope}%
\pgfsys@transformshift{2.156068in}{2.715602in}%
\pgfsys@useobject{currentmarker}{}%
\end{pgfscope}%
\begin{pgfscope}%
\pgfsys@transformshift{2.166556in}{2.715922in}%
\pgfsys@useobject{currentmarker}{}%
\end{pgfscope}%
\begin{pgfscope}%
\pgfsys@transformshift{2.178524in}{2.715067in}%
\pgfsys@useobject{currentmarker}{}%
\end{pgfscope}%
\begin{pgfscope}%
\pgfsys@transformshift{2.185090in}{2.715734in}%
\pgfsys@useobject{currentmarker}{}%
\end{pgfscope}%
\begin{pgfscope}%
\pgfsys@transformshift{2.192144in}{2.716221in}%
\pgfsys@useobject{currentmarker}{}%
\end{pgfscope}%
\begin{pgfscope}%
\pgfsys@transformshift{2.200945in}{2.716542in}%
\pgfsys@useobject{currentmarker}{}%
\end{pgfscope}%
\begin{pgfscope}%
\pgfsys@transformshift{2.210721in}{2.715705in}%
\pgfsys@useobject{currentmarker}{}%
\end{pgfscope}%
\begin{pgfscope}%
\pgfsys@transformshift{2.221138in}{2.715123in}%
\pgfsys@useobject{currentmarker}{}%
\end{pgfscope}%
\begin{pgfscope}%
\pgfsys@transformshift{2.226851in}{2.714594in}%
\pgfsys@useobject{currentmarker}{}%
\end{pgfscope}%
\begin{pgfscope}%
\pgfsys@transformshift{2.234458in}{2.714990in}%
\pgfsys@useobject{currentmarker}{}%
\end{pgfscope}%
\begin{pgfscope}%
\pgfsys@transformshift{2.238628in}{2.714587in}%
\pgfsys@useobject{currentmarker}{}%
\end{pgfscope}%
\begin{pgfscope}%
\pgfsys@transformshift{2.243770in}{2.714392in}%
\pgfsys@useobject{currentmarker}{}%
\end{pgfscope}%
\begin{pgfscope}%
\pgfsys@transformshift{2.250283in}{2.713818in}%
\pgfsys@useobject{currentmarker}{}%
\end{pgfscope}%
\begin{pgfscope}%
\pgfsys@transformshift{2.258997in}{2.712554in}%
\pgfsys@useobject{currentmarker}{}%
\end{pgfscope}%
\begin{pgfscope}%
\pgfsys@transformshift{2.263825in}{2.712937in}%
\pgfsys@useobject{currentmarker}{}%
\end{pgfscope}%
\begin{pgfscope}%
\pgfsys@transformshift{2.271843in}{2.712437in}%
\pgfsys@useobject{currentmarker}{}%
\end{pgfscope}%
\begin{pgfscope}%
\pgfsys@transformshift{2.282620in}{2.712441in}%
\pgfsys@useobject{currentmarker}{}%
\end{pgfscope}%
\begin{pgfscope}%
\pgfsys@transformshift{2.294381in}{2.710885in}%
\pgfsys@useobject{currentmarker}{}%
\end{pgfscope}%
\begin{pgfscope}%
\pgfsys@transformshift{2.306848in}{2.711448in}%
\pgfsys@useobject{currentmarker}{}%
\end{pgfscope}%
\begin{pgfscope}%
\pgfsys@transformshift{2.313711in}{2.711473in}%
\pgfsys@useobject{currentmarker}{}%
\end{pgfscope}%
\begin{pgfscope}%
\pgfsys@transformshift{2.322525in}{2.711849in}%
\pgfsys@useobject{currentmarker}{}%
\end{pgfscope}%
\begin{pgfscope}%
\pgfsys@transformshift{2.331962in}{2.709986in}%
\pgfsys@useobject{currentmarker}{}%
\end{pgfscope}%
\begin{pgfscope}%
\pgfsys@transformshift{2.342793in}{2.709083in}%
\pgfsys@useobject{currentmarker}{}%
\end{pgfscope}%
\begin{pgfscope}%
\pgfsys@transformshift{2.354650in}{2.709493in}%
\pgfsys@useobject{currentmarker}{}%
\end{pgfscope}%
\begin{pgfscope}%
\pgfsys@transformshift{2.361175in}{2.709538in}%
\pgfsys@useobject{currentmarker}{}%
\end{pgfscope}%
\begin{pgfscope}%
\pgfsys@transformshift{2.369026in}{2.709347in}%
\pgfsys@useobject{currentmarker}{}%
\end{pgfscope}%
\begin{pgfscope}%
\pgfsys@transformshift{2.378079in}{2.709324in}%
\pgfsys@useobject{currentmarker}{}%
\end{pgfscope}%
\begin{pgfscope}%
\pgfsys@transformshift{2.383034in}{2.708839in}%
\pgfsys@useobject{currentmarker}{}%
\end{pgfscope}%
\begin{pgfscope}%
\pgfsys@transformshift{2.388874in}{2.709632in}%
\pgfsys@useobject{currentmarker}{}%
\end{pgfscope}%
\begin{pgfscope}%
\pgfsys@transformshift{2.396947in}{2.709128in}%
\pgfsys@useobject{currentmarker}{}%
\end{pgfscope}%
\begin{pgfscope}%
\pgfsys@transformshift{2.406822in}{2.709553in}%
\pgfsys@useobject{currentmarker}{}%
\end{pgfscope}%
\begin{pgfscope}%
\pgfsys@transformshift{2.418245in}{2.710775in}%
\pgfsys@useobject{currentmarker}{}%
\end{pgfscope}%
\begin{pgfscope}%
\pgfsys@transformshift{2.424503in}{2.711649in}%
\pgfsys@useobject{currentmarker}{}%
\end{pgfscope}%
\begin{pgfscope}%
\pgfsys@transformshift{2.432005in}{2.713469in}%
\pgfsys@useobject{currentmarker}{}%
\end{pgfscope}%
\begin{pgfscope}%
\pgfsys@transformshift{2.440316in}{2.713587in}%
\pgfsys@useobject{currentmarker}{}%
\end{pgfscope}%
\begin{pgfscope}%
\pgfsys@transformshift{2.449268in}{2.713297in}%
\pgfsys@useobject{currentmarker}{}%
\end{pgfscope}%
\begin{pgfscope}%
\pgfsys@transformshift{2.458818in}{2.713793in}%
\pgfsys@useobject{currentmarker}{}%
\end{pgfscope}%
\begin{pgfscope}%
\pgfsys@transformshift{2.469314in}{2.713012in}%
\pgfsys@useobject{currentmarker}{}%
\end{pgfscope}%
\begin{pgfscope}%
\pgfsys@transformshift{2.475016in}{2.714015in}%
\pgfsys@useobject{currentmarker}{}%
\end{pgfscope}%
\begin{pgfscope}%
\pgfsys@transformshift{2.478200in}{2.714027in}%
\pgfsys@useobject{currentmarker}{}%
\end{pgfscope}%
\begin{pgfscope}%
\pgfsys@transformshift{2.482963in}{2.714103in}%
\pgfsys@useobject{currentmarker}{}%
\end{pgfscope}%
\begin{pgfscope}%
\pgfsys@transformshift{2.489358in}{2.713482in}%
\pgfsys@useobject{currentmarker}{}%
\end{pgfscope}%
\begin{pgfscope}%
\pgfsys@transformshift{2.499161in}{2.713678in}%
\pgfsys@useobject{currentmarker}{}%
\end{pgfscope}%
\begin{pgfscope}%
\pgfsys@transformshift{2.511701in}{2.713396in}%
\pgfsys@useobject{currentmarker}{}%
\end{pgfscope}%
\begin{pgfscope}%
\pgfsys@transformshift{2.525714in}{2.715395in}%
\pgfsys@useobject{currentmarker}{}%
\end{pgfscope}%
\begin{pgfscope}%
\pgfsys@transformshift{2.541499in}{2.716012in}%
\pgfsys@useobject{currentmarker}{}%
\end{pgfscope}%
\begin{pgfscope}%
\pgfsys@transformshift{2.559208in}{2.716665in}%
\pgfsys@useobject{currentmarker}{}%
\end{pgfscope}%
\begin{pgfscope}%
\pgfsys@transformshift{2.578171in}{2.715882in}%
\pgfsys@useobject{currentmarker}{}%
\end{pgfscope}%
\begin{pgfscope}%
\pgfsys@transformshift{2.600873in}{2.720264in}%
\pgfsys@useobject{currentmarker}{}%
\end{pgfscope}%
\begin{pgfscope}%
\pgfsys@transformshift{2.624888in}{2.724378in}%
\pgfsys@useobject{currentmarker}{}%
\end{pgfscope}%
\begin{pgfscope}%
\pgfsys@transformshift{2.650576in}{2.727692in}%
\pgfsys@useobject{currentmarker}{}%
\end{pgfscope}%
\begin{pgfscope}%
\pgfsys@transformshift{2.677415in}{2.728038in}%
\pgfsys@useobject{currentmarker}{}%
\end{pgfscope}%
\begin{pgfscope}%
\pgfsys@transformshift{2.704773in}{2.729913in}%
\pgfsys@useobject{currentmarker}{}%
\end{pgfscope}%
\begin{pgfscope}%
\pgfsys@transformshift{2.735398in}{2.731471in}%
\pgfsys@useobject{currentmarker}{}%
\end{pgfscope}%
\begin{pgfscope}%
\pgfsys@transformshift{2.767578in}{2.735848in}%
\pgfsys@useobject{currentmarker}{}%
\end{pgfscope}%
\begin{pgfscope}%
\pgfsys@transformshift{2.801351in}{2.736370in}%
\pgfsys@useobject{currentmarker}{}%
\end{pgfscope}%
\begin{pgfscope}%
\pgfsys@transformshift{2.819909in}{2.737210in}%
\pgfsys@useobject{currentmarker}{}%
\end{pgfscope}%
\begin{pgfscope}%
\pgfsys@transformshift{2.839874in}{2.736134in}%
\pgfsys@useobject{currentmarker}{}%
\end{pgfscope}%
\begin{pgfscope}%
\pgfsys@transformshift{2.860853in}{2.738835in}%
\pgfsys@useobject{currentmarker}{}%
\end{pgfscope}%
\begin{pgfscope}%
\pgfsys@transformshift{2.872310in}{2.736810in}%
\pgfsys@useobject{currentmarker}{}%
\end{pgfscope}%
\begin{pgfscope}%
\pgfsys@transformshift{2.878400in}{2.738774in}%
\pgfsys@useobject{currentmarker}{}%
\end{pgfscope}%
\begin{pgfscope}%
\pgfsys@transformshift{2.881918in}{2.738692in}%
\pgfsys@useobject{currentmarker}{}%
\end{pgfscope}%
\begin{pgfscope}%
\pgfsys@transformshift{2.883806in}{2.739120in}%
\pgfsys@useobject{currentmarker}{}%
\end{pgfscope}%
\begin{pgfscope}%
\pgfsys@transformshift{2.886603in}{2.739033in}%
\pgfsys@useobject{currentmarker}{}%
\end{pgfscope}%
\begin{pgfscope}%
\pgfsys@transformshift{2.888138in}{2.739131in}%
\pgfsys@useobject{currentmarker}{}%
\end{pgfscope}%
\begin{pgfscope}%
\pgfsys@transformshift{2.890363in}{2.739122in}%
\pgfsys@useobject{currentmarker}{}%
\end{pgfscope}%
\begin{pgfscope}%
\pgfsys@transformshift{2.893623in}{2.739618in}%
\pgfsys@useobject{currentmarker}{}%
\end{pgfscope}%
\begin{pgfscope}%
\pgfsys@transformshift{2.897551in}{2.740064in}%
\pgfsys@useobject{currentmarker}{}%
\end{pgfscope}%
\begin{pgfscope}%
\pgfsys@transformshift{2.899702in}{2.740382in}%
\pgfsys@useobject{currentmarker}{}%
\end{pgfscope}%
\begin{pgfscope}%
\pgfsys@transformshift{2.900890in}{2.740252in}%
\pgfsys@useobject{currentmarker}{}%
\end{pgfscope}%
\begin{pgfscope}%
\pgfsys@transformshift{2.903870in}{2.740244in}%
\pgfsys@useobject{currentmarker}{}%
\end{pgfscope}%
\begin{pgfscope}%
\pgfsys@transformshift{2.908327in}{2.738852in}%
\pgfsys@useobject{currentmarker}{}%
\end{pgfscope}%
\begin{pgfscope}%
\pgfsys@transformshift{2.913491in}{2.737178in}%
\pgfsys@useobject{currentmarker}{}%
\end{pgfscope}%
\begin{pgfscope}%
\pgfsys@transformshift{2.917792in}{2.732799in}%
\pgfsys@useobject{currentmarker}{}%
\end{pgfscope}%
\begin{pgfscope}%
\pgfsys@transformshift{2.924079in}{2.728645in}%
\pgfsys@useobject{currentmarker}{}%
\end{pgfscope}%
\begin{pgfscope}%
\pgfsys@transformshift{2.927464in}{2.721315in}%
\pgfsys@useobject{currentmarker}{}%
\end{pgfscope}%
\begin{pgfscope}%
\pgfsys@transformshift{2.930122in}{2.717758in}%
\pgfsys@useobject{currentmarker}{}%
\end{pgfscope}%
\begin{pgfscope}%
\pgfsys@transformshift{2.930782in}{2.715406in}%
\pgfsys@useobject{currentmarker}{}%
\end{pgfscope}%
\begin{pgfscope}%
\pgfsys@transformshift{2.931139in}{2.712364in}%
\pgfsys@useobject{currentmarker}{}%
\end{pgfscope}%
\begin{pgfscope}%
\pgfsys@transformshift{2.930490in}{2.708391in}%
\pgfsys@useobject{currentmarker}{}%
\end{pgfscope}%
\begin{pgfscope}%
\pgfsys@transformshift{2.929717in}{2.703805in}%
\pgfsys@useobject{currentmarker}{}%
\end{pgfscope}%
\begin{pgfscope}%
\pgfsys@transformshift{2.928922in}{2.698747in}%
\pgfsys@useobject{currentmarker}{}%
\end{pgfscope}%
\begin{pgfscope}%
\pgfsys@transformshift{2.928574in}{2.695953in}%
\pgfsys@useobject{currentmarker}{}%
\end{pgfscope}%
\begin{pgfscope}%
\pgfsys@transformshift{2.928454in}{2.694408in}%
\pgfsys@useobject{currentmarker}{}%
\end{pgfscope}%
\begin{pgfscope}%
\pgfsys@transformshift{2.928295in}{2.693572in}%
\pgfsys@useobject{currentmarker}{}%
\end{pgfscope}%
\begin{pgfscope}%
\pgfsys@transformshift{2.927911in}{2.691986in}%
\pgfsys@useobject{currentmarker}{}%
\end{pgfscope}%
\begin{pgfscope}%
\pgfsys@transformshift{2.928344in}{2.688554in}%
\pgfsys@useobject{currentmarker}{}%
\end{pgfscope}%
\begin{pgfscope}%
\pgfsys@transformshift{2.928309in}{2.686652in}%
\pgfsys@useobject{currentmarker}{}%
\end{pgfscope}%
\begin{pgfscope}%
\pgfsys@transformshift{2.927639in}{2.681662in}%
\pgfsys@useobject{currentmarker}{}%
\end{pgfscope}%
\begin{pgfscope}%
\pgfsys@transformshift{2.926665in}{2.675936in}%
\pgfsys@useobject{currentmarker}{}%
\end{pgfscope}%
\begin{pgfscope}%
\pgfsys@transformshift{2.926754in}{2.668473in}%
\pgfsys@useobject{currentmarker}{}%
\end{pgfscope}%
\begin{pgfscope}%
\pgfsys@transformshift{2.927418in}{2.660055in}%
\pgfsys@useobject{currentmarker}{}%
\end{pgfscope}%
\begin{pgfscope}%
\pgfsys@transformshift{2.926025in}{2.650986in}%
\pgfsys@useobject{currentmarker}{}%
\end{pgfscope}%
\begin{pgfscope}%
\pgfsys@transformshift{2.924137in}{2.640511in}%
\pgfsys@useobject{currentmarker}{}%
\end{pgfscope}%
\begin{pgfscope}%
\pgfsys@transformshift{2.924314in}{2.634660in}%
\pgfsys@useobject{currentmarker}{}%
\end{pgfscope}%
\begin{pgfscope}%
\pgfsys@transformshift{2.924805in}{2.627579in}%
\pgfsys@useobject{currentmarker}{}%
\end{pgfscope}%
\begin{pgfscope}%
\pgfsys@transformshift{2.924324in}{2.623705in}%
\pgfsys@useobject{currentmarker}{}%
\end{pgfscope}%
\begin{pgfscope}%
\pgfsys@transformshift{2.923829in}{2.618863in}%
\pgfsys@useobject{currentmarker}{}%
\end{pgfscope}%
\begin{pgfscope}%
\pgfsys@transformshift{2.923872in}{2.613489in}%
\pgfsys@useobject{currentmarker}{}%
\end{pgfscope}%
\begin{pgfscope}%
\pgfsys@transformshift{2.924357in}{2.607234in}%
\pgfsys@useobject{currentmarker}{}%
\end{pgfscope}%
\begin{pgfscope}%
\pgfsys@transformshift{2.924726in}{2.600511in}%
\pgfsys@useobject{currentmarker}{}%
\end{pgfscope}%
\begin{pgfscope}%
\pgfsys@transformshift{2.923355in}{2.591747in}%
\pgfsys@useobject{currentmarker}{}%
\end{pgfscope}%
\begin{pgfscope}%
\pgfsys@transformshift{2.923034in}{2.586878in}%
\pgfsys@useobject{currentmarker}{}%
\end{pgfscope}%
\begin{pgfscope}%
\pgfsys@transformshift{2.922633in}{2.580944in}%
\pgfsys@useobject{currentmarker}{}%
\end{pgfscope}%
\begin{pgfscope}%
\pgfsys@transformshift{2.923031in}{2.577697in}%
\pgfsys@useobject{currentmarker}{}%
\end{pgfscope}%
\begin{pgfscope}%
\pgfsys@transformshift{2.922570in}{2.573116in}%
\pgfsys@useobject{currentmarker}{}%
\end{pgfscope}%
\begin{pgfscope}%
\pgfsys@transformshift{2.922336in}{2.570594in}%
\pgfsys@useobject{currentmarker}{}%
\end{pgfscope}%
\begin{pgfscope}%
\pgfsys@transformshift{2.922411in}{2.569204in}%
\pgfsys@useobject{currentmarker}{}%
\end{pgfscope}%
\begin{pgfscope}%
\pgfsys@transformshift{2.922604in}{2.566795in}%
\pgfsys@useobject{currentmarker}{}%
\end{pgfscope}%
\begin{pgfscope}%
\pgfsys@transformshift{2.922140in}{2.563369in}%
\pgfsys@useobject{currentmarker}{}%
\end{pgfscope}%
\begin{pgfscope}%
\pgfsys@transformshift{2.921539in}{2.559249in}%
\pgfsys@useobject{currentmarker}{}%
\end{pgfscope}%
\begin{pgfscope}%
\pgfsys@transformshift{2.921439in}{2.556962in}%
\pgfsys@useobject{currentmarker}{}%
\end{pgfscope}%
\begin{pgfscope}%
\pgfsys@transformshift{2.921107in}{2.552915in}%
\pgfsys@useobject{currentmarker}{}%
\end{pgfscope}%
\begin{pgfscope}%
\pgfsys@transformshift{2.921378in}{2.548272in}%
\pgfsys@useobject{currentmarker}{}%
\end{pgfscope}%
\begin{pgfscope}%
\pgfsys@transformshift{2.920360in}{2.542177in}%
\pgfsys@useobject{currentmarker}{}%
\end{pgfscope}%
\begin{pgfscope}%
\pgfsys@transformshift{2.920185in}{2.538783in}%
\pgfsys@useobject{currentmarker}{}%
\end{pgfscope}%
\begin{pgfscope}%
\pgfsys@transformshift{2.919871in}{2.534285in}%
\pgfsys@useobject{currentmarker}{}%
\end{pgfscope}%
\begin{pgfscope}%
\pgfsys@transformshift{2.921017in}{2.528794in}%
\pgfsys@useobject{currentmarker}{}%
\end{pgfscope}%
\begin{pgfscope}%
\pgfsys@transformshift{2.920617in}{2.525735in}%
\pgfsys@useobject{currentmarker}{}%
\end{pgfscope}%
\begin{pgfscope}%
\pgfsys@transformshift{2.920554in}{2.524039in}%
\pgfsys@useobject{currentmarker}{}%
\end{pgfscope}%
\begin{pgfscope}%
\pgfsys@transformshift{2.920449in}{2.523112in}%
\pgfsys@useobject{currentmarker}{}%
\end{pgfscope}%
\begin{pgfscope}%
\pgfsys@transformshift{2.921072in}{2.519635in}%
\pgfsys@useobject{currentmarker}{}%
\end{pgfscope}%
\begin{pgfscope}%
\pgfsys@transformshift{2.921327in}{2.515632in}%
\pgfsys@useobject{currentmarker}{}%
\end{pgfscope}%
\begin{pgfscope}%
\pgfsys@transformshift{2.921179in}{2.513430in}%
\pgfsys@useobject{currentmarker}{}%
\end{pgfscope}%
\begin{pgfscope}%
\pgfsys@transformshift{2.921326in}{2.512226in}%
\pgfsys@useobject{currentmarker}{}%
\end{pgfscope}%
\begin{pgfscope}%
\pgfsys@transformshift{2.921384in}{2.511561in}%
\pgfsys@useobject{currentmarker}{}%
\end{pgfscope}%
\begin{pgfscope}%
\pgfsys@transformshift{2.921352in}{2.511195in}%
\pgfsys@useobject{currentmarker}{}%
\end{pgfscope}%
\begin{pgfscope}%
\pgfsys@transformshift{2.921396in}{2.510998in}%
\pgfsys@useobject{currentmarker}{}%
\end{pgfscope}%
\begin{pgfscope}%
\pgfsys@transformshift{2.921437in}{2.510895in}%
\pgfsys@useobject{currentmarker}{}%
\end{pgfscope}%
\begin{pgfscope}%
\pgfsys@transformshift{2.921755in}{2.510015in}%
\pgfsys@useobject{currentmarker}{}%
\end{pgfscope}%
\begin{pgfscope}%
\pgfsys@transformshift{2.921350in}{2.507373in}%
\pgfsys@useobject{currentmarker}{}%
\end{pgfscope}%
\begin{pgfscope}%
\pgfsys@transformshift{2.921568in}{2.505918in}%
\pgfsys@useobject{currentmarker}{}%
\end{pgfscope}%
\begin{pgfscope}%
\pgfsys@transformshift{2.922136in}{2.501524in}%
\pgfsys@useobject{currentmarker}{}%
\end{pgfscope}%
\begin{pgfscope}%
\pgfsys@transformshift{2.922153in}{2.496193in}%
\pgfsys@useobject{currentmarker}{}%
\end{pgfscope}%
\begin{pgfscope}%
\pgfsys@transformshift{2.920299in}{2.488616in}%
\pgfsys@useobject{currentmarker}{}%
\end{pgfscope}%
\begin{pgfscope}%
\pgfsys@transformshift{2.920453in}{2.484329in}%
\pgfsys@useobject{currentmarker}{}%
\end{pgfscope}%
\begin{pgfscope}%
\pgfsys@transformshift{2.920609in}{2.479465in}%
\pgfsys@useobject{currentmarker}{}%
\end{pgfscope}%
\begin{pgfscope}%
\pgfsys@transformshift{2.920890in}{2.476803in}%
\pgfsys@useobject{currentmarker}{}%
\end{pgfscope}%
\begin{pgfscope}%
\pgfsys@transformshift{2.920076in}{2.470711in}%
\pgfsys@useobject{currentmarker}{}%
\end{pgfscope}%
\begin{pgfscope}%
\pgfsys@transformshift{2.918805in}{2.464160in}%
\pgfsys@useobject{currentmarker}{}%
\end{pgfscope}%
\begin{pgfscope}%
\pgfsys@transformshift{2.919511in}{2.455850in}%
\pgfsys@useobject{currentmarker}{}%
\end{pgfscope}%
\begin{pgfscope}%
\pgfsys@transformshift{2.920056in}{2.451296in}%
\pgfsys@useobject{currentmarker}{}%
\end{pgfscope}%
\begin{pgfscope}%
\pgfsys@transformshift{2.919111in}{2.444589in}%
\pgfsys@useobject{currentmarker}{}%
\end{pgfscope}%
\begin{pgfscope}%
\pgfsys@transformshift{2.918504in}{2.440914in}%
\pgfsys@useobject{currentmarker}{}%
\end{pgfscope}%
\begin{pgfscope}%
\pgfsys@transformshift{2.918675in}{2.438872in}%
\pgfsys@useobject{currentmarker}{}%
\end{pgfscope}%
\begin{pgfscope}%
\pgfsys@transformshift{2.918936in}{2.436123in}%
\pgfsys@useobject{currentmarker}{}%
\end{pgfscope}%
\begin{pgfscope}%
\pgfsys@transformshift{2.918944in}{2.434605in}%
\pgfsys@useobject{currentmarker}{}%
\end{pgfscope}%
\begin{pgfscope}%
\pgfsys@transformshift{2.918428in}{2.432223in}%
\pgfsys@useobject{currentmarker}{}%
\end{pgfscope}%
\begin{pgfscope}%
\pgfsys@transformshift{2.918567in}{2.430891in}%
\pgfsys@useobject{currentmarker}{}%
\end{pgfscope}%
\begin{pgfscope}%
\pgfsys@transformshift{2.918747in}{2.428329in}%
\pgfsys@useobject{currentmarker}{}%
\end{pgfscope}%
\begin{pgfscope}%
\pgfsys@transformshift{2.919201in}{2.424971in}%
\pgfsys@useobject{currentmarker}{}%
\end{pgfscope}%
\begin{pgfscope}%
\pgfsys@transformshift{2.918443in}{2.419152in}%
\pgfsys@useobject{currentmarker}{}%
\end{pgfscope}%
\begin{pgfscope}%
\pgfsys@transformshift{2.918226in}{2.415933in}%
\pgfsys@useobject{currentmarker}{}%
\end{pgfscope}%
\begin{pgfscope}%
\pgfsys@transformshift{2.918005in}{2.414171in}%
\pgfsys@useobject{currentmarker}{}%
\end{pgfscope}%
\begin{pgfscope}%
\pgfsys@transformshift{2.918231in}{2.411144in}%
\pgfsys@useobject{currentmarker}{}%
\end{pgfscope}%
\begin{pgfscope}%
\pgfsys@transformshift{2.918075in}{2.409481in}%
\pgfsys@useobject{currentmarker}{}%
\end{pgfscope}%
\begin{pgfscope}%
\pgfsys@transformshift{2.917900in}{2.405958in}%
\pgfsys@useobject{currentmarker}{}%
\end{pgfscope}%
\begin{pgfscope}%
\pgfsys@transformshift{2.917269in}{2.401952in}%
\pgfsys@useobject{currentmarker}{}%
\end{pgfscope}%
\begin{pgfscope}%
\pgfsys@transformshift{2.917224in}{2.399722in}%
\pgfsys@useobject{currentmarker}{}%
\end{pgfscope}%
\begin{pgfscope}%
\pgfsys@transformshift{2.917294in}{2.398498in}%
\pgfsys@useobject{currentmarker}{}%
\end{pgfscope}%
\begin{pgfscope}%
\pgfsys@transformshift{2.917229in}{2.397826in}%
\pgfsys@useobject{currentmarker}{}%
\end{pgfscope}%
\begin{pgfscope}%
\pgfsys@transformshift{2.917176in}{2.397459in}%
\pgfsys@useobject{currentmarker}{}%
\end{pgfscope}%
\begin{pgfscope}%
\pgfsys@transformshift{2.917105in}{2.396412in}%
\pgfsys@useobject{currentmarker}{}%
\end{pgfscope}%
\begin{pgfscope}%
\pgfsys@transformshift{2.917121in}{2.391965in}%
\pgfsys@useobject{currentmarker}{}%
\end{pgfscope}%
\begin{pgfscope}%
\pgfsys@transformshift{2.917264in}{2.389523in}%
\pgfsys@useobject{currentmarker}{}%
\end{pgfscope}%
\begin{pgfscope}%
\pgfsys@transformshift{2.916376in}{2.385130in}%
\pgfsys@useobject{currentmarker}{}%
\end{pgfscope}%
\begin{pgfscope}%
\pgfsys@transformshift{2.916235in}{2.382669in}%
\pgfsys@useobject{currentmarker}{}%
\end{pgfscope}%
\begin{pgfscope}%
\pgfsys@transformshift{2.915547in}{2.378957in}%
\pgfsys@useobject{currentmarker}{}%
\end{pgfscope}%
\begin{pgfscope}%
\pgfsys@transformshift{2.916318in}{2.373751in}%
\pgfsys@useobject{currentmarker}{}%
\end{pgfscope}%
\begin{pgfscope}%
\pgfsys@transformshift{2.915010in}{2.367485in}%
\pgfsys@useobject{currentmarker}{}%
\end{pgfscope}%
\begin{pgfscope}%
\pgfsys@transformshift{2.913825in}{2.360674in}%
\pgfsys@useobject{currentmarker}{}%
\end{pgfscope}%
\begin{pgfscope}%
\pgfsys@transformshift{2.913308in}{2.353317in}%
\pgfsys@useobject{currentmarker}{}%
\end{pgfscope}%
\begin{pgfscope}%
\pgfsys@transformshift{2.913480in}{2.343709in}%
\pgfsys@useobject{currentmarker}{}%
\end{pgfscope}%
\begin{pgfscope}%
\pgfsys@transformshift{2.913471in}{2.338424in}%
\pgfsys@useobject{currentmarker}{}%
\end{pgfscope}%
\begin{pgfscope}%
\pgfsys@transformshift{2.912304in}{2.332008in}%
\pgfsys@useobject{currentmarker}{}%
\end{pgfscope}%
\begin{pgfscope}%
\pgfsys@transformshift{2.911687in}{2.324776in}%
\pgfsys@useobject{currentmarker}{}%
\end{pgfscope}%
\begin{pgfscope}%
\pgfsys@transformshift{2.910668in}{2.315879in}%
\pgfsys@useobject{currentmarker}{}%
\end{pgfscope}%
\begin{pgfscope}%
\pgfsys@transformshift{2.911582in}{2.306142in}%
\pgfsys@useobject{currentmarker}{}%
\end{pgfscope}%
\begin{pgfscope}%
\pgfsys@transformshift{2.910568in}{2.300860in}%
\pgfsys@useobject{currentmarker}{}%
\end{pgfscope}%
\begin{pgfscope}%
\pgfsys@transformshift{2.909921in}{2.294570in}%
\pgfsys@useobject{currentmarker}{}%
\end{pgfscope}%
\begin{pgfscope}%
\pgfsys@transformshift{2.909272in}{2.287763in}%
\pgfsys@useobject{currentmarker}{}%
\end{pgfscope}%
\begin{pgfscope}%
\pgfsys@transformshift{2.908939in}{2.279306in}%
\pgfsys@useobject{currentmarker}{}%
\end{pgfscope}%
\begin{pgfscope}%
\pgfsys@transformshift{2.909283in}{2.274664in}%
\pgfsys@useobject{currentmarker}{}%
\end{pgfscope}%
\begin{pgfscope}%
\pgfsys@transformshift{2.908131in}{2.268207in}%
\pgfsys@useobject{currentmarker}{}%
\end{pgfscope}%
\begin{pgfscope}%
\pgfsys@transformshift{2.907522in}{2.260926in}%
\pgfsys@useobject{currentmarker}{}%
\end{pgfscope}%
\begin{pgfscope}%
\pgfsys@transformshift{2.907042in}{2.252441in}%
\pgfsys@useobject{currentmarker}{}%
\end{pgfscope}%
\begin{pgfscope}%
\pgfsys@transformshift{2.909873in}{2.243024in}%
\pgfsys@useobject{currentmarker}{}%
\end{pgfscope}%
\begin{pgfscope}%
\pgfsys@transformshift{2.907623in}{2.231456in}%
\pgfsys@useobject{currentmarker}{}%
\end{pgfscope}%
\begin{pgfscope}%
\pgfsys@transformshift{2.907558in}{2.224975in}%
\pgfsys@useobject{currentmarker}{}%
\end{pgfscope}%
\begin{pgfscope}%
\pgfsys@transformshift{2.907341in}{2.221417in}%
\pgfsys@useobject{currentmarker}{}%
\end{pgfscope}%
\begin{pgfscope}%
\pgfsys@transformshift{2.908410in}{2.216387in}%
\pgfsys@useobject{currentmarker}{}%
\end{pgfscope}%
\begin{pgfscope}%
\pgfsys@transformshift{2.908296in}{2.213562in}%
\pgfsys@useobject{currentmarker}{}%
\end{pgfscope}%
\begin{pgfscope}%
\pgfsys@transformshift{2.908173in}{2.210123in}%
\pgfsys@useobject{currentmarker}{}%
\end{pgfscope}%
\begin{pgfscope}%
\pgfsys@transformshift{2.908329in}{2.208237in}%
\pgfsys@useobject{currentmarker}{}%
\end{pgfscope}%
\begin{pgfscope}%
\pgfsys@transformshift{2.908896in}{2.205577in}%
\pgfsys@useobject{currentmarker}{}%
\end{pgfscope}%
\begin{pgfscope}%
\pgfsys@transformshift{2.909144in}{2.202261in}%
\pgfsys@useobject{currentmarker}{}%
\end{pgfscope}%
\begin{pgfscope}%
\pgfsys@transformshift{2.909333in}{2.198423in}%
\pgfsys@useobject{currentmarker}{}%
\end{pgfscope}%
\begin{pgfscope}%
\pgfsys@transformshift{2.909446in}{2.193973in}%
\pgfsys@useobject{currentmarker}{}%
\end{pgfscope}%
\begin{pgfscope}%
\pgfsys@transformshift{2.909518in}{2.188811in}%
\pgfsys@useobject{currentmarker}{}%
\end{pgfscope}%
\begin{pgfscope}%
\pgfsys@transformshift{2.910067in}{2.186025in}%
\pgfsys@useobject{currentmarker}{}%
\end{pgfscope}%
\begin{pgfscope}%
\pgfsys@transformshift{2.910047in}{2.184463in}%
\pgfsys@useobject{currentmarker}{}%
\end{pgfscope}%
\begin{pgfscope}%
\pgfsys@transformshift{2.910041in}{2.183604in}%
\pgfsys@useobject{currentmarker}{}%
\end{pgfscope}%
\begin{pgfscope}%
\pgfsys@transformshift{2.910043in}{2.183132in}%
\pgfsys@useobject{currentmarker}{}%
\end{pgfscope}%
\begin{pgfscope}%
\pgfsys@transformshift{2.910082in}{2.182875in}%
\pgfsys@useobject{currentmarker}{}%
\end{pgfscope}%
\begin{pgfscope}%
\pgfsys@transformshift{2.910081in}{2.182732in}%
\pgfsys@useobject{currentmarker}{}%
\end{pgfscope}%
\begin{pgfscope}%
\pgfsys@transformshift{2.910078in}{2.182653in}%
\pgfsys@useobject{currentmarker}{}%
\end{pgfscope}%
\begin{pgfscope}%
\pgfsys@transformshift{2.910076in}{2.182610in}%
\pgfsys@useobject{currentmarker}{}%
\end{pgfscope}%
\begin{pgfscope}%
\pgfsys@transformshift{2.910075in}{2.182586in}%
\pgfsys@useobject{currentmarker}{}%
\end{pgfscope}%
\begin{pgfscope}%
\pgfsys@transformshift{2.910077in}{2.182573in}%
\pgfsys@useobject{currentmarker}{}%
\end{pgfscope}%
\begin{pgfscope}%
\pgfsys@transformshift{2.909808in}{2.181131in}%
\pgfsys@useobject{currentmarker}{}%
\end{pgfscope}%
\begin{pgfscope}%
\pgfsys@transformshift{2.909715in}{2.179036in}%
\pgfsys@useobject{currentmarker}{}%
\end{pgfscope}%
\begin{pgfscope}%
\pgfsys@transformshift{2.909549in}{2.177895in}%
\pgfsys@useobject{currentmarker}{}%
\end{pgfscope}%
\begin{pgfscope}%
\pgfsys@transformshift{2.909916in}{2.175131in}%
\pgfsys@useobject{currentmarker}{}%
\end{pgfscope}%
\begin{pgfscope}%
\pgfsys@transformshift{2.909860in}{2.173599in}%
\pgfsys@useobject{currentmarker}{}%
\end{pgfscope}%
\begin{pgfscope}%
\pgfsys@transformshift{2.909635in}{2.170374in}%
\pgfsys@useobject{currentmarker}{}%
\end{pgfscope}%
\begin{pgfscope}%
\pgfsys@transformshift{2.909012in}{2.166697in}%
\pgfsys@useobject{currentmarker}{}%
\end{pgfscope}%
\begin{pgfscope}%
\pgfsys@transformshift{2.909602in}{2.162112in}%
\pgfsys@useobject{currentmarker}{}%
\end{pgfscope}%
\begin{pgfscope}%
\pgfsys@transformshift{2.911100in}{2.155683in}%
\pgfsys@useobject{currentmarker}{}%
\end{pgfscope}%
\begin{pgfscope}%
\pgfsys@transformshift{2.910587in}{2.148356in}%
\pgfsys@useobject{currentmarker}{}%
\end{pgfscope}%
\begin{pgfscope}%
\pgfsys@transformshift{2.909241in}{2.140185in}%
\pgfsys@useobject{currentmarker}{}%
\end{pgfscope}%
\begin{pgfscope}%
\pgfsys@transformshift{2.910749in}{2.131169in}%
\pgfsys@useobject{currentmarker}{}%
\end{pgfscope}%
\begin{pgfscope}%
\pgfsys@transformshift{2.913280in}{2.120233in}%
\pgfsys@useobject{currentmarker}{}%
\end{pgfscope}%
\begin{pgfscope}%
\pgfsys@transformshift{2.912312in}{2.107298in}%
\pgfsys@useobject{currentmarker}{}%
\end{pgfscope}%
\begin{pgfscope}%
\pgfsys@transformshift{2.911128in}{2.093107in}%
\pgfsys@useobject{currentmarker}{}%
\end{pgfscope}%
\begin{pgfscope}%
\pgfsys@transformshift{2.911205in}{2.085276in}%
\pgfsys@useobject{currentmarker}{}%
\end{pgfscope}%
\begin{pgfscope}%
\pgfsys@transformshift{2.912773in}{2.076050in}%
\pgfsys@useobject{currentmarker}{}%
\end{pgfscope}%
\begin{pgfscope}%
\pgfsys@transformshift{2.912527in}{2.070910in}%
\pgfsys@useobject{currentmarker}{}%
\end{pgfscope}%
\begin{pgfscope}%
\pgfsys@transformshift{2.912521in}{2.068079in}%
\pgfsys@useobject{currentmarker}{}%
\end{pgfscope}%
\begin{pgfscope}%
\pgfsys@transformshift{2.912698in}{2.066532in}%
\pgfsys@useobject{currentmarker}{}%
\end{pgfscope}%
\begin{pgfscope}%
\pgfsys@transformshift{2.913402in}{2.064249in}%
\pgfsys@useobject{currentmarker}{}%
\end{pgfscope}%
\begin{pgfscope}%
\pgfsys@transformshift{2.913326in}{2.062937in}%
\pgfsys@useobject{currentmarker}{}%
\end{pgfscope}%
\begin{pgfscope}%
\pgfsys@transformshift{2.913266in}{2.062217in}%
\pgfsys@useobject{currentmarker}{}%
\end{pgfscope}%
\begin{pgfscope}%
\pgfsys@transformshift{2.913304in}{2.061821in}%
\pgfsys@useobject{currentmarker}{}%
\end{pgfscope}%
\begin{pgfscope}%
\pgfsys@transformshift{2.913534in}{2.060613in}%
\pgfsys@useobject{currentmarker}{}%
\end{pgfscope}%
\begin{pgfscope}%
\pgfsys@transformshift{2.913628in}{2.058656in}%
\pgfsys@useobject{currentmarker}{}%
\end{pgfscope}%
\begin{pgfscope}%
\pgfsys@transformshift{2.913585in}{2.057579in}%
\pgfsys@useobject{currentmarker}{}%
\end{pgfscope}%
\begin{pgfscope}%
\pgfsys@transformshift{2.913817in}{2.056052in}%
\pgfsys@useobject{currentmarker}{}%
\end{pgfscope}%
\begin{pgfscope}%
\pgfsys@transformshift{2.914271in}{2.052951in}%
\pgfsys@useobject{currentmarker}{}%
\end{pgfscope}%
\begin{pgfscope}%
\pgfsys@transformshift{2.914427in}{2.051235in}%
\pgfsys@useobject{currentmarker}{}%
\end{pgfscope}%
\begin{pgfscope}%
\pgfsys@transformshift{2.914477in}{2.048896in}%
\pgfsys@useobject{currentmarker}{}%
\end{pgfscope}%
\begin{pgfscope}%
\pgfsys@transformshift{2.914558in}{2.047612in}%
\pgfsys@useobject{currentmarker}{}%
\end{pgfscope}%
\begin{pgfscope}%
\pgfsys@transformshift{2.914710in}{2.044602in}%
\pgfsys@useobject{currentmarker}{}%
\end{pgfscope}%
\begin{pgfscope}%
\pgfsys@transformshift{2.915003in}{2.041087in}%
\pgfsys@useobject{currentmarker}{}%
\end{pgfscope}%
\begin{pgfscope}%
\pgfsys@transformshift{2.914801in}{2.036507in}%
\pgfsys@useobject{currentmarker}{}%
\end{pgfscope}%
\begin{pgfscope}%
\pgfsys@transformshift{2.914830in}{2.033986in}%
\pgfsys@useobject{currentmarker}{}%
\end{pgfscope}%
\begin{pgfscope}%
\pgfsys@transformshift{2.915266in}{2.030696in}%
\pgfsys@useobject{currentmarker}{}%
\end{pgfscope}%
\begin{pgfscope}%
\pgfsys@transformshift{2.915644in}{2.026896in}%
\pgfsys@useobject{currentmarker}{}%
\end{pgfscope}%
\begin{pgfscope}%
\pgfsys@transformshift{2.915586in}{2.024797in}%
\pgfsys@useobject{currentmarker}{}%
\end{pgfscope}%
\begin{pgfscope}%
\pgfsys@transformshift{2.915935in}{2.022113in}%
\pgfsys@useobject{currentmarker}{}%
\end{pgfscope}%
\begin{pgfscope}%
\pgfsys@transformshift{2.916897in}{2.017477in}%
\pgfsys@useobject{currentmarker}{}%
\end{pgfscope}%
\begin{pgfscope}%
\pgfsys@transformshift{2.918189in}{2.012288in}%
\pgfsys@useobject{currentmarker}{}%
\end{pgfscope}%
\begin{pgfscope}%
\pgfsys@transformshift{2.917725in}{2.005141in}%
\pgfsys@useobject{currentmarker}{}%
\end{pgfscope}%
\begin{pgfscope}%
\pgfsys@transformshift{2.918047in}{1.997490in}%
\pgfsys@useobject{currentmarker}{}%
\end{pgfscope}%
\begin{pgfscope}%
\pgfsys@transformshift{2.919457in}{1.988128in}%
\pgfsys@useobject{currentmarker}{}%
\end{pgfscope}%
\begin{pgfscope}%
\pgfsys@transformshift{2.921708in}{1.978442in}%
\pgfsys@useobject{currentmarker}{}%
\end{pgfscope}%
\begin{pgfscope}%
\pgfsys@transformshift{2.921906in}{1.966028in}%
\pgfsys@useobject{currentmarker}{}%
\end{pgfscope}%
\begin{pgfscope}%
\pgfsys@transformshift{2.922422in}{1.953004in}%
\pgfsys@useobject{currentmarker}{}%
\end{pgfscope}%
\begin{pgfscope}%
\pgfsys@transformshift{2.923946in}{1.939055in}%
\pgfsys@useobject{currentmarker}{}%
\end{pgfscope}%
\begin{pgfscope}%
\pgfsys@transformshift{2.928736in}{1.925148in}%
\pgfsys@useobject{currentmarker}{}%
\end{pgfscope}%
\begin{pgfscope}%
\pgfsys@transformshift{2.929177in}{1.909103in}%
\pgfsys@useobject{currentmarker}{}%
\end{pgfscope}%
\begin{pgfscope}%
\pgfsys@transformshift{2.928641in}{1.900291in}%
\pgfsys@useobject{currentmarker}{}%
\end{pgfscope}%
\begin{pgfscope}%
\pgfsys@transformshift{2.929235in}{1.895473in}%
\pgfsys@useobject{currentmarker}{}%
\end{pgfscope}%
\begin{pgfscope}%
\pgfsys@transformshift{2.929491in}{1.889884in}%
\pgfsys@useobject{currentmarker}{}%
\end{pgfscope}%
\begin{pgfscope}%
\pgfsys@transformshift{2.931455in}{1.883991in}%
\pgfsys@useobject{currentmarker}{}%
\end{pgfscope}%
\begin{pgfscope}%
\pgfsys@transformshift{2.930943in}{1.875560in}%
\pgfsys@useobject{currentmarker}{}%
\end{pgfscope}%
\begin{pgfscope}%
\pgfsys@transformshift{2.930519in}{1.870933in}%
\pgfsys@useobject{currentmarker}{}%
\end{pgfscope}%
\begin{pgfscope}%
\pgfsys@transformshift{2.931395in}{1.865662in}%
\pgfsys@useobject{currentmarker}{}%
\end{pgfscope}%
\begin{pgfscope}%
\pgfsys@transformshift{2.933312in}{1.859139in}%
\pgfsys@useobject{currentmarker}{}%
\end{pgfscope}%
\begin{pgfscope}%
\pgfsys@transformshift{2.932405in}{1.849927in}%
\pgfsys@useobject{currentmarker}{}%
\end{pgfscope}%
\begin{pgfscope}%
\pgfsys@transformshift{2.930521in}{1.839591in}%
\pgfsys@useobject{currentmarker}{}%
\end{pgfscope}%
\begin{pgfscope}%
\pgfsys@transformshift{2.930826in}{1.828477in}%
\pgfsys@useobject{currentmarker}{}%
\end{pgfscope}%
\begin{pgfscope}%
\pgfsys@transformshift{2.935407in}{1.815721in}%
\pgfsys@useobject{currentmarker}{}%
\end{pgfscope}%
\begin{pgfscope}%
\pgfsys@transformshift{2.934962in}{1.808279in}%
\pgfsys@useobject{currentmarker}{}%
\end{pgfscope}%
\begin{pgfscope}%
\pgfsys@transformshift{2.934872in}{1.804180in}%
\pgfsys@useobject{currentmarker}{}%
\end{pgfscope}%
\begin{pgfscope}%
\pgfsys@transformshift{2.935216in}{1.799608in}%
\pgfsys@useobject{currentmarker}{}%
\end{pgfscope}%
\begin{pgfscope}%
\pgfsys@transformshift{2.936969in}{1.794035in}%
\pgfsys@useobject{currentmarker}{}%
\end{pgfscope}%
\begin{pgfscope}%
\pgfsys@transformshift{2.936585in}{1.787467in}%
\pgfsys@useobject{currentmarker}{}%
\end{pgfscope}%
\begin{pgfscope}%
\pgfsys@transformshift{2.936300in}{1.783860in}%
\pgfsys@useobject{currentmarker}{}%
\end{pgfscope}%
\begin{pgfscope}%
\pgfsys@transformshift{2.936558in}{1.781887in}%
\pgfsys@useobject{currentmarker}{}%
\end{pgfscope}%
\begin{pgfscope}%
\pgfsys@transformshift{2.937862in}{1.778321in}%
\pgfsys@useobject{currentmarker}{}%
\end{pgfscope}%
\begin{pgfscope}%
\pgfsys@transformshift{2.936938in}{1.772301in}%
\pgfsys@useobject{currentmarker}{}%
\end{pgfscope}%
\begin{pgfscope}%
\pgfsys@transformshift{2.935926in}{1.765120in}%
\pgfsys@useobject{currentmarker}{}%
\end{pgfscope}%
\begin{pgfscope}%
\pgfsys@transformshift{2.937007in}{1.755734in}%
\pgfsys@useobject{currentmarker}{}%
\end{pgfscope}%
\begin{pgfscope}%
\pgfsys@transformshift{2.940863in}{1.744450in}%
\pgfsys@useobject{currentmarker}{}%
\end{pgfscope}%
\begin{pgfscope}%
\pgfsys@transformshift{2.938390in}{1.730509in}%
\pgfsys@useobject{currentmarker}{}%
\end{pgfscope}%
\begin{pgfscope}%
\pgfsys@transformshift{2.936007in}{1.715650in}%
\pgfsys@useobject{currentmarker}{}%
\end{pgfscope}%
\begin{pgfscope}%
\pgfsys@transformshift{2.937822in}{1.699773in}%
\pgfsys@useobject{currentmarker}{}%
\end{pgfscope}%
\begin{pgfscope}%
\pgfsys@transformshift{2.942421in}{1.682316in}%
\pgfsys@useobject{currentmarker}{}%
\end{pgfscope}%
\begin{pgfscope}%
\pgfsys@transformshift{2.942568in}{1.661904in}%
\pgfsys@useobject{currentmarker}{}%
\end{pgfscope}%
\begin{pgfscope}%
\pgfsys@transformshift{2.941527in}{1.640876in}%
\pgfsys@useobject{currentmarker}{}%
\end{pgfscope}%
\begin{pgfscope}%
\pgfsys@transformshift{2.945744in}{1.618665in}%
\pgfsys@useobject{currentmarker}{}%
\end{pgfscope}%
\begin{pgfscope}%
\pgfsys@transformshift{2.949396in}{1.606779in}%
\pgfsys@useobject{currentmarker}{}%
\end{pgfscope}%
\begin{pgfscope}%
\pgfsys@transformshift{2.950180in}{1.593886in}%
\pgfsys@useobject{currentmarker}{}%
\end{pgfscope}%
\begin{pgfscope}%
\pgfsys@transformshift{2.950375in}{1.586784in}%
\pgfsys@useobject{currentmarker}{}%
\end{pgfscope}%
\begin{pgfscope}%
\pgfsys@transformshift{2.951640in}{1.578483in}%
\pgfsys@useobject{currentmarker}{}%
\end{pgfscope}%
\begin{pgfscope}%
\pgfsys@transformshift{2.953643in}{1.574321in}%
\pgfsys@useobject{currentmarker}{}%
\end{pgfscope}%
\begin{pgfscope}%
\pgfsys@transformshift{2.953757in}{1.571783in}%
\pgfsys@useobject{currentmarker}{}%
\end{pgfscope}%
\begin{pgfscope}%
\pgfsys@transformshift{2.953813in}{1.570387in}%
\pgfsys@useobject{currentmarker}{}%
\end{pgfscope}%
\begin{pgfscope}%
\pgfsys@transformshift{2.953850in}{1.569620in}%
\pgfsys@useobject{currentmarker}{}%
\end{pgfscope}%
\begin{pgfscope}%
\pgfsys@transformshift{2.953987in}{1.569220in}%
\pgfsys@useobject{currentmarker}{}%
\end{pgfscope}%
\begin{pgfscope}%
\pgfsys@transformshift{2.953912in}{1.567874in}%
\pgfsys@useobject{currentmarker}{}%
\end{pgfscope}%
\begin{pgfscope}%
\pgfsys@transformshift{2.953928in}{1.567132in}%
\pgfsys@useobject{currentmarker}{}%
\end{pgfscope}%
\begin{pgfscope}%
\pgfsys@transformshift{2.953961in}{1.566726in}%
\pgfsys@useobject{currentmarker}{}%
\end{pgfscope}%
\begin{pgfscope}%
\pgfsys@transformshift{2.954023in}{1.566510in}%
\pgfsys@useobject{currentmarker}{}%
\end{pgfscope}%
\begin{pgfscope}%
\pgfsys@transformshift{2.954015in}{1.566387in}%
\pgfsys@useobject{currentmarker}{}%
\end{pgfscope}%
\begin{pgfscope}%
\pgfsys@transformshift{2.954017in}{1.566319in}%
\pgfsys@useobject{currentmarker}{}%
\end{pgfscope}%
\begin{pgfscope}%
\pgfsys@transformshift{2.954016in}{1.566282in}%
\pgfsys@useobject{currentmarker}{}%
\end{pgfscope}%
\begin{pgfscope}%
\pgfsys@transformshift{2.954021in}{1.566262in}%
\pgfsys@useobject{currentmarker}{}%
\end{pgfscope}%
\begin{pgfscope}%
\pgfsys@transformshift{2.954020in}{1.566250in}%
\pgfsys@useobject{currentmarker}{}%
\end{pgfscope}%
\begin{pgfscope}%
\pgfsys@transformshift{2.954020in}{1.566244in}%
\pgfsys@useobject{currentmarker}{}%
\end{pgfscope}%
\begin{pgfscope}%
\pgfsys@transformshift{2.954020in}{1.566241in}%
\pgfsys@useobject{currentmarker}{}%
\end{pgfscope}%
\begin{pgfscope}%
\pgfsys@transformshift{2.954512in}{1.564105in}%
\pgfsys@useobject{currentmarker}{}%
\end{pgfscope}%
\begin{pgfscope}%
\pgfsys@transformshift{2.954984in}{1.561353in}%
\pgfsys@useobject{currentmarker}{}%
\end{pgfscope}%
\begin{pgfscope}%
\pgfsys@transformshift{2.954821in}{1.555934in}%
\pgfsys@useobject{currentmarker}{}%
\end{pgfscope}%
\begin{pgfscope}%
\pgfsys@transformshift{2.954609in}{1.552960in}%
\pgfsys@useobject{currentmarker}{}%
\end{pgfscope}%
\begin{pgfscope}%
\pgfsys@transformshift{2.955249in}{1.548427in}%
\pgfsys@useobject{currentmarker}{}%
\end{pgfscope}%
\begin{pgfscope}%
\pgfsys@transformshift{2.956370in}{1.543486in}%
\pgfsys@useobject{currentmarker}{}%
\end{pgfscope}%
\begin{pgfscope}%
\pgfsys@transformshift{2.956734in}{1.537398in}%
\pgfsys@useobject{currentmarker}{}%
\end{pgfscope}%
\begin{pgfscope}%
\pgfsys@transformshift{2.956758in}{1.534043in}%
\pgfsys@useobject{currentmarker}{}%
\end{pgfscope}%
\begin{pgfscope}%
\pgfsys@transformshift{2.956863in}{1.532201in}%
\pgfsys@useobject{currentmarker}{}%
\end{pgfscope}%
\begin{pgfscope}%
\pgfsys@transformshift{2.957443in}{1.527930in}%
\pgfsys@useobject{currentmarker}{}%
\end{pgfscope}%
\begin{pgfscope}%
\pgfsys@transformshift{2.957658in}{1.522961in}%
\pgfsys@useobject{currentmarker}{}%
\end{pgfscope}%
\begin{pgfscope}%
\pgfsys@transformshift{2.957425in}{1.517452in}%
\pgfsys@useobject{currentmarker}{}%
\end{pgfscope}%
\begin{pgfscope}%
\pgfsys@transformshift{2.957574in}{1.514423in}%
\pgfsys@useobject{currentmarker}{}%
\end{pgfscope}%
\begin{pgfscope}%
\pgfsys@transformshift{2.958157in}{1.508812in}%
\pgfsys@useobject{currentmarker}{}%
\end{pgfscope}%
\begin{pgfscope}%
\pgfsys@transformshift{2.958552in}{1.505735in}%
\pgfsys@useobject{currentmarker}{}%
\end{pgfscope}%
\begin{pgfscope}%
\pgfsys@transformshift{2.958466in}{1.501526in}%
\pgfsys@useobject{currentmarker}{}%
\end{pgfscope}%
\begin{pgfscope}%
\pgfsys@transformshift{2.958429in}{1.499211in}%
\pgfsys@useobject{currentmarker}{}%
\end{pgfscope}%
\begin{pgfscope}%
\pgfsys@transformshift{2.959380in}{1.494151in}%
\pgfsys@useobject{currentmarker}{}%
\end{pgfscope}%
\begin{pgfscope}%
\pgfsys@transformshift{2.959808in}{1.491352in}%
\pgfsys@useobject{currentmarker}{}%
\end{pgfscope}%
\begin{pgfscope}%
\pgfsys@transformshift{2.959675in}{1.487233in}%
\pgfsys@useobject{currentmarker}{}%
\end{pgfscope}%
\begin{pgfscope}%
\pgfsys@transformshift{2.959603in}{1.484967in}%
\pgfsys@useobject{currentmarker}{}%
\end{pgfscope}%
\begin{pgfscope}%
\pgfsys@transformshift{2.960206in}{1.481386in}%
\pgfsys@useobject{currentmarker}{}%
\end{pgfscope}%
\begin{pgfscope}%
\pgfsys@transformshift{2.961684in}{1.475538in}%
\pgfsys@useobject{currentmarker}{}%
\end{pgfscope}%
\begin{pgfscope}%
\pgfsys@transformshift{2.961181in}{1.468288in}%
\pgfsys@useobject{currentmarker}{}%
\end{pgfscope}%
\begin{pgfscope}%
\pgfsys@transformshift{2.960566in}{1.460363in}%
\pgfsys@useobject{currentmarker}{}%
\end{pgfscope}%
\begin{pgfscope}%
\pgfsys@transformshift{2.960894in}{1.451844in}%
\pgfsys@useobject{currentmarker}{}%
\end{pgfscope}%
\begin{pgfscope}%
\pgfsys@transformshift{2.961921in}{1.442246in}%
\pgfsys@useobject{currentmarker}{}%
\end{pgfscope}%
\begin{pgfscope}%
\pgfsys@transformshift{2.962102in}{1.436940in}%
\pgfsys@useobject{currentmarker}{}%
\end{pgfscope}%
\begin{pgfscope}%
\pgfsys@transformshift{2.961348in}{1.430932in}%
\pgfsys@useobject{currentmarker}{}%
\end{pgfscope}%
\begin{pgfscope}%
\pgfsys@transformshift{2.961780in}{1.427629in}%
\pgfsys@useobject{currentmarker}{}%
\end{pgfscope}%
\begin{pgfscope}%
\pgfsys@transformshift{2.961926in}{1.421315in}%
\pgfsys@useobject{currentmarker}{}%
\end{pgfscope}%
\begin{pgfscope}%
\pgfsys@transformshift{2.962422in}{1.417877in}%
\pgfsys@useobject{currentmarker}{}%
\end{pgfscope}%
\begin{pgfscope}%
\pgfsys@transformshift{2.962372in}{1.412638in}%
\pgfsys@useobject{currentmarker}{}%
\end{pgfscope}%
\begin{pgfscope}%
\pgfsys@transformshift{2.962597in}{1.409765in}%
\pgfsys@useobject{currentmarker}{}%
\end{pgfscope}%
\begin{pgfscope}%
\pgfsys@transformshift{2.963783in}{1.404637in}%
\pgfsys@useobject{currentmarker}{}%
\end{pgfscope}%
\begin{pgfscope}%
\pgfsys@transformshift{2.964139in}{1.401763in}%
\pgfsys@useobject{currentmarker}{}%
\end{pgfscope}%
\begin{pgfscope}%
\pgfsys@transformshift{2.964119in}{1.397691in}%
\pgfsys@useobject{currentmarker}{}%
\end{pgfscope}%
\begin{pgfscope}%
\pgfsys@transformshift{2.963592in}{1.393020in}%
\pgfsys@useobject{currentmarker}{}%
\end{pgfscope}%
\begin{pgfscope}%
\pgfsys@transformshift{2.965098in}{1.384989in}%
\pgfsys@useobject{currentmarker}{}%
\end{pgfscope}%
\begin{pgfscope}%
\pgfsys@transformshift{2.966961in}{1.376162in}%
\pgfsys@useobject{currentmarker}{}%
\end{pgfscope}%
\begin{pgfscope}%
\pgfsys@transformshift{2.965746in}{1.364684in}%
\pgfsys@useobject{currentmarker}{}%
\end{pgfscope}%
\begin{pgfscope}%
\pgfsys@transformshift{2.965650in}{1.358336in}%
\pgfsys@useobject{currentmarker}{}%
\end{pgfscope}%
\begin{pgfscope}%
\pgfsys@transformshift{2.966901in}{1.350341in}%
\pgfsys@useobject{currentmarker}{}%
\end{pgfscope}%
\begin{pgfscope}%
\pgfsys@transformshift{2.968696in}{1.340868in}%
\pgfsys@useobject{currentmarker}{}%
\end{pgfscope}%
\begin{pgfscope}%
\pgfsys@transformshift{2.968635in}{1.328272in}%
\pgfsys@useobject{currentmarker}{}%
\end{pgfscope}%
\begin{pgfscope}%
\pgfsys@transformshift{2.966216in}{1.314156in}%
\pgfsys@useobject{currentmarker}{}%
\end{pgfscope}%
\begin{pgfscope}%
\pgfsys@transformshift{2.970128in}{1.299538in}%
\pgfsys@useobject{currentmarker}{}%
\end{pgfscope}%
\begin{pgfscope}%
\pgfsys@transformshift{2.975476in}{1.282663in}%
\pgfsys@useobject{currentmarker}{}%
\end{pgfscope}%
\begin{pgfscope}%
\pgfsys@transformshift{2.975492in}{1.261841in}%
\pgfsys@useobject{currentmarker}{}%
\end{pgfscope}%
\begin{pgfscope}%
\pgfsys@transformshift{2.972387in}{1.240650in}%
\pgfsys@useobject{currentmarker}{}%
\end{pgfscope}%
\begin{pgfscope}%
\pgfsys@transformshift{2.977862in}{1.217861in}%
\pgfsys@useobject{currentmarker}{}%
\end{pgfscope}%
\begin{pgfscope}%
\pgfsys@transformshift{2.986372in}{1.193585in}%
\pgfsys@useobject{currentmarker}{}%
\end{pgfscope}%
\begin{pgfscope}%
\pgfsys@transformshift{2.985531in}{1.165137in}%
\pgfsys@useobject{currentmarker}{}%
\end{pgfscope}%
\begin{pgfscope}%
\pgfsys@transformshift{2.985737in}{1.149485in}%
\pgfsys@useobject{currentmarker}{}%
\end{pgfscope}%
\begin{pgfscope}%
\pgfsys@transformshift{2.985573in}{1.140877in}%
\pgfsys@useobject{currentmarker}{}%
\end{pgfscope}%
\begin{pgfscope}%
\pgfsys@transformshift{2.985724in}{1.136144in}%
\pgfsys@useobject{currentmarker}{}%
\end{pgfscope}%
\begin{pgfscope}%
\pgfsys@transformshift{2.986675in}{1.130867in}%
\pgfsys@useobject{currentmarker}{}%
\end{pgfscope}%
\begin{pgfscope}%
\pgfsys@transformshift{2.986810in}{1.127921in}%
\pgfsys@useobject{currentmarker}{}%
\end{pgfscope}%
\begin{pgfscope}%
\pgfsys@transformshift{2.986943in}{1.126304in}%
\pgfsys@useobject{currentmarker}{}%
\end{pgfscope}%
\begin{pgfscope}%
\pgfsys@transformshift{2.986482in}{1.123528in}%
\pgfsys@useobject{currentmarker}{}%
\end{pgfscope}%
\begin{pgfscope}%
\pgfsys@transformshift{2.986175in}{1.122011in}%
\pgfsys@useobject{currentmarker}{}%
\end{pgfscope}%
\begin{pgfscope}%
\pgfsys@transformshift{2.984727in}{1.120275in}%
\pgfsys@useobject{currentmarker}{}%
\end{pgfscope}%
\begin{pgfscope}%
\pgfsys@transformshift{2.981102in}{1.117292in}%
\pgfsys@useobject{currentmarker}{}%
\end{pgfscope}%
\begin{pgfscope}%
\pgfsys@transformshift{2.976080in}{1.115162in}%
\pgfsys@useobject{currentmarker}{}%
\end{pgfscope}%
\begin{pgfscope}%
\pgfsys@transformshift{2.969965in}{1.113661in}%
\pgfsys@useobject{currentmarker}{}%
\end{pgfscope}%
\begin{pgfscope}%
\pgfsys@transformshift{2.966502in}{1.113720in}%
\pgfsys@useobject{currentmarker}{}%
\end{pgfscope}%
\begin{pgfscope}%
\pgfsys@transformshift{2.964695in}{1.114323in}%
\pgfsys@useobject{currentmarker}{}%
\end{pgfscope}%
\begin{pgfscope}%
\pgfsys@transformshift{2.961861in}{1.115119in}%
\pgfsys@useobject{currentmarker}{}%
\end{pgfscope}%
\begin{pgfscope}%
\pgfsys@transformshift{2.958150in}{1.115650in}%
\pgfsys@useobject{currentmarker}{}%
\end{pgfscope}%
\begin{pgfscope}%
\pgfsys@transformshift{2.953546in}{1.116259in}%
\pgfsys@useobject{currentmarker}{}%
\end{pgfscope}%
\begin{pgfscope}%
\pgfsys@transformshift{2.951001in}{1.116465in}%
\pgfsys@useobject{currentmarker}{}%
\end{pgfscope}%
\begin{pgfscope}%
\pgfsys@transformshift{2.949596in}{1.116472in}%
\pgfsys@useobject{currentmarker}{}%
\end{pgfscope}%
\begin{pgfscope}%
\pgfsys@transformshift{2.948835in}{1.116604in}%
\pgfsys@useobject{currentmarker}{}%
\end{pgfscope}%
\begin{pgfscope}%
\pgfsys@transformshift{2.947442in}{1.116634in}%
\pgfsys@useobject{currentmarker}{}%
\end{pgfscope}%
\begin{pgfscope}%
\pgfsys@transformshift{2.944953in}{1.116220in}%
\pgfsys@useobject{currentmarker}{}%
\end{pgfscope}%
\begin{pgfscope}%
\pgfsys@transformshift{2.941910in}{1.116402in}%
\pgfsys@useobject{currentmarker}{}%
\end{pgfscope}%
\begin{pgfscope}%
\pgfsys@transformshift{2.933744in}{1.116216in}%
\pgfsys@useobject{currentmarker}{}%
\end{pgfscope}%
\begin{pgfscope}%
\pgfsys@transformshift{2.924206in}{1.116223in}%
\pgfsys@useobject{currentmarker}{}%
\end{pgfscope}%
\begin{pgfscope}%
\pgfsys@transformshift{2.912050in}{1.114733in}%
\pgfsys@useobject{currentmarker}{}%
\end{pgfscope}%
\begin{pgfscope}%
\pgfsys@transformshift{2.899054in}{1.114963in}%
\pgfsys@useobject{currentmarker}{}%
\end{pgfscope}%
\begin{pgfscope}%
\pgfsys@transformshift{2.882053in}{1.115400in}%
\pgfsys@useobject{currentmarker}{}%
\end{pgfscope}%
\begin{pgfscope}%
\pgfsys@transformshift{2.863038in}{1.114926in}%
\pgfsys@useobject{currentmarker}{}%
\end{pgfscope}%
\begin{pgfscope}%
\pgfsys@transformshift{2.842907in}{1.112554in}%
\pgfsys@useobject{currentmarker}{}%
\end{pgfscope}%
\begin{pgfscope}%
\pgfsys@transformshift{2.822047in}{1.112088in}%
\pgfsys@useobject{currentmarker}{}%
\end{pgfscope}%
\begin{pgfscope}%
\pgfsys@transformshift{2.795206in}{1.110589in}%
\pgfsys@useobject{currentmarker}{}%
\end{pgfscope}%
\begin{pgfscope}%
\pgfsys@transformshift{2.766343in}{1.106586in}%
\pgfsys@useobject{currentmarker}{}%
\end{pgfscope}%
\begin{pgfscope}%
\pgfsys@transformshift{2.736384in}{1.099969in}%
\pgfsys@useobject{currentmarker}{}%
\end{pgfscope}%
\begin{pgfscope}%
\pgfsys@transformshift{2.705082in}{1.099117in}%
\pgfsys@useobject{currentmarker}{}%
\end{pgfscope}%
\begin{pgfscope}%
\pgfsys@transformshift{2.668378in}{1.094871in}%
\pgfsys@useobject{currentmarker}{}%
\end{pgfscope}%
\begin{pgfscope}%
\pgfsys@transformshift{2.632537in}{1.081900in}%
\pgfsys@useobject{currentmarker}{}%
\end{pgfscope}%
\begin{pgfscope}%
\pgfsys@transformshift{2.593376in}{1.074484in}%
\pgfsys@useobject{currentmarker}{}%
\end{pgfscope}%
\begin{pgfscope}%
\pgfsys@transformshift{2.553075in}{1.073439in}%
\pgfsys@useobject{currentmarker}{}%
\end{pgfscope}%
\begin{pgfscope}%
\pgfsys@transformshift{2.507328in}{1.071726in}%
\pgfsys@useobject{currentmarker}{}%
\end{pgfscope}%
\begin{pgfscope}%
\pgfsys@transformshift{2.482176in}{1.070550in}%
\pgfsys@useobject{currentmarker}{}%
\end{pgfscope}%
\begin{pgfscope}%
\pgfsys@transformshift{2.453837in}{1.068060in}%
\pgfsys@useobject{currentmarker}{}%
\end{pgfscope}%
\begin{pgfscope}%
\pgfsys@transformshift{2.438220in}{1.067091in}%
\pgfsys@useobject{currentmarker}{}%
\end{pgfscope}%
\begin{pgfscope}%
\pgfsys@transformshift{2.420886in}{1.066525in}%
\pgfsys@useobject{currentmarker}{}%
\end{pgfscope}%
\begin{pgfscope}%
\pgfsys@transformshift{2.403039in}{1.066912in}%
\pgfsys@useobject{currentmarker}{}%
\end{pgfscope}%
\begin{pgfscope}%
\pgfsys@transformshift{2.384208in}{1.065472in}%
\pgfsys@useobject{currentmarker}{}%
\end{pgfscope}%
\begin{pgfscope}%
\pgfsys@transformshift{2.373822in}{1.065660in}%
\pgfsys@useobject{currentmarker}{}%
\end{pgfscope}%
\begin{pgfscope}%
\pgfsys@transformshift{2.360467in}{1.065394in}%
\pgfsys@useobject{currentmarker}{}%
\end{pgfscope}%
\begin{pgfscope}%
\pgfsys@transformshift{2.353123in}{1.065221in}%
\pgfsys@useobject{currentmarker}{}%
\end{pgfscope}%
\begin{pgfscope}%
\pgfsys@transformshift{2.345161in}{1.064050in}%
\pgfsys@useobject{currentmarker}{}%
\end{pgfscope}%
\begin{pgfscope}%
\pgfsys@transformshift{2.340741in}{1.063816in}%
\pgfsys@useobject{currentmarker}{}%
\end{pgfscope}%
\begin{pgfscope}%
\pgfsys@transformshift{2.333435in}{1.063786in}%
\pgfsys@useobject{currentmarker}{}%
\end{pgfscope}%
\begin{pgfscope}%
\pgfsys@transformshift{2.329418in}{1.063701in}%
\pgfsys@useobject{currentmarker}{}%
\end{pgfscope}%
\begin{pgfscope}%
\pgfsys@transformshift{2.323758in}{1.062547in}%
\pgfsys@useobject{currentmarker}{}%
\end{pgfscope}%
\begin{pgfscope}%
\pgfsys@transformshift{2.320582in}{1.062509in}%
\pgfsys@useobject{currentmarker}{}%
\end{pgfscope}%
\begin{pgfscope}%
\pgfsys@transformshift{2.315502in}{1.062318in}%
\pgfsys@useobject{currentmarker}{}%
\end{pgfscope}%
\begin{pgfscope}%
\pgfsys@transformshift{2.309701in}{1.062102in}%
\pgfsys@useobject{currentmarker}{}%
\end{pgfscope}%
\begin{pgfscope}%
\pgfsys@transformshift{2.302436in}{1.060873in}%
\pgfsys@useobject{currentmarker}{}%
\end{pgfscope}%
\begin{pgfscope}%
\pgfsys@transformshift{2.298388in}{1.060686in}%
\pgfsys@useobject{currentmarker}{}%
\end{pgfscope}%
\begin{pgfscope}%
\pgfsys@transformshift{2.292907in}{1.060511in}%
\pgfsys@useobject{currentmarker}{}%
\end{pgfscope}%
\begin{pgfscope}%
\pgfsys@transformshift{2.286879in}{1.059872in}%
\pgfsys@useobject{currentmarker}{}%
\end{pgfscope}%
\begin{pgfscope}%
\pgfsys@transformshift{2.279770in}{1.059882in}%
\pgfsys@useobject{currentmarker}{}%
\end{pgfscope}%
\begin{pgfscope}%
\pgfsys@transformshift{2.271148in}{1.058557in}%
\pgfsys@useobject{currentmarker}{}%
\end{pgfscope}%
\begin{pgfscope}%
\pgfsys@transformshift{2.266361in}{1.058234in}%
\pgfsys@useobject{currentmarker}{}%
\end{pgfscope}%
\begin{pgfscope}%
\pgfsys@transformshift{2.260255in}{1.057831in}%
\pgfsys@useobject{currentmarker}{}%
\end{pgfscope}%
\begin{pgfscope}%
\pgfsys@transformshift{2.253629in}{1.057766in}%
\pgfsys@useobject{currentmarker}{}%
\end{pgfscope}%
\begin{pgfscope}%
\pgfsys@transformshift{2.245521in}{1.056611in}%
\pgfsys@useobject{currentmarker}{}%
\end{pgfscope}%
\begin{pgfscope}%
\pgfsys@transformshift{2.241020in}{1.056428in}%
\pgfsys@useobject{currentmarker}{}%
\end{pgfscope}%
\begin{pgfscope}%
\pgfsys@transformshift{2.233838in}{1.056467in}%
\pgfsys@useobject{currentmarker}{}%
\end{pgfscope}%
\begin{pgfscope}%
\pgfsys@transformshift{2.229910in}{1.056054in}%
\pgfsys@useobject{currentmarker}{}%
\end{pgfscope}%
\begin{pgfscope}%
\pgfsys@transformshift{2.224609in}{1.055758in}%
\pgfsys@useobject{currentmarker}{}%
\end{pgfscope}%
\begin{pgfscope}%
\pgfsys@transformshift{2.221689in}{1.055723in}%
\pgfsys@useobject{currentmarker}{}%
\end{pgfscope}%
\begin{pgfscope}%
\pgfsys@transformshift{2.216983in}{1.055061in}%
\pgfsys@useobject{currentmarker}{}%
\end{pgfscope}%
\begin{pgfscope}%
\pgfsys@transformshift{2.211352in}{1.054370in}%
\pgfsys@useobject{currentmarker}{}%
\end{pgfscope}%
\begin{pgfscope}%
\pgfsys@transformshift{2.204872in}{1.053637in}%
\pgfsys@useobject{currentmarker}{}%
\end{pgfscope}%
\begin{pgfscope}%
\pgfsys@transformshift{2.201296in}{1.053359in}%
\pgfsys@useobject{currentmarker}{}%
\end{pgfscope}%
\begin{pgfscope}%
\pgfsys@transformshift{2.194258in}{1.053600in}%
\pgfsys@useobject{currentmarker}{}%
\end{pgfscope}%
\begin{pgfscope}%
\pgfsys@transformshift{2.190388in}{1.053429in}%
\pgfsys@useobject{currentmarker}{}%
\end{pgfscope}%
\begin{pgfscope}%
\pgfsys@transformshift{2.184835in}{1.052497in}%
\pgfsys@useobject{currentmarker}{}%
\end{pgfscope}%
\begin{pgfscope}%
\pgfsys@transformshift{2.181743in}{1.052306in}%
\pgfsys@useobject{currentmarker}{}%
\end{pgfscope}%
\begin{pgfscope}%
\pgfsys@transformshift{2.175467in}{1.052207in}%
\pgfsys@useobject{currentmarker}{}%
\end{pgfscope}%
\begin{pgfscope}%
\pgfsys@transformshift{2.172026in}{1.051935in}%
\pgfsys@useobject{currentmarker}{}%
\end{pgfscope}%
\begin{pgfscope}%
\pgfsys@transformshift{2.166236in}{1.051175in}%
\pgfsys@useobject{currentmarker}{}%
\end{pgfscope}%
\begin{pgfscope}%
\pgfsys@transformshift{2.159855in}{1.050733in}%
\pgfsys@useobject{currentmarker}{}%
\end{pgfscope}%
\begin{pgfscope}%
\pgfsys@transformshift{2.151500in}{1.050443in}%
\pgfsys@useobject{currentmarker}{}%
\end{pgfscope}%
\begin{pgfscope}%
\pgfsys@transformshift{2.142577in}{1.049331in}%
\pgfsys@useobject{currentmarker}{}%
\end{pgfscope}%
\begin{pgfscope}%
\pgfsys@transformshift{2.132090in}{1.046941in}%
\pgfsys@useobject{currentmarker}{}%
\end{pgfscope}%
\begin{pgfscope}%
\pgfsys@transformshift{2.126176in}{1.046805in}%
\pgfsys@useobject{currentmarker}{}%
\end{pgfscope}%
\begin{pgfscope}%
\pgfsys@transformshift{2.118284in}{1.045834in}%
\pgfsys@useobject{currentmarker}{}%
\end{pgfscope}%
\begin{pgfscope}%
\pgfsys@transformshift{2.113914in}{1.045682in}%
\pgfsys@useobject{currentmarker}{}%
\end{pgfscope}%
\begin{pgfscope}%
\pgfsys@transformshift{2.108219in}{1.044546in}%
\pgfsys@useobject{currentmarker}{}%
\end{pgfscope}%
\begin{pgfscope}%
\pgfsys@transformshift{2.105027in}{1.044637in}%
\pgfsys@useobject{currentmarker}{}%
\end{pgfscope}%
\begin{pgfscope}%
\pgfsys@transformshift{2.100044in}{1.044254in}%
\pgfsys@useobject{currentmarker}{}%
\end{pgfscope}%
\begin{pgfscope}%
\pgfsys@transformshift{2.097296in}{1.044247in}%
\pgfsys@useobject{currentmarker}{}%
\end{pgfscope}%
\begin{pgfscope}%
\pgfsys@transformshift{2.093552in}{1.043341in}%
\pgfsys@useobject{currentmarker}{}%
\end{pgfscope}%
\begin{pgfscope}%
\pgfsys@transformshift{2.091437in}{1.043211in}%
\pgfsys@useobject{currentmarker}{}%
\end{pgfscope}%
\begin{pgfscope}%
\pgfsys@transformshift{2.087577in}{1.042828in}%
\pgfsys@useobject{currentmarker}{}%
\end{pgfscope}%
\begin{pgfscope}%
\pgfsys@transformshift{2.083243in}{1.042517in}%
\pgfsys@useobject{currentmarker}{}%
\end{pgfscope}%
\begin{pgfscope}%
\pgfsys@transformshift{2.078073in}{1.041340in}%
\pgfsys@useobject{currentmarker}{}%
\end{pgfscope}%
\begin{pgfscope}%
\pgfsys@transformshift{2.071250in}{1.040341in}%
\pgfsys@useobject{currentmarker}{}%
\end{pgfscope}%
\begin{pgfscope}%
\pgfsys@transformshift{2.063472in}{1.040008in}%
\pgfsys@useobject{currentmarker}{}%
\end{pgfscope}%
\begin{pgfscope}%
\pgfsys@transformshift{2.053886in}{1.039676in}%
\pgfsys@useobject{currentmarker}{}%
\end{pgfscope}%
\begin{pgfscope}%
\pgfsys@transformshift{2.043615in}{1.038727in}%
\pgfsys@useobject{currentmarker}{}%
\end{pgfscope}%
\begin{pgfscope}%
\pgfsys@transformshift{2.030705in}{1.038179in}%
\pgfsys@useobject{currentmarker}{}%
\end{pgfscope}%
\begin{pgfscope}%
\pgfsys@transformshift{2.023632in}{1.037480in}%
\pgfsys@useobject{currentmarker}{}%
\end{pgfscope}%
\begin{pgfscope}%
\pgfsys@transformshift{2.015592in}{1.036806in}%
\pgfsys@useobject{currentmarker}{}%
\end{pgfscope}%
\begin{pgfscope}%
\pgfsys@transformshift{2.006970in}{1.035984in}%
\pgfsys@useobject{currentmarker}{}%
\end{pgfscope}%
\begin{pgfscope}%
\pgfsys@transformshift{1.996726in}{1.034569in}%
\pgfsys@useobject{currentmarker}{}%
\end{pgfscope}%
\begin{pgfscope}%
\pgfsys@transformshift{1.985934in}{1.033900in}%
\pgfsys@useobject{currentmarker}{}%
\end{pgfscope}%
\begin{pgfscope}%
\pgfsys@transformshift{1.973016in}{1.033135in}%
\pgfsys@useobject{currentmarker}{}%
\end{pgfscope}%
\begin{pgfscope}%
\pgfsys@transformshift{1.965946in}{1.032316in}%
\pgfsys@useobject{currentmarker}{}%
\end{pgfscope}%
\begin{pgfscope}%
\pgfsys@transformshift{1.956060in}{1.030600in}%
\pgfsys@useobject{currentmarker}{}%
\end{pgfscope}%
\begin{pgfscope}%
\pgfsys@transformshift{1.944763in}{1.029835in}%
\pgfsys@useobject{currentmarker}{}%
\end{pgfscope}%
\begin{pgfscope}%
\pgfsys@transformshift{1.930949in}{1.028777in}%
\pgfsys@useobject{currentmarker}{}%
\end{pgfscope}%
\begin{pgfscope}%
\pgfsys@transformshift{1.916619in}{1.028347in}%
\pgfsys@useobject{currentmarker}{}%
\end{pgfscope}%
\begin{pgfscope}%
\pgfsys@transformshift{1.901960in}{1.026365in}%
\pgfsys@useobject{currentmarker}{}%
\end{pgfscope}%
\begin{pgfscope}%
\pgfsys@transformshift{1.886681in}{1.025401in}%
\pgfsys@useobject{currentmarker}{}%
\end{pgfscope}%
\begin{pgfscope}%
\pgfsys@transformshift{1.868161in}{1.025525in}%
\pgfsys@useobject{currentmarker}{}%
\end{pgfscope}%
\begin{pgfscope}%
\pgfsys@transformshift{1.858022in}{1.024551in}%
\pgfsys@useobject{currentmarker}{}%
\end{pgfscope}%
\begin{pgfscope}%
\pgfsys@transformshift{1.845662in}{1.022908in}%
\pgfsys@useobject{currentmarker}{}%
\end{pgfscope}%
\begin{pgfscope}%
\pgfsys@transformshift{1.838810in}{1.022626in}%
\pgfsys@useobject{currentmarker}{}%
\end{pgfscope}%
\begin{pgfscope}%
\pgfsys@transformshift{1.830507in}{1.021690in}%
\pgfsys@useobject{currentmarker}{}%
\end{pgfscope}%
\begin{pgfscope}%
\pgfsys@transformshift{1.821228in}{1.020199in}%
\pgfsys@useobject{currentmarker}{}%
\end{pgfscope}%
\begin{pgfscope}%
\pgfsys@transformshift{1.809543in}{1.018085in}%
\pgfsys@useobject{currentmarker}{}%
\end{pgfscope}%
\begin{pgfscope}%
\pgfsys@transformshift{1.797062in}{1.017032in}%
\pgfsys@useobject{currentmarker}{}%
\end{pgfscope}%
\begin{pgfscope}%
\pgfsys@transformshift{1.782408in}{1.014986in}%
\pgfsys@useobject{currentmarker}{}%
\end{pgfscope}%
\begin{pgfscope}%
\pgfsys@transformshift{1.766957in}{1.011761in}%
\pgfsys@useobject{currentmarker}{}%
\end{pgfscope}%
\begin{pgfscope}%
\pgfsys@transformshift{1.749242in}{1.008416in}%
\pgfsys@useobject{currentmarker}{}%
\end{pgfscope}%
\begin{pgfscope}%
\pgfsys@transformshift{1.730626in}{1.007164in}%
\pgfsys@useobject{currentmarker}{}%
\end{pgfscope}%
\begin{pgfscope}%
\pgfsys@transformshift{1.709948in}{1.005627in}%
\pgfsys@useobject{currentmarker}{}%
\end{pgfscope}%
\begin{pgfscope}%
\pgfsys@transformshift{1.688763in}{1.002096in}%
\pgfsys@useobject{currentmarker}{}%
\end{pgfscope}%
\begin{pgfscope}%
\pgfsys@transformshift{1.665407in}{0.999343in}%
\pgfsys@useobject{currentmarker}{}%
\end{pgfscope}%
\begin{pgfscope}%
\pgfsys@transformshift{1.652498in}{0.998534in}%
\pgfsys@useobject{currentmarker}{}%
\end{pgfscope}%
\begin{pgfscope}%
\pgfsys@transformshift{1.638454in}{0.996722in}%
\pgfsys@useobject{currentmarker}{}%
\end{pgfscope}%
\begin{pgfscope}%
\pgfsys@transformshift{1.623687in}{0.993426in}%
\pgfsys@useobject{currentmarker}{}%
\end{pgfscope}%
\begin{pgfscope}%
\pgfsys@transformshift{1.607442in}{0.991036in}%
\pgfsys@useobject{currentmarker}{}%
\end{pgfscope}%
\begin{pgfscope}%
\pgfsys@transformshift{1.590226in}{0.990868in}%
\pgfsys@useobject{currentmarker}{}%
\end{pgfscope}%
\begin{pgfscope}%
\pgfsys@transformshift{1.571809in}{0.988659in}%
\pgfsys@useobject{currentmarker}{}%
\end{pgfscope}%
\begin{pgfscope}%
\pgfsys@transformshift{1.552171in}{0.987168in}%
\pgfsys@useobject{currentmarker}{}%
\end{pgfscope}%
\begin{pgfscope}%
\pgfsys@transformshift{1.531516in}{0.983932in}%
\pgfsys@useobject{currentmarker}{}%
\end{pgfscope}%
\begin{pgfscope}%
\pgfsys@transformshift{1.510141in}{0.983199in}%
\pgfsys@useobject{currentmarker}{}%
\end{pgfscope}%
\begin{pgfscope}%
\pgfsys@transformshift{1.487431in}{0.981669in}%
\pgfsys@useobject{currentmarker}{}%
\end{pgfscope}%
\begin{pgfscope}%
\pgfsys@transformshift{1.464046in}{0.981687in}%
\pgfsys@useobject{currentmarker}{}%
\end{pgfscope}%
\begin{pgfscope}%
\pgfsys@transformshift{1.439548in}{0.975844in}%
\pgfsys@useobject{currentmarker}{}%
\end{pgfscope}%
\begin{pgfscope}%
\pgfsys@transformshift{1.425700in}{0.975509in}%
\pgfsys@useobject{currentmarker}{}%
\end{pgfscope}%
\begin{pgfscope}%
\pgfsys@transformshift{1.410310in}{0.976155in}%
\pgfsys@useobject{currentmarker}{}%
\end{pgfscope}%
\begin{pgfscope}%
\pgfsys@transformshift{1.402028in}{0.977944in}%
\pgfsys@useobject{currentmarker}{}%
\end{pgfscope}%
\begin{pgfscope}%
\pgfsys@transformshift{1.398341in}{0.980794in}%
\pgfsys@useobject{currentmarker}{}%
\end{pgfscope}%
\begin{pgfscope}%
\pgfsys@transformshift{1.396077in}{0.987512in}%
\pgfsys@useobject{currentmarker}{}%
\end{pgfscope}%
\begin{pgfscope}%
\pgfsys@transformshift{1.396250in}{0.995539in}%
\pgfsys@useobject{currentmarker}{}%
\end{pgfscope}%
\begin{pgfscope}%
\pgfsys@transformshift{1.395137in}{1.004853in}%
\pgfsys@useobject{currentmarker}{}%
\end{pgfscope}%
\begin{pgfscope}%
\pgfsys@transformshift{1.396513in}{1.015699in}%
\pgfsys@useobject{currentmarker}{}%
\end{pgfscope}%
\begin{pgfscope}%
\pgfsys@transformshift{1.396476in}{1.027673in}%
\pgfsys@useobject{currentmarker}{}%
\end{pgfscope}%
\begin{pgfscope}%
\pgfsys@transformshift{1.396464in}{1.041085in}%
\pgfsys@useobject{currentmarker}{}%
\end{pgfscope}%
\begin{pgfscope}%
\pgfsys@transformshift{1.396840in}{1.048453in}%
\pgfsys@useobject{currentmarker}{}%
\end{pgfscope}%
\begin{pgfscope}%
\pgfsys@transformshift{1.396268in}{1.057544in}%
\pgfsys@useobject{currentmarker}{}%
\end{pgfscope}%
\begin{pgfscope}%
\pgfsys@transformshift{1.395411in}{1.067862in}%
\pgfsys@useobject{currentmarker}{}%
\end{pgfscope}%
\begin{pgfscope}%
\pgfsys@transformshift{1.393472in}{1.078675in}%
\pgfsys@useobject{currentmarker}{}%
\end{pgfscope}%
\begin{pgfscope}%
\pgfsys@transformshift{1.397328in}{1.092056in}%
\pgfsys@useobject{currentmarker}{}%
\end{pgfscope}%
\begin{pgfscope}%
\pgfsys@transformshift{1.396379in}{1.099655in}%
\pgfsys@useobject{currentmarker}{}%
\end{pgfscope}%
\begin{pgfscope}%
\pgfsys@transformshift{1.394977in}{1.107664in}%
\pgfsys@useobject{currentmarker}{}%
\end{pgfscope}%
\begin{pgfscope}%
\pgfsys@transformshift{1.398372in}{1.119273in}%
\pgfsys@useobject{currentmarker}{}%
\end{pgfscope}%
\begin{pgfscope}%
\pgfsys@transformshift{1.394757in}{1.132779in}%
\pgfsys@useobject{currentmarker}{}%
\end{pgfscope}%
\begin{pgfscope}%
\pgfsys@transformshift{1.393578in}{1.147447in}%
\pgfsys@useobject{currentmarker}{}%
\end{pgfscope}%
\begin{pgfscope}%
\pgfsys@transformshift{1.394229in}{1.167748in}%
\pgfsys@useobject{currentmarker}{}%
\end{pgfscope}%
\begin{pgfscope}%
\pgfsys@transformshift{1.397874in}{1.188326in}%
\pgfsys@useobject{currentmarker}{}%
\end{pgfscope}%
\begin{pgfscope}%
\pgfsys@transformshift{1.392326in}{1.209234in}%
\pgfsys@useobject{currentmarker}{}%
\end{pgfscope}%
\begin{pgfscope}%
\pgfsys@transformshift{1.392644in}{1.221127in}%
\pgfsys@useobject{currentmarker}{}%
\end{pgfscope}%
\begin{pgfscope}%
\pgfsys@transformshift{1.396073in}{1.233964in}%
\pgfsys@useobject{currentmarker}{}%
\end{pgfscope}%
\begin{pgfscope}%
\pgfsys@transformshift{1.394211in}{1.252246in}%
\pgfsys@useobject{currentmarker}{}%
\end{pgfscope}%
\begin{pgfscope}%
\pgfsys@transformshift{1.393650in}{1.271278in}%
\pgfsys@useobject{currentmarker}{}%
\end{pgfscope}%
\begin{pgfscope}%
\pgfsys@transformshift{1.393328in}{1.294803in}%
\pgfsys@useobject{currentmarker}{}%
\end{pgfscope}%
\begin{pgfscope}%
\pgfsys@transformshift{1.394428in}{1.307697in}%
\pgfsys@useobject{currentmarker}{}%
\end{pgfscope}%
\begin{pgfscope}%
\pgfsys@transformshift{1.389796in}{1.321852in}%
\pgfsys@useobject{currentmarker}{}%
\end{pgfscope}%
\begin{pgfscope}%
\pgfsys@transformshift{1.387420in}{1.339247in}%
\pgfsys@useobject{currentmarker}{}%
\end{pgfscope}%
\begin{pgfscope}%
\pgfsys@transformshift{1.391070in}{1.357685in}%
\pgfsys@useobject{currentmarker}{}%
\end{pgfscope}%
\begin{pgfscope}%
\pgfsys@transformshift{1.382763in}{1.379357in}%
\pgfsys@useobject{currentmarker}{}%
\end{pgfscope}%
\begin{pgfscope}%
\pgfsys@transformshift{1.380046in}{1.391830in}%
\pgfsys@useobject{currentmarker}{}%
\end{pgfscope}%
\begin{pgfscope}%
\pgfsys@transformshift{1.384208in}{1.408390in}%
\pgfsys@useobject{currentmarker}{}%
\end{pgfscope}%
\begin{pgfscope}%
\pgfsys@transformshift{1.380005in}{1.428540in}%
\pgfsys@useobject{currentmarker}{}%
\end{pgfscope}%
\begin{pgfscope}%
\pgfsys@transformshift{1.378766in}{1.449556in}%
\pgfsys@useobject{currentmarker}{}%
\end{pgfscope}%
\begin{pgfscope}%
\pgfsys@transformshift{1.380161in}{1.475397in}%
\pgfsys@useobject{currentmarker}{}%
\end{pgfscope}%
\begin{pgfscope}%
\pgfsys@transformshift{1.383007in}{1.501583in}%
\pgfsys@useobject{currentmarker}{}%
\end{pgfscope}%
\begin{pgfscope}%
\pgfsys@transformshift{1.375757in}{1.527666in}%
\pgfsys@useobject{currentmarker}{}%
\end{pgfscope}%
\begin{pgfscope}%
\pgfsys@transformshift{1.373804in}{1.542427in}%
\pgfsys@useobject{currentmarker}{}%
\end{pgfscope}%
\begin{pgfscope}%
\pgfsys@transformshift{1.378805in}{1.558238in}%
\pgfsys@useobject{currentmarker}{}%
\end{pgfscope}%
\begin{pgfscope}%
\pgfsys@transformshift{1.374190in}{1.579562in}%
\pgfsys@useobject{currentmarker}{}%
\end{pgfscope}%
\begin{pgfscope}%
\pgfsys@transformshift{1.375529in}{1.591486in}%
\pgfsys@useobject{currentmarker}{}%
\end{pgfscope}%
\begin{pgfscope}%
\pgfsys@transformshift{1.380043in}{1.607482in}%
\pgfsys@useobject{currentmarker}{}%
\end{pgfscope}%
\begin{pgfscope}%
\pgfsys@transformshift{1.379780in}{1.625471in}%
\pgfsys@useobject{currentmarker}{}%
\end{pgfscope}%
\begin{pgfscope}%
\pgfsys@transformshift{1.375280in}{1.643785in}%
\pgfsys@useobject{currentmarker}{}%
\end{pgfscope}%
\begin{pgfscope}%
\pgfsys@transformshift{1.375918in}{1.665784in}%
\pgfsys@useobject{currentmarker}{}%
\end{pgfscope}%
\begin{pgfscope}%
\pgfsys@transformshift{1.381574in}{1.688257in}%
\pgfsys@useobject{currentmarker}{}%
\end{pgfscope}%
\begin{pgfscope}%
\pgfsys@transformshift{1.376546in}{1.714166in}%
\pgfsys@useobject{currentmarker}{}%
\end{pgfscope}%
\begin{pgfscope}%
\pgfsys@transformshift{1.375745in}{1.728659in}%
\pgfsys@useobject{currentmarker}{}%
\end{pgfscope}%
\begin{pgfscope}%
\pgfsys@transformshift{1.380391in}{1.745881in}%
\pgfsys@useobject{currentmarker}{}%
\end{pgfscope}%
\begin{pgfscope}%
\pgfsys@transformshift{1.377328in}{1.766670in}%
\pgfsys@useobject{currentmarker}{}%
\end{pgfscope}%
\begin{pgfscope}%
\pgfsys@transformshift{1.375184in}{1.778026in}%
\pgfsys@useobject{currentmarker}{}%
\end{pgfscope}%
\begin{pgfscope}%
\pgfsys@transformshift{1.376712in}{1.793091in}%
\pgfsys@useobject{currentmarker}{}%
\end{pgfscope}%
\begin{pgfscope}%
\pgfsys@transformshift{1.377207in}{1.801404in}%
\pgfsys@useobject{currentmarker}{}%
\end{pgfscope}%
\begin{pgfscope}%
\pgfsys@transformshift{1.374124in}{1.812415in}%
\pgfsys@useobject{currentmarker}{}%
\end{pgfscope}%
\begin{pgfscope}%
\pgfsys@transformshift{1.372816in}{1.818566in}%
\pgfsys@useobject{currentmarker}{}%
\end{pgfscope}%
\begin{pgfscope}%
\pgfsys@transformshift{1.375147in}{1.828578in}%
\pgfsys@useobject{currentmarker}{}%
\end{pgfscope}%
\begin{pgfscope}%
\pgfsys@transformshift{1.374227in}{1.840085in}%
\pgfsys@useobject{currentmarker}{}%
\end{pgfscope}%
\begin{pgfscope}%
\pgfsys@transformshift{1.373136in}{1.846340in}%
\pgfsys@useobject{currentmarker}{}%
\end{pgfscope}%
\begin{pgfscope}%
\pgfsys@transformshift{1.372960in}{1.856062in}%
\pgfsys@useobject{currentmarker}{}%
\end{pgfscope}%
\begin{pgfscope}%
\pgfsys@transformshift{1.375508in}{1.865964in}%
\pgfsys@useobject{currentmarker}{}%
\end{pgfscope}%
\begin{pgfscope}%
\pgfsys@transformshift{1.372327in}{1.879295in}%
\pgfsys@useobject{currentmarker}{}%
\end{pgfscope}%
\begin{pgfscope}%
\pgfsys@transformshift{1.371587in}{1.886797in}%
\pgfsys@useobject{currentmarker}{}%
\end{pgfscope}%
\begin{pgfscope}%
\pgfsys@transformshift{1.374220in}{1.897015in}%
\pgfsys@useobject{currentmarker}{}%
\end{pgfscope}%
\begin{pgfscope}%
\pgfsys@transformshift{1.371277in}{1.911480in}%
\pgfsys@useobject{currentmarker}{}%
\end{pgfscope}%
\begin{pgfscope}%
\pgfsys@transformshift{1.368372in}{1.926447in}%
\pgfsys@useobject{currentmarker}{}%
\end{pgfscope}%
\begin{pgfscope}%
\pgfsys@transformshift{1.374884in}{1.946460in}%
\pgfsys@useobject{currentmarker}{}%
\end{pgfscope}%
\begin{pgfscope}%
\pgfsys@transformshift{1.372591in}{1.968555in}%
\pgfsys@useobject{currentmarker}{}%
\end{pgfscope}%
\begin{pgfscope}%
\pgfsys@transformshift{1.369257in}{1.980309in}%
\pgfsys@useobject{currentmarker}{}%
\end{pgfscope}%
\begin{pgfscope}%
\pgfsys@transformshift{1.369899in}{1.995982in}%
\pgfsys@useobject{currentmarker}{}%
\end{pgfscope}%
\begin{pgfscope}%
\pgfsys@transformshift{1.374934in}{2.013106in}%
\pgfsys@useobject{currentmarker}{}%
\end{pgfscope}%
\begin{pgfscope}%
\pgfsys@transformshift{1.369373in}{2.034824in}%
\pgfsys@useobject{currentmarker}{}%
\end{pgfscope}%
\begin{pgfscope}%
\pgfsys@transformshift{1.370474in}{2.047105in}%
\pgfsys@useobject{currentmarker}{}%
\end{pgfscope}%
\begin{pgfscope}%
\pgfsys@transformshift{1.374840in}{2.064321in}%
\pgfsys@useobject{currentmarker}{}%
\end{pgfscope}%
\begin{pgfscope}%
\pgfsys@transformshift{1.375718in}{2.083022in}%
\pgfsys@useobject{currentmarker}{}%
\end{pgfscope}%
\begin{pgfscope}%
\pgfsys@transformshift{1.371004in}{2.101679in}%
\pgfsys@useobject{currentmarker}{}%
\end{pgfscope}%
\begin{pgfscope}%
\pgfsys@transformshift{1.370064in}{2.122357in}%
\pgfsys@useobject{currentmarker}{}%
\end{pgfscope}%
\begin{pgfscope}%
\pgfsys@transformshift{1.373719in}{2.144057in}%
\pgfsys@useobject{currentmarker}{}%
\end{pgfscope}%
\begin{pgfscope}%
\pgfsys@transformshift{1.368271in}{2.170929in}%
\pgfsys@useobject{currentmarker}{}%
\end{pgfscope}%
\begin{pgfscope}%
\pgfsys@transformshift{1.367734in}{2.186000in}%
\pgfsys@useobject{currentmarker}{}%
\end{pgfscope}%
\begin{pgfscope}%
\pgfsys@transformshift{1.363396in}{2.203532in}%
\pgfsys@useobject{currentmarker}{}%
\end{pgfscope}%
\begin{pgfscope}%
\pgfsys@transformshift{1.366217in}{2.213056in}%
\pgfsys@useobject{currentmarker}{}%
\end{pgfscope}%
\begin{pgfscope}%
\pgfsys@transformshift{1.362780in}{2.228093in}%
\pgfsys@useobject{currentmarker}{}%
\end{pgfscope}%
\begin{pgfscope}%
\pgfsys@transformshift{1.362827in}{2.236576in}%
\pgfsys@useobject{currentmarker}{}%
\end{pgfscope}%
\begin{pgfscope}%
\pgfsys@transformshift{1.365293in}{2.249551in}%
\pgfsys@useobject{currentmarker}{}%
\end{pgfscope}%
\begin{pgfscope}%
\pgfsys@transformshift{1.365416in}{2.256814in}%
\pgfsys@useobject{currentmarker}{}%
\end{pgfscope}%
\begin{pgfscope}%
\pgfsys@transformshift{1.362957in}{2.265442in}%
\pgfsys@useobject{currentmarker}{}%
\end{pgfscope}%
\begin{pgfscope}%
\pgfsys@transformshift{1.363239in}{2.274874in}%
\pgfsys@useobject{currentmarker}{}%
\end{pgfscope}%
\begin{pgfscope}%
\pgfsys@transformshift{1.366751in}{2.285839in}%
\pgfsys@useobject{currentmarker}{}%
\end{pgfscope}%
\begin{pgfscope}%
\pgfsys@transformshift{1.365038in}{2.300051in}%
\pgfsys@useobject{currentmarker}{}%
\end{pgfscope}%
\begin{pgfscope}%
\pgfsys@transformshift{1.360832in}{2.315061in}%
\pgfsys@useobject{currentmarker}{}%
\end{pgfscope}%
\begin{pgfscope}%
\pgfsys@transformshift{1.362131in}{2.333017in}%
\pgfsys@useobject{currentmarker}{}%
\end{pgfscope}%
\begin{pgfscope}%
\pgfsys@transformshift{1.365608in}{2.351284in}%
\pgfsys@useobject{currentmarker}{}%
\end{pgfscope}%
\begin{pgfscope}%
\pgfsys@transformshift{1.360549in}{2.372266in}%
\pgfsys@useobject{currentmarker}{}%
\end{pgfscope}%
\begin{pgfscope}%
\pgfsys@transformshift{1.360570in}{2.384136in}%
\pgfsys@useobject{currentmarker}{}%
\end{pgfscope}%
\begin{pgfscope}%
\pgfsys@transformshift{1.362989in}{2.398576in}%
\pgfsys@useobject{currentmarker}{}%
\end{pgfscope}%
\begin{pgfscope}%
\pgfsys@transformshift{1.360868in}{2.416192in}%
\pgfsys@useobject{currentmarker}{}%
\end{pgfscope}%
\begin{pgfscope}%
\pgfsys@transformshift{1.356419in}{2.433979in}%
\pgfsys@useobject{currentmarker}{}%
\end{pgfscope}%
\begin{pgfscope}%
\pgfsys@transformshift{1.356195in}{2.455486in}%
\pgfsys@useobject{currentmarker}{}%
\end{pgfscope}%
\begin{pgfscope}%
\pgfsys@transformshift{1.357847in}{2.467199in}%
\pgfsys@useobject{currentmarker}{}%
\end{pgfscope}%
\begin{pgfscope}%
\pgfsys@transformshift{1.352662in}{2.483072in}%
\pgfsys@useobject{currentmarker}{}%
\end{pgfscope}%
\begin{pgfscope}%
\pgfsys@transformshift{1.352100in}{2.492239in}%
\pgfsys@useobject{currentmarker}{}%
\end{pgfscope}%
\begin{pgfscope}%
\pgfsys@transformshift{1.354940in}{2.505303in}%
\pgfsys@useobject{currentmarker}{}%
\end{pgfscope}%
\begin{pgfscope}%
\pgfsys@transformshift{1.353838in}{2.512573in}%
\pgfsys@useobject{currentmarker}{}%
\end{pgfscope}%
\begin{pgfscope}%
\pgfsys@transformshift{1.353096in}{2.516549in}%
\pgfsys@useobject{currentmarker}{}%
\end{pgfscope}%
\begin{pgfscope}%
\pgfsys@transformshift{1.351991in}{2.521463in}%
\pgfsys@useobject{currentmarker}{}%
\end{pgfscope}%
\begin{pgfscope}%
\pgfsys@transformshift{1.351468in}{2.530382in}%
\pgfsys@useobject{currentmarker}{}%
\end{pgfscope}%
\begin{pgfscope}%
\pgfsys@transformshift{1.352686in}{2.539723in}%
\pgfsys@useobject{currentmarker}{}%
\end{pgfscope}%
\begin{pgfscope}%
\pgfsys@transformshift{1.348897in}{2.552016in}%
\pgfsys@useobject{currentmarker}{}%
\end{pgfscope}%
\begin{pgfscope}%
\pgfsys@transformshift{1.348823in}{2.559091in}%
\pgfsys@useobject{currentmarker}{}%
\end{pgfscope}%
\begin{pgfscope}%
\pgfsys@transformshift{1.351197in}{2.568104in}%
\pgfsys@useobject{currentmarker}{}%
\end{pgfscope}%
\begin{pgfscope}%
\pgfsys@transformshift{1.349014in}{2.580602in}%
\pgfsys@useobject{currentmarker}{}%
\end{pgfscope}%
\begin{pgfscope}%
\pgfsys@transformshift{1.347389in}{2.587388in}%
\pgfsys@useobject{currentmarker}{}%
\end{pgfscope}%
\begin{pgfscope}%
\pgfsys@transformshift{1.347882in}{2.597905in}%
\pgfsys@useobject{currentmarker}{}%
\end{pgfscope}%
\begin{pgfscope}%
\pgfsys@transformshift{1.349257in}{2.609005in}%
\pgfsys@useobject{currentmarker}{}%
\end{pgfscope}%
\begin{pgfscope}%
\pgfsys@transformshift{1.345054in}{2.623272in}%
\pgfsys@useobject{currentmarker}{}%
\end{pgfscope}%
\begin{pgfscope}%
\pgfsys@transformshift{1.345000in}{2.631451in}%
\pgfsys@useobject{currentmarker}{}%
\end{pgfscope}%
\begin{pgfscope}%
\pgfsys@transformshift{1.347961in}{2.643143in}%
\pgfsys@useobject{currentmarker}{}%
\end{pgfscope}%
\begin{pgfscope}%
\pgfsys@transformshift{1.345828in}{2.655702in}%
\pgfsys@useobject{currentmarker}{}%
\end{pgfscope}%
\begin{pgfscope}%
\pgfsys@transformshift{1.342499in}{2.668646in}%
\pgfsys@useobject{currentmarker}{}%
\end{pgfscope}%
\begin{pgfscope}%
\pgfsys@transformshift{1.341404in}{2.683713in}%
\pgfsys@useobject{currentmarker}{}%
\end{pgfscope}%
\begin{pgfscope}%
\pgfsys@transformshift{1.342752in}{2.691911in}%
\pgfsys@useobject{currentmarker}{}%
\end{pgfscope}%
\begin{pgfscope}%
\pgfsys@transformshift{1.338473in}{2.703886in}%
\pgfsys@useobject{currentmarker}{}%
\end{pgfscope}%
\begin{pgfscope}%
\pgfsys@transformshift{1.337897in}{2.710856in}%
\pgfsys@useobject{currentmarker}{}%
\end{pgfscope}%
\begin{pgfscope}%
\pgfsys@transformshift{1.340732in}{2.721140in}%
\pgfsys@useobject{currentmarker}{}%
\end{pgfscope}%
\begin{pgfscope}%
\pgfsys@transformshift{1.339970in}{2.726958in}%
\pgfsys@useobject{currentmarker}{}%
\end{pgfscope}%
\begin{pgfscope}%
\pgfsys@transformshift{1.339116in}{2.730069in}%
\pgfsys@useobject{currentmarker}{}%
\end{pgfscope}%
\begin{pgfscope}%
\pgfsys@transformshift{1.338927in}{2.735778in}%
\pgfsys@useobject{currentmarker}{}%
\end{pgfscope}%
\begin{pgfscope}%
\pgfsys@transformshift{1.340098in}{2.741953in}%
\pgfsys@useobject{currentmarker}{}%
\end{pgfscope}%
\begin{pgfscope}%
\pgfsys@transformshift{1.336688in}{2.752359in}%
\pgfsys@useobject{currentmarker}{}%
\end{pgfscope}%
\begin{pgfscope}%
\pgfsys@transformshift{1.335820in}{2.763733in}%
\pgfsys@useobject{currentmarker}{}%
\end{pgfscope}%
\begin{pgfscope}%
\pgfsys@transformshift{1.339168in}{2.778984in}%
\pgfsys@useobject{currentmarker}{}%
\end{pgfscope}%
\begin{pgfscope}%
\pgfsys@transformshift{1.338395in}{2.787538in}%
\pgfsys@useobject{currentmarker}{}%
\end{pgfscope}%
\begin{pgfscope}%
\pgfsys@transformshift{1.337172in}{2.792100in}%
\pgfsys@useobject{currentmarker}{}%
\end{pgfscope}%
\begin{pgfscope}%
\pgfsys@transformshift{1.336212in}{2.798761in}%
\pgfsys@useobject{currentmarker}{}%
\end{pgfscope}%
\begin{pgfscope}%
\pgfsys@transformshift{1.338348in}{2.807704in}%
\pgfsys@useobject{currentmarker}{}%
\end{pgfscope}%
\begin{pgfscope}%
\pgfsys@transformshift{1.335694in}{2.820464in}%
\pgfsys@useobject{currentmarker}{}%
\end{pgfscope}%
\begin{pgfscope}%
\pgfsys@transformshift{1.334179in}{2.827471in}%
\pgfsys@useobject{currentmarker}{}%
\end{pgfscope}%
\begin{pgfscope}%
\pgfsys@transformshift{1.333998in}{2.838511in}%
\pgfsys@useobject{currentmarker}{}%
\end{pgfscope}%
\begin{pgfscope}%
\pgfsys@transformshift{1.335174in}{2.850609in}%
\pgfsys@useobject{currentmarker}{}%
\end{pgfscope}%
\begin{pgfscope}%
\pgfsys@transformshift{1.330937in}{2.866142in}%
\pgfsys@useobject{currentmarker}{}%
\end{pgfscope}%
\begin{pgfscope}%
\pgfsys@transformshift{1.330687in}{2.874993in}%
\pgfsys@useobject{currentmarker}{}%
\end{pgfscope}%
\begin{pgfscope}%
\pgfsys@transformshift{1.331530in}{2.887611in}%
\pgfsys@useobject{currentmarker}{}%
\end{pgfscope}%
\begin{pgfscope}%
\pgfsys@transformshift{1.332845in}{2.900809in}%
\pgfsys@useobject{currentmarker}{}%
\end{pgfscope}%
\begin{pgfscope}%
\pgfsys@transformshift{1.330870in}{2.915200in}%
\pgfsys@useobject{currentmarker}{}%
\end{pgfscope}%
\begin{pgfscope}%
\pgfsys@transformshift{1.331029in}{2.923187in}%
\pgfsys@useobject{currentmarker}{}%
\end{pgfscope}%
\begin{pgfscope}%
\pgfsys@transformshift{1.330657in}{2.927566in}%
\pgfsys@useobject{currentmarker}{}%
\end{pgfscope}%
\begin{pgfscope}%
\pgfsys@transformshift{1.330813in}{2.929977in}%
\pgfsys@useobject{currentmarker}{}%
\end{pgfscope}%
\begin{pgfscope}%
\pgfsys@transformshift{1.330883in}{2.932858in}%
\pgfsys@useobject{currentmarker}{}%
\end{pgfscope}%
\begin{pgfscope}%
\pgfsys@transformshift{1.332600in}{2.935794in}%
\pgfsys@useobject{currentmarker}{}%
\end{pgfscope}%
\begin{pgfscope}%
\pgfsys@transformshift{1.335664in}{2.938690in}%
\pgfsys@useobject{currentmarker}{}%
\end{pgfscope}%
\begin{pgfscope}%
\pgfsys@transformshift{1.340466in}{2.940277in}%
\pgfsys@useobject{currentmarker}{}%
\end{pgfscope}%
\begin{pgfscope}%
\pgfsys@transformshift{1.343128in}{2.941081in}%
\pgfsys@useobject{currentmarker}{}%
\end{pgfscope}%
\begin{pgfscope}%
\pgfsys@transformshift{1.346457in}{2.941564in}%
\pgfsys@useobject{currentmarker}{}%
\end{pgfscope}%
\begin{pgfscope}%
\pgfsys@transformshift{1.348258in}{2.941988in}%
\pgfsys@useobject{currentmarker}{}%
\end{pgfscope}%
\begin{pgfscope}%
\pgfsys@transformshift{1.350549in}{2.942306in}%
\pgfsys@useobject{currentmarker}{}%
\end{pgfscope}%
\begin{pgfscope}%
\pgfsys@transformshift{1.353891in}{2.942852in}%
\pgfsys@useobject{currentmarker}{}%
\end{pgfscope}%
\begin{pgfscope}%
\pgfsys@transformshift{1.358227in}{2.943150in}%
\pgfsys@useobject{currentmarker}{}%
\end{pgfscope}%
\begin{pgfscope}%
\pgfsys@transformshift{1.363309in}{2.943768in}%
\pgfsys@useobject{currentmarker}{}%
\end{pgfscope}%
\begin{pgfscope}%
\pgfsys@transformshift{1.368933in}{2.943881in}%
\pgfsys@useobject{currentmarker}{}%
\end{pgfscope}%
\begin{pgfscope}%
\pgfsys@transformshift{1.372016in}{2.944128in}%
\pgfsys@useobject{currentmarker}{}%
\end{pgfscope}%
\begin{pgfscope}%
\pgfsys@transformshift{1.373717in}{2.944189in}%
\pgfsys@useobject{currentmarker}{}%
\end{pgfscope}%
\begin{pgfscope}%
\pgfsys@transformshift{1.374652in}{2.944228in}%
\pgfsys@useobject{currentmarker}{}%
\end{pgfscope}%
\begin{pgfscope}%
\pgfsys@transformshift{1.375166in}{2.944245in}%
\pgfsys@useobject{currentmarker}{}%
\end{pgfscope}%
\begin{pgfscope}%
\pgfsys@transformshift{1.375449in}{2.944250in}%
\pgfsys@useobject{currentmarker}{}%
\end{pgfscope}%
\begin{pgfscope}%
\pgfsys@transformshift{1.375605in}{2.944244in}%
\pgfsys@useobject{currentmarker}{}%
\end{pgfscope}%
\begin{pgfscope}%
\pgfsys@transformshift{1.375689in}{2.944256in}%
\pgfsys@useobject{currentmarker}{}%
\end{pgfscope}%
\begin{pgfscope}%
\pgfsys@transformshift{1.375736in}{2.944259in}%
\pgfsys@useobject{currentmarker}{}%
\end{pgfscope}%
\begin{pgfscope}%
\pgfsys@transformshift{1.379217in}{2.944309in}%
\pgfsys@useobject{currentmarker}{}%
\end{pgfscope}%
\begin{pgfscope}%
\pgfsys@transformshift{1.383622in}{2.944291in}%
\pgfsys@useobject{currentmarker}{}%
\end{pgfscope}%
\begin{pgfscope}%
\pgfsys@transformshift{1.388701in}{2.944824in}%
\pgfsys@useobject{currentmarker}{}%
\end{pgfscope}%
\begin{pgfscope}%
\pgfsys@transformshift{1.395110in}{2.944546in}%
\pgfsys@useobject{currentmarker}{}%
\end{pgfscope}%
\begin{pgfscope}%
\pgfsys@transformshift{1.406107in}{2.944872in}%
\pgfsys@useobject{currentmarker}{}%
\end{pgfscope}%
\begin{pgfscope}%
\pgfsys@transformshift{1.417997in}{2.946275in}%
\pgfsys@useobject{currentmarker}{}%
\end{pgfscope}%
\begin{pgfscope}%
\pgfsys@transformshift{1.430636in}{2.946428in}%
\pgfsys@useobject{currentmarker}{}%
\end{pgfscope}%
\begin{pgfscope}%
\pgfsys@transformshift{1.445901in}{2.946659in}%
\pgfsys@useobject{currentmarker}{}%
\end{pgfscope}%
\begin{pgfscope}%
\pgfsys@transformshift{1.463105in}{2.946756in}%
\pgfsys@useobject{currentmarker}{}%
\end{pgfscope}%
\begin{pgfscope}%
\pgfsys@transformshift{1.481947in}{2.947436in}%
\pgfsys@useobject{currentmarker}{}%
\end{pgfscope}%
\begin{pgfscope}%
\pgfsys@transformshift{1.501735in}{2.947722in}%
\pgfsys@useobject{currentmarker}{}%
\end{pgfscope}%
\begin{pgfscope}%
\pgfsys@transformshift{1.524083in}{2.947372in}%
\pgfsys@useobject{currentmarker}{}%
\end{pgfscope}%
\begin{pgfscope}%
\pgfsys@transformshift{1.547282in}{2.946385in}%
\pgfsys@useobject{currentmarker}{}%
\end{pgfscope}%
\begin{pgfscope}%
\pgfsys@transformshift{1.570985in}{2.949454in}%
\pgfsys@useobject{currentmarker}{}%
\end{pgfscope}%
\begin{pgfscope}%
\pgfsys@transformshift{1.595554in}{2.947450in}%
\pgfsys@useobject{currentmarker}{}%
\end{pgfscope}%
\begin{pgfscope}%
\pgfsys@transformshift{1.621428in}{2.945990in}%
\pgfsys@useobject{currentmarker}{}%
\end{pgfscope}%
\begin{pgfscope}%
\pgfsys@transformshift{1.649567in}{2.944832in}%
\pgfsys@useobject{currentmarker}{}%
\end{pgfscope}%
\begin{pgfscope}%
\pgfsys@transformshift{1.665040in}{2.945548in}%
\pgfsys@useobject{currentmarker}{}%
\end{pgfscope}%
\begin{pgfscope}%
\pgfsys@transformshift{1.681614in}{2.945931in}%
\pgfsys@useobject{currentmarker}{}%
\end{pgfscope}%
\begin{pgfscope}%
\pgfsys@transformshift{1.698678in}{2.946858in}%
\pgfsys@useobject{currentmarker}{}%
\end{pgfscope}%
\begin{pgfscope}%
\pgfsys@transformshift{1.708074in}{2.947102in}%
\pgfsys@useobject{currentmarker}{}%
\end{pgfscope}%
\begin{pgfscope}%
\pgfsys@transformshift{1.718968in}{2.947561in}%
\pgfsys@useobject{currentmarker}{}%
\end{pgfscope}%
\begin{pgfscope}%
\pgfsys@transformshift{1.730698in}{2.949477in}%
\pgfsys@useobject{currentmarker}{}%
\end{pgfscope}%
\begin{pgfscope}%
\pgfsys@transformshift{1.743009in}{2.950442in}%
\pgfsys@useobject{currentmarker}{}%
\end{pgfscope}%
\begin{pgfscope}%
\pgfsys@transformshift{1.755876in}{2.950988in}%
\pgfsys@useobject{currentmarker}{}%
\end{pgfscope}%
\begin{pgfscope}%
\pgfsys@transformshift{1.769722in}{2.953383in}%
\pgfsys@useobject{currentmarker}{}%
\end{pgfscope}%
\begin{pgfscope}%
\pgfsys@transformshift{1.777435in}{2.953869in}%
\pgfsys@useobject{currentmarker}{}%
\end{pgfscope}%
\begin{pgfscope}%
\pgfsys@transformshift{1.785780in}{2.954550in}%
\pgfsys@useobject{currentmarker}{}%
\end{pgfscope}%
\begin{pgfscope}%
\pgfsys@transformshift{1.790385in}{2.954551in}%
\pgfsys@useobject{currentmarker}{}%
\end{pgfscope}%
\begin{pgfscope}%
\pgfsys@transformshift{1.792891in}{2.954915in}%
\pgfsys@useobject{currentmarker}{}%
\end{pgfscope}%
\begin{pgfscope}%
\pgfsys@transformshift{1.795917in}{2.955129in}%
\pgfsys@useobject{currentmarker}{}%
\end{pgfscope}%
\begin{pgfscope}%
\pgfsys@transformshift{1.797578in}{2.955298in}%
\pgfsys@useobject{currentmarker}{}%
\end{pgfscope}%
\begin{pgfscope}%
\pgfsys@transformshift{1.798495in}{2.955322in}%
\pgfsys@useobject{currentmarker}{}%
\end{pgfscope}%
\begin{pgfscope}%
\pgfsys@transformshift{1.798998in}{2.955366in}%
\pgfsys@useobject{currentmarker}{}%
\end{pgfscope}%
\begin{pgfscope}%
\pgfsys@transformshift{1.800259in}{2.955499in}%
\pgfsys@useobject{currentmarker}{}%
\end{pgfscope}%
\begin{pgfscope}%
\pgfsys@transformshift{1.802176in}{2.955588in}%
\pgfsys@useobject{currentmarker}{}%
\end{pgfscope}%
\begin{pgfscope}%
\pgfsys@transformshift{1.803231in}{2.955577in}%
\pgfsys@useobject{currentmarker}{}%
\end{pgfscope}%
\begin{pgfscope}%
\pgfsys@transformshift{1.805205in}{2.955685in}%
\pgfsys@useobject{currentmarker}{}%
\end{pgfscope}%
\begin{pgfscope}%
\pgfsys@transformshift{1.809557in}{2.956951in}%
\pgfsys@useobject{currentmarker}{}%
\end{pgfscope}%
\begin{pgfscope}%
\pgfsys@transformshift{1.815556in}{2.957235in}%
\pgfsys@useobject{currentmarker}{}%
\end{pgfscope}%
\begin{pgfscope}%
\pgfsys@transformshift{1.824272in}{2.957802in}%
\pgfsys@useobject{currentmarker}{}%
\end{pgfscope}%
\begin{pgfscope}%
\pgfsys@transformshift{1.834152in}{2.958007in}%
\pgfsys@useobject{currentmarker}{}%
\end{pgfscope}%
\begin{pgfscope}%
\pgfsys@transformshift{1.845242in}{2.960034in}%
\pgfsys@useobject{currentmarker}{}%
\end{pgfscope}%
\begin{pgfscope}%
\pgfsys@transformshift{1.851443in}{2.960078in}%
\pgfsys@useobject{currentmarker}{}%
\end{pgfscope}%
\begin{pgfscope}%
\pgfsys@transformshift{1.858236in}{2.960501in}%
\pgfsys@useobject{currentmarker}{}%
\end{pgfscope}%
\begin{pgfscope}%
\pgfsys@transformshift{1.866418in}{2.960648in}%
\pgfsys@useobject{currentmarker}{}%
\end{pgfscope}%
\begin{pgfscope}%
\pgfsys@transformshift{1.870871in}{2.961302in}%
\pgfsys@useobject{currentmarker}{}%
\end{pgfscope}%
\begin{pgfscope}%
\pgfsys@transformshift{1.873346in}{2.961318in}%
\pgfsys@useobject{currentmarker}{}%
\end{pgfscope}%
\begin{pgfscope}%
\pgfsys@transformshift{1.876975in}{2.961670in}%
\pgfsys@useobject{currentmarker}{}%
\end{pgfscope}%
\begin{pgfscope}%
\pgfsys@transformshift{1.881896in}{2.961507in}%
\pgfsys@useobject{currentmarker}{}%
\end{pgfscope}%
\begin{pgfscope}%
\pgfsys@transformshift{1.884555in}{2.962017in}%
\pgfsys@useobject{currentmarker}{}%
\end{pgfscope}%
\begin{pgfscope}%
\pgfsys@transformshift{1.888137in}{2.962114in}%
\pgfsys@useobject{currentmarker}{}%
\end{pgfscope}%
\begin{pgfscope}%
\pgfsys@transformshift{1.890100in}{2.962284in}%
\pgfsys@useobject{currentmarker}{}%
\end{pgfscope}%
\begin{pgfscope}%
\pgfsys@transformshift{1.891180in}{2.962370in}%
\pgfsys@useobject{currentmarker}{}%
\end{pgfscope}%
\begin{pgfscope}%
\pgfsys@transformshift{1.894363in}{2.962600in}%
\pgfsys@useobject{currentmarker}{}%
\end{pgfscope}%
\begin{pgfscope}%
\pgfsys@transformshift{1.900459in}{2.962920in}%
\pgfsys@useobject{currentmarker}{}%
\end{pgfscope}%
\begin{pgfscope}%
\pgfsys@transformshift{1.909322in}{2.964424in}%
\pgfsys@useobject{currentmarker}{}%
\end{pgfscope}%
\begin{pgfscope}%
\pgfsys@transformshift{1.920784in}{2.964527in}%
\pgfsys@useobject{currentmarker}{}%
\end{pgfscope}%
\begin{pgfscope}%
\pgfsys@transformshift{1.933830in}{2.965685in}%
\pgfsys@useobject{currentmarker}{}%
\end{pgfscope}%
\begin{pgfscope}%
\pgfsys@transformshift{1.941033in}{2.965791in}%
\pgfsys@useobject{currentmarker}{}%
\end{pgfscope}%
\begin{pgfscope}%
\pgfsys@transformshift{1.948912in}{2.966369in}%
\pgfsys@useobject{currentmarker}{}%
\end{pgfscope}%
\begin{pgfscope}%
\pgfsys@transformshift{1.958323in}{2.967230in}%
\pgfsys@useobject{currentmarker}{}%
\end{pgfscope}%
\begin{pgfscope}%
\pgfsys@transformshift{1.969226in}{2.967701in}%
\pgfsys@useobject{currentmarker}{}%
\end{pgfscope}%
\begin{pgfscope}%
\pgfsys@transformshift{1.981301in}{2.968151in}%
\pgfsys@useobject{currentmarker}{}%
\end{pgfscope}%
\begin{pgfscope}%
\pgfsys@transformshift{1.995566in}{2.968849in}%
\pgfsys@useobject{currentmarker}{}%
\end{pgfscope}%
\begin{pgfscope}%
\pgfsys@transformshift{2.003386in}{2.969588in}%
\pgfsys@useobject{currentmarker}{}%
\end{pgfscope}%
\begin{pgfscope}%
\pgfsys@transformshift{2.012824in}{2.971089in}%
\pgfsys@useobject{currentmarker}{}%
\end{pgfscope}%
\begin{pgfscope}%
\pgfsys@transformshift{2.018056in}{2.971597in}%
\pgfsys@useobject{currentmarker}{}%
\end{pgfscope}%
\begin{pgfscope}%
\pgfsys@transformshift{2.020943in}{2.971750in}%
\pgfsys@useobject{currentmarker}{}%
\end{pgfscope}%
\begin{pgfscope}%
\pgfsys@transformshift{2.022515in}{2.971990in}%
\pgfsys@useobject{currentmarker}{}%
\end{pgfscope}%
\begin{pgfscope}%
\pgfsys@transformshift{2.023380in}{2.972116in}%
\pgfsys@useobject{currentmarker}{}%
\end{pgfscope}%
\begin{pgfscope}%
\pgfsys@transformshift{2.026700in}{2.972415in}%
\pgfsys@useobject{currentmarker}{}%
\end{pgfscope}%
\begin{pgfscope}%
\pgfsys@transformshift{2.031720in}{2.973024in}%
\pgfsys@useobject{currentmarker}{}%
\end{pgfscope}%
\begin{pgfscope}%
\pgfsys@transformshift{2.037449in}{2.973670in}%
\pgfsys@useobject{currentmarker}{}%
\end{pgfscope}%
\begin{pgfscope}%
\pgfsys@transformshift{2.044705in}{2.974256in}%
\pgfsys@useobject{currentmarker}{}%
\end{pgfscope}%
\begin{pgfscope}%
\pgfsys@transformshift{2.052736in}{2.975414in}%
\pgfsys@useobject{currentmarker}{}%
\end{pgfscope}%
\begin{pgfscope}%
\pgfsys@transformshift{2.065725in}{2.976729in}%
\pgfsys@useobject{currentmarker}{}%
\end{pgfscope}%
\begin{pgfscope}%
\pgfsys@transformshift{2.081817in}{2.979834in}%
\pgfsys@useobject{currentmarker}{}%
\end{pgfscope}%
\begin{pgfscope}%
\pgfsys@transformshift{2.099570in}{2.981675in}%
\pgfsys@useobject{currentmarker}{}%
\end{pgfscope}%
\begin{pgfscope}%
\pgfsys@transformshift{2.118778in}{2.983798in}%
\pgfsys@useobject{currentmarker}{}%
\end{pgfscope}%
\begin{pgfscope}%
\pgfsys@transformshift{2.139330in}{2.985211in}%
\pgfsys@useobject{currentmarker}{}%
\end{pgfscope}%
\begin{pgfscope}%
\pgfsys@transformshift{2.162573in}{2.986661in}%
\pgfsys@useobject{currentmarker}{}%
\end{pgfscope}%
\begin{pgfscope}%
\pgfsys@transformshift{2.185999in}{2.990766in}%
\pgfsys@useobject{currentmarker}{}%
\end{pgfscope}%
\begin{pgfscope}%
\pgfsys@transformshift{2.198839in}{2.993262in}%
\pgfsys@useobject{currentmarker}{}%
\end{pgfscope}%
\begin{pgfscope}%
\pgfsys@transformshift{2.212966in}{2.994865in}%
\pgfsys@useobject{currentmarker}{}%
\end{pgfscope}%
\begin{pgfscope}%
\pgfsys@transformshift{2.228980in}{2.995613in}%
\pgfsys@useobject{currentmarker}{}%
\end{pgfscope}%
\begin{pgfscope}%
\pgfsys@transformshift{2.251235in}{3.002817in}%
\pgfsys@useobject{currentmarker}{}%
\end{pgfscope}%
\begin{pgfscope}%
\pgfsys@transformshift{2.275459in}{3.006150in}%
\pgfsys@useobject{currentmarker}{}%
\end{pgfscope}%
\begin{pgfscope}%
\pgfsys@transformshift{2.300322in}{3.010885in}%
\pgfsys@useobject{currentmarker}{}%
\end{pgfscope}%
\begin{pgfscope}%
\pgfsys@transformshift{2.326123in}{3.015234in}%
\pgfsys@useobject{currentmarker}{}%
\end{pgfscope}%
\begin{pgfscope}%
\pgfsys@transformshift{2.353117in}{3.018221in}%
\pgfsys@useobject{currentmarker}{}%
\end{pgfscope}%
\begin{pgfscope}%
\pgfsys@transformshift{2.382137in}{3.019987in}%
\pgfsys@useobject{currentmarker}{}%
\end{pgfscope}%
\begin{pgfscope}%
\pgfsys@transformshift{2.411940in}{3.020468in}%
\pgfsys@useobject{currentmarker}{}%
\end{pgfscope}%
\begin{pgfscope}%
\pgfsys@transformshift{2.443086in}{3.022527in}%
\pgfsys@useobject{currentmarker}{}%
\end{pgfscope}%
\begin{pgfscope}%
\pgfsys@transformshift{2.476257in}{3.023962in}%
\pgfsys@useobject{currentmarker}{}%
\end{pgfscope}%
\begin{pgfscope}%
\pgfsys@transformshift{2.512047in}{3.026376in}%
\pgfsys@useobject{currentmarker}{}%
\end{pgfscope}%
\begin{pgfscope}%
\pgfsys@transformshift{2.549715in}{3.027218in}%
\pgfsys@useobject{currentmarker}{}%
\end{pgfscope}%
\begin{pgfscope}%
\pgfsys@transformshift{2.587904in}{3.027921in}%
\pgfsys@useobject{currentmarker}{}%
\end{pgfscope}%
\begin{pgfscope}%
\pgfsys@transformshift{2.626832in}{3.029581in}%
\pgfsys@useobject{currentmarker}{}%
\end{pgfscope}%
\begin{pgfscope}%
\pgfsys@transformshift{2.667081in}{3.030899in}%
\pgfsys@useobject{currentmarker}{}%
\end{pgfscope}%
\begin{pgfscope}%
\pgfsys@transformshift{2.709128in}{3.033900in}%
\pgfsys@useobject{currentmarker}{}%
\end{pgfscope}%
\begin{pgfscope}%
\pgfsys@transformshift{2.752871in}{3.034430in}%
\pgfsys@useobject{currentmarker}{}%
\end{pgfscope}%
\begin{pgfscope}%
\pgfsys@transformshift{2.776911in}{3.035442in}%
\pgfsys@useobject{currentmarker}{}%
\end{pgfscope}%
\begin{pgfscope}%
\pgfsys@transformshift{2.803519in}{3.036755in}%
\pgfsys@useobject{currentmarker}{}%
\end{pgfscope}%
\begin{pgfscope}%
\pgfsys@transformshift{2.832021in}{3.037623in}%
\pgfsys@useobject{currentmarker}{}%
\end{pgfscope}%
\begin{pgfscope}%
\pgfsys@transformshift{2.861331in}{3.040731in}%
\pgfsys@useobject{currentmarker}{}%
\end{pgfscope}%
\begin{pgfscope}%
\pgfsys@transformshift{2.877537in}{3.041129in}%
\pgfsys@useobject{currentmarker}{}%
\end{pgfscope}%
\begin{pgfscope}%
\pgfsys@transformshift{2.895023in}{3.042024in}%
\pgfsys@useobject{currentmarker}{}%
\end{pgfscope}%
\begin{pgfscope}%
\pgfsys@transformshift{2.914574in}{3.042440in}%
\pgfsys@useobject{currentmarker}{}%
\end{pgfscope}%
\begin{pgfscope}%
\pgfsys@transformshift{2.925313in}{3.043038in}%
\pgfsys@useobject{currentmarker}{}%
\end{pgfscope}%
\begin{pgfscope}%
\pgfsys@transformshift{2.937625in}{3.043933in}%
\pgfsys@useobject{currentmarker}{}%
\end{pgfscope}%
\begin{pgfscope}%
\pgfsys@transformshift{2.952478in}{3.044898in}%
\pgfsys@useobject{currentmarker}{}%
\end{pgfscope}%
\begin{pgfscope}%
\pgfsys@transformshift{2.968456in}{3.044586in}%
\pgfsys@useobject{currentmarker}{}%
\end{pgfscope}%
\begin{pgfscope}%
\pgfsys@transformshift{2.985499in}{3.044181in}%
\pgfsys@useobject{currentmarker}{}%
\end{pgfscope}%
\begin{pgfscope}%
\pgfsys@transformshift{3.004017in}{3.047633in}%
\pgfsys@useobject{currentmarker}{}%
\end{pgfscope}%
\begin{pgfscope}%
\pgfsys@transformshift{3.014349in}{3.046868in}%
\pgfsys@useobject{currentmarker}{}%
\end{pgfscope}%
\begin{pgfscope}%
\pgfsys@transformshift{3.020044in}{3.047076in}%
\pgfsys@useobject{currentmarker}{}%
\end{pgfscope}%
\begin{pgfscope}%
\pgfsys@transformshift{3.026622in}{3.047137in}%
\pgfsys@useobject{currentmarker}{}%
\end{pgfscope}%
\begin{pgfscope}%
\pgfsys@transformshift{3.030236in}{3.047316in}%
\pgfsys@useobject{currentmarker}{}%
\end{pgfscope}%
\begin{pgfscope}%
\pgfsys@transformshift{3.035296in}{3.047211in}%
\pgfsys@useobject{currentmarker}{}%
\end{pgfscope}%
\begin{pgfscope}%
\pgfsys@transformshift{3.041891in}{3.047341in}%
\pgfsys@useobject{currentmarker}{}%
\end{pgfscope}%
\begin{pgfscope}%
\pgfsys@transformshift{3.050385in}{3.047720in}%
\pgfsys@useobject{currentmarker}{}%
\end{pgfscope}%
\begin{pgfscope}%
\pgfsys@transformshift{3.061933in}{3.048046in}%
\pgfsys@useobject{currentmarker}{}%
\end{pgfscope}%
\begin{pgfscope}%
\pgfsys@transformshift{3.074346in}{3.046976in}%
\pgfsys@useobject{currentmarker}{}%
\end{pgfscope}%
\begin{pgfscope}%
\pgfsys@transformshift{3.087657in}{3.049199in}%
\pgfsys@useobject{currentmarker}{}%
\end{pgfscope}%
\begin{pgfscope}%
\pgfsys@transformshift{3.102538in}{3.049222in}%
\pgfsys@useobject{currentmarker}{}%
\end{pgfscope}%
\begin{pgfscope}%
\pgfsys@transformshift{3.118369in}{3.049649in}%
\pgfsys@useobject{currentmarker}{}%
\end{pgfscope}%
\begin{pgfscope}%
\pgfsys@transformshift{3.134976in}{3.048813in}%
\pgfsys@useobject{currentmarker}{}%
\end{pgfscope}%
\begin{pgfscope}%
\pgfsys@transformshift{3.155933in}{3.048803in}%
\pgfsys@useobject{currentmarker}{}%
\end{pgfscope}%
\begin{pgfscope}%
\pgfsys@transformshift{3.179976in}{3.047084in}%
\pgfsys@useobject{currentmarker}{}%
\end{pgfscope}%
\begin{pgfscope}%
\pgfsys@transformshift{3.205872in}{3.048924in}%
\pgfsys@useobject{currentmarker}{}%
\end{pgfscope}%
\begin{pgfscope}%
\pgfsys@transformshift{3.220094in}{3.047645in}%
\pgfsys@useobject{currentmarker}{}%
\end{pgfscope}%
\begin{pgfscope}%
\pgfsys@transformshift{3.235148in}{3.047197in}%
\pgfsys@useobject{currentmarker}{}%
\end{pgfscope}%
\begin{pgfscope}%
\pgfsys@transformshift{3.252301in}{3.047527in}%
\pgfsys@useobject{currentmarker}{}%
\end{pgfscope}%
\begin{pgfscope}%
\pgfsys@transformshift{3.270466in}{3.047537in}%
\pgfsys@useobject{currentmarker}{}%
\end{pgfscope}%
\begin{pgfscope}%
\pgfsys@transformshift{3.291035in}{3.048021in}%
\pgfsys@useobject{currentmarker}{}%
\end{pgfscope}%
\begin{pgfscope}%
\pgfsys@transformshift{3.313576in}{3.046184in}%
\pgfsys@useobject{currentmarker}{}%
\end{pgfscope}%
\begin{pgfscope}%
\pgfsys@transformshift{3.338280in}{3.045567in}%
\pgfsys@useobject{currentmarker}{}%
\end{pgfscope}%
\begin{pgfscope}%
\pgfsys@transformshift{3.366096in}{3.045387in}%
\pgfsys@useobject{currentmarker}{}%
\end{pgfscope}%
\begin{pgfscope}%
\pgfsys@transformshift{3.396462in}{3.045240in}%
\pgfsys@useobject{currentmarker}{}%
\end{pgfscope}%
\begin{pgfscope}%
\pgfsys@transformshift{3.428194in}{3.045613in}%
\pgfsys@useobject{currentmarker}{}%
\end{pgfscope}%
\begin{pgfscope}%
\pgfsys@transformshift{3.461051in}{3.045553in}%
\pgfsys@useobject{currentmarker}{}%
\end{pgfscope}%
\begin{pgfscope}%
\pgfsys@transformshift{3.495425in}{3.048343in}%
\pgfsys@useobject{currentmarker}{}%
\end{pgfscope}%
\begin{pgfscope}%
\pgfsys@transformshift{3.530538in}{3.050775in}%
\pgfsys@useobject{currentmarker}{}%
\end{pgfscope}%
\begin{pgfscope}%
\pgfsys@transformshift{3.549668in}{3.047816in}%
\pgfsys@useobject{currentmarker}{}%
\end{pgfscope}%
\begin{pgfscope}%
\pgfsys@transformshift{3.568008in}{3.038196in}%
\pgfsys@useobject{currentmarker}{}%
\end{pgfscope}%
\begin{pgfscope}%
\pgfsys@transformshift{3.576515in}{3.030622in}%
\pgfsys@useobject{currentmarker}{}%
\end{pgfscope}%
\begin{pgfscope}%
\pgfsys@transformshift{3.578949in}{3.024849in}%
\pgfsys@useobject{currentmarker}{}%
\end{pgfscope}%
\begin{pgfscope}%
\pgfsys@transformshift{3.579224in}{3.018082in}%
\pgfsys@useobject{currentmarker}{}%
\end{pgfscope}%
\begin{pgfscope}%
\pgfsys@transformshift{3.578248in}{3.010190in}%
\pgfsys@useobject{currentmarker}{}%
\end{pgfscope}%
\begin{pgfscope}%
\pgfsys@transformshift{3.578694in}{3.001296in}%
\pgfsys@useobject{currentmarker}{}%
\end{pgfscope}%
\begin{pgfscope}%
\pgfsys@transformshift{3.578131in}{2.991794in}%
\pgfsys@useobject{currentmarker}{}%
\end{pgfscope}%
\begin{pgfscope}%
\pgfsys@transformshift{3.577455in}{2.981641in}%
\pgfsys@useobject{currentmarker}{}%
\end{pgfscope}%
\begin{pgfscope}%
\pgfsys@transformshift{3.579760in}{2.970996in}%
\pgfsys@useobject{currentmarker}{}%
\end{pgfscope}%
\begin{pgfscope}%
\pgfsys@transformshift{3.580687in}{2.958472in}%
\pgfsys@useobject{currentmarker}{}%
\end{pgfscope}%
\begin{pgfscope}%
\pgfsys@transformshift{3.578833in}{2.945258in}%
\pgfsys@useobject{currentmarker}{}%
\end{pgfscope}%
\begin{pgfscope}%
\pgfsys@transformshift{3.576402in}{2.938334in}%
\pgfsys@useobject{currentmarker}{}%
\end{pgfscope}%
\begin{pgfscope}%
\pgfsys@transformshift{3.576547in}{2.929109in}%
\pgfsys@useobject{currentmarker}{}%
\end{pgfscope}%
\begin{pgfscope}%
\pgfsys@transformshift{3.575428in}{2.924160in}%
\pgfsys@useobject{currentmarker}{}%
\end{pgfscope}%
\begin{pgfscope}%
\pgfsys@transformshift{3.573364in}{2.917237in}%
\pgfsys@useobject{currentmarker}{}%
\end{pgfscope}%
\begin{pgfscope}%
\pgfsys@transformshift{3.572133in}{2.908093in}%
\pgfsys@useobject{currentmarker}{}%
\end{pgfscope}%
\begin{pgfscope}%
\pgfsys@transformshift{3.572534in}{2.903035in}%
\pgfsys@useobject{currentmarker}{}%
\end{pgfscope}%
\begin{pgfscope}%
\pgfsys@transformshift{3.570881in}{2.895050in}%
\pgfsys@useobject{currentmarker}{}%
\end{pgfscope}%
\begin{pgfscope}%
\pgfsys@transformshift{3.571087in}{2.890570in}%
\pgfsys@useobject{currentmarker}{}%
\end{pgfscope}%
\begin{pgfscope}%
\pgfsys@transformshift{3.571945in}{2.885075in}%
\pgfsys@useobject{currentmarker}{}%
\end{pgfscope}%
\begin{pgfscope}%
\pgfsys@transformshift{3.571601in}{2.882035in}%
\pgfsys@useobject{currentmarker}{}%
\end{pgfscope}%
\begin{pgfscope}%
\pgfsys@transformshift{3.571360in}{2.880370in}%
\pgfsys@useobject{currentmarker}{}%
\end{pgfscope}%
\begin{pgfscope}%
\pgfsys@transformshift{3.571455in}{2.877427in}%
\pgfsys@useobject{currentmarker}{}%
\end{pgfscope}%
\begin{pgfscope}%
\pgfsys@transformshift{3.571714in}{2.873894in}%
\pgfsys@useobject{currentmarker}{}%
\end{pgfscope}%
\begin{pgfscope}%
\pgfsys@transformshift{3.571199in}{2.869490in}%
\pgfsys@useobject{currentmarker}{}%
\end{pgfscope}%
\begin{pgfscope}%
\pgfsys@transformshift{3.570523in}{2.863563in}%
\pgfsys@useobject{currentmarker}{}%
\end{pgfscope}%
\begin{pgfscope}%
\pgfsys@transformshift{3.570927in}{2.856989in}%
\pgfsys@useobject{currentmarker}{}%
\end{pgfscope}%
\begin{pgfscope}%
\pgfsys@transformshift{3.571450in}{2.853404in}%
\pgfsys@useobject{currentmarker}{}%
\end{pgfscope}%
\begin{pgfscope}%
\pgfsys@transformshift{3.570572in}{2.847938in}%
\pgfsys@useobject{currentmarker}{}%
\end{pgfscope}%
\begin{pgfscope}%
\pgfsys@transformshift{3.570427in}{2.844897in}%
\pgfsys@useobject{currentmarker}{}%
\end{pgfscope}%
\begin{pgfscope}%
\pgfsys@transformshift{3.569962in}{2.839787in}%
\pgfsys@useobject{currentmarker}{}%
\end{pgfscope}%
\begin{pgfscope}%
\pgfsys@transformshift{3.569999in}{2.836965in}%
\pgfsys@useobject{currentmarker}{}%
\end{pgfscope}%
\begin{pgfscope}%
\pgfsys@transformshift{3.569560in}{2.833082in}%
\pgfsys@useobject{currentmarker}{}%
\end{pgfscope}%
\begin{pgfscope}%
\pgfsys@transformshift{3.569550in}{2.828549in}%
\pgfsys@useobject{currentmarker}{}%
\end{pgfscope}%
\begin{pgfscope}%
\pgfsys@transformshift{3.569140in}{2.820933in}%
\pgfsys@useobject{currentmarker}{}%
\end{pgfscope}%
\begin{pgfscope}%
\pgfsys@transformshift{3.569388in}{2.816746in}%
\pgfsys@useobject{currentmarker}{}%
\end{pgfscope}%
\begin{pgfscope}%
\pgfsys@transformshift{3.568623in}{2.811814in}%
\pgfsys@useobject{currentmarker}{}%
\end{pgfscope}%
\begin{pgfscope}%
\pgfsys@transformshift{3.568402in}{2.809077in}%
\pgfsys@useobject{currentmarker}{}%
\end{pgfscope}%
\begin{pgfscope}%
\pgfsys@transformshift{3.568499in}{2.805012in}%
\pgfsys@useobject{currentmarker}{}%
\end{pgfscope}%
\begin{pgfscope}%
\pgfsys@transformshift{3.568284in}{2.802786in}%
\pgfsys@useobject{currentmarker}{}%
\end{pgfscope}%
\begin{pgfscope}%
\pgfsys@transformshift{3.567865in}{2.798808in}%
\pgfsys@useobject{currentmarker}{}%
\end{pgfscope}%
\begin{pgfscope}%
\pgfsys@transformshift{3.568195in}{2.794124in}%
\pgfsys@useobject{currentmarker}{}%
\end{pgfscope}%
\begin{pgfscope}%
\pgfsys@transformshift{3.569423in}{2.787967in}%
\pgfsys@useobject{currentmarker}{}%
\end{pgfscope}%
\begin{pgfscope}%
\pgfsys@transformshift{3.568626in}{2.779086in}%
\pgfsys@useobject{currentmarker}{}%
\end{pgfscope}%
\begin{pgfscope}%
\pgfsys@transformshift{3.567885in}{2.774238in}%
\pgfsys@useobject{currentmarker}{}%
\end{pgfscope}%
\begin{pgfscope}%
\pgfsys@transformshift{3.567626in}{2.767723in}%
\pgfsys@useobject{currentmarker}{}%
\end{pgfscope}%
\begin{pgfscope}%
\pgfsys@transformshift{3.567530in}{2.764138in}%
\pgfsys@useobject{currentmarker}{}%
\end{pgfscope}%
\begin{pgfscope}%
\pgfsys@transformshift{3.567290in}{2.759288in}%
\pgfsys@useobject{currentmarker}{}%
\end{pgfscope}%
\begin{pgfscope}%
\pgfsys@transformshift{3.567331in}{2.756617in}%
\pgfsys@useobject{currentmarker}{}%
\end{pgfscope}%
\begin{pgfscope}%
\pgfsys@transformshift{3.568066in}{2.751629in}%
\pgfsys@useobject{currentmarker}{}%
\end{pgfscope}%
\begin{pgfscope}%
\pgfsys@transformshift{3.567417in}{2.744415in}%
\pgfsys@useobject{currentmarker}{}%
\end{pgfscope}%
\begin{pgfscope}%
\pgfsys@transformshift{3.566919in}{2.740463in}%
\pgfsys@useobject{currentmarker}{}%
\end{pgfscope}%
\begin{pgfscope}%
\pgfsys@transformshift{3.566927in}{2.734843in}%
\pgfsys@useobject{currentmarker}{}%
\end{pgfscope}%
\begin{pgfscope}%
\pgfsys@transformshift{3.567322in}{2.731777in}%
\pgfsys@useobject{currentmarker}{}%
\end{pgfscope}%
\begin{pgfscope}%
\pgfsys@transformshift{3.566870in}{2.726047in}%
\pgfsys@useobject{currentmarker}{}%
\end{pgfscope}%
\begin{pgfscope}%
\pgfsys@transformshift{3.567182in}{2.719348in}%
\pgfsys@useobject{currentmarker}{}%
\end{pgfscope}%
\begin{pgfscope}%
\pgfsys@transformshift{3.568662in}{2.708663in}%
\pgfsys@useobject{currentmarker}{}%
\end{pgfscope}%
\begin{pgfscope}%
\pgfsys@transformshift{3.569507in}{2.702790in}%
\pgfsys@useobject{currentmarker}{}%
\end{pgfscope}%
\begin{pgfscope}%
\pgfsys@transformshift{3.568957in}{2.695412in}%
\pgfsys@useobject{currentmarker}{}%
\end{pgfscope}%
\begin{pgfscope}%
\pgfsys@transformshift{3.569848in}{2.687509in}%
\pgfsys@useobject{currentmarker}{}%
\end{pgfscope}%
\begin{pgfscope}%
\pgfsys@transformshift{3.571900in}{2.678434in}%
\pgfsys@useobject{currentmarker}{}%
\end{pgfscope}%
\begin{pgfscope}%
\pgfsys@transformshift{3.570850in}{2.665493in}%
\pgfsys@useobject{currentmarker}{}%
\end{pgfscope}%
\begin{pgfscope}%
\pgfsys@transformshift{3.570600in}{2.651234in}%
\pgfsys@useobject{currentmarker}{}%
\end{pgfscope}%
\begin{pgfscope}%
\pgfsys@transformshift{3.571577in}{2.636231in}%
\pgfsys@useobject{currentmarker}{}%
\end{pgfscope}%
\begin{pgfscope}%
\pgfsys@transformshift{3.574319in}{2.619591in}%
\pgfsys@useobject{currentmarker}{}%
\end{pgfscope}%
\begin{pgfscope}%
\pgfsys@transformshift{3.572513in}{2.598411in}%
\pgfsys@useobject{currentmarker}{}%
\end{pgfscope}%
\begin{pgfscope}%
\pgfsys@transformshift{3.573165in}{2.586738in}%
\pgfsys@useobject{currentmarker}{}%
\end{pgfscope}%
\begin{pgfscope}%
\pgfsys@transformshift{3.575581in}{2.573303in}%
\pgfsys@useobject{currentmarker}{}%
\end{pgfscope}%
\begin{pgfscope}%
\pgfsys@transformshift{3.576209in}{2.556615in}%
\pgfsys@useobject{currentmarker}{}%
\end{pgfscope}%
\begin{pgfscope}%
\pgfsys@transformshift{3.574505in}{2.539203in}%
\pgfsys@useobject{currentmarker}{}%
\end{pgfscope}%
\begin{pgfscope}%
\pgfsys@transformshift{3.574833in}{2.520080in}%
\pgfsys@useobject{currentmarker}{}%
\end{pgfscope}%
\begin{pgfscope}%
\pgfsys@transformshift{3.575541in}{2.509584in}%
\pgfsys@useobject{currentmarker}{}%
\end{pgfscope}%
\begin{pgfscope}%
\pgfsys@transformshift{3.574044in}{2.495320in}%
\pgfsys@useobject{currentmarker}{}%
\end{pgfscope}%
\begin{pgfscope}%
\pgfsys@transformshift{3.573884in}{2.487433in}%
\pgfsys@useobject{currentmarker}{}%
\end{pgfscope}%
\begin{pgfscope}%
\pgfsys@transformshift{3.575424in}{2.477800in}%
\pgfsys@useobject{currentmarker}{}%
\end{pgfscope}%
\begin{pgfscope}%
\pgfsys@transformshift{3.575200in}{2.464868in}%
\pgfsys@useobject{currentmarker}{}%
\end{pgfscope}%
\begin{pgfscope}%
\pgfsys@transformshift{3.573360in}{2.451581in}%
\pgfsys@useobject{currentmarker}{}%
\end{pgfscope}%
\begin{pgfscope}%
\pgfsys@transformshift{3.574665in}{2.436813in}%
\pgfsys@useobject{currentmarker}{}%
\end{pgfscope}%
\begin{pgfscope}%
\pgfsys@transformshift{3.578741in}{2.420947in}%
\pgfsys@useobject{currentmarker}{}%
\end{pgfscope}%
\begin{pgfscope}%
\pgfsys@transformshift{3.576464in}{2.401188in}%
\pgfsys@useobject{currentmarker}{}%
\end{pgfscope}%
\begin{pgfscope}%
\pgfsys@transformshift{3.575317in}{2.390308in}%
\pgfsys@useobject{currentmarker}{}%
\end{pgfscope}%
\begin{pgfscope}%
\pgfsys@transformshift{3.575021in}{2.376780in}%
\pgfsys@useobject{currentmarker}{}%
\end{pgfscope}%
\begin{pgfscope}%
\pgfsys@transformshift{3.576642in}{2.362839in}%
\pgfsys@useobject{currentmarker}{}%
\end{pgfscope}%
\begin{pgfscope}%
\pgfsys@transformshift{3.575013in}{2.345125in}%
\pgfsys@useobject{currentmarker}{}%
\end{pgfscope}%
\begin{pgfscope}%
\pgfsys@transformshift{3.577456in}{2.335650in}%
\pgfsys@useobject{currentmarker}{}%
\end{pgfscope}%
\begin{pgfscope}%
\pgfsys@transformshift{3.581319in}{2.324562in}%
\pgfsys@useobject{currentmarker}{}%
\end{pgfscope}%
\begin{pgfscope}%
\pgfsys@transformshift{3.579075in}{2.309009in}%
\pgfsys@useobject{currentmarker}{}%
\end{pgfscope}%
\begin{pgfscope}%
\pgfsys@transformshift{3.576919in}{2.291220in}%
\pgfsys@useobject{currentmarker}{}%
\end{pgfscope}%
\begin{pgfscope}%
\pgfsys@transformshift{3.577559in}{2.272349in}%
\pgfsys@useobject{currentmarker}{}%
\end{pgfscope}%
\begin{pgfscope}%
\pgfsys@transformshift{3.578205in}{2.261984in}%
\pgfsys@useobject{currentmarker}{}%
\end{pgfscope}%
\begin{pgfscope}%
\pgfsys@transformshift{3.576338in}{2.247233in}%
\pgfsys@useobject{currentmarker}{}%
\end{pgfscope}%
\begin{pgfscope}%
\pgfsys@transformshift{3.578902in}{2.239467in}%
\pgfsys@useobject{currentmarker}{}%
\end{pgfscope}%
\begin{pgfscope}%
\pgfsys@transformshift{3.582596in}{2.228550in}%
\pgfsys@useobject{currentmarker}{}%
\end{pgfscope}%
\begin{pgfscope}%
\pgfsys@transformshift{3.582676in}{2.213725in}%
\pgfsys@useobject{currentmarker}{}%
\end{pgfscope}%
\begin{pgfscope}%
\pgfsys@transformshift{3.580886in}{2.197050in}%
\pgfsys@useobject{currentmarker}{}%
\end{pgfscope}%
\begin{pgfscope}%
\pgfsys@transformshift{3.580137in}{2.176622in}%
\pgfsys@useobject{currentmarker}{}%
\end{pgfscope}%
\begin{pgfscope}%
\pgfsys@transformshift{3.580847in}{2.165402in}%
\pgfsys@useobject{currentmarker}{}%
\end{pgfscope}%
\begin{pgfscope}%
\pgfsys@transformshift{3.579975in}{2.152509in}%
\pgfsys@useobject{currentmarker}{}%
\end{pgfscope}%
\begin{pgfscope}%
\pgfsys@transformshift{3.582039in}{2.138577in}%
\pgfsys@useobject{currentmarker}{}%
\end{pgfscope}%
\begin{pgfscope}%
\pgfsys@transformshift{3.587632in}{2.122087in}%
\pgfsys@useobject{currentmarker}{}%
\end{pgfscope}%
\begin{pgfscope}%
\pgfsys@transformshift{3.586158in}{2.101516in}%
\pgfsys@useobject{currentmarker}{}%
\end{pgfscope}%
\begin{pgfscope}%
\pgfsys@transformshift{3.587094in}{2.080392in}%
\pgfsys@useobject{currentmarker}{}%
\end{pgfscope}%
\begin{pgfscope}%
\pgfsys@transformshift{3.588362in}{2.056305in}%
\pgfsys@useobject{currentmarker}{}%
\end{pgfscope}%
\begin{pgfscope}%
\pgfsys@transformshift{3.589991in}{2.043139in}%
\pgfsys@useobject{currentmarker}{}%
\end{pgfscope}%
\begin{pgfscope}%
\pgfsys@transformshift{3.588965in}{2.028783in}%
\pgfsys@useobject{currentmarker}{}%
\end{pgfscope}%
\begin{pgfscope}%
\pgfsys@transformshift{3.591944in}{2.013978in}%
\pgfsys@useobject{currentmarker}{}%
\end{pgfscope}%
\begin{pgfscope}%
\pgfsys@transformshift{3.595429in}{1.996550in}%
\pgfsys@useobject{currentmarker}{}%
\end{pgfscope}%
\begin{pgfscope}%
\pgfsys@transformshift{3.594204in}{1.974907in}%
\pgfsys@useobject{currentmarker}{}%
\end{pgfscope}%
\begin{pgfscope}%
\pgfsys@transformshift{3.593392in}{1.952446in}%
\pgfsys@useobject{currentmarker}{}%
\end{pgfscope}%
\begin{pgfscope}%
\pgfsys@transformshift{3.596735in}{1.928812in}%
\pgfsys@useobject{currentmarker}{}%
\end{pgfscope}%
\begin{pgfscope}%
\pgfsys@transformshift{3.600598in}{1.904505in}%
\pgfsys@useobject{currentmarker}{}%
\end{pgfscope}%
\begin{pgfscope}%
\pgfsys@transformshift{3.599018in}{1.891061in}%
\pgfsys@useobject{currentmarker}{}%
\end{pgfscope}%
\begin{pgfscope}%
\pgfsys@transformshift{3.600680in}{1.875778in}%
\pgfsys@useobject{currentmarker}{}%
\end{pgfscope}%
\begin{pgfscope}%
\pgfsys@transformshift{3.602150in}{1.856947in}%
\pgfsys@useobject{currentmarker}{}%
\end{pgfscope}%
\begin{pgfscope}%
\pgfsys@transformshift{3.601483in}{1.834480in}%
\pgfsys@useobject{currentmarker}{}%
\end{pgfscope}%
\begin{pgfscope}%
\pgfsys@transformshift{3.601627in}{1.822118in}%
\pgfsys@useobject{currentmarker}{}%
\end{pgfscope}%
\begin{pgfscope}%
\pgfsys@transformshift{3.604424in}{1.807740in}%
\pgfsys@useobject{currentmarker}{}%
\end{pgfscope}%
\begin{pgfscope}%
\pgfsys@transformshift{3.603167in}{1.789336in}%
\pgfsys@useobject{currentmarker}{}%
\end{pgfscope}%
\begin{pgfscope}%
\pgfsys@transformshift{3.599125in}{1.770410in}%
\pgfsys@useobject{currentmarker}{}%
\end{pgfscope}%
\begin{pgfscope}%
\pgfsys@transformshift{3.598368in}{1.750008in}%
\pgfsys@useobject{currentmarker}{}%
\end{pgfscope}%
\begin{pgfscope}%
\pgfsys@transformshift{3.600506in}{1.738984in}%
\pgfsys@useobject{currentmarker}{}%
\end{pgfscope}%
\begin{pgfscope}%
\pgfsys@transformshift{3.598971in}{1.726992in}%
\pgfsys@useobject{currentmarker}{}%
\end{pgfscope}%
\begin{pgfscope}%
\pgfsys@transformshift{3.599431in}{1.720359in}%
\pgfsys@useobject{currentmarker}{}%
\end{pgfscope}%
\begin{pgfscope}%
\pgfsys@transformshift{3.602232in}{1.711024in}%
\pgfsys@useobject{currentmarker}{}%
\end{pgfscope}%
\begin{pgfscope}%
\pgfsys@transformshift{3.600163in}{1.699344in}%
\pgfsys@useobject{currentmarker}{}%
\end{pgfscope}%
\begin{pgfscope}%
\pgfsys@transformshift{3.598083in}{1.685698in}%
\pgfsys@useobject{currentmarker}{}%
\end{pgfscope}%
\begin{pgfscope}%
\pgfsys@transformshift{3.602024in}{1.670058in}%
\pgfsys@useobject{currentmarker}{}%
\end{pgfscope}%
\begin{pgfscope}%
\pgfsys@transformshift{3.607883in}{1.651848in}%
\pgfsys@useobject{currentmarker}{}%
\end{pgfscope}%
\begin{pgfscope}%
\pgfsys@transformshift{3.603675in}{1.629105in}%
\pgfsys@useobject{currentmarker}{}%
\end{pgfscope}%
\begin{pgfscope}%
\pgfsys@transformshift{3.600500in}{1.604794in}%
\pgfsys@useobject{currentmarker}{}%
\end{pgfscope}%
\begin{pgfscope}%
\pgfsys@transformshift{3.601346in}{1.576972in}%
\pgfsys@useobject{currentmarker}{}%
\end{pgfscope}%
\begin{pgfscope}%
\pgfsys@transformshift{3.602734in}{1.561726in}%
\pgfsys@useobject{currentmarker}{}%
\end{pgfscope}%
\begin{pgfscope}%
\pgfsys@transformshift{3.601847in}{1.545182in}%
\pgfsys@useobject{currentmarker}{}%
\end{pgfscope}%
\begin{pgfscope}%
\pgfsys@transformshift{3.602346in}{1.536083in}%
\pgfsys@useobject{currentmarker}{}%
\end{pgfscope}%
\begin{pgfscope}%
\pgfsys@transformshift{3.604647in}{1.523498in}%
\pgfsys@useobject{currentmarker}{}%
\end{pgfscope}%
\begin{pgfscope}%
\pgfsys@transformshift{3.602034in}{1.508349in}%
\pgfsys@useobject{currentmarker}{}%
\end{pgfscope}%
\begin{pgfscope}%
\pgfsys@transformshift{3.599627in}{1.491578in}%
\pgfsys@useobject{currentmarker}{}%
\end{pgfscope}%
\begin{pgfscope}%
\pgfsys@transformshift{3.603010in}{1.473078in}%
\pgfsys@useobject{currentmarker}{}%
\end{pgfscope}%
\begin{pgfscope}%
\pgfsys@transformshift{3.607806in}{1.452170in}%
\pgfsys@useobject{currentmarker}{}%
\end{pgfscope}%
\begin{pgfscope}%
\pgfsys@transformshift{3.605649in}{1.426888in}%
\pgfsys@useobject{currentmarker}{}%
\end{pgfscope}%
\begin{pgfscope}%
\pgfsys@transformshift{3.607239in}{1.400852in}%
\pgfsys@useobject{currentmarker}{}%
\end{pgfscope}%
\begin{pgfscope}%
\pgfsys@transformshift{3.611975in}{1.371724in}%
\pgfsys@useobject{currentmarker}{}%
\end{pgfscope}%
\begin{pgfscope}%
\pgfsys@transformshift{3.612062in}{1.341268in}%
\pgfsys@useobject{currentmarker}{}%
\end{pgfscope}%
\begin{pgfscope}%
\pgfsys@transformshift{3.607463in}{1.310226in}%
\pgfsys@useobject{currentmarker}{}%
\end{pgfscope}%
\begin{pgfscope}%
\pgfsys@transformshift{3.611151in}{1.276926in}%
\pgfsys@useobject{currentmarker}{}%
\end{pgfscope}%
\begin{pgfscope}%
\pgfsys@transformshift{3.610721in}{1.242345in}%
\pgfsys@useobject{currentmarker}{}%
\end{pgfscope}%
\begin{pgfscope}%
\pgfsys@transformshift{3.615346in}{1.206285in}%
\pgfsys@useobject{currentmarker}{}%
\end{pgfscope}%
\begin{pgfscope}%
\pgfsys@transformshift{3.611733in}{1.164892in}%
\pgfsys@useobject{currentmarker}{}%
\end{pgfscope}%
\begin{pgfscope}%
\pgfsys@transformshift{3.604498in}{1.122095in}%
\pgfsys@useobject{currentmarker}{}%
\end{pgfscope}%
\begin{pgfscope}%
\pgfsys@transformshift{3.604147in}{1.077044in}%
\pgfsys@useobject{currentmarker}{}%
\end{pgfscope}%
\begin{pgfscope}%
\pgfsys@transformshift{3.599335in}{1.052737in}%
\pgfsys@useobject{currentmarker}{}%
\end{pgfscope}%
\begin{pgfscope}%
\pgfsys@transformshift{3.592685in}{1.040841in}%
\pgfsys@useobject{currentmarker}{}%
\end{pgfscope}%
\begin{pgfscope}%
\pgfsys@transformshift{3.579976in}{1.031929in}%
\pgfsys@useobject{currentmarker}{}%
\end{pgfscope}%
\begin{pgfscope}%
\pgfsys@transformshift{3.571910in}{1.029132in}%
\pgfsys@useobject{currentmarker}{}%
\end{pgfscope}%
\begin{pgfscope}%
\pgfsys@transformshift{3.561891in}{1.030518in}%
\pgfsys@useobject{currentmarker}{}%
\end{pgfscope}%
\begin{pgfscope}%
\pgfsys@transformshift{3.551208in}{1.030350in}%
\pgfsys@useobject{currentmarker}{}%
\end{pgfscope}%
\begin{pgfscope}%
\pgfsys@transformshift{3.545661in}{1.032289in}%
\pgfsys@useobject{currentmarker}{}%
\end{pgfscope}%
\begin{pgfscope}%
\pgfsys@transformshift{3.542463in}{1.032757in}%
\pgfsys@useobject{currentmarker}{}%
\end{pgfscope}%
\begin{pgfscope}%
\pgfsys@transformshift{3.538616in}{1.034226in}%
\pgfsys@useobject{currentmarker}{}%
\end{pgfscope}%
\begin{pgfscope}%
\pgfsys@transformshift{3.534065in}{1.035126in}%
\pgfsys@useobject{currentmarker}{}%
\end{pgfscope}%
\begin{pgfscope}%
\pgfsys@transformshift{3.528961in}{1.036523in}%
\pgfsys@useobject{currentmarker}{}%
\end{pgfscope}%
\begin{pgfscope}%
\pgfsys@transformshift{3.523209in}{1.038149in}%
\pgfsys@useobject{currentmarker}{}%
\end{pgfscope}%
\begin{pgfscope}%
\pgfsys@transformshift{3.516623in}{1.038192in}%
\pgfsys@useobject{currentmarker}{}%
\end{pgfscope}%
\begin{pgfscope}%
\pgfsys@transformshift{3.506195in}{1.039012in}%
\pgfsys@useobject{currentmarker}{}%
\end{pgfscope}%
\begin{pgfscope}%
\pgfsys@transformshift{3.495272in}{1.040817in}%
\pgfsys@useobject{currentmarker}{}%
\end{pgfscope}%
\begin{pgfscope}%
\pgfsys@transformshift{3.482187in}{1.039925in}%
\pgfsys@useobject{currentmarker}{}%
\end{pgfscope}%
\begin{pgfscope}%
\pgfsys@transformshift{3.467600in}{1.034263in}%
\pgfsys@useobject{currentmarker}{}%
\end{pgfscope}%
\begin{pgfscope}%
\pgfsys@transformshift{3.450584in}{1.033372in}%
\pgfsys@useobject{currentmarker}{}%
\end{pgfscope}%
\begin{pgfscope}%
\pgfsys@transformshift{3.427452in}{1.032471in}%
\pgfsys@useobject{currentmarker}{}%
\end{pgfscope}%
\begin{pgfscope}%
\pgfsys@transformshift{3.405059in}{1.024336in}%
\pgfsys@useobject{currentmarker}{}%
\end{pgfscope}%
\begin{pgfscope}%
\pgfsys@transformshift{3.378829in}{1.022984in}%
\pgfsys@useobject{currentmarker}{}%
\end{pgfscope}%
\begin{pgfscope}%
\pgfsys@transformshift{3.347884in}{1.021213in}%
\pgfsys@useobject{currentmarker}{}%
\end{pgfscope}%
\begin{pgfscope}%
\pgfsys@transformshift{3.314373in}{1.018685in}%
\pgfsys@useobject{currentmarker}{}%
\end{pgfscope}%
\begin{pgfscope}%
\pgfsys@transformshift{3.279065in}{1.008745in}%
\pgfsys@useobject{currentmarker}{}%
\end{pgfscope}%
\begin{pgfscope}%
\pgfsys@transformshift{3.240155in}{1.009391in}%
\pgfsys@useobject{currentmarker}{}%
\end{pgfscope}%
\begin{pgfscope}%
\pgfsys@transformshift{3.197064in}{1.009076in}%
\pgfsys@useobject{currentmarker}{}%
\end{pgfscope}%
\begin{pgfscope}%
\pgfsys@transformshift{3.153223in}{1.003494in}%
\pgfsys@useobject{currentmarker}{}%
\end{pgfscope}%
\begin{pgfscope}%
\pgfsys@transformshift{3.104724in}{0.997750in}%
\pgfsys@useobject{currentmarker}{}%
\end{pgfscope}%
\begin{pgfscope}%
\pgfsys@transformshift{3.055557in}{0.993754in}%
\pgfsys@useobject{currentmarker}{}%
\end{pgfscope}%
\begin{pgfscope}%
\pgfsys@transformshift{3.000769in}{0.996145in}%
\pgfsys@useobject{currentmarker}{}%
\end{pgfscope}%
\begin{pgfscope}%
\pgfsys@transformshift{2.970722in}{0.993517in}%
\pgfsys@useobject{currentmarker}{}%
\end{pgfscope}%
\begin{pgfscope}%
\pgfsys@transformshift{2.937767in}{0.985131in}%
\pgfsys@useobject{currentmarker}{}%
\end{pgfscope}%
\begin{pgfscope}%
\pgfsys@transformshift{2.919101in}{0.983961in}%
\pgfsys@useobject{currentmarker}{}%
\end{pgfscope}%
\begin{pgfscope}%
\pgfsys@transformshift{2.894530in}{0.984263in}%
\pgfsys@useobject{currentmarker}{}%
\end{pgfscope}%
\begin{pgfscope}%
\pgfsys@transformshift{2.869445in}{0.982036in}%
\pgfsys@useobject{currentmarker}{}%
\end{pgfscope}%
\begin{pgfscope}%
\pgfsys@transformshift{2.842310in}{0.976196in}%
\pgfsys@useobject{currentmarker}{}%
\end{pgfscope}%
\begin{pgfscope}%
\pgfsys@transformshift{2.813634in}{0.977199in}%
\pgfsys@useobject{currentmarker}{}%
\end{pgfscope}%
\begin{pgfscope}%
\pgfsys@transformshift{2.780670in}{0.974685in}%
\pgfsys@useobject{currentmarker}{}%
\end{pgfscope}%
\begin{pgfscope}%
\pgfsys@transformshift{2.762911in}{0.970783in}%
\pgfsys@useobject{currentmarker}{}%
\end{pgfscope}%
\begin{pgfscope}%
\pgfsys@transformshift{2.740591in}{0.969180in}%
\pgfsys@useobject{currentmarker}{}%
\end{pgfscope}%
\begin{pgfscope}%
\pgfsys@transformshift{2.716818in}{0.971285in}%
\pgfsys@useobject{currentmarker}{}%
\end{pgfscope}%
\begin{pgfscope}%
\pgfsys@transformshift{2.689741in}{0.968668in}%
\pgfsys@useobject{currentmarker}{}%
\end{pgfscope}%
\begin{pgfscope}%
\pgfsys@transformshift{2.661728in}{0.968244in}%
\pgfsys@useobject{currentmarker}{}%
\end{pgfscope}%
\begin{pgfscope}%
\pgfsys@transformshift{2.629942in}{0.965709in}%
\pgfsys@useobject{currentmarker}{}%
\end{pgfscope}%
\begin{pgfscope}%
\pgfsys@transformshift{2.612429in}{0.966655in}%
\pgfsys@useobject{currentmarker}{}%
\end{pgfscope}%
\begin{pgfscope}%
\pgfsys@transformshift{2.591065in}{0.965260in}%
\pgfsys@useobject{currentmarker}{}%
\end{pgfscope}%
\begin{pgfscope}%
\pgfsys@transformshift{2.569647in}{0.958362in}%
\pgfsys@useobject{currentmarker}{}%
\end{pgfscope}%
\begin{pgfscope}%
\pgfsys@transformshift{2.544877in}{0.957925in}%
\pgfsys@useobject{currentmarker}{}%
\end{pgfscope}%
\begin{pgfscope}%
\pgfsys@transformshift{2.517184in}{0.957037in}%
\pgfsys@useobject{currentmarker}{}%
\end{pgfscope}%
\begin{pgfscope}%
\pgfsys@transformshift{2.488332in}{0.950610in}%
\pgfsys@useobject{currentmarker}{}%
\end{pgfscope}%
\begin{pgfscope}%
\pgfsys@transformshift{2.458249in}{0.942347in}%
\pgfsys@useobject{currentmarker}{}%
\end{pgfscope}%
\begin{pgfscope}%
\pgfsys@transformshift{2.424906in}{0.937899in}%
\pgfsys@useobject{currentmarker}{}%
\end{pgfscope}%
\begin{pgfscope}%
\pgfsys@transformshift{2.386924in}{0.933874in}%
\pgfsys@useobject{currentmarker}{}%
\end{pgfscope}%
\begin{pgfscope}%
\pgfsys@transformshift{2.365953in}{0.932641in}%
\pgfsys@useobject{currentmarker}{}%
\end{pgfscope}%
\begin{pgfscope}%
\pgfsys@transformshift{2.343289in}{0.924153in}%
\pgfsys@useobject{currentmarker}{}%
\end{pgfscope}%
\begin{pgfscope}%
\pgfsys@transformshift{2.318510in}{0.920792in}%
\pgfsys@useobject{currentmarker}{}%
\end{pgfscope}%
\begin{pgfscope}%
\pgfsys@transformshift{2.288654in}{0.918733in}%
\pgfsys@useobject{currentmarker}{}%
\end{pgfscope}%
\begin{pgfscope}%
\pgfsys@transformshift{2.258351in}{0.915010in}%
\pgfsys@useobject{currentmarker}{}%
\end{pgfscope}%
\begin{pgfscope}%
\pgfsys@transformshift{2.226405in}{0.907144in}%
\pgfsys@useobject{currentmarker}{}%
\end{pgfscope}%
\begin{pgfscope}%
\pgfsys@transformshift{2.208318in}{0.906602in}%
\pgfsys@useobject{currentmarker}{}%
\end{pgfscope}%
\begin{pgfscope}%
\pgfsys@transformshift{2.185631in}{0.906772in}%
\pgfsys@useobject{currentmarker}{}%
\end{pgfscope}%
\begin{pgfscope}%
\pgfsys@transformshift{2.161766in}{0.904081in}%
\pgfsys@useobject{currentmarker}{}%
\end{pgfscope}%
\begin{pgfscope}%
\pgfsys@transformshift{2.136956in}{0.892633in}%
\pgfsys@useobject{currentmarker}{}%
\end{pgfscope}%
\begin{pgfscope}%
\pgfsys@transformshift{2.106457in}{0.890732in}%
\pgfsys@useobject{currentmarker}{}%
\end{pgfscope}%
\begin{pgfscope}%
\pgfsys@transformshift{2.074298in}{0.891304in}%
\pgfsys@useobject{currentmarker}{}%
\end{pgfscope}%
\begin{pgfscope}%
\pgfsys@transformshift{2.038306in}{0.889833in}%
\pgfsys@useobject{currentmarker}{}%
\end{pgfscope}%
\begin{pgfscope}%
\pgfsys@transformshift{2.002481in}{0.880949in}%
\pgfsys@useobject{currentmarker}{}%
\end{pgfscope}%
\begin{pgfscope}%
\pgfsys@transformshift{1.960913in}{0.877311in}%
\pgfsys@useobject{currentmarker}{}%
\end{pgfscope}%
\begin{pgfscope}%
\pgfsys@transformshift{1.918729in}{0.876292in}%
\pgfsys@useobject{currentmarker}{}%
\end{pgfscope}%
\begin{pgfscope}%
\pgfsys@transformshift{1.872058in}{0.870751in}%
\pgfsys@useobject{currentmarker}{}%
\end{pgfscope}%
\begin{pgfscope}%
\pgfsys@transformshift{1.847588in}{0.862417in}%
\pgfsys@useobject{currentmarker}{}%
\end{pgfscope}%
\begin{pgfscope}%
\pgfsys@transformshift{1.817791in}{0.859472in}%
\pgfsys@useobject{currentmarker}{}%
\end{pgfscope}%
\begin{pgfscope}%
\pgfsys@transformshift{1.801334in}{0.858854in}%
\pgfsys@useobject{currentmarker}{}%
\end{pgfscope}%
\begin{pgfscope}%
\pgfsys@transformshift{1.779991in}{0.857227in}%
\pgfsys@useobject{currentmarker}{}%
\end{pgfscope}%
\begin{pgfscope}%
\pgfsys@transformshift{1.768265in}{0.856164in}%
\pgfsys@useobject{currentmarker}{}%
\end{pgfscope}%
\begin{pgfscope}%
\pgfsys@transformshift{1.753639in}{0.857684in}%
\pgfsys@useobject{currentmarker}{}%
\end{pgfscope}%
\begin{pgfscope}%
\pgfsys@transformshift{1.745680in}{0.859126in}%
\pgfsys@useobject{currentmarker}{}%
\end{pgfscope}%
\begin{pgfscope}%
\pgfsys@transformshift{1.733528in}{0.861834in}%
\pgfsys@useobject{currentmarker}{}%
\end{pgfscope}%
\begin{pgfscope}%
\pgfsys@transformshift{1.726796in}{0.863087in}%
\pgfsys@useobject{currentmarker}{}%
\end{pgfscope}%
\begin{pgfscope}%
\pgfsys@transformshift{1.717815in}{0.864903in}%
\pgfsys@useobject{currentmarker}{}%
\end{pgfscope}%
\begin{pgfscope}%
\pgfsys@transformshift{1.712940in}{0.866179in}%
\pgfsys@useobject{currentmarker}{}%
\end{pgfscope}%
\begin{pgfscope}%
\pgfsys@transformshift{1.704575in}{0.867243in}%
\pgfsys@useobject{currentmarker}{}%
\end{pgfscope}%
\begin{pgfscope}%
\pgfsys@transformshift{1.700034in}{0.868183in}%
\pgfsys@useobject{currentmarker}{}%
\end{pgfscope}%
\begin{pgfscope}%
\pgfsys@transformshift{1.694216in}{0.868202in}%
\pgfsys@useobject{currentmarker}{}%
\end{pgfscope}%
\begin{pgfscope}%
\pgfsys@transformshift{1.687876in}{0.869778in}%
\pgfsys@useobject{currentmarker}{}%
\end{pgfscope}%
\begin{pgfscope}%
\pgfsys@transformshift{1.679639in}{0.870397in}%
\pgfsys@useobject{currentmarker}{}%
\end{pgfscope}%
\begin{pgfscope}%
\pgfsys@transformshift{1.675170in}{0.871208in}%
\pgfsys@useobject{currentmarker}{}%
\end{pgfscope}%
\begin{pgfscope}%
\pgfsys@transformshift{1.669884in}{0.870718in}%
\pgfsys@useobject{currentmarker}{}%
\end{pgfscope}%
\begin{pgfscope}%
\pgfsys@transformshift{1.664133in}{0.871704in}%
\pgfsys@useobject{currentmarker}{}%
\end{pgfscope}%
\begin{pgfscope}%
\pgfsys@transformshift{1.655030in}{0.872623in}%
\pgfsys@useobject{currentmarker}{}%
\end{pgfscope}%
\begin{pgfscope}%
\pgfsys@transformshift{1.644510in}{0.872607in}%
\pgfsys@useobject{currentmarker}{}%
\end{pgfscope}%
\begin{pgfscope}%
\pgfsys@transformshift{1.632153in}{0.872634in}%
\pgfsys@useobject{currentmarker}{}%
\end{pgfscope}%
\begin{pgfscope}%
\pgfsys@transformshift{1.617864in}{0.874650in}%
\pgfsys@useobject{currentmarker}{}%
\end{pgfscope}%
\begin{pgfscope}%
\pgfsys@transformshift{1.600601in}{0.874582in}%
\pgfsys@useobject{currentmarker}{}%
\end{pgfscope}%
\begin{pgfscope}%
\pgfsys@transformshift{1.591109in}{0.874340in}%
\pgfsys@useobject{currentmarker}{}%
\end{pgfscope}%
\begin{pgfscope}%
\pgfsys@transformshift{1.580001in}{0.874717in}%
\pgfsys@useobject{currentmarker}{}%
\end{pgfscope}%
\begin{pgfscope}%
\pgfsys@transformshift{1.567598in}{0.875656in}%
\pgfsys@useobject{currentmarker}{}%
\end{pgfscope}%
\begin{pgfscope}%
\pgfsys@transformshift{1.553633in}{0.875904in}%
\pgfsys@useobject{currentmarker}{}%
\end{pgfscope}%
\begin{pgfscope}%
\pgfsys@transformshift{1.545963in}{0.876338in}%
\pgfsys@useobject{currentmarker}{}%
\end{pgfscope}%
\begin{pgfscope}%
\pgfsys@transformshift{1.537147in}{0.877309in}%
\pgfsys@useobject{currentmarker}{}%
\end{pgfscope}%
\begin{pgfscope}%
\pgfsys@transformshift{1.532949in}{0.879792in}%
\pgfsys@useobject{currentmarker}{}%
\end{pgfscope}%
\begin{pgfscope}%
\pgfsys@transformshift{1.528103in}{0.882675in}%
\pgfsys@useobject{currentmarker}{}%
\end{pgfscope}%
\begin{pgfscope}%
\pgfsys@transformshift{1.525591in}{0.884492in}%
\pgfsys@useobject{currentmarker}{}%
\end{pgfscope}%
\begin{pgfscope}%
\pgfsys@transformshift{1.524823in}{0.886015in}%
\pgfsys@useobject{currentmarker}{}%
\end{pgfscope}%
\begin{pgfscope}%
\pgfsys@transformshift{1.523581in}{0.888239in}%
\pgfsys@useobject{currentmarker}{}%
\end{pgfscope}%
\begin{pgfscope}%
\pgfsys@transformshift{1.523592in}{0.889639in}%
\pgfsys@useobject{currentmarker}{}%
\end{pgfscope}%
\begin{pgfscope}%
\pgfsys@transformshift{1.523440in}{0.891811in}%
\pgfsys@useobject{currentmarker}{}%
\end{pgfscope}%
\begin{pgfscope}%
\pgfsys@transformshift{1.523575in}{0.895295in}%
\pgfsys@useobject{currentmarker}{}%
\end{pgfscope}%
\begin{pgfscope}%
\pgfsys@transformshift{1.523925in}{0.899613in}%
\pgfsys@useobject{currentmarker}{}%
\end{pgfscope}%
\begin{pgfscope}%
\pgfsys@transformshift{1.524169in}{0.904554in}%
\pgfsys@useobject{currentmarker}{}%
\end{pgfscope}%
\begin{pgfscope}%
\pgfsys@transformshift{1.524302in}{0.907272in}%
\pgfsys@useobject{currentmarker}{}%
\end{pgfscope}%
\begin{pgfscope}%
\pgfsys@transformshift{1.524404in}{0.908765in}%
\pgfsys@useobject{currentmarker}{}%
\end{pgfscope}%
\begin{pgfscope}%
\pgfsys@transformshift{1.524186in}{0.910804in}%
\pgfsys@useobject{currentmarker}{}%
\end{pgfscope}%
\begin{pgfscope}%
\pgfsys@transformshift{1.524026in}{0.913350in}%
\pgfsys@useobject{currentmarker}{}%
\end{pgfscope}%
\begin{pgfscope}%
\pgfsys@transformshift{1.523835in}{0.916492in}%
\pgfsys@useobject{currentmarker}{}%
\end{pgfscope}%
\begin{pgfscope}%
\pgfsys@transformshift{1.523858in}{0.918223in}%
\pgfsys@useobject{currentmarker}{}%
\end{pgfscope}%
\begin{pgfscope}%
\pgfsys@transformshift{1.523676in}{0.920594in}%
\pgfsys@useobject{currentmarker}{}%
\end{pgfscope}%
\begin{pgfscope}%
\pgfsys@transformshift{1.523737in}{0.921900in}%
\pgfsys@useobject{currentmarker}{}%
\end{pgfscope}%
\begin{pgfscope}%
\pgfsys@transformshift{1.523478in}{0.924112in}%
\pgfsys@useobject{currentmarker}{}%
\end{pgfscope}%
\begin{pgfscope}%
\pgfsys@transformshift{1.523569in}{0.926949in}%
\pgfsys@useobject{currentmarker}{}%
\end{pgfscope}%
\begin{pgfscope}%
\pgfsys@transformshift{1.523451in}{0.928506in}%
\pgfsys@useobject{currentmarker}{}%
\end{pgfscope}%
\begin{pgfscope}%
\pgfsys@transformshift{1.523487in}{0.929364in}%
\pgfsys@useobject{currentmarker}{}%
\end{pgfscope}%
\begin{pgfscope}%
\pgfsys@transformshift{1.523344in}{0.930712in}%
\pgfsys@useobject{currentmarker}{}%
\end{pgfscope}%
\begin{pgfscope}%
\pgfsys@transformshift{1.523459in}{0.932525in}%
\pgfsys@useobject{currentmarker}{}%
\end{pgfscope}%
\begin{pgfscope}%
\pgfsys@transformshift{1.523218in}{0.934884in}%
\pgfsys@useobject{currentmarker}{}%
\end{pgfscope}%
\begin{pgfscope}%
\pgfsys@transformshift{1.523292in}{0.936186in}%
\pgfsys@useobject{currentmarker}{}%
\end{pgfscope}%
\begin{pgfscope}%
\pgfsys@transformshift{1.523219in}{0.936899in}%
\pgfsys@useobject{currentmarker}{}%
\end{pgfscope}%
\begin{pgfscope}%
\pgfsys@transformshift{1.523250in}{0.937292in}%
\pgfsys@useobject{currentmarker}{}%
\end{pgfscope}%
\begin{pgfscope}%
\pgfsys@transformshift{1.523227in}{0.937508in}%
\pgfsys@useobject{currentmarker}{}%
\end{pgfscope}%
\begin{pgfscope}%
\pgfsys@transformshift{1.523236in}{0.937627in}%
\pgfsys@useobject{currentmarker}{}%
\end{pgfscope}%
\begin{pgfscope}%
\pgfsys@transformshift{1.523228in}{0.937692in}%
\pgfsys@useobject{currentmarker}{}%
\end{pgfscope}%
\begin{pgfscope}%
\pgfsys@transformshift{1.523275in}{0.938238in}%
\pgfsys@useobject{currentmarker}{}%
\end{pgfscope}%
\begin{pgfscope}%
\pgfsys@transformshift{1.523191in}{0.939270in}%
\pgfsys@useobject{currentmarker}{}%
\end{pgfscope}%
\begin{pgfscope}%
\pgfsys@transformshift{1.523231in}{0.939838in}%
\pgfsys@useobject{currentmarker}{}%
\end{pgfscope}%
\begin{pgfscope}%
\pgfsys@transformshift{1.523202in}{0.940150in}%
\pgfsys@useobject{currentmarker}{}%
\end{pgfscope}%
\begin{pgfscope}%
\pgfsys@transformshift{1.523213in}{0.940322in}%
\pgfsys@useobject{currentmarker}{}%
\end{pgfscope}%
\begin{pgfscope}%
\pgfsys@transformshift{1.523205in}{0.940416in}%
\pgfsys@useobject{currentmarker}{}%
\end{pgfscope}%
\begin{pgfscope}%
\pgfsys@transformshift{1.523208in}{0.940468in}%
\pgfsys@useobject{currentmarker}{}%
\end{pgfscope}%
\begin{pgfscope}%
\pgfsys@transformshift{1.523206in}{0.940497in}%
\pgfsys@useobject{currentmarker}{}%
\end{pgfscope}%
\begin{pgfscope}%
\pgfsys@transformshift{1.523207in}{0.940513in}%
\pgfsys@useobject{currentmarker}{}%
\end{pgfscope}%
\begin{pgfscope}%
\pgfsys@transformshift{1.523206in}{0.940521in}%
\pgfsys@useobject{currentmarker}{}%
\end{pgfscope}%
\begin{pgfscope}%
\pgfsys@transformshift{1.523206in}{0.940526in}%
\pgfsys@useobject{currentmarker}{}%
\end{pgfscope}%
\begin{pgfscope}%
\pgfsys@transformshift{1.523206in}{0.940529in}%
\pgfsys@useobject{currentmarker}{}%
\end{pgfscope}%
\begin{pgfscope}%
\pgfsys@transformshift{1.523206in}{0.940530in}%
\pgfsys@useobject{currentmarker}{}%
\end{pgfscope}%
\begin{pgfscope}%
\pgfsys@transformshift{1.523206in}{0.940531in}%
\pgfsys@useobject{currentmarker}{}%
\end{pgfscope}%
\begin{pgfscope}%
\pgfsys@transformshift{1.523206in}{0.940531in}%
\pgfsys@useobject{currentmarker}{}%
\end{pgfscope}%
\begin{pgfscope}%
\pgfsys@transformshift{1.523206in}{0.940531in}%
\pgfsys@useobject{currentmarker}{}%
\end{pgfscope}%
\begin{pgfscope}%
\pgfsys@transformshift{1.522952in}{0.941287in}%
\pgfsys@useobject{currentmarker}{}%
\end{pgfscope}%
\begin{pgfscope}%
\pgfsys@transformshift{1.523238in}{0.944973in}%
\pgfsys@useobject{currentmarker}{}%
\end{pgfscope}%
\begin{pgfscope}%
\pgfsys@transformshift{1.523533in}{0.949311in}%
\pgfsys@useobject{currentmarker}{}%
\end{pgfscope}%
\begin{pgfscope}%
\pgfsys@transformshift{1.521795in}{0.955312in}%
\pgfsys@useobject{currentmarker}{}%
\end{pgfscope}%
\begin{pgfscope}%
\pgfsys@transformshift{1.521101in}{0.962075in}%
\pgfsys@useobject{currentmarker}{}%
\end{pgfscope}%
\begin{pgfscope}%
\pgfsys@transformshift{1.521989in}{0.965708in}%
\pgfsys@useobject{currentmarker}{}%
\end{pgfscope}%
\begin{pgfscope}%
\pgfsys@transformshift{1.520132in}{0.972385in}%
\pgfsys@useobject{currentmarker}{}%
\end{pgfscope}%
\begin{pgfscope}%
\pgfsys@transformshift{1.520707in}{0.976153in}%
\pgfsys@useobject{currentmarker}{}%
\end{pgfscope}%
\begin{pgfscope}%
\pgfsys@transformshift{1.520826in}{0.978246in}%
\pgfsys@useobject{currentmarker}{}%
\end{pgfscope}%
\begin{pgfscope}%
\pgfsys@transformshift{1.520138in}{0.984317in}%
\pgfsys@useobject{currentmarker}{}%
\end{pgfscope}%
\begin{pgfscope}%
\pgfsys@transformshift{1.521140in}{0.990843in}%
\pgfsys@useobject{currentmarker}{}%
\end{pgfscope}%
\begin{pgfscope}%
\pgfsys@transformshift{1.523776in}{1.000735in}%
\pgfsys@useobject{currentmarker}{}%
\end{pgfscope}%
\begin{pgfscope}%
\pgfsys@transformshift{1.521775in}{1.013781in}%
\pgfsys@useobject{currentmarker}{}%
\end{pgfscope}%
\begin{pgfscope}%
\pgfsys@transformshift{1.523206in}{1.027429in}%
\pgfsys@useobject{currentmarker}{}%
\end{pgfscope}%
\begin{pgfscope}%
\pgfsys@transformshift{1.524164in}{1.044657in}%
\pgfsys@useobject{currentmarker}{}%
\end{pgfscope}%
\begin{pgfscope}%
\pgfsys@transformshift{1.526577in}{1.062352in}%
\pgfsys@useobject{currentmarker}{}%
\end{pgfscope}%
\begin{pgfscope}%
\pgfsys@transformshift{1.524583in}{1.071969in}%
\pgfsys@useobject{currentmarker}{}%
\end{pgfscope}%
\begin{pgfscope}%
\pgfsys@transformshift{1.525435in}{1.083147in}%
\pgfsys@useobject{currentmarker}{}%
\end{pgfscope}%
\begin{pgfscope}%
\pgfsys@transformshift{1.527681in}{1.094900in}%
\pgfsys@useobject{currentmarker}{}%
\end{pgfscope}%
\begin{pgfscope}%
\pgfsys@transformshift{1.524910in}{1.109677in}%
\pgfsys@useobject{currentmarker}{}%
\end{pgfscope}%
\begin{pgfscope}%
\pgfsys@transformshift{1.525457in}{1.125348in}%
\pgfsys@useobject{currentmarker}{}%
\end{pgfscope}%
\begin{pgfscope}%
\pgfsys@transformshift{1.530103in}{1.141468in}%
\pgfsys@useobject{currentmarker}{}%
\end{pgfscope}%
\begin{pgfscope}%
\pgfsys@transformshift{1.526662in}{1.163395in}%
\pgfsys@useobject{currentmarker}{}%
\end{pgfscope}%
\begin{pgfscope}%
\pgfsys@transformshift{1.525682in}{1.186751in}%
\pgfsys@useobject{currentmarker}{}%
\end{pgfscope}%
\begin{pgfscope}%
\pgfsys@transformshift{1.517118in}{1.211858in}%
\pgfsys@useobject{currentmarker}{}%
\end{pgfscope}%
\begin{pgfscope}%
\pgfsys@transformshift{1.521363in}{1.225818in}%
\pgfsys@useobject{currentmarker}{}%
\end{pgfscope}%
\begin{pgfscope}%
\pgfsys@transformshift{1.518630in}{1.245733in}%
\pgfsys@useobject{currentmarker}{}%
\end{pgfscope}%
\begin{pgfscope}%
\pgfsys@transformshift{1.518290in}{1.256784in}%
\pgfsys@useobject{currentmarker}{}%
\end{pgfscope}%
\begin{pgfscope}%
\pgfsys@transformshift{1.522022in}{1.270890in}%
\pgfsys@useobject{currentmarker}{}%
\end{pgfscope}%
\begin{pgfscope}%
\pgfsys@transformshift{1.521317in}{1.287224in}%
\pgfsys@useobject{currentmarker}{}%
\end{pgfscope}%
\begin{pgfscope}%
\pgfsys@transformshift{1.520731in}{1.296196in}%
\pgfsys@useobject{currentmarker}{}%
\end{pgfscope}%
\begin{pgfscope}%
\pgfsys@transformshift{1.520599in}{1.308367in}%
\pgfsys@useobject{currentmarker}{}%
\end{pgfscope}%
\begin{pgfscope}%
\pgfsys@transformshift{1.522345in}{1.314830in}%
\pgfsys@useobject{currentmarker}{}%
\end{pgfscope}%
\begin{pgfscope}%
\pgfsys@transformshift{1.520472in}{1.324082in}%
\pgfsys@useobject{currentmarker}{}%
\end{pgfscope}%
\begin{pgfscope}%
\pgfsys@transformshift{1.520398in}{1.329273in}%
\pgfsys@useobject{currentmarker}{}%
\end{pgfscope}%
\begin{pgfscope}%
\pgfsys@transformshift{1.523216in}{1.338701in}%
\pgfsys@useobject{currentmarker}{}%
\end{pgfscope}%
\begin{pgfscope}%
\pgfsys@transformshift{1.522178in}{1.351675in}%
\pgfsys@useobject{currentmarker}{}%
\end{pgfscope}%
\begin{pgfscope}%
\pgfsys@transformshift{1.523543in}{1.365201in}%
\pgfsys@useobject{currentmarker}{}%
\end{pgfscope}%
\begin{pgfscope}%
\pgfsys@transformshift{1.530648in}{1.382343in}%
\pgfsys@useobject{currentmarker}{}%
\end{pgfscope}%
\begin{pgfscope}%
\pgfsys@transformshift{1.530707in}{1.401606in}%
\pgfsys@useobject{currentmarker}{}%
\end{pgfscope}%
\begin{pgfscope}%
\pgfsys@transformshift{1.526143in}{1.420960in}%
\pgfsys@useobject{currentmarker}{}%
\end{pgfscope}%
\begin{pgfscope}%
\pgfsys@transformshift{1.534430in}{1.444777in}%
\pgfsys@useobject{currentmarker}{}%
\end{pgfscope}%
\begin{pgfscope}%
\pgfsys@transformshift{1.535922in}{1.458567in}%
\pgfsys@useobject{currentmarker}{}%
\end{pgfscope}%
\begin{pgfscope}%
\pgfsys@transformshift{1.535253in}{1.473015in}%
\pgfsys@useobject{currentmarker}{}%
\end{pgfscope}%
\begin{pgfscope}%
\pgfsys@transformshift{1.537358in}{1.490827in}%
\pgfsys@useobject{currentmarker}{}%
\end{pgfscope}%
\begin{pgfscope}%
\pgfsys@transformshift{1.540146in}{1.509060in}%
\pgfsys@useobject{currentmarker}{}%
\end{pgfscope}%
\begin{pgfscope}%
\pgfsys@transformshift{1.535713in}{1.529583in}%
\pgfsys@useobject{currentmarker}{}%
\end{pgfscope}%
\begin{pgfscope}%
\pgfsys@transformshift{1.536757in}{1.551250in}%
\pgfsys@useobject{currentmarker}{}%
\end{pgfscope}%
\begin{pgfscope}%
\pgfsys@transformshift{1.542287in}{1.573384in}%
\pgfsys@useobject{currentmarker}{}%
\end{pgfscope}%
\begin{pgfscope}%
\pgfsys@transformshift{1.538332in}{1.601627in}%
\pgfsys@useobject{currentmarker}{}%
\end{pgfscope}%
\begin{pgfscope}%
\pgfsys@transformshift{1.538085in}{1.617310in}%
\pgfsys@useobject{currentmarker}{}%
\end{pgfscope}%
\begin{pgfscope}%
\pgfsys@transformshift{1.540313in}{1.636189in}%
\pgfsys@useobject{currentmarker}{}%
\end{pgfscope}%
\begin{pgfscope}%
\pgfsys@transformshift{1.541929in}{1.646518in}%
\pgfsys@useobject{currentmarker}{}%
\end{pgfscope}%
\begin{pgfscope}%
\pgfsys@transformshift{1.539817in}{1.657724in}%
\pgfsys@useobject{currentmarker}{}%
\end{pgfscope}%
\begin{pgfscope}%
\pgfsys@transformshift{1.539689in}{1.670670in}%
\pgfsys@useobject{currentmarker}{}%
\end{pgfscope}%
\begin{pgfscope}%
\pgfsys@transformshift{1.544305in}{1.684995in}%
\pgfsys@useobject{currentmarker}{}%
\end{pgfscope}%
\begin{pgfscope}%
\pgfsys@transformshift{1.541653in}{1.702957in}%
\pgfsys@useobject{currentmarker}{}%
\end{pgfscope}%
\begin{pgfscope}%
\pgfsys@transformshift{1.542243in}{1.712926in}%
\pgfsys@useobject{currentmarker}{}%
\end{pgfscope}%
\begin{pgfscope}%
\pgfsys@transformshift{1.547187in}{1.726191in}%
\pgfsys@useobject{currentmarker}{}%
\end{pgfscope}%
\begin{pgfscope}%
\pgfsys@transformshift{1.548739in}{1.743707in}%
\pgfsys@useobject{currentmarker}{}%
\end{pgfscope}%
\begin{pgfscope}%
\pgfsys@transformshift{1.548790in}{1.753379in}%
\pgfsys@useobject{currentmarker}{}%
\end{pgfscope}%
\begin{pgfscope}%
\pgfsys@transformshift{1.551682in}{1.766840in}%
\pgfsys@useobject{currentmarker}{}%
\end{pgfscope}%
\begin{pgfscope}%
\pgfsys@transformshift{1.552582in}{1.781054in}%
\pgfsys@useobject{currentmarker}{}%
\end{pgfscope}%
\begin{pgfscope}%
\pgfsys@transformshift{1.549181in}{1.796571in}%
\pgfsys@useobject{currentmarker}{}%
\end{pgfscope}%
\begin{pgfscope}%
\pgfsys@transformshift{1.548603in}{1.805289in}%
\pgfsys@useobject{currentmarker}{}%
\end{pgfscope}%
\begin{pgfscope}%
\pgfsys@transformshift{1.550334in}{1.814502in}%
\pgfsys@useobject{currentmarker}{}%
\end{pgfscope}%
\begin{pgfscope}%
\pgfsys@transformshift{1.547513in}{1.829257in}%
\pgfsys@useobject{currentmarker}{}%
\end{pgfscope}%
\begin{pgfscope}%
\pgfsys@transformshift{1.548027in}{1.837503in}%
\pgfsys@useobject{currentmarker}{}%
\end{pgfscope}%
\begin{pgfscope}%
\pgfsys@transformshift{1.550113in}{1.849254in}%
\pgfsys@useobject{currentmarker}{}%
\end{pgfscope}%
\begin{pgfscope}%
\pgfsys@transformshift{1.546453in}{1.864547in}%
\pgfsys@useobject{currentmarker}{}%
\end{pgfscope}%
\begin{pgfscope}%
\pgfsys@transformshift{1.546585in}{1.873195in}%
\pgfsys@useobject{currentmarker}{}%
\end{pgfscope}%
\begin{pgfscope}%
\pgfsys@transformshift{1.549781in}{1.885480in}%
\pgfsys@useobject{currentmarker}{}%
\end{pgfscope}%
\begin{pgfscope}%
\pgfsys@transformshift{1.546225in}{1.898590in}%
\pgfsys@useobject{currentmarker}{}%
\end{pgfscope}%
\begin{pgfscope}%
\pgfsys@transformshift{1.544384in}{1.912934in}%
\pgfsys@useobject{currentmarker}{}%
\end{pgfscope}%
\begin{pgfscope}%
\pgfsys@transformshift{1.543419in}{1.930835in}%
\pgfsys@useobject{currentmarker}{}%
\end{pgfscope}%
\begin{pgfscope}%
\pgfsys@transformshift{1.545746in}{1.949350in}%
\pgfsys@useobject{currentmarker}{}%
\end{pgfscope}%
\begin{pgfscope}%
\pgfsys@transformshift{1.538105in}{1.969264in}%
\pgfsys@useobject{currentmarker}{}%
\end{pgfscope}%
\begin{pgfscope}%
\pgfsys@transformshift{1.537223in}{1.991289in}%
\pgfsys@useobject{currentmarker}{}%
\end{pgfscope}%
\begin{pgfscope}%
\pgfsys@transformshift{1.541728in}{2.014690in}%
\pgfsys@useobject{currentmarker}{}%
\end{pgfscope}%
\begin{pgfscope}%
\pgfsys@transformshift{1.532299in}{2.041963in}%
\pgfsys@useobject{currentmarker}{}%
\end{pgfscope}%
\begin{pgfscope}%
\pgfsys@transformshift{1.531909in}{2.057830in}%
\pgfsys@useobject{currentmarker}{}%
\end{pgfscope}%
\begin{pgfscope}%
\pgfsys@transformshift{1.534936in}{2.077138in}%
\pgfsys@useobject{currentmarker}{}%
\end{pgfscope}%
\begin{pgfscope}%
\pgfsys@transformshift{1.535609in}{2.097380in}%
\pgfsys@useobject{currentmarker}{}%
\end{pgfscope}%
\begin{pgfscope}%
\pgfsys@transformshift{1.529521in}{2.117293in}%
\pgfsys@useobject{currentmarker}{}%
\end{pgfscope}%
\begin{pgfscope}%
\pgfsys@transformshift{1.529829in}{2.140252in}%
\pgfsys@useobject{currentmarker}{}%
\end{pgfscope}%
\begin{pgfscope}%
\pgfsys@transformshift{1.536431in}{2.165121in}%
\pgfsys@useobject{currentmarker}{}%
\end{pgfscope}%
\begin{pgfscope}%
\pgfsys@transformshift{1.530385in}{2.193093in}%
\pgfsys@useobject{currentmarker}{}%
\end{pgfscope}%
\begin{pgfscope}%
\pgfsys@transformshift{1.529559in}{2.208811in}%
\pgfsys@useobject{currentmarker}{}%
\end{pgfscope}%
\begin{pgfscope}%
\pgfsys@transformshift{1.535245in}{2.228031in}%
\pgfsys@useobject{currentmarker}{}%
\end{pgfscope}%
\begin{pgfscope}%
\pgfsys@transformshift{1.528661in}{2.250083in}%
\pgfsys@useobject{currentmarker}{}%
\end{pgfscope}%
\begin{pgfscope}%
\pgfsys@transformshift{1.522735in}{2.273225in}%
\pgfsys@useobject{currentmarker}{}%
\end{pgfscope}%
\begin{pgfscope}%
\pgfsys@transformshift{1.526343in}{2.301727in}%
\pgfsys@useobject{currentmarker}{}%
\end{pgfscope}%
\begin{pgfscope}%
\pgfsys@transformshift{1.528139in}{2.331481in}%
\pgfsys@useobject{currentmarker}{}%
\end{pgfscope}%
\begin{pgfscope}%
\pgfsys@transformshift{1.518470in}{2.360401in}%
\pgfsys@useobject{currentmarker}{}%
\end{pgfscope}%
\begin{pgfscope}%
\pgfsys@transformshift{1.526448in}{2.393686in}%
\pgfsys@useobject{currentmarker}{}%
\end{pgfscope}%
\begin{pgfscope}%
\pgfsys@transformshift{1.525275in}{2.412475in}%
\pgfsys@useobject{currentmarker}{}%
\end{pgfscope}%
\begin{pgfscope}%
\pgfsys@transformshift{1.518295in}{2.432218in}%
\pgfsys@useobject{currentmarker}{}%
\end{pgfscope}%
\begin{pgfscope}%
\pgfsys@transformshift{1.518846in}{2.443722in}%
\pgfsys@useobject{currentmarker}{}%
\end{pgfscope}%
\begin{pgfscope}%
\pgfsys@transformshift{1.519626in}{2.456614in}%
\pgfsys@useobject{currentmarker}{}%
\end{pgfscope}%
\begin{pgfscope}%
\pgfsys@transformshift{1.513837in}{2.474148in}%
\pgfsys@useobject{currentmarker}{}%
\end{pgfscope}%
\begin{pgfscope}%
\pgfsys@transformshift{1.513413in}{2.484294in}%
\pgfsys@useobject{currentmarker}{}%
\end{pgfscope}%
\begin{pgfscope}%
\pgfsys@transformshift{1.518083in}{2.497140in}%
\pgfsys@useobject{currentmarker}{}%
\end{pgfscope}%
\begin{pgfscope}%
\pgfsys@transformshift{1.514406in}{2.515035in}%
\pgfsys@useobject{currentmarker}{}%
\end{pgfscope}%
\begin{pgfscope}%
\pgfsys@transformshift{1.513579in}{2.525048in}%
\pgfsys@useobject{currentmarker}{}%
\end{pgfscope}%
\begin{pgfscope}%
\pgfsys@transformshift{1.518471in}{2.538535in}%
\pgfsys@useobject{currentmarker}{}%
\end{pgfscope}%
\begin{pgfscope}%
\pgfsys@transformshift{1.514190in}{2.556442in}%
\pgfsys@useobject{currentmarker}{}%
\end{pgfscope}%
\begin{pgfscope}%
\pgfsys@transformshift{1.511943in}{2.575777in}%
\pgfsys@useobject{currentmarker}{}%
\end{pgfscope}%
\begin{pgfscope}%
\pgfsys@transformshift{1.512263in}{2.599146in}%
\pgfsys@useobject{currentmarker}{}%
\end{pgfscope}%
\begin{pgfscope}%
\pgfsys@transformshift{1.512286in}{2.612000in}%
\pgfsys@useobject{currentmarker}{}%
\end{pgfscope}%
\begin{pgfscope}%
\pgfsys@transformshift{1.510004in}{2.618692in}%
\pgfsys@useobject{currentmarker}{}%
\end{pgfscope}%
\begin{pgfscope}%
\pgfsys@transformshift{1.510650in}{2.628716in}%
\pgfsys@useobject{currentmarker}{}%
\end{pgfscope}%
\begin{pgfscope}%
\pgfsys@transformshift{1.514150in}{2.639287in}%
\pgfsys@useobject{currentmarker}{}%
\end{pgfscope}%
\begin{pgfscope}%
\pgfsys@transformshift{1.511762in}{2.653134in}%
\pgfsys@useobject{currentmarker}{}%
\end{pgfscope}%
\begin{pgfscope}%
\pgfsys@transformshift{1.511629in}{2.667660in}%
\pgfsys@useobject{currentmarker}{}%
\end{pgfscope}%
\begin{pgfscope}%
\pgfsys@transformshift{1.516003in}{2.685333in}%
\pgfsys@useobject{currentmarker}{}%
\end{pgfscope}%
\begin{pgfscope}%
\pgfsys@transformshift{1.512260in}{2.706632in}%
\pgfsys@useobject{currentmarker}{}%
\end{pgfscope}%
\begin{pgfscope}%
\pgfsys@transformshift{1.511411in}{2.718495in}%
\pgfsys@useobject{currentmarker}{}%
\end{pgfscope}%
\begin{pgfscope}%
\pgfsys@transformshift{1.515397in}{2.734678in}%
\pgfsys@useobject{currentmarker}{}%
\end{pgfscope}%
\begin{pgfscope}%
\pgfsys@transformshift{1.514783in}{2.743824in}%
\pgfsys@useobject{currentmarker}{}%
\end{pgfscope}%
\begin{pgfscope}%
\pgfsys@transformshift{1.513963in}{2.748799in}%
\pgfsys@useobject{currentmarker}{}%
\end{pgfscope}%
\begin{pgfscope}%
\pgfsys@transformshift{1.513519in}{2.756585in}%
\pgfsys@useobject{currentmarker}{}%
\end{pgfscope}%
\begin{pgfscope}%
\pgfsys@transformshift{1.514254in}{2.760811in}%
\pgfsys@useobject{currentmarker}{}%
\end{pgfscope}%
\begin{pgfscope}%
\pgfsys@transformshift{1.512827in}{2.768483in}%
\pgfsys@useobject{currentmarker}{}%
\end{pgfscope}%
\begin{pgfscope}%
\pgfsys@transformshift{1.513039in}{2.772770in}%
\pgfsys@useobject{currentmarker}{}%
\end{pgfscope}%
\begin{pgfscope}%
\pgfsys@transformshift{1.515750in}{2.781178in}%
\pgfsys@useobject{currentmarker}{}%
\end{pgfscope}%
\begin{pgfscope}%
\pgfsys@transformshift{1.514253in}{2.793107in}%
\pgfsys@useobject{currentmarker}{}%
\end{pgfscope}%
\begin{pgfscope}%
\pgfsys@transformshift{1.512460in}{2.799472in}%
\pgfsys@useobject{currentmarker}{}%
\end{pgfscope}%
\begin{pgfscope}%
\pgfsys@transformshift{1.515758in}{2.809946in}%
\pgfsys@useobject{currentmarker}{}%
\end{pgfscope}%
\begin{pgfscope}%
\pgfsys@transformshift{1.515793in}{2.821708in}%
\pgfsys@useobject{currentmarker}{}%
\end{pgfscope}%
\begin{pgfscope}%
\pgfsys@transformshift{1.515177in}{2.828148in}%
\pgfsys@useobject{currentmarker}{}%
\end{pgfscope}%
\begin{pgfscope}%
\pgfsys@transformshift{1.515283in}{2.837890in}%
\pgfsys@useobject{currentmarker}{}%
\end{pgfscope}%
\begin{pgfscope}%
\pgfsys@transformshift{1.516619in}{2.843079in}%
\pgfsys@useobject{currentmarker}{}%
\end{pgfscope}%
\begin{pgfscope}%
\pgfsys@transformshift{1.515380in}{2.851662in}%
\pgfsys@useobject{currentmarker}{}%
\end{pgfscope}%
\begin{pgfscope}%
\pgfsys@transformshift{1.514803in}{2.860904in}%
\pgfsys@useobject{currentmarker}{}%
\end{pgfscope}%
\begin{pgfscope}%
\pgfsys@transformshift{1.518441in}{2.871605in}%
\pgfsys@useobject{currentmarker}{}%
\end{pgfscope}%
\begin{pgfscope}%
\pgfsys@transformshift{1.516801in}{2.887827in}%
\pgfsys@useobject{currentmarker}{}%
\end{pgfscope}%
\begin{pgfscope}%
\pgfsys@transformshift{1.513697in}{2.904349in}%
\pgfsys@useobject{currentmarker}{}%
\end{pgfscope}%
\begin{pgfscope}%
\pgfsys@transformshift{1.520302in}{2.924171in}%
\pgfsys@useobject{currentmarker}{}%
\end{pgfscope}%
\begin{pgfscope}%
\pgfsys@transformshift{1.520502in}{2.946416in}%
\pgfsys@useobject{currentmarker}{}%
\end{pgfscope}%
\begin{pgfscope}%
\pgfsys@transformshift{1.518463in}{2.958480in}%
\pgfsys@useobject{currentmarker}{}%
\end{pgfscope}%
\begin{pgfscope}%
\pgfsys@transformshift{1.519214in}{2.973743in}%
\pgfsys@useobject{currentmarker}{}%
\end{pgfscope}%
\begin{pgfscope}%
\pgfsys@transformshift{1.520759in}{2.989614in}%
\pgfsys@useobject{currentmarker}{}%
\end{pgfscope}%
\begin{pgfscope}%
\pgfsys@transformshift{1.518949in}{3.007305in}%
\pgfsys@useobject{currentmarker}{}%
\end{pgfscope}%
\begin{pgfscope}%
\pgfsys@transformshift{1.519633in}{3.025660in}%
\pgfsys@useobject{currentmarker}{}%
\end{pgfscope}%
\begin{pgfscope}%
\pgfsys@transformshift{1.528149in}{3.044677in}%
\pgfsys@useobject{currentmarker}{}%
\end{pgfscope}%
\begin{pgfscope}%
\pgfsys@transformshift{1.527281in}{3.069079in}%
\pgfsys@useobject{currentmarker}{}%
\end{pgfscope}%
\begin{pgfscope}%
\pgfsys@transformshift{1.528214in}{3.082476in}%
\pgfsys@useobject{currentmarker}{}%
\end{pgfscope}%
\begin{pgfscope}%
\pgfsys@transformshift{1.535179in}{3.100275in}%
\pgfsys@useobject{currentmarker}{}%
\end{pgfscope}%
\begin{pgfscope}%
\pgfsys@transformshift{1.532379in}{3.120468in}%
\pgfsys@useobject{currentmarker}{}%
\end{pgfscope}%
\begin{pgfscope}%
\pgfsys@transformshift{1.530604in}{3.131539in}%
\pgfsys@useobject{currentmarker}{}%
\end{pgfscope}%
\begin{pgfscope}%
\pgfsys@transformshift{1.530278in}{3.146029in}%
\pgfsys@useobject{currentmarker}{}%
\end{pgfscope}%
\begin{pgfscope}%
\pgfsys@transformshift{1.531750in}{3.153864in}%
\pgfsys@useobject{currentmarker}{}%
\end{pgfscope}%
\begin{pgfscope}%
\pgfsys@transformshift{1.529620in}{3.165501in}%
\pgfsys@useobject{currentmarker}{}%
\end{pgfscope}%
\begin{pgfscope}%
\pgfsys@transformshift{1.529789in}{3.178122in}%
\pgfsys@useobject{currentmarker}{}%
\end{pgfscope}%
\begin{pgfscope}%
\pgfsys@transformshift{1.534788in}{3.194254in}%
\pgfsys@useobject{currentmarker}{}%
\end{pgfscope}%
\begin{pgfscope}%
\pgfsys@transformshift{1.533507in}{3.213227in}%
\pgfsys@useobject{currentmarker}{}%
\end{pgfscope}%
\begin{pgfscope}%
\pgfsys@transformshift{1.531794in}{3.232801in}%
\pgfsys@useobject{currentmarker}{}%
\end{pgfscope}%
\begin{pgfscope}%
\pgfsys@transformshift{1.535706in}{3.255486in}%
\pgfsys@useobject{currentmarker}{}%
\end{pgfscope}%
\begin{pgfscope}%
\pgfsys@transformshift{1.546105in}{3.277079in}%
\pgfsys@useobject{currentmarker}{}%
\end{pgfscope}%
\begin{pgfscope}%
\pgfsys@transformshift{1.555856in}{3.299974in}%
\pgfsys@useobject{currentmarker}{}%
\end{pgfscope}%
\begin{pgfscope}%
\pgfsys@transformshift{1.577063in}{3.314300in}%
\pgfsys@useobject{currentmarker}{}%
\end{pgfscope}%
\begin{pgfscope}%
\pgfsys@transformshift{1.599694in}{3.327981in}%
\pgfsys@useobject{currentmarker}{}%
\end{pgfscope}%
\begin{pgfscope}%
\pgfsys@transformshift{1.627220in}{3.332244in}%
\pgfsys@useobject{currentmarker}{}%
\end{pgfscope}%
\begin{pgfscope}%
\pgfsys@transformshift{1.655211in}{3.339614in}%
\pgfsys@useobject{currentmarker}{}%
\end{pgfscope}%
\begin{pgfscope}%
\pgfsys@transformshift{1.684817in}{3.338337in}%
\pgfsys@useobject{currentmarker}{}%
\end{pgfscope}%
\begin{pgfscope}%
\pgfsys@transformshift{1.700785in}{3.341603in}%
\pgfsys@useobject{currentmarker}{}%
\end{pgfscope}%
\begin{pgfscope}%
\pgfsys@transformshift{1.709749in}{3.341553in}%
\pgfsys@useobject{currentmarker}{}%
\end{pgfscope}%
\begin{pgfscope}%
\pgfsys@transformshift{1.719256in}{3.342401in}%
\pgfsys@useobject{currentmarker}{}%
\end{pgfscope}%
\begin{pgfscope}%
\pgfsys@transformshift{1.731188in}{3.340415in}%
\pgfsys@useobject{currentmarker}{}%
\end{pgfscope}%
\begin{pgfscope}%
\pgfsys@transformshift{1.744701in}{3.341290in}%
\pgfsys@useobject{currentmarker}{}%
\end{pgfscope}%
\begin{pgfscope}%
\pgfsys@transformshift{1.759186in}{3.339391in}%
\pgfsys@useobject{currentmarker}{}%
\end{pgfscope}%
\begin{pgfscope}%
\pgfsys@transformshift{1.767207in}{3.338925in}%
\pgfsys@useobject{currentmarker}{}%
\end{pgfscope}%
\begin{pgfscope}%
\pgfsys@transformshift{1.771619in}{3.338676in}%
\pgfsys@useobject{currentmarker}{}%
\end{pgfscope}%
\begin{pgfscope}%
\pgfsys@transformshift{1.778303in}{3.337700in}%
\pgfsys@useobject{currentmarker}{}%
\end{pgfscope}%
\begin{pgfscope}%
\pgfsys@transformshift{1.781995in}{3.337279in}%
\pgfsys@useobject{currentmarker}{}%
\end{pgfscope}%
\begin{pgfscope}%
\pgfsys@transformshift{1.784008in}{3.336928in}%
\pgfsys@useobject{currentmarker}{}%
\end{pgfscope}%
\begin{pgfscope}%
\pgfsys@transformshift{1.787056in}{3.336605in}%
\pgfsys@useobject{currentmarker}{}%
\end{pgfscope}%
\begin{pgfscope}%
\pgfsys@transformshift{1.790672in}{3.336711in}%
\pgfsys@useobject{currentmarker}{}%
\end{pgfscope}%
\begin{pgfscope}%
\pgfsys@transformshift{1.799518in}{3.336161in}%
\pgfsys@useobject{currentmarker}{}%
\end{pgfscope}%
\begin{pgfscope}%
\pgfsys@transformshift{1.810353in}{3.337143in}%
\pgfsys@useobject{currentmarker}{}%
\end{pgfscope}%
\begin{pgfscope}%
\pgfsys@transformshift{1.822031in}{3.335871in}%
\pgfsys@useobject{currentmarker}{}%
\end{pgfscope}%
\begin{pgfscope}%
\pgfsys@transformshift{1.828492in}{3.335808in}%
\pgfsys@useobject{currentmarker}{}%
\end{pgfscope}%
\begin{pgfscope}%
\pgfsys@transformshift{1.837087in}{3.338524in}%
\pgfsys@useobject{currentmarker}{}%
\end{pgfscope}%
\begin{pgfscope}%
\pgfsys@transformshift{1.842032in}{3.338869in}%
\pgfsys@useobject{currentmarker}{}%
\end{pgfscope}%
\begin{pgfscope}%
\pgfsys@transformshift{1.848102in}{3.338953in}%
\pgfsys@useobject{currentmarker}{}%
\end{pgfscope}%
\begin{pgfscope}%
\pgfsys@transformshift{1.855177in}{3.338529in}%
\pgfsys@useobject{currentmarker}{}%
\end{pgfscope}%
\begin{pgfscope}%
\pgfsys@transformshift{1.859033in}{3.339106in}%
\pgfsys@useobject{currentmarker}{}%
\end{pgfscope}%
\begin{pgfscope}%
\pgfsys@transformshift{1.861175in}{3.339187in}%
\pgfsys@useobject{currentmarker}{}%
\end{pgfscope}%
\begin{pgfscope}%
\pgfsys@transformshift{1.864715in}{3.338966in}%
\pgfsys@useobject{currentmarker}{}%
\end{pgfscope}%
\begin{pgfscope}%
\pgfsys@transformshift{1.866664in}{3.338866in}%
\pgfsys@useobject{currentmarker}{}%
\end{pgfscope}%
\begin{pgfscope}%
\pgfsys@transformshift{1.869722in}{3.338957in}%
\pgfsys@useobject{currentmarker}{}%
\end{pgfscope}%
\begin{pgfscope}%
\pgfsys@transformshift{1.871404in}{3.338997in}%
\pgfsys@useobject{currentmarker}{}%
\end{pgfscope}%
\begin{pgfscope}%
\pgfsys@transformshift{1.874760in}{3.338830in}%
\pgfsys@useobject{currentmarker}{}%
\end{pgfscope}%
\begin{pgfscope}%
\pgfsys@transformshift{1.876606in}{3.338736in}%
\pgfsys@useobject{currentmarker}{}%
\end{pgfscope}%
\begin{pgfscope}%
\pgfsys@transformshift{1.877602in}{3.338536in}%
\pgfsys@useobject{currentmarker}{}%
\end{pgfscope}%
\begin{pgfscope}%
\pgfsys@transformshift{1.880103in}{3.338436in}%
\pgfsys@useobject{currentmarker}{}%
\end{pgfscope}%
\begin{pgfscope}%
\pgfsys@transformshift{1.886606in}{3.337913in}%
\pgfsys@useobject{currentmarker}{}%
\end{pgfscope}%
\begin{pgfscope}%
\pgfsys@transformshift{1.893641in}{3.337873in}%
\pgfsys@useobject{currentmarker}{}%
\end{pgfscope}%
\begin{pgfscope}%
\pgfsys@transformshift{1.897431in}{3.337094in}%
\pgfsys@useobject{currentmarker}{}%
\end{pgfscope}%
\begin{pgfscope}%
\pgfsys@transformshift{1.899542in}{3.336826in}%
\pgfsys@useobject{currentmarker}{}%
\end{pgfscope}%
\begin{pgfscope}%
\pgfsys@transformshift{1.902127in}{3.336287in}%
\pgfsys@useobject{currentmarker}{}%
\end{pgfscope}%
\begin{pgfscope}%
\pgfsys@transformshift{1.907273in}{3.336220in}%
\pgfsys@useobject{currentmarker}{}%
\end{pgfscope}%
\begin{pgfscope}%
\pgfsys@transformshift{1.915526in}{3.336086in}%
\pgfsys@useobject{currentmarker}{}%
\end{pgfscope}%
\begin{pgfscope}%
\pgfsys@transformshift{1.925849in}{3.338031in}%
\pgfsys@useobject{currentmarker}{}%
\end{pgfscope}%
\begin{pgfscope}%
\pgfsys@transformshift{1.937387in}{3.337404in}%
\pgfsys@useobject{currentmarker}{}%
\end{pgfscope}%
\begin{pgfscope}%
\pgfsys@transformshift{1.949796in}{3.339791in}%
\pgfsys@useobject{currentmarker}{}%
\end{pgfscope}%
\begin{pgfscope}%
\pgfsys@transformshift{1.964680in}{3.342682in}%
\pgfsys@useobject{currentmarker}{}%
\end{pgfscope}%
\begin{pgfscope}%
\pgfsys@transformshift{1.981264in}{3.344712in}%
\pgfsys@useobject{currentmarker}{}%
\end{pgfscope}%
\begin{pgfscope}%
\pgfsys@transformshift{1.990442in}{3.344263in}%
\pgfsys@useobject{currentmarker}{}%
\end{pgfscope}%
\begin{pgfscope}%
\pgfsys@transformshift{1.995481in}{3.343872in}%
\pgfsys@useobject{currentmarker}{}%
\end{pgfscope}%
\begin{pgfscope}%
\pgfsys@transformshift{2.001385in}{3.343254in}%
\pgfsys@useobject{currentmarker}{}%
\end{pgfscope}%
\begin{pgfscope}%
\pgfsys@transformshift{2.004563in}{3.344001in}%
\pgfsys@useobject{currentmarker}{}%
\end{pgfscope}%
\begin{pgfscope}%
\pgfsys@transformshift{2.006349in}{3.343809in}%
\pgfsys@useobject{currentmarker}{}%
\end{pgfscope}%
\begin{pgfscope}%
\pgfsys@transformshift{2.008648in}{3.343803in}%
\pgfsys@useobject{currentmarker}{}%
\end{pgfscope}%
\begin{pgfscope}%
\pgfsys@transformshift{2.009896in}{3.343605in}%
\pgfsys@useobject{currentmarker}{}%
\end{pgfscope}%
\begin{pgfscope}%
\pgfsys@transformshift{2.011795in}{3.343683in}%
\pgfsys@useobject{currentmarker}{}%
\end{pgfscope}%
\begin{pgfscope}%
\pgfsys@transformshift{2.012838in}{3.343754in}%
\pgfsys@useobject{currentmarker}{}%
\end{pgfscope}%
\begin{pgfscope}%
\pgfsys@transformshift{2.014715in}{3.343530in}%
\pgfsys@useobject{currentmarker}{}%
\end{pgfscope}%
\begin{pgfscope}%
\pgfsys@transformshift{2.017071in}{3.343505in}%
\pgfsys@useobject{currentmarker}{}%
\end{pgfscope}%
\begin{pgfscope}%
\pgfsys@transformshift{2.022553in}{3.343258in}%
\pgfsys@useobject{currentmarker}{}%
\end{pgfscope}%
\begin{pgfscope}%
\pgfsys@transformshift{2.025555in}{3.343562in}%
\pgfsys@useobject{currentmarker}{}%
\end{pgfscope}%
\begin{pgfscope}%
\pgfsys@transformshift{2.027214in}{3.343508in}%
\pgfsys@useobject{currentmarker}{}%
\end{pgfscope}%
\begin{pgfscope}%
\pgfsys@transformshift{2.030368in}{3.343427in}%
\pgfsys@useobject{currentmarker}{}%
\end{pgfscope}%
\begin{pgfscope}%
\pgfsys@transformshift{2.034504in}{3.343374in}%
\pgfsys@useobject{currentmarker}{}%
\end{pgfscope}%
\begin{pgfscope}%
\pgfsys@transformshift{2.036772in}{3.343190in}%
\pgfsys@useobject{currentmarker}{}%
\end{pgfscope}%
\begin{pgfscope}%
\pgfsys@transformshift{2.037994in}{3.342919in}%
\pgfsys@useobject{currentmarker}{}%
\end{pgfscope}%
\begin{pgfscope}%
\pgfsys@transformshift{2.040252in}{3.342925in}%
\pgfsys@useobject{currentmarker}{}%
\end{pgfscope}%
\begin{pgfscope}%
\pgfsys@transformshift{2.043421in}{3.342820in}%
\pgfsys@useobject{currentmarker}{}%
\end{pgfscope}%
\begin{pgfscope}%
\pgfsys@transformshift{2.045164in}{3.342893in}%
\pgfsys@useobject{currentmarker}{}%
\end{pgfscope}%
\begin{pgfscope}%
\pgfsys@transformshift{2.046123in}{3.342880in}%
\pgfsys@useobject{currentmarker}{}%
\end{pgfscope}%
\begin{pgfscope}%
\pgfsys@transformshift{2.048487in}{3.342765in}%
\pgfsys@useobject{currentmarker}{}%
\end{pgfscope}%
\begin{pgfscope}%
\pgfsys@transformshift{2.051464in}{3.342562in}%
\pgfsys@useobject{currentmarker}{}%
\end{pgfscope}%
\begin{pgfscope}%
\pgfsys@transformshift{2.055960in}{3.341693in}%
\pgfsys@useobject{currentmarker}{}%
\end{pgfscope}%
\begin{pgfscope}%
\pgfsys@transformshift{2.062814in}{3.340918in}%
\pgfsys@useobject{currentmarker}{}%
\end{pgfscope}%
\begin{pgfscope}%
\pgfsys@transformshift{2.066585in}{3.340502in}%
\pgfsys@useobject{currentmarker}{}%
\end{pgfscope}%
\begin{pgfscope}%
\pgfsys@transformshift{2.071503in}{3.340189in}%
\pgfsys@useobject{currentmarker}{}%
\end{pgfscope}%
\begin{pgfscope}%
\pgfsys@transformshift{2.074194in}{3.339865in}%
\pgfsys@useobject{currentmarker}{}%
\end{pgfscope}%
\begin{pgfscope}%
\pgfsys@transformshift{2.078821in}{3.340275in}%
\pgfsys@useobject{currentmarker}{}%
\end{pgfscope}%
\begin{pgfscope}%
\pgfsys@transformshift{2.084171in}{3.338769in}%
\pgfsys@useobject{currentmarker}{}%
\end{pgfscope}%
\begin{pgfscope}%
\pgfsys@transformshift{2.090506in}{3.338813in}%
\pgfsys@useobject{currentmarker}{}%
\end{pgfscope}%
\begin{pgfscope}%
\pgfsys@transformshift{2.098365in}{3.339030in}%
\pgfsys@useobject{currentmarker}{}%
\end{pgfscope}%
\begin{pgfscope}%
\pgfsys@transformshift{2.107855in}{3.339872in}%
\pgfsys@useobject{currentmarker}{}%
\end{pgfscope}%
\begin{pgfscope}%
\pgfsys@transformshift{2.113074in}{3.339406in}%
\pgfsys@useobject{currentmarker}{}%
\end{pgfscope}%
\begin{pgfscope}%
\pgfsys@transformshift{2.118771in}{3.338815in}%
\pgfsys@useobject{currentmarker}{}%
\end{pgfscope}%
\begin{pgfscope}%
\pgfsys@transformshift{2.121916in}{3.338634in}%
\pgfsys@useobject{currentmarker}{}%
\end{pgfscope}%
\begin{pgfscope}%
\pgfsys@transformshift{2.126232in}{3.338679in}%
\pgfsys@useobject{currentmarker}{}%
\end{pgfscope}%
\begin{pgfscope}%
\pgfsys@transformshift{2.128548in}{3.338159in}%
\pgfsys@useobject{currentmarker}{}%
\end{pgfscope}%
\begin{pgfscope}%
\pgfsys@transformshift{2.129845in}{3.338011in}%
\pgfsys@useobject{currentmarker}{}%
\end{pgfscope}%
\begin{pgfscope}%
\pgfsys@transformshift{2.133080in}{3.338133in}%
\pgfsys@useobject{currentmarker}{}%
\end{pgfscope}%
\begin{pgfscope}%
\pgfsys@transformshift{2.137318in}{3.337327in}%
\pgfsys@useobject{currentmarker}{}%
\end{pgfscope}%
\begin{pgfscope}%
\pgfsys@transformshift{2.139654in}{3.336909in}%
\pgfsys@useobject{currentmarker}{}%
\end{pgfscope}%
\begin{pgfscope}%
\pgfsys@transformshift{2.142523in}{3.336444in}%
\pgfsys@useobject{currentmarker}{}%
\end{pgfscope}%
\begin{pgfscope}%
\pgfsys@transformshift{2.145977in}{3.336352in}%
\pgfsys@useobject{currentmarker}{}%
\end{pgfscope}%
\begin{pgfscope}%
\pgfsys@transformshift{2.147861in}{3.336099in}%
\pgfsys@useobject{currentmarker}{}%
\end{pgfscope}%
\begin{pgfscope}%
\pgfsys@transformshift{2.148894in}{3.335944in}%
\pgfsys@useobject{currentmarker}{}%
\end{pgfscope}%
\begin{pgfscope}%
\pgfsys@transformshift{2.150897in}{3.336130in}%
\pgfsys@useobject{currentmarker}{}%
\end{pgfscope}%
\begin{pgfscope}%
\pgfsys@transformshift{2.154600in}{3.336286in}%
\pgfsys@useobject{currentmarker}{}%
\end{pgfscope}%
\begin{pgfscope}%
\pgfsys@transformshift{2.156637in}{3.336329in}%
\pgfsys@useobject{currentmarker}{}%
\end{pgfscope}%
\begin{pgfscope}%
\pgfsys@transformshift{2.157735in}{3.336105in}%
\pgfsys@useobject{currentmarker}{}%
\end{pgfscope}%
\begin{pgfscope}%
\pgfsys@transformshift{2.158350in}{3.336051in}%
\pgfsys@useobject{currentmarker}{}%
\end{pgfscope}%
\begin{pgfscope}%
\pgfsys@transformshift{2.160507in}{3.335961in}%
\pgfsys@useobject{currentmarker}{}%
\end{pgfscope}%
\begin{pgfscope}%
\pgfsys@transformshift{2.161689in}{3.336075in}%
\pgfsys@useobject{currentmarker}{}%
\end{pgfscope}%
\begin{pgfscope}%
\pgfsys@transformshift{2.163877in}{3.335806in}%
\pgfsys@useobject{currentmarker}{}%
\end{pgfscope}%
\begin{pgfscope}%
\pgfsys@transformshift{2.165087in}{3.335888in}%
\pgfsys@useobject{currentmarker}{}%
\end{pgfscope}%
\begin{pgfscope}%
\pgfsys@transformshift{2.168230in}{3.336328in}%
\pgfsys@useobject{currentmarker}{}%
\end{pgfscope}%
\begin{pgfscope}%
\pgfsys@transformshift{2.169957in}{3.336079in}%
\pgfsys@useobject{currentmarker}{}%
\end{pgfscope}%
\begin{pgfscope}%
\pgfsys@transformshift{2.170913in}{3.335988in}%
\pgfsys@useobject{currentmarker}{}%
\end{pgfscope}%
\begin{pgfscope}%
\pgfsys@transformshift{2.172655in}{3.335968in}%
\pgfsys@useobject{currentmarker}{}%
\end{pgfscope}%
\begin{pgfscope}%
\pgfsys@transformshift{2.173614in}{3.335974in}%
\pgfsys@useobject{currentmarker}{}%
\end{pgfscope}%
\begin{pgfscope}%
\pgfsys@transformshift{2.175836in}{3.335458in}%
\pgfsys@useobject{currentmarker}{}%
\end{pgfscope}%
\begin{pgfscope}%
\pgfsys@transformshift{2.177079in}{3.335295in}%
\pgfsys@useobject{currentmarker}{}%
\end{pgfscope}%
\begin{pgfscope}%
\pgfsys@transformshift{2.177749in}{3.335128in}%
\pgfsys@useobject{currentmarker}{}%
\end{pgfscope}%
\begin{pgfscope}%
\pgfsys@transformshift{2.178122in}{3.335061in}%
\pgfsys@useobject{currentmarker}{}%
\end{pgfscope}%
\begin{pgfscope}%
\pgfsys@transformshift{2.178325in}{3.335012in}%
\pgfsys@useobject{currentmarker}{}%
\end{pgfscope}%
\begin{pgfscope}%
\pgfsys@transformshift{2.179195in}{3.334965in}%
\pgfsys@useobject{currentmarker}{}%
\end{pgfscope}%
\begin{pgfscope}%
\pgfsys@transformshift{2.179672in}{3.334918in}%
\pgfsys@useobject{currentmarker}{}%
\end{pgfscope}%
\begin{pgfscope}%
\pgfsys@transformshift{2.179934in}{3.334888in}%
\pgfsys@useobject{currentmarker}{}%
\end{pgfscope}%
\begin{pgfscope}%
\pgfsys@transformshift{2.180704in}{3.334844in}%
\pgfsys@useobject{currentmarker}{}%
\end{pgfscope}%
\begin{pgfscope}%
\pgfsys@transformshift{2.182424in}{3.334793in}%
\pgfsys@useobject{currentmarker}{}%
\end{pgfscope}%
\begin{pgfscope}%
\pgfsys@transformshift{2.184706in}{3.334351in}%
\pgfsys@useobject{currentmarker}{}%
\end{pgfscope}%
\begin{pgfscope}%
\pgfsys@transformshift{2.187732in}{3.333943in}%
\pgfsys@useobject{currentmarker}{}%
\end{pgfscope}%
\begin{pgfscope}%
\pgfsys@transformshift{2.191238in}{3.333635in}%
\pgfsys@useobject{currentmarker}{}%
\end{pgfscope}%
\begin{pgfscope}%
\pgfsys@transformshift{2.193172in}{3.333578in}%
\pgfsys@useobject{currentmarker}{}%
\end{pgfscope}%
\begin{pgfscope}%
\pgfsys@transformshift{2.196281in}{3.333130in}%
\pgfsys@useobject{currentmarker}{}%
\end{pgfscope}%
\begin{pgfscope}%
\pgfsys@transformshift{2.198001in}{3.332966in}%
\pgfsys@useobject{currentmarker}{}%
\end{pgfscope}%
\begin{pgfscope}%
\pgfsys@transformshift{2.198951in}{3.332966in}%
\pgfsys@useobject{currentmarker}{}%
\end{pgfscope}%
\begin{pgfscope}%
\pgfsys@transformshift{2.200856in}{3.333131in}%
\pgfsys@useobject{currentmarker}{}%
\end{pgfscope}%
\begin{pgfscope}%
\pgfsys@transformshift{2.201895in}{3.332963in}%
\pgfsys@useobject{currentmarker}{}%
\end{pgfscope}%
\begin{pgfscope}%
\pgfsys@transformshift{2.202473in}{3.332956in}%
\pgfsys@useobject{currentmarker}{}%
\end{pgfscope}%
\begin{pgfscope}%
\pgfsys@transformshift{2.202789in}{3.332921in}%
\pgfsys@useobject{currentmarker}{}%
\end{pgfscope}%
\begin{pgfscope}%
\pgfsys@transformshift{2.202963in}{3.332941in}%
\pgfsys@useobject{currentmarker}{}%
\end{pgfscope}%
\begin{pgfscope}%
\pgfsys@transformshift{2.203059in}{3.332932in}%
\pgfsys@useobject{currentmarker}{}%
\end{pgfscope}%
\begin{pgfscope}%
\pgfsys@transformshift{2.204759in}{3.332878in}%
\pgfsys@useobject{currentmarker}{}%
\end{pgfscope}%
\begin{pgfscope}%
\pgfsys@transformshift{2.208303in}{3.332738in}%
\pgfsys@useobject{currentmarker}{}%
\end{pgfscope}%
\begin{pgfscope}%
\pgfsys@transformshift{2.210252in}{3.332814in}%
\pgfsys@useobject{currentmarker}{}%
\end{pgfscope}%
\begin{pgfscope}%
\pgfsys@transformshift{2.211322in}{3.332731in}%
\pgfsys@useobject{currentmarker}{}%
\end{pgfscope}%
\begin{pgfscope}%
\pgfsys@transformshift{2.212977in}{3.332761in}%
\pgfsys@useobject{currentmarker}{}%
\end{pgfscope}%
\begin{pgfscope}%
\pgfsys@transformshift{2.215845in}{3.332793in}%
\pgfsys@useobject{currentmarker}{}%
\end{pgfscope}%
\begin{pgfscope}%
\pgfsys@transformshift{2.217422in}{3.332776in}%
\pgfsys@useobject{currentmarker}{}%
\end{pgfscope}%
\begin{pgfscope}%
\pgfsys@transformshift{2.218288in}{3.332731in}%
\pgfsys@useobject{currentmarker}{}%
\end{pgfscope}%
\begin{pgfscope}%
\pgfsys@transformshift{2.218764in}{3.332700in}%
\pgfsys@useobject{currentmarker}{}%
\end{pgfscope}%
\begin{pgfscope}%
\pgfsys@transformshift{2.221148in}{3.332623in}%
\pgfsys@useobject{currentmarker}{}%
\end{pgfscope}%
\begin{pgfscope}%
\pgfsys@transformshift{2.222458in}{3.332577in}%
\pgfsys@useobject{currentmarker}{}%
\end{pgfscope}%
\begin{pgfscope}%
\pgfsys@transformshift{2.223175in}{3.332493in}%
\pgfsys@useobject{currentmarker}{}%
\end{pgfscope}%
\begin{pgfscope}%
\pgfsys@transformshift{2.223570in}{3.332460in}%
\pgfsys@useobject{currentmarker}{}%
\end{pgfscope}%
\begin{pgfscope}%
\pgfsys@transformshift{2.223786in}{3.332431in}%
\pgfsys@useobject{currentmarker}{}%
\end{pgfscope}%
\begin{pgfscope}%
\pgfsys@transformshift{2.223906in}{3.332421in}%
\pgfsys@useobject{currentmarker}{}%
\end{pgfscope}%
\begin{pgfscope}%
\pgfsys@transformshift{2.223970in}{3.332405in}%
\pgfsys@useobject{currentmarker}{}%
\end{pgfscope}%
\begin{pgfscope}%
\pgfsys@transformshift{2.224645in}{3.332495in}%
\pgfsys@useobject{currentmarker}{}%
\end{pgfscope}%
\begin{pgfscope}%
\pgfsys@transformshift{2.225019in}{3.332470in}%
\pgfsys@useobject{currentmarker}{}%
\end{pgfscope}%
\begin{pgfscope}%
\pgfsys@transformshift{2.227227in}{3.331456in}%
\pgfsys@useobject{currentmarker}{}%
\end{pgfscope}%
\begin{pgfscope}%
\pgfsys@transformshift{2.228561in}{3.331374in}%
\pgfsys@useobject{currentmarker}{}%
\end{pgfscope}%
\begin{pgfscope}%
\pgfsys@transformshift{2.231898in}{3.331110in}%
\pgfsys@useobject{currentmarker}{}%
\end{pgfscope}%
\begin{pgfscope}%
\pgfsys@transformshift{2.237987in}{3.330987in}%
\pgfsys@useobject{currentmarker}{}%
\end{pgfscope}%
\begin{pgfscope}%
\pgfsys@transformshift{2.247621in}{3.330058in}%
\pgfsys@useobject{currentmarker}{}%
\end{pgfscope}%
\begin{pgfscope}%
\pgfsys@transformshift{2.258596in}{3.333559in}%
\pgfsys@useobject{currentmarker}{}%
\end{pgfscope}%
\begin{pgfscope}%
\pgfsys@transformshift{2.264920in}{3.333956in}%
\pgfsys@useobject{currentmarker}{}%
\end{pgfscope}%
\begin{pgfscope}%
\pgfsys@transformshift{2.277501in}{3.334454in}%
\pgfsys@useobject{currentmarker}{}%
\end{pgfscope}%
\begin{pgfscope}%
\pgfsys@transformshift{2.292868in}{3.331809in}%
\pgfsys@useobject{currentmarker}{}%
\end{pgfscope}%
\begin{pgfscope}%
\pgfsys@transformshift{2.309001in}{3.330195in}%
\pgfsys@useobject{currentmarker}{}%
\end{pgfscope}%
\begin{pgfscope}%
\pgfsys@transformshift{2.330145in}{3.324235in}%
\pgfsys@useobject{currentmarker}{}%
\end{pgfscope}%
\begin{pgfscope}%
\pgfsys@transformshift{2.353436in}{3.325724in}%
\pgfsys@useobject{currentmarker}{}%
\end{pgfscope}%
\begin{pgfscope}%
\pgfsys@transformshift{2.366235in}{3.326689in}%
\pgfsys@useobject{currentmarker}{}%
\end{pgfscope}%
\begin{pgfscope}%
\pgfsys@transformshift{2.383471in}{3.324434in}%
\pgfsys@useobject{currentmarker}{}%
\end{pgfscope}%
\begin{pgfscope}%
\pgfsys@transformshift{2.402965in}{3.322410in}%
\pgfsys@useobject{currentmarker}{}%
\end{pgfscope}%
\begin{pgfscope}%
\pgfsys@transformshift{2.426868in}{3.323444in}%
\pgfsys@useobject{currentmarker}{}%
\end{pgfscope}%
\begin{pgfscope}%
\pgfsys@transformshift{2.450070in}{3.315684in}%
\pgfsys@useobject{currentmarker}{}%
\end{pgfscope}%
\begin{pgfscope}%
\pgfsys@transformshift{2.473984in}{3.324922in}%
\pgfsys@useobject{currentmarker}{}%
\end{pgfscope}%
\begin{pgfscope}%
\pgfsys@transformshift{2.487864in}{3.327406in}%
\pgfsys@useobject{currentmarker}{}%
\end{pgfscope}%
\begin{pgfscope}%
\pgfsys@transformshift{2.505968in}{3.326655in}%
\pgfsys@useobject{currentmarker}{}%
\end{pgfscope}%
\begin{pgfscope}%
\pgfsys@transformshift{2.529826in}{3.323842in}%
\pgfsys@useobject{currentmarker}{}%
\end{pgfscope}%
\begin{pgfscope}%
\pgfsys@transformshift{2.553727in}{3.318433in}%
\pgfsys@useobject{currentmarker}{}%
\end{pgfscope}%
\begin{pgfscope}%
\pgfsys@transformshift{2.579054in}{3.323573in}%
\pgfsys@useobject{currentmarker}{}%
\end{pgfscope}%
\begin{pgfscope}%
\pgfsys@transformshift{2.606882in}{3.327607in}%
\pgfsys@useobject{currentmarker}{}%
\end{pgfscope}%
\begin{pgfscope}%
\pgfsys@transformshift{2.636772in}{3.338027in}%
\pgfsys@useobject{currentmarker}{}%
\end{pgfscope}%
\begin{pgfscope}%
\pgfsys@transformshift{2.653910in}{3.341091in}%
\pgfsys@useobject{currentmarker}{}%
\end{pgfscope}%
\begin{pgfscope}%
\pgfsys@transformshift{2.663405in}{3.342337in}%
\pgfsys@useobject{currentmarker}{}%
\end{pgfscope}%
\begin{pgfscope}%
\pgfsys@transformshift{2.668572in}{3.343352in}%
\pgfsys@useobject{currentmarker}{}%
\end{pgfscope}%
\begin{pgfscope}%
\pgfsys@transformshift{2.676243in}{3.341955in}%
\pgfsys@useobject{currentmarker}{}%
\end{pgfscope}%
\begin{pgfscope}%
\pgfsys@transformshift{2.686389in}{3.343898in}%
\pgfsys@useobject{currentmarker}{}%
\end{pgfscope}%
\begin{pgfscope}%
\pgfsys@transformshift{2.698928in}{3.341725in}%
\pgfsys@useobject{currentmarker}{}%
\end{pgfscope}%
\begin{pgfscope}%
\pgfsys@transformshift{2.705399in}{3.339057in}%
\pgfsys@useobject{currentmarker}{}%
\end{pgfscope}%
\begin{pgfscope}%
\pgfsys@transformshift{2.714014in}{3.339327in}%
\pgfsys@useobject{currentmarker}{}%
\end{pgfscope}%
\begin{pgfscope}%
\pgfsys@transformshift{2.718558in}{3.337977in}%
\pgfsys@useobject{currentmarker}{}%
\end{pgfscope}%
\begin{pgfscope}%
\pgfsys@transformshift{2.721110in}{3.337442in}%
\pgfsys@useobject{currentmarker}{}%
\end{pgfscope}%
\begin{pgfscope}%
\pgfsys@transformshift{2.730174in}{3.335832in}%
\pgfsys@useobject{currentmarker}{}%
\end{pgfscope}%
\begin{pgfscope}%
\pgfsys@transformshift{2.739973in}{3.338638in}%
\pgfsys@useobject{currentmarker}{}%
\end{pgfscope}%
\begin{pgfscope}%
\pgfsys@transformshift{2.750555in}{3.339879in}%
\pgfsys@useobject{currentmarker}{}%
\end{pgfscope}%
\begin{pgfscope}%
\pgfsys@transformshift{2.762902in}{3.337038in}%
\pgfsys@useobject{currentmarker}{}%
\end{pgfscope}%
\begin{pgfscope}%
\pgfsys@transformshift{2.777608in}{3.332841in}%
\pgfsys@useobject{currentmarker}{}%
\end{pgfscope}%
\begin{pgfscope}%
\pgfsys@transformshift{2.795472in}{3.328743in}%
\pgfsys@useobject{currentmarker}{}%
\end{pgfscope}%
\begin{pgfscope}%
\pgfsys@transformshift{2.805395in}{3.330521in}%
\pgfsys@useobject{currentmarker}{}%
\end{pgfscope}%
\begin{pgfscope}%
\pgfsys@transformshift{2.810894in}{3.331228in}%
\pgfsys@useobject{currentmarker}{}%
\end{pgfscope}%
\begin{pgfscope}%
\pgfsys@transformshift{2.818818in}{3.336414in}%
\pgfsys@useobject{currentmarker}{}%
\end{pgfscope}%
\begin{pgfscope}%
\pgfsys@transformshift{2.823559in}{3.334255in}%
\pgfsys@useobject{currentmarker}{}%
\end{pgfscope}%
\begin{pgfscope}%
\pgfsys@transformshift{2.829384in}{3.333904in}%
\pgfsys@useobject{currentmarker}{}%
\end{pgfscope}%
\begin{pgfscope}%
\pgfsys@transformshift{2.835990in}{3.334319in}%
\pgfsys@useobject{currentmarker}{}%
\end{pgfscope}%
\begin{pgfscope}%
\pgfsys@transformshift{2.844216in}{3.333973in}%
\pgfsys@useobject{currentmarker}{}%
\end{pgfscope}%
\begin{pgfscope}%
\pgfsys@transformshift{2.853839in}{3.336038in}%
\pgfsys@useobject{currentmarker}{}%
\end{pgfscope}%
\begin{pgfscope}%
\pgfsys@transformshift{2.867556in}{3.337588in}%
\pgfsys@useobject{currentmarker}{}%
\end{pgfscope}%
\begin{pgfscope}%
\pgfsys@transformshift{2.874977in}{3.335982in}%
\pgfsys@useobject{currentmarker}{}%
\end{pgfscope}%
\begin{pgfscope}%
\pgfsys@transformshift{2.879110in}{3.336578in}%
\pgfsys@useobject{currentmarker}{}%
\end{pgfscope}%
\begin{pgfscope}%
\pgfsys@transformshift{2.884422in}{3.335417in}%
\pgfsys@useobject{currentmarker}{}%
\end{pgfscope}%
\begin{pgfscope}%
\pgfsys@transformshift{2.892882in}{3.339427in}%
\pgfsys@useobject{currentmarker}{}%
\end{pgfscope}%
\begin{pgfscope}%
\pgfsys@transformshift{2.908090in}{3.344203in}%
\pgfsys@useobject{currentmarker}{}%
\end{pgfscope}%
\begin{pgfscope}%
\pgfsys@transformshift{2.925997in}{3.349267in}%
\pgfsys@useobject{currentmarker}{}%
\end{pgfscope}%
\begin{pgfscope}%
\pgfsys@transformshift{2.942968in}{3.364644in}%
\pgfsys@useobject{currentmarker}{}%
\end{pgfscope}%
\begin{pgfscope}%
\pgfsys@transformshift{2.968955in}{3.365270in}%
\pgfsys@useobject{currentmarker}{}%
\end{pgfscope}%
\begin{pgfscope}%
\pgfsys@transformshift{2.997865in}{3.362661in}%
\pgfsys@useobject{currentmarker}{}%
\end{pgfscope}%
\begin{pgfscope}%
\pgfsys@transformshift{3.035151in}{3.373634in}%
\pgfsys@useobject{currentmarker}{}%
\end{pgfscope}%
\begin{pgfscope}%
\pgfsys@transformshift{3.046629in}{3.355599in}%
\pgfsys@useobject{currentmarker}{}%
\end{pgfscope}%
\begin{pgfscope}%
\pgfsys@transformshift{3.068689in}{3.342325in}%
\pgfsys@useobject{currentmarker}{}%
\end{pgfscope}%
\begin{pgfscope}%
\pgfsys@transformshift{3.095247in}{3.352567in}%
\pgfsys@useobject{currentmarker}{}%
\end{pgfscope}%
\begin{pgfscope}%
\pgfsys@transformshift{3.128334in}{3.352791in}%
\pgfsys@useobject{currentmarker}{}%
\end{pgfscope}%
\begin{pgfscope}%
\pgfsys@transformshift{3.169740in}{3.355339in}%
\pgfsys@useobject{currentmarker}{}%
\end{pgfscope}%
\begin{pgfscope}%
\pgfsys@transformshift{3.212975in}{3.356329in}%
\pgfsys@useobject{currentmarker}{}%
\end{pgfscope}%
\begin{pgfscope}%
\pgfsys@transformshift{3.256819in}{3.351287in}%
\pgfsys@useobject{currentmarker}{}%
\end{pgfscope}%
\begin{pgfscope}%
\pgfsys@transformshift{3.304043in}{3.340448in}%
\pgfsys@useobject{currentmarker}{}%
\end{pgfscope}%
\begin{pgfscope}%
\pgfsys@transformshift{3.351505in}{3.325535in}%
\pgfsys@useobject{currentmarker}{}%
\end{pgfscope}%
\begin{pgfscope}%
\pgfsys@transformshift{3.399884in}{3.300482in}%
\pgfsys@useobject{currentmarker}{}%
\end{pgfscope}%
\begin{pgfscope}%
\pgfsys@transformshift{3.454386in}{3.287534in}%
\pgfsys@useobject{currentmarker}{}%
\end{pgfscope}%
\begin{pgfscope}%
\pgfsys@transformshift{3.506702in}{3.312483in}%
\pgfsys@useobject{currentmarker}{}%
\end{pgfscope}%
\begin{pgfscope}%
\pgfsys@transformshift{3.562363in}{3.332160in}%
\pgfsys@useobject{currentmarker}{}%
\end{pgfscope}%
\begin{pgfscope}%
\pgfsys@transformshift{3.618516in}{3.354763in}%
\pgfsys@useobject{currentmarker}{}%
\end{pgfscope}%
\begin{pgfscope}%
\pgfsys@transformshift{3.682195in}{3.351862in}%
\pgfsys@useobject{currentmarker}{}%
\end{pgfscope}%
\begin{pgfscope}%
\pgfsys@transformshift{3.746285in}{3.344104in}%
\pgfsys@useobject{currentmarker}{}%
\end{pgfscope}%
\begin{pgfscope}%
\pgfsys@transformshift{3.781695in}{3.341489in}%
\pgfsys@useobject{currentmarker}{}%
\end{pgfscope}%
\begin{pgfscope}%
\pgfsys@transformshift{3.818177in}{3.339847in}%
\pgfsys@useobject{currentmarker}{}%
\end{pgfscope}%
\begin{pgfscope}%
\pgfsys@transformshift{3.858122in}{3.341291in}%
\pgfsys@useobject{currentmarker}{}%
\end{pgfscope}%
\begin{pgfscope}%
\pgfsys@transformshift{3.879973in}{3.338882in}%
\pgfsys@useobject{currentmarker}{}%
\end{pgfscope}%
\begin{pgfscope}%
\pgfsys@transformshift{3.902030in}{3.334293in}%
\pgfsys@useobject{currentmarker}{}%
\end{pgfscope}%
\begin{pgfscope}%
\pgfsys@transformshift{3.918502in}{3.318250in}%
\pgfsys@useobject{currentmarker}{}%
\end{pgfscope}%
\begin{pgfscope}%
\pgfsys@transformshift{3.939147in}{3.307037in}%
\pgfsys@useobject{currentmarker}{}%
\end{pgfscope}%
\begin{pgfscope}%
\pgfsys@transformshift{3.956940in}{3.289206in}%
\pgfsys@useobject{currentmarker}{}%
\end{pgfscope}%
\begin{pgfscope}%
\pgfsys@transformshift{3.983218in}{3.283330in}%
\pgfsys@useobject{currentmarker}{}%
\end{pgfscope}%
\begin{pgfscope}%
\pgfsys@transformshift{4.007133in}{3.269239in}%
\pgfsys@useobject{currentmarker}{}%
\end{pgfscope}%
\begin{pgfscope}%
\pgfsys@transformshift{4.031267in}{3.253839in}%
\pgfsys@useobject{currentmarker}{}%
\end{pgfscope}%
\begin{pgfscope}%
\pgfsys@transformshift{4.041316in}{3.241717in}%
\pgfsys@useobject{currentmarker}{}%
\end{pgfscope}%
\begin{pgfscope}%
\pgfsys@transformshift{4.041118in}{3.224738in}%
\pgfsys@useobject{currentmarker}{}%
\end{pgfscope}%
\begin{pgfscope}%
\pgfsys@transformshift{4.045603in}{3.206328in}%
\pgfsys@useobject{currentmarker}{}%
\end{pgfscope}%
\begin{pgfscope}%
\pgfsys@transformshift{4.046986in}{3.186570in}%
\pgfsys@useobject{currentmarker}{}%
\end{pgfscope}%
\begin{pgfscope}%
\pgfsys@transformshift{4.053774in}{3.166406in}%
\pgfsys@useobject{currentmarker}{}%
\end{pgfscope}%
\begin{pgfscope}%
\pgfsys@transformshift{4.056947in}{3.144483in}%
\pgfsys@useobject{currentmarker}{}%
\end{pgfscope}%
\begin{pgfscope}%
\pgfsys@transformshift{4.060480in}{3.122106in}%
\pgfsys@useobject{currentmarker}{}%
\end{pgfscope}%
\begin{pgfscope}%
\pgfsys@transformshift{4.057293in}{3.097804in}%
\pgfsys@useobject{currentmarker}{}%
\end{pgfscope}%
\begin{pgfscope}%
\pgfsys@transformshift{4.056950in}{3.071934in}%
\pgfsys@useobject{currentmarker}{}%
\end{pgfscope}%
\begin{pgfscope}%
\pgfsys@transformshift{4.053655in}{3.045297in}%
\pgfsys@useobject{currentmarker}{}%
\end{pgfscope}%
\begin{pgfscope}%
\pgfsys@transformshift{4.055851in}{3.017478in}%
\pgfsys@useobject{currentmarker}{}%
\end{pgfscope}%
\begin{pgfscope}%
\pgfsys@transformshift{4.048660in}{2.989749in}%
\pgfsys@useobject{currentmarker}{}%
\end{pgfscope}%
\begin{pgfscope}%
\pgfsys@transformshift{4.053523in}{2.960997in}%
\pgfsys@useobject{currentmarker}{}%
\end{pgfscope}%
\begin{pgfscope}%
\pgfsys@transformshift{4.044212in}{2.930105in}%
\pgfsys@useobject{currentmarker}{}%
\end{pgfscope}%
\begin{pgfscope}%
\pgfsys@transformshift{4.045636in}{2.912416in}%
\pgfsys@useobject{currentmarker}{}%
\end{pgfscope}%
\begin{pgfscope}%
\pgfsys@transformshift{4.047197in}{2.893917in}%
\pgfsys@useobject{currentmarker}{}%
\end{pgfscope}%
\begin{pgfscope}%
\pgfsys@transformshift{4.042137in}{2.872722in}%
\pgfsys@useobject{currentmarker}{}%
\end{pgfscope}%
\begin{pgfscope}%
\pgfsys@transformshift{4.041487in}{2.860755in}%
\pgfsys@useobject{currentmarker}{}%
\end{pgfscope}%
\begin{pgfscope}%
\pgfsys@transformshift{4.041524in}{2.846279in}%
\pgfsys@useobject{currentmarker}{}%
\end{pgfscope}%
\begin{pgfscope}%
\pgfsys@transformshift{4.041367in}{2.838319in}%
\pgfsys@useobject{currentmarker}{}%
\end{pgfscope}%
\begin{pgfscope}%
\pgfsys@transformshift{4.038712in}{2.829437in}%
\pgfsys@useobject{currentmarker}{}%
\end{pgfscope}%
\begin{pgfscope}%
\pgfsys@transformshift{4.040746in}{2.816100in}%
\pgfsys@useobject{currentmarker}{}%
\end{pgfscope}%
\begin{pgfscope}%
\pgfsys@transformshift{4.045039in}{2.800689in}%
\pgfsys@useobject{currentmarker}{}%
\end{pgfscope}%
\begin{pgfscope}%
\pgfsys@transformshift{4.043659in}{2.782109in}%
\pgfsys@useobject{currentmarker}{}%
\end{pgfscope}%
\begin{pgfscope}%
\pgfsys@transformshift{4.037820in}{2.763711in}%
\pgfsys@useobject{currentmarker}{}%
\end{pgfscope}%
\begin{pgfscope}%
\pgfsys@transformshift{4.042461in}{2.738622in}%
\pgfsys@useobject{currentmarker}{}%
\end{pgfscope}%
\begin{pgfscope}%
\pgfsys@transformshift{4.043144in}{2.724606in}%
\pgfsys@useobject{currentmarker}{}%
\end{pgfscope}%
\begin{pgfscope}%
\pgfsys@transformshift{4.041903in}{2.710049in}%
\pgfsys@useobject{currentmarker}{}%
\end{pgfscope}%
\begin{pgfscope}%
\pgfsys@transformshift{4.038285in}{2.695105in}%
\pgfsys@useobject{currentmarker}{}%
\end{pgfscope}%
\begin{pgfscope}%
\pgfsys@transformshift{4.041855in}{2.676681in}%
\pgfsys@useobject{currentmarker}{}%
\end{pgfscope}%
\begin{pgfscope}%
\pgfsys@transformshift{4.041438in}{2.666367in}%
\pgfsys@useobject{currentmarker}{}%
\end{pgfscope}%
\begin{pgfscope}%
\pgfsys@transformshift{4.040449in}{2.653216in}%
\pgfsys@useobject{currentmarker}{}%
\end{pgfscope}%
\begin{pgfscope}%
\pgfsys@transformshift{4.040956in}{2.645981in}%
\pgfsys@useobject{currentmarker}{}%
\end{pgfscope}%
\begin{pgfscope}%
\pgfsys@transformshift{4.041978in}{2.642125in}%
\pgfsys@useobject{currentmarker}{}%
\end{pgfscope}%
\begin{pgfscope}%
\pgfsys@transformshift{4.041697in}{2.639949in}%
\pgfsys@useobject{currentmarker}{}%
\end{pgfscope}%
\begin{pgfscope}%
\pgfsys@transformshift{4.040527in}{2.635125in}%
\pgfsys@useobject{currentmarker}{}%
\end{pgfscope}%
\begin{pgfscope}%
\pgfsys@transformshift{4.041903in}{2.628293in}%
\pgfsys@useobject{currentmarker}{}%
\end{pgfscope}%
\begin{pgfscope}%
\pgfsys@transformshift{4.044892in}{2.620463in}%
\pgfsys@useobject{currentmarker}{}%
\end{pgfscope}%
\begin{pgfscope}%
\pgfsys@transformshift{4.043432in}{2.608893in}%
\pgfsys@useobject{currentmarker}{}%
\end{pgfscope}%
\begin{pgfscope}%
\pgfsys@transformshift{4.044380in}{2.595779in}%
\pgfsys@useobject{currentmarker}{}%
\end{pgfscope}%
\begin{pgfscope}%
\pgfsys@transformshift{4.048503in}{2.577626in}%
\pgfsys@useobject{currentmarker}{}%
\end{pgfscope}%
\begin{pgfscope}%
\pgfsys@transformshift{4.049588in}{2.567445in}%
\pgfsys@useobject{currentmarker}{}%
\end{pgfscope}%
\begin{pgfscope}%
\pgfsys@transformshift{4.048022in}{2.553932in}%
\pgfsys@useobject{currentmarker}{}%
\end{pgfscope}%
\begin{pgfscope}%
\pgfsys@transformshift{4.048365in}{2.539427in}%
\pgfsys@useobject{currentmarker}{}%
\end{pgfscope}%
\begin{pgfscope}%
\pgfsys@transformshift{4.052485in}{2.522127in}%
\pgfsys@useobject{currentmarker}{}%
\end{pgfscope}%
\begin{pgfscope}%
\pgfsys@transformshift{4.050845in}{2.503513in}%
\pgfsys@useobject{currentmarker}{}%
\end{pgfscope}%
\begin{pgfscope}%
\pgfsys@transformshift{4.044131in}{2.482494in}%
\pgfsys@useobject{currentmarker}{}%
\end{pgfscope}%
\begin{pgfscope}%
\pgfsys@transformshift{4.051820in}{2.457846in}%
\pgfsys@useobject{currentmarker}{}%
\end{pgfscope}%
\begin{pgfscope}%
\pgfsys@transformshift{4.057357in}{2.431658in}%
\pgfsys@useobject{currentmarker}{}%
\end{pgfscope}%
\begin{pgfscope}%
\pgfsys@transformshift{4.056684in}{2.401349in}%
\pgfsys@useobject{currentmarker}{}%
\end{pgfscope}%
\begin{pgfscope}%
\pgfsys@transformshift{4.054810in}{2.370598in}%
\pgfsys@useobject{currentmarker}{}%
\end{pgfscope}%
\begin{pgfscope}%
\pgfsys@transformshift{4.062216in}{2.337359in}%
\pgfsys@useobject{currentmarker}{}%
\end{pgfscope}%
\begin{pgfscope}%
\pgfsys@transformshift{4.067875in}{2.303119in}%
\pgfsys@useobject{currentmarker}{}%
\end{pgfscope}%
\begin{pgfscope}%
\pgfsys@transformshift{4.060013in}{2.265234in}%
\pgfsys@useobject{currentmarker}{}%
\end{pgfscope}%
\begin{pgfscope}%
\pgfsys@transformshift{4.067740in}{2.226582in}%
\pgfsys@useobject{currentmarker}{}%
\end{pgfscope}%
\begin{pgfscope}%
\pgfsys@transformshift{4.076764in}{2.187450in}%
\pgfsys@useobject{currentmarker}{}%
\end{pgfscope}%
\begin{pgfscope}%
\pgfsys@transformshift{4.080125in}{2.144176in}%
\pgfsys@useobject{currentmarker}{}%
\end{pgfscope}%
\begin{pgfscope}%
\pgfsys@transformshift{4.070849in}{2.098987in}%
\pgfsys@useobject{currentmarker}{}%
\end{pgfscope}%
\begin{pgfscope}%
\pgfsys@transformshift{4.087997in}{2.051953in}%
\pgfsys@useobject{currentmarker}{}%
\end{pgfscope}%
\begin{pgfscope}%
\pgfsys@transformshift{4.094354in}{2.001524in}%
\pgfsys@useobject{currentmarker}{}%
\end{pgfscope}%
\begin{pgfscope}%
\pgfsys@transformshift{4.084097in}{1.947333in}%
\pgfsys@useobject{currentmarker}{}%
\end{pgfscope}%
\begin{pgfscope}%
\pgfsys@transformshift{4.115654in}{1.900323in}%
\pgfsys@useobject{currentmarker}{}%
\end{pgfscope}%
\begin{pgfscope}%
\pgfsys@transformshift{4.144489in}{1.850919in}%
\pgfsys@useobject{currentmarker}{}%
\end{pgfscope}%
\begin{pgfscope}%
\pgfsys@transformshift{4.144188in}{1.792481in}%
\pgfsys@useobject{currentmarker}{}%
\end{pgfscope}%
\begin{pgfscope}%
\pgfsys@transformshift{4.136776in}{1.729914in}%
\pgfsys@useobject{currentmarker}{}%
\end{pgfscope}%
\begin{pgfscope}%
\pgfsys@transformshift{4.167857in}{1.672695in}%
\pgfsys@useobject{currentmarker}{}%
\end{pgfscope}%
\begin{pgfscope}%
\pgfsys@transformshift{4.200254in}{1.615528in}%
\pgfsys@useobject{currentmarker}{}%
\end{pgfscope}%
\begin{pgfscope}%
\pgfsys@transformshift{4.203671in}{1.545578in}%
\pgfsys@useobject{currentmarker}{}%
\end{pgfscope}%
\begin{pgfscope}%
\pgfsys@transformshift{4.198961in}{1.474352in}%
\pgfsys@useobject{currentmarker}{}%
\end{pgfscope}%
\begin{pgfscope}%
\pgfsys@transformshift{4.220374in}{1.404323in}%
\pgfsys@useobject{currentmarker}{}%
\end{pgfscope}%
\begin{pgfscope}%
\pgfsys@transformshift{4.224994in}{1.364313in}%
\pgfsys@useobject{currentmarker}{}%
\end{pgfscope}%
\begin{pgfscope}%
\pgfsys@transformshift{4.222162in}{1.319958in}%
\pgfsys@useobject{currentmarker}{}%
\end{pgfscope}%
\begin{pgfscope}%
\pgfsys@transformshift{4.236797in}{1.276615in}%
\pgfsys@useobject{currentmarker}{}%
\end{pgfscope}%
\begin{pgfscope}%
\pgfsys@transformshift{4.245558in}{1.253028in}%
\pgfsys@useobject{currentmarker}{}%
\end{pgfscope}%
\begin{pgfscope}%
\pgfsys@transformshift{4.254887in}{1.228641in}%
\pgfsys@useobject{currentmarker}{}%
\end{pgfscope}%
\begin{pgfscope}%
\pgfsys@transformshift{4.249539in}{1.199264in}%
\pgfsys@useobject{currentmarker}{}%
\end{pgfscope}%
\begin{pgfscope}%
\pgfsys@transformshift{4.263438in}{1.169059in}%
\pgfsys@useobject{currentmarker}{}%
\end{pgfscope}%
\begin{pgfscope}%
\pgfsys@transformshift{4.268117in}{1.151380in}%
\pgfsys@useobject{currentmarker}{}%
\end{pgfscope}%
\begin{pgfscope}%
\pgfsys@transformshift{4.271444in}{1.130851in}%
\pgfsys@useobject{currentmarker}{}%
\end{pgfscope}%
\begin{pgfscope}%
\pgfsys@transformshift{4.267581in}{1.108350in}%
\pgfsys@useobject{currentmarker}{}%
\end{pgfscope}%
\begin{pgfscope}%
\pgfsys@transformshift{4.278306in}{1.083202in}%
\pgfsys@useobject{currentmarker}{}%
\end{pgfscope}%
\begin{pgfscope}%
\pgfsys@transformshift{4.283879in}{1.055848in}%
\pgfsys@useobject{currentmarker}{}%
\end{pgfscope}%
\begin{pgfscope}%
\pgfsys@transformshift{4.281740in}{1.025400in}%
\pgfsys@useobject{currentmarker}{}%
\end{pgfscope}%
\begin{pgfscope}%
\pgfsys@transformshift{4.282283in}{1.008620in}%
\pgfsys@useobject{currentmarker}{}%
\end{pgfscope}%
\begin{pgfscope}%
\pgfsys@transformshift{4.290028in}{0.991861in}%
\pgfsys@useobject{currentmarker}{}%
\end{pgfscope}%
\begin{pgfscope}%
\pgfsys@transformshift{4.291531in}{0.972715in}%
\pgfsys@useobject{currentmarker}{}%
\end{pgfscope}%
\begin{pgfscope}%
\pgfsys@transformshift{4.289697in}{0.952214in}%
\pgfsys@useobject{currentmarker}{}%
\end{pgfscope}%
\begin{pgfscope}%
\pgfsys@transformshift{4.289682in}{0.940893in}%
\pgfsys@useobject{currentmarker}{}%
\end{pgfscope}%
\begin{pgfscope}%
\pgfsys@transformshift{4.292897in}{0.928700in}%
\pgfsys@useobject{currentmarker}{}%
\end{pgfscope}%
\begin{pgfscope}%
\pgfsys@transformshift{4.293309in}{0.921777in}%
\pgfsys@useobject{currentmarker}{}%
\end{pgfscope}%
\begin{pgfscope}%
\pgfsys@transformshift{4.292159in}{0.913008in}%
\pgfsys@useobject{currentmarker}{}%
\end{pgfscope}%
\begin{pgfscope}%
\pgfsys@transformshift{4.294940in}{0.903689in}%
\pgfsys@useobject{currentmarker}{}%
\end{pgfscope}%
\begin{pgfscope}%
\pgfsys@transformshift{4.299252in}{0.893223in}%
\pgfsys@useobject{currentmarker}{}%
\end{pgfscope}%
\begin{pgfscope}%
\pgfsys@transformshift{4.299326in}{0.886998in}%
\pgfsys@useobject{currentmarker}{}%
\end{pgfscope}%
\begin{pgfscope}%
\pgfsys@transformshift{4.298962in}{0.879175in}%
\pgfsys@useobject{currentmarker}{}%
\end{pgfscope}%
\begin{pgfscope}%
\pgfsys@transformshift{4.303238in}{0.868458in}%
\pgfsys@useobject{currentmarker}{}%
\end{pgfscope}%
\begin{pgfscope}%
\pgfsys@transformshift{4.307840in}{0.856597in}%
\pgfsys@useobject{currentmarker}{}%
\end{pgfscope}%
\begin{pgfscope}%
\pgfsys@transformshift{4.310980in}{0.843428in}%
\pgfsys@useobject{currentmarker}{}%
\end{pgfscope}%
\begin{pgfscope}%
\pgfsys@transformshift{4.308955in}{0.828524in}%
\pgfsys@useobject{currentmarker}{}%
\end{pgfscope}%
\begin{pgfscope}%
\pgfsys@transformshift{4.314078in}{0.811038in}%
\pgfsys@useobject{currentmarker}{}%
\end{pgfscope}%
\begin{pgfscope}%
\pgfsys@transformshift{4.321402in}{0.792573in}%
\pgfsys@useobject{currentmarker}{}%
\end{pgfscope}%
\begin{pgfscope}%
\pgfsys@transformshift{4.325748in}{0.771586in}%
\pgfsys@useobject{currentmarker}{}%
\end{pgfscope}%
\begin{pgfscope}%
\pgfsys@transformshift{4.327667in}{0.749501in}%
\pgfsys@useobject{currentmarker}{}%
\end{pgfscope}%
\begin{pgfscope}%
\pgfsys@transformshift{4.327987in}{0.737313in}%
\pgfsys@useobject{currentmarker}{}%
\end{pgfscope}%
\begin{pgfscope}%
\pgfsys@transformshift{4.327093in}{0.724569in}%
\pgfsys@useobject{currentmarker}{}%
\end{pgfscope}%
\begin{pgfscope}%
\pgfsys@transformshift{4.325336in}{0.711135in}%
\pgfsys@useobject{currentmarker}{}%
\end{pgfscope}%
\begin{pgfscope}%
\pgfsys@transformshift{4.321874in}{0.697423in}%
\pgfsys@useobject{currentmarker}{}%
\end{pgfscope}%
\begin{pgfscope}%
\pgfsys@transformshift{4.311114in}{0.686557in}%
\pgfsys@useobject{currentmarker}{}%
\end{pgfscope}%
\begin{pgfscope}%
\pgfsys@transformshift{4.294311in}{0.683179in}%
\pgfsys@useobject{currentmarker}{}%
\end{pgfscope}%
\begin{pgfscope}%
\pgfsys@transformshift{4.275235in}{0.683000in}%
\pgfsys@useobject{currentmarker}{}%
\end{pgfscope}%
\begin{pgfscope}%
\pgfsys@transformshift{4.256146in}{0.688526in}%
\pgfsys@useobject{currentmarker}{}%
\end{pgfscope}%
\begin{pgfscope}%
\pgfsys@transformshift{4.233982in}{0.691715in}%
\pgfsys@useobject{currentmarker}{}%
\end{pgfscope}%
\begin{pgfscope}%
\pgfsys@transformshift{4.212004in}{0.698993in}%
\pgfsys@useobject{currentmarker}{}%
\end{pgfscope}%
\begin{pgfscope}%
\pgfsys@transformshift{4.187077in}{0.701000in}%
\pgfsys@useobject{currentmarker}{}%
\end{pgfscope}%
\begin{pgfscope}%
\pgfsys@transformshift{4.162301in}{0.710854in}%
\pgfsys@useobject{currentmarker}{}%
\end{pgfscope}%
\begin{pgfscope}%
\pgfsys@transformshift{4.134133in}{0.709881in}%
\pgfsys@useobject{currentmarker}{}%
\end{pgfscope}%
\begin{pgfscope}%
\pgfsys@transformshift{4.105556in}{0.715571in}%
\pgfsys@useobject{currentmarker}{}%
\end{pgfscope}%
\begin{pgfscope}%
\pgfsys@transformshift{4.070458in}{0.719524in}%
\pgfsys@useobject{currentmarker}{}%
\end{pgfscope}%
\begin{pgfscope}%
\pgfsys@transformshift{4.033493in}{0.724567in}%
\pgfsys@useobject{currentmarker}{}%
\end{pgfscope}%
\begin{pgfscope}%
\pgfsys@transformshift{4.001769in}{0.752041in}%
\pgfsys@useobject{currentmarker}{}%
\end{pgfscope}%
\begin{pgfscope}%
\pgfsys@transformshift{3.958808in}{0.755417in}%
\pgfsys@useobject{currentmarker}{}%
\end{pgfscope}%
\begin{pgfscope}%
\pgfsys@transformshift{3.916191in}{0.767547in}%
\pgfsys@useobject{currentmarker}{}%
\end{pgfscope}%
\begin{pgfscope}%
\pgfsys@transformshift{3.867819in}{0.769496in}%
\pgfsys@useobject{currentmarker}{}%
\end{pgfscope}%
\begin{pgfscope}%
\pgfsys@transformshift{3.819138in}{0.773808in}%
\pgfsys@useobject{currentmarker}{}%
\end{pgfscope}%
\begin{pgfscope}%
\pgfsys@transformshift{3.766554in}{0.770606in}%
\pgfsys@useobject{currentmarker}{}%
\end{pgfscope}%
\begin{pgfscope}%
\pgfsys@transformshift{3.711650in}{0.777035in}%
\pgfsys@useobject{currentmarker}{}%
\end{pgfscope}%
\begin{pgfscope}%
\pgfsys@transformshift{3.655753in}{0.783973in}%
\pgfsys@useobject{currentmarker}{}%
\end{pgfscope}%
\begin{pgfscope}%
\pgfsys@transformshift{3.595629in}{0.786856in}%
\pgfsys@useobject{currentmarker}{}%
\end{pgfscope}%
\begin{pgfscope}%
\pgfsys@transformshift{3.535266in}{0.796944in}%
\pgfsys@useobject{currentmarker}{}%
\end{pgfscope}%
\begin{pgfscope}%
\pgfsys@transformshift{3.470353in}{0.807467in}%
\pgfsys@useobject{currentmarker}{}%
\end{pgfscope}%
\begin{pgfscope}%
\pgfsys@transformshift{3.404846in}{0.818899in}%
\pgfsys@useobject{currentmarker}{}%
\end{pgfscope}%
\begin{pgfscope}%
\pgfsys@transformshift{3.333084in}{0.823641in}%
\pgfsys@useobject{currentmarker}{}%
\end{pgfscope}%
\begin{pgfscope}%
\pgfsys@transformshift{3.293599in}{0.825984in}%
\pgfsys@useobject{currentmarker}{}%
\end{pgfscope}%
\begin{pgfscope}%
\pgfsys@transformshift{3.251228in}{0.840307in}%
\pgfsys@useobject{currentmarker}{}%
\end{pgfscope}%
\begin{pgfscope}%
\pgfsys@transformshift{3.205447in}{0.848110in}%
\pgfsys@useobject{currentmarker}{}%
\end{pgfscope}%
\begin{pgfscope}%
\pgfsys@transformshift{3.154488in}{0.850927in}%
\pgfsys@useobject{currentmarker}{}%
\end{pgfscope}%
\begin{pgfscope}%
\pgfsys@transformshift{3.102753in}{0.859821in}%
\pgfsys@useobject{currentmarker}{}%
\end{pgfscope}%
\begin{pgfscope}%
\pgfsys@transformshift{3.046412in}{0.868267in}%
\pgfsys@useobject{currentmarker}{}%
\end{pgfscope}%
\begin{pgfscope}%
\pgfsys@transformshift{2.989258in}{0.884181in}%
\pgfsys@useobject{currentmarker}{}%
\end{pgfscope}%
\begin{pgfscope}%
\pgfsys@transformshift{2.929048in}{0.891075in}%
\pgfsys@useobject{currentmarker}{}%
\end{pgfscope}%
\begin{pgfscope}%
\pgfsys@transformshift{2.863382in}{0.894493in}%
\pgfsys@useobject{currentmarker}{}%
\end{pgfscope}%
\begin{pgfscope}%
\pgfsys@transformshift{2.796599in}{0.893493in}%
\pgfsys@useobject{currentmarker}{}%
\end{pgfscope}%
\begin{pgfscope}%
\pgfsys@transformshift{2.727656in}{0.907216in}%
\pgfsys@useobject{currentmarker}{}%
\end{pgfscope}%
\begin{pgfscope}%
\pgfsys@transformshift{2.656627in}{0.912870in}%
\pgfsys@useobject{currentmarker}{}%
\end{pgfscope}%
\begin{pgfscope}%
\pgfsys@transformshift{2.580130in}{0.913546in}%
\pgfsys@useobject{currentmarker}{}%
\end{pgfscope}%
\begin{pgfscope}%
\pgfsys@transformshift{2.505288in}{0.936685in}%
\pgfsys@useobject{currentmarker}{}%
\end{pgfscope}%
\begin{pgfscope}%
\pgfsys@transformshift{2.424047in}{0.938082in}%
\pgfsys@useobject{currentmarker}{}%
\end{pgfscope}%
\begin{pgfscope}%
\pgfsys@transformshift{2.342048in}{0.945411in}%
\pgfsys@useobject{currentmarker}{}%
\end{pgfscope}%
\begin{pgfscope}%
\pgfsys@transformshift{2.254255in}{0.948616in}%
\pgfsys@useobject{currentmarker}{}%
\end{pgfscope}%
\begin{pgfscope}%
\pgfsys@transformshift{2.166401in}{0.963109in}%
\pgfsys@useobject{currentmarker}{}%
\end{pgfscope}%
\begin{pgfscope}%
\pgfsys@transformshift{2.073384in}{0.955317in}%
\pgfsys@useobject{currentmarker}{}%
\end{pgfscope}%
\begin{pgfscope}%
\pgfsys@transformshift{1.979188in}{0.959200in}%
\pgfsys@useobject{currentmarker}{}%
\end{pgfscope}%
\begin{pgfscope}%
\pgfsys@transformshift{1.878948in}{0.959214in}%
\pgfsys@useobject{currentmarker}{}%
\end{pgfscope}%
\begin{pgfscope}%
\pgfsys@transformshift{1.823998in}{0.963683in}%
\pgfsys@useobject{currentmarker}{}%
\end{pgfscope}%
\begin{pgfscope}%
\pgfsys@transformshift{1.767022in}{0.983691in}%
\pgfsys@useobject{currentmarker}{}%
\end{pgfscope}%
\begin{pgfscope}%
\pgfsys@transformshift{1.733940in}{0.986634in}%
\pgfsys@useobject{currentmarker}{}%
\end{pgfscope}%
\begin{pgfscope}%
\pgfsys@transformshift{1.694809in}{0.986296in}%
\pgfsys@useobject{currentmarker}{}%
\end{pgfscope}%
\begin{pgfscope}%
\pgfsys@transformshift{1.652361in}{0.988340in}%
\pgfsys@useobject{currentmarker}{}%
\end{pgfscope}%
\begin{pgfscope}%
\pgfsys@transformshift{1.608111in}{0.989775in}%
\pgfsys@useobject{currentmarker}{}%
\end{pgfscope}%
\begin{pgfscope}%
\pgfsys@transformshift{1.559384in}{0.997980in}%
\pgfsys@useobject{currentmarker}{}%
\end{pgfscope}%
\begin{pgfscope}%
\pgfsys@transformshift{1.509607in}{1.002345in}%
\pgfsys@useobject{currentmarker}{}%
\end{pgfscope}%
\begin{pgfscope}%
\pgfsys@transformshift{1.455648in}{0.999447in}%
\pgfsys@useobject{currentmarker}{}%
\end{pgfscope}%
\begin{pgfscope}%
\pgfsys@transformshift{1.400943in}{1.000410in}%
\pgfsys@useobject{currentmarker}{}%
\end{pgfscope}%
\begin{pgfscope}%
\pgfsys@transformshift{1.343498in}{0.995454in}%
\pgfsys@useobject{currentmarker}{}%
\end{pgfscope}%
\begin{pgfscope}%
\pgfsys@transformshift{1.285219in}{0.999084in}%
\pgfsys@useobject{currentmarker}{}%
\end{pgfscope}%
\begin{pgfscope}%
\pgfsys@transformshift{1.262209in}{1.001273in}%
\pgfsys@useobject{currentmarker}{}%
\end{pgfscope}%
\begin{pgfscope}%
\pgfsys@transformshift{1.302984in}{1.044860in}%
\pgfsys@useobject{currentmarker}{}%
\end{pgfscope}%
\begin{pgfscope}%
\pgfsys@transformshift{1.317309in}{1.105295in}%
\pgfsys@useobject{currentmarker}{}%
\end{pgfscope}%
\begin{pgfscope}%
\pgfsys@transformshift{1.315540in}{1.139409in}%
\pgfsys@useobject{currentmarker}{}%
\end{pgfscope}%
\begin{pgfscope}%
\pgfsys@transformshift{1.307730in}{1.174652in}%
\pgfsys@useobject{currentmarker}{}%
\end{pgfscope}%
\begin{pgfscope}%
\pgfsys@transformshift{1.309042in}{1.218440in}%
\pgfsys@useobject{currentmarker}{}%
\end{pgfscope}%
\begin{pgfscope}%
\pgfsys@transformshift{1.300254in}{1.262742in}%
\pgfsys@useobject{currentmarker}{}%
\end{pgfscope}%
\begin{pgfscope}%
\pgfsys@transformshift{1.293597in}{1.308372in}%
\pgfsys@useobject{currentmarker}{}%
\end{pgfscope}%
\begin{pgfscope}%
\pgfsys@transformshift{1.305440in}{1.354344in}%
\pgfsys@useobject{currentmarker}{}%
\end{pgfscope}%
\begin{pgfscope}%
\pgfsys@transformshift{1.300552in}{1.409055in}%
\pgfsys@useobject{currentmarker}{}%
\end{pgfscope}%
\begin{pgfscope}%
\pgfsys@transformshift{1.291934in}{1.438010in}%
\pgfsys@useobject{currentmarker}{}%
\end{pgfscope}%
\begin{pgfscope}%
\pgfsys@transformshift{1.292506in}{1.472279in}%
\pgfsys@useobject{currentmarker}{}%
\end{pgfscope}%
\begin{pgfscope}%
\pgfsys@transformshift{1.287255in}{1.508870in}%
\pgfsys@useobject{currentmarker}{}%
\end{pgfscope}%
\begin{pgfscope}%
\pgfsys@transformshift{1.282261in}{1.546481in}%
\pgfsys@useobject{currentmarker}{}%
\end{pgfscope}%
\begin{pgfscope}%
\pgfsys@transformshift{1.293226in}{1.587196in}%
\pgfsys@useobject{currentmarker}{}%
\end{pgfscope}%
\begin{pgfscope}%
\pgfsys@transformshift{1.292654in}{1.610380in}%
\pgfsys@useobject{currentmarker}{}%
\end{pgfscope}%
\begin{pgfscope}%
\pgfsys@transformshift{1.284258in}{1.632756in}%
\pgfsys@useobject{currentmarker}{}%
\end{pgfscope}%
\begin{pgfscope}%
\pgfsys@transformshift{1.293010in}{1.662699in}%
\pgfsys@useobject{currentmarker}{}%
\end{pgfscope}%
\begin{pgfscope}%
\pgfsys@transformshift{1.301124in}{1.696190in}%
\pgfsys@useobject{currentmarker}{}%
\end{pgfscope}%
\begin{pgfscope}%
\pgfsys@transformshift{1.293224in}{1.734078in}%
\pgfsys@useobject{currentmarker}{}%
\end{pgfscope}%
\begin{pgfscope}%
\pgfsys@transformshift{1.300635in}{1.776299in}%
\pgfsys@useobject{currentmarker}{}%
\end{pgfscope}%
\begin{pgfscope}%
\pgfsys@transformshift{1.311033in}{1.818915in}%
\pgfsys@useobject{currentmarker}{}%
\end{pgfscope}%
\begin{pgfscope}%
\pgfsys@transformshift{1.313475in}{1.868738in}%
\pgfsys@useobject{currentmarker}{}%
\end{pgfscope}%
\begin{pgfscope}%
\pgfsys@transformshift{1.298812in}{1.919449in}%
\pgfsys@useobject{currentmarker}{}%
\end{pgfscope}%
\begin{pgfscope}%
\pgfsys@transformshift{1.319142in}{1.974095in}%
\pgfsys@useobject{currentmarker}{}%
\end{pgfscope}%
\begin{pgfscope}%
\pgfsys@transformshift{1.335589in}{2.031309in}%
\pgfsys@useobject{currentmarker}{}%
\end{pgfscope}%
\begin{pgfscope}%
\pgfsys@transformshift{1.326248in}{2.096602in}%
\pgfsys@useobject{currentmarker}{}%
\end{pgfscope}%
\begin{pgfscope}%
\pgfsys@transformshift{1.330344in}{2.162951in}%
\pgfsys@useobject{currentmarker}{}%
\end{pgfscope}%
\begin{pgfscope}%
\pgfsys@transformshift{1.343317in}{2.197133in}%
\pgfsys@useobject{currentmarker}{}%
\end{pgfscope}%
\begin{pgfscope}%
\pgfsys@transformshift{1.339489in}{2.239120in}%
\pgfsys@useobject{currentmarker}{}%
\end{pgfscope}%
\begin{pgfscope}%
\pgfsys@transformshift{1.330608in}{2.283099in}%
\pgfsys@useobject{currentmarker}{}%
\end{pgfscope}%
\begin{pgfscope}%
\pgfsys@transformshift{1.342325in}{2.329360in}%
\pgfsys@useobject{currentmarker}{}%
\end{pgfscope}%
\begin{pgfscope}%
\pgfsys@transformshift{1.347101in}{2.355169in}%
\pgfsys@useobject{currentmarker}{}%
\end{pgfscope}%
\begin{pgfscope}%
\pgfsys@transformshift{1.340720in}{2.385224in}%
\pgfsys@useobject{currentmarker}{}%
\end{pgfscope}%
\begin{pgfscope}%
\pgfsys@transformshift{1.342829in}{2.416641in}%
\pgfsys@useobject{currentmarker}{}%
\end{pgfscope}%
\begin{pgfscope}%
\pgfsys@transformshift{1.345873in}{2.433689in}%
\pgfsys@useobject{currentmarker}{}%
\end{pgfscope}%
\begin{pgfscope}%
\pgfsys@transformshift{1.342820in}{2.455713in}%
\pgfsys@useobject{currentmarker}{}%
\end{pgfscope}%
\begin{pgfscope}%
\pgfsys@transformshift{1.340155in}{2.467648in}%
\pgfsys@useobject{currentmarker}{}%
\end{pgfscope}%
\begin{pgfscope}%
\pgfsys@transformshift{1.343820in}{2.483319in}%
\pgfsys@useobject{currentmarker}{}%
\end{pgfscope}%
\begin{pgfscope}%
\pgfsys@transformshift{1.341633in}{2.499851in}%
\pgfsys@useobject{currentmarker}{}%
\end{pgfscope}%
\begin{pgfscope}%
\pgfsys@transformshift{1.334256in}{2.516947in}%
\pgfsys@useobject{currentmarker}{}%
\end{pgfscope}%
\begin{pgfscope}%
\pgfsys@transformshift{1.338741in}{2.540963in}%
\pgfsys@useobject{currentmarker}{}%
\end{pgfscope}%
\begin{pgfscope}%
\pgfsys@transformshift{1.344867in}{2.565422in}%
\pgfsys@useobject{currentmarker}{}%
\end{pgfscope}%
\begin{pgfscope}%
\pgfsys@transformshift{1.336016in}{2.594217in}%
\pgfsys@useobject{currentmarker}{}%
\end{pgfscope}%
\begin{pgfscope}%
\pgfsys@transformshift{1.339257in}{2.625387in}%
\pgfsys@useobject{currentmarker}{}%
\end{pgfscope}%
\begin{pgfscope}%
\pgfsys@transformshift{1.344734in}{2.641730in}%
\pgfsys@useobject{currentmarker}{}%
\end{pgfscope}%
\begin{pgfscope}%
\pgfsys@transformshift{1.339335in}{2.662183in}%
\pgfsys@useobject{currentmarker}{}%
\end{pgfscope}%
\begin{pgfscope}%
\pgfsys@transformshift{1.345533in}{2.684791in}%
\pgfsys@useobject{currentmarker}{}%
\end{pgfscope}%
\begin{pgfscope}%
\pgfsys@transformshift{1.348070in}{2.697432in}%
\pgfsys@useobject{currentmarker}{}%
\end{pgfscope}%
\begin{pgfscope}%
\pgfsys@transformshift{1.346273in}{2.716066in}%
\pgfsys@useobject{currentmarker}{}%
\end{pgfscope}%
\begin{pgfscope}%
\pgfsys@transformshift{1.340372in}{2.737577in}%
\pgfsys@useobject{currentmarker}{}%
\end{pgfscope}%
\begin{pgfscope}%
\pgfsys@transformshift{1.349308in}{2.761759in}%
\pgfsys@useobject{currentmarker}{}%
\end{pgfscope}%
\begin{pgfscope}%
\pgfsys@transformshift{1.350777in}{2.775862in}%
\pgfsys@useobject{currentmarker}{}%
\end{pgfscope}%
\begin{pgfscope}%
\pgfsys@transformshift{1.344801in}{2.793808in}%
\pgfsys@useobject{currentmarker}{}%
\end{pgfscope}%
\begin{pgfscope}%
\pgfsys@transformshift{1.347273in}{2.813862in}%
\pgfsys@useobject{currentmarker}{}%
\end{pgfscope}%
\begin{pgfscope}%
\pgfsys@transformshift{1.352714in}{2.833956in}%
\pgfsys@useobject{currentmarker}{}%
\end{pgfscope}%
\begin{pgfscope}%
\pgfsys@transformshift{1.354610in}{2.857570in}%
\pgfsys@useobject{currentmarker}{}%
\end{pgfscope}%
\begin{pgfscope}%
\pgfsys@transformshift{1.345887in}{2.882212in}%
\pgfsys@useobject{currentmarker}{}%
\end{pgfscope}%
\begin{pgfscope}%
\pgfsys@transformshift{1.353347in}{2.912270in}%
\pgfsys@useobject{currentmarker}{}%
\end{pgfscope}%
\begin{pgfscope}%
\pgfsys@transformshift{1.356209in}{2.929061in}%
\pgfsys@useobject{currentmarker}{}%
\end{pgfscope}%
\begin{pgfscope}%
\pgfsys@transformshift{1.351196in}{2.950450in}%
\pgfsys@useobject{currentmarker}{}%
\end{pgfscope}%
\begin{pgfscope}%
\pgfsys@transformshift{1.351725in}{2.962521in}%
\pgfsys@useobject{currentmarker}{}%
\end{pgfscope}%
\begin{pgfscope}%
\pgfsys@transformshift{1.354219in}{2.968681in}%
\pgfsys@useobject{currentmarker}{}%
\end{pgfscope}%
\begin{pgfscope}%
\pgfsys@transformshift{1.353195in}{2.977821in}%
\pgfsys@useobject{currentmarker}{}%
\end{pgfscope}%
\begin{pgfscope}%
\pgfsys@transformshift{1.350355in}{2.987689in}%
\pgfsys@useobject{currentmarker}{}%
\end{pgfscope}%
\begin{pgfscope}%
\pgfsys@transformshift{1.354626in}{3.003358in}%
\pgfsys@useobject{currentmarker}{}%
\end{pgfscope}%
\begin{pgfscope}%
\pgfsys@transformshift{1.358518in}{3.019695in}%
\pgfsys@useobject{currentmarker}{}%
\end{pgfscope}%
\begin{pgfscope}%
\pgfsys@transformshift{1.352973in}{3.040734in}%
\pgfsys@useobject{currentmarker}{}%
\end{pgfscope}%
\begin{pgfscope}%
\pgfsys@transformshift{1.353148in}{3.063128in}%
\pgfsys@useobject{currentmarker}{}%
\end{pgfscope}%
\begin{pgfscope}%
\pgfsys@transformshift{1.363570in}{3.086350in}%
\pgfsys@useobject{currentmarker}{}%
\end{pgfscope}%
\begin{pgfscope}%
\pgfsys@transformshift{1.362750in}{3.112700in}%
\pgfsys@useobject{currentmarker}{}%
\end{pgfscope}%
\begin{pgfscope}%
\pgfsys@transformshift{1.356900in}{3.139658in}%
\pgfsys@useobject{currentmarker}{}%
\end{pgfscope}%
\begin{pgfscope}%
\pgfsys@transformshift{1.367525in}{3.171647in}%
\pgfsys@useobject{currentmarker}{}%
\end{pgfscope}%
\begin{pgfscope}%
\pgfsys@transformshift{1.371905in}{3.206361in}%
\pgfsys@useobject{currentmarker}{}%
\end{pgfscope}%
\begin{pgfscope}%
\pgfsys@transformshift{1.371109in}{3.244194in}%
\pgfsys@useobject{currentmarker}{}%
\end{pgfscope}%
\begin{pgfscope}%
\pgfsys@transformshift{1.368114in}{3.264790in}%
\pgfsys@useobject{currentmarker}{}%
\end{pgfscope}%
\begin{pgfscope}%
\pgfsys@transformshift{1.374961in}{3.289564in}%
\pgfsys@useobject{currentmarker}{}%
\end{pgfscope}%
\begin{pgfscope}%
\pgfsys@transformshift{1.372735in}{3.303525in}%
\pgfsys@useobject{currentmarker}{}%
\end{pgfscope}%
\begin{pgfscope}%
\pgfsys@transformshift{1.368284in}{3.318746in}%
\pgfsys@useobject{currentmarker}{}%
\end{pgfscope}%
\begin{pgfscope}%
\pgfsys@transformshift{1.372399in}{3.338870in}%
\pgfsys@useobject{currentmarker}{}%
\end{pgfscope}%
\begin{pgfscope}%
\pgfsys@transformshift{1.372278in}{3.359891in}%
\pgfsys@useobject{currentmarker}{}%
\end{pgfscope}%
\begin{pgfscope}%
\pgfsys@transformshift{1.369774in}{3.384666in}%
\pgfsys@useobject{currentmarker}{}%
\end{pgfscope}%
\begin{pgfscope}%
\pgfsys@transformshift{1.368031in}{3.398250in}%
\pgfsys@useobject{currentmarker}{}%
\end{pgfscope}%
\begin{pgfscope}%
\pgfsys@transformshift{1.372194in}{3.412045in}%
\pgfsys@useobject{currentmarker}{}%
\end{pgfscope}%
\begin{pgfscope}%
\pgfsys@transformshift{1.371468in}{3.419937in}%
\pgfsys@useobject{currentmarker}{}%
\end{pgfscope}%
\begin{pgfscope}%
\pgfsys@transformshift{1.368825in}{3.428246in}%
\pgfsys@useobject{currentmarker}{}%
\end{pgfscope}%
\begin{pgfscope}%
\pgfsys@transformshift{1.373781in}{3.443950in}%
\pgfsys@useobject{currentmarker}{}%
\end{pgfscope}%
\begin{pgfscope}%
\pgfsys@transformshift{1.376330in}{3.460807in}%
\pgfsys@useobject{currentmarker}{}%
\end{pgfscope}%
\begin{pgfscope}%
\pgfsys@transformshift{1.370670in}{3.480987in}%
\pgfsys@useobject{currentmarker}{}%
\end{pgfscope}%
\begin{pgfscope}%
\pgfsys@transformshift{1.374818in}{3.503775in}%
\pgfsys@useobject{currentmarker}{}%
\end{pgfscope}%
\begin{pgfscope}%
\pgfsys@transformshift{1.380195in}{3.515324in}%
\pgfsys@useobject{currentmarker}{}%
\end{pgfscope}%
\begin{pgfscope}%
\pgfsys@transformshift{1.376104in}{3.532906in}%
\pgfsys@useobject{currentmarker}{}%
\end{pgfscope}%
\begin{pgfscope}%
\pgfsys@transformshift{1.375336in}{3.551485in}%
\pgfsys@useobject{currentmarker}{}%
\end{pgfscope}%
\begin{pgfscope}%
\pgfsys@transformshift{1.385668in}{3.571903in}%
\pgfsys@useobject{currentmarker}{}%
\end{pgfscope}%
\begin{pgfscope}%
\pgfsys@transformshift{1.382080in}{3.595543in}%
\pgfsys@useobject{currentmarker}{}%
\end{pgfscope}%
\begin{pgfscope}%
\pgfsys@transformshift{1.374616in}{3.619499in}%
\pgfsys@useobject{currentmarker}{}%
\end{pgfscope}%
\begin{pgfscope}%
\pgfsys@transformshift{1.383940in}{3.650625in}%
\pgfsys@useobject{currentmarker}{}%
\end{pgfscope}%
\begin{pgfscope}%
\pgfsys@transformshift{1.389558in}{3.683498in}%
\pgfsys@useobject{currentmarker}{}%
\end{pgfscope}%
\begin{pgfscope}%
\pgfsys@transformshift{1.380672in}{3.720156in}%
\pgfsys@useobject{currentmarker}{}%
\end{pgfscope}%
\begin{pgfscope}%
\pgfsys@transformshift{1.380308in}{3.740899in}%
\pgfsys@useobject{currentmarker}{}%
\end{pgfscope}%
\begin{pgfscope}%
\pgfsys@transformshift{1.388355in}{3.762230in}%
\pgfsys@useobject{currentmarker}{}%
\end{pgfscope}%
\begin{pgfscope}%
\pgfsys@transformshift{1.384104in}{3.788907in}%
\pgfsys@useobject{currentmarker}{}%
\end{pgfscope}%
\begin{pgfscope}%
\pgfsys@transformshift{1.377835in}{3.816316in}%
\pgfsys@useobject{currentmarker}{}%
\end{pgfscope}%
\begin{pgfscope}%
\pgfsys@transformshift{1.386234in}{3.848020in}%
\pgfsys@useobject{currentmarker}{}%
\end{pgfscope}%
\begin{pgfscope}%
\pgfsys@transformshift{1.386803in}{3.866050in}%
\pgfsys@useobject{currentmarker}{}%
\end{pgfscope}%
\begin{pgfscope}%
\pgfsys@transformshift{1.379194in}{3.886512in}%
\pgfsys@useobject{currentmarker}{}%
\end{pgfscope}%
\begin{pgfscope}%
\pgfsys@transformshift{1.386116in}{3.910287in}%
\pgfsys@useobject{currentmarker}{}%
\end{pgfscope}%
\begin{pgfscope}%
\pgfsys@transformshift{1.393234in}{3.934872in}%
\pgfsys@useobject{currentmarker}{}%
\end{pgfscope}%
\begin{pgfscope}%
\pgfsys@transformshift{1.396721in}{3.948511in}%
\pgfsys@useobject{currentmarker}{}%
\end{pgfscope}%
\begin{pgfscope}%
\pgfsys@transformshift{1.397531in}{3.956211in}%
\pgfsys@useobject{currentmarker}{}%
\end{pgfscope}%
\begin{pgfscope}%
\pgfsys@transformshift{1.397714in}{3.960466in}%
\pgfsys@useobject{currentmarker}{}%
\end{pgfscope}%
\begin{pgfscope}%
\pgfsys@transformshift{1.397280in}{3.962767in}%
\pgfsys@useobject{currentmarker}{}%
\end{pgfscope}%
\begin{pgfscope}%
\pgfsys@transformshift{1.396717in}{3.963926in}%
\pgfsys@useobject{currentmarker}{}%
\end{pgfscope}%
\begin{pgfscope}%
\pgfsys@transformshift{1.395425in}{3.965875in}%
\pgfsys@useobject{currentmarker}{}%
\end{pgfscope}%
\begin{pgfscope}%
\pgfsys@transformshift{1.392673in}{3.967739in}%
\pgfsys@useobject{currentmarker}{}%
\end{pgfscope}%
\begin{pgfscope}%
\pgfsys@transformshift{1.391122in}{3.968706in}%
\pgfsys@useobject{currentmarker}{}%
\end{pgfscope}%
\begin{pgfscope}%
\pgfsys@transformshift{1.388681in}{3.969389in}%
\pgfsys@useobject{currentmarker}{}%
\end{pgfscope}%
\begin{pgfscope}%
\pgfsys@transformshift{1.385693in}{3.970320in}%
\pgfsys@useobject{currentmarker}{}%
\end{pgfscope}%
\begin{pgfscope}%
\pgfsys@transformshift{1.381543in}{3.970682in}%
\pgfsys@useobject{currentmarker}{}%
\end{pgfscope}%
\begin{pgfscope}%
\pgfsys@transformshift{1.376942in}{3.971511in}%
\pgfsys@useobject{currentmarker}{}%
\end{pgfscope}%
\begin{pgfscope}%
\pgfsys@transformshift{1.371907in}{3.970209in}%
\pgfsys@useobject{currentmarker}{}%
\end{pgfscope}%
\begin{pgfscope}%
\pgfsys@transformshift{1.363866in}{3.971527in}%
\pgfsys@useobject{currentmarker}{}%
\end{pgfscope}%
\begin{pgfscope}%
\pgfsys@transformshift{1.359408in}{3.971076in}%
\pgfsys@useobject{currentmarker}{}%
\end{pgfscope}%
\begin{pgfscope}%
\pgfsys@transformshift{1.353823in}{3.971761in}%
\pgfsys@useobject{currentmarker}{}%
\end{pgfscope}%
\begin{pgfscope}%
\pgfsys@transformshift{1.346992in}{3.971158in}%
\pgfsys@useobject{currentmarker}{}%
\end{pgfscope}%
\begin{pgfscope}%
\pgfsys@transformshift{1.343256in}{3.971680in}%
\pgfsys@useobject{currentmarker}{}%
\end{pgfscope}%
\begin{pgfscope}%
\pgfsys@transformshift{1.336430in}{3.971582in}%
\pgfsys@useobject{currentmarker}{}%
\end{pgfscope}%
\begin{pgfscope}%
\pgfsys@transformshift{1.322104in}{3.971247in}%
\pgfsys@useobject{currentmarker}{}%
\end{pgfscope}%
\begin{pgfscope}%
\pgfsys@transformshift{1.303867in}{3.974111in}%
\pgfsys@useobject{currentmarker}{}%
\end{pgfscope}%
\begin{pgfscope}%
\pgfsys@transformshift{1.293717in}{3.974366in}%
\pgfsys@useobject{currentmarker}{}%
\end{pgfscope}%
\begin{pgfscope}%
\pgfsys@transformshift{1.282150in}{3.976472in}%
\pgfsys@useobject{currentmarker}{}%
\end{pgfscope}%
\begin{pgfscope}%
\pgfsys@transformshift{1.267623in}{3.976911in}%
\pgfsys@useobject{currentmarker}{}%
\end{pgfscope}%
\begin{pgfscope}%
\pgfsys@transformshift{1.251804in}{3.978731in}%
\pgfsys@useobject{currentmarker}{}%
\end{pgfscope}%
\begin{pgfscope}%
\pgfsys@transformshift{1.254035in}{3.979210in}%
\pgfsys@useobject{currentmarker}{}%
\end{pgfscope}%
\begin{pgfscope}%
\pgfsys@transformshift{1.274994in}{3.979107in}%
\pgfsys@useobject{currentmarker}{}%
\end{pgfscope}%
\begin{pgfscope}%
\pgfsys@transformshift{1.301462in}{3.975377in}%
\pgfsys@useobject{currentmarker}{}%
\end{pgfscope}%
\begin{pgfscope}%
\pgfsys@transformshift{1.331300in}{3.981080in}%
\pgfsys@useobject{currentmarker}{}%
\end{pgfscope}%
\begin{pgfscope}%
\pgfsys@transformshift{1.362121in}{3.979530in}%
\pgfsys@useobject{currentmarker}{}%
\end{pgfscope}%
\begin{pgfscope}%
\pgfsys@transformshift{1.396497in}{3.981535in}%
\pgfsys@useobject{currentmarker}{}%
\end{pgfscope}%
\begin{pgfscope}%
\pgfsys@transformshift{1.435024in}{3.985441in}%
\pgfsys@useobject{currentmarker}{}%
\end{pgfscope}%
\begin{pgfscope}%
\pgfsys@transformshift{1.474497in}{3.989546in}%
\pgfsys@useobject{currentmarker}{}%
\end{pgfscope}%
\begin{pgfscope}%
\pgfsys@transformshift{1.517581in}{3.987235in}%
\pgfsys@useobject{currentmarker}{}%
\end{pgfscope}%
\begin{pgfscope}%
\pgfsys@transformshift{1.566715in}{3.991626in}%
\pgfsys@useobject{currentmarker}{}%
\end{pgfscope}%
\begin{pgfscope}%
\pgfsys@transformshift{1.616963in}{3.996460in}%
\pgfsys@useobject{currentmarker}{}%
\end{pgfscope}%
\begin{pgfscope}%
\pgfsys@transformshift{1.670850in}{4.002322in}%
\pgfsys@useobject{currentmarker}{}%
\end{pgfscope}%
\begin{pgfscope}%
\pgfsys@transformshift{1.729996in}{4.001404in}%
\pgfsys@useobject{currentmarker}{}%
\end{pgfscope}%
\begin{pgfscope}%
\pgfsys@transformshift{1.791113in}{4.007141in}%
\pgfsys@useobject{currentmarker}{}%
\end{pgfscope}%
\begin{pgfscope}%
\pgfsys@transformshift{1.853524in}{4.008851in}%
\pgfsys@useobject{currentmarker}{}%
\end{pgfscope}%
\begin{pgfscope}%
\pgfsys@transformshift{1.919046in}{4.006906in}%
\pgfsys@useobject{currentmarker}{}%
\end{pgfscope}%
\begin{pgfscope}%
\pgfsys@transformshift{1.990102in}{3.987200in}%
\pgfsys@useobject{currentmarker}{}%
\end{pgfscope}%
\begin{pgfscope}%
\pgfsys@transformshift{2.063446in}{4.011510in}%
\pgfsys@useobject{currentmarker}{}%
\end{pgfscope}%
\begin{pgfscope}%
\pgfsys@transformshift{2.141423in}{4.020244in}%
\pgfsys@useobject{currentmarker}{}%
\end{pgfscope}%
\begin{pgfscope}%
\pgfsys@transformshift{2.221030in}{4.036535in}%
\pgfsys@useobject{currentmarker}{}%
\end{pgfscope}%
\begin{pgfscope}%
\pgfsys@transformshift{2.304591in}{4.041848in}%
\pgfsys@useobject{currentmarker}{}%
\end{pgfscope}%
\begin{pgfscope}%
\pgfsys@transformshift{2.350625in}{4.040591in}%
\pgfsys@useobject{currentmarker}{}%
\end{pgfscope}%
\begin{pgfscope}%
\pgfsys@transformshift{2.375948in}{4.040068in}%
\pgfsys@useobject{currentmarker}{}%
\end{pgfscope}%
\begin{pgfscope}%
\pgfsys@transformshift{2.405282in}{4.042348in}%
\pgfsys@useobject{currentmarker}{}%
\end{pgfscope}%
\begin{pgfscope}%
\pgfsys@transformshift{2.437168in}{4.043000in}%
\pgfsys@useobject{currentmarker}{}%
\end{pgfscope}%
\begin{pgfscope}%
\pgfsys@transformshift{2.470627in}{4.039538in}%
\pgfsys@useobject{currentmarker}{}%
\end{pgfscope}%
\begin{pgfscope}%
\pgfsys@transformshift{2.504877in}{4.033691in}%
\pgfsys@useobject{currentmarker}{}%
\end{pgfscope}%
\begin{pgfscope}%
\pgfsys@transformshift{2.540203in}{4.031098in}%
\pgfsys@useobject{currentmarker}{}%
\end{pgfscope}%
\begin{pgfscope}%
\pgfsys@transformshift{2.577003in}{4.027361in}%
\pgfsys@useobject{currentmarker}{}%
\end{pgfscope}%
\begin{pgfscope}%
\pgfsys@transformshift{2.614824in}{4.028011in}%
\pgfsys@useobject{currentmarker}{}%
\end{pgfscope}%
\begin{pgfscope}%
\pgfsys@transformshift{2.653925in}{4.023428in}%
\pgfsys@useobject{currentmarker}{}%
\end{pgfscope}%
\begin{pgfscope}%
\pgfsys@transformshift{2.694025in}{4.006116in}%
\pgfsys@useobject{currentmarker}{}%
\end{pgfscope}%
\begin{pgfscope}%
\pgfsys@transformshift{2.741970in}{4.008515in}%
\pgfsys@useobject{currentmarker}{}%
\end{pgfscope}%
\begin{pgfscope}%
\pgfsys@transformshift{2.787517in}{3.990538in}%
\pgfsys@useobject{currentmarker}{}%
\end{pgfscope}%
\begin{pgfscope}%
\pgfsys@transformshift{2.833961in}{4.010157in}%
\pgfsys@useobject{currentmarker}{}%
\end{pgfscope}%
\begin{pgfscope}%
\pgfsys@transformshift{2.887570in}{4.000264in}%
\pgfsys@useobject{currentmarker}{}%
\end{pgfscope}%
\begin{pgfscope}%
\pgfsys@transformshift{2.943059in}{4.005563in}%
\pgfsys@useobject{currentmarker}{}%
\end{pgfscope}%
\begin{pgfscope}%
\pgfsys@transformshift{3.002342in}{4.009348in}%
\pgfsys@useobject{currentmarker}{}%
\end{pgfscope}%
\begin{pgfscope}%
\pgfsys@transformshift{3.064749in}{4.009328in}%
\pgfsys@useobject{currentmarker}{}%
\end{pgfscope}%
\begin{pgfscope}%
\pgfsys@transformshift{3.128666in}{4.008449in}%
\pgfsys@useobject{currentmarker}{}%
\end{pgfscope}%
\begin{pgfscope}%
\pgfsys@transformshift{3.196682in}{4.002066in}%
\pgfsys@useobject{currentmarker}{}%
\end{pgfscope}%
\begin{pgfscope}%
\pgfsys@transformshift{3.269852in}{3.984188in}%
\pgfsys@useobject{currentmarker}{}%
\end{pgfscope}%
\begin{pgfscope}%
\pgfsys@transformshift{3.347331in}{3.981686in}%
\pgfsys@useobject{currentmarker}{}%
\end{pgfscope}%
\begin{pgfscope}%
\pgfsys@transformshift{3.425783in}{3.980439in}%
\pgfsys@useobject{currentmarker}{}%
\end{pgfscope}%
\begin{pgfscope}%
\pgfsys@transformshift{3.468852in}{3.983132in}%
\pgfsys@useobject{currentmarker}{}%
\end{pgfscope}%
\begin{pgfscope}%
\pgfsys@transformshift{3.510071in}{3.999036in}%
\pgfsys@useobject{currentmarker}{}%
\end{pgfscope}%
\begin{pgfscope}%
\pgfsys@transformshift{3.555812in}{4.003488in}%
\pgfsys@useobject{currentmarker}{}%
\end{pgfscope}%
\begin{pgfscope}%
\pgfsys@transformshift{3.581006in}{4.001451in}%
\pgfsys@useobject{currentmarker}{}%
\end{pgfscope}%
\begin{pgfscope}%
\pgfsys@transformshift{3.610156in}{3.997794in}%
\pgfsys@useobject{currentmarker}{}%
\end{pgfscope}%
\begin{pgfscope}%
\pgfsys@transformshift{3.644710in}{3.988020in}%
\pgfsys@useobject{currentmarker}{}%
\end{pgfscope}%
\begin{pgfscope}%
\pgfsys@transformshift{3.683094in}{3.988391in}%
\pgfsys@useobject{currentmarker}{}%
\end{pgfscope}%
\begin{pgfscope}%
\pgfsys@transformshift{3.721617in}{3.994483in}%
\pgfsys@useobject{currentmarker}{}%
\end{pgfscope}%
\begin{pgfscope}%
\pgfsys@transformshift{3.762273in}{3.995280in}%
\pgfsys@useobject{currentmarker}{}%
\end{pgfscope}%
\begin{pgfscope}%
\pgfsys@transformshift{3.804233in}{4.001661in}%
\pgfsys@useobject{currentmarker}{}%
\end{pgfscope}%
\begin{pgfscope}%
\pgfsys@transformshift{3.847542in}{3.998243in}%
\pgfsys@useobject{currentmarker}{}%
\end{pgfscope}%
\begin{pgfscope}%
\pgfsys@transformshift{3.871353in}{4.000245in}%
\pgfsys@useobject{currentmarker}{}%
\end{pgfscope}%
\begin{pgfscope}%
\pgfsys@transformshift{3.897839in}{4.000359in}%
\pgfsys@useobject{currentmarker}{}%
\end{pgfscope}%
\begin{pgfscope}%
\pgfsys@transformshift{3.929191in}{3.992475in}%
\pgfsys@useobject{currentmarker}{}%
\end{pgfscope}%
\begin{pgfscope}%
\pgfsys@transformshift{3.962507in}{4.003331in}%
\pgfsys@useobject{currentmarker}{}%
\end{pgfscope}%
\begin{pgfscope}%
\pgfsys@transformshift{3.998683in}{3.998121in}%
\pgfsys@useobject{currentmarker}{}%
\end{pgfscope}%
\begin{pgfscope}%
\pgfsys@transformshift{4.037746in}{4.003300in}%
\pgfsys@useobject{currentmarker}{}%
\end{pgfscope}%
\begin{pgfscope}%
\pgfsys@transformshift{4.082887in}{4.010024in}%
\pgfsys@useobject{currentmarker}{}%
\end{pgfscope}%
\begin{pgfscope}%
\pgfsys@transformshift{4.132843in}{4.013381in}%
\pgfsys@useobject{currentmarker}{}%
\end{pgfscope}%
\begin{pgfscope}%
\pgfsys@transformshift{4.160350in}{4.012080in}%
\pgfsys@useobject{currentmarker}{}%
\end{pgfscope}%
\begin{pgfscope}%
\pgfsys@transformshift{4.192353in}{4.005075in}%
\pgfsys@useobject{currentmarker}{}%
\end{pgfscope}%
\begin{pgfscope}%
\pgfsys@transformshift{4.228679in}{3.998986in}%
\pgfsys@useobject{currentmarker}{}%
\end{pgfscope}%
\begin{pgfscope}%
\pgfsys@transformshift{4.248851in}{3.997120in}%
\pgfsys@useobject{currentmarker}{}%
\end{pgfscope}%
\begin{pgfscope}%
\pgfsys@transformshift{4.271111in}{3.992198in}%
\pgfsys@useobject{currentmarker}{}%
\end{pgfscope}%
\begin{pgfscope}%
\pgfsys@transformshift{4.283361in}{3.989525in}%
\pgfsys@useobject{currentmarker}{}%
\end{pgfscope}%
\begin{pgfscope}%
\pgfsys@transformshift{4.294530in}{3.981518in}%
\pgfsys@useobject{currentmarker}{}%
\end{pgfscope}%
\begin{pgfscope}%
\pgfsys@transformshift{4.299927in}{3.976226in}%
\pgfsys@useobject{currentmarker}{}%
\end{pgfscope}%
\begin{pgfscope}%
\pgfsys@transformshift{4.302597in}{3.968314in}%
\pgfsys@useobject{currentmarker}{}%
\end{pgfscope}%
\begin{pgfscope}%
\pgfsys@transformshift{4.305716in}{3.959593in}%
\pgfsys@useobject{currentmarker}{}%
\end{pgfscope}%
\begin{pgfscope}%
\pgfsys@transformshift{4.305522in}{3.949095in}%
\pgfsys@useobject{currentmarker}{}%
\end{pgfscope}%
\begin{pgfscope}%
\pgfsys@transformshift{4.306636in}{3.937293in}%
\pgfsys@useobject{currentmarker}{}%
\end{pgfscope}%
\begin{pgfscope}%
\pgfsys@transformshift{4.304527in}{3.925074in}%
\pgfsys@useobject{currentmarker}{}%
\end{pgfscope}%
\begin{pgfscope}%
\pgfsys@transformshift{4.300578in}{3.910558in}%
\pgfsys@useobject{currentmarker}{}%
\end{pgfscope}%
\begin{pgfscope}%
\pgfsys@transformshift{4.298874in}{3.894423in}%
\pgfsys@useobject{currentmarker}{}%
\end{pgfscope}%
\begin{pgfscope}%
\pgfsys@transformshift{4.296761in}{3.877768in}%
\pgfsys@useobject{currentmarker}{}%
\end{pgfscope}%
\begin{pgfscope}%
\pgfsys@transformshift{4.296832in}{3.868535in}%
\pgfsys@useobject{currentmarker}{}%
\end{pgfscope}%
\begin{pgfscope}%
\pgfsys@transformshift{4.297479in}{3.858564in}%
\pgfsys@useobject{currentmarker}{}%
\end{pgfscope}%
\begin{pgfscope}%
\pgfsys@transformshift{4.294598in}{3.848333in}%
\pgfsys@useobject{currentmarker}{}%
\end{pgfscope}%
\begin{pgfscope}%
\pgfsys@transformshift{4.297643in}{3.837207in}%
\pgfsys@useobject{currentmarker}{}%
\end{pgfscope}%
\begin{pgfscope}%
\pgfsys@transformshift{4.299092in}{3.831031in}%
\pgfsys@useobject{currentmarker}{}%
\end{pgfscope}%
\begin{pgfscope}%
\pgfsys@transformshift{4.295425in}{3.820531in}%
\pgfsys@useobject{currentmarker}{}%
\end{pgfscope}%
\begin{pgfscope}%
\pgfsys@transformshift{4.298468in}{3.815225in}%
\pgfsys@useobject{currentmarker}{}%
\end{pgfscope}%
\begin{pgfscope}%
\pgfsys@transformshift{4.302183in}{3.805978in}%
\pgfsys@useobject{currentmarker}{}%
\end{pgfscope}%
\begin{pgfscope}%
\pgfsys@transformshift{4.300634in}{3.792840in}%
\pgfsys@useobject{currentmarker}{}%
\end{pgfscope}%
\begin{pgfscope}%
\pgfsys@transformshift{4.298320in}{3.778441in}%
\pgfsys@useobject{currentmarker}{}%
\end{pgfscope}%
\begin{pgfscope}%
\pgfsys@transformshift{4.302082in}{3.763864in}%
\pgfsys@useobject{currentmarker}{}%
\end{pgfscope}%
\begin{pgfscope}%
\pgfsys@transformshift{4.306147in}{3.747595in}%
\pgfsys@useobject{currentmarker}{}%
\end{pgfscope}%
\begin{pgfscope}%
\pgfsys@transformshift{4.304920in}{3.725533in}%
\pgfsys@useobject{currentmarker}{}%
\end{pgfscope}%
\begin{pgfscope}%
\pgfsys@transformshift{4.303401in}{3.713475in}%
\pgfsys@useobject{currentmarker}{}%
\end{pgfscope}%
\begin{pgfscope}%
\pgfsys@transformshift{4.305461in}{3.707117in}%
\pgfsys@useobject{currentmarker}{}%
\end{pgfscope}%
\begin{pgfscope}%
\pgfsys@transformshift{4.307802in}{3.698811in}%
\pgfsys@useobject{currentmarker}{}%
\end{pgfscope}%
\begin{pgfscope}%
\pgfsys@transformshift{4.305046in}{3.684873in}%
\pgfsys@useobject{currentmarker}{}%
\end{pgfscope}%
\begin{pgfscope}%
\pgfsys@transformshift{4.307259in}{3.677379in}%
\pgfsys@useobject{currentmarker}{}%
\end{pgfscope}%
\begin{pgfscope}%
\pgfsys@transformshift{4.306905in}{3.673096in}%
\pgfsys@useobject{currentmarker}{}%
\end{pgfscope}%
\begin{pgfscope}%
\pgfsys@transformshift{4.307949in}{3.664423in}%
\pgfsys@useobject{currentmarker}{}%
\end{pgfscope}%
\begin{pgfscope}%
\pgfsys@transformshift{4.305642in}{3.653929in}%
\pgfsys@useobject{currentmarker}{}%
\end{pgfscope}%
\begin{pgfscope}%
\pgfsys@transformshift{4.309617in}{3.641855in}%
\pgfsys@useobject{currentmarker}{}%
\end{pgfscope}%
\begin{pgfscope}%
\pgfsys@transformshift{4.315683in}{3.626838in}%
\pgfsys@useobject{currentmarker}{}%
\end{pgfscope}%
\begin{pgfscope}%
\pgfsys@transformshift{4.314611in}{3.606127in}%
\pgfsys@useobject{currentmarker}{}%
\end{pgfscope}%
\begin{pgfscope}%
\pgfsys@transformshift{4.318686in}{3.595474in}%
\pgfsys@useobject{currentmarker}{}%
\end{pgfscope}%
\begin{pgfscope}%
\pgfsys@transformshift{4.325277in}{3.584263in}%
\pgfsys@useobject{currentmarker}{}%
\end{pgfscope}%
\begin{pgfscope}%
\pgfsys@transformshift{4.325406in}{3.565434in}%
\pgfsys@useobject{currentmarker}{}%
\end{pgfscope}%
\begin{pgfscope}%
\pgfsys@transformshift{4.327110in}{3.546176in}%
\pgfsys@useobject{currentmarker}{}%
\end{pgfscope}%
\begin{pgfscope}%
\pgfsys@transformshift{4.331295in}{3.525234in}%
\pgfsys@useobject{currentmarker}{}%
\end{pgfscope}%
\begin{pgfscope}%
\pgfsys@transformshift{4.340988in}{3.504983in}%
\pgfsys@useobject{currentmarker}{}%
\end{pgfscope}%
\begin{pgfscope}%
\pgfsys@transformshift{4.339445in}{3.476555in}%
\pgfsys@useobject{currentmarker}{}%
\end{pgfscope}%
\begin{pgfscope}%
\pgfsys@transformshift{4.351838in}{3.450221in}%
\pgfsys@useobject{currentmarker}{}%
\end{pgfscope}%
\begin{pgfscope}%
\pgfsys@transformshift{4.362074in}{3.421943in}%
\pgfsys@useobject{currentmarker}{}%
\end{pgfscope}%
\begin{pgfscope}%
\pgfsys@transformshift{4.364448in}{3.386033in}%
\pgfsys@useobject{currentmarker}{}%
\end{pgfscope}%
\begin{pgfscope}%
\pgfsys@transformshift{4.364910in}{3.347489in}%
\pgfsys@useobject{currentmarker}{}%
\end{pgfscope}%
\begin{pgfscope}%
\pgfsys@transformshift{4.377442in}{3.306136in}%
\pgfsys@useobject{currentmarker}{}%
\end{pgfscope}%
\begin{pgfscope}%
\pgfsys@transformshift{4.369690in}{3.261083in}%
\pgfsys@useobject{currentmarker}{}%
\end{pgfscope}%
\begin{pgfscope}%
\pgfsys@transformshift{4.368203in}{3.213280in}%
\pgfsys@useobject{currentmarker}{}%
\end{pgfscope}%
\begin{pgfscope}%
\pgfsys@transformshift{4.390153in}{3.168381in}%
\pgfsys@useobject{currentmarker}{}%
\end{pgfscope}%
\begin{pgfscope}%
\pgfsys@transformshift{4.409983in}{3.120964in}%
\pgfsys@useobject{currentmarker}{}%
\end{pgfscope}%
\begin{pgfscope}%
\pgfsys@transformshift{4.402282in}{3.066175in}%
\pgfsys@useobject{currentmarker}{}%
\end{pgfscope}%
\begin{pgfscope}%
\pgfsys@transformshift{4.419274in}{3.012522in}%
\pgfsys@useobject{currentmarker}{}%
\end{pgfscope}%
\begin{pgfscope}%
\pgfsys@transformshift{4.448654in}{2.962151in}%
\pgfsys@useobject{currentmarker}{}%
\end{pgfscope}%
\begin{pgfscope}%
\pgfsys@transformshift{4.436551in}{2.899705in}%
\pgfsys@useobject{currentmarker}{}%
\end{pgfscope}%
\begin{pgfscope}%
\pgfsys@transformshift{4.437409in}{2.832800in}%
\pgfsys@useobject{currentmarker}{}%
\end{pgfscope}%
\begin{pgfscope}%
\pgfsys@transformshift{4.452934in}{2.766789in}%
\pgfsys@useobject{currentmarker}{}%
\end{pgfscope}%
\begin{pgfscope}%
\pgfsys@transformshift{4.475363in}{2.700285in}%
\pgfsys@useobject{currentmarker}{}%
\end{pgfscope}%
\begin{pgfscope}%
\pgfsys@transformshift{4.462465in}{2.626340in}%
\pgfsys@useobject{currentmarker}{}%
\end{pgfscope}%
\begin{pgfscope}%
\pgfsys@transformshift{4.501316in}{2.561515in}%
\pgfsys@useobject{currentmarker}{}%
\end{pgfscope}%
\begin{pgfscope}%
\pgfsys@transformshift{4.535700in}{2.492384in}%
\pgfsys@useobject{currentmarker}{}%
\end{pgfscope}%
\begin{pgfscope}%
\pgfsys@transformshift{4.539784in}{2.410638in}%
\pgfsys@useobject{currentmarker}{}%
\end{pgfscope}%
\begin{pgfscope}%
\pgfsys@transformshift{4.515779in}{2.327072in}%
\pgfsys@useobject{currentmarker}{}%
\end{pgfscope}%
\begin{pgfscope}%
\pgfsys@transformshift{4.544751in}{2.289028in}%
\pgfsys@useobject{currentmarker}{}%
\end{pgfscope}%
\begin{pgfscope}%
\pgfsys@transformshift{4.576654in}{2.249194in}%
\pgfsys@useobject{currentmarker}{}%
\end{pgfscope}%
\begin{pgfscope}%
\pgfsys@transformshift{4.577734in}{2.193625in}%
\pgfsys@useobject{currentmarker}{}%
\end{pgfscope}%
\begin{pgfscope}%
\pgfsys@transformshift{4.585634in}{2.164094in}%
\pgfsys@useobject{currentmarker}{}%
\end{pgfscope}%
\begin{pgfscope}%
\pgfsys@transformshift{4.588015in}{2.129798in}%
\pgfsys@useobject{currentmarker}{}%
\end{pgfscope}%
\begin{pgfscope}%
\pgfsys@transformshift{4.590570in}{2.111063in}%
\pgfsys@useobject{currentmarker}{}%
\end{pgfscope}%
\begin{pgfscope}%
\pgfsys@transformshift{4.587957in}{2.086909in}%
\pgfsys@useobject{currentmarker}{}%
\end{pgfscope}%
\begin{pgfscope}%
\pgfsys@transformshift{4.600516in}{2.064989in}%
\pgfsys@useobject{currentmarker}{}%
\end{pgfscope}%
\begin{pgfscope}%
\pgfsys@transformshift{4.614228in}{2.042132in}%
\pgfsys@useobject{currentmarker}{}%
\end{pgfscope}%
\begin{pgfscope}%
\pgfsys@transformshift{4.605166in}{2.011366in}%
\pgfsys@useobject{currentmarker}{}%
\end{pgfscope}%
\begin{pgfscope}%
\pgfsys@transformshift{4.596346in}{1.977206in}%
\pgfsys@useobject{currentmarker}{}%
\end{pgfscope}%
\begin{pgfscope}%
\pgfsys@transformshift{4.609604in}{1.943484in}%
\pgfsys@useobject{currentmarker}{}%
\end{pgfscope}%
\begin{pgfscope}%
\pgfsys@transformshift{4.627173in}{1.910528in}%
\pgfsys@useobject{currentmarker}{}%
\end{pgfscope}%
\begin{pgfscope}%
\pgfsys@transformshift{4.624092in}{1.869767in}%
\pgfsys@useobject{currentmarker}{}%
\end{pgfscope}%
\begin{pgfscope}%
\pgfsys@transformshift{4.629431in}{1.847928in}%
\pgfsys@useobject{currentmarker}{}%
\end{pgfscope}%
\begin{pgfscope}%
\pgfsys@transformshift{4.641815in}{1.824579in}%
\pgfsys@useobject{currentmarker}{}%
\end{pgfscope}%
\begin{pgfscope}%
\pgfsys@transformshift{4.639703in}{1.795459in}%
\pgfsys@useobject{currentmarker}{}%
\end{pgfscope}%
\begin{pgfscope}%
\pgfsys@transformshift{4.639641in}{1.764181in}%
\pgfsys@useobject{currentmarker}{}%
\end{pgfscope}%
\begin{pgfscope}%
\pgfsys@transformshift{4.645788in}{1.725956in}%
\pgfsys@useobject{currentmarker}{}%
\end{pgfscope}%
\begin{pgfscope}%
\pgfsys@transformshift{4.648361in}{1.704819in}%
\pgfsys@useobject{currentmarker}{}%
\end{pgfscope}%
\begin{pgfscope}%
\pgfsys@transformshift{4.644517in}{1.679581in}%
\pgfsys@useobject{currentmarker}{}%
\end{pgfscope}%
\begin{pgfscope}%
\pgfsys@transformshift{4.661597in}{1.659936in}%
\pgfsys@useobject{currentmarker}{}%
\end{pgfscope}%
\begin{pgfscope}%
\pgfsys@transformshift{4.678050in}{1.636234in}%
\pgfsys@useobject{currentmarker}{}%
\end{pgfscope}%
\begin{pgfscope}%
\pgfsys@transformshift{4.684630in}{1.602585in}%
\pgfsys@useobject{currentmarker}{}%
\end{pgfscope}%
\begin{pgfscope}%
\pgfsys@transformshift{4.688669in}{1.584164in}%
\pgfsys@useobject{currentmarker}{}%
\end{pgfscope}%
\begin{pgfscope}%
\pgfsys@transformshift{4.698661in}{1.564025in}%
\pgfsys@useobject{currentmarker}{}%
\end{pgfscope}%
\begin{pgfscope}%
\pgfsys@transformshift{4.698104in}{1.539548in}%
\pgfsys@useobject{currentmarker}{}%
\end{pgfscope}%
\begin{pgfscope}%
\pgfsys@transformshift{4.694065in}{1.512417in}%
\pgfsys@useobject{currentmarker}{}%
\end{pgfscope}%
\begin{pgfscope}%
\pgfsys@transformshift{4.704736in}{1.484114in}%
\pgfsys@useobject{currentmarker}{}%
\end{pgfscope}%
\begin{pgfscope}%
\pgfsys@transformshift{4.718989in}{1.454967in}%
\pgfsys@useobject{currentmarker}{}%
\end{pgfscope}%
\begin{pgfscope}%
\pgfsys@transformshift{4.717694in}{1.419440in}%
\pgfsys@useobject{currentmarker}{}%
\end{pgfscope}%
\begin{pgfscope}%
\pgfsys@transformshift{4.724847in}{1.383798in}%
\pgfsys@useobject{currentmarker}{}%
\end{pgfscope}%
\begin{pgfscope}%
\pgfsys@transformshift{4.742165in}{1.350872in}%
\pgfsys@useobject{currentmarker}{}%
\end{pgfscope}%
\begin{pgfscope}%
\pgfsys@transformshift{4.742139in}{1.309581in}%
\pgfsys@useobject{currentmarker}{}%
\end{pgfscope}%
\begin{pgfscope}%
\pgfsys@transformshift{4.749171in}{1.267691in}%
\pgfsys@useobject{currentmarker}{}%
\end{pgfscope}%
\begin{pgfscope}%
\pgfsys@transformshift{4.755854in}{1.225133in}%
\pgfsys@useobject{currentmarker}{}%
\end{pgfscope}%
\begin{pgfscope}%
\pgfsys@transformshift{4.764697in}{1.182440in}%
\pgfsys@useobject{currentmarker}{}%
\end{pgfscope}%
\begin{pgfscope}%
\pgfsys@transformshift{4.767403in}{1.158614in}%
\pgfsys@useobject{currentmarker}{}%
\end{pgfscope}%
\begin{pgfscope}%
\pgfsys@transformshift{4.770547in}{1.134358in}%
\pgfsys@useobject{currentmarker}{}%
\end{pgfscope}%
\begin{pgfscope}%
\pgfsys@transformshift{4.774165in}{1.109661in}%
\pgfsys@useobject{currentmarker}{}%
\end{pgfscope}%
\begin{pgfscope}%
\pgfsys@transformshift{4.776954in}{1.084395in}%
\pgfsys@useobject{currentmarker}{}%
\end{pgfscope}%
\begin{pgfscope}%
\pgfsys@transformshift{4.772974in}{1.070992in}%
\pgfsys@useobject{currentmarker}{}%
\end{pgfscope}%
\begin{pgfscope}%
\pgfsys@transformshift{4.756023in}{1.060980in}%
\pgfsys@useobject{currentmarker}{}%
\end{pgfscope}%
\begin{pgfscope}%
\pgfsys@transformshift{4.735152in}{1.060709in}%
\pgfsys@useobject{currentmarker}{}%
\end{pgfscope}%
\begin{pgfscope}%
\pgfsys@transformshift{4.713402in}{1.060393in}%
\pgfsys@useobject{currentmarker}{}%
\end{pgfscope}%
\begin{pgfscope}%
\pgfsys@transformshift{4.688192in}{1.065166in}%
\pgfsys@useobject{currentmarker}{}%
\end{pgfscope}%
\begin{pgfscope}%
\pgfsys@transformshift{4.661750in}{1.067419in}%
\pgfsys@useobject{currentmarker}{}%
\end{pgfscope}%
\begin{pgfscope}%
\pgfsys@transformshift{4.647506in}{1.070606in}%
\pgfsys@useobject{currentmarker}{}%
\end{pgfscope}%
\begin{pgfscope}%
\pgfsys@transformshift{4.631810in}{1.070354in}%
\pgfsys@useobject{currentmarker}{}%
\end{pgfscope}%
\begin{pgfscope}%
\pgfsys@transformshift{4.616658in}{1.076218in}%
\pgfsys@useobject{currentmarker}{}%
\end{pgfscope}%
\begin{pgfscope}%
\pgfsys@transformshift{4.599534in}{1.079132in}%
\pgfsys@useobject{currentmarker}{}%
\end{pgfscope}%
\begin{pgfscope}%
\pgfsys@transformshift{4.582491in}{1.085944in}%
\pgfsys@useobject{currentmarker}{}%
\end{pgfscope}%
\begin{pgfscope}%
\pgfsys@transformshift{4.563514in}{1.088252in}%
\pgfsys@useobject{currentmarker}{}%
\end{pgfscope}%
\begin{pgfscope}%
\pgfsys@transformshift{4.554119in}{1.092972in}%
\pgfsys@useobject{currentmarker}{}%
\end{pgfscope}%
\begin{pgfscope}%
\pgfsys@transformshift{4.542698in}{1.093499in}%
\pgfsys@useobject{currentmarker}{}%
\end{pgfscope}%
\begin{pgfscope}%
\pgfsys@transformshift{4.531705in}{1.098734in}%
\pgfsys@useobject{currentmarker}{}%
\end{pgfscope}%
\begin{pgfscope}%
\pgfsys@transformshift{4.518513in}{1.100594in}%
\pgfsys@useobject{currentmarker}{}%
\end{pgfscope}%
\begin{pgfscope}%
\pgfsys@transformshift{4.511203in}{1.101097in}%
\pgfsys@useobject{currentmarker}{}%
\end{pgfscope}%
\begin{pgfscope}%
\pgfsys@transformshift{4.500966in}{1.098920in}%
\pgfsys@useobject{currentmarker}{}%
\end{pgfscope}%
\begin{pgfscope}%
\pgfsys@transformshift{4.495246in}{1.099565in}%
\pgfsys@useobject{currentmarker}{}%
\end{pgfscope}%
\begin{pgfscope}%
\pgfsys@transformshift{4.481552in}{1.099129in}%
\pgfsys@useobject{currentmarker}{}%
\end{pgfscope}%
\begin{pgfscope}%
\pgfsys@transformshift{4.466164in}{1.098030in}%
\pgfsys@useobject{currentmarker}{}%
\end{pgfscope}%
\begin{pgfscope}%
\pgfsys@transformshift{4.449387in}{1.094520in}%
\pgfsys@useobject{currentmarker}{}%
\end{pgfscope}%
\begin{pgfscope}%
\pgfsys@transformshift{4.430787in}{1.094203in}%
\pgfsys@useobject{currentmarker}{}%
\end{pgfscope}%
\begin{pgfscope}%
\pgfsys@transformshift{4.404852in}{1.092196in}%
\pgfsys@useobject{currentmarker}{}%
\end{pgfscope}%
\begin{pgfscope}%
\pgfsys@transformshift{4.390571in}{1.093051in}%
\pgfsys@useobject{currentmarker}{}%
\end{pgfscope}%
\begin{pgfscope}%
\pgfsys@transformshift{4.376243in}{1.085570in}%
\pgfsys@useobject{currentmarker}{}%
\end{pgfscope}%
\begin{pgfscope}%
\pgfsys@transformshift{4.367441in}{1.084322in}%
\pgfsys@useobject{currentmarker}{}%
\end{pgfscope}%
\begin{pgfscope}%
\pgfsys@transformshift{4.349905in}{1.089520in}%
\pgfsys@useobject{currentmarker}{}%
\end{pgfscope}%
\begin{pgfscope}%
\pgfsys@transformshift{4.328618in}{1.089753in}%
\pgfsys@useobject{currentmarker}{}%
\end{pgfscope}%
\begin{pgfscope}%
\pgfsys@transformshift{4.308049in}{1.080022in}%
\pgfsys@useobject{currentmarker}{}%
\end{pgfscope}%
\begin{pgfscope}%
\pgfsys@transformshift{4.284254in}{1.079554in}%
\pgfsys@useobject{currentmarker}{}%
\end{pgfscope}%
\begin{pgfscope}%
\pgfsys@transformshift{4.253722in}{1.082771in}%
\pgfsys@useobject{currentmarker}{}%
\end{pgfscope}%
\begin{pgfscope}%
\pgfsys@transformshift{4.220290in}{1.081029in}%
\pgfsys@useobject{currentmarker}{}%
\end{pgfscope}%
\begin{pgfscope}%
\pgfsys@transformshift{4.184289in}{1.070441in}%
\pgfsys@useobject{currentmarker}{}%
\end{pgfscope}%
\begin{pgfscope}%
\pgfsys@transformshift{4.144199in}{1.066159in}%
\pgfsys@useobject{currentmarker}{}%
\end{pgfscope}%
\begin{pgfscope}%
\pgfsys@transformshift{4.100687in}{1.064405in}%
\pgfsys@useobject{currentmarker}{}%
\end{pgfscope}%
\begin{pgfscope}%
\pgfsys@transformshift{4.054655in}{1.063496in}%
\pgfsys@useobject{currentmarker}{}%
\end{pgfscope}%
\begin{pgfscope}%
\pgfsys@transformshift{4.006934in}{1.047499in}%
\pgfsys@useobject{currentmarker}{}%
\end{pgfscope}%
\begin{pgfscope}%
\pgfsys@transformshift{3.953775in}{1.039550in}%
\pgfsys@useobject{currentmarker}{}%
\end{pgfscope}%
\begin{pgfscope}%
\pgfsys@transformshift{3.899543in}{1.033151in}%
\pgfsys@useobject{currentmarker}{}%
\end{pgfscope}%
\begin{pgfscope}%
\pgfsys@transformshift{3.844055in}{1.034965in}%
\pgfsys@useobject{currentmarker}{}%
\end{pgfscope}%
\begin{pgfscope}%
\pgfsys@transformshift{3.787418in}{1.031632in}%
\pgfsys@useobject{currentmarker}{}%
\end{pgfscope}%
\begin{pgfscope}%
\pgfsys@transformshift{3.731911in}{1.003711in}%
\pgfsys@useobject{currentmarker}{}%
\end{pgfscope}%
\begin{pgfscope}%
\pgfsys@transformshift{3.667140in}{0.986254in}%
\pgfsys@useobject{currentmarker}{}%
\end{pgfscope}%
\begin{pgfscope}%
\pgfsys@transformshift{3.599731in}{0.978951in}%
\pgfsys@useobject{currentmarker}{}%
\end{pgfscope}%
\begin{pgfscope}%
\pgfsys@transformshift{3.527835in}{0.973519in}%
\pgfsys@useobject{currentmarker}{}%
\end{pgfscope}%
\begin{pgfscope}%
\pgfsys@transformshift{3.452075in}{0.969251in}%
\pgfsys@useobject{currentmarker}{}%
\end{pgfscope}%
\begin{pgfscope}%
\pgfsys@transformshift{3.374985in}{0.949909in}%
\pgfsys@useobject{currentmarker}{}%
\end{pgfscope}%
\begin{pgfscope}%
\pgfsys@transformshift{3.294731in}{0.945013in}%
\pgfsys@useobject{currentmarker}{}%
\end{pgfscope}%
\begin{pgfscope}%
\pgfsys@transformshift{3.209567in}{0.935570in}%
\pgfsys@useobject{currentmarker}{}%
\end{pgfscope}%
\begin{pgfscope}%
\pgfsys@transformshift{3.126409in}{0.903425in}%
\pgfsys@useobject{currentmarker}{}%
\end{pgfscope}%
\begin{pgfscope}%
\pgfsys@transformshift{3.034941in}{0.899089in}%
\pgfsys@useobject{currentmarker}{}%
\end{pgfscope}%
\begin{pgfscope}%
\pgfsys@transformshift{2.942780in}{0.890204in}%
\pgfsys@useobject{currentmarker}{}%
\end{pgfscope}%
\begin{pgfscope}%
\pgfsys@transformshift{2.845619in}{0.892690in}%
\pgfsys@useobject{currentmarker}{}%
\end{pgfscope}%
\begin{pgfscope}%
\pgfsys@transformshift{2.753971in}{0.855043in}%
\pgfsys@useobject{currentmarker}{}%
\end{pgfscope}%
\begin{pgfscope}%
\pgfsys@transformshift{2.651847in}{0.850738in}%
\pgfsys@useobject{currentmarker}{}%
\end{pgfscope}%
\begin{pgfscope}%
\pgfsys@transformshift{2.547764in}{0.842944in}%
\pgfsys@useobject{currentmarker}{}%
\end{pgfscope}%
\begin{pgfscope}%
\pgfsys@transformshift{2.440766in}{0.835383in}%
\pgfsys@useobject{currentmarker}{}%
\end{pgfscope}%
\begin{pgfscope}%
\pgfsys@transformshift{2.335751in}{0.798622in}%
\pgfsys@useobject{currentmarker}{}%
\end{pgfscope}%
\begin{pgfscope}%
\pgfsys@transformshift{2.222003in}{0.789860in}%
\pgfsys@useobject{currentmarker}{}%
\end{pgfscope}%
\begin{pgfscope}%
\pgfsys@transformshift{2.102536in}{0.797856in}%
\pgfsys@useobject{currentmarker}{}%
\end{pgfscope}%
\begin{pgfscope}%
\pgfsys@transformshift{1.981797in}{0.801437in}%
\pgfsys@useobject{currentmarker}{}%
\end{pgfscope}%
\begin{pgfscope}%
\pgfsys@transformshift{1.861984in}{0.770587in}%
\pgfsys@useobject{currentmarker}{}%
\end{pgfscope}%
\begin{pgfscope}%
\pgfsys@transformshift{1.736395in}{0.778906in}%
\pgfsys@useobject{currentmarker}{}%
\end{pgfscope}%
\begin{pgfscope}%
\pgfsys@transformshift{1.606360in}{0.788880in}%
\pgfsys@useobject{currentmarker}{}%
\end{pgfscope}%
\begin{pgfscope}%
\pgfsys@transformshift{1.476250in}{0.809773in}%
\pgfsys@useobject{currentmarker}{}%
\end{pgfscope}%
\begin{pgfscope}%
\pgfsys@transformshift{1.344521in}{0.828357in}%
\pgfsys@useobject{currentmarker}{}%
\end{pgfscope}%
\begin{pgfscope}%
\pgfsys@transformshift{1.272667in}{0.842163in}%
\pgfsys@useobject{currentmarker}{}%
\end{pgfscope}%
\begin{pgfscope}%
\pgfsys@transformshift{1.290814in}{0.846316in}%
\pgfsys@useobject{currentmarker}{}%
\end{pgfscope}%
\begin{pgfscope}%
\pgfsys@transformshift{1.353341in}{0.890229in}%
\pgfsys@useobject{currentmarker}{}%
\end{pgfscope}%
\begin{pgfscope}%
\pgfsys@transformshift{1.424752in}{0.922581in}%
\pgfsys@useobject{currentmarker}{}%
\end{pgfscope}%
\begin{pgfscope}%
\pgfsys@transformshift{1.495226in}{0.961804in}%
\pgfsys@useobject{currentmarker}{}%
\end{pgfscope}%
\begin{pgfscope}%
\pgfsys@transformshift{1.525121in}{0.994578in}%
\pgfsys@useobject{currentmarker}{}%
\end{pgfscope}%
\begin{pgfscope}%
\pgfsys@transformshift{1.558665in}{1.024968in}%
\pgfsys@useobject{currentmarker}{}%
\end{pgfscope}%
\begin{pgfscope}%
\pgfsys@transformshift{1.593300in}{1.056347in}%
\pgfsys@useobject{currentmarker}{}%
\end{pgfscope}%
\end{pgfscope}%
\begin{pgfscope}%
\pgfsetbuttcap%
\pgfsetroundjoin%
\definecolor{currentfill}{rgb}{0.000000,0.000000,0.000000}%
\pgfsetfillcolor{currentfill}%
\pgfsetlinewidth{0.803000pt}%
\definecolor{currentstroke}{rgb}{0.000000,0.000000,0.000000}%
\pgfsetstrokecolor{currentstroke}%
\pgfsetdash{}{0pt}%
\pgfsys@defobject{currentmarker}{\pgfqpoint{0.000000in}{-0.048611in}}{\pgfqpoint{0.000000in}{0.000000in}}{%
\pgfpathmoveto{\pgfqpoint{0.000000in}{0.000000in}}%
\pgfpathlineto{\pgfqpoint{0.000000in}{-0.048611in}}%
\pgfusepath{stroke,fill}%
}%
\begin{pgfscope}%
\pgfsys@transformshift{1.244158in}{0.515000in}%
\pgfsys@useobject{currentmarker}{}%
\end{pgfscope}%
\end{pgfscope}%
\begin{pgfscope}%
\definecolor{textcolor}{rgb}{0.000000,0.000000,0.000000}%
\pgfsetstrokecolor{textcolor}%
\pgfsetfillcolor{textcolor}%
\pgftext[x=1.244158in,y=0.417777in,,top]{\color{textcolor}\rmfamily\fontsize{10.000000}{12.000000}\selectfont \(\displaystyle {0}\)}%
\end{pgfscope}%
\begin{pgfscope}%
\pgfsetbuttcap%
\pgfsetroundjoin%
\definecolor{currentfill}{rgb}{0.000000,0.000000,0.000000}%
\pgfsetfillcolor{currentfill}%
\pgfsetlinewidth{0.803000pt}%
\definecolor{currentstroke}{rgb}{0.000000,0.000000,0.000000}%
\pgfsetstrokecolor{currentstroke}%
\pgfsetdash{}{0pt}%
\pgfsys@defobject{currentmarker}{\pgfqpoint{0.000000in}{-0.048611in}}{\pgfqpoint{0.000000in}{0.000000in}}{%
\pgfpathmoveto{\pgfqpoint{0.000000in}{0.000000in}}%
\pgfpathlineto{\pgfqpoint{0.000000in}{-0.048611in}}%
\pgfusepath{stroke,fill}%
}%
\begin{pgfscope}%
\pgfsys@transformshift{2.151020in}{0.515000in}%
\pgfsys@useobject{currentmarker}{}%
\end{pgfscope}%
\end{pgfscope}%
\begin{pgfscope}%
\definecolor{textcolor}{rgb}{0.000000,0.000000,0.000000}%
\pgfsetstrokecolor{textcolor}%
\pgfsetfillcolor{textcolor}%
\pgftext[x=2.151020in,y=0.417777in,,top]{\color{textcolor}\rmfamily\fontsize{10.000000}{12.000000}\selectfont \(\displaystyle {10}\)}%
\end{pgfscope}%
\begin{pgfscope}%
\pgfsetbuttcap%
\pgfsetroundjoin%
\definecolor{currentfill}{rgb}{0.000000,0.000000,0.000000}%
\pgfsetfillcolor{currentfill}%
\pgfsetlinewidth{0.803000pt}%
\definecolor{currentstroke}{rgb}{0.000000,0.000000,0.000000}%
\pgfsetstrokecolor{currentstroke}%
\pgfsetdash{}{0pt}%
\pgfsys@defobject{currentmarker}{\pgfqpoint{0.000000in}{-0.048611in}}{\pgfqpoint{0.000000in}{0.000000in}}{%
\pgfpathmoveto{\pgfqpoint{0.000000in}{0.000000in}}%
\pgfpathlineto{\pgfqpoint{0.000000in}{-0.048611in}}%
\pgfusepath{stroke,fill}%
}%
\begin{pgfscope}%
\pgfsys@transformshift{3.057882in}{0.515000in}%
\pgfsys@useobject{currentmarker}{}%
\end{pgfscope}%
\end{pgfscope}%
\begin{pgfscope}%
\definecolor{textcolor}{rgb}{0.000000,0.000000,0.000000}%
\pgfsetstrokecolor{textcolor}%
\pgfsetfillcolor{textcolor}%
\pgftext[x=3.057882in,y=0.417777in,,top]{\color{textcolor}\rmfamily\fontsize{10.000000}{12.000000}\selectfont \(\displaystyle {20}\)}%
\end{pgfscope}%
\begin{pgfscope}%
\pgfsetbuttcap%
\pgfsetroundjoin%
\definecolor{currentfill}{rgb}{0.000000,0.000000,0.000000}%
\pgfsetfillcolor{currentfill}%
\pgfsetlinewidth{0.803000pt}%
\definecolor{currentstroke}{rgb}{0.000000,0.000000,0.000000}%
\pgfsetstrokecolor{currentstroke}%
\pgfsetdash{}{0pt}%
\pgfsys@defobject{currentmarker}{\pgfqpoint{0.000000in}{-0.048611in}}{\pgfqpoint{0.000000in}{0.000000in}}{%
\pgfpathmoveto{\pgfqpoint{0.000000in}{0.000000in}}%
\pgfpathlineto{\pgfqpoint{0.000000in}{-0.048611in}}%
\pgfusepath{stroke,fill}%
}%
\begin{pgfscope}%
\pgfsys@transformshift{3.964744in}{0.515000in}%
\pgfsys@useobject{currentmarker}{}%
\end{pgfscope}%
\end{pgfscope}%
\begin{pgfscope}%
\definecolor{textcolor}{rgb}{0.000000,0.000000,0.000000}%
\pgfsetstrokecolor{textcolor}%
\pgfsetfillcolor{textcolor}%
\pgftext[x=3.964744in,y=0.417777in,,top]{\color{textcolor}\rmfamily\fontsize{10.000000}{12.000000}\selectfont \(\displaystyle {30}\)}%
\end{pgfscope}%
\begin{pgfscope}%
\pgfsetbuttcap%
\pgfsetroundjoin%
\definecolor{currentfill}{rgb}{0.000000,0.000000,0.000000}%
\pgfsetfillcolor{currentfill}%
\pgfsetlinewidth{0.803000pt}%
\definecolor{currentstroke}{rgb}{0.000000,0.000000,0.000000}%
\pgfsetstrokecolor{currentstroke}%
\pgfsetdash{}{0pt}%
\pgfsys@defobject{currentmarker}{\pgfqpoint{0.000000in}{-0.048611in}}{\pgfqpoint{0.000000in}{0.000000in}}{%
\pgfpathmoveto{\pgfqpoint{0.000000in}{0.000000in}}%
\pgfpathlineto{\pgfqpoint{0.000000in}{-0.048611in}}%
\pgfusepath{stroke,fill}%
}%
\begin{pgfscope}%
\pgfsys@transformshift{4.871606in}{0.515000in}%
\pgfsys@useobject{currentmarker}{}%
\end{pgfscope}%
\end{pgfscope}%
\begin{pgfscope}%
\definecolor{textcolor}{rgb}{0.000000,0.000000,0.000000}%
\pgfsetstrokecolor{textcolor}%
\pgfsetfillcolor{textcolor}%
\pgftext[x=4.871606in,y=0.417777in,,top]{\color{textcolor}\rmfamily\fontsize{10.000000}{12.000000}\selectfont \(\displaystyle {40}\)}%
\end{pgfscope}%
\begin{pgfscope}%
\definecolor{textcolor}{rgb}{0.000000,0.000000,0.000000}%
\pgfsetstrokecolor{textcolor}%
\pgfsetfillcolor{textcolor}%
\pgftext[x=3.010556in,y=0.238889in,,top]{\color{textcolor}\rmfamily\fontsize{10.000000}{12.000000}\selectfont Position X [\(\displaystyle m\)]}%
\end{pgfscope}%
\begin{pgfscope}%
\pgfsetbuttcap%
\pgfsetroundjoin%
\definecolor{currentfill}{rgb}{0.000000,0.000000,0.000000}%
\pgfsetfillcolor{currentfill}%
\pgfsetlinewidth{0.803000pt}%
\definecolor{currentstroke}{rgb}{0.000000,0.000000,0.000000}%
\pgfsetstrokecolor{currentstroke}%
\pgfsetdash{}{0pt}%
\pgfsys@defobject{currentmarker}{\pgfqpoint{-0.048611in}{0.000000in}}{\pgfqpoint{-0.000000in}{0.000000in}}{%
\pgfpathmoveto{\pgfqpoint{-0.000000in}{0.000000in}}%
\pgfpathlineto{\pgfqpoint{-0.048611in}{0.000000in}}%
\pgfusepath{stroke,fill}%
}%
\begin{pgfscope}%
\pgfsys@transformshift{0.530556in}{0.884903in}%
\pgfsys@useobject{currentmarker}{}%
\end{pgfscope}%
\end{pgfscope}%
\begin{pgfscope}%
\definecolor{textcolor}{rgb}{0.000000,0.000000,0.000000}%
\pgfsetstrokecolor{textcolor}%
\pgfsetfillcolor{textcolor}%
\pgftext[x=0.363889in, y=0.836709in, left, base]{\color{textcolor}\rmfamily\fontsize{10.000000}{12.000000}\selectfont \(\displaystyle {0}\)}%
\end{pgfscope}%
\begin{pgfscope}%
\pgfsetbuttcap%
\pgfsetroundjoin%
\definecolor{currentfill}{rgb}{0.000000,0.000000,0.000000}%
\pgfsetfillcolor{currentfill}%
\pgfsetlinewidth{0.803000pt}%
\definecolor{currentstroke}{rgb}{0.000000,0.000000,0.000000}%
\pgfsetstrokecolor{currentstroke}%
\pgfsetdash{}{0pt}%
\pgfsys@defobject{currentmarker}{\pgfqpoint{-0.048611in}{0.000000in}}{\pgfqpoint{-0.000000in}{0.000000in}}{%
\pgfpathmoveto{\pgfqpoint{-0.000000in}{0.000000in}}%
\pgfpathlineto{\pgfqpoint{-0.048611in}{0.000000in}}%
\pgfusepath{stroke,fill}%
}%
\begin{pgfscope}%
\pgfsys@transformshift{0.530556in}{1.338334in}%
\pgfsys@useobject{currentmarker}{}%
\end{pgfscope}%
\end{pgfscope}%
\begin{pgfscope}%
\definecolor{textcolor}{rgb}{0.000000,0.000000,0.000000}%
\pgfsetstrokecolor{textcolor}%
\pgfsetfillcolor{textcolor}%
\pgftext[x=0.363889in, y=1.290140in, left, base]{\color{textcolor}\rmfamily\fontsize{10.000000}{12.000000}\selectfont \(\displaystyle {5}\)}%
\end{pgfscope}%
\begin{pgfscope}%
\pgfsetbuttcap%
\pgfsetroundjoin%
\definecolor{currentfill}{rgb}{0.000000,0.000000,0.000000}%
\pgfsetfillcolor{currentfill}%
\pgfsetlinewidth{0.803000pt}%
\definecolor{currentstroke}{rgb}{0.000000,0.000000,0.000000}%
\pgfsetstrokecolor{currentstroke}%
\pgfsetdash{}{0pt}%
\pgfsys@defobject{currentmarker}{\pgfqpoint{-0.048611in}{0.000000in}}{\pgfqpoint{-0.000000in}{0.000000in}}{%
\pgfpathmoveto{\pgfqpoint{-0.000000in}{0.000000in}}%
\pgfpathlineto{\pgfqpoint{-0.048611in}{0.000000in}}%
\pgfusepath{stroke,fill}%
}%
\begin{pgfscope}%
\pgfsys@transformshift{0.530556in}{1.791765in}%
\pgfsys@useobject{currentmarker}{}%
\end{pgfscope}%
\end{pgfscope}%
\begin{pgfscope}%
\definecolor{textcolor}{rgb}{0.000000,0.000000,0.000000}%
\pgfsetstrokecolor{textcolor}%
\pgfsetfillcolor{textcolor}%
\pgftext[x=0.294444in, y=1.743571in, left, base]{\color{textcolor}\rmfamily\fontsize{10.000000}{12.000000}\selectfont \(\displaystyle {10}\)}%
\end{pgfscope}%
\begin{pgfscope}%
\pgfsetbuttcap%
\pgfsetroundjoin%
\definecolor{currentfill}{rgb}{0.000000,0.000000,0.000000}%
\pgfsetfillcolor{currentfill}%
\pgfsetlinewidth{0.803000pt}%
\definecolor{currentstroke}{rgb}{0.000000,0.000000,0.000000}%
\pgfsetstrokecolor{currentstroke}%
\pgfsetdash{}{0pt}%
\pgfsys@defobject{currentmarker}{\pgfqpoint{-0.048611in}{0.000000in}}{\pgfqpoint{-0.000000in}{0.000000in}}{%
\pgfpathmoveto{\pgfqpoint{-0.000000in}{0.000000in}}%
\pgfpathlineto{\pgfqpoint{-0.048611in}{0.000000in}}%
\pgfusepath{stroke,fill}%
}%
\begin{pgfscope}%
\pgfsys@transformshift{0.530556in}{2.245196in}%
\pgfsys@useobject{currentmarker}{}%
\end{pgfscope}%
\end{pgfscope}%
\begin{pgfscope}%
\definecolor{textcolor}{rgb}{0.000000,0.000000,0.000000}%
\pgfsetstrokecolor{textcolor}%
\pgfsetfillcolor{textcolor}%
\pgftext[x=0.294444in, y=2.197002in, left, base]{\color{textcolor}\rmfamily\fontsize{10.000000}{12.000000}\selectfont \(\displaystyle {15}\)}%
\end{pgfscope}%
\begin{pgfscope}%
\pgfsetbuttcap%
\pgfsetroundjoin%
\definecolor{currentfill}{rgb}{0.000000,0.000000,0.000000}%
\pgfsetfillcolor{currentfill}%
\pgfsetlinewidth{0.803000pt}%
\definecolor{currentstroke}{rgb}{0.000000,0.000000,0.000000}%
\pgfsetstrokecolor{currentstroke}%
\pgfsetdash{}{0pt}%
\pgfsys@defobject{currentmarker}{\pgfqpoint{-0.048611in}{0.000000in}}{\pgfqpoint{-0.000000in}{0.000000in}}{%
\pgfpathmoveto{\pgfqpoint{-0.000000in}{0.000000in}}%
\pgfpathlineto{\pgfqpoint{-0.048611in}{0.000000in}}%
\pgfusepath{stroke,fill}%
}%
\begin{pgfscope}%
\pgfsys@transformshift{0.530556in}{2.698627in}%
\pgfsys@useobject{currentmarker}{}%
\end{pgfscope}%
\end{pgfscope}%
\begin{pgfscope}%
\definecolor{textcolor}{rgb}{0.000000,0.000000,0.000000}%
\pgfsetstrokecolor{textcolor}%
\pgfsetfillcolor{textcolor}%
\pgftext[x=0.294444in, y=2.650433in, left, base]{\color{textcolor}\rmfamily\fontsize{10.000000}{12.000000}\selectfont \(\displaystyle {20}\)}%
\end{pgfscope}%
\begin{pgfscope}%
\pgfsetbuttcap%
\pgfsetroundjoin%
\definecolor{currentfill}{rgb}{0.000000,0.000000,0.000000}%
\pgfsetfillcolor{currentfill}%
\pgfsetlinewidth{0.803000pt}%
\definecolor{currentstroke}{rgb}{0.000000,0.000000,0.000000}%
\pgfsetstrokecolor{currentstroke}%
\pgfsetdash{}{0pt}%
\pgfsys@defobject{currentmarker}{\pgfqpoint{-0.048611in}{0.000000in}}{\pgfqpoint{-0.000000in}{0.000000in}}{%
\pgfpathmoveto{\pgfqpoint{-0.000000in}{0.000000in}}%
\pgfpathlineto{\pgfqpoint{-0.048611in}{0.000000in}}%
\pgfusepath{stroke,fill}%
}%
\begin{pgfscope}%
\pgfsys@transformshift{0.530556in}{3.152059in}%
\pgfsys@useobject{currentmarker}{}%
\end{pgfscope}%
\end{pgfscope}%
\begin{pgfscope}%
\definecolor{textcolor}{rgb}{0.000000,0.000000,0.000000}%
\pgfsetstrokecolor{textcolor}%
\pgfsetfillcolor{textcolor}%
\pgftext[x=0.294444in, y=3.103864in, left, base]{\color{textcolor}\rmfamily\fontsize{10.000000}{12.000000}\selectfont \(\displaystyle {25}\)}%
\end{pgfscope}%
\begin{pgfscope}%
\pgfsetbuttcap%
\pgfsetroundjoin%
\definecolor{currentfill}{rgb}{0.000000,0.000000,0.000000}%
\pgfsetfillcolor{currentfill}%
\pgfsetlinewidth{0.803000pt}%
\definecolor{currentstroke}{rgb}{0.000000,0.000000,0.000000}%
\pgfsetstrokecolor{currentstroke}%
\pgfsetdash{}{0pt}%
\pgfsys@defobject{currentmarker}{\pgfqpoint{-0.048611in}{0.000000in}}{\pgfqpoint{-0.000000in}{0.000000in}}{%
\pgfpathmoveto{\pgfqpoint{-0.000000in}{0.000000in}}%
\pgfpathlineto{\pgfqpoint{-0.048611in}{0.000000in}}%
\pgfusepath{stroke,fill}%
}%
\begin{pgfscope}%
\pgfsys@transformshift{0.530556in}{3.605490in}%
\pgfsys@useobject{currentmarker}{}%
\end{pgfscope}%
\end{pgfscope}%
\begin{pgfscope}%
\definecolor{textcolor}{rgb}{0.000000,0.000000,0.000000}%
\pgfsetstrokecolor{textcolor}%
\pgfsetfillcolor{textcolor}%
\pgftext[x=0.294444in, y=3.557295in, left, base]{\color{textcolor}\rmfamily\fontsize{10.000000}{12.000000}\selectfont \(\displaystyle {30}\)}%
\end{pgfscope}%
\begin{pgfscope}%
\pgfsetbuttcap%
\pgfsetroundjoin%
\definecolor{currentfill}{rgb}{0.000000,0.000000,0.000000}%
\pgfsetfillcolor{currentfill}%
\pgfsetlinewidth{0.803000pt}%
\definecolor{currentstroke}{rgb}{0.000000,0.000000,0.000000}%
\pgfsetstrokecolor{currentstroke}%
\pgfsetdash{}{0pt}%
\pgfsys@defobject{currentmarker}{\pgfqpoint{-0.048611in}{0.000000in}}{\pgfqpoint{-0.000000in}{0.000000in}}{%
\pgfpathmoveto{\pgfqpoint{-0.000000in}{0.000000in}}%
\pgfpathlineto{\pgfqpoint{-0.048611in}{0.000000in}}%
\pgfusepath{stroke,fill}%
}%
\begin{pgfscope}%
\pgfsys@transformshift{0.530556in}{4.058921in}%
\pgfsys@useobject{currentmarker}{}%
\end{pgfscope}%
\end{pgfscope}%
\begin{pgfscope}%
\definecolor{textcolor}{rgb}{0.000000,0.000000,0.000000}%
\pgfsetstrokecolor{textcolor}%
\pgfsetfillcolor{textcolor}%
\pgftext[x=0.294444in, y=4.010726in, left, base]{\color{textcolor}\rmfamily\fontsize{10.000000}{12.000000}\selectfont \(\displaystyle {35}\)}%
\end{pgfscope}%
\begin{pgfscope}%
\definecolor{textcolor}{rgb}{0.000000,0.000000,0.000000}%
\pgfsetstrokecolor{textcolor}%
\pgfsetfillcolor{textcolor}%
\pgftext[x=0.238889in,y=2.363000in,,bottom,rotate=90.000000]{\color{textcolor}\rmfamily\fontsize{10.000000}{12.000000}\selectfont Position Y [\(\displaystyle m\)]}%
\end{pgfscope}%
\begin{pgfscope}%
\pgfpathrectangle{\pgfqpoint{0.530556in}{0.515000in}}{\pgfqpoint{4.960000in}{3.696000in}}%
\pgfusepath{clip}%
\pgfsetrectcap%
\pgfsetroundjoin%
\pgfsetlinewidth{1.505625pt}%
\definecolor{currentstroke}{rgb}{0.121569,0.466667,0.705882}%
\pgfsetstrokecolor{currentstroke}%
\pgfsetdash{}{0pt}%
\pgfpathmoveto{\pgfqpoint{1.244158in}{0.884903in}}%
\pgfpathlineto{\pgfqpoint{1.244158in}{1.247648in}}%
\pgfpathlineto{\pgfqpoint{1.606902in}{1.247648in}}%
\pgfpathlineto{\pgfqpoint{1.606902in}{0.884903in}}%
\pgfpathlineto{\pgfqpoint{1.244158in}{0.884903in}}%
\pgfpathlineto{\pgfqpoint{1.244158in}{1.610393in}}%
\pgfpathlineto{\pgfqpoint{1.969647in}{1.610393in}}%
\pgfpathlineto{\pgfqpoint{1.969647in}{0.884903in}}%
\pgfpathlineto{\pgfqpoint{1.244158in}{0.884903in}}%
\pgfpathlineto{\pgfqpoint{1.244158in}{1.973138in}}%
\pgfpathlineto{\pgfqpoint{2.332392in}{1.973138in}}%
\pgfpathlineto{\pgfqpoint{2.332392in}{0.884903in}}%
\pgfpathlineto{\pgfqpoint{1.244158in}{0.884903in}}%
\pgfpathlineto{\pgfqpoint{1.244158in}{2.335883in}}%
\pgfpathlineto{\pgfqpoint{2.695137in}{2.335883in}}%
\pgfpathlineto{\pgfqpoint{2.695137in}{0.884903in}}%
\pgfpathlineto{\pgfqpoint{1.244158in}{0.884903in}}%
\pgfpathlineto{\pgfqpoint{1.244158in}{2.698627in}}%
\pgfpathlineto{\pgfqpoint{3.057882in}{2.698627in}}%
\pgfpathlineto{\pgfqpoint{3.057882in}{0.884903in}}%
\pgfpathlineto{\pgfqpoint{1.244158in}{0.884903in}}%
\pgfpathlineto{\pgfqpoint{1.244158in}{3.061372in}}%
\pgfpathlineto{\pgfqpoint{3.420627in}{3.061372in}}%
\pgfpathlineto{\pgfqpoint{3.420627in}{0.884903in}}%
\pgfpathlineto{\pgfqpoint{1.244158in}{0.884903in}}%
\pgfpathlineto{\pgfqpoint{1.244158in}{3.424117in}}%
\pgfpathlineto{\pgfqpoint{3.783372in}{3.424117in}}%
\pgfpathlineto{\pgfqpoint{3.783372in}{0.884903in}}%
\pgfpathlineto{\pgfqpoint{1.244158in}{0.884903in}}%
\pgfusepath{stroke}%
\end{pgfscope}%
\begin{pgfscope}%
\pgfsetrectcap%
\pgfsetmiterjoin%
\pgfsetlinewidth{0.803000pt}%
\definecolor{currentstroke}{rgb}{0.000000,0.000000,0.000000}%
\pgfsetstrokecolor{currentstroke}%
\pgfsetdash{}{0pt}%
\pgfpathmoveto{\pgfqpoint{0.530556in}{0.515000in}}%
\pgfpathlineto{\pgfqpoint{0.530556in}{4.211000in}}%
\pgfusepath{stroke}%
\end{pgfscope}%
\begin{pgfscope}%
\pgfsetrectcap%
\pgfsetmiterjoin%
\pgfsetlinewidth{0.803000pt}%
\definecolor{currentstroke}{rgb}{0.000000,0.000000,0.000000}%
\pgfsetstrokecolor{currentstroke}%
\pgfsetdash{}{0pt}%
\pgfpathmoveto{\pgfqpoint{5.490556in}{0.515000in}}%
\pgfpathlineto{\pgfqpoint{5.490556in}{4.211000in}}%
\pgfusepath{stroke}%
\end{pgfscope}%
\begin{pgfscope}%
\pgfsetrectcap%
\pgfsetmiterjoin%
\pgfsetlinewidth{0.803000pt}%
\definecolor{currentstroke}{rgb}{0.000000,0.000000,0.000000}%
\pgfsetstrokecolor{currentstroke}%
\pgfsetdash{}{0pt}%
\pgfpathmoveto{\pgfqpoint{0.530556in}{0.515000in}}%
\pgfpathlineto{\pgfqpoint{5.490556in}{0.515000in}}%
\pgfusepath{stroke}%
\end{pgfscope}%
\begin{pgfscope}%
\pgfsetrectcap%
\pgfsetmiterjoin%
\pgfsetlinewidth{0.803000pt}%
\definecolor{currentstroke}{rgb}{0.000000,0.000000,0.000000}%
\pgfsetstrokecolor{currentstroke}%
\pgfsetdash{}{0pt}%
\pgfpathmoveto{\pgfqpoint{0.530556in}{4.211000in}}%
\pgfpathlineto{\pgfqpoint{5.490556in}{4.211000in}}%
\pgfusepath{stroke}%
\end{pgfscope}%
\begin{pgfscope}%
\pgfsetbuttcap%
\pgfsetmiterjoin%
\definecolor{currentfill}{rgb}{1.000000,1.000000,1.000000}%
\pgfsetfillcolor{currentfill}%
\pgfsetfillopacity{0.800000}%
\pgfsetlinewidth{1.003750pt}%
\definecolor{currentstroke}{rgb}{0.800000,0.800000,0.800000}%
\pgfsetstrokecolor{currentstroke}%
\pgfsetstrokeopacity{0.800000}%
\pgfsetdash{}{0pt}%
\pgfpathmoveto{\pgfqpoint{3.799444in}{2.148555in}}%
\pgfpathlineto{\pgfqpoint{5.393333in}{2.148555in}}%
\pgfpathquadraticcurveto{\pgfqpoint{5.421111in}{2.148555in}}{\pgfqpoint{5.421111in}{2.176333in}}%
\pgfpathlineto{\pgfqpoint{5.421111in}{2.549666in}}%
\pgfpathquadraticcurveto{\pgfqpoint{5.421111in}{2.577444in}}{\pgfqpoint{5.393333in}{2.577444in}}%
\pgfpathlineto{\pgfqpoint{3.799444in}{2.577444in}}%
\pgfpathquadraticcurveto{\pgfqpoint{3.771667in}{2.577444in}}{\pgfqpoint{3.771667in}{2.549666in}}%
\pgfpathlineto{\pgfqpoint{3.771667in}{2.176333in}}%
\pgfpathquadraticcurveto{\pgfqpoint{3.771667in}{2.148555in}}{\pgfqpoint{3.799444in}{2.148555in}}%
\pgfpathclose%
\pgfusepath{stroke,fill}%
\end{pgfscope}%
\begin{pgfscope}%
\pgfsetrectcap%
\pgfsetroundjoin%
\pgfsetlinewidth{1.505625pt}%
\definecolor{currentstroke}{rgb}{0.121569,0.466667,0.705882}%
\pgfsetstrokecolor{currentstroke}%
\pgfsetdash{}{0pt}%
\pgfpathmoveto{\pgfqpoint{3.827222in}{2.473277in}}%
\pgfpathlineto{\pgfqpoint{4.105000in}{2.473277in}}%
\pgfusepath{stroke}%
\end{pgfscope}%
\begin{pgfscope}%
\definecolor{textcolor}{rgb}{0.000000,0.000000,0.000000}%
\pgfsetstrokecolor{textcolor}%
\pgfsetfillcolor{textcolor}%
\pgftext[x=4.216111in,y=2.424666in,left,base]{\color{textcolor}\rmfamily\fontsize{10.000000}{12.000000}\selectfont Ground truth}%
\end{pgfscope}%
\begin{pgfscope}%
\pgfsetbuttcap%
\pgfsetroundjoin%
\definecolor{currentfill}{rgb}{0.121569,0.466667,0.705882}%
\pgfsetfillcolor{currentfill}%
\pgfsetlinewidth{1.003750pt}%
\definecolor{currentstroke}{rgb}{0.121569,0.466667,0.705882}%
\pgfsetstrokecolor{currentstroke}%
\pgfsetdash{}{0pt}%
\pgfsys@defobject{currentmarker}{\pgfqpoint{-0.041667in}{-0.041667in}}{\pgfqpoint{0.041667in}{0.041667in}}{%
\pgfpathmoveto{\pgfqpoint{0.000000in}{-0.041667in}}%
\pgfpathcurveto{\pgfqpoint{0.011050in}{-0.041667in}}{\pgfqpoint{0.021649in}{-0.037276in}}{\pgfqpoint{0.029463in}{-0.029463in}}%
\pgfpathcurveto{\pgfqpoint{0.037276in}{-0.021649in}}{\pgfqpoint{0.041667in}{-0.011050in}}{\pgfqpoint{0.041667in}{0.000000in}}%
\pgfpathcurveto{\pgfqpoint{0.041667in}{0.011050in}}{\pgfqpoint{0.037276in}{0.021649in}}{\pgfqpoint{0.029463in}{0.029463in}}%
\pgfpathcurveto{\pgfqpoint{0.021649in}{0.037276in}}{\pgfqpoint{0.011050in}{0.041667in}}{\pgfqpoint{0.000000in}{0.041667in}}%
\pgfpathcurveto{\pgfqpoint{-0.011050in}{0.041667in}}{\pgfqpoint{-0.021649in}{0.037276in}}{\pgfqpoint{-0.029463in}{0.029463in}}%
\pgfpathcurveto{\pgfqpoint{-0.037276in}{0.021649in}}{\pgfqpoint{-0.041667in}{0.011050in}}{\pgfqpoint{-0.041667in}{0.000000in}}%
\pgfpathcurveto{\pgfqpoint{-0.041667in}{-0.011050in}}{\pgfqpoint{-0.037276in}{-0.021649in}}{\pgfqpoint{-0.029463in}{-0.029463in}}%
\pgfpathcurveto{\pgfqpoint{-0.021649in}{-0.037276in}}{\pgfqpoint{-0.011050in}{-0.041667in}}{\pgfqpoint{0.000000in}{-0.041667in}}%
\pgfpathclose%
\pgfusepath{stroke,fill}%
}%
\begin{pgfscope}%
\pgfsys@transformshift{3.966111in}{2.267513in}%
\pgfsys@useobject{currentmarker}{}%
\end{pgfscope}%
\end{pgfscope}%
\begin{pgfscope}%
\definecolor{textcolor}{rgb}{0.000000,0.000000,0.000000}%
\pgfsetstrokecolor{textcolor}%
\pgfsetfillcolor{textcolor}%
\pgftext[x=4.216111in,y=2.231055in,left,base]{\color{textcolor}\rmfamily\fontsize{10.000000}{12.000000}\selectfont Estimated position}%
\end{pgfscope}%
\end{pgfpicture}%
\makeatother%
\endgroup%
}
%         \caption{Davenport's 3D position estimation had the lowest displacement error for the squares experiment.}
%         \label{fig:squares2D}
%     \end{subfigure}
%     \begin{subfigure}{0.49\textwidth}
%         \centering
%         \resizebox{1\linewidth}{!}{\input{plots/square/squares3D.pgf}}
%         \caption{ROLEQ's 3D position estimation had the lowest turn error for the spiral experiment.}
%         \label{fig:squares3D}
%     \end{subfigure}
%     \caption{Position estimation by the best performing algorithms in the squares experiment.}
%     \label{fig:squares}
% \end{figure}
